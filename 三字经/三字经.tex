% 三字经
% 三字经.tex

\documentclass[a4paper,12pt,UTF8,twoside]{ctexbook}

% 设置纸张信息。
\RequirePackage[a4paper]{geometry}
\geometry{
	%textwidth=138mm,
	%textheight=215mm,
	%left=27mm,
	%right=27mm,
	%top=25.4mm, 
	%bottom=25.4mm,
	%headheight=2.17cm,
	%headsep=4mm,
	%footskip=12mm,
	%heightrounded,
	inner=1in,
	outer=1.25in
}

% 设置字体,并解决显示难检字问题。
\xeCJKsetup{AutoFallBack=true}
\setCJKmainfont{SimSun}[BoldFont=SimHei, ItalicFont=KaiTi, FallBack=SimSun-ExtB]

% 目录 chapter 级别加点(.)。
\usepackage{titletoc}
\titlecontents{chapter}[0pt]{\vspace{3mm}\bf\addvspace{2pt}\filright}{\contentspush{\thecontentslabel\hspace{0.8em}}}{}{\titlerule*[8pt]{.}\contentspage}

% 设置 part 和 chapter 标题格式。
\ctexset{
	chapter/name={},
	chapter/number={}
}

% 设置古文原文格式。
\newenvironment{yuanwen}{\bfseries\zihao{4}}

% 设置署名格式。
\newenvironment{shuming}{\hfill\bfseries\zihao{4}}

\title{\heiti\zihao{0} 三字经}
\author{王应麟}
\date{南宋}

\begin{document}

\maketitle
\tableofcontents

\frontmatter
\chapter{前言}

《三字经》是中国的传统启蒙教材。在中国古代经典当中,《三字经》是最浅显易懂的读本之一。《三字经》取材典范,包括中国传统文化的文学、历史、哲学、天文地理、人伦义理、忠孝节义等等,而核心思想又包括了“仁,义,诚,敬,孝。”背诵《三字经》的同时,就了解了常识、传统国学及历史故事,以及故事内涵中的做人做事道理。

在格式上,三字一句朗朗上口,因其文通俗、顺口、易记等特点,使其与《百家姓》、《千字文》并称为中国传统蒙学三大读物,合称“三百千”。

《三字经》是中华民族珍贵的文化遗产,它短小精悍、琅琅上口,千百年来,家喻户晓。其内容涵盖了历史、天文、地理、道德以及一些民间传说,所谓“熟读《三字经》,可知千古事”。基于历史原因,《三字经》难免含有一些精神糟粕、艺术瑕疵,但其独特的思想价值和文化魅力仍然为世人所公认,被历代中国人奉为经典并不断流传。

\chapter{历史背景}

关于《三字经》的成书年代和作者历代说法不一,大多数后代学者倾向的观点意见是“宋儒王伯厚先生作《三字经》,以课家塾”,即王应麟为了更好的教育本族子弟读书,于是编写了融会经史子集的三字歌诀。

王应麟是南宋人,《三字经》原著中的历史部分只截至到宋朝为止。随着历史的发展,为了体现时代变迁,各朝代都有人对《三字经》不断地加以补充,例如清道光年间贺兴思增补了关于元、明、清三代的历史,共计二十四句话。

\chapter{作品赏析}

《三字经》的内容分为六个部分,每一部分有一个中心。

从“人之初,性本善”到“人不学,不知义”,讲述的是教育和学习对儿童成长的重要性,后天教育及时,方法正确,可以使儿童成为有用之材;

从“为人子,方少时”至“首孝悌,次见闻”强调儿童要懂礼仪要孝敬父母、尊敬兄长,并举了黄香和孔融的例子;

从“知某数,识某文”到“此十义,人所同”介绍的是生活中的一些名物常识,有数字、三才、三光、三纲、四时、四方、五行、五常、六谷、六畜、七情、八音、九族、十义,方方面面,一应俱全,而且简单明了;

从“凡训蒙,须讲究”到“文中子,及老庄”介绍中国古代的重要典籍和儿童读书的程序,这部分列举的书籍有四书、六经、三易、四诗、三传、五子,基本包括了儒家的典籍和部分先秦诸子的著作;

从“经子通,读诸史”到“通古今,若亲目”讲述的是从三皇至清代的朝代变革,一部中国史的基本面貌尽在其中;

从“口而诵,心而维”至“戒之哉,宜勉力”强调学习要勤奋刻苦、孜孜不倦,只有从小打下良好的学习基础,长大才能有所作为,“上致君,下泽民”。

《三字经》内容的排列顺序极有章法,体现了作者的教育思想。作者认为教育儿童要重在礼仪孝悌,端正孩子们的思想,知识的传授则在其次,即“首孝悌,次见闻”。训导儿童要先从小学入手,即先识字,然后读经、子两类的典籍。经部子部书读过后,再学习史书,书中说:“经子通,读诸史”。《三字经》最后强调学习的态度和目的。可以说,《三字经》既是一部儿童识字课本,同时也是作者论述启蒙教育的著作,这在阅读时需加注意。《三字经》用典多,知识性强,是一部在儒家思想指导下编成的读物,充满了积极向上的精神。

\chapter{版本变化}

《三字经》是宋朝王应麟(存疑)先生所作,内容大都采用韵文,每三字一句,四句一组,像一首诗一样,背诵起来如同唱儿歌,用来教育子女朗朗上口十分有趣,又能启迪心智。时人觉得本书内容很好,纷纷翻印,因此广为流传,历久不衰,成为历朝历代最重要的童蒙养正教材之一。

随着时间的推移,在《三字经》的内容上,不同历史时期皆有所修改或增加。迄今为止所见的就有宋末元初的1068字本,明代的1092字本,明末的1122字本,清初的1140字本及1170字本等多个版本。并出现相关注解本、插图本。如明清时期就有《增补三字经》、《节增三字经》和《广三字经》等。清末民初的著名学者章太炎〈字炳麟〉先生的《三字经》增订本,是近一个世纪以来流传最广的版本。

1949年后,又对《三字经》进行修订。主要涉及一些民族史观内容的修订。如:

1.关于元代历史的“莅中国,兼戎狄,九十年,国祚废。”改为“舆图广,超前代,九十年,国祚废。”

2.关于清代历史的“清顺治,据神京。至十传,宣统逊。举总统,共和成。复汉土,民国兴。廿二史,全在兹,载治乱,知兴衰。”改为“清世祖,膺景命,靖四方,克大定。由康雍,历乾嘉,民安富,治绩夸。道咸间,变乱起,始英法,扰都鄙。同光后,宣统弱,传九帝,满清殁。革命兴,废帝制,立宪法,建民国。古今史,全在兹,载治乱,知兴衰。”

\chapter{作品影响}

中国

《三字经》是中国传统的儿童启蒙读物,知名度极高。古代儿童都是通过背诵《三字经》来识字知理的。《三字经》用简洁通俗的白话讲出了亘古不变的哲理,脍炙人口、广为流传;不受文字限制,用通俗的文字将经史子集等各部类的知识揉合在一起,全文用典极多,全篇充满乐观精神;在《三字经》出现之前,蒙学读物都是四个字一句,《三字经》则以三言形式出现,读起来轻松愉快,更符合儿歌特点,明朝赵南星称其“句短而易读,殊便于开蒙”,故此为蒙学第一书。《三字经》在古代被称为“小纲鉴”,可以将零散的知识贯穿起来,使读书积累的百科知识,得以纳入一个清晰知识体系。

国际

从明朝开始,《三字经》就已流传至中国以外的国家。

根据记载,世界上最早的《三字经》翻译本是拉丁文。1579年,历史上第一位研究汉学的欧洲人罗明坚,到澳门学习中文,他从1581年就开始着手翻译《三字经》,并将译文寄回意大利。

1727年,沙俄政府派遣一批人士到中国学习儒家文化,首先研读的就是《三字经》。其中一位学生罗索兴将它翻译为俄文,后入选培训教材,成为俄国文化界的流行读物。1779年,彼得堡帝俄科学院又公开出版了列昂节夫(1716~1786年)翻译的《三字经及名贤集合刊本》,因其内容与当时女皇叶卡捷林娜二世推行的讲求秩序的“开明专制”等政治策略不谋而合,政府遂正式“推荐给俄国公众”并走向民间。“俄国汉学之父”俾丘林(1777~1853年)曾在北京生活14年,深谙经史,更明晓《三字经》的文化内涵和社会影响,他在1829年推出《汉俄对照三字经》,并称《三字经》是“十二世纪的百科全书”。当时俄国教育界在讨论儿童教育问题,于是《三字经》成为“俄国人阅读中文翻译本的指南”,成为当时社会流行读物。普希金细读后,在作序时称赞此书是“三字圣经”。普希金研读过《四书》、《五经》,但对《三字经》情有独钟,如今普希金故居还珍藏着当年他读过的《三字经》。喀山大学和彼得堡大学的东方学系都以《三字经》为初级教材,而大多数入华商团和驻华使者的培训多以《三字经》为首选教材,因而,《三字经》在俄国文化历史上留下了自己深深的印记。

韩国、日本也对《三字经》也非常重视。日本早在江户时代(1603~1868年)已印行由中国商船带来的各种版本的《三字经》。从江户时代到明治初年(1868~1921年),日本的私塾已采用《三字经》,后更大量出现各种仿制本,如《本朝三字经》、《皇朝三字经》等,多达二十多种,其中影响最大的是三字押韵,介绍日本历史地理文化道德的《本朝三字经》。

英国的马礼逊(1782~1834年)翻译的第一本中国传统经典就是《三字经》。1812年,他出版《中国春秋》英文版,包括《三字经》和《大学》。修订后,1917年又在伦敦再版。

美国传教士裨治文在他主办的《中国丛报》上刊载《三字经》、《千字文》等启蒙读物。

在法国,犹太籍汉学家儒莲(1797年~1873年),在1827年担任法兰西研究院图书馆副馆长后翻译出《孟子》、《三字经》、《西厢记》、《白蛇传》、《老子道德经》、《天工开物》等中国典籍。

1989年,新加坡出版潘世兹翻译的英文本《三字经》,被推荐参加“法兰克福国际书展”,并成为新加坡的教科书。

1990年,《三字经》被联合国教科组织选编入《儿童道德丛书》,向世界各地儿童推介学习,成为一本世界著名的启蒙读物。 [8]

\chapter{关于作者}

原典作者

关于《三字经》的成书年代和作者历代说法不一,但是大多数学者的意见倾向于“宋儒王伯厚先生作《三字经》,以课家塾”。王应麟(1223—1296),南宋官员、学者。字伯厚,号深宁居士,又号厚斋。庆元府鄞县(今浙江鄞县)人。理宗淳祐元年进士,宝祐四年复中博学宏词科。历官太常寺主簿、通判台州,召为秘节监、权中书舍人,知徽州、礼部尚书兼给事中等职。其为人正直敢言,屡次冒犯权臣丁大全、贾似道而遭罢斥,后辞官回乡,专意著述二十年。为学宗朱熹,涉猎经史百家、天文地理,熟悉掌故制度,长于考证。一生著述颇富,计有二十余种、六百多卷,相传《三字经》为其所著。

王应麟晚年教育本族子弟读书的时候,编写了一本融会经史子集的三字歌诀,据传就是《三字经》。

关于《三字经》的作者及成书时间,还有其他说法。

一说是宋代人区适子。明末清初屈大均在“广东新语”卷十一中记载:“童蒙所诵三字经乃宋末区适子所撰。适子,顺德登洲人,字正叔,入元抗节不仕”,认为广东顺德人区适子才是《三字经》的真正作者。

一说是明代人黎贞。清代邵晋涵诗:“读得贞黎三字训”,自注:“《三字经》,南海黎贞撰。”即以为明代黎贞撰。

吴蒙校点《三字经》时,结合《三字经》中提到《四书》以及鼓励仕进等的内容,指出该书“似当作成于元延祐年恢复科举,规定考试程式中《四书》用朱氏集注之后”。同时又提及南宋陈淳用三字句写成的《启蒙初诵》疑似为《三字经》的先河,指出《三字经》从雏形到更定,经历了相当长的时间。

学者张如安根据南宋慈溪人桂氏《家训》中的相关记载认为,《三字经》应成书于南宋绍熙(1190年-1194年)至嘉定(1208年-1224年)年间,其时代要早于王应麟(1223年-1296年)和区适子。而宁波是目前已知的《三字经》最早流传的地区。

增改作者

原典内容之外,后经章太炎等人多次增改,故三字经内容在“叙史”部分,也已包含元、明、清、民国时期。

各版本题名也有差异,例如“三字经注解备要”、“重订三字经”等。

\mainmatter

\chapter{正 \quad 文}

\begin{yuanwen}
人之初\footnote{初始。这里指人初生下来时。},性\footnote{生性,天性。}本善。性相近,习\footnote{指人在成长过程中,因为后天的环境、教育不同,所形成的习性、习惯。}相远。

苟\footnote{如果。}不教,性乃迁\footnote{迁移,变化。}。教之道\footnote{方法。},贵\footnote{最宝贵的。这里指重视、注重。}以专\footnote{专一,始终不懈。}。
\end{yuanwen}

人出生之初,禀性本身都是善良的,天性也都相差不多,只是后天所处的环境不同和所受教育不同,彼此的习性才形成了巨大的差别。

如果从小不好好教育,善良的本性就会变坏。为了使人不变坏,最重要的方法就是要专心一致地去教育孩子。

\begin{yuanwen}
昔\footnote{往昔,过去。}孟母\footnote{孟子的母亲。孟子,名轲,战国著名思想家,儒家尊其为“亚圣”。},择\footnote{选择。}邻处\footnote{ch\v{u},指安家居住。据说孟母为培养孟子,曾三次搬家。}。子不学,断机杼\footnote{zh\`u,织布机上穿引纬线的梭子。}。

窦燕山\footnote{五代后周时人,名禹钧。因家居渔阳(今北京地区),地处燕山脚下,故号燕山。},有义方\footnote{指良好的家教。}。教五子,名俱\footnote{text}扬。
\end{yuanwen}

战国时,孟子的母亲,曾三次搬家,是为了使孟子有个好的学习环境。小孩子不肯好好学习,孟母就折断了织布的机杼来教育孟子。

五代时,燕山人窦禹钧教育儿子很有方法,他教育的五个儿子都很有成就,同时科举成名。

\begin{yuanwen}
养不教,父之过\footnote{text}。教不严,师之惰\footnote{text}。

子不学,非所宜\footnote{text}。幼不学,老何为\footnote{text}?
\end{yuanwen}

仅仅是供养儿女吃穿,而不好好教育,是父母的过错。只是教育,但不严格要求就是做老师的懒惰了。

小孩子不肯好好学习,是很不应该的。一个人倘若小时候不好好学习,到老的时候既不懂做人的道理,又无知识,那么到老的时候都很难有所作为的。

\begin{yuanwen}
玉不琢\footnote{text},不成器。人不学,不知义\footnote{text}。为人子,方少时。亲\footnote{text}师友,习礼仪。
\end{yuanwen}

玉不打磨雕刻,不会成为精美的器物;而人要是不懂得学习,以自己的知识、技能来实现自己的价值,就不懂得礼仪,不能成才。

做儿女的,从小时候就要亲近老师和朋友,以便从他们那里学习到许多为人处事的礼节和知识。

\begin{yuanwen}
香\footnote{text}九龄\footnote{text},能温席\footnote{text}。孝于亲,所当执\footnote{text}。

融\footnote{text}四岁,能让\footnote{text}梨。弟\footnote{text}于长\footnote{text},宜先知\footnote{text}。
\end{yuanwen}

东汉人黄香九岁时就知道孝敬父亲,替父亲暖被窝。每个孝顺父母的人都应该实行和效仿。

汉代人孔融四岁时,就知道把大的梨让给哥哥吃,这种尊敬和友爱兄长的道理,是每个人从小就应该知道的。

\begin{yuanwen}
首\footnote{text}孝弟\footnote{text},次见闻。知某数\footnote{text},识某文\footnote{text}。

一而十,十而百。百而千,千而万\footnote{text}。
\end{yuanwen}

人生急当首务者,莫大于孝悌,其次一等,多见天下之事,以广其所知,多闻古今之理,以广其所学。知十百千万之数为某数,识古今圣贤之事为某文也。

中国采用十进位算术方法:一到十是基本的数字,然后十个十是一百,十个一百是一千,十个一千是一万……一直变化下去。

\begin{yuanwen}
三才\footnote{text}者,天地人。三光者,日月星。

三纲\footnote{text}者,君臣义\footnote{text}。父子亲,夫妇顺\footnote{text}。
\end{yuanwen}

三才指的是天、地、人三个方面。三光就是太阳、月亮、星星。

三纲是人与人之间关系应该遵守的三个行为准则,君王与臣子的言行要合乎义理,父母子女之间相亲相爱,夫妻之间和顺相处。

\begin{yuanwen}
曰春夏,曰秋冬。此四时,运\footnote{text}不穷\footnote{text}。

曰南北,曰西东。此四方,应\footnote{text}乎中\footnote{text}。
\end{yuanwen}

春、夏、秋、冬叫做四季。季节不断变化,春去夏来,秋去冬来,如此循环往复,永不停止。

南、北、西、东,叫做“四方”,是指各个方向的位置。这四个方位,必须有个中央位置对应,才能把各个方位定出来。

\begin{yuanwen}
曰水火,木金土。此五行\footnote{text},本\footnote{text}乎数\footnote{text}。

十干者,甲至癸。十二支,子至亥。
\end{yuanwen}

水、火、木、金、土叫做“五行”。这是中国古代用来指宇宙各种事物的抽象概念,是根据一、二、三、四、五这五个数字和组合变化而产生的。

“十干”指的是甲、乙、丙、丁、戊、己、庚、辛、壬、癸,又叫“天干”;“十二支”指的是子、丑、寅、卯、辰、巳、午、未、申、酉、戌、亥,又叫“地支”。

\begin{yuanwen}
曰黄道,日所躔。曰赤道,当中权。
\end{yuanwen}

黄道指的是太阳行走的轨迹赤道指的是大地所在的平面。

\begin{yuanwen}
赤道下,温暖极。我中华,在东北。
\end{yuanwen}

在赤道地区,温度最高,气候特别炎热,从赤道向南北两个方向,气温逐渐变低。我们中国地处地球的东北边。

\begin{yuanwen}
寒燠均,霜露改。右高原,左大海。
\end{yuanwen}



\begin{yuanwen}
曰江河,曰淮济。此四渎,水之纪。
\end{yuanwen}

中国直接流入大海的有长江、黄河、还有淮河和济水,这四条大河是中国河流的代表。

\begin{yuanwen}
曰岱华,嵩恒衡。此五岳,山之名。
\end{yuanwen}

东岳泰山、西岳华山中岳嵩山、南岳衡山、北岳恒山,是中国的五大名山,称为“五岳”,这五座山是中国大山的代表。

\begin{yuanwen}
古九州,今改制。称行省,三十五。
\end{yuanwen}

\begin{yuanwen}
曰士农,曰工商。此四民,国之良。
\end{yuanwen}

知识分子、农民、工人和商人,称为“四民”,是国家不可缺少的栋梁,这是社会重要的组成部分。

\begin{yuanwen}
曰仁\footnote{text}义\footnote{text},礼\footnote{text}智\footnote{text}信\footnote{text}。此五常\footnote{text},不容紊\footnote{text}。
\end{yuanwen}

仁、义、礼、智、信叫做“五常”,这五种不变的法则是处事做人的标准,每个人都应遵守,不可怠慢疏忽。

\begin{yuanwen}
地所生,有草木。此植物,遍水陆。
\end{yuanwen}

大地上生长的,有花草树木,这些属于植物,在陆地上和水里到处都有。

\begin{yuanwen}
有虫鱼,有鸟兽。此动物,能飞走。
\end{yuanwen}

虫、鱼、鸟、兽属于动物,这些动物有的能在天空中飞,有的能在陆地上走,有的能在水里游。

\begin{yuanwen}
稻粱\footnote{text}菽\footnote{text},麦黍\footnote{text}稷\footnote{text}。此六谷,人所食。

马牛羊,鸡犬豕\footnote{text}。此六畜,人所饲。
\end{yuanwen}

稻米、小米、豆类、小麦、玉米、高粱为“六谷”,这些是日常生活的重要食品。

在动物中有马、牛、羊、鸡、狗和猪,这叫“六畜”。这些动物和六谷一样本来都是野生的。后来被人们渐渐驯化后,才成为人类日常生活的必需品。

\begin{yuanwen}
曰喜怒,曰哀惧。爱恶\footnote{text}欲\footnote{text},七情\footnote{text}具\footnote{text}。
\end{yuanwen}

高兴叫做喜,生气叫做怒,伤心叫做哀,害怕叫做惧,心里喜欢叫爱,讨厌叫恶,内心很贪恋叫做欲,合起来叫“七情”。这是人生下来就有的七种感情。

\begin{yuanwen}
青赤黄,及黑白。此五色,目所识。
\end{yuanwen}

青色、黄色、赤色、黑色和白色,这是中国古代传统的五行中的五种颜色,是人们的肉眼能够识别的。

\begin{yuanwen}
酸苦甘,及辛咸。此五味,口所含。
\end{yuanwen}

酸、甜、苦、辣和咸,称为“五味”,是我们能用嘴巴分辨出来的味道。

\begin{yuanwen}
膻焦香,及腥朽。此五臭,鼻所嗅。
\end{yuanwen}

羊膻味、烧焦味、香味、鱼腥味和腐朽味,称为“五臭”,是鼻子可以闻出的气味。

\begin{yuanwen}
匏\footnote{text}土\footnote{text}革\footnote{text},木\footnote{text}石\footnote{text}金\footnote{text}。丝\footnote{text}与竹\footnote{text},乃八音\footnote{text}。
\end{yuanwen}

匏瓜、黏土、皮革、木块、石头、金属、丝线与竹子,称为“八音”,是中国古代人制造乐器的材料。

\begin{yuanwen}
曰平上,曰去入。此四声,宜调协。
\end{yuanwen}

平、上、去、入,被称为“四音”,四音的运用必须和谐,听起来才能使人舒畅。

\begin{yuanwen}
高\footnote{text}曾\footnote{text}祖\footnote{text},父而身\footnote{text}。身而子,子而孙。

自子孙,至玄\footnote{text}曾\footnote{text}。乃九族\footnote{text},人之伦\footnote{text}。
\end{yuanwen}

由高祖父生曾祖父,曾祖父生祖父,祖父生父亲,父亲生自己本身,自己生儿子,儿子再生孙子。

由自己的儿子、孙子再接下去,就是曾孙和玄孙。从高祖父到玄孙称为“九族”。这“九族”代表着人的长幼尊卑秩序和家族血统的承续关系。

\begin{yuanwen}
父子恩\footnote{text},夫妇从\footnote{text}。兄则友\footnote{text},弟则恭\footnote{text}。

长幼序\footnote{text},友与朋\footnote{text}。君则敬\footnote{text},臣则忠。

此十义\footnote{text},人所同。当师(顺)叙,勿违背。
\end{yuanwen}

父亲与儿子之间要注重相互的恩情,夫妻之间的感情要和顺,哥哥对弟弟要友爱,弟弟对哥哥则要尊敬。

年长的和年幼的交往要注意长幼尊卑的次序;朋友相处应该互相讲信用。如果君主能尊重他的臣子,官吏们就会对他忠心耿耿了。

前面提到的“十义”:父慈、子孝、夫和、妻顺、兄友、弟恭、朋信、友义、君敬、臣忠,对于每个人都相同,人人都应遵守,千万不能违背。

\begin{yuanwen}
斩齐衰,大小功。至缌麻,五服终。
\end{yuanwen}

斩衰、齐衰、大功、小功还有缌麻,称为“五服”,是中国古代亲族中不同的人死去时穿的五种孝服。

\begin{yuanwen}
礼乐射,御书数。古六艺,今不具。
\end{yuanwen}

礼法、音乐、射箭、驾车、书法和算数称为“六艺”,是古代读书人必须学习的六种技艺,这六种技艺到现已经没有人能同时具备了。

\begin{yuanwen}
惟书学,人共遵。既识字,讲说文。
\end{yuanwen}

在“六艺”中,只有书法现今社会还是每个人都推崇的。当一个人认识字以后,就可以去研究《说文解字》,这样对于研究高深的学问是有帮助的。

\begin{yuanwen}
有古文,大小篆。隶草继,不可乱。
\end{yuanwen}

先有古文、然后有大篆、小篆隶书、草书紧随其后文字发展的顺序一定要认清楚,不可搞混乱了。

\begin{yuanwen}
若广学,惧其繁。但略说,能知原。
\end{yuanwen}

假如想广泛地学习知识,实在是不容易的事,也无从下手,但如能做大体研究,还是能了解到许多基本的道理。

\begin{yuanwen}
凡训\footnote{text}蒙\footnote{text}\footnote{text},须讲\footnote{text}究\footnote{text}。详\footnote{text}训诂\footnote{text},名句读\footnote{text}。

为学\footnote{text}者,必有初\footnote{text}。小学\footnote{text}终,至四书\footnote{text}。
\end{yuanwen}

凡是教导刚入学的儿童的老师,必须把每个字都讲清楚,每句话都要解释明白,并且使学童读书时懂得断句。

作为一个学者,求学的初期打好基础,把小学知识学透了,才可以读“四书”。

\begin{yuanwen}
论语者,二十篇。群弟子,记善言\footnote{text}。

孟子者,七篇止。讲道德,说仁义。
\end{yuanwen}

《论语》这本书,共有二十篇。是孔子的弟子们,以及弟子的弟子们,记载有关孔子言论的一部书。

《孟子》这本书是孟轲所作,共分七篇。内容也是有关品行修养、发扬道德仁义等优良德行的言论。

\begin{yuanwen}
作中庸\footnote{text},子思\footnote{text}笔。中\footnote{text}不偏,庸\footnote{text}不易\footnote{text}。

作大学\footnote{text},乃曾子\footnote{text}。自修\footnote{text}齐\footnote{text},至平\footnote{text}治\footnote{text}。
\end{yuanwen}

创作《中庸》这本书的人是子思,即孔伋,“中”是不偏的意思,“庸”是不变的意思。

创作《大学》这本书的人是曾子,他提出了先修身齐家,才能治国平天下。

\begin{yuanwen}
孝经\footnote{text}通,四书熟。如六经\footnote{text},始可读。

诗\footnote{text}书\footnote{text}易\footnote{text},礼\footnote{text}春秋\footnote{text}。号六经,当讲求。
\end{yuanwen}

把孝经的道理弄明白了,把四书读熟了,像六经这样深奥的书,才可以开始研读。

《诗》、《书》、《易》、《礼》、《春秋》,再加上《乐》称“六经”,这是中国古代儒家的重要经典,应当仔细阅读。

\begin{yuanwen}
有连山\footnote{text},有归藏\footnote{text}。有周易,三易\footnote{text}详\footnote{text}。

有典\footnote{text}谟\footnote{text},有训\footnote{text}诰\footnote{text}。有誓\footnote{text}命\footnote{text},书\footnote{text}之奥\footnote{text}。
\end{yuanwen}

《连山》、《归藏》、《周易》,是我国古代的三部书,这三部书合称“三易”,“三易”是用“卦”的形式来说明宇宙间万事万物循环变化的道理的书籍。

一典,是立国的基本原则;二谟,即治国计划;三训,即大臣的态度;四诰,即国君的通告;五誓,起兵文告;六命,国君的命令,是《书经》的主要内容。

\begin{yuanwen}
我周公\footnote{text},作周礼。著六官\footnote{text},存\footnote{text}治体\footnote{text}。

大小戴\footnote{text},注礼记\footnote{text}。述圣言,礼乐备\footnote{text}。
\end{yuanwen}

周公著作了《周礼》,其中记载着当时六宫的官制以及国家的组成情况。

戴德和戴圣,整理并且注释《礼记》,传述和阐扬了圣贤的著作,这使后代人知道了前代的典章制度和有关礼乐的情形。

\begin{yuanwen}
曰国风\footnote{text},曰雅\footnote{text}颂\footnote{text}。号四诗\footnote{text},当讽咏\footnote{text}。
\end{yuanwen}

《国风》、《大雅》、《小雅》、《颂》,合称为四诗,它是一种内容丰富、感情深切的诗歌,值得去朗诵。

\begin{yuanwen}
诗既亡,春秋作。寓\footnote{text}褒贬\footnote{text},别善恶。

三传\footnote{text}者,有公羊\footnote{text}。有左氏\footnote{text},有毂梁\footnote{text}。
\end{yuanwen}

由于周朝的衰落,诗经也就跟着被冷落了,所以孔子就作《春秋》,在这本书中隐含着对现实政治的褒贬以及对各国善恶行为的分辨。

三传就是羊高所著的《公羊传》,左丘明所著的《左传》和谷梁赤所著的《谷梁传》,这些是解释《春秋》的书。

\begin{yuanwen}
(此处有多个版本,版本1,无新增内容)
\end{yuanwen}

\begin{yuanwen}
(版本2)

尔雅者,善辨言,求经训,此莫先。

古圣著,先贤传,注疏备,十三经。

左传外,有国语,合群经,数十五。
\end{yuanwen}

\begin{yuanwen}
(版本3)

尔雅者,善辨言,求经训,此莫先。

注疏备,十三经,惟“大戴”,疏未成。

《左传》外,有《国语》,合群经,数十五。
\end{yuanwen}

\begin{yuanwen}
经\footnote{text}既明,方读子\footnote{text}。撮\footnote{text}其要\footnote{text},记其事。

五子者,有荀\footnote{text}扬\footnote{text}。文中子\footnote{text},及老\footnote{text}庄\footnote{text}。
\end{yuanwen}

经传都读熟了然后读子书。子书繁杂,必须选择比较重要的来读,并且要记住每件事的本末因果。

五子是指荀子、扬子、文中子、老子和庄子。他们所写的书,便称为子书。

\begin{yuanwen}
经子通,读诸史。考\footnote{text}世系\footnote{text},知终始\footnote{text}。

自羲\footnote{text}农\footnote{text},至黄帝\footnote{text}。号三皇\footnote{text},居上世\footnote{text}。
\end{yuanwen}

经书和子书读熟了以后,再读史书、读史时必须要考究各朝各代的世系,明白他们盛衰的原因,才能从历史中记取教训。

自伏羲氏、神农氏到黄帝,后人尊称他们为“三皇”,这三位上古时代的帝王都能勤政爱民、非常伟大。

\begin{yuanwen}
唐\footnote{text}有虞\footnote{text},号二帝。相揖逊\footnote{text},称盛世。

夏有禹\footnote{text},商有汤\footnote{text}。周文武\footnote{text},称三王。
\end{yuanwen}

黄帝之后,有唐尧和虞舜二位帝王,后人尊称为“二帝”,尧认为自己的儿子不肖,而把帝位传给了才德兼备的舜,在两位帝王治理下,天下太平,人人称颂。

夏朝的开国君主是禹,商朝的开国君主是汤,周武王起兵灭掉商朝,这几个德才兼备的君王被后人称为三王。

\begin{yuanwen}
夏传子\footnote{text},家天下\footnote{text}。四百载\footnote{text},迁\footnote{text}夏社\footnote{text}。

汤伐\footnote{text}夏,国号商。六百载,至纣\footnote{text}亡。
\end{yuanwen}

禹把帝位传给自己的儿子,从此天下就成为一个家族所有的了。经过四百多年,夏被汤灭掉,从而结束了它的统治。

汤朝征讨夏朝,定国号为商,过了六百多年,直到纣的灭亡。

\begin{yuanwen}
周武王,始诛\footnote{text}纣。八百载\footnote{text},最长久。

周辙东\footnote{text},王纲\footnote{text}坠\footnote{text}。逞干戈\footnote{text},尚\footnote{text}游说\footnote{text}。
\end{yuanwen}

周武王起兵灭掉商朝,杀死纣王,建立周朝。周朝前后延续了八百多年,持续的历史最长。

自从周平王东迁国都后,对诸侯的控制力就越来越弱了。诸侯国之间时常发生战争,而游说之士也开始大行其道。

\begin{yuanwen}
始春秋\footnote{text},终战国\footnote{text}。五霸\footnote{text}强,七雄\footnote{text}出。

嬴秦氏\footnote{text},始兼并\footnote{text}。传二世\footnote{text},楚\footnote{text}汉\footnote{text}争\footnote{text}。
\end{yuanwen}

东周分为两个阶段,始于春秋时期,终于战国时期。春秋时的齐桓公、宋襄公、晋文公、秦穆公和楚庄王号称五霸。战国的七雄分别为齐楚燕韩赵魏秦。

战国末年,秦国的势力日渐强大,把其他诸侯国都灭掉了,建立了统一的秦朝。秦传到二世胡亥,天下又开始大乱,最后,形成楚汉相争的局面。

\begin{yuanwen}
高祖\footnote{text}兴\footnote{text},汉业建。至孝平\footnote{text},王莽\footnote{text}篡。

光武\footnote{text}兴,为东汉。四百年,终于献\footnote{text}。
\end{yuanwen}

汉高祖打败了项羽,建立汉朝。汉朝的帝位传了两百多年,到了孝平帝时,就被王莽篡夺了。

王莽篡权,改国号为新,天下大乱,刘秀推翻更始帝,恢复国号为汉,史称东汉光武帝,东汉延续四百年,到汉献帝的时候灭亡。

\begin{yuanwen}
魏\footnote{text}蜀\footnote{text}吴\footnote{text},争汉鼎\footnote{text}。号三国,迄\footnote{text}两晋。

宋\footnote{text}齐\footnote{text}继,梁\footnote{text}陈\footnote{text}承。为南朝\footnote{text},都金陵\footnote{text}。
\end{yuanwen}

东汉末年,魏国、蜀国、吴国三个国家争夺天下,形成三国相争的局面。后来魏灭了蜀国和吴国,但被司马炎篡夺了帝位,建立了晋朝,晋又分为东晋和西晋两个时期。

晋朝王室南迁以后,不久就衰亡了,继之而起的是南北朝时期。宋、齐和梁陈四个政权统称为“南朝”,定都在金陵。

\begin{yuanwen}
北\footnote{text}元魏\footnote{text},分东西\footnote{text}。宇文周\footnote{text},与高齐\footnote{text}。

迨\footnote{text}至隋\footnote{text},一\footnote{text}土宇\footnote{text}。不再传\footnote{text},失统绪\footnote{text}。
\end{yuanwen}

北朝则指的是元魏。元魏后来也分裂成东魏和西魏,西魏被宇文觉篡了位,建立了北周;东魏被高洋篡了位,建立了北齐。

等到隋朝,杨坚重新统一了中国,历史上称为隋文帝。他的儿子隋炀帝杨广即位后,荒淫无道,隋朝很快就灭亡了。

\begin{yuanwen}
唐高祖\footnote{text},起义师。除隋乱,创国基\footnote{text}。

二十传\footnote{text},三百载\footnote{text}。梁\footnote{text}灭之,国乃改。
\end{yuanwen}

唐高祖李渊率领正义之师反隋,他战胜了各路的反隋义军,取得了天下,建立起唐朝。

唐朝总共传了二十位皇帝,其统治近三百年。到唐昭宣帝时被朱全忠所灭,建立了梁朝,唐朝从此灭亡,改朝换代。

\begin{yuanwen}
梁\footnote{text}唐\footnote{text}晋\footnote{text},及汉\footnote{text}周\footnote{text}。称五代\footnote{text},皆有由\footnote{text}。

炎宋\footnote{text}兴,受周禅\footnote{text}。十八传\footnote{text},南北混\footnote{text}。
\end{yuanwen}

为和南北朝时期的梁相区别,历史上称为后梁。后梁、后唐、后晋、后汉和后周五个朝代的更替时期,历史上称作“五代”,这五个朝代的更替都有着一定的原因。

赵匡胤建立了宋朝,其接受了后周“禅让”的帝位。宋朝相传了十八个皇帝之后,北方的少数民族南下侵扰,结果又成了南北混战的局面。

\begin{yuanwen}
(版本1)辽与金,皆称帝。元灭金,绝宋世。

舆图广,超前代。九十年,国祚废。
\end{yuanwen}

北方的辽人、金人和蒙古人都建立了国家,自称皇帝,最后蒙古人灭了金朝和宋朝,建立了元朝,重又统一了中国。

元朝的疆域很广大,所统治的领土,超过了以前的每一个朝代。然而它只维持了短短九十年,就被农民起义推翻了。

\begin{yuanwen}
(版本2)辽\footnote{text}与金\footnote{text},帝号纷\footnote{text}。迨灭辽,宋犹存\footnote{text}。

至元\footnote{text}兴,金绪\footnote{text}歇\footnote{text}\footnote{text}。有宋世,一同灭\footnote{text}。

并中国\footnote{text},兼\footnote{text}戎狄\footnote{text}。九十年\footnote{text},国祚\footnote{text}废\footnote{text}。
\end{yuanwen}

\begin{yuanwen}
(版本1)明太祖\footnote{text},久亲师\footnote{text}。传建文\footnote{text},方四祀\footnote{text}。

迁北京,永乐\footnote{text}嗣。迨崇祯\footnote{text},煤山逝。
\end{yuanwen}

\begin{yuanwen}
(版本2)太祖兴,国大明。号洪武,都金陵。

迨成祖,迁燕京。十六世,至崇祯。

权阉肆,寇如林。李闯出,神器焚。
\end{yuanwen}

元朝末年,明太祖朱元璋起义,最后推翻元朝统治,统一全国,建立大明,年号洪武,定都在金陵。

到明成祖即位后,把国都由金陵迁到北方的燕京。明朝共传了十六个皇帝,直到崇祯皇帝为止,明朝就灭亡了。

明朝末年,宦官专权,天下大乱,老百姓纷纷起义,以闯王李自成为首的起义军攻破北京,迫使崇祯皇帝自杀,烧毁明陵,明朝最后灭亡。

\begin{yuanwen}
(版本1)清太祖\footnote{text},膺\footnote{text}景命\footnote{text}。靖\footnote{text}四方,克\footnote{text}大定\footnote{text}。

至世祖\footnote{text},乃大同\footnote{text}。十二世\footnote{text},清祚\footnote{text}终\footnote{text}。
\end{yuanwen}

\begin{yuanwen}
(版本2)清世祖,膺景命。靖四方,克大定。

由康雍,历乾嘉。民安富,治绩夸。

道咸间,变乱起。始英法,扰都鄙。

同光后,宣统弱。传九帝,满清殁。

革命兴,废帝制。立宪法,建民国。
\end{yuanwen}

清军入关后,清世祖顺治皇帝,“顺应天命”在北京登上帝座,平定了各地的混乱局面,使得老百姓可以重新安定地生活。

顺治皇帝以后,分别是康熙、雍正、乾隆和嘉庆四位皇帝,在此期间,天下太平,人民生活比较安定,国家也比较强盛。

清朝道光、咸丰年间,发生了变乱,英军挑起鸦片战争。英法两国分别以亚罗号事件和法国神父被杀为由组成联军,直攻北京。

同治、光绪皇帝以后,当传到宣统皇帝时,清朝的国势已经破败不堪,清朝只传递了九代,就被孙中山领导的辛亥革命推翻了。

孙中山领导的辛亥革命,推翻了清朝政府的统治,废除了帝制,建立了宪法,成立了中华民国政府,孙中山任临时大总统。

\begin{yuanwen}
古今史,全在兹。载治乱,知兴衰。
\end{yuanwen}

三皇五帝到建立民国的古今历史,全都列在此处了,通过对历史的学习,可以了解各朝各代的治乱兴衰,领悟到许多有益的东西。

\begin{yuanwen}
史虽繁,读有次。史记一,汉书二。
\end{yuanwen}

中国的历史书虽然纷繁、复杂,但在读的时候应该有次序:先读《史记》,然后读《汉书》。

\begin{yuanwen}
后汉三,国志四。兼证经,参通鉴。
\end{yuanwen}

第三读《后汉书》,第四读《三国志》,读的同时,还要参照经书,参考《资治通鉴》,这样就可以更好地了解历史的治乱兴衰了。

\begin{yuanwen}
读史者,考实录\footnote{text}。通古今,若亲目\footnote{text}。

口而诵,心而惟\footnote{text}。朝\footnote{text}于斯\footnote{text},夕\footnote{text}于斯。
\end{yuanwen}

读历史的人应该更进一步地去翻阅历史资料,了解古往今来事情的前因后果,就好像是自己亲眼所见一样。

\begin{yuanwen}
昔仲尼\footnote{text},师项橐\footnote{text}。古圣贤,尚勤学。

赵中令\footnote{text},读鲁论\footnote{text}。彼既仕\footnote{text},学且勤。
\end{yuanwen}

从前,孔子是个十分好学的人,当时鲁国有一位神童名叫项橐,孔子就曾向他学习。像孔子这样的圣贤,尚不忘勤学。

宋朝时赵中令——赵普,他官已经做到了中书令了,天天还手不释卷地阅读论语,不因为自己已经当了高官,而忘记勤奋学习。

\begin{yuanwen}
披蒲编\footnote{text},削竹简\footnote{text}。彼无书,且知勉。

头悬梁\footnote{text},锥刺股\footnote{text}。彼不教\footnote{text},自勤苦。
\end{yuanwen}

西汉时路温舒把文字抄在蒲草上阅读。公孙弘将春秋刻在竹子削成的竹片上。两人都很穷,买不起书,但还不忘勤奋学习。

东汉的孙敬读书时把自己的头发拴在屋梁上,以免打瞌睡。战国时苏秦读书每到疲倦时就用锥子刺大腿,他们不用别人督促而自觉勤奋苦读。

\begin{yuanwen}
如囊萤\footnote{text},如映雪\footnote{text}。家虽贫,学不辍\footnote{text}。

如负薪\footnote{text},如挂角\footnote{text}。身虽劳,犹苦卓\footnote{text}。
\end{yuanwen}

晋朝人车胤,把萤火虫放在纱袋里当照明读书。孙康则利用积雪的反光来读书。他们两人家境贫苦,却能在艰苦条件下继续求学。

汉朝的朱买臣,以砍柴维持生活,每天边担柴边读书。隋朝李密放牛把书挂在牛角上,有时间就读。他们在艰苦的环境里仍坚持读书。

\begin{yuanwen}
苏老泉\footnote{text},二十七。始发愤,读书籍。

彼既老,犹悔迟。尔\footnote{text}小生\footnote{text},宜早思。
\end{yuanwen}

唐宋八大家之一的苏洵,号老泉,小时候不想念书,到了二十七岁的时候,才开始下决心努力学习,后来成了大学问家。

像苏老泉上了年纪,才后悔当初没好好读书,而我们应该趁着年轻的时候,更应该把握大好时光,发奋读书,才不至于将来后悔。

\begin{yuanwen}
若梁灏\footnote{text},八十二。对大廷\footnote{text},魁\footnote{text}多士。

彼既成,众称异。尔小生,宜立志。
\end{yuanwen}

宋朝有个梁灏,在八十二岁时才考中状元,在金殿上对皇帝提出的问题对答如流,所有参加考试的人都不如他。

梁灏这么大年纪,尚能获得成功,不能不使大家感到惊异,钦佩他的好学不倦。而我们应该趁着年轻的时候,立定志向,努力用功就一定前途无量。

\begin{yuanwen}
莹\footnote{text}八岁,能咏诗。泌\footnote{text}七岁,能赋棋。

彼颖悟\footnote{text},人称奇。尔幼学,当效\footnote{text}之。
\end{yuanwen}

北齐有个叫祖莹的人,八岁的时候,就能吟诗,后来当了秘书监著作郎。另外唐朝有个叫李泌的人,七岁时,就能以下棋为题而作出诗赋。

他们两个人的聪明和才智,在当时很受人们的赞赏和称奇,我们正值求学的开始,应该效法他们,努力用功读书。

\begin{yuanwen}
蔡文姬\footnote{text},能辨琴。谢道韫\footnote{text},能咏吟。

彼女子,且聪敏。尔男子,当自警\footnote{text}。
\end{yuanwen}

在古代有许多出色的女能人。像东汉末年的蔡文姬,能分辨琴声好坏,晋朝的才女谢道韫,则能出口成诗。

像这样的两个女孩子,一个懂音乐,一个会做诗,天资如此聪慧;身为一个男子汉,更要时时警惕,充实自己才对。

\begin{yuanwen}
唐刘晏\footnote{text},方七岁。举\footnote{text}神童,作正字\footnote{text}。

彼虽幼,身已仕。尔幼学,勉而致\footnote{text}。
\end{yuanwen}

唐玄宗时,有一个名叫刘晏的小孩子,才只有七岁,就被推举为神童,并且做了负责刊正文字的官。读书学习,要有恒心,要一边读,一边用心去思考。只有早早晚晚都把心思用到学习上,才能真正学好。

\begin{yuanwen}
有为者,亦若是\footnote{text}。
\end{yuanwen}

刘晏虽然年纪这么小,但却已经做官来,担当国家给他的重任,要想成为一个有用的人,只要勤奋好学,也可以和刘晏一样名扬后世。

\begin{yuanwen}
犬守夜,鸡司\footnote{text}晨。苟\footnote{text}不学,曷\footnote{text}为人?

蚕吐丝,蜂酿蜜。人不学,不如物。
\end{yuanwen}

狗在夜间会替人看守家门,鸡在每天早晨天亮时报晓,人如果不能用心学习、迷迷糊糊过日子,有什么资格称为人呢。

蚕吐丝以供做衣料,蜜蜂可以酿制蜂蜜,供人们食用。而人要是不懂得学习,以自己的知识、技能来实现自己的价值,真不如小动物。

\begin{yuanwen}
幼而学,壮\footnote{text}而行\footnote{text}。上致君\footnote{text},下泽民\footnote{text}。

扬名声,显父母。光于前,裕\footnote{text}于后。
\end{yuanwen}

要在幼年时努力学习不断充实自己,长大后能够学以致用,上替国家效力,下为人民谋福利。

如果为人民做出应有的贡献,就会得到赞扬,自己的父母也可以得到荣耀,给祖先增添了光彩,也给下代留下了好的榜样。

\begin{yuanwen}
人遗\footnote{留下。}子,金满籯\footnote{y\'ing,竹箱,竹筐。}。我教子,唯一经\footnote{泛指经典、经书。这里是作者对自己《三字经》的自称。}。

勤有功,戏无益。戒之哉,宜勉力。
\end{yuanwen}

有的人遗留给子孙后代的是金银钱财,而我并不这样,我只希望他们能精于读书学习,长大后做个有所作为的人。

凡是勤奋上进的人,都会有好的收获,而只顾贪玩,浪费了大好时光是一定要后悔的。要时刻提醒自己,勉励自己好好学习。

\end{document}