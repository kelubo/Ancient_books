\documentclass[UTF8]{ctexbook}

\usepackage{titletoc}
\titlecontents{chapter}[0pt]{\vspace{3mm}\bf\addvspace{2pt}\filright}{\contentspush{\thecontentslabel\hspace{0.8em}}}{}{\titlerule*[8pt]{.}\contentspage}

\title{孙子兵法}
\author{孙武}
\date{}

\begin{document}
	
	\maketitle
	\thispagestyle{empty}
	
	\newpage
	\tableofcontents
	
	\chapter{始计篇}
	
	孙子曰:兵者,国之大事,死生之地,存亡之道,不可不察也。
	
	故经之以五事,校之以计,而索其情:一曰道,二曰天,三曰地,四曰将、五曰法。道者,令民与上同意也,故可以与之死,可以与之生,而不畏危。天者,阴阳,寒暑、时制也。地者,远近、险易、广狭、死生也。将者,智、信、仁、勇、严也。法者,曲制、官道、主用也。凡此五者,将莫不闻,知之者胜,不知者不胜。故校之以计,而索其情,曰:主孰有道?将孰有能?天地孰得?法令孰行?兵众孰强?士卒孰练?赏罚孰明?吾以此知胜负矣。
	
	将听吾计,用之必胜,留之;将不听吾计,用之必败,去之。计利以听,乃为之势,以佐其外。势者,因利而制权也。
	
	兵者,诡道也。故能而示之不能,用而示之不用,近而示之远,远而示之近;利而诱之,乱而取之,实而备之,强而避之,怒而挠之,卑而骄之,佚而劳之,亲而离之。攻其无备,出其不意。此兵家之胜,不可先传也。
	
	夫未战而庙算胜者,得算多也;未战而庙算不胜者,得算少也。多算胜,少算不胜,而况于无算乎?吾以此观之,胜负见矣。
	
	\chapter{作战篇}
	
	孙子曰:凡用兵之法,驰车千驷,革车千乘,带甲十万,千里馈粮,则内外之费,宾客之用,胶漆之材,车甲之奉,日费千金,然后十万之师举矣。
	其用战也胜,久则钝兵挫锐,攻城则力屈,久暴师则国用不足。夫钝兵挫锐,屈力殚货,则诸侯乘其弊而起,虽有智者,不能善其后矣。故兵闻拙速,未睹巧之久也。夫兵久而国利者,未之有也。故不尽知用兵之害者,则不能尽知用兵之利也。
	
	善用兵者,役不再籍,粮不三载;取用于国,因粮于敌,故军食可足也。国之贫于师者远输,远输则百姓贫。近于师者贵卖,贵卖则百姓财竭,财竭则急于丘役。力屈、财殚,中原内虚于家。百姓之费,十去其七;公家之费,破车罢马,甲胄矢弩。戟楯蔽橹,丘牛大车,十去其六。
	
	故智将务食于敌。食敌一钟,当吾二十钟;芑秆一石,当吾二十石。
	
	故杀敌者,怒也;取敌之利者,货也。故车战,得车十乘已上,赏其先得者,而更其旌旗,车杂而乘之,卒善而养之,是谓胜敌而益强。
	
	故兵贵胜,不贵久。故知兵之将,生民之司命,国家安危之主也。
	
	\chapter{谋攻篇}
	
	孙子曰:夫用兵之法,全国为上,破国次之,全军为上,破军次之;全旅为上,破旅次之;全卒为上,破卒次之;全伍为上,破伍次之。是故百战百胜,非善之善者也;不战而屈人之兵,善之善者也。
	
	故上兵伐谋,其次伐交,其次伐兵,其下攻城。攻城之法为不得已。修橹轒辒,具器械,三月而后成,距堙,又三月而后已。将不胜其忿而蚁附之,杀士三分之一而城不拔者,此攻之灾也。
	
	故善用兵者,屈人之兵而非战也,拔人之城而非攻也,毁人之国而非久也,必以全争于天下。故兵不顿而利可全,此谋攻之法也。
	
	故用兵之法,十则围之,五则攻之,倍则分之,敌则能战之,少则能逃之,不若则能避之。故小敌之坚,大敌之擒也。
	
	夫将者,国之辅也。辅周,则国必强;辅隙,则国必弱。
	
	故君之所以患于军者三:不知军之不可以进而谓之进,不知军之不可以退而谓之退,是谓“縻军”;不知三军之事,而同三军之政者,则军士惑矣;不知三军之权,而同三军之任,则军士疑矣。三军既惑且疑,则诸侯之难至矣,是谓“乱军引胜”。
	
	故知胜有五:知可以战与不可以战者胜,识众寡之用者胜,上下同欲者胜,以虞待不虞者胜,将能而君不御者胜。此五者,知胜之道也。
	
	故曰:知彼知己者,百战不殆;不知彼而知己,一胜一负,不知彼,不知己,每战必殆。
	
	\chapter{军形篇}
	
	孙子曰:昔之善战者,先为不可胜,以待敌之可胜。不可胜在己,可胜在敌。故善战者,能为不可胜,不能使敌之可胜。故曰:胜可知,而不可为。
	
	不可胜者,守也;可胜者,攻也。守则不足,攻则有余(竹简为:守则有余,攻则不足)。善守者,藏于九地之下,善攻者,动于九天之上,故能自保而全胜也。
	
	见胜不过众人之所知,非善之善者也;战胜而天下曰善,非善之善者也。故举秋毫不为多力,见日月不为明目,闻雷霆不为聪耳。古之所谓善战者,胜于易胜者也。故善战者之胜也,无智名,无勇功。故其战胜不忒,不忒者,其所措必胜,胜已败者也。故善战者,立于不败之地,而不失敌之败也。是故胜兵先胜而后求战,败兵先战而后求胜。善用兵者,修道而保法,故能为胜败之政。
	
	兵法:一曰度,二曰量,三曰数,四曰称,五曰胜。地生度,度生量,量生数,数生称,称生胜。故胜兵若以镒称铢,败兵若以铢称镒。胜者之战民也,若决积水于千仞之溪者,形也。
	
	\chapter{兵势篇}
	
	孙子曰:凡治众如治寡,分数是也;斗众如斗寡,形名是也;三军之众,可使必受敌而无败者,奇正是也;兵之所加,如以碫投卵者,虚实是也。
	
	凡战者,以正合,以奇胜。故善出奇者,无穷如天地,不竭如江海。终而复始,日月是也。死而更生,四时是也。声不过五,五声之变,不可胜听也;色不过五,五色之变,不可胜观也;味不过五,五味之变,不可胜尝也;战势不过奇正,奇正之变,不可胜穷也。奇正相生,如循环之无端,孰能穷之哉!
	
	激水之疾,至于漂石者,势也;鸷鸟之疾,至于毁折者,节也。故善战者,其势险,其节短。势如扩弩,节如发机。纷纷纭纭,斗乱而不可乱;浑浑沌沌,形圆而不可败。乱生于治,怯生于勇,弱生于强。治乱,数也;勇怯,势也;强弱,形也。
	
	故善动敌者,形之,敌必从之;予之,敌必取之。以利动之,以卒待之。故善战者,求之于势,不责于人故能择人而任势。任势者,其战人也,如转木石。木石之性,安则静,危则动,方则止,圆则行。
	
	故善战人之势,如转圆石于千仞之山者,势也。
	
	\chapter{虚实篇}
	
	孙子曰:凡先处战地而待敌者佚,后处战地而趋战者劳,故善战者,致人而不致于人。能使敌人自至者,利之也;能使敌人不得至者,害之也,故敌佚能劳之,饱能饥之,安能动之。出其所不趋,趋其所不意。行千里而不劳者,行于无人之地也。
	
	攻而必取者,攻其所不守也;守而必固者,守其所不攻也。故善攻者,敌不知其所守;善守者,敌不知其所攻。微乎微乎,至于无形。神乎神乎,至于无声,故能为敌之司命。进而不可御者,冲其虚也;退而不可追者。速而不可及也。故我欲战,敌虽高垒深沟,不得不与我战者,攻其所必救也;我不欲战,画地而守之,敌不得与我战者,乖其所之也。
	
	故形人而我无形,则我专而敌分。我专为一,敌分为十,是以十攻其一也,则我众而敌寡;能以众击寡者,则吾之所与战者,约矣。吾所与战之地不可知,不可知,则敌所备者多;敌所备者多,则吾所与战者,寡矣。
	
	故备前则后寡,备后则前寡,备左则右寡,备右则左寡,无所不备,则无所不寡。寡者,备人者也;众者,使人备己者也。
	
	故知战之地,知战之日,则可千里而会战。不知战地,不知战日,则左不能救右,右不能救左,前不能救后,后不能救前,而况远者数十里,近者数里乎?
	
	以吾度之,越人之兵虽多,亦奚益于胜败哉?故曰:胜可为也。敌虽众,可使无斗。故策之而知得失之计,作之而知动静之理,形之而知死生之地,角之而知有余不足之处。故形兵之极,至于无形。无形,则深间不能窥,智者不能谋。因形而错胜于众,众不能知;人皆知我所以胜之形,而莫知吾所以制胜之形。故其战胜不复,而应形于无穷。
	
	夫兵形象水,水之形,避高而趋下,兵之形,避实而击虚。水因地而制流,兵因敌而制胜。故兵无常势,水无常形,能因敌变化而取胜者,谓之神。
	
	故五行无常胜,四时无常位,日有短长,月有死生。
	
	\chapter{军争篇}
	
	孙子曰:凡用兵之法,将受命于君,合军聚众,交和而舍,莫难于军争。军争之难者,以迂为直,以患为利。
	
	故迂其途,而诱之以利,后人发,先人至,此知迂直之计者也。军争为利,军争为危。举军而争利则不及,委军而争利则辎重捐。是故卷甲而趋,日夜不处,倍道兼行,百里而争利,则擒三将军,劲者先,疲者后,其法十一而至;五十里而争利,则蹶上将军,其法半至;三十里而争利,则三分之二至。是故军无辎重则亡,无粮食则亡,无委积则亡。故不知诸侯之谋者,不能豫交;不知山林、险阻、沮泽之形者,不能行军;不用乡导者,不能得地利。故兵以诈立,以利动,以分和为变者也。故其疾如风,其徐如林,侵掠如火,不动如山,难知如阴,动如雷震。掠乡分众,廓地分利,悬权而动。先知迂直之计者胜,此军争之法也。
	
	《军政》曰:“言不相闻,故为之金鼓;视不相见,故为之旌旗。”夫金鼓旌旗者,所以一民之耳目也。民既专一,则勇者不得独进,怯者不得独退,此用众之法也。故夜战多金鼓,昼战多旌旗,所以变人之耳目也。
	
	三军可夺气,将军可夺心。是故朝气锐,昼气惰,暮气归。善用兵者,避其锐气,击其惰归,此治气者也。以治待乱,以静待哗,此治心者也。以近待远,以佚待劳,以饱待饥,此治力者也。无邀正正之旗,勿击堂堂之阵,此治变者也。
	
	故用兵之法,高陵勿向,背丘勿逆,佯北勿从,锐卒勿攻,饵兵勿食,归师勿遏,围师遗阙,穷寇勿迫,此用兵之法也。
	
	\chapter{九变篇}
	
	孙子曰:凡用兵之法,将受命于君,合军聚众。圮地无舍,衢地交合,绝地无留,围地则谋,死地则战,途有所不由,军有所不击,城有所不攻,地有所不争,君命有所不受。
	
	故将通于九变之利者,知用兵矣;将不通九变之利,虽知地形,不能得地之利矣;治兵不知九变之术,虽知五利,不能得人之用矣。
	
	是故智者之虑,必杂于利害,杂于利而务可信也,杂于害而患可解也。是故屈诸侯者以害,役诸侯者以业,趋诸侯者以利。故用兵之法,无恃其不来,恃吾有以待之;无恃其不攻,恃吾有所不可攻也。
	
	故将有五危,必死可杀,必生可虏,忿速可侮,廉洁可辱,爱民可烦。凡此五者,将之过也,用兵之灾也。覆军杀将,必以五危,不可不察也。
	
	\chapter{行军篇}
	
	孙子曰:凡处军相敌:绝山依谷,视生处高,战隆无登,此处山之军也。绝水必远水;客绝水而来,勿迎之于水内,令半济而击之,利;欲战者,无附于水而迎客;视生处高,无迎水流,此处水上之军也。绝斥泽,惟亟去无留;若交军于斥泽之中,必依水草而背众树,此处斥泽之军也。平陆处易,而右背高,前死后生,此处平陆之军也。凡此四军之利,黄帝之所以胜四帝也。
	
	凡军好高而恶下,贵阳而贱阴,养生而处实,军无百疾,是谓必胜。丘陵堤防,必处其阳,而右背之。此兵之利,地之助也。
	
	上雨,水沫至,欲涉者,待其定也。
	
	凡地有绝涧、天井、天牢、天罗、天陷、天隙,必亟去之,勿近也。吾远之,敌近之;吾迎之,敌背之。
	
	军行有险阻、潢井、葭苇、山林、蘙荟者,必谨覆索之,此伏奸之所处也。
	
	敌近而静者,恃其险也;远而挑战者,欲人之进也;其所居易者,利也。
	
	众树动者,来也;众草多障者,疑也;鸟起者,伏也;兽骇者,覆也;尘高而锐者,车来也;卑而广者,徒来也;散而条达者,樵采也;少而往来者,营军也。
	
	辞卑而益备者,进也;辞强而进驱者,退也;轻车先出居其侧者,陈也;无约而请和者,谋也;奔走而陈兵车者,期也;半进半退者,诱也。
	
	杖而立者,饥也;汲而先饮者,渴也;见利而不进者,劳也;鸟集者,虚也;夜呼者,恐也;军扰者,将不重也;旌旗动者,乱也;吏怒者,倦也;粟马肉食,军无悬缻,不返其舍者,穷寇也;谆谆翕翕,徐与人言者,失众也;数赏者,窘也;数罚者,困也;先暴而后畏其众者,不精之至也;来委谢者,欲休息也。兵怒而相迎,久而不合,又不相去,必谨察之。
	
	兵非益多也,惟无武进,足以并力、料敌、取人而已。夫惟无虑而易敌者,必擒于人。
	
	卒未亲附而罚之,则不服,不服则难用也。卒已亲附而罚不行,则不可用也。故令之以文,齐之以武,是谓必取。令素行以教其民,则民服;令不素行以教其民,则民不服。令素行者,与众相得也。
	
	\chapter{地形篇}
	
	孙子曰:地形有通者,有挂者,有支者,有隘者,有险者,有远者。我可以往,彼可以来,曰通;通形者,先居高阳,利粮道,以战则利。可以往,难以返,曰挂;挂形者,敌无备,出而胜之;敌若有备,出而不胜,难以返,不利。我出而不利,彼出而不利,曰支;支形者,敌虽利我,我无出也;引而去之,令敌半出而击之,利。隘形者,我先居之,必盈之以待敌;若敌先居之,盈而勿从,不盈而从之。险形者,我先居之,必居高阳以待敌;若敌先居之,引而去之,勿从也。远形者,势均,难以挑战,战而不利。凡此六者,地之道也;将之至任,不可不察也。
	
	故兵有走者,有弛者,有陷者,有崩者,有乱者,有北者。凡此六者,非天之灾,将之过也。夫势均,以一击十,曰走;卒强吏弱,曰弛,吏强卒弱,曰陷;大吏怒而不服,遇敌怼而自战,将不知其能,曰崩;将弱不严,教道不明,吏卒无常,陈兵纵横,曰乱;将不能料敌,以少合众,以弱击强,兵无选锋,曰北。凡此六者,败之道也;将之至任,不可不察也。
	
	夫地形者,兵之助也。料敌制胜,计险厄远近,上将之道也。知此而用战者必胜,不知此而用战者必败。
	
	故战道必胜,主曰无战,必战可也;战道不胜,主曰必战,无战可也。故进不求名,退不避罪,唯人是保,而利合于主,国之宝也。
	
	视卒如婴儿,故可与之赴深溪;视卒如爱子,故可与之俱死。厚而不能使,爱而不能令,乱而不能治,譬若骄子,不可用也。
	
	知吾卒之可以击,而不知敌之不可击,胜之半也;知敌之可击,而不知吾卒之不可以击,胜之半也;知敌之可击,知吾卒之可以击,而不知地形之不可以战,胜之半也。故知兵者,动而不迷,举而不穷。故曰:知彼知己,胜乃不殆;知天知地,胜乃不穷。
	
	\chapter{九地篇}
	
	孙子曰:用兵之法,有散地,有轻地,有争地,有交地,有衢地,有重地,有圮地,有围地,有死地。诸侯自战其地,为散地。入人之地不深者,为轻地。我得则利,彼得亦利者,为争地。我可以往,彼可以来者,为交地。诸侯之地三属,先至而得天下之众者,为衢地。入人之地深,背城邑多者,为重地。行山林、险阻、沮泽,凡难行之道者,为圮地。所由入者隘,所从归者迂,彼寡可以击吾之众者,为围地。疾战则存,不疾战则亡者,为死地。是故散地则无战,轻地则无止,争地则无攻,交地则无绝,衢地则合交,重地则掠,圮地则行,围地则谋,死地则战。
	
	所谓古之善用兵者,能使敌人前后不相及,众寡不相恃,贵贱不相救,上下不相收,卒离而不集,兵合而不齐。合于利而动,不合于利而止。敢问:“敌众整而将来,待之若何?”曰:“先夺其所爱,则听矣。”
	
	兵之情主速,乘人之不及,由不虞之道,攻其所不戒也。
	
	凡为客之道:深入则专,主人不克;掠于饶野,三军足食;谨养而勿劳,并气积力,运兵计谋,为不可测。投之无所往,死且不北,死焉不得,士人尽力。兵士甚陷则不惧,无所往则固。深入则拘,不得已则斗。是故其兵不修而戒,不求而得,不约而亲,不令而信,禁祥去疑,至死无所之。吾士无余财,非恶货也;无余命,非恶寿也。令发之日,士卒坐者涕沾襟。偃卧者涕交颐。投之无所往者,诸、刿之勇也。
	
	故善用兵者,譬如率然;率然者,常山之蛇也。击其首则尾至,击其尾则首至,击其中则首尾俱至。敢问:“兵可使如率然乎?”曰:“可。”夫吴人与越人相恶也,当其同舟而济,遇风,其相救也如左右手。是故方马埋轮,未足恃也;齐勇若一,政之道也;刚柔皆得,地之理也。故善用兵者,携手若使一人,不得已也。
	
	将军之事:静以幽,正以治。能愚士卒之耳目,使之无知。易其事,革其谋,使人无识;易其居,迂其途,使人不得虑。帅与之期,如登高而去其梯;帅与之深入诸侯之地,而发其机,焚舟破釜,若驱群羊,驱而往,驱而来,莫知所之。聚三军之众,投之于险,此谓将军之事也。九地之变,屈伸之利,人情之理,不可不察。
	
	凡为客之道:深则专,浅则散。去国越境而师者,绝地也;四达者,衢地也;入深者,重地也;入浅者,轻地也;背固前隘者,围地也;无所往者,死地也。
	
	是故散地,吾将一其志;轻地,吾将使之属;争地,吾将趋其后;交地,吾将谨其守;衢地,吾将固其结;重地,吾将继其食;圮地,吾将进其涂;围地,“吾将塞其阙;死地,吾将示之以不活。
	
	故兵之情,围则御,不得已则斗,过则从。是故不知诸侯之谋者,不能预交;不知山林、险阻、沮泽之形者,不能行军;不用乡导者,不能得地利。四五者,不知一,非霸王之兵也。夫霸王之兵,伐大国,则其众不得聚;威加于敌,则其交不得合。是故不争天下之交,不养天下之权,信己之私,威加于敌,故其城可拔,其国可隳。施无法之赏,悬无政之令,犯三军之众,若使一人。犯之以事,勿告以言;犯之以利,勿告以害。
	
	投之亡地然后存,陷之死地然后生。夫众陷于害,然后能为胜败。
	
	故为兵之事,在于顺详敌之意,并敌一向,千里杀将,此谓巧能成事者也。
	
	是故政举之日,夷关折符,无通其使;厉于廊庙之上,以诛其事。敌人开阖,必亟入之。先其所爱,微与之期。践墨随敌,以决战事。是故始如处女,敌人开户,后如脱兔,敌不及拒。
	
	\chapter{火攻篇}
	
	孙子曰:凡火攻有五:一曰火人,二曰火积,三曰火辎,四曰火库,五曰火队。行火必有因,烟火必素具。发火有时,起火有日。时者,天之燥也;日者,月在箕、壁、翼、轸也。凡此四宿者,风起之日也。
	
	凡火攻,必因五火之变而应之。火发于内,则早应之于外。火发兵静者,待而勿攻,极其火力,可从而从之,不可从而止。火可发于外,无待于内,以时发之。火发上风,无攻下风。昼风久,夜风止。凡军必知有五火之变,以数守之。
	
	故以火佐攻者明,以水佐攻者强。水可以绝,不可以夺。夫战胜攻取,而不修其功者凶,命曰费留。故曰:明主虑之,良将修之。非利不动,非得不用,非危不战。主不可以怒而兴师,将不可以愠而致战;合于利而动,不合于利而止。怒可以复喜,愠可以复悦;亡国不可以复存,死者不可以复生。故明君慎之,良将警之,此安国全军之道也。
	
	\chapter{用间篇}
	
	孙子曰:凡兴师十万,出征千里,百姓之费,公家之奉,日费千金;内外骚动,怠于道路,不得操事者,七十万家。相守数年,以争一日之胜,而爱爵禄百金,不知敌之情者,不仁之至也,非人之将也,非主之佐也,非胜之主也。故明君贤将,所以动而胜人,成功出于众者,先知也。先知者,不可取于鬼神,不可象于事,不可验于度,必取于人,知敌之情者也。
	
	故用间有五:有因间,有内间,有反间,有死间,有生间。五间俱起,莫知其道,是谓神纪,人君之宝也。因间者,因其乡人而用之。内间者,因其官人而用之。反间者,因其敌间而用之。死间者,为诳事于外,令吾间知之,而传于敌间也。生间者,反报也。
	
	故三军之事,莫亲于间,赏莫厚于间,事莫密于间。非圣智不能用间,非仁义不能使间,非微妙不能得间之实。微哉!微哉!无所不用间也。间事未发,而先闻者,间与所告者皆死。(莫亲于间:指没有比间谍更应成为亲信了。赏莫厚于间:指没有比间谍更应该得到丰富的奖赏了。事莫密于间:没有经间谍的事更应该保守机密了。间事未发:用间之事还没有开始进行。间与所告者皆死:间谍和告知用间之事的人都要处死。)
	
	凡军之所欲击,城之所欲攻,人之所欲杀,必先知其守将,左右,谒者,门者,舍人之姓名,令吾间必索知之。
	
	必索敌人之间来间我者,因而利之,导而舍之,故反间可得而用也。因是而知之,故乡间、内间可得而使也;因是而知之,故死间为诳事,可使告敌。因是而知之,故生间可使如期。五间之事,主必知之,知之必在于反间,故反间不可不厚也。
	
	昔殷之兴也,伊挚在夏;周之兴也,吕牙在殷。故惟明君贤将,能以上智为间者,必成大功。此兵之要,三军之所恃而动也。	

\end{document}