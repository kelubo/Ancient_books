%-*- coding: UTF-8 -*-
% 孙子兵法
% 孙子兵法.tex

\documentclass[a4paper,12pt,UTF8,twoside]{ctexbook}

% 设置纸张信息。
\RequirePackage[a4paper]{geometry}
\geometry{
	%textwidth=138mm,
	%textheight=215mm,
	%left=27mm,
	%right=27mm,
	%top=25.4mm, 
	%bottom=25.4mm,
	%headheight=2.17cm,
	%headsep=4mm,
	%footskip=12mm,
	%heightrounded,
	inner=1in,
	outer=1.25in
}

% 设置字体,并解决显示难检字问题。
\xeCJKsetup{AutoFallBack=true}
\setCJKmainfont{SimSun}[BoldFont=SimHei, ItalicFont=KaiTi, FallBack=SimSun-ExtB]

% 目录 chapter 级别加点(.)。
\usepackage{titletoc}
\titlecontents{chapter}[0pt]{\vspace{3mm}\bf\addvspace{2pt}\filright}{\contentspush{\thecontentslabel\hspace{0.8em}}}{}{\titlerule*[8pt]{.}\contentspage}

% 设置 part 和 chapter 标题格式。
\ctexset{
	chapter/name={第},
	chapter/number={\chinese{chapter}}
}

% 设置古文原文格式。
\newenvironment{yuanwen}{\bfseries\zihao{4}}

\title{\heiti\zihao{0} 孙子兵法}
\author{孙武}
\date{}

\begin{document}
	
	\maketitle
	\tableofcontents

	\frontmatter
    \chapter{前言}
    
    2. 注释
    
    3. 译文
    
    4. 拼音修改。
    
    5. 前言增加
    
    《孙子兵法》,俗称为《孙子》,《汉书·艺文志》著录为《吴孙子兵法》。是现存我国历史上的第一部兵书。

    \mainmatter
	\chapter{(始)计篇}
	
	《说文》云:计,会也,筭\footnote{suan4}(算)也。“筭”是一种原始的计数工具,与筹、策是同类性质的东西。古代出兵打仗之前都要先在庙堂上用这种工具计算敌我优势,叫“庙算”。战争是关系到国家生死存亡的大事,所以在出兵作战前,一定要认真分析敌我作战的基本条件。本篇以“计”名篇,论述了决定战争胜败的五个基本原则。最后着重指出,对站前的作战计划、作战意图必须深思熟虑。

	\begin{yuanwen}
	孙子曰:兵\footnote{指战争。}者,国之大事,死生之地,存亡之道,不可不察\footnote{考察,研究。}也。
    \end{yuanwen}
    
    孙子说:战争是国家的大事,地形上的死地、生地,战场上的存亡胜败,是不能不慎重分析研究的。
    
    \begin{yuanwen}
	故经\footnote{衡量。这里是分析研究的意思。}之以五事,校\footnote{jiao4,比较。}之以计,而索\footnote{探索。}其情\footnote{指实际情况。}。一曰道\footnote{道义、品德。这里指在政治方面是否得民心。},二曰天\footnote{指天时(气象、时令)方面的条件。},三曰地\footnote{指地理条件。如距离远近、险要平坦、广阔狭隘等地形、地势。},四曰将\footnote{将令。这里指统率作战的将帅的谋略、智能以及指挥士兵作战的方法等。},五曰法\footnote{军法、法令。这里指军队的编制、将帅的职掌、军备物资的供给等。}。道者,令民与上\footnote{指国君。}同意\footnote{同心。}也,故可以与之死,可以与之生,而不畏危\footnote{不惧怕危险。}也。天者,阴阳\footnote{这里指昼夜、晴雨等气象变化。}、寒暑、时制也\footnote{指春、夏、秋、冬四季。}。地者,远近、险易\footnote{地势的险阻平坦。}、广狭、死生\footnote{这里指死地与生地,即地形是否有利于攻守进退。}也。将者,智、信、仁、勇、严也。法者,曲制\footnote{指军队编制制度。}、官道\footnote{指各级将吏的职责划分以及统辖管理制度。}、主\footnote{掌管。}用\footnote{物资费用。}\footnote{掌管物资费用的后勤管理制度。}也。凡此五者,将莫不闻\footnote{知道。},知之者胜,不知者不胜。
	\end{yuanwen}
	
	所以,要通过以下五个方面来研究,比较分析双方的各种条件,考察双方的实际情况,来预测战争胜负的可能性。一是道义,二是天时,三是地理,四是将帅,五是法规。所谓“道义”,就是使民众和君主的心意相同,这样才可以同生共死,而不惧怕危险。所谓“天时”,就是指阴阳、寒暑、春夏秋冬四季。所谓“地形”,就是指路程的远近、地势的险要平坦、战场的广阔狭窄、是生地还是死地等地理条件。所谓“将帅”,就是指将帅的智谋才能、赏罚有信、对部下仁慈关爱、果断勇敢、军纪严明。所谓“法规”,就是指军队组织的编制、将吏责权的划分、军需物资的掌管和供给。对于这五个方面,身为将领要深刻了解。了解的就能胜利,否则就不能胜利。

	\begin{yuanwen}
	故校之以计,而索其情。曰:主孰\footnote{疑问代词,谁,哪一方。}有道?将孰有能?天地孰得?法令孰行?兵众孰强?士卒孰练?赏罚孰明?吾以此知胜负矣。
    \end{yuanwen}
	
	所以,要通过双方的考察分析,掌握实际情况,并据此加以比较,从而来预测战争胜负的情形。就要问:哪一方的君主能得民心?哪一方的将领更有能力?哪一方占有天时地利?哪一方的法规、法令更能严格执行?哪一方的兵力更强大?哪一方的士卒训练更加有素?哪一方的赏罚更公正严明?我根据这些分析比较,就可以判明双方的胜负了。
	
	\begin{yuanwen}
	将\footnote{虚词,表示假设。}听吾计,用之必胜,留之;将不听吾计,用之必败,去\footnote{离开。}之。
	\end{yuanwen}
	
	如果采用我的计策,指挥作战必胜,我就留下;如果不采用我的计策,指挥作战必败,我就离开。
	
	\begin{yuanwen}
	计利\footnote{分析后有优势,有利于我。}以听\footnote{听从,采纳。},乃为之势\footnote{含有态势之意。如战略形势、战术态势、战场优势等。它与“形”相反,多指随机的、能动的东西,如指挥的灵活、士气的勇怯等。},以佐\footnote{辅助。}其外\footnote{指国境之外。古时用兵多在境外,如《管子·七法》里说:“故凡攻伐以为道也,计必先定于内,然后出兵乎境。计未定于内而兵出乎境,是则战之自胜,攻之自毁也。”}。势者,因利而制权\footnote{权变,灵活处置。}\footnote{指根据实际利害关系而灵活应变。}也。
	\end{yuanwen}
	
	筹划有利并且能得到执行,还要设法造“势”,来协助在外的军事行动。所谓“势”,就是根据实际利害关系而采取相应的措施。
	
	\begin{yuanwen}
	兵者,诡\footnote{诡秘、诡诈。意思是说用兵打仗是一种诡诈的行为。}道也。故能而示\footnote{显示、表示。这里也含有伪装的意思。}之不能,用而示之不用,近而示之远,远而示之近。利而诱之,乱而取\footnote{指攻取。}之,实而备之,强而避之,怒而挠\footnote{打扰,扰乱。}之,卑\footnote{这里可理解为卑弱而谨慎。}而骄之,佚\footnote{通“逸”,安逸、安稳。这里是指休整充分的意思。}而劳\footnote{使动用法,使疲劳。}之,亲而离\footnote{离间。}之。攻其无备,出其不意。此兵家之胜,不可先传\footnote{传授,在这里可引申为“规定”。}也。
	\end{yuanwen}
	
	用兵作战,就是一种诡诈的行为。因此,能攻却要装出不能攻,要打却要装出不去打;欲从近处攻打却要装出从远处攻打,欲从远处攻打却要装出从近处攻打;对方贪利就要用利益来诱惑他,对方混乱就要趁机攻取他;对方充实就要防备他,对方强大就要躲避他;丢分暴躁易恼怒就要骚扰他,对方自卑谨慎就要使他骄傲自大;对方休整充分就要使其劳累,对方内部团结就要设法离间他。攻打对方没有防备之处,在对方没有料到的时机发动进攻。这些都是军事家克敌制胜的诀窍,要在战争中根据实际情况灵活应用,不可能事先作出死板的规定。
	
	\begin{yuanwen}
	夫未战而庙算\footnote{古时出兵作战之前,都要到宗庙里举行仪式,商讨作战计划,这就叫“庙算”。}胜者,得算\footnote{本指计数用的筹码,这里引申指取得胜利的条件。}多\footnote{得算多,指取得胜利的条件多。}也;未战而庙算不胜者,得算少也。多算胜,少算不胜,而况于无算乎!吾以此观之,胜负见矣。
	\end{yuanwen}
	
	还没有出兵交战,就在“庙算”上先已获胜,是由于得到的“算筹”较多。还没有出兵交战,就在“庙算”上先已失败,是由于得到的“算筹”较少。得到“算筹”较多的就可能取胜,得到“算筹”较少的就不会取胜,更何况那些没有得到“算筹”的呢?我根据“庙算”的结果来观察,胜负之分就显而易见了。
	
	\chapter{作战篇}
	
	本篇主要从战争对人力、物力、财力的依赖关系出发,论述了“持久作战”会给国家带来危害的观点,提出了“兵贵胜,不贵久”的速战速决的军事思想。任何一场战争都是双方军事实力的较量,而军事实力所依赖的则是综合国力的强弱。军队长期在外作战,国家的财力终会枯竭,长期消耗,必然会导致战争的失败。为减轻作战的负担,进一步提出“取用于国,因粮于敌”的主张,减少远程运输,节约作战开支,这在当时来说是一种很了不起的军事思想。
	
	\begin{yuanwen}
	孙子曰:凡用兵之法,驰车\footnote{一种轻型战车。}千驷\footnote{古代四马拉一车为“驷”。},革车\footnote{古代运载辎重的战车。}千乘\footnote{辆。},带甲\footnote{指用甲胄武装起来的士卒。}十万,千里馈\footnote{运送。}粮,则内外之费,宾客之用,胶漆之材\footnote{泛指制作和维护作战器械所需的材料。},车\footnote{车辆。}甲\footnote{盔甲。}\footnote{这里泛指各种军事装备。}之奉\footnote{供给,补充。},日费千金,然后十万之师举\footnote{出兵作战。}矣。
	\end{yuanwen}
	
	孙子说:凡兴兵作战,需要出动轻车千辆,重车千乘,兵士十万,并千里迢迢运送粮食。这样一来,前方后方的各种开支,招待宾客策士的费用,物资器材,战车、甲胄的供给等,每天要花费千金,之后十万大军方才出兵作战。
	
	\begin{yuanwen}
	其用战\footnote{用兵作战。}也胜,久则钝兵\footnote{钝了刃的刀。}挫锐\footnote{挫了尖的矛。}\footnote{这里比喻军队疲惫,锐气挫伤。},攻城则力屈\footnote{指战斗力衰竭。},久暴\footnote{pu4}师\footnote{军队长期在外作战。}则国用\footnote{国家的开支。}不足。夫钝兵挫锐,屈力殚\footnote{枯竭。}货\footnote{财货。}\footnote{耗尽物力财力。},则诸侯乘其弊\footnote{指疲惫。}而起,虽有智者,不能善其后矣。故兵闻拙\footnote{笨拙。}速,未睹巧\footnote{巧妙。}之久也。夫兵久\footnote{指作战时间长。}而国利\footnote{对国家有利。}者,未之有也。故不尽知用兵之害者,则不能尽知用兵之利也。
	\end{yuanwen}
	
	用这样庞大的军队作战务求速胜,持久就会使军队疲惫,锐气挫伤。攻城就会耗尽兵力,军队长期在外作战,必然导致国家财用不足。如果军队疲惫不堪、锐气受挫、军队实力耗尽、国内资源枯竭,那么诸侯就会乘机向我发起进攻,即使有智谋之士也无法挽救如此危局。所以在实际作战中只听说过宁拙而求速胜,没见过求巧而久拖的。战争旷日持久而对国家有利的,从来没有过。所以,不能完全了解用兵危害的将领,就不能完全了解用兵的益处。

	\begin{yuanwen}
	善用兵者,役不再籍\footnote{指户籍。古代按户籍征兵。},粮不三载\footnote{运输。};取用于国,因\footnote{依靠。}粮于敌,故军食可足也。
	\end{yuanwen}
	
	善于用兵的人,不用再三从国内征兵,不用再三从国内运粮。武器装备由国内供应,从敌人那里夺取粮食,这样,军队的粮草就可以充足了。
	
	\begin{yuanwen}
	国之贫于师者远输\footnote{远道运输。},远输则百姓贫。近于师者贵卖,贵卖则百姓财竭,财竭则急于丘役\footnote{按丘征集的赋税徭役。丘,是古代的地方行政单位。据《周礼》,在古代九家为井,四井为邑,四邑为丘。}。力屈、财殚,中原\footnote{这里指国内。}内虚于家\footnote{text}。百姓之费,十去其七;公家之费,破车罢\footnote{pi2,通“疲”。}马,甲胄矢弩,戟楯\footnote{同“盾”。}蔽橹\footnote{大盾牌。},丘牛\footnote{指从“丘”征集来的牛。}大车\footnote{指辎重车。},十去其六。
	\end{yuanwen}
	
	国家因作战而贫困,是由于军队远途运输,远途运输就会导致百姓贫穷。靠近驻军的地方物价必然会暴涨,物价暴涨就会导致百姓的财物枯竭,财物枯竭国家就会急于征集赋税和劳役。军力耗尽,财源枯竭,国内空虚。百姓的私家财产损耗掉十分之七;公家的财产,由于战车破损,战马疲惫,甲胄、弓箭、矛戟、盾牌、拉辎重的牛车也损耗掉十分之六。
	
	\begin{yuanwen}
	故智将务\footnote{必须。}食于敌。食敌一钟\footnote{古代的容量单位。六十四斗为一钟。},当吾二十钟;芑\footnote{qi2,通“萁”,豆秸。}秆\footnote{禾茎。}一石\footnote{古代的重量单位。一百二十斤为一石。},当吾二十石。
	\end{yuanwen}
	
	所以明智的将军一定要靠敌国解决粮草,从敌国搞到一钟的粮食就相当于从本国运来二十钟,在当地取得草料一石,就相当于从本国运来二十石。

	\begin{yuanwen}
	故杀敌者,怒\footnote{指激起士兵对敌人的愤怒。}也;取敌之利者,货\footnote{财货。指用来奖赏士兵的财物。}也。故车战得车十乘已上,赏其先得者。而更\footnote{更换。}其旌旗\footnote{旗帜。},车杂\footnote{混合,搀杂。}而乘之,卒善而养之,是谓胜敌而益强。
	\end{yuanwen}
	
	所以,要使士兵拼死杀敌,就必须激起士兵对敌人的愤怒。要使士兵勇于夺取敌方的军需物资,就必须用缴获的财物来奖赏士兵。所以,在车战中,凡夺取敌军战车十辆以上的,就奖赏最先夺取战车的人。而夺得的战车要立即换去上面的旗帜,编入我方车队而夹杂在一起乘用。对俘虏来的士卒要优待他们、供养他们,这就是所谓战胜敌人而使自己日益强大的原因。

	\begin{yuanwen}
	故兵贵胜,不贵久。故知兵之将,生民之司命,国家安危之主\footnote{主宰。}也。
    \end{yuanwen}
	
	所以,用兵作战贵在速胜,最不宜的是旷日持久。深知用兵之道的将帅,是民众命运的掌握者,是国家安危的主宰者。
	
	\chapter{谋攻篇}
	
	“谋攻”就是用计谋来征服敌人。孙武认为理想的作战结果是“全国为上”、“全军为上”、“全旅为上”、“劝阻微商”、“全伍为上”,“破国”、“破军”、“破旅”、“破卒”、“破伍”则次之。但最理想的作战结果是“不战而屈人之兵”。如何才能达到最佳的作战效果?就是要用计谋去战胜敌人。本篇主要论述了“上兵伐谋”的思想、国君与将帅的关系、致胜的条件和致败的原因,最后提出了著名的“知己知彼,百战不殆”的思想,作战的谋略必须建立在了解敌我双方情况的基础上。
	
	\begin{yuanwen}
	孙子曰:凡用兵之法,全国为上,破国次之;全军\footnote{古代军队的编制单位。据《周礼》,一万两千五百人为一军。}为上,破军次之;全旅\footnote{古代军队的编制单位。据《周礼》,五百人为一旅。}为上,破旅次之;全卒\footnote{古代军队的编制单位。据《周礼》,百人为一卒。}为上,破卒次之;全伍\footnote{古代军队的编制单位。据《周礼》,五人为一伍。}为上,破伍次之。是故百战百胜,非善之善者也;不战而屈人之兵,善之善者也。
	\end{yuanwen}
	
	孙子说:大凡作战的原则是:使整个敌国屈服是上策,用武力攻破敌国使之屈服就差一些;使敌人全军降服是上策,击破敌军就差一些;使敌人全旅降服是上策,击破敌旅就差一些;使敌人全卒降服是上策,击破敌卒就差一些;使敌人全伍降服是上策,击破敌伍就差一些。所以说,百战百胜,算不上是好中之最好的;不通过交战就使敌人的军队降服,这才是好中之最好的。

	\begin{yuanwen}
	故上兵\footnote{最好的军事手段。}伐谋\footnote{用谋略讨伐。},其次伐交\footnote{用外交手段去讨伐。},其次伐兵\footnote{用武力去讨伐。},其下攻城。攻城之法为不得已。修\footnote{制造。}橹\footnote{古代的一种攻城工具,即“楼橹”。}轒辒\footnote{fen2,古代攻城用的一种四轮车具。},具\footnote{准备。}器械,三月而后成;距闉\footnote{yin1,是指堆积攻城用的土山。闉,通“堙”,土山。},又三月而后已。将不胜其忿而蚁附\footnote{指士兵像蚂蚁一样的爬城。}之,杀士\footnote{指士卒伤亡。}三分之一而城不拔\footnote{指城被攻下。}者,此攻之灾也。
	\end{yuanwen}
	
	所以,好中之最好的军事行动是用谋略挫败敌人,其次就是用外交手段战胜敌人,再次就是用武力击败敌军,最下之策是攻打敌人的城池。攻城是迫不得已采取的方法。制造橹、轒辒等各种攻城工具,准备所有的攻城器械,三个月才能完成。堆筑攻城的土山,又得三个月才能完成。如果将领难以抑制焦躁情绪,命令士兵像蚂蚁一样爬梯攻城,尽管士兵死伤三分之一,而城池仍然攻不下来,这就是攻城所带来的灾难。

	\begin{yuanwen}
	故善用兵者,屈人之兵而非战\footnote{指不用交战的方法。}也,拔人之城而非攻\footnote{指不用强攻的方法。}也,毁人之国而非久\footnote{指战争不要旷日持久。}也,必以全争于天下。故兵不顿\footnote{通“钝”,疲惫,受挫。}而利可全,此谋攻之法也。
	\end{yuanwen}
	
	所以善于用兵打仗的人,不通过打仗就使敌人屈服,不通过攻城就使敌城投降,摧毁敌国不需长期作战;一定要用“全胜”的策略争胜于天下,这样既不使国力兵力疲惫,又获得了全面胜利的利益,这就是谋攻的法则。

	\begin{yuanwen}
	故用兵之法,十则围之,五则攻之,倍则分之,敌\footnote{指与敌人兵力相等,势均力敌。}则能战之,少则能逃\footnote{摆脱,逃离。}之,不若\footnote{指条件不如敌人。}则能避之。故小敌之坚,大敌之擒也。
	\end{yuanwen}
	
	所以,用兵作战的原则是:十倍于敌就围歼敌人,五倍于敌就进攻敌人,一倍于敌就要设法分散敌人,势均力敌就要设法战胜敌人,兵力少于敌就设法摆脱敌人,如果各种条件不如敌人就要避免作战。所以,弱小的军队如果坚持硬拼,那就会被强大的敌人所俘虏。

	\begin{yuanwen}
	夫将者,国\footnote{指国君。}之辅\footnote{辅佐。}也。辅周\footnote{辅佐周到。},则国必强;辅隙\footnote{辅佐有漏洞、缺陷。},则国必弱。
	\end{yuanwen}
	
	将帅是国君的辅佐,辅佐得缜密周祥,国家就必然会强大,辅佐得有疏漏失当,国家就必然会衰弱。

	\begin{yuanwen}
	故君之所以患\footnote{危害,贻害。}于军者三:不知军之不可以进而谓\footnote{告诉。这里有命令的意思。}之进,不知军之不可以退而谓之退,是谓縻\footnote{mi2,羁縻,牵制。这里是指军队受到束缚。}军;不知三军之事,而同\footnote{参与。这里有干涉的意思。}三军之政\footnote{行政。指军队的工作。}者,则军士惑矣;不知三军之权\footnote{权谋,权变。},而同三军之任\footnote{指挥。},则军士疑矣。三军既惑且疑,则诸侯之难至矣,是谓乱军\footnote{把自己的军心扰乱。}引胜\footnote{导致敌人胜利。}。
	\end{yuanwen}
	
	所以,国君给军队造成的危害有三种:不知道军队不可以前进而下令前进,不知道军队不可以后退而下令后退,这叫束缚牵制军队;不懂得三军战守之事而要参与和干涉三军之政,将士们就会迷惑而无所适从;不懂得三军战略战术的权宜变化而要参与和干涉三军的指挥,将士们就会产生疑虑。军队既无所适从,又疑虑重重,各诸侯就会趁机兴兵作难。这就会把自己的军心扰乱而导致敌人胜利。

	\begin{yuanwen}
	故知胜有五:知可以战与不可以战者胜,识众寡\footnote{指军队力量配备的多少。}之用者胜,上下同欲\footnote{指同心、齐心。}者胜,以虞\footnote{准备。}待不虞者胜,将能而君不御\footnote{驾御。这里引申为牵制、干预的意思。}者胜。此五者,知胜之道也。
	\end{yuanwen}
	
	所以,有五个方面可以预见胜利:能够准确判断仗能打或不能打的会取得胜利;能够知道根据敌我双方而配备兵力的会取得胜利;全军上下同心协力的会取得胜利;有充分准备的对付毫无准备的会取得胜利;将领的才能精通军事、精于权变而君主又不加干涉的会取得胜利。以上五条就是预见胜利的方法。

	\begin{yuanwen}
	故曰:知彼知己者,百战不殆\footnote{危险,失败。};不知彼而知己,一胜一负;不知彼,不知己,每战必殆。
    \end{yuanwen}
    
	所以说:了解对方也了解自己,每次战斗都不会失败;不了解对方只了解自己,就可能胜负各占一半;既不了解对方又不了解自己,那就会每战必败。
	
	\chapter{军形篇}
	
	该篇以“形”字命题,所谓“形”,与下篇所论的“势”是一对矛盾的概念。据《汉书·艺文志·兵书略》中记载,任宏当年论次兵书有四种,其中第二种为“形势”,而“形势”则属于战术学的范畴,其特征是“雷动风举,后发而先至,离合背乡,变化无常,以轻疾制敌者也”。任宏所说的“形势”是一个合成词,“形”和“势”似乎无别,都是指人为造成的态势。但在《孙子兵法》中“形”和“势”是有明显区别的。“形”含有形象、形体等义,是指战争中客观、经常、易见的诸多因素。从本篇内容来看,主要是指军事的实力、力量的强弱等。所以本篇反复论述的内容主要是“胜可知而不可为”,“故善战者立于不败之地,而不失敌之败也”。并且把战争的物资准备(“地生度,度生量,量生数,数生称,称生胜”)看作是取胜的根本条件。决定战争胜利的因素,主要是军事实力和战略战术的谋划。善战者首先是创造出不被敌人打败的条件,然后再伺机打败敌人。
	
	\begin{yuanwen}
	孙子曰:昔之善战者,先为不可胜\footnote{不可战胜。},以待敌之可胜\footnote{可战胜。}。不可胜在己,可胜在敌。故善战者,能为不可胜,不能使敌之可胜。故曰:胜可知而不可为\footnote{指在条件不具备的情况下不能硬做。}。
	\end{yuanwen}
	
	孙子说:过去善于用兵作战的人,总是首先创造自己不可战胜的条件,然后等待可以战胜敌人的机会。使不被敌人战胜的主动权掌握在自己手中,能否战胜敌人,在于敌人是否给以可乘之机。所以,善于作战的人能做到自己不被敌人战胜,却不能做到是敌人一定会为我所胜。从这个意义上说,胜利可以预见,但在条件不具备的情况下不能强为。

	\begin{yuanwen}
	不可胜者,守\footnote{采取防守。}也;可胜者,攻\footnote{采取进攻。}也。守则不足,攻则有余\footnote{竹简为:守则有余,攻则不足。}。善守者,藏于九地之下\footnote{是说隐藏的深不可知。古人常用“九”来表示数的极点。},善攻者,动于九天\footnote{极言高不可及。}之上,故能自保而全胜也。
	\end{yuanwen}

	\begin{yuanwen}
	见胜不过\footnote{指没有超过。}众人之所知,非善之善者也;战胜而天下曰善,非善之善者也。故举秋毫\footnote{本指秋天鸟兽的细毛。这里比喻非常细微的东西。}不为多力,见日月不为明目\footnote{眼睛明亮。},闻雷霆不为聪耳\footnote{指耳朵灵。}。古之所谓善战者,胜于易胜者\footnote{指容易被战胜的人。}也。故善战者之胜也,无智名\footnote{指有智慧的名声。},无勇功\footnote{指勇敢杀敌的名声。}。故其战胜不忒\footnote{te4,没有差错。}。不忒者,其所措\footnote{指作战措施。}必胜,胜已败者\footnote{指已经处于失败地位的敌人。}也。故善战者,立于不败之地,而不失敌之败\footnote{指使敌人致败的时机。}也。是故胜兵先胜\footnote{指先造成的取胜条件。}而后求战,败兵先战而后求胜。善用兵者,修道而保法,故能为胜败之政\footnote{这里指主宰战争的胜负。}。
	\end{yuanwen}

	\begin{yuanwen}
	兵法:一曰度\footnote{text},二曰量\footnote{text},三曰数\footnote{text},四曰称\footnote{text},五曰胜\footnote{text}。地生度,度生量,量生数,数生称,称生胜。故胜兵若以镒称铢\footnote{text},败兵若以铢称镒。胜者之战民\footnote{text}也,若决积水于千仞\footnote{text}之谿者,形\footnote{text}也。
    \end{yuanwen}
	
	\chapter{(兵)势篇}
	
	\begin{yuanwen}
	孙子曰:凡治\footnote{text}众如治寡,分数\footnote{text}是也;斗众如斗寡,形名\footnote{text}是也;三军之众,可使毕受敌\footnote{text}而无败者,奇正\footnote{text}是也;兵之所加,如以碫\footnote{text}投卵\footnote{text}\footnote{text}者,虚实\footnote{text}是也。
	\end{yuanwen}

	\begin{yuanwen}
	凡战者,以正合\footnote{text},以奇胜。故善出奇者,无穷如天地,不竭\footnote{text}如江河。终而复始,日月是也。死而复生,四时是也。声\footnote{text}不过五,五声之变,不可胜\footnote{text}听也;色\footnote{text}不过五,五色之变,不可胜观也;味\footnote{text}不过五,五味之变,不可胜尝也;战势\footnote{text}不过奇正,奇正之变,不可胜穷也。奇正相生,如循环之无端\footnote{text},孰能穷之?
	\end{yuanwen}

	\begin{yuanwen}
	激水之疾\footnote{text},至于漂石者,势也;鸷鸟\footnote{text}之疾,至于毁折者,节\footnote{text}也。故善战者,其势险,其节短。势如彍弩\footnote{text},节如发机\footnote{text}。
	\end{yuanwen}
	
	\begin{yuanwen}
	纷纷纭纭\footnote{text},斗乱\footnote{text}而不可乱也;浑浑沌沌\footnote{text},形圆\footnote{text}而不可败也。乱生于治,怯生于勇,弱生于强。治乱,数\footnote{text}也;勇怯,势\footnote{text}也;强弱,形也。故善动\footnote{text}敌者,形\footnote{text}之,敌必从之;予\footnote{text}之,敌必取之。以利动之,以卒\footnote{text}待之。
	\end{yuanwen}

	\begin{yuanwen}
	故善战者,求之于势,不责\footnote{text}于人,故能择人而任势。任势者,其战人\footnote{text}也如转木石。木石之性,安则静,危\footnote{text}则动,方则止,圆则行。故善战人之势,如转圆石于千仞之山者,势\footnote{text}也。
	\end{yuanwen}
	
	\chapter{虚实篇}
	
	\begin{yuanwen}
	孙子曰:凡先处\footnote{text}\footnote{text}战地而待敌者佚\footnote{text},后处战地而趋\footnote{text}战\footnote{text}者劳\footnote{text}。故善战者,致人\footnote{text}而不致于人。能使敌人自至者,利之也;能使敌人不得至者,害\footnote{text}之也。故敌佚能劳之,饱能饥之,安能动之。
	\end{yuanwen}

	\begin{yuanwen}
	出其所必趋\footnote{text}\footnote{text},趋其所不意\footnote{text}。行千里而不劳者,行于无人之地也。攻而必取者,攻其所不守也。守而必固者,守其所不攻也。故善攻者,敌不知其所守;善守者,敌不知其所攻。微乎微乎\footnote{text},至于无形;神乎神乎\footnote{text},至于无声,故能为敌之司命\footnote{text}。
	\end{yuanwen}
	
	\begin{yuanwen}
	进而不可御\footnote{text}者,冲其虚也;退而不可追者,速而不可及\footnote{text}也。故我欲战,敌虽高垒深沟,不得不与我战者,攻其所必救也。我不欲战,画地而守之,敌不得与我战者,乖\footnote{text}其所之也。
	\end{yuanwen}
	
	\begin{yuanwen}
	故形人\footnote{text}而我无形,则我专\footnote{text}而敌分\footnote{text}。我专为一,敌分为十,是以十攻其一也,则我众敌寡。能以众击寡者,则吾之所与战者约\footnote{text}矣。吾所与战之地不可知,不可知则敌所备者多。敌所备者多,则吾所与战者寡矣。故备前则后寡,备后则前寡,备左则右寡,备右则左寡,无所不备,则无所不寡。寡者,备人者也;众者,使人备己者也。
	\end{yuanwen}

	\begin{yuanwen}
	故知战之地,知战之日,则可千里而会战。不知战地,不知战日,则左不能救右,右不能救左,前不能救后,后不能救前,而况远者数十里,近者数里乎?以吾度\footnote{text}之,越人\footnote{text}之兵虽多,亦奚\footnote{text}益于胜哉?故曰:胜可为也。敌虽众,可使无斗\footnote{text}。
	\end{yuanwen}

	\begin{yuanwen}
	故策\footnote{text}之而知得失之计,作\footnote{text}之而知动静之理,形\footnote{text}之而知死生之地,角\footnote{text}之而知有余不足之处。故形兵之极\footnote{text},至于无形。无形,则深间\footnote{text}不能窥\footnote{text},智者不能谋。因形而错胜\footnote{text}于众,众不能知;人皆知我所以胜之形,而莫知吾所以制胜之形。故其战胜不复\footnote{text},而应形于无穷。
	\end{yuanwen}

	\begin{yuanwen}
	夫兵形象水,水之形,避高而趋下。兵之形,避实而击虚。水因地而制流,兵因敌而制胜。故兵无常势\footnote{text},水无常形。能因敌变化而取胜者,谓之神\footnote{text}。故五行\footnote{text}无常胜,四时无常位,日有短长,月有死生\footnote{text}。
    \end{yuanwen}
	
	\chapter{军争篇}
	
	\begin{yuanwen}
	孙子曰:凡用兵之法,将受命于君,合军聚众,交和而舍\footnote{text}\footnote{text}\footnote{text},莫难于军争\footnote{text}。军争之难者,以迂\footnote{text}为直\footnote{text}\footnote{text},以患\footnote{text}为利。故迂其途而诱之以利,后人发,先人至,此知迂直之计者也。
	\end{yuanwen}

	\begin{yuanwen}
	故军争为\footnote{text}利\footnote{text},军争为危。举军\footnote{text}而争利\footnote{text},则不及;委\footnote{text}军而争利\footnote{text},则辎重\footnote{text}捐\footnote{text}\footnote{text}。是故卷甲而趋\footnote{text},日夜不处\footnote{text},倍\footnote{text}道兼行\footnote{text},百里而争利,则擒\footnote{text}三军将,劲者\footnote{text}先,疲者后,其法十一而至。五十里而争利,则蹶\footnote{text}上将军\footnote{text},其法半至。三十里而争利,则三分之二至。是故军无辎重则亡,无粮食则亡,无委积\footnote{text}则亡。
	\end{yuanwen}
	
	\begin{yuanwen}
	故不知诸侯之谋者,不能豫交\footnote{text};不知山林、险阻\footnote{text}、沮泽\footnote{text}之形者,不能行军;不用乡导\footnote{text}者,不能得地利。
	\end{yuanwen}
	
	\begin{yuanwen}
	故兵以诈立\footnote{text},以利动,以分合为变\footnote{text}者也。故其疾如风\footnote{text},其徐如林\footnote{text},侵掠如火\footnote{text},不动如山,难知如阴\footnote{text},动如雷震。掠乡\footnote{text}分众\footnote{text},廓\footnote{text}地分利,悬权\footnote{text}而动。先知迂直之计者胜,此军争之法也。
	\end{yuanwen}

	\begin{yuanwen}
	《军政》\footnote{text}曰:“言不相闻,故为金鼓\footnote{text};视不相见,故为旌旗\footnote{text}。”夫金鼓旌旗者,所以一\footnote{text}人之耳目也。人既专一,则勇者不得独进,怯者不得独退,此用众之法也。故夜战多火鼓\footnote{text},昼战多旌旗,所以变人之耳目也。
	\end{yuanwen}

	\begin{yuanwen}
	故三军可夺\footnote{text}气\footnote{text},将军可夺\footnote{text}心。是故朝\footnote{text}气锐\footnote{text}\footnote{text},昼气惰\footnote{text},暮气归\footnote{text}。故善用兵者,避其锐气,击其惰归,此治气\footnote{text}者也。以治待乱,以静待哗,此治心者也。以近待远,以佚待劳,以饱待饥,此治力者也。无邀\footnote{text}正正之旗,勿击堂堂之陈\footnote{text},此治变者也。
	\end{yuanwen}

	\begin{yuanwen}
	故用兵之法,高陵\footnote{text}勿向\footnote{text},背\footnote{text}丘勿逆\footnote{text},佯\footnote{text}北\footnote{text}勿从\footnote{text},锐卒\footnote{text}勿攻,饵兵\footnote{text}勿食\footnote{text},归师勿遏\footnote{text},围师必阙\footnote{text}\footnote{text},穷寇\footnote{text}勿迫,此用兵之法也。
    \end{yuanwen}
	
	\chapter{九变篇}
	
	\begin{yuanwen}
	孙子曰:凡用兵之法,将受命于君\footnote{text},合军聚众。圮地无舍\footnote{text},衢地交合\footnote{text},绝地无留\footnote{text},围地则谋\footnote{text},死地则战\footnote{text},途\footnote{text}有所不由\footnote{text},军有所不击,城有所不攻,地有所不争,君命有所不受\footnote{text}。
	\end{yuanwen}

	\begin{yuanwen}
	故将通于九变之地利\footnote{text}者,知用兵矣;将不通于九变之利者,虽知地形,不能得地之利矣。治兵不知九变之术,虽知五利\footnote{text},不能得人之用矣。
	\end{yuanwen}

	\begin{yuanwen}
	是故智者\footnote{text}之虑,必杂\footnote{text}于利害\footnote{text}。杂于利而务\footnote{text}可信\footnote{text}\footnote{text}也,杂于害而患可解\footnote{text}也。
	\end{yuanwen}
	
	\begin{yuanwen}
	是故屈\footnote{text}诸侯者以害\footnote{text},役\footnote{text}诸侯者以业\footnote{text},趋\footnote{text}诸侯者以利。
	\end{yuanwen}
	
	\begin{yuanwen}
	故用兵之法,无恃\footnote{text}其不来,恃吾有以待之;无恃其不攻,恃吾有所不可攻也。
	\end{yuanwen}

	\begin{yuanwen}
	故将有五危:必死\footnote{text},可杀\footnote{text}也;必生\footnote{text},可虏\footnote{text}也;忿速\footnote{text}\footnote{text},可侮\footnote{text}也;廉洁\footnote{text},可辱\footnote{text}也;爱民,可烦也。凡此五者,将之过也,用兵之灾也。覆军杀将,必以五危,不可不察也。
    \end{yuanwen}
	
	\chapter{行军篇}
	
	\begin{yuanwen}
	孙子曰:凡处军\footnote{text}相敌\footnote{text},绝山依谷\footnote{text},视生处高\footnote{text},战隆无登\footnote{text},此处山之军也。绝水必远水\footnote{text};客\footnote{text}绝水而来,勿迎之于水内,令半济而击之\footnote{text},利;欲战者,无附\footnote{text}于水而迎\footnote{text}客;视生处高,无迎水流,此处水上之军也。绝斥泽\footnote{text},惟亟\footnote{text}去无留;若交军于斥泽之中,必依水草而背众树\footnote{text},此处斥泽之军也。平陆处易\footnote{text}而右背高,前死后生\footnote{text},此处平陆之军也。凡此四军之利,黄帝之所以胜四帝也\footnote{text}。
	\end{yuanwen}

	\begin{yuanwen}
	凡军好\footnote{text}高而恶\footnote{text}下,贵阳\footnote{text}而贱阴,养生\footnote{text}而处实\footnote{text},军无百疾,是谓必胜。丘陵堤防,必处其阳而右背之\footnote{text}。此兵之利,地之助\footnote{text}也。上雨,水沫至\footnote{text},欲涉\footnote{text}者,待其定也。
	\end{yuanwen}

	\begin{yuanwen}
	凡地有绝涧\footnote{text}、天井、天牢、天罗、天陷、天隙,必亟\footnote{text}去\footnote{text}之,勿近也。吾远之,敌近之;吾迎之,敌背之。军行有险阻、潢井\footnote{text}、葭\footnote{text}苇、山林翳荟\footnote{text}者,必谨\footnote{text}复\footnote{text}索\footnote{text}之,此伏奸之所处也。
	\end{yuanwen}

	\begin{yuanwen}
	敌近而静者,恃\footnote{text}其险也;远而挑战者,欲人之进也;其所居易\footnote{text}者,利\footnote{text}也。众树动者,来也;众草多障\footnote{text}者,疑\footnote{text}也;鸟起者,伏也;兽骇\footnote{text}者,覆\footnote{text}也。尘高而锐\footnote{text}者,车来也;卑而广者,徒来也;散而条达\footnote{text}者,樵采\footnote{text}也;少而往来者,营军\footnote{text}也。辞卑\footnote{text}而益备\footnote{text}者,进也;辞强而进驱者,退也;轻车先出居其侧者,陈\footnote{text}也;无约而请和者,谋也;奔走而陈兵车者,期\footnote{text}也;半进半退者,诱也。杖而立者\footnote{text},饥\footnote{text}也;汲\footnote{text}而先饮者,渴也;见利而不进者,劳也;鸟集者,虚也;夜呼者,恐也;军扰者,将不重也\footnote{text};旌旗动者,乱也;吏怒者,倦也;粟马肉食\footnote{text},军无悬缻\footnote{text}\footnote{text},不返其舍者,穷寇也;谆谆\footnote{text}翕翕\footnote{text}\footnote{text},徐\footnote{text}与人言者,失众也;数赏\footnote{text}者,窘\footnote{text}也;数罚\footnote{text}者,困\footnote{text}也;先暴而后畏其众者,不精\footnote{text}之至也;来委谢\footnote{text}者,欲休息\footnote{text}也。
	\end{yuanwen}

	\begin{yuanwen}
	兵怒而相迎,久而不合\footnote{text},又不相去,必谨察之。
	\end{yuanwen}
	
	\begin{yuanwen}
	兵非益多也,惟\footnote{text}无武进\footnote{text},足以并\footnote{text}力、料敌\footnote{text}、取人\footnote{text}而已。夫惟无虑而易敌\footnote{text}者,必擒于人。
	\end{yuanwen}

	\begin{yuanwen}
	卒未亲附\footnote{text}而罚之则不服,不服则难用也。卒已亲附而罚不行,则不可用也。故令之以文,齐之以武\footnote{text},是谓必取\footnote{text}。令素\footnote{text}行\footnote{text}以教其民,则民服;令素不行以教其民,则民不服。令素行者,与众相得\footnote{text}也。
    \end{yuanwen}
	
	\chapter{地形篇}
	
	\begin{yuanwen}
	孙子曰:地形有通者\footnote{text},有挂者\footnote{text},有支者\footnote{text},有隘者\footnote{text},有险者,有远者。我可以往,彼可以来,曰通;通形者,先居高阳,利粮道,以战则利。可以往,难以返,曰挂;挂形者,敌无备,出而胜之;敌若有备,出而不胜,难以返,不利。我出而不利,彼出而不利,曰支;支形者,敌虽利我,我无出也;引而去之,令敌半出而击之,利。隘形者,我先居之,必盈之以待敌\footnote{text};若敌先居之,盈而勿从,不盈而从之。险形者\footnote{text},我先居之,必居高阳以待敌;若敌先居之,引而去之,勿从也。远形者\footnote{text},势均难以挑战\footnote{text},战而不利。凡此六者,地之道也\footnote{text},将之至任,不可不察也。
	\end{yuanwen}

	\begin{yuanwen}
	故兵有走者\footnote{text},有弛者\footnote{text},有陷者\footnote{text},有崩者\footnote{text},有乱者,有北者\footnote{text}。凡此六者,非天之灾,将之过也。夫势均,以一击十,曰走。卒强吏弱,曰弛。吏强卒弱,曰陷。大吏怒而不服\footnote{text},遇敌怼而自战\footnote{text},将不知其能,曰崩。将弱不严,教道不明\footnote{text},吏卒无常,陈兵纵横\footnote{text},曰乱。将不能料敌\footnote{text},以少合众,以弱击强,兵无选锋\footnote{text},曰北。凡此六者,败之道也,将之至任,不可不察也。
	\end{yuanwen}

	\begin{yuanwen}
	夫地形者,兵之助\footnote{text}也。料敌制胜,计险厄远近\footnote{text},上将之道也\footnote{text}。知此而用战者必胜,不知此而用战者必败。故战道必胜\footnote{text},主曰无战,必战可也;战道不胜,主曰必战,无战可也。故进不求名,退不避罪,唯人是保\footnote{text},而利合于主\footnote{text},国之宝也。
	\end{yuanwen}

	\begin{yuanwen}
	视卒如婴儿\footnote{text},故可与之赴深谿\footnote{text};视卒如爱子,故可与之俱死。厚而不能使\footnote{text},爱而不能令\footnote{text},乱而不能治,譬若骄子\footnote{text},不可用也。
	\end{yuanwen}

	\begin{yuanwen}
	知吾卒之可以击,而不知敌之不可击,胜之半也;知敌之可击,而不知吾卒之不可以击,胜之半也;知敌之可击,知吾卒之可以击,而不知地形之不可以战,胜之半也。故知兵\footnote{text}者,动而不迷\footnote{text},举而不穷\footnote{text}。故曰:知彼知己,胜乃不殆\footnote{text};知天知地,胜乃不穷。
    \end{yuanwen}
	
	\chapter{九地篇}
	
	\begin{yuanwen}
	孙子曰:用兵之法,有散地,有轻地,有争地,有交地,有衢地,有重地,有圮地,有围地,有死地。诸侯自战其地者\footnote{text},为散地\footnote{text}。入人之地而不深者,为轻地\footnote{text}。我得亦利,彼得亦利者,为争地\footnote{text}。我可以往,彼可以来者,为交地\footnote{text}。诸侯之地三属\footnote{text},先至而得天下众者,为衢地。入人之地深,背城邑多者,为重地\footnote{text}。行山林、险阻、沮泽,凡难行之道者,为圮地\footnote{text}。所由入者隘\footnote{text},所从归者迂\footnote{text},彼寡可以击吾之众者,为围地。疾战则存,不疾战则亡者,为死地。是故散地则无战\footnote{text},轻地则无止\footnote{text},争地则无攻,交地则无绝\footnote{text},衢地则合交\footnote{text},重地则掠\footnote{text},圮地则行\footnote{text},围地则谋\footnote{text},死地则战。
	\end{yuanwen}

	\begin{yuanwen}
	所谓古之善用兵者,能使敌人前后不相及\footnote{text},众寡不相恃\footnote{text},贵贱不相救\footnote{text},上下不相收\footnote{text},卒离而不集,兵合而不齐。合于利而动,不合于利而止。敢问:“敌众整而将来,待之若何?”曰:“先夺其所爱\footnote{text},则听矣。”兵之情主速\footnote{text},乘人之不及,由不虞之道\footnote{text},攻其所不戒也\footnote{text}。
	\end{yuanwen}

	\begin{yuanwen}
	凡为客之道\footnote{text},深入则专\footnote{text}。主人不克\footnote{text},掠于饶野\footnote{text},三军足食;谨养而勿劳\footnote{text},并气积力;运兵计谋\footnote{text},为不可测。投之无所往\footnote{text},死且不北。死焉不得\footnote{text},士人尽力。兵士甚陷则不惧,无所往则固\footnote{text},深入则拘\footnote{text},不得已则斗。是故其兵不修而戒\footnote{text},不求而得,不约而亲\footnote{text},不令而信\footnote{text}。禁祥去疑\footnote{text},至死无所之。
	\end{yuanwen}
	
	\begin{yuanwen}
	吾士无余财,非恶货也\footnote{text};无余命,非恶寿也。令发之日,士卒坐者涕沾襟\footnote{text},偃卧者涕交颐\footnote{text}。投之无所往,诸、刿之勇也\footnote{text}。
	\end{yuanwen}

	\begin{yuanwen}
	故善用兵者,譬如率然\footnote{text};率然者,常山之蛇\footnote{text}也。击其首则尾至,击其尾则首至,击其中则首尾俱至。敢问:“兵可使如率然乎?”曰:“可。”夫吴人与越人相恶也,当其同舟而济\footnote{text},遇风,其相救也如左右手。是故方马埋轮,未足恃也\footnote{text};齐勇若一,政之道也\footnote{text};刚柔皆得\footnote{text},地之理也。故善用兵者,携手若使一人,不得已也。
	\end{yuanwen}

	\begin{yuanwen}
	将军之事,静以幽\footnote{text},正以治\footnote{text}。能愚士卒之耳目\footnote{text},使之无知。易其事\footnote{text},革其谋\footnote{text},使人无识;易其居,迂其途,使人不得虑。帅与之期\footnote{text},如登高而去其梯。帅与之深入诸侯之地,而发其机\footnote{text},焚舟破釜\footnote{text},若驱群羊,驱而往,驱而来,莫知所之。聚三军之众,投之于险,此谓将军之事也。九地之变\footnote{text},屈伸之利\footnote{text},人情之理\footnote{text},不可不察。
	\end{yuanwen}

	\begin{yuanwen}
	凡为客之道,深则专,浅则散。去国越境而师者\footnote{text},绝地也;四达者,衢地也;入深者,重地也;入浅者,轻地也;背固前隘者,围地也;无所往者,死地也。是故散地,吾将一其志;轻地,吾将使之属\footnote{text};争地,吾将趋其后\footnote{text};交地,吾将谨其守;衢地,吾将固其结\footnote{text};重地,吾将继其食;圮地,吾将进其塗\footnote{text};围地,吾将塞其阙\footnote{text};死地,吾将示之以不活。故兵之情,围则御,不得已则斗,过则从\footnote{text}。
	\end{yuanwen}

	\begin{yuanwen}
	是故不知诸侯之谋者,不能预交\footnote{text};不知山林、险阻、沮泽之形者,不能行军;不用乡导者\footnote{text},不能得地利。四五者,不知一\footnote{text},非霸王之兵也\footnote{text}。夫霸王之兵,伐大国,则其众不得聚;威加于敌,则其交不得合\footnote{text}。是故不争天下之交,不养天下之权\footnote{text},信己之私,威加于敌,故其城可拔,其国可隳\footnote{text}。
	\end{yuanwen}

	\begin{yuanwen}
	施无法之赏\footnote{text},悬无政之令\footnote{text},犯三军之众\footnote{text},若使一人。犯之以事,勿告以言;犯之以利,勿告以害。投之亡地然后存,陷之死地然后生。夫众陷于害,然后能为胜败。故为兵之事,在于顺详敌之意\footnote{text},并敌一向\footnote{text},千里杀将,此谓巧能成事者也。
	\end{yuanwen}

	\begin{yuanwen}
	是故政举之日\footnote{text},夷关折符\footnote{text},无通其使\footnote{text};厉于廊庙之上\footnote{text},以诛其事\footnote{text}。敌人开阖\footnote{text},必亟入之。先其所爱\footnote{text},微与之期\footnote{text}。践墨随敌\footnote{text},以决战事。是故始如处女,敌人开户;后如脱兔\footnote{text},敌不及拒。
    \end{yuanwen}
	
	\chapter{火攻篇}
	
	\begin{yuanwen}
	孙子曰:凡火攻有五:一曰火人\footnote{text},二曰火积\footnote{text},三曰火辎\footnote{text},四曰火库\footnote{text},五曰火队\footnote{text}。行火必有因\footnote{text},烟火必素具\footnote{text}。发火有时,起火有日。时者,天之燥也;日者,月在箕\footnote{text}、壁\footnote{text}、翼\footnote{text}、轸\footnote{text}也。凡此四宿者,风起之日也。
	\end{yuanwen}

	\begin{yuanwen}
	凡火攻,必因五火之变而应之\footnote{text}。火发于内,则早应之于外。火发兵静者,待而勿攻,极其火力\footnote{text},可从而从之\footnote{text},不可从而止。火可发于外,无待于内,以时发之。火发上风,无攻下风。昼风久,夜风止。凡军必知有五火之变,以数守之\footnote{text}。
	\end{yuanwen}

	\begin{yuanwen}
	故以火佐攻者明\footnote{text},以水佐攻者强。水可以绝\footnote{text},不可以夺\footnote{text}。
    \end{yuanwen}
    
    \begin{yuanwen}
    夫战胜攻取,而不修其功者凶\footnote{text},命曰费留\footnote{text}。故曰:明主虑之\footnote{text},良将修之\footnote{text}。非利不动,非得不用\footnote{text},非危不战。主不可以怒而兴师\footnote{text},将不可以愠而致战\footnote{text}。合于利而动,不合于利而止。怒可以复喜,愠可以复悦。亡国不可以复存,死者不可以复生。故明君慎之,良将警之,此安国全军之道也\footnote{text}。
    \end{yuanwen}
	
	\chapter{用间篇}
	
	\begin{yuanwen}
	孙子曰:凡兴师十万,出征千里,百姓之费,公家之奉\footnote{text},日费千金;内外骚动,怠于道路\footnote{text},不得操事者\footnote{text},七十万家\footnote{text}。相守数年\footnote{text},以争一日之胜,而爱爵禄百金\footnote{text},不知敌之情者,不仁之至也\footnote{text},非人之将也,非主之佐也,非胜之主也。
	\end{yuanwen}
	
	\begin{yuanwen}
	故明君贤将,所以动而胜人,成功出于众者\footnote{text},先知也\footnote{text}。先知者,不可取于鬼神\footnote{text},不可象于事\footnote{text},不可验于度\footnote{text},必取于人,知敌之情者也。
	\end{yuanwen}

	\begin{yuanwen}
	故用间有五:有因间\footnote{text},有内间,有反间,有死间,有生间。五间俱起\footnote{text},莫知其道\footnote{text},是谓神纪\footnote{text},人君之宝也。乡间者,因其乡人而用之\footnote{text}。内间者,因其官人而用之。反间者,因其敌间而用之\footnote{text}。死间者,为诳事于外\footnote{text},令吾间知之,而传于敌间也。生间者,反报也\footnote{text}。
	\end{yuanwen}

	\begin{yuanwen}
	故三军之事,莫亲于间\footnote{text},赏莫厚于间,事莫密于间\footnote{text},非圣智不能用间\footnote{text},非仁义不能使间\footnote{text},非微妙不能得间之实\footnote{text}。微哉!微哉!无所不用间也。间事未发而先闻者\footnote{text},间与所告者皆死。
	\end{yuanwen}

	\begin{yuanwen}
	凡军之所欲击,城之所欲攻,人之所欲杀,必先知其守将\footnote{text},左右\footnote{text},谒者\footnote{text},门者\footnote{text},舍人之姓名\footnote{text},令吾间必索知之\footnote{text}。
	\end{yuanwen}

	\begin{yuanwen}
	必索敌人之间来间我者,因而利之\footnote{text},导而舍之\footnote{text},故反间可得而用也。因是而知之\footnote{text},故乡间、内间可得而使也。因是而知之,故死间为诳事,可使告敌。因是而知之,故生间可使如期\footnote{text}。五间之事,主必知之,知之必在于反间,故反间不可不厚也\footnote{text}。
	\end{yuanwen}

	\begin{yuanwen}
	昔殷之兴也\footnote{text},伊挚在夏\footnote{text};周之兴也\footnote{text},吕牙在殷\footnote{text}。故惟明君贤将,能以上智为间者\footnote{text},必成大功。此兵之要,三军之所恃而动也\footnote{text}。	
    \end{yuanwen}

\end{document}