% 西游记
% 西游记.tex

\documentclass[12pt,UTF8]{ctexbook}

% 设置纸张信息。
\usepackage[a4paper,twoside]{geometry}
\geometry{
	left=25mm,
	right=25mm,
	bottom=25.4mm,
	bindingoffset=10mm
}

% 设置字体,并解决显示难检字问题。
\xeCJKsetup{AutoFallBack=true}
\setCJKmainfont{SimSun}[BoldFont=SimHei, ItalicFont=KaiTi, FallBack=SimSun-ExtB]

% 目录 chapter 级别加点(.)。
\usepackage{titletoc}
\titlecontents{chapter}[0pt]{\vspace{3mm}\bf\addvspace{2pt}\filright}{\contentspush{\thecontentslabel\hspace{0.8em}}}{}{\titlerule*[8pt]{.}\contentspage}

% 设置 chapter 标题格式。
\usepackage{varwidth}
\ctexset{
	chapter/name={第,回},
	chapter/titleformat= \chaptertitleformat
}
\newcommand\chaptertitleformat[1]{
	\begin{varwidth}
		[t]{.7\linewidth}#1
	\end{varwidth}
}

% 设置古文原文格式。
\newenvironment{yuanwen}{\bfseries}

% 设置署名格式。
\newenvironment{shuming}{\hfill\bfseries\zihao{4}}

% 注脚每页重新编号,避免编号过大。
\usepackage[perpage]{footmisc}

\title{\heiti\zihao{0} 西游记}
\author{吴承恩}
\date{明}

\begin{document}

\maketitle
\tableofcontents

\frontmatter

\chapter{前言}

《西游记》的故事在唐玄奘取经之后,经历数百年的演变,逐渐发展成型,最迟于明代集撰成为小说。现存最早的百回本《西游记》约刊行于明嘉靖至万历年间,为明刊金陵世德堂本(简称世本)。

世本现存四套,一套藏于中国,三套藏于日本,后者中直到近年才有天理本和浅野本两套得以在国内公开。

明代李评本、唐僧本、杨闽斋本、闽斋堂本,清代新说本、证道书等。

\mainmatter

\chapter[灵根育孕源流出\ 心性修持大道生]{灵根育孕源流出\\心性修持大道生}

诗曰:

\begin{quotation}
混沌未分天地乱,茫茫渺渺无人见。

自从盘古破鸿蒙,开辟从兹清浊辨。

覆载群生仰至仁,发明万物皆成善。

欲知造化会元\footnote{会元:会、元,皆为时间单位,来自宋代思想家邵雍提出的一种宇宙观,以元、会、运、世为单位推演宇宙变化。三十年为一世,十二世为一运,三十运为一会,十二会为一元。每一元即为一次宇宙生灭周期。下文“盖闻天地之数”一节,即在解释这一观念。}功,须看西游释厄传。
\end{quotation}

盖闻天地之数,有十二万九千六百岁为一元。将一元分为十二会,乃子、丑、寅、卯、辰、巳、午、未、申、酉、戌、亥之十二支也。每会该一万八百岁。且就一日而论:子时得阳气,而丑则鸡鸣;寅不通光,而卯则日出;辰时食后,而巳则挨排;日午天中,而未则西蹉\footnote{cu\=o,倾斜、下坠。};申时晡\footnote{b\=u}而日落酉;戌黄昏而人定亥。\footnote{“子时”句:古用“夜半、鸡鸣、平旦、日出、食时、隅中、日中、日昳、晡时、日入、黄昏、人定”命名十二时辰,与十二地支命名一一相对。}譬于大数\footnote{自然法则;气数。},若到戌会之终,则天地昏曚\footnote{m\'eng}而万物否\footnote{p\v{i},闭塞。}矣。再去五千四百岁,交亥会之初,则当黑暗,而两间人物俱无矣,故曰混沌\footnote{d\`un}。又五千四百岁,亥会将终,贞下起元\footnote{《周易》有“乾,元亨利贞”句。后世以“元、亨、利、贞”象征阴阳演化的一个循环。“贞下起元”即“贞”终而“元”始。这里指“亥会”和“子会”的接续。},近子之会,而复逐渐开明。邵康节\footnote{即邵雍。北宋哲学家。字尧夫,谥康节。}曰:“冬至子之半,天心无改移。一阳初动处,万物未生时。\footnote{“冬至”四句:大意为,冬至是子月(农历十一月)过了一半,天道规律并没有改变。此时阳气刚刚萌动,万物尚未生长。冬至后白昼渐长,传统哲学认为此时正是阴极阳生之时。}”到此,天始有根。再五千四百岁,正当子会,轻清上腾,有日,有月,有星,有辰。日、月、星、辰,谓之四象。故曰,天开于子。又经五千四百岁,子会将终,近丑之会,而遂渐坚实。《易》曰:“大哉乾元!至哉坤元!万物资生,乃顺承天。”至此,地始凝结。再五千四百岁,正当丑会,重浊下凝,有水,有火,有山,有石,有土。水、火、山、石、土,谓之五形。故曰,地辟于丑。又经五千四百岁,丑会终而寅会之初,发生万物。历曰:“天气下降,地气上升;天地交合,群物皆生。”至此,天清地爽,阴阳交合。再五千四百岁,正当寅会,生人,生兽,生禽,正谓天地人,三才\footnote{古人以“天、地、人”为三才。}定位。故曰,人生于寅。

感盘古开辟,三皇治世,五帝定伦,世界之间,遂分为四大部洲:曰东胜神洲,曰西牛贺洲,曰南赡部洲,曰北俱芦洲。\footnote{四大部洲的概念源自古印度,在其神话中,人类居住世界的中心位于须弥山,须弥山被咸海包围,海中分布有四大部洲。}这部书单表东胜神洲。海外有一国土,名曰傲来国。国近大海,海中有一座名山,唤为花果山。此山乃十洲之祖脉,三岛之来龙\footnote{十洲、三岛,道教称神仙在大海中的居所。十洲包括祖洲、瀛洲等处。三岛指蓬莱、方丈、瀛洲。祖脉、来龙,堪舆学术语,指山的源头。},自开清浊而立,鸿蒙判后而成。真个好山!有词赋为证。赋曰:

\begin{quotation}
势镇汪洋,威宁瑶海。势镇汪洋,潮涌银山鱼入穴;威宁瑶海,波翻雪浪蜃离渊。木火方隅高积土\footnote{意指花果山在大海的东南方。木火,指东南方。五行观念中,东方属木,南方属火。隅(yú),角落。},东海之处耸崇巅。丹崖怪石,削壁奇峰。丹崖上彩凤双鸣;削壁前麒麟独卧。峰头时听锦鸡鸣,石窟每观龙出入。林中有寿鹿仙狐,树上有灵禽玄鹤。瑶草奇花不谢,青松翠柏长春。仙桃常结果,修竹每留云。一条涧壑藤萝密,四面原堤草色新。正是百川会处擎天柱,万劫无移大地根。
\end{quotation}

那座山正当顶上,有一块仙石。其石有三丈六尺五寸高,有二丈四尺围圆。三丈六尺五寸高,按周天三百六十五度;二丈四尺围圆,按政历\footnote{即历法。}二十四气。上有九窍八孔,按九宫八卦。四面更无树木遮阴,左右倒有芝兰相衬。盖自开辟以来,每受天真地秀,日精月华,感之既久,遂有灵通之意。内育仙胞,一日迸裂,产一石卵,似圆球样大。因见风,化作一个石猴。五官俱备,四肢皆全。便就学爬学走,拜了四方\footnote{有些畜类初生时因习走路而摔倒,像是朝拜动作,故民间有“拜四方”之说。}。目运两道金光,射冲斗府\footnote{指斗宿星宫。}。惊动高天上圣大慈仁者玉皇大天尊玄穹高上帝,驾座金阙云宫灵霄宝殿,聚集仙卿,见有金光焰焰,即命千里眼、顺风耳开南天门观看。二将果奉旨出门外,看的真,听的明,须臾回报道:“臣奉旨观听金光之处,乃东胜神洲海东傲来小国之界,有一座花果山,山上有一仙石,石产一卵,见风化一石猴,在那里拜四方,眼运金光,射冲斗府。如今服饵水食\footnote{大意指吃了人间的水和食物。饵,泛指食物,也指服食、吃。},金光将潜息矣。”玉帝垂赐恩慈曰:“下方之物,乃天地精华所生,不足为异。”

那猴在山中,却会行走跳跃,食草木,饮涧泉,采山花,觅树果;与狼虫为伴,虎豹为群,獐鹿为友,猕猿为亲;夜宿石崖之下,朝游峰洞之中。真是“山中无甲子\footnote{中国传统干支纪年中,每六十年为一个循环,首年为甲子。这里引申为时间。},寒尽不知年”。一朝天气炎热,与群猴避暑,都在松阴之下顽耍。你看他一个个:

\begin{quotation}
跳树攀枝,采花觅果。抛弹子,邷麽儿\footnote{邷麽儿:一种儿童游戏。把瓦片磨成许多小块抓着玩。邷(w\v{a}),抓。}。跑\footnote{同“刨”。}沙窝,砌宝塔。赶蜻蜓,扑𧈢蜡\footnote{m\`a,𧈢蜡,北京方言,蝗虫的意思,就是蚂蚱。也作“八蜡”。}。参老天,拜菩萨。扯葛藤,编草帓\footnote{w\`a:同“袜”,袜子。}。捉虱子,咬又掐。理毛衣,剔指甲。挨的挨,擦的擦。推的推,压的压。扯的扯,拉的拉。青松林下任他顽,绿水涧边随洗濯\footnote{zhu\'o}。
\end{quotation}

一群猴子耍了一会,却去那山涧中洗澡。见那股涧水奔流,真个似滚瓜涌溅。古云:“禽有禽言,兽有兽语。”众猴都道:“这股水不知是那里的水。我们今日赶闲无事,顺涧边往上溜头\footnote{上游,河川的上水方向。}寻看源流,耍子去耶!”喊一声,都拖男挈\footnote{qi\`e,带,领。}女,唤弟呼兄,一齐跑来。顺涧爬山,直至源流之处,乃是一股瀑布飞泉。但见那:

\begin{quotation}
一派白虹起,千寻雷浪飞\footnote{形容瀑布极高而声势浩大。古以八尺为一寻,千寻形容极高或极长。雷,形容如雷的水声。}。海风吹不断,江月照还依。

冷气分青嶂,余流润翠微。潺湲\footnote{ch\'an yu\'an,形容河水慢慢流动的样子。}名瀑布,真似挂帘帷。
\end{quotation}


众猴拍手称扬道:“好水!好水!原来此处远通山脚之下,直接大海之波。”又道:“那一个有本事的,钻进去寻个源头出来,不伤身体者,我等即拜他为王。”连呼了三声,忽见丛杂中跳出一个石猴,应声高叫道:“我进去!我进去!”好猴!也是他:

\begin{quotation}
今日芳名显,时来大运通。有缘居此地,天遣入仙宫。
\end{quotation}

你看他瞑目蹲身,将身一纵,径跳入瀑布泉中,忽睁睛抬头观看,那里边却无水无波,明明朗朗的一架桥梁。他住了身,定了神,仔细再看,原来是座铁板桥。桥下之水,冲贯于石窍\footnote{石洞。}之间,倒挂流出去,遮闭了桥门。却又欠身上桥头,再走再看,却似有人家住处一般,真个好所在。但见那:

\begin{quotation}
翠藓堆蓝,白云浮玉,光摇片片烟霞。虚窗静室,滑凳板生花。乳窟\footnote{生长有钟乳石的洞穴。}龙珠倚挂,萦回满地奇葩。锅灶傍崖存火迹,樽罍\footnote{z\=un l\'ei,一种盛酒器。}靠案见肴渣。石座石床真可爱,石盆石碗更堪夸。
\end{quotation}

又见那:

\begin{quotation}
一竿两竿修竹,三点五点梅花。几树青松常带雨,浑然像个人家。
\end{quotation}

看罢多时,跳过桥中间,左右观看,只见正当中有一石碣\footnote{圆顶碑石。}。碣上有一行楷书大字,镌着:

\begin{quotation}
花果山福地,水帘洞洞天。
\end{quotation}

石猿喜不自胜,急抽身往外便走,复瞑目蹲身,跳出水外,打了两个呵呵道:“大造化\footnote{福分,运气。}!大造化!”众猴把他围住,问道:“里面怎么样?水有多深?”石猴道:“没水!没水!原来是一座铁板桥。桥那边是一座天造地设的家当。”众猴道:“怎见得是个家当?”石猴笑道:“这股水乃是桥下冲贯石窍,倒挂下来遮闭门户的。桥边有花有树,乃是一座石房。房内有石锅、石灶、石碗、石盆、石床、石凳,中间一块石碣,上镌着‘花果山福地,水帘洞洞天’,真个是我们安身之处。里面且是宽阔,容得千百口老小。我们都进去住,也省得受老天之气。这里边:

\begin{quotation}
刮风有处躲,下雨好存身。霜雪全无惧,雷声永不闻。

烟霞常照耀,祥瑞每蒸薰。松竹年年秀,奇花日日新。”

\end{quotation}
众猴听得,个个欢喜,都道:“你还先走,带我们进去,进去!”石猴却又瞑目蹲身,往里一跳,叫道:“都随我进来!进来!”那些猴有胆大的,都跳进去了;胆小的,一个个伸头缩颈,抓耳挠腮,大声叫喊,缠一会,也都进去了。跳过桥头,一个个抢盆夺碗,占灶争床,搬过来,移过去,正是猴性顽劣,再无一个宁时,只搬得力倦神疲方止。

石猿端坐上面道:“列位呵,‘人而无信,不知其可’\footnote{语出《论语》,意为一个人若没有信用,不知道他还可以做什么。}。你们才说有本事进得来,出得去,不伤身体者,就拜他为王。我如今进来又出去,出去又进来,寻了这一个洞天与列位安眠稳睡,各享成家之福,何不拜我为王?”众猴听说,即拱伏无违。一个个序齿排班\footnote{按年龄长幼排定等第次序。齿,年龄。},朝上礼拜,都称“千岁大王”。自此,石猿高登王位,将“石”字儿隐了,遂称“美猴王”。有诗为证。诗曰:

\begin{quotation}
三阳交泰\footnote{原为《周易》卦象,后成为一种象征好运的祝福语。《周易》以月份匹配卦象,因正月为泰卦,此时三阳生,故名。也作“三阳开泰”。古人认为此时冬去春来,阴消阳长,有吉亨之象。}产群生,仙石胞含日月精。

借卵化猴完大道,假他名姓配丹成。

内观不识因无相,外合明知作有形。

历代人人皆属此,称王称圣任纵横。
\end{quotation}

美猴王领一群猿猴、猕猴、马猴等,分派了君臣佐使,朝游花果山,暮宿水帘洞,合契同情\footnote{意气相投,同心共志。合契,融洽,意气相投。同情,犹同心,一心。},不入飞鸟之丛,不从走兽之类,独自为王,不胜欢乐。是以:

\begin{quotation}
春采百花为饮食,夏寻诸果作生涯。

秋收芋栗延时节,冬觅黄精\footnote{一种草药,又名老虎姜、鸡头参,中医以根茎入药。}度岁华。
\end{quotation}

美猴王享乐天真,何期有三五百载。一日,与群猴喜宴之间,忽然忧恼,堕下泪来。众猴慌忙罗拜道:“大王何为烦恼?”猴王道:“我虽在欢喜之时,却有一点儿远虑,故此烦恼。”众猴又笑道:“大王好不知足!我等日日欢会,在仙山福地,古洞神洲,不伏麒麟辖,不伏凤凰管,又不伏人间王位所拘束,自由自在,乃无量之福,为何远虑而忧也?”猴王道:“今日虽不归人王法律,不惧禽兽威严,将来年老血衰,暗中有阎王老子管着,一旦身亡,可不枉生世界之中,不得久注天人之内?”众猴闻此言,一个个掩面悲啼,俱以无常为虑。

只见那班部中,忽跳出一个通臂猿猴\footnote{长臂猿。通臂,长臂。一说为传说中的猿,两臂相通,一臂缩短时另一臂伸长。},厉声高叫道:“大王若是这般远虑,真所谓道心\footnote{佛教语。菩提心,悟道之心。}开发也!如今五虫\footnote{古人把动物分为五类,即羽虫(禽类)、毛虫(兽类)、甲虫(昆虫类)、鳞虫(鱼类)、裸虫(人类),合称“五虫”。虫,动物的通称。}之内,惟有三等名色\footnote{名目,名称。}不伏阎王老子所管。”猴王道:“你知那三等人?”猿猴道:“乃是佛与仙与神圣三者,躲过轮回,不生不灭,与天地山川齐寿。”猴王道:“此三者居于何所?”猿猴道:“他只在阎浮世界\footnote{指人世间。阎浮,“阎浮提”的省称,即“南赡部洲”。}之中,古洞仙山之内。”猴王闻之,满心欢喜,道:“我明日就辞汝等下山,云游海角,远涉天涯,务必访此三者,学一个不老长生,常躲过阎君之难。”噫!这句话,顿教跳出轮回网,致使齐天大圣成。众猴鼓掌称扬,都道:“善哉!善哉!我等明日越岭登山,广寻些果品,大设筵\footnote{y\'an,指宴会。}宴送大王也。”

次日,众猴果去采仙桃,摘异果,刨山药,劚\footnote{zh\v{u},挖。}黄精,芝兰香蕙,瑶草奇花,般般件件,整整齐齐,摆开石凳石桌,排列仙酒仙肴。但见那:

\begin{quotation}
金丸珠弹,红绽黄肥。金丸珠弹,腊樱桃色真甘美;红绽黄肥,熟梅子味果香酸。鲜龙眼肉甜皮薄,火荔枝核小囊红。林檎碧实连枝献,枇杷缃苞\footnote{浅浅的黄色一丛。缃,浅黄色。}带叶擎。兔头梨子鸡心枣,消渴除烦更解酲\footnote{ch\'eng,酒醉后神志不清。解酲即解酒。}。香桃烂杏,美甘甘似玉液琼浆;脆李杨梅,酸荫荫如脂酥膏酪。红瓤黑子熟西瓜,四瓣黄皮大柿子。石榴裂破,丹砂粒现火晶珠;芋栗剖开,坚硬肉团金玛瑙。胡桃银杏可传茶,椰子葡萄能做酒。榛松榧柰\footnote{f\v{e}i n\`ai,榧,坚果类,两端尖,仁可食。柰,果木名,俗称柰子,苹果的一种。}满盘盛,橘蔗柑橙盈案摆。熟煨山药,烂煮黄精。捣碎茯苓并薏苡\footnote{一种寄生于松根的菌类。薏苡:草本植物,其仁含淀粉,可食用、酿酒、入药。},石锅微火漫炊羹。人间纵有珍馐味,怎比山猴乐更宁?
\end{quotation}

群猴尊美猴王上坐,各依齿序排于下边,一个个轮流上前,奉酒,奉花,奉果,痛饮了一日。次日,美猴王早起,教:“小的们,替我折些枯松,编作筏子,取个竹竿作篙,收拾些果品之类,我将去也。”果独自登筏,尽力撑开,飘飘荡荡,径向大海波中,趁天风,来渡南赡部洲地界。这一去,正是那:

\begin{quotation}
天产仙猴道行隆,离山驾筏趁天风。

飘洋过海寻仙道,立志潜心建大功。

有分有缘休俗愿,无忧无虑会元龙。

料应必遇知音者,说破源流万法通。
\end{quotation}

也是他运至时来,自登木筏之后,连日东南风紧,将他送到西北岸前,乃是南赡部洲地界。持篙试水,偶得浅水,弃了筏子跳上岸来,只见海边上有人捕鱼打雁、穵\footnote{w\=a,同“挖”。}蛤淘盐。他走近前,弄个把戏装个𡤫虎\footnote{qi\=a,𡤫虎,吓唬人的人。},吓得那些人丢筐弃网,四散奔跑。将那跑不动的拿住一个,剥了他的衣裳,也学人穿在身上,摇摇摆摆,穿州过府,在于市廛\footnote{chán,指店铺集中的市区。}中,学人礼,学人话。朝餐夜宿,一心里访问佛仙神圣之道,觅个长生不老之方。见世人都是为名为利之徒,更无一个为身命者。正是那:

\begin{quotation}
争名夺利几时休,早起迟眠不自由。

骑着驴骡思骏马,官居宰相望王侯。

只愁衣食耽\footnote{沉湎。}劳碌,何怕阎君就取勾\footnote{即勾取,传讯、提审犯人,借指鬼卒勾摄人的魂灵。}。

继子荫孙图富贵,更无一个肯回头。
\end{quotation}

猴王参访仙道,无缘得遇。在于南赡部洲,串长城,游小县,不觉八九年余。忽行至西洋大海,他想着海外必有神仙。独自个依前作筏,又飘过西海,直至西牛贺洲地界。登岸遍访多时,忽见一座高山秀丽,林麓幽深。他也不怕狼虫,不惧虎豹,登在山顶上观看。果是好山:

\begin{quotation}
千峰排戟,万仞开屏。日映岚光轻锁翠,雨收黛色冷含青。瘦藤缠老树,古渡界幽程。奇花瑞草,修竹乔松。修竹乔松,万载常青欺\footnote{遮蔽。}福地;奇花瑞草,四时不谢赛蓬瀛。幽鸟啼声近,源泉响溜清。重重谷壑芝兰绕,处处巉崖\footnote{高耸险峻的山崖。巉,ch\'an,险峻陡峭。}苔藓生。起伏峦头龙脉好,必有高人隐姓名。
\end{quotation}

正观看间,忽闻得林深之处有人言语,急忙趋步,穿入林中,侧耳而听,原来是歌唱之声。歌曰:

\begin{quotation}
观棋柯烂\footnote{斧柄朽烂。典出晋人伐木遇仙的故事,说晋代王质在山中伐木时遇下棋的童子,送他一枚枣核大小的食物。王质口含而不觉饥,片刻后,手中斧柄已烂,回到家中才发现物是人非,已去数年。柯,斧子的柄。},伐木丁丁\footnote{zh\=eng,指伐木声。},云边谷口徐行。卖薪沽酒,狂笑自陶情。苍径秋高,对月枕松根,一觉天明。认旧林,登崖过岭,持斧断枯藤。收来成一担,行歌市上,易米三升。更无些子争竞,时价平平。不会机谋巧算,没荣辱,恬淡延生。相逢处,非仙即道,静坐讲《黄庭》。
\end{quotation}

美猴王听得此言,满心欢喜道:“神仙原来藏在这里!”即忙跳入里面,仔细再看,乃是一个樵子,在那里举斧砍柴。但看他打扮非常:

\begin{quotation}
头上戴箬笠\footnote{ru\`o l\`i,用箬竹叶及篾编成的宽边帽,即用竹篾、箬叶编织的斗笠。},乃是新笋初脱之箨\footnote{tu\`o,竹笋壳。包在新竹外面的皮叶,竹长成逐渐脱落。}。身上穿布衣,乃是木绵捻就之纱。腰间系环绦,乃是老蚕口吐之丝。足下踏草履,乃是枯莎搓就之爽\footnote{通“𦄍”,草鞋上的绞绳。}。手执衠钢\footnote{纯钢。衠zh\=un,纯、真。}斧,担挽火麻\footnote{即大麻,又名线麻。纤维长而坚韧,可纺线制绳索、织渔网,或织麻布、造纸等。}绳。扳松劈枯树,争似此樵能。
\end{quotation}

猴王近前叫道:“老神仙!弟子起手\footnote{即稽首。古代跪拜礼的一种,也称道士举手向人行礼的动作。}。”

那樵汉慌忙丢了斧,转身回礼道:“不当人\footnote{也作“不当人子”。不当价。犹言罪过。}!不当人!我拙汉衣食不全,怎敢当‘神仙’二字?”

猴王道:“你不是神仙,如何说出神仙的话来?”

樵夫道:“我说甚么神仙话?”

猴王道:“我才来至林边,只听的你说:‘相逢处,非仙即道,静坐讲《黄庭》。’《黄庭》乃道德真言,非神仙而何?”

樵夫笑道:“实不瞒你说,这个词名做《满庭芳》,乃一神仙教我的。那神仙与我舍下相邻。他见我家事劳苦,日常烦恼,教我遇烦恼时,即把这词儿念念,一则散心,二则解困。我才有些不足处思虑,故此念念。不期被你听了。”

猴王道:“你家既与神仙相邻,何不从他修行?学得个不老之方,却不是好?”

樵夫道:“我一生命苦,自幼蒙父母养育,至八九岁才知人事,不幸父丧,母亲居孀。再无兄弟姊妹,只我一人,没奈何,早晚侍奉。如今母老,一发不敢抛离。却又田园荒芜,衣食不足,只得斫\footnote{zhu\'o,砍。}两束柴薪,挑向市廛之间,货\footnote{卖。}几文钱,籴\footnote{d\'i,买进谷物。}几升米,自炊自造,安排些茶饭,供养老母,所以不能修行。”

猴王道:“据你说起来,乃是一个行孝的君子,向后必有好处。但望你指与我那神仙住处,却好拜访去也。”

樵夫道:“不远,不远。此山叫做灵台方寸山\footnote{此处原注“灵台方寸,心也”。},山中有座斜月三星洞\footnote{此处原注“斜月象一勾,三星象三点也。是心。言学仙不必在远,只在此心”。}。那洞中有一个神仙,称名须菩提祖师。那祖师出去的徒弟也不计其数,见今还有三四十人从他修行。你顺那条小路儿,向南行七八里远近,即是他家了。”

猴王用手扯住樵夫道:“老兄,你便同我去去。若还得了好处,决不忘你指引之恩。”

樵夫道:“你这汉子,甚不通变。我方才这般与你说了,你还不省?假若我与你去了,却不误了我的生意?老母何人奉养?我要斫柴,你自去,自去。”

猴王听说,只得相辞。出深林,找上路径,过一山坡,约有七八里远,果然望见一座洞府。挺身观看,真好去处!但见:

\begin{quotation}
烟霞散彩,日月摇光。千株老柏,万节修篁。千株老柏,带雨半空青冉冉;万节修篁,含烟一壑色苍苍。门外奇花布锦,桥边瑶草喷香。石崖突兀青苔润,悬壁高张翠藓长。时闻仙鹤唳,每见凤凰翔。仙鹤唳时,声振九皋\footnote{曲折深远的沼泽。}霄汉远;凤凰翔起,翎毛五色彩云光。玄猿白鹿随隐见,金狮玉象任行藏。细观灵福地,真个赛天堂!
\end{quotation}

又见那洞门紧闭,静悄悄杳无人迹。忽回头,见崖头立一石碑,约有三丈余高,八尺余阔,上有一行十个大字,乃是:

\begin{quotation}
灵台方寸山,斜月三星洞。
\end{quotation}

美猴王十分欢喜,道:“此间人果是朴实,果有此山此洞。”看够多时,不敢敲门。且去跳上松枝梢头,摘松子吃了顽耍。

少顷间,只听得呀的一声,洞门开处,里面走出一个仙童,真是丰姿英伟,像貌清奇,比寻常俗子不同。但见他:

\begin{quotation}
髽髻\footnote{zhu\=a j\`i,梳在头顶两旁或脑后的发髻。}双丝绾,宽袍两袖风。貌和身自别,心与相俱空。

物外长年客,山中永寿童。一尘全不染,甲子任翻腾。
\end{quotation}

那童子出得门来,高叫道:“甚么人在此搔扰?”

猴王扑的跳下树来,上前躬身道:“仙童,我是个访道学仙之弟子,更不敢在此搔扰。”

仙童笑道:“你是个访道的么?”

猴王道:“是。”

童子道:“我家师父正才下榻,登坛讲道,还未说出原由,就教我出来开门,说,‘外面有个修行的来了,可去接待接待。’想必就是你了?”

猴王笑道:“是我,是我。”

童子道:“你跟我进来。”

这猴王整衣端肃,随童子径入洞天深处观看:一层层深阁琼楼,一进进珠宫贝阙,说不尽那静室幽居。直至瑶台之下。见那菩提祖师端坐在台上,两边有三十个小仙侍立台下。果然是:

\begin{quotation}
大觉金仙\footnote{因宋徽宗时期废除佛教而对释迦牟尼佛的改称。这里形容须菩提祖师。}没垢姿,西方妙相祖菩提。

不生不灭三三行,全气全神万万慈。

空寂自然随变化,真如本性任为之。

与天同寿庄严体,历劫\footnote{佛教语。经历宇宙的成毁。后形容经历各种灾难。劫,宇宙在时间上的一成一毁。}明心大法师。
\end{quotation}

美猴王一见,倒身下拜,磕头不计其数,口中只道:“师父!师父!我弟子志心朝礼!志心朝礼!”

祖师道:“你是那方人氏?且说个乡贯姓名明白,再拜。”

猴王道:“弟子乃东胜神洲傲来国花果山水帘洞人氏。”

祖师喝令:“赶出去!他本是个撒诈捣虚之徒,那里修甚么道果!”

猴王慌忙磕头不住道:“弟子是老实之言,决无虚诈。”

祖师道:“你既老实,怎么说东胜神洲?那去处到我这里,隔两重大海,一座南赡部洲,如何就得到此?”

猴王叩头道:“弟子飘洋过海,登界游方\footnote{登上天界,游历四方。指周游世界。},有十数个年头,方才访到此处。”

祖师道:“既是逐渐行来的,也罢。你姓甚么?”

猴王又道:“我无性。人若骂我,我也不恼;若打我,我也不嗔,只是赔个礼儿就罢了。一生无性。”

祖师道:“不是这个性。你父母原来姓甚么?”

猴王道:“我也无父母。”

祖师道:“既无父母,想是树上生的?”

猴王道:“我虽不是树上生,却是石里长的。我只记得花果山上有一块仙石,其年石破,我便生也。”

祖师闻言暗喜,道:“这等说,却是个天地生成的。你起来走走我看。”

猴王纵身跳起,拐呀拐的,走了两遍。祖师笑道:“你身躯虽是鄙陋,却像个食松果的猢狲。我与你就身上取个姓氏,意思教你姓‘猢’。‘猢’字去了个兽旁,乃是个古月。古者,老也;月者,阴也。老阴不能化育\footnote{产生发育、变化生长。}。教你姓‘狲’倒好。‘狲’字去了兽旁,乃是个子系\footnote{二字合起来即繁体的“孫”字。}。子者,儿男也;系者,婴细也。正合婴儿之本论。教你姓‘孙’罢。”

猴王听说,满心欢喜,朝上叩头道:“好!好!好!今日方知姓也。万望师父慈悲,既然有姓,再乞赐个名字,却好呼唤。”

祖师道:“我门中有十二个字,分派起名,到你乃第十辈之小徒矣。”

猴王道:“那十二个字?”

祖师道:“乃‘广、大、智、慧、真、如、性、海、颖、悟、圆、觉’十二字。排到你,正当‘悟’字。与你起个法名叫做‘孙悟空’,好么?”

猴王笑道:“好!好!好!自今就叫做孙悟空也!”正是:

\begin{quotation}
鸿蒙初辟原无姓,打破顽空\footnote{佛道教语。指一种僵死、没有生机的修炼状态。}须悟空。
\end{quotation}

毕竟不知向后修些甚么道果,且听下回分解。

\chapter[悟彻菩提真妙理\ 断魔归本合元神]{悟彻菩提真妙理\\断魔归本合元神}

话表美猴王得了姓名,怡然踊跃,对菩提前作礼启谢。那祖师即命大众引孙悟空出二门外,教他洒扫应对、进退周旋之节。众仙奉行而出。悟空到门外,又拜了大众师兄,就于廊庑之间,安排寝处。次早,与众师兄学言语礼貌,讲经论道,习字焚香,每自如此。闲时即扫地锄园,养花修树,寻柴燃火,挑水运浆。凡所用之物,无一不备。在洞中不觉倏\footnote{sh\=u,副词。相当于“迅速”、“忽然”。}六七年。一日,祖师登坛高坐,唤集诸仙,开讲大道。真个是:

\begin{quotation}
天花乱坠,地涌金莲。妙演三乘\footnote{指佛教引导众生达到解脱的三种方法、途径或教义。乘,比喻载众生到达彼岸的教法。}教,精微万法全。慢摇麈尾\footnote{用麈尾做的一类工具,形似扇子,也可用于驱虫、拂尘。麈zh\v{u},鹿类动物。}喷珠玉,响振雷霆动九天。说一会道,讲一会禅,三家\footnote{指儒、释、道。}配合本如然。开明一字皈诚理,指引无生了性玄。
\end{quotation}

孙悟空在旁闻讲,喜得他抓耳挠腮,眉花眼笑。忍不住手之舞之,足之蹈之。忽被祖师看见,叫孙悟空道:“你在班中,怎么颠狂跃舞,不听我讲?”

悟空道:“弟子诚心听讲,听到老师父妙音处,喜不自胜,故不觉作此踊跃之状。望师父恕罪!”

祖师道:“你既识妙音,我且问你,你到洞中多少时了?”

悟空道:“弟子本来懵懂,不知多少时节。只记得灶下无火,常去山后打柴,见一山好桃树,我在那里吃了七次饱桃矣。”

祖师道:“那山唤名烂桃山。你既吃七次,想是七年了。你今要从我学些甚么道?”

悟空道:“但凭尊师教诲,只是有些道气儿,弟子便就学了。”

祖师道:“‘道’字门中有三百六十旁门,旁门皆有正果。不知你学那一门哩?”

悟空道:“凭尊师意思。弟子倾心听从。”

祖师道:“我教你个‘术’字门中之道,如何?”

悟空道:“术门之道怎么说?”

祖师道:“‘术’字门中,乃是些请仙扶鸾\footnote{即扶乩(j\=i),一种占卜活动。术士制丁字形木架,其直端顶部悬锥下垂。架放在沙盘上,由两人各以食指分扶横木两端,依法请神,木架的下垂部分即在沙上画成文字,作为神的启示。传说神仙来时驾凤乘鸾,故名。},问卜揲蓍\footnote{sh\'e sh\=i,数蓍草。古代问卜的一种方式。},能知趋吉避凶之理。”

悟空道:“似这般可得长生么?”

祖师道:“不能,不能。”

悟空道:“不学,不学。”

祖师又道:“教你‘流’字门中之道,如何?”

悟空又问:“‘流’字门中,是甚义理?”

祖师道:“‘流’字门中,乃是儒家、释家、道家、阴阳家、墨家、医家,或看经,或念佛,并朝真降圣\footnote{朝真,道教谓朝见真人。降圣,谓帝王诞生。}之类。”

悟空道:“似这般可得长生么?”

祖师道:“若要长生,也似‘壁里安柱’。”

悟空道:“师父,我是个老实人,不晓得打市语\footnote{行话。}。怎么谓之‘壁里安柱’?”

祖师道:“人家盖房,欲图坚固,将墙壁之间立一顶柱,有日大厦将颓,他必朽矣。”

悟空道:“据此说,也不长久。不学!不学!”

祖师道:“教你‘静’字门中之道,如何?”

悟空道:“‘静’字门中,是甚正果?”

祖师道:“此是休粮守谷,清静无为,参禅打坐,戒语持斋,或睡功,或立功,并入定坐关之类。”

悟空道:“这般也能长生么?”

祖师道:“也似‘窑头土坯’。”

悟空笑道:“师父果有些滴澾。一行说我不会打市语。怎么谓之‘窑头土坯’?”

祖师道:“就如那窑头上,造成砖瓦之坯,虽已成形,尚未经水火煅炼,一朝大雨滂沱,他必烂矣。”

悟空道:“也不长远。不学!不学!”

祖师道:“教你‘动’字门\footnote{此处主要指道家房中术。}中之道,如何?”

悟空道:“动门之道,却又怎么?”

祖师道:“此是有为有作,采阴补阳,攀弓踏弩,摩脐过气,用方炮制,烧茅打鼎,进红铅\footnote{旧时术士称妇女初次月经或其炼取物为红铅。},炼秋石\footnote{丹药名,从童男童女尿液中提炼的春药。},并服妇乳之类。”

悟空道:“似这等也得长生么?”

祖师道:“此欲长生,亦如‘水中捞月’。”

悟空道:“师父又来了!怎么叫做‘水中捞月’?”

祖师道:“月在长空,水中有影,虽然看见,只是无捞摸处,到底只成空耳。”

悟空道:“也不学!不学!”

祖师闻言,咄的一声,跳下高台,手持戒尺,指定悟空道:“你这猢狲,这般不学,那般不学,却待怎么?”

走上前,将悟空头上打了三下,倒背着手,走入里面,将中门关了,撇下大众而去。諕\footnote{xi\`a,古同“吓”,使人害怕。}得那一班听讲的人人惊惧,皆怨悟空道:“你这泼猴,十分无状!师父传你道法,如何不学?却与师父顶嘴!这番冲撞了他,不知几时才出来呵!”

此时俱甚报怨\footnote{义同“抱怨”。}他,又鄙贱嫌恶他。悟空一些儿也不恼,只是满脸陪笑\footnote{义同“赔笑”。}。原来那猴王已打破盘中之谜,暗暗在心,所以不与众人争竞,只是忍耐无言。祖师打他三下者,教他三更时分存心;倒背着手,走入里面,将中门关上者,教他从后门进步\footnote{向前行步。},秘处传他道也。

当日悟空与众等喜喜欢欢,在三星仙洞之前盼望天色,急不能到晚。及黄昏时,却与众就寝,假合眼,定息存神。山中又没支更传箭\footnote{打更报时。传箭,古用铜壶滴漏计时,通过水平面箭上的刻度来判断时刻。},不知时分,只自家将鼻孔中出入之气调定。约到子时前后,轻轻的起来,穿了衣服,偷开前门,躲离大众,走出外,抬头观看。正是那:

\begin{quotation}
月明清露冷,八极迥无尘\footnote{八极,八方极远之地。迥,遥远。无尘,表示超尘脱俗。}。深树幽禽宿,源头水溜汾\footnote{li\`u f\'en,形容水大而湍急。溜,湍急。汾,大。}。

飞萤光散影,过雁字排云。正直三更候,应该访道真。
\end{quotation}

你看他从旧路径至后门外,只见那门儿半开半掩。悟空喜道:“老师父果然注意与我传道,故此开着门也。”即拽步近前,侧身进得门里,直走到祖师寝榻之下。见祖师踡跼\footnote{qu\'an j\'u,屈曲。}身躯,朝里睡着了。悟空不敢惊动,即跪在榻前。那祖师不多时觉来,舒开两足,口中自吟道:

\begin{quotation}
难难难!道最玄,莫把金丹作等闲。

不遇至人\footnote{道家指超凡脱俗,达到无我境界的人。}传妙诀,空言口困舌头干。
\end{quotation}

悟空应声叫道:“师父,弟子在此跪候多时。”

祖师闻得声音是悟空,即起披衣盘坐,喝道:“这猢狲!你不在前边去睡,却来我这后边作甚?”

悟空道:“师父昨日坛前对众相允,教弟子三更时候,从后门里传我道理,故此大胆,径拜老爷榻下。”

祖师听说,十分欢喜,暗自寻思道:“这厮果然是个天地生成的!不然,何就打破我盘中之暗谜也?”

悟空道:“此间更无六耳\footnote{指第三者。},止是弟子一人,望师父大舍慈悲,传与我长生之道罢,永不忘恩!”

祖师道:“你今有缘,我亦喜说。既识得盘中暗谜,你近前来,仔细听之,当传与你长生之妙道也。”

悟空叩头谢了,洗耳用心,跪于榻下。

祖师云:

\begin{quotation}
显密圆通真妙诀,惜修性命\footnote{性,指人的精神、意识等层面;命,对应肉体、身体机能等层面。}无他说。都来总是精炁神\footnote{道教认为先天的精、炁、神,是修炼的基础。炁q\`i,同“气”。},谨固牢藏休漏泄。休漏泄,体中藏,汝受吾传道自昌。口诀记来多有益,屏除邪欲得清凉。得清凉,光皎洁,好向丹台\footnote{道教指神仙的居处。}赏明月。月藏玉兔日藏乌,自有龟蛇相盘结。相盘结,性命坚,却能火里种金莲。攒簇五行颠倒用,工\footnote{通“功”。}完随作佛和仙。
\end{quotation}

此时说破根源,悟空心灵福至,切切记了口诀,对祖师拜谢深恩,即出后门观看。但见东方天色微舒白,西路金光大显明。依旧路,转到前门,轻轻的推开进去,坐在原寝之处,故将床铺摇响道:“天光了!天光了!起耶!”

那大众还正睡哩,不知悟空已得了好事。当日起来打混\footnote{含混过日,做事不认真。},暗暗维持,子前午后,自己调息。

却早过了三年,祖师复登宝座,与众说法。谈的是公案比语\footnote{公案,佛教禅宗指前辈祖师的言行范例。比语,比方的话或故事。},论的是外像包皮\footnote{外像,佛教语,指显露在外表上的善恶美丑和言语行动。包皮,指表面现象。}。忽问:“悟空何在?”

悟空近前跪下:“弟子有。”

祖师道:“你这一向修些甚么道来?”

悟空道:“弟子近来法性颇通,根源亦渐坚固矣。”

祖师道:“你既通法性,会得根源,已注神体,却只是防备着‘三灾利害’。”

悟空听说,沉吟良久道:“师父之言谬矣。我尝闻道高德隆,与天同寿;水火既济,百病不生,却怎么有个‘三灾利害’?”

祖师道:“此乃非常之道:夺天地之造化,侵日月之玄机;丹成之后,鬼神难容。虽驻颜益寿,但到了五百年后,天降雷灾打你,须要见性明心,预先躲避。躲得过,寿与天齐;躲不过,就此绝命。再五百年后,天降火灾烧你。这火不是天火,亦不是凡火,唤做‘阴火’。自本身涌泉穴下烧起,直透泥垣宫\footnote{即泥丸宫,指人脑。古人称脑神名精根,字泥丸,居于泥丸宫。},五脏成灰,四肢皆朽,把千年苦行,俱为虚幻。再五百年,又降风灾吹你。这风不是东西南北风,不是和薰金朔风\footnote{一年四季,风各有名。春为和风,夏为薰风,秋为金风,冬为朔风。},亦不是花柳松竹风,唤做‘赑风’\footnote{巨风。佛教所称大三灾之一的风灾名。b\`i}。自囟门\footnote{婴儿头顶骨未合缝的地方,在头顶的前部中央,也叫脑门、顶门。x\`in}中吹入六腑,过丹田,穿九窍,骨肉消疏,其身自解。所以都要躲过。”

悟空闻说,毛骨悚然,叩头礼拜道:“万望老爷垂悯,传与躲避三灾之法,到底不敢忘恩。”

祖师道:“此亦无难,只是你比他人不同,故传不得。”

悟空道:“我也头圆顶天,足方履地,一般有九窍四肢,五脏六腑,何以比人不同?”

祖师道:“你虽然像人,却比人少腮。”

原来那猴子孤拐面\footnote{上部凸出、下部尖削的脸。},凹脸尖嘴。悟空伸手一摸,笑道:“师父没成算!我虽少腮,却比人多这个嗉袋\footnote{猿猴类、啮齿类动物的颊囊,用于暂时存放食物。},亦可准折过也。”

祖师说:“也罢,你要学那一般?有一般天罡[106]数,该三十六般变化;有一般地煞数,该七十二般变化。”悟空道:“弟子愿多里捞摸[107],学一个地煞变化罢。”祖师道:“既如此,上前来,传与你口诀。”遂附耳低言,不知说了些甚么妙法。这猴王——也是他一窍通时百窍通,当时习了口诀,自修自炼,将七十二般变化都学成了。

忽一日,祖师与众门人在三星洞前戏玩晚景。祖师道:“悟空,事成了未曾?”悟空道:“多蒙师父海恩,弟子功果完备,已能霞举飞升也。”祖师道:“你试飞举我看。”悟空弄本事,将身一耸,打了个连扯跟头[108],跳离地有五六丈,踏云霞去够有顿饭之时,返复不尚三里远近,落在面前,扠手[109]道:“师父,这就是飞举腾云了。”祖师笑道:“这个算不得腾云,只算得爬云而已。自古道:‘神仙朝游北海暮苍梧。’似你这半日,去不上三里,即爬云也还算不得哩!”悟空道:“怎么为‘朝游北海暮苍梧’?”祖师道:“凡腾云之辈,早辰起自北海,游过东海、西海、南海,复转苍梧,苍梧者,却是南海零陵之语话也。将四海之外,一日都游遍,方算得腾云。”悟空道:“这个却难!却难!”祖师道:“世上无难事,只怕有心人。”

悟空闻得此言,叩头礼拜,启道:“师父,‘为人须为彻’,索性舍个大慈悲,将此腾云之法,一发传与我罢,决不敢忘恩。”祖师道:“凡诸仙腾云,皆跌足[110]而起,你却不是这般。我才见你去,连扯方才跳上。我今只就你这个势,传你个‘筋斗云’罢。”悟空又礼拜恳求,祖师却又传个口诀道:“这朵云,捻着诀,念动真言,攒紧了拳,将身一抖,跳将起来,一筋斗就有十万八千里路哩!”大众听说,一个个嘻嘻笑道:“悟空造化!若会这个法儿,与人家当铺兵[111],送文书,递报单[112],不管那里都寻了饭吃!”师徒们天昏各归洞府。这一夜,悟空即运神炼法,会了筋斗云。逐日家[113]无拘无束,自在逍遥,此亦长生之美。

一日,春归夏至,大众都在松树下会讲多时。大众道:“悟空,你是那世修来的缘法,前日老师父附耳低言,传与你的躲三灾变化之法,可都会么?”悟空笑道:“不瞒诸兄长说,一则是师父传授,二来也是我昼夜殷勤,那几般儿都会了。”大众道:“趁此良时,你试演演,让我等看看。”悟空闻说,抖擞精神,卖弄手段道:“众师兄请出个题目,要我变化甚么?”大众道:“就变棵松树罢。”悟空捻着诀,念动咒语,摇身一变,就变做一棵松树。真个是:
\begin{quotation}
郁郁含烟贯四时,凌云直上秀贞姿。

全无一点妖猴像,尽是经霜耐雪枝。
\end{quotation}
大众见了,鼓掌呵呵大笑,都道:“好猴儿!好猴儿!”

不觉的嚷闹惊动了祖师。祖师急拽杖出门来问道:“是何人在此喧哗?”大众闻呼,慌忙检束,整衣向前。悟空也现了本相,杂在丛中道:“启上尊师,我等在此会讲,更无外姓喧哗。”祖师怒喝道:“你等大呼小叫,全不像个修行的体段!修行的人,口开神气散,舌动是非生。如何在此嚷笑?”大众道:“不敢瞒师父,适才孙悟空演变化耍子,教他变棵松树,果然是棵松树,弟子每[114]俱称扬喝采,故高声惊冒尊师,望乞恕罪。”祖师道:“你等起去。叫悟空过来!——我问你,弄甚么精神[115]?变甚么松树?这个工夫,可好在人前卖弄?假如你见别人有,不要求他?别人见你有,必然求你。你若畏祸,却要传他;若不传他,必然加害:你之性命又不可保。”悟空叩头道:“只望师父恕罪!”

祖师道:“我也不罪你,但只是你去罢。”悟空闻此言,满眼堕泪道:“师父教我往那里去?”祖师道:“你从那里来,便从那里去就是了。”悟空顿然醒悟道:“我自东胜神洲傲来国花果山水帘洞来的。”祖师道:“你快回去,全你性命;若在此间,断然不可!”悟空领罪:“上告尊师,我也离家有二十年矣,虽是回顾旧日儿孙,但念师父厚恩未报,不敢去。”祖师道:“那里甚么恩义?你只不惹祸,不牵带我就罢了。”

悟空见没奈何,只得拜辞,与众相别。祖师道:“你这去,定生不良。凭你怎么惹祸行凶,却不许说是我的徒弟。你说出半个字来,我就知之,把你这猢狲剥皮锉骨,将神魂贬在九幽之处[116],教你万劫不得翻身!”悟空道:“决不敢题起师父一字,只说是我自家会的便罢。”悟空谢了,即抽身,捻着诀,丢[117]个连扯,纵起筋斗云,径回东胜。

那里消一个时辰,早看见花果山水帘洞。美猴王自知快乐,暗暗的自称道:
\begin{quotation}
去时凡骨凡胎重,得道身轻体亦轻。

举世无人肯立志,立志修玄玄自明。

当年过海波难进,今日回来甚易行。

别语叮咛还在耳,何期顷刻见东溟。
\end{quotation}
悟空按下云头,直至花果山。找路而走,忽听得鹤唳猿啼,鹤唳声冲霄汉外,猿啼悲切甚伤情,即开口叫道:“孩儿们,我来了也!”

那崖下石坎边,花草中,树木里,若大若小之猴,跳出千千万万,把个美猴王围在当中,叩头叫道:“大王,你好宽心!怎么一去许久?把我们俱闪在这里,望你诚如饥渴!近来被一妖魔在此欺虐,强要占我们水帘洞府,是我等舍死忘生,与他争斗。这些时,被那厮抢了我们家火[118],捉了许多子侄,教我们昼夜无眠,看守家业。幸得大王来了!大王若再年载不来,我等连山洞尽属他人矣!”悟空闻说,心中大怒道:“是甚么妖魔,辄敢无状!你且细细说来,待我寻他报仇。”众猴叩头:“告上大王,那厮自称混世魔王,住居在直北下。”悟空道:“此间到他那里,有多少路程?”众猴道:“他来时云,去时雾,或风或雨,或电或雷,我等不知有多少路。”悟空道:“既如此,你们休怕,且自顽耍,等我寻他去来!”

好猴王,将身一纵,跳起去,一路筋斗,直至北下观看,见一座高山,真是十分崄峻。好山:
\begin{quotation}
笔峰挺立,曲涧深沉。笔峰挺立透空霄,曲涧深沉通地户。两崖花木争奇,几处松篁斗翠。左边龙,熟熟驯驯;右边虎,平平伏伏。每见铁牛耕,常有金钱种。幽禽睍睆[119]声,丹凤朝阳立。石磷磷,波净净,古怪跷蹊真恶狞。世上名山无数多,花开花谢蘩还众。争如此景永长存,八节[120]四时浑不动。诚为三界坎[121]源山,滋养五行水脏洞。
\end{quotation}
美猴王正然观看景致,只听得有人言语。径自下山寻觅,原来那陡崖之前,乃是那水脏洞。洞门外有几个小妖跳舞,见了悟空就走。

悟空道:“休走!借你口中言,传我心内事。我乃正南方花果山水帘洞洞主。你家甚么混世鸟魔,屡次欺我儿孙,我特寻来,要与他见个上下!”那小妖听说,疾忙跑入洞里,报道:“大王!祸事了!”魔王道:“有甚祸事?”小妖道:“洞外有猴头,称为花果山水帘洞洞主。他说你屡次欺他儿孙,特来寻你,见个上下哩。”魔王笑道:“我常闻得那些猴精说他有个大王,出家修行去,想是今番来了。你们见他怎生打扮?有甚器械?”小妖道:“他也没甚么器械,光着个头,穿一领红色衣,勒一条黄丝绦,足下踏一对乌靴,不僧不俗,又不像道士神仙,赤手空拳,在门外叫哩。”魔王闻说:“取我披挂、兵器来!”那小妖即时取出。那魔王穿了甲胄,绰刀在手,与众妖出得门来,即高声叫道:“那个是水帘洞洞主?”悟空急睁睛观看,只见那魔王:

头戴乌金盔,映日光明;身挂皂罗[122]袍,迎风飘荡。下穿着黑铁甲,紧勒皮条;足踏着花褶靴,雄如上将。腰广十围,身高三丈。手执一口刀,锋刃多明亮。称为混世魔,磊落凶模样。

猴王喝道:“这泼魔这般眼大,看不见老孙!”魔王见了,笑道:“你身不满四尺,年不过三旬,手内又无兵器,怎么大胆猖狂,要寻我见甚么上下?”悟空骂道:“你这泼魔,原来没眼!你量我小,要大却也不难。你量我无兵器,我两只手够着天边月哩!你不要怕,只吃老孙一拳!”纵一纵,跳上去,劈脸就打。那魔王伸手架住道:“你这般矬矮,我这般高长,你要使拳,我要使刀,使刀就杀了你,也吃人笑[123],待我放下刀,与你使路拳看。”悟空道:“说得是。好汉子,走来!”那魔王丢开架子[124]便打,这悟空钻进去相撞相迎。

他两个拳捶脚踢,一冲一撞。原来长拳空大,短簇坚牢。那魔王被悟空掏短胁,撞丫裆[125],几下筋节[126],把他打重了。他闪过,拿起那板大的钢刀,望悟空劈头就砍。悟空急撤身,他砍了一个空。悟空见他凶猛,即使身外身法,拔一把毫毛,丢在口中嚼碎,望空喷去,叫一声“变”,即变做三二百个小猴,周围攒簇。原来人得仙体,出神变化无方。不知这猴王自从了道之后,身上有八万四千毛羽,根根能变,应物随心。那些小猴,眼乖会跳,刀来砍不着,枪去不能伤。你看他前踊后跃,钻上去,把个魔王围绕,抱的抱,扯的扯,钻裆的钻裆,扳脚的扳脚,踢打,挦毛[127],抠眼睛,捻鼻子,抬鼓弄[128],直打做一个攒盘[129]。这悟空才去夺得他的刀来,分开小猴,照顶门一下,砍为两段。领众杀进洞中,将那大小妖精,尽皆剿灭。却把毫毛一抖,收上身来。

又见那收不上身者,却是那魔王在水帘洞擒去的小猴。悟空道:“汝等何为到此?”约有三五十个,都含泪道:“我等因大王修仙去后,这两年被他争吵,把我们都摄将来,那不是我们洞中的家火?石盆、石碗都被这厮拿来也。”悟空道:“既是我们的家火,你们都搬出外去。”随即洞里放起火来,把那水脏洞烧得枯干,尽归了一体。对众道:“汝等跟我回去。”众猴道:“大王,我们来时,只听得耳边风响,虚飘飘到于此地,更不识路径,今怎得回乡?”悟空道:“这是他弄的个术法儿,有何难也!我如今一窍通,百窍通,我也会弄。你们都合了眼,休怕!”

好猴王,念声咒语,驾阵狂风,云头落下,叫:“孩儿们,睁眼。”众猴脚躧[130]实地,认得是家乡,个个欢喜,都奔洞门旧路。那在洞众猴,都一齐簇拥同入,分班序齿,礼拜猴王。安排酒果,接风贺喜,启问降魔救子之事。悟空备细言了一遍,众猴称扬不尽道:“大王去到那方,不意学得这般手段!”悟空又道:“我当年别汝等,随波逐流,飘过东洋大海,径至南赡部洲,学成人像,着此衣,穿此履,摆摆摇摇,云游了八九年余,更不曾有道;又渡西洋大海,到西牛贺洲地界,访问多时,幸遇一老祖,传了我与天同寿的真功果,不死长生的大法门。”众猴称贺,都道:“万劫难逢也!”悟空又笑道:“小的们,又喜我这一门皆有姓氏。”众猴道:“大王姓甚?”悟空道:“我今姓孙,法名悟空。”众猴闻说,鼓掌忻[131]然道:“大王是老孙,我们都是二孙、三孙、细孙、小孙,一家孙、一国孙、一窝孙矣!”都来奉承老孙,大盆小碗的椰子酒、葡萄酒、仙花、仙果,真个是合家欢

\begin{quotation}
贯通一姓身归本,只待荣迁仙箓[132]名。
\end{quotation}
毕竟不知怎生结果,居此界终始如何,且听下回分解。

\chapter[四海千山皆拱伏\ 九幽十类尽除名]{四海千山皆拱伏\\九幽十类尽除名}

却说美猴王荣归故里,自剿了混世魔王,夺了一口大刀,逐日操演武艺,教小猴砍竹为标[133],削木为刀,治旗幡,打哨子,一进一退,安营下寨,顽耍多时。忽然静坐处,思想道:“我等在此,恐作耍成真,或惊动人王,或有禽王、兽王认此犯头[134],说我们操兵造反,兴师来相杀,汝等都是竹竿木刀,如何对敌?须得锋利剑戟方可。如今奈何?”众猴闻说,个个惊恐道:“大王所见甚长,只是无处可取。”正说间,转上四个老猴,两个是赤尻马猴[135],两个是通臂猿猴,走在面前道:“大王若要治锋利器械,甚是容易。”悟空道:“怎见容易?”四猴道:“我们这山向东去有二百里水面,那厢乃傲来国界。那国界中有一王位,满城中军民无数,必有金银铜铁等匠作。大王若去那里,或买或造些兵器,教演我等,守护山场,诚所谓保泰长久之机也。”悟空闻说,满心欢喜道:“汝等在此顽耍,待我去来。”

好猴王,即纵筋斗云,霎时间过了二百里水面。果见那厢有座城池,六街三巷,万户千门,来来往往,人都在光天化日之下。悟空心中想道:“这里定有现成的兵器,我待下去买他几件,还不如使个神通觅他几件倒好。”他就捻起诀来,念动咒语,向巽地[136]上吸一口气,呼的吹将去,便是一阵狂风,飞沙走石,好惊人也:
\begin{quotation}
炮云[137]起处荡乾坤,黑雾阴霾大地昏。

江海波翻鱼蟹怕,山林树折虎狼奔。

诸般买卖无商旅,各样生涯不见人。

殿上君王归内院,阶前文武转衙门。

千秋宝座都吹倒,五凤高楼幌动根。
\end{quotation}
风起处,惊散了那傲来国君王,三市六街,都慌得关门闭户,无人敢走。悟空才按下云头,径闯入朝门里。直寻到兵器馆武库中,打开门扇看时,那里面无数器械:刀枪剑戟、斧钺毛镰、鞭钯挝简[138]、弓弩叉矛,件件俱备。一见甚喜道:“我一人能拿几何?还使个分身法搬将去罢。”好猴王,即拔一把毫毛,入口嚼烂,喷将出去,念动咒语,叫声“变”,变做千百个小猴,都乱搬乱抢;有力的拿五七件,力小的拿三二件,尽数搬个罄净。径踏云头,弄个摄法,唤转狂风,带领小猴俱回本处。

却说那花果山大小儿猴,正在那洞门外顽耍,忽听得风声响处,见半空中,丫丫叉叉[139],无边无岸的猴精,諕得都乱跑乱躲。少时,美猴王按落云头,收了云雾,将身一抖,收了毫毛,将兵器都乱堆在山前,叫道:“小的们!都来领兵器!”众猴看时,只见悟空独立在平阳之地,俱跑来叩头问故。悟空将前使狂风、搬兵器一应事说了一遍。众猴称谢毕,都去抢刀夺剑,挝[140]斧争枪,扯弓扳弩,吆吆喝喝,耍了一日。

次日,依旧排营。悟空会聚群猴,计有四万七千余口。早惊动满山怪兽,都是些狼虫虎豹、麖麂獐、狐狸獾狢、狮象狻猊、猩猩熊鹿、野豕山牛、羚羊青兕、狡儿神獒[141]……各样妖王,共有七十二洞,都来参拜猴王为尊。每年献贡,四时点卯[142]。也有随班操备的,也有随节征粮的,齐齐整整,把一座花果山造得似铁桶金城。各路妖王,又有进金鼓,进彩旗,进盔甲的,纷纷攘攘,日逐家[143]习武兴师。

美猴王正喜间,忽对众说道:“汝等弓弩熟谙,兵器精通,奈我这口刀着实榔槺[144],不遂我意,奈何?”四老猴上前启奏道:“大王乃是仙圣,凡兵是不堪用;但不知大王水里可能去得?”悟空道:“我自闻道之后,有七十二般地煞变化之功;筋斗云有莫大的神通;善能隐身遁身,起法摄法;上天有路,入地有门;步日月无影,入金石无碍;水不能溺,火不能焚。那些儿去不得?”四猴道:“大王既有此神通,我们这铁板桥下,水通东海龙宫。大王若肯下去,寻着老龙王,问他要件甚么兵器,却不趁心?”悟空闻言甚喜道:“等我去来。”

好猴王,跳至桥头,使一个闭水法,捻着诀,扑的钻入波中,分开水路,径入东洋海底。正行间,忽见一个巡海的夜叉,挡住问道:“那推水来的,是何神圣?说个明白,好通报迎接。”悟空道:“吾乃花果山天生圣人孙悟空,是你老龙王的紧邻,为何不识?”那夜叉听说,急转水晶宫传报道:“大王,外面有个花果山天生圣人孙悟空,口称是大王紧邻,将到宫也。”东海龙王敖广即忙起身,与龙子龙孙、虾兵蟹将出宫迎道:“上仙请进,请进。”直至宫里相见,上坐献茶毕,问道:“上仙几时得道,受何仙术?”悟空道:“我自生身之后,出家修行,得一个无生无灭之体。近因教演儿孙,守护山洞,奈何没件兵器。久闻贤邻享乐瑶宫贝阙,必有多余神器,特来告求一件。”

龙王见说,不好推辞,即着鳜都司取出一把大杆刀奉上。悟空道:“老孙不会使刀,乞另赐一件。”龙王又着鲌太尉,领鳝力士,抬出一杆九股叉来。悟空跳下来,接在手中,使了一路,放下道:“轻!轻!轻!又不趁手!再乞另赐一件。”龙王笑道:“上仙,你不曾看这叉,有三千六百斤重哩!”悟空道:“不趁手!不趁手!”龙王心中恐惧,又着鳊提督、鲤总兵抬出一柄画杆方天戟。那戟有七千二百斤重。悟空见了,跑近前接在手中,丢几个架子,撒两个解数[145],插在中间道:“也还轻!轻!轻!”老龙王一发害怕道:“上仙,我宫中只有这根戟重,再没甚么兵器了。”悟空笑道:“古人云:‘愁海龙王没宝哩!’你再去寻寻看。若有可意的,一一奉价[146]。”龙王道:“委的再无。”

正说处,后面闪过龙婆、龙女道:“大王,观看此圣,决非小可。我们这海藏[147]中,那一块天河定底的神珍铁,这几日霞光艳艳,瑞气腾腾,敢莫是该出现遇此圣也?”龙王道:“那是大禹治水之时,定江海浅深的一个定子,是一块神铁,能中何用?”龙婆道:“莫管他用不用,且送与他,凭他怎么改造,送出宫门便了。”老龙王依言,尽向悟空说了。悟空道:“拿出来我看。”龙王摇手道:“扛不动!抬不动!须上仙亲去看看。”悟空道:“在何处?你引我去。”

龙王果引导至海藏中间,忽见金光万道。龙王指定道:“那放光的便是。”悟空撩衣上前,摸了一把,乃是一根铁柱子,约有斗来粗,二丈有余长。他尽力两手挝过道:“忒粗忒长些!再短细些方可用。”说毕,那宝贝就短了几尺,细了一围。悟空又掂一掂道:“再细些更好!”那宝贝真个又细了几分。悟空十分欢喜,拿出海藏看时,原来两头是两个金箍,中间乃一段乌铁;紧挨箍有镌成的一行字,唤做“如意金箍棒,重一万三千五百斤”。心中暗喜道:“想必这宝贝如人意!”一边走,一边心思口念,手掂着道:“再短细些更妙!”拿出外面,只有二丈长短,碗口粗细。你看他弄神通,丢开解数,打转水晶宫里,諕得老龙王胆战心惊,小龙子魂飞魄散;龟鳖鼋鼍[148]皆缩颈,鱼虾鳌[149]蟹尽藏头。

悟空将宝贝执在手中,坐在水晶宫殿上,对龙王笑道:“多谢贤邻厚意。”龙王道:“不敢,不敢。”悟空道:“这块铁虽然好用,还有一说。”龙王道:“上仙还有甚说?”悟空道:“当时若无此铁,倒也罢了;如今手中既拿着他,身上更无衣服相趁[150],奈何?你这里若有披挂[151],索性送我一副,一总奉谢。”龙王道:“这个却是没有。”悟空道:“‘一客不犯二主’,若没有,我也定不出此门。”龙王道:“烦上仙再转一海,或者有之。”悟空又道:“‘走三家不如坐一家’,千万告求一副。”龙王道:“委的没有;如有即当奉承[152]。”悟空道:“真个没有,就和你试试此铁!”龙王慌了道:“上仙,切莫动手!切莫动手!待我看舍弟处可有,当送一副。”悟空道:“令弟何在?”龙王道:“舍弟乃南海龙王敖钦、北海龙王敖顺、西海龙王敖闰是也。”悟空道:“我老孙不去!不去!俗语谓‘赊三不敌见二’[153],只望你随高就低[154]的送一副便了。”老龙道:“不须上仙去。我这里有一面铁鼓,一口金钟;凡有紧急事,擂得鼓响,撞得钟鸣,舍弟们就顷刻而至。”悟空道:“既是如此,快些去擂鼓撞钟!”真个那鼍将便去撞钟,鳖帅即来擂鼓。

霎时,钟鼓响处,果然惊动那三海龙王,须臾来到,一齐在外面会着敖广道:“大哥,有甚紧事,擂鼓撞钟?”老龙道:“贤弟,不好说!有一个花果山甚么天生圣人,早间来认我做邻居,后要求一件兵器,献钢叉嫌小,奉画戟嫌轻。将一块天河定底神珍铁,自己拿出,丢了些解数。如今坐在宫中,又要索甚么披挂。我处无有,故响钟鸣鼓,请贤弟来。你们可有甚么披挂,送他一副,打发他出门去罢了。”敖钦闻言,大怒道:“我兄弟们,点起兵,拿他不是!”老龙道:“莫说拿!莫说拿!那块铁,挽着些儿就死,磕着些儿就亡;挨挨儿皮破,擦擦儿筋伤!”

西海龙王敖闰说:“二哥不可与他动手;且只凑副披挂与他,打发他出了门,启表奏上上天,天自诛也。”北海龙王敖顺道:“说的是。我这里有一双藕丝步云履哩。”西海龙王敖闰道:“我带了一副锁子黄金甲哩。”南海龙王敖钦道:“我有一顶凤翅紫金冠哩。”老龙大喜,引入水晶宫相见了,以此奉上。悟空将金冠、金甲、云履都穿戴停当,使动如意棒,一路打出去,对众龙道:“聒噪[155]!聒噪!”四海龙王甚是不平,一边商议进表上奏不题。

你看这猴王,分开水道,径回铁板桥头,蹿将上来,只见四个老猴领着众猴,都在桥边等候。忽然见悟空跳出波外,身上更无一点水湿,金灿灿的走上桥来,諕得众猴一齐跪下道:“大王,好华彩耶!好华彩耶!”悟空满面春风,高登宝座,将铁棒竖在当中。那些猴不知好歹,都来拿那宝贝,却便似蜻蜓撼铁树,分毫也不能禁动[156],一个个咬指伸舌道:“爷爷呀!这般重,亏你怎的拿来也!”悟空近前,舒开手,一把挝起,对众笑道:“物各有主。这宝贝镇于海藏中,也不知几千百年,可可的[157]今岁放光。龙王只认做是块黑铁,又唤做天河镇底神珍。那厮每都扛抬不动,请我亲去拿之。那时此宝有二丈多长,斗来粗细;被我挝他一把,意思嫌大,他就小了许多;再教小些,他又小了许多;再教小些,他又小了许多;急对天光看处,上有一行字,乃‘如意金箍棒,重一万三千五百斤’。你都站开,等我再叫他变一变看。”他将那宝贝掂在手中,叫:“小!小!小!”即时就小做一个绣花针儿相似,可以揌[158]在耳朵里面藏下。众猴骇然,叫道:“大王,还拿出来耍耍!”猴王真个去耳朵里拿出,托放掌上叫:“大!大!大!”即又大做斗来粗细,二丈长短。

他弄到欢喜处,跳上桥,走出洞外,将宝贝揝[159]在手中,使一个法天象地的神通,把腰一躬,叫声:“长!”他就长的高万丈,头如太山,腰如峻岭,眼如闪电,口似血盆,牙如剑戟;手中那棒,上抵三十三天,下至十八层地狱,把些虎豹狼虫,满山群怪,七十二洞妖王,都諕得磕头礼拜,战兢兢魄散魂飞。霎时收了法象,将宝贝还变做个绣花针儿,藏在耳内,复归洞府。慌得那各洞妖王,都来参贺。

此时遂大开旗鼓,响振铜锣。广设珍馐百味,满斟椰液萄浆,与众饮宴多时。却又依前教演。猴王将那四个老猴封为健将:将两个赤尻马猴唤做马、流[160]二元帅;两个通臂猿猴唤做崩、芭二将军。将那安营下寨、赏罚诸事,都付与四健将维持。他放下心,日逐腾云驾雾,遨游四海,行乐千山。施武艺,遍访英豪;弄神通,广交贤友。此时又会了个七弟兄,乃牛魔王、蛟魔王、鹏魔王、狮王、猕猴王、狨[161]王,连自家美猴王七个。日逐讲文论武,走斝[162]传觞,弦歌吹舞,朝去暮回,无般儿不乐。把那万里之遥,只当庭闱[163]之路,所谓点头径过三千里,扭腰八百有余程。

一日,在本洞分付四健将,安排筵宴,请六王赴饮,杀牛宰马,祭天享地。着众怪跳舞欢歌,俱吃得酩酊大醉。送六王出去,却又赏劳[164]大小头目,欹[165]在铁板桥边松阴之下,霎时间睡着。四健将领众围护,不敢高声。

只见那美猴王睡里见两人拿一张批文,上有“孙悟空”三字,走近身,不容分说,套上绳,就把美猴王的魂灵儿索了去,踉踉跄跄,直带到一座城边。猴王渐觉酒醒,忽抬头观看,那城上有一铁牌,牌上有三个大字,乃“幽冥界”。美猴王顿然醒悟道:“幽冥界乃阎王所居,何为到此?”那两人道:“你今阳寿该终,我两人领批,勾你来也。”猴王听说,道:“我老孙超出三界外,不在五行中,已不伏他管辖,怎么朦胧[166],又敢来勾我?”那两个勾死人[167]只管扯扯拉拉,定要拖他进去。这猴王恼起性来,耳朵中掣出宝贝,幌一幌,碗来粗细;略举手,把两个勾死人打为肉酱。自解其索,丢开手,抡着棒,打入城中。諕得那牛头鬼[168]东躲西藏,马面鬼南奔北跑,众鬼卒奔上森罗殿[169],报着:“大王!祸事!祸事!外面有一个毛脸雷公打将来了!”

慌得那十殿冥王[170]急整衣来看,见他相貌凶恶,即排下班次,应声高叫道:“上仙留名!上仙留名!”猴王道:“你既认不得我,怎么差人来勾我?”十王道:“不敢!不敢!想是差人差了。”猴王道:“我本是花果山水帘洞天生圣人孙悟空。你等是甚么官位?”十王躬身道:“我等是阴间天子十殿冥王。”悟空道:“快报名来,免打!”十王道:“我等是秦广王、楚江王、宋帝王、忤官王、阎罗王、平等王、泰山王、都市王、卞城王、转轮王。”悟空道:“汝等既登王位,乃灵显感应之类,为何不知好歹?我老孙修仙了道,与天齐寿,超升三界之外,跳出五行之中,为何着人拘我?”十王道:“上仙息怒。普天下同名同姓者多,敢是那勾死人错走了也?”悟空道:“胡说!胡说!常言道:‘官差吏差,来人不差[171]。’你快取生死簿子来我看!”十王闻言,即请上殿查看。

悟空执着如意棒,径登森罗殿上,正中间南面坐下。十王即命掌案的判官取出文簿来查。那判官不敢怠慢,便到司房里,捧出五六簿文书并十类[172]簿子,逐一查看。裸虫、毛虫、羽虫、昆虫、鳞介之属,俱无他名。又看到猴属之类,原来这猴似人相,不入人名;似裸虫,不居国界;似走兽,不伏麒麟管;似飞禽,不受凤凰辖,另有个簿子。悟空亲自检阅,直到那“槐”字一千三百五十号上,方注着孙悟空名字,乃天产石猴,该寿三百四十二岁,善终。悟空道:“我也不记寿数几何,且只销了名字便罢!取笔过来!”那判官慌忙捧笔,饱掭[173]浓墨。悟空拿过簿子,把猴属之类,但有名者,一概勾之,捽[174]下簿子道:“了帐!了帐!今番不伏你管了!”一路棒打出幽冥界。那十王不敢相近,都去翠云宫同拜地藏王菩萨,商量启表奏闻上天,不在话下。

这猴王打出城中,忽然绊着一个草纥繨[175],跌了个躘踵[176],猛的醒来,乃是南柯一梦。才觉伸腰,只闻得四健将与众猴高叫道:“大王,吃了多少酒,睡这一夜,还不醒来?”悟空道:“睡还小可,我梦见两个人,来此勾我,把我带到幽冥界城门之外,却才醒悟。是我显神通,直嚷到森罗殿,与那十王争吵,将我们的生死簿子看了,但有我等名号,俱是我勾了,都不伏那厮所辖也。”众猴磕头礼谢。自此,山猴多有不老者,以阴司无名故也。美猴王言毕前事,四健将报知各洞妖王,都来贺喜。不几日,六个义兄弟又来拜贺,一闻销名之故,又个个欢喜,每日聚乐不题。

却表启那高天上圣大慈仁者玉皇大天尊玄穹高上帝,一日,驾坐金阙云宫灵霄宝殿,聚集文武仙卿早朝之际,忽有丘弘济真人[177]启奏道:“万岁,通明殿外,有东海龙王敖广进表,听天尊宣诏。”玉皇传旨:“着宣来。”敖广宣至灵霄殿下,礼拜毕。旁有引奏仙童接上表文,玉皇从头看过。表曰:

水元下界东胜神洲东海小龙臣敖广启奏大天圣主玄穹高上帝君:近因花果山生、水帘洞住妖仙孙悟空者,欺虐小龙,强坐水宅,索兵器,施法施威;要披挂,骋凶骋势。惊伤水族,諕走龟鼍。南海龙战战兢兢,西海龙凄凄惨惨,北海龙缩首归降。臣敖广舒身下拜,献神珍之铁棒,凤翅之金冠,与那锁子甲、步云履,以礼送出。他仍弄武艺,显神通,但云:‘聒噪!聒噪!’果然无敌,甚为难制。臣今启奏,伏望圣裁。恳乞天兵,收此妖孽,庶使海岳清宁,下元安泰。谨奏。

圣帝览毕,传旨:“着龙神回海,朕即遣将擒拿。”老龙王顿首谢去。下面又有葛仙翁天师[178]启奏道:“万岁,有冥司秦广王赍奉[179]幽冥教主地藏王菩萨表文进上。”旁有传言玉女接上表文,玉皇亦从头看过。表曰:

幽冥境界,乃地之阴司。天有神而地有鬼,阴阳轮转;禽有生而兽有死,反复雌雄。生生化化,孕女成男,此自然之数,不能易也。今有花果山水帘洞天产妖猴孙悟空,逞恶行凶,不服拘唤。弄神通,打绝九幽鬼使;恃势力,惊伤十殿慈王。大闹森罗,强销名号。致使猴属之类无拘,猕猴之畜多寿;寂灭轮回,各无生死。贫僧具表,冒渎[180]天威。伏乞调遣神兵,收降此妖,整理阴阳,永安地府。谨奏。

玉皇览毕,传旨:“着冥君回归地府,朕即遣将擒拿。”秦广王亦顿首谢去。

大天尊宣众文武仙卿,问曰:“这妖猴是几年产育,何代出身,却就这般有道?”一言未已,班中闪出千里眼、顺风耳道:“这猴乃三百年前天产石猴。当时不以为然,不知这几年在何方修炼成仙,降龙伏虎,强销死籍也。”玉帝道:“那路神将下界收伏?”言未已,班中闪出太白长庚星[181],俯伏启奏道:“上圣,三界中凡有九窍者,皆可修仙。奈此猴乃天地育成之体,日月孕就之身,他也顶天履地,服露餐霞;今既修成仙道,有降龙伏虎之能,与人何以异哉?臣启陛下,可念生化之慈恩,降一道招安圣旨,把他宣来上界,授他一个大小官职,与他籍名在箓,拘束此间,若受天命,后再升赏;若违天命,就此擒拿。一则不动众劳师,二则收仙有道也。”玉帝闻言甚喜,道:“依卿所奏。”即着文曲星官修诏,着太白金星招安。

金星领了旨,出南天门外,按下祥云,直至花果山水帘洞,对众小猴道:“我乃天差天使,有圣旨在此,请你大王上界。快快报知!”洞外小猴一层层传至洞天深处,道:“大王,外面有一老人,背着一角[182]文书,言是上天差来的天使,有圣旨请你也。”美猴王听得大喜,道:“我这两日正思量要上天走走,却就有天使来请。”叫:“快请进来!”猴王急整衣冠,门外迎接。金星径入当中,面南立定道:“我是西方太白金星,奉玉帝招安圣旨下界,请你上天,拜受仙箓。”悟空笑道:“多感老星降临。”教:“小的们!安排筵宴款待。”金星道:“圣旨在身,不敢久留,就请大王同往,待荣迁之后,再从容叙也。”悟空道:“承光顾,空退[183]!空退!”即唤四健将,分付:“谨慎教演儿孙,待我上天去看看路,却好带你们上去同居住也。”四健将领诺。这猴王与金星纵起云头,升在空霄之上。正是那:

\begin{quotation}
高迁上品天仙位,名列云班宝箓中。
\end{quotation}

毕竟不知授个甚么官爵,且听下回分解。

\chapter[官封弼马心何足\ 名注齐天意未宁]{官封弼马心何足\\名注齐天意未宁}

那太白金星与美猴王同出了洞天深处一齐驾云而起。原来悟空筋斗云比众不同十分快疾把个金星撇在脑后先至南天门外。正欲收云前进被增长天王领着庞、刘、苟、毕、邓、辛、张、陶一路大力天丁枪刀剑戟挡住天门不肯放进。猴王道:“这个金星老儿乃奸诈之徒!既请老孙如何教人动刀动枪阻塞门路?”正嚷间金星倏到。悟空就觌面狠道:“你这老儿怎么哄我?被你说奉玉帝招安旨意来请却怎么教这些人阻住天门不放老孙进去?”金星笑道:“大王息怒。你自来未曾到此天堂却又无名众天丁又与你素不相识他怎肯放你擅入?等如今见了天尊授了仙录注了官名向后随你出入谁复挡也?”悟空道:“这等说也罢我不进去了。”金星又用手扯住道:“你还同我进去。”

将近天门金星高叫道:“那天门天将大小吏兵放开路者。此乃下界仙人我奉玉帝圣旨宣他来也。”这增长天王与众天丁俱才敛兵退避。猴王始信其言。同金星缓步入里观看。真个是:

初登上界乍入天堂。金光万道滚红霓瑞气千条喷紫雾。只见那南天门碧沉沉琉璃造就;明幌幌宝玉妆成。两边摆数十员镇天元帅一员员顶梁靠柱持铣拥旄;四下列十数个金甲神人一个个执戟悬鞭持刀仗剑。外厢犹可入内惊人:里壁厢有几根大柱柱上缠绕着金鳞耀日赤须龙;又有几座长桥桥上盘旋着彩羽凌空丹顶凤。

明霞幌幌映天光碧雾蒙蒙遮斗口。这天上有三十三座天宫乃遣云宫、毗沙宫、五明宫、太阳宫、花药宫、……一宫宫脊吞金稳兽;又有七十二重宝殿乃朝会殿、凌虚殿、宝光殿、天王殿、灵官殿、……一殿殿柱列玉麒麟。寿星台上有千千年不卸的名花;炼药炉边有万万载常青的绣草。又至那朝圣楼前绛纱衣星辰灿烂;芙蓉冠金璧辉煌。玉簪珠履紫绶金章。金钟撞动三曹神表进丹墀;天鼓鸣时万圣朝王参玉帝。又至那灵霄宝殿金钉攒玉户彩凤舞朱门。

复道回廊处处玲珑剔透;三檐四簇层层龙凤翱翔。上面有个紫巍巍明幌幌圆丢丢亮灼灼大金葫芦顶;下面有天妃悬掌扇玉女捧仙巾。恶狠狠掌朝的天将;气昂昂护驾的仙卿。正中间琉璃盘内放许多重重叠叠太乙丹;玛瑙瓶中插几枝弯弯曲曲珊瑚树。正是天宫异物般般有世上如他件件无。金阙银銮并紫府琪花瑶草暨琼葩。朝王玉兔坛边过参圣金乌着底飞。猴王有分来天境不堕人间点污泥。

太白金星领着美猴王到于灵霄殿外。不等宣诏直至御前朝上礼拜。悟空挺身在旁且不朝礼但侧耳以听金星启奏。金星奏道:“臣领圣旨已宣妖仙到了。”玉帝垂帘问曰:“那个是妖仙?”悟空却才躬身答道:“老孙便是!”仙卿们都大惊失色道:“这个野猴!怎么不拜伏参见辄敢这等答应道:‘老孙便是!’却该死了!该死了!”玉帝传旨道:“那孙悟空乃下界妖仙初得人身不知朝礼且姑恕罪。”众仙卿叫声“谢恩!”猴王却才朝上唱个大喏。玉帝宣文选武选仙卿看那处少甚官职着孙悟空去除授。旁边转过武曲星君启奏道:“天宫里各宫各殿各方各处都不少官只是御马监缺个正堂管事。”玉帝传旨道:“就除他做个‘弼马温’罢。”众臣叫谢恩他也只朝上唱个大喏。玉帝又差木德星君送他去御马监到任。

当时猴王欢欢喜喜与木德星官径去到任。事毕木德星官回宫。他在监里会聚了监丞、监副、典簿、力士大小官员人等查明本监事务止有天马千匹。乃是:

骅骝骐骥騄駬纤离;龙媒紫燕挟翼骕骦;駃騠银騔騕褭飞黄;騊駼翻羽赤兔光;逾辉弥景腾雾胜黄;追风绝地飞翻奔霄;逸飘赤电铜爵浮云;骢珑虎〔马剌〕绝尘紫鳞;四极大宛八骏九逸千里绝群:——此等良马一个个嘶风逐电精神壮踏雾登云气力长。

这猴王查看了文簿点明了马数。本监中典簿管征备草料;力士官管刷洗马匹、扎草、饮水、煮料;监丞、监副辅佐催办;弼马昼夜不睡滋养马匹。日间舞弄犹可夜间看管殷勤但是马睡的赶起来吃草;走的捉将来靠槽。那些天马见了他泯耳攒蹄倒养得肉膘肥满。不觉的半月有馀一朝闲暇众监官都安排酒席一则与他接风二则与他贺喜。

正在欢饮之间猴王忽停杯问曰:“我这‘弼马温’是个甚么官衔?”众曰:“官名就是此了。”又问:“此官是个几品?”众道:“没有品从。”猴王道:“没品想是大之极也。”众道:“不大不大只唤做‘未入流’。”猴王道:“怎么叫做‘未入流’?”众道:“末等。这样官儿最低最小只可与他看马。似堂尊到任之后这等殷勤喂得马肥只落得道声‘好’字如稍有些尪羸还要见责;再十分伤损还要罚赎问罪。”猴王闻此不觉心头火起咬牙大怒道:“这般藐视老孙!老孙在花果山称王称祖怎么哄我来替他养马?养马者乃后生小辈下贱之役岂是待我的?不做他!不做他!我将去也!”忽喇的一声把公案推倒耳中取出宝贝幌一幌碗来粗细一路解数直打出御马监径至南天门。众天丁知他受了仙录乃是个弼马温不敢阻当让他打出天门去了。

须臾按落云头回至花果山上。只见那四健将与各洞妖王在那里操演兵卒。这猴王厉声高叫道:“小的们!老孙来了!”一群猴都来叩头迎接进洞天深处请猴王高登宝位一壁厢办酒接风都道:“恭喜大王上界去十数年想必得意荣归也?”猴王道:“我才半月有馀那里有十数年?”众猴道:“大王你在天上不觉时辰。天上一日就是下界一年哩。请问大王官居何职?”猴王摇手道:“不好说!不好说!活活的羞杀人!那玉帝不会用人他见老孙这般模样封我做个甚么‘弼马温’原来是与他养马未入流品之类。我初到任时不知只在御马监中顽耍。及今日问我同寮始知是这等卑贱。老孙心中大恼推倒席面不受官衔因此走下来了。”众猴道:“来得好!来得好!大王在这福地洞天之处为王多少尊重快乐怎么肯去与他做马夫?”教:“小的们!快办酒来与大王释闷。”

正饮酒欢会间有人来报道:“大王门外有两个独角鬼王要见大王。”猴王道:“教他进来。”那鬼王整衣跑入洞中倒身下拜。美猴王问他:“你见我何干?”鬼王道:“久闻大王招贤无由得见;今见大王授了天录得意荣归特献赭黄袍一件与大王称庆。肯不弃鄙贱收纳小人亦得效犬马之劳。”猴王大喜将赭黄袍穿起众等欣然排班朝拜即将鬼王封为前部总督先锋。鬼王谢恩毕复启道:“大王在天许久所授何职?”猴王道:“玉帝轻贤封我做个甚么‘弼马温’!”鬼王听言又奏道:“大王有此神通如何与他养马?就做个‘齐天大圣’有何不可?”猴王闻说欢喜不胜连道几个“好!好!好!”教四健将:“就替我快置个旌旗旗上写‘齐天大圣’四大字立竿张挂。自此以后只称我为齐天大圣不许再称大王。亦可传与各洞妖王一体知悉。”此不在话下。

却说那玉帝次日设朝只见张天师引御马监监丞、监副在丹墀下拜奏道:“万岁新任弼马温孙悟空因嫌官小昨日反下天宫去了。”正说间又见南天门外增长天王领众天丁亦奏道:“弼马温不知何故走出天门去了。”玉帝闻言即传旨:“着两路神元各归本职朕遣天兵擒拿此怪。”班部中闪上托塔李天王与哪吒三太子越班奏上道:“万岁微臣不才请旨降此妖怪。”玉帝大喜即封托塔天王李靖为降魔大元帅哪吒三太子为三坛海会大神即刻兴师下界。

李天王与哪吒叩头谢辞径至本宫点起三军帅众头目着巨灵神为先锋鱼肚将掠后药叉将催兵。一霎时出南天门外径来到花果山。选平阳处安了营寨传令教巨灵神挑战。巨灵神得令结束整齐轮着宣花斧到了水帘洞外。只见小洞门外许多妖魔都是些狼虫虎豹之类丫丫叉叉轮枪舞剑在那里跳斗咆哮。这巨灵神喝道:“那业畜!快早去报与弼马温知道吾乃上天大将奉玉帝旨意到此收伏;教他早早出来受降免致汝等皆伤残也。”那些怪奔奔波波传报洞中道:“祸事了!祸事了!”猴王问:“有甚祸事?”众妖道:“门外有一员天将口称大圣官衔道:奉玉帝圣旨来此收伏;教早早出去受降免伤我等性命。”猴王听说教:“取我披挂来!”就戴上紫金冠贯上黄金甲登上步云鞋手执如意金箍棒领众出门摆开阵势。这巨灵神睁睛观看真好猴王:

身穿金甲亮堂堂头戴金冠光映映。手举金箍棒一根足踏云鞋皆相称。

一双怪眼似明星两耳过肩查又硬。挺挺身才变化多声音响亮如钟磬。

尖嘴咨牙弼马温心高要做齐天圣。

巨灵神厉声高叫道:“那泼猴!你认得我么?”大圣听言急问道:“你是那路毛神老孙不曾会你你快报名来。”巨灵神道:“我把你那欺心的猢狲!你是认不得我!我乃高上神灵托塔李天王部下先锋巨灵天将!今奉玉帝圣旨到此收降你。你快卸了装束归顺天恩免得这满山诸畜遭诛;若道半个‘不’字教你顷刻化为齑粉!”猴王听说心中大怒道:“泼毛神休夸大口少弄长舌!我本待一棒打死你恐无人去报信;且留你性命快早回天对玉皇说:他甚不用贤!老孙有无穷的本事为何教我替他养马?你看我这旌旗上字号。若依此字号升官我就不动刀兵自然的天地清泰;如若不依时间就打上灵霄宝殿教他龙床定坐不成!”这巨灵神闻此言急睁睛迎风观看果见门外竖一高竿竿上有旌旗一面上写着“齐天大圣”四大字。巨灵神冷笑三声道:“这泼猴这等不知人事辄敢无状你就要做齐天大圣!好好的吃吾一斧!”劈头就砍将去。那猴王正是会家不忙将金箍棒应手相迎。这一场好杀:

棒名如意斧号宣花。他两个乍相逢不知深浅;斧和棒左右交加。一个暗藏神妙一个大口称夸。使动法喷云嗳雾;展开手播土扬沙。天将神通就有道猴王变化实无涯。棒举却如龙戏水斧来犹似凤穿花。巨灵名望传天下原来本事不如他;大圣轻轻轮铁棒着头一下满身麻。巨灵神抵敌他不住被猴王劈头一棒慌忙将斧架隔呵嚓的一声把个斧柄打做两截急撤身败阵逃生。猴王笑道:“脓包!脓包!我已饶了你你快去报信!快去报信!”

巨灵神回至营门径见托塔天王忙哈哈下跪道:“弼马温果是神通广大!末将战他不得败阵回来请罪。”李天王怒道:“这厮锉吾锐气推出斩之!”旁边闪出哪吒太子拜告:“父王息怒且恕巨灵之罪待孩儿出师一遭便知深浅。”天王听谏且教回营待罪管事。

这哪吒太子甲胄齐整跳出营盘撞至水帘洞外。那悟空正来收兵见哪吒来的勇猛。好太子:

总角才遮囟披毛未盖肩。神奇多敏悟骨秀更清妍。诚为天上麒麟子果是烟霞彩凤仙。龙种自然非俗相妙龄端不类尘凡。身带六般神器械飞腾变化广无边。今受玉皇金口诏敕射海会号三坛。悟空迎近前来问曰:“你是谁家小哥?闯近吾门有何事干?”哪吒喝道:“泼妖猴!岂不认得我?我乃托塔天王三太子哪吒是也。今奉玉帝钦差至此捉你。”悟空笑道:“小太子你的奶牙尚未退胎毛尚未干怎敢说这般大话?我且留你的性命不打你。你只看我旌旗上的是甚么字号拜上玉帝:是这般官衔再也不须动众我自皈依;若是不遂我心定要打上灵霄宝殿。”哪吒抬头看处乃“齐天大圣”四字。哪吒道:“这妖猴能有多大神通就敢称此名号!不要怕!吃吾一剑!”悟空道:“我只站下不动任你砍几剑罢。”那哪吒奋怒大喝一声叫“变!”即变做三头六臂恶狠狠手持着六般兵器乃是斩妖剑、砍妖刀、缚妖索、降妖杵、绣球儿、火轮儿丫丫叉叉扑面打来。悟空见了心惊道:“这小哥倒也会弄些手段!莫无礼看我神通!”好大圣喝声“变”也变做三头六臂;把金箍棒幌一幌也变作三条;六只手拿着三条棒架住。这场斗真是个地动山摇好杀也:

六臂哪吒太子天生美石猴王相逢真对手正遇本源流。那一个蒙差来下界这一个欺心闹斗牛。斩妖宝剑锋芒快砍妖刀狠鬼神愁;缚妖索子如飞蟒降妖大杵似狼头;火轮掣电烘烘艳往往来来滚绣球。大圣三条如意棒前遮后挡运机谋。苦争数合无高下太子心中不肯休。把那六件兵器多教变百千万亿照头丢。猴王不惧呵呵笑铁棒翻腾自运筹。以一化千千化万满空乱舞赛飞虬。唬得各洞妖王都闭户遍山鬼怪尽藏头。神兵怒气云惨惨金箍铁棒响飕飕。那壁厢天丁呐喊人人怕;这壁厢猴怪摇旗个个忧。狠两家齐斗勇

不知那个刚强那个柔。三太子与悟空各骋神威斗了个三十回合。那太子六般兵器变做千千万万;孙悟空金箍棒变作万万千千。半空中似雨点流星不分胜负。原来悟空手疾正在那混乱之时他拔下一根毫毛叫声“变!”就变做他的本相手挺着棒演着哪吒;他的真身却一纵赶至哪吒脑后着左膊上一棒打来。哪吒正使法间听得棒头风响急躲闪时不能措手被他着了一下负痛逃走;收了法把六件兵器依旧归身败阵而回。

那阵上李天王早已看见急欲提兵助战。不觉太子倏至面前战兢兢报道:“父王!弼马温真个有本事!孩儿这般法力也战他不过已被他打伤膊也。”天王大惊失色道:“这厮恁的神通如何取胜?”太子道:“他洞门外竖一竿旗上写‘齐天大圣’四字亲口夸称教玉帝就封他做齐天大圣万事俱休;若还不是此号定要打上灵霄宝殿哩!”天王道:“既然如此且不要与他相持且去上界将此言回奏再多遣天兵围捉这厮未为迟也。”太子负痛不能复战故同天王回天启奏不题。

你看那猴王得胜归山那七十二洞妖王与那六弟兄俱来贺喜。在洞天福地饮乐无比。他却对六弟兄说:“小弟既称齐天大圣你们亦可以大圣称之。”内有牛魔王忽然高声叫道:“贤弟言之有理我即称做个平天大圣。”蛟魔王道:“我称覆海大圣。”鹏魔王道:“我称混天大圣。”狮犭它王道:“我称移山大圣。”猕猴王道:“我称通风大圣。”犭狨王道:“我称驱神大圣。”此时七大圣自作自为自称自号耍乐一日各散讫。

却说那李天王与三太子领着众将直至灵霄殿。启奏道:“臣等奉圣旨出师下界收伏妖仙孙悟空不期他神通广大不能取胜仍望万岁添兵剿除。”玉帝道:“谅一妖猴有多少本事还要添兵?”太子又近前奏道:“望万岁赦臣死罪!那妖猴使一条铁棒先败了巨灵神又打伤臣臂膊。洞门外立一竿旗上书‘齐天大圣’四字道是封他这官职即便休兵来投;若不是此官还要打上灵霄宝殿也。”玉帝闻言惊讶道:“这妖猴何敢这般狂妄!着众将即刻诛之。”正说间班部中又闪出太白金星奏道:“那妖猴只知出言不知大小。欲加兵与他争斗想一时不能收伏反又劳师。不若万岁大舍恩慈还降招安旨意就教他做个齐天大圣。只是加他个空衔有官无禄便了。”玉帝道:“怎么唤做‘有官无禄’?”金星道:“名是齐天大圣只不与他事管不与他俸禄且养在天壤之间收他的邪心使不生狂妄庶乾坤安靖海宇得清宁也。”玉帝闻言道:“依卿所奏。”即命降了诏书仍着金星领去。

金星复出南天门直至花果山水帘洞外观看。这番比前不同威风凛凛杀气森森各样妖精无般不有。一个个都执剑拈枪拿刀弄杖的在那里咆哮跳跃。一见金星皆上前动手。金星道:“那众头目来!累你去报你大圣知之。吾乃上帝遣来天使有圣旨在此请他。”众妖即跑入报道:“外面有一老者他说是上界天使有旨意请你。”悟空道:“来得好!来得好!想是前番来的那太白金星。那次请我上界虽是官爵不堪却也天上走了一次认得那天门内外之路。今番又来定有好意。”教众头目大开旗鼓摆队迎接。大圣即带引群猴顶冠贯甲甲上罩了赭黄袍足踏云履急出洞门躬身施礼高叫道:“老星请进恕我失迎之罪。”

金星趋步向前径入洞内面南立着道:“今告大圣前者因大圣嫌恶官小躲离御马监当有本监中大小官员奏了玉帝。玉帝传旨道:‘凡授官者皆由卑而尊为何嫌小?’即有李天王领哪吒下界取战。不知大圣神通故遭败北回天奏道:‘大圣立一竿旗要做“齐天大圣”。’众武将还要支吾是老汉力为大圣冒罪奏闻免兴师旅请大王授录。玉帝准奏因此来请。”悟空笑道:“前番勤劳今又蒙爱多谢!多谢!但不知上天可有此‘齐天大圣’之官衔也?”金星道:“老汉以此衔奏准方敢岭旨而来;如有不遂只坐罪老汉便是。”

悟空大喜恳留饮宴不肯遂与金星纵着祥云到南天门外。那些天丁天将都拱手相迎。径入灵霄殿下。金星拜奏道:“臣奉诏宣弼马温孙悟空已到。”玉帝道:“那孙悟空过来。今宣你做个‘齐天大圣’官品极矣但切不可胡为。”这猴亦止朝上唱个喏道声谢恩。玉帝即命工干官——张、鲁二班——在蟠桃园右起一座齐天大圣府府内设个二司:一名安静司一名宁神司。司俱有仙吏左右扶持。又差五斗星君送悟空去到任外赐御酒二瓶金花十朵着他安心定志再勿胡为。那猴王信受奉行即日与五斗星君到府打开酒瓶同众尽饮。送星官回转本宫他才遂心满意喜地欢天在于天宫快乐无挂无碍。正是:仙名永注长生录不堕轮回万古传。毕竟不知向后如何且听下回分解。





附录 陈光蕊赴任逢灾 江流僧复仇报本

话表陕西大国长安城乃历代帝王建都之地。自周、秦、汉以来三州花似锦八水绕城流真个是名胜之邦。彼时是大唐太宗皇帝登基改元贞观已登极十三年岁在己巳天下太平八方进贡四海称臣。忽一日太宗登位聚集文武众官朝拜礼毕有魏征丞相出班奏道:“方今天下太平八方宁静应依古法开立选场招取贤士擢用人材以资化理。”太宗道:

“贤卿所奏有理。”就传招贤文榜颁布天下:各府州县不拘军民人等但有读书儒流文义明畅三场精通者前赴长安应试。

此榜行至海州地方有一人姓陈名萼表字光蕊见了此榜即时回家对母张氏道:“朝廷颁下黄榜诏开南省考取贤才孩儿意欲前去应试。倘得一官半职显亲扬名封妻荫子光耀门闾乃儿之志也。特此禀告母亲前去。”张氏道:“我儿读书人‘幼而学壮而行’正该如此。但去赴举路上须要小心得了官早早回来。”光蕊便吩咐家僮收拾行李即拜辞母亲趱程前进。到了长安正值大开选场光蕊就进场。考毕中选及廷试三策唐王御笔亲赐状元跨马游街三日。不期游到丞相殷开山门有丞相所生一女名唤温娇又名满堂娇未曾婚配正高结彩楼抛打绣球卜婿。适值陈光蕊在楼下经过小姐一见光蕊人材出众知是新科状元心内十分欢喜就将绣球抛下恰打着光蕊的乌纱帽。猛听得一派笙箫细乐十数个婢妾走下楼来把光蕊马头挽住迎状元入相府成婚。那丞相和夫人即时出堂唤宾人赞礼将小姐配与光蕊。拜了天地夫妻交拜毕又拜了岳丈岳母。丞相吩咐安排酒席欢饮一宵。

二人同携素手共入兰房。次日五更三点太宗驾坐金銮宝殿文武众臣趋朝。太宗同道:“新科状元陈光蕊应授何官?”魏征丞相奏道:“臣查所属州郡有江州缺官。乞我主授他此职。”太宗就命为江州州主即令收拾起身勿误限期。光蕊谢恩出朝回到相府与妻商议拜辞岳丈岳母同妻前赴江州之任。

离了长安登途正是暮春天气和风吹柳绿细雨点花红。

光蕊便道回家同妻交拜母亲张氏。张氏道:“恭喜我儿且又娶亲回来。”光蕊道:“孩儿叨赖母亲福庇忝中状元钦赐游街经过丞相殷府门前遇抛打绣球适中蒙丞相即将小姐招孩儿为婿。朝廷除孩儿为江州州主今来接取母亲同去赴任。”张氏大喜收拾行程。在路数日前至万花店刘小二家安下张氏身体忽然染病与光蕊道:“我身上不安且在店中调养两日再去。”光蕊遵命。至次日早晨见店门前有一人提着个金色鲤鱼叫卖光蕊即将一贯钱买了欲待烹与母亲吃只见鲤鱼闪闪咪眼光蕊惊异道:“闻说鱼蛇咪眼必不是等闲之物!”遂问渔人道:“这鱼那里打来的?”渔人道:“离府十五里洪江内打来的。”光蕊就把鱼送在洪江里去放了生。回店对母亲道知此事张氏道:“放生好事我心甚喜。”光蕊道:“此店已住三日了钦限紧急孩儿意欲明日起身不知母亲身体好否?”

张氏道:“我身子不快此时路上炎热恐添疾病。你可这里赁间房屋与我暂住。付些盘缠在此你两口儿先上任去候秋凉却来接我。”光蕊与妻商议就租了屋宇付了盘缠与母亲同妻拜辞前去。

途路艰苦晓行夜宿不觉已到洪江渡口。只见稍水刘洪、李彪二人撑船到岸迎接。也是光蕊前生合当有此灾难撞着这冤家。光蕊令家僮将行李搬上船去夫妻正齐齐上船那刘洪睁眼看见殷小姐面如满月眼似秋波樱桃小口绿柳蛮腰真个有沉鱼落雁之容闭月羞花之貌陡起狼心遂与李彪设计将船撑至没人烟处候至夜静三更先将家僮杀死次将光蕊打死把尸都推在水里去了。小姐见他打死了丈夫也便将身赴水刘洪一把抱住道:“你若从我万事皆休!若不从时一刀两断!”那小姐寻思无计只得权时应承顺了刘洪。那贼把船渡到南岸将船付与李彪自管他就穿了光蕊衣冠带了官凭同小姐往江州上任去了。

却说刘洪杀死的家僮尸顺水流去惟有陈光蕊的尸沉在水底不动。有洪江口巡海夜叉见了星飞报入龙宫正值龙王升殿夜叉报道:“今洪江口不知甚人把一个读书士子打死将尸撇在水底。”龙王叫将尸抬来放在面前仔细一看道:“此人正是救我的恩人如何被人谋死?常言道恩将恩报。

我今日须索救他性命以报日前之恩。”即写下牒文一道差夜叉径往洪州城隍土地处投下要取秀才魂魄来救他的性命。

城隍土地遂唤小鬼把陈光蕊的魂魄交付与夜叉去夜叉带了魂魄到水晶宫禀见了龙王。龙王问道:“你这秀才姓甚名谁?

何方人氏?因甚到此被人打死?”光蕊施礼道:“小生陈萼表字光蕊系海州弘农县人。忝中新科状元叨授江州州主同妻赴任行至江边上船不料稍子刘洪贪谋我妻将我打死抛尸乞大王救我一救!”龙王闻言道:“原来如此先生你前者所放金色鲤鱼即我也你是救我的恩人你今有难我岂有不救你之理?”就把光蕊尸身安置一壁口内含一颗定颜珠休教损坏了日后好还魂报仇。又道:“汝今真魂权且在我水府中做个都领。”光蕊叩头拜谢龙王设宴相待不题。

却说殷小姐痛恨刘贼恨不食肉寝皮只因身怀有孕未知男女万不得已权且勉强相从。转盼之间不觉已到江州。

吏书门皂俱来迎接。所属官员公堂设宴相叙。刘洪道:“学生到此全赖诸公大力匡持。”属官答道:“堂尊大魁高才自然视民如子讼简刑清。我等合属有赖何必过谦?”公宴已罢众人各散。

光阴迅。一日刘洪公事远出小姐在衙思念婆婆、丈夫在花亭上感叹忽然身体困倦腹内疼痛晕闷在地不觉生下一子。耳边有人嘱曰:“满堂娇听吾叮嘱。吾乃南极星君奉观音菩萨法旨特送此子与你异日声名远大非比等闲。刘贼若回必害此子汝可用心保护。汝夫已得龙王相救日后夫妻相会子母团圆雪冤报仇有日也。谨记吾言快醒快醒!”言讫而去。小姐醒来句句记得将子抱定无计可施。忽然刘洪回来一见此子便要淹杀小姐道:“今日天色已晚容待明日抛去江中。”幸喜次早刘洪忽有紧急公事远出小姐暗思:“此子若待贼人回来性命休矣!不如及早抛弃江中听其生死。倘或皇天见怜有人救得收养此子他日还得相逢。”但恐难以识认即咬破手指写下血书一纸将父母姓名、跟脚原由备细开载;又将此子左脚上一个小指用口咬下以为记验。取贴身汗衫一件包裹此子乘空抱出衙门。幸喜官衙离江不远小姐到了江边大哭一场。正欲抛弃忽见江岸岸侧飘起一片木板小姐即朝天拜祷将此子安在板上用带缚住血书系在胸前推放江中听其所之。小姐含泪回衙不题。

却说此子在木板上顺水流去一直流到金山寺脚下停住。那金山寺长老叫做法明和尚修真悟道已得无生妙诀。正当打坐参禅忽闻得小儿啼哭之声一时心动急到江边观看只见涯边一片木板上睡着一个婴儿长老慌忙救起。见了怀中血书方知来历取个乳名叫做江流托人抚养血书紧紧收藏。光阴似箭日月如梭不觉江流年长一十八岁。长老就叫他削修行取法名为玄奘摩顶受戒坚心修道。

一日暮春天气众人同在松阴之下讲经参禅谈说奥妙。那酒肉和尚恰被玄奘难倒和尚大怒骂道:“你这业畜姓名也不知父母也不识还在此捣甚么鬼!”玄奘被他骂出这般言语入寺跪告师父眼泪双流道:“人生于天地之间禀阴阳而资五行尽由父生母养岂有为人在世而无父母者乎?”再三哀告求问父母姓名。长老道:“你真个要寻父母可随我到方丈里来。”玄奘就跟到方丈长老到重梁之上取下一个小匣儿打开来取出血书一纸汗衫一件付与玄奘。玄奘将血书拆开读之才备细晓得父母姓名并冤仇事迹。玄奘读罢不觉哭倒在地道:“父母之仇不能报复何以为人?十八年来不识生身父母至今日方知有母亲。此身若非师父捞救抚养安有今日?容弟子去寻见母亲然后头顶香盆重建殿宇报答师父之深恩也!”师父道:“你要去寻母可带这血书与汗衫前去只做化缘径往江州私衙才得你母亲相见。”

玄奘领了师父言语就做化缘的和尚径至江州。适值刘洪有事出外也是天教他母子相会玄奘就直至私衙门口抄化。那殷小姐原来夜间得了一梦梦见月缺再圆暗想道:“我婆婆不知音信我丈夫被这贼谋杀我的儿子抛在江中倘若有人收养算来有十八岁矣或今日天教相会亦未可知。”正沉吟间忽听私衙前有人念经连叫“抄化”小姐又乘便出来问道:“你是何处来的?”玄奘答道:“贫僧乃是金山寺法明长老的徒弟。”小姐道:“你既是金山寺长老的徒弟——”叫进衙来将斋饭与玄奘吃。仔细看他举止言谈好似与丈夫一般小姐将从婢打开去问道:“你这小师父还是自幼出家的?还是中年出家的?姓甚名谁?可有父母否?”玄奘答道:“我也不是自幼出家我也不是中年出家我说起来冤有天来大仇有海样深!我父被人谋死我母亲被贼人占了。我师父法明长老教我在江州衙内寻取母亲。”小姐问道:“你母姓甚?”玄奘道:“我母姓殷名唤温娇我父姓陈名光蕊我小名叫做江流法名取为玄奘。”小姐道:“温娇就是我。但你今有何凭据?”玄奘听说是他母亲双膝跪下哀哀大哭:“我娘若不信见有血书汗衫为证!”温娇取过一看果然是真母子相抱而哭就叫:“我儿快去!”玄奘道:“十八年不识生身父母今朝才见母亲教孩儿如何割舍?”小姐道:“我儿你火抽身前去!刘贼若回他必害你性命!我明日假装一病只说先年曾许舍百双僧鞋来你寺中还愿。那时节我有话与你说。”玄奘依言拜别。

却说小姐自见儿子之后心内一忧一喜忽一日推病茶饭不吃卧于床上。刘洪归衙问其原故小姐道:“我幼时曾许下一愿许舍僧鞋一百双。昨五日之前梦见个和尚手执利刃要索僧鞋便觉身子不快。”刘洪道:“这些小事何不早说?”随升堂吩咐王左衙、李右衙:江州城内百姓每家要办僧鞋一双限五日内完纳。百姓俱依派完纳讫。小姐对刘洪道:

“僧鞋做完这里有甚么寺院好去还愿?”刘洪道:“这江州有个金山寺、焦山寺听你在那个寺里去。”小姐道:“久闻金山寺好个寺院我就往金山寺去。”刘洪即唤王、李二衙办下船只。

小姐带了心腹人同上了船稍水将船撑开就投金山寺去。

却说玄奘回寺见法明长老把前项说了一遍长老甚喜。

次日只见一个丫鬟先到说夫人来寺还愿众僧都出寺迎接。

小姐径进寺门参了菩萨大设斋衬唤丫鬟将僧鞋暑袜托于盘内。来到法堂小姐复拈心香礼拜就教法明长老分表与众僧去讫。玄奘见众僧散了法堂上更无一人他却近前跪下。小姐叫他脱了鞋袜看时那左脚上果然少了一个小指头。当时两个又抱住而哭拜谢长老养育之恩。法明道:“汝今母子相会恐奸贼知之可抽身回去庶免其祸。”小姐道:“我儿我与你一只香环你径到洪州西北地方约有一千五百里之程那里有个万花店当时留下婆婆张氏在那里是你父亲生身之母。我再写一封书与你径到唐王皇城之内金殿左边殷开山丞相家是你母生身之父母。你将我的书递与外公叫外公奏上唐王统领人马擒杀此贼与父报仇那时才救得老娘的身子出来。我今不敢久停诚恐贼汉怪我归迟。”便出寺登舟而去。

玄奘哭回寺中告过师父即时拜别径往洪州。来到万花店问那店主刘小二道:“昔年江州陈客官有一母亲住在你店中如今好么?”刘小二道:“他原在我店中后来昏了眼三四年并无店租还我如今在南门头一个破瓦窑里每日上街叫化度日。那客官一去许久到如今杳无信息不知为何。”玄奘听罢即时问到南门头破瓦窑寻着婆婆。婆婆道:“你声音好似我儿陈光蕊。”玄奘道:“我不是陈光蕊我是陈光蕊的儿子。温娇小姐是我的娘。”婆婆道:“你爹娘怎么不来?”玄奘道:“我爹爹被强盗打死了我娘被强盗霸占为妻。”婆婆道:“你怎么晓得来寻我?”玄奘道:“是我娘着我来寻婆婆。我娘有书在此又有香环一只。”那婆婆接了书并香环放声痛哭道:“我儿为功名到此我只道他背义忘恩那知他被人谋死!且喜得皇天怜念不绝我儿之后今日还有孙子来寻我。”玄奘问:“婆婆的眼如何都昏了?”婆婆道:“我因思量你父亲终日悬望不见他来因此上哭得两眼都昏了。”玄奘便跪倒向天祷告道:“念玄奘一十八岁父母之仇不能报复。今日领母命来寻婆婆天若怜鉴弟子诚意保我婆婆双眼复明!”祝罢就将舌尖与婆婆舔眼。须臾之间双眼舔开仍复如初。婆婆觑了小和尚道:

“你果是我的孙子!恰和我儿子光蕊形容无二!”婆婆又喜又悲。玄奘就领婆婆出了窑门还到刘小二店内将些房钱赁屋一间与婆婆栖身又将盘缠与婆婆道:“我此去只月余就回。”

随即辞了婆婆径往京城。寻到皇城东街殷丞相府上与门上人道:“小僧是亲戚来探相公。”门上人禀知丞相丞相道:“我与和尚并无亲眷。”夫人道:“我昨夜梦见我女儿满堂娇来家莫不是女婿有书信回来也。”丞相便教请小和尚来到厅上。小和尚见了丞相与夫人哭拜在地就怀中取出一封书来递与丞相。丞相拆开从头读罢放声痛哭。夫人问道:“相公有何事故?”丞相道:“这和尚是我与你的外甥。女婿陈光蕊被贼谋死满堂娇被贼强占为妻。”夫人听罢亦痛哭不止。丞相道:“夫人休得烦恼来朝奏知主上亲自统兵定要与女婿报仇。”

次日丞相入朝启奏唐王曰:“今有臣婿状元陈光蕊带领家小江州赴任被稍水刘洪打死占女为妻假冒臣婿为官多年事属异变。乞陛下立人马剿除贼寇。”唐王见奏大怒就御林军六万着殷丞相督兵前去。丞相领旨出朝即往教场内点了兵径往江州进。晓行夜宿星落鸟飞不觉已到江州。殷丞相兵马俱在北岸下了营寨。星夜令金牌下户唤到江州同知、州判二人丞相对他说知此事叫他提兵相助一同过江而去。天尚未明就把刘洪衙门围了。刘洪正在梦中听得火炮一响金鼓齐鸣众兵杀进私衙刘洪措手不及早被擒住。丞相传下军令将刘洪一干人犯绑赴法场令众军俱在城外安营去了。

丞相直入衙内正厅坐下请小姐出来相见。小姐欲待要出羞见父亲就要自缢。玄奘闻知急急将母解救双膝跪下对母道:“儿与外公统兵至此与父报仇。今日贼已擒捉母亲何故反要寻死?母亲若死孩儿岂能存乎?”丞相亦进衙劝解。

小姐道:“吾闻妇人从一而终。痛夫已被贼人所杀岂可靦颜从贼?止因遗腹在身只得忍耻偷生。今幸儿已长大又见老父提兵报仇为女儿者有何面目相见!惟有一死以报丈夫耳!”

丞相道:“此非我儿以盛衰改节皆因出乎不得已何得为耻!”

父子相抱而哭玄奘亦哀哀不止。丞相拭泪道:“你二人且休烦恼我今已擒捉仇贼且去落去来。”即起身到法场恰好江州同知亦差哨兵拿获水贼李彪解到。丞相大喜就令军牢押过刘洪、李彪每人痛打一百大棍取了供状招了先年不合谋死陈光蕊情由先将李彪钉在木驴上推去市曹剐了千刀枭示众讫;把刘洪拿到洪江渡口先年打死陈光蕊处丞相与小姐、玄奘三人亲到江边望空祭奠活剜取刘洪心肝祭了光蕊烧了祭文一道。

三人望江痛哭早已惊动水府。有巡海夜叉将祭文呈与龙王。龙王看罢就差鳖无帅去请光蕊来到道:“先生恭喜!

恭喜!今有先生夫人公子同岳丈俱在江边祭你我今送你还魂去也。再有如意珠一颗走盘珠二颗绞绡十端明珠玉带一条奉送。你今日便可夫妻子母相会也。”光蕊再三拜谢。龙王就令夜叉将光蕊身尸送出江口还魂夜叉领命而去。

却说殷小姐哭奠丈夫一番又欲将身赴水而死慌得玄奘拚命扯住。正在仓皇之际忽见水面上一个死尸浮来靠近江岸之旁。小姐忙向前认看认得是丈夫的尸一嚎啕大哭不已。众人俱来观看只见光蕊舒拳伸脚身子渐渐展动忽地爬将起来坐下众人不胜惊骇。光蕊睁开眼早见殷小姐与丈人殷丞相同着小和尚俱在身边啼哭。光蕊道:“你们为何在此?”小姐道:“因汝被贼人打死后来妾身生下此子幸遇金山寺长老抚养长大寻我相会。我教他去寻外公父亲得知奏闻朝廷统兵到此拿住贼人。适才生取心肝望空祭奠我夫不知我夫怎生又得还魂。”光蕊道:“皆因我与你昔年在万花店时买放了那尾金色鲤鱼谁知那鲤鱼就是此处龙王。后来逆贼把我推在水中全亏得他救我方才又赐我还魂送我宝物俱在身上。更不想你生下这儿子又得岳丈为我报仇。真是苦尽甘来莫大之喜!”

众官闻知都来贺喜。丞相就令安排酒席答谢所属官员即日军马回程。来到万花店那丞相传令安营。光蕊便同玄奘到刘家店寻婆婆。那婆婆当夜得了一梦梦见枯木开花屋后喜鹊频频喧噪想道:“莫不是我孙儿来也?”说犹未了只见店门外光蕊父子齐到。小和尚指道:“这不是俺婆婆?”光蕊见了老母连忙拜倒。母子抱头痛哭一场把上项事说了一遍。算还了小二店钱起程回到京城。进了相府光蕊同小姐与婆婆、玄奘都来见了夫人。夫人不胜之喜吩咐家僮大排筵宴庆贺。

丞相道:“今日此宴可取名为团圆会。”真正合家欢乐。

次日早朝唐王登殿殷丞相出班将前后事情备细启奏并荐光蕊才可大用。唐王准奏即命升陈萼为学士之职随朝理政。玄奘立意安禅送在洪福寺内修行。后来殷小姐毕竟从容自尽玄奘自到金山寺中报答法明长老。不知后来事体若何且听下回分解。


----------

第二十回 黄风岭唐僧有难 半山中八戒争先

偈曰:“法本从心生还是从心灭。(WWW.mianhuatang.la 好看的小说)生灭尽由谁请君自辨别。既然皆己心何用别人说?只须下苦功扭出铁中血。绒绳着鼻穿挽定虚空结。拴在无为树不使他颠劣。莫认贼为子心法都忘绝。休教他瞒我一拳先打彻。现心亦无心现法法也辍。人牛不见时碧天光皎洁。秋月一般圆彼此难分别。”

这一篇偈子乃是玄奘法师悟彻了《多心经》打开了门户那长老常念常存一点灵光自透。

且说他三众在路餐风宿水带月披星早又至夏景炎天。

但见那:花尽蝶无情叙树高蝉有声喧。野蚕成茧火榴妍沼内新荷出现。那日正行时忽然天晚又见山路旁边有一村舍。

三藏道:“悟空你看那日落西山藏火镜月升东海现冰轮。幸而道旁有一人家我们且借宿一宵明日再走。”八戒道:“说得是我老猪也有些饿了且到人家化些斋吃有力气好挑行李。”行者道:“这个恋家鬼!你离了家几日就生报怨!”八戒道:“哥啊似不得你这喝风呵烟的人。我从跟了师父这几日长忍半肚饥你可晓得?”三藏闻之道:“悟能你若是在家心重呵不是个出家的了你还回去罢。那呆子慌得跪下道:“师父你莫听师兄之言。他有些赃埋人。我不曾报怨甚的他就说我报怨。我是个直肠的痴汉我说道肚内饥了好寻个人家化斋他就骂我是恋家鬼。师父啊我受了菩萨的戒行又承师父怜悯情愿要伏侍师父往西天去誓无退悔这叫做恨苦修行怎的说不是出家的话!”三藏道:“既是如此你且起来。”

那呆子纵身跳起口里絮絮叨叨的挑着担子只得死心塌地跟着前来。早到了路旁人家门三藏下马行者接了缰绳八戒歇了行李都伫立绿荫之下。三藏拄着九环锡杖按按藤缠篾织斗篷先奔门前只见一老者斜倚竹床之上口里嘤嘤的念佛。三藏不敢高言慢慢的叫一声:“施主问讯了。”那老者一骨鲁跳将起来忙敛衣襟出门还礼道:“长老失迎。你自那方来的?到我寒门何故?”三藏道:“贫僧是东土大唐和尚奉圣旨上雷音寺拜佛求经。适至宝方天晚意投檀府告借一宵万祈方便方便。”那老儿摆手摇头道:“去不得西天难取经。要取经往东天去罢。”三藏口中不语意下沉吟:“菩萨指道西去怎么此老说往东行?东边那得有经?”腼腆难言半晌不答。却说行者索性凶顽忍不住上前高叫道:“那老儿你这们大年纪全不晓事。我出家人远来借宿就把这厌钝的话虎唬我。十分你家窄狭没处睡时我们在树底下好道也坐一夜不打搅你。”那老者扯住三藏道:“师父你倒不言语你那个徒弟那般拐子脸、别颏腮、雷公嘴、红眼睛的一个痨病魔鬼怎么反冲撞我这年老之人!”行者笑道:“你这个老儿忒也没眼色!似那俊刮些儿的叫做中看不中吃。想我老孙虽小颇结实皮裹一团筋哩。”那老者道:“你想必有些手段。”行者道:“不敢夸言也将就看得过。”老者道:“你家居何处?因甚事削为僧?”行者道:“老孙祖贯东胜神洲海东傲来国花果山水帘洞居住。自小儿学做妖怪称名悟空凭本事挣了一个齐天大圣。只因不受天禄大反天宫惹了一场灾愆。如今脱难消灾转拜沙门前求正果保我这唐朝驾下的师父上西天拜佛走遭怕甚么山高路险水阔波狂!我老孙也捉得怪降得魔。

伏虎擒龙踢天弄井都晓得些儿。倘若府上有甚么丢砖打瓦锅叫门开老孙便能安镇。”那老儿听得这篇言语哈哈笑道:

“原来是个撞头化缘的熟嘴儿和尚。”行者道:“你儿子便是熟嘴!我这些时只因跟我师父走路辛苦还懒说话哩。”那老儿道:“若是你不辛苦不懒说话好道活活的聒杀我!你既有这样手段西方也还去得去得。你一行几众?请至茅舍里安宿。”

三藏道:“多蒙老施主不叱之恩我一行三众。”老者道:“那一众在那里?”行者指着道:“这老儿眼花那绿荫下站的不是?”

老儿果然眼花忽抬头细看一见八戒这般嘴脸就唬得一步一跌往屋里乱跑只叫:“关门!关门!妖怪来了!”行者赶上扯住道:“老儿莫怕他不是妖怪是我师弟。”老者战兢兢的道:“好!好!好!一个丑似一个的和尚!”八戒上前道:“老官儿你若以相貌取人干净差了。我们丑自丑却都有用。”

那老者正在门前与三个和尚相讲只见那庄南边有两个少年人带着一个老妈妈三四个小男女敛衣赤脚插秧而回。他看见一匹白马一担行李都在他家门喧哗不知是甚来历都一拥上前问道:“做甚么的?”八戒调过头来把耳朵摆了几摆长嘴伸了一伸吓得那些人东倒西歪乱跄乱跌。慌得那三藏满口招呼道:“莫怕!莫怕!我们不是歹人我们是取经的和尚。”那老儿才出了门搀着妈妈道:“婆婆起来少要惊恐。这师父是唐朝来的只是他徒弟脸嘴丑些却也面恶人善。带男女们家去。”那妈妈才扯着老儿二少年领着儿女进去。三藏却坐在他们楼里竹床之上埋怨道:“徒弟呀你两个相貌既丑言语又粗把这一家儿吓得七损八伤都替我身造罪哩!”八戒道:“不瞒师父说老猪自从跟了你这些时俊了许多哩。若象往常在高老庄走时把嘴朝前一掬把耳两头一摆常吓杀二三十人哩。”行者笑道:“呆子不要乱说把那丑也收拾起些。”三藏道:“你看悟空说的话!相貌是生成的你教他怎么收拾?”行者道:“把那个耙子嘴揣在怀里莫拿出来;把那蒲扇耳贴在后面不要摇动这就是收拾了。”那八戒真个把嘴揣了把耳贴了拱着头立于左右。行者将行李拿入门里将白马拴在桩上。

只见那老儿才引个少年拿一个板盘儿托三杯清茶来献。茶罢又吩咐办斋。那少年又拿一张有窟窿无漆水的旧桌端两条破头折脚的凳子放在天井中请三众凉处坐下。三藏方问道:“老施主高姓?”老者道:“在下姓王。”“有几位令嗣?”

道:“有两个小儿三个小孙。”三藏道:“恭喜恭喜。”又问:“年寿几何?”道:“痴长六十一岁。”行者道:“好!好!好!花甲重逢矣。”三藏复问道:“老施主始初说西天经难取者何也?”老者道:“经非难取只是道中艰涩难行。我们这向西去只有三十里远近有一座山叫做八百里黄风岭那山中多有妖怪。故言难取者此也。若论此位小长老说有许多手段却也去得。”

行者道:“不妨!不妨!有了老孙与我这师弟任他是甚么妖怪不敢惹我。”正说处又见儿子拿将饭来摆在桌上道声“请斋。”三藏就合掌讽起斋经八戒早已吞了一碗。长老的几句经还未了那呆子又吃彀三碗。行者道:“这个馕糠!好道撞着饿鬼了!”那老王倒也知趣见他吃得快道:“这个长老想着实饿了快添饭来。”那呆子真个食肠大看他不抬头一连就吃有十数碗。三藏、行者俱各吃不上两碗呆子不住便还吃哩。

老王道:“仓卒无肴不敢苦劝请再进一筋。”三藏、行者俱道:

“彀了。”八戒道:“老儿滴答甚么谁和你课说甚么五爻六爻!有饭只管添将来就是。”呆子一顿把他一家子饭都吃得罄尽还只说才得半饱。却才收了家火在那门楼下安排了竹床板铺睡下。

次日天晓行者去背马八戒去整担老王又教妈妈整治些点心汤水管待三众方致谢告行。老者道:“此去倘路间有甚不虞是必还来茅舍。”行者道:“老儿莫说哈话。我们出家人不走回头路。”遂此策马挑担西行。噫!这一去果无好路朝西域定有邪魔降大灾。三众前来不上半日果逢一座高山说起来十分险峻。三藏马到临崖斜挑宝镫观看果然那:高的是山峻的是岭;陡的是崖深的是壑;响的是泉鲜的是花。那山高不高顶上接青霄;这涧深不深底中见地府。山前面有骨都都白云屹嶝嶝怪石说不尽千丈万丈挟魂崖。崖后有弯弯曲曲藏龙洞洞中有叮叮当当滴水岩。又见些丫丫叉叉带角鹿泥泥痴痴看人獐;盘盘曲曲红鳞蟒耍耍顽顽白面猿。至晚巴山寻穴虎带晓翻波出水龙登的洞门唿喇喇响。草里飞禽扑轳轳起;林中走兽掬律律行。猛然一阵狼虫过吓得人心趷蹬蹬惊。正是那当倒洞当当倒洞洞当当倒洞当山。青岱染成千丈玉碧纱笼罩万堆烟。那师父缓促银骢孙大圣停云慢步猪悟能磨担徐行。正看那山忽闻得一阵旋风大作三藏在马上心惊道:“悟空风起了!”行者道:“风却怕他怎的!此乃天家四时之气有何惧哉!”三藏道:“此风其恶比那天风不同。”行者道:“怎见得不比天风?”三藏道:“你看这风:巍巍荡荡飒飘飘渺渺茫茫出碧霄。过岭只闻千树吼入林但见万竿摇。岸边摆柳连根动园内吹花带叶飘。收网渔舟皆紧缆落篷客艇尽抛锚。途半征夫迷失路山中樵子担难挑。仙果林间猴子散奇花丛内鹿儿逃。崖前桧柏颗颗倒涧下松篁叶叶凋。播土扬尘沙迸迸翻江搅海浪涛涛。”八戒上前一把扯住行者道:“师兄十分风大!我们且躲一躲儿干净。”行者笑道:“兄弟不济!

风大时就躲倘或亲面撞见妖精怎的是好?”八戒道:“哥啊你不曾闻得避色如避仇避风如避箭哩!我们躲一躲也不亏人。”行者道:“且莫言语等我把这风抓一把来闻一闻看。”八戒笑道:“师兄又扯空头谎了风又好抓得过来闻?就是抓得来使也钻了去了。”行者道:“兄弟你不知道老孙有个抓风之法。”好大圣让过风头把那风尾抓过来闻了一闻有些腥气道:“果然不是好风!这风的味道不是虎风定是怪风断乎有些蹊跷。”

说不了只见那山坡下剪尾跑蹄跳出一只斑斓猛虎慌得那三藏坐不稳雕鞍翻根头跌下白马斜倚在路旁真个是魂飞魄散。八戒丢了行李掣钉钯不让行者走上前大喝一声道:“孽畜!那里走!”赶将去劈头就筑。那只虎直挺挺站将起来把那前左爪轮起抠住自家的胸膛往下一抓唿剌的一声把个皮剥将下来站立道旁。你看他怎生恶相!咦那模样:

血津津的赤剥身躯红姢姢的弯环腿足。火焰焰的两鬓蓬松硬搠搠的双眉直竖。白森森的四个钢牙光耀耀的一双金眼。

气昂昂的努力大哮雄纠纠的厉声高喊。喊道:“慢来!慢来!

吾党不是别人乃是黄风大王部下的前路先锋。今奉大王严命在山巡逻要拿几个凡夫去做案酒。你是那里来的和尚敢擅动兵器伤我?”八戒骂道:“我把你这个孽畜!你是认不得我!

我等不是那过路的凡夫乃东土大唐御弟三藏之弟子奉旨上西方拜佛求经者。你早早的远避他方让开大路休惊了我师父饶你性命。若似前猖獗钯举处却不留情!”那妖精那容分说急近步丢一个架子望八戒劈脸来抓。这八戒忙闪过轮钯就筑。那怪手无兵器下头就走八戒随后赶来。那怪到了山坡下乱石丛中取出两口赤铜刀急轮起转身来迎。两个在这坡前一往一来一冲一撞的赌斗。那里孙行者搀起唐僧道:

“师父你莫害怕且坐住等老孙去助助八戒打倒那怪好走。”三藏才坐将起来战兢兢的口里念着《多心经》不题。那行者掣了铁棒喝声叫“拿了!”此时八戒抖擞精神那怪败下阵去。行者道:“莫饶他!务要赶上!”他两个轮钉钯举铁棒赶下山来。那怪慌了手脚使个金蝉脱壳计打个滚现了原身依然是一只猛虎。行者与八戒那里肯舍赶着那虎定要除根。那怪见他赶得至近却又抠着胸膛剥下皮来苫盖在那卧虎石上脱真身化一阵狂风径回路口。路口上那师父正念《多心经》被他一把拿住驾长风摄将去了。可怜那三藏啊:江流注定多磨折寂灭门中功行难。

那怪把唐僧擒来洞口按住狂风对把门的道:“你去报大王说前路虎先锋拿了一个和尚在门外听令。”那洞主传令教:“拿进来。”那虎先锋腰撇着两口赤铜刀双手捧着唐僧上前跪下道:“大王小将不才蒙钧令差往山上巡逻忽遇一个和尚他是东土大唐驾下御弟三藏法师上西方拜佛求经被我擒来奉上聊具一馔。”那洞主闻得此言吃了一惊道:“我闻得前后有人传说:三藏法师乃大唐奉旨意取经的神僧他手下有一个徒弟名唤孙行者神通广大智力高强。你怎么能彀捉得他来?”先锋道:“他有两个徒弟:先来的使一柄九齿钉钯他生得嘴长耳大;又一个使一根金箍铁棒他生得火眼金睛。正赶着小将争持被小将使一个金蝉脱壳之计撤身得空把这和尚拿来奉献大王聊表一餐之敬。”洞主道:“且莫吃他着。”先锋道:“大王见食不食呼为劣蹶。”洞主道:“你不晓得吃了他不打紧只恐怕他那两个徒弟上门吵闹未为稳便且把他绑在后园定风桩上待三五日他两个不来搅扰那时节一则图他身子干净二来不动口舌却不任我们心意?或煮或蒸或煎或炒慢慢的自在受用不迟。”先锋大喜道:“大王深谋远虑说得有理。”教:“小的们拿了去。”旁边拥上七八个绑缚手将唐僧拿去好便似鹰拿燕雀索绑绳缠。这的是苦命江流思行者遇难神僧想悟能道声:“徒弟啊!不知你在那山擒怪何处降妖我却被魔头拿来遭此毒害几时再得相见?好苦啊!你们若早些儿来还救得我命;若十分迟了断然不能保矣!”一边嗟叹一边泪落如雨。

却说那行者、八戒赶那虎下山坡只见那虎跑倒了塌伏在崖前行者举棒尽力一打转震得自己手疼。八戒复筑了一钯亦将钯齿迸起原来是一张虎皮盖着一块卧虎石。行者大惊道:“不好了!不好了!中了他计也!”八戒道:“中他甚计?”

行者道:“这个叫做金蝉脱壳计他将虎皮苫在此他却走了。

我们且回去看看师父莫遭毒手。”两个急急转来早已不见了三藏。行者大叫如雷道:“怎的好!师父已被他擒去了。”八戒即便牵着马眼中滴泪道:“天哪!天哪!却往那里找寻!”行者抬着头跳道:“莫哭!莫哭!一哭就挫了锐气。横竖想只在此山我们寻寻去来。”

他两个果奔入山中穿岗越岭行彀多时只见那石崖之下耸出一座洞府。两人定步观瞻果然凶险但见那:迭障尖峰回峦古道。青松翠竹依依绿柳碧梧冉冉。崖前有怪石双双林内有幽禽对对。涧水远流冲石壁山泉细滴漫沙堤。野云片片瑶草芊芊。妖狐狡兔乱撺梭角鹿香獐齐斗勇。劈崖斜挂万年藤深壑半悬千岁柏。奕奕巍巍欺华岳落花啼鸟赛天台。行者道:“贤弟你可将行李歇在藏风山凹之间撒放马匹不要出头。等老孙去他门与他赌斗必须拿住妖精方才救得师父。”八戒道:“不消吩咐请快去。”行者整一整直裰束一束虎裙掣了棒撞至那门前只见那门上有六个大字乃“黄风岭黄风洞”却便丁字脚站定执着棒高叫道:“妖怪!趁早儿送我师父出来省得掀翻了你窝巢躧平了你住处!”那小怪闻言一个个害怕战兢兢的跑入里面报道:“大王!祸事了!”那黄风怪正坐间问:“有何事?”小妖道:“洞门外来了一个雷公嘴毛脸的和尚手持着一根许大粗的铁棒要他师父哩!”那洞主惊张即唤虎先锋道:“我教你去巡山只该拿些山牛、野彘、肥鹿、胡羊怎么拿那唐僧来却惹他那徒弟来此闹吵怎生区处?”先锋道:“大王放心稳便高枕勿忧。小将不才愿带领五十个小妖校出去把那甚么孙行者拿来凑吃。”洞主道:“我这里除了大小头目还有五七百名小校凭你选择领多少去。只要拿住那行者我们才自自在在吃那和尚一块肉情愿与你拜为兄弟;但恐拿他不得反伤了你那时休得埋怨我也。”虎怪道:“放心!放心!等我去来。”果然点起五十名精壮小妖擂鼓摇旗缠两口赤铜刀腾出门来厉声高叫道:“你是那里来的个猴和尚敢在此间大呼小叫的做甚?”行者骂道:

“你这个剥皮的畜生!你弄甚么脱壳法儿把我师父摄了倒转问我做甚!趁早好好送我师父出来还饶你这个性命!”虎怪道:“你师父是我拿了要与我大王做顿下饭。你识起倒回去罢!不然拿住你一齐凑吃却不是买一个又饶一个?”行者闻言心中大怒扢迸迸钢牙错啮;滴流流火眼睁圆。掣铁棒喝道:“你多大欺心敢说这等大话!休走!看棍!”那先锋急持刀按住。这一场果然不善他两个各显威能。好杀:那怪是个真鹅卵悟空是个鹅卵石。赤铜刀架美猴王浑如垒卵来击石。鸟鹊怎与凤凰争?鹁鸽敢和鹰鹞敌?那怪喷风灰满山悟空吐雾云迷日。来往不禁三五回先锋腰软全无力。转身败了要逃生却被悟空抵死逼。

那虎怪撑持不住回头就走。他原来在那洞主面前说了嘴不敢回洞径往山坡上逃生。行者那里肯放执着棒只情赶来呼呼吼吼喊声不绝却赶到那藏风山凹之间。正抬头见八戒在那里放马。八戒忽听见呼呼声喊回头观看乃是行者赶败的虎怪就丢了马举起钯刺斜着头一筑。可怜那先锋脱身要跳黄丝网岂知又遇罩鱼人却被八戒一钯筑得九个窟窿鲜血冒一头脑髓尽流干。有诗为证诗曰:三五年前归正宗持斋把素悟真空。诚心要保唐三藏初秉沙门立此功。那呆子一脚躧住他的脊背两手轮钯又筑。行者见了大喜道:

“兄弟正是这等!他领了几十个小妖敢与老孙赌斗被我打败了他转不往洞跑却跑来这里寻死。亏你接着;不然又走了。”八戒道:“弄风摄师父去的可是他?”行者道:“正是正是。”八戒道:“你可曾问他师父的下落么?”行者道:“这怪把师父拿在洞里要与他甚么鸟大王做下饭。是老孙恼了就与他斗将这里来却着你送了性命。兄弟啊这个功劳算你的你可还守着马与行李等我把这死怪拖了去再到那洞口索战。须是拿得那老妖方才救得师父。”八戒道:“哥哥说得有理。你去你去若是打败了这老妖还赶将这里来等老猪截住杀他。”好行者一只手提着铁棒一只手拖着死虎径至他洞口。

正是:法师有难逢妖怪情性相和伏乱魔。毕竟不知此去可降得妖怪救得唐僧且听下回分解。
------------


------------

第二十二回 八戒大战流沙河 木叉奉法收悟净

话说唐僧师徒三众脱难前来不一日行过了八百黄风岭进西却是一脉平阳之地。光阴迅历夏经秋见了些寒蝉鸣败柳大火向西流。正行处只见一道大水狂澜浑波涌浪。

三藏在马上忙呼道:“徒弟你看那前边水势宽阔怎不见船只行走我们从那里过去?”八戒见了道:“果是狂澜无舟可渡。”

那行者跳在空中用手搭凉篷而看他也心惊道:“师父啊真个是难真个是难!这条河若论老孙去呵只消把腰儿扭一扭就过去了;若师父诚千分难渡万载难行。”三藏道:“我这里一望无边端的有多少宽阔?”行者道:“径过有八百里远近。”

八戒道:“哥哥怎的定得个远近之数?”行者道:“不瞒贤弟说老孙这双眼白日里常看得千里路上的吉凶。却才在空中看出:此河上下不知多远但只见这径过足有八百里。”长老忧嗟烦恼兜回马忽见岸上有一通石碑。三众齐来看时见上有三个篆字乃流沙河腹上有小小的四行真字云:“八百流沙界三千弱水深。鹅毛飘不起芦花定底沉。”师徒们正看碑文只听得那浪涌如山波翻若岭河当中滑辣的钻出一个妖精十分凶丑:一头红焰蓬松两只圆睛亮似灯。不黑不青蓝靛脸如雷如鼓老龙声。身披一领鹅黄氅腰束双攒露白藤。项下骷髅悬九个手持宝杖甚峥嵘。那怪一个旋风奔上岸来径抢唐僧慌得行者把师父抱住急登高岸回身走脱。那八戒放下担子掣出铁钯望妖精便筑那怪使宝杖架住。他两个在流沙河岸各逞英雄。这一场好斗:九齿钯降妖杖二人相敌河岸上。

这个是总督大天蓬那个是谪下卷帘将。昔年曾会在灵霄今日争持赌猛壮。这一个钯去探爪龙那一个杖架磨牙象。伸开大四平钻入迎风戗。这个没头没脸抓那个无乱无空放。一个是久占流沙界吃人精一个是秉教迦持修行将。他两个来来往往战经二十回合不分胜负。

那大圣护了唐僧牵着马守定行李见八戒与那怪交战就恨得咬牙切齿擦掌磨拳忍不住要去打他掣出棒来道:

“师父你坐着莫怕。等老孙和他耍耍儿来。”那师父苦留不住。他打个唿哨跳到前边。原来那怪与八戒正战到好处难解难分被行者轮起铁棒望那怪着头一下那怪急转身慌忙躲过径钻入流沙河里。气得个八戒乱跳道:“哥啊!谁着你来的!那怪渐渐手慢难架我钯再不上三五合我就擒住他了!

他见你凶险败阵而逃怎生是好!”行者笑道:“兄弟实不瞒你说自从降了黄风怪下山来这个把月不曾耍棍我见你和他战的甜美我就忍不住脚痒故就跳将来耍耍的。那知那怪不识耍就走了。”

他两个搀着手说说笑笑转回见了唐僧。唐僧道:“可曾捉得妖怪?”行者道:“那妖怪不奈战败回钻入水去也。”三藏道:“徒弟这怪久住于此他知道浅深。似这般无边的弱水又没了舟楫须是得个知水性的引领引领才好哩。”行者道:“正是这等说。常言道近朱者赤近墨者黑。那怪在此断知水性。

我们如今拿住他且不要打杀只教他送师父过河再做理会。”八戒道:“哥哥不必迟疑让你先去拿他等老猪看守师父。”行者笑道:“贤弟呀这桩儿我不敢说嘴。水里勾当老孙不大十分熟。若是空走还要捻诀又念念避水咒方才走得。

不然就要变化做甚么鱼虾蟹鳖之类我才去得。若论赌手段凭你在高山云里干甚么蹊跷异样事儿老孙都会只是水里的买卖有些儿榔杭。”八戒道:“老猪当年总督天河掌管了八万水兵大众倒学得知些水性却只怕那水里有甚么眷族老小七窝八代的都来我就弄他不过一时不被他捞去耶?”行者道:“你若到他水中与他交战却不要恋战许败不许胜把他引将出来等老孙下手助你。”八戒道:“言得是我去耶。”说声去就剥了青锦直裰脱了鞋双手舞钯分开水路使出那当年的旧手段跃浪翻波撞将进去径至水底之下往前正走。

却说那怪败了阵回方才喘定又听得有人推得水响忽起身观看原来是八戒执了钯推水。那怪举杖当面高呼道:“那和尚那里走!仔细看打!”八戒使钯架住道:“你是个甚么妖精敢在此间挡路?”那妖道:“你是也不认得我。我不是那妖魔鬼怪也不是少姓无名。”八戒道:“你既不是邪妖鬼怪却怎生在此伤生?你端的甚么姓名实实说来我饶你性命。”那怪道:

“我自小生来神气壮乾坤万里曾游荡。英雄天下显威名豪杰人家做模样。万国九州任我行五湖四海从吾撞。皆因学道荡天涯只为寻师游地旷。常年衣钵谨随身每日心神不可放。沿地云游数十遭到处闲行百余趟。因此才得遇真人引开大道金光亮。先将婴儿姹女收后把木母金公放。明堂肾水入华池重楼肝火投心脏。三千功满拜天颜志心朝礼明华向。玉皇大帝便加升亲口封为卷帘将。南天门里我为尊灵霄殿前吾称上。腰间悬挂虎头牌手中执定降妖杖。头顶金盔晃日光身披铠甲明霞亮。往来护驾我当先出入随朝予在上。只因王母降蟠桃设宴瑶池邀众将。失手打破玉玻璃天神个个魂飞丧。

玉皇即便怒生嗔却令掌朝左辅相:卸冠脱甲摘官衔将身推在杀场上。多亏赤脚大天仙越班启奏将吾放。饶死回生不典刑遭贬流沙东岸上。饱时困卧此山中饿去翻波寻食饷。樵子逢吾命不存渔翁见我身皆丧。来来往往吃人多翻翻复复伤生瘴。你敢行凶到我门今日肚皮有所望。莫言粗糙不堪尝拿住消停剁鲊酱!”八戒闻言大怒骂道:“你这泼物全没一些儿眼色!我老猪还掐出水沫儿来哩你怎敢说我粗糙要剁鲊酱!看起来你把我认做个老走硝哩。休得无礼!吃你祖宗这一钯!”那怪见钯来使一个凤点头躲过。两个在水中打出水面各人踏浪登波。这一场赌斗比前不同你看那:卷帘将天蓬帅各显神通真可爱。那个降妖宝杖着头轮这个九齿钉钯随手快。跃浪振山川推波昏世界。凶如太岁撞幛幡恶似丧门掀宝盖。这一个赤心凛凛保唐僧那一个犯罪滔滔为水怪。

钯抓一下九条痕杖打之时魂魄败。努力喜相持用心要赌赛。

算来只为取经人怒气冲天不忍耐。搅得那鯾鲌鲤鳜退鲜鳞龟鳖鼋鼍伤嫩盖;红虾紫蟹命皆亡水府诸神朝上拜。只听得波翻浪滚似雷轰日月无光天地怪。二人整斗有两个时辰不分胜败。这才是铜盆逢铁帚玉磬对金钟。

却说那大圣保着唐僧立于左右眼巴巴的望着他两个在水上争持只是他不好动手。只见那八戒虚幌一钯佯输诈败转回头往东岸上走。那怪随后赶来将近到了岸边这行者忍耐不住撇了师父掣铁棒跳到河边望妖精劈头就打。那妖物不敢相迎飕的又钻入河内。八戒嚷道:“你这弼马温真是个急猴子!你再缓缓些儿等我哄他到了高处你却阻住河边教他不能回呵却不拿住他也!他这进去几时又肯出来?”

行者笑道:“呆子莫嚷!莫嚷!我们且回去见师父去来。(WWW.mianhuatang.la 好看的小说)”八戒却同行者到高岸上见了三藏。三藏欠身道:“徒弟辛苦呀。”八戒道:“且不说辛苦只是降了妖精送得你过河方是万全之策。”三藏道:“你才与妖精交战何如?”八戒道:“那妖的手段与老猪是个对手。正战处使一个诈败他才赶到岸上。见师兄举着棍子他就跑了。”三藏道:“如此怎生奈何?”行者道:

“师父放心且莫焦恼。如今天色又晚且坐在这崖次之下待老孙去化些斋饭来你吃了睡去待明日再处。”八戒道:“说得是你快去快来。”行者急纵云跳起去正到直北下人家化了一钵素斋回献师父。师父见他来得甚快便叫:“悟空我们去化斋的人家求问他一个过河之策不强似与这怪争持?”行者笑道:“这家子远得很哩!相去有五七千里之路。他那里得知水性?问他何益?”八戒道:“哥哥又来扯谎了。五七千里路你怎么这等去来得快?”行者道:“你那里晓得老孙的觔斗云一纵有十万八千里。象这五七千路只消把头点上两点把腰躬上一躬就是个往回有何难哉!”八戒道:“哥啊既是这般容易你把师父背着只消点点头躬躬腰跳过去罢了何必苦苦的与他厮战?”行者道:“你不会驾云?你把师父驮过去不是?”八戒道:“师父的骨肉凡胎重似泰山我这驾云的怎称得起?须是你的觔斗方可。”行者道:“我的觔斗好道也是驾云只是去的有远近些儿。你是驮不动我却如何驮得动?自古道遣泰山轻如芥子携凡夫难脱红尘。象这泼魔毒怪使摄法弄风头却是扯扯拉拉就地而行不能带得空中而去。象那样法儿老孙也会使会弄。还有那隐身法、缩地法老孙件件皆知。

但只是师父要穷历异邦不能彀脱苦海所以寸步难行也。

我和你只做得个拥护保得他身在命在替不得这些苦恼也取不得经来就是有能先去见了佛那佛也不肯把经善与你我。正叫做若将容易得便作等闲看。”那呆子闻言喏喏听受。

遂吃了些无菜的素食师徒们歇在流沙河东崖次之下。

次早三藏道:“悟空今日怎生区处?”行者道:“没甚区处还须八戒下水。”八戒道:“哥哥你要图干净只作成我下水。”行者道:“贤弟这番我再不急性了只让你引他上来我拦住河沿不让他回去务要将他擒了。”好八戒抹抹脸抖擞精神双手拿钯到河沿分开水路依然又下至窝巢。那怪方才睡醒忽听推得水响急回头睁睛看看见八戒执钯下至他跳出来当头阻住喝道:“慢来!慢来!看杖!”八戒举钯架住道:

“你是个甚么哭丧杖叫你祖宗看杖!”那怪道:“你这厮甚不晓得哩!我这宝杖原来名誉大本是月里梭罗派。吴刚伐下一枝来鲁班制造工夫盖。里边一条金趁心外边万道珠丝玠。名称宝杖善降妖永镇灵霄能伏怪。只因官拜大将军玉皇赐我随身带。或长或短任吾心要细要粗凭意态。也曾护驾宴蟠桃也曾随朝居上界。值殿曾经众圣参卷帘曾见诸仙拜。养成灵性一神兵不是人间凡器械。自从遭贬下天门任意纵横游海外。不当大胆自称夸天下枪刀难比赛。看你那个锈钉钯只好锄田与筑菜!”八戒笑道:“我把你少打的泼物!且莫管甚么筑菜只怕荡了一下儿教你没处贴膏药九个眼子一齐流血!

纵然不死也是个到老的破伤风!”那怪丢开架子在那水底下与八戒依然打出水面。这一番斗比前果更不同你看他:

宝杖轮钉钯筑言语不通非眷属。只因木母克刀圭致令两下相战触。没输赢无反复翻波淘浪不和睦。这个怒气怎含容?

那个伤心难忍辱。钯来杖架逞英雄水滚流沙能恶毒。气昂昂劳碌碌多因三藏朝西域。钉钯老大凶宝杖十分熟。这个揪住要往岸上拖那个抓来就将水里沃。声如霹雳动鱼龙云暗天昏神鬼伏。这一场来来往往斗经三十回合不见强弱。八戒又使个佯输计拖了钯走。那怪随后又赶来拥波捉浪赶至崖边。八戒骂道:“我把你这个泼怪!你上来!这高处脚踏实地好打!”那妖骂道:“你这厮哄我上去又教那帮手来哩。你下来还在水里相斗。”原来那妖乖了再不肯上岸只在河沿与八戒闹吵。

却说行者见他不肯上岸急得他心焦性爆恨不得一把捉来。行者道:“师父!你自坐下等我与他个饿鹰雕食。”就纵筋斗跳在半空刷的落下来要抓那妖。那妖正与八戒嚷闹忽听得风响急回头见是行者落下云来却又收了那杖一头淬下水隐迹潜踪渺然不见。行者伫立岸上对八戒说:“兄弟呀这妖也弄得滑了。他再不肯上岸如之奈何?”八戒道:“难!

难!难!战不胜他就把吃奶的气力也使尽了只绷得个手平。”

行者道:“且见师父去。”

二人又到高岸见了唐僧备言难捉。那长老满眼下泪道:

“似此艰难怎生得渡!”行者道:“师父莫要烦恼。这怪深潜水底其实难行。八戒你只在此保守师父再莫与他厮斗等老孙往南海走走去来。”八戒道:“哥呵你去南海何干?”行者道:

“这取经的勾当原是观音菩萨;及脱解我等也是观音菩萨。

今日路阻流沙河不能前进不得他怎生处治?等我去请他还强如和这妖精相斗。”八戒道:“也是也是。师兄你去时千万与我上复一声:向日多承指教。”三藏道:“悟空若是去请菩萨却也不必迟疑快去赶来。”

行者即纵筋斗云径上南海。咦!那消半个时辰早望见普陀山境。须臾间坠下筋斗到紫竹林外又只见那二十四路诸天上前迎着道:“大圣何来?”行者道:“我师有难特来谒见菩萨。”诸天道:“请坐容报。”那轮日的诸天径至潮音洞口报道:“孙悟空有事朝见。”菩萨正与捧珠龙女在宝莲池畔扶栏看花闻报即转云岩开门唤入。大圣端肃皈依参拜菩萨问曰:

“你怎么不保唐僧?为甚事又来见我?”行者启上道:“菩萨我师父前在高老庄又收了一个徒弟唤名猪八戒多蒙菩萨又赐法讳悟能。才行过黄风岭今至八百里流沙河乃是弱水三千师父已是难渡。河中又有个妖怪武艺高强甚亏了悟能与他水面上大战三次只是不能取胜被他拦阻不能渡河。因此特告菩萨望垂怜悯。济渡他一济渡。”菩萨道:“你这猴子又逞自满不肯说出保唐僧的话来么?”行者道:“我们只是要拿住他教他送我师父渡河。水里事我又弄不得精细只是悟能寻着他窝巢与他打话想是不曾说出取经的勾当。”菩萨道:

“那流沙河的妖怪乃是卷帘大将临凡也是我劝化的善信教他保护取经之辈。你若肯说出是东土取经人呵他决不与你争持断然归顺矣。”行者道:“那怪如今怯战不肯上崖只在水里潜踪如何得他归顺?我师如何得渡弱水?”

菩萨即唤惠岸袖中取出一个红葫芦儿吩咐道:“你可将此葫芦同孙悟空到流沙河水面上只叫悟净他就出来了。先要引他归依了唐僧然后把他那九个骷髅穿在一处按九宫布列却把这葫芦安在当中就是法船一只能渡唐僧过流沙河界。”惠岸闻言谨遵师命当时与大圣捧葫芦出了潮音洞奉法旨辞了紫竹林。有诗为证诗曰:五行匹配合天真认得从前旧主人。炼已立基为妙用辨明邪正见原因。金来归性还同类木去求情共复沦。二土全功成寂寞调和水火没纤尘。

他两个不多时按落云头早来到流沙河岸。猪八戒认得是木叉行者引师父上前迎接。那木叉与三藏礼毕又与八戒相见。八戒道:“向蒙尊者指示得见菩萨我老猪果遵法教今喜拜了沙门。这一向在途中奔碌未及致谢恕罪恕罪。”行者道:

“且莫叙阔我们叫唤那厮去来。”三藏道:“叫谁?”行者道:“老孙见菩萨备陈前事。菩萨说:这流沙河的妖怪乃是卷帘大将临凡因为在天有罪堕落此河忘形作怪。他曾被菩萨劝化愿归师父往西天去的。但是我们不曾说出取经的事情故此苦苦争斗。菩萨今差木叉将此葫芦要与这厮结作法船渡你过去哩。”三藏闻言顶礼不尽对木叉作礼道:“万望尊者作一行。”那木叉捧定葫芦半云半雾径到了流沙河水面上厉声高叫道:“悟净!悟净!取经人在此久矣你怎么还不归顺!”却说那怪惧怕猴王回于水底正在窝中歇息只听得叫他法名情知是观音菩萨;又闻得说“取经人在此”他也不惧斧钺急翻波伸出头来又认得是木叉行者。你看他笑盈盈上前作礼道:“尊者失迎菩萨今在何处?”木叉道:“我师未来先差我来吩咐你早跟唐僧做个徒弟。叫把你项下挂的骷髅与这个葫芦按九宫结做一只法船渡他过此弱水。”悟净道:“取经人却在那里?”木叉用手指道:“那东岸上坐的不是?”悟净看见了八戒道:“他不知是那里来的个泼物与我整斗了这两日何曾言着一个取经的字儿?”又看见行者道:“这个主子是他的帮手好不利害!我不去了。”木叉道:“那是猪八戒这是孙行者俱是唐僧的徒弟俱是菩萨劝化的怕他怎的?我且和你见唐僧去。”那悟净才收了宝杖整一整黄锦直裰跳上岸来对唐僧双膝跪下道:“师父弟子有眼无珠不认得师父的尊容多有冲撞万望恕罪。”八戒道:“你这脓包怎的早不皈依只管要与我打?是何说话!”行者笑道:“兄弟你莫怪他还是我们不曾说出取经的事样与姓名耳。”长老道:“你果肯诚心皈依吾教么?”悟净道:“弟子向蒙菩萨教化指河为姓与我起了法名唤做沙悟净岂有不从师父之理!”三藏道:“既如此”叫:“悟空取戒刀来与他落了。”大圣依言即将戒刀与他剃了头。

又来拜了三藏拜了行者与八戒分了大小。三藏见他行礼真象个和尚家风故又叫他做沙和尚。木叉道:“既秉了迦持不必叙烦早与作法船去来。”那悟净不敢怠慢即将颈项下挂的骷髅取下用索子结作九宫把菩萨葫芦安在当中请师父下岸。那长老遂登法船坐于上面果然稳似轻舟。左有八戒扶持右有悟净捧托孙行者在后面牵了龙马半云半雾相跟头直上又有木叉拥护那师父才飘然稳渡流沙河界浪静风平过弱河。真个也如飞似箭不多时身登彼岸得脱洪波又不拖泥带水幸喜脚干手燥清净无为师徒们脚踏实地。那木叉按祥云收了葫芦又只见那骷髅一时解化作九股阴风寂然不见。三藏拜谢了木叉顶礼了菩萨。正是木叉径回东洋海三藏上马却投西。毕竟不知几时才得正果求经且听下回分解。
------------


------------

第二十四回 万寿山大仙留故友 五庄观行者窃人参

却说那三人穿林入里只见那呆子绷在树上声声叫喊痛苦难禁。行者上前笑道:“好女婿呀!这早晚还不起来谢亲又不到师父处报喜还在这里卖解儿耍子哩!咄!你娘呢?你老婆呢?好个绷巴吊拷的女婿呀!”那呆子见他来抢白着羞咬着牙忍着疼不敢叫喊。沙僧见了老大不忍放下行李上前解了绳索救下。呆子对他们只是磕头礼拜其实羞耻难当有《西江月》为证:色乃伤身之剑贪之必定遭殃。佳人二八好容妆更比夜叉凶壮。只有一个原本再无微利添囊。好将资本谨收藏坚守休教放荡。那八戒撮土焚香望空礼拜。行者道:

“你可认得那些菩萨么?”八戒道:“我已此晕倒昏迷眼花撩乱那认得是谁?”行者把那简帖儿递与八戒八戒见了是颂子更加惭愧。沙僧笑道:“二哥有这般好处哩感得四位菩萨来与你做亲!”八戒道:“兄弟再莫题起不当人子了!从今后再也不敢妄为。就是累折骨头也只是摩肩压担随师父西域去也。”三藏道:“既如此说才是。”

行者遂领师父上了大路。在路餐风宿水行罢多时忽见有高山挡路三藏勒马停鞭道:“徒弟前面一山必须仔细恐有妖魔作耗侵害吾党。”行者道:“马前但有我等三人怕甚妖魔?”因此长老安心前进。只见那座山真是好山:高山峻极大势峥嵘。根接昆仑脉顶摩霄汉中。白鹤每来栖桧柏玄猿时复挂藤萝。日映晴林迭迭千条红雾绕;风生阴壑飘飘万道彩云飞。幽鸟乱啼青竹里锦鸡齐斗野花间。只见那千年峰、五福峰、芙蓉峰巍巍凛凛放毫光;万岁石、虎牙石、三尖石突突磷磷生瑞气。崖前草秀岭上梅香。荆棘密森森芝兰清淡淡。深林鹰凤聚千禽古洞麒麟辖万兽。涧水有情曲曲弯弯多绕顾;峰峦不断重重迭迭自周回。又见那绿的槐斑的竹青的松依依千载斗秾华;白的李、红的桃翠的柳灼灼三春争艳丽。龙吟虎啸鹤舞猿啼。麋鹿从花出青鸾对日鸣。乃是仙山真福地蓬莱阆苑只如然。又见些花开花谢山头景云去云来岭上峰。三藏在马上欢喜道:“徒弟我一向西来经历许多山水都是那嵯峨险峻之处更不似此山好景果然的幽趣非常。若是相近雷音不远路我们好整肃端严见世尊。”行者笑道:“早哩!早哩!正好不得到哩!”沙僧道:“师兄我们到雷音有多少远?”行者道:“十万八千里十停中还不曾走了一停哩。”八戒道:“哥啊要走几年才得到?”行者道:“这些路若论二位贤弟便十来日也可到;若论我走一日也好走五十遭还见日色;若论师父走莫想!莫想!”唐僧道:“悟空你说得几时方可到?”行者道:“你自小时走到老老了再小老小千番也还难。只要你见性志诚念念回处即是灵山。”沙僧道:“师兄此间虽不是雷音观此景致必有个好人居止。”行者道:“此言却当。这里决无邪祟一定是个圣僧仙辈之乡我们游玩慢行。”不题。

却说这座山名唤万寿山山中有一座观名唤五庄观观里有一尊仙道号镇元子混名与世同君。那观里出一般异宝乃是混沌初分鸿蒙始判天地未开之际产成这颗灵根。盖天下四大部洲惟西牛贺洲五庄观出此唤名草还丹又名人参果。三千年一开花三千年一结果再三千年才得熟短头一万年方得吃。似这万年只结得三十个果子。果子的模样就如三朝未满的小孩相似四肢俱全五官咸备。人若有缘得那果子闻了一闻就活三百六十岁;吃一个就活四万七千年。

当日镇元大仙得元始天尊的筒帖邀他到上清天上弥罗宫中听讲混元道果。大仙门下出的散仙也不计其数见如今还有四十八个徒弟都是得道的全真。当日带领四十六个上界去听讲留下两个绝小的看家:一个唤做清风一个唤做明月。

清风只有一千三百二十岁明月才交一千二百岁。镇元子吩咐二童道:“不可违了大天尊的简帖要往弥罗宫听讲你两个在家仔细。不日有一个故人从此经过却莫怠慢了他可将我人参果打两个与他吃权表旧日之情。”二童道:“师父的故人是谁?望说与弟子好接待。”大仙道:“他是东土大唐驾下的圣僧道号三藏今往西天拜佛求经的和尚。”二童笑道:“孔子云道不同不相为谋。我等是太乙玄门怎么与那和尚做甚相识!”大仙道:“你那里得知。那和尚乃金蝉子转生西方圣老如来佛第二个徒弟。五百年前我与他在兰盆会上相识他曾亲手传茶佛子敬我故此是为故人也。”二仙童闻言谨遵师命。

那大仙临行又叮咛嘱咐道:“我那果子有数只许与他两个不得多费。”清风道:“开园时大众共吃了两个还有二十八个在树不敢多费。”大仙道:“唐三藏虽是故人须要防备他手下人罗唣不可惊动他知。”二童领命讫那大仙承众徒弟飞升径朝天界。

却说唐僧四众在山游玩忽抬头见那松篁一簇楼阁数层。唐僧道:“悟空你看那里是甚么去处?”行者看了道:“那所在不是观宇定是寺院。我们走动些到那厢方知端的。”不一时来于门观看见那松坡冷淡竹径清幽。往来白鹤送浮云上下猿猴时献果。那门前池宽树影长石裂苔花破。宫殿森罗紫极高楼台缥缈丹霞堕。真个是福地灵区蓬莱云洞。清虚人事少寂静道心生。青鸟每传王母信紫鸾常寄老君经。看不尽那巍巍道德之风果然漠漠神仙之宅。三藏离鞍下马又见那山门左边有一通碑碑上有十个大字乃是“万寿山福地五庄观洞天”。长老道:“徒弟真个是一座观宇。”沙僧道:“师父观此景鲜明观里必有好人居住。我们进去看看若行满东回此间也是一景。”行者道:“说得好。”遂都一齐进去又见那二门上有一对春联:长生不老神仙府与天同寿道人家。行者笑道:“这道士说大话唬人。我老孙五百年前大闹天宫时在那太上老君门也不曾见有此话说。”八戒道:“且莫管他进去进去或者这道士有些德行未可知也。”

及至二层门里只见那里面急急忙忙走出两个小童儿来。看他怎生打扮:骨清神爽容颜丽顶结丫髻短鬅。道服自然襟绕雾羽衣偏是袖飘风。环绦紧束龙头结芒履轻缠蚕口绒。丰采异常非俗辈正是那清风明月二仙童。那童子控背躬身出来迎接道:“老师父失迎请坐。”长老欢喜遂与二童子上了正殿观看。原来是向南的五间大殿都是上明下暗的雕花格子。那仙童推开格子请唐僧入殿只见那壁中间挂着五彩装成的“天地”二大字设一张朱红雕漆的香几几上有一副黄金炉瓶炉边有方便整香。

唐僧上前以左手拈香注炉三匝礼拜拜毕回头道:“仙童你五庄观真是西方仙界何不供养三清、四帝、罗天诸宰只将天地二字侍奉香火?”童子笑道:“不瞒老师说这两个字上头的礼上还当;下边的还受不得我们的香火。是家师父谄佞出来的。”三藏道:“何为谄佞?”童子道:“三清是家师的朋友四帝是家师的故人九曜是家师的晚辈元辰是家师的下宾。”那行者闻言就笑得打跌八戒道:“哥啊你笑怎的?”行者道:“只讲老孙会捣鬼原来这道童会捆风!”三藏道:“令师何在?”童子道:“家师元始天尊降简请到上清天弥罗宫听讲混元道果去了不在家。”行者闻言忍不住喝了一声道:“这个臊道童!人也不认得你在那个面前捣鬼扯甚么空心架子!那弥罗宫有谁是太乙天仙?请你这泼牛蹄子去讲甚么!”三藏见他怒恐怕那童子回言斗起祸来便道:“悟空且休争竞我们既进来就出去显得没了方情。常言道鹭鸶不吃鹭鸶肉。

他师既是不在搅扰他做甚?你去山门前放马沙僧看守行李教八戒解包袱取些米粮借他锅灶做顿饭吃待临行送他几文柴钱便罢了。各依执事让我在此歇息歇息饭毕就行。”

他三人果各依执事而去。

那明月、清风暗自夸称不尽道:“好和尚!真个是西方爱圣临凡真元不昧。师父命我们接待唐僧将人参果与他吃以表故旧之情又教防着他手下人罗唣。果然那三个嘴脸凶顽性情粗糙幸得就把他们调开了。若在边前却不与他人参果见面。”清风道:“兄弟还不知那和尚可是师父的故人问他一问看莫要错了。”二童子又上前道:“启问老师可是大唐往西天取经的唐三藏?”长老回礼道:“贫僧就是仙童为何知我贱名?”童子道:“我师临行曾吩咐教弟子远接。不期车驾来促有失迎迓。老师请坐待弟子办茶来奉。”三藏道:“不敢。”那明月急转本房取一杯香茶献与长老。茶毕清风道:“兄弟不可违了师命我和你去取果子来。”

二童别了三藏同到房中一个拿了金击子一个拿了丹盘又多将丝帕垫着盘底径到人参园内。那清风爬上树去使金击子敲果;明月在树下以丹盘等接。须臾敲下两个果来接在盘中径至前殿奉献道:“唐师父我五庄观土僻山荒无物可奉土仪素果二枚权为解渴。”那长老见了战战兢兢远离三尺道:“善哉!善哉!今岁倒也年丰时稔怎么这观里作荒吃人?这个是三朝未满的孩童如何与我解渴?”清风暗道:“这和尚在那口舌场中是非海里弄得眼肉胎凡不识我仙家异宝。”明月上前道:“老师此物叫做人参果吃一个儿不妨。”三藏道:“胡说!胡说!他那父母怀胎不知受了多少苦楚方生下未及三日怎么就把他拿来当果子?”清风道:“实是树上结的。”长老道:“乱谈!乱谈!树上又会结出人来?拿过去不当人子!”那两个童儿见千推万阻不吃只得拿着盘子转回本房。那果子却也跷蹊久放不得若放多时即僵了不中吃。二人到于房中一家一个坐在床边上只情吃起。

噫!原来有这般事哩!他那道房与那厨房紧紧的间壁这边悄悄的言语那边即便听见。八戒正在厨房里做饭先前听见说取金击子拿丹盘他已在心;又听见他说唐僧不认得是人参果即拿在房里自吃口里忍不住流涎道:“怎得一个儿尝新!”自家身子又狼犺不能彀得动只等行者来与他计较。

他在那锅门前更无心烧火不时的伸头探脑出来观看。不多时见行者牵将马来拴在槐树上径往后走那呆子用手乱招道:“这里来!这里来!”行者转身到于厨声门道:“呆子你嚷甚的?想是饭不彀吃且让老和尚吃饱我们前边大人家再化吃去罢。”八戒道:“你进来不是饭少。这观里有一件宝贝你可晓得?”行者道:“甚么宝贝?”八戒笑道:“说与你你不曾见;

拿与你你不认得。”行者道:“这呆子笑话我老孙。老孙五百年前因访仙道时也曾云游在海角天涯那般儿不曾见?”八戒道:“哥啊人参果你曾见么?”行者惊道:“这个真不曾见。但只常闻得人说人参果乃是草还丹人吃了极能延寿。如今那里有得?”八戒道:“他这里有。那童子拿两个与师父吃那老和尚不认得道是三朝未满的孩儿不曾敢吃。那童子老大惫懒师父既不吃便该让我们他就瞒着我们才自在这隔壁房里一家一个啯啅啯啅的吃了出去就急得我口里水泱。怎么得一个儿尝新?我想你有些溜撒去他那园子里偷几个来尝尝如何?”行者道:“这个容易老孙去手到擒来。”急抽身往前就走八戒一把扯住道:“哥啊我听得他在这房里说要拿甚么金击子去打哩。须是干得停当不可走露风声。”行者道:“我晓得我晓得。”

那大圣使一个隐身法闪进道房看时原来那两个道童吃了果子上殿与唐僧说话不在房里。行者四下里观看看有甚么金击子但只见窗棂上挂着一条赤金:有二尺长短有指头粗细;底下是一个蒜疙疸的头子;上边有眼系着一根绿绒绳儿。他道:“想必就是此物叫做金击子。”他却取下来出了道房径入后边去推开两扇门抬头观看呀!却是一座花园!但见:朱栏宝槛曲砌峰山。奇花与丽日争妍翠竹共青天斗碧。

流杯亭外一弯绿柳似拖烟;赏月台前数簇乔松如泼靛。红拂拂锦巢榴;绿依依绣墩草。青茸茸碧砂兰;攸荡荡临溪水。

丹桂映金井梧桐锦槐傍朱栏玉砌。有或红或白千叶桃有或香或黄九秋菊。荼蘼架映着牡丹亭;木槿台相连芍药圃。看不尽傲霜君子竹欺雪大夫松。更有那鹤庄鹿宅方沼圆池;泉流碎玉地萼堆金。朔风触绽梅花白春来点破海棠红。诚所谓人间第一仙景西方魁花丛。那行者观看不尽又见一层门推开看处却是一座菜园:布种四时蔬菜菠芹莙荙姜苔。

笋薯瓜瓠茭白葱蒜芫荽韭薤。窝蕖童蒿苦荬葫芦茄子须栽。

蔓菁萝卜羊头埋红苋青菘紫芥。行者笑道:“他也是个自种自吃的道士。”走过菜园又见一层门。推开看处呀!只见那正中间有根大树真个是青枝馥郁绿叶阴森那叶儿却似芭蕉模样直上去有千尺余高根下有七八丈围圆。那行者倚在树下往上一看只见向南的枝上露出一个人参果真个象孩儿一般。原来尾间上是个扢蒂看他丁在枝头手脚乱动点头幌脑风过处似乎有声。行者欢喜不尽暗自夸称道:“好东西呀!

果然罕见!果然罕见!”他倚着树飕的一声撺将上去。

那猴子原来第一会爬树偷果子。他把金击子敲了一下那果子扑的落将下来。他也随跳下来跟寻寂然不见四下里草中找寻更无踪影。行者道:“跷蹊!跷蹊!想是有脚的会走就走也跳不出墙去。我知道了想是花园中土地不许老孙偷他果子他收了去也。”他就捻着诀念一口“唵”字咒拘得那花园土地前来对行者施礼道:“大圣呼唤小神有何吩咐?”行者道:“你不知老孙是盖天下有名的贼头。我当年偷蟠桃、盗御酒、窃灵丹也不曾有人敢与我分用怎么今日偷他一个果子你就抽了我的头分去了!这果子是树上结的空中过鸟也该有分老孙就吃他一个有何大害?怎么刚打下来你就捞了去?”

土地道:“大圣错怪了小神也。这宝贝乃是地仙之物小神是个鬼仙怎么敢拿去?就是闻也无福闻闻。”行者道:“你既不曾拿去如何打下来就不见了?”土地道:“大圣只知这宝贝延寿更不知他的出处哩。”行者道:“有甚出处?”土地道:“这宝贝三千年一开花三千年一结果再三千年方得成熟。短头一万年只结得三十个。有缘的闻一闻就活三百六十岁;吃一个就活四万七千年。却是只与五行相畏。”行者道:“怎么与五行相畏?”土地道:“这果子遇金而落遇木而枯遇水而化遇火而焦遇土而入。敲时必用金器方得下来。打下来却将盘儿用丝帕衬垫方可;若受些木器就枯了就吃也不得延寿。吃他须用磁器清水化开食用遇火即焦而无用。遇土而入者大圣方才打落地上他即钻下土去了。这个土有四万七千年就是钢钻钻他也钻不动些须比生铁也还硬三四分人若吃了所以长生。大圣不信时可把这地下打打儿看。”行者即掣金箍棒筑了一下响一声迸起棒来土上更无痕迹。行者道:“果然!果然!我这棍打石头如粉碎撞生铁也有痕怎么这一下打不伤些儿?这等说我却错怪了你了你回去罢。”那土地即回本庙去讫。

大圣却有算计:爬上树一只手使击子一只手将锦布直裰的襟儿扯起来做个兜子等住他却串枝分叶敲了三个果兜在襟中跳下树一直前来径到厨房里去。那八戒笑道:“哥哥可有么?”行者道:“这不是?老孙的手到擒来。这个果子也莫背了沙僧可叫他一声。”八戒即招手叫道:“悟净你来。”

那沙僧撇下行李跑进厨房道:“哥哥叫我怎的?”行者放开衣兜道:“兄弟你看这个是甚的东西?”沙僧见了道:“是人参果。”行者道:“好啊!你倒认得你曾在那里吃过的?”沙僧道:

“小弟虽不曾吃但旧时做卷帘大将扶侍鸾舆赴蟠桃宴尝见海外诸仙将此果与王母上寿。见便曾见却未曾吃。哥哥可与我些儿尝尝?”行者道:“不消讲兄弟们一家一个。”他三人将三个果各各受用。那八戒食肠大口又大一则是听见童子吃时便觉馋虫拱动却才见了果子拿过来张开口毂辘的囫囵吞咽下肚却白着眼胡赖向行者、沙僧道:“你两个吃的是甚么?”沙僧道:“人参果。”八戒道:“甚么味道?”行者道:“悟净不要睬他!你倒先吃了又来问谁?”八戒道:“哥哥吃的忙了些不象你们细嚼细咽尝出些滋味。我也不知有核无核就吞下去了。哥啊为人为彻。已经调动我这馋虫再去弄个儿来老猪细细的吃吃。”行者道:“兄弟你好不知止足这个东西比不得那米食面食撞着尽饱。象这一万年只结得三十个我们吃他这一个也是大有缘法不等小可。罢罢罢!彀了!”

他欠起身来把一个金击子瞒窗眼儿丢进他道房里竟不睬他。

那呆子只管絮絮叨叨的唧哝不期那两个道童复进房来取茶去献只听得八戒还嚷甚么“人参果吃得不快活再得一个儿吃吃才好。”清风听见心疑道:“明月你听那长嘴和尚讲人参果还要个吃吃。师父别时叮咛教防他手下人罗唣莫敢是他偷了我们宝贝么?”明月回头道:“哥耶不好了!不好了!

金击子如何落在地下?我们去园里看看来!”他两个急急忙忙的走去只见花园开了清风道:“这门是我关的如何开了?”

又急转过花园只见菜园门也开了。忙入人参园里倚在树下望上查数;颠倒来往只得二十二个。明月道:“你可会算帐?”

清风道:“我会你说将来。”明月道:“果子原是三十个。师父开园分吃了两个还有二十八个;适才打两个与唐僧吃还有二十六个;如今止剩得二十二个却不少了四个?不消讲不消讲定是那伙恶人偷了我们只骂唐僧去来。”两个出了园门径来殿上指着唐僧秃前秃后秽语污言不绝口的乱骂;贼头鼠脑臭短臊长没好气的胡嚷。唐僧听不过道:“仙童啊你闹的是甚么?消停些儿有话慢说不妨不要胡说散道的。”清风说:“你的耳聋?我是蛮话你不省得?你偷吃了人参果怎么不容我说。”唐僧道:“人参果怎么模样?”明月道:“才拿来与你吃你说象孩童的不是?”唐僧道:“阿弥陀佛!那东西一见我就心惊胆战还敢偷他吃哩!就是害了馋痞也不敢干这贼事。

不要错怪了人。”清风道:“你虽不曾吃还有手下人要偷吃的哩。”三藏道:“这等也说得是你且莫嚷等我问他们看。果若是偷了教他赔你。”明月道:“赔呀!就有钱那里去买?”三藏道:“纵有钱没处买呵常言道仁义值千金。教他陪你个礼便罢了。也还不知是他不是他哩。”明月道:“怎的不是他?他那里分不均还在那里嚷哩。”三藏叫声:“徒弟且都来。”沙僧听见道:“不好了!决撒了!老师父叫我们小道童胡厮骂不是旧话儿走了风却是甚的?”行者道:“活羞杀人!这个不过是饮食之类。若说出来就是我们偷嘴了只是莫认。”八戒道:“正是正是昧了罢。”他三人只得出了厨房走上殿去。咦!毕竟不知怎么与他抵赖且听下回分解。
------------

第二十五回 镇元仙赶捉取经僧 孙行者大闹五庄观

却说他兄弟三众到了殿上对师父道:“饭将熟了叫我们怎的?”三藏道:“徒弟不是问饭。他这观里有甚么人参果似孩子一般的东西你们是那一个偷他的吃了?”八戒道:“我老实不晓得不曾见。”清风道:“笑的就是他!笑的就是他!”

行者喝道:“我老孙生的是这个笑容儿莫成为你不见了甚么果子就不容我笑?”三藏道:“徒弟息怒我们是出家人休打诳语莫吃昧心食果然吃了他的陪他个礼罢何苦这般抵赖?”行者见师父说得有理他就实说道:“师父不干我事是八戒隔壁听见那两个道童吃甚么人参果他想一个儿尝新着老孙去打了三个我兄弟各人吃了一个。如今吃也吃了待要怎么?”明月道:“偷了我四个这和尚还说不是贼哩!”八戒道:

“阿弥陀佛!既是偷了四个怎么只拿出三个来分预先就打起一个偏手?”那呆子倒转胡嚷。二仙童问得是实越加毁骂。就恨得个大圣钢牙咬响火眼睁圆把条金箍棒揝了又揝忍了又忍道:“这童子这样可恶只说当面打人也罢受他些气儿等我送他一个绝后计教他大家都吃不成!”好行者把脑后的毫毛拔了一根吹口仙气叫“变!”变做个假行者跟定唐僧陪着悟能、悟净忍受着道童嚷骂;他的真身出一个神纵云头跳将起去径到人参园里掣金箍棒往树上乒乓一下又使个推山移岭的神力把树一推推倒。可怜叶落枒开根出土道人断绝草还丹!那大圣推倒树却在枝儿上寻果子那里得有半个?原来这宝贝遇金而落他的棒刃头却是金裹之物况铁又是五金之类所以敲着就振下来既下来又遇土而入因此上边再没一个果子。他道:“好!好!好!大家散火!”他收了铁棒径往前来把毫毛一抖收上身来。那些人肉眼凡胎看不明白。

却说那仙童骂彀多时清风道:“明月这些和尚也受得气哩我们就象骂鸡一般骂了这半会通没个招声想必他不曾偷吃。倘或树高叶密数得不明不要诳骂了他!我和你再去查查。”明月道:“也说得是。”他两个果又到园中只见那树倒枒开果无叶落唬得清风脚软跌根头明月腰酥打骸垢。那两个魂飞魄散有诗为证诗曰:三藏西临万寿山悟空断送草还丹。枒开叶落仙根露明月清风心胆寒。他两个倒在尘埃语言颠倒只叫:“怎的好!怎的好!害了我五庄观里的丹头断绝我仙家的苗裔!师父来家我两个怎的回话?”明月道:“师兄莫嚷我们且整了衣冠莫要惊张了这几个和尚。这个没有别人定是那个毛脸雷公嘴的那厮他来出神弄法坏了我们的宝贝。若是与他分说那厮毕竟抵赖定要与他相争争起来就要交手相打你想我们两个怎么敌得过他四个?且不如去哄他一哄只说果子不少我们错数了转与他陪个不是。他们的饭已熟了等他吃饭时再贴他些儿小菜。他一家拿着一个碗你却站在门左我却站在门右扑的把门关倒把锁锁住将这几层门都锁了不要放他待师父来家凭他怎的处置。他又是师父的故人饶了他也是师父的人情;不饶他我们也拿住个贼在庶几可以免我等之罪。”清风闻言道:“有理!有理!”

他两个强打精神勉生欢喜从后园中径来殿上对唐僧控背躬身道:“师父适间言语粗俗多有冲撞莫怪莫怪。”三藏问道:“怎么说?”清风道:“果子不少只因树高叶密不曾看得明白。才然又去查查还是原数。”那八戒就趁脚儿跷道:“你这个童儿年幼不知事体就来乱骂白口咀咒枉赖了我们也!不当人子!”行者心上明白口里不言心中暗想道:“是谎是谎!果子已是了帐怎的说这般话?想必有起死回生之法。”

三藏道:“既如此盛将饭来我们吃了去罢。”那八戒便去盛饭沙僧安放桌椅。二童忙取小菜却是些酱瓜、酱茄、糟萝卜、醋豆角、腌窝蕖、绰芥菜共排了七八碟儿与师徒们吃饭;又提一壶好茶两个茶钟伺候左右。mianhuatang.la [棉花糖小说网]那师徒四众却才拿起碗来这童儿一边一个扑的把门关上插上一把两鐄铜锁。八戒笑道:“这童子差了。你这里风俗不好却怎的关了门里吃饭?”

明月道:“正是正是好歹吃了饭儿开门。”清风骂道:“我把你这个害馋劳、偷嘴的秃贼!你偷吃了我的仙果已该一个擅食田园瓜果之罪却又把我的仙树推倒坏了我五庄观里仙根你还要说嘴哩!若能彀到得西方参佛面只除是转背摇车再托生!”三藏闻言丢下饭碗把个石头放在心上。那童子将那前山门、二山门通都上了锁却又来正殿门恶语恶言贼前贼后只骂到天色将晚才去吃饭。饭毕归房去了。

唐僧埋怨行者道:“你这个猴头番番撞祸!你偷吃了他的果子就受他些气儿让他骂几句便也罢了。怎么又推倒他的树!若论这般情由告起状来就是你老子做官也说不通。”行者道:“师父莫闹那童儿都睡去了只等他睡着了我们连夜起身。”沙僧道:“哥啊几层门都上了锁闭得甚紧如何走么?”行者笑道:“莫管!莫管!老孙自有法儿。”八戒道:“愁你没有法儿哩!你一变变甚么虫蛭儿瞒格子眼里就飞将出去只苦了我们不会变的便在此顶缸受罪哩!”唐僧道:“他若干出这个勾当不同你我出去啊我就念起旧话经儿他却怎生消受!”八戒闻言又愁又笑道:“师父你说的那里话?我只听得佛教中有卷《楞严经》、《法华经》、《孔雀经》、《观音经》、《金刚经》不曾听见个甚那旧话儿经啊。”行者道:“兄弟你不知道我顶上戴的这个箍儿是观音菩萨赐与我师父的。师父哄我戴了就如生根的一般莫想拿得下来叫做《紧箍儿咒》又叫做《紧箍儿经》。他旧话儿经即此是也。但若念动我就头疼故有这个法儿难我。师父你莫念我决不负你管情大家一齐出去。”说话之间都已天昏不觉东方月上。行者道:“此时万籁无声冰轮明显正好走了去罢。”八戒道:“哥啊不要捣鬼门俱锁闭往那里走?”行者道:“你看手段!”好行者把金箍棒捻在手中使一个解锁法往门上一指只听得突蹡的一声响几层门双鐄俱落唿喇的开了门扇。八戒笑道:“好本事!

就是叫小炉儿匠使掭子便也不象这等爽利!”行者道:“这个门儿有甚稀罕!就是南天门指一指也开了。”却请师父出了门上了马八戒挑着担沙僧拢着马径投西路而去。行者道:

“你们且慢行等老孙去照顾那两个童儿睡一个月。”三藏道:

“徒弟不可伤他性命;不然又一个得财伤人的罪了。”行者道:“我晓得。”行者复进去来到那童儿睡的房门外。他腰里有带的瞌睡虫儿原来在东天门与增长天王猜枚耍子赢的。他摸出两个来瞒窗眼儿弹将进去径奔到那童子脸上鼾鼾沉睡再莫想得醒。他才拽开云步赶上唐僧顺大路一直西奔这一夜马不停蹄只行到天晓三藏道:“这个猴头弄杀我也!你因为嘴带累我一夜无眠!”行者道:“不要只管埋怨。天色明了你且在这路旁边树林中将就歇歇养养精神再走。”那长老只得下马倚松根权作禅床坐下沙僧歇了担子打盹八戒枕着石睡觉。孙大圣偏有心肠你看他跳树扳枝顽耍。四众歇息不题。

却说那大仙自元始宫散会领众小仙出离兜率径下瑶天坠祥云早来到万寿山五庄观门。看时只见观门大开地上干净大仙道:“清风、明月却也中用。常时节日高三丈腰也不伸今日我们不在他倒肯起早开门扫地。”众小仙俱悦。行至殿上香火全无人踪俱寂那里有明月、清风!众仙道:“他两个想是因我们不在拐了东西走了。”大仙道:“岂有此理!修仙的人敢有这般坏心的事!想是昨晚忘却关门就去睡了今早还未醒哩。”众仙到他房门看处真个关着房门鼾鼾沉睡。这外边打门乱叫那里叫得醒来?众仙撬开门板着手扯下床来也只是不醒。大仙笑道:“好仙童啊!成仙的人神满再不思睡却怎么这般困倦?莫不是有人做弄了他也?快取水来。”一童急取水半盏递与大仙。大仙念动咒语噀一口水喷在脸上随即解了睡魔。

二人方醒忽睁睛抹抹脸抬头观看认得是仙师与世同君和仙兄等众慌得那清风顿明月叩头道:“师父啊!你的故人原是东来的和尚一伙强盗十分凶狠!”大仙笑道:“莫惊恐慢慢的说来。”清风道:“师父啊当日别后不久果有个东土唐僧一行有四个和尚连马五口。弟子不敢违了师命问及来因将人参果取了两个奉上。那长老俗眼愚心不识我们仙家的宝贝。他说是三朝未满的孩童再三不吃是弟子各吃了一个。不期他那手下有三个徒弟有一个姓孙的名悟空行者先偷四个果子吃了。是弟子们向伊理说实实的言语了几句他却不容暗自里弄了个出神的手段苦啊!”二童子说到此处止不住腮边泪落。众仙道:“那和尚打你来?”明月道:“不曾打只是把我们人参树打倒了。”大仙闻言更不恼怒道:

“莫哭!莫哭!你不知那姓孙的也是个太乙散仙也曾大闹天宫神通广大。既然打倒了宝树你可认得那些和尚?”清风道:

“都认得。”大仙道:“既认得都跟我来。众徒弟们都收拾下刑具等我回来打他。”

众仙领命。大仙与明月、清风纵起祥光来赶三藏顷刻间就有千里之遥。大仙在云端里向西观看不见唐僧;及转头向东看时倒多赶了九百余里。原来那长老一夜马不停蹄只行了一百二十里路大仙的云头一纵赶过了九百余里。仙童道:

“师父那路旁树下坐的是唐僧。”大仙道:“我已见了。你两个回去安排下绳索等我自家拿他。”清风先回不题。

那大仙按落云头摇身一变变作个行脚全真。你道他怎生模样:穿一领百衲袍系一条吕公绦。手摇塵尾渔鼓轻敲。

三耳草鞋登脚下九阳巾子把头包。飘飘风满袖口唱《月儿高》。径直来到树下对唐僧高叫道:“长老贫道起手了。”那长老忙忙答礼道:“失瞻!失瞻!”大仙问:“长老是那方来的?为何在途中打坐?”三藏道:“贫僧乃东土大唐差往西天取经者。

路过此间权为一歇。”大仙佯讶道:“长老东来可曾在荒山经过?”长老道:“不知仙宫是何宝山?”大仙道:“万寿山五庄观便是贫道栖止处。”行者闻言他心中有物的人忙答道:“不曾!不曾!我们是打上路来的。”那大仙指定笑道:“我把你这个泼猴!你瞒谁哩?你倒在我观里把我人参果树打倒你连夜走在此间还不招认遮饰甚么?不要走!趁早去还我树来!”

那行者闻言心中恼怒掣铁棒不容分说望大仙劈头就打。大仙侧身躲过踏祥光径到空中。行者也腾云急赶上去。大仙在半空现了本相你看他怎生打扮:头戴紫金冠无忧鹤氅穿。

履鞋登足下丝带束腰间。体如童子貌面似美人颜。三须飘颔下鸦瓴叠鬓边。相迎行者无兵器止将玉塵手中拈。那行者没高没低的棍子乱打。大仙把玉塵左遮右挡奈了他两三回合使一个袖里乾坤的手段在云端里把袍袖迎风轻轻的一展刷地前来把四僧连马一袖子笼住。八戒道:“不好了!我们都装在拉縺里了!”行者道:“呆子不是拉縺我们被他笼在衣袖中哩。”八戒道:“这个不打紧等我一顿钉钯筑他个窟窿脱将下去只说他不小心笼不牢吊的了罢。”那呆子使钯乱筑那里筑得动?手捻着虽然是个软的筑起来就比铁还硬。

那大仙转祥云径落五庄观坐下叫徒弟拿绳来。众小仙一一伺候。你看他从袖子里却象撮傀儡一般把唐僧拿出缚在正殿檐柱上;又拿出他三个每一根柱上绑了一个;将马也拿出拴在庭下与他些草料行李抛在廊下。又道:“徒弟这和尚是出家人不可用刀枪不可加铁钺且与我取出皮鞭来打他一顿与我人参果出气!”众仙即忙取出一条鞭不是甚么牛皮、羊皮、麂皮、犊皮的原来是龙皮做的七星鞭着水浸在那里。令一个有力量的小仙把鞭执定道:“师父先打那个?”大仙道:“唐三藏做大不尊先打他。”行者闻言心中暗道:“我那老和尚不禁打假若一顿鞭打坏了啊却不是我造的业?”他忍不住开言道:“先生差了。偷果子是我吃果子是我推倒树也是我怎么不先打我打他做甚?”大仙笑道:“这泼猴倒言语膂烈。这等便先打他。”小仙问:“打多少?”大仙道:“照依果数打三十鞭。”那小仙轮鞭就打。行者恐仙家法大睁圆眼瞅定看他打那里。原来打腿行者就把腰扭一扭叫声“变!”变作两条熟铁腿看他怎么打。那小仙一下一下的打了三十天早向午了。大仙又吩咐道:“还该打三藏训教不严纵放顽徒撒泼。”那仙又轮鞭来打行者道:“先生又差了。偷果子时我师父不知他在殿上与你二童讲话是我兄弟们做的勾当。纵是有教训不严之罪我为弟子的也当替打再打我罢。”大仙笑道:“这泼猴虽是狡猾奸顽却倒也有些孝意。既这等还打他罢。”小仙又打了三十。行者低头看看两只腿似明镜一般通打亮了更不知些疼痒。此时天色将晚大仙道:“且把鞭子浸在水里待明朝再拷打他。”小仙且收鞭去浸各各归房。晚斋已毕尽皆安寝不题。

那长老泪眼双垂怨他三个徒弟道:“你等闯出祸来却带累我在此受罪这是怎的起?”行者道:“且休报怨打便先打我你又不曾吃打倒转嗟呀怎的?”唐僧道:“虽然不曾打却也绑得身上疼哩。”沙僧道:“师父还有陪绑的在这里哩。”行者道:“都莫要嚷再停会儿走路。”八戒道:“哥哥又弄虚头了。

这里麻绳喷水紧紧的绑着还比关在殿上被你使解锁法搠开门走哩!”行者道:“不是夸口说那怕他三股的麻绳喷上了水就是碗粗的棕缆也只好当秋风!”正话处早已万籁无声正是天街人静。好行者把身子小一小脱下索来道:“师父去哑!”沙僧慌了道:“哥哥也救我们一救!”行者道:“悄言!悄言!”他却解了三藏放下八戒、沙僧整束了褊衫扣背了马匹廊下拿了行李一齐出了观门。又教八戒:“你去把那崖边柳树伐四颗来。”八戒道:“要他怎的?”行者道:“有用处快快取来!”那呆子有些夯力走了去一嘴一颗就拱了四颗一抱抱来。行者将枝梢折了将兄弟二人复进去将原绳照旧绑在柱上。那大圣念动咒语咬破舌尖将血喷在树上叫“变!”一根变作长老一根变作自身那两根变作沙僧、八戒都变得容貌一般相貌皆同问他也就说话叫名也就答应。他两个却才放开步赶上师父。这一夜依旧马不停蹄躲离了五庄观。只走到天明那长老在马上摇桩打盹行者见了叫道:“师父不济!出家人怎的这般辛苦?我老孙千夜不眠也不晓得困倦。

且下马来莫教走路的人看见笑你权在山坡下藏风聚气处歇歇再走。”

不说他师徒在路暂住。且说那大仙天明起来吃了早斋出在殿上教拿鞭来:“今日却该打唐三藏了。”那小仙轮着鞭望唐僧道:“打你哩。”那柳树也应道:“打么。”乒乓打了三十。

轮过鞭来对八戒道:“打你哩。”那柳树也应道:“打么。”及打沙僧也应道“打么。”及打到行者那行者在路偶然打个寒噤道:“不好了!”三藏问道:“怎么说?”行者道:“我将四颗柳树变作我师徒四众我只说他昨日打了我两顿今日想不打了。却又打我的化身所以我真身打噤收了法罢。”那行者慌忙念咒收法。

你看那些道童害怕丢了皮鞭报道:“师父啊为头打的是大唐和尚这一会打的都是柳树之根!”大仙闻言呵呵冷笑夸不尽道:“孙行者真是一个好猴王!曾闻他大闹天宫布地网天罗拿他不住果有此理。你走了便也罢却怎么绑些柳树在此冒名顶替?决莫饶他赶去来!”那大仙说声赶纵起云头往西一望只见那和尚挑包策马正然走路。大仙低下云头叫声:“孙行者!往那里走!还我人参树来!”八戒听见道:

“罢了!对头又来了!”行者道:“师父且把善字儿包起让我们使些凶恶一结果了他脱身去罢。”唐僧闻言战战兢兢未曾答应沙僧掣宝杖八戒举钉钯大圣使铁棒一齐上前把大仙围住在空中乱打乱筑。这场恶斗有诗为证诗曰:悟空不识镇元仙与世同君妙更玄。三件神兵施猛烈一根塵尾自飘然。左遮右挡随来往后架前迎任转旋。夜去朝来难脱体淹留何日到西天!

他兄弟三众各举神兵一齐攻打那大仙只把蝇帚儿演架。那里有半个时辰他将袍袖一展依然将四僧一马并行李一袖笼去返云头又到观里。众仙接着仙师坐于殿上却又在袖儿里一个个搬出将唐僧绑在阶下矮槐树上八戒、沙僧各绑在两边树上。将行者捆倒行者道:“想是调问哩。”不一时捆绑停当教把长头布取十匹来。行者笑道:“八戒!这先生好意思拿出布来与我们做中袖哩!减省些儿做个一口中罢了。”那小仙将家机布搬将出来。大仙道:“把唐三藏、猪八戒、沙和尚都使布裹了!”众仙一齐上前裹了。行者笑道:“好!

好!好!夹活儿就大殓了!”须臾缠裹已毕又教拿出漆来。众仙即忙取了些自收自晒的生熟漆把他三个布裹的漆了浑身俱裹漆上留着头脸在外。八戒道:“先生上头倒不打紧只是下面还留孔儿我们好出恭。”那大仙又教把大锅抬出来。行者笑道:“八戒造化!抬出锅来想是煮饭我们吃哩。”八戒道:

“也罢了让我们吃些饭儿做个饱死的鬼也好看。”众仙果抬出一口大锅支在阶下。大仙叫架起干柴起烈火教:“把清油熬上一锅烧得滚了将孙行者下油锅扎他一扎与我人参树报仇!”行者闻言暗喜道:“正可老孙之意。这一向不曾洗澡有些儿皮肤燥痒好歹荡荡足感盛情。”顷刻间那油锅将滚。

大圣却又留心恐他仙法难参油锅里难做手脚急回头四顾只见那台下东边是一座日规台西边是一个石狮子。行者将身一纵滚到西边咬破舌尖把石狮子喷了一口叫声“变!”变作他本身模样也这般捆作一团他却出了元神起在云端里低头看着道士。

只见那小仙报道:“师父油锅滚透了。”大仙教“把孙行者抬下去!”四个仙童抬不动八个来也抬不动又加四个也抬不动。众仙道:“这猴子恋土难移小自小倒也结实。”却教二十个小仙扛将起来往锅里一掼烹的响了一声溅起些滚油点子把那小道士们脸上烫了几个燎浆大泡!只听得烧火的小童喊道:“锅漏了!锅漏了!”说不了油漏得罄尽锅底打破原来是一个石狮子放在里面。大仙大怒道:“这个泼猴着然无礼!教他当面做了手脚!你走了便罢怎么又捣了我的灶?这泼猴枉自也拿他不住就拿住他也似抟砂弄汞捉影捕风。

罢!罢!罢!饶他去罢。且将唐三藏解下另换新锅把他扎一扎与人参树报报仇罢。”那小仙真个动手拆解布漆。行者在半空里听得明白他想着:“师父不济他若到了油锅里一滚就死二滚就焦到三五滚他就弄做个稀烂的和尚了!我还去救他一救。”好大圣按落云头上前叉手道“莫要拆坏了布漆我来下油锅了。”那大仙惊骂道:“你这猢猴!怎么弄手段捣了我的灶?”行者笑道:“你遇着我就该倒灶干我甚事?我才自也要领你些油汤油水之爱但只是大小便急了若在锅里开风恐怕污了你的熟油不好调菜吃如今大小便通干净了才好下锅。不要扎我师父还来扎我。”那大仙闻言呵呵冷笑走出殿来一把扯住。毕竟不知有何话说端的怎么脱身且听下回分解。
------------

第二十六回 孙悟空三岛求方 观世音甘泉活树

诗曰:处世须存心上刃修身切记寸边而。常言刃字为生意但要三思戒怒欺。上士无争传亘古圣人怀德继当时。刚强更有刚强辈究竟终成空与非。却说那镇元大仙用手搀着行者道:“我也知道你的本事我也闻得你的英名只是你今番越理欺心纵有腾那脱不得我手。我就和你讲到西天见了你那佛祖也少不得还我人参果树。你莫弄神通!”行者笑道:“你这先生好小家子样!若要树活有甚疑难!早说这话可不省了一场争竞?”大仙道:“不争竞我肯善自饶你?”行者道:“你解了我师父我还你一颗活树如何?”大仙道:“你若有此神通医得树活我与你八拜为交结为兄弟。”行者道:“不打紧放了他们老孙管教还你活树。”大仙谅他走不脱即命解放了三藏、八戒、沙僧。沙僧道:“师父啊不知师兄捣得是甚么鬼哩。”

八戒道:“甚么鬼!这叫做当面人情鬼!树死了又可医得活?

他弄个光皮散儿好看者着求医治树单单了脱身走路还顾得你和我哩!”三藏道:“他决不敢撒了我们我们问他那里求医去。”遂叫道:“悟空你怎么哄了仙长解放我等?”行者道:

“老孙是真言实语怎么哄他?”三藏道:“你往何处去求方?”行者道:“古人云方从海上来。我今要上东洋大海遍游三岛十洲访问仙翁圣老求一个起死回生之法管教医得他树活。”

三藏道:“此去几时可回?”行者道:“只消三日。”三藏道:“既如此就依你说与你三日之限。三日里来便罢若三日之外不来我就念那话儿经了。”行者道:“遵命遵命。”你看他急整虎皮裙出门来对大仙道:“先生放心我就去就来。你却要好生伏侍我师父逐日家三茶六饭不可欠缺。若少了些儿老孙回来和你算帐先捣塌你的锅底。衣服禳了与他浆洗浆洗。脸儿黄了些儿我不要;若瘦了些儿不出门。”那大仙道:“你去你去定不教他忍饿。”

好猴王急纵觔斗云别了五庄观径上东洋大海。在半空中快如掣电疾如流星早到蓬莱仙境。按云头仔细观看真个好去处!有诗为证诗曰:大地仙乡列圣曹蓬莱分合镇波涛。瑶台影蘸天心冷巨阙光浮海面高。五色烟霞含玉籁九霄星月射金鳌。西池王母常来此奉祝三仙几次桃。那行者看不尽仙景径入蓬莱。正然走处见白云洞外松阴之下有三个老儿围棋:观局者是寿星对局者是福星、禄星。行者上前叫道:“老弟们作揖了。”那三星见了拂退棋枰回礼道:“大圣何来?”行者道:“特来寻你们耍子。”寿星道:“我闻大圣弃道从释脱性命保护唐僧往西天取经遂日奔波山路那些儿得闲却来耍子?”行者道:“实不瞒列位说老孙因往西方行在半路有些儿阻滞特来小事欲干不知肯否?”福星道:“是甚地方?是何阻滞?乞为明示吾好裁处。”行者道:“因路过万寿山五庄观有阻。”三老惊讶道:“五庄观是镇元大仙的仙宫。你莫不是把他人参果偷吃了”行者笑道:“偷吃了能值甚么?”三老道:“你这猴子不知好歹。那果子闻一闻活三百六十岁;吃一个活四万七千年叫做万寿草还丹。我们的道不及他多矣!

他得之甚易就可与天齐寿;我们还要养精、炼气、存神调和龙虎捉坎填离不知费多少工夫。你怎么说他的能值甚紧?天下只有此种灵根!”行者道:“灵根!灵根!我已弄了他个断根哩!”三老惊道:“怎的断根?”行者道:“我们前日在他观里那大仙不在家只有两个小童接待了我师父却将两个人参果奉与我师。mianhuatang.la [棉花糖小说网]我师不认得只说是三朝未满的孩童再三不吃。那童子就拿去吃了不曾让得我们。是老孙就去偷了他三个我三兄弟吃了。那童子不知高低贼前贼后的骂个不住。是老孙恼了把他树打了一棍推倒在地树上果子全无桠开叶落根出枝伤已枯死了。不想那童子关住我们又被老孙扭开锁走了。次日清晨那先生回家赶来问答间语言不和遂与他赌斗被他闪一闪把袍袖展开一袖子都笼去了。绳缠索绑拷问鞭敲就打了一日。是夜又逃了他又赶上依旧笼去。他身无寸铁只是把个塵尾遮架我兄弟这等三般兵器莫想打得着他。这一番仍旧摆布将布裹滚了我师父与两师弟却将我下油锅。我又做了个脱身本事走了把他锅都打破。他见拿我不住尽有几分醋我。是我又与他好讲教他放了我师父、师弟我与他医树管活两家才得安宁。我想着方从海上来故此特游仙境访三位老弟有甚医树的方儿传我一个急救唐僧脱苦。”三星闻言心中也闷道:“你这猴儿全不识人。那镇元子乃地仙之祖我等乃神仙之宗;你虽得了天仙还是太乙散数未入真流你怎么脱得他手?若是大圣打杀了走兽飞禽蜾虫鳞长只用我黍米之丹可以救活。那人参果乃仙木之根如何医治?没方没方。”那行者见说无方却就眉峰双锁额蹙千痕。福星道:“大圣此处无方他处或有怎么就生烦恼?”行者道:“无方别访果然容易就是游遍海角天涯转透三十六天亦是小可;只是我那唐长老法严量窄止与了我三日期限。三日以外不到他就要念那《紧箍儿咒》哩。”三星笑道:“好!好!

好!若不是这个法儿拘束你你又钻天了。”寿星道:“大圣放心不须烦恼。那大仙虽称上辈却也与我等有识。一则久别不曾拜望;二来是大圣的人情。如今我三人同去望他一望就与你道达此情教那唐和尚莫念《紧箍儿咒》休说三日五日只等你求得方来我们才别。”行者道:感激!感激!就请三位老弟行行我去也。”大圣辞别三星不题。

却说这三星驾起祥光即往五庄观而来。那观中合众人等忽听得长天鹤唳原来是三老光临。但见那:盈空蔼蔼祥光簇霄汉纷纷香馥郁。彩雾千条护羽衣轻云一朵擎仙足。青鸾飞丹凤鷫袖引香风满地扑。拄杖悬龙喜笑生皓髯垂玉胸前拂。童颜欢悦更无忧壮体雄威多有福。执星筹添海屋腰挂葫芦并宝箓。万纪千旬福寿长十洲三岛随缘宿。常来世上送千祥每向人间增百福。概乾坤荣福禄福寿无疆今喜得。

三老乘祥谒大仙福堂和气皆无极。那仙童看见即忙报道:

“师父海上三星来了。”镇元子正与唐僧师弟闲叙闻报即降阶奉迎。那八戒见了寿星近前扯住笑道:“你这肉头老儿许久不见还是这般脱洒帽儿也不带个来。”遂把自家一个僧帽扑的套在他头上扑着手呵呵大笑道:“好!好!好!真是加冠进禄也!”那寿星将帽子掼了骂道:“你这个夯货老大不知高低!”八戒道:“我不是夯货你等真是奴才!”福星道:“你倒是个夯货反敢骂人是奴才!”八戒又笑道:“既不是人家奴才好道叫做添寿、添福、添禄?”那三藏喝退了八戒急整衣拜了三星。那三星以晚辈之礼见了大仙方才叙坐。坐定禄星道:“我们一向久阔尊颜有失恭敬今因孙大圣搅扰仙山特来相见。(WWW.mianhuatang.la 好看的小说)”大仙道:“孙行者到蓬莱去的?”寿星道:“是因为伤了大仙的丹树他来我处求方医治我辈无方他又到别处求访但恐违了圣僧三日之限要念《紧箍儿咒》。我辈一来奉拜二来讨个宽限。”三藏闻言连声应道:“不敢念不敢念。”正说处八戒又跑进来扯住福星要讨果子吃。他去袖里乱摸腰里乱吞不住的揭他衣服搜检。三藏笑道:“那八戒是甚么规矩!”八戒道:“不是没规矩此叫做番番是福。”三藏又叱令出去那呆子喑雒懦蜃鸥P茄鄄蛔Φ暮莞P堑溃骸昂换酰∥夷抢锬樟四憷茨阏獾群尬遥俊卑私涞溃骸安皇呛弈阏饨谢赝吠!!蹦谴糇映龅妹爬粗患桓鲂⊥昧怂陌巡璩追饺パ按溉」床璞凰话讯峁苌系钅米判№喽檬致仪寐掖蛄酵吠嫠!4笙傻溃骸罢飧龊蜕性讲蛔鹬亓耍 卑私湫Φ溃骸安皇遣蛔鹬卣饨凶鏊氖奔臁!

且不说八戒打诨乱缠却表行者纵祥云离了蓬莱又早到方丈仙山。这山真好去处有诗为证诗曰:方丈巍峨别是天太元宫府会神仙。紫台光照三清路花木香浮五色烟。金凤自多槃蕊阙玉膏谁逼灌芝田?碧桃紫李新成熟又换仙人信万年。那行者按落云头无心玩景正走处只闻得香风馥馥玄鹤声鸣那壁厢有个神仙但见:盈空万道霞光现彩雾飘飖光不断。丹凤衔花也更鲜青鸾飞舞声娇艳。福如东海寿如山貌似小童身体健。壶隐洞天不老丹腰悬与日长生篆。人间数次降祯祥世上几番消厄愿。武帝曾宣加寿龄瑶池每赴蟠桃宴。教化众僧脱俗缘指开大道明如电。也曾跨海祝千秋常去灵山参佛面。圣号东华大帝君烟霞第一神仙眷。孙行者觌面相迎叫声:“帝君起手了。”那帝君慌忙回礼道:“大圣失迎。请荒居奉茶。”遂与行者搀手而入。果然是贝阙仙宫看不尽瑶池琼阁。方坐待茶只见翠屏后转出一个童儿。他怎生打扮:身穿道服飘霞烁腰束丝绦光错落。头戴纶巾布斗星足登芒履游仙岳。炼元真脱本壳功行成时遂意乐。识破原流精气神主人认得无虚错。逃名今喜寿无疆甲子周天管不着。转回廊登宝阁天上蟠桃三度摸。缥缈香云出翠屏小仙乃是东方朔。行者见了笑道:“这个小贼在这里哩!帝君处没有桃子你偷吃!”东方朔朝上进礼答道:“老贼你来这里怎的?我师父没有仙丹你偷吃。”帝君叫道:“曼倩休乱言看茶来也。”曼倩原是东方朔的道名他急入里取茶二杯。饮讫行者道:“老孙此来有一事奉干未知允否?”帝君道:“何事?自当领教。”

行者道:“近因保唐僧西行路过万寿山五庄观因他那小童无状是我一时怒把他人参果树推倒因此阻滞唐僧不得脱身特来尊处求赐一方医治万望慨然。”帝君道:“你这猴子不管一二到处里闯祸。那五庄观镇元子圣号与世同君乃地仙之祖。你怎么就冲撞出他?他那人参果树乃草还丹。你偷吃了尚说有罪;却又连树推倒他肯干休?”行者道:“正是呢我们走脱了被他赶上把我们就当汗巾儿一般一袖子都笼了去所以角气。没奈何许他求方医治故此拜求。”帝君道:

“我有一粒九转太乙还丹但能治世间生灵却不能医树。树乃水土之灵天滋地润。若是凡间的果木医治还可;这万寿山乃先天福地五庄观乃贺洲洞天人参果又是天开地辟之灵根如何可治?无方!无方!”

行者道:“既然无方老孙告别。”帝君仍欲留奉玉液一杯行者道:“急救事紧不敢久滞。”遂驾云至瀛洲海岛。也好去处有诗为证诗曰:珠树玲珑照紫烟瀛洲宫阙接诸天。青山绿水琪花艳玉液锟鋘铁石坚。五色碧鸡啼海日千年丹凤吸朱烟。世人罔究壶中景象外春光亿万年。那大圣至瀛洲只见那丹崖珠树之下有几个皓皤髯之辈童颜鹤鬓之仙在那里着棋饮酒谈笑讴歌。真个是:祥云光满瑞霭香浮。彩鸾鸣洞口玄鹤舞山头。碧藕水桃为按酒交梨火枣寿千秋。一个个丹诏无闻仙符有籍;逍遥随浪荡散淡任清幽。周天甲子难拘管大地乾坤只自由献果玄猿对对参随多美爱;衔花白鹿双双拱伏甚绸缪。那些老儿正然洒乐这行者厉声高叫道:

“带我耍耍儿便怎的!”众仙见了急忙趋步相迎。有诗为证诗曰:人参果树灵根折大圣访仙求妙诀。缭绕丹霞出宝林瀛洲九老来相接。行者认得是九老笑道:“老兄弟们自在哩!”九老道:“大圣当年若存正不闹天宫比我们还自在哩。如今好了闻你归真向西拜佛如何得暇至此?”行者将那医树求方之事具陈了一遍。九老也大惊道:“你也忒惹祸!惹祸!我等实是无方。”行者道:“既是无方我且奉别。”

九老又留他饮琼浆食碧藕。行者定不肯坐止立饮了他一杯浆吃了一块藕急急离了瀛洲径转东洋大海。早望见落伽山不远遂落下云头直到普陀岩上见观音菩萨在紫竹林中与诸天大神、木叉、龙女讲经说法。有诗为证诗曰:海主城高瑞气浓更观奇异事无穷。须知隐约千般外尽出希微一品中。四圣授时成正果六凡听后脱樊笼。少林别有真滋味花果馨香满树红。

那菩萨早已看见行者来到即命守山大神去迎。那大神出林来叫声:“孙悟空那里去?”行者抬头喝道:“你这个熊罴!

我是你叫的悟空?当初不是老孙饶了你你已此做了黑风山的尸鬼矣。今日跟了菩萨受了善果居此仙山常听法教你叫不得我一声老爷?”那黑熊真个得了正果在菩萨处镇守普陀称为大神是也亏了行者。他只得陪笑道:“大圣古人云君子不念旧恶只管题他怎的!菩萨着我来迎你哩。”这行者就端肃尊诚与大神到了紫竹林里参拜菩萨。菩萨道:“悟空唐僧行到何处也?”行者道:“行到西牛贺洲万寿山了。”菩萨道:“那万寿山有座五庄观镇元大仙你曾会他么?”行者顿道:“因是在五庄观弟子不识镇元大仙毁伤了他的人参果树冲撞了他他就困滞了我师父不得前进。”那菩萨情知怪道:“你这泼猴不知好歹!他那人参果树乃天开地辟的灵根。镇元子乃地仙之祖我也让他三分你怎么就打伤他树!”行者再拜道:“弟子实是不知。那一日他不在家只有两个仙童候待我等。是猪悟能晓得他有果子要一个尝新弟子委偷了他三个兄弟们分吃了。那童子知觉骂我等无已是弟子怒遂将他树推倒。他次日回来赶上将我等一袖子笼去绳绑鞭抽拷打了一日。我等当夜走脱又被他赶上依然笼了。三番两次其实难逃已允了与他医树。却才自海上求方遍游三岛众神仙都没有本事。弟子因此志心朝礼特拜告菩萨伏望慈悯俯赐一方以救唐僧早早西去。”菩萨道:“你怎么不早来见我却往岛上去寻找?”行者闻得此言心中暗喜道:“造化了!造化了!

菩萨一定有方也!”他又上前恳求菩萨道:“我这净瓶底的甘露水善治得仙树灵苗。”行者道:“可曾经验过么?”菩萨道:

“经验过的。”行者问:“有何经验?”菩萨道:“当年太上老君曾与我赌胜:他把我的杨柳枝拔了去放在炼丹炉里炙得焦干送来还我。是我拿了插在瓶中一昼夜复得青枝绿叶与旧相同。”行者笑道:“真造化了!真造化了!烘焦了的尚能医活况此推倒的有何难哉!”菩萨吩咐大众:“看守林中我去去来。”

遂手托净瓶白鹦哥前边巧啭孙大圣随后相从。有诗为证诗曰:玉毫金象世难论正是慈悲救苦尊。过去劫逢无垢佛至今成得有为身。几生欲海澄清浪一片心田绝点尘。甘露久经真妙法管教宝树永长春。

却说那观里大仙与三老正然清话忽见孙大圣按落云头叫道:“菩萨来了快接快接!”慌得那三星与镇元子共三藏师徒一齐迎出宝殿。菩萨才住了祥云先与镇元子陪了话后与三星作礼。礼毕上坐那阶前行者引唐僧、八戒、沙僧都拜了。

那观中诸仙也来拜见。行者道:“大仙不必迟疑趁早儿陈设香案请菩萨替你治那甚么果树去。”大仙躬身谢菩萨道:“小可的勾当怎么敢劳菩萨下降?”菩萨道:“唐僧乃我之弟子孙悟空冲撞了先生理当赔偿宝树。”三老道:“既如此不须谦讲了。请菩萨都到园中去看看。”

那大仙即命设具香案打扫后园请菩萨先行三老随后。

三藏师徒与本观众仙都到园内观看时那棵树倒在地下土开根现叶落枝枯。菩萨叫:“悟空伸手来。”那行者将左手伸开。菩萨将杨柳枝蘸出瓶中甘露把行者手心里画了一道起死回生的符字教他放在树根之下但看水出为度。那行者捏着拳头往那树根底下揣着须臾有清泉一汪。菩萨道:“那个水不许犯五行之器须用玉瓢舀出扶起树来从头浇下自然根皮相合叶长芽生枝青果出。”行者道:“小道士们快取玉瓢来。”镇元子道:“贫道荒山没有玉瓢只有玉茶盏、玉酒杯可用得么?”菩萨道:“但是玉器可舀得水的便罢取将来看。”

大仙即命小童子取出有二三十个茶盏四五十个酒盏却将那根下清泉舀出。行者、八戒、沙僧扛起树来扶得周正拥上土将玉器内甘泉一瓯瓯捧与菩萨。菩萨将杨柳枝细细洒上口中又念着经咒。不多时洒净那舀出之水只见那树果然依旧青枝绿叶浓郁阴森上有二十三个人参果。清风、明月二童子道:“前日不见了果子时颠倒只数得二十二个今日回生怎么又多了一个?”行者道:“日久见人心。前日老孙只偷了三个那一个落下地来土地说这宝遇土而入八戒只嚷我打了偏手故走了风信只缠到如今才见明白。”菩萨道:“我方才不用五行之器者知道此物与五行相畏故耳。”那大仙十分欢喜急令取金击子来把果子敲下十个请菩萨与三老复回宝殿一则谢劳二来做个人参果会。众小仙遂调开桌椅铺设丹盘请菩萨坐了上面正席三老左席唐僧右席镇元子前席相陪各食了一个。有诗为证诗曰:万寿山中古洞天人参一熟九千年。灵根现出芽枝损甘露滋生果叶全。三老喜逢皆旧契四僧幸遇是前缘。自今会服人参果尽是长生不老仙。此时菩萨与三老各吃了一个唐僧始知是仙家宝贝也吃了一个悟空三人亦各吃一个镇元子陪了一个本观仙众分吃了一个。

行者才谢了菩萨回上普陀岩送三星径转蓬莱岛。镇元子却又安排蔬酒与行者结为兄弟。这才是不打不成相识两家合了一家。师徒四众喜喜欢欢天晚歇了。那长老才是:有缘吃得草还丹长寿苦捱妖怪难。毕竟到明日如何作别且听下回分解。
------------

第二十七回 尸魔三戏唐三藏 圣僧恨逐美猴王

却说三藏师徒次日天明收拾前进。mianhuatang.la [棉花糖小说网]那镇元子与行者结为兄弟两人情投意合决不肯放又安排管待一连住了五六日。那长老自服了草还丹真似脱胎换骨神爽体健。他取经心重那里肯淹留无已遂行。

师徒别了上路早见一座高山。三藏道:“徒弟前面有山险峻恐马不能前大家须仔细仔细。”行者道:“师父放心我等自然理会。”好猴王他在那马前横担着棒剖开山路上了高崖看不尽:峰岩重叠涧壑湾环。虎狼成阵走麂鹿作群行。

无数獐豝钻簇簇满山狐兔聚丛丛。千尺大蟒万丈长蛇。大蟒喷愁雾长蛇吐怪风。道旁荆棘牵漫岭上松楠秀丽。薜萝满目芳草连天。影落沧溟北云开斗柄南。万古常含元气老千峰巍列日光寒。那长老马上心惊孙大圣布施手段舞着铁棒哮吼一声唬得那狼虫颠窜虎豹奔逃。师徒们入此山正行到嵯峨之处三藏道:“悟空我这一日肚中饥了你去那里化些斋吃?”行者陪笑道:“师父好不聪明。这等半山之中前不巴村后不着店有钱也没买处教往那里寻斋?”三藏心中不快口里骂道:“你这猴子!想你在两界山被如来压在石匣之内口能言足不能行也亏我救你性命摩顶受戒做了我的徒弟。怎么不肯努力常怀懒惰之心!”行者道:“弟子亦颇殷勤何尝懒惰?”三藏道:“你既殷勤何不化斋我吃?我肚饥怎行?况此地山岚瘴气怎么得上雷音?”行者道:“师父休怪少要言语。我知你尊性高傲十分违慢了你便要念那话儿咒。你下马稳坐等我寻那里有人家处化斋去。”行者将身一纵跳上云端里手搭凉篷睁眼观看。可怜西方路甚是寂寞更无庄堡人家正是多逢树木少见人烟去处。看多时只见正南上有一座高山那山向阳处有一片鲜红的点子。行者按下云头道:

“师父有吃的了。”那长老问甚东西行者道:“这里没人家化饭那南山有一片红的想必是熟透了的山桃我去摘几个来你充饥。”三藏喜道:“出家人若有桃子吃就为上分了快去!”

行者取了钵盂纵起祥光你看他觔斗幌幌冷气飕飕须臾间奔南山摘桃不题。

却说常言有云:山高必有怪岭峻却生精。果然这山上有一个妖精孙大圣去时惊动那怪。他在云端里踏着阴风看见长老坐在地下就不胜欢喜道:“造化!造化!几年家人都讲东土的唐和尚取大乘他本是金蝉子化身十世修行的原体。

有人吃他一块肉长寿长生。真个今日到了。”那妖精上前就要拿他只见长老左右手下有两员大将护持不敢拢身。他说两员大将是谁?说是八戒、沙僧。八戒、沙僧虽没甚么大本事然八戒是天蓬元帅沙僧是卷帘大将他的威气尚不曾泄故不敢拢身。妖精说:“等我且戏他戏看怎么说。”

好妖精停下阴风在那山凹里摇身一变变做个月貌花容的女儿说不尽那眉清目秀齿白唇红左手提着一个青砂罐儿右手提着一个绿磁瓶儿从西向东径奔唐僧。圣僧歇马在山岩忽见裙钗女近前。翠袖轻摇笼玉笋湘裙斜拽显金莲。

汗流粉面花含露尘拂峨眉柳带烟。仔细定睛观看处看看行至到身边。三藏见了叫:“八戒沙僧悟空才说这里旷野无人你看那里不走出一个人来了?”八戒道:“师父你与沙僧坐着等老猪去看看来。”那呆子放下钉钯整整直裰摆摆摇摇充作个斯文气象一直的觌面相迎。真个是远看未实近看分明那女子生得:冰肌藏玉骨衫领露酥胸。柳眉积翠黛杏眼闪银星。月样容仪俏天然性格清。体似燕藏柳声如莺啭林。

半放海棠笼晓日才开芍药弄春晴。那八戒见他生得俊俏呆子就动了凡心忍不住胡言乱语叫道:“女菩萨往那里去?手里提着是甚么东西?”分明是个妖怪他却不能认得。那女子连声答应道:“长老我这青罐里是香米饭绿瓶里是炒面筋特来此处无他故因还誓愿要斋僧。”八戒闻言满心欢喜急抽身就跑了个猪颠风报与三藏道:“师父!吉人自有天报!师父饿了教师兄去化斋那猴子不知那里摘桃儿耍子去了。桃子吃多了也有些嘈人又有些下坠。你看那不是个斋僧的来了?”唐僧不信道:“你这个夯货胡缠!我们走了这向好人也不曾遇着一个斋僧的从何而来!”八戒道:“师父这不到了?”

三藏一见连忙跳起身来合掌当胸道:“女菩萨你府上在何处住?是甚人家?有甚愿心来此斋僧?”分明是个妖精那长老也不认得。那妖精见唐僧问他来历他立地就起个虚情花言巧语来赚哄道:“师父此山叫做蛇回兽怕的白虎岭正西下面是我家。我父母在堂看经好善广斋方上远近僧人只因无子求福作福生了奴奴欲扳门第配嫁他人又恐老来无倚只得将奴招了一个女婿养老送终。”三藏闻言道:“女菩萨你语言差了。圣经云:父母在不远游游必有方。你既有父母在堂又与你招了女婿有愿心教你男子还便也罢怎么自家在山行走?又没个侍儿随从。这个是不遵妇道了。”

那女子笑吟吟忙陪俏语道:“师父我丈夫在山北凹里带几个客子锄田。这是奴奴煮的午饭送与那些人吃的。只为五黄六月无人使唤父母又年老所以亲身来送。忽遇三位远来却思父母好善故将此饭斋僧如不弃嫌愿表芹献。”三藏道:

“善哉!善哉!我有徒弟摘果子去了就来我不敢吃。假如我和尚吃了你饭你丈夫晓得骂你却不罪坐贫僧也?”那女子见唐僧不肯吃却又满面春生道:“师父啊我父母斋僧还是小可;我丈夫更是个善人一生好的是修桥补路爱老怜贫。但听见说这饭送与师父吃了他与我夫妻情上比寻常更是不同。”三藏也只是不吃旁边却恼坏了八戒。那呆子努着嘴口里埋怨道:“天下和尚也无数不曾象我这个老和尚罢软!现成的饭三分儿倒不吃只等那猴子来做四分才吃!”他不容分说一嘴把个罐子拱倒就要动口。

只见那行者自南山顶上摘了几个桃子托着钵盂一筋斗点将回来睁火眼金睛观看认得那女子是个妖精放下钵盂掣铁棒当头就打。唬得个长老用手扯住道:“悟空!你走将来打谁?”行者道:“师父你面前这个女子莫当做个好人。

他是个妖精要来骗你哩。”三藏道:“你这猴头当时倒也有些眼力今日如何乱道!这女菩萨有此善心将这饭要斋我等你怎么说他是个妖精?”行者笑道:“师父你那里认得!老孙在水帘洞里做妖魔时若想人肉吃便是这等:或变金银或变庄台或变醉人或变女色。有那等痴心的爱上我我就迷他到洞里尽意随心或蒸或煮受用;吃不了还要晒干了防天阴哩!师父我若来迟你定入他套子遭他毒手!”那唐僧那里肯信只说是个好人。行者道:“师父我知道你了你见他那等容貌必然动了凡心。若果有此意叫八戒伐几棵树来沙僧寻些草来我做木匠就在这里搭个窝铺你与他圆房成事我们大家散了却不是件事业?何必又跋涉取甚经去!”那长老原是个软善的人那里吃得他这句言语羞得个光头彻耳通红。三藏正在此羞惭行者又起性来掣铁棒望妖精劈脸一下。那怪物有些手段使个解尸法见行者棍子来时他却抖擞精神预先走了把一个假尸打死在地下。唬得个长老战战兢兢口中作念道:“这猴着然无礼!屡劝不从无故伤人性命!”行者道:“师父莫怪你且来看看这罐子里是甚东西。”沙僧搀着长老近前看时那里是甚香米饭却是一罐子拖尾巴的长蛆也不是面筋却是几个青蛙、癞虾蟆满地乱跳。长老才有三分儿信了怎禁猪八戒气不忿在旁漏八分儿唆嘴道:“师父说起这个女子他是此间农妇因为送饭下田路遇我等却怎么栽他是个妖怪?哥哥的棍重走将来试手打他一下不期就打杀了;怕你念甚么《紧箍儿咒》故意的使个障眼法儿变做这等样东西演幌你眼使不念咒哩。”

三藏自此一言就是晦气到了:果然信那呆子撺唆手中捻诀口里念咒行者就叫:“头疼!头疼!莫念!莫念!有话便说。”唐僧道:“有甚话说!出家人时时常要方便念念不离善心扫地恐伤蝼蚁命爱惜飞蛾纱罩灯。你怎么步步行凶打死这个无故平人取将经来何用?你回去罢!”行者道:“师父你教我回那里去?”唐僧道:“我不要你做徒弟。”行者道:“你不要我做徒弟只怕你西天路去不成。”唐僧道:“我命在天该那个妖精蒸了吃就是煮了也算不过。终不然你救得我的大限?

你快回去!”行者道:“师父我回去便也罢了只是不曾报得你的恩哩。”唐僧道:“我与你有甚恩?”那大圣闻言连忙跪下叩头道:“老孙因大闹天宫致下了伤身之难被我佛压在两界山幸观音菩萨与我受了戒行幸师父救脱吾身若不与你同上西天显得我知恩不报非君子万古千秋作骂名。”原来这唐僧是个慈悯的圣僧他见行者哀告却也回心转意道:“既如此说且饶你这一次再休无礼。如若仍前作恶这咒语颠倒就念二十遍!”行者道:“三十遍也由你只是我不打人了。”却才伏侍唐僧上马又将摘来桃子奉上。唐僧在马上也吃了几个权且充饥。

却说那妖精脱命升空。原来行者那一棒不曾打杀妖精妖精出神去了。他在那云端里咬牙切齿暗恨行者道:“几年只闻得讲他手段今日果然话不虚传。那唐僧已此不认得我将要吃饭。若低头闻一闻儿我就一把捞住却不是我的人了?

不期被他走来弄破我这勾当又几乎被他打了一棒。若饶了这个和尚诚然是劳而无功也我还下去戏他一戏。”

好妖精按落阴云在那前山坡下摇身一变变作个老妇人年满八旬手拄着一根弯头竹杖一步一声的哭着走来。八戒见了大惊道:“师父!不好了!那妈妈儿来寻人了!”唐僧道:

“寻甚人?”八戒道:“师兄打杀的定是他女儿。这个定是他娘寻将来了。”行者道:“兄弟莫要胡说!那女子十八岁这老妇有八十岁怎么六十多岁还生产?断乎是个假的等老孙去看来。”好行者拽开步走近前观看那怪物:假变一婆婆两鬓如冰雪。走路慢腾腾行步虚怯怯。弱体瘦伶仃脸如枯菜叶。

颧骨望上翘嘴唇往下别。老年不比少年时满脸都是荷叶摺。

行者认得他是妖精更不理论举棒照头便打。那怪见棍子起时依然抖擞又出化了元神脱真儿去了把个假尸又打死在山路之下。唐僧一见惊下马来睡在路旁更无二话只是把《紧箍儿咒》颠倒足足念了二十遍。可怜把个行者头勒得似个亚腰儿葫芦十分疼痛难忍滚将来哀告道:“师父莫念了!

有甚话说了罢!”唐僧道:“有甚话说!出家人耳听善言不堕地狱。我这般劝化你你怎么只是行凶?把平人打死一个又打死一个此是何说?”行者道:“他是妖精。”唐僧道:“这个猴子胡说!就有这许多妖怪!你是个无心向善之辈有意作恶之人你去罢!”行者道:“师父又教我去回去便也回去了只是一件不相应。”唐僧道:“你有甚么不相应处?”八戒道:“师父他要和你分行李哩。跟着你做了这几年和尚不成空着手回去?你把那包袱里的甚么旧褊衫破帽子分两件与他罢。”行者闻言气得暴跳道:“我把你这个尖嘴的夯货!老孙一向秉教沙门更无一毫嫉妒之意贪恋之心怎么要分甚么行李?”唐僧道:“你既不嫉妒贪恋如何不去?”行者道:“实不瞒师父说老孙五百年前居花果山水帘洞大展英雄之际收降七十二洞邪魔手下有四万七千群怪头戴的是紫金冠身穿的是赭黄袍腰系的是蓝田带足踏的是步云履手执的是如意金箍棒着实也曾为人。自从涅槃罪度削秉正沙门跟你做了徒弟把这个金箍儿勒在我头上若回去却也难见故乡人。师父果若不要我把那个《松箍儿咒》念一念退下这个箍子交付与你套在别人头上我就快活相应了也是跟你一场。莫不成这些人意儿也没有了?”唐僧大惊道:“悟空我当时只是菩萨暗受一卷《紧箍儿咒》却没有甚么松箍儿咒。”行者道:“若无《松箍儿咒》你还带我去走走罢。”长老又没奈何道:“你且起来我再饶你这一次却不可再行凶了。”行者道:“再不敢了再不敢了。”又伏侍师父上马剖路前进。

却说那妖精原来行者第二棍也不曾打杀他。那怪物在半空中夸奖不尽道:“好个猴王着然有眼!我那般变了去他也还认得我。这些和尚他去得快若过此山西下四十里就不伏我所管了。若是被别处妖魔捞了去好道就笑破他人口使碎自家心我还下去戏他一戏。”好妖怪按耸阴风在山坡下摇身一变变成一个老公公真个是:白如彭祖苍髯赛寿星耳中鸣玉磬眼里幌金星。手拄龙头拐身穿鹤氅轻。数珠掐在手口诵南无经。唐僧在马上见了心中欢喜道:“阿弥陀佛!西方真是福地!那公公路也走不上来逼法的还念经哩。”

八戒道:“师父你且莫要夸奖那个是祸的根哩。”唐僧道:“怎么是祸根?”八戒道:“行者打杀他的女儿又打杀他的婆子这个正是他的老儿寻将来了。我们若撞在他的怀里呵师父你便偿命该个死罪;把老猪为从问个充军;沙僧喝令问个摆站;那行者使个遁法走了却不苦了我们三个顶缸?”行者听见道:“这个呆根这等胡说可不唬了师父?等老孙再去看看。”

他把棍藏在身边走上前迎着怪物叫声:“老官儿往那里去?

怎么又走路又念经?”那妖精错认了定盘星把孙大圣也当做个等闲的遂答道:“长老啊我老汉祖居此地一生好善斋僧看经念佛。命里无儿止生得一个小女招了个女婿今早送饭下田想是遭逢虎口。老妻先来找寻也不见回去全然不知下落老汉特来寻看。果然是伤残他命也没奈何将他骸骨收拾回去安葬茔中。”行者笑道:“我是个做吓虎的祖宗你怎么袖子里笼了个鬼儿来哄我?你瞒了诸人瞒不过我!我认得你是个妖精!”那妖精唬得顿口无言。行者掣出棒来自忖思道:“若要不打他显得他倒弄个风儿;若要打他又怕师父念那话儿咒语。”又思量道:“不打杀他他一时间抄空儿把师父捞了去却不又费心劳力去救他?还打的是!就一棍子打杀他师父念起那咒常言道虎毒不吃儿。凭着我巧言花语嘴伶舌便哄他一哄好道也罢了。”好大圣念动咒语叫当坊土地、本处山神道:“这妖精三番来戏弄我师父这一番却要打杀他。你与我在半空中作证不许走了。”众神听令谁敢不从?都在云端里照应。那大圣棍起处打倒妖魔才断绝了灵光。

那唐僧在马上又唬得战战兢兢口不能言。八戒在旁边又笑道:“好行者!风了!只行了半日路倒打死三个人!”唐僧正要念咒行者急到马前叫道:“师父莫念!莫念!你且来看看他的模样。”却是一堆粉骷髅在那里。唐僧大惊道:“悟空这个人才死了怎么就化作一堆骷髅?”行者道:“他是个潜灵作怪的僵尸在此迷人败本被我打杀他就现了本相。他那脊梁上有一行字叫做白骨夫人。”唐僧闻说倒也信了怎禁那八戒旁边唆嘴道:“师父他的手重棍凶把人打死只怕你念那话儿故意变化这个模样掩你的眼目哩!”唐僧果然耳软又信了他随复念起。行者禁不得疼痛跪于路旁只叫:“莫念!莫念!有话快说了罢!”唐僧道:“猴头!还有甚说话!出家人行善如春园之草不见其长日有所增;行恶之人如磨刀之石不见其损日有所亏。你在这荒郊野外一连打死三人还是无人检举没有对头;倘到城市之中人烟凑集之所你拿了那哭丧棒一时不知好歹乱打起人来撞出大祸教我怎的脱身?你回去罢!”行者道:“师父错怪了我也。这厮分明是个妖魔他实有心害你。我倒打死他替你除了害你却不认得反信了那呆子谗言冷语屡次逐我。常言道事不过三。我若不去真是个下流无耻之徒。我去我去!去便去了只是你手下无人。”唐僧怒道:“这泼猴越无礼!看起来只你是人那悟能、悟净就不是人?”那大圣一闻得说他两个是人止不住伤情凄惨对唐僧道声:“苦啊!你那时节出了长安有刘伯钦送你上路;到两界山救我出来投拜你为师我曾穿古洞入深林擒魔捉怪收八戒得沙僧吃尽千辛万苦。今日昧着惺惺使糊涂只教我回去:这才是鸟尽弓藏兔死狗烹!罢罢罢!但只是多了那《紧箍儿咒》。”唐僧道:“我再不念了。”行者道:“这个难说。若到那毒魔苦难处不得脱身八戒沙僧救不得你那时节想起我来忍不住又念诵起来就是十万里路我的头也是疼的;假如再来见你不如不作此意。”唐僧见他言言语语越添恼怒滚鞍下马来叫沙僧包袱内取出纸笔即于涧下取水石上磨墨写了一纸贬书递于行者道:“猴头!执此为照再不要你做徒弟了!如再与你相见我就堕了阿鼻地狱!”

行者连忙接了贬书道:“师父不消誓老孙去罢。”他将书摺了留在袖中却又软款唐僧道:“师父我也是跟你一场又蒙菩萨指教今日半途而废不曾成得功果你请坐受我一拜我也去得放心。”唐僧转回身不睬口里唧唧哝哝的道:“我是个好和尚不受你歹人的礼!”大圣见他不睬又使个身外法把脑后毫毛拔了三根吹口仙气叫“变!”即变了三个行者连本身四个四面围住师父下拜。那长老左右躲不脱好道也受了一拜。

大圣跳起来把身一抖收上毫毛却又吩咐沙僧道:“贤弟你是个好人却只要留心防着八戒言语途中更要仔细。倘一时有妖精拿住师父你就说老孙是他大徒弟。西方毛怪闻我的手段不敢伤我师父。”唐僧道:“我是个好和尚不题你这歹人的名字你回去罢。”那大圣见长老三番两复不肯转意回心没奈何才去。你看他:噙泪叩头辞长老含悲留意嘱沙僧。

一头拭迸坡前草两脚蹬翻地上藤。上天下地如轮转跨海飞山第一能。顷刻之间不见影霎时疾返旧途程。你看他忍气别了师父纵筋斗云径回花果山水帘洞去了。独自个凄凄惨惨忽闻得水声聒耳大圣在那半空里看时原来是东洋大海潮的声响。一见了又想起唐僧止不住腮边泪坠停云住步良久方去。毕竟不知此去反复何如且听下回分解。
------------

第二十八回 花果山群妖聚义 黑松林三藏逢魔

却说那大圣虽被唐僧逐赶然犹思念感叹不已早望见东洋大海道:“我不走此路者已五百年矣!”只见那海水:烟波荡荡巨浪悠悠。烟波荡荡接天河巨浪悠悠通地脉。潮来汹涌水浸湾环。潮来汹涌犹如霹雳吼三春;水浸湾环却似狂风吹九夏。乘龙福老往来必定皱眉行;跨鹤仙童反复果然忧虑过。近岸无村社傍水少渔舟。浪卷千年雪风生六月秋。

野禽凭出没沙鸟任沉浮眼前无钓客耳畔只闻鸥。海底游鱼乐天边过雁愁。那行者将身一纵跳过了东洋大海早至花果山。按落云头睁睛观看那山上花草俱无烟霞尽绝;峰岩倒塌林树焦枯。你道怎么这等?只因他闹了天宫拿上界去此山被显圣二郎神率领那梅山七弟兄放火烧坏了。这大圣倍加凄惨有一篇败山颓景的古风为证古风云:回顾仙山两泪垂对山凄惨更伤悲。当时只道山无损今日方知地有亏。可恨二郎将我灭堪嗔小圣把人欺。行凶掘你先灵墓无干破尔祖坟基。满天霞雾皆消荡遍地风云尽散稀。东岭不闻斑虎啸西山那见白猿啼?北溪狐兔无踪迹南谷獐豝没影遗。青石烧成千块土碧砂化作一堆泥。洞外乔松皆倚倒崖前翠柏尽稀少。椿杉槐桧栗檀焦桃杏李梅梨枣了。柘绝桑无怎养蚕?柳稀竹少难栖鸟。峰头巧石化为尘涧底泉干都是草。崖前土黑没芝兰路畔泥红藤薜攀。往日飞禽飞那处?当时走兽走何山?

豹嫌蟒恶倾颓所鹤避蛇回败坏间。想是日前行恶念致令目下受艰难。

那大圣正当悲切只听得那芳草坡前、曼荆凹里响一声跳出七八个小猴一拥上前围住叩头高叫道:“大圣爷爷!今日来家了?”美猴王道:“你们因何不耍不顽一个个都潜踪隐迹?我来多时了不见你们形影何也?”群猴听说一个个垂泪告道:“自大圣擒拿上界我们被猎人之苦着实难捱!怎禁他硬弩强弓黄鹰劣犬网扣枪钩故此各惜性命不敢出头顽耍只是深潜洞府远避窝巢饥去坡前偷草食渴来涧下吸清泉。却才听得大圣爷爷声音特来接见伏望扶持。”那大圣闻得此言愈加凄惨便问:“你们还有多少在此山上?”群猴道:

“老者小者只有千把。”大圣道:“我当时共有四万七千群妖如今都往那里去了?”群猴道:“自从爷爷去后这山被二郎菩萨点上火烧杀了大半。我们蹲在井里钻在涧内藏于铁板桥下得了性命。及至火灭烟消出来时又没花果养赡难以存活别处又去了一半。我们这一半捱苦的住在山中这两年又被些打猎的抢了一半去也。”行者道:“他抢你去何干?”群猴道:“说起这猎户可恨!他把我们中箭着枪的中毒打死的拿了去剥皮剔骨酱煮醋蒸油煎盐炒当做下饭食用。或有那遭网的遇扣的夹活儿拿去了教他跳圈做戏翻筋斗竖蜻蜓当街上筛锣擂鼓无所不为的顽耍。”大圣闻此言更十分恼怒道“洞中有甚么人执事?”群妖道:“还有马流二元帅奔芭二将军管着哩。”大圣道:“你们去报他知道说我来了。”那些小妖撞入门里报道:“大圣爷爷来家了。”那马流奔芭闻报忙出门叩头迎接进洞。大圣坐在中间群怪罗拜于前启道:“大圣爷爷近闻得你得了性命保唐僧往西天取经如何不走西方却回本山?”大圣道:“小的们你不知道那唐三藏不识贤愚。我为他一路上捉怪擒魔使尽了平生的手段几番家打杀妖精他说我行凶作恶不要我做徒弟把我逐赶回来写立贬书为照永不听用了。”众猴鼓掌大笑道:“造化!造化!做甚么和尚且家来带携我们耍子几年罢!”叫:“快安排椰子酒来与爷爷接风。”大圣道:“且莫饮酒我问你那打猎的人几时来我山上一度?”马流道:“大圣不论甚么时度他逐日家在这里缠扰。”

大圣道:“他怎么今日不来?”马流道:“看待来耶。”大圣吩咐:

“小的们都出去把那山上烧酥了的碎石头与我搬将起来堆着。或二三十个一推或五六十个一堆堆着我有用处。”那些小猴都是一窝峰一个个跳天搠地乱搬了许多堆集。大圣看了教:“小的们都往洞内藏躲让老孙作法。”

那大圣上了山巅看处只见那南半边冬冬鼓响噹噹锣鸣闪上有千余人马都架着鹰犬持着刀枪。猴王仔细看那些人来得凶险。好男子真个骁勇!但见:狐皮苫肩顶锦绮裹腰胸。袋插狼牙箭胯挂宝雕弓。人似搜山虎马如跳涧龙。成群引着犬满膀架其鹰。荆筐抬火炮带定海东青。粘竿百十担兔叉有千根。牛头拦路网阎王扣子绳一齐乱吆喝散撒满天星。大圣见那些人布上他的山来心中大怒手里捻诀口内念念有词往那巽地上吸了一口气呼的吹将去便是一阵狂风。好风!但见:扬尘播土倒树摧林。海浪如山耸浑波万迭侵。乾坤昏荡荡日月暗沉沉。一阵摇松如虎啸忽然入竹似龙吟。万窍怒号天噫气飞砂走石乱伤人。大圣作起这大风将那碎石乘风乱飞乱舞可怜把那些千余人马一个个石打乌头粉碎沙飞海马俱伤。人参官桂岭前忙血染朱砂地上。附子难归故里槟榔怎得还乡?尸骸轻粉卧山场红娘子家中盼望。有诗为证:人亡马死怎归家?野鬼孤魂乱似麻。可怜抖擞英雄将不辨贤愚血染沙。

大圣按落云头鼓掌大笑道:“造化!造化!自从归顺唐僧做了和尚他每每劝我话道:千日行善善犹不足;一日行恶恶自有余。真有此话!我跟着他打杀几个妖精他就怪我行凶今日来家却结果了这许多猎户。”叫:“小的们出来!”那群猴狂风过去听得大圣呼唤一个个跳将出来。大圣道:“你们去南山下把那打死的猎户衣服剥得来家洗净血迹穿了遮寒;把死人的尸都推在那万丈深潭里;把死倒的马拖将来剥了皮做靴穿将肉腌着慢慢的食用;把那些弓箭枪刀与你们操演武艺;将那杂色旗号收来我用。”群猴一个个领诺。

那大圣把旗拆洗总斗做一面杂彩花旗上写着“重修花果山复整水帘洞齐天大圣”十四字竖起杆子将旗挂于洞外逐日招魔聚兽积草屯粮不题和尚二字。他的人情又大手段又高便去四海龙王借些甘霖仙水把山洗青了。前栽榆柳后种松楠桃李枣梅无所不备逍遥自在乐业安居不题。

却说唐僧听信狡性纵放心猿攀鞍上马八戒前边开路沙僧挑着行李西行。过了白虎岭忽见一带林丘真个是藤攀葛绕柏翠松青。三藏叫道:“徒弟呀山路崎岖甚是难走却又松林丛簇树木森罗切须仔细恐有妖邪妖兽。”你看那呆子抖擞精神叫沙僧带着马他使钉钯开路领唐僧径入松林之内。正行处那长老兜住马道:“八戒我这一日其实饥了那里寻些斋饭我吃?”八戒道:“师父请下马在此等老猎去寻。”

长老下了马沙僧歇了担取出钵盂递与八戒。八戒道:“我去也。”长老问:“那里去?”八戒道:“莫管我这一去钻冰取火寻斋至压雪求油化饭来。”你看他出了松林往西行经十余里更不曾撞着一个人家真是有狼虎无人烟的去处。那呆子走得辛苦心内沉吟道:“当年行者在日老和尚要的就有今日轮到我的身上诚所谓当家才知柴米价养子方晓父娘恩公道没去化处。”却又走得瞌睡上来思道:“我若就回去对老和尚说没处化斋他也不信我走了这许多路。须是再多幌个时辰才好去回话。也罢也罢且往这草科里睡睡。”呆子就把头拱在草里睡下当时也只说朦胧朦胧就起来岂知走路辛苦的人丢倒头只管齁齁睡起。

且不言八戒在此睡觉却说长老在那林间耳热眼跳身心不安急回叫沙僧道:“悟能去化斋怎么这早晚还不回?”沙僧道:“师父你还不晓得哩他见这西方上人家斋僧的多他肚子又大他管你?只等他吃饱了才来哩。”三藏道:“正是呀倘或他在那里贪着吃斋我们那里会他?天色晚了此间不是个住处须要寻个下处方好哩。”沙僧道:“不打紧师父你且坐在这里等我去寻他来。”三藏道:“正是正是。有斋没斋罢了只是寻下处要紧。”沙僧绰了宝杖径出松林来找八戒。

长老独坐林中十分闷倦只得强打精神跳将起来把行李攒在一处将马拴在树上取下戴的斗笠插定了锡杖整一整缁衣徐步幽林权为散闷。那长老看遍了野草山花听不得归巢鸟噪。原来那林子内都是些草深路小的去处只因他情思紊乱却走错了。他一来也是要散散闷。二来也是要寻八戒沙僧。不期他两个走的是直西路长老转了一会却走向南边去了。出得松林忽抬头见那壁厢金光闪烁彩气腾腾仔细看处原来是一座宝塔金顶放光。这是那西落的日色映着那金顶放亮。他道:“我弟子却没缘法哩!自离东土愿逢庙烧香见佛拜佛遇塔扫塔。那放光的不是一座黄金宝塔?怎么就不曾走那条路?塔下必有寺院院内必有僧家且等我走走。这行李、白马料此处无人行走却也无事。那里若有方便处待徒弟们来一同借歇。”噫!长老一时晦气到了。你看他拽开步竟至塔边但见那:石崖高万丈山大接青霄。根连地厚峰插天高。两边杂树数千颗前后藤缠百余里。花映草梢风有影水流云窦月无根。倒木横担深涧枯藤结挂光峰。石桥下流滚滚清泉;台座上长明明白粉。远观一似三岛天堂近看有如蓬莱胜境。香松紫竹绕山溪鸦鹊猿猴穿峻岭。洞门外有一来一往的走兽成行;树林里有或出或入的飞禽作队。青青香草秀艳艳野花开。这所在分明是恶境那长老晦气撞将来。那长老举步进前才来到塔门之下只见一个斑竹帘儿挂在里面。他破步入门揭起来往里就进猛抬头见那石床上侧睡着一个妖魔。你道他怎生模样:青靛脸白獠牙一张大口呀呀。两边乱蓬蓬的鬓毛却都是些胭脂染色;三四紫巍巍的髭髯恍疑是那荔枝排芽。鹦嘴般的鼻儿拱拱曙星样的眼儿巴巴。两个拳头和尚钵盂模样;一双蓝脚悬崖榾柮枒槎。斜披着淡黄袍帐赛过那织锦袈裟。拿的一口刀精光耀映;眠的一块石细润无瑕。他也曾小妖排蚁阵他也曾老怪坐蜂衙你看他威风凛凛大家吆喝叫一声爷。他也曾月作三人壶酌酒他也曾风生两腋盏倾茶你看他神通浩浩霎着下眼游遍天涯。

荒林喧鸟雀深莽宿龙蛇。仙子种田生白玉道人伏火养丹砂。

小小洞门虽到不得那阿鼻地狱;楞楞妖怪却就是一个牛头夜叉。

那长老看见他这般模样唬得打了一个倒退遍体酥麻两腿酸软即忙的抽身便走。刚刚转了一个身那妖魔他的灵性着实是强大撑开着一双金睛鬼眼叫声:“小的们你看门外是甚么人!”一个小妖就伸头望门外一看看见是个光头的长老连忙跑将进去报道:“大王外面是个和尚哩团头大面两耳垂肩嫩刮刮的一身肉细娇娇的一张皮:且是好个和尚!”那妖闻言呵声笑道:“这叫做个蛇头上苍蝇自来的衣食。你众小的们疾忙赶上去与我拿将来我这里重重有赏!”

那些小妖就是一窝蜂齐齐拥上。三藏见了虽则是一心忙似箭两脚走如飞终是心惊胆颤腿软脚麻况且是山路崎岖林深日暮步儿那里移得动?被那些小妖平抬将去正是:龙游浅水遭虾戏虎落平原被犬欺。纵然好事多磨障谁象唐僧西向时?

你看那众小妖抬得长老放在那竹帘儿外欢欢喜喜报声道:“大王拿得和尚进来了。”那老妖他也偷眼瞧一瞧只见三藏头直上貌堂堂果然好一个和尚他便心中想道:“这等好和尚必是上方人物不当小可的若不做个威风他怎肯服降哩?”陡然间就狐假虎威红须倒竖血朝天眼睛迸裂大喝一声道:“带那和尚进来!”众妖们大家响响的答应了一声“是!”就把三藏望里面只是一推。这是既在矮檐下怎敢不低头!三藏只得双手合着与他见个礼那妖道:“你是那里和尚?从那里来?到那里去?”快快说明!”三藏道:“我本是唐朝僧人奉大唐皇帝敕命前往西方访求经偈经过贵山特来塔下谒圣不期惊动威严望乞恕罪。待往西方取得经回东土永注高名也。”那妖闻言呵呵大笑道:“我说是上邦人物果然是你。正要吃你哩却来的甚好!甚好!不然却不错放过了?

你该是我口里的食自然要撞将来就放也放不去就走也走不脱!”叫小妖:“把那和尚拿去绑了!”果然那些小妖一拥上前把个长老绳缠索绑缚在那定魂桩上。老妖持刀又问道:

“和尚你一行有几个?终不然一人敢上西天?”三藏见他持刀又老实说道:“大王我有两个徒弟叫做猪八戒、沙和尚都出松林化斋去了。还有一担行李一匹白马都在松林里放着哩。”老妖道:“又造化了!两个徒弟连你三个连马四个彀吃一顿了!”小妖道:“我们去捉他来。”老妖道:“不要出去把前门关了。他两个化斋来一定寻师父吃寻不着一定寻着我门上。常言道上门的买卖好做且等慢慢的捉他。”众小妖把前门闭了。

且不言三藏逢灾。却说那沙僧出林找八戒直有十余里远近不曾见个庄村。他却站在高埠上正然观看只听得草中有人言语急使杖拨开深草看时原来是呆子在里面说梦话哩。

被沙僧揪着耳朵方叫醒了道:“好呆子啊!师父教你化斋许你在此睡觉的?”那呆子冒冒失失的醒来道:“兄弟有甚时候了?”沙僧道:“快起来!师父说有斋没斋也罢教你我那里寻下住处去哩。”呆子懵懵懂懂的托着钵盂拑着钉钯与沙僧径直回来到林中看时不见了师父。沙僧埋怨道:“都是你这呆子化斋不来必有妖精拿师父也。”八戒笑道:“兄弟莫要胡说。那林子里是个清雅的去处决然没有妖精。想是老和尚坐不住往那里观风去了。我们寻他去来。”二人只得牵马挑担收拾了斗篷锡杖出松林寻找师父。

这一回也是唐僧不该死。他两个寻一会不见忽见那正南下有金光闪灼八戒道:“兄弟啊有福的只是有福。你看师父往他家去了那放光的是座宝塔谁敢怠慢?一定要安排斋饭留他在那里受用。我们还不走动些也赶上去吃些斋儿。”

沙僧道:“哥啊定不得吉凶哩。我们且去看来。”二人雄纠纠的到了门前呀!闭着门哩。只见那门上横安了一块白玉石板上镌着六个大字:“碗子山波月洞”。沙僧道:“哥啊这不是甚么寺院是一座妖精洞府也。我师父在这里也见不得哩。”八戒道:“兄弟莫怕你且拴下马匹守着行李待我问他的信看。”那呆子举着钯上前高叫:“开门!开门!”那洞内有把门的小妖开了门忽见他两个的模样急抽身跑入里面报道:“大王!买卖来了!”老妖道:“那里买卖?”小妖道:“洞门外有一个长嘴大耳的和尚与一个晦气色的和尚来叫门了!”老妖大喜道:“是猪八戒与沙僧寻将来也!噫他也会寻哩!怎么就寻到我这门上?既然嘴脸凶顽却莫要怠慢了他。”叫:“取披挂来!”

小妖抬来就结束了绰刀在手径出门来。

却说那八戒、沙僧在门前正等只见妖魔来得凶险。你道他怎生打扮:青脸红须赤飘黄金铠甲亮光饶。裹肚衬腰磲石带攀胸勒甲步云绦。闲立山前风吼吼闷游海外浪滔滔。一双蓝靛焦筋手执定追魂取命刀。要知此物名和姓声扬二字唤黄袍。那黄袍老怪出得门来便问:“你是那方和尚在我门吆喝?”八戒道:“我儿子你不认得?我是你老爷!我是大唐差往西天去的!我师父是那御弟三藏。若在你家里趁早送出来省了我钉钯筑进去!”那怪笑道:“是是是有一个唐僧在我家。我也不曾怠慢他安排些人肉包儿与他吃哩。你们也进去吃一个儿何如?”这呆子认真就要进去沙僧一把扯住道:

“哥啊他哄你哩你几时又吃人肉哩?”呆子却才省悟掣钉钯望妖怪劈脸就筑。那怪物侧身躲过使钢刀急架相迎。两个都显神通纵云头跳在空中厮杀。沙僧撇了行李白马举宝杖急急帮攻。此时两个狠和尚一个泼妖魔在云端里这一场好杀正是那:杖起刀迎钯来刀架。一员魔将施威两个神僧显化。九齿钯真个英雄降妖伐诚然凶咤。没前后左右齐来那黄袍公然不怕。你看他蘸钢刀晃亮如银其实的那神通也为广大。只杀得满空中雾绕云迷、半山里崖崩岭咋。一个为声名怎肯干休?一个为师父断然不怕。他三个在半空中往往来来战经数十回合不分胜负。各因性命要紧其实难解难分。

毕竟不知怎救唐僧且听下回分解。
------------

第二十九回 脱难江流来国土 承恩八戒转山林

诗曰:妄想不复强灭真如何必希求?本原自性佛前修迷悟岂居前后?悟即刹那成正迷而万劫沉流。若能一念合真修灭尽恒沙罪垢。却说那八戒、沙僧与怪斗经个三十回合不分胜负。你道怎么不分胜负?若论赌手段莫说两个和尚就是二十个也敌不过那妖精。只为唐僧命不该死暗中有那护法神祇保着他空中又有那六丁六甲、五方揭谛、四值功曹、一十八位护教伽蓝助着八戒沙僧。

且不言他三人战斗却说那长老在洞里悲啼思量他那徒弟眼中流泪道:“悟能啊不知你在那个村中逢了善友贪着斋供!悟净啊你又不知在那里寻他可能得会?岂知我遇妖魔在此受难!几时得会你们脱了大难早赴灵山!”正当悲啼烦恼忽见那洞里走出一个妇人来扶着定魂桩叫道:“那长老你从何来?为何被他缚在此处?”长老闻言泪眼偷看那妇人约有三十年纪遂道:“女菩萨不消问了我已是该死的走进你家门来也。要吃就吃了罢又问怎的?”那妇人道:“我不是吃人的。我家离此西下有三百余里。那里有座城叫做宝象国。我是那国王的第三个公主乳名叫做百花羞。只因十三年前八月十五日夜玩月中间被这妖魔一阵狂风摄将来与他做了十三年夫妻。在此生儿育女杳无音信回朝思量我那父母不能相见。你从何来被他拿住?”唐僧道:“贫僧乃是差往西天取经者不期闲步误撞在此。如今要拿住我两个徒弟一齐蒸吃理。”那公主陪笑道:“长老宽心你既是取经的我救得你。那宝象国是你西方去的大路你与我捎一封书儿去拜上我那父母我就教他饶了你罢。”三藏点头道:“女菩萨若还救得贫僧命愿做捎书寄信人。”那公主急转后面即修了一纸家书封固停当到桩前解放了唐僧将书付与。唐僧得解脱捧书在手道:“女菩萨多谢你活命之恩。贫僧这一去过贵处定送国王处。只恐日久年深你父母不肯相认奈何?切莫怪我贫僧打了诳语。”公主道:“不妨我父王无子止生我三个姊妹若见此书必有相看之意。三藏紧紧袖了家书谢了公主就往外走被公主扯住道:“前门里你出不去!那些大小妖精都在门外摇旗呐喊擂鼓筛锣助着大王与你徒弟厮杀哩。你往后门里去罢若是大王拿住还审问审问;只恐小妖儿捉了不分好歹挟生儿伤了你的性命。等我去他面前说个方便。若是大王放了你啊待你徒弟讨个示下寻着你一同好走。”三藏闻言磕了头谨依吩咐辞别公主躲离后门之外不敢自行将身藏在荆棘丛中。

却说公主娘娘心生巧计急往前来出门外分开了大小群妖只听得叮叮噹兵刃乱响原来是八戒沙僧与那怪在半空里厮杀哩。这公主厉声高叫道:“黄袍郎!”那妖王听得公主叫唤即丢了八戒沙僧按落云头揪了钢刀搀着公主道:“浑家有甚话说?”公主道:“郎君啊我才时睡在罗帏之内梦魂中忽见个金甲神人。”妖魔道:“那个金甲神?上我门怎的?”公主道:“是我幼时在宫里对神暗许下一桩心愿:若得招个贤郎驸马上名山拜仙府斋僧布施。自从配了你夫妻们欢会到今不曾题起。那金甲神人来讨誓愿喝我醒来却是南柯一梦。

因此急整容来郎君处诉知不期那桩上绑着一个僧人万望郎君慈悯看我薄意饶了那个和尚罢只当与我斋僧还愿不知郎君肯否?”那怪道:“浑家你却多心呐!甚么打紧之事。我要吃人那里不捞几个吃吃?这个把和尚到得那里放他去罢。”公主道:“郎君放他从后门里去罢。”妖魔道:“奈烦哩放他去便罢又管他甚么后门前门哩。”他遂绰了钢刀高叫道:

“那猪八戒你过来。我不是怕你不与你战看着我浑家的分上饶了你师父也。趁早去后门寻着他往西方去罢。若再来犯我境界断乎不饶!”

那八戒与沙僧闻得此言就如鬼门关上放回来的一般即忙牵马挑担鼠窜而行转过那波月洞后门之外叫声“师父!”

那长老认得声音就在那荆棘中答应。沙僧就剖开草径搀着师父慌忙的上马。这里狠毒险遭青面鬼殷勤幸有百花羞。鳌鱼脱却金钩钓摆尾摇头游。

八戒当头领路沙僧后随出了那松林上了大路。你看他两个哜哜嘈嘈埋埋怨怨三藏只是解和。遇晚先投宿鸡鸣早看天一程一程长亭短亭不觉的就走了二百九十九里。猛抬头只见一座好城就是宝象国。真好个处所也:云渺渺路迢迢。地虽千里外景物一般饶。瑞霭祥烟笼罩清风明月招摇。

嵂嵂崒崒的远山大开图画;潺潺湲湲的流水碎溅琼瑶。可耕的连阡带陌足食的密蕙新苗。渔钓的几家三涧曲樵采的一担两峰椒。廓的廓城的城金汤巩固;家的家户的户只斗逍遥。九重的高阁如殿宇万丈的层台似锦标。也有那太极殿、华盖殿、烧香殿、观文殿、宣政殿、延英殿一殿殿的玉陛金阶摆列着文冠武弁;也有那大明宫、昭阳宫、长乐宫、华清宫、建章宫、未央宫一宫宫的钟鼓管籥撒抹了闺怨春愁。也有禁苑的露花匀嫩脸;也有御沟的风柳舞纤腰。通衢上也有个顶冠束带的盛仪容乘五马;幽僻中也有个持弓挟矢的拨云雾贯双雕。花柳的巷管弦的楼春风不让洛阳桥。取经的长老回大唐肝胆裂;伴师的徒弟息肩小驿梦魂消。看不尽宝象国的景致。师徒三众收拾行李、马匹安歇馆驿中。

唐僧步行至朝门外对阁门大使道:“有唐朝僧人特来面驾倒换文牒乞为转奏转奏。”那黄门奏事官连忙走至白玉阶前奏道:“万岁唐朝有个高僧欲求见驾倒换文牒。”那国王闻知是唐朝大国且又说是个方上圣僧心中甚喜即时准奏叫:“宣他进来。”把三藏宣至金阶舞蹈山呼礼毕。两边文武多官无不叹道:“上邦人物礼乐雍容如此!”那国王道:“长老你到我国中何事?”三藏道:“小僧是唐朝释子承我天子敕旨前往西方取经。原领有文牒到陛下上国理合倒换。故此不识进退惊动龙颜。”国王道:“既有唐天子文牒取上来看。”

三藏双手捧上去展开放在御案上。牒云:“南赡部洲大唐国奉天承运唐天子牒行:切惟朕以凉德嗣续丕基事神治民临深履薄朝夕是惴。前者失救泾河老龙获谴于我皇皇后帝三魂七魄倏忽阴司已作无常之客。因有阳寿未绝感冥君放送回生广陈善会修建度亡道场。感蒙救苦观世音菩萨金身出现指示西方有佛有经可度幽亡脱孤魂。特着法师玄奘远历千山询求经偈。倘到西邦诸国不灭善缘照牒放行。须至牒者。大唐贞观一十三年秋吉日御前文牒。”(上有宝印九颗)国王见了取本国玉宝用了花押递与三藏。

三藏谢了恩收了文牒又奏道:“贫僧一来倒换文牒二来与陛下寄有家书。”国王大喜道:“有甚书?”三藏道:“陛下第三位公主娘娘被碗子山波月洞黄袍妖摄将去贫僧偶尔相遇故寄书来也。”国王闻言满眼垂泪道:“自十三年前不见了公主两班文武官也不知贬退了多少宫内宫外大小婢子太监也不知打死了多少只说是走出皇宫迷失路径无处找寻满城中百姓人家也盘诘了无数更无下落。怎知道是妖怪摄了去!今日乍听得这句话故此伤情流泪。”三藏袖中取出书来献上。国王接了见有平安二字一手软拆不开书传旨宣翰林院大学士上殿读书。学士随即上殿殿前有文武多官殿后有后妃宫女俱侧耳听书。学士拆开朗诵上写着:“不孝女百花羞顿百拜大德父王万岁龙凤殿前暨三宫母后昭阳宫下及举朝文武贤卿台次:拙女幸托坤宫感激劬劳万种不能竭力怡颜尽心奉孝。乃于十三年前八月十五日良夜佳辰蒙父王恩旨着各宫排宴赏玩月华共乐清霄盛会。正欢娱之间不觉一阵香风闪出个金睛蓝面青魔王将女擒住驾祥光直带至半野山中无人处难分难辨被妖倚强霸占为妻。

是以无奈捱了一十三年产下两个妖儿尽是妖魔之种。论此真是败坏人伦有伤风化不当传书玷辱;但恐女死之后不显分明。正含怨思忆父母不期唐朝圣僧亦被魔王擒住。是女滴泪修书大胆放脱特托寄此片楮以表寸心。伏望父王垂悯遣上将早至碗子山波月洞捉获黄袍怪救女回朝深为恩念。草草欠恭面听不一。逆女百花羞再顿顿。’那学士读罢家书国王大哭三宫滴泪文武伤情前前后后无不哀念。

国王哭之许久便问两班文武:“那个敢兴兵领将与寡人捉获妖魔救我百花公主?”连问数声更无一人敢答真是木雕成的武将泥塑就的文官。那国王心生烦恼泪若涌泉。只见那多官齐俯伏奏道:“陛下且休烦恼公主已失至今一十三载无音。偶遇唐朝圣僧寄书来此未知的否。况臣等俱是凡人凡马习学兵书武略止可布阵安营保国家无侵陵之患。那妖精乃云来雾去之辈不得与他觌面相见何以征救?想东土取经者乃上邦圣僧。这和尚道高龙虎伏德重鬼神钦必有降妖之术。自古道来说是非者就是是非人。可就请这长老降妖邪救公主庶为万全之策。”那国王闻言急回头便请三藏道:“长老若有手段放法力捉了妖魔救我孩儿回朝也不须上西方拜佛长留头朕与你结为兄弟同坐龙床共享富贵如何?”三藏慌忙启上道:“贫僧粗知念佛其实不会降妖。”国王道:“你既不会降妖怎么敢上西天拜佛?”那长老瞒不过说出两个徒弟来了奏道:“陛下贫僧一人实难到此。贫僧有两个徒弟善能逢山开路遇水迭桥保贫僧到此。”国王怪道:

“你这和尚大没理既有徒弟怎么不与他一同进来见朕?若到朝中虽无中意赏赐必有随分斋供。”三藏道:“贫僧那徒弟丑陋不敢擅自入朝但恐惊伤了陛下的龙体。”国王笑道:“你看你这和尚说话终不然朕当怕他?”三藏道:“不敢说。我那大徒弟姓猪法名悟能八戒他生得长嘴獠牙刚鬃扇耳身粗肚大行路生风。第二个徒弟姓沙法名悟净和尚他生得身长丈二臂阔三停脸如蓝靛口似血盆眼光闪灼牙齿排钉。他都是这等个模样所以不敢擅领入朝。”国王道:“你既这等样说了一遍寡人怕他怎的?宣进来。”随即着金牌至馆驿相请。

那呆子听见来请对沙僧道:“兄弟你还不教下书哩这才见了下书的好处。想是师父下了书国王道:捎书人不可怠慢一定整治筵宴待他。他的食肠不济有你我之心举出名来故此着金牌来请。大家吃一顿明日好行。”沙僧道:“哥啊知道是甚缘故我们且去来。”遂将行李马匹俱交付驿丞各带随身兵器随金牌入朝。早行到白玉阶前左右立下朝上唱个喏再也不动。那文武多官无人不怕都说道:“这两个和尚貌丑也罢只是粗俗太甚!怎么见我王更不下拜喏毕平身挺然而立可怪可怪!”八戒听见道:“列位莫要议论我们是这般。乍看果有些丑只是看下些时来却也耐看。”

那国王见他丑陋已是心惊及听得那呆子说出话来越胆颤就坐不稳跌下龙床幸有近侍官员扶起。慌得个唐僧跪在殿前不住的叩头道:“陛下贫僧该万死万死!我说徒弟丑陋不敢朝见恐伤龙体果然惊了驾也。”那国王战兢兢走近前搀起道:“长老还亏你先说过了;若未说猛然见他寡人一定唬杀了也!”国王定性多时便问:“猪长老沙长老是那一位善于降妖?”那呆子不知好歹答道:“老猪会降。”国王道:

“怎么家降?”八戒道:“我乃是天蓬元帅只因罪犯天条堕落下世幸今皈正为僧。自从东土来此第一会降妖的是我。”国王道:“既是天将临凡必然善能变化。”八戒道:“不敢不敢也将就晓得几个变化儿。”国王道:“你试变一个我看看。”八戒道:“请出题目照依样子好变。”国王道:“变一个大的罢。”那八戒他也有三十六般变化就在阶前卖弄手段却便捻诀念咒喝一声叫“长!”把腰一躬就长了有八九丈长却似个开路神一般。吓得那两班文武战战兢兢;一国君臣呆呆挣挣。时有镇殿将军问道:“长老似这等变得身高必定长到甚么去处才有止极?”那呆子又说出呆话来道:“看风东风犹可西风也将就;若是南风起把青天也拱个大窟窿!”那国王大惊道:“收了神通罢晓得是这般变化了。”八戒把身一矬依然现了本相侍立阶前。国王又问道:“长老此去有何兵器与他交战?”八戒腰里掣出钯来道:“老猪使的是钉钯。”国王笑道:“可败坏门面!我这里有的是鞭简瓜锤刀枪钺斧剑戟矛镰随你选称手的拿一件去。那钯算做甚么兵器?”八戒道:“陛下不知我这钯虽然粗夯实是自幼随身之器。曾在天河水府为帅辖押八万水兵全仗此钯之力。今临凡世保护吾师逢山筑破虎狼窝遇水掀翻龙蜃穴皆是此钯。”国王闻得此言十分欢喜心信。即命九嫔妃子:“将朕亲用的御酒整瓶取来权与长老送行。”遂满斟一爵奉与八戒道:“长老这杯酒聊引奉劳之意。待捉得妖魔救回小女自有大宴相酬千金重谢。”那呆子接杯在手人物虽是粗鲁行事倒有斯文对三藏唱个大喏道:

“师父这酒本该从你饮起但君王赐我不敢违背让老猪先吃了助助兴头好捉妖怪。”那呆子一饮而干才斟一爵递与师父。三藏道:“我不饮酒你兄弟们吃罢。”沙僧近前接了。八戒就足下生云直上空里国王见了道:“猪长老又会腾云!”呆子去了沙僧将酒亦一饮而干道:“师父!那黄袍怪拿住你时我两个与他交战只战个手平。今二哥独去恐战不过他。”三藏道:“正是徒弟啊你可去与他帮帮功。”沙僧闻言也纵云跳将起去。那国王慌了扯住唐僧道:“长老你且陪寡人坐坐也莫腾云去了。”唐僧道:“可怜可怜!我半步儿也去不得!”此时二人在殿上叙话不题。

却说那沙僧赶上八戒道:“哥哥我来了。”八戒道:“兄弟你来怎的?”沙僧道:“师父叫我来帮帮功的。”八戒大喜道:“说得是来得好。我两个努力齐心去捉那怪物虽不怎的也在此国扬扬姓名。”你看他:叆叇祥光辞国界氤氲瑞气出京城。

领王旨意来山洞努力齐心捉怪灵。他两个不多时到了洞口按落云头。八戒掣钯往那波月洞的门上尽力气一筑把他那石门筑了斗来大小的个窟窿。吓得那把门的小妖开门看见是他两个急跑进去报道:“大王不好了!那长嘴大耳的和尚与那晦气脸的和尚又来把门都打破了!”那怪惊道:“这个还是猪八戒、沙和尚二人。我饶了他师父怎么又敢复来打我的门!”小妖道:“想是忘了甚么物件来取的。”老怪咄的一声道:

“胡缠!忘了物件就敢打上门来?必有缘故!”急整束了披挂绰了钢刀走出来问道:“那和尚我既饶了你师父你怎么又敢来打上我门?”八戒道:“你这泼怪干得好事儿!”老魔道:“甚么事?”八戒道:“你把宝象国三公主骗来洞内倚强霸占为妻住了一十三载也该还他了。我奉国王旨意特来擒你。你快快进去自家把绳子绑缚出来还免得老猪动手!”那老怪闻言十分怒。你看他屹迸迸咬响钢牙;滴溜溜睁圆环眼;雄纠纠举起刀来;赤淋淋拦头便砍。八戒侧身躲过使钉钯劈面迎来随后又有沙僧举宝杖赶上前齐打。这一场在山头上赌斗比前不同真个是:言差语错招人恼意毒情伤怒气生。这魔王大钢刀着头便砍;那八戒九齿钯对面来迎。沙悟净丢开宝杖那魔王抵架神兵。一猛怪二神僧来来往往甚消停。这个说:“你骗国理该死罪!”那个说:“你罗闲事报不平!”这个说:“你强婚公主伤国体!”那个说:“不干你事莫闲争!”算来只为捎书故致使僧魔两不宁。他们在那山坡前战经八九个回合八戒渐渐不济将来钉钯难举气力不加。你道如何这等战他不过?当时初相战斗有那护法诸神为唐僧在洞暗助八戒沙僧故仅得个手平;此时诸神都在宝象国护定唐僧所以二人难敌。那呆子道:“沙僧你且上前来与他斗着让老猪出恭来。”他就顾不得沙僧一溜往那蒿草薜萝荆棘葛藤里不分好歹一顿钻进那管刮破头皮搠伤嘴脸一毂辘睡倒再也不敢出来但留半边耳朵听着梆声。那怪见八戒走了就奔沙僧。沙僧措手不及被怪一把抓住捉进洞去小妖将沙僧四马攒蹄捆住。毕竟不知端的性命如何且听下回分解。
------------

第三十回 邪魔侵正法 意马忆心猿

却说那怪把沙僧捆住也不来杀他也不曾打他骂也不曾骂他一句绰起钢刀心中暗想道:“唐僧乃上邦人物必知礼义终不然我饶了他性命又着他徒弟拿我不成?噫!这多是我浑家有甚么书信到他那国里走了风讯!等我去问他一问。”那怪陡起凶性要杀公主。

却说那公主不知梳妆方毕移步前来只见那怪怒目攒眉咬牙切齿。那公主还陪笑脸迎道:“郎君有何事这等烦恼?”

那怪咄的一声骂道:“你这狗心贱妇全没人伦!我当初带你到此更无半点儿说话。你穿的锦戴的金缺少东西我去寻四时受用每日情深。你怎么只想你父母更无一点夫妇心?”那公主闻说吓得跪倒在地道:“郎君啊你怎么今日说起这分离的话?”那怪道:“不知是我分离是你分离哩!我把那唐僧拿来算计要他受用你怎么不先告过我就放了他?原来是你暗地里修了书信教他替你传寄;不然怎么这两个和尚又来打上我门教还你回去?这不是你干的事?”公主道:“郎君你差怪我了我何尝有甚书去?”老怪道:“你还强嘴哩!现拿住一个对头在此却不是证见?”公主道:“是谁?”老妖道:“是唐僧第二个徒弟沙和尚。”原来人到了死处谁肯认死只得与他放赖。公主道:“郎君且息怒我和你去问他一声。果然有书就打死了我也甘心;假若无书却不枉杀了奴奴也?”那怪闻言不容分说轮开一只簸箕大小的蓝靛手抓住那金枝玉叶的万根把公主揪上前捽在地下执着钢刀却来审沙僧咄的一声道:“沙和尚!你两个辄敢擅打上我们门来可是这女子有书到他那国国王教你们来的?”沙僧已捆在那里见妖精凶恶之甚把公主掼倒在地持刀要杀。他心中暗想道:“分明是他有书去救了我师父此是莫大之恩。我若一口说出他就把公主杀了此却不是恩将仇报?罢罢罢!想老沙跟我师父一场也没寸功报效今日已此被缚就将此性命与师父报了恩罢。”

遂喝道:“那妖怪不要无礼!他有甚么书来你这等枉他要害他性命!我们来此问你要公主有个缘故只因你把我师父捉在洞中我师父曾看见公主的模样动静。及至宝象国倒换关文那皇帝将公主画影图形前后访问因将公主的形影问我师父沿途可曾看见我师父遂将公主说起他故知是他儿女赐了我等御酒教我们来拿你要他公主还宫。此情是实何尝有甚书信?你要杀就杀了我老沙不可枉害平人大亏天理!”

那妖见沙僧说得雄壮遂丢了刀双手抱起公主道:“是我一时粗卤多有冲撞莫怪莫怪。”遂与他挽了青丝扶上宝髻软款温柔怡颜悦色撮哄着他进去了又请上坐陪礼那公主是妇人家水性见他错敬遂回心转意道:“郎君啊你若念夫妇的恩爱可把那沙僧的绳子略放松些儿。”老妖闻言即命小的们把沙僧解了绳子锁在那里。沙僧见解缚锁住立起来心中暗喜道:“古人云与人方便自己方便。我若不方便了他他怎肯教把我松放松放?”

那老妖又教安排酒席与公主陪礼压惊。吃酒到半酣老妖忽的又换了一件鲜明的衣服取了一口宝刀佩在腰里转过手摸着公主道:“浑家你且在家吃酒看着两个孩儿不要放了沙和尚。趁那唐僧在那国里我也赶早儿去认认亲也。”公主道:“你认甚亲?”老妖道:“认你父王。我是他驸马他是我丈人怎么不去认认?”公主道:“你去不得。’老妖道:“怎么去不得?”公主道:“我父王不是马挣力战的江山他本是祖宗遗留的社稷。自幼儿是太子登基城门也不曾远出没有见你这等凶汉。你这嘴脸相貌生得这等丑陋若见了他恐怕吓了他反为不美却不如不去认的还好。”老妖道:“既如此说我变个俊的儿去便罢。”公主道:“你试变来我看看。”好怪物他在那酒席间摇身一变就变做一个俊俏之人真个生得:形容典雅体段峥嵘。言语多官样行藏正妙龄。才如子建成诗易貌似潘安掷果轻。头上戴一顶鹊尾冠乌云敛伏;身上穿一件玉罗褶广袖飘迎。足下乌靴花摺腰间鸾带光明。丰神真是奇男子耸壑轩昂美俊英。公主见了十分欢喜。那妖笑道:“浑家可是变得好么?”公主道:“变得好!变得好!你这一进朝啊我父王是亲不灭一定着文武多官留你饮宴。倘吃酒中间千千仔细万万个小心却莫要现出原嘴脸来露出马脚走了风讯就不斯文了。”老妖道:“不消吩咐自有道理。’你看他纵云头早到了宝象国按落云光行至朝门之外对阁门大使道:“三驸马特来见驾乞为转奏转奏。”那黄门奏事官来至白玉阶前奏道:“万岁有三驸马来见驾现在朝门外听宣。”那国王正与唐僧叙话忽听得三驸马便问多官道:

“寡人只有两个驸马怎么又有个三驸马?”多官道:“三驸马必定是妖怪来了。”国王道:“可好宣他进来?”那长老心惊道:

“陛下妖精啊不精者不灵。他能知过去未来他能腾云驾雾宣他也进来不宣他也进来倒不如宣他进来还省些口面。”

国王准奏叫宣把怪宣至金阶他一般的也舞蹈山呼的行礼。

多官见他生得俊丽也不敢认他是妖精他都是些肉眼凡胎却当做好人。那国王见他耸壑昂霄以为济世之梁栋便问他:

“驸马你家在那里居住?是何方人氏?几时得我公主配合?怎么今日才来认亲?”那老妖叩头道:“主公臣是城东碗子山波月庄人家。”国王道:“你那山离此处多远?”老妖道:“不远只有三百里。”国王道:“三百里路我公主如何得到那里与你匹配?”那妖精巧语花言虚情假意的答道:“主公微臣自幼儿好习弓马采猎为生。那十三年前带领家童数十放鹰逐犬忽见一只斑斓猛虎身驮着一个女子往山坡下走。是微臣兜弓一箭射倒猛虎将女子带上本庄把温水温汤灌醒救了他性命。因问他是那里人家他更不曾题公主二字。早说是万岁的三公主怎敢欺心擅自配合?当得进上金殿大小讨一个官职荣身。只因他说是民家之女才被微臣留在庄所女貌郎才两相情愿故配合至此多年。当时配合之后欲将那虎宰了邀请诸亲却是公主娘娘教且莫杀。其不杀之故有几句言词道得甚好说道托天托地成夫妇无媒无证配婚姻。前世赤绳曾系足今将老虎做媒人。臣因此言故将虎解了索子饶了他性命。那虎带着箭伤跑蹄剪尾而去。不知他得了性命在那山中修了这几年炼体成精专一迷人害人。臣闻得昔年也有几次取经的都说是大唐来的唐僧想是这虎害了唐僧得了他文引变作那取经的模样今在朝中哄骗主公。主公啊那绣墩上坐的正是那十三年前驮公主的猛虎不是真正取经之人!”

你看那水性的君王愚迷肉眼不识妖精转把他一片虚词当了真实道:“贤驸马你怎的认得这和尚是驮公主的老虎?”那妖道:“主公臣在山中吃的是老虎穿的也是老虎与他同眠同起怎么不认得?”国王道:“你既认得可教他现出本相来看。”怪物道:“借半盏净水臣就教他现了本相。”国王命官取水递与驸马。那怪接水在手纵起身来走上前使个黑眼定身法念了咒语将一口水望唐僧喷去叫声“变!”那长老的真身隐在殿上真个变作一只斑斓猛虎。此时君臣同眼观看那只虎生得:白额圆头花身电目。四只蹄挺直峥嵘;二十爪钩弯锋利。锯牙包口尖耳连眉。狞狰壮若大猫形猛烈雄如黄犊样。刚须直直插银条刺舌騂騂喷恶气。果然是只猛斑斓阵阵威风吹宝殿。国王一见魄散魂飞唬得那多官尽皆躲避。有几个大胆的武将领着将军校尉一拥上前使各项兵器乱砍这一番不是唐僧该有命不死就是二十个僧人也打为肉酱。此时幸有丁甲、揭谛、功曹、护教诸神暗在半空中护佑所以那些人兵器皆不能打伤。众臣嚷到天晚才把那虎活活的捉了用铁绳锁了放在铁笼里收于朝房之内。

那国王却传旨教光禄寺大排筵宴谢驸马救拔之恩不然险被那和尚害了。当晚众臣朝散那妖魔进了银安殿。又选十八个宫娥彩女吹弹歌舞劝妖魔饮酒作乐。那怪物独坐上席左右排列的都是那艳质娇姿你看他受用。饮酒至二更时分醉将上来忍不住胡为跳起身大笑一声现了本相陡凶心伸开簸箕大手把一个弹琵琶的女子抓将过来扢咋的把头咬了一口。吓得那十七个宫娥没命的前后乱跑乱藏你看那:宫娥悚惧彩女忙惊。宫娥悚惧一似雨打芙蓉笼夜雨;彩女忙惊就如风吹芍药舞春风。捽碎琵琶顾命跌伤琴瑟逃生。出门那分南北离殿不管西东。磕损玉面撞破娇容。人人逃命走各各奔残生。那些人出去又不敢吆喝夜深了又不敢惊驾都躲在那短墙檐下战战兢兢不题。

却说那怪物坐在上面自斟自酌。喝一盏扳过人来血淋淋的啃上两口。他在里面受用外面人尽传道:“唐僧是个虎精!”乱传乱嚷嚷到金亭馆驿。此时驿里无人止有白马在槽上吃草吃料。他本是西海小龙王因犯天条锯角退鳞变白马驮唐僧往西方取经忽闻人讲唐僧是个虎精他也心中暗想道:“我师父分明是个好人必然被怪把他变做虎精害了师父。怎的好!怎的好?大师兄去得久了八戒、沙僧又无音信!”

他只捱到二更时分万籁无声却才跳将起来道:“我今若不救唐僧这功果休矣!休矣!”他忍不住顿绝缰绳抖松鞍辔急纵身忙显化依然化作龙驾起乌云直上九霄空里观看。有诗为证诗曰:三藏西来拜世尊途中偏有恶妖氛。今宵化虎灾难脱白马垂缰救主人。

小龙王在半空里只见银安殿内灯烛辉煌原来那八个满堂红上点着八根蜡烛。低下云头仔细看处那妖魔独自个在上面逼法的饮酒吃人肉哩。小龙笑道:“这厮不济!走了马脚识破风讯躧匾秤铊了吃人可是个长进的!却不知我师父下落何如倒遇着这个泼怪。且等我去戏他一戏若得手拿住妖精再救师父不迟。”好龙王他就摇身一变也变做个宫娥真个身体轻盈仪容娇媚忙移步走入里面对妖魔道声万福:

“驸马啊你莫伤我性命我来替你把盏。”那妖道:“斟酒来。”

小龙接过壶来将酒斟在他盏中酒比锺高出三五分来更不漫出这是小龙使的逼水法。那怪见了不识心中喜道:“你有这般手段!”小龙道:“还斟得有几分高哩。”那怪道:“再斟上!

再斟上!”他举着壶只情斟那酒只情高就如十三层宝塔一般尖尖满满更不漫出些须。那怪物伸过嘴来吃了一锺扳着死人吃了一口道:“会唱么?”小龙道:“也略晓得些儿。”依腔韵唱了一个小曲又奉了一锺。那怪道:“你会舞么?”小龙道:“也略晓得些儿但只是素手舞得不好看。”那怪揭起衣服解下腰间所佩宝剑掣出鞘来递与小龙。小龙接了刀就留心在那酒席前上三下四、左五右六丢开了花刀法。那怪看得眼咤小龙丢了花字望妖精劈一刀来。好怪物侧身躲过慌了手脚举起一根满堂红架住宝刀。那满堂红原是熟铁打造的连柄有八九十斤。两个出了银安殿小龙现了本相却驾起云头与那妖魔在那半空中相杀。这一场黑地里好杀!怎见得:那一个是碗子山生成的怪物这一个是西洋海罚下的真龙。一个放毫光如喷白电:一个生锐气如迸红云。一个好似白牙老象走人间一个就如金爪狸猫飞下界。一个是擎天玉柱一个是架海金梁。银龙飞舞黄鬼翻腾。左右宝刀无怠慢往来不歇满堂红。他两个在云端里战彀八九回合小龙的手软筋麻老魔的身强力壮。小龙抵敌不住飞起刀去砍那妖怪妖怪有接刀之法一只手接了宝刀一只手抛下满堂红便打小龙措手不及被他把后腿上着了一下急慌慌按落云头多亏了御水河救了性命。小龙一头钻下水去那妖魔赶来寻他不见执了宝刀拿了满堂红回上银安殿照旧吃酒睡觉不题。

却说那小龙潜于水底半个时辰听不见声息方才咬着牙忍着腿疼跳将起去踏着乌云径转馆驿还变作依旧马匹伏于槽下。可怜浑身是水腿有伤痕那时节:意马心猿都失散金公木母尽凋零。黄婆伤损通分别道义消疏怎得成!

且不言三藏逢灾小龙败战却说那猪八戒从离了沙僧一头藏在草科里拱了一个猪浑塘。这一觉直睡到半夜时候才醒。醒来时又不知是甚么去处摸摸眼定了神思侧耳才听噫!正是那山深无犬吠野旷少鸡鸣。他见那星移斗转约莫有三更时分心中想道:“我要回救沙僧诚然是单丝不线孤掌难鸣。罢!罢!罢!我且进城去见了师父奏准当今再选些骁勇人马助着老猪明日来救沙僧罢。”

那呆子急纵云头径回城里半霎时到了馆驿。此时人静月明两廊下寻不见师父只见白马睡在那厢浑身水湿后腿有盘子大小一点青痕。八戒失惊道:“双晦气了!这亡人又不曾走路怎么身上有汗腿有青痕?想是歹人打劫师父把马打坏了。”那白马认得是八戒忽然口吐人言叫声“师兄!”这呆子吓了一跌扒起来往外要走被那马探探身一口咬住皂衣道:“哥啊你莫怕我。”八戒战兢兢的道:“兄弟你怎么今日说起话来了?你但说话必有大不祥之事。”小龙道:“你知师父有难么!”八戒道:“我不知。”小龙道:“你是不知!你与沙僧在皇帝面前弄了本事思量拿倒妖魔请功求赏不想妖魔本领大你们手段不济禁他不过。好道着一个回来说个信息是却更不闻音。那妖精变做一个俊俏文人撞入朝中与皇帝认了亲眷把我师父变作一个斑斓猛虎见被众臣捉住锁在朝房铁笼里面。我听得这般苦恼心如刀割。你两日又不在不知恐一时伤了性命。只得化龙身去救不期到朝里又寻不见师父。

及到银安殿外遇见妖精我又变做个宫娥模样哄那怪物。那怪叫我舞刀他看遂尔留心砍他一刀早被他闪过双手举个满堂红把我战败。我又飞刀砍去他又把刀接了捽下满堂红把我后腿上着了一下故此钻在御水河逃得性命。腿上青是他满堂红打的。”八戒闻言道:“真个有这样事?”小龙道:“莫成我哄你了!”八戒道:“怎的好?怎的好!你可挣得动么?”小龙道:“我挣得动便怎的?”八戒道:“你挣得动便挣下海去罢。

把行李等老猪挑去高老庄上回炉做女婿去呀。”小龙闻说一口咬住他直裰子那里肯放止不住眼中滴泪道:“师兄啊!你千万休生懒惰!”八戒道:“不懒惰便怎么?沙兄弟已被他拿住我是战不过他不趁此散火还等甚么?”小龙沉吟半晌又滴泪道:“师兄啊莫说散火的话若要救得师父你只去请个人来。”八戒道:“教我请谁么?”小龙道:“你趁早儿驾云回上花果山请大师兄孙行者来。他还有降妖的大法力管教救了师父也与你我报得这败阵之仇。”八戒道:“兄弟另请一个儿便罢了那猴子与我有些不睦。前者在白虎岭上打杀了那白骨夫人他怪我撺掇师父念《紧箍儿咒》。我也只当耍子不想那老和尚当真的念起来就把他赶逐回去他不知怎么样的恼我他也决不肯来。倘或言语上略不相对他那哭丧棒又重假若不知高低捞上几下我怎的活得成么?”小龙道:“他决不打你他是个有仁有义的猴王。你见了他且莫说师父有难只说师父想你哩把他哄将来到此处见这样个情节他必然不忿断乎要与那妖精比并管情拿得那妖精救得我师父。”八戒道:“也罢也罢你倒这等尽心我若不去显得我不尽心了。我这一去果然行者肯来我就与他一路来了;他若不来你却也不要望我我也不来了。”小龙道:“你去你去管情他来也。”

真个呆子收拾了钉钯整束了直裰跳将起去踏着云径往东来。这一回也是唐僧有命那呆子正遇顺风撑起两个耳朵好便似风篷一般早过了东洋大海按落云头。不觉的太阳星上他却入山寻路。正行之际忽闻得有人言语。八戒仔细看时看来是行者在山凹里聚集群妖。他坐在一块石头崖上面前有一千二百多猴子分序排班口称“万岁!大圣爷爷!”八戒道:“且是好受用且是好受用!怪道他不肯做和尚只要来家哩!原来有这些好处许大的家业又有这多的小猴伏侍!若是老猪有这一座山场也不做甚么和尚了。如今既到这里却怎么好?必定要见他一见是。”那呆子有些怕他又不敢明明的见他却往草崖边溜阿溜的溜在那一千二三百猴子当中挤着也跟那些猴子磕头。

不知孙大圣坐得高眼又乖滑看得他明白便问:“那班部中乱拜的是个夷人是那里来的?拿上来!”说不了那些小猴一窝蜂把个八戒推将上来按倒在地。行者道:“你是那里来的夷人?”八戒低着头道:“不敢承问了。不是夷人是熟人熟人。”行者道:“我这大圣部下的群猴都是一般模样。你这嘴脸生得各样相貌有些雷堆定是别处来的妖魔。既是别处来的若要投我部下先来递个脚色手本报了名字我好留你在这随班点扎。若不留你你敢在这里乱拜!”八戒低着头拱着嘴道:“不羞就拿出这副嘴脸来了!我和你兄弟也做了几年又推认不得说是甚么夷人!”行者笑道:“抬起头来我看。”那呆子把嘴往上一伸道:“你看么!你认不得我好道认得嘴耶!”行者忍不住笑道:“猪八戒。”他听见一声叫就一毂辘跳将起来道:“正是!正是!我是猪八戒!”他又思量道:“认得就好说话了。”行者道:“你不跟唐僧取经去却来这里怎的?想是你冲撞了师父师父也贬你回来了?有甚贬书拿来我看。”八戒道:

“不曾冲撞他他也没甚么贬书也不曾赶我。”行者道:“既无贬书又不曾赶你你来我这里怎的?”八戒道:“师父想你着我来请你的。”行者道:“他也不请我他也不想我。他那日对天誓亲笔写了贬书怎么又肯想我又肯着你远来请我?我断然也是不好去的。”八戒就地扯个谎忙道:“委实想你!委是想你!”行者道:“他怎的想我来?”八戒道:“师父在马上正行叫声徒弟我不曾听见沙僧又推耳聋。师父就想起你来说我们不济说你还是个聪明伶俐之人常时声叫声应问一答十。因这般想你专专教我来请你的万望你去走走一则不孤他仰望之心二来也不负我远来之意。”行者闻言跳下崖来用手搀住八戒道:“贤弟累你远来且和我耍耍儿去。”八戒道:“哥啊这个所在路远恐师父盼望去迟我不耍子了。”行者道:

“你也是到此一场看看我的山景何如。”那呆子不敢苦辞只得随他走走。

二人携手相搀概众小妖随后上那花果山极巅之处。好山!自是那大圣回家这几日收拾得复旧如新但见那:青如削翠高似摩云。周围有虎踞龙蟠四面多猿啼鹤唳。朝出云封山顶暮观日挂林间。流水潺潺鸣玉珮涧泉滴滴奏瑶琴。山前有崖峰峭壁山后有花木秾华。上连玉女洗头盆下接天河分派水。乾坤结秀赛蓬莱清浊育成真洞府。丹青妙笔画时难仙子天机描不就。玲珑怪石石玲珑玲珑结彩岭头峰。日影动千条紫艳瑞气摇万道红霞。洞天福地人间有遍山新树与新花。八戒观之不尽满心欢喜道:“哥啊好去处!果然是天下第一名山!”行者道:“贤弟可过得日子么?”八戒笑道:“你看师兄说的话宝山乃洞天福地之处怎么说度日之言也?“二人谈笑多时下了山只见路旁有几个小猴捧着紫巍巍的葡萄香喷喷的梨枣黄森森的枇杷红艳艳的杨梅跪在路旁叫道:

“大圣爷爷请进早膳。”行者笑道:“我猪弟食肠大却不是以果子作膳的。也罢也罢莫嫌菲薄将就吃个儿当点心罢。”八戒道:“我虽食肠大却也随乡入乡是。拿来拿来我也吃几个儿尝新。”二人吃了果子渐渐日高。那呆子恐怕误了救唐僧只管催促道:“哥哥师父在那里盼望我和你哩。望你和我早早儿去罢。”行者道:“贤弟请你往水帘洞里去耍耍。”八戒坚辞道:“多感老兄盛意奈何师父久等不劳进洞罢。”行者道:“既如此不敢久留请就此处奉别。”八戒道:“哥哥你不去了?”

行者道:“我往哪里去?我这里天不收地不管自由自在不耍子儿做甚么和尚?我是不去你自去罢。但上复唐僧:既赶退了再莫想我。”呆子闻言不敢苦逼只恐逼他性子一时打上两棍无奈只得喏喏告辞找路而去。行者见他去了即差两个溜撒的小猴跟着八戒听他说些甚么。真个那呆子下了山不上三四里路回头指着行者口里骂道:“这个猴子不做和尚倒做妖怪!这个猢狲我好意来请他他却不去!你不去便罢!”走几步又骂几声。那两个小猴急跑回来报道:“大圣爷爷那猪八戒不大老实他走走儿骂几声。”行者大怒叫:

“拿将来!”那众猴满地飞来赶上把个八戒扛翻倒了抓鬃扯耳拉尾揪毛捉将回去毕竟不知怎么处治性命死活若何且听下回分解。
------------


------------

第三十二回 平顶山功曹传信 莲花洞木母逢灾

话说唐僧复得了孙行者师徒们一心同体共诣西方。自宝象国救了公主承君臣送出城西说不尽沿路饥餐渴饮。夜住晓行。却又值三春景候那时节:轻风吹柳绿如丝佳景最堪题。时催鸟语暖烘花遍地芳菲。海棠庭院来双燕正是赏春时。红尘紫陌绮罗弦管斗草传卮。师徒们正行赏间又见一山挡路。唐僧道:“徒弟们仔细前遇山高恐有虎狼阻挡。”

行者道:“师父出家人莫说在家话。你记得那乌巢和尚的《心经》云心无挂碍无挂碍方无恐怖远离颠倒梦想之言?但只是扫除心上垢洗净耳边尘。不受苦中苦难为人上人。你莫生忧虑但有老孙就是塌下天来可保无事。怕甚么虎狼!”长老勒回马道:“我当年奉旨出长安只忆西来拜佛颜。舍利国中金象彩浮屠塔里玉毫斑。寻穷天下无名水历遍人间不到山。

逐逐烟波重迭迭几时能彀此身闲?”行者闻说笑呵呵道:“师要身闲有何难事?若功成之后万缘都罢诸法皆空。那时节自然而然却不是身闲也?”长老闻言只得乐以忘忧。放辔催银駔兜缰趱玉龙。师徒们上得山来十分险峻真个嵯峨好山:巍巍峻岭削削尖峰。湾环深涧下孤峻陡崖边。湾环深涧下只听得唿喇喇戏水蟒翻身;孤峻陡崖边但见那崒嵂嵂出林虎剪尾。往上看峦头突兀透青霄;回眼观壑下深沉邻碧落。上高来似梯似凳;下低行如堑如坑。真个是古怪巅峰岭果然是连尖削壁崖。巅峰岭上采药人寻思怕走:削壁崖前打柴夫寸步难行。胡羊野马乱撺梭狡兔山牛如布阵。山高蔽日遮星斗时逢妖兽与苍狼。草径迷漫难进马怎得雷音见佛王?

长老勒马观山正在难行之处。只见那绿莎坡上佇立着一个樵夫。你道他怎生打扮:头戴一顶老蓝毡笠身穿一领毛皂衲衣。老蓝毡笠遮烟盖日果稀奇;毛皂衲衣乐以忘忧真罕见。

手持钢斧快磨明刀伐干柴收束紧。担头春色幽然四序融融;

身外闲情常是三星淡淡。到老只于随分过有何荣辱暂关山?

那樵子正在坡前伐朽柴忽逢长老自东来。停柯住斧出林外趋步将身上石崖对长老厉声高叫道:“那西进的长老!暂停片时。我有一言奉告:此山有一伙毒魔狠怪专吃你东来西去的人哩。”长老闻言魂飞魄散战兢兢坐不稳雕鞍急回头忙呼徒弟道:“你听那樵夫报道此山有毒魔狠怪谁敢去细问他一问?”行者道:“师父放心等老孙去问他一个端的。”

好行者拽开步径上山来对樵子叫声“大哥”道个问讯。樵夫答礼道:“长老啊你们有何缘故来此?”行者道:“不瞒大哥说我们是东土差来西天取经的那马上是我的师父他有些胆小。适蒙见教说有甚么毒魔狠怪故此我来奉问一声:

那魔是几年之魔怪是几年之怪?还是个把势还是个雏儿?烦大哥老实说说我好着山神土地递解他起身。”樵子闻言仰天大笑道:“你原来是个风和尚。”行者道:“我不风啊这是老实话。”樵子道:“你说是老实便怎敢说把他递解起身?”行者道:

“你这等长他那威风胡言乱语的拦路报信莫不是与他有亲?

不亲必邻不邻必友。”樵子笑道:“你这个风泼和尚忒没道理。我倒是好意特来报与你们教你们走路时早晚间防备你倒转赖在我身上。且莫说我不晓得妖魔出处就晓得啊你敢把他怎么的递解?解往何处?”行者道:“若是天魔解与玉帝;若是土魔解与土府。西方的归佛东方的归圣。北方的解与真武南方的解与火德。是蛟精解与海主是鬼祟解与阎王各有地头方向。我老孙到处里人熟一张批文把他连夜解着飞跑。”那樵子止不住呵呵冷笑道:“你这个风泼和尚想是在方上云游学了些书符咒水的法术只可驱邪缚鬼还不曾撞见这等狠毒的怪哩。”行者道:“怎见他狠毒?”樵子道:“此山径过有六百里远近名唤平顶山。山中有一洞名唤莲花洞。洞里有两个魔头他画影图形要捉和尚;抄名访姓要吃唐僧。

你若别处来的还好但犯了一个唐字儿莫想去得去得!”行者道:“我们正是唐朝来的。”樵子道:“他正要吃你们哩。”行者道:“造化!造化!但不知他怎的样吃哩?”樵子道:“你要他怎的吃?”行者道:“若是先吃头还好耍子;若是先吃脚就难为了。”樵子道:“先吃头怎么说?先吃脚怎么说?”行者道:“你还不曾经着哩。若是先吃头一口将他咬下我已死了凭他怎么煎炒熬煮我也不知疼痛;若是先吃脚他啃了孤拐嚼了腿亭吃到腰截骨我还急忙不死却不是零零碎碎受苦?此所以难为也。”樵子道:“和尚他那里有这许多工夫?只是把你拿住捆在笼里囫囵蒸吃了。”行者笑道:“这个更好!更好!疼倒不忍疼只是受些闷气罢了。”樵子道:“和尚不要调嘴。那妖怪随身有五件宝贝神通极大极广。就是擎天的玉柱架海的金梁若保得唐朝和尚去也须要昏是。”行者道:“几个昏么?”樵子道:“要三四个昏是。”行者道:“不打紧不打紧。

我们一年常七八百个昏儿这三四个昏儿易得儿就过去了。”

好大圣全然无惧一心只是要保唐僧捽脱樵夫拽步而转径至山坡马头前道:“师父没甚大事。有便有个把妖精儿只是这里人胆小放他在心上。有我哩怕他怎的?走路!走路!”长老见说只得放怀随行。正行处早不见了那樵夫。长老道:“那报信的樵子如何就不见了?”八戒道:“我们造化低撞见日里鬼了。”行者道:“想是他钻进林子里寻柴去了。等我看看来。”好大圣睁开火眼金睛漫山越岭的望处却无踪迹。

忽抬头往云端里一看看见是日值功曹他就纵云赶上骂了几声毛鬼道:“你怎么有话不来直说却那般变化了演样老孙?”慌得那功曹施礼道:“大圣报信来迟勿罪勿罪。那怪果然神通广大变化多端。只看你腾那乖巧运动神机仔细保你师父;假若怠慢了些儿西天路莫想去得。”

行者闻言把功曹叱退切切在心按云头径来山上。只见长老与八戒、沙僧簇拥前进他却暗想:“我若把功曹的言语实实告诵师父师父他不济事必就哭了;假若不与他实说梦着头带着他走常言道乍入芦圩不知深浅。倘或被妖魔捞去却不又要老孙费心?且等我照顾八戒一照顾先着他出头与那怪打一仗看。若是打得过他就算他一功;若是没手段被怪拿去等老孙再去救他不迟却好显我本事出名。”正自家计较以心问心道:“只恐八戒躲懒便不肯出头师父又有些护短等老孙羁勒他羁勒。”好大圣你看他弄个虚头把眼揉了一揉揉出些泪来迎着师父往前径走。八戒看见连忙叫:

“沙和尚歇下担子拿出行李来我两个分了罢!”沙僧道:“二哥分怎的?”八戒道:“分了罢!你往流沙河还做妖怪老猪往高老庄上盼盼浑家。把白马卖了买口棺木与师父送老大家散火还往西天去哩?”长老在马上听见道:“这个夯货!正走路怎么又胡说了?”八戒道:“你儿子便胡说!你不看见孙行者那里哭将来了?他是个钻天入地、斧砍火烧、下油锅都不怕的好汉如今戴了个愁帽泪汪汪的哭来必是那山险峻妖怪凶狠。似我们这样软弱的人儿怎么去得?”长老道:“你且休胡谈待我问他一声看是怎么说话。”问道:“悟空有甚话当面计较你怎么自家烦恼?这般样个哭包脸是虎唬我也!”行者道:“师父啊刚才那个报信的是日值功曹。他说妖精凶狠此处难行果然的山高路峻不能前进改日再去罢。”长老闻言恐惶悚惧扯住他虎皮裙子道:“徒弟呀我们三停路已走了停半因何说退悔之言?”行者道:“我没个不尽心的但只恐魔多力弱行势孤单。纵然是块铁下炉能打得几根钉?”长老道:

“徒弟啊你也说得是果然一个人也难。兵书云寡不可敌众。

我这里还有八戒沙僧都是徒弟凭你调度使用或为护将帮手协力同心扫清山径领我过山却不都还了正果?”那行者这一场扭捏只逗出长老这几句话来他揾了泪道:“师父啊若要过得此山须是猪八戒依得我两件事儿才有三分去得;

假若不依我言替不得我手半分儿也莫想过去。”八戒道:“师兄不去就散火罢不要攀我。”长老道:“徒弟且问你师兄看他教你做甚么。”呆子真个对行者说道:“哥哥你教我做甚事?”行者道:“第一件是看师父第二件是去巡山。”八戒道:

“看师父是坐巡山去是走。终不然教我坐一会又走走一会又坐两处怎么顾盼得来?”行者道:“不是教你两件齐干只是领了一件便罢。”八戒又笑道:“这等也好计较。但不知看师父是怎样巡山是怎样你先与我讲讲等我依个相应些儿的去干罢。”行者道:“看师父啊:师父去出恭你伺候;师父要走路你扶持;师父要吃斋你化斋。若他饿了些儿你该打;黄了些儿脸皮你该打;瘦了些儿形骸你该打。”八戒慌了道:“这个难!

难!难!伺候扶持通不打紧就是不离身驮着也还容易;假若教我去乡下化斋他这西方路上不识我是取经的和尚只道是那山里走出来的一个半壮不壮的健猪伙上许多人叉钯扫帚把老猪围倒拿家去宰了腌着过年这个却不就遭瘟了?”行者道:“巡山去罢。”八戒道:“巡山便怎么样儿?”行者道:“就入此山打听有多少妖怪是甚么山是甚么洞我们好过去。”八戒道:“这个小可老猪去巡山罢。”那呆子就撒起衣裙挺着钉钯雄纠纠径入深山;气昂昂奔上大路。

行者在旁忍不住嘻嘻冷笑。长老骂道:“你这个泼猴!兄弟们全无爱怜之意常怀嫉妒之心。你做出这样獐智巧言令色撮弄他去甚么巡山却又在这里笑他!”行者道:“不是笑他我这笑中有味。你看猪八戒这一去决不巡山也不敢见妖怪不知往那里去躲闪半会捏一个谎来哄我们也。”长老道:

“你怎么就晓得他?”行者道:“我估出他是这等不信等我跟他去看看听他一听一则帮副他手段降妖二来看他可有个诚心拜佛。”长老道:“好好好你却莫去捉弄他。”行者应诺了径直赶上山坡摇身一变变作个蟭蟟虫儿。其实变得轻巧但见他:翅薄舞风不用力腰尖细小如针。穿蒲抹草过花阴疾似流星还甚。眼睛明映映声气渺喑喑。昆虫之类惟他小亭亭款款机深。几番闲日歇幽林一身浑不见千眼莫能寻。嘤的一翅飞将去赶上八戒钉在他耳朵后面鬃根底下。那呆子只管走路怎知道身上有人行有七八里路把钉钯撇下吊转头来望着唐僧指手画脚的骂道:“你罢软的老和尚捉掐的弼马温面弱的沙和尚!他都在那里自在捉弄我老猪来跄路!大家取经都要望成正果偏是教我来巡甚么山!哈哈哈!晓得有妖怪躲着些儿走。还不彀一半却教我去寻他这等晦气哩!我往那里睡觉去睡一觉回去含含糊糊的答应他只说是巡了山就了其帐也。”那呆子一时间侥幸搴着钯又走。只见山凹里一弯红草坡他一头钻得进去使钉钯扑个地铺毂辘的睡下把腰伸了一伸道声“快活!就是那弼马温也不得象我这般自在!”原来行者在他耳根后句句儿听着哩忍不住飞将起来又捉弄他一捉弄。又摇身一变变作个啄木虫儿但见:铁嘴尖尖红溜翠翎艳艳光明。一双钢爪利如钉腹馁何妨林静。最爱枯槎朽烂偏嫌老树伶仃。圜睛决尾性丢灵辟剥之声堪听。

这虫鹥不大不小的上秤称只有二三两重红铜嘴黑铁脚刷剌的一翅飞下来。那八戒丢倒头正睡着了被他照嘴唇上扢揸的一下。那呆子慌得爬将起来口里乱嚷道:“有妖怪!

有妖怪!把我戳了一枪去了!嘴上好不疼呀!”伸手摸摸泱出血来了他道:“蹭蹬啊!我又没甚喜事怎么嘴上挂了红耶?”

他看着这血手口里絮絮叨叨的两边乱看却不见动静道:

“无甚妖怪怎么戳我一枪么?”忽抬头往上看时原来是个啄木虫在半空中飞哩。呆子咬牙骂道:“这个亡人!弼马温欺负我罢了你也来欺负我!我晓得了他一定不认我是个人只把我嘴当一段黑朽枯烂的树内中生了虫寻虫儿吃的将我啄了这一下也等我把嘴揣在怀里睡罢。”那呆子毂辘的依然睡倒行者又飞来着耳根后又啄了一下。呆子慌得爬起来道:

“这个亡人却打搅得我狠!想必这里是他的窠巢生蛋布雏怕我占了故此这般打搅。罢!罢!罢!不睡他了!”搴着钯径出红草坡找路又走。可不喜坏了孙行者笑倒个美猴王行者道:“这夯货大睁着两个眼连自家人也认不得!”好大圣摇身又一变还变做个蟭蟟虫钉在他耳朵后面不离他身上。那呆子入深山又行有四五里只见山凹中有桌面大的四四方方三块青石头。呆子放下钯对石头唱个大喏。行者暗笑道:“这呆子!石头又不是人又不会说话又不会还礼的唱他喏怎的可不是个瞎帐?”原来那呆子把石头当着唐僧沙僧行者三人朝着他演习哩。他道:“我这回去见了师父若问有妖怪就说有妖怪。他问甚么山我若说是泥捏的土做的锡打的铜铸的面蒸的纸糊的笔画的他们见说我呆哩若讲这话一说呆了我只说是石头山。他问甚么洞也只说是石头洞。

他问甚么门却说是钉钉的铁叶门。他问里边有多远只说入内有三层。十分再搜寻问门上钉子多少只说老猪心忙记不真。此间编造停当哄那弼马温去!”

那呆子捏合了拖着钯径回本路怎知行者在耳朵后一一听得明白。行者见他回来即腾两翅预先回去现原身见了师父。师父道:“悟空你来了悟能怎不见回?”行者笑道:“他在那里编谎哩就待来也。”长老道:“他两个耳朵盖着眼愚拙之人也他会编甚么谎?又是你捏合甚么鬼话赖他哩。”行者道:“师父你只是这等护短这是有对问的话。”把他那钻在草里睡觉被啄木虫叮醒朝石头唱喏编造甚么石头山、石头洞、铁叶门、有妖精的话预先说了。说毕不多时那呆子走将来又怕忘了那谎低着头口里温习。被行者喝了一声道:“呆子!念甚么哩?”八戒掀起耳朵来看看道:“我到了地头了!”那呆子上前跪倒长老搀起道:“徒弟辛苦啊。”八戒道:“正是。

走路的人爬山的人第一辛苦了。”长老道:“可有妖怪么?”八戒道:“有妖怪!有妖怪!一堆妖怪哩!”长老道:“怎么打你来?”八戒说:“他叫我做猪祖宗猪外公安排些粉汤素食教我吃了一顿说道摆旗鼓送我们过山哩。”行者道:“想是在草里睡着了说得是梦话?”呆子闻言就吓得矮了三寸道:“爷爷呀!我睡他怎么晓得?”行者上前一把揪住道:“你过来等我问你。”呆子又慌了战战兢兢的道:“问便罢了揪扯怎的?”行者道:“是甚么山?”八戒道:“是石头山。”“甚么洞?”道:“是石头洞。”“甚么门?”道:“是钉钉铁叶门。”“里边有多远?”道:“入内是三层。”行者道:“你不消说了后半截我记得真。恐师父不信我替你说了罢。”八戒道:“嘴脸!你又不曾去你晓得那些儿要替我说?”行者笑道:“门上钉子有多少只说老猪心忙记不真。可是么?”那呆子即慌忙跪倒。行者道:“朝着石头唱喏当做我三人对他一问一答可是么?又说等我编得谎儿停当哄那弼马温去!可是么?”那呆子连忙只是磕头道:“师兄我去巡山你莫成跟我去听的?”行者骂道:“我把你个馕糠的夯货!这般要紧的所在教你去巡山你却去睡觉!不是啄木虫叮你醒来你还在那里睡哩。及叮醒又编这样大谎可不误了大事?你快伸过孤拐来打五棍记心!”八戒慌了道:“那个哭丧棒重擦一擦儿皮塌挽一挽儿筋伤若打五下就是死了!”

行者道:“你怕打却怎么扯谎?”八戒道:“哥哥呀只是这一遭儿以后再不敢了。”行者道:“一遭便打三棍罢。”八戒道:“爷爷呀半棍儿也禁不得!”呆子没计奈何扯住师父道:“你替我说个方便儿。”长老道:“悟空说你编谎我还不信。今果如此其实该打。但如今过山少人使唤悟空你且饶他待过了山再打罢。”行者道:“古人云顺父母言情呼为大孝。师父说不打我就且饶你。你再去与他巡山若再说谎误事我定一下也不饶你!”那呆子只得爬起来奔上大路又去。你看他疑心生暗鬼步步只疑是行者变化了跟住他故见一物即疑是行者。走有七八里见一只老虎从山坡上跑过他也不怕举着钉钯道:

“师兄来听说谎的这遭不编了。”又走处那山风来得甚猛呼的一声把颗枯木刮倒滚至面前他又跌脚捶胸的道:“哥啊!

这是怎的起!一行说不敢编谎罢了又变甚么树来打人!”又走向前只见一个白颈老鸦当头喳喳的连叫几声他又道:“哥哥不羞!不羞!我说不编就不编了只管又变着老鸦怎的?你来听么?”原来这一番行者却不曾跟他去他那里却自惊自怪乱疑乱猜故无往而不疑是行者随他身也。呆子惊疑且不题。

却说那山叫做平顶山那洞叫做莲花洞。洞里两妖:一唤金角大王一唤银角大王。金角正坐对银角说:“兄弟我们多少时不巡山了?”银角道:“有半个月了。”金角道:“兄弟你今日与我去巡巡。”银角道:“今日巡山怎的?”金角道:“你不知近闻得东土唐朝差个御弟唐僧往西方拜佛一行四众叫做孙行者、猪八戒、沙和尚连马五口。你看他在那处与我把他拿来。”银角道:“我们要吃人那里不捞几个?这和尚到得那里让他去罢。”金角道:“你不晓得。我当年出天界尝闻得人言:

唐僧乃金蝉长老临凡十世修行的好人一点元阳未泄有人吃他肉延寿长生哩。”银角道:“若是吃了他肉就可以延寿长生我们打甚么坐立甚么功炼甚么龙与虎配甚么雌与雄?

只该吃他去了。等我去拿他来。”金角道:“兄弟你有些性急且莫忙着。你若走出门不管好歹但是和尚就拿将来假如不是唐僧却也不当人子?我记得他的模样曾将他师徒画了一个影图了一个形你可拿去。但遇着和尚以此照验照验。”又将某人是某名字一一说了。银角得了图像知道姓名即出洞点起三十名小怪便来山上巡逻。

却说八戒运拙正行处可可的撞见群魔当面挡住道:

“那来的甚么人?”呆子才抬起头来掀着耳朵看见是些妖魔他就慌了心中暗道:“我若说是取经的和尚他就捞了去只是说走路的。”小妖回报道:“大王是走路的。”那三十名小怪中间有认得的有不认得的旁边有听着指点说话的道:“大王这个和尚象这图中猪八戒模样。”叫挂起影神图来八戒看见大惊道:“怪道这些时没精神哩!原来是他把我的影神传将来也!”小妖用枪挑着银角用手指道:“这骑白马的是唐僧这毛脸的是孙行者。”八戒听见道:“城隍没我便也罢了猪头三牲清醮二十四分。”口里唠叨只管许愿。那怪又道:“这黑长的是沙和尚这长嘴大耳的是猪八戒。”呆子听见说他慌得把个嘴揣在怀里藏了。那怪叫:“和尚伸出嘴来!”八戒道:“胎里病伸不出来。”那怪令小妖使钩子钩出来。八戒慌得把个嘴伸出道:“小家形罢了这不是?你要看便就看钩怎的?”那怪认得是八戒掣出宝刀上前就砍。这呆子举钉钯按住道:“我的儿休无礼!看钯!”那怪笑道:“这和尚是半路出家的。”八戒道:“好儿子!有些灵性!你怎么就晓得老爷是半路出家的?”

那怪道:“你会使这钯一定是在人家园圃中筑地把他这钯偷将来也。”八戒道:“我的儿你那里认得老爷这钯。我不比那筑地之钯这是:巨齿铸来如龙爪渗金妆就似虎形。若逢对敌寒风洒但遇相持火焰生。能替唐僧消障碍西天路上捉妖精。轮动烟霞遮日月使起昏云暗斗星。筑倒泰山老虎怕掀翻大海老龙惊。饶你这妖有手段一钯九个血窟窿!”那怪闻言那里肯让使七星剑丢开解数与八戒一往一来在山中赌斗有二十回合不分胜负。八戒起狠来舍死的相迎。那怪见他捽耳朵喷粘涎舞钉钯口里吆吆喝喝的也尽有些悚惧即回头招呼小怪一齐动手。若是一个打一个其实还好。他见那些小妖齐上慌了手脚遮架不住败了阵回头就跑。原来是道路不平未曾细看忽被蓏萝藤绊了个踉跄。挣起来正走又被个小妖睡倒在地扳着他脚跟扑的又跌了个狗吃屎被一群赶上按住抓鬃毛揪耳朵扯着脚拉着尾扛扛抬抬擒进洞去。咦!正是:一身魔难消灭万种灾生不易除。毕竟不知猪八戒性命如何且听下回分解。
------------

第三十三回 外道迷真性 元神助本心

却说那怪将八戒拿进洞去道:“哥哥啊拿将一个来了。”

老魔喜道:“拿来我看。”二魔道:“这不是?”老魔道:“兄弟错拿了这个和尚没用。”八戒就绰经说道:“大王没用的和尚放他出去罢不当人子!”二魔道:“哥哥不要放他虽然没用也是唐僧一起的叫做猪八戒。把他且浸在后边净水池中浸退了毛衣使盐腌着晒干了等天阴下酒。”八戒听言道:“蹭蹬啊!撞着个贩腌腊的妖怪了!”那小妖把八戒抬进去抛在水里不题。

却说三藏坐在坡前耳热眼跳身体不安叫声:“悟空!怎么悟能这番巡山去之久而不来?”行者道:“师父还不晓得他的心哩。”三藏道:“他有甚心?”行者道:“师父啊此山若是有怪他半步难行一定虚张声势跑将回来报我;想是无怪路途平静他一直去了。”三藏道:“假若真个去了却在那里相会?此间乃是山野空阔之处比不得那店市城井之间。”行者道:“师父莫虑且请上马。那呆子有些懒惰断然走的迟慢。你把马打动些儿我们定赶上他一同去罢。”真个唐僧上马沙僧挑担行者前面引路上山。

却说那老怪又唤二魔道:“兄弟你既拿了八戒断乎就有唐僧。再去巡巡山来切莫放过他去。”二魔道:“就行就行。”

你看他急点起五十名小妖上山巡逻。正走处只见祥云缥缈瑞气盘旋二魔道:“唐僧来了。”众妖道:“唐僧在那里?”二魔道:“好人头上祥云照顶恶人头上黑气冲天。那唐僧原是金蝉长老临凡十世修行的好人所以有这样云缥缈。”众怪都不看见二魔用手指道:“那不是?”那三藏就在马上打了一个寒噤又一指又打个寒噤。一连指了三指他就一连打了三个寒噤心神不宁道:“徒弟啊我怎么打寒噤么?”沙僧道:“打寒噤想是伤食病了。行者道:“胡说师父是走着这深山峻岭必然小心虚惊。莫怕!莫怕!等老孙把棒打一路与你压压惊。”好行者理开棒在马前丢几个解数上三下四左五右六尽按那六韬三略使起神通。那长老在马上观之真个是寰中少有世上全无。剖开路一直前行险些儿不唬倒那怪物。他在山顶上看见魂飞魄丧忽失声道:“几年间闻说孙行者今日才知话不虚传果是真。”众怪上前道:“大王怎么长他人之志气灭自己之威风?你夸谁哩?”二魔道:“孙行者神通广大那唐僧吃他不成。”众怪道:“大王你没手段等我们着几个去报大大王教他点起本洞大小兵来摆开阵势合力齐心怕他走了那里去!”二魔道:“你们不曾见他那条铁棒有万夫不当之勇我洞中不过有四五百兵怎禁得他那一棒?”众妖道:“这等说唐僧吃不成却不把猪八戒错拿了?如今送还他罢。”二魔道:“拿便也不曾错拿送便也不好轻送。唐僧终是要吃只是眼下还尚不能。”众妖道:“这般说还过几年么?”二魔道:“也不消几年。我看见那唐僧只可善图不可恶取。若要倚势拿他闻也不得一闻只可以善去感他赚得他心与我心相合却就善中取计可以图之。”众妖道:“大王如定计拿他可用我等?”二魔道:“你们都各回本寨但不许报与大王知道。若是惊动了他必然走了风讯败了我计策。我自有个神通变化可以拿他。”

众妖散去他独跳下山来在那道路之旁摇身一变变做个年老的道者真个是怎生打扮?但见他:星冠晃亮鹤蓬松。羽衣围绣带云履缀黄棕。神清目朗如仙客体健身轻似寿翁。说甚么清牛道士也强如素券先生。妆成假象如真象捏作虚情似实情。他在那大路旁妆做个跌折腿的道士脚上血淋津口里哼哼的只叫“救人!救人!”

却说这三藏仗着孙大圣与沙僧欢喜前来正行处只听得叫“师父救人!”三藏闻得道:“善哉!善哉!这旷野山中四下里更无村舍是甚么人叫?想必是虎豹狼虫唬倒的。”这长老兜回俊马叫道:“那有难者是甚人?可出来。”这怪从草科里爬出对长老马前乒乓的只情磕头。三藏在马上见他是个道者却又年纪高大甚不过意连忙下马搀道:“请起请起。”那怪道:“疼!疼!疼!”丢了手看处只见他脚上流血三藏惊问道:

“先生啊你从那里来?因甚伤了尊足?”那怪巧语花言虚情假意道:“师父啊此山西去有一座清幽观宇我是那观里的道士。”三藏道:“你不在本观中侍奉香火演习经法为何在此闲行?”那魔道:“因前日山南里施主家邀道众禳星散福来晚我师徒二人一路而行。行至深衢忽遇着一只斑斓猛虎将我徒弟衔去贫道战兢兢亡命走一跤跌在乱石坡上伤了腿足不知回路。今日大有天缘得遇师父万望师父大慈悲救我一命。若得到观中就是典身卖命一定重谢深恩。”三藏闻言认为真实道:“先生啊你我都是一命之人我是僧你是道衣冠虽别修行之理则同。我不救你啊就不是出家之辈。救便救你你却走不得路哩。”那怪道:“立也立不起来怎生走路?”三藏道:“也罢也罢。我还走得路将马让与你骑一程到你上宫还我马去罢。”那怪道:“师父感蒙厚情只是腿胯跌伤不能骑马。”三藏道:“正是。”叫沙和尚:“你把行李捎在我马上你驮他一程罢。”沙僧道:“我驮他。”那怪急回头抹了他一眼道:“师父啊我被那猛虎唬怕了见这晦气色脸的师父愈加惊怕不敢要他驮。”三藏叫道:“悟空你驮罢。”行者连声答应道:“我驮我驮!”那妖就认定了行者顺顺的要他驮再不言语。沙僧笑道:“这个没眼色的老道!我驮着不好颠倒要他驮。他若看不见师父时三尖石上把筋都掼断了你的哩!”行者驮了口中笑道:“你这个泼魔怎么敢来惹我?你也问问老孙是几年的人儿!你这般鬼话儿只好瞒唐僧又好来瞒我?我认得你是这山中的怪物想是要吃我师父哩。我师父又非是等闲之辈是你吃的!你要吃他也须是分多一半与老孙是。”那魔闻得行者口中念诵道:“师父我是好人家儿孙做了道士。

今日不幸遇着虎狼之厄我不是妖怪。”行者道:“你既怕虎狼怎么不念《北斗经》?”三藏正然上马闻得此言骂道:“这个泼猴!救人一命胜造七级浮屠。你驮他驮儿便罢了且讲甚么北斗经南斗经!”行者闻言道:“这厮造化哩!我那师父是个慈悲好善之人又有些外好里枒槎。我待不驮你他就怪我。

驮便驮须要与你讲开:若是大小便先和我说。若在脊梁上淋下来臊气不堪且污了我的衣服没人浆洗。”那怪道:“我这般一把子年纪岂不知你的话说?”行者才拉将起来背在身上同长老、沙僧奔大路西行。那山上高低不平之处行者留心慢走让唐僧前去。行不上三五里路师父与沙僧下了山凹之中行者却望不见心中埋怨道:“师父偌大年纪再不晓得事体。这等远路就是空身子也还嫌手重恨不得捽了却又教我驮着这个妖怪!莫说他是妖怪就是好人这们年纪也死得着了掼杀他罢驮他怎的?”这大圣正算计要掼原来那怪就知道了且会遣山就使一个移山倒海的法术就在行者背上捻诀念动真言把一座须弥山遣在空中劈头来压行者。这大圣慌的把头偏一偏压在左肩背上笑道:“我的儿你使甚么重身法来压老孙哩?这个倒也不怕只是正担好挑偏担儿难挨。”那魔道:“一座山压他不住!”却又念咒语把一座峨眉山遣在空中来压。行者又把头偏一偏压在右肩背上。看他挑着两座大山飞星来赶师父!那魔头看见就吓得浑身是汗遍体生津道:“他却会担山!”又整性情把真言念动将一座泰山遣在空中劈头压住行者。那大圣力软筋麻遭逢他这泰山下顶之法只压得三尸神咋七窍喷红。

好妖魔使神通压倒行者却疾驾长风去赶唐三藏就于云端里伸下手来马上挝人。慌得个沙僧丢了行李掣出降妖棒当头挡住。那妖魔举一口七星剑对面来迎。这一场好杀:

七星剑降妖杖万映金光如闪亮。这个圜眼凶如黑杀神那个铁脸真是卷帘将。那怪山前大显能一心要捉唐三藏。这个努力保真僧一心宁死不肯放。他两个喷云嗳雾照天宫播土扬尘遮斗象。杀得那一轮红日淡无光大地乾坤昏荡荡。来往相持八九回不期战败沙和尚。那魔十分凶猛使口宝剑流星的解数滚来把个沙僧战得软弱难搪回头要走早被他逼住宝杖轮开大手挝住沙僧挟在左胁下将右手去马上拿了三藏脚尖儿钩着行李张开口咬着马鬃使起摄法把他们一阵风都拿到莲花洞里厉声高叫道:“哥哥!这和尚都拿来了!”老魔闻言大喜道:“拿来我看。”二魔道:“这不是?”老魔道:“贤弟呀又错拿来了也。”二魔道:“你说拿唐僧的。”老魔道:“是便就是唐僧只是还不曾拿住那有手段的孙行者。须是拿住他才好吃唐僧哩。若不曾拿得他切莫动他的人。那猴王神通广大变化多般我们若吃了他师父他肯甘心?来那门前吵闹莫想能得安生。”二魔笑道:“哥啊你也忒会抬举人。

若依你夸奖他天上少有地下全无自我观之也只如此没甚手段。”老魔道:“你拿住了?”二魔道:“他已被我遣三座大山压在山下寸步不能举移所以才把唐僧、沙和尚连马行李都摄将来也。”那老魔闻言满心欢喜道:“造化!造化!拿住这厮唐僧才是我们口里的食哩。”叫小妖:“快安排酒来且与你二大王奉一个得功的杯儿。”二魔道:“哥哥且不要吃酒叫小的们把猪八戒捞上水来吊起。”遂把八戒吊在东廊沙僧吊在西边唐僧吊在中间白马送在槽上行李收将进去。老魔笑道:

“贤弟好手段!两次捉了三个和尚。但孙行者虽是有山压住也须要作个法怎么拿他来凑蒸才好哩。”二魔道:“兄长请坐。

若要拿孙行者不消我们动身只教两个小妖拿两件宝贝把他装将来罢。”老魔道:“拿甚么宝贝去?”二魔道:“拿我的紫金红葫芦你的羊脂玉净瓶。”老魔将宝贝取出道:“差那两个去?”二魔道:“差精细鬼、伶俐虫二人去。”吩咐道:“你两个拿着这宝贝径至高山绝顶将底儿朝天口儿朝地叫一声孙行者!他若应了就已装在里面随即贴上太上老君急急如律令奉敕的帖儿他就一时三刻化为脓了。”二小妖叩头将宝贝领出去拿行者不题。

却说那大圣被魔使法压住在山根之下遇苦思三藏逢灾念圣僧厉声叫道:“师父啊!想当时你到两界山揭了压帖老孙脱了大难秉教沙门感菩萨赐与法旨我和你同住同修同缘同相同见同知乍想到了此处遭逢魔障又被他遣山压了。可怜!可怜!你死该当只难为沙僧八戒与那小龙化马一场!这正是树大招风风撼树人为名高名丧人!”叹罢那珠泪如雨。早惊了山神土地与五方揭谛神众会金头揭谛道:“这山是谁的?”土地道:“是我们的。”“你山下压的是谁?”土地道:

“不知是谁。”揭谛道:“你等原来不知。这压的是五百年前大闹天宫的齐天大圣孙悟空行者如今皈依正果跟唐僧做了徒弟。你怎么把山借与妖魔压他?你们是死了。他若有一日脱身出来他肯饶你!就是从轻土地也问个摆站山神也问个充军我们也领个大不应是。”那山神、土地才怕道:“委实不知不知只听得那魔头念起遣山咒法我们就把山移将来了谁晓得是孙大圣?”揭谛道:“你且休怕律上有云不知者不坐。我与你计较放他出来不要教他动手打你们。”土地道:“就没理了既放出来又打?”揭谛道:“你不知他有一条如意金箍棒十分利害:打着的就死挽着的就伤。磕一磕儿筋断擦一擦儿皮塌哩!”那土地山神心中恐惧与五方揭谛商议了却来到三山门外叫道:“大圣!山神土地五方揭谛来见。”好行者他虎瘦雄心还在自然的气象昂昂声音朗朗道:“见我怎的?”土地道:“告大圣得知遣开山请大圣出来赦小神不恭之罪。”行者道:“遣开山不打你。”喝声“起去!”就如官府放一般。那众神念动真言咒语把山仍遣归本位放起行者。行者跳将起来抖抖土束束裙耳后掣出棒来叫山神土地:“都伸过孤拐来每人先打两下与老孙散散闷!”众神大惊道:“刚才大圣已吩咐恕我等之罪怎么出来就变了言语要打?”行者道:“好土地!好山神!你倒不怕老孙却怕妖怪!”土地道:“那魔神通广大法术高强念动真言咒语拘唤我等在他洞里一日一个轮流当值哩!”行者听见当值二字却也心惊仰面朝天高声大叫道:“苍天!苍天!自那混沌初分天开地辟花果山生了我我也曾遍访明师传授长生秘诀。想我那随风变化伏虎降龙大闹天宫名称大圣更不曾把山神、土地欺心使唤。今日这个妖魔无状怎敢把山神、土地唤为奴仆替他轮流当值?天啊!

既生老孙怎么又生此辈?”

那大圣正感叹间又见山凹里霞光焰焰而来行者道:“山神土地你既在这洞中当值那放光的是甚物件?”土地道:“那是妖魔的宝贝放光想是有妖精拿宝贝来降你。”行者道:“这个却好耍子儿啊!我且问你他这洞中有甚人与他相往?”土地道:“他爱的是烧丹炼药喜的是全真道人。”行者道:“怪道他变个老道士把我师父骗去了。既这等你都且记打回去罢等老孙自家拿他。”那众神俱腾空而散。这大圣摇身一变变做个老真人。你道他怎生打扮:头挽双髽髻身穿百衲衣。手敲渔鼓简腰系吕公绦。斜倚大路下专候小魔妖。顷刻妖来到猴王暗放刁。不多时那两个小妖到了。行者将金箍棒伸开那妖不曾防备绊着脚扑的一跌。爬起来才看见行者口里嚷道:“惫懒!惫懒!若不是我大王敬重你这行人就和比较起来。”行者陪笑道:“比较甚么?道人见道人都是一家人。”那怪道:“你怎么睡在这里绊我一跌?”行者道:“小道童见我这老道人要跌一跌儿做见面钱。”那妖道:“我大王见面钱只要几两银子你怎么跌一跌儿做见面钱?你别是一乡风决不是我这里道士。”行者道:“我当真不是我是蓬莱山来的。”那妖道:

“蓬莱山是海岛神仙境界。”行者道:“我不是神仙谁是神仙?”

那妖却回嗔作喜上前道:“老神仙老神仙!我等肉眼凡胎不能识认言语冲撞莫怪莫怪。”行者道:“我不怪你常言道仙体不踏凡地你怎知之?我今日到你山上要度一个成仙了道的好人。那个肯跟我去?”精细鬼道:“师父我跟你去。”伶俐虫道:“师父我跟你去。”行者明知故问道:“你二位从那里来的?”那怪道:“自莲花洞来的。”要往那里去?”那怪道:“奉我大王教命拿孙行者去的。”行者道:“拿那个?”那怪又道:“拿孙行者。”孙行者道:“可是跟唐僧取经的那个孙行者么?”那妖道:“正是正是。你也认得他?”行者道:“那猴子有些无礼。我认得他我也有些恼他我与你同拿他去就当与你助功。”那怪道:“师父不须你助功我二大王有些法术遣了三座大山把他压在山下寸步难移教我两个拿宝贝来装他的。”行者道:“是甚宝贝?”精细鬼道:“我的是红葫芦他的是玉净瓶。”

行者道:“怎么样装他?”小妖道:“把这宝贝的底儿朝天口儿朝地叫他一声他若应了就装在里面贴上一张太上老君急急如律令奉敕的帖子他就一时三刻化为脓了。”行者见说心中暗惊道:“利害!利害!当时日值功曹报信说有五件宝贝这是两件了不知那三件又是甚么东西?”行者笑道:“二位你把宝贝借我看看。”那小妖那知甚么诀窍就于袖中取出两件宝贝双手递与行者。行者见了心中暗喜道:“好东西!好东西!我若把尾子一抉飕的跳起走了只当是送老孙。”忽又思道:“不好!不好!抢便抢去只是坏了老孙的名头这叫做白日抢夺了。”复递与他去道:“你还不曾见我的宝贝哩。”那怪道:“师父有甚宝贝?也借与我凡人看看压灾。”好行者伸下手把尾上毫毛拔了一根捻一捻叫“变”!即变做一个一尺七寸长的大紫金红葫芦自腰里拿将出来道:“你看我的葫芦么?”

那伶俐虫接在手看了道:“师父你这葫芦长大有样范好看却只是不中用。”行者道:“怎的不中用?”那怪道:“我这两件宝贝每一个可装千人哩。”行者道:“你这装人的何足稀罕?我这葫芦连天都装在里面哩!”那怪道:“就可以装天?”行者道:“当真的装天。”那怪道:“只怕是谎。就装与我们看看才信不然决不信你。”行者道:“天若恼着我一月之间常装他七八遭;不恼着我就半年也不装他一次。”伶俐虫道:“哥啊装天的宝贝与他换了罢。”精细鬼道:“他装天的怎肯与我装人的相换?伶俐虫道:“若不肯啊贴他这个净瓶也罢。”行者心中暗喜道:“葫芦换葫芦余外贴净瓶一件换两件其实甚相应!”即上前扯住那伶俐虫道:“装天可换么?”那怪道:“但装天就换不换我是你的儿子!”行者道:“也罢也罢我装与你们看看。”

好大圣低头捻诀念个咒语叫那日游神、夜游神、五方揭谛神:“即去与我奏上玉帝说老孙皈依正果保唐僧去西天取经路阻高山师逢苦厄。妖魔那宝吾欲诱他换之万千拜上将天借与老孙装闭半个时辰以助成功。若道半声不肯即上灵霄殿动起刀兵!”那日游神径至南天门里灵霄殿下启奏玉帝备言前事玉帝道:“这泼猴头出言无状前者观音来说放了他保护唐僧朕这里又差五方揭谛、四值功曹轮流护持如今又借天装天可装乎?”才说装不得那班中闪出哪吒三太子奏道:“万岁天也装得。”玉帝道:“天怎样装?”哪吒道:“自混沌初分以轻清为天重浊为地。天是一团清气而扶托瑶天宫阙以理论之其实难装;但只孙行者保唐僧西去取经诚所谓泰山之福缘海深之善庆今日当助他成功。”玉帝道:“卿有何助?”哪吒道:“请降旨意往北天门问真武借皂雕旗在南天门上一展把那日月星辰闭了。对面不见人捉白不见黑哄那怪道只说装了天以助行者成功。”玉帝闻言:“依卿所奏。”那太子奉旨前来北天门见真武备言前事。那祖师随将旗付太子。

早有游神急降大圣耳边道:“哪吒太子来助功了。”行者仰面观之只见祥云缭绕果是有神却回头对小妖道:“装天罢。”小妖道:“要装就装只管阿绵花屎怎的?”行者道:“我方才运神念咒来。”那小妖都睁着眼看他怎么样装天。这行者将一个假葫芦儿抛将上去。你想这是一根毫毛变的能有多重?

被那山顶上风吹去飘飘荡荡足有半个时辰方才落下。只见那南天门上哪吒太子把皂旗拨喇喇展开把日月星辰俱遮闭了真是乾坤墨染就宇宙靛装成。二小妖大惊道:“才说话时只好向午却怎么就黄昏了?”行者道:“天既装了不辨时候怎不黄昏!”“如何又这等样黑?”行者道:“日月星辰都装在里面外却无光怎么不黑!”小妖道:“师父你在那厢说话哩?”

行者道:“我在你面前不是?”小妖伸手摸着道:“只见说话更不见面目。师父此间是甚么去处?”行者又哄他道:“不要动脚此间乃是渤海岸上若塌了脚落下去啊七八日还不得到底哩!”小妖大惊道:“罢!罢!罢!放了天罢。我们晓得是这样装了。若弄一会子落下海去不得归家!”好行者见他认了真实又念咒语惊动太子把旗卷起却早见日光正午。小妖笑道:“妙啊!妙啊!这样好宝贝若不换啊诚为不是养家的儿子!”那精细鬼交了葫芦伶俐虫拿出净瓶一齐儿递与行者行者却将假葫芦儿递与那怪。行者既换了宝贝却又干事找绝:脐下拔一根毫毛吹口仙气变作一个铜钱叫道:“小童你拿这个钱去买张纸来。”小妖道:“何用?”行者道:“我与你写个合同文书。你将这两件装人的宝贝换了我一件装天的宝贝恐人心不平向后去日久年深有甚反悔不便故写此各执为照。”小妖道:“此间又无笔墨写甚文书?我与你赌个咒罢。”行者道:“怎么样赌?”小妖道:“我两件装人之宝贴换你一件装天之宝若有反悔一年四季遭瘟。”行者笑道:“我是决不反悔如有反悔也照你四季遭瘟。”说了誓将身一纵把尾子翘了一翘跳在南天门前谢了哪吒太子麾旗相助之功。太子回宫缴旨将旗送还真武不题。这行者伫立霄汉之间观看那个小妖。毕竟不知怎生区处且听下回分解。
------------

第三十四回 魔王巧算困心猿 大圣腾那骗宝贝

却说那两个小妖将假葫芦拿在手中争看一会忽抬头不见了行者。伶俐虫道:“哥啊神仙也会打诳语他说换了宝贝度我等成仙怎么不辞就去了?”精细鬼道:“我们相应便宜的多哩他敢去得成?拿过葫芦来等我装装天也试演试演看。”真个把葫芦往上一抛扑的就落将下来慌得个伶俐虫道:“怎么不装!不装!莫是孙行者假变神仙将假葫芦换了我们的真的去耶?”精细鬼道:“不要胡说!孙行者是那三座山压住了怎生得出?拿过来等我念他那几句咒儿装了看。”这怪也把葫芦儿望空丢起口中念道:“若有半声不肯就上灵霄殿上动起刀兵!”念不了扑的又落将下来。两妖道:“不装不装!

一定是个假的。”正嚷处孙大圣在半空里听得明白看得真实恐怕他弄得时辰多了紧要处走了风讯将身一抖把那变葫芦的毫毛收上身来弄得那两妖四手皆空。精细鬼道:“兄弟拿葫芦来。”伶俐虫道:“你拿着的。天呀!怎么不见了?”都去地下乱摸草里胡寻吞袖子揣腰间那里得有?二妖吓得呆呆挣挣道:“怎的好!怎的好!当时大王将宝贝付与我们教拿孙行者今行者既不曾拿得连宝贝都不见了。我们怎敢去回话?这一顿直直的打死了也!怎的好!怎的好!”伶俐虫道:

“我们走了罢。”精细鬼道:“往那里走么?”伶俐虫道:“不管那里走罢。若回去说没宝贝断然是送命了。”精细鬼道:“不要走还回去。二大王平日看你甚好我推一句儿在你身上。他若肯将就留得性命说不过就打死还在此间莫弄得两头不着去来去来!”那怪商议了转步回山。

行者在半空中见他回去又摇身一变变作苍蝇儿飞下去跟着小妖。你道他既变了苍蝇那宝贝却放在何处?如丢在路上藏在草里被人看见拿去却不是劳而无功?他还带在身上。带在身上啊苍蝇不过豆粒大小如何容得?原来他那宝贝与他金箍棒相同叫做如意佛宝随身变化可以大可以小故身上亦可容得。他嘤的一声飞下去跟定那怪不一时到了洞里。只见那两个魔头坐在那里饮酒。小妖朝上跪下行者就钉在那门柜上侧耳听着。小妖道:“大王。”二老魔即停杯道:“你们来了?”小妖道:“来了。”又问:“拿着孙行者否?”小妖叩头不敢声言。老魔又问又不敢应只是叩头。问之再三小妖俯伏在地:“赦小的万千死罪!赦小的万千死罪!

我等执着宝贝走到半山之中忽遇着蓬莱山一个神仙。他问我们那里去我们答道拿孙行者去。那神仙听见说孙行者他也恼他要与我们帮功。是我们不曾叫他帮功却将拿宝贝装人的情由与他说了。那神仙也有个葫芦善能装天。我们也是妄想之心养家之意:他的装天我的装人与他换了罢。原说葫芦换葫芦伶俐虫又贴他个净瓶。谁想他仙家之物近不得凡人之手正试演处就连人都不见了。万望饶小的们死罪!”老魔听说暴躁如雷道:“罢了!罢了!这就是孙行者假妆神仙骗哄去了!那猴头神通广大处处人熟不知那个毛神放他出来骗去宝贝!”二魔道:“兄长息怒。叵耐那猴头着然无礼既有手段便走了也罢怎么又骗宝贝?我若没本事拿他永不在西方路上为怪!”老魔道:“怎生拿他?”二魔道:“我们有五件宝贝去了两件还有三件务要拿住他。”老魔道:“还有那三件?”二魔道:“还有七星剑与芭蕉扇在我身边那一条幌金绳在压龙山压龙洞老母亲那里收着哩。如今差两个小妖去请母亲来吃唐僧肉就教他带幌金绳来拿孙行者。”老魔道:

“差那个去?”二魔道:“不差这样废物去!”将精细鬼、伶俐虫一声喝起。二人道:“造化!造化!打也不曾打骂也不曾骂却就饶了。”二魔道:“叫那常随的伴当巴山虎、倚海龙来。”二人跪下二魔吩咐道:“你却要小心。”俱应道:“小心。”“却要仔细。”俱应道:“仔细。”又问道:“你认得老奶奶家么?”又俱应道:“认得。”“你既认得你快早走动到老奶奶处多多拜上说请吃唐僧肉哩。就着带幌金绳来要拿孙行者。”

二怪领命疾走怎知那行者在旁一一听得明白。他展开翅飞将去赶上巴山虎钉在他身上。行经二三里就要打杀他两个。又思道:“打死他有何难事?但他奶奶身边有那幌金绳又不知住在何处等我且问他一问再打。”好行者嘤的一声躲离小妖让他先行有百十步却又摇身一变也变做个小妖儿戴一顶狐皮帽子将虎皮裙子倒插上来勒住赶上道:

“走路的等我一等。”那倚海龙回头问道:“是那里来的?”行者道:“好哥啊连自家人也认不得?”小妖道:“我家没有你。”行者道:“怎么没我?你再认认看。”小妖道:“面生面生不曾相会。”行者道:“正是你们不曾会着我我是外班的。”小妖道:

“外班长官是不曾会。你往那里去?”行者道:“大王说差你二位请老奶奶来吃唐僧肉教他就带幌金绳来拿孙行者。恐你二位走得缓有些贪顽误了正事又差我来催你们快去。”小妖见说着海底眼更不疑惑把行者果认做一家人急急忙忙往前飞跑一气又跑有八九里。行者道:“忒走快了些我们离家有多少路了?”小怪道:“有十五六里了。”行者道:“还有多远?”

倚海龙用手一指道:“乌林子里就是。”行者抬头见一带黑林不远料得那老怪只在林子里外却立定步让那小怪前走即取出铁棒走上前着脚后一刮。可怜忒不禁打就把两个小妖刮做一团肉饼却拖着脚藏在路旁深草科里。即便拔下一根毫毛吹口仙气叫“变!”变做个巴山虎自身却变做个倚海龙假妆做两个小妖径往那压龙洞请老奶奶。这叫做七十二变神通大指物腾那手段高。

三五步跳到林子里正找寻处只见有两扇石门半开半掩不敢擅入只得吆叫一声:“开门!开门!”早惊动那把门的一个女怪将那半扇儿开了道:“你是那里来的?”行者道:“我是平顶山莲花洞里差来请老***。”那女怪道:“进去。”到了二层门下闪着头往里观看又见那正当中高坐着一个老妈妈儿。你道他怎生模样?但见:雪鬓蓬松星光晃亮。脸皮红润皱文多牙齿稀疏神气壮。貌似菊残霜里色形如松老雨余颜。

头缠白练攒丝帕耳坠黄金嵌宝环。孙大圣见了不敢进去只在二门外仵着脸脱脱的哭起来你道他哭怎的莫成是怕他?

就怕也便不哭况先哄了他的宝贝又打杀他的小妖却为何而哭?他当时曾下九鼎油锅就煠了七八日也不曾有一点泪儿只为想起唐僧取经的苦恼他就泪出痛肠放眼便哭心却想道:“老孙既显手段变做小妖来请这老怪没有个直直的站了说话之理一定见他磕头才是。我为人做了一场好汉止拜了三个人:西天拜佛祖南海拜观音两界山师父救了我我拜了他四拜。为他使碎六叶连肝肺用尽三毛七孔心。一卷经能值几何?今日却教我去拜此怪。若不跪拜必定走了风讯。

苦啊!算来只为师父受困故使我受辱于人!”到此际也没及奈何撞将进去朝上跪下道:“奶奶磕头。”那怪道:“我儿起来。”行者暗道:“好!好!好!叫得结实!”老怪问道:“你是那里来的?”行者道:“平顶山莲花洞蒙二位大王有令差来请奶奶去吃唐僧肉教带幌金绳要拿孙行者哩。”老怪大喜道:“好孝顺的儿子!”就去叫抬出轿来。行者道:“我的儿啊!妖精也抬轿!”后壁厢即有两个女怪抬出一顶香藤轿放在门外挂上青绢纬幔。老怪起身出洞坐在轿里后有几个小女怪捧着减妆端着镜架提着手巾托着香盒跟随左右。那老怪道:

“你们来怎的?我往自家儿子去处愁那里没人伏侍要你们去献勤塌嘴?都回去!关了门看家!”那几个小妖果俱回去止有两个抬轿的。老怪问道:“那差来的叫做甚么名字?”行者连忙答应道:“他叫做巴山虎我叫做倚海龙。”老怪道:“你两个前走与我开路。”行者暗想道:“可是晦气!经倒不曾取得且来替他做皂隶!”却又不敢抵强只得向前引路大四声喝起。

行了五六里远近他就坐在石崖上等候那抬轿的到了行者道:“略歇歇如何?压得肩头疼啊。”小怪那知甚么诀窍就把轿子歇下。行者在轿后胸脯上拔下一根毫毛变做一个大烧饼抱着啃。轿夫道:“长官你吃的是甚么?”行者道:“不好说。这远的路来请奶奶没些儿赏赐肚里饥了原带来的干粮等我吃些儿再走。”轿夫道:“把些儿我们吃吃。”行者笑道:

“来么都是一家人怎么计较?”那小妖不知好歹围着行者分其干粮被行者掣出棒着头一磨一个汤着的打得稀烂;

一个擦着的不死还哼。那老怪听得人哼轿子里伸出头来看时被行者跳到轿前劈头一棍打了个窟窿脑浆迸流鲜血直冒拖出轿来看处原是个九尾狐狸。行者笑道:“造孽畜!叫甚么老奶奶!你叫老奶奶就该称老孙做上太祖公公是!”好猴王把他那幌金绳搜出来笼在袖里欢喜道:“那泼魔纵有手段已此三件儿宝贝姓孙了!”却又拔两根毫毛变做个巴山虎、倚海龙又拔两根变做两个抬轿的他却变做老奶奶模样坐在轿里。将轿子抬起径回本路。不多时到了莲花洞口那毫毛变的小妖俱在前道:“开门!开门!”内有把门的小妖开了门道:“巴山虎、倚海龙来了?”毫毛道:“来了。”“你们请的奶奶呢?”毫毛用手指道:“那轿内的不是?”小怪道:“你且住等我进去先报。”报道:“大王奶奶来耶。”两个魔头闻说即命排香案来接。行者听得暗喜道:“造化!也轮到我为人了!我先变小妖去请老怪磕了他一个头。这番来我变老怪是他母亲定行四拜之礼。虽不怎的好道也赚他两个头儿!”好大圣下了轿子抖抖衣服把那四根毫毛收在身上。那把门的小妖把空轿抬入门里他却随后徐行那般娇娇啻啻扭扭捏捏就象那老怪的行动径自进去。又只见大小群妖都来跪接鼓乐箫韶一派响喨;博山炉里霭霭香烟。他到正厅中南面坐下两个魔头双膝跪倒朝上叩头叫道:“母亲孩儿拜揖。”行者道:“我儿起来。”

却说猪八戒吊在梁上哈哈的笑了一声。沙僧道:“二哥好啊!吊出笑来也!”八戒道:“兄弟我笑中有故。”沙僧道:“甚故?”八戒道:“我们只怕是奶奶来了就要蒸吃;原来不是奶奶是旧话来了。”沙僧道:“甚么旧话?”八戒笑道:“弼马温来了。”沙僧道:“你怎么认得是他?”八戒道:“弯倒腰叫我儿起来那后面就掬起猴尾巴子。我比你吊得高所以看得明也。”

沙僧道:“且不要言语听他说甚么话。”八戒道:“正是正是。”

那孙大圣坐在中间问道:“我儿请我来有何事干?”魔头道:

“母亲啊连日儿等少礼不曾孝顺得。今早愚兄弟拿得东土唐僧不敢擅吃请母亲来献献生好蒸与母亲吃了延寿。”行者道:“我儿唐僧的肉我倒不吃听见有个猪八戒的耳朵甚好可割将下来整治整治我下酒。”那八戒听见慌了道:“遭瘟的!

你来为割我耳朵的!我喊出来不好听啊!”

噫只为呆子一句通情话走了猴王变化的风。那里有几个巡山的小怪把门的众妖都撞将进来报道:“大王祸事了!孙行者打杀奶奶假妆来耶!”魔头闻此言那容分说掣七星宝剑望行者劈脸砍来。好大圣将身一幌只见满洞红光预先走了。似这般手段着实好耍子正是那聚则成形散则成气。唬得个老魔头魂飞魄散众群精噬指摇头。老魔道:“兄弟把唐僧与沙僧、八戒、白马、行李都送还那孙行者闭了是非之门罢。”二魔道:“哥哥你说那里话?我不知费了多少辛勤施这计策将那和尚都摄将来。如今似你这等怕惧孙行者的诡谲就俱送去还他真所谓畏刀避剑之人岂大丈夫之所为也?

你且请坐勿惧。我闻你说孙行者神通广大我虽与他相会一场却不曾与他比试。取披挂来等我寻他交战三合。假若他三合胜我不过唐僧还是我们之食;如三战我不能胜他那时再送唐僧与他未迟。”老魔道:“贤弟说得是。”教:“取披挂。”众妖抬出披挂二魔结束齐整执宝剑出门外叫声:“孙行者!你往那里走了?”此时大圣已在云端里闻得叫他名字急回头观看原来是那二魔。你看他怎生打扮:头戴凤盔欺腊雪身披战甲幌镔铁。腰间带是蟒龙筋粉皮靴靿梅花摺。颜如灌口活真君貌比巨灵无二别。七星宝剑手中擎怒气冲霄威烈烈。二魔高叫道:“孙行者!快还我宝贝与我母亲来我饶你唐僧取经去!”大圣忍不住骂道:“这泼怪物错认了你孙外公!赶早儿送还我师父师弟白马行囊仍打我些盘缠往西走路。若牙缝里道半个不字就自家搓根绳儿去罢也免得你外公动手。”二魔闻言急纵云跳在空中轮宝剑来刺行者掣铁棒劈手相迎。

他两个在半空中这场好杀:棋逢对手将遇良才。棋逢对手难藏兴将遇良才可用功。那两员神将相交好便似南山虎斗北海龙争。龙争处鳞甲生辉;虎斗时爪牙乱落。爪牙乱落撒银钩鳞甲生辉支铁叶。这一个翻翻复复有千般解数;那一个来来往往无半点放闲。金箍棒离顶门只隔三分;七星剑向心窝惟争一蹍。那个威风逼得斗牛寒这个怒气胜如雷电险。他两个战了有三十回合不分胜负。

行者暗喜道:“这泼怪倒也架得住老孙的铁棒!我已得了他三件宝贝却这般苦苦的与他厮杀可不误了我的工夫?不若拿葫芦或净瓶装他去多少是好。”又想道:“不好!不好!常言道:物随主便。倘若我叫他不答应却又不误了事业?且使幌金绳扣头罢。”好大圣一只手使棒架住他的宝剑;一只手把那绳抛起刷喇的扣了魔头。原来那魔头有个《紧绳咒》有个《松绳咒》。若扣住别人就念《紧绳咒》莫能得脱;若扣住自家人就念《松绳咒》不得伤身。他认得是自家的宝贝即念《松绳咒》把绳松动便脱出来反望行者抛将去却早扣住了大圣。大圣正要使“瘦身法”想要脱身却被那魔念动《紧绳咒》紧紧扣住怎能得脱?褪至颈项之下原是一个金圈子套住。那怪将绳一扯扯将下来照光头上砍了七八宝剑行者头皮儿也不曾红了一红。那魔道:“这猴子你这等头硬我不砍你且带你回去再打你。将我那两件宝贝趁早还我!”行者道:

“我拿你甚么宝贝你问我要?”那魔头将身上细细搜检却将那葫芦、净瓶都搜出来又把绳子牵着带至洞里道:“兄长拿将来了。”老魔道:“拿了谁来?”二魔道:“孙行者。你来看你来看。”老魔一见认得是行者满面欢喜道:“是他!是他!把他长长的绳儿拴在柱枓上耍子!”真个把行者拴住两个魔头却进后面堂里饮酒。那大圣在柱根下爬蹉忽惊动八戒。那呆子吊在梁上哈哈的笑道:“哥哥啊耳朵吃不成了!”行者道:“呆子可吊得自在么?我如今就出去管情救了你们。”八戒道:

“不羞!不羞!本身难脱还想救人罢罢罢!师徒们都在一处死了好到阴司里问路!”行者道:“不要胡说!你看我出去。”八戒道:“我看你怎么出去。”那大圣口里与八戒说话眼里却抹着那些妖怪。见他在里边吃酒有几个小妖拿盘拿盏执壶酾酒不住的两头乱跑关防的略松了些儿。他见面前无人就弄神通:顺出棒来吹口仙气叫“变!”即变做一个纯钢的锉儿扳过那颈项的圈子三五锉锉做两段;扳开锉口脱将出来拔了一根毫毛叫变做一个假身拴在那里真身却幌一幌变做个小妖立在旁边。八戒又在梁上喊道:“不好了!不好了!

拴的是假货吊的是正身!”老魔停杯便问:“那猪八戒吆喝的是甚么?”行者已变做小妖上前道:“猪八戒撺道孙行者教变化走了罢他不肯走在那里吆喝哩。”二魔道:“还说猪八戒老实原来这等不老实!该打二十多嘴棍!”这行者就去拿条棍来打八戒道:“你打轻些儿若重了些儿我又喊起我认得你!”

行者道:“老孙变化也只为你们你怎么倒走了风息?这一洞里妖精都认不得怎的偏你认得?”八戒道:“你虽变了头脸还不曾变得屁股。那屁股上两块红不是?我因此认得是你。”

行者随往后面演到厨中锅底上摸了一把将两臀擦黑行至前边。八戒看见又笑道:“那个猴子去那里混了这一会弄做个黑屁股来了。”

行者仍站在跟前要偷他宝贝真个甚有见识:走上厅对那怪扯个腿子道:“大王你看那孙行者拴在柱上左右爬蹉磨坏那根金绳得一根粗壮些的绳子换将下来才好。”老魔道:

“说得是。”即将腰间的狮蛮带解下递与行者。行者接了带把假妆的行者拴住换下那条绳子一窝儿窝儿笼在袖内又拔一根毫毛吹口仙气变作一根假幌金绳双手送与那怪。那怪只因贪酒那曾细看就便收下。这个是大圣腾那弄本事毫毛又换幌金绳。

得了这件宝贝急转身跳出门外现了原身高叫:“妖怪!”

那把门的小妖问道:“你是甚人在此呼喝?”行者道:“你快早进去报与你那泼魔说者行孙来了。”那小妖如言报告老魔大惊道:“拿住孙行者又怎么有个者行孙?”二魔道:“哥哥怕他怎的?宝贝都在我手里等我拿那葫芦出去把他装将来。”老魔道:“兄弟仔细。”二魔拿了葫芦走出山门忽看见与孙行者模样一般只是略矮些儿问道:“你是那里来的”行者道:“我是孙行者的兄弟闻说你拿了我家兄却来与你寻事的。”二魔道:“是我拿了锁在洞中。你今既来必要索战。我也不与你交兵我且叫你一声你敢应我么?”行者道:“可怕你叫上千声我就答应你万声!”那魔执了宝贝跳在空中把底儿朝天口儿朝地叫声“者行孙。”行者却不敢答应心中暗想道:“若是应了就装进去哩。”那魔道:“你怎么不应我?”行者道:“我有些耳闭不曾听见。你高叫。”那怪物又叫声“者行孙。”行者在底下掐着指头算了一算道:“我真名字叫做孙行者起的鬼名字叫做者行孙。真名字可以装得鬼名字好道装不得。”却就忍不住应了他一声飕的被他吸进葫芦去贴上帖儿。原来那宝贝那管甚么名字真假但绰个应的气儿就装了去也。大圣到他葫芦里浑然乌黑把头往上一顶那里顶得动且是塞得甚紧却才心中焦躁道:“当时我在山上遇着那两个小妖他曾告诵我说:不拘葫芦净瓶把人装在里面只消一时三刻就化为脓了敢莫化了我么?”一条心又想着道:“没事!化不得我!老孙五百年前大闹天宫被太上老君放在八卦炉中炼了四十九日炼成个金子心肝银子肺腑铜头铁背火眼金睛那里一时三刻就化得我?且跟他进去看他怎的!”

二魔拿入里面道:“哥哥拿来了。”老魔道:“拿了谁?”二魔道:“者行孙是我装在葫芦里也。”老魔欢喜道:“贤弟请坐。

不要动只等摇得响再揭帖儿。”行者听得道:“我这般一个身子怎么便摇得响?只除化成稀汁才摇得响是。等我撒泡溺罢他若摇得响时一定揭帖起盖。我乘空走他娘罢!”又思道“不好不好!溺虽可响只是污了这直裰。等他摇时我但聚些唾津漱口稀漓呼喇的哄他揭开老孙再走罢。”大圣作了准备那怪贪酒不摇。大圣作个法意思只是哄他来摇忽然叫道:“天呀!孤拐都化了!”那魔也不摇。大圣又叫道:“娘啊!连腰截骨都化了!”老魔道:“化至腰时都化尽矣揭起帖儿看看。”那大圣闻言就拔了一根毫毛。叫“变!”变作个半截的身子在葫芦底上真身却变做个蟭蟟虫儿钉在那葫芦口边。只见那二魔揭起帖子看时大圣早已飞出打个滚又变做个倚海龙。倚海龙却是原去请老***那个小妖他变了站在旁边。那老魔扳着葫芦口张了一张见是个半截身子动耽他也不认真假慌忙叫:“兄弟盖上!盖上!还不曾化得了哩!”二魔依旧贴上。大圣在旁暗笑道:“不知老孙已在此矣!”

那老魔拿了壶满满的斟了一杯酒近前双手递与二魔道:“贤弟我与你递个锺儿。”二魔道:“兄长我们已吃了这半会酒又递甚锺?”老魔道:“你拿住唐僧、八戒、沙僧犹可又索了孙行者装了者行孙如此功劳该与你多递几锺。”二魔见哥哥恭敬怎敢不接但一只手托着葫芦一只手不敢去接却把葫芦递与倚海龙双手去接杯不知那倚海龙是孙行者变的。你看他端葫芦殷勤奉侍。二魔接酒吃了也要回奉一杯老魔道:“不消回酒我这里陪你一杯罢。”两人只管谦逊。行者顶着葫芦眼不转睛看他两个左右传杯全无计较他就把个葫芦揌入衣袖拔根毫毛变个假葫芦一样无二捧在手中。那魔递了一会酒也不看真假一把接过宝贝各上席安然坐下依然叙饮。孙大圣撤身走过得了宝贝心中暗喜道:“饶这魔头有手段毕竟葫芦还姓孙!”毕竟不知向后怎样施为方得救师灭怪且听下回分解。
------------

第三十五回 外道施威欺正性 心猿获宝伏邪魔

“本性圆明道自通翻身跳出网罗中。修成变化非容易炼就长生岂俗同?清浊几番随运转辟开数劫任西东。逍遥万亿年无计一点神光永注空。”此诗暗合孙大圣的道妙。他自得了那魔真宝笼在袖中喜道:“泼魔苦苦用心拿我诚所谓水中捞月;老孙若要擒你就好似火上弄冰。”藏着葫芦密密的溜出门外现了本相厉声高叫道:“精怪开门!”旁有小妖道:“你又是甚人敢来吆喝?”行者道:“快报与你那老泼魔吾乃行者孙来也。”那小妖急入里报道:“大王门外有个甚么行者孙来了。”老魔大惊道:“贤弟不好了!惹动他一窝风了!幌金绳现拴着孙行者葫芦里现装着者行孙怎么又有个甚么行者孙?

想是他几个兄弟都来了。”二魔道:兄长放心我这葫芦装下一千人哩。我才装了者行孙一个又怕那甚么行者孙!等我出去看看一装来。”老魔道:“兄弟仔细。”

你看那二魔拿着个假葫芦还象前番雄纠纠、气昂昂走出门高呼道:“你是那里人氏敢在此间吆喝?”行者道:“你认不得我?家居花果山祖贯水帘洞。只为闹天宫多时罢争竞。如今幸脱灾弃道从僧用。秉教上雷音求经归觉正。相逢野泼魔却把神通弄。还我大唐僧上西参佛圣。两家罢战争各守平安境。休惹老孙焦伤残老性命!”那魔道:“你且过来我不与你相打但我叫你一声你敢应么?”行者笑道:“你叫我我就应了;我若叫你你可应么?”那魔道:“我叫你是我有个宝贝葫芦可以装人;你叫我却有何物?”行者道:“我也有个葫芦儿。”那魔道:“既有拿出来我看。”行者就于袖中取出葫芦道:“泼魔你看!”幌一幌复藏在袖中恐他来抢。那魔见了大惊道:“他葫芦是那里来的?怎么就与我的一般?纵是一根藤上结的也有个大小不同偏正不一却怎么一般无二?”他便正色叫道:“行者孙你那葫芦是那里来的?”行者委的不知来历接过口来就问他一句道:“你那葫芦是那里来的?”那魔不知是个见识只道是句老实言语就将根本从头说出道:“我这葫芦是混沌初分天开地辟有一位太上老祖解化女娲之名炼石补天普救阎浮世界;补到乾宫夬地见一座昆仑山脚下有一缕仙藤上结着这个紫金红葫芦却便是老君留下到如今者。”大圣闻言就绰了他口气道:“我的葫芦也是那里来的。”

魔头道:“怎见得?”大圣道:“自清浊初开天不满西北地不满东南太上道祖解化女娲补完天缺行至昆仑山下有根仙藤藤结有两个葫芦。我得一个是雄的你那个却是雌的。”那怪道:“莫说雌雄但只装得人的就是好宝贝。”大圣道:“你也说得是我就让你先装。”那怪甚喜急纵身跳将起去到空中执着葫芦叫一声“行者孙。”大圣听得却就不歇气连应了八九声只是不能装去。那魔坠将下来跌脚捶胸道:“天那!只说世情不改变哩!这样个宝贝也怕老公雌见了雄就不敢装了!”行者笑道:“你且收起轮到老孙该叫你哩。”急纵筋斗跳起去将葫芦底儿朝天口儿朝地照定妖魔叫声“银角大王”。那怪不敢闭口只得应了一声倏的装在里面被行者贴上“太上老君急急如律令奉敕”的帖子心中暗喜道:“我的儿你今日也来试试新了!”

他就按落云头拿着葫芦心心念念只是要救师父又往莲花洞口而来。那山上都是些洼踏不平之路况他又是个圈盘腿拐呀拐的走着摇的那葫芦里漷漷索索响声不绝。(WWW.mianhuatang.la 好看的小说)你道他怎么便有响声?原来孙大圣是熬炼过的身体急切化他不得那怪虽也能腾云驾雾不过是些法术大端是凡胎未脱到于宝贝里就化了。行者还不当他就化了笑道:“我儿子啊不知是撒尿耶不知是漱口哩这是老孙干过的买卖。不等到七八日化成稀汁我也不揭盖来看。忙怎的?有甚要紧?想着我出来的容易就该千年不看才好!”他拿着葫芦说着话不觉的到了洞口把那葫芦摇摇一响了他道:“这个象课的筒子响倒好课。等老孙一课看师父甚么时才得出门。”你看他手里不住的摇口里不住的念道:“周易文王、孔子圣人、桃花女先生、鬼谷子先生。”那洞里小妖看见道:“大王祸事了!行者孙把二大王爷爷装在葫芦里课哩!”那老魔闻得此言。唬得魂飞魄散骨软筋麻扑的跌倒在地放声大哭道:“贤弟呀!我和你私离上界转托尘凡指望同享荣华永为山洞之主。怎知为这和尚伤了你的性命断吾手足之情!”满洞群妖一齐痛哭。

猪八戒吊在梁上听得他一家子齐哭忍不住叫道:“妖精你且莫哭等老猪讲与你听。先来的孙行者次来的者行孙后来的行者孙返复三字都是我师兄一人。他有七十二变化腾那进来盗了宝贝装了令弟。令弟已是死了不必这等扛丧快些儿刷净锅灶办些香蕈、蘑菇、茶芽、竹笋、豆腐、面筋、木耳、蔬菜请我师徒们下来与你令弟念卷受生经。”那老魔闻言心中大怒道:“只说猪八戒老实原来甚不老实!他倒作笑话儿打觑我!”叫小妖:“且休举哀把猪八戒解下来蒸得稀烂等我吃饱了再去拿孙行者报仇。”沙僧埋怨八戒道:“好么!我说教你莫多话多话的要先蒸吃哩!”那呆子也尽有几分悚惧。旁一小妖道:“大王猪八戒不好蒸。”八戒道:“阿弥陀佛!是那位哥哥积阴德的?果是不好蒸。”又有一个妖道:“将他皮剥了就好蒸。”八戒慌了道:“好蒸!好蒸!皮骨虽然粗糙汤滚就烂棬户!棬户!”正嚷处只见前门外一个小妖报道:

“行者孙又骂上门来了!”那老魔又大惊道:“这厮轻我无人!”

叫:“小的们且把猪八戒照旧吊起查一查还有几件宝贝。”管家的小妖道:“洞中还有三件宝贝哩。”老魔问:“是那三件?”管家的道:“还有七星剑、芭蕉扇与净瓶。”老魔道:“那瓶子不中用原是叫人人应了就装得转把个口诀儿教了那孙行者倒把自家兄弟装去了。不用他放在家里快将剑与扇子拿来。”

那管家的即将两件宝贝献与老魔。老魔将芭蕉扇插在后项衣领把七星剑提在手中又点起大小群妖有三百多名都教一个个拈枪弄棒理索轮刀。这老魔却顶盔贯甲罩一领赤焰焰的丝袍。群妖摆出阵去要拿孙大圣。那孙大圣早已知二魔化在葫芦里面却将他紧紧拴扣停当撒在腰间手持着金箍棒准备厮杀。只见那老妖红旗招展跳出门来。却怎生打扮?头上盔缨光焰焰腰间带束彩霞鲜。身穿铠甲龙鳞砌上罩红袍烈火然。圆眼睁开光掣电钢须飘起乱飞烟。七星宝剑轻提手芭蕉扇子半遮肩。行似流云离海岳声如霹雳震山川。威风凛凛欺天将怒帅群妖出洞前。那老魔急令小妖摆开阵势骂道:

“你这猴子十分无礼!害我兄弟伤我手足着然可恨!”行者骂道:“你这讨死的怪物!你一个妖精的性命舍不得似我师父、师弟、连马四个生灵平白的吊在洞里我心何忍!情理何甘!

快快的送将出来还我多多贴些盘费喜喜欢欢打老孙起身还饶了你这个老妖的狗命!”那怪那容分说举宝剑劈头就砍这大圣使铁棒举手相迎。这一场在洞门外好杀!咦!金箍棒与七星剑对撞霞光如闪电。悠悠冷气逼人寒荡荡昏云遮岭堰。那个皆因手足情些儿不放善;这个只为取经僧毫厘不容缓。两家各恨一般仇二处每怀生怒怨。只杀得天昏地暗鬼神惊日淡烟浓龙虎战。这个咬牙锉玉钉那个怒目飞金焰。一来一往逞英雄不住翻腾棒与剑。这老魔与大圣战经二十回合不分胜负他把那剑梢一指叫声“小妖齐来!”那三百余精一齐拥上把行者围在垓心。好大圣公然无惧使一条棒左冲右撞后抵前遮。那小妖都有手段越打越上一似绵絮缠身搂腰扯腿莫肯退后大圣慌了即使个身外身法将左胁下毫毛拔了一把嚼碎喷去喝声叫“变!”一根根都变做行者。你看他长的使棒短的轮拳再小的没处下手抱着孤拐啃筋把那小妖都打得星落云散齐声喊道:“大王啊事不谐矣!

难矣乎哉!满地盈山皆是孙行者了!”被这身外法把群妖打退止撇得老魔围困中间赶得东奔西走出路无门。

那魔慌了将左手擎着宝剑右手伸于项后取出芭蕉扇子望东南丙丁火正对离宫唿喇的一扇子搧将下来只见那就地上火光焰焰。原来这般宝贝平白地搧出火来。那怪物着实无情:一连搧了七八扇子熯天炽地烈火飞腾。好火:

那火不是天上火不是炉中火也不是山头火也不是灶底火乃是五行中自然取出的一点灵光火。这扇也不是凡间常有之物也不是人工造就之物乃是自开辟混沌以来产成的珍宝之物。用此扇搧此火、煌煌烨烨就如电掣红绡;灼灼辉辉却似霞飞绛绮。更无一缕青烟尽是满山赤焰只烧得岭上松翻成火树崖前柏变作灯笼。那窝中走兽贪性命西撞东奔;这林内飞禽惜羽毛高飞远举。这场神火飘空燎只烧得石烂溪干遍地红!大圣见此恶火却也心惊胆颤道声“不好了!我本身可处毫毛不济一落这火中岂不真如燎毛之易?”将身一抖遂将毫毛收上身来只将一根变作假身子避火逃灾他的真身捻着避火诀纵筋斗跳将起去脱离了大火之中径奔他莲花洞里想着要救师父。急到门前把云头按落又见那洞门外有百十个小妖都破头折脚肉绽皮开原来都是他分身法打伤了的都在这里声声唤唤忍疼而立。大圣见了按不住恶性凶顽轮起铁棒一路打将进去。可怜把那苦炼人身的功果息依然是块旧皮毛!

那大圣打绝了小妖撞入洞里要解师父又见那内面有火光焰焰唬得他手慌脚忙道:“罢了!罢了!这火从后门口烧起来老孙却难救师父也!”正悚惧处仔细看时呀!原来不是火光却是一道金光。他正了性往里视之乃羊脂玉净瓶放光却自心中欢喜道:“好宝贝耶!这瓶子曾是那小妖拿在山上放光老孙得了不想那怪又复搜去。今日藏在这里原来也放光。”你看他窃了这瓶子喜喜欢欢且不救师父急抽身往洞外而走。才出门只见那妖魔提着宝剑拿着扇子从南而来。

孙大圣回避不及被那老魔举剑劈头就砍。大圣急纵筋斗云跳将起去无影无踪的逃了不题。

却说那怪到得门口但见尸横满地就是他手下的群精慌得仰天长叹止不住放声大哭道:“苦哉!痛哉!”有诗为证诗曰:可恨猿乖马劣顽灵胎转托降尘凡。只因错念离天阙致使忘形落此山。鸿雁失群情切切妖兵绝族泪潺潺。何时孽满开愆锁返本还原上御关?那老魔惭惶不已一步一声哭入洞内只见那什物家火俱在只落得静悄悄没个人形;悲切切愈加凄惨。独自个坐在洞中蹋伏在那石案之上将宝剑斜倚案边把扇子插于肩后昏昏默默睡着了这正是人逢喜事精神爽闷上心来瞌睡多。

话说孙大圣拨转筋斗云佇立山前想着要救师父把那净瓶儿牢扣腰间径来洞口打探。见那门开两扇静悄悄的不闻消耗随即轻轻移步潜入里边只见那魔斜倚石案呼呼睡着芭蕉扇褪出肩衣半盖着脑后七星剑还斜倚案边却被他轻轻的走上前拔了扇子急回头呼的一声跑将出去。原来这扇柄儿刮着那怪的头早惊醒他。抬头看时是孙行者偷了急慌忙执剑来赶。那大圣早已跳出门前将扇子撒在腰间双手轮开铁棒与那魔抵敌。这一场好杀:恼坏泼妖王怒冲冠志。恨不过挝来囫囵吞难解心头气。恶口骂猢狲:“你老大将人戏伤我若干生还来偷宝贝!这场决不容定见存亡计!”大圣喝妖魔:“你好不知趣!徒弟要与老师争累卵焉能击石碎?”

宝剑来铁棒去两家更不留仁义。一翻二复赌输赢三转四回施武艺。盖为取经僧灵山参佛位致令金火不相投五行拨乱伤和气。扬威耀武显神通走石飞沙弄本事。交锋渐渐日将晡魔头力怯先回避。那老魔与大圣战经三四十合天将晚矣抵敌不住败下阵来径往西南上投奔压龙洞去不题。

这大圣才按落云头闯入莲花洞里解下唐僧与八戒、沙和尚来。他三人脱得灾危谢了行者却问:“妖魔那里去了?”

行者道:“二魔已装在葫芦里想是这会子已化了;大魔才然一阵战败往西南压龙山去讫。概洞小妖被老孙分身法打死一半还有些败残回的又被老孙杀绝方才得入此处解放你们。”唐僧谢之不尽道:“徒弟啊多亏你受了劳苦!”行者笑道:

“诚然劳苦。你们还只是吊着受疼我老孙再不曾住脚比急递铺的铺兵还甚反复里外奔波无已。因是偷了他的宝贝方能平退妖魔。”猪八戒道:“师兄你把那葫芦儿拿出来与我们看看。只怕那二魔已化了也。”大圣先将净瓶解下又将金绳与扇子取出然后把葫芦儿拿在手道:“莫看莫看!他先曾装了老孙被老孙漱口哄得他扬开盖子老孙方得走了。我等切莫揭盖只怕他也会弄喧走了。”师徒们喜喜欢欢将他那洞中的米面菜蔬寻出。烧刷了锅灶安排些素斋吃了饱餐一顿安寝洞中。一夜无词早又天晓。

却说那老魔径投压龙山会聚了大小女怪备言打杀母亲装了兄弟绝灭妖兵偷骗宝贝之事众女怪一齐大哭。哀痛多时道:“你等且休凄惨。我身边还有这口七星剑欲会汝等女兵都去压龙山后会借外家亲戚断要拿住那孙行者报仇。”说不了有门外小妖报道:“大王山后老舅爷帅领若干兵卒来也。”老魔闻言急换了缟素孝服躬身迎接。原来那老舅爷是他母亲之弟名唤狐阿七大王因闻得哨山的妖兵报道他姐姐被孙行者打死假变姐形盗了外甥宝贝连日在平顶山拒敌。他却帅本洞妖兵二百余名特来助阵故此先拢姐家问信。才进门见老魔挂了孝服二人大哭。哭久老魔拜下备言前事。那阿七大怒即命老魔换了孝服提了宝剑尽点女妖合同一处纵风云径投东北而来。

这大圣却教沙僧整顿早斋吃了走路忽听得风声走出门看乃是一伙妖兵自西南上来。行者大惊急抽身忙呼八戒道:“兄弟妖精又请救兵来也。”三藏闻言惊恐失色道:“徒弟似此如何?”行者笑道:“放心!放心!”把他这宝贝都拿来与我。”大圣将葫芦、净瓶系在腰间金绳笼于袖内芭蕉扇插在肩后双手轮着铁棒教沙僧保守师父稳坐洞中着八戒执钉钯同出洞外迎敌。那怪物摆开阵势只见当头的是阿七大王。

他生的玉面长髯钢眉刀耳头戴金炼盔身穿锁子甲手执方天戟高声骂道:“我把你个大胆的泼猴!怎敢这等欺人!偷了宝贝伤了眷族杀了神兵又敢久占洞府!赶早儿一个个引颈受死雪我姐家之仇!”行者骂道:“你这伙作死的毛团不识你孙外公的手段!不要走!领吾一棒!”那怪物侧身躲过使方天戟劈面相印。两个在山头一来一往战经三四回合那怪力软败阵回走。行者赶来却被老魔接住又斗了三合只见那狐阿七复转来攻。这壁厢八戒见了急掣九齿钯挡住。一个抵一个战经多时不分胜败那老魔喝了一声众妖兵一齐围上。

却说那三藏坐在莲花洞里听得喊声振地便叫:“沙和尚你出去看你师兄胜负如何。”沙僧果举降妖杖出来喝一声撞将出去打退群妖。阿七见事势不利回头就走被八戒赶上照背后一钯就筑得九点鲜红往外冒可怜一灵真性赴前程。急拖来剥了衣服看处原来也是个狐狸精。那老魔见伤了他老舅丢了行者提宝剑就劈八戒八戒使钯架住。正赌斗间沙僧撞近前来举杖便打那妖抵敌不住纵风云往南逃走八戒沙僧紧紧赶来。大圣见了急纵云跳在空中解下净瓶罩定老魔叫声“金角大王!”那怪只道是自家败残的小妖呼叫就回头应了一声飕的装将进去被行者贴上“太上老君急急如律令奉敕”的帖子。只见那七星剑坠落尘埃也归了行者。八戒迎着道:“哥哥宝剑你得了精怪何在?”行者笑道:了了!已装在我这瓶儿里也。”沙僧听说与八戒十分欢喜。

当时通扫净诸邪回至洞里与三藏报喜道:“山已净妖已无矣请师父上马走路。”三藏喜不自胜。师徒们吃了早斋收拾了行李马匹奔西找路。正行处猛见路旁闪出一个瞽者走上前扯住三藏马道:“和尚那里去?还我宝贝来!”八戒大惊道:“罢了!这是老妖来讨宝贝了!”行者仔细观看原来是太上李老君慌得近前施礼道:“老官儿那里去?”那老祖急升玉局宝座九霄空里佇立叫:“孙行者还我宝贝。”大圣起到空中道:“甚么宝贝?”老君道:“葫芦是我盛丹的净瓶是我盛水的宝剑是我炼魔的扇子是我搧火的绳子是我一根勒袍的带。

那两个怪:一个是我看金炉的童子一个是我看银炉的童子只因他偷了我的宝贝走下界来正无觅处却是你今拿住得了功绩。”大圣道:“你这老官儿着实无礼纵放家属为邪该问个钤束不严的罪名。”老君道:“不干我事不可错怪了人。此乃海上菩萨问我借了三次送他在此托化妖魔看你师徒可有真心往西去也。”大圣闻言心中作念道:“这菩萨也老大惫懒!

当时解脱老孙教保唐僧西去取经我说路途艰涩难行他曾许我到急难处亲来相救。如今反使精邪掯害语言不的该他一世无夫!若不是老官儿亲来我决不与他。既是你这等说拿去罢。”那老君收得五件宝贝揭开葫芦与净瓶盖口倒出两股仙气用手一指仍化为金、银二童子相随左右。只见那霞光万道咦!缥缈同归兜率院逍遥直上大罗天。毕竟不知此后又有甚事孙大圣怎生保护唐僧几时得到西天且听下回分解。
------------


------------

第三十七回 鬼王夜谒唐三藏 悟空神化引婴儿

却说三藏坐于宝林寺禅堂中灯下念一会《梁皇水忏》看一会《孔雀真经》只坐到三更时候却才把经本包在囊里正欲起身去睡只听得门外扑剌剌一声响喨淅零零刮阵狂风。

那长老恐吹灭了灯慌忙将褊衫袖子遮住又见那灯或明或暗便觉有些心惊胆战。此时又困倦上来伏在经案上盹睡虽是合眼朦胧却还心中明白耳内嘤嘤听着那窗外阴风飒飒。

好风真个那淅淅潇潇飘飘荡荡。淅淅潇潇飞落叶飘飘荡荡卷浮云。满天星斗皆昏昧遍地尘沙尽洒纷。一阵家猛一阵家纯。纯时松竹敲清韵猛处江湖波浪浑。刮得那山鸟难栖声哽哽海鱼不定跳喷喷。东西馆阁门窗脱前后房廊神鬼。佛殿花瓶吹堕地琉璃摇落慧灯昏。香炉鞍+倒香灰迸烛架歪斜烛焰横。幢幡宝盖都摇拆钟鼓楼台撼动根。

那长老昏梦中听着风声一时过处又闻得禅堂外隐隐的叫一声“师父!”忽抬头梦中观看门外站着一条汉子浑身上下水淋淋的眼中垂泪口里不住叫:“师父!师父!”三藏欠身道:“你莫是魍魉妖魅神怪邪魔至夜深时来此戏我?我却不是那贪欲贪嗔之类。我本是个光明正大之僧奉东土大唐旨意上西天拜佛求经者。我手下有三个徒弟都是降龙伏虎之英豪扫怪除魔之壮士。他若见了你碎尸粉骨化作微尘。此是我大慈悲之意方便之心。你趁早儿潜身远遁莫上我的禅门来。”那人倚定禅堂道:“师父我不是妖魔鬼怪亦不是魍魉邪神。”三藏道:“你既不是此类却深夜来此何为?”那人道:

“师父你舍眼看我一看。”长老果仔细定睛看处呀!只见他头戴一顶冲天冠腰束一条碧玉带身穿一领飞龙舞凤赭黄袍足踏一双云头绣口无忧履手执一柄列斗罗星白玉圭。面如东岳长生帝形似文昌开化君。三藏见了大惊失色急躬身厉声高叫道:“是那一朝陛下?请坐。”用手忙搀扑了个空虚回身坐定。再看处还是那个人。长老便问:“陛下你是那里皇王?

何邦帝主?想必是国土不宁谗臣欺虐半夜逃生至此。有何话说说与我听。”这人才泪滴腮边谈旧事愁攒眉上诉前因道:“师父啊我家住在正西离此只有四十里远近。那厢有座城池便是兴基之处。”三藏道:“叫做甚么地名?”那人道:“不瞒师父说便是朕当时创立家邦改号乌鸡国。”三藏道:“陛下这等惊慌却因甚事至此?”那人道:“师父啊我这里五年前天年干旱草子不生民皆饥死甚是伤情。”三藏闻言点头叹道:“陛下啊古人云国正天心顺。想必是你不慈恤万民既遭荒歉怎么就躲离城郭?且去开了仓库赈济黎民;悔过前非重兴今善放赦了那枉法冤人。自然天心和合雨顺风调。”那人道:“我国中仓禀空虚钱粮尽绝文武两班停俸禄寡人膳食亦无荤。仿效禹王治水与万民同受甘苦沐浴斋戒昼夜焚香祈祷。如此三年只干得河枯井涸。正都在危急之处忽然锺南山来了一个全真能呼风唤雨点石成金。先见我文武多官后来见朕当即请他登坛祈祷果然有应只见令牌响处顷刻间大雨滂沱。寡人只望三尺雨足矣他说久旱不能润泽又多下了二寸。朕见他如此尚义就与他八拜为交以兄弟称之。”三藏道:“此陛下万千之喜也。”那人道:“喜自何来?”三藏道:“那全真既有这等本事若要雨时就教他下雨若要金时就教他点金。还有那些不足却离了城阙来此?”那人道:“朕与他同寝食者只得二年。又遇着阳春天气红杏夭桃开花绽蕊家家士女处处王孙俱去游春赏玩。那时节文武归衙嫔妃转院。朕与那全真携手缓步至御花园里忽行到八角琉璃井边不知他抛下些甚么物件井中有万道金光。哄朕到井边看甚么宝贝他陡起凶心扑通的把寡人推下井内将石板盖住井口拥上泥土移一株芭蕉栽在上面。可怜我啊已死去三年是一个落井伤生的冤屈之鬼也!”

唐僧见说是鬼唬得筋力酥软毛骨耸然没奈何只得将言又问他道:“陛下你说的这话全不在理。既死三年那文武多官三宫皇后遇三朝见驾殿上怎么就不寻你?”那人道:

“师父啊说起他的本事果然世间罕有!自从害了朕他当时在花园内摇身一变就变做朕的模样更无差别。(WWW.mianhuatang.la 好看的小说)现今占了我的江山暗侵了我的国土。他把我两班文武四百朝官三宫皇后六院嫔妃尽属了他矣。”三藏道:“陛下你忒也懦。”那人道:“何懦?”三藏道:“陛下那怪倒有些神通变作你的模样侵占你的乾坤文武不能识后妃不能晓只有你死的明白。你何不在阴司阎王处具告把你的屈情伸诉伸诉?”那人道:“他的神通广大官吏情熟都城隍常与他会酒海龙王尽与他有亲东岳天齐是他的好朋友十代阎罗是他的异兄弟。因此这般我也无门投告。”三藏道:“陛下你阴司里既没本事告他却来我阳世间作甚?”那人道:“师父啊我这一点冤魂怎敢上你的门来?山门前有那护法诸天、六丁六甲、五方揭谛、四值功曹、一十八位护教伽蓝紧随鞍马。却才被夜游神一阵神风把我送将进来他说我三年水灾该满着我来拜谒师父。他说你手下有一个大徒弟是齐天大圣极能斩怪降魔。今来志心拜恳千乞到我国中拿住妖魔辨明邪正朕当结草衔环报酬师恩也!”三藏道:“陛下你此来是请我徒弟与你去除却那妖怪么?”那人道:“正是!正是!”三藏道:“我徒弟干别的事不济但说降妖捉怪正合他宜。陛下啊虽是着他拿怪但恐理上难行。”那人道:“怎么难行?”三藏道:“那怪既神通广大变得与你相同满朝文武一个个言和心顺;三宫妃嫔一个个意合情投。我徒弟纵有手段决不敢轻动干戈。倘被多官拿住说我们欺邦灭国问一款大逆之罪困陷城中却不是画虎刻鹄也?”那人道:“我朝中还有人哩。”三藏道:“却好!却好!想必是一代亲王侍长付何处镇守去了?”那人道:“不是。我本宫有个太子是我亲生的储君。”三藏道:“那太子想必被妖魔贬了?”那人道:“不曾他只在金銮殿上五凤楼中或与学士讲书或共全真登位。自此三年禁太子不入皇宫不能彀与娘娘相见。”三藏道:“此是何故?”那人道:“此是妖怪使下的计策只恐他母子相见闲中论出长短怕走了消息。故此两不会面他得永住常存也。”三藏道:“你的灾屯想应天付却与我相类。当时我父曾被水贼伤生我母被水贼欺占经三个月分娩了我。我在水中逃了性命幸金山寺恩师救养成*人。记得我幼年无父母此间那太子失双亲惭惶不已!”又问道:“你纵有太子在朝我怎的与他相见?”那人道:“如何不得见?”三藏道:

“他被妖魔拘辖连一个生身之母尚不得见我一个和尚欲见何由?”那人道:“他明早出朝来也。”三藏问:“出朝作甚?”那人道:“明日早朝领三千人马架鹰犬出城采猎师父断得与他相见。见时肯将我的言语说与他他便信了。”三藏道:“他本是肉眼凡胎被妖魔哄在殿上那一日不叫他几声父王?他怎肯信我的言语?”那人道:“既恐他不信我留下一件表记与你罢。”三藏问:“是何物件?”那人把手中执的金厢白玉圭放下道:“此物可以为记。”三藏道:“此物何如?”那人道:“全真自从变作我的模样只是少变了这件宝贝。他到宫中说那求雨的全真拐了此圭去了自此三年还没此物。我太子若看见他睹物思人此仇必报。”三藏道:“也罢等我留下着徒弟与你处置。却在那里等么?”那人道:“我也不敢等。我这去还央求夜游神再使一阵神风把我送进皇宫内院托一梦与我那正宫皇后教他母子们合意你师徒们同心。”三藏点头应承道:“你去罢。”

那冤魂叩头拜别举步相送不知怎么踢了脚跌了一个筋斗把三藏惊醒却原来是南柯一梦慌得对着那盏昏灯连忙叫:“徒弟!徒弟!”八戒醒来道:“甚么土地土地?当时我做好汉专一吃人度日受用腥膻其实快活偏你出家教我们保护你跑路!原说只做和尚如今拿做奴才日间挑包袱牵马夜间提尿瓶务脚!这早晚不睡又叫徒弟作甚?”三藏道:“徒弟我刚才伏在案上打盹做了一个怪梦。”行者跳将起来道:

“师父梦从想中来。你未曾上山先怕妖怪又愁雷音路远不能得到思念长安不知何日回程所以心多梦多。似老孙一点真心专要西方见佛更无一个梦儿到我。”三藏道:“徒弟我这桩梦不是思乡之梦。才然合眼见一阵狂风过处禅房门外有一朝皇帝自言是乌鸡国王浑身水湿满眼泪垂。”这等这等如此如此将那梦中话一一的说与行者。行者笑道:“不消说了他来托梦与你分明是照顾老孙一场生意。必然是个妖怪在那里篡位谋国等我与他辨个真假。想那妖魔棍到处立要成功。”三藏道:“徒弟他说那怪神通广大哩。”行者道:“怕他甚么广大!早知老孙到教他即走无方!”三藏道:“我又记得留下一件宝贝做表记。”八戒答道:“师父莫要胡缠做个梦便罢了怎么只管当真?”沙僧道:“不信直中直须防仁不仁。我们打起火开了门看看如何便是。”行者果然开门一齐看处只见星月光中阶檐上真个放着一柄金厢白玉圭。八戒近前拿起道:“哥哥这是甚么东西?”行者道:“这是国王手中执的宝贝名唤玉圭。师父啊既有此物想此事是真。明日拿妖全都在老孙身上只是要你三桩儿造化低哩。”八戒道:“好好好!

做个梦罢了又告诵他。他那些儿不会作弄人哩?就教你三桩儿造化低。”三藏回入里面道:“是那三桩?”行者道:“明日要你顶缸、受气、遭瘟。”八戒笑道:一桩儿也是难的三桩儿却怎么耽得?”唐僧是个聪明的长老便问:“徒弟啊此三事如何讲?”

行者道:“也不消讲等我先与你二件物。”

好大圣拔了一根毫毛吹口仙气叫声“变!”变做一个红金漆匣儿把白玉圭放在内盛着道:“师父你将此物捧在手中到天晓时穿上锦襕袈裟去正殿坐着念经等我去看看他那城池。端的是个妖怪就打杀他也在此间立个功绩;假若不是且休撞祸。”三藏道:“正是!正是!”行者道:“那太子不出城便罢若真个应梦出城来我定引他来见你。”三藏道:“见了我如何迎答?”行者道:“来到时我先报知你把那匣盖儿扯开些等我变作二寸长的一个小和尚钻在匣儿里你连我捧在手中。那太子进了寺来必然拜佛你尽他怎的下拜只是不睬他。他见你不动身一定教拿你你凭他拿下去打也由他绑也由他杀也由他。”三藏道:“呀!他的军令大真个杀了我怎么好?”行者道:“没事有我哩若到那紧关处我自然护你。他若问时你说是东土钦差上西天拜佛取经进宝的和尚。他道有甚宝贝?你却把锦襕袈裟对他说一遍说道:‘此是三等宝贝还有头一等、第二等的好物哩’。但问处就说这匣内有一件宝贝上知五百年下知五百年中知五百年共一千五百年过去未来之事俱尽晓得却把老孙放出来。我将那梦中话告诵那太子他若肯信就去拿了那妖魔一则与他父王报仇二来我们立个名节;他若不信再将白玉圭拿与他看。只恐他年幼还不认得哩。”三藏闻言大喜道:“徒弟啊此计绝妙!但说这宝贝一个叫做锦襕袈裟一个叫做白玉圭你变的宝贝却叫做甚名?”行者道:“就叫做立帝货罢。”三藏依言记在心上。师徒们一夜那曾得睡。盼到天明恨不得点头唤出扶桑日喷气吹散满天星。

不多时东方白。行者又吩咐了八戒、沙僧教他两个:

“不可搅扰僧人出来乱走。待我成功之后共汝等同行。”才别了唐僧打了唿哨一筋斗跳在空中睁火眼平西看处果见有一座城池。你道怎么就看见了?当时说那城池离寺只有四十里故此凭高就望见了。行者近前仔细看处又见那怪雾愁云漠漠妖风怨气纷纷。行者在空中赞叹道:“若是真王登宝座自有祥光五色云;只因妖怪侵龙位腾腾黑气锁金门。”行者正然感叹。忽听得炮声响喨又只见东门开处闪出一路人马真个是采猎之军果然势勇但见晓出禁城东分围浅草中。彩旗开映日白马骤迎风。鼍鼓冬冬擂标枪对对冲。架鹰军猛烈牵犬将骁雄。火炮连天振粘竿映日红。人人支弩箭个个挎雕弓。张网山坡下铺绳小径中。一声惊霹雳千骑拥貔熊。狡兔身难保乖獐智亦穷。狐狸该命尽麋鹿丧当终。山雉难飞脱野鸡怎避凶?他都要捡占山场擒猛兽摧残林木射飞虫。那些人出得城来散步东郊不多时有二十里向高田地又只见中军营里有小小的一个将军顶着盔贯着甲果肚花十八札手执青锋宝剑坐下黄骠马腰带满弦弓真个是隐隐君王象昂昂帝主容。规模非小辈行动显真龙。行者在空暗喜道:

“不须说那个就是皇帝的太子了。等我戏他一戏。”好大圣按落云头撞入军中太子马前摇身一变变作一个白兔儿只在太子马前乱跑。太子看见正合欢心拈起箭拽满弓一箭正中了那兔儿。原来是那大圣故意教他中了却眼乖手疾一把接住那箭头把箭翎花落在前边丢开脚步跑了。那太子见箭中了玉兔兜开马独自争先来赶。不知马行的快行者如风;

马行的迟行者慢走只在他面前不远。看他一程一程将太子哄到宝林寺山门之下行者现了本身不见兔儿只见一枝箭插在门槛上。径撞进去见唐僧道:“师父来了!来了!”却又一变变做二寸长短的小和尚儿钻在红匣之内。

却说那太子赶到山门前不见了白兔只见门槛上插住一枝雕翎箭。太子大惊失色道:“怪哉!怪哉!分明我箭中了玉兔玉兔怎么不见只见箭在此间!想是年多日久成了精魅也。”拔了箭抬头看处山门上有五个大字写着敕建宝林寺。

太子道:“我知之矣。向年间曾记得我父王在金銮殿上差官赍些金帛与这和尚修理佛殿佛象不期今日到此。正是因过道院逢僧话又得浮生半日闲我且进去走走。”

那太子跳下马来正要进去只见那保驾的官将与三千人马赶上簇簇拥拥都入山门里面。慌得那本寺众僧都来叩头拜接接入正殿中间参拜佛象。却才举目观瞻又欲游廊玩景忽见正当中坐着一个和尚太子大怒道:“这个和尚无礼!

我今半朝銮驾进山虽无旨意知会不当远接此时军马临门也该起身怎么还坐着不动?”教:“拿下来!”说声拿字两边校尉一齐下手把唐僧抓将下来急理绳索便捆。行者在匣里默默的念咒教道:“护法诸天、六丁六甲我今设法降妖这太子不能知识将绳要捆我师父汝等即早护持若真捆了汝等都该有罪!”那大圣暗中吩咐谁敢不遵却将三藏护持定了:有些人摸也摸不着他光头好似一壁墙挡住难拢其身。那太子道:“你是那方来的使这般隐身法欺我!”三藏上前施礼道:

“贫僧无隐身法乃是东土唐僧上雷音寺拜佛求经进宝的和尚。”太子道:“你那东土虽是中原其穷无比有甚宝贝你说来我听。”三藏道:“我身上穿的这袈裟是第三样宝贝。还有第一等、第二等更好的物哩!”太子道:“你那衣服半边苫身半边露臂能值多少物敢称宝贝!”三藏道:“这袈裟虽不全体有诗几句诗曰:佛衣偏袒不须论内隐真如脱世尘。万线千针成正果九珠八宝合元神。仙娥圣女恭修制遗赐禅僧静垢身。

见驾不迎犹自可你的父冤未报枉为人!”太子闻言心中大怒道:“这泼和尚胡说!你那半片衣凭着你口能舌便夸好夸强。

我的父冤从何未报你说来我听。”三藏进前一步合掌问道:

“殿下为人生在天地之间能有几恩?”太子道:“有四恩。”三藏道:“那四恩?”太子道:“感天地盖载之恩日月照临之恩国王水土之恩父母养育之恩。”三藏笑曰:“殿下言之有失人只有天地盖载日月照临国王水土那得个父母养育来?”太子怒道:“和尚是那游手游食削逆君之徒!人不得父母养育身从何来?”三藏道:“殿下贫僧不知。但只这红匣内有一件宝贝叫做立帝货他上知五百年中知五百年下知五百年共知一千五百年过去未来之事便知无父母养育之恩令贫僧在此久等多时矣。”

太子闻说教:“拿来我看。”三藏扯开匣盖儿那行者跳将出来呀呀的两边乱走。太子道:“这星星小人儿能知甚事?”行者闻言嫌小却就使个神通把腰伸一伸就长了有三尺四五寸。众军士吃惊道:“若是这般快长不消几日就撑破天也。”行者长到原身就不长了。太子才问道:“立帝货这老和尚说你能知未来过去吉凶你却有龟作卜?有蓍作筮?凭书句断人祸福?”行者道:“我一毫不用只是全凭三寸舌万事尽皆知。”太子道:“这厮又是胡说。自古以来《周易》之书极其玄妙断尽天下吉凶使人知所趋避故龟所以卜蓍所以筮。

听汝之言凭据何理妄言祸福扇惑人心!”行者道:“殿下且莫忙等我说与你听。你本是乌鸡国王的太子你那里五年前年程荒旱万民遭苦你家皇帝共臣子秉心祈祷。正无点雨之时锺南山来了一个道士他善呼风唤雨点石为金。君王忒也爱小就与他拜为兄弟。这桩事有么?”太子道:“有有有!你再说说。”行者道:“后三年不见全真称孤的却是谁?”太子道:

“果是有个全真父王与他拜为兄弟食则同食寝则同寝。三年前在御花园里玩景被他一阵神风把父王手中金厢白玉圭摄回锺南山去了至今父王还思慕他。因不见他遂无心赏玩把花园紧闭了已三年矣。做皇帝的非我父王而何?”行者闻言哂笑不绝。太子再问不答只是哂笑。太子怒道:“这厮当言不言如何这等哂笑?”行者又道:“还有许多话哩!奈何左右人众不是说处。”太子见他言语有因将袍袖一展教军士且退。那驾上官将急传令将三千人马都出门外住札。此时殿上无人太子坐在上面长老立在前边左手旁立着行者。本寺诸僧皆退行者才正色上前道:“殿下化风去的是你生身之父母见坐位的是那祈雨之全真。”太子道:“胡说!胡说!我父自全真去后风调雨顺国泰民安。照依你说就不是我父王了。还是我年孺容得你;若我父王听见你这番话拿了去碎尸万段!”把行者咄的喝下来。行者对唐僧道:“何如?我说他不信果然!果然!如今却拿那宝贝进与他倒换关文往西方去罢。”三藏即将红匣子递与行者。行者接过来将身一抖那匣儿卒不见了原是他毫毛变的被他收上身去。却将白玉圭双手捧上献与太子。

太子见了道:“好和尚!好和尚!你五年前本是个全真来骗了我家的宝贝如今又妆做和尚来进献!”叫:“拿了!”一声传令把长老唬得慌忙指着行者道:“你这弼马温!专撞空头祸带累我哩!”行者近前一齐拦住道:“休嚷!莫走了风!我不叫做立帝货还有真名哩。”太子怒道:“你上来!我问你个真名字好送法司定罪!”行者道:“我是那长老的大徒弟名唤悟空孙行者因与我师父上西天取经昨宵到此觅宿。我师父夜读经卷至三更时分得一梦梦见你父王道他被那全真欺害推在御花园八角琉璃井内全真变作他的模样。满朝官不能知你年幼亦无分晓禁你入宫关了花园大端怕漏了消息。你父王今夜特来请我降魔我恐不是妖邪自空中看了果然是个妖精。正要动手拿他不期你出城打猎。你箭中的玉兔就是老孙。老孙把你引到寺里见师父诉此衷肠句句是实。你既然认得白玉圭怎么不念鞠养恩情替亲报仇?”那太子闻言心中惨慽暗自伤愁道:“若不信此言语他却有三分儿真实;

若信了怎奈殿上见是我父王?”这才是进退两难心问口三思忍耐口问心。行者见他疑惑不定又上前道:“殿下不必心疑请殿下驾回本国问你国母娘娘一声看他夫妻恩爱之情比三年前如何。只此一问便知真假矣。”那太子回心道:“正是!

且待我问我母亲去来。”他跳起身笼了玉圭就走。行者扯住道:“你这些人马都回却不走漏消息我难成功?但要你单人独马进城不可扬名卖弄莫入正阳门须从后宰门进去。到宫中见你母亲切休高声大气须是悄语低言。恐那怪神通广大一时走了消息你娘儿们性命俱难保也。”太子谨遵教命出山门吩咐将官:“稳在此札营不得移动。我有一事待我去了就来一同进城。”看他:指挥号令屯军士上马如飞即转城。这一去不知见了娘娘有何话说且听下回分解。
------------

第三十八回 婴儿问母知邪正 金木参玄见假真

逢君只说受生因便作如来会上人。一念静观尘世佛十方同看降威神。欲知今日真明主须问当年嫡母身。别有世间曾未见一行一步一花新。却说那乌鸡国王太子自别大圣不多时回至城中果然不奔朝门不敢报传宣诏径至后宰门见几个太监在那里把守。见太子来不敢阻滞让他进去了。好太子夹一夹马撞入里面忽至锦香亭下只见那正宫娘娘坐在锦香亭上两边有数十个嫔妃掌扇那娘娘倚雕栏儿流泪哩。你道他流泪怎的?原来他四更时也做了一梦记得一半含糊了一半沉沉思想。这太子下马跪于亭下叫:“母亲!”那娘娘强整欢容叫声“孩儿喜呀!喜呀!这二三年在前殿与你父王开讲不得相见我甚思量今日如何得暇来看我一面?诚万千之喜!诚万千之喜!孩儿你怎么声音悲惨?你父王年纪高迈有一日龙归碧海凤返丹霄你就传了帝位还有甚么不悦?”太子叩头道:“母亲我问你:即位登龙是那个?称孤道寡果何人?”娘娘闻言道:“这孩儿风了!做皇帝的是你父王你问怎的?”太子叩头道:“万望母亲敕子无罪敢问;不敕不敢问。”娘娘道:“子母家有何罪?敕你敕你快快说来。”太子道:

“母亲我问你三年前夫妻宫里之事与后三年恩爱同否如何?”娘娘见说魂飘魄散急下亭抱起紧搂在怀眼中滴泪道:“孩儿!我与你久不相见怎么今日来宫问此?”太子怒道:“母亲有话早说不说时且误了大事。”娘娘才喝退左右泪眼低声道:“这桩事孩儿不问我到九泉之下也不得明白。

既问时听我说:三载之前温又暖三年之后冷如冰。枕边切切将言问他说老迈身衰事不兴!”太子闻言撒手脱身攀鞍上马。那娘娘一把扯住道:“孩儿你有甚事话不终就走?”太子跪在面前道:“母亲不敢说!今日早期蒙钦差架鹰逐犬出城打猎偶遇东土驾下来的个取经圣僧有大徒弟乃孙行者极善降妖。原来我父王死在御花园八角琉璃井内这全真假变父王侵了龙位。今夜三更父王托梦请他到城捉怪。孩儿不敢尽信特来问母母亲才说出这等言语必然是个妖精。”那娘娘道:“儿啊外人之言你怎么就信为实?”太子道:“儿还不敢认实父王遗下表记与他了。”娘娘问是何物太子袖中取出那金厢白玉圭递与娘娘。那娘娘认得是当时国王之宝止不住泪如泉涌叫声:“主公!你怎么死去三年不来见我却先见圣僧后来见我?”太子道:“母亲这话是怎的说?”娘娘道:“儿啊我四更时分也做了一梦梦见你父王水淋淋的站在我跟前亲说他死了鬼魂儿拜请了唐僧降假皇帝救他前身。记便记得是这等言语只是一半儿不得分明正在这里狐疑怎知今日你又来说这话又将宝贝拿出。我且收下你且去请那圣僧急急为之。果然扫荡妖氛辨明邪正庶报你父王养育之恩也。”

太子急忙上马出后宰门躲离城池真个是噙泪叩头辞国母含悲顿复唐僧。不多时出了城门径至宝林寺山门前下马。众军士接着太子又见红轮将坠。太子传令不许军士乱动他又独自个入了山门整束衣冠拜请行者。只见那猴王从正殿摇摇摆摆走来那太子双膝跪下道:“师父我来了。”行者上前搀住道:“请起你到城中可曾问谁么?”太子道:“问母亲来。”将前言尽说了一遍。行者微微笑道:“若是那般冷啊想是个甚么冰冷的东西变的。不打紧!不打紧!等我老孙与你扫荡。却只是今日晚了不好行事。你先回去待明早我来。”

太子跪地叩拜道:“师父我只在此伺候到明日同师父一路去罢。”行者道:“不好!不好!若是与你一同入城那怪物生疑不说是我撞着你却说是你请老孙却不惹他反怪你也?”太子道:“我如今进城他也怪我。”行者道:“怪你怎么?”太子道:

“我自早朝蒙差带领若干人马鹰犬出城今一日更无一件野物怎么见驾?若问我个不才之罪监陷羑里你明日进城却将何倚?况那班部中更没个相知人也。”行者道:“这甚打紧!你肯早说时却不寻下些等你?”

好大圣!你看他就在太子面前显个手段将身一纵跳在云端里捻着诀念一声“唵蓝净法界”的真言拘得那山神土地在半空中施礼道:“大圣呼唤小神有何使令?”行者道:“老孙保护唐僧到此欲拿邪魔奈何那太子打猎无物不敢回朝。

问汝等讨个人情快将獐鹿兔走兽飞禽各寻些来打他回去。”山神土地闻言敢不承命?又问各要几何。大圣道:“不拘多少取些来便罢。”那各神即着本处阴兵刮一阵聚兽阴风捉了些野鸡山雉角鹿肥獐狐獾狢兔虎豹狼虫共有百千余只献与行者。行者道:“老孙不要你可把他都捻就了筋单摆在那四十里路上两旁教那些人不纵鹰犬拿回城去算了汝等之功。”众神依言散了阴风摆在左右。行者才按云头对太子道:“殿下请回路上已有物了你自收去。”太子见他在半空中弄此神通如何不信只得叩头拜别出山门传了令教军士们回城。只见那路旁果有无限的野物军士们不放鹰犬一个个俱着手擒捉喝采俱道是千岁殿下的洪福怎知是老孙的神功?你听凯歌声唱一拥回城。

这行者保护了三藏那本寺中的和尚见他们与太子这样绸缪怎不恭敬?却又安排斋供管待了唐僧依然还歇在禅堂里。将近有一更时分行者心中有事急睡不着。他一毂辘爬起来到唐僧床前叫:“师父。”此时长老还未睡哩他晓得行者会失惊打怪的推睡不应。行者摸着他的光头乱摇道:“师父怎睡着了?”唐僧怒道:“这个顽皮!这早晚还不睡吆喝甚么?”

行者道:“师父有一桩事儿和你计较计较。”长老道:“甚么事?”行者道:“我日间与那太子夸口说我的手段比山还高比海还深拿那妖精如探囊取物一般伸了手去就拿将转来却也睡不着想起来有些难哩。”唐僧道:“你说难便就不拿了罢。”行者道:“拿是还要拿只是理上不顺。”唐僧道:“这猴头乱说!妖精夺了人君位怎么叫做理上不顺!”行者道:“你老人家只知念经拜佛打坐参禅那曾见那萧何的律法?常言道拿贼拿赃。那怪物做了三年皇帝又不曾走了马脚漏了风声。他与三宫妃后同眠又和两班文武共乐我老孙就有本事拿住他也不好定个罪名。”唐僧道:“怎么不好定罪?”行者道:“他就是个没嘴的葫芦也与你滚上几滚。他敢道:我是乌鸡国王有甚逆天之事你来拿我?将甚执照与他折辩?”唐僧道:“凭你怎生裁处?”行者笑道:“老孙的计已成了只是干碍着你老人家有些儿护短。”唐僧道:“我怎么护短?”行者道:“八戒生得夯你有些儿偏向他。”唐僧道:“我怎么向他?”行者道:“你若不向他啊且如今把胆放大些与沙僧只在这里。待老孙与八戒趁此时先入那乌鸡国城中寻着御花园打开琉璃井把那皇帝尸捞将上来包在我们包袱里。明日进城且不管甚么倒换文牒见了那怪掣棍子就打。他但有言语就将骨榇与他看说你杀的是这个人!却教太子上来哭父皇后出来认夫文武多官见主我老孙与兄弟们动手。这才是有对头的官事好打。”唐僧闻言暗喜道:“只怕八戒不肯去。”行者笑道:“如何?

我说你护短你怎么就知他不肯去?你只象我叫你时不答应半个时辰便了!我这去但凭三寸不烂之舌莫说是猪八戒就是猪九戒也有本事教他跟着我走。”唐僧道:“也罢随你去叫他。”

行者离了师父径到八戒床边叫:“八戒!八戒!”那呆子是走路辛苦的人丢倒头只情打呼那里叫得醒?行者揪着耳朵抓着鬃把他一拉拉起来叫声“八戒。(WWW.mianhuatang.la 好看的小说)”那呆子还打棱挣行者又叫一声呆子道:“睡了罢莫顽!明日要走路哩!”行者道:“不是顽有一桩买卖我和你做去。”八戒道:“甚么买卖?”

行者道:“你可曾听得那太子说么?”八戒道:“我不曾见面不曾听见说甚么。”行者说:“那太子告诵我说那妖精有件宝贝万夫不当之勇。我们明日进朝不免与他争敌倘那怪执了宝贝降倒我们却不反成不美我想着打人不过不如先下手。

我和你去偷他的来却不是好?”八戒道:“哥哥你哄我去做贼哩。这个买卖我也去得果是晓得实实的帮寸我也与你讲个明白:偷了宝贝降了妖精我却不奈烦甚么小家罕气的分宝贝我就要了。”行者道:“你要作甚?”八戒道:“我不如你们乖巧能言人面前化得出斋来老猪身子又夯言语又粗不能念经若到那无济无生处可好换斋吃么!”行者道:“老孙只要图名那里图甚宝贝就与你罢便了。”那呆子听见说都与他他就满心欢喜一毂辘爬将起来套上衣服就和行者走路。这正是清酒红人面黄金动道心。两个密密的开了门躲离三藏纵祥光径奔那城。

不多时到了按落云头只听得楼头方二鼓矣。行者道:

“兄弟二更时分了。”八戒道:“正好!正好!人都在头觉里正浓睡也。”二人不奔正阳门径到后宰门只听得梆铃声响。

行者道:“兄弟前后门皆紧急如何得入?”八戒道:“那见做贼的从门里走么?瞒墙跳过便罢。”行者依言将身一纵跳上里罗城墙八戒也跳上去。二人潜入里面找着门路径寻那御花园。正行时只见有一座三檐白簇的门楼上有三个亮灼灼的大字映着那星月光辉乃是御花园。行者近前看了有几重封皮公然将锁门锈住了即命八戒动手。那呆子掣铁钯尽力一筑把门筑得粉碎。行者先举步插入忍不住跳将起来大呼小叫唬得八戒上前扯住道:“哥呀害杀我也!那见做贼的乱嚷似这般吆喝!惊醒了人把我们拿住到官司就不该死罪也要解回原籍充军。”行者道:“兄弟啊你却不知我急为何你看这:“彩画雕栏狼狈宝妆亭阁敧歪。莎汀蓼岸尽尘埋芍药荼蘼俱败。茉莉玫瑰香暗牡丹百合空开。芙蓉木槿草垓垓异卉奇葩壅坏。巧石山峰俱倒池塘水涸鱼衰。青松紫竹似干柴满路茸茸蒿艾。丹桂碧桃枝损海榴棠棣根歪。桥头曲径有苍苔冷落花园境界!”八戒道:“且叹他做甚?快干我们的买卖去来!”行者虽然感慨却留心想起唐僧的梦来说芭蕉树下方是井。正行处果见一株芭蕉生得茂盛比众花木不同真是:一种灵苗秀天生体性空。枝枝抽片纸叶叶卷芳丛。翠缕千条细丹心一点红。凄凉愁夜雨憔悴怯秋风。长养元丁力栽培造化工。缄书成妙用挥洒有奇功。凤翎宁得似鸾尾迥相同。薄露瀼瀼滴轻烟淡淡笼。青阴遮户牖碧影上帘栊。不许栖鸿雁何堪系玉骢。霜天形槁悴月夜色朦胧。仅可消炎暑犹宜避日烘。愧无桃李色冷落粉墙东。行者道:“八戒动手么!宝贝在芭蕉树下埋着哩。”那呆子双手举钯筑倒了芭蕉然后用嘴一拱拱了有三四尺深见一块石板盖住。呆子欢喜道:“哥呀!造化了!果有宝贝是一片石板盖着哩!不知是坛儿盛着是柜儿装着哩。”行者道:“你掀起来看看。”那呆子果又一嘴拱开看处又见有霞光灼灼白气明明。八戒笑道:

“造化!造化!宝贝放光哩!”又近前细看时呀!原来是星月之光映得那井中水亮。八戒道:“哥呀你但干事便要留根。”

行者道:“我怎留根?”八戒道:“这是一眼井。你在寺里早说是井中有宝贝我却带将两条捆包袱的绳来怎么作个法儿把老猪放下去。如今空手这里面东西怎么得下去上来耶?”行者道:“你下去么?”八戒道:“正是要下去只是没绳索。”行者笑道:“你脱了衣服我与你个手段。”八戒道:“有甚么好衣服?

解了这直裰子就是了。”

好大圣把金箍棒拿出来两头一扯叫“长!”足有七八丈长。教:“八戒你抱着一头儿把你放下井去。”八戒道:“哥呀放便放下去若到水边就住了罢。”行者道:“我晓得。”那呆子抱着铁棒被行者轻轻提将起来将他放下去。不多时放至水边八戒道:“到水了!”行者听见他说却将棒往下一按。那呆子扑通的一个没头蹲丢了铁棒便就负水口里哺哺的嚷道:

“这天杀的!我说到水莫放他却就把我一按!”行者擎上棒来笑道:“兄弟可有宝贝么?”八戒道:“见甚么宝贝只是一井水!”行者道:“宝贝沉在水底下哩你下去摸一摸来。”呆子真个深知水性却就打个猛子淬将下去呀!那井底深得紧!他却着实又一淬忽睁眼见有一座牌楼上有水晶宫三个字。八戒大惊道:“罢了!罢了!错走了路了!蹡下海来也!海内有个水晶宫井里如何有之?”原来八戒不知此是井龙王的水晶宫。

八戒正叙话处早有一个巡水的夜叉开了门看见他的模样急抽身进去报道:“大王祸事了!井上落一个长嘴大耳的和尚来了!赤淋淋的衣服全无还不死逼法说话哩。”那井龙王忽闻此言心中大惊道:“这是天蓬元帅来也。昨夜夜游神奉上敕旨来取乌鸡国王魂灵去拜见唐僧请齐天大圣降妖。

这怕是齐天大圣、天蓬元帅来了却不可怠慢他快接他去也。”那龙王整衣冠领众水族出门来厉声高叫道:“天蓬元帅请里面坐。”八戒却才欢喜道:“原来是个故知。”那呆子不管好歹径入水晶宫里。其实不知上下赤淋淋的就坐在上面。龙王道:“元帅近闻你得了性命皈依释教保唐僧西天取经如何得到此处?”八戒道:“正为此说我师兄孙悟空多多拜上着我来问你取甚么宝贝哩。”龙王道:“可怜我这里怎么得个宝贝?比不得那江河淮济的龙王飞腾变化便有宝贝。我久困于此日月且不能长见宝贝果何自而来也?”八戒道:“不要推辞有便拿出来罢。”龙王道:“有便有一件宝贝只是拿不出来就元帅亲自来看看何如?”八戒道:“妙妙妙!须是看看来也。”那龙王前走这呆子随后转过了水晶宫殿只见廊庑下横軃着一个六尺长躯。龙王用手指定道:“元帅那厢就是宝贝了。”八戒上前看了呀!原来是个死皇帝戴着冲天冠穿着赭黄袍踏着无忧履系着蓝田带直挺挺睡在那厢。八戒笑道:“难难难!算不得宝贝!想老猪在山为怪时时常将此物当饭且莫说见的多少吃也吃够无数那里叫做甚么宝贝!”龙王道:“元帅原来不知他本是乌鸡国王的尸自到井中我与他定颜珠定住不曾得坏。你若肯驮他出去见了齐天大圣假有起死回生之意啊莫说宝贝凭你要甚么东西都有。”八戒道:“既这等说我与你驮出去只说把多少烧埋钱与我?”龙王道“其实无钱。”八戒道:“你好白使人?果然没钱不驮!”龙王道:“不驮请行。”八戒就走。龙王差两个有力量的夜叉把尸抬将出去送到水晶宫门外丢在那厢摘了辟水珠就有水响。

八戒急回头看不见水晶宫门一把摸着那皇帝的尸慌得他脚软筋麻撺出水面扳着井墙叫道:“师兄!伸下棒来救我一救!”行者道:“可有宝贝么?”八戒道:“那里有!只是水底下有一个井龙王教我驮死人我不曾驮他就把我送出门来就不见那水晶宫了只摸着那个尸唬得我手软筋麻挣搓不动了!哥呀!好歹救我救儿!”行者道:“那个就是宝贝如何不驮上来?”八戒道:“知他死了多少时了我驮他怎的?”行者道:“你不驮我回去耶。”八戒道:“你回那里去?”行者道:

“我回寺中同师父睡觉去。”八戒道:“我就不去了?”行者道:

“你爬得上来便带你去爬不上来便罢。”八戒慌了:“怎生爬得动!你想城墙也难上这井肚子大口儿小壁陡的圈墙又是几年不曾打水的井团团都长的是苔痕好不滑也教我怎爬?哥哥不要失了兄弟们和气等我驮上来罢。”行者道:“正是快快驮上来我同你回去睡觉。”那呆子又一个猛子淬将下去摸着尸拽过来背在身上撺出水面扶井墙道:“哥哥驮上来了。”那行者睁睛看处真个的背在身上却才把金箍棒伸下井底那呆子着了恼的人张开口咬着铁棒被行者轻轻的提将出来。八戒将尸放下捞过衣服穿了。行者看时那皇帝容颜依旧似生时未改分毫。行者道:“兄弟啊这人死了三年怎么还容颜不坏?”八戒道:“你不知之这井龙王对我说他使了定颜珠定住了尸未曾坏得。”行者道:“造化!造化!一则是他的冤仇未报二来该我们成功兄弟快把他驮了去。”八戒道:“驮往那里去?”行者道:“驮了去见师父。”八戒口中作念道:“怎的起!怎的起!好好睡觉的人被这猢狲花言巧语哄我教做甚么买卖如今却干这等事教我驮死人!驮着他腌脏臭水淋将下来污了衣服没人与我浆洗。上面有几个补丁天阴潮如何穿么?”行者道:“你只管驮了去到寺里我与你换衣服。”八戒道:“不羞!连你穿的也没有又替我换!”

行者道:“这般弄嘴便不驮罢!”八戒道:“不驮!”“便伸过孤拐来打二十棒!”八戒慌了道:“哥哥那棒子重若是打上二十我与这皇帝一般了。”行者道:“怕打时趁早儿驮着走路!”八戒果然怕打没好气把尸拽将过来背在身上拽步出园就走。

好大圣捻着诀念声咒语往巽地上吸一口气吹将去就是一阵狂风把八戒撮出皇宫内院躲离了城池息了风头二人落地徐徐却走将来。那呆子心中暗恼算计要报恨行者道:

“这猴子捉弄我我到寺里也捉弄他捉弄撺唆师父只说他医得活;医不活教师父念《紧箍儿咒》把这猴子的脑浆勒出来方趁我心!”走着路再再寻思道:“不好!不好!若教他医人却是容易:他去阎王家讨将魂灵儿来就医活了。只说不许赴阴司阳世间就能医活这法儿才好。”说不了却到了山门前径直进去将尸丢在那禅堂门前道:“师父起来看邪。”那唐僧睡不着正与沙僧讲行者哄了八戒去久不回之事忽听得他来叫了一声唐僧连忙起身道:“徒弟看甚么?”八戒道:“行者的外公教老猪驮将来了。”行者道:“你这馕糟的呆子!我那里有甚么外公?”八戒道:“哥不是你外公却教老猪驮他来怎么?也不知费了多少力了!”那唐僧与沙僧开门看处那皇帝容颜未改似活的一般。长老忽然惨凄道:“陛下你不知那世里冤家今生遇着他暗丧其身抛妻别子致令文武不知多官不晓!可怜你妻子昏蒙谁曾见焚香献茶?”忽失声泪如雨下。

八戒笑道:“师父他死了可干你事?又不是你家父祖哭他怎的!”三藏道:“徒弟啊出家人慈悲为本方便为门你怎的这等心硬?”八戒道:“不是心硬师兄和我说来他能医得活。若是医不活我也不驮他来了。”那长老原来是一头水的被那呆子摇动了也便就叫:“悟空若果有手段医活这个皇帝正是救人一命胜造七级浮图我等也强似灵山拜佛。”行者道:“师父你怎么信这呆子乱谈!人若死了或三七五七尽七七日受满了阳间罪过就转生去了如今已死三年如何救得!”三藏闻其言道:“也罢了。”八戒苦恨不息道:“师父你莫被他瞒了他有些夹脑风。你只念念那话儿管他还你一个活人。”真个唐僧就念《紧箍儿咒》勒得那猴子眼胀头疼。毕竟不知怎生医救且听下回分解。
------------

第三十九回 一粒金丹天上得 三年故主世间生

话说那孙大圣头痛难禁哀告道:“师父莫念!莫念!等我医罢!”长老问:“怎么医?”行者道:“只除过阴司查勘那个阎王家有他魂灵请将来救他。”八戒道:“师父莫信他。他原说不用过阴司阳世间就能医活方见手段哩。”那长老信邪风又念《紧箍儿咒》慌得行者满口招承道:“阳世间医罢!阳世间医罢!”八戒道:“莫要住!只管念!只管念!”行者骂道:“你这呆孽畜撺道师父咒我哩!”八戒笑得打跌道:“哥耶!哥耶!你只晓得捉弄我不晓得我也捉弄你捉弄!”行者道:“师父莫念!莫念!待老孙阳世间医罢。”三藏道:“阳世间怎么医?”行者道:“我如今一筋斗云撞入南天门里不进斗牛宫不入灵霄殿径到那三十三天之上离恨天宫兜率院内见太上老君把他九转还魂丹求得一粒来管取救活他也。”三藏闻言大喜道:“就去快来。”行者道:“如今有三更时候罢了投到回来好天明了。只是这个人睡在这里冷淡冷淡不象个模样;须得举哀人看着他哭便才好哩。”八戒道:“不消讲这猴子一定是要我哭哩。”行者道:“怕你不哭!你若不哭我也医不成!”八戒道:“哥哥你自去我自哭罢了。”行者道:“哭有几样:若干着口喊谓之嚎扭搜出些眼泪儿来谓之啕。又要哭得有眼泪又要哭得有心肠才算着嚎啕痛哭哩。”八戒道:“我且哭个样子你看看。”他不知那里扯个纸条拈作一个纸拈儿往鼻孔里通了两通打了几个涕喷你看他眼泪汪汪粘涎答答的哭将起来口里不住的絮絮叨叨数黄道黑真个象死了人的一般。哭到那伤情之处唐长老也泪滴心酸。行者笑道:“正是那样哀痛再不许住声。你这呆子哄得我去了你就不哭我还听哩!

若是这等哭便罢若略住住声儿定打二十个孤拐!”八戒笑道:“你去你去!我这一哭动头有两日哭哩。”沙僧见他数落便去寻几枝香来烧献行者笑道:“好好好!一家儿都有些敬意老孙才好用功。”

好大圣此时有半夜时分别了他师徒三众纵筋斗云只入南天门里果然也不谒灵霄宝殿不上那斗牛天宫一路云光径来到三十三天离恨天兜率宫中。才入门只见那太上老君正坐在那丹房中与众仙童执芭蕉扇扇火炼丹哩。他见行者来时即吩咐看丹的童儿:“各要仔细偷丹的贼又来也。”行者作礼笑道:“老官儿这等没搭撒防备我怎的?我如今不干那样事了。”老君道:“你那猴子五百年前大闹天宫把我灵丹偷吃无数着小圣二郎捉拿上界送在我丹炉炼了四十九日炭也不知费了多少。你如今幸得脱身皈依佛果保唐僧往西天取经前者在平顶山上降魔弄刁难不与我宝贝今日又来做甚?”行者道:“前日事老孙更没稽迟将你那五件宝贝当时交还你反疑心怪我?”老君道:“你不走路潜入吾宫怎的?”行者道:“自别后西过一方名乌鸡国。那国王被一妖精假妆道士呼风唤雨阴害了国王那妖假变国王相貌现坐金銮殿上。是我师父夜坐宝林寺看经那国王鬼魂参拜我师敦请老孙与他降妖辨明邪正。正是老孙思无指实与弟八戒夜入园中打破花园寻着埋藏之所乃是一眼八角琉璃井内捞上他的尸容颜不改。到寺中见了我师他慈悲着老孙医救不许去赴阴司里求索灵魂只教在阳世间救治。我想着无处回生特来参谒万望道祖垂怜把九转还魂丹借得一千丸儿与我老孙搭救他也。”老君道:“这猴子胡说!甚么一千丸二千丸!

当饭吃哩!是那里土块捘的这等容易?咄!快去!没有!”行者笑道:“百十丸儿也罢。”老君道:“也没有。”行者道:“十来丸也罢。”老君怒道:“这泼猴却也缠帐!没有没有!出去出去!”

行者笑道:“真个没有我问别处去救罢。”老君喝道:“去!去!

去!”这大圣拽转步往前就走。老君忽的寻思道:“这猴子惫懒哩说去就去只怕溜进来就偷。”即命仙童叫回来道:“你这猴子手脚不稳我把这还魂丹送你一丸罢。”行者道:“老官儿既然晓得老孙的手段快把金丹拿出来与我四六分分还是你的造化哩;不然就送你个皮笊篱一捞个罄尽。”那老祖取过葫芦来倒吊过底子倾出一粒金丹递与行者道:“止有此了拿去拿去!送你这一粒医活那皇帝只算你的功果罢。”

行者接了道:“且休忙等我尝尝看只怕是假的莫被他哄了。”扑的往口里一丢慌得那老祖上前扯住一把揪着顶瓜皮揝着拳头骂道:“这泼猴若要咽下去就直打杀了!”行者笑道:“嘴脸!小家子样!那个吃你的哩!能值几个钱?虚多实少的在这里不是?”原来那猴子颏下有嗉袋儿他把那金丹噙在嗉袋里被老祖捻着道:“去罢!去罢!再休来此缠绕!”这大圣才谢了老祖出离了兜率天宫。

你看他千条瑞霭离瑶阙万道祥云降世尘须臾间下了南天门回到东观早见那太阳星上。按云头径至宝林寺山门外只听得八戒还哭哩忽近前叫声:“师父。”三藏喜道:“悟空来了可有丹药?”行者道:“有。”八戒道:“怎么得没有?他偷也去偷人家些来!”行者笑道:“兄弟你过去罢用不着你了。你揩揩眼泪别处哭去。”教:“沙和尚取些水来我用。”沙僧急忙往后面井上有个方便吊桶即将半钵盂水递与行者。行者接了水口中吐出丹来安在那皇帝唇里两手扳开牙齿用一口清水把金丹冲灌下肚。有半个时辰只听他肚里呼呼的乱响只是身体不能转移。行者道:“师父弄我金丹也不能救活可是掯杀老孙么!”三藏道:“岂有不活之理。似这般久死之尸如何吞得水下?此乃金丹之仙力也。自金丹入腹却就肠鸣了肠鸣乃血脉和动但气绝不能回伸。莫说人在井里浸了三年就是生铁也上锈了只是元气尽绝得个人度他一口气便好。”

那八戒上前就要度气三藏一把扯住道:“使不得!还教悟空来。”那师父甚有主张:原来猪八戒自幼儿伤生作孽吃人是一口浊气;惟行者从小修持咬松嚼柏吃桃果为生是一口清气。这大圣上前把个雷公嘴噙着那皇帝口唇呼的一口气收入咽喉度下重楼转明堂径至丹田从涌泉倒返泥垣宫。呼的一声响喨那君王气聚神归便翻身轮拳曲足叫了一声“师父!”双膝跪在尘埃道:“记得昨夜鬼魂拜谒怎知道今朝天晓返阳神!”三藏慌忙搀起道:“陛下不干我事你且谢我徒弟。”行者笑道:“师父说那里话?常言道家无二主你受他一拜儿不亏。”三藏甚不过意搀起那皇帝来同入禅堂又与八戒、行者、沙僧拜见了方才按座。只见那本寺的僧人整顿了早斋却欲来奉献;忽见那个水衣皇帝个个惊张人人疑说。

孙行者跳出来道:“那和尚不要这等惊疑这本是乌鸡国王乃汝之真主也。三年前被怪害了性命是老孙今夜救活如今进他城去要辨明邪正。若有了斋摆将来等我们吃了走路。”

众僧即奉献汤水与他洗了面换了衣服。把那皇帝赭黄袍脱了本寺僧官将两领布直裰与他穿了;解下蓝田带将一条黄丝绦子与他系了;褪下无忧履与他一双旧僧鞋撒了。却才都吃了早斋扣背马匹。

行者问:“八戒你行李有多重?”八戒道:“哥哥这行李日逐挑着倒也不知有多重。”行者道:“你把那一担儿分为两担将一担儿你挑着将一担儿与这皇帝挑我们赶早进城干事。”

八戒欢喜道:“造化!造化!当时驮他来不知费了多少力如今医活了原来是个替身。”那呆子就弄玄虚将行李分开就问寺中取条匾担轻些的自己挑了重些的教那皇帝挑着。行者笑道:“陛下着你那般打扮挑着担子跟我们走走可亏你么?”那国王慌忙跪下道:“师父你是我重生父母一般莫说挑担情愿执鞭坠镫伏侍老爷同行上西天去也。”行者道:“不要你去西天我内中有个缘故。你只挑得四十里进城待捉了妖精你还做你的皇帝我们还取我们的经也。”八戒听言道:

“这等说他只挑四十里路我老猪还是长工!”行者道:“兄弟不要胡说趁早外边引路。”真个八戒领那皇帝前行沙僧伏侍师父上马行者随后只见那本寺五百僧人齐齐整整吹打着细乐都送出山门之外。行者笑道:“和尚们不必远送但恐官家有人知觉泄漏我的事机反为不美。快回去!快回去!但把那皇帝的衣服冠带整顿干净或是今晚明早送进城来我讨些封赡赏赐谢你。”众僧依命各回讫。行者搀开大步赶上师父一直前来正是:西方有诀好寻真金木和同却炼神。丹母空怀懞懂梦婴儿长恨杌樗身。必须井底求明主还要天堂拜老君。悟得色空还本性诚为佛度有缘人。

师徒们在路上那消半日早望见城池相近三藏道:“悟空前面想是乌鸡国了。”行者道:“正是我们快赶进城干事。”

那师徒进得城来只见街市上人物齐整风光闹热早又见凤阁龙楼十分壮丽。有诗为证诗曰:海外宫楼如上邦人间歌舞若前唐。花迎宝扇红云绕日照鲜袍翠雾光。孔雀屏开香霭出珍珠帘卷彩旗张。太平景象真堪贺静列多官没奏章。三藏下马道:“徒弟啊我们就此进朝倒换关文省得又拢那个衙门费事。”行者道:“说得有理我兄弟们都进去人多才好说话。”唐僧道:“都进去莫要撒村先行了君臣礼然后再讲。”

行者道:“行君臣礼就要下拜哩。”三藏道:“正是要行五拜三叩头的大礼。”行者笑道:“师父不济若是对他行礼诚为不智。你且让我先走到里边自有处置。等他若有言语让我对答。我若拜你们也拜;我若蹲你们也蹲。”你看那惹祸的猴王引至朝门与阁门大使言道:“我等是东土大唐驾下差来上西天拜佛求经者今到此倒换关文烦大人转达是谓不误善果。”那黄门官即入端门跪下丹墀启奏道:“朝门外有五众僧人言是东土唐国钦差上西天拜佛求经今至此倒换关文不敢擅入现在门外听宣。”

那魔王即令传宣。唐僧却同入朝门里面那回生的国主随行。正行忍不住腮边堕泪心中暗道:“可怜!我的铜斗儿江山铁围的社稷谁知被他阴占了!”行者道:“陛下切莫伤感恐走漏消息。这棍子在我耳朵里跳哩如今决要见功管取打杀妖魔扫荡邪物这江山不久就还归你也。”那君王不敢违言只得扯衣揩泪舍死相生径来到金銮殿下。又见那两班文武四百朝官一个个威严端肃像貌轩昂。这行者引唐僧站立在白玉阶前挺身不动那阶下众官无不悚惧道:“这和尚十分愚浊!怎么见我王便不下拜亦不开言呼祝?喏也不唱一个好大胆无礼!”说不了只听得那魔王开口问道:“那和尚是那方来的?”行者昂然答道:“我是南赡部洲东土大唐国奉钦差前往西域天竺国大雷音寺拜活佛求真经者今到此方不敢空度特来倒换通关文牒。”那魔王闻说心中作怒道:“你东土便怎么!我不在你朝进贡不与你国相通你怎么见吾抗礼不行参拜!”行者笑道:“我东土古立天朝久称上国汝等乃下土边邦。自古道上邦皇帝为父为君;下邦皇帝为臣为子。你倒未曾接我且敢争我不拜?”那魔王大怒教文武官:“拿下这野和尚去!”说声叫“拿”你看那多官一齐踊跃。这行者喝了一声用手一指教:“莫来!”那一指就使个定身法众官俱莫能行动真个是校尉阶前如木偶将军殿上似泥人。

那魔王见他定住了文武多官急纵身跳下龙床就要来拿。猴王暗喜道:“好!正合老孙之意这一来就是个生铁铸的头汤着棍子也打个窟窿!”正动身不期旁边转出一个救命星来。你道是谁原来是乌鸡国王的太子急上前扯住那魔王的朝服跪在面前道:“父王息怒。”妖精问:“孩儿怎么说?”太子道:“启父王得知三年前闻得人说有个东土唐朝驾下钦差圣僧往西天拜佛求经不期今日才来到我邦。父王尊性威烈若将这和尚拿去斩只恐大唐有日得此消息必生嗔怒。你想那李世民自称王位一统江山心尚未足又兴过海征伐。若知我王害了他御弟圣僧一定兴兵马来与我王争敌。奈何兵少将微那时悔之晚矣。父王依儿所奏且把那四个和尚问他个来历分明先定他一段不参王驾然后方可问罪。”

这一篇原来是太子小心恐怕来伤了唐僧故意留住妖魔更不知行者安排着要打。那魔王果信其言立在龙床前面大喝一声道:“那和尚是几时离了东土?唐王因甚事着你求经?”行者昂然而答道:“我师父乃唐王御弟号曰三藏。因唐王驾下有一丞相姓魏名徵奉天条梦斩泾河老龙。大唐王梦游阴司地府复得回生之后大开水陆道场普度冤魂孽鬼。因我师父敷演经文广运慈悲忽得南海观世音菩萨指教来西。我师父大弘愿情欣意美报国尽忠蒙唐王赐与文牒。那时正是大唐贞观十三年九月望前三日。离了东土前至两界山收了我做大徒弟姓孙名悟空行者;又到乌斯国界高家庄收了二徒弟姓猪名悟能八戒;流沙河界又收了三徒弟姓沙名悟净和尚;前日在敕建宝林寺又新收个挑担的行童道人。”魔王闻说又没法搜检那唐僧弄巧计盘诘行者怒目问道:“那和尚你起初时一个人离东土又收了四众那三僧可让这一道难容。那行童断然是拐来的。他叫做甚么名字?有度牒是无度牒?拿他上来取供。”唬得那皇帝战战兢兢道:“师父啊!

我却怎的供?”孙行者捻他一把道:“你休怕等我替你供。”好大圣趋步上前对怪物厉声高叫道:“陛下这老道是一个瘖痖之人却又有些耳聋。只因他年幼间曾走过西天认得道路他的一节儿起落根本我尽知之望陛下宽恕待我替他供罢。”魔王道:“趁早实实的替他供来免得取罪。”行者道:“供罪行童年且迈痴聋瘖痖家私坏。祖居原是此间人五载之前遭破败。天无雨民干坏君王黎庶都斋戒。焚香沐浴告天公万里全无云叆叇。百姓饥荒若倒悬锺南忽降全真怪。呼风唤雨显神通然后暗将他命害。推下花园水井中阴侵龙位人难解。幸吾来功果大起死回生无挂碍。情愿皈依作行童与僧同去朝西界。假变君王是道人道人转是真王代。”那魔王在金銮殿上闻得这一篇言语唬得他心头撞小鹿面上起红云急抽身就要走路奈何手内无一兵器转回头只见一个镇殿将军腰挎一口宝刀被行者使了定身法直挺挺如痴如痖立在那里他近前夺了这宝刀就驾云头望空而去。气得沙和尚爆躁如雷猪八戒高声喊叫埋怨行者是一个急猴子:“你就慢说些儿却不稳住他了?如今他驾云逃走却往何处追寻?”行者笑道:“兄弟们且莫乱嚷。我等叫那太子下来拜父嫔后出来拜夫。”却又念个咒语解了定身法“教那多官苏醒回来拜君方知是真实皇帝教诉前情才见分晓我再去寻他。好大圣吩咐八戒、沙僧:“好生保护他君臣父子嫔后与我师父!”只听说声去就不见形影。

他原来跳在九霄云里睁眼四望看那魔王哩。只见那畜果逃了性命径往东北上走哩。行者赶得将近喝道:“那怪物那里去!老孙来了也!”那魔王急回头掣出宝刀高叫道:“孙行者你好惫懒!我来占别人的帝位与你无干你怎么来抱不平泄漏我的机密!”行者呵呵笑道:“我把你大胆的泼怪!皇帝又许你做?你既知我是老孙就该远遁;怎么还刁难我师父要取甚么供状!适才那供状是也不是?你不要走!好汉吃我老孙这一棒!”那魔侧身躲过掣宝刀劈面相还。他两个搭上手这一场好杀真是:猴王猛魔王强刀迎棒架敢相当。一天云雾迷三界只为当朝立帝王。他两个战经数合那妖魔抵不住猴王急回头复从旧路跳入城里闯在白玉阶前两班文武丛中摇身一变即变得与唐三藏一般模样并搀手立在阶前。

这大圣赶上就欲举棒来打那怪道:“徒弟莫打是我!”急掣棒要打那个唐僧却又道:“徒弟莫打是我!”一样两个唐僧实难辨认。“倘若一棒打杀妖怪变的唐僧这个也成了功果;假若一棒打杀我的真实师父却怎么好!”只得停手叫八戒、沙僧问道:“果然那一个是怪那一个是我的师父?你指与我我好打他。”八戒道:“你在半空中相打相嚷我瞥瞥眼就见两个师父也不知谁真谁假。”行者闻言捻诀念声咒语叫那护法诸天、六丁六甲、五方揭谛、四值功曹、一十八位护驾伽蓝、当坊土地、本境山神道:“老孙至此降妖妖魔变作我师父气体相同实难辨认。汝等暗中知会者请师父上殿让我擒魔。”原来那妖怪善腾云雾听得行者言语急撒手跳上金銮宝殿。这行者举起棒望唐僧就打。可怜!若不是唤那几位神来这一下就是二千个唐僧也打为肉酱!多亏众神架住铁棒道:“大圣那怪会腾云先上殿去了。”行者赶上殿他又跳将下来扯住唐僧在人丛里又混了一混依然难认。

行者心中不快又见那八戒在旁冷笑行者大怒道:“你这夯货怎的?如今有两个师父你有得叫有得应有得伏侍哩你这般欢喜得紧!”八戒笑道:“哥啊说我呆你比我又呆哩!

师父既不认得何劳费力?你且忍些头疼叫我师父念念那话儿我与沙僧各搀一个听着。若不会念的必是妖怪有何难也?”行者道:“兄弟亏你也正是那话儿只有三人记得。原是我佛如来心苗上所传与观世音菩萨菩萨又传与我师父便再没人知道。也罢师父念念。”真个那唐僧就念起来。那魔王怎么知得口里胡哼乱哼。八戒道:“这哼的却是妖怪了!”

他放了手举钯就筑。那魔王纵身跳起踏着云头便走。好八戒喝一声也驾云头赶上慌得那沙和尚丢了唐僧也掣出宝杖来打唐僧才停了咒语。孙大圣忍着头疼揝着铁棒赶在空中。呀!这一场三个狠和尚围住一个泼妖魔。那魔王被八戒沙僧使钉钯宝杖左右攻住了行者笑道:“我要再去当面打他他却有些怕我只恐他又走了。等我老孙跳高些与他个捣蒜打结果了他罢。”

这大圣纵祥光起在九霄正欲下个切手只见那东北上一朵彩云里面厉声叫道:“孙悟空且休下手!”行者回头看处原来文殊菩萨急收棒上前施礼道:“菩萨那里去?”文殊道:“我来替你收这个妖怪的。”行者谢道:“累烦了。”那菩萨袖中取出照妖镜照住了那怪的原身。行者才招呼八戒、沙僧齐来见了菩萨。却将镜子里看处那魔王生得好不凶恶:眼似琉璃盏头若炼炒缸。浑身三伏靛四爪九秋霜。搭拉两个耳一尾扫帚长。青毛生锐气红眼放金光。匾牙排玉板圆须挺硬枪。镜里观真象原是文殊一个狮猁王。行者道:“菩萨这是你坐下的一个青毛狮子却怎么走将来成精你就不收服他?”

菩萨道:“悟空他不曾走他是佛旨差来的。”行者道:“这畜类成精侵夺帝位还奉佛旨差来。似老孙保唐僧受苦就该领几道敕书!”菩萨道:“你不知道;当初这乌鸡国王好善斋僧佛差我来度他归西早证金身罗汉。因是不可原身相见变做一种凡僧问他化些斋供。被吾几句言语相难他不识我是个好人把我一条绳捆了送在那御水河中浸了我三日三夜。多亏六甲金身救我归西奏与如来、如来将此怪令到此处推他下井浸他三年以报吾三日水灾之恨。一饮一啄莫非前定。今得汝等来此成了功绩。”行者道:“你虽报了甚么一饮一啄的私仇但那怪物不知害了多少人也。”菩萨道:“也不曾害人自他到后这三年间风调雨顺国泰民安何害人之有?”行者道:“固然如此但只三宫娘娘与他同眠同起点污了他的身体坏了多少纲常伦理还叫做不曾害人?”菩萨道:“点污他不得他是个骗了的狮子。”八戒闻言走近前就摸了一把笑道:“这妖精真个是糟鼻子不吃酒——枉担其名了!”行者道:

“既如此收了去罢。若不是菩萨亲来决不饶他性命。”那菩萨却念个咒喝道:“畜生还不皈正更待何时!”那魔王才现了原身。菩萨放莲花罩定妖魔坐在背上踏祥光辞了行者。咦!

径转五台山上去宝莲座下听谈经。毕竟不知那唐僧师徒怎的出城且听下回分解。
------------

第四十回 婴儿戏化禅心乱 猿马刀归木母空

却说那孙大圣兄弟三人按下云头径至朝内只见那君臣储后几班儿拜接谢恩。行者将菩萨降魔收怪的那一节陈诉与他君臣听了一个个顶礼不尽。正都在贺喜之间又听得黄门官来奏:“主公外面又有四个和尚来也。”八戒慌了道:

“哥哥莫是妖精弄法假捏文殊菩萨哄了我等却又变作和尚来与我们斗智哩?”行者道:“岂有此理!”即命宣进来看。众文武传令着他进来。行者看时原来是那宝林寺僧人捧着那冲天冠、碧玉带、赭黄袍、无忧履进得来也。行者大喜道:“来得好!来得好!”且教道人过来摘下包巾戴上冲天冠;脱了布衣穿上赭黄袍;解了绦子系上碧玉带;褪了僧鞋登上无忧履。教太子拿出白玉圭来与他执在手里早请上殿称孤正是自古道:“朝廷不可一日无君。”那皇帝那里肯坐哭啼啼跪在阶心道:“我已死三年今蒙师父救我回生怎么又敢妄自称尊?请那一位师父为君我情愿领妻子城外为民足矣。”那三藏那里肯受一心只是要拜佛求经。又请行者行者笑道:“不瞒列位说老孙若肯做皇帝天下万国九州皇帝都做遍了。只是我们做惯了和尚是这般懒散。若做了皇帝就要留头长黄昏不睡五鼓不眠听有边报心神不安;见有灾荒忧愁无奈。

我们怎么弄得惯?你还做你的皇帝我还做我的和尚修功行去也。”那国王苦让不过只得上了宝殿南面称孤大赦天下封赠了宝林寺僧人回去。却才开东阁筵宴唐僧一壁厢传旨宣召丹青写下唐师徒四位喜容供养在金銮殿上。

那师徒们安了邦国不肯久停欲辞王驾投西。那皇帝与三宫妃后、太子诸臣将镇国的宝贝金银缎帛献与师父酬恩。那三藏分毫不受只是倒换关文催悟空等背马早行。那国王甚不过意摆整朝銮驾请唐僧上坐着两班文武引导他与三宫妃后并太子一家儿捧毂推轮送出城廓却才下龙辇与众相别。国王道:“师父啊到西天经回之日是必还到寡人界内一顾。”三藏道:“弟子领命。”那皇帝阁泪汪汪遂与众臣回去了。

那唐僧一行四僧上了羊肠大路一心里专拜灵山。正值秋尽冬初时节但见霜凋红叶林林瘦雨熟黄粱处处盈。日暖岭梅开晓色风摇山竹动寒声。师徒们离了乌鸡国夜住晓行将半月有余忽又见一座高山真个是摩天碍日。三藏马上心惊急兜缰忙呼行者。行者道:“师父有何吩咐?”三藏道:“你看前面又有大山峻岭须要仔细堤防恐一时又有邪物来侵我也。”行者笑道:“只管走路莫再多心老孙自有防护。”那长老只得宽怀加鞭策马奔至山岩果然也十分险峻。但见得:高不高顶上接青霄;深不深涧中如地府。山前常见骨都都白云扢腾腾黑雾。红梅翠竹绿柏青松。山后有千万丈挟魂灵台台后有古古怪怪藏魔洞洞中有叮叮狢狢滴水泉泉下更有弯弯曲曲流水涧。又见那跳天搠地献果猿丫丫叉叉带角鹿呢呢痴痴看人獐。至晚巴山寻穴虎待晓翻波出水龙。登得洞门唿喇的响惊得飞禽扑鲁的起看那林中走兽鞠律律的行。见此一伙禽和兽吓得人心扢磴磴惊。堂倒洞堂堂倒洞洞堂当倒洞当仙。青石染成千块玉碧纱笼罩万堆烟。师徒们正当悚惧又只见那山凹里有一朵红云直冒到九霄空内结聚了一团火气。行者大惊走近前把唐僧搊着脚推下马来叫:“兄弟们不要走了妖怪来矣。”慌得个八戒急掣钉钯沙僧忙轮宝杖把唐僧围护在当中。

话分两头。却说红光里真是个妖精。他数年前闻得人讲:“东土唐僧往西天取经乃是金蝉长老转生十世修行的好人。有人吃他一块肉延生长寿与天地同休。”他朝朝在山间等候不期今日到了。他在那半空里正然观看只见三个徒弟把唐僧围护在马上各各准备。这精灵夸赞不尽道:“好和尚!我才看着一个白面胖和尚骑了马真是那唐朝圣僧却怎么被三个丑和尚护持住了!一个个伸拳敛袖各执兵器似乎要与人打的一般。噫!不知是那个有眼力的想应认得我了似此模样莫想得那唐僧的肉吃。”沉吟半晌以心问心的自家商量道:“若要倚势而擒莫能得近;或者以善迷他却到得手。

但哄得他心迷惑待我在善内生机断然拿了。且下去戏他一戏。”好妖怪即散红光按云头落下去那山坡里摇身一变变作七岁顽童赤条条的身上无衣将麻绳捆了手足高吊在那松树梢头口口声声只叫“救人!救人!”

却说那孙大圣忽抬头再看处只见那红云散尽火气全无便叫:“师父请上马走路。”唐僧道:“你说妖怪来了怎么又敢走路?”行者道:“我才然间见一朵红云从地而起到空中结做一团火气断然是妖精。这一会红云散了想是个过路的妖精不敢伤人我们去耶!”八戒笑道:“师兄说话最巧妖精又有个甚么过路的?”行者道:“你那里知道若是那山那洞的魔王设宴邀请那诸山各洞之精赴会却就有东南西北四路的精灵都来赴会故此他只有心赴会无意伤人。此乃过路之妖精也。”三藏闻言也似信不信的只得攀鞍在马顺路奔山前进。正行时只听得叫声“救人!”长老大惊道:“徒弟呀这半山中是那里甚么人叫?”行者上前道:“师父只管走路莫缠甚么人轿骡轿明轿睡轿。这所在就有轿也没个人抬你。”唐僧道:“不是扛抬之轿乃是叫唤之叫。”行者笑道:“我晓得莫管闲事且走路。”

三藏依言策马又进行不上一里之遥又听得叫声“救人!”长老道:“徒弟这个叫声不是鬼魅妖邪;若是鬼魅妖邪但有出声无有回声。你听他叫一声又叫一声想必是个有难之人我们可去救他一救。”行者道:“师父今日且把这慈悲心略收起收起待过了此山再慈悲罢。这去处凶多吉少你知道那倚草附木之说是物可以成精。诸般还可只有一般蟒蛇但修得年远日深成了精魅善能知人小名儿。他若在草科里或山凹中叫人一声人不答应还可;若答应一声他就把人元神绰去当夜跟来断然伤人性命。且走!且走!古人云脱得去谢神明切不可听他。”长老只得依他又加鞭催马而去行者心中暗想:“这泼怪不知在那里只管叫阿叫的。等我老孙送他一个卯酉星法教他两不见面。”好大圣叫沙和尚前来:“拢着马慢慢走着让老孙解解手。”你看他让唐僧先行几步却念个咒语使个移山缩地之法把金箍棒往后一指他师徒过此峰头往前走了却把那怪物撇下他再拽开步赶上唐僧一路奔山。只见那三藏又听得那山背后叫声“救人!”长老道:

“徒弟呀那有难的人大没缘法不曾得遇着我们。我们走过他了你听他在山后叫哩。”八戒道:“在便还在山前只是如今风转了也。”行者道:“管他甚么转风不转风且走路。”因此遂都无言语恨不得一步插过此山不题话下。

却说那妖精在山坡里连叫了三四声更无人到他心中思量道:“我等唐僧在此望见他离不上三里却怎么这半晌还不到?想是抄下路去了。”他抖一抖身躯脱了绳索又纵红光上空再看。不觉孙大圣仰面回观识得是妖怪又把唐僧撮着脚推下马来道:“兄弟们仔细!仔细!那妖精又来也!”慌得那八戒、沙僧各持兵刀将唐僧又围护在中间。那精灵见了在半空中称羡不已道:“好和尚!我才见那白面和尚坐在马上却怎么又被他三人藏了?这一去见面方知。先把那有眼力的弄倒了方才捉得唐僧。不然啊徒费心机难获物枉劳情兴总成空。”却又按下云头恰似前番变化高吊在松树山头等候这番却不上半里之地。

却说那孙大圣抬头再看只见那红云又散复请师父上马前行。三藏道:“你说妖精又来如何又请走路?”行者道:“这还是个过路的妖精不敢惹我们。”长老又怀怒道:“这个泼猴十分弄我!正当有妖魔处却说无事;似这般清平之所却又恐吓我不时的嚷道有甚妖精。虚多实少不管轻重将我搊着脚捽下马来如今却解说甚么过路的妖精。假若跌伤了我却也过意不去!这等这等!”行者道:“师父莫怪若是跌伤了你的手足却还好医治;若是被妖精捞了去却何处跟寻?”三藏大怒哏哏的要念《紧箍儿咒》却是沙僧苦劝只得上马又行。

还未曾坐得稳只听又叫“师父救人啊!”长老抬头看时原来是个小孩童赤条条的吊在那树上兜住缰便骂行者道:“这泼猴多大惫懒!全无有一些儿善良之意心心只是要撒泼行凶哩!我那般说叫唤的是个人声他就千言万语只嚷是妖怪!你看那树上吊的不是个人么?”大圣见师父怪下来了却又觌面看见模样一则做不得手脚二来又怕念《紧箍儿咒》低着头再也不敢回言让唐僧到了树下。那长老将鞭梢指着问道:“你是那家孩儿?因有甚事吊在此间?说与我好救你。”噫!分明他是个精灵变化得这等那师父却是个肉眼凡胎不能相识。

那妖魔见他下问越弄虚头眼中噙泪叫道:“师父呀山西去有一条枯松涧涧那边有一庄村我是那里人家。我祖公公姓红只因广积金银家私巨万混名唤做红百万。年老归世已久家产遗与我父。近来人事奢侈家私渐废改名唤做红十万专一结交四路豪杰将金银借放希图利息。怎知那无籍之人设骗了去啊本利无归。我父了洪誓分文不借。那借金银人身贫无计结成凶党明火执杖白日杀上我门将我财帛尽情劫掳把我父亲杀了见我母亲有些颜色拐将去做甚么压寨夫人。那时节我母亲舍不得我把我抱在怀里哭哀哀战兢兢跟随贼寇不期到此山中又要杀我多亏我母亲哀告免教我刀下身亡却将绳子吊我在树上只教冻饿而死那些贼将我母亲不知掠往那里去了。我在此已吊三日三夜更没一个人来行走。不知那世里修积今生得遇老师父若肯舍大慈悲救我一命回家就典身卖命也酬谢师恩致使黄沙盖面更不敢忘也。”三藏闻言认了真实就教八戒解放绳索救他下来。那呆子也不识人便要上前动手行者在旁忍不住喝了一声道:“那泼物!有认得你的在这里哩!莫要只管架空捣鬼说谎哄人!你既家私被劫父被贼伤母被人掳救你去交与谁人?你将何物与我作谢?这谎脱节了耶!”那怪闻言心中害怕就知大圣是个能人暗将他放在心上却又战战兢兢滴泪而言曰:“师父虽然我父母空亡家财尽绝还有些田产未动亲戚皆存。”行者道:“你有甚么亲戚?”妖怪道:“我外公家在山南姑娘住居岭北。涧头李四是我姨夫;林内红三是我族伯。还有堂叔堂兄都住在本庄左右。老师父若肯救我到了庄上见了诸亲将老师父拯救之恩一一对众言说典卖些田产重重酬谢也。”八戒听说扛住行者道:“哥哥这等一个小孩子家你只管盘诘他怎的!他说得是强盗只打劫他些浮财莫成连房屋田产也劫得去?若与他亲戚们说了我们纵有广大食肠也吃不了他十亩田价。救他下来罢。”呆子只是想着吃食那里管甚么好歹使戒刀挑断绳索放下怪来。那怪对唐僧马下泪汪汪只情磕头。长老心慈便叫:“孩儿你上马来我带你去。”那怪道:“师父啊我手脚都吊麻了腰胯疼痛一则是乡下人家不惯骑马。”唐僧叫八戒驮着那妖怪抹了一眼道:“师父我的皮肤都冻熟了不敢要这位师父驮。他的嘴长耳大脑后鬃硬搠得我慌。”唐僧道:“教沙和尚驮着。”那怪也抹了一眼道:“师父那些贼来打劫我家时一个个都搽了花脸带假胡子拿刀弄杖的。我被他唬怕了见这位晦气脸的师父一没了魂了也不敢要他驮。”唐僧教孙行者驮着行者呵呵笑道:“我驮!我驮!”那怪物暗自欢喜顺顺当当的要行者驮他。行者把他扯在路旁边试了一试只好有三斤十来两重。

行者笑道:“你这个泼怪物今日该死了怎么在老孙面前捣鬼!我认得你是个那话儿呵。”妖怪道:“师父我是好人家儿女不幸遭此大难我怎么是个甚么那话儿?”行者道:“你既是好人家儿女怎么这等骨头轻?”妖怪道:“我骨格儿小。”行者道:“你今年几岁了?”那怪道:“我七岁了。”行者笑道:“一岁长一斤也该七斤你怎么不满四斤重么?”那怪道:“我小时失乳。”行者说:“也罢我驮着你若要尿尿把把须和我说。”三藏才与八戒、沙僧前走行者背着孩儿随后一行径投西去。有诗为证诗曰:道德高隆魔障高禅机本静静生妖。心君正直行中道木母痴顽躧外趫。意马不言怀爱欲黄婆无语自忧焦。客邪得志空欢喜毕竟还从正处消。孙大圣驮着妖魔心中埋怨唐僧不知艰苦“行此险峻山场空身也难走却教老孙驮人。

这厮莫说他是妖怪就是好人他没了父母不知将他驮与何人倒不如掼杀他罢。”那怪物却早知觉了便就使个神通往四下里吸了四口气吹在行者背上便觉重有千斤。行者笑道:

“我儿啊你弄重身法压我老爷哩!”那怪闻言恐怕大圣伤他却就解尸出了元神跳将起去佇立在九霄空里这行者背上越重了。猴王怒抓过他来往那路旁边赖石头上滑辣的一掼将尸骸掼得象个肉饼一般还恐他又无礼索性将四肢扯下丢在路两边俱粉碎了。

那物在空中明明看着忍不住心头火起道:“这猴和尚十分惫懒!就作我是个妖魔要害你师父却还不曾见怎么下手哩你怎么就把我这等伤损!早是我有算计出神走了不然是无故伤生也。若不趁此时拿了唐僧再让一番越教他停留长智。”好怪物就在半空里弄了一阵旋风呼的一声响亮走石扬沙诚然凶狠。好风:淘淘怒卷水云腥黑气腾腾闭日明。岭树连根通拔尽野梅带干悉皆平。黄沙迷目人难走怪石伤残路怎平。滚滚团团平地暗遍山禽兽哮声。刮得那三藏马上难存八戒不敢仰视沙僧低头掩面。孙大圣情知是怪物弄风急纵步来赶时那怪已骋风头将唐僧摄去了无踪无影不知摄向何方无处跟寻。

一时间风声暂息日色光明。行者上前观看只见白龙马战兢兢喊声嘶行李担丢在路下八戒伏于崖下呻吟沙僧蹲在坡前叫唤。行者喊:“八戒!”那呆子听见是行者的声音却抬头看时狂风已静爬起来扯住行者道:“哥哥好大风啊!”

沙僧却也上前道:“哥哥这是一阵旋风。”又问:“师父在那里?”八戒道:“风来得紧我们都藏头遮眼各自躲风师父也伏在马上的。”行者道:“如今却往那里去了?”沙僧道:“是个灯草做的想被一风卷去也。”行者道:“兄弟们我等自此就该散了!”八戒道:“正是趁早散了各寻头路多少是好。那西天路无穷无尽几时能到得!”沙僧闻言打了一个失惊浑身麻木道:“师兄你都说的是那里话。我等因为前生有罪感蒙观世音菩萨劝化与我们摩顶受戒改换法名皈依佛果情愿保护唐僧上西方拜佛求经将功折罪。今日到此一旦俱休说出这等各寻头路的话来可不违了菩萨的善果坏了自己的德行惹人耻笑说我们有始无终也!”行者道:“兄弟你说的也是奈何师父不听人说我老孙火眼金睛认得好歹才然这风是那树上吊的孩儿弄的。我认得他是个妖精你们不识那师父也不识认作是好人家儿女教我驮着他走。是老孙算计要摆布他他就弄个重身法压我。是我把他掼得粉碎他想是又使解尸之法弄阵旋风把我师父摄去也。因此上怪他每每不听我说。故我意懒心灰说各人散了。既是贤弟有此诚意教老孙进退两难。八戒你端的要怎的处?”八戒道:“我才自失口乱说了几句其实也不该散。哥哥没及奈何还信沙弟之言去寻那妖怪救师父去。”行者却回嗔作喜道:“兄弟们还要来结同心收拾了行李马匹上山找寻怪物搭救师父去。”三个人附葛扳藤寻坡转涧行经有五七十里却也没个音信那山上飞禽走兽全无老柏乔松常见。孙大圣着实心焦将身一纵跳上那巅险峰头喝一声叫“变!”变作三头六臂似那大闹天宫的本象将金箍棒幌一幌变作三根金箍棒劈哩扑辣的往东打一路往西打一路两边不住的乱打。八戒见了道:“沙和尚不好了师兄是寻不着师父恼出气心风来了。”

那行者打了一会打出一伙穷神来都披一片挂一片裩无裆裤无口的跪在山前叫:“大圣山神土地来见。”行者道:“怎么就有许多山神土地?”众神叩头道:“上告大圣此山唤做六百里钻头号山。我等是十里一山神十里一土地共该三十名山神三十名土地。昨日已此闻大圣来了只因一时会不齐故此接迟致令大圣怒万望恕罪。”行者道:“我且饶你罪名。我问你:这山上有多少妖精?”众神道:“爷爷呀只有得一个妖精把我们头也摩光了弄得我们少香没纸血食全无一个个衣不充身食不充口还吃得有多少妖精哩!”行者道:“这妖精在山前住是山后住?”众神道:“他也不在山前山后。这山中有一条涧叫做枯松涧涧边有一座洞叫做火云洞那洞里有一个魔王神通广大常常的把我们山神土地拿了去烧火顶门黑夜与他提铃喝号。小妖儿又讨甚么常例钱。”行者道:“汝等乃是阴鬼之仙有何钱钞?”众神道:“正是没钱与他只得捉几个山獐野鹿早晚间打点群精;若是没物相送就要来拆庙宇剥衣裳搅得我等不得安生!万望大圣与我等剿除此怪拯救山上生灵。”行者道:“你等既受他节制常在他洞下可知他是那里妖精叫做甚么名字?”众神道:“说起他来或者大圣也知道。他是牛魔王的儿子罗刹女养的。他曾在火焰山修行了三百年炼成三昧真火却也神通广大。牛魔王使他来镇守号山乳名叫做红孩儿号叫做圣婴大王。”行者闻言满心欢喜喝退了土地山神却现了本象跳下峰头对八戒沙僧道:“兄弟们放心再不须思念师父决不伤生妖精与老孙有亲。”八戒笑道:“哥哥莫要说谎。你在东胜神洲他这里是西牛贺洲路程遥远隔着万水千山海洋也有两道怎的与你有亲?”行者道:“刚才这伙人都是本境土地山神。我问他妖怪的原因他道是牛魔王的儿子罗刹女养的名字唤做红孩儿号圣婴大王。想我老孙五百年前大闹天宫时遍游天下名山寻访大地豪杰那牛魔王曾与老孙结七弟兄。一般五六个魔王止有老孙生得小巧故此把牛魔王称为大哥。这妖精是牛魔王的儿子我与他父亲相识若论将起来还是他老叔哩他怎敢害我师父?我们趁早去来。”沙和尚笑道:“哥啊常言道:三年不上门当亲也不亲哩。你与他相别五六百年又不曾往还杯酒又没有个节礼相邀他那里与你认甚么亲耶?”

行者道:“你怎么这等量人!常言道一叶浮萍归大海为人何处不相逢!纵然他不认亲好道也不伤我师父。不望他相留酒席必定也还我个囫囵唐僧。”三兄弟各办虔心牵着白马马上驮着行李找大路一直前进。无分昼夜行了百十里远近忽见一松林林中有一条曲涧涧下有碧澄澄的活水飞流那涧梢头有一座石板桥通着那厢洞府。行者道:“兄弟你看那壁厢有石崖磷磷想必是妖精住处了。我等从众商议那个管看守行李马匹那个肯跟我过去降妖?”八戒道:“哥哥老猪没甚坐性我随你去罢。”行者道:“好!好!”教沙僧:“将马匹行李俱潜在树林深处小心守护待我两个上门去寻师父耶。”那沙僧依命八戒相随与行者各持兵器前来。正是:未炼婴儿邪火胜心猿木母共扶持。毕竟不知这一去吉凶何如且听下回分解。
------------

第四十一回 心猿遭火败 木母被魔擒

善恶一时忘念荣枯都不关心。晦明隐现任浮沉随分饥餐渴饮。神静湛然常寂昏冥便有魔侵。五行蹭蹬破禅林风动必然寒凛。却说那孙大圣引八戒别了沙僧跳过枯松涧径来到那怪石崖前果见有一座洞府真个也景致非凡。但见回銮古道幽还静风月也听玄鹤弄。白云透出满川光流水过桥仙意兴。猿啸鸟啼花木奇藤萝石蹬芝兰胜。苍摇崖壑散烟霞翠染松篁招彩凤。远列巅峰似插屏山朝涧绕真仙洞。昆仑地脉来龙有分有缘方受用。将近行到门前见有一座石碣上镌八个大字乃是“号山枯松涧火云洞”。那壁厢一群小妖在那里轮枪舞剑的跳风顽耍。孙大圣厉声高叫道:“那小的们趁早去报与洞主知道教他送出我唐僧师父来免你这一洞精灵的性命!牙迸半个不字我就掀翻了你的山场躧平了你的洞府!”那些小妖闻有此言慌忙急转身各归洞里关了两扇石门到里边来报:“大王祸事了!”

却说那怪自把三藏拿到洞中选剥了衣服四马攒蹄捆在后院里着小妖打干净水刷洗要上笼蒸吃哩急听得报声祸事且不刷洗便来前庭上问:“有何祸事?”小妖道:“有个毛脸雷公嘴的和尚带一个长嘴大耳的和尚在门前要甚么唐僧师父哩。但若牙迸半个不字就要掀翻山场躧平洞府。”魔王微微冷笑道:“这是孙行者与猪八戒他却也会寻哩。他拿他师父自半山中到此有百五十里却怎么就寻上门来?”教:“小的们把管车的推出车去!”那一班几个小妖推出五辆小车儿来开了前门。八戒望见道:“哥哥这妖精想是怕我们推出车子往那厢搬哩。”行者道:“不是且看他放在那里。”只见那小妖将车子按金、木、水、火、土安下着五个看着五个进去通报。那魔王问:“停当了?”答应:“停当了。”教:“取过枪来。”有那一伙管兵器的小妖着两个抬出一杆丈八长的火尖枪递与妖王。妖王轮枪拽步也无甚么盔甲只是腰间束一条锦绣战裙赤着脚走出门前。行者与八戒抬头观看但见那怪物:面如傅粉三分白唇若涂朱一表才。鬓挽青云欺靛染眉分新月似刀裁。战裙巧绣盘龙凤形比哪吒更富胎。双手绰枪威凛冽祥光护体出门来。哏声响若春雷吼暴眼明如掣电乖。要识此魔真姓氏名扬千古唤红孩。那红孩儿怪出得门来高叫道:

“是甚么人在我这里吆喝!”行者近前笑道:“我贤侄莫弄虚头你今早在山路旁高吊在松树梢头是那般一个瘦怯怯的黄病孩儿哄了我师父。我倒好意驮着你你就弄风儿把我师父摄将来。你如今又弄这个样子我岂不认得你?趁早送出我师父不要白了面皮失了亲情恐你令尊知道怪我老孙以长欺幼不象模样。”那怪闻言心中大怒咄的一声喝道:“那泼猴头!我与你有甚亲情?你在这里满口胡柴绰甚声经儿!那个是你贤侄?”行者道:“哥哥是你也不晓得。当年我与你令尊做弟兄时你还不知在那里哩。”那怪道:“这猴子一胡说!你是那里人我是那里人怎么得与我父亲做兄弟?”行者道:“你是不知我乃五百年前大闹天宫的齐天大圣孙悟空是也。我当初未闹天宫时遍游海角天涯四大部洲无方不到。那时节专慕豪杰你令尊叫做牛魔王称为平天大圣与我老孙结为七弟兄让他做了大哥;还有个蛟魔王称为复海大圣做了二哥;又有个大鹏魔王称为混天大圣做了三哥;又有个狮狔王称为移山大圣做了四哥;又有个猕猴王称为通风大圣做了五哥;又有个獝狨王称为驱神大圣做了六哥;惟有老孙身小称为齐天大圣排行第七。我老弟兄们那时节耍子时还不曾生你哩!”

那怪物闻言那里肯信举起火尖枪就刺。行者正是那会家不忙又使了一个身法闪过枪头轮起铁棒骂道:“你这小畜生不识高低!看棍!”那妖精也使身法让过铁棒道:“泼猢狲不达时务!看枪!”他两个也不论亲情一齐变脸各使神通跳在云端里好杀:行者名声大魔王手段强。一个横举金箍棒一个直挺火尖枪。吐雾遮三界喷云照四方。一天杀气凶声吼日月星辰不见光。语言无逊让情意两乖张。那一个欺心失礼仪这一个变脸没纲常。棒架威风长枪来野性狂。一个是混元真大圣一个是正果善财郎。二人努力争强胜只为唐僧拜法王。那妖魔与孙大圣战经二十合不分胜败。猪八戒在旁边看得明白:妖精虽不败降却只是遮拦隔架全无攻杀之能;行者纵不赢他棒法精强来往只在那妖精头上不离了左右。八戒暗想道:“不好啊行者溜撒一时间丢个破绽哄那妖魔钻进来一铁棒打倒就没了我的功劳。”你看他抖擞精神举着九齿钯在空里望妖精劈头就筑。那怪见了心惊急拖枪败下阵来。行者喝教八戒:“赶上!赶上!”

二人赶到他洞门前只见妖精一只手举着火尖枪站在那中间一辆小车儿上一只手捏着拳头往自家鼻子上捶了两拳。八戒笑道:“这厮放赖不羞!你好道捶破鼻子淌出些血来搽红了脸往那里告我们去耶?”那妖魔捶了两拳念个咒语口里喷出火来鼻子里浓烟迸出闸闸眼火焰齐生。那五辆车子上火光涌出。连喷了几口只见那红焰焰、大火烧空把一座火云洞被那烟火迷漫真个是熯天炽地。八戒慌了道:“哥哥不停当!这一钻在火里莫想得活把老猪弄做个烧熟的加上香料尽他受用哩!快走!快走!”说声走他也不顾行者跑过涧去了。这行者神通广大捏着避火诀撞入火中寻那妖怪。那妖怪见行者来又吐上几口那火比前更胜。好火:炎炎烈烈盈空燎赫赫威威遍地红。却似火轮飞上下犹如炭屑舞西东。这火不是燧人钻木又不是老子炮丹非天火非野火乃是妖魔修炼成真三昧火。五辆车儿合五行五行生化火煎成。肝木能生心火旺心火致令脾土平。脾土生金金化水水能生木彻通灵。生生化化皆因火火遍长空万物荣。妖邪久悟呼三昧永镇西方第一名。行者被他烟火飞腾不能寻怪看不见他洞门前路径抽身跳出火中。那妖精在门看得明白他见行者走了却才收了火具帅群妖转于洞内闭了石门以为得胜着小的排宴奏乐、欢笑不题。

却说行者跳过枯松涧按下云头只听得八戒与沙僧朗朗的在松间讲话。行者上前喝八戒道:“你这呆子全无人气!你就惧怕妖火败走逃生却把老孙丢下早是我有些南北哩!”

八戒笑道:“哥啊你被那妖精说着了果然不达时务。古人云:

识得时务者呼为俊杰。那妖精不与你亲你强要认亲;既与你赌斗放出那般无情的火来又不走还要与他恋战哩!”行者道:“那怪物的手段比我何如?”八戒道:“不济。”“枪法比我何如?”八戒道:“也不济。老猪见他撑持不住却来助你一钯不期他不识耍就败下阵来没天理就放火了。”行者道:“正是你不该来。我再与他斗几合我取巧儿捞他一棒却不是好?”

他两个只管论那妖精的手段讲那妖精的火毒沙和尚倚着松根笑得呆了。行者看见道:“兄弟你笑怎么?你好道有甚手段擒得那妖魔破得那火阵?这桩事也是大家有益的事。常言道众毛攒毬。你若拿得妖魔救了师父也是你的一件大功绩。”沙僧道:“我也没甚手段也不能降妖。我笑你两个都着了忙也。”行者道:“我怎么着忙?”沙僧道:“那妖精手段不如你枪法不如你只是多了些火势故不能取胜。若依小弟说以相生相克拿他有甚难处?”行者闻言呵呵笑道:“兄弟说得有理。果然我们着忙了忘了这事。若以相生相克之理论之须是以水克火却往那里寻些水来泼灭这妖火可不救了师父?”沙僧道:“正是这般不必迟疑。”行者道:“你两个只在此间莫与他索战待老孙去东洋大海求借龙兵将些水来泼息妖火捉这泼怪。”八戒道:“哥哥放心前去我等理会得。”

好大圣纵云离此地顷刻到东洋却也无心看玩海景使个逼水法分开波浪。正行时见一个巡海夜叉相撞看见是孙大圣急回到水晶宫里报知那老龙王。敖广即率龙子、龙孙、虾兵、蟹卒一齐出门迎接请里面坐。坐定礼毕告茶行者道:

“不劳茶有一事相烦。我因师父唐僧往西天拜佛取经经过号山枯松涧火云洞有个红孩儿妖精号圣婴大王把我师父拿了去。是老孙寻到洞边与他交战他却放出火来。我们禁不得他想着水能克火特来问你求些水去与我下场大雨泼灭了妖火救唐僧一难。”那龙王道:“大圣差了若要求取雨水不该来问我。”行者道:“你是四海龙王主司雨泽不来问你却去问谁?”龙王道:“我虽司雨不敢擅专须得玉帝旨意吩咐在那地方要几尺几寸甚么时辰起住还要三官举笔太乙移文会令了雷公电母风伯云童俗语云龙无云而不行哩。”

行者道:“我也不用着风云雷电只是要些雨水灭火。”龙王道:

“大圣不用风云雷电但我一人也不能助力着舍弟们同助大圣一功如何?”行者道:“令弟何在?”龙王道:“南海龙王敖钦、北海龙王敖闰、西海龙王敖顺。”行者笑道:“我若再游过三海不如上界去求玉帝旨意了。”龙王道:“不消大圣去只我这里撞动铁鼓金钟他自顷刻而至。”行者闻其言道:“老龙王快撞钟鼓。”

须臾间三海龙王拥至问:“大哥有何事命弟等?”敖广道:“孙大圣在这里借雨助力降妖。”三弟即引进见毕行者备言借水之事众神个个欢从即点起鲨鱼骁勇为前部鳠痴口大作先锋。鲤元帅翻波跳浪鯾提督吐雾喷风。鲭太尉东方打哨鲌都司西路催征。红眼马郎南面舞黑甲将军北下冲。鱑把总中军掌号五方兵处处英雄。纵横机巧鼋枢密妙算玄微龟相分。有谋有智鼍丞相多变多能鳖总戎。横行蟹士轮长剑直跳虾婆扯硬弓。鲇外郎查明文簿点龙兵出离波中。

诗曰:四海龙王喜助功齐天大圣请相从。只因三藏途中难借水前来灭火红。

那行者领着龙兵不多时早到号山枯松涧上。行者道:“敖氏昆玉有烦远涉。此间乃妖魔之处汝等且停于空中不要出头露面。让老孙与他赌斗若赢了他不须列位捉拿;若输与他也不用列位助阵。只是他但放火时可听我呼唤一齐喷雨。”龙王俱如号令。

行者却按云头入松林里见了八戒、沙僧叫声:“兄弟。”

八戒道:“哥哥来得快哑!可曾请得龙王来?”行者道:“俱来了。

你两个切须仔细只怕雨大莫湿了行李待老孙与他打去。”

沙僧道:“师兄放心前去我等俱理会得了。”行者跳过涧到了门叫声“开门!”那些小妖又去报道:“孙行者又来了。”红孩仰面笑道:“那猴子想是火中不曾烧了他故此又来。这一来切莫饶他断然烧个皮焦肉烂才罢!”急纵身挺着长枪教:“小的们推出火车子来!”他出门前对行者道:“你又来怎的?”行者道:“还我师父来。”那怪道:“你这猴头忒不通变。那唐僧与你做得师父也与我做得按酒你还思量要他哩莫想莫想!”

行者闻言十分恼怒掣金箍棒劈头就打。那妖精使火尖枪急架相迎。这一场赌斗比前不同好杀:怒泼妖魔恼急猴王将。这一个专救取经僧那一个要吃唐三藏。心变没亲情情疏无义让。这个恨不得捉住活剥皮那个恨不得拿来生蘸酱真个忒英雄果然多猛壮。棒来枪架赌输赢枪去棒迎争下上。举手相轮二十回两家本事一般样。那妖王与行者战经二十回合见得不能取胜虚幌一枪怎抽身捏着拳头又将鼻子捶了两下却就喷出火来。那门前车子上烟火迸起;口眼中赤焰飞腾。孙大圣回头叫道:“龙王何在?”那龙王兄弟帅众水族望妖精火光里喷下雨来。好雨!真个是:潇潇洒洒密密沉沉。潇潇洒洒如天边坠落星辰;密密沉沉似海口倒悬浪滚。起初时如拳大小次后来瓮泼盆倾。满地浇流鸭顶绿高山洗出佛头青。沟壑水飞千丈玉涧泉波涨万条银。三叉路口看看满九曲溪中渐渐平。这个是唐僧有难神龙助扳倒天河往下倾。那雨淙综大小莫能止息那妖精的火势。原来龙王私雨只好泼得凡火妖精的三昧真火如何泼得?好一似火上浇油越泼越灼。大圣道:“等我捻着诀。钻入火中!”轮铁棒寻妖要打。那妖见他来到将一口烟劈脸喷来。行者急回头煼得眼花雀乱忍不住泪落如雨。原来这大圣不怕火只怕烟。当年因大闹天宫时被老君放在八封炉中锻过一番他幸在那巽位安身不曾烧坏只是风搅得烟来把他煼做火眼金睛故至今只是怕烟。那妖又喷一口行者当不得纵云头走了。那妖王却又收了火具回归洞府。

这大圣一身烟火炮燥难禁径投于涧水内救火。怎知被冷水一逼弄得火气攻心三魂出舍可怜气塞胸堂喉舌冷魂飞魄散丧残生!慌得那四海龙王在半空里收了雨泽高声大叫:“天蓬元帅!卷帘将军!休在林中藏隐且寻你师兄出来!”

八戒与沙僧听得呼他圣号急忙解了马、挑着担奔出林来也不顾泥泞顺涧边找寻只见那上溜头翻波滚浪急流中淌下一个人来。沙僧见了连衣跳下水中抱上岸来却是孙大圣身躯。噫!你看他踡跼四肢伸不得浑身上下冷如冰。沙和尚满眼垂泪道:“师兄!可惜了你亿万年不老长生客如今化作个中途短命人!”八戒笑道:“兄弟莫哭这猴子佯推死吓我们哩。你摸他摸胸前还有一点热气没有?”沙僧道:“浑身都冷了就有一点儿热气怎的就是回生?”八戒道:“他有七十二般变化就有七十二条性命。你扯着脚等我摆布他。”真个那沙僧扯着脚八戒扶着头把他拽个直推上脚来盘膝坐定。八戒将两手搓热仵住他的七窍使一个按摩禅法。原来那行者被冷水逼了气阻丹田不能出声却幸得八戒按摸揉擦须臾间气透三关转明堂冲开孔窍叫了一声:“师父啊!”沙僧道:“哥啊你生为师父死也还在口里且苏醒我们在这里哩。”行者睁开眼道:“兄弟们在这里?老孙吃了亏也!”八戒笑道:“你才子昏的若不是老猪救你啊已此了帐了还不谢我哩!”行者却才起身仰面道:“敖氏弟兄何在?”那四海龙王在半空中答应道:“小龙在此伺候。”行者道:“累你远劳不曾成得功果且请回去改日再谢。”龙王帅水族泱泱而回不在话下。

沙僧搀着行者一同到松林之下坐定。少时间却定神顺气止不住泪滴腮边又叫:“师父啊!忆昔当年出大唐岩前救我脱灾殃。三山六水遭魔障万苦千辛割寸肠。托钵朝餐随厚薄参禅暮宿或林庄。一心指望成功果今日安知痛受伤!”沙僧道:“哥哥且休烦恼我们早安计策去那里请兵助力搭救师父耶?”行者道:“那里请救么?”沙僧道:“当初菩萨吩咐着我等保护唐僧他曾许我们叫天天应叫地地应。那里请救去?”行者道:“想老孙大闹天宫时那些神兵都禁不得我。这妖精神通不小须是比老孙手段大些的才降得他哩。天神不济地煞不能若要拿此妖魔须是去请观音菩萨才好。奈何我皮肉酸麻腰膝疼痛驾不起筋斗云怎生请得?”八戒道:“有甚话吩咐等我去请。”行者笑道:“也罢你是去得。若见了菩萨切休仰视只可低头礼拜。等他问时你却将地名、妖名说与他再请教师父之事。他若肯来定取擒了怪物。”八戒闻言即便驾了云雾向南而去。

却说那个妖王在洞里欢喜道:“小的们孙行者吃了亏去了。这一阵虽不得他死好道也个大昏。咦只怕他又请救兵来也快开门等我去看他请谁。”众妖开了门妖精就跳在空里观看只见八戒往南去了。妖精想着南边再无他处断然是请观音菩萨急按下云叫:“小的们把我那皮袋寻出来。多时不用只恐口绳不牢与我换上一条放在二门之下等我去把八戒赚将回来装于袋内蒸得稀烂犒劳你们。”原来那妖精有一个如意的皮袋。众小妖拿出来、换了口绳安于洞门内不题。

却说那妖王久居于此俱是熟游之地他晓得那条路上南海去近那条去远。他从那近路上一驾云头赶过了八戒端坐在壁岩之上变作一个“假观世音”模样等候着八戒。那呆子正纵云行处忽然望见菩萨他那里识得真假?这才是见象作佛。呆子停云下拜道:“菩萨弟子猪悟能叩头。”妖精道:“你不保唐僧去取经却见我有何事干?”八戒道:“弟子因与师父行至中途遇着号山枯松涧火云洞有个红孩儿妖精他把我师父摄了去。是弟子与师兄等寻上他门与他交战。他原来会放火头一阵不曾得赢;第二阵请龙王助雨也不能灭火。

师兄被他烧坏了不能行动着弟子来请菩萨万望垂慈救我师父一难!”妖精道:“那火云洞洞主不是个伤生的一定是你们冲撞了他也。”八戒道:“我不曾冲撞他是师兄悟空冲撞他的。他变作一个小孩子吊在树上试我师父。师父甚有善心教我解下来着师兄驮他一程。是师兄掼了他一掼他就弄风儿把师父摄去了。”妖精道:“你起来跟我进那洞里见洞主与你说个人情你陪一个礼把你师父讨出来罢。”八戒道:“菩萨呀若肯还我师父就磕他一个头也罢。”妖王道:“你跟来。”

那呆子不知好歹就跟着他径回旧路却不向南洋海随赴火云门顷刻间到了门。妖精进去道:“你休疑忌他是我的故人你进来。”呆子只得举步入门。众妖一齐呐喊将八戒捉倒装于袋内束紧了口绳高吊在驮梁之上。妖精现了本象坐在当中道:“猪八戒你有甚么手段就敢保唐僧取经就敢请菩萨降我?你大睁着两个眼还不认得我是圣婴大王哩!如今拿你吊得三五日蒸熟了赏赐小妖权为案酒!”八戒听言在里面骂道:“泼怪物!十分无礼!若论你百计千方骗了我吃管教你一个个遭肿头天瘟!”呆子骂了又骂嚷了又嚷不题。

却说孙大圣与沙僧正坐只见一阵腥风刮面而过他就打了一个喷嚏道:“不好!不好!这阵风凶多吉少。想是猪八戒走错路也。”沙僧道:“他错了路不会问人?”行者道:“想必撞见妖精了。”沙僧道:“撞见妖精他不会跑回?”行者道:“不停当。你坐在这里看守等我跑过涧去打听打听。”沙僧道:“师兄腰疼只恐又着他手等小弟去罢。”行者道:“你不济事还让我去。”好行者咬着牙忍着疼捻着铁棒走过涧到那火云洞前叫声“泼怪!”那把门的小妖又急入里报:“孙行者又在门叫哩!”那妖王传令叫拿那伙小妖枪刀簇拥齐声呐喊即开门都道:“拿住!拿住!”行者果然疲倦不敢相迎将身钻在路旁念个咒语叫“变!”即变做一个销金包袱。小妖看见报道:“大王孙行者怕了只见说一声拿字慌得把包袱丢下走了。”妖王笑道:“那包袱也无甚么值钱之物左右是和尚的破褊衫旧帽子背进来拆洗做补衬。”一个小妖果将包袱背进不知是行者变的。行者道:“好了!这个销金包袱背着了!”那妖精不以为事丢在门内。

好行者假中又假虚里还虚:即拔一根毫毛吹口仙气变作个包袱一样;他的真身却又变作一个苍蝇儿钉在门枢上。只听得八戒在那里哼哩哼的声音不清却似一个瘟猪。行者嘤的飞了去寻时原来他吊在皮袋里也。行者钉在皮袋又听得他恶言恶语骂道妖怪长妖怪短“你怎么假变作个观音菩萨哄我回来吊我在此还说要吃我!有一日我师兄大展齐天无量法满山泼怪登时擒!解开皮袋放我出筑你千钯方趁心!”行者闻言暗笑道:“这呆子虽然在这里面受闷气却还不倒了旗枪。老孙一定要拿了此怪若不如此怎生雪恨!”正欲设法拯救八戒出来只听那妖王叫道:“六健将何在?”时有六个小妖是他知己的精灵封为健将都有名字:一个叫做云里雾一个叫做雾里云一个叫做急如火一个叫做快如风一个叫做兴烘掀一个叫做掀烘兴。六健将上前跪下妖王道:

“你们认得老大王家么?”六健将道:“认得。”妖王道:“你与我星夜去请老大王来说我这里捉唐僧蒸与他吃寿延千纪。”六怪领命一个个厮拖厮扯径出门去了。行者嘤的一声飞下袋来跟定那六怪躲离洞中。毕竟不知怎的请来且听下回分解。
------------


------------

第四十三回 黑河妖孽擒僧去 西洋龙子捉鼍回

却说那菩萨念了几遍却才住口那妖精就不疼了。又正性起身看处颈项里与手足上都是金箍勒得疼痛便就除那箍儿时莫想褪得动分毫这宝贝已此是见肉生根越抹越痛。

行者笑道:“我那乖乖菩萨恐你养不大与你戴个颈圈镯头哩。”那童子闻此言又生烦恼就此绰起枪来望行者乱刺。行者急闪身立在菩萨后面叫:“念咒!念咒!”那菩萨将杨柳枝儿蘸了一点甘露洒将去叫声“合!”只见他丢了枪一双手合掌当胸再也不能开放至今留了一个观音扭即此意也。那童子开不得手拿不得枪方知是法力深微没奈何才纳头下拜。菩萨念动真言把净瓶敧倒将那一海水依然收去更无半点存留对行者道:“悟空这妖精已是降了却只是野心不定等我教他一步一拜只拜到落伽山方才收法。你如今快早去洞中救你师父去来!”行者转身叩头道:“有劳菩萨远涉弟子当送一程。”菩萨道:“你不消送恐怕误了你师父性命。”行者闻言欢喜叩别。那妖精早归了正果五十三参参拜观音且不题善菩萨收了童子。

却说那沙僧久坐林间盼望行者不到将行李捎在马上一只手执着降妖宝杖一只手牵着缰绳出松林向南观看。只见行者欣喜而来。沙僧迎着道:“哥哥你怎么去请菩萨此时才来!焦杀我也!”行者道:“你还做梦哩老孙已请了菩萨降了妖怪。”行者却将菩萨的法力备陈了一遍。沙僧十分欢喜道:“救师父去也!”他两个才跳过涧去撞到门前拴下马匹举兵器齐打入洞里剿净了群妖解下皮袋放出八戒来。那呆子谢了行者道:“哥哥那妖精在那里?等我去筑他几钯出出气来!”行者道:“且寻师父去。”三人径至后边只见师父赤条条捆在院中哭哩。沙僧连忙解绳行者即取衣服穿上三人跪在面前道:“师父吃苦了。”三藏谢道:“贤徒啊多累你等怎生降得妖魔也?”行者又将请菩萨、收童子之言备陈一遍。三藏听得即忙跪下朝南礼拜。行者道:“不消谢他转是我们与他作福收了一个童子。”如今说童子拜观音五十三参参参见佛即此是也。教沙僧将洞内宝物收了且寻米粮安排斋饭管待了师父。那长老得性命全亏孙大圣取真经只靠美猴精。

师徒们出洞来攀鞍上马找大路笃志投西。

行经一个多月忽听得水声振耳三藏大惊道!徒弟呀又是那里水声?”行者笑道:“你这老师父忒也多疑做不得和尚。我们一同四众偏你听见甚么水声。你把那《多心经》又忘了也?”唐僧道:“多心经乃浮屠山乌巢禅师口授共五十四句二百七十个字。我当时耳传至今常念你知我忘了那句儿?”

行者道:“老师父你忘了‘无眼耳鼻舌身意’。我等出家人眼不视色耳不听声鼻不嗅香舌不尝味身不知寒暑意不存妄想如此谓之祛褪六贼。你如今为求经念念在意怕妖魔不肯舍身要斋吃动舌喜香甜嗅鼻闻声音惊耳睹事物凝眸招来这六贼纷纷怎生得西天见佛?”三藏闻言默然沉虑道:

“徒弟啊我一自当年别圣君奔波昼夜甚殷勤。芒鞋踏破山头雾竹笠冲开岭上云。夜静猿啼殊可叹月明鸟噪不堪闻。何时满足三三行得取如来妙法文?”行者听毕忍不住鼓掌大笑道:“这师父原来只是思乡难息!若要那三三行满有何难哉!

常言道功到自然成哩。”八戒回头道:“哥啊若照依这般魔障凶高就走上一千年也不得成功!”沙僧道:“二哥你和我一般拙口钝腮不要惹大哥热擦。且只捱肩磨担终须有日成功也。”

师徒们正话间脚走不停马蹄正疾见前面有一道黑水滔天马不能进。四众停立岸边仔细观看但见那:层层浓浪迭迭浑波层层浓浪翻乌潦迭迭浑波卷黑油。近观不照人身影远望难寻树木形。滚滚一地墨滔滔千里灰。水沫浮来如积炭浪花飘起似翻煤。牛羊不饮鸦鹊难飞。牛羊不饮嫌深黑鸦鹊难飞怕渺弥。只是岸上芦蘋知节令滩头花草斗青奇。

湖泊江河天下有溪源泽洞世间多。人生皆有相逢处谁见西方黑水河!唐僧下马道:“徒弟这水怎么如此浑黑?”八戒道:

“是那家泼了靛缸了。mianhuatang.la [棉花糖小说网]”沙僧道:“不然是谁家洗笔砚哩。”行者道:“你们且休胡猜乱道且设法保师父过去。”八戒道:“这河若是老猪过去不难或是驾了云头或是下河负水不消顿饭时我就过去了。”沙僧道:“若教我老沙也只消纵云躧水顷刻而过。”行者道:“我等容易只是师父难哩。”三藏道:“徒弟啊这河有多少宽么?”八戒道:“约摸有十来里宽。”三藏道:

“你三个计较着那个驮我过去罢。”行者道:“八戒驮得。”八戒道:“不好驮。若是驮着腾云三尺也不能离地常言道背凡人重若丘山。若是驮着负水转连我坠下水去了。”

师徒们在河边正都商议只见那上溜头有一人棹下一只小船儿来。唐僧喜道:“徒弟有船来了。叫他渡我们过去。”

沙僧厉声高叫道:棹船的来渡人!来渡人!”船上人道:“我不是渡船如何渡人?”沙僧道:“天上人间方便第一。你虽不是渡船我们也不是常来打搅你的。我等是东土钦差取经的佛子你可方便方便渡我们过去谢你。”那人闻言却把船儿棹近岸边扶着桨道:“师父啊我这船小你们人多怎能全渡?”

三藏近前看了那船儿原来是一段木头刻的中间只有一个舱口只好坐下两个人。三藏道:“怎生是好?”沙僧道:“这般啊两遭儿渡罢。”八戒就使心术要躲懒讨乖道:“悟净你与大哥在这边看着行李马匹等我保师父先过去却再来渡马。教大哥跳过去罢。”行者点头道:“你说的是。”

那呆子扶着唐僧那梢公撑开船举棹冲流一直而去。方才行到中间只听得一声响喨卷浪翻波遮天迷目。那阵狂风十分利害!好风:当空一片炮云起中溜千层黑浪高。两岸飞沙迷日色四边树倒振天号。翻江搅海龙神怕播土扬尘花木凋。呼呼响若春雷吼阵阵凶如饿虎哮。蟹鳖鱼虾朝上拜飞禽走兽失窝巢。五湖船户皆遭难四海人家命不牢。溪内渔翁难把钩河间梢子怎撑篙?揭瓦翻砖房屋倒惊天动地泰山摇。

这阵风原来就是那棹船人弄的他本是黑水河中怪物。眼看着那唐僧与猪八戒连船儿淬在水里无影无形不知摄了那方去也。

这岸上沙僧与行者心慌道:“怎么好?老师父步步逢灾才脱了魔障幸得这一路平安又遇着黑水迍邅!”沙僧道:“莫是翻了船我们往下溜头找寻去。”行者道:“不是翻船。若翻船八戒会水他必然保师父负水而出。我才见那个棹船的有些不正气想必就是这厮弄风把师父拖下水去了。”沙僧闻言道:“哥哥何不早说你看着马与行李等我下水找寻去来。”行者道:“这水色不正恐你不能去。”沙僧道:“这水比我那流沙河如何?去得!去得!”

好和尚脱了褊衫札抹了手脚轮着降妖宝杖扑的一声分开水路钻入波中大踏步行将进去。正走处只听得有人言语。沙僧闪在旁边偷睛观看那壁厢有一座亭台台门外横封了八个大字乃是“衡阳峪黑水河神府”。又听得那怪物坐在上面道:“一向辛苦今日方能得物。这和尚乃十世修行的好人但得吃他一块肉便做长生不老人。我为他也等够多时今朝却不负我志。”教:“小的们!快把铁笼抬出来将这两个和尚囫囵蒸熟具柬去请二舅爷来与他暖寿。”沙僧闻言按不住心头火起掣宝杖将门乱打口中骂道:“那泼物快送我唐僧师父与八戒师兄出来!”唬得那门内妖邪急跑去报:“祸事了!”老怪问:“甚么祸事?”小妖道:“外面有一个晦气色脸的和尚打着前门骂要人哩!”那怪闻言即唤取披挂。小妖抬出披挂老妖结束整齐手提一根竹节钢鞭走出门来真个是凶顽毒像。但见:方面圜睛霞彩亮卷唇巨口血盆红。几根铁线稀髯摆两鬓朱砂乱蓬。形似显灵真太岁貌如怒狠雷公。身披铁甲团花灿头戴金盔嵌宝浓。竹节钢鞭提手内行时滚滚拽狂风。生来本是波中物脱去原流变化凶。要问妖邪真姓字前身唤做小鼍龙。那怪喝道:“是甚人在此打我门哩!”沙僧道:

“我把你个无知的泼怪!你怎么弄玄虚变作梢公架船将我师父摄来?快早送还饶你性命!”那怪呵呵笑道:“这和尚不知死活!你师父是我拿了如今要蒸熟了请人哩!你上来与我见个雌雄!三合敌得我啊还你师父;如三合敌不得连你一都蒸吃了休想西天去也!”沙僧闻言大怒轮宝杖劈头就打。那怪举钢鞭急架相迎。两个在水底下这场好杀:降妖杖、竹节鞭二人怒各争先。一个是黑水河中千载怪一个是灵霄殿外旧时仙。那个因贪三藏肉中吃这个为保唐僧命可怜。都来水底相争斗各要功成两不然。杀得虾鱼对对摇头躲蟹鳖双双缩潜。只听水府群妖齐擂鼓门前众怪乱争喧。好个沙门真悟净单身独力展威权!跃浪翻波无胜败鞭迎杖架两牵连。

算来只为唐和尚欲取真经拜佛天。他二人战经三十回合不见高低。沙僧暗想道:“这怪物是我的对手枉自不能取胜且引他出去教师兄打他。”这沙僧虚丢了个架子拖着宝杖就走。那妖精更不赶来道:“你去罢我不与你斗了我且具柬帖儿去请客哩。”

沙僧气呼呼跳出水来见了行者道:“哥哥这怪物无礼。”

行者问:“你下去许多时才出来端的是甚妖邪?可曾寻见师父?”沙僧道:“他这里边有一座亭台台门外横书八个大字唤做‘衡阳峪黑水河神府’。我闪在旁边听着他在里面说话教小的们刷洗铁笼待要把师父与八戒蒸熟了去请他舅爷来暖寿。是我起怒来就去打门。那怪物提一条竹节钢鞭走出来与我斗了这半日约有三十合不分胜负。我却使个佯输法要引他出来着你助阵。那怪物乖得紧他不来赶我只要回去具柬请客我才上来了。”行者道:“不知是个甚么妖邪?”

沙僧道:“那模样象一个大鳖;不然便是个鼍龙也。”行者道:

“不知那个是他舅爷?”说不了只见那下湾里走出一个老人远远的跪下叫:“大圣黑水河河神叩头。”行者道:“你莫是那棹船的妖邪又来骗我么?”那老人磕头滴泪道:“大圣我不是妖邪我是这河内真神。那妖精旧年五月间从西洋海趁大潮来于此处就与小神交斗。奈我年迈身衰敌他不过把我坐的那衡阳峪黑水河神府就占夺去住了又伤了我许多水族。我却没奈何径往海内告他。原来西海龙王是他的母舅不准我的状子教我让与他住。我欲启奏上天奈何神微职小不能得见玉帝。今闻得大圣到此特来参拜投生万望大圣与我出力报冤!”行者闻言道:“这等说四海龙王都该有罪。他如今摄了我师父与师弟扬言要蒸熟了去请他舅爷暖寿我正要拿他幸得你来报信。这等啊你陪着沙僧在此看守等我去海中先把那龙王捉来教他擒此怪物。”河神道:“深感大圣大恩!”

行者即驾云径至西洋大海按筋斗捻了避水诀分开波浪。正然走处撞见一个黑鱼精棒着一个浑金的请书匣儿从下流头似箭如梭钻将上来被行者扑个满面掣铁棒分顶一下可怜就打得脑浆迸出腮骨查开嗗都的一声飘出水面。他却揭开匣儿看处里边有一张简帖上写着:“愚甥鼍洁顿百拜启上二舅爷敖老大人台下:向承佳惠感感。今因获得二物乃东土僧人实为世间之罕物。甥不敢自用。因念舅爷圣诞在迩特设菲筵预祝千寿。万望车驾临是荷!”行者笑道:

“这厮却把供状先递与老孙也!”正才袖了帖子往前再行。早有一个探海的夜叉望见行者急抽身撞上水晶宫报大王:“齐天大圣孙爷爷来了!”那龙王敖顺即领众水族出宫迎接道:“大圣请入小宫少座献茶。”行者道:“我还不曾吃你的茶你倒先吃了我的酒也!”龙王笑道:“大圣一向皈依佛门不动荤酒却几时请我吃酒来?”行者道:“你便不曾去吃酒只是惹下一个吃酒的罪名了。”敖顺大惊道:“小龙为何有罪?”行者袖中取出简帖儿递与龙王。龙王见了魂飞魄散慌忙跪下叩头道:

“大圣恕罪!那厮是舍妹第九个儿子。因妹夫错行了风雨刻减了雨数被天曹降旨着人曹官魏征丞相梦里斩了。舍妹无处安身是小龙带他到此恩养成*人。前年不幸舍妹疾故惟他无方居住我着他在黑水河养性修真不期他作此恶孽小龙即差人去擒他来也。”行者道:“你令妹共有几个贤郎?都在那里作怪?”龙王道:“舍妹有九个儿子。那八个都是好的。第一个小黄龙见居淮渎;第二个小骊龙见住济渎;第三个青背龙占了江渎;第四个赤髯龙镇守河渎;第五个徒劳龙与佛祖司钟;第六个稳兽龙与神官镇脊;第七个敬仲龙与玉帝守擎天华表;第八个蜃龙在大家兄处砥据太岳。此乃第九个鼍龙因年幼无甚执事自旧年才着他居黑水河养性待成名别迁调用谁知他不遵吾旨冲撞大圣也。”行者闻言笑道:“你妹妹有几个妹丈?”敖顺道:“只嫁得一个妹丈乃泾河龙王。向年已此被斩舍妹孀居于此前年疾故了。”行者道:“一夫一妻如何生这几个杂种?”敖顺道:“此正谓龙生九种九种各别。”

行者道:“我才心中烦恼欲将简帖为证上奏天庭问你个通同作怪抢夺人口之罪。据你所言是那厮不遵教诲我且饶你这次:一则是看你昆玉分上二来只该怪那厮年幼无知你也不甚知情。你快差人擒来救我师父!再作区处。”敖顺即唤太子摩昂:“快点五百虾鱼壮兵将小鼍捉来问罪!”一壁厢安排酒席与大圣陪礼。行者道:“龙王再勿多心既讲开饶了你便罢又何须办酒?我今须与你令郎同回:一则老师父遭愆二则我师弟盼望。”那老龙苦留不住又见龙女捧茶来献。行者立饮他一盏香茶别了老龙随与摩昂领兵离了西海。早到黑水河中行者道:“贤太子好生捉怪我上岸去也。”摩昂道:“大圣宽心小龙子将他拿上来先见了大圣惩治了他罪名把师父送上来才敢带回海内见我家父。”行者欣然相别捏了避水诀跳出波津径到了东边崖上。沙僧与那河神迎着道:“师兄你去时从空而去怎么回来却自河内而回?”行者把那打死鱼精得简帖见龙王与太子同领兵来之事备陈了一遍。沙僧十分欢喜。都立在岸边候接师父不题。

却说那摩昂太子着介士先到他水府门前报与妖怪道:

“西海老龙王太子摩昂来也。”那怪正坐忽闻摩昂来心中疑惑道:“我差黑鱼精投简帖拜请二舅爷这早晚不见回话怎么舅爷不来却是表兄来耶?”正说间只见那巡河的小怪又来报:“大王河内有一枝兵屯于水府之西旗号上书着‘西海储君摩昂小帅’。”妖怪道:“这表兄却也狂妄:想是舅爷不得来命他来赴宴既是赴宴如何又领兵劳士?咳!但恐其间有故。”

教:“小的们将我的披挂钢鞭伺候恐一时变暴待我且出去迎他看是何如。”众妖领命一个个擦掌摩拳准备。

这鼍龙出得门来真个见一枝海兵札营在右只见:征旗飘绣带画戟列明霞。宝剑凝光彩长枪缨绕花。弓弯如月小箭插似狼牙。大刀光灿灿短棍硬沙沙。鲸鳌并蛤蚌蟹鳖共鱼虾。大小齐齐摆干戈似密麻。不是元戎令谁敢乱爬喳!鼍怪见了径至那营门前厉声高叫:“大表兄小弟在此拱候有请。”有一个巡营的螺螺急至中军帐:“报千岁殿下外有鼍龙叫请哩。”太子按一按顶上金盔束一束腰间宝带手提一根三棱简拽开步跑出营去道:“你来请我怎么?”鼍龙进礼道:“小弟今早有简帖拜请舅爷想是舅爷见弃着表兄来的兄长既来赴席如何又劳师动众不入水府札营在此又贯甲提兵何也?”太子道:“你请舅爷做甚?”妖怪道:“小弟一向蒙恩赐居于此久别尊颜未得孝顺。昨日捉得一个东土僧人我闻他是十世修行的元体人吃了他可以延寿欲请舅爷看过上铁笼蒸熟与舅爷暖寿哩。”太子喝道:“你这厮十分懵懂!你道僧人是谁?”妖怪道:“他是唐朝来的僧人往西天取经的和尚。”太子道:“你只知他是唐僧不知他手下徒弟利害哩。”妖怪道:

“他有一个长嘴的和尚唤做个猪八戒我也把他捉住了要与唐和尚一同蒸吃。还有一个徒弟唤做沙和尚乃是一条黑汉子晦气色脸使一根宝杖昨日在这门外与我讨师父被我帅出河兵一顿钢鞭战得他败阵逃生也不见怎的利害。”太子道:“原来是你不知!他还有一个大徒弟是五百年前大闹天宫上方太乙金仙齐天大圣如今保护唐僧往西天拜佛求经是普陀岩大慈大悲观音菩萨劝善与他改名唤做孙悟空行者。你怎么没得做撞出这件祸来?他又在我海内遇着你的差人夺了请帖径入水晶宫拿捏我父子们有结连妖邪抢夺人口之罪。你快把唐僧、八戒送上河边交还了孙大圣凭着我与他陪礼你还好得性命若有半个不字休想得全生居于此也!”那怪鼍闻此言心中大怒道:“我与你嫡亲的姑表你倒反护他人?听你所言就教把唐僧送出天地间那里有这等容易事也!

你便怕他莫成我也怕他?他若有手段敢来我水府门前与我交战三合我才与他师父若敌不过我就连他也拿来一齐蒸熟也没甚么亲人也不去请客自家关了门教小的们唱唱舞舞我坐在上面自自在在吃他娘不是!”太子见说开口骂道:“这泼邪果然无状!且不要教孙大圣与你对敌你敢与我相持么?”那怪道:“要做好汉怕甚么相持!”教:“取披挂!”呼唤一声众小妖跟随左右献上披挂捧上钢鞭。他两个变了脸各逞英雄;传号令一齐擂鼓。这一场比与沙僧争斗甚是不同但见那:旌旗照耀戈戟摇光。这壁厢营盘解散那壁厢门户开张。摩昂太子提金简鼍怪轮鞭急架偿。一声炮响河兵烈三棒锣鸣海士狂。虾与虾争蟹与蟹斗。鲸鳌吞赤鲤鯾鲌起黄鱨。鲨鲻吃鮆鲭鱼走牡蛎擒蛏蛤蚌慌少扬刺硬如铁棍鱑司针利似锋芒。鲆鱑追白蟮鲈鲙捉乌鲳。一河水怪争高下两处龙兵定弱强。混战多时波浪滚摩昂太子赛金刚。喝声金简当头重拿住妖鼍作怪王。这太子将三棱简闪了一个破绽那妖精不知是诈钻将进来被他使个解数把妖精右臂只一简打了个躘踵赶上前又一拍脚跌倒在地。众海兵一拥上前揪翻住将绳子背绑了双手将铁索穿了琵琶骨拿上岸来押至孙行者面前道:“大圣小龙子捉住妖鼍请大圣定夺。”

行者与沙僧见了道:“你这厮不遵旨令你舅爷原着你在此居住教你养性存身待你名成之日别有迁用。你怎么强占水神之宅倚势行凶欺心诳上弄玄虚骗我师父、师弟?我待要打你这一棒奈何老孙这棒子甚重略打打儿就了了性命。

你将我师父安在何处哩?”那怪叩头不住道:“大圣小鼍不知大圣大名却才逆了表兄骋强背理被表兄把我拿住。今见大圣幸蒙大圣不杀之恩感谢不尽。你师父还捆在那水府之间望大圣解了我的铁索放了我手等我到河中送他出来。”摩昂在旁道:“大圣这厮是个逆怪他极奸诈若放了他恐生恶念。”沙和尚道:“我认得他那里等我寻师父去。”他两个跳入水中径至水府门前那里门扇大开更无一个小卒。直入亭台里面见唐僧八戒赤条条都捆在那里。沙僧即忙解了师父河神亦随解了八戒一家背着一个出水面径至岸边。猪八戒见那妖精锁绑在侧急掣钯上前就筑口里骂道:“泼邪畜!你如今不吃我了?”行者扯住道:“兄弟且饶他死罪罢看敖顺贤父子之情。”摩昂进礼道:“大圣小龙子不敢久停。既然救得你师父我带这厮去见家父;虽大圣饶了他死罪家父决不饶他活罪定有落处置仍回复大圣谢罪。”行者道:“既如此你领他去罢多多拜上令尊尚容面谢。”那太子押着那妖鼍投水中帅领海兵径转西洋大海不题。

却说那黑水河神谢了行者道:“多蒙大圣复得水府之恩!”

唐僧道:“徒弟啊如今还在东岸如何渡此河也?”河神道:“老爷勿虑且请上马小神开路引老爷过河。”那师父才骑了白马八戒采着缰绳沙和尚挑了行李孙行者扶持左右只见河神作起阻水的法术将上流挡住。须臾下流撤干开出一条大路。师徒们行过西边谢了河神登崖上路。这正是:禅僧有救来西域彻地无波过黑河。毕竟不知怎生得拜佛求经且听下回分解。
------------

第四十四回 法身元运逢车力 心正妖邪度脊关

诗曰:求经脱障向西游无数名山不尽休。兔走乌飞催昼夜鸟啼花落自春秋。微尘眼底三千界锡杖头边四百州。宿水餐风登紫陌未期何日是回头。话说唐三藏幸亏龙子降妖黑水河神开路师徒们过了黑水河找大路一直西来。真个是迎风冒雪戴月披星行彀多时又值早春天气但见三阳转运万物生辉。三阳转运满天明媚开图画;万物生辉遍地芳菲设绣茵。梅残数点雪麦涨一川云。渐开冰解山泉溜尽放萌芽没烧痕。正是那太昊乘震勾芒御辰花香风气暖云淡日光新。道旁杨柳舒青眼膏雨滋生万象春。师徒们在路上游观景色缓马而行忽听得一声吆喝好便似千万人呐喊之声。唐三藏心中害怕兜住马不能前进急回头道:悟空是那里这等响振?”八戒道:“好一似地裂山崩。”沙僧道:“也就如雷声霹雳。”三藏道:“还是人喊马嘶。”孙行者笑道:“你们都猜不着且住待老孙看是何如。”

好行者将身一纵踏云光起在空中睁眼观看远见一座城池。又近觑倒也祥光隐隐不见甚么凶气纷纷。行者暗自沉吟道:“好去处!如何有响声振耳?那城中又无旌旗闪灼戈戟光明又不是炮声响振何以若人马喧哗?”正议间只见那城门外有一块沙滩空地攒簇了许多和尚在那里扯车儿哩。

原来是一齐着力打号齐喊“大力王菩萨”所以惊动唐僧。行者渐渐按下云头来看处呀!那车子装的都是砖瓦木植土坯之类;滩头上坡坂最高又有一道夹脊小路两座大关关下之路都是直立壁陡之崖那车儿怎么拽得上去?虽是天色和暖那些人却也衣衫蓝缕看此象十分窘迫。行者心疑道:“想是修盖寺院。他这里五谷丰登寻不出杂工人来所以这和尚亲自努力。”正自猜疑未定只见那城门里摇摇摆摆走出两个少年道士来。你看他怎生打扮但见他:头戴星冠身披锦绣。头戴星冠光耀耀身披锦绣彩霞飘。足踏云头履腰系熟丝绦。面如满月多聪俊形似瑶天仙客娇。那些和尚见道士来一个个心惊胆战加倍着力恨苦的拽那车子。行者就晓得了:“咦!想必这和尚们怕那道士。不然啊怎么这等着力拽扯?我曾听得人言西方路上有个敬道灭僧之处断乎此间是也。我待要回报师父奈何事不明白返惹他怪敢道这等一个伶俐之人就不能探个实信?且等下去问得明白好回师父话。

你道他来问谁?好大圣按落云头去郡城脚下摇身一变变做个游方的云水全真左臂上挂着一个水火篮儿手敲着渔鼓口唱着道情词近城门迎着两个道士当面躬身道:

“道长贫道起手。”那道士还礼道:“先生那里来的?”行者道:

“我弟子云游于海角浪荡在天涯;今朝来此处欲募善人家。

动问二位道长这城中那条街上好道?那个巷里好贤?我贫道好去化些斋吃。”那道士笑道:“你这先生怎么说这等败兴的话?”行者道:“何为败兴?”道士道:“你要化些斋吃却不是败兴?”行者道:“出家人以乞化为由却不化斋吃怎生有钱买?”

道士笑道:“你是远方来的不知我这城中之事。我这城中且休说文武官员好道富民长者爱贤大男小女见我等拜请奉斋这般都不须挂齿头一等就是万岁君王好道爱贤。”行者道:“我贫道一则年幼二则是远方乍来实是不知。烦二位道长将这里地名、君王好道爱贤之事细说一遍足见同道之情。”道士说:“此城名唤车迟国宝殿上君王与我们有亲。”行者闻言呵呵笑道:“想是道士做了皇帝?”他道:“不是。只因这二十年前民遭亢旱天无点雨地绝谷苗不论君臣黎庶大小人家家家沐浴焚香户户拜天求雨。正都在倒悬捱命之处忽然天降下三个仙长来俯救生灵。”行者问道:“是那三个仙长?”道士说:“便是我家师父。”行者道:“尊师甚号?”道士云:

“我大师父号做虎力大仙;二师父鹿力大仙;三师父羊力大仙。”行者问曰:“三位尊师有多少法力?”道士云:“我那师父呼风唤雨只在翻掌之间指水为油点石成金却如转身之易。所以有这般法力能夺天地之造化换星斗之玄微君臣相敬与我们结为亲也。”行者道:“这皇帝十分造化。常言道术动公卿。老师父有这般手段结了亲其实不亏他。噫不知我贫道可有星星缘法得见那老师父一面哩?”道士笑曰:“你要见我师父。有何难处!我两个是他靠胸贴肉的徒弟我师父却又好道爱贤只听见说个道字就也接出大门。若是我两个引进你乃吹灰之力。”行者深深的唱个大喏道:“多承举荐就此进去罢。”道士说:“且少待片时你在这里坐下等我两个把公事干了来和你进去。”行者道:“出家人无拘无束自由自在有甚公干?”道士用手指定那沙滩上僧人:“他做的是我家生活恐他躲懒我们去点他一卯就来。’行者笑道:“道长差了!

僧道之辈都是出家人为何他替我们做活伏我们点卯?”道士云:“你不知道因当年求雨之时僧人在一边拜佛道士在一边告斗都请朝廷的粮饷;谁知那和尚不中用空念空经不能济事。后来我师父一到唤雨呼风拔济了万民涂炭。却才恼了朝廷说那和尚无用拆了他的山门毁了他的佛像追了他的度牒不放他回乡御赐与我们家做活就当小厮一般。我家里烧火的也是他扫地的也是他顶门的也是他。因为后边还有住房未曾完备着这和尚来拽砖瓦拖木植起盖房宇。只恐他贪顽躲懒不肯拽车所以着我两个去查点查点。”行者闻言。扯住道士滴泪道:“我说我无缘真个无缘不得见老师父尊面!”道士云:“如何不得见面?”行者道:“我贫道在方上云游一则是为性命二则也为寻亲。”道士问:“你有甚么亲?”行者道:“我有一个叔父自幼出家削为僧向日年程饥馑也来外面求乞。这几年不见回家我念祖上之恩特来顺便寻访想必是羁迟在此等地方不能脱身未可知也。我怎的寻着他见一面才可与你进城?”道士云:“这般却是容易。我两个且坐下即烦你去沙滩上替我一查只点头目有五百名数目便罢看内中那个是你令叔。果若有呀我们看道中情分放他去了却与你进城好么?”

行者顶谢不尽长揖一声别了道士敲着渔鼓径往沙滩之上。过了双关转下夹脊那和尚一齐跪下磕头道:“爷爷我等不曾躲懒五百名半个不少都在此扯车哩。”行者看见暗笑道:“这些和尚被道士打怕了见我这假道士就这般悚惧若是个真道士好道也活不成了。”行者又摇手道:“不要跪休怕。我不是监工的我来此是寻亲的。”众僧们听说认亲就把他圈子阵围将上来一个个出头露面咳嗽打响巴不得要认出去。道:“不知那个是他亲哩。”行者认了一会呵呵笑将起来众僧道:“老爷不认亲如何笑?”行者道:“你们知我笑甚么?笑你这些和尚全不长俊!父母生下你来皆因命犯华盖妨爷克娘或是不招姊妹才把你舍断了出家。你怎的不遵三宝不敬佛法不去看经拜忏却怎么与道士佣工作奴婢使唤?”众僧道:“老爷你来羞我们哩!你老人家想是个外边来的不知我这里利害。”行者道:“果是外方来的其实不知你这里有甚利害。”众僧滴泪道:“我们这一国君王偏心无道只喜得是老爷等辈恼的是我们佛子。”行者道:“为何来?”众僧道:

“只因呼风唤雨三个仙长来此处灭了我等哄信君王把我们寺拆了度牒追了不放归乡亦不许补役当差赐与那仙长家使用苦楚难当!但有个游方道者至此即请拜王领赏;若是和尚来不分远近就拿来与仙长家佣工。”行者道:“想必那道士还有甚么巧法术诱了君王?若只是呼风唤雨也都是旁门小法术耳安能动得君心?”众僧道:“他会抟砂炼汞打坐存神点水为油点石成金。如今兴盖三清观宇对天地昼夜看经忏悔祈君王万年不老所以就把君心惑动了。”行者道:“原来这般你们都走了便罢。”众僧道:“老爷走不脱!那仙长奏准君王把我们画了影身图四下里长川张挂。他这车迟国地界也宽各府州县乡村店集之方都有一张和尚图上面是御笔亲题。若有官职的拿得一个和尚高升三级;无官职的拿得一个和尚就赏白银五十两所以走不脱。且莫说是和尚就是剪鬃、秃子、毛稀的都也难逃。四下里快手又多缉事的又广凭你怎么也是难脱。我们没奈何只得在此苦捱。”行者道:“既然如此你们死了便罢。”众僧道:“老爷有死的。到处捉来与本处和尚也共有二千余众到此熬不得苦楚受不得爊煎忍不得寒冷服不得水土死了有六七百自尽了有七八百只有我这五百个不得死。”行者道:“怎么不得死?”众僧道:“悬梁绳断刀刎不疼投河的飘起不沉服药的身安不损。”行者道:

“你却造化天赐汝等长寿哩!”众僧道:“老爷呀你少了一个字儿是长受罪哩!我等日食三餐乃是糙米熬得稀粥到晚就在沙滩上冒露安身才合眼就有神人拥护。”行者道:“想是累苦了见鬼么?”众僧道:“不是鬼乃是六丁六甲、护教伽蓝但至夜就来保护。但有要死的就保着不教他死。”行者道:“这些神却也没理只该教你们早死早升天却来保护怎的?”众僧道:“他在梦寐中劝解我们教不要寻死且苦捱着等那东土大唐圣僧往西天取经的罗汉。他手下有个徒弟乃齐天大圣神通广大专秉忠良之心与人间报不平之事济困扶危恤孤念寡。只等他来显神通灭了道士还敬你们沙门禅教哩。”

行者闻得此言心中暗笑道:“莫说老孙无手段预先神圣早传名。”他急抽身敲着渔鼓别了众僧径来城门口见了道士。那道士迎着道:“先生那一位是令亲?”行者道:“五百个都与我有亲。”两个道士笑道:“你怎么就有许多亲?”行者道:“一百个是我左邻一百个是我右舍一百个是我父党一百个是我母党一百个是我交契。你若肯把这五百人都放了我便与你进去;不放我不去了。”道士云:“你想有些风病一时间就胡说了。那些和尚乃国王御赐若放一二名还要在师父处递了病状然后补个死状才了得哩。怎么说都放了?此理不通!

不通!且不要说我家没人使唤就是朝廷也要怪。他那里长要差官查勘或时御驾也亲来点札怎么敢放?”行者道:“不放么?”道士说:“不放!”行者连问三声就怒将起来把耳朵里铁棒取出迎风捻了一捻就碗来粗细幌了一幌照道士脸上一刮可怜就打得头破血流身倒地皮开颈折脑浆倾!那滩上僧人远远望见他打杀了两个道士丢了车儿跑将上来道:“不好了!不好了!打杀皇亲了!”行者道:“那个是皇亲?”众僧把他簸箕阵围了道:“他师父上殿不参王下殿不辞主朝廷常称做国师兄长先生。你怎么到这里闯祸?他徒弟出来监工与你无干你怎么把他来打死?那仙长不说是你来打杀只说是来此监工我们害了他性命我等怎了?且与你进城去会了人命出来。”行者笑道:“列位休嚷我不是云水全真我是来救你们的。”众僧道:“你倒打杀人害了我们添了担儿如何是救我们的?”行者道:“我是大唐圣僧徒弟孙悟空行者特特来此救你们性命。”众僧道:“不是!不是!那老爷我们认得他。”行者道:“又不曾会他如何认得?”众僧道:“我们梦中尝见一个老者自言太白金星常教诲我等说那孙行者的模样莫教错认了。”行者道:“他和你怎么说来?”众僧道:“他说那大圣:磕额金睛幌亮圆头毛脸无腮。咨牙尖嘴性情乖貌比雷公古怪。惯使金箍铁棒曾将天阙攻开。如今皈正保僧来专救人间灾害。”行者闻言又嗔又喜喜道替老孙传名!嗔道那老贼惫懒把我的元身都说与这伙凡人!忽失声道:“列位诚然认我不是孙行者我是孙行者的门人来此处学闯祸耍子的。那里不是孙行者来了?”用手向东一指哄得众僧回头他却现了本相众僧们方才认得一个个倒身下拜道:“爷爷!我等凡胎肉眼不知是爷爷显化。望爷爷与我们雪恨消灾早进城降邪从正也!”行者道:“你们且跟我来。”众僧紧随左右。

那大圣径至沙滩上使个神通将车儿拽过两关穿过夹脊提起来捽得粉碎把那些砖瓦木植尽抛下坡坂喝教众僧:“散!莫在我手脚边等我明日见这皇帝灭那道士!”众僧道:“爷爷呀我等不敢远走但恐在官人拿住解来却又吃打赎返又生灾。”行者道:“既如此我与你个护身法儿。”好大圣把毫毛拔了一把嚼得粉碎每一个和尚与他一截都教他:“捻在无名指甲里捻着拳头只情走路。无人敢拿你便罢;

若有人拿你攒紧了拳头叫一声齐天大圣我就来护你。”众僧道:“爷爷倘若去得远了看不见你叫你不应怎么是好?”

行者道:“你只管放心就是万里之遥可保全无事。”众僧有胆量大的捻着拳头悄悄的叫声“齐天大圣!”只见一个雷公站在面前手执铁棒就是千军万马也不能近身。此时有百十众齐叫足有百十个大圣护持众僧叩头道:“爷爷!果然灵显!”

行者又吩咐:“叫声寂字还你收了。”真个是叫声“寂!”依然还是毫毛在那指甲缝里。众和尚却才欢喜逃生一齐而散。行者道:“不可十分远遁听我城中消息。但有招僧榜出就进城还我毫毛也。”五百个和尚东的东西的西走的走立的立四散不题。

却说那唐僧在路旁等不得行者回话教猪八戒引马投西遇着些僧人奔走将近城边见行者还与十数个未散的和尚在那里。三藏勒马道:“悟空你怎么来打听个响声许久不回?”行者引了十数个和尚对唐僧马前施礼将上项事说了一遍。三藏大惊道:“这般啊我们怎了?”那十数个和尚道:“老爷放心孙大圣爷爷乃天神降的神通广大定保老爷无虞。我等是这城里敕建智渊寺内僧人。因这寺是先王太祖御造的现有先王太祖神象在内未曾拆毁城中寺院大小尽皆拆了。我等请老爷赶早进城到我荒山安下。待明日早朝孙大圣必有处置。”行者道:“汝等说得是。也罢趁早进城去来。”那长老却才下马行到城门之下此时已太阳西坠。过吊桥进了三层门里街上人见智渊寺的和尚牵马挑包尽皆回避。正行时却到山门前但见那门上高悬着一面金字大匾乃敕建智渊寺。众僧推开门穿过金刚殿把正殿门开了。唐僧取袈裟披起拜毕金身方入。众僧叫:“看家的!”老和尚走出来看见行者就拜道“爷爷!你来了?”行者道:“你认得我是那个爷爷就是这等呼拜?”那和尚道:“我认得你是齐天大圣孙爷爷我们夜夜梦中见你。太白金星常常来托梦说道只等你来我们才得性命。

今日果见尊颜与梦中无异。爷爷呀喜得早来!再迟一两日我等已俱做鬼矣!”行者笑道:“请起请起明日就有分晓。”众僧安排了斋饭他师徒们吃了打扫乾净方丈安寝一宿。

二更时候孙大圣心中有事偏睡不着只听那里吹打悄悄的爬起来穿了衣服跳在空中观看原来是正南上灯烛荧煌。低下云头仔细再看却是三清观道士禳星哩。但见那灵区高殿福地真堂。灵区高殿巍巍壮似蓬壶景;福地真堂隐隐清如化乐宫。两边道士奏笙簧正面高公擎玉简。宣理《消灾忏》开讲《道德经》。扬尘几度尽传符表白一番皆俯伏。咒水檄烛焰飘摇冲上界;查罡布斗香烟馥郁透清霄。案头有供献新鲜桌上有斋筵丰盛。殿门前挂一联黄绫织锦的对句绣着二十二个大字云:“雨顺风调愿祝天尊无量法;河清海晏祈求万岁有余年。”行者见三个老道士披了法衣想是那虎力、鹿力、羊力大仙。下面有七八百个散众司鼓司钟侍香表白尽都侍立两边。行者暗自喜道:“我欲下去与他混一混奈何单丝不线孤掌难鸣且回去照顾八戒沙僧一同来耍耍。”

按落祥云径至方丈中原来八戒与沙僧通脚睡着。行者先叫悟净沙和尚醒来道:“哥哥你还不曾睡哩?”行者道:“你且起来我和你受用些来。”沙僧道:“半夜三更口枯眼涩有甚受用?”行者道:“这城里果有一座三清观。观里道士们修蘸三清殿上有许多供养:馒头足有斗大烧果有五六十斤一个衬饭无数果品新鲜。和你受用去来!”那猪八戒睡梦里听见说吃好东西就醒了道:“哥哥就不带挈我些儿?”行者道:“兄弟你要吃东西不要大呼小叫惊醒了师父都跟我来。”他两个套上衣服悄悄的走出门前随行者踏了云头跳将起去。那呆子看见灯光就要下手行者扯住道:“且休忙待他散了方可下去。”八戒道:“他才念到兴头上却怎么肯散?”行者道:

“等我弄个法儿他就散了。”好大圣捻着诀念个咒语往巽地上吸一口气呼的吹去便是一阵狂风径直卷进那三清殿上把他些花瓶烛台四壁上悬挂的功德一齐刮倒遂而灯火无光。众道士心惊胆战虎力大仙道:“徒弟们且散这阵神风所过吹灭了灯烛香花各人归寝明朝早起多念几卷经文补数。”众道士果各退回。

这行者却引八戒沙僧按落云头闯上三清殿。呆子不论生熟拿过烧果来张口就啃行者掣铁棒着手便打。八戒缩手躲过道:“还不曾尝着甚么滋味就打!”行者道:“莫要小家子行且叙礼坐下受用。”八戒道:“不羞!偷东西吃还要叙礼!

若是请将来却要如何?”行者道:“这上面坐的是甚么菩萨?”

八戒笑道:“三清也认不得却认做甚么菩萨!”行者道:“那三清?”八戒道:“中间的是元始天尊左边的是灵宝道君右边的是太上老君。”行者道:“都要变得这般模样才吃得安稳哩。”

那呆子急了闻得那香喷喷供养要吃爬上高台把老君一嘴拱下去道:“老官儿你也坐得彀了让我老猪坐坐。”八戒变做太上老君行者变做元始天尊沙僧变作灵宝道君把原象都推下去。及坐下时八戒就抢大馒头吃行者道:“莫忙哩!”八戒道:“哥哥变得如此还不吃等甚?”行者道:“兄弟呀吃东西事小泄漏天机事大。这圣象都推在地下倘有起早的道士来撞钟扫地或绊一个根头却不走漏消息?你把他藏过一边来。”八戒道:“此处路生摸门不着却那里藏他?”行者道:“我才进来时那右手下有一重小门儿那里面秽气畜人想必是个五谷轮回之所。你把他送在那里去罢。”这呆子有些夯力量跳下来把三个圣像拿在肩膊上扛将出来。到那厢用脚登开门看时原来是个大东厕笑道:“这个弼马温着然会弄嘴弄舌!把个毛坑也与他起个道号叫做甚么五谷轮回之所!”那呆子扛在肩上且不丢了去口里啯啯哝哝的祷道:“三清三清我说你听:远方到此惯灭妖精欲享供养无处安宁。借你坐位略略少停。你等坐久也且暂下毛坑。你平日家受用无穷做个清净道士;今日里不免享些秽物也做个受臭气的天尊!”祝罢烹的望里一捽灒了半衣襟臭水走上殿来。行者道:“可藏得好么?”八戒道:“藏便藏得好;只是灒起些水来污了衣服有些腌脏臭气你休恶心。”行者笑道:“也罢你且来受用但不知可得个干净身子出门哩。”那呆子还变做老君。三人坐下尽情受用先吃了大馒头后吃簇盘、衬饭、点心、拖炉、饼锭、油煠、蒸酥那里管甚么冷热任情吃起。原来孙行者不大吃烟火食只吃几个果子陪他两个。那一顿如流星赶月风卷残云吃得罄尽已此没得吃了还不走路且在那里闲讲消食耍子。

噫!有这般事!原来那东廊下有一个小道士才睡下忽然起来道:“我的手铃儿忘记在殿上若失落了明日师父见责。”

与那同睡者道“你睡着等我寻去。”急忙中不穿底衣。止扯一领直裰径到正殿中寻铃。摸来摸去铃儿摸着了正欲回头只听得有呼吸之声道士害怕。急拽步往外走时不知怎的躧着一个荔枝核子扑的滑了一跌狢的一声把个铃儿跌得粉碎。猪八戒忍不住呵呵大笑出来把个小道士唬走了三魂惊回了七魄一步一跌撞到后方丈外打着门叫:“师公!不好了!祸事了!”三个老道士还未曾睡即开门问:“有甚祸事?”他战战兢兢道:“弟子忘失了手铃儿因去殿上寻铃只听得有人呵呵大笑险些儿唬杀我也!”老道士闻言即叫:“掌灯来!看是甚么邪物?”一声传令惊动那两廊的道士大大小小都爬起来点灯着火往正殿上观看。不知端的何如且听下回分解。
------------

第四十五回 三清观大圣留名 车迟国猴王显法

却说孙大圣左手把沙和尚捻一把右手把猪八戒捻一把他二人却就省悟坐在高处倥着脸不言不语凭那些道士点灯着火前后照看他三个就如泥塑金装一般模样。虎力大仙道:“没有歹人如何把供献都吃了?”鹿力大仙道:“却象人吃的勾当有皮的都剥了皮有核的都吐出核却怎么不见人形?”羊力大仙道:“师兄勿疑想是我们虔心敬意在此昼夜诵经前后申文又是朝廷名号断然惊动天尊。想是三清爷爷圣驾降临受用了这些供养。趁今仙从未返鹤驾在斯我等可拜告天尊恳求些圣水金丹进与陛下却不是长生永寿见我们的功果也?”虎力大仙道:“说的是。”教:“徒弟们动乐诵经!一壁厢取法衣来等我步罡拜祷。”那些小道士俱遵命两班儿摆列齐整当的一声磬响齐念一卷《黄庭道德真经》。虎力大仙披了法衣擎着玉简对面前舞蹈扬尘拜伏于地朝上启奏道:“诚惶诚恐稽归依。臣等兴教仰望清虚。灭僧鄙俚敬道光辉。敕修宝殿御制庭闱。广陈供养高挂龙旗。通宵秉烛镇日香菲。一诚达上寸敬虔归。今蒙降驾未返仙车。望赐些金丹圣水进与朝廷寿比南山。”八戒闻言心中忐忑默对行者道:“这是我们的不是。吃了东西且不走路只等这般祷祝却怎么答应?”行者又捻一把忽地开口叫声:“晚辈小仙且休拜祝我等自蟠桃会上来的不曾带得金丹圣水待改日再来垂赐。”那些大小道士听见说出话来一个个抖衣而战道:“爷爷呀!活天尊临凡是必莫放好歹求个长生的法儿!”

鹿力大仙上前又拜云:“扬尘顿谨办丹诚。微臣归命俯仰三清。自来此界兴道除僧。国王心喜敬重玄龄。罗天大醮彻夜看经。幸天尊之不弃降圣驾而临庭。俯求垂念仰望恩荣。是必留些圣水与弟子们延寿长生。”沙僧捻着行者默默的道:“哥呀要得紧又来祷告了。”行者道:“与他些罢。”八戒寂寂道:“那里有得?”行者道:“你只看着我我有时你们也都有了。”那道士吹打已毕行者开言道:“那晚辈小仙不须拜伏。我欲不留些圣水与你们恐灭了苗裔;若要与你又忒容易了。”众道闻言一齐俯伏叩头道:“万望天尊念弟子恭敬之意千乞喜赐些须。我弟子广宣道德奏国王普敬玄门。”行者道:

“既如此取器皿来。”那道士一齐顿谢恩。虎力大仙爱强就抬一口大缸放在殿上;鹿力大仙端一砂盆安在供桌之上;羊力大仙把花瓶摘了花移在中间。行者道:“你们都出殿前掩上格子不可泄了天机好留与你些圣水。”众道一齐跪伏丹墀之下掩了殿门。

那行者立将起来掀着虎皮裙撒了一花瓶臊溺。猪八戒见了欢喜道:“哥啊我把你做这几年兄弟只这些儿不曾弄我。我才吃了些东西道要干这个事儿哩。”那呆子揭衣服忽喇喇就似吕梁洪倒下坂来沙沙的溺了一砂盆沙和尚却也撒了半缸依旧整衣端坐在上道:“小仙领圣水。”那些道士推开格子磕头礼拜谢恩抬出缸去将那瓶盆总归一处教:“徒弟取个锺子来尝尝。”小道士即便拿了一个茶钟递与老道士。道士舀出一锺来喝下口去只情抹唇咂嘴鹿力大仙道:

“师兄好吃么?”老道士努着嘴道:“不甚好吃有些酣郸之味。”

羊力大仙道:“等我尝尝。”也喝了一口道:“有些猪溺臊气。”

行者坐在上面听见说出这话儿来已此识破了道:“我弄个手段索性留个名罢。”大叫云:“道号道号你好胡思!那个三清肯降凡基?吾将真姓说与你知。大唐僧众奉旨来西。良宵无事下降宫闱。吃了供养闲坐嬉嬉。蒙你叩拜何以答之?

那里是甚么圣水你们吃的都是我一溺之尿!”那道士闻得此言拦住门一齐动叉钯扫帚瓦块石头没头没脸往里面乱打。

好行者左手挟了沙僧右手挟了八戒闯出门驾着祥光径转智渊寺方丈不敢惊动师父三人又复睡下。

早是五鼓三点那国王设朝聚集两班文武四百朝官但见绛纱灯火光明宝鼎香云叆叇。此时唐三藏醒来叫:“徒弟徒弟伏侍我倒换关文去来。”行者与沙僧、八戒急起身穿了衣服侍立左右道:“上告师父这昏君信着那些道士兴道灭僧恐言语差错不肯倒换关文我等护持师父都进朝去也。”唐僧大喜披了锦襕袈裟。行者带了通关文牒教悟净捧着钵盂悟能拿了锡杖将行囊马匹交与智渊寺僧看守径到五凤楼前对黄门官作礼报了姓名言是东土大唐取经的和尚来此倒换关文烦为转奏。那阁门大使进朝俯伏金阶奏曰:“外面有四个和尚说是东土大唐取经的欲来倒换关文现在五凤楼前候旨。”国王闻奏道:“这和尚没处寻死却来这里寻死!那巡捕官员怎么不拿他解来?”旁边闪过当驾的太师启奏道:

“东土大唐乃南赡部洲号曰中华大国到此有万里之遥路多妖怪。这和尚一定有些法力方敢西来。望陛下看中华之远僧且召来验牒放行庶不失善缘之意。”国王准奏把唐僧等宣至金銮殿下。师徒们排列阶前捧关文递与国王。国王展开方看又见黄门官来奏:“三位国师来也。”慌得国王收了关文急下龙座着近侍的设了绣墩躬身迎接。三藏等回头观看见那大仙摇摇摆摆后带着一双丫髻蓬头的小童儿往里直进两班官控背躬身不敢仰视。他上了金銮殿对国王径不行礼。

那国王道:“国师朕未曾奉请今日如何肯降?”老道士云:“有一事奉告故来也。那四个和尚是那国来的?”国王道:“是东土大唐差去西天取经的来此倒换关文。”那三道士鼓掌大笑道:

“我说他走了原来还在这里!”国王惊道:“国师有何话说?他才来报了姓名正欲拿送国师使用怎奈当驾太师所奏有理朕因看远来之意不灭中华善缘方才召入验牒。不期国师有此问想是他冒犯尊颜有得罪处也?”道士笑云:“陛下不知他昨日来的在东门外打杀了我两个徒弟放了五百个囚僧捽碎车辆夜间闯进观来把三清圣象毁坏偷吃了御赐供养。

我等被他蒙蔽了只道是天尊下降求些圣水金丹进与陛下指望延寿长生;不期他遗些小便哄瞒我等。我等各喝了一口尝出滋味正欲下手擒拿他却走了。今日还在此间正所谓冤家路儿窄也!”那国王闻言怒欲诛四众。孙大圣合掌开言厉声高叫道:“陛下暂息雷霆之怒容僧等启奏。”国王道:“你冲撞了国师国师之言岂有差谬!”行者道:“他说我昨日到城外打杀他两个徒弟是谁知证?我等且屈认了着两个和尚偿命还放两个去取经。他又说我捽碎车辆放了囚僧此事亦无见证料不该死再着一个和尚领罪罢了。他说我毁了三清闹了观宇这又是栽害我也。”国王道:“怎见栽害?”行者道:“我僧乃东土之人乍来此处街道尚且不通如何夜里就知他观中之事?既遗下小便就该当时捉住却这早晚坐名害人。天下假名托姓的无限怎么就说是我?望陛下回嗔详察。”那国王本来昏乱被行者说了一遍他就决断不定。

正疑惑之间又见黄门官来奏:“陛下门外有许多乡老听宣。”国王道:“有何事干?”即命宣来。宣至殿前有三四十名乡老朝上磕头道:“万岁今年一春无雨但恐夏月干荒特来启奏请那位国师爷爷祈一场甘雨普济黎民。”国王道:“乡老且退就有雨来也。”乡老谢恩而出。国王道:“唐朝僧众朕敬道灭僧为何?只为当年求雨我朝僧人更未尝求得一点;幸天降国师拯援涂炭。你今远来冒犯国师本当即时问罪。姑且恕你敢与我国师赌胜求雨么?若祈得一场甘雨济度万民朕即饶你罪名倒换关文放你西去。若赌不过无雨就将汝等推赴杀场典刑示众。”行者笑道:“小和尚也晓得些儿求祷。”国王见说即命打扫坛场一壁厢教:“摆驾寡人亲上五凤楼观看。”当时多官摆驾须臾上楼坐了。唐三藏随着行者、沙僧、八戒侍立楼下那三道士陪国王坐在楼上。少时间一员官飞马来报:“坛场诸色皆备请国师爷爷登坛。”

那虎力大仙欠身拱手辞了国王径下楼来。行者向前拦住道:“先生那里去?”大仙道:“登坛祈雨。”行者道:“你也忒自重了更不让我远乡之僧。也罢这正是强龙不压地头蛇。先生先去必须对君前讲开。”大仙道:“讲甚么?”行者道:“我与你都上坛祈雨知雨是你的是我的?不见是谁的功绩了。”国王在上听见心中暗喜道:“那小和尚说话倒有些筋节。”沙僧听见暗笑道:“不知一肚子筋节还不曾拿出来哩!”大仙道:

“不消讲陛下自然知之。”行者道:“虽然知之奈我远来之僧未曾与你相会。那时彼此混赖不成勾当须讲开方好行事。”

大仙道:“这一上坛只看我的令牌为号:一声令牌响风来二声响云起三声响雷闪齐鸣四声响雨至五声响云散雨收。”

行者笑道:“妙啊!我僧是不曾见!请了!请了!”

大仙拽开步前进三藏等随后径到了坛门外。抬头观看那里有一座高台约有三丈多高。台左右插着二十八宿旗号顶上放一张桌子桌上有一个香炉炉中香烟霭霭。两边有两只烛台台上风烛煌煌。炉边靠着一个金牌牌上镌的是雷神名号。底下有五个大缸都注着满缸清水水上浮着杨柳枝。杨柳枝上托着一面铁牌牌上书的是雷霆都司的符字。左右有五个大桩桩上写着五方蛮雷使者的名录。每一桩边立两个道士各执铁锤伺候着打桩。台后面有许多道士在那里写作文书。正中间设一架纸炉又有几个象生的人物都是那执符使者、土地赞教之神。

那大仙走进去更不谦逊直上高台立定。旁边有个小道士捧了几张黄纸书就的符字一口宝剑递与大仙。大仙执着宝剑念声咒语将一道符在烛上烧了。那底下两三个道士拿过一个执符的象生一道文书亦点火焚之。那上面乒的一声令牌响只见那半空里悠悠的风色飘来猪八戒口里作念道:

“不好了!不好了!这道士果然有本事!令牌响了一下果然就刮风!”行者道:“兄弟悄悄的你们再莫与我说话只管护持师父等我干事去来。”好大圣拔下一根毫毛吹口仙气叫“变!”就变作一个“假行者”立在唐僧手下。他的真身出了元神赶到半空中高叫:“那司风的是那个?”慌得那风婆婆捻住布袋巽二郎札住口绳上前施礼。行者道:“我保护唐朝圣僧西天取经路过车迟国与那妖道赌胜祈雨你怎么不助老孙反助那道士?我且饶你把风收了。若有一些风儿把那道士的胡子吹得动动各打二十铁棒!”风婆婆道:“不敢不敢!”遂而没些风气。八戒忍不住乱嚷道:“那先儿请退!令牌已响怎么不见一些风儿?你下来让我们上去!”

那道士又执令牌烧了符檄扑的又打了一下只见那空中云雾遮满。孙大圣又当头叫道:“布云的是那个?”慌得那推云童子、布雾郎君当面施礼。行者又将前事说了一遍那云童、雾子也收了云雾放出太阳星耀耀一天万里更无云。八戒笑道:“这先儿只好哄这皇帝搪塞黎民全没些真实本事!令牌响了两下如何又不见云生?”

那道士心中焦躁仗宝剑解散了头念着咒烧了符再一令牌打将下去只见那南天门里邓天君领着雷公电母到当空迎着行者施礼。行者又将前项事说了一遍道:“你们怎么来的志诚!是何法旨?”天君道:“那道士五雷法是个真的。他了文书烧了文檄惊动玉帝玉帝掷下旨意径至九天应元雷声普化天尊府下。我等奉旨前来助雷电下雨。”行者道:“既如此且都住了同候老孙行事。”果然雷也不鸣电也不灼。

那道士愈加着忙又添香、烧符、念咒、打下令牌。半空中又有四海龙王一齐拥至。行者当头喝道:“敖广!那里去?”那敖广、敖顺、敖钦、敖闰上前施礼。行者又将前项事说了一遍道:“向日有劳未曾成功;今日之事望为助力。”龙王道:“遵命!遵命!”行者又谢了敖顺道:“前日亏令郎缚怪搭救师父。”

龙王道:“那厮还锁在海中未敢擅便正欲请大圣落。”行者道:“凭你怎么处治了罢如今且助我一功。那道士四声令牌已毕却轮到老孙下去干事了。但我不会符烧檄打甚令牌你列位却要助我行行。”邓天君道:“大圣吩咐谁敢不从!但只是得一个号令方敢依令而行;不然雷雨乱了显得大圣无款也。”行者道:“我将棍子为号罢。”那雷公大惊道:“爷爷呀!我们怎吃得这棍子?”行者道:“不是打你们但看我这棍子往上一指就要刮风。”那风婆婆、巽二郎没口的答应道:“就放风!”

“棍子第二指就要布云。”那推云童子、布雾郎君道:“就布云!

就布云!”“棍子第三指就要雷鸣电灼。”那雷公、电母道:“奉承!奉承!”“棍子第四指就要下雨。”那龙王道:“遵命!遵命!”

“棍子第五指就要大日晴天却莫违误。”吩咐已毕遂按下云头把毫毛一抖收上身来。那些人肉眼凡胎那里晓得?行者遂在旁边高叫道:“先生请了四声令牌俱已响毕更没有风云雷雨该让我了。”那道士无奈不敢久占只得下了台让他努着嘴径往楼上见驾。行者道:“等我跟他去看他说些甚的。”

只听得那国王问道:“寡人这里洗耳诚听你那里四声令响不见风雨何也?”道士云:“今日龙神都不在家。”行者厉声道:

“陛下龙神俱在家只是这国师法不灵请他不来。等和尚请来你看。”国王道:“即去登坛寡人还在此候雨。”行者得旨急抽身到坛所扯着唐僧道:“师父请上台。”唐僧道:“徒弟我却不会祈雨。”八戒笑道:“他害你了若还没雨拿上柴蓬一把火了帐!”行者道:“你不会求雨好的会念经等我助你。”那长老才举步登坛到上面端然坐下定性归神默念那《密多心经》。正坐处忽见一员官飞马来问:“那和尚怎么不打令牌不烧符檄?”行者高声答道:“不用!不用!我们是静功祈祷。”

那官去回奏不题。

行者听得老师父经文念尽却去耳朵内取出铁棒迎风幌了一幌就有丈二长短碗来粗细将棍望空一指那风婆婆见了急忙扯开皮袋巽二郎解放口绳:只听得呼呼风响满城中揭瓦翻砖扬砂走石。看起来真个好风却比那寻常之风不同也但见:折柳伤花摧林倒树。九重殿损壁崩墙五凤楼摇梁撼柱。天边红日无光地下黄砂有翅。演武厅前武将惊会文阁内文官惧。三宫粉黛乱青丝六院嫔妃蓬宝髻。侯伯金冠落绣缨宰相乌纱飘展翅。当驾有言不敢谈黄门执本无由递。金鱼玉带不依班象简罗衫无品叙。彩阁翠屏尽损伤绿窗朱户皆狼狈。金銮殿瓦走砖飞锦云堂门歪槅碎。这阵狂风果是凶刮得那君王父子难相会;六街三市没人踪万户千门皆紧闭!

正是那狂风大作孙行者又显神通把金箍棒钻一钻望空又一指只见那:推云童子布雾郎君。推云童子显神威骨都都触石遮天;布雾郎君施法力浓漠漠飞烟盖地。茫茫三市暗冉冉六街昏。因风离海上随雨出昆仑。顷刻漫天地须臾蔽世尘。宛然如混沌不见凤楼门。此时昏雾朦胧浓云叆叇。孙行者又把金箍棒钻一钻望空又一指慌得那:雷公奋怒电母生嗔。雷公奋怒倒骑火兽下天关电母生嗔乱掣金蛇离斗府。唿喇喇施霹雳振碎了铁叉山;淅沥沥闪红绡飞出了东洋海。呼呼隐隐滚车声烨烨煌煌飘稻米。万萌万物精神改多少昆虫蛰已开。君臣楼上心惊骇商贾闻声胆怯忙。那沉雷护闪乒乒乓乓一似那地裂山崩之势唬得那满城人户户焚香家家化纸。孙行者高呼:“老邓!仔细替我看那贪赃坏法之官忤逆不孝之子多打死几个示众!”那雷越振响起来。行者却又把铁棒望上一指只见那:龙施号令雨漫乾坤。势如银汉倾天堑疾似云流过海门。楼头声滴滴窗外响潇潇。天上银河泻街前白浪滔。淙淙如瓮捡滚滚似盆浇。孤庄将漫屋野岸欲平桥。真个桑田变沧海霎时陆岸滚波涛。神龙借此来相助抬起长江望下浇。这场雨自辰时下起只下到午时前后下得那车迟城里里外外水漫了街衢。那国王传旨道:“雨彀了!雨彀了!十分再多又渰坏了禾苗反为不美。”五凤楼下听事官策马冒雨来报:“圣僧雨彀了。”行者闻言将金箍棒往上又一指只见霎时间雷收风息雨散云收。国王满心欢喜文武尽皆称赞道:“好和尚!这正是强中更有强中手!就是我国师求雨虽灵若要晴细雨儿还下半日便不清爽。怎么这和尚要晴就晴顷刻间杲杲日出万里就无云也?”

国王教回銮倒换关文打唐僧过去。正用御宝时又被那三个道士上前阻住道:“陛下这场雨全非和尚之功还是我道门之力。”国王道:“你才说龙王不在家不曾有雨他走上去以静功祈祷就雨下来怎么又与他争功何也?”虎力大仙道:“我上坛了文书烧了符檄击了令牌那龙王谁敢不来?

想是别方召请风云雷雨五司俱不在一闻我令随赶而来适遇着我下他上一时撞着这个机会所以就雨。从根算来还是我请的龙下的雨怎么算作他的功果?”那国王昏乱听此言却又疑惑未定。行者近前一步合掌奏道:“陛下这些旁门法术也不成个功果算不得我的他的。如今有四海龙王现在空中我僧未曾放他还不敢遽退。那国师若能叫得龙王现身就算他的功劳。”国王大喜道:“寡人做了二十三年皇帝更不曾看见活龙是怎么模样。你两家各显法力不论僧道但叫得来的就是有功;叫不出的有罪。”那道士怎么有那样本事?就叫那龙王见大圣在此也不敢出头。道士云:“我辈不能你是叫来。”那大圣仰面朝空厉声高叫:“敖广何在?弟兄们都现原身来看!”那龙王听唤即忙现了本身。四条龙在半空中度雾穿云飞舞向金銮殿上但见:飞腾变化绕雾盘云。玉爪垂钩白银鳞舞镜明。髯飘素练根根爽角耸轩昂挺挺清。磕额崔巍圆睛幌亮。隐显莫能测飞扬不可评。祷雨随时布雨求晴即便天晴。这才是有灵有圣真龙象祥瑞缤纷绕殿庭。那国王在殿上焚香。众公卿在阶前礼拜。国王道:“有劳贵体降临请回寡人改日醮谢。”行者道:“列位众神各自归去这国王改日醮谢哩。”那龙王径自归海众神各各回天。这正是:广大无边真妙法至真了性劈旁门。毕竟不知怎么除邪且听下回分解。
------------


------------

第四十七回 圣僧夜阻通天水 金木垂慈救小童

却说那国王倚着龙床泪如泉涌只哭到天晚不住。行者上前高呼道:“你怎么这等昏乱!见放着那道士的尸骸一个是虎一个是鹿那羊力是一个羚羊。不信时捞上骨头来看那里人有那样骷髅?他本是成精的山兽同心到此害你因见气数还旺不敢下手。若再过二年你气数衰败他就害了你性命把你江山一股儿尽属他了。幸我等早来除妖邪救了你命你还哭甚?哭甚!急打关文送我出去。”国王闻此方才省悟。那文武多官俱奏道:“死者果然是白鹿黄虎油锅里果是羊骨。圣僧之言不可不听。”国王道:“既是这等感谢圣僧。今日天晚教太师且请圣僧至智渊寺。明日早朝大开东阁教光禄寺安排素净筵宴酬谢。”果送至寺里安歇。次日五更时候国王设朝聚集多官传旨:“快出招僧榜文四门各路张挂。”一壁厢大排筵宴摆驾出朝至智渊寺门外请了三藏等共入东阁赴宴不在话下。却说那脱命的和尚闻有招僧榜个个欣然都入城来寻孙大圣交纳毫毛谢恩。这长老散了宴那国王换了关文同皇后嫔妃两班文武送出朝门。只见那些和尚跪拜道旁口称:“齐天大圣爷爷!我等是沙滩上脱命僧人。闻知爷爷扫除妖孽救拔我等又蒙我王出榜招僧特来交纳毫毛叩谢天恩。”行者笑道:“汝等来了几何?”僧人道:“五百名半个不少。”行者将身一抖收了毫毛对君臣僧俗人说道:“这些和尚实是老孙放了车辆是老孙运转双关穿夹脊捽碎了那两个妖道也是老孙打死了。今日灭了妖邪方知是禅门有道向后来再不可胡为乱信。望你把三教归一也敬僧也敬道也养育人才我保你江山永固。”国王依言感谢不尽遂送唐僧出城去讫。

这一去只为殷勤经三藏努力修持光一元。晓行夜住渴饮饥餐不觉的春尽夏残又是秋光天气。一日天色已晚唐僧勒马道:“徒弟今宵何处安身也?”行者道:“师父出家人莫说那在家人的话。”三藏道:“在家人怎么?出家人怎么?”行者道:“在家人这时候温床暖被怀中抱子脚后蹬妻自自在在睡觉;我等出家人那里能够!便是要带月披星餐风宿水有路且行无路方住。”八戒道:“哥哥你只知其一不知其二。如今路多险峻我挑着重担着实难走须要寻个去处好眠一觉养养精神明日方好捱担不然却不累倒我也?”行者道:

“趁月光再走一程到有人家之所再住。”师徒们没奈何只得相随行者往前。

又行不多时只听得滔滔浪响。八戒道:“罢了!来到尽头路了!”沙僧道:“是一股水挡住也。”唐僧道:“却怎生得渡?”八戒道:“等我试之看深浅何如。”三藏道:“悟能你休乱谈水之浅深如何试得?”八戒道:“寻一个鹅卵石抛在当中。若是溅起水泡来是浅若是骨都都沉下有声是深。”行者道:“你去试试看。”那呆子在路旁摸了一块顽石望水中抛去只听得骨都都泛起鱼津沉下水底。他道:“深深深!去不得!”唐僧道:

“你虽试得深浅却不知有多少宽阔。”八戒道:“这个却不知不知。”行者道:“等我看看。”好大圣纵筋斗云跳在空中定睛观看但见那:洋洋光浸月浩浩影浮天。灵派吞华岳长流贯百川。千层汹浪滚万迭峻波颠。岸口无渔火沙头有鹭眠。

茫然浑似海一望更无边。急收云头按落河边道:“师父宽哩宽哩!去不得!老孙火眼金睛白日里常看千里凶吉晓得是夜里也还看三五百里。如今通看不见边岸怎定得宽阔之数?”

三藏大惊口不能言声音哽咽道:“徒弟啊似这等怎了?”沙僧道:“师父莫哭你看那水边立的可不是个人么。”行者道:

“想是扳罾的渔人等我问他去来。”拿了铁棒两三步跑到面前看处呀!不是人是一面石碑。碑上有三个篆文大字下边两行有十个小字。三个大字乃“通天河”十个小字乃“径过八百里亘古少人行”。行者叫:“师父你来看看。”三藏看见滴泪道:“徒弟呀我当年别了长安只说西天易走那知道妖魔阻隔山水迢遥!”八戒道:“师父你且听是那里鼓钹声音?想是做斋的人家。我们且去赶些斋饭吃问个渡口寻船明日过去罢。”三藏马上听得果然有鼓钹之声“却不是道家乐器足是我僧家举事。我等去来。”行者在前引马一行闻响而来。那里有甚正路没高没低漫过沙滩望见一簇人家住处约摸有四五百家却也都住得好但见倚山通路傍岸临溪。处处柴扉掩家家竹院关。沙头宿鹭梦魂清柳外啼鹃喉舌冷。短笛无声寒砧不韵。红蓼枝摇月黄芦叶斗风。陌头村犬吠疏篱渡口老渔眠钓艇。灯火稀人烟静半空皎月如悬镜。忽闻一阵白蘋香却是西风隔岸送。

三藏下马只见那路头上有一家儿门外竖一幢幡内里有灯烛荧煌香烟馥郁。三藏道:“悟空此处比那山凹河边却是不同。在人间屋檐下可以遮得冷露放心稳睡。你都莫来让我先到那斋公门告求。若肯留我我就招呼汝等;假若不留你却休要撒泼。汝等脸嘴丑陋只恐唬了人闯出祸来却倒无住处矣。”行者道:“说得有理。请师父先去我们在此守待。”那长老才摘了斗笠光着头抖抖褊衫拖着锡杖径来到人家门外见那门半开半掩三藏不敢擅入。聊站片时只见里面走出一个老者项下挂着数珠口念阿弥陀佛径自来关门慌得这长老合掌高叫:“老施主贫僧问讯了。”那老者还礼道:

“你这和尚却来迟了。”三藏道:“怎么说?”老者道:“来迟无物了。早来啊我舍下斋僧尽饱吃饭熟米三升白布一段铜钱十文。你怎么这时才来?”三藏躬身道:“老施主贫僧不是赶斋的。”老者道:“既不赶斋来此何干?”三藏道:“我是东土大唐钦差往西天取经者今到贵处天色已晚听得府上鼓钹之声特来告借一宿天明就行也。”那老者摇手道:“和尚出家人休打诳语。东土大唐到我这里有五万四千里路你这等单身如何来得?”三藏道:“老施主见得最是但我还有三个小徒逢山开路遇水迭桥保护贫僧方得到此。”老者道:“既有徒弟何不同来?”教:“请请我舍下有处安歇。”三藏回头叫声:“徒弟这里来。”那行者本来性急八戒生来粗鲁沙僧却也莽撞三个人听得师父招呼牵着马挑着担不问好歹一阵风闯将进去。那老者看见唬得跌倒在地口里只说是“妖怪来了!妖怪来了!”三藏搀起道:“施主莫怕不是妖怪是我徒弟。”老者战兢兢道:“这般好俊师父怎么寻这样丑徒弟!”三藏道:“虽然相貌不中却倒会降龙伏虎捉怪擒妖。”老者似信不信的扶着唐僧慢走。

却说那三个凶顽闯入厅房上拴了马丢下行李。那厅中原有几个和尚念经八戒掬着长嘴喝道:“那和尚念的是甚么经?”那些和尚听见问了一声忽然抬头观看外来人嘴长耳朵大。身粗背膊宽声响如雷咋。行者与沙僧容貌更丑陋。厅堂几众僧无人不害怕。阇黎还念经班教行罢。难顾磬和铃佛象且丢下。一齐吹息灯惊散光乍乍。跌跌与爬爬门槛何曾跨!你头撞我头似倒葫芦架。清清好道场翻成大笑话。

这兄弟三人见那些人跌跌爬爬鼓着掌哈哈大笑。那些僧越加悚惧磕头撞脑各顾性命通跑净了三藏搀那老者走上厅堂灯火全无三人嘻嘻哈哈的还笑。唐僧骂道:“这泼物十分不善!我朝朝教诲日日叮咛。古人云不教而善非圣而何!

教而后善非贤而何!教亦不善非愚而何!汝等这般撒泼诚为至下至愚之类!走进门不知高低唬倒了老施主惊散了念经僧把人家好事都搅坏了却不是堕罪与我?”说得他们不敢回言。那老者方信是他徒弟急回头作礼道:“老爷没大事没大事才然关了灯散了花佛事将收也。”八戒道:“既是了帐摆出满散的斋来我们吃了睡觉。”老者叫:“掌灯来!掌灯来!”

家里人听得大惊小怪道:“厅上念经有许多香烛如何又教掌灯?”几个僮仆出来看时这个黑洞洞的即便点火把灯笼一拥而至忽抬头见八戒沙僧慌得丢了火把忽抽身关了中门往里嚷道:“妖怪来了!妖怪来了!”

行者拿起火把点上灯烛扯过一张交椅请唐僧坐在上面他兄弟们坐在两旁那老者坐在前面。(WWW.mianhuatang.la 好看的小说)正叙坐间只听得里面门开处又走出一个老者拄着拐杖道:“是甚么邪魔黑夜里来我善门之家?”前面坐的老者急起身迎到屏门后道:“哥哥莫嚷不是邪魔乃东土大唐取经的罗汉。徒弟们相貌虽凶果然是相恶人善。”那老者方才放下拄杖与他四位行礼。礼毕也坐了面前叫:“看茶来排斋。”连叫数声几个僮仆战战兢兢不敢拢帐。八戒忍不住问道:“老者你这盛价两边走怎的?”老者道:“教他们捧斋来侍奉老爷。”八戒道:“几个人伏侍?”老者道:“八个人。”八戒道:“这八个人伏侍那个?”老者道:“伏侍你四位。”八戒道:“那白面师父只消一个人;毛脸雷公嘴的只消两个人;那晦气脸的要八个人;我得二十个人伏侍方彀。”老者道:“这等说想是你的食肠大些。”八戒道:“也将就看得过。”老者道:“有人有人。”七大八小就叫出有三四十人出来。

那和尚与老者一问一答的讲话众人方才不怕。却将上面排了一张桌请唐僧上坐;两边摆了三张桌请他三位坐;前面一张桌坐了二位老者。先排上素果品菜蔬然后是面饭、米饭、闲食、粉汤排得齐齐整整。唐长老举起箸来先念一卷《启斋经》。那呆子一则有些急吞二来有些饿了那里等唐僧经完拿过红漆木碗来把一碗白米饭扑的丢下口去就了了。

旁边小的道:“这位老爷忒没算计不笼馒头怎的把饭笼了却不污了衣服?”八戒笑道:“不曾笼吃了。”小的道:“你不曾举口怎么就吃了?”八戒道:“儿子们便说谎!分明吃了;不信再吃与你看。”那小的们又端了碗盛一碗递与八戒。呆子幌一幌又丢下口去就了了。众僮仆见了道:“爷爷呀!你是磨砖砌的喉咙着实又光又溜!”那唐僧一卷经还未完他已五六碗过手了然后却才同举箸一齐吃斋。呆子不论米饭面饭果品闲食只情一捞乱噇口里还嚷:“添饭!添饭!”渐渐不见来了!

行者叫道:“贤弟少吃些罢也强似在山凹里忍饿将就彀得半饱也好了。”八戒道:“嘴脸!常言道斋僧不饱不如活埋哩。”行者教:“收了家火莫睬他!”二老者躬身道:“不瞒老爷说白日里倒也不怕似这大肚子长老也斋得起百十众;只是晚了收了残斋只蒸得一石面饭、五斗米饭与几桌素食要请几个亲邻与众僧们散福。不期你列位来唬得众僧跑了连亲邻也不曾敢请尽数都供奉了列位。如不饱再教蒸去。”八戒道:“再蒸去!再蒸去!”话毕收了家火桌席三藏拱身谢了斋供才问:“老施主高姓?”老者道:“姓陈。”三藏合掌道:“这是我贫僧华宗了。”老者道:“老爷也姓陈?”三藏道:“是俗家也姓陈请问适才做的甚么斋事?”八戒笑道:“师父问他怎的!岂不知道?必然是青苗斋、平安斋、了场斋罢了。”老者道:“不是不是。”三藏又问:“端的为何?”老者道:“是一场预修亡斋。”八戒笑得打跌道:“公公忒没眼力!我们是扯谎架桥哄人的大王你怎么把这谎话哄我!和尚家岂不知斋事?只有个预修寄库斋、预修填还斋那里有个预修亡斋的?你家人又不曾有死的做甚亡斋?”

行者闻言暗喜道:“这呆子乖了些也。老公公你是错说了怎么叫做预修亡斋?”那二位欠身道:“你等取经怎么不走正路却蹡到我这里来?”行者道:“走的是正路只见一股水挡住不能得渡因闻鼓钹之声特来造府借宿。”老者道:“你们到水边可曾见些甚么?”行者道:“止见一面石碑上书通天河三字下书‘径过八百里亘古少人行’十字再无别物。”老者道:“再往上岸走走好的离那碑记只有里许有一座灵感大王庙你不曾见?”行者道:“未见请公公说说何为灵感?”那两个老者一齐垂泪道:“老爷啊!那大王:感应一方兴庙宇威灵千里祐黎民。年年庄上施甘露岁岁村中落庆云。”行者道:“施甘雨落庆云也是好意思你却这等伤情烦恼何也?”那老者跌脚捶胸哏了一声道:“老爷啊!虽则恩多还有怨纵然慈惠却伤人。只因要吃童男女不是昭彰正直神。”行者道:“要吃童男女么?”老者道:“正是。”行者道:“想必轮到你家了?”老者道:“今年正到舍下。我们这里有百家人家居住。此处属车迟国元会县所管唤做陈家庄。这大王一年一次祭赛要一个童男一个童女猪羊牲醴供献他。他一顿吃了保我们风调雨顺;若不祭赛就来降祸生灾。”行者道:“你府上几位令郎?”老者捶胸道:“可怜!可怜!说甚么令郎羞杀我等!这个是我舍弟名唤陈清老拙叫做陈澄。我今年六十三岁他今年五十八岁儿女上都艰难。我五十岁上还没儿子亲友们劝我纳了一妾没奈何寻下一房生得一女今年才交八岁取名唤做一秤金。”八戒道:“好贵名!怎么叫做一秤金?”老者道:“我因儿女艰难修桥补路建寺立塔布施斋僧有一本帐目那里使三两那里使五两到生女之年却好用过有三十斤黄金。三十斤为一秤所以唤做一秤金。”行者道:“那个的儿子么?”老者道:

“舍弟有个儿子也是偏出今年七岁了取各唤做陈关保。”行者问:“何取此名?”老者道:“家下供养关圣爷爷因在关爷之位下求得这个儿子故名关保我兄弟二人年岁百二止得这两个人种不期轮次到我家祭赛所以不敢不献。故此父子之情难割难舍先与孩儿做个生道场故曰预修亡斋者此也。”三藏闻言止不住腮边泪下道:“这正是古人云黄梅不落青梅落老天偏害没儿人。”行者笑道:“等我再问他。老公公你府上有多大家当?”二老道:“颇有些儿水田有四五十顷旱田有六七十顷草场有八九十处水黄牛有二三百头驴马有三二十匹猪羊鸡鹅无数。舍下也有吃不着的陈粮穿不了的衣服。家财产业也尽得数。”行者道:“你这等家业也亏你省将起来的。”老者道:“怎见我省?”行者道:“既有这家私怎么舍得亲生儿女祭赛?拚了五十两银子可买一个童男;拚了一百两银子可买一个童女连绞缠不过二百两之数可就留下自己儿女后代却不是好?”二老滴泪道:“老爷!你不知道那大王甚是灵感常来我们人家行走。”行者道:“他来行走你们看见他是甚么嘴脸?有几多长短?”二老道:“不见其形只闻得一阵香风就知是大王爷爷来了即忙满斗焚香老少望风下拜。他把我们这人家匙大碗小之事他都知道老幼生时年月他都记得。只要亲生儿女他方受用。不要说二三百两没处买就是几千万两也没处买这般一模一样同年同月的儿女。”行者道:“原来这等也罢也罢你且抱你令郎出来我看看。”那陈清急入里面将关保儿抱出厅上放在灯前。小孩儿那知死活笼着两袖果子跳跳舞舞的吃着耍子。行者见了默默念声咒语摇身一变变作那关保儿一般模样。两个孩儿搀着手在灯前跳舞唬得那老者谎忙跪着唐僧道:“老爷不当人子!不当人子!这位老爷才然说话怎么就变作我儿一般模样叫他一声齐应齐走!却折了我们年寿!请现本相!请现本相!行者把脸抹了一把现了本相。那老者跪在面前道:

“老爷原来有这样本事。”行者笑道:“可象你儿子么?”老者道:

“象象象!果然一般嘴脸一般声音一般衣服一般长短。”行者道:“你还没细看哩取秤来称称可与他一般轻重。”老者道:是是是是一般重。”行者道:“似这等可祭赛得过么?”老者道:“忒好忒好!祭得过了!”行者道:“我今替这个孩儿性命留下你家香烟后代我去祭赛那大王去也。”那陈清跪地磕头道:

“老爷果若慈悲替得我送白银一千两与唐老爷做盘缠往西天去。”行者道:“就不谢谢老孙?”老者道:“你已替祭没了你也。”行者道:“怎的得没了?”老者道:“那大王吃了。”行者道:

“他敢吃我?”老者道:“不吃你好道嫌腥。”行者笑道:“任从天命吃了我是我的命短;不吃是我的造化。我与你祭赛去。”

那陈清只管磕头相谢又允送银五百两惟陈澄也不磕头也不说谢只是倚着那屏门痛哭。行者知之上前扯住道:

“老大你这不允我不谢我想是舍不得你女儿么?”陈澄才跪下道:“是舍不得敢蒙老爷盛情救替了我侄子也彀了。但只是老拙无儿止此一女就是我死之后他也哭得痛切怎么舍得!”行者道:“你快去蒸上五斗米的饭整治些好素菜与我那长嘴师父吃教他变作你的女儿我兄弟同去祭赛索性行个阴骘救你两个儿女性命如何?”那八戒听得此言心中大惊道:“哥哥你要弄精神不管我死活就要攀扯我。”行者道:

“贤弟常言道鸡儿不吃无工之食。你我进门感承盛斋你还嚷吃不饱哩怎么就不与人家救些患难?”八戒道:“哥啊你便会变化我却不会哩。”行者道:“你也有三十六般变化怎么不会?”唐僧叫:“悟能你师兄说得最是处得甚当。常言救人一命胜造七级浮屠。一则感谢厚情二来当积阴德况凉夜无事你兄弟耍耍去来。”八戒道:“你看师父说的话!我只会变山变树变石头变癞象变水牛变大胖汉还可若变小女儿有几分难哩。”行者道:“老大莫信他抱出你令爱来看。”那陈澄急入里边抱将一秤金孩儿到了厅上。一家子妻妾大小不分老幼内外都出来磕头礼拜只请救孩儿性命。那女儿头上戴一个八宝垂珠的花翠箍身上穿一件红闪黄的纻丝袄上套着一件官绿缎子棋盘领的披风;腰间系一条大红花绢裙脚下踏一双虾蟆头浅红纻丝鞋腿上系两只绡金膝裤儿也袖着果子吃哩。行者道:“八戒这就是女孩儿你快变的象他我们祭赛去。”八戒道:“哥呀似这般小巧俊秀怎变?”行者叫:“快些!

莫讨打!”八戒谎了道:“哥哥不要打等我变了看。”这呆子念动咒语把头摇了几摇叫“变!”真个变过头来就也象女孩儿面目只是肚子胖大郎伉不象。行者笑道:“再变变!”八戒道:

“凭你打了罢!变不过来奈何?”行者道:“莫成是丫头的头和尚的身子?弄的这等不男不女却怎生是好?你可布起罡来。”

他就吹他一口仙气果然即时把身子变过与那孩儿一般。便教:“二位老者带你宝眷与令郎令爱进去不要错了。一会家我兄弟躲懒讨乖走进去转难识认。你将好果子与他吃不可教他哭叫恐大王一时知觉走了风讯等我两人耍子去也!”

好大圣吩咐沙僧保护唐僧他变作陈关保八戒变作一秤金。二人俱停当了却问:“怎么供献?还是捆了去是绑了去?蒸熟了去是剁碎了去?”八戒道:“哥哥莫要弄我我没这个手段。”老者道:“不敢不敢!只是用两个红漆丹盘请二位坐在盘内放在桌上着两个后生抬一张桌子把你们抬上庙去。”行者道:“好好好!拿盘子出来我们试试。”那老者即取出两个丹盘行者与八戒坐上四个后生抬起两张桌子往天井里走走儿又抬回放在堂上。行者欢喜道:“八戒象这般子走走耍耍我们也是上台盘的和尚了。”八戒道:“若是抬了去还抬回来两头抬到天明我也不怕;只是抬到庙里就要吃哩这个却不是耍子!”行者道:“你只看着我划着吃我时你就走了罢。”八戒道:“知他怎么吃哩?如先吃童男我便好跑;如先吃童女我却如何?”老者道:“常年祭赛时我这里有胆大的钻在庙后或在供桌底下看见他先吃童男后吃童女。”八戒道:“造化!造化!兄弟正然谈论只听得外面锣鼓喧天灯火照耀同庄众人打开前门叫:“抬出童男童女来!”这老者哭哭啼啼那四个后生将他二人抬将出去。端的不知性命何如且听下回分解。
------------

第四十八回 魔弄寒风飘大雪 僧思拜佛履层冰

话说陈家庄众信人等将猪羊牲醴与行者八戒喧喧嚷嚷直抬至灵感庙里排下将童男女设在上。行者回头看见那供桌上香花蜡烛正面一个金字牌位上写灵感大王之神更无别的神象。众信摆列停当一齐朝上叩头道:“大王爷爷今年今月今日今时陈家庄祭主陈澄等众信年甲不齐谨遵年例供献童男一名陈关保童女一名陈一秤金猪羊牲醴如数奉上大王享用保祐风调雨顺五谷丰登。”祝罢烧了纸马各回本宅不题。

那八戒见人散了对行者道:“我们家去罢。”行者道:“你家在那里?”八戒道:“往老陈家睡觉去。”行者道:“呆子又乱谈了既允了他须与他了这愿心才是哩。”八戒道:“你倒不是呆子反说我是呆子!只哄他耍耍便罢怎么就与他祭赛当起真来!”行者道:“莫胡说为人为彻一定等那大王来吃了才是个全始全终;不然又教他降灾贻害反为不美。”正说间只听得呼呼风响。八戒道:“不好了!风响是那话儿来了!”行者只叫:“莫言语等我答应。”顷刻间庙门外来了一个妖邪你看他怎生模样:金甲金盔灿烂新腰缠宝带绕红云。眼如晚出明星皎牙似重排锯齿分。足下烟霞飘荡荡身边雾霭暖熏熏。行时阵阵阴风冷立处层层煞气温。却似卷帘扶驾将犹如镇寺大门神。那怪物拦住庙门问道:“今年祭祀的是那家?”行者笑吟吟的答道:“承下问庄头是陈澄、陈清家。”那怪闻答心中疑似道:“这童男胆大言谈伶俐常来供养受用的问一声不言语再问声唬了魂用手去捉已是死人。怎么今日这童男善能应对?”怪物不敢来拿又问:“童男女叫甚名字?”行者笑道:“童男陈关保童女一秤金。”怪物道:“这祭赛乃上年旧规如今供献我当吃你。”行者道:“不敢抗拒请自在受用。”怪物听说又不敢动手拦住门喝道:“你莫顶嘴!我常年先吃童男今年倒要先吃童女!”八戒慌了道:“大王还照旧罢不要吃坏例子。”

那怪不容分说放开手就捉八戒。呆子扑的跳下来现了本相掣钉钯劈手一筑那怪物缩了手往前就走只听得当的一声响。八戒道:“筑破甲了!”行者也现本相看处原来是冰盘大小两个鱼鳞喝声“赶上!”二人跳到空中。那怪物因来赴会不曾带得兵器空手在云端里问道:“你是那方和尚到此欺人破了我的香火坏了我的名声!”行者道:“这泼物原来不知我等乃东土大唐圣僧三藏奉钦差西天取经之徒弟。昨因夜寓陈家闻有邪魔假号灵感年年要童男女祭赛是我等慈悲拯救生灵捉你这泼物!趁早实实供来!一年吃两个童男女你在这里称了几年大王吃了多少男女?一个个算还我饶你死罪!”那怪闻言就走被八戒又一钉钯未曾打着他化一阵狂风钻入通天河内。行者道:“不消赶他了这怪想是河中之物。且待明日设法拿他送我师父过河。”八戒依言径回庙里把那猪羊祭醴连桌面一齐搬到陈家。此时唐长老、沙和尚共陈家兄弟正在厅中候信忽见他二人将猪羊等物都丢在天井里。三藏迎来问道:“悟空祭赛之事何如?”行者将那称名赶怪钻入河中之事说了一遍二老十分欢喜即命打扫厢房安排床铺请他师徒就寝不题。

却说那怪得命回归水内坐在宫中默默无言水中大小眷族问题:“大王每年享祭回来欢喜怎么今日烦恼?”那怪道:“常年享毕还带些余物与汝等受用今日连我也不曾吃得。造化低撞着一个对头几乎伤了性命。”众水族问:“大王是那个?”那怪道:“是一个东土大唐圣僧的徒弟往西天拜佛求经者假变男女坐在庙里。我被他现出本相险些儿伤了性命。一向闻得人讲:唐三藏乃十世修行好人但得吃他一块肉延寿长生。不期他手下有这般徒弟我被他坏了名声破了香火有心要捉唐僧只怕不得能彀。”那水族中闪上一个斑衣鳜婆对怪物跬跬拜拜笑道:“大王要捉唐僧有何难处!但不知捉住他可赏我些酒肉?”那怪道:“你若有谋合同用力捉了唐僧与你拜为兄妹共席享之。”鳜婆拜谢了道:“久知大王有呼风唤雨之神通搅海翻江之势力不知可会降雪?”那怪道:“会降。”又道:“既会降雪不知可会作冷结冰?”那怪道:

“更会!”鳜婆鼓掌笑道:“如此极易!极易!”那怪道:“你且将极易之功讲来我听。”鳜婆道:“今夜有三更天气大王不必迟疑趁早作法起一阵寒风下一阵大雪把通天河尽皆冻结。

着我等善变化者变作几个人形在于路口背包持伞担担推车不住的在冰上行走。那唐僧取经之心甚急看见如此人行断然踏冰而渡。大王稳坐河心待他脚踪响处迸裂寒冰连他那徒弟们一齐坠落水中一鼓可得也!”那怪闻言。满心欢喜道:“甚妙!甚妙!”即出水府踏长空兴风作雪结冷凝冻成冰不题。

却说唐长老师徒四人歇在陈家将近天晓师徒们衾寒枕冷。八戒咳歌打战睡不得叫道:“师兄冷啊!”行者道:“你这呆子忒不长俊!出家人寒暑不侵怎么怕冷?”三藏道:“徒弟果然冷。你看就是那重衾无暖气袖手似揣冰。此时败叶垂霜蕊苍松挂冻铃。地裂因寒甚池平为水凝。渔舟不见叟山寺怎逢僧?樵子愁柴少王孙喜炭增。征人须似铁诗客笔如菱。皮袄犹嫌薄貂裘尚恨轻。蒲团僵老衲纸帐旅魂惊。绣被重裀褥浑身战抖铃。”师徒们都睡不得爬起来穿了衣服开门看处呀!外面白茫茫的原来下雪哩!行者道:“怪道你们害冷哩却是这般大雪!”四人眼同观看好雪!但见那:彤云密布惨雾重浸。彤云密布朔风凛凛号空;惨雾重浸大雪纷纷盖地。真个是六出花片片飞琼;千林树株株带玉。须臾积粉顷刻成盐。白鹦歌失素皓鹤羽毛同。平添吴楚千江水压倒东南几树梅。却便似战退玉龙三百万果然如败鳞残甲满天飞。那里得东郭履袁安卧孙康映读;更不见子猷舟王恭币苏武餐毡。但只是几家村舍如银砌万里江山似玉团。好雪!

柳絮漫桥梨花盖舍。柳絮漫桥桥边渔叟挂蓑衣;梨花盖舍舍下野翁煨榾柮。客子难沽酒苍头苦觅梅。洒洒潇潇裁蝶翘飘飘荡荡剪鹅衣。团团滚滚随风势迭迭层层道路迷。阵阵寒威穿小幕飕飕冷气透幽帏。丰年祥瑞从天降堪贺人间好事宜。那场雪纷纷洒洒果如剪玉飞绵。师徒们叹玩多时只见陈家老者着两个僮仆扫开道路又两个送出热汤洗面。须臾又送滚茶乳饼又抬出炭火俱到厢房师徒们叙坐。长老问道:“老施主贵处时令不知可分春夏秋冬?”陈老笑道:“此间虽是僻地但只风俗人物与上国不同至于诸凡谷苗牲畜都是同天共日岂有不分四时之理?”三藏道:“既分四时怎么如今就有这般大雪这般寒冷?”陈老道:“此时虽是七月昨日已交白露就是八月节了。我这里常年八月间就有霜雪。”三藏道:“甚比我东土不同我那里交冬节方有之。”

正话间又见僮仆来安桌子请吃粥。粥罢之后雪比早间又大须臾平地有二尺来深。三藏心焦垂泪陈老道:“老爷放心莫见雪深忧虑。我舍下颇有几石粮食供养得老爷们半生。”三藏道:“老施主不知贫僧之苦。我当年蒙圣恩赐了旨意摆大驾亲送出关唐王御手擎杯奉饯问道几时可回?贫僧不知有山川之险顺口回奏只消三年可取经回国。自别后今已七八个年头还未见佛面恐违了钦限又怕的是妖魔凶狠所以焦虑。今日有缘得寓潭府昨夜愚徒们略施小惠报答实指望求一船只渡河。不期天降大雪道路迷漫不知几时才得功成回故土也!”陈老道:“老爷放心正是多的日子过了那里在这几日?且待天晴化了冰老拙倾家费产必处置送老爷过河。”只见一僮又请进早斋。到厅上吃毕叙不多时又午斋相继而进。三藏见品物丰盛再四不安道:“既蒙见留只可以家常相待。”陈老道:“老爷感蒙替祭救命之恩虽逐日设筵奉款也难酬难谢。”

此后大雪方住就有人行走。陈老见三藏不快又打扫花园大盆架火请去雪洞里闲耍散闷。八戒笑道:“那老儿忒没算计!春二三月好赏花园这等大雪又冷赏玩何物!”行者道:

“呆子不知事!雪景自然幽静一则游赏二来与师父宽怀。”陈老道:“正是正是。”遂此邀请到园但见:景值三秋风光如腊。苍松结玉蕊衰柳挂银花。阶下玉苔堆粉屑窗前翠竹吐琼芽。巧石山头养鱼池内。巧石山头削削尖峰排玉笋;养鱼池内清清活水作冰盘。临岸芙蓉娇色浅傍崖木槿嫩枝垂。秋海棠全然压倒;腊梅树聊新枝。牡丹亭、海榴亭、丹桂亭亭亭尽鹅毛堆积;放怀处、款客处、遣兴处处处皆蝶翅铺漫。

两篱黄菊玉绡金几树丹枫红间白。无数闲庭冷难到且观雪洞冷如冰。那里边放一个兽面象足铜火盆热烘烘炭火才生;

那上下有几张虎皮搭苫漆交椅软温温纸窗铺设。四壁上挂几轴名公古画却是那七贤过关寒江独钓迭嶂层峦团雪景;苏武餐毡折梅逢使琼林玉树写寒文。说不尽那家近水亭鱼易买雪迷山径酒难沽。真个可堪容膝处算来何用访蓬壶?众人观玩良久就于雪洞里坐下对邻叟道取经之事又捧香茶饮毕。陈老问:列位老爷可饮酒么?”三藏道:“贫僧不饮小徒略饮几杯素酒。”陈老大喜即命:“取素果品炖暖酒与列位汤寒。”那僮仆即抬桌围炉与两个邻叟各饮了几杯收了家火。

不觉天色将晚又仍请到厅上晚斋只听得街上行人都说:“好冷天啊!把通天河冻住了!”三藏闻言道:“悟空冻住河我们怎生是好?”陈老道:“乍寒乍冷想是近河边浅水处冻结。”那行人道:“把八百里都冻的似镜面一般路口上有人走哩!”三藏听说有人走就要去看。陈老道:“老爷莫忙今日晚了明日去看。”遂此别却邻叟又晚斋毕依然歇在厢房。

及次日天晓八戒起来道:“师兄今夜更冷想必河冻住也。”三藏迎着门朝天礼拜道:“众位护教大神弟子一向西来虔心拜佛苦历山川更无一声报怨。今至于此感得皇天祐助结冻河水弟子空心权谢待得经回奏上唐皇竭诚酬答。”礼拜毕遂教悟净背马趁冰过河。陈老又道:“莫忙待几日雪融冰解老拙这里办船相送。”沙僧道:“就行也不是话再住也不是话口说无凭耳闻不如眼见。我背了马且请师父亲去看看。”陈老道:“言之有理。”教:“小的们快去背我们六匹马来!且莫背唐僧老爷马。”就有六个小价跟随一行人径往河边来看真个是雪积如山耸云收破晓晴。寒凝楚塞千峰瘦冰结江湖一片平。朔风凛凛滑冻棱棱。池鱼偎密藻野鸟恋枯槎。塞外征夫俱坠指江头梢子乱敲牙。裂蛇腹断鸟足果然冰山千百尺。万壑冷浮银一川寒浸玉。东方自信出僵蚕北地果然有鼠窟。王祥卧光武渡一夜溪桥连底固。曲沼结棱层深渊重迭沍。通天阔水更无波皎洁冰漫如陆路。三藏与一行人到了河边勒马观看真个那路口上有人行走。三藏问道:“施主那些人上冰往那里去?”陈老道:“河那边乃西梁女国这起人都是做买卖的。我这边百钱之物到那边可值万钱;那边百钱之物到这边亦可值万钱。利重本轻所以人不顾生死而去。常年家有五七人一船或十数人一船飘洋而过。见如今河道冻住故舍命而步行也。”三藏道:“世间事惟名利最重。似他为利的舍死忘生我弟子奉旨全忠也只是为名与他能差几何!”教:“悟空快回施主家收拾行囊叩背马匹趁此层冰早奔西方去也。”行者笑吟吟答应。沙僧道:“师父啊常言道千日吃了千升米。今已托赖陈府上且再住几日待天晴化冻办船而过忙中恐有错也。”三藏道:“悟净怎么这等愚见!若是正二月一日暖似一日可以待得冻解。此时乃八月一日冷似一日如何可便望解冻!却不又误了半载行程?”

八戒跳下马来:“你们且休讲闲口等老猪试看有多少厚薄。”

行者道:“呆子前夜试水能去抛石如今冰冻重漫怎生试得?”八戒道:“师兄不知等我举钉钯筑他一下。假若筑破就是冰薄且不敢行;若筑不动便是冰厚如何不行?”三藏道:

“正是说得有理。”那呆子撩衣拽步走上河边双手举钯尽力一筑只听扑的一声筑了九个白迹手也振得生疼。呆子笑道:“去得!去得!连底都锢住了。”

三藏闻言十分欢喜与众同回陈家只教收拾走路。那两个老者苦留不住只得安排些干粮烘炒做些烧饼馍馍相送。

一家子磕头礼拜又捧出一盘子散碎金银跪在面前道:“多蒙老爷活子之恩聊表途中一饭之敬。”三藏摆手摇头只是不受道:“贫僧出家人财帛何用?就途中也不敢取出。只是以化斋度日为正事收了干粮足矣。”二老又再三央求行者用指尖儿捻了一小块约有四五钱重递与唐僧道:“师父也只当些衬钱莫教空负二老之意。”遂此相向而别径至河边冰上那马蹄滑了一滑险些儿把三藏跌下马来。沙僧道:“师父难行!”

八戒道:“且住!问陈老官讨个稻草来我用。”行者道:“要稻草何用?”八戒道:“你那里得知要稻草包着马蹄方才不滑免教跌下师父来也。”陈老在岸上听言急命人家中取一束稻草却请唐僧上岸下马。八戒将草包裹马足然后踏冰而行。

别陈老离河边行有三四里远近八戒把九环锡杖递与唐僧道:“师父你横此在马上。”行者道:“这呆子奸诈!锡杖原是你挑的如何又叫师父拿着?”八戒道:“你不曾走过冰凌不晓得。凡是冰冻之上必有凌眼倘或躧着凌眼脱将下去若没横担之物骨都的落水就如一个大锅盖盖住如何钻得上来!

须是如此架住方可。”行者暗笑道:“这呆子倒是个积年走冰的!”果然都依了他。长老横担着锡杖行者横担着铁棒沙僧横担着降妖宝杖八戒肩挑着行李腰横着钉钯师徒们放心前进。这一直行到天晚吃了些干粮却又不敢久停对着星月光华观的冰冻上亮灼灼、白茫茫只情奔走果然是马不停蹄师徒们莫能合眼走了一夜。天明又吃些干粮望西又进。

正行时只听得冰底下扑喇喇一声响喨险些儿唬倒了白马。

三藏大惊道:“徒弟呀!怎么这般响喨?”八戒道:“这河忒也冻得结实地凌响了或者这半中间连底通锢住了也。”三藏闻言又惊又喜策马前进趱行不题。

却说那妖邪自从回归水府引众精在于冰下。等候多时只听得马蹄响处他在底下弄个神通滑喇的迸开冰冻慌得孙大圣跳上空中早把那白马落于水内三人尽皆脱下。那妖邪将三藏捉住引群精径回水府厉声高叫:“鳜妹何在?”老鳜婆迎门施礼道:“大王不敢不敢!”妖邪道:“贤妹何出此言!一言既出驷马难追。原说听从汝计捉了唐僧与你拜为兄妹。

今日果成妙计捉了唐僧就好味了前言?”教:“小的们抬过案桌磨快刀来把这和尚剖腹剜心剥皮剐肉一壁厢响动乐器与贤妹共而食之延寿长生也。”鳜婆道:“大王且休吃他恐他徒弟们寻来吵闹。且宁耐两日让那厮不来寻然后剖开请大王上坐众眷族环列吹弹歌舞奉上大王从容自在享用却不好也?”那怪依言把唐僧藏于宫后使一个六尺长的石匣盖在中间不题。

却说八戒、沙僧在水里捞着行囊放在白马身上驮了分开水路涌浪翻波负水而出只见行者在半空中看见问道:

“师父何在?”八戒道:“师父姓陈名到底了如今没处找寻且上岸再作区处。”原来八戒本是天蓬元帅临凡他当年掌管天河八万水兵大众沙和尚是流沙河内出身白马本是西海龙孙:故此能知水性。大圣在空中指引须臾回转东崖晒刷了马匹靦掠了衣裳大圣云头按落一同到于陈家庄上。早有人报与二老道:“四个取经的老爷如今只剩了三个来也。”兄弟即忙接出门外果见衣裳还湿道:“老爷们我等那般苦留却不肯住只要这样方休。怎么不见三藏老爷?”八戒道:“不叫做三藏了改名叫做陈到底也。”二老垂泪道:“可怜!可怜!我说等雪融备船相送坚执不从致令丧了性命!”行者道:“老儿莫替古人耽忧我师父管他不死长命。老孙知道决然是那灵感大王弄法算计去了。你且放心与我们浆浆衣服晒晒关文取草料喂着白马等我弟兄寻着那厮救出师父索性剪草除根替你一庄人除了后患庶几永永得安生也。”陈老闻言满心欢喜即命安排斋供。兄弟三人饱餐一顿将马匹行囊交与陈家看守各整兵器径赴道边寻师擒怪。正是:误踏层冰伤本性大丹脱漏怎周全?毕竟不知怎么救得唐僧且听下回分解。
------------

第四十九回 三藏有灾沉水宅 观音救难现鱼篮

却说孙大圣与八戒、沙僧辞陈老来至河边道:“兄弟你两个议定那一个先下水。(WWW.mianhuatang.la 好看的小说)”八戒道:“哥啊我两个手段不见怎的还得你先下水。”行者道:“不瞒贤弟说若是山里妖精全不用你们费力水中之事我去不得。就是下海行江我须要捻着避水诀或者变化甚么鱼蟹之形才去得。若是那般捻诀却轮不得铁棒使不得神通打不得妖怪。我久知你两个乃惯水之人所以要你两个下去。”沙僧道:“哥啊小弟虽是去得但不知水底如何。我等大家都去哥哥变作甚么模样或是我驮着你分开水道寻着妖圣的巢穴你先进去打听打听。若是师父不曾伤损还在那里我们好努力征讨。假若不是这怪弄法或者渰杀师父或者被妖吃了我等不须苦求早早的别寻道路何如?”行者道:“贤弟说得有理你们那个驮我?”八戒暗喜道:“这猴子不知捉弄了我多少今番原来不会水等老猪驮他也捉弄他捉弄!”呆子笑嘻嘻的叫道:“哥哥我驮你。”行者就知有意却便将计就计道:“是也好你比悟净还有些膂力。”八戒就背着他。沙僧剖开水路弟兄们同入通天河内。向水底下行有百十里远近那呆子要捉弄行者行者随即拔下一根毫毛变做假身伏在八戒背上真身变作一个猪虱子紧紧的贴在他耳朵里。八戒正行忽然打个躘踵得故子把行者往前一掼扑的跌了一跤。原来那个假身本是毫毛变的却就飘起去无影无形。沙僧道:“二哥你是怎么说?不好生走路就跌在泥里便也罢了却把大哥不知跌在那里去了!”八戒道:

“那猴子不禁跌一跌就跌化了。兄弟莫管他死活我和你且去寻师父去。”沙僧道:“不好还得他来他虽水性不知他比我们乖巧。若无他来我不与你去。”行者在八戒耳朵里忍不住高叫道:“悟净!老孙在这里也。”沙僧听得笑道:“罢了!这呆子是死了!你怎么就敢捉弄他!如今弄得闻声不见面却怎是好?”八戒慌得跪在泥里磕头道:“哥哥是我不是了待救了师父上岸陪礼。你在那里做声?就影杀我也!你请现原身出来我驮着你再不敢冲撞你了。”行者道:“是你还驮着我哩。

我不弄你你快走!快走!”那呆子絮絮叨叨只管念诵着陪礼爬起来与沙僧又进。

行了又有百十里远近忽抬头望见一座楼台上有“水鼋之第”四个大字。沙僧道:“这厢想是妖精住处我两个不知虚实怎么上门索战?”行者道:“悟净那门里外可有水么?”沙僧道:“无水。”行者道:“既无水你再藏隐在左右待老孙去打听打听。”好大圣爬离了八戒耳朵里却又摇身一变变作个长脚虾婆两三跳跳到门里。睁眼看时只见那怪坐在上面众水族摆列两边有个斑衣鳜婆坐于侧手都商议要吃唐僧。行者留心两边寻找不见忽看见一个大肚虾婆走将来径往西廊下立定。行者跳到面前称呼道:“姆姆大王与众商议要吃唐僧唐僧却在那里?”虾婆道:“唐僧被大王降雪结冰昨日拿在宫后石匣中间只等明日他徒弟们不来吵闹就奏乐享用也。”

行者闻言演了一会径直寻到宫后看果有一个石匣却象人家槽房里的猪槽又似人间一口石棺材之样量量足有六尺长短;却伏在上面听了一会只听得三藏在里面嘤嘤的哭哩。行者不言语侧耳再听那师父挫得牙响哏了一声道:“自恨江流命有愆生时多少水灾缠。出娘胎腹淘波浪拜佛西天堕渺渊。前遇黑河身有难今逢冰解命归泉。不知徒弟能来否可得真经返故园?”行者忍不住叫道:“师父莫恨水灾经云土乃五行之母水乃五行之源。无土不生无水不长。老孙来了!”

三藏闻得道:“徒弟啊救我耶!”行者道:“你且放心待我们擒住妖精管教你脱难。”三藏道:“快些儿下手!再停一日足足闷杀我也!”行者道:“没事没事!我去也!”急回头跳将出去到门外现了原身叫:“八戒!”那呆子与沙僧近道:“哥哥如何?”行者道:“正是此怪骗了师父。师父未曾伤损被怪物盖在石匣之下。你两个快早挑战让老孙先出水面。你若擒得他就擒;擒不得做个佯输引他出水等我打他。”沙僧道:“哥哥放心先去待小弟们鉴貌辨色。”这行者捻着避水法钻出波中停立岸边等候不题。

你看那猪八戒行凶闯至门前厉声高叫:“泼怪物!送我师父出来!”慌得那门里小妖急报:“大王门外有人要师父哩!”妖邪道:“这定是那泼和尚来了。”教:“快取披挂兵器来!”

众小妖连忙取出。妖邪结束了执兵器在手即命开门走将出来。八戒与沙僧对列左右见妖邪怎生披挂。好怪物!你看他:

头戴金盔晃且辉身披金甲掣虹霓。腰围宝带团珠翠足踏烟黄靴样奇。鼻准高隆如峤耸天庭广阔若龙仪。眼光闪灼圆还暴牙齿钢锋尖又齐。短蓬松飘火焰长须潇洒挺金锥。口咬一枝青嫩藻手拿九瓣赤铜锤。一声咿哑门开处响似三春惊蛰雷。这等形容人世少敢称灵显大王威。

妖邪出得门来随后有百十个小妖一个个轮枪舞剑摆开两哨对八戒道:“你是那寺里和尚为甚到此喧嚷?”八戒喝道:“我把你这打不死的泼物!你前夜与我顶嘴今日如何推不知来问我?我本是东土大唐圣僧之徒弟往西天拜佛求经者。

你弄玄虚假做甚么灵感大王专在陈家庄要吃童男童女我本是陈清家一秤金你不认得我么?”那妖邪道:“你这和尚甚没道理!你变做一秤金该一个冒名顶替之罪。我倒不曾吃你反被你伤了我手背已此让了你你怎么又寻上我的门来?”八戒道:“你既让我却怎么又弄冷风下大雪冻结坚冰害我师父?快早送我师父出来万事皆休!牙迸半个不字你只看看手中钯决不饶你!”妖邪闻言微微冷笑道:“这和尚卖此长舌胡夸大口。果然是我作冷下雪冻河摄你师父。你今嚷上门来思量取讨只怕这一番不比那一番了。那时节我因赴会不曾带得兵器误中你伤。你如今且休要走我与你交敌三合三合敌得我过还你师父;敌不过连你一吃了。”八戒道:“好乖儿子!正是这等说!仔细看钯!”妖邪道:“你原来是半路上出家的和尚。”八戒道:“我的儿你真个有些灵感怎么就晓得我是半路出家的?”妖邪道:“你会使钯想是雇在那里种园把他钉钯拐将来也。”八戒道:“儿子我这钯不是那筑地之钯你看巨齿铸就如龙爪逊金妆来似蟒形。若逢对敌寒风洒但遇相持火焰生。能与圣僧除怪物西方路上捉妖精。轮动烟云遮日月使开霞彩照分明。筑倒太山千虎怕掀翻大海万龙惊。饶你威灵有手段一筑须教九窟窿!”

那个妖邪那里肯信举铜锤劈头就打八戒使钉钯架住道:“你这泼物原来也是半路上成精的邪魔!”那怪道:“你怎么认得我是半路上成精的?”八戒道:“你会使铜锤想是雇在那个银匠家扯炉被你得了手偷将出来的。”妖邪道:“这不是打银之锤你看九瓣攒成花骨朵一竿虚孔万年青。原来不比凡间物出处还从仙苑名。绿房紫菂瑶池老素质清香碧沼生。

因我用功抟炼过坚如钢锐彻通灵。枪刀剑戟浑难赛钺斧戈矛莫敢经。纵让你钯能利刃汤着吾锤迸折钉!”

沙和尚见他两个攀话忍不住近前高叫道:“那怪物休得浪言!古人云口说无凭做出便见。不要走!且吃我一杖!”

妖邪使锤杆架住道:“你也是半路里出家的和尚。”沙僧道:“你怎么认得?”妖邪道:“你这个模样象一个磨博士出身。”沙僧道:“如何认得我象个磨博士?”妖邪道:“你不是磨博士怎么会使赶面杖?”沙僧骂道:“你这孽障是也不曾见!这般兵器人间少故此难知宝杖名。出自月宫无影处梭罗仙木琢磨成。外边嵌宝霞光耀内里钻金瑞气凝。(WWW.mianhuatang.la 好看的小说)先日也曾陪御宴今朝秉正保唐僧。西方路上无知识上界宫中有大名。唤做降妖真宝杖管教一下碎天灵!”那妖邪不容分说三家变脸这一场在水底下好杀:铜锤宝杖与钉钯悟能悟净战妖邪。一个是天蓬临世界一个是上将降天涯。他两个夹攻水怪施威武这一个独抵神僧势可夸。有分有缘成大道相生相克秉恒沙。土克水水干见底;水生木木旺开花。禅法参修归一体还丹炮炼伏三家。土是母金芽金生神水产婴娃;水为本润木华木有辉煌烈火霞。攒簇五行皆别异故然变脸各争差。看他那铜锤九瓣光明好宝杖千丝彩绣佳。钯按阴阳分九曜不明解数乱如麻。捐躯弃命因僧难舍死忘生为释迦。致使铜锤忙不坠左遮宝杖右遮钯。三人在水底下斗经两个时辰不分胜败。猪八戒料道不得赢他对沙僧丢了个眼色二人诈败佯输各拖兵器回头就走。那怪物教:“小的们扎住在此等我赶上这厮捉将来与汝等凑吃哑!”你看他如风吹败叶似雨打残花将他两个赶出水面。

那孙大圣在东岸上眼不转睛只望着河边水势忽然见波浪翻腾喊声号吼八戒先跳上岸道:“来了!来了!”沙僧也到岸边道:“来了!来了!”那妖邪随后叫:“那里走!”才出头被行者喝道:“看棍!”那妖邪闪身躲过使铜锤急架相还。一个在河边涌浪一个在岸上施威。搭上手未经三合那妖遮架不住打个花又淬于水里遂此风平浪息。行者回转高崖道:“兄弟们辛苦啊。”沙僧道:“哥啊这妖精他在岸上觉到不济在水底也尽利害哩!我与二哥左右齐攻只战得个两平却怎么处置救师父也?”行者道:“不必疑迟恐被他伤了师父。”八戒道:

“哥哥我这一去哄他出来你莫做声但只在半空中等候估着他钻出头来却使个捣蒜打照他顶门上着着实实一下!纵然打不死他好道也护疼晕却等老猪赶上一钯管教他了帐!”行者道:“正是!正是!这叫做‘里迎外合’方可济事。”他两个复入水中不题。

却说那妖邪败阵逃生回归本宅众妖接到宫中鳜婆上前问道:“大王赶那两个和尚到那方来?”妖邪道:“那和尚原来还有一个帮手。他两个跳上岸去那帮手轮一条铁棒打我我闪过与他相持。也不知他那棍子有多少斤重我的铜锤莫想架得他住战未三合我却败回来也。”鳜婆道:“大王可记得那帮手是甚相貌?”妖邪道:“是一个毛脸雷公嘴查耳朵折鼻梁火眼金睛和尚。”鳜婆闻说打了一个寒噤道:“大王啊!亏了你识俊逃了性命!若再三合决然不得全生!那和尚我认得他。”妖邪道:“你认得他是谁?”鳜婆道:“我当年在东洋海内曾闻得老龙王说他的名誉乃是五百年前大闹天宫、混元一气上方太乙金仙美猴王齐天大圣如今归依佛教保唐僧往西天取经改名唤做孙悟空行者。他的神通广大变化多端大王你怎么惹他!今后再莫与他战了。”

说不了只见门里小妖来报:“大王那两个和尚又来门前索战哩!”妖精道:“贤妹所见甚长再不出去看他怎么。”急传令教:“小的们把门关紧了正是任君门外叫只是不开门。

让他缠两日性摊了回去时我们却不自在受用唐僧也?”那小妖一齐都搬石头塞泥块把门闭杀。八戒与沙僧连叫不出呆子心焦就使钉钯筑门。那门已此紧闭牢关莫想能彀;被他七八钯筑破门扇里面却都是泥土石块高迭千层。沙僧见了道:“二哥这怪物惧怕之甚闭门不出我和你且回上河崖再与大哥计较去来。”八戒依言径转东岸。

那行者半云半雾提着铁棒等哩。看见他两个上来不见妖怪即按云头迎至岸边问道:“兄弟那话儿怎么不上来?”

沙僧道:“那怪物紧闭宅门再不出来见面被二哥打破门扇看时那里面都使些泥土石块实实的迭住了。故此不能得战却来与哥哥计议再怎么设法去救师父。”行者道:“似这般却也无法可治。你两个只在河岸上巡视着不可放他往别处走了待我去来。”八戒道:“哥哥你往那里去?”行者道:“我上普陀岩拜问菩萨看这妖怪是那里出身姓甚名谁。寻着他的祖居拿了他的家属捉了他的四邻却来此擒怪救师。”八戒笑道:

“哥啊这等干只是忒费事担搁了时辰了。”行者道:“管你不费事不担搁!我去就来!”

好大圣急纵祥光躲离河口径赴南海。那里消半个时辰早望见落伽山不远低下云头径至普陀崖上。只见那二十四路诸天与守山大神、木叉行者、善财童子、捧珠龙女一齐上前迎着施礼道:“大圣何来?”行者道:“有事要见菩萨。”众神道:“菩萨今早出洞不许人随自入竹林里观玩。知大圣今日必来吩咐我等在此候接大圣不可就见。请在翠岩前聊坐片时待菩萨出来自有道理。”行者依言还未坐下又见那善财童子上前施礼道:“孙大圣前蒙盛意幸菩萨不弃收留早晚不离左右专侍莲台之下甚得善慈。行者知是红孩儿笑道:

“你那时节魔业迷心今朝得成正果才知老孙是好人也。”

行者久等不见心焦道:“列位与我传报传报但迟了恐伤吾师之命。”诸天道:“不敢报菩萨吩咐只等他自出来哩。”

行者性急那里等得急纵身往里便走。噫!这个美猴王性急能鹊薄。诸天留不住要往里边皐。拽步入深林睁眼偷觑着。

远观救苦尊盘坐衬残箬。懒散怕梳妆容颜多绰约。散挽一窝丝未曾戴缨络。不挂素蓝袍贴身小袄缚。漫腰束锦裙赤了一双脚。披肩绣带无精光两臂膊。玉手执钢刀正把竹皮削。行者见了忍不住厉声高叫道:“菩萨弟子孙悟空志心朝礼。”菩萨教:“外面俟候。”行者叩头道:“菩萨我师父有难特来拜问通天河妖怪根源。”菩萨道:“你且出去待我出来。”行者不敢强只得走出竹林对众诸天道:“菩萨今日又重置家事哩怎么不坐莲台不妆饰不喜欢在林里削篾做甚?”诸天道:“我等却不知。今早出洞未曾妆束就入林中去了又教我等在此接候大圣必然为大圣有事。”行者没奈何只得等候。

不多时只见菩萨手提一个紫竹篮儿出林道:“悟空我与你救唐僧去来。”行者慌忙跪下道:“弟子不敢催促且请菩萨着衣登座。”菩萨道:“不消着衣就此去也。”那菩萨撇下诸天纵祥云腾空而去孙大圣只得相随。顷刻间到了通天河界八戒与沙僧看见道:“师兄性急不知在南海怎么乱嚷乱叫把一个未梳妆的菩萨逼将来也。”说不了到于河岸。二人下拜道:

“菩萨我等擅干有罪!有罪!”菩萨即解下一根束袄的丝绦将篮儿拴定提着丝绦半踏云彩抛在河中往上溜头扯着口念颂子道:“死的去活的住死的去活的住!”念了七遍提起篮儿但见那篮里亮灼灼一尾金鱼还斩眼动鳞。菩萨叫:

“悟空快下水救你师父耶。”行者道:“未曾拿住妖邪如何救得师父?”菩萨道:“这篮儿里不是?”八戒与沙僧拜问道:“这鱼儿怎生有那等手段。菩萨道:“他本是我莲花池里养大的金鱼每日浮头听经修成手段。那一柄九瓣铜锤乃是一枝未开的菡萏被他运炼成兵。不知是那一日海潮泛涨走到此间。我今早扶栏看花却不见这厮出拜掐指巡纹算着他在此成精害你师父故此未及梳妆运神功织个竹篮儿擒他。”行者道:

“菩萨既然如此且待片时我等叫陈家庄众信人等看看菩萨的金面:一则留恩二来说此收怪之事好教凡人信心供养。”菩萨道:“也罢你快去叫来。”那八戒与沙僧一齐飞跑至庄前高呼道:“都来看活观音菩萨!都来看活观音菩萨!”一庄老幼男女都向河边也不顾泥水都跪在里面磕头礼拜。内中有善图画者传下影神这才是鱼篮观音现身。当时菩萨就归南海。

八戒与沙僧分开水道径往那水鼋之第找寻师父。原来那里边水怪鱼精尽皆死烂。却入后宫揭开石匣驮着唐僧出离波津与众相见。那陈清兄弟叩头称谢道:“老爷不依小人劝留致令如此受苦。”行者道:“不消说了。你们这里人家下年再不用祭赛那大王已此除根永无伤害。陈老儿如今才好累你快寻一只船儿送我们过河去也。”那陈清道:“有!有!

有!”就教解板打船众庄客闻得此言无不喜舍。那个道我买桅篷这个道我办篙桨有的说我出绳索有的说我雇水手。正都在河边上吵闹忽听得河中间高叫:“孙大圣不要打船花费人家财物我送你师徒们过去。”众人听说个个心惊胆小的走了回家胆大的战兢兢贪看。须臾那水里钻出一个怪来你道怎生模样:方头神物非凡品九助灵机号水仙。曳尾能延千纪寿潜身静隐百川渊。翻波跳浪冲江岸向日朝风卧海边。养气含灵真有道多年粉盖癞头鼋。那老鼋又叫:“大圣不要打船我送你师徒过去。”行者轮着铁棒道:“我把你这个孽畜!若到边前这一棒就打死你!”老鼋道:“我感大圣之恩情愿办好心送你师徒你怎么反要打我?”行者道:“与你有甚恩惠?”老鼋道:“大圣你不知这底下水鼋之第乃是我的住宅自历代以来祖上传留到我。我因省悟本根养成灵气在此处修行被我将祖居翻盖了一遍立做一个水鼋之第。那妖邪乃九年前海啸波翻他赶潮头来于此处仗逞凶顽与我争斗被他伤了我许多儿女夺了我许多眷族。我斗他不过将巢穴白白的被他占了。今蒙大圣至此搭救唐师父请了观音菩萨扫净妖氛收去怪物将第宅还归于我我如今团圞老小再不须挨土帮泥得居旧舍。此恩重若丘山深如大海。且不但我等蒙惠只这一庄上人免得年年祭赛全了多少人家儿女此诚所谓一举而两得之恩也!敢不报答?”行者闻言心中暗喜收了铁棒道:“你端的是真实之情么?”老鼋道:“因大圣恩德洪深怎敢虚谬?”行者道:“既是真情你朝天赌咒。”那老鼋张着红口朝天誓道:“我若真情不送唐僧过此通天河将身化为血水!”行者笑道:“你上来你上来。”老鼋却才负近岸边将身一纵爬上河崖。众人近前观看有四丈围圆的一个大白盖。行者道:“师父我们上他身渡过去也。”三藏道:“徒弟呀那层冰厚冻尚且迍邅况此鼋背恐不稳便。”老鼋道:“师父放心我比那层冰厚冻稳得紧哩但歪一歪不成功果!”行者道:

“师父啊凡诸众生会说人话决不打诳语。”教:“兄弟们快牵马来。”

到了河边陈家庄老幼男女一齐来拜送。行者教把马牵在白鼋盖上请唐僧站在马的颈项左边沙僧站在右边八戒站在马后行者站在马前又恐那鼋无礼解下虎筋绦子穿在老鼋的鼻之内扯起来象一条缰绳却使一只脚踏在盖上一只脚登在头上一只手执着铁棒一只手扯着缰绳叫道:“老鼋慢慢走啊歪一歪儿就照头一下!”老鼋道:“不敢!不敢!”

他却蹬开四足踏水面如行平地。众人都在岸上焚香叩头都念南无阿弥陀佛这正是真罗汉临凡活菩萨出现。众人只拜的望不见形影方回不题。

却说那师父驾着白鼋那消一日行过了八百里通天河界干手干脚的登岸。三藏上崖合手称谢道:“老鼋累你无物可赠待我取经回谢你罢。”老鼋道:“不劳师父赐谢。我闻得西天佛祖无灭无生能知过去未来之事。我在此间整修行了一千三百余年虽然延寿身轻会说人语只是难脱本壳。万望老师父到西天与我问佛祖一声看我几时得脱本壳可得一个人身。”三藏响允道:“我问我问。”那老鼋才淬水中去了。行者遂伏侍唐僧上马八戒挑着行囊沙僧跟随左右师徒们找大路一直奔西。这的是:圣僧奉旨拜弥陀水远山遥灾难多。意志心诚不惧死白鼋驮渡过天河。毕竟不知此后还有多少路程还有甚么凶吉且听下回分解。
------------

第五十回 情乱性从因爱欲 神昏心动遇魔头

词曰:心地频频扫尘情细细除莫教坑堑陷毗卢。(wwW.mianhuatang.la 无弹窗广告)本体常清净方可论元初。性烛须挑剔曹溪任吸呼勿令猿马气声粗。昼夜绵绵息方显是功夫。这一词牌名《南柯子》。单道着唐僧脱却通天河寒冰之灾踏白鼋负登彼岸。四众奔西正遇严冬之景但见那林光漠漠烟中淡山骨棱棱水外清。师徒们正当行处忽然又遇一座大山阻住去道路窄崖高石多岭峻人马难行。三藏在马上兜住缰绳叫声“徒弟。”那孙行者引八戒、沙僧近前侍立道:“师父有何吩咐?”三藏道:“你看那前面山高只恐有虎狼作怪妖兽伤人今番是必仔细!”行者道:“师父放心莫虑我等兄弟三人性和意合归正求真使出荡怪降妖之法怕甚么虎狼妖兽!”三藏闻言只得放怀前进到于谷口促马登崖抬头观看好山:嵯峨矗矗峦削巍巍。嵯峨矗矗冲霄汉峦削巍巍碍碧空。怪石乱堆如坐虎苍松斜挂似飞龙。岭上鸟啼娇韵美崖前梅放异香浓。涧水潺湲流出冷巅云黯淡过来凶。又见那飘飘雪凛凛风咆哮饿虎吼山中。寒鸦拣树无栖处野鹿寻窝没定踪。可叹行人难进步皱眉愁脸把头蒙。

师徒四众冒雪冲寒战澌澌行过那巅峰峻岭远望见山凹中有楼台高耸房舍清幽。唐僧马上欣然道:“徒弟啊这一日又饥又寒幸得那山凹里有楼台房舍断乎是庄户人家庵观寺院且去化些斋饭吃了再走。”行者闻言急睁睛看只见那壁厢凶云隐隐恶气纷纷回对唐僧道:“师父那厢不是好处。”三藏道:“见有楼台亭宇如何不是好处?”行者笑道:

“师父啊你那里知道?西方路上多有妖怪邪魔善能点化庄宅不拘甚么楼台房舍馆阁亭宇俱能指化了哄人。你知道龙生九种内有一种名‘蜃’蜃气放出就如楼阁浅池。若遇大江昏迷蜃现此势倘有鸟鹊飞腾定来歇翅那怕你上万论千尽被他一气吞之。此意害人最重那壁厢气色凶恶断不可入。”三藏道:“既不可入我却着实饥了。”行者道:“师父果饥且请下马就在这平处坐下待我别处化些斋来你吃。”三藏依言下马。八戒采定缰绳沙僧放下行李即去解开包裹取出钵盂递与行者。行者接钵盂在手吩咐沙僧道:“贤弟却不可前进好生保护师父稳坐于此待我化斋回来再往西去。”沙僧领诺。行者又向三藏道:“师父这去处少吉多凶切莫要动身别往老孙化斋去也。”唐僧道:“不必多言但要你快去快来我在这里等你。”行者转身欲行却又回来道:“师父我知你没甚坐性我与你个安身法儿。”即取金箍棒幌了一幌将那平地下周围画了一道圈子请唐僧坐在中间着八戒沙僧侍立左右把马与行李都放在近身对唐僧合掌道:“老孙画的这圈强似那铜墙铁壁凭他甚么虎豹狼虫妖魔鬼怪俱莫敢近。但只不许你们走出圈外只在中间稳坐保你无虞;但若出了圈儿定遭毒手。千万千万!至嘱至嘱!”三藏依言师徒俱端然坐下。

行者才起云头寻庄化斋一直南行忽见那古树参天乃一村庄舍。按下云头仔细观看但只见:雪欺衰柳冰结方塘。

疏疏修竹摇青郁郁乔松凝翠。几间茅屋半装银一座小桥斜砌粉。篱边微吐水仙花檐下长垂冰冻箸。飒飒寒风送异香雪漫不见梅开处。行者随步观看庄景只听得呀的一声柴扉响处走出一个老者手拖藜杖头顶羊裘身穿破衲足踏蒲鞋拄着杖仰身朝天道:“西北风起明日晴了。(WWW.mianhuatang.la 好看的小说)”说不了后边跑出一个哈巴狗儿来望着行者汪汪的乱吠。老者却才转过头来看见行者捧着钵盂打个问讯道:“老施主我和尚是东土大唐钦差上西天拜佛求经者适路过宝方我师父腹中饥馁特造尊府募化一斋。”老者闻言点头顿杖道:“长老你且休化斋你走错路了。”行者道:“不错。”老者道:“往西天大路在那直北下此间到那里有千里之遥还不去找大路而行?”行者笑道:“正是直北下我师父现在大路上端坐等我化斋哩。”

那老者道:“这和尚胡说了。你师父在大路上等你化斋似这千里之遥就会走路也须得六七日走回去又要六七日却不饿坏他也?”行者笑道:“不瞒老施主说我才然离了师父还不上一盏热茶之时却就走到此处。如今化了斋还要趁去作午斋哩。”老者见说心中害怕道:“这和尚是鬼!是鬼!”急抽身往里就走。行者一把扯住道:“施主那里去?有斋快化些儿。”老者道:“不方便!不方便!别转一家儿罢!”行者道:“你这施主好不会事!你说我离此有千里之遥若再转一家却不又有千里?

真是饿杀我师父也。”那老者道:“实不瞒你说我家老小六七口才淘了三升米下锅还未曾煮熟。你且到别处去转转再来。”行者道:“古人云走三家不如坐一家。我贫僧在此等一等罢。”那老者见缠得紧恼了举藜杖就打。行者公然不惧被他照光头上打了七八下只当与他拂痒。那老者道:“这是个撞头的和尚!”行者笑道:“老官儿凭你怎么打只要记得杖数明白一杖一升米慢慢量来。”那老者闻言急丢了藜杖跑进去把门关了只嚷:“有鬼!有鬼!”慌得那一家儿战战兢兢把前后门俱关上。行者见他关了门心中暗想:“这老贼才说淘米下锅不知是虚是实。常言道道化贤良释化愚。且等老孙进去看看。”好大圣捻着诀使个隐身遁法径走入厨中看处果然那锅里气腾腾的煮了半锅干饭。就把钵盂往里一桠满满的桠了一钵盂即驾云回转不题。

却说唐僧坐在圈子里等待多时。不见行者回来欠身怅望道:“这猴子往那里化斋去了?”八戒在旁笑道:“知他往那里耍子去来!化甚么斋却教我们在此坐牢!”三藏道:“怎么谓之坐牢?”八戒道:“师父你原来不知。古人划地为牢他将棍子划了圈儿强似铁壁铜墙假如有虎狼妖兽来时如何挡得他住?只好白白的送与他吃罢子。”三藏道:“悟能凭你怎么处治?”八戒道:“此间又不藏风又不避冷若依老猪只该顺着路往西且行。师兄化了斋驾了云必然来快让他赶来。如有斋吃了再走。如今坐了这一会老大脚冷!”三藏闻此言就是晦气星进宫遂依呆子一齐出了圈外。沙僧牵了马八戒担了担那长老顺路步行前进不一时到了那楼阁之所原来是坐北向南之家。门外八字粉墙有一座倒垂莲升斗门楼都是五色装的那门儿半开半掩。八戒就把马拴在门枕石鼓上沙僧歇了担子三藏畏风坐于门限之上。八戒道:“师父这所在想是公侯之宅相辅之家。前门外无人想必都在里面烘火。你们坐着让我进去看看。”唐僧道:“仔细耶!莫要冲撞了人家。”

呆子道:“我晓得自从归正禅门这一向也学了些礼数不比那村莽之夫也。”

那呆子把钉钯撒在腰里整一整青锦直裰斯斯文文走入门里只见是三间大厅帘栊高控静悄悄全无人迹也无桌椅家火。转过屏门往里又走乃是一座穿堂堂后有一座大楼楼上窗格半开隐隐见一顶黄绫帐幔。呆子道:“想是有人怕冷还睡哩。”他也不分内外拽步走上楼来用手掀开看时把呆子唬了一个躘踵。原来那帐里象牙床上白媸媸的一堆骸骨骷髅有巴斗大腿挺骨有四五尺长。呆子定了性止不住腮边泪落对骷髅点头叹云:“你不知是那代那朝元帅体何邦何国大将军。当时豪杰争强胜今日凄凉露骨筋。不见妻儿来侍奉那逢士卒把香焚?谩观这等真堪叹可惜兴王霸业人。”八戒正才感叹只见那帐幔后有火光一幌。呆子道:“想是有侍奉香火之人在后面哩。”急转步过帐观看却是穿楼的窗扇透光。

那壁厢有一张彩漆的桌子桌子上乱搭着几件锦绣绵衣。呆子提起来看时却是三件纳锦背心儿。他也不管好歹拿下楼来出厅房径到门外道:“师父这里全没人烟是一所亡灵之宅。

老猪走进里面直至高楼之上黄绫帐内有一堆骸骨。串楼旁有三件纳锦的背心被我拿来了也是我们一程儿造化此时天气寒冷正当用处。师父且脱了褊衫把他且穿在底下受用受用免得吃冷。”三藏道:“不可不可!律云:公取窃取皆为盗。倘或有人知觉赶上我们到了当官断然是一个窃盗之罪。还不送进去与他搭在原处!我们在此避风坐一坐等悟空来时走路出家人不要这等爱小。”八戒道:“四顾无人虽鸡犬亦不知之但只我们知道谁人告我?有何证见?就如拾到的一般那里论甚么公取窃取也!”三藏道:“你胡做啊!虽是人不知之天何盖焉!玄帝垂训云暗室亏心神目如电。趁早送去还他莫爱非礼之物。”那呆子莫想肯听对唐僧笑道:“师父啊我自为人也穿了几件背心不曾见这等纳锦的。你不穿且待老猪穿一穿试试新晤晤脊背。等师兄来脱了还他走路。”沙僧道:“既如此说我也穿一件儿。”两个齐脱了上盖直裰将背心套上。才紧带子不知怎么立站不稳扑的一跌。原来这背心儿赛过绑缚手霎时间把他两个背剪手贴心捆了。

慌得个三藏跌足报怨急忙上前来解那里便解得开?三个人在那里吆喝之声不绝却早惊动了魔头也。

话说那座楼房果是妖精点化的终日在此拿人。他在洞里正坐忽闻得怨恨之声急出门来看果见捆住几个人了。妖魔即唤小妖同到那厢收了楼台房屋之形把唐僧搀住牵了白马挑了行李将八戒沙僧一齐捉到洞里。老妖魔登台高坐众小妖把唐僧推近台边跪伏于地。妖魔问道:“你是那方和尚?

怎么这般胆大白日里偷盗我的衣服?”三藏滴泪告曰:“贫僧是东土大唐钦差往西天取经的因腹中饥馁着大徒弟去化斋未回不曾依得他的言语误撞仙庭避风。不期我这两个徒弟爱小拿出这衣物贫僧决不敢坏心当教送还本处。他不听吾言要穿此晤晤脊背不料中了大王机会把贫僧拿来。万望慈悯留我残生求取真经永注大王恩情回东土千古传扬也!”

那妖魔笑道:“我这里常听得人言:有人吃了唐僧一块肉白还黑齿落更生幸今日不请自来还指望饶你哩!你那大徒弟叫做甚么名字?往何方化斋?”八戒闻言即开口称扬道:“我师兄乃五百年前大闹天宫齐天大圣孙悟空也。”那妖魔听说是齐天大圣孙悟空老大有些悚惧口内不言心中暗想道:“久闻那厮神通广大如今不期而会。”教:“小的们把唐僧捆了将那两个解下宝贝换两条绳子也捆了。且抬在后边待我拿住他大徒弟一刷洗却好凑笼蒸吃。”众小妖答应一声把三人一齐捆了抬在后边将白马拴在槽头行李挑在屋里。众妖都磨兵器准备擒拿行者不题。

却说孙行者自南庄人家摄了一钵盂斋饭驾云回返旧路。

径至山坡平处按下云头早已不见唐僧不知何往棍划的圈子还在只是人马都不见了。回看那楼台处所亦俱无矣惟见山根怪石。行者心惊道:“不消说了!他们定是遭那毒手也!”

急依路看着马蹄向西而赶。行有五六里正在凄怆之际只闻得北坡外有人言语。看时乃一个老翁毡衣苫体暖帽蒙头足下踏一双半新半旧的油靴手持着一根龙头拐棒后边跟一个年幼的僮仆折一枝腊梅花自坡前念歌而走。行者放下钵盂觌面道个问讯叫:“老公公贫僧问讯了。”那老翁即便回礼道:“长老那里来的?”行者道:“我们东土来的往西天拜佛求经一行师徒四众。我因师父饥了特去化斋教他三众坐在那山坡平处相候。及回来不见不知往那条路上去了。动问公公可曾看见?”老者闻言呵呵冷笑道:“你那三众可有一个长嘴大耳的么?”行者道:“有有有!”“又有一个晦气色脸的牵着一匹白马领着一个白脸的胖和尚么?”行者道:“是是是!”

老翁道:“你们走错路了你休寻他各个顾命去也。”行者道:

“那白脸者是我师父那怪样者是我师弟。我与他共虔心要往西天取经如何不寻他去!”老翁道:“我才然从此过时看见他错走了路径闯入妖魔口里去了。”行者道:“烦公公指教指教是个甚么妖魔居于何方我好上门取索他等往西天去也。”老翁道:“这座山叫做金皘山山前有个金皘洞那洞中有个独角兕大王。那大王神通广大威武高强。那三众此回断没命了你若去寻只怕连你也难保不如不去之为愈也。我也不敢阻你也不敢留你只凭你心中度量”行者再拜称谢道:“多蒙公公指教我岂有不寻之理!”把这斋饭倒与他将这空钵盂自家收拾。那老翁放下拐棒接了钵盂递与僮仆现出本象双双跪下叩头叫:“大圣小神不敢隐瞒我们两个就是此山山神土地在此候接大圣。这斋饭连钵盂小神收下让大圣身轻好施法力。待救唐僧出难将此斋还奉唐僧方显得大丝至恭至孝。”行者喝道:“你这毛鬼讨打!既知我到何不早迎?却又这般藏头露尾是甚道理?”土地道:“大圣性急小神不敢造次恐犯威颜故此隐象告知。”行者息怒道:“你且记打!好生与我收着钵盂!待我拿那妖精去来!”土地山神遵领。

这大圣却才束一束虎筋绦拽起虎皮裙执着金箍棒径奔山前找寻妖洞。转过山崖只见那乱石磷磷翠崖边有两扇石门门外有许多小妖在那里轮枪舞剑真个是:烟云凝瑞苔藓堆青。崚嶒怪石列崎岖曲道萦。猿啸鸟啼风景丽鸾飞凤舞若蓬瀛。向阳几树梅初放弄暖千竿竹自青。陡崖之下深涧之中陡崖之下雪堆粉深涧之中水结冰。两林松柏千年秀几簇山茶一样红。这大圣观看不尽拽开步径至门前厉声高叫道:“那小妖你快进去与你那洞主说我本是唐朝圣僧徒弟齐天大圣孙悟空快教他送我师父出来免教你等丧了性命!”那伙小妖急入洞里报道:“大王前面有一个毛脸勾嘴的和尚称是齐天大圣孙悟空来要他师父哩。”那魔王闻得此言满心欢喜道:“正要他来哩!我自离了本宫下降尘世更不曾试试武艺。今日他来必是个对手。”即命:“小的们!取出兵器。”那洞中大小群魔一个个精神抖擞即忙抬出一根丈二长的点钢枪递与老怪。老怪传令教:“小的们各要整齐进前者赏退后者诛!”众妖得令随着老怪腾出门来叫道:“那个是孙悟空?”行者在旁闪过见那魔王生得好不凶丑:独角参差双眸幌亮。顶上粗皮突耳根黑肉光。舌长时搅鼻口阔版牙黄。毛皮青似靛筋挛硬如钢。比犀难照水象牯不耕荒。全无喘月犁云用倒有欺天振地强。两只焦筋蓝靛手雄威直挺点钢枪。细看这等凶模样不枉名称兕大王!孙大圣上前道:

“你孙外公在这里也!快早还我师父两无毁伤!若道半个不字我教你死无葬身之地!”那魔喝道:“我把你这个大胆泼猴精!你有些甚么手段敢出这般大言!”行者道:“你这泼物是也不曾见我老孙的手段!”那妖魔道:“你师父偷盗我的衣服实是我拿住了如今待要蒸吃。你是个甚么好汉就敢上我的门来取讨!”行者道:“我师父乃忠良正直之僧岂有偷你甚么妖物之理?”妖魔道:“我在山路边点化一座仙庄你师父潜入里面心爱情欲将我三领纳锦绵装背心儿偷穿在身只有赃证故此我才拿他。你今果有手段即与我比势假若三合敌得我饶了你师之命;如敌不过我教你一路归阴!”行者笑道:

“泼物!不须讲口!但说比势正合老孙之意。走上来吃吾之棒!”那怪物那怕甚么赌斗挺钢枪劈面迎来。这一场好杀!你看那:金箍棒举长杆枪迎。金箍棒举亮藿藿似电掣金蛇;长杆枪迎明幌幌如龙离黑海。那门前小妖擂鼓排开阵势助威风;这壁厢大圣施功使出纵横逞本事。他那里一杆枪精神抖擞;我这里一条棒武艺高强。正是英雄相遇英雄汉果然对手才逢对手人。那魔王口喷紫气盘烟雾这大圣眼放光华结绣云。只为大唐僧有难两家无义苦争轮。他两个战经三十合不分胜负。那魔王见孙悟空棍法齐整一往一来全无些破绽喜得他连声喝采道:“好猴儿!好猴儿!真个是那闹天官的本事!”这大圣也爱他枪法不乱右遮左挡甚有解数也叫道:

“好妖精!好妖精!果然是一个偷丹的魔头!”二人又斗了一二十合。那魔王把枪尖点地喝令小妖齐来。那些泼怪一个个拿刀弄杖执剑轮枪把个孙大圣围在中间。行者公然不惧只叫:“来得好!来得好!正合吾意!”使一条金箍棒前迎后架东挡西除那伙群妖莫想肯退。行者忍不住焦躁把金箍棒丢将起去喝声“变!”即变作千百条铁棒好便似飞蛇走蟒盈空里乱落下来。那伙妖精见了一个个魄散魂飞抱头缩颈尽往洞中逃命。老魔王唏唏冷笑道:“那猴不要无礼!看手段!”即忙袖中取出一个亮灼灼白森森的圈子来望空抛起叫声“着!”唿喇一下把金箍棒收做一条套将去了。弄得孙大圣赤手空拳翻筋斗逃了性命。那妖魔得胜回归洞行者朦胧失主张这正是:道高一尺魔高一丈性乱情昏错认家。可恨法身无坐位当时行动念头差。毕竟不知这番怎么结果且听下回分解。
------------

第五十一回 心猿空用千般计 水火无功难炼魔

话说齐天大圣空着手败了阵来坐于金皘山后扑梭梭两眼滴泪叫道:“师父啊!指望和你:佛恩有德有和融同幼同生意莫穷。同住同修同解脱同慈同念显灵功。同缘同相心真契同见同知道转通。岂料如今无主杖空拳赤脚怎兴隆!”大圣凄惨多时心中暗想道:“那妖精认得我。我记得他在阵上夸奖道:‘真个是闹天宫之类!’这等啊决不是凡间怪物定然是天上凶星。想因思凡下界又不知是那里降下来魔头且须上界去查勘查勘。”

行者这才是以心问心自张自主急翻身纵起祥云直至南天门外忽抬头见广目天王当面迎着长揖道:“大圣何往?”

行者道:“有事要见玉帝你在此何干?”广目道:今日轮该巡视南天门。”说未了又见那马赵温关四大元帅作礼道:“大圣失迎请待茶。”行者道:“有事哩。”遂辞了广目并四元帅径入南天门里直至灵霄殿外果又见张道陵、葛仙翁、许旌阳、丘弘济四天师并南斗六司、北斗七元都在殿前迎着行者一齐起手道:“大圣如何到此?”又问:“保唐僧之功完否?”行者道:“早哩早哩!路遥魔广才有一半之功见如今阻住在金皘山金皘洞。

有一个兕怪把唐师父拿于洞里是老孙寻上门与他交战一场那厮的神通广大把老孙的金箍棒抢去了因此难缚魔王。

疑是上界那个凶星思凡下界又不知是那里降来的魔头老孙因此来寻寻玉帝问他个钳束不严。”许旌阳笑道:“这猴头还是如此放刁!”行者道:“不是放刁我老孙一生是这口儿紧些才寻的着个头儿。”张道陵道:“不消多说只与他传报便了。”

行者道:“多谢多谢!”当时四天师传奏灵霄引见玉陛。行者朝上唱个大喏道:“老官儿累你累你!我老孙保护唐僧往西天取经一路凶多吉少也不消说。于今来在金山兜山金山兜洞有一兕怪把唐僧拿在洞里不知是要蒸要煮要晒。是老孙寻上他门与他交战那怪却就有些认得老孙卓是神通广大把老孙的金箍棒抢去因此难缚妖魔。疑是上天凶星思凡下界为此老孙特来启奏伏乞天尊垂慈洞鉴降旨查勘凶星兵收剿妖魔老孙不胜战栗屏营之至!”却又打个深躬道:“以闻。”旁有葛仙翁笑道:“猴子是何前倨后恭?”行者道:“不敢不敢!不是甚前倨后恭老孙于今是没棒弄了。”

彼时玉皇天尊闻奏即忙降旨可韩司知道:“既如悟空所奏可随查诸天星斗各宿神王有无思凡下界随即复奏施行以闻。”可韩丈人真君领旨当时即同大圣去查。先查了四天门门上神王官吏;次查了三微垣垣中大小群真;又查了雷霆官将陶张辛邓苟毕庞刘;最后才查三十三天天天自在;又查二十八宿:东七宿角亢氏房参尾箕西七宿斗牛女虚危室壁南七宿北七宿宿宿安宁;又查了太阳太阴水火木金土七政;罗睺计都噹孛四余。满天星斗并无思凡下界。行者道:“既是如此我老孙也不消上那灵霄宝殿打搅玉皇大帝深为不便。你自回旨去罢我只在此等你回话便了。”那可韩丈人真君依命。

孙行者等候良久作诗纪兴曰:“风清云霁乐升平神静星明显瑞祯。河汉安宁天地泰五方八极偃戈旌。”

那可韩司丈人真君历历查勘回奏玉帝道:“满天星宿不少各方神将皆存并无思凡下界者。”玉帝闻奏:“着孙悟空挑选几员天将下界擒魔去也。”四大天师奉旨意即出灵霄宝殿对行者道:“大圣啊玉帝宽恩言天宫无神思凡着你挑选几员天将擒魔去哩。”行者低头暗想道:“天上将不如老孙者多胜似老孙者少。想我闹天宫时玉帝遣十万天兵布天罗地网更不曾有一将敢与我比手。向后来调了小圣二郎方是我的对手。如今那怪物手段又强似老孙却怎么得能彀取胜?”许旌阳道:“此一时彼一时大不同也。常言道一物降一物哩你好违了旨意?但凭高见选用天将勿得迟疑误事。”行者道:

“既然如此深感上恩。果是不好违旨。一则老孙又不可空走这遭烦旌阳转奏玉帝只教托塔李天王与哪吒太子他还有几件降妖兵器且下界与那怪见一仗以看如何。果若能擒得他是老孙之幸;若不能那时再作区处。”

真个那天师启奏了玉帝玉帝即令李天王父子率领众部天兵与行者助力。那天王即奉旨来会行者行者又对天师道:

“蒙玉帝遣差天王谢谢不尽。还有一事再烦转达:但得两个雷公使用等天王战斗之时教雷公在云端里下个雷捎照顶门上锭死那妖魔深为良计也。”天师笑道:“好!好!好!”天师又奏玉帝传旨教九天府下点邓化、张蕃二雷公与天王合力缚妖救难。遂与天王、孙大圣径下南天门外。

顷刻而到行者道:“此山便是金皘山山中间乃是金皘洞。列位商议却教那个先去索战?”天王停下云头扎住天兵在于山南坡下道:“大圣素知小儿哪吒曾降九十六洞妖魔善能变化随身有降妖兵器须教他先去出阵。”行者道:“既如此等老孙引太子去来。”那太子抖擞雄威与大圣跳在高山径至洞口但见那洞门紧闭崖下无精。行者上前高叫:“泼魔!

快开门!还我师父来也!”那洞里把门的小妖看见急报道:“大王孙行者领着一个小童男在门前叫战哩。”那魔王道:“这猴子铁棒被我夺了空手难争想是请得救兵来也。”叫:“取兵器!”魔王绰枪在手走到门外观看那小童男生得相貌清奇十分精壮。真个是:玉面娇容如满月朱唇方口露银牙。眼光掣电睛珠暴额阔凝霞髻髽。绣带舞风飞彩焰锦袍映日放金花。环绦灼灼攀心镜宝甲辉辉衬战靴。身小声洪多壮丽三天护教恶哪吒。魔王笑道:“你是李天王第三个孩儿名唤做哪吒太子却如何到我这门前呼喝?”太子道:“因你这泼魔作乱困害东土圣僧奉玉帝金旨特来拿你!”魔王大怒道:“你想是孙悟空请来的。我就是那圣僧的魔头哩!量你这小儿曹有何武艺敢出浪言!不要走!吃吾一枪!”这太子使斩妖剑劈手相迎。他两个搭上手却才赌斗那大圣急转山坡叫:“雷公何在?快早去着妖魔下个雷捎助太子降伏来也!”邓张二公即踏云光正欲下手只见那太子使出法来将身一变变作三头六臂手持六般兵器望妖魔砍来那魔王也变作三头六臂三柄长枪抵住。这太子又弄出降妖法力将六般兵器抛将起去是那六般兵器?却是砍妖剑、斩妖刀、缚妖索、降魔杵、绣球、火轮儿大叫一声“变!”一变十十变百百变千千变万都是一般兵器如骤雨冰雹纷纷密密望妖魔打将去。那魔王公然不惧一只手取出那白森森的圈子来望空抛起叫声“着!”唿喇的一下把六般兵器套将下来慌得那哪吒太子赤手逃生魔王得胜而回。

邓张二雷公在空中暗笑道:“早是我先看头势不曾放了雷捎假若被他套将去却怎么回见天尊?”二公按落云头与太子来山南坡下对李天王道:“妖魔果神通广大!”悟空在旁笑道:“那厮神通也只如此争奈那个圈子利害。不知是甚么宝贝丢起来善套诸物。”哪吒恨道:“这大圣甚不成*人!我等折兵败阵十分烦恼都只为你你反喜笑何也!”行者道:“你说烦恼终然我老孙不烦恼?我如今没计奈何哭不得所以只得笑也。”天王道:“似此怎生结果?”行者道:“凭你等再怎计较只是圈子套不去的就可拿住他了。”天王道:“套不去者惟水火最利。常言道水火无情。”行者闻言道:“说得有理!你且稳坐在此待老孙再上天走走来。”邓、张二公道:“又去做甚的?”行者道:“老孙这去不消启奏玉帝只到南天门里上彤华宫请荧惑火德星君来此放火烧那怪物一场或者连那圈子烧做灰烬捉住妖魔。一则取兵器还汝等归天二则可解脱吾师之难。”太子闻言甚喜道:“不必迟疑请大圣早去早来我等只在此拱候。”

行者纵起祥光又至南天门外那广目与四将迎道:“大圣如何又来?”行者道:“李天王着太子出师只一阵被那魔王把六件兵器捞了去了。我如今要到彤华宫请火德星君助阵哩。”

四将不敢久留让他进去。至彤华宫只见那火部众神即入报道:“孙悟空欲见主公。”那南方三噹火德星君整衣出门迎进道:“昨日可韩司查点小宫更无一人思凡。”行者道:“已知但李天王与太子败阵失了兵器特来请你救援救援。”星君道:

“那哪吒乃三坛海会大神他出身时曾降九十六洞妖魔神通广大若他不能小神又怎敢望也?”行者道:“因与李天王计议天地间至利者惟水火也。那怪物有一个圈子善能套人的物件不知是甚么宝贝故此说火能灭诸物特请星君领火部到下方纵火烧那妖魔救我师父一难。”火德星君闻言即点本部神兵同行者到金皘山南坡下与天王、雷公等相见了。天王道:“孙大圣你还去叫那厮出来等我与他交战待他拿动圈子我却闪过教火德帅众烧他。”行者笑道:“正是我和你去来。”火德共太子、邓、张二公立于高峰之上与他挑战。

这大圣到了金皘洞口叫声“开门!快早还我师父!”那妖又急通报道:“孙悟空又来了!”那魔帅众出洞见了行者道:

“你这泼猴又请了甚么兵来耶?”这壁厢转上托塔天王喝道:

“泼魔头!认得我么?”魔王笑道:“李天王想是要与你令郎报仇欲讨兵器么?”天王道:“一则报仇要兵器二来是拿你救唐僧!不要走!吃吾一刀!”那怪物侧身躲过挺长枪随手相迎。

他两个在洞前这场好杀!你看那:天王刀砍妖怪枪迎。刀砍霜光喷烈火枪迎锐气迸愁云。一个是金皘山生成的恶怪一个是灵霄殿差下的天神。那一个因欺禅性施威武这一个为救师灾展大伦。天王使法飞沙石魔怪争强播土尘。播土能教天地暗飞沙善着海江浑。两家努力争功绩皆为唐僧拜世尊。

那孙大圣见他两个交战即转身跳上高峰对火德星君道:“三噹用心者!”你看那个妖魔与天王正斗到好处却又取出圈子来天王看见即拨祥光败阵而走。这高峰上火德星君忙传号令教众部火神一齐放火。这一场真个利害。好火:

经云“南方者火之精也。”虽星星之火能烧万顷之田;乃三噹之威能变百端之火。今有火枪、火刀、火弓、火箭各部神祇所用不一但见那半空中火鸦飞噪;满山头火马奔腾。双双赤鼠对对火龙。双双赤鼠喷烈焰万里通红;对对火龙吐浓烟千方共黑。火车儿推出火葫芦撒开。火旗摇动一天霞火棒搅行盈地燎。说甚么宁戚鞭牛胜强似周郎赤壁。这个是天火非凡真利害烘烘焃焃火风红!那妖魔见火来时全无恐惧将圈子望空抛起唿喇一声把这火龙火马火鸦火鼠火枪火刀火弓火箭一圈子又套将下去转回本洞得胜收兵。

这火德星君手执着一杆空旗招回众将会合天王等坐于山南坡下对行者道:“大圣啊这个凶魔真是罕见!我今折了火具怎生是好?”行者笑道:“不须报怨列位且请宽坐坐待老孙再去去来。”天王道:“你又往那里去?”行者道:“那怪物既不怕火断然怕水。常言道水能克火。等老孙去北天门里请水德星君施布水势往他洞里一灌把魔王渰死取物件还你们。”天王道:“此计虽妙但恐连你师父都渰杀也。”行者道:

“没事!渰死我师我自有个法儿教他活来。如今稽迟列位甚是不当。”火德道:“既如此且请行请行。”

好大圣又驾筋斗云径到北天门外忽抬头见多闻天王向前施礼道:“孙大圣何往?”行者道:“有一事要入乌浩宫见水德星君你在此作甚?”多闻道:“今日轮该巡视。”正说处又见那庞刘苟毕四大天将进礼邀茶。行者道:“不劳不劳!我事急矣!”遂别却诸神直至乌浩宫着水部众神即时通报。众神报道:“齐天大圣孙悟空来了。”水德星君闻言即将查点四海五湖、八河四渎、三江九派并各处龙王俱遣退整冠束带接出宫门迎进宫内道:“昨日可韩司查勘小宫恐有本部之神思凡作怪正在此点查江海河渎之神尚未完也”行者道:“那魔王不是江河之神此乃广大之精。先蒙玉帝差李天王父子并两个雷公下界擒拿被他弄个圈子将六件神兵套去。老孙无奈又上彤华宫请火德星君帅火部众神放火又将火龙火马等物一圈子套去。我想此物既不怕火必然怕水特来告请星君施水势与我捉那妖精取兵器归还天将。吾师之难亦可救也。”水德闻言即令黄河水伯神王:“随大圣去助功。”水伯自衣袖中取出一个白玉盂儿道:“我有此物盛水。”行者道:“看这盂儿能盛几何?妖魔如何渰得?”水伯道:“不瞒大圣说。我这一盂乃是黄河之水。半盂就是半河一盂就是一河。”行者喜道:“只消半盂足矣。”遂辞别水德与黄河神急离天阙。

那水伯将盂儿望黄河舀了半盂跟大圣至金嶒山向南坡下见了天王、太子、雷公、火德具言前事行者道:“不必细讲且教水伯跟我去。待我叫开他门不要等他出来就将水往门里一倒那怪物一窝子可都渰死我却去捞师父的尸再救活不迟。”那水伯依命紧随行者转山坡径至洞口叫声“妖怪开门!”那把门的小妖听得是孙大圣的声音急又去报道:

“孙悟空又来矣!”那魔闻说带了宝贝绰枪就走响一声开了石门。这水伯将白玉盂向里一倾那妖见是水来撒了长枪即忙取出圈子撑住二门。只见那股水骨都都的都往外泛将出来慌得孙大圣急纵筋斗与水伯跳在高峰。那天王同众都驾云停于高峰之前观看那水波涛泛涨着实狂澜。好水!真个是:一勺之多果然不测。盖唯神功运化利万物而流涨百川。

只听得那潺潺声振谷又见那滔滔势漫天。雄威响若雷奔走猛涌波如雪卷颠。千丈波高漫路道万层涛激泛山岩。冷冷如漱玉滚滚似鸣弦。触石沧沧喷碎玉回湍渺渺漩窝圆。低低凹凹随流荡满涧平沟上下连。行者见了心慌道:“不好啊!水漫四野渰了民田未曾灌在他的洞里曾奈之何?”唤水伯急忙收水。水伯道:“小神只会放水却不会收水常言道泼水难收。”咦!那座山却也高峻这场水只奔低流。须臾间四散而归涧壑。

又只见那洞外跳出几个小妖在外边吆吆喝喝伸拳逻袖弄棒拈枪依旧喜喜欢欢耍子。天王道:“这水原来不曾灌入洞内枉费一场之功也!”行者忍不住心中怒双手轮拳闯至妖魔门喝道:“那里走!看打!”唬得那几个小妖丢了枪棒跑入洞里战兢兢的报道:“大王打将来了!”魔王挺长枪迎出门前道:“这泼猴老大惫懒!你几番家敌不过我纵水火亦不能近怎么又踵将来送命?”行者道:“这儿子反说了哩!

不知是我送命是你送命!走过来吃老外公一拳!”那妖魔笑道:“这猴儿强勉缠帐!我倒使枪他却使拳。那般一个筋骷子拳头只好有个核桃儿大小怎么称得个锤子起也?罢!罢!罢!

我且把枪放下与你走一路拳看看!”行者笑道:“说得是!走上来!”那妖撩衣进步丢了个架子举起两个拳来真似打油的铁锤模样。这大圣展足挪身摆开解数在那洞门前与那魔王递走拳势。这一场好打!咦!拽开大四平踢起双飞脚。韬胁劈胸墩剜心摘胆着。仙人指路老子骑鹤。饿虎扑食最伤人蛟龙戏水能凶恶。魔王使个蟒翻身大圣却施鹿解角。翘跟淬地龙扭腕拿天橐。青狮张口来鲤鱼跌脊跃。盖顶撒花绕腰贯索。迎风贴扇儿急雨催花落。妖精便使观音掌行者就对罗汉脚。长掌开阔自然松怎比短拳多紧削?两个相持数十回一般本事无强弱。他两个在那洞门前厮打只见这高峰头喜得个李天王厉声喝采火德星鼓掌夸称。那两个雷公与哪吒太子帅众神跳到跟前都要来相助;这壁厢群妖摇旗擂鼓舞剑轮刀一齐护。孙大圣见事不谐将毫毛拔下一把望空撒起叫“变!”即变做三五十个小猴一拥上前把那妖缠住抱腿的抱腿扯腰的扯腰抓眼的抓眼挦毛的挦毛。那怪物慌了急把圈子拿将出来。大圣与天王等见他弄出圈套拨转云头走上高峰逃阵。那妖把圈子往上抛起唿喇的一声把那三五十个毫毛变的小猴收为本相套入洞中得了胜领兵闭门贺喜而去。

这太子道:“孙大圣还是个好汉!这一路拳走得似锦上添花。使分身法正是人前显贵。”行者笑道:“列位在此远观那怪的本事比老孙如何?”李天王道:“他拳松脚慢不如大圣的紧疾他见我们去时也就着忙;又见你使出分身法来他就急了所以大弄个圈套。”行者道:“魔王好治只是套子难降。”火德与水伯道:“若还取胜除非得了他那宝贝然后可擒。”行者道:“他那宝贝如何可得?只除是偷去来。”邓张二公笑道:“若要行偷礼除大圣再无能者想当年大闹天宫时偷御酒偷蟠桃偷龙肝凤髓及老君之丹那是何等手段!今日正该拿此处用也。”行者道:“好说好说!既如此你们且坐等老孙打听去来。”好大圣跳下峰头私至洞口摇身一变变做个麻苍蝇儿。

真个秀溜!你看他:翎翅薄如竹膜身躯小似花心。手足比毛更奘星星眼窟明明。善自闻香逐气飞时迅乘风。称来刚压定盘星可爱些些有用。轻轻的飞在门上爬到门缝边钻进去只见那大小群妖舞的舞唱的唱排列两旁;老魔王高坐台上面前摆着些蛇肉、鹿脯、熊掌、驼峰、山蔬果品有一把青磁酒壶香喷喷的羊酪椰醪大碗家宽怀畅饮。行者落于小妖丛里又变做一个獾头精慢慢的演近台边看彀多时全不见宝贝放在何方。急抽身转至台后又见那后厅上高吊着火龙吟啸火马号嘶。忽抬头见他的那金箍棒靠在东壁喜得他心痒难挝忘记了更容变象走上前拿了铁棒现原身丢开解数一路棒打将出去。慌得那群妖胆战心惊老魔王措手不及却被他推倒三个放倒两个打开一条血路径自出了洞门。这才是:魔头骄傲无防备主杖还归与本人。毕竟不知吉凶如何且听下回分解。
------------

第五十二回 悟空大闹金山兜洞 如来暗示主人公

话说孙大圣得了金箍棒打出门前跳上高峰对众神满心欢喜。(WWW.mianhuatang.la 好看的小说)李天王道:“你这场如何”行者道:“老孙变化进他洞去那怪物越唱唱舞舞的吃得胜酒哩更不曾打听得他的宝贝在那里。我转他后面忽听得马叫龙吟知是火部之物。东壁厢靠着我的金箍棒是老孙拿在手中一路打将出来也。”众神道:“你的宝贝得了我们的宝贝何时到手?”行者道:“不难!

不难!我有了这根铁棒不管怎的也要打倒他取宝贝还你。”

正讲处只听得那山坡下锣鼓齐鸣喊声振地原来是兕大王帅众精灵来赶行者。行者见了叫道:“好!好!好!正合吾意!

列位请坐待老孙再去捉他。”

好大圣举铁棒劈面迎来喝道:“泼魔那里走!看棍!”那怪使枪支住骂道:“贼猴头!着实无礼!你怎么白昼劫吾物件?”行者道:“我把你这个不知死的孽畜!你倒弄圈套白昼抢夺我物!那件儿是你的?不要走!吃老爷一棍!”那怪物轮枪隔架。这一场好战:大圣施威猛妖魔不顺柔。两家齐斗勇那个肯干休!这一个铁棒如龙尾那一个长枪似蟒头。这一个棒来解数如风响那一个枪架雄威似水流。只见那彩雾朦朦山岭暗祥云叆叆树林愁。满空飞鸟皆停翅四野狼虫尽缩头。那阵上小妖呐喊这壁厢行者抖擞。一条铁棒无人敌打遍西方万里游。那杆长枪真对手永镇金皘称上筹。相遇这场无好散不见高低誓不休。那魔王与孙大圣战经三个时辰不分胜败早又见天色将晚。妖魔支着长枪道:“悟空你住了天昏地暗不是个赌斗之时且各歇息歇息明朝再与你比迸。”行者骂道:“泼畜休言!老孙的兴头才来管甚么天晚!是必与你定个输赢!”那怪物喝一声虚幌一枪逃了性命帅群妖收转干戈入洞中将门紧紧闭了。

这大圣拽棍方回天神在岸头贺喜都道:“是有能有力的大齐天无量无边的真本事!”行者笑道:“承过奖!承过奖!”李天王近前道:“此言实非褒奖真是一条好汉子!这一阵也不亚当时瞒地网罩天罗也!”行者道:“且休题夙话。那妖魔被老孙打了这一场必然疲倦。我也说不得辛苦你们都放怀坐坐等我再进洞去打听他的圈子务要偷了他的捉住那怪寻取兵器奉还汝等归天。”太子道:“今已天晚不若安眠一宿明早去罢。”行者笑道:“这小郎不知世事!那见做贼的好白日里下手?似这等掏摸的必须夜去夜来不知不觉才是买卖哩。”火德与雷公道:“三太子休言这件事我们不知大圣是个惯家熟套须教他趁此时候一则魔头困倦二来夜黑无防就请快去!快去!”

好大圣笑唏唏的将铁棒藏了跳下高峰又至洞口摇身一变变作一个促织儿真个嘴硬须长皮黑眼明爪脚丫叉。

风清月明叫墙涯夜静如同人话。泣露凄凉景色声音断续堪夸。客窗旅思怕闻他偏在空阶床下。蹬开大腿三五跳跳到门边自门缝里钻将进去蹲在那壁根下迎着里面灯光仔细观看。只见那大小群妖一个个狼餐虎咽正都吃东西哩。行者揲揲锤锤的叫了一遍。少时间收了家火又都去安排窝铺各各安身。约摸有一更时分行者才到他后边房里只听那老魔传令教:“各门上小的醒睡!恐孙悟空又变甚么私入家偷盗。”又有些该班坐夜的涤涤托托梆铃齐响这大圣越好行事钻入房门见有一架石床左右列几个抹粉搽胭的山精树鬼展铺盖伏侍老魔脱脚的脱脚解衣的解衣。只见那魔王宽了衣服左肐膊上白森森的套着那个圈子原来象一个连珠镯头模样。你看他更不取下转往上抹了两抹紧紧的勒在肐膊上方才睡下。行者见了将身又变变作一个黄皮虼蚤跳上石床钻入被里爬在那怪的肐膊上着实一口叮的那怪翻身骂道:“这些少打的奴才!被也不抖床也不拂不知甚么东西咬了我这一下!”他却把圈子又捋上两捋依然睡下。mianhuatang.la [棉花糖小说网]行者爬上那圈子又咬一口。那怪睡不得又翻过身来道:“刺闹杀我也!”

行者见他关防得紧宝贝又随身不肯除下料偷他的不得。跳下床来还变做促织儿出了房门径至后面又听得龙吟马嘶原来那层门紧锁火龙火马都吊在里面。行者现了原身走近门前使个解锁法念动咒语用手一抹扢扠一声那锁双鐄俱就脱落推开门闯将进去观看原来那里面被火器照得明晃晃的如白日一般。忽见东西两边斜靠着几件兵器都是太子的砍妖刀等物并那火德的火弓火箭等物。行者映火光周围看了一遍又见那门背后一张石桌子上有一个篾丝盘儿放着一把毫毛。大圣满心欢喜将毫毛拿起来呵了两口热气叫声“变!”即变作三五十个小猴教他都拿了刀、剑、杵、索、球、轮及弓、箭枪、车、葫芦、火鸦、火鼠、火马一应套去之物骑了火龙纵起火势从里边往外烧来。只听得烘烘焃焃扑扑乒乒好便似咋雷连炮之声。慌得那些大小妖精梦梦查查的披着被朦着头喊的喊哭的哭一个个走头无路被这火烧死大半。美猴王得胜回来只好有三更时候。却说那高峰上李天王众位忽见火光幌亮一拥前来见行者骑着龙喝喝呼呼纵着小猴径上峰头厉声高叫道:“来收兵器!来收兵器!”火德与哪吒答应一声这行者将身一抖那把毫毛复上身来。哪吒太子收了他六件兵器火德星君着众火部收了火龙等物都笑吟吟赞贺行者不题。

却说那金皘洞里火焰纷纷唬得个兕大王魂不附体急欠身开了房门双手拿看圈子东推东火灭西推西火消满空中冒烟突火执着宝贝跑了一遍四下里烟火俱熄。急忙收救群妖已此烧杀大半男男女女收不上百十余丁;又查看藏兵之内各件皆无;又去后面看处见八戒、沙僧与长老还捆住未解白龙马还在槽上行李担亦在屋里。妖魔遂恨道:“不知是那个小妖不仔细失了火致令如此!”旁有近侍的告道:“大王这火不干本家之事多是个偷营劫寨之贼放了那火部之物盗了神兵去也。”老魔方然省悟道:“没有别人断乎是孙悟空那贼!怪道我临睡时不得安稳!想是那贼猴变化进来在我这肐膊叮了两口。一定是要偷我的宝贝见我抹勒得紧不能下手故此盗了兵器纵着火龙放此狠毒之心意欲烧杀我也。贼猴啊!你枉使机关不知我的本事!我但带了这件宝贝就是入大海而不能溺赴火池而不能焚哩!这番若拿住那贼只把刮了点垛方趁我心!”说着话懊恼多时不觉的鸡鸣天晓。

那高峰上太子得了六件兵器对行者道:“大圣天色已明不须怠慢。我们趁那妖魔挫了锐气与火部等扶住你再去力战庶几这次可擒拿也。”行者笑道:“说得有理。我们齐了心耍子儿去耶!”一个个抖擞威风喜弄武艺径至洞口。行者叫道:“泼魔出来!与老孙打者!”原来那里两扇石门被火气化成灰烬门里边有几个小妖正然扫地撮灰忽见众圣齐来慌得丢了扫帚撇下灰耙跑入里面又报道:“孙悟空领着许多天神又在门外骂战哩!”那兕怪闻报大惊扢迸迸钢牙咬响;

滴溜溜环眼睁圆挺着长枪带了宝贝走出门来泼口乱骂道:“我把你这个偷营放火的贼猴!你有多大手段敢这等藐视我也?”行者笑脸儿骂道:“泼怪物!你要知我的手段且上前来我说与你听:自小生来手段强乾坤万里有名扬。当时颖悟修仙道昔日传来不老方。立志拜投方寸地虔心参见圣人乡。

学成变化无量法宇宙长空任我狂。闲在山前将虎伏闷来海内把龙降。祖居花果称王位水帘洞里逞刚强。几番有意图天界数次无知夺上方。御赐齐天名大圣敕封又赠美猴王。只因宴设蟠桃会无简相邀我性刚。暗闯瑶池偷玉液私行空阁饮琼浆;龙肝凤髓曾偷吃百味珍馐我窃尝;千载蟠桃随受用万年丹药任充肠。天宫异物般般取圣府奇珍件件藏。玉帝访我有手段即天兵摆战场。九曜恶星遭我贬五方凶宿被吾伤。普天神将皆无敌十万雄师不敢当。威逼玉皇传旨意灌江小圣把兵扬。相持七十单二变各弄精神个个强。南海观音来助战净瓶杨柳也相帮。老君又使金刚套把我擒拿到上方。

绑见玉皇张大帝曹官拷较罪该当。即差大力开刀斩刀砍头皮火焰光。百计千方弄不死将吾押赴老君堂。六丁神火炉中炼炼得浑身硬似钢。七七数完开鼎看我身跳出又凶张。诸神闭户无遮挡众圣商量把佛央。其实如来多法力果然智慧广无量。手中赌赛翻筋斗将山压我不能强。玉皇才设安天会西域方称极乐场。压困老孙五百载一些茶饭不曾尝。金蝉长老临凡世东土差他拜佛乡。欲取真经回上国大唐帝主度先亡。观音劝我皈依善秉教迦持不放狂。解脱高山根下难如今西去取经章。泼魔休弄獐狐智还我唐僧拜法王!”那怪闻言指着行者道:“你原来是个偷天的大贼!不要走!吃吾一枪!”这大圣使棒来迎。两个正自相持这壁厢哪吒太子生嗔火德星君狠即将那六件神兵火部等物望妖魔身上抛来孙大圣更加雄势。一边又雷公使捎天王举刀不分上下一拥齐来。那魔头巍巍冷笑袖子中暗暗将宝贝取出撒手抛起空中叫声“着!”唿喇的一下把六件神兵、火部等物、雷公捎、天王刀、行者棒尽情又都捞去众神灵依然赤手孙大圣仍是空拳。妖魔得胜回身叫:“小的们搬石砌门动土修造从新整理房廊。待齐备了杀唐僧三众来谢土大家散福受用。”众小妖领命维持不题。

却说那李天王帅众回上高峰火德怨哪吒性急雷公怪天王放刁惟水伯在旁无语。行者见他们面不厮睹心有萦思没奈何怀恨强欢对众笑道:“列位不须烦恼自古道胜败兵家之常。我和他论武艺也只如此。但只是他多了这个圈子所以为害把我等兵器又套将去了。你且放心待老孙再去查查他的脚色来也。”太子道:“你前启奏玉帝查勘满天世界更无一点踪迹如今却又何处去查?”行者道:“我想起来佛法无边如今且上西天问我佛如来教他着慧眼观看大地四部洲看这怪是那方生长何处乡贯住居圈子是件甚么宝贝。不管怎的一定要拿他与列位出气还汝等欢喜归天。”众神道:

“既有此意不须久停快去快去!”

好行者说声去就纵筋斗云早至灵山落下祥光四方观看好去处:灵峰疏杰迭嶂清佳仙岳顶巅摩碧汉。西天瞻巨镇形势压中华。元气流通天地远威风飞彻满台花。时闻钟磬音长每听经声明朗。又见那青松之下优婆讲翠柏之间罗汉行。白鹤有情来鹫岭青鸾着意佇闲亭。玄猴对对擎仙果寿鹿双双献紫英。幽鸟声频如诉语奇花色绚不知名。回峦盘绕重重顾古道湾环处处平。正是清虚灵秀地庄严大觉佛家风。那行者正然点看山景忽听得有人叫道:“孙悟空从那里来?往何处去?”急回头看原来是比丘尼尊者。大圣作礼道:

“正有一事欲见如来。”比丘尼道:“你这个顽皮!既然要见如来怎么不登宝刹且在这里看山?”行者道:“初来贵地故此大胆。”比丘尼道:“你快跟我来也。”这行者紧随至雷音寺山门下又见那八大金刚雄纠纠的两边挡住比丘尼道。“悟空暂候片时等我与你奏上去来。”行者只得住立门外。那比丘尼至佛前合掌道:“孙悟空有事要见如来。”如来传旨令入金刚才闪路放行。行者低头礼拜毕如来问道:“悟空前闻得观音尊者解脱汝身皈依释教保唐僧来此求经你怎么独自到此?有何事故?”行者顿道:“上告我佛弟子自秉迦持与唐朝师父西来行至金皘山金皘洞遇着一个恶魔头名唤兕大王神通广大把师父与师弟等摄入洞中。弟子向伊求取没好意两家比迸被他将一个白森森的一个圈子抢了我的铁棒。我恐他是天将思凡急上界查勘不出。蒙玉帝差遣李天王父子助援又被他抢了太子的六般兵器。及请火德星君放火烧他又被他将火具抢去。又请水德星君放水渰他一毫又渰他不着弟子费若干精神气力将那铁棒等物偷出复去索战又被他将前物依然套去无法收降因此特告我佛望垂慈与弟子看看果然是何物出身我好去拿他家属四邻擒此魔头救我师父合拱虔诚拜求正果。”如来听说将慧眼遥观早已知识对行者道:“那怪物我虽知之但不可与你说。你这猴儿口敞一传道是我说他他就不与你斗定要嚷上灵山反遗祸于我也。我这里着法力助你擒他去罢。”行者再拜称谢道:“如来助我甚么法力”如来即令十八尊罗汉开宝库取十八粒“金丹砂”与悟空助力。行者道:“金丹砂却如何?”如来道:“你去洞外叫那妖魔比试。演他出来却教罗汉放砂陷住他使他动不得身拔不得脚凭你揪打便了。”行者笑道:“妙!妙!妙!趁早去来!”那罗汉不敢迟延即取金丹砂出门行者又谢了如来。一路查看止有十六尊罗汉行者嚷道:“这是那个去处却卖放人!”众罗汉道:“那个卖放?”行者道:“原差十八尊今怎么只得十六尊?”

说不了里边走出降龙、伏虎二尊上前道:“悟空怎么就这等放刁?我两个在后听如来吩咐话的。”行者道:“忒卖法!忒卖法!才自若嚷迟了些儿你敢就不出来了。”众罗汉笑呵呵驾起祥云。

不多时到了金皘山界。那李天王见了帅众相迎备言前事。罗汉道:“不必絮繁快去叫他出来。”这大圣捻着拳头来于洞口骂道:“泼怪物快出来与你孙外公见个上下!”那小妖又飞跑去报魔王怒道:“这贼猴又不知请谁来猖獗也!”小妖道:“更无甚将止他一人。”魔王道:“那根棒子已被我收来怎么却又一人到此?敢是又要走拳?”随带了宝贝绰枪在手叫小妖搬开石块跳出门来骂道:“贼猴!你几番家不得便宜就该回避如何又来吆喝?”行者道:“这泼魔不识好歹!若要你外公不来除非你服了降陪了礼送出我师父师弟我就饶你!”

那怪道:“你那三个和尚已被我洗净了不久便要宰杀你还不识起倒!去了罢!”行者听说宰杀二字扢蹬蹬腮边火按不住心头之怒丢了架子轮着拳斜行抅步望妖魔使个挂面。

那怪展长枪劈手相迎。行者左跳右跳哄那妖魔。妖魔不是是计赶离洞口南来。行者即招呼罗汉把金丹砂望妖魔一齐抛下共显神通好砂!正是那:似雾如烟初散漫纷纷霭霭下天涯。白茫茫到处迷人眼;昏漠漠飞时找路差。打柴的樵子失了伴采药的仙童不见家。细细轻飘如麦面粗粗翻复似芝麻。

世界朦胧山顶暗长空迷没太阳遮。不比嚣尘随骏马难言轻软衬香车。此砂本是无情物盖地遮天把怪拿。只为妖魔侵正道阿罗奉法逞豪华。手中就有明珠现等时刮得眼生花。那妖魔见飞砂迷目把头低了一低足下就有三尺余深慌得他将身一纵跳在浮上一层未曾立得稳须臾又有二尺余深。

那怪急了拔出脚来即忙取圈子往上一撇叫声“着!”唿喇的一下把十八粒金丹砂又尽套去拽回步径归本洞。

那罗汉一个个空手停云。行者近前问道:“众罗汉怎么不下砂了?”罗汉道:“适才响了一声金丹砂就不见矣。”行者笑道:“又是那话儿套将去了。”天王等众道:“这般难伏啊却怎么捉得他何日归天何颜见帝也!”旁有降龙、伏虎二罗汉对行者道:“悟空你晓得我两个出门迟滞何也?”行者道:“老孙只怪你躲避不来却不知有甚话说。”罗汉道:“如来吩咐我两个说那妖魔神通广大如失了金丹砂就教孙悟空上离恨天兜率宫太上老君处寻他的踪迹庶几可一鼓而擒也。”行者闻言道:“可恨!可恨!如来却也闪赚老孙!当时就该对我说了却不免教汝等远涉!”李天王道:“既是如来有此明示大圣就当早起。”

好行者说声去就纵一道筋斗云直入南天门里。时有四大元帅擎拳拱手道:“擒怪事如何?”行者且行且答道:“未哩!

未哩!如今有处寻根去也。”四将不敢留阻让他进了天门不上灵屑殿不入斗牛宫径至三十三天之外离恨天兜率宫前见两仙童侍立他也不通姓名一直径走慌得两童扯住道:

“你是何人?待往何处去?”行者才说:“我是齐天大圣欲寻李老君哩。”仙童道:“你怎这样粗鲁?且住下让我们通报。”行者那容分说喝了一声往里径走忽见老君自内而出撞个满怀。行者躬身唱个喏道:“老官一向少看。”老君笑道:“这猴儿不去取经却来我处何干?”行者道:“取经取经昼夜无停;有些阻碍到此行行。”老君道:“西天路阻与我何干?”行者道:

“西天西天你且休言;寻着踪迹与你缠缠。”老君道:“我这里乃是无上仙宫有甚踪迹可寻?”行者入里眼不转睛东张西看走过几层廊宇忽见那牛栏边一个童儿盹睡青牛不在栏中。行者道:“老官走了牛也!走了牛也!”老君大惊道:“这孽畜几时走了?”正嚷间那童儿方醒跪于当面道:“爷爷弟子睡着不知是几时走的。”老君骂道:“你这厮如何盹睡?”童儿叩头道:“弟子在丹房里拾得一粒丹当时吃了就在此睡着。”

老君道:“想是前日炼的七返火丹吊了一粒被这厮拾吃了。

那丹吃一粒该睡七日哩那孽畜因你睡着无人看管遂乘机走下界去今亦是七日矣。”即查可曾偷甚宝贝。行者道:“无甚宝贝只见他有一个圈子甚是利害。”老君急查看时诸般俱在止不见了金刚琢。老君道:“是这孽畜偷了我金刚琢去了!”

行者道:“原来是这件宝贝!当时打着老孙的是他!如今在下界张狂不知套了我等多少物件!”老君道:“这孽畜在甚地方?”行者道:“现住金皘山金皘洞。他捉了我唐僧进去抢了我金箍棒。请天兵相助又抢了太子的神兵。及请火德星君又抢了他的火具。惟水伯虽不能渰死他倒还不曾抢他物件。至请如来着罗汉下砂又将金丹砂抢去。似你这老官纵放怪物抢夺伤人该当何罪?”老君道:“我那金刚琢乃是我过函关化胡之器自幼炼成之宝。凭你甚么兵器水火俱莫能近他。若偷去我的芭蕉扇儿连我也不能奈他何矣。”

大圣才欢欢喜喜随着老君。老君执了芭蕉扇驾着祥云同行出了仙宫南天门外低下云头径至金皘山界见了十八尊罗汉、雷公、水伯、火德、李天王父子备言前事一遍。老君道:“孙悟空还去诱他出来我好收他。”这行者跳下峰头又高声骂道:“北泼孽畜!趁早出来受死!”那小妖又去报知老魔道:“这贼猴又不知请谁来也。”急绰枪举宝迎出门来。行者骂道:“你这泼魔今番坐定是死了!不要走!吃吾一掌!”急纵身跳个满怀劈脸打了一个耳括子回头就跑。那魔轮枪就赶只听得高峰上叫道:“那牛儿还不归家可待何日?”那魔抬头看见是太上老君就唬得心惊胆战道:“这贼猴真个是个地里鬼!

却怎么就访得我的主公来也?”老君念个咒语将扇子搧了一下那怪将圈子丢来被老君一把接住;又一搧那怪物力软筋麻现了本相原来是一只青牛。老君将金钢琢吹口仙气穿了那怪的鼻子解下勒袍带系于琢上牵在手中。至今留下个拴牛鼻的拘儿又名宾郎职此之谓。老君辞了众神跨上青牛背上驾彩云径归兜率院;缚妖怪高升离恨天。孙大圣才同天王等众打入洞里把那百十个小妖尽皆打死各取兵器谢了天王父子回天雷公入府火德归宫水伯回河罗汉向西;然后才解放唐僧八戒沙僧拿了铁棒。他三人又谢了行者收拾马匹行装师徒们离洞找大路方走。正走间只听得路旁叫:

“唐圣僧吃了斋饭去。”那长老心惊。不知是甚么人叫唤且听下回分解。
------------


------------

第五十四回 法性西来逢女国 心猿定计脱烟花

话说三藏师徒别了村舍人家依路西进不上三四十里早到西梁国界。唐僧在马上指道:“悟空前面城池相近市井上人语喧哗想是西梁女国。汝等须要仔细谨慎规矩切休放荡情怀紊乱法门教旨。”三人闻言谨遵严命。言未尽却至东关厢街口。那里人都是长裙短袄粉面油头不分老少尽是妇女正在两街上做买做卖忽见他四众来时一齐都鼓掌呵呵整容欢笑道:“人种来了!人种来了!”慌得那三藏勒马难行须臾间就塞满街道惟闻笑语。八戒口里乱嚷道:“我是个销猪!

我是个销猪!”行者道:“呆子莫胡谈拿出旧嘴脸便是。”八戒真个把头摇上两摇竖起一双蒲扇耳扭动莲蓬吊搭唇一声喊把那些妇女们唬得跌跌爬爬。有诗为证诗曰:圣僧拜佛到西梁国内衠阴世少阳。农士工商皆女辈渔樵耕牧尽红妆。

娇娥满路呼人种幼妇盈街接粉郎。不是悟能施丑相烟花围困苦难当!遂此众皆恐惧不敢上前一个个都捻手矬腰摇头咬指战战兢兢排塞街旁路下都看唐僧。孙大圣却也弄出丑相开路。沙僧也装吓虎维持八戒采着马掬着嘴摆着耳朵。

一行前进又见那市井上房屋齐整铺面轩昂一般有卖盐卖米、酒肆茶房鼓角楼台通货殖旗亭候馆挂帘栊。师徒们转湾抹角忽见有一女官侍立街下高声叫道:“远来的使客不可擅入城门请投馆驿注名上簿待下官执名奏驾验引放行。”

三藏闻言下马观看那衙门上有一匾上书迎阳驿三字。长老道:“悟空那村舍人家传言是实果有迎阳之驿。”沙僧笑道:

“二哥你却去照胎泉边照照看可有双影。”八戒道:“莫弄我!

我自吃了那盏儿落胎泉水已此打下胎来了还照他怎的?”三藏回头吩咐道:“悟能谨言!谨言!”遂上前与那女官作礼。女官引路请他们都进驿内正厅坐下即唤看茶。又见那手下人尽是三绺梳头、两截穿衣之类你看他拿茶的也笑。少顷茶罢女官欠身问曰:“使客何来?”行者道:“我等乃东土大唐王驾下钦差上西天拜佛求经者。我师父便是唐王御弟号曰唐三藏我乃他大徒弟孙悟空这两个是我师弟猪悟能沙悟净一行连马五口。随身有通关文牒乞为照验放行。”那女官执笔写罢下来叩头道:“老爷恕罪下官乃迎阳驿驿丞实不知上邦老爷知当远接。”拜毕起身即令管事的安排饮馔道:“爷爷们宽坐一时待下官进城启奏我王倒换关文打领给送老爷们西进。”三藏欣然而坐不题。

且说那驿丞整了衣冠径入城中五凤楼前对黄门官道:

“我是迎阳馆驿丞有事见驾。”黄门即时启奏降旨传宣至殿问曰:“驿丞有何事来奏?”驿丞道:“微臣在驿接得东土大唐王御弟唐三藏有三个徒弟名唤孙悟空、猪悟能、沙悟净连马五口欲上西天拜佛取经。特来启奏主公可许他倒换关文放行?“女王闻奏满心欢喜对众文武道:“寡人夜来梦见金屏生彩艳玉镜展光明乃是今日之喜兆也。”众女官拥拜丹墀道:“主公怎见得是今日之喜兆?”女王道:“东土男人乃唐朝御弟。我国中自混沌开辟之时累代帝王更不曾见个男人至此。幸今唐王御弟下降想是天赐来的。寡人以一国之富愿招御弟为王我愿为后与他阴阳配合生子生孙永传帝业却不是今日之喜兆也?”众女官拜舞称扬无不欢悦。驿丞又奏道:“主公之论乃万代传家之好。但只是御弟三徒凶恶不成相貌。”女王道:“卿见御弟怎生模样?他徒弟怎生凶丑?”驿丞道:“御弟相貌堂堂丰姿英俊诚是天朝上国之男儿南赡中华之人物。那三徒却是形容狞恶相貌如精。(wwW.mianhuatang.la 无弹窗广告)”女王道:“既如此把他徒弟与他领给倒换关文打他往西天只留下御弟有何不可?”众官拜奏道:“主公之言极当臣等钦此钦遵。

但只是匹配之事无媒不可自古道姻缘配合凭红叶月老夫妻系赤绳。”女王道:“依卿所奏就着当驾太师作媒迎阳驿丞主婚先去驿中与御弟求亲。待他许可寡人却摆驾出城迎接。”那太师驿丞领旨出朝。

却说三藏师徒们在驿厅上正享斋饭只见外面人报:“当驾太师与我们本官老姆来了。”三藏道:“太师来却是何意?”八戒道:“怕是女王请我们也。”行者道:“不是相请就是说亲。”

三藏道:“悟空假如不放强逼成亲却怎么是好?”行者道:

“师父只管允他老孙自有处治。”

说不了二女官早至对长老下拜。长老一一还礼道:“贫僧出家人有何德能敢劳大人下拜?”那太师见长老相貌轩昂心中暗喜道:“我国中实有造化这个男子却也做得我王之夫。”二官拜毕起来侍立左右道:“御弟爷爷万千之喜了!”

三藏道:“我出家人喜从何来?”太师躬身道:“此处乃西梁女国国中自来没个男子。今幸御弟爷爷降临臣奉我王旨意特来求亲。”三藏道:“善哉!善哉!我贫僧只身来到贵地又无儿女相随止有顽徒三个不知大人求的是那个亲事?”驿丞道:

“下官才进朝启奏我王十分欢喜道夜来得一吉梦梦见金屏生彩艳玉镜展光明知御弟乃中华上国男儿我王愿以一国之富招赘御弟爷爷为夫坐南面称孤我王愿为帝后。传旨着太师作媒下官主婚故此特来求这亲事也。”三藏闻言低头不语。太师道:“大丈夫遇时不可错过似此招赘之事天下虽有;托国之富世上实稀。请御弟允庶好回奏。”长老越加痴哑。八戒在旁掬着碓挺嘴叫道:“太师你去上复国王:我师父乃久修得道的罗汉决不爱你托国之富也不爱你倾国之容快些儿倒换关文打他往西去留我在此招赘如何?”太师闻说胆战心惊不敢回话。驿丞道:“你虽是个男身但只形容丑陋不中我王之意。”八戒笑道:“你甚不通变常言道粗柳簸箕细柳斗世上谁见男儿丑。”行者道:“呆子勿得胡谈任师父尊意可行则行可止则止莫要担阁了媒妁工夫。”三藏道:“悟空凭你怎么说好!”行者道:“依老孙说你在这里也好自古道千里姻缘似线牵哩那里再有这般相应处?”三藏道:“徒弟我们在这里贪图富贵谁却去西天取经?那不望坏了我大唐之帝主也?”太师道:“御弟在上微臣不敢隐言。我王旨意原只教求御弟为亲教你三位徒弟赴了会亲筵宴付领给倒换关文往西天取经去哩。”行者道:“太师说得有理我等不必作难情愿留下师父与你主为夫快换关文打我们西去待取经回来好到此拜爷娘讨盘缠回大唐也。”那太师与驿丞对行者作礼道:“多谢老师玉成之恩!”八戒道:“太师切莫要口里摆菜碟儿既然我们许诺且教你主先安排一席与我们吃锺肯酒如何?”太师道:“有有有就教摆设筵宴来也。”那驿丞与太师欢天喜地回奏女主不题。

却说唐长老一把扯住行者骂道:“你这猴头弄杀我也!

怎么说出这般话来教我在此招婚你们西天拜佛我就死也不敢如此。”行者道:“师父放心老孙岂不知你性情但只是到此地遇此人不得不将计就计!”三藏道:“怎么叫做将计就计?”行者道:“你若使住法儿不允他他便不肯倒换关文不放我们走路。倘或意恶心毒喝令多人割了你肉做甚么香袋啊我等岂有善报?一定要使出降魔荡怪的神通。你知我们的手脚又重器械又凶但动动手儿这一国的人尽打杀了。他虽然阻当我等却不是怪物妖精还是一国人身;你又平素是个好善慈悲的人在路上一灵不损若打杀无限的平人你心何忍!

诚为不善了也。”三藏听说道:“悟空此论最善。但恐女主招我进去要行夫妇之礼我怎肯丧元阳败坏了佛家德行;走真精坠落了本教人身?”行者道:“今日允了亲事他一定以皇帝礼摆驾出城接你。你更不要推辞就坐他凤辇龙车登宝殿面南坐下问女王取出御宝印信来宣我们兄弟进朝把通关文牒用了印再请女王写个手字花押佥押了交付与我们。一壁厢教摆筵宴就当与女王会喜就与我们送行。待筵宴已毕再叫排驾只说送我们三人出城回来与女王配合。哄得他君臣欢悦更无阻挡之心亦不起毒恶之念却待送出城外你下了龙车凤辇教沙僧伺候左右伏侍你骑上白马老孙却使个定身法儿教他君臣人等皆不能动我们顺大路只管西行。行得一昼夜我却念个咒解了术法还教他君臣们苏醒回城。一则不伤了他的性命二来不损了你的元神。这叫做假亲脱网之计岂非一举两全之美也?”三藏闻言如醉方醒似梦初觉乐以忘忧称谢不尽道:“深感贤徒高见。”四众同心合意正自商量不题。

却说那太师与驿丞不等宣诏直入朝门白玉阶前奏道:

“主公佳梦最准鱼水之欢就矣。”女王闻奏卷珠帘下龙床启樱唇露银齿笑吟吟娇声问曰:“贤卿见御弟怎么说来?”

太师道:“臣等到驿拜见御弟毕即备言求亲之事。御弟还有推托之辞幸亏他大徒弟慨然见允愿留他师父与我王为夫面南称帝只教先倒换关文打他三人西去;取得经回好到此拜认爷娘讨盘费回大唐也。”女王笑道:“御弟再有何说。”

太师奏道:“御弟不言愿配我主只是他那二徒弟先要吃席肯酒?”女王闻言即传旨教光禄寺排宴一壁厢排大驾出城迎接夫君。众女官即钦遵王命打扫宫殿铺设庭台。一班儿摆宴的火安排;一班儿摆驾的流星整备。你看那西梁国虽是妇女之邦那銮舆不亚中华之盛但见:六龙喷彩双凤生祥。六龙喷彩扶车出双凤生祥驾辇来。馥蘛异香蔼氤氲瑞气开。金鱼玉佩多官拥宝髻云鬟众女排。鸳鸯掌扇遮銮驾翡翠珠帘影凤钗。笙歌音美弦管声谐。一片欢情冲碧汉无边喜气出灵台。三檐罗盖摇天宇五色旌旗映御阶。此地自来无合卺女王今日配男才。

不多时大驾出城早到迎阳馆驿。忽有人报三藏师徒道:

“驾到了。”三藏闻言即与三徒整衣出厅迎驾。女王卷帘下辇道:“那一位是唐朝御弟?”太师指道:“那驿门外香案前穿襕衣者便是。”女王闪凤目簇蛾眉仔细观看果然一表非凡你看他:丰姿英伟相貌轩昂。齿白如银砌唇红口四方。顶平额阔天仓满目秀眉清地阁长。两耳有轮真杰士一身不俗是才郎。

好个妙龄聪俊风流子堪配西梁窈窕娘。女王看到那心欢意美之外不觉淫情汲汲爱欲恣恣展放樱桃小口呼道:“大唐御弟还不来占凤乘鸾也?”三藏闻言耳红面赤羞答答不敢抬头。猪八戒在旁掬着嘴饧眼观看那女王却也袅娜真个眉如翠羽肌似羊脂。脸衬桃花瓣鬟堆金凤丝。秋波湛湛妖娆态春笋纤纤妖媚姿。斜軃红绡飘彩艳高簪珠翠显光辉。说甚么昭君美貌果然是赛过西施。柳腰微展鸣金珮莲步轻移动玉肢。月里嫦娥难到此九天仙子怎如斯。宫妆巧样非凡类诚然王母降瑶池。那呆子看到好处忍不住口嘴流涎心头撞鹿一时间骨软筋麻好便似雪狮子向火不觉的都化去也。

只见那女王走近前来一把扯住三藏俏语娇声叫道:

“御弟哥哥请上龙车和我同上金銮宝殿匹配夫妇去来。”这长老战兢兢立站不住似醉如痴。行者在侧教道:“师父不必太谦请共师娘上辇快快倒换关文等我们取经去罢。”长老不敢回言把行者抹了两抹止不住落下泪来行者道:“师父切莫烦恼这般富贵不受用还待怎么哩?”三藏没及奈何只得依从揩了眼泪强整欢容移步近前与女主:同携素手共坐龙车。那女主喜孜孜欲配夫妻这长老忧惶惶只思拜佛。一个要洞房花烛交鸳侣一个要西宇灵山见世尊。女帝真情圣僧假意。女帝真情指望和谐同到老;圣僧假意牢藏情意养元神。一个喜见男身恨不得白昼并头谐伉俪;一个怕逢女色只思量即时脱网上雷音。二人和会同登辇岂料唐僧各有心!

那些文武官见主公与长老同登凤辇并肩而坐一个个眉花眼笑拨转仪从复入城中。孙大圣才教沙僧挑着行李牵着白马随大驾后边同行。猪八戒往前乱跑先到五凤楼前嚷道:“好自在!好现成呀!这个弄不成!这个弄不成!吃了喜酒进亲才是!”唬得些执仪从引导的女官一个个回至驾边道:

“主公那一个长嘴大耳的在五凤楼前嚷道要喜酒吃哩。”女主闻奏与长老倚香肩偎并桃腮开檀口俏声叫道:“御弟哥哥长嘴大耳的是你那个高徒?”三藏道:“是我第二个徒弟他生得食肠宽大一生要图口肥。须是先安排些酒食与他吃了方可行事。”女主急问:“光禄寺安排筵宴完否?”女官奏道:“已完设了荤素两样在东阁上哩。”女王又问:“怎么两样?”女官奏道:“臣恐唐朝御弟与高徒等平素吃斋故有荤素两样。”女王却又笑吟吟偎着长老的香腮道:“御弟哥哥你吃荤吃素?”

三藏道:“贫僧吃素但是未曾戒酒须得几杯素酒与我二徒弟吃些。”说未了太师启奏:“请赴东阁会宴今宵吉日良辰就可与御弟爷爷成亲明日天开黄道请御弟爷爷登宝殿面南改年号即位。”女王大喜即与长老携手相搀下了龙车共入端门里但见那:风飘仙乐下楼台阊阖中间翠辇来。凤阙大开光蔼蔼皇宫不闭锦排排。麒麟殿内炉烟袅孔雀屏边房影回。亭阁峥嵘如上国玉堂金马更奇哉!

既至东阁之下又闻得一派笙歌声韵美又见两行红粉貌娇娆。正中堂排设两般盛宴:左边上是素筵右边上是荤筵下两路尽是单席。那女王敛袍袖十指尖尖奉着玉杯便来安席。行者近前道:“我师徒都是吃素。先请师父坐了左手素席转下三席分左右我兄弟们好坐。”太师喜道:“正是正是。师徒即父子也不可并肩。”众女官连忙调了席面。女王一一传杯安了他弟兄三位。行者又与唐僧丢个眼色教师父回礼。三藏下来却也擎玉杯与女王安席。那些文武官朝上拜谢了皇恩各依品从分坐两边才住了音乐请酒。那八戒那管好歹放开肚子只情吃起。也不管甚么玉屑米饭、蒸饼、糖糕、蘑菇、香蕈、笋芽木耳、黄花菜、石花菜、紫菜、蔓菁、芋头、萝菔、山药、黄精、一骨辣噇了个罄尽喝了五七杯酒。口里嚷道:

“看添换来!拿大觥来!再吃几觥各人干事去。”沙僧问道:

“好筵席不吃还要干甚事?”呆子笑道:“古人云造弓的造弓造箭的造箭。我们如今招的招嫁的嫁取经的还去取经走路的还去走路莫只管贪杯误事快早儿打关文正是将军不下马各自奔前程。”女王闻说即命取大杯来。近侍官连忙取几个鹦鹉杯、鸬鹚杓、金叵罗、银凿落、玻璃盏、水晶盆、蓬莱碗、琥珀锺满斟玉液连注琼浆果然都各饮一巡。

三藏欠身而起对女王合掌道:“陛下多蒙盛设酒已彀了。请登宝殿倒换关文赶天早送他三人出城罢。”女王依言携着长老散了筵宴上金銮宝殿即让长老即位。三藏道:

“不可!不可!适太师言过明日天开黄道贫僧才敢即位称孤。

今日即印关文打他去也。”女王依言仍坐了龙床即取金交椅一张放在龙床左手请唐僧坐了叫徒弟们拿上通关文牒来。大圣便教沙僧解开包袱取出关文。大圣将关文双手捧上。那女王细看一番上有大唐皇帝宝印九颗下有宝象国印乌鸡国印车迟国印。女王看罢娇滴滴笑语道:“御弟哥哥又姓陈?”三藏道:“俗家姓陈法名玄奘。因我唐王圣恩认为御弟赐姓我为唐也。”女王道:“关文上如何没有高徒之名?”三藏道:“三个顽徒不是我唐朝人物。”女王道:“既不是你唐朝人物为何肯随你来?”三藏道:“大的个徒弟祖贯东胜神洲傲来国人氏;第二个乃西牛贺洲乌斯庄人氏;第三个乃流沙河人氏。他三人都因罪犯天条南海观世音菩萨解脱他苦秉善皈依将功折罪情愿保护我上西天取经。皆是途中收得故此未注法名在牒。”女王道:“我与你添注法名好么?”三藏道:“但凭陛下尊意。”女王即令取笔砚来浓磨香翰饱润香毫牒文之后写上孙悟空、猪悟能、沙悟净三人名讳却才取出御印端端正正印了又画个手字花押传将下去。孙大圣接了教沙僧包裹停当。那女王又赐出碎金碎银一盘下龙床递与行者道:“你三人将此权为路费早上西天。待汝等取经回来寡人还有重谢。”行者道:“我们出家人不受金银途中自有乞化之处。”女王见他不受又取出绫锦十匹对行者道:“汝等行色匆匆裁制不及将此路上做件衣服遮寒”行者道:“出家人穿不得绫锦自有护体布衣。”女王见他不受教:“取御米三升在路权为一饭。”八戒听说个饭字便就接了捎在包袱之间。行者道:“兄弟行李见今沉重且倒有气力挑米?”八戒笑道:“你那里知道米好的是个日消货只消一顿饭就了帐也。”遂此合掌谢恩。

三藏道:“敢烦陛下相同贫僧送他三人出城待我嘱付他们几句教他好生西去我却回来与陛下永受荣华无挂无牵方可会鸾交凤友也。”女王不知是计便传旨摆驾与三藏并倚香肩同登凤辇出西城而去。满城中都盏添净水炉降真香一则看女王銮驾二来看御弟男身。没老没小尽是粉容娇面、绿鬓云鬟之辈。不多时大驾出城到西关之处行者、八戒、沙僧、同心合意结束整齐径迎着銮舆厉声高叫道:“那女王不必远送我等就此拜别。”长老慢下龙车对女王拱手道:“陛下请回让贫僧取经去也。”女王闻言大惊失色扯住唐僧道:“御弟哥哥我愿将一国之富招你为夫明日高登宝位即位称君我愿为君之后喜筵通皆吃了如何却又变卦?”

八戒听说起个风来把嘴乱扭耳朵乱摇闯至驾前嚷道:

“我们和尚家和你这粉骷髅做甚夫妻!放我师父走路!”那女王见他那等撒泼弄丑唬得魂飞魄散跌入辇驾之中。沙僧却把三藏抢出人丛伏侍上马。只见那路旁闪出一个女子喝道:

“唐御弟那里走!我和你耍风月儿去来!”沙僧骂道:“贼辈无知!”掣宝杖劈头就打。那女子弄阵旋风呜的一声把唐僧摄将去了无影无踪不知下落何处。咦!正是:脱得烟花网又遇风月魔。毕竟不知那女子是人是怪老师父的性命得死得生且听下回分解。
------------

第五十五回 色邪淫戏唐三藏 性正修持不坏身

却说孙大圣与猪八戒正要使法定那些妇女忽闻得风响处沙僧嚷闹急回头时不见了唐僧。行者道:“是甚人来抢师父去了?”沙僧道:“是一个女子弄阵旋风把师父摄了去也。”

行者闻言唿哨跳在云端里用手搭凉篷四下里观看只见一阵灰尘风滚滚往西北上去了急回头叫道:“兄弟们快驾云同我赶师父去来!”八戒与沙僧即把行囊捎在马上响一声都跳在半空里去。慌得那西梁国君臣女辈跪在尘埃都道:

“是白日飞升的罗汉我主不必惊疑。唐御弟也是个有道的禅僧我们都有眼无珠错认了中华男子枉费了这场神思。请主公上辇回朝也。”女王自觉惭愧多官都一齐回国不题。

却说孙大圣兄弟三人腾空踏雾望着那阵旋风一直赶来前至一座高山只见灰尘息静风头散了更不知怪向何方。兄弟们按落云雾找路寻访忽见一壁厢青石光明却似个屏风模样。三人牵着马转过石屏石屏后有两扇石门门上有六个大字乃是“毒敌山琵琶洞”。八戒无知上前就使钉钯筑门行者急止住道:“兄弟莫忙我们随旋风赶便赶到这里寻了这会方遇此门又不知深浅如何。倘不是这个门儿却不惹他见怪?你两个且牵了马还转石屏前立等片时待老孙进去打听打听察个有无虚实却好行事。”沙僧听说大喜道:

“好!好!好!正是粗中有细果然急处从宽。”他二人牵马回头。

孙大圣显个神通捻着诀念个咒语摇身一变变作蜜蜂儿真个轻巧!你看他:翅薄随风软腰轻映日纤。嘴甜曾觅蕊尾利善降蟾。酿蜜功何浅投衙礼自谦。如今施巧计飞舞入门檐。行者自门瑕处钻将进去飞过二层门里只见正当中花亭子上端坐着一个女怪左右列几个彩衣绣服、丫髻两揫的女童都欢天喜地正不知讲论甚么。这行者轻轻的飞上去钉在那花亭格子上侧耳才听又见两个总角蓬头女子捧两盘热腾腾的面食上亭来道:“奶奶一盘是人肉馅的荤馍馍一盘是邓沙馅的素馍馍。”那女怪笑道:“小的们搀出唐御弟来。”

几个彩衣绣服的女童走向后房把唐僧扶出。那师父面黄唇白眼红泪滴行者在暗中嗟叹道:“师父中毒了!”

那怪走下亭露春葱十指纤纤扯住长老道:“御弟宽心我这里虽不是西梁女国的宫殿不比富贵奢华其实却也清闲自在正好念佛看经。我与你做个道伴儿真个是百岁和谐也。”三藏不语那怪道:“且休烦恼。我知你在女国中赴宴之时不曾进得饮食。这里荤素面饭两盘凭你受用些儿压惊。”

三藏沉思默想道:“我待不说话不吃东西此怪比那女王不同女王还是人身行动以礼;此怪乃是妖神恐为加害奈何?

我三个徒弟不知我困陷在于这里倘或加害却不枉丢性命?”以心问心无计所奈只得强打精神开口道:“荤的何如?

素的何如?”女怪道:“荤的是人肉馅馍馍素的是邓沙馅馍馍。”三藏道:“贫僧吃素。”那怪笑道:“女童看热茶来与你家长爷爷吃素馍馍。”一女童果捧着香茶一盏放在长老面前。

那怪将一个素馍馍劈破递与三藏。三藏将个荤馍馍囫囵递与女怪。女怪笑道:“御弟你怎么不劈破与我?”三藏合掌道:“我出家人不敢破荤。”那女怪道:“你出家人不敢破荤怎么前日在子母河边吃水高今日又好吃邓沙馅?”三藏道:“水高船去急沙陷马行迟。”行者在格子眼听着两个言语相攀恐怕师父乱了真性忍不住现了本相掣铁棒喝道:“孽畜无礼!”那女怪见了口喷一道烟光把花亭子罩住教:“小的们收了御弟!”他却拿一柄三股钢叉跳出亭门骂道:“泼猴惫懒!怎么敢私入吾家偷窥我容貌!不要走!吃老娘一叉!”这大圣使铁棒架住且战且退。

二人打出洞外那八戒、沙僧正在石屏前等候忽见他两人争持慌得八戒将白马牵过道:“沙僧你只管看守行李马匹等老猪去帮打帮打。”好呆子双手举钯赶上前叫道:“师兄靠后让我打这泼贱!”那怪见八戒来他又使个手段呼了一声鼻中出火口内生烟把身子抖了一抖三股叉飞舞冲迎。那女怪也不知有几只手没头没脸的滚将来。这行者与八戒两边攻住。那怪道:“孙悟空你好不识进退!我便认得你你是不认得我。你那雷音寺里佛如来也还怕我哩量你这两个毛人到得那里!都上来一个个仔细看打!”这一场怎见得好战:女怪威风长猴王气概兴。天蓬元帅争功绩乱举钉钯要显能。那一个手多叉紧烟光绕这两个性急兵强雾气腾。女怪只因求配偶男僧怎肯泄元精!阴阳不对相持斗各逞雄才恨苦争。阴静养荣思动动阳收息卫爱清清。致令两处无和睦叉钯铁棒赌输赢。这个棒有力钯更能女怪钢叉丁对丁。毒敌山前三不让琵琶洞外两无情。那一个喜得唐僧谐凤侣这两个必随长老取真经。惊天动地来相战只杀得日月无光星斗更!三个斗罢多时不分胜负。那女怪将身一纵使出个倒马毒桩不觉的把大圣头皮上扎了一下。行者叫声“苦啊!”忍耐不得负痛败阵而走。八戒见事不谐拖着钯彻身而退。那怪得了胜收了钢叉。

行者抱头皱眉苦面叫声“利害!利害!”八戒到跟前问道:“哥哥你怎么正战到好处却就叫苦连天的走了?”行者抱着头只叫:“疼!疼!疼!”沙僧道:“想是你头风了?”行者跳道:“不是!不是!”八戒道:“哥哥我不曾见你受伤却头疼何也?”行者哼哼的道:“了不得!了不得!我与他正然打处他见我破了他的叉势他就把身子一纵不知是件甚么兵器着我头上扎了一下就这般头疼难禁故此败了阵来。”八戒笑道:

“只这等静处常夸口说你的头是修炼过的。却怎么就不禁这一下儿?”行者道:“正是我这头自从修炼成真盗食了蟠桃仙酒老子金丹大闹天宫时又被玉帝差大力鬼王、二十八宿押赴斗牛宫处处斩那些神将使刀斧锤剑雷打火烧及老子把我安于八卦炉锻炼四十九日俱未伤损。今日不知这妇人用的是甚么兵器把老孙头弄伤也!”沙僧道:“你放了手等我看看。莫破了!”行者道:“不破!不破!”八戒道:“我去西梁国讨个膏药你贴贴。”行者道:“又不肿不破怎么贴得膏药?”八戒笑道:“哥啊我的胎前产后病倒不曾有你倒弄了个脑门痈了。”沙僧道:“二哥且休取笑。如今天色晚矣大哥伤了头师父又不知死活怎的是好!”行者哼道:“师父没事。我进去时变作蜜蜂儿飞入里面见那妇人坐在花亭子上。少顷两个丫鬟捧两盘馍馍:一盘是人肉馅荤的;一盘是邓沙馅素的。又着两个女童扶师父出来吃一个压惊又要与师父做甚么道伴儿。师父始初不与那妇人答话也不吃馍馍后见他甜言美语不知怎么就开口说话却说吃素的。那妇人就将一个素的劈开递与师父师父将个囫囵荤的递与那妇人。妇人道:‘怎不劈破?’师父道:‘出家人不敢破荤。’那妇人道:‘既不破荤前日怎么在子母河边饮水高今日又好吃邓沙馅?’师父不解其意答他两句道:‘水高船去急沙陷马行迟。’我在格子上听见恐怕师父乱性便就现了原身掣棒就打。他也使神通喷出烟雾叫收了御弟就轮钢叉与老孙打出洞来也。”沙僧听说咬指道:“这泼贱也不知从那里就随将我们来把上项事都知道了!”八戒道:“这等说便我们安歇不成?莫管甚么黄昏半夜且去他门上索战嚷嚷闹闹搅他个不睡莫教他捉弄了我师父。”行者道:“头疼去不得!”沙僧道:“不须索战。一则师兄头痛二来我师父是个真僧决不以色空乱性且就在山坡下闭风处坐这一夜养养精神待天明再作理会。”遂此三个弟兄拴牢白马守护行囊就在坡下安歇不题。

却说那女怪放下凶恶之心重整欢愉之色叫:“小的们把前后门都关紧了。”又使两个支更防守行者但听门响即时通报。却又教:“女童将卧房收拾齐整掌烛焚香请唐御弟来我与他交欢。”遂把长老从后边搀出。那女怪弄出十分娇媚之态携定唐僧道:“常言黄金未为贵安乐值钱多。且和你做会夫妻儿耍子去也。”这长老咬定牙关声也不透。欲待不去恐他生心害命只得战兢兢跟着他步入香房却如痴如哑那里抬头举目更不曾看他房里是甚床铺幔帐也不知有甚箱笼梳妆那女怪说出的雨意云情亦漠然无听。好和尚真是那:

目不视恶色耳不听淫声。他把这锦绣娇容如粪土金珠美貌若灰尘。一生只爱参禅半步不离佛地。那里会惜玉怜香只晓得修真养性。那女怪活泼泼春意无边;这长老死丁丁禅机有在。一个似软玉温香一个如死灰槁木。那一个展鸳衾淫兴浓浓;这一个束褊衫丹心耿耿。那个要贴胸交股和鸾凤这个要画壁归山访达摩。女怪解衣卖弄他肌香肤腻;唐僧敛衽紧藏了糙肉粗皮。女怪道:“我枕剩衾闲何不睡?”唐僧道:“我头光服异怎相陪!”那个道:“我愿作前朝柳翠翠。”这个道:“贫僧不是月阇黎。”女怪道:“我美若西施还袅娜。”唐僧道:“我越王因此久埋尸。”女怪道:“御弟你记得宁教花下死做鬼也风流?”唐僧道:“我的真阳为至宝怎肯轻与你这粉骷髅。”他两个散言碎语的直斗到更深唐长老全不动念。那女怪扯扯拉拉的不放这师父只是老老成成的不肯。直缠到有半夜时候把那怪弄得恼了叫:“小的们拿绳来!”可怜将一个心爱的人儿一条绳捆的象个猱狮模样又教拖在房廊下去却吹灭银灯各归寝处。

一夜无词不觉的鸡声三唱。那山坡下孙大圣欠身道:“我这头疼了一会到如今也不疼不麻只是有些作痒。”八戒笑道:“痒便再教他扎一下何如?”行者啐了一口道:“放放放!”

八戒又笑道:“放放放!我师父这一夜倒浪浪浪!”沙僧道:“且莫斗口天亮了快赶早儿捉妖怪去。”行者道:“兄弟你只管在此守马休得动身。猪八戒跟我去。”那呆子抖擞精神束一束皂锦直裰相随行者各带了兵器跳上山崖径至石屏之下。行者道:“你且立住只怕这怪物夜里伤了师父先等我进去打听打听。倘若被他哄了丧了元阳真个亏了德行却就大家散火;若不乱性情禅心未动却好努力相持打死精怪救师西去。”八戒道:“你好痴哑!常言道干鱼可好与猫儿作枕头?就不如此就不如此也要抓你儿把是!”行者道:“莫胡疑乱说待我看去。”

好大圣转石屏别了八戒摇身还变个蜜蜂儿飞入门里见那门里有两个丫鬟头枕着梆铃正然睡哩。却到花亭子观看那妖精原来弄了半夜都辛苦了一个个都不知天晓还睡着哩。行者飞来后面隐隐的只听见唐僧声唤忽抬头见那步廊下四马攒蹄捆着师父。行者轻轻的钉在唐僧头上叫:“师父。”唐僧认得声音道:“悟空来了?快救我命!”行者道:“夜来好事如何?”三藏咬牙道:“我宁死也不肯如此!”行者道:“昨日我见他有相怜相爱之意却怎么今日把你这般挫折?”三藏道:

“他把我缠了半夜我衣不解带身未沾床。他见我不肯相从才捆我在此。你千万救我取经去也!”他师徒们正然问答早惊醒了那个妖精。妖精虽是下狠却还有流连不舍之意一觉翻身只听见“取经去也”一句他就滚下床来厉声高叫道:“好夫妻不做却取甚么经去!”行者慌了撇却师父急展翅飞将出去现了本相叫声“八戒。”那呆子转过石屏道:“那话儿成了否?”行者笑道:“不曾!不曾!老师父被他摩弄不从恼了捆在那里正与我诉说前情那怪惊醒了我慌得出来也。”八戒道:“师父曾说甚来?”行者道:“他只说衣不解带身未沾床。”八戒笑道:“好!好!好!还是个真和尚!我们救他去!”

呆子粗鲁不容分说举钉钯望他那石头门上尽力气一钯唿喇喇筑做几块。唬得那几个枕梆铃睡的丫环跑至二层门外叫声:“开门!前门被昨日那两个丑男人打破了!”那女怪正出房门只见四五个丫鬟跑进去报道:“奶奶昨日那两个丑男人又来把前门已打碎矣。”那怪闻言即忙叫:“小的们!快烧汤洗面梳妆!”叫:“把御弟连绳抬在后房收了等我打他去!”好妖精走出来举着三股叉骂道:“泼猴!野彘!老大无知!你怎敢打破我门!”八戒骂道:“滥淫贱货!你倒困陷我师父返敢硬嘴!我师父是你哄将来做老公的快快送出饶你!敢再说半个不字老猪一顿钯连山也筑倒你的!”那妖精那容分说抖擞身躯依前弄法鼻口内喷烟冒火举钢叉就刺八戒。八戒侧身躲过着钯就筑孙大圣使铁棒并力相帮。那怪又弄神通也不知是几只手左右遮拦交锋三五个回合不知是甚兵器把八戒嘴唇上也又扎了一下。那呆子拖着钯侮着嘴负痛逃生。

行者却也有些醋他虚丢一棒败阵而走。那妖精得胜而回叫小的们搬石块垒迭了前门不题。

却说那沙和尚正在坡前放马只听得那里猪哼忽抬头见八戒侮着嘴哼将来。沙僧道:“怎的说?”呆子哼道:“了不得!了不得!疼疼疼!”说不了行者也到跟前笑道:“好呆子啊!

昨日咒我是脑门痈今日却也弄做个肿嘴瘟了!”八戒哼道:

“难忍难忍!疼得紧!利害利害!”三人正然难处只见一个老妈妈儿左手提着一个青竹篮儿自南山路上挑菜而来。沙僧道:“大哥那妈妈来得近了等我问他个信儿看这个是甚妖精是甚兵器这般伤人。”行者道:“你且住等老孙问他去来。”行者急睁睛看只见头直上有祥云盖顶左右有香雾笼身。行者认得即叫:“兄弟们还不来叩头!那妈妈是菩萨来也。”慌得猪八戒忍疼下拜沙和尚牵马躬身孙大圣合掌跪下叫声“南无大慈大悲救苦救难灵感观世音菩萨。”那菩萨见他们认得元光即踏祥云起在半空现了真象原来是鱼篮之象。行者赶到空中拜告道:“菩萨恕弟子失迎之罪!我等努力救师不知菩萨下降今遇魔难难收万望菩萨搭救搭救!”

菩萨道:“这妖精十分利害他那三股叉是生成的两只钳脚。扎人痛者是尾上一个钩子唤做倒马毒。本身是个蝎子精。他前者在雷音寺听佛谈经如来见了不合用手推他一把他就转过钩子把如来左手中拇指上扎了一下如来也疼难禁即着金刚拿他他却在这里。若要救得唐僧除是别告一位方好我也是近他不得。”行者再拜道:“望菩萨指示指示别告那位去好弟子即去请他也。”菩萨道:“你去东天门里光明宫告求昴日星官方能降伏。”言罢遂化作一道金光径回南海。

孙大圣才按云头对八戒沙僧道:“兄弟放心师父有救星了。”沙僧道:“是那里救星?”行者道:“才然菩萨指示教我告请昴日星官老孙去来。”八戒侮着嘴哼道:“哥啊!就问星官讨些止疼的药饵来!”行者笑道:“不须用药只似昨日疼过夜就好了。”沙僧道:“不必烦叙快早去罢。”好行者急忙驾筋斗云须臾到东天门外。忽见增长天王当面作礼道:“大圣何往?”

行者道:“因保唐僧西方取经路遇魔障缠身要到光明宫见昴日星官走走。”忽又见陶张辛邓四大元帅也问何往行者道:

“要寻昴日星官去降妖救师。”四元帅道:“星官今早奉玉帝旨意上观星台巡札去了。”行者道:“可有这话?”辛天君道:“小将等与他同下斗牛宫岂敢说假?”陶天君道:“今已许久或将回矣。大圣还先去光明宫如未回再去观星台可也。”大圣遂喜即别他们至光明宫门果是无人复抽身就走只见那壁厢有一行兵士摆列后面星官来了。那星官还穿的是拜驾朝衣一身金缕但见他:冠簪五岳金光彩笏执山河玉色琼。袍挂七星云叆叇腰围八极宝环明。叮当珮响如敲韵迅风声似摆铃。翠羽扇开来昴宿天香飘袭满门庭。

前行的兵士看见行者立于光明宫外急转身报道:“主公孙大圣在这里也。”那星官敛云雾整束朝衣停执事分开左右上前作礼道:“大圣何来?”行者道:“专来拜烦救师父一难。”星官道:“何难?在何地方?”行者道:“在西梁国毒敌山琵琶洞。”星官道:“那山洞有甚妖怪却来呼唤小神?”行者道:

“观音菩萨适才显化说是一个蝎子精特举先生方能治得因此来请。”星官道:“本欲回奏玉帝奈大圣至此又感菩萨举荐恐迟误事小神不敢请献茶且和你去降妖精却再来回旨罢。”大圣闻言即同出东天门直至西梁国。望见毒敌山不远行者指道:“此山便是。”星官按下云头同行者至石屏前山坡之下。沙僧见了道:“二哥起来大哥请得星官来了。”那呆子还侮着嘴道:“恕罪恕罪!有病在身不能行礼。”星官道:“你是修行之人何病之有?”八戒道:“早间与那妖精交战被他着我唇上扎了一下至今还疼呀。”星官道:“你上来我与你医治医治。”呆子才放了手口里哼哼喷喷道:“千万治治!待好了谢你。”那星官用手把嘴唇上摸了一摸吹一口气就不疼了。呆子欢喜下拜道:“妙啊!妙啊!”行者笑道:“烦星官也把我头上摸摸。”星官道:“你未遭毒摸他何为?”行者道:“昨日也曾遭过只是过了夜才不疼如今还有些麻痒只恐天阴也烦治治。”星官真个也把头上摸了一摸吹口气也就解了余毒不麻不痒了。八戒狠道:“哥哥去打那泼贱去!”星官道:“正是正是你两个叫他出来等我好降他。”

行者与八戒跳上山坡又至石屏之后。呆子口里乱骂手似捞钩一顿钉钯把那洞门外垒迭的石块爬开闯至一层门又一钉钯将二门筑得粉碎。慌得那门里小妖飞报:“奶奶!那两个丑男人又把二层门也打破了!”那怪正教解放唐僧讨素茶饭与他吃哩听见打破二门即便跳出花亭子轮叉来刺八戒。八戒使钉钯迎架行者在旁又使铁棒来打。那怪赶至身边要下毒手他两个识得方法回头就走。那怪赶过石屏之后行者叫声:“昴宿何在?”只见那星官立于山坡上现出本相原来是一只双冠子大公鸡昂起头来约有六七尺高对着妖精叫一声那怪即时就现了本象是个琵琶来大小的蝎子精。星官再叫一声那怪浑身酥软死在坡前。有诗为证诗曰:

花冠绣颈若团缨爪硬距长目怒睛。踊跃雄威全五德峥嵘壮势羡三鸣。岂如凡鸟啼茅屋本是天星显圣名。毒蝎枉修人道行还原反本见真形。八戒上前一只脚躧住那怪的胸背道:

“孽畜!今番使不得倒马毒了!”那怪动也不动被呆子一顿钉钯捣作一团烂酱。那星官复聚金光驾云而去。行者与八戒沙僧朝天拱谢道:“有累有累!改日赴宫拜酬。”三人谢毕却才收拾行李马匹都进洞里见那大小丫环两边跪下拜道:“爷爷我们不是妖邪都是西梁国女人前者被这妖精摄来的。你师父在后边香房里坐着哭哩。”行者闻言仔细观看果然不见妖气遂入后边叫道:“师父!”那唐僧见众齐来十分欢喜道:

“贤徒累及你们了!那妇人何如也?”八戒道:“那厮原是个大母蝎子。幸得观音菩萨指示大哥去天宫里请得那昴日星官下降把那厮收伏。才被老猪筑做个泥了方敢深入于此得见师父之面。”唐僧谢之不尽。又寻些素米、素面安排了饮食吃了一顿把那些摄将来的女子赶下山指与回家之路。点上一把火把几间房宇烧毁罄尽请唐僧上马找寻大路西行。正是:

割断尘缘离色相推干金海悟禅心。毕竟不知几年上才得成真且听下回分解。
------------

第五十六回 神狂诛草寇 道昧放心猿

诗曰:灵台无物谓之清寂寂全无一念生。(wwW.mianhuatang.la 无弹窗广告)猿马牢收休放荡精神谨慎莫峥嵘。除六贼悟三乘万缘都罢自分明。色邪永灭真界坐享西方极乐城。话说唐三藏咬钉嚼铁以死命留得一个不坏之身感蒙行者等打死蝎子精救出琵琶洞。一路无词又早是朱明时节但见那:熏风时送野兰香濯雨才晴新竹凉。艾叶满山无客采蒲花盈涧自争芳。海榴娇艳游蜂喜溪柳阴浓黄雀狂。长路那能包角黍龙舟应吊汨罗江。他师徒们行赏端阳之景虚度中天之节忽又见一座高山阻路。长老勒马回头叫道:“悟空前面有山恐又生妖怪是必谨防。”行者等道:“师父放心我等皈命投诚怕甚妖怪!”长老闻言甚喜加鞭催骏马放辔趱蛟龙。须臾上了山崖举头观看真个是:顶巅松柏接云青石壁荆榛挂野藤。万丈崔巍千层悬削。

万丈崔巍峰岭峻千层悬削壑崖深。苍苔碧藓铺阴石古桧高槐结大林。林深处听幽禽巧声襕睆实堪吟。涧内水流如泻玉路旁花落似堆金。山势恶不堪行十步全无半步平。狐狸糜鹿成双遇白鹿玄猿作对迎。忽闻虎啸惊人胆鹤鸣振耳透天庭。黄梅红杏堪供食野草闲花不识名。

四众进山缓行良久过了山头下西坡乃是一段平阳之地。猪八戒卖弄精神教沙和尚挑着担子他双手举钯上前赶马。那马更不惧他凭那呆子嗒笞笞的赶只是缓行不紧。行者道:“兄弟你赶他怎的?让他慢慢走罢了。”八戒道:“天色将晚自上山行了这一日肚里饿了大家走动些寻个人家化些斋吃。”行者闻言道:“既如此等我教他快走。”把金箍棒幌一幌喝了一声那马溜了缰如飞似箭顺平路往前去了。你说马不怕八戒只怕行者何也?行者五百年前曾受玉帝封在大罗天御马监养马官名弼马温故此传留至今是马皆惧猴子。那长老挽不住缰口只扳紧着鞍桥让他放了一路辔头有二十里向开田地方才缓步而行。

正走处忽听得一棒锣声路两边闪出三十多人一个个枪刀棍棒拦住路口道:“和尚!那里走!”唬得个唐僧战兢兢坐不稳跌下马来蹲在路旁草科里只叫:“大王饶命!大王饶命!”那为头的两个大汉道:“不打你只是有盘缠留下。”长老方才省悟知他是伙强人却欠身抬头观看但见他:一个青脸獠牙欺太岁一个暴睛圆眼赛丧门。鬓边红如飘火颔下黄须似插针。他两个头戴虎皮花磕脑腰系貂裘彩战裙。一个手中执着狼牙棒一个肩上横担扢挞藤。果然不亚巴山虎真个犹如出水龙。三藏见他这般凶恶只得走起来合掌当胸道:

“大王贫僧是东土唐王差往西天取经者自别了长安年深日久就有些盘缠也使尽了。出家人专以乞化为由那得个财帛?

万望大王方便方便让贫僧过去罢!”那两个贼帅众向前道:

“我们在这里起一片虎心截住要路专要些财帛甚么方便方便?你果无财帛快早脱下衣服留下白马放你过去!”三藏道:“阿弥陀佛!贫僧这件衣服是东家化布西家化针零零碎碎化来的。你若剥去可不害杀我也?只是这世里做得好汉那世里变畜生哩!”那贼闻言大怒掣大棍上前就打。这长老口内不言心中暗想道:“可怜!你只说你的棍子还不知我徒弟的棍子哩!”那贼那容分说举着棒没头没脸的打来。长老一生不会说谎遇着这急难处没奈何只得打个诳语道:“二位大王且莫动手我有个小徒弟在后面就到。他身上有几两银子把与你罢。”那贼道:“这和尚是也吃不得亏且捆起来。”

众娄罗一齐下手把一条绳捆了高高吊在树上。

却说三个撞祸精随后赶来。八戒呵呵大笑道:“师父去得好快不知在那里等我们哩。”忽见长老在树上他又说:“你看师父等便罢了却又有这般心肠爬上树去扯着藤儿打秋千耍子哩!”行者见了道:“呆子莫乱谈。(WWW.mianhuatang.la 好看的小说)师父吊在那里不是?你两个慢来等我去看看。”好大圣急登高坡细看认得是伙强人心中暗喜道:“造化!造化!买卖上门了!”即转步摇身一变变做个干干净净的小和尚穿一领缁衣年纪只有二八肩上背着一个蓝布包袱拽开步来到前边叫道:“师父这是怎么说话?这都是些甚么歹人?”三藏道:“徒弟呀还不救我一救还问甚的?”行者道:“是干甚勾当的?”三藏道:“这一伙拦路的把我拦住要买路钱。因身边无物遂把我吊在这里只等你来计较计较不然把这匹马送与他罢。”行者闻言笑道:

“师父不济天下也有和尚似你这样皮松的却少。唐太宗差你往西天见佛谁教你把这龙马送人?”三藏道:“徒弟呀似这等吊起来打着要怎生是好?”行者道:“你怎么与他说来?”三藏道:“他打的我急了没奈何把你供出来也。”行者道:“师父你好没搭撒你供我怎的?”三藏道:“我说你身边有些盘缠且教道莫打我是一时救难的话儿。”行者道:“好!好!好!承你抬举正是这样供。若肯一个月供得七八十遭老孙越有买卖。”

那伙贼见行者与他师父讲话撒开势围将上来道:“小和尚你师父说你腰里有盘缠趁早拿出来饶你们性命!若道半个不字就都送了你的残生!”行者放下包袱道:“列位长官不要嚷。盘缠有些在此包袱不多只有马蹄金二十来锭粉面银二三十锭散碎的未曾见数。要时就连包儿拿去切莫打我师父。古书云德者本也财者末也此是末事。我等出家人自有化处。若遇着个斋僧的长者衬钱也有衣服也有能用几何?只望放下我师父来我就一并奉承。”那伙贼闻言都甚欢喜道:“这老和尚悭吝这小和尚倒还慷慨。”教:“放下来。”那长老得了性命跳上马顾不得行者操着鞭一直跑回旧路。

行者忙叫道:“走错路了。”提着包袱就要追去。那伙贼拦住道:“那里走?将盘缠留下免得动刑!”行者笑道:“说开盘缠须三分分之。”那贼头道:“这小和尚忒乖就要瞒着他师父留起些儿。也罢拿出来看。若多时也分些与你背地里买果子吃。”行者道:“哥呀不是这等说。我那里有甚盘缠?说你两个打劫别人的金银是必分些与我。”那贼闻言大怒骂道:“这和尚不知死活!你倒不肯与我返回我要!不要走!看打!”轮起一条扢挞藤棍照行者光头上打了七八下。行者只当不知且满面陪笑道:“哥呀若是这等打就打到来年打罢春也是不当真的。”那贼大惊道:“这和尚好硬头!”行者笑道:“不敢不敢承过奖了也将就看得过。”那贼那容分说两三个一齐乱打行者道:“列位息怒等我拿出来。”好大圣耳中摸一摸拔出一个绣花针儿道:“列位我出家人果然不曾带得盘缠只这个针儿送你罢。”那贼道:“晦气呀!把一个富贵和尚放了却拿住这个穷秃驴!你好道会做裁缝?我要针做甚的?”行者听说不要就拈在手中幌了一幌变作碗来粗细的一条棍子。那贼害怕道:“这和尚生得小倒会弄术法儿。”行者将棍子插在地下道:“列位拿得动就送你罢。”两个贼上前抢夺可怜就如蜻蜓撼石柱莫想弄动半分毫。这条棍本是如意金箍棒天秤称的一万三千五百斤重那伙贼怎么知得?大圣走上前轻轻的拿起丢一个蟒翻身拗步势指着强人道:“你都造化低遇着我老孙了!”那贼上前来又打了五六十下。行者笑道:“你也打得手困了且让老孙打一棒儿却休当真。”你看他展开棍子幌一幌有井栏粗细七八丈长短荡的一棍把一个打倒在地嘴唇揞土再不做声。那一个开言骂道:“这秃厮老大无礼!盘缠没有转伤我一个人!”行者笑道:“且消停且消停!待我一个个打来一教你断了根罢!”荡的又一棍把第二个又打死了唬得那众娄罗撇枪弃棍四路逃生而走。(wwW.mianhuatang.la 无弹窗广告)

却说唐僧骑着马往东正跑八戒、沙僧拦住道:“师父往那里去?错走路了。”长老兜马道:“徒弟啊趁早去与你师兄说教他棍下留情莫要打杀那些强盗。”八戒道:“师父住下等我去来。”呆子一路跑到前边厉声高叫道:“哥哥师父教你莫打人哩。”行者道:“兄弟那曾打人?”八戒道:“那强盗往那里去了?”行者道:“别个都散了只是两个头儿在这里睡觉哩。”八戒笑道:“你两个遭瘟的好道是熬了夜这般辛苦不往别处睡却睡在此处!”呆子行到身边看看道:“倒与我是一起的干净张着口睡淌出些粘涎来了。”行者道:“是老孙一棍子打出豆腐来了。”八戒道:“人头上又有豆腐?”行者道:“打出脑子来了!”八戒听说打出脑子来慌忙跑转去对唐僧道:“散了伙也!”三藏道:“善哉!善哉!往那条路上去了?”八戒道:

“打也打得直了脚又会往那里去走哩!”三藏道:“你怎么说散伙?”八戒道:“打杀了不是散伙是甚的?”三藏问:“打的怎么模样?”八戒道:“头上打了两个大窟窿。”三藏教:“解开包取几文衬钱快去那里讨两个膏药与他两个贴贴。”八戒笑道:

“师父好没正经膏药只好贴得活人的疮肿那里好贴得死人的窟窿?”三藏道:“真打死了?”就恼起来口里不住的絮絮叨叨猢狲长猴子短兜转马与沙僧、八戒至死人前见那血淋淋的倒卧山坡之下。

这长老甚不忍见即着八戒:“快使钉钯筑个坑子埋了我与他念卷倒头经。”八戒道:“师父左使了人也。行者打杀人还该教他去烧埋怎么教老猪做土工?”行者被师父骂恼了喝着八戒道:“泼懒夯货!趁早儿去埋!迟了些儿就是一棍!”呆子慌了往山坡下筑了有三尺深下面都是石脚石根扛住钯齿呆子丢了钯便把嘴拱拱到软处一嘴有二尺五两嘴有五尺深把两个贼尸埋了盘作一个坟堆。三藏叫:“悟空取香烛来待我祷祝好念经。”行者努着嘴道:“好不知趣!这半山之中前不巴村后不着店那讨香烛?就有钱也无处去买。”三藏恨恨的道:“猴头过去!等我撮土焚香祷告。”这是三藏离鞍悲野冢圣僧善念祝荒坟祝云:“拜惟好汉听祷原因:念我弟子东土唐人。奉太宗皇帝旨意上西方求取经文。适来此地逢尔多人不知是何府、何州、何县都在此山内结党成群。我以好话哀告殷勤。尔等不听返善生嗔。却遭行者棍下伤身。

切念尸骸暴露吾随掩土盘坟。折青竹为香烛无光彩有心勤;取顽石作施食无滋味有诚真。你到森罗殿下兴词倒树寻根他姓孙我姓陈各居异姓。冤有头债有主切莫告我取经僧人。”八戒笑道:“师父推了干净他打时却也没有我们两个。”三藏真个又撮土祷告道:“好汉告状只告行者也不干八戒、沙僧之事。”大圣闻言忍不住笑道:“师父你老人家忒没情义。为你取经我费了多少殷勤劳苦如今打死这两个毛贼你倒教他去告老孙。虽是我动手打却也只是为你。你不往西天取经我不与你做徒弟怎么会来这里会打杀人!索性等我祝他一祝。”攥着铁棒望那坟上捣了三下道:“遭瘟的强盗你听着!我被你前七八棍后七八棍打得我不疼不痒的触恼了性子一差二误将你打死了尽你到那里去告我老孙实是不怕:玉帝认得我天王随得我;二十八宿惧我九曜星官怕我;府县城隍跪我东岳天齐怖我;十代阎君曾与我为仆从五路猖神曾与我当后生;不论三界五司十方诸宰都与我情深面熟随你那里去告!”三藏见说出这般恶话却又心惊道:“徒弟呀我这祷祝是教你体好生之德为良善之人你怎么就认真起来?”行者道:“师父这不是好耍子的勾当且和你赶早寻宿去。”那长老只得怀嗔上马。

孙大圣有不睦之心八戒、沙僧亦有嫉妒之意师徒都面是背非依大路向西正走忽见路北下有一座庄院。三藏用鞭指定道:“我们到那里借宿去。”八戒道:“正是。”遂行至庄舍边下马。看时却也好个住场但见:野花盈径杂树遮扉。远岸流山水平畦种麦葵。蒹葭露润轻鸥宿杨柳风微倦鸟栖。青柏间松争翠碧红蓬映蓼斗芳菲。村犬吠晚鸡啼牛羊食饱牧童归。爨烟结雾黄粱熟正是山家入暮时。长老向前忽见那村舍门里走出一个老者即与相见道了问讯。那老者问道:

“僧家从那里来?”三藏道:“贫僧乃东土大唐钦差往西天求经者。适路过宝方天色将晚特来檀府告宿一宵。”老者笑道:

“你贵处到我这里程途迢递怎么涉水登山独自到此?”三藏道:“贫僧还有三个徒弟同来。”老者问:“高徒何在?”三藏用手指道:“那大路旁立的便是。”老者猛抬头看见他们面貌丑陋急回身往里就走被三藏扯住道:“老施主千万慈悲告借一宿!”老者战兢兢钳口难言摇着头摆着手道:“不不不不象人模样!是是是几个妖精!”三藏陪笑道:“施主切休恐惧我徒弟生得是这等相貌不是妖精!”老者道:“爷爷呀一个夜叉一个马面一个雷公!”行者闻言厉声高叫道:“雷公是我孙子夜叉是我重孙马面是我玄孙哩!”那老者听见魄散魂飞面容失色只要进去。三藏搀住他同到草堂陪笑道:“老施主不要怕他。他都是这等粗鲁不会说话。”

正劝解处只见后面走出一个婆婆携着五六岁的一个小孩儿道:“爷爷为何这般惊恐?”老者才叫:“妈妈看茶来。”

那婆婆真个丢了孩儿入里面捧出二锺茶来。茶罢三藏却转下来对婆婆作礼道:“贫僧是东土大唐差往西天取经的才到贵处拜求尊府借宿因是我三个徒弟貌丑老家长见了虚惊也。”婆婆道:“见貌丑的就这等虚惊若见了老虎豺狼却怎么好?”老者道:“妈妈呀人面丑陋还可只是言语一吓人。我说他象夜叉马面雷公他吆喝道雷公是他孙子夜叉是他重孙马面是他玄孙。我听此言故然悚惧。”唐僧道:“不是不是象雷公的是我大徒孙悟空象马面的是我二徒猪悟能象夜叉的是我三徒沙悟净。他们虽是丑陋却也秉教沙门皈依善果不是甚么恶魔毒怪怕他怎么!”公婆两个闻说他名号皈正沙门之言却才定性回惊教:“请来请来。”长老出门叫来又吩咐道:“适才这老者甚恶你等今进去相见切勿抗礼各要尊重些。”八戒道:“我俊秀我斯文不比师兄撒泼。”行者笑道:

“不是嘴长耳大、脸丑便也是一个好男子。”沙僧道:“莫争讲这里不是那抓乖弄俏之处且进去!且进去!”

遂此把行囊马匹都到草堂上齐同唱了个喏坐定。那妈妈儿贤慧即便携转小儿咐吩煮饭安排一顿素斋他师徒吃了。渐渐晚了又掌起灯来都在草堂上闲叙。长老才问:“施主高姓?”老者道:“姓杨。”又问年纪。老者道:“七十四岁。”又问:“几位令郎?”老者道:“止得一个适才妈妈携的是小孙。”

长老:“请令郎相见拜揖。”老者道:“那厮不中拜。老拙命苦养不着他如今不在家了。”三藏道:“何方生理?”老者点头而叹:

“可怜!可怜!若肯何方生理是吾之幸也!那厮专生恶念不务本等专好打家截道杀人放火!相交的都是些狐群狗党!自五日之前出去至今未回。”三藏闻说不敢言喘心中暗想道:

“或者悟空打杀的就是也。”长老神思不安欠身道:“善哉!善哉!如此贤父母何生恶逆儿!”行者近前道:“老官儿似这等不良不肖、奸盗邪淫之子连累父母要他何用!等我替你寻他来打杀了罢。”老者道:“我待也要送了他奈何再无以次人丁纵是不才一定还留他与老汉掩土。”沙僧与八戒笑道:“师兄莫管闲事你我不是官府。他家不肖与我何干!且告施主见赐一束草儿在那厢打铺睡觉天明走路。”老者即起身着沙僧到后园里拿两个稻草教他们在园中草团瓢内安歇。行者牵了马八戒挑了行李同长老俱到团瓢内安歇不题。

却说那伙贼内果有老杨的儿子。自天早在山前被行者打死两个贼他们都四散逃生约摸到四更时候又结坐一伙在门前打门。老者听得门响即披衣道:“妈妈那厮们来也。”

妈妈道:“既来你去开门放他来家。”老者方才开门只见那一伙贼都嚷道:“饿了!饿了!”这老杨的儿子忙入里面叫起他妻来打米煮饭。却厨下无柴往后园里拿柴到厨房里问妻道:“后园里白马是那里的?”其妻道:“是东土取经的和尚昨晚至此借宿公公婆婆管待他一顿晚斋教他在草团瓢内睡哩。”那厮闻言走出草堂拍手打掌笑道:“兄弟们造化!造化!冤家在我家里也!”众贼道:“那个冤家?”那厮道:“却是打死我们头儿的和尚来我家借宿现睡在草团瓢里。”众贼道:

“却好!却好!拿住这些秃驴一个个剁成肉酱一则得那行囊白马二来与我们头儿报仇!”那厮道:“且莫忙你们且去磨刀。等我煮饭熟了大家吃饱些一齐下手。”真个那些贼磨刀的磨刀磨枪的磨枪。那老儿听得此言悄悄的走到后园叫起唐僧四位道:“那厮领众来了知得汝等在此意欲图害我老拙念你远来不忍伤害快早收拾行李我送你往后门出去罢!”三藏听说战兢兢的叩头谢了老者即唤八戒牵马沙僧挑担行者拿了九环锡杖。老者开后门放他去了依旧悄悄的来前睡下。

却说那厮们磨快了刀枪吃饱了饭食时已五更天气一齐来到园中看处却不见了。即忙点灯着火寻彀多时四无踪迹但见后门开着都道:“从后门走了!走了!”一声喊“赶将上拿来。”一个个如飞似箭直赶到东方日出却才望见唐僧。那长老忽听得喊声回头观看后面有二三十人枪刀簇簇而来便叫:“徒弟啊贼兵追至怎生奈何!”行者道:“放心!放心!老孙了他去来!”三藏勒马道:“悟空切莫伤人只吓退他便罢。”行者那肯听信急掣棒回相迎道:“列位那里去?”众贼骂道:“秃厮无礼!还我大王的命来!”那厮们圈子阵把行者围在中间举枪刀乱砍乱搠。这大圣把金箍棒幌一幌碗来粗细把那伙贼打得星落云散汤着的就死挽着的就亡;搕着的骨折擦着的皮伤乖些的跑脱几个痴些的都见阎王!

三藏在马上见打倒许多人慌的放马奔西。猪八戒与沙和尚紧随鞭镫而去。行者问那不死带伤的贼人道:“那个是那杨老儿的儿子?”那贼哼哼的告道:“爷爷那穿黄的是!”行者上前夺过刀来把个穿黄的割下头来血淋淋提在手中收了铁棒拽开云步赶到唐僧马前提着头道:“师父这是杨老儿的逆子被老孙取将级来也。”三藏见了大惊失色慌得跌下马来骂道:“这泼猢狲唬杀我也!快拿过!快拿过!”八戒上前将人头一脚踢下路旁使钉钯筑些土盖了。沙僧放下担子搀着唐僧道:“师父请起。”那长老在地下正了性心中念起《紧箍儿咒》来把个行者勒得耳红面赤眼胀头昏在地下打滚只教:“莫念!莫念!”那长老念彀有十余遍还不住口。行者翻筋斗竖蜻蜓疼痛难禁只叫:“师父饶我罪罢!有话便说莫念!莫念!”三藏却才住口道:“没话说我不要你跟了你回去罢!”行者忍疼磕头道:“师父怎的就赶我去耶?”三藏道:“你这泼猴凶恶太甚不是个取经之人。昨日在山坡下打死那两个贼头我已怪你不仁。及晚了到老者之家蒙他赐斋借宿又蒙他开后门放我等逃了性命虽然他的儿子不肖与我无干也不该就枭他况又杀死多人坏了多少生命伤了天地多少和气。屡次劝你更无一毫善念要你何为!快走!快走!免得又念真言!”行者害怕只教:“莫念莫念!我去也!”说声去一路筋斗云无影无踪遂不见了。咦!这正是:心有凶狂丹不熟神无定位道难成。毕竟不知那大圣投向何方且听下回分解。
------------

第五十七回 真行者落伽山诉苦 假猴王水帘洞誊文

却说孙大圣恼恼闷闷起在空中欲待回花果山水帘洞恐本洞小妖见笑笑我出乎尔反乎尔不是个大丈夫之器;欲待要投奔天宫又恐天宫内不容久住;欲待要投海岛却又羞见那三岛诸仙;欲待要奔龙宫又不伏气求告龙王。真个是无依无倚苦自忖量道:“罢!罢!罢!我还去见我师父还是正果。”遂按下云头径至三藏马前侍立道:“师父恕弟子这遭!

向后再不敢行凶一一受师父教诲千万还得我保你西天去也。”唐僧见了更不答应兜住马即念《紧箍儿咒》颠来倒去又念有二十余遍把大圣咒倒在地箍儿陷在肉里有一寸来深浅方才住口道:“你不回去又来缠我怎的?”行者只教:

“莫念!莫念!我是有处过日子的只怕你无我去不得西天。”

三藏怒道:“你这猢狲杀生害命连累了我多少如今实不要你了!我去得去不得不干你事!快走快走!迟了些儿我又念真言这番决不住口把你脑浆都勒出来哩!”大圣疼痛难忍见师父更不回心没奈何只得又驾筋斗云起在空中忽然省悟道:“这和尚负了我心我且向普陀崖告诉观音菩萨去来。”

好大圣拨回筋斗那消一个时辰早至南洋大海住下祥光直至落伽山上撞入紫竹林中忽见木叉行者迎面作礼道:

“大圣何往?”行者道:“要见菩萨。”木叉即引行者至潮音洞口又见善财童子作礼道:“大圣何来?”行者道:“有事要告菩萨。”

善财听见一个告字笑道:“好刁嘴猴儿!还象当时我拿住唐僧被你欺哩!我菩萨是个大慈大悲大愿大乘救苦救难无边无量的圣善菩萨有甚不是处你要告他?”行者满怀闷气一闻此言心中怒咄的一声把善财童子喝了个倒退道:“这个背义忘恩的小畜生着实愚鲁!你那时节作怪成精我请菩萨收了你皈正迦持如今得这等极乐长生自在逍遥与天同寿还不拜谢老孙转倒这般侮慢!我是有事来告求菩萨却怎么说我刁嘴要告菩萨?”善财陪笑道:“还是个急猴子我与你作笑耍子你怎么就变脸了?”

正讲处只见白鹦哥飞来飞去知是菩萨呼唤木叉与善财遂向前引导至宝莲台下。行者望见菩萨倒身下拜止不住泪如泉涌放声大哭。菩萨教木叉与善财扶起道:“悟空有甚伤感之事明明说来莫哭莫哭我与你救苦消灾也。”行者垂泪再拜道:“当年弟子为人曾受那个气来?自蒙菩萨解脱天灾秉教沙门保护唐僧往西天拜佛求经我弟子舍身拚命救解他的魔障就如老虎口里夺脆骨蛟龙背上揭生鳞。只指望归真正果洗业除邪怎知那长老背义忘恩直迷了一片善缘更不察皂白之苦!”菩萨道:“且说那皂白原因来我听。”行者即将那打杀草寇前后始终细陈了一遍。却说唐僧因他打死多人心生怨恨不分皂白遂念《紧箍儿咒》赶他几次上天无路入地无门特来告诉菩萨。菩萨道:“唐三藏奉旨投西一心要秉善为僧决不轻伤性命。似你有无量神通何苦打死许多草寇!草寇虽是不良到底是个人身不该打死比那妖禽怪兽、鬼魅精魔不同。那个打死是你的功绩;这人身打死还是你的不仁。但祛退散自然救了你师父据我公论还是你的不善。”行者噙泪叩头道:“纵是弟子不善也当将功折罪不该这般逐我。万望菩萨舍大慈悲将《松箍儿咒》念念褪下金箍交还与你放我仍往水帘洞逃生去罢!”菩萨笑道:“《紧箍儿咒》本是如来传我的。(wwW.mianhuatang.la 无弹窗广告)当年差我上东土寻取经人赐我三件宝贝乃是锦襕袈裟、九环锡杖、金紧禁三个箍儿秘授与咒语三篇却无甚么《松箍儿咒》。”行者道:“既如此我告辞菩萨去也。”

菩萨道:“你辞我往那里去?”行者道:“我上西天拜告如来求念《松箍儿咒》去也。”菩萨道:“你且住我与你看看祥晦如何。”行者道:“不消看只这样不祥也彀了。”菩萨道:“我不看你看唐僧的祥晦。”好菩萨端坐莲台运心三界慧眼遥观遍周宇宙霎时间开口道:“悟空你那师父顷刻之际就有伤身之难不久便来寻你。你只在此处待我与唐僧说教他还同你去取经了成正果。”孙大圣只得皈依不敢造次侍立于宝莲台下不题。

却说唐长老自赶回行者教八戒引马沙僧挑担连马四口奔西走不上五十里远近三藏勒马道:“徒弟自五更时出了村舍又被那弼马温着了气恼这半日饥又饥渴又渴那个去化些斋来我吃?”八戒道:“师父且请下马等我看可有邻近的庄村化斋去也。”三藏闻言滚下马来。呆子纵起云头半空中仔细观看一望尽是山岭莫想有个人家。八戒按下云来对三藏道:“却是没处化斋一望之间全无庄舍。”三藏道:“既无化斋之处且得些水来解渴也可。”八戒道:“等我去南山涧下取些水来。”沙僧即取钵盂递与八戒八戒托着钵盂驾起云雾而去。那长老坐在路旁等彀多时不见回来可怜口干舌苦难熬。有诗为证诗曰:保神养气谓之精情性原来一禀形。心乱神昏诸病作形衰精败道元倾。三花不就空劳碌四大萧条枉费争。土木无功金水绝法身疏懒几时成!沙僧在旁见三藏饥渴难忍八戒又取水不来只得稳了行囊拴牢了白马道:

“师父你自在着等我去催水来。”长老含泪无言但点头相答。沙僧急驾云光也向南山而去。

那师父独炼自熬困苦太甚正在怆惶之际忽听得一声响亮唬得长老欠身看处原来是孙行者跪在路旁双手捧着一个磁杯道:“师父没有老孙你连水也不能彀哩。这一杯好凉水你且吃口水解渴待我再去化斋。”长老道:“我不吃你的水!立地渴死我当任命!不要你了!你去罢!”行者道:“无我你去不得西天也。”三藏道:“去得去不得不干你事!泼猢狲!

只管来缠我做甚!”那行者变了脸怒生嗔喝骂长老道:“你这个狠心的泼秃十分贱我!”轮铁棒丢了磁杯望长老脊背上砑了一下那长老昏晕在地不能言语被他把两个青毡包袱提在手中驾筋斗云不知去向。

却说八戒托着钵盂只奔山南坡下忽见山凹之间有一座草舍人家。原来在先看时被山高遮住未曾见得;今来到边前方知是个人家。呆子暗想道:“我若是这等丑嘴脸决然怕我枉劳神思断然化不得斋饭。须是变好!须是变好!”好呆子捻着诀念个咒把身摇了七八摇变作一个食痨病黄胖和尚口里哼哼喷喷的挨近门前叫道:“施主厨中有剩饭路上有饥人。贫僧是东土来往西天取经的我师父在路饥渴了家中有锅巴冷饭千万化些儿救口。”原来那家子男人不在都去插秧种谷去了只有两个女人在家正才煮了午饭盛起两盆却收拾送下田锅里还有些饭与锅巴未曾盛了。那女人见他这等病容却又说东土往西天去的话只恐他是病昏了胡说又怕跌倒死在门只得哄哄翕翕将些剩饭锅巴满满的与了一钵。呆子拿转来现了本象径回旧路。正走间听得有人叫“八戒”。八戒抬头看时却是沙僧站在山崖上喊道:“这里来!这里来!”及下崖迎至面前道:“这涧里好清水不舀你往那里去的?”八戒笑道:“我到这里见山凹子有个人家我去化了这一钵干饭来了。”沙僧道:“饭也用着只是师父渴得紧了怎得水去?”八戒道:“要水也容易你将衣襟来兜着这饭等我使钵盂去舀水。”

二人欢欢喜喜回至路上只见三藏面磕地倒在尘埃白马撒缰在路旁长嘶跑跳行李担不见踪影。慌得八戒跌脚捶胸大呼小叫道:“不消讲!不消讲!这还是孙行者赶走的余党来此打杀师父抢了行李去了!”沙僧道:“且去把马拴住!”只叫:“怎么好!怎么好!这诚所谓半途而废中道而止也!”叫一声:“师父!”满眼抛珠伤心痛哭。八戒道:“兄弟且休哭如今事已到此取经之事且莫说了。你看着师父的尸灵等我把马骑到那个府州县乡村店集卖几两银子买口棺木把师父埋了我两个各寻道路散伙。”沙僧实不忍舍将唐僧扳转身体以脸温脸哭一声:“苦命的师父!”只见那长老口鼻中吐出热气胸前温暖连叫:“八戒你来!师父未伤命哩!”那呆子才近前扶起。长老苏醒呻吟一会骂道:“好泼猢狲打杀我也!”沙僧、八戒问道:“是那个猢狲?”长老不言只是叹息却讨水吃了几口才说:“徒弟你们刚去那悟空更来缠我。是我坚执不收他遂将我打了一棒青毡包袱都抢去了。”八戒听说咬响口中牙起心头火道:“叵耐这泼猴子怎敢这般无礼!”教沙僧道:“你伏侍师父等我到他家讨包袱去!”沙僧道:“你且休怒我们扶师父到那山凹人家化些热茶汤将先化的饭热热调理师父再去寻他。”八戒依言把师父扶上马拿着钵盂兜着冷饭直至那家门只见那家止有个老婆子在家忽见他们慌忙躲过。沙僧合掌道:“老母亲我等是东土唐朝差往西天去者师父有些不快特拜府上化口热茶汤与他吃饭。”那妈妈道:“适才有个食痨病和尚说是东土差来的已化斋去了又有个甚么东土的。我没人在家请别转转。”长老闻言扶着八戒下马躬身道:“老婆婆我弟子有三个徒弟合意同心保护我上天竺国大雷音拜佛求经。只因我大徒弟唤孙悟空一生凶恶不遵善道是我逐回。不期他暗暗走来着我背上打了一棒将我行囊衣钵抢去。如今要着一个徒弟寻他取讨因在那空路上不是坐处特来老婆婆府上权安息一时。待讨将行李来就行决不敢久住。”那妈妈道:“刚才一个食痨病黄胖和尚他化斋去了也说是东土往西天去的怎么又有一起?”

八戒忍不住笑道:“就是我。因我生得嘴长耳大恐你家害怕不肯与斋故变作那等模样。你不信我兄弟衣兜里不是你家锅巴饭?”那妈妈认得果是他与的饭遂不拒他留他们坐了却烧了一確热茶递与沙僧泡饭。沙僧即将冷饭泡了递与师父。师父吃了几口定性多时道:“那个去讨行李?”八戒道:

“我前年因师父赶他回去我曾寻他一次认得他花果山水帘洞等我去!等我去!”长老道:“你去不得。那猢狲原与你不和你又说话粗鲁或一言两句之间有些差池他就要打你。着悟净去罢。”沙僧应承道:“我去我去。”长老又吩咐沙僧道:“你到那里须看个头势。他若肯与你包袱你就假谢谢拿来;若不肯切莫与他争竞径至南海菩萨处将此情告诉请菩萨去问他要。”沙僧一一听从向八戒道:“我今寻他去你千万莫僝僽好生供养师父。这人家亦不可撒泼恐他不肯供饭我去就回。”八戒点头道:“我理会得。但你去讨得讨不得次早回来不要弄做尖担担柴两头脱也。”沙僧遂捻了诀驾起云光直奔东胜神洲而去。真个是:身在神飞不守舍有炉无火怎烧丹。黄婆别主求金老木母延师奈病颜。此去不知何日返这回难量几时还。五行生克情无顺只待心猿复进关。

那沙僧在半空里行经三昼夜方到了东洋大海忽闻波浪之声低头观看真个是黑雾涨天阴气盛沧溟衔日晓光寒。

他也无心观玩望仙山渡过瀛洲向东方直抵花果山界。乘海风踏水势又多时却望见高峰排戟峻壁悬屏即至峰头按云找路下山寻水帘洞。步近前只听得一派喧声见那山中无数猴精滔滔乱嚷。沙僧又近前仔细再看原来是孙行者高坐石台之上双手扯着一张纸朗朗的念道:“东土大唐王皇帝李驾前敕命御弟圣僧陈玄奘法师上西方天竺国娑婆灵山大雷音寺专拜如来佛祖求经。朕因促病侵身魂游地府幸有阳数臻长感冥君放送回生广陈善会修建度亡道场。盛蒙救苦救难观世音菩萨金身出现指示西方有佛有经可度幽亡脱特着法师玄奘远历千山询求经偈。倘过西邦诸国不灭善缘照牒施行。大唐贞观一十三年秋吉日御前文牒。自别大国以来经度诸邦中途收得大徒弟孙悟空行者二徒弟猪悟能八戒三徒弟沙悟净和尚。”念了从头又念。沙僧听得是通关文牒止不住近前厉声高叫:“师兄师父的关文你念他怎的?”

那行者闻言急抬头不认得是沙僧叫:“拿来!拿来!”众猴一齐围绕把沙僧拖拖扯扯拿近前来喝道:“你是何人擅敢近吾仙洞?”沙僧见他变了脸不肯相认只得朝上行礼道:“上告师兄前者实是师父性暴错怪了师兄把师兄咒了几遍逐赶回家。一则弟等未曾劝解二来又为师父饥渴去寻水化斋。不意师兄好意复来又怪师父执法不留遂把师父打倒昏晕在地将行李抢去。后救转师父特来拜兄若不恨师父还念昔日解脱之恩同小弟将行李回见师父共上西天了此正果。倘怨恨之深不肯同去千万把包袱赐弟兄在深山乐桑榆晚景亦诚两全其美也。”

行者闻言呵呵冷笑道:“贤弟此论甚不合我意。我打唐僧抢行李不因我不上西方亦不因我爱居此地。我今熟读了牒文我自己上西方拜佛求经送上东土我独成功教那南赡部洲人立我为祖万代传名也。”沙僧笑道:“师兄言之欠当自来没个孙行者取经之说。我佛如来造下三藏真经原着观音菩萨向东土寻取经人求经要我们苦历千山询求诸国保护那取经人。菩萨曾言:取经人乃如来门生号曰金蝉长老只因他不听佛祖谈经贬下灵山转生东土教他果正西方复修大道。遇路上该有这般魔障解脱我等三人与他做护法。兄若不得唐僧去那个佛祖肯传经与你!却不是空劳一场神思也?”

那行者道:“贤弟你原来懞懂但知其一不知其二。谅你说你有唐僧同我保护我就没有唐僧?我这里另选个有道的真僧在此老孙独力扶持有何不可!已选明日起身去矣。你不信待我请来你看。”叫:“小的们快请老师父出来。”果跑进去牵出一匹白马请出一个唐三藏跟着一个八戒挑着行李;一个沙僧拿着锡杖。这沙僧见了大怒道:“我老沙行不更名坐不改姓那里又有一个沙和尚!不要无礼!吃我一杖!”好沙僧双手举降妖杖把一个假沙僧劈头一下打死原来这是一个猴精。那行者恼了轮金箍棒帅众猴把沙僧围了。沙僧东冲西撞打出路口纵云雾逃生道:“这泼猴如此惫懒我告菩萨去来!”那行者见沙僧打死一个猴精把沙和尚逼得走了他也不来追赶回洞教小的们把打死的妖尸拖在一边剥了皮取肉煎炒将椰子酒、葡萄酒同众猴都吃了。另选一个会变化的妖猴还变一个沙和尚从新教道要上西方不题。

沙僧一驾云离了东海行经一昼夜到了南海。正行时早见落伽山不远急至前低停云雾观看。好去处!果然是:包乾之奥括坤之区。会百川而浴日滔星归众流而生风漾月。潮腾凌大鲲化波翻浩荡巨鳌游。水通西北海浪合正东洋。四海相连同地脉仙方洲岛各仙宫。休言满地蓬莱且看普陀云洞。好景致!山头霞彩壮元精岩下祥风漾月晶。紫竹林中飞孔雀绿杨枝上语灵鹦。琪花瑶草年年秀宝树金莲岁岁生。白鹤几番朝顶上素鸾数次到山亭。游鱼也解修真性跃浪穿波听讲经。沙僧徐步落伽山玩看仙境只见木叉行者当面相迎道:“沙悟净你不保唐僧取经却来此何干?”沙僧作礼毕道:

“有一事特来朝见菩萨烦为引见引见。”木叉情知是寻行者更不题起即先进去对菩萨道:“外有唐僧的小徒弟沙悟净朝拜。”孙行者在台下听见笑道:“这定是唐僧有难沙僧来请菩萨的。”菩萨即命木叉门外叫进。这沙僧倒身下拜拜罢抬头正欲告诉前事忽见孙行者站在旁边等不得说话就掣降妖杖望行者劈脸便打。这行者更不回手彻身躲过。沙僧口里乱骂道:“我把你个犯十恶造反的泼猴!你又来影瞒菩萨哩!”菩萨喝道:“悟净不要动手有甚事先与我说。”沙僧收了宝杖再拜台下气冲冲的对菩萨道:“这猴一路行凶不可数计。前日在山坡下打杀两个剪路的强人师父怪他。不期晚间就宿在贼窝主家里又把一伙贼人尽情打死又血淋淋提一个人头来与师父看。师父唬得跌下马来骂了他几句赶他回来。分别之后师父饥渴太甚教八戒去寻水久等不来又教我去寻他。不期孙行者见我二人不在复回来把师父打一铁棍将两个青毡包袱抢去。我等回来将师父救醒特来他水帘洞寻他讨包袱不想他变了脸不肯认我将师父关文念了又念。我问他念了做甚他说不保唐僧他要自上西天取经送上东土算他的功果立他为祖万古传扬。我又说:没唐僧那肯传经与你?他说他选了一个有道的真僧。及请出果是一匹白马一个唐僧后跟着八戒、沙僧。我道我便是沙和尚那里又有个沙和尚?是我赶上前打了他一宝杖原来是个猴精。他就帅众拿我是我特来告请菩萨。不知他会使筋斗云预先到此处又不知他将甚巧语花言影瞒菩萨也。”菩萨道:“悟净不要赖人悟空到此今已四日我更不曾放他回去他那里有另请唐僧、自去取经之意?”沙僧道:“见如今水帘洞有一个孙行者怎敢欺诳?”

菩萨道:“既如此你休急教悟空与你同去花果山看看。是真难灭是假易除到那里自见分晓。”这大圣闻言即与沙僧辞了菩萨。这一去到那花果山前分皂白水帘洞口辨真邪。毕竟不知如何分辨且听下回分解。
------------

第五十八回 二心搅乱大乾坤 一体难修真寂灭

这行者与沙僧拜辞了菩萨纵起两道祥光离了南海。原来行者筋斗云快沙和尚仙云觉迟行者就要先行。沙僧扯住道:“大哥不必这等藏头露尾先去安根待小弟与你一同走。”

大圣本是良心沙僧却有疑意真个二人同驾云而去。不多时果见花果山按下云头二人洞外细看果见一个行者高坐石台之上与群猴饮酒作乐。模样与大圣无异:也是黄金箍金睛火眼;身穿也是锦布直裰腰系虎皮裙;手中也拿一条儿金箍铁棒足下也踏一双麂皮靴;也是这等毛脸雷公嘴朔腮别土星查耳额颅阔獠牙向外生。这大圣怒一撒手撇了沙和尚掣铁棒上前骂道:“你是何等妖邪敢变我的相貌敢占我的儿孙擅居吾仙洞擅作这威福!”那行者见了公然不答也使铁棒来迎。二行者在一处果是不分真假好打呀:两条棒二猴精这场相敌实非轻。都要护持唐御弟各施功绩立英名。真猴实受沙门教假怪虚称佛子情。盖为神通多变化无真无假两相平。一个是混元一气齐天圣一个是久炼千灵缩地精。这个是如意金箍棒那个是随心铁杆兵。隔架遮拦无胜败撑持抵敌没输赢。先前交手在洞外少顷争持起半空。他两个各踏云光跳斗上九霄云内。沙僧在旁不敢下手见他们战此一场诚然难认真假欲待拔刀相助又恐伤了真的。忍耐良久且纵身跳下山崖使降妖宝杖打近水帘洞外惊散群妖掀翻石凳把饮酒食肉的器皿尽情打碎寻他的青毡包袱四下里全然不见。原来他水帘洞本是一股瀑布飞泉遮挂洞门远看似一条白布帘儿近看乃是一股水脉故曰水帘洞。沙僧不知进步来历故此难寻。即便纵云赶到九霄云里轮着宝杖又不好下手。大圣道:“沙僧你既助不得力且回复师父说我等这般这般等老孙与此妖打上南海落伽山菩萨前辨个真假。”道罢那行者也如此说。沙僧见两个相貌、声音更无一毫差别皂白难分只得依言拨转云头回复唐僧不题。

你看那两个行者且行且斗直嚷到南海径至落伽山打打骂骂喊声不绝。早惊动护法诸天即报入潮音洞里道:“菩萨果然两个孙悟空打将来也。”那菩萨与木叉行者、善财童子、龙女降莲台出门喝道:“那孽畜那里走!”这两个递相揪住道:“菩萨这厮果然象弟子模样。才自水帘洞打起战斗多时不分胜负。沙悟净肉眼愚蒙不能分识有力难助是弟子教他回西路去回复师父我与这厮打到宝山借菩萨慧眼与弟子认个真假辨明邪正。”道罢那行者也如此说一遍。众诸天与菩萨都看良久莫想能认。菩萨道:“且放了手两边站下等我再看。”果然撒手两边站定。这边说:“我是真的!”那边说:“他是假的!”

菩萨唤木叉与善财上前悄悄吩咐:“你一个帮住一个等我暗念《紧箍儿咒》看那个害疼的便是真不疼的便是假。”他二人果各帮一个。菩萨暗念真言两个一齐喊疼都抱着头地下打滚只叫:“莫念!莫念!”菩萨不念他两个又一齐揪住照旧嚷斗。菩萨无计奈何即令诸天木叉上前助力。众神恐伤真的亦不敢下手。菩萨叫声“孙悟空”两个一齐答应。菩萨道:“你当年官拜弼马温大闹天宫时神将皆认得你你且上界去分辨回话。”这大圣谢恩那行者也谢恩。

二人扯扯拉拉口里不住的嚷斗径至南天门外慌得那广目天王帅马赵温关四大天将及把门大小众神各使兵器挡住道:“那里走!此间可是争斗之处?”大圣道:“我因保护唐僧往西天取经在路上打杀贼徒那三藏赶我回去我径到普陀崖见观音菩萨诉苦不想这妖精几时就变作我的模样打倒唐僧抢去包袱。有沙僧至花果山寻讨只见这妖精占了我的巢穴后到普陀崖告请菩萨又见我侍立台下沙僧诳说是我驾筋斗云又先在菩萨处遮饰。菩萨却是个正明不听沙僧之言命我同他到花果山看验。原来这妖精果象老孙模样才自水帘洞打到普陀山见菩萨菩萨也难识认故打至此间烦诸天眼力与我认个真假。”说罢那行者也似这般这般说了一遍。众天神看彀多时也不能辨他两个吆喝道:“你们既不能认让开路等我们去见玉帝!”众神搪抵不住放开天门直至灵霄宝殿马元帅同张葛许邱四天师奏道:“下界有一般两个孙悟空打进天门口称见王。”说不了两个直嚷将进来唬得那玉帝即降立宝殿问曰:“你两个因甚事擅闹天宫嚷至朕前寻死!”大圣口称:“万岁!万岁!臣今皈命秉教沙门再不敢欺心诳上只因这个妖精变作臣的模样。”如此如彼把前情备陈了一遍“指望与臣辨个真假!”那行者也如此陈了一遍。玉帝即传旨宣托塔李天王教:“把照妖镜来照这厮谁真谁假教他假灭真存。”天王即取镜照住请玉帝同众神观看。镜中乃是两个孙悟空的影子金箍衣服毫不差。玉帝亦辨不出赶出殿外。这大圣呵呵冷笑那行者也哈哈欢喜揪头抹颈复打出天门坠落西方路上道:“我和你见师父去!我和你见师父去!”

却说那沙僧自花果山辞他两个又行了三昼夜回至本庄把前事对唐僧说了一遍。唐僧自家悔恨道:“当时只说是孙悟空打我一棍抢去包袱岂知却是妖精假变的行者!”沙僧又告道:“这妖又假变一个长老一匹白马又有一个八戒挑着我们包袱又有一个变作是我。我忍不住恼怒一杖打死原是一个猴精。因此惊散又到菩萨处诉苦。菩萨着我与师兄又同去识认那妖果与师兄一般模样。我难助力故先来回复师父。”

三藏闻言大惊失色。八戒哈哈大笑道:“好好好!应了这施主家婆婆之言了!他说有几起取经的这却不又是一起?”那家子老老小小的都来问沙僧:“你这几日往何处讨盘缠去的?”沙僧笑道:“我往东胜神洲花果山寻大师兄取讨行李又到南海普陀山拜见观音菩萨却又到花果山方才转回至此。”那老者又问:“往返有多少路程?”沙僧道:“约有二十余万里。”老者道:“爷爷呀似这几日就走了这许多路只除是驾云方能彀得到!”八戒道:“不是驾云如何过海?”沙僧道:“我们那算得走路若是我大师兄只消一二日可往回也。”那家子听言都说是神仙八戒道:“我们虽不是神仙神仙还是我们的晚辈哩!”

正说间只听半空中喧哗人嚷慌得都出来看却是两个行者打将来。八戒见了忍不住手痒道:“等我去认认看。”好呆子急纵身跳起望空高叫道:“师兄莫嚷我老猪来也!”那两个一齐应道:“兄弟来打妖精!来打妖精!”那家子又惊又喜道:“是几位腾云驾雾的罗汉歇在我家!就是愿斋僧的也斋不着这等好人!”更不计较茶饭愈加供养又说:“这两个行者只怕斗出不好来地覆天翻作祸在那里!”三藏见那老者当面是喜背后是忧即开言道:“老施主放心莫生忧叹。贫僧收伏了徒弟去恶归善自然谢你。”那老者满口回答道:“不敢!不敢!”沙僧道:“施主休讲师父可坐在这里等我和二哥去一家扯一个来到你面前你就念念那话儿看那个害疼的就是真的不疼的就是假的。”三藏道:“言之极当。”沙僧果起在半空道:“二位住了手我同你到师父面前辨个真假去。”这大圣放了手那行者也放了手。沙僧搀住一个叫道:“二哥你也搀住一个。”果然搀住落下云头径至草舍门外。三藏见了就念《紧箍儿咒》二人一齐叫苦道:“我们这等苦斗你还咒我怎的?莫念!莫念!”那长老本心慈善遂住了口不念却也不认得真假。他两个挣脱手依然又打。这大圣道:“兄弟们保着师父等我与他打到阎王前折辨去也!”那行者也如此说二人抓抓挜挜须臾又不见了。八戒道:“沙僧你既到水帘洞看见假八戒挑着行李怎么不抢将来?”沙僧道:“那妖精见我使宝杖打他假沙僧他就乱围上来要拿是我顾性命走了。及告菩萨与行者复至洞口他两个打在空中是我去掀翻他的石凳打散他的小妖只见一股瀑布泉水流竟不知洞门开在何处寻不着行李所以空手回复师命也。”八戒道:“你原来不晓得。

我前年请他去时先在洞门外相见后被我说泛了他他就跳下去洞里换衣来时我看见他将身往水里一钻那一股瀑布水流就是洞门。想必那怪将我们包袱收在那里面也。”三藏道:“你既知此门你可趁他都不在家可先到他洞里取出包袱我们往西天去罢。他就来我也不用他了。”八戒道:“我去。”沙僧说:“二哥他那洞前有千数小猴你一人恐弄他不过反为不美。”八戒笑道:“不怕!不怕!”急出门纵着云雾径上花果山寻取行李不题。

却说那两个行者又打嚷到阴山背后唬得那满山鬼战战兢兢藏藏躲躲。有先跑的撞入阴司门里报上森罗宝殿道:

“大王背阴山上有两个齐天大圣打得来也!”慌得那第一殿秦广王传报与二殿楚江王、三殿宋帝王、四殿卞城王五殿阎罗王、六殿平等王、七殿泰山王、八殿都市王、九殿忤官王、十殿转轮王。一殿转一殿霎时间十王会齐又着人飞报与地藏王。尽在森罗殿上点聚阴兵等擒真假。只听得那强风滚滚惨雾漫漫二行者一翻一滚的打至森罗殿下。阴君近前挡住道:“大圣有何事闹我幽冥?”这大圣道:“我因保唐僧西天取经路过西梁国至一山有强贼截劫我师是老孙打死几个师父怪我把我逐回。我随到南海菩萨处诉告不知那妖精怎么就绰着口气假变作我的模样在半路上打倒师父抢夺了行李。师弟沙僧向我本山取讨包袱这妖假立师名要往西天取经。沙僧跑遁至南海见菩萨我正在侧他备说原因菩萨又命我同他至花果山观看果被这厮占了我巢穴。我与他争辨到菩萨处其实相貌、言语等俱一般菩萨也难辨真假。又与这厮打上天堂众神亦果难辨因见我师我师念《紧箍咒》试验与我一般疼痛。故此闹至幽冥望阴君与我查看生死簿见假行者是何出身快早追他魂魄免教二心沌乱。”那怪亦如此说一遍。阴君闻言即唤管簿判官一一从头查勘更无个假行者之名。再看毛虫文簿那猴子一百三十条已是孙大圣幼年得道之时大闹阴司消死名一笔勾之自后来凡是猴属尽无名号。

查勘毕当殿回报阴君各执笏对行者说:“大圣幽冥处既无名号可查你还到阳间去折辨。”正说处只听得地藏王菩萨道:

“且住!且住!等我着谛听与你听个真假。”原来那谛听是地藏菩萨经案下伏的一个兽名。他若伏在地下一霎时将四大部洲山川社稷、洞天福地之间蠃虫鳞虫毛虫羽虫昆虫天仙地仙神仙人仙鬼仙可以顾鉴善恶察听贤愚。那兽奉地藏钧旨就于森罗庭院之中俯伏在地须臾抬起头来对地藏道:“怪名虽有但不可当面说破又不能助力擒他。”地藏道:“当面说出便怎么?”谛听道:“当面说出恐妖精恶搔扰宝殿致令阴府不安。”又问:“何为不能助力擒拿?”谛听道:“妖精神通与孙大圣无二。幽冥之神能有多少法力?故此不能擒拿。”地藏道:“似这般怎生祛除?”谛听言:“佛法无边。”地藏早已省悟即对行者道:“你两个形容如一神通无二若要辨明须到雷音寺释迦如来那里方得明白。”两个一齐嚷道:“说的是!说的是!我和你西天佛祖之前折辨去!”那十殿阴君送出谢了地藏回上翠云宫着鬼使闭了幽冥关隘不题。

看那两个行者飞云奔雾打上西天。有诗为证诗曰:人有二心生祸灾天涯海角致疑猜。欲思宝马三公位又忆金銮一品台南征北讨无休歇东挡西除未定哉。禅门须学无心诀静养婴儿结圣胎。他两个在那半空里扯扯拉拉抓抓挜挜且行且斗直嚷至大西天灵鹫仙山雷音宝刹之外。早见那四大菩萨、八大金刚、五百阿罗、三千揭谛、比丘尼、比丘僧、优婆塞、优婆夷诸大圣众都到七宝莲台之下各听如来说法。那如来正讲到这:不有中有不无中无。不色中色不空中空。非有为有非无为无。非色为色非空为空。空即是空色即是色。色无定色色即是空。空无定空空即是色。知空不空知色不色。

名为照了始达妙音。概众稽皈依流通诵读之际如来降天花普散缤纷即离宝座对大众道:“汝等俱是一心且看二心竞斗而来也。”大众举目看之果是两个行者吆天喝地打至雷音胜境。慌得那八大金刚上前挡住道:“汝等欲往那里去?”

这大圣道:“妖精变作我的模样欲至宝莲台下烦如来为我辨个虚实也。”众金刚抵挡不住直嚷至台下跪于佛祖之前拜告道:“弟子保护唐僧来造宝山求取真经一路上炼魔缚怪不知费了多少精神。前至中途偶遇强徒劫掳委是弟子二次打伤几人师父怪我赶回不容同拜如来金身。弟子无奈只得投奔南海见观音诉苦。不期这个妖精假变弟子声音相貌将师父打倒把行李抢去。师弟悟净寻至我山被这妖假捏巧言说有真僧取经之故。悟净脱身至南海备说详细。观音知之遂令弟子同悟净再至我山。因此两人比并真假打至南海又打到天宫又曾打见唐僧打见冥府俱莫能辨认。故此大胆轻造千乞大开方便之门广垂慈悯之念与弟子辨明邪正庶好保护唐僧亲拜金身取经回东土永扬大教。”大众听他两张口一样声俱说一遍众亦莫辨惟如来则通知之。正欲道破忽见南下彩云之间来了观音参拜我佛。

我佛合掌道:“观音尊者你看那两个行者谁是真假?”菩萨道:“前日在弟子荒境委不能辨。他又至天宫地府亦俱难认特来拜告如来千万与他辨明辨明。”如来笑道:“汝等法力广大只能普阅周天之事不能遍识周天之物亦不能广会周天之种类也。”菩萨又请示周天种类如来才道:“周天之内有五仙乃天地神人鬼;有五虫乃蠃鳞毛羽昆。这厮非天非地非神非人非鬼亦非蠃非鳞非毛非羽非昆。又有四猴混世不入十类之种。”菩萨道:“敢问是那四猴?”如来道:“第一是灵明石猴通变化识天时知地利移星换斗。第二是赤尻马猴晓阴阳会人事善出入避死延生。第三是通臂猿猴拿日月缩千山辨休咎乾坤摩弄。第四是六耳猕猴善聆音能察理知前后万物皆明。此四猴者不入十类之种不达两间之名。我观假悟空乃六耳猕猴也。此猴若立一处能知千里外之事凡人说话亦能知之故此善聆音能察理知前后万物皆明。与真悟空同象同音者六耳猕猴也。”那猕猴闻得如来说出他的本象胆战心惊急纵身跳起来就走。如来见他走时即令大众下手早有四菩萨、八金刚、五百阿罗、三千揭谛、比丘僧、比丘尼、优婆塞、优婆夷、观音、木叉一齐围绕。孙大圣也要上前如来道:“悟空休动手待我与你擒他。”那猕猴毛骨悚然料着难脱即忙摇身一变变作个蜜蜂儿往上便飞。如来将金钵盂撇起去正盖着那蜂儿落下来。大众不知以为走了如来笑云:“大众休言妖精未走见在我这钵盂之下。”大众一上前把钵盂揭起果然见了本象是一个六耳猕猴。孙大圣忍不住轮起铁棒劈头一下打死至今绝此一种。如来不忍道声:

“善哉!善哉!”大圣道:“如来不该慈悯他他打伤我师父抢夺我包袱依律问他个得财伤人白昼抢夺也该个斩罪哩!”如来道:“你自快去保护唐僧来此求经罢。”大圣叩头谢道:“上告如来得知那师父定是不要我我此去若不收留却不又劳一番神思!望如来方便把松箍儿咒念一念褪下这个金箍交还如来放我还俗去罢。”如来道:“你休乱想切莫放刁。我教观音送你去不怕他不收。好生保护他去那时功成归极乐汝亦坐莲台。”

那观音在旁听说即合掌谢了圣恩领悟空辄驾云而去随后木叉行者、白鹦哥一同赶上。不多时到了中途草舍人家沙和尚看见急请师父拜门迎接。菩萨道:“唐僧前日打你的乃假行者六耳猕猴也幸如来知识已被悟空打死。你今须是收留悟空一路上魔障未消须得他保护你才得到灵山见佛取经再休嗔怪。”三藏叩头道:“谨遵教旨。”正拜谢时只听得正东上狂风滚滚众目视之乃猪八戒背着两个包袱驾风而至。呆子见了菩萨倒身下拜道:“弟子前日别了师父至花果山水帘洞寻得包袱果见一个假唐僧假八戒都被弟子打死原是两个猴身。却入里方寻着包袱当时查点一物不少。却驾风转此更不知两行者下落如何。”菩萨把如来识怪之事说了一遍。那呆子十分欢喜称谢不尽。师徒们拜谢了菩萨回海却都照旧合意同心洗冤解怒。又谢了那村舍人家整束行囊马匹找大路而西。正是:中道分离乱五行降妖聚会合元明。神归心舍禅方定六识祛降丹自成。毕竟这去不知三藏几时得面佛求经且听下回分解。
------------

第五十九回 唐三藏路阻火焰山 孙行者一调芭蕉扇

若干种性本来同海纳无穷。千思万虑终成妄般般色色和融。有日功完行满圆明法性高隆。休教差别走西东紧锁牢靴。收来安放丹炉内炼得金乌一样红。朗朗辉辉娇艳任教出入乘龙。话表三藏遵菩萨教旨收了行者与八戒沙僧剪断二心锁鑨猿马同心戮力赶奔西天。说不尽光阴似箭日月如梭历过了夏月炎天却又值三秋霜景但见那:薄云断绝西风紧鹤鸣远岫霜林锦。光景正苍凉山长水更长。征鸿来北塞玄鸟归南陌。客路怯孤单衲衣容易寒。师徒四众进前行处渐觉热气蒸人。三藏勒马道:“如今正是秋天却怎返有热气?”八戒道:“原来不知西方路上有个斯哈哩国乃日落之处俗呼为天尽头。若到申酉时国王差人上城擂鼓吹角混杂海沸之严。日乃太阳真火落于西海之间如火淬水接声滚沸;若无鼓角之声混耳即振杀城中小儿。此地热气蒸人想必到日落之处也。”大圣听说忍不住笑道:“呆子莫乱谈!若论斯哈哩国正好早哩。似师父朝三暮二的这等担阁就从小至老老了又小老小三生也还不到。”八戒道:“哥啊据你说不是日落之处为何这等酷热?”沙僧道:“想是天时不正秋行夏令故也。”他三个正都争讲只见那路旁有座庄院乃是红瓦盖的房舍红砖砌的垣墙红油门扇红漆板榻一片都是红的。三藏下马道:“悟空你去那人家问个消息看那炎热之故何也。”

大圣收了金箍棒整肃衣裳扭捏作个斯文气象绰下大路径至门前观看。那门里忽然走出一个老者但见他:穿一领黄不黄、红不红的葛布深衣戴一顶青不青、皂不皂的篾丝凉帽。手中拄一根弯不弯、直不直、暴节竹杖足下踏一双新不新、旧不旧、搫靸靴鞋。面似红铜须如白练。两道寿眉遮碧眼一张吮口露金牙。那老者猛抬头看见行者吃了一惊拄着竹杖喝道:“你是那里来的怪人?在我这门何干?”行者答礼道:“老施主休怕我我不是甚么怪人贫僧是东土大唐钦差上西方求经者。师徒四人适至宝方见天气蒸热一则不解其故二来不地知名特拜问指教一二。”那老者却才放心笑云:

“长老勿罪我老汉一时眼花不识尊颜。”行者道:“不敢。”老者又问:“令师在那条路上?”行者道:“那南大路上立的不是!”老者教:“请来请来。”行者欢喜把手一招三藏即同八戒、沙僧牵白马挑行李近前都对老者作礼。老者见三藏丰姿标致八戒沙僧相貌奇稀又惊又喜只得请入里坐教小的们看茶一壁厢办饭。三藏闻言起身称谢道:“敢问公公贵处遇秋何返炎热?”老者道:“敝地唤做火焰山无春无秋四季皆热。”三藏道:“火焰山却在那边?可阻西去之路老者道:“西方却去不得。那山离此有六十里远正是西方必由之路却有八百里火焰四周围寸草不生。若过得山就是铜脑盖铁身躯也要化成汁哩。”三藏闻言大惊失色不敢再问。

只见门外一个少年男子推一辆红车儿住在门旁叫声“卖糕!”大圣拔根毫毛变个铜钱问那人买糕。那人接了钱不论好歹揭开车儿上衣裹热气腾腾拿出一块糕递与行者。

行者托在手中好似火盆里的灼炭煤炉内的红钉。你看他左手倒在右手右手换在左手只道:“热热热!难吃难吃!”那男子笑道:“怕热莫来这里这里是这等热。”行者道:“你这汉子好不明理常言道不冷不热五谷不结。他这等热得很你这糕粉自何而来?”那人道:“若知糕粉米敬求铁扇仙。”行者道:“铁扇仙怎的?”那人道:“铁扇仙有柄芭蕉扇。求得来一扇息火二扇生风三扇下雨我们就布种及时收割故得五谷养生。不然诚寸草不能生也。”行者闻言急抽身走入里面将糕递与三藏道:“师父放心且莫隔年焦着吃了糕我与你说。”长老接糕在手向本宅老者道:“公公请糕。(WWW.mianhuatang.la 好看的小说)”老者道:“我家的茶饭未奉敢吃你糕?”行者笑道:“老人家茶饭倒不必赐我问你:铁扇仙在那里住?”老者道:“你问他怎的?”行者道:“适才那卖糕人说此仙有柄芭蕉扇求将来一扇息火二扇生风三扇下雨你这方布种收割才得五谷养生。我欲寻他讨来扇息火焰山过去且使这方依时收种得安生也。”老者道:“固有此说。你们却无礼物恐那圣贤不肯来也。”三藏道:

“他要甚礼物?”老者道:“我这里人家十年拜求一度。四猪四羊花红表里异香时果鸡鹅美酒沐浴虔诚拜到那仙山请他出洞至此施为。”行者道:“那山坐落何处?唤甚地名?有几多里数?等我问他要扇子去。”老者道:“那山在西南方名唤翠云山。山中有一仙洞名唤芭蕉洞。我这里众信人等去拜仙山往回要走一月计有一千四百五六十里。”行者笑道:“不打紧就去就来。”那老者道:“且住吃些茶饭办些干粮须得两人做伴。那路上没有人家又多狼虎非一日可到莫当耍子。”行者笑道:“不用不用我去也!”说一声忽然不见。那老者慌张道:“爷爷呀!原来是腾云驾雾的神人也!”

且不说这家子供奉唐僧加倍却说那行者霎时径到翠云山按住祥光正自找寻洞口忽然闻得丁丁之声乃是山林内一个樵夫伐木。行者即趋步至前又闻得他道:“云际依依认旧林断崖荒草路难寻。西山望见朝来雨南涧归时渡处深。”行者近前作礼道:“樵哥问讯了。”那樵子撇了柯斧答礼道:“长老何往?”行者道:“敢问樵哥这可是翠云山?”樵子道:“正是。”行者道:“有个铁扇仙的芭蕉洞在何处?”樵子笑道:“这芭蕉洞虽有却无个铁扇仙只有个铁扇公主又名罗刹女。”

行者道:“人言他有一柄芭蕉扇能熄得火焰山敢是他么?”樵子道:“正是正是这圣贤有这件宝贝善能熄火保护那方人家故此称为铁扇仙。我这里人家用不着他只知他叫做罗刹女乃大力牛魔王妻也。”行者闻言大惊失色心中暗想道:

“又是冤家了!当年伏了红孩儿说是这厮养的。前在那解阳山破儿洞遇他叔子尚且不肯与水要作报仇之意今又遇他父母怎生借得这扇子耶?”樵子见行者沉思默虑嗟叹不已便笑道:“长老你出家人有何忧疑?这条小路儿向东去不上五六里就是芭蕉洞休得心焦。”行者道:“不瞒樵哥说我是东土唐朝差往西天求经的唐僧大徒弟。前年在火云洞曾与罗刹之子红孩儿有些言语但恐罗刹怀仇不与故生忧疑。”樵子道:“大丈夫鉴貌辨色只以求扇为名莫认往时之溲话管情借得。”行者闻言深深唱个大喏道:“谢樵哥教诲我去也。”

遂别了樵夫径至芭蕉洞口但见那两扇门紧闭牢关洞外风光秀丽。好去处!正是那:山以石为骨石作土之精。烟霞含宿润苔藓助新青。嵯峨势耸欺蓬岛幽静花香若海瀛。几树乔松栖野鹤数株衰柳语山莺。诚然是千年古迹万载仙踪。

碧梧鸣彩凤活水隐苍龙。曲径荜萝垂挂石梯藤葛攀笼。猿啸翠岩忻月上鸟啼高树喜晴空。两林竹荫凉如雨一径花浓没绣绒。时见白云来远岫略无定体漫随风。行者上前叫:“牛大哥开门!开门!”呀的一声洞门开了里边走出一个毛儿女手中提着花篮肩上担着锄子真个是一身蓝缕无妆饰满面精神有道心。行者上前迎着合掌道:“女童累你转报公主一声。我本是取经的和尚在西方路上难过火焰山特来拜借芭蕉扇一用。”那毛女道:“你是那寺里和尚?叫甚名字?我好与你通报。”行者道:“我是东土来的叫做孙悟空和尚。”

那毛女即便回身转于洞内对罗刹跪下道:“奶奶洞门外有个东土来的孙悟空和尚要见奶奶拜求芭蕉扇过火焰山一用。(WWW.mianhuatang.la 好看的小说)”那罗刹听见孙悟空三字便以撮盐入火火上浇油;

骨都都红生脸上恶狠狠怒心头口中骂道:“这泼猴!今日来了!”叫:“丫鬟取披挂拿兵器来!”随即取了披挂拿两口青锋宝剑整束出来。行者在洞外闪过偷看怎生打扮只见他:头裹团花手帕身穿纳锦云袍。腰间双束虎筋绦微露绣裙偏绡。凤嘴弓鞋三寸龙须膝裤金销。手提宝剑怒声高凶比月婆容貌。那罗刹出门高叫道:“孙悟空何在?”行者上前躬身施礼道:“嫂嫂老孙在此奉揖。”罗刹咄的一声道:“谁是你的嫂嫂!那个要你奉揖!”行者道:“尊府牛魔王当初曾与老孙结义乃七兄弟之亲。今闻公主是牛大哥令正安得不以嫂嫂称之!”罗刹道:“你这泼猴!既有兄弟之亲如何坑陷我子?”行者佯问道:“令郎是谁?”罗刹道:“我儿是号山枯松涧火云洞圣婴大王红孩儿被你倾了。我们正没处寻你报仇你今上门纳命我肯饶你!”行者满脸陪笑道:“嫂嫂原来不察理错怪了老孙。你令郎因是捉了师父要蒸要煮幸亏了观音菩萨收他去救出我师。他如今现在菩萨处做善财童子实受了菩萨正果不生不灭不垢不净与天地同寿日月同庚。你倒不谢老孙保命之恩返怪老孙是何道理!”罗刹道:“你这个巧嘴的泼猴!

我那儿虽不伤命再怎生得到我的跟前几时能见一面?”行者笑道:“嫂嫂要见令郎有何难处?你且把扇子借我扇息了火送我师父过去我就到南海菩萨处请他来见你就送扇子还你有何不可!那时节你看他可曾损伤一毫?如有些须之伤你也怪得有理如比旧时标致还当谢我。”罗刹道:“泼猴少要饶舌!伸过头来等我砍上几剑!若受得疼痛就借扇子与你;若忍耐不得教你早见阎君!”行者叉手向前笑道:“嫂嫂切莫多言老孙伸着光头任尊意砍上多少但没气力便罢是必借扇子用用。”那罗刹不容分说双手轮剑照行者头上乒乒乓乓砍有十数下这行者全不认真。罗刹害怕回头要走行者道:“嫂嫂那里去?快借我使使!”那罗刹道:“我的宝贝原不轻借。”行者道:“既不肯借吃你老叔一棒!”好猴王一只手扯住一只手去耳内掣出棒来幌一幌有碗来粗细。那罗刹挣脱手举剑来迎行者随又轮棒便打。两个在翠云山前不论亲情却只讲仇隙。这一场好杀:裙钗本是修成怪为子怀仇恨泼猴。行者虽然生狠怒因师路阻让娥流。先言拜借芭蕉扇不展骁雄耐性柔。罗刹无知轮剑砍猴王有意说亲由。女流怎与男儿斗到底男刚压女流。这个金箍铁棒多凶猛那个霜刃青锋甚紧稠。劈面打照头丢恨苦相持不罢休。左挡右遮施武艺前迎后架骋奇谋。却才斗到沉酣处不觉西方坠日头。罗刹忙将真扇了一扇挥动鬼神愁!那罗刹女与行者相持到晚见行者棒重却又解数周密料斗他不过即便取出芭蕉扇幌一幌一扇阴风把行者扇得无影无形莫想收留得住。这罗刹得胜回归。

那大圣飘飘荡荡左沉不能落地右坠不得存身就如旋风翻败叶流水淌残花滚了一夜直至天明方才落在一座山上双手抱住一块峰石。定性良久仔细观看却才认得是小须弥山。大圣长叹一声道:“好利害妇人!怎么就把老孙送到这里来了?我当年曾记得在此处告求灵吉菩萨降黄风怪救我师父。那黄风岭至此直南上有三千余里今在西路转来乃东南方隅不知有几万里。等我下去问灵吉菩萨一个消息好回旧路。”正踌躇间又听得钟声响亮急下山坡径至禅院。那门前道人认得行者的形容即入里面报道:“前年来请菩萨去降黄风怪的那个毛脸大圣又来了。”菩萨知是悟空连忙下宝座相迎入内施礼道:“恭喜!取经来耶?”悟空答道:“正好未到!早哩早哩!”灵吉道:“既未曾得到雷音何以回顾荒山?”行者道:

“自上年蒙盛情降了黄风怪一路上不知历过多少苦楚。今到火焰山不能前进询问土人说有个铁扇仙芭蕉扇扇得火灭老孙特去寻访原来那仙是牛魔王的妻红孩儿的母。他说我把他儿子做了观音菩萨的童子不得常见跟我为仇不肯借扇与我争斗。他见我的棒重难撑遂将扇子把我一扇扇得我悠悠荡荡直至于此方才落住。故此轻造禅院问个归路此处到火焰山不知有多少里数?”灵吉笑道:“那妇人唤名罗刹女又叫做铁扇公主。他的那芭蕉扇本是昆仑山后自混沌开辟以来天地产成的一个灵宝乃太阳之精叶故能灭火气。

假若扇着人要飘八万四千里方息阴风。我这山到火焰山只有五万余里此还是大圣有留云之能故止住了。若是凡人正好不得住也。”行者道:“利害利害!我师父却怎生得度那方?”

灵吉道:“大圣放心此一来也是唐僧的缘法合教大圣成功。”行者道:“怎见成功?”灵吉道:“我当年受如来教旨赐我一粒定风丹一柄飞龙杖。飞龙杖已降了风魔这定风丹尚未曾见用如今送了大圣管教那厮扇你不动你却要了扇子扇息火却不就立此功也?”行者低头作礼感谢不尽。那菩萨即于衣袖中取出一个锦袋儿将那一粒定风丹与行者安在衣领里边将针线紧紧缝了送行者出门道:“不及留款往西北上去就是罗刹的山场也。”

行者辞了灵吉驾筋斗云径返翠云山顷刻而至使铁棒打着洞门叫道:“开门!开门!老孙来借扇子使使哩!”慌得那门里女童即忙来报:“奶奶借扇子的又来了!”罗刹闻言心中悚惧道:“这泼猴真有本事!我的宝贝扇着人要去八万四千里方能停止他怎么才吹去就回来也?这番等我一连扇他两三扇教他找不着归路!”急纵身结束整齐双手提剑走出门来道:“孙行者!你不怕我又来寻死!”行者笑道:“嫂嫂勿得悭吝是必借我使使。保得唐僧过山就送还你。我是个志诚有余的君子不是那借物不还的小人。”罗刹又骂道:“泼猢狲!好没道理没分晓!夺子之仇尚未报得:借扇之意岂得如心!你不要走!吃我老娘一剑!”大圣公然不惧使铁棒劈手相迎。他两个往往来来战经五七回合罗刹女手软难轮孙行者身强善敌。他见事势不谐即取扇子望行者扇了一扇行者巍然不动。行者收了铁棒笑吟吟的道:“这番不比那番!任你怎么搧来老孙若动一动就不算汉子!”那罗刹又搧两搧。果然不动。

罗刹慌了急收宝贝转回走入洞里将门紧紧关上。

行者见他闭了门却就弄个手段拆开衣领把定风丹噙在口中摇身一变变作一个蟭蟟虫儿从他门隙处钻进。只见罗刹叫道:“渴了!渴了!快拿茶来!”近侍女童即将香茶一壶沙沙的满斟一碗冲起茶沫漕漕。行者见了欢喜嘤的一翅飞在茶沫之下。那罗刹渴极接过茶两三气都喝了。行者已到他肚腹之内现原身厉声高叫道:“嫂嫂借扇子我使使!”罗刹大惊失色叫:“小的们关了前门否?”俱说:“关了。”他又说:

“既关了门孙行者如何在家里叫唤?”女童道:“在你身上叫哩。”罗刹道:“孙行者你在那里弄术哩?”行者道:“老孙一生不会弄术都是些真手段实本事已在尊嫂尊腹之内耍子已见其肺肝矣。我知你也饥渴了我先送你个坐碗儿解渴!”却就把脚往下一登。那罗刹小腹之中疼痛难禁坐于地下叫苦。行者道:“嫂嫂休得推辞我再送你个点心充饥!”又把头往上一顶。那罗刹心痛难禁只在地上打滚疼得他面黄唇白只叫“孙叔叔饶命!”行者却才收了手脚道:“你才认得叔叔么?我看牛大哥情上且饶你性命快将扇子拿来我使使。”罗刹道:“叔叔有扇!有扇!你出来拿了去!”行者道:“拿扇子我看了出来。”罗刹即叫女童拿一柄芭蕉扇执在旁边。行者探到喉咙之上见了道:“嫂嫂我既饶你性命不在腰肋之下搠个窟窿出来还自口出。你把口张三张儿。”那罗刹果张开口。行者还作个蟭蟟虫先飞出来丁在芭蕉扇上。那罗刹不知连张三次叫:“叔叔出来罢。”行者化原身拿了扇子叫道:“我在此间不是?谢借了!谢借了!”拽开步往前便走小的们连忙开了门放他出洞。

这大圣拨转云头径回东路霎时按落云头立在红砖壁下。八戒见了欢喜道:“师父师兄来了!来了!”三藏即与本庄老者同沙僧出门接着同至舍内。把芭蕉扇靠在旁边道:“老官儿可是这个扇子?”老者道:“正是!正是!”唐僧喜道:“贤徒有莫大之功求此宝贝甚劳苦了。”行者道:“劳苦倒也不说。那铁扇仙你道是谁?那厮原来是牛魔王的妻红孩儿的母名唤罗刹女又唤铁扇公主。我寻到洞外借扇他就与我讲起仇隙把我砍了几剑。是我使棒吓他他就把扇子扇了我一下飘飘荡荡直刮到小须弥山。幸见灵吉菩萨送了我一粒定风丹指与归路复至翠云山。又见罗刹女罗刹女又使扇子搧我不动他就回洞。是老孙变作一个蟭蟟虫飞入洞去。那厮正讨茶吃是我又钻在茶沫之下到他肚里做起手脚。他疼痛难禁不住口的叫我做叔叔饶命情愿将扇借与我我却饶了他拿将扇来待过了火焰山仍送还他。”三藏闻言感谢不尽师徒们俱拜辞老者。

一路西来约行有四十里远近渐渐酷热蒸人。沙僧只叫:

“脚底烙得慌!”八戒又道:“爪子烫得痛!”马比寻常又快只因地热难停十分难进。行者道:“师父且请下马兄弟们莫走等我搧息了火待风雨之后地土冷些再过山去。”行者果举扇径至火边尽力一扇那山上火光烘烘腾起再一扇更着百倍又一扇那火足有千丈之高渐渐烧着身体。行者急回已将两股毫毛烧净径跑至唐僧面前叫:“快回去快回去!火来了火来了!”那师父爬上马与八戒沙僧复东来有二十余里方才歇下道:“悟空如何了呀!”行者丢下扇子道:“不停当!不停当!被那厮哄了!”三藏听说愁促眉尖闷添心上止不住两泪交流只道:“怎生是好!”八戒道:“哥哥你急急忙忙叫回去是怎么说?”行者道:“我将扇子搧了一下火光烘烘;第二扇火气愈盛;第三扇火头飞有千丈之高。若是跑得不快把毫毛都烧尽矣!”八戒笑道:“你常说雷打不伤火烧不损如今何又怕火?”行者道:“你这呆子全不知事!那时节用心防备故此不伤;今日只为搧息火光不曾捻避火诀又未使护身法所以把两股毫毛烧了。”沙僧道:“似这般火盛无路通西怎生是好?”八戒道:“只拣无火处走便罢。”三藏道:“那方无火?”八戒道:“东方南方北方俱无火。”又问:“那方有经?”八戒道:“西方有经。”三藏道:“我只欲往有经处去哩!”沙僧道:“有经处有火无火处无经诚是进退两难!”师徒们正自胡谈乱讲只听得有人叫道:“大圣不须烦恼且来吃些斋饭再议。”四众回看时见一老人身披飘风氅头顶偃月冠手持龙头杖只踏铁靿靴后带着一个雕嘴鱼腮鬼鬼头上顶着一个铜盆盆内有些蒸饼糕糜黄粮米饭在于西路下躬身道:“我本是火焰山土地知大圣保护圣僧不能前进特献一斋。”行者道:“吃斋小可这火光几时灭得让我师父过去?”土地道:“要灭火光须求罗刹女借芭蕉扇。”行者去路旁拾起扇子道:“这不是?那火光越扇越着何也?”土地看了笑道:“此扇不是真的被他哄了。”行者道:“如何方得真的?”那土地又控背躬身微微笑道:

“若还要借真蕉扇须是寻求大力王。”毕竟不知大力王有甚缘故且听下回分解。
------------

第六十回 牛魔王罢战赴华筵 孙行者二调芭蕉扇

土地说:“大力王即牛魔王也。”行者道:“这山本是牛魔王放的火假名火焰山?”土地道:“不是不是大圣若肯赦小神之罪方敢直言。”行者道:“你有何罪?直说无妨。”土地道:“这火原是大圣放的。”行者怒道:“我在那里你这等乱谈!我可是放火之辈?”土地道:“是你也认不得我了。此间原无这座山因大圣五百年前大闹天宫时被显圣擒了压赴老君将大圣安于八卦炉内煅炼之后开鼎被你蹬倒丹炉落了几个砖来内有余火到此处化为火焰山。我本是兜率宫守炉的道人当被老君怪我失守降下此间就做了火焰山土地也。”猪八戒闻言恨道:“怪道你这等打扮!原来是道士变的土地!”行者半信不信道:“你且说早寻大力王何故?’土地道:“大力王乃罗刹女丈夫。他这向撇了罗刹现在积雷山摩云洞。有个万岁狐王那狐王死了遗下一个女儿叫做玉面公主。那公主有百万家私无人掌管二年前访着牛魔王神通广大情愿倒陪家私招赘为夫。那牛王弃了罗刹久不回顾。若大圣寻着牛王拜求来此方借得真扇。一则扇息火焰可保师父前进;二来永除火患可保此地生灵;三者赦我归天回缴老君法旨。”行者道:

“积雪山坐落何处?到彼有多少程途?”土地道:“在正南方。此间到彼有三千余里。”行者闻言即吩咐沙僧、八戒保护师父又教土地陪伴勿回随即忽的一声渺然不见。

那里消半个时辰早见一座高山凌汉。按落云头停立巅峰之上观看真是好山:高不高顶摩碧汉;大不大根扎黄泉。

山前日暖岭后风寒。山前日暖有三冬草木无知;岭后风寒见九夏冰霜不化。龙潭接涧水长流虎穴依崖花放早。水流千派似飞琼花放一心如布锦。湾环岭上湾环树扢扠石外扢扠松。真个是高的山峻的岭陡的崖深的涧香的花美的果红的藤紫的竹青的松翠的柳:八节四时颜不改千年万古色如龙。大圣看彀多时步下尖峰入深山找寻路径。正自没个消息忽见松阴下有一女子手折了一枝香兰袅袅娜娜而来。大圣闪在怪石之旁定睛观看那女子怎生模样:娇娇倾国色缓缓步移莲。貌若王嫱颜如楚女。如花解语似玉生香。

高髻堆青軃碧鸦双睛蘸绿横秋水。湘裙半露弓鞋小翠袖微舒粉腕长。说甚么暮雨朝云真个是朱唇皓齿。锦江滑腻蛾眉秀赛过文君与薛涛。那女子渐渐走近石边大圣躬身施礼缓缓而言曰:“女菩萨何往?”那女子未曾观看听得叫问却自抬头忽见大圣的相貌丑陋老大心惊欲退难退欲行难行只得战兢兢勉强答道:“你是何方来者?敢在此间问谁?”大圣沉思道:“我若说出取经求扇之事恐这厮与牛王有亲且只以假亲托意来请魔王之言而答方可。”那女子见他不语变了颜色怒声喝道:“你是何人敢来问我!”大圣躬身陪笑道:“我是翠云山来的初到贵处不知路径。敢问菩萨此间可是积雷山?”那女子道:“正是。”大圣道:“有个摩云洞坐落何处?”那女子道:“你寻那洞做甚?”大圣道:“我是翠云山芭蕉洞铁扇公主央来请牛魔王的。”那女子一听铁扇公主请牛魔王之言心中大怒彻耳根子通红泼口骂道:“这贱婢着实无知!牛王自到我家未及二载也不知送了他多少珠翠金银绫罗缎匹。年供柴月供米自自在在受用还不识羞又来请他怎的!”大圣闻言情知是玉面公主故意子掣出铁棒大喝一声道:“你这泼贱将家私买住牛王诚然是陪钱嫁汉!你倒不羞却敢骂谁!”

那女子见了唬得魄散魂飞没好步乱躧金莲战兢兢回头便走这大圣吆吆喝喝随后相跟。原来穿过松阴就是摩云洞口女子跑进去扑的把门关了。大圣却收了铁棒咳咳停步看时好所在:树林森密崖削崚嶒。薜萝阴冉冉兰蕙味馨馨。流泉漱玉穿修竹巧石知机带落英。烟霞笼远岫日月照云屏。龙吟虎啸鹤唳莺鸣。一片清幽真可爱琪花瑶草景常明。不亚天台仙洞胜如海上蓬瀛。

且不言行者这里观看景致却说那女子跑得粉汗淋淋唬得兰心吸吸径入书房里面。原来牛魔王正在那里静玩丹书这女子没好气倒在怀里抓耳挠腮放声大哭。牛王满面陪笑道:“美人休得烦恼。有甚话说?”那女子跳天索地口中骂道:

“泼魔害杀我也!”牛王笑道:“你为甚事骂我?”女子道:“我因父母无依招你护身养命。江湖中说你是条好汉你原来是个惧内的庸夫!”牛王闻说将女子抱住道:“美人我有那些不是处你且慢慢说来我与你陪礼。”女子道:“适才我在洞外闲步花阴折兰采蕙忽有一个毛脸雷公嘴的和尚猛地前来施礼把我吓了个呆挣。及定性问是何人他说是铁扇公主央他来请牛魔王的。被我说了两句他倒骂了我一场将一根棍子赶着我打。若不是走得快些几乎被他打死!这不是招你为祸?害杀我也!”牛王闻言却与他整容陪礼温存良久女子方才息气。魔王却狠道:“美人在上不敢相瞒那芭蕉洞虽是僻静却清幽自在。我山妻自幼修持也是个得道的女仙却是家门严谨内无一尺之童焉得有雷公嘴的男子央来这想是那里来的怪妖或者假绰名声至此访我等我出去看看。”好魔王拽开步出了书房上大厅取了披挂结束了拿了一条混铁棍出门高叫道:“是谁人在我这里无状?”行者在旁见他那模样与五百年前又大不同只见;头上戴一顶水磨银亮熟铁盔身上贯一副绒穿锦绣黄金甲足下踏一双卷尖粉底麂皮靴腰间束一条攒丝三股狮蛮带。一双眼光如明镜两道眉艳似红霓。口若血盆齿排铜板。吼声响震山神怕行动威风恶鬼慌。

四海有名称混世西方大力号魔王。这大圣整衣上前深深的唱个大喏道:“长兄还认得小弟么?”牛王答礼道:“你是齐天大圣孙悟空么?”大圣道:“正是正是一向久别未拜。适才到此问一女子方得见兄丰采果胜常真可贺也!”牛王喝道:

“且休巧舌!我闻你闹了天宫被佛祖降压在五行山下近解脱天灾保护唐僧西天见佛求经怎么在号山枯松涧火云洞把我小儿牛圣婴害了?正在这里恼你你却怎么又来寻我?”大圣作礼道:“长兄勿得误怪小弟。当时令郎捉住吾师要食其肉小弟近他不得幸观音菩萨欲救我师劝他归正。现今做了善财童子比兄长还高享极乐之门堂受逍遥之永寿有何不可返怪我耶?”牛王骂道:“这个乖嘴的猢狲!害子之情被你说过你才欺我爱妾打上我门何也?”大圣笑道:“我因拜谒长兄不见向那女子拜问不知就是二嫂嫂;因他骂了我几句是小弟一时粗卤惊了嫂嫂。望长兄宽恕宽恕!”牛王道:“既如此说我看故旧之情饶你去罢。”大圣道:“既蒙宽恩感谢不尽但尚有一事奉渎万望周济周济。”牛王骂道:“这猢狲不识起倒!饶了你倒还不走反来缠我!甚么周济周济!”大圣道:

“实不瞒长兄小弟因保唐僧西进路阻火焰山不能前进。询问土人知尊嫂罗刹女有一柄芭蒲扇欲求一用。昨到旧府奉拜嫂嫂嫂嫂坚执不借是以特求长兄。望兄长开天地之心同小弟到大嫂处一行千万借扇扇灭火焰保得唐僧过山即时完璧。”牛王闻言心如火咬响钢牙骂道:“你说你不无礼你原来是借扇之故!一定先欺我山妻山妻想是不肯故来寻我!且又赶我爱妾!常言道朋友妻不可欺;朋友妾不可灭。mianhuatang.la [棉花糖小说网]

你既欺我妻又灭我妾多大无礼?上来吃我一棍!”大圣道:

“哥要说打弟也不惧但求宝贝是我真心万乞借我使使!”

牛王道:“你若三合敌得我我着山妻借你;如敌不过打死你与我雪恨!”大圣道:“哥说得是小弟这一向疏懒不曾与兄相会不知这几年武艺比昔日如何我兄弟们请演演棍看。”这牛王那容分说掣混铁棍劈头就打。这大圣持金箍棒随手相迎。

两个这场好斗:金箍棒混铁棍变脸不以朋友论。那个说:“正怪你这猢狲害子情!”这个说:“你令郎已得道休嗔恨!”那个说:“你无知怎敢上我门?”这个说:“我有因特地来相问。”一个要求扇子保唐僧一个不借芭蕉忒鄙吝。语去言来失旧情举家无义皆生忿。牛王棍起赛蛟龙大圣棒迎神鬼遁。初时争斗在山前后来齐驾祥云进。半空之内显神通五彩光中施妙运。

两条棍响振天关不见输赢皆傍寸。这大圣与那牛王斗经百十回合不分胜负。正在难解难分之际只听得山峰上有人叫道:

“牛爷爷我大王多多拜上幸赐早临好安座也。”牛王闻说使混铁棍支住金箍棒叫道:“猢狲你且住了等我去一个朋友家赴会来者!”言毕按下云头径至洞里。对玉面公主道:

“美人才那雷公嘴的男子乃孙悟空猢狲被我一顿棍打走了再不敢来你放心耍子。我到一个朋友处吃酒去也。”他才卸了盔甲穿一领鸦青剪绒袄子走出门跨上辟水金睛兽着小的们看守门庭半云半雾一直向西北方而去。

大圣在高峰上看着心中暗想道:“这老牛不知又结识了甚么朋友往那里去赴会等老孙跟他走走。”好行者将身幌一幌变作一阵清风赶上随着同走。不多时到了一座山中那牛王寂然不见。大圣聚了原身入山寻看那山中有一面清水深潭潭边有一座石碣碣上有六个大字乃乱石山碧波潭。

大圣暗想道:“老牛断然下水去了。水底之精若不是蛟精必是龙精鱼精或是龟鳖鼋鼍之精等老孙也下去看看。

好大圣捻着诀念个咒语摇身一变变作一个螃蟹不大不小的有三十六斤重扑的跳在水中径沉潭底。忽见一座玲珑剔透的牌楼楼下拴着那个辟水金睛兽进牌楼里面却就没水。大圣爬进去仔细看时只见那壁厢一派音乐之声但见:朱宫贝阙与世不殊。黄金为屋瓦白玉作门枢。屏开玳瑁甲槛砌珊瑚珠。祥云瑞蔼辉莲座上接三光下八衢。非是天宫并海藏果然此处赛蓬壶。高堂设宴罗宾主大小官员冠冕珠。忙呼玉女捧牙槃催唤仙娥调律吕。长鲸鸣巨蟹舞鳖吹笙鼍击鼓骊颔之珠照樽俎。鸟篆之文列翠屏虾须之帘挂廊庑。八音迭奏杂仙韶宫商响彻遏云霄。青头鲈妓抚瑶瑟红眼马郎品玉箫。鳜婆顶献香獐脯龙女头簪金凤翘。吃的是天厨八宝珍羞味;饮的是紫府琼浆熟酝醪。那上面坐的是牛魔王左右有三四个蛟精前面坐着一个老龙精两边乃龙子龙孙龙婆龙女。正在那里觥筹交错之际孙大圣一直走将上去被老龙看见即命:“拿下那个野蟹来!”龙子龙孙一拥上前把大圣拿住。大圣忽作人言只叫:“饶命!饶命!”老龙道:

“你是那里来的野蟹?怎么敢上厅堂在尊客之前横行乱走?

快早供来免汝死罪!”好大圣假捏虚言对众供道:“生自湖中为活傍崖作窟权居。盖因日久得身舒官受横行介士。踏草拖泥落索从来未习行仪。不知法度冒王威伏望尊慈恕罪!”座上众精闻言都拱身对老龙作礼道:“蟹介士初入瑶宫不知王礼望尊公饶他去罢。”老龙称谢了。众精即教:“放了那厮且记打外面伺候。”大圣应了一声往外逃命径至牌楼之下心中暗想道:“这牛王在此贪杯那里等得他散?就是散了也不肯借扇与我。不如偷了他的金睛兽变做牛魔王去哄那罗刹女骗他扇子送我师父过山为妙。”

好大圣即现本象将金睛兽解了缰绳扑一把跨上雕鞍径直骑出水底。到于潭外将身变作牛王模样打着兽纵着云不多时已至翠云山芭蕉洞口叫声“开门!”那洞门里有两个女童闻得声音开了门看见是牛魔王嘴脸即入报:“奶奶爷爷来家了。”那罗刹听言忙整云鬟急移莲步出门迎接。这大圣下雕鞍牵进金睛兽;弄大胆诓骗女佳人。罗刹女肉眼认他不出即携手而入。着丫鬟设座看茶一家子见是主公无不敬谨。须臾间叙及寒温。“牛王”道:“夫人久阔。”罗刹道:

“大王万福。”又云:“大王宠幸新婚抛撇奴家今日是那阵风儿吹你来的?’大圣笑道:“非敢抛撇只因玉面公主招后家事繁冗朋友多顾是以稽留在外却也又治得一个家当了。”又道:“近闻悟空那厮保唐僧将近火焰山界恐他来问你借扇子。我恨那厮害子之仇未报但来时可差人报我等我拿他分尸万段以雪我夫妻之恨。”罗刹闻言滴泪告道:“大王常言说男儿无妇财无主女子无夫身无主。我的性命险些儿不着这猢狲害了!”大圣听得故意怒骂道:“那泼猴几时过去了?”罗刹道:“还未去昨日到我这里借扇子我因他害孩儿之故披挂了轮宝剑出门就砍那猢狲。他忍着疼叫我做嫂嫂说大王曾与他结义。”大圣道:“是五百年前曾拜为七兄弟。”罗刹道:“被我骂也不敢回言砍也不敢动手后被我一扇子扇去;不知在那里寻得个定风法儿今早又在门外叫唤。是我又使扇扇莫想得动。急轮剑砍时他就不让我了。我怕他棒重就走入洞里紧关上门。不知他又从何处钻在我肚腹之内险被他害了性命!是我叫他几声叔叔将扇与他去也。”大圣又假意捶胸道:“可惜可惜!夫人错了怎么就把这宝贝与那猢狲?

恼杀我也!”罗刹笑道:“大王息怒。与他的是假扇但哄他去了。”大圣问:“真扇在于何处?”罗刹道:“放心放心!我收着哩。”叫丫鬟整酒接风贺喜遂擎杯奉上道:“大王燕尔新婚千万莫忘结且吃一杯乡中之水。”大圣不敢不接只得笑吟吟举觞在手道:“夫人先饱我因图治外产久别夫人早晚蒙护守家门权为酬谢。”罗刹复接杯斟起递与大王道:“自古道妻者齐也夫乃养身之父讲甚么谢。”两人谦谦讲讲方才坐下巡酒。大圣不敢破荤只吃几个果子与他言言语语。

酒至数巡罗刹觉有半酣色*情微动就和孙大圣挨挨擦擦搭搭拈拈携着手俏语温存并着肩低声俯就。将一杯酒你喝一口我喝一口却又哺果。大圣假意虚情相陪相笑没奈何也与他相倚相偎。果然是:钓诗钩扫愁帚破除万事无过酒。男儿立节放襟怀女子忘情开笑口。面赤似夭桃身摇如嫩柳。絮絮叨叨话语多捻捻掐掐风情有。时见掠云鬟又见轮尖手。几番常把脚儿跷数次每将衣袖抖。粉项自然低蛮腰渐觉扭。合欢言语不曾丢酥胸半露松金钮。醉来真个玉山颓饧眼摩娑几弄丑。大圣见他这等酣然暗自留心挑斗道:“夫人真扇子你收在那里?早晚仔细。但恐孙行者变化多端却又来骗去。”罗刹笑嘻嘻的口中吐出只有一个杏叶儿大小递与大圣道:“这个不是宝贝?”大圣接在手中却又不信暗想着:“这些些儿怎生扇得火灭?怕又是假的。”罗刹见他看着宝贝沉思忍不住上前将粉面揾在行者脸上叫道:

“亲亲你收了宝贝吃酒罢只管出神想甚么哩?”大圣就趁脚儿跷问他一句道:“这般小小之物如何扇得八百里火焰?”罗刹酒陶真性无忌惮就说出方法道:“大王与你别了二载你想是昼夜贪欢被那玉面公主弄伤了神思怎么自家的宝贝事情也都忘了?只将左手大指头捻着那柄儿上第七缕红丝念一声哃嘘呵吸嘻吹呼即长一丈二尺长短。这宝贝变化无穷!

那怕他八万里火焰可一扇而消也。”大圣闻言切切记在心上却把扇儿也噙在口里把脸抹一抹现了本象厉声高叫道:“罗刹女!你看看我可是你亲老公!就把我缠了这许多丑勾当!不羞!不羞!”那女子一见是孙行者慌得推倒桌席跌落尘埃羞愧无比只叫“气杀我也!气杀我也!”

这大圣不管他死活捽脱手拽大步径出了芭蕉洞正是无心贪美色得意笑颜回。将身一纵踏祥云跳上高山将扇子吐出来演演方法。将左手大指头捻着那柄上第七缕红丝念了一声哃嘘呵吸嘻吹呼果然长了有一丈二尺长短。拿在手中仔细看了又看比前番假的果是不同只见祥光幌幌瑞气纷纷上有三十六缕红丝穿经度络表里相联。原来行者只讨了个长的方法不曾讨他个小的口诀左右只是那等长短。没奈何只得搴在肩上找旧路而回不题。

却说那牛魔王在碧波潭底与众精散了筵席出得门来不见了辟水金睛兽。老龙王聚众精问道:“是谁偷放牛爷的金睛兽也?”众精跪下道:“没人敢偷我等俱在筵前供酒捧盘供唱奏乐更无一人在前。”老龙道:“家乐儿断乎不敢可曾有甚生人进来?”龙子龙孙道:“适才安座之时有个蟹精到此那个便是生人。”牛王闻说顿然省悟道:“不消讲了!早间贤友着人邀我时有个孙悟空保唐僧取经路遇火焰山难过曾问我求借芭蕉扇。我不曾与他他和我赌斗一场未分胜负我却丢了他径赴盛会。那猴子千般伶俐万样机关断乎是那厮变作蟹精来此打探消息偷了我兽去山妻处骗了那一把芭蕉扇儿也!”众精见说一个个胆战心惊问道:“可是那大闹天宫的孙悟空么?”牛王道:“正是。列公若在西天路上有不是处切要躲避他些儿。”老龙道:“似这般说大王的骏骑却如之何?”牛王笑道:“不妨不妨列公各散等我赶他去来。”遂而分开水路跳出潭底驾黄云径至翠云山芭蕉洞只听得罗刹女跌脚捶胸大呼小叫推开门又见辟水金睛兽拴在下边牛王高叫:“夫人孙悟空那厢去了?”众女童看见牛魔一齐跪下道:

“爷爷来了?”罗刹女扯住牛王磕头撞脑口里骂道:“泼老天杀的!怎样这般不谨慎着那猢狲偷了金睛兽变作你的模样到此骗我!”牛王切齿道:“猢狲那厢去了?”罗刹捶着胸膛骂道:“那泼猴赚了我的宝贝现出原身走了!气杀我也!”牛王道:“夫人保重勿得心焦等我赶上猢狲夺了宝贝剥了他皮铿碎他骨摆出他的心肝与你出气!”叫:“拿兵器来!”女童道:“爷爷的兵器不在这里。”牛王道:“拿你***兵器来罢!”侍婢将两把青锋宝剑捧出。牛王脱了那赴宴的鸦青绒袄束一束贴身的小衣双手绰剑走出芭蕉洞径奔火焰山上赶来。正是那:忘恩汉骗了痴心妇;烈性魔来近木叉人。毕竟不知此去吉凶如何且听下回分解。
------------

第六十一回 猪八戒助力败魔王 孙行者三调芭蕉扇

话表牛魔王赶上孙大圣只见他肩膊上掮着那柄芭蕉扇怡颜悦色而行。魔王大惊道:“猢狲原来把运用的方法儿也叨餂得来了。我若当面问他索取他定然不与。倘若扇我一扇要去十万八千里远却不遂了他意?我闻得唐僧在那大路上等候。他二徒弟猪精三徒弟沙流精我当年做妖怪时也曾会他且变作猪精的模样返骗他一场。料猢狲以得意为喜必不详细提防。”好魔王他也有七十二变武艺也与大圣一般只是身子狼犺些欠钻疾不活达些;把宝剑藏了念个咒语摇身一变即变作八戒一般嘴脸抄下路当面迎着大圣叫道:

“师兄我来也!”这大圣果然欢喜古人云得胜的猫儿欢似虎也只倚着强能更不察来人的意思见是个八戒的模样便就叫道:“兄弟你往那里去?”牛魔王绰着经儿道:“师父见你许久不回恐牛魔王手段大你斗他不过难得他的宝贝教我来迎你的。”行者笑道:“不必费心我已得了手了。”牛王又问道:

“你怎么得的?”行者道:“那老牛与我战经百十合不分胜负。

他就撇了我去那乱石山碧波潭底与一伙蛟精龙精饮酒。是我暗跟他去变作个螃蟹偷了他所骑的辟水金睛兽变了老牛的模样径至芭蕉洞哄那罗刹女。那女子与老孙结了一场干夫妻是老孙设法骗将来的。”牛王道:“却是生受了哥哥劳碌太甚可把扇子我拿。”孙大圣那知真假也虑不及此遂将扇子递与他。

原来那牛王他知那扇子收放的根本接过手不知捻个甚么诀儿依然小似一片杏叶现出本象开言骂道:“泼猢狲!

认得我么?”行者见了心中自悔道:“是我的不是了!”恨了一声跌足高呼道:“咦!逐年家打雁今却被小雁儿鹐了眼睛。”

狠得他爆躁如雷掣铁棒劈头便打那魔王就使扇子搧他一下不知那大圣先前变蟭蟟虫入罗刹女腹中之时将定风丹噙在口里不觉的咽下肚里所以五脏皆牢皮骨皆固凭他怎么搧再也搧他不动。牛王慌了把宝贝丢入口中双手轮剑就砍。那两个在那半空中这一场好杀:齐天孙大圣混世泼牛王只为芭蕉扇相逢各骋强。粗心大圣将人骗大胆牛王把扇诓。

这一个金箍棒起无情义;那一个双刃青锋有智量。大圣施威喷彩雾牛王放泼吐毫光。齐斗勇两不良咬牙锉齿气昂昂。

播土扬尘天地暗飞砂走石鬼神藏。这个说:“你敢无知返骗我!”那个说:“我妻许你共相将!”言村语泼性烈情刚。那个说:“你哄人妻女真该死!告到官司有罪殃!”伶俐的齐天圣凶顽的大力王一心只要杀更不待商量。棒打剑迎齐努力有些松慢见阎王。

且不说他两个相斗难分却表唐僧坐在途中一则火气蒸人二来心焦口渴对火焰山土地道:“敢问尊神那牛魔王法力如何?”土地道:“那牛王神通不小法力无边正是孙大圣的敌手。”三藏道:“悟空是个会走路的往常家二千里路一霎时便回怎么如今去了一日?断是与那牛王赌斗。”叫:“悟能悟净!你两个那一个去迎你师兄一迎?倘或遇敌就当用力相助求得扇子来解我烦躁早早过山赶路去也。”八戒道:“今日天晚我想着要去接他但只是不认得积雷山路。”土地道:

“小神认得。且教卷帘将军与你师父做伴我与你去来。”三藏大喜道:“有劳尊神功成再谢。”

那八戒抖擞精神束一束皂锦直裰搴着钯即与土地纵起云雾径回东方而去。正行时忽听得喊杀声高狂风滚滚。

八戒按住云头看时原来孙行者与牛王厮杀哩。土地道:“天蓬还不上前怎的?”呆子掣钉钯厉声高叫道:“师兄我来也!”行者恨道:“你这夯货误了我多少大事!”八戒道:“师父教我来迎你因认不得山路商议良久教土地引我故此来迟;如何误了大事?”行者道:“不是怪你来迟这泼牛十分无礼!我向罗刹处弄得扇子来却被这厮变作你的模样口称迎我我一时欢悦转把扇子递在他手他却现了本象与老孙在此比并所以误了大事也。”八戒闻言大怒举钉钯当面骂道:“我把你这血皮胀的遭瘟!你怎敢变作你祖宗的模样骗我师兄使我兄弟不睦!”你看他没头没脸的使钉钯乱筑那牛王一则是与行者斗了一日力倦神疲;二则是见八戒的钉钯凶猛遮架不住败阵就走。只见那火焰山土地帅领阴兵当面挡住道:“大力王且住手唐三藏西天取经无神不保无天不佑三界通知十方拥护。快将芭蕉扇来搧息火焰教他无灾无障早过山去;

不然上天责你罪愆定遭诛也。”牛王道:“你这土地全不察理!那泼猴夺我子欺我妾骗我妻番番无道我恨不得囫囵吞他下肚化作大便喂狗怎么肯将宝贝借他!”说不了八戒赶上骂道:“我把你个结心癀!快拿出扇来饶你性命!”那牛王只得回头使宝剑又战八戒孙大圣举棒相帮这一场在那里好杀:成精豕作怪牛兼上偷天得道猴。禅性自来能战炼必当用土合元由。钉钯九齿尖还利宝剑双锋快更柔。铁棒卷舒为主仗土神助力结丹头。三家刑克相争竞各展雄才要运筹。

捉牛耕地金钱长唤豕归炉木气收。心不在焉何作道神常守舍要拴猴。胡乱嚷苦相求三般兵刃响搜搜。钯筑剑伤无好意金箍棒起有因由。只杀得星不光兮月不皎一天寒雾黑悠悠!那魔王奋勇争强且行且斗斗了一夜不分上下早又天明。前面是他的积雷山摩云洞口他三个与土地阴兵又喧哗振耳惊动那玉面公主唤丫鬟看是那里人嚷。只见守门小妖来报:“是我家爷爷与昨日那雷公嘴汉子并一个长嘴大耳的和尚同火焰山土地等众厮杀哩!”玉面公主听言即命外护的大小头目各执枪刀助力。前后点起七长八短有百十余口一个个卖弄精神拈枪弄棒齐告:“大王爷爷我等奉奶奶内旨特来助力也!”牛王大喜道:“来得好!来得好!”众妖一齐上前乱砍。八戒措手不及倒拽着钯败阵而走大圣纵筋斗云跳出重围众阴兵亦四散奔走。老牛得胜聚众妖归洞紧闭了洞门不题。

行者道:“这厮骁勇!自昨日申时前后与老孙战起直到今夜未定输赢却得你两个来接力。如此苦斗半日一夜他更不见劳困。才这一伙小妖却又莽壮。他将洞门紧闭不出如之奈何?”八戒道:“哥哥你昨日巳时离了师父怎么到申时才与他斗起?你那两三个时辰在那里的?”行者道:“别你后顷刻就到这座山上见一个女子问讯原来就是他爱妾玉面公主。被我使铁棒唬他一唬他就跑进洞叫出那牛王来。与老孙狔言狔语嚷了一会又与他交手斗了有一个时辰。正打处有人请他赴宴去了。是我跟他到那乱石山碧波潭底变作一个螃蟹探了消息偷了他辟水金睛兽假变牛王模样复至翠云山芭蕉洞骗了罗刹女哄得他扇子。出门试演试演方法把扇子弄长了只是不会收小。正掮了走处被他假变做你的嘴脸返骗了去故此耽搁两三个时辰也。”八戒道:“这正是俗语云大海里翻了豆腐船汤里来水里去。如今难得他扇子如何保得师父过山?且回去转路走他娘罢!”土地道:“大圣休焦恼天蓬莫懈怠。但说转路就是入了旁门不成个修行之类古语云行不由径岂可转走?你那师父在正路上坐着眼巴巴只望你们成功哩!”行者狠道:“正是正是呆子莫要胡谈!土地说得有理我们正要与他赌输赢弄手段等我施为地煞变。自到西方无对头牛王本是心猿变。今番正好会源流断要相持借宝扇。趁清凉息火焰打破顽空参佛面。行满升极乐天大家同赴龙华宴!”那八戒听言便生努力殷勤道:

“是是是!去去去!管甚牛王会不会木生在亥配为猪牵转牛儿归土类。申下生金本是猴无刑无克多和气。用芭蕉为水意焰火消除成既济。昼夜休离苦尽功功完赶赴盂兰会。”

他两个领着土地阴兵一齐上前使钉钯轮铁棒乒乒乓乓把一座摩云洞的前门打得粉碎。唬得那外护头目战战兢兢闯入里边报道:“大王!孙悟空率众打破前门也!”那牛王正与玉面公主备言其事懊恨孙行者哩听说打破前门十分怒急披挂拿了铁棍从里边骂出来道:“泼猢狲!你是多大个人儿敢这等上门撒泼打破我门扇?”八戒近前乱骂道:“泼老剥皮!你是个甚样人物敢量那个大小!不要走!看钯!”牛王喝道:“你这个囔糟食的夯货不见怎的!快叫那猴儿上来!”行者道:“不知好歹的盞草!我昨日还与你论兄弟今日就是仇人了!仔细吃吾一棒!”那牛王奋勇而迎。这场比前番更胜。三个英雄厮混在一处。好杀:钉钯铁棒逞神威同帅阴兵战老牺牺牲独展凶强性遍满同天法力恢。使钯筑着棍擂铁棒英雄又出奇。三般兵器叮当响隔架遮拦谁让谁?他道他为我道我夺魁。士兵为证难分解木土相煎上下随。这两个说:

“你如何不借芭蕉扇!”那一个道:“你焉敢欺心骗我妻!赶妾害儿仇未报敲门打户又惊疑!”这个说:“你仔细堤防如意棒擦着些儿就破皮!”那个说:“好生躲避钯头齿一伤九孔血淋漓!”牛魔不怕施威猛铁棍高擎有见机。翻云覆雨随来往吐雾喷风任挥。恨苦这场都拚命各怀恶念喜相持。丢架子让高低前迎后挡总无亏。兄弟二人齐努力单身一棍独施为。

卯时战到辰时后战罢牛魔束手回。他三个含死忘生又斗有百十余合。八戒起呆性仗着行者神通举钯乱筑。牛王遮架不住败阵回头就奔洞门却被土地阴兵拦住洞门喝道:

“大力王那里走!吾等在此!”那老牛不得进洞急抽身又见八戒、行者赶来慌得卸了盔甲丢了铁棍摇身一变变做一只天鹅望空飞走。行者看见笑道:“八戒!老牛去了。”那呆子漠然不知土地亦不能晓一个个东张西觑只在积雷山前后乱找。行者指道:“那空中飞的不是?”八戒道:“那是一只天鹅。”行者道:“正是老牛变的。”土地道:“既如此却怎生么?”

行者道:“你两个打进此门把群妖尽情剿除拆了他的窝巢绝了他的归路等老孙与他赌变化去。”那八戒与土地依言攻破洞门不题。

这大圣收了金箍棒捻诀念咒摇身一变变作一个海东青飕的一翅钻在云眼里倒飞下来落在天鹅身上抱住颈项嗛眼。那牛王也知是孙行者变化急忙抖抖翅变作一只黄鹰返来嗛海东青。行者又变作一个乌凤专一赶黄鹰。牛王识得又变作一只白鹤长唳一声向南飞去。行者立定抖抖翎毛又变作一只丹凤高鸣一声。那白鹤见凤是鸟王诸禽不敢妄动刷的一翅淬下山崖将身一变变作一只香獐乜乜些些在崖前吃草。行者认得也就落下翅来变作一只饿虎剪尾跑蹄要来赶獐作食。魔王慌了手脚又变作一只金钱花斑的大豹要伤饿虎。行者见了迎着风把头一幌又变作一只金眼狻猊声如霹雳铁额铜头复转身要食大豹。牛王着了急又变作一个人熊放开脚就来擒那狻猊。行者打个滚就变作一只赖象鼻似长蛇牙如竹笋撒开鼻子要去卷那人熊。牛王嘻嘻的笑了一笑现出原身一只大白牛头如峻岭眼若闪光两只角似两座铁塔牙排利刃。连头至尾有千余丈长短自蹄至背有八百丈高下对行者高叫道:“泼猢狲!你如今将奈我何?”行者也就现了原身抽出金箍棒来把腰一躬喝声叫:“长!”长得身高万丈头如泰山眼如日月口似血池牙似门扇手执一条铁棒着头就打。那牛王硬着头使角来触。这一场真个是撼岭摇山惊天动地!有诗为证诗曰:道高一尺魔千丈奇巧心猿用力降。若得火山无烈焰必须宝扇有清凉。黄婆矢志扶元老木母留情扫荡妖。和睦五行归正果炼魔涤垢上西方。他两个大展神通在半山中赌斗惊得那过往虚空一切神众与金头揭谛、六甲六丁、一十八位护教伽蓝都来围困魔王。那魔王公然不惧你看他东一头西一头直挺挺光耀耀的两只铁角往来抵触;南一撞北一撞毛森森筋暴暴的一条硬尾左右敲摇。孙大圣当面迎众多神四面打牛王急了就地一滚复本象便投芭蕉洞去。行者也收了法象与众多神随后追袭。那魔王闯入洞里闭门不出概众把一座翠云山围得水泄不通。

正都上门攻打忽听得八戒与土地阴兵嚷嚷而至。行者见了问曰:“那摩云洞事体如何?”八戒笑道:“那老牛的娘子被我一钯筑死剥开衣看原来是个玉面狸精。那伙群妖俱是些驴骡犊特、獾狐狢獐、羊虎麋鹿等类已此尽皆剿戮又将他洞府房廊放火烧了。土地说他还有一处家小住居此山故又来这里扫荡也。”行者道:“贤弟有功可喜!可喜!老孙空与那老牛赌变化未曾得胜。他变做无大不大的白牛我变了法天象地的身量正和他抵触之间幸蒙诸神下降围困多时他却复原身走进洞去矣。”八戒道:“那可是芭蕉洞么?”行者道:“正是!

正是!罗刹女正在此间。”八戒狠道:“既是这般怎么不打进去剿除那厮问他要扇子倒让他停留长智两口儿叙情!”好呆子抖擞威风举钯照门一筑忽辣的一声将那石崖连门筑倒了一边。慌得那女童忙报:“爷爷!不知甚人把前门都打坏了!”牛王方跑进去喘嘘嘘的正告诉罗刹女与孙行者夺扇子赌斗之事闻报心中大怒就口中吐出扇子递与罗刹女。罗刹女接扇在手满眼垂泪道:“大王!把这扇子送与那猢狲教他退兵去罢。”牛王道:“夫人啊物虽小而恨则深。你且坐着等我再和他比并去来。”那魔重整披挂又选两口宝剑走出门来正遇着八戒使钯筑门老牛更不打话掣剑劈脸便砍。八戒举钯迎着向后倒退了几步出门来早有大圣轮棒当头。那牛魔即驾狂风跳离洞府又都在那翠云山上相持。众多神四面围绕土地兵左右攻击。这一场又好杀哩:云迷世界雾罩乾坤。飒飒阴风砂石滚巍巍怒气海波浑。重磨剑二口复挂甲全身。结冤深似海怀恨越生嗔。你看齐天大圣因功绩不讲当年老故人。八戒施威求扇子众神护法捉牛君。牛王双手无停息左遮右挡弄精神。只杀得那过鸟难飞皆敛翅游鱼不跃尽潜鳞;鬼泣神嚎天地暗龙愁虎怕日光昏!

那牛王拚命捐躯斗经五十余合抵敌不住败了阵往北就走。早有五台山秘魔岩神通广大泼法金刚阻住道:“牛魔你往那里去!我等乃释迦牟尼佛祖差来布列天罗地网至此擒汝也!”正说间随后有大圣、八戒、众神赶来。那魔王慌转身向南走又撞着峨眉山清凉洞法力无量胜至金刚挡住喝道:“吾奉佛旨在此正要拿住你也!”牛王心慌脚软急抽身往东便走却逢着须弥山摩耳崖毗卢沙门大力金刚迎住道:“你老牛何往!我蒙如来密令教来捕获你也!”牛王又悚然而退向西就走又遇着昆仑山金霞岭不坏尊王永住金刚敌住喝道:“这厮又将安走!我领西天大雷音寺佛老亲言在此把截谁放你也!”那老牛心惊胆战悔之不及。见那四面八方都是佛兵天将真个似罗网高张不能脱命。正在仓惶之际又闻得行者帅众赶来他就驾云头望上便走。却好有托塔李天王并哪吒太子领鱼肚药叉、巨灵神将幔住空中叫道:“慢来!慢来!吾奉玉帝旨意特来此剿除你也!”牛王急了依前摇身一变还变做一只大白牛使两只铁角去触天王天王使刀来砍。随后孙行者又到哪吒太子厉声高叫:“大圣衣甲在身不能为礼。

愚父子昨日见佛如来檄奏闻玉帝言唐僧路阻火焰山孙大圣难伏牛魔王玉帝传旨特差我父王领众助力。”行者道:

“这厮神通不小!又变作这等身躯却怎奈何?”太子笑道:“大圣勿疑你看我擒他。”这太子即喝一声“变!”变得三头六臂飞身跳在牛王背上使斩妖剑望颈项上一挥不觉得把个牛头斩下。天王收刀却才与行者相见。那牛王腔子里又钻出一个头来口吐黑气眼放金光。被哪吒又砍一剑头落处又钻出一个头来。一连砍了十数剑随即长出十数个头。哪吒取出火轮儿挂在那老牛的角上便吹真火焰焰烘烘把牛王烧得张狂哮吼摇头摆尾。才要变化脱身又被托塔天王将照妖镜照住本象腾那不动无计逃生只叫“莫伤我命!情愿归顺佛家也!”哪吒道:“既惜身命快拿扇子出来!”牛王道:“扇子在我山妻处收着哩。”

哪吒见说将缚妖索子解下跨在他那颈项上一把拿住鼻头将索穿在鼻孔里用手牵来。孙行者却会聚了四大金刚、六丁六甲、护教伽蓝、托塔天王、巨灵神将并八戒、土地、阴兵簇拥着白牛回至芭蕉洞口。老牛叫道:“夫人将扇子出来救我性命!”罗刹听叫急卸了钗环脱了色服挽青丝如道姑穿缟素似比丘双手捧那柄丈二长短的芭蕉扇子走出门又见有金刚众圣与天王父子慌忙跪在地下磕头礼拜道:“望菩萨饶我夫妻之命愿将此扇奉承孙叔叔成功去也!”行者近前接了扇同大众共驾祥云径回东路。

却说那三藏与沙僧立一会坐一会盼望行者许久不回何等忧虑!忽见祥云满空瑞光满地飘飘飖飖盖众神行将近这长老害怕道:“悟净!那壁厢是谁神兵来也?”沙僧认得道:“师父啊那是四大金刚、金头揭谛、六甲六丁、护教伽蓝与过往众神。牵牛的是哪吒三太子拿镜的是托塔李天王大师兄执着芭蕉扇二师兄并土地随后其余的都是护卫神兵。”三藏听说换了毗卢帽穿了袈裟与悟净拜迎众圣称谢道:“我弟子有何德能敢劳列位尊圣临凡也!”四大金刚道:“圣僧喜了十分功行将完!吾等奉佛旨差来助汝汝当竭力修持勿得须臾怠情。”三藏叩齿叩头受身受命。

孙大圣执着扇子行近山边尽气力挥了一扇那火焰山平平息焰寂寂除光;行者喜喜欢欢又搧一扇只闻得习习潇潇清风微动;第三扇满天云漠漠细雨落霏霏。有诗为证诗曰:火焰山遥八百程火光大地有声名。火煎五漏丹难熟火燎三关道不清。时借芭蕉施雨露幸蒙天将助神功。牵牛归佛休颠劣水火相联性自平。此时三藏解燥除烦清心了意。四众皈依谢了金刚各转宝山。六丁六甲升空保护过往神祇四散天王太子牵牛径归佛地回缴。止有本山土地押着罗刹女在旁伺候。行者道:“那罗刹你不走路还立在此等甚?”罗刹跪道:“万望大圣垂慈将扇子还了我罢。”八戒喝道:“泼贱人不知高低!饶了你的性命就彀了还要讨甚么扇子我们拿过山去不会卖钱买点心吃?费了这许多精神力气又肯与你!雨蒙蒙的还不回去哩!”罗刹再拜道:“大圣原说扇息了火还我。

今此一场诚悔之晚矣。只因不倜傥致令劳师动众。我等也修成*人道只是未归正果见今真身现象归西我再不敢妄作。

愿赐本扇从立自新修身养命去也。”土地道:“大圣!趁此女深知息火之法断绝火根还他扇子小神居此苟安拯救这方生民;求些血食诚为恩便。”行者道:“我当时问着乡人说这山扇息火只收得一年五谷便又火!”如何治得除根?”罗刹道:“要是断绝火根只消连扇四十九扇永远再不了。”行者闻言执扇子使尽筋力。望山头连扇四十九扇那山上大雨淙淙果然是宝贝:有火处下雨无火处天晴。他师徒们立在这无火处不遭雨湿。坐了一夜次早才收拾马匹行李把扇子还了罗刹又道:“老孙若不与你恐人说我言而无信。你将扇子回山再休生事。看你得了人身饶你去罢!”那罗刹接了扇子。念个咒语捏做个杏叶儿噙在口里拜谢了众圣隐姓修行后来也得了正果经藏中万古流名。罗刹、土地俱感激谢恩随后相送。行者、八戒、沙僧保着三藏遂此前进真个是身体清凉足下滋润。诚所谓:坎离既济真元合水火均平大道成。毕竟不知几年才回东土且听下回分解。
------------

第六十二回 涤垢洗心惟扫塔 缚魔归正乃修身

十二时中忘不得行功百刻全收。五年十万八千周休教神水涸莫纵火光愁。水火调停无损处五行联络如钩。阴阳和合上云楼乘鸾登紫府跨鹤赴瀛洲。这一篇词牌名《临江仙》。单道唐三藏师徒四众水火既济本性清凉借得纯阴宝扇扇息燥火过山不一日行过了八百之程师徒们散诞逍遥向西而去正值秋末冬初时序见了些:野菊残英落新梅嫩蕊生。村村纳禾稼处处食香羹。平林木落远山现曲涧霜浓幽壑清。应锺气闭蛰营纯阴阳月帝玄溟盛水德舜日怜晴。

地气下降天气上升。虹藏不见影池沼渐生冰。悬崖挂索藤花败松竹凝寒色更青。四众行彀多时前又遇城池相近。唐僧勒住马叫徒弟:“悟空你看那厢楼阁峥嵘是个甚么去处?”

行者抬头观看乃是一座城池。真个是:龙蟠形势虎踞金城。

四垂华盖近百转紫墟平。玉石桥栏排巧兽黄金台座列贤明。

真个是神洲都会天府瑶京。万里邦畿固千年帝业隆。蛮夷拱服君恩远海岳朝元圣会盈。御阶洁净辇路清宁。酒肆歌声闹花楼喜气生。未央宫外长春树应许朝阳彩凤鸣。

行者道:“师父那座城池是一国帝王之所。”八戒笑道:

“天下府有府城县有县城怎么就见是帝王之所?”行者道:

“你不知帝王之居与府县自是不同。你看他四面有十数座门周围有百十余里楼台高耸云雾缤纷。非帝京邦国何以有此壮丽?”沙僧道:“哥哥眼明虽识得是帝王之处却唤做甚么名色?”行者道:“又无牌匾旌号何以知之?须到城中询问方可知也。”长老策马须臾到门。下马过桥进门观看只见六街三市货殖通财又见衣冠隆盛人物豪华。正行时忽见有十数个和尚一个个披枷戴锁沿门乞化着实的蓝缕不堪。三藏叹曰:“兔死狐悲物伤其类。”叫:“悟空你上前去问他一声为何这等遭罪?”行者依言即叫:“那和尚你是那寺里的?为甚事披枷戴锁?”众僧跪倒道:“爷爷我等是金光寺负屈的和尚。”行者道:“金光寺坐落何方?”众僧道:“转过隅头就是。”行者将他带在唐僧前问道:“怎生负屈你说我听。”众僧道:“爷爷不知你们是那方来的我等似有些面善。此问不敢在此奉告请到荒山具说苦楚。”长老道:“也是我们且到他那寺中去仔细询问缘由。”同至山门门上横写七个金字:“敕建护国金光寺”。师徒们进得门来观看但见那:古殿香灯冷虚廊叶扫风。凌云千尺塔养性几株松。满地落花无客过檐前蛛网任攀笼。空架鼓枉悬钟绘壁尘多彩象朦。讲座幽然僧不见禅堂静矣鸟常逢。凄凉堪叹息寂寞苦无穷。佛前虽有香炉设灰冷花残事事空。三藏心酸止不住眼中出泪。众僧们顶着枷锁将正殿推开请长老上殿拜佛。长老进殿奉上心香叩齿三咂。却转于后面见那方丈檐柱上又锁着六七个小和尚三藏甚不忍见。及到方丈众僧俱来叩头问道:“列位老爷象貌不一可是东土大唐来的么?”行者笑道:“这和尚有甚未卜先知之法?我们正是。你怎么认得?”众僧道:“爷爷我等有甚未卜先知之法只是痛负了屈苦无处分明日逐家只是叫天叫地。

想是惊动天神昨日夜间各人都得一梦说有个东土大唐来的圣僧救得我等性命庶此冤苦可伸。今日果见老爷这般异象。故认得也。”三藏闻言大喜道:“你这里是何地方?有何冤屈?”众僧跪告:“爷爷此城名唤祭赛国乃西邦大去处。当年有四夷朝贡:南月陀国北高昌国东西梁国西本钵国年年进贡美玉明珠娇妃骏马。我这里不动干戈不去征讨他那里自然拜为上邦。”三藏道:“既拜为上邦想是你这国王有道文武贤良。”众僧道:“爷爷文也不贤武也不良国君也不是有道。我这金光寺自来宝塔上祥云笼罩瑞霭高升夜放霞光万里有人曾见;昼喷彩气四国无不同瞻。故此以为天府神京四夷朝贡。只是三年之前孟秋朔日夜半子时下了一场血雨。天明时家家害怕户户生悲。众公卿奏上国王不知天公甚事见责。当时延请道士打醮和尚看经答天谢地。谁晓得我这寺里黄金宝塔污了这两年外国不来朝贡。我王欲要征伐众臣谏道:“我寺里僧人偷了塔上宝贝所以无祥云瑞霭外国不朝。”昏君更不察理那些赃官将我僧众拿了去千般拷打万样追求。当时我这里有三辈和尚前两辈已被拷打不过死了如今又捉我辈问罪枷锁。老爷在上我等怎敢欺心盗取塔中之宝!万望爷爷怜念方以类聚物以群分舍大慈大悲广施法力拯救我等性命!”

三藏闻言点头叹道:“这桩事暗昧难明。一则是朝廷失政二来是汝等有灾。既然天降血雨污了宝塔那时节何不启本奏君致令受苦?”众僧道:“爷爷我等凡人怎知天意?况前辈俱未辨得我等如何处之!”三藏道:“悟空今日甚时分了?”

行者道:“有申时前后。”三藏道:“我欲面君倒换关文奈何这众僧之事不得明白难以对君奏言。我当时离了长安在法门寺里立愿:上西方逢庙烧香遇寺拜佛见塔扫塔。今日至此遇有受屈僧人乃因宝塔之累。你与我办一把新笤帚待我沐浴了上去扫扫即看这污秽之事何如不放光之故何如访着端的方好面君奏言解救他们这苦难也。”这些枷锁的和尚听说连忙去厨房取把厨刀递与八戒道:“爷爷你将此刀打开那柱子上锁的小和尚铁锁放他去安排斋饭香汤伏侍老爷进斋沐浴。我等且上街化把新笤帚来与老爷扫塔。”八戒笑道:

“开锁有何难哉?不用刀斧教我那一位毛脸老爷他是开锁的积年。”行者真个近前使个解锁法用手一抹几把锁俱退落下。那小和尚俱跑到厨中净刷锅灶安排茶饭。三藏师徒们吃了斋渐渐天昏只见那枷锁的和尚拿了两把笤帚进来三藏甚喜。

正说处一个小和尚点了灯来请洗澡。此时满天星月光辉谯楼上更鼓齐正是那:四壁寒风起万家灯火明。六街关户牖三市闭门庭。钓艇归深树耕犁罢短绳。樵夫柯斧歇学子诵书声。三藏沐浴毕穿了小袖褊衫束了环绦足下换一双软公鞋手里拿一把新笤帚对众僧道:“你等安寝待我扫塔去来。”行者道:“塔上既被血雨所污又况日久无光恐生恶物一则夜静风寒又没个伴侣自去恐有差池老孙与你同上如何?”三藏道:“甚好!甚好!”两人各持一把先到大殿上点起琉璃灯烧了香佛前拜道:“弟子陈玄奘奉东土大唐差往灵山参见我佛如来取经今至祭赛国金光寺遇本僧言宝塔被污国王疑僧盗宝衔冤取罪上下难明。弟子竭诚扫塔望我佛威灵早示污塔之原因莫致凡夫之冤屈。”祝罢与行者开了塔门自下层望上而扫。只见这塔真是峥嵘倚汉突兀凌空。正唤做五色琉璃塔千金舍利峰。梯转如穿窟门开似出笼。宝瓶影射天边月金铎声传海上风。但见那虚檐拱斗绝顶留云。虚檐拱斗作成巧石穿花凤;绝顶留云造就浮屠绕雾龙。远眺可观千里外高登似在九霄中。(WWW.mianhuatang.la 好看的小说)层层门上琉璃灯有尘无火;步步檐前白玉栏积垢飞虫。塔心里佛座上香烟尽绝;窗棂外神面前蛛网牵蒙。炉中多鼠粪盏内少油熔。只因暗失中间宝苦杀僧人命落空。三藏心将塔扫管教重见旧时容。唐僧用帚子扫了一层又上一层。如此扫至第七层上却早二更时分。那长老渐觉困倦行者道:“困了你且坐下等老孙替你扫罢。”三藏道:“这塔是多少层数?”行者道:“怕不有十三层哩。”长老耽着劳倦道:“是必扫了方趁本愿。”又扫了三层腰酸腿痛就于十层上坐倒道:“悟空你替我把那三层扫净下来罢。”行者抖擞精神登上第十一层霎时又上到第十二层。正扫处只听得塔顶上有人言语行者道:“怪哉!怪哉!

这早晚有三更时分怎么得有人在这顶上言语?断乎是邪物也!且看看去。”

好猴王轻轻的挟着笤帚撒起衣服钻出前门踏着云头观看只见第十三层塔心里坐着两个妖精面前放一盘下饭一只碗一把壶在那里猜拳吃酒哩。行者使个神通丢了笤帚掣出金箍棒拦住塔门喝道:“好怪物!偷塔上宝贝的原来是你!”两个怪物慌了急起身拿壶拿碗乱掼被行者横铁棒拦住道:“我若打死你没人供状。”只把棒逼将去。那怪贴在壁上莫想挣扎得动口里只叫:“饶命饶命!不干我事!自有偷宝贝的在那里也。”行者使个拿法一只手抓将过来径拿下第十层塔中。报道:“师父拿住偷宝贝之贼了!”三藏正自盹睡忽闻此言又惊又喜道:“是那里拿来的?”行者把怪物揪到面前跪下道:“他在塔顶上猜拳吃酒耍子是老孙听得喧哗一纵云跳到顶上拦住未曾着力。但恐一棒打死没人供状故此轻轻捉来。师父可取他个口词看他是那里妖精偷的宝贝在于何处。”那怪物战战兢兢口叫“饶命!”遂从实供道:“我两个是乱石山碧波潭万圣龙王差来巡塔的。他叫做奔波儿灞我叫做灞波儿奔。他是鲇鱼怪我是黑鱼精。因我万圣老龙生了一个女儿就唤做万圣公主。那公主花容月貌有二十分人才招得一个驸马唤做九头驸马神通广大。前年与龙王来此显大法力下了一阵血雨污了宝塔偷了塔中的舍利子佛宝。公主又去大罗天上灵霄殿前偷了王母娘娘的九叶灵芝草养在那潭底下金光霞彩昼夜光明。近日闻得有个孙悟空往西天取经说他神通广大沿路上专一寻人的不是所以这些时常差我等来此巡拦若还有那孙悟空到时好准备也。”行者闻言嘻嘻冷笑道:“那孽畜等这等无礼怪道前日请牛魔王在那里赴会!原来他结交这伙泼魔专干不良之事!”

说未了只见八戒与两三个小和尚自塔下提着两个灯笼走上来道:“师父扫了塔不去睡觉在这里讲甚么哩?”行者道:“师弟你来正好。塔上的宝贝乃是万圣老龙偷了去。今着这两个小妖巡塔探听我等来的消息却才被我拿住也。”八戒道:“叫做甚么名字甚么妖精?”行者道:“才然供了口词一个叫做奔波儿灞一个叫做灞波儿奔;一个是鲇鱼怪一个是黑鱼精。”八戒掣钯就打道:“既是妖精取了口词不打死何待?”行者道:“你不知且留着活的好去见皇帝讲话又好做凿眼去寻贼追宝。”好呆子真个收了钯一家一个都抓下塔来。那怪只叫:“饶命!”八戒道:“正要你鲇鱼黑鱼做些鲜汤与那负冤屈的和尚吃哩!”两三个小和尚喜喜欢欢提着灯笼引长老下了塔。一个先跑报众僧道:“好了!好了!我们得见青天了!偷宝贝的妖怪已是爷爷们捉将来矣!”行者教:“拿铁索来穿了琵琶骨锁在这里。汝等看守我们睡觉去明日再做理会”那些和尚都紧紧的守着让三藏们安寝。

不觉的天晓长老道:“我与悟空入朝倒换关文去来。”长老即穿了锦襕袈裟戴了毗卢帽整束威仪拽步前进。行者也束一束虎皮裙整一整绵布直裰取了关文同去。八戒道:“怎么不带这两个妖贼?”行者道:“待我们奏过了自有驾帖着人来提他。”遂行至朝门外看不尽那朱雀黄龙清都绛阙。三藏到东华门对阁门大使作礼道:“烦大人转奏贫僧是东土大唐差去西天取经者意欲面君倒换关文。”那黄门官果与通报至阶前奏道:“外面有两个异容异服僧人称言南赡部洲东土唐朝差往西方拜佛求经欲朝我王倒换关文。”国王闻言传旨教宣长老即引行者入朝。文武百官见了行者无不惊怕有的说是猴和尚有的说是雷公嘴和尚个个悚然不敢久视。

长老在阶前舞蹈山呼的行拜大圣叉着手斜立在旁公然不动。长老启奏道:“臣僧乃南赡部洲东土大唐国差来拜西方天竺国大雷音寺佛求取真经者路经宝方不敢擅过有随身关文乞倒验方行。”那国王闻言大喜。传旨教宣唐朝圣僧上金銮殿安绣墩赐坐。长老独自上殿先将关文捧上然后谢恩敢坐。那国王将关文看了一遍心中喜悦道:“似你大唐王有疾能选高僧不避路途遥远拜我佛取经;寡人这里和尚专心只是做贼败国倾君!”三藏闻言合掌道:“怎见得败国倾君?”国王道:“寡人这国乃是西域上邦常有四夷朝贡皆因国内有个金光寺寺内有座黄金宝塔塔上有光彩冲天近被本寺贼僧暗窃了其中之宝三年无有光彩外国这二年也不来朝寡人心痛恨之。”三藏合掌笑道:“万岁差之毫厘失之千里矣。

贫僧昨晚到于天府一进城门就见十数个枷纽之僧。问及何罪他道是金光寺负冤屈者。因到寺细审更不干本寺僧人之事。贫僧入夜扫塔已获那偷宝之妖贼矣。”国王大喜道:“妖贼安在?”三藏道:“现被小徒锁在金光寺里。”那国王急降金牌:

“着锦衣卫快到金光寺取妖贼来寡人亲审。”三藏又奏道:“万岁虽有锦衣卫还得小徒去方可。”国王道:“高徒在那里?”三藏用手指道:“那玉阶旁立者便是。”国王见了大惊道:“圣僧如此丰姿高徒怎么这等象貌?”孙大圣听见了厉声高叫道:

“陛下人不可貌相海水不可斗量。若爱丰姿者如何捉得妖贼也?”国王闻言回惊作喜道:“圣僧说的是朕这里不选人材只要获贼得宝归塔为上。”再着当驾官看车盖教锦衣卫好生伏侍圣僧去取妖贼来。那当驾官即备大轿一乘黄伞一柄锦衣卫点起校尉将行者八抬八绰大四声喝路径至金光寺。

自此惊动满城百姓无处无一人不来看圣僧及那妖贼。

八戒、沙僧听得喝道只说是国王差官急出迎接原来是行者坐在轿上。呆子当面笑道:“哥哥你得了本身也!”行者下了轿搀着八戒道:“我怎么得了本身?”八戒道:“你打着黄伞抬着八人轿却不是猴王之职分?故说你得了本身。”行者道:

“且莫取笑。”遂解下两个妖物押见国王。沙僧道:“哥哥也带挈小弟带挈。”行者道:“你只在此看守行李马匹。”那枷锁之僧道:“爷爷们都去承受皇恩等我们在此看守。”行者道:“既如此等我去奏过国王却来放你。”八戒揪着一个妖贼沙僧揪着一个妖贼孙大圣依旧坐了轿摆开头搭将两个妖怪押赴当朝。须臾至白玉阶对国王道:“那妖贼已取来了。”国王遂降龙床与唐僧及文武多官同目视之那怪一个是暴腮乌甲尖嘴利牙;一个是滑皮大肚巨口长须虽然是有足能行大抵是变成的人象。国王问曰:“你是何方贼怪那处妖精几年侵吾国土何年盗我宝贝一盘共有多少贼徒都唤做甚么名字从实一一供来!”二怪朝上跪下颈内血淋淋的更不知疼痛供道:“三载之外七月初一有个万圣龙王帅领许多亲戚住居在本国东南离此处路有百十潭号碧波山名乱石。生女多娇妖娆美色招赘一个九头驸马神通无敌。他知你塔上珍奇与龙王合盘做贼先下血雨一场后把舍利偷讫。见如今照耀龙宫纵黑夜明如白日。公主施能寂寂密密又偷了王母灵芝在潭中温养宝物。我两个不是贼头乃龙王差来小卒。今夜被擒所供是实。”国王道:“既取了供如何不供自家名字?”

那怪道:“我唤做奔波儿灞他唤做灞波儿奔奔波儿灞是个鲇鱼怪灞波儿奔是个黑鱼精。”国王教锦衣卫好生收监传旨:

“赦了金光寺众僧的枷锁快教光禄寺排宴就于麒麟殿上谢圣僧获贼之功议请圣僧捕擒贼。”

光禄寺即时备了荤素两样筵席国王请唐僧四众上麒麟殿叙坐问道:“圣僧尊号?”唐僧合掌道:“贫僧俗家姓陈法名玄奘。蒙君赐姓唐贱号三藏。”国王又问:“圣僧高徒何号?”三藏道:“小徒俱无号第一个名孙悟空第二个名猪悟能第三个名沙悟净此乃南海观世音菩萨起的名字。因拜贫僧为师贫僧又将悟空叫做行者悟能叫做八戒悟净叫做和尚。”国王听毕请三藏坐了上席孙行者坐了侧左席猪八戒沙和尚坐了侧右席俱是素果、素菜、素茶、素饭。前面一席荤的坐了国王下有百十席荤的坐了文武多官。众臣谢了君恩徒告了师罪坐定。国王把盏三藏不敢饮酒他三个各受了安席酒。下边只听得管弦齐奏乃是教坊司动乐。你看八戒放开食嗓真个是虎咽狼吞将一席果菜之类吃得罄尽。少顷间添换汤饭又来又吃得一毫不剩巡酒的来又杯杯不辞。这场筵席直乐到午后方散。三藏谢了盛宴国王又留住道:“这一席聊表圣僧获怪之功。”教光禄寺:“快翻席到建章宫里再请圣僧定捕贼取宝归塔之计。”三藏道:“既要捕贼取宝不劳再宴贫僧等就此辞王就擒捉妖怪去也。”国王不肯一定请到建章宫又吃了一席。国王举酒道:“那位圣僧帅众出师降妖捕贼?”三藏道:“教大徒弟孙悟空去。”大圣拱手应承。国王道:

“孙长老既去用多少人马?几时出城?”八戒忍不住高声叫道:

“那里用甚么人马!又那里管甚么时辰!趁如今酒醉饭饱我共师兄去手到擒来!”三藏甚喜道:“八戒这一向勤紧啊!”行者道:“既如此着沙僧弟保护师父我两个去来。”那国王道:

“二位长老既不用人马可用兵器?”八戒笑道:“你家的兵器我们用不得。我弟兄自有随身器械。”国王闻说即取大觥来与二位长老送行。孙大圣道:“酒不吃了只教锦衣卫把两个小妖拿来我们带了他去做凿眼。”国王传旨即时提出。二人挟着两个小妖驾风头使个摄法径上东南去了。噫!他那君臣一见腾风雾才识师徒是圣僧。毕竟不知此去如何擒获且听下回分解。
------------

第六十三回 二僧荡怪闹龙宫 群圣除邪获宝贝

却说祭赛国王与大小公卿见孙大圣与八戒腾云驾雾提着两个小妖飘然而去一个个朝天礼拜道:“话不虚传!今日方知有此辈神仙活佛!”又见他远去无踪却拜谢三藏、沙僧道:“寡人肉眼凡胎只知高徒有力量拿住妖贼便了岂知乃腾云驾雾之上仙也。”三藏道:“贫僧无些法力一路上多亏这三个小徒。”沙僧道:“不瞒陛下说我大师兄乃齐天大圣皈依。

他曾大闹天宫使一条金箍棒十万天兵无一个对手只闹得太上老君害怕玉皇大帝心惊。我二师兄乃天蓬元帅果正他也曾掌管天河八万水兵大众。惟我弟子无法力乃卷帘大将受戒。愚弟兄若干别事无能若说擒妖缚怪拿贼捕亡伏虎降龙踢天弄井以至搅海翻江之类略通一二。这腾云驾雾唤雨呼风与那换斗移星担山赶月特余事耳何足道哉!”国王闻说愈十分加敬请唐僧上坐口口称为老佛将沙僧等皆称为菩萨。满朝文武欣然一国黎民顶礼不题。

却说孙大圣与八戒驾着狂风把两个小妖摄到乱石山碧波潭住定云头将金箍棒吹了一口仙气叫“变!”变作一把戒刀将一个黑鱼怪割了耳朵鲇鱼精割了下唇撇在水里喝道:“快早去对那万圣龙王报知说我齐天大圣孙爷爷在此着他即送祭赛国金光寺塔上的宝贝出来免他一家性命!若迸半个不字我将这潭水搅净教他一门儿老幼遭诛!”那两个小妖得了命负痛逃生拖着锁索淬入水内唬得那些鼋鼍龟鳖虾蟹鱼精都来围住问道:“你两个为何拖绳带索?”一个掩着耳摇头摆尾一个侮着嘴跌脚捶胸;都嚷嚷闹闹径上龙王宫殿报:“大王祸事了!”那万圣龙王正与九头驸马饮酒忽见他两个来即停杯问何祸事。那两个即告道:“昨夜巡拦被唐僧、孙行者扫塔捉获用铁索拴锁。今早见国王又被那行者与猪八戒抓着我两个一个割了耳朵一个割了嘴唇抛在水中着我来报要索那塔顶宝贝。”遂将前后事细说了一遍。那老龙听说是孙行者齐天大圣唬得魂不附体魄散九霄战兢兢对驸马道:“贤婿啊别个来还好计较若果是他却不善也!”驸马笑道:“太岳放心愚婿自幼学了些武艺四海之内也曾会过几个豪杰怕他做甚!等我出去与他交战三合管取那厮缩归降不敢仰视。”

好妖怪急纵身披挂了使一般兵器叫做月牙铲步出宫分开水道在水面上叫道:“是甚么齐天大圣!快上来纳命!”行者与八戒立在岸边观看那妖精怎生打扮:戴一顶烂银盔光欺白雪;贯一副兜鍪甲亮敌秋霜。上罩着锦征袍真个是彩云笼玉;腰束着犀纹带果然象花蟒缠金。手执着月牙铲霞飞电掣;脚穿着猪皮靴水利波分。远看时一头一面近睹处四面皆人。前有眼后有眼八方通见;左也口右也口九口言论。一声吆喝长空振似鹤飞鸣贯九宸。他见无人对答又叫一声:“那个是齐天大圣?”行者按一按金箍理一理铁棒道:

“老孙便是。”那怪道:“你家居何处?身出何方!怎生得到祭赛国与那国王守塔却大胆获我头目又敢行凶上吾宝山索战?”行者骂道:“你这贼怪原来不识你孙爷爷哩!你上前听我道:老孙祖住花果山大海之间水帘洞。自幼修成不坏身玉皇封我齐天圣。只因大闹斗牛宫天上诸神难取胜。当请如来展妙高无边智慧非凡用。为翻筋斗赌神通手化为山压我重。

整到如今五百年观者劝解方逃命。大唐三藏上西天远拜灵山求佛颂。解脱吾身保护他炼魔净怪从修行。路逢西域祭赛城。屈害僧人三代命。我等慈悲问旧情乃因塔上无光映。吾师扫塔探分明夜至三更天籁静。捉住鱼精取实供他言汝等偷宝珍。合盘为盗有龙王公主连名称万圣。血雨浇淋塔上光将他宝贝偷来用。殿前供状更无虚我奉君言驰此境。所以相寻索战争不须再问孙爷姓。快将宝贝献还他免汝老少全家命。敢若无知骋胜强教你水涸山颓都蹭蹬!”那驸马闻言微微冷笑道:“你原来是取经的和尚没要紧罗织管事!我偷他的宝贝你取佛的经文与你何干却来厮斗!”行者道:“这贼怪甚不达理!我虽不受国王的恩惠不食他的水米不该与他出力。但是你偷他的宝贝污他的宝塔屡年屈苦金光寺僧人他是我一门同气我怎么不与他出力辨明冤枉?”驸马道:“你既如此想是要行赌赛。常言道武不善作但只怕起手处不得留情一时间伤了你的性命误了你去取经!”行者大怒骂道:

“这泼贼怪有甚强能敢开大口!走上来吃老爷一棒!”那驸马更不心慌把月牙铲架住铁棒就在那乱石山头这一场真个好杀妖魔盗宝塔无光行者擒妖报国王。小怪逃生回水内老龙破胆各商量。九头驸马施威武披挂前来展素强。怒齐天孙大圣金箍棒起十分刚。那怪物九个头颅十八眼前前后后放毫光;这行者一双铁臂千斤力蔼蔼纷纷并瑞祥。铲似一阳初现月棒如万里遍飞霜。他说“你无干休把不平报!”我道“你有意偷宝真不良!”那泼贼少轻狂还他宝贝得安康!棒迎铲架争高下不见输赢练战场。

他两个往往来来斗经三十余合不分胜负。猪八戒立在山前见他们战到酣美之处举着钉钯从妖精背后一筑。原来那怪九个头转转都是眼睛看得明白见八戒在背后来时即使铲鐏架着钉钯铲头抵着铁棒。又耐战五七合挡不得前后齐轮他却打个滚腾空跳起现了本象乃是一个九头虫观其形象十分恶见此身模怕杀人!他生得:毛羽铺锦团身结絮。方圆有丈二规模长短似鼋鼍样致。两只脚尖利如钩九个头攒环一处。展开翅极善飞扬纵大鹏无他力气;起声远振天涯比仙鹤还能高唳。眼多闪灼幌金光气傲不同凡鸟类。

猪八戒看见心惊道:“哥啊!我自为人也不曾见这等个恶物!

是甚血气生此禽兽也?”行者道:“真个罕有!真个罕有!等我赶上打去!”好大圣急纵祥云跳在空中使铁棒照头便打。那怪物大显身展翅斜飞飕的打个转身掠到山前半腰里又伸出一个头来张开口如血盆相似把八戒一口咬着鬃半拖半扯捉下碧波潭水内而去。及至龙宫外还变作前番模样将八戒掷之于地叫:“小的们何在?”那里面鲭鲌鲤鳜之鱼精龟鳖鼋鼍之介怪一拥齐来道声“有!”驸马道:“把这个和尚绑在那里与我巡拦的小卒报仇!”众精推推嚷嚷抬进八戒去时那老龙王欢喜迎出道:“贤婿有功怎生捉他来也?”那驸马把上项原故说了一遍老龙即命排酒贺功不题。

却说孙行者见妖精擒了八戒心中惧道:“这厮恁般利害!

我待回朝见师恐那国王笑我。待要开言骂战曾奈我又单身况水面之事不惯。且等我变化了进去看那怪把呆子怎生摆布若得便且偷他出来干事。”好大圣捻着诀摇身一变还变做一个螃蟹淬于水内径至牌楼之前。原来这条路是他前番袭牛魔王盗金睛兽走熟了的直至那宫阙之下横爬过去又见那老龙王与九头虫合家儿欢喜饮酒。行者不敢相近爬过东廊之下见几个虾精蟹精纷纷纭纭耍子。行者听了一会言谈却就学语学话问道:“驸马爷爷拿来的那长嘴和尚这会死了不曾?”众精道:“不曾死缚在那西廊下哼的不是?”行者听说又轻轻的爬过西廊真个那呆子绑在柱上哼哩。mianhuatang.la [棉花糖小说网]行者近前道:“八戒认得我么?”八戒听得声音知是行者道:“哥哥怎么了!反被这厮捉住我也!”行者四顾无人将钳咬断索子叫走那呆子脱了手道:“哥哥我的兵器被他收了又奈何?”行者道:“你可知道收在那里?”八戒道:“当被那怪拿上宫殿去了。”行者道:“你先去牌楼下等我。”八戒逃生悄悄的溜出。行者复身爬上宫殿观看左下有光彩森森乃是八戒的钉钯放光使个隐身法将钯偷出到牌楼下叫声:“八戒!接兵器!”

呆子得了钯便道:“哥哥你先走等老猪打进宫殿。若得胜就捉住他一家子;若不胜败出来你在这潭岸上救应。”行者大喜只教仔细八戒道:“不怕他!水里本事我略有些儿。”行者丢了他负出水面不题。

这八戒束了皂直裰双手缠钯一声喊打将进去。慌得那大小水族奔奔波波跑上宫殿吆喝道:“不好了!长嘴和尚挣断绳返打进来了!”那老龙与九头虫并一家子俱措手不及跳起来藏藏躲躲。这呆子不顾死活闯上宫殿一路钯筑破门扇打破桌椅把些吃酒的家火之类尽皆打碎。有诗为证诗曰:木母遭逢水怪擒心猿不舍苦相寻。暗施巧计偷开锁大显神威怒恨深。驸马忙携公主躲龙王战栗绝声音。水宫绛阙门窗损龙子龙孙尽没魂。这一场被八戒把玳瑁屏打得粉碎珊瑚树掼得凋零。那九头虫将公主安藏在内急取月牙铲赶至前宫喝道:“泼夯豕彘!怎敢欺心惊吾眷族!”八戒骂道:“这贼怪你焉敢将我捉来!这场不干我事是你请我来家打的!快拿宝贝还我回见国王了事;不然决不饶你一家命也!”那怪那肯容情咬定牙齿与八戒交锋。那老龙才定了神思领龙子龙孙各执枪刀齐来攻取。八戒见事体不谐虚幌一钯撤身便走那老龙帅众追来。须臾撺出水中都到潭面上翻腾。却说孙行者立于潭岸等候忽见他们追赶八戒出离水中就半踏云雾掣铁棒喝声“休走!”只一下把个老龙头打得稀烂。

可怜血溅潭中红水泛尸飘浪上败鳞浮!唬得那龙子龙孙各各逃命九头驸马收龙尸转宫而去。

行者与八戒且不追袭回上岸备言前事。八戒道:“这厮锐气挫了!被我那一路钯打进去时打得落花流水魂散魄飞!正与那驸马厮斗却被老龙王赶着却亏了你打死。那厮们回去一定停丧挂孝决不肯出来。今又天色晚了却怎奈何?”行者道:“管甚么天晚!乘此机会你还下去攻战务必取出宝贝方可回朝。”那呆子意懒情疏徉徉推托行者催逼道:

“兄弟不必多疑还象刚才引出来等我打他。”两人正自商量只听得狂风滚滚惨雾阴阴忽从东方径往南去。行者仔细观看乃二郎显圣领梅山六兄弟架着鹰犬挑着狐兔抬着獐鹿一个个腰挎弯弓手持利刃纵风雾踊跃而来。行者道:“八戒那是我七圣兄弟倒好留请他们与我助战。若得成功倒是一场大机会也。”八戒道:“既是兄弟极该留请。”行者道:

“但内有显圣大哥我曾受他降伏不好见他。你去拦住云头叫道:‘真君且略住住。齐天大圣在此进拜。’他若听见是我断然住了。待他安下我却好见。”那呆子急纵云头上山拦住厉声高叫道:“真君且慢车驾有齐天大圣请见哩。”那爷爷见说即传令就停住六兄弟与八戒相见毕问:“齐天大圣何在?”八戒道:“现在山下听呼唤。”二郎道:“兄弟们快去请来。”六兄弟乃是康、张、姚、李、郭、直各各出营叫道:“孙悟空哥哥大哥有请。”行者上前对众作礼遂同上山。二郎爷爷迎见携手相搀一同相见道:“大圣你去脱大难受戒沙门刻日功完高登莲座可贺!可贺!”行者道:“不敢向蒙莫大之恩未展斯须之报。虽然脱难西行未知功行何如。今因路遇祭赛国搭救僧灾在此擒妖索宝。偶见兄长车驾大胆请留一助未审兄长自何而来肯见爱否。”二郎笑道:“我因闲暇无事同众兄弟采猎而回幸蒙大圣不弃留会足感故旧之情。若命挟力降妖敢不如命!却不知此地是何怪贼?”六圣道:“大哥忘了?此间是乱石山山下乃碧波潭万圣之龙宫也。”二郎惊呀道:“万圣老龙却不生事怎么敢偷塔宝?”行者道:“他近日招了一个驸马乃是九头虫成精。他郎丈两个做贼将祭赛国下了一场血雨把金光寺塔顶舍利佛宝偷来。那国王不解其意苦拿着僧人拷打。是我师父慈悲夜来扫搭当被我在塔上拿住两个小妖是他差来巡探的。今早押赴朝中实实供招了。

那国王就请我师收降师命我等到此。先一场战被九头虫腰里伸出一个头来把八戒衔了去我却又变化下水解了八戒。

才然大战一场是我把老龙打死那厮们收尸挂孝去了。我两个正议索战却见兄长仪仗降临故此轻渎也。”二郎道:“既伤了老龙正好与他攻击使那厮不能措手却不连窝巢都灭绝了?”八戒道:“虽是如此奈天晚何?”二郎道:“兵家云征不待时何怕天晚!”康姚郭直道:“大哥莫忙那厮家眷在此料无处去。孙二哥也是贵客猪刚鬣又归了正果我们营内有随带的酒肴教小的们取火就此铺设:一则与二位贺喜二来也当叙情。且欢会这一夜待天明索战何迟?”二郎大喜道:“贤弟说得极当。”却命小校安排行者道:“列位盛情不敢固却。但自做和尚都是斋戒恐荤素不便。”二郎道:“有素果品酒也是素的。”众兄弟在星月光前幕天席地举杯叙旧。

正是寂寞更长欢娱夜短早不觉东方白。那八戒几锺酒吃得兴抖抖的道:“天将明了等老猪下水去索战也。”二郎道:“元帅仔细只要引他出来我兄弟们好下手。”八戒笑道:

“我晓得!我晓得!你看他敛衣缠钯使分水法跳将下去径至那牌楼下声喊打入殿内。此时那龙子披了麻看着龙尸哭龙孙与那驸马在后面收拾棺材哩。这八戒骂上前手起处钯头着重把个龙子夹脑连头一钯筑了九个窟窿唬得那龙婆与众往里乱跑哭道:“长嘴和尚又把我儿打死了!”那驸马闻言即使月牙铲带龙孙往外杀来。这八戒举钯迎敌且战且退跳出水中。这岸上齐天大圣与七兄弟一拥上前枪刀乱扎把个龙孙剁成几断肉饼。那驸马见不停当在山前打个滚又现了本象展开翅旋绕飞腾。二郎即取金弓安上银弹扯满弓往上就打。那怪急铩翅掠到边前要咬二郎;半腰里才伸出一个头来被那头细犬撺上去汪的一口把头血淋淋的咬将下来。那怪物负痛逃生径投北海而去。八戒便要赶去行者止住道:“且莫赶他正是穷寇勿追他被细犬咬了头必定是多死少生。等我变做他的模样你分开水路赶我进去寻那宫主诈他宝贝来也。”二郎与六圣道:“不赶他倒也罢了只是遗这种类在世必为后人之害。”至今有个九头虫滴血是遗种也。

那八戒依言分开水路行者变作怪象前走八戒吆吆喝喝后追。渐渐追至龙宫只见那万圣宫主道:“驸马怎么这等慌张?”行者道:“那八戒得胜把我赶将进来觉道不能敌他。

你快把宝贝好生藏了!”那宫主急忙难识真假即于后殿里取出一个浑金匣子来递与行者道:“这是佛宝。”又取出一个白玉匣子也递与行者道:“这是九叶灵芝。你拿这宝贝藏去等我与猪八戒斗上两三合挡住他你将宝贝收好了再出来与他合战。”行者将两个匣儿收在身边把脸一抹现了本象道:

“宫主你看我可是驸马么?”宫主慌了便要抢夺匣子被八戒跑上去着背一钯筑倒在地。还有一个老龙婆撤身就走被八戒扯住举钯才筑行者道:“且住!莫打死他留个活的好去国内见功。”遂将龙婆提出水面。行者随后捧着两个匣子上岸对二郎道:“感兄长威力得了宝贝扫净妖贼也。”二郎道:“一则是那国王洪福齐天二则是贤昆玉神通无量我何功之有!”

兄弟们俱道:“孙二哥既已功成我们就此告别。”行者感谢不尽欲留同见国王。诸公不肯遂帅众回灌口去讫。

行者捧着匣子八戒拖着龙婆半云半雾顷刻间到了国内。原来那金光寺解脱的和尚都在城外迎接忽见他两个云雾定时近前磕头礼拜接入城中。那国王与唐僧正在殿上讲论这里有先走的和尚礼仗着胆入朝门奏道:“万岁孙猪二老爷擒贼获宝而来也。”那国王听说连忙下殿共唐僧沙僧迎着称谢神功不尽随命排筵谢恩。三藏道:“且不须赐饮着小徒归了塔中之宝方可饮宴。”三藏又问行者道:“汝等昨日离国怎么今日才来?”行者把那战驸马打龙王逢真君败妖怪及变化诈宝贝之事细说了一遍。三藏与国王大小文武俱喜之不胜。国王又问:“龙婆能人言语否?”八戒道:“乃是龙王之妻生了许多龙子龙孙岂不知人言?”国王道:“既知人言快早说前后做贼之事。”龙婆道:“偷佛宝我全不知都是我那夫君龙鬼与那驸马九头虫知你塔上之光乃是佛家舍利子三年前下了血雨乘机盗去。”又问:“灵芝草是怎么偷的?”

龙婆道:“只是我小女万圣宫主私入大罗天上灵霄殿前偷的王母娘娘九叶灵芝草。那舍利子得这草的仙气温养着千年不坏万载生光去地下或田中扫一扫即有万道霞光千条瑞气。如今被你夺来弄得我夫死子绝婿丧女亡千万饶了我的命罢!”八戒道:“正不饶你哩!”行者道:“家无全犯我便饶你只便要你长远替我看塔。”龙婆道:“好死不如恶活。但留我命凭你教做甚么。”行者叫取铁索来当驾官即取铁索一条把龙婆琵琶骨穿了教沙僧:“请国王来看我们安塔去。”那国王即忙排驾遂同三藏携手出朝并文武多官随至金光寺上塔。将舍利子安在第十三层塔顶宝瓶中间把龙婆锁在塔心柱上念动真言唤出本国土地、城隍与本寺伽蓝每三日送饮食一餐与这龙婆度口少有差讹即行处斩众神暗中领诺。行者却将芝草把十三层塔层层扫过安在瓶内温养舍利子。这才是整旧如新霞光万道瑞气千条依然八方共睹四国同瞻。下了塔门国王就谢道:“不是老佛与三位菩萨到此怎生得明此事也!”行者道:“陛下金光二字不好不是久住之物:金乃流动之物光乃熌灼之气。贫僧为你劳碌这场将此寺改作伏龙寺教你永远常存。”那国王即命换了字号悬上新匾乃是“敕建护国伏龙寺”。一壁厢安排御宴一壁厢召丹青写下四众生形五凤楼注了名号。国王摆銮驾送唐僧师徒赐金玉酬答师徒们坚辞一毫不受。这真个是:邪怪剪除万境静宝塔回光大地明。毕竟不知此去前路如何且听下回分解。
------------

第六十四回 荆棘岭悟能努力 木仙庵三藏谈诗

话表祭赛国王谢了唐三藏师徒获宝擒怪之恩所赠金玉分毫不受却命当驾官照依四位常穿的衣服各做两套鞋袜各做两双绦环各做两条外备干粮烘炒倒换了通关文牒大排銮驾并文武多官满城百姓伏龙寺僧人大吹大打送四众出城。约有二十里先辞了国王。众人又送二十里辞回。伏龙寺僧人送有五六十里不回有的要同上西天有的要修行伏侍。行者见都不肯回去遂弄个手段把毫毛拔了三四十根吹口仙气叫“变!”都变作斑斓猛虎拦住前路哮吼踊跃。众僧方惧不敢前进大圣才引师父策马而去。少时间去得远了众僧人放声大哭都喊:“有恩有义的老爷!我等无缘不肯度我们也!”

且不说众僧啼哭却说师徒四众走上大路却才收回毫毛一直西去。正是时序易迁又早冬残春至不暖不寒正好逍遥行路。忽见一条长岭岭顶上是路。三藏勒马观看那岭上荆棘丫叉薜萝牵绕虽是有道路的痕迹左右却都是荆刺棘针。唐僧叫:“徒弟这路怎生走得?”行者道:“怎么走不得?”

又道:“徒弟啊路痕在下荆棘在上只除是蛇虫伏地而游方可去了。若你们走腰也难伸教我如何乘马?”八戒道:“不打紧等我使出钯柴手来把钉钯分开荆棘莫说乘马就抬轿也包你过去。”三藏道:“你虽有力长远难熬却不知有多少远近怎生费得这许多精神!”行者道:“不须商量等我去看看。”

将身一纵跳在半空看时一望无际。真个是:匝地远天凝烟带雨。夹道柔茵乱漫山翠盖张。密密搓搓初叶攀攀扯扯正芬芳。遥望不知何所尽近观一似绿云茫。蒙蒙茸茸郁郁苍苍。风声飘索索日影映煌煌。那中间有松有柏还有竹多梅多柳更多桑。薜萝缠古树藤葛绕垂杨。盘团似架联络如床。有处花开真布锦无端卉远生香。为人谁不遭荆棘那见西方荆棘长!行者看罢多时将云头按下道:“师父这去处远哩!”三藏问:“有多少远?”行者道:“一望无际似有千里之遥。”三藏大惊道:“怎生是好?”沙僧笑道:“师父莫愁我们也学烧荒的放上一把火烧绝了荆棘过去。”八戒道:“莫乱谈!

烧荒的须在十来月草衰木枯方好引火。如今正是蕃盛之时怎么烧得!”行者道:“就是烧得也怕人子。”三藏道:“这般怎生得度?”八戒笑道:“要得度还依我。”好呆子捻个诀念个咒语把腰躬一躬叫“长!”就长了有二十丈高下的身躯把钉钯幌一幌教“变!”就变了有三十丈长短的钯柄拽开步双手使钯将荆棘左右搂开:“请师父跟我来也!”三藏见了甚喜即策马紧随。后面沙僧挑着行李行者也使铁棒拨开。这一日未曾住手行有百十里将次天晚见有一块空阔之处当路上有一通石碣上有三个大字乃“荆棘岭”;下有两行十四个小字乃“荆棘蓬攀八百里古来有路少人行”。八戒见了笑道:“等我老猪与他添上两句:自今八戒能开破直透西方路尽平!”三藏欣然下马道:“徒弟啊累了你也!我们就在此住过了今宵待明日天光再走。”八戒道:“师父莫住趁此天色晴明我等有兴连夜搂开路走他娘!”那长老只得相从。

八戒上前努力师徒们人不住手马不停蹄又行了一日一夜却又天色晚矣。那前面蓬蓬结结又闻得风敲竹韵飒飒松声。却好又有一段空地中间乃是一座古庙庙门之外有松柏凝青桃梅斗丽。三藏下马与三个徒弟同看只见岩前古庙枕寒流落目荒烟锁废丘。白鹤丛中深岁月绿芜台下自春秋。

竹摇青珮疑闻语鸟弄余音似诉愁。鸡犬不通人迹少闲花野蔓绕墙头。行者看了道:“此地少吉多凶不宜久坐。”沙僧道:

“师兄差疑了似这杳无人烟之处又无个怪兽妖禽怕他怎的?”说不了忽见一阵阴风庙门后转出一个老者头戴角巾身穿淡服手持拐杖足踏芒鞋后跟着一个青脸獠牙、红须赤身鬼使头顶着一盘面饼跪下道:“大圣小神乃荆棘岭土地知大圣到此无以接待特备蒸饼一盘奉上老师父各请一餐。此地八百里更无人家聊吃些儿充饥。”八戒欢喜上前舒手就欲取饼。不知行者端详已久喝一声:“且住!这厮不是好人!休得无礼!你是甚么土地来诳老孙!看棍!”那老者见他打来将身一转化作一阵阴风呼的一声把个长老摄将起去飘飘荡荡不知摄去何所。慌得那大圣没跟寻处八戒沙僧俱相顾失色白马亦只自惊吟。三兄弟连马四口恍恍忽忽远望高张并无一毫下落前后找寻不题。

却说那老者同鬼使把长老抬到一座烟霞石屋之前轻轻放下与他携手相搀道:“圣僧休怕我等不是歹人乃荆棘岭十八公是也。因风清月霁之宵特请你来会友谈诗消遣情怀故耳。”那长老却才定性睁眼仔细观看真个是:漠漠烟云去所清清仙境人家。正好洁身修炼堪宜种竹栽花。每见翠岩来鹤时闻青沼鸣蛙。更赛天台丹灶仍期华岳明霞。说甚耕云钓月此间隐逸堪夸。坐久幽怀如海朦胧月上窗纱。三藏正自点看渐觉月明星朗只听得人语相谈都道:“十八公请得圣僧来也。”长老抬头观看乃是三个老者:前一个霜姿丰采第二个绿鬓婆娑第三个虚心黛色。各各面貌、衣服俱不相同都来与三藏作礼。长老还了礼道:“弟子有何德行敢劳列位仙翁下爱?”十八公笑道:“一向闻知圣僧有道等待多时今幸一遇。如果不吝珠玉宽坐叙怀足见禅机真派。”三藏躬身道:“敢问仙翁尊号?”十八公道:“霜姿者号孤直公绿鬓者号凌空子虚心者号拂云叟老拙号曰劲节。”三藏道:“四翁尊寿几何?”孤直公道:“我岁今经千岁古撑天叶茂四时春。香枝郁郁龙蛇状碎影重重霜雪身。自幼坚刚能耐老从今正直喜修真。乌栖凤宿非凡辈落落森森远俗尘。”凌空子笑道:“吾年千载傲风霜高干灵枝力自刚。夜静有声如雨滴秋晴荫影似云张。盘根已得长生诀受命尤宜不老方留鹤化龙非俗辈苍苍爽爽近仙乡。”拂云叟笑道:“岁寒虚度有千秋老景潇然清更幽。不杂嚣尘终冷淡饱经霜雪自风流。七贤作侣同谈道六逸为朋共唱酬。戛玉敲金非琐琐天然情性与仙游。”劲节十八公笑道:“我亦千年约有余苍然贞秀自如如。堪怜雨露生成力借得乾坤造化机。万壑风烟惟我盛四时洒落让吾疏。盖张翠影留仙客博弈调琴讲道书。”三藏称谢道:“四位仙翁俱享高寿但劲节翁又千岁余矣。高年得道丰采清奇得非汉时之四皓乎?”四老道:“承过奖!承过奖!吾等非四皓乃深山之四操也。敢问圣僧妙龄几何?”三藏合掌躬身答曰:“四十年前出母胎未产之时命已灾。逃生落水随波滚幸遇金山脱本骸。

养性看经无懈怠诚心拜佛敢俄捱?今蒙皇上差西去路遇仙翁下爱来。”四老俱称道:“圣僧自出娘胎即从佛教果然是从小修行真中正有道之上僧也。我等幸接台颜敢求大教望以禅法指教一二足慰生平。”长老闻言慨然不惧即对众言曰:

“禅者静也法者度也。静中之度非悟不成。悟者洗心涤虑脱俗离尘是也。夫人身难得中土难生正法难遇:全此三者幸莫大焉。至德妙道渺漠希夷六根六识遂可扫除。(WWW.mianhuatang.la 好看的小说)菩提者不死不生无余无欠空色包罗圣凡俱遣。访真了元始钳锤悟实了牟尼手段。挥象罔踏碎涅槃。必须觉中觉了悟中悟一点灵光全保护。放开烈焰照婆娑法界纵横独显露。至幽微更守固玄关口说谁人度?我本元修大觉禅有缘有志方记悟。”

四老侧耳受了无边喜悦一个个稽皈依躬身拜谢道:

“圣僧乃禅机之悟本也!”拂云叟道:“禅虽静法虽度须要性定心诚纵为大觉真仙终坐无生之道。我等之玄又大不同也。”三藏云:“道乃非常体用合一如何不同?”拂云叟笑云:

“我等生来坚实体用比尔不同。感天地以生身蒙雨露而滋色。笑傲风霜消磨日月。一叶不凋千枝节操。似这话不叩冲虚你执持梵语。道也者本安中国反来求证西方。空费了草鞋不知寻个甚么?石狮子剜了心肝野狐涎灌彻骨髓。忘本参禅妄求佛果都似我荆棘岭葛藤谜语萝蓏浑言。此般君子怎生接引?这等规模如何印授?必须要检点见前面目静中自有生涯。没底竹篮汲水无根铁树生花。灵宝峰头牢着脚归来雅会上龙华。”三藏闻言叩头拜谢十八公用手搀扶孤直公将身扯起凌空子打个哈哈道:“拂云之言分明漏泄。圣僧请起不可尽信。我等趁此月明原不为讲论修持且自吟哦逍遥放荡襟怀也。”拂云叟笑指石屋道:“若要吟哦且入小庵一茶何如?”

长老真个欠身向石屋前观看门上有三个大字乃“木仙庵”。遂此同入又叙了坐次忽见那赤身鬼使捧一盘茯苓膏将五盏香汤奉上。四老请唐僧先吃三藏惊疑不敢便吃。那四老一齐享用三藏却才吃了两块各饮香汤收去。三藏留心偷看只见那里玲珑光彩如月下一般:“水自石边流出香从花里飘来。满座清虚雅致全无半点尘埃。那长老见此仙境。

以为得意情乐怀开十分欢喜忍不住念了一句道:“禅心似月迥无尘。”劲节老笑而即联道:“诗兴如天青更新。”孤直公道:“好句漫裁抟锦绣。”凌空子道:“佳文不点唾奇珍。”拂云叟道:“六朝一洗繁华尽四始重删雅颂分。”三藏道:“弟子一时失口胡谈几字诚所谓班门弄斧。适闻列仙之言清新飘逸真诗翁也。”劲节老道:“圣僧不必闲叙出家人全始全终。既有起句何无结句?望卒成之。”三藏道:“弟子不能烦十八公结而成篇为妙。”劲节道:“你好心肠!你起的句如何不肯结果?

悭吝珠玑非道理也。”三藏只得续后二句云:“半枕松风茶未熟吟怀潇洒满腔春。”

十八公道:“好个吟怀潇洒满腔春!”孤直公道:“劲节你深知诗味所以只管咀嚼何不再起一篇?”十八公亦慨然不辞道:“我却是顶针字起:春不荣华冬不枯云来雾往只如无。”凌空子道:“我亦体前顶针二句:无风摇拽婆娑影有客欣怜福寿图。”拂云叟亦顶针道:“图似西山坚节老清如南国没心夫。”

孤直公亦顶针道:“夫因侧叶称梁栋台为横柯作宪乌。”

长老听了赞叹不已道:“真是阳春白雪浩气冲霄!弟子不才敢再起两句。”孤直公道:“圣僧乃有道之士大养之人也。不必再相联句请赐教全篇庶我等亦好勉强而和。”三藏无已只得笑吟一律曰:“杖锡西来拜法王愿求妙典远传扬。

金芝三秀诗坛瑞宝树千花莲蕊香。百尺竿头须进步十方世界立行藏。修成玉象庄严体极乐门前是道场。”四老听毕俱极赞扬。十八公道:“老拙无能大胆搀越也勉和一。”云:

“劲节孤高笑木王灵椿不似我名扬。山空百丈龙蛇影。泉泌千年琥珀香。解与乾坤生气概喜因风雨化行藏。衰残自愧无仙骨惟有苓膏结寿场。”孤直公道:“此诗起句豪雄联句有力但结句自谦太过矣堪羡!堪羡!老拙也和一。”云:“霜姿常喜宿禽王四绝堂前大器扬。露重珠缨蒙翠盖风轻石齿碎寒香。长廊夜静吟声细古殿秋阴淡影藏。元日迎春曾献寿老来寄傲在山场。”凌空子笑而言曰:“好诗!好诗!真个是月胁天心老拙何能为和?但不可空过也须扯谈几句。”曰:“梁栋之材近帝王太清宫外有声扬。晴轩恍若来青气暗壁寻常度翠香。壮节凛然千古秀深根结矣九泉藏。凌云势盖婆娑影不在群芳艳丽场。”拂云叟道:“三公之诗高雅清淡正是放开锦绣之囊也。我身无力我腹无才得三公之教茅塞顿开无已也打油几句幸勿哂焉。”诗曰:“淇澳园中乐圣王渭川千亩任分扬。翠筠不染湘娥泪班箨堪传汉史香。霜叶自来颜不改烟梢从此色何藏?子猷去世知音少亘古留名翰墨场。”

三藏道:“众仙老之诗真个是吐凤喷珠游夏莫赞。厚爱高情感之极矣。但夜已深沉三个小徒不知在何处等我。意者弟子不能久留敢此告回寻访尤天穷之至爱也望老仙指示归路。”四老笑道:“圣僧勿虑我等也是千载奇逢况天光晴爽虽夜深却月明如昼再宽坐坐待天晓自当远送过岭高徒一定可相会也。”

正话间只见石屋之外有两个青衣女童挑一对绛纱灯笼后引着一个仙女。那仙女拈着一枝杏花笑吟吟进门相见。

那仙女怎生模样?他生得:青姿妆翡翠丹脸赛胭脂。星眼光还彩蛾眉秀又齐。下衬一条五色梅浅红裙子上穿一件烟里火比甲轻衣。弓鞋弯凤嘴绫袜锦绣泥。妖娆娇似天台女不亚当年俏妲姬。四老欠身问道:“杏仙何来?”那女子对众道了万福道:“知有佳客在此赓酬特来相访敢求一见。”十八公指着唐僧道:“佳客在此何劳求见!”三藏躬身不敢言语。那女子叫:“快献茶来。”又有两个黄衣女童捧一个红漆丹盘盘内有六个细磁茶盂盂内设几品异果横担着匙儿提一把白铁嵌黄铜的茶壶壶内香茶喷鼻。斟了茶那女子微露春葱捧磁盂先奉三藏次奉四老然后一盏自取而陪。

凌空子道:“杏仙为何不坐?”那女子方才去坐。茶毕欠身问道:“仙翁今宵盛乐佳句请教一二如何?”拂云叟道:“我等皆鄙俚之言惟圣僧真盛唐之作甚可嘉羡。”那女子道:“如不吝教乞赐一观。”四老即以长老前诗后诗并禅法论宣了一遍。那女子满面春风对众道:“妾身不才不当献丑。但聆此佳句似不可虚也勉强将后诗奉和一律如何?”遂朗吟道:“上盖留名汉武王周时孔子立坛场。董仙爱我成林积孙楚曾怜寒食香。雨润红姿娇且嫩烟蒸翠色显还藏。自知过熟微酸意落处年年伴麦场。”四老闻诗人人称贺都道:“清雅脱尘句内包含春意。好个雨润红姿娇且嫩雨润红姿娇且嫩!”那女子笑而悄答道:“惶恐!惶恐!适闻圣僧之章诚然锦心绣口如不吝珠玉赐教一阕如何?”唐僧不敢答应。那女子渐有见爱之情挨挨轧轧渐近坐边低声悄语呼道:“佳客莫者趁此良宵不耍子待要怎的?人生光景能有几何?”十八公道:“杏仙尽有仰高之情圣僧岂可无俯就之意?如不见怜是不知趣了也。”孤直公道:“圣僧乃有道有名之士决不苟且行事。如此样举措是我等取罪过了。污人名坏人德非远达也。果是杏仙有意可教拂云叟与十八公做媒我与凌空子保亲成此姻眷何不美哉!”

三藏听言遂变了颜色跳起来高叫道:“汝等皆是一类邪物这般诱我!当时只以砥砺之言谈玄谈道可也如今怎么以美人局来骗害贫僧!是何道理!”四老见三藏怒一个个咬指担惊再不复言。那赤身鬼使暴躁如雷道:“这和尚好不识抬举!我这姐姐那些儿不好?他人材俊雅玉质娇姿不必说那女工针指只这一段诗才也配得过你。你怎么这等推辞!休错过了!孤直公之言甚当如果不可苟合待我再与你主婚。”

三藏大惊失色凭他们怎么胡谈乱讲只是不从。鬼使又道:

“你这和尚我们好言好语你不听从若是我们起村野之性还把你摄了去教你和尚不得做老婆不得娶却不枉为人一世也?”那长老心如金石坚执不从。暗想道:“我徒弟们不知在那里寻我哩!”说一声止不住眼中堕泪。那女子陪着笑挨至身边翠袖中取出一个蜜合绫汗巾儿与他揩泪道:“佳客勿得烦恼我与你倚玉偎香耍子去来。”长老咄的一声吆喝跳起身来就走被那些人扯扯拽拽嚷到天明。

忽听得那里叫声:“师父!师父!你在那方言语也?”原来那孙大圣与八戒沙僧牵着马挑着担一夜不曾住脚穿荆度棘东寻西找却好半云半雾的过了八百里荆棘岭西下听得唐僧吆喝却就喊了一声。那长老挣出门来叫声:“悟空我在这里哩快来救我!快来救我!”那四老与鬼使那女子与女童幌一幌都不见了。须臾间八戒、沙僧俱到边前道:“师父你怎么得到此也?”三藏扯住行者道:“徒弟啊多累了你们了!昨日晚间见的那个老者言说土地送斋一事是你喝声要打他就把我抬到此方。他与我携手相搀走入门又见三个老者来此会我俱道我做圣僧一个个言谈清雅极善吟诗。我与他赓和相攀觉有夜半时候又见一个美貌女子执灯火也来这里会我吟了一诗称我做佳客。因见我相貌欲求配偶我方省悟正不从时又被他做媒的做媒保亲的保亲主婚的主婚我立誓不肯正欲挣着要走与他嚷闹不期你们到了。一则天明二来还是怕你只才还扯扯拽拽忽然就不见了。”行者道:

“你既与他叙话谈诗就不曾问他个名字?”三藏道:“我曾问他之号那老者唤做十八公号劲节;第二个号孤直公;第三个号凌空子;第四个号拂云叟;那女子人称他做杏仙。”八戒道:

“此物在于何处?才往那方去了?”三藏道:“去向之方不知何所但只谈诗之处去此不远。”

他三人同师父看处只见一座石崖崖上有木仙庵三字。

三藏道:“此间正是。”行者仔细观之却原来是一株大桧树一株老柏一株老松一株老竹竹后有一株丹枫。再看崖那边还有一株老杏二株腊梅二株丹桂。行者笑道:“你可曾看见妖怪?”八戒道:“不曾。”行者道:“你不知就是这几株树木在此成精也。”八戒道:“哥哥怎得知成精者是树?”行者道:“十八公乃松树孤直公乃柏树凌空子乃桧树拂云叟乃竹竿赤身鬼乃枫树杏仙即杏树女童即丹桂、腊梅也。”八戒闻言不论好歹一顿钉钯三五长嘴连拱带筑把两颗腊梅、丹桂、老杏、枫杨俱挥倒在地果然那根下俱鲜血淋漓。三藏近前扯住道:“悟能不可伤了他!他虽成了气候却不曾伤我我等找路去罢。”行者道:“师父不可惜他恐日后成了大怪害人不浅也。”那呆子索性一顿钯将松柏桧竹一齐皆筑倒却才请师父上马顺大路一齐西行。毕竟不知前去如何且听下回分解。
------------

第六十五回 妖邪假设小雷音 四众皆遭大厄难

这回因果劝人为善切休作恶。一念生神明照鉴任他为作。拙蠢乖能君怎学两般还是无心药。趁生前有道正该修莫浪泊。认根源脱本壳。访长生须把捉。要时时明见醍醐斟酌。贯彻三关填黑海管教善者乘鸾鹤。那其间愍故更慈悲登极乐。话表唐三藏一念虔诚且休言天神保护似这草木之灵尚来引送雅会一宵脱出荆棘针刺再无萝蓏攀缠。四众西进行彀多时又值冬残正是那三春之日:物华交泰斗柄回寅。草芽遍地绿柳眼满堤青。一岭桃花红锦涴半溪烟水碧罗明。几多风雨无限心情。日晒花心艳燕衔苔蕊轻。山色王维画浓淡鸟声季子舌纵横。芳菲铺绣无人赏蝶舞蜂歌却有情。师徒们也自寻芳踏翠缓随马步正行之间忽见一座高山远望着与天相接。三藏扬鞭指道:“悟空那座山也不知有多少高可便似接着青天透冲碧汉。”行者道:“古诗不云只有天在上更无山与齐。但言山之极高无可与他比并岂有接天之理!”八戒道:“若不接天如何把昆仑山号为天柱?”行者道:“你不知自古天不满西北。昆仑山在西北乾位上故有顶天塞空之意遂名天柱。”沙僧笑道:“大哥把这好话儿莫与他说他听了去又降别人。我们且走路等上了那山就知高下也。”

那呆子赶着沙僧厮耍厮斗老师父马快如飞须臾到那山崖之边。一步步往上行来只见那山:林中风飒飒涧底水潺潺。鸦雀飞不过神仙也道难。千崖万壑亿曲百湾。尘埃滚滚无人到怪石森森不厌看。有处有云如水滉是方是树鸟声繁。鹿衔芝去猿摘桃还。狐貉往来崖上跳麖獐出入岭头顽。

忽闻虎啸惊人胆斑豹苍狼把路拦。唐三藏一见心惊孙行者神通广大你看他一条金箍棒哮吼一声吓过了狼虫虎豹剖开路引师父直上高山。行过岭头下西平处忽见祥光蔼蔼彩雾纷纷有一所楼台殿阁隐隐的钟磬悠扬。三藏道:“徒弟们看是个甚么去处。”行者抬头用手搭凉篷仔细观看那壁厢好个所在!真个是:珍楼宝座上刹名方。谷虚繁地籁境寂散天香。青松带雨遮高阁翠竹留云护讲堂。霞光缥缈龙宫显彩色飘飖沙界长。朱栏玉户画栋雕梁。谈经香满座语箓月当窗。鸟啼丹树内鹤饮石泉旁。四围花琪园秀三面门开舍卫光。楼台突兀门迎嶂钟磬虚徐声韵长。窗开风细帘卷烟茫。有僧情散淡无俗意和昌。红尘不到真仙境静土招提好道场。行者看罢回复道:“师父那去处是便是座寺院却不知禅光瑞蔼之中又有些凶气何也。观此景象也似雷音却又路道差池。我们到那厢决不可擅入恐遭毒手。”唐僧道:“既有雷音之景莫不就是灵山?你休误了我诚心担搁了我来意。”行者道:“不是不是!灵山之路我也走过几遍那是这路途!”八戒道:“纵然不是也必有个好人居住。”沙僧道:“不必多疑此条路未免从那门过是不是一见可知也。”行者道:

“悟净说得有理。”

那长老策马加鞭至山门前见雷音寺三个大字慌得滚下马来倒在地下口里骂道:“泼猢狲!害杀我也!现是雷音寺还哄我哩!”行者陪笑道:“师父莫恼你再看看。山门上乃四个字你怎么只念出三个来倒还怪我?”长老战兢兢的爬起来再看真个是四个字乃小雷音寺。三藏道:“就是小雷音寺必定也有个佛祖在内。经上言三千诸佛想是不在一方:似观音在南海普贤在峨眉文殊在五台。(WWW.mianhuatang.la 好看的小说)这不知是那一位佛祖的道场。

古人云有佛有经无方无宝我们可进去来。”行者道:“不可进去此处少吉多凶若有祸患你莫怪我。”三藏道:“就是无佛也必有个佛象。我弟子心愿遇佛拜佛如何怪你。”即命八戒取袈裟换僧帽结束了衣冠举步前进。只听得山门里有人叫道:“唐僧你自东土来拜见我佛怎么还这等怠慢?”三藏闻言即便下拜八戒也磕头沙僧也跪倒惟大圣牵马收拾行李在后。方入到二层门内就见如来大殿。殿门外宝台之下摆列着五百罗汉、三千揭谛、四金刚、八菩萨、比丘尼、优婆塞、无数的圣僧、道者真个也香花艳丽瑞气缤纷。慌得那长老与八戒沙僧一步一拜拜上灵台之间行者公然不拜。又闻得莲台座上厉声高叫道:“那孙悟空见如来怎么不拜?”不知行者又仔细观看见得是假遂丢了马匹行囊掣棒在手喝道:“你这伙孽畜十分胆大!怎么假倚佛名败坏如来清德!不要走!”

双手轮棒上前便打。只听得半空中叮狢一声撇下一副金铙把行者连头带足合在金铙之内。慌得个猪八戒、沙和尚连忙使起钯杖就被些阿罗揭谛、圣僧道者一拥近前围绕。他两个措手不及尽被拿了将三藏捉住一齐都绳缠索绑紧缚牢栓。

原来那莲花座上装佛祖者乃是个妖王众阿罗等都是些小怪。遂收了佛祖体象依然现出妖身将三众抬入后边收藏把行者合在金铙之中永不开放只搁在宝台之上限三昼夜化为脓血。化后才将铁笼蒸他三个受用。这正是:碧眼猢儿识假真禅机见象拜金身。黄婆盲目同参礼木母痴心共话论。邪怪生强欺本性魔头怀恶诈天人。诚为道小魔头大错入旁门枉费身。那时群妖将唐僧三众收藏在后把马拴在后边把他的袈裟僧帽安在行李担内亦收藏了一壁厢严紧不题。

却说行者合在金铙里黑洞洞的燥得满身流汗左拱右撞不能得出急得他使铁棒乱打莫想得动分毫。他心里没了算计将身往外一挣却要挣破那金铙遂捻着一个诀就长有千百丈高那金铙也随他身长全无一些瑕缝光明。却又捻诀把身子往下一小小如芥菜子儿那铙也就随身小了更没些些孔窍。他又把铁棒吹口仙气叫“变!”即变做幡竿一样撑住金铙。他却把脑后毫毛选长的拔下两根叫“变!”即变做梅花头五瓣钻儿挨着棒下钻有千百下只钻得苍苍响魀再不钻动一些。行者急了却捻个诀念一声“唵囒静法界乾元亨利贞”的咒语拘得那五方揭谛六丁六甲、一十八位护教伽蓝都在金铙之外道:“大圣我等俱保护着师父不教妖魔伤害你又拘唤我等做甚?”行者道:“我那师父不听我劝解就弄死他也不亏!但只你等怎么快作法将这铙钹掀开放我出来再作处治。这里面不通光亮满身暴燥却不闷杀我也?”众神真个掀铙就如长就的一般莫想揭得分毫。金头揭谛道:“大圣这铙钹不知是件甚么宝贝连上带下合成一块。小神力薄不能掀动。”行者道:“我在里面不知使了多少神通也不得动。”

揭谛闻言即着六丁神保护着唐僧六甲神看守着金铙众伽蓝前后照察他却纵起祥光须臾间闯入南天门里不待宣召直上灵霄宝殿之下见玉帝俯伏启奏道:“主公臣乃五方揭谛使。今有齐天大圣保唐僧取经路遇一山名小雷音寺。

唐僧错认灵山进拜原来是妖魔假设困陷他师徒将大圣合在一副金铙之内进退无门看看至死特来启奏。”即传旨:

“差二十八宿星辰快去释厄降妖。”那星宿不敢少缓随同揭谛出了天门至山门之内。有二更时分那些大小妖精因获了唐僧老妖俱犒赏了各去睡觉。众星宿更不惊张都到铙钹之外报道:“大圣我等是玉帝差来二十八宿到此救你。”行者听说大喜便教:“动兵器打破老孙就出来了!”众星宿道:“不敢打此物乃浑金之宝打着必响;响时惊动妖魔却难救拔。

等我们用兵器捎他你那里但见有一些光处就走。”行者道:

“正是。”你看他们使枪的使枪使剑的使剑使刀的使刀使斧的使斧;扛的扛抬的抬掀的掀捎的捎弄到有三更天气漠然不动就是铸成了囫囵的一般。那行者在里边东张张西望望爬过来滚过去莫想看见一些光亮。亢金龙道:“大圣啊且休焦躁观此宝定是个如意之物断然也能变化。你在那里面于那合缝之处用手摸着等我使角尖儿拱进来你可变化了顺松处脱身。”行者依言真个在里面乱摸。这星宿把身变小了那角尖儿就似个针尖一样顺着钹合缝口上伸将进去可怜用尽千斤之力方能穿透里面。却将本身与角使法象叫“长!长!长!”角就长有碗来粗细。那钹口倒也不象金铸的好似皮肉长成的顺着亢金龙的角紧紧噙住四下里更无一丝拔缝。行者摸着他的角叫道:“不济事!上下没有一毫松处!

没奈何你忍着些儿疼带我出去。”好大圣即将金箍棒变作一把钢钻儿将他那角尖上钻了一个孔窍把身子变得似个芥菜子儿拱在那钻眼里蹲着叫:“扯出角去!扯出角去!”这星宿又不知费了多少力方才拔出使得力尽筋柔倒在地下。

行者却自他角尖钻眼里钻出现了原身掣出铁棒照铙钹当的一声打去就如崩倒铜山咋开金铙可惜把个佛门之器打做个千百块散碎之金!唬得那二十八宿惊张五方揭谛竖大小群妖皆梦醒。老妖王睡里慌张急起来披衣擂鼓聚点群妖各执器械。此时天将黎明一拥赶到宝台之下只见孙行者与列宿围在碎破金铙之外大惊失色即令:“小的们!紧关了前门不要放出人去!”行者听说即携星众驾云跳在九霄空里。那妖王收了碎金排开妖卒列在山门外。妖王怀恨没奈何披挂了使一根短软狼牙棒出营高叫:“孙行者!好男子不可远走高飞!快向前与我交战三合!”行者忍不住即引星众按落云头观看那妖精怎生模样但见他:蓬着头勒一条扁薄金箍;光着眼簇两道黄眉的竖。悬胆鼻孔窃开查;四方口牙齿尖利。穿一副叩结连环铠勒一条生丝攒穗绦。脚踏乌喇鞋一对手执狼牙棒一根。此形似兽不如兽相貌非人却似人。行者挺着铁棒喝道:“你是个甚么怪物擅敢假装佛祖侵占山头虚设小雷音寺!”那妖王道:“这猴儿是也不知我的姓名故来冒犯仙山。此处唤做小西天因我修行得了正果天赐与我的宝阁珍楼。我名乃是黄眉老佛这里人不知但称我为黄眉大王、黄眉爷爷。一向久知你往西去有些手段故此设象显能诱你师父进来要和你打个赌赛。如若斗得过我饶你师徒让汝等成个正果;如若不能将汝等打死等我去见如来取经果正中华也。”行者笑道:“妖精不必海口既要赌快上来领棒!”那妖王喜孜孜使狼牙棒抵住。这一场好杀:两条棒不一样说将起来有形状:一条短软佛家兵一条坚硬藏海藏。都有随心变化功今番相遇争强壮。短软狼牙杂锦妆坚硬金箍蛟龙象。若粗若细实可夸要短要长甚停当。猴与魔齐打仗这场真个无虚诳。驯猴秉教作心猿泼怪欺天弄假象。

嗔嗔恨恨各无情恶恶凶凶都有样。那一个当头手起不放松这一个架丢劈面难推让。喷云照日昏吐雾遮峰嶂。棒来棒去两相迎忘生忘死因三藏。看他两个斗经五十回合不见输赢。

那山门口鸣锣擂鼓众妖精呐喊摇旗。这壁厢有二十八宿天兵共五方揭谛众圣各掮器械吆喝一声把那魔头围在中间吓得那山门外群妖难擂鼓战兢兢手软不敲锣。老妖魔公然不惧一只手使狼牙棒架着众兵一只手去腰间解下一条旧白布搭包儿往上一抛滑的一声响喨把孙大圣、二十八宿与五方揭谛一搭包儿通装将去挎在肩上拽步回身众小妖个个欢然得胜而回。老妖教小的们取了三五十条麻索解开搭包拿一个捆一个一个个都骨软筋麻皮肤窊皱。捆了抬去后边不分好歹俱掷之于地。妖王又命排筵畅饮自旦至暮方散各归寝处不题。

却说孙大圣与众神捆至夜半忽闻有悲泣之声。侧耳听时却原来是三藏声音哭道:“悟空啊!我自恨当时不听伊致令今日受灾危。金铙之内伤了你麻绳捆我有谁知。四人遭逢缘命苦三千功行尽倾颓。何由解得迍邅难坦荡西方去复归!”行者听言暗自怜悯道:“那师父虽是未听吾言今遭此毒然于患难之中还有忆念老孙之意。趁此夜静妖眠无人防备且去解脱众等逃生也。”好大圣使了个遁身法将身一小脱下绳来走近唐僧身边叫声“师父。”长老认得声音叫道:

“你为何到此?”行者悄悄的把前项事告诉了一遍长老甚喜道:“徒弟!快救我一救!向后事但凭你处再不强了!”行者才动手先解了师父放了八戒沙僧又将二十八宿、五方揭谛个个解了又牵过马来教快先走出去。方出门却不知行李在何处又来找寻。亢金龙道:“你好重物轻人!既救了你师父就彀了又还寻甚行李?”行者道:“人固要紧衣钵尤要紧。包袱中有通关文牒、锦襕袈裟、紫金钵盂俱是佛门至宝如何不要!”

八戒道:“哥哥你去找寻我等先去路上等你。”你看那星众簇拥着唐僧使个摄法共弄神通一阵风撮出垣围奔大路下了山坡却屯于平处等候。

约有三更时分孙大圣轻挪慢步走入里面原来一层层门户甚紧。他就爬上高楼看时窗牖皆关欲要下去又恐怕窗棂儿响不敢推动。捻着诀摇身一变变做一个仙鼠俗名蝙蝠。你道他怎生模样:头尖还似鼠眼亮亦如之。有翅黄昏出无光白昼居。藏身穿瓦穴觅食扑蚊儿。偏喜晴明月飞腾最识时。他顺着不封瓦口椽子之下钻将进去越门过户到了中间看时只见那第三重楼窗之下熌灼灼一道毫光也不是灯烛之光香火之光又不是飞霞之光掣电之光。他半飞半跳近于光前看时却是包袱放光。那妖精把唐僧的袈裟脱了不曾折就乱乱的揌在包袱之内。那袈裟本是佛宝上边有如意珠、摩尼珠、红玛瑙、紫珊瑚、舍利子、夜明珠所以透的光彩。

他见了此衣钵心中一喜就现了本象拿将过来也不管担绳偏正抬上肩往下就走不期脱了一头扑的落在楼板上唿喇的一声响喨。噫!有这般事:可可的老妖精在楼下睡觉一声响把他惊醒跳起来乱叫道:“有人了!有人了!”那些大小妖都起来点灯打火一齐吆喝前后去看。有的来报道:“唐僧走了!”又有的来报道:“行者众人俱走了!”老妖急传号令教:

“拿!各门上谨慎!”行者听言恐又遭他罗网挑不成包袱纵筋斗就跳出楼窗外走了。

那妖精前前后后寻不着唐僧等又见天色将明取了棒帅众来赶只见那二十八宿与五方揭谛等神云雾腾腾屯住山坡之下。妖王喝了一声“那里去!吾来也!”角木蛟急唤:“兄弟们!怪物来了!”亢金龙、女土蝠、房日兔、心月狐、尾火虎、箕水豹、斗木獬、牛金牛、氐土貉、虚日鼠、危月燕、室火猪、壁水獝、奎木狼、娄金狗、胃土彘、昴日鸡、毕月乌、觜火猴、参水猿、井木犴、鬼金羊、柳土獐、星日马、张月鹿、翼火蛇、轸水蚓领着金头揭谛、银头揭谛、六甲、六丁等神、护教伽蓝同八戒沙僧不领唐三藏丢了白龙马各执兵器一拥而上。这妖王见了呵呵冷笑叫一声哨子有四五千大小妖精一个个威强力胜浑战在西山坡上。好杀:魔头泼恶欺真性真性温柔怎奈魔。百计施为难脱苦千方妙用不能和。诸天来拥护众圣助干戈。留情亏木母定志感黄婆。浑战惊天并振地强争设网与张罗。那壁厢摇旗呐喊这壁厢擂鼓筛锣。枪刀密密寒光荡剑戟纷纷杀气多。妖卒凶还勇神兵怎奈何!愁云遮日月惨雾罩山河。苦掤苦拽来相战皆因三藏拜弥陀。那妖精倍加勇猛帅众上前掩杀。正在那不分胜败之际只闻得行者叱咤一声道:“老孙来了!”八戒迎着道:“行李如何?”行者道:“老孙的性命几乎难免却便说甚么行李!”沙僧执着宝杖道:“且休叙话快去打妖精也!”那星宿、揭谛、丁甲等神被群妖围在垓心浑杀老妖使棒来打他三个。这行者、八戒、沙僧丢开棍杖、轮着钉钯抵住。真个是地暗天昏不能取胜只杀得太阳星西没山根;太阴星东生海峤。那妖见天晚打个哨子教群妖各各留心他却取出宝贝。孙行者看得分明那怪解下搭包拿在手中。行者道声“不好了!走啊!”他就顾不得八戒沙僧、诸天等众一路筋斗跳上九霄空里。众神、八戒、沙僧不解其意被他抛起去又都装在里面只是走了行者。那妖王收兵回寺又教取出绳索照旧绑了。将唐僧、八戒、沙僧悬梁高吊白马拴在后边诸神亦俱绑缚抬在地窖子内封了盖锁。那众妖遵依一一收了不题。

却说行者跳在九霄全了性命见妖兵回转不张旗号已知众等遭擒。他却按下祥光落在那东山顶上咬牙恨怪物滴泪想唐僧仰面朝天望悲嗟忽失声叫道:“师父啊!你是那世里造下这迍邅难今生里步步遇妖精似这般苦楚难逃怎生是好!”独自一个嗟叹多时复又宁神思虑以心问心道:“这妖魔不知是个甚么搭包子那般装得许多物件?如今将天神天将许多人又都装进去了我待求救于天奈恐玉帝见怪。我记得有个北方真武号曰荡魔天尊他如今现在南赡部洲武当山上等我去请他来搭救师父一难。”正是:仙道未成猿马散心神无主五行枯。毕竟不知此去端的如何且听下回分解。
------------

第六十六回 诸神遭毒手 弥勒缚妖魔

话表孙大圣无计可施纵一朵祥云驾筋斗径转南赡部洲去拜武当山参请荡魔天尊解释三藏、八戒、沙僧、天兵等众之灾。他在半空里无停止不一日早望见祖师仙境轻轻按落云头定睛观看好去处:巨镇东南中天神岳。芙蓉峰竦杰紫盖岭巍峨。九江水尽荆扬远百越山连翼轸多。上有太虚之宝洞朱陆之灵台。三十六宫金磬响百千万客进香来。舜巡禹祷玉简金书。楼阁飞青鸟幢幡摆赤裾。地设名山雄宇宙天开仙境透空虚。几树榔梅花正放满山瑶草色皆舒。龙潜涧底虎伏崖中。幽含如诉语驯鹿近人行。白鹤伴云栖老桧青鸾丹凤向阳鸣。玉虚师相真仙地金阙仁慈治世门。上帝祖师乃净乐国王与善胜皇后梦吞日光觉而有孕怀胎一十四个月于开皇元年甲辰之岁三月初一日午时降诞于王宫。那爷爷;幼而勇猛长而神灵。不统王位惟务修行。父母难禁弃舍皇宫。参玄入定在此山中。功完行满白日飞升。玉皇敕号真武之名。玄虚上应龟蛇合形。周天六合皆称万灵。无幽不察无显不成。劫终劫始剪伐魔精。

孙大圣玩着仙境景致早来到一天门、二天门、三天门却至太和宫外忽见那祥光瑞气之间簇拥着五百灵官。那灵官上前迎着道:“那来的是谁?”大圣道:“我乃齐天大圣孙悟空要见师相。”众灵官听说随报。祖师即下殿迎到太和宫。行者作礼道:“我有一事奉劳。”问:“何事?”行者道:“保唐僧西天取经路遭险难。至西牛贺洲有座山唤小西天小雷音寺有一妖魔。我师父进得山门见有阿罗揭谛比丘圣僧排列以为真佛倒身才拜忽被他拿住绑了。我又失于防闲被他抛一副金铙将我罩在里面无纤毫之缝口合如钳。甚亏金头揭谛请奏玉帝钦差二十八宿当夜下界掀揭不起。幸得亢金龙将角透入铙内将我度出被我打碎金铙惊醒怪物。赶战之间又被撒一个白布搭包儿将我与二十八宿并五方揭谛尽皆装去复用绳捆了。是我当夜脱逃救了星辰等众与我唐僧等。后为找寻衣钵又惊醒那妖与天兵赶战。那怪又拿出搭包儿理弄之时我却知道前音遂走了众等被他依然装去。我无计可施特来拜求师相一助力也。”祖师道:“我当年威镇北方统摄真武之位剪伐天下妖邪乃奉玉帝敕旨。后又披跣足踏腾蛇神龟领五雷神将、巨虬狮子、猛兽毒龙收降东北方黑气妖氛乃奉元始天尊符召。今日静享武当山安逸太和殿一向海岳平宁乾坤清泰。奈何我南赡部洲并北俱芦洲之地妖魔剪伐邪鬼潜踪。今蒙大圣下降不得不行;只是上界无有旨意不敢擅动干戈。假若法遣众神又恐玉帝见罪十分却了大圣又是我逆了人情。我谅着那西路上纵有妖邪也不为大害。我今着龟、蛇二将并五大神龙与你助力管教擒妖精救你师之难。”行者拜谢了祖师即同龟、蛇、龙神各带精锐之兵复转西洲之界。不一日到了小雷音寺按下云头径至山门外叫战。

却说那黄眉大王聚众怪在宝阁下说:“孙行者这两日不来又不知往何方去借兵也。”说不了只见前门上小妖报道:

“行者引几个龙蛇龟相在门外叫战!”妖魔道:“这猴儿怎么得个龙蛇龟相?此等之类却是何方来者?”随即披挂走出山门高叫:“汝等是那路龙神敢来造吾仙境?”五龙二将相貌峥嵘精神抖擞喝道:“那泼怪!我乃武当山太和宫混元教主荡魔天尊之前五位龙神、龟、蛇二将。今蒙齐天大圣相邀我天尊符召到此捕你这妖精快送唐僧与天星等出来免你一死!不然将这一山之怪碎劈其尸;几间之房烧为灰烬!”那怪闻言心中大怒道:“这畜生有何法力敢出大言!不要走!吃吾一棒!”这五条龙翻云使雨那两员将播土扬沙各执枪刀剑戟一拥而攻孙大圣又使铁棒随后。这一场好杀:凶魔施武行者求兵。凶魔施武擅据珍楼施佛象;行者求兵远参宝境借龙神。龟蛇生水火妖怪动刀兵。五龙奉旨来西路行者因师在后收。剑戟光明摇彩电枪刀晃亮闪霓虹。这个狼牙棒强能短软;那个金箍棒随意如心。只听得扢扑响声如爆竹叮当音韵似敲金。水火齐来征怪物刀兵共簇绕精灵。喊杀惊狼虎喧哗振鬼神。浑战正当无胜处妖魔又取宝和珍。行者帅五龙二将与妖魔战经半个时辰那妖精即解下搭包在手。行者见了心惊叫道:“列位仔细!”那龙神蛇龟不知甚么仔细一个个都停住兵近前抵挡。那妖精幌的一声把搭包儿撇将起去。孙大圣顾不得五龙二将驾筋斗跳在九霄逃脱。他把个龙神龟蛇一搭包子又装将去了。妖精得胜回寺也将绳捆了抬在地窖子里盖住不题。

你看那大圣落下云头斜敧在山巅之上没精没采懊恨道:“这怪物十分利害!”不觉的合着眼似睡一般猛听得有人叫道:“大圣休推睡快早上紧求救。你师父性命只在须臾间矣!”行者急睁睛跳起来看原来是日值功曹。行者喝道:“你这毛神这向在那方贪图血食不来点卯今日却来惊我!伸过孤拐来让老孙打两棒解闷!”功曹慌忙施礼道:“大圣你是人间之喜仙何闷之有!我等早奉菩萨旨令教我等暗中护佑唐僧乃同土地等神不敢暂离左右是以不得常来参见怎么反见责也?”行者道:“你既是保护如今那众星、揭谛、伽蓝并我师等被妖精困在何方?受甚罪苦?”功曹道:“你师父师弟都吊在宝殿廊下星辰等众都收在地窖之间受罪。这两日不闻大圣消息却才见妖精又拿了神龙、龟、蛇又送在地窖里去了方知是大圣请来之兵小神特来寻大圣。大圣莫辞劳倦千万再急急去求救援。”行者闻言及此不觉对功曹滴泪道:“我如今愧上天宫羞临海藏!怕问菩萨之原由愁见如来之玉象!才拿去者乃真武师相之龟、蛇、五龙圣众。教我再无方求救奈何?”功曹笑道:“大圣宽怀小神想起一处精兵请来断然可降。适才大圣至武当是南赡部洲之地。这枝兵也在南赡部洲盱眙山蠙城即今泗洲是也。那里有个大圣国师王菩萨神通广大。他手下有一个徒弟唤名小张太子还有四大神将昔年曾降伏水母娘娘。你今若去请他他来施恩相助准可捉怪救师也。”行者心喜道:“你且去保护我师父勿令伤他待老孙去请也。”

行者纵起筋斗云躲离怪处直奔盱眙山。不一日早到细观真好去处:南近江津北临淮水。东通海峤西接封浮。山顶上有楼观峥嵘山凹里有涧泉浩涌。嵯峨怪石槃秀乔松。百般果品应时新千样花枝迎日放。人如蚁阵往来多船似雁行归去广。上边有瑞岩观、东岳宫、五显祠、龟山寺钟韵香烟冲碧汉;又有玻璃泉、五塔峪、八仙台、杏花园山光树色映蠙城。

白云横不度幽鸟倦还鸣。说甚泰嵩衡华秀此间仙景若蓬瀛。(wwW.mianhuatang.la 无弹窗广告)

大圣点玩不尽径过了淮河入蠙城之内到大圣禅寺山门外又见那殿宇轩昂长廊彩丽有一座宝塔峥嵘。真是:插云倚汉高千丈仰视金瓶透碧空。上下有光凝宇宙东西无影映帘栊。

风吹宝铎闻天乐日映冰虬对梵宫。飞宿灵禽时诉语遥瞻淮水渺无穷。

行者且观且走直至二层门下。那国师王菩萨早已知之即与小张太子出门迎迓。相见叙礼毕行者道:“我保唐僧西天取经路上有个小雷音寺那里有个黄眉怪假充佛祖。我师父不辨真伪就下拜被他拿了。又将金铙把我罩了幸亏天降星辰救出。是我打碎金铙与他赌斗又将一个布搭包儿把天神、揭谛、伽蓝与我师父、师弟尽皆装了进去。我前去武当山请玄天上帝救援他差五龙龟蛇拿怪又被他一搭包子装去。弟子无依无倚故来拜请菩萨大展威力将那收水母之神通拯生民之妙用同弟子去救师父一难!取得经回永传中国扬我佛之智慧兴般若之波罗也。”国师王道:“你今日之事诚我佛教之兴隆理当亲去奈时值初夏正淮水泛涨之时新收了水猿大圣那厮遇水即兴恐我去后他乘空生顽无神可治。今着小徒领四将和你去助力炼魔收伏罢。”行者称谢即同四将并小张太子又驾云回小西天直至小雷音寺。小张太子使一条楮白枪四大将轮四把锟鋘剑和孙大圣上前骂战。小妖又去报知那妖王复帅群妖鼓噪而出道:“猢狲!你今又请得何人来也?”说不了小张太子指挥四将上前喝道:“泼妖精!你面上无肉不认得我等在此!”妖王道:“是那方小将敢来与他助力?”太子道:“吾乃泗州大圣国师王菩萨弟子帅领四大神将奉令擒你!”妖王笑道:“你这孩儿有甚武艺擅敢到此轻薄?”

太子道:“你要知我武艺等我道来:祖居西土流沙国我父原为沙国王。自幼一身多疾苦命干华盖恶星妨。因师远慕长生诀有分相逢舍药方。半粒丹砂祛病退愿从修行不为王。学成不老同天寿容颜永似少年郎。也曾赶赴龙华会也曾腾云到佛堂。捉雾拿风收水怪擒龙伏虎镇山场。抚民高立浮屠塔静海深明舍利光。楮白枪尖能缚怪淡缁衣袖把妖降。如今静乐蠙城内大地扬名说小张!”妖王听说微微冷笑道:“那太子你舍了国家从那国师王菩萨修的是甚么长生不老之术?

只好收捕淮河水怪却怎么听信孙行者诳谬之言千山万水来此纳命!看你可长生可不老也!”小张闻言心中大怒缠枪当面便刺四大将一拥齐攻孙大圣使铁棒上前又打。好妖精公然不惧轮着他那短软狼牙棒左遮右架直挺横冲。这场好杀:小太子楮白枪四柄锟鋘剑更强。悟空又使金箍棒齐心围绕杀妖王。妖王其实神通大不惧分毫左右搪。狼牙棒是佛中宝剑砍枪轮莫可伤。只听狂风声吼吼又观恶气混茫茫。那个有意思凡弄本事这个专心拜佛取经章。几番驰骋数次张狂。喷云雾闭三光奋怒怀嗔各不良。多时三乘无上法致令百艺苦相将。概众争战多时不分胜负那妖精又解搭包儿。行者又叫:“列位仔细!”太子并众等不知“仔细”之意。那怪滑的一声把四大将与太子一搭包又装将进去只是行者预先知觉走了那妖王得胜回寺又教取绳捆了送在地窖牢封固锁不题。

这行者纵筋斗云起在空中见那怪回兵闭门方才按下祥光立于西山坡上怅望悲啼道:“师父啊!我自从秉教入禅林感荷菩萨脱难深。保你西来求大道相同辅助上雷音。只言平坦羊肠路岂料崔巍怪物侵。百计千方难救你东求西告枉劳心!”大圣正当凄惨之时忽见那西南上一朵彩云坠地满山头大雨缤纷有人叫道:“悟空认得我么?”行者急走前看处那个人:大耳横颐方面相肩查腹满身躯胖。一腔春意喜盈盈两眼秋波光荡荡。敞袖飘然福气多芒鞋洒落精神壮。极乐场中第一尊南无弥勒笑和尚。行者见了连忙下拜道:“东来佛祖那里去?弟子失回避了万罪!万罪!”佛祖道:“我此来专为这小雷音妖怪也。”行者道:“多蒙老爷盛德大恩。敢问那妖是那方怪物何处精魔不知他那搭包儿是件甚么宝贝烦老爷指示指示。”佛祖道:“他是我面前司磬的一个黄眉童儿。

三月三日我因赴元始会去留他在宫看守他把我这几件宝贝拐来假佛成精。那搭包儿是我的后天袋子俗名唤做人种袋。那条狼牙棒是个敲磬的槌儿。”行者听说高叫一声道:“好个笑和尚!你走了这童儿教他诳称佛祖陷害老孙未免有个家法不谨之过!”弥勒道:“一则是我不谨走失人口二则是你师徒们魔障未完故此百灵下界应该受难。我今来与你收他去也。”行者道:“这妖精神通广大你又无些兵器何以收之?”

弥勒笑道:“我在这山坡下设一草庵种一田瓜果在此你去与他索战。交战之时许败不许胜引他到我这瓜田里。我别的瓜都是生的你却变做一个大熟瓜。他来定要瓜吃我却将你与他吃。吃下肚中任你怎么在内摆布他那时等我取了他的搭包儿装他回去。”行者道:“此计虽妙你却怎么认得变的熟瓜?他怎么就肯跟我来此?”弥勒笑道:“我为治世之尊慧眼高明岂不认得你!凭你变作甚物我皆知之但恐那怪不肯跟来耳。我却教你一个法术。”行者道:“他断然是以搭包儿装我怎肯跟来!有何法术可来也?”弥勒笑道:“你伸手来。”行者即舒左手递将过去弥勒将右手食指蘸着口中神水在行者掌上写了一个禁字教他捏着拳头见妖精当面放手他就跟来。

行者揝拳欣然领教一只手轮着铁棒直至山门外高叫道:“妖魔你孙爷爷又来了!可快出来与你见个上下!”小妖又忙忙奔告妖王问道:“他又领多少兵来叫战?”小妖道:“别无甚兵止他一个。”妖王笑道:“那猴儿计穷力竭无处求人断然是送命来也。”随又结束整齐带了宝贝举着那轻软狼牙棒走出站来叫道:“孙悟空今番挣挫不得了!”行者骂道:“泼怪物!我怎么挣挫不得?”妖王道:“我见你计穷力竭无处求人独自个强来支持如今拿住再没个甚么神兵救拔此所以说你挣挫不得也。”行者道:“这怪不知死活!莫说嘴!吃吾一棒!”那妖王见他一只手轮棒忍不住笑道:“这猴儿你看他弄巧!怎么一只手使棒支吾?”行者道:“儿子!你禁不得我两只手打!若是不使搭包子再着三五个也打不过老孙这一只手!”妖王闻言道:“也罢!也罢!我如今不使宝贝只与你实打比个雌雄。”即举狼牙棒上前来斗。孙行者迎着面把拳头一放双手轮棒。那妖精着了禁不思退步果然不弄搭包只顾使棒来赶。行者虚幌一下败阵就走那妖精直赶到西山坡下。

行者见有瓜田打个滚钻入里面即变做一个大熟瓜又熟又甜。那妖精停身四望不知行者那方去了他却赶至庵边叫道:

“瓜是谁人种的?”弥勒变作一个种瓜叟出草庵答道:“大王瓜是小人种的。”妖王道:“可有熟瓜么?”弥勒道:“有熟的。”妖王叫:“摘个熟的来我解渴。”弥勒即把行者变的那瓜双手递与妖王。妖王更不察情到此接过手张口便啃。那行者乘此机会一毂辘钻入咽喉之下等不得好歹就弄手脚抓肠蒯腹翻根头竖蜻蜓任他在里面摆布。那妖精疼得傞牙倈嘴眼泪汪汪把一块种瓜之地滚得似个打麦之场口中只叫:“罢了!

罢了!谁人救我一救!”弥勒却现了本象嘻嘻笑叫道:“孽畜!

认得我么?”那妖抬头看见慌忙跪倒在地双手揉着肚子磕头撞脑只叫:“主人公!饶我命罢!饶我命罢!再不敢了!”弥勒上前一把揪住解了他的后天袋儿夺了他的敲磬槌儿叫:

“孙悟空看我面上饶他命罢。”行者十分恨苦却又左一拳右一脚在里面乱掏乱捣。那怪万分疼痛难忍倒在地下。弥勒又道:“悟空他也彀了你饶他罢。”行者才叫:“你张大口等老孙出来。”那怪虽是肚腹绞痛还未伤心。俗语云人未伤心不得死花残叶落是根枯。他听见叫张口即便忍着疼把口大张。行者方才跳出现了本象急掣棒还要打时早被佛祖把妖精装在袋里斜跨在腰间手执着磬槌骂道:“孽畜!金铙偷了那里去了?”那怪却只要怜生在后天袋内哼哼喷喷的道:

“金铙是孙悟空打破了。”佛祖道:“铙破还我金来。”那怪道:

“碎金堆在殿莲台上哩。”那佛祖提着袋子执着磬槌嘻嘻笑叫道:“悟空我和你去寻金还我。”行者见此法力怎敢违误只得引佛上山回至寺内收取金碴。只见那山门紧闭佛祖使槌一指门开入里看时那些小妖已得知老妖被擒各自收拾囊底都要逃生四散。被行者见一个打一个;见两个打两个把五七百个小妖尽皆打死各现原身都是些山精树怪兽孽禽魔。佛祖将金收攒一处吹口仙气念声咒语即时返本还原复得金铙一副别了行者驾祥云径转极乐世界。

这大圣却才解下唐僧、八戒、沙僧。那呆子吊了几日饿得慌了且不谢大圣却就虾着腰跑到厨房寻饭吃。原来那怪正安排了午饭因行者索战还未得吃。这呆子看见即吃了半锅却拿出两钵头叫师父、师弟们各吃了两碗然后才谢了行者。问及妖怪原由行者把先请祖师龟、蛇后请大圣借太子并弥勒收降之事细陈了一遍。三藏闻言谢之不尽顶礼了诸天道:“徒弟这些神圣困于何所?”行者道:“昨日日值功曹对老孙说都在地窖之内。”叫:“八戒我与你去解脱他等。”

那呆子得食力壮抖擞精神寻着他的钉钯即同大圣到后面打开地窖将众等解了绳请出珍楼之下。三藏披了袈裟朝上一一拜谢。这大圣才送五龙二将回武当送小张太子与四将回蠙城后送二十八宿归天府放揭谛伽蓝各回境。

师徒们却宽住了半日喂饱了白马收拾行囊至次早登程。临行时放上一把火将那些珍楼、宝座、高阁、讲堂俱尽烧为灰烬。这里才无挂无牵逃难去消灾消障脱身行。毕竟不知几时才到大雷音且听下回分解。
------------

第六十七回 拯救驼罗禅性稳 脱离秽污道心清

话说三藏四众躲离了小西天欣然上路。行经个月程途正是春深花放之时见了几处园林皆绿暗一番风雨又黄昏。

三藏勒马道:“徒弟啊天色晚矣往那条路上求宿去?”行者笑道:“师父放心若是没有借宿处我三人都有些本事叫八戒砍草沙和尚扳松老孙会做木匠就在这路上搭个蓬庵好道也住得年把你忙怎的!”八戒道:“哥呀这个所在岂是住场!

满山多虎豹狼虫遍地有魑魅魍魉。白日里尚且难行黑夜里怎生敢宿?”行者道:“呆子!越不长进了!不是老孙海口只这条棒子揝在手里就是塌下天来也撑得住!”

师徒们正然讲论忽见一座山庄不远。行者道:“好了!有宿处了!”长老问:“在何处?”行者指道:“那树丛里不是个人家?我们去借宿一宵明早走路。”长老欣然促马至庄门外下马。只见那柴扉紧闭长老敲门道:“开门开门。”里面有一老者手拖藜杖足踏蒲鞋头顶乌巾身穿素服开了门便问:

“是甚人在此大呼小叫?”三藏合掌当胸躬身施礼道:“老施主贫僧乃东土差往西天取经者。适到贵地天晚特造尊府假宿一宵万望方便方便。”老者道:“和尚你要西行却是去不得啊。此处乃小西天若到大西天路途甚远。且休道前去艰难只这个地方已此难过。”三藏问:“怎么难过?”老者用手指道:“我这庄村西去三十余里有一条稀柿衕山名七绝。”三藏道:“何为七绝?”老者道:“这山径过有八百里满山尽是柿果。

古云柿树有七绝:一益寿二多阴三无鸟巢四无虫五霜叶可玩六嘉实七枝叶肥大故名七绝山。我这敝处地阔人稀那深山亘古无人走到。每年家熟烂柿子落在路上将一条夹石胡同尽皆填满;又被雨露雪霜经霉过夏作成一路污秽。这方人家俗呼为稀屎衕。但刮西风有一股秽气就是淘东圊也不似这般恶臭。如今正值春深东南风大作所以还不闻见也。”三藏心中烦闷不言。行者忍不住高叫道:“你这老儿甚不通便!我等远来投宿你就说出这许多话来唬人!十分你家窄逼没处睡我等在此树下蹲一蹲也就过了此宵何故这般絮聒?”那老者见了他相貌丑陋便也拧住口惊嘬嘬的硬着胆喝了一声用藜杖指定道:“你这厮骨挝脸磕额头塌鼻子凹颉腮毛眼毛睛痨病鬼不知高低尖着个嘴敢来冲撞我老人家!”行者陪笑道:“老官儿你原来有眼无珠不识我这痨病鬼哩!相法云形容古怪石中有美玉之藏。你若以言貌取人干净差了我虽丑便丑却倒有些手段。”老者道:“你是那方人氏?姓甚名谁?有何手段?”行者笑道:“我祖居东胜大神洲花果山前自幼修。身拜灵台方寸祖学成武艺甚全周。也能搅海降龙母善会担山赶日头;缚怪擒魔称第一移星换斗鬼神愁。

偷天转地英名大我是变化无穷美石猴!”老者闻言回嗔作喜躬着身便教:请入寒舍安置。遂此四众牵马挑担一齐进去只见那荆针棘刺铺设两边;二层门是砖石垒的墙壁又是荆棘苫盖入里才是三间瓦房。老者便扯椅安坐待茶又叫办饭。少顷移过桌子摆着许多面筋、豆腐、芋苗、萝白、辣芥、蔓菁、香稻米饭、醋烧葵汤师徒们尽饱一餐。吃毕八戒扯过行者背云:“师兄这老儿始初不肯留宿今返设此盛斋何也?”

行者道:“这个能值多少钱!到明日还要他十果十菜的送我们哩!”八戒道:“不羞!凭你那几句大话哄他一顿饭吃了明日却要跑路他又管待送你怎的?”行者道:“不要忙我自有个处治。(wwW.mianhuatang.la 无弹窗广告)”

不多时渐渐黄昏老者又叫掌灯。行者躬身问道:“公公高姓?”老者道:“姓李。”行者道:“贵地想就是李家庄?”老者道:“不是这里唤做驼罗庄共有五百多人家居住。别姓俱多惟我姓李。”行者道:“李施主府上有何善意赐我等盛斋?”那老者起身道:“才闻得你说会拿妖怪我这里却有个妖怪累你替我们拿拿自有重谢。”行者就朝上唱个喏道:“承照顾了!”

八戒道:“你看他惹祸!听见说拿妖怪就是他外公也不这般亲热预先就唱个喏!”行者道:“贤弟你不知我唱个喏就是下了个定钱他再不去请别人了。”三藏闻言道:“这猴儿凡事便要自专倘或那妖精神通广大你拿他不住可不是我出家人打诳语么?”行者笑道:“师父莫怪等我再问了看。”那老者道:

“还问甚?”行者道:“你这贵处地势清平又许多人家居住更不是偏僻之方有甚么妖精敢上你这高门大户?”老者道:“实不瞒你说我这里久矣康宁。只这三年六月间忽然一阵风起那时人家甚忙打麦的在场上插秧的在田里俱着了慌只说是天变了。谁知风过处有个妖精将人家牧放的牛马吃了猪羊吃了见鸡鹅囫囵咽遇男女夹活吞。自从那次这二年常来伤害。长老啊你若有手段拿了他扫净此土我等决然重谢不敢轻慢。”行者道:“这个却是难拿。”八戒道:“真是难拿难拿!我们乃行脚僧借宿一宵明日走路拿甚么妖精!”老者道:“你原来是骗饭吃的和尚!初见时夸口弄舌说会换斗移星降妖缚怪及说起此事就推却难拿!”行者道:“老儿妖精好拿。只是你这方人家不齐心所以难拿。”老者道:“怎见得人心不齐?”行者道:“妖精搅扰了三年也不知伤害了多少生灵。

我想着每家只出银一两五百家可凑五百两银子不拘到那里也寻一个法官把妖拿了却怎么就甘受他三年磨折?”老者道:“若论说使钱好道也羞杀人!我们那家不花费三五两银子!前年音访着山南里有个和尚请他到此拿妖未曾得胜。”

行者道:“那和尚怎的拿来?”老者道:“那个僧伽披领袈裟。先谈《孔雀》后念《法华》。香焚炉内手把铃拿。正然念处惊动妖邪。风生云起径至庄家。僧和怪斗其实堪夸:一递一拳捣一递一把抓。和尚还相应相应没头。须臾妖怪胜径直返烟霞原来晒干疤。我等近前看光头打的似个烂西瓜!”行者笑道:“这等说吃了亏也。”老者道:“他只拚得一命还是我们吃亏:与他买棺木殡葬又把些银子与他徒弟。那徒弟心还不歇至今还要告状不得干净!”行者道:“再可曾请甚么人拿他?”老者道:“旧年又请了一个道士。”行者道:“那道士怎么拿他?”老者道:“那道士:头戴金冠身穿法衣。令牌敲响符水施为。驱神使将拘到妖魑。狂风滚滚黑雾迷迷。即与道士两个相持。斗到天晚怪返云霓。乾坤清朗朗我等众人齐。出来寻道士渰死在山溪。捞得上来大家看却如一个落汤鸡!”

行者笑道:“这等说也吃亏了。”老者道:“他也只舍得一命我们又使彀闷数钱粮。”行者道:“不打紧不打紧等我替你拿他来。”老者道:“你若果有手段拿得他我请几个本庄长者与你写个文书。若得胜凭你要多少银子相谢半分不少;如若有亏切莫和我等放赖各听天命。”行者笑道:“这老儿被人赖怕了。我等不是那样人快请长者去。”

那老者满心欢喜即命家僮请几个左邻右舍表弟姨兄亲家朋友共有八九位老者都来相见。会了唐僧言及拿妖一事无不欣然。众老问:“是那一位高徒去拿?”行者叉手道:“是我小和尚。”众老悚然道:“不济!不济!那妖精神通广大身体狼犺。你这个长老瘦瘦小小还不彀他填牙齿缝哩!”行者笑道:“老官儿你估不出人来。我小自小结实都是吃了磨刀水的秀气在内哩!”众老见说只得依从道:“长老拿住妖精你要多少谢礼?”行者道:“何必说要甚么谢礼!俗语云说金子幌眼说银子傻白说铜钱腥气!我等乃积德的和尚决不要钱。”

众老道:“既如此说都是受戒的高僧。既不要钱岂有空劳之理!我等各家俱以鱼田为活若果降了妖孽净了地方我等每家送你两亩良田共凑一千亩坐落一处你师徒们在上起盖寺院打坐参禅强似方上云游。”行者又笑道:“越不停当!但说要了田就要养马当差纳粮办草黄昏不得睡五鼓不得眠好倒弄杀人也!”众老道:“诸般不要却将何谢?”行者道:

“我出家人但只是一茶一饭便是谢了。”众老喜道:“这个容易但不知你怎么拿他。”行者道:“他但来我就拿住他。”众老道:“那怪大着哩!上拄天下拄地;来时风去时雾你却怎生近得他?”行者笑道:“若论呼风驾雾的妖精我把他当孙子罢了;若说身体长大有那手段打他!”

正讲处只听得呼呼风响慌得那八九个老者战战兢兢道:“这和尚盐酱口!说妖精妖精就来了!”那老李开了腰门把几个亲戚连唐僧都叫:“进来!进来!妖怪来了!”唬得那八戒也要进去沙僧也要进去。行者两只手扯住两个道:“你们忒不循理!出家人怎么不分内外!站住!不要走!跟我去天井里看看是个甚么妖精。”八戒道:“哥啊他们都是经过帐的风响便是妖来。他都去躲我们又不与他有亲又不相识又不是交契故人看他做甚?”原来行者力量大不容说一把拉在天井里站下。那阵风越大了好风:倒树摧林狼虎忧播江搅海鬼神愁。掀翻华岳三峰石提起乾坤四部洲。村舍人家皆闭户满庄儿女尽藏头。黑云漠漠遮星汉灯火无光遍地幽。慌得那八戒战战兢兢伏之于地把嘴拱开土埋在地下却如钉了钉一般。沙僧蒙着头脸眼也难睁。

行者闻风认怪一霎时风头过处只见那半空中隐隐的两盏灯来即低头叫道:“兄弟们!风过了起来看!”那呆子扯出嘴来抖抖灰土仰着脸朝天一望见有两盏灯光忽失声笑道:“好耍子!好耍子!原来是个有行止的妖精!该和他做朋友!”沙僧道:“这般黑夜又不曾觌面相逢怎么就知好歹?”八戒道:“古人云夜行以烛无烛则止。你看他打一对灯笼引路必定是个好的。”沙僧道:“你错看了那不是一对灯笼是妖精的两只眼亮。”这呆子就唬矮了三寸道:“爷爷呀!眼有这般大啊不知口有多少大哩!”行者道:“贤弟莫怕。你两个护持着师父待老孙上去讨他个口气看他是甚妖精。”八戒道:“哥哥不要供出我们来。”好行者纵身打个唿哨跳到空中执铁棒厉声高叫道:“慢来!慢来!有吾在此!”那怪见了挺住身躯将一根长枪乱舞。行者执了棍势问道:“你是那方妖怪?何处精灵?”那怪更不答应只是舞枪。行者又问又不答只是舞枪。

行者暗笑道:“好是耳聋口哑!不要走!看棍!”那怪更不怕乱舞枪遮拦。在那半空中一来一往一上一下斗到三更时分未见胜败。八戒沙僧在李家天井里看得明白原来那怪只是舞枪遮架更无半分儿攻杀行者一条棒不离那怪的头上。八戒笑道:“沙僧你在这里护持让老猪去帮打帮打莫教那猴子独干这功领头一钟酒。”好呆子就跳起云头赶上就筑那怪物又使一条枪抵住。两条枪就如飞蛇掣电。八戒夸奖道:“这妖精好枪法!不是山后枪乃是缠丝枪也不是马家枪却叫做个软柄枪!”行者道:“呆子莫胡谈!那里有个甚么软柄枪!”八戒道:“你看他使出枪尖来架住我们不见枪柄不知收在何处。”行者道:“或者是个软柄枪。但这怪物还不会说话想是还未归人道阴气还重只怕天明时阳气胜他必要走。但走时一定赶上不可放他。”八戒道:“正是!正是!”

又斗多时不觉东方白那怪不敢恋战回头就走。行者与八戒一齐赶来忽闻得污秽之气旭人乃是七绝山稀柿衕也。八戒道:“是那家淘毛厕哩!哏!臭气难闻!”行者侮着鼻子只叫:“快快赶妖精!快快赶妖精!”那怪物撺过山去现了本象乃是一条红鳞大蟒。你看他:眼射晓星鼻喷朝雾。密密牙排钢剑弯弯爪曲金钩。头戴一条肉角好便似千千块玛瑙攒成;身披一派红鳞却就如万万片胭脂砌就。盘地只疑为锦被飞空错认作虹霓。歇卧处有腥气冲天行动时有赤云罩体。大不大两边人不见东西;长不长一座山跨占南北。八戒道:“原来是这般一个长蛇!若要吃人啊一顿也得五百个还不饱足!”行者道:“那软柄枪乃是两条信菾。我们赶他软了从后打出去!”这八戒纵身赶上将钯便筑。那怪物一头钻进窟里还有七八尺长尾巴丢在外边。八戒放下钯一把挝住道:“着手!

着手!”尽力气往外乱扯莫想扯得动一毫。行者笑道:“呆子!

放他进去自有处置不要这等倒扯蛇。”八戒真个撒了手那怪缩进去了。八戒怨道:“才不放手时半截子已是我们的了!

是这般缩了却怎么得他出来?这不是叫做没蛇弄了?”行者道:“这厮身体狼犺窟穴窄小断然转身不得一定是个照直撺的定有个后门出头。你快去后门外拦住等我在前门外打。”那呆子真个一溜烟跑过山去果见有个孔窟他就扎定脚。还不曾站稳不期行者在前门外使棍子往里一捣那怪物护疼径往后门撺出。八戒未曾防备被他一尾巴打了一跌莫能挣挫得起睡在地下忍疼。行者见窟中无物搴着棍穿进去叫赶妖怪。那八戒听得吆喝自己害羞忍着疼爬起来使钯乱扑。行者见了笑道:“妖怪走了你还扑甚的了?”八戒道:“老猪在此打草惊蛇哩!”行者道:“活呆子!快赶上!”

二人赶过涧去见那怪盘做一团竖起头来张开巨口要吞八戒八戒慌得往后便退。这行者反迎上前被他一口吞之。

八戒捶胸跌脚大叫道:“哥耶!倾了你也!”行者在妖精肚里支着铁棒道:“八戒莫愁我叫他搭个桥儿你看!”那怪物躬起腰来就似一道路东虹八戒道:“虽是象桥只是没人敢走。”行者道:“我再叫他变做个船儿你看!”在肚里将铁棒撑着肚皮。

那怪物肚皮贴地翘起头来就似一只赣保船八戒道:“虽是象船只是没有桅篷不好使风。”行者道:“你让开路等我叫他使个风你看。”又在里面尽着力把铁棒从脊背上一搠将出去约有五七丈长就似一根桅杆。那厮忍疼挣命往前一撺比使风更快撺回旧路下了山有二十余里却才倒在尘埃动荡不得呜呼丧矣。八戒随后赶上来又举钯乱筑。行者把那物穿了一个大洞钻将出来道:“呆子!他死也死了你还筑他怎的?”八戒道:“哥啊你不知我老猪一生好打死蛇?”遂此收了兵器抓着尾巴倒拉将来。

却说那驼罗庄上李老儿与众等对唐僧道:“你那两个徒弟一夜不回断然倾了命也。”三藏道:“决不妨事我们出去看看。”须臾间只见行者与八戒拖着一条大蟒吆吆喝喝前来众人却才欢喜。满庄上老幼男女都来跪拜道:“爷爷!正是这个妖精在此伤人!今幸老爷施法斩怪除邪我辈庶各得安生也!”众家都是感激东请西邀各各酬谢。师徒们被留住五七日苦辞无奈方肯放行。又各家见他不要钱物都办些干粮果品骑骡压马花红彩旗尽来饯行。此处五百人家到有七八百人相送。

一路上喜喜欢欢不时到了七绝山稀柿同口。三藏闻得那般恶秽又见路道填塞道:“悟空似此怎生度得?”行者侮着鼻子道:“这个却难也。”三藏见行者说难便就眼中垂泪。李老儿与众上前道:“老爷勿得心焦。我等送到此处都已约定意思了。令高徒与我们降了妖精除了一庄祸害我们各办虔心另开一条好路送老爷过去。”行者笑道:“你这老儿俱言之欠当。你初然说这山径过有八百里你等又不是大禹的神兵那里会开山凿路!若要我师父过去还得我们着力你们都成不得。”三藏下马道:“悟空怎生着力么!”行者笑道:“眼下就要过山却也是难若说再开条路却又难也。须是还从旧胡同过去只恐无人管饭。”李老儿道:“长老说那里话!凭你四位担搁多少时我等俱养得起怎么说无人管饭!”行者道:“既如此你们去办得两石米的干饭再做些蒸饼馍馍来等我那长嘴和尚吃饱了变了大猪拱开旧路我师父骑在马上我等扶持着管情过去了。”八戒闻言道:“哥哥你们都要图个干净怎么独教老猪出臭?”三藏道:“悟能你果有本事拱开胡同领我过山注你这场头功。”八戒笑道:“师父在上列位施主们都在此休笑话我老猪本来有三十六般变化若说变轻巧华丽飞腾之物委实不能;若说变山变树变石块变土墩变赖象、科猪、水牛、骆驼真个全会。只是身体变得大肚肠越大须是吃得饱了才好干事。”众人道:“有东西!有东西!我们都带得有干粮果品烧饼馉饳在此。原要开山相送的且都拿出来凭你受用。待变化了行动之时我们再着人回去做饭送来。”八戒满心欢喜脱了皂直裰丢了九齿钯对众道:“休笑话看老猪干这场臭功。”好呆子捻着诀摇身一变果然变做一个大猪真个是嘴长毛短半脂膘自幼山中食药苗。黑面环睛如日月圆头大耳似芭蕉。修成坚骨同天寿炼就粗皮比铁牢。齆齆鼻音呱诂叫喳喳喉响喷喁哮。白蹄四只高千尺剑鬣长身百丈饶。从见人间肥豕彘未观今日老猪魈。唐僧等众齐称赞羡美天蓬法力高。孙行者见八戒变得如此即命那些相送人等快将干粮等物推攒一处叫八戒受用。那呆子不分生熟一涝食之却上前拱路。行者叫沙僧脱了脚好生挑担请师父稳坐雕鞍他也脱了靴鞋吩咐众人回去:“若有情快早送些饭来与我师弟接力。”那些人有七八百相送随行多一半有骡马的飞星回庄做饭;还有三百人步行的立于山下遥望他行。原来此庄至山有三十余里待回取饭来又三十余里往回担搁约有百里之遥他师徒们已此去得远了。众人不舍催趱骡马进胡同连夜赶至次日方才赶上叫道:“取经的老爷慢行慢行!我等送饭来也!”长老闻言谢之不尽道:“真是善信之人!”叫八戒住了再吃些饭食壮神。那呆子拱了两日正在饥饿之际那许多人何止有七八石饭食他也不论米饭、面饭收积来一涝用之饱餐一顿却又上前拱路。三藏与行者、沙僧谢了众人分手两别。正是:驼罗庄客回家去八戒开山过同来。

三藏心诚神力拥悟空法显怪魔衰。千年稀柿今朝净七绝胡同此日开。六欲尘情皆剪绝平安无阻拜莲台。这一去不知还有多少路程还遇甚么妖怪且听下回分解。
------------

第六十八回 朱紫国唐僧论前世 孙行者施为三折肱

善正万缘收名誉传扬四部洲。mianhuatang.la [棉花糖小说网]智慧光明登彼岸飕飕叆叆云生天际头。诸佛共相酬永住瑶台万万秋。打破人间蝴蝶梦休休涤净尘氛不惹愁。话表三藏师徒洗污秽之胡同上逍遥之道路光阴迅又值炎天正是:海榴舒锦弹荷叶绽青盘。两路绿杨藏乳燕行人避暑扇摇绔。进前行处忽见有一城池相近。三藏勒马叫:“徒弟们你看那是甚么去处?”行者道:“师父原来不识字亏你怎么领唐王旨意离朝也!”三藏道:

“我自幼为僧千经万典皆通怎么说我不识字?”行者道:“既识字怎么那城头上杏黄旗明书三个大字就不认得却问是甚去处何也?”三藏喝道:“这泼猴胡说!那旗被风吹得乱摆纵有字也看不明白!”行者道:“老孙偏怎看见?”八戒沙僧道:“师父莫听师兄捣鬼。这般遥望城池尚不明白如何就见是甚字号?”行者道:“却不是朱紫国三字?”三藏道:“朱紫国必是西邦王位却要倒换关文。”行者道:“不消讲了。”

不多时至城门下马过桥入进三层门里真个好个皇州!

但见:门楼高耸垛迭齐排。周围活水通流南北高山相对。六街三市货资多万户千家生意盛。果然是个帝王都会处天府大京城。绝域梯航至遐方玉帛盈。形胜连山远宫垣接汉清。

三关严锁钥万古乐升平。师徒们在那大街市上行时但见人物轩昂衣冠齐整言语清朗真不亚大唐世界。那两边做买做卖的忽见猪八戒相貌丑陋沙和尚面黑身长孙行者脸毛额廓丢了买卖都来争看。三藏只叫:“不要撞祸!低着头走!”

八戒遵依把个莲蓬嘴揣在怀里沙僧不敢仰视惟行者东张西望紧随唐僧左右。那些人有知事的看看儿就回去了。有那游手好闲的并那顽童们烘烘笑笑都上前抛瓦丢砖与八戒作戏。唐僧捏着一把汗只教:“莫要生事!”那呆子不敢抬头。

不多时转过隅头忽见一座门墙上有会同馆三字。唐僧道:“徒弟我们进这衙门去也。”行者道:“进去怎的?”唐僧道:

“会同馆乃天下通会通同之所我们也打搅得且到里面歇下。

待我见驾倒换了关文再赶出城走路。”八戒闻言掣出嘴来把那些随看的人唬倒了数十个他上前道:“师父说的是我们且到里边藏下免得这伙鸟人吵嚷。”遂进馆去那些人方渐渐而退。

却说那馆中有两个馆使乃是一正一副都在厅上查点人夫要往那里接官忽见唐僧来到个个心惊齐道:“是甚么人?是甚么人?往那里走?”三藏合掌道:“贫僧乃东土大唐驾下差往西天取经者今到宝方不敢私过有关文欲倒验放行权借高衙暂歇。”那两个馆使听言屏退左右一个个整冠束带下厅迎上相见即命打扫客房安歇教办清素支应三藏谢了。二官带领人夫出厅而去。手下人请老爷客房安歇三藏便走行者恨道:“这厮惫懒!怎么不让老孙在正厅?”三藏道:“他这里不服我大唐管属又不与我国相连况不时又有上司过客往来所以不好留此相待。”行者道:“这等说我偏要他相待!”正说处有管事的送支应来乃是一盘白米、一盘白面、两把青菜、四块豆腐、两个面筋、一盘干笋、一盘木耳。三藏教徒弟收了谢了管事的管事的道:“西房里有干净锅灶柴火方便请自去做饭。”三藏道:“我问你一声国王可在殿上么?”

管事的道:“我万岁爷爷久不上朝今日乃黄道良辰正与文武多官议出黄榜。你若要倒换关文趁此急去还赶上。到明日就不能彀了不知还有多少时伺候哩。”三藏道:“悟空你们在此安排斋饭等我急急去验了关文回来吃了走路。”八戒急取出袈裟关文。三藏整束了进朝只是吩咐徒弟们切不可出外去生事。

不一时已到五凤楼前说不尽那殿阁峥嵘楼台壮丽。直至端门外烦奏事官转达天廷欲倒验关文。那黄门官果至玉阶前启奏道:“朝门外有东土大唐钦差一员僧前往西天雷音寺拜佛求经欲倒换通关文牒听宣。”国王闻言喜道:“寡人久病不曾登基今上殿出榜招医就有高僧来国!”即传旨宣至阶下三藏即礼拜俯伏。国王又宣上金殿赐坐命光禄寺办斋三藏谢了恩将关文献上。国王看毕十分欢喜道:“法师你那大唐几朝君正?几辈臣贤?至于唐王因甚作疾回生着你远涉山川求经?”这长老因问即欠身合掌道:“贫僧那里三皇治世五帝分伦。尧舜正位禹汤安民。成周子众各立乾坤。倚强欺弱分国称君。邦君十八分野边尘。后成十二宇宙安淳。

因无车马却又相吞。七雄争胜六国归秦。天生鲁沛各怀不仁。江山属汉约法钦遵。汉归司马晋又纷纭。南北十二宋齐梁陈。列祖相继大隋绍真。赏花无道涂炭多民。我王李氏国号唐君。高祖晏驾当今世民。河清海晏大德宽仁。兹因长安城北有个怪水龙神刻减甘雨应该损身。夜间托梦告王救迍。王言准赦早召贤臣。款留殿内慢把棋轮。时当日午那贤臣梦斩龙身。”国王闻言忽作呻吟之声问道:“法师那贤臣是那邦来者?”三藏道:“就是我王驾前丞相姓魏名徵。他识天文知地理辨阴阳乃安邦立国之大宰辅也。因他梦斩了泾河龙王那龙王告到阴司说我王许救又杀之故我王遂得促病渐觉身危。魏徵又写书一封与我王带至冥司寄与酆都城判官崔玨。少时唐王身死至三日复得回生。亏了魏徵感崔判官改了文书加王二十年寿。今要做水陆大会故遣贫僧远涉道途询求诸国拜佛祖取大乘经三藏度孽苦升天也。”那国王又呻吟叹道:“诚乃是天朝大国君正臣贤!似我寡人久病多时并无一臣拯救。”长老听说偷睛观看见那皇帝面黄肌瘦形脱神衰。长老正欲启问有光禄寺官奏请唐僧奉斋。王传旨教:“在披香殿连朕之膳摆下与法师同享。”

三藏谢了恩与王同进膳进斋不题。

却说行者在会同馆中着沙僧安排茶饭并整治素菜。沙僧道:“茶饭易煮蔬菜不好安排。”行者问道:“如何?”沙僧道:

“油盐酱醋俱无也。”行者道:“我这里有几文衬钱教八戒上街买去。”那呆子躲懒道:“我不敢去嘴脸欠俊恐惹下祸来师父怪我。”行者道:“公平交易又不化他又不抢他何祸之有!”八戒道:“你才不曾看见獐智?在这门前扯出嘴来把人唬倒了十来个;若到闹市丛中也不知唬杀多少人是!”行者道:

“你只知闹市丛中你可曾看见那市上卖的是甚么东西?”八戒道:“师父只教我低着头莫撞祸实是不曾看见。”行者道:“酒店、米铺、磨坊并绫罗杂货不消说着然又好茶房、面店大烧饼、大馍馍饭店又有好汤饭好椒料、好蔬菜与那异品的糖糕、蒸酥、点心、卷子、油食、蜜食无数好东西我去买些儿请你如何?”那呆子闻说口内流涎喉咙里啯啯的咽唾跳起来道:“哥哥!这遭我扰你待下次趱钱我也请你回席。”行者暗笑道:“沙僧好生煮饭等我们去买调和来。”沙僧也知是耍呆子只得顺口应承道:“你们去须是多买些吃饱了来。”那呆子捞个碗盏拿了就跟行者出门。有两个在官人问道:“长老那里去?”行者道:“买调和。”那人道:“这条街往西去转过拐角鼓楼那郑家杂货店凭你买多少油盐酱醋、姜椒茶叶俱全。”

他二人携手相搀径上街西而去。行者过了几处茶房几家饭店当买的不买当吃的不吃。八戒叫道:“师兄这里将就买些用罢。”那行者原是耍他那里肯买道:“贤弟你好不经纪!再走走拣大的买吃。”两个人说说话儿又领了许多人跟随争看。不时到了鼓楼边只见那楼下无数人喧嚷挤挤挨挨填街塞路。八戒见了道:“哥哥我不去了那里人嚷得紧只怕是拿和尚的。又况是面生可疑之人拿了去怎的了?”行者道:“胡谈!和尚又不犯法拿我怎的?我们走过去到郑家店买些调和来。”八戒道:“罢罢罢!我不撞祸。这一挤到人丛里把耳朵躧了两躧唬得他跌跌爬爬跌死几个我倒偿命哩!”行者道:“既然如此你在这壁根下站定等我过去买了回来与你买素面烧饼吃罢。”那呆子将碗盏递与行者把嘴拄着墙根背着脸死也不动。这行者走至楼边果然挤塞直挨入人丛里听时原来是那皇榜张挂楼下故多人争看。行者挤到近处闪开火眼金睛仔细看时那榜上却云:“朕西牛贺洲朱紫国王自立业以来四方平服百姓清安。近因国事不祥沉疴伏枕淹延日久难痊。本国太医院屡选良方未能调治。今出此榜文普招天下贤士。不拘北往东来中华外国若有精医药者请登宝殿疗理朕躬。稍得病愈愿将社稷平分决不虚示。为此出给张挂须至榜者。”览毕满心欢喜道:“古人云行动有三分财气。早是不在馆中呆坐。即此不必买甚调和且把取经事宁耐一日等老孙做个医生耍耍。”好大圣弯倒腰丢了碗盏拈一撮土往上洒去念声咒语使个隐身法轻轻的上前揭了榜又朝着巽地上吸口仙气吹来那阵旋风起处他却回身径到八戒站处只见那呆子嘴拄着墙根却是睡着了一般。行者更不惊他将榜文折了轻轻揣在他怀里拽转步先往会同馆去了不题。

却说那楼下众人见风起时各各蒙头闭眼。不觉风过时没了皇榜众皆悚惧。那榜原有十二个太监十二个校尉早朝领出才挂不上三个时辰被风吹去战兢兢左右追寻忽见猪八戒怀中露出个纸边儿来众人近前道:“你揭了榜来耶?”那呆子猛抬头把嘴一噘唬得那几个校尉踉踉蹡蹡跌倒在地。

他却转身要走又被面前几个胆大的扯住道:“你揭了招医的皇榜还不进朝医治我万岁去却待何往?”那呆子慌慌张张道:“你儿子便揭了皇榜!你孙子便会医治!”校尉道:“你怀中揣的是甚?”呆子却才低头看时真个有一张字纸展开一看咬着牙骂道:“那猢狲害杀我也!”恨一声便要扯破早被众人架住道:“你是死了!此乃当今国王出的榜文谁敢扯坏?你既揭在怀中必有医国之手快同我去!”八戒喝道:“汝等不知这榜不是我揭的是我师兄孙悟空揭的。他暗暗揣在我怀中他却丢下我去了。若得此事明白我与你寻他去。”众人道:“说甚么乱话现钟不打打铸钟?你现揭了榜文教我们寻谁!不管你!扯了去见主上!”那伙人不分清白将呆子推推扯扯。这呆子立定脚就如生了根一般十来个人也弄他不动。八戒道:

“汝等不知高低!再扯一会扯得我呆性子了你却休怪!”

不多时闹动了街人将他围绕内有两个年老的太监道:

“你这相貌稀奇声音不对是那里来的这般村强?”八戒道:

“我们是东土差往西天取经的我师父乃唐王御弟法师却才入朝倒换关文去了。我与师兄来此买办调和我见楼下人多未曾敢去是我师兄教我在此等候。他原来见有榜文弄阵旋风揭了暗揣我怀内先去了。”那太监道:“我头前见个白面胖和尚径奔朝门而去想就是你师父?”八戒道:“正是正是。”太监道:“你师兄往那里去了?”八戒道:“我们一行四众师父去倒换关文我三众并行囊马匹俱歇在会同馆。师兄弄了我他先回馆中去了。”太监道:“校尉不要扯他我等同到馆中便知端的。”八戒道:“你这两个奶奶知事。”众校尉道:“这和尚委不识货!怎么赶着公公叫起奶奶来耶?”八戒笑道:“不羞!你这反了阴阳的!他二位老妈妈儿不叫他做婆婆奶奶倒叫他做公公!”众人道:“莫弄嘴!快寻你师兄去。”那街上人吵吵闹闹何止三五百共扛到馆门。八戒道:“列位住了我师兄却不比我任你们作戏他却是个猛烈认真之士。汝等见了须要行个大礼叫他声孙老爷他就招架了。不然啊他就变了嘴脸这事却弄不成也。”众太监校尉俱道:“你师兄果有手段医好国王他也该有一半江山我等合该下拜。”

那些闲杂人都在门外喧哗八戒领着一行太监校尉径入馆中只听得行者与沙僧在客房里正说那揭榜之事耍笑哩。八戒上前扯住乱嚷道:“你可成个人!哄我去买素面、烧饼、馍馍我吃原来都是空头!又弄旋风揭了甚么皇榜暗暗的揣在我怀里拿我装胖!这可成个弟兄!”行者笑道:“你这呆子想是错了路走向别处去。我过鼓楼买了调和急回来寻你不见我先来了在那里揭甚皇榜?”八戒道:“现在看榜的官员在此。”说不了只见那几个太监校尉朝上礼拜道:孙老爷今日我王有缘天遣老爷下降是必大展经纶手微施三折肱治得我王病愈江山有分社稷平分也。”行者闻言正了声色接了八戒的榜文对众道:“你们想是看榜的官么?”太监叩头道:

“奴婢乃司礼监内臣这几个是锦衣校尉。”行者道:“这招医榜委是我揭的故遣我师弟引见。既然你主有病常言道药不跟卖病不讨医。你去教那国王亲来请我我有手到病除之功。”太监闻言无不惊骇校尉道:“口出大言必有度量。我等着一半在此哑请着一半入朝启奏。”当分了四个太监六个校尉更不待宣召径入朝当阶奏道:“主公万千之喜!”那国王正与三藏膳毕清谈忽闻此奏问道:“喜自何来?”太监奏道:“奴婢等早领出招医皇榜鼓楼下张挂有东土大唐远来取经的一个圣僧孙长老揭了现在会同馆内要王亲自去请他他有手到病除之功故此特来启奏。”国王闻言满心欢喜就问唐僧道:“法师有几位高徒?”三藏合掌答曰:“贫僧有三个顽徒。”国王问:“那一位高徒善医?”三藏道:“实不瞒陛下说我那顽徒俱是山野庸才只会挑包背马转涧寻波带领贫僧登山涉岭或者到峻险之处可以伏魔擒怪捉虎降龙而已更无一个能知药性者。”国王道:“法师何必太谦?朕当今日登殿幸遇法师来朝诚天缘也。高徒既不知医他怎肯揭我榜文教寡人亲迎?断然有医国之能也。”叫:“文武众卿寡人身虚力怯不敢乘辇;汝等可替寡人俱到朝外敦请孙长老看朕之病。汝等见他切不可轻慢称他做神僧孙长老皆以君臣之礼相见。”那众臣领旨与看榜的太监、校尉径至会同馆排班参拜。唬得那八戒躲在厢房沙僧闪于壁下。那大圣看他坐在当中端然不动八戒暗地里怨恶道:“这猢狲活活的折杀也!怎么这许多官员礼拜更不还礼也不站将起来!”不多时礼拜毕分班启奏道:“上告神僧孙长老我等俱朱紫国王之臣今奉王旨敬以洁礼参请神僧入朝看病。”行者方才立起身来对众道:“你王如何不来?”众臣道:“我王身虚力怯不敢乘辇特令臣等行代君之礼拜请神僧也。”行者道:“既如此说列位请前行我当随至。”众臣各依品从作队而走。行者整衣而起八戒道:“哥哥切莫攀出我们来。”行者道:“我不攀你只要你两个与我收药。”沙僧道:“收甚么药?”行者道:“凡有人送药来与我照数收下待我回来取用。”二人领诺不题。

这行者即同多官顷间便到。众臣先走奏知那国王高卷珠帘闪龙睛凤目开金口御言便问:“那一位是神僧孙长老?”

行者进前一步厉声道:“老孙便是。”那国王听得声音凶狠又见相貌刁钻唬得战兢兢跌在龙床之上。慌得那女官内宦急扶入宫中道:“唬杀寡人也!”众官都嗔怨行者道:“这和尚怎么这等粗鲁村疏!怎敢就擅揭榜!”行者闻言笑道:“列位错怪了我也。若象这等慢人你国王之病就是一千年也不得好。”

众臣道:“人生能有几多阳寿?就一千年也还不好?”行者道:

“他如今是个病君死了是个病鬼再转世也还是个病人却不是一千年也还不好?”众臣怒曰:“你这和尚甚不知礼!怎么敢这等满口胡柴!”行者笑道:“不是胡柴你都听我道来:医门理法至微玄大要心中有转旋。望闻问切四般事缺一之时不备全:第一望他神气色润枯肥瘦起和眠;第二闻声清与浊听他真语及狂言;三问病原经几日如何饮食怎生便;四才切脉明经络浮沉表里是何般。我不望闻并问切今生莫想得安然。”

那两班文武丛中有太医院官一闻此言对众称扬道:“这和尚也说得有理。就是神仙看病也须望闻问切谨合着神圣功巧也。”众官依此言着近侍传奏道:“长老要用望闻问切之理方可认病用药。”那国王睡在龙床上声声唤道:“叫他去罢!寡人见不得生人面了!”近侍的出宫来道:“那和尚我王旨意教你去罢见不得生人面哩。”行者道:“若见不得生人面啊我会悬丝诊脉。”众官暗喜道:“悬丝诊脉我等耳闻不曾眼见。再奏去来。”那近侍的又入宫奏道:“主公那孙长老不见主公之面他会悬丝诊脉。”国王心中暗想道:“寡人病了三年未曾试此宣他进来。”近侍的即忙传出道:“主公已许他悬丝诊脉快宣孙长老进宫诊视。”行者却就上了宝殿唐僧迎着骂道:“你这泼猴害了我也!”行者笑道:“好师父我倒与你壮观你返说我害你?”三藏喝道:“你跟我这几年那曾见你医好谁来!你连药性也不知医书也未读怎么大胆撞这个大祸!”行者笑道:

“师父你原来不晓得。我有几个草头方儿能治大病管情医得他好便是。就是医死了也只问得个庸医杀人罪名也不该死你怕怎的!不打紧不打紧你且坐下看我的脉理如何。”长老又道:“你那曾见《素问》、《难经》、《本草》、《脉诀》是甚般章句怎生注解就这等胡说散道会甚么悬丝诊脉!”行者笑道:

“我有金线在身你不曾见哩。”即伸手下去尾上拔了三根毫毛捻一把叫声“变!”即变作三条丝线每条各长二丈四尺按二十四气托于手内对唐僧道:“这不是我的金线?”近侍宦官在旁道:“长老且休讲口请入宫中诊视去来。”行者别了唐僧随着近侍入宫看病。正是那:心有秘方能治国内藏妙诀注长生。毕竟这去不知看出甚么病来用甚么药品。欲知端的且听下回分解。
------------

第六十九回 心主夜间修药物 君王筵上论妖邪

话表孙大圣同近侍宦官到于皇宫内院直至寝宫门外立定将三条金线与宦官拿入里面吩咐:“教内宫妃后或近侍太监先系在圣躬左手腕下按寸关尺三部上却将线头从窗棂儿穿出与我。”真个那宦官依此言请国王坐在龙床按寸关尺以金线一头系了一头理出窗外。行者接了线头以自己右手大指先托着食指看了寸脉;次将中指按大指看了关脉;又将大指托定无名指看了尺脉;调停自家呼吸分定四气五郁、七表八里九候、浮中沉、沉中浮辨明了虚实之端;又教解下左手依前系在右手腕下部位。行者即以左手指一一从头诊视毕却将身抖了一抖把金线收上身来厉声高呼道:“陛下左手寸脉强而紧关脉涩而缓尺脉芤且沉;右手寸脉浮而滑关脉迟而结尺脉数而牢。夫左寸强而紧者中虚心痛也;关涩而缓者汗出肌麻也;尺芤而沉者小便赤而大便带血也。右手寸脉浮而滑者内结经闭也;关迟而结者宿食留饮也;尺数而牢者烦满虚寒相持也。诊此贵恙是一个惊恐忧思号为双鸟失群之证。”那国王在内闻言满心欢喜打起精神高声应道:“指下明白!指下明白!果是此疾!请出外面用药来也。”大圣却才缓步出宫。早有在旁听见的太监已先对众报知。须臾行者出来唐僧即问如何行者道:“诊了脉如今对证制药哩。”众官上前道:“神僧长老适才说双鸟失群之证何也?”行者笑道:“有雌雄二鸟原在一处同飞忽被暴风骤雨惊散雌不能见雄雄不能见雌雌乃想雄雄亦想雌:这不是双鸟失群也?”

众官闻说齐声喝采道:“真是神僧!真是神医!”称赞不已。当有太医官问道:“病势已看出矣但不知用何药治之?”行者道:

“不必执方见药就要。”医官道:“经云药有八百八味人有四百四病。病不在一人之身药岂有全用之理!如何见药就要?”

行者道:“古人云药不执方合宜而用故此全征药品而随便加减也。”那医官不复再言即出朝门之外差本衙当值之人遍晓满城生熟药铺即将药品每味各办三斤送与行者。行者道:“此间不是制药处可将诸药之数并制药一应器皿都送入会同馆交与我师弟二人收下。”医官听命即将八百八味每味三斤及药碾、药磨、药罗、药乳并乳钵、乳槌之类都送至馆中一一交付收讫。

行者往殿上请师父同至馆中制药。那长老正自起身忽见内宫传旨教阁下留住法师同宿文华殿待明朝服药之后病痊酬谢倒换关文送行。三藏大惊道:“徒弟啊此意是留我做当头哩。若医得好欢喜起送;若医不好我命休矣。你须仔细上心精虔制度也!”行者笑道:“师父放心在此受用老孙自有医国之手。”

好大圣别了三藏辞了众臣径至馆中。八戒迎着笑道:

“师兄我知道你了。”行者道:“你知甚么?”八戒道:“知你取经之事不果欲作生涯无本今日见此处富庶设法要开药铺哩。”行者喝道:“莫胡说!医好国王得意处辞朝走路开甚么药铺!”八戒道:“终不然这八百八味药每味三斤共计二千四百二十四斤只医一人能用多少?不知多少年代方吃得了哩!”行者道:“那里用得许多?他那太医院官都是些愚盲之辈所以取这许多药品教他没处捉摸不知我用的是那几味难识我神妙之方也。”正说处只见两个馆使当面跪下道:“请神僧老爷进晚斋。”行者道:“早间那般待我如今却跪而请之何也?”馆使叩头道:“老爷来时下官有眼无珠不识尊颜。今闻老爷大展三折之肱治我一国之主若主上病愈老爷江山有分我辈皆臣子也礼当拜请。”行者见说欣然登堂上坐八戒、沙僧分坐左右摆上斋来。沙僧便问道:“师兄师父在那里哩?”行者笑道:“师父被国王留住作当头哩只待医好了病方才酬谢送行。”沙僧又问:“可有些受用么?”行者道:“国王岂无受用!我来时他已有三个阁老陪侍左右请入文华殿去也。”

八戒道:“这等说还是师父大哩。他倒有阁老陪侍我们只得两个馆使奉承。且莫管他让老猪吃顿饱饭也。”兄弟们遂自在受用一番。

天色已晚行者叫馆使:“收了家火多办些油蜡我等到夜静时方好制药。”馆使果送若干油蜡各命散讫。至半夜天街人静万籁无声。八戒道:“哥哥制何药?赶早干事。我瞌睡了。”行者道:“你将大黄取一两来碾为细末。”沙僧乃道:

“大黄味苦性寒无毒其性沉而不浮其用走而不守夺诸郁而无壅滞定祸乱而致太平名之曰将军。此行药耳但恐久病虚弱不可用此。”行者笑道:“贤弟不知此药利痰顺气荡肚中凝滞之寒热。你莫管我你去取一两巴豆去壳去膜捶去油毒碾为细末来。”八戒道:“巴豆味辛性热有毒削坚积荡肺腑之沉寒通闭塞利水谷之道路乃斩关夺门之将不可轻用。”行者道:“贤弟你也不知此药破结宣肠能理心膨水胀。

快制来我还有佐使之味辅之也。”他二人即时将二药碾细道:

“师兄还用那几十味?”行者道:“不用了。”八戒道:“八百八味每味三斤只用此二两诚为起夺人了。”行者将一个花磁盏子道:“贤弟莫讲你拿这个盏儿将锅脐灰刮半盏过来。”八戒道:“要怎的?”行者道:“药内要用。”沙僧道:“小弟不曾见药内用锅灰。”行者道:“锅灰名为百草霜能调百病你不知道。”

那呆子真个刮了半盏又碾细了。行者又将盏子递与他道:

“你再去把我们的马尿等半盏来。”八戒道:“要他怎的?”行者道:“要丸药。”沙僧又笑道:“哥哥这事不是耍子。马尿腥臊如何入得药品?我只见醋糊为丸陈米糊为丸炼蜜为丸或只是清水为丸那曾见马尿为丸?那东西腥腥臊臊脾虚的人一闻就吐;再服巴豆大黄弄得人上吐下泻可是耍子?”行者道:

“你不知就里我那马不是凡马他本是西海龙身。若得他肯去便溺凭你何疾服之即愈但急不可得耳。”八戒闻言真个去到马边。那马斜伏地下睡哩呆子一顿脚踢起衬在肚下等了半会全不见撒尿。他跑将来对行者说:“哥啊且莫去医皇帝且快去医医马来。那亡人干结了莫想尿得出一点儿!”行者笑道:“我和你去。”沙僧道:“我也去看看。”三人都到马边那马跳将起来口吐人言厉声高叫道:“师兄你岂不知?我本是西海飞龙因为犯了天条观音菩萨救了我将我锯了角退了鳞变作马驮师父往西天取经将功折罪。我若过水撒尿水中游鱼食了成龙;过山撒尿山中草头得味变作灵芝仙僮采去长寿。我怎肯在此尘俗之处轻抛却也?”行者道:“兄弟谨言此间乃西方国王非尘俗也亦非轻抛弃也。常言道众毛攒裘要与本国之王治病哩。医得好时大家光辉不然恐惧不得善离此地也。”那马才叫声“等着!”你看他往前扑了一扑往后蹲了一蹲咬得那满口牙龁支支的响喨仅努出几点儿将身立起。八戒道:“这个亡人!就是金汁子再撒些儿也罢!”那行者见有少半盏道:“彀了!彀了!拿去罢。”沙僧方才欢喜。

三人回至厅上把前项药饵搅和一处搓了三个大丸子。行者道:“兄弟忒大了。”八戒道:“只有核桃大若论我吃还不彀一口哩!”遂此收在一个小盒儿里。兄弟们连衣睡下一夜无词。

早是天晓却说那国王耽病设朝请唐僧见了即命众官快往会同馆参拜神僧孙长老取药去。多官随至馆中对行者拜伏于地道:“我王特命臣等拜领妙剂。”行者叫八戒取盒儿揭开盖子递与多官。(WWW.mianhuatang.la 好看的小说)多官启问:“此药何名?好见王回话。”行者道:“此名乌金丹。”八戒二人暗中作笑道:“锅灰拌的怎么不是乌金!”多官又问道:“用何引子?”行者道:“药引儿两般都下得。有一般易取者乃六物煎汤送下。”多官问:“是何六物?”行者道:“半空飞的老鸦屁紧水负的鲤鱼尿王母娘娘搽脸粉老君炉里炼丹灰玉皇戴破的头巾要三块还要五根困龙须:

六物煎汤送此药你王忧病等时除。”多官闻言道:“此物乃世间所无者请问那一般引子是何?”行者道:“用无根水送下。”

众官笑道:“这个易取。”行者道:“怎见得易取?”多官道:“我这里人家俗论;若用无根水将一个碗盏到井边或河下舀了水急转步更不落地亦不回头到家与病人吃药便是。”行者道:“井中河内之水俱是有根的。我这无根水非此之论乃是天上落下者不沾地就吃才叫做无根水。”多官又道:“这也容易。等到天阴下雨时再吃药便罢了。”遂拜谢了行者将药持回献上。国王大喜即命近侍接上来。看了道:“此是甚么丸子?”多官道:“神僧说是乌金丹用无根水送下。”国王便教宫人取无根水众官道:“神僧说无根水不是井河中者乃是天上落下不沾地的才是。”国王即唤当驾官传旨教请法官求雨。

众官遵依出榜不题。

却说行者在会同馆厅上叫猪八戒道:“适间允他天落之水才可用药此时急忙怎么得个雨水?我看这王倒也是个大贤大德之君我与你助他些儿雨下药如何?”八戒道:“怎么样助?”行者道:“你在我左边立下做个辅星。”又叫沙僧“你在我右边立下做个弼宿等老孙助他些无根水儿。”好大圣步了罡诀念声咒语早见那正东上一朵乌云渐近于头顶上。叫道:“大圣东海龙王敖广来见。”行者道:“无事不敢捻烦请你来助些无根水与国王下药。”龙王道:“大圣呼唤时不曾说用水小龙只身来了不曾带得雨器亦未有风云雷电怎生降雨?”行者道:“如今用不着风云雷电亦不须多雨只要些须引药之水便了。”龙王道:“既如此待我打两个喷涕吐些涎津溢与他吃药罢。”行者大喜道:“最好!最好!不必迟疑趁早行事。”那老龙在空中渐渐低下乌云直至皇宫之上隐身潜象噀一口津唾遂化作甘霖。那满朝官齐声喝采道:“我主万千之喜!天公降下甘雨来也!”国王即传旨教:“取器皿盛着不拘宫内外及官大小都要等贮仙水拯救寡人。”你看那文武多官并三宫六院妃嫔与三千彩女八百娇娥一个个擎杯托盏举碗持盘等接甘雨。那老龙在半空运化津涎不离了王宫前后将有一个时辰龙王辞了大圣回海。众臣将杯盂碗盏收来也有等着一点两点者也有等着三点五点者也有一点不曾等着者共合一处约有三盏之多总献至御案。真个是异香满袭金銮殿佳味熏子庭!

那国王辞了法师将着乌金丹并甘雨至宫中先吞了一丸吃了一盏甘雨;再吞了一丸又饮了一盏甘雨;三次三丸俱吞了三盏甘雨俱送下。不多时腹中作响如辘轳之声不绝即取净桶连行了三五次服了些米饮敧倒在龙床之上。

有两个妃子将净桶捡看说不尽那秽污痰涎内有糯米饭块一团。妃子近龙床前来报:“病根都行下来也!”国王闻此言甚喜又进一次米饭。少顷渐觉心胸宽泰气血调和就精神抖擞脚力强健。下了龙床穿上朝服即登宝殿见了唐僧辄倒身下拜。那长老忙忙还礼。拜毕以御手搀着便教阁下:“快具简帖帖上写朕再拜顿字样差官奉请法师高徒三位。一壁厢大开东阁光禄寺排宴酬谢。”多官领旨具简的具简排宴的排宴正是国家有倒山之力霎时俱完。

却说八戒见官投简喜不自胜道:“哥啊果是好妙药!今来酬谢乃兄长之功。”沙僧道:“二哥说那里话!常言道一人有福带挈一屋。我们在此合药俱是有功之人只管受用去再休多话。”咦!你看他弟兄们俱欢欢喜喜径入朝来。众官接引上了东阁早见唐僧、国王、阁老已都在那里安排筵宴哩。

这行者与八戒、沙僧对师父唱了个喏随后众官都至只见那上面有四张素桌面都是吃一看十的筵席;前面有一张荤桌面也是吃一看十的珍馐。左右有四五百张单桌面真个排得齐整:古云珍馐百味美禄千锺。琼膏酥酪锦缕肥红。宝妆花彩艳果品味香浓。斗糖龙缠列狮仙饼锭拖炉摆凤侣。荤有猪羊鸡鹅鱼鸭般般肉素有蔬肴笋芽木耳并蘑菇。几样香汤饼数次透酥糖。滑软黄粱饭清新菰米糊。色色粉汤香又辣般般添换美还甜。君臣举盏方安席名分品级慢传壶。那国王御手擎杯先与唐僧安坐三藏道:“贫僧不会饮酒。”国王道:

“素酒法师饮此一杯何如?”三藏道:“酒乃僧家第一戒。”国王甚不过意道:“法师戒饮却以何物为敬?”三藏道:“顽徒三众代饮罢。”国王却才欢喜转金卮递与行者。行者接了酒对众礼毕吃了一杯。国王见他吃得爽利又奉一杯。行者不辞又吃了。国王笑道:“吃个三宝锺儿。”行者不辞又吃了。国王又叫斟上“吃个四季杯儿。”八戒在旁见酒不到他忍得他啯啯咽唾又见那国王苦劝行者他就叫将起来道:“陛下吃的药也亏了我那药里有马——”这行者听说恐怕呆子走了消息却将手中酒递与八戒。八戒接着就吃却不言语。国王问道:“神僧说药里有马是甚么马?”行者接过口来道:“我这兄弟是这般口敞但有个经验的好方儿他就要说与人。陛下早间吃药内有马兜铃。”国王问众官道:“马兜铃是何品味?能医何证?”时有太医院官在旁道:“主公:兜铃味苦寒无毒定喘消痰大有功。通气最能除血盅补虚宁嗽又宽中。”国王笑道:“用得当!用得当!猪长老再饮一杯。”呆子亦不言语却也吃了个三宝锺。国王又递了沙僧酒也吃了三杯却俱叙坐。

饮宴多时国王又擎大爵奉与行者。行者道:“陛下请坐老孙依巡痛饮决不敢推辞。”国王道:“神僧恩重如山寡人酬谢不尽好歹进此一巨觥朕有话说。”行者道:“有甚话说了老孙好饮。”国王道:“寡人有数载忧疑病被神僧一贴灵丹打通所以就好了。”行者笑道:“昨日老孙看了陛下已知是忧疑之疾但不知忧惊何事?”国王道:“古人云家丑不可外谈奈神僧是朕恩主惟不笑方可告之。”行者道:“怎敢笑话请说无妨。”国王道:“神僧东来不知经过几个邦国?”行者道:“经有五六处。”又问:“他国之后不知是何称呼。”行者道:“国王之后都称为正宫、东宫、西宫。”国王道:“寡人不是这等称呼:将正宫称为金圣宫东宫称为玉圣宫西宫称为银圣宫。现今只有银、玉二后在宫。”行者道:“金圣宫因何不在宫中?”国王滴泪道:“不在已三年矣。”行者道:“向那厢去了?”国王道:“三年前正值端阳之节朕与嫔后都在御花园海榴亭下解粽插艾饮菖蒲雄黄酒看斗龙舟。忽然一阵风至半空中现出一个妖精自称赛太岁说他在麒麟山獬豸洞居住洞中少个夫人访得我金圣宫生得貌美姿娇要做个夫人教朕快早送出。如若三声不献出来就要先吃寡人后吃众臣将满城黎民尽皆吃绝。那时节朕却忧国忧民无奈将金圣宫推出海榴亭外被那妖响一声摄将去了。寡人为此着了惊恐把那粽子凝滞在内况又昼夜忧思不息所以成此苦疾三年。今得神僧灵丹服后行了数次尽是那三年前积滞之物所以这会体健身轻精神如旧。今日之命皆是神僧所赐岂但如泰山之重而已乎!”行者闻得此言满心喜悦将那巨觥之酒两口吞之笑问国王曰:“陛下原来是这等惊忧!今遇老孙幸而获愈但不知可要金圣宫回国?”那国王滴泪道:“朕切切思思无昼无夜但只是没一个能获得妖精的。岂有不要他回国之理!”行者道:“我老孙与你去伏妖邪那时何如?”国王跪下道:“若救得朕后朕愿领三宫九嫔出城为民将一国江山尽付神僧让你为帝。”八戒在旁见出此言行此礼忍不住呵呵大笑道:“这皇帝失了体统!怎么为老婆就不要江山跪着和尚?”行者急上前将国王搀起道:“陛下那妖精自得金圣宫去后这一向可曾再来?”国王道:“他前年五月节摄了金圣宫至十月间来要取两个宫娥是说伏侍娘娘朕即献出两个。至旧年三月间又来要两个宫娥;七月间又要去两个;今年二月里又要去两个;不知到几时又要来也。”行者道:“似他这等频来你们可怕他么?”国王道:“寡人见他来得多遭一则惧怕二来又恐有伤害之意旧年四月内是朕命工起了一座避妖楼但闻风响知是他来即与二后九嫔入楼躲避。”行者道:“陛下不弃可携老孙去看那避妖楼一番何如?”那国王即将左手携着行者出席众官亦皆起身。猪八戒道:“哥哥你不达理!这般御酒不吃摇席破坐的且去看甚么哩?”国王闻说情知八戒是为嘴即命当驾官抬两张素桌面看酒在避妖楼外伺候。呆子却才不嚷同师父沙僧笑道:“翻席去也。”

一行文武官引导那国王并行者相搀穿过皇宫到了御花园后更不见楼台殿阁。行者道:“避妖楼何在?”说不了只见两个太监拿两根红漆扛子往那空地上掬起一块四方石板。

国王道:“此间便是。这底下有三丈多深槃成的九间朝殿内有四个大缸缸内满注清油点着灯火昼夜不息。寡人听得风响就入里边躲避外面着人盖上石板。”行者笑道:“那妖精还是不害你若要害你这里如何躲得?”正说间只见那正南上呼呼的吹得风响播土扬尘唬得那多官齐声报怨道:“这和尚盐酱口讲起甚么妖精妖精就来了!”慌得那国王丢了行者即钻入地穴唐僧也就跟入众官亦躲个干净。八戒、沙僧也都要躲被行者左右手扯住他两个道“兄弟们不要怕得我和你认他一认看是个甚么妖精。”八戒道:“可是扯淡!认他怎的?众官躲了师父藏了国王避了我们不去了罢炫的是那家世!”那呆子左挣右挣挣不得脱手被行者拿定多时只见那半空里闪出一个妖精。你看他怎生模样:九尺长身多恶狞一双环眼闪金灯。两轮查耳如撑扇四个钢牙似插钉。鬓绕红毛眉竖焰鼻垂精准孔开明髭髯几缕朱砂线颧骨崚嶒满面青。两臂红筋蓝靛手十条尖爪把枪擎。豹皮裙子腰间系赤脚蓬头若鬼形。行者见了道:“沙僧你可认得他?”沙僧道:

“我又不曾与他相识那里认得!”又问:“八戒你可认得他?”

八戒道:“我又不曾与他会茶会酒又不是宾朋邻里我怎么认得他!”行者道:“他却象东岳天齐手下把门的那个醮面金睛鬼。”八戒道:“不是!不是!”行者道:“你怎知他不是?”八戒道:

“我岂不知鬼乃阴灵也一日至晚交申酉戌亥时方出。今日还在巳时那里有鬼敢出来?就是鬼也不会驾云。纵会弄风也只是一阵旋风耳有这等狂风?或者他就是赛太岁也。”行者笑道:“好呆子!倒也有些论头!既如此说你两个护持在此等老孙去问他个名号好与国王救取金圣宫来朝。”八戒道:

“你去自去切莫供出我们来。”行者昂然不答急纵祥光跳将上去。咦!正是:安邦先却君王病守道须除爱恶心。毕竟不知此去到于空中胜败如何怎么擒得妖怪救得金圣宫且听下回分解。
------------

第七十回 妖魔宝放烟沙火 悟空计盗紫金铃

却说那孙行者抖擞神威持着铁棒踏祥光起在空中迎面喝道:“你是那里来的邪魔待往何方猖獗!”那怪物厉声高叫道:“吾党不是别人乃麒麟山獬豸洞赛太岁大王爷爷部下先锋今奉大王令到此取宫女二名伏侍金圣娘娘。你是何人敢来问我!”行者道:“吾乃齐天大圣孙悟空因保东土唐僧西天拜佛路过此国知你这伙邪魔欺主特展雄才治国祛邪。正没处寻你却来此送命!”那怪闻言不知好歹展长枪就刺行者。行者举铁棒劈面相迎在半空里这一场好杀:棍是龙宫镇海珍枪乃人间转炼铁。凡兵怎敢比仙兵擦着些儿神气泄。大圣原来太乙仙妖精本是邪魔孽。鬼祟焉能近正人一正之时邪就灭。那个弄风播土唬皇王这个踏雾腾云遮日月。

丢开架子赌输赢无能谁敢夸豪杰!还是齐天大圣能乒乓一棍枪先折。那妖精被行者一铁棒把根枪打做两截慌得顾性命拨转风头径往西方败走。

行者且不赶他按下云头来至避妖楼地穴之外叫道:“师父请同陛下出来怪物已赶去矣。”那唐僧才扶着君王同出穴外见满天清朗更无妖邪之气。那皇帝即至酒席前自己拿壶把盏满斟金杯奉与行者道:“神僧权谢!权谢!”这行者接杯在手还未回言只听得朝门外有官来报:“西门上火起了!”

行者闻说将金杯连酒望空一撇当的一声响喨那个金杯落地。君王着了忙躬身施礼道:“神僧恕罪!恕罪!是寡人不是了!礼当请上殿拜谢只因有这方便酒在此故就奉耳。神僧却把杯子撇了却不是有见怪之意?”行者笑道:“不是这话不是这话。”少顷间又有官来报:“好雨呀!才西门上起火被一场大雨把火灭了。满街上流水尽都是酒气。”行者又笑道:

“陛下你见我撇杯疑有见怪之意非也。那妖败走西方我不曾赶他他就放起火来。这一杯酒却是我灭了妖火救了西城里外人家岂有他意!”国王更十分欢喜加敬。即请三藏四众同上宝殿就有推位让国之意。行者笑道:“陛下才那妖精他称是赛太岁部下先锋来此取宫女的。他如今战败而回定然报与那厮那厮定要来与我相争。我恐他一时兴师帅众未免又惊伤百姓恐唬陛下。欲去迎他一迎就在那半空中擒了他取回圣后。但不知向那方去这里到他那山洞有多少远近?”国王道:“寡人曾差夜不收军马到那里探听声息往来要行五十余日。坐落南方约有三千余里。”行者闻言叫:“八戒沙僧护持在此老孙去来。”国王扯住道:“神僧且从容一日待安排些干粮烘炒与你些盘缠银两选一匹快马方才可去。”行者笑道:“陛下说得是巴山转岭步行之话。我老孙不瞒你说似这三千里路斟酒在锺不冷就打个往回。”国王道:“神僧你不要怪我说。你这尊貌却象个猿猴一般怎生有这等法力会走路也?”行者道:“我身虽是猿猴数自幼打开生死路。遍访明师把道传山前修炼无朝暮。倚天为顶地为炉两般药物团乌兔。采取阴阳水火交时间顿把玄关悟。全仗天罡搬运功也凭斗柄迁移步。退炉进火最依时抽铅添汞相交顾。攒簇五行造化生合和四象分时度。二气归于黄道间三家会在金丹路。悟通法律归四肢本来筋斗如神助。一纵纵过太行山一打打过凌云渡。何愁峻岭几千重不怕长江百十数。只因变化没遮拦一打十万八千路!”那国王见说又惊又喜笑吟吟捧着一杯御酒递与行者道:“神僧远劳进此一杯引意。”这大圣一心要去降妖那里有心吃酒只叫:“且放下等我去了回来再饮。”好行者说声去唿哨一声寂然不见。那一国君臣皆惊讶不题。

却说行者将身一纵早见一座高山阻住雾角即按云头立在那巅峰之上仔细观看好山:冲天占地碍日生云。冲天处尖峰矗矗;占地处远脉迢迢。碍日的乃岭头松郁郁;生云的乃崖下石磷磷。松郁郁四时八节常青;石磷磷万载千年不改。林中每听夜猿啼涧内常闻妖蟒过。山禽声咽咽山兽吼呼呼。山獐山鹿成双作对纷纷走;山鸦山鹊打阵攒群密密飞。山草山花看不尽山桃山果映时新。虽然倚险不堪行却是妖仙隐逸处。这大圣看看不厌正欲找寻洞口只见那山凹里烘烘火光飞出霎时间扑天红焰红焰之中冒出一股恶烟比火更毒好烟!但见那:火光迸万点金灯火焰飞千条红虹。

那烟不是灶筒烟不是草木烟烟却有五色:青红白黑黄。熏着南天门外柱燎着灵霄殿上梁。烧得那窝中走兽连皮烂林内飞禽羽尽光。但看这烟如此恶怎入深山伏怪王!大圣正自恐惧又见那山中迸出一道沙来。好沙真个是遮天蔽日!你看:

纷纷絯絯遍天涯邓邓浑浑大地遮。细尘到处迷人目粗灰满谷滚芝麻。采药仙僮迷失伴打柴樵子没寻家。手中就有明珠现时间刮得眼生花。

这行者只顾看玩不觉沙灰飞入鼻内痒斯斯的打了两个喷嚏即回头伸手在岩下摸了两个鹅卵石塞住鼻子摇身一变变做一个攒火的鹞子飞入烟火中间蓦了几蓦却就没了沙灰烟火也息了。急现本象下来。又看时只听得丁丁东东的一个铜锣声响却道:“我走错了路也!这里不是妖精住处。锣声似铺兵之锣想是通国的大路有铺兵去下文书。且等老孙去问他一问。”

正走处忽见是个小妖儿担着黄旗背着文书敲着锣儿急走如飞而来行者笑道:“原来是这厮打锣。他不知送的是甚么书信等我听他一听。”好大圣摇身一变变做个猛虫儿轻轻的飞在他书包之上只听得那妖精敲着锣绪绪聒聒的自念自诵道:“我家大王忒也心毒三年前到朱紫国强夺了金圣皇后一向无缘未得沾身只苦了要来的宫女顶缸。两个来弄杀了四个来也弄杀了。前年要了去年又要今年又要今年还要却撞个对头来了。那个要宫女的先锋被个甚么孙行者打败了不宫女。我大王因此怒要与他国争持教我去下甚么战书。这一去那国王不战则可战必不利。我大王使烟火飞沙那国王君臣百姓等莫想一个得活。那时我等占了他的城池大王称帝我等称臣虽然也有个大小官爵只是天理难容也!”行者听了暗喜道:“妖精也有存心好的似他后边这两句话说天理难容却不是个好的?但只说金圣皇后一向无缘未得沾身此话却不解其意。等我问他一问。”嘤的一声一翅飞离了妖精转向前路有十数里地摇身一变又变做一个道童:头挽双抓髻身穿百衲衣。手敲鱼鼓简口唱道情词。转山坡迎着小妖打个起手道:“长官那里去?送的是甚么公文?”那妖物就象认得他的一般住了锣槌笑嘻嘻的还礼道:

“我大王差我到朱紫国下战书的。”行者接口问道:“朱紫国那话儿可曾与大王配合哩?”小妖道:“自前年摄得来当时就有一个神仙送一件五彩仙衣与金圣宫妆新。他自穿了那衣就浑身上下都生了针刺我大王摸也不敢摸他一摸。但挽着些儿手心就痛不知是甚缘故自始至今尚未沾身。早间差先锋去要宫女伏侍被一个甚么孙行者战败了。大王奋怒所以教我去下战书明日与他交战也。”行者道:“怎的大王却着恼呵?”小妖道:“正在那里着恼哩。你去与他唱个道情词儿解解闷也好。”

行者拱手抽身就走那妖依旧敲锣前行。行者就行起凶来掣出棒复转身望小妖脑后一下可怜就打得头烂血流浆迸出皮开颈折命倾之!收了棍子却又自悔道:“急了些儿!不曾问他叫做甚么名字罢了!”却去取下他的战书藏于袖内将他黄旗、铜锣藏在路旁草里捽时只听当的一声腰间露出一个镶金的牙牌牌上有字写道:“心腹小校一名有来有去。五短身材扢挞脸无须。长用悬挂无牌即假。”行者笑道:“这厮名字叫做有来有去这一棍子打得有去无来也!”将牙牌解下带在腰间欲要捽下尸骸却又思量起烟火之毒且不敢寻他洞府即将棍子举起着小妖胸前捣了一下挑在空中径回本国且当报一个头功。你看他自思自念唿哨一声到了国界。

那八戒在金銮殿前正护持着王师忽回头看见行者半空中将个妖精挑来他却怨道:“嗳!不打紧的买卖!早知老猪去拿来却不算我一功?”说未毕行者按落云头将妖精捽在阶下。八戒跑上去就筑了一钯道:“此是老猪之功!”行者道:“是你甚功?”八戒道:“莫赖我我有证见!你不看一钯筑了九个眼子哩!”行者道:“你看看可有头没头。”八戒笑道:“原来是没头的!我道如何筑他也不动动儿。”行者道:“师父在那里?”八戒道:“在殿里与王叙话哩。”行者道:“你且去请他出来。”八戒急上殿点点头三藏即便起身下殿迎着行者。行者将一封战书揣在三藏袖里道:“师父收下且莫与国王看见。”说不了那国王也下殿迎着行者道:“神僧孙长老来了!拿妖之事如何?”行者用手指道:“那阶下不是妖精?被老孙打杀了也。”国王见了道:“是便是个妖尸却不是赛太岁。赛太岁寡人亲见他两次:

身长丈八膊阔五停面似金光声如霹雳那里是这般鄙矮。”

行者笑道:“陛下认得果然不是这是一个报事的小妖撞见老孙却先打死挑回来报功。”国王大喜道:“好!好!好!该算头功!寡人这里常差人去打探更不曾得个的实。似神僧一出就捉了一个回来真神通也!”叫:“看暖酒来!与长老贺功。”行者道:“吃酒还是小事我问陛下金圣宫别时可曾留下个甚么表记?你与我些儿。”那国王听说表记二字却似刀剑剜心忍不住失声泪下说道:“当年佳节庆朱明太岁凶妖喊声。

强夺御妻为压寨寡人献出为苍生。更无会话并离话那有长亭共短亭!表记香囊全没影至今撇我苦伶仃!”行者道:“陛下在迩何以为恼?那娘娘既无表记他在宫内可有甚么心爱之物与我一件也罢。”国王道:“你要怎的?”行者道:“那妖王实有神通我见他放烟、放火、放沙果是难收。纵收了又恐娘娘见我面生不肯跟我回国。须是得他平日心爱之物一件他方信我我好带他回来为此故要带去。”国王道:“昭阳宫里梳妆阁上有一双黄金宝串原是金圣宫手上带的只因那日端午要缚五色彩线故此褪下不曾带上。此乃是他心爱之物如今现收在简妆盒里。寡人见他遭此离别更不忍见;一见即如见他玉容病又重几分也。”行者道:“且休题这话且将金串取来。如舍得都与我拿去;如不舍只拿一只去也。”国正遂命玉圣宫取出取出即递与国王。国王见了叫了几声知疼着热的娘娘遂递与行者。行者接了套在肐膊上。

好大圣不吃得功酒且驾筋斗云唿哨一声又至麒麟山上无心玩景径寻洞府而去。正行时只听得人语喧嚷即佇立凝睛观看原来那獬豸洞口把门的大小头目约摸有五百名在那里:森森罗列密密挨排。森森罗列执干戈映日光明;

密密挨排展旌旗迎风飘闪。虎将熊师能变化豹头彪帅弄精神。苍狼多猛烈。獭象更骁雄。狡兔乖獐轮剑戟长蛇大蟒挎刀弓。猩猩能解人言语引阵安营识汛风。行者见了不敢前进抽身径转旧路。你道他抽身怎么?不是怕他他却至那打死小妖之处寻出黄旗铜锣迎风捏诀想象腾那即摇身一变变做那有来有去的模样乒乓敲着锣大踏步一直前来径撞至獬豸洞正欲看看洞景只闻得猩猩出语道:“有来有去你回来了?”行者只得答应道:“来了。”猩猩道:“快走!大王爷爷正在剥皮亭上等你回话哩。”行者闻言拽开步敲着锣径入前门里看处原来是悬崖削壁石屋虚堂左右有琪花瑶草前后多古柏乔松。不觉又至二门之内忽抬头见一座八窗明亮的亭子亭子中间有一张戗金的交椅椅子上端坐着一个魔王真个生得恶象。但见他:幌幌霞光生顶上威威杀气迸胸前。口外獠牙排利刃鬓边焦放红烟。嘴上髭须如插箭遍体昂毛似迭毡。眼突铜铃欺太岁手持铁杵若摩天。行者见了公然傲慢那妖精更不循一些儿礼法调转脸朝着外只管敲锣。妖王问道:“你来了?”行者不答又问:“有来有去你来了?”也不答应妖王上前扯住道:“你怎么到了家还筛锣?问之又不答何也?”行者把锣往地下一掼道:“甚么何也何也!我说我不去你却教我去。行到那厢只见无数的人马列成阵势见了我就都叫拿妖精!拿妖精!把我揪揪扯扯拽拽扛扛拿进城去见了那国王国王便教斩了幸亏那两班谋士道两家相争不斩来使把我饶了收了战书又押出城外对军前打了三十顺腿放我来回话。他那里不久就要来此与你交战哩。”

妖王道:“这等说是你吃亏了怪不道问你更不言语。”行者道:“却不是怎的只为护疼所以不曾答应。”妖王道:“那里有多少人马?”行者道:“我也唬昏了又吃他打怕了那里曾查他人马数目!只见那里森森兵器摆列着:弓箭刀枪甲与衣干戈剑戟并缨旗。剽枪月铲兜鍪铠大斧团牌铁蒺藜。长闷棍短窝槌钢叉铳铇及头盔。打扮得靴鞋护顶并胖袄简鞭袖弹与铜锤。”那王听了笑道:“不打紧!不打紧!似这般兵器一火皆空。你且去报与金圣娘娘得知教他莫恼。今早他听见我狠要去战斗他就眼泪汪汪的不干。你如今去说那里人马骁勇必然胜我且宽他一时之心。”

行者闻言十分欢喜道:“正中老孙之意!”你看他偏是路熟转过角门穿过厅堂。那里边尽都是高堂大厦更不似前边的模样直到后面宫里远见彩门壮丽乃是金圣娘娘住处。直入里面看时有两班妖狐妖鹿一个个都妆成美女之形侍立左右正中间坐着那个娘娘手托着香腮双眸滴泪果然是玉容娇嫩美貌妖娆。懒梳妆散鬓堆鸦;怕打扮钗环不戴。面无粉冷淡了胭脂;无油蓬松了云鬓。努樱唇紧咬银牙;皱蛾眉泪淹星眼。一片心只忆着朱紫君王;一时间恨不离天罗地网。诚然是:自古红颜多薄命恹恹无语对东风!行者上前打了个问讯道:“接喏。”那娘娘道:“这泼村怪十分无状!想我在那朱紫国中与王同享荣华之时那太师宰相见了就俯伏尘埃不敢仰视。这野怪怎么叫声接喏?是那里来的这般村泼?”众侍婢上前道:“太太息怒他是大王爷爷心腹的小校唤名有来有去。今早差下战书的是他。”娘娘听说忍怒问曰:“你下战书可曾到朱紫国界?”行者道:“我持书直至城里到于金銮殿面见君王已讨回音来也。”娘娘道:“你面君君有何言?”行者道:“那君王敌战之言与排兵布阵之事才与大王说了。只是那君王有思想娘娘之意有一句合心的话儿特来上禀奈何左右人众不是说处。”娘娘闻言喝退两班狐鹿。行者掩上宫门把脸一抹现了本象对娘娘道:“你休怕我我是东土大唐差往大西天天竺国雷音寺见佛求经的和尚。我师父是唐王御弟唐三藏我是他大徒弟孙悟空。因过你国倒换关文见你君臣出榜招医是我大施三折之肱把他相思之病治好了。排宴谢我饮酒之间说出你被妖摄来我会降龙伏虎特请我来捉怪救你回国。那战败先锋是我打死小妖也是我。我见他门外凶狂是我变作有来有去模样舍身到此与你通信。”那娘娘听说沉吟不语。行者取出宝串双手奉上道:“你若不信看此物何来?”娘娘一见垂泪下座拜谢道:“长老你果是救得我回朝没齿不忘大恩!”行者道:“我且问你他那放火、放烟、放沙的是件甚么宝贝?”娘娘道:“那里是甚宝贝!乃是三个金铃。他将头一个幌一幌有三百丈火光烧人;第二个幌一幌有三百丈烟光熏人;第三个幌一幌有三百丈黄沙迷人。烟火还不打紧只是黄沙最毒若钻入人鼻孔就伤了性命。”行者道:“利害!利害!我曾经着打了两个嚏喷却不知他的铃儿放在何处?”娘娘道:“他那肯放下只是带在腰间行住坐卧再不离身。”行者道:“你若有意于朱紫国还要相会国王把那烦恼忧愁都且权解使出个风流喜悦之容与他叙个夫妻之情教他把铃儿与你收贮。待我取便偷了降了这妖怪那时节好带你回去重谐鸾凤共享安宁也。”那娘娘依言。

这行者还变作心腹小校开了宫门唤进左右侍婢。娘娘叫:“有来有去快往前亭请你大王来与他说话。”好行者应了一声即至剥皮亭对妖精道:“大王圣宫娘娘有请。”妖王欢喜道:“娘娘常时只骂怎么今日有请?”行者道:“那娘娘问朱紫国王之事是我说他不要你了他国中另扶了皇后。娘娘听说故此没了想头方才命我来奉请。”妖王大喜道:“你却中用。待我剿除了他国封你为个随朝的太宰。”行者顺口谢恩疾与妖王来至后宫门。那娘娘欢容迎接就去用手相搀那妖王喏喏而退道:“不敢不敢!多承娘娘下爱我怕手痛不敢相傍。”娘娘道:“大王请坐我与你说。”妖王道:“有话但说不妨。”娘娘道:“我蒙大王辱爱今已三年未得共枕同衾也是前世之缘做了这场夫妻谁知大王有外我之意不以夫妻相待。我想着当时在朱紫国为后外邦凡有进贡之宝君看毕一定与后收之。你这里更无甚么宝贝左右穿的是貂裘吃的是血食那曾见绫锦金珠!只一味铺皮盖毯或者就有些宝贝你因外我也不教我看见也不与我收着。且如闻得你有三个铃铛想就是件宝贝你怎么走也带着坐也带着?你就拿与我收着待你用时取出未为不可。此也是做夫妻一场也有个心腹相托之意。如此不相托付非外我而何?”妖王大笑陪礼道:“娘娘怪得是!怪得是!宝贝在此今日就当付你收之。”便即揭衣取宝。行者在旁眼不转睛看着那怪揭起两三层衣服贴身带着三个铃儿。他解下来将些绵花塞了口儿把一块豹皮作一个包袱儿包了递与娘娘道:“物虽微贱却要用心收藏切不可摇幌着他。”娘娘接过手道:“我晓得。安在这妆台之上无人摇动。”叫:“小的们安排酒来我与大王交欢会喜饮几杯儿。”众侍婢闻言即铺排果菜摆上些獐鹿兔之肉将椰子酒斟来奉上。那娘娘做出妖娆之态哄着精灵。

孙行者在旁取事但挨挨摸摸行近妆台把三个金铃轻轻拿过慢慢移步溜出宫门径离洞府。到了剥皮亭前无人处展开豹皮幅子看时中间一个有茶锺大两头两个有拳头大。他不知利害就把绵花扯了只闻得当的一声响喨骨都都的迸出烟火黄沙急收不住满亭中烘烘火起。唬得那把门精怪一拥撞入后宫惊动了妖王慌忙教:“去救火!救火!”出来看时原来是有来有去拿了金铃儿哩。妖王上前喝道:“好贱奴!怎么偷了我的金铃宝贝在此胡弄!”叫:“拿来!拿来!”那门前虎将、熊师、豹头、彪帅、獭象、苍狼、乖獐、狡兔、长蛇、大蟒、猩猩帅众妖一齐攒簇。那行者慌了手脚丢了金铃现出本象掣出金箍如意棒撒开解数往前乱打。那妖王收了宝贝传号令教:“关了前门!”众妖听了关门的关门打仗的打仗。那行者难得脱身收了棒摇身一变变作个痴苍蝇儿钉在那无火处石壁上。众妖寻不见报道:“大王走了贼也!走了贼也!”妖王问:“可曾自门里走出去?”众妖都说:“前门紧锁牢拴在此不曾走出。”妖王只说:“仔细搜寻!”有的取水泼火有的仔细搜寻更无踪迹。妖王怒道:“是个甚么贼子好大胆变作有来有去的模样进来见我回话又跟在身边乘机盗我宝贝!早是不曾拿将出去!若拿出山头见了天风怎生是好?”

虎将上前道:“大王的洪福齐天我等的气数不尽故此知觉了。”熊师上前道:“大王这贼不是别人定是那战败先锋的那个孙悟空。想必路上遇着有来有去伤了性命夺了黄旗、铜锣、牙牌变作他的模样到此欺骗了大王也。”妖王道:“正是!

正是!见得有理!”叫:“小的们仔细搜求防避切莫开门放出走了!”这才是个有分教:弄巧翻成拙作耍却为真。毕竟不知孙行者怎么脱得妖门且听下回分解。
------------

第七十一回 行者假名降怪犼 观音现象伏妖王

色即空兮自古空言是色如然。人能悟彻色空禅何用丹砂炮炼。德行全修休懈工夫苦用熬煎。有时行满始朝天永驻仙颜不变。话说那赛太岁紧关了前后门户搜寻行者直嚷到黄昏时分不见踪迹。坐在那剥皮亭上点聚群妖号施令都教各门上提铃喝号击鼓敲梆一个个弓上弦刀出鞘支更坐夜。原来孙大圣变做个痴苍蝇钉在门旁见前面防备甚紧他即抖开翅飞入后宫门看处见金圣娘娘伏在御案上清清滴泪隐隐声悲。行者飞进门去轻轻的落在他那乌云散髻之上听他哭的甚么。少顷间那娘娘忽失声道:“主公啊!

我和你:前生烧了断头香今世遭逢泼怪王。拆凤三年何日会?

分鸳两处致悲伤。差来长老才通信惊散佳姻一命亡。只为金铃难解识相思又比旧时狂。”行者闻言即移身到他耳根后悄悄的叫道:“圣宫娘娘你休恐惧我还是你国差来的神僧孙长老未曾伤命。只因自家性急近妆台偷了金铃你与妖王吃酒之时我却脱身私出了前亭忍不住打开看看。不期扯动那塞口的绵花那铃响一声迸出烟火黄沙。我就慌了手脚把金铃丢了现出原身使铁棒苦战不出恐遭毒手故变作一个苍蝇儿钉在门枢上躲到如今。那妖王愈加严紧不肯开门。

你可去再以夫妻之礼哄他进来安寝我好脱身行事别作区处救你也。”娘娘一闻此言战兢兢似神揪虚怯怯心如杵筑泪汪汪的道:“你如今是人是鬼?”行者道:“我也不是人我也不是鬼如今变作个苍蝇儿在此。你休怕快去请那妖王也。”娘娘不信泪滴滴悄语低声道:“你莫魇寐我。”行者道:

“我岂敢魇寐你?你若不信展开手等我跳下来你看。”那娘娘真个把左手张开行者轻轻飞下落在他玉掌之间好便似:菡萏蕊头钉黑豆牡丹花上歇游蜂;绣球心里葡萄落百合枝边黑点浓。金圣宫高擎玉掌叫声神僧行者嘤嘤的应道:“我是神僧变的。”那娘娘方才信了悄悄的道:“我去请那妖王来时你却怎生行事?”行者道:“古人云断送一生惟有酒。又云破除万事无过酒。酒之为用多端你只以饮酒为上你将那贴身的侍婢唤一个进来指与我看我就变作他的模样在旁边伏侍却好下手。”那娘娘真个依言即叫:“春娇何在?”那屏风后转出一个玉面狐狸来跪下道:“娘娘唤春娇有何使令?”娘娘道:“你去叫他们来点纱灯焚脑麝扶我上前庭请大王安寝也。”那春娇即转前面叫了七八个怪鹿妖狐打着两对灯龙一对提炉摆列左右。娘娘欠身叉手那大圣早已飞去。好行者展开翅径飞到那玉面狐狸头上拔下一根毫毛吹口仙气叫“变!”变作一个瞌睡虫轻轻的放在他脸上。原来瞌睡虫到了人脸上往鼻孔里爬爬进孔中即瞌睡了。那春娇果然渐觉困倦立不住脚摇桩打盹即忙寻着原睡处丢倒头只情呼呼的睡起。行者跳下来摇身一变变做那春娇一般模样转屏风与众排立不题。

却说那金圣宫娘娘往前正走有小妖看见即报赛太岁道:“大王娘娘来了。”那妖王急出剥皮亭外迎迓娘娘道:“大王啊烟火既息贼已无踪深夜之际特请大王安置。”那妖满心欢喜道:“娘娘珍重却才那贼乃是孙悟空。他败了我先锋打杀我小校变化进来哄了我们我们这般搜检他却渺无踪迹故此心上不安。”娘娘道:“那厮想是走脱了。大王放心勿虑且自安寝去也。”妖精见娘娘侍立敬请不敢坚辞只得吩咐群妖各要小心火烛谨防盗贼遂与娘娘径往后宫。行者假变春娇从两班侍婢引入。娘娘叫:“安排酒来与大王解劳。”妖王笑道:“正是正是快将酒来我与娘娘压惊。”假春娇即同众怪铺排了果品整顿些腥肉调开桌椅。那娘娘擎杯这妖王也以一杯奉上二人穿换了酒杯。假春娇在旁执着酒壶道:“大王与娘娘今夜才递交杯盏请各饮干穿个双喜杯儿。”真个又各斟上又饮干了。假春娇又道:“大王娘娘喜会众侍婢会唱的供唱善舞的起舞来耶。”说未毕只听得一派歌声齐调音律唱的唱舞的舞。他两个又饮了许多。娘娘叫住了歌舞。众侍婢分班出屏风外摆列惟有假春娇执壶上下奉酒。娘娘与那妖王专说得是夫妻之话。你看那娘娘一片云情雨意哄得那妖王骨软筋麻只是没福不得沾身。可怜!真是猫咬尿胞空欢喜!

叙了一会笑了一会娘娘问道:“大王宝贝不曾伤损么?”妖王道:“这宝贝乃先天抟铸之物如何得损!只是被那贼扯开塞口之绵烧了豹皮包袱也。”娘娘说:“怎生收拾?”妖王道:“不用收拾我带在腰间哩。”假春娇闻得此言即拔下毫毛一把嚼得粉碎轻轻挨近妖王将那毫毛放在他身上吹了三口仙气暗暗的叫“变!”那些毫毛即变做三样恶物乃虱子、虼蚤、臭虫攻入妖王身内挨着皮肤乱咬。那妖王燥痒难禁伸手入怀揣摸揉痒用指头捏出几个虱子来拿近灯前观看。娘娘见了含忖道:“大王想是衬衣禳了久不曾浆洗故生此物耳。”妖王惭愧道:“我从来不生此物可可的今宵出丑。”娘娘笑道:“大王何为出丑?常言道皇帝身上也有三个御虱哩。且脱下衣服来等我替你捉捉。”妖王真个解带脱衣。假春娇在旁着意看着那妖王身上衣服层层皆有虼蚤跳件件皆排大臭虫;子母虱密密浓浓就如蝼蚁出窝中。不觉的揭到第三层见肉之处那金铃上纷纷垓垓的也不胜其数。假春娇道:“大王拿铃子来等我也与你捉捉虱子。”那妖王一则羞二则慌却也不认得真假将三个铃儿递与假春娇。假春娇接在手中卖弄多时见那妖王低着头抖这衣服他即将金铃藏了拔下一根毫毛变作三个铃儿一般无二拿向灯前翻检;却又把身子扭扭捏捏的抖了一抖将那虱子、臭虫、虼蚤收了归在身上把假金铃儿递与那怪。那怪接在手中一朦胧无措那里认得甚么真假双手托着那铃儿递与娘娘道:“今番你却收好了却要仔细仔细不要象前一番。”那娘娘接过来轻轻的揭开衣箱把那假铃收了用黄金锁锁了却又与妖王叙饮了几杯酒教侍婢:“净拂牙床展开锦被我与大王同寝。”那妖王诺诺连声道:“没福!没福!不敢奉陪我还带个宫女往西宫里睡去娘娘请自安置。”遂此各归寝处不题。

却说假春娇得了手将他宝贝带在腰间现了本象把身子抖一抖收去那个瞌睡虫儿径往前走只听得梆铃齐响紧打三更。好行者捏着诀念动真言使个隐身法直至门边。又见那门上拴锁甚密却就取出金箍棒望门一指使出那解锁之法那门就轻轻开了急拽步出门站下厉声高叫道:“赛太岁!还我金圣娘娘来!”连叫两三遍惊动大小群妖急急看处前门开了即忙掌灯寻锁把门儿依然锁上着几个跑入里边去报道:“大王!有人在大门外呼唤大王尊号要金圣娘娘哩!”

那里边侍婢即出宫门悄悄的传言道:“莫吆喝大王才睡着了。”行者又在门前高叫那小妖又不敢去惊动。如此者三四遍俱不敢去通报。那大圣在外嚷嚷闹闹的直弄到天晓忍不住手轮着铁棒上前打门。慌得那大小群妖顶门的顶门报信的报信。那妖王一觉方醒只闻得乱撺撺的喧哗起身穿了衣服即出罗帐之外问道:“嚷甚么?”众侍婢才跪下道:“爷爷不知是甚人在洞外叫骂了半夜如今却又打门。”妖王走出宫门只见那几个传报的小妖慌张张的磕头道:“外面有人叫骂要金圣宫娘娘哩!若说半个不字他就说出无数的歪话甚不中听。见天晓大王不出逼得打门也。”那妖道:“且休开门你去问他是那里来的姓甚名谁快来回报。”小妖急出去隔门问道:“打门的是谁?”行者道:“我是朱紫国拜请来的外公来取圣宫娘娘回国哩!”那小妖听得即以此言回报。那妖随往后宫查问来历。原来那娘娘才起来还未梳洗早见侍婢来报:

“爷爷来了。”那娘娘急整衣散挽黑云出宫迎迓。才坐下还未及问又听得小妖来报:“那来的外公已将门打破矣。”那妖笑道:“娘娘你朝中有多少将帅?”娘娘道:“在朝有四十八卫人马良将千员各边上元帅总兵不计其数。”妖王道:“可有个姓外的么?”娘娘道:“我在宫只知内里辅助君王早晚教诲妃嫔外事无边我怎记得名姓!”妖王道:“这来者称为外公我想着百家姓上更无个姓外的。娘娘赋性聪明出身高贵居皇宫之中必多览书籍。记得那本书上有此姓也?”娘娘道:“止千字文上有句外受傅训想必就是此矣。”

妖王喜道:“定是!定是!”即起身辞了娘娘到剥皮亭上结束整齐点出妖兵开了门直至外面手持一柄宣花钺斧厉声高叫道:“那个是朱紫国来的外公?”行者把金箍棒攥在右手将左手指定道:“贤甥叫我怎的?”那妖王见了心中大怒道:“你这厮:相貌若猴子嘴脸似猢狲。七分真是鬼大胆敢欺人!”行者笑道:“你这个诳上欺君的泼怪原来没眼!想我五百年前大闹天宫时九天神将见了我无一个老字不敢称呼你叫我声外公那里亏了你!”妖王喝道:“快早说出姓甚名谁有些甚么武艺敢到我这里猖獗!”行者道:“你若不问姓名犹可若要我说出姓名只怕你立身无地!你上来站稳着听我道:

生身父母是天地日月精华结圣胎。仙石怀抱无岁数灵根孕育甚奇哉。当年产我三阳泰今日归真万会谐。曾聚众妖称帅能降众怪拜丹崖。玉皇大帝传宣旨太白金星捧诏来。请我上天承职裔官封弼马不开怀。初心造反谋山洞大胆兴兵闹御阶。托塔天王并太子交锋一阵尽猥衰。金星复奏玄穹帝再降招安敕旨来。封做齐天真大圣那时方称栋梁材。又因搅乱蟠桃会仗酒偷丹惹下灾。太上老君亲奏驾西池王母拜瑶台。情知是我欺王法即点天兵火牌。十万凶星并恶曜干戈剑戟密排排。天罗地网漫山布齐举刀兵大会垓。恶斗一场无胜败观音推荐二郎来两家对敌分高下他有梅山兄弟侪。

各逞英雄施变化天门三圣拨云开。老君丢了金钢套众神擒我到金阶。不须详允书供状罪犯凌迟杀斩灾。斧剁锤敲难损命刀轮剑砍怎伤怀!火烧雷打只如此无计摧残长寿胎。押赴太清兜率院炉中煅炼尽安排。日期满足才开鼎我向当中跳出来。手挺这条如意棒翻身打上玉龙台。各星各象皆潜躲大闹天宫任我歪。巡视灵官忙请佛释伽与我逞英才。手心之内翻筋斗游遍周天去复来。佛使先知赚哄法被他压住在天崖。到今五百余年矣解脱微躯又弄乖。特保唐僧西域去悟空行者甚明白。西方路上降妖怪那个妖邪不惧哉!”那妖王听他说出悟空行者遂道:“你原来是大闹天宫的那厮你既脱身保唐僧西去你走你的路去便罢了。怎么罗织管事替那朱紫国为奴却到我这里寻死!”行者喝道:“贼泼怪!说话无知!我受朱紫国拜请之礼又蒙他称呼管待之恩我老孙比那王位还高千倍他敬之如父母事之如神明你怎么说出为奴二字!我把你这诳上欺君之怪不要走!吃外公一棒!”那妖慌了手脚即闪身躲过使宣花斧劈面相迎。这一场好杀!你看:金箍如意棒风刃宣花斧。一个咬牙狠凶一个切齿施威武。这个是齐天大圣降临凡那个是作怪妖王来下土。两个喷云嗳雾照天宫真是走石扬沙遮斗府。往往来来解数多翻翻复复金光吐。齐将本事施各把神通赌。这个要取娘娘转帝都那个喜同皇后居山坞。这场都是没来由舍死忘生因国主。他两个战经五十回合不分胜负。那妖王见行者手段高强料不能取胜将斧架住他的铁棒道:“孙行者你且住了。我今日还未早膳待我进了膳再来与你定雌雄。”行者情知是要取铃铛收了铁棒道:“好汉子不赶乏兔儿你去你去!吃饱些好来领死!”

那妖急转身闯入里边对娘娘道:“快将宝贝拿来!”娘娘道:“要宝贝何干?”妖王道:“今早叫战者乃是取经的和尚之徒叫做孙悟空行者假称外公。我与他战到此时不分胜负。

等我拿宝贝出去放些烟火烧这猴头。”娘娘见说心中怛突:

欲不取出铃儿恐他见疑;欲取出铃儿又恐伤了孙行者性命。

正自踌躇未定那妖王又催逼道:“快拿出来!”这娘娘无奈只得将锁钥开了把三个铃儿递与妖王。妖王拿了就走出洞。娘娘坐在宫中泪如雨下思量行者不知可能逃得性命。两人却俱不知是假铃也。那妖出了门就占起上风叫道:“孙行者休走!看我摇摇铃儿!”行者笑道:“你有铃我就没铃?你会摇我就不会摇?”妖王道:“你有甚么铃儿拿出来我看。”行者将铁棒捏做个绣花针儿藏在耳内却去腰间解下三个真宝贝来对妖王说:“这不是我的紫金铃儿?”妖王见了心惊道:“跷蹊!跷蹊!他的铃儿怎么与我的铃儿就一般无二!纵然是一个模子铸的好道打磨不到也有多个瘢儿少个蒂儿却怎么这等一毫不差?”又问:“你那铃儿是那里来的?”行者道:“贤甥你那铃儿却是那里来的。”妖王老实便就说道:“我这铃儿是:太清仙君道源深八卦炉中久炼金。结就铃儿称至宝老君留下到如今。”行者笑道:“老孙的铃儿也是那时来的。”妖王道:“怎生出处?”行者道:“我这铃儿是:道祖烧丹兜率宫金铃抟炼在炉中。二三如六循环宝我的雌来你的雄。”妖王道:“铃儿乃金丹之宝又不是飞禽走兽如何辨得雌雄?但只是摇出宝来就是好的!”行者道:“口说无凭做出便见且让你先摇。”那妖王真个将头一个铃儿幌了三幌不见火出;第二个幌了三幌不见烟出;第三个幌了三幌也不见沙出。妖王慌了手脚道:“怪哉!怪哉!世情变了!这铃儿想是惧内雄见了雌所以不出来了。”行者道:“贤甥住了手等我也摇摇你看。”好猴子一把攥了三个铃儿一齐摇起。你看那红火、青烟、黄沙一齐滚出骨都都燎树烧山!大圣口里又念个咒语望巽地上叫:“风来!”真个是风催火势火挟风威红焰焰黑沉沉满天烟火遍地黄沙!把那赛太岁唬得魄散魂飞走头无路在那火当中怎逃性命!

只闻得半空中厉声高叫:“孙悟空!我来了也!”行者急回头上望原来是观音菩萨左手托着净瓶右手拿着杨柳洒下甘露救火哩慌得行者把铃儿藏在腰间即合掌倒身下拜。那菩萨将柳枝连拂几点甘露霎时间烟火俱无黄沙绝迹。行者叩头道:“不知大慈临凡有失回避。敢问菩萨何往?”菩萨道:

“我特来收寻这个妖怪。”行者道:“这怪是何来历敢劳金身下降收之?”菩萨道:“他是我跨的个金毛犼。因牧童盹睡失于防守这孽畜咬断铁索走来却与朱紫国王消灾也。”行者闻言急欠身道:“菩萨反说了他在这里欺君骗后败俗伤风与那国王生灾却说是消灾何也?”菩萨道:“你不知之当时朱紫国先王在位之时这个王还做东宫太子未曾登基他年幼间极好射猎。他率领人马纵放鹰犬正来到落凤坡前有西方佛母孔雀大明王菩萨所生二子乃雌雄两个雀雏停翅在山坡之下被此王弓开处射伤了雄孔雀那雌孔雀也带箭归西。佛母忏悔以后吩咐教他拆凤三年身耽啾疾。那时节我跨着这犼同听此言不期这孽畜留心故来骗了皇后与王消灾。至今三年冤愆满足幸你来救治王患我特来收妖邪也。”行者道:“菩萨虽是这般故事奈何他玷污了皇后败俗伤风坏伦乱法却是该他死罪。今蒙菩萨亲临饶得他死罪却饶不得他活罪。让我打他二十棒与你带去罢。”菩萨道:“悟空你既知我临凡就当看我分上一都饶了罢也算你一番降妖之功。

若是动了棍子他也就是死了。”行者不敢违言只得拜道:“菩萨既收他回海再不可令他私降人间贻害不浅!”那菩萨才喝了一声:“孽畜!还不还原待何时也!”只见那怪打个滚现了原身将毛衣抖抖菩萨骑上。菩萨又望项下一看不见那三个金铃。菩萨道:“悟空还我铃来。”行者道:“老孙不知。”菩萨喝道:“你这贼猴!若不是你偷了这铃莫说一个悟空就是十个也不敢近身!快拿出来!”行者笑道:“实不曾见。”菩萨道:“既不曾见等我念念《紧箍儿咒》。”那行者慌了只教:“莫念莫念!铃儿在这里哩!”这正是:犼项金铃何人解?解铃人还问系铃人。菩萨将铃儿套在犼项下飞身高坐。你看他四足莲花生焰焰满身金缕迸森森大慈悲回南海不题。

却说孙大圣整束了衣裙轮铁棒打进獬豸洞去把群妖众怪尽情打死。剿除干净。直至宫中请圣宫娘娘回国那娘娘顶礼不尽。行者将菩萨降妖并拆凤原由备说了一遍寻些软草扎了一条草龙教:“娘娘跨上合着眼莫怕我带你回朝见主也。”那娘娘谨遵吩咐行者使起神通只听得耳内风响。半个时辰带进城按落云头叫:“娘娘开眼。”那皇后睁开眼看认得是凤阁龙楼心中欢喜撇了草龙与行者同登宝殿。那国王见了急下龙床就来扯娘娘玉手欲诉离情猛然跌倒在地只叫:“手疼!手疼!”八戒哈哈大笑道:“嘴脸!没福消受!

一见面就蛰杀了也!”行者道:“呆子你敢扯他扯儿么?”八戒道:“就扯他扯儿便怎的?”行者道:“娘娘身上生了毒刺手上有蜇阳之毒。自到麒麟山与那赛太岁三年那妖更不曾沾身但沾身就害身疼但沾手就害手疼。”众官听说道:“似此怎生奈何?”此时外面众官忧疑内里妃嫔悚惧旁有玉圣、银圣二宫将君王扶起。俱正在仓皇之际忽听得那半空中有人叫道:“大圣我来也。”行者抬头观看只见那:肃肃冲天鹤唳飘飘径至朝前。缭绕祥光道道氤氲瑞气翩翩。棕衣苫体放云烟足踏芒鞋罕见。手执龙须蝇帚丝绦腰下围缠。乾坤处处结人缘大地逍遥游遍。此乃是大罗天上紫云仙今日临凡解魇。行者上前迎住道:“张紫阳何往?”紫阳真人直至殿前躬身施礼道:“大圣小仙张伯端起手。”行者答礼道:“你从何来?”真人道:“小仙三年前曾赴佛会因打这里经过见朱紫国王有拆凤之忧我恐那妖将皇后玷辱有坏人伦后日难与国王复合。是我将一件旧棕衣变作一领新霞裳光生五彩进与妖王教皇后穿了妆新。那皇后穿上身即生一身毒刺毒刺者乃棕毛也。今知大圣成功特来解魇。”行者道:“既如此累你远来且快解脱。”真人走向前对娘娘用手一指即脱下那件棕衣那娘娘遍体如旧。真人将衣抖一抖披在身上对行者道:“大圣勿罪小仙告辞。”行者道:“且住待君王谢谢。”真人笑道:“不劳不劳。”遂长揖一声腾空而去慌得那皇帝、皇后及大小众臣一个个望空礼拜。

拜毕即命大开东阁酬谢四僧。那君王领众跪拜夫妻才得重谐。正当欢宴时行者叫:“师父拿那战书来。”长老袖中取出递与行者行者递与国王道:“此书乃那怪差小校送来者。

那小校已先被我打死送来报功。后复至山中变作小校进洞回复因得见娘娘盗出金铃几乎被他拿住;又变化复偷出与他对敌。幸遇观音菩萨将他收去又与我说拆凤之故。”从头至尾细说了一遍。那举国君臣内外无一人不感谢称赞。唐僧道:“一则是贤王之福二来是小徒之功。今蒙盛宴至矣!至矣!就此拜别不要误贫僧向西去也。”那国王恳留不得遂换了关文大排銮驾请唐僧稳坐龙车那君王妃后俱捧毂推轮相送而别。正是:有缘洗尽忧疑病绝念无思心自宁。毕竟这去后面再有甚么吉凶之事且听下回分解。
------------

第七十二回 盘丝洞七情迷本 濯垢泉八戒忘形

话表三藏别了朱紫国王整顿鞍马西进。行彀多少山原历尽无穷水道不觉的秋去冬残又值春光明媚。师徒们正在路踏青玩景忽见一座庵林三藏滚鞍下马站立大道之旁。行者问道:“师父这条路平坦无邪因何不走?”八戒道:“师兄好不通情!师父在马上坐得困了也让他下来关关风是。”三藏道:“不是关风我看那里是个人家意欲自去化些斋吃。”行者笑道:“你看师父说的是那里话。你要吃斋我自去化俗语云一日为师终身为父岂有为弟子者高坐教师父去化斋之理?”三藏道:“不是这等说。平日间一望无边无际你们没远没近的去化斋今日人家逼近可以叫应也让我去化一个来。”

八戒道:“师父没主张。常言道三人出外小的儿苦你况是个父辈我等俱是弟子。古书云有事弟子服其劳等我老猪去。”

三藏道:“徒弟啊今日天气晴明与那风雨之时不同。那时节汝等必定远去此个人家等我去有斋无斋可以就回走路。”

沙僧在旁笑道:“师兄不必多讲师父的心性如此不必违拗。

若恼了他就化将斋来他也不吃。”

八戒依言即取出钵盂与他换了衣帽。拽开步直至那庄前观看却也好座住场但见:石桥高耸古树森齐。石桥高耸潺潺流水接长溪;古树森齐聒聒幽禽鸣远岱。桥那边有数椽茅屋清清雅雅若仙庵;又有那一座蓬窗白白明明欺道院。窗前忽见四佳人都在那里刺凤描鸾做针线。长老见那人家没个男儿只有四个女子不敢进去将身立定闪在乔林之下只见那女子一个个:闺心坚似石兰性喜如春。娇脸红霞衬朱唇绛脂匀。蛾眉横月小蝉鬓迭云新。若到花间立游蜂错认真。少停有半个时辰一静悄悄鸡犬无声。自家思虑道:

“我若没本事化顿斋饭也惹那徒弟笑我敢道为师的化不出斋来为徒的怎能去拜佛。”长老没计奈何也带了几分不是趋步上桥又走了几步只见那茅屋里面有一座木香亭子亭子下又有三个女子在那里踢气球哩。你看那三个女子比那四个又生得不同但见那:飘扬翠袖摇拽缃裙。飘扬翠袖低笼着玉笋纤纤;摇拽缃裙半露出金莲窄窄。形容体势十分全动静脚跟千样躧。拿头过论有高低张泛送来真又楷。转身踢个出墙花退步翻成大过海。轻接一团泥单枪急对拐。明珠上佛头实捏来尖靴。窄砖偏会拿卧鱼将脚跘。平腰折膝蹲扭顶翘跟躧。扳凳能喧泛披肩甚脱洒。绞裆任往来锁项随摇摆。踢的是黄河水倒流金鱼滩上买。那个错认是头儿这个转身就打拐。端然捧上臁周正尖来潠。提跟潠草鞋倒插回头采。退步泛肩妆钩儿只一歹。版篓下来长便把夺门揣。踢到美心时佳人齐喝采。一个个汗流粉腻透罗裳兴懒情疏方叫海。

言不尽又有诗为证诗曰:蹴踘当场三月天仙风吹下素婵娟。汗沾粉面花含露尘染蛾眉柳带烟。翠袖低垂笼玉笋缃裙斜拽露金莲。几回踢罢娇无力云鬓蓬松宝髻偏。三藏看得时辰久了只得走上桥头应声高叫道:“女菩萨贫僧这里随缘布施些儿斋吃。”那些女子听见一个个喜喜欢欢抛了针线撇了气球都笑笑吟吟的接出门来道:“长老失迎了今到荒庄决不敢拦路斋僧请里面坐。”三藏闻言心中暗道:“善哉善哉!西方正是佛地!女流尚且注意斋僧男子岂不虔心向佛?”长老向前问讯了相随众女入茅屋过木香亭看处呀!

原来那里边没甚房廊只见那:峦头高耸地脉遥长。峦头高耸接云烟地脉遥长通海岳。门近石桥九曲九湾流水顾;园栽桃李千株千颗斗秾华。藤薜挂悬三五树芝兰香散万千花。远观洞府欺蓬岛近睹山林压太华。正是妖仙寻隐处更无邻舍独成家。有一女子上前把石头门推开两扇请唐僧里面坐。那长老只得进去忽抬头看时铺设的都是石桌、石凳冷气阴阴。长老心惊暗自思忖道:“这去处少吉多凶断然不善。”众女子喜笑吟吟都道:“长老请坐。”长老没奈何只得坐了少时间打个冷禁。众女子问道:“长老是何宝山?化甚么缘?还是修桥补路建寺礼塔还是造佛印经?请缘簿出来看看。”长老道:“我不是化缘的和尚。”女子道:“既不化缘到此何干?”长老道:“我是东土大唐差去西天大雷音求经者。适过宝方腹间饥馁特造檀府募化一斋贫僧就行也。”众女子道:“好!好!

好!常言道远来的和尚好看经。妹妹们!不可怠慢快办斋来。”

此时有三个女子陪着言来语去论说些因缘。那四个到厨中撩衣敛袖炊火刷锅。你道他安排的是些甚么东西?原来是人油炒炼人肉煎熬熬得黑糊充作面筋样子剜的人脑煎作豆腐块片。两盘儿捧到石桌上放下对长老道:“请了仓卒间不曾备得好斋且将就吃些充腹后面还有添换来也。”那长老闻了一闻见那腥膻不敢开口欠身合掌道:“女菩萨贫僧是胎里素。”众女子笑道:“长老此是素的。”长老道:“阿弥陀佛!若象这等素的啊我和尚吃了莫想见得世尊取得经卷。”众女子道:“长老你出家人切莫拣人布施。”长老道:“怎敢怎敢!我和尚奉大唐旨意一路西来微生不损见苦就救遇谷粒手拈入口逢丝缕联缀遮身怎敢拣主布施!”众女子笑道:“长老虽不拣人布施却只有些上门怪人。莫嫌粗淡吃些儿罢。”长老道:“实是不敢吃恐破了戒望菩萨养生不若放生放我和尚出去罢。”那长老挣着要走那女子拦住门怎么肯放俱道:“上门的买卖倒不好做!放了屁儿却使手掩你往那里去?”他一个个都会些武艺手脚又活把长老扯住顺手牵羊扑的掼倒在地。众人按住将绳子捆了悬梁高吊这吊有个名色叫做“仙人指路”。原来是一只手向前牵丝吊起;

一只手拦腰捆住将绳吊起两只脚向后一条绳吊起三条绳把长老吊在梁上却是脊背朝上肚皮朝下。那长老忍着疼噙着泪心中暗恨道:“我和尚这等命苦!只说是好人家化顿斋吃岂知道落了火坑!徒弟啊!来救我还得见面但迟两个时辰我命休矣!”那长老虽然苦恼却还留心看着那些女子。

那些女子把他吊得停当便去脱剥衣服。长老心惊暗自忖道:

“这一脱了衣服是要打我的情了或者夹生儿吃我的情也有哩。”原来那女子们只解了上身罗衫露出肚腹各显神通:一个个腰眼中冒出丝绳有鸭蛋粗细骨都都的迸玉飞银时下把庄门瞒了不题。

却说那行者、八戒、沙僧都在大道之旁。他二人都放马看担惟行者是个顽皮他且跳树攀枝摘叶寻果忽回头只见一片光亮慌得跳下树来吆喝道:“不好不好!师父造化低了!”行者用手指道:“你看那庄院如何?”八戒沙僧共目视之那一片如雪又亮如雪似银又光似银。八戒道:“罢了罢了!师父遇着妖精了!我们快去救他也!”行者道:“贤弟莫嚷你都不见怎的等老孙去来。”沙僧道:“哥哥仔细。”行者道:“我自有处。”好大圣束一束虎皮裙掣出金箍棒拽开脚两三步跑到前边看见那丝绳缠了有千百层厚穿穿道道却似经纬之势用手按了一按有些粘软沾人。行者更不知是甚么东西他即举棒道:“这一棒莫说是几千层就有几万层也打断了!”正欲打又停住手道:“若是硬的便可打断这个软的只好打匾罢了。假如惊了他缠住老孙反为不美。等我且问他一问再打。”你道他问谁?即捻一个诀念一个咒拘得个土地老儿在庙里似推磨的一般乱转。土地婆儿道:“老儿你转怎的?好道是羊儿风了!”土地道:“你不知!你不知!有一个齐天大圣来了我不曾接他他那里拘我哩。”婆儿道:“你去见他便了却如何在这里打转?”土地道:“若去见他他那棍子好不重他管你好歹就打哩!”婆儿道:“他见你这等老了那里就打你?”

土地道:“他一生好吃没钱酒偏打老年人。”两口儿讲一会没奈何只得走出去战兢兢的跪在路旁叫道:“大圣当境土地叩头。”行者道:“你且起来不要假忙我且不打你寄下在那里。(WWW.mianhuatang.la 好看的小说)

我问你此间是甚地方?”土地道:“大圣从那厢来?”行者道:

“我自东土往西来的。”土地道:“大圣东来可曾在那山岭上?”

行者道:“正在那山岭上我们行李马匹还都歇在那岭上不是!”土地道:“那岭叫做盘丝岭岭下有洞叫做盘丝洞洞里有七个妖精。”行者道:“是男怪女怪?”土地道:“是女怪。”行者道:“他有多大神通?”土地道:“小神力薄威短不知他有多大手段只知那正南上离此有三里之遥有一座濯垢泉乃天生的热水原是上方七仙姑的浴池。自妖精到此居住占了他的濯垢泉仙姑更不曾与他争竞平白地就让与他了。我见天仙不惹妖魔怪必定精灵有大能。”行者道:“占了此泉何干?”土地道:“这怪占了浴池一日三遭出来洗澡。如今巳时已过午时将来哑。”行者听言道:“土地你且回去等我自家拿他罢。”

那土地老儿磕了一个头战兢兢的回本庙去了。

这大圣独显神通摇身一变变作个麻苍蝇儿钉在路旁草梢上等待。须臾间只听得呼呼吸吸之声犹如蚕食叶却似海生潮。只好有半盏茶时丝绳皆尽依然现出庄村还象当初模样。又听得呀的一声柴扉响处里边笑语喧哗走出七个女子。行者在暗中细看见他一个个携手相搀挨肩执袂有说有笑的走过桥来果是标致。但见:比玉香尤胜如花语更真。柳眉横远岫檀口破樱唇。钗头翘翡翠金莲闪绛裙。却似嫦娥临下界仙子落凡尘。行者笑道:“怪不得我师父要来化斋原来是这一般好处。这七个美人儿假若留住我师父要吃也不彀一顿吃要用也不彀两日用要动手轮流一摆布就是死了。

且等我去听他一听看他怎的算计。”好大圣嘤的一声飞在那前面走的女子云髻上钉住。才过桥来后边的走向前来呼道:“姐姐我们洗了澡来蒸那胖和尚吃去。”行者暗笑道:“这怪物好没算计!煮还省些柴怎么转要蒸了吃!”那些女子采花斗草向南来不多时到了浴池。但见一座门墙十分壮丽遍地野花香艳艳满旁兰蕙密森森。后面一个女子走上前唿哨的一声把两扇门儿推开那中间果有一塘热水。这水自开辟以来太阳星原贞有十后被羿善开弓射落九乌坠地止存金乌一星乃太阳之真火也。天地有九处汤泉俱是众乌所化。那九阳泉乃香冷泉、伴山泉、温泉、东合泉、满山泉、孝安泉、广汾泉、汤泉此泉乃濯垢泉。有诗为证诗曰:一气无冬夏三秋永注春。炎波如鼎沸热浪似汤新。分溜滋禾稼停流荡俗尘。

涓涓珠泪泛滚滚玉团津。润滑原非酿清平还自温。瑞祥本地秀造化乃天真。佳人洗处冰肌滑涤荡尘烦玉体新。那浴池约有五丈余阔十丈多长内有四尺深浅但见水清彻底。底下水一似滚珠泛玉骨都都冒将上来四面有六七个孔窍通流。

流去二三里之遥淌到田里还是温水。池上又有三间亭子亭子中近后壁放着一张八只脚的板凳。两山头放着两个描金彩漆的衣架。行者暗中喜嘤嘤的一翅飞在那衣架头上钉住。

那些女子见水又清又热便要洗浴即一齐脱了衣服搭在衣架上。一齐下去被行者看见:褪放纽扣儿解开罗带结。

酥胸白似银玉体浑如雪。肘膊赛凝胭香肩疑粉捏。肚皮软又绵脊背光还洁。膝腕半围团金莲三寸窄。中间一段情露出风流穴。那女子都跳下水去一个个跃浪翻波负水顽耍。行者道:“我若打他啊只消把这棍子往池中一搅就叫做滚汤泼老鼠一窝儿都是死。可怜!可怜!打便打死他只是低了老孙的名头。常言道男不与女斗我这般一个汉子打杀这几个丫头着实不济。不要打他只送他一个绝后计教他动不得身出不得水多少是好。”好大圣捏着诀念个咒摇身一变变作一个饿老鹰但见:毛犹霜雪眼若明星。妖狐见处魂皆丧狡兔逢时胆尽惊。钢爪锋芒快雄姿猛气横。会使老拳供口腹不辞亲手逐飞腾。万里寒空随上下穿云检物任他行。呼的一翅飞向前轮并利爪把他那衣架上搭的七套衣服尽情雕去径转岭头现出本相来见八戒、沙僧道:“你看。”那呆子迎着对沙僧笑道:“师父原来是典当铺里拿了去的。”沙僧道:

“怎见得?”八戒道:“你不见师兄把他些衣服都抢将来也?”行者放下道:“此是妖精穿的衣服。”八戒道:“怎么就有这许多?”

行者道:“七套。”八戒道:“如何这般剥得容易又剥得干净?”

行者道:“那曾用剥。原来此处唤做盘丝岭那庄村唤做盘丝洞。洞中有七个女怪把我师父拿住吊在洞里都向濯垢泉去洗浴。那泉却是天地产成的一塘子热水。他都算计着洗了澡要把师父蒸吃。是我跟到那里见他脱了衣服下水我要打他恐怕污了棍子又怕低了名头是以不曾动棍只变做一个饿老鹰雕了他的衣服。他都忍辱含羞不敢出头蹲在水中哩。

我等快去解下师父走路罢。”八戒笑道:“师兄你凡干事只要留根。既见妖精如何不打杀他却就去解师父!他如今纵然藏羞不出到晚间必定出来。他家里还有旧衣服穿上一套来赶我们。纵然不赶他久住在此我们取了经还从那条路回去。常言道宁少路边钱莫少路边拳。那时节他拦住了吵闹却不是个仇人也?”行者道:“凭你如何主张?”八戒道:“依我先打杀了妖精再去解放师父此乃斩草除根之计。”行者道:

“我是不打他。你要打你去打他。”

八戒抖擞精神欢天喜地举着钉钯拽开步径直跑到那里。忽的推开门看时只见那七个女子蹲在水里口中乱骂那鹰哩道:“这个匾毛畜生!猫嚼头的亡人!把我们衣服都雕去了教我们怎的动手!”八戒忍不住笑道:“女菩萨在这里洗澡哩也携带我和尚洗洗何如?”那怪见了作怒道:“你这和尚十分无礼!我们是在家的女流你是个出家的男子。古书云:七年男女不同席你好和我们同塘洗澡?”八戒道:“天气炎热没奈何将就容我洗洗儿罢。那里调甚么书担儿同席不同席!”

呆子不容说丢了钉钯脱了皂锦直裰扑的跳下水来那怪心中烦恼一齐上前要打。不知八戒水势极熟到水里摇身一变变做一个鲇鱼精。那怪就都摸鱼赶上拿他不住:东边摸忽的又渍了西去;西边摸忽的又渍了东去;滑扢虀的只在那腿裆里乱钻。原来那水有搀胸之深水上盘了一会又盘在水底都盘倒了喘嘘嘘的精神倦怠。八戒却才跳将上来现了本相穿了直裰执着钉钯喝道:“我是那个?你把我当鲇鱼精哩!”那怪见了心惊胆战对八戒道:“你先来是个和尚到水里变作鲇鱼及拿你不住却又这般打扮你端的是从何到此?是必留名。”八戒道:“这伙泼怪当真的不认得我!我是东土大唐取经的唐长老之徒弟乃天蓬元帅悟能八戒是也。你把我师父吊在洞里算计要蒸他受用!我的师父又好蒸吃?快早伸过头来各筑一钯教你断根!”那些妖闻此言魂飞魄散就在水中跪拜道:“望老爷方便方便!我等有眼无珠误捉了你师父虽然吊在那里不曾敢加刑受苦。望慈悲饶了我的性命情愿贴些盘费送你师父往西天去也。”八戒摇头道:“莫说这话!俗语说得好曾着卖糖君子哄到今不信口甜人。是便筑一钯各人走路!”呆子一味粗夯显手段那有怜香惜玉之心举着钯不分好歹赶上前乱筑。那怪慌了手脚那里顾甚么羞耻只是性命要紧随用手侮着羞处跳出水来都跑在亭子里站立作出法来:脐孔中骨都都冒出丝绳瞒天搭了个大丝篷把八戒罩在当中。那呆子忽抬头不见天日即抽身往外便走那里举得脚步!原来放了绊脚索满地都是丝绳动动脚跌个躘踵:左边去一个面磕地;右边去一个倒栽葱;急转身又跌了个嘴揾地;忙爬起又跌了个竖蜻蜓。也不知跌了多少跟头把个呆子跌得身麻脚软头晕眼花爬也爬不动只睡在地下呻吟。那怪物却将他困住也不打他也不伤他一个个跳出门来将丝篷遮住天光各回本洞。到了石桥上站下念动真言霎时间把丝篷收了赤条条的跑入洞里侮着那话从唐僧面前笑嘻嘻的跑过去。走入石房取几件旧衣穿了径至后门口立定叫:“孩儿们何在?”原来那妖精一个有一个儿子却不是他养的都是他结拜的干儿子。有名唤做蜜、蚂、蜍、班、蜢、蜡、蜻:蜜是蜜蜂蚂是蚂蜂蜍是蜍蜂班是班毛蜢是牛蜢蜡是抹蜡蜻是蜻蜓。原来那妖精幔天结网掳住这七般虫蛭却要吃他。古云禽有禽言兽有兽语当时这些虫哀告饶命愿拜为母遂此春采百花供怪物夏寻诸卉孝妖精。忽闻一声呼唤都到面前问:“母亲有何使令?”众怪道:“儿啊早间我们错惹了唐朝来的和尚才然被他徒弟拦在池里出了多少丑几乎丧了性命!

汝等努力快出门前去退他一退。如得胜后可到你舅舅家来会我。”那些怪既得逃生往他师兄处孽嘴生灾不题。你看这些虫蛭一个个摩拳擦掌出来迎敌。

却说八戒跌得昏头昏脑猛抬头见丝篷丝索俱无他才一步一探爬将起来忍着疼找回原路见了行者用手扯住道:

“哥哥我的头可肿、脸可青么?”行者道:“你怎的来?”八戒道:

“我被那厮将丝绳罩住放了绊脚索不知跌了多少跟头跌得我腰拖背折寸步难移。却才丝篷索子俱空方得了性命回来也。”沙僧见了道:“罢了罢了!你闯下祸来也!那怪一定往洞里去伤害师父、我等快去救他!”行者闻言急拽步便走八戒牵着马急急来到庄前但见那石桥上有七个小妖儿挡住道:“慢来慢来!吾等在此!”行者看了道:“好笑!干净都是些小人儿!

长的也只有二尺五六寸不满三尺;重的也只有八九斤不满十斤。”喝道:“你是谁?”那怪道:“我乃七仙姑的儿子。你把我母亲欺辱了还敢无知打上我门!不要走!仔细!”好怪物!一个个手之舞之足之蹈之乱打将来。八戒见了生嗔本是跌恼了的性子又见那伙虫蛭小巧就狠举钯来筑。

那些怪见呆子凶猛一个个现了本象飞将起去叫声“变!”须臾间一个变十个十个变百个百个变千个千个变万个个个都变成无穷之数。只见:满天飞抹蜡遍地舞蜻蜓。

蜜蚂追头额蜍蜂扎眼睛。班毛前后咬牛蜢上下叮。扑面漫漫黑翛翛神鬼惊。八戒慌了道:“哥啊只说经好取西方路上虫儿也欺负人哩!”行者道:“兄弟不要怕快上前打!”八戒道:“扑头扑脸浑身上下都叮有十数层厚却怎么打?”行者道:“没事!没事!我自有手段!”沙僧道:“哥啊有甚手段快使出来罢!一会子光头上都叮肿了!”好大圣拔了一把毫毛嚼得粉碎喷将出去即变做些黄、麻、鴏、白、雕、鱼、鹞。八戒道:“师兄又打甚么市语黄啊、麻啊哩?”行者道:“你不知黄是黄鹰麻是麻鹰鴏是鴏鹰白是白鹰雕是雕鹰鱼是鱼鹰鹞是鹞鹰。那妖精的儿子是七样虫我的毫毛是七样鹰。”

鹰最能、虫一嘴一个爪打翅敲须臾打得罄尽满空无迹地积尺余。

三兄弟方才闯过桥去径入洞里只见老师父吊在那里哼哼的哭哩。八戒近前道:“师父你是要来这里吊了耍子不知作成我跌了多少跟头哩!”沙僧道:“且解下师父再说。”行者即将绳索挑断放下唐僧都问道:“妖精那里去了?”唐僧道:“那七个怪都赤条条的往后边叫儿子去了。”行者道:“兄弟们跟我来寻去。”三人各持兵器往后园里寻处不见踪迹。都到那桃李树上寻遍不见八戒道:“去了!去了!”沙僧道:“不必寻他等我扶师父去也。”弟兄们复来前面请唐僧上马道:“师父下次化斋还让我们去。”唐僧道:“徒弟呵以后就是饿死也再不自专了。”八戒道:“你们扶师父走着等老猪一顿钯筑倒他这房子教他来时没处安身。”行者笑道:“筑还费力不若寻些柴来与他个断根罢。”好呆子寻了些朽松破竹干柳枯藤点上一把火烘烘的都烧得干净。师徒却才放心前来。咦!毕竟这去不知那怪的吉凶如何且听下回分解。
------------

第七十三回 情因旧恨生灾毒 心主遭魔幸破光

话说孙大圣扶持着唐僧与八戒、沙僧奔上大路一直西来。(wwW.mianhuatang.la 无弹窗广告)不半晌忽见一处楼阁重重宫殿巍巍。唐僧勒马道:“徒弟你看那是个甚么去处?”行者举头观看忽然见:山环楼阁溪绕亭台。门前杂树密森森宅外野花香艳艳。柳间栖白鹭浑如烟里玉无瑕;桃内啭黄莺却似火中金有色。双双野鹿忘情闲踏绿莎茵;对对山禽飞语高鸣红树杪。真如刘阮天台洞不亚神仙阆苑家。行者报道:“师父那所在也不是王侯第宅也不是豪富人家却象一个庵观寺院到那里方知端的。”三藏闻言加鞭促马。师徒们来至门前观看门上嵌着一块石板上有黄花观三字。三藏下马八戒道:“黄花观乃道士之家我们进去会他一会也好他与我们衣冠虽别修行一般。”沙僧道:

“说得是一则进去看看景致二来也当撒货头口。看方便处安排些斋饭与师父吃。”长老依言四众共入但见二门上有一对春联:“黄芽白雪神仙府瑶草琪花羽士家。”行者笑道:“这个是烧茅炼药弄炉火提罐子的道士。”三藏捻他一把道:“谨言!谨言!我们不与他相识又不认亲左右暂时一会管他怎的?”说不了进了二门只见那正殿谨闭东廊下坐着一个道士在那里丸药。你看他怎生打扮:戴一顶红艳艳戗金冠穿一领黑淄淄乌皂服踏一双绿阵阵云头履系一条黄拂拂吕公绦。面如瓜铁目若朗星。准头高大类回回唇口翻张如达达。

道心一片隐轰雷伏虎降龙真羽士。三藏见了厉声高叫道:

“老神仙贫僧问讯了。”那道士猛抬头一见心惊丢了手中之药按簪儿整衣服降阶迎接道:“老师父失迎了请里面坐。”

长老欢喜上殿推开门见有三清圣象供桌有炉有香即拈香注炉礼拜三匝方与道士行礼。遂至客位中同徒弟们坐下。

急唤仙童看茶当有两个小童即入里边寻茶盘洗茶盏擦茶匙办茶果。忙忙的乱走早惊动那几个冤家。

原来那盘丝洞七个女怪与这道士同堂学艺自从穿了旧衣唤出儿子径来此处。正在后面裁剪衣服忽见那童子看茶便问道:“童儿有甚客来了这般忙冗?”仙童道:“适间有四个和尚进来师父教来看茶。”女怪道:“可有个白胖和尚?”

道:“有。”又问:“可有个长嘴大耳朵的?”道:“有。”女怪道:“你快去递了茶对你师父丢个眼色着他进来我有要紧的话说。”果然那仙童将五杯茶拿出去。道士敛衣双手拿一杯递与三藏然后与八戒、沙僧、行者。茶罢收锺小童丢个眼色那道士就欠身道:“列位请坐。”教:“童儿放了茶盘陪侍等我去去就来。”此时长老与徒弟们并一个小童出殿上观玩不题。

却说道士走进方丈中只见七个女子齐齐跪倒叫:“师兄!师兄!听小妹子一言!”道士用手搀起道:“你们早间来时要与我说甚么话可可的今日丸药这枝药忌见阴人所以不曾答你。如今又有客在外面有话且慢慢说罢。”众怪道:“告禀师兄这桩事专为客来方敢告诉若客去了纵说也没用了。”

道士笑道:“你看贤妹说话怎么专为客来才说?却不疯了?且莫说我是个清静修仙之辈就是个俗人家有妻子老小家务事也等客去了再处。怎么这等不贤替我装幌子哩!且让我出去。”众怪又一齐扯住道:“师兄息怒我问你前边那客是那方来的?”道士唾着脸不答应众怪道:“方才小童进来取茶我闻得他说是四个和尚。”道士作怒道:“和尚便怎么?”众怪道:“四个和尚内有一个白面胖的有一个长嘴大耳的师兄可曾问他是那里来的?”道士道:“内中是有这两个你怎么知道?想是在那里见他来?”女子道:“师兄原不知这个委曲。那和尚乃唐朝差往西天取经去的今早到我洞里化斋委是妹子们闻得唐僧之名将他拿了。”道士道:“你拿他怎的?”女子道:

“我等久闻人说唐僧乃十世修行的真体有人吃他一块肉延寿长生故此拿了他。后被那个长嘴大耳朵的和尚把我们拦在濯垢泉里先抢了衣服后弄本事强要同我等洗浴也止他不住。他就跳下水变作一个鲇鱼在我们腿裆里钻来钻去欲行奸骗之事果有十分惫懒!他又跳出水去现了本相见我们不肯相从他就使一柄九齿钉钯要伤我们性命。若不是我们有些见识几乎遭他毒手。故此战兢兢逃生又着你愚外甥与他敌斗不知存亡如何。我们特来投兄长望兄长念昔日同窗之雅与我今日做个报冤之人!”那道士闻此言却就恼恨遂变了声色道:“这和尚原来这等无礼!这等惫懒!你们都放心等我摆布他!”众女子谢道:“师兄如若动手等我们都来相帮打他。”道士道:“不用打!不用打!常言道一打三分低你们都跟我来。”

众女子相随左右。他入房内取了梯子转过床后爬上屋梁拿下一个小皮箱儿。那箱儿有八寸高下一尺长短四寸宽窄上有一把小铜锁儿锁住。即于袖中拿出一方鹅黄绫汗巾儿来汗巾须上系着一把小钥匙儿。开了锁取出一包儿药来此药乃是:山中百鸟粪扫积上千斤。是用铜锅煮煎熬火候匀。

千斤熬一杓一杓炼三分。三分还要炒再锻再重熏。制成此毒药贵似宝和珍。如若尝他味入口见阎君!道士对七个女子道:“妹妹我这宝贝若与凡人吃只消一厘入腹就死;若与神仙吃也只消三厘就绝。这些和尚只怕也有些道行须得三厘。快取等子来。”内一女子急拿了一把等子道:“称出一分二厘分作四分。”却拿了十二个红枣儿将枣掐破些儿揌上一厘分在四个茶锺内;又将两个黑枣儿做一个茶锺着一个托盘安了对众女说:“等我去问他。不是唐朝的便罢;若是唐朝来的就教换茶你却将此茶令童儿拿出。但吃了个个身亡就与你报了此仇解了烦恼也。”七女感激不尽。

那道士换了一件衣服虚礼谦恭走将出去请唐僧等又至客位坐下道:“老师父莫怪适间去后面吩咐小徒教他们挑些青菜萝卜安排一顿素斋供养所以失陪。”三藏道:“贫僧素手进拜怎么敢劳赐斋?”道士笑云:“你我都是出家人见山门就有三升俸粮何言素手?敢问老师父是何宝山?到此何干?”

三藏道:“贫僧乃东土大唐驾下差往西天大雷音寺取经者。却才路过仙宫竭诚进拜。”道士闻言满面生春道:“老师乃忠诚大德之佛小道不知失于远候恕罪!恕罪!”叫:“童儿快去换茶来一厢作办斋。”那小童走将进去众女子招呼他来道:“这里有现成好茶拿出去。”那童子果然将五锺茶拿出。道士连忙双手拿一个红枣儿茶锺奉与唐僧。他见八戒身躯大就认做大徒弟沙僧认做二徒弟见行者身量小认做三徒弟所以第四锺才奉与行者。行者眼乖接了茶锺早已见盘子里那茶锺是两个黑枣儿他道:“先生我与你穿换一杯。”道士笑道:“不瞒长老说山野中贫道士茶果一时不备。才然在后面亲自寻果子止有这十二个红枣做四锺茶奉敬。小道又不可空陪所以将两个下色枣儿作一杯奉陪此乃贫道恭敬之意也。”行者笑道:“说那里话?古人云在家不是贫路上贫杀人。

你是住家儿的何以言贫!象我们这行脚僧才是真贫哩。我和你换换我和你换换。”三藏闻言道:“悟空这仙长实乃爱客之意你吃了罢换怎的?”行者无奈将左手接了右手盖住看着他们。

却说那八戒一则饥二则渴原来是食肠大大的见那锺子里有三个红枣儿拿起来锺的都咽在肚里。师父也吃了沙僧也吃了。一霎时只见八戒脸上变色沙僧满眼流泪唐僧口中吐沫他们都坐不住晕倒在地。这大圣情知是毒将茶锺手举起来望道士劈脸一掼。道士将袍袖隔起当的一声把个锺子跌得粉碎。道士怒道:“你这和尚十分村卤!怎么把我锺子碎了?”行者骂道:“你这畜生!你看我那三个人是怎么说!我与你有甚相干你却将毒药茶药倒我的人?”道士道:“你这个村畜生闯下祸来你岂不知?”行者道:“我们才进你门方叙了坐次道及乡贯又不曾有个高言那里闯下甚祸?”道士道:

“你可曾在盘丝洞化斋么?你可曾在濯垢泉洗澡么?”行者道:

“濯垢泉乃七个女怪。你既说出这话必定与他苟合必定也是妖精!不要走!吃我一棒!”好大圣去耳朵里摸出金箍棒幌一幌碗来粗细望道士劈脸打来。那道士急转身躲过取一口宝剑来迎。他两个厮骂厮打早惊动那里边的女怪。他七个一拥出来叫道:“师兄且莫劳心待小妹子拿他。”行者见了越生嗔怒双手轮铁棒丢开解数滚将进去乱打。只见那七个敞开怀腆着雪白肚子脐孔中作出法来:骨都都丝绳乱冒搭起一个天篷把行者盖在底下。行者见事不谐即翻身念声咒语打个筋斗扑的撞破天篷走了忍着性气淤淤的立在空中看处见那怪丝绳幌亮穿穿道道却是穿梭的经纬顷刻间把黄花观的楼台殿阁都遮得无影无形。行者道:“利害!利害!早是不曾着他手!怪道猪八戒跌了若干!似这般怎生是好!我师父与师弟却又中了毒药。这伙怪合意同心却不知是个甚来历待我还去问那土地神也。”

好大圣按落云头捻着诀念声“唵”字真言把个土地老儿又拘来了战兢兢跪下路旁叩头道:“大圣你去救你师父的为何又转来也?”行者道:“早间救了师父前去不远遇一座黄花观。我与师父等进去看看那观主迎接。才叙话间被他把毒药茶药倒我师父等。我幸不曾吃茶使棒就打他却说出盘丝洞化斋、濯垢泉洗澡之事我就知那厮是怪。才举手相敌只见那七个女子跑出吐放丝绳老孙亏有见识走了。我想你在此间为神定知他的来历。是个甚么妖精老实说来免打!”土地叩头道:“那妖精到此住不上十年。小神自三年前检点之后方见他的本相乃是七个蜘蛛精。他吐那些丝绳乃是蛛丝。”行者闻言十分欢喜道:“据你说却是小可。既这般你回去等我作法降他也。”那土地叩头而去。行者却到黄花观外将尾巴上毛捋下七十根吹口仙气叫“变!”即变做七十个小行者;又将金箍棒吹口仙气叫“变!”即变做七十个双角叉儿棒。每一个小行者与他一根。他自家使一根站在外边将叉儿搅那丝绳一齐着力打个号子把那丝绳都搅断各搅了有十余斤。里面拖出七个蜘蛛足有巴斗大的身躯一个个攒着手脚索着头只叫:“饶命!饶命!”此时七十个小行者按住七个蜘蛛那里肯放。行者道:“且不要打他只教还我师父师弟来。”那怪厉声高叫道:“师兄还他唐僧救我命也!”那道士从里边跑出道:“妹妹我要吃唐僧哩救不得你了。”行者闻言大怒道:“你既不还我师父且看你妹妹的样子!”好大圣把叉儿棒幌一幌复了一根铁棒双手举起把七个蜘蛛精尽情打烂却似七个劖肉布袋儿脓血淋淋却又将尾巴摇了两摇收了毫毛单身轮棒赶入里边来打道士。

那道士见他打死了师妹心甚不忍即狠举剑来迎。这一场各怀忿怒一个个大展神通这一场好杀:妖精轮宝剑大圣举金箍。都为唐朝三藏先教七女呜呼。如今大展经纶手施威弄法逞金吾大圣神光壮妖仙胆气粗。浑身解数如花锦双手腾那似辘轳。乒乓剑棒响。惨淡野云浮。劖言语使机谋一来一往如画图。杀得风响沙飞狼虎怕天昏地暗斗星无。那道士与大圣战经五六十合渐觉手软一时间松了筋节便解开衣带忽辣的响一声脱了皂袍。行者笑道:“我儿子!打不过人就脱剥了也是不能彀的!”原来这道士剥了衣裳把手一齐抬起只见那两胁下有一千只眼眼中迸放金光十分利害:

森森黄雾艳艳金光森森黄雾两边胁下似喷云;艳艳金光千只眼中如放火。左右却如金桶东西犹似铜钟。此乃妖仙施法力道士显神通幌眼迷天遮日月罩人爆燥气朦胧;把个齐天孙大圣困在金光黄雾中。行者慌了手脚只在那金光影里乱转向前不能举步退后不能动脚却便似在个桶里转的一般。无奈又爆燥不过他急了往上着实一跳却撞破金光扑的跌了一个倒栽葱觉道撞的头疼急伸头摸摸把顶梁皮都撞软了自家心焦道:“晦气!晦气!这颗头今日也不济了!常时刀砍斧剁莫能伤损却怎么被这金光撞软了皮肉?久以后定要贡脓纵然好了也是个破伤风。”一会家爆燥难禁却又自家计较道:“前去不得后退不得左行不得右行不得往上又撞不得却怎么好?往下走他娘罢!”

好大圣念个咒语摇身一变变做个穿山甲又名鲮鲤鳞。真个是:四只铁爪钻山碎石如挝粉;满身鳞甲破岭穿岩似切葱。两眼光明好便似双星幌亮;一嘴尖利胜强如钢钻金锥。药中有性穿山甲俗语呼为鲮鲤鳞。你看他硬着头往地下一钻就钻了有二十余里方才出头。原来那金光只罩得十余里。出来现了本相力软筋麻浑身疼痛止不住眼中流泪忽失声叫道:“师父啊!当年秉教出山中共往西来苦用工。大海洪波无恐惧阳沟之内却遭风!”

美猴王正当悲切忽听得山背后有人啼哭即欠身揩了眼泪回头观看。但见一个妇人身穿重孝左手托一盏凉浆水饭右手执几张烧纸黄钱从那厢一步一声哭着走来。行者点头嗟叹道:“正是流泪眼逢流泪眼断肠人遇断肠人!这一个妇人不知所哭何事待我问他一问。”那妇人不一时走上路来迎着行者。行者躬身问道:“女菩萨你哭的是甚人?”妇人噙泪道:“我丈夫因与黄花观观主买竹竿争讲被他将毒药茶药死我将这陌纸钱烧化以报夫妇之情。”行者听言眼中泪下。那妇女见了作怒道:“你甚无知!我为丈夫烦恼生悲你怎么泪眼愁眉欺心戏我?”行者躬身道:“女菩萨息怒我本是东土大唐钦差御弟唐三藏大徒弟孙悟空行者。因往西天行过黄花观歇马。那观中道士不知是个甚么妖精他与七个蜘蛛精结为兄妹。蜘蛛精在盘丝洞要害我师父是我与师弟八戒、沙僧救解得脱。那蜘蛛精走到他这里背了是非说我等有欺骗之意。道士将毒药茶药倒我师父师弟共三人连马四口陷在他观里。

惟我不曾吃他茶将茶锺掼碎他就与我相打。正嚷时那七个蜘蛛精跑出来吐放丝绳将我捆住是我使法力走脱。问及土地说他本相我却又使分身法搅绝丝绳拖出妖来一顿棒打死。这道士即与他报仇举宝剑与我相斗。斗经六十回合他败了阵随脱了衣裳两胁下放出千只眼有万道金光把我罩定。所以进退两难才变做一个鲮鲤鳞从地下钻出来。正自悲切忽听得你哭故此相问。因见你为丈夫有此纸钱报答我师父丧身更无一物相酬所以自怨生悲岂敢相戏!”那妇女放下水饭纸钱对行者陪礼道:“莫怪莫怪我不知你是被难者。才据你说将起来你不认得那道士。他本是个百眼魔君又唤做多目怪。你既然有此变化脱得金光战得许久必定有大神通却只是还近不得那厮。我教你去请一位圣贤他能破得金光降得道士。”行者闻言连忙唱喏道:“女菩萨知此来历烦为指教指教。果是那位圣贤我去请求救我师父之难就报你丈夫之仇。”妇人道:“我就说出来你去请他降了道士只可报仇而已恐不能救你师父。”行者道:“怎不能救?”妇人道:“那厮毒药最狠:药倒人三日之间骨髓俱烂。你此往回恐迟了故不能救。”行者道:“我会走路;凭他多远千里只消半日。”女子道:“你既会走路听我说:此处到那里有千里之遥。那厢有一座山名唤紫云山山中有个千花洞。洞里有位圣贤唤做毗蓝婆。他能降得此怪。”行者道:“那山坐落何方?

却从何方去?”女子用手指定道:“那直南上便是。”行者回头看时那女子早不见了。行者慌忙礼拜道:“是那位菩萨?我弟子钻昏了不能相识千乞留名好谢!”只见那半空中叫道:“大圣是我。”行者急抬头看处原是黎山老姆赶至空中谢道:

“老姆从何来指教我也?”老姆道:“我才自龙华会上回来见你师父有难假做孝妇借夫丧之名免他一死。你快去请他但不可说出是我指教那圣贤有些多怪人。”

行者谢了辞别把筋斗云一纵随到紫云山上按定云头就见那千花洞。那洞外:青松遮胜境翠柏绕仙居。绿柳盈山道奇花满涧渠。香兰围石屋芳草映岩嵎。流水连溪碧云封古树虚。野禽声聒聒幽鹿步徐徐。修竹枝枝秀红梅叶叶舒。寒鸦栖古树春鸟嗓高樗。夏麦盈田广秋禾遍地余。四时无叶落八节有花如。每生瑞霭连霄汉常放祥云接太虚。这大圣喜喜欢欢走将进去一程一节看不尽天边的景致。直入里面更没个人儿见静静悄悄的鸡犬之声也无心中暗道:

“这圣贤想是不在家了。”又进数里看时见一个女道姑坐在榻上。你看他怎生模样:头戴五花纳锦帽身穿一领织金袍。脚踏云尖凤头履腰系攒丝双穗绦。面似秋容霜后老声如春燕社前娇。腹中久谙三乘法心上常修四谛饶。悟出空空真正果炼成了了自逍遥。正是千花洞里佛毗蓝菩萨姓名高。行者止不住脚近前叫道:“毗蓝婆菩萨问讯了。”那菩萨即下榻合掌回礼道:“大圣失迎了你从那里来的?”行者道:“你怎么就认得我是大圣?”毗蓝婆道:“你当年大闹天宫时普地里传了你的形象谁人不知那个不识?”行者道:“正是好事不出门恶事传千里象我如今皈正佛门你就不晓的了!”毗蓝道:“几时皈正?恭喜!恭喜!”行者道:“近能脱命保师父唐僧上西天取经师父遇黄花观道士将毒药茶药倒。我与那厮赌斗他就放金光罩住我是我使神通走脱了。闻菩萨能灭他的金光特来拜请。”菩萨道:“是谁与你说的?我自赴了盂兰会到今三百余年不曾出门。我隐姓埋名更无一人知得你却怎么得知?”

行者道:“我是个地里鬼不管那里自家都会访着。”毗蓝道:

“也罢也罢我本当不去奈蒙大圣下临不可灭了求经之善我和你去来。”行者称谢了道:“我忒无知擅自催促但不知曾带甚么兵器。”菩萨道:“我有个绣花针儿能破那厮。”行者忍不住道:“老姆误了我早知是绣花针不须劳你就问老孙要一担也是有的。”毗蓝道:“你那绣花针无非是钢铁金针用不得。我这宝贝非钢非铁非金乃我小儿日眼里炼成的。”

行者道:“令郎是谁?”毗蓝道:“小儿乃昴日星官。”行者惊骇不已。早望见金光艳艳即回向毗蓝道:“金光处便是黄花观也。”

毗蓝随于衣领里取出一个绣花针似眉毛粗细有五六分长短拈在手望空抛去。少时间响一声破了金光。行者喜道:

“菩萨妙哉妙哉!寻针寻针!”毗蓝托在手掌内道:“这不是?”

行者却同按下云头走入观里只见那道士合了眼不能举步。

行者骂道:“你这泼怪装瞎子哩!”耳朵里取出棒来就打。毗蓝扯住道:“大圣莫打且看你师父去。”行者径至后面客位里看时他三人都睡在地上吐痰吐沫哩。行者垂泪道:“却怎么好!

却怎么好”!毗蓝道:“大圣休悲也是我今日出门一场索性积个阴德我这里有解毒丹送你三丸。”行者转身拜求。那菩萨袖中取出一个破纸包儿内将三粒红丸子递与行者教放入口里。行者把药扳开他们牙关每人揌了一丸。须臾药味入腹便就一齐呕哕遂吐出毒味得了性命。那八戒先爬起道:“闷杀我也!”三藏沙僧俱醒了道:“好晕也!”行者道:“你们那茶里中了毒了亏这毗蓝菩萨搭救快都来拜谢。”三藏欠身整衣谢了。八戒道:“师兄那道士在那里?等我问他一问为何这般害我!”行者把蜘蛛精上项事说了一遍八戒狠道:“这厮既与蜘蛛为姊妹定是妖精!”行者指道:“他在那殿外立定装瞎子哩。”八戒拿钯就筑又被毗蓝止住道:“天蓬息怒大圣知我洞里无人待我收他去看守门户也。”行者道:“感蒙大德岂不奉承!但只是教他现本象我们看看。”毗蓝道:“容易。”即上前用手一指那道士扑的倒在尘埃现了原身乃是一条七尺长短的大蜈蚣精。毗蓝使小指头挑起驾祥云径转千花洞去。八戒打仰道:“这妈妈儿却也利害怎么就降这般恶物?”行者笑道:“我问他有甚兵器破他金光他道有个绣花针儿是他儿子在日眼里炼的。及问他令郎是谁他道是昴日星官。我想昴日星是只公鸡这老妈妈子必定是个母鸡。鸡最能降蜈蚣所以能收伏也。”三藏闻言顶礼不尽教:“徒弟们收拾去罢。”那沙僧即在里面寻了些米粮安排了些斋俱饱餐一顿。牵马挑担请师父出门。行者从他厨中放了一把火把一座观霎时烧得煨烬却拽步长行。正是唐僧得命感毗蓝了性消除多目怪。毕竟向前去还有甚么事体且听下回分解。
------------

第七十四回 长庚传报魔头狠 行者施为变化能

情欲原因总一般有情有欲自如然。沙门修炼纷纷士断欲忘情即是禅。须着意要心坚一尘不染月当天。行功进步休教错行满功完大觉仙。话表三藏师徒们打开欲网跳出情牢放马西行。走多时又是夏尽秋初新凉透体但见那:急雨收残暑梧桐一叶惊。萤飞莎径晚蛩语月华明。黄葵开映露红蓼遍沙汀。蒲柳先零落寒蝉应律鸣。三藏正然行处忽见一座高山峰插碧空真个是摩星碍日。长老心中害怕叫悟空道:“你看前面这山十分高耸但不知有路通行否。”行者笑道:“师父说那里话。自古道山高自有客行路水深自有渡船人岂无通达之理?可放心前去。”长老闻言喜笑花生扬鞭策马而进径上高岩。

行不数里见一老者鬓蓬松白飘搔;须稀朗银丝摆动。项挂一串数珠子手持拐杖现龙头。远远的立在那山坡上高呼:“西进的长老且暂住骅骝紧兜玉勒。这山上有一伙妖魔吃尽了阎浮世上人不可前进!”三藏闻言大惊失色。一是马的足下不平二是坐个雕鞍不稳扑的跌下马来挣挫不动睡在草里哼哩。行者近前搀起道:“莫怕莫怕!有我哩!”长老道:“你听那高岩上老者报道这山上有伙妖魔吃尽阎浮世上人谁敢去问他一个真实端的?”行者道:“你且坐地等我去问他。”三藏道:“你的相貌丑陋言语粗俗怕冲撞了他问不出个实信。”行者笑道:“我变个俊些儿的去问他。”三藏道:“你是变了我看。”好大圣捻着诀摇身一变变做个干干净净的小和尚几真个是目秀眉清头圆脸正行动有斯文之气象开口无俗类之言辞抖一抖锦衣直裰拽步上前向唐僧道:“师父我可变得好么?”三藏见了大喜道:“变得好!”八戒道:“怎么不好!只是把我们都比下去了。老猪就滚上二三年也变不得这等俊俏!”

好大圣躲离了他们径直近前对那老者躬身道:“老公公贫僧问讯了。”那老儿见他生得俊雅年少身轻待答不答的还了他个礼用手摸着他头儿笑嘻嘻问道:“小和尚你是那里来的?”行者道:“我们是东土大唐来的特上西天拜佛求经。

适到此间闻得公公报道有妖怪我师父胆小怕惧着我来问一声:端的是甚妖精他敢这般短路!烦公公细说与我知之我好把他贬解起身。”那老儿笑道:“你这小和尚年幼不知好歹言不帮衬。那妖魔神通广大得紧怎敢就说贬解他起身!”行者笑道:“据你之言似有护他之意必定与他有亲或是紧邻契友。不然怎么长他的威智兴他的节概不肯倾心吐胆说他个来历?”公公点头笑道:“这和尚倒会弄嘴!”想是跟你师父游方到处儿学些法术或者会驱缚魍魉与人家镇宅降邪你不曾撞见十分狠怪哩!”行者道:“怎的狠?”公公道:“那妖精一封书到灵出五百阿罗都来迎接;一纸简上天宫十一大曜个个相钦。四海龙曾与他为友八洞仙常与他作会十地阎君以兄弟相称社令城隍以宾朋相爱。”大圣闻言忍不住呵呵大笑用手扯着老者道:“不要说!不要说!那妖精与我后生小厮为兄弟朋友也不见十分高作。若知是我小和尚来啊他连夜就搬起身去了!”公公道:“你这小和尚胡说!不当人子!那个神圣是你的后生小厮?”行者笑道:“实不瞒你说我小和尚祖居傲来国花果山水帘洞姓孙名悟空。当年也曾做过妖精干过大事。曾因会众魔多饮了几杯酒睡着梦中见二人将批勾我去到阴司。一时怒将金箍棒打伤鬼判唬倒阎王几乎掀翻了森罗殿。吓得那掌案的判官拿纸十阎王佥名画字教我饶他打情愿与我做后生小厮。”那公公闻说道:“阿弥陀佛!这和尚说了这过头话莫想再长得大了。”行者道:“官儿似我这般大也彀了。”公公道:“你年几岁了?”行者道:“你猜猜看。”老者道:“有七八岁罢了。”行者笑道:“有一万个七八岁!我把旧嘴脸拿出来你看看你即莫怪。”公公道:“怎么又有个嘴脸?”行者道:“我小和尚有七十二副嘴脸哩。”那公公不识窍只管问他他就把脸抹一抹即现出本象咨牙倈嘴两股通红腰间系一条虎皮裙手里执一根金箍棒立在石崖之下就象个活雷公。那老者见了吓得面容失色腿脚酸麻站不稳扑的一跌;爬起来又一个躘蹲。大圣上前道:“老官儿不要虚惊我等面恶人善。莫怕!莫怕!适间蒙你好意报有妖魔。委的有多少怪一累你说说我好谢你。”那老儿战战兢兢口不能言又推耳聋一句不应。

行者见他不言即抽身回坡。长老道:“悟空你来了?所问如何?”行者笑道:“不打紧!不打紧!西天有便有个把妖精儿只是这里人胆小把他放在心上。没事没事!有我哩!”长老道:“你可曾问他此处是甚么山甚么洞有多少妖怪那条路通得雷音?”八戒道:“师父莫怪我说。若论赌变化使促掐捉弄人我们三五个也不如师兄;若论老实象师兄就摆一队伍也不如我。“唐僧道:“正是!正是!你还老实。”八戒道:

“他不知怎么钻过头不顾尾的问了两声不狤不魀的就跑回来了。等老猪去问他个实信来。”唐僧道:“悟能你仔细着。”

好呆子把钉钯撒在腰里整一整皂直裰扭扭捏捏奔上山坡对老者叫道:“公公唱喏了。”那老儿见行者回去方拄着杖挣得起来战战兢兢的要走忽见八戒愈觉惊怕道:“爷爷呀!今夜做的甚么恶梦遇着这伙恶人!为先的那和尚丑便丑还有三分人相;这个和尚怎么这等个碓梃嘴蒲扇耳朵铁片脸毧毛颈项一分人气儿也没有了!”八戒笑道:“你这老公公不高兴有些儿好褒贬人你是怎的看我哩?丑便丑奈看再停一时就俊了。”那老者见他说出人话来只得开言问他:“你是那里来的?”八戒道:“我是唐僧第二个徒弟法名叫做悟能八戒。才自先问的叫做悟空行者是我师兄。师父怪他冲撞了公公不曾问得实信所以特着我来拜问。此处果是甚山甚洞洞里果是甚妖精那里是西去大路烦公公指示指示。”老者道:“可老实么?”八戒道:“我生平不敢有一毫虚的。”

老者道:“你莫象才来的那个和尚走花弄水的胡缠。”八戒道:

“我不象他。”

公公拄着杖对八戒说:“此山叫做八百里狮驼岭中间有座狮驼洞洞里有三个魔头。”八戒啐了一声:“你这老儿却也多心!三个妖魔也费心劳力的来报遭信!”公公道:“你不怕么?”八戒道:“不瞒你说这三个妖魔我师兄一棍就打死一个我一钯就筑死一个我还有个师弟他一降妖杖又打死一个。三个都打死我师父就过去了有何难哉!”那老者笑道:

“这和尚不知深浅!那三个魔头神通广大得紧哩!他手下小妖南岭上有五千北岭上有五千东路口有一万西路口有一万;巡哨的有四五千把门的也有一万;烧火的无数打柴的也无数:共计算有四万七八千。这都是有名字带牌儿的专在此吃人。”那呆子闻得此言战兢兢跑将转来相近唐僧且不回话放下钯在那里出恭。行者见了喝道:“你不回话却蹲在那里怎的?”八戒道:“唬出屎来了!如今也不消说赶早儿各自顾命去罢!”行者道:“这个呆根!我问信偏不惊恐你去问就这等慌张失智!”长老道:“端的何如?”八戒道:“这老儿说:此山叫做八百里狮驼山中间有座狮驼洞洞里有三个老妖有四万八千小妖专在那里吃人。我们若躧着他些山边儿就是他口里食了莫想去得!”三藏闻言战兢兢毛骨悚然道:“悟空如何是好?”行者笑道:“师父放心没大事。想是这里有便有几个妖精只是这里人胆小把他就说出许多人许多大所以自惊自怪。有我哩!”八戒道:“哥哥说的是那里话!我比你不同我问的是实决无虚谬之言。满出满谷都是妖魔怎生前进?”行者笑道:“呆子嘴脸不要虚惊!若论满山满谷之魔只消老孙一路棒半夜打个罄尽!”八戒道:“不羞不羞莫说大话!那些妖精点卯也得七八日怎么就打得罄尽?”行者道:“你说怎样打?”八戒道:“凭你抓倒捆倒使定身法定倒也没有这等快的。”行者笑道:“不用甚么抓拿捆缚。我把这棍子两头一扯叫长就有四十丈长短;幌一幌叫粗就有八丈围圆粗细。(WWW.mianhuatang.la 好看的小说)往山南一滚滚杀五千;山北一滚滚杀五千;从东往西一滚只怕四五万砑做肉泥烂酱!”八戒道:“哥哥若是这等赶面打或者二更时也都了了。”沙僧在旁笑道:“师父有大师兄恁样神通怕他怎的!请上马走啊。”唐僧见他们讲论手段没奈何只得宽心上马而走。

正行间不见了那报信的老者沙僧道:“他就是妖怪故意狐假虎威的来传报恐唬我们哩。”行者道:“不要忙等我去看看。”好大圣跳上高峰四顾无迹急转面见半空中有彩霞幌亮即纵云赶上看时乃是太白金星。走到身边用手扯住口口声声只叫他的小名道:“李长庚!李长庚!你好惫懒!有甚话当面来说便好怎么装做个山林之老魇样混我!”金星慌忙施礼道:“大圣报信来迟乞勿罪!乞勿罪!这魔头果是神通广大势要峥嵘只看你挪移变化乖巧机谋可便过去;如若怠慢些儿其实难去。”行者谢道:“感激!感激!果然此处难行望老星上界与玉帝说声借些天兵帮助老孙帮助。”金星道:“有!有!有!你只口信带去就是十万天兵也是有的。”

大圣别了金星按落云头见了三藏道:“适才那个老儿原是太白星来与我们报信的。”长老合掌道:“徒弟快赶上他问他那里另有个路我们转了去罢。”行者道:“转不得此山径过有八百里四周围不知更有多少路哩怎么转得?”三藏闻言止不住眼中流泪道:“徒弟似此艰难怎生拜佛!”行者道:“莫哭莫哭!一哭便脓包行了!他这报信必有几分虚话只是要我们着意留心诚所谓以告者过也。你且下马来坐着。”八戒道:

“又有甚商议?”行者道:“没甚商议你且在这里用心保守师父沙僧好生看守行李马匹等老孙先上岭打听打听看前后共有多少妖怪拿住一个问他个详细教他写个执结开个花名把他老老小小一一查明吩咐他关了洞门不许阻路却请师父静静悄悄的过去方显得老孙手段!”沙僧只教:“仔细!

仔细!”行者笑道:“不消嘱咐我这一去就是东洋大海也荡开路就是铁裹银山也撞透门!”

好大圣唿哨一声纵筋斗云跳上高峰扳藤负葛平山观看那山里静悄无人。忽失声道:“错了!错了!不该放这金星老儿去了他原来恐唬我这里那有个甚么妖精!他就出来跳风顽耍必定拈枪弄棒操演武艺如何没有一个?”正自家揣度只听得山背后叮叮当当、辟辟剥剥梆铃之声。急回头看处原来是个小妖儿掮着一杆“令”字旗腰间悬着铃子手里敲着梆子从北向南而走。仔细看他有一丈二尺的身子。行者暗笑道:“他必是个铺兵想是送公文下报帖的。且等我去听他一听看他说些甚话。”好大圣捻着诀念个咒摇身一变变做个苍蝇儿轻轻飞在他帽子上侧耳听之。只见那小妖走上大路敲着梆摇着铃口里作念道:“我等寻山的各人是谨慎堤防孙行者:他会变苍蝇!”行者闻言暗自惊疑道:“这厮看见我了若未看见怎么就知我的名字又知我会变苍蝇!”原来那小妖也不曾见他只是那魔头不知怎么就吩咐他这话却是个谣言着他这等胡念。行者不知反疑他看见就要取出棒来打他却又停住暗想道:“曾记得八戒问金星时他说老妖三个小妖有四万七八千名。似这小妖再多几万也不打紧却不知这三个老魔有多大手段。等我问他一问动手不迟。”好大圣!你道他怎么去问?跳下他的帽子来钉在树头上让那小妖先行几步急转身腾那也变做个小妖儿照依他敲着梆摇着铃掮着旗一般衣服只是比他略长了三五寸口里也那般念着赶上前叫道:“走路的等我一等。”那小妖回头道:“你是那里来的?”行者笑道:“好人呀!一家人也不认得!”小妖道:

“我家没你呀。”行者道:“怎的没我?你认认看。”小妖道:“面生认不得!认不得!”行者道:“可知道面生我是烧火的你会得我少。”小妖摇头道:“没有!没有!我洞里就是烧火的那些兄弟也没有这个嘴尖的。”行者暗想道:“这个嘴好的变尖了些了。”即低头把手侮着嘴揉一揉道:“我的嘴不尖啊。”真个就不尖了。那小妖道:“你刚才是个尖嘴怎么揉一揉就不尖了?疑惑人子!大不好认!不是我一家的!少会少会!可疑可疑!我那大王家法甚严烧火的只管烧火巡山的只管巡山终不然教你烧火又教你来巡山?”行者口乖就趁过来道:“你不知道大王见我烧得火好就升我来巡山。”小妖道:“也罢!

我们这巡山的一班有四十名十班共四百名各自年貌各自名色。大王怕我们乱了班次不好点卯一家与我们一个牌儿为号。你可有牌儿?”行者只见他那般打扮那般报事遂照他的模样变了因不曾看见他的牌儿所以身上没有。好大圣更不说没有就满口应承道:“我怎么没牌?但只是刚才领的新牌。拿你的出来我看。”那小妖那里知这个机括即揭起衣服贴身带着个金漆牌儿穿条绒线绳儿扯与行者看看。行者见那牌背是个威镇诸魔的金牌正面有三个真字是小钻风他却心中暗想道:“不消说了!但是巡山的必有个风字坠脚。”便道:“你且放下衣走过等我拿牌儿你看。”即转身插下手将尾巴梢儿的小毫毛拔下一根捻他把叫“变!”即变做个金漆牌儿也穿上个绿绒绳儿上书三个真字乃总钻风拿出来递与他看了。小妖大惊道:“我们都叫做个小钻风偏你又叫做个甚么总钻风!”行者干事找绝说话合宜就道:“你实不知大王见我烧得火好把我升个巡风又与我个新牌叫做总巡风教我管你这一班四十名兄弟也。”那妖闻言即忙唱喏道:

“长官长官新点出来的实是面生言语冲撞莫怪!”行者还着礼笑道:“怪便不怪你只是一件:见面钱却要哩。每人拿出五两来罢。”小妖道:“长官不要忙待我向南岭头会了我这一班的人一总打罢。”行者道:“既如此我和你同去。”那小妖真个前走大圣随后相跟。

不数里忽见一座笔峰。何以谓之笔峰?那山头上长出一条峰来约有四五丈高如笔插在架上一般故以为名。行者到边前把尾巴掬一掬跳上去坐在峰尖儿上叫道:“钻风!都过来!”那些小钻风在下面躬身道:“长官伺候。”行者道:“你可知大王点我出来之故?”小妖道:“不知。”行者道:“大王要吃唐僧只怕孙行者神通广大说他会变化只恐他变作小钻风来这里躧着路径打探消息把我升作总钻风来查勘你们这一班可有假的。”小钻风连声应道:“长官我们俱是真的。”行者道:“你既是真的大王有甚本事你可晓得?”小钻风道:“我晓得。”行者道:“你晓得快说来我听。如若说得合着我便是真的;若说差了一些儿便是假的我定拿去见大王处治。”那小钻风见他坐在高处弄獐弄智呼呼喝喝的没奈何只得实说道:“我大王神通广大本事高强一口曾吞了十万天兵。”行者闻说吐出一声道:“你是假的!”小钻风慌了道:“长官老爷我是真的怎么说是假的?”行者道:“你既是真的如何胡说!大王身子能有多大一口都吞了十万天兵?”小钻风道:“长官原来不知我大王会变化:要大能撑天堂要小就如菜子。因那年王母娘娘设蟠桃大会邀请诸仙他不曾具柬来请我大王意欲争天被玉皇差十万天兵来降我大王是我大王变化法身张开大口似城门一般用力吞将去唬得众天兵不敢交锋关了南天门故此是一口曾吞十万兵。”行者闻言暗笑道:“若是讲手头之话老孙也曾干过。”又应声道:“二大王有何本事?”

小钻风道:“二大王身高三丈卧蚕眉丹凤眼美人声匾担牙鼻似蛟龙。若与人争斗只消一鼻子卷去就是铁背铜身也就魂亡魄丧!”行者道:“鼻子卷人的妖精也好拿。”又应声道:“三大王也有几多手段?”小钻风道:“我三大王不是凡间之怪物名号云程万里鹏行动时抟风运海振北图南。随身有一件儿宝贝唤做阴阳二气瓶。假若是把人装在瓶中一时三刻化为浆水。”行者听说心中暗惊道:“妖魔倒也不怕只是仔细防他瓶儿。”又应声道:“三个大王的本事你倒也说得不差与我知道的一样。但只是那个大王要吃唐僧哩?”小钻风道:“长官你不知道?”行者喝道:“我比你不知些儿!因恐汝等不知底细吩咐我来着实盘问你哩!”小钻风道:“我大大王与二大王久住在狮驼岭狮驼洞。三大王不在这里住他原住处离此西下有四百里远近。那厢有座城唤做狮驼国。他五百年前吃了这城国王及文武官僚满城大小男女也尽被他吃了干净因此上夺了他的江山如今尽是些妖怪。不知那一年打听得东土唐朝差一个僧人去西天取经说那唐僧乃十世修行的好人有人吃他一块肉就延寿长生不老。只因怕他一个徒弟孙行者十分利害自家一个难为径来此处与我这两个大王结为兄弟合意同心打伙儿捉那个唐僧也。”行者闻言心中大怒道:

“这泼魔十分无礼!我保唐僧成正果他怎么算计要吃我的人!”恨一声咬响钢牙掣出铁棒跳下高峰把棍子望小妖头上砑了一砑可怜就砑得象一个肉陀!自家见了又不忍道:

“咦!他倒是个好意把些家常话儿都与我说了我怎么却这一下子就结果了他?也罢也罢左右是左右!”好大圣只为师父阻路没奈何干出这件事来。就把他牌儿解下带在自家腰里将“令”字旗掮在背上腰间挂了铃手里敲着梆子迎风捻个诀口里念个咒语摇身一变变的就象小钻风模样拽回步径转旧路找寻洞府去打探那三个老妖魔的虚实。这正是:千般变化美猴王万样腾那真本事。

闯入深山依着旧路正走处忽听得人喊马嘶之声即举目观之原来是狮驼洞口有万数小妖排列着枪刀剑戟旗帜旌旄。这大圣心中暗喜道:“李长庚之言真是不妄!真是不妄!”

原来这摆列的有些路数:二百五十名作一大队伍。他只见有四十名杂彩长旗迎风乱舞就知有万名人马却又自揣自度道:

“老孙变作小钻风这一进去那老魔若问我巡山的话我必随机答应。倘或一时言语差讹认得我啊怎生脱体?就要往外跑时那伙把门的挡住如何出得门去?要拿洞里妖王必先除了门前众怪!”你道他怎么除得众怪?好大圣想着:“那老魔不曾与我会面就知我老孙的名头我且倚着我的这个名头仗着威风说些大话吓他一吓看。果然中土众僧有缘有分取得经回这一去只消我几句英雄之言就吓退那门前若干之怪;

假若众僧无缘无分取不得真经啊就是纵然说得莲花现也除不得西方洞外精。”心问口口问心思量此计敲着梆摇着铃径直闯到狮驼洞口早被前营上小妖挡住道:“小钻风来了?”行者不应低着头就走。走至二层营里又被小妖扯住道:

“小钻风来了?”行者道:“来了。”众妖道:“你今早巡风去可曾撞见甚么孙行者么?”行者道:“撞见的正在那里磨扛子哩。”

众妖害怕道:“他怎么个模样?磨甚么扛子?”行者道:“他蹲在那涧边还似个开路神;若站起来好道有十数丈长!手里拿着一条铁棒就似碗来粗细的一根大扛子在那石崖上抄一把水磨一磨口里又念着:“扛子啊!这一向不曾拿你出来显显神通这一去就有十万妖精也都替我打死!等我杀了那三个魔头祭你!他要磨得明了先打死你门前一万精哩!”那些小妖闻得此言一个个心惊胆战魂散魄飞。行者又道:“列位那唐僧的肉也不多几斤也分不到我处我们替他顶这个缸怎的!

不如我们各自散一散罢。”众妖都道:“说得是我们各自顾命去来。”假若是些军民人等服了圣化就死也不敢走。原来此辈都是些狼虫虎豹走兽飞禽呜的一声都哄然而去了。这个倒不象孙大圣几句铺头话却就如楚歌声吹散了八千兵!行者暗自喜道:“好了!老妖是死了!闻言就走怎敢觌面相逢?这进去还似此言方好;若说差了才这伙小妖有一两个倒走进去听见却不走了风讯?”你看他存心来古洞仗胆入深门。毕竟不知见那个老魔头有甚吉凶且听下回分解。
------------

第七十五回 心猿钻透阴阳窍 魔王还归大道真

却说孙大圣进于洞口两边观看只见:骷髅若岭骸骨如林。人头躧成毡片人皮肉烂作泥尘。人筋缠在树上干焦晃亮如银。真个是尸山血海果然腥臭难闻。东边小妖将活人拿了剐肉;西下泼魔把人肉鲜煮鲜烹。若非美猴王如此英雄胆第二个凡夫也进不得他门。不多时行入二层门里看时呀!这里却比外面不同:清奇幽雅秀丽宽平;左右有瑶草仙花前后有乔松翠竹。又行七八里远近才到三层门。闪着身偷着眼看处那上面高坐三个老妖十分狞恶。中间的那个生得:凿牙锯齿圆头方面。声吼若雷眼光如电。仰鼻朝天赤眉飘焰。但行处百兽心慌;若坐下群魔胆战。这一个是兽中王青毛狮子怪。左手下那个生得:凤目金睛黄牙粗腿。长鼻银毛看头似尾。圆额皱眉身躯磊磊。细声如窃窕佳人玉面似牛头恶鬼。这一个是藏齿修身多年的黄牙老象。右手下那一个生得:金翅鲲头星睛豹眼。振北图南刚强勇敢。变生翱翔鷃笑龙惨。抟风翮百鸟藏头舒利爪诸禽丧胆。这个是云程九万的大鹏雕。那两下列着有百十大小头目一个个全装披挂介胄整齐威风凛凛杀气腾腾。行者见了心中欢喜一些儿不怕大踏步径直进门把梆铃卸下朝上叫声“大王。”三个老魔笑呵呵问道:“小钻风你来了?”行者应声道:“来了。”你去巡山打听孙行者的下落何如?”行者道:“大王在上我也不敢说起。”老魔道:“怎么不敢说?”行者道:“我奉大王命敲着梆铃正然走处猛抬头只看见一个人蹲在那里磨扛子还象个开路神若站将起来足有十数丈长短。他就着那涧崖石上抄一把水磨一磨口里又念一声说他那扛子到此还不曾显个神通他要磨明就来打大王。我因此知他是孙行者特来报知。”那老魔闻此言浑身是汗唬得战呵呵的道:“兄弟我说莫惹唐僧。他徒弟神通广大预先作了准备磨棍打我们却怎生是好?”教:“小的们把洞外大小俱叫进来关了门让他过去罢。”那头目中有知道的报:“大王门外小妖已都散了。”老魔道:“怎么都散了?想是闻得风声不好也快早关门!快早关门!”众妖乒乓把前后门尽皆牢拴紧闭。行者自心惊道:“这一关了门他再问我家长里短的事我对不来却不弄走了风被他拿住?且再唬他一唬教他开着门好跑。”又上前道:“大王他还说得不好。”老魔道:“他又说甚么?”行者道:“他说拿大大王剥皮二大王剐骨三大王抽筋。你们若关了门不出去啊他会变化一时变了个苍蝇儿自门缝里飞进把我们都拿出去却怎生是好?”老魔道:“兄弟们仔细我这洞里递年家没个苍蝇但是有苍蝇进来就是孙行者。”行者暗笑道:“就变个苍蝇唬他一唬好开门。”大圣闪在旁边伸手去脑后拔了一根毫毛吹一口仙气叫“变!”即变做一个金苍蝇飞去望老魔劈脸撞了一头。那老怪慌了道:“兄弟!不停当!那话儿进门来了!”

惊得那大小群妖一个个丫钯扫帚都上前乱扑苍蝇。这大圣忍不住赥赥的笑出声来。干净他不宜笑这一笑笑出原嘴脸来了却被那第三个老妖魔跳上前一把扯住道:“哥哥险些儿被他瞒了!”老魔道:“贤弟谁瞒谁?”三怪道:“刚才这个回话的小妖不是小钻风他就是孙行者。必定撞见小钻风不知是他怎么打杀了却变化来哄我们哩。”行者慌了道:“他认得我了!”即把手摸摸对老怪道:“我怎么是孙行者?我是小钻风大王错认了。”老魔笑道:“兄弟他是小钻风。他一日三次在面前点卯我认得他。”又问:“你有牌儿么?”行者道:“有。”

掳着衣服就拿出牌子。老怪一认实道:“兄弟莫屈了他。”

三怪道:“哥哥你不曾看见他他才子闪着身笑了一声我见他就露出个雷公嘴来。见我扯住时他又变作个这等模样。”

叫:“小的们拿绳来!”众头目即取绳索。三怪把行者扳翻倒四马攒蹄捆住揭起衣裳看时足足是个弼马温。原来行者有七十二般变化若是变飞禽、走兽、花木、器皿、昆虫之类却就连身子滚去了;但变人物却只是头脸变了身子变不过来果然一身黄毛两块红股一条尾巴。老妖看着道:“是孙行者的身子小钻风的脸皮是他了!”教:“小的们先安排酒来与你三大王递个得功之杯。既拿倒了孙行者唐僧坐定是我们口里食也。”三怪道:“且不要吃酒。孙行者溜撒他会逃遁之法只怕走了。教小的们抬出瓶来把孙行者装在瓶里我们才好吃酒。”老魔大笑道:“正是!正是!”即点三十六个小妖入里面开了库房门抬出瓶来。你说那瓶有多大?只得二尺四寸高。怎么用得三十六个人抬?那瓶乃阴阳二气之宝内有七宝八卦、二十四气要三十六人按天罡之数才抬得动。不一时将宝瓶抬出放在三层门外展得干净揭开盖把行者解了绳索剥了衣服就着那瓶中仙气飕的一声吸入里面将盖子盖上贴了封皮却去吃酒道:“猴儿今番入我宝瓶之中再莫想那西方之路!若还能彀拜佛求经除是转背摇车再去投胎夺舍是。”你看那大小群妖一个个笑呵呵都去贺功不题。

却说大圣到了瓶中被那宝贝将身束得小了索性变化蹲在当中。半晌倒还荫凉忽失声笑道:“这妖精外有虚名内无实事。怎么告诵人说这瓶装了人一时三刻化为脓血?若似这般凉快就住上七八年也无事!”咦!大圣原来不知那宝贝根由:假若装了人一年不语一年荫凉但闻得人言就有火来烧了。大圣未曾说完只见满瓶都是火焰。幸得他有本事坐在中间捻着避火诀全然不惧。耐到半个时辰四周围钻出四十条蛇来咬。行者轮开手抓将过来尽力气一揝揝做八十段。少时间又有三条火龙出来把行者上下盘绕着实难禁自觉慌张无措道:“别事好处这三条火龙难为。再过一会不出弄得火气攻心怎了?”他想道:“我把身子长一长券破罢。”好大圣捻着诀念声咒叫“长!”即长了丈数高下那瓶紧靠着身也就长起去他把身子往下一小那瓶儿也就小下来了。行者心惊道:“难!难!难!怎么我长他也长我小他也小?如之奈何!”说不了孤拐上有些疼痛急伸手摸摸却被火烧软了自己心焦道:“怎么好?孤拐烧软了!弄做个残疾之人了!”忍不住吊下泪来这正是:遭魔遇苦怀三藏着难临危虑圣僧道:“师父啊!当年皈正蒙观音菩萨劝善脱离天灾我与你苦历诸山收殄多怪降八戒得沙僧千辛万苦指望同证西方共成正果。何期今日遭此毒魔老孙误入于此倾了性命撇你在半山之中不能前进!想是我昔日名高故有今朝之难!”正此凄怆忽想起菩萨当年在蛇盘山曾赐我三根救命毫毛不知有无且等我寻一寻看。即伸手浑身摸了一把只见脑后有三根毫毛十分挺硬忽喜道:“身上毛都如彼软熟只此三根如此硬枪必然是救我命的。”即便咬着牙忍着疼拔下毛吹口仙气叫“变!”一根即变作金钢钻一根变作竹片一根变作绵绳。扳张篾片弓儿牵着那钻照瓶底下飕飕的一顿钻钻成一个眼孔诱进光亮喜道:“造化!造化!却好出去也!”才变化出身那瓶复荫凉了。怎么就凉?原来被他钻了把阴阳之气泄了故此遂凉。

好大圣收了毫毛将身一小就变做个蟭蟟虫儿十分轻巧细如须长似眉毛自孔中钻出且还不走径飞在老魔头上钉着。那老魔正饮酒猛然放下杯儿道:“三弟孙行者这回化了么?”三魔笑道:“还到此时哩?”老魔教传令抬上瓶来。

那下面三十六个小妖即便抬瓶瓶就轻了许多慌得众小妖报道:“大王瓶轻了!”老魔喝道:“胡说!宝贝乃阴阳二气之全功如何轻了!”内中有一个勉强的小妖把瓶提上来道:“你看这不轻了?”老魔揭盖看时只见里面透亮忍不住失声叫道:

“这瓶里空者控也!”大圣在他头上也忍不住道一声“我的儿啊搜者走也!”众怪听见道:“走了走了!”即传令:“关门关门!”

那行者将身一抖收了剥去的衣服现本相跳出洞外。(WWW.mianhuatang.la 好看的小说)回头骂道:“妖精不要无礼!瓶子钻破装不得人了只好拿了出恭!”喜喜欢欢嚷嚷闹闹踏着云头径转唐僧处。那长老正在那里撮土为香望空祷祝行者且停云头听他祷祝甚的。那长老合掌朝天道:“祈请云霞众位仙六丁六甲与诸天。愿保贤徒孙行者神通广大法无边。”大圣听得这般言语更加努力收敛云光近前叫道:“师父我来了!”长老搀住道:“悟空劳碌你远探高山许久不回我甚忧虑。端的这山中有何吉凶?”行者笑道:“师父才这一去一则是东土众僧有缘有分二来是师父功德无量无边三也亏弟子法力!”将前项妆钻风、陷瓶里及脱身之事细陈了一遍“今得见尊师之面实为两世之人也!”长老感谢不尽道:“你这番不曾与妖精赌斗么?”行者道:

“不曾。”长老道:“这等保不得我过山了?”行者是个好胜的人叫喊道:“我怎么保你过山不得?”长老道:“不曾与他见个胜负只这般含糊我怎敢前进!”大圣笑道:“师父你也忒不通变。常言道单丝不线孤掌难鸣。那魔三个小妖千万教老孙一人怎生与他赌斗?”长老道:“寡不敌众是你一人也难处。八戒、沙僧他也都有本事教他们都去与你协力同心扫净山路保我过去罢。”行者沉吟道:“师言最当着沙僧保护你着八戒跟我去罢。“那呆子慌了道:“哥哥没眼色!我又粗夯无甚本事走路扛风跟你何益?”行者道:“兄弟你虽无甚本事好道也是个人。俗云放屁添风你也可壮我些胆气。”八戒道:“也罢也罢望你带挈带挈。但只急溜处莫捉弄我。”长老道:“八戒在意我与沙僧在此。”

那呆子抖擞神威与行者纵着狂风驾着云雾跳上高山即至洞口早见那洞门紧闭四顾无人。行者上前执铁棒厉声高叫道:“妖怪开门!快出来与老孙打耶!”那洞里小妖报入老魔心惊胆战道:“几年都说猴儿狠话不虚传果是真!”二老怪在旁问道:“哥哥怎么说?”老魔道:“那行者早间变小钻风混进来我等不能相识。幸三贤弟认得把他装在瓶里。他弄本事钻破瓶儿却又摄去衣服走了。如今在外叫战谁敢与他打个头仗?”更无一人答应又问又无人答都是那装聋推哑。老魔怒道:“我等在西方大路上忝着个丑名今日孙行者这般藐视若不出去与他见阵也低了名头。等我舍了这老性命去与他战上三合!三合战得过唐僧还是我们口里食;战不过那时关了门让他过去罢。”遂取披挂结束了开门前走。

行者与八戒在门旁观看真是好一个怪物:铁额铜头戴宝盔盔缨飘舞甚光辉。辉辉掣电双睛亮亮亮铺霞两鬓飞。勾爪如银尖且利锯牙似凿密还齐。身披金甲无丝缝腰束龙绦有见机。手执钢刀明晃晃英雄威武世间稀。一声吆喝如雷震问道“敲门者是谁?”大圣转身道:是你孙老爷齐天大圣也。”老魔笑道:“你是孙行者?大胆泼猴!我不惹你你却为何在此叫战?”行者道:“有风方起浪无潮水自平。你不惹我我好寻你?

只因你狐群狗党结为一伙算计吃我师父所以来此施为。”

老魔道:“你这等雄纠纠的嚷上我门莫不是要打么?”行者道:“正是。”老魔道:“你休猖獗!我若调出妖兵摆开阵势摇旗擂鼓与你交战显得我是坐家虎欺负你了。我只与你一个对一个不许帮丁!”行者闻言叫:“猪八戒走过看他把老孙怎的!”那呆子真个闪在一边。老魔道:“你过来先与我做个桩儿让我尽力气着光头砍上三刀就让你唐僧过去;假若禁不得快送你唐僧来与我做一顿下饭!”行者闻言笑道:“妖怪你洞里若有纸笔取出来与你立个合同。自今日起就砍到明年我也不与你当真!”那老魔抖擞威风丁字步站定双手举刀望大圣劈顶就砍。这大圣把头往上一迎只闻扢扠一声响头皮儿红也不红。那老魔大惊道:“这猴子好个硬头儿!”大圣笑道:“你不知老孙是:生就铜头铁脑盖天地乾坤世上无。斧砍锤敲不得碎幼年曾入老君炉。四斗星官监临适二十八宿用工夫。水浸几番不得坏周围扢搭板筋铺。唐僧还恐不坚固预先又上紫金箍。”老魔道:“猴儿不要说嘴!看我这二刀来决不容你性命!”行者道:“不见怎的左右也只这般砍罢了。”老魔道:“猴儿你不知这刀:金火炉中造神功百炼熬。锋刃依三略刚强按六韬。却似苍蝇尾犹如白蟒腰。入山云荡荡下海浪滔滔。琢磨无遍数煎熬几百遭。深山古洞放上阵有功劳。

搀着你这和尚天灵盖一削就是两个瓢!”大圣笑道:“这妖精没眼色!把老孙认做个瓢头哩!也罢误砍误让教你再砍一刀看怎么。”那老魔举刀又砍大圣把头迎一迎乒乓的劈做两半个;大圣就地打个滚变做两个身子。那妖一见慌了手按下钢刀。猪八戒远远望见笑道:“老魔好砍两刀的!却不是四个人了?”老魔指定行者道:“闻你能使分身法怎么把这法儿拿出在我面前使!”大圣道:“何为分身法?”老魔道:“为甚么先砍你一刀不动如今砍你一刀就是两个人?”大圣笑道:“妖怪你切莫害怕。砍上一万刀还你二万个人!”老魔道:“你这猴儿你只会分身不会收身。你若有本事收做一个打我一棍去罢。”大圣道:“不许说谎你要砍三刀只砍了我两刀;教我打一棍若打了棍半就不姓孙!”老魔道:“正是正是。”

好大圣就把身搂上来打个滚依然一个身子掣棒劈头就打那老魔举刀架住道:“泼猴无礼!甚么样个哭丧棒敢上门打人?”大圣喝道:“你若问我这条棍天上地下都有名声。”

老魔道:“怎见名声?”他道:“棒是九转镔铁炼老君亲手炉中煅。禹王求得号神珍四海八河为定验。中间星斗暗铺陈两头箝裹黄金片。花纹密布鬼神惊上造龙纹与凤篆。名号灵阳棒一条深藏海藏人难见。成形变化要飞腾飘飖五色霞光现。

老孙得道取归山无穷变化多经验。时间要大瓮来粗或小些微如铁线。粗如南岳细如针长短随吾心意变。轻轻举动彩云生亮亮飞腾如闪电。攸攸冷气逼人寒条条杀雾空中现。降龙伏虎谨随身天涯海角都游遍。曾将此棍闹天宫威风打散蟠桃宴。天王赌斗未曾赢哪吒对敌难交战。棍打诸神没躲藏天兵十万都逃窜。雷霆众将护灵霄飞身打上通明殿。掌朝天使尽皆惊护驾仙卿俱搅乱。举棒掀翻北斗宫回振开南极院。金阙天皇见棍凶特请如来与我见。兵家胜负自如然困苦灾危无可辨。整整挨排五百年亏了南海菩萨劝。大唐有个出家僧对天下洪誓愿。枉死城中度鬼魂灵山会上求经卷。

西方一路有妖魔行动甚是不方便。已知铁棒世无双央我途中为侣伴。邪魔汤着赴幽冥肉化红尘骨化面。处处妖精棒下亡论万成千无打算。上方击坏斗牛宫下方压损森罗殿。天将曾将九曜追地府打伤催命判。半空丢下振山川胜如太岁新华剑。全凭此棍保唐僧天下妖魔都打遍!”

那魔闻言战兢兢舍着性命举刀就砍。猴王笑吟吟使铁棒前迎。他两个先时在洞前撑持然后跳起去都在半空里厮杀。这一场好杀:天河定底神珍棒棒名如意世间高。夸称手段魔头恼大捍刀擎法力豪。门外争持还可近空中赌斗怎相饶!一个随心更面目一个立地长身腰。杀得满天云气重遍野雾飘飘。那一个几番立意吃三藏这一个广施法力保唐朝。

都因佛祖传经典邪正分明恨苦交。那老魔与大圣斗经二十余合不分输赢。原来八戒在底下见他两个战到好处忍不住掣钯架风跳将起去望妖魔劈脸就筑。那魔慌了不知八戒是个呼头性子冒冒失失的唬人他只道嘴长耳大手硬钯凶败了阵丢了刀回头就走。大圣喝道:“赶上!赶上!”这呆子仗着威风举着钉钯即忙赶下怪去。老魔见他赶的相近在坡前立定迎着风头幌一幌现了原身张开大口就要来吞八戒。八戒害怕急抽身往草里一钻也管不得荆针棘刺也顾不得刮破头疼战兢兢的在草里听着梆声。随后行者赶到那怪也张口来吞却中了他的机关收了铁棒迎将上去被老魔一口吞之。唬得个呆子在草里囊囊咄咄的埋怨道:“这个弼马温不识进退!那怪来吃你你如何不走反去迎他!这一口吞在肚中今日还是个和尚明日就是个大恭也!”那魔得胜而去。这呆子才钻出草来溜回旧路。

却说三藏在那山坡下正与沙僧盼望只见八戒喘呵呵的跑来。三藏大惊道:“八戒你怎么这等狼狈?悟空如何不见?”

呆子哭哭啼啼道:“师兄被妖精一口吞下肚去了!”三藏听言唬倒在地半晌间跌脚拳胸道:“徒弟呀!只说你善会降妖领我西天见佛怎知今日死于此怪之手!苦哉苦哉!我弟子同众的功劳如今都化作尘土矣!’那师父十分苦痛。你看那呆子他也不来劝解师父却叫:“沙和尚你拿将行李来我两个分了罢。”沙僧道:“二哥分怎的?”八戒道:”分开了各人散火:你往流沙河还去吃人;我往高老庄看看我浑家。将白马卖了与师父买个寿器送终。”长老气呼呼的闻得此言叫皇天放声大哭。且不题。

却说那老魔吞了行者以为得计径回本洞。众妖迎问出战之功老魔道:“拿了一个来了。”二魔喜道:“哥哥拿的是谁?”老魔道:“是孙行者。”二魔道:“拿在何处?”老魔道:“被我一口吞在腹中哩。”第三个魔头大惊道:“大哥啊我就不曾吩咐你孙行者不中吃!”那大圣肚里道:“忒中吃!又禁饥再不得饿”慌得那小妖道:“大王不好了!孙行者在你肚里说话哩!”老魔道:“怕他说话!有本事吃了他没本事摆布他不成?

你们快去烧些盐白汤等我灌下肚去把他哕出来慢慢的煎了吃酒。”小妖真个冲了半盆盐汤。老怪一饮而干洼着口着实一呕那大圣在肚里生了根动也不动却又拦着喉咙往外又吐吐得头晕眼花黄胆都破了行者越不动。老魔喘息了叫声:“孙行者你不出来?”行者道:“早哩!正好不出来哩!”老魔道:“你怎么不出?”行者道:“你这妖精甚不通变。我自做和尚十分淡薄:如今秋凉我还穿个单直裰。这肚里倒暖又不透风等我住过冬才好出来。”众妖听说都道:“大王孙行者要在你肚里过冬哩!”老魔道:“他要过冬我就打起禅来使个搬运法一冬不吃饭就饿杀那弼马温!”大圣道:“我儿子你不知事!老孙保唐僧取经从广里过带了个折迭锅儿进来煮杂碎吃。将你这里边的肝肠肚肺细细儿受用还彀盘缠到清明哩!”那二魔大惊道:“哥啊这猴子他干得出来!”

三魔道:“哥啊吃了杂碎也罢不知在那里支锅。”行者道:“三叉骨上好支锅。”三魔道:“不好了!假若支起锅烧动火烟煼到鼻孔里打嚏喷么?”行者笑道:“没事!等老孙把金箍棒往顶门里一搠搠个窟窿:一则当天窗二来当烟洞。”老魔听说虽说不怕却也心惊只得硬着胆叫:“兄弟们莫怕把我那药酒拿来等我吃几锺下去把猴儿药杀了罢!”行者暗笑道:“老孙五百年前大闹天宫时吃老君丹玉皇酒王母桃及凤髓龙肝那样东西我不曾吃过?是甚么药酒敢来药我?”那小妖真个将药酒筛了两壶满满斟了一锺递与老魔。老魔接在手中大圣在肚里就闻得酒香道:“不要与他吃!”好大圣把头一扭变做个喇叭口子张在他喉咙之下。那怪啯的咽下被行者啯的接吃了。第二锺咽下被行者啯的又接吃了。一连咽了七八锺都是他接吃了。老魔放下锺道:“不吃了这酒常时吃两锺腹中如火却才吃了七八锺脸上红也不红!”原来这大圣吃不多酒接了他七八锺吃了在肚里撒起酒风来不住的支架子跌四平踢飞脚抓住肝花打秋千竖蜻艇翻根头乱舞。

那怪物疼痛难禁倒在地下。毕竟不知死活如何且听下回分解。
------------

第七十六回 心神居舍魔归性 木母同降怪体真

话表孙大圣在老魔肚里支吾一会那魔头倒在尘埃无声无气若不言语想是死了却又把手放放。(WWW.mianhuatang.la 好看的小说)魔头回过气来叫一声:“大慈大悲齐天大圣菩萨!”行者听见道:“儿子莫废工夫省几个字儿只叫孙外公罢。”那妖魔惜命真个叫:“外公!

外公!是我的不是了!一差二误吞了你你如今却反害我。万望大圣慈悲可怜蝼蚁贪生之意饶了我命愿送你师父过山也。”大圣虽英雄甚为唐僧进步他见妖魔哀告好奉承的人也就回了善念叫道:“妖怪我饶你你怎么送我师父?”老魔道:“我这里也没甚么金银、珠翠、玛瑙、珊瑚、琉璃、琥珀、玳瑁珍奇之宝相送我兄弟三个抬一乘香藤轿儿把你师父送过此山。”行者笑道:“既是抬轿相送强如要宝。你张开口我出来。”那魔头真个就张开口。那三魔走近前悄悄的对老魔道:

“大哥等他出来时把口往下一咬将猴儿嚼碎咽下肚却不得磨害你了。”原来行者在里面听得便不先出去却把金箍棒伸出试他一试。那怪果往下一口扢喳的一声把个门牙都迸碎了。行者抽回棒道:“好妖怪!我倒饶你性命出来你反咬我要害我命!我不出来活活的只弄杀你!不出来!不出来!”老魔报怨三魔道:“兄弟你是自家人弄自家人了。且是请他出来好了你却教我咬他。他倒不曾咬着却迸得我牙龈疼痛这是怎么起的!“三魔见老魔怪他他又作个激将法厉声高叫道:

“孙行者闻你名如轰雷贯耳说你在南天门外施威灵霄殿下逞势。如今在西天路上降妖缚怪原来是个小辈的猴头!”行者道:“我何为小辈?”三怪道:“好汉千里客万里去传名。你出来我与你赌斗才是好汉;怎么在人肚里做勾当!非小辈而何?”行者闻言心中暗想道:“是是是!我若如今扯断他肠揌破他肝弄杀这怪有何难哉?但真是坏了我的名头。也罢!也罢!你张口我出来与你比并。但只是你这洞口窄逼不好使家火须往宽处去。”三魔闻说即点大小怪前前后后有三万多精都执着精锐器械出洞摆开一个三才阵势专等行者出口一齐上阵。那二怪搀着老魔径至门外叫道:“孙行者!好汉出来!此间有战场好斗!”

大圣在他肚里闻得外面鸦鸣鹊噪鹤唳风声知道是宽阔之处却想着:“我不出去是失信与他;若出去这妖精人面兽心。先时说送我师父哄我出来咬我今又调兵在此。也罢也罢与他个两全其美:出去便出去还与他肚里生下一个根儿。”即转手将尾上毫毛拔了一根吹口仙气叫“变!”即变一条绳儿只有头粗细倒有四十丈长短。那绳儿理出去见风就长粗了。把一头拴着妖怪的心肝系上打做个活扣儿那扣儿不扯不紧扯紧就痛。却拿着一头笑道:“这一出去他送我师父便罢;如若不送乱动刀兵我也没工夫与他打只消扯此绳儿就如我在肚里一般!”又将身子变得小小的往外爬爬到咽喉之下见妖精大张着方口上下钢牙排如利刃忽思量道:“不好!不好!若从口里出去扯这绳儿他怕疼往下一嚼却不咬断了?我打他没牙齿的所在出去。”好大圣理着绳儿从他那上腭子往前爬爬到他鼻孔里。那老魔鼻子痒“阿口妻”的一声打了个喷嚏却迸出行者。行者见了风把腰躬一躬就长了有三丈长短一只手扯着绳儿一只手拿着铁棒。那魔头不知好歹见他出来了就举钢刀劈脸来砍这大圣一只手使铁棒相迎。又见那二怪使枪三怪使戟没头没脸的乱上。大圣放松了绳收了铁棒急纵身驾云走了原来怕那伙小妖围绕不好干事。他却跳出营外去那空阔山头上落下云双手把绳尽力一扯老魔心里才疼。他害疼往上一挣大圣复往下一扯。众小妖远远看见齐声高叫道:“大王莫惹他!让他去罢!这猴儿不按时景清明还未到他却那里放风筝也!”

大圣闻言着力气蹬了一蹬那老魔从空中拍刺刺似纺车儿一般跌落尘埃就把那山坡下死硬的黄土跌做个二尺浅深之坑。慌得那二怪三怪一齐按下云头上前拿住绳儿跪在坡下哀告道:“大圣啊只说你是个宽洪海量之仙谁知是个鼠腹蜗肠之辈。实实的哄你出来与你见阵不期你在我家兄心上拴了一根绳子!”行者笑道:“你这伙泼魔十分无礼!前番哄我出去便就咬我这番哄我出来却又摆阵敌我。似这几万妖兵战我一个理上也不通扯了去!扯了去见我师父!”那怪一齐叩头道“大圣慈悲饶我性命愿送老师父过山!”行者笑道:“你要性命只消拿刀把绳子割断罢了。”老魔道:“爷爷呀割断外边的这里边的拴在心上喉咙里又菾菾的恶心怎生是好?”

行者道:“既如此张开口等我再进去解出绳来。”老魔慌了道:“这一进去又不肯出来却难也!却难也!”行者道:“我有本事外边就可以解得里面绳头也解了可实实的送我师父么?”老魔道:“但解就送决不敢打诳语。”大圣审得是实即便将身一抖收了毫毛那怪的心就不疼了。这是孙大圣掩样的法儿使毫毛拴着他的心收了毫毛所以就不害疼也。三个妖纵身而起谢道:“大圣请回上复唐僧收拾下行李我们就抬轿来送。”众怪偃干戈尽皆归洞。

大圣收绳子径转山东远远的看见唐僧睡在地下打滚痛哭猪八戒与沙僧解了包袱将行李搭分儿在那里分哩。行者暗暗嗟叹道:“不消讲了这定是八戒对师父说我被妖精吃了师父舍不得我痛哭那呆子却分东西散火哩。咦!不知可是此意且等我叫他一声看。”落下云头叫道:“师父!”沙僧听见报怨八戒道:“你是个棺材座子专一害人!师兄不曾死你却说他死了在这里干这个勾当!那里不叫将来了?”八戒道:“我分明看见他被妖精一口吞了。想是日辰不好那猴子来显魂哩。”

行者到跟前一把挝住八戒脸一个巴掌打了个踉跄道:“夯货!我显甚么魂?”呆子侮着脸道:“哥哥你实是那怪吃了你、你怎么又活了?”行者道:“象你这个不济事的脓包!他吃了我我就抓他肠捏他肺又把这条绳儿穿住地的心扯他疼痛难禁一个个叩头哀告我才饶了他性命。如今抬轿来送我师父过山也。”那三藏闻言一骨鲁爬起来对行者躬身道:“徒弟啊累杀你了!若信悟能之言我已绝矣!”行者轮拳打着八戒骂道:“这个馕糠的呆子十分懈怠甚不成*人!师父你切莫恼那怪就来送你也。”沙僧也甚生惭愧连忙遮掩收拾行李扣背马匹都在途中等候不题。

却说三个魔头帅群精回洞二怪道:“哥哥我只道是个九头八尾的孙行者原来是恁的个小小猴儿!你不该吞他只与他斗时他那里斗得过你我!洞里这几万妖精吐唾沫也可湅{杀他。你却将他吞在肚里他便弄起法来教你受苦怎么敢与他比较?才自说送唐僧都是假意实为兄长性命要紧所以哄他出来。决不送他!”老魔道:“贤弟不送之故何也?”二怪道:
	
	“你与我三千小妖摆开阵势我有本事拿住这个猴头!”老魔道:“莫说三千凭你起老营去只是拿住他便大家有功。”那二魔即点三千小妖径到大路旁摆开着一个蓝旗手往来传报教:“孙行者!赶早出来与我二大王爷爷交战!”八戒听见笑道:“哥啊常言道说谎不瞒当乡人就来弄虚头捣鬼!怎么说降了妖精就抬轿来送师父却又来叫战何也?”行者道:“老怪已被我降了不敢出头闻着个孙字儿也害头疼。这定是二妖魔不伏气送我们故此叫战。我道兄弟这妖精有弟兄三个这般义气;我弟兄也是三个就没些义气?我已降了大魔二魔出来你就与他战战未为不可。”八戒道:“怕他怎的!等我去打他一仗来!”行者道:“要去便去罢。”八戒笑道:“哥啊去便去你把那绳儿借与我使使。”行者道:“你要怎的?你又没本事钻在肚里你又没本事拴在他心上要他何用?”八戒道:“我要扣在这腰间做个救命索。你与沙僧扯住后手放我出去与他交战。估着赢了他你便放松我把他拿住;若是输与他你把我扯回来莫教他拉了去。”真个行者暗笑道:“也是捉弄呆子一番!”就把绳儿扣在他腰里撮弄他出战。(wwW.mianhuatang.la 无弹窗广告)
	
	那呆子举钉钯跑上山崖叫道:“妖精出来!与你猪祖宗打来!”那蓝旗手急报道:“大王有一个长嘴大耳朵的和尚来了。”二怪即出营见了八戒更不打话挺枪劈面刺来。这呆子举钯上前迎住。他两个在山坡前搭上手斗不上七八回合呆子手软架不得妖魔急回头叫:“师兄不好了!扯扯救命索扯扯救命索!”这壁厢大圣闻言转把绳子放松了抛将去。那呆子败了阵住后就跪。原来那绳子拖着走还不觉转回来因松了倒有些绊脚自家绊倒了一跌爬起来又一跌。始初还跌个躘踵后面就跌了个嘴抢地。被妖精赶上捽开鼻子就如蛟龙一般把八戒一鼻子卷住得胜回洞。众妖凯歌齐唱一拥而归。
	
	这坡下三藏看见又恼行者道:“悟空怪不得悟能咒你死哩!原来你兄弟全无相亲相爱之意专怀相嫉相妒之心!他那般说教你扯扯救命索你怎么不扯还将索子丢去?如今教他被害却如之何?”行者笑道:“师父也忒护短忒偏心!罢了象老孙拿去时你略不挂念左右是舍命之材;这呆子才自遭擒你就怪我。也教他受些苦恼方见取经之难。”三藏道:“徒弟啊你去我岂不挂念?想着你会变化断然不至伤身。那呆子生得狼犺又不会腾那这一去少吉多凶你还去救他一救。”
	
	行者道:“师父不得报怨等我去救他一救。”急纵身赶上山暗中恨道:“这呆子咒我死且莫与他个快活!且跟去看那妖精怎么摆布他等他受些罪再去救他。”即捻诀念起真言摇身一变即变做个蟭蟟虫飞将去钉在八戒耳朵根上同那妖精到了洞里。二魔帅三千小怪大吹大打的至洞口屯下自将八戒拿入里边道:“哥哥我拿了一个来也。”老怪道:“拿来我看。”
	
	他把鼻子放松捽下八戒道:“这不是?”老怪道:“这厮没用。”
	
	八戒闻言道:“大王没用的放出去寻那有用的捉来罢。”三怪道:“虽是没用也是唐僧的徒弟猪八戒。且捆了送在后边池塘里浸着待浸退了毛破开肚子使盐腌了晒干等天阴下酒。”八戒大惊道:“罢了罢了!撞见那贩腌的妖怪也!”众怪一齐下手把呆子四马攒蹄捆住扛扛抬抬送至池塘边往中间一推尽皆转去。
	
	大圣却飞起来看处那呆子四肢朝上掘着嘴半浮半沉嘴里呼呼的着然好笑倒象八九月经霜落了子儿的一个大黑莲蓬。大圣见他那嘴脸又恨他又怜他说道:“怎的好么?他也是龙华会上的一个人但只恨他动不动分行李散火又要撺掇师父念《紧箍咒》咒我。我前日曾闻得沙僧说也攒了些私房不知可有否等我且吓他一吓看。”好大圣飞近他耳边假捏声音叫声:“猪悟能!猪悟能!”八戒慌了道:“晦气呀!我这悟能是观世音菩萨起的自跟了唐僧又呼做八戒此间怎么有人知道我叫做悟能?”呆子忍不住问道:“是那个叫我的法名?”行者道:“是我。”呆子道:“你是那个?”行者道:“我是勾司人。”那呆子慌了道:“长官你是那里来的?”行者道:“我是五阎王差来勾你的。”那呆子道:“长官你且回去上复五阎王他与我师兄孙悟空交得甚好教他让我一日儿明日来勾罢。”
	
	行者道:“胡说!阎王注定三更死谁敢留人到四更!趁早跟我去免得套上绳子扯拉!”呆子道:”长官那里不是方便看我这般嘴脸还想活哩。死是一定死只等一日这妖精连我师父们都拿来会一会就都了帐也。”行者暗笑道:“也罢我这批上有三十个人都在这中前后等我拘将来就你便有一日耽阁。你可有盘缠把些儿我去。”八戒道:“可怜啊!出家人那里有甚么盘缠?”行者道:“若无盘缠索了去!跟着我走!”呆子慌了道:“长官不要索我晓得你这绳儿叫做追命绳索上就要断气。有有有!有便有些儿只是不多。”行者道:“在那里?快拿出来!”八戒道:“可怜可怜!我自做了和尚到如今有些善信的人家斋僧见我食肠大衬钱比他们略多些儿我拿了攒在这里零零碎碎有五钱银子因不好收拾前者到城中央了个银匠煎在一处他又没天理偷了我几分只得四钱六分一块儿你拿了去罢。”行者暗笑道:“这呆子裤子也没得穿却藏在何处?咄!你银子在那里?”八戒道:“在我左耳朵眼儿里揌着哩。我捆了拿不得你自家拿了去罢。”行者闻言即伸手在耳朵窍中摸出真个是块马鞍儿银子足有四钱五六分重拿在手里忍不住哈哈的大笑一声。那呆子认是行者声音在水里乱骂道:“天杀的弼马温!到这们苦处还来打诈财物哩!”行者又笑道:“我把你这馕糟的!老孙保师父不知受了多少苦难你到攒下私房!”八戒道:“嘴脸!这是甚么私房!都是牙齿上刮下来的我不舍得买了嘴吃留了买匹布儿做件衣服你却吓了我的。还分些儿与我。”行者道:“半分也没得与你!”八戒骂道:“买命钱让与你罢好道也救我出去是。”行者道:“莫急等我救你。”将银子藏了即现原身掣铁棒把呆子划拢用手提着脚扯上来解了绳。八戒跳起来脱下衣裳整干了水抖一抖潮漉漉的披在身上道:“哥哥开后门走了罢。”行者道:“后门里走可是个长进的?还打前门上去。”八戒道:“我的脚捆麻了跑不动。”行者道:“快跟我来。”
	
	好大圣把铁棒一路丢开解数打将出去。那呆子忍着麻只得跟定他只看见二门下靠着的是他的钉钯走上前推开小妖捞过来往前乱筑与行者打出三四层门不知打杀了多少小妖。那老魔听见对二魔道:“拿得好人!拿得好人!你看孙行者劫了猪八戒门上打伤小妖也!”那二魔急纵身绰枪在手赶出门来应声骂道:“泼猢狲!这般无礼!怎敢渺视我等!”
	
	大圣听得即应声站下。那怪物不容讲使枪便刺。行者正是会家不忙掣铁棒劈面相迎。他两个在洞门外这一场好杀:
	
	黄牙老象变人形义结狮王为弟兄。因为大魔来说合同心计算吃唐僧。齐天大圣神通广辅正除邪要灭精。八戒无能遭毒手悟空拯救出门行。妖王赶上施英猛枪棒交加各显能。那一个枪来好似穿林蟒这一个棒起犹如出海龙。龙出海门云霭霭蟒穿林树雾腾腾。算来都为唐和尚恨苦相持太没情。那八戒见大圣与妖精交战他在山嘴上竖着钉钯不来帮打只管呆呆的看着。那妖精见行者棒重满身解数全无破绽就把枪架住捽开鼻子要来卷他。行者知道他的勾当双手把金箍棒横起来往上一举被妖精一鼻子卷住腰胯不曾卷手。你看他两只手在妖精鼻头上丢花棒儿耍子。八戒见了捶胸道:
	
	“咦!那妖怪晦气呀!卷我这夯的连手都卷住了不能得动卷那们滑的倒不卷手。他那两只手拿着棒只消往鼻里一搠那孔子里害疼流涕怎能卷得他住?”行者原无此意倒是八戒教了他。他就把棒幌一幌小如鸡子长有丈余真个往他鼻孔里一搠。那妖精害怕沙的一声把鼻子捽放被行者转手过来一把挝住用气力往前一拉那妖精护疼随着手举步跟来。八戒方才敢近拿钉钯望妖精胯子上乱筑。行者道:“不好!
	
	不好!那钯齿儿尖恐筑破皮淌出血来师父看见又说我们伤生只调柄子来打罢。”真个呆子举钯柄走一步打一下行者牵着鼻子就似两个象奴牵至坡下只见三藏凝睛盼望见他两个嚷嚷闹闹而来即唤:“悟净你看悟空牵的是甚么?”沙僧见了笑道:“师父大师兄把妖精揪着鼻子拉来真爱杀人也!”
	
	三藏道:“善哉!善哉!那般大个妖精!那般长个鼻子!你且问他:他若喜喜欢欢送我等过山呵饶了他莫伤他性命。”沙僧急纵前迎着高声叫道:“师父说:那怪果送师父过山教不要伤他命哩。”那怪闻说连忙跪下口里呜呜的答应原来被行者揪着鼻子捏儾了就如重伤风一般叫道:“唐老爷若肯饶命即便抬轿相送。”行者道:“我师徒俱是善胜之人依你言且饶你命快抬轿来。如再变卦拿住决不再饶!”那怪得脱手磕头而去。行者同八戒见唐僧备言前事。八戒惭愧不胜在坡前晾晒衣服等候不题。
	
	那二魔战战兢兢回洞未到时已有小妖报知老魔三魔说二魔被行者揪着鼻子拉去。老魔悚惧与三魔帅众方出见二魔独回又皆接入问及放回之故。二魔把三藏慈悯善胜之言对众说了一遍一个个面面相觑更不敢言。二魔道:“哥哥可送唐僧么?”老魔道:“兄弟你说那里话孙行者是个广施仁义的猴头他先在我肚里若肯害我性命一千个也被他弄杀了。却才揪住你鼻子若是扯了去不放回只捏破你的鼻子头儿却也惶恐。快早安排送他去罢。”三魔笑道:“送!送!送!”
	
	老魔道:“贤弟这话却又象尚气的了。你不送我两个送去罢。”三魔又笑道:“二位兄长在上那和尚倘不要我们送只这等瞒过去还是他的造化;若要送不知正中了我的调虎离山之计哩。”老怪道:“何为调虎离山?”三怪道:“如今把满洞群妖点将起来万中选千千中选百百中选十六个又选三十个。”
	
	老怪道:“怎么既要十六又要三十?”三怪道:“要三十个会烹煮的与他些精米、细面、竹笋、茶芽、香蕈、蘑菇、豆腐、面筋着他二十里或三十里搭下窝铺安排茶饭管待唐僧。”老怪道:“又要十六个何用?”三怪道:“着八个抬八个喝路。我弟兄相随左右送他一程。此去向西四百余里就是我的城池我那里自有接应的人马若至城边如此如此着他师徒尾不能相顾。要捉唐僧全在此十六个鬼成功。”老怪闻言欢欣不已真是如醉方醒似梦方觉道:“好!好!好!”即点众妖先选三十与他物件;又选十六抬一顶香藤轿子同出门来又吩咐众妖:“俱不许上山闲走!孙行者是个多心的猴子若见汝等往来他必生疑识破此计。”
	
	老怪遂帅众至大路旁高叫道:“唐老爷今日不犯红沙请老爷早早过山。”三藏闻言道:“悟空是甚人叫我?”行者指定道:“那厢是老孙降伏的妖精抬轿来送你哩。”三藏合掌朝天道:“善哉!善哉!若不是贤徒如此之能我怎生得去?”径直向前对众妖作礼道:“多承列位之爱我弟子取经东回向长安当传扬善果也。”众妖叩道:“请老爷上轿。”那三藏肉眼凡胎不知是计;孙大圣又是太乙金仙忠正之性只以为擒纵之功降了妖怪亦岂期他都有异谋?却也不曾详察尽着师父之意即命八戒将行囊捎在马上与沙僧紧随他使铁棒向前开路顾盼吉凶。八个抬起轿子八个一递一声喝道。三个妖扶着轿扛师父喜喜欢欢的端坐轿上上了高山依大路而行。
	
	此一去岂知欢喜之间愁又至经云泰极否还生时运相逢真太岁又值丧门吊客星。那伙妖魔同心合意的侍卫左右早晚殷勤。行经三十里献斋五十里又斋未晚请歇沿路齐齐整整。一日三餐遂心满意;良宵一宿好处安身。西进有四百里余程忽见城池相近。大圣举铁棒离轿仅有一里之遥见城池把他吓了一跌挣挫不起。你道他只这般大胆如何见此着唬原来望见那城中有许多恶气乃是:攒攒簇簇妖魔怪四门都是狼精灵。斑斓老虎为都管白面雄彪作总兵。丫叉角鹿传文引伶俐狐狸当道行。千尺大蟒围城走万丈长蛇占路程。楼下苍狼呼令使台前花豹作人声。摇旗擂鼓皆妖怪巡更坐铺尽山精。狡兔开门弄买卖野猪挑担干营生。先年原是天朝国如今翻作虎狼城。那大圣正当悚惧只听得耳后风响急回头观看原来是三魔双手举一柄画杆方天戟往大圣头上打来。大圣急翻身爬起使金箍棒劈面相迎。他两个各怀恼怒气呼呼更不打话;咬着牙各要相争。又见那老魔头传声号令举钢刀便砍八戒。八戒慌得丢了马轮着钯向前乱筑。那二魔缠长枪望沙僧刺来沙僧使降妖杖支开架子敌住。三个魔头与三个和尚一个敌一个在那山头舍死忘生苦战。那十六个小妖却遵号令各各效能:抢了白马行囊把三藏一拥抬着轿子径至城边高叫道:“大王爷爷定计已拿得唐僧来了!”那城上大小妖精一个个跑下将城门大开吩咐各营卷旗息鼓不许呐喊筛锣说:“大王原有令在前不许吓了唐僧。唐僧禁不得恐吓一吓就肉酸不中吃了。”众精都欢天喜地邀三藏控背躬身接主僧。把唐僧一轿子抬上金銮殿请他坐在当中一壁厢献茶献饭左右旋绕。那长老昏昏沉沉举眼无亲。毕竟不知性命何如且听下回分解。
	------------
	
	第七十七回 群魔欺本性 一体拜真如
	
	且不言唐长老困苦却说那三个魔头齐心竭力与大圣兄弟三人在城东半山内努力争持。这一场正是那铁刷帚刷铜锅家家挺硬。好杀:六般体相六般兵六样形骸六样情。六恶六根缘六欲六门六道赌输赢。三十六宫春自在六六形色恨有名。这一个金箍棒千般解数;那一个方天戟百样峥嵘。八戒钉钯凶更猛二怪长枪俊又能。小沙僧宝杖非凡有心打死;
	
	老魔头钢刀快利举手无情。这三个是护卫真僧无敌将那三个是乱法欺君泼野精。起初犹可向后弥凶。六枚都使升空法云端里面各翻腾。一时间吐雾喷云天地暗哮哮吼吼只闻声。
	
	他六个斗罢多时渐渐天晚。却又是风雾漫漫霎时间就黑暗了。原来八戒耳大盖着眼皮越昏蒙手脚慢又遮架不住拖着钯败阵就走被老魔举刀砍去几乎伤命幸躲过头脑被口刀削断几根鬃毛赶上张开口咬着领头拿入城中丢与小怪捆在金銮殿。老妖又驾云起在半空助力。沙和尚见事不谐虚幌着宝杖顾本身回头便走被二怪捽开鼻子响一声连手卷住拿到城里也叫小妖捆在殿下却又腾空去叫拿行者。行者见两个兄弟遭擒他自家独力难撑正是好手不敌双拳双拳难敌四手。他喊一声把棍子隔开三个妖魔的兵器纵筋斗驾云走了。三怪见行者驾筋斗时即抖抖身现了本象扇开两翅赶上大圣。你道他怎能赶上?当时如行者闹天宫十万天兵也拿他不住者以他会驾筋斗云一去有十万八千里路所以诸神不能赶上。这妖精搧一翅就有九万里两搧就赶过了所以被他一把挝住拿在手中左右挣挫不得。欲思要走莫能逃脱即使变化法遁法又往来难行:变大些儿他就放松了挝住;变小些儿他又揝紧了挝住。复拿了径回城内放了手捽下尘埃吩咐群妖也照八戒、沙僧捆在一处。那老魔、二魔俱下来迎接。三个魔头同上宝殿。噫!这一番倒不是捆住行者分明是与他送行。
	
	此时有二更时候众怪一齐相见毕把唐僧推下殿来。那长老于灯光前忽见三个徒弟都捆在地下老师父伏于行者身边哭道:“徒弟啊!常时逢难你却在外运用神通到那里取救降魔今番你亦遭擒我贫僧怎么得命!”八戒、沙僧听见师父这般苦楚便也一齐放声痛哭。行者微微笑道:“师父放心兄弟莫哭!凭他怎的决然无伤。等那老魔安静了我们走路。”
	
	八戒道:“哥啊又来捣鬼了!麻绳捆住松些儿还着水喷想你这瘦人儿不觉我这胖的遭瘟哩!不信你看两膊上入肉已有二寸如何脱身?”行者笑道:“莫说是麻绳捆的就是碗粗的棕缆只也当秋风过耳何足罕哉!”师徒们正说处只闻得那老魔道:“三贤弟有力量有智谋果成妙计拿将唐僧来了!”叫:
	
	“小的们着五个打水七个刷锅十个烧火二十个抬出铁笼来把那四个和尚蒸熟我兄弟们受用各散一块儿与小的们吃也教他个个长生。”八戒听见战兢兢的道:“哥哥你听那妖精计较要蒸我们吃哩!”行者道:“不要怕等我看他是维儿妖精是把势妖精。”沙和尚哭道:“哥呀!且不要说宽话如今已与阎王隔壁哩且讲甚么雏儿把势!”说不了又听得二怪说:“猪八戒不好蒸。”八戒欢喜道:“阿弥陀佛是那个积阴骘的说我不好蒸?”三怪道:“不好蒸剥了皮蒸。”八戒慌了厉声喊道:“不要剥皮!粗自粗汤响就烂了!”老怪道:“不好蒸的安在底下一格。”行者笑道:“八戒莫怕是雏儿不是把势。”沙僧道:“怎么认得?”行者道:“大凡蒸东西都从上边起。
	
	不好蒸的安在上头一格多烧把火圆了气就好了;若安在底下一住了气就烧半年也是不得气上的。他说八戒不好蒸安在底下不是雏儿是甚的!”八戒道:“哥啊依你说就活活的弄杀人了!他打紧见不上气抬开了把我翻转过来再烧起火弄得我两边俱熟中间不夹生了?”正讲时又见小妖来报:
	
	“汤滚了。”老怪传令叫抬。众妖一齐上手将八戒抬在底下一格沙僧抬在二格。行者估着来抬他他就脱身道:“此灯光前好做手脚!”拔下一根毫毛吹口仙气叫声“变!”即变做一个行者捆了麻绳将真身出神跳在半空里低头看着。那群妖那知真假见人就抬把个“假行者”抬在上三格;才将唐僧揪翻倒捆住抬上第四格。干柴架起烈火气焰腾腾。大圣在云端里嗟叹道:“我那八戒沙僧还捱得两滚我那师父只消一滚就烂。若不用法救他顷刻丧矣!”好行者在空中捻着诀念一声“唵蓝净法界乾元亨利贞”的咒语拘唤得北海龙王早至。只见那云端里一朵乌云应声高叫道:“北海小龙敖顺叩头。”行者道:“请起!请起!无事不敢相烦今与唐师父到此被毒魔拿住上铁笼蒸哩。你去与我护持护持莫教蒸坏了。”
	
	龙王随即将身变作一阵冷风吹入锅下盘旋围护更没火气烧锅。他三人方不损命。
	
	将有三更尽时只闻得老魔放道:“手下的我等用计劳形拿了唐僧四众又因相送辛苦四昼夜未曾得睡。今已捆在笼里料应难脱汝等用心看守着十个小妖轮流烧火让我们退宫略略安寝。到五更天色将明必然烂了可安排下蒜泥盐醋请我们起来空心受用。”众妖各各遵命三个魔头却各转寝宫而去。行者在云端里明明听着这等吩咐却低下云头不听见笼里人声。他想着:“火气上腾必然也热他们怎么不怕又无言语?哼喷!莫敢是蒸死了?等我近前再听。”好大圣踏着云摇身一变变作一个黑苍蝇儿钉在铁笼格外听时只闻得八戒在里面道:“晦气晦气!不知是闷气蒸又不知是出气蒸哩。”沙僧道:“二哥怎么叫做闷气、出气?”八戒道:“闷气蒸是盖了笼头出气蒸不盖。”三藏在浮上一层应声道:“徒弟不曾盖。”八戒道:“造化!今夜还不得死!这是出气蒸了!”行者听得他三人都说话未曾伤命便就飞了去把个铁笼盖轻轻儿盖上。三藏慌了道:“徒弟!盖上了!”八戒道:“罢了!这个是闷气蒸今夜必是死了!”沙僧与长老嘤嘤的啼哭。八戒道:
	
	“且不要哭这一会烧火的换了班了。”沙僧道:“你怎么知道?”
	
	八戒道:“早先抬上来时正合我意:我有些儿寒湿气的病要他腾腾。这会子反冷气上来了。咦!烧火的长官添上些柴便怎的?要了你的哩!”行者听见忍不住暗笑道:“这个夯货!冷还好捱若热就要伤命。再说两遭一定走了风了快早救他。
	
	且住!要救他须是要现本相。假如现了这十个烧火的看见一齐乱喊惊动老怪却不又费事?等我先送他个法儿。”忽想起:“我当初做大圣时曾在北天门与护国天王猜枚耍子赢得他瞌睡虫儿还有几个送了他罢。”即往腰间顺带里摸摸还有十二个。“送他十个还留两个做种。”即将虫儿抛了去散在十个小妖脸上钻入鼻孔渐渐打盹都睡倒了。只有一个拿火叉的睡不稳揉头搓脸把鼻子左捏右捏不住的打喷嚏。行者道:“这厮晓得勾当了我再与他个双掭灯。”又将一个虫儿抛在他脸上。“两个虫儿左进右出右出左进谅有一个安住。”那小妖两三个大呵欠把腰伸一伸丢了火叉也扑的睡倒再不翻身。
	
	行者道:“这法儿真是妙而且灵!”即现原身走近前叫声“师父。”唐僧听见道:“悟空救我啊!”沙僧道:“哥哥你在外面叫哩?”行者道:“我不在外面好和你们在里边受罪?”八戒道:“哥啊溜撒的溜了我们都是顶缸的在此受闷气哩!”行者笑道:“呆子莫嚷我来救你。”八戒道:“哥啊救便要脱根救莫又要复蒸笼。”行者却揭开笼头解了师父将假变的毫毛抖了一抖收上身来又一层层放了沙僧放了八戒。那呆子才解了巴不得就要跑。行者道:“莫忙!莫忙!”却又念声咒语放了龙神才对八戒道:“我们这去到西天还有高山峻岭师父没脚力难行等我还将马来。!你看他轻手轻脚走到金銮殿下见那些大小群妖俱睡熟了却解了缰绳更不惊动。
	
	那马原是龙马若是生人飞踢两脚便嘶几声行者曾养过马授弼马温之官又是自家一伙所以不跳不叫。悄悄的牵来束紧了肚带扣备停当请师父上马。长老战兢兢的骑上也就要走行者道:“也且莫忙我们西去还有国王须要关文方才去得不然将甚执照?等我还去寻行李来。”唐僧道:“我记得进门时众怪将行李放在金殿左手下担儿也在那一边。”行者道:“我晓得了。”即抽身跳在宝殿寻时忽见光彩飘飖。行者知是行李怎么就知?以唐僧的锦襕袈裟上有夜明珠故此放光。
	
	急到前见担儿原封未动连忙拿下去付与沙僧挑着。八戒牵着马他引了路径奔正阳门。只听得梆铃乱响门上有锁锁上贴了封皮。行者道:“这等防守如何去得?”八戒道:“后门里去罢。”行者引路径奔后门:“后宰门外也有梆铃之声门上也有封锁却怎生是好?我这一番若不为唐僧是个凡体我三人不管怎的也驾云弄风走了。只为唐僧未三界外见在五行中一身都是父母浊骨所以不得升驾难逃。”八戒道:“哥哥不消商量我们到那没梆铃不防卫处撮着师父爬过墙去罢。”
	
	行者笑道:“这个不好。此时无奈撮他过去;到取经回来你这呆子口敞延地里就对人说我们是爬墙头的和尚了。”八戒道:“此时也顾不得行检且逃命去罢。”行者也没奈何只得依他到那净墙边算计爬出。
	
	噫!有这般事!也是三藏灾星未脱。那三个魔头在宫中正睡忽然惊觉。说走了唐僧一个个披衣忙起急登宝殿问曰:“唐僧蒸了几滚了?”那些烧火的小妖已是有睡魔虫都睡着了就是打也莫想打得一个醒来。其余没执事的惊醒几个冒冒失失的答应道:“七……七……七……七滚了!”急跑近锅边只见笼格子乱丢在地下烧火的还都睡着慌得又来报道:
	
	“大王走……走……走……走了!”三个魔头都下殿近锅前仔细看时果见那笼格子乱丢在地下汤锅尽冷火脚俱无那烧火的俱呼呼鼾睡如泥。慌得众怪一齐呐喊都叫:“快拿唐僧!快拿唐僧!”这一片喊声振起把些前前后后、大大小小妖精都惊起来。刀枪簇拥至正阳门下见那封锁不动梆铃不绝问外边巡夜的道:“唐僧从那里走了?”俱道:“不曾走出人来。”急赶至后宰门封锁梆铃一如前门。复乱抢抢的灯笼火把焙天通红就如白日却明明的照见他四众爬墙哩!老魔赶近喝声:“那里走!”那长老唬得脚软筋麻跌下墙来被老魔拿住。二魔捉了沙僧三魔擒倒八戒众妖抢了行李白马只是走了行者。那八戒口里口国口国哝哝的报怨行者道:“天杀的”我说要救便脱根救如今却又复笼蒸了!”众魔把唐僧擒至殿上却不蒸了。二怪吩咐把八戒绑在殿前檐柱上三怪吩咐把沙僧绑在殿后檐柱上惟老魔把唐僧抱住不放。三怪道:“大哥你抱住他怎的?终不然就活吃?却也没些趣味。此物比不得那愚夫俗子拿了可以当饭。此是上邦稀奇之物必须待天阴闲暇之时拿他出来整制精洁猜枚行令细吹细打的吃方可。”
	
	老魔笑道:“贤弟之言虽当但孙行者又要来偷哩。”三魔道:
	
	“我这皇宫里面有一座锦香亭子亭子内有一个铁柜。依着我把唐僧藏在柜里关了亭子却传出谣言说唐僧已被我们夹生吃了。令小妖满城讲说那行者必然来探听消息若听见这话他必死心塌地而去。待三五日不来搅扰却拿出来慢慢受用如何?”老怪二怪俱大喜道:“是是是!兄弟说得有理!”可怜把个唐僧连夜拿将进去藏在柜中闭了亭子。传出谣言满城里都乱讲不题。
	
	却说行者自夜半顾不得唐僧驾云走脱径至狮驼洞里一路棍把那万数小妖尽情剿绝。急回来东方日出到城边不敢叫战正是单丝不线孤掌难鸣。他落下云头摇身一变变作个小妖儿演入门里大街小巷缉访消息。满城里俱道:
	
	“唐僧被大王夹生儿连夜吃了。”前前后后都是这等说。行者着实心焦行至金銮殿前观看那里边有许多精灵都戴着皮金帽子穿着黄布直身手拿着红漆棍腰挂象牙牌一往一来不住的乱走。行者暗想道:“此必是穿宫的妖怪。就变做这个模样进去打听打听。”好大圣果然变得一般无二混入金门。正走处只见八戒绑在殿前柱上哼哩。行者近前叫声“悟能。”那呆子认得声音道:“师兄你来了?救我一救!”行者道:
	
	“我救你你可知师父在那里?”八戒道:“师父没了昨夜被妖精夹生儿吃了。”行者闻言忽失声泪似泉涌。八戒道:“哥哥莫哭我也是听得小妖乱讲未曾眼见。你休误了再去寻问寻问。”这行者却才收泪又往里面找寻。忽见沙僧绑在后檐柱上即近前摸着他胸脯子叫道:“悟净。”沙僧也识得声音道:
	
	“师兄你变化进来了?救我!救我!”行者道:“救你容易你可知师父在那里?”沙僧滴泪道:“哥啊!师父被妖精等不得蒸就夹生儿吃了!”大圣听得两个言语相同心如刀搅泪似水流急纵身望空跳起且不救八戒沙僧回至城东山上按落云头放声大哭叫道:“师父啊!恨我欺天困网罗师来救我脱沉疴。
	
	潜心笃志同参佛努力修身共炼魔。岂料今朝遭蜇害不能保你上婆娑。西方胜境无缘到气散魂消怎奈何!”行者凄凄惨惨的自思自忖以心问心道:“这都是我佛如来坐在那极乐之境没得事干弄了那三藏之经!若果有心劝善理当送上东土却不是个万古流传?只是舍不得送去却教我等来取。怎知道苦历千山今朝到此丧命!罢!罢!罢!老孙且驾个筋斗云去见如来备言前事。若肯把经与我送上东土一则传扬善果二则了我等心愿;若不肯与我教他把松箍儿咒念念退下这个箍子交还与他老孙还归本洞称王道寡耍子儿去罢。”
	
	好大圣急翻身驾起筋斗云径投天竺。那里消一个时辰早望见灵山不远。须臾间按落云头直至鹫峰之下忽抬头见四大金刚挡住道:“那里走?”行者施礼道:“有事要见如来。”
	
	当头又有昆仑山金霞岭不坏尊王永住金刚喝道:“这泼猴甚是粗狂!前者大困牛魔我等为汝努力今日面见全不为礼!有事且待先奏奉召方行。这里比南天门不同教你进去出来两边乱走!咄!还不靠开!”那大圣正是烦恼处又遭此抢白气得哮吼如雷忍不住大呼小叫早惊动如来。如来佛祖正端坐在九品宝莲台上与十八尊轮世的阿罗汉讲经即开口道:“孙悟空来了汝等出去接待接待。”大众阿罗遵佛旨两路幢幡宝盖即出山门应声道:“孙大圣如来有旨相唤哩。”那山门口四大金刚却才闪开路让行者前进。众阿罗引至宝莲台下见如来倒身下拜两泪悲啼。如来道:“悟空有何事这等悲啼?”
	
	行者道:“弟子屡蒙教训之恩托庇在佛爷爷之门下自归正果保护唐僧拜为师范一路上苦不可言!今至狮驼山狮驼洞狮驼城有三个毒魔乃狮王、象王、大鹏把我师父捉将去连弟子一概遭迍都捆在蒸笼里受汤火之灾。幸弟子脱逃唤龙王救免。是夜偷出师等不料灾星难脱复又擒回。及至天明入城打听叵耐那魔十分狠毒万样骁勇把师父连夜夹生吃了如今骨肉无存。又况师弟悟能悟净见绑在那厢不久性命亦皆倾矣。弟子没及奈何特地到此参拜如来。望大慈悲将松箍咒儿念念退下我这头上箍儿交还如来放我弟子回花果山宽闲耍子去罢!”说未了泪如泉涌悲声不绝。如来笑道:
	
	“悟空少得烦恼。那妖精神通广大你胜不得他所以这等心痛。”行者跪在下面捶着胸膛道:“不瞒如来说弟子当年闹天宫称大圣自为人以来不曾吃亏今番却遭这毒魔之手!”如来闻言道:“你且休恨那妖精我认得他。”行者猛然失声道:
	
	“如来!我听见人讲说那妖精与你有亲哩。”如来道:“这个刁猢狲!怎么个妖精与我有亲?”行者笑道:“不与你有亲如何认得?”如来道:“我慧眼观之故此认得。那老怪与二怪有主。”叫阿傩迦叶来:“你两个分头驾云去五台山、峨眉山宣文殊、普贤来见。”二尊者即奉旨而去。如来道:“这是老魔、二怪之主。
	
	但那三怪说将起来也是与我有些亲处。”行者道:“亲是父党?母党?”如来道:“自那混沌分时天开于子地辟于丑人生于寅天地再交合万物尽皆生。万物有走兽飞禽走兽以麒麟为之长飞禽以凤凰为之长。那凤凰又得交合之气育生孔雀、大鹏。孔雀出世之时最恶能吃人四十五里路把人一口吸之。
	
	我在雪山顶上修成丈六金身早被他也把我吸下肚去。我欲从他便门而出恐污真身是我剖开他脊背跨上灵山。欲伤他命当被诸佛劝解伤孔雀如伤我母故此留他在灵山会上封他做佛母孔雀大明王菩萨。大鹏与他是一母所生故此有些亲处。”行者闻言笑道:“如来若这般比论你还是妖精的外甥哩。”如来道:“那怪须是我去方可收得。”行者叩头启上如来:“千万望玉趾一降!”
	
	如来即下莲台同诸佛众径出山门又见阿傩、迦叶引文殊、普贤来见。二菩萨对佛礼拜如来道:“菩萨之兽下山多少时了?”文殊道:“七日了。”如来道:“山中方七日世上几千年。
	
	不知在那厢伤了多少生灵快随我收他去。”二菩萨相随左右同众飞空。只见那:满天缥缈瑞云分我佛慈悲降法门。明示开天生物理细言辟地化身文。面前五百阿罗汉脑后三千揭谛神。迦叶阿傩随左右普文菩萨殄妖氛。大圣有此人情请得佛祖与众前来不多时早望见城池。行者报道:“如来那放黑气的乃是狮驼国也。”如来道:“你先下去到那城中与妖精交战许败不许胜。败上来我自收他。”大圣即按云头径至城上脚踏着垛儿骂道:“泼孽畜!快出来与老孙交战!”慌得那城楼上小妖急跳下城中报道:“大王孙行者在城上叫战哩。”老妖道:“这猴儿两三日不来今朝却又叫战莫不是请了些救兵来耶?”三怪道:“怕他怎的!我们都去看来。”三个魔头各持兵器赶上城来见了行者更不打话举兵器一齐乱刺行者轮铁棒掣手相迎。斗经七八回合行者佯输而走。那妖王喊声大振叫道:“那里走!”大圣筋斗一纵跳上半空三个精即驾云来赶。行者将身一闪藏在佛爷爷金光影里全然不见。只见那过去、未来、见在的三尊佛像与五百阿罗汉、三千揭谛神布散左右把那三个妖王围住水泄不通。老魔慌了手脚叫道:“兄弟不好了!那猴子真是个地里鬼!那里请得个主人公来也!”
	
	三魔道:“大哥休得悚惧我们一齐上前使枪刀搠倒如来夺他那雷音宝刹!”这魔头不识起倒真个举刀上前乱砍却被文殊、普贤念动真言喝道:“这孽畜还不皈正更待怎生!”唬得老怪、二怪不敢撑持丢了兵器打个滚现了本相。二菩萨将莲花台抛在那怪的脊背上飞身跨坐二怪遂泯耳皈依。
	
	二菩萨既收了青狮、白象只有那第三个妖魔不伏腾开翅丢了方天戟扶摇直上轮利爪要刁捉猴王。原来大圣藏在光中他怎敢近?如来情知此意即闪金光把那鹊巢贯顶之头迎风一幌变做鲜红的一块血肉。妖精轮利爪刁他一下被佛爷把手往上一指那妖翅膊上鞦了筋。飞不去只在佛顶上不能远遁现了本相乃是一个大鹏金翅雕即开口对佛应声叫道:“如来你怎么使大法力困住我也?”如来道:“你在此处多生孽障跟我去有进益之功。”妖精道:“你那里持斋把素极贫极苦;我这里吃人肉受用无穷!你若饿坏了我你有罪愆。”如来道:“我管四大部洲无数众生瞻仰凡做好事我教他先祭汝口。”那大鹏欲脱难脱要走怎走?是以没奈何只得皈依。行者方才转出向如来叩头道:“佛爷你今收了妖精除了大害只是没了我师父也。”大鹏咬着牙恨道:“泼猴头!寻这等狠人困我!你那老和尚几曾吃他?如今在那锦香亭铁柜里不是?”行者闻言忙叩头谢了佛祖。佛祖不敢松放了大鹏也只教他在光焰上做个护法引众回云径归宝刹。
	
	行者却按落云头直入城里。那城里一个小妖儿也没有了正是蛇无头而不行鸟无翅而不飞。他见佛祖收了妖王各自逃生而去。行者才解救了八戒、沙僧寻着行李马匹与他二人说:“师父不曾吃都跟我来。”引他两个径入内院找着锦香亭打开门看内有一个铁柜只听得三藏有啼哭之声。沙僧使降妖杖打开铁锁揭开柜盖叫声:“师父!”三藏见了放声大哭道:“徒弟啊!怎生降得妖魔?如何得到此寻着我也?”行者把上项事从头至尾细陈了一遍三藏感谢不尽。师徒们在那宫殿里寻了些米粮安排些茶饭饱吃一餐收拾出城找大路投西而去。正是:真经必得真人取意嚷心劳总是虚。毕竟这一去不知几时得面如来且听下回分解。
	------------
	
	第七十八回 比丘怜子遣阴神 金殿识魔谈道德
	
	一念才生动百魔修持最苦奈他何!但凭洗涤无尘垢也用收拴有琢磨。扫退万缘归寂灭荡除千怪莫蹉跎。管教跳出樊笼套行满飞升上大罗。话说孙大圣用尽心机请如来收了众怪解脱三藏师徒之难离狮驼城西行。又经数月早值冬天但见那岭梅将破玉池水渐成冰。红叶俱飘落青松色更新。淡云飞欲雪枯草伏山平。满目寒光迥阴阴诱骨泠。师徒们冲寒冒冷宿雨餐风正行间又见一座城池。三藏问道:
	
	“悟空那厢又是甚么所在?”行者道:“到跟前自知若是西邸王位须要倒换关文;若是府州县径过。”师徒言语未毕早至城门之外。三藏下马一行四众进了月城见一个老军在向阳墙下偎风而睡。行者近前摇他一下叫声:“长官。”那老军猛然惊觉麻麻糊糊的睁开眼看见行者连忙跪下磕头叫:“爷爷!”行者道:“你休胡惊作怪我又不是甚么恶神你叫爷爷怎的!”老军磕头道:“你是雷公爷爷!”行者道:“胡说!吾乃东土去西天取经的僧人。适才到此不知地名问你一声的。”那老军闻言却才正了心打个呵欠爬起来伸伸腰道:“长老长老恕小人之罪。此处地方原唤比丘国今改作小子城。”行者道:“国中有帝王否?”老军道:“有!有!有!”行者却转身对唐僧道:“师父此处原是比丘国今改小子城。但不知改名之意何故也。”唐僧疑惑道:“既云比丘又何云小子?”八戒道:“想是比丘王崩了新立王位的是个小子故名小子城。”唐僧道:
	
	“无此理!无此理!我们且进去到街坊上再问。”沙僧道:“正是那老军一则不知二则被大哥唬得胡说且入城去询问。”
	
	又入三层门里到通衢大市观看倒也衣冠济楚人物清秀。但见那:酒楼歌馆语声喧彩铺茶房高挂帘。万户千门生意好六街三市广财源。买金贩锦人如蚁夺利争名只为钱。礼貌庄严风景盛河清海晏太平年。师徒四众牵着马挑着担在街市上行彀多时看不尽繁华气概但只见家家门口一个鹅笼。三藏道:“徒弟啊此处人家都将鹅笼放在门何也?”八戒听说左右观之果是鹅笼排列五色彩缎遮幔。呆子笑道:“师父今日想是黄道良辰宜结婚姻会友都行礼哩。”行者道:“胡谈!
	
	那里就家家都行礼!其间必有缘故等我上前看看。”三藏扯住道:“你莫去你嘴脸丑陋怕人怪你。”行者道:“我变化个儿去来。”好大圣捻着诀念声咒语摇身一变变作一个蜜蜂儿展开翅飞近边前钻进幔里观看原来里面坐的是个小孩儿。
	
	再去第二家笼里看也是个小孩儿。连看八九家都是个小孩儿却是男身更无女子。有的坐在笼中顽耍有的坐在里边啼哭有的吃果子有的或睡坐。行者看罢现原身回报唐僧道:
	
	“那笼里是些小孩子大者不满七岁小者只有五岁不知何故。”三藏见说疑思不定。忽转街见一衙门乃金亭馆驿。长老喜道:“徒弟我们且进这驿里去一则问他地方二则撒喂马匹三则天晚投宿。”沙僧道:“正是正是快进去耶。”四众欣然而入。只见那在官人果报与驿丞接入门各各相见。叙坐定驿丞问:“长老自何方来?”三藏言:“贫僧东土大唐差往西天取经者今到贵处有关文理当照验权借高衙一歇。”驿丞即命看茶茶毕即办支应命当直的安排管待。(WWW.mianhuatang.la 好看的小说)三藏称谢又问:“今日可得入朝见驾照验关文?”驿丞道:“今晚不能须待明日早朝。今晚且于敝衙门宽住一宵。”
	
	少顷安排停当驿丞即请四众同吃了斋供又教手下人打归客房安歇。三藏感谢不尽。既坐下长老道:“贫僧有一件不明之事请教烦为指示。贵处养孩儿不知怎生看待。”驿丞道:“天无二日人无二理。养育孩童父精母血怀胎十月待时而生生下乳哺三年渐成体相岂有不知之理!”三藏道:
	
	“据尊言与敝邦无异。但贫僧进城时见街坊人家各设一鹅笼都藏小儿在内。此事不明故敢动问。”驿丞附耳低言道:
	
	“长老莫管他莫问他也莫理他、说他。请安置明早走路。”长老闻言一把扯住驿丞定要问个明白。驿丞摇头摇手只叫:
	
	“谨言!”三藏一不放执死定要问个详细。驿丞无奈只得屏去一应在官人等独在灯光之下悄悄而言道:“适所问鹅笼之事乃是当今国主无道之事。你只管问他怎的!”三藏道:“何为无道?必见教明白我方得放心。”驿丞道:“此国原是比丘国近有民谣改作小子城。三年前有一老人打扮做道人模样携一小女子年方一十六岁其女形容娇俊貌若观音进贡与当今陛下爱其色美宠幸在宫号为美后。近来把三宫娘娘六院妃子全无正眼相觑不分昼夜贪欢不已。如今弄得精神瘦倦身体尪羸饮食少进命在须臾。太医院检尽良方不能疗治。那进女子的道人受我主诰封称为国丈。国丈有海外秘方甚能延寿前者去十洲、三岛采将药来俱已完备。但只是药引子利害:单用着一千一百一十一个小儿的心肝煎汤服药服后有千年不老之功。这些鹅笼里的小儿俱是选就的养在里面。人家父母惧怕王法俱不敢啼哭遂传播谣言叫做小儿城。此非无道而何?长老明早到朝只去倒换关文不得言及此事。”言毕抽身而退。唬得个长老骨软筋麻止不住腮边泪堕忽失声叫道:“昏君昏君!为你贪欢爱美弄出病来怎么屈伤这许多小儿性命!苦哉!苦哉!痛杀我也!”有诗为证诗曰:邪主无知失正真贪欢不省暗伤身。因求永寿戕童命为解天灾杀小民。僧慈悲难割舍官言利害不堪闻。灯前洒泪长吁叹痛倒参禅向佛人。八戒近前道:“师父你是怎的起哩?
	
	专把别人棺材抬在自家家里哭!不要烦恼!常言道君教臣死臣不死不忠;父教子亡子不亡不孝。他伤的是他的子民与你何干!且来宽衣服睡觉莫替古人耽忧。”三藏滴泪道:“徒弟啊你是一个不慈悯的!我出家人积功累行第一要行方便。
	
	怎么这昏君一味胡行!从来也不见吃人心肝可以延寿。这都是无道之事教我怎不伤悲!”沙僧道:“师父且莫伤悲等明早倒换关文觌面与国王讲过。如若不从看他是怎么模样的一个国丈。或恐那国丈是个妖精欲吃人的心肝故设此法未可知也。”行者道:“悟净说得有理。师父你且睡觉明日等老孙同你进朝看国丈的好歹。如若是人只恐他走了旁门不知正道徒以采药为真待老孙将先天之要旨化他皈正;若是妖邪我把他拿住与这国王看看教他宽欲养身断不教他伤了那些孩童性命。”三藏闻言急躬身反对行者施礼道:“徒弟啊此论极妙!极妙!但只是见了昏君不可便问此事恐那昏君不分远近并作谣言见罪却怎生区处?”行者笑道:“老孙自有法力如今先将鹅笼小儿摄离此城教他明日无物取心。地方官自然奏表那昏君必有旨意或与国丈商量或者另行选报。
	
	那时节借此举奏决不致罪坐于我也。”三藏甚喜又道:“如今怎得小儿离城?若果能脱得真贤徒天大之德!可为之略迟缓些恐无及也。”行者抖擞神威即起身吩咐八戒沙僧:
	
	同师父坐着等我施为你看但有阴风刮动就是小儿出城了“他三人一齐俱念:“南无救生药师佛!南无救生药师佛!”
	
	这大圣出得门外打个唿哨起在半空捻了诀念动真言叫声“唵净法界”拘得那城隍、土地、社令、真官并五方揭谛、四值功曹、六丁六甲与护教伽蓝等众都到空中对他施礼道:“大圣夜唤吾等有何急事?”行者道:“今因路过比丘国那国王无道听信妖邪要取小儿心肝做药引子指望长生。我师父十分不忍欲要救生灭怪故老孙特请列位各使神通与我把这城中各街坊人家鹅笼里的小儿连笼都摄出城外山凹中或树林深处收藏一二日与他些果子食用不得饿损;再暗的护持不得使他惊恐啼哭。待我除了邪治了国劝正君王临行时送来还我。”众神听令即便各使神通按下云头满城中阴风滚滚惨雾漫漫:阴风刮暗一天星惨雾遮昏千里月。
	
	起初时还荡荡悠悠;次后来就轰轰烈烈。悠悠荡荡各寻门户救孩童;烈烈轰轰都看鹅笼援骨血。冷气侵人怎出头寒威透体衣如铁。父母徒张皇兄嫂皆悲切。满地卷阴风笼儿被神摄。此夜纵孤恓天明尽欢悦。有诗为证诗曰:释门慈悯古来多正善成功说摩诃。万圣千真皆积德三皈五戒要从和。比丘一国非君乱小子千名是命讹。行者因师同救护这场阴骘胜波罗。当夜有三更时分众神祇把鹅笼摄去各处安藏。
	
	行者按下祥光径至驿庭上只听得他三人还念“南无救生药师佛”哩。他也心中暗喜近前叫:“师父我来也。阴风之起何如?”八戒道:“好阴风!”三藏道:“救儿之事却怎么说?”
	
	行者道:“已一一救他出去待我们起身时送还。”长老谢了又谢方才就寝。
	
	至天晓三藏醒来遂结束齐备道:“悟空我趁早朝倒换关文去也。”行者道:“师父你自家去恐不济事待老孙和你同去看那国丈邪正如何。”三藏道:“你去却不肯行礼恐国王见怪。”行者道:“我不现身暗中跟随你就当保护。”三藏甚喜吩咐八戒沙僧看守行李马匹却才举步这驿丞又来相见。看这长老打扮起来比昨日又甚不同但见他:身上穿一领锦襕异宝佛袈裟头戴金顶毗卢帽。九环锡杖手中拿胸藏一点神光妙。通关文牒紧随身包裹袋中缠锦套。行似阿罗降世间诚如活佛真容貌。那驿丞相见礼毕附耳低言只教莫管闲事三藏点头应声。大圣闪在门旁念个咒语摇身一变变做个蟭蟟虫儿嘤的一声飞在三藏帽儿上出了馆驿径奔朝中。及到朝门外见有黄门官即施礼道:“贫僧乃东土大唐差往西天取经者今到贵地理当倒换关文。意欲见驾伏乞转奏转奏。”
	
	那黄门官果为传奏国王喜道:“远来之僧必有道行。”教请进来。黄门官复奉旨将长老请入。长老阶下朝见毕复请上殿赐坐。长老又谢恩坐了只见那国王相貌尪羸精神倦怠:举手处揖让差池;开言时声音断续。长老将文牒献上那国王眼目昏朦看了又看方才取宝印用了花押递与长老长老收讫。
	
	那国王正要问取经原因只听得当驾官奏道:“国丈爷爷来矣。”那国王即扶着近侍小宦挣下龙床躬身迎接慌得那长老急起身侧立于旁。回头观看原来是一个老道者自玉阶前摇摇摆摆而进。但见他:头上戴一顶淡鹅黄九锡云锦纱巾身上穿一领箸顶梅沉香绵丝鹤氅。腰间系一条纫蓝三股攒绒带足下踏一对麻经葛纬云头履。手中拄一根九节枯藤盘龙拐杖胸前挂一个描龙刺凤团花锦囊。玉面多光润苍髯颔下飘。
	
	金睛飞火焰长目过眉梢。行动云随步逍遥香雾饶。阶下众官都拱接齐呼国丈进王朝。那国丈到宝殿前更不行礼昂昂烈烈径到殿上。国王欠身道:“国丈仙踪今喜早降。”就请左手绣墩上坐。三藏起一步躬身施礼道:“国丈大人贫僧问讯了。”那国丈端然高坐亦不回礼转面向国王道:“僧家何来?”
	
	国王道:“东土唐朝差上西天取经者今来倒验关文。”国丈笑道:“西方之路黑漫漫有甚好处!”三藏道:“自古西方乃极乐之胜境如何不好?”那国王问道:“朕闻上古有云僧是佛家弟子端的不知为僧可能不死向佛可能长生?”三藏闻言急合掌应道:“为僧者万缘都罢;了性者诸法皆空。大智闲闲澹泊在不生之内;真机默默逍遥于寂灭之中。三界空而百端治六根净而千种穷。若乃坚诚知觉须当识心:心净则孤明独照心存则万境皆清。真容无欠亦无余生前可见;幻相有形终有坏分外何求?行功打坐乃为入定之原;布惠施恩诚是修行之本。大巧若拙还知事事无为;善计非筹必须头头放下。但使一心不行万行自全;若云采阴补阳诚为谬语服饵长寿实乃虚词。只要尘尘缘总弃物物色皆空。素素纯纯寡爱欲自然享寿永无穷。”那国丈闻言付之一笑用手指定唐僧道:
	
	“呵!呵!呵!你这和尚满口胡柴!寂灭门中须云认性你不知那性从何而灭!枯坐参禅尽是些盲修瞎炼。俗语云坐坐坐你的屁股破!火熬煎反成祸。更不知我这修仙者骨之坚秀;达道者神之最灵。携箪瓢而入山访友采百药而临世济人。摘仙花以砌笠折香蕙以铺裀。歌之鼓掌舞罢眠云。阐道法扬太上之正教;施符水除人世之妖氛。夺天地之秀气采日月之华精。运阴阳而丹结按水火而胎凝。二八阴消兮若恍若惚;三九阳长兮如杳如冥。应四时而采取药物养九转而修炼丹成。跨青鸾升紫府;骑白鹤上瑶京。参满天之华采表妙道之殷勤。比你那静禅释教寂灭阴神涅槃遗臭壳又不脱凡尘!三教之中无上品古来惟道独称尊!”那国王听说十分欢喜满朝官都喝采道“好个惟道独称尊!惟道独称尊”长老见人都赞他不胜羞愧。国王又叫光禄寺安排素斋待那远来之僧出城西去。三藏谢恩而退才下殿往外正走行者飞下帽顶儿来在耳边叫道:“师父这国丈是个妖邪国王受了妖气。你先去驿中等斋待老孙在这里听他消息。”三藏知会了独出朝门不题。
	
	看那行者一翅飞在金銮殿翡翠屏中钉下只见那班部中闪出五城兵马官奏道:“我主今夜一阵冷风将各坊各家鹅笼里小儿连笼都刮去了更无踪迹。”国王闻奏又惊又恼对国丈道:“此事乃天灭朕也!连月病重御医无效。幸国丈赐仙方专待今日午时开刀取此小儿心肝作引何期被冷风刮去。非天欲灭朕而何?”国丈笑道:“陛下且休烦恼。此儿刮去正是天送长生与陛下也。”国王道:“见把笼中之儿刮去何以返说天送长生?”国丈道:“我才入朝来见了一个绝妙的药引强似那一千一百一十一个小儿之心。那小儿之心只延得陛下千年之寿;此引子吃了我的仙药就可延万万年也。”国王漠然不知是何药引请问再三国丈才说:“那东土差去取经的和尚我观他器宇清净容颜齐整乃是个十世修行的真体。自幼为僧元阳未泄比那小儿更强万倍若得他的心肝煎汤服我的仙药足保万年之寿。”那昏君闻言十分听信对国丈道:“何不早说?若果如此有效适才留住不放他去了。”国丈道:“此何难哉!适才吩咐光禄寺办斋待他他必吃了斋方才出城。如今急传旨将各门紧闭点兵围了金亭馆驿将那和尚拿来必以礼求其心。如果相从即时剖而取出遂御葬其尸还与他立庙享祭;如若不从就与他个武不善作即时捆住剖开取之。有何难事!“那昏君如其言即传旨把各门闭了。又差羽林卫大小官军围住馆驿。行者听得这个消息一翅飞奔馆驿现了本相对唐僧道:“师父祸事了!祸事了!”那三藏才与八戒、沙僧领御斋忽闻此言唬得三尸神散七窍烟生倒在尘埃浑身是汗眼不定睛口不能言。慌得沙僧上前搀住只叫:“师父苏醒!师父苏醒!”八戒道:“有甚祸事?有甚祸事?你慢些儿说便也罢却唬得师父如此!”行者道:“自师父出朝老孙回视那国丈是个妖精。少顷有五城兵马来奏冷风刮去小儿之事。
	
	国王方恼他却转教喜欢道这是天送长生与你要取师父的心肝做药引可延万年之寿。那昏君听信诬言所以点精兵来围馆驿差锦衣官来请师父求心也。”八戒笑道:“行的好慈悯!
	
	救的好小儿!刮的好阴风今番却撞出祸来了!”三藏战兢兢的爬起来扯着行者哀告道:“贤徒啊!此事如何是好?”行者道:
	
	“若要好大做小。”沙僧道:“怎么叫做大做小?”行者道:“若要全命师作徒徒作师方可保全。”三藏道:“你若救得我命情愿与你做徒子徒孙也。”行者道:“既如此不必迟疑。”教:“八戒快和些泥来。”那呆子即使钉钯筑了些土又不敢外面去取水后就掳起衣服撒溺和了一团臊泥递与行者。行者没奈何将泥扑作一片往自家脸上一安做下个猴象的脸子叫唐僧站起休动再莫言语贴在唐僧脸上念动真言吹口仙气叫“变!”那长老即变做个行者模样脱了他的衣服以行者的衣服穿上。行者却将师父的衣服穿了捻着诀念个咒语摇身变作唐僧的嘴脸八戒沙僧也难识认。正当合心装扮停当只听得锣鼓齐鸣又见那枪刀簇拥。原来是羽林卫官领三千兵把馆驿围了。又见一个锦衣官走进驿庭问道:“东土唐朝长老在那里?”慌得那驿丞战兢兢的跪下指道:“在下面客房里。”
	
	锦衣官即至客房里道:“唐长老我王有请。”八戒沙僧左右护持假行者只见假唐僧出门施礼道:“锦衣大人陛下召贫僧有何话说?”锦衣官上前一把扯住道:“我与你进朝去想必有取用也。”咦!这正是:妖诬胜慈善慈善反招凶。毕竟不知此去端的性命何如且听下回分解。
	------------
	
	第七十九回 寻洞擒妖逢老寿 当朝正主救婴儿
	
	却说那锦衣官把假唐僧扯出馆驿与羽林军围围绕绕直至朝门外对黄门官言:“我等已请唐僧到此烦为转奏。”黄门官急进朝依言奏上昏君遂请进去。众官都在阶下跪拜惟假唐僧挺立阶心口中高叫:“比丘王请我贫僧何说?”君王笑道:“朕得一疾缠绵日久不愈。幸国丈赐得一方药饵俱已完备只少一味引子特请长老求些药引。若得病愈与长老修建祠堂四时奉祭永为传国之香火。”假唐僧道:“我乃出家人只身至此不知陛下问国丈要甚东西作引。”昏君道:“特求长老的心肝。”假唐僧道:“不瞒陛下说心便有几个儿不知要的甚么色样。”那国丈在旁指定道:“那和尚要你的黑心。”假唐僧道:“既如此快取刀来。剖开胸腹若有黑心谨当奉命。”那昏君欢喜相谢即着当驾官取一把牛耳短刀递与假僧。假僧接刀在手解开衣服挺起胸膛将左手抹腹右手持刀唿喇的响一声把腹皮剖开那里头就骨都都的滚出一堆心来。唬得文官失色武将身麻。国丈在殿上见了道:“这是个多心的和尚!”假僧将那些心血淋淋的一个个捡开与众观看却都是些红心、白心、黄心、悭贪心、利名心、嫉妒心、计较心、好胜心、望高心、侮慢心、杀害心、狠毒心、恐怖心、谨慎心、邪妄心、无名隐暗之心、种种不善之心更无一个黑心。那昏君唬得呆呆挣挣口不能言战兢兢的教:“收了去!收了去!”那假唐僧忍耐不住收了法现出本相对昏君道:“陛下全无眼力!我和尚家都是一片好心惟你这国丈是个黑心好做药引。你不信等我替你取他的出来看看。”那国丈听见急睁睛仔细观看见那和尚变了面皮不是那般模样。咦!认得当年孙大圣五百年前旧有名。却抽身腾云就起被行者翻筋斗跳在空中喝道:
	
	“那里走!吃吾一棒!”那国丈即使蟠龙拐杖来迎。他两个在半空中这场好杀如意棒蟠龙拐虚空一片云叆叆。原来国丈是妖精故将怪女称娇色。国主贪欢病染身妖邪要把儿童宰。相逢大圣显神通捉怪救人将难解。铁棒当头着实凶拐棍迎来堪喝采。杀得那满天雾气暗城池城里人家都失色。文武多官魂魄飞嫔妃绣女容颜改。唬得那比丘昏主乱身藏战战兢兢没布摆。棒起犹如虎出山拐轮却似龙离海。今番大闹比丘城致令邪正分明白。那妖精与行者苦战二十余合蟠龙拐抵不住金箍棒虚幌了一拐将身化作一道寒光落入皇宫内院把进贡的妖后带出宫门并化寒光不知去向。
	
	大圣按落云头到了宫殿下对多官道:“你们的好国丈啊!”多官一齐礼拜感谢神僧行者道:“且休拜且去看你那昏主何在。”多官道:“我主见争战时惊恐潜藏不知向那座宫中去也。”行者即命:“快寻!莫被美后拐去!”多官听言不分内外同行者先奔美后宫漠然无踪连美后也通不见了。正宫、东宫、西宫、六院概众后妃都来拜谢大圣。大圣道:“且请起不到谢处哩且去寻你主公。”少时见四五个太监搀着那昏君自谨身殿后面而来。众臣俯伏在地齐声启奏道:“主公!主公!感得神僧到此辨明真假。那国丈乃是个妖邪连美后亦不见矣。”国王闻言即请行者出皇宫到宝殿拜谢了道:“长老你早间来的模样那般俊伟这时如何就改了形容?”行者笑道:“不瞒陛下说早间来者是我师父乃唐朝御弟三藏。我是他徒弟孙悟空还有两个师弟猪悟能沙悟净见在金亭馆驿。因知你信了妖言要取我师父心肝做药引是老孙变作师父模样特来此降妖也。”那国王闻说即传旨着阁下太宰快去驿中请师众来朝。
	
	那三藏听见行者现了相在空中降妖吓得魂飞魄散幸有八戒沙僧护持他又脸上戴着一片子臊泥正闷闷不快只听得人叫道:“法师我等乃比丘国王差来的阁下太宰特请入朝谢恩也。”八戒笑道:“师父。莫怕莫怕!这不是又请你取心想是师兄得胜请你酬谢哩。”三藏道:“虽是得胜来请但我这个臊脸怎么见人?”八戒道:“没奈何我们且去见了师兄自有解释。”真个那长老无计只得扶着八戒沙僧挑着担牵着马同去驿庭之上。那太宰见了害怕道:“爷爷呀!这都相似妖头怪脑之类!”沙僧道:“朝士休怪丑陋我等乃是生成的遗体。若我师父来见了我师兄他就俊了。”他三人与众来朝不待宣召直至殿下。行者看见即转身下殿迎着面把师父的泥脸子抓下吹口仙气叫“正!”那唐僧即时复了原身精神愈觉爽利。国王下殿亲迎口称:“法师老佛。”师徒们将马拴住都上殿来相见。行者道:“陛下可知那怪来自何方?等老孙去与你一并擒来剪除后患。”三宫六院诸嫔群妃都在那翡翠屏后听见行者说剪除后患也不避内外男女之嫌一齐出来拜告道:“万望神僧老佛大施法力斩草除根把他剪除尽绝诚为莫大之恩自当重报!”行者忙忙答礼只教国王说他住居。
	
	国王含羞告道:“三年前他到时朕曾问他。他说离城不远只在向南去七十里路有一座柳林坡湾华庄上。国丈年老无儿止后妻生一女年方十六不曾配人愿进与朕。朕因那女貌娉婷遂纳了宠幸在宫。不期得疾太医屡药无功。他说我有仙方止用小儿心煎汤为引。是朕不才轻信其言遂选民间小儿选定今日午时开刀取心。不料神僧下降恰恰又遇笼儿都不见了。他就说神僧十世修真元阳未泄得其心比小儿心更加万倍。一时误犯不知神僧识透妖魔。敢望广施大法剪其后患朕以倾国之资酬谢!”行者笑道:“实不相瞒笼中小儿是我师慈悲着我藏了。你且休题甚么资财相谢待我捉了妖怪是我的功行。”叫:“八戒跟我去来。”八戒道:“谨依兄命。
	
	但只是腹中空虚不好着力。”国王即传旨教:“光禄寺快办斋供。”不一时斋到。八戒尽饱一餐抖擞精神随行者驾云而起。
	
	唬得那国王、妃后并文武多官一个个朝空礼拜都道:“是真仙真佛降临凡也!”那大圣携着八戒径到南方七十里之地住下风云找寻妖处。但只见一股清溪两边夹岸岸上有千千万万的杨柳更不知清华庄在于何处。正是那:万顷野田观不尽千堤烟柳隐无踪。
	
	孙大圣寻觅不着即捻诀念一声“唵”字真言拘出一个当坊土地战兢兢近前跪下叫道:“大圣柳林坡土地叩头。”行者道:“你休怕我不打你。我问你:柳林坡有个清华庄在于何方?”土地道:“此间有个清华洞不曾有个清华庄。小神知道了大圣想是自比丘国来的?”行者道:“正是正是。比丘国王被一个妖精哄了是老孙到那厢识得是妖怪当时战退那怪化一道寒光不知去向。及问比丘王他说三年前进美女时曾问其由怪言居住城南七十里柳林坡清华庄。适寻到此只见林坡不见清华庄是以问你。”土地叩头道:“望大圣恕罪。比丘王亦我地之主也小神理当鉴察奈何妖精神威法大如我泄漏他事就来欺凌故此未获。大圣今来只去那南岸九叉头一颗杨树根下左转三转右转三转用两手齐扑树上连叫三声开门即现清华洞府。”
	
	大圣闻言即令土地回去与八戒跳过溪来寻那颗杨树。
	
	果然有九条叉枝总在一颗根上。行者吩咐八戒:“你且远远的站定待我叫开门寻着那怪赶将出来你却接应。”八戒闻命即离树有半里远近立下。这大圣依土地之言绕树根左转三转右转三转双手齐扑其树叫:“开门!开门!”霎时间一声响喨唿喇喇的门开两扇更不见树的踪迹。那里边光明霞采亦无人烟。行者趁神威撞将进去但见那里好个去处:烟霞幌亮日月偷明。白云常出洞翠藓乱漫庭。一径奇花争艳丽遍阶瑶草斗芳荣。温暖气景常春浑如阆苑不亚蓬瀛。滑凳攀长蔓平桥挂乱藤。蜂衔红蕊来岩窟蝶戏幽兰过石屏。行者急拽步行近前边细看见石屏上有四个大字:“清华仙府”。
	
	他忍不住跳过石屏看处只见那老怪怀中搂着个美女喘嘘嘘的正讲比丘国事齐声叫道:“好机会来!三年事今日得完被那猴头破了!”行者跑近身掣棒高叫道:“我把你这伙毛团甚么好机会!吃吾一棒!”那老怪丢放美人轮起蟠龙拐急架相迎。他两个在洞前这场好杀比前又甚不同:棒举迸金光拐轮凶气。那怪道:“你无知敢进我门来!”行者道:“我有意降邪怪!”那怪道:“我恋国主你无干怎的欺心来展抹?”行者道:“僧修政教本慈悲不忍儿童活见杀。”语去言来各恨仇棒迎拐架当心札。促损琪花为顾生踢破翠苔因把滑。只杀得那洞中霞采欠光明岩上芳菲俱掩压。乒乓惊得鸟难飞吆喝吓得美人散。只存老怪与猴王呼呼卷地狂风刮。看看杀出洞门来又撞悟能呆性。原来八戒在外边听见他们里面嚷闹激得他心痒难挠掣钉钯把一棵九叉杨树刨倒使钯筑了几下筑得那鲜血直冒嘤嘤的似乎有声。他道:“这棵树成了精也!这棵树成了精也!”按在地下又正筑处只见行者引怪出来。那呆子不打话赶上前举钯就筑。那老怪战行者已是难敌见八戒钯来愈觉心慌败了阵将身一幌化道寒光径投东走。他两个决不放松向东赶来。
	
	正当喊杀之际又闻得鸾鹤声鸣祥光缥缈举目视之乃南极老人星也那老人把寒光罩住叫道:“大圣慢来天蓬休赶老道在此施礼哩。”行者即答礼道:“寿星兄弟那里来”?八戒笑道:“肉头老儿罩住寒光必定捉住妖怪了。”寿星陪笑道:“在这里在这里望二公饶他命罢。”行者道:“老怪不与老弟相干为何来说人情?”寿星笑道:“他是我的一副脚力不意走将来成此妖怪。”行者道:“既是老弟之物只教他现出本相来看看。”寿星闻言即把寒光放出喝道:“孽畜!快现本相饶你死罪!”那怪打个转身原来是只白鹿寿星拿起拐杖道:“这孽畜!连我的拐棒也偷来也!”那只鹿俯伏在地口不能言只管叩头滴泪。但见他:一身如玉简斑斑两角参差七汊湾。几度饥时寻药圃有朝渴处饮云潺。年深学得飞腾法日久修成变化颜。今见主人呼唤处现身珉耳伏尘寰。寿星谢了行者就跨鹿而行被行者一把扯住道:“老弟且慢走还有两件事未完哩。”寿星道:“还有甚么未完之事?”行者道:“还有美人未获不知是个甚么怪物;还又要同到比丘城见那昏君现相回旨也。”寿星道:“既这等说我且宁耐。你与天蓬下洞擒捉那美人来同去现相可也。”行者道:“老弟略等等儿我们去了就来。”那八戒抖擞精神随行者径入清华仙府呐声喊叫:“拿妖精!拿妖精!”那美人战战兢兢正自难逃又听得喊声大振即转石屏之内又没个后门出头被八戒喝声:“那里走!我把你这个哄汉子的臊精!看钯”!那美人手中又无兵器不能迎敌将身一闪化道寒光往外就走被大圣抵住寒光乒乓一棒那怪立不住脚倒在尘埃现了本相原来是一个白面狐狸。呆子忍不住手举钯照头一筑可怜把那个倾城倾国千般笑化作毛团狐狸形!行者叫道:“莫打烂他且留他此身去见昏君。”
	
	那呆子不嫌秽污一把揪住尾子拖拖扯扯跟随行者出得门来。只见那寿星老儿手摸着鹿头骂道:“好孽畜啊!你怎么背主逃去在此成精!若不是我来孙大圣定打死你了。”行者跳出来道:“老弟说甚么?”寿星道:“我嘱鹿哩!我嘱鹿哩!”八戒将个死狐狸掼在鹿的面前道:“这可是你的女儿么?”那鹿点头幌脑伸着嘴闻他几闻呦呦声似有眷恋不舍之意被寿星劈头扑了一掌道:“孽畜!你得命足矣又闻他怎的?”即解下勒袍腰带把鹿扣住颈项牵将起来道:“大圣我和你比丘国相见去也。”行者道:“且住!索性把这边都扫个干净庶免他年复生妖孽。”八戒闻言举钯将柳树乱筑。行者又念声“唵”字真言依然拘出当坊土地叫:“寻些枯柴点起烈火与你这方消除妖患以免欺凌。”那土地即转身阴风飒飒帅起阴兵搬取了些迎霜草、秋青草、蓼节草、山蕊草、篓蒿柴、龙骨柴、芦荻柴都是隔年干透的枯焦之物见火如同油腻一般。行者叫:
	
	“八戒不必筑树但得此物填塞洞里放起火来烧得个干净。”火一起果然把一座清华妖怪宅烧作火池坑。
	
	这里才喝退土地同寿星牵着鹿拖着狐狸一齐回到殿前对国王道:“这是你的美后与他耍子儿么?”那国王胆战心惊。又只见孙大圣引着寿星牵着白鹿都到殿前唬得那国里君臣妃后一齐下拜。行者近前搀住国王笑道:“且休拜我这鹿儿却是国丈你只拜他便是。”那国王羞愧无地只道:“感谢神僧救我一国小儿真天恩也!”即传旨教光禄寺安排素宴大开东阁请南极老人与唐僧四众共坐谢恩。三藏拜见了寿星沙僧亦以礼见都问道:“白鹿既是老寿星之物如何得到此间为害?”寿星笑道:“前者东华帝君过我荒山我留坐着棋一局未终这孽畜走了。及客去寻他不见我因屈指询算知他走在此处特来寻他正遇着孙大圣施威。若果来迟此畜休矣。”
	
	叙不了只见报道:“宴已完备。”好素宴:五彩盈门异香满座。
	
	桌挂绣纬生锦艳地铺红毯幌霞光。宝鸭内沉檀香袅;御筵前蔬品香馨。看盘高果砌楼台龙缠斗糖摆走兽。鸳鸯锭狮仙糖似模似样;鹦鹉杯鹭鹚杓如相如形。席前果品般般盛案上斋肴件件精。魁圆茧栗鲜荔桃子。枣儿柿饼味甘甜松子葡萄香腻酒。几般蜜食数品蒸酥。油札糖浇花团锦砌。金盘高垒大馍馍银碗满盛香稻饭。辣煼煼汤水粉条长香喷喷相连添换美。说不尽蘑菇、木耳、嫩笋、黄精十香素菜百味珍馐。往来绰摸不曾停进退诸般皆盛设。当时叙了坐次寿星席长老次席国王前席行者、八戒、沙僧侧席旁又有两三个太师相陪左右。即命教坊司动乐国王擎着紫霞杯一一奉酒惟唐僧不饮。八戒向行者道:“师兄果子让你汤饭等须请让我受用受用。”那呆子不分好歹一齐乱上但来的吃个精空。一席筵宴已毕寿星告辞。那国王又近前跪拜寿星求祛病延年之法寿星笑道:“我因寻鹿未带丹药。欲传你修养之方你又筋衰神败不能还丹。我这衣袖中只有三个枣儿是与东华帝君献茶的我未曾吃今送你罢。”国王吞之渐觉身轻病退。后得长生者皆原于此。八戒看见就叫道:“老寿有火枣送我几个吃吃。”寿星道:“未曾带得待改日我送你几斤。”遂出了东阁道了谢意将白鹿一声喝起飞跨背上踏云而去。这朝中君王妃后城中黎庶居民各各焚香礼拜不题。
	
	三藏叫:“徒弟收拾辞王。”那国王又苦留求教行者道:
	
	“陛下从此色欲少贪阴功多积。凡百事将长补短自足以祛病延年就是教也。”遂拿出两盘散金碎银奉为路费。唐僧坚辞分文不受。国王无已命摆銮驾请唐僧端坐凤辇龙车王与嫔后俱推轮转毂方送出朝。六街三市百姓群黎亦皆盏添净水炉降真香又送出城。忽听得半空中一声风响路两边落下一千一百一十一个鹅笼内有小儿啼哭暗中有原护的城隍、土地、社令、真官、五方揭谛、四值功曹、六丁六甲、护教伽蓝等众应声高叫道:“大圣我等前蒙吩咐摄去小儿鹅笼今知大圣功成起行一一送来也。”那国王妃后与一应臣民又俱下拜。行者望空道:“有劳列位请各归祠我着民间祭祀谢你。”呼呼淅淅阴风又起而退。行者叫城里人家来认领小儿。
	
	当时传播俱来各认出笼中之儿欢欢喜喜抱出叫哥哥叫肉儿跳的跳笑的笑都叫:“扯住唐朝爷爷到我家奉谢救儿之恩!”无大无小若男若女都不怕他相貌之丑抬着猪八戒扛着沙和尚顶着孙大圣撮着唐三藏牵着马挑着担一拥回城那国王也不能禁止。这家也开宴那家也设席。请不及的或做僧帽、僧鞋、褊衫、布袜里里外外大小衣裳都来相送。
	
	如此盘桓将有个月才得离城。又有传下影神立起牌位顶礼焚香供养。这才是:阴功高垒恩山重救活千千万万人。毕竟不知向后又有甚么事体且听下回分解。
	------------
	
	第八十回 姹女育阳求配偶 心猿护主识妖邪
	
	却说比丘国君臣黎庶送唐僧四众出城有二十里之远还不肯舍。(wwW.mianhuatang.la 无弹窗广告)三藏勉强下辇乘马辞别而行目送者直至望不见踪影方回。四众行彀多时又过了冬残春尽看不了野花山树景物芳菲前面又见一座高山峻岭。三藏心惊问道:“徒弟前面高山有路无路是必小心!”行者笑道:“师父这话也不象个走长路的却似个公子王孙坐井观天之类。自古道:山不碍路路自通山。何以言有路无路?”三藏道:“虽然是山不碍路但恐险峻之间生怪物密林深处出妖精。”八戒道:“放心放心!这里来相近极乐不远管取太平无事!”师徒正说不觉的到了山脚下。行者取出金箍棒走上石崖叫道:“师父此间乃转山的路儿忒好步快来快来!”长老只得放怀策马。沙僧教:
	
	“二哥你把担子挑一肩儿。”真个八戒接了担子挑上。沙僧拢着缰绳老师父稳坐雕鞍随行者都奔山崖上大路。但见那山:
	
	云雾笼峰顶潺湲涌涧中。百花香满路万树密丛丛。梅青李白柳绿桃红。杜鹃啼处春将暮紫燕呢喃社已终。峨峨石翠盖松。崎岖岭道突兀玲珑。削壁悬崖峻藤萝草木秾。千岩竞秀如排戟万壑争流远浪洪。老师父缓观山景忽闻啼鸟之声又起思乡之念。兜马叫道:“徒弟!我自天牌传旨意锦屏风下领关文。观灯十五离东土才与唐王天地分甫能龙虎风云会却又师徒拗马军。行尽巫山峰十二何时对子见当今?”
	
	行者道:“师父你常以思乡为念全不似个出家人。放心且走莫要多忧古人云欲求生富贵须下死工夫。”三藏道:“徒弟虽然说得有理但不知西天路还在那里哩!”八戒道:“师父我佛如来舍不得那三藏经知我们要取去想是搬了;不然如何只管不到?”沙僧道:“莫胡谈!只管跟着大哥走只把工夫捱他终须有个到之之日。”
	
	师徒正自闲叙又见一派黑松大林。唐僧害怕又叫道:
	
	“悟空我们才过了那崎岖山路怎么又遇这个深黑松林?是必在意。”行者道:“怕他怎的!”三藏道:“说那里话!不信直中直须防仁不仁。我也与你走过好几处松林不似这林深远。你看:
	
	东西密摆南北成行。东西密摆彻云霄南北成行侵碧汉。密查荆棘周围结蓼却缠枝上下盘。藤来缠葛葛去缠藤。藤来缠葛东西客旅难行;葛去缠藤南北经商怎进。这林中住半年那分日月;行数里不见斗星。你看那背阴之处千般景向阳之所万丛花。又有那千年槐万载桧耐寒松山桃果、野芍药旱芙蓉一攒攒密砌重堆乱纷纷神仙难画。又听得百鸟声:鹦鹉哨杜鹃啼喜鹊穿枝乌鸦反哺黄鹂飞舞百舌调音鹧鸪鸣紫燕语八哥儿学人说话画眉郎也会看经。又见那大虫摆尾老虎磕牙多年狐狢妆娘子日久苍狼吼振林。就是托塔天王来到此纵会降妖也失魂!”孙大圣公然不惧使铁棒上前臂开大路引唐僧径入深林逍逍遥遥行经半日未见出林之路。唐僧叫道:“徒弟一向西来无数的山林崎险幸得此间清雅一路太平。这林中奇花异卉其实可人情意!我要在此坐坐:一则歇马二则腹中饥了你去那里化些斋来我吃。”行者道:“师父请下马老孙化斋去来。”那长老果然下了马。八戒将马拴在树上沙僧歇下行李取了钵盂递与行者。
	
	行者道:“师父稳坐莫要惊怕我去了就来。”三藏端坐松阴之下八戒沙僧却去寻风觅果闲耍。
	
	却说大圣纵筋斗到了半空伫定云光回头观看只见松林中祥云缥缈瑞霭氤氲他忽失声叫道:“好啊!好啊!”你道他叫好做甚?原来夸奖唐僧说他是金蝉长老转世十世修行的好人所以有此祥瑞罩头。“若我老孙方五百年前大闹天宫之时云游海角放荡天涯聚群精自称齐天大圣降龙伏虎消了死籍;头戴着三额金冠身穿着黄金铠甲手执着金箍棒足踏着步云履手下有四万七千群怪都称我做大圣爷爷着实为人。如今脱却天灾。做小伏低与你做了徒弟想师父头顶上有祥云瑞霭罩定径回东土必定有些好处老孙也必定得个正果。”正自家这等夸念中间忽然见林南下有一股子黑气骨都都的冒将上来。行者大惊道:“那黑气里必定有邪了!
	
	我那八戒沙僧却不会放甚黑气。”那大圣在半空中详察不定。
	
	却说三藏坐在林中明心见性讽念那《摩诃般若波罗密多心经》忽听得嘤嘤的叫声“救人”。三藏大惊道:“善哉!善哉!这等深林里有甚么人叫?想是狼虫虎豹唬倒的待我看看。”那长老起身挪步穿过千年柏隔起万年松附葛攀藤近前视之只见那大树上绑着一个女子上半截使葛藤绑在树上下半截埋在土里。长老立定脚问他一句道:“女菩萨你有甚事绑在此间?”咦!分明这厮是个妖怪长老肉眼凡胎却不能认得。那怪见他来问泪如泉涌。你看他桃腮垂泪有沉鱼落雁之容;星眼含悲有闭月羞花之貌。长老实不敢近前又开口问道:“女菩萨你端的有何罪过?说与贫僧却好救你。”那妖精巧语花言虚情假意忙忙的答应道:“师父我家住在贫婆国。离此有二百余里。父母在堂十分好善一生的和亲爱友。时遇清明邀请诸亲及本家老小拜扫先茔一行轿马都到了荒郊野外。至茔前摆开祭礼刚烧化纸马只闻得锣鸣鼓响跑出一伙强人持刀弄杖喊杀前来慌得我们魂飞魄散。
	
	父母诸亲得马得轿的各自逃了性命;奴奴年幼跑不动唬倒在地被众强人拐来山内大大王要做夫人二大王要做妻室第三第四个都爱我美色七八十家一齐争吵大家都不忿气所以把奴奴绑在林间众强人散盘而去。今已五日五夜看看命尽不久身亡!不知是那世里祖宗积德今日遇着老师父到此。千万大慈悲救我一命九泉之下决不忘恩!”说罢泪下如雨。三藏真个慈心也就忍不住吊下泪来声音哽咽叫道:“徒弟”。那八戒沙僧正在林中寻花觅果猛听得师父叫得凄怆呆子道:“沙和尚师父在此认了亲耶。”沙僧笑道:“二哥胡缠!我们走了这些时好人也不曾撞见一个亲从何来?”八戒道:“不是亲师父那里与人哭么?我和你去看来。”沙僧真个回转旧处牵了马挑了担至跟前叫:“师父怎么说?”唐僧用手指定那树上叫:“八戒解下那女菩萨来救他一命。”呆子不分好歹就去动手。
	
	却说那大圣在半空中又见那黑气浓厚把祥光尽情盖了道声:“不好不好!黑气罩暗祥光怕不是妖邪害俺师父!
	
	化斋还是小事且去看我师父去。”即返云头按落林里只见八戒乱解绳儿。行者上前一把揪住耳朵扑的捽了一跌。呆子抬头看见爬起来说道:“师父教我救人你怎么恃你有力将我掼这一跌!”行者笑道:“兄弟莫解他。他是个妖怪弄喧儿骗我们哩。”三藏喝道:“你这泼猴又来胡说了!怎么这等一个女子就认得他是个妖怪!”行者道:“师父原来不知。这都是老孙干过的买卖想人肉吃的法儿你那里认得!”八戒唝着嘴道:“师父莫信这弼马温哄你!这女子乃是此间人家。(WWW.mianhuatang.la 好看的小说)我们东土远来不与相较又不是亲眷如何说他是妖精!他打我们丢了前去他却翻筋斗弄神法转来和他干巧事儿倒踏门也!”行者喝道:“夯货!莫乱谈!我老孙一向西来那里有甚惫懒处?似你这个重色轻生见利忘义的馕糟不识好歹替人家哄了招女婿绑在树上哩!”三藏道:“也罢也罢。八戒啊你师兄常时也看得不差。既这等说不要管他我们去罢。”行者大喜道:“好了!师父是有命的了!请上马出松林外有人家化斋你吃。”四人果一路前进把那怪撇了。
	
	却说那怪绑在树上咬牙恨齿道:“几年家闻人说孙悟空神通广大今日见他果然话不虚传。那唐僧乃童身修行一点元阳未泄正欲拿他去配合成太乙金仙不知被此猴识破吾法将他救去了。若是解了绳放我下来随手捉将去却不是我的人儿也?今被他一篇散言碎语带去却又不是劳而无功?
	
	等我再叫他两声看是如何。”好妖精不动绳索把几声善言善语用一阵顺风嘤嘤的吹在唐僧耳内。你道叫的甚么?他叫道:“师父啊你放着活人的性命还不救昧心拜佛取何经?”
	
	唐僧在马上听得又这般叫唤即勒马叫:“悟空去救那女子下来罢。”行者道:“师父走路怎么又想起他来了?”唐僧道:“他又在那里叫哩。”行者问:“八戒你听见么?”八戒道:“耳大遮住了不曾听见。”又问:“沙僧你听见么?”沙僧道:“我挑担前走不曾在心也不曾听见。”行者道:“老孙也不曾听见。师父他叫甚么?偏你听见。”唐僧道:“他叫得有理说道活人性命还不救昧心拜佛取何经?救人一命胜造七级浮屠。快去救他下来强似取经拜佛。”行者笑道:“师父要善将起来就没药医。你想你离了东土一路西来却也过了几重山场遇着许多妖怪常把你拿将进洞老孙来救你使铁棒常打死千千万万;今日一个妖精的性命舍不得要去救他?”唐僧道:“徒弟呀古人云勿以善小而不为勿以恶小而为之还去救他救罢。”行者道:“师父既然如此只是这个担儿老孙却担不起。
	
	你要救他我也不敢苦劝你劝一会你又恼了。任你去救。”唐僧道:“猴头莫多话!你坐着等我和八戒救他去。”
	
	唐僧回至林里教八戒解了上半截绳子用钯筑出下半截身子。那怪跌跌鞋束束裙喜孜孜跟着唐僧出松林见了行者行者只是冷笑不止。唐僧骂道:“泼猴头!你笑怎的?”行者道:“我笑你时来逢好友运去遇佳人。”三藏又骂道:“泼猢狲!
	
	胡说!我自出娘肚皮就做和尚。如今奉旨西来虔心礼佛求经又不是利禄之辈有甚运退时!”行者笑道:“师父你虽是自幼为僧却只会看经念佛不曾见王法条律。这女子生得年少标致我和你乃出家人同他一路行走倘或遇着歹人把我们拿送官司不论甚么取经拜佛且都打做奸情;纵无此事也要问个拐带人口。师父追了度牒打个小死;八戒该问充军;沙僧也问摆站;我老孙也不得干净饶我口能怎么折辩也要问个不应。”三藏喝道:“莫胡说!终不然我救他性命有甚贻累不成!带了他去凡有事都在我身上。”行者道:“师父虽说有事在你却小知你不是救他反是害他。”三藏道:“我救他出林得其活命怎么反是害他?”行者道:“他当时绑在林间或三五日十日半月没饭吃饿死了还得个完全身体归阴;如今带他出来你坐得是个快马行路如风我们只得随你那女子脚小挪步艰难怎么跟得上走?一时把他丢下若遇着狼虫虎豹一口吞之却不是反害其生也?”三藏道:“正是呀这件事却亏你想如何处置?”行者笑道:“抱他上来和你同骑着马走罢。”三藏沉吟道:“我那里好与他同马!……他怎生得去?”三藏道:“教八戒驮他走罢。”行者笑道:“呆子造化到了!”八戒道:“远路没轻担教我驮人有甚造化?”行者道:“你那嘴长驮着他转过嘴来计较私情话儿却不便益?”八戒闻此言捶胸爆跳道:“不好!不好!师父要打我几下宁可忍疼背着他决不得干净师兄一生会赃埋人。我驮不成!”三藏道:“也罢也罢。我也还走得几步等我下来慢慢的同走着八戒牵着空马罢。”行者大笑道:“呆子倒有买卖师父照顾你牵马哩。”三藏道:“这猴头又胡说了!古人云马行千里无人不能自往。假如我在路上慢走你好丢了我去?我若慢你们也慢。大家一处同这女菩萨走下山去或到庵观寺院有人家之处留他在那里也是我们救他一场。”行者道:“师父说得有理快请前进。”三藏拽步前走沙僧挑担八戒牵着空马行者拿着棒引着女子一行前进。不上二三十里天色将晚又见一座楼台殿阁。三藏道:“徒弟那里必定是座庵观寺院就此借宿了明日早行。”行者道:“师父说得是各各走动些。”霎时到了门。吩咐道:“你们略站远些等我先去借宿。若有方便处着人来叫你。”众人俱立在柳阴之下惟行者拿铁棒辖着那女子。
	
	长老拽步近前只见那门东倒西歪零零落落。推开看时忍不住心中凄惨:长廊寂静古刹萧疏;苔藓盈庭蒿蓁满径;
	
	惟萤火之飞灯只蛙声而代漏。长老忽然吊下泪来真个是:殿宇凋零倒塌廊房寂寞倾颓。断砖破瓦十余堆尽是些歪梁折柱。前后尽生青草尘埋朽烂香厨。钟楼崩坏鼓无皮琉璃香灯破损。佛祖金身没色罗汉倒卧东西。观音淋坏尽成泥杨柳净瓶坠地。日内并无僧入夜间尽宿狐狸只听风响吼如雷都是虎豹藏身之处。四下墙垣皆倒亦无门扇关居。有诗为证诗曰:多年古刹没人修狼狈凋零倒更休。猛风吹裂伽蓝面大雨浇残佛象头。金刚跌损随淋洒土地无房夜不收。更有两般堪叹处铜钟着地没悬楼。三藏硬着胆走进二层门见那钟鼓楼俱倒了止有一口铜钟札在地下。上半截如雪之白下半截如靛之青原来是日久年深上边被雨淋白下边是土气上的铜青。三藏用手摸着钟高叫道:“钟啊!你也曾悬挂高楼吼也曾鸣远彩梁声。也曾鸡啼就报晓也曾天晚送黄昏。不知化铜的道人归何处铸铜匠作那边存。想他二命归阴府他无踪迹你无声。”长老高声赞叹不觉的惊动寺里之人。那里边有一个侍奉香火的道人他听见人语扒起来拾一块断砖照钟上打将去。那钟当的响了一声把个长老唬了一跌挣起身要走又绊着树根扑的又是一跌。长老倒在地下抬头又叫道:“钟啊!贫僧正然感叹你忽的叮当响一声。想是西天路上无人到日久多年变作精。”那道人赶上前一把搀住道:“老爷请起。不干钟成精之事却才是我打得钟响。”三藏抬头见他的模样丑黑道:“你莫是魍魉妖邪?我不是寻常之人我是大唐来的我手下有降龙伏虎的徒弟。你若撞着他性命难存也!”道人跪下道:“老爷休怕我不是妖邪我是这寺里侍奉香火的道人。却才听见老爷善言相赞就欲出来迎接;恐怕是个邪鬼敲门故此拾一块断砖把钟打一下压惊方敢出来。老爷请起。”那唐僧方然正性道:“住持险些儿唬杀我也你带我进去。”
	
	那道人引定唐僧直至三层门里看处比外边甚是不同但见那:青砖砌就彩云墙绿瓦盖成琉璃殿。黄金装圣象白玉造阶台。大雄殿上舞青光毗罗阁下生锐气。文殊殿结采飞云:轮藏堂描花堆翠。三檐顶上宝瓶尖五福楼中平绣盖。千株翠竹摇禅榻万种青松映佛门。碧云宫里放金光紫雾丛中飘瑞霭。朝闻四野香风远暮听山高画鼓鸣。应有朝阳补破衲岂无对月了残经?又只见半壁灯光明后院一行香雾照中庭。
	
	三藏见了不敢进去叫:“道人你这前边十分狼狈后边这等齐整何也?”道人笑道:“老爷这山中多有妖邪强寇天色清明沿山打劫天阴就来寺里藏身被他把佛象推倒垫坐木植搬来烧火。本寺僧人软弱不敢与他讲论因此把这前边破房都舍与那些强人安歇从新另化了些施主盖得那一所寺院。
	
	清混各一这是西方的事情。”三藏道:“原来是如此。
	
	正行间又见山门上有五个大字乃镇海禅林寺。才举步跨入门里忽见一个和尚走来。你看他怎生模样:头戴左笄绒锦帽一对铜圈坠耳根。身着颇罗毛线服一双白眼亮如银。手中摇着播郎鼓口念番经听不真。三藏原来不认得这是西方路上喇嘛僧。那喇嘛和尚走出门来看见三藏眉清目秀额阔顶平耳垂肩手过膝好似罗汉临凡十分俊雅。他走上前扯住满面笑唏唏的与他捻手捻脚摸他鼻子揪他耳朵以示亲近之意。携至方丈中行礼毕却问:“老师父何来?”三藏道:“弟子乃东土大唐驾下钦差往西方天竺国大雷音寺拜佛取经者。
	
	适行至宝方天晚特奔上刹借宿一宵明日早行望垂方便一二。”那和尚笑道:“不当人子!不当人子!我们不是好意要出家的皆因父母生身命犯华盖家里养不住才舍断了出家既做了佛门弟子切莫说脱空之话。”三藏道:“我是老实话。”
	
	和尚道:“那东土到西天有多少路程!路上有山山中有洞洞内有精。象你这个单身又生得娇嫩那里象个取经的!”三藏道:“院主也见得是贫僧一人岂能到此?我有三个徒弟逢山开路遇水叠桥保我弟子所以到得上刹。”那和尚道:“三位高徒何在?”三藏道:“现在山门外伺候。”那和尚慌了道:“师父你不知我这里有虎狼、妖贼、鬼怪伤人。白日里不敢远出未经天晚就关了门户。这早晚把人放在外边!”叫:“徒弟快去请将进来。”
	
	有两个小喇嘛儿跑出外去看见行者唬了一跌见了八戒又是一跌扒起来往后飞跑道:“爷爷!造化低了!你的徒弟不见只有三四个妖怪站在那门也。”三藏问道:“怎么模样?”
	
	小和尚道:“一个雷公嘴一个碓挺嘴一个青脸獠牙。旁有一个女子倒是个油头粉面。”三藏笑道:“你不认得。那三个丑的是我徒弟那一个女子是我打松林里救命来的。”那喇嘛道:“爷爷呀这们好俊师父怎么寻这般丑徒弟?”三藏道:“他丑自丑却俱有用。你快请他进来若再迟了些儿那雷公嘴的有些闯祸不是个人生父母养的他就打进来也。”那小和尚即忙跑出战兢兢的跪下道:“列位老爷唐老爷请哩。”八戒笑道:“哥啊他请便罢了却这般战兢兢的何也?”行者道:“看见我们丑陋害怕。”八戒道:“可是扯淡!我们乃生成的那个是好要丑哩!”行者道:“把那丑且略收拾收拾!呆子真个把嘴揣在怀里低着头牵着马沙僧挑着担行者在后面拿着棒辖着那女子一行进去。穿过了倒塌房廊入三层门里。拴了马歇了担进方丈中与喇嘛僧相见分了坐次。那和尚入里边引出七八十个小喇嘛来见礼毕收拾办斋管待。正是:积功须在慈悲念佛法兴时僧赞僧。毕竟不知怎生离寺且听下回分解。
	------------
	
	第八十一回 镇海寺心猿知怪 黑松林三众寻师
	
	话表三藏师徒到镇海禅林寺众僧相见安排斋供。四众食毕那女子也得些食力。渐渐天昏方丈里点起灯来众僧一则是问唐僧取经来历二则是贪看那女子都攒攒簇簇排列灯下。三藏对那初见的喇嘛僧道:“院主明日离了宝山西去的路途如何?”那僧双膝跪下慌得长老一把扯住道:“院主请起我问你个路程你为何行礼?”那僧道:“老师父明日西行路途平正不须费心。只是眼下有件事儿不尴魀一进门就要说恐怕冒犯洪威却才斋罢方敢大胆奉告:老师东来路遥辛苦都在小和尚房中安歇甚好;只是这位女菩萨不方便不知请他那里睡好。”三藏道:“院主你不要生疑说我师徒们有甚邪意。早间打黑松林过撞见这个女子绑在树上。小徒孙悟空不肯救他是我菩提心将他救了到此随院主送他那里睡去。”那僧谢道:“既老师宽厚请他到天王殿里就在天王爷爷身后安排个草铺教他睡罢。”三藏道:“甚好甚好。”遂此时众小和尚引那女子往殿后睡去。长老就在方丈中请众院主自在遂各散去。三藏吩咐悟空:“辛苦了早睡早起!”遂一处都睡了不敢离侧护着师父。渐入夜深正是那:玉兔高升万籁宁天街寂静断人行。银河耿耿星光灿鼓谯楼趱换更。
	
	一宵晚话不题。及天明了行者起来教八戒沙僧收拾行囊马匹却请师父走路。此时长老还贪睡未醒行者近前叫声“师父。”那师父把头抬了一抬又不曾答应得出。行者问:“师父怎么说?”长老呻吟道:“我怎么这般头悬眼胀浑身皮骨皆疼?”八戒听说伸手去摸摸身上有些热。呆子笑道:“我晓得了这是昨晚见没钱的饭多吃了几碗倒沁着头睡伤食了。”行者喝道:“胡说!等我问师父端的何如。”三藏道:“我半夜之间起来解手不曾戴得帽子想是风吹了。”行者道:“这还说得是如今可走得路么?”三藏道:“我如今起坐不得怎么上马?但只误了路啊!”行者道:“师父说那里话!常言道一日为师终身为父。我等与你做徒弟就是儿子一般。又说道养儿不用阿金溺银只是见景生情便好。你既身子不快说甚么误了行程便宁耐几日何妨!”兄弟们都伏侍着师父不觉的早尽午来昏又至良宵才过又侵晨。
	
	光阴迅早过了三日。那一日师父欠身起来叫道:“悟空这两日病体沉疴不曾问得你那个脱命的女菩萨可曾有人送些饭与他吃?”行者笑道:“你管他怎的且顾了自家的病着。”三藏道:“正是正是。你且扶我起来取出我的纸、笔、墨寺里借个砚台来使使。”行者道:“要怎的?”长老道:“我要修一封书并关文封在一处你替我送上长安驾下见太宗皇帝一面。”行者道:“这个容易我老孙别事无能若说送书:人间第一。你把书收拾停当与我我一筋斗送到长安递与唐王再一筋斗转将回来你的笔砚还不干哩。但只是你寄书怎的?且把书意念念我听念了再写不迟。”长老滴泪道:“我写着:臣僧稽三顿万岁山呼拜圣君;文武两班同入目公卿四百共知闻:当年奉旨离东土指望灵山见世尊。不料途中遭厄难何期半路有灾迍。僧病沉疴难进步佛门深远接天门。有经无命空劳碌启奏当今别遣人。”行者听得此言忍不住呵呵大笑道:
	
	“师父你忒不济略有些病儿就起这个意念。你若是病重要死要活只消问我。我老孙自有个本事问道‘那个阎王敢起心?那个判官敢出票?那个鬼使来勾取?’若恼了我我拿出那大闹天宫之性子又一路棍打入幽冥捉住十代阎王一个个抽了他的筋还不饶他哩!”三藏道:“徒弟呀我病重了切莫说这大话。”八戒上前道:“师兄师父说不好你只管说好十分不尴魀。mianhuatang.la [棉花糖小说网]我们趁早商量先卖了马典了行囊买棺木送终散火。”行者道:“呆子又胡说了!你不知道师父是我佛如来第二个徒弟原叫做金蝉长老只因他轻慢佛法该有这场大难。”
	
	八戒道:“哥啊师父既是轻慢佛法贬回东土在是非海内口舌场中托化做人身愿往西天拜佛求经遇妖精就捆逢魔头就吊受诸苦恼也彀了怎么又叫他害病?”行者道:“你那里晓得老师父不曾听佛讲法打了一个盹往下一失左脚下躧了一粒米下界来该有这三日病。”八戒惊道:“象老猪吃东西泼泼撒撒的也不知害多少年代病是!”行者道:“兄弟佛不与你众生为念。你又不知人云锄禾日当午汗滴禾下土。谁知盘中餐粒粒皆辛苦!师父只今日一日明日就好了。”三藏道:
	
	“我今日比昨不同咽喉里十分作渴。你去那里有凉水寻些来我吃。”行者道:“好了!师父要水吃便是好了。等我取水去。”
	
	即时取了钵盂往寺后面香积厨取水。忽见那些和尚一个个眼儿通红悲啼哽咽只是不敢放声大哭。行者道:“你们这些和尚忒小家子样!我们住几日临行谢你柴火钱照日算还。怎么这等脓包!”众僧慌跪下道:“不敢!不敢!”行者道:
	
	“怎么不敢?想是我那长嘴和尚食肠大吃伤了你的本儿也?”
	
	众僧道:“老爷我这荒山大大小小也有百十众和尚每一人养老爷一日也养得起百十日。怎么敢欺心计较甚么食用!”
	
	行者道:“既不计较你却为甚么啼哭?”众僧道:“老爷不知是那山里来的妖邪在这寺里。我们晚夜间着两个小和尚去撞钟打鼓只听得钟鼓响罢再不见人回。至次日找寻只见僧帽僧鞋丢在后边园里骸骨尚存将人吃了。你们住了三日我寺里不见了六个和尚。故此我兄弟们不由的不怕不由的不伤。
	
	因见你老师父贵慈不敢传说忍不住泪珠偷垂也。”行者闻言又惊又喜道:“不消说了必定是妖魔在此伤人也等我与你剿除他。”众僧道:“老爷妖精不精者不灵一定会腾云驾雾一定会出幽入冥。古人道得好莫信直中直须防仁不仁。
	
	老爷你莫怪我们说:你若拿得他住哩便与我荒山除了这条祸根正是三生有幸了;若还拿他不住啊却有好些儿不便处。”行者道:“怎叫做好些不便处?”那众僧道:“直不相瞒老爷说。我这荒山虽有百十众和尚却都只是自小儿出家的长寻刀削衣单破衲缝。早晨起来洗着脸叉手躬身皈依大道;
	
	夜来收拾烧着香虔心叩齿念的弥陀。举头看见佛莲九品秇三乘慈航共法云愿见叆园释世尊;低头看见心受五戒度大千生生万法中愿悟顽空与色空。诸檀越来啊老的、小的、长的、矮的、胖的、瘦的一个个敲木鱼击金磬挨挨拶拶两卷《法华经》一策《梁王忏》;诸檀越不来啊新的、旧的、生的、熟的、村的、俏的一个个合着掌瞑着目悄悄冥冥入定蒲团上牢关月下门。一任他莺啼鸟语闲争斗不上我方便慈悲大法乘。因此上也不会伏虎也不会降龙;也不识的怪也不识的精。你老爷若还惹起那妖魔啊我百十个和尚只彀他斋一饱一则堕落我众生轮回二则灭抹了这禅林古迹三则如来会上全没半点儿光辉。这却是好些儿不便处。”行者闻得众和尚说出这一端的话语他便怒从心上起恶向胆边生高叫一声:“你这众和尚好呆哩!只晓得那妖精就不晓得我老孙的行止么?”众僧轻轻的答道:“实不晓得。”行者道:“我今日略节说说你们听着:我也曾花果山伏虎降龙我也曾上天堂大闹天宫。饥时把老君的丹略略咬了两三颗;渴时把玉帝的酒轻轻呼了六七钟。(WWW.mianhuatang.la 好看的小说)睁着一双不白不黑的金睛眼天惨淡月朦胧;
	
	拿着一条不短不长的金箍棒来无影去无踪。说甚么大精小怪那怕他惫懒膭脓!一赶赶上去跑的跑颤的颤躲的躲慌的慌;一捉捉将来锉的锉烧的烧磨的磨舂的舂。正是八仙同过海独自显神通!众和尚我拿这妖精与你看看你才认得我老孙!”众僧听着暗点头道:“这贼秃开大口话大话想是有些来历。”都一个个诺诺连声只有那喇嘛僧道:“且住!你老师父贵恙你拿这妖精不至紧。俗语道公子登筵不醉便饱;
	
	壮士临阵不死即伤。你两下里角斗之时倘贻累你师父不当稳便。”行者道:“有理!有理!我且送凉水与师父吃了再来。”
	
	掇起钵盂着上凉水转出香积厨就到方丈叫声:“师父吃凉水哩。”三藏正当烦渴之时便抬起头来捧着水只是一吸真个渴时一滴如甘露药到真方病即除。行者见长老精神渐爽眉目舒开就问道:“师父可吃些汤饭么?”三藏道:“这凉水就是灵丹一般这病儿减了一半有汤饭也吃得些。”行者连声高高叫道:“我师父好了要汤饭吃哩。”教那些和尚忙忙的安排。淘米煮饭捍面烙饼蒸馍馍做粉汤抬了四五桌。唐僧只吃得半碗儿米汤行者沙僧止用了一席其余的都是八戒一肚餐之。家火收去点起灯来众僧各散。”
	
	三藏道:“我们今住几日了?”行者道:“三整日矣。明朝向晚便就是四个日头。”三藏道:“三日误了许多路程。”行者道:
	
	“师父也算不得路程明日去罢。”三藏道:“正是就带几分病儿也没奈何。”行者道:“既是明日要去且让我今晚捉了妖精者。”三藏惊道:“又捉甚么妖精?”行者道:“有个妖精在这寺里等老孙替他捉捉。”唐僧道:“徒弟呀我的病身未可你怎么又兴此念!倘那怪有神通你拿他不住啊却又不是害我?”
	
	行者道:“你好灭人威风!老孙到处降妖你见我弱与谁的?只是不动手动手就要赢。”三藏扯住道:“徒弟常言说得好遇方便时行方便得饶人处且饶人。操心怎似存心好争气何如忍气高!”孙大圣见师父苦苦劝他不许降妖他说出老实话来道:“师父实不瞒你说那妖在此吃了人了。”唐僧大惊道:“吃了甚么人?”行者说道:“我们住了三日已是吃了这寺里六个小和尚了。”长老道:“兔死狐悲物伤其类。他既吃了寺内之僧我亦僧也我放你去只但用心仔细些。”行者道:“不消说老孙的手到就消除了。”
	
	你看他灯光前吩咐八戒沙僧看守师父他喜孜孜跳出方丈径来佛殿看时天上有星月还未上那殿里黑暗暗的。他就吹出真火点起琉璃东边打鼓西边撞钟。响罢摇身一变变做个小和尚儿年纪只有十二三岁披着黄绢褊衫白布直裰手敲着木鱼口里念经。等到一更时分不见动静。二更时分残月才升只听见呼呼的一阵风响。好风:黑雾遮天暗愁云照地昏。四方如泼墨一派靛妆浑。先刮时扬尘播土次后来倒树摧林。扬尘播土星光现倒树摧林月色昏。只刮得嫦娥紧抱梭罗树玉兔团团找药盆。九曜星官皆闭户四海龙王尽掩门。庙里城隍觅小鬼空中仙子怎腾云?地府阎罗寻马面判官乱跑赶头巾。刮动昆仑顶上石卷得江湖波浪混。那风才然过处猛闻得兰麝香熏环珮声响即欠身抬头观看呀!却是一个美貌佳人径上佛殿。行者口里呜哩呜喇只情念经。那女子走近前一把搂住道:“小长老念的甚么经?”行者道:“许下的。”女子道:“别人都自在睡觉你还念经怎么?”行者道:
	
	“许下的如何不念?”女子搂住与他亲个嘴道:“我与你到后面耍耍去。”行者故意的扭过头去道:“你有些不晓事!”女子道:“你会相面?”行者道:“也晓得些儿。”女子道:“你相我怎的样子?”行者道:“我相你有些儿偷生搲熟被公婆赶出来的。”
	
	女子道:“相不着!相不着!我不是公婆赶逐不因搲熟偷生。
	
	奈我前生命薄投配男子年轻。不会洞房花烛避夫逃走之情。
	
	趁如今星光月皎也是有缘千里来相会我和你到后园中交欢配鸾俦去也。”行者闻言暗点头道:“那几个愚僧。都被色欲引诱所以伤了性命他如今也来哄我。”就随口答应道:“娘子我出家人年纪尚幼却不知甚么交欢之事。”女子道:“你跟我去我教你。”行者暗笑道:“也罢我跟他去看他怎生摆布。”
	
	他两个搂着肩携着手出了佛殿径至后边园里。那怪把行者使个绊子腿跌倒在地口里“心肝哥哥”的乱叫将手就去掐他的臊根。行者道:“我的儿真个要吃老孙哩!”却被行者接住他手使个小坐跌法把那怪一辘轳掀翻在地上。那怪口里还叫道:“心肝哥哥你倒会跌你的娘哩!”行者暗算道:“不趁此时下手他还到几时!正是先下手为强后下手遭殃。”就把手一叉腰一躬一跳跳起来现出原身法象轮起金箍铁棒劈头就打。那怪倒也吃了一惊他心想道:“这个小和尚这等利害!”打开眼一看原来是那唐长老的徒弟姓孙的他也不惧他。你说这精怪是甚么精怪:金作鼻雪铺毛。地道为门屋安身处处牢。养成三百年前气曾向灵山走几遭。一饱香花和蜡烛如来吩咐下天曹。托塔天王恩爱女哪吒太子认同胞。也不是个填海鸟也不是个戴山鳌。也不怕的雷焕剑也不怕的吕虔刀。往往来来一任他水流江汉阔;上上下下那论他山耸泰恒高?你看他月貌花容娇滴滴谁识得是个鼠老成精逞黠豪!他自恃的神通广大便随手架起双股剑玎玎珰珰的响左遮右格随东倒西。行者虽强些却也捞他不倒。阴风四起残月无光你看他两人后园中一场好杀:阴风从地起残月荡微光。阒静梵王宇阑珊小鬼廊。后园里一片战争场孙大士天上圣毛姹女女中王赌赛神通未肯降。一个儿扭转芳心嗔黑秃一个儿圆睁慧眼恨新妆。两手剑飞那认得女菩萨;一根棍打狠似个活金刚。响处金箍如电掣霎时铁白耀星芒。玉楼抓翡翠金殿碎鸳鸯。猿啼巴月小雁叫楚天长。十八尊罗汉暗暗喝采;三十二诸天个个慌张。
	
	那孙大圣精神抖擞棍儿没半点差池。妖精自料敌他不住猛可的眉头一蹙计上心来抽身便走。行者喝道:“泼货!
	
	那走!快快来降!”那妖精只是不理直往后退。等行者赶到紧急之时即将左脚上花鞋脱下来吹口仙气念个咒语叫一声“变!”就变做本身模样使两口剑舞将来真身一幌化阵清风而去。这却不是三藏的灾星?他便径撞到方丈里把唐三藏摄将去云头上杳杳冥冥霎霎眼就到了陷空山进了无底洞叫小的们安排素筵席成亲不题。
	
	却说行者斗得心焦性燥闪一个空一棍把那妖精打落下来乃是一只花鞋。行者晓得中了他计连忙转身来看师父。那有个师父?只见那呆子和沙僧口里呜哩呜哪说甚么。行者怒气填胸也不管好歹捞起棍来一片打连声叫道:“打死你们!
	
	打死你们!”那呆子慌得走也没路沙僧却是个灵山大将见得事多就软款温柔近前跪下道:“兄长我知道了想你要打杀我两个也不去救师父径自回家去哩。”行者道:“我打杀你两个我自去救他!”沙僧笑道:“兄长说那里话!无我两个真是单丝不线孤掌难鸣。兄啊这行囊马匹谁与看顾?宁学管鲍分金休仿孙庞斗智。自古道打虎还得亲兄弟上阵须教父子兵望兄长且饶打待天明和你同心戮力寻师去也。”行者虽是神通广大却也明理识时见沙僧苦苦哀告便就回心道:
	
	“八戒沙僧你都起来。明日找寻师父却要用力。”那呆子听见饶了恨不得天也许下半边道:“哥啊这个都在老猪身上。”兄弟们思思想想那曾得睡恨不得点头唤出扶桑日一口吹散满天星。
	
	三众只坐到天晓收拾要行早有寺僧拦门来问:“老爷那里去?”行者笑道:“不好说昨日对众夸口说与他们拿妖精妖精未曾拿得倒把我个师父不见了。我们寻师父去哩。”众僧害怕道:“老爷小可的事倒带累老师却往那里去寻?”行者道:“有处寻他。”众僧又道:“既去莫忙且吃些早斋。”连忙的端了两三盆汤饭。八戒尽力吃个干净道:“好和尚!我们寻着师父再到你这里来耍子。”行者道:“还到这里吃他饭哩!你去天王殿里看看那女子在否。”众僧道:“老爷不在了不在了。
	
	自是当晚宿了一夜第二日就不见了。”
	
	行者喜喜欢欢的辞了众僧着八戒、沙僧牵马挑担径回东走。八戒道:“哥哥差了怎么又往东行?”行者道:“你岂知道!前日在那黑松林绑的那个女子老孙火眼金睛把他认透了你们都认做好人。今日吃和尚的也是他摄师父的也是他!
	
	你们救得好女菩萨!今既摄了师父还从旧路上找寻去也。”二人叹服道:“好好好!真是粗中有细!去来去来!”三人急急到于林内只见那:云蔼蔼雾漫漫;石层层路盘盘。狐踪兔迹交加走虎豹豺狼往复钻。林内更无妖怪影不知三藏在何端。行者心焦掣出棒来。摇身一变变作大闹天宫的本相三头六臂六只手理着三根棒在林里辟哩拨喇的乱打。八戒见了道:“沙僧师兄着了恼寻不着师父弄做个气心风了。”原来行者打了一路打出两个老头儿来一个是山神一个是土地上前跪下道:“大圣山神土地来见。”八戒道:“好灵根啊!打了一路打出两个山神土地若再打一路连太岁都打出来也。”
	
	行者问道:“山神土地汝等这般无礼!在此处专一结伙强盗强盗得了手买些猪羊祭赛你又与妖精结掳打伙儿把我师父摄来!如今藏在何处?快快的从实供来免打!”二神慌了道:
	
	“大圣错怪了我耶。妖精不在小神山上不伏小神管辖但只夜间风响处小神略知一二。”行者道:“既知一一说来!”土地道:“那妖精摄你师父去在那正南下离此有千里之遥。那厢有座山唤做陷空山山中有个洞叫做无底洞。是那山里妖精到此变化摄去也。”行者听言暗自惊心喝退了山神土地收了法身现出本相与八戒沙僧道:“师父去得远了。”八戒道:“远便腾云赶去!”好呆子一纵狂风先起随后是沙僧驾云那白马原是龙子出身驮了行李也踏了风雾。大圣即起筋斗一直南来。不多时早见一座大山阻住云脚。三人采住马都按定云头见那山:顶摩碧汉峰接青霄。周围杂树万万千来往飞禽喳喳噪。虎豹成阵走獐鹿打丛行。向阳处琪花瑶草馨香;背阴方腊雪顽冰不化。崎岖峻岭削壁悬崖。直立高峰湾环深涧。松郁郁石磷磷行人见了悚其心。打柴樵子全无影采药仙童不见踪。眼前虎豹能兴雾遍地狐狸乱弄风。八戒道:“哥啊这山如此险峻必有妖邪。”行者道:“不消说了山高原有怪岭峻岂无精!”叫:“沙僧我和你且在此着八戒先下山凹里打听打听看那条路好走端的可有洞府再看是那里开门俱细细打探我们好一齐去寻师父救他。”八戒道:
	
	“老猪晦气!先拿我顶缸!”行者道:“你夜来说都在你身上如何打仰?”八戒道:“不要嚷等我去。”呆子放下钯抖抖衣裳空着手跳下高山找寻路径。这一去毕竟不知好歹如何且听下回分解。
	------------
	
	第八十二回 姹女求阳 元神护道
	
	却说八戒跳下山寻着一条小路依路前行有五六里远近忽见二个女怪在那井上打水。他怎么认得是两个女怪?见他头上戴一顶一尺二三寸高的篾丝鬏髻甚不时兴。呆子走近前叫声妖怪那怪闻言大怒两人互相说道:“这和尚惫懒!我们又不与他相识平时又没有调得嘴惯他怎么叫我们做妖怪!”那怪恼了轮起抬水的杠子劈头就打。这呆子手无兵器遮架不得被他捞了几下侮着头跑上山来道:“哥啊回去罢!
	
	妖怪凶!”行者道:“怎么凶?”八戒道:“山凹里两个女妖精在井上打水我只叫了他一声就被他打了我三四杠子!”行者道:
	
	“你叫他做甚么的?”八戒道:“我叫他做妖怪。”行者笑道:“打得还少。”八戒道:“谢你照顾!头都打肿了还说少哩!”行者道:“‘温柔天下去得刚强寸步难移’。他们是此地之怪我们是远来之僧你一身都是手也要略温存。你就去叫他做妖怪他不打你打我?人将礼乐为先。”八戒道:“一不晓得!”行者道:“你自幼在山中吃人你晓得有两样木么?”八戒道:“不知是甚么木?”行者道:“一样是杨木一样是檀木。杨木性格甚软巧匠取来或雕圣象或刻如来装金立粉嵌玉装花万人烧香礼拜受了多少无量之福。那檀木性格刚硬油房里取了去做柞撒使铁箍箍了头又使铁锤往下打只因刚强所以受此苦楚。”八戒道:“哥啊你这好话儿早与我说说也好却不受他打了。”行者道:“你还去问他个端的。”八戒道:“这去他认得我了。”行者道:“你变化了去。”八戒道:“哥啊且如我变了却怎么问么?”行者道:“你变了去到他跟前行个礼儿看他多大年纪若与我们差不多叫他声姑娘;若比我们老些儿叫他声奶奶。”八戒笑道:“可是蹭蹬!这般许远的田地认得是甚么亲!”行者道:“不是认亲要套他的话哩。若是他拿了师父就好下手;若不是他却不误了我们别处干事?”八戒道:
	
	“说得有理等我再去。”好呆子把钉钯撒在腰里下山凹摇身一变变做个黑胖和尚摇摇摆摆走近怪前深深唱个大喏道:“奶奶贫僧稽了。”那两个喜道:“这个和尚却好会唱个喏儿又会称道一声儿。”问道:“长老那里来的?”八戒道:“那里来的。”又问:“那里去的?”又道:“那里去的。”又问:“你叫做甚么名字?”又答道:“我叫做甚么名字。”那怪笑道:“这和尚好便好只是没来历会说顺口话儿。”八戒道:“奶奶你们打水怎的?”那怪道:“和尚你不知道。我家老夫人今夜里摄了一个唐僧在洞内要管待他我洞中水不干净差我两个来此打这阴阳交媾的好水安排素果素菜的筵席与唐僧吃了晚间要成亲哩。”那呆子闻得此言急抽身跑上山叫:“沙和尚快拿将行李来我们分了罢!”沙僧道:“二哥又分怎的?”八戒道:“分了便你还去流沙河吃人我去高老庄探亲哥哥去花果山称圣白龙马归大海成龙师父已在这妖精洞内成亲哩!我们都各安生理去也!”行者道:“这呆子又胡说了!”八戒道:“你的儿子胡说!才那两个抬水的妖精说安排素筵席与唐僧吃了成亲哩!”行者道:“那妖精把师父困在洞里师父眼巴巴的望我们去救你却在此说这样话!”八戒道:“怎么救?”行者道:“你两个牵着马挑着担我们跟着那两个女怪做个引子引到那门前一齐下手。”真个呆子只得随行。行者远远的标着那两怪渐入深山有一二十里远近忽然不见。八戒惊道:“师父是日里鬼拿去了!”行者道:“你好眼力!怎么就看出他本相来?”八戒道:“那两个怪正抬着水走忽然不见却不是个日里鬼?”
	
	行者道:“想是钻进洞去了等我去看。”
	
	好大圣急睁火眼金睛漫山看处果然不见动静只见那陡崖前有一座玲珑剔透细妆花、堆五采、三檐四簇的牌楼。他与八戒沙僧近前观看上有六个大字乃陷空山无底洞。(wwW.mianhuatang.la 无弹窗广告)行者道:“兄弟呀这妖精把个架子支在这里这不知门向那里开哩。”沙僧说:“不远!不远!好生寻!”都转身看时牌楼下山脚下有一块大石约有十余里方圆;正中间有缸口大的一个洞儿爬得光溜溜的。八戒道:“哥啊这就是妖精出入洞也。”行者看了道:“怪哉!我老孙自保唐僧瞒不得你两个妖精也拿了些却不见这样洞府。八戒你先下去试试看有多少浅深我好进去救师父。”八戒摇头道:“这个难!这个难!我老猪身子夯夯的若塌了脚吊下去不知二三年可得到底哩!”行者道:“就有多深么?”八戒道:“你看!”大圣伏在洞边上仔细往下看处咦!深啊!周围足有三百余里回头道:“兄弟果然深得紧!”八戒道:“你便回去罢。师父救不得耶!”行者道:“你说那里话!莫生懒惰意休起怠荒心且将行李歇下把马拴在牌楼柱上你使钉钯沙僧使杖拦住洞门让我进去打听打听。
	
	若师父果在里面我将铁棒把妖精从里打出跑至门口你两个却在外面挡住这是里应外合。打死精灵才救得师父。”二人遵命。
	
	行者却将身一纵跳入洞中足下彩云生万道身边瑞气护千层。不多时到于深远之间那里边明明朗朗一般的有日色有风声又有花草果木。行者喜道:“好去处啊!想老孙出世天赐与水帘洞这里也是个洞天福地!”正看时又见有一座二滴水的门楼团团都是松竹内有许多房舍又想道:“此必是妖精的住处了我且到那里边去打听打听。且住!若是这般去啊他认得我了且变化了去。”摇身捻诀就变做个苍蝇儿轻轻的飞在门楼上听听。只见那怪高坐在草亭内他那模样比在松林里救他寺里拿他便是不同越打扮得俊了:
	
	盘云髻似堆鸦身着绿绒花比甲。一对金莲刚半折十指如同春笋。团团粉面若银盆朱唇一似樱桃滑。端端正正美人姿月里嫦娥还喜恰。今朝拿住取经僧便要欢娱同枕榻。行者且不言语听他说甚话。少时绽破樱桃喜孜孜的叫道:“小的们快排素筵席来。我与唐僧哥哥吃了成亲。”行者暗笑道:
	
	“真个有这话!我只道八戒作耍子乱说哩!等我且飞进去寻寻看师父在那里。不知他的心性如何。假若被他摩弄动了啊留他在这里也罢。”即展翅飞到里边看处那东廊下上明下暗的红纸格子里面坐着唐僧哩。行者一头撞破格子眼飞在唐僧光头上丁着叫声“师父。”三藏认得声音叫道:“徒弟救我命啊!”行者道:“师父不济呀!那妖精安排筵宴与你吃了成亲哩。或生下一男半女也是你和尚之后代你愁怎的?”长老闻言咬牙切齿道:“徒弟我自出了长安到两界山中收你一向西来那个时辰动荤?那一日子有甚歪意?今被这妖精拿住要求配偶我若把真阳丧了我就身堕轮回打在那阴山背后永世不得翻身!”行者笑道:“莫誓既有真心往西天取经老孙带你去罢。”三藏道:“进来的路儿我通忘了。”行者道:“莫说你忘了。他这洞不比走进来走出去的是打上头往下钻。如今救了你要打底下往上钻。若是造化高钻着洞口儿就出去了;若是造化低钻不着还有个闷杀的日子了。”三藏满眼垂泪道:“似此艰难怎生是好?”行者道:“没事!没事!那妖精整治酒与你吃没奈何也吃他一锺;只要斟得急些儿斟起一个喜花儿来等我变作个蟭蟟虫儿飞在酒泡之下他把我一口吞下肚去我就捻破他的心肝扯断他的肺腑弄死那妖精你才得脱身出去。”三藏道:“徒弟这等说只是不当人子。”行者道:“只管行起善来你命休矣。妖精乃害人之物你惜他怎的!”三藏道:“也罢也罢!你只是要跟着我。”正是那孙大圣护定唐三藏取经僧全靠美猴王。
	
	他师徒两个商量未定早是那妖精安排停当走近东廊外开了门锁叫声:“长老。(WWW.mianhuatang.la 好看的小说)”唐僧不敢答应。又叫一声又不敢答应。他不敢答应者何意?想着口开神气散舌动是非生。却又一条心儿想着若死住法儿不开口怕他心狠顷刻间就害了性命。正是那进退两难心问口三思忍耐口问心正自狐疑那怪又叫一声“长老。”唐僧没奈何应他一声道:“娘子有。”
	
	那长老应出这一句言来真是肉落千斤。人都说唐僧是个真心的和尚往西天拜佛求经怎么与这女妖精答话?不知此时正是危急存亡之秋万分出于无奈虽是外有所答其实内无所欲。妖精见长老应了一声他推开门把唐僧搀起来和他携手挨背交头接耳你看他做出那千般娇态万种风情岂知三藏一腔子烦恼!行者暗中笑道:“我师父被他这般哄诱只怕一时动心。”正是:真僧魔苦遇娇娃妖怪娉婷实可夸。淡淡翠眉分柳叶盈盈丹脸衬桃花。绣鞋微露双钩凤云髻高盘两鬓鸦。含笑与师携手处香飘兰麝满袈裟。妖精挽着三藏行近草亭道:
	
	“长老我办了一杯酒和你酌酌。”唐僧道:“娘子贫僧自不用荤。”妖精道:“我知你不吃荤因洞中水不洁净特命山头上取阴阳交媾的净水做些素果素菜筵席和你耍子。”唐僧跟他进去观看果然见那:盈门下绣缠彩结;满庭中香喷金猊。摆列着黑油垒钿桌朱漆篾丝盘。垒钿桌上有异样珍羞;篾丝盘中盛稀奇素物。林檎、橄榄、莲肉、葡萄、榧、柰、榛、松、荔枝、龙眼、山栗、风菱、枣儿、柿子、胡桃、银杏、金桔、香橙果子随山有;蔬菜更时新:豆腐、面筋、木耳、鲜笋、蘑菇、香蕈、山药、黄精。石花菜、黄花菜青油煎炒;扁豆角、豇豆角熟酱调成。
	
	王瓜、瓠子白果、蔓菁。镟皮茄子鹌鹑做剔种冬瓜方旦名。烂煨芋头糖拌着白煮萝卜醋浇烹。椒姜辛辣般般美咸淡调和色色平。那妖精露尖尖之玉指捧晃晃之金杯满斟美酒递与唐僧口里叫道:“长老哥哥妙人请一杯交欢酒儿。”三藏羞答答的接了酒望空浇奠心中暗祝道:“护法诸天、五方揭谛、四值功曹:弟子陈玄奘自离东土蒙观世音菩萨差遣列位众神暗中保护拜雷音见佛求经今在途中被妖精拿住强逼成亲将这一杯酒递与我吃。此酒果是素酒弟子勉强吃了还得见佛成功;若是荤酒破了弟子之戒永堕轮回之苦!”孙大圣他却变得轻巧在耳根后若象一个耳报但他说话惟三藏听见别人不闻。他知师父平日好吃葡萄做的素酒教吃他一锺。
	
	那师父没奈何吃了急将酒满斟一锺回与妖怪果然斟起有一个喜花儿。行者变作个蟭蟟虫儿轻轻的飞入喜花之下。那妖精接在手且不吃把杯儿放住与唐僧拜了两拜口里娇娇怯怯叙了几句情话。却才举杯那花儿已散就露出虫来。妖精也认不得是行者变的只以为虫儿用小指挑起往下一弹。
	
	行者见事不谐料难入他腹即变做个饿老鹰。真个是:玉爪金睛铁翮雄姿猛气抟云。妖狐狡兔见他昏千里山河时遁。饥处迎风逐雀饱来高贴天门。老拳钢硬最伤人得志凌霄嫌近。
	
	飞起来轮开玉爪响一声掀翻桌席把些素果素菜、盘碟家火尽皆捽碎撇却唐僧飞将出去。唬得妖精心胆皆裂唐僧的骨肉通酥。妖精战战兢兢搂住唐僧道:“长老哥哥此物是那里来的?”三藏道:“贫僧不知。”妖精道:“我费了许多心安排这个素宴与你耍耍却不知这个扁毛畜生从那里飞来把我的家火打碎!”众小妖道:“夫人打碎家火犹可将些素品都泼散在地秽了怎用?”三藏分明晓得是行者弄法他那里敢说。那妖精道:“小的们我知道了想必是我把唐僧困住天地不容故降此物。你们将碎家火拾出去另安排些酒肴不拘荤素我指天为媒指地作订然后再与唐僧成亲。”依然把长老送在东廊里坐下不题。
	
	却说行者飞出去现了本相到于洞口叫声“开门”八戒笑道:“沙僧哥哥来了。”他二人撒开兵器。行者跳出八戒上前扯住道:“可有妖精?可有师父?”行者道:“有!有!有!”八戒道:“师父在里边受罪哩?绑着是捆着?要蒸是要煮?”行者道:“这个事倒没有只是安排素宴要与他干那个事哩。”八戒道:“你造化你造化!你吃了陪亲酒来了!”行者道:“呆子啊!
	
	师父的性命也难保吃甚么陪亲酒!”八戒道:“你怎的就来了?”行者把见唐僧施变化的上项事说了一遍道:“兄弟们再休胡思乱想。师父已在此间老孙这一去一定救他出来。”复翻身入里面还变做个苍蝇儿丁在门楼上听之只闻得这妖怪气呼呼的在亭子上吩咐:“小的们不论荤素拿来烧纸。借烦天地为媒订务要与他成亲。”行者听见暗笑道:“这妖精全没一些儿廉耻!青天白日的把个和尚关在家里摆布。且不要忙等老孙再进去看看。”嘤的一声飞在东廊之下见那师父坐在里边清滴滴腮边泪淌。行者钻将进去丁在他头上又叫声“师父。长老认得声音跳起来咬牙恨道:“猢狲啊!别人胆大还是身包胆;你的胆大就是胆包身!你弄变化神通打破家火能值几何!斗得那妖精淫兴了那里不分荤素安排定要与我交媾此事怎了!”行者暗中陪笑道:“师父莫怪有救你处。”唐僧道:“那里救得我?”行者道:“我才一翅飞起去时见他后边有个花园。你哄他往园里去耍子我救了你罢。”唐僧道:“园里怎么样救?”行者道:“你与他到园里走到桃树边就莫走了。等我飞上桃枝变作个红桃子。你要吃果子先拣红的儿摘下来。红的是我他必然也要摘一个你把红的定要让他。他若一口吃了我却在他肚里等我捣破他的皮袋扯断他的肝肠弄死他你就脱身了。”三藏道:“你若有手段就与他赌斗便了只要钻在他肚里怎么?”行者道:“师父你不知趣。
	
	他这个洞若好出入便可与他赌斗;只为出入不便曲道难行若就动手他这一窝子老老小小连我都扯住却怎么了?
	
	须是这般捽手干大家才得干净。”三藏点头听信只叫:“你跟定我。”行者道:“晓得!晓得!我在你头上。”
	
	师徒们商量定了三藏才欠起身来双手扶着那格子叫道:“娘子娘子。”那妖精听见笑唏唏的跑近跟前道:“妙人哥哥有甚话说?”三藏道:“娘子我出了长安一路西来无日不山无日不水。昨在镇海寺投宿偶得伤风重疾今日出了汗略才好些;又蒙娘子盛情携入仙府只得坐了这一日又觉心神不爽。你带我往那里略散散心耍耍儿去么?”那妖精十分欢喜道:“妙人哥哥倒有些兴趣我和你去花园里耍耍。”叫:“小的们拿钥匙来开了园门打扫路径。”众妖都跑去开门收拾。
	
	这妖精开了格子搀出唐僧。你看那许多小妖都是油头粉面嬝娜娉婷簇簇拥拥与唐僧径上花园而去。好和尚!他在这绮罗队里无他故锦绣丛中作哑聋若不是这铁打的心肠朝佛去。第二个酒色凡夫也取不得经。一行都到了花园之外那妖精俏语低声叫道:“妙人哥哥这里耍耍真可散心释闷。”唐僧与他携手相搀同入园内抬头观看其实好个去处。但见那:
	
	萦回曲径纷纷尽点苍苔;窈窕绮窗处处暗笼绣箔。微风初动轻飘飘展开蜀锦吴绫;细雨才收娇滴滴露出冰肌玉质。日灼鲜杏红如仙子晒霓裳;月映芭蕉青似太真摇羽扇。粉墙四面万株杨柳啭黄鹂;闲馆周围满院海棠飞粉蝶。更看那凝香阁;青蛾阁、解酲阁、相思阁层层卷映朱帘上钩控虾须;又见那养酸亭、披素亭、画眉亭、四雨亭、个个峥嵘华扁上字书鸟篆。看那浴鹤池、洗觞池、怡月池、濯缨池青萍绿藻耀金鳞;
	
	又有墨花轩、异箱轩、适趣轩、慕云轩玉斗琼卮浮绿蚁。池亭上下有太湖石、紫英石、鹦落石、锦川石青青栽着虎须蒲;轩阁东西有木假山、翠屏山、啸风山、玉芝山处处丛生凤尾竹。
	
	荼蘼架、蔷薇架近着秋千架浑如锦帐罗帏;松柏亭、辛夷亭对着木香亭却似碧城绣幕。芍药栏牡丹丛朱朱紫紫斗秾华;夜合台茉藜槛岁岁年年生妩媚。涓涓滴露紫含笑堪画堪描艳艳烧空红拂桑宜题宜赋。论景致休夸阆苑蓬莱;较芳菲不数姚黄魏紫。若到三春闲斗草园中只少玉琼花。长老携着那怪步赏花园看不尽的奇葩异卉。行过了许多亭阁真个是渐入佳境。忽抬头到了桃树林边行者把师父头上一掐那长老就知。
	
	行者飞在桃树枝儿上摇身一变变作个红桃儿其实红得可爱。长老对妖精道:“娘子你这苑内花香枝头果熟苑内花香蜂竞采枝头果熟鸟争衔。怎么这桃树上果子青红不一何也?”妖精笑道:“天无阴阳日月不明;地无阴阳草木不生;
	
	人无阴阳不分男女。这桃树上果子向阳处有日色相烘者先熟故红;背阴处无日者还生故青:此阴阳之道理也。”三藏道“谢娘子指教其实贫僧不知。”即向前伸手摘了个红桃。妖精也去摘了一个青桃。三藏躬身将红桃奉与妖怪道:“娘子你爱色请吃这个红桃拿青的来我吃。”妖精真个换了且暗喜道:“好和尚啊!果是个真人!一日夫妻未做却就有这般恩爱也。”那妖精喜喜欢欢的把唐僧亲敬。这唐僧把青桃拿过来就吃那妖精喜相陪把红桃儿张口便咬。启朱唇露银牙未曾下口原来孙行者十分性急毂辘一个跟头翻入他咽喉之下径到肚腹之中。妖精害怕对三藏道:“长老啊这个果子利害。
	
	怎么不容咬破就滚下去了?”三藏道:“娘子新开园的果子爱吃所以去得快了。”妖精道:“未曾吐出核子他就撺下去了。”
	
	三藏道:“娘子意美情佳喜吃之甚所以不及吐核就下去了。”行者在他肚里复了本相叫声:“师父不要与他答嘴老孙已得了手也!”三藏道:“徒弟方便着些。”妖精听见道:“你和那个说话哩?”三藏道:“和我徒弟孙悟空说话哩。”妖精道:“孙悟空在那里?”三藏道:“在你肚里哩却才吃的那个红桃子不是?”妖精慌了道:“罢了罢了!这猴头钻在我肚里我是死也!
	
	孙行者!你千方百计的钻在我肚里怎的?”行者在里边恨道:
	
	“也不怎的!只是吃了你的六叶连肝肺三毛七孔心;五脏都淘净弄做个梆子精!”妖精听说唬得魂飞魄散战战兢兢的把唐僧抱住道:“长老啊!我只道夙世前缘系赤绳鱼水相和两意浓。不料鸳鸯今拆散何期鸾凤又西东!蓝桥水涨难成事佛庙烟沉嘉会空。着意一场今又别何年与你再相逢!行者在他肚里听见说时只怕长老慈心又被他哄了便就轮拳跳脚支架子理四平几乎把个皮装儿捣破了。那妖精忍不得疼痛倒在尘埃半晌家不敢言语。行者见不言语想是死了却把手略松一松他又回过气来叫:“小的们!在那里?”原来那些小妖自进园门来各人知趣都不在一处各自去采花斗草任意随心耍子让那妖精与唐僧两个自在叙情儿。忽听得叫却才都跑将来又见妖精倒在地上面容改色口里哼哼的爬不动连忙搀起围在一处道:“夫人怎的不好?想是急心疼了?”妖精道:“不是!不是!你莫要问我肚里已有了人也!快把这和尚送出去留我性命!”那些小妖真个都来扛抬。行者在肚里叫道:“那个敢抬!要便是你自家献我师父出去出到外边我饶你命!”那怪精没计奈何只是惜命之心急挣起来把唐僧背在身上拽开步往外就走。小妖跟随道:“老夫人往那里去?”
	
	妖精道:“留得五湖明月在何愁没处下金钩!把这厮送出去等我别寻一个头儿罢!”好妖精一纵云光直到洞口。又闻得叮叮当当兵刃乱响三藏道:“徒弟外面兵器响哩。”行者道:
	
	“是八戒揉钯哩你叫他一声。”三藏便叫:“八戒!”八戒听见道:“沙和尚!师父出来也!”二人掣开钯杖妖精把唐僧驮出。
	
	咦!正是:心猿里应降邪怪土木司门接圣僧。毕竟不知那妖精性命如何且听下回分解。
	------------
	
	第八十三回 心猿识得丹头 姹女还归本性
	
	却说三藏着妖精送出洞外沙和尚近前问曰:“师父出来师兄何在?”八戒道:“他有算计必定贴换师父出来也。”三藏用手指着妖精道:“你师兄在他肚里哩。”八戒笑道:“腌脏杀人!在肚里做甚?出来罢!”行者在里边叫道:“张开口等我出来!”那怪真个把口张开。行者变得小小的睮在咽喉之内正欲出来又恐他无理来咬即将铁棒取出吹口仙气叫“变!”
	
	变作个枣核钉儿撑住他的上腭子把身一纵跳出口外就把铁棒顺手带出把腰一躬还是原身法象举起棒来就打。那妖精也随手取出两口宝剑丁当架住。两个在山头上这场好杀:
	
	双舞剑飞当面架金箍棒起照头来。一个是天生猴属心猿体一个是地产精灵姹女骸。他两个恨冲怀喜处生仇大会垓。那个要取元阳成配偶这个要战纯阴结圣胎。棒举一天寒雾漫剑迎满地黑尘筛。因长老拜如来恨苦相争显大才水火不投母道损阴阳难合各分开。两家斗罢多时节地动山摇树木摧。
	
	八戒见他们赌斗口里絮絮叨叨返恨行者转身对沙僧道:
	
	“兄弟师兄胡缠!才子在他肚里轮起拳来送他一个满肚红扒开肚皮钻出来却不了帐?怎么又从他口里出来却与他争战让他这等猖狂!”沙僧道:“正是却也亏了师兄深洞中救出师父返又与妖精厮战。且请师父自家坐着我和你各持兵器助助大哥打倒妖精去来。”八戒摆手道:“不不不!他有神通我们不济。”沙僧道:“说那里话!都是大家有益之事虽说不济却也放屁添风。”那呆子一时兴掣了钉钯叫声“去来!”他两个不顾师父一拥驾风赶上举钉钯使宝杖望妖精乱打。那妖精战行者一个已是不能又见他二人怎生抵敌急回头抽身就走。行者喝道:“兄弟们赶上!”那妖精见他们赶得紧即将右脚上花鞋脱下来吹口仙气念个咒语叫“变!”即变作本身模样使两口剑舞将来将身一幌化一阵清风径直回去。这番也只说战他们不过顾命而回岂知又有这般样事!
	
	也是三藏灾星未退:他到了洞门前牌楼下却见唐僧在那里独坐他就近前一把抱住抢了行李咬断缰绳连人和马复又摄将进去不题。
	
	且说八戒闪个空一钯把妖精打落地乃是一只花鞋。行者看见道:“你这两个呆子!看着师父罢了谁要你来帮甚么功!”八戒道:“沙和尚如何么!我说莫来。这猴子好的有些夹脑风我们替他降了妖怪返落得他生报怨!”行者道:“在那里降了妖怪?那妖怪昨日与我战时使了一个遗鞋计哄了。你们走了不知师父如何我们快去看看!”三人急回来果然没了师父连行李白马一并无踪。慌得个八戒两头乱跑沙僧前后跟寻孙大圣亦心焦性燥。正寻觅处只见那路旁边斜軃着半截儿缰绳。他一把拿起止不住眼中流泪放声叫道:“师父啊!
	
	我去时辞别人和马回来只见这些绳!”正是那见鞍思俊马滴泪想亲人。八戒见他垂泪忍不住仰天大笑。行者骂道:“你这个夯货!又是要散火哩!”八戒又笑道:“哥啊不是这话师父一定又被妖精摄进洞去了。常言道事无三不成你进洞两遭了再进去一遭管情救出师父来也。”行者揩了眼泪道:“也罢到此地位势不容己我还进去。你两个没了行李马匹耽心却好生把守洞口。”
	
	好大圣即转身跳入里面不施变化就将本身法相。真个是:古怪别腮心里强自小为怪神力壮。高低面赛马鞍鞒眼放金光如火亮。浑身毛硬似钢针虎皮裙系明花响。上天撞散万云飞下海混起千层浪。当天倚力打天王挡退十万八千将。官封大圣美猴精手中惯使金箍棒。今日西天任显能复来洞内扶三藏。你看他停住云光径到了妖精宅外见那门楼门关了不分好歹轮铁棒一下打开闯将进去。那里边静悄悄全无人迹东廊下不见唐僧亭子上桌椅与各处家火一件也无。原来他的洞里周围有三百余里妖精窠穴甚多。前番摄唐僧在此被行者寻着今番摄了又怕行者来寻当时搬了不知去向。
	
	恼得这行者跌脚捶胸放声高叫道:“师父啊!你是个晦气转成的唐三藏灾殃铸就的取经僧!噫!这条路且是走熟了如何不在?却教老孙那里寻找也!”正自吆喝爆燥之间忽闻得一阵香烟扑鼻他回了性道:“这香烟是从后面飘出想是在后头哩。”拽开步提着铁棒走将进去看时也不见动静。只见有三间倒坐儿近后壁却铺一张龙吞口雕漆供桌桌上有一个大流金香炉炉内有香烟馥郁。那上面供养着一个大金字牌牌上写着“尊父李天王之位”略次些儿写着“尊兄哪吒三太子位”。
	
	行者见了满心欢喜也不去搜妖怪找唐僧把铁棒捻作个绣花针儿揌在耳朵里轮开手把那牌子并香炉拿将起来返云光径出门去。至洞口唏唏哈哈笑声不绝。八戒沙僧听见掣放洞口迎着行者道:“哥哥这等欢喜想是救出师父也?”行者笑道:“不消我们救只问这牌子要人。”八戒道:“哥啊这牌子不是妖精又不会说话怎么问他要人?”行者放在地下道:
	
	“你们看!”沙僧近前看时上写着“尊父李天王之位”、“尊兄哪吒三太子位”。沙僧道:“此意何也?”行者道:“这是那妖精家供养的。我闯入他住居之所见人迹俱无惟有此牌。想是李天王之女三太子之妹思凡下界假扮妖邪将我师父摄去。不问他要人却问谁要?你两个且在此把守等老孙执此牌位径上天堂玉帝前告个御状教天王爷儿们还我师父。”八戒道:
	
	“哥啊常言道告人死罪得死罪须是理顺方可为之。况御状又岂是可轻易告的?你且与我说怎的告他?”行者笑道:“我有主张我把这牌位香炉做个证见另外再备纸状儿。”八戒道:
	
	“状儿上怎么写?你且念念我听。”行者道:“告状人孙悟空年甲在牒系东土唐朝西天取经僧唐三藏徒弟。告为假妖摄陷人口事。今有托塔天王李靖同男哪吒太子闺门不谨走出亲女在下方陷空山无底洞变化妖邪迷害人命无数。今将吾师摄陷曲邃之所渺无寻处。若不状告切思伊父子不仁故纵女氏成精害众。伏乞怜准行拘至案收邪救师明正其罪深为恩便。
	
	有此上告。”八戒沙僧闻其言十分欢喜道:“哥啊告的有理必得上风。切须早来稍迟恐妖精伤了师父性命。”行者道:“我快!我快!多时饭熟少时茶滚就回。”
	
	好大圣执着这牌位香炉将身一纵驾祥云直至南天门外。时有把天门的大力天王与护国天王见了行者一个个都控背躬身不敢拦阻让他进去。直至通明殿下有张葛许邱四大天师迎面作礼道:“大圣何来?”行者道:“有纸状儿要告两个人哩。”天师吃惊道:“这个赖皮不知要告那个。”无奈将他引入灵霄殿下启奏。蒙旨宣进行者将牌位香炉放下朝上礼毕将状子呈上。葛仙翁接了铺在御案。玉帝从头看了见这等这等即将原状批作圣旨宣西方长庚太白金星领旨到云楼宫宣托塔李天王见驾。行者上前奏道:“望天主好生惩治不然又别生事端。”玉帝又吩咐:“原告也去。”行者道:“老孙也去?”
	
	四天师道:“万岁已出了旨意你可同金星去来。”行者真个随着金星纵云头早至云楼宫。原来是天王住宅号云楼宫。金星见宫门有个童子侍立那童子认得金星即入里报道:“太白金星老爷来了”天王遂出迎迓又见金星捧着旨意即命焚香。及转身又见行者跟入天王即又作怒。你道他作怒为何?
	
	当年行者大闹天宫时玉帝曾封天王为降魔大元帅封哪吒太子为三坛海会之神帅领天兵收降行者屡战不能取胜。还是五百年前败阵的仇气有些恼他故此作怒。他且忍不住道:
	
	“老长庚你赍得是甚么旨意?”金星道:“是孙大圣告你的状子。”那天王本是烦恼听见说个“告”字一雷霆大怒道:“他告我怎的?”金星道:“告你假妖摄陷人口事。你焚了香请自家开读。”那天王气呼呼的设了香案望空谢恩。拜毕展开旨意看了原来是这般这般如此如此恨得他手扑着香案道:“这个猴头!他也错告我了!”金星道:“且息怒现有牌位香炉在御前作证说是你亲女哩。”天王道:“我止有三个儿子一个女儿。大小儿名金吒侍奉如来做前部护法。二小儿名木叉在南海随观世音做徒弟。三小儿得名哪吒在我身边早晚随朝护驾。一女年方七岁名贞英人事尚未省得如何会做妖精!
	
	不信抱出来你看。这猴头着实无礼!且莫说我是天上元勋封受先斩后奏之职就是下界小民也不可诬告。律云:诬告加三等。”叫手下:“将缚妖索把这猴头捆了!”那庭下摆列着巨灵神、鱼肚将、药叉雄帅一拥上前把行者捆了。金星道:“李天王莫闯祸啊!我在御前同他领旨意来宣你的人。你那索儿颇重一时捆坏他阁气。”天王道:“金星啊似他这等诈伪告扰怎该容他!你且坐下待我取砍妖刀砍了这个猴头然后与你见驾回旨!”金星见他取刀心惊胆战对行者道:“你干事差了御状可是轻易告的?你也不访的实似这般乱弄伤其性命怎生是好?”行者全然不惧笑吟吟的道:“老官儿放心一些没事。老孙的买卖原是这等做一定先输后赢。”
	
	说不了天王轮过刀来望行者劈头就砍。早有那三太子赶上前将斩腰剑架住叫道:“父王息怒。”天王大惊失色。噫!
	
	父见子以剑架刀就当喝退怎么返大惊失色?原来天王生此子时他左手掌上有个“哪”字右手掌上有个“吒”字故名哪吒。这太子三朝儿就下海净身闯祸踏倒水晶宫捉住蛟龙要抽筋为绦子。天王知道恐生后患欲杀之。哪吒奋怒将刀在手割肉还母剔骨还父还了父精母血一点灵魂径到西方极乐世界告佛。佛正与众菩萨讲经只闻得幢幡宝盖有人叫道:“救命!”佛慧眼一看知是哪吒之魂即将碧藕为骨荷叶为衣念动起死回生真言哪吒遂得了性命。运用神力法降九十六洞妖魔神通广大后来要杀天王报那剔骨之仇。天王无奈告求我佛如来。如来以和为尚赐他一座玲珑剔透舍利子如意黄金宝塔那塔上层层有佛艳艳光明。唤哪吒以佛为父解释了冤仇。所以称为托塔李天王者此也。今日因闲在家未曾托着那塔恐哪吒有报仇之意故吓个大惊失色。却即回手向塔座上取了黄金宝塔托在手间问哪吒道:“孩儿你以剑架住我刀有何话说?”哪吒弃剑叩头道:“父王是有女儿在下界哩。”天王道:“孩儿我只生了你姊妹四个那里又有个女儿哩?”哪吒道:“父王忘了那女儿原是个妖精三百年前成怪在灵山偷食了如来的香花宝烛如来差我父子天兵将他拿住。拿住时只该打死如来吩咐道积水养鱼终不钓深山喂鹿望长生当时饶了他性命。积此恩念拜父王为父拜孩儿为兄在下方供设牌位侍奉香火。不期他又成精陷害唐僧却被孙行者搜寻到巢穴之间将牌位拿来就做名告了御状。
	
	此是结拜之恩女非我同胞之亲妹也。”天王闻言悚然惊讶道:
	
	“孩儿我实忘了他叫做甚么名字?”太子道:“他有三个名字:
	
	他的本身出处唤做金鼻白毛老鼠精;因偷香花宝烛改名唤做半截观音;如今饶他下界又改了唤做地涌夫人是也。”天王却才省悟放下宝塔便亲手来解行者。行者就放起刁来道:
	
	“那个敢解我!要便连绳儿抬去见驾老孙的官事才赢!”慌得天王手软太子无言众家将委委而退。那大圣打滚撒赖只要天王去见驾。天王无计可施哀求金星说个方便。金星道:“古人云万事从宽。你干事忒紧了些儿就把他捆住又要杀他。
	
	这猴子是个有名的赖皮你如今教我怎的处!若论你令郎讲起来虽是恩女不是亲女却也晚亲义重不拘怎生折辨你也有个罪名。”天王道:“老星怎说个方便就没罪了。”金星道:
	
	“我也要和解你们却只是无情可说。”天王笑道:“你把那奏招安授官衔的事说说他也罢了。”真个金星上前将手摸着行者道:“大圣看我薄面解了绳好去见驾。”行者道:“老官儿不用解我会滚法一路滚就滚到也。”金星笑道:“你这猴忒恁寡情我昔日也曾有些恩义儿到你你这些些事儿就不依我?”
	
	行者道:“你与我有甚恩义?”金星道:“你当年在花果山为怪伏虎降龙强消死籍聚群妖大肆猖狂上天欲要擒你是老身力奏降旨招安把你宣上天堂封你做弼马温。你吃了玉帝仙酒后又招安也是老身力奏封你做齐天大圣。你又不守本分偷桃盗酒窃老君之丹如此如此才得个无灭无生。若不是我你如何得到今日?”行者道:“古人说得好死了莫与老头儿同墓干净会揭挑人!我也只是做弼马温闹天宫罢了再无甚大事。也罢也罢看你老人家面皮还教他自己来解。”天王才敢向前解了缚请行者着衣上坐一一上前施礼。
	
	行者朝了金星道:“老官儿何如?我说先输后赢买卖儿原是这等做。快催他去见驾莫误了我的师父。”金星道:“莫忙弄了这一会也吃锤茶儿去。”行者道:“你吃他的茶受他的私卖放犯人轻慢圣旨你得何罪?”金星道:“不吃茶!不吃茶!连我也赖将起来了!李天王快走快走!”天王那里敢去怕他没的说做有的放起刁来口里胡说乱道怎生与他折辨没奈何又央金星教说方便。金星道:“我有一句话儿你可依我?”行者道:“绳捆刀砍之事我也通看你面还有甚话?你说!
	
	你说!说得好就依你;说得不好莫怪。”金星道:“一日官事十日打你告了御状说妖精是天王的女儿天王说不是你两个只管在御前折辨反复不已我说天上一日下界就是一年。这一年之间那妖精把你师父陷在洞中莫说成亲若有个喜花下儿子也生了一个小和尚儿却不误了大事?”行者低头想道:“是啊!我离八戒沙僧只说多时饭熟、少时茶滚就回今已弄了这半会却不迟了?老官儿既依你说这旨意如何回缴?”
	
	金星道:“教李天王点兵同你下去降妖我去回旨。”行者道:
	
	“你怎么样回?”金星道:“我只说原告脱逃被告免提。”行者笑道:“好啊!我倒看你面情罢了你倒说我脱逃!教他点兵在南天门外等我我即和你回旨缴状去。”天王害怕道:“他这一去若有言语是臣背君也。”行者道:“你把老孙当甚么样人?我也是个大丈夫!一言既出驷马难追岂又有污言顶你?”天王即谢了行者行者与金星回旨。天王点起本部天兵径出南天门外。金星与行者回见玉帝道:“陷唐僧者乃金鼻白毛老鼠成精假设天王父子牌位。天王知之已点兵收怪去了望天尊赦罪。”玉帝已知此情降天恩免究。行者即返云光到南天门外见天王、太子布列天兵等候。噫!那些神将风滚滚雾腾腾接住大圣一齐坠下云头早到了陷空山上。
	
	八戒沙僧眼巴巴正等只见天兵与行者来了。呆子迎着天王施礼道:“累及!累及!”天王道:“天蓬元帅你却不知只因我父子受他一炷香致令妖精无理困了你师父来迟莫怪。这个山就是陷空山了?但不知他的洞门还向那边开?”行者道:
	
	“我这条路且是走熟了。只是这个洞叫做个无底洞周围有三百余里妖精窠穴甚多。前番我师父在那两滴水的门楼里今番静悄悄鬼影也没个不知又搬在何处去也。”天王道:“任他设尽千般计难脱天罗地网中。到洞门前再作道理。”大家就行。咦约有十余里就到了那大石边。行者指那缸口大的门儿道:“兀的便是也。”天王道:“不入虎穴安得虎子!谁敢当先”行者道:“我当先。”三太子道:“我奉旨降妖我当先。”那呆子便莽撞起来高声叫道:“当头还要我老猪!”天王道:“不须罗噪但依我分摆:孙大圣和太子同领着兵将下去我们三人在口上把守做个里应外合教他上天无路入地无门才显些些手段。”众人都答应了一声“是”。
	
	你看那行者和三太子领了兵将望洞里只是一溜。驾起云光闪闪烁烁抬头一望果然好个洞啊:依旧双轮日月照般一望山川。珠渊玉井暖韬烟更有许多堪羡。迭迭朱楼画阁嶷嶷赤壁青田。三春杨柳九秋莲兀的洞天罕见。顷刻间停住了云光径到那妖精旧宅。挨门儿搜寻吆吆喝喝一重又一重一处又一处把那三百里地草都踏光了那见个妖精?那见个三藏?都只说:“这孽畜一定是早出了这洞远远去哩。”那晓得在那东南黑角落上望下去另有个小洞。洞里一重小小门一间矮矮屋盆栽了几种花檐傍着数竿竹黑气氲氲暗香馥馥老怪摄了三藏搬在这里逼住成亲只说行者再也找不着。
	
	谁知他命合该休那些小怪在里面一个个哜哜嘈嘈挨挨簇簇。中间有个大胆些的伸起颈来望洞外略看一看一头撞着个天兵一声嚷道:“在这里!”那行者恼起性来捻着金箍棒一下闯将进去那里边窄小窝着一窟妖精。三太子纵起天兵一齐拥上一个个那里去躲?行者寻着唐僧和那龙马和那行李。那老怪寻思无路看着哪吒太子只是磕头求命。太子道:
	
	“这是玉旨来拿你不当小可。我父子只为受了一炷香。险些儿和尚拖木头做出了寺!”啈声“天兵取下缚妖索把那些妖精都捆了!”老怪也少不得吃场苦楚。返云光一齐出洞。行者口里嘻嘻嘎嘎。天王掣开洞口迎着行者道:“今番却见你师父也。”行者道:“多谢了!多谢了!”就引三藏拜谢天王次及太子。沙僧八戒只是要碎剐那老精天王道:“他是奉玉旨拿的轻易不得。我们还要去回旨哩。”一边天王同三太子领着天兵神将押住妖精去奏天曹听候落;一边行者拥着唐僧沙僧收拾行李八戒拢马请唐僧骑马齐上大路。这正是:割断丝萝干金海打开玉锁出樊笼。毕竟不知前去何如且听下回分解。
	------------
	
	第八十四回 难灭伽持圆大觉 法王成正体天然
	
	话说唐三藏固住元阳出离了烟花苦套随行者投西前进。不觉夏时正值那熏风初动梅雨丝丝好光景:冉冉绿阴密风轻燕引雏。新荷翻沼面修竹渐扶苏。芳草连天碧山花遍地铺。溪边蒲插剑榴火壮行图。师徒四众耽炎受热正行处忽见那路旁有两行高柳柳阴中走出一个老母右手下搀着一个小孩儿对唐僧高叫道:“和尚不要走了快早儿拨马东回进西去都是死路。“唬得个三藏跳下马来打个问讯道:
	
	“老菩萨古人云海阔从鱼跃天空任鸟飞怎么西进便没路了?”那老母用手朝西指道:“那里去有五六里远近乃是灭法国。那国王前生那世里结下冤仇今世里无端造罪。二年前许下一个罗天大愿要杀一万个和尚这两年陆陆续续杀彀了九千九百九十六个无名和尚只要等四个有名的和尚凑成一万好做圆满哩。你们去若到城中都是送命王菩萨!”三藏闻言心中害怕战兢兢的道:“老菩萨深感盛情感谢不尽!但请问可有不进城的方便路儿我贫僧转过去罢。”那老母笑道:
	
	“转不过去转不过去只除是会飞的就过去了也。”八戒在旁边卖嘴道:“妈妈儿莫说黑话我们都会飞哩。”行者火眼金睛其实认得好歹那老母搀着孩儿原是观音菩萨与善财童子慌得倒身下拜叫道:“菩萨弟子失迎!失迎!”那菩萨一朵祥云轻轻驾起吓得个唐长老立身无地只情跪着磕头。八戒沙僧也慌跪下朝天礼拜。一时间祥云缥缈径回南海而去。行者起来扶着师父道:“请起来菩萨已回宝山也。”三藏起来道:“悟空你既认得是菩萨何不早说?”行者笑道:“你还问话不了我即下拜怎么还是不早哩?”八戒沙僧对行者道:“感蒙菩萨指示前边必是灭法国要杀和尚我等怎生奈何?”行者道:“呆子休怕!我们曾遭着那毒魔狠怪虎穴龙潭更不曾伤损?此间乃是一国凡人有何惧哉?只奈这里不是住处。天色将晚且有乡村人家上城买卖回来的看见我们是和尚嚷出名去不当稳便。且引师父找下大路寻个僻静之处却好商议。”真个三藏依言一行都闪下路来到一个坑坎之下坐定。
	
	行者道:“兄弟你两个好生保守师父待老孙变化了去那城中看看寻一条僻路连夜去也。”三藏叮嘱道:“徒弟啊莫当小可王法不容你须仔细!”行者笑道:“放心!放心!老孙自有道理。”
	
	好大圣话毕将身一纵唿哨的跳在空中。怪哉:上面无绳扯下头没棍撑一般同父母他便骨头轻。佇立在云端里、往下观看只见那城中喜气冲融祥光荡漾。行者道:“好个去处为何灭法?”看一会渐渐天昏又见那:十字街灯光灿烂九重殿香蔼钟鸣。七点皎星照碧汉八方客旅卸行踪。六军营隐隐的画角才吹;五鼓楼点点的铜壶初滴。四边宿雾昏昏三市寒烟蔼蔼。两两夫妻归绣幕一轮明月上东方。他想着:“我要下去到街坊打看路径这般个嘴脸撞见人必定说是和尚等我变一变了。”捻着诀念动真言摇身一变变做个扑灯蛾儿:
	
	形细翼硗轻巧灭灯扑烛投明。本来面目化生成腐草中间灵应。每爱炎光触焰忙忙飞绕无停。紫衣香翅赶流萤最喜夜深风静。但见他翩翩翻翻飞向六街三市。傍房檐近屋角正行时忽见那隅头拐角上一湾子人家人家门挂着个灯笼儿。他道:“这人家过元宵哩?怎么挨排儿都点灯笼?”他硬硬翅飞近前来仔细观看正当中一家子方灯笼上写着安歇往来商贾六字下面又写着王小二店四字行者才知是开饭店的。又伸头打一看看见有八九个人都吃了晚饭宽了衣服卸了头巾洗了脚手各各上床睡了。行者暗喜道:“师父过得去了。”你道他怎么就知过得去?他要起个不良之心等那些人睡着要偷他的衣服头巾装做俗人进城。
	
	噫有这般不遂意的事!正思忖处只见那小二走向前吩咐:“列位官人仔细些我这里君子小人不同各人的衣物行李都要小心着。”你想那在外做买卖的人那样不仔细?又听得店家吩咐越谨慎。他都爬起来道:“主人家说得有理我们走路的人辛苦只怕睡着急忙不醒一时失所奈何?你将这衣服头巾、搭联都收进去待天将明交付与我们起身。”那王小二真个把些衣物之类尽情都搬进他屋里去了。行者性急展开翅就飞入里面丁在一个头巾架上。又见王小二去门摘了灯笼放下吊搭关了门窗却才进房脱衣睡下。那王小二有个婆婆带了两个孩子哇哇聒噪急忙不睡。那婆子又拿了一件破衣补补纳纲也不见睡。行者暗想道:“若等这婆子睡下下手却不误了师父?”又恐更深城门闭了他就忍不住飞下去望灯上一扑真是舍身投火焰焦额探残生那盏灯早已息了。他又摇身一变变作个老鼠睳睳哇哇的叫了两声跳下来拿着衣服头巾往外就走。那婆子慌慌张张的道:“老头子!
	
	不好了!夜耗子成精也!”行者闻言又弄手段拦着门厉声高叫道:“王小二莫听你婆子胡说我不是夜耗子成精。明人不做暗事吾乃齐天大圣临凡保唐僧往西天取经。你这国王无道特来借此衣冠装扮我师父。一时过了城去就便送还。”那王小二听言一毂辘起来黑天摸地又是着忙的人捞着裤子当衫子左穿也穿不上右套也套不上。
	
	那大圣使个摄法早已驾云出去复翻身径至路下坑坎边前。三藏见星光月皎探身凝望见是行者来至近前即开口叫道:“徒弟可过得灭法国么?”行者上前放下衣物道:“师父要过灭法国和尚做不成。”八戒道:“哥你勒掯那个哩?不做和尚也容易只消半年不剃头就长出毛来也。”行者道:“那里等得半年!眼下就都要做俗人哩!”那呆子慌了道:“但你说话通不察理。我们如今都是和尚眼下要做俗人却怎么戴得头巾?就是边儿勒住也没收顶绳处。”三藏喝道:“不要打花且干正事!端的何如?”行者道:“师父他这城池我已看了。虽是国王无道杀僧却倒是个真天子城头上有祥光喜气。城中的街道我也认得这里的乡谈我也省得会说。却才在饭店内借了这几件衣服头巾我们且扮作俗人进城去借了宿至四更天就起来教店家安排了斋吃;捱到五更时候挨城门而去奔大路西行就有人撞见扯住也好折辨只说是上邦钦差的灭法王不敢阻滞放我们来的。”沙僧道:“师兄处的最当且依他行。”真个长老无奈脱了褊衫去了僧帽穿了俗人的衣服戴了头巾。沙僧也换了八戒的头大戴不得巾儿被行者取了些针线把头巾扯开两顶缝做一顶与他搭在头上拣件宽大的衣服与他穿了然后自家也换上一套道:“列位这一去把师父徒弟四个字儿且收起。”八戒道:“除了此四字怎的称呼?”行者道:“都要做弟兄称呼:师父叫做唐大官儿你叫做朱三官儿沙僧叫做沙四官儿我叫做孙二官儿。但到店中你们切休言语只让我一个开口答话。等他问甚么买卖只说是贩马的客人。把这白马做个样子说我们是十弟兄我四个先来赁店房卖马。那店家必然款待我们我们受用了临行时等我拾块瓦查儿变块银子谢他却就走路。”长老无奈只得曲从。
	
	四众忙忙的牵马挑担跑过那边。此处是个太平境界入更时分尚未关门径直进去行到王小二店门只听得里边叫哩。有的说:“我不见了头巾!”有的说:“我不见了衣服!”行者只推不知引着他们往斜对门一家安歇。那家子还未收灯笼即近门叫道:“店家可有闲房儿我们安歇?”那里边有个妇人答应道:“有有有请官人们上楼。”说不了就有一个汉子来牵马。行者把马儿递与牵进去他引着师父从灯影儿后面径上楼门。那楼上有方便的桌椅推开窗格映月光齐齐坐下。
	
	只见有人点上灯来行者拦门一口吹息道:“这般月亮不用灯。”那人才下去又一个丫环拿四碗清茶。行者接住楼下又走上一个妇人来约有五十七八岁的模样一直上楼站着旁边问道:“列位客官那里来的?有甚宝货?”行者道:“我们是北方来的有几匹粗马贩卖。”那妇人道:“贩马的客人尚还小。”
	
	行者道:“这一位是唐大官这一位是朱三官这一位是沙四官我学生是孙二官。”妇人笑道:“异姓。”行者道:“正是异姓同居。我们共有十个弟兄我四个先来赁店房打火;还有六个在城外借歇领着一群马因天晚不好进城。待我们赁了房子明早都进来只等卖了马才回。”那妇人道:“一群有多少马?”
	
	行者道:“大小有百十匹儿都象我这个马的身子却只是毛片不一。”妇人笑道:“孙二官人诚然是个客纲客纪。早是来到舍下第二个人家也不敢留你。我舍下院落宽阔槽札齐备草料又有凭你几百匹马都养得下。却一件:我舍下在此开店多年也有个贱名。先夫姓赵不幸去世久矣我唤做赵寡妇店。我店里三样儿待客。如今先小人后君子先把房钱讲定后好算帐。”行者道:“说得是。你府上是那三样待客?常言道货有高低三等价客无远近一般看你怎么说三样待客?你可试说说我听。”赵寡妇道:“我这里是上、中、下三样。上样者:五果五菜的筵席狮仙斗糖桌面二位一张请小娘儿来陪唱陪歇每位该银五钱连房钱在内。”行者笑道:“相应啊!我那里五钱银子还不彀请小娘儿哩。”寡妇又道:“中样者:合盘桌儿只是水果、热酒筛来凭自家猜枚行令不用小娘儿每位只该二钱银子。”行者道:“一相应!下样儿怎么?”妇人道:“不敢在尊客面前说。”行者道:“也说说无妨我们好拣相应的干。”妇人道:
	
	“下样者:没人伏侍锅里有方便的饭凭他怎么吃:吃饱了拿个草儿打个地铺方便处睡觉天光时凭赐几文饭钱决不争竞。”八戒听说道:“造化造化!老朱的买卖到了!等我看着锅吃饱了饭灶门前睡他娘!”行者道:“兄弟说那里话!你我在江湖上那里不赚几两银子!把上样的安排将来。”那妇人满心欢喜即叫:“看好茶来厨下快整治东西。”遂下楼去忙叫:
	
	“宰鸡宰鹅煮腌下饭。”又叫:“杀猪杀羊今日用不了明日也可用。看好酒拿白米做饭白面捍饼。”三藏在楼上听见道:
	
	“孙二官怎好?他去宰鸡鹅杀猪羊倘送将来我们都是长斋那个敢吃?”行者道:“我有主张。”去那楼门边跌跌脚道:
	
	“赵妈妈你上来。”那妈妈上来道:“二官人有甚吩咐?”行者道:“今日且莫杀生我们今日斋戒。”寡妇惊讶道:“官人们是长斋是月斋?”行者道:“俱不是我们唤做庚申斋。今朝乃是庚申日当斋只过三更后就是辛酉便开斋了你明日杀生罢。如今且去安排些素的来定照上样价钱奉上。”那妇人越欢喜跑下去教:“莫宰!莫宰!取些木耳、闽笋、豆腐、面筋园里拔些青菜做粉汤面蒸卷子再煮白米饭烧香茶。”咦!
	
	那些当厨的庖丁都是每日家做惯的手段霎时间就安排停当摆在楼上。又有现成的狮仙糖果四众任情受用。又问:
	
	“可吃素酒?”行者道:“止唐大官不用我们也吃几杯。”寡妇又取了一壶暖酒他三个方才斟上忽听得乒乓板响行者道:
	
	“妈妈底下倒了甚么家火了?”寡妇道:“不是是我小庄上几个客子送租米来晚了教他在底下睡。因客官到没人使用教他们抬轿子去院中请小娘儿陪你们想是轿杠撞得楼板响。”
	
	行者道:“早是说哩快不要去请。一则斋戒日期二则兄弟们未到。索性明日进来一家请个表子在府上耍耍时待卖了马起身。”寡妇道:“好人!好人!又不失了和气又养了精神。”教:
	
	“抬进轿子来不要请去。”四众吃了酒饭收了家火都散讫。
	
	三藏在行者耳根边悄悄的道:“那里睡?”行者道:“就在楼上睡。”三藏道:“不稳便。我们都辛辛苦苦的倘或睡着这家子一时再有人来收拾见我们或滚了帽子露出光头认得是和尚嚷将起来却怎么好?”行者道:“是啊!”又去楼前跌跌脚。寡妇又上来道:“孙官人又有甚吩咐?”行者道:“我们在那里睡?”妇人道:“楼上好睡又没蚊子又是南风大开着窗子忒好睡觉。”行者道:“睡不得我这朱三官儿有些寒湿气沙四官儿有些漏肩风唐大哥只要在黑处睡我也有些儿羞明。此间不是睡处。”那妈妈走下去倚着柜栏叹气。他有个女儿抱着个孩子近前道:“母亲常言道十日滩头坐一日行九滩如今炎天虽没甚买卖到交秋时还做不了的生意哩你嗟叹怎么?”妇人道:“儿啊不是愁没买卖。今日晚间已是将收铺子入更时分有这四个马贩子来赁店房他要上样管待。实指望赚他几钱银子他却吃斋又赚不得他钱故此嗟叹。”那女儿道:“他既吃了饭不好往别人家去。明日还好安排荤酒如何赚不得他钱?”妇人又道:“他都有病怕风羞亮都要在黑处睡。你想家中都是些单浪瓦儿的房子那里去寻黑暗处?不若舍一顿饭与他吃了教他往别家去罢。”女儿道:“母亲我家有个黑处又无风色甚好甚好。”妇人道:“是那里?”女儿道:
	
	“父亲在日曾做了一张大柜。那柜有四尺宽七尺长三尺高下里面可睡六七个人。教他们往柜里睡去罢。”妇人道:“不知可好等我问他一声。孙官人舍下蜗居更无黑处止有一张大柜不透风又不透亮往柜里睡去如何?”行者道:“好!好!
	
	好!”即着几个客子把柜抬出打开盖儿请他们下楼。行者引着师父沙僧拿担顺灯影后径到柜边。八戒不管好歹就先睮进柜去沙僧把行李递入搀着唐僧进去沙僧也到里边。行者道:“我的马在那里?”旁有伏侍的道:“马在后屋拴着吃草料哩。”行者道:“牵来把糟抬来紧挨着柜儿拴住。”方才进去叫:“赵妈妈盖上盖儿插上锁钉锁上锁子还替我们看看那里透亮使些纸儿糊糊明日早些儿来开。”寡妇道:“忒小心了!”遂此各各关门去睡不题。
	
	却说他四个到了柜里可怜啊!一则乍戴个头巾二来天气炎热又闷住了气略不透风他都摘了头巾脱了衣服又没把扇子只将僧帽扑扑扇扇。你挨着我我挤着你直到有二更时分却都睡着惟行者有心闯祸偏他睡不着伸过手将八戒腿上一捻。那呆子缩了脚口里哼哼的道:“睡了罢!辛辛苦苦的有甚么心肠还捻手捻脚的耍子?”行者捣鬼道:“我们原来的本身是五千两前者马卖了三千两如今两搭联里现有四千两这一群马还卖他三千两也有一本一利彀了!彀了!”八戒要睡的人那里答对。岂知他这店里走堂的挑水的烧火的素与强盗一伙听见行者说有许多银子他就着几个溜出去伙了二十多个贼明火执杖的来打劫马贩子。冲开门进来唬得那赵寡妇娘女们战战兢兢的关了房门尽他外边收拾。原来那贼不要店中家火只寻客人。到楼上不见形迹打着火把四下照看只见天井中一张大柜柜脚上拴着一匹白马柜盖紧锁掀翻不动。众贼道:“走江湖的人都有手眼看这柜势重必是行囊财帛锁在里面。我们偷了马抬柜出城打开分用却不是好?”那些贼果找起绳扛把柜抬着就走幌阿幌的。八戒醒了道:“哥哥睡罢摇甚么?”行者道:“莫言语!没人摇。”三藏与沙僧忽地也醒了道:“是甚人抬着我们哩?”行者道:“莫嚷莫嚷!等他抬!抬到西天也省得走路。”
	
	那贼得了手不往西去倒抬向城东杀了守门的军打开城门出去。当时就惊动六街三市各铺上火甲人夫都报与巡城总兵、东城兵马司。那总兵、兵马事当干己即点人马弓兵出城赶贼。那贼见官军势大不敢抵敌放下大柜丢了白马各自落草逃走。众官军不曾拿得半个强盗只是夺下柜捉住马得胜而回。总兵在灯光下见那马好马:鬃分银线尾軃玉条。说甚么八骏龙驹赛过了骕骦款段。千金市骨万里追风。
	
	登山每与青云合啸月浑如白雪匀。真是蛟龙离海岛人间喜有玉麒麟。总兵官把自家马儿不骑就骑上这个白马帅军兵进城把柜子抬在总府同兵马写个封皮封了令人巡守待天明启奏请旨定夺。官军散讫不题。
	
	却说唐长老在柜里埋怨行者道:“你这个猴头害杀我也!
	
	若在外边被人拿住送与灭法国王还好折辨;如今锁在柜里被贼劫去又被官军夺来明日见了国王现现成成的开刀请杀却不凑了他一万之数?”行者道:“外面有人!打开柜拿出来不是捆着便是吊着。且忍耐些儿免了捆吊。明日见那昏君老孙自有对答管你一毫儿也不伤且放心睡睡。”挨到三更时分行者弄个手段顺出棒来吹口仙气叫“变!”即变做三尖头的钻儿挨柜脚两三钻钻了一个眼子。收了钻摇身一变变做个蝼蚁儿睮将出去现原身踏起云头径入皇宫门外。那国王正在睡浓之际他使个大分身普会神法将左臂上毫毛都拔下来吹口仙气叫“变!”都变做小行者。右臂上毛也都拔下来吹口仙气叫“变!”都变做瞌睡虫;念一声“唵”字真言教当坊土地领众布散皇宫内院五府六部各衙门大小官员宅内但有品职者都与他一个瞌睡虫人人稳睡不许翻身。又将金箍棒取在手中掂一掂幌一幌叫声“宝贝变!”即变做千百口剃头刀儿他拿一把吩咐小行者各拿一把都去皇宫内院、五府六部、各衙门里剃头。咦!这才是:法王灭法法无穷法贯乾坤大道通。万法原因归一体三乘妙相本来同。钻开玉柜明消息布散金毫破蔽蒙。管取法王成正果不生不灭去来空。这半夜剃削成功念动咒语喝退土地神祇将身一抖两臂上毫毛归伏将剃头刀总捻成真依然认了本性还是一条金箍棒收来些小之形藏于耳内。复翻身还做蝼蚁钻入柜内!现了本相与唐僧守困不题。
	
	却说那皇宫内院宫娥彩女天不亮起来梳洗一个个都没了头。穿宫的大小太监也都没了头一拥齐来到于寝宫外奏乐惊寝个个噙泪不敢传言。少时那三宫皇后醒来也没了头忙移灯到龙床下看处锦被窝中睡着一个和尚皇后忍不住言语出来惊醒国王。那国王急睁睛见皇后的头光他连忙爬起来道:“梓童你如何这等?”皇后道:“主公亦如此也。”那皇帝摸摸头唬得三尸呻咋七魄飞空道:“朕当怎的来耶!”正慌忙处只见那六院嫔妃宫娥彩女大小太监皆光着头跪下道:“主公我们做了和尚耶!”国王见了眼中流泪道:“想是寡人杀害和尚。”即传旨吩咐:“汝等不得说出落之事恐文武群臣褒贬国家不正且都上殿设朝。”
	
	却说那五府六部合衙门大小官员天不明都要去朝王拜阙。原来这半夜一个个也没了头各人都写表启奏此事。只听那:静鞭三响朝皇帝表奏当今剃因。毕竟不知那总兵官夺下柜里贼赃如何与唐僧四众的性命如何且听下回分解。
	------------
	
	第八十五回 心猿妒木母 魔主计吞禅
	
	话说那国王早朝文武多官俱执表章启奏道:“主公望赦臣等失仪之罪。”国王道:“众卿礼貌如常有何失仪?”众卿道:
	
	“主公啊不知何故臣等一夜把头都没了。”国王执了这没头之表下龙床对群臣道:“果然不知何故朕宫中大小人等一夜也尽没了头。”君臣们都各汪汪滴泪道:“从此后再不敢杀戮和尚也。”王复上龙位众官各立本班。王又道:“有事出班来奏无事卷帘散朝。”只见那武班中闪出巡城总兵官文班中走出东城兵马使当阶叩头道:“臣蒙圣旨巡城夜来获得贼赃一柜白马一匹。微臣不敢擅专请旨定夺。”国王大喜道:
	
	“连柜取来。”二臣即退至本衙点起齐整军士将柜抬出。三藏在内魂不附体道:“徒弟们这一到国王前如何理说?”行者笑道:“莫嚷!我已打点停当了。开柜时他就拜我们为师哩只教八戒不要争竞长短。”八戒道:“但只免杀就是无量之福还敢争竞哩!”说不了抬至朝外入五凤楼放在丹墀之下。二臣请国王开看国王即命打开。方揭了盖猪八戒就忍不住往外一跳唬得那多官胆战口不能言又见孙行者搀出唐僧沙和尚搬出行李。八戒见总兵官牵着马走上前咄的一声道:
	
	“马是我的!拿过来!”吓得那官儿翻跟头跌倒在地。四众俱立在阶中。那国王看见是四个和尚忙下龙床宣召三宫妃后下金銮宝殿同群臣拜问道:“长老何来?”三藏道:“是东土大唐驾下差往西方天竺国大雷音寺拜活佛取真经的。”国王道:
	
	“老师远来为何在这柜里安歇?”三藏道:“贫僧知陛下有愿心杀和尚不敢明投上国扮俗人夜至宝方饭店里借宿。因怕人识破原身故此在柜中安歇。不幸被贼偷出被总兵捉获抬来今得见陛下龙颜所谓拨云见日。望陛下赦放贫僧海深恩便也!”国王道:“老师是天朝上国高僧朕失迎迓。朕常年有愿杀僧者曾因僧谤了朕朕许天愿要杀一万和尚做圆满。不期今夜皈依教朕等为僧。如今君臣后妃都剃落了望老师勿吝高贤愿为门下。”八戒听言呵呵大笑道:“既要拜为门徒有何贽见之礼?”国王道:“师若肯从愿将国中财宝献上。”行者道:“莫说财宝我和尚是有道之僧。你只把关文倒换了送我们出城保你皇图永固福寿长臻。”那国王听说即着光禄寺大排筵宴君臣合同拜归于一即时倒换关文求三藏改换国号。行者道:“陛下法国之名甚好但只灭字不通自经我过可改号钦法国管教你海晏河清千代胜风调雨顺万方安。”国王谢了恩摆整朝銮驾送唐僧四众出城西去。君臣们秉善归真不题。
	
	却说长老辞别了钦法国王在马上欣然道:“悟空此一法甚善大有功也。”沙僧道:“哥啊是那里寻这许多整容匠连夜剃这许多头?”行者把那施变化弄神通的事说了一遍师徒们都笑不合口。正欢喜处忽见一座高山阻路唐僧勒马道:
	
	“徒弟们你看这面前山势崔巍切须仔细!”行者笑道:“放心!
	
	放心!保你无事!”三藏道:“休言无事。我见那山峰挺立远远的有些凶气暴云飞出渐觉惊煌满身麻木神思不安。”行者笑道:“你把乌巢禅师的《多心经》早已忘了?”三藏道:“我记得。”行者道:“你虽记得还有四句颂子你却忘了哩。”三藏道:“那四句?”行者道:“佛在灵山莫远求灵山只在汝心头。人人有个灵山塔好向灵山塔下修。”三藏道:“徒弟我岂不知?
	
	若依此四句千经万典也只是修心。”行者道:“不消说了心净孤明独照心存万境皆清。差错些儿成惰懈千年万载不成功。但要一片志诚雷音只在眼下。似你这般恐惧惊惶神思不安大道远矣雷音亦远矣。且莫胡疑随我去。”那长老闻言心神顿爽万虑皆休。
	
	四众一同前进。不几步到于山上举目看时:那山真好山细看色班班。顶上云飘荡崖前树影寒。飞禽淅沥走兽凶顽。林内松千干峦头竹几竿。吼叫是苍狼夺食咆哮是饿虎争餐。野猿长啸寻鲜果麋鹿攀花上翠岚。风洒洒水潺潺时闻幽鸟语间关。几处藤萝牵又扯满溪瑶草杂香兰。磷磷怪石削削峰岩。狐狢成群走猴猿作队顽。行客正愁多险峻奈何古道又湾还!师徒们怯怯惊惊正行之时只听得呼呼一阵风起。三藏害怕道:“风起了!”行者道:“春有和风夏有熏风秋有金风冬有朔风:四时皆有风风起怕怎的?”三藏道:“这风来得甚急决然不是天风。”行者道:“自古来风从地起云自山出怎么得个天风?”说不了又见一阵雾起。那雾真个是:漠漠连天暗蒙蒙匝地昏。日色全无影鸟声无处闻。宛然如混沌仿佛似飞尘。不见山头树那逢采药人?三藏一心惊道:
	
	“悟空风还未定如何又这般雾起?”行者道:“且莫忙请师父下马你兄弟二个在此保守等我去看看是何吉凶。”
	
	好大圣把腰一躬就到半空用手搭在眉上圆睁火眼向下观之果见那悬岩边坐着一个妖精。你看他怎生模样:炳炳文斑多采艳昂昂雄势甚抖擞。坚牙出口如钢钻利爪藏蹄似玉钩。金眼圆睛禽兽怕银须倒竖鬼神愁。张狂哮吼施威猛嗳雾喷风运智谋。又见那左右手下有三四十个小妖摆列他在那里逼法的喷风嗳雾。行者暗笑道:“我师父也有些儿先兆。他说不是天风果然不是却是个妖精在这里弄喧儿哩。若老孙使铁棒往下就打这叫做捣蒜打打便打死了只是坏了老孙的名头。”那行者一生豪杰再不晓得暗算计人。他道:“我且回去照顾猪八戒照顾教他来先与这妖精见一仗。若是八戒有本事打倒这妖算他一功;若无手段被这妖拿去等我再去救他才好出名。他想道八戒有些躲懒不肯出头却只是有些口紧好吃东西。等我哄他一哄看他怎么说。”即时落下云头到三藏前。三藏问道:“悟空风雾处吉凶何如?”行者道:
	
	“这会子明净了没甚风雾。”三藏道:“正是觉到退下些去了。”行者笑道:“师父我常时间还看得好这番却看错了。我只说风雾之中恐有妖怪原来不是。”三藏道:“是甚么?”行者道:“前面不远乃是一庄村。村上人家好善蒸的白米干饭白面馍馍斋僧哩。这些雾想是那些人家蒸笼之气也是积善之应。”八戒听说认了真实扯过行者悄悄的道:“哥哥你先吃了他的斋来的?”行者道:“吃不多儿因那菜蔬太咸酌了些不喜多吃。”八戒道:“啐!凭他怎么咸我也尽肚吃他一饱!十分作渴便回来吃水。”行者道:“你要吃么?”八戒道:“正是我肚里有些饥了先要去吃些儿不知如何?”行者道:“兄弟莫题古书云父在子不得自专。师父又在此谁敢先去?”八戒笑道:“你若不言语我就去了。”行者道:“我不言语看你怎么得去。”那呆子吃嘴的见识偏有走上前唱个大喏道:“师父适才师兄说前村里有人家斋僧。你看这马有些要打搅人家便要草要料却不费事?幸如今风雾明净你们且略坐坐等我去寻些嫩草儿先喂喂马然后再往那家子化斋去罢。”唐僧欢喜道:“好啊!你今日却怎肯这等勤谨?快去快来。”那呆子暗暗笑着便走行者赶上扯住道:“兄弟他那里斋僧只斋俊的不斋丑的。”八戒道:“这等说又要变化是。”行者道:“正是你变变儿去。”好呆子他也有三十六般变化走到山凹里捻着诀念动咒语摇身一变变做个矮胖和尚手里敲个木鱼口里哼阿哼的又不会念经只哼的是“上大人”。
	
	却说那怪物收风敛雾号令群妖在于大路口上摆开一个圈子阵专等行客。这呆子晦气不多时撞到当中被群妖围住这个扯住衣服那个扯着丝绦推推拥拥一齐下手。八戒道:“不要扯等我一家家吃将来。”群妖道:“和尚你要吃甚的?”八戒道:“你们这里斋僧我来吃斋的。”群妖道:“你想这里斋僧不知我这里专要吃僧。我们都是山中得道的妖仙专要把你们和尚拿到家里上蒸笼蒸熟吃哩你倒还想来吃斋!”
	
	八戒闻言心中害怕才报怨行者道:“这个弼马温其实惫懒!
	
	他哄我说是这村里斋僧这里那得村庄人家那里斋甚么僧却原来是些妖精!”那呆子被他扯急了即便现出原身腰间掣钉钯一顿乱筑筑退那些小妖。小妖急跑去报与老怪道:“大王祸事了!”老修道:“有甚祸事?”小妖道:“山前来了一个和尚且是生得干净。我说拿家来蒸他吃若吃不了留些儿防天阴不想他会变化。”老妖道:“变化甚的模样?”小妖道:“那里成个人相!长嘴大耳朵背后又有鬃双手轮一根钉钯没头没脸的乱筑唬得我们跑回来报大王也。”老怪道:“莫怕等我去看。”轮着一条铁杵走近前看时见呆子果然丑恶。他生得:碓嘴初长三尺零獠牙觜出赛银钉。一双圆眼光如电两耳扇风唿唿声。脑后鬃长排铁箭浑身皮糙癞还青。手中使件蹊跷物九齿钉钯个个惊。妖精硬着胆喝道:“你是那里来的叫甚名字?快早说来饶你性命!”八戒笑道:“我的儿你是也不认得你猪祖宗哩!上前来说与你听:巨口獠牙神力大玉皇升我天蓬帅。掌管天河八万兵天宫快乐多自在。只因酒醉戏宫娥那时就把英雄卖。一嘴拱倒斗牛宫吃了王母灵芝菜。玉皇亲打二千锤把吾贬下三天界。教吾立志养元神下方却又为妖怪。正在高庄喜结亲命低撞着孙兄到。金箍棒下受他降低头才把沙门拜。背马挑包做夯工前生少了唐僧债。铁脚天蓬本姓猪法名改作猪八戒。”那妖精闻言喝道:“你原来是唐僧的徒弟。我一向闻得唐僧的肉好吃正要拿你哩你却撞得来我肯饶你?不要走!看杵!”八戒道:“孽畜你原来是个染博士出身!”妖精道:“我怎么是染博士?”八戒道:“不是染博士怎么会使棒槌?”那怪那容分说近前乱打。他两个在山凹里这一场好杀:九齿钉钯一条铁棒。钯丢解数滚狂风杵运机谋飞骤雨。一个是无名恶怪阻山程一个是有罪天蓬扶性主。性正何愁怪与魔山高不得金生土。那个杵架犹如蟒出潭这个钯来却似龙离浦。喊声叱咤振山川吆喝雄威惊地府。两个英雄各逞能舍身却把神通赌。八戒长起威风与妖精厮斗那怪喝令小妖把八戒一齐围住不题。
	
	却说行者在唐僧背后忽失声冷笑。沙僧道:“哥哥冷笑何也?”行者道:“猪八戒真个呆呀!听见说斋僧就被我哄去了这早晚还不见回来。若是一顿钯打退妖精你看他得胜而回争嚷功果;若战他不过被他拿去却是我的晦气背前面后不知骂了多少弼马温哩!悟净你休言语等我去看看。”好大圣他也不使长老知道悄悄的脑后拔了一根毫毛吹口仙气叫“变!”即变做本身模样陪着沙僧随着长老。他的真身出个神跳在空中观看但见那呆子被怪围绕钉钯势乱渐渐的难敌。行者忍不住按落云头厉声高叫道:“八戒不要忙老孙来了!”那呆子听得是行者声音仗着势愈长威风一顿钯向前乱筑那妖精抵敌不住道:“这和尚先前不济这会子怎么又起狠来。”八戒道:“我的儿不可欺负我!我家里人来也!”一向前没头没脸筑去。那妖精抵架不住领群妖败阵去了。行者见妖精败去他就不曾近前拨转云头径回本处把毫毛一抖收上身来。长老的肉眼凡胎那里认得。
	
	不一时呆子得胜也自转来累得那粘涎鼻涕白沫生生气呼呼的走将来叫声“师父!”长老见了惊讶道:“八戒你去打马草的怎么这般狼狈回来?想是山上人家有人看护不容你打草么?”呆子放下钯捶胸跌脚道:“师父!莫要问!说起来就活活羞杀人!”长老道:“为甚么羞来?”八戒道:“师兄捉弄我!他先头说风雾里不是妖精没甚凶兆是一庄村人家好善蒸白米干饭、白面馍馍斋僧的我就当真想着肚里饥了先去吃些儿假倚打草为名岂知若干妖怪把我围了苦战了这一会若不是师兄的哭丧棒相助我也莫想得脱罗网回来也!”行者在旁笑道:“这呆子胡说!你若做了贼就攀上一牢人。是我在这里看着师父何曾侧离?”长老道:“是啊悟空不曾离我。”那呆子跳着嚷道:“师父!你不晓得!他有替身!”长老道:“悟空端的可有怪么?”行者瞒不过躬身笑道:“是有个把小妖儿他不敢惹我们。八戒你过来一照顾你照顾。我们既保师父走过险峻山路就似行军的一般。”八戒道:“行军便怎的?”行者道:“你做个开路将军在前剖路。那妖精不来便罢若来时你与他赌斗打倒妖精算你的功果。”八戒量着那妖精手段与他差不多却说:“我就死在他手内也罢等我先走!”行者笑道:“这呆子先说晦气话怎么得长进!”八戒道:
	
	“哥啊你知道公子登筵不醉即饱;壮士临阵不死带伤?先说句错话儿后便有威风。”行者欢喜即忙背了马请师父骑上沙僧挑着行李相随八戒一路入山不题。
	
	却说那妖精帅几个败残的小妖径回本洞高坐在那石崖上默默无言。洞中还有许多看家的小妖都上前问道:“大王常时出去喜喜欢欢回来今日如何烦恼?”老妖道:“小的们我往常出洞巡山不管那里的人与兽定捞几个来家养赡汝等今日造化低撞见一个对头。”小妖问:“是那个对头?”老妖道:“是一个和尚乃东土唐僧取经的徒弟名唤猪八戒。我被他一顿钉钯把我筑得败下阵来。好恼啊!我这一向常闻得人说唐僧乃十世修行的罗汉有人吃他一块肉可以延寿长生。
	
	不期他今日到我山里正好拿住他蒸吃不知他手下有这等徒弟!”说不了班部丛中闪上一个小妖对老妖哽哽咽咽哭了三声又嘻嘻哈哈的笑了三声。老妖喝道:“你又哭又笑何也?”
	
	小妖跪下道:“大王才说要吃唐僧唐僧的肉不中吃。”老妖道:
	
	“人都说吃他一块肉可以长生不老与天同寿怎么说他不中吃?”小妖道:“若是中吃也到不得这里别处妖精也都吃了。
	
	他手下有三个徒弟哩。”老妖道:“你知是那三个?”小妖道:“他大徒弟是孙行者三徒弟是沙和尚这个是他二徒弟猪八戒。”
	
	老妖道:“沙和尚比猪八戒如何?”小妖道:“也差不多儿。”“那个孙行者比他如何?”小妖吐舌道:“不敢说!那孙行者神通广大变化多端!他五百年前曾大闹天宫上方二十八宿、九曜星官、十二元辰、五卿四相、东西星斗、南北二神、五岳四渎、普天神将也不曾惹得他过你怎敢要吃唐僧?”老妖道:“你怎么晓得他这等详细?”小妖道:“我当初在狮驼岭狮驼洞与那大王居住那大王不知好歹要吃唐僧被孙行者使一条金箍棒打进门来可怜就打得犯了骨牌名都断么绝六还亏我有些见识从后门走了来到此处蒙大王收留故此知他手段。”老妖听言大惊失色这正是大将军怕谶语他闻得自家人这等说安得不惊?正都在悚惧之际又一个小妖上前道:“大王莫恼莫怕。常言道事从缓来若是要吃唐僧等我定个计策拿他。”老妖道:“你有何计?”小妖道:“我有个分瓣梅花计。”老妖道:“怎么叫做分瓣梅花计?”小妖道:“如今把洞中大小群妖点将起来千中选百百中选十十中只选三个须是有能干、会变化的都变做大王的模样顶大王之盔贯大王之甲执大王之杵三处埋伏。先着一个战猪八戒再着一个战孙行者再着一个战沙和尚:舍着三个小妖调开他弟兄三个大王却在半空伸下拿云手去捉这唐僧就如探囊取物就如鱼水盆内捻苍蝇有何难哉!”老妖闻此言满心欢喜道:“此计绝妙!绝妙!这一去拿不得唐僧便罢;若是拿了唐僧决不轻你就封你做个前部先锋。”小妖叩头谢恩叫点妖怪即将洞中大小妖精点起果然选出三个有能的小妖俱变做老妖各执铁杵埋伏等待唐僧不题。
	
	却说这唐长老无虑无忧相随八戒上大路行彀多时只见那路旁边扑喇的一声响喨跳出一个小妖奔向前边要捉长老。孙行者叫道:“八戒!妖精来了何不动手?”那呆子不认真假掣钉钯赶上乱筑那妖精使铁杵急架相迎。他两个一往一来的在山坡下正然赌斗又见那草科里响一声又跳出个怪来就奔唐僧。行者道:“师父!不好了!八戒的眼拙放那妖精来拿你了等老孙打他去!”急掣棒迎上前喝道:“那里去!
	
	看棒!”那妖精更不打话举杵来迎。他两个在草坡下一撞一冲正相持处又听得山背后呼的风响又跳出个妖精来径奔唐僧。沙僧见了大惊道:“师父!大哥与二哥的眼都花了把妖精放将来拿你了!你坐在马上等老沙拿他去!”这和尚也不分好歹即掣杖对面挡住那妖精铁杵恨苦相持。吆吆喝喝乱嚷乱斗渐渐的调远。那老怪在半空中见唐僧独坐马上伸下五爪钢钩把唐僧一把挝住。那师父丢了马脱了镫被妖精一阵风径摄去了。可怜!这正是禅性遭魔难正果江流又遇苦灾星!
	
	老妖按下风头把唐僧拿到洞里叫:“先锋!”那定计的小妖上前跪倒口中道:“不敢!不敢!”老妖道:“何出此言?大将军一言既出如白染皂。当时说拿不得唐僧便罢拿了唐僧封你为前部先锋。今日你果妙计成功岂可失信于你?你可把唐僧拿来着小的们挑水刷锅搬柴烧火把他蒸一蒸我和你都吃他一块肉以图延寿长生也。先锋道:“大王且不可吃。”老怪道:“既拿来怎么不可吃?”先锋道:“大王吃了他不打紧猪八戒也做得人情沙和尚也做得人情但恐孙行者那主子刮毒。他若晓得是我们吃了他也不来和我们厮打他只把那金箍棒往山腰里一搠搠个窟窿连山都掬倒了我们安身之处也无之矣!”老怪道:“先锋凭你有何高见?”先锋道:“依着我把唐僧送在后园绑在树上两三日不要与他饭吃一则图他里面干净;二则等他三人不来门前寻找打听得他们回去了我们却把他拿出来自自在在的受用却不是好?”老怪笑道:
	
	“正是正是!先锋说得有理!”一声号令把唐僧拿入后园一条绳绑在树上众小妖都去前面去听候。你看那长老苦捱着绳缠索绑紧缚牢拴止不住腮边流泪叫道:“徒弟呀!你们在那山中擒怪甚路里赶妖?我被泼魔捉来此处受灾何日相会?
	
	痛杀我也!”正自两泪交流只见对面树上有人叫道:“长老你也进来了!”长老正了性道:“你是何人?”那人道:“我是本山中的樵子被那山主前日拿来绑在此间今已三日算计要吃我哩。”长老滴泪道:“樵夫啊你死只是一身无甚挂碍我却死得不甚干净。”樵子道:“长老你是个出家人上无父母下无妻子死便死了有甚么不干净?”长老道:“我本是东土往西天取经去的奉唐朝太宗皇帝御旨拜活佛取真经要度那幽冥无主的孤魂。今若丧了性命可不盼杀那君王孤负那臣子?
	
	那枉死城中无限的冤魂却不大失所望永世不得生?一场功果尽化作风尘这却怎么得干净也?”樵子闻言眼中堕泪道:“长老你死也只如此我死又更伤情。我自幼失父与母鳏居更无家业止靠着打柴为生。老母今年八十三岁只我一人奉养。倘若身丧谁与他埋尸送老?苦哉苦哉!痛杀我也!”长老闻言放声大哭道:“可怜可怜!山人尚有思亲意空教贫僧会念经!事君事亲皆同一理。你为亲恩我为君恩。”正是那流泪眼观流泪眼断肠人送断肠人!且不言三藏身遭困苦却说孙行者在草坡下战退小妖急回来路旁边不见了师父止存白马行囊。慌得他牵马挑担向山头找寻。咦!正是那:有难的江流专遇难降魔的大圣亦遭魔。毕竟不知寻找师父下落如何且听下回分解。
	------------
	
	第八十六回 木母助威征怪物 金公施法灭妖邪
	
	话说孙大圣牵着马挑着担满山头寻叫师父忽见猪八戒气呼呼的跑将来道:“哥哥你喊怎的?”行者道:“师父不见了你可曾看见?”八戒道:“我原来只跟唐僧做和尚的你又捉弄我教做甚么将军!我舍着命与那妖精战了一会得命回来。师父是你与沙僧看着的反来问我?”行者道:“兄弟我不怪你。你不知怎么眼花了把妖精放回来拿师父。我去打那妖精教沙和尚看着师父的如今连沙和尚也不见了。”八戒笑道:“想是沙和尚带师父那里出恭去了。”说不了只见沙僧来到。行者问道:“沙僧师父那里去了?”沙僧道:“你两个眼都昏了把妖精放将来拿师父老沙去打那妖精的师父自家在马上坐来。”行者气得暴跳道:“中他计了!中他计了!”沙僧道:
	
	“中他甚么计?”行者道:“这是分瓣梅花计把我弟兄们调开他劈心里捞了师父去了。天天天!却怎么好!”止不住腮边泪滴。八戒道:“不要哭!一哭就脓包了!横竖不远只在这座山上我们寻去来。”
	
	三人没计奈何只得入山找寻行了有二十里远近只见那悬崖之下有一座洞府:削峰掩映怪石嵯峨。奇花瑶草馨香红杏碧桃艳丽。崖前古树霜皮溜雨四十围;门外苍松黛色参天二千尺。双双野鹤常来洞口舞清风;对对山禽每向枝头啼白昼。簇簇黄藤如挂索行行烟柳似垂金。方塘积水深穴依山。方塘积水隐穷鳞未变的蛟龙;深穴依山住多年吃人的老怪。果然不亚神仙境真是藏风聚气巢。行者见了两三步跳到门前看处那石门紧闭门上横安着一块石版石版上有八个大字乃隐雾山折岳连环洞。行者道:“八戒动手啊!此间乃妖精住处师父必在他家也。”那呆子仗势行凶举钉钯尽力筑将去把他那石头门筑了一个大窟窿叫道:“妖怪!快送出我师父来免得钉钯筑倒门一家子都是了帐!”守门的小妖急急跑入报道:“大王闯出祸来了!”老怪道:“有甚祸?”小妖道:“门前有人把门打破嚷道要师父哩!”老怪大惊道:“不知是那个寻将来也?”先锋道:“莫怕!等我出去看看。”那小妖奔至前门从那打破的窟窿处歪着头往外张见是个长嘴大耳朵即回头高叫:“大王莫怕他!这个是猪八戒没甚本事不敢无理。他若无理。开了门拿他进来凑蒸。怕便只怕那毛脸雷公嘴的和尚。”八戒在外边听见道:“哥啊他不怕我只怕你哩。师父定在他家了你快上前。”行者骂道:“泼孽畜!你孙外公在这里!送我师父出来饶你命罢!”先锋道:“大王不好了!
	
	孙行者也寻将来了!”老怪报怨道:“都是你定的甚么分瓣分瓣却惹得祸事临门!怎生结果?”先锋道“大王放心且休埋怨。我记得孙行者是个宽洪海量的猴头虽则他神通广大却好奉承。我们拿个假人头出去哄他一哄奉承他几句只说他师父是我们吃了。若还哄得他去了唐僧还是我们受用;哄不过再作理会。”老怪道:“那里得个假人头?”先锋道:“等我做一个儿看。”好妖怪将一把衠钢刀斧把柳树根砍做个人头模样喷上些人血糊糊涂涂的着一个小怪使漆盘儿拿至门下叫道:“大圣爷爷息怒容禀。”孙行者果好奉承听见叫声大圣爷爷便就止住八戒:“且莫动手看他有甚话说。”拿盘的小怪道:“你师父被我大王拿进洞来洞里小妖村顽不识好歹这个来吞那个来啃抓的抓咬的咬把你师父吃了只剩了一个头在这里也。”行者道:“既吃了便罢只拿出人头来我看是真是假。”那小怪从门窟里抛出那个头来猪八戒见了就哭道:
	
	“可怜啊!那们个师父进去弄做这们个师父出来也!”行者道:
	
	“呆子你且认认是真是假就哭!”八戒道:“不羞人头有个真假的?”行者道:“这是个假人头。”八戒道:“怎认得是假?”行者道:“真人头抛出来扑搭不响假人头抛得象梆子声。你不信等我抛了你听。”拿起来往石头上一掼当的一声响亮。沙和尚道:“哥哥响哩!”行者道:“响便是个假的我教他现出本相来你看。”急掣金箍棒扑的一下打破了。八戒看时乃是个柳树根。呆子忍不住骂起来道:“我把你这伙毛团!你将我师父藏在洞里拿个柳树根哄你猪祖宗莫成我师父是柳树精变的!”
	
	慌得那拿盘的小怪战兢兢跑去报道:“难难难!难难难!”
	
	老妖道:“怎么有许多难?”小妖道:“猪八戒与沙和尚倒哄过了孙行者却是个贩古董的——识货!识货!他就认得是个假人头。如今得个真人头与他或者他就去了。”老怪道:“怎么得个真人头?我们那剥皮亭内有吃不了的人头选一个来。”众妖即至亭内拣了个新鲜的头教啃净头皮滑塔塔的还使盘儿拿出叫:“大圣爷爷先前委是个假头。这个真正是唐老爷的头我大王留了镇宅子的今特献出来也。”扑通的把个人头又从门窟里抛出血滴滴的乱滚。孙行者认得是个真人头没奈何就哭八戒沙僧也一齐放声大哭。八戒噙着泪道:“哥哥且莫哭天气不是好天气恐一时弄臭了。等我拿将去乘生气埋下再哭。”行者道:“也说得是。”那呆子不嫌秽污把个头抱在怀里跑上山崖。向阳处寻了个藏风聚气的所在取钉钯筑了一个坑把头埋了又筑起一个坟冢才叫沙僧:“你与哥哥哭着等我去寻些甚么供养供养。”他就走向涧边攀几根大柳枝拾几块鹅卵石回至坟前把柳枝儿插在左右鹅卵石堆在面前。行者问道:“这是怎么说?”八戒道:“这柳枝权为松柏与师父遮遮坟顶;这石子权当点心与师父供养供养。”行者喝道:“夯货!人已死了还将石子儿供他!”八戒道:“表表生人意权为孝道心。”行者道:“且休胡弄!教沙僧在此:一则庐墓二则看守行李马匹。我和你去打破他的洞府拿住妖魔碎尸万段与师父报仇去来。”沙和尚滴泪道:“大哥言之极当。你两个着意我在此处看守。”
	
	好八戒即脱了皂锦直裰束一束着体小衣举钯随着行者。二人努力向前不容分辨径自把他石门打破喊声振天叫道:“还我活唐僧来耶!”那洞里大小群妖一个个魂飞魄散都报怨先锋的不是。老妖问先锋道:“这些和尚打进门来却怎处治?”先锋道:“古人说得好手插鱼篮避不得腥。一不做二不休左右帅领家兵杀那和尚去来!”老怪闻言无计可奈真个传令叫:“小的们各要齐心将精锐器械跟我去出征。”果然一齐呐喊杀出洞门。这大圣与八戒急退几步到那山场平处抵住群妖喝道:“那个是出名的头儿?那个是拿我师父的妖怪?”那群妖扎下营盘将一面锦绣花旗闪一闪老怪持铁杵应声高呼道:“那泼和尚你认不得我?我乃南山大王数百年放荡于此。你唐僧已是我拿吃了你敢如何?”行者骂道:“这个大胆的毛团!你能有多少的年纪敢称南山二字?李老君乃开天辟地之祖尚坐于太清之右;佛如来是治世之尊还坐于大鹏之下;孔圣人是儒教之尊亦仅呼为夫子。你这个孽畜敢称甚么南山大王数百年之放荡!不要走!吃你外公老爷一棒!”那妖精侧身闪过使杵抵住铁棒睁圆眼问道:“你这嘴脸象个猴儿模样敢将许多言语压我!你有甚么手段在吾门下猖狂?”行者笑道:“我把你个无名的孽畜!是也不知老孙!你站住硬着胆且听我说:祖居东胜大神洲天地包含几万秋。
	
	花果山头仙石卵卵开产化我根苗。生来不比凡胎类圣体原从日月俦。本性自修非小可天姿颖悟大丹头。官封大圣居云府倚势行凶斗斗牛。十万神兵难近我满天星宿易为收。名扬宇宙方方晓;智贯乾坤处处留。今幸皈依从释教扶持长老向西游。逢山开路无人阻遇水支桥有怪愁。林内施威擒虎豹崖前复手捉貔貅。东方果正来西域那个妖邪敢出头!孽畜伤师真可恨管教时下命将休!”那怪闻言又惊又恨。咬着牙跳近前来使铁杵望行者就打。行者轻轻的用棒架住还要与他讲话那八戒忍不住掣钯乱筑那怪的先锋。先锋帅众齐来。这一场在山中平地处混战真是好杀:东土大邦上国僧西方极乐取真经。南山大豹喷风雾路阻深山独显能。施巧计弄乖伶无知误捉大唐僧。相逢行者神通广更遭八戒有声名。群妖混战山平处尘土纷飞天不清。那阵上小妖呼哮枪刀乱举;
	
	这壁厢神僧叱喝钯棒齐兴。大圣英雄无敌手悟能精壮喜神生。南禺老怪部下先锋都为唐僧一块肉致令舍死又亡生。
	
	这两个因师性命成仇隙那两个为要唐僧忒恶情。往来斗经多半会冲冲撞撞没输赢。孙大圣见那些小妖勇猛连打不退。即使个分身法把毫毛拔下一把嚼在口中喷出去叫声“变!”
	
	都变做本身模样一个使一条金箍棒从前边往里打进。那一二百个小妖顾前不能顾后遮左不能遮右一个个各自逃生败走归洞。这行者与八戒从阵里往外杀来。可怜那些不识俊的妖精搪着钯九孔血出;挽着棒骨肉如泥!唬得那南山大王滚风生雾得命逃回。那先锋不能变化早被行者一棒打倒现出本相乃是个铁背苍狼怪。八戒上前扯着脚翻过来看了道“这厮从小儿也不知偷了人家多少猪牙子、羊羔儿吃了!”行者将身一抖收上毫毛道:“呆子!不可迟慢!快赶老怪讨师父的命去来!”八戒回头就不见那些小行者道:“哥哥的法相儿都去了!”行者道:“我已收来也。”八戒道:“妙啊!妙啊!”两个喜喜欢欢得胜而回。
	
	却说那老怪逃了命回洞吩咐小妖搬石块挑土把前门堵了。那些得命的小妖一个个战兢兢的把门都堵了再不敢出头。这行者引八戒赶至门吆喝内无人答应。八戒使钯筑时莫想得动。行者知之道:“八戒莫费气力他把门已堵了。”八戒道:“堵了门师仇怎报?”行者道:“且回上墓前看看沙僧去。”二人复至本处见沙僧还哭哩。八戒越伤悲丢了钯伏在坟上手扑着土哭道:“苦命的师父啊!远乡的师父啊!
	
	那里再得见你耶!”行者道:“兄弟且莫悲切。这妖精把前门堵了一定有个后门出入。你两个只在此间等我再去寻看。”八戒滴泪道:“哥啊!仔细着!莫连你也捞去了我们不好哭得:
	
	哭一声师父哭一声师兄就要哭得乱了。”行者道:“没事!我自有手段!”
	
	好大圣收了棒束束裙拽开步转过山坡忽听得潺潺水响且回头看处原来是涧中水响上溜头冲泄下来。又见涧那边有座门儿门左边有一个出水的暗沟沟中流出红水来。
	
	他道:“不消讲!那就是后门了。若要是原嘴脸恐有小妖开门看见认得等我变作个水蛇儿过去。且住!变水蛇恐师父的阴灵儿知道怪我出家人变蛇缠长变作个小螃蟹儿过去罢。也不好恐师父怪我出家人脚多。”即做一个水老鼠飕的一声撺过去从那出水的沟中钻至里面天井中。探着头儿观看只见那向阳处有几个小妖拿些人肉巴子一块块的理着晒哩。行者道:“我的儿啊!那想是师父的肉吃不了晒干巴子防天阴的。我要现本相赶上前一棍子打杀显得我有勇无谋且再变化进去寻那老怪看是何如。”跳出沟摇身又一变变做个有翅的蚂蚁儿。真个是:力微身小号玄驹日久藏修有翅飞。闲渡桥边排阵势喜来床下斗仙机。善知雨至常封穴垒积尘多遂作灰。巧巧轻轻能爽利几番不觉过柴扉。他展开翅无声无影一直飞入中堂只见那老怪烦烦恼恼正坐有一个小妖从后面跳将来报道:“大王万千之喜!”老妖道:“喜从何来?”小妖道:“我才在后门外涧头上探看忽听得有人大哭。即睮上峰头望望原来是猪八戒、孙行者、沙和尚在那里拜坟痛哭。想是把那个人头认做唐僧的头葬下睺作坟墓哭哩。”行者在暗中听说心内欢喜道:“若出此言我师父还藏在那里未曾吃哩。
	
	等我再去寻寻看死活如何再与他说话。”好大圣飞在中堂东张西看见旁边有个小门儿关得甚紧即从门缝儿里钻去看时原是个大园子隐隐的听得悲声。径飞入深处但见一丛大树树底下绑着两个人一人正是唐僧。行者见了心痒难挠忍不住现了本相近前叫声“师父。”那长老认得滴泪道:
	
	“悟空你来了?快救我一救!悟空!悟空!”行者道:“师父莫只管叫名字面前有人怕走了风讯。你既有命我可救得你。
	
	那怪只说已将你吃了拿个假人头哄我我们与他恨苦相持。
	
	师父放心且再熬熬儿等我把那妖精弄倒方好来解救。”
	
	大圣念声咒语却又摇身还变做个蚂蚁儿复入中堂丁在正梁之上。只见那些未伤命的小妖簇簇攒攒纷纷嚷嚷。内中忽跳出一个小妖告道:“大王他们见堵了门攻打不开死心塌地舍了唐僧将假人头弄做个坟墓。今日哭一日明日再哭一日后日复了三好道回去。打听得他们散了啊把唐僧拿出来碎劖碎剁把些大料煎了香喷喷的大家吃一块儿也得个延年长寿。”又一个小妖拍着手道:“莫说莫说!还是蒸了吃的有味!”又一个说:“煮了吃还省柴。”又一个道:“他本是个稀奇之物还着些盐儿腌腌吃得长久。”行者在那梁中听见心中大怒道:“我师父与你有甚毒情这般算计吃他!”即将毫毛拔了一把口中嚼碎轻轻吹出暗念咒语都教变做瞌睡虫儿往那众妖脸上抛去。一个个钻入鼻中小妖渐渐打盹不一时都睡倒了。只有那个老妖睡不稳他两只手揉头搓脸不住的打涕喷捏鼻子。行者道:“莫是他晓得了?与他个双掭灯!”
	
	又拔一根毫毛依母儿做了抛在他脸上钻于鼻孔内。两个虫儿一个从左进一个从右入。那老妖睮起来伸伸腰打两个呵欠呼呼的也睡倒了。行者暗喜才跳下来现出本相。耳朵里取出棒来幌一幌有鸭蛋粗细当的一声把旁门打破跑至后园高叫:“师父!”长老道:“徒弟快来解解绳儿绑坏我了!”行者道:“师父不要忙等我打杀妖精再来解你。”急抽身跑至中堂。正举棍要打又滞住手道:“不好!等解了师父来打。”复至园中又思量道:“等打了来救。”如此者两三番却才跳跳舞舞的到园里。长老见了悲中作喜道:“猴儿想是看见我不曾伤命所以欢喜得没是处故这等作跳舞也?”行者才至前将绳解了挽着师父就走又听得对面树上绑的人叫道:
	
	“老爷舍大慈悲也救我一命!”长老立定身叫:“悟空那个人也解他一解。”行者道:“他是甚么人?”长老道:“他比我先拿进一日。他是个樵子说有母亲年老甚是思想倒是个尽孝的一连他都救了罢。”
	
	行者依言也解了绳索一同带出后门睮上石崖过了陡涧。长老谢道:“贤徒亏你教了他与我命!悟能悟净都在何处?”行者道:“他两个都在那里哭你哩你可叫他一声。”长老果厉声高叫道:“八戒!八戒!”那呆子哭得昏头昏脑的揩揩鼻涕眼泪道:“沙和尚师父回家来显魂哩!在那里叫我们不是?”
	
	行者上前喝了一声道:“夯货!显甚么魂?这不是师父来了?”
	
	那沙僧抬头见了忙忙跪在面前道:“师父你受了多少苦啊!
	
	哥哥怎生救得你来也?”行者把上项事说了一遍。八戒闻言咬牙恨齿忍不住举起钯把那坟冢一顿筑倒掘出那人头一顿筑得稀烂。唐僧道:“你筑他为何?”八戒道“师父啊不知他是那家的亡人教我朝着他哭!”长老道:“亏他救了我命哩。你兄弟们打上他门嚷着要我想是拿他来搪塞不然啊就杀了我也。还把他埋一埋见我们出家人之意。”那呆子听长老此言遂将一包稀烂骨肉埋下也劖起个坟墓。行者却笑道:“师父你请略坐坐等我剿除去来。”即又跳下石崖过涧入洞把那绑唐僧与樵子的绳索拿入中堂那老妖还睡着了即将他四马攒蹄捆倒使金箍棒掬起来握在肩上径出后门。猪八戒远远的望见道:“哥哥好干这握头事!再寻一个儿趁头挑着不好?”
	
	行者到跟前放下八戒举钯就筑。行者道:“且住!洞里还有小妖怪未拿哩。”八戒道:“哥啊有便带我进去打他。”行者道:
	
	“打又费工夫了不若寻些柴教他断根罢。”那樵子闻言即引八戒去东凹里寻了些破梢竹、败叶松、空心柳、断根藤、黄蒿、老荻、芦苇、干桑挑了若干送入后门里。行者点上火八戒两耳扇起风。那大圣将身跳上抖一抖收了瞌睡虫的毫毛。那些小妖及醒来烟火齐着可怜!莫想有半个得命。连洞府烧得精空却回见师父。师父听见老妖方醒声唤便叫:“徒弟妖精醒了。”八戒上前一钯把老怪筑死现出本相原来是个艾叶花皮豹子精。行者道:“花皮会吃老虎如今又会变人这顿打死才绝了后患也!”长老谢之不尽攀鞍上马。那樵子道:
	
	“老爷向西南去不远就是舍下。请老爷到舍见见家母叩谢老爷活命之恩送老爷上路。”长老欣然遂不骑马与樵子并四众同行向西南迤逶前来不多路果见那:石径重漫苔藓柴门篷络藤花。四面山光连接一林鸟雀喧哗。密密松篁交翠纷纷异卉奇葩。地僻云深之处竹篱茅舍人家。远见一个老妪倚着柴扉眼泪汪汪的儿天儿地的痛哭。这樵子看见是他母亲丢了长老急忙忙先跑到柴扉前跪下叫道:“母亲!儿来也!”老妪一把抱住道:“儿啊!你这几日不来家我只说是山主拿你去害了性命是我心疼难忍。你既不曾被害何以今日才来?你绳担、柯斧俱在何处?”樵子叩头道:“母亲儿已被山主拿去绑在树上实是难得性命幸亏这几位老爷!这老爷是东土唐朝往西天取经的罗汉。那老爷倒也被山主拿去绑在树上他那三位徒弟老爷神通广大把山主一顿打死却是个艾叶花皮豹子精;概众小妖俱尽烧死却将那老老爷解下救出连孩儿都解救出来此诚天高地厚之恩!不是他们孩儿也死无疑了。如今山上太平孩儿彻夜行走也无事矣。”那老妪听言一步一拜拜接长老四众都入柴扉茅舍中坐下。娘儿两个磕头称谢不尽慌慌忙忙的安排些素斋酬谢。八戒道:“樵哥我见你府上也寒薄只可将就一饭切莫费心大摆布。”樵子道“不瞒老爷说我这山间实是寒薄没甚么香蕈、蘑菰、川椒、大料只是几品野菜奉献老爷权表寸心。”八戒笑道:“聒噪聒噪放快些儿就是我们肚中饥了。”樵子道:“就有!就有!”果然不多时展抹桌凳摆将上来果是几盘野菜。但见那:嫩焯黄花菜酸虀白鼓丁。浮蔷马齿苋江荠雁肠英。燕子不来香且嫩芽儿拳小脆还青。烂煮马蓝头白熝狗脚迹。猫耳朵野落荜灰条熟烂能中吃;剪刀股牛塘利倒灌窝螺操帚荠。碎米荠莴菜荠几品青香又滑腻。油炒乌英花菱科甚可夸;蒲根菜并茭儿菜四般近水实清华。看麦娘娇且佳;破破纳不穿他苦麻台下藩篱架。雀儿绵单猢狲脚迹油灼灼煎来只好吃。斜蒿青蒿抱娘蒿灯娥儿飞上板荞荞。羊耳秃枸杞头加上乌蓝不用油。几般野菜一餐饭樵子虔心为谢酬。
	
	师徒们饱餐一顿收拾起程。那樵子不敢久留请母亲出来再拜再谢。樵子只是磕头取了一条枣木棍结束了衣裙出门相送。沙僧牵马八戒挑担行者紧随左右长老在马上拱手道:“樵哥烦先引路到大路上相别。”一齐登高下坂转涧寻坡。长老在马上思量道:“徒弟啊!自从别主来西域递递迢迢去路遥。水水山山灾不脱妖妖怪怪命难逃。心心只为经三藏念念仍求上九霄。碌碌劳劳何日了几时行满转唐朝!”樵子闻言道:“老爷切莫忧思。这条大路向西方不满千里就是天竺国极乐之乡也。”长老闻言鄱身下马道:“有劳远涉。既是大路请樵哥回府多多拜上令堂老安人:适间厚扰盛斋贫僧无甚相谢只是早晚诵经保佑你母子平安百年长寿。”那樵子喏喏相辞复回本路师徒遂一直投西。正是:降怪解冤离苦厄受恩上路用心行。毕竟不知还有几日得到西天且听下回分解。
	------------
	
	第八十七回 凤仙郡冒天止雨 孙大圣劝善施霖
	
	大道幽深如何消息说破鬼神惊骇。挟藏宇宙剖判玄光真乐世间无赛。灵鹫峰前宝珠拈出明映五般光彩。照乾坤上下群生知者寿同山海。却说三藏师徒四众别樵子下了隐雾山奔上大路。行经数日忽见一座城池相近三藏道:“悟空你看那前面城池可是天竺国么?”行者摇手道:“不是!不是!如来处虽称极乐却没有城池乃是一座大山山中有楼台殿阁唤做灵山大雷音寺。就到了天竺国也不是如来住处天竺国还不知离灵山有多少路哩。那城想是天竺之外郡到前边方知明白。”
	
	不一时至城外三藏下马入到三层门里见那民事荒凉街衢冷落。又到市口之间见许多穿青衣者左右摆列有几个冠带者立于房檐之下。他四众顺街行走那些人更不逊避。猪八戒村愚把长嘴掬一掬叫道:“让路!让路!”那些人猛抬头看见模样一个个骨软筋麻跌跌蹡蹡都道:“妖精来了!妖精来了!”唬得那檐下冠带者战兢兢躬身问道:“那方来者?”三藏恐他们闯祸一力当先对众道:“贫僧乃东土大唐驾下拜天竺国大雷音寺佛祖求经者。路过宝方一则不知地名二则未落人家才进城甚失回避望列公恕罪。那官人却才施礼道:“此处乃天竺外郡地名凤仙郡。连年干旱郡侯差我等在此出榜招求法师祈雨救民也。”行者闻言道:“你的榜文何在?”众官道:“榜文在此适间才打扫廊檐还未张挂。”行者道:“拿来我看看。”众官即将榜文展开挂在檐下。行者四众上前同看。榜上写着:“大天竺国凤仙郡郡侯上官。为榜聘明师招求大法事。慈因郡土宽弘军民殷实连年亢旱累岁干荒民田菑而军地薄河道浅而沟浍空。井中无水泉底无津。富室聊以全生穷民难以活命。斗粟百金之价束薪五两之资。十岁女易米三升五岁男随人带去。城中惧法典衣当物以存身;乡下欺公打劫吃人而顾命。为此出给榜文仰望十方贤哲祷雨救民恩当重报。愿以千金奉谢决不虚言。须至榜者。”行者看罢对众官道:“郡侯上官何也?”众官道:“上官乃是姓此我郡侯之姓也。”行者笑道:“此姓却少。”八戒道:“哥哥不曾读书百家姓后有一句上官欧阳。”三藏道:“徒弟们且休闲讲。那个会求雨与他求一场甘雨以济民瘼此乃万善之事;如不会就行莫误了走路。”行者道:“祈雨有甚难事!我老孙翻江搅海换斗移星踢天弄井吐雾喷云担山赶月唤雨呼风那一件儿不是幼年耍子的勾当!何为稀罕!”
	
	众官听说着两个急去郡中报道:“老爷万千之喜至也!”
	
	那郡侯正焚香默祝听得报声喜至即问:“何喜?”那官道:“今日领榜方至市口张挂即有四个和尚称是东土大唐差往天竺国大雷音拜佛求经者见榜即道能祈甘雨特来报知。”那郡侯即整衣步行不用轿马多人径至市口以礼敦请。忽有人报道:“郡侯老爷来了。”众人闪过那郡侯一见唐僧不怕他徒弟丑恶当街心倒身下拜道:“下官乃凤仙郡郡侯上官氏熏沐拜请老师祈雨救民。望师大舍慈悲运神功拔济拔济!”三藏答礼道:“此间不是讲话处待贫僧到那寺观却好行事。”郡侯道:“老师同到小衙自有洁净之处”师徒们遂牵马挑担径至府中一一相见。郡侯即命看茶摆斋。少顷斋至那八戒放量吞餐如同饿虎唬得那些捧盘的心惊胆战一往一来添汤添饭就如走马灯儿一般刚刚供上直吃得饱满方休。斋毕唐僧谢了斋却问:“郡侯大人贵处干旱几时了?”郡侯道:“敝地大邦天竺国凤仙外郡吾司牧。一连三载遇干荒草子不生绝五谷。大小人家买卖难十门九户俱啼哭。三停饿死二停人一停还似风中烛。下官出榜遍求贤幸遇真僧来我国。若施寸雨济黎民愿奉千金酬厚德!”行者听说满面喜生呵呵的笑道:
	
	“莫说!莫说!若说千金为谢半点甘雨全无。但论积功累德老孙送你一场大雨。”那郡侯原来十分清正贤良爱民心重即请行者上坐低头下拜道:“老师果舍慈悲下官必不敢悖德。”
	
	行者道:“且莫讲话请起。但烦你好生看着我师父等老孙行事。”沙僧道:“哥哥怎么行事?”行者道:“你和八戒过来就在他这堂下随着我做个羽翼等老孙唤龙来行雨。”八戒、沙僧谨依使令三个人都在堂下郡侯焚香礼拜三藏坐着念经。
	
	行者念动真言诵动咒语即时见正东上一朵乌云渐渐落至堂前乃是东海老龙王敖广。那敖广收了云脚化作人形走向前对行者躬身施礼道:“大圣唤小龙来那方使用?”行者道:“请起累你远来别无甚事。此间乃凤仙郡连年干旱问你如何不来下雨?”老龙道:“启上大圣得知我虽能行雨乃上天遣用之辈。上天不差岂敢擅自来此行雨?”行者道:“我因路过此方见久旱民苦特着你来此施雨救济如何推托?”龙王道:“岂敢推托?但大圣念真言呼唤不敢不来。一则未奉上天御旨二则未曾带得行雨神将怎么动得雨部?大圣既有拔济之心容小龙回海点兵烦大圣到天宫奏准请一道降雨的圣旨请水官放出龙来我却好照旨意数目下雨。”行者见他说出理来只得放老龙回海。他即跳出罡斗对唐僧备言龙王之事唐僧道:“既然如此你去为之切莫打诳语。”行者即吩咐八戒沙僧:“保着师父我上天宫去也。”好大圣说声去寂然不见。那郡侯胆战心惊道:“孙老爷那里去了?”八戒笑道:“驾云上天去了。”郡侯十分恭敬传出飞报教满城大街小巷不拘公卿士庶军民人等家家供养龙王牌位门设清水缸缸插杨柳枝侍奉香火拜天不题。
	
	却说行者一路筋斗云径到西天门外早见护国天王引天丁力士上前迎接道:“大圣取经之事完乎?”行者道:“也差不远矣。今行至天竺国界有一外郡名凤仙郡。彼处三年不雨民甚艰苦老孙欲祈雨拯救呼得龙王到彼他言无旨不敢私自为之特来朝见玉帝请旨。”天王道:“那壁厢敢是不该下雨哩。我向时闻得说那郡侯撒泼冒犯天地上帝见罪立有米山、面山、黄金大锁直等此三事倒断才该下雨。”行者不知此意是何要见玉帝。天王不敢拦阻让他进去径至通明殿外又见四大天师迎道:“大圣到此何干?”行者道:“因保唐僧路至天竺国界凤仙郡无雨郡侯召师祈雨。老孙呼得龙王意命降雨他说未奉玉帝旨意不敢擅行特来求旨以苏民困。”四大天师道:“那方不该下雨。”行者笑道:“该与不该烦为引奏引奏看老孙的人情何如。”葛仙翁道:“俗语云苍蝇包网儿好大面皮!”许旌阳道:“不要乱谈且只带他进去。”邱洪济、张道陵与葛、许四真人引至灵霄殿下启奏道:“万岁有孙悟空路至天竺国凤仙郡欲与求雨特来请旨。”玉帝道:“那厮三年前十二月二十五日朕出行监观万天浮游三界驾至他方见那上官正不仁将斋天素供推倒喂狗口出秽言造有冒犯之罪朕即立以三事在于披香殿内。汝等引孙悟空去看若三事倒断即降旨与他;如不倒断且休管闲事。”四天师即引行者至披香殿里看时见有一座米山约有十丈高下;一座面山约有二十丈高下。米山边有一只拳大之鸡在那里紧一嘴慢一嘴嗛那米吃。面山边有一只金毛哈巴狗儿在那里长一舌短一舌餂那面吃。左边悬一座铁架子架上挂一把金锁约有一尺三四寸长短锁梃有指头粗细下面有一盏明灯灯焰儿燎着那锁梃。行者不知其意回头问天师曰:“此何意也?”天师道:“那厮触犯了上天玉帝立此三事直等鸡嗛了米尽狗餂得面尽灯焰燎断锁梃那方才该下雨哩。”行者闻言大惊失色再不敢启奏走出殿满面含羞。四大天师笑道:“大圣不必烦恼这事只宜作善可解。若有一念善慈惊动上天那米、面山即时就倒锁梃即时就断。你去劝他归善福自来矣。”行者依言不上灵霄辞玉帝径来下界复凡夫。须臾到西天门又见护国天王天王道:“请旨如何?”行者将米山、面山、金锁之事说了一遍道:“果依你言不肯传旨。适间天师送我教劝那厮归善即福原也。”遂相别降云下界。
	
	全文字版小说阅读更新更快尽在支持文学支持!那郡侯同三藏、八戒、沙僧、大小官员人等接着都簇簇攒攒来问。行者将郡侯喝了一声道:“只因你这斯三年前十二月二十五日冒犯了天地致令黎民有难如今不肯降雨!”郡侯慌得跪伏在地道:“老师如何得知三年前事?”行者道:“你把那斋天的素供怎么推倒喂狗?可实实说来!”那郡侯不敢隐瞒道:
	
	“三年前十二月二十五日献供斋天在于本衙之内因妻不贤恶言相斗一时怒无知推倒供桌泼了素馔果是唤狗来吃了。这两年忆念在心神思恍惚无处可以解释不知上天见罪遗害黎民。今遇老师降临万望明示上界怎么样计较。”
	
	行者道:“那一日正是玉皇下界之日见你将斋供喂狗又口出秽言玉帝即立三事记汝。”八戒问道:“哥是那三事?”行者道:“披香殿立一座米山约有十丈高下;一座面山约有二十丈高下。米山边有拳大的一只小鸡在那里紧一嘴慢一嘴的嗛那米吃;面山边有一个金毛哈巴狗儿在那里长一舌短一舌的餂那面吃。左边又一座铁架子架上挂一把黄金大锁锁梃儿有指头粗细下面有一盏明灯灯焰儿燎着那锁梃。直等那鸡嗛米尽狗餂面尽灯燎断锁梃他这里方才该下雨哩。”
	
	八戒笑道:“不打紧!不打紧!哥肯带我去变出法身来一顿把他的米面都吃了锁梃弄断了管取下雨。”行者道:“呆子莫胡说!此乃上天所设之计你怎么得见?”三藏道:“似这等说怎生是好?”行者道:“不难!不难!我临行时四天师曾对我言但只作善可解。”那郡侯拜伏在地哀告道:“但凭老师指教下官一一皈依也。”行者道:“你若回心向善趁早儿念佛看经我还替你作为;汝若仍前不改我亦不能解释不久天即诛之性命不能保矣。”那郡侯磕头礼拜誓愿皈依。当时召请本处僧道启建道场各各写文书申奏三天。郡侯领众拈香瞻拜答天谢地引罪自责三藏也与他念经。一壁厢又出飞报教城里城外大家小户不论男女人等都要烧香念佛。自此时一片善声盈耳。行者却才欢喜对八戒沙僧道:“你两个好生护持师父等老孙再与他去去来。”八戒道:“哥哥又往那里去?”行者道:“这郡侯听信老孙之言果然受教恭敬善慈诚心念佛我这去再奏玉帝求些雨来。”沙僧道:“哥哥既要去不必迟疑且耽搁我们行路必求雨一坛庶成我们之正果也。”
	
	好大圣又纵云头直至天门外还遇着护国天王。天王道:“你今又来做甚?”行者道:“那郡侯已归善矣。”天王亦喜。
	
	正说处早见直符使者捧定了道家文书僧家关牒到天门外传递。那符使见了行者施礼道:“此意乃大圣劝善之功。”行者道:“你将此文牒送去何处?”符使道:“直送至通明殿上与天师传递到玉皇大天尊前。”行者道:“如此你先行我当随后而去。”那符使入天门去了。”护国天王道:“大圣不消见玉帝了。
	
	你只往九天应元府下借点雷神径自声雷掣电还他就有雨下也。”真个行者依言入天门里不上灵霄殿求请旨意转云步径往九天应元府见那雷门使者、纠录典者、廉访典者都来迎着施礼道:“大圣何来?”行者道:“有事要见天尊。”三使者即为传奏天尊随下九凤丹霞之扆整衣出迎。相见礼毕行者道:“有一事特来奉求。”天尊道:“何事?”行者道:“我因保唐僧至凤仙郡见那干旱之甚已许他求雨特来告借贵部官将到彼声雷。”天尊道:“我知那郡侯冒犯上天立有三事不知可该下雨哩。”行者笑道:“我昨日已见玉帝请旨。玉帝着天师引我去披香殿看那三事乃是米山、面山、金锁只要三事倒断方该下雨。我愁难得倒断天师教我劝化郡侯等众作善以为人有善念天必从之庶几可以回天心解灾难也。今已善念顿生善声盈耳。适间直符使者已将改行从善的文牒奏上玉帝去了老孙因特造尊府告借雷部官将相助相助。”天尊道:“既如此差邓辛张陶帅领闪电娘子即随大圣下降凤仙郡声雷。”
	
	那四将同大圣不多时至于凤仙境界即于半空中作起法来。只听得唿鲁鲁的雷声又见那淅沥沥的闪电真个是:电掣紫金蛇雷轰群蛰哄。荧煌飞火光霹雳崩山洞。列缺满天明震惊连地纵。红销一闪萌芽万里江山都撼动。那凤仙郡城里城外大小官员军民人等整三年不曾听见雷电今日见有雷声霍闪一齐跪下头顶着香炉有的手拈着柳枝都念:
	
	“南无阿弥陀佛!南无阿弥陀佛!”这一声善念果然惊动上天正是那古诗云:“人心生一念天地悉皆知善恶若无报乾坤必有私。”
	
	且不说孙大圣指挥雷将掣电轰雷于凤仙郡人人归善。
	
	却说那上界直符使者将僧道两家的文牒送至通明殿四天师传奏灵霄殿。玉帝见了道:“那厮们既有善念看三事如何。”
	
	正说处忽有披香殿看管的将官报道:“所立米、面山俱倒了霎时间米面皆无锁梃亦断。”奏未毕又有当驾天官引凤仙郡土地、城隍、社令等神齐来拜奏道:“本郡郡主并满城大小黎庶之家无一家一人不皈依善果礼佛敬天。今启垂慈普降甘雨救济黎民。”玉帝闻言大喜即传旨:“着风部、云部、雨部各遵号令去下方按凤仙郡界即于今日今时声雷布云降雨三尺零四十二点。”时有四大天师奉旨传与各部随时下界各逞神威一齐振作。
	
	行者正与邓辛张陶令闪电娘子在空中调弄只见众神都到合会一天。那其间风云际会甘雨滂沱好雨:漠漠浓云蒙蒙黑雾。雷车轰轰闪电灼灼。滚滚狂风淙淙骤雨。所谓一念回天万民满望。全亏大圣施元运万里江山处处阴。好雨倾河倒海蔽野迷空。檐前垂瀑布窗外响玲珑。万户千门人念佛六街三市水流洪。东西河道条条满南北溪湾处处通。槁苗得润枯木回生。田畴麻麦盛村堡豆粮升。客旅喜通贩卖农夫爱尔耘耕。从今黍稷多条畅自然稼穑得丰登。风调雨顺民安乐海晏河清享太平。一日雨下足了三尺零四十二点众神祇渐渐收回。孙大圣厉声高叫道:“那四部众神且暂停云从待老孙去叫郡侯拜谢列位。列位可拨开云雾各现真身与这凡夫亲眼看看他才信心供奉也。”众神听说只得都停在空中。这行者按落云头径至郡里早见三藏、八戒、沙僧都来迎接那郡侯一步一拜来谢。行者道:“且慢谢我我已留住四部神祇你可传召多人同此拜谢。教他向后好来降雨。”郡侯随传飞报召众同酬都一个个拈香朝拜只见那四部神祇开明云雾各现真身。四部者乃雨部、雷部、云部、风部只见那龙王显象雷将舒身。云童出现风伯垂真。龙王显象银须苍貌世无双。雷将舒身钩嘴威颜诚莫比。云童出现谁如玉面金冠;
	
	风伯垂真曾似燥眉环眼。齐齐显露青霄上各各挨排观圣仪。
	
	凤仙郡界人才信顶礼拈香恶性回。今日仰朝天上将洗心向善尽皈依。众神祇宁待了一个时辰人民拜之不已。孙行者又起在云端对众作礼道:“有劳!有劳!请列位各归本部。老孙还教郡界中人家供养高真遇时节醮谢。列位从此后五日一风十日一雨还来拯救拯救。”众神依言各各转部不题。
	
	却说大圣坠落云头与三藏道:“事毕民安可收拾走路矣。”那郡侯闻言急忙行礼道:“孙老爷说那里话!今此一场乃无量无边之恩德。下官这里差人办备小宴奉答厚恩。仍买治民间田地与老爷起建寺院立老爷生祠勒碑刻名四时享祀。虽刻骨镂心难报万一怎么就说走路的话!”三藏道:“大人之言虽当但我等乃西方挂搭行脚之僧不敢久住。一二日间定走无疑。”那郡侯那里肯放连夜差多人治办酒席起盖祠宇。
	
	次日大开佳宴请唐僧高坐孙大圣与八戒沙僧列坐郡侯同本郡大小官员部臣把杯献馔细吹细打款待了一日。这场果是欣然有诗为证:田畴久旱逢甘雨河道经商处处通。深感神僧来郡界多蒙大圣上天宫。解除三事从前恶一念皈依善果弘。此后愿如尧舜世五风十雨万年丰。
	
	一日筵二日宴今日酬明日谢扳留将有半月只等寺院生祠完备。一日郡侯请四众往观唐僧惊讶道:“工程浩大何成之如此耶?”郡侯道:“下官催趱人工昼夜不息急急命完特请列位老爷看看。”行者笑道:“果是贤才能干的好贤侯也!”即时都到新寺见那殿阁巍峨山门壮丽俱称赞不已。行者请师父留一寺名三藏道:“有留名当唤做甘霖普济寺。”郡侯称道:“甚好!甚好!”用金贴广招僧众侍奉香火。殿左边立起四众生祠每年四时祭祀;又起盖雷神、龙神等庙以答神功。看毕即命趱行。那一郡人民知久留不住各备赆仪分文不受。因此合郡官员人等盛张鼓乐大展旌幢送有三十里远近犹不忍别遂掩泪目送直至望不见方回。这正是:硕德神僧留普济齐天大圣广施恩。毕竟不知此去还有几日方见如来且听下回分解。
	------------
	
	第八十八回 禅到玉华施法会 心猿木母授门人
	
	话说唐僧喜喜欢欢别了郡侯在马上向行者道:“贤徒这一场善果真胜似比丘国搭救儿童皆尔之功也。”沙僧道:“比丘国只救得一千一百一十一个小儿怎似这场大雨滂沱浸润活彀者万万千千性命!弟子也暗自称赞大师兄的法力通天慈恩盖地也。”八戒笑道:“哥的恩也有善也有却只是外施仁义内包祸心。但与老猪走就要作践人。”行者道:“我在那里作践你?”八戒道:“也彀了!也彀了!常照顾我捆照顾我吊照顾我煮照顾我蒸!今在凤仙郡施了恩惠与万万之人就该住上半年带挈我吃几顿自在饱饭却只管催趱行路!”长老闻言喝道:“这个呆子怎么只思量掳嘴!快走路再莫斗口!”
	
	八戒不敢言掬掬嘴挑着行囊打着哈哈师徒们奔上大路。
	
	此时光景如梭又值深秋之候但见:水痕收山骨瘦。红叶纷飞黄花时候。霜晴觉夜长月白穿窗透。家家烟火夕阳多处处湖光寒水溜。白蘋香红蓼茂。桔绿橙黄柳衰谷秀。荒村雁落碎芦花野店鸡声收菽豆。四众行彀多时又见城垣影影长老举鞭遥指叫:“悟空你看那里又有一座城池却不知是甚去处。”行者道:“你我俱未曾到何以知之?且行至边前问人。”
	
	说不了忽见树丛里走出一个老者手持竹杖身着轻衣足踏一对棕鞋腰束一条扁带慌得唐僧滚鞍下马上前道个问讯。
	
	那老者扶杖还礼道:“长老那方来的?”唐僧合掌道:“贫僧东土唐朝差往雷音拜佛求经者今至宝方遥望城垣不知是甚去处特问老施主指教。”那老者闻言口称:“有道禅师我这敝处乃天竺国下郡地名玉华县。县中城主就是天竺皇帝之宗室封为玉华王。此王甚贤专敬僧道重爱黎民。老禅师若去相见必有重敬。”三藏谢了那老者径穿树林而去。
	
	三藏才转身对徒弟备言前事。他三人欣喜扶师父上马。
	
	三藏道:“没多路不须乘马。”四众遂步至城边街道观看。原来那关厢人家做买做卖的人烟凑集生意亦甚茂盛。观其声音相貌与中华无异。三藏吩咐:“徒弟们谨慎切不可放肆。那八戒低了头沙僧掩着脸惟孙行者搀着师父。两边人都来争看齐声叫道:“我这里只有降龙伏虎的高僧不曾见降猪伏猴的和尚。”八戒忍不住把嘴一掬道:“你们可曾看见降猪王的和尚。”唬得满街上人跌跌睮睮都往两边闪过。行者笑道:“呆子快藏了嘴莫装扮仔细脚下过桥。”那呆子低着头只是笑。过了吊桥入城门内又见那大街上酒楼歌馆热闹繁华果然是神州都邑。有诗为证诗曰:锦城铁瓮万年坚临水依山色色鲜。百货通湖船入市千家沽酒店垂帘。楼台处处人烟广巷陌朝朝客贾喧。不亚长安风景好鸡鸣犬吠亦般般。三藏心中暗喜道:“人言西域诸番更不曾到此。细观此景与我大唐何异!所为极乐世界诚此之谓也。”又听得人说白米四钱一石麻油八厘一斤真是五谷丰登之处。行彀多时方到玉华王府府门左右有长史府、审理厅、典膳所、待客馆。三藏道:“徒弟此间是府等我进去朝王验牒而行。”八戒道:“师父进去我们可好在衙门前站立?”三藏道:“你不看这门上是待客馆三字!你们都去那里坐下看有草料买些喂马。我见了王倘或赐斋便来唤你等同享。”行者道:“师父放心前去老孙自当理会。”那沙僧把行李挑至馆中。馆中有看馆的人役见他们面貌丑陋也不敢问他也不敢教他出去只得让他坐下不题。
	
	却说老师父换了衣帽拿了关文径至王府前早见引礼官迎着问道:“长老何来?”三藏道:“东土大唐差来大雷音拜佛祖求经之僧今到贵地欲倒换关文特来朝参千岁。”引礼官即为传奏那王子果然贤达即传旨召进。三藏至殿下施礼王子即请上殿赐坐。三藏将关文献上王子看了又见有各国印信手押也就欣然将宝印了押了花字收折在案。问道:“国师长老自你那大唐至此历遍诸邦共有几多路程?”三藏道:
	
	“贫僧也未记程途。但先年蒙观音菩萨在我王御前显身曾留了颂子言西方十万八千里。贫僧在路已经过一十四遍寒暑矣。”王子笑道:“十四遍寒暑即十四年了。想是途中有甚耽搁。”三藏道:“一言难尽!万蛰千魔也不知受了多少苦楚才到得宝方!”那王子十分欢喜。即着典膳官备素斋管待。三藏:
	
	“启上殿下贫僧有三个小徒在外等候不敢领斋但恐迟误行程。”王子教:“当殿官快去请长老三位徒弟进府同斋。”当殿官随出外相请都道:“未曾见未曾见。”有跟随的人道:“待客馆中坐着三个丑貌和尚想必是也。”当殿官同众至馆中即问看馆的道:“那个是大唐取经僧的高徒?我主有旨请吃斋也。”八戒正坐打盹听见一个斋字忍不住跳起身来答道:“我们是!我们是!”当殿官一见了魂飞魄丧都战战的道:“是个猪魈!猪魈!”行者听见一把扯住八戒道:“兄弟放斯文些莫撒村野。”那众官见了行者又道:“是个猴精!猴精!”沙僧拱手道:“列位休得惊恐。我三人都是唐僧的徒弟。”众官见了又道:“灶君!灶君!”孙行者即教八戒牵马沙僧挑担同众入玉华王府。当殿官先入启知那王子举目见那等丑恶却也心中害怕。三藏合掌道:“千岁放心顽徒虽是貌丑却都心良。”八戒朝上唱个喏道:“贫僧问讯了。”王子愈觉心惊。三藏道:“顽徒都是山野中收来的不会行礼万望赦罪。”王子奈着惊恐教典膳官请众僧官去暴纱亭吃斋三藏谢了恩辞王下殿同至亭内埋怨八戒道:“你这夯货全不知一毫礼体!索性不开口便也罢了怎么那般粗鲁!一句话足足冲倒泰山!”行者笑道:“还是我不唱喏的好也省些力气。”沙僧道:“他唱喏又不等齐预先就抒着个嘴吆喝。”八戒道:“活淘气!活淘气!师父前日教我见人打个问讯儿是礼。今日打问讯又说不好教我怎的干么!”三藏道:“我教你见了人打个问讯不曾教你见王子就此歪缠!常言道物有几等物人有几等人如何不分个贵贱?”正说处见那典膳官带领人役调开桌椅摆上斋来师徒们却不言语各各吃斋。
	
	却说那王子退殿进宫宫中有三个小王子见他面容改色即问道:“父王今日为何有此惊恐?”王子道:“适才有东土大唐差来拜佛取经的一个和尚倒换关文却一表非凡。我留他吃斋他说有徒弟在府前我即命请。少时进来见我不行大礼打个问讯我已不快。及抬头看时一个个丑似妖魔心中不觉惊骇故此面容改色。”原来那三个小王子比众不同一个个好武好强便就伸拳掳袖道:“莫敢是那山里走来的妖精假装人象待我们拿兵器出去看来!”好王子大的个拿一条齐眉棍第二个轮一把九齿钯第三个使一根乌油黑棒子雄纠纠、气昂昂的走出王府吆喝道:“甚么取经的和尚!在那里?”时有典膳官员人等跪下道:“小王他们在这暴纱亭吃斋哩。”小王子不分好歹闯将进去喝道:“汝等是人是怪快早说来饶你性命!”唬得三藏面容失色丢下饭碗躬着身道:“贫僧乃唐朝来取经者人也非怪也。”小王子道:“你便还象个人那三个丑的断然是怪!”八戒只管吃饭不睬。沙僧与行者欠身道:“我等俱是人面虽丑而心良身虽夯而性善。汝三个却是何来却这样海口轻狂?”旁有典膳等官道:“三位是我王之子小殿下。”
	
	八戒丢了碗道:“小殿下各拿兵器怎么?莫是要与我们打哩?”
	
	二王子掣开步双手舞钯便要打八戒。八戒嘻嘻笑道:“你那钯只好与我这钯做孙子罢了!”即揭衣腰间取出钯来幌一幌金光万道丢了解数有瑞气千条把个王子唬得手软筋麻不敢舞弄。行者见大的个使一条齐眉棍跳阿跳的即耳朵里取出金箍棒来幌一幌碗来粗细有丈二三长短着地下一捣捣了有三尺深浅竖在那里笑道:“我把这棍子送你罢!”
	
	那王子听言即丢了自己棍去取那棒双手尽气力一拔莫想得动分毫再又端一端摇一摇就如生根一般。第三个撒起莽性使乌油杆棒来打被沙僧一手劈开取出降妖宝杖拈一拈艳艳光生纷纷霞亮唬得那典膳等官一个个呆呆挣挣口不能言。三个小王子一齐下拜道:“神师!神师!我等凡人不识万望施展一番我等好拜授也。”行者走近前轻轻的把棒拿将起来道:“这里窄狭不好展手等我跳在空中耍一路儿你们看看。”好大圣唿哨一声将筋斗一纵两只脚踏着五色祥云起在半空离地约有三百步高下把金箍棒丢开个撒花盖顶黄龙转身一上一下左旋右转。起初时人与棒似锦上添花次后来不见人只见一天棒滚。八戒在底下喝声采也忍不住手脚厉声喊道:“等老猪也去耍耍来!”好呆子驾起风头也到半空丢开钯上三下四左五右六前七后八满身解数只听得呼呼风响。正使到热闹处沙僧对长老道:“师父也等老沙去操演操演。”好和尚双着脚一跳轮着杖也起在空中只见那锐气氤氲金光缥缈双手使降妖杖丢一个丹凤朝阳饿虎扑食紧迎慢挡捷转忙撺。弟兄三个即展神通都在那半空中一齐扬威耀武。这才是:真禅景象不凡同大道缘由满太空。金木施威盈法界刀圭展转合圆通。神兵精锐随时显丹器花生到处崇。天竺虽高还戒性玉华王子总归中。唬得那三个小王子跪在尘埃。暴纱亭大小人员并王府里老王子满城中军民男女僧尼道俗一应人等家家念佛磕头户户拈香礼拜。果然是:见象归真度众僧人间作福享清平。从今果正菩提路尽是参禅拜佛人。他三个各逞雄才使了一路按下祥云把兵器收了到唐僧面前问讯谢了师恩各各坐下不题。
	
	那三个小王子急回宫里告奏老王道:“父王万千之喜!今有莫大之功也!适才可曾看见半空中舞弄么?”老王道:“我才见半空霞彩就于宫院内同你母亲等众焚香启拜更不知是那里神仙降聚也。”小王子道:“不是那里神仙就是那取经僧三个丑徒弟。一个使金箍铁棒一个使九齿钉钯一个使降妖宝杖把我三个的兵器比的通没有分毫。我们教他使一路他嫌地上窄狭不好支吾等我起在空中使一路你看。他就各驾云头满空中祥云缥缈瑞气氤氲。才然落下都坐在暴纱亭里。
	
	做儿的十分欢喜欲要拜他为师学他手段保护我邦此诚莫大之功!不知父王以为何如?”老王闻言信心从愿。
	
	当时父子四人不摆驾不张盖步行到暴纱亭。他四众收拾行李欲进府谢斋辞王起行偶见玉华王父子上亭来倒身下拜慌得长老舒身扑地还礼行者等闪过旁边微微冷笑。
	
	众拜毕请四众进府堂上坐。四众欣然而入老王起身道:“唐老师父孤有一事奉求不知三位高徒可能容否?”三藏道:
	
	“但凭千岁吩咐小徒不敢不从。”老王道:“孤先见列位时只以为唐朝远来行脚僧其实肉眼凡胎多致轻亵。适见孙师、猪师、沙师起舞在空方知是仙是佛。孤三个犬子一生好弄武艺今谨虔心欲拜为门徒学些武艺。万望老师开天地之心普运慈舟传度小儿必以倾城之资奉谢。”行者闻言忍不住呵呵笑道:“你这殿下好不会事!我等出家人巴不得要传几个徒弟。你令郎既有从善之心切不可说起分毫之利但只以情相处足为爱也。”王子听言十分欢喜随命大排筵宴就于本府正堂摆列。噫!一声旨意即刻俱完。但见那:结彩飘飖香烟馥郁。戗金桌子挂绞绡幌人眼目;彩漆椅儿铺锦绣添座风光。树果新鲜茶汤香喷。三五道闲食清甜一两餐馒头丰洁。蒸酥蜜煎更奇哉油札糖浇真美矣。有几瓶香糯素酒斟出来赛过琼浆;献几番阳羡仙茶捧到手香欺丹桂。般般品品皆齐备色色行行尽出奇。一壁厢叫承应的歌舞吹弹撮弄演戏。他师徒们并王父子尽乐一日。不觉天晚散了酒席又叫即于暴纱亭铺设床帏请师安宿待明早竭诚焚香再拜求传武艺。众皆听从即备香汤请师沐浴众却归寝。此时那:
	
	众鸟高栖万簌沉诗人下榻罢哦吟。银河光显天弥亮野径荒凉草更深。砧杵叮咚敲别院关山杳窎动乡心。寒蛩声朗知人意呖呖床头破梦魂。
	
	一宵晚景题过明早那老王父子又来相见这长老。昨日相见还是王礼今日就行师礼。那三个小王子对行者、八戒、沙僧当面叩头拜问道:“尊师之兵器还借出与弟子们看看。”
	
	八戒闻言欣然取出钉钯抛在地下。沙僧将宝杖抛出倚在墙边。二王子与三王子跳起去便拿就如蜻蜓撼石柱一个个挣得红头赤脸莫想拿动半分毫。大王子见了叫道:“兄弟莫费力了。师父的兵器俱是神兵不知有多少重哩!”八戒笑道:
	
	“我的钯也没多重只有一藏之数连柄五千零四十八斤。”三王子问沙僧道:“师父宝杖多重?”沙僧笑道:“也是五千零四十八斤。”大王子求行者的金箍棒看。行者去耳朵里取出一个针儿来迎风幌一幌就有碗来粗细直直的竖立面前。那王父子都皆悚惧众官员个个心惊。三个小王子礼拜道:“猪师、沙师之兵俱随身带在衣下即可取之。孙师为何自耳中取出?见风即长何也?”行者笑道:“你不知我这棒不是凡间等闲可有者。这棒是:鸿蒙初判陶镕铁大禹神人亲所设。湖海江河浅共深曾将此棒知之切。开山治水太平时流落东洋镇海阙。日久年深放彩霞能消能长能光洁。老孙有分取将来变化无方随口诀。要大弥于宇宙间要小却似针儿节。棒名如意号金箍天上人间称一绝。重该一万三千五百斤或粗或细能生灭。也曾助我闹天宫也曾随我攻地阙。伏虎降龙处处通炼魔荡怪方方彻。举头一指太阳昏天地鬼神皆胆怯。混沌仙传到至今原来不是凡间铁。”那王子听言个个顶礼不尽。三个向前重重拜礼虔心求授行者道:“你三人不知学那般武艺。”王子道:
	
	“愿使棍的就学棍惯使钯的就学钯爱用杖的就学杖。”行者笑道:“教便也容易只是你等无力量使不得我们的兵器恐学之不精如画虎不成反类狗也。古人云教训不严师之惰学问无成子之罪。汝等既有诚心可去焚香来拜了天地我先传你些神力然后可授武艺。”三个小王子闻言满心欢喜即便亲抬香案沐手焚香朝天礼拜。拜毕请师传法行者转下身来对唐僧行礼道:“告尊师恕弟子之罪。自当年在两界山蒙师父大德救脱弟子秉教沙门一向西来虽不曾重报师恩却也曾渡水登山竭尽心力。今来佛国之乡幸遇贤王三子投拜我等欲学武艺。彼既为我等之徒弟即为我师之徒孙也。谨禀过我师庶好传授。”三藏十分大喜。八戒、沙僧见行者行礼也那转身朝三藏磕头道:“师父我等愚鲁拙口钝腮不会说话望师父高坐法位也让我两个各招个徒弟耍耍也是西方路上之忆念。”三藏俱欣然允之。
	
	行者才教三个王子就于暴纱亭后静室之间画了罡斗教三人都俯伏在内一个个瞑目宁神。这里却暗暗念动真言诵动咒语将仙气吹入他三人心腹之中把元神收归本舍传与口诀各授得万千之膂力运添了火候却象个脱胎换骨之法。运遍了子午周天那三个小王子方才苏醒一齐爬将起来抹抹脸精神抖擞一个个骨壮筋强:大王子就拿得金箍棒二王子就轮得九齿钯三王子就举得降妖杖。老王见了欢喜不胜又排素宴启谢他师徒四众。就在筵前各传各授:学棍的演棍学钯的演钯学杖的演杖。虽然打几个转身丢几般解数终是有些着力走一路便喘气嘘嘘不能耐久;盖他那兵器都有变化其进退攻扬随消随长皆有变化自然之妙此等终是凡夫岂能以遽及也?当日散了筵宴。
	
	次日三个王子又来称谢道:“感蒙神师授赐了膂力纵然轮得师的神器只是转换艰难。意欲命工匠依师神器式样减削斤两打造一般未知师父肯容否?”八戒道:“好!好!好!说得象话。我们的器械一则你们使不得二则我们要护法降魔正该另造另造。”王子又随宣召铁匠买办钢铁万斤就于王府内前院搭厂支炉铸造。先一日将钢铁炼熟次日请行者三人将金箍棒、九齿钯、降妖杖都取出放在篷厂之间看样造作遂此昼夜不收。
	
	噫!这兵器原是他们随身之宝一刻不可离者各藏在身自有许多光彩护体。今放在厂院中几日那霞光有万道冲天瑞气有千般罩地。其夜有一妖精离城只有七十里远近山唤豹头山洞唤虎口洞夜坐之间忽见霞光瑞气即驾云头而看。原是州城之光彩他按下云来近前观看乃是这三般兵器放光。妖精又喜又爱道:“好宝贝!好宝贝!这是甚人用的今放在此?也是我的缘法拿了去呀!拿了去呀!”他爱心一动弄起威风将三般兵器一股收之径转本洞。正是那:道不须臾离可离非道也。神兵尽落空枉费参修者。毕竟不知怎生寻得这兵器且听下回分解。
	------------
	
	第八十九回 黄狮精虚设钉钯宴 金木土计闹豹头山
	
	却说那院中几个铁匠因连日辛苦夜间俱自睡了。及天明起来打造篷下不见了三般兵器一个个呆挣神惊四下寻找。只见那三个王子出宫来看那铁匠一齐磕头道:“小主啊神师的三般兵器都不知那里去了!”小王子听言心惊胆战道:“想是师父今夜收拾去了。”急奔暴纱亭看时见白马尚在廊下忍不住叫道:“师父还睡哩!”沙僧道:“起来了。”即将房门开了让王子进里看时不见兵器慌慌张张问道:“师父的兵器都收来了?”行者跳起道:“不曾收啊!”王子道:“三般兵器今夜都不见了。”八戒连忙爬起道:“我的钯在么?”小王道:
	
	“适才我等出来只见众人前后找寻不见弟子恐是师父收了却才来问。老师的宝贝俱是能长能消想必藏在身边哄弟子哩。”行者道:“委的未收都寻去来。”随至院中篷下果然不见踪影。八戒道:“定是这伙铁匠偷了!快拿出来!略迟了些儿就都打死!打死!”那铁匠慌得磕头滴泪道:“爷爷!我们连日辛苦夜间睡着乃至天明起来遂不见了。我等乃一概凡人怎么拿得动望爷爷饶命!饶命!”行者无语暗恨道:“还是我们的不是既然看了式样就该收在身边怎么却丢放在此!那宝贝霞彩光生想是惊动甚么歹人今夜窃去也。”八戒不信道:
	
	“哥哥说那里话!这般个太平境界又不是旷野深山怎得个歹人来!定是铁匠欺心他见我们的兵器光彩认得是三件宝贝连夜走出王府伙些人来抬的抬拉的拉偷出去了!拿过来打呀!打呀!”众匠只是磕头誓。正嚷处只见老王子出来问及前事却也面无人色沉吟半晌道:“神师兵器本不同凡就有百十余人也禁挫不动;况孤在此城今已五代不是大胆海口孤也颇有个贤名在外这城中军民匠作人等也颇惧孤之法度断是不敢欺心望神师再思可矣。”行者笑道:“不用再思也不须苦赖铁匠。我问殿下:你这州城四面可有甚么山林妖怪?”王子道:“神师此问甚是有理。孤这州城之北有一座豹头山山中有一座虎口洞。往往人言洞内有仙又言有虎狼又言有妖怪。孤未曾访得端的不知果是何物。”行者笑道:
	
	“不消讲了定是那方歹人知道俱是宝贝一夜偷将去了。”
	
	叫:“八戒沙僧你都在此保着师父护着城池等老孙寻访去来。”又叫铁匠们不可住了炉火一一炼造。
	
	好猴王辞了三藏唿哨一声形影不见早跨到豹头山上。原来那城相去只有七十里一瞬即到。径上山峰观看果然有些妖气真是:龙脉悠长地形远大。尖峰挺挺插天高陡涧沉沉流水紧。山前有瑶草铺茵山后有奇花布锦。乔松老柏古树修复出鸦山鹊乱飞鸣野鹤野猿皆啸唳。悬崖下麋鹿双双;峭壁前獾狐对对。一起一伏远来龙九曲九湾潜地脉。埂头相接玉华州万古千秋兴胜处。行者正然看时忽听得山背后有人言语急回头视之乃两个狼头怪妖朗朗的说着话向西北上走。行者揣道:“这定是巡山的怪物等老孙跟他去听听看他说些甚的。”捻着诀念个咒摇身一变变做个蝴蝶儿展开翅翩翩翻翻径自赶上。果然变得有样范:一双粉翅两道银须。乘风飞去急映日舞来徐。渡水过墙能疾俏偷香弄絮甚欢娱。体轻偏爱鲜花味雅态芳情任卷舒。他飞在那个妖精头直上飘飘荡荡听他说话。那妖猛的叫道:“二哥我大王连日侥幸。前月里得了一个美人儿在洞内盘桓十分快乐。
	
	昨夜里又得了三般兵器果然是无价之宝。明朝开宴庆钉钯会唱我们都有受用。”这个道:“我们也有些侥幸。拿这二十两银子买猪羊去如今到了乾方集上先吃几壶酒儿把东西开个花帐儿落他二三两银子买件绵衣过寒却不是好?”两个怪说说笑笑的上大路急走如飞。行者听得要庆钉钯会心中暗喜;欲要打杀他争奈不管他事况手中又无兵器。他即飞向前边现了本相在路口上立定。那怪看看走到身边被他一口法唾喷将去念一声“唵吽咤唎”即使个定身法把两个狼头精定住。眼睁睁口也难开;直挺挺双脚站住。又将他扳翻倒揭衣搜捡果是有二十两银子着一条搭包儿打在腰间裙带上又各挂着一个粉漆牌儿一个上写着“刁钻古怪”一个上写着“古怪刁钻”。
	
	好大圣取了他银子解了他牌儿返跨步回至州城。到王府中见了王子、唐僧并大小官员、匠作人等具言前事。八戒笑道:“想是老猪的宝贝霞彩光明所以买猪羊治筵席庆贺哩。但如今怎得他来?”行者道:“我兄弟三人俱去这银子是买办猪羊的且将这银子赏了匠人教殿下寻几个猪羊。八戒你变做刁钻古怪我变做古怪刁钻沙僧装做个贩猪羊的客人走进那虎口洞里得便处各人拿了兵器打绝那妖邪回来却收拾走路。”沙僧笑道:“妙妙妙!不宜迟!快走!”老王果依此计即教管事的买办了七八口猪四五腔羊。
	
	他三人辞了师父在城外大显神通。八戒道:“哥哥我未曾看见那刁钻古怪怎生变得他模样?”行者道:“那怪被老孙使了定身法定住在那里直到明日此时方醒。我记得他的模样你站下等我教你变。如此如彼就是他的模样了。”那呆子真个口里念着咒行者吹口仙气霎时就变得与那刁钻古怪一般无二将一个粉牌儿带在腰间。行者即变做古怪刁钻腰间也带了一个牌儿。沙僧打扮得象个贩猪羊的客人一起儿赶着猪羊上大路径奔山来。不多时进了山凹里又遇见一个小妖。他生得嘴脸也恁地凶恶!看那:圆滴溜两只眼如灯幌亮;
	
	红剌瞔一头毛似火飘光。糟鼻子猱猍口獠牙尖利;查耳朵砍额头青脸泡浮。身穿一件浅黄衣足踏一双莎蒲履。雄雄纠纠若凶神急急忙忙如恶鬼。那怪左胁下挟着一个彩漆的请书匣儿迎着行者三人叫道:“古怪刁钻你两个来了?买了几口猪羊?”行者道:“这赶的不是?”那怪朝沙僧道:“此位是谁?”
	
	行者道:“就是贩猪羊的客人还少他几两银子带他来家取的。你往那里去?”那怪道:“我往竹节山去请老大王明早赴会。”行者绰他的口气儿就问:“共请多少人?”那怪道:“请老大王坐席连本山大王共头目等众约有四十多位。”正说处八戒道:“去罢去罢!猪羊都四散走了!”行者道:“你去邀着等我讨他帖儿看看。”那怪见自家人即揭开取出递与行者。行者展开看时上写着:“明辰敬治肴酌庆钉钯嘉会屈尊过山一叙幸勿外至感!右启祖翁九灵元圣老大人尊前。(wwW.mianhuatang.la 无弹窗广告)门下孙黄狮顿百拜。”行者看毕仍递与那怪。那怪放在匣内径往东南上去了。
	
	沙僧问道:“哥哥帖儿上是甚么话头?”行者道:“乃庆钉钯会的请帖名字写着门下孙黄狮顿百拜请的是祖翁九灵元圣老大人。”沙僧笑道:“黄狮想必是个金毛狮子成精但不知九灵元圣是个何物。”八戒听言笑道:“是老猪的货了!”行者道:“怎见得是你的货?”八戒道:“古人云癞母猪专赶金毛狮子故知是老猪之货物也。”他三人说说笑笑赶着猪羊却就望见虎口洞门。但见那门儿外:周围山绕翠一脉气连城。峭壁扳青蔓高崖挂紫荆。鸟声深树匝花影洞门迎。不亚桃源洞堪宜避世情。
	
	渐渐近于门口又见一丛大大小小的杂项妖精在那花树之下顽耍忽听得八戒“呵!呵!”赶猪羊到时都来迎接便就捉猪的捉猪捉羊的捉羊一齐捆倒。早惊动里面妖王领十数个小妖出来问道:“你两个来了?买了多少猪羊?”行者道:“买了八口猪七腔羊共十五个牲口。猪银该一十六两羊银该九两前者领银二十两仍欠五两。这个就是客人跟来找银子的。”妖王听说即唤:“小的们取五两银子打他去。”行者道:“这客人一则来找银子二来要看看嘉会。”那妖大怒骂道:“你这个刁钻儿惫懒!你买东西罢了又与人说甚么会不会!”八戒上前道:“主人公得了宝贝诚是天下之奇珍就教他看看怕怎的?”那怪咄的一声道:“你这古怪也可恶!我这宝贝乃是玉华州城中得来的倘这客人看了去那州中传说说得人知那王子一时来访求却如之何?”行者道:“主公这个客人乃乾方集后边的人去州许远又不是他城中人也那里去传说?二则他肚里也饥了我两个也未曾吃饭。家中有现成酒饭赏他些吃了打他去罢。”说不了有一小妖取了五两银子递与行者。行者将银子递与沙僧道:“客人收了银子我与你进后面去吃些饭来。”沙僧仗着胆同八戒、行者进于洞内到二层厂厅之上只见正中间桌上高高的供养着一柄九齿钉钯真个是光彩映目东山头靠着一条金箍棒西山头靠着一条降妖杖。那怪王随后跟着道:“客人那中间放光亮的就是钉钯。你看便看只是出去千万莫与人说。”沙僧点头称谢了。
	
	噫!这正是物见主必定取那八戒一生是个鲁夯的人他见了钉钯那里与他叙甚么情节跑上去拿下来轮在手中现了本相丢了解数望妖精劈脸就筑。这行者、沙僧也奔至两山头各拿器械现了原身。三兄弟一齐乱打慌得那怪王急抽身闪过转入后边取一柄四明铲杆长鐏利赶到天井中支住他三般兵器厉声喝道:“你是甚么人敢弄虚头骗我宝贝!”行者骂道:“我把你这个贼毛团!你是认我不得!我们乃东土圣僧唐三藏的徒弟。因至玉华州倒换关文蒙贤王教他三个王子拜我们为师学习武艺将我们宝贝作样打造如式兵器。因放在院中被你这贼毛团夤夜入城偷来倒说我弄虚头骗你宝贝!不要走!就把我们这三件兵器各奉承你几下尝尝!”那妖精就举铲来敌。这一场从天井中斗出前门。看他三僧攒一怪!好杀:
	
	呼呼棒若风滚滚钯如雨。降妖杖举满天霞四明铲伸云生绮。
	
	好似三仙炼大丹火光彩幌惊神鬼。行者施威甚有能妖精盗宝多无礼!天蓬八戒显神通大将沙僧英更美。兄弟合意运机谋虎口洞中兴斗起。那怪豪强弄巧乖四个英雄堪厮比。当时杀至日头西妖邪力软难相抵。他们在豹头山战斗多时那妖精抵敌不住向沙僧前喊一声:“看铲!”沙僧让个身法躲过妖精得空而走向东南巽宫上乘风飞去。八戒拽步要赶行者道:“且让他去自古道穷寇勿追。且只来断他归路。”八戒依言。三人径至洞口把那百十个若大若小的妖精尽皆打死原来都是些虎狼彪豹马鹿山羊。被大圣使个手法将他那洞里细软物件并打死的杂项兽身与赶来的猪羊通皆带出。沙僧就取出干柴放起火来八戒使两个耳朵扇风把一个巢穴霎时烧得干净却将带出的诸物即转州城。
	
	此时城门尚开人家未睡老王父子与唐僧俱在暴纱亭盼望。只见他们扑哩扑剌的丢下一院子死兽、猪羊及细软物件一齐叫道:“师父我们已得胜回来也!”那殿下喏喏相谢唐长老满心欢喜三个小王子跪拜于地沙僧搀起道:“且莫谢都近前看看那物件。”王子道:“此物俱是何来?”行者笑道:“那虎狼彪豹马鹿山羊都是成精的妖怪。被我们取了兵器打出门来。那老妖是个金毛狮子他使一柄四明铲与我等战到天晚败阵逃生往东南上走了。我等不曾赶他却扫除他归路打杀这些群妖搜寻他这些物件带将来的。”老王听说又喜又忧。
	
	喜的是得胜而回忧的是那妖日后报仇。行者道:“殿下放心我已虑之熟处之当矣。一定与你扫除尽绝方才起行决不至贻害于后。我午间去时撞见一个青脸红毛的小妖送请书我看他帖子上写着‘明辰敬治肴酌庆钉钯嘉会屈尊车从过山一叙。幸勿外至感!右启祖翁九灵元圣老大人尊前。’名字是门下孙黄狮顿百拜。才子那妖精败阵必然向他祖翁处去会话。明辰断然寻我们报仇当情与你扫荡干净。”老王称谢了摆上晚斋。师徒们斋毕各归寝处不题。
	
	却说那妖精果然向东南方奔到竹节山。那山中有一座洞天之处唤名九曲盘桓洞。洞中的九灵元圣是他的祖翁。当夜足不停风行至五更时分到于洞口敲门而进。小妖见了道:
	
	“大王昨晚有青脸儿下请书老爷留他住到今早欲同他去赴你钉钯会你怎么又绝早亲来邀请?”妖精道:“不好说不好说!会成不得了!”正说处见青脸儿从里边走出道:“大王你来怎的?老大王爷爷起来就同我去赴会哩。”妖精慌张张的只是摇手不言。少顷老妖起来了唤入。这妖精丢了兵器倒身下拜止不住腮边泪落。老妖道:“贤孙你昨日下柬今早正欲来赴会你又亲来为何悲烦恼?”妖精叩头道:“小孙前夜对月闲行只见玉华州城中有光彩冲空。急去看时乃是王府院中三般兵器放光:一件是九齿渗金钉钯一件是宝杖一件是金箍棒。小孙即使神法摄来立名钉钯嘉会着小的们买猪羊果品等物设宴庆会请祖爷爷赏之以为一乐。昨差青脸来送柬之后只见原差买猪羊的刁钻儿等赶着几个猪羊又带了一个贩卖的客人来找银子。他定要看看会去是小孙恐他外面传说不容他看。他又说肚中饥饿讨些饭吃因教他后边吃饭。
	
	他走到里边看见兵器说是他的。三人就各抢去一件现出原身:一个是毛脸雷公嘴的和尚一个是长嘴大耳朵的和尚一个是晦气色脸的和尚他都不分好歹喊一声乱打。是小孙急取四明铲赶出与他相持问是甚么人敢弄虚头。他道是东土大唐差往西天去的唐僧之徒弟因过州城倒换关文被王子留住习学武艺将他这三件兵器作样子打造放在院内被我偷来遂此不忿相持。不知那三个和尚叫做甚名却真有本事。小孙一人敌他三个不过所以败走祖爷处。望拔刀相助拿那和尚报仇庶见我祖爱孙之意也!”老妖闻言默想片时笑道:
	
	“原来是他。我贤孙你错惹了他也!”妖精道:“祖爷知他是谁?”老妖道:“那长嘴大耳者乃猪八戒晦气色脸者乃沙和尚这两个犹可。那毛脸雷公嘴者叫做孙行者这个人其实神通广大五百年前曾大闹天宫十万天兵也不曾拿得住。他专意寻人的他便就是个搜山揭海、破洞攻城、闯祸的个都头!你怎么惹他?也罢等我和你去把那厮连玉华王子都擒来替你出气!”那妖精听说即叩头而谢。
	
	当时老妖点猱狮、雪狮、狻猊、白泽、伏狸、抟象诸孙各执锋利器械黄狮引领各纵狂风径至豹头山界。只闻得烟火之气扑鼻又闻得有哭泣之声。仔细看时原来是刁钻、古怪二人在那里叫主公哭主公哩。妖精近前喝道:“你是真刁钻儿假刁钻儿?”二怪跪倒噙泪叩头道:“我们怎是假的?昨日这早晚领了银子去买猪羊走至山西边大冲之内见一个毛脸雷公嘴的和尚他啐了我们一口我们就脚软口强不能言语不能移步被他扳倒把银子搜了去牌儿解了去我两个昏昏沉沉直到此时才醒。及到家见烟火未息房舍尽皆烧了又不见主公并大小头目故在此伤心痛哭。不知这火是怎生起的!”那妖精闻言止不住泪如泉涌双脚齐跌喊声振天恨道:“那秃厮!十分作恶!怎么干出这般毒事把我洞府烧尽美人烧死家当老小一空!气杀我也气杀我也!”老妖叫猱狮扯他过来道:“贤孙事已至此徒恼无益。且养全锐气到州城里拿那和尚去。”那妖精犹不肯住哭道:“老爷!我那们个山场非一日治的今被这秃厮尽毁我却要此命做甚的!”挣起来往石崖上撞头磕脑被雪狮、猱狮等苦劝方止。当时丢了此处都奔州城。
	
	只听得那风滚滚雾腾腾来得甚近唬得那城外各关厢人等拖男挟女顾不得家私都往州城中走走入城门将门闭了。有人报入王府中道:“祸事!祸事!”那王子唐僧等正在暴纱亭吃早斋听得人报祸事却出门来问。众人道:“一群妖精飞沙走石喷雾掀风的来近城了!”老王大惊道:“怎么好?”行者笑道:“都放心!都放心!这是虎口洞妖精昨日败阵往东南方去伙了那甚么九灵元圣儿来也。等我同兄弟们出去吩咐教关了四门汝等点人夫看守城池。”那王子果传令把四门闭了点起人夫上城。他父子并唐僧在城楼上点札旌旗蔽日炮火连天。行者三人却半云半雾出城迎敌。这正是:失却慧兵缘不谨顿教魔起众邪凶。毕竟不知这场胜败如何且听下回分解。
	------------
	
	第九十回 师狮授受同归一 盗道缠禅静九灵山
	
	却说孙大圣同八戒、沙僧出城头觌面相迎见那伙妖精都是些杂毛狮子:黄狮精在前引领狻猊狮、抟象狮在左白泽狮、伏狸狮在右猱狮、雪狮在后中间却是一个九头狮子。那青脸儿怪执一面锦锈团花宝幢紧挨着九头狮子刁钻古怪儿、古怪刁钻儿打两面红旗齐齐的都布在坎宫之地。八戒莽撞走近前骂道:“偷宝贝的贼怪!你去那里伙这几个毛团来此怎的?”黄狮精切齿骂道:“泼狠秃厮!昨日三个敌我一个我败回去让你为人罢了;你怎么这般狠恶烧了我的洞府损了我的山场伤了我的眷族!我和你冤仇深如大海!不要走!吃你老爷一铲!”好八戒举钯就迎。两个才交手还未见高低那猱狮精轮一根铁蒺藜雪狮精使一条三楞简径来奔打。八戒一声喊道:“来得好!”你看他横冲直抵斗在一处。这壁厢沙和尚急掣降妖杖近前相助又见那狻猊精、白泽精与抟象、伏狸二精一拥齐上。这里孙大圣使金箍棒架住群精狻猊使闷棍白泽使铜锤抟象使钢枪伏狸使钺斧。那七个狮子精这三个狠和尚好杀:棍锤枪斧三楞简蒺藜骨朵四明铲。七狮七器甚锋芒围战三僧齐呐喊。大圣金箍铁棒凶沙僧宝杖人间罕。八戒颠风骋势雄钉钯幌亮光华惨。前遮后挡各施功左架右迎都勇敢。城头王子助威风擂鼓筛锣齐壮胆。投来抢去弄神通杀得昏濛天地反”那一伙妖精齐与大圣三人战经半日不觉天晚。八戒口吐粘涎看看脚软虚幌一钯败下阵去被那雪狮、猱狮二精喝道:“那里走”看打!”呆子躲闪不及被他照脊梁上打了一简睡在地下只叫:“罢了!罢了!”两个精把八戒采鬃拖尾扛将去见那九头狮子报道:“祖爷我等拿了一个来也。”说不了沙僧行者也都战败。众妖精一齐赶来被行者拔一把毫毛嚼碎喷将去叫声“变!”即变做百十个小行者围围绕绕将那白泽、狻猊、抟象、伏狸并金毛狮怪围裹在中。沙僧行者却又上前攒打。到晚拿住狻猊、白泽走了伏狸、抟象。金毛报知老妖老怪见失了二狮吩咐:“把猪八戒捆了不可伤他性命。待他还我二狮却将八戒与他。他若无知坏了我二狮即将八戒杀了对命!”当晚群妖安歇城外不题。
	
	却说孙大圣把两个狮子精抬近城边老王见了即传令开门差二三十个校尉拿绳扛出门绑了狮精扛入城里。孙大圣收了法毛同沙僧径至城楼上见了唐僧。唐僧道:“这场事甚是利害呀!悟能性命不知有无?”行者道:“没事!我们把这两个妖精拿了他那里断不敢伤。且将二精牢拴紧缚待明早抵换八戒也。”三个小王子对行者叩头道:“师父先前赌斗只见一身及后佯输而回却怎么就有百十位师身?及至拿住妖精近城来还是一身此是甚么法力?”行者笑道:“我身上有八万四千毫毛以一化十以十化百百千万亿之变化皆身外身之法也。”那王子一个个顶礼即时摆上斋来就在城楼上吃了。各垛口上都要灯笼旗帜梆铃锣鼓支更传箭放炮呐喊。
	
	早又天明。老怪即唤黄狮精定计道:“汝等今日用心拿那行者、沙僧等我暗自飞空上城拿他那师父并那老王父子先转九曲盘桓洞待你得胜回报。”黄狮领计便引猱狮、雪狮、抟象、伏狸各执兵器到城处滚风酿雾的索战。这里行者与沙僧跳出城头厉声骂道:“贼泼怪!快将我师弟八戒送还我饶你性命!不然都教你粉骨碎尸!”那妖精那容分说一拥齐来。这大圣弟兄两个各运机谋挡住五个狮子。这杀比昨日又甚不同:呼呼刮地狂风恶暗暗遮天黑雾浓。走石飞沙神鬼怕推林倒树虎狼惊。钢枪狠狠钺斧明棍铲铜锤太毒情。恨不得囫囵吞行者活活泼泼擒住小沙僧。这大圣一条如意棒卷舒收放甚精灵。沙僧那柄降妖杖灵霄殿外有名声。今番干运神通广西域施功扫荡精。这五个杂毛狮子精与行者、沙僧正自杀到好处那老怪驾着黑云径直腾至城楼上摇一摇头唬得那城上文武大小官员并守城人夫等都滚下城去被他奔入楼中张开口把三藏与老王父子一顿噙出复至坎宫地下将八戒也着口噙之。原来他九个头就有九张口一口噙着唐僧一口噙着八戒一口噙着老王一口噙着大王子一口噙着二王子一口噙着三王子六口噙着六人还空了三张口声喊叫道:“我先去也!”这五个小狮精见他祖得胜一个个愈展雄才。行者闻得城上人喊嚷情知中了他计急唤沙僧仔细;他却把臂膊上毫毛尽皆拔下入口嚼烂喷出变作千百个小行者一拥攻上当时拖倒猱狮活捉了雪狮拿住了抟象狮扛翻了伏狸狮将黄狮打死烘烘的嚷到州城之下倒转走脱了青脸儿与刁钻古怪、古怪刁钻儿二怪。那城上官看见却又开门将绳把五个狮精又捆了抬进城去。还未落只见那王妃哭哭啼啼对行者礼拜道:“神师啊我殿下父子并你师父性命休矣!这孤城怎生是好?”大圣收了法毛对王妃作礼道:“贤后莫愁只因我拿他七个狮精那老妖弄摄法定将我师父与殿下父子摄去料必无伤。待明日绝早我兄弟二人去那山中管情捉住老妖还你四个王子。”那王妃一簇女眷闻得此言都对行者下拜道:“愿求殿下父子全生皇图坚固!”拜毕一个个含泪还宫。
	
	行者吩咐各官:“将打死那黄狮精剥了皮六个活狮精牢牢拴锁。取些斋饭来我们吃了睡觉你们都放心保你无事。”
	
	至次日大圣领沙僧驾起祥云不多时到子竹节山头。按云头观看好座高山!但见:峰排突兀岭峻崎岖。深涧下潺湅水漱陡崖前锦锈花香。回峦重迭古道湾环。真是鹤来松有伴果然云去石无依。玄猿觅果向晴晖麋鹿寻花欢日暖。青鸾声晰呖黄鸟语绵蛮。春来桃李争妍夏至柳槐竞茂。秋到黄花布锦冬交白雪飞绵。四时八节好风光不亚瀛洲仙景象。
	
	他两个正在山头上看景忽见那青脸儿手拿一条短棍径跑出崖谷之间。行者喝道:“那里走!老孙来也!”唬得那小妖一翻一滚的跑下崖谷。他两个一直追来又不见踪迹向前又转几步却是一座洞府两扇花斑石门紧紧关闭。门楟上横嵌着一块石版楷镌了十个大字乃是万灵竹节山九曲盘桓洞。那小妖原来跑进洞去即把洞门闭了到中间对老妖道:“爷爷外面又有两个和尚来了。”老妖道:“你大王并猱狮、雪狮、抟象、伏狸可曾来?”小妖道:“不见!不见!只是两个和尚在山峰高处眺望。我看见回头就跑他赶将来我却闭门来也。”老妖听说低头不语半晌忽的吊下泪来叫声:“苦啊!我黄狮孙死了!猱狮孙等又尽被和尚捉进城去矣!此恨怎生报得!”
	
	八戒捆在旁边与王父子唐僧俱攒在一处恓恓惶惶受苦听见老妖说声“众孙被和尚捉进城去”暗暗喜道:“师父莫怕殿下休愁我师兄已得胜捉了众妖寻到此间救拔吾等也。”说罢又听得老妖叫:“小的们好生在此看守等我出去拿那两个和尚进来一惩治。”你看他身无披挂手不拈兵大踏步走到前边只闻得孙行者吆喝哩。他就大开了洞门不答话径奔行者。行者使铁棒当头支住沙僧轮宝杖就打。那老妖把头摇一摇左右八个头一齐张开口把行者、沙僧轻轻的又衔于洞内教:“取绳索来!”那刁钻古怪、古怪刁钻与青脸儿是昨夜逃生而回者即拿两条绳把他二人着实捆了。老妖问道:“你这泼猴把我那七个儿孙捉了我今拿住你和尚四个王子四个也足以抵得我儿孙之命!小的们选荆条柳棍来且打这猴头一顿与我黄狮孙报报冤仇!”那三个小妖各执柳棍专打行者。行者本是熬炼过的身体那些些柳棍儿只好与他拂痒他那里做声?凭他怎么捶打略不介意。八戒、唐僧与王子见了一个个毛骨悚然。少时打折了柳棍直打到天晚也不计其数。沙僧见打得多了甚不过意道:“我替他打百十下罢。”老妖道:你且莫忙明日就打到你了一个个挨次儿打将来。”八戒着忙道:“后日就打到我老猪也!”打一会渐渐的天昏了老妖叫:“小的们且住点起灯火来你们吃些饮食让我到锦云窝略睡睡去。汝三人都是遭过害的却用心看守待明早再打。”三个小妖移过灯来拿柳棍又打行者脑盖就象敲梆子一般剔剔托托托剔紧几下慢几下。夜将深了却都盹睡。
	
	行者就使个遁法将身一小脱出绳来抖一抖毫毛整束了衣服耳朵内取出棒来幌一幌有吊桶粗细二丈长短朝着三个小妖道:“你这孽畜把你老爷就打了许多棍子!老爷还只照旧老爷也把这棍子略桠你桠看道如何!”把三个小妖轻轻一桠就桠做三个肉饼却又剔亮了灯解放沙僧。八戒捆急了忍不住大声叫道:“哥哥!我的手脚都捆肿了倒不来先解放我!”这呆子喊了一声却早惊动老妖。老妖一毂辘爬起来道:“是谁人解放?”那行者听见一口吹息灯也顾不得沙僧等众使铁棒打破几重门走了。那老妖到中堂里叫:“小的们怎么没了灯光?只莫走了人也?”叫一声没人答应;又叫一声又没人答应。及取灯火来看时只见地下血淋淋的三块肉饼老王父子及唐僧、八戒俱在只不见了行者、沙僧。点着火前后赶看忽见沙僧还背贴在廊下站哩被他一把拿住捽倒照旧捆了。又找寻行者但见几层门尽皆破损情知是行者打破走了也不去追赶将破门补的补遮的遮固守家业不题。
	
	却说孙大圣出了那九曲盘桓洞跨祥云径转玉华州但见那城头上各厢的土地神祇与城隍之神迎空拜接。行者道:“汝等怎么今夜才见?”城隍道:“小神等知大圣下降玉华州因有贤王款留故不敢见。今知王等遇怪大圣降魔特来叩接。”行者正在嗔怪处又见金头揭谛、六甲六丁神将押着一尊土地跪在面前道:“大圣吾等捉得这个地里鬼来也。”行者喝道:
	
	“汝等不在竹节出护我师父却怎么嚷到这里?”丁甲神道:“大圣那妖精自你逃时复捉住卷帘大将依然捆了。我等见他法力甚大却将竹节山土地押解至此。他知那妖精的根由乞大圣问他一问便好处治以救圣僧贤王之苦。”行者听言甚喜那土地战兢兢叩头道:“那老妖前年下降竹节山。那九曲盘桓洞原是六狮之窝那六个狮子自得老妖至此就都拜为祖翁。
	
	祖翁乃是个九头狮子号为九灵元圣。若得他灭须去到东极妙岩宫请他主人公来方可收伏。他人莫想擒也。”行者闻言思忆半晌道:“东极妙岩宫是太乙救苦天尊啊。他坐下正是个九头狮子。这等说——”便教:“揭谛、金甲还同土地回去暗中护祐师父、师弟并州王父子。本处城隍守护城池走出去来。”众神各各遵守去讫。
	
	这大圣纵筋斗云连夜前行。约有寅时分到了东天门外正撞着广目天王与天丁、力士一行仪从。众皆停住拱手迎道:
	
	“大圣何往?”行者对众礼毕道:“前去妙岩宫走走。”天王道:
	
	“西天路不走却又东天来做甚?”行者道:“因到玉华州蒙州王相款遣三子拜我等弟兄为师习学武艺不期遇着一伙狮怪。今访得妙岩宫太乙救苦天尊乃怪之主人公也欲请他为我降怪救师。”天王道:“那厢因你欲为人师所以惹出这一窝狮子来也。”行者笑道:“正为此!正为此!”众天丁、力士一个个拱手让道而行。大圣进了东天门不多时到妙岩宫前但见:彩云重迭紫气茏葱。瓦漾金波焰门排玉兽崇。花盈双阙红霞绕日映骞林翠雾笼。果然是万真环拱千圣兴隆。殿阁层层锦窗轩处处通。苍龙盘护神光蔼黄道光辉瑞气浓。这的是青华长乐界东极妙岩宫。那宫门里立着一个穿霓帔的仙童忽见孙大圣即入宫报道:“爷爷外面是闹天宫的齐天大圣来了。”太乙救苦天尊听得即唤侍卫众仙迎接。迎至宫中只见天尊高坐九色莲花座上百亿瑞光之中见了行者下座来相见。行者朝上施礼天尊答礼道:“大圣这几年不见前闻得你弃道归佛保唐僧西天取经想是功行完了?”行者道:“功行未完却也将近。但如今因保唐僧到玉华州蒙王子遣三子拜老孙等为师习学武艺把我们三件神兵照样打造不期夜间被贼偷去。及天明寻找原是城北豹头山虎口洞一个金毛狮子成精盗去。老孙用计取出那精就伙了若干狮精与老孙大闹。内有一个九头狮子神通广大将我师父与八戒并王父子四人都衔去到一竹节山九曲盘桓洞。次日老孙与沙僧跟寻亦被衔去。老孙被他捆打无数幸而弄法走了他们正在彼处受罪。问及当坊土地始知天尊是他主人特来奉请收降解救。”天尊闻言即令仙将到狮子房唤出狮奴来问?”那狮奴熟睡被众将推摇方醒揪至中厅来见。天尊问道:“狮兽何在?”那奴儿垂泪叩头只教:“饶命!饶命!”天尊道:“孙大圣在此且不打你。你快说为何不谨走了九头狮子。”狮奴道:“爷爷我前日在大千甘露殿中见一瓶酒不知偷去吃了不觉沉醉睡着失于拴锁是以走了。”天尊道:“那酒是太上老君送的唤做轮回琼液你吃了该醉三日不醒。那狮兽今走几日了?”大圣道:“据土地说他前年下降到今二三年矣。”天尊笑道:“是了!是了!天宫里一日在凡世就是一年。”叫狮奴道:“你且起来饶你死罪跟我与大圣下方去收他来。汝众仙都回去不用跟随。”
	
	天尊遂与大圣、狮奴踏云径至竹节山只见那五方揭谛、六丁六甲、本山土地都来跪接。行者道:“汝等护祐可曾伤着我师?”众神道:“妖精着了恼睡了更不曾动甚刑罚。”天尊道:
	
	“我那元圣儿也是一个久修得道的真灵:他喊一声上通三圣下彻九泉等闲也便不伤生。孙大圣你去他门索战引他出来我好收之。”行者听言果掣棒跳近洞口高骂道:“泼妖精还我人来也!泼妖精还我人来也!”连叫了数声那老妖睡着了无人答应。行者性急起来轮铁棒往里打进口中不住的喊骂。那老妖方才惊醒心中大怒爬起来喝一声“赶战!”摇摇头便张口来衔。行者回头跳出。妖精赶到外边骂道:“贼猴!那里走!”行者立在高崖上笑道:“你还敢这等大胆无礼!你死活也不知哩!这不是你老爷主公在此?”那妖精赶到崖前早被天尊念声咒语喝道:“元圣儿!我来了!”那妖认得是主人不敢展挣四只脚伏之于地只是磕头。旁边跑过狮奴儿一把挝住项毛用拳着项上打彀百十口里骂道:“你这畜生如何偷走教我受罪!”那狮兽合口无言不敢摇动。狮奴儿打得手困方才住了即将锦韂安在他身上天尊骑了喝声教走。他就纵声驾起彩云径转妙岩宫去。
	
	大圣望空称谢了却入洞中先解玉华王次解唐三藏次又解了八戒、沙僧并三王子共搜他洞里物件逍逍停停将众领出门外。八戒就取了若干枯柴前后堆上放起火来把一个九曲盘桓洞烧做了乌焦破瓦窑!大圣又放了众神还教土地在此镇守却令八戒、沙僧各各使法把王父子背驮回州他搀着唐僧。不多时到了州城天色渐晚当有妃后官员都来接见了。摆上斋筵共坐享之。长老师徒还在暴纱亭安歇王子们入宫各寝。一宵无话。
	
	次日王又传旨大开素宴合府大小官员一一谢恩。行者又叫屠子来把那六个活狮子杀了共那黄狮子都剥了皮将肉安排将来受用。殿下十分欢喜即命杀了把一个留在本府内外人用一个与王府长史等官分用把五个都剁做一二两重的块子差校尉散给州城内外军民人等各吃些须:一则尝尝滋味二则押押惊恐。那些家家户户无不瞻仰。又见那铁匠人等造成了三般兵器对行者磕头道:“爷爷小的们工都完了。”问道:“各重多少斤两?”铁匠道:“金箍棒有千斤九齿钯与降妖杖各有八百斤。”行者道:“也罢了。”叫请三位王子出来各人执兵器。三子对老王道:“父王今日兵器完矣。”老王道:“为此兵器几乎伤了我父子之命。”小王子道:“幸蒙神师施法救出我等却又扫荡妖邪除了后患诚所谓海晏河清太平之世界也!”当时老王父子赏劳了匠作又至暴纱亭拜谢了师恩。
	
	三藏又教大圣等快传武艺莫误行程。他三人就各轮兵器在王府院中一一传授。不数日那三个王子尽皆操演精熟其余攻退之方紧慢之法各有七十二到解数无不知之。
	
	一则那诸王子心坚二则亏孙大圣先授了神力此所以那千斤之棒八百斤之钯杖俱能举能运较之初时自家弄的武艺真天渊也!有诗为证诗曰:缘因善庆遇神师习武何期动怪狮。
	
	扫荡群邪安社稷皈依一体定边夷。九灵数合元阳理四面精通道果之。授受心明遗万古玉华永乐太平时。那王子又大开筵宴谢了师教又取出一大盘金银用答微情。行者笑道:“快拿进去!快拿进去!我们出家人要他何用?”八戒在旁道:“金银实不敢受奈何我这件衣服被那些狮子精扯拉破了但与我们换件衣服足为爱也。”那王子随命针工照依色样取青锦、红锦、茶褐锦各数匹与三位各做了一件。三人欣然领受各穿了锦布直裰收拾了行装起程只见那城里城外若大若小无一人不称是罗汉临凡活佛下界鼓乐之声旌旗之色盈街塞道。正是家家户外焚香火处处门前献彩灯来至许远才回他四众方得离城西去。这一去顿脱群思潜心正果。才是:无虑无忧来佛界诚心诚意上雷音。毕竟不知到灵山还有几多路程何时行满且听下回分解。
	------------
	
	第九十一回 金平府元夜观灯 玄英洞唐僧供状
	
	修禅何处用工夫?马劣猿颠剪除。牢捉牢拴生五彩暂停暂住堕三途。若教自在神丹漏才放从容玉性枯。喜怒忧思须扫净得玄得妙恰如无。话表唐僧师徒四众离了玉华城一路平稳诚所谓极乐之乡。去有五六日程途又见一座城池唐僧问行者道:“此又是甚么处所?”行者道:“是座城池但城上有杆无旗不知地方俟近前再问。”及至关东厢见那两边茶坊酒肆喧哗米市油房热闹。街衢中有几个无事闲游的浪子见猪八戒嘴长沙和尚脸黑孙行者眼红都拥拥簇簇的争看只是不敢近前而问。唐僧捏着一把汗惟恐他们惹祸。又走过几条巷口还不到城忽见有一座山门门上有慈云寺三字唐僧道:“此处略进去歇歇马打一个斋如何?”行者道:“好!好!”
	
	四众遂一齐而入。但见那里边:珍楼壮丽宝座峥嵘。佛阁高云外僧房静月中。丹霞缥缈浮屠挺碧树阴森轮藏清。真净土假龙宫大雄殿上紫云笼。两廊不绝闲人戏一塔常开有客登。炉中香火时时爇台上灯花夜夜荧。忽闻方丈金钟韵应佛僧人朗诵经。四众正看时又见廊下走出一个和尚对唐僧作礼道:“老师何来?”唐僧道:“弟子中华唐朝来者。”那和尚倒身下拜慌得唐僧搀起道:“院主何为行此大礼?”那和尚合掌道:“我这里向善的人看经念佛都指望修到你中华地托生。
	
	才见老师丰采衣冠果然是前生修到的方得此受用故当下拜。”唐僧笑道:“惶恐!惶恐!我弟子乃行脚僧有何受用!若院主在此闲养自在才是享福哩。”那和尚领唐僧入正殿拜了佛像。唐僧方才招呼:“徒弟来耶。”原来行者三人自见那和尚与师父讲话他都背着脸牵着马守着担立在一处和尚不曾在心。忽的闻唐僧叫徒弟他三人方才转面那和尚见了慌得叫:“爷爷呀!你高徒如何恁般丑样?”唐僧道:“丑则虽丑倒颇有些法力我一路甚亏他们保护。”正说处里面又走出几个和尚作礼。先见的那和尚对后的说道:“这老师是中华大唐来的人物那三位是他高徒。”众僧且喜且惧道:“老师中华大国到此何为?”唐僧言:“我奉唐王圣旨向灵山拜佛求经。适过宝方特奔上刹一则求问地方二则打顿斋食就行。”那僧人个个欢喜又邀入方丈方丈里又有几个与人家做斋的和尚。这先进去的又叫道:“你们都来看看中华人物。原来中华有俊的有丑的俊的真个难描难画丑的却十分古怪。”那许多僧同斋主都来相见。见毕各坐下。茶罢唐僧问道:“贵处是何地名?”
	
	众僧道:“我这里乃天竺国外郡金平府是也。”唐僧道:“贵府至灵山还有许多远近?”众僧道:“此间到都下有二千里这是我等走过的。西去到灵山我们未走不知还有多少路不敢妄对。”唐僧谢了。
	
	少时摆上斋来。斋罢唐僧要行却被众僧并斋主款留道:“老师宽住一二日过了元宵耍耍去不妨。”唐僧惊问道:
	
	“弟子在路只知有山有水怕的是逢怪逢魔把光阴都错过了不知几时是元宵佳节。”众僧笑道:“老师拜佛与悟禅心重故不以此为念。今日乃正月十三到晚就试灯后日十五上元直至十八九方才谢灯。我这里人家好事本府太守老爷爱民各地方俱高张灯火彻夜笙箫。还有个金灯桥乃上古传留至今丰盛。老爷们宽住数日我荒山颇管待得起。”唐僧无奈遂俱住下。当晚只听得佛殿上钟鼓喧天乃是街坊众信人等送灯来献佛唐僧等都出方丈来看了灯各自归寝。
	
	次日寺僧又献斋。吃罢同步后园闲要。果然好个去处正是:时维正月岁届新春。园林幽雅景物妍森。四时花木争奇一派峰峦迭翠。芳草阶前萌动老梅枝上生馨。红入桃花嫩青归柳色新。金谷园富丽休夸《辋川图》流风慢说。水流一道野凫出没无常;竹种千竿墨客推敲未定。芍药花、牡丹花、紫薇花、含笑花天机方醒;山茶花、红梅花、迎春花、瑞香花艳质先开。阴崖积雪犹含冻远树浮烟已带春。又见那鹿向池边照影鹤来松下听琴。东几厦西几亭客来留宿;南几堂北几塔僧静安禅。花卉中有一两座养性楼重檐高拱;山水内有三四处炼魔室静几明窗。真个是天然堪隐逸又何须他处觅蓬瀛。师徒们玩赏一日殿上看了灯又都去看灯游戏。
	
	但见那:玛瑙花城琉璃仙洞水晶云母诸宫:似重重锦绣迭迭玲珑。星桥影幌乾坤动看数株火树摇红。六街箫鼓千门璧月万户香风。几处鳌峰高耸有鱼龙出海鸾凤腾空。羡灯光月色和气融融。绮罗队里人人喜听笙歌车马轰轰。看不尽花容玉貌风流豪侠佳景无穷。众等既在本寺里看了灯又到东门厢各街上游戏。到二更时方才回转安置。
	
	次日唐僧对众僧道:“弟子原有扫塔之愿趁今日上元佳节请院主开了塔门让弟子了此愿心。”众僧随开了门。沙僧取了袈裟随从唐僧到了一层就披了袈裟拜佛祷祝毕即将笤帚扫了一层卸了袈裟付与沙僧又扫二层一层层直扫上绝顶。那塔上层层有佛处处开窗扫一层赏玩赞美一层。
	
	扫毕下来已此天晚又都点上灯火。此夜正是十五元宵众僧道:“老师父我们前晚只在荒山与关厢看灯。今晚正节进城里看看金灯如何?”唐僧欣然从之同行者三人及本寺多僧进城看灯。正是:三五良宵节上元春色和。花灯悬闹市齐唱太平歌。又见那六街三市灯亮半空一鉴初升。那月如冯夷推上烂银盘这灯似仙女织成铺地锦。灯映月增一倍光辉;月照灯添十分灿烂。观不尽铁锁星桥看不了灯花火树。雪花灯、梅花灯春冰剪碎;绣屏灯、画屏灯五彩攒成。核桃灯、荷花灯灯楼高挂;青狮灯、白象灯灯架高檠。虾儿灯、鳖儿灯棚前高弄;羊儿灯、兔儿灯檐下精神。鹰儿灯、凤儿灯相连相并;虎儿灯、马儿灯同走同行。仙鹤灯、白鹿灯寿星骑坐;金鱼灯、长鲸灯李白高乘。鳌山灯神仙聚会;走马灯武将交锋。万千家灯火楼台十数里云烟世界。那壁厢索琅琅玉韂飞来;这壁厢毂辘辘香车辇过。看那红妆楼上倚着栏隔着帘并着肩携着手双双美女贪欢;绿水桥边闹吵吵锦簇簇醉醺醺笑呵呵对对游人戏彩。满城中箫鼓喧哗彻夜里笙歌不断。有诗为证诗曰:锦绣场中唱彩莲太平境内簇人烟。灯明月皎元宵夜雨顺风调大有年。
	
	此时正是金吾不禁乱烘烘的无数人烟有那跳舞的躧跷的装鬼的骑象的东一攒西一簇看之不尽。却才到金灯桥上唐僧与众僧近前看处原来是三盏金灯。那灯有缸来大上照着玲珑剔透的两层楼阁都是细金丝儿编成;内托着琉璃薄片其光幌月其油喷香。唐僧回问众僧道:“此灯是甚油?怎么这等异香扑鼻?”众僧道:“老师不知我这府后有一县名唤旻天县县有二百四十里。每年审造差徭共有二百四十家灯油大户。府县的各项差徭犹可惟有此大户甚是吃累每家当一年要使二百多两银子。此油不是寻常之油乃是酥合香油。
	
	这油每一两值价银二两每一斤值三十二两银子。三盏灯每缸有五百斤三缸共一千五百斤共该银四万八千两。还有杂项缴缠使用将有五万余两只点得三夜。”行者道:“这许多油三夜何以就点得尽?”众僧道:“这缸内每缸有四十九个大灯马都是灯草扎的把裹了丝绵有鸡子粗细只点过今夜见佛爷现了身明夜油也没了灯就昏了。”八戒在旁笑道:“想是佛爷连油都收去了。”众僧道:“正是此说满城里人家自古及今皆是这等传说。但油干了人俱说是佛祖收了灯自然五谷丰登;若有一年不干却就年成荒旱风雨不调。所以人家都要这供献。”
	
	正说处只听得半空中呼呼风响唬得些看灯的人尽皆四散。那些和尚也立不住脚道:“老师父回去罢风来了。是佛爷降祥到此看灯也。”唐僧道:“怎见得是佛来看灯?”众僧道:
	
	“年年如此不上三更就有风来知道是诸佛降祥所以人皆回避。”唐僧道:“我弟子原是思佛念佛拜佛的人今逢佳景果有诸佛降临就此拜拜多少是好。”众僧连请不回。少时风中果现出三位佛身近灯来了。慌得那唐僧跑上桥顶倒身下拜。行者急忙扯起道:“师父不是好人必定是妖邪也。”说不了见灯光昏暗呼的一声把唐僧抱起驾风而去。噫!不知是那山那洞真妖怪积年假佛看金灯。唬得那八戒两边寻找沙僧左右招呼。行者叫道:“兄弟!不须在此叫唤师父乐极生悲已被妖精摄去了!”那几个和尚害怕道:“爷爷怎见得是妖精摄去?”行者笑道:“原来你这伙凡人累年不识故被妖邪惑了只说是真佛降祥受此灯供。刚才风到处现佛身者就是三个妖精。我师父亦不能识上桥顶就拜却被他侮暗灯光将器皿盛了油连我师父都摄去。我略走迟了些儿所以他三个化风而遁。”沙僧道:“师兄这般却如之何?”行者道:“不必迟疑。你两个同众回寺看守马匹行李等老孙趁此风追赶去也。”
	
	好大圣急纵筋斗云起在半空闻着那腥风之气往东北上径赶。赶至天晓倐尔风息见有一座大山十分险峻着实嵯峨。好山:重重丘壑曲曲源泉。藤萝悬削壁松柏挺虚岩。
	
	鹤鸣晨雾里雁唳晓云间。峨峨矗矗峰排戟突突磷磷石砌磐。
	
	顶巅高万仞峻岭迭千湾。野花佳木知春杜宇黄莺应景妍。
	
	能巍奕实巉岩古怪崎岖险又艰。停玩多时人不语只听虎豹有声鼾。香獐白鹿随来往玉兔青狼去复还。深涧水流千万里回湍激石响潺潺。大圣在山崖上正自找寻路径只见四个人赶着三只羊从西坡下齐吆喝“开泰”。大圣闪火眼金睛仔细观看认得是年、月、日、时四值功曹使者隐像化形而来。大圣即掣出铁棒幌一幌碗来粗细有丈二长短跳下崖来喝道:
	
	“你都藏头缩颈的那里走!”四值功曹见他说出风息慌得喝散三羊现了本相闪下路旁施礼道:“大圣恕罪!恕罪!”行者道:“这一向也不曾用着你们你们见老孙宽慢都一个个弄懈怠了见也不来见我一见!是怎么说!你们不在暗中保祐吾师都往那里去?”功曹道:“你师父宽了禅性在于金平府慈云寺贪欢所以泰极生否乐盛成悲今被妖邪捕获。他身边有护法伽蓝保着哩吾等知大圣连夜追寻恐大圣不识山林特来传报。”行者道:“你既传报怎么隐姓埋名赶着三个羊儿吆吆喝喝作甚?”功曹道:“设此三羊以应开泰之言唤做三阳开泰破解你师之否塞也。”行者恨恨的要打见有此意却就免之收了棒回嗔作喜道:“这座山可是妖精之处?”功曹道:
	
	“正是正是。此山名青龙山内有洞名玄英洞洞中有三个妖精:大的个名辟寒大王第二个号辟暑大王第三个号辟尘大王这妖精在此有千年了。他自幼儿爱食酥合香油。当年成精到此假装佛像哄了金平府官员人等设立金灯灯油用酥合香油。他年年到正月半变佛像收油;今年见你师父他认得是圣僧之身连你师父都摄在洞内不日要割剐你师之肉使酥合香油煎吃哩。你快用工夫救援去也。”行者闻言喝退四功曹转过山崖找寻洞府。行未数里只见那涧边有一石崖崖下是座石屋屋有两扇石门半开半掩。门旁立有石碣上有六字却是青龙山玄英洞。行者不敢擅入立定步叫声:“妖怪!
	
	快送我师父出来!”那里唿喇一声大开了门跑出一阵牛头精邓邓呆呆的问道:“你是谁敢在这里呼唤!”行者道:“我本是东土大唐取经的圣僧唐三藏之大徒弟路过金平府观灯我师被你家魔头摄来快早送还免汝等性命!如或不然掀翻你窝巢教你群精都化为脓血!”
	
	那些小妖听言急入里边报道:“大王!祸事了!祸事了!”
	
	三个老妖正把唐僧拿在那洞中深远处那里问甚么青红皂白教小的选剥了衣裳汲湍中清水洗净算计要细切细锉着酥合香油煎吃忽闻得报声“祸事”老大着惊问是何故。小妖道:“大门前有一个毛脸雷公嘴的和尚嚷道:大王摄了他师父来教快送出去免吾等性命;不然就要掀翻窝巢教我们都化为脓血哩!”那老妖听说个个心惊道:“才拿了这厮还不曾问他个姓名来历。小的们且把衣服与他穿了带过来审他一审端是何人何自而来也。”众妖一拥上前把唐僧解了索穿了衣服推至座前唬得唐僧战兢兢的跪在下面只叫:“大王饶命饶命!”三个妖精异口同声道:“你是那方来的和尚?怎么见佛像不躲却冲撞我的云路?”唐僧磕头道:“贫僧是东土大唐驾下差来的前往天竺国大雷音寺拜佛祖取经的。因到金平府慈云寺打斋蒙那寺僧留过元宵看灯。正在金灯桥上见大王显现佛像贫僧乃肉眼凡胎见佛就拜故此冲撞大王云路。”那妖精道:“你那东土到此路程甚远一行共有几众都叫甚名字快实实供来我饶你性命。”唐僧道:“贫僧俗名陈玄奘自幼在金山寺为僧。后蒙唐皇敕赐在长安洪福寺为僧官。
	
	又因魏徵丞相梦斩泾河老龙唐王游地府回生阳世开设水陆大会度阴魂蒙唐王又选赐贫僧为坛主大阐都纲。幸观世音菩萨出现指化贫僧说西天大雷音寺有三藏真经可以度亡者升天差贫僧来取因赐号三藏即倚唐为姓所以人都呼我为唐三藏。我有三个徒弟大的个姓孙名悟空行者乃齐天大圣归正。”群妖闻得此名着了一惊道:“这个齐天大圣可是五百年前大闹天宫的?”唐僧道:“正是正是。第二个姓猪名悟能八戒乃天蓬大元帅转世。第三个姓沙名悟净和尚乃卷帘大将临凡。”三个妖王听说个个心惊道:“早是不曾吃他。小的们且把唐僧将铁链锁在后面待拿他三个徒弟来凑吃。”遂点了一群山牛精、水牛精、黄牛精各持兵器走出门掌了号头摇旗擂鼓。
	
	三个妖披挂整齐都到门外喝道:“是谁人敢在我这里吆喝!”行者闪在石崖上仔细观看那妖精生得:彩面环睛二角峥嵘。尖尖四只耳灵窍闪光明。一体花纹如彩画满身锦绣若蜚英。第一个头顶狐裘花帽暖一脸昂毛热气腾;第二个身挂轻纱飞烈焰四蹄花莹玉玲玲;第三个威雄声吼如雷振獠牙尖利赛银针。个个勇而猛手持三样兵:一个使钺斧一个大刀能;但看第三个肩上横担扢挞藤。又见那七长八短、七肥八瘦的大大小小妖精都是些牛头鬼怪各执枪棒。有三面大旗旗上明明书着“辟寒大王”、“辟暑大王”、“辟尘大王”。孙行者看了一会忍耐不得上前高叫道:“泼贼怪!认得老孙么?”
	
	那妖喝道:“你是那闹天宫的孙悟空?真个是闻名不曾见面见面羞杀天神!你原来是这等个猢狲儿敢说大话!”行者大怒骂道:“我把你这个偷灯油的贼!油嘴妖怪不要胡谈!快还我师父来!”赶近前轮铁棒就打。那三个老妖举三般兵器急架相迎。这一场在山凹中好杀:钺斧钢刀扢挞藤猴王一棒敢来迎。辟寒辟暑辟尘怪认得齐天大圣名。棒起致令神鬼怕斧来刀砍乱飞腾。好一个混元有法真空像!抵住三妖假佛形。那三个偷油润鼻今年犯务捉钦差驾下僧。这个因师不惧山程远那个为嘴常年设献灯。乒乓只听刀斧响劈朴惟闻棒有声。
	
	冲冲撞撞三攒一架架遮遮各显能。一朝斗至天将晚不知那个亏输那个赢。孙行者一条棒与那三个妖魔斗经百五十合天色将晚胜负未分。只见那辟尘大王把扢挞藤闪一闪跳过阵前将旗摇了一摇那伙牛头怪簇拥上前把行者围在垓心各轮兵器乱打将来。行者见事不谐唿喇的纵起筋斗云败阵而走。那妖更不来赶招回群妖安排些晚食众各吃了。也叫小妖送一碗与唐僧只待拿住孙行者等才要整治。那师父一则长斋二则愁苦哭啼啼的未敢沾唇不题。
	
	却说行者驾云回至慈云寺内叫声“师弟!”那八戒沙僧正自盼望商量听得叫时一齐出接道:“哥哥如何去这一日方回?端的师父下落何如?”行者笑道:“昨夜闻风而赶至天晓到一山不见。幸四值功曹传信道:那山叫做青龙山山中有一玄英洞。洞中有三个妖精唤做辟寒大王、辟暑大王、辟尘大王。
	
	原来积年在此偷油假变佛像哄了金平府官员人等。今年遇见我们他不知好歹反连师父都摄去。老孙审得此情吩咐功曹等众暗中保护师父我寻近门前叫骂。那三怪齐出都象牛头鬼形。大的个使钺斧第二个使大刀第三个使藤棍后引一窝子牛头鬼怪摇旗擂鼓与老孙斗了一日杀个手平。那妖王摇动旗小妖都来我见天晚恐不能取胜所以驾筋斗回来也。”八戒道:“那里想是酆都城鬼王弄喧。”沙僧道:“你怎么就猜道是酆都城鬼王弄喧?”八戒笑道:“哥哥说是牛头鬼怪故知之耳。”行者道:“不是!不是!若论老孙看那怪是三只犀牛成的精。”八戒道:“若是犀牛且拿住他锯下角来倒值好几两银子哩!”正说处众僧道:“孙老爷可吃晚斋?”行者道:“方便吃些儿不吃也罢。”众僧道:“老爷征战这一日岂不饥了?”
	
	行者笑道:“这日把儿那里便得饥!老孙曾五百年不吃饮食哩!”众僧不知是实只以为说笑。须臾拿来行者也吃了道:
	
	“且收拾睡觉待明日我等都去相持拿住妖王庶可救师父也。”沙僧在旁道:“哥哥说那里话!常言道停留长智。那妖精倘或今晚不睡把师父害了却如之何?不若如今就去嚷得他措手不及方才好救师父。少迟恐有失也。”八戒闻言抖擞神威道:“沙兄弟说得是!我们都趁此月光去降魔耶!”行者依言即吩咐寺僧:“看守行李马匹待我等把妖精捉来对本府刺史证其假佛免却灯油以苏概县小民之困却不是好?”众僧领诺称谢不已。他三个遂纵起祥云出城而去。正是那:懒散无拘禅性乱灾危有分道心蒙。毕竟不知此去胜败何如且听下回分解。
	------------
	
	第九十二回 三僧大战青龙山 四星挟捉犀牛怪
	
	却说孙大圣挟同二弟滚着风驾着云向东北艮地上顷刻至青龙山玄英洞口按落云头。(WWW.mianhuatang.la 好看的小说)八戒就欲筑门行者道:“且消停待我进去看看师父生死如何再好与他争持。”沙僧道:
	
	“这门闭紧如何得进?”行者道:“我自有法力。”好大圣收了棒捻着诀念声咒语叫“变!”即变做个火焰虫儿。真个也疾伶!你看他:展翅星流光灿古云腐草为萤。神通变化不非轻自有徘徊之性。飞近石门悬看旁边瑕缝穿风。将身一纵到幽庭打探妖魔动静。他自飞入只见几只牛横敧直倒一个个呼吼如雷尽皆睡熟。又至中厅里面全无消息。四下门户通关不知那三个妖精睡在何处。才转过厅房向后又照只闻得啼泣之声乃是唐僧锁在后房檐柱上哭哩。行者暗暗听他哭甚只见他哭道:“一别长安十数年登山涉水苦熬煎。幸来西域逢佳节喜到金平遇上元。不识灯中假佛像概因命里有灾愆。贤徒追袭施威武但愿英雄展大权。”行者闻言满心欢喜展开翅飞近师前。唐僧揩泪道:“呀!西方景象不同此时正月蛰虫始振为何就有萤飞?”行者忍不住叫声:“师父我来了!”
	
	唐僧喜道:“悟空我心说正月怎得萤火原来是你。”行者即现了本相道:“师父啊为你不识真假误了多少路程费了多少心力。我一行说不是好人你就下拜却被这怪侮暗灯光盗取酥合香油连你都摄将来了。我当吩咐八戒沙僧回寺看守我即闻风追至此间不识地名幸遇四值功曹传报说此山名青龙山玄英洞。我日间与此怪斗至天晚方回与师弟辈细道此情却就不曾睡同他两个来此。我恐夜深不便交战又不知师父下落所以变化进来打听师情。”唐僧喜道:“八戒沙僧如今在外边哩?”行者道:“在外边才子老孙看时妖精都睡着。我且解了锁搠开门带你出去罢。”唐僧点头称谢。
	
	行者使个解锁法用手一抹那锁早自开了领着师父往前正走忽听得妖王在中厅内房里叫道:“小的们紧闭门户小心火烛。这会怎么不叫更巡逻梆铃都不响了?”原来那伙小妖征战一日俱辛辛苦苦睡着听见叫唤却才醒了。梆铃响处有几个执器械的敲着锣从后而走可可的撞着他师徒两个。众小妖一齐喊道:“好和尚啊!扭开锁往那里去!”行者不容分说掣出棒幌一幌碗来粗细就打。棒起处打死两个其余的丢了器械近中厅打着门叫:“大王!不好了!不好了!毛脸和尚在家里打杀人了!”那三怪听见一毂辘爬将起来只教“拿住!拿住!”唬得个唐僧手软脚软。行者也不顾师父一路棒滚向前来。众小妖遮架不住被他放倒三两个推倒两三个打开几层门径自出来叫道:“兄弟们何在?”八戒沙僧正举着钯杖等待道:“哥哥如何了?”行者将变化入里解放师父正走被妖惊觉顾不得师父打出来的事讲说一遍不题。
	
	那妖王把唐僧捉住依然使铁索锁了执着刀轮着斧灯火齐明问道:“你这厮怎样开锁那猴子如何得进快早供来饶你之命!不然就一刀两段!”慌得那唐僧战战兢兢的跪道:
	
	“大王爷爷!我徒弟孙悟空他会七十二般变化。才变个火焰虫儿飞进来救我。不期大王知觉被小大王等撞见是我徒弟不知好歹打伤两个众皆喊叫举兵着火他遂顾不得我走出去了。”三个妖王呵呵大笑道:“早是惊觉未曾走了!”叫小的们把前后门紧紧关闭亦不喧哗。沙僧道:“闭门不喧哗想是暗弄我师父我们动手耶!”行者道:“说的是快早打门。”那呆子卖弄神通举钯尽力筑去把那石门筑得粉碎却又厉声喊骂道:“偷油的贼怪!快送吾师出来也!”唬得那门内小妖滚将进去报道:“大王不好了!不好了!前门被和尚打破了!”三个妖王十分烦恼道:“这厮着实无礼!”即命取披挂结束了各持兵器帅小妖出门迎敌。此时约有三更时候半天中月明如昼。走出来更不打话便就轮兵。这里行者抵住钺斧八戒敌住大刀沙僧迎住大棍。这场好杀:僧三众棍杖钯三个妖魔胆气加。钺斧钢刀藤纥褡只闻风响并尘沙。初交几合喷愁雾次后飞腾散彩霞钉钯解数随身滚铁棒英豪更可夸。降妖宝杖人间少妖怪顽心不让他。钺斧口明尖鐏利藤条节懞一身花。大刀幌亮如门扇和尚神通偏赛他。这壁厢因师性命狠打那壁厢不放唐僧劈脸挝。斧剁棒迎争胜负钯轮刀砍两交搽。扢挞藤条降怪杖翻翻复复逞豪华。三僧三怪赌斗多时不见输赢。那辟寒大王喊一声叫:“小的们上来!”众精各执兵刃齐来早把个八戒绊倒在地被几个水牛精揪揪扯扯拖入洞里捆了。沙僧见没了八戒只见那群牛喊咙声。即掣宝杖望辟尘大王虚丢了架子要走又被群精一拥而来拉了个躘踵急挣不起也被捉去捆了。行者觉道难为纵筋斗云脱身而去。当时把八戒沙僧拖至唐僧前。唐僧见了满眼垂泪道:
	
	“可怜你二人也遭了毒手!悟空何在?”沙僧道:“师兄见捉住我们他就走了。”唐僧道:“他既走了必然那里去求救。但我等不知何日方得脱网。”师徒们凄凄惨惨不题。
	
	却说行者驾筋斗云复至慈云寺寺僧接着来问:“唐老爷救得否?”行者道:“难救!难救!那妖精神通广大我弟兄三个与他三个斗了多时被他呼小妖先捉了八戒后捉了沙僧老孙幸走脱了。”众僧害怕道:“爷爷这般会腾云驾雾还捉获不得想老师父被倾害也。”行者道:“不妨!不妨!我师父自有伽蓝、揭谛、丁甲等神暗中护佑却也曾吃过草还丹料不伤命只是那妖精有本事。汝等可好看马匹行李等老孙上天去求救兵来。”众僧胆怯道:“爷爷又能上天?”行者笑道:“天宫原是我的旧家。当年我做齐天大圣因为乱了蟠桃会被我佛收降如今没奈何保唐僧取经将功折罪。一路上辅正除邪我师父该有此难汝等却不知也。”众僧听此言又磕头礼拜。行者出得门打个唿哨即时不见。
	
	好大圣早至西天门外忽见太白金星与增长天王殷、朱、陶、许四大灵官讲话。他见行者来都慌忙施礼道:“大圣那里去?”行者道:“因保唐僧行至天竺国东界金平府旻天县我师被本县慈云寺僧留赏元宵。比至金灯桥有金灯三盏点灯用酥合香油价贵白金五万余两年年有诸佛降祥受用。正看时果有三尊佛像降临我师不识好歹上桥就拜。我说不是好人早被他侮暗灯光连油并我师一风摄去。我随风追袭至天晓到一山幸四功曹报道那山名青龙山山有玄英洞洞有三怪名辟寒大王、辟暑大王、辟尘大王。老孙急上门寻讨与他赌斗一阵未胜。是我变化入里见师父锁住未伤随解了欲出又被他知觉我遂走了。后又同八戒沙僧苦战复被他将二人也捉去捆了。老孙因此特启玉帝查他来历请命将降之。”
	
	金星呵呵冷笑道:“大圣既与妖怪相持岂看不出他的出处?”
	
	行者道:“认便认得是一伙牛精。只是他大有神通急不能降也。”金星道:“那是三个犀牛之精。他因有天文之象累年修悟成真亦能飞云步雾。其怪极爱干净常嫌自己影身每欲下水洗浴。他的名色也多:有兕犀有雄犀有牯犀有斑犀又有胡冒犀、堕罗犀、通天花文犀都是一孔三毛二角行于江海之中能开水道。似那辟寒、辟暑、辟尘都是角有贵气故以此为名而称大王也。若要拿他只是四木禽星见面就伏。”行者连忙唱喏问道:“是那四木禽星?烦长庚老一一明示明示。(WWW.mianhuatang.la 好看的小说)”金星笑道:“此星在斗牛宫外罗布乾坤。你去奏闻玉帝便见分晓。”
	
	行者拱拱手称谢径入天门里去。
	
	不一时到于通明殿下先见葛邱张许四大天师。天师问道:“何往?”行者道:“近行至金平府地方因我师宽放禅性元夜观灯遇妖魔摄去。老孙不能收降特来奏闻玉帝求救。”四天师即领行者至灵霄宝殿启奏。各各礼毕备言其事玉帝传旨:“教点那路天兵相助?”行者奏道:“老孙才到西天门遇长庚星说那怪是犀牛成精惟四木禽星可以降伏。”玉帝即差许天师同行者去斗牛宫点四木禽星下界收降。
	
	及至宫外早有二十八宿星辰来接天师道:“吾奉圣旨教点四木禽星与孙大圣下界降妖。”旁即闪过角木蛟、斗木獬、奎木狼、井木犴应声呼道:“孙大圣点我等何处降妖?”行者笑道:“原来是你。这长庚老儿却隐匿我不解其意早说是二十八宿中的四木老孙径来相请又何必劳烦旨意?”四木道:“大圣说那里话!我等不奉旨意谁敢擅离?端的是那方?快早去来。”行者道:“在金平府东北艮地青龙山玄英洞犀牛成精。”
	
	斗木獬、奎木狼、角木蛟道:“若果是犀牛成精不须我们只消井宿去罢。他能上山吃虎下海擒犀。”行者道:“那犀不比望月之犀乃是修行得道都有千年之寿者。须得四位同去才好切勿推调倘一时一位拿他不住却不又费事了?”天师道:“你们说得是甚话!旨意着你四人岂可不去?趁早飞行我回旨去也。”那天师遂别行者而去。四木道:“大圣不必迟疑你先去索战引他出来我们随后动手。”行者即近前骂道:“偷油的贼怪!还我师来!”原来那门被八戒筑破几个小妖弄了几块板儿搪住在里边听得骂詈急跑进报道:“大王孙和尚在外面骂哩!”辟尘儿道:“他败阵去了这一日怎么又来?想是那里求些救兵来了。”辟寒、辟暑道:“怕他甚么救兵!快取披挂来!小的们都要用心围绕休放他走了。”那伙精不知死活一个个各执枪刀摇旗擂鼓走出洞来对行者喝道:“你个不怕打的猢狲儿你又来了!”行者最恼得是这猢狲二字咬牙狠举铁棒就打。三个妖王调小妖跑个圈子阵把行者圈在垓心。那壁厢四木禽星一个个各轮兵刃道:“孽畜!休动手!”那三个妖王看他四星自然害怕俱道:“不好了!不好了!他寻将降手儿来了!小的们各顾性命走耶!”只听得呼呼吼吼喘喘呵呵众小妖都现了本身:原来是那山牛精、水牛精、黄牛精满山乱跑。那三个妖王也现了本相放下手来还是四只蹄子就如铁炮一般径往东北上跑。这大圣帅井木犴、角木蛟紧追急赶略不放松。惟有斗木獬、奎木狼在东山凹里、山头上、山涧中、山谷内把些牛精打死的、活捉的尽皆收净。却向玄英洞里解了唐僧、八戒、沙僧。沙僧认得是二星随同拜谢因问:“二位如何到此相救?”二星道:“吾等是孙大圣奏玉帝请旨调来收怪救你也。”唐僧又滴泪道:“我悟空徒弟怎么不见进来?”二星道:“那三个老怪是三只犀牛他见吾等各各顾命向东北艮方逃遁。孙大圣帅井木犴、角木蛟追赶去了。我二星扫荡群牛到此特来解放圣僧。”唐僧复又顿拜谢朝天又拜八戒搀起道:“师父礼多必诈不须只管拜了。四星官一则是玉帝圣旨二则是师兄人情。今既扫荡群妖还不知老妖如何降伏我们且收拾些细软东西出来掀翻此洞以绝其根回寺等候师兄罢。”奎木狼道:“天蓬元帅说得有理。你与卷帘大将保护你师回寺安歇待吾等还去艮方迎敌。”八戒道:“正是正是你二位还协同一捉必须剿尽方好回旨。”二星官即时追袭。八戒与沙僧将他洞内细软宝贝有许多珊瑚、玛瑙、珍珠、琥珀、琗琚、宝贝、美玉、良金搜出一石搬在外面请师父到山崖上坐了他又进去放起火来把一座洞烧成灰烬却才领唐僧找路回金平慈云寺去。正是:经云泰极还生否好处逢凶实有之。
	
	爱赏花灯禅性乱喜游美景道心漓。大丹自古宜长守一失原来到底亏。紧闭牢拴休旷荡须臾懈怠见参差。
	
	且不言他三众得命回寺却表斗木獬、奎木狼二星官驾云直向东北艮方赶妖怪来。二人在那半空中寻看不见直到西洋大海远望见孙大圣在海上吆喝。他两个按落云头道:“大圣妖怪那里去了?”行者恨道:“你两个怎么不来追降?这会子却冒冒失失的问甚?”斗木獬道:“我见大圣与井、角二星战败妖魔追赶料必擒拿。我二人却就扫荡群精入玄英洞救出你师父、师弟。搜了山烧了洞把你师父付托与你二弟领回府城慈云寺。多时不见车驾回转故又追寻到此也。”行者闻言方才喜谢道:“如此却是有功多累!多累!但那三个妖魔被我赶到此间他就钻下海去。当有井、角二星紧紧追拿教老孙在岸边抵挡。你两个既来且在岸边把截等老孙也再去来。”
	
	好大圣轮着棒捻着诀辟开水径直入波涛深处只见那三个妖魔在水底下与井木犴、角木蛟舍死忘生苦斗哩。他跳近前喊道:“老孙来也!”那妖精抵住二星官措手不及正在危难之处忽听得行者叫喊顾残生拨转头往海心里飞跑。原来这怪头上角极能分水只闻得花花花冲开明路。这后边二星官并孙大圣并力追之。
	
	却说西海中有个探海的夜叉巡海的介士远见犀牛分开水势又认得孙大圣与二天星即赴水晶宫对龙王慌慌张张报道:“大王!有三只犀牛被齐天大圣和二位天星赶来也!”老龙王敖顺听言即唤太子摩昂:“快点水兵想是犀牛精辟寒、辟暑、辟尘儿三个惹了孙行者。今既至海快快拔刀相助。”敖摩昂得令即忙点兵。顷刻间龟鳖鼋鼍鯾鱼白鳜鲤与虾兵蟹卒等各执枪刀一齐呐喊腾出水晶宫外挡住犀牛精。犀牛精不能前进急退后又有井、角二星并大圣拦阻慌得他失了群各各逃生四散奔走早把个辟尘儿被老龙王领兵围住。孙大圣见了心欢叫道:“消停消停!捉活的不要死的。”摩昂听令一拥上前将辟尘儿扳翻在地用铁钩子穿了鼻攒蹄捆倒。
	
	老龙王又传号令教分兵赶那两个协助二星官擒拿。即时小龙王帅众前来只见井木犴现原身按住辟寒儿大口小口的啃着吃哩。摩昂高叫道:“井宿!井宿!莫咬死他孙大圣要活的不要死的哩。”连喊数喊已是被他把颈项咬断了。摩昂吩咐虾兵蟹卒将个死犀牛抬转水晶宫却又与井木犴向前追赶。只见角木蛟把那辟暑儿倒赶回来只撞着井宿。摩昂帅龟鳖鼋鼍撒开簸箕阵围住那怪只教:“饶命!饶命!”井木犴走近前一把揪住耳朵夺了他的刀叫道:“不杀你!不杀你!
	
	拿与孙大圣落去来。”当即倒干戈复至水晶宫外报道:“都捉来也。”行者见一个断了头血淋津的倒在地下一个被井木犴拖着耳朵推跪在地近前仔细看了道:“这头不是兵刀伤的啊。”摩昂笑道:“不是我喊得紧连身子都着井星官吃了。”行者道:“既是如此也罢取锯子来锯下他的这两只角剥了皮带去。犀牛肉还留与龙王贤父子享之。”又把辟尘儿穿了鼻教角木蛟牵着;辟暑儿也穿了鼻教井木犴牵着:“带他上金平府见那刺史官明究其由问他个积年假佛害民然后的决。”
	
	众等遵言辞龙王父子都出西海牵着犀牛会着奎、斗二星驾云雾径转金平府。行者足踏祥光半空中叫道:“金平府刺史、各佐贰郎官并府城内外军民人等听着:吾乃东土大唐差往西天取经的圣僧。你这府县每年家供献金灯假充诸佛降祥者即此犀牛之怪。我等过此因元夜观灯见这怪将灯油并我师父摄去是我请天神收伏。今已扫清山洞剿尽妖魔不得为害以后你府县再不可供献金灯劳民伤财也。”那慈云寺里八戒沙僧方保唐僧进得山门只听见行者在半空言语即便撇了师父丢下担子纵风云起到空中问行者降妖之事。行者道:“那一只被井星咬死已锯角剥皮带来两只活拿在此。”
	
	八戒道:“这两个索性推下此城与官员人等看看也认得我们是圣是神左右累四位星官收云下地同到府堂将这怪的决。
	
	已此情真罪当再有甚讲!”四星道:“天蓬帅近来知理明律却好呀!”八戒道:“因做了这几年和尚也略学得些儿。”
	
	众神果推落犀牛一簇彩云降至府堂之上。唬得这府县官员城里城外人等都家家设香案户户拜天神。少时间慈云寺僧把长老用轿抬进府门会着行者口中不离“谢”字道:
	
	“有劳上宿星官救出我等因不见贤徒悬悬在念今幸得胜而回!然此怪不知赶向何方才捕获也!”行者道:“自前日别了尊师老孙上天查访蒙太白金星识得妖魔是犀牛指示请四木禽星。当时奏闻玉帝蒙旨差委直至洞**战。妖王走了又蒙斗、奎二宿救出尊师。老孙与井、角二宿并力追妖直赶到西洋大海又亏龙王遣子帅兵相助所以捕获到此审究也。”长老赞扬称谢不已。又见那府县正官并佐贰领都在那里高烧宝烛满斗焚香朝上礼拜。少顷间八戒起性来掣出戒刀将辟尘儿头一刀砍下又一刀把辟暑儿头也砍下随即取锯子锯下四只角来。孙大圣更有主张就教:“四位星官将此四只犀角拿上界去进贡玉帝回缴圣旨。”把自己带来的二只:“留一只在府堂镇库以作向后免征灯油之证;我们带一只去献灵山佛祖。”四星心中大喜即时拜别大圣忽驾彩云回奏而去。
	
	府县官留住他师徒四众大排素宴遍请乡官陪奉。一壁厢出给告示晓谕军民人等下年不许点设金灯永蠲买油大户之役;一壁厢叫屠子宰剥犀牛之皮硝熟熏干制造铠甲把肉普给官员人等;又一壁厢动支枉罚无碍钱粮买民间空地起建四星降妖之庙;又为唐僧四众建立生祠各各树碑刻文用传千古以为报谢。师徒们索性宽怀领受又被那二百四十家灯油大户这家酬那家请略无虚刻。八戒遂心满意受用把洞里搜来的宝物每样各笼些须在袖以为各家斋筵之赏。
	
	住经个月犹不得起身长老吩咐:“悟空将余剩的宝物尽送慈云寺僧以为酬礼。瞒着那些大户人家天不明走罢。恐只管贪乐误了取经惹佛祖见罪、又生灾厄深为不便。”行者随将前件一一处分。
	
	次日五更早起唤八戒备马。那呆子吃了自在酒饭睡得梦梦乍道:“这早备马怎的?”行者喝道:“师父教走路哩!”呆子抹抹脸道:“又是这长老没正经!二百四十家大户都请才吃了有三十几顿饱斋怎么又弄老猪忍饿!”长老听言骂道:“馕糟的夯货!莫胡说!快早起来!再若强嘴教悟空拿金箍棒打牙!”
	
	那呆子听见说打慌了手脚道:“师父今番变了常时疼我爱我念我蠢夯护我哥要打时他又劝解;今日怎么狠转教打么?”行者道:“师父怪你为嘴误了路程快早收拾行李备马免打!”那呆子真个怕打跳起来穿了衣服吆喝沙僧:“快起来!
	
	打将来了!”沙僧也随跳起各各收拾皆完。长老摇手道:“寂寂悄悄的不要惊动寺僧。”连忙上马开了山门找路而去。这一去正所谓:暗放玉笼飞彩凤私开金锁走蛟龙。毕竟不知天明时酬谢之家端的如何且听下回分解。
	------------
	
	第九十三回 给孤园问古谈因 天竺国朝王遇偶
	
	起念断然有爱留情必定生灾。灵明何事辨三台?行满自归元海。不论成仙成佛须从个里安排。清清净净绝尘埃果正飞升上界。却说寺僧天明不见了三藏师徒都道:“不曾留得不曾别得不曾求告得清清的把个活菩萨放得走了!”正说处只见南关厢有几个大户来请众僧扑掌道:“昨晚不曾防御今夜都驾云去了。”众人齐望空拜谢。此言一讲满城中官员人等尽皆知之叫此大户人家俱治办五牲花果往生祠祭献酬恩不题。
	
	却说唐僧四众餐风宿水一路平宁行有半个多月。忽一日见座高山唐僧又悚惧道:“徒弟那前面山岭峻峭是必小心!”行者笑道:“这边路上将近佛地断乎无甚妖邪师父放怀勿虑。”唐僧道:“徒弟虽然佛地不远。但前日那寺僧说到天竺国都下有二千里还不知是有多少路哩。”行者道:“师父你好是又把乌巢禅师《心经》忘记了也?”三藏道:“《般若心经》是我随身衣钵。自那乌巢禅师教后那一日不念那一时得忘?颠倒也念得来怎会忘得!”行者道:“师父只是念得不曾求那师父解得。”三藏说:“猴头!怎又说我不曾解得!你解得么?”行者道:“我解得我解得。”自此三藏、行者再不作声。旁边笑倒一个八戒喜坏一个沙僧说道:“嘴脸!替我一般的做妖精出身又不是那里禅和子听过讲经那里应佛僧也曾见过说法?弄虚头找架子说甚么晓得解得!怎么就不作声?听讲!
	
	请解!”沙僧说:“二哥你也信他。大哥扯长话哄师父走路。他晓得弄棒罢了他那里晓得讲经!”三藏道:“悟能悟净休要乱说悟空解得是无言语文字乃是真解。”
	
	他师徒们正说话间却倒也走过许多路程离了几个山冈路旁早见一座大寺。三藏道:“悟空前面是座寺啊你看那寺倒也不小不大却也是琉璃碧瓦;半新半旧却也是八字红墙。隐隐见苍松偃盖也不知是几千百年间故物到于今;潺潺听流水鸣弦也不道是那朝代时分开山留得在。山门上大书着布金禅寺;悬扁上留题着上古遗迹。”行者看得是布金禅寺八戒也道是布金禅寺三藏在马上沉思道:“布金布金这莫不是舍卫国界了么?”八戒道:“师父奇啊!我跟师父几年再不曾见识得路今日也识得路了。”三藏说道:“不是我常看经诵典说是佛在舍卫城祇树给孤园。这园说是给孤独长者问太子买了请佛讲经。太子说:‘我这园不卖。他若要买我的时除非黄金满布园地。’给孤独长者听说随以黄金为砖布满园地才买得太子祇园才请得世尊说法。我想这布金寺莫非就是这个故事?”八戒笑道:“造化!若是就是这个故事我们也去摸他块把砖儿送人。”大家又笑了一会三藏才下得马来。
	
	进得山门只见山门下挑担的背包的推车的整车坐下;也有睡的去睡讲的去讲。忽见他们师徒四众俊的又俊丑的又丑大家有些害怕却也就让开些路儿。三藏生怕惹事口中不住只叫:“斯文!斯文!”这时节却也大家收敛。转过金刚殿后早有一位禅僧走出却也威仪不俗。真是:面如满月光身似菩提树。拥锡袖飘风芒鞋石头路。三藏见了问讯。那僧即忙还礼道:“师从何来?”三藏道:“弟子陈玄奘奉东土大唐皇帝之旨差往西天拜佛求经。路过宝方造次奉谒便求借一宿明日就行。”那僧道:“荒山十方常住都可随喜况长老东土神僧但得供养幸甚。”三藏谢了随即唤他三人同行过了回廊香积径入方丈。相见礼毕分宾主坐定行者三人亦垂手坐了。
	
	话说这时寺中听说到了东土大唐取经僧人寺中若大若小不问长住、挂榻、长老、行童一一都来参见。茶罢摆上斋供。这时长老还正开斋念偈八戒早是要紧馒头、素食、粉汤一搅直下。这时方丈却也人多有知识的赞说三藏威仪好耍子的都看八戒吃饭。却说沙僧眼溜看见头底暗把八戒捏了一把说道:“斯文!”八戒着忙急的叫将起来说道:“斯文斯文!肚里空空!”沙僧笑道:“二哥你不晓的天下多少斯文若论起肚子里来正替你我一般哩。”八戒方才肯住。三藏念了结斋左右彻了席面三藏称谢。
	
	寺僧问起东土来因三藏说到古迹才问布金寺名之由。
	
	那僧答曰:“这寺原是舍卫国给孤独园寺又名祇园。因是给孤独长者请佛讲经金砖布地又易今名。我这寺一望之前乃是舍卫国那时给孤独长者正在舍卫国居住。我荒山原是长者之祇园因此遂名给孤布金寺寺后边还有祇园基址。近年间若遇时雨滂沱还淋出金银珠儿有造化的每每拾着。”三藏道:
	
	“话不虚传果是真!”又问道:“才进宝山见门下两廊有许多骡马车担的行商为何在此歇宿?”众僧道:“我这山唤做百脚山。
	
	先年且是太平近因天气循环不知怎的生几个蜈蚣精常在路下伤人。虽不至于伤命其实人不敢走。山下有一座关唤做鸡鸣关但到鸡鸣之时才敢过去。那些客人因到晚了惟恐不便权借荒山一宿等鸡鸣后便行。”三藏道:“我们也等鸡鸣后去罢。”师徒们正说处又见拿上斋来却与唐僧等吃毕。此时上弦月皎三藏与行者步月闲行又见个道人来报道:“我们老师爷要见见中华人物。”三藏急转身见一个老和尚手持竹杖向前作礼道:“此位就是中华来的师父?”三藏答礼道:“不敢。”老僧称赞不已。因问:“老师高寿?”三藏道:“虚度四十五年矣敢问老院主尊寿?”老僧笑道:“比老师痴长一花甲也。”
	
	行者道:“今年是一百零五岁了你看我有多少年纪?”老僧道:
	
	“师家貌古神清况月夜眼花急看不出来。”叙了一会又向后廊看看。三藏道:“才说给孤园基址果在何处?”老僧道:“后门外就是。”快教开门但见是一块空地还有些碎石迭的墙脚。
	
	三藏合掌叹曰:“忆昔檀那须达多曾将金宝济贫疴。祇园千古留名在长者何方伴觉罗?”
	
	他都玩着月缓缓而行行近后门外至台上又坐了一坐。
	
	忽闻得有啼哭之声三藏静心诚听哭的是爷娘不知苦痛之言。他就感触心酸不觉泪堕回问众僧道:“是甚人在何处悲切?”老僧见问即命众僧先回去煎茶见无人方才对唐僧行者下拜。三藏搀起道:“老院主为何行此礼?”老僧道:“弟子年岁百余略通人事。每于禅静之间也曾见过几番景象。若老爷师徒弟子聊知一二与他人不同。若言悲切之事非这位师家明辨不得。”行者道:“你且说是甚事?”老僧道:“旧年今日弟子正明性月之时忽闻一阵风响就有悲怨之声。弟子下榻到祇园基上看处乃是一个美貌端正之女。我问他:‘你是谁家女子?为甚到于此地?’那女子道:‘我是天竺国国王的公主。因为月下观花被风刮来的。’我将他锁在一间敝空房里将那房砌作个监房模样门上止留一小孔仅递得碗过。当日与众僧传道是个妖邪被我捆了但我僧家乃慈悲之人不肯伤他性命。每日与他两顿粗茶粗饭吃着度命。那女子也聪明即解吾意恐为众僧点污就装风作怪尿里眠屎里卧。白日家说胡话呆呆邓邓的;到夜静处却思量父母啼哭。我几番家进城乞化打探公主之事全然无损。故此坚收紧锁更不放出。今幸老师来国万望到了国中广施法力辨明辨明一则救拔良善二则昭显神通也。”三藏与行者听罢切切在心。正说处只见两个小和尚请吃茶安置遂而回去。
	
	八戒与沙僧在方丈中突突哝哝的道:“明日要鸡鸣走路此时还不来睡!”行者道:“呆子又说甚么?”八戒道:“睡了罢这等夜深还看甚么景致。”因此老僧散去唐僧就寝。正是那:人静月沉花梦悄暖风微透壁窝纱。铜壶点点看三汲银汉明明照九华。
	
	当夜睡还未久即听鸡鸣那前边行商烘烘皆起引灯造饭。这长老也唤醒八戒沙僧扣马收拾行者叫点灯来。那寺僧已先起来安排茶汤点心在后候敬。八戒欢喜吃了一盘馍馍把行李马匹牵出。三藏、行者对众辞谢老僧又向行者道:
	
	“悲切之事在心在心!”行者笑道:“谨领谨领!我到城中自能聆音而察理见貌而辨色也。”那伙行商哄哄嚷嚷的也一同上了大路将有寅时过了鸡鸣关。至巳时方见城垣真是铁瓮金城神洲天府。那城:虎踞龙蟠形势高凤楼麟阁彩光摇。
	
	御沟流水如环带福地依山插锦标。晓日旌旗明辇路春风箫鼓遍溪桥。国王有道衣冠胜五谷丰登显俊豪。
	
	当日入于东市街众商各投旅店。他师徒们进城正走处有一个会同馆驿三藏等径入驿内。那驿内管事的即报驿丞道:“外面有四个异样的和尚牵一匹白马进来了。”驿丞听说有马就知是官差的出厅迎迓。三藏施礼道:“贫僧是东土唐朝钦差灵山大雷音见佛求经的随身有关文入朝照验。借大人高衙一歇事毕就行。”驿丞答礼道:“此衙门原设待使客之处理当款迓请进请进。”三藏喜悦教徒弟们都来相见。那驿丞看见嘴脸丑陋暗自心惊不知是人是鬼战兢兢的只得看茶摆斋。三藏见他惊怕道:“大人勿惊我等三个徒弟相貌虽丑心地俱良俗谓山恶人善何以惧为!”驿丞闻言方才定了心性问道:“国师唐朝在于何方?”三藏道:“在南赡部洲中华之地。”又问:“几时离家?”三藏道:“贞观十三年今已历过十四载苦经了些万水千山方到此处。”驿丞道:“神僧!神僧!”三藏问道:“上国天年几何?”驿丞道:“我敝处乃大天竺国自太祖太宗传到今已五百余年。现在位的爷爷爱山水花卉号做怡宗皇帝改元靖宴今已二十八年了。”三藏道:“今日贫僧要去见驾倒换关文不知可得遇朝?”驿丞道:“好!好!
	
	正好!近因国王的公主娘娘年登二十青春正在十字街头高结彩楼抛打绣球撞天婚招驸马。今日正当热闹之际想我国王爷爷还未退期若欲倒换关文趁此时好去。”三藏欣然要走只见摆上斋来遂与驿丞、行者等吃了。
	
	时已过午三藏道:“我好去了。”行者道:“我保师父去。”
	
	八戒道:“我去。”沙僧道:“二哥罢么你的嘴脸不见怎的莫到朝门外装胖还教大哥去。”三藏道:“悟净说得好呆子粗夯悟空还有些细腻。”那呆子掬着嘴道:“除了师父我三个的嘴脸也差不多儿。”三藏却穿了袈裟行者拿了引袋同去。只见街坊上士农工商文人墨客愚夫俗子齐咳咳都道:“看抛绣球去也!”三藏立于道旁对行者道:“他这里人物衣冠宫室器用言语谈吐也与我大唐一般。我想着我俗家先母也是抛打绣球遇旧姻缘结了夫妇。此处亦有此等风俗。”行者道:“我们也去看看如何?”三藏道:“不可!不可!你我服色不便恐有嫌疑。”
	
	行者道:“师父你忘了那给孤布金寺老僧之言:一则去看彩楼二则去辨真假。似这般忙忙的那皇帝必听公主之喜报那里视朝理事?且去去来!”三藏听说真与行者相随见各项人等俱在那里看打绣球。呀!那知此去却是渔翁抛下钩和线从今钓出是非来。
	
	话表那个天竺国王因爱山水花卉前年带后妃、公主在御花园月夜赏玩惹动一个妖邪把真公主摄去他却变做一个假公主。知得唐僧今年今月今日今时到此他假借国家之富搭起彩楼欲招唐僧为偶采取元阳真气以成太乙上仙。
	
	正当午时三刻三藏与行者杂入人丛行近楼下那公主才拈香焚起祝告天地。左右有五七十胭娇绣女近侍的捧着绣球。
	
	那楼八窗玲珑公主转睛观看见唐僧来得至近将绣球取过来亲手抛在唐僧头上。唐僧着了一惊把个毗卢帽子打歪双手忙扶着那球那球毂辘的滚在他衣袖之内。那楼上齐声喊道:“打着个和尚了!打着个和尚了!”噫!十字街头那些客商人等济济哄哄都来奔抢绣球被行者喝一声把牙傞一傞把腰躬一躬长了有三丈高使个神威弄出丑脸唬得些人跌跌爬爬不敢相近。霎时人散行者还现了本象。那楼上绣女宫娥并大小太监都来对唐僧下拜道:“贵人!贵人!请入朝堂贺喜。”三藏急还礼扶起众人回头埋怨行者道:“你这猴头又是撮弄我也!”行者笑道:“绣球儿打在你头上滚在你袖里干我何事?埋怨怎么?”三藏道:“似此怎生区处?”行者道:“师父你且放心。便入朝见驾我回驿报与八戒沙僧等候。若是公主不招你便罢倒换了关文就行;如必欲招你你对国王说召我徒弟来我要吩咐他一声。那时召我三个入朝我其间自能辨别真假。此是倚婚降怪之计。”唐僧无已从言行者转身回驿。
	
	那长老被众宫娥等撮拥至楼前。公主下楼玉手相搀同登宝辇摆开仪从回转朝门。早有黄门官先奏道:“万岁公主娘娘搀着一个和尚想是绣球打着现在午门外候旨。”那国王见说心甚不喜意欲赶退又不知公主之意何如只得含情宣入。公主与唐僧遂至金銮殿下正是一对夫妻呼万岁两门邪正拜千秋。礼毕又宣至殿上开言问道:“僧人何来遇朕女抛球得中?”唐僧俯伏奏道:“贫僧乃南赡部洲大唐皇帝差往西天大雷音寺拜佛求经的因有长路关文特来朝王倒换。路过十字街彩楼之下不期公主娘娘抛绣球打在贫僧头上。贫僧是出家异教之人怎敢与玉叶金枝为偶!万望赦贫僧死罪倒换关文打早赴灵山见佛求经回我国土永注陛下之天恩也!”国王道:“你乃东土圣僧正是千里姻缘使线牵。寡人公主今登二十岁未婚因择今日年月日时俱利所以结彩楼抛绣球以求佳偶。可可的你来抛着朕虽不喜却不知公主之意如何。”那公主叩头道:“父王常言嫁鸡逐鸡嫁犬逐犬。女有誓愿在先结了这球告奏天地神明撞天婚抛打。今日打着圣僧即是前世之缘遂得今生之遇岂敢更移!愿招他为驸马。”
	
	国王方喜即宣钦天监正台官选择日期一壁厢收拾妆奁又出旨晓谕天下。三藏闻言更不谢恩只教“放赦!放赦!”国王道:“这和尚甚不通理。朕以一国之富招你做驸马为何不在此停用念念只要取经!再若推辞教锦衣官校推出斩了!”长老唬得魂不附体只得战兢兢叩头启奏道:“感蒙陛下天恩但贫僧一行四众还有三个徒弟在外今当领纳只是不曾吩咐得一言万望召他到此倒换关文教他早去不误了西来之意。”国王遂准奏道:“你徒弟在何处?”三藏道:“都在会同馆驿。”随即差官召圣僧徒弟领关文西去留圣僧在此为驸马长老只得起身侍立。有诗为证:大丹不漏要三全苦行难成恨恶缘。道在圣传修在己善由人积福由天。休逞六根多贪欲顿开一性本来原。无爱无思自清净管教解脱得然。当时差官至会同馆驿宣召唐僧徒弟不题。
	
	却说行者自彩楼下别了唐僧走两步笑两声喜喜欢欢的回驿。八戒沙僧迎着道:“哥哥你怎么那般喜笑?师父如何不见?”行者道:“师父喜了。”八戒道:“还未到地头又不曾见佛取得经回是何来之喜?”行者笑道:“我与师父只走至十字街彩楼之下可可的被当朝公主抛绣球打中了师父师父被些宫娥、彩女、太监推拥至楼前同公主坐辇入朝招为驸马此非喜而何?”八戒听说跌脚捶胸道:“早知我去好来!都是那沙僧惫懒!你不阻我啊我径奔彩楼之下一绣球打着我老猪那公主招了我却不美哉妙哉!俊刮标致停当大家造化耍子儿何等有趣!”沙僧上前把他脸上一抹道:“不羞!不羞!好个嘴巴骨子!三钱银子买了老驴自夸骑得!要是一绣球打着你就连夜烧退送纸也还道迟了敢惹你这晦气进门!”八戒道:“你这黑子不知趣!丑自丑还有些风味。自古道皮肉粗糙骨格坚强各有一得可取。”行者道:“呆子莫胡谈!且收拾行李。但恐师父着了急来叫我们却好进朝保护他。”八戒道:
	
	“哥哥又说差了。师父做了驸马到宫中与皇帝的女儿交欢又不是爬山蹱路遇怪逢魔要你保护他怎的!他那样一把子年纪岂不知被窝里之事要你去扶揝?”行者一把揪住耳朵轮拳骂道:“你这个淫心不断的夯货!说那甚胡话!”正吵闹间只见驿丞来报道:“圣上有旨差官来请三位神僧。”八戒道:“端的请我们为何?”驿丞道:“老神僧幸遇公主娘娘打中绣球招为驸马故此差官来请。”行者道:“差官在那里?教他进来。”那官看行者施礼。礼毕不敢仰视只管暗念诵道:“是鬼是怪?
	
	是雷公夜叉?”行者道:“那官儿有话不说为何沉吟?”那官儿慌得战战兢兢的双手举着圣旨口里乱道:“我公主有请会亲我主公会亲有请!”八戒道:“我这里没刑具不打你你慢慢说不要怕。”行者道:“莫成道怕你打?怕你那脸哩!快收拾挑担牵马进朝见师父议事去也!”这正是:路逢狭道难回避定教恩爱反为仇。毕竟不知见了国王有何话说且听下回分解。
	------------
	
	第九十四回 四僧宴乐御花园 一怪空怀情欲喜
	
	话表孙行者三人随着宣召官至午门外黄门官即时传奏宣进。他三个齐齐站定更不下拜国王问道:“那三位是圣僧驸马之高徒?姓甚名谁?何方居住?因甚事出家?取何经卷?”
	
	行者即近前意欲上殿旁有护驾的喝道:“不要走!有甚话立下奏来。”行者笑道:“我们出家人得一步就进一步。”随后八戒沙僧亦俱近前。长老恐他村鲁惊驾便起身叫道:“徒弟啊陛下问你来因你即奏上。”行者见他那师父在旁侍立忍不住大叫一声道:“陛下轻人重己!既招我师为驸马如何教他侍立?世间称女夫谓之贵人岂有贵人不坐之理!”国王听说大惊失色欲退殿恐失了观瞻只得硬着胆教近侍的取绣墩来请唐僧坐了。行者才奏道:“老孙祖居东胜神洲傲来国花果山水帘洞。父天母地石裂吾生。曾拜至人学成大道。复转仙乡啸聚在洞天福地。下海降龙登山擒兽。消死名上生籍官拜齐天大圣。玩赏琼楼喜游宝阁。会天仙日日歌欢;居圣境朝朝快乐。只因乱却蟠桃宴大反天宫被佛擒伏。困压在五行山下饥餐铁弹渴饮铜汁五百年未尝茶饭。幸我师出东土拜西方观音教令脱天灾离大难皈正在瑜伽门下。旧讳悟空称名行者。”国王闻得这般名重慌得下了龙床走将来以御手挽定长老道:“驸马也是朕之天缘得遇你这仙姻仙眷。”三藏满口谢恩请国王登位。复问:“那位是第二高徒?”八戒掬嘴扬威道:“老猪先世为人贪欢爱懒。一生混沌乱性迷心。未识天高地厚难明海阔山遥。正在幽闲之际忽然遇一真人。半句话解开业网;两三言劈破灾门。当时省悟立地投师谨修二八之工夫敬炼三三之前后。行满飞升得天府。荷蒙玉帝厚恩官赐天蓬元帅管押河兵逍遥汉阙。只因蟠桃酒醉戏弄嫦娥谪官衔遭贬临凡;错投胎托生猪象。住福陵山造恶无边。遇观音指明善道。皈依佛教保护唐僧。
	
	径往西天拜求妙典。法讳悟能称为八戒。”国王听言胆战心惊不敢观觑。这呆子越弄精神摇着头掬着嘴撑起耳朵呵呵大笑。三藏又怕惊驾即叱道:“八戒收敛!”方才叉手拱立假扭斯文。又问:“第三位高徒因甚皈依?”沙和尚合掌道:“老沙原系凡夫因怕轮回访道。云游海角浪荡天涯。常得衣钵随身每炼心神在舍。因此虔诚得逢仙侣。养就孩儿配缘姹女。工满三千合和四相。天界拜玄穹官授卷帘大将侍御凤辇龙车封号将军。也为蟠桃会上失手打破玻璃盏贬在流沙河改头换面造孽伤生。幸喜菩萨远游东土劝我皈依等候唐朝佛子往西天求经果正。从立自新复修大觉指河为姓。法讳悟净称名沙僧。”国王见说多惊多喜喜的是女儿招了活佛惊的是三个实乃妖神。正在惊喜之间忽有正台阴阳官奏道:“婚期已定本年本月十二日。壬子辰良周堂通利宜配婚姻。”国王道:“今日是何日辰?”阴阳官奏:“今日初八乃戊申之日猿猴献果正宜进贤纳事。”国王大喜即着当驾官打扫御花园馆阁楼亭且请驸马同三位高徒安歇待后安排合卺佳筵着公主匹配。众等钦遵国王退朝多官皆散不题。
	
	却说三藏师徒们都到御花园天色渐晚摆了素膳。八戒喜道:“这一日也该吃饭了。”管办人即将素米饭、面饭等物整担挑来。那八戒吃了又添添了又吃直吃得撑肠拄腹方才住手。少顷又点上灯设铺盖各自归寝。长老见左右无人却恨责行者怒声叫道:“悟空!你这猢狲番番害我!我说只去倒换关文莫向彩楼前去你怎么直要引我去看看?如今看得好么!却惹出这般事来怎生是好?”行者陪笑道:“师父说先母也是抛打绣球遇旧缘成其夫妇。似有慕古之意老孙才引你去又想着那个给孤布金寺长老之言就此检视真假。适见那国王之面略有些晦暗之色但只未见公主何如耳。”长老道:“你见公主便怎的?”行者道:“老孙的火眼金睛但见面就认得真假善恶富贵贫穷却好施为辨明邪正。”沙僧与八戒笑道:“哥哥近日又学得会相面了。”行者道:“相面之士当我孙子罢了。”三藏喝道:“且休调嘴!只是他如今定要招我果何以处之?”行者道:“且到十二日会喜之时必定那公主出来参拜父母等老孙在旁观看。若还是个真女人你就做了驸马享用国内之荣华也罢。”三藏闻言越生嗔怒骂道:“好猢狲!你还害我哩!却是悟能说的我们十节儿已上了九节七八分了你还把热舌头铎我?快早夹着你休开那臭口!再若无礼我就念起咒来教你了当不得!”行者听说念咒慌得跪在面前道:“莫念莫念!若是真女人待拜堂时我们一齐大闹皇宫领你去也。”师徒说话不觉早已入更。正是:沉沉宫漏荫荫花香。绣户垂珠箔闲庭绝火光。秋千索冷空留影羌笛声残静四方。绕屋有花笼月灿隔空无树显星芒。杜鹃啼歇蝴蝶梦长。银汉横天宇白云归故乡。正是离人情切处风摇嫩柳更凄凉。八戒道:“师父夜深了有事明早再议且睡!且睡!”师徒们果然安歇。
	
	一宵夜景已题早又金鸡唱晓。五更三点国王即登殿设朝但见:宫殿开轩紫气高风吹御乐透青霄。云移豹尾旌旗动日射螭头玉佩摇。香雾细添宫柳绿露珠微润苑花娇。山呼舞蹈千官列海晏河清一统朝。众文武百官朝罢又宣光禄寺安排十二日会喜佳筵今日且整春罍请驸马在御花园中款玩。吩咐仪制司领三位贤亲去会同馆少坐着光禄寺安排三席素宴去彼奉陪。两处俱着教坊司奏乐伏侍赏春景消迟日也。
	
	八戒闻得应声道:“陛下我师徒自相会更无一刻相离。今日既在御花园饮宴带我们去耍两日好教师父替你家做驸马;
	
	不然这个买卖生意弄不成。”那国王见他丑陋说话粗俗又见他扭头捏颈掬嘴巴摇耳朵即象有些风气犹恐搅破亲事只得依从便教:“在永镇华夷阁里安排二席我与驸马同坐。(wwW.mianhuatang.la 无弹窗广告)留春亭上安排三席请三位别坐恐他师徒们坐次不便。”
	
	那呆子才朝上唱个喏叫声多谢各各而退。又传旨教内宫官排宴着三宫六院后妃与公主上头就为添妆餪子以待十二日佳配。
	
	将有巳时前后那国王排驾请唐僧都到御花园内观看。
	
	好去处:径铺彩石槛凿雕栏。径铺彩石径边石畔长奇葩;槛凿雕栏槛外栏中生异卉。夭桃迷翡翠嫩柳闪黄鹂。步觉幽香来袖满行沾清味上衣多。凤台龙沼竹阁松轩。凤台之上吹箫引凤来仪;龙沼之间养鱼化龙而去。竹阁有诗费尽推敲裁白雪;松轩文集考成珠玉注青编。假山拳石翠曲水碧波深。牡丹亭蔷薇架迭锦铺绒;茉藜槛海棠畦堆霞砌玉。芍药异香蜀葵奇艳。白梨红杏斗芳菲紫蕙金萱争烂熳。丽春花、木笔花、杜鹃花夭夭灼灼;含笑花、凤仙花、玉簪花战战巍巍。一处处红透胭脂润一丛丛芳浓锦绣围。更喜东风回暖日满园娇媚逞光辉。
	
	一行君王几位观之良久。早有仪制司官邀请行者三人入留春亭国王携唐僧上华夷阁各自饮宴。那歌舞吹弹铺张陈设真是:峥嵘阊阖曙光生凤阁龙楼瑞霭横。春色细铺花草绣天光遥射锦袍明。笙歌缭绕如仙宴杯斝飞传玉液清。君悦臣欢同玩赏华夷永镇世康宁。此时长老见那国王敬重无计可奈只得勉强随喜诚是外喜而内忧也。坐间见壁上挂着四面金屏屏上画着春夏秋冬四景皆有题咏皆是翰林名士之诗:《春景诗》曰:“周天一气转洪钧大地熙熙万象新。桃李争妍花烂熳燕来画栋迭香尘。”《夏景诗》曰:“熏风拂拂思迟迟宫院榴葵映日辉。玉笛音调惊午梦芰荷香散到庭帏。”《秋景诗》曰:“金井梧桐一叶黄珠帘不卷夜来霜。燕知社日辞巢去雁折芦花过别乡。”《冬景诗》曰:“天雨飞云暗淡寒朔风吹雪积千山。深宫自有红炉暖报道梅开玉满栏。”
	
	那国王见唐僧恣意看诗便道:“驸马喜玩诗中之味心定善于吟哦如不吝珠玉请依韵各和一如何?”长老是个对景忘情、明心见性之意见国王钦重命和前韵他不觉忽谈一句道:“日暖冰消大地钧。”国王大喜即召侍卫官:“取文房四宝请驸马和完录下俟朕缓缓味之。”长老欣然不辞举笔而和。
	
	和《春景诗》曰:“日暖冰消大地钧御园花卉又更新。和风膏雨民沾泽海晏河清绝俗尘。”和《夏景诗》曰:“斗指南方白昼迟槐云榴火斗光辉。黄鹂紫燕啼宫柳巧转双声入绛帏。”和《秋景诗》曰:“香飘橘绿与橙黄松柏青青喜降霜。篱菊半开攒锦绣笙歌韵彻水云乡。”和《冬景诗》曰:“瑞雪初晴气味寒奇峰巧石玉团山。炉烧兽炭煨酥酪袖手高歌倚翠栏。”国王见和大喜称唱道:“好个袖手高歌倚翠栏!”遂命教坊司以新诗奏乐尽日而散。
	
	行者三人在留春亭亦尽受用各饮了几杯也都有些酣意正欲去寻长老只见长老已同国王在一阁。八戒呆性作应声叫道:“好快活!好自在!今日也受用这一下了!却该趁饱儿睡觉去也!”沙僧笑道:“二哥忒没修养这气饱饫如何睡觉?”八戒道:“你那里知俗语云吃了饭儿不挺尸肚里没板脂哩!”唐僧与国王相别只谨言只谨言既至亭内嗔责他三人道:“这夯货越村了!这是甚么去处只管大呼小叫!倘或恼着国王却不被他伤害性命?”八戒道:“没事没事!我们与他亲家礼道的他便不好生怪。常言道打不断的亲骂不断的邻。大家耍子怕他怎的?”长老叱道教:“拿过呆子来打他二十禅杖!”行者果一把揪翻长老举杖就打呆子喊叫道:“驸马爷爷!饶罪饶罪!”旁有陪宴官劝住呆子爬将起来突突囔囔的道:“好贵人!好驸马!亲还未成就行起王法来了!”行者侮着他嘴道:“莫胡说!莫胡说!快早睡去。”他们又在留春亭住了一宿。到明早依旧宴乐。
	
	不觉乐了三四日正值十二日佳辰有光禄寺三部各官回奏道:“臣等自八日奉旨驸马府已修完专等妆奁铺设。合卺宴亦已完备荤素共五百余席。”国王心喜正欲请驸马赴席忽有内宫官对御前启奏道:“万岁正宫娘娘有请。”国王遂退入内宫只见那三宫皇后六院嫔妃引领着公主都在昭阳宫谈笑。真个是花团锦簇!那一片富丽妖娆真胜似天堂月殿不亚于仙府瑶宫。有《喜会佳姻》新词四为证。《喜词》云:喜!
	
	喜!喜!欣然乐矣!结婚姻恩爱美。巧样宫妆嫦娥怎比。龙钗与凤镵艳艳飞金缕。樱唇皓齿朱颜嬝娜如花轻体。锦重重五彩丛中;香拂佛千金队里。《会词》云:会!会!会!妖娆娇媚。赛毛嫱欺楚妹。倾国倾城比花比玉。妆饰更鲜妍钗环多艳丽。兰心蕙性清高粉脸冰肌荣贵。黛眉一线远山微窈窕嫣姌攒锦队。《佳词》云:佳!佳!佳!玉女仙娃。深可爱实堪夸。异香馥郁脂粉交加。天台福地远怎似国王家。笑语纷然娇态笙歌缭绕喧哗。花堆锦砌千般美看遍人间怎若他。《姻词云》:姻!姻!姻!兰麝香喷。仙子阵美人群。嫔妃换彩公主妆新。云鬓堆鸦髻霓裳压凤裙。一派仙音嘹喨两行朱紫缤纷。当年曾结乘鸾信今朝幸喜会佳姻。
	
	却说国王驾到那后妃引着公主并彩女宫娥都来迎接。
	
	国王喜孜孜进了昭阳宫坐下。后妃等朝拜毕国王道:“公主贤女自初八日结彩抛球幸遇圣僧想是心愿已足。各衙门官又能体朕心各项事俱已完备。今日正是佳期可早赴合卺之宴不要错过时辰。”那公主走近前倒身下拜奏道:“父王乞赦小女万千之罪。有一言启奏:这几日闻得宫官传说唐圣僧有三个徒弟他生得十分丑恶小女不敢见他恐见时必生恐惧。万望父王将他放出城方好不然惊伤弱体反为祸害也。”国王道:“孩儿不说朕几乎忘了果然生得有些丑恶连日教他在御花园里留春亭管待。趁今日就上殿打他关文教他出城却好会宴。”公主叩头谢了恩国王即出驾上殿传旨:“请驸马共他三位。”原来那唐僧捏指头儿算日子熬至十二日天未明就与他三人计较道:“今日却是十二了这事如何区处?”行者道:“那国王我已识得他有些晦气还未沾身不为大害但只不得公主见面若得出来老孙一觑就知真假方才动作你只管放心。他如今一定来请打我等出城你自应承莫怕。我闪闪身儿就来紧紧随护你也。”师徒们正讲果见当驾官同仪制司来请。行者笑道:“去来!去来”必定是与我们送行好留师父会合。”八戒道:“送行必定有千百两黄金白银我们也好买些人事回去到我那丈人家也再会亲耍子儿去耶。”沙僧道:“二哥箝着口休乱说只凭大哥主张”遂此将行李马匹俱随那些官到于丹墀下。国王见了教请行者三位近前道:“汝等将关文拿上来朕当用宝花押交付汝等外多备盘缠送你三位早去灵山见佛若取经回来还有重谢。留驸马在此勿得悬念。”行者称谢遂教沙僧取出关文递上。国王看了即用了印押了花字又取黄金十锭白金二十锭聊达亲礼。八戒原来财色心重即去接了。行者朝上唱个喏道:“聒噪聒噪!”便转身要走慌着个三藏一毂辘爬起扯住行者咬响牙根道:“你们都不顾我就去了!”行者把手捏着三藏手掌丢个眼色道:“你在这里宽怀欢会我等取了经回来看你。”那长老似信不信的不肯放手。多官都看见以为实是相别而去。
	
	早见国王又请驸马上殿着多官送三位出城长老只得放了手上殿。
	
	行者三人同众出了朝门各自相别。八戒道:“我们当真的走哩?”行者不言语只管走至驿中。驿丞接入看茶摆饭。行者对八戒沙僧道:“你两个只在此切莫出头。但驿丞问甚么事情且含糊答应莫与我说话我保师父去也。”好大圣拔一根毫毛吹口仙气叫“变!”即变作本身模样与八戒沙僧同在驿内真身却幌的跳在半空变作一个蜜蜂儿其实小巧。但见:
	
	翅黄口甜尾利随风飘舞颠狂。最能摘蕊与偷香度柳穿花摇荡。辛苦几番淘染飞来飞去空忙。酿成浓美自何尝只好留存名状。你看他轻轻的飞入朝中。远见那唐僧在国王左边绣墩上坐着愁眉不展心存焦燥。径飞至他毗卢帽上悄悄的爬及耳边叫道:“师父我来了切莫忧虑。”这句话只有唐僧听见那伙凡人莫想知觉。唐僧听见始觉心宽。不一时宫官来请道:“万岁合卺嘉筵已排设在鳷鹊宫中娘娘与公主俱在宫伺候专请万岁同贵人会亲也。”国王喜之不尽即同驸马进宫而去。正是那:邪主爱花花作祸禅心动念念生愁。毕竟不知唐僧在内宫怎生解脱且听下回分解。
	------------
	
	第九十五回 假合真形擒玉兔 真阴归正会灵元
	
	却说那唐僧忧忧愁愁随着国王至后宫只听得鼓乐喧天随闻得异香扑鼻低着头不敢仰视。行者暗里欣然丁在那毗卢帽顶上运神光睁火眼金睛观看又只见那两班彩女摆列的似蕊宫仙府胜强似锦帐春风。真个是:娉婷嬝娜玉质冰肌。一双双娇欺楚女一对对美赛西施。云髻高盘飞彩凤娥眉微显远山低。笙簧杂奏箫鼓频吹。宫商角徵羽抑扬高下齐。清歌妙舞常堪爱锦砌花团色色怡。行者见师父全不动念暗自里咂嘴夸称道:“好和尚!好和尚!身居锦绣心无爱足步琼瑶意不迷。”
	
	少时皇后嫔妃簇拥着公主出鳷鹊宫一齐迎接都道声:
	
	“我王万岁万万岁!”慌的个长老战战兢兢莫知所措。行者早已知识见那公主头顶上微露出一点妖氛却也不十分凶恶即忙爬近耳朵叫道:“师父公主是个假的。”长老道:“是假的却如何教他现相。”行者道:“使出法身就此拿他也。”长老道:
	
	“不可!不可!恐惊了主驾且待君后退散再使法力。”那行者一生性急那里容得大咤一声现了本相赶上前揪住公主骂道:“好孽畜!你在这里弄假成真只在此这等受用也尽彀了心尚不足还要骗我师父破他的真阳遂你的淫性哩!”唬得那国王呆呆挣挣后妃跌跌爬爬宫娥彩女无一个不东躲西藏各顾性命。好便似:春风荡荡秋气潇潇。春风荡荡过园林千花摆动;秋气潇潇来径苑万叶飘摇。刮折牡丹敧槛下吹歪芍药卧栏边。沼岸芙蓉乱撼台基菊蕊铺堆。海棠无力倒尘埃玫瑰有香眠野径。春风吹折芰荷楟冬雪压歪梅嫩蕊。石榴花瓣乱落在内院东西;岸柳枝条斜垂在皇宫南北。好花风雨一宵狂无数残红铺地锦。三藏一慌了手脚战兢兢抱住国王只叫:“陛下莫怕!莫怕!此是我顽徒使法力辨真假也。”
	
	却说那妖精见事不谐挣脱了手解剥了衣裳捽捽头摇落了钗环饰即跑到御花园土地庙里取出一条碓嘴样的短棍急转身来乱打行者。行者随即跟来使铁棒劈面相迎。他两个吆吆喝喝就在花园斗起后却大显神通各驾云雾杀在空中。这一场:金箍铁棒有名声碓嘴短棍无人识。一个因取真经到此方一个为爱奇花来住迹。那怪久知唐圣僧要求配合元精液。旧年摄去真公主变作人身钦爱惜。今逢大圣认妖氛救援活命分虚实。短棍行凶着顶丢铁棒施威迎面击。喧喧嚷嚷两相持云雾满天遮白日。他两个杀在半空赌斗吓得那满城中百姓心慌尽朝里多官胆怕。长老扶着国王只叫:
	
	“休惊!请劝娘娘与众等莫怕。你公主是个假作真形的等我徒弟拿住他方知好歹也。”那些妃子有胆大的把那衣服钗环拿与皇后看了道:“这是公主穿的戴的今都丢下精着身子与那和尚在天上争打必定是个妖邪。”此时国王后妃人等才正了性望空仰视不题。
	
	却说那妖精与大圣斗经半日不分胜败。行者把棒丢起叫一声“变!”就以一变十以十变百以百变千半天里好似蛇游蟒搅乱打妖邪。妖邪慌了手脚将身一闪化道清风即奔碧空之上逃走。行者念声咒语将铁棒收做一根纵祥光一直赶来。将近西天门望见那旌旗熌灼行者厉声高叫道:“把天门的挡住妖精不要放他走了!”真个那天门上有护国天王帅领着庞刘苟毕四大元帅各展兵器拦阻。mianhuatang.la [棉花糖小说网]妖邪不能前进急回头舍死忘生使短棍又与行者相持。这大圣用心力轮铁棒仔细迎着看时见那短棍儿一头壮一头细却似春碓臼的杵头模样叱咤一声喝道:“孽畜!你拿的是甚么器械敢与老孙抵敌!快早降伏免得这一棒打碎你的天灵!”那妖邪咬着牙道:“你也不知我这兵器!听我道:仙根是段羊脂玉磨琢成形不计年。混沌开时吾已得洪蒙判处我当先。源流非比凡间物本性生来在上天。一体金光和四相五行瑞气合三元。随吾久住蟾宫内伴我常居桂殿边。因为爱花垂世境故来天竺假婵娟。与君共乐无他意欲配唐僧了宿缘。你怎欺心破佳偶死寻赶战逞凶顽!这般器械名头大在你金箍棒子前。广寒宫里捣药杵打人一下命归泉!”行者闻说呵呵冷笑道:“好孽畜啊!你既住在蟾宫之内就不知老孙的手段?你还敢在此支吾?
	
	快早现相降伏饶你性命!”那怪道:“我认得你是五百年前大闹天宫的弼马温理当让你。但只是破人亲事如杀父母之仇故此情理不甘要打你欺天罔上的弼马温!”那大圣恼得是弼马温三字他听得此言心中大怒举铁棒劈面就打。那妖邪轮杵来迎就于西天门前狠相持。这一场:金箍棒捣药杵两般仙器真堪比。那个为结婚姻降世间这个因保唐僧到这里。
	
	原来是国王没正经爱花引得妖邪喜。致使如今恨苦争两家都把顽心起。一冲一撞赌输赢劖语劖言齐斗嘴。药杵英雄世罕稀铁棒神威还更美。金光湛湛幌天门彩雾辉辉连地里。来往战经十数回妖邪力弱难搪抵。那妖精与行者又斗了十数回见行者的棒势紧密料难取胜虚丢一杵将身幌一幌金光万道径奔正南上败走大圣随后追袭忽至一座大山妖精按金光钻入山洞寂然不见。又恐他遁身回国暗害唐僧他认了这山的规模返云头径转国内。
	
	此时有申时矣。那国王正扯着三藏战战兢兢只叫:“圣僧救我!”那些嫔妃皇后也正怆惶只见大圣自云端里落将下来叫道:“师父我来也!”三藏道:“悟空立住不可惊了圣躬。我问你:假公主之事端的如何?”行者立于鳷鹊宫外叉手当胸道:“假公主是个妖邪。初时与他打了半日他战不过我化道清风径往天门上跑是我吆喝天神挡住。他现了相又与我斗到十数合又将身化作金光败回正南上一座山上。我急追至山无处寻觅恐怕他来此害你特地回顾也。”国王听说扯着唐僧问道:“既然假公主是个妖邪我真公主在于何处?”行者应声道:“待我拿住假公主你那真公主自然来也。”那后妃等闻得此言都解了恐惧一个个上前拜告道:“望圣僧救得我真公主来分了明暗必当重谢”行者道:“此间不是我们说话处请陛下与我师出宫上殿娘娘等各转各宫召我师弟八戒沙僧来保护师父我却好去降妖。一则分了内外二则免我悬心谨当辨明以表我一场心力。”国王依言感谢不已遂与唐僧携手出宫径至殿上众后妃各各回宫。一壁厢教备素膳一壁厢请八戒沙僧。须臾间二人早至。行者备言前事教他两个用心护持。这大圣纵筋斗云飞空而去那殿前多官一个个望空礼拜不题。
	
	孙大圣径至正南方那座山上寻找。原来那妖邪败了阵到此山钻入窝中将门儿使石块挡塞虚怯怯藏隐不出。行者寻一会不见动静心甚焦恼捻着诀念动真言唤出那山中土地山神审问。少时二神至了叫头道:“不知不知知当远接。万望恕罪!”行者道:“我且不打你我问你:这山叫做甚么名字?
	
	此处有多少妖精?从实说来饶你罪过。”二神告道:“大圣此山唤做毛颖山山中只有三处兔穴。亘古至今没甚妖精乃五环之福地也。大圣要寻妖精还是西天路上去有。”行者道:“老孙到了西天天竺国那国王有个公主被个妖精摄去抛在荒野他就变做公主模样戏哄国王结彩楼抛绣球欲招驸马。
	
	我保唐僧至其楼下被他有心打着唐僧欲为配偶诱取元阳。
	
	是我识破就于宫中现身捉获。他就脱了人衣、饰使一条短棍唤名捣药杵与我斗了半日他就化清风而去。被老孙赶至西天门又斗有十数合他料不能胜复化金光逃至此处如何不见?”二神听说即引行者去那三窟中寻找始于山脚下窟边看处亦有几个草兔儿也惊得走了。寻至绝顶上窟中看时只见两块大石头将窟门挡住。土地道:“此间必是妖邪赶急钻进去也。行者即使铁棒捎开石块那妖邪果藏在里面呼的一声就跳将出来举药杵来打。行者轮起铁棒架住唬得那山神倒退土地忙奔。那妖邪口里囔囔突突的骂着山神土地道:
	
	“谁教你引着他往这里来找寻!”他支支撑撑的抵着铁棒且战且退奔至空中。正在危急之际却又天色晚了。这行者愈狠性下毒手恨不得一棒打杀忽听得九霄碧汉之间有人叫道:“大圣莫动手!莫动手!棍下留情!”行者回头看时原来是太阴星君后带着姮娥仙子降彩云到于当面。慌得行者收了铁棒躬身施礼道:“老太阴那里来的?老孙失回避了。太**:“与你对敌的这个妖邪是我广寒宫捣玄霜仙药之玉兔也。他私自偷开玉关金锁走出宫来经今一载。我算他目下有伤命之灾特来救他性命望大圣看老身饶他罢。”行者喏喏连声只道:“不敢!不敢!怪道他会使捣药杵!原来是个玉兔儿!
	
	老太阴不知他摄藏了天竺国王之公主却又假合真形欲破我圣僧师父之元阳。其情其罪其实何甘!怎么便可轻恕饶他?”太**:“你亦不知。那国王之公主也不是凡人原是蟾宫中之素娥。十八年前他曾把玉兔儿打了一掌却就思凡下界。一灵之光遂投胎于国王正宫皇后之腹当时得以降生。这玉兔儿怀那一掌之仇故于旧年走出广寒抛素娥于荒野。但只是不该欲配唐僧此罪真不可逭。幸汝留心识破真假却也未曾伤损你师。万望看我面上恕他之罪我收他去也。”行者笑道:“既有这些因果老孙也不敢抗违。但只是你收了玉兔儿恐那国王不信敢烦太阴君同众仙妹将玉兔儿拿到那厢对国王明证明证一则显老孙之手段二来说那素娥下降之因由然后着那国王取素娥公主之身以见显报之意也。”太阴君信其言用手指定妖邪喝道:“那孽畜还不归正同来!”玉兔儿打个滚现了原身。真个是:缺唇尖齿长耳稀须。团身一块毛如玉展足千山蹄若飞。直鼻垂酥果赛霜华填粉腻;双睛红映犹欺雪上点胭脂。伏在地白穰穰一堆素练;伸开腰白铎铎一架银丝。几番家吸残清露瑶天晓捣药长生玉杵奇。
	
	那大圣见了不胜欣喜踏云光向前引导那太阴君领着众姮娥仙子带着玉兔儿径转天竺国界。此时正黄昏看看月上到城边闻得谯楼上擂鼓。那国王与唐僧尚在殿内八戒沙僧与多官都在阶前方议退朝只见正南上一片彩霞光明如昼。众抬头看处又闻得孙大圣厉声高叫道:“天竺陛下请出你那皇后嫔妃看者。这宝幢下乃月宫太阴星君两边的仙妹是月里嫦娥。这个玉兔儿却是你家的假公主今现真相也。”那国王急召皇后嫔妃与宫娥彩女等众朝天礼拜他和唐僧及多官亦俱望空拜谢。满城中各家各户也无一人不设香案叩头念佛。正此观看处猪八戒动了欲心忍不住跳在空中把霓裳仙子抱住道:“姐姐我与你是旧相识我和你耍子儿去也。”行者上前揪着八戒打了两掌骂道:“你这个村泼呆子!此是甚么去处敢动淫心!”八戒道:“拉闲散闷耍子而已!”那太阴君令转仙幢与众嫦娥收回玉兔径上月宫而去。行者把八戒揪落尘埃。这国王在殿上谢了行者又问前因道:“多感神僧大法力捉了假公主朕之真公主却在何处所也?”行者道:“你那真公主也不是凡胎就是月宫里素娥仙子。因十八年前他将玉兔儿打了一掌就思凡下界投胎在你正宫腹内生下身来。那玉兔儿怀恨前仇所以于旧年间偷开玉关金锁走下来把素娥摄抛荒野他却变形哄你。这段因果是太阴君亲口才与我说的。今日既去其假者明日请御驾去寻其真者。”国王闻说又心意惭惶止不住腮边流泪道:“孩儿!我自幼登基虽城门也不曾出去却教我那里去寻你也!”行者笑道:“不须烦恼你公主现在给孤布金寺里装风。今且各散到天明我还你个真公主便是。”
	
	众官又拜伏奏道:“我王且心宽这几位神僧乃腾云驾雾之神佛必知未来过去之因由。明日即烦神僧四众同去一寻便知端的。”国王依言即请至留春亭摆斋安歇。此时已近二更正是那:铜壶滴漏月华明金铎叮当风送声。杜宇正啼春去半落花无路近三更。御园寂寞秋千影碧落空浮银汉横。三市六街无客走一天星斗夜光晴。当夜各寝不题。
	
	这一夜国王退了妖气陡长精神至五更三点复出临朝。
	
	朝毕命请唐僧四众议寻公主。长老随至朝上行礼。大圣三人一同打个问讯。国王欠身道:“昨所云公主孩儿敢烦神僧为一寻救。”长老道:“贫僧前日自东来行至天晚见一座给孤布金寺特进求宿幸那寺僧相待。当晚斋罢步月闲行行至布金旧园观看基址忽闻悲声入耳。询问其由本寺一老僧年已百岁之外他屏退左右细细的对我说了一遍道:‘悲声者乃旧年春深时我正明性月忽然一阵风生就有悲怨之声。下榻到捽园基上看处乃是一个女子。询问其故那女子道我是天竺国国王公主。因为夜间玩月观花被风刮至于此。’那老僧多知人礼即将公主锁在一间僻静房中惟恐本寺顽僧污染只说是妖精被我锁住。公主识得此意日间胡言乱语讨些茶饭吃了;夜深无人处思量父母悲啼。那老僧也曾来国打听几番见公主在宫无恙所以不敢声言举奏。因见我徒弟有些神通那老僧千叮万嘱教贫僧到此查访。不期他原是蟾宫玉兔为妖假合真形变作公主模样他却又有心要破我元阳。幸亏我徒弟施威显法认出真假今已被太阴星收去。贤公主见在布金寺装风也。”国王见说此详细放声大哭。早惊动三宫六院都来问及前因。无一人不痛哭者。良久国王又问:
	
	“布金寺离城多远?”三藏道:“只有六十里路。”国王遂传旨:
	
	“着东西二宫守殿掌朝太师卫国朕同正宫皇后帅多官、四神僧去寺取公主也。”
	
	当时摆驾一行出朝。你看那行者就跳在空中把腰一扭先到了寺里。众僧慌忙跪接道:“老爷去时与众步行今日何从天上下来?”行者笑道:“你那老师在于何处?快叫他出来排设香案接驾。天竺国王、皇后、多官与我师父都来了。”众僧不解其意即请出那老僧老僧见了行者倒身下拜道:“老爷公主之事如何?”行者把那假公主抛绣球欲配唐僧并赶捉赌斗与太阴星收去玉兔之言备陈了一遍。那老僧又磕头拜谢行者搀起道:“且莫拜且莫拜快安排接驾。”众僧才知后房里锁得是个女子。一个个惊惊喜喜便都设了香案摆列山门之外穿了袈裟撞起钟鼓等候。不多时圣驾早到果然是:缤纷瑞霭满天香一座荒山倏被祥。虹流千载清河海电绕长春赛禹汤。草木沾恩添秀色野花得润有余芳。古来长者留遗迹今喜明君降宝堂。国王到于山门之外只见那众僧齐齐整整俯伏接拜又见孙行者立在中间国王道:“神僧何先到此?”行者笑道:“老孙把腰略扭一扭儿就到了你们怎么就走这半日?”随后唐僧等俱到。长老引驾到于后面房边那公主还装风胡说。老僧跪指道:“此房内就是旧年风吹来的公主娘娘。”
	
	国王即令开门。随即打开铁锁开了门。国王与皇后见了公主认得形容不顾秽污近前一把搂抱道:“我的受苦的儿啊!你怎么遭这等折磨在此受罪!”真是父母子女相逢比他人不同三人抱头大哭。哭了一会叙毕离情即令取香汤教公主沐浴更衣上辇回国。
	
	行者又对国王拱手道:“老孙还有一事奉上。”国王答礼道:“神僧有事吩咐朕即从之。”行者道:“他这山名为百脚山。近来说有蜈蚣成精黑夜伤人往来行旅甚为不便。我思蜈蚣惟鸡可以降伏可选绝大雄鸡千只撒放山中除此毒虫。
	
	就将此山名改换改换。赐文一道敕封就当谢此僧存养公主之恩也。”国王甚喜领诺随差官进城取鸡;又改山名为宝华山仍着工部办料重修赐与封号唤做“敕建宝华山给孤布金寺。”把那老僧封为“报国僧官”永远世袭赐俸三十六石。僧众谢了恩送驾回朝。公主入宫各各相见安排筵宴与公主释闷贺喜。后妃母子复聚团圞国王君臣亦共喜饮宴一宵不题。
	
	次早国王传旨召丹青图下圣僧四众喜容供养在华夷楼上又请公主新妆重整出殿谢唐僧四众救苦之恩。谢毕唐僧辞王西去。那国王那里肯放大设佳宴一连吃了五六日着实好了呆子尽力放开肚量受用。国王见他们拜佛心重苦留不住遂取金银二百锭宝贝各一盘奉谢师徒们一毫不受。教摆銮驾请老师父登辇差官远送那后妃并臣民人等俱各叩谢不尽。及至前途又见众僧叩送俱不忍相别。行者见送者不肯回去无已捻诀往巽地上吹口仙气一阵暗风把送的人都迷了眼目方才得脱身而去。这正是:沐净恩波归了性出离金海悟真空。毕竟不知前路如何且听下回分解。
	------------
	
	第九十六回 寇员外喜待高僧 唐长老不贪富贵
	
	色色原无色空空亦非空。mianhuatang.la [棉花糖小说网]静喧语默本来同梦里何劳说梦。有用用中无用无功功里施功。还如果熟自然红莫问如何修种。话表唐僧师众使法力阻住那布金寺僧。僧见黑风过处不见他师徒以为活佛临凡磕头而回不题。他师徒们西行正是春尽夏初时节:清和天气爽池沼芰荷生。梅逐雨余熟麦随风里成。草香花落处莺老柳枝轻。江燕携雏习山鸡哺子鸣。斗南当日永万物显光明说不尽那朝餐暮宿转涧寻坡。在那平安路上行经半月前边又见一城垣相近。三藏问道:“徒弟此又是甚么去处!”行者道:“不知不知。”八戒笑道:“这路是你行过的怎说不知!却是又有些儿跷蹊。故意推不认得捉弄我们哩。”行者道:“这呆子全不察理!这路虽是走过几遍那时只在九霄空里驾云而来驾云而去何曾落在此地?事不关心查他做甚此所以不知。却有甚跷蹊又捉弄你也?”
	
	说话间不觉已至边前三藏下马过吊桥径入门里。长街上只见廊下坐着两个老儿叙话。三藏叫:“徒弟你们在那街心里站住低着头不要放肆等我去那廊下问个地方。”行者等果依言立住长老近前合掌叫声“老施主贫僧问讯了。”
	
	那二老正在那里闲讲闲论说甚么兴衰得失谁圣谁贤当时的英雄事业而今安在诚可谓大叹息忽听得道声问讯随答礼道:“长老有何话说?”三藏道:“贫僧乃远方来拜佛祖的适到宝方不知是甚地名那里有向善的人家化斋一顿?”老者道:“我敝处是铜台府府后有一县叫做地灵县。长老若要吃斋不须募化过此牌坊南北街坐西向东者有一个虎坐门楼乃是寇员外家他门前有个万僧不阻之牌。似你这远方僧尽着受用。去!去!去!莫打断我们的话头。”三藏谢了转身对行者道:“此处乃铜台府地灵县。那二老道:‘过此牌坊南北街向东虎坐门楼有个寇员外家他门前有个万僧不阻之牌。’教我到他家去吃斋哩。”沙僧道:“西方乃佛家之地真个有斋僧的。此间既是府县不必照验关文我们去化些斋吃了就好走路。长老与三人缓步长街又惹得那市口里人都惊惊恐恐猜猜疑疑的。围绕争看他们相貌。长老吩咐闭口只教“莫放肆!莫放肆!”三人果低着头不取仰视。转过拐角果见一条南北大街。正行时见一个虎坐门楼门里边影壁上挂着一面大牌书着万僧不阻四字。三藏道:“西方佛地贤者愚者俱无诈伪。那二老说时我犹不信至此果如其言。”八戒村野就要进去。行者道:“呆子且住待有人出来问及何如方好进去。”沙僧道:“大哥说得有理恐一时不分内外惹施主烦恼。”
	
	在门口歇下马匹行李。须臾间有个苍头出来提着一把秤一只篮儿猛然看见慌的丢了倒跑进去报道:“主公!外面有四个异样僧家来也!”那员外拄着拐正在天井中闲走口里不住的念佛一闻报道就丢了拐出来迎接见他四众也不怕丑恶只叫:“请进请进。”三藏谦谦逊逊一同都入。转过一条巷子员外引路至一座房里说道:“此上手房宇乃管待老爷们的佛堂、经堂、斋堂、下手的是我弟子老小居住。”三藏称赞不已随取袈裟穿了拜佛举步登堂观看但见那:香云叆叇烛焰光辉。满堂中锦簇花攒四下里金铺彩绚。朱红架高挂紫金钟;彩漆檠对设花腔鼓。几对幡绣成八宝;千尊佛尽戗黄金。古铜炉;古铜瓶;雕漆桌雕漆盒。古铜炉内常常不断沉檀;古铜瓶中每有莲花现彩。雕漆桌上五云鲜雕漆盒中香瓣积。玻璃盏净水澄清;瑠璃灯;香油明亮。一声金磬响韵虚徐。真个是红尘不到赛珍楼家奉佛堂欺上刹。长老净了手拈了香叩头拜毕却转回与员外行礼。员外道:“且住!请到经堂中相见。”又见那:“方台竖柜玉匣金函。方台竖柜堆积着无数经文;玉匣金函收贮着许多简札。彩漆桌上有纸墨笔砚都是些精精致致的文房;椒粉屏前有书画琴棋尽是些妙妙玄玄的真趣。放一口轻玉浮金之仙磬挂一柄披风披月之龙髯。清气令人神气爽斋心自觉道心闲。长老到此正欲行礼那员外又搀住道:“请宽佛衣”。三藏脱了袈裟才与长老见了又请行者三人见了又叫把马喂了行李安在廊下方问起居。
	
	三藏道:“贫僧是东土大唐钦差诣宝方谒灵山见佛祖求真经者。闻知尊府敬僧故此拜见求一斋就行。”员外面生喜色笑吟吟的道:“弟子贱名寇洪字大宽虚度六十四岁。自四十岁上许斋万僧才做圆满。今已斋了二十四年有一簿斋僧的帐目。连日无事把斋过的僧名算一算已斋过九千九百九十六员止少四众不得圆满。今日可可的天降老师四位完足万僧之数请留尊讳好歹宽住月余待做了圆满弟子着轿马送老师上山。此间到灵山只有八百里路苦不远也。”三藏闻言十分欢喜都就权且应承不题。
	
	他那几个大小家僮往宅里搬柴打水取米面蔬菜整治斋供忽惊动员外妈妈问道:“是那里来的僧这等上紧?”僮仆道:“才有四位高僧爹爹问他起居他说是东土大唐皇帝差来的往灵山拜佛爷爷到我们这里不知有多少路程。爹爹说是天降的吩咐我们快整斋供养他也。”那老妪听说也喜叫丫鬟:“取衣服来我穿我也去看看。”僮仆道:“奶奶只一位看得那三位看不得形容丑得狠哩。老妪道:“汝等不知但形容丑陋古怪清奇必是天人下界。快先去报你爹爹知道。”那僮仆跑至经堂对员外道:“奶奶来了要拜见东土老爷哩。”三藏听见即起身下座。说不了老妪已至堂前举目见唐僧相貌轩昂丰姿英伟。转面见行者三人模样非凡虽知他是天人下界却也有几分悚惧朝上跪拜。三藏急急还礼道:“有劳菩萨错敬。”老妪问员外说道:“四位师父怎不并坐?”八戒掬着嘴道:
	
	“我三个是徒弟。”噫!他这一声就如深山虎啸那妈妈一害怕。
	
	:bsp;
	
	正说处又见一个家僮来报道:“两个叔叔也来了。三藏急转身看时原来是两个少年秀才。那秀才走上经堂对长老倒身下拜慌得三藏急便还礼。员外上前扯住道:“这是我两个小儿唤名寇梁、寇栋在书房里读书方回来吃午饭知老师下降故来拜也。(WWW.mianhuatang.la 好看的小说)”三藏喜道:“贤哉!贤哉!正是欲高门第须为善要好儿孙在读书。”二秀才启上父亲道:“这老爷是那里来的?”
	
	员外笑道:“来路远哩南赡部洲东土大唐皇帝钦差到灵山拜佛祖爷爷取经的。”秀才道:“我看《事林广记》上盖天下只有四大部洲。我们这里叫做西牛贺洲还有个东胜神洲。想南赡部洲至此不知走了多少年代?”三藏笑道:“贫僧在路耽阁的日子多行的日子少。常遭毒魔狠怪万苦千辛甚亏我三个徒弟保护共计一十四遍寒暑方得至宝方。”秀才闻言称奖不尽道:“真是神僧!真是神僧!说未毕又有个小的来请道:“斋筵已摆请老爷进斋。”员外着妈妈与儿子转宅他却陪四众进斋堂吃斋。那里铺设的齐整但见:金漆桌案黑漆交椅。前面是五色高果俱巧匠新装成的时样。第二行五盘小菜第三行五碟水果第四行五大盘闲食。般般甜美件件馨香。素汤米饭蒸卷馒头辣辣灶灶腾腾尽皆可口真足充肠。七八个僮仆往来奔奉四五个庖丁不住手。你看那上汤的上汤添饭的添饭一往一来真如流星赶月。这猪八戒一口一碗就是风卷残云师徒们尽受用了一顿。长老起身对员外谢了斋就欲走路。那员外拦住道:“老师放心住几日儿。常言道起头容易结梢难。只等我做过了圆满方敢送程。”三藏见他心诚意恳没奈何住了。
	
	早经过五七遍朝夕那员外才请了本处应佛僧二十四员办做圆满道场。众僧们写作有三四日选定良辰开启佛事他那里与大唐的世情一般却倒也:大扬幡铺设金容;齐秉烛烧香供养。擂鼓敲铙。吹笙捻管。云锣儿横笛音清也都是尺工字样。打一回吹一荡朗言齐语开经藏。先安土地次请神将。了文书拜了佛像。谈一部《孔雀经》句句消灾障;点一架药师灯焰焰辉光亮。拜水忏解冤愆;讽《华严》。除诽谤。
	
	三乘妙法甚精勤一二沙门皆一样。如此做了三昼夜道场已毕。唐僧想着雷音一心要去又相辞谢。员外道:“老师辞别甚急想是连日佛事冗忙多致简慢有见怪之意。”三藏道:
	
	“深扰尊府不知何以为报怎敢言怪!但只当时圣君送我出关问几时可回我就误答三年可回不期在路耽阁今已十四年矣!取经未知有无及回又得十二三年岂不违背圣旨?罪何可当!望老员外让贫僧前去待取得经回再造府久住些时有何不可!”八戒忍不住高叫道:“师父忒也不从人愿!不近人情!老员外大家巨富许下这等斋僧之愿今已圆满又况留得至诚须住年把也不妨事只管要去怎的?放了这等现成好斋不吃却往人家化募!前头有你甚老爷、老娘家哩?”长老咄的喝了一声道:“你这夯货只知要吃更不管回向之因正是那槽里吃食胃里擦痒的畜生!汝等既要贪此嗔痴明日等我自家去罢。”行者见师父变了脸即揪住八戒着头打一顿拳骂道:“呆子不知好歹惹得师父连我们都怪了!”沙僧笑道:“打得好!打得好!只这等不说话还惹人嫌且又插嘴!”那呆子气呼呼的立在旁边再不敢言。员外见他师徒们生恼只得满面陪笑道:“老师莫焦燥今日且少宽容待明日我办些旗鼓请几个邻里亲戚送你们起程。”
	
	正讲处那老妪又出来道:“老师父既蒙到舍不必苦辞。
	
	今到几日了?”三藏道:“已半月矣。”老妪道:“这半月算我员外的功德老身也有些针线钱儿也愿斋老师父半月。”说不了寇栋兄弟又出来道:“四位老爷家父斋僧二十余年更不曾遇着好人今幸圆满四位下降诚然是蓬屋生辉。学生年幼不知因果常闻得有云公修公得婆修婆得不修不得。我家父家母各欲献芹者正是各求得些因果何必苦辞?就是愚兄弟也省得有些束修钱儿也只望供养老爷半月方才送行。”三藏道:“令堂老菩萨盛情已不敢领怎么又承贤昆玉厚爱?决不敢领。今朝定要起身万勿见罪不然久违钦限罪不容诛矣。”那老妪与二子见他执一不住便生起恼来道:“好意留他他这等固执要去要去便就去了罢!只管劳叨甚么!”母子遂抽身进去。八戒忍不住口又对唐僧道:“师父不要拿过了班儿。
	
	常言道留得在落得怪。我们且住一个月儿了了他母子的愿心也罢了只管忙怎的?”唐僧又咄了一声喝道那呆子就自家把嘴打了两下道:“啐!啐!啐!”说道:“莫多话!又做声了!”
	
	行者与沙僧欷欷的笑在一边。唐僧又怪行者道:“你笑甚么?”
	
	即捻诀要念紧箍儿咒慌得个行者跪下道:“师父我不曾笑我不曾笑!千万莫念莫念!”员外又见他师徒们渐生烦恼再也不敢苦留只叫:“老师不必吵闹准于明早送行。”遂此出了经堂吩咐书办写了百十个简帖儿邀请邻里亲戚明早奉送唐朝老师西行;一壁厢又叫庖人安排饯行的筵宴;一壁厢又叫管办的做二十对彩旗觅一班吹鼓手乐人南来寺里请一班和尚东岳观里请一班道士限明日已时各项俱要整齐。众执事领命去讫不多时天又晚了。吃了晚斋各归寝处正是那:几点归鸦过别村楼头钟鼓远相闻。六街三市人烟静万户千门灯火昏。月皎风清花弄影银河惨淡映星辰。子规啼处更深矣天籁无声大地钧。当时三四更天气各管事的家僮尽皆早起买办各项物件。你看那办筵席的厨上慌忙置彩旗的堂前吵闹请僧道的两脚奔波叫鼓乐的一声急纵送简帖的东走西跑备轿马的上呼下应。这半夜直嚷至天明将已时前后各项俱完也只是有钱不过。
	
	却表唐僧师徒们早起又有那一班人供奉。长老吩咐收拾行李扣备马匹。呆子听说要走又努嘴胖唇唧唧哝哝只得将衣钵收拾找启高肩担子。沙僧刷鞄马匹套起鞍辔伺候。行者将九环杖递在师父手里他将通关文牒的引袋儿挂在胸前只是一齐要走。员外又都请至后面大厂厅内那里面又铺设了筵宴比斋堂中相待的更是不同。但见那:帘幕高挂屏围四绕正中间挂一幅寿山福海之图;两壁厢列四轴春夏秋冬之景。龙文鼎内香飘霭鹊尾炉中瑞气生。看盘簇彩宝妆花色色鲜明;排桌堆金狮仙糖齐齐摆列。阶前鼓舞按宫商堂上果肴铺锦绣。素汤素饭甚清奇香酒香茶多美艳。虽然是百姓之家却不亚王侯之宅。只听得一片欢声真个也惊天动地。长老正与员外作礼。只见家僮来报:“客俱到了。”却是那请来的左邻、右舍、妻弟、姨兄、姐夫、妹丈又有那些同道的斋公念佛的善友一齐都向长老礼拜。拜毕各各叙坐只见堂下面鼓瑟吹笙堂上边弦歌酒宴。这一席盛宴八戒留心对沙僧道:
	
	“兄弟放怀放量吃些儿。离了寇家再没这好丰盛的东西了!”
	
	沙僧笑道:“二哥说那里话!常言道珍馐百味一饱便休。只有私房路那有私房肚!”八戒道:“你也忒不济!不济!我这一顿尽饱吃了就是三日也急忙不饿。行者听见道:“呆子莫胀破了肚子!如今要走路哩!”
	
	说不了日将中矣长老在上举箸念揭斋经。八戒慌了拿过添饭来一口一碗又丢彀有五六碗把那馒头、卷儿、饼子、烧果没好没歹的满满笼了两袖才跟师父起身。长老谢了员外又谢了众人一同出门。你看那门外摆着彩旗宝盖鼓手乐人。又见那两班僧道方来员外笑道:“列位来迟老师去急不及奉斋俟回来谢罢。”众等让叙道路抬轿的抬轿骑马的骑马步行的步行都让长老四众前行。只闻得鼓乐喧天旗幡蔽日人烟凑集车马骈填都来看寇员外迎送唐僧。这一场富贵真赛过珠围翠绕诚不亚锦帐藏春!那一班僧打一套佛曲;那一班道吹一道玄音俱送出府城之外。行至十里长亭又设着箪食壶浆擎杯把盏相饮而别。那员外犹不忍舍噙着泪道:“老师取经回来是必到舍再住几日以了我寇洪之心。”
	
	三藏感之不尽谢之无已道:“我若到灵山得见佛祖表员外之大德。回时定踵门叩谢叩谢!”说说话儿不觉的又有二三里路长老恳切拜辞那员外又放声大哭而转。这正是“有愿斋僧归妙觉无缘得见佛如来。
	
	且不说寇员外送至十里长亭同众回家。却说他师徒四众行有四五十里之地天色将晚。长老道:“天晚了何方借宿?”八戒挑着担努着嘴道:“放了现成茶饭不吃清凉瓦屋不住却要走甚么路象抢丧踵魂的!如今天晚倘下起雨来却如之何!”三藏骂道:“泼孽畜又来报怨了!常言道长安虽好不是久恋之家。待我们有缘拜了佛祖取得真经那时回转大唐奏过主公将那御厨里饭凭你吃上几年胀死你这孽畜教你做个饱鬼!”那呆子吓吓的暗笑不敢复言。行者举目遥观只见大路旁有几间房宇急请师父道:“那里安歇那里安歇。”长老至前见是一座倒塌的牌坊坊上有一旧扁扁上有落颜色积尘的四个大字乃华光行院。长老下了马道:“华光菩萨是火焰五光佛的徒弟因剿除毒火鬼王降了职化做五显灵官此间必有庙祝。”遂一齐进去但见廊房俱倒墙壁皆倾更不见人之踪迹只是些杂草丛菁。欲抽身而出不期天上黑云盖顶大雨淋漓。没奈何却在那破房之下拣遮得风雨处将身躲避。密密寂寂不敢高声恐有妖邪知觉。坐的坐站的站苦捱了一夜未睡。咦!真个是:泰极还生否乐处又逢悲。
	
	毕竟不知天晓向前去还是如何且听下回分解。
	------------
	
	第九十七回 金酬外护遭魔毒 圣显幽魂救本原
	
	且不言唐僧等在华光破屋中苦奈夜雨存身。却说铜台府地灵县城内有伙凶徒因宿娼、饮酒、赌博花费了家私无计过活遂伙了十数人做贼算道本城那家是第一个财主那家是第二个财主去打劫些金银用度。内有一人道:“也不用缉访也不须算计只有今日送那唐朝和尚的寇员外家十分富厚。我们乘此夜雨街上人也不防备火甲等也不巡逻就此下手劫他些资本我们再去嫖赌儿耍子岂不美哉!众贼欢喜齐了心都带了短刀、蒺藜、拐子、闷棍、麻绳、火把冒雨前来打开寇家大门呐喊杀入。慌得他家里若大若小是男是女俱躲个干净。妈妈儿躲在床底老头儿闪在门后寇梁、寇栋与着亲的几个儿女都战战兢兢的四散逃走顾命。那伙贼拿着刀点着火将他家箱笼打开把些金银宝贝饰衣裳器皿家火尽情搜劫。那员外割舍不得拚了命走出门来对众强人哀告道:“列位大王彀你用的便罢还留几件衣物与我老汉送终”那众强人那容分说赶上前把寇员外撩阴一脚踢翻在地可怜三魂渺渺归阴府七魄悠悠别世人!众贼得了手走出寇家顺城脚做了软梯漫城墙一一系出冒着雨连夜奔西而去。
	
	那寇家僮仆、见贼退了方才出头。及看时老员外已死在地下放声哭道:“天呀!主人公已打死了!”众皆伏尸而哭悲悲啼啼。
	
	将四更时那妈妈想恨唐僧等不受他的斋供因为花扑扑的送他惹出这场灾祸便生妒害之心欲陷他四众扶着寇梁道:“儿啊不须哭了。你老子今日也斋僧明日也斋僧岂知今日做圆满斋着那一伙送命的僧也!”他兄弟道:“母亲怎么是送命的僧?”妈妈道:“贼势凶勇杀进房来我就躲在床下战兢兢的留心向灯火处看得明白你说是谁?点火的是唐僧持刀的是猪八戒搬金银的是沙和尚打死你老子的是孙行者。”
	
	二子听言认了真实道:“母亲既然看得明白必定是了。他四人在我家住了半月将我家门户墙垣窗棂巷道俱看熟了财动人心所以乘此夜雨复到我家既劫去财物又害了父亲此情何毒!待天明到府里递失状坐名告他。”寇栋道:“失状如何写?”寇梁道:“就依母亲之言。”写道:“唐僧点着火八戒叫杀人。沙和尚劫出金银去孙行者打死我父亲。”一家子吵吵闹闹不觉天晓。一壁厢传请亲人置办棺木;一壁厢寇梁兄弟赴府投词。原来这铜台府刺史正堂大人平生正直素性贤良。
	
	少年向雪案攻书早岁在金銮对策。常怀忠义之心每切仁慈之念。名扬青史播千年龚黄再见;声振黄堂传万古卓鲁重生。当时坐了堂放了一应事务即令抬出放告牌。这寇梁兄弟抱牌而入跪倒高叫道:“爷爷小的们是告强盗得财杀伤人命重情事。”刺史接上状去看了这般这的如此如彼即问道:“昨日有人传说你家斋僧圆满斋得四众高僧乃东土唐朝的罗汉花扑扑的满街鼓乐送行怎么却有这般事情?”寇梁等磕头道:“爷爷小的父亲寇洪斋僧二十四年因这四僧远来恰足万僧之数因此做了圆满留他住了半月。他就将路道、门窗都看熟了。当日送出当晚复回乘黑夜风雨遂明火执杖杀进房来劫去金银财宝衣服饰又将父打死在地。
	
	望爷爷与小民做主!”刺史闻言即点起马步快手并民壮人役共有百五十人各执锋利器械出西门一直来赶唐僧四众。
	
	却说他师徒们在那华光行院破屋下挨至天晓方才出门上路奔西。可可的那些强盗当夜打劫了寇家系出城外也向西方大路上行经天晓走过华光院西去有二十里远近藏于山凹中分拨金银等物。分还未了忽见唐僧四众顺路而来众贼心犹不歇指定唐僧道:“那不是昨日送行的和尚来了!”众贼笑道:“来得好!来得好!我们也是干这般没天理的买卖。这些和尚缘路来又在寇家许久不知身边有多少东西我们索性去截住他夺了盘缠抢了白马凑分却不是遂心满意之事?”众贼遂持兵器呐一声喊跑上大路一字儿摆开叫道:
	
	“和尚不要走!快留下买路钱饶你性命!牙迸半个不字一刀一个决不留存!”唬得个唐僧在马上乱战沙僧与八戒心慌对行者道:“怎的了!怎的了!苦奈得半夜雨天又早遇强徒断路诚所谓祸不单行也!”行者笑道:!师父莫怕兄弟勿忧。等老孙去问他一问。”
	
	好大圣束一束虎皮裙子抖一抖锦布直裰走近前叉手当胸道:“列位是做甚么的?”贼徒喝道:“这厮不知死活敢来问我!你额颅下没眼不认得我是大王爷爷!快将买路钱来放你过去!”行者闻言满面陪笑道:“你原来是剪径的强盗!”
	
	贼徒狠叫:“杀了!”行者假假的惊恐道:“大王!大王!我是乡村中的和尚不会说话冲撞莫怪莫怪!若要买路钱不要问那三个只消问我。(wwW.mianhuatang.la 无弹窗广告)我是个管帐的凡有经钱、衬钱那里化缘的、布施的都在包袱中尽是我管出入那个骑马的虽是我的师父他却只会念经不管闲事财色俱忘一毫没有。那个黑脸的是我半路上收的个后生只会养马。那个长嘴的是我雇的长工只会挑担。你把三个放过去我将盘缠衣钵尽情送你。”众贼听说:“这个和尚倒是个老实头儿。既如此饶了你命教那三个丢下行李放他过去。”行者回头使个眼色沙僧就丢了行李担子与师父牵着马同八戒往西径走。行者低头打开包袱就地挝把尘土往上一洒念个咒语乃是个定身之法喝一声“住!”那伙贼共有三十来名一个个咬着牙睁着眼撒着手直直的站定莫能言语不得动身。行者跳出路口叫道:“师父回来!回来!”八戒慌了道:“不好不好!师兄供出我们来了!他身上又无钱财包袱里又无金银必定是叫师父要马哩叫我们是剥衣服了。”沙僧笑道:“二哥莫乱说!大哥是个了得的向者那般毒魔狠怪也能收服怕这几个毛贼?他那里招呼必有话说快回去看看。”长老听言欣然转马回至边前叫道:“悟空有甚事叫回来也?”行者者:“你们看这些贼是怎的说?”八戒近前推着他叫道:“强盗你怎的不动弹了?”
	
	那贼浑然无知不言不语。八戒道:“好的痴哑了!”行者笑道:
	
	“是老孙使个定身法定住也。”八戒道:“既定了身未曾定口怎么连声也不做?“行者道:“师父请下马坐着。常言道只有错拿没有错放。兄弟你们把贼都扳翻倒捆了教他供一个供状看他是个雏儿强盗把势强盗。”沙僧道:“没绳索哩。”行者即拔下些毫毛吹口仙气变作三十条绳索一齐下手把贼扳翻都四马攒蹄捆住却又念念解咒那伙贼渐渐苏醒。
	
	行者请唐僧坐在上他三人各执兵器喝道:“毛贼你们一起有多少人?做了几年买卖?打劫了有多少东西?可曾杀伤人口?还是初犯却是二犯三犯?”众贼开口道:“爷爷饶命!”行者道:“莫叫唤!从实供来!”众贼道:“老爷我们不是久惯做贼的都是好人家子弟。只因不才吃酒赌钱宿娼顽耍将父祖家业尽花费了一向无干又无钱用。访知铜台府城中寇员外家资财豪富昨日合伙当晚乘夜雨昏黑就去打劫。劫的有些金银服饰在这路北下山凹里正自分赃忽见老爷们来。内中有认得是寇员外送行的必定身边有物;又见行李沉重白马快走人心不足故又来邀截。岂知老爷有大神通法力将我们困住。万望老爷慈悲收去那劫的财物饶了我的性命也!”三藏听说是寇家劫的财物猛然吃了一惊慌忙站起道:“悟空寇老员外十分好善如何招此灾厄?”行者笑道:“只为送我们起身那等彩帐花幢盛张鼓乐惊动了人眼目所以这伙光棍就去下手他家。今又幸遇着我们夺下他这许多金银服饰。三藏道:“我们扰他半月感激厚恩无以为报不如将此财物护送他家却不是一件好事?”行者依言即与八戒、沙僧去山凹里取将那些赃物收拾了驮在马上。又教八戒挑了一担金银沙僧挑着自己行李。行者欲将这伙强盗一棍尽情打死又恐唐僧怪他伤人性命只得将身一抖收上毫毛。那伙贼松了手脚爬起来一个个落草逃生而去。这唐僧转步回身将财物送还员外。这一去却似飞蛾投火反受其殃。有诗为证诗曰:恩将恩报人间少反把恩慈变作仇。下水救人终有失三思行事却无忧。
	
	三藏师徒们将着金银服饰拿转正行处忽见那枪刀簇簇而来。三藏大惊道:“徒弟你看那兵器簇拥相临是甚好歹?”
	
	八戒道:“祸来了祸来了!这是那放去的强盗他取了兵器又伙了些人转过路来与我们斗杀也!”沙僧道:“二哥那来的不是贼势。大哥你仔细观之。”行者悄悄的向沙僧道:“师父的灾星又到了此必是官兵捕贼之意。”说不了众兵卒至边前撒开个圈子阵把他师徒围住道:“好和尚打劫了人家东西还在这里摇摆哩!”一拥上前先把唐僧抓下马来用绳捆了又把行者三人也一齐捆了穿上扛子两个抬一个赶着马夺了担径转府城。(WWW.mianhuatang.la 好看的小说)只见那:唐三藏战战兢兢滴泪难言。猪八戒絮絮叨叨心中报怨。沙和尚囊突突意下踌躇。孙行者笑唏唏要施手段。众官兵攒拥扛抬须臾间拿到城里径自解上黄堂报道:“老爷民快人等捕获强盗来了。”那刺史端坐堂上赏劳了民快捡看了贼赃当叫寇家领去却将三藏等提近厅前问道:“你这起和尚口称是东土远来向西天拜佛却原来是些设法躧看门路打家劫舍之贼!”三藏道:“大人容告:贫僧实不是贼决不敢假随身现有通关文牒可照。只因寇员外家斋我等半月情意深重我等路遇强盗夺转打劫寇家的财物因送还寇家报恩不期民快人等捉获以为是贼实不是贼。望大人详察。”刺史道:“你这厮见官兵捕获却巧言报恩。
	
	既是路遇强盗何不连他捉来报官报恩?如何只是你四众!你看!寇梁递得失状坐名告你你还敢展挣?”三藏闻言一似大海烹舟魂飞魄丧叫:“悟空你何不上来折辨!”行者道:“有赃是实折辨何为!”刺史道:“正是啊!赃证现存还敢抵赖?”
	
	叫手下:“拿脑箍来把这秃贼的光头箍他一箍然后再打!”行者慌了心中暗想道:“虽是我师父该有此难还不可教他十分受苦。”他见那皂隶们收拾索子结脑箍即便开口道:“大人且莫箍那个和尚。昨夜打劫寇家点火的也是我持刀的也是我劫财的也是我杀人的也是我。我是个贼头要打只打我与他们无干但只不放我便是。”刺史闻言就教:“先箍起这个来。”
	
	皂隶们齐来上手把行者套上脑箍收紧了一勒扢扑的把索子断了。又结又箍又扢扑的断了。一连箍了三四次他的头皮皱也不曾皱一些儿。却又换索子再结时只听得有人来报道:“老爷都下陈少保爷爷到了请老爷出郭迎接。”那刺史即命刑房吏:“把贼收监好生看辖待我接过上司再行拷问。”
	
	刑房吏遂将唐僧四众推进监门。八戒、沙僧将自己行李担进随身。
	
	三藏道:“徒弟这是怎么起的?“行者笑道:“师父进去进去!这里边没狗叫倒好耍子!”可怜把四众捉将进去一个个都推入辖床扣拽了滚肚、敌脑、攀胸禁子们又来乱打。三藏苦痛难禁只叫:“悟空!怎的好!怎的好!”行者道:“他打是要钱哩。常言道好处安身苦处用钱。如今与他些钱便罢了。”
	
	三藏道:“我的钱自何来?”行者道:“若没钱衣物也是把那袈裟与了他罢。”三藏听说就如刀刺其心一时间见他打不过只得开言道:“悟空随你罢。”行者便叫:“列位长官不必打了。
	
	我们担进来的那两个包袱中有一件锦襕袈裟价值千金。你们解开拿了去罢。”众禁子听言一齐动手把两个包袱解看。
	
	虽有几件布衣虽有个引袋俱不值钱只见几层油纸包裹着一物霞光焰焰知是好物。抖开看时但只见:巧妙明珠缀稀奇佛宝攒。盘龙铺绣结飞凤锦沿边。众皆争看又惊动本司狱官走来喝道:“你们在此嚷甚的?”禁子们跪道:“老爹才子却提控送下四个和尚乃是大伙强盗。他见我们打了他几下把这两个包袱与我。我们打开看时见有此物无可处置。若众人扯破分之其实可惜;若独归一人众人无利。幸老爹来凭老爹做个劈着。”狱官见了乃是一件袈裟又将别项衣服并引袋儿通检看了又打开袋内关文一看见有各国的宝印花押道:“早是我来看呀!不然你们都撞出事来了。这和尚不是强盗切莫动他衣物待明日太爷再审方知端的。”众禁子听言将包袱还与他照旧包裹交与狱官收讫。
	
	渐渐天晚听得楼头起鼓火甲巡更。捱至四更三点行者见他们都不呻吟尽皆睡着他暗想道:“师父该有这一夜牢狱之灾老孙不开口折辨不使法力者盖为此耳。如今四更将尽灾将满矣我须去打点打点天明好出牢门。”你看他弄本事将身小一小脱出辖床摇身一变变做个蜢虫儿从房檐瓦缝里飞出。见那星光月皎正是清和夜静之天他认了方向径飞向寇家门只见那街西下一家儿灯火明亮。又飞近他门口看时原来是个做豆腐的见一个老头儿烧火妈妈儿挤浆。
	
	那老儿忽的叫声:“妈妈寇大官且是有子有财只是没寿。我和他小时同学读书我还大他五岁。他老子叫做寇铭当时也不上千亩田地放些租帐也讨不起。他到二十岁时那铭老儿死了他掌着家当其实也是他一步好运。娶的妻是那张旺之女小名叫做穿针儿却倒旺夫。自进他门种田又收放帐又起;买着的有利做着的赚钱被他如今挣了有十万家私。他到四十岁上就回心向善斋了万僧不期昨夜被强盗踢死。可怜!今年才六十四岁正好享用何期这等向善不得好报乃死于非命?可叹!可叹!”
	
	行者一一听之却早五更初点。他就飞入寇家只见那堂屋里已停着棺材材头边点着灯摆列着香烛花果妈妈在旁啼哭;又见他两个儿子也来拜哭两个媳妇拿两盏饭儿供献。
	
	行者就钉在他材头上咳嗽了一声唬得那两个媳妇查手舞脚的往外跑寇梁兄弟伏在地下不敢动只叫:“爹爹!口乐!
	
	口乐!
	
	口乐!”那妈妈子胆大把材头扑了一把道:“老员外你活了?”
	
	行者学着那员外的声音道:“我不曾活。”两个儿子一慌了不住的叩头垂泪只叫:“爹爹!口乐!
	
	口乐!
	
	口乐!”妈妈子硬着胆又问道:“员外你不曾活如何说话?”行者道:“我是阎王差鬼使押将来家与你们讲话的。”说道:“那张氏穿针儿枉口诳舌陷害无辜。”那妈妈子听见叫他小名慌得跪倒磕头道:“好老儿啊!
	
	这等大年纪还叫我的小名儿!我那些枉口诳舌害甚么无辜?”
	
	行者喝道:“那里有个甚么唐僧点着火八戒叫杀人沙僧劫出金银去行者打死你父亲?只因你诳言把那好人受难。那唐朝四位老师路遇强徒夺将财物送来谢我是何等好意!你却假捻失状着儿子们官官府又未细审又如今把他们监禁那狱神、土地、城隍俱慌了坐立不宁报与阎王。阎王转差鬼使押解我来家教你们趁早解放他去;不然教我在家搅闹一月将合门老幼并鸡狗之类一个也不存留!”寇梁兄弟又磕头哀告道:“爹爹请回切莫伤残老幼待天明就去本府投递解状愿认招回只求存殁均安也。”行者听了即叫:“烧纸我去呀!”他一家儿都来烧纸。
	
	行者一翅飞起径又飞至刺史住宅里面。低头观看那房内里已有灯光见刺史已起来了。他就飞进中堂看时只见中间后壁挂着一轴画儿是一个官儿骑着一匹点子马有几个从人打着一把青伞搴着一张交床更不识是甚么故事行者就钉在中间。忽然那刺史自房里出来湾着腰梳洗。行者猛的里咳嗽一声把刺史唬得慌慌张张走入房内梳洗毕穿了大衣即出来对着画儿焚香祷告道:“伯考姜公乾一神位孝侄姜坤三蒙祖上德荫忝中甲科今叨受铜台府刺史旦夕侍奉香火不绝为何今日声?切勿为邪为祟恐唬家众。”行者暗笑道:
	
	“此是他大爷的神子!”却就绰着经儿叫道:“坤三贤侄你做官虽承祖荫一向清廉怎的昨日无知把四个圣僧当贼不审来因囚于禁内!那狱神、土地、城隍不安报与阎君阎君差鬼使押我来对你说教你推情察理快快解放他;不然就教你去阴司折证也。”刺史听说心中悚惧道:“大爷请回小侄升堂当就释放。”行者道:“既如此烧纸来我去见阎君回话。”刺史复添香烧纸拜谢。
	
	行者又飞出来看时东方早已白。及飞到地灵县又见那合县官却都在堂上他思道:“蜢虫儿说话被人看见露出马脚来不好。”他就半空中改了个大法身从空里伸下一只脚来把个县堂躧满口中叫道:“众官听着:吾乃玉帝差来的浪荡游神。说你这府监里屈打了取经的佛子惊动三界诸神不安教吾传说趁早放他;若有差池教我再来一脚先踢死合府县官后躧死四境居民把城池都踏为灰烬!”概县官吏人等慌得一齐跪倒磕头礼拜道:“上圣请回。我们如今进府禀上府尊即教放出千万莫动脚惊唬死下官。”行者才收了法身仍变做个蜢虫儿从监房瓦缝儿飞入依旧钻在辖床中间睡着。
	
	却说那刺史升堂才抬出投文牌去早有寇梁兄弟抱牌跪门叫喊。刺史着令进来二人将解状递上。刺史见了怒道:
	
	“你昨日递了失状就与你拿了贼来你又领了赃去怎么今日又来递解状?”二人滴泪道:“老爷今夜小的父亲显魂道:‘唐朝圣僧原将贼徒拿住夺获财物放了贼去好意将财物送还我家报恩怎么反将他当贼拿在狱中受苦!狱中土地城隍俱不安报了阎王阎王差鬼使押解我来教你赴府再告释放唐僧庶免灾咎不然老幼皆亡。’因此特来递个解词望老爷方便!方便!”刺史听他说了这话却暗想道:“他那父亲乃是热尸新鬼显魂报应犹可;我伯父死去五六年了却怎么今夜也来显魂教我审放?看起来必是冤枉。”正忖度间只见那地灵县知县等官急急跑上堂乱道:“老大人不好了!不好了!适才玉帝差浪荡游神下界教你快放狱中好人。昨日拿的那些和尚不是强盗都是取经的佛子。若少迟延就要踢杀我等官员还要把城池连百姓俱尽踏为灰烬。”刺史又大惊失色即叫刑房吏火写牌提出。当时开了监门提出八戒愁道:“今日又不知怎的打哩。行者笑道:“管你一下儿也不敢打老孙俱已干办停当。上堂切不可下跪他还要下来请我们上坐却等我问他要行李要马匹。少了一些儿等我打他你看。”说不了已至堂口那刺史、知县并府县大小官员一见都下来迎接道:“圣僧昨日来时一则接上司忙迫二则又见了所获之赃未及细问端的。”唐僧合掌躬身又将前情细陈了一遍。众官满口认称都道:“错了错了!莫怪莫怪!”又问狱中可曾有甚疏失行者近前努目睁看厉声高叫道:“我的白马是堂上人得了行李是狱中人得了快快还我!今日却该我拷较你们了!枉拿平人做贼你们该个甚罪?”府县官见他作恶无一个不怕即便叫收马的牵马来收行李的取行李来一一交付明白。你看他三人一个个逞凶众官只以寇家遮饰。三藏劝解了道:“徒弟是也不得明白。我们且到寇家去一则吊问二来与他对证对证看是何人见我做贼。”行者道:“说得是等老孙把那死的叫起来看是那个打他。”沙僧就在府堂上把唐僧撮上马吆吆喝喝一拥而出。那些府县多官也一一俱到寇家唬得那寇梁兄弟在门前不住的磕头接进厅。只见他孝堂之中一家儿都在孝幔里啼哭行者叫道:“那打诳语栽害平人的妈妈子且莫哭!等老孙叫你老公来看他说是那个打死的羞他一羞!”众官员只道孙行者说的是笑话。行者道:“列位大人略陪我师父坐坐。八戒、沙僧好生保护等我去了就来。”好大圣跳出门望空就起只见那遍地彩霞笼住宅一天瑞气护元神。众等方才认得是个腾云驾雾之仙起死回生之圣这里一一焚香礼拜不题。
	
	那大圣一路筋斗云直至幽冥地界径撞入森罗殿上慌得那十代阎君拱手接五方鬼判叩头迎。千株剑树皆敧侧万迭刀山尽坦平。枉死城中魑魅化奈河桥下鬼生。正是那神光一照如天赦黑暗阴司处处明。十阎王接下大圣相见了问及何来何干。行者道:“铜台府地灵县斋僧的寇洪之鬼是那个收了?快点查来与我。”十阎王道:“寇洪善士也不曾有鬼使勾他他自家到此遇着地藏王的金衣童子他引见地藏也。”行者即别了径至翠云宫见地藏王菩萨。菩萨与他礼毕具言前事菩萨喜道:“寇洪阳寿止该卦数命终不染床席弃世而来。我因他斋僧是个善士收他做个掌善缘簿子的案长。既大圣来取我再延他阳寿一纪教他跟大圣去。金衣童子遂领出寇洪寇洪见了行者声声叫道:“老师!老师!救我一救!”
	
	行者道:“你被强盗踢死。此乃阴司地藏王菩萨之处我老孙特来取你到阳世间对明此事既蒙菩萨放回又延你阳寿一纪待十二年之后你再来也。”那员外顶礼不尽。行者谢辞了菩萨将他吹化为气掉于衣袖之间同去幽府复返阳间。驾云头到了寇家即唤八戒捎开材盖把他魂灵儿推付本身。须臾间透出气来活了那员外爬出材来对唐僧四众磕头道:“师父!师父!寇洪死于非命蒙师父至阴司救活乃再造之恩!”
	
	言谢不已。及回头见各官罗列即又磕头道:“列位老爹都如何在舍?”那刺史道:“你儿子始初递失状坐名告了圣僧我即差人捕获;不期圣僧路遇杀劫你家之贼夺取财物送还你家。是我下人误捉未得详审当送监禁。今夜被你显魂我先伯亦来家诉告县中又蒙浪荡游神下界一时就有这许多显应所以放出圣僧圣僧却又去救活你也。”那员外跪道:“老爹其实枉了这四位圣僧!那夜有三十多名强盗明火执杖劫去家私是我难舍向贼理说不期被他一脚撩阴踢死与这四位何干!”
	
	叫过妻子来“是谁人踢死你等辄敢妄告?请老爹定罪。”当时一家老小只是磕头刺史宽恩免其罪过。寇洪教安排筵宴酬谢府县厚恩个个未坐回衙。至次日再挂斋僧牌又款留三藏三藏决不肯住。却又请亲友办旌幢如前送行而去。咦!
	
	这正是:地辟能存凶恶事天高不负善心人。逍遥稳步如来径只到灵山极乐门毕竟不知见佛何如且听下回分解。
	------------
	
	第098回 猿熟马驯方脱壳   功成行满见真如
	
	话表寇员外既得回生,复整理了幢幡鼓乐,僧道亲友,依旧送行不题。却说唐僧四众,上了大路,果然西方佛地,与他处不同。见了些琪花、瑶草、古柏、苍松,所过地方,家家向善,户户斋僧,每逢山下人修行,又见林间客诵经。师徒们夜宿晓行,又经有六七日,忽见一带高楼,几层杰阁,真个是冲天百尺,耸汉凌空。低头观落日,引手摘飞星。豁达窗轩吞宇宙,嵯峨栋宇接云屏,黄鹤信来秋树老,彩鸾书到晚风清。此乃是灵宫宝阙,琳馆珠庭。真堂谈道,宇宙传经。花向春来美,松临雨过青。
	
	紫芝仙果年年秀,丹凤仪翔万感灵。三藏举鞭遥指道:“悟空,好去处耶!”行者道:“师父,你在那假境界假佛象处,倒强要下拜;今日到了这真境界真佛象处,倒还不下马,是怎的说?”三藏闻言,慌得翻身跳下来,已到了那楼阁门首。只见一个道童,斜立山门之前叫道:“那来的莫非东土取经人么?”长老急整衣,抬头观看,见他身披锦衣,手摇玉。身披锦衣,宝阁瑶池常赴宴;手摇玉-,丹台紫府每挥尘。肘悬仙-,足踏履鞋。飘然真羽士,秀丽实奇哉。炼就长生居胜境,修成永寿脱尘埃。圣僧不识灵山客,当年金顶大仙来。孙大圣认得他,即叫:“师父,此乃是灵山脚下玉真观金顶大仙,他来接我们哩。”三藏方才醒悟,进前施礼。大仙笑道:“圣僧今年才到,我被观音菩萨哄了。他十年前领佛金旨,向东土寻取经人,原说二三年就到我处。我年年等候,渺无消息,不意今年才相逢也。”三藏合掌道:
	
	“有劳大仙盛意,感激!感激!”遂此四众牵马挑担,同入观里,却又与大仙一一相见。即命看茶摆斋,又叫小童儿烧香汤与圣僧沐浴了,好登佛地。正是那:功满行完宜沐浴,炼驯本性合天真。千辛万苦今方息,九戒三皈始自新。魔尽果然登佛地,灾消故得见沙门。洗尘涤垢全无染,反本还原不坏身。师徒们沐浴了,不觉天色将晚,就于玉真观安歇。
	
	次早,唐僧换了衣服,披上锦-袈裟,戴了毗卢帽,手持锡杖,登堂拜辞大仙。大仙笑道:“昨日褴缕,今日鲜明,观此相真佛子也。”三藏拜别就行,大仙道;“且住,等我送你。”行者道:“不必你送,老孙认得路。大仙道:“你认得的是云路。圣僧还未登云路,当从本路而行。”行者道:“这个讲得是,老孙虽走了几遭,只是云来云去,实不曾踏着此地。既有本路,还烦你送送,我师父拜佛心重,幸勿迟疑。那大仙笑吟吟,携着唐僧手,接引旃坛上法门。原来这条路不出山门,就自观宇中堂穿出后门便是。大仙指着灵山道:“圣僧,你看那半天中有祥光五色,瑞蔼千重的,就是灵鹫高峰,佛祖之圣境也。”唐僧见了就拜,行者笑道:“师父,还不到拜处哩。常言道望山走倒马,离此镇还有许远,如何就拜!若拜到顶上,得多少头磕是?”大仙道:
	
	“圣僧,你与大圣、天蓬、卷帘四位,已此到于福地,望见灵山,我回去也。”三藏遂拜辞而去。
	
	大圣引着唐僧等,徐徐缓步,登了灵山,不上五六里,见了一道活水,滚浪飞流,约有八九里宽阔,四无人迹。三藏心惊道:“悟空,这路来得差了,敢莫大仙错指了?此水这般宽阔,这般汹涌,又不见舟楫,如何可渡?”行者笑道:“不差!你看那壁厢不是一座大桥?要从那桥上行过去,方成正果哩。”长老等又近前看时,桥边有一扁,扁上有凌云渡三字,原来是一根独木桥。正是:远看横空如玉栋,近观断水一枯槎。维河架海还容易,独木单梁人怎-!万丈虹霓平卧影,千寻白练接天涯。十分细滑浑难渡,除是神仙步彩霞。三藏心惊胆战道:“悟空,这桥不是人走的,我们别寻路径去来。”行者笑道:“正是路!正是路!八戒慌了道:“这是路,那个敢走?水面又宽,波浪又涌,独独一根木头,又细又滑,怎生动脚?”行者道:“你都站下,等老孙走个儿你看。”好大圣,拽开步跳上独木桥,摇摇摆摆,须臾跑将过去,在那边招呼道:“过来!过来!”唐僧摇手,八戒沙僧咬指道:“难!难!难!”行者又从那边跑过来,拉着八戒道:“呆子,跟我走,跟我走!”那八戒卧倒在地道:“滑!滑!滑!走不得!你饶我罢!让我驾风雾过去!”行者按住道:“这是甚么去处,许你驾风雾?必须从此桥上走过,方可成佛。”八戒道:“哥啊,佛做不成也罢,实是走不得!”
	
	他两个在那桥边,滚滚爬爬,扯扯拉拉的耍斗。沙僧走去劝解,才撒脱了手。三藏回头,忽见那下溜中有一人撑一只船来,叫道:“上渡!上渡!”长老大喜道:“徒弟,休得乱顽。那里有只渡船儿来了。”他三个跳起来站定,同眼观看,那船儿来得至近,原来是一只无底的船儿。行者火眼金睛,早已认得是接引佛祖,又称为南无宝幢光王佛。行者却不题破,只管叫:“这里来!撑拢来!”霎时撑近岸边,又叫:“上渡!上渡!”三藏见了,又心惊道:“你这无底的破船儿,如何渡人?”佛祖道:“我这船鸿蒙初判有声名,幸我撑来不变更。有浪有风还自稳,无终无始乐升平。六尘不染能归一,万劫安然自在行。无底船儿难过海,今来古往渡群生。”孙大圣合掌称谢道:“承盛意接引吾师。
	
	师父,上船去,他这船儿虽是无底,却稳;纵有风浪,也不得翻。”长老还自惊疑,行者叉着膊子,往上一推。那师父踏不住脚,毂辘的跌在水里,早被撑船人一把扯起,站在船上。师父还抖衣服,垛鞋脚,抱怨行者。行者却引沙僧八戒,牵马挑担,也上了船,都立在舟旱舟唐之上。那佛祖轻轻用力撑开,只见上溜头泱下一个死尸。长老见了大惊,行者笑道:“师父莫怕,那个原来是你。”八戒也道:“是你是你!”沙僧拍着手也道:“是你是你!”那撑船的打着号子也说:“那是你!可贺可贺!”
	
	他们三人,也一齐声相和。撑着船,不一时稳稳当当的过了凌云仙渡。三藏才转身,轻轻的跳上彼岸。有诗为证,诗曰:
	
	脱却胎胞骨肉身,相亲相爱是元神。今朝行满方成佛,洗净当年六六尘。此诚所谓广大智慧,登彼岸无极之法。四众上岸回头,连无底船儿却不知去向,行者方说是接引佛祖。三藏方才省悟,急转身,反谢了三个徒弟,行者道:“两不相谢,彼此皆扶持也。我等亏师父解脱,借门路修功,幸成了正果;师父也赖我等保护,秉教伽持,喜脱了凡胎。师父,你看这面前花草松篁,鸾凤鹤鹿之胜境,比那妖邪显化之处,孰美孰恶?何善何凶?”
	
	三藏称谢不已。一个个身轻体快,步上灵山,早见那雷音古刹:
	
	顶摩霄汉中,根接须弥脉。巧峰排列,怪石参差。悬崖下瑶草琪花,曲径旁紫芝香蕙。仙猿摘果入桃林,却似火烧金;白鹤牺松立枝头,浑如烟捧玉。彩凤双双,青鸾对对。彩凤双双,向日一鸣天下瑞;青鸾对对,迎风耀舞世间稀。又见那黄森森金瓦迭鸳鸯,明幌幌花砖铺玛瑙。东一行,西一行,尽都是蕊宫珠阙;南一带,北一带,看不了宝阁珍楼。天王殿上放霞光,护法堂前喷紫焰。浮屠塔显,优钵花香、正是地胜疑天别,云闲觉昼长。红尘不到诸缘尽,万劫无亏大法堂。师徒们逍逍遥遥,走上灵山之巅,又见青松林下列优婆,翠柏丛中排善士。长老就便施礼,慌得那优婆塞、优婆夷、比丘僧、比丘尼合掌道:“圣僧且休行礼,待见了牟尼,却来相叙。行者笑道:“早哩!早哩!且去拜上位者。”
	
	那长老手舞足蹈,随着行者,直至雷音寺山门之外。那厢有四大金刚迎住道:“圣僧来耶?”三藏躬身道:“是弟子玄奘到了。”答毕就欲进门,金刚道:“圣僧少待,容禀过再进。”那金刚着一个转山门报与二门上四大金刚,说唐僧到了;二门上又传入三门上,说唐僧到了;三山门内原是打供的神僧,闻得唐僧到时,急至大雄殿下,报与如来至尊释迦牟尼文佛说:“唐朝圣僧到于宝山取经来了。”佛爷爷大喜,即召聚八菩萨、四金刚、五百阿罗、三千揭谛、十一大曜、十八伽蓝,两行排列,却传金旨,召唐僧进。那里边,一层一节,钦依佛旨,叫:“圣僧进来。”
	
	这唐僧循规蹈矩,同悟空、悟能、悟净,牵马挑担,径入山门。正是:当年奋志奉钦差,领牒辞王出玉阶。清晓登山迎雾露,黄昏枕石卧云霾。挑禅远步三千水,飞锡长行万里崖。念念在心求正果,今朝始得见如来。
	
	四众到大雄宝殿殿前,对如来倒身下拜。拜罢,又向左右再拜。各各三匝已遍,复向佛祖长跪,将通关文牒奉上,如来一一看了,还递与三藏。三藏俯囱作礼,启上道:“弟子玄奘,奉东土大唐皇帝旨意,遥诣宝山,拜求真经,以济众生。望我佛祖垂恩,早赐回国。”如来方开怜悯之口,大发慈悲之心,对三藏言曰:“你那东土乃南赡部洲,只因天高地厚,物广人稠,多贪多杀,多淫多诳,多欺多诈;不遵佛教,不向善缘,不敬三光,不重五谷;不忠不孝,不义不仁,瞒心昧己,大斗小秤,害命杀牲。造下无边之孽,罪盈恶满,致有地狱之灾,所以永堕幽冥,受那许多碓捣磨舂之苦,变化畜类。有那许多披毛顶角之形,将身还债,将肉饲人。其永堕阿鼻,不得超升者,皆此之故也。虽有孔氏在彼立下仁义礼智之教,帝王相继,治有徒流绞斩之刑,其如愚昧不明,放纵无忌之辈何耶!我今有经三藏,可以超脱苦恼,解释灾愆。三藏:有法一藏,谈天;有论一藏,说地;有经一藏,度鬼。共计三十五部,该一万五千一百四十四卷。真是修真之径,正善之门,凡天下四大部洲之天文、地理、人物、鸟兽、花木、器用、人事,无般不载。汝等远来,待要全付与汝取去,但那方之人,愚蠢村强,毁谤真言,不识我沙门之奥旨。”叫:“阿傩、伽叶,你两个引他四众,到珍楼之下,先将斋食待他。斋罢,开了宝阁,将我那三藏经中三十五部之内,各检几卷与他,教他传流东土,永注洪恩。”二尊者即奉佛旨,将他四众领至楼下,看不尽那奇珍异宝,摆列无穷。只见那设供的诸神,铺排斋宴,并皆是仙品、仙肴、仙茶、仙果,珍馐百味,与凡世不同。师徒们顶礼了佛恩,随心享用,其实是:宝焰金光映目明,异香奇品更微精。千层金阁无穷丽,一派仙音入耳清。素味仙花人罕见,香茶异食得长生。向来受尽千般苦,今日荣华喜道成。
	
	这番造化了八戒,便宜了沙僧,佛祖处正寿长生,脱胎换骨之馔,尽着他受用。二尊者陪奉四众餐毕,却入宝阁,开门登看。那厢有霞光瑞气,笼罩千重;彩雾祥云,遮漫万道。经柜上,宝箧外,都贴了红签,楷书着经卷名目。乃是:《涅-经》一部,七百四十八卷;《菩萨经》一部,一千二十一卷;《虚空藏经》一部,四百卷;《首楞严经》一部,一百一十卷;《恩意经大集》一部,五十卷;《决定经》一部,一百四十卷;《宝藏经》一部,四十五卷;《华严经》一部,五百卷;《礼真如经》一部,九十卷;《大般若经》一部,九百一十六卷;《大光明经》一部,三百卷;《未曾有经》一部,一千一百一十卷;《维摩经》一部,一百七十卷;《三论别经》一部,二百七十卷;《金刚经》一部,一百卷;《正法论经》一部,一百二十卷;《佛本行经》一部,八百卷;《五龙经》一部,三十二卷;《菩萨戒经》一部,一百一十六卷;《大集经》一部,一百三十卷;《摩竭经》一部,三百五十卷;《法华经》一部,一百卷;《瑜伽经》一部,一百卷;《宝常经》一部,二百二十卷;《西天论经》一部,一百三十卷;《僧-经》一部,一百五十七卷;《佛国杂经》一部,一千九百五十卷;《起信论经》一部,一千卷;《大智度经》一部,一千八十卷;《宝威经》一部,一千二百八十卷;《本阁经》一部,八百五十卷;《正律文经》一部,二百卷;《大孔雀经》一部,二百二十卷;《维识论经》一部,一百卷;《具舍论经》一部,二百卷。阿傩、伽叶引唐僧看遍经名,对唐僧道:“圣僧东土到此,有些甚么人事送我们?快拿出来,好传经与你去。三藏闻言道:“弟子玄奘,来路迢遥,不曾备得。”二尊者笑道:
	
	“好,好,好!白手传经继世,后人当饿死矣!”行者见他讲口扭捏,不肯传经,他忍不住叫噪道:“师父,我们去告如来,教他自家来把经与老孙也。”阿傩道:“莫嚷!此是甚么去处,你还撒野放刁!到这边来接着经。”八戒沙僧耐住了性子,劝住了行者,转身来接。一卷卷收在包里,驮在马上,又捆了两担,八戒与沙僧挑着,却来宝座前叩头,谢了如来,一直出门。逢一位佛祖,拜两拜;见一尊菩萨,拜两拜。又到大门,拜了比丘僧、尼,优婆夷、塞,一一相辞,下山奔路不题。
	
	却说那宝阁上有一尊燃灯古佛,他在阁上,暗暗的听着那传经之事,心中甚明,原是阿傩、伽叶将无字之经传去,却自笑云:东土众僧愚迷,不识无字之经,却不枉费了圣僧这场跋涉?
	
	问:“座边有谁在此?”只见白雄尊者闪出。古佛吩咐道:“你可作起神威,飞星赶上唐僧,把那无字之经夺了,教他再来求取有字真经。”白雄尊者,即驾狂风,滚离了雷音寺山门之外,大作神威。那阵好风,真个是:佛前勇士,不比巽二风神。仙窍怒号,远赛吹嘘少女。这一阵,鱼龙皆失穴,江海逆波涛。玄猿捧果难来献,黄鹤回云找旧巢。丹凤清音鸣不美,锦鸡喔运叫声嘈。青松枝折,优钵花飘。翠竹竿竿倒,金莲朵朵摇。钟声远送三千里,经韵轻飞万壑高。崖下奇花残MeiSe,路旁瑶草偃鲜苗。彩鸾难舞翅,白鹿躲山崖。荡荡异香漫宇宙,清清风气彻云霄。那唐长老正行间,忽闻香风滚滚,只道是佛祖之祯祥,未曾提防。又闻得响一声,半空中伸下一只手来,将马驮的经,轻轻抢去,唬得个三藏捶xiong叫唤,八戒滚地来追,沙和尚护守着经担,孙行者急赶去如飞。那白雄尊者,见行者赶得将近,恐他棍头上没眼,一时间不分好歹,打伤身体,即将经包-碎,抛落尘埃。行者见经包破落,又被香风吹得飘零,却就按下云头,顾经不去追赶。那白雄尊者收风敛雾,回报古佛不题。
	
	八戒去追赶,见经本落下,遂与行者收拾背着,来见唐僧。
	
	唐僧满眼垂泪道:“徒弟呀!这个极乐世界,也还有凶魔欺害哩!”沙僧接了抱着的散经,打开看时,原来雪白,并无半点字迹,慌忙递与三藏道:“师父,这一卷没字。”行者又打开一卷看时,也无字。八戒打开一卷,也无字。三藏叫:“通打开来看看。”
	
	卷卷俱是白纸。长老短叹长吁的道:“我东土人果是没福!似这般无字的空本,取去何用?怎么敢见唐王!诳君之罪,诚不容诛也!”行者早已知之,对唐僧道:“师父,不消说了,这就是阿傩、伽叶那厮,问我要人事没有,故将此白纸本子与我们来了。快回去告在如来之前,问他-财作弊之罪。”八戒嚷道:“正是!正是!告他去来!”四众急急回山,无好步,忙忙又转上雷音。不多时,到于山门之外,众皆拱手相迎,笑道:“圣僧是换经来的?”三藏点头称谢。众金刚也不阻挡,让他进去,直至大雄殿前。行者嚷道:“如来!我师徒们受了万蜇千魔,千辛万苦,自东土拜到此处,蒙如来吩咐传经,被阿傩、伽叶-财不遂,通同作弊,故意将无字的白纸本儿教我们拿去,我们拿他去何用!望如来敕治!”佛祖笑道:“你且休嚷,他两个问你要人事之情,我已知矣。但只是经不可轻传,亦不可以空取,向时众比丘圣僧下山,曾将此经在舍卫国赵长者家与他诵了一遍,保他家生者安全,亡者超脱,只讨得他三斗三升米粒黄金回来,我还说他们忒卖贱了,教后代儿孙没钱使用。你如今空手来取,是以传了白本。白本者,乃无字真经,倒也是好的。因你那东土众生,愚迷不悟,只可以此传之耳。”即叫:“阿傩、伽叶,快将有字的真经,每部中各检几卷与他,来此报数。”
	
	二尊者复领四众,到珍楼宝阁之下,仍问唐僧要些人事。
	
	三藏无物奉承,即命沙僧取出紫金钵盂,双手奉上道:“弟子委是穷寒路遥,不曾备得人事。这钵盂乃唐王亲手所赐,教弟子持此,沿路化斋。今特奉上,聊表寸心,万望尊者不鄙轻亵,将此收下,待回朝奏上唐王,定有厚谢。只是以有字真经赐下,庶不孤钦差之意,远涉之劳也。”那阿傩接了,但微微而笑。被那些管珍楼的力士,管香积的庖丁,看阁的尊者,你抹他脸,我扑他背,弹指的,扭唇的,一个个笑道:“不羞!不羞!需索取经的人事!”须臾把脸皮都羞皱了,只是拿着钵盂不放。伽叶却才进阁检经,一一查与三藏,三藏却叫:“徒弟们,你们都好生看看,莫似前番。”他三人接一卷,看一卷,却都是有字的。传了五千零四十八卷,乃一藏之数,收拾齐整驮在马上,剩下的还装了一担,八戒挑着。自己行囊,沙僧挑着。行者牵了马,唐僧拿了锡杖,按一按毗卢帽,抖一抖锦袈裟,才喜喜欢欢,到我佛如来之前、正是那:大藏真经滋味甜,如来造就甚精严。须知玄奘登山苦,可笑阿傩却爱钱。先次未详亏古佛,后来真实始安然。至今得意传东土,大众均将雨露沾。
	
	阿傩、伽叶引唐僧来见如来,如来高升莲座,指令降龙、伏虎二大罗汉敲响云磬,遍请三千诸佛、三千揭谛、八金刚、四菩萨、五百尊罗汉、八百比丘僧、大众优婆塞、比丘尼、优婆夷,各天各洞,福地灵山,大小尊者圣僧,该坐的请登宝座,该立的侍立两旁。一时间,天乐遥闻,仙音嘹-,满空中祥光迭迭,瑞气重重,诸佛毕集,参见了如来。如来问:“阿傩、伽叶,传了多少经卷与他?可一一报数。”二尊者即开报:“现付去唐朝《涅-经》四百卷,《菩萨经》三百六十卷,《虚空藏经》二十卷,《首楞严经》三十卷,《恩意经大集》四十卷,《决定经》四十卷,《宝藏经》二十卷,《华严经》八十一卷,《礼真如经》三十卷,《大般若经》六百卷,《金光明品经》五十卷,《未曾有经》五百五十卷,《维摩经》三十卷,《三论别经》四十二卷,《金刚经》一卷,《正法论经》二十卷,《佛本行经》一百一十六卷,《五龙经》二十卷,《菩萨戒经》六十卷,《大集经》三十卷,《摩竭经》一百四十卷,《法华经》十卷,《瑜伽经》三十卷,《宝常经》一百七十卷,《西天论经》三十卷,《僧-经》一百一十卷,《佛国杂经》一千六百三十八卷,《起信论经》五十卷,《大智度经》九十卷;《宝威经》一百四十卷,《本阁经》五十六卷,《正律文经》十卷,《大孔雀经》十四卷,《维识论经》十卷,《具舍论经》十卷。在藏总经,共三十五部,各部中检出五千零四十八卷,与东土圣僧传留在唐。现俱收拾整顿于人马驮担之上,专等谢恩。”
	
	三藏四众拴了马,歇了担,一个个合掌躬身,朝上礼拜。如来对唐僧言曰:“此经功德,不可称量,虽为我门之龟鉴,实乃三教之源流。若到你那南赡部洲,示与一切众生,不可轻慢,非沐浴斋戒,不可开卷,宝之重之!盖此内有成仙了道之奥妙,有发明万化之奇方也。”三藏叩头谢恩,信受奉行,依然对佛祖遍礼三匝,承谨归诚,领经而去。去到三山门,一一又谢了众圣不题。
	
	如来因打发唐僧去后,才散了传经之会。旁又闪上观世音菩萨合掌启佛祖道:“弟子当年领金旨向东土寻取经之人,今已成功,共计得一十四年,乃五千零四十日,还少八日,不合藏数。望我世尊,早赐圣僧回东转西,须在八日之内,庶完藏数,准弟子缴还金旨。”如来大喜道:“所言甚当,准缴金旨。”即叫八大金刚吩咐道:“汝等快使神威,驾送圣僧回东,把真经传留,即引圣僧西回、须在八日之内,以完一藏之数,勿得迟违。”
	
	金刚随即赶上唐僧,叫道:“取经的,跟我来!”唐僧等俱身轻体健,荡荡飘飘,随着金刚,驾云而起。这才是:见性明心参佛祖,功完行满即飞升。毕竟不知回东土怎生传授,且听下回分解——
	
	输入:中华古籍old126.com
	
	转载请保留
	
	.
	------------
	
	第099回 九九数完魔刬尽   三三行满道归根
	
	话表八金刚既送唐僧回国不题。那三层门下,有五方揭谛、四值功曹、六丁六甲、护教伽蓝,走向观音菩萨前启道:“弟子等向蒙菩萨法旨,暗中保护圣僧,今日圣僧行满,菩萨缴了佛祖金旨,我等望菩萨准缴法旨。”菩萨亦甚喜道:“准缴,准缴。”又问道:“那唐僧四众,一路上心行何如?”诸神道:“委实心虔志诚,料不能逃菩萨洞察。但只是唐僧受过之苦,真不可言。他一路上历过的灾愆患难,弟子已谨记在此,这就是他灾难的簿子。”菩萨从头看了一遍。上写着:“蒙差揭谛皈依旨,谨记唐僧难数清:金蝉遭贬第一难,出胎几杀第二难,满月抛江第三难,寻亲报冤第四难,出城逢虎第五难,落坑折从第六难,双叉岭上第七难,两界山头第八难,陡涧换马第九难,夜被火烧第十难,失却袈裟十一难,收降八戒十二难,黄风怪阻十三难,请求灵吉十四难,流沙难渡十五难,收得沙僧十六难,四圣显化十七难,五庄观中十八难,难活人参十九难,贬退心猿二十难,黑松林失散二十一难,宝象国捎书二十二难,金銮殿变虎二十三难,平顶山逢魔二十四难,莲花洞高悬二十五难,乌鸡国救主二十六难,被魔化身二十七难,号山逢怪二十八难,风摄圣僧二十九难,心猿遭害三十难,请圣降妖三十一难,黑河沉没三十二难,搬运车迟三十三难,大赌输赢三十四难,祛道兴僧三十五难,路逢大水三十六难,身落天河三十七难,鱼篮现身三十八难,金山遇怪三十九难,普天神难伏四十难,问佛根源四十一难,吃水遭毒四十二难,西梁国留婚四十三难,琵琶洞受苦四十四难,再贬心猿四十五难,难辨猕猴四十六难,路阻火焰山四十七难,求取芭蕉扇四十八难,收缚魔王四十九难,赛城扫塔五十难,取宝救僧五十一难,棘林吟咏五十二难,小雷音遇难五十三难,诸天神遭困五十四难,稀柿-秽阻五十五难,朱紫国行医五十六难,拯救疲癃五十七难,降妖取后五十八难,七情迷没五十九难,多目遭伤六十难,路阻狮驼六十一难,怪分三色六十二难,城里遇灾六十三难,请佛收魔六十四难,比丘救子六十五难,辨认真邪六十六难,松林救怪六十七难,僧房卧病六十八难,无底洞遭困六十九难,灭法国难行七十难,隐雾山遇魔七十一难,凤仙郡求雨七十二难,失落兵器七十三难,会庆钉钯七十四难,竹节山遭难七十五难,玄英洞受苦七十六难,赶捉犀牛七十七难,天竺招婚七十八难,铜台府监禁七十九难,凌云渡脱胎八十难,路经十万八千里,圣僧历难簿分明。”菩萨将难簿目过了一遍,急传声道:“佛门中九九归真,圣僧受过八十难,还少一难,不得完成此数。”即令揭谛,“赶上金刚,还生一难者。”这揭谛得令,飞云一驾向东来。一昼夜赶上八大金刚,附耳低言道:“如此如此,谨遵菩萨法旨,不得违误。”八金刚闻得此言,刷的把风按下,将他四众,连马与经,坠落下地。噫!正是那:九九归真道行难,坚持笃志立玄关。必须苦练邪魔退,定要修持正法还。莫把经章当容易,圣僧难过许多般。古来妙合参同契,毫发差殊不结丹。
	
	三藏脚踏了凡地,自觉心惊。八戒呵呵大笑道:“好!好!
	
	好!这正是要快得迟。”沙僧道:“好!好!好!因是我们走快了些儿,教我们在此歇歇哩。”大圣道:“俗语云,十日滩头坐,一日行九滩。”三藏道:“你三个且休斗嘴,认认方向,看这是甚么地方。”沙僧转头四望道:“是这里!是这里!师父,你听听水响。”行者道:“水响想是你的祖家了。”八戒道:“他祖家乃流沙河。”沙僧道:“不是,不是,此通天河也。”三藏道:“徒弟啊,仔细看在那岸。”行者纵身跳起,用手搭凉篷仔细看了,下来道:
	
	“师父,此是通天河西岸。”三藏道:“我记起来了,东岸边原有个陈家庄。那年到此,亏你救了他儿女,深感我们,要造船相送,幸白鼋伏渡。我记得西岸上,四无人烟,这番如何是好?”八戒道:“只说凡人会作弊,原来这佛面前的金刚也会作弊。他奉佛旨,教送我们东回,怎么到此半路上就丢下我们?如今岂不进退两难!怎生过去!”沙僧道:“二哥休报怨。我的师父已得了道,前在凌云渡已脱了凡胎,今番断不落水。教师兄同你我都作起摄法,把师父驾过去也。”行者频频的暗笑道:“驾不去!”驾不去!”你看他怎么就说个驾不去?若肯使出神通,说破飞升之奥妙,师徒们就一千个河也过去了;只因心里明白,知道唐僧九九之数未完,还该有一难,故羁留于此。师徒们口里纷纷的讲,足下徐徐的行,直至水边,忽听得有人叫道:“唐圣僧,唐圣僧!这里来,这里来!”四众皆惊。举头观看,四无人迹,又没舟船,却是一个大白赖头鼋在岸边探着头叫道:“老师父,我等了你这几年,却才回也?”行者笑道:“老鼋,向年累你,今岁又得相逢。”三藏与八戒、沙僧都欢喜不尽。行者道:“老鼋,你果有接待之心,可上岸来。”那鼋即纵身爬上河来。行者叫把马牵上他身,八戒还蹲在马尾之后,唐僧站在马颈左边,沙僧站在右边,行者一脚踏着老鼋的项,一脚踏着老鼋的头叫道:
	
	“老鼋,好生走稳着。”那老鼋蹬开四足,踏水面如行平地,将他师徒四众,连马五口,驮在身上,径回东岸而来。诚所谓:不二门中法奥玄,诸魔战退识人天。本来面目今方见,一体原因始得全。秉证三乘随出入,丹成九转任周旋。挑包飞杖通休讲,幸喜还元遇老鼋。老鼋驮着他们,-波踏浪,行经多半日,将次天晚,好近东岸,忽然问曰:“老师父,我向年曾央到西方见我佛如来,与我问声归着之事,还有多少年寿,果曾问否?”原来那长老自到西天玉真观沐浴,凌云渡脱胎,步上灵山,专心拜佛及参诸佛菩萨圣僧等众,意念只在取经,他事一毫不理,所以不曾问得老鼋年寿,无言可答,却又不敢欺,打诳语,沉吟半晌,不曾答应。老鼋即知不曾替问,他就将身一幌,唿喇的淬下水去,把他四众连马并经,通皆落水。咦!还喜得唐僧脱了胎,成了道,若似前番,已经沉底。又幸白马是龙,八戒、沙僧会水,行者笑巍巍显大神通,把唐僧扶驾出水,登彼东岸。只是经包、衣服、鞍辔俱shi了。
	
	师徒方登岸整理,忽又一阵狂风,天色昏暗,雷烟俱作,走石飞沙。但见那:一阵风,乾坤播荡;一声雷,振动山川。一个-,钻云飞火;一天雾,大地遮漫。风气呼号,雷声激烈-掣红绡,雾迷星月。风鼓的尘沙扑面,雷惊的虎豹藏形,-幌的飞禽叫噪,雾漫的树木无踪。那风搅得个通天河波浪翻腾,那雷振得个通天河鱼龙丧胆,那-照得个通天河彻底光明,那雾盖得个通天河岸崖昏惨。好风!颓山烈石松篁倒。好雷!惊蛰伤人威势豪。好-!流天照野金蛇走。好雾!混混漫空蔽九霄。唬得那三藏按住了经包,沙僧压住了经担,八戒牵住了白马,行者却双手轮起铁.棒,左右护持。原来那风、雾、雷、-乃是些阴魔作号,欲夺所取之经,劳攘了一夜,直到天明,却才止息。长老一身水衣,战兢兢的道:“悟空,这是怎的起?”行者气呼呼的道:“师父,你不知就里,我等保护你取获此经,乃是夺天地造化之功,可以与乾坤并久,日月同明,寿享长春,法身不朽,此所以为天地不容,鬼神所忌,欲来暗夺之耳。一则这经是水shi透了,二则是你的正法身压住,雷不能轰,电不能照,雾不能迷,又是老孙轮着铁.棒,使纯阳之性,护持住了,及至天明,阳气又盛,所以不能夺去。”三藏、八戒、沙僧方才省悟,各谢不尽。少顷,太阳高照,却移经于高崖上,开包晒晾,至今彼处晒经之石尚存。他们又将衣鞋都晒在崖旁,立的立,坐的坐,跳的跳。真个是:一体纯阳喜向阳,阴魔不敢逞强梁。须知水胜真经伏,不怕风雷-雾光。自此清平归正觉,从今安泰到仙乡。晒经石上留踪迹,千古无魔到此方。
	
	他四众检看经本,一一晒晾,早见几个打鱼人,来过河边,抬头看见,内有认得的道:“老师父可是前年过此河往西天取经的?”八戒道:“正是,正是,你是那里人?怎么认得我们?”渔人道:“我们是陈家庄上人。”八戒道:“陈家庄离此有多远?”渔人道:“过此冲南有二十里,就是也。”八戒道:“师父,我们把经搬到陈家庄上晒去。他那里有住坐,又有得吃,就教他家与我们浆浆衣服,却不是好?”三藏道:“不去罢,在此晒干了,就收拾找路回也。”那几个渔人行过南冲,恰遇着陈澄,叫道:“二老官,前年在你家替祭儿子的师父回来了。”陈澄道:“你在那里看见?”渔人回指道:“都在那石上晒经哩。”陈澄随带了几个佃户,走过冲来望见,跑近前跪下道:“老爷取经回来,功成行满,怎么不到舍下,却在这里盘弄?快请,快请到舍。”行者道:“等晒干了经,和你去。”陈澄又问道:“老爷的经典、衣物,如何shi了?”三藏道:“昔年亏白鼋驮渡河西,今年又蒙他驮渡河东。已将近岸,被他问昔年托问佛祖寿年之事,我本未曾问得,他遂淬在水内,故此shi了。”又将前后事细说了一遍。那陈澄拜请甚恳,三藏无已,遂收拾经卷。不期石上把佛本行经沾住了几卷,遂将经尾沾破了,所以至今本行经不全,晒经石上犹有字迹。
	
	三藏懊悔道:“是我们怠慢了,不曾看顾得!”行者笑道:“不在此!不在此!盖天地不全,这经原是全全的,今沾破了,乃是应不全之奥妙也,岂人力所能与耶!”师徒们果收拾毕,同陈澄赴庄。
	
	那庄上人家,一个传十,十个传百,百个传千,若老若幼,都来接看。陈清闻说,就摆香案在门前迎迓,又命鼓乐吹打。少顷到了迎入,陈清领合家人眷俱出来拜见,拜谢昔日救女儿之恩,随命看茶摆斋。三藏自受了佛祖的仙品仙肴,又脱了凡胎成佛,全不思凡间之食。二老苦劝,没奈何,略见他意。孙大圣自来不吃烟火食,也道:“彀了。”沙僧也不甚吃,八戒也不似前番,就放下碗。行者道:“呆子也不吃了?”八戒道:“不知怎么,脾胃一时就弱了。”遂此收了斋筵,却又问取经之事。三藏又将先至玉真观沐浴,凌云渡脱胎,及至雷音寺参如来,蒙珍楼赐宴,宝阁传经,始被二尊者索人事未遂,故传无字之经,后复拜告如来,始得授一藏之数,并白鼋淬水,阴魔暗夺之事,细细陈了一遍,就欲拜别。那二老举家,如何肯放,且道:“向蒙救拔儿女,深恩莫报,已创建一座院宇,名曰救生寺,专侍奉香火不绝。”又唤出原替祭之儿女陈关保、一秤金叩谢,复请至寺观看。三藏却又将经包儿收在他家堂前,与他念了一卷《宝常经》。后至寺中,只见陈家又设馔在此。还不曾坐下,又一起来请;还不曾举箸,又一起来请,络绎不绝,争不上手。三藏俱不敢辞,略略见意。只见那座寺果盖得齐整:山门HongFen腻,多赖施主功。一座楼台从此立,两廊房宇自今兴。朱红隔扇,七宝玲珑。香气飘云汉,清光满太空。几株嫩柏还浇水,数干乔松未结丛。活水迎前,通天迭迭翻波浪;高崖倚后,山脉重重接地龙。三藏看毕,才上高楼,楼上果装塑着他四众之象。八戒看见,扯着行者道:“兄长的相儿甚象。”沙僧道:“二哥,你的又象得紧。只是师父的又忒俊了些儿。”三藏道:“却好!却好!”遂下楼来,下面前殿后廊,还有摆斋的候请。行者却问:“向日大王庙儿如何了?”众老道:“那庙当年拆了。老爷,这寺自建立之后,年年成熟,岁岁丰登,却是老爷之福庇。”行者笑道:“此天赐耳,与我们何与!但只我们自今去后,保你这一庄上人家,子孙繁衍,六畜安生,年年风调雨顺,岁岁雨顺风调。”众等却叩头拜谢。只见那前前后后,更有献果献斋的,无限人家。八戒笑道:“我的蹭蹬!那时节吃得,却没人家连请十请;今日吃不得,却一家不了,又是一家。”饶他气满,略动手又吃过八九盘素食;纵然胃伤,又吃了二三十个馒头,已皆尽饱又有人来相邀,三藏道:“弟子何能,感蒙至爱!望今夕暂停,明早再领。”
	
	时已深夜,三藏守定真经,不敢暂离,就于楼下打坐看守。
	
	将及三更,三藏悄悄的叫道:“悟空,这里人家,识得我们道成事完了。自古道,真人不露相,露相不真人。恐为久淹,失了大事。”行者道:“师父说得有理,我们趁此深夜,人皆熟睡,寂寂的去了罢。”八戒却也知觉,沙僧尽自分明,白马也能会意。遂此起了身,轻轻的抬上驮垛,挑着担,从庑廊驮出。到于山门,只见门上有锁。行者又使个解锁法,开了二门、大门,找路望东而去。只听得半空中有八大金刚叫道:“逃走的,跟我来!”那长老闻得香风荡荡,起在空中。这正是:丹成识得本来面,体健如如拜主人。毕竟不知怎生见那唐王,且听下回分解——
	
	输入:中华古籍old126.com
	
	转载请保留
	
	.
	------------
	
	第100回 径回东土      五圣成真
	
	且不言他四众脱身,随金刚驾风而起,却说陈家庄救生寺内多人,天晓起来,仍治果肴来献,至楼下,不见了唐僧。这个也来问,那个也来寻,俱慌慌张张,莫知所措,叫苦连天的道:
	
	“清清把个活佛放去了!”一会家无计,将办来的品物,俱抬在楼上祭祀烧纸。以后每年四大祭,二十四小祭。还有那告病的,保安的,求亲许愿,求财求子的,无时无日不来烧香祭赛,真个是金炉不断千年火,玉盏常明万载灯,不题。
	
	却说八大金刚使第二阵香风,把他四众,不一日送至东土,渐渐望见长安。原来那太宗自贞观十三年九月望前三日送唐僧出城,至十六年,即差工部官在西安关外起建了望经楼接经,太宗年年亲至其地。恰好那一日出驾复到楼上,忽见正西方满天瑞霭,阵阵香风,金刚停在空中叫道:“圣僧,此间乃长安城了。我们不好下去,这里人伶俐,恐泄漏吾像。孙大圣三位也不消去,汝自去传了经与汝主,即便回来。我在霄汉中等你,与你一同缴旨。”大圣道:“尊者之言虽当,但吾师如何挑得经担?如何牵得这马?须得我等同去一送。烦你在空少等,谅不敢误。”金刚道:“前日观音菩萨启过如来,往来只在八日,方完藏数。今已经四日有余,只怕八戒贪图富贵,误了期限。”八戒笑道:“师父成佛,我也望成佛,岂有贪图之理!泼大粗人!都在此等我,待交了经,就来与你回向也。”呆子挑着担,沙僧牵着马,行者领着圣僧,都按下云头,落于望经楼边。太宗同多官一齐见了,即下楼相迎道:“御弟来也?”唐僧即倒身下拜,太宗搀起,又问:“此三者何人?”唐僧道:“是途中收的徒弟。”太宗大喜,即命侍官:“将朕御车马扣背,请御弟上马,同朕回朝。”
	
	唐僧谢了恩,骑上马,大圣轮金箍棒紧随,八戒、沙僧俱扶马挑担,随驾后共入长安。真个是:当年清宴乐升平,文武安然显俊英。水陆场中僧演法,金銮殿上主差卿。关文敕赐唐三藏,经卷原因配五行。苦炼凶魔种种灭,功成今喜上朝京。
	
	唐僧四众,随驾入朝,满城中无一不知是取经人来了。却说那长安唐僧旧住的洪福寺大小僧人,看见几株松树一颗颗头俱向东,惊讶道:“怪哉!怪哉!今夜未曾刮风,如何这树头都扭过来了?”内有三藏的旧徒道:“快拿衣服来!取经的老师父来了!”众僧问道:“你何以知之?”旧徒曰:“当年师父去时,曾有言道:‘我去之后,或三五年,或六七年,但看松树枝头若是东向,我即回矣。’我师父佛口圣言,故此知之。”急披衣而出,至西街时,早已有人传播说:“取经的人适才方到,万岁爷爷接入城来了。”众僧听说,又急急跑来,却就遇着,一见大驾,不敢近前,随后跟至朝门之外。唐僧下马,同众进朝。唐僧将龙马与经担,同行者、八戒、沙僧,站在玉阶之下。太宗传宣:
	
	“御弟上殿。”赐坐,唐僧又谢恩坐了,教把经卷抬来。行者等取出,近侍官传上。太宗又问:“多少经数?怎生取来?”三藏道:
	
	“臣僧到了灵山,参见佛祖,蒙差阿傩、伽叶二尊者先引至珍楼内赐斋,次到宝阁内传经。那尊者需索人事,因未曾备得,不曾送他,他遂以经与了。当谢佛祖之恩东行,忽被妖风抢了经去,幸小徒有些神通赶夺,却俱抛掷散漫。因展看,皆是无字空本。
	
	臣等着惊,复去拜告恳求,佛祖道:‘此经成就之时,有比丘圣僧将下山与舍卫国赵长者家看诵了一遍,保-他家生者安全,亡者超脱,止讨了他三斗三升米粒黄金,意思还嫌卖贱了,后来子孙没钱使用。’我等知二尊者需索人事,佛祖明知,只得将钦赐紫金钵盂送他,方传了有字真经。此经有三十五部,各部中检了几卷传来,共计五千零四十八卷,此数盖合一藏也。”太宗更喜,教:“光禄寺设宴,开东阁酬谢。”忽见他三徒立在阶下,容貌异常,便问:“高徒果外国人耶?”长老俯伏道:“大徒弟姓孙,法名悟空,臣又呼他为孙行者。他出身原是东胜神洲傲来国花果山水帘洞人氏,因五百年前大闹天宫,被佛祖困压在西番两界山石匣之内,蒙观音菩萨劝善,情愿皈依,是臣到彼救出,甚亏此徒保护。二徒弟姓猪,法名悟能,臣又呼他为猪八戒。他出身原是福陵山云栈洞人氏,因在乌斯藏高老庄上作怪,即蒙菩萨劝善,亏行者收之,一路上挑担有力,涉水有功。
	
	三徒弟姓沙,法名悟净,臣又呼他为沙和尚。他出身原是流沙河作怪者,也蒙菩萨劝善,秉教沙门。那匹马不是主公所赐者。”太宗道:“MaoPian相同,如何不是?”三藏道:“臣到蛇盘山鹰愁涧涉水,原马被此马吞之,亏行者请菩萨问此马来历,原是西海龙王之了,因有罪,也蒙菩萨救解,教他与臣作脚力。当时变作原马,MaoPian相同。幸亏他登山越岭,跋涉崎岖,去时骑坐,来时驮经,亦甚赖其力也。”太宗闻言,称赞不已,又问:“远涉西方,端的路程多少?”三藏道:“总记菩萨之言,有十万八千里之远。途中未曾记数,只知经过了一十四遍寒暑。Ri山,Ri岭,遇林不小,遇水宽洪。还经几座国王,俱有照验印信。”
	
	叫:“徒弟,将通关文牒取上来,对主公缴纳。”当时递上。太宗看了,乃贞观一十三年九月望前三日给。太宗笑道:“久劳远涉,今已贞观二十七年矣。”牒文上有宝象国印,乌鸡国印,车迟国印,西梁女国印,祭赛国印,朱紫国印,狮驼国印,比丘国印,灭法国印;又有凤仙郡印,玉华州印,金平府印。太宗览毕,收了。
	
	早有当驾官请宴,即下殿携手而行,又问:“高徒能礼貌乎?”三藏道:“小徒俱是山村旷野之妖身,未谙中华圣朝之礼数,万望主公赦罪。”太宗笑道:“不罪他,不罪他,都同请东阁赴宴去也。”三藏又谢了恩,招呼他三众,都到阁内观看。果是中华大国,比寻常不同。你看那:门悬彩绣,地衬红毡。异香馥郁,奇品新鲜。琥珀杯,玻璃盏,镶金点翠;黄金盘,白玉碗,嵌锦花缠。烂煮蔓菁,糖浇香芋。蘑菇甜美,海菜清奇。几次添来姜辣笋,数番办上蜜调葵。面筋椿树叶,木耳豆腐皮。石花仙菜,蕨粉干薇。花椒煮莱菔,芥末拌瓜丝。几盘素品还犹可,数种奇稀果夺魁。核桃柿饼,龙眼荔枝。宣州茧栗山东枣,江南银杏兔头梨。榛松莲肉葡.萄大,榧子瓜仁菱米齐。橄榄林檎,苹婆沙果。慈菇嫩藕,脆李杨梅。无般不备,无件不齐。还有些蒸酥蜜食兼嘉馔,更有那美酒香茶与异奇。说不尽百味珍馐真上品,果然是中华大国异西夷。师徒四众与文武多官俱侍列左右,太宗皇帝仍正坐当中,歌舞吹弹,整齐严肃,遂尽乐一日。正是:君王嘉会赛唐虞,取得真经福有余。千古流传千古盛,佛光普照帝王居。当日天晚,谢恩宴散。太宗回宫,多官回宅,唐僧等归于洪福寺,只见寺僧磕头迎接。方进山门,众僧报道:“师父,这树头儿今早俱忽然向东。我们记得师父之言,遂出城来接,果然到了!”长老喜之不胜,遂入方丈。此时八戒也不嚷茶饭,也不弄喧头,行者、沙僧个个稳重。只因道果完成,自然安静。当晚睡了。
	
	次早,太宗升朝,对群臣言曰:“朕思御弟之功,至深至大,无以为酬。一夜无寐,口占几句俚谈,权表谢意,但未曾写出。”
	
	叫:“中书官来,朕念与你,你一一写之。”其文云:“盖闻二仪有象,显覆载以含生;四时无形,潜寒暑以化物。是以窥天鉴地,庸愚皆识其端;明阴洞阳,贤哲罕穷其数。然天地包乎阴阳,而易识者,以其有象也;阴阳处乎天地,而难穷者,以其无形也。
	
	故知象显可征,虽愚不惑;形潜莫睹,在智犹迷。况乎佛道崇虚,乘幽控寂。弘济万品,典御十方。举威灵而无上,抑神力而无下;大之则弥于宇宙,细之则摄于毫厘。无灭无生,历千劫而亘古;若隐若显,运百福而长今。妙道凝玄,遵之莫知其际;法流湛寂,挹之莫测其源。故知蠢蠢凡愚,区区庸鄙,投其旨趣,能无疑惑者哉!然则大教之兴,基乎西土。腾汉庭而皎梦,照东域而流慈。古者分形分迹之时,言未驰而成化;当常见常隐之世,民仰德而知遵。及乎晦影归真,迁移越世,金容掩色,不镜三千之光;丽象开图,空端四八之相。于是微言广被,拯禽类于三途;遗训遐宣,导群生于十地。佛有经,能分大小之乘,更有法,传讹邪正之术。我僧玄奘法师者,法门之领袖也。幼怀慎敏,早悟三空之功;长契神清,先包四忍之行。松风水月,未足比其清华;仙露明珠,讵能方其朗润!故以智通无累,神测未形。超六尘而迥出,使千古而传芳。凝心内境,悲正法之陵迟;
	
	栖虑玄门,慨深文之讹谬。思欲分条振理,广彼前闻;截伪续真,开兹后学。是以翘心净土,法游西域。乘危远迈,策杖孤征。
	
	积雪晨飞,途间失地;惊沙夕起,空外迷天。万里山川,拨烟霞而进步;百重寒暑,蹑霜雨而前踪。诚重劳轻,求深欲达。周游西宇,十有四年。穷历异邦,询求正教。双林八水,味道餐风;
	
	鹿苑鹫峰,瞻奇仰异。承至言于先圣,受真教于上贤。探赜妙门,精穷奥业。三乘六律之道,驰骤于心田;一藏百箧之文,波涛于海口。爰自所历之国无涯,求取之经有数。总得大乘要文,凡三十五部,计五千四十八卷,译布中华,宣扬胜业。引慈云于西极,注法雨于东陲。圣教缺而复全,苍生罪而还福。shi火宅之干焰,共拔迷途;朗金水之昏波,同臻彼岸。是知恶因业坠,善以缘升。升坠之端,惟人自作。譬之桂生高岭,云露方得泫其花;莲出绿波,飞尘不能染其叶。非莲性自洁而桂质本贞,良由所附者高,则微物不能累;所凭者净,则浊类不能沾。夫以卉木无知,犹资善而成善,矧乎人伦有识,宁不缘庆而成庆?方冀真经传布,并日月而无穷;景福遐敷,与乾坤而永大也欤!”写毕,即召圣僧。此时长老已在朝门外候谢,闻宣急入,行俯伏之礼。太宗传请上殿,将文字递与长老览遍。复下谢恩,奏道:
	
	“主公文辞高古,理趣渊微,但不知是何名目。”太宗道:“朕夜口占,答谢御弟之意,名曰圣教序,不知好否。”长老叩头,称谢不已。太宗又曰:“朕才愧圭璋,言惭金石。至于内典,尤所未闻。口占叙文,诚为鄙拙。秽翰墨于金简,标瓦砾于珠林。循躬省虑,-面恧心。甚不足称,虚劳致谢。”
	
	当时多官齐贺,顶礼圣教御文,遍传内外。太宗道:“御弟将真经演诵一番,何如?”长老道:“主公,若演真经,须寻佛地,宝殿非可诵之处。”太宗甚喜,即问当驾官:“长安城寺,有那座寺院洁净?”班中闪上大学士萧-奏道:“城中有一雁塔寺洁净。”太宗即令多官:“把真经各虔捧几卷,同朕到雁塔寺,请御弟谈经去来。”多官遂各各捧着,随太宗驾幸寺中,搭起高台,铺设齐整。长老仍命:“八戒沙僧牵龙马,理行囊,行者在我左右。”又向太宗道:“主公欲将真经传流天下,须当誉录副本,方可布散。原本还当珍藏,不可轻亵。”太宗又笑道:“御弟之言甚当!甚当!”随召翰林院及中书科各官誉写真经。又建一寺,在城之东,名曰誊黄寺。
	
	长老捧几卷登台,方欲讽诵,忽闻得香风缭绕,半空中有八大金刚现身高叫道:“诵经的,放下经卷,跟我回西去也。”这底下行者三人,连白马平地而起,长老亦将经卷丢下,也从台上起于九霄,相随腾空而去,慌得那太宗与多官望空下拜。这正是:圣僧努力取经编,西宇周流十四年。苦历程途遭患难,多经山水受。功完八九还加九,行满三千及大千。大觉妙文回上国,至今东土永留传。太宗与多官拜毕,即选高僧,就于雁塔寺里,修建水陆大会,看诵《大藏真经》,超脱幽冥孽鬼,普施善庆,将誊录过经文,传布天下不题。
	
	却说八大金刚,驾香风,引着长老四众,连马五口,复转灵山,连去连来,适在八日之内。此时灵山诸神,都在佛前听讲。
	
	八金刚引他师徒进去,对如来道:“弟子前奉金旨,驾送圣僧等,已到唐国,将经交纳,今特缴旨。”遂叫唐僧等近前受职。如来道:“圣僧,汝前世原是我之二徒,名唤金蝉子。因为汝不听说法,轻慢我之大教,故贬汝之真灵,转生东土。今喜皈依,秉我迦持,又乘吾教,取去真经,甚有功果,加升大职正果,汝为旃檀功德佛。孙悟空,汝因大闹天宫,吾以甚深法力,压在五行山下,幸天灾满足,归于释教,且喜汝隐恶扬善,在途中炼魔降怪有功,全终全始,加升大职正果,汝为斗战胜佛。猪悟能,汝本天河水神,天蓬元帅,为汝蟠桃会上酗酒戏了仙娥,贬汝下界投胎,身如畜类,幸汝记爱人身,在福陵山云栈洞造孽,喜归大教,入吾沙门,保圣僧在路,却又有顽心,SeQing未泯,因汝挑担有功,加升汝职正果,做净坛使者。”八戒口中嚷道:“他们都成佛,如何把我做个净坛使者?”如来道:“因汝口壮身慵,食肠宽大。盖天下四大部洲,瞻仰吾教者甚多,凡诸佛事,教汝净坛,乃是个有受用的品级,如何不好!沙悟净,汝本是卷帘大将,先因蟠桃会上打碎玻璃盏,贬汝下界,汝落于流沙河,伤生吃人造孽,幸皈吾教,诚敬迦持、保护圣僧,登山牵马有功,加升大职正果,为金身罗汉。”又叫那白马:“汝本是西洋大海广晋龙王之子,因汝违逆父命,犯了不孝之罪,幸得皈身皈法,皈我沙门,每日家亏你驮负圣僧来西,又亏你驮负圣经去东,亦有功者,加升汝职正果,为八部天龙马。”长老四众,俱各叩头谢恩。马亦谢恩讫,仍命揭谛引了马下灵山后崖化龙池边,将马推入池中。须臾间,那马打个展身,即退了毛皮,换了头角,浑身上长起金鳞,腮颔下生出银须,一身瑞气,四爪祥云,飞出化龙池,盘绕在山门里擎天华表柱上,诸佛赞扬如来的大法。
	
	孙行者却又对唐僧道:“师父,此时我已成佛,与你一般,莫成还戴金箍儿,你还念甚么《紧箍咒》儿-勒我?趁早儿念个松箍儿咒,脱下来,打得粉碎,切莫叫那甚么菩萨再去捉弄他人。唐僧道:“当时只为你难管,故以此法制之。今已成佛,自然去矣,岂有还在你头上之理!你试摸摸看。”行者举手去摸一摸,果然无之。此时旃檀佛、斗战佛、净坛使者、金身罗汉,俱正果了本位,天龙马亦自归真。有诗为证,诗曰:一体真如转落尘,合和四相复修身。五行论色空还寂,百怪虚名总莫论。正果旃檀皈大觉,完成品职脱沉沦。经传天下恩光阔,五圣高居不二门。
	
	五圣果位之时,诸众佛祖、菩萨、圣僧、罗汉、揭谛、比丘、优婆夷塞,各山各洞的神仙、大神、丁甲、功曹、伽蓝、土地,一切得道的师仙,始初俱来听讲,至此各归方位。你看那:灵鹫峰头聚霞彩,极乐世界集祥云。金龙稳卧,玉虎安然。乌兔任随来往,龟蛇凭汝盘旋。丹凤青鸾情爽爽,玄猿白鹿意怡怡。八节奇花,四时仙果。乔松古桧,翠柏修篁。五色梅时开时结,万年桃时熟时新。千果千花争秀,一天瑞霭纷纭。大众合掌皈依,都念:南无燃灯上古佛。南无药师琉璃光王佛。南无释迦牟尼佛。南无过去未来现在佛。南无清净喜佛。南无毗卢尸佛。南无宝幢王佛。南无弥勒尊佛。南无阿弥陀佛。南无无量寿佛。
	
	南无接引归真佛。南无金刚不坏佛。南无宝光佛。南无龙尊王佛。南无精进善佛。南无宝月光佛。南无现无愚佛。南无婆留那佛。南无那罗延佛。南无功德华佛。南无才功德佛。南无善游步佛。南无旃檀光佛。南无摩尼幢佛。南无慧炬照佛。
	
	南无海德光明佛。南无大慈光佛。南无慈力王佛。南无贤善首佛。南无广主严佛。南无金华光佛。南无才光明佛。南无智慧胜佛。南无世静光佛。南无日月光佛。南无日月珠光佛。
	
	南无慧幢胜王佛。南无妙音声佛。南无常光幢佛。南无观世灯佛。南无法胜王佛。南无须弥光佛。南无大慧力王佛。南无金海光佛。南无大通光佛。南无才光佛。南无旃檀功德佛。
	
	南无斗战胜佛。南无观世音菩萨。南无大势至菩萨。南无文殊菩萨。南无普贤菩萨。南无清净大海众菩萨。南无莲池海会佛菩萨。南无西天极乐诸菩萨。南无三千揭谛大菩萨。南无五百阿罗大菩萨。南无比丘夷塞尼菩萨。南无无边无量法菩萨。南无金刚大士圣菩萨。南无净坛使者菩萨。南无八宝金身罗汉菩萨。南无八部天龙广力菩萨。如是等一切世界诸佛,愿以此功德,庄严佛净土。上报四重恩,下济三途苦。若有见闻者,悉发菩提心。同生极乐国,尽报此一身。十方三世一切佛,诸尊菩萨摩诃萨,摩诃般若波罗密。”《西游记》至此终——
	
	输入:中华古籍old126.com
	
	转载请保留
	
	.
	------------
	
	附录 陈光蕊赴任逢灾 江流僧复仇报本
	
	话表陕西大国长安城,乃历代帝王建都之地。自周、秦、汉以来,三州花似锦,八水绕城流,真个是名胜之邦。彼时是大唐太宗皇帝登基,改元贞观,已登极十三年,岁在己巳,天下太平,八方进贡,四海称臣。忽一日,太宗登位,聚集文武众官,朝拜礼毕,有魏征丞相出班奏道:“方今天下太平,八方宁静,应依古法,开立选场,招取贤士,擢用人材,以资化理。”太宗道:
	
	“贤卿所奏有理。”就传招贤文榜,颁布天下:各府州县,不拘军民人等,但有读书儒流,文义明畅,三场精通者,前赴长安应试。
	
	此榜行至海州地方,有一人姓陈名萼,表字光蕊,见了此榜,即时回家,对母张氏道:“朝廷颁下黄榜,诏开南省,考取贤才,孩儿意欲前去应试。倘得一官半职,显亲扬名,封妻荫子,光耀门闾,乃儿之志也。特此禀告母亲前去。”张氏道:“我儿读书人,‘幼而学,壮而行’,正该如此。但去赴举,路上须要小心,得了官,早早回来。”光蕊便吩咐家僮收拾行李,即拜辞母亲,趱程前进。到了长安,正值大开选场,光蕊就进场。考毕中选,及廷试三策,唐王御笔亲赐状元,跨马游街三日。不期游到丞相殷开山门首,有丞相所生一女,名唤温娇,又名满堂娇,未曾婚配,正高结彩楼,抛打绣球卜婿。适值陈光蕊在楼下经过,小姐一见光蕊人材出众,知是新科状元,心内十分欢喜,就将绣球抛下,恰打着光蕊的乌纱帽。猛听得一派笙箫细乐,十数个婢妾走下楼来,把光蕊马头挽住,迎状元入相府成婚。那丞相和夫人,即时出堂,唤宾人赞礼,将小姐配与光蕊。拜了天地,夫妻交拜毕,又拜了岳丈岳母。丞相吩咐安排酒席,欢饮一宵。
	
	二人同携素手,共入兰房。次日五更三点,太宗驾坐金銮宝殿,文武众臣趋朝。太宗同道:“新科状元陈光蕊应授何官?”魏征丞相奏道:“臣查所属州郡,有江州缺官。乞我主授他此职。”太宗就命为江州州主,即令收拾起身,勿误限期。光蕊谢恩出朝,回到相府,与妻商议,拜辞岳丈岳母,同妻前赴江州之任。
	
	离了长安登途,正是暮春天气,和风吹柳绿,细雨点花红。
	
	光蕊便道回家,同妻交拜母亲张氏。张氏道:“恭喜我儿,且又娶亲回来。”光蕊道:“孩儿叨赖母亲福庇,忝中状元,钦赐游街,经过丞相殷府门前,遇抛打绣球适中,蒙丞相即将小姐招孩儿为婿。朝廷除孩儿为江州州主,今来接取母亲,同去赴任。”张氏大喜,收拾行程。在路数日,前至万花店刘小二家安下,张氏身体忽然染病,与光蕊道:“我身上不安,且在店中调养两日再去。”光蕊遵命。至次日早晨,见店门前有一人提着个金色鲤鱼叫卖,光蕊即将一贯钱买了,欲待烹与母亲吃,只见鲤鱼闪闪咪眼,光蕊惊异道:“闻说鱼蛇咪眼,必不是等闲之物!”遂问渔人道:“这鱼那里打来的?”渔人道:“离府十五里洪江内打来的。”光蕊就把鱼送在洪江里去放了生。回店对母亲道知此事,张氏道:“放生好事,我心甚喜。”光蕊道:“此店已住三日了,钦限紧急,孩儿意欲明日起身,不知母亲身体好否?”
	
	张氏道:“我身子不快,此时路上炎热,恐添疾病。你可这里赁间房屋,与我暂住。付些盘缠在此,你两口儿先上任去,候秋凉却来接我。”光蕊与妻商议,就租了屋宇,付了盘缠与母亲,同妻拜辞前去。
	
	途路艰苦,晓行夜宿,不觉已到洪江渡口。只见稍水刘洪、李彪二人,撑船到岸迎接。也是光蕊前生合当有此灾难,撞着这冤家。光蕊令家僮将行李搬上船去,夫妻正齐齐上船,那刘洪睁眼看见殷小姐面如满月,眼似秋波,ying桃小口,绿柳蛮腰,真个有沉鱼落雁之容,闭月羞花之貌,陡起狼心,遂与李彪设计,将船撑至没人烟处,候至夜静三更,先将家僮杀死,次将光蕊打死,把尸首都推在水里去了。小姐见他打死了丈夫,也便将身赴水,刘洪一把抱住道:“你若从我,万事皆休!若不从时,一刀两断!”那小姐寻思无计,只得权时应承,顺了刘洪。那贼把船渡到南岸,将船付与李彪自管,他就穿了光蕊衣冠,带了官凭,同小姐往江州上任去了。
	
	却说刘洪杀死的家僮尸首,顺水流去,惟有陈光蕊的尸首,沉在水底不动。有洪江口巡海夜叉见了,星飞报入龙宫,正值龙王升殿,夜叉报道:“今洪江口不知甚人把一个读书士子打死,将尸撇在水底。”龙王叫将尸抬来,放在面前,仔细一看道:“此人正是救我的恩人,如何被人谋死?常言道,恩将恩报。
	
	我今日须索救他性命,以报日前之恩。”即写下牒文一道,差夜叉径往洪州城隍土地处投下,要取秀才魂魄来,救他的性命。
	
	城隍土地遂唤小鬼把陈光蕊的魂魄交付与夜叉去,夜叉带了魂魄到水晶宫,禀见了龙王。龙王问道:“你这秀才,姓甚名谁?
	
	何方人氏?因甚到此,被人打死?”光蕊施礼道:“小生陈萼,表字光蕊,系海州弘农县人。忝中新科状元,叨授江州州主,同妻赴任,行至江边上船,不料稍子刘洪,贪谋我妻,将我打死抛尸,乞大王救我一救!”龙王闻言道:“原来如此,先生,你前者所放金色鲤鱼即我也,你是救我的恩人,你今有难,我岂有不救你之理?”就把光蕊尸身安置一壁,口内含一颗定颜珠,休教损坏了,日后好还魂报仇。又道:“汝今真魂,权且在我水府中做个都领。”光蕊叩头拜谢,龙王设宴相待不题。
	
	却说殷小姐痛恨刘贼,恨不食肉寝皮,只因身怀有孕,未知男女,万不得已,权且勉强相从。转盼之间,不觉已到江州。
	
	吏书门皂,俱来迎接。所属官员,公堂设宴相叙。刘洪道:“学生到此,全赖诸公大力匡持。”属官答道:“堂尊大魁高才,自然视民如子,讼简刑清。我等合属有赖,何必过谦?”公宴已罢,众人各散。
	
	光阴迅速。一日,刘洪公事远出,小姐在衙思念婆婆、丈夫,在花亭上感叹,忽然身体困倦,腹内疼痛,晕闷在地,不觉生下一子。耳边有人嘱曰:“满堂娇,听吾叮嘱。吾乃南极星君,奉观音菩萨法旨,特送此子与你,异日声名远大,非比等闲。刘贼若回,必害此子,汝可用心保护。汝夫已得龙王相救,日后夫妻相会,子母团圆,雪冤报仇有日也。谨记吾言,快醒快醒!”言讫而去。小姐醒来,句句记得,将子抱定,无计可施。忽然刘洪回来,一见此子,便要淹杀,小姐道:“今日天色已晚,容待明日抛去江中。”幸喜次早刘洪忽有紧急公事远出,小姐暗思:“此子若待贼人回来,性命休矣!不如及早抛弃江中,听其生死。倘或皇天见怜,有人救得,收养此子,他日还得相逢。”但恐难以识认,即咬破手指,写下血书一纸,将父母姓名、跟脚原由,备细开载;又将此子左脚上一个小指,用口咬下,以为记验。取贴身汗衫一件,包裹此子,乘空抱出衙门。幸喜官衙离江不远,小姐到了江边,大哭一场。正欲抛弃,忽见江岸岸侧飘起一片木板,小姐即朝天拜祷,将此子安在板上,用带缚住,血书系在xiong前,推放江中,听其所之。小姐含泪回衙不题。
	
	却说此子在木板上,顺水流去,一直流到金山寺脚下停住。那金山寺长老叫做法明和尚,修真悟道,已得无生妙诀。正当打坐参禅,忽闻得小儿啼哭之声,一时心动,急到江边观看,只见涯边一片木板上,睡着一个婴儿,长老慌忙救起。见了怀中血书,方知来历,取个乳名,叫做江流,托人抚养,血书紧紧收藏。光阴似箭,日月如梭,不觉江流年长一十八岁。长老就叫他削发修行,取法名为玄奘,摩顶受戒,坚心修道。
	
	一日,暮春天气,众人同在松阴之下,讲经参禅,谈说奥妙。那酒肉和尚恰被玄奘难倒,和尚大怒骂道:“你这业畜,姓名也不知,父母也不识,还在此捣甚么鬼!”玄奘被他骂出这般言语,入寺跪告师父,眼泪双流道:“人生于天地之间,禀阴阳而资五行,尽由父生母养,岂有为人在世而无父母者乎?”再三哀告,求问父母姓名。长老道:“你真个要寻父母,可随我到方丈里来。”玄奘就跟到方丈,长老到重梁之上,取下一个小匣儿,打开来取出血书一纸,汗衫一件,付与玄奘。玄奘将血书拆开读之,才备细晓得父母姓名,并冤仇事迹。玄奘读罢,不觉哭倒在地道:“父母之仇,不能报复,何以为人?十八年来,不识生身父母,至今日方知有母亲。此身若非师父捞救抚养,安有今日?容弟子去寻见母亲,然后头顶香盆,重建殿宇,报答师父之深恩也!”师父道:“你要去寻母,可带这血书与汗衫前去,只做化缘,径往江州私衙,才得你母亲相见。”
	
	玄奘领了师父言语,就做化缘的和尚,径至江州。适值刘洪有事出外,也是天教他母子相会,玄奘就直至私衙门口抄化。那殷小姐原来夜间得了一梦,梦见月缺再圆,暗想道:“我婆婆不知音信,我丈夫被这贼谋杀,我的儿子抛在江中,倘若有人收养,算来有十八岁矣,或今日天教相会,亦未可知。”正沉吟间,忽听私衙前有人念经,连叫“抄化”,小姐又乘便出来问道:“你是何处来的?”玄奘答道:“贫僧乃是金山寺法明长老的徒弟。”小姐道:“你既是金山寺长老的徒弟——”叫进衙来,将斋饭与玄奘吃。仔细看他举止言谈,好似与丈夫一般,小姐将从婢打发开去,问道:“你这小师父,还是自幼出家的?还是中年出家的?姓甚名谁?可有父母否?”玄奘答道:“我也不是自幼出家,我也不是中年出家,我说起来,冤有天来大,仇有海样深!我父被人谋死,我母亲被贼人占了。我师父法明长老教我在江州衙内寻取母亲。”小姐问道:“你母姓甚?”玄奘道:“我母姓殷名唤温娇,我父姓陈名光蕊,我小名叫做江流,法名取为玄奘。”小姐道:“温娇就是我。但你今有何凭据?”玄奘听说是他母亲,双膝跪下,哀哀大哭:“我娘若不信,见有血书汗衫为证!”温娇取过一看,果然是真,母子相抱而哭,就叫:“我儿快去!”玄奘道:“十八年不识生身父母,今朝才见母亲,教孩儿如何割舍?”小姐道:“我儿,你火速抽身前去!刘贼若回,他必害你性命!我明日假装一病,只说先年曾许舍百双僧鞋,来你寺中还愿。那时节,我有话与你说。”玄奘依言拜别。
	
	却说小姐自见儿子之后,心内一忧一喜,忽一日推病,茶饭不吃,卧于chuang上。刘洪归衙,问其原故,小姐道:“我幼时曾许下一愿,许舍僧鞋一百双。昨五日之前,梦见个和尚,手执利刃,要索僧鞋,便觉身子不快。”刘洪道:“这些小事,何不早说?”随升堂吩咐王左衙、李右衙:江州城内百姓,每家要办僧鞋一双,限五日内完纳。百姓俱依派完纳讫。小姐对刘洪道:
	
	“僧鞋做完,这里有甚么寺院,好去还愿?”刘洪道:“这江州有个金山寺、焦山寺,听你在那个寺里去。”小姐道:“久闻金山寺好个寺院,我就往金山寺去。”刘洪即唤王、李二衙办下船只。
	
	小姐带了心腹人,同上了船,稍水将船撑开,就投金山寺去。
	
	却说玄奘回寺,见法明长老,把前项说了一遍,长老甚喜。
	
	次日,只见一个丫鬟先到,说夫人来寺还愿,众僧都出寺迎接。
	
	小姐径进寺门,参了菩萨,大设斋衬,唤丫鬟将僧鞋暑袜,托于盘内。来到法堂,小姐复拈心香礼拜,就教法明长老分表与众僧去讫。玄奘见众僧散了,法堂上更无一人,他却近前跪下。小姐叫他脱了鞋袜看时,那左脚上果然少了一个小指头。当时两个又抱住而哭,拜谢长老养育之恩。法明道:“汝今母子相会,恐奸贼知之,可速速抽身回去,庶免其祸。”小姐道:“我儿,我与你一只香环,你径到洪州西北地方,约有一千五百里之程,那里有个万花店,当时留下婆婆张氏在那里,是你父亲生身之母。我再写一封书与你,径到唐王皇城之内,金殿左边,殷开山丞相家,是你母生身之父母。你将我的书递与外公,叫外公奏上唐王,统领人马,擒杀此贼,与父报仇,那时才救得老娘的身子出来。我今不敢久停,诚恐贼汉怪我归迟。”便出寺登舟而去。
	
	玄奘哭回寺中,告过师父,即时拜别,径往洪州。来到万花店,问那店主刘小二道:“昔年江州陈客官有一母亲住在你店中,如今好么?”刘小二道:“他原在我店中,后来昏了眼,三四年并无店租还我,如今在南门头一个破瓦窑里,每日上街叫化度日。那客官一去许久,到如今杳无信息,不知为何。”玄奘听罢,即时问到南门头破瓦窑,寻着婆婆。婆婆道:“你声音好似我儿陈光蕊。”玄奘道:“我不是陈光蕊,我是陈光蕊的儿子。温JiaoXiao姐是我的娘。”婆婆道:“你爹娘怎么不来?”玄奘道:“我爹爹被强盗打死了,我娘被强盗霸占为妻。”婆婆道:“你怎么晓得来寻我?”玄奘道:“是我娘着我来寻婆婆。我娘有书在此,又有香环一只。”那婆婆接了书并香环,放声痛哭道:“我儿为功名到此,我只道他背义忘恩,那知他被人谋死!且喜得皇天怜念,不绝我儿之后,今日还有孙子来寻我。”玄奘问:“婆婆的眼,如何都昏了?”婆婆道:“我因思量你父亲,终日悬望,不见他来,因此上哭得两眼都昏了。”玄奘便跪倒向天祷告道:“念玄奘一十八岁,父母之仇不能报复。今日领母命来寻婆婆,天若怜鉴弟子诚意,保我婆婆双眼复明!”祝罢,就将舌尖与婆婆舔眼。须臾之间,双眼舔开,仍复如初。婆婆觑了小和尚道:
	
	“你果是我的孙子!恰和我儿子光蕊形容无二!”婆婆又喜又悲。玄奘就领婆婆出了窑门,还到刘小二店内,将些房钱赁屋一间与婆婆栖身,又将盘缠与婆婆道:“我此去只月余就回。”
	
	随即辞了婆婆,径往京城。寻到皇城东街殷丞相府上,与门上人道:“小僧是亲戚,来探相公。”门上人禀知丞相,丞相道:“我与和尚并无亲眷。”夫人道:“我昨夜梦见我女儿满堂娇来家,莫不是女婿有书信回来也。”丞相便教请小和尚来到厅上。小和尚见了丞相与夫人,哭拜在地,就怀中取出一封书来,递与丞相。丞相拆开,从头读罢,放声痛哭。夫人问道:“相公,有何事故?”丞相道:“这和尚是我与你的外甥。女婿陈光蕊被贼谋死,满堂娇被贼强占为妻。”夫人听罢,亦痛哭不止。丞相道:“夫人休得烦恼,来朝奏知主上,亲自统兵,定要与女婿报仇。”
	
	次日,丞相入朝,启奏唐王曰:“今有臣婿状元陈光蕊,带领家小江州赴任,被稍水刘洪打死,占女为妻,假冒臣婿,为官多年,事属异变。乞陛下立发人马,剿除贼寇。”唐王见奏大怒,就发御林军六万,着殷丞相督兵前去。丞相领旨出朝,即往教场内点了兵,径往江州进发。晓行夜宿,星落鸟飞,不觉已到江州。殷丞相兵马,俱在北岸下了营寨。星夜令金牌下户唤到江州同知、州判二人,丞相对他说知此事,叫他提兵相助,一同过江而去。天尚未明,就把刘洪衙门围了。刘洪正在梦中,听得火炮一响,金鼓齐鸣,众兵杀进私衙,刘洪措手不及,早被擒住。丞相传下军令,将刘洪一干人犯,绑赴法场,令众军俱在城外安营去了。
	
	丞相直入衙内正厅坐下,请小姐出来相见。小姐欲待要出,羞见父亲,就要自缢。玄奘闻知,急急将母解救,双膝跪下,对母道:“儿与外公,统兵至此,与父报仇。今日贼已擒捉,母亲何故反要寻死?母亲若死,孩儿岂能存乎?”丞相亦进衙劝解。
	
	小姐道:“吾闻妇人从一而终。痛夫已被贼人所杀,岂可-颜从贼?止因遗腹在身,只得忍耻偷生。今幸儿已长大,又见老父提兵报仇,为女儿者,有何面目相见!惟有一死以报丈夫耳!”
	
	丞相道:“此非我儿以盛衰改节,皆因出乎不得已,何得为耻!”
	
	父子相抱而哭,玄奘亦哀哀不止。丞相拭泪道:“你二人且休烦恼,我今已擒捉仇贼,且去发落去来。”即起身到法场,恰好江州同知亦差哨兵拿获水贼李彪解到。丞相大喜,就令军牢押过刘洪、李彪,每人痛打一百大棍,取了供状,招了先年不合谋死陈光蕊情由,先将李彪钉在木驴上,推去市曹,剐了千刀,枭首示众讫;把刘洪拿到洪江渡口先年打死陈光蕊处,丞相与小姐、玄奘,三人亲到江边,望空祭奠,活剜取刘洪心肝,祭了光蕊,烧了祭文一道。
	
	三人望江痛哭,早已惊动水府。有巡海夜叉,将祭文呈与龙王。龙王看罢,就差鳖无帅去请光蕊来到,道:“先生,恭喜!
	
	恭喜!今有先生夫人,公子同岳丈俱在江边祭你,我今送你还魂去也。再有如意珠一颗,走盘珠二颗,绞绡十端,明珠玉带一条奉送。你今日便可夫妻子母相会也。”光蕊再三拜谢。龙王就令夜叉将光蕊身尸送出江口还魂,夜叉领命而去。
	
	却说殷小姐哭奠丈夫一番,又欲将身赴水而死,慌得玄奘拚命扯住。正在仓皇之际,忽见水面上一个死尸浮来,靠近江岸之旁。小姐忙向前认看,认得是丈夫的尸首,一发嚎啕大哭不已。众人俱来观看,只见光蕊舒拳伸脚,身子渐渐展动,忽地爬将起来坐下,众人不胜惊骇。光蕊睁开眼,早见殷小姐与丈人殷丞相同着小和尚俱在身边啼哭。光蕊道:“你们为何在此?”小姐道:“因汝被贼人打死,后来妾身生下此子,幸遇金山寺长老抚养长大,寻我相会。我教他去寻外公,父亲得知,奏闻朝廷,统兵到此,拿住贼人。适才生取心肝,望空祭奠我夫,不知我夫怎生又得还魂。”光蕊道:“皆因我与你昔年在万花店时,买放了那尾金色鲤鱼,谁知那鲤鱼就是此处龙王。后来逆贼把我推在水中,全亏得他救我,方才又赐我还魂,送我宝物,俱在身上。更不想你生下这儿子,又得岳丈为我报仇。真是苦尽甘来,莫大之喜!”
	
	众官闻知,都来贺喜。丞相就令安排酒席,答谢所属官员,即日军马回程。来到万花店,那丞相传令安营。光蕊便同玄奘到刘家店寻婆婆。那婆婆当夜得了一梦,梦见枯木开花,屋后喜鹊频频喧噪,想道:“莫不是我孙儿来也?”说犹未了,只见店门外,光蕊父子齐到。小和尚指道:“这不是俺婆婆?”光蕊见了老母,连忙拜倒。母子抱头痛哭一场,把上项事说了一遍。算还了小二店钱,起程回到京城。进了相府,光蕊同小姐与婆婆、玄奘都来见了夫人。夫人不胜之喜,吩咐家僮,大排筵宴庆贺。
	
	丞相道:“今日此宴可取名为团圆会。”真正合家欢乐。
	
	次日早朝,唐王登殿,殷丞相出班,将前后事情备细启奏,并荐光蕊才可大用。唐王准奏,即命升陈萼为学士之职,随朝理政。玄奘立意安禅,送在洪福寺内修行。后来殷小姐毕竟从容自尽,玄奘自到金山寺中报答法明长老。不知后来事体若何,且听下回分解。
	


\chapter[乱蟠桃大圣偷丹\ 反天宫诸神捉怪]{乱蟠桃大圣偷丹\\反天宫诸神捉怪}

------------

第五回 乱蟠桃大圣偷丹 反天宫诸神捉怪

话表齐天大圣到底是个妖猴更不知官衔品从也不较俸禄高低但只注名便了。(wwW.mianhuatang.la 无弹窗广告)那齐天府下二司仙吏早晚扶侍只知日食三餐夜眠一榻无事牵萦自由自在。闲时节会友游宫交朋结义。见三清称个“老”字;逢四帝道个“陛下”。与那九曜星、五方将、二十八宿、四大天王、十二元辰、五方五老、普天星相、河汉群神俱只以弟兄相待彼此称呼。今日东游明日西荡云去云来行踪不定。

一日玉帝早朝班部中闪出许旌阳真人俯囟启奏道:“今有齐天大圣无事闲游结交天上众星宿不论高低俱称朋友。恐后闲中生事不若与他一件事管庶免别生事端。”玉帝闻言即时宣诏。那猴王欣欣然而至道:“陛下诏老孙有何升赏?”玉帝道:“朕见你身闲无事与你件执事。你且权管那蟠桃园早晚好生在意。”大圣欢喜谢恩朝上唱喏而退。

他等不得穷忙即入蟠桃园内查勘。本园中有个土地拦住问道:“大圣何往?”大圣道:“吾奉玉帝点差代管蟠桃园今来查勘也。”那土地连忙施礼即呼那一班锄树力士、运水力士、修桃力士、打扫力士都来见大圣磕头引他进去。但见那:

夭夭灼灼颗颗株株。夭夭灼灼花盈树颗颗株株果压枝。果压枝头垂锦弹花盈树上簇胭脂。时开时结千年熟无夏无冬万载迟。先熟的酡颜醉脸;还生的带蒂青皮。凝烟肌带绿映日显丹姿。树下奇葩并异卉四时不谢色齐齐。左右楼台并馆舍盘空常见罩云霓。

不是玄都凡俗种瑶池王母自栽培。大圣看玩多时问土地道:“此树有多少株数?”土地道:“有三千六百株:前面一千二百株花微果小三千年一熟人吃了成仙了道体健身轻。中间一千二百株层花甘实六千年一熟人吃了霞举飞升长生不老。后面一千二百株紫纹缃核九千年一熟人吃了与天地齐寿日月同庚。”大圣闻言欢喜无任当日查明了株数点看了亭阁回府。自此后三五日一次赏玩也不交友也不他游。

一日见那老树枝头桃熟大半他心里要吃个尝新。奈何本园土地、力士并齐天府仙吏紧随不便。忽设一计道:“汝等且出门外伺候让我在这亭上少憩片时。”那众仙果退。只见那猴王脱了冠着服爬上大树拣那熟透的大桃摘了许多就在树枝上自在受用。吃了一饱却跳下来簪冠著服唤众等仪从回府。迟三二日又去设法偷桃尽他享用。

一朝王母娘娘设宴大开宝阁瑶池中做“蟠桃胜会”即着那红衣仙女、素衣仙女、青衣仙女、皂衣仙女、紫衣仙女、黄衣仙女、绿衣仙女各顶花篮去蟠桃园摘桃建会。七衣仙女直至园门只见蟠桃园土地、力士同齐天府二司仙吏都在那里把门。仙女近前道:“我等奉王母懿旨到此携桃设宴。”土地道:“仙娥且住。今岁不比往年了玉帝点差齐天大圣在此督理须是报大圣得知方敢开园。”仙女道:“大圣何在?”土地道:“大圣在园内因困倦自家在亭子上睡哩。”仙女道:“既如此寻他去来不可延误。”土地即与同进。寻至花亭不见只有衣冠在亭不知何往。四下里都没寻处。原来大圣耍了一会吃了几个桃子变做二寸长的个人儿在那大树梢头浓叶之下睡着了。七衣仙女道:“我等奉旨前来寻不见大圣怎敢空回?”旁有仙吏道:“仙娥既奉旨来不必迟疑。我大圣闲游惯了想是出园会友去了。汝等且去摘桃我们替你回话便是。”那仙女依言入树林之下摘桃。先在前树摘了二篮又在中树摘了三篮;到后树上摘取只见那树上花果稀疏止有几个毛蒂青皮的。原来熟的都是猴王吃了。七仙女张望东西只见南枝上止有一个半红半白的桃子。青衣女用手扯下枝来红衣女摘了却将枝子望上一放。原来那大圣变化了正睡在此枝被他惊醒。大圣即现本相耳朵内掣出金箍棒幌一幌碗来粗细咄的一声道:“你是那方怪物敢大胆偷摘我桃!”慌得那七仙女一齐跪下道:“大圣息怒。我等不是妖怪乃王母娘娘差来的七衣仙女摘取仙桃大开宝阁做‘蟠桃胜会’。适至此间先见了本园土地等神寻大圣不见。我等恐迟了王母懿旨是以等不得大圣故先在此摘桃万望恕罪。”大圣闻言回嗔作喜道:“仙娥请起。王母开阁设宴请的是谁?”仙女道:“上会自有旧规。请的是西天佛老、菩萨、罗汉南方南极观音东方崇恩圣帝十洲三岛仙翁北方北极玄灵中央黄极黄角大仙这个是五方五老。还有五斗星君上八洞三清、四帝、太乙天仙等众中八洞玉皇、九垒、海岳神仙下八洞幽冥教主、注世地仙。各宫各殿大小尊神俱一齐赴蟠桃嘉会。”大圣笑道:“可请我么?”仙女说:“不曾听得说。”大圣道:“我乃齐天大圣就请我老孙做个尊席有何不可?”仙女道:“此是上会会规今会不知如何。”大圣道:“此言也是难怪汝等。你且立下待老孙先去打听个消息看可请老孙不请。”

好大圣捻着诀念声咒语对众仙女道:“住!住!住!”这原来是个定身法把那七衣仙女一个个睖睖睁睁白着眼都站在桃树之下。大圣纵朵祥云跳出园内竟奔瑶池路上而去。正行时只见那壁厢:

一天瑞霭光摇曳五色祥云飞不绝。白鹤声鸣振九皋紫芝色秀分千叶。

中间现出一尊仙相貌天然丰采别。神舞虹霓幌汉霄腰悬宝录无生灭。

名称赤脚大罗仙特赴蟠桃添寿节。那赤脚大仙觌面撞见大圣大圣低头定计赚哄真仙他要暗去赴会却问:“老道何往?”大仙道:“蒙王母见招去赴蟠桃嘉会。”大圣道:“老道不知。玉帝因老孙筋斗云疾着老孙五路邀请列位先至通明殿下演礼后方去赴宴。”大仙是个光明正大之人就以他的诳语作真。道:“常年就在瑶池演礼谢恩如何先去通明殿演礼方去瑶池赴会?”无奈只得拨转祥云径往通明殿去了。

大圣驾着云念声咒语摇身一变就变做赤脚大仙模样前奔瑶池。不多时直至宝阁按住云头轻轻移步走入里面。只见那里:

琼香缭绕瑞霭缤纷瑶台铺彩结宝阁散氤氲。凤翥鸾腾形缥缈金花玉萼影浮沉。上排着九凤丹霞扆八宝紫霓墩。五彩描金桌千花碧玉盆。桌上有龙肝和凤髓熊掌与猩唇。珍馐百味般般美异果嘉肴色色新。

那里铺设得齐齐整整却还未有仙来。这大圣点看不尽忽闻得一阵酒香扑鼻;忽转头见右壁厢长廊之下有几个造酒的仙官盘糟的力士领几个运水的道人烧火的童子在那里洗缸刷瓮已造成了玉液琼浆香醪佳酿。大圣止不住口角流涎就要去吃奈何那些人都在这里。他就弄个神通把毫毛拔下几根丢入口中嚼碎喷将出去念声咒语叫“变!”即变做几个瞌睡虫奔在众人脸上。你看那伙人手软头低闭眉合眼丢了执事都去盹睡。大圣却拿了些百味珍馐佳肴异品走入长廊里面就着缸挨着瓮放开量痛饮一番。吃勾了多时酕醄醉了。自揣自摸道:“不好!不好!再过会请的客来却不怪我?一时拿住怎生是好?不如早回府中睡去也。”

好大圣:摇摇摆摆仗着酒任情乱撞一会把路差了;不是齐天府却是兜率天宫。一见了顿然醒悟道:“兜率宫是三十三天之上乃离恨天太上老君之处如何错到此间?——也罢!也罢!一向要来望此老不曾得来今趁此残步就望他一望也好。”即整衣撞进去那里不见老君四无人迹。原来那老君与燃灯古佛在三层高阁朱陵丹台上讲道众仙童、仙将、仙官、仙吏都侍立左右听讲。这大圣直至丹房里面寻访不遇但见丹灶之旁炉中有火。炉左右安放着五个葫芦葫芦里都是炼就的金丹。大圣喜道:“此物乃仙家之至宝老孙自了道以来识破了内外相同之理也要些金丹济入不期到家无暇;今日有缘却又撞着此物趁老子不在等我吃他几丸尝新。”他就把那葫芦都倾出来就都吃了如吃炒豆相似。

一时间丹满酒醒又自己揣度道:“不好!不好!这场祸比天还大;若惊动玉帝性命难存。走!走!走!不如下界为王去也!”他就跑出兜率宫不行旧路从西天门使个隐身法逃去。即按云头回至花果山界。但见那旌旗闪灼戈戟光辉原来是四健将与七十二洞妖王在那里演习武艺。大圣高叫道:“小的们!我来也!”众怪丢了器械跪倒道:“大圣好宽心!丢下我等许久不来相顾!”大圣道:“没多时!没多时!”且说且行径入洞天深处。四健将打扫安歇叩头礼拜毕。俱道:“大圣在天这百十年实受何职?”大圣笑道:“我记得才半年光景怎么就说百十年话?”健将道:“在天一日即在下方一年也。”大圣道:“且喜这番玉帝相爱果封做‘齐天大圣’起一座齐天府又设安静、宁神二司司设仙吏侍卫。向后见我无事着我看管蟠桃园。近因王母娘娘设‘蟠桃大会’未曾请我是我不待他请先赴瑶池把他那仙品、仙酒都是我偷吃了。走出瑶池踉踉跄跄误入老君宫阙又把他五个葫芦金丹也偷吃了。但恐玉帝见罪方才走出天门来也。”

众怪闻言大喜。即安排酒果接风将椰酒满斟一石碗奉上大圣喝了一口即咨牙咧嘴道:“不好吃!不好吃!”崩、巴二将道:“大圣在天宫吃了仙酒、仙肴是以椰酒不甚美口。常言道:‘美不美乡中水。’”大圣道:“你们就是‘亲不亲故乡人。’我今早在瑶池中受用时见那长廊之下有许多瓶罐都是那玉液琼浆。你们都不曾尝着。待我再去偷他几瓶回来你们各饮半杯一个个也长生不老。”众猴欢喜不胜。大圣即出洞门又翻一筋斗使个隐身法径至蟠桃会上。进瑶池宫阙只见那几个造酒、盘糟、运水、烧火的还鼾睡未醒。他将大的从左右胁下挟了两个两手提了两个即拨转云头回来会众猴在于洞中就做个“仙酒会”各饮了几杯快乐不题。

却说那七衣仙女自受了大圣的定身法术一周天方能解脱。各提花篮回奏王母说道:“齐天大圣使法术困住我等故此来迟。”王母问道:“你等摘了多少蟠桃?”仙女道:“只有两篮小桃三篮中桃。至后面大桃半个也无想都是大圣偷吃了。及正寻间不期大圣走将出来行凶挖打又问设宴请谁。我等把上会事说了一遍他就定住我等不知去向。只到如今才得醒解回来。”

王母闻言即去见玉帝备陈前事。说不了又见那造酒的一班人同仙官等来奏:“不知甚么人搅乱了‘蟠桃大会’偷吃了玉液琼浆其八珍百味亦俱偷吃了。”又有四个大天师来奏上:“太上道祖来了。”玉帝即同王母出迎。老君朝礼毕道:“老道宫中炼了些‘九转金丹’伺候陛下做‘丹元大会’不期被贼偷去特启陛下知之。”玉帝见奏悚惧。少时又有齐天府仙吏叩头道:“孙大圣不守执事自昨日出游至今未转更不知去向。”玉帝又添疑思。只见那赤脚大仙又俯囟上奏道:“臣蒙王母诏昨日赴会偶遇齐天大圣对臣言万岁有旨着他邀臣等先赴通明殿演礼方去赴会。臣依他言语即返至通明殿外不见万岁龙车凤辇又急来此俟候。”玉帝越大惊道:“这厮假传旨意赚哄贤卿快着纠察灵官缉访这厮踪迹!”

灵官领旨即出殿遍访尽得其详细。回奏道:“搅乱天宫者乃齐天大圣也。”又将前事尽诉一番。玉帝大恼。即差四大天王协同李天王并哪吒太子点二十八宿、九曜星官、十二元辰、五方揭谛、四值功曹、东西星斗、南北二神、五岳四渎、普天星相共十万天兵布一十八架天罗地网下界去花果山围困定捉获那厮处治。众神即时兴师离了天宫。这一去但见那:

黄风滚滚遮天暗紫雾腾腾罩地昏。只为妖猴欺上帝致令众圣降凡尘。四大天王五方揭谛:四大天王权总制五方揭谛调多兵。李托塔中军掌号恶哪吒前部先锋。罗猴星为头检点计都星随后峥嵘。太阴星精神抖擞太阳星照耀分明。五行星偏能豪杰九曜星最喜相争。元辰星子午卯酉一个个都是大力天丁。五瘟五岳东西摆六丁六甲左右行。四渎龙神分上下二十八宿密层层。角亢氐房为总领奎娄胃昴惯翻腾。斗牛女虚危室壁心尾箕星个个能井鬼柳星张翼轸轮枪舞剑显威灵。停云降雾临凡世花果山前扎下营。

诗曰:

天产猴王变化多偷丹偷酒乐山窝。

只因搅乱蟠桃会十万天兵布网罗。

当时李天王传了令着众天兵扎了营把那花果山围得水泄不通。上下布了十八架天罗地网先差九曜恶星出战。九曜即提兵径至洞外只见那洞外大小群猴跳跃顽耍。星官厉声高叫道:“那小妖!你那大圣在那里?我等乃上界差调的天神到此降你这造反的大圣。教他快快来归降;若道半个‘不’字教汝等一概遭诛!”那小妖慌忙传入道:“大圣祸事了!祸事了!外面有九个凶神口称上界来的天神收降大圣。”

那大圣正与七十二洞妖王并四健将分饮仙酒一闻此报公然不理道:“今朝有酒今朝醉莫管门前是与非!”说不了一起小妖又跳来道:“那九个凶神恶言泼语在门前骂战哩!”大圣笑道:“莫睬他。‘诗酒且图今日乐功名休问几时成。’”说犹未了又一起小妖来报:“爷爷!那九个凶神已把门打破了杀进来也!”大圣怒道:“这泼毛神老大无礼!本来不与他计较如何上门来欺我?”即命独角鬼王领帅七十二洞妖王出阵老孙领四健将随后。那鬼王疾帅妖兵出门迎敌却被九曜恶星一齐掩杀抵住在铁板桥头莫能得出。

正嚷间大圣到了。叫一声“开路!”掣开铁棒幌一幌碗来粗细丈二长短丢开架子打将出来。九曜星那个敢抵一时打退。那九曜星立住阵势道:“你这不知死活的弼马温!你犯了十恶之罪先偷桃后偷酒搅乱了蟠桃大会又窃了老君仙丹又将御酒偷来此处享乐。你罪上加罪岂不知之?”大圣笑道:“这几椿事实有!实有!但如今你怎么?”九曜星道:“吾奉玉帝金旨帅众到此收降你快早皈依!免教这些生灵纳命。不然就屣平了此山掀翻了此洞也!”大圣大怒道:“量你这些毛神有何法力敢出浪言不要走请吃老孙一棒!”这九曜星一齐踊跃。那美猴王不惧分毫轮起金箍棒左遮右挡把那九曜星战得筋疲力软一个个倒拖器械败阵而走急入中军帐下对托塔天王道:“那猴王果十分骁勇!我等战他不过败阵来了。”李天王即调四大天王与二十八宿一路出师来斗。大圣也公然不惧调出独角鬼王、七十二洞妖王与四个健将于洞门外列成阵势。你看这场混战好惊人也:

寒风飒飒怪雾阴阴。那壁廊旌旗飞彩这壁厢戈戟生辉。滚滚盔明层层甲亮。滚滚盔明映太阳如撞天的银磬;层层甲亮砌岩崖似压地的冰山。大捍刀飞云掣电楮白枪度雾穿云。方天戟虎眼鞭麻林摆列;青铜剑四明铲密树排阵。弯弓硬弩雕翎箭短棍蛇矛挟了魂。大圣一条如意棒翻来覆去战天神。杀得那空中无鸟过山内虎狼奔。扬砂走石乾坤黑播土飞尘宇宙昏。只听兵兵扑扑惊天地煞煞威威振鬼神。

这一场自辰时布阵混杀到日落西山。那独角鬼王与七十二洞妖怪尽被众天神捉拿去了止走了四健将与那群猴深藏在水帘洞底。这大圣一条棒抵住了四大天神与李托塔、哪吒太子俱在半空中——杀勾多时大圣见天色将晚即拉毫毛一把丢在口中嚼将出去叫声“变!”就变了千百个大圣都使的是金箍棒打退了哪吒太子战败了五个天王。

大圣得胜收了毫毛急转身回洞早又见铁板桥头四个健将领众叩迎那大圣哽哽咽咽大哭三声又唏唏哈哈大笑三声。大圣道:“汝等见了我又哭又笑何也?”四健将道:“今早帅众将与天王交战把七十二洞妖王与独角鬼王尽被众神捉了我等逃生故此该哭。这见大圣得胜回来未曾伤损故此该笑。”大圣道:“胜负乃兵家之常。古人云:‘杀人一万自损三千。’况捉了去的头目乃是虎、豹、狼虫、獾獐、狐骆之类我同类者未伤一个何须烦恼?他虽被我使个分身法杀退他还要安营在我山脚下。我等且紧紧防守饱食一顿安心睡觉养养精神。天明看我使个大神通拿这些天将与众报仇。”四将与众猴将椰酒吃了几碗安心睡觉不题。

那四大天王收兵罢战众各报功:有拿住虎豹的有拿住狮象的有拿住狼虫狐骆的更不曾捉着一个猴精。当时果又安辕营下大寨赏劳了得功之将吩咐了天罗地网之兵个个提铃喝号围困了花果山专待明早大战。各人得令一处处谨守。此正是:妖猴作乱惊天地布网张罗昼夜看。毕竟天晓后如何处治且听下回分解。

\chapter[观音赴会问原因\ 小圣施威降大圣]{观音赴会问原因\\小圣施威降大圣}

且不言天神围绕大圣安歇。话表南海普陀落伽山大慈大悲救苦救难灵感观世音菩萨自王母娘娘请赴蟠桃大会与大徒弟惠岸行者同登宝阁瑶池见那里荒荒凉凉席面残乱;虽有几位天仙俱不就座都在那里乱纷纷讲论。菩萨与众仙相见毕众仙备言前事。菩萨道:“既无盛会又不传杯汝等可跟贫僧去见玉帝。”众仙怡然随往。至通明殿前早有四大天师、赤脚大仙等众俱在此迎着菩萨即道玉帝烦恼调遣天兵擒怪未回等因。菩萨道:“我要见见玉帝烦为转奏。”天师邱弘济即入灵霄宝殿启知宣入。时有太上老君在上王母娘娘在后。

菩萨引众同入里面与玉帝礼毕又与老君、王母相见各坐下。便问:“蟠桃盛会如何?”玉帝道:“每年请会喜喜欢欢今年被妖猴作乱甚是虚邀也。”菩萨道:“妖猴是何出处?”玉帝道:“妖猴乃东胜神洲傲来国花果山石卵化生的。当时生出即目运金光射冲斗府。始不介意继而成精降龙伏虎自削死籍。当有龙王、阎王启奏。朕欲擒拿是长庚星启奏道:‘三界之间凡有九窍者可以成仙。’朕即施教育贤宣他上界封为御马监弼马温官。那厮嫌恶官小反了天宫。即差李天王与哪吒太子收降又降诏抚安宣至上界就封他做个‘齐天大圣’只是有官无禄。他因没事干管理东游西荡。朕又恐别生事端着他代管蟠桃园。他又不遵法律将老树大桃尽行偷吃。及至设会他乃无禄人员不曾请他他就设计赚哄赤脚大仙却自变他相貌入会将仙肴仙酒尽偷吃了又偷老君仙丹又偷御酒若干去与本山众猴享乐。朕心为此烦恼故调十万天兵天罗地网收伏。这一日不见回报不知胜负如何。”

菩萨闻言即命惠岸行者道:“你可快下天宫到花果山打探军情如何。如遇相敌可就相助一功务必的实回话。”惠岸行者整整衣裙执一条铁棍架云离阙径至山前。见那天罗地网密密层层各营门提铃喝号将那山围绕的水泄不通。惠岸立住叫:“把营门的天丁烦你传报。我乃李天王二太子木叉南海观音大徒弟惠岸特来打探军情。”那营里五岳神兵即传入辕门之内。早有虚日鼠、昴日鸡、星日马、房日兔将言传到中军帐下。李天王下令旗教开天罗地网放他进来。此时东方才亮。惠岸随旗进入见四大天王与李天王下拜。拜讫李天王道:“孩儿你自那厢来者?”惠岸道:“愚男随菩萨赴蟠桃会菩萨见胜会荒凉瑶池寂寞引众仙并愚男去见玉帝。玉帝备言父王等下界收伏妖猴一日不见回报胜负未知菩萨因命愚男到此打听虚实。”李天王道:“昨日到此安营下寨着九曜星挑战;被这厮大弄神通九曜星俱败走而回。后我等亲自提兵那厮也排开阵势。我等十万天兵与他混战至晚他使个分身法战退。及收兵查勘时止捉得些狼虫虎豹之类不曾捉得他半个妖猴。今日还未出战。”

说不了只见辕门外有人来报道:“那大圣引一群猴精在外面叫喊。”四大天王与李天王并太子正议出兵。木叉道:“父王愚男蒙菩萨吩咐下来打探消息就说若遇战时可助一功。今不才愿往看他怎么个大圣!”天王道:“孩儿你随菩萨修行这几年想必也有些神通切须在意。”

好太子双手轮着铁棍束一束绣衣跳出辕门高叫:“那个是齐天大圣?”大圣挺如意棒应声道:“老孙便是。你是甚人辄敢问我?”木叉道:“吾乃李天王第二太子叉今在观音菩萨宝座前为徒弟护教法名惠岸是也。”大圣道:“你不在南海修行却来此见我做甚?”木叉道:“我蒙师父差来打探军情见你这般猖獗特来擒你!”大圣道:“你敢说那等大话!且休走!吃老孙这一棒!”木叉全然不惧使铁棒劈手相迎。他两个立那半山中辕门外这场好斗:

棍虽对棍铁各异兵纵交兵人不同。一个是太乙散仙呼大圣一个是

观音徒弟正元龙。浑铁棍乃千锤打六丁六甲运神功;如意棒是天河

定镇海神珍法力洪。两个相逢真对手往来解数实无穷这个的阵

手棍万千凶绕腰贯索疾如风;那个的夹枪棒不放空左遮右挡

怎相容?那阵上旌旗闪闪这阵上驼鼎冬冬。万员天将团团绕一洞

妖猴簇簇丛。怪雾愁云漫地府狼烟煞气射天宫。昨朝混战还犹可

今日争持更又凶。堪羡猴王真本事木叉复败又逃生。

这大圣与惠岸战经五六十合惠岸臂膊酸麻不能迎敌虚幌一幌败阵而走。大圣也收了猴兵安扎在洞门之外。只见天王营门外大小天兵接住了太子让开大路径入辕门对四天王、李托塔、哪吒气哈哈的喘息未定:“好大圣!好大圣!着实神通广大!孩儿战不过又败阵而来也!”李天王见了心惊即命写表求助便差大力鬼王与木叉太子上天启奏。

二人当时不敢停留闯出天罗地网驾起瑞霭祥云。须臾径至通明殿下见了四大天师引至灵霄宝殿呈上表章。惠岸又见菩萨施礼。菩萨道:“你打探的如何?”惠岸道:“始领命到花果山叫开天罗地网门见了父亲道师父差命之意。父王道:‘昨日与那猴王战了一场止捉得他虎豹狮象之类更未捉他一个猴精。’正讲间他又索战是弟子使铁棍与他战经五六十合不能取胜败走回营。父亲因此差大力鬼王同弟子上界求助。”菩萨低头思忖。

却说玉帝拆开表章见有求助之言笑道:“叵耐这个猴精能有多大手段就敢敌过十万天兵!李天王又来求助却将那路神兵助之?”言未毕观音合掌启奏:“陛下宽心贫僧举一神可擒这猴。”玉帝道:“所举者何神?”菩萨道:“乃陛下令甥显圣二郎真君现居灌洲灌江口享受下方香火。他昔日曾力诛六怪又有梅山兄弟与帐前一千二百草头神神通广大。奈他只是听调不听宣陛下可降一道调兵旨意着他助力便可擒也。”玉帝闻言即传调兵的旨意就差大力鬼王赍调。

那鬼王领了旨即驾起云径至灌江口。不消半个时辰直至真君之庙。早有把门的鬼判传报至里道:“外有天使捧旨而至。”二郎即与众兄弟出门迎接旨意焚香开读旨意。上云:

“花果山妖猴齐天大圣作乱。因在宫偷桃、偷酒、偷丹搅乱蟠桃大

会现着十万天兵一十八架天罗地网围山收伏未曾得胜今特

调贤甥同义兄弟即赴花果山助力剿除。成功之后高升重赏。”

真君大喜道:“天使请回吾当就去拔刀相助也。”

鬼王回奏不题。

这真君即唤梅山六兄弟——乃康、张、姚、李四太尉郭申、直健二将军聚集殿前道:“适才玉帝调遣我等往花果山收降妖猴同去去来。”众兄弟俱忻然愿往。即点本部神兵驾鹰牵犬搭弩张弓纵狂风霎时过了东洋大海径至花果山。见那天罗地网密密层层不能前进。因叫道:“把天罗地网的神将听着:吾乃二郎显圣真君蒙玉帝调来擒拿妖猴者快开营门放行。”一时各神一层层传入。四大天王与李天王俱出辕门迎接相见毕问及胜败之事天王将上项事备陈一遍。真君笑道:“小圣来此必须与他斗个变化列公将天罗地网不要幔了顶上只四围紧密让我赌斗。若我输与他不必列公相助我自有兄弟扶持;若赢了他也不必列公绑缚我自有兄弟动手。只请托塔天王与我使个照妖镜住立空中。恐他一时败阵逃窜他方切须与我照耀明白勿走了他。”天王各居四维众天兵各挨排列阵去讫。这真君领着四太尉、二将军连本身七兄弟出营挑战;分付众将紧守营盘收全了鹰犬。众草头神得令真君只到那水帘洞外见那一群猴齐齐整整排作个蟠龙阵势;中军里立一竿旗上书“齐天大圣”四字。真君道:“那泼猴怎么称得起齐天之职?”梅山六弟道:“且休赞叹叫战去来。”那营口小猴见了真君急走去报知。那猴王即掣金箍棒整黄金甲登步云履按一按紫金冠腾出营门急睁眼观看那真君的相貌果是清奇打扮得又秀气。真是个:

仪容清秀貌堂堂两耳垂肩目有光。头戴三山飞凤帽身穿一领淡鹅黄。

缕金靴衬盘龙袜玉带团花八宝妆。腰挎弹弓新月样手执三尖两刃枪。

斧劈桃山曾救母弹打棕罗双凤凰。力诛八怪声名远义结梅山七圣行。

心高不认天家眷性傲归神住灌江。赤城昭惠英灵圣显化无边号二郎。大圣见了笑嘻嘻的将金箍棒掣起高叫道:“你是何方小将辄敢大胆到此挑战?”真君喝道:“你这厮有眼无珠认不得我么!吾乃玉帝外甥敕封昭惠灵王二郎是也。今蒙上命到此擒你这造反天宫的弼马温猢狲你还不知死活!”大圣道:“我记得玉帝妹子思凡下界配合杨君生一男子曾使斧劈桃山的是你么?我行要骂你几声曾奈无甚冤仇待要打你一棒可惜了你的性命。你这郎君小辈可急急回去唤你四大天王出来。”真君闻言心中大怒道:“泼猴!休得无礼!吃吾一刀!”大圣侧身躲过疾举金箍棒劈手相还。他两个这场好杀:

昭惠二郎神齐天孙大圣这个心高欺敌美猴王那个面生压伏真梁

栋。两个乍相逢个人皆睹兴。从来未识浅和深今日方之轻与重。

铁棒赛飞龙神锋如舞凤左挡右攻前迎后映。这阵上梅山六弟助

威风那阵上马流四将传军令。摇旗擂鼓各齐心呐喊筛锣都助兴。

两个钢刀有见机一来一往无丝缝。金箍棒是海中珍变化飞腾能取

胜;若还身慢命该休但要差汽为蹭蹬。

真君与大圣斗经三百馀合不知胜负。那真君抖擞神威摇身一变变得身高万丈两只手举着三尖两刃神锋好便似华山顶上之峰青脸獠牙朱红头恶狠狠望大圣着头就砍。这大圣也使神通变得与二郎身躯一样嘴脸一般举一条如意金箍棒却就是昆仑顶上擎天之柱抵住二郎神唬得那马、流元帅战兢兢摇不得旌旗;崩、巴二将虚怯怯使不得刀剑。这阵上康、张、姚、李、郭申、直健传号令撒放草头神向他那水帘洞外纵着鹰犬搭弩张弓一齐掩杀。可怜冲散妖猴四健将捉拿灵怪二三千!那些猴抛戈弃甲撇剑抛枪;跑的跑喊的喊;上山的上山归洞的归洞;好似夜猫惊宿鸟飞洒满天星。众兄弟得胜不题。

却说真君与大圣变做法天象地的规模正斗时大圣忽见本营中妖猴惊散自觉心慌收了法象掣棒抽身就起。真君见他败走大步赶上道:“那里走趁早归降饶你性命!”大圣不恋战只情跑起将近洞口正撞着康、张、姚、李四太尉郭申、直健二将军一齐帅众挡住道:“泼猴!那里走!”大圣慌了手脚就把金箍棒捏做绣花针藏在耳内摇身一变变作个麻雀儿飞在树稍头钉住。那六兄弟慌慌张张前后寻觅不见一齐吆喝道:“走了这猴精也!走了这猴精也!”

正嚷间真君到了问:“兄弟们赶到那厢不见了?”众神道:“才在这里围住就不见了。”二郎圆睁凤眼观看见大圣变了麻雀儿钉在树上就收了法象撇了神锋卸下弹弓摇身一变变作个雀鹰儿抖开翅飞将去扑打。大圣见了搜的一翅飞起去变作一只大鹚老冲天而去。二郎见了急抖翎毛摇身一变变作一只大海鹤钻上云霄来衔。大圣又将身按下入涧中变作一个鱼儿淬入水内。二郎赶至涧边不见踪迹。心中暗想道:“这猢狲必然下水去也。定变作鱼虾之类。等我再变变拿他。”果一变变作个鱼鹰儿飘荡在下溜头波面上。等待片时那大圣变鱼儿顺水正游忽见一只飞禽似青鹞毛片不青;似鹭鸶顶上无缨;似老鹳腿又不红:“想是二郎变化了等我哩!……”急转头打个花就走。二郎看见道:“打花的鱼儿似鲤鱼尾巴不红;似鳜鱼花鳞不见;似黑鱼头上无星;似鲂鱼腮上无针。他怎么见了我就回去了?必然是那猴变的。”赶上来刷的啄一嘴。那大圣就撺出水中一变变作一条水蛇游近岸钻入草中。二郎因衔他不着他见水响中见一条蛇撺出去认得是大圣急转身又变了一只朱绣顶的灰鹤伸着一个长嘴与一把尖头铁钳子相似径来吃这水蛇。水蛇跳一跳又变做一只花鸨木木樗樗的立在蓼汀之上。二郎见他变得低贱——花鸨乃鸟中至贱至淫之物不拘鸾、凤、鹰、鸦都与交群——故此不去拢傍即现原身走将去取过弹弓拽满一弹子把他打个〔足龙〕踵。

那大圣趁着机会滚下山崖伏在那里又变变一座土地庙儿;大张着口似个庙门;牙齿变做门扇舌头变做菩萨眼睛变做窗棂。只有尾巴不好收拾竖在后面变做一根旗竿。真君赶到崖下不见打倒的鸨鸟只有一间小庙急睁凤眼仔细看之见旗竿立在后面笑道:“是这猢狲了!他今又在那里哄我。我也曾见庙宇更不曾见一个旗竿竖在后面的。断是这畜生弄谊!他若哄我进去他便一口咬住。我怎肯进去?等我掣拳先捣窗棂后踢门扇!”大圣听得心惊道:“好狠!好狠!门扇是我牙齿窗棂是我眼睛;若打了牙捣了眼却怎么是好?”扑的一个虎跳又冒在空中不见。

真君前前后后乱赶只见四太尉、二将军一齐拥至道:“兄长拿住大圣了么?”真君笑道:“那猴儿才自变座庙宇哄我。我正要捣他窗棂踢他门扇他就纵一纵又渺无踪迹。可怪!可怪!”众皆愕然四望更无形影。真君道:“兄弟们在此看守巡逻等我上去寻他。”即纵身驾云起在半空。见那李天王高擎照妖镜与哪吒住立云端真君道:“天王曾见那猴王么?”天王道:“不曾上来。我这里照着他哩。”真君把那睹变化弄神通拿群猴一事说毕却道:“他变庙宇正打处就走了。”李天王闻言又把照妖镜四方一照呵呵的笑道:“真君快去!快去!那猴使了个隐身法走出营围往你那灌江口去也。”二郎听说即取神锋回灌江口来赶。

却说那大圣已至灌江口摇身一变变作二郎爷爷的模样按下云头径入庙里。鬼判不能相认一个个磕头迎接。他坐中间点查香火:见李虎拜还的三牲张龙许下的保福赵甲求子的文书钱丙告病的良愿。正看处有人报:“又一个爷爷来了。”众鬼判急急观看无不惊心。真君却道:“有个甚么齐天大圣才来这里否?”众鬼判道:“不曾见甚么大圣只有一个爷爷在里面查点哩。”真君撞进门大圣见了现出本相道:“郎君不消嚷庙宇已姓孙了。”这真君即举三尖两刃神锋劈脸就砍。那猴王使个身法让过神锋掣出那绣花针儿幌一幌碗来粗细赶到前对面相还。两个嚷嚷闹闹打出庙门半雾半云且行且战复打到花果山慌得那四大天王等众提防愈紧。这康、张太尉等迎着真君合力努力把那美猴王围绕不题。

话表大力鬼王既调了真君与六兄弟提兵擒魔去后却上界回奏。玉帝与观音菩萨、王母并众仙卿正在灵霄殿讲话道:“既是二郎已去赴战这一日还不见回报。”观音合掌道:“贫僧请陛下同道祖出南天门外亲去看看虚实如何?”玉帝道:“言之有理。”即摆驾同道祖、观音、王母与众仙卿至南天门。早有些天丁、力士接着开门遥观只见众天丁布罗网围住四面;李天王与哪吒擎照妖镜立在空中;真君把大圣围绕中间纷纷赌斗呢。菩萨开口对老君说:“贫僧所举二郎神如何?——果有神通已把那大圣围困只是未得擒拿。我如今助他一功决拿住他也。”老君道:“菩萨将甚兵器?怎能助他?”菩萨道:“我将那净瓶杨柳抛下去打那猴头;即不能打死也打一跌教二郎小圣好去拿他。”老君道:“你这瓶是个磁器准打着他便好;如打不着他的头或撞着他的铁棒却不打碎了?你且莫动手等我老君助他一功。”菩萨道:“你有甚么兵器?”老君道:“有有有。”捋起衣袖左膊上取下一个圈子说道:“这件兵器乃锟钢抟炼的被我将还丹点成养就一身灵气善能变化水火不侵又能套诸物;一名‘金钢琢’又名‘金钢套’。当年过函关化胡为佛甚是亏他。早晚最可防身。等我丢下去打他一下。”

话毕自天门上往下一掼滴流流径落花果山营盘里可可的着猴王头上一下。猴王只顾苦战七圣却不知天上坠下这兵器打中了天灵立不稳脚跌了一跤爬将起来就跑;被二郎爷爷的细犬赶上照腿肚子上一口又扯了一跌。他睡倒在地骂道:“这个亡人!你不去妨家长却来咬老孙!”急翻身爬不起来被七圣一拥按住即将绳索捆绑使勾刀穿了琵琶骨再不能变化。

那老君收了金钢琢请玉帝同观音、王母、众仙等俱回灵霄殿。这下面四大天王与李天王诸神俱收兵拔寨近前向小圣贺喜道:“此小圣之功也!”小圣道:“此乃天尊洪福众神威权我何功之有?”康、张、姚、李道:“兄长不必多叙且押这厮去上界见玉帝请旨落去也。”真君道:“贤弟汝等未受天录不得面见玉帝。教天甲神兵押着我同天王等上届回旨。你们帅众在此搜山搜净之后仍回灌口。待我请了赏讨了功回来同乐。”四太尉、二将军依言领诺。这真君与众即驾云头唱凯歌得胜朝天。不多时到通明殿外。天师启奏道:“四大天王等众已捉了妖猴齐天大圣了。来此听宣。”玉帝传旨即命大力鬼王与天丁等众押至斩妖台将这厮碎剁其尸。咦!正是:欺诳今遭刑宪苦英雄气概等时休。毕竟不知那猴王性命如何且听下回分解。

\chapter[八卦炉中逃大圣\ 五行山下定心猿]{八卦炉中逃大圣\\五行山下定心猿}

第七回 八卦炉中逃大圣 五行山下定心猿

富贵功名前缘分定为人切莫欺心。正大光明忠良善果弥深。

些些狂妄天加谴眼前不遇待时临。问东君因甚如今祸害相侵。

只为心高图罔极不分上下乱规箴。

话表齐天大圣被众天兵押去斩妖台下绑在降妖柱上刀砍斧剁枪刺剑刳莫想伤及其身。南斗星奋令火部众神放火煨烧亦不能烧着。又着雷部众神以雷屑钉打越不能伤损一毫。那大力鬼王与众启奏道:“万岁这大圣不知是何处学得这护身之法臣等用刀砍斧剁雷打火烧一毫不能伤损却如之何?”玉帝闻言道:“这厮这等这等如何处治?”太上老君即奏道:“那猴吃了蟠桃饮了御酒又盗了仙丹——我那五壶丹有生有熟被他都吃在肚里。运用三昧火煅成一块所以浑做金钢之躯急不能伤。不若与老道领去放在‘八卦炉’中以文武火煅炼。炼出我的丹来他身自为灰烬矣。”玉帝闻言即教六丁、六甲将他解下付与老君。老君领旨去讫。一壁厢宣二郎显圣赏赐金花百朵御酒百瓶还丹百粒异宝明珠锦绣等件教与义兄弟分享。真君谢恩回灌江口不题。

那老君到兜率宫将大圣解去绳索放了穿琵琶骨之器推入八卦炉中命看炉的道人架火的童子将火煽起煅炼。原来那炉是乾、坎、艮、震、巽、离、坤、兑八卦。他即将身钻在“巽宫”位下。巽乃风也有风则无火。只是风搅得烟来把一双眼熏红了弄做个老害眼病故唤作“火眼金睛”。

真个光阴迅不觉七七四十九日老君的火候俱全。忽一日开炉取丹那大圣双手侮着眼正自搓*揉流涕只听得炉头声响。猛睁眼看见光明他就忍不住将身一纵跳出丹炉忽喇的一声蹬倒八卦炉往外就走。慌得那架火、看炉与丁甲一班人来扯被他一个个都放倒好似癫痫的白额虎风狂的独角龙。老君赶上抓一把被他一捽捽了个倒栽葱脱身走了。即去耳中掣出如意棒迎风幌一幌碗来粗细依然拿在手中不分好歹却又大乱天宫打得那九曜星闭门闭户四天王无影无形。好猴精!有诗为证。诗曰:

混元体正合先天万劫千番只自然。渺渺无为浑太乙如如不动号初玄。

炉中久炼非铅汞物外长生是本仙。变化无穷还变化三皈五戒总休言。

又诗:

一点灵光彻太虚那条拄杖亦如之:或长或短随人用横竖横排任卷舒。

又诗:

猿猴道体假人心心即猿猴意思深。大圣齐天非假论官封弼马岂知音?

马猿合作心和意紧缚拴牢莫外寻。万相归真从一理如来同契住双林。

这一番猴王不分上下使铁棒东打西敌更无一神可挡。只打到通明殿里灵霄殿外。幸有佑圣真君的佐使王灵官执殿。他见大圣纵横掣金鞭近前挡住道:“泼猴何往!有吾在此切莫猖狂!”这大圣不由分说举棒就打。那灵官鞭起相迎。两个在灵霄殿前厮浑一处。好杀:

赤胆忠良名誉大欺天诳上声名坏。一低一好幸相持豪杰英雄同赌赛。铁棒凶金鞭快正直无私怎忍耐?这个是太乙雷声应化尊那个是齐天大圣猿猴怪。金鞭铁棒两家能都是神宫仙器械。今日在灵霄宝殿弄威风各展雄才真可爱。一个欺心要夺斗牛宫一个竭力匡扶玄圣界。苦争不让显神通鞭棒往来无胜败。他两个斗在一处胜败未分早有佑圣真君又差将佐文到雷府调三十六员雷将齐来把大圣围在垓心各骋凶恶鏖战。那大圣全无一毫惧色使一条如意棒左遮右挡后架前迎。一时见那众雷将的刀枪剑戟、鞭简挝锤、钺斧金瓜、旄镰月铲来的甚紧他即摇身一变变做三头六臂;把如意棒幌一幌变作三条;六只手使开三条棒好便似纺车儿一般滴流流在那垓心里飞舞。众雷神莫能相近。真个是:

圆陀陀光灼灼亘古常存人怎学?入火不能焚入水何曾溺?光明一颗摩尼珠剑戟刀枪伤不着。也能善也能恶眼前善恶凭他作。

善时成佛与成仙恶处披毛并带角。无穷变化闹天宫雷将神兵不可捉。当时众神把大圣攒在一处却不能近身乱嚷乱斗早惊动玉帝。遂传旨着游弈灵官同翊圣真君上西方请佛老降伏。

那二圣得了旨径到灵山胜境雷音宝刹之前对四金刚、八菩萨礼毕即烦转达。众神随至宝莲台下启知如来召请。二圣礼佛三匝侍立台下。如来问:“玉帝何事烦二圣下凡?”二圣即启道:“向时花果山产一猴在那里弄神通聚众猴搅乱世界。玉帝降招安旨封为‘弼马温’他嫌官小反去。当遣李天王、哪吒太子擒拿未获复招安他封做‘齐天大圣’先有官无禄。着他代管蟠桃园;他即偷桃;又走至瑶池偷肴偷酒搅乱大会;仗酒又暗入兜率宫偷老君仙丹反出天宫。玉帝复遣十万天兵亦不能收伏。后观世音举二郎真君同他义兄弟追杀他变化多端亏老君抛金钢琢打重二郎方得拿住。解赴御前即命斩之。刀砍斧剁火烧雷打俱不能伤老君准奏领去以火煅炼。四十九日开鼎他却又跳出八卦炉打退天丁径入通明殿里灵霄殿外;被佑圣真君的佐使王灵官挡住苦战又调三十六员雷将把他困在垓心终不能相近。事在紧急因此玉帝特请如来救驾。”如来闻说即对众菩萨道:“汝等在此稳坐法庭休得乱了禅位待我炼魔救驾去来。”

如来即唤阿傩、迦叶二尊者相随离了雷音径至灵霄门外。忽听得喊声振耳乃三十六员雷将围困着大圣哩。佛祖传法旨:“教雷将停息干戈放开营所叫那大圣出来等我问他有何法力。”众将果退。大圣也收了法象现出原身近前怒气昂昂厉声高叫道:“你是那方善士?敢来止住刀兵问我?”如来笑道:“我是西方极乐世界释迦牟尼尊者阿弥陀佛。今闻你猖狂村野屡反天宫不知是何方生长何年得道为何这等暴横?”大圣道:“我本:

天地生成灵混仙花果山中一老猿。水帘洞里为家业拜友寻师悟太玄。

炼就长生多少法学来变化广无边。在因凡间嫌地窄立心端要住瑶天。

灵霄宝殿非他久历代人王有分传。强者为尊该让我英雄只此敢争先。”佛祖听言呵呵冷笑道:“你那厮乃是个猴子成精焉敢欺心要夺玉皇上帝尊位?他自幼修持苦历过一千七百五十劫。每劫该十二万九千六百年。你算他该多少年数方能享受此无极大道?你那个初世为人的畜生如何出此大言!不当人子!不当人子!折了你的寿算!趁早皈依切莫胡说!但恐遭了毒手性命顷刻而休可惜了你的本来面目!”大圣道:“他虽年久修长也不应久占在此。常言道:‘皇帝轮流做明年到我家。’只教他搬出去将天宫让与我变罢了。若还不让定要搅乱永不清平!”佛祖道:“你除了生长变化之法在有何能敢占天宫胜境?”大圣道:“我的手段多哩!我有七十二般变化万劫不老长生。会驾筋斗云一纵十万八千里。如何坐不得天位?”佛祖道:“我与你打个赌赛;你若有本事一筋斗打出我这右手掌中算你赢再不用动刀兵苦争战就请玉帝到西方居住把天宫让你;若不能打出手掌你还下界为妖再修几劫却来争吵。”

那大圣闻言暗笑道:“这如来十分好呆!我老孙一筋斗去十万八千里。他那手掌方圆不满一尺如何跳不出去?”急声道:“既如此说你可做得主张?”佛祖道:“做得!做得!”伸开右手却似个荷叶大小。那大圣收了如意棒抖擞神威将身一纵站在佛祖手心里却道声:“我出去也!”你看他一路云光无影无形去了。佛祖慧眼观看见那猴王风车子一般相似不住只管前进。大圣行时忽见有五根肉红柱子撑着一股青气。他道:“此间乃尽头路了。这番回去如来作证灵霄殿定是我坐也。”又思量说:“且住!等我留下些记号方好与如来说话。”拔下一根毫毛吹口仙气叫“变!”变作一管浓墨双毫笔在那中间柱子上写一行大字云:“齐天大圣到此一游。”写毕收了毫毛。又不庄尊却在第一根柱子根下撒了一泡猴尿。翻转筋斗云径回本处站在如来掌:“我已去今来了。你教玉帝让天宫与我。”

如来骂道:“我把你这个尿精猴子!你正好不曾离了我掌哩!”大圣道:“你是不知。我去到天尽头见五根肉红柱撑着一股青气我留个记在那里你敢和我同去看么?”如来道:“不消去你只自低头看看。”那大圣睁圆火眼金睛低头看时原来佛祖右手中指写着“齐天大圣到此一游。”大指丫里还有些猴尿臊气。大圣大吃了一惊道:“有这等事!有这等事!我将此字写在撑天柱子上如何却在他手指上?莫非有个未卜先知的法术?我决不信!不信!等我再去来!”

好大圣急纵身又要跳出被佛祖翻掌一扑把这猴王推出西天门外将五指化作金、木、水、火、土五座联山唤名“五行山”轻轻的把他压住。众雷神与阿傩、迦叶一个个合掌称扬道:“善哉!善哉!

当年卵化学为人立志修行果道真。万劫无移居胜境一朝有变散精神。

欺天罔上思高位凌圣偷丹乱大伦。恶贯满盈今有报不知何日得翻身。”

如来佛祖殄灭了妖猴即唤傩、迦叶同转西方极乐世界。时有天蓬、天佑急出灵霄宝殿道:“请如来少待我主大驾来也。”佛祖闻言回瞻仰。须臾果见八景鸾舆九光宝盖;声奏玄歌妙乐咏哦无量神章;散宝花喷真香直至佛前谢曰:“多蒙大法收殄妖邪。望如来少停一日请诸仙做一会筵奉谢。”如来不敢违悖即合掌谢道:“老僧承大天尊宣命来此有何法力?还是天尊与众神洪福敢劳致谢?”玉帝传旨即着云部众神分头请三清、四御、五老、六司、七元、八极、九曜、十都、千真万圣来此赴会同谢佛恩。又命四大天师、九天仙女大开玉京金阙、太玄宝宫、洞阳玉馆请如来高坐七宝灵台。调设各班座位安排龙肝凤髓玉液蟠桃。

不一时那玉清元始天尊、上清灵宝天尊、太清道德天尊、五气真君、五斗星君、三官四圣、九曜真君、左辅、右弼、天王、哪吒、元虚一应灵通对对旌旗双双幡盖都摔着明珠异宝寿果奇花向佛前拜献曰:“感如来无量法力收伏妖猴。蒙大天尊设宴呼唤我等皆来陈谢。请如来将此会立一名如何?”如来领众神之托曰:“今欲立名可作个‘安天大会’。”各仙老异口同声俱道:“好个‘安天大会’!好个‘安天大会’!”言讫个坐座位走吅传觞簪花鼓瑟果好会也。有诗为证。诗曰:

宴设蟠桃猴搅乱安天大会胜蟠桃。龙旗鸾辂祥光蔼宝节幢幡瑞气飘。

仙乐玄歌音韵美凤箫玉管响声高。琼香缭绕群仙集宇宙清平贺圣朝。

众皆畅然喜会只见王母娘娘引一班仙子、仙娥、美姬、美女飘飘荡荡舞向佛前施礼曰:“前被妖猴搅乱蟠桃一会今蒙如来大法链锁顽猴喜庆‘安天大会’无物可谢今是我净手亲摘大株蟠桃数枚奉献。”真个是:

半红半绿喷甘香艳丽仙根万载长。堪笑武陵源上种争如天府更奇强!

紫纹娇嫩寰中少缃核清甜世莫双。延寿延年能易体有缘食者自非常。

佛祖合掌向王母谢讫。王母又着仙姬、仙子唱的唱舞的舞。满会群仙又皆赏赞。正是:

缥缈天香满座缤纷仙蕊仙花。玉京金阙大荣华异品奇珍无价。

对对与天齐寿双双万劫增加。桑田沧海任更差他自无惊无讶。

王母正着仙姬仙子歌舞觥筹交错不多时忽又闻得:

一阵异香来鼻嗅惊动满堂星与宿。天仙佛祖把杯停各各抬头迎目候。

霄汉中间现老人手捧灵芝飞蔼绣。葫芦藏蓄万年丹宝录名书千纪寿。

洞里乾坤任自由壶中日月随成就。遨游四海乐清闲散淡十洲容辐辏。

曾赴蟠桃醉几遭醒时明月还依旧。长头大耳短身躯南极之方称老寿。

寿星又到。见玉帝礼毕又见如来申谢道:“始闻那妖猴被老君引至兜率宫煅炼以为必致平安不期他又反出。干如来善伏此怪设宴奉谢故此闻风而来。更无他物可献特具紫芝瑶草碧藕金丹奉上。”诗曰:

碧藕金丹奉释迦如来万寿若恒沙。清平永乐三乘锦康泰长生九品花。

无相门中真法王色空天上是仙家。乾坤大地皆称祖丈六金身福寿赊。

如来欣然领谢。寿星得座依然走吅传觞。只见赤脚大仙又至。向玉帝前俯囟礼毕又对佛祖谢道:“深感法力降伏妖猴。无物可以表敬特具交梨二颗火枣数枚奉献。”诗曰:

大仙赤脚枣梨香敬献弥陀寿算长。七宝莲台山样稳千金花座锦般妆。

寿同天地言非谬福比洪波话岂狂。福寿如期真个是清闲极乐那西方。

如来又称谢了。叫阿傩、迦叶将各所献之物一一收起方向玉帝前谢宴。众各酩酊。只见个巡视灵官来报道:“那大圣伸出头来了。”佛祖道:“不妨不妨。”袖中只抽出一张帖子上有六个金字:“唵、嘛、呢、叭、〔口迷〕、吽”。递与阿傩叫贴在那山顶上。这尊者即领帖子拿出天门到那五行山顶上紧紧的贴在一块四方石上。那座山即生根合缝可运用呼吸之气手儿爬出可以摇挣摇挣。阿傩回报道:“已将帖子贴了。”

如来即辞了玉帝众神与二尊者出天门之外又一个慈悲心念动真言咒语将五行山召一尊土地神祗会同五方揭谛居住此山监押。但他饥时与他铁丸子吃;渴时与他溶化的铜汁饮。待他灾愆满日自有人救他。正是:

妖猴大胆反天宫却被如来伏手降。渴饮溶铜捱岁月饥餐铁弹度时光。

天灾苦困遭磨折人事凄凉喜命长。若得英雄重展挣他年奉佛上西方。

又诗曰:

伏逞豪强大事兴降龙伏虎弄乖能。偷桃偷酒游天府受录承恩在玉京。

恶贯满盈身受困善根不绝气还升。果然脱得如来手且待唐朝出圣僧。

毕竟不知何年何月方满灾殃且听下回分解。

\chapter[我佛造经传极乐\ 观音奉旨上长安]{我佛造经传极乐\\观音奉旨上长安}

我佛造经传极乐 观音奉旨上长安

试问禅关参求无数往往到头虚老。磨砖作镜积雪为粮迷了几多年少?毛吞大海芥纳须弥金色头防微笑。悟时十地三乘凝滞了四生六道。谁听得绝想崖前无阴树下杜宇一声春晓?曹溪路险暨岭云深此处故人音沓。千丈冰崖五叶莲开古殿帘垂香袅。那时节识破源流便见龙王三宝。

这一篇词名《苏武慢》。话表我佛如来辞别了玉帝回至雷音宝刹但见那三千诸佛、五百阿罗、八大金刚、无边菩萨一个个都执着幢幡宝盖异宝仙花摆列在灵山仙境.婆罗双林之下接迎。如来驾住祥云对众道:“我以甚深般苦遍现三界。根本性原毕竟寂灭。同虚空相一无所有。殄伏乖猴.是事莫识。名生死始法相如。”说罢放舍利之光满空有白虹四十二道南北通连。大众见了皈身礼拜。少顷间聚庆云彩雾登上品莲台端然坐下。那三千诸佛、五百罗汉、八金刚、四菩萨合掌近前礼毕问日:“闹天宫搅乱皤桃者谁也?”如来道:“那厮乃花果山产的一妖猴罪恶滔天不可名状。概天神将俱莫能降伏虽二郎捉获。老君用火锻炼亦莫能伤损。我去时正在雷将中间扬威耀武卖弄精神被我止住兵戈问他来历。他言有神通会变化又驾筋斗云.一去十万八千里。我与他打了个赌赛他出不得我手却将他一把抓住指化五行山封压他在那里。五帝大开金阙瑶宫请我坐了席.立安天大会谢我却方辞驾而回。”

大众听言喜悦极口称扬。谢罢各分班而退各执乃事共乐天真。果然是:

瑞霭漫天竺虹光拥世尊。西方称第一无相法王门!常见玄猿献果糜鹿衔花;青鸾舞彩凤鸣;灵龟捧寿仙鹤擒芝。安享净土袛园受用龙宫法界。日日开花时时果熟习静归真参禅果正。不灭不生不增不减。烟霞缥缈随来往寒暑无侵不记年。

诗曰:

去来自在任优游也无恐怖也无愁。极乐场中俱坦荡大千之处没春秋。

佛祖居一月灵山大雷音宝刹之间一日唤聚诸佛阿罗、揭谛。菩萨、金刚、比丘增、尼等众曰:“自伏乖猿安天之后我处不知年月料凡间有半千年矣今值孟秋望日。我有一宝盆.具设百样花千般异果等物与法等享此‘孟兰盆会’如何?”慨众一个个合掌礼佛三匝.领会。如来却将宝盆中花果品物着阿傩捧走着迎叶布散、大众感激。各献诗伸谢。

福诗曰:

福圣光耀性尊前福纳弥深远更绵。福德无疆同地久福缘有庆与天连。福田广种年年盛福海洪深岁岁坚。福满乾坤多福荫福增无量永周全。

禄诗曰:

禄重如山彩凤鸣禄随时泰视长庚。禄添万斛身康健禄享千钟也太平。禄俸齐天还永固禄名似海更澄清。禄思远继多瞻仰禄爵无边万国荣。

寿诗曰:

寿星献彩对如来.寿域光华自此开。寿果满盘生瑞霭寿花新采插莲台。寿诗清雅多奇妙寿曲调音按美才。寿命延长同日月寿如山海更悠哉。

众菩萨献毕因请如来明示根本指解源流。那如来微开善口敷演大法宣扬正果讲的是三乘妙典五蕴得严。但见那天龙同绕花雨缤纷。正是:“禅心朗照千江月真性情涵万里天。”如来讲罢对众言回:“我现四大部洲众生善恶各方不一:东胜神洲者.敬天礼地。心爽气平;北巨芦洲者虽好亲生只因糊口性拙情流.无多作践;我西牛贺洲者不贪不杀养气潜灵虽无上真人人固寿;但那南赠部洲者贪淫乐祸多杀多争正所谓口舌凶场是非恶海。我今有三藏真经可以劝人为善。”

诸菩萨闻言合掌皈依向佛前问曰:“如来有哪三藏真经7”如来回:“我有法一藏谈天;论一藏说地;经一藏度鬼;三藏共计三十五部该一万五千一百四十四卷乃是修真之径正善之门。我待要送上东土叵耐那方众生愚蠢毁谤真言不识我法门之要旨怠慢了瑜迦之正宗。怎么得一个有法力的去东土寻一个善信.教他苦历千山远经万水到我处求取真经永传东土劝他众生却乃是个山大的福缘海深的善庆、谁肯去走一遭来?”当有观音菩萨行近莲台.礼佛三匝道:“弟子不才愿上东土寻一个取经人来也。”诸众抬头观看那菩萨:

理圆四德智满金身。缨络垂珠翠香环结宝明乌云巧叠盘龙警绣带轻飘彩凤翎。碧玉纽素罗袍祥光笼罩;锦城裙金落索瑞气遮迎。眉如小月眼似双星。五面天生喜朱唇一点红。净瓶甘露年年盛斜插垂杨岁岁青。解八难度群生大慈悯:故镇大山居南海救苦寻声万称万应千圣千灵。兰心欣紫竹意性爱香藤。他是落伽山上慈悲主潮音洞里活观音。

如来见了心中大喜道:“别个是也去不得须是观音尊者、神通广大方可去得。”菩萨道;“弟子此去东土有甚言语吩咐?”如来道;“这一去。要踏看路道不许在霄汉中行须是要半云半雾;目过山水谨记程途远近之数叮咛那取经人。但恐善信难行我与你五件宝贝。”即命阿傩、迦叶取出“锦澜袈裟”一领“九环锡杖”一根对菩萨言回;“这袈裟、锡杖。可与那取经人亲用。若肯坚心来此穿我的袈裟免堕轮回;持我的锡枚不遭毒害。”

这菩萨皈依拜领如来又取三个箍儿递与菩萨道:“此宝唤做‘紧箍儿’虽是一样三个但只是用各不同。我有‘金紧禁’的咒语三篇。假若路上撞见神通广大的妖魔。你须是劝他学好跟那取经人做个徒弟。他若不伏使唤可将此箍儿与他带在头上自然见肉生根。各依所用的咒语念一念眼胀头痛脑门皆裂管教他入我门来。”

那菩萨闻言踊跃作礼而退即唤惠岸行者随行。那惠岸使一条浑铁棍重有千斤只在菩萨左右作一个降魔的大力士。菩萨遂将镜湖袈裟作一个包裹令他背了。菩萨将金箍藏了执了锡枚径下灵山。这一去有分教:佛子还来归本愿金蝉长老裹¥檀。

那菩萨到山脚下有玉真观金顶大仙在观门接住请菩萨献茶。菩萨不敢久停对大仙曰;“今领如来法旨上东土寻取经人去。”大仙道:“取经人几时方到?”菩萨道:“未定约莫二三年间或可至此。”遂辞了大仙半云半雾约记程途。有诗为证。诗曰:万里相寻自不言却云谁得意难全?求人忽若浑如此是我平生岂偶然?传道有方成是语说明无信也虚传。愿倾肝胆寻相识料想前头必有缘。

师徒二人正走间.忽然见弱水三千乃是流沙河界。菩萨道:

“徒弟呀.此处却是难行。取经人浊骨凡胎如何得渡了”惠岸道:“师父你看河有多远?”那菩萨停云步看时.只见:

东连沙碛两抵诸番;南达乌戈北通鞑靼。径过有八百里遥.上下有千万里远。水流一似地翻身浪滚却如山耸背。洋洋浩浩漠漠茫茫十里遥闻万丈洪。仙槎难到此莲叶莫能浮。衰草斜阳流曲浦黄云影日暗长堤。那里得客商来往?何曾有渔叟依栖?平沙无雁落远岸有猿啼。只是红蓼花絮知景色白苹香细任依依。

菩萨正然点看只见那河中泼刺一声响亮水波里跳出一个妖魔来十分丑恶。他生得:

青不青黑不黑晦气色脸;长不长短不短赤脚筋躯。眼光闪烁好似灶底双灯;口角角丫叉.就如屠家火钵。撩牙撑剑刃红乱蓬松。一声叱咤如雷吼两脚奔波似滚风。

那怪物手执一根宝杖走上岸就捉菩萨.却被惠岸掣浑铁棒挡住喝声:“休走!”那怪物就持定杖来迎。两个在流沙河边。这一场恶杀真个惊人:

木吒浑铁棒护法显神通;怪物降妖杖努力逞英雄。双条银蟒河边舞一对神谱岸上冲。那一个威镇流沙施本事这一个力保观音建大功。那一个翻波跃浪.这一个吐雾喷云。翻波跃浪乾坤暗吐雾喷云日月昏。那个降妖杖好便似出山的白虎;这个浑铁棒却就如卧道的黄龙。那个使将来.寻蛇拨草;这个丢开去扑鹞分松。只杀得昏漠漠星辰灿烂;雾腾腾天地腾胧。那个久住弱水惟他狠。这个初出灵山第一功。

他两个来来往往战上数十合不分胜负。那怪物架住了铁棒道;“你是哪里和尚敢来与我抵敌?”木吒道:“我是托塔天王二太子木吒惠岸行者今保我师父往东土寻取经人去。你是何怪敢大胆阻路?”那怪方才醒悟道:“我记得你踉南海观音在紫竹林中修行你为何来此?”木呼道:“那岸上不是我师父?”

怪物闻言连声喏喏收了宝杖让木吒揪了去见观音。纳头下拜告道:“菩萨恕我之罪待我诉告。我不是妖邪我是灵霄殿下侍銮舆的卷帘大将。只因在蟋桃会上失手打碎了玻璃盏玉帝把我打了八百贬下界来变得这般模样;又教七日一次将飞剑来穿我胸胁百余下方回故此这般苦恼。没奈何饥寒难忍三二日间出波涛寻一个行人食用。不期今日无知冲撞了大慈菩萨。”菩萨道:“你在天有罪既贬下来今又这等伤生正所谓罪上加罪。我今领了佛旨.上东上寻取经人。你何不入我门来皈依善果跟那取经人做个徒弟上西天拜佛求经?我教飞剑不来穿你。那时节功成免罪复你本职心下如何?”

那怪道:“我愿皈正果。”乃向前道:“菩萨我在此间吃人无数向来有几次取经人来都被我吃了。凡吃的人头抛落流沙竟沉水底(这个水鹅毛也不能浮)惟有九个取经人的骷髅浮在水面再不能沉。我以为异物将索儿穿在一处闲时拿来顽耍这去但恐取经人不得到此却不是反误了我的前程也?”菩萨日:“岂有不到之理?你可将骷髅地挂在头顶下等候取经入自有用处。”怪物道:“既然如此愿领教诲。”菩萨方与他摩项受戒指沙为姓就姓了沙起个法名叫做个沙悟净。当时入了沙门送菩萨过了河他洗心涤虑.再不伤生专等取经人。

菩萨与他别了同木吒径奔东土。行了多时又见一座高山山上有恶气遮漫不能步上。正欲驾云过山不觉狂风起处又闪上一个妖魔。他生得又甚凶险:

卷上莲蓬吊搭嘴耳如蒲扇显金睛。獠牙锋利如钢挫长嘴张开似火盆。金盔紧系腮边带勒甲丝绦蟒退鳞。手执钉把龙探爪腰挎弯弓月十轮。纠纠威风欺太岁昂昂志气压天神。他撞上来不分好歹望菩萨举钉把就筑被木呼行者挡住大喝一声道:“那泼怪休得无礼!看棒!”妖魔道:“这和尚不知死活!看钯!”两个在山底下一冲一撞赌斗输赢。真个好杀;

妖魔凶猛惠岸威能。铁棒分心捣钉钻劈面迎。播土扬尘天地暗飞砂走石鬼神惊。九齿钯光耀耀双环响亮;一条棒黑悠悠两手飞腾。这个是天王太子那个是元帅精灵。一个在普陀为护法一个在山洞作妖精。这场相遇争高下不知那个亏输那个赢。

他两个正杀到好处观世音在半空中抛下莲花隔开钯杖。怪物见了心惊便问:“你是哪里和尚敢弄甚么‘眼前花’哄我?”木吒道:“我把你这个肉眼凡胎的泼物!我是南海菩萨的徒弟。这是我师父抛来的莲花你也不认得哩!”那怪道:“南海菩萨可是扫三灾救八难的观世音么?”木吐道:“不是他是谁?”怪物撇了钉把纳头下礼道;“老兄菩萨在哪里?累烦你引见一引见。”木吐仰面指道:“哪不是?”怪物朝上磕头厉声高叫道:“菩萨恕罪!恕罪!”

观音按下云头前来问道:“你是那里成精的野豕何方作怪的老彘敢在此间挡我?”那怪道:“我不是野豕亦不是老彘我本是天河里天蓬元帅。只因带酒戏弄嫦娥玉帝把我打了二千锤贬下尘凡;一灵真性竟来夺舍投胎不期错了道路投在个母猪胎里变得这般模样。是我咬杀母猪打死群彘在此处占了山场吃人度日。不期撞着菩萨万望拨救拔救。”塔萨道:“此山叫做甚么山?”怪物道:“叫做福陵山。山中有一洞叫做云栈洞。洞里原有个卵二姐。

他见我有些武艺把我做个家长又唤做‘倒查门’。不上一年他死了将一洞的家当尽归我受用。在此日久年深没有个赡身的勾当.菩萨道:“古人云:‘若要有前程莫做没前程。’你既上界违法.今又不改凶心伤生造孽却不是二罪俱罚?”那怪道:“前程!前程!若依你教我喝风!常言道:‘依着官法打杀依着佛法饿杀。’去也!去也!还不如捉个行人肥腻腻的吃他家娘!管甚么二罪三罪千罪万罪!”菩萨道:“‘人有善愿天必从之。’汝若肯皈依正果自有养身之处。世有五谷尽能济饥为何吃人度日?

怪物闻言似梦方觉向菩萨道:“我欲从正奈何‘获罪于天无所祷也’!”菩萨道:“我领了佛旨上东土寻取经人。你可跟他做个徒弟往西天走一遭来将功折罪管教你脱离灾瘴。”那怪满口道:“愿随!愿随!”菩萨才与他摩顶受戒指身为姓就姓了猪替他起个法名就叫做猪悟能。遂此领命归真持斋把素断绝了五荤三厌专候那取经人。

菩萨却与木吒辞了悟能半兴云雾前来、正走处只见空中有一条玉龙叫唤。菩萨近前问日:“你是何龙在此受罪?”那龙道:“我是西海龙王敖闰之子。因纵火烧了殿上明珠我父王表奏天庭告了忤逆。五帝把我吊在空中。打了三百不日遭诛。望菩萨搭救搭救。”

观音闻言。即与木吒撞上南天门里。早有丘、张二天师接着问道:“何往?”菩萨道:“贫僧要见玉帝一面。”二天师即忙上奏。玉帝遂下殿迎接。菩萨上前礼毕道:“贫僧领佛旨上东土寻取经人路遇孽龙悬吊特来启奏饶地性命赐与贫僧教他与取经人做个脚力。”五帝闻言即传旨赦宥差天将解放送与菩萨。菩萨谢恩而出。这小龙叩头谢活命之恩听从菩萨使唤。菩萨把他送在深涧之中只等取经人来变做白马上西方立功。小龙领命潜身不题。

菩萨带引木吒行者过了此山又奔东土。行不多时忽见金光万道瑞气千条。木吒道:“师父那放光之处乃是五行山了:见有如来的‘压帖’在那里。”菩萨道:“此却是那搅乱皤桃会大闹天宫的齐天大圣今乃压在此也。”木吒道:“正是正是。”师徒俱上山来观看帖子乃是“唵嘛呢叭[口迷]吽”六字真言。菩萨看罢叹惜不已作诗一。诗曰:

堪叹妖猴不奉公当年狂妄逞英雄。欺心搅乱皤桃会大胆私行兜率宫。十万军中无敌手.九重天上有威风。自遭我佛如来困何日舒伸再显功!

师徒们正说话处早惊动了那大圣。大全在山根下高叫道:

“是那个在山上吟诗揭我的短哩?”菩萨闻言径下山来寻着。只见那石崖之下有土地、山神、监押大圣的天将都来拜接了菩萨引至那大圣面前。看时他原来压于石匣之中口能言身不能动。菩萨道:“姓孙的你认得我么?”大圣睁开火眼金睛点着头儿高叫道;“我怎么不认得你。你好的是那南海普陀落伽山救苦救难大慈大悲南无观世音菩萨。承看顾!承看顾!我在此度日如年更无一个相知的来看我一看。你从哪里来也?”菩萨道:“我奉佛旨上东土寻取经人去从此经过特留残步看你。”大圣道:“如来哄了我把我压在此山五百余年了不能展挣万望菩萨方便一二救我老孙一救!”菩萨道;“你这厮罪业弥深救你出来恐你又生祸害。反为不美。”大圣道:“我已知悔了但愿大慈悲指条门路情愿修行。”这才是:

人心生一念天地尽皆知。善恶若无报乾坤必有私。

那菩萨闻得此言满心欢喜对大圣道:“圣经云:‘出其言善。

则千里之外应之;出其言不善则千里之外适之。’你既有此心待我到了东土大唐国寻一个取经的人来教他救你。你可跟他做个徒弟秉教伽持入我佛门。再修正果如何?”大圣声声道:“愿去!愿去!”菩萨道:“既有善果我与你起个法名。”大圣道:“我已有名了叫做孙悟空。”菩萨又喜道:“我前面也有二人归降正是‘悟’字排行。你今也是‘悟’字却与他相合甚好甚好。这等也不消叮嘱我去也。”那大圣见性明心归佛教这菩萨留情在意访神谱。

他与木吒离了此处一直东来不一日就到了长安大唐国。敛雾收云师徒们变作两个疥癫游憎入长安城里竟不觉天晚。行至大市街旁见一座土地庙祠二人径进唬得那土地心慌鬼兵胆战。知是菩萨叩头接入。那土地又急跑报与城隍社令及满长安城各庙神抵都来参见告道:“菩萨恕众神接迟之罪。”菩萨道:“汝等不可走漏消息。我奉佛旨特来此处寻访取经人。借你庙宇权住几日待访着真僧即回。”众神各归本处把个土地赶到城隍庙里暂住他师徒们隐遁真形。

毕竟不知寻出那个取经来且听下回分解。

\chapter[陈光蕊赴任逢灾\ 江流僧复仇报本]{陈光蕊赴任逢灾\\江流僧复仇报本}
\chapter[袁守诚妙算无私曲\ 老龙王拙计犯天条]{袁守诚妙算无私曲\\老龙王拙计犯天条}

诗曰:都城大国实堪观八水周流绕四山。多少帝王兴此处古来天下说长安。此单表陕西大国长安城乃历代帝王建都之地。自周、秦、汉以来三州花似锦八水绕城流。三十六条花柳巷七十二座管弦楼。华夷图上看天下最为头真是奇胜之方。今却是大唐太宗文皇帝登基改元龙集贞观。此时已登极十三年岁在己巳。且不说他驾前有安邦定国的英豪。与那创业争疆的杰士。

却说长安城外泾河岸边有两个贤人:一个是渔翁名唤张稍;一个是樵子名唤李定。他两个是不登科的进士能识字的山人。一日在长安城里卖了肩上柴货了篮中鲤同入酒馆之中吃了半酣各携一瓶顺泾河岸边徐步而回。张稍道:

“李兄我想那争名的因名丧体;夺利的为利亡身;受爵的抱虎而眠;承恩的袖蛇而去。算起来还不如我们水秀山青逍遥自在甘淡薄随缘而过。”李定道:“张兄说得有理。但只是你那水秀不如我的山青。”张稍道:“你山青不如我的水秀。

有一《蝶恋花》词为证词曰:烟波万里扁舟小静依孤篷西施声音绕。涤虑洗心名利少闲攀蓼穗蒹葭草。数点沙鸥堪乐道柳岸芦湾妻子同欢笑。一觉安眠风浪俏无荣无辱无烦恼。”

李定道:“你的水秀不如我的山青。也有个《蝶恋花》词为证词曰:云林一段松花满默听莺啼巧舌如调管。红瘦绿肥春正暖倏然夏至光阴转。又值秋来容易换黄花香堪供玩。迅严冬如指拈逍遥四季无人管。”渔翁道:“你山青不如我水秀受用些好物有一《鹧鸪天》为证仙乡云水足生涯摆橹横舟便是家。活剖鲜鳞烹绿鳖旋蒸紫蟹煮红虾。青芦笋水荇芽菱角鸡头更可夸。娇藕老莲芹叶嫩慈菇茭白鸟英花。”樵夫道:“你水秀不如我山青受用些好物亦有一《鹧鸪天》为证:

崔巍峻岭接天涯草舍茅庵是我家。腌腊鸡鹅强蟹鳖獐豝兔鹿胜鱼虾。香椿叶黄楝芽竹笋山茶更可夸。紫李红桃梅杏熟甜梨酸枣木樨花。”渔翁道:“你山青真个不如我的水秀又有《天仙子》一:一叶小舟随所寓万迭烟波无恐惧。垂钩撒网捉鲜鳞没酱腻偏有味老妻稚子团圆会。鱼多又货长安市换得香醪吃个醉。蓑衣当被卧秋江鼾鼾睡无忧虑不恋人间荣与贵。”樵子道:“你水秀还不如我的山青也有《天仙子》一:茆舍数椽山下盖松竹梅兰真可爱。穿林越岭觅干柴没人怪从我卖或少或多凭世界。将钱沽酒随心快瓦钵磁瓯殊自在。酕醄醉了卧松阴无挂碍无利害不管人间兴与败。”渔翁道:“李兄你山中不如我水上生意快活有一《西江月》为证:“红蓼花繁映月黄芦叶乱摇风。碧天清远楚江空牵搅一潭星动。入网大鱼作队吞钩小鳜成丛。得来烹煮味偏浓笑傲江湖打哄。”樵夫道:“张兄你水上还不如我山中的生意快活亦有《西江月》为证。败叶枯藤满路破梢老竹盈山。女萝干葛乱牵攀折取收绳杀担。虫蛀空心榆柳风吹断头松楠。

采来堆积备冬寒换酒换钱从俺。”渔翁道:“你山中虽可比过还不如我水秀的幽雅有一《临江仙》为证:潮落旋移孤艇去夜深罢棹歌来。蓑衣残月甚幽哉宿鸥惊不起天际彩云开。困卧芦洲无个事三竿日上还捱。随心尽意自安排朝臣寒待漏争似我宽怀?”樵夫道:“你水秀的幽雅还不如我山青更幽雅亦有《临江仙》可证:苍径秋高拽斧去晚凉抬担回来。野花插鬓更奇哉拨云寻路出待月叫门开。稚子山妻欣笑接草床木枕敧捱。蒸梨炊黍旋铺排瓮中新酿熟真个壮幽怀!”渔翁道:

“这都是我两个生意赡身的勾当你却没有我闲时节的好处有诗为证诗曰:闲看天边白鹤飞停舟溪畔掩苍扉。倚篷教子搓钓线罢棹同妻晒网围。性定果然知浪静身安自是觉风微。

绿蓑青笠随时着胜挂朝中紫绶衣。”樵夫道:“你那闲时又不如我的闲时好也亦有诗为证诗曰:闲观缥缈白云飞独坐茅庵掩竹扉。无事训儿开卷读有时对客把棋围。喜来策杖歌芳径兴到携琴上翠微。草履麻绦粗布被心宽强似着罗衣。”

张稍道:“李定我两个真是微吟可相狎不须檀板共金樽。但散道词章不为稀罕且各联几句看我们渔樵攀话何如?”李定道:“张兄言之最妙请兄先吟。”“舟停绿水烟波内家住深山旷野中。偏爱溪桥春水涨最怜岩岫晓云蒙。龙门鲜鲤时烹煮虫蛀干柴日燎烘。钓网多般堪赡老担绳二事可容终。小舟仰卧观飞雁草径斜敧听唳鸿。口舌场中无我分是非海内少吾踪。溪边挂晒缯如锦石上重磨斧似锋。秋月晖晖常独钓春山寂寂没人逢。鱼多换酒同妻饮柴剩沽壶共子丛。

自唱自斟随放荡长歌长叹任颠风。呼兄唤弟邀船伙挈友携朋聚野翁。行令猜拳频递盏拆牌道字漫传钟。烹虾煮蟹朝朝乐炒鸭爊鸡日日丰。愚妇煎茶情散诞山妻造饭意从容。晓来举杖淘轻浪日出担柴过大冲。雨后披蓑擒活鲤风前弄斧伐枯松。潜踪避世妆痴蠢隐姓埋名作哑聋。”张稍道:“李兄我才僭先起句今到我兄也先起一联小弟亦当续之。”“风月佯狂山野汉江湖寄傲老余丁。清闲有分随潇洒口舌无闻喜太平。月夜身眠茅屋稳天昏体盖箬蓑轻。忘情结识松梅友乐意相交鸥鹭盟。名利心头无算计干戈耳畔不闻声。随时一酌香醪酒度日三餐野菜羹。两束柴薪为活计一竿钓线是营生。闲呼稚子磨钢斧静唤憨儿补旧缯。春到爱观杨柳绿时融喜看荻芦青。夏天避暑修新竹六月乘凉摘嫩菱。霜降鸡肥常日宰重阳蟹壮及时烹。冬来日上还沉睡数九天高自不蒸。

八节山中随放性四时湖里任陶情。采薪自有仙家兴垂钓全无世俗形。门外野花香艳艳船头绿水浪平平。身安不说三公位性定强如十里城。十里城高防阃令三公位显听宣声。乐山乐水真是罕谢天谢地谢神明。(wwW.mianhuatang.la 无弹窗广告)”他二人既各道词章又相联诗句行到那分路去处躬身作别。张稍道:“李兄呵途中保重!上山仔细看虎。假若有些凶险正是明日街头少故人!”李定闻言大怒道:“你这厮惫懒!好朋友也替得生死你怎么咒我?我若遇虎遭害你必遇浪翻江!”张稍道:“我永世也不得翻江。”李定道:“天有不测风云人有暂时祸福。你怎么就保得无事?”张稍道:“李兄你虽这等说你还没捉摸;不若我的生意有捉摸定不遭此等事。”李定道:“你那水面上营生极凶极险隐隐暗暗有甚么捉摸?”张稍道:“你是不晓得。这长安城里西门街上有一个卖卦的先生。我每日送他一尾金色鲤他就与我袖传一课依方位百下百着。今日我又去买卦他教我在泾河湾头东边下网西岸抛钓定获满载鱼虾而归。明日上城来卖钱沽酒再与老兄相叙。”二人从此叙别。

这正是路上说话草里有人。原来这泾河水府有一个巡水的夜叉听见了百下百着之言急转水晶宫慌忙报与龙王道:

“祸事了!祸事了!”龙王问:“有甚祸事?”夜叉道:“臣巡水去到河边只听得两个渔樵攀话。相别时言语甚是利害。那渔翁说:长安城里西门街上有个卖卦先生算得最准。他每日送他鲤鱼一尾他就袖传一课教他百下百着。若依此等算准却不将水族尽情打了?何以壮观水府何以跃浪翻波辅助大王威力?”龙王甚怒急提了剑就要上长安城诛灭这卖卦的。旁边闪过龙子龙孙、虾臣蟹士、鲥军师鳜少卿鲤太宰一齐启奏道:

“大王且息怒。常言道过耳之言不可听信。大王此去必有云从必有雨助恐惊了长安黎庶上天见责。大王隐显莫测变化无方但只变一秀士到长安城内访问一番。果有此辈容加诛灭不迟;若无此辈可不是妄害他人也?”龙王依奏遂弃宝剑也不兴云雨出岸上摇身一变变作一个白衣秀士真个丰姿英伟耸壑昂霄。步履端祥循规蹈矩。语言遵孔孟礼貌体周文。身穿玉色罗襕服头戴逍遥一字巾。上路来拽开云步径到长安城西门大街上。只见一簇人挤挤杂杂闹闹哄哄内有高谈阔论的道:“属龙的本命属虎的相冲。寅辰巳亥虽称合局但只怕的是日犯岁君。”龙王闻言情知是那卖卜之处走上前分开众人望里观看只见:四壁珠玑满堂绮绣。

宝鸭香无断磁瓶水恁清。两边罗列王维画座上高悬鬼谷形。

端溪砚金烟墨相衬着霜毫大笔;火珠林郭璞数谨对了台政新经。六爻熟谙八卦精通。能知天地理善晓鬼神情。一槃子午安排定满腹星辰布列清。真个那未来事过去事观如月镜;几家兴几家败鉴若神明。知凶定吉断死言生。开谈风雨迅下笔鬼神惊。招牌有字书名姓神课先生袁守诚。此人是谁?原来是当朝钦天监台正先生袁天罡的叔父袁守诚是也。那先生果然相貌稀奇仪容秀丽名扬大国术冠长安。龙王入门来与先生相见。礼毕请龙上坐童子献茶。先生问曰:

“公来问何事?”龙王曰:“请卜天上阴晴事如何。”先生即袖传一课断曰:“云迷山顶雾罩林梢。若占雨泽准在明朝。”龙王曰:“明日甚时下雨?雨有多少尺寸?”先生道:“明日辰时布云已时雷午时下雨未时雨足共得水三尺三寸零四十八点”。龙王笑曰:“此言不可作戏。如是明日有雨依你断的时辰数目我送课金五十两奉谢。若无雨或不按时辰数目我与你实说定要打坏你的门面扯碎你的招牌即时赶出长安不许在此惑众!”先生欣然而答:“这个一定任你。请了请了明朝雨后来会。”

龙王辞别出长安回水府。大小水神接着问曰:“大王访那卖卦的如何?”龙王道:“有有有!”但是一个掉嘴口讨春的先生。我问他几时下雨他就说明日下雨;问他甚么时辰甚么雨数他就说辰时布云已时雷午时下雨未时雨足得水三尺三寸零四十八点我与他打了个赌赛;若果如他言送他谢金五十两;如略差些就打破他门面赶他起身不许在长安惑众。”众水族笑曰:“大王是八河都总管司雨大龙神有雨无雨惟大王知之他怎敢这等胡言?那卖卦的定是输了!定是输了!”

此时龙子龙孙与那鱼鲫蟹士正欢笑谈此事未毕只听得半空中叫:“泾河龙王接旨。”众抬头上看是一个金衣力士手擎玉帝敕旨径投水府而来。慌得龙王整衣端肃焚香接了旨。

金衣力士回空而去。龙王谢恩拆封看时上写着:“敕命八河总驱雷掣电行;明朝施雨泽普济长安城。”旨意上时辰数目与那先生判断者毫不差唬得那龙王魂飞魄散。少顷苏醒对众水族曰:“尘世上有此灵人!真个是能通天彻地却不输与他呵!”鲥军师奏云:“大王放心。要赢他有何难处?臣有小计管教灭那厮的口嘴。”龙王问计军师道:“行雨差了时辰少些点数就是那厮断卦不准怕不赢他?那时捽碎招牌赶他跑路果何难也?”龙王依他所奏果不担忧。

至次日点札风伯、雷公、云童、电母直至长安城九霄空上。他挨到那巳时方布云午时雷未时落雨申时雨止却只得三尺零四十点改了他一个时辰克了他三寸八点雨后放众将班师。他又按落云头还变作白衣秀士到那西门里大街上撞入袁守诚卦铺不容分说就把他招牌、笔、砚等一齐捽碎。那先生坐在椅上公然不动。这龙王又轮起门板便打、骂道:“这妄言祸福的妖人擅惑众心的泼汉!你卦又不灵言又狂谬!说今日下雨的时辰点数俱不相对你还危然高坐趁早去饶你死罪!”守诚犹公然不惧分毫仰面朝天冷笑道:“我不怕!我不怕!我无死罪只怕你倒有个死罪哩!别人好瞒只是难瞒我也。我认得你你不是秀士乃是泾河龙王。你违了玉帝敕旨改了时辰克了点数犯了天条。你在那剐龙台上恐难免一刀你还在此骂我?”龙王见说心惊胆战毛骨悚然急丢了门板整衣伏礼向先生跪下道:“先生休怪。前言戏之耳岂知弄假成真果然违犯天条奈何?望先生救我一救!

不然我死也不放你。”守诚曰:“我救你不得只是指条生路与你投生便了。”龙曰:“愿求指教。”先生曰:“你明日午时三刻该赴人曹官魏征处听斩。你果要性命须当急急去告当今唐太宗皇帝方好。那魏征是唐王驾下的丞相若是讨他个人情方保无事。”龙王闻言拜辞含泪而去。不觉红日西沉太阴星上但见:烟凝山紫归鸦倦远路行人投旅店。渡头新雁宿眭沙银河现。催更筹孤村灯火光无焰。风袅炉烟清道院蝴蝶梦中人不见。月移花影上栏杆星光乱。漏声换不觉深沉夜已半。

这泾河龙王也不回水府只在空中等到子时前后收了云头敛了雾角径来皇宫门。此时唐王正梦出宫门之外步月花阴忽然龙王变作人相上前跪拜。口叫“陛下救我!救我!”

太宗云:“你是何人?朕当救你。”龙王云:“陛下是真龙臣是业龙。臣因犯了天条该陛下贤臣人曹官魏征处斩故来拜求望陛下救我一救!”太宗曰:“既是魏征处斩朕可以救你。你放心前去。”龙王欢喜叩谢而去。

却说那太宗梦醒后念念在心。早已至五鼓三点太宗设朝聚集两班文武官员。但见那:

烟笼凤阙香蔼龙楼。光摇丹扆动云拂翠华流。君臣相契同尧舜礼乐威严近汉周。侍臣灯宫女扇双双映彩;孔雀屏麒麟殿处处光浮。山呼万岁华祝千秋。静鞭三下响衣冠拜冕旒。宫花灿烂天香袭堤柳轻柔御乐讴。珍珠帘翡翠帘金钩高控;龙凤扇山河扇宝辇停留。文官英秀武将抖搜。御道分高下丹墀列品流。金章紫绶乘三象地久天长万万秋。众官朝贺已毕各各分班。唐王闪凤目龙睛一一从头观看只见那文官内是房玄龄、杜如晦、徐世勣、许敬宗、王珪等武官内是马三宝、段志贤、殷开山、程咬金、刘洪纪、胡敬德、秦叔宝等一个个威仪端肃却不见魏征丞相。唐王召徐世勣上殿道:“朕夜间得一怪梦梦见一人迎面拜谒口称是泾河龙王犯了天条该人曹官魏征处斩拜告寡人救他朕已许诺。今日班前独不见魏征何也?”世勣对曰:“此梦告准须臾魏征来朝陛下不要放他出门。过此一日可救梦中之龙。”唐王大喜即传旨着当驾官宣魏征入朝。

却说魏征丞相在府夜观乾象正爇宝香只闻得九霄鹤唳却是天差仙使捧玉帝金旨一道着他午时三刻梦斩泾河老龙。这丞相谢了天恩斋戒沐浴在府中试慧剑运元神故此不曾入朝。一见当驾官赍旨来宣惶惧无任又不敢违迟君命只得急急整衣束带同旨入朝在御前叩头请罪。唐王出旨道:“赦卿无罪。”那时诸臣尚未退朝至此却命卷帘散朝独留魏征宣上金銮召入便殿先议论安邦之策定国之谋。将近巳末午初时候却命宫人取过大棋来“朕与贤卿对弈一局。”众嫔妃随取棋枰铺设御案。魏征谢了恩即与唐王对弈。

毕竟不知胜负如何且听下回分解。

\chapter[二将军宫门镇鬼\ 唐太宗地府还魂]{二将军宫门镇鬼\\唐太宗地府还魂}

第十回 二将军宫门镇鬼 唐太宗地府还魂

却说太宗与魏征在便殿对弈一递一着摆开阵势。正合《烂柯经》云:博弈之道贵乎严谨。高者在腹下者在边中者在角此棋家之常法。法曰:宁输一子不失一先。击左则视右攻后则瞻前。有先而后有后而先。两生勿断皆活勿连。阔不可太疏密不可太促。与其恋子以求生不若弃之而取胜;与其无事而独行不若固之而自补。彼众我寡先谋其生;我众彼寡务张其势。善胜者不争善阵者不战;善战者不败善败者不乱。夫棋始以正合终以奇胜。凡敌无事而自补者有侵绝之意;弃小而不救者有图大之心。随手而下者无谋之人;不思而应者取败之道。《诗》云:“惴惴小心如临于谷。”此之谓也。诗曰

棋盘为地子为天色按阴阳造化全。

下到玄微通变处笑夸当日烂柯仙。

君臣两个对弈此棋正下到午时三刻一盘残局未终魏征忽然踏伏在案边鼾鼾盹睡。太宗笑曰:“贤卿真是匡扶社稷之心劳创立江山之力倦所以不觉盹睡。”太宗任他睡着更不呼唤不多时魏征醒来俯伏在地道:“臣该万死!臣该万死!却才晕困不知所为望陛下赦臣慢君之罪。”太宗道:“卿有何慢罪?且起来拂退残棋与卿从新更着。”魏征谢了恩却才拈子在手只听得朝门外大呼小叫。原来是秦叔宝、徐茂功等将着一个血淋的龙头掷在帝前启奏道:“陛下海浅河枯曾有见这般异事却无闻。”太宗与魏征起身道:“此物何来?”

叔宝、茂功道:“千步廊南十字街头云端里落下这颗龙头微臣不敢不奏。”唐王惊问魏征:“此是何说?”魏征转身叩头道:

“是臣才一梦斩的。”唐王闻言大惊道:“贤卿盹睡之时又不曾见动身动手又无刀剑如何却斩此龙?”魏征奏道:“主公臣的身在君前梦离陛下。身在君前对残局合眼朦胧;梦离陛下乘瑞云出神抖搜。那条龙在剐龙台上被天兵将绑缚其中。是臣道:‘你犯天条合当死罪。我奉天命斩汝残生。’龙闻哀苦臣抖精神。龙闻哀苦伏爪收鳞甘受死;臣抖精神撩衣进步举霜锋。扢扠一声刀过处龙头因此落虚空。”太宗闻言心中悲喜不一。喜者夸奖魏征好臣朝中有此豪杰愁甚江山不稳?悲者谓梦中曾许救龙不期竟致遭诛。只得强打精神传旨着叔宝将龙头悬挂市曹晓谕长安黎庶一壁厢赏了魏征众官散讫。当晚回宫心中只是忧闷想那梦中之龙哭啼啼哀告求生岂知无常难免此患。思念多时渐觉神魂倦怠身体不安。当夜二更时分只听得宫门外有号泣之声太宗愈加惊恐。正朦胧睡间又见那泾河龙王手提着一颗血淋淋的级高叫:“唐太宗!还我命来!还我命来!你昨夜满口许诺救我怎么天明时反宣人曹官来斩我?你出来你出来!我与你到阎君处折辨折辨!”他扯住太宗再三嚷闹不放太宗箝口难言只挣得汗流遍体。正在那难分难解之时只见正南上香云缭绕彩雾飘飘有一个女真人上前将杨柳枝用手一摆那没头的龙悲悲啼啼径往西北而去。原来这是观音菩萨领佛旨上东土寻取经人此住长安城都土地庙里夜闻鬼泣神号特来喝退业龙救脱皇帝。那龙径到阴司地狱具告不题。

却说太宗苏醒回来只叫“有鬼!有鬼!”慌得那三宫皇后六院嫔妃与近侍太监战兢兢一夜无眠。(wwW.mianhuatang.la 无弹窗广告)不觉五更三点那满朝文武多官都在朝门外候朝。等到天明犹不见临朝唬得一个个惊惧踌躇。及日上三竿方有旨意出来道:“朕心不快众官免朝。”不觉倏五七日众官忧惶都正要撞门见驾问安只见太后有旨召医官入宫用药众人在朝门等候讨信。少时医官出来众问何疾。医官道:“皇上脉气不正虚而又数狂言见鬼又诊得十动一代五脏无气恐不讳只在七日之内矣。”众官闻言大惊失色。正怆惶间又听得太后有旨宣徐茂功、护国公、尉迟公见驾。三公奉旨急入到分宫楼下。拜毕太宗正色强言道:“贤卿寡人十九岁领兵南征北伐东挡西除苦历数载更不曾见半点邪崇今日却反见鬼!”尉迟公道:“创立江山杀人无数何怕鬼乎?”太宗道:“卿是不信。朕这寝宫门外入夜就抛砖弄瓦鬼魅呼号着然难处。白日犹可昏夜难禁。”

叔宝道:“陛下宽心今晚臣与敬德把守宫门看有甚么鬼祟。”

太宗准奏茂功谢恩而出。当日天晚各取披挂他两个介胄整齐执金瓜钺斧在宫门外把守。好将军!你看他怎生打扮:头戴金盔光烁烁身披铠甲龙鳞。护心宝镜幌祥云狮蛮收紧扣绣带彩霞新。这一个凤眼朝天星斗怕那一个环睛映电月光浮。他本是英雄豪杰旧勋臣只落得千年称户尉万古作门神。

二将军侍立门旁一夜天晚更不曾见一点邪崇。是夜太宗在宫安寝无事晓来宣二将军重重赏劳道:“朕自得疾数日不能得睡今夜仗二将军威势甚安。卿且请出安息安息待晚间再一护卫。”二将谢恩而出。遂此二三夜把守俱安只是御膳减损病转觉重。太宗又不忍二将辛苦又宣叔宝、敬德与杜、房诸公入宫吩咐道:“这两日朕虽得安却只难为秦、胡二将军彻夜辛苦。朕欲召巧手丹青传二将军真容贴于门上免得劳他如何?”众臣即依旨选两个会写真的着胡、秦二公依前披挂照样画了贴在门上夜间也即无事。

如此二三日又听得后宰门乒乓乒乓砖瓦乱响晓来急宣众臣曰:“连日前门幸喜无事今夜后门又响却不又惊杀寡人也!”茂功进前奏道:“前门不安是敬德、叔宝护卫;后门不安该着魏征护卫。”太宗准奏又宣魏征今夜把守后门。征领旨当夜结束整齐提着那诛龙的宝剑侍立在后宰门前真个的好英雄也!他怎生打扮:熟绢青巾抹额锦袍玉带垂腰兜风氅袖采霜飘压赛垒荼神貌。脚踏乌靴坐折手持利刃凶骁。圆睁两眼四边瞧那个邪神敢到?一夜通明也无鬼魅。虽是前后门无事只是身体渐重。一日太后又传旨召众臣商议殡殓后事。太宗又宣徐茂功吩咐国家大事叮嘱仿刘蜀主托孤之意。言毕沐浴更衣待时而已。旁闪魏征手扯龙衣奏道:

“陛下宽心臣有一事管保陛下长生。”太宗道:“病势已入膏肓命将危矣如何保得?”征云:“臣有书一封进与陛下捎去到冥司付酆都判官崔珪。”太宗道:“崔珪是谁?”征云:“崔珪乃是太上先皇帝驾前之臣先受兹州令后升礼部侍郎。在日与臣八拜为交相知甚厚。他如今已死现在阴司做掌生死文簿的酆都判官梦中常与臣相会。此去若将此书付与他他念微臣薄分必然放陛下回来管教魂魄还阳世定取龙颜转帝都。”太宗闻言接在手中笼入袖里遂瞑目而亡。那三宫六院、皇后嫔妃、侍长储君及两班文武俱举哀戴孝又在白虎殿上停着梓宫不题。

却说太宗渺渺茫茫魂灵径出五凤楼前只见那御林军马请大驾出朝采猎。太宗欣然从之缥渺而去。行多时人马俱无。独自个散步荒郊草野之间。正惊惶难寻道路只见那一边有一人高声大叫道:“大唐皇帝往这里来!往这里来!”太宗闻言抬头观看只见那人:头顶乌纱腰围犀角。头顶乌纱飘软带腰围犀角显金厢。手擎牙笏凝祥霭身着罗袍隐瑞光。

脚踏一双粉底靴登云促雾;怀揣一本生死簿注定存亡。鬓蓬松飘耳上胡须飞舞绕腮旁。昔日曾为唐国相如今掌案侍阎王。太宗行到那边只见他跪拜路旁口称“陛下赦臣失悮远迎之罪!”太宗问曰:“你是何人?因甚事前来接拜?”那人道:

“微臣半月前在森罗殿上见泾河鬼龙告陛下许救反诛之故第一殿秦广大王即差鬼使催请陛下要三曹对案。臣已知之故来此间候接不期今日来迟望乞恕罪恕罪。”太宗道:“你姓甚名谁?是何官职?”那人道:“微臣存日在阳曹侍先君驾前为兹州令后拜礼部侍郎姓崔名珪。今在阴司得受酆都掌案判官。”太宗大喜近前来御手忙搀道:“先生远劳。朕驾前魏征有书一封正寄与先生却好相遇。”判官谢恩问书在何处。太宗即向袖中取出递与崔珪。珪拜接了拆封而看。其书曰:辱爱弟魏征顿书拜大都案契兄崔老先生台下:忆昔交游音容如在。倏尔数载不闻清教。常只是遇节令设蔬品奉祭未卜享否?又承不弃梦中临示始知我兄长大人高迁。奈何阴阳两隔天各一方不能面觌。今因我太宗文皇帝倏然而故料是对案三曹必然得与兄长相会。万祈俯念生日交情方便一二放我陛下回阳殊为爱也。容再修谢。不尽。”那判官看了书满心欢喜道:“魏人曹前日梦斩老龙一事臣已早知甚是夸奖不尽。又蒙他早晚看顾臣的子孙今日既有书来陛下宽心微臣管送陛下还阳重登玉阙。”太宗称谢了。

二人正说间只见那边有一对青衣童子执幢幡宝盖高叫道:“阎王有请有请。”太宗遂与崔判官并二童子举步前进。

忽见一座城城门上挂着一面大牌上写着“幽冥地府鬼门关”七个大金字。那青衣将幢幡摇动引太宗径入城中顺街而走。

只见那街旁边有先主李渊先兄建成故弟元吉上前道:“世民来了!世民来了!”那建成、元吉就来揪打索命。太宗躲闪不及被他扯住。幸有崔判官唤一青面獠牙鬼使喝退了建成、元吉太宗方得脱身而去。行不数里见一座碧瓦楼台真个壮丽但见:飘飘万迭彩霞堆隐隐千条红雾现。耿耿檐飞怪兽头辉辉瓦迭鸳鸯片。门钻几路赤金钉槛设一横白玉段。窗牖近光放晓烟帘栊幌亮穿红电。楼台高耸接青霄廊庑平排连宝院。兽鼎香云袭御衣绛纱灯火明宫扇。左边猛烈摆牛头右下峥嵘罗马面。接亡送鬼转金牌引魄招魂垂素练。唤作阴司总会门下方阎老森罗殿。太宗正在外面观看只见那壁厢环珮叮噹仙香奇异外有两对提烛后面却是十代阎王降阶而至。是那十代阎君:秦广王、楚江王、宋帝王、仵官王、阎罗王、平等王、泰山王、都市王、卞城王、转轮王。

十王出在森罗宝殿控背躬身迎迓太宗。太宗谦下不敢前行十王道:“陛下是阳间人王我等是阴间鬼王分所当然何须过让?”太宗道:“朕得罪麾下岂敢论阴阳人鬼之道?”逊之不已。太宗前行径入森罗殿上与十王礼毕分宾主坐定。

约有片时秦广王拱手而进言曰:“泾河鬼龙告陛下许救而反杀之何也?”太宗道:“朕曾夜梦老龙求救实是允他无事不期他犯罪当刑该我那人曹官魏征处斩。朕宣魏征在殿着棋不知他一梦而斩。这是那人曹官出没神机又是那龙王犯罪当死岂是朕之过也?”十王闻言伏礼道:“自那龙未生之前南斗星死簿上已注定该遭杀于人曹之手我等早已知之。但只是他在此折辩定要陛下来此三曹对案是我等将他送入轮藏转生去了。今又有劳陛下降临望乞恕我催促之罪。”言毕命掌生死簿判官:“急取簿子来看陛下阳寿天禄该有几何?”崔判官急转司房将天下万国国王天禄总簿先逐一检阅只见南赡部洲大唐太宗皇帝注定贞观一十三年。崔判官吃了一惊急取浓墨大笔将“一”字上添了两画却将簿子呈上。十王从头看时见太宗名下注定三十三年阎王惊问:“陛下登基多少年了?”太宗道:“朕即位今一十三年了。”阎王道:“陛下宽心勿虑还有二十年阳寿。此一来已是对案明白请返本还阳。”

太宗闻言躬身称谢。十阎王差崔判官、朱太尉二人送太宗还魂。太宗出森罗殿又起手问十王道:“朕宫中老少安否如何?”

十王道:“俱安但恐御妹寿似不永。”太宗又再拜启谢:“朕回阳世无物可酬谢惟答瓜果而已。”十王喜曰:“我处颇有东瓜西瓜只少南瓜。”太宗道:“朕回去即送来即送来。”从此遂相揖而别。

那太尉执一引魂幡在前引路崔判官随后保着太宗径出幽司。太宗举目而看不是旧路问判官曰:“此路差矣?”

判官道:“不差。阴司里是这般有去路无来路。如今送陛下自转轮藏出身一则请陛下游观地府一则教陛下转托生。”

太宗只得随他两个引路前来。径行数里忽见一座高山阴云垂地黑雾迷空。太宗道:“崔先生那厢是甚么山?”判官道:

“乃幽冥背阴山。”太宗悚惧道:“朕如何去得?”判官道:“陛下宽心有臣等引领。”太宗战战兢兢相随二人上得山岩抬头观看只见:形多凸凹势更崎岖。峻如蜀岭高似庐岩。非阳世之名山实阴司之险地。荆棘丛丛藏鬼怪石崖磷磷隐邪魔。

耳畔不闻兽鸟噪眼前惟见鬼妖行。阴风飒飒黑雾漫漫。阴风飒飒是神兵口内哨来烟;黑雾漫漫是鬼祟暗中喷出气。一望高低无景色相看左右尽猖亡。那里山也有峰也有岭也有洞也有涧也有;只是山不生草峰不插天岭不行客洞不纳云涧不流水。岸前皆魍魉岭下尽神魔。洞中收野鬼涧底隐邪魂。山前山后牛头马面乱喧呼;半掩半藏饿鬼穷魂时对泣。催命的判官急急忙忙传信票;追魂的太尉吆吆喝喝趱公文。急脚子旋风滚滚勾司人黑雾纷纷。太宗全靠着那判官保护过了阴山。前进又历了许多衙门一处处俱是悲声振耳恶怪惊心。太宗又道:“此是何处?”判官道:“此是阴山背后一十八层地狱。”太宗道:“是那十八层?”判官道:“你听我说:吊筋狱、幽枉狱、火坑狱寂寂寥寥烦烦恼恼尽皆是生前作下千般业死后通来受罪名。酆都狱、拔舌狱、剥皮狱哭哭啼啼凄凄惨惨只因不忠不孝伤天理佛口蛇心堕此门。磨捱狱、碓捣狱、车崩狱皮开肉绽抹嘴咨牙乃是瞒心昧己不公道巧语花言暗损人。寒冰狱、脱壳狱、抽肠狱垢面蓬头愁眉皱眼都是大斗小秤欺痴蠢致使灾屯累自身。油锅狱、黑暗狱、刀山狱战战兢兢悲悲切切皆因强暴欺良善藏头缩颈苦伶仃。

血池狱、阿鼻狱、秤杆狱脱皮露骨折臂断筋也只为谋财害命宰畜屠生堕落千年难解释沉沦永世下翻身。一个个紧缚牢栓绳缠索绑差些赤鬼、黑脸鬼长枪短剑;牛头鬼、马面鬼铁简铜锤。只打得皱眉苦面血淋淋叫地叫天无救应。正是人生却莫把心欺神鬼昭彰放过谁?善恶到头终有报只争来早与来迟。”太宗听说心中惊惨。

进前又走不多时见一伙鬼卒各执幢幡路旁跪下道:

“桥梁使者来接。”判官喝令起去上前引着太宗从金桥而过。

太宗又见那一边有一座银桥桥上行几个忠孝贤良之辈公平正大之人亦有幢幡接引;那壁厢又有一桥寒风滚滚血浪滔滔号泣之声不绝。太宗问道:“那座桥是何名色?”判官道:“陛下那叫做奈河桥。若到阳间切须传记那桥下都是些奔流浩浩之水险峻窄窄之路。俨如匹练搭长江却似火坑浮上界。阴气逼人寒透骨腥风扑鼻味钻心。波翻浪滚往来并没渡人船;

赤脚蓬头出入尽皆作业鬼。桥长数里阔只三皻高有百尺深却千重。上无扶手栏杆下有抢人恶怪。枷杻缠身打上奈河险路。你看那桥边神将甚凶顽河内孽魂真苦恼桠杈树上挂的是肯红黄紫色丝衣;壁斗崖前蹲的是毁骂公婆淫泼妇。

铜蛇铁狗任争餐永堕奈河无出路。诗曰:时闻鬼哭与神号血水浑波万丈高。无数牛头并马面狰狞把守奈河桥。”正说间那几个桥梁使者早已回去了。太宗心又惊惶点头暗叹默默悲伤相随着判官、太尉早过了奈河恶水血盆苦界。前又到枉死城只听哄哄人嚷分明说“李世民来了!李世民来了!”太宗听叫心惊胆战。见一伙拖腰折臂、有足无头的鬼魅上前拦住都叫道:还我命来!还我命来!”慌得那太宗藏藏躲躲只叫“崔先生救我!崔先生救我!”判官道:陛下那些人都是那六十四处烟尘七十二处草寇众王子、众头目的鬼魂;尽是枉死的冤业无收无管不得生又无钱钞盘缠都是孤寒饿鬼。陛下得些钱钞与他我才救得哩。”太宗道:“寡人空身到此却那里得有钱钞?”判官道:“陛下阳间有一人金银若干在我这阴司里寄放。陛下可出名立一约小判可作保且借他一库给散这些饿鬼方得过去。”太宗问曰:“此人是谁?”判官道:“他是河南开封府人氏姓相名良他有十三库金银在此。陛下若借用过他的到阳间还他便了。”太宗甚喜情愿出名借用。遂立了文书与判官借他金银一库着太尉尽行给散。判官复吩咐道:“这些金银汝等可均分用度放你大唐爷爷过去他的阳寿还早哩。我领了十王钧语送他还魂教他到阳间做一个水陆大会度汝等生再休生事。”众鬼闻言得了金银俱唯唯而退。判官令太尉摇动引魂幡领太宗出离了枉死城中奔上平阳大路飘飘荡荡而去。毕竟不知从那条路出身且听下回分解。

\chapter[还受生唐王遵善果\ 度孤魂萧瑀正空门]{还受生唐王遵善果\\度孤魂萧瑀正空门}

第十一回 还受生唐王遵善果 度孤魂萧瑀正空门

诗曰:百岁光阴似水流一生事业等浮沤。昨朝面上桃花色今日头边雪片浮。白蚁阵残方是幻子规声切想回头。古来阴能延寿善不求怜天自周。却说唐太宗随着崔判官、朱太尉自脱了冤家债主前进多时却来到“六道轮回”之所又见那腾云的身披霞帔受箓的腰挂金鱼僧尼道俗走兽飞禽魑魅魍魉滔滔都奔走那轮回之下各进其道。唐王问曰:“此意何如?”判官道:“陛下明心见性是必记了传与阳间人知。

这唤做六道轮回:行善的升化仙道尽忠的生贵道行孝的再生福道公平的还生人道积德的转生富道恶毒的沉沦鬼道。”唐王听说点头叹曰:“善哉真善哉!作善果无灾!善心常切切善道大开开。莫教兴恶念是必少刁乖。休言不报应神鬼有安排。”判官送唐王直至那生贵道门拜呼唐王道:

“陛下呵此间乃出头之处小判告回着朱太尉再送一程。”唐王谢道:“有劳先生远涉。”判官道:“陛下到阳间千万做个水陆大会度那无主的冤魂切勿忘了。若是阴司里无报怨之声阳世间方得享太平之庆。凡百不善之处俱可一一改过普谕世人为善管教你后代绵长江山永固。”唐王一一准奏辞了崔判官随着朱太尉同入门来。那太尉见门里有一匹海骝马鞍韂齐备急请唐王上马太尉左右扶持。马行如箭早到了渭水河边只见那水面上有一对金色鲤鱼在河里翻波跳斗。

唐王见了心喜兜马贪看不舍太尉道:“陛下趱动些趁早赶时辰进城去也。”那唐王只管贪看不肯前行被太尉撮着脚高呼道:“还不走等甚!”扑的一声望那渭河推下马去却就脱了阴司径回阳世。

却说那唐朝驾下有徐茂功、秦叔宝、胡敬德、段志贤、马三宝、程咬金、高士廉、虞世南、房玄龄、杜如晦、萧瑀、傅奕、张道源、张士衡、王珪等两班文武俱保着那东宫太子与皇后、嫔妃、宫娥、侍长都在那白虎殿上举哀一壁厢议传哀诏要晓谕天下欲扶太子登基。时有魏征在旁道:“列位且住不可!不可!假若惊动州县恐生不测。且再按候一日我主必还魂也。”

下边闪上许敬宗道:“魏丞相言之甚谬。自古云泼水难收人逝不返你怎么还说这等虚言惑乱人心是何道理!”魏征道:

“不瞒许先生说下官自幼得授仙术推算最明管取陛下不死。”正讲处只听得棺中连声大叫道:“渰杀我耶!渰杀我耶”唬得个文官武将心慌皇后嫔妃胆战。一个个面如秋后黄桑叶腰似春前嫩柳条。储君脚软难扶丧杖尽哀仪;侍长魂飞怎戴梁冠遵孝礼?嫔妃打跌彩女欹斜。嫔妃打跌却如狂风吹倒败芙蓉;彩女欹斜好似骤雨冲歪娇菡萏。众臣悚惧骨软筋麻。战战兢兢痴痴痖痖。把一座白虎殿却象断梁桥闹丧台就如倒塌寺。此时众宫人走得精光那个敢近灵扶柩。多亏了正直的徐茂功理烈的魏丞相有胆量的秦琼忒猛撞的敬德上前来扶着棺材叫道:“陛下有甚么放不下心处说与我等不要弄鬼惊骇了眷族。”魏征道:“不是弄鬼此乃陛下还魂也。快取器械来!”打开棺盖果见太宗坐在里面还叫“渰死我了!是谁救捞?”茂功等上前扶起道:“陛下苏醒莫怕臣等都在此护驾哩。”唐王方才开眼道:“朕适才好苦躲过阴司恶鬼难又遭水面丧身灾。”众臣道:“陛下宽心勿惧有甚水灾来?”

唐王道:“朕骑着马正行至渭水河边见双头鱼戏被朱太尉欺心将朕推下马来跌落河中几乎渰死。”魏征道:“陛下鬼气尚未解。”急着太医院进安神定魄汤药又安排粥膳。连服一二次方才反本还原知得人事。一计唐王死去已三昼夜复回阳间为君。诗曰:万古江山几变更历来数代败和成。周秦汉晋多奇事谁似唐王死复生?当日天色已晚众臣请王归寝各各散讫。次早脱却孝衣换了彩服一个个红袍乌帽一个个紫绶金章在那朝门外等候宣召。

却说太宗自服了安神定魄之剂连进了数次粥汤被众臣扶入寝室一夜稳睡保养精神直至天明方起抖擞威仪你看他怎生打扮;戴一顶冲天冠穿一领赭黄袍。系一条蓝田碧玉带踏一对创业无忧履。貌堂堂赛过当朝;威烈烈重兴今日。好一个清平有道的大唐王起死回生的李陛下!唐王上金銮宝殿聚集两班文武山呼已毕依品分班。只听得传旨道:

“有事出班来奏无事退朝。”那东厢闪过徐茂功、魏征、王珪、杜如晦、房玄龄、袁天罡、李淳风、许敬宗等西厢闪过殷开山、刘洪基、马三宝、段志贤、程咬金、秦叔宝、胡敬德、薛仁贵等一齐上前在白玉阶前俯伏启奏道:“陛下前朝一梦如何许久方觉?”太宗道:“日前接得魏征书朕觉神魂出殿只见羽林军请朕出猎。正行时人马无踪又见那先君父王与先兄弟争嚷。

正难解处见一人乌帽皂袍乃是判官崔珪喝退先兄弟朕将魏征书传递与他。正看时又见青衣者执幢幡引朕入内到森罗殿上与十代阎王叙坐。他说那泾河龙诬告我许救转杀之事是朕将前言陈具一遍。他说已三曹对过案了急命取生死文簿检看我的阳寿。时有崔判官传上簿子阎王看了道寡人有三十三年天禄才过得一十三年还该我二十年阳寿即着朱太尉、崔判官、送朕回来。朕与十王作别允了送他瓜果谢恩。自出了森罗殿见那阴司里不忠不孝、非礼非义、作践五谷、明欺暗骗、大斗小秤、奸盗诈伪、淫邪欺罔之徒受那些磨烧舂锉之苦煎熬吊剥之刑有千千万万看之不足。又过着枉死城中有无数的冤魂。尽都是六十四处烟尘的叛贼七十二处草寇的魂灵挡住了朕之来路。幸亏崔判官作保借得河南相老儿的金银一库买转鬼魂方得前行。崔判官教朕回阳世千万作一场水陆大会度那无主的孤魂将此言叮咛分别。

出了那六道轮回之下有朱太尉请朕上马飞也相似行到渭水河边我看见那水面上有双头鱼戏。正欢喜处他将我撮着脚推下水中朕方得还魂也。”众臣闻此言无不称贺遂此编行传报天下各府县官员上表称庆不题。

却说太宗又传旨赦天下罪人又查狱中重犯。时有审官将刑部绞斩罪人查有四百余名呈上。太宗放赦回家拜辞父母兄弟托产与亲戚子侄明年今日赴曹仍领应得之罪。众犯谢恩而退。又出恤孤榜文又查宫中老幼彩女共有三千人出旨配军。自此内外俱善有诗为证诗曰:大国唐王恩德洪道过尧舜万民丰。死囚四百皆离狱怨女三千放出宫。天下多官称上寿朝中众宰贺元龙。善心一念天应佑福荫应传十七宗。太宗既放宫女、出死囚已毕又出御制榜文遍传天下。榜曰:“乾坤浩大日月照鉴分明;宇宙宽洪天地不容奸党。使心用术果报只在今生;善布浅求获福休言后世。千般巧计不如本分为人;万种强徒怎似随缘节俭。心行慈善何须努力看经?意欲损人空读如来一藏!”

自此时盖天下无一人不行善者。一壁厢又出招贤榜招人进瓜果到阴司里去;一壁厢将宝藏库金银一库差鄂国公胡敬德上河南开封府访相良还债。榜张数日有一赴命进瓜果的贤者本是均州人姓刘名全家有万贯之资。只因妻李翠莲在门拔金钗斋僧刘全骂了他几句说他不遵妇道擅出闺门。李氏忍气不过自缢而死。撇下一双儿女年幼昼夜悲啼。

刘全又不忍见无奈遂舍了性命弃了家缘撇了儿女情愿以死进瓜将皇榜揭了来见唐王。王传旨意教他去金亭馆里头顶一对南瓜袖带黄钱口噙药物。

那刘全果服毒而死一点魂灵顶着瓜果早到鬼门关上。

把门的鬼使喝道:“你是甚人敢来此处?”刘全道:“我奉大唐太宗皇帝钦差特进瓜果与十代阎王受用的。”那鬼使欣然接引。刘全径至森罗宝殿见了阎王将瓜果进上道:“奉唐王旨意远进瓜果以谢十王宽宥之恩。”阎王大喜道:“好一个有信有德的太宗皇帝!”遂此收了瓜果。便问那进瓜的人姓名那方人氏刘全道:“小人是均州城民籍姓刘名全。因妻李氏缢死撇下儿女无人看管小人情愿舍家弃子捐躯报国特与我王进贡瓜果谢众大王厚恩。”十王闻言即命查勘刘全妻李氏。

那鬼使取来在森罗殿下与刘全夫妻相会。诉罢前言回谢十王恩宥那阎王却检生死簿子看时他夫妻们都有登仙之寿急差鬼使送回。鬼使启上道:“李翠莲归阴日久尸无存魂将何附?”阎王道:“唐御妹李玉英今该促死;你可借他尸教他还魂去也。”那鬼使领命即将刘全夫妻二人还魂。带定出了阴司那阴风绕绕径到了长安大国将刘全的魂灵推入金亭馆里;将翠莲的灵魂带进皇宫内院只见那玉英宫主正在花阴下徐步绿苔而行被鬼使扑个满怀推倒在地活捉了他魂却将翠莲的魂灵推入玉英身内。鬼使回转阴司不题。

却说宫院中的大小侍婢见玉英跌死急走金銮殿报与三宫皇后道:“宫主娘娘跌死也!”皇后大惊随报太宗太宗闻言点头叹曰:“此事信有之也。朕曾问十代阎君:‘老幼安乎?’他道:‘俱安但恐御妹寿促。’果中其言。”合宫人都来悲切尽到花阴下看时只见那宫主微微有气。唐王道:“莫哭!莫哭!

休惊了他。”遂上前将御手扶起头来叫道:“御妹苏醒苏醒。”

那宫主忽的翻身叫:“丈夫慢行等我一等!”太宗道:“御妹是我等在此。”宫主抬头睁眼观看道:“你是谁人敢来扯我?”

太宗道:“是你皇兄、皇嫂。”宫主道:“我那里得个甚么皇兄、皇嫂!我娘家姓李我的乳名唤做李翠莲我丈夫姓刘名全两口儿都是均州人氏。因为我三个月前拔金钗在门斋僧我丈夫怪我擅出内门不遵妇道骂了我几句是我气塞胸堂将白绫带悬梁缢死撇下一双儿女昼夜悲啼。今因我丈夫被唐王钦差赴阴司进瓜果阎王怜悯放我夫妻回来。他在前走因我来迟赶不上他我绊了一跌。你等无礼!不知姓名怎敢扯我!”太宗闻言与众宫人道:“想是御妹跌昏了胡说哩。”传旨教太医院进汤药将玉英扶入宫中。

唐王当殿忽有当驾官奏道:“万岁今有进瓜果人刘全还魂在朝门外等旨。”唐王大惊急传旨将刘全召进俯伏丹墀。

太宗问道:“进瓜果之事何如?”刘全道:“臣顶瓜果径至鬼门关引上森罗殿见了那十代阎君将瓜果奉上备言我王殷勤致谢之意。阎君甚喜多多拜上我王道:‘真是个有信有德的太宗皇帝’!”唐王道:“你在阴司见些甚么来?”刘全道:“臣不曾远行没见甚的只闻得阎王问臣乡贯、姓名。臣将弃家舍子、因妻缢死、愿来进瓜之事说了一遍他急差鬼使引过我妻就在森罗殿下相会。一壁厢又检看死生文簿说我夫妻都有登仙之寿便差鬼使送回。臣在前走我妻后行幸得还魂。但不知妻投何所。”唐王惊问道:“那阎王可曾说你妻甚么?”刘全道:“阎王不曾说甚么只听得鬼使说‘李翠莲归阴日久尸无存。’阎王道:‘唐御妹李玉英今该促死教翠莲即借玉英尸还魂去罢。’臣不知唐御妹是甚地方家居何处我还未曾得去找寻哩。”唐王闻奏满心欢喜当对多官道:“朕别阎君曾问宫中之事他言老幼俱安但恐御妹寿促。却才御妹玉英花阴下跌死朕急扶看须臾苏醒口叫‘丈夫慢行等我一等!’朕只道是他跌昏了胡言。又问他详细他说的话与刘全一般。”

魏征奏道:“御妹偶尔寿促少苏醒即说此言此是刘全妻借尸还魂之事。此事也有可请宫主出来看他有甚话说。”唐王道:

“朕才命太医院去进药不知何如。”便教妃嫔入宫去请。那宫主在里面乱嚷道:“我吃甚么药?这里那是我家!我家是清凉瓦屋不象这个害黄病的房子花狸狐哨的门扇!放我出去!放我出去!”正嚷处只见四五个女官两三个太监扶着他直至殿上。唐王道:“你可认得你丈夫么?”玉英道:“说那里话我两个从小儿的结夫妻与他生男育女怎的不认得?”唐王叫内官搀他下去。那宫主下了宝殿直至白玉阶前见了刘全一把扯住道:“丈夫你往那里去就不等我一等!我跌了一跤被那些没道理的人围住我嚷这是怎的说!”那刘全听他说的话是妻之言观其人非妻之面不敢相认。唐王道:“这正是山崩地裂有人见捉生替死却难逢!”好一个有道的君王即将御妹的妆奁、衣物、饰尽赏赐了刘全就如陪嫁一般又赐与他永免差徭的御旨着他带领御妹回去。他夫妻两个便在阶前谢了恩欢欢喜喜还乡。有诗为证:人生人死是前缘短短长长各有年。刘全进瓜回阳世借尸还魂李翠莲。他两个辞了君王径来均州城里见旧家业儿女俱好两口儿宣扬善果不题。

却说那尉迟公将金银一库上河南开封府访看相良原来卖水为活同妻张氏在门贩卖乌盆瓦器营生但赚得些钱儿只以盘缠为足其多少斋僧布施买金银纸锭记库焚烧故有此善果臻身。阳世间是一条好善的穷汉那世里却是个积玉堆金的长者。尉迟公将金银送上他门唬得那相公、相婆魂飞魄散;又兼有本府官员茅舍外车马骈集那老两口子如痴如哑跪在地下只是磕头礼拜。尉迟公道:“老人家请起。我虽是个钦差官却赍着我王的金银送来还你。”他战兢兢的答道:“小的没有甚么金银放债如何敢受这不明之财?”尉迟公道:“我也访得你是个穷汉只是你斋僧布施尽其所用就买办金银纸锭烧记阴司阴司里有你积下的钱钞。是我太宗皇帝死去三日还魂复生曾在那阴司里借了你一库金银今此照数送还与你。你可一一收下等我好去回旨。”那相良两口儿只是朝天礼拜那里敢受道:“小的若受了这些金银就死得快了。虽然是烧纸记库此乃冥冥之事;况万岁爷爷那世里借了金银有何凭据?我决不敢受。”尉迟公道:“陛下说借你的东西有崔判官作保可证你收下罢。”相良道:“就死也是不敢受的。”尉迟公见他苦苦推辞只得具本差人启奏。太宗见了本知相良不受金银道:“此诚为善良长者!”即传旨教胡敬德将金银与他修理寺院起盖生祠请僧作善就当还他一般。旨意到日敬德望阙谢恩宣旨众皆知之。遂将金银买到城里军民无碍的地基一段周围有五十亩宽阔在上兴工起盖寺院名“敕建相国寺”。左有相公相婆的生祠镌碑刻石上写着“尉迟公监造”即今大相国寺是也。

工完回奏太宗甚喜。却又聚集多官出榜招僧修建水陆大会度冥府孤魂。榜行天下着各处官员推选有道的高僧上长安做会。那消个月之期天下多僧俱到。唐王传旨着太史丞傅奕选举高僧修建佛事。傅奕闻旨即上疏止浮图以言无佛。表曰:“西域之法无君臣父子以三途六道蒙诱愚蠢追既往之罪窥将来之福口诵梵言以图偷免。且生死寿夭本诸自然;刑德威福系之人主。今闻俗徒矫托皆云由佛。自五帝三王未有佛法君明臣忠年祚长久。至汉明帝始立胡神然惟西域桑门自传其教实乃夷犯中国不足为信。”太宗闻言遂将此表掷付群臣议之。时有宰相萧瑀出班俯囟奏曰:

“佛法兴自屡朝弘善遏恶冥助国家理无废弃。佛圣人也。

非圣者无法请置严刑。”傅奕与萧瑀论辨言礼本于事亲事君而佛背亲出家以匹夫抗天子以继体悖所亲萧瑀不生于空桑乃遵无父之教正所谓非孝者无亲。萧瑀但合掌曰:“地狱之设正为是人。”太宗召太仆卿张道源、中书令张士衡问佛事营福其应何如。二臣对曰:“佛在清净仁恕果正佛空。周武帝以三教分次:大慧禅师有赞幽远历众供养而无不显;五祖投胎达摩现象。自古以来皆云三教至尊而不可毁不可废。伏乞陛下圣鉴明裁。”太宗甚喜道:“卿之言合理。再有所陈者罪之。”遂着魏征与萧瑀、张道源邀请诸佛选举一名有大德行者作坛主设建道场众皆顿谢恩而退。自此时出了法律:但有毁僧谤佛者断其臂。

次日三位朝臣聚众僧在那山川坛里逐一从头查选内中选得一名有德行的高僧。你道他是谁人?灵通本讳号金蝉只为无心听佛讲转托尘凡苦受磨降生世俗遭罗网。投胎落地就逢凶未出之前临恶党。父是海州陈状元外公总管当朝长。出身命犯落江星顺水随波泱。海岛金山有大缘迁安和尚将他养。年方十八认亲娘特赴京都求外长。总管开山调大军洪州剿寇诛凶党。状元光蕊脱天罗子父相逢堪贺奖。复谒当今受主恩凌烟阁上贤名响。恩官不受愿为僧洪福沙门将道访。小字江流古佛儿法名唤做陈玄奘。当日对众举出玄奘法师。这个人自幼为僧出娘胎就持斋受戒。他外公见是当朝一路总管殷开山他父亲陈光蕊中状元官拜文渊殿大学士。一心不爱荣华只喜修持寂灭。查得他根源又好德行又高。千经万典无所不通:佛号仙音无般不会。当时三位引至御前扬尘舞蹈拜罢奏曰:“臣瑀等蒙圣旨选得高僧一名陈玄奘。”太宗闻其名沉思良久道:“可是学士陈光蕊之儿玄奘否?”江流儿叩头曰:“臣正是。”太宗喜道:“果然举之不错诚为有德行有禅心的和尚。朕赐你左僧纲、右僧纲、天下大阐都僧纲之职。”玄奘顿谢恩受了大阐官爵。又赐五彩织金袈裟一件毗卢帽一顶。教他用心再拜明僧排次阇黎班书办旨意前赴化生寺择定吉日良时开演经法。玄奘再拜领旨而出遂到化生寺里聚集多僧打造禅榻装修功德整理音乐。选得大小明僧共计一千二百名分派上中下三堂。诸所佛前物件皆齐头头有次。选到本年九月初三日黄道良辰开启做七七四十九日水陆大会。即具表申奏太宗及文武国戚皇亲俱至期赴会拈香听讲。毕竟不知圣意如何且听下回分解。

\chapter[玄奘秉诚建大会\ 观音显相化金蝉]{玄奘秉诚建大会\\观音显相化金蝉}

第十二回 玄奘秉诚建大会 观音显象化金蝉

诗曰:龙集贞观正十三王宣大众把经谈。道场开演无量法云雾光乘大愿龛。御敕垂恩修上刹金蝉脱壳化西涵。普施善果沉没秉教宣扬前后三。贞观十三年岁次己巳九月甲戌初三日癸卯良辰。陈玄奘大阐法师聚集一千二百名高僧都在长安城化生寺开演诸品妙经。那皇帝早朝已毕帅文武多官乘凤辇龙车出离金銮宝殿径上寺来拈香。怎见那銮驾?真个是:一天瑞气万道祥光。仁风轻淡荡化日丽非常。

千官环佩分前后五卫旌旗列两旁。执金瓜擎斧钺双双对对;绛纱烛御炉香霭霭堂堂。龙飞凤舞鹗荐鹰扬。圣明天子正忠义大臣良。介福千年过舜禹升平万代赛尧汤。又见那曲柄伞滚龙袍辉光相射;玉连环彩凤扇瑞霭飘扬。珠冠玉带紫绶金章。护驾军千队扶舆将两行。这皇帝沐浴虔诚尊敬佛皈依善果喜拈香。唐王大驾早到寺前吩咐住了音乐响器下了车辇引着多官。拜佛拈香。三匝已毕抬头观看果然好座道场但见:幢幡飘舞宝盖飞辉。幢幡飘舞凝空道道彩霞摇;宝盖飞辉映日翩翩红电彻。世尊金象貌臻臻罗汉玉容威烈烈。瓶插仙花炉焚檀降。瓶插仙花锦树辉辉漫宝刹;炉焚檀降香云霭霭透清霄。时新果品砌朱盘奇样糖酥堆彩案。高僧罗列诵真经愿拔孤魂离苦难。太宗文武俱各拈香拜了佛祖金身参了罗汉。又见那大阐都纲陈玄奘法师引众僧罗拜唐王。礼毕分班各安禅位法师献上济孤榜文与太宗看榜曰:“至德渺茫禅宗寂灭。清净灵通周流三界。千变万化统摄阴阳。体用真常无穷极矣。观彼孤魂深宜哀愍。此奉太宗圣命:选集诸僧参禅讲法。大开方便门庭广运慈悲舟楫普济苦海群生脱免沉疴六趣。引归真路普玩鸿蒙;动止无为混成纯素。仗此良因邀赏清都绛阙;乘吾胜会脱离地狱凡笼。早登极乐任逍遥来往西方随自在。诗曰:一炉永寿香几卷生箓。无边妙法宣无际天恩沐。冤孽尽消除孤魂皆出狱。愿保我邦家清平万年福。”太宗看了满心欢喜对众僧道:“汝等秉立丹衷切休怠慢佛事。待后功成完备各各福有所归朕当重赏决不空劳。”那一千二百僧一齐顿称谢。

当日三斋已毕唐王驾回。待七日正会复请拈香。时天色将晚各官俱退。怎见得好晚?你看那:万里长空淡落辉归鸦数点下栖迟。满城灯火人烟静正是禅僧入定时。一宿晚景题过。

次早法师又升坐聚众诵经不题。

却说南海普陀山观世音菩萨自领了如来佛旨在长安城访察取经的善人日久未逢真实有德行者。忽闻得太宗宣扬善果选举高僧开建大会又见得法师坛主乃是江流儿和尚正是极乐中降来的佛子又是他原引送投胎的长老菩萨十分欢喜就将佛赐的宝贝捧上长街与木叉货卖。你道他是何宝贝?有一件锦襕异宝袈裟、九环锡杖还有那金紧禁三个箍儿密密藏收以俟后用只将袈裟、锡杖出卖。长安城里有那选不中的愚僧倒有几贯村钞。见菩萨变化个疥癞形容身穿破衲赤脚光头将袈裟捧定艳艳生光他上前问道:“那癞和尚你的袈裟要卖多少价钱?”菩萨道:“袈裟价值五千两锡杖价值二千两。”那愚僧笑道:“这两个癞和尚是疯子!是傻子!这两件粗物就卖得七千两银子?只是除非穿上身长生不老就得成佛作祖也值不得这许多!拿了去!卖不成!”那菩萨更不争吵与木叉往前又走。行勾多时来到东华门前正撞着宰相萧瑀散朝而回众头踏喝开街道。那菩萨公然不避当街上拿着袈裟径迎着宰相。宰相勒马观看见袈裟艳艳生光着手下人问那卖袈裟的要价几何。菩萨道:“袈裟要五千两锡杖要二千两。”萧瑀道:“有何好处值这般高价?”菩萨道:“袈裟有好处有不好处;有要钱处有不要钱处。”萧瑀道:“何为好?何为不好?”菩萨道:“着了我袈裟不入沉沦不堕地狱不遭恶毒之难不遇虎狼之穴便是好处;若贪淫乐祸的愚僧不斋不戒的和尚毁经谤佛的凡夫难见我袈裟之面这便是不好处。”

又问道:“何为要钱不要钱?”菩萨道:“不遵佛法不敬三宝强买袈裟、锡杖定要卖他七千两这便是要钱;若敬重三宝见善随喜皈依我佛承受得起我将袈裟、锡杖情愿送他与我结个善缘这便是不要钱。”萧瑀闻言倍添春色知他是个好人即便下马与菩萨以礼相见口称:“大法长老恕我萧瑀之罪。我大唐皇帝十分好善满朝的文武无不奉行。即今起建水陆大会这袈裟正好与大都阐陈玄奘法师穿用。我和你入朝见驾去来。”

菩萨欣然从之拽转步径进东华门里。黄门官转奏蒙旨宣至宝殿。见萧瑀引着两个疥癞僧人立于阶下唐王问曰:

“萧瑀来奏何事?”萧瑀俯伏阶前道:“臣出了东华门前偶遇二僧乃卖袈裟与锡杖者。臣思法师玄奘可着此服故领僧人启见。”太宗大喜便问那袈裟价值几何。菩萨与木叉侍立阶下更不行礼因问袈裟之价答道:“袈裟五千两锡杖二千两。”

太宗道:“那袈裟有何好处就值许多?”菩萨道:“这袈裟龙披一缕免大鹏蚕噬之灾;鹤挂一丝得凡入圣之妙。但坐处有万神朝礼;凡举动有七佛随身。这袈裟是冰蚕造练抽丝巧匠翻腾为线。仙娥织就神女机成。方方簇幅绣花缝片片相帮堆锦簆。玲珑散碎斗妆花色亮飘光喷宝艳。穿上满身红雾绕脱来一段彩云飞。三天门外透玄光五岳山前生宝气。重重嵌就西番莲灼灼悬珠星斗象。四角上有夜明珠攒顶间一颗祖母绿。虽无全照原本体也有生光八宝攒。这袈裟闲时折迭遇圣才穿。闲时折迭千层包裹透虹霓;遇圣才穿惊动诸天神鬼怕。上边有如意珠、摩尼珠、辟尘珠、定风珠;又有那红玛瑙、紫珊瑚、夜明珠、舍利子。偷月沁白与日争红。条条仙气盈空朵朵祥光捧圣。条条仙气盈空照彻了天关;朵朵祥光捧圣影遍了世界。照山川惊虎豹;影海岛动鱼龙。沿边两道销金锁叩领连环白玉琮。诗曰:三宝巍巍道可尊四生六道尽评论。明心解养人天法见性能传智慧灯。护体庄严金世界身心清净玉壶冰。自从佛制袈裟后万劫谁能敢断僧?”

唐王在那宝殿上闻言十分欢喜又问:“那和尚九环杖有甚好处?”菩萨道:“我这锡杖是那铜镶铁造九连环九节仙藤永驻颜。入手厌看青骨瘦下山轻带白云还。摩呵五祖游天阙罗卜寻娘破地关。不染红尘些子秽喜伴神僧上玉山。”唐王闻言即命展开袈裟从头细看果然是件好物道:“大法长老实不瞒你朕今大开善教广种福田见在那化生寺聚集多僧敷演经法。内中有一个大有德行者法名玄奘。朕买你这两件宝物赐他受用。你端的要价几何?”菩萨闻言与木叉合掌皈依道声佛号躬身上启道:“既有德行贫僧情愿送他决不要钱。”说罢抽身便走。唐王急着萧瑀扯住欠身立于殿上问曰:“你原说袈裟五千两锡杖二千两你见朕要买就不要钱敢是说朕心倚恃君位强要你的物件?更无此理。朕照你原价奉偿却不可推避。”菩萨起手道:“贫僧有愿在前原说果有敬重三宝见善随喜皈依我佛不要钱愿送与他。今见陛下明德止善敬我佛门况又高僧有德有行宣扬大法理当奉上决不要钱。贫僧愿留下此物告回。”唐王见他这等勤恳甚喜随命光禄寺大排素宴酬谢。菩萨又坚辞不受畅然而去依旧望都土地庙中隐避不题。

却说太宗设午朝着魏征赍旨宣玄奘入朝。那法师正聚众登坛讽经诵偈一闻有旨随下坛整衣与魏征同往见驾。

太宗道:“求证善事有劳法师无物酬谢。早间萧瑀迎着二僧愿送锦襕异宝袈裟一件九环锡杖一条。今特召法师领去受用。”玄奘叩头谢恩。太宗道:“法师如不弃可穿上与朕看看。”

长老遂将袈裟抖开披在身上手持锡杖侍立阶前。君臣个个欣然。诚为如来佛子你看他:凛凛威颜多雅秀佛衣可体如裁就。辉光艳艳满乾坤结彩纷纷凝宇宙。朗朗明珠上下排层层金线穿前后。兜罗四面锦沿边万样稀奇铺绮绣。八宝妆花缚钮丝金环束领攀绒扣。佛天大小列高低星象尊卑分左右。

玄奘法师大有缘现前此物堪承受。浑如极乐活罗汉赛过西方真觉秀。锡杖叮噹斗九环毗卢帽映多丰厚。诚为佛子不虚传胜似菩提无诈谬。当时文武阶前喝采太宗喜之不胜即着法师穿了袈裟持了宝杖又赐两队仪从着多官送出朝门教他上大街行道往寺里去就如中状元夸官的一般。这位玄奘再拜谢恩在那大街上烈烈轰轰摇摇摆摆。你看那长安城里行商坐贾、公子王孙、墨客文人、大男小女无不争看夸奖俱道:“好个法师!真是个活罗汉下降活菩萨临凡。”玄奘直至寺里僧人下榻来迎。一见他披此袈裟执此锡杖都道是地藏王来了各各归依侍于左右。玄奘上殿炷香礼佛又对众感述圣恩已毕各归禅座。又不觉红轮西坠正是那:日落烟迷草树帝都钟鼓初鸣。叮叮三响断人行前后御前寂静。上刹辉煌灯火孤村冷落无声。禅僧入定理残经正好炼魔养性。

光阴拈指却当七日正会玄奘又具表请唐王拈香。此时善声遍满天下。太宗即排驾率文武多官、后妃国戚早赴寺里。那一城人无论大小尊卑俱诣寺听讲。当有菩萨与木叉道:“今日是水陆正会以一七继七七可矣了。我和你杂在众人丛中一则看他那会何如二则看金蝉子可有福穿我的宝贝三则也听他讲的是那一门经法。”两人随投寺里。正是有缘得遇旧相识般若还归本道场。入到寺里观看真个是天朝大国果胜裟婆赛过祇园舍卫也不亚上刹招提。那一派仙音响亮佛号喧哗。这菩萨直至多宝台边果然是明智金蝉之相。诗曰:万象澄明绝点埃大典玄奘坐高台。生孤魂暗中到听法高流市上来。施物应机心路远出生随意藏门开。对看讲出无量法老幼人人放喜怀。又诗曰:因游法界讲堂中逢见相知不俗同。尽说目前千万事又谈尘劫许多功。法云容曳舒群岳教网张罗满太空。检点人生归善念纷纷天雨落花红。那法师在台上念一会《受生度亡经》谈一会《安邦天宝篆》又宣一会《劝修功卷》。这菩萨近前来拍着宝台厉声高叫道:“那和尚你只会谈小乘教法可会谈大乘么?”玄奘闻言心中大喜翻身跳下台来对菩萨起手道:“老师父弟子失瞻多罪。见前的盖众僧人都讲的是小乘教法却不知大乘教法如何。”菩萨道:“你这小乘教法度不得亡者升只可浑俗和光而已。我有大乘佛法三藏能亡者升天能度难人脱苦能修无量寿身能作无来无去。”

正讲处有那司香巡堂官急奏唐王道:“法师正讲谈妙法被两个疥癞游僧扯下来乱说胡话。”王令擒来只见许多人将二僧推拥进后法堂。见了太宗那僧人手也不起拜也不拜仰面道:“陛下问我何事?”唐王却认得他道:“你是前日送袈裟的和尚?”菩萨道:“正是。”太宗道:“你既来此处听讲只该吃些斋便了为何与我法师乱讲扰乱经堂误我佛事?”菩萨道:

“你那法师讲的是小乘教法度不得亡者升天。我有大乘佛法三藏可以度亡脱苦寿身无坏。”太宗正色喜问道:“你那大乘佛法在于何处?”菩萨道:“在大西天天竺国大雷音寺我佛如来处能解百冤之结能消无妄之灾。”太宗道:“你可记得么?”

菩萨道:“我记得。”太宗大喜道:“教法师引去请上台开讲。”

那菩萨带了木叉飞上高台遂踏祥云直至九霄现出救苦原身托了净瓶杨柳。左边是木叉惠岸执着棍抖擞精神。

喜的个唐王朝天礼拜众文武跪地焚香满寺中僧尼道俗士人工贾无一人不拜祷道:“好菩萨!好菩萨!”有词为证但见那:瑞霭散缤纷祥光护法身。九霄华汉里现出女真人。那菩萨头上戴一顶金叶纽翠花铺放金光生锐气的垂珠缨络;

身上穿一领淡淡色浅浅妆盘金龙飞彩凤的结素蓝袍;胸前挂一面对月明舞清风杂宝珠攒翠玉的砌香环珮;腰间系一条冰蚕丝织金边登彩云促瑶海的锦绣绒裙;面前又领一个飞东洋游普世感恩行孝黄毛红嘴白鹦哥;手内托着一个施恩济世的宝瓶瓶内插着一枝洒青霄撒大恶扫开残雾垂杨柳。玉环穿绣扣金莲足下深。三天许出入这才是救苦救难观世音。喜的个唐太宗忘了江山;爱的那文武官失却朝礼;

盖众多人都念“南无观世音菩萨”。太宗即传旨:教巧手丹青描下菩萨真象。旨意一声选出个图神写圣远见高明的吴道子此人即后图功臣于凌烟阁者。当时展开妙笔图写真形。那菩萨祥云渐远霎时间不见了金光。只见那半空中滴溜溜落下一张简帖上有几句颂子写得明白。颂曰:“礼上大唐君西方有妙文。程途十万八千里大乘进殷勤。此经回上国能鬼出群。若有肯去者求正果金身。”太宗见了颂子即命众僧:

“且收胜会待我差人取得大乘经来再秉丹诚重修善果。”众官无不遵依。当时在寺中问曰:“谁肯领朕旨意上西天拜佛求经?”问不了旁边闪过法师帝前施礼道:“贫僧不才愿效犬马之劳与陛下求取真经祈保我王江山永固。”唐王大喜上前将御手扶起道:“法师果能尽此忠贤不怕程途遥远跋涉山川朕情愿与你拜为兄弟。”玄奘顿谢恩。唐王果是十分贤德就去那寺里佛前与玄奘拜了四拜口称“御弟圣僧”。玄奘感谢不尽道:“陛下贫僧有何德何能敢蒙天恩眷顾如此?我这一去定要捐躯努力直至西天。如不到西天不得真经即死也不敢回国永堕沉沦地狱。”随在佛前拈香以此为誓。唐王甚喜即命回銮待选良利日辰牒出行遂此驾回各散。

玄奘亦回洪福寺里。那本寺多僧与几个徒弟早闻取经之事都来相见因问:“誓愿上西天实否?”玄奘道:“是实。”

他徒弟道:“师父呵尝闻人言西天路远更多虎豹妖魔。只怕有去无回难保身命。”玄奘道:“我已了弘誓大愿不取真经永堕沉沦地狱。大抵是受王恩宠不得不尽忠以报国耳。我此去真是渺渺茫茫吉凶难定。”又道:“徒弟们我去之后或三二年或五七年但看那山门里松枝头向东我即回来;不然断不回矣。”众徒将此言切切而记。

次早太宗设朝聚集文武写了取经文牒用了通行宝印。有钦天监奏曰:“今日是人专吉星堪宜出行远路。”唐王大喜。又见黄门官奏道:“御弟法师朝门外候旨。”随即宣上宝殿道:“御弟今日是出行吉日。这是通关文牒。朕又有一个紫金钵盂送你途中化斋而用。再选两个长行的从者又银駔的马一匹送为远行脚力。你可就此行程。”玄奘大喜即便谢了恩领了物事更无留滞之意。唐王排驾与多官同送至关外只见那洪福寺僧与诸徒将玄奘的冬夏衣服俱送在关外相等。唐王见了先教收拾行囊马匹然后着官人执壶酌酒。太宗举爵又问曰:“御弟雅号甚称?”玄奘道:“贫僧出家人未敢称号。”太宗道:“当时菩萨说西天有经三藏。御弟可指经取号号作三藏何如?”玄奘又谢恩接了御酒道:“陛下酒乃僧家头一戒贫僧自为人不会饮酒。”太宗道:“今日之行比他事不同。此乃素酒只饮此一杯以尽朕奉饯之意。”三藏不敢不受。接了酒方待要饮只见太宗低头将御指拾一撮尘土弹入酒中。

三藏不解其意太宗笑道:“御弟呵这一去到西天几时可回?”三藏道:“只在三年径回上国。”太宗道:“日久年深山遥路远御弟可进此酒:宁恋本乡一捻土莫爱他乡万两金。”三藏方悟捻土之意复谢恩饮尽辞谢出关而去。唐王驾回。毕竟不知此去何如且听下回分解。

\chapter[陷虎穴金星解厄\ 双叉岭伯钦留僧]{陷虎穴金星解厄\\双叉岭伯钦留僧}

第十三回 陷虎穴金星解厄 双叉岭伯钦留僧

诗曰:大有唐王降敕封钦差玄奘问禅宗。坚心磨琢寻龙穴着意修持上鹫峰。边界远游多少国云山前度万千重。自今别驾投西去秉教迦持悟大空。却说三藏自贞观十三年九月望前三日蒙唐王与多官送出长安关外。一二日马不停蹄早至法门寺。本寺住持上房长老带领众僧有五百余人两边罗列接至里面相见献茶。茶罢进斋斋后不觉天晚正是那:影动星河近月明无点尘。雁声鸣远汉砧韵响西邻。归鸟栖枯树禅僧讲梵音。蒲团一榻上坐到夜将分。众僧们灯下议论佛门定旨上西天取经的原由。有的说水远山高有的说路多虎豹有的说峻岭陡崖难度有的说毒魔恶怪难降。三藏钳口不言但以手指自心点头几度。众僧们莫解其意合掌请问道:“法师指心点头者何也?”三藏答曰:“心生种种魔生;心灭种种魔灭。我弟子曾在化生寺对佛设下洪誓大愿不由我不尽此心。这一去定要到西天见佛求经使我们法轮回转愿圣主皇图永固。”众僧闻得此言人人称羡个个宣扬都叫一声“忠心赤胆大阐法师”夸赞不尽请师入榻安寐。

早又是竹敲残月落鸡唱晓云生。那众僧起来收拾茶水早斋。玄奘遂穿了袈裟上正殿佛前礼拜道:“弟子陈玄奘前往西天取经但肉眼愚迷不识活佛真形。今愿立誓:路中逢庙烧香遇佛拜佛遇塔扫塔。但愿我佛慈悲早现丈六金身赐真经留传东土。”祝罢回方丈进斋。斋毕那二从者整顿了鞍马促趱行程。三藏出了山门辞别众僧。众僧不忍分别直送有十里之遥噙泪而返三藏遂直西前进。正是那季秋天气但见:数村木落芦花碎几树枫杨红叶坠。路途烟雨故人稀黄菊丽山骨细水寒荷破人憔悴。白蘋红蓼霜天雪落霞孤鹜长空坠。依稀黯淡野云飞玄鸟去宾鸿至嘹嘹呖呖声宵碎。

师徒们行了数日到了巩州城。早有巩州合属官吏人等迎接入城中。安歇一夜次早出城前去。一路饥餐渴饮夜住晓行两三日又至河州卫。此乃是大唐的山河边界。早有镇边的总兵与本处僧道闻得是钦差御弟法师上西方见佛无不恭敬接至里面供给了着僧纲请往福原寺安歇。本寺僧人一一参见安排晚斋。斋毕吩咐二从者饱喂马匹天不明就行。

及鸡方鸣随唤从者却又惊动寺僧整治茶汤斋供。斋罢出离边界。

这长老心忙太起早了。原来此时秋深时节鸡鸣得早只好有四更天气。一行三人连马四口迎着清霜看着明月行有数十里远近见一山岭只得拨草寻路说不尽崎岖难走又恐怕错了路径。正疑思之间忽然失足三人连马都跌落坑坎之中。三藏心慌从者胆战。却才悚惧又闻得里面哮吼高呼叫:“拿将来!拿将来!”只见狂风滚滚拥出五六十个妖邪将三藏、从者揪了上去。这法师战战兢兢的偷眼观看上面坐的那魔王十分凶恶真个是:雄威身凛凛猛气貌堂堂。电目飞光艳雷声振四方。锯牙舒口外凿齿露腮旁。锦绣围身体文斑裹脊梁。钢须稀见肉钩爪利如霜。东海黄公惧南山白额王。唬得个三藏魂飞魄散二从者骨软筋麻。魔王喝令绑了众妖一齐将三人用绳索绑缚。正要安排吞食只听得外面喧哗有人来报:“熊山君与特处士二位来也。”三藏闻言抬头观看前走的是一条黑汉你道他是怎生模样:雄豪多胆量轻健夯身躯。涉水惟凶力跑林逞怒威。向来符吉梦今独露英姿。

绿树能攀折知寒善谕时。准灵惟显处故此号山君。又见那后边来的是一条胖汉你道怎生模样:嵯峨双角冠端肃耸肩背。性服青衣稳蹄步多迟滞。宗名父作牯原号母称牸。能为田者功因名特处士。

这两个摇摇摆摆走入里面慌得那魔王奔出迎接。熊山君道:“寅将军一向得意可贺!可贺!”特处士道:“寅将军丰姿胜常真可喜!真可喜!”魔王道:“二公连日如何?”山君道:“惟守素耳。”处士道:“惟随时耳。”三个叙罢各坐谈笑。

只见那从者绑得痛切悲啼那黑汉道:“此三者何来?”魔王道:“自送上门来者。”处士笑云:“可能待客否?”魔王道:“奉承!奉承!”山君道:“不可尽用食其二留其一可也。”魔王领诺即呼左左将二从者剖腹剜心剁碎其尸将级与心肝奉献二客将四肢自食其余骨肉分给各妖。只听得啯啅之声真似虎啖羊羔霎时食尽。把一个长老几乎唬死。这才是初出长安第一场苦难。

正怆慌之间渐惭的东方白那二怪至天晓方散俱道:

“今日厚扰容日竭诚奉酬。”方一拥而退。不一时红日高升。

三藏昏昏沉沉也辨不得东西南北正在那不得命处忽然见一老叟手持拄杖而来。走上前用手一拂绳索皆断对面吹了一口气三藏方苏跪拜于地道:“多谢老公公!搭救贫僧性命!”老叟答礼道:“你起来。你可曾疏失了甚么东西?”三藏道:

“贫僧的从人已是被怪食了只不知行李马匹在于何处?”老叟用杖指定道:“那厢不是一匹马、两个包袱?”三藏回头看时果是他的物件并不曾失落心才略放下些问老叟曰:“老公公此处是甚所在?公公何由在此?”老叟道:“此是双叉岭乃虎狼巢穴处。你为何堕此?”三藏道:“贫僧鸡鸣时出河州卫界不料起得早了冒霜拨露忽失落此地。见一魔王凶顽太甚将贫僧与二从者绑了。又见一条黑汉称是熊山君;一条胖汉称是特处士走进来称那魔王是寅将军。他三个把我二从者吃了天光才散。不想我是那里有这大缘大分感得老公公来此救我?”老叟道:“处士者是个野牛精山君者是个熊罴精寅将军者是个老虎精。左右妖邪尽都是山精树鬼怪兽苍狼。

只因你的本性元明所以吃不得你。你跟我来引你上路。”三藏不胜感激将包袱捎在马上牵著缰绳相随老叟径出了坑坎之中走上大路。却将马拴在道旁草头上转身拜谢那公公那公公遂化作一阵清风跨一只朱顶白鹤腾空而去。只见风飘飘遗下一张简帖书上四句颂子颂子云:“吾乃西天太白星特来搭救汝生灵。前行自有神徒助莫为艰难报怨经。”三藏看了对天礼拜道:“多谢金星度脱此难。”拜毕牵了马匹独自个孤孤凄凄往前苦进。这岭上真个是寒飒飒雨林风响潺潺涧下水。香馥馥野花开密丛丛乱石磊。闹嚷嚷鹿与猿一队队獐和麂。喧杂杂鸟声多静悄悄人事靡。那长老战兢兢心不宁;这马儿力怯怯蹄难举。三藏舍身拚命上了那峻岭之间。行经半日更不见个人烟村舍。一则腹中饥了二则路又不平正在危急之际只见前面有两只猛虎咆哮后边有几条长蛇盘绕。左有毒虫右有怪兽三藏孤身无策只得放下身心听天所命。又无奈那马腰软蹄弯即便跪下伏倒在地打又打不起牵又牵不动。苦得个法师衬身无地真个有万分凄楚已自分必死莫可奈何。却说他虽有灾迍却有救应。正在那不得命处忽然见毒虫奔走妖兽飞逃;猛虎潜踪长蛇隐迹。三藏抬头看时只见一人手执钢叉腰悬弓箭自那山坡前转出果然是一条好汉。你看他:头上戴一顶艾叶花斑豹皮帽身上穿一领羊绒织锦叵罗衣腰间束一条狮蛮带。脚下躧一对麂皮靴。环眼圆睛如吊客圈须乱扰似河奎。悬一囊毒药弓矢拿一杆点钢大叉。雷声震破山虫胆勇猛惊残野雉魂。三藏见他来得渐近跪在路旁合掌高叫道:“大王救命!大王救命!”那条汉到跟前放下钢叉用手搀起道:“长老休怕。我不是歹人我是这山中的猎户姓刘名伯钦绰号镇山太保。我才自来要寻两只山虫食用不期遇著你多有冲撞。”三藏道:

“贫僧是大唐驾下钦差往西天拜佛求经的和尚。适间来到此处遇著些狼虎蛇虫四边围绕不能前进。忽见太保来众兽皆走救了贫僧性命多谢!多谢!”伯钦道:“我在这里住人专倚打些狼虎为生捉些蛇虫过活故此众兽怕我走了。你既是唐朝来的与我都是乡里。此间还是大唐的地界我也是唐朝的百姓我和你同食皇王的水土诚然是一国之人。你休怕跟我来到我舍下歇马明朝我送你上路。”三藏闻言满心欢喜谢了伯钦牵马随行。

过了山坡又听得呼呼风响。伯钦道:“长老休走坐在此间。风响处是个山猫来了等我拿他家去管待你。”三藏见说又胆战心惊不敢举步。那太保执了钢叉拽开步迎将上去。

只见一只斑斓虎对面撞见他看见伯钦急回头就步。这太保霹雳一声咄道:“那业畜!那里走!”那虎见赶得急转身轮爪扑来。这太保三股叉举手迎敌唬得个三藏软瘫在草地。这和尚自出娘肚皮那曾见这样凶险的勾当?太保与那虎在那山坡下人虎相持果是一场好斗。但见:怒气纷纷狂风滚滚。怒气纷纷太保冲冠多膂力;狂风滚滚斑彪逞势喷红尘。那一个张牙舞爪这一个转步回身。三股叉擎天幌日千花尾扰雾飞云。这一个当胸乱刺那一个劈面来吞。闪过的再生人道撞着的定见阎君。只听得那斑彪哮吼太保声哏。斑彪哮吼振裂山川惊鸟兽;太保声哏喝开天府现星辰。那一个金睛怒出这一个壮胆生嗔。可爱镇山刘太保堪夸据地兽之君。人虎贪生争胜负些儿有慢丧三魂。他两个斗了有一个时辰只见那虎爪慢腰松被太保举叉平胸刺倒可怜呵钢叉尖穿透心肝霎时间血流满地。揪著耳朵拖上路来好男子!气不连喘面不改色对三藏道:“造化!造化!这只山猫彀长老食用几日。”

三藏夸赞不尽道:“太保真山神也!”伯钦道:“有何本事敢劳过奖?这个是长老的洪福。去来!赶早儿剥了皮煮些肉管待你也。”他一只手执着叉一只手拖着虎在前引路。三藏牵着马随后而行迤逶行过山坡忽见一座山庄。那门前真个是:参天古树漫路荒藤。万壑风尘冷千崖气象奇。一径野花香袭体数竿幽竹绿依依。草门楼篱笆院堪描堪画;石板桥白土壁真乐真稀。秋容萧索爽气孤高。道旁黄叶落岭上白云飘。疏林内山禽聒聒庄门外细犬嘹嘹。伯钦到了门将死虎掷下叫:“小的们何在?”只见走出三四个家僮都是怪形恶相之类上前拖拖拉拉把只虎扛将进去。伯钦吩咐教:“赶早剥了皮安排将来待客。”复回头迎接三藏进内。彼此相见三藏又拜谢伯钦厚恩怜悯救命伯钦道:“同乡之人何劳致谢。”坐定茶罢有一老妪领着一个媳妇对三藏进礼。伯钦道:“此是家母、山妻。”三藏道:“请令堂上坐贫僧奉拜。”老妪道:“长老远客各请自珍不劳拜罢。”伯钦道:“母亲呵他是唐王驾下差往西天见佛求经者。适间在岭头上遇着孩儿孩儿念一国之人请他来家歇马明日送他上路。”老妪闻言十分欢喜道:“好!好!好!就是请他不得这般恰好明日你父亲周忌就浼长老做些好事念卷经文到后日送他去罢。”这刘伯钦虽是一个杀虎手镇山的太保他却有些孝顺之心闻得母言就要安排香纸留住三藏。

说话间不觉的天色将晚。小的们排开桌凳拿几盘烂熟虎肉热腾腾的放在上面。伯钦请三藏权用再另办饭。三藏合掌当胸道:“善哉!贫僧不瞒太保说自出娘胎就做和尚更不晓得吃荤。”伯钦闻得此说沉吟了半晌道:“长老寒家历代以来不晓得吃素。就是有些竹笋采些木耳寻些干菜做些豆腐也都是獐鹿虎豹的油煎却无甚素处。有两眼锅灶也都是油腻透了这等奈何?反是我请长老的不是。”三藏道:“太保不必多心请自受用。我贫僧就是三五日不吃饭也可忍饿只是不敢破了斋戒。”伯钦道:“倘或饿死却如之何?”三藏道:

“感得太保天恩搭救出虎狼丛里就是饿死也强如喂虎。”伯钦的母亲闻说叫道:“孩儿不要与长老闲讲我自有素物可以管待。”伯钦道:“素物何来?”母亲道:“你莫管我我自有素的。”叫媳妇将小锅取下着火烧了油腻刷了又刷洗了又洗却仍安在灶上。先烧半锅滚水别用却又将些山地榆叶子着水煎作茶汤然后将些黄粱粟米煮起饭来又把些干菜煮熟盛了两碗拿出来铺在桌上。老母对着三藏道:“长老请斋这是老身与儿妇亲自动手整理的些极洁极净的茶饭。”三藏下来谢了方才上坐。那伯钦另设一处铺排些没盐没酱的老虎肉、香獐肉、蟒蛇肉、狐狸肉、兔肉点剁鹿肉干巴满盘满碗的陪着三藏吃斋。方坐下心欲举著只见三藏合掌诵经唬得个伯钦不敢动著急起身立在旁边。三藏念不数句却教“请斋”。伯钦道:“你是个念短头经的和尚?”三藏道:“此非是经乃是一卷揭斋之咒。”伯钦道:“你们出家人偏有许多计较吃饭便也念诵念诵。”

吃了斋饭收了盘碗渐渐天晚伯钦引着三藏出中宅到后边走走穿过夹道有一座草亭。推开门入到里面只见那四壁上挂几张强弓硬弩插几壶箭过梁上搭两块血腥的虎皮墙根头插着许多枪刀叉棒正中间设两张坐器。伯钦请三藏坐坐。三藏见这般凶险腌脏不敢久坐遂出了草亭。又往后再行是一座大园子却看不尽那丛丛菊蕊堆黄树树枫杨挂赤;又见呼的一声跑出十来只肥鹿一大阵黄獐见了人呢呢痴痴更不恐惧。三藏道:“这獐鹿想是太保养家了的?”伯钦道:“似你那长安城中人家有钱的集财宝有庄的集聚稻粮似我们这打猎的只得聚养些野兽备天阴耳。”他两个说话闲行不觉黄昏复转前宅安歇。

次早那合家老小都起来就整素斋管待长老请开启念经。这长老净了手同太保家堂前拈了香拜了家堂。三藏方敲响木鱼先念了净口业的真言又念了净身心的神咒然后开《度亡经》一卷。诵毕伯钦又请写荐亡疏一道再开念《金刚经》、《观音经》一一朗音高诵。诵毕吃了午斋又念《法华经》、《弥陀经》。各诵几卷又念一卷《孔雀经》及谈苾蒭洗业的故事早又天晚。献过了种种香火化了众神纸马烧了荐亡文疏佛事已毕又各安寝。

却说那伯钦的父亲之灵荐得脱沉沦鬼魂儿早来到东家宅内托一梦与合宅长幼道:“我在阴司里苦难难脱日久不得生。今幸得圣僧念了经卷消了我的罪业阎王差人送我上中华富地长者人家托生去了。你们可好生谢送长老不要怠慢、不要怠慢。我去也。”这才是:万法庄严端有意荐亡离苦出沉沦。那合家儿梦醒又早太阳东上伯钦的娘子道:“太保我今夜梦见公公来家说他在阴司苦难难脱日久不得生。今幸得圣僧念了经卷消了他的罪业阎王差人送他上中华富地长者人家托生去教我们好生谢那长老不得怠慢。他说罢径出门徉徜去了。我们叫他不应留他不住醒来却是一梦。”伯钦道:“我也是那等一梦与你一般。我们起去对母亲说去。”他两口子正欲去说只见老母叫道:“伯钦孩儿你来我与你说话。”二人至前老母坐在床上道:“儿呵我今夜得了个喜梦梦见你父亲来家说多亏了长老度已消了罪业上中华富地长者家去托生。”夫妻们俱呵呵大笑道:“我与媳妇皆有此梦正来告禀不期母亲呼唤也是此梦。”遂叫一家大小起来安排谢意替他收拾马匹都至前拜谢道:“多谢长老荐我亡父脱难生报答不尽!”三藏道:“贫僧有何能处敢劳致谢!”

伯钦把三口儿的梦话对三藏陈诉一遍三藏也喜。早供给了素斋又具白银一两为谢。三藏分文不受。一家儿又恳恳拜央三藏毕竟分文未受但道:“是你肯慈悲送我一程足感至爱。”伯钦与母妻无奈急做了些粗面烧饼干粮叫伯钦远送三藏欢喜收纳。太保领了母命又唤两三个家僮各带捕猎的器械同上大路看不尽那山中野景岭上风光。行经半日只见对面处有一座大山真个是高接青霄崔巍险峻。三藏不一时到了边前。那太保登此山如行平地。正走到半山之中伯钦回身立于路下道:“长老你自前进我却告回。”三藏闻言滚鞍下马道:“千万敢劳太保再送一程!”伯钦道:“长老不知此山唤做两界山东半边属我大唐所管西半边乃是鞑靼的地界。那厢狼虎不伏我降我却也不能过界你自去罢。”三藏心惊轮开手牵衣执袂滴泪难分。正在那叮咛拜别之际只听得山脚下叫喊如雷道:“我师父来也!我师父来也!”唬得个三藏痴呆伯钦打挣。毕竟不知是甚人叫喊且听下回分解。

\chapter[心猿归正\ 六贼无踪]{心猿归正\\六贼无踪}

第十四回 心猿归正 六贼无踪

诗曰:佛即心兮心即佛心佛从来皆要物。若知无物又无心便是真如法身佛。法身佛没模样一颗圆光涵万象。无体之体即真体无相之相即实相。非色非空非不空不来不向不回向。无异无同无有无难舍难取难听望。内外灵光到处同一佛国在一沙中。一粒沙含大千界一个身心万法同。知之须会无心诀不染不滞为净业。善恶千端无所为便是南无释迦叶。却说那刘伯钦与唐三藏惊惊慌慌又闻得叫声师父来也。

众家僮道:“这叫的必是那山脚下石匣中老猿。”太保道:“是他!是他!”三藏问:“是甚么老猿?”太保道:“这山旧名五行山因我大唐王征西定国改名两界山。先年间曾闻得老人家说:

‘王莽篡汉之时天降此山下压着一个神猴不怕寒暑不吃饮食自有土神监押教他饥餐铁丸渴饮铜汁。自昔到今冻饿不死。’这叫必定是他。长老莫怕我们下山去看来。”三藏只得依从牵马下山。行不数里只见那石匣之间果有一猴露着头伸着手乱招手道:“师父你怎么此时才来?来得好!来得好!救我出来我保你上西天去也!”这长老近前细看你道他是怎生模样:尖嘴缩腮金睛火眼。头上堆苔藓耳中生薜萝。鬓边少多青草颔下无须有绿莎。眉间土鼻凹泥十分狼狈指头粗手掌厚尘垢余多。还喜得眼睛转动喉舌声和。

语言虽利便身体莫能那。正是五百年前孙大圣今朝难满脱天罗。

这太保诚然胆大走上前来与他拔去了鬓边草颔下莎问道:“你有甚么说话?”那猴道:“我没话说教那个师父上来我问他一问。”三藏道:“你问我甚么?”那猴道:“你可是东土大王差往西天取经去的么?”三藏道:“我正是你问怎么?”那猴道:“我是五百年前大闹天宫的齐天大圣只因犯了诳上之罪被佛祖压于此处。前者有个观音菩萨领佛旨意上东土寻取经人。我教他救我一救他劝我再莫行凶归依佛法尽殷勤保护取经人往西方拜佛功成后自有好处。故此昼夜提心晨昏吊胆只等师父来救我脱身。我愿保你取经与你做个徒弟。”

三藏闻言满心欢喜道:“你虽有此善心又蒙菩萨教诲愿入沙门只是我又没斧凿如何救得你出?”那猴道:“不用斧凿你但肯救我我自出来也。”三藏道:“我自救你你怎得出来?”

那猴道:“这山顶上有我佛如来的金字压帖。你只上出去将帖儿揭起我就出来了。”三藏依言回头央浼刘伯钦道:“太保啊我与你上出走一遭。”伯钦道:“不知真假何如!”那猴高叫道:“是真!决不敢虚谬!”伯钦只得呼唤家僮牵了马匹。他却扶着三藏复上高山攀藤附葛只行到那极巅之处果然见金光万道瑞气千条有块四方大石石上贴着一封皮却是“唵、嘛、呢、叭、、吽”六个金字。三藏近前跪下朝石头看着金字拜了几拜望西祷祝道:“弟子陈玄奘特奉旨意求经果有徒弟之分揭得金字救出神猴同证灵山;若无徒弟之分此辈是个凶顽怪物哄赚弟子不成吉庆便揭不得起。”祝罢又拜。拜毕上前将六个金字轻轻揭下。只闻得一阵香风劈手把压帖儿刮在空中叫道:“吾乃监押大圣者。今日他的难满吾等回见如来缴此封皮去也。”吓得个三藏与伯钦一行人望空礼拜。径下高山又至石匣边对那猴道:“揭了压帖矣你出来么。”那猴欢喜叫道:“师父你请走开些我好出来莫惊了你。”伯钦听说领着三藏一行人回东即走。走了五七里远近又听得那猴高叫道:“再走!再走!”三藏又行了许远下了山只闻得一声响亮真个是地裂山崩。众人尽皆悚惧只见那猴早到了三藏的马前赤淋淋跪下道声“师父我出来也!”对三藏拜了四拜急起身与伯钦唱个大喏道:“有劳大哥送我师父又承大哥替我脸上薅草。”谢毕就去收拾行李扣背马匹。

那马见了他腰软蹄矬战兢兢的立站不住。盖因那猴原是弼马温在天上看养龙马的有些法则故此凡马见他害怕。

三藏见他意思实有好心真个象沙门中的人物便叫:

“徒弟啊你姓甚么?”猴王道:“我姓孙。”三藏道:“我与你起个法名却好呼唤。”猴王道:“不劳师父盛意我原有个法名叫做孙悟空。”三藏欢喜道:“也正合我们的宗派。你这个模样就象那小头陀一般我再与你起个混名称为行者好么?”悟空道:“好!好!好!”自此时又称为孙行者。那伯钦见孙行者一心收拾要行却转身对三藏唱个喏道:“长老你幸此间收得个好徒甚喜甚喜此人果然去得。我却告回。”三藏躬身作礼相谢道:“多有拖步感激不胜。回府多多致意令堂老夫人令荆夫人贫僧在府多扰容回时踵谢。”伯钦回礼遂此两下分别。

却说那孙行者请三藏上马他在前边背着行李赤条条拐步而行。不多时过了两界山忽然见一只猛虎咆哮剪尾而来三藏在马上惊心。行者在路旁欢喜道:“师父莫怕他他是送衣服与我的。”放下行李耳朵里拔出一个针儿迎着风幌一幌原来是个碗来粗细一条铁棒。他拿在手中笑道:“这宝贝五百余年不曾用着他今日拿出来挣件衣服儿穿穿。”你看他拽开步迎着猛虎道声“业畜!那里去!”那只虎蹲着身伏在尘埃动也不敢动动。却被他照头一棒就打的脑浆迸万点桃红牙齿喷几点玉块唬得那陈玄奘滚鞍落马咬指道声“天哪!天哪!刘太保前日打的斑斓虎还与他斗了半日;今日孙悟空不用争持把这虎一棒打得稀烂正是强中更有强中手!”

行者拖将虎来道:“师父略坐一坐等我脱下他的衣服来穿了走路。”三藏道:“他那里有甚衣服?”行者道:“师父莫管我我自有处置。”好猴王把毫毛拔下一根吹口仙气叫“变!”变作一把牛耳尖刀从那虎腹上挑开皮往下一剥剥下个囫囵皮来剁去了爪甲割下头来割个四四方方一块虎皮提起来量了一量道:“阔了些儿一幅可作两幅。”拿过刀来又裁为两幅。收起一幅把一幅围在腰间路旁揪了一条葛藤紧紧束定遮了下体道:“师父且去!且去!到了人家借些针线再缝不迟。”他把条铁棒捻一捻依旧象个针儿收在耳里背着行李请师父上马。

两个前进长老在马上问道:“悟空你才打虎的铁棒如何不见?”行者笑道:“师父你不晓得。我这棍本是东洋大海龙宫里得来的唤做天河镇底神珍铁又唤做如意金箍棒。当年大反天宫甚是亏他。随身变化要大就大要小就小。刚才变做一个绣花针儿模样收在耳内矣。但用时方可取出。”三藏闻言暗喜。又问道:“方才那只虎见了你怎么就不动动让自在打他何说?悟空道:“不瞒师父说莫道是只虎就是一条龙见了我也不敢无礼。我老孙颇有降龙伏虎的手段翻江搅海的神通见貌辨色聆音察理大之则量于宇宙小之则摄于毫毛!变化无端隐显莫测。剥这个虎皮何为稀罕?见到那疑难处看展本事么!”三藏闻得此言愈加放怀无虑策马前行。师徒两个走着路说着话不觉得太阳星坠但见:焰焰斜辉返照天涯海角归云。千出鸟雀噪声频觅宿投林成阵。野兽双双对对回窝族族群群。一勾新月破黄昏万点明星光晕。

行者道:师父走动些天色晚了。那壁厢树木森森想必是人家庄院我们赶早投宿去来。”三藏果策马而行径奔人家到了庄院前下马。行者撇了行李走上前叫声“开门!开门!”那里面有一老者扶筇而出唿喇的开了门看见行者这般恶相腰系着一块虎皮好似个雷公模样唬得脚软身麻口出谵语道:

“鬼来了!鬼来了!”三藏近前搀住叫道:“老施主休怕。他是我贫僧的徒弟不是鬼怪。”老者抬头见了三藏的面貌清奇方然立定问道:“你是那寺里来的和尚带这恶人上我门来?”

三藏道:“我贫僧是唐朝来的往西天拜佛求经适路过此间天晚特造檀府借宿一宵明早不犯天光就行。万望方便一二。”老者道:“你虽是个唐人那个恶的却非唐人。”悟空厉声高呼道:“你这个老儿全没眼色!唐人是我师父我是他徒弟!

我也不是甚糖人蜜人我是齐天大圣。你们这里人家也有认得我的我也曾见你来。”那老者道:“你在那里见我?”悟空道:

“你小时不曾在我面前扒柴?不曾在我脸上挑菜?”老者道:“这厮胡说!你在那里住?我在那里住?我来你面前扒柴挑菜!”

悟空道:“我儿子便胡说!你是认不得我了我本是这两界山石匣中的大圣。你再认认看。”老者方才省悟道:“你倒有些象他但你是怎么得出来的?”悟空将菩萨劝善、令我等待唐僧揭贴脱身之事对那老者细说了一遍。老者却才下拜将唐僧请到里面即唤老妻与儿女都来相见具言前事个个欣喜。又命看茶茶罢问悟空道:“大圣啊你也有年纪了?”悟空道:“你今年几岁了?”老者道:“我痴长一百三十岁了。”行者道:“还是我重子重孙哩!我那生身的年纪我不记得是几时但只在这山脚下已五百余年了。”老者道:“是有是有。我曾记得祖公公说此山乃从天降下就压了一个神猴。只到如今你才脱体。

我那小时见你是你头上有草脸上有泥还不怕你;如今脸上无了泥头上无了草却象瘦了些腰间又苫了一块大虎皮与鬼怪能差多少?”

一家儿听得这般话说都呵呵大笑。这老儿颇贤即今安排斋饭。饭后悟空道:“你家姓甚?”老者道:“舍下姓陈。”三藏闻言即下来起手道:“老施主与贫僧是华宗。”行者道:“师父你是唐姓怎的和他是华宗?”三藏道:“我俗家也姓陈乃是唐朝海州弘农郡聚贤庄人氏。我的法名叫做陈玄奘。只因我大唐太宗皇帝赐我做御弟三藏指唐为姓故名唐僧也。”那老者见说同姓又十分欢喜。行者道:“老陈左右打搅你家。我有五百多年不洗澡了你可去烧些汤来与我师徒们洗浴洗浴一临行谢你。”那老儿即令烧汤拿盆掌上灯火。师徒浴罢坐在灯前行者道:“老陈还有一事累你有针线借我用用。”那老儿道:“有有有。”即教妈妈取针线来递与行者。行者又有眼色见师父洗浴脱下一件白布短小直裰未穿他即扯过来披在身上却将那虎皮脱下联接一处打一个马面样的折子围在腰间勒了藤条走到师父面前道:“老孙今日这等打扮比昨日如何?”三藏道:“好!好!好!这等样才象个行者。”三藏道:“徒弟你不嫌残旧那件直裰儿你就穿了罢。”悟空唱个喏道:“承赐!承赐!”他又去寻些草料喂了马。此时各各事毕师徒与那老儿亦各归寝。

次早悟空起来请师父走路。三藏着衣教行者收拾铺盖行李。正欲告辞只见那老儿早具脸汤又具斋饭。斋罢方才起身。三藏上马行者引路不觉饥餐渴饮夜宿晓行又值初冬时候但见那:霜凋红叶千林瘦岭上几株松柏秀。未开梅蕊散香幽暖短昼小春候菊残荷尽山茶茂。寒桥古树争枝斗曲涧涓涓泉水溜。淡云欲雪满天浮朔风骤牵衣袖向晚寒威人怎受?师徒们正走多时忽见路旁唿哨一声闯出六个人来各执长枪短剑利刃强弓大咤一声道:“那和尚!那里走!赶早留下马匹放下行李饶你性命过去!”唬得那三藏魂飞魄散跌下马来不能言语。行者用手扶起道:“师父放心没些儿事这都是送衣服送盘缠与我们的。”三藏道:“悟空你想有些耳闭?他说教我们留马匹、行李你倒问他要甚么衣服、盘缠?”行者道:“你管守着衣服、行李、马匹待老孙与他争持一场看是何如。”三藏道:“好手不敌双拳双拳不如四手。他那里六条大汉你这般小小的一个人儿怎么敢与他争持?”

行者的胆量原大那容分说走上前来叉手当胸对那六个人施礼道:“列位有甚么缘故阻我贫僧的去路?”那人道:

“我等是剪径的大王行好心的山主。大名久播你量不知早早的留下东西放你过去;若道半个不字教你碎尸粉骨!”行者道:“我也是祖传的大王积年的山主却不曾闻得列位有甚大名。”那人道:“你是不知我说与你听:一个唤做眼看喜一个唤做耳听怒一个唤做鼻嗅爱一个唤作舌尝思一个唤作意见欲一个唤作身本忧。”悟空笑道:“原来是六个毛贼!你却不认得我这出家人是你的主人公你倒来挡路。把那打劫的珍宝拿出来我与你作七分儿均分饶了你罢!”那贼闻言喜的喜怒的怒爱的爱思的思欲的欲忧的忧一齐上前乱嚷道:“这和尚无礼!你的东西全然没有转来和我等要分东西!”

他轮枪舞剑一拥前来照行者劈头乱砍乒乒乓乓砍有七八十下。悟空停立中间只当不知。那贼道:“好和尚!真个的头硬!”行者笑道:“将就看得过罢了!你们也打得手困了却该老孙取出个针儿来耍耍。”那贼道:“这和尚是一个行针灸的郎中变的。我们又无病症说甚么动针的话!”行者伸手去耳朵里拔出一根绣花针儿迎风一幌却是一条铁棒足有碗来粗细拿在手中道:“不要走!也让老孙打一棍儿试试手!”唬得这六个贼四散逃走被他拽开步团团赶上一个个尽皆打死。剥了他的衣服夺了他的盘缠笑吟吟走将来道:“师父请行那贼已被老孙剿了。”三藏道:“你十分撞祸!他虽是剪径的强徒就是拿到官司也不该死罪;你纵有手段只可退他去便了怎么就都打死?这却是无故伤人的性命如何做得和尚?出家人扫地恐伤蝼蚁命爱惜飞蛾纱罩灯。你怎么不分皂白一顿打死?全无一点慈悲好善之心!早还是山野中无人查考;若到城市倘有人一时冲撞了你你也行凶执着棍子乱打伤人我可做得白客怎能脱身?”悟空道:“师父我若不打死他他却要打死你哩。”三藏道:“我这出家人宁死决不敢行凶。我就死也只是一身你却杀了他六人如何理说?此事若告到官就是你老子做官也说不过去。”行者道:“不瞒师父说我老孙五百年前据花果山称王为怪的时节也不知打死多少人。假似你说这般到官倒也得些状告是。”三藏道:“只因你没收没管暴横人间欺天诳上才受这五百年前之难。今既入了沙门若是还象当时行凶一味伤生去不得西天做不得和尚!忒恶!忒恶!”原来这猴子一生受不得人气他见三藏只管绪绪叨叨按不住心头火道:“你既是这等说我做不得和尚上不得西天不必惩般绪咶恶我我回去便了!”那三藏却不曾答应他就使一个性子将身一纵说一声“老孙去也!”三藏急抬头早已不见只闻得呼的一声回东而去。撇得那长老孤孤零零点头自叹悲怨不已道:“这厮!这等不受教诲!我但说他几句他怎么就无形无影的径回去了?罢!罢!罢!也是我命里不该招徒弟进人口!如今欲寻他无处寻欲叫他叫不应去来!

去来!”正是舍身拚命归西去莫倚旁人自主张。

那长老只得收拾行李捎在马上也不骑马一只手柱着锡杖一只手揪着缰绳凄凄凉凉往西前进。行不多时只见山路前面有一个年高的老母捧一件绵衣绵衣上有一顶花帽。三藏见他来得至近慌忙牵马立于右侧让行。那老母问道:“你是那里来的长老孤孤凄凄独行于此?”三藏道:“弟子乃东土大唐奉圣旨往西天拜活佛求真经者。”老母道:“西方佛乃大雷音寺天竺国界此去有十万八千里路。你这等单人独马又无个伴侣又无个徒弟你如何去得!”三藏道:“弟子日前收得一个徒弟他性泼凶顽是我说了他几句他不受教遂渺然而去也。”老母道:“我有这一领绵布直裰一顶嵌金花帽原是我儿子用的。他只做了三日和尚不幸命短身亡。我才去他寺里哭了一场辞了他师父将这两件衣帽拿来做个忆念。长老啊你既有徒弟我把这衣帽送了你罢。”三藏道:“承老母盛赐但只是我徒弟已走了不敢领受。”老母道:“他那厢去了?”三藏道:“我听得呼的一声他回东去了。”老母道:“东边不远就是我家想必往我家去了。我那里还有一篇咒儿唤做定心真言又名做紧箍儿咒。你可暗暗的念熟牢记心头再莫泄漏一人知道。我去赶上他叫他还来跟你你却将此衣帽与他穿戴。他若不服你使唤你就默念此咒他再不敢行凶也再不敢去了。”三藏闻言低头拜谢。那老母化一道金光回东而去。三藏情知是观音菩萨授此真言急忙撮土焚香望东恳恳礼拜。拜罢收了衣帽藏在包袱中间却坐于路旁诵习那定心真言。来回念了几遍念得烂熟牢记心胸不题。

却说那悟空别了师父一筋斗云径转东洋大海。按住云头分开水道径至水晶宫前。早惊动龙王出来迎接接至宫里坐下礼毕、龙王道:“近闻得大圣难满失贺!想必是重整仙山复归古洞矣。”悟空道:“我也有此心性只是又做了和尚了。”龙王道:“做甚和尚?”行者道:“我亏了南海菩萨劝善教我正果随东土唐僧上西方拜佛皈依沙门又唤为行者了。”

龙王道:“这等真是可贺!可贺!这才叫做改邪归正惩创善心。

既如此怎么不西去复东回何也?”行者笑道:“那是唐僧不识人性。有几个毛贼剪径是我将他打死唐僧就绪绪叨叨说了我若干的不是你想老孙可是受得闷气的?是我撇了他欲回本山故此先来望你一望求钟茶吃。”龙王道:“承降!承降!”

当时龙子龙孙即捧香茶来献。

茶毕行者回头一看见后壁上挂著一幅圯桥进履的画儿。行者道:“这是甚么景致?”龙王道:“大圣在先此事在后故你不认得。这叫做圯桥三进履。”行者道:“怎的是三进履?”

龙王道:“此仙乃是黄石公此子乃是汉世张良。石公坐在圯桥上忽然失履于桥下遂唤张良取来。此子即忙取来跪献于前。如此三度张良略无一毫倨傲怠慢之心石公遂爱他勤谨夜授天书着他扶汉。后果然运筹帷幄之中决胜千里之外。太平后弃职归山从赤松子游悟成仙道。大圣你若不保唐僧不尽勤劳不受教诲到底是个妖仙休想得成正果。”悟空闻言沉吟半晌不语。龙王道:“大圣自当裁处不可图自在误了前程。”悟空道:“莫多话老孙还去保他便了。”龙王欣喜道:

“既如此不敢久留请大圣早慈悲莫要疏久了你师父。”行者见他催促请行急耸身出离海藏驾着云别了龙王。正走却遇着南海菩萨。菩萨道:“孙悟空你怎么不受教诲不保唐僧来此处何干?”慌得个行者在云端里施礼道:“向蒙菩萨善言果有唐朝僧到揭了压帖救了我命跟他做了徒弟。他却怪我凶顽我才闪了他一闪如今就去保他也。”菩萨道:“赶早去莫错过了念头。”言毕各回。

这行者须臾间看见唐僧在路旁闷坐。他上前道:“师父!

怎么不走路?还在此做甚?”三藏抬头道:“你往那里去来?教我行又不敢行动又不敢动只管在此等你。”行者道:“我往东洋大海老龙王家讨茶吃吃。”三藏道:“徒弟啊出家人不要说谎。你离了我没多一个时辰就说到龙王家吃茶?”行者笑道:

“不瞒师父说我会驾筋斗云一个筋斗有十万八千里路故此得即去即来。”三藏道:“我略略的言语重了些儿你就怪我使个性子丢了我去。象你这有本事的讨得茶吃;象我这去不得的只管在此忍饿你也过意不去呀!”行者道:“师父你若饿了我便去与你化些斋吃。”三藏道:“不用化斋。我那包袱里还有些干粮是刘太保母亲送的你去拿钵盂寻些水来等我吃些儿走路罢。”行者去解开包袱在那包裹中间见有几个粗面烧饼拿出来递与师父。又见那光艳艳的一领绵布直裰一顶嵌金花帽行者道:“这衣帽是东土带来的?”三藏就顺口儿答应道:“是我小时穿戴的。这帽子若戴了不用教经就会念经;这衣服若穿了不用演礼就会行礼。”行者道:“好师父把与我穿戴了罢。”三藏道:“只怕长短不一你若穿得就穿了罢。”行者遂脱下旧白布直裰将绵布直裰穿上也就是比量着身体裁的一般把帽儿戴上。三藏见他戴上帽子就不吃干粮却默默的念那紧箍咒一遍。行者叫道:“头痛!头痛!”那师父不住的又念了几遍把个行者痛得打滚抓破了嵌金的花帽。

三藏又恐怕扯断金箍住了口不念。不念时他就不痛了。伸手去头上摸摸似一条金线儿模样紧紧的勒在上面取不下揪不断已此生了根了。他就耳里取出针儿来插入箍里往外乱捎。三藏又恐怕他捎断了口中又念起来他依旧生痛痛得竖蜻蜓翻筋斗耳红面赤眼胀身麻。那师父见他这等又不忍不舍复住了口他的头又不痛了。行者道:“我这头原来是师父咒我的。”三藏道:“我念得是紧箍经何曾咒你?”行者道:

“你再念念看。”三藏真个又念行者真个又痛只教:“莫念!莫念!念动我就痛了!这是怎么说?”三藏道:“你今番可听我教诲了?”行者道:“听教了!”“你再可无礼了?”行者道:“不敢了!”他口里虽然答应心上还怀不善把那针儿幌一幌碗来粗细望唐僧就欲下手慌得长老口中又念了两三遍这猴子跌倒在地丢了铁棒不能举手只教:“师父!我晓得了!再莫念!再莫念!”三藏道:“你怎么欺心就敢打我?”行者道:“我不曾敢打我问师父你这法儿是谁教你的?”三藏道:“是适间一个老母传授我的。”行者大怒道:“不消讲了!这个老母坐定是那个观世音!他怎么那等害我!等我上南海打他去!”三藏道:

“此法既是他授与我他必然先晓得了。你若寻他他念起来你却不是死了?”行者见说得有理真个不敢动身只得回心跪下哀告道:“师父!这是他奈何我的法儿教我随你西去。我也不去惹他你也莫当常言只管念诵。我愿保你再无退悔之意了。”三藏道:“既如此伏侍我上马去也。”那行者才死心塌地抖擞精神束一束绵布直裰扣背马匹收拾行李奔西而进。毕竟这一去后面又有甚话说且听下回分解。

\chapter[蛇盘山诸神暗佑\ 鹰愁涧意马收缰]{蛇盘山诸神暗佑\\鹰愁涧意马收缰}

第十五回 蛇盘山诸神暗佑 鹰愁涧意马收缰

却说行者伏侍唐僧西进行经数日正是那腊月寒天朔风凛凛滑冻凌凌去的是些悬崖峭壁崎岖路迭岭层峦险峻山。三藏在马上遥闻唿喇喇水声聒耳回头叫:“悟空是那里水响?”行者道:“我记得此处叫做蛇盘山鹰愁涧想必是涧里水响。”说不了马到涧边三藏勒缰观看但见:涓涓寒脉穿云过湛湛清波映日红。声摇夜雨闻幽谷彩朝霞眩太空。千仞浪飞喷碎玉一泓水响吼清风。流归万顷烟波去鸥鹭相忘没钓逢。师徒两个正然看处只见那涧当中响一声钻出一条龙来推波掀浪撺出崖山就抢长老。慌得个行者丢了行李把师父抱下马来回头便走。那条龙就赶不上把他的白马连鞍辔一口吞下肚去依然伏水潜踪。行者把师父送在那高阜上坐了却来牵马挑担止存得一担行李不见了马匹。他将行李担送到师父面前道:“师父那孽龙也不见踪影只是惊走我的马了。”三藏道:“徒弟啊却怎生寻得马着么?”行者道:“放心放心等我去看来。”

他打个唿哨跳在空中火眼金睛用手搭凉篷四下里观看更不见马的踪迹。按落云头报道:“师父我们的马断乎是那龙吃了四下里再看不见。”三藏道:“徒弟呀那厮能有多大口却将那匹大马连鞍辔都吃了?想是惊张溜缰走在那山凹之中。你再仔细看看。”行者道:“你也不知我的本事。我这双眼白日里常看一千里路的吉凶。象那千里之内蜻蜓儿展翅我也看见何期那匹大马我就不见!”三藏道:“既是他吃了我如何前进!可怜啊!这万水千山怎生走得!”说着话泪如雨落。行者见他哭将起来他那里忍得住暴燥声喊道:“师父莫要这等脓包形么!你坐着!坐着!等老孙去寻着那厮教他还我马匹便了。”三藏却才扯住道:“徒弟啊你那里去寻他?

只怕他暗地里撺将出来却不又连我都害了?那时节人马两亡怎生是好!”行者闻得这话越加嗔怒就叫喊如雷道:“你忒不济!不济!又要马骑又不放我去似这般看着行李坐到老罢!”哏哏的吆喝正难息怒只听得空中有人言语叫道:

“孙大圣莫恼唐御弟休哭。我等是观音菩萨差来的一路神祇特来暗中保取经者。”那长老闻言慌忙礼拜。行者道:“你等是那几个?可报名来我好点卯。”众神道:“我等是六丁六甲、五方揭谛、四值功曹、一十八位护教伽蓝各各轮流值日听候。”

行者道:“今日先从谁起?”众揭谛道:“丁甲、功曹、伽蓝轮次。

我五方揭谛惟金头揭谛昼夜不离左右。”行者道:“既如此不当值者且退留下六丁神将与日值功曹和众揭谛保守着我师父。等老孙寻那涧中的孽龙教他还我马来。”众神遵令。三藏才放下心坐在石崖之上吩咐行者仔细行者道:“只管宽心。”好猴王束一束绵布直裰撩起虎皮裙子揝着金箍铁棒抖擞精神径临涧壑半云半雾的在那水面上高叫道:“泼泥鳅还我马来!还我马来!”

却说那龙吃了三藏的白马伏在那涧底中间潜灵养性。

只听得有人叫骂索马他按不住心中火急纵身跃浪翻波跳将上来道:“是那个敢在这里海口伤吾?”行者见了他大咤一声“休走!还我马来!”轮着棍劈头就打。那条龙张牙舞爪来抓。他两个在涧边前这一场赌斗果是骁雄但见那:龙舒利爪猴举金箍。那个须垂白玉线这个服幌赤金灯。那个须下明珠喷彩雾这个手中铁棒舞狂风。那个是迷爷娘的业子这个是欺天将的妖精。他两个都因有难遭磨折今要成功各显能。来来往往战罢多时盘旋良久那条龙力软筋麻不能抵敌打一个转身又撺于水内深潜涧底再不出头被猴王骂詈不绝他也只推耳聋。

行者没及奈何只得回见三藏道:“师父这个怪被老孙骂将出来他与我赌斗多时怯战而走只躲在水中间再不出来了。”三藏道:“不知端的可是他吃了我马?”行者道:“你看你说的话!不是他吃了他还肯出来招声与老孙犯对?”三藏道:

“你前日打虎时曾说有降龙伏虎的手段今日如何便不能降他?”原来那猴子吃不得人急他见三藏抢白了他这一句他就起神威道:“不要说!不要说!等我与他再见个上下!”

这猴王拽开步跳到涧边使出那翻江搅海的神通把一条鹰愁陡涧彻底澄清的水搅得似那九曲黄河泛涨的波。那孽龙在于深涧中坐卧宁心中思想道:“这才是福无双降祸不单行。我才脱了天条死难不上一年在此随缘度日又撞着这般个泼魔他来害我!”你看他越思越恼受不得屈气咬着牙跳将出去骂道:“你是那里来的泼魔这等欺我!”行者道:“你莫管我那里不那里你只还了马我就饶你性命!”那龙道:“你的马是我吞下肚去如何吐得出来!不还你便待怎的!”行者道“不还马时看棍!只打杀你偿了我马的性命便罢!”他两个又在那山崖下苦斗。斗不数合小龙委实难搪将身一幌变作一条水蛇儿钻入草科中去了。

猴王拿着棍赶上前来拨草寻蛇那里得些影响?急得他三尸神咋七窍烟生念了一声唵字咒语即唤出当坊土地、本处山神一齐来跪下道:“山神土地来见。”行者道:“伸过孤拐来各打五棍见面与老孙散散心!”二神叩头哀告道:“望大圣方便容小神诉告。”行者道:“你说甚么?”二神道:“大圣一向久困小神不知几时出来所以不曾接得万望恕罪。”行者道:

“既如此我且不打你。我问你:鹰愁涧里是那方来的怪龙?他怎么抢了我师父的白马吃了?”二神道:“大圣自来不曾有师父原来是个不伏天不伏地混元上真如何得有甚么师父的马来?”行者道:“你等是也不知。我只为那诳上的勾当整受了这五百年的苦难。今蒙观音菩萨劝善着唐朝驾下真僧救出我来教我跟他做徒弟往西天去拜佛求经。因路过此处失了我师父的白马。”二神道:“原来是如此。这涧中自来无邪只是深陡宽阔水光彻底澄清鸦鹊不敢飞过因水清照见自己的形影便认做同群之鸟往往身掷于水内故名鹰愁陡涧。只是向年间观音菩萨因为寻访取经人去救了一条玉龙送他在此教他等候那取经人不许为非作歹他只是饥了时上岸来扑些鸟鹊吃或是捉些獐鹿食用。不知他怎么无知今日冲撞了大圣。”行者道:“先一次他还与老孙侮手盘旋了几合;后一次是老孙叫骂他再不出因此使了一个翻江搅海的法儿搅混了他涧水他就撺将上来还要争持。不知老孙的棍重他遮架不住就变做一条水蛇钻在草里。我赶来寻他却无踪迹。”

土地道:“大圣不知这条涧千万个孔窍相通故此这波澜深远。想是此间也有一孔他钻将下去。也不须大圣怒在此找寻要擒此物只消请将观世音来自然伏了。”

行者见说唤山神土地同来见了三藏具言前事。三藏道:

“若要去请菩萨几时才得回来?我贫僧饥寒怎忍!”说不了只听得暗空中有金头揭谛叫道:“大圣你不须动身小神去请菩萨来也。”行者大喜道声“有累有累!快行快行!”那揭谛急纵云头径上南海。行者吩咐山神、土地守护师父日值功曹去寻斋供他又去涧边巡绕不题。

却说金头揭谛一驾云早到了南海按祥光直至落伽山紫竹林中托那金甲诸天与木叉惠岸转达得见菩萨。菩萨道:

“汝来何干?”揭谛道:“唐僧在蛇盘山鹰愁陡涧失了马急得孙大圣进退两难。及问本处土神说是菩萨送在那里的孽龙吞了那大圣着小神来告请菩萨降这孽龙还他马匹。”菩萨闻言道:“这厮本是西海敖闰之子。他为纵火烧了殿上明珠他父告他忤逆天庭上犯了死罪是我亲见玉帝讨他下来教他与唐僧做个脚力。他怎么反吃了唐僧的马?这等说等我去来。”那菩萨降莲台径离仙洞与揭谛驾着祥光过了南海而来。有诗为证诗曰:佛说蜜多三藏经菩萨扬善满长城。摩诃妙语通天地般若真言救鬼灵。致使金蝉重脱壳故令玄奘再修行。只因路阻鹰愁涧龙子归真化马形。那菩萨与揭谛不多时到了蛇盘山。却在那半空里留住祥云低头观看。只见孙行者正在涧边叫骂。菩萨着揭谛唤他来。那揭谛按落云头不经由三藏直至涧边对行者道:“菩萨来也。”行者闻得急纵云跳到空中对他大叫道:“你这个七佛之师慈悲的教主!你怎么生方法儿害我!”菩萨道:“我把你这个大胆的马流村愚的赤尻!我倒再三尽意度得个取经人来叮咛教他救你性命你怎么不来谢我活命之恩反来与我嚷闹?”行者道:“你弄得我好哩!你既放我出来让我逍遥自在耍子便了你前日在海上迎着我伤了我几句教我来尽心竭力伏侍唐僧便罢了;你怎么送他一顶花帽哄我戴在头上受苦?把这个箍子长在老孙头上又教他念一卷甚么紧箍儿咒着那老和尚念了又念教我这头上疼了又疼这不是你害我也?”菩萨笑道:“你这猴子!你不遵教令不受正果若不如此拘系你你又诳上欺天知甚好歹!再似从前撞出祸来有谁收管?须是得这个魔头你才肯入我瑜伽之门路哩!”行者道:“这桩事作做是我的魔头罢你怎么又把那有罪的孽龙送在此处成精教他吃了我师父的马匹?此又是纵放歹人为恶太不善也!”菩萨道:“那条龙是我亲奏玉帝讨他在此专为求经人做个脚力。你想那东土来的凡马怎历得这万水千山?怎到得那灵山佛地?须是得这个龙马方才去得。”行者道:“象他这般惧怕老孙潜躲不出如之奈何?”菩萨叫揭谛道:“你去涧边叫一声‘敖闰龙王玉龙三太子你出来有南海菩萨在此。’他就出来了。”那揭谛果去涧边叫了两遍。那小龙翻波跳浪跳出水来变作一个人象踏了云头到空中对菩萨礼拜道:“向蒙菩萨解脱活命之恩在此久等更不闻取经人的音信。”菩萨指着行者道:“这不是取经人的大徒弟?”小龙见了道:“菩萨这是我的对头。我昨日腹中饥馁果然吃了他的马匹。他倚着有些力量将我斗得力怯而回又骂得我闭门不敢出来他更不曾提着一个取经的字样。”行者道:

“你又不曾问我姓甚名谁我怎么就说?”小龙道:“我不曾问你是那里来的泼魔?你嚷道:‘管甚么那里不那里只还我马来!’何曾说出半个唐字!”菩萨道:“那猴头专倚自强那肯称赞别人?今番前去还有归顺的哩若问时先提起取经的字来却也不用劳心自然拱伏。”行者欢喜领教。菩萨上前把那小龙的项下明珠摘了将杨柳枝蘸出甘露往他身上拂了一拂吹口仙气喝声叫“变!”那龙即变做他原来的马匹毛片又将言语吩咐道:“你须用心了还业障功成后越凡龙还你个金身正果。”那小龙口衔着横骨心心领诺。菩萨教悟空领他去见三藏“我回海上去也。”行者扯住菩萨不放道:“我不去了!我不去了!西方路这等崎岖保这个凡僧几时得到?似这等多磨多折老孙的性命也难全如何成得甚么功果!我不去了!我不去了!”菩萨道:“你当年未成*人道且肯尽心修悟;你今日脱了天灾怎么倒生懒惰?我门中以寂灭成真须是要信心正果。

假若到了那伤身苦磨之处我许你叫天天应叫地地灵。十分再到那难脱之际我也亲来救你。你过来我再赠你一般本事。”菩萨将杨柳叶儿摘下三个放在行者的脑后喝声“变”!

即变做三根救命的毫毛教他:“若到那无济无主的时节可以随机应变救得你急苦之灾。”行者闻了这许多好言才谢了大慈大悲的菩萨。那菩萨香风绕绕彩雾飘飘径转普陀而去。

这行者才按落云头揪着那龙马的顶鬃来见三藏道:“师父马有了也。”三藏一见大喜道:“徒弟这马怎么比前反肥盛了些?在何处寻着的?”行者道:“师父你还做梦哩!却才是金头揭谛请了菩萨来把那涧里龙化作我们的白马。其毛片相同只是少了鞍辔着老孙揪将来也。”三藏大惊道:“菩萨何在?待我去拜谢他。”行者道:“菩萨此时已到南海不耐烦矣。”

三藏就撮土焚香望南礼拜拜罢起身即与行者收拾前进。行者喝退了山神土地吩咐了揭谛功曹却请师父上马。三藏道:

“那无鞍辔的马怎生骑得?且待寻船渡过涧去再作区处。”行者道:“这个师父好不知时务!这个旷野山中船从何来?这匹马他在此久住必知水势就骑着他做个船儿过去罢。”三藏无奈只得依言跨了刬马。行者挑着行囊到了涧边。只见那上流头有一个渔翁撑着一个枯木的筏子顺流而下。行者见了用手招呼道:“那老渔你来你来。我是东土取经去的我师父到此难过你来渡他一渡。”渔翁闻言即忙撑拢。行者请师父下了马扶持左右。三藏上了筏子揪上马匹安了行李。

那老渔撑开筏子如风似箭不觉的过了鹰愁陡涧上了西岸。

三藏教行者解开包袱取出大唐的几文钱钞送与老渔。老渔把筏子一篙撑开道:“不要钱不要钱。”向中流渺渺茫茫而去。

三藏甚不过意只管合掌称谢。行者道:“师父休致意了。你不认得他?他是此涧里的水神。不曾来接得我老孙老孙还要打他哩。只如今免打就彀了他的怎敢要钱!”那师父也似信不信只得又跨刬着马随着行者径投大路奔西而去。这正是:

广大真如登彼岸诚心了性上灵山。同师前进不觉的红日沉西天光渐晚但见:淡云撩乱山月昏蒙。满天霜色生寒四面风声透体。孤鸟去时苍渚阔落霞明处远山低。疏林千树吼空岭独猿啼。长途不见行人迹万里归舟入夜时。三藏在马上遥观忽见路旁一座庄院。三藏道:“悟空前面人家可以借宿明早再行。”行者抬头看见道:“师父不是人家庄院。”三藏道:“如何不是?”行者道:“人家庄院却没飞鱼稳兽之脊这断是个庙宇庵院。”

师徒们说着话早已到了门。三藏下了马只见那门上有三个大字乃里社祠遂入门里。那里边有一个老者:顶挂着数珠儿合掌来迎叫声“师父请坐。”三藏慌忙答礼上殿去参拜了圣象那老者即呼童子献茶。茶罢三藏问老者道:“此庙何为里社?”老者道:“敝处乃西番哈咇国界。这庙后有一庄人家共虔心立此庙宇。里者乃一乡里地;社者乃一社上神。每遇春耕、夏耘、秋收、冬藏之日各办三牲花果来此祭社以保四时清吉、五谷丰登、六畜茂盛故也。”三藏闻言点头夸赞:“正是离家三里远别是一乡风。我那里人家更无此善。”老者却问:“师父仙乡是何处?”三藏道:“贫僧是东土大唐国奉旨意上西天拜佛求经的。路过宝坊天色将晚特投圣祠告宿一宵天光即行。”那老者十分欢喜道了几声失迎又叫童子办饭。三藏吃毕谢了。行者的眼乖见他房檐下有一条搭衣的绳子走将去一把扯断将马脚系住。那老者笑道:“这马是那里偷来的?”行者怒道:“你那老头子说话不知高低!我们是拜佛的圣僧又会偷马?”老儿笑道:“不是偷的如何没有鞍辔缰绳却来扯断我晒衣的索子?”三藏陪礼道:“这个顽皮只是性燥。你要拴马好生问老人家讨条绳子如何就扯断他的衣索?老先休怪休怪。我这马实不瞒你说不是偷的:昨日东来至鹰愁陡涧原有骑的一匹白马鞍辔俱全。不期那涧里有条孽龙在彼成精他把我的马连鞍辔一口吞之。幸亏我徒弟有些本事又感得观音菩萨来涧边擒住那龙教他就变做我原骑的白马毛片俱同驮我上西天拜佛。今此过涧未经一日却到了老先的圣祠还不曾置得鞍辔哩。”那老者道:“师父休怪我老汉作笑耍子谁知你高徒认真。我小时也有几个村钱也好骑匹骏马只因累岁迍邅遭丧失火到此没了下梢故充为庙祝侍奉香火幸亏这后庄施主家募化度日。我那里倒还有一副鞍辔是我平日心爱之物就是这等贫穷也不曾舍得卖了。才听老师父之言菩萨尚且救护神龙教他化马驮你我老汉却不能少有周济明日将那鞍辔取来愿送老师父扣背前去乞为笑纳。”三藏闻言称谢不尽。早又见童子拿出晚斋斋罢掌上灯安了铺各各寝歇。

至次早行者起来道:“师父那庙祝老儿昨晚许我们鞍辔问他要不要饶他。”说未了只见那老儿果擎着一副鞍辔、衬屉缰笼之类凡马上一切用的无不全备放在廊下道:

“师父鞍辔奉上。”三藏见了欢喜领受教行者拿了背上马看可相称否。行者走上前一件件的取起看了果然是些好物。有诗为证诗曰:雕鞍彩晃柬银星宝凳光飞金线明。衬屉几层绒苫迭牵疆三股紫丝绳。辔头皮札团花粲云扇描金舞兽形。环嚼叩成磨炼铁两垂蘸水结毛缨。行者心中暗喜将鞍辔背在马上就似量着做的一般。三藏拜谢那老那老慌忙搀起道:“惶恐!惶恐!何劳致谢?”那老者也不再留请三藏上马。那长老出得门来攀鞍上马行者担着行李。那老儿复袖中取出一条鞭儿来却是皮丁儿寸札的香藤柄子虎筋丝穿结的梢儿在路旁拱手奉上道:“圣僧我还有一条挽手儿一送了你罢。”那三藏在马上接了道:“多承布施!多承布施!”正打问讯却早不见了那老儿及回看那里社祠是一片光地。只听得半空中有人言语道:“圣僧多简慢你。我是落伽山山神土地蒙菩萨差送鞍辔与汝等的。汝等可努力西行却莫一时怠慢。”慌得个三藏滚鞍下马望空礼拜道:“弟子肉眼凡胎不识尊神尊面望乞恕罪。烦转达菩萨深蒙恩佑。”你看他只管朝天磕头也不计其数路旁边活活的笑倒个孙大圣孜孜的喜坏个美猴王上前来扯住唐僧道:“师父你起来罢他已去得远了听不见你祷祝看不见你磕头。只管拜怎的?”长老道:

“徒弟呀我这等磕头你也就不拜他一拜且立在旁边只管哂笑是何道理?”行者道:“你那里知道象他这个藏头露尾的本该打他一顿只为看菩萨面上饶他打尽彀了他还敢受我老孙之拜?老孙自小儿做好汉不晓得拜人就是见了玉皇大帝、太上老君我也只是唱个喏便罢了。”三藏道:“不当人子!莫说这空头话!快起来莫误了走路。”那师父才起来收拾投西而去。

此去行有两个月太平之路相遇的都是些虏虏、回回狼虫虎豹。光阴迅又值早春时候但见山林锦翠色草木青芽;梅英落尽柳眼初开。师徒们行玩春光又见太阳西坠。三藏勒马遥观山凹里有楼台影影殿阁沉沉。三藏道:“悟空你看那里是甚么去处?”行者抬头看了道:“不是殿宇定是寺院。我们赶起些那里借宿去。”三藏欣然从之放开龙马径奔前来。毕竟不知此去是甚么去处且听下回分解。

\chapter[观音院僧谋宝贝\ 黑风山怪窃袈裟]{观音院僧谋宝贝\\黑风山怪窃袈裟}

第十六回 观音院僧谋宝贝 黑风山怪窃袈裟

却说他师徒两个策马前来直至山门观看果然是一座寺院。但见那层层殿阁选迭廊房三山门外巍巍万道彩云遮;五福堂前艳艳千条红雾绕。两路松篁一林桧柏。两路松篁无年无纪自清幽;一林桧柏有色有颜随傲丽。又见那钟鼓楼高浮屠塔峻。安禅僧定性啼树鸟音闲。寂寞无尘真寂寞清虚有道果清虚。诗曰:上刹祇园隐翠窝招提胜景赛婆婆。果然净土人间少天下名山僧占多。长老下了马行者歇了担正欲进门只见那门里走出一众僧来。你看他怎生模样:头戴左笄帽身穿无垢衣。铜环双坠耳绢带束腰围。草履行来稳木鱼手内提。口中常作念般若总皈依。三藏见了侍立门旁道个问讯那和尚连忙答礼笑道失瞻问:“是那里来的?请入方丈献茶。”三藏道:“我弟子乃东土钦差上雷音寺拜佛求经。至此处天色将晚欲借上刹一宵。”那和尚道:“请进里坐请进里坐。”三藏方唤行者牵马进来。那和尚忽见行者相貌有些害怕便问:“那牵马的是个甚么东西?”三藏道:“悄言!悄言!他的性急若听见你说是甚么东西他就恼了。他是我的徒弟。”

那和尚打了个寒噤咬着指头道:“这般一个丑头怪脑的好招他做徒弟?”三藏道:“你看不出来哩丑自丑甚是有用。”

那和尚只得同三藏与行者进了山门。山门里。又见那正殿上书四个大字是观音禅院。三藏又大喜道:“弟子屡感菩萨圣恩未及叩谢。今遇禅院就如见菩萨一般甚好拜谢。”那和尚闻言即命道人开了殿门请三藏朝拜。那行者拴了马丢了行李同三藏上殿。三藏展背舒身铺胸纳地望金象叩头。那和尚便去打鼓行者就去撞钟。三藏俯伏台前倾心祷祝。祝拜已毕那和尚住了鼓行者还只管撞钟不歇或紧或慢撞了许久那道人道:“拜已毕了还撞钟怎么?”行者方丢了钟杵笑道:“你那里晓得我这是做一日和尚撞一日钟的。”此时却惊动那寺里大小僧人、上下房长老听得钟声乱响一齐拥出道:“那个野人在这里乱敲钟鼓?”行者跳将出来咄的一声道:

“是你孙外公撞了耍子的!”那些和尚一见了唬得跌跌滚滚都爬在地下道:“雷公爷爷!”行者道:“雷公是我的重孙儿哩!

起来起来不要怕我们是东土大唐来的老爷。”众僧方才礼拜见了三藏都才放心不怕。内有本寺院主请道:“老爷们到后方丈中奉茶。”遂而解缰牵马抬了行李转过正殿径入后房序了坐次。

那院主献了茶又安排斋供。天光尚早三藏称谢未毕只见那后面有两个小童搀着一个老僧出来。看他怎生打扮:头上戴一顶毗卢方帽猫睛石的宝顶光辉;身上穿一领锦绒褊衫翡翠毛的金边晃亮。一对僧鞋攒八宝一根拄杖嵌云星。满面皱痕好似骊山老母;一双昏眼却如东海龙君。口不关风因齿落腰驼背屈为筋挛。众僧道:“师祖来了。”三藏躬身施礼迎接道:“老院主弟子拜揖。”那老僧还了礼又各叙坐。老僧道:

“适间小的们说东土唐朝来的老爷我才出来奉见。”三藏道:

“轻造宝山不知好歹恕罪恕罪!”老僧道:“不敢不敢!”因问:

“老爷东土到此有多少路程?”三藏道:“出长安边界有五千余里;过两界山收了一个小徒一路来行过西番哈咇国经两个月又有五六千里才到了贵处。”老僧道:“也有万里之遥了。我弟子虚度一生山门也不曾出去诚所谓坐井观天樗朽之辈。”三藏又问:“老院主高寿几何?”老僧道:“痴长二百七十岁了。”行者听见道:“这还是我万代孙儿哩?”三藏瞅了他一眼道:“谨言!莫要不识高低冲撞人。”那和尚便问:老爷你有多少年纪了?”行者道;“不敢说。”那老僧也只当一句疯话便不介意也不再回只叫献茶。有一个小幸童拿出一个羊脂玉的盘儿有三个法蓝镶金的茶锺;又一童提一把白铜壶儿斟了三杯香茶。真个是色欺榴蕊艳味胜桂花香。三藏见了夸爱不尽道:“好物件!好物件!真是美食美器!”那老僧道:“污眼污眼!老爷乃天朝上国广览奇珍似这般器具何足过奖?老爷自上邦来可有甚么宝贝借与弟子一观?”三藏道:“可怜!

我那东土无甚宝贝就有时路程遥远也不能带得。”行者在旁道:“师父我前日在包袱里曾见那领袈裟不是件宝贝?拿与他看看如何?”众僧听说袈裟一个个冷笑。行者道:“你笑怎的?”院主道:“老爷才说袈裟是件宝贝言实可笑。若说袈裟似我等辈者不止二三十件;若论我师祖在此处做了二百五六十年和尚足有七八百件!”叫:“拿出来看看。”那老和尚也是他一时卖弄便叫道人开库房头陀抬柜子就抬出十二柜放在天井中开了锁两边设下衣架四围牵了绳子将袈裟一件件抖开挂起请三藏观看。果然是满堂绮绣四壁绫罗!行者一一观之都是些穿花纳锦刺绣销金之物笑道:“好好好收起收起!把我们的也取出来看看。”三藏把行者扯住悄悄的道:“徒弟莫要与人斗富。你我是单身在外只恐有错。”

行者道:“看看袈裟有何差错?”三藏道:“你不曾理会得古人有云珍奇玩好之物不可使见贪婪奸伪之人。倘若一经入目必动其心;既动其心必生其计。汝是个畏祸的索之而必应其求可也;不然则殒身灭命皆起于此事不小矣。”行者道:“放心放心!都在老孙身上!”你看他不由分说急急的走了去把个包袱解开早有霞光迸迸尚有两层油纸裹定去了纸取出袈裟!抖开时红光满室彩气盈庭。众僧见了无一个不心欢口赞。真个好袈裟!上头有:千般巧妙明珠坠万样稀奇佛宝攒。上下龙须铺彩绮兜罗四面锦沿边。体挂魍魉从此灭身披魑魅入黄泉。托化天仙亲手制不是真僧不敢穿。

那老和尚见了这般宝贝果然动了奸心走上前对三藏跪下眼中垂泪道:“我弟子真是没缘!”三藏搀起道:“老院师有何话说?”他道:“老爷这件宝贝方才展开天色晚了奈何眼目昏花不能看得明白岂不是无缘!”三藏教:“掌上灯来让你再看。”那老僧道:“爷爷的宝贝已是光亮再点了灯一晃眼莫想看得仔细。”行者道:“你要怎的看才好?”老僧道:

“老爷若是宽恩放心教弟子拿到后房细细的看一夜明早送还老爷西去不知尊意何如?”三藏听说吃了一惊埋怨行者道:“都是你!都是你!”行者笑道:“怕他怎的?等我包起来教他拿了去看。但有疏虞尽是老孙管整。”那三藏阻当不住他把袈裟递与老僧道:“凭你看去只是明早照旧还我不得损污些须。”老僧喜喜欢欢着幸童将袈裟拿进去却吩咐众僧将前面禅堂扫净取两张藤床安设铺盖请二位老爷安歇;一壁厢又教安排明早斋送行遂而各散。师徒们关了禅堂睡下不题。

却说那和尚把袈裟骗到手拿在后房灯下对袈裟号啕痛哭慌得那本寺僧不敢先睡。小幸童也不知为何却去报与众僧道:“公公哭到二更时候还不歇声。”有两个徒孙是他心爱之人上前问道:“师公你哭怎的?”老僧道:“我哭无缘看不得唐僧宝贝!”小和尚道:“公公年纪高大过了他的袈裟放在你面前你只消解开看便罢了何须痛哭?”老僧道:“看的不长久。我今年二百七十岁空挣了几百件袈裟怎么得有他这一件?怎么得做个唐僧?”小和尚道:“师公差了。唐僧乃是离乡背井的一个行脚僧。你这等年高享用也彀了倒要象他做行脚僧何也?”老僧道:“我虽是坐家自在乐乎晚景却不得他这袈裟穿穿。若教我穿得一日儿就死也闭眼也是我来阳世间为僧一场!”众僧道:“好没正经!你要穿他的有何难处?

我们明日留他住一日你就穿他一日留他住十日你就穿他十日便罢了。何苦这般痛哭?”老僧道:“纵然留他住了半载也只穿得半载到底也不得气长。他要去时只得与他去怎生留得长远?”

正说话处有一个小和尚名唤广智出头道:“公公要得长远也容易。mianhuatang.la [棉花糖小说网]”老僧闻言就欢喜起来道:“我儿你有甚么高见?”广智道:“那唐僧两个是走路的人辛苦之甚如今已睡着了。我们想几个有力量的拿了枪刀打开禅堂将他杀了把尸埋在后园只我一家知道却又谋了他的白马、行囊却把那袈裟留下以为传家之宝岂非子孙长久之计耶?”老和尚见说满心欢喜却才揩了眼泪道:“好!好!好!此计绝妙!”即便收拾枪刀。内中又有一个小和尚名唤广谋就是那广智的师弟上前来道:“此计不妙。若要杀他须要看看动静。那个白脸的似易那个毛脸的似难。万一杀他不得却不反招己祸?

我有一个不动刀枪之法不知你尊意如何?”老僧道:“我儿你有何法?”广谋道:“依小孙之见如今唤聚东山大小房头每人要干柴一束舍了那三间禅堂放起火来教他欲走无门连马一火焚之。就是山前山后人家看见只说是他自不小心走了火将我禅堂都烧了。那两个和尚却不都烧死?又好掩人耳目。袈裟岂不是我们传家之宝?”那些和尚闻言无不欢喜都道:“强!强!强!此计更妙!更妙!”遂教各房头搬柴来。唉!

这一计正是弄得个高寿老僧该尽命观音禅院化为尘!原来他那寺里有七八十个房头大小有二百余众。当夜一拥搬柴把个禅堂前前后后四面围绕不通安排放火不题。

却说三藏师徒安歇已定。那行者却是个灵猴虽然睡下只是存神炼气朦胧着醒眼。忽听得外面不住的人走揸揸的柴响风生他心疑惑道:“此时夜静如何有人行得脚步之声?

莫敢是贼盗谋害我们的?”他就一骨鲁跳起欲要开门出看又恐惊醒师父。你看他弄个精神摇身一变变做一个蜜蜂儿真个是:口甜尾毒腰细身轻。穿花度柳飞如箭粘絮寻香似落星。小小微躯能负重嚣嚣薄翅会乘风。却自椽棱下钻出看分明。只见那众僧们搬柴运草已围住禅堂放火哩。行者暗笑道:“果依我师父之言他要害我们性命谋我的袈裟故起这等毒心。我待要拿棍打他啊可怜又不禁打一顿棍都打死了师父又怪我行凶。罢罢罢!与他个顺手牵羊将计就计教他住不成罢!”好行者一筋斗跳上南天门里唬得个庞刘苟毕躬身马赵温关控背俱道:“不好了!不好了!那闹天宫的主子又来了!”行者摇着手道:“列位免礼休惊我来寻广目天王的。”说不了却遇天王早到迎着行者道:“久阔久阔。前闻得观音菩萨来见玉帝借了四值功曹、六丁六甲并揭谛等保护唐僧往西天取经去说你与他做了徒弟今日怎么得闲到此?”行者道:“且休叙阔。唐僧路遇歹人放火烧他事在万分紧急特来寻你借辟火罩儿救他一救。快些拿来使使即刻返上。”天王道:“你差了既是歹人放火只该借水救他如何要辟火罩?”行者道:“你那里晓得就里。借水救之却烧不起来倒相应了他;只是借此罩护住了唐僧无伤其余管他尽他烧去快些快些!此时恐已无及莫误了我下边干事!”那天王笑道:“这猴子还是这等起不善之心只顾了自家就不管别人。”

行者道:“快着快着莫要调嘴害了大事!”那天王不敢不借遂将罩儿递与行者。

行者拿了按着云头径到禅堂房脊上罩住了唐僧与白马、行李他却去那后面老和尚住的方丈房上头坐着意保护那袈裟。看那些人放起火来他转捻诀念咒望巽地上吸一口气吹将去一阵风起把那火转刮得烘烘乱着。好火!好火!但见:黑烟漠漠红焰腾腾。黑烟漠漠长空不见一天星;红焰腾腾大地有光千里赤。起初时灼灼金蛇;次后来威威血马。南方三炁逞英雄回禄大神施法力。燥干柴烧烈火性说甚么燧人钻木;熟油门前飘彩焰赛过了老祖开炉。正是那无情火怎禁这有意行凶不去弭灾反行助虐。风随火势焰飞有千丈余高;火趁风威灰迸上九霄云外。乒乒乓乓好便似残年爆竹;泼泼喇喇却就如军中炮声。烧得那当场佛象莫能逃东院伽蓝无处躲。胜如赤壁夜鏖兵赛过阿房宫内火!这正是星星之火能烧万顷之田。须臾间风狂火盛把一座观音院处处通红。你看那众和尚搬箱抬笼抢桌端锅满院里叫苦连天。

孙行者护住了后边方丈辟火罩罩住了前面禅堂其余前后火光大真个是照天红焰辉煌透壁金光照耀!

不期火起之时惊动了一山兽怪。这观音院正南二十里远近有座黑风山山中有一个黑风洞洞中有一个妖精正在睡醒翻身只见那窗门透亮只道是天明。起来看时却是正北下的火光晃亮妖精大惊道:“呀!这必是观音院里失了火!这些和尚好不小心!我看时与他救一救来。”好妖精纵起云头即至烟火之下果然冲天之火前面殿宇皆空两廊烟火方灼。他大拽步撞将进去正呼唤叫取水来只见那后房无火房脊上有一人放风。他却情知如此急入里面看时见那方丈中间有些霞光彩气台案上有一个青毡包袱。他解开一看见是一领锦襕袈裟乃佛门之异宝。正是财动人心他也不救火他也不叫水拿着那袈裟趁哄打劫拽回云步径转东山而去。

那场火只烧到五更天明方才灭息。你看那众僧们赤赤精精啼啼哭哭都去那灰内寻铜铁拨腐炭扑金银。有的在墙筐里苫搭窝棚;有的赤壁根头支锅造饭。叫冤叫屈乱嚷乱闹不题。

却说行者取了辟火罩一筋斗送上南天门交与广目天王道:“谢借!谢借!”天王收了道:“大圣至诚了。我正愁你不还我的宝贝无处寻讨且喜就送来也。”行者道:“老孙可是那当面骗物之人?这叫做好借好还再借不难。”天王道:“许久不面请到宫少坐一时何如?”行者道:“老孙比在前不同烂板凳高谈阔论了;如今保唐僧不得身闲。容叙!容叙!”急辞别坠云又见那太阳星上径来到禅堂前摇身一变变做个蜜蜂儿飞将进去现了本象看时那师父还沉睡哩。行者叫道:“师父天亮了起来罢。”三藏才醒觉翻身道:“正是。”穿了衣服开门出来忽抬头只见些倒壁红墙不见了楼台殿宇大惊道:

“呀!怎么这殿宇俱无?都是红墙何也?”行者道:“你还做梦哩!今夜走了火的。”三藏道:“我怎不知?”行者道:“是老孙护了禅堂见师父浓睡不曾惊动。”三藏道:“你有本事护了禅堂如何就不救别房之火?”行者笑道:“好教师父得知。果然依你昨日之言他爱上我们的袈裟算计要烧杀我们。若不是老孙知觉到如今皆成灰骨矣!”三藏闻言害怕道:“是他们放的火么?”行者道:“不是他是谁?”三藏道:“莫不是怠慢了你你干的这个勾当?”行者道:“老孙是这等惫懒之人干这等不良之事?实实是他家放的。老孙见他心毒果是不曾与他救火只是与他略略助些风的。”三藏道:“天那!天那!火起时只该助水怎转助风?”行者道:“你可知古人云人没伤虎心虎没伤人意。他不弄火我怎肯弄风?”三藏道:“袈裟何在?敢莫是烧坏了也?”行者道:“没事!没事!烧不坏!那放袈裟的方丈无火。”三藏恨道:“我不管你!但是有些儿伤损我只把那话儿念动念动你就是死了!”行者慌了道:“师父莫念!莫念!管寻还你袈裟就是了。等我去拿来走路。”三藏才牵着马行者挑了担出了禅堂径往后方丈去。

却说那些和尚正悲切间忽的看见他师徒牵马挑担而来唬得一个个魂飞魄散道:“冤魂索命来了!”行者喝道:“甚么冤魂索命?快还我袈裟来!”众僧一齐跪倒叩头道:“爷爷呀!

冤有冤家债有债主。要索命不干我们事都是广谋与老和尚定计害你的莫问我们讨命。”行者咄的一声道:“我把你这些该死的畜生!那个问你讨甚么命!只拿袈裟来还我走路!”其间有两个胆量大的和尚道:“老爷你们在禅堂里已烧死了如今又来讨袈裟端的还是人是鬼?”行者笑道:“这伙孽畜!那里有甚么火来?你去前面看看禅堂再来说话!”众僧们爬起来往前观看那禅堂外面的门窗槅扇更不曾燎灼了半分。众人悚惧才认得三藏是位神僧行者是尊护法一齐上前叩头道:

“我等有眼无珠不识真人下界!你的袈裟在后面方丈中老师祖处哩。”三藏行过了三五层败壁破墙嗟叹不已。只见方丈果然无火众僧抢入里面叫道:“公公!唐僧乃是神人未曾烧死如今反害了自己家当!趁早拿出袈裟还他去也。”

原来这老和尚寻不见袈裟又烧了本寺的房屋正在万分烦恼焦燥之处一闻此言怎敢答应?因寻思无计进退无方拽开步躬着腰往那墙上着实撞了一头可怜只撞得脑破血流魂魄散咽喉气断染红沙!有诗为证诗曰:堪叹老衲性愚蒙枉作人间一寿翁。欲得袈裟传远世岂知佛宝不凡同!但将容易为长久定是萧条取败功。广智广谋成甚用?损人利己一场空!慌得个众僧哭道:“师公已撞杀了又不见袈裟怎生是好?”行者道:“想是汝等盗藏起也!都出来!开具花名手本等老孙逐一查点!”那上下房的院主将本寺和尚、头陀、幸童、道人尽行开具手本二张大小人等共计二百三十名。行者请师父高坐他却一一从头唱名搜检都要解放衣襟分明点过更无袈裟。又将那各房头搬抢出去的箱笼物件从头细细寻遍那里得有踪迹。三藏心中烦恼懊恨行者不尽却坐在上面念动那咒。行者扑的跌倒在地抱着头十分难禁只教“莫念!

莫念!管寻还了袈裟!”那众僧见了一个个战兢兢的上前跪下劝解三藏才合口不念。行者一骨鲁跳起来耳朵里掣出铁棒要打那些和尚被三藏喝住道:“这猴头!你头痛还不怕还要无礼?休动手!且莫伤人!再与我审问一问!”众僧们磕头礼拜哀告三藏道:“老爷饶命!我等委实的不曾看见。这都是那老死鬼的不是。他昨晚看着你的袈裟只哭到更深时候看也不曾敢看思量要图长久做个传家之宝设计定策要烧杀老爷。自火起之候狂风大作各人只顾救火搬抢物件更不知袈裟去向。”

行者大怒走进方丈屋里把那触死鬼尸抬出选剥了细看浑身更无那件宝贝就把个方丈掘地三尺也无踪影。行者忖量半晌问道:“你这里可有甚么妖怪成精么?”院主道:

“老爷不问莫想得知。我这里正东南有座黑风山黑风洞内有一个黑大王。我这老死鬼常与他讲道他便是个妖精。别无甚物。”行者道:“那山离此有多远近?”院主道:“只有二十里那望见山头的就是。”行者笑道:“师父放心不须讲了一定是那黑怪偷去无疑。”三藏道:“他那厢离此有二十里如何就断得是他?”行者道:“你不曾见夜间那火光腾万里亮透三天且休说二十里就是二百里也照见了!坐定是他见火光焜耀趁着机会暗暗的来到这里看见我们袈裟是件宝贝必然趁哄掳去也。等老孙去寻他一寻。”三藏道:“你去了时我却何倚?”

行者道:“这个放心暗中自有神灵保护明中等我叫那些和尚伏侍。”即唤众和尚过来道:“汝等着几个去埋那老鬼着几个伏侍我师父看守我白马!”众僧领诺。行者又道:“汝等莫顺口儿答应等我去了你就不来奉承。看师父的要怡颜悦色;养白马的要水草调匀。假有一毫儿差了照依这个样棍与你们看看!”他掣出棍子照那火烧的砖墙扑的一下把那墙打得粉碎又震倒了有七八层墙。众僧见了个个骨软身麻跪着磕头滴泪道:“爷爷宽心前去我等竭力虔心供奉老爷决不敢一毫怠慢!”好行者急纵筋斗云径上黑风山寻找这袈裟。正是那:金禅求正出京畿仗锡投西涉翠微。虎豹狼虫行处有工商士客见时稀。路逢异国愚僧妒全仗齐天大圣威。火风生禅院废黑熊夜盗锦襕衣。毕竟此去不知袈裟有无吉凶如何且听下回分解。

\chapter[孙行者大闹黑风山\ 观世音收伏熊罴怪]{孙行者大闹黑风山\\观世音收伏熊罴怪}

第十七回 孙行者大闹黑风山 观世音收伏熊罴怪

话说孙行者一筋斗跳将起去唬得那观音院大小和尚并头陀、幸童、道人等一个个朝天礼拜道:“爷爷呀!原来是腾云驾雾的神圣下界怪道火不能伤!恨我那个不识人的老剥皮使心用心今日反害了自己!”三藏道:“列位请起不须恨了。

这去寻着袈裟万事皆休;但恐找寻不着我那徒弟性子有些不好汝等性命不知如何恐一人不能脱也。”众僧闻得此言一个个提心吊胆告天许愿只要寻得袈裟各全性命不题。

却说孙大圣到空中把腰儿扭了一扭早来到黑风山上。

住了云头仔细看果然是座好山。况正值春光时节但见:万壑争流千崖竞秀。鸟啼人不见花落树犹香。雨过天连青壁润风来松卷翠屏张。山草野花开悬崖峭嶂;薛萝生佳木丽峻岭平岗。不遇幽人那寻樵子?涧边双鹤饮石上野猿狂。

矗矗堆螺排黛色巍巍拥翠弄岚光。那行者正观山景忽听得芳草坡前有人言语。他却轻步潜踪闪在那石崖之下偷睛观看。原来是三个妖魔席地而坐:上的是一条黑汉左下是一个道人右下是一个白衣秀士都在那里高谈阔论。讲的是立鼎安炉持砂炼汞白雪黄芽旁门外道。正说中间那黑汉笑道:“后日是我母难之日二公可光顾光顾?”白衣秀士道:

年年与大王上寺今年岂有不来之理?”黑汉道:“我夜来得了一件宝贝名唤锦襕佛衣诚然是件玩好之物。我明日就以他为寿大开筵宴邀请各山道官庆贺佛衣就称为佛衣会如何?”道人笑道:“妙!妙!妙!我明日先来拜寿后日再来赴宴。”

行者闻得佛衣之言定以为是他宝贝他就忍不住怒气跳出石崖双手举起金箍棒高叫道:“我把你这伙贼怪!你偷了我的袈裟要做甚么佛衣会!趁早儿将来还我!”喝一声“休走!”

轮起棒照头一下慌得那黑汉化风而逃道人驾云而走只把个白衣秀士一棒打死拖将过来看处却是一条白花蛇怪。索性提起来捽做五七断径入深山找寻那个黑汉。转过尖峰抹过峻岭又见那壁陡崖前耸出一座洞府但见那:烟霞渺渺松柏森森。烟霞渺渺采盈门松柏森森青绕户。桥踏枯槎木峰巅绕薛萝。鸟衔红蕊来云壑鹿践芳丛上石台。那门前时催花风送花香。临堤绿柳转黄鹂傍岸夭桃翻粉蝶。虽然旷野不堪夸却赛蓬莱山下景。

行者到于门又见那两扇石门关得甚紧门上有一横石板明书六个大字乃“黑风山黑风洞”即便轮棒叫声“开门!”那里面有把门的小妖开了门出来问道:“你是何人敢来击吾仙洞?”行者骂道:“你个作死的孽畜!甚么个去处敢称仙洞!仙字是你称的?快进去报与你那黑汉教他快送老爷的袈裟出来饶你一窝性命!”小妖急急跑到里面报道:“大王!

佛衣会做不成了!门外有一个毛脸雷公嘴的和尚来讨袈裟哩!”那黑汉被行者在芳草坡前赶将来却才关了门坐还未稳又听得那话心中暗想道:“这厮不知是那里来的这般无礼他敢嚷上我的门来!”教:“取披挂!”随结束了绰一杆黑缨枪走出门来。这行者闪在门外执着铁棒睁睛观看只见那怪果生得凶险:碗子铁盔火漆光乌金铠甲亮辉煌。皂罗袍罩风兜袖黑绿丝绦軃穗长。手执黑缨枪一杆足踏乌皮靴一双。

眼幌金睛如掣电正是山中黑风王。行者暗笑道:“这厮真个如烧窑的一般筑煤的无二!想必是在此处刷炭为生怎么这等一身乌黑?”那怪厉声高叫道:“你是个甚么和尚敢在我这里大胆?”行者执铁棒撞至面前大咤一声道:“不要闲讲!快还你老外公的袈裟来!”那怪道:“你是那寺里和尚?你的袈裟在那里失落了敢来我这里索取?”行者道:“我的袈裟在直北观音院后方丈里放着。只因那院里失了火你这厮趁哄掳掠盗了来要做佛衣会庆寿怎敢抵赖?快快还我饶你性命!若牙迸半个不字我推倒了黑风山躧平了黑风洞把你这一洞妖邪都碾为齑粉!”那怪闻言呵呵冷笑道:“你这个泼物!原来昨夜那火就是你放的!你在那方丈屋上行凶招风是我把一件袈裟拿来了你待怎么!你是那里来的?姓甚名谁?有多大手段敢那等海口浪言!”行者道:“是你也认不得你老外公哩!

你老外公乃大唐上国驾前御弟三藏法师之徒弟姓孙名悟空行者。若问老孙的手段说出来教你魂飞魄散死在眼前!”那怪道:“我不曾会你有甚么手段说来我听。”行者笑道:“我儿子你站稳着仔细听了!我:自小神通手段高随风变化逞英豪。养性修真熬日月跳出轮回把命逃。一点诚心曾访道灵台山上采药苗。那山有个老仙长寿年十万八千高。老孙拜他为师父指我长生路一条。他说身内有丹药外边采取枉徒劳。

得传大品天仙诀若无根本实难熬。回光内照宁心坐身中日月坎离交。万事不思全寡欲六根清净体坚牢。返老还童容易得凡入圣路非遥。三年无漏成仙体不同俗辈受煎熬。十洲三岛还游戏海角天涯转一遭。活该三百多余岁不得飞升上九霄。下海降龙真宝贝才有金箍棒一条。花果山前为帅水帘洞里聚群妖。玉皇大帝传宣诏封我齐天极品高。几番大闹灵霄殿数次曾偷王母桃。天兵十万来降我层层密密布枪刀。战退天王归上界哪吒负痛领兵逃。显圣真君能变化老孙硬赌跌平交。道祖观音同玉帝南天门上看降妖。却被老君助一阵二郎擒我到天曹。将身绑在降妖柱即命神兵把枭。

刀砍锤敲不得坏又教雷打火来烧。老孙其实有手段全然不怕半分毫。送在老君炉里炼六丁神火慢煎熬。日满开炉我跳出手持铁棒绕天跑。纵横到处无遮挡三十三天闹一遭。我佛如来施法力五行山压老孙腰。整整压该五百载幸逢三藏出唐朝。吾今皈正西方去转上雷音见玉毫。你去乾坤四海问一问我是历代驰名第一妖!”

那怪闻言笑道:“你原来是那闹天宫的弼马温么?”行者最恼的是人叫他弼马温听见这一声心中大怒骂道:“你这贼怪!偷了袈裟不还倒伤老爷!不要走!看棍!”那黑汉侧身躲过绰长枪劈手来迎。两家这场好杀:如意棒黑缨枪二人洞口逞刚强。分心劈脸刺着臂照头伤。这个横丢阴棍手那个直拈急三枪。白虎爬山来探爪黄龙卧道转身忙。喷彩雾吐毫光两个妖仙不可量:一个是修正齐天圣一个是成精黑大王。这场山里相争处只为袈裟各不良。那怪与行者斗了十数回合不分胜负。渐渐红日当午那黑汉举枪架住铁棒道:“孙行者我两个且收兵等我进了膳来再与你赌斗。”行者道:

“你这个孽畜教做汉子?好汉子半日儿就要吃饭?似老孙在山根下整压了五百余年也未曾尝些汤水那里便饿哩?莫推故休走!还我袈裟来方让你去吃饭!”那怪虚幌一枪撤身入洞关了石门收回小怪且安排筵宴书写请帖邀请各山魔王庆会不题。

却说行者攻门不开也只得回观音院。那本寺僧人已葬埋了那老和尚都在方丈里伏侍唐僧。早斋已毕又摆上午斋正那里添汤换水只见行者从空降下众僧礼拜接入方丈见了三藏。三藏道:“悟空你来了袈裟如何?”行者道:“已有了根由。早是不曾冤了这些和尚原来是那黑风山妖怪偷了。老孙去暗暗的寻他只见他与一个白衣秀士一个老道人坐在那芳草坡前讲话。也是个不打自招的怪物他忽然说出道:后日是他母难之日邀请诸邪来做生日夜来得了一件锦襕佛衣要以此为寿作一大宴唤做庆赏佛衣会。是老孙抢到面前打了一棍那黑汉化风而走。道人也不见了只把个白衣秀士打死乃是一条白花蛇成精。我又急急赶到他洞口叫他出来与他赌斗。他已承认了是他拿回。战彀这半日不分胜负。那怪回洞却要吃饭关了石门惧战不出。老孙却来回看师父先报此信已是有了袈裟的下落不怕他不还我。”众僧闻言合掌的合掌磕头的磕头都念声“南无阿弥陀佛!今日寻着下落我等方有了性命矣!”行者道:“你且休喜欢畅快我还未曾到手师父还未曾出门哩。只等有了袈裟打得我师父好好的出门才是你们的安乐处;若稍有些须不虞老孙可是好惹的主子!可曾有好茶饭与我师父吃?可曾有好草料喂马?”众僧俱满口答应道:“有!有!有!更不曾一毫有怠慢了老爷。”

三藏道:“自你去了这半日我已吃过了三次茶汤两餐斋供了他俱不曾敢慢我。但只是你还尽心竭力去寻取袈裟回来。”

行者道:“莫忙!既有下落管情拿住这厮还你原物。放心放心!”

正说处那上房院主又整治素供请孙老爷吃斋。行者却吃了些须复驾祥云又去找寻。正行间只见一个小怪左胁下夹着一个花梨木匣儿从大路而来。行者度他匣内必有甚么柬札举起棒劈头一下可怜不禁打就打得似个肉饼一般却拖在路旁揭开匣儿观看果然是一封请帖。帖上写着:“侍生熊罴顿拜启上大阐金池老上人丹房:屡承佳惠感激渊深。夜观回禄之难有失救护谅仙机必无他害。生偶得佛衣一件欲作雅会谨具花酌奉扳清赏。至期千乞仙驾过临一叙。是荷。先二日具。”行者见了呵呵大笑道:“那个老剥皮死得他一毫儿也不亏!他原来与妖精结党!怪道他也活了二百七十岁。想是那个妖精传他些甚么服气的小法儿故有此寿。老孙还记得他的模样等我就变做那和尚往他洞里走走看我那袈裟放在何处。假若得手即便拿回却也省力。”

好大圣念动咒语迎着风一变果然就象那老和尚一般藏了铁棒拽开步径来洞口叫声开门。那小妖开了门见是这般模样急转身报道:“大王金池长老来了。”那怪大惊道:

“刚才差了小的去下简帖请他这时候还未到那里哩如何他就来得这等迅?想是小的不曾撞着他断是孙行者呼他来讨袈裟的。管事的可把佛衣藏了莫教他看见。”行者进了前门但见那天井中松篁交翠桃李争妍丛丛花簇簇兰香却也是个洞天之处。又见那二门上有一联对子写着:“静隐深山无俗虑幽居仙洞乐天真。”行者暗道:“这厮也是个脱垢离尘、知命的怪物。”入门里往前又进到于三层门里都是些画栋雕梁明窗彩户。只见那黑汉子穿的是黑绿纻丝袢袄罩一领鸦青花绫披风戴一顶乌角软巾穿一双麂皮皂靴见行者进来整顿衣巾降阶迎接道:“金池老友连日欠亲。请坐请坐。”行者以礼相见见毕而坐坐定而茶。茶罢妖精欠身道:

“适有小简奉启后日一叙何老友今日就下顾也?”行者道:

“正来进拜不期路遇华翰见有佛衣雅会故此急急奔来愿求见见。”那怪笑道:“老友差矣。这袈裟本是唐僧的他在你处住札你岂不曾看见反来就我看看?”行者道:“贫僧借来因夜晚还不曾展看不期被大王取来又被火烧了荒山失落了家私。那唐僧的徒弟又有些骁勇乱忙中四下里都寻觅不见。原来是大王的洪福收来故特来一见。”

正讲处只见有一个巡山的小妖来报道:“大王!祸事了!

下请书的小校被孙行者打死在大路旁边他绰着经儿变化做金池长老来骗佛衣也!”那怪闻言暗道:“我说那长老怎么今日就来又来得迅果然是他!”急纵身拿过枪来就刺行者。行者耳朵里急掣出棍子现了本相架住枪尖就在他那中厅里跳出自天井中斗到前门外唬得那洞里群魔都丧胆家间老幼尽无魂。这场在山头好赌斗比前番更是不同。好杀:

那猴王胆大充和尚这黑汉心灵隐佛衣。语去言来机会巧随机应变不差池。袈裟欲见无由见宝贝玄微真妙微。小怪寻山言祸事老妖怒显神威。翻身打出黑风洞枪棒争持辨是非。

棒架长枪声响亮枪迎铁棒放光辉。悟空变化人间少妖怪神通世上稀。这个要把佛衣来庆寿那个不得袈裟肯善归?这番苦战难分手就是活佛临凡也解不得围。他两个从洞口打上山头自山头杀在云外吐雾喷风飞砂走石只斗到红日沉西不分胜败。那怪道:“姓孙的你且住了手。今日天晚不好相持。你去你去!待明早来与你定个死活。”行者叫道:“儿子莫走!要战便象个战的不可以天晚相推。”看他没头没脸的只情使棍子打来这黑汉又化阵清风转回本洞紧闭石门不出。

行者却无计策奈何只得也回观音院里按落云头道声“师父”。那三藏眼儿巴巴的正望他哩忽见到了面前甚喜;

又见他手里没有袈裟又惧。问道:“怎么这番还不曾有袈裟来?”行者袖中取出个简帖儿来递与三藏道:“师父那怪物与这死的老剥皮原是朋友。他着一个小妖送此帖来还请他去赴佛衣会。是老孙就把那小妖打死变做那老和尚进他洞去骗了一钟茶吃欲问他讨袈裟看看他不肯拿出。正坐间忽被一个甚么巡山的走了风信他就与我打将起来。只斗到这早晚不分上下。他见天晚闪回洞去紧闭石门。老孙无奈也暂回来。”三藏道:“你手段比他何如?”行者道:“我也硬不多儿只战个手平。”三藏才看了简帖又递与那院主道:“你师父敢莫也是妖精么?”那院主慌忙跪下道:“老爷我师父是人。只因那黑大王修成*人道常来寺里与我师父讲经他传了我师父些养神服气之术故以朋友相称。”行者道:“这伙和尚没甚妖气他一个个头圆顶天足方履地但比老孙肥胖长大些儿非妖精也。你看那帖儿上写着侍生熊罴此物必定是个黑熊成精。”三藏道:“我闻得古人云熊与猩猩相类都是兽类他却怎么成精?”行者笑道:“老孙是兽类见做了齐天大圣与他何异?大抵世间之物凡有九窍者皆可以修行成仙。”三藏又道:

“你才说他本事与你手平你却怎生得胜取我袈裟回来?”行者道:“莫管莫管我有处治。”

正商议间众僧摆上晚斋请他师徒们吃了。三藏教掌灯仍去前面禅堂安歇。众僧都挨墙倚壁苫搭窝棚各各睡下只把个后方丈让与那上下院主安身。此时夜静但见:银河现影玉宇无尘。满天星灿烂一水浪收痕。万籁声宁千山鸟绝。溪边渔火息塔上佛灯昏。昨夜阇黎钟鼓响今宵一遍哭声闻。

是夜在禅堂歇宿。那三藏想着袈裟那里得稳睡?忽翻身见窗外透白急起叫道:“悟空天明了快寻袈裟去。”行者一骨鲁跳将起来早见众僧侍立供奉汤水行者道:“你等用心伏侍我师父老孙去也。”三藏下床扯住道:“你往那里去?”行者道“我想这桩事都是观音菩萨没理他有这个禅院在此受了这里人家香火又容那妖精邻住。我去南海寻他与他讲一讲教他亲来问妖精讨袈裟还我。”三藏道:“你这去几时回来?”行者道:“时少只在饭罢时多只在晌午就成功了。那些和尚可好伏侍老孙去也。”说声去早已无踪。须臾间到了南海停云观看但见那:汪洋海远水势连天。祥光笼宇宙瑞气照山川。千层雪浪吼青霄万迭烟波滔白昼。水飞四野浪滚周遭。水飞四野振轰雷浪滚周遭鸣霹雳。休言水势且看中间。五色朦胧宝迭山红黄紫皂绿和蓝。才见观音真胜境试看南海落伽山。好去处!山峰高耸顶透虚空。中间有千样奇花百般瑞草。风摇宝树日映金莲。观音殿瓦盖琉璃潮音洞门铺玳瑁。绿杨影里语鹦哥紫竹林中啼孔雀。罗纹石上护法威严;玛瑙滩前木叉雄壮。这行者观不尽那异景非常径直按云头到竹林之下。早有诸天迎接道:“菩萨前者对众言大圣归善甚是宣扬。今保唐僧如何得暇到此?”行者道:“因保唐僧路逢一事特见菩萨烦为通报。”诸天遂来洞口报知。菩萨唤入行者遵法而行至宝莲台下拜了。菩萨问曰:“你来何干?”行者道:“我师父路遇你的禅院你受了人间香火容一个黑熊精在那里邻住着他偷了我师父袈裟屡次取讨不与今特来问你要的。”菩萨道:“这猴子说话这等无状!既是熊精偷了你的袈裟你怎来问我取讨?都是你这个孽猴大胆将宝贝卖弄拿与小人看见你却又行凶唤风火烧了我的留云下院反来我处放刁!”行者见菩萨说出这话知他晓得过去未来之事慌忙礼拜道:“菩萨乞恕弟子之罪果是这般这等。但恨那怪物不肯与我袈裟师父又要念那话儿咒语老孙忍不得头疼故此来拜烦菩萨。望菩萨慈悲之心助我去拿那妖精取衣西进也。”菩萨道:“那怪物有许多神通却也不亚于你。也罢我看唐僧面上和你去走一遭。”行者闻言谢恩再拜。即请菩萨出门遂同驾祥云早到黑风山坠落云头依路找洞。

正行处只见那山坡前走出一个道人手拿着一个玻璃盘儿盘内安着两粒仙丹往前正走被行者撞个满怀掣出棒就照头一下打得脑里浆流出腔中血进撺。菩萨大惊道:

“你这个猴子还是这等放泼!他又不曾偷你袈裟又不与你相识又无甚冤仇你怎么就将他打死?”行者道:“菩萨你认他不得。他是那黑熊精的朋友。他昨日和一个白衣秀士都在芳草坡前坐讲。后日是黑精的生日请他们来庆佛衣会。今日他先来拜寿明日来庆佛衣会所以我认得定是今日替那妖去上寿。”菩萨说:“既是这等说来也罢。”行者才去把那道人提起来看却是一只苍狼。旁边那个盘儿底下却有字刻道:凌虚子制。行者见了笑道:“造化!造化!”老孙也是便益菩萨也是省力。这怪叫做不打自招那怪教他今日了劣。”菩萨说道:

“悟空这教怎么说?”行者道:“菩萨我悟空有一句话儿叫做将计就计不知菩萨可肯依我?”菩萨道:“你说。”行者说道:

“菩萨你看这盘儿中是两粒仙丹便是我们与那妖魔的贽见;

这盘儿后面刻的四个字说凌虚子制便是我们与那妖魔的勾头。菩萨若要依得我时我好替你作个计较也就不须动得干戈也不须劳得征战妖魔眼下遭瘟佛衣眼下出现;菩萨要不依我时菩萨往西我悟空往东佛衣只当相送唐三藏只当落空。”菩萨笑道:“这猴熟嘴!”行者道:“不敢倒是一个计较。”

菩萨说:“你这计较怎说?”行者道:这盘上刻那凌虚子制想这道人就叫做凌虚子。菩萨你要依我时可就变做这个道人我把这丹吃了一粒变上一粒略大些儿。菩萨你就捧了这个盘儿两颗仙丹去与那妖上寿把这丸大些的让与那妖。待那妖一口吞之老孙便于中取事他若不肯献出佛衣老孙将他肚肠就也织将一件出来。”

菩萨没法只得也点点头儿。行者笑道:“如何?”尔时菩萨乃以广大慈悲无边法力亿万化身以心会意以意会身恍惚之间变作凌虚仙子:鹤氅仙风飒飘飖欲步虚。苍颜松柏老秀色古今无。去去还无住如如自有殊。总来归一法只是隔邪躯。行者看道:“妙啊!妙啊!还是妖精菩萨还是菩萨妖精?”菩萨笑道:“悟空菩萨妖精总是一念。若论本来皆属无有。”行者心下顿悟转身却就变做一粒仙丹:走盘无不定圆明未有方。三三勾漏合六六少翁商。瓦铄黄金焰牟尼白昼光。外边铅与汞未许易论量。行者变了那颗丹终是略大些儿。菩萨认定拿了那个玻璃盘儿径到妖洞门口看时果然是:崖深岫险云生岭上;柏苍松翠风飒林间。崖深岫险果是妖邪出没人烟少;柏苍松翠也可仙真修隐道情多。山有涧涧有泉潺潺流水咽鸣琴便堪洗耳;崖有鹿林有鹤幽幽仙籁动间岑亦可赏心。这是妖仙有分降菩提弘誓无边垂恻隐。菩萨看了心中暗喜道:“这孽畜占了这座山洞却是也有些道分。”因此心中已是有个慈悲。

走到洞口只见守洞小妖都有些认得道:凌虚仙长来了。”一边传报一边接引。那妖早已迎出二门道:“凌虚有劳仙驾珍顾蓬荜有辉。”菩萨道:“小道敬献一粒仙丹敢称千寿。”他二人拜毕方才坐定又叙起他昨日之事。菩萨不答连忙拿丹盘道:“大王且见小道鄙意。”觑定一粒大的推与那妖道:“愿大王千寿!”那妖亦推一粒递与菩萨道:“愿与凌虚子同之。”让毕那妖才待要咽那药顺口儿一直滚下。现了本相理起四平那妖滚倒在地。菩萨现相问妖取了佛衣行者早已从鼻孔中出去。菩萨又怕那妖无礼却把一个箍儿丢在那妖头上。那妖起来提枪要刺行者、菩萨早已起在空中菩萨将真言念起。那怪依旧头疼丢了枪满地乱滚。半空里笑倒个美猴王平地下滚坏个黑熊怪。菩萨道:“孽畜!你如今可皈依么?”那怪满口道:“心愿皈依只望饶命!”行者恐耽搁了工夫意欲就打菩萨急止住道:“休伤他命我有用他处哩。”行者道:“这样怪物不打死他反留他在何处用哩?”菩萨道:“我那落伽山后无人看管我要带他去做个守山大神。”行者笑道:

“诚然是个救苦慈尊一灵不损。若是老孙有这样咒语就念上他娘千遍!这回儿就有许多黑熊都教他了帐!”却说那怪苏醒多时公道难禁疼痛只得跪在地下哀告道:“但饶性命愿皈正果!”菩萨方坠落祥光又与他摩顶受戒教他执了长枪跟随左右。那黑熊才一片野心今日定无穷顽性此时收。菩萨吩咐道:“悟空你回去罢。好生伏侍唐僧以后再休懈惰生事。”

行者道:“深感菩萨远来弟子还当回送回送。”菩萨道:“免送。”行者才捧着袈裟叩头而别。菩萨亦带了熊罴径回大海。

有诗为证诗曰:祥光霭霭凝金象万道缤纷实可夸。普济世人垂悯恤遍观法界现金莲。今来多为传经意此去原无落点瑕。

降怪成真归大海空门复得锦袈裟。毕竟不知向后事情如何且听下回分解。

\chapter[观音院唐僧脱难\ 高老庄大圣降魔]{观音院唐僧脱难\\高老庄大圣降魔}

第十八回 观音院唐僧脱难 高老庄行者降魔

行者辞了菩萨按落云头将袈裟挂在香楠树上掣出棒来打入黑风洞里。那洞里那得一个小妖?原来是他见菩萨出现降得那老怪就地打滚急急都散走了。行者一行凶将他那几层门上都积了干柴前前后后一齐火把个黑风洞烧做个红风洞却拿了袈裟驾祥光转回直北。

话说那三藏望行者急忙不来心甚疑惑不知是请菩萨不至不知是行者托故而逃正在那胡猜乱想之中只见半空中彩雾灿灿行者忽坠阶前叫道:“师父袈裟来了。”三藏大喜众僧亦无不欢悦道:“好了!好了!我等性命今日方才得全了。”三藏接了袈裟道:“悟空你早间去时原约到饭罢晌午如何此时日西方回?”行者将那请菩萨施变化降妖的事情备陈了一遍三藏闻言遂设香案朝南礼拜罢道:“徒弟啊既然有了佛衣可快收拾包裹去也。”行者道:“莫忙莫忙。今日将晚不是走路的时候且待明日早行。”众僧们一齐跪下道:

“孙老爷说得是。一则天晚二来我等有些愿心儿今幸平安有了宝贝待我还了愿请老爷散了福明早再送西行。”行者道:“正是正是。”你看那些和尚都倾囊倒底把那火里抢出的余资各出所有整顿了些斋供烧了些平安无事的纸念了几卷消灾解厄的经。当晚事毕。

次早方刷扮了马匹包裹了行囊出门。众僧远送方回。行者引路而去正是那春融时节但见那:草衬玉骢蹄迹软柳摇金线露华新。桃杏满林争艳丽薜萝绕径放精神。沙堤日暖鸳鸯睡山涧花香蛱蝶驯。这般秋去冬残春过半不知何年行满得真文。师徒们行了五七日荒路忽一日天色将晚远远的望见一村人家。三藏道:“悟空你看那壁厢有座山庄相近我们去告宿一宵明日再行何如?”行者道:“且等老孙去看看吉凶再作区处。”那师父挽住丝缰这行者定睛观看真个是:竹篱密密茅屋重重。参天野树迎门曲水溪桥映户。道旁杨柳绿依依园内花开香馥馥。此时那夕照沉西处处山林喧鸟雀;晚烟出爨条条道径转牛羊。又见那食饱鸡豚眠屋角醉酣邻叟唱歌来。行者看罢道:“师父请行定是一村好人家正可借宿。”那长老催动白马早到街衢之口。又见一个少年头裹绵布身穿蓝袄持伞背包敛裩扎裤脚踏着一双三耳草鞋雄纠纠的出街忙步。行者顺手一把扯住道:“那里去?我问你一个信儿:此间是甚么地方?”那个人只管苦挣口里嚷道:“我庄上没人只是我好回信?”行者陪着笑道:“施主莫恼与人方便自己方便。你就与我说说地名何害?我也可解得你的烦恼。”那人挣不脱手气得乱跳道:“蹭蹬!蹭蹬!家长的屈气受不了又撞着这个光头受他的清气!”行者道:“你有本事劈开我的手你便就去了也罢。”那人左扭右扭那里扭得动却似一把铁钤拑住一般气得他丢了包袱撇了伞两只手雨点似来抓行者。行者把一只手扶着行李一只手抵住那人凭他怎么支吾只是不能抓着。行者愈加不放急得爆燥如雷。三藏道:“悟空那里不有人来了?你再问那人就是只管扯住他怎的?放他去罢。”行者笑道:“师父不知若是问了别人没趣须是问他才有买卖。”那人被行者扯住不过只得说出道:“此处乃是乌斯藏国界之地唤做高老庄。一庄人家有大半姓高故此唤做高老庄。你放了我去罢。”行者又道:“你这样行装不是个走近路的。你实与我说你要往那里去端的所干何事我才放你。”这人无奈只得以实情告诉道:“我是高太公的家人名叫高才。我那太公有一个女儿年方二十岁更不曾配人三年前被一个妖精占了。那妖整做了这三年女婿我太公不悦说道女儿招了妖精不是长法一则败坏家门二则没个亲家来往一向要退这妖精。那妖精那里肯退转把女儿关在他后宅将有半年再不放出与家内人相见。我太公与了我几两银子教我寻访法师拿那妖怪。我这些时不曾住脚前前后后请了有三四个人都是不济的和尚脓包的道士降不得那妖精。刚才骂了我一场说我不会干事又与了我五钱银子做盘缠教我再去请好法师降他。不期撞着你这个纥刺星扯住误了我走路故此里外受气我无奈才与你叫喊。不想你又有些拿法我挣不过你所以说此实情。你放我走罢。”行者道:“你的造化我有营生这才是凑四合六的勾当。你也不须远行莫要化费了银子。我们不是那不济的和尚脓包的道士其实有些手段惯会拿妖。这正是一来照顾郎中二来又医得眼好烦你回去上复你那家主说我们是东土驾下差来的御弟圣僧往西天拜佛求经者善能降妖缚怪。”高才道:“你莫误了我。我是一肚子气的人你若哄了我没甚手段拿不住那妖精却不又带累我来受气?”行者道:“管教不误了你。你引我到你家门去来。”那人也无计奈何真个提着包袱拿了伞转步回身领他师徒到于门道:“二位长老你且在马台上略坐坐等我进去报主人知道。”行者才放了手落担牵马师徒们坐立门旁等候。

那高才入了大门径往中堂上走可可的撞见高太公。太公骂道:“你那个蛮皮畜生怎么不去寻人又回来做甚?”高才放下包伞道:“上告主人公得知小人才行出街口忽撞见两个和尚:一个骑马一个挑担。他扯住我不放问我那里去。我再三不曾与他说及他缠得没奈何不得脱手遂将主人公的事情一一说与他知。他却十分欢喜要与我们拿那妖怪哩。”高老道:“是那里来的?”高才道:“他说是东土驾下差来的御弟圣僧前往西天拜佛求经的。”太公道:“既是远来的和尚怕不真有些手段。他如今在那里?”高才道:“现在门外等候。”那太公即忙换了衣服与高才出来迎接叫声“长老”。三藏听见急转身早已到了面前。那老者戴一顶乌绫巾穿一领葱白蜀锦衣踏一双糙米皮的犊子靴系一条黑绿绦子出来笑语相迎便叫:“二位长老作揖了。”三藏还了礼行者站着不动。那老者见他相貌凶丑便就不敢与他作揖。行者道:“怎么不唱老孙喏?”那老儿有几分害怕叫高才道:“你这小厮却不弄杀我也?

家里现有一个丑头怪脑的女婿打不开怎么又引这个雷公来害我?”行者道:“老高你空长了许大年纪还不省事!若专以相貌取人干净错了。我老孙丑自丑却有些本事替你家擒得妖精捉得鬼魅拿住你那女婿还了你女儿便是好事何必谆谆以相貌为言!”太公见说战兢兢的只得强打精神叫声“请进”。这行者见请才牵了白马教高才挑着行李与三藏进去。他也不管好歹就把马拴在敞厅柱上扯过一张退光漆交椅叫三藏坐下。他又扯过一张椅子坐在旁边。那高老道:

“这个小长老倒也家怀。”行者道:“你若肯留我住得半年还家怀哩。”

坐定高老问道:“适间小价说二位长老是东土来的?”三藏道:“便是。贫僧奉朝命往西天拜佛求经因过宝庄特借一宿明日早行。”高老道:“二位原是借宿的怎么说会拿怪?”行者道:“因是借宿顺便拿几个妖怪儿耍耍的。动问府上有多少妖怪?”高老道:“天哪!还吃得有多少哩!只这一个妖怪女婿已彀他磨慌了!”行者道:“你把那妖怪的始末有多大手段从头儿说说我听我好替你拿他。”高老道:“我们这庄上自古至今也不晓得有甚么鬼祟魍魉邪魔作耗。只是老拙不幸不曾有子止生三个女儿:大的唤名香兰第二的名玉兰第三的名翠兰。那两个从小儿配与本庄人家止有小的个要招个女婿指望他与我同家过活做个养老女婿撑门抵户做活当差。不期三年前有一个汉子模样儿倒也精致他说是福陵山上人家姓猪上无父母下无兄弟愿与人家做个女婿。我老拙见是这般一个无羁无绊的人就招了他。一进门时倒也勤谨:耕田耙地不用牛具;收割田禾不用刀杖。昏去明来其实也好只是一件有些会变嘴脸。”行者道:“怎么变么?”高老道:“初来时是一条黑胖汉后来就变做一个长嘴大耳朵的呆子脑后又有一溜鬃毛身体粗糙怕人头脸就象个猪的模样。食肠却又甚大:一顿要吃三五斗米饭早间点心也得百十个烧饼才彀。喜得还吃斋素若再吃荤酒便是老拙这些家业田产之类不上半年就吃个罄净!”三藏道:“只因他做得所以吃得。”高老道:“吃还是件小事他如今又会弄风云来雾去走石飞砂唬得我一家并左邻右舍俱不得安生。又把那翠兰小女关在后宅子里一半年也不曾见面更不知死活如何。因此知他是个妖怪要请个法师与他去退去退。”行者道:“这个何难?老儿你管放心今夜管情与你拿住教他写了退亲文书还你女儿如何?”高老大喜道:“我为招了他不打紧坏了我多少清名疏了我多少亲眷。但得拿住他要甚么文书?就烦与我除了根罢。”行者道:“容易容易!入夜之时就见好歹。”

老儿十分欢喜才教展抹桌椅摆列斋供。斋罢将晚老儿问道:“要甚兵器?要多少人随?趁早好备。”行者道:“兵器我自有。”老儿道:“二位只是那根锡杖锡杖怎么打得妖精?”行者随于耳内取出一个绣花针来捻在手中迎风幌了一幌就是碗来粗细的一根金箍铁棒对着高老道:“你看这条棍子比你家兵器如何?可打得这怪否?”高老又道:“既有兵器可要人跟?”行者道:“我不用人只是要几个年高有德的老儿陪我师父清坐闲叙我好撇他而去。等我把那妖精拿来对众取供替你除了根罢。”那老儿即唤家僮请了几个亲故朋友。一时都到相见已毕行者道:“师父你放心稳坐老孙去也。”

你看他揝着铁棒扯着高老道:“你引我去后宅子里妖精的住处看看。”高老遂引他到后宅门行者道:“你去取钥匙来。”高老道:“你且看看若是用得钥匙却不请你了。”行者笑道:“你那老儿年纪虽大却不识耍。我把这话儿哄你一哄你就当真。”走上前摸了一摸原来是铜汁灌的锁子。狠得他将金箍棒一捣捣开门扇里面却黑洞洞的。行者道:“老高你去叫你女儿一声看他可在里面。”那老儿硬着胆叫道:“三姐姐!”那女儿认得是他父亲的声音才少气无力的应了一声道:

“爹爹我在这里哩。”行者闪金睛向黑影里仔细看时你道他怎生模样?但见那:云鬓乱堆无掠玉容未洗尘淄。一片兰心依旧十分娇态倾颓。樱唇全无气血腰肢屈屈偎偎。愁蹙蹙蛾眉淡瘦怯怯语声低。他走来看见高老一把扯住抱头大哭。行者道:“且莫哭!且莫哭”!我问你妖怪往那里去了?”

女子道:“不知往那里走。这些时天明就去入夜方来云云雾雾往回不知何所。因是晓得父亲要祛退他他也常常防备故此昏来朝去。”行者道:“不消说了老儿你带令爱往前边宅里慢慢的叙阔让老孙在此等他。他若不来你却莫怪;他若来了定与你剪草除根。”那老高欢欢喜喜的把女儿带将前去。

行者却弄神通摇身一变变得就如那女子一般独自个坐在房里等那妖精。不多时一阵风来真个是走石飞砂。好风:起初时微微荡荡向后来渺渺茫茫。微微荡荡乾坤大渺渺茫茫无阻碍。凋花折柳胜揌麻倒树摧林如拔菜。翻江搅海鬼神愁裂石崩山天地怪。衔花糜鹿失来踪摘果猿猴迷在外。七层铁塔侵佛头八面幢幡伤宝盖。金梁玉柱起根摇房上瓦飞如燕块。举棹梢公许愿心开船忙把猪羊赛。当坊土地弃祠堂四海龙王朝上拜。海边撞损夜叉船长城刮倒半边塞。那阵狂风过处只见半空里来了一个妖精果然生得丑陋:黑脸短毛长喙大耳穿一领青不青、蓝不蓝的梭布直裰系一条花布手巾。行者暗笑道:“原来是这个买卖!”好行者却不迎他也不问他且睡在床上推病口里哼哼喷喷的不绝。那怪不识真假走进房一把搂住就要亲嘴。行者暗笑道:“真个要来弄老孙哩!”即使个拿法托着那怪的长嘴叫做个小跌。漫头一料扑的掼下床来。那怪爬起来扶着床边道:“姐姐你怎么今日有些怪我?想是我来得迟了?”行者道:“不怪!不怪!”那妖道:

“既不怪我怎么就丢我这一跌?”行者道:“你怎么就这等样小家子就搂我亲嘴?我因今日有些不自在若每常好时便起来开门等你了。你可脱了衣服睡是。”那怪不解其意真个就去脱衣。行者跳起来坐在净桶上。那怪依旧复来床上摸一把摸不着人叫道:“姐姐你往那里去了?请脱衣服睡罢。”行者道:

“你先睡等我出个恭来”那怪果先解衣上床。行者忽然叹口气道声“造化低了!”那怪道:“你恼怎的?造化怎么得低的?我得到了你家虽是吃了些茶饭却也不曾白吃你的:我也曾替你家扫地通沟搬砖运瓦筑土打墙耕田耙地种麦插秧创家立业。如今你身上穿的锦戴的金四时有花果享用八节有蔬菜烹煎你还有那些儿不趁心处这般短叹长吁说甚么造化低了?”行者道:“不是这等说。今日我的父母隔着墙丢砖料瓦的甚是打我骂我哩。”那怪道:“他打骂你怎的?”行者道:

“他说我和你做了夫妻你是他门下一个女婿全没些儿礼体。

这样个丑嘴脸的人又会不得姨夫又见不得亲戚又不知你云来雾去端的是那里人家姓甚名谁败坏他清德玷辱他门风故此这般打骂所以烦恼。”那怪道:“我虽是有些儿丑陋若要俊却也不难。我一来时曾与他讲过他愿意方才招我今日怎么又说起这话!我家住在福陵山云栈洞。我以相貌为姓故姓猪官名叫做猪刚鬣。他若再来问你你就以此话与他说便了。”行者暗喜道:“那怪却也老实不用动刑就供得这等明白。既有了地方姓名不管怎的也拿住他。”行者道:“他要请法师来拿你哩。”那怪笑道:“睡着!睡着!莫睬他!我有天罡数的变化九齿的钉钯怕甚么法师、和尚、道士?就是你老子有虔心请下九天荡魔祖师下界我也曾与他做过相识他也不敢怎的我。”行者道:“他说请一个五百年前大闹天宫姓孙的齐天大圣要来拿你哩。”那怪闻得这个名头就有三分害怕道:“既是这等说我去了罢两口子做不成了。”行者道:“你怎的就去?”那怪道:“你不知道那闹天宫的弼马温有些本事只恐我弄他不过低了名头不象模样。”他套上衣服开了门往外就走被行者一把扯住将自己脸上抹了一抹现出原身喝道:“好妖怪那里走!你抬头看看我是那个?”那怪转过眼来看见行者咨牙俫嘴火眼金睛磕头毛脸就是个活雷公相似慌得他手麻脚软划剌的一声挣破了衣服化狂风脱身而去。行者急上前掣铁棒望风打了一下。那怪化万道火光径转本山而去。行者驾云随后赶来叫声:“那里走!你若上天我就赶到斗牛宫!你若入地我就追至枉死狱!”咦!毕竟不知这一去赶至何方有何胜败且听下回分解。

\chapter[云栈洞悟空收八戒\ 浮屠山玄奘受心经]{云栈洞悟空收八戒\\浮屠山玄奘受心经}
\chapter[黄风岭唐僧有难\ 半山中八戒争先]{黄风岭唐僧有难\\半山中八戒争先}
\chapter[护法设庄留大圣\ 须弥灵吉定风魔]{护法设庄留大圣\\须弥灵吉定风魔}

第二十一回 护法设庄留大圣 须弥灵吉定风魔

却说那五十个败残的小妖拿着些破旗破鼓撞入洞里报道:“大王虎先锋战不过那毛脸和尚被他赶下东山坡去了。”老妖闻说十分烦恼正低头不语默思计策又有把前门的小妖道:“大王虎先锋被那毛脸和尚打杀了拖在门口骂战哩。”那老妖闻言愈加烦恼道:“这厮却也无知!我倒不曾吃他师父他转打杀我家先锋可恨!可恨!”叫:“取披挂来。我也只闻得讲甚么孙行者等我出去看是个甚么九头八尾的和尚拿他进来与我虎先锋对命。”众小妖急急抬出披挂。老妖结束齐整绰一杆三股钢叉帅群妖跳出本洞。那大圣停立门外见那怪走将出来着实骁勇。看他怎生打扮但见:金盔晃日金甲凝光。盔上缨飘山雉尾罗袍罩甲淡鹅黄。勒甲绦盘龙耀彩护心镜绕眼辉煌。鹿皮靴槐花染色;锦围裙柳叶绒妆。手持三股钢叉利不亚当年显圣郎。

那老妖出得门来厉声高叫道:“那个是孙行者?”这行者脚躧着虎怪的皮囊手执着如意的铁棒答道:“你孙外公在此送出我师父来!”那怪仔细观看见行者身躯鄙猥面容羸瘦不满四尺笑道:“可怜!可怜!我只道是怎么样扳翻不倒的好汉原来是这般一个骷髅的病鬼!”行者笑道:“你这个儿子忒没眼色!你外公虽是小小的你若肯照头打一叉柄就长三尺。”那怪道:“你硬着头吃吾一柄。”大圣公然不惧。那怪果打一下来他把腰躬一躬足长了三尺有一丈长短慌得那妖把钢叉按住喝道:“孙行者你怎么把这护身的变化法儿拿来我门前使唤!莫弄虚头!走上来我与你见见手段!”行者笑道:“儿子啊!常言道:留情不举手举手不留情。你外公手儿重重的只怕你捱不起这一棒!”那怪那容分说拈转钢叉望行者当胸就刺。这大圣正是会家不忙忙家不会理开铁棒使一个乌龙掠地势拨开钢叉又照头便打。他二人在那黄风洞口这一场好杀:妖王怒大圣施威。妖王怒要拿行者抵先锋;大圣施威欲捉精灵救长老。叉来棒架棒去叉迎。一个是镇山都总帅一个是护法美猴王。初时还在尘埃战后来各起在中央。点钢叉尖明锐利;如意棒身黑箍黄。戳着的魂归冥府打着的定见阎王。全凭着手疾必须要力壮身强。两家舍死忘生战不知那个平安那个伤!

那老妖与大圣斗经三十回合不分胜败。这行者要见功绩使一个身外身的手段:把毫毛揪下一把用口嚼得粉碎望上一喷叫声“变!”变有百十个行者都是一样打扮各执一根铁棒把那怪围在空中。那怪害怕也使一般本事:急回头望着巽地上把口张了三张嘑的一口气吹将出去忽然间一阵黄风从空刮起。好风!真个利害:冷冷飕飕天地变无影无形黄沙旋。穿林折岭倒松梅播土扬尘崩岭坫。黄河浪泼彻底浑湘江水涌翻波转。碧天振动斗牛宫争些刮倒森罗殿。五百罗汉闹喧天八大金刚齐嚷乱。文殊走了青毛狮普贤白象难寻见。真武龟蛇失了群梓橦骡子飘其韂。行商喊叫告苍天梢公拜许诸般愿。烟波性命浪中流名利残生随水办。仙山洞府黑攸攸海岛蓬莱昏暗暗。老君难顾炼丹炉寿星收了龙须扇。

王母正去赴蟠桃一风吹断裙腰钏。二郎迷失灌州城哪吒难取匣中剑。天王不见手心塔鲁班吊了金头钻。雷音宝阙倒三层赵州石桥崩两断。一轮红日荡无光满天星斗皆昏乱。南山鸟往北山飞东湖水向西湖漫。雌雄拆对不相呼子母分离难叫唤。龙王遍海找夜叉雷公到处寻闪电。十代阎王觅判官地府牛头追马面。这风吹倒普陀山卷起观音经一卷。白莲花卸海边飞欢倒菩萨十二院。盘古至今曾见风不似这风来不善。唿喇喇乾坤险不炸崩开万里江山都是颤!那妖怪使出这阵狂风就把孙大圣毫毛变的小行者刮得在那半空中却似纺车儿一般乱转莫想轮得棒如何拢得身?慌得行者将毫毛一抖收上身来独自个举着铁棒上前来打又被那怪劈脸喷了一口黄风把两只火眼金睛刮得紧紧闭合莫能睁开因此难使铁棒遂败下阵来。那妖收风回洞不题。

却说猪八戒见那黄风大作天地无光牵着马守着担伏在山凹之间也不敢睁眼不敢抬头口里不住的念佛许愿又不知行者胜负何如师父死活何如。正在那疑思之时却早风定天晴忽抬头往那洞门前看处却也不见兵戈不闻锣鼓。呆子又不敢上他门又没人看守马匹、行李果是进退两难怆惶不已。忧虑间只听得孙大圣从西边吆喝而来他才欠身迎着道:“哥哥好大风啊!你从那里走来?”行者摆手道:“利害!利害!我老孙自为人不曾见这大风。那老妖使一柄三股钢叉来与老孙交战战到有三十余合是老孙使一个身外身的本事把他围打他甚着急故弄出这阵风来果是凶恶刮得我站立不住收了本事冒风而逃。哏好风!哏好风!老孙也会呼风也会唤雨不曾似这个妖精的风恶!”八戒道:“师兄那妖精的武艺如何?”行者道:“也看得过叉法儿倒也齐整与老孙也战个手平。却只是风恶了难得赢他。”八戒道:“似这般怎生救得师父?”行者道:“救师父且等再处不知这里可有眼科先生且教他把我眼医治医治。”八戒道:“你眼怎的来?”行者道:“我被那怪一口风喷将来吹得我眼珠酸痛这会子冷泪常流。”八戒道:“哥啊这半山中天色又晚且莫说要甚么眼科连宿处也没有了!”行者道:“要宿处不难。我料着那妖精还不敢伤我师父我们且找上大路寻个人家住下过此一宵明日天光再来降妖罢。”八戒道:“正是正是。”

他却牵了马挑了担出山凹行上路口。此时渐渐黄昏只听得那路南山坡下有犬吠之声。二人停身观看乃是一家庄院影影的有灯火光明。他两个也不管有路无路漫草而行直至那家门但见:紫芝翳翳白石苍苍。紫芝翳翳多青草白石苍苍半绿苔。数点小萤光灼灼一林野树密排排。香兰馥郁嫩竹新栽。清泉流曲涧古柏倚深崖。地僻更无游客到门前惟有野花开。他两个不敢擅入只得叫一声:“开门开门!”

那里有一老者带几个年幼的农夫叉钯扫帚齐来问道:“甚么人?甚么人?”行者躬身道:“我们是东土大唐圣僧的徒弟因往西方拜佛求经路过此山被黄风大王拿了我师父去了我们还未曾救得。天色已晚特来府上告借一宵万望方便方便。”那老者答礼道:“失迎失迎。此间乃云多人少之处却才闻得叫门恐怕是妖狐老虎及山中强盗等类故此小介愚顽多有冲撞不知是二位长老。请进请进。”他兄弟们牵马挑担而入径至里边拴马歇担与庄老拜见叙坐。又有苍头献茶茶罢捧出几碗胡麻饭。饭毕命设铺就寝行者道:“不睡还可敢问善人贵地可有卖眼药的?”老者道:“是那位长老害眼?”

行者道:“不瞒你老人家说我们出家人自来无病从不晓得害眼。”老人道:“既不害眼如何讨药?”行者道:“我们今日在黄风洞口救我师父不期被那怪将一口风喷来吹得我眼珠酸痛。今有些眼泪汪汪故此要寻眼药。”那老者道:“善哉!善哉!

你这个长老小小的年纪怎么说谎?那黄风大圣风最利害。他那风比不得甚么春秋风、松竹风与那东西南北风。”八戒道:

“想必是夹脑风、羊耳风、大麻风、偏正头风?”长者道:“不是不是。(WWW.mianhuatang.la 好看的小说)他叫做三昧神风。”行者道:“怎见得?”老者道:“那风能吹天地暗善刮鬼神愁裂石崩崖恶吹人命即休。你们若遇着他那风吹了呵还想得活哩!只除是神仙方可得无事。”行者道:“果然!果然!我们虽不是神仙神仙还是我的晚辈这条命急切难休却只是吹得我眼珠酸痛!”那老者道:“既如此说也是个有来头的人。我这敝处却无卖眼药的老汉也有些迎风冷泪曾遇异人传了一方名唤三花九子膏能治一切风眼。”

行者闻言低头唱喏道:“愿求些儿点试点试。”那老者应承即走进去取出一个玛瑙石的小罐儿来拔开塞口用玉簪儿蘸出少许与行者点上教他不得睁开宁心睡觉明早就好。点毕收了石罐径领小介们退于里面。八戒解包袱展开铺盖请行者安置。行者闭着眼乱摸八戒笑道:“先生你的明杖儿呢?”行者道:“你这个馕糟的呆子!你照顾我做瞎子哩!”那呆子哑哑的暗笑而睡。行者坐在铺上转运神功直到有三更后方才睡下。

不觉又是五更将晓行者抹抹脸睁开眼道:“果然好药!

比常更有百分光明!”却转头后边望望呀!那里得甚房舍窗门但只见些老槐高柳兄弟们都睡在那绿莎茵上。那八戒醒来道:“哥哥你嚷怎的?”行者道:“你睁开眼看看。”呆子忽抬头见没了人家慌得一毂辘爬将起来道:“我的马哩?”行者道:“树上拴的不是?”“行李呢?”行者道:“你头边放的不是?”

八戒道:“这家子惫懒也。他搬了怎么就不叫我们一声?通得老猪知道也好与你送些茶果。想是躲门户的恐怕里长晓得却就连夜搬了。噫!我们也忒睡得死!怎么他家拆房子响也不听见响响?”行者吸吸的笑道:“呆子不要乱嚷你看那树上是个甚么纸帖儿。”八戒走上前用手揭了原来上面四句颂子云:“庄居非是俗人居护法伽蓝点化庐。妙药与君医眼痛尽心降怪莫踌躇。”行者道:“这伙强神自换了龙马一向不曾点他他倒又来弄虚头!”八戒道:“哥哥莫扯架子他怎么伏你点札?”行者道:“兄弟你还不知哩。这护教伽蓝、六丁六甲、五方揭谛、四值功曹奉菩萨的法旨暗保我师父者。自那日报了名只为这一向有了你再不曾用他们故不曾点札罢了。”八戒道:“哥哥他既奉法旨暗保师父所以不能现身明显故此点化仙庄。你莫怪他昨日也亏他与你点眼又亏他管了我们一顿斋饭亦可谓尽心矣。你莫怪他我们且去救师父来。”行者道:“兄弟说得是。此处到那黄风洞口不远。你且莫动身只在林子里看马守担等老孙去洞里打听打听看师父下落如何再与他争战。”八戒道:“正是这等讨一个死活的实信。假若师父死了各人好寻头干事;若是未死我们好竭力尽心。”行者道:“莫乱谈我去也!”

他将身一纵径到他门门尚关着睡觉。行者不叫门且不惊动妖怪捻着诀念个咒语摇身一变变做一个花脚蚊虫真个小巧!有诗为证诗曰:扰扰微形利喙嘤嘤声细如雷。

兰房纱帐善通随正爱炎天暖气。只怕熏烟扑扇偏怜灯火光辉。轻轻小小忒钻疾飞入妖精洞里。只见那把门的小妖正打鼾睡行者往他脸上叮了一口那小妖翻身醒了道:“我爷哑!好大蚊子!一口就叮了一个大疙疸!”忽睁眼道:“天亮了。”

又听得支的一声二门开了。行者嘤嘤的飞将进去只见那老妖吩咐各门上谨慎一壁厢收拾兵器:“只怕昨日那阵风不曾刮死孙行者他今日必定还来来时定教他一命休矣。”行者听说又飞过那厅堂径来后面。但见层门关得甚紧行者漫门缝儿钻将进去原来是个大空园子那壁厢定风桩上绳缠索绑着唐僧哩。那师父纷纷泪落心心只念着悟空、悟能不知都在何处。行者停翅叮在他光头上叫声“师父”。那长老认得他的声音道:“悟空啊想杀我也!你在那里叫我哩?”行者道:“师父我在你头上哩。你莫要心焦少得烦恼我们务必拿住妖精方才救得你的性命。”唐僧道:“徒弟啊几时才拿得妖精么?”行者道:“拿你的那虎怪已被八戒打死了只是老妖的风势利害。料着只在今日管取拿他。你放心莫哭我去哑。”

说声去嘤嘤的飞到前面只见那老妖坐在上面正点札各路头目。又见那洞前有一个小妖把个令字旗磨一磨撞上厅来报道:“大王小的巡山才出门见一个长嘴大耳朵的和尚坐在林里若不是我跑得快些几乎被他捉住。却不见昨日那个毛脸和尚。”老妖道:“孙行者不在想必是风吹死也再不便去那里求救兵去了!”众妖道:“大王若果吹杀了他是我们的造化只恐吹不死他他去请些神兵来却怎生是好?”老妖道:“怕他怎的怕那甚么神兵!若还定得我的风势只除了灵吉菩萨来是其余何足惧也!”行者在屋梁上只听得他这一句言语不胜欢喜即抽身飞出现本相来至林中叫声“兄弟!”

八戒道:“哥你往那里去来?刚才一个打令字旗的妖精被我赶了去也。”行者笑道:“亏你!亏你!老孙变做蚊虫儿进他洞去探看师父原来师父被他绑在定风桩上哭哩。是老孙吩咐教他莫哭又飞在屋梁上听了一听。只见那拿令字旗的喘嘘嘘的走进去报道:只是被你赶他却不见我。老妖乱猜乱说说老孙是风吹杀了又说是请神兵去了。他却自家供出一个人来甚妙!甚妙!”八戒道:“他供的是谁?”行者道:“他说怕甚么神兵那个能定他的风势!只除是灵吉菩萨来是。但不知灵吉住在何处?”

正商议处只见大路旁走出一个老公公来。你看他怎生模样:身健不扶拐杖冰髯雪鬓蓬蓬。金花耀眼意朦胧瘦骨衰筋强硬。屈背低头缓步庞眉赤脸如童。看他容貌是人称却似寿星出洞。八戒望见大喜道:“师兄常言道要知山下路须问去来人。你上前问他一声何如?”真个大圣藏了铁棒放下衣襟上前叫道:“老公公问讯了。”那老者半答不答的还了个礼道:“你是那里和尚?这旷野处有何事干?”行者道:“我们是取经的圣僧昨日在此失了师父特来动问公公一声灵吉菩萨在那里住?”老者道:“灵吉在直南上到那里还有二千里路。有一山呼名小须弥山。山中有个道场乃是菩萨讲经禅院。汝等是取他的经去了?”行者道:“不是取他的经我有一事烦他不知从那条路去。”老者用手向南指道:“这条羊肠路就是了。”哄得那孙大圣回头看路那公公化作清风寂然不见只是路旁边下一张简帖上有四句颂子云:“上复齐天大圣听老人乃是李长庚。须弥山有飞龙杖灵吉当年受佛兵。”行者执了帖儿转身下路。八戒道:“哥啊我们连日造化低了。这两日忏日里见鬼!那个化风去的老儿是谁?”行者把帖儿递与八戒念了一遍道:“李长庚是那个?”行者道:“是西方太白金星的名号。”八戒慌得望空下拜道:“恩人!恩人!老猪若不亏金星奏准玉帝呵性命也不知化作甚的了!”行者道:“兄弟你却也知感恩。但莫要出头只藏在这树林深处仔细看守行李、马匹等老孙寻须弥山请菩萨去耶。”八戒道:“晓得!晓得!你只管快快前去!老猪学得个乌龟法得缩头时且缩头。”

孙大圣跳在空中纵觔斗云径往直南上去果然快。他点头经过三千里扭腰八百有余程。须臾见一座高山半中间有祥云出现瑞霭纷纷山凹里果有一座禅院只听得钟磬悠扬又见那香烟缥缈。大圣直至门前见一道人项挂数珠口中念佛。行者道:“道人作揖。”那道人躬身答礼道:“那里来的老爷?”行者道:“这可是灵吉菩萨讲经处么?”道人道:“此间正是有何话说?”行者道:“累烦你老人家与我传答传答:我是东土大唐驾下御弟三藏法师的徒弟齐天大圣孙悟空行者。今有一事要见菩萨。”道人笑道:“老爷字多话多我不能全记。”行者道:“你只说是唐僧徒弟孙悟空来了。”道人依言上讲堂传报。那菩萨即穿袈裟添香迎接。

这大圣才举步入门往里观看只见那满堂锦绣一屋威严。众门人齐诵《法华经》老班轻敲金铸磬。佛前供养尽是仙果仙花;案上安排皆是素肴素品。辉煌宝烛条条金焰射虹霓;馥郁真香道道玉烟飞彩雾。正是那讲罢心闲方入定白云片片绕松梢。静收慧剑魔头绝般若波罗善会高。那菩萨整衣出迓行者登堂坐了客位随命看茶。行者道:“茶不劳赐但我师父在黄风山有难特请菩萨施大法力降怪救师。”菩萨道:“我受了如来法令在此镇押黄风怪。如来赐了我一颗定风丹一柄飞龙宝杖。当时被我拿住饶了他的性命放他去隐性归山不许伤生造孽不知他今日欲害令师有违教令我之罪也。”那菩萨欲留行者治斋相叙行者恳辞随取了飞龙杖与大圣一齐驾云。不多时至黄风山上。菩萨道:“大圣这妖怪有些怕我我只在云端里住定你下去与他索战诱他出来我好施法力。”行者依言按落云头不容分说掣铁棒把他洞门打破叫道:“妖怪还我师父来也!”慌得那把门小妖急忙传报。那怪道:“这泼猴着实无礼!再不伏善反打破我门!这一出去使阵神风定要吹死!”仍前披挂手绰钢叉又走出门来见了行者更不打话拈叉当胸就刺。大圣侧身躲过举棒对面相还。战不数合那怪吊回头望巽地上才待要张口呼风只见那半空里灵吉菩萨将飞龙宝杖丢将下来不知念了些甚么咒语却是一条八爪金龙拨喇的轮开两爪一把抓住妖精提着头两三捽捽在山石崖边现了本相却是一个黄毛貂鼠。行者赶上举棒就打被菩萨拦住道:“大圣莫伤他命我还要带他去见如来。”对行者道:“他本是灵山脚下的得道老鼠因为偷了琉璃盏内的清油灯火昏暗恐怕金刚拿他故此走了却在此处成精作怪。如来照见了他不该死罪故着我辖押但他伤生造孽拿上灵山;今又冲撞大圣陷害唐僧我拿他去见如来明正其罪才算这场功绩哩。”行者闻言却谢了菩萨。菩萨西归不题。

却说猪八戒在那林内正思量行者只听得山坂下叫声“悟能兄弟牵马挑担来耶。”那呆子认得是行者声音急收拾跑出林外见了行者道:“哥哥怎的干事来?”行者道:“请灵吉菩萨使一条飞龙杖拿住妖精原来是个黄毛貂鼠成精被他带去灵山见如来去了。我和你洞里去救师父。”那呆子才欢欢喜喜。二人撞入里面把那一窝狡兔、妖狐、香獐、角鹿一顿钉钯铁棒尽情打死却往后园拜救师父。师父出得门来问道:

“你两人怎生捉得妖精?如何方救得我?”行者将那请灵吉降妖的事情陈了一遍师父谢之不尽。他兄弟们把洞中素物安排些茶饭吃了方才出门找大路向西而去。毕竟不知向后如何且听下回分解。

\chapter[八戒大战流沙河\ 木叉奉法收悟净]{八戒大战流沙河\\木叉奉法收悟净}
\chapter[三藏不忘本\ 四圣试禅心]{三藏不忘本\\四圣试禅心}

第二十三回 三藏不忘本 四圣试禅心

诗曰:奉法西来道路赊秋风渐浙落霜花。乖猿牢锁绳休解劣马勤兜鞭莫加。木母金公原自合黄婆赤子本无差。咬开铁弹真消息般若波罗到彼家。这回书盖言取经之道不离乎一身务本之道也。却说他师徒四众了悟真如顿开尘锁自跳出性海流沙浑无挂碍径投大路西来。历遍了青山绿水看不尽野草闲花。真个也光阴迅又值九秋但见了些枫叶满山红黄花耐晚风。老蝉吟渐懒愁蟋思无穷。荷破青绔扇橙香金弹丛。可怜数行雁点点远排空。

正走处不觉天晚。三藏道:“徒弟如今天色又晚却往那里安歇?”行者道:“师父说话差了出家人餐风宿水卧月眠霜随处是家。又问那里安歇何也?”猪八戒道:“哥啊你只知道你走路轻省那里管别人累坠?自过了流沙河这一向爬山过岭身挑着重担老大难挨也!须是寻个人家一则化些茶饭二则养养精神才是个道理。”行者道:“呆子你这般言语似有报怨之心。还象在高老庄倚懒不求福的自在恐不能也。

既是秉正沙门须是要吃辛受苦才做得徒弟哩。”八戒道:“哥哥你看这担行李多重?”行者道:“兄弟自从有了你与沙僧我又不曾挑着那知多重?”八戒道:“哥啊你看看数儿么:四片黄藤蔑长短八条绳。又要防阴雨毡包三四层。匾担还愁滑两头钉上钉。铜镶铁打九环杖篾丝藤缠大斗篷。似这般许多行李难为老猪一个逐日家担着走偏你跟师父做徒弟拿我做长工!”行者笑道:“呆子你和谁说哩?”八戒道:“哥哥与你说哩。”行者道:“错和我说了。老孙只管师父好歹你与沙僧专管行李马匹。但若怠慢了些儿孤拐上先是一顿粗棍!”

八戒道:“哥啊不要说打打就是以力欺人。我晓得你的尊性高傲你是定不肯挑;但师父骑的马那般高大肥盛只驮着老和尚一个教他带几件儿也是弟兄之情。”行者道:“你说他是马哩!他不是凡马本是西海龙王敖闰之子唤名龙马三太子。

只因纵火烧了殿上明珠被他父亲告了忤逆身犯天条多亏观音菩萨救了他的性命他在那鹰愁陡涧久等师父又幸得菩萨亲临却将他退鳞去角摘了项下珠才变做这匹马愿驮师父往西天拜佛。这个都是各人的功果你莫攀他。”那沙僧闻言道:“哥哥真个是龙么?”行者道:“是龙。”八戒道:“哥啊我闻得古人云龙能喷云暧雾播土扬沙。有巴山捎岭的手段有翻江搅海的神通。怎么他今日这等慢慢而走?”行者道:“你要他快走我教他快走个儿你看。”好大圣把金箍棒揝一揝万道彩云生。那马看见拿棒恐怕打来慌得四只蹄疾如飞电飕的跑将去了。那师父手软勒不住尽他劣性奔上山崖才大达辿步走。师父喘息始定抬头远见一簇松阴内有几间房舍着实轩昂但见:门垂翠柏宅近青山。几株松冉冉数茎竹斑斑。

篱边野菊凝霜艳桥畔幽兰映水丹。粉泥墙壁砖砌围圜。高堂多壮丽大厦甚清安。牛羊不见无鸡犬想是秋收农事闲。

那师父正按辔徐观又见悟空兄弟方到。悟净道:“师父不曾跌下马来么?”长老骂道:“悟空这泼猴他把马儿惊了早是我还骑得住哩!”行者陪笑道:“师父莫骂我都是猪八戒说马行迟故此着他快些。”那呆子因赶马走急了些儿喘气嘘嘘口里唧唧哝哝的闹道:“罢了!罢了!见自肚别腰松担子沉重挑不上来又弄我奔奔波波的赶马!”长老道:“徒弟啊你且看那壁厢有一座庄院我们却好借宿去也。”行者闻言急抬头举目而看果见那半空中庆云笼罩瑞霭遮盈情知定是佛仙点化他却不敢泄漏天机只道:“好!好!好!我们借宿去来。”

长老连忙下马见一座门楼乃是垂莲象鼻画栋雕梁。沙僧歇了担子八戒牵了马匹道:“这个人家是过当的富实之家。”行者就要进去三藏道:“不可你我出家人各自避些嫌疑切莫擅入。且自等他有人出来以礼求宿方可。”八戒拴了马斜倚墙根之下三藏坐在石鼓上行者、沙僧坐在台基边。久无人出行者性急跳起身入门里看处:原来有向南的三间大厅帘栊高控。屏门上挂一轴寿山福海的横披画;两边金漆柱上贴着一幅大红纸的春联上写着:丝飘弱柳平桥晚雪点香梅小院春。正中间设一张退光黑漆的香几几上放一个古铜兽炉。

上有六张交椅两山头挂着四季吊屏。

行者正然偷看处忽听得后门内有脚步之声走出一个半老不老的妇人来娇声问道:“是甚么人擅入我寡妇之门?”慌得个大圣喏喏连声道:“小僧是东土大唐来的奉旨向西方拜佛求经。一行四众路过宝方天色已晚特奔老菩萨檀府告借一宵。”那妇人笑语相迎道:“长老那三位在那里?请来。”行者高声叫道:“师父请进来耶。”三藏才与八戒、沙僧牵马挑担而入只见那妇人出厅迎接。八戒饧眼偷看你道他怎生打扮:

穿一件织金官绿纻丝袄上罩着浅红比甲;系一条结彩鹅黄锦绣裙下映着高底花鞋。时样鬘髻皂纱漫相衬着二色盘龙;

宫样牙梳朱翠晃斜簪着两股赤金钗。云鬓半苍飞凤翅耳环双坠宝珠排。脂粉不施犹自美风流还似少年才。

那妇人见了他三众更加欣喜以礼邀入厅房一一相见礼毕请各叙坐看茶。那屏风后忽有一个丫髻垂丝的女童托着黄金盘、白玉盏香茶喷暖气异果散幽香。那人绰彩袖春笋纤长;擎玉盏传茶上奉。对他们一一拜了。茶毕又吩咐办斋。三藏启手道:“老菩萨高姓?贵地是甚地名?”妇人道:“此间乃西牛贺洲之地。小妇人娘家姓贾夫家姓莫。幼年不幸公姑早亡与丈夫守承祖业有家资万贯良田千顷。夫妻们命里无子止生了三个女孩儿前年大不幸又丧了丈夫小妇居孀今岁服满。空遗下田产家业再无个眷族亲人只是我娘女们承领。欲嫁他人又难舍家业。适承长老下降想是师徒四众。小妇娘女四人意欲坐山招夫四位恰好不知尊意肯否如何。”三藏闻言推聋妆哑瞑目宁心寂然不答。那妇人道:“舍下有水田三百余顷旱田三百余顷山场果木三百余顷;黄水牛有一千余只况骡马成群猪羊无数。东南西北庄堡草场共有六七十处。家下有八九年用不着的米谷十来年穿不着的绫罗;一生有使不着的金银胜强似那锦帐藏春说甚么金钗两行。你师徒们若肯回心转意招赘在寒家自自在在享用荣华却不强如往西劳碌?”那三藏也只是如痴如蠢默默无言。

那妇人道:“我是丁亥年三月初三日酉时生。故夫比我年大三岁我今年四十五岁。大女儿名真真今年二十岁;次女名爱爱今年十八岁;三小女名怜怜今年十六岁俱不曾许配人家。虽是小妇人丑陋却幸小女俱有几分颜色女工针指无所不会。因是先夫无子即把他们当儿子看养小时也曾教他读些儒书也都晓得些吟诗作对。虽然居住山庄也不是那十分粗俗之类料想也配得过列位长老若肯放开怀抱长留头与舍下做个家长穿绫着锦胜强如那瓦钵缁衣雪鞋云笠!”

三藏坐在上面好便似雷惊的孩子雨淋的虾蟆只是呆呆挣挣翻白眼儿打仰。那八戒闻得这般富贵这般美色他却心痒难挠坐在那椅子上一似针戳屁股左扭右扭的忍耐不住走上前扯了师父一把道:“师父!这娘子告诵你话你怎么佯佯不睬?好道也做个理会是。”那师父猛抬头咄的一声喝退了八戒道:“你这个孽畜!我们是个出家人岂以富贵动心美色留意成得个甚么道理!”那妇人笑道:“可怜!可怜!出家人有何好处?”三藏道:“女菩萨你在家人却有何好处?”那妇人道:“长老请坐等我把在家人好处说与你听。怎见得?有诗为证诗曰:春裁方胜着新罗夏换轻纱赏绿荷;秋有新蒭香糯酒冬来暖阁醉颜酡。四时受用般般有八节珍羞件件多;衬锦铺绫花烛夜强如行脚礼弥陀。”三藏道:“女菩萨你在家人享荣华受富贵有可穿有可吃儿女团圆果然是好。但不知我出家的人也有一段好处。怎见得?有诗为证诗曰:出家立志本非常推倒从前恩爱堂。外物不生闲口舌身中自有好阴阳。

功完行满朝金阙见性明心返故乡。胜似在家贪血食老来坠落臭皮囊。”

那妇人闻言大怒道:“这泼和尚无礼!我若不看你东土远来就该叱出。我倒是个真心实意要把家缘招赘汝等你倒反将言语伤我。你就是受了戒了愿永不还俗好道你手下人我家也招得一个。你怎么这般执法?”三藏见他怒只得者者谦谦叫道:“悟空你在这里罢。”行者道:“我从小儿不晓得干那般事教八戒在这里罢。”八戒道:“哥啊不要栽人么。

大家从长计较。”三藏道:“你两个不肯便教悟净在这里罢。”

沙僧道:“你看师父说的话。弟子蒙菩萨劝化受了戒行等候师父。自蒙师父收了我又承教诲跟着师父还不上两月更不曾进得半分功果怎敢图此富贵!宁死也要往西天去决不干此欺心之事。”那妇人见他们推辞不肯急抽身转进屏风扑的把腰门关上。师徒们撇在外面茶饭全无再没人出。八戒心中焦燥埋怨唐僧道:“师父忒不会干事把话通说杀了。你好道还活着些脚儿只含糊答应哄他些斋饭吃了今晚落得一宵快活明日肯与不肯在乎你我了。似这般关门不出我们这清灰冷灶一夜怎过!”悟净道:“二哥你在他家做个女婿罢。”

八戒道:“兄弟不要栽人。从长计较。”行者道:“计较甚的?你要肯便就教师父与那妇人做个亲家你就做个倒踏门的女婿。他家这等有财有宝一定倒陪妆奁整治个会亲的筵席我们也落些受用。你在此间还俗却不是两全其美?”八戒道:“话便也是这等说却只是我脱俗又还俗停妻再娶妻了。”沙僧道:“二哥原来是有嫂子的?”行者道:“你还不知他哩他本是乌斯藏高老儿庄高太公的女婿。因被老孙降了他也曾受菩萨戒行没及奈何被我捉他来做个和尚所以弃了前妻投师父往西拜佛。他想是离别的久了又想起那个勾当却才听见这个勾当断然又有此心。呆子你与这家子做了女婿罢只是多拜老孙几拜我不检举你就罢了。”那呆子道:“胡说!胡说!大家都有此心独拿老猪出丑。常言道:和尚是色中饿鬼。那个不要如此?都这们扭扭捏捏的拿班儿把好事都弄得裂了。这如今茶水不得见面灯火也无人管虽熬了这一夜但那匹马明日又要驮人又要走路再若饿上这一夜只好剥皮罢了。你们坐着等老猪去放放马来。”那呆子虎急急的解了缰绳拉出马去。行者道:“沙僧你且陪师父坐这里等老孙跟他去看他往那里放马。”三藏道:“悟空你看便去看他但只不可只管嘲他了。”行者道:“我晓得。”这大圣走出厅房摇身一变变作个红蜻蜓儿飞出前门赶上八戒。

那呆子拉着马有草处且不教吃草嗒嗒嗤嗤的赶着马转到后门去只见那妇人带了三个女子在后门外闲立着看菊花儿耍子。他娘女们看见八戒来时三个女儿闪将进去那妇人伫立门道:“小长老那里去?”这呆子丢了缰绳上前唱个喏道声:“娘!我来放马的。”那妇人道:“你师父忒弄精细在我家招了女婿却不强似做挂搭僧往西跄路?”八戒笑道:“他们是奉了唐王的旨意不敢有违君命不肯干这件事。

刚才都在前厅上栽我我又有些奈上祝下的只恐娘嫌我嘴长耳大。”那妇人道:“我也不嫌只是家下无个家长招一个倒也罢了但恐小女儿有些儿嫌丑。”八戒道:“娘你上复令爱不要这等拣汉。想我那唐僧人才虽俊其实不中用。我丑自丑有几句口号儿。”妇人道:“你怎的说么?”八戒道:“我虽然人物丑勤紧有些功。若言千顷地不用使牛耕。只消一顿钯布种及时生。没雨能求雨无风会唤风。房舍若嫌矮起上二三层。

地下不扫扫一扫阴沟不通通一通。家长里短诸般事踢天弄井我皆能。”那妇人道:“既然干得家事你再去与你师父商量商量看不尴尬便招你罢。”八戒道:“不用商量!他又不是我的生身父母干与不干都在于我。”妇人道:“也罢也罢等我与小女说。”看他闪进去扑的掩上后门。八戒也不放马将马拉向前来。怎知孙大圣已一一尽知他转翅飞来现了本相先见唐僧道:“师父悟能牵马来了。”长老道:“马若不牵恐怕撒欢走了。”行者笑将起来把那妇人与八戒说的勾当从头说了一遍三藏也似信不信的。

少时间见呆子拉将马来拴下长老道:“你马放了?”八戒道:“无甚好草没处放马。”行者道:“没处放马可有处牵马么?”呆子闻得此言情知走了消息也就垂头扭颈努嘴皱眉半晌不言。又听得呀的一声腰门开了有两对红灯一副提壶香云霭霭环珮叮叮那妇人带着三个女儿走将出来叫真真、爱爱、怜怜拜见那取经的人物。那女子排立厅中朝上礼拜。果然也生得标致但见他:一个个蛾眉横翠粉面生春。

妖娆倾国色窈窕动人心。花钿显现多娇态绣带飘飖迥绝尘。

半含笑处樱桃绽缓步行时兰麝喷。满头珠翠颤巍巍无数宝钗簪;遍体幽香娇滴滴有花金缕细。说甚么楚娃美貌西子娇容?真个是九天仙女从天降月里嫦娥出广寒!那三藏合掌低头孙大圣佯佯不睬这沙僧转背回身。你看那猪八戒眼不转睛淫心紊乱色胆纵横扭捏出悄语低声道:“有劳仙子下降。

娘请姐姐们去耶。”那三个女子转入屏风将一对纱灯留下。

妇人道:“四位长老可肯留心着那个配我小女么?”悟净道:

“我们已商议了着那个姓猪的招赘门下。”八戒道:“兄弟不要栽我还从众计较。”行者道:“还计较甚么?你已是在后门说合的停停当当娘都叫了又有甚么计较?师父做个男亲家这婆儿做个女亲家等老孙做个保亲沙僧做个媒人。也不必看通书今朝是个天恩上吉日你来拜了师父进去做了女婿罢。”八戒道:“弄不成!弄不成!那里好干这个勾当!”行者道:

“呆子不要者嚣你那口里娘也不知叫了多少又是甚么弄不成?快快的应成带携我们吃些喜酒也是好处。”他一只手揪着八戒一只手扯住妇人道:“亲家母带你女婿进去。”那呆子脚儿趄趄的要往那里走那妇人即唤童子:“展抹桌椅铺排晚斋管待三位亲家。我领姑夫房里去也。”一壁厢又吩咐庖丁排筵设宴明晨会亲那几个童子又领命讫。他三众吃了斋急急铺铺都在客座里安歇不题。

却说那八戒跟着丈母行入里面一层层也不知多少房舍磕磕撞撞尽都是门槛绊脚。呆子道:“娘慢些儿走我这里边路生你带我带儿。”那妇人道:“这都是仓房、库房、碾房各房还不曾到那厨房边哩。”八戒道:“好大人家!”磕磕撞撞转湾抹角又走了半会才是内堂房屋。那妇人道:“女婿你师兄说今朝是天恩上吉日就教你招进来了。却只是仓卒间不曾请得个阴阳拜堂撒帐你可朝上拜八拜儿罢。”八戒道:

“娘娘说得是你请上坐等我也拜几拜就当拜堂就当谢亲两当一儿却不省事?”他丈母笑道:“也罢也罢果然是个省事干家的女婿。我坐着你拜么。”咦!满堂中银烛辉煌这呆子朝上礼拜拜毕道:“娘你把那个姐姐配我哩?”他丈母道:“正是这些儿疑难:我要把大女儿配你恐二女怪;要把二女配你恐三女怪;欲将三女配你又恐大女怪;所以终疑未定。”八戒道:“娘既怕相争都与我罢省得闹闹吵吵乱了家法。”他丈母道:“岂有此理!你一人就占我三个女儿不成!”八戒道:“你看娘说的话。那个没有三房四妾?就再多几个你女婿也笑纳了。我幼年间也曾学得个熬战之法管情一个个伏侍得他欢喜。”那妇人道:“不好!不好!我这里有一方手帕你顶在头上遮了脸撞个天婚教我女儿从你跟前走过你伸开手扯倒那个就把那个配了你罢。”呆子依言接了手帕顶在头上。有诗为证诗曰:痴愚不识本原由色剑伤身暗自休。从来信有周公礼今日新郎顶盖头。那呆子顶裹停当道:“娘请姐姐们出来么。”他丈母叫:“真真、爱爱、怜怜都来撞天婚配与你女婿。”只听得环珮响亮兰麝馨香似有仙子来往那呆子真个伸手去捞人。两边乱扑左也撞不着右也撞不着。来来往往不知有多少女子行动只是莫想捞着一个。东扑抱着柱科西扑摸着板壁两头跑晕了立站不稳只是打跌。前来蹬着门扇后去汤着砖墙磕磕撞撞跌得嘴肿头青坐在地下喘气呼呼的道:“娘啊你女儿这等乖滑得紧捞不着一个奈何!奈何!”那妇人与他揭了盖头道:“女婿不是我女儿乖滑他们大家谦让不肯招你。”八戒道:“娘啊既是他们不肯招我啊你招了我罢。”那妇人道:“好女婿呀!这等没大没小的连丈母也都要了!我这三个女儿心性最巧他一人结了一个珍珠篏锦汗衫儿。你若穿得那个的就教那个招你罢。”八戒道:

“好!好!好!把三件儿都拿来我穿了看。若都穿得就教都招了罢。”那妇人转进房里止取出一件来递与八戒。那呆子脱下青锦布直裰取过衫儿就穿在身上还未曾系上带子扑的一蹻跌倒在地原来是几条绳紧紧绷住。那呆子疼痛难禁这些人早已不见了。

却说三藏、行者、沙僧一觉睡醒不觉的东方白。忽睁睛抬头观看。那里得那大厦高堂也不是雕梁画栋一个个都睡在松柏林中。慌得那长老忙呼行者沙僧道:“哥哥罢了!罢了!我们遇着鬼了!”孙大圣心中明白微微的笑道:“怎么说?”

长老道:“你看我们睡在那里耶!”行者道:“这松林下落得快活但不知那呆子在那里受罪哩。”长老道:“那个受罪?”行者笑道:“昨日这家子娘女们不知是那里菩萨在此显化我等想是半夜里去了只苦了猪八戒受罪。”三藏闻言合掌顶礼又只见那后边古柏树上飘飘荡荡的挂着一张简帖儿。沙僧急去取来与师父看时却是八句颂子云:“黎山老母不思凡南海菩萨请下山。普贤文殊皆是客化成美女在林间。圣僧有德还无俗八戒无禅更有凡。从此静心须改过若生怠慢路途难!”那长老、行者、沙僧正然唱念此颂只听得林深处高声叫道:“师父啊绷杀我了!救我一救!下次再不敢了!”三藏道:

“悟空那叫唤的可是悟能么?”沙僧道:“正是。”行者道:“兄弟莫睬他我们去罢。”三藏道:“那呆子虽是心性愚顽却只是一味懞直倒也有些膂力挑得行李还看当日菩萨之念救他随我们去罢料他以后再不敢了。”那沙和尚却卷起铺盖收拾了担子;孙大圣解缰牵马引唐僧入林寻看。咦!这正是:从正修持须谨慎扫除爱欲自归真。毕竟不知那呆子凶吉如何且听下回分解。

\chapter[万寿山大仙留故友\ 五庄观行者窃人参]{万寿山大仙留故友\\五庄观行者窃人参}
\chapter[镇元仙赶捉取经僧\ 孙行者大闹五庄观]{镇元仙赶捉取经僧\\孙行者大闹五庄观}
\chapter[孙悟空三岛求方\ 观世音甘泉活树]{孙悟空三岛求方\\观世音甘泉活树}
\chapter[尸魔三戏唐三藏\ 圣僧恨逐美猴王]{尸魔三戏唐三藏\\圣僧恨逐美猴王}
\chapter[花果山群妖聚义\ 黑松林三藏逢魔]{花果山群妖聚义\\黑松林三藏逢魔}
\chapter[脱难江流来国土\ 承恩八戒转山林]{脱难江流来国土\\承恩八戒转山林}
\chapter[邪魔侵正法\ 意马忆心猿]{邪魔侵正法\\意马忆心猿}
\chapter[猪八戒义激猴王\ 孙行者智降妖怪]{猪八戒义激猴王\\孙行者智降妖怪}

第三十一回 猪八戒义激猴王 孙行者智降妖怪

义结孔怀法归本性。金顺木驯成正果心猿木母合丹元。

共登极乐世界同来不二法门。经乃修行之总径佛配自己之元神。兄和弟会成三契妖与魔色应五行。剪除六门趣即赴大雷音。却说那呆子被一窝猴子捉住了扛抬扯拉把一件直裰子揪破口里劳劳叨叨的自家念诵道:“罢了!罢了!这一去有个打杀的情了!”不一时到洞口。那大圣坐在石崖之上骂道:“你这馕糠的劣货!你去便罢了怎么骂我?”八戒跪在地下道:“哥啊我不曾骂你若骂你就嚼了舌头根。我只说哥哥不去我自去报师父便了怎敢骂你?”行者道:“你怎么瞒得过我?我这左耳往上一扯晓得三十三天人说话;我这右耳往下一扯晓得十代阎王与判官算帐。你今走路把我骂我岂不听见?”八戒道:“哥啊我晓得你贼头鼠脑的一定又变作个甚么东西儿跟着我听的。”行者叫:“小的们选大棍来!先打二十个见面孤拐再打二十个背花然后等我使铁棒与他送行!”八戒慌得磕头道:“哥哥千万看师父面上饶了我罢!”行者道:

“我想那师父好仁义儿哩!”八戒又道:“哥哥不看师父啊请看海上菩萨之面饶了我罢!”

行者见说起菩萨却有三分儿转意道:“兄弟既这等说我且不打你你却老实说不要瞒我。那唐僧在那里有难你却来此哄我?”八戒道:“哥哥没甚难处实是想你。”行者骂道:

“这个好打的劣货!你怎么还要者嚣?我老孙身回水帘洞心逐取经僧。那师父步步有难处处该灾你趁早儿告诵我免打!”八戒闻得此言叩头上告道:“哥啊分明要瞒着你请你去的不期你这等样灵。饶我打放我起来说罢。”行者道:“也罢起来说。”众猴撒开手那呆子跳得起来两边乱张行者道:“你张甚么?”八戒道:“看看那条路儿空阔好跑。”行者道:

“你跑到那里?我就让你先走三日老孙自有本事赶转你来!快早说来这一恼我的性子断不饶你!”八戒道:“实不瞒哥哥说自你回后我与沙僧保师父前行。只见一座黑松林师父下马教我化斋。我因许远无一个人家辛苦了略在草里睡睡。

不想沙僧别了师父又来寻我。你晓得师父没有坐性他独步林间玩景出得林见一座黄金宝塔放光他只当寺院不期塔下有个妖精名唤黄袍被他拿住。后边我与沙僧回寻止见白马行囊不见师父随寻至洞口与那怪厮杀。师父在洞幸亏了一个救星原是宝象国王第三个公主被那怪摄来者。他修了一封家书托师父寄去遂说方便解放了师父。到了国中递了书子那国王就请师父降妖取回公主。哥啊你晓得那老和尚可会降妖?我二人复去与战。不知那怪神通广大将沙僧又捉了我败阵而走伏在草中。那怪变做个俊俏文人入朝与国王认亲把师父变作老虎。又亏了白龙马夜现龙身去寻师父。师父倒不曾寻见却遇着那怪在银安殿饮酒。他变一宫娥与他巡酒舞刀欲乘机而砍反被他用满堂红打伤马腿。就是他教我来请师兄的说道:“师兄是个有仁有义的君子君子不念旧恶一定肯来救师父一难。’万望哥哥念一日为师、终身为父之情千万救他一救!”行者道:“你这个呆子!我临别之时曾叮咛又叮咛说道:‘若有妖魔捉住师父你就说老孙是他大徒弟。’怎么却不说我?”八戒又思量道:“请将不如激将等我激他一激。”道:“哥啊不说你还好哩只为说你他一无状!”行者道:“怎么说?”八戒道:“我说:‘妖精你不要无礼莫害我师父!我还有个大师兄叫做孙行者。他神通广大善能降妖。他来时教你死无葬身之地!’那怪闻言越加忿怒骂道:‘是个甚么孙行者我可怕他?他若来我剥了他皮抽了他筋啃了他骨吃了他心!饶他猴子瘦我也把他剁碎着油烹!’”行者闻言就气得抓耳挠腮暴躁乱跳道:“是那个敢这等骂我!”八戒道:“哥哥息怒是那黄袍怪这等骂来我故学与你听也。”行者道:“贤弟你起来。不是我去不成既是妖精敢骂我我就不能不降他我和你去。老孙五百年前大闹天宫普天的神将看见我一个个控背躬身口口称呼大圣。这妖怪无礼他敢背前面后骂我!我这去把他拿住碎尸万段以报骂我之仇!报毕我即回来。”八戒道:“哥哥正是你只去拿了妖精报了你仇那时来与不来任从尊意。”

那猴才跳下崖撞入洞里脱了妖衣整一整锦直裰束一束虎皮裙执了铁棒径出门来。慌得那群猴拦住道:“大圣爷爷你往那里去?带挈我们耍子几年也好。”行者道:“小的们你说那里话!我保唐僧的这桩事天上地下都晓得孙悟空是唐僧的徒弟。他倒不是赶我回来倒是教我来家看看送我来家自在耍子。如今只因这件事你们却都要仔细看守家业依时插柳栽松毋得废坠待我还去保唐僧取经回东土。功成之后仍回来与你们共乐天真。”众猴各各领命。

那大圣才和八戒携手驾云离了洞过了东洋大海至西岸住云光叫道:“兄弟你且在此慢行等我下海去净净身子。”八戒道:“忙忙的走路且净甚么身子?”行者道:“你那里知道我自从回来这几日弄得身上有些妖精气了。师父是个爱干净的恐怕嫌我。”八戒于此始识得行者是片真心更无他意。须臾洗毕复驾云西进只见那金塔放光八戒指道:“那不是黄袍怪家?沙僧还在他家里。”行者道:“你在空中等我下去看看那门前如何好与妖精见阵。”八戒道:“不要去妖精不在家。”行者道:“我晓得。”好猴王按落祥光径至洞门外观看只见有两个小孩子在那里使弯头棍打毛球抢窝耍子哩。一个有十来岁一个有八九岁了。正戏处被行者赶上前也不管他是张家李家的一把抓着顶搭子提将过来。那孩子吃了唬口里夹骂带哭的乱嚷惊动那波月洞的小妖急报与公主道:

“奶奶不知甚人把二位公子抢去也!”原来那两个孩子是公主与那怪生的。公主闻言忙忙走出洞门来只见行者提着两个孩子站在那高崖之上意欲往下掼慌得那公主厉声高叫道:

“那汉子我与你没甚相干怎么把我儿子拿去?他老子利害有些差错决不与你干休!”行者道:“你不认得我?我是那唐僧的大徒弟孙悟空行者。我有个师弟沙和尚在你洞里你去放他出来我把这两个孩儿还你似这般两个换一个还是你便宜。”那公主闻言急往里面喝退那几个把门的小妖亲动手把沙僧解了。沙僧道:“公主你莫解我恐你那怪来家问你要人带累你受气。”公主道:“长老啊你是我的恩人你替我折辩了家书救了我一命我也留心放你。不期洞门之外你有个大师兄孙悟空来了叫我放你哩。”噫!那沙僧一闻孙悟空的三个字好便似醍醐灌顶甘露滋心一面天生喜满腔都是春也不似闻得个人来就如拾着一方金玉一般。你看他捽手佛衣走出门来对行者施礼道:“哥哥你真是从天而降也!万乞救我一救!”行者笑道:“你这个沙尼!师父念《紧箍儿咒》可肯替我方便一声?都弄嘴施展!要保师父如何不走西方路却在这里蹲甚么?”沙僧道:“哥哥不必说了君子既往不咎。我等是个败军之将不可语勇救我救儿罢!”行者道:“你上来。”

沙僧才纵身跳上石崖。

却说那八戒停立空中看见沙僧出洞即按下云头叫声:

“沙兄弟心忍!心忍!”沙僧见身道:“二哥你从那里来?”八戒道:“我昨日败阵夜间进城会了白马知师父有难被黄袍使法变做个老虎。那白马与我商议请师兄来的。”行者道:“呆子且休叙阔把这两个孩子你两人抱着先进那宝象城去激那怪来等我在这里打他。”沙僧道:“哥啊怎么样激他?”行者道:“你两个驾起云站在那金銮殿上莫分好歹把那孩子往那白玉阶前一掼。有人问你是甚人你便说是黄袍妖精的儿子被我两个拿将来也。那怪听见管情回来我却不须进城与他斗了。若在城上厮杀必要喷云嗳雾播土扬尘惊扰那朝廷与多官黎庶俱不安也。”八戒笑道:“哥哥你但干事就左我们。”行者道:“如何为左你?”八戒道:“这两个孩子被你抓来已此唬破胆了这一会声都哭哑再一会必死无疑。我们拿他往下一掼掼做个肉糰子那怪赶上肯放?定要我两个偿命。你却还不是个干净人?连见证也没你你却不是左我们?”行者道:“他若扯你你两个就与他打将这里来。这里有战场宽阔我在此等候打他。”沙僧道:“正是正是大哥说得有理。我们去来。”他两个才倚仗威风将孩子拿去。

行者即跳下石崖到他塔门之下那公主道:“你这和尚全无信义!你说放了你师弟就与我孩儿怎么你师弟放去把我孩儿又留反来我门做甚?”行者陪笑道:“公主休怪你来的日子已久带你令郎去认他外公去哩。”公主道:“和尚莫无礼我那黄袍郎比众不同。你若唬了我的孩儿与他挪挪惊是。”行者笑道:“公主啊为人生在天地之间怎么便是得罪?”

公主道:“我晓得。”行者道:“你女流家晓得甚么?”公主道:

“我自幼在宫曾受父母教训。记得古书云五刑之属三千而罪莫大于不孝。”行者道:“你正是个不孝之人。盖父兮生我母兮鞠我。哀哀父母生我劬劳!故孝者百行之原万善之本却怎么将身陪伴妖精更不思念父母?非得不孝之罪如何?”公主闻此正言半晌家耳红面赤惭愧无地忽失口道:“长老之言最善我岂不思念父母?只因这妖精将我摄骗在此他的法令又谨我的步履又难路远山遥无人可传音信。欲要自尽又恐父母疑我逃走事终不明。故没奈何苟延残喘诚为天地间一大罪人也!”说罢泪如泉涌。行者道:“公主不必伤悲。猪八戒曾告诉我说你有一封书曾救了我师父一命你书上也有思念父母之意。老孙来管与你拿了妖精带你回朝见驾别寻个佳偶侍奉双亲到老你意如何?”公主道:“和尚啊你莫要寻死。昨者你两个师弟那样好汉也不曾打得过我黄袍郎。

你这般一个筋多骨少的瘦鬼一似个螃蟹模样骨头都长在外面有甚本事你敢说拿妖魔之话?”行者笑道:“你原来没眼色认不得人。俗语云:尿泡虽大无斤两秤铊虽小压千斤。他们相貌空大无用走路抗风穿衣费布种火心空顶门腰软吃食无功。咱老孙小自小筋节。”那公主道:“你真个有手段么?”行者道:“我的手段你是也不曾看见绝会降妖极能伏怪。”公主道:“你却莫误了我耶。”行者道:“决然误你不得。”公主道:“你既会降妖伏怪如今却怎样拿他?”行者说:“你且回避回避莫在我这眼前倘他来时不好动手脚只恐你与他情浓了舍不得他。”公主道:“我怎的舍不得他?其稽留于此者不得已耳!”行者道:“你与他做了十三年夫妻岂无情意?我若见了他不与他儿戏一棍便是一棍一拳便是一拳须要打倒他才得你回朝见驾。”那公主果然依行者之言往僻静处躲避也是他姻缘该尽故遇着大圣来临。那猴王把公主藏了他却摇身一变就变做公主一般模样回转洞中专候那怪。

却说八戒、沙僧把两个孩子拿到宝象国中往那白玉阶前捽下可怜都掼做个肉饼相似鲜血迸流骨骸粉碎慌得那满朝多官报道:“不好了!不好了!天上掼下两个人来了!”八戒厉声高叫道:“那孩子是黄袍妖精的儿子被老猪与沙弟拿将来也!”那怪还在银安殿宿酒未醒正睡梦间听得有人叫他名字他就翻身抬头观看只见那云端里是猪八戒沙和尚二人吆喝。妖怪心中暗想道:“猪八戒便也罢了沙和尚是我绑在家里他怎么得出来?我的浑家怎么肯放他?我的孩儿怎么得到他手?这怕是猪八戒不得我出去与他交战故将此计来羁我。我若认了这个泛头就与他打啊噫!我却还害酒哩!假若被他筑上一钯却不灭了这个威风识破了那个关窍且等我回家看看是我的儿子不是我的儿子再与他说话不迟。”好妖怪他也不辞王驾转山林径去洞中查信息。此时朝中已知他是个妖怪了原来他夜里吃了一个宫娥还有十七个脱命去的五更时奏了国王说他如此如此。又因他不辞而去越知他是怪那国王即着多官看守着假老虎不题。

却说那怪径回洞口。行者见他来时设法哄他把眼挤了一挤扑簌簌泪如雨落儿天儿地的跌脚捶胸于此洞里嚎啕痛哭。那怪一时间那里认得?上前搂住道:“浑家你有何事这般烦恼?”那大圣编成的鬼话捏出的虚词泪汪汪的告道:

“郎君啊!常言道男子无妻财没主妇女无夫身落空!你昨日进朝认亲怎不回来?今早被猪八戒劫了沙和尚又把我两个孩儿抢去是我苦告更不肯饶。他说拿去朝中认认外公这半日不见孩儿又不知存亡如何你又不见来家教我怎生割舍?

故此止不住伤心痛哭。”那怪闻言心中大怒道:“真个是我的儿子?”行者道:“正是被猪八戒抢去了。”那妖魔气得乱跳道:

“罢了!罢了!我儿被他掼杀了!已是不可活也!只好拿那和尚来与我儿子偿命报仇罢!浑家你且莫哭你如今心里觉道怎么?且医治一医治。”行者道:“我不怎的只是舍不得孩儿哭得我有些心疼。”妖魔道:“不打紧你请起来我这里有件宝贝只在你那疼上摸一摸儿就不疼了。却要仔细休使大指儿弹着若使大指儿弹着啊就看出我本相来了”行者闻言心中暗笑道:“这泼怪倒也老实不动刑法就自家供了。等他拿出宝贝来我试弹他一弹看他是个甚么妖怪。”那怪携着行者一直行到洞里深远密闭之处。却从口中吐出一件宝贝有鸡子大小是一颗舍利子玲珑内丹。行者心中暗喜道:“好东西耶!

这件物不知打了多少坐工炼了几年磨难配了几转雌雄炼成这颗内丹舍利。今日大有缘法遇着老孙。”那猴子拿将过来那里有甚么疼处特故意摸了一摸一指头弹将去。那妖慌了劈手来抢你思量那猴子好不溜撒把那宝贝一口吸在肚里。那妖魔攥着拳头就打被行者一手隔住把脸抹了一抹现出本相道声“妖怪!不要无礼!你且认认看我是谁?”那妖怪见了大惊道:“呀!浑家你怎么拿出这一副嘴脸来耶?”行者骂道:“我把你这个泼怪!谁是你浑家?连你祖宗也还不认得哩?”那怪忽然省悟道:“我象有些认得你哩。”行者道:“我且不打你你再认认看。”那怪道:“我虽见你眼熟一时间却想不起姓名。你果是谁从那里来的?你把我浑家估倒在何处却来我家诈诱我的宝贝?着实无礼!可恶!”行者道:“你是也不认得我。我是唐僧的大徒弟叫做孙悟空行者。我是你五百年前的旧祖宗哩!”那怪道:“没有这话!没有这话!我拿住唐僧时止知他有两个徒弟叫做猪八戒、沙和尚何曾见有人说个姓孙的。你不知是那里来的个怪物到此骗我!”行者道:“我不曾同他二人来是我师父因老孙惯打妖怪杀伤甚多他是个慈悲好善之人将我逐回故不曾同他一路行走。你是不知你祖宗名姓。”那怪道:“你好不丈夫啊!既受了师父赶逐却有甚么嘴脸又来见人!”行者道:“你这个泼怪岂知一日为师终身为父父子无隔宿之仇!你伤害我师父我怎么不来救他?你害他便也罢却又背前面后骂我是怎的说?”妖怪道:“我何尝骂你?”行者道:“是猪八戒说的。”那怪道:“你不要信他那个猪八戒尖着嘴有些会学老婆舌头你怎听他?”行者道:“且不必讲此闲话只说老孙今日到你家里你好怠慢了远客。虽无酒馔款待头却是有的快快将头伸过来等老孙打一棍儿当茶!”那怪闻得说打呵呵大笑道:“孙行者你差了计较了!你既说要打不该跟我进来。我这里大小群妖还有百十饶你满身是手也打不出我的门去。”行者道:“不要胡说!莫说百十个就有几千、几万只要一个个查明白了好打棍棍无空教你断根绝迹!”那怪闻言急传号令把那山前山后群妖洞里洞外诸怪一齐点起各执器械把那三四层门密密拦阻不放。行者见了满心欢喜双手理棍喝声叫“变!”变的三头六臂把金箍棒幌一幌变做三根金箍棒。你看他六只手使着三根棒一路打将去好便似虎入羊群鹰来鸡栅可怜那小怪汤着的头如粉碎;刮着的血似水流!往来纵横如入无人之境。止剩一个老妖赶出门来骂道:“你这泼猴其实惫懒!怎么上门子欺负人家!”行者急回头用手招呼道:“你来!你来!

打倒你才是功绩!”

那怪物举宝刀分头便砍好行者掣铁棒觌面相迎。这一场在那山顶上半云半雾的杀哩:大圣神通大妖魔本事高。

这个横理生金棒那个斜举蘸钢刀。悠悠刀起明霞亮轻轻棒架彩云飘。往来护顶翻多次反复浑身转数遭。一个随风更面目一个立地把身摇。那个大睁火眼伸猿膊这个明幌金睛折虎腰。你来我去交锋战刀迎棒架不相饶。猴王铁棍依三略怪物钢刀按六韬。一个惯行手段为魔主一个广施法力保唐僧。猛烈的猴王添猛烈英豪的怪物长英豪。死生不顾空中打都为唐僧拜佛遥。他两个战有五六十合不分胜负。行者心中暗喜道:“这个泼怪他那口刀倒也抵得住老孙的这根棒。等老孙丢个破绽与他看他可认得。”好猴王双手举棍使一个高探马的势子。那怪不识是计见有空儿舞着宝刀径奔下三路砍被行者急转个大中平挑开他那口刀又使个叶底偷桃势望妖精头顶一棍就打得他无影无踪急收棍子看处不见了妖精行者大惊道:“我儿啊不禁打就打得不见了。果是打死好道也有些脓血如何没一毫踪影?想是走了。”急纵身跳在云端里看处四边更无动静。“老孙这双眼睛不管那里一抹都见却怎么走得这等溜撒?我晓得了:那怪说有些儿认得我想必不是凡间的怪多是天上来的精。”

那大圣一时忍不住怒攥着铁棒打个筋斗只跳到南天门上。慌得那庞刘苟毕、张陶邓辛等众两边躬身控背不敢拦阻让他打入天门直至通明殿下。早有张葛许邱四大天师问道:“大圣何来?”行者道:“因保唐僧至宝象国有一妖魔欺骗国女伤害吾师老孙与他赌斗。正斗间不见了这怪。想那怪不是凡间之怪多是天上之精特来查勘那一路走了甚么妖神。”天师闻言即进灵霄殿上启奏蒙差查勘九曜星官、十二元辰、东西南北中央五斗、河汉群辰、五岳四渎、普天神圣都在天上更无一个敢离方位。又查那斗牛宫外二十八宿颠倒只有二十七位内独少了奎星。天师回奏道:“奎木狼下界了。”

玉帝道:“多少时不在天了?”天师道:“四卯不到。三日点卯一次今已十三日了。”玉帝道:“天上十三日下界已是十三年。”

即命本部收他上界。那二十七宿星员领了旨意出了天门各念咒语惊动奎星。你道他在那里躲避?他原来是孙大圣大闹天宫时打怕了的神将闪在那山涧里潜灾被水气隐住妖云所以不曾看见他。他听得本部星员念咒方敢出头随众上界。

被大圣拦住天门要打幸亏众星劝住押见玉帝。那怪腰间取出金牌在殿下叩头纳罪玉帝道:“奎木狼上界有无边的胜景你不受用却私走一方何也?”奎宿叩头奏道:“万岁赦臣死罪。那宝象国王公主非凡人也。他本是披香殿侍香的玉女因欲与臣私通臣恐点污了天宫胜境他思凡先下界去托生于皇宫内院是臣不负前期变作妖魔占了名山摄他到洞府与他配了一十三年夫妻。一饮一啄莫非前定今被孙大圣到此成功。”玉帝闻言收了金牌贬他去兜率宫与太上老君烧火带俸差操有功复职无功重加其罪。行者见玉帝如此放心中欢喜朝上唱个大喏又向众神道:“列位起动了。”天师笑道:“那个猴子还是这等村俗替他收了怪神也倒不谢天恩却就喏喏而退。”玉帝道:“只得他无事落得天上清平是幸。”

那大圣按落祥光径转碗子山波月洞寻出公主将那思凡下界收妖的言语正然陈诉只听得半空中八戒、沙僧厉声高叫道:“师兄有妖精留几个儿我们打耶。”行者道:“妖精已尽绝矣。”沙僧道:“既把妖精打绝无甚挂碍将公主引入朝中去罢。不要睁眼兄弟们使个缩地法来。”那公主只闻得耳内风响霎时间径回城里。他三人将公主带上金銮殿上那公主参拜了父王、母后会了姊妹各官俱来拜见。那公主才启奏道:

“多亏孙长老法力无边降了黄袍怪救奴回国。”那国王问曰:

“黄袍是个甚怪?”行者道:“陛下的驸马是上界的奎星令爱乃侍香的玉女因思凡降落人间不非小可都因前世前缘该有这些姻眷。那怪被老孙上天宫启奏玉帝玉帝查得他四卯不到下界十三日就是十三年了盖天上一日下界一年。随差本部星宿收他上界贬在兜率宫立功去讫老孙却救得令爱来也。”那国王谢了行者的恩德便教:“看你师父去来。”

他三人径下宝殿与众官到朝房里抬出铁笼将假虎解了铁索。别人看他是虎独行者看他是人。原来那师父被妖术魇住不能行走心上明白只是口眼难开。行者笑道:“师父啊你是个好和尚怎么弄出这般个恶模样来也?你怪我行凶作恶赶我回去你要一心向善怎么一旦弄出个这等嘴脸?”

八戒道:“哥啊救他一救罢不要只管揭挑他了。”行者道:“你凡事撺唆是他个得意的好徒弟你不救他又寻老孙怎的?原与你说来待降了妖精报了骂我之仇就回去的。”沙僧近前跪下道:“哥啊古人云不看僧面看佛面。兄长既是到此万望救他一救。若是我们能救也不敢许远的来奉请你也。”行者用手挽起道:“我岂有安心不救之理?快取水来。”那八戒飞星去驿中取了行李马匹将紫金钵盂取出盛水半盂递与行者。

行者接水在手念动真言望那虎劈头一口喷上退了妖术解了虎气。长老现了原身定性睁睛才认得是行者一把搀住道:“悟空!你从那里来也?”沙僧侍立左右把那请行者降妖精救公主解虎气并回朝上项事备陈了一遍。三藏谢之不尽道:“贤徒亏了你也!亏了你也!这一去早诣西方径回东土奏唐王你的功劳第一。”行者笑道:“莫说莫说!但不念那话儿足感爱厚之情也。”国王闻此言又劝谢了他四众整治素筵大开东阁。他师徒受了皇恩辞王西去国王又率多官远送。这正是:君回宝殿定江山僧去雷音参佛祖。毕竟不知此后又有甚事几时得到西天且听下回分解。

\chapter[平顶山功曹传信\ 莲花洞木母逢灾]{平顶山功曹传信\\莲花洞木母逢灾}
\chapter[外道迷真性\ 元神助本心]{外道迷真性\\元神助本心}
\chapter[魔头巧算困心猿\ 大圣腾那骗宝贝]{魔头巧算困心猿\\大圣腾那骗宝贝}
\chapter[外道施威欺正性\ 心猿获宝伏邪魔]{外道施威欺正性\\心猿获宝伏邪魔}
\chapter[心猿正处诸缘伏\ 劈破旁门见月明]{心猿正处诸缘伏\\劈破旁门见月明}

第三十六回 心猿正处诸缘伏 劈破旁门见月明

却说孙行者按落云头对师父备言菩萨借童子、老君收去宝贝之事。三藏称谢不已死心塌地办虔诚舍命投西攀鞍上马猪八戒挑着行李沙和尚拢着马头孙行者执了铁棒剖开路径下高山前进。说不尽那水宿风餐披霜冒露师徒们行罢多时前又一山阻路。三藏在那马上高叫:“徒弟啊你看那里山势崔巍须是要仔细提防恐又有魔障侵身也。”行者道:

“师父休要胡思乱想只要定性存神自然无事。”三藏道:“徒弟呀西天怎么这等难行?我记得离了长安城在路上春尽夏来秋残冬至有四五个年头怎么还不能得到?”行者闻言呵呵笑道:“早哩!早哩!还不曾出大门哩!”八戒道:“哥哥不要扯谎人间就有这般大门?”行者道:“兄弟我们还在堂屋里转哩!”沙僧笑道:“师兄少说大话吓我那里就有这般大堂屋却也没处买这般大过梁啊。”行者道:“兄弟若依老孙看时把这青天为屋瓦日月作窗棂四山五岳为梁柱天地犹如一敞厅!”八戒听说道:“罢了!罢了!我们只当转些时回去罢。”行者道:“不必乱谈只管跟着老孙走路。”

好大圣横担了铁棒领定了唐僧剖开山路一直前进。

那师父在马上遥观好一座山景真个是:山顶嵯峨摩斗柄树梢仿佛接云霄。青烟堆里时闻得谷口猿啼;乱翠阴中每听得松间鹤唳。啸风山魅立溪间戏弄樵夫;成器狐狸坐崖畔惊张猎户。好山!看那八面崖巍四围险峻。古怪乔松盘翠盖枯摧老树挂藤萝。泉水飞流寒气透人毛冷;巅峰屹崒清风射眼梦魂惊。时听大虫哮吼每闻山鸟时鸣。麂鹿成群穿荆棘往来跳跃;獐兔结党寻野食前后奔跑。佇立草坡一望并无客旅;行来深凹四边俱有豺狼。应非佛祖修行处尽是飞禽走兽场。那师父战战兢兢进此深山心中凄惨兜住马叫声:“悟空啊!我自从益智登山盟王不留行送出城。路上相逢三棱子途中催趱马兜铃。寻坡转涧求荆芥迈岭登山拜茯苓。防己一身如竹沥茴香何日拜朝廷?”孙大圣闻言呵呵冷笑道:“师父不必挂念少要心焦且自放心前进还你个功到自然成也。”

师徒们玩着山景信步行时早不觉红轮西坠正是:十里长亭无客走九重天上现星辰。八河船只皆收港七千州县尽关门。

六宫五府回官宰四海三江罢钓纶。两座楼头钟鼓响一轮明月满乾坤。

那长老在马上遥观只见那山凹里有楼台迭迭殿阁重重。三藏道:“徒弟此时天色已晚幸得那壁厢有楼阁不远想必是庵观寺院我们都到那里借宿一宵明日再行罢。”行者道:“师父说得是。不要忙等我且看好歹如何。”那大圣跳在空中仔细观看果然是座山门但见八字砖墙泥红粉两边门上钉金钉。迭迭楼台藏岭畔层层宫阙隐山中。万佛阁对如来殿朝阳楼应大雄门。七层塔屯云宿雾三尊佛神现光荣。文殊台对伽蓝舍弥勒殿靠大慈厅。看山楼外青光舞步虚阁上紫云生。松关竹院依依绿方丈禅堂处处清。雅雅幽幽供乐事川川道道喜回迎。参禅处有禅僧讲演乐房多乐器鸣。妙高台上昙花坠说法坛前贝叶生。正是那林遮三宝地山拥梵王宫。半壁灯烟光闪灼一行香霭雾朦胧。孙大圣按下云头报与三藏道:“师父果然是一座寺院却好借宿我们去来。”

这长老放开马一直前来径到了山门之外。行者道:“师父这一座是甚么寺?”三藏道:“我的马蹄才然停住脚尖还未出镫就问我是甚么寺好没分晓!”行者道:“你老人家自幼为僧须曾讲过儒书方才去演经法文理皆通然后受唐王的恩宥门上有那般大字如何不认得?”长老骂道:“泼猢狲!说话无知!我才面西催马被那太阳影射奈何门虽有字又被尘垢朦胧所以未曾看见。”行者闻言把腰儿躬一躬长了二丈余高用手展去灰尘道:“师父请看。”上有五个大字乃是敕建宝林寺。行者收了法身道:“师父这寺里谁进去借宿?”三藏道:“我进去。你们的嘴脸丑陋言语粗疏性刚气傲倘或冲撞了本处僧人不容借宿反为不美。”行者道:“既如此请师父进去不必多言。”

那长老却丢了锡杖解下斗篷整衣合掌径入山门只见两边红漆栏杆里面高坐着一对金刚装塑的威仪恶丑:一个铁面钢须似活容一个燥眉圜眼若玲珑。左边的拳头骨突如生铁右边的手掌崚嶒赛赤铜。金甲连环光灿烂明盔绣带映飘风。西方真个多供佛石鼎中间香火红。三藏见了点头长叹道:“我那东土若有人也将泥胎塑这等大菩萨烧香供养啊我弟子也不往西天去矣。”正叹息处又到了二层山门之内见有四大天王之相乃是持国、多闻、增长、广目按东北西南风调雨顺之意。进了二层门里又见有乔松四树一树树翠盖蓬蓬却如伞状忽抬头乃是大雄宝殿。那长老合掌皈依舒身下拜。拜罢起来转过佛台到于后门之下又见有倒座观音普度南海之相。那壁上都是良工巧匠装塑的那些虾鱼蟹鳖出头露尾跳海水波潮耍子。长老又点头三五度感叹万千声道:

“可怜啊!鳞甲众生都拜佛为人何不肯修行!”正赞叹间又见三门里走出一个道人。那道人忽见三藏相貌稀奇丰姿非俗急趋步上前施礼道:“师父那里来的?”三藏道:“弟子是东土大唐驾下差来上西天拜佛求经的今到宝方天色将晚告借一宿。”那道人道:“师父莫怪我做不得主。我是这里扫地撞钟打勤劳的道人里面还有个管家的老师父哩待我进去禀他一声。他若留你我就出来奉请;若不留你我却不敢羁迟。”三藏道:“累及你了。”

那道人急到方丈报道:“老爷外面有个人来了。”那僧官即起身换了衣服按一按毗卢帽披上袈裟急开门迎接问道人:“那里人来?”道人用手指定道:“那正殿后边不是一个人?”那三藏光着一个头穿一领二十五条达摩衣足下登一双拖泥带水的达公鞋斜倚在那后门。僧官见了大怒道:“道人少打!你岂不知我是僧官但只有城上来的士夫降香我方出来迎接。这等个和尚你怎么多虚少实报我接他!看他那嘴脸不是个诚实的多是云游方上僧今日天晚想是要来借宿。我们方丈中岂容他打搅!教他往前廊下蹲罢了报我怎么!”抽身转去。长老闻言满眼垂泪道:“可怜!可怜!这才是人离乡贱!我弟子从小儿出家做了和尚又不曾拜谶吃荤生歹意看经怀怒坏禅心;又不曾丢瓦抛砖伤佛殿阿罗脸上剥真金。噫!可怜啊!不知是那世里触伤天地教我今生常遇不良人!和尚你不留我们宿便罢了怎么又说这等惫懒话教我们在前道廊下去蹲?此话不与行者说还好若说了那猴子进来一顿铁棒把孤拐都打断你的!”长老道:“也罢也罢常言道人将礼乐为先。我且进去问他一声看意下如何。”

那师父踏脚迹跟他进方丈门里只见那僧官脱了衣服气呼呼的坐在那里不知是念经又不知是与人家写法事见那桌案上有些纸札堆积。唐僧不敢深入就立于天井里躬身高叫道:“老院主弟子问讯了!”那和尚就有些不耐烦他进里边来的意思半答不答的还了个礼道:“你是那里来的?”三藏道:“弟子乃东土大唐驾下差来上西天拜活佛求经的经过宝方天晚求借一宿明日不犯天光就行了。万望老院主方便方便。”那僧官才欠起身来道:“你是那唐三藏么?”三藏道:“不敢弟子便是。”僧官道:“你既往西天取经怎么路也不会走?”

三藏道:“弟子更不曾走贵处的路。”他道:“正西去只有四五里远近有一座三十里店店上有卖饭的人家方便好宿。我这里不便不好留你们远来的僧。”三藏合掌道:“院主古人有云庵观寺院都是我方上人的馆驿见山门就有三升米分。你怎么不留我却是何情?”僧官怒声叫道:“你这游方的和尚便是有些油嘴油舌的说话!”三藏道:“何为油嘴油舌?”僧官道:

“古人云老虎进了城家家都闭门。虽然不咬人日前坏了名。”三藏道:“怎么日前坏了名?”他道:“向年有几众行脚僧来于山门口坐下是我见他寒薄一个个衣破鞋无光头赤脚我叹他那般褴褛即忙请入方丈延之上坐。款待了斋饭又将故衣各借一件与他就留他住了几日。怎知他贪图自在衣食更不思量起身就住了七八个年头。住便也罢又干出许多不公的事来。”三藏道:“有甚么不公的事?”僧官道:“你听我说:

闲时沿墙抛瓦闷来壁上扳钉。冷天向火折窗棂夏日拖门拦径。幡布扯为脚带牙香偷换蔓菁。常将琉璃把油倾夺碗夺锅赌胜。”三藏听言心中暗道:“可怜啊!我弟子可是那等样没脊骨的和尚?”欲待要哭又恐那寺里的老和尚笑他但暗暗扯衣揩泪忍气吞声急走出去见了三个徒弟。那行者见师父面上含怒向前问:“师父寺里和尚打你来?”唐僧道:“不曾打。”

八戒说:“一定打来不是怎么还有些哭包声?”那行者道:“骂你来?”唐僧道:“也不曾骂。”行者道:“既不曾打又不曾骂你这般苦恼怎么?好道是思乡哩?”唐僧道:“徒弟他这里不方便。”行者笑道:“这里想是道士?”唐僧怒道:“观里才有道士寺里只是和尚。”行者道:“你不济事但是和尚即与我们一般。常言道既在佛会下都是有缘人。你且坐等我进去看看。”

好行者按一按顶上金箍束一束腰间裙子执着铁棒径到大雄宝殿上指着那三尊佛像道:“你本是泥塑金装假像内里岂无感应?我老孙保领大唐圣僧往西天拜佛求取真经今晚特来此处投宿趁早与我报名!假若不留我等就一顿棍打碎金身教你还现本相泥土!”这大圣正在前边狠捣叉子乱说只见一个烧晚香的道人点了几枝香来佛前炉里插被行者咄的一声唬了一跌爬起来看见脸又是一跌吓得滚滚蹡蹡跑入方丈里报道:“老爷!外面有个和尚来了!”那僧官道:

“你这伙道人都少打!一行说教他往前廊下去蹲又报甚么!再说打二十!”道人说:“老爷这个和尚比那个和尚不同生得恶躁没脊骨。”僧官道:“怎的模样?”道人道:“是个圆眼睛查耳朵满面毛雷公嘴。手执一根棍子咬牙恨恨的要寻人打哩。”僧官道:“等我出去看。”他即开门只见行者撞进来了真个生得丑陋:七高八低孤拐脸两只黄眼睛一个磕额头;獠牙往外生就象属螃蟹的肉在里面骨在外面。那老和尚慌得把方丈门关了。行者赶上扑的打破门扇道:“赶早将干净房子打扫一千间老孙睡觉!”僧官躲在房里对道人说:“怪他生得丑么原来是说大话折作的这般嘴脸。我这里连方丈、佛殿、钟鼓楼、两廊共总也不上三百间他却要一千间睡觉却打那里来?”道人说:“师父我也是吓破胆的人了凭你怎么答应他罢。”那僧官战索索的高叫道:“那借宿的长老我这小荒山不方便不敢奉留往别处去宿罢。”行者将棍子变得盆来粗细直壁壁的竖在天井里道:“和尚不方便你就搬出去!”僧官道:“我们从小儿住的寺师公传与师父师父传与我辈我辈要远继儿孙。他不知是那里勾当冒冒实实的教我们搬哩。”

道人说:“老爷十分不狤魀搬出去也罢扛子打进门来了。”

僧官道:“你莫胡说!我们老少众大四五百名和尚往那里搬?

搬出去却也没处住。”行者听见道:“和尚没处搬便着一个出来打样棍!”老和尚叫:“道人你出去与我打个样棍来。”那道人慌了道:“爷爷呀!那等个大扛子教我去打样棍!”老和尚道:“养军千日用军一朝。你怎么不出去?”道人说:“那扛子莫说打来若倒下来压也压个肉泥!”老和尚道:“也莫要说压只道竖在天井里夜晚间走路不记得啊一头也撞个大窟窿!”道人说:“师父你晓得这般重却教我出去打甚么样棍?”

他自家里面转闹起来行者听见道:“是也禁不得假若就一棍打杀一个我师父又怪我行凶了。且等我另寻一个甚么打与你看看。”忽抬头只见方丈门外有一个石狮子却就举起棍来乒乓一下打得粉乱麻碎。那和尚在窗眼儿里看见就吓得骨软筋麻慌忙往床下拱道人就往锅门里钻口中不住叫:“爷爷棍重棍重!禁不得!方便方便!”行者道:“和尚我不打你。我问你:“这寺里有多少和尚?”僧官战索索的道:“前后是二百八十五房头共有五百个有度牒的和尚。”行者道:“你快去把那五百个和尚都点得齐齐整整穿了长衣服出去把我那唐朝的师父接进来就不打你了。”僧官道:“爷爷若是不打便抬也抬进来。”行者道:“趁早去!”僧官叫:“道人你莫说吓破了胆就是吓破了心便也去与我叫这些人来接唐僧老爷爷来。”

那道人没奈何舍了性命不敢撞门从后边狗洞里钻将出去径到正殿上东边打鼓西边撞钟。钟鼓一齐响处惊动了两廊大小僧众上殿问道:“这早还下晚哩撞钟打鼓做甚?”

道人说:“快换衣服随老师父排班出山门外迎接唐朝来的老爷。”那众和尚真个齐齐整整摆班出门迎接。有的披了袈裟有的着了褊衫无的穿着个一口钟直裰十分穷的没有长衣服就把腰裙接起两条披在身上。行者看见道:“和尚你穿的是甚么衣服?”和尚见他丑恶道:“爷爷不要打等我说。这是我们城中化的布此间没有裁缝是自家做的个一裹穷。”

行者闻言暗笑押着众僧出山门下跪下。那僧官磕头高叫道:“唐老爷请方丈里坐。”八戒看见道:“师父老大不济事你进去时泪汪汪嘴上挂得油瓶。师兄怎么就有此獐智教他们磕头来接?”三藏道:“你这个呆子好不晓礼!常言道鬼也怕恶人哩。”唐僧见他们磕头礼拜甚是不过意上前叫:“列位请起。”众僧叩头道:“老爷若和你徒弟说声方便不动扛子就跪一个月也罢。”唐僧叫:“悟空莫要打他。”行者道:“不曾打若打这会已打断了根矣。”那些和尚却才起身牵马的牵马挑担的挑担抬着唐僧驮着八戒挽着沙僧一齐都进山门里去却到后面方丈中依叙坐下。众僧却又礼拜三藏道:

“院主请起再不必行礼作践贫僧我和你都是佛门弟子。”僧官道:“老爷是上国钦差小和尚有失迎接。今到荒山奈何俗眼不识尊仪与老爷邂逅相逢。动问老爷:一路上是吃素?是吃荤?我们好去办饭。”三藏道:“吃素。”僧官道:“徒弟这个爷爷好的吃荤。”行者道:“我们也吃素都是胎里素。”那和尚道:

“爷爷呀这等凶汉也吃素!”有一个胆量大的和尚近前又问:

“老爷既然吃素煮多少米的饭方彀吃?”八戒道:“小家子和尚!问甚么!一家煮上一石米。”那和尚都慌了便去刷洗锅灶各房中安排茶饭高掌明灯调开桌椅管待唐僧。

师徒们都吃罢了晚斋众僧收拾了家火三藏称谢道:“老院主打搅宝山了。”僧官道:“不敢不敢怠慢怠慢。”三藏道:

“我师徒却在那里安歇?”僧官道:“老爷不要忙小和尚自有区处。”叫道人:“那壁厢有几个人听使令的?”道人说:“师父有。”僧官吩咐道:“你们着两个去安排草料与唐老爷喂马;着几个去前面把那三间禅堂打扫干净铺设床帐快请老爷安歇。”那些道人听命各各整顿齐备却来请唐老爷安寝。他师徒们牵马挑担出方丈径至禅堂门看处只见那里面灯火光明两梢间铺着四张藤屉床。行者见了唤那办草料的道人将草料抬来放在禅堂里面拴下白马教道人都出去。三藏坐在中间灯下两班儿立五百个和尚都伺候着不敢侧离。三藏欠身道:“列位请回贫僧好自在安寝也。”众僧决不敢退。僧官上前吩咐大众:“伏侍老爷安置了再回。”三藏道:“即此就是安置了都就请回。”众人却才敢散去讫。

唐僧举步出门小解只见明月当天叫:“徒弟。”行者、八戒沙僧都出来侍立。因感这月清光皎洁玉宇深沉真是一轮高照大地分明对月怀归口占一古风长篇。诗云:“皓魄当空宝镜悬山河摇影十分全。琼楼玉宇清光满冰鉴银盘爽气旋。万里此时同皎洁一年今夜最明鲜。浑如霜饼离沧海却似冰轮挂碧天。别馆寒窗孤客闷山村野店老翁眠。乍临汉苑惊秋鬓才到秦楼促晚奁。庾亮有诗传晋史袁宏不寐泛江船。

光浮杯面寒无力清映庭中健有仙。处处窗轩吟白雪家家院宇弄冰弦。今宵静玩来山寺何日相同返故园?”行者闻言近前答曰:“师父啊你只知月色光华心怀故里更不知月中之意乃先天法象之规绳也。月至三十日阳魂之金散尽阴魄之水盈轮故纯黑而无光乃曰晦。此时与日相交在晦朔两日之间感阳光而有孕。至初三日一阳现初八日二阳生魄中魂半其平如绳故曰上弦。至今十五日三阳备足是以团圆故曰望。至十六日一阴生二十二日二阴生此时魂中魄半其平如绳故曰下弦。至三十日三阴备足亦当晦。此乃先天采炼之意。我等若能温养二八九九成功那时节见佛容易返故田亦易也。诗曰:前弦之后后弦前药味平平气象全。采得归来炉里炼志心功果即西天。”那长老听说一时解悟明彻真言满心欢喜称谢了悟空。沙僧在旁笑道:“师兄此言虽当只说的是弦前属阳弦后属阴阴中阳半得水之金;更不道水火相搀各有缘全凭土母配如然。三家同会无争竞水在长江月在天。”那长老闻得亦开茅塞。正是理明一窍通千窍说破无生即是仙。八戒上前扯住长老道:“师父莫听乱讲误了睡觉。

这月啊:缺之不久又团圆似我生来不十全。吃饭嫌我肚子大拿碗又说有粘涎。他都伶俐修来福我自痴愚积下缘。我说你取经还满三途业摆尾摇头直上天!”三藏道:“也罢徒弟们走路辛苦先去睡下等我把这卷经来念一念。”行者道:“师父差了你自幼出家做了和尚小时的经文那本不熟?却又领了唐王旨意上西天见佛求取大乘真典。如今功未完成佛未得见经未曾取你念的是那卷经儿?”三藏道:“我自出长安朝朝跋涉日日奔波小时的经文恐怕生了;幸今夜得闲等我温习温习。”行者道:“既这等说我们先去睡也。”他三人各往一张藤床上睡下。长老掩上禅堂门高剔银缸铺开经本默默看念。正是那:楼头初鼓人烟静野浦渔舟火灭时。毕竟不知那长老怎么样离寺且听下回分解。

\chapter[鬼王夜谒唐三藏\ 悟空神化引婴儿]{鬼王夜谒唐三藏\\悟空神化引婴儿}
\chapter[婴儿问母知邪正\ 金木参玄见假真]{婴儿问母知邪正\\金木参玄见假真}
\chapter[一粒金丹天上得\ 三年故主世间生]{一粒金丹天上得\\三年故主世间生}
\chapter[婴儿戏化禅心乱\ 猿马刀圭木母空]{婴儿戏化禅心乱\\猿马刀圭木母空}
\chapter[心猿遭火败\ 木母被魔擒]{心猿遭火败\\木母被魔擒}
\chapter[大圣殷勤拜南海\ 观音慈善缚红孩]{大圣殷勤拜南海\\观音慈善缚红孩}

话说那六健将出洞门径往西南上依路而走。行者心中暗想道:“他要请老大王吃我师父老大王断是牛魔王。我老孙当年与他相会真个意合情投交游甚厚至如今我归正道他还是邪魔。虽则久别还记得他模样且等老孙变作牛魔王哄他一哄看是何如。”好行者躲离了六个小妖展开翅飞向前边离小妖有十数里远近摇身一变变作个牛魔王拔下几根毫毛叫“变!”即变作几个小妖。在那山凹里驾鹰牵犬搭驽张弓充作打围的样子等候那六健将。那一伙厮拖厮扯正行时忽然看见牛魔王坐在中间慌得兴烘掀、掀烘兴扑的跪下道:“老大王爷爷在这里也。”那云里雾、雾里云、急如火、快如风都是肉眼凡胎那里认得真假也就一同跪倒磕头道:“爷爷!小的们是火云洞圣婴大王处差来请老大王爷爷去吃唐僧肉寿延千纪哩。”行者借口答道:“孩儿们起来同我回家去换了衣服来也。”小妖叩头道:“望爷爷方便不消回府罢。路程遥远恐我大王见责小的们就此请行。”行者笑道:“好乖儿女也罢也罢向前开路我和你去来。”六怪抖擞精神向前喝路大圣随后而来。

不多时早到了本处。快如风、急如火撞进洞里报:“大王老大王爷爷来了。”妖王欢喜道:“你们却中用这等来的快。”

即便叫:“各路头目摆队伍开旗鼓迎接老大王爷爷。”满洞群妖遵依旨令齐齐整整摆将出去。这行者昂昂烈烈挺着胸脯把身子抖了一抖却将那架鹰犬的毫毛都收回身上拽开大步径走入门里坐在南面当中。红孩儿当面跪下朝上叩头道:“父王孩儿拜揖。”行者道:“孩儿免礼。”那妖王四大拜拜毕立于下手。行者道:“我儿请我来有何事?”妖王躬身道:

“孩儿不才昨日获得一人乃东土大唐和尚。常听得人讲他是一个十世修行之人有人吃他一块肉寿似蓬瀛不老仙。愚男不敢自食特请父王同享唐僧之肉寿延千纪。”行者闻言打了个失惊道:“我儿是那个唐僧?”妖王道:“是往西天取经的人也。”行者道:“我儿可是孙行者师父么?”妖王道:“正是。”行者摆手摇头道:“莫惹他!莫惹他!别的还好惹孙行者是那样人哩我贤郎你不曾会他?那猴子神通广大变化多端。他曾大闹天宫玉皇上帝差十万天兵布下天罗地网也不曾捉得他。你怎么敢吃他师父!快早送出去还他不要惹那猴子。他若打听着你吃了他师父他也不来和你打他只把那金箍棒往山腰里搠个窟窿连山都掬了去。我儿弄得你何处安身教我倚靠何人养老!”妖王道:“父王说那里话长他人志气灭孩儿的威风。那孙行者共有兄弟三人领唐僧在我半山之中被我使个变化将他师父摄来。他与那猪八戒当时寻到我的门前讲甚么攀亲托熟之言被我怒冲天与他交战几合也只如此不见甚么高作。那猪八戒刺邪里就来助战是孩儿吐出三昧真火把他烧败了一阵。慌得他去请四海龙王助雨又不能灭得我三昧真火被我烧了一个小昏连忙着猪八戒去请南海观音菩萨。是我假变观音把猪八戒赚来见吊在如意袋中也要蒸他与众小的们吃哩。那行者今早又来我的门吆喝我传令教拿他慌得他把包袱都丢下走了。却才去请父王来看看唐僧活像方可蒸与你吃延寿长生不老也。”行者笑道:“我贤郎啊你只知有三昧火赢得他不知他有七十二般变化哩!”妖王道:“凭他怎么变化我也认得谅他决不敢进我门来。”行者道:“我儿你虽然认得他他却不变大的如狼犺大象恐进不得你门;他若变作小的你却难认。”妖王道:

“凭他变甚小的我这里每一层门上有四五个小妖把守他怎生得入!”行者道:“你是不知他会变苍蝇、蚊子、虼蚤或是蜜蜂、蝴蝶并蟭蟟虫等项又会变我模样你却那里认得?”妖王道:“勿虑他就是铁胆铜心也不敢近我门来也。”行者道:“既如此说贤郎甚有手段实是敌得他过方来请我吃唐僧的肉奈何我今日还不吃哩。”妖王道:“如何不吃?”行者道:“我近来年老你母亲常劝我作些善事。我想无甚作善且持些斋戒。”

妖王道:“不知父王是长斋是月斋?”行者道:“也不是长斋也不是月斋唤做雷斋每月只该四日。”妖王问:“是那四日?”行者道:“三辛逢初六。今朝是辛酉日一则当斋二来酉不会客。

且等明日我去亲自刷洗蒸他与儿等同享罢。”那妖王闻言心中暗想道:“我父王平日吃人为生今活彀有一千余岁怎么如今又吃起斋来了?想当初作恶多端这三四日斋戒那里就积得过来?此言有假可疑!可疑!”即抽身走出二门之下叫六健将来问:“你们老大王是那里请来的?”小妖道:“是半路请来的。”妖王道:“我说你们来的快不曾到家么?”小妖道:“是不曾到家。”妖王道:“不好了!着了他假也!这不是老大王!”小妖一齐跪下道:“大王自家父亲也认不得?”妖王道:“观其形容动静都象只是言语不象只怕着了他假吃了人亏。你们都要仔细:会使刀的刀要出鞘会使枪的枪要磨明会使棍的使棍会使绳的使绳。待我再去问他看他言语如何。若果是老大王莫说今日不吃明日不吃便迟个月何妨!假若言语不对只听我哏的一声就一齐下手。”群魔各各领命讫。

这妖王复转身到于里面对行者当面又拜。行者道:“孩儿家无常礼不须拜但有甚话只管说来。”妖王伏于地下道:“愚男一则请来奉献唐僧之肉二来有句话儿上请。我前日闲行驾祥光直至九霄空内忽逢着祖延道龄张先生。”行者道:“可是做天师的张道龄么?”妖王道:“正是。”行者问曰:“有甚话说?”妖王道:“他见孩儿生得五官周正三停平等他问我是几年、那月、那日、那时出世儿因年幼记得不真。先生子平精熟要与我推看五星今请父王正欲问此。倘或下次再得会他好烦他推算。”行者闻言坐在上面暗笑道:“好妖怪呀!老孙自归佛果保唐师父一路上也捉了几个妖精不似这厮克剥。他问我甚么家长礼短少米无柴的话说我也好信口捏脓答他。他如今问我生年月日我却怎么知道!”好猴王也十分乖巧巍巍端坐中间也无一些儿惧色面上反喜盈盈的笑道:

“贤郎请起我因年老连日有事不遂心怀把你生时果偶然忘了。且等到明日回家问你母亲便知。”妖王道:“父王把我八个字时常不离口论说说我有同天不老之寿怎么今日一旦忘了!岂有此理!必是假的!”哏的一声群妖枪刀簇拥望行者没头没脸的札来。这大圣使金箍棒架住了现出本象对妖精道:“贤郎你却没理。那里儿子好打爷的?”那妖王满面羞惭。

不敢回视。行者化金光走出他的洞府。小妖道:“大王孙行者走了。”妖王道:“罢罢罢!让他走了罢!我吃他这一场亏也!

且关了门莫与他打话只来刷洗唐僧蒸吃便罢。”

却说那行者搴着铁棒呵呵大笑自涧那边而来。沙僧听见急出林迎着道:“哥啊这半日方回如何这等哂笑想救出师父来也?”行者道:“兄弟虽不曾救得师父老孙却得个上风来了。”沙僧道:“甚么上风?”行者道:“原来猪八戒被那怪假变观音哄将回来吊于皮袋之内。我欲设法救援不期他着甚么六健将去请老大王来吃师父肉。是老孙想着他老大王必是牛魔王就变了他的模样充将进去坐在中间。他叫父王我就应他;他便叩头我就直受着实快活!果然得了上风!”沙僧道:“哥啊你便图这般小便宜恐师父性命难保。”行者道:“不须虑等我去请菩萨来。”沙僧道:“你还腰疼哩。”行者道:“我不疼了。古人云人逢喜事精神爽。你看着行李马匹等我去。”

沙僧道:“你置下仇了恐他害我师父。你须快去快来。”行者道:“我来得快只消顿饭时就回来矣。”

好大圣说话间躲离了沙僧纵筋斗云径投南海。在那半空里那消半个时辰望见普陀山景。须臾按下云头直至落伽崖上端肃正行只见二十四路诸天迎着道:“大圣那里去?”

行者作礼毕道:“要见菩萨。”诸天道:“少停容通报。”时有鬼子母诸天来潮音洞外报道:“菩萨得知孙悟空特来参见。”菩萨闻报即命进去。大圣敛衣皈命捉定步径入里边见菩萨倒身下拜。菩萨道:“悟空你不领金蝉子西方求经去却来此何干?”行者道:“上告菩萨弟子保护唐僧前行至一方乃号山枯松涧火云洞。有一个红孩儿妖精唤作圣婴大王把我师父摄去是弟子与猪悟能等寻至门前与他交战。他放出三昧火来我等不能取胜救不出师父。急上东洋大海请到四海龙王施雨水又不能胜火把弟子都熏坏了几乎丧了残生。”菩萨道:“既他是三昧火神通广大怎么去请龙王不来请我?”

行者道:“本欲来的只是弟子被烟熏了不能驾云却教猪八戒来请菩萨。”菩萨道:“悟能不曾来呀。”行者道:“正是。未曾到得宝山被那妖精假变做菩萨模样把猪八戒又赚入洞中现吊在一个皮袋里也要蒸吃哩。”菩萨听说心中大怒道:“那泼妖敢变我的模样!”恨了一声将手中宝珠净瓶往海心里扑的一掼唬得那行者毛骨竦然即起身侍立下面道:“这菩萨火性不退好是怪老孙说的话不好坏了他的德行就把净瓶掼了。可惜!可惜!早知送了我老孙却不是一件大人事?”说不了只见那海当中翻波跳浪钻出个瓶来原来是一个怪物驮着出来。行者仔细看那驮瓶的怪物怎生模样:根源出处号帮泥水底增光独显威。世隐能知天地性安藏偏晓鬼神机。藏身一缩无头尾展足能行快似飞。文王画卦曾元卜常纳庭台伴伏羲。云龙透出千般俏号水推波把浪吹。条条金线穿成甲点点装成彩玳瑁。九宫八卦袍披定散碎铺遮绿灿衣。生前好勇龙王幸死后还驮佛祖碑。要知此物名和姓兴风作浪恶乌龟。那龟驮着净瓶爬上崖边对菩萨点头二十四点权为二十四拜。行者见了暗笑道:“原来是看瓶的想是不见瓶就问他要。”菩萨道:“悟空你在下面说甚么?”行者道:“没说甚么。”

菩萨教:“拿上瓶来。”这行者即去拿瓶唉!莫想拿得他动。好便似蜻蜓撼石柱怎生摇得半分毫?行者上前跪下道:“菩萨弟子拿不动。”菩萨道:“你这猴头只会说嘴瓶儿你也拿不动怎么去降妖缚怪?”行者道:“不瞒菩萨说平日拿得动今日拿不动。想是吃了妖精亏筋力弱了。”菩萨道:“常时是个空瓶如今是净瓶抛下海去这一时间转过了三江五湖八海四渎溪源潭洞之间共借了一海水在里面。你那里有架海的斤量?此所以拿不动也。”行者合掌道:“是弟子不知。”那菩萨走上前将右手轻轻的提起净瓶托在左手掌上。只见那龟点点头钻下水去了。行者道:“原来是个养家看瓶的夯货!”菩萨坐定道:“悟空我这瓶中甘露水浆比那龙王的私雨不同能灭那妖精的三昧火。待要与你拿了去你却拿不动;待要着善财龙女与你同去你却又不是好心专一只会骗人。你见我这龙女貌美净瓶又是个宝物你假若骗了去却那有工夫又来寻你?你须是留些甚么东西作当。”行者道:“可怜!菩萨这等多心我弟子自秉沙门一向不干那样事了。你教我留些当头却将何物?我身上这件绵布直裰还是你老人家赐的。这条虎皮裙子能值几个铜钱?这根铁棒早晚却要护身。但只是头上这个箍儿是个金的却又被你弄了个方法儿长在我头上取不下来。你今要当头情愿将此为当你念个松箍儿咒将此除去罢不然将何物为当?”菩萨道:“你好自在啊!我也不要你的衣服、铁棒、金箍只将你那脑后救命的毫毛拔一根与我作当罢。”行者道:“这毫毛也是你老人家与我的。但恐拔下一根就拆破群了又不能救我性命。”菩萨骂道:“你这猴子!你便一毛也不拔教我这善财也难舍。”行者笑道:“菩萨你却也多疑。正是不看僧面看佛面千万救我师父一难罢!”那菩萨逍遥欣喜下莲台云步香飘上石崖。只为圣僧遭障害要降妖怪救回来。孙大圣十分欢喜请观音出了潮音仙洞。诸天大神都列在普陀岩上。菩萨道:“悟空过海。”行者躬身道:“请菩萨先行。”菩萨道:“你先过去。”行者磕头道:“弟子不敢在菩萨面前施展。若驾筋斗云啊掀露身体恐菩萨怪我不敬。”菩萨闻言即着善财龙女去莲花池里劈一瓣莲花放在石岩下边水上教行者:“你上那莲花瓣儿我渡你过海。”行者见了道:“菩萨这花瓣儿又轻又薄如何载得我起!这一躧翻跌下水去却不湿了虎皮裙?走了硝天冷怎穿!”菩萨喝道:“你且上去看!”行者不敢推辞舍命往上跳。果然先见轻小到上面比海船还大三分行者欢喜道:“菩萨载得我了。”菩萨道:“既载得如何不过去?”行者道:“又没了篙桨篷桅怎生得过?”菩萨道:“不用。”只把他一口气吹开吸拢又着实一口气吹过南洋苦海得登彼岸。行者却脚躧实地笑道:“这菩萨卖弄神通把老孙这等呼来喝去全不费力也!”

那菩萨吩咐概众诸天各守仙境着善财龙女闭了洞门他却纵祥云躲离普陀岩到那边叫:“惠岸何在?”惠岸乃托塔李天王第二个太子俗名木叉是也乃菩萨亲传授的徒弟不离左左称为护法惠岸行者即对菩萨合掌伺候。菩萨道:“你快上界去见你父王问他借王罡刀来一用。”惠岸道:“师父用着几何?”菩萨道:“全副都要。”惠岸领命即驾云头径入南天门里到云楼宫殿见父王下拜。天王见了问:“儿从何来?”木叉道:“师父是孙悟空请来降妖着儿拜上父王将天罡刀借了一用。”天王即唤哪吒将刀取三十六把递与木叉。木叉对哪吒说:“兄弟你回去多拜上母亲:我事紧急等送刀来再磕头罢。”忙忙相别按落祥光径至南海将刀捧与菩萨。菩萨接在手中抛将去念个咒语只见那刀化作一座千叶莲台。菩萨纵身上去端坐在中间。行者在旁暗笑道:“这菩萨省使俭用那莲花池里有五色宝莲台舍不得坐将来却又问别人去借。”菩萨道:“悟空休言语跟我来也。”却才都驾着云头离了海上。

白鹦哥展翅前飞孙大圣与惠岸随后。

顷刻间早见一座山头行者道:“这山就是号山了。从此处到那妖精门约摸有四百余里。”菩萨闻言即命住下祥云在那山头上念一声“唵”字咒语只见那山左山右走出许多神鬼却乃是本山土地众神都到菩萨宝莲座下磕头。菩萨道:“汝等俱莫惊张我今来擒此魔王。你与我把这团围打扫干净要三百里远近地方不许一个生灵在地。将那窝中小兽窟内雏虫都送在巅峰之上安生。”众神遵依而退。须臾间又来回复菩萨道:“既然干净俱各回祠。”遂把净瓶扳倒唿喇喇倾出水来就如雷响。真个是:漫过山头冲开石壁。漫过山头如海势冲开石壁似汪洋。黑雾涨天全水气沧波影日幌寒光。

遍崖冲玉浪满海长金莲。菩萨大展降魔法袖中取出定身禅。

化做落伽仙景界真如南海一般般。秀蒲挺出昙花嫩香草舒开贝叶鲜。紫竹几竿鹦鹉歇青松数簇鹧鸪喧。万迭波涛连四野只闻风吼水漫天。孙大圣见了暗中赞叹道:“果然是一个大慈大悲的菩萨!若老孙有此法力将瓶儿望山一倒管甚么禽兽蛇虫哩!”菩萨叫:“悟空伸手过来。”行者即忙敛袖将左手伸出。菩萨拔杨柳枝蘸甘露把他手心里写一个迷字教他:“捏着拳头快去与那妖精索战许败不许胜。败将来我这跟前我自有法力收他。”行者领命返云光径来至洞口一只手使拳一只手使棒高叫道:“妖怪开门!”那些小妖又进去报道:“孙行者又来了!”妖王道:“紧关了门!莫睬他!”行者叫道:“好儿子!把老子赶在门外还不开门!”小妖又报道:“孙行者骂出那话儿来了!”妖王只教:“莫睬他!”行者叫两次见不开门心中大怒举铁棒将门一下打了一个窟窿。慌得那小妖跌将进去道:“孙行者打破门了!”妖王见报几次又听说打破前门急纵身跳将出去挺长枪对行者骂道:“这猴子老大不识起倒!我让你得些便宜你还不知尽足又来欺我!打破我门你该个甚么罪名?”行者道:“我儿你赶老子出门你该个甚么罪名?”那妖王羞怒绰长枪劈胸便刺;这行者举铁棒架隔相还。一番搭上手斗经四五个回合行者捏着拳头拖着棒败将下来。那妖王立在山前道:“我要刷洗唐僧去哩!”行者道:“好儿子天看着你哩!你来!”那妖精闻言愈加嗔怒喝一声赶到面前挺枪又刺。这行者轮棒又战几合败阵又走。那妖王骂道:“猴子你在前有二三十合的本事你怎么如今正斗时就要走了何也?”行者笑道:“贤郎老子怕你放火。”妖精道:“我不放火了你上来。”行者道:“既不放火走开些好汉子莫在家门前打人。”那妖精不知是诈真个举枪又赶。行者拖了棒放了拳头那妖王着了迷乱只情追赶。前走的如流星过度后走的如弩箭离弦。

不一时望见那菩萨了。行者道:“妖精我怕你了你饶我罢。你如今赶至南海观音菩萨处怎么还不回去?”那妖王不信咬着牙只管赶来。行者将身一幌藏在那菩萨的神光影里。这妖精见没了行者走近前睁圆眼对菩萨道:“你是孙行者请来的救兵么?”菩萨不答应。妖王拈转长枪喝道:“咄!你是孙行者请来的救兵么?”菩萨也不答应。妖精望菩萨劈心刺一枪来那菩萨化道金光径走上九霄空内。行者跟定道:“菩萨你好欺伏我罢了!那妖精再三问你你怎么推聋装哑不敢做声被他一枪搠走了却把那个莲台都丢下耶!”菩萨只教:

“莫言语看他再要怎的。”此时行者与木叉俱在空中并肩同看。只见那妖呵呵冷笑道:“泼猴头错认了我也!他不知把我圣婴当作个甚人。几番家战我不过又去请个甚么脓包菩萨来却被我一枪搠得无形无影去了又把个宝莲台儿丢了且等我上去坐坐。”好妖精他也学菩萨盘手盘脚的坐在当中。

行者看见道:“好!好!好!莲花台儿好送人了!”菩萨道:“悟空你又说甚么?”行者道:“说甚?说甚?莲台送了人了!”那妖精坐放臀下终不得你还要哩?”菩萨道:“正要他坐哩。”行者道:“他的身躯小巧比你还坐得稳当。”菩萨叫:“莫言语且看法力。”他将杨柳枝往下指定叫一声“退!”只见那莲台花彩俱无祥光尽散原来那妖王坐在刀尖之上。即命木叉:“使降妖杵把刀柄儿打打去来。”那木叉按下云头将降魔杵如筑墙一般筑了有千百余下。那妖精穿通两腿刀尖出血流成汪皮肉开。好怪物你看他咬着牙忍着痛且丢了长枪用手将刀乱拔。行者却道:“菩萨啊那怪物不怕痛还拔刀哩。”菩萨见了唤上木叉“且莫伤他生命。”却又把杨柳枝垂下念声“唵”字咒语那天罡刀都变做倒须钩儿狼牙一般莫能褪得。那妖精却才慌了扳着刀尖痛声苦告道:“菩萨我弟子有眼无珠不识你广大法力。千乞垂慈饶我性命!再不敢恃恶愿入法门戒行也。”菩萨闻言却与二行者、白鹦哥低下金光到了妖精面前问道:“你可受吾戒行么?”妖王点头滴泪道:“若饶性命愿受戒行。”菩萨道:“你可入我门么?”妖王道:“果饶性命愿入法门。”菩萨道:“既如此我与你摩顶受戒。”就袖中取出一把金剃头刀儿近前去把那怪分顶剃了几刀剃作一个太山压顶与他留下三个顶搭挽起三个窝角揪儿。行者在旁笑道:“这妖精大晦气!弄得不男不女不知象个甚么东西!”菩萨道:“你今既受我戒我却也不慢你称你做善财童子如何?”

那妖点头受持只望饶命。菩萨却用手一指叫声“退!”撞的一声天罡刀都脱落尘埃那童子身躯不损。菩萨叫:“惠岸你将刀送上天宫还你父王莫来接我先到普陀岩会众诸天等候。”那木叉领命送刀上界回海不题。

却说那童子野性不定见那腿疼处不疼臀破处不破头挽了三个揪儿他走去绰起长枪望菩萨道:“那里有甚真法力降我!原来是个掩样术法儿!不受甚戒看枪!”望菩萨劈脸刺来。恨得个行者轮铁棒要打菩萨只叫:“莫打我自有惩治。”

却又袖中取出一个金箍儿来道:“这宝贝原是我佛如来赐我往东土寻取经人的金紧禁三个箍儿。紧箍儿先与你戴了禁箍儿收了守山大神这个金箍儿未曾舍得与人今观此怪无礼与他罢。”好菩萨将箍儿迎风一幌叫声“变!”即变作五个箍儿望童子身上抛了去喝声“着!”一个套在他头顶上两个套在他左右手上两个套在他左右脚上。菩萨道:“悟空走开些等我念念《金箍儿咒》。”行者慌了道:“菩萨呀请你来此降妖如何却要咒我?”菩萨道:“这篇咒不是《紧箍儿咒》咒你的是《金箍儿咒》咒那童子的。”行者却才放心紧随左右听得他念咒。菩萨捻着诀默默的念了几遍那妖精搓耳揉腮攒蹄打滚。正是:一句能通遍沙界广大无边法力深。毕竟不知那童子怎的皈依且听下回分解。

\chapter[黑河妖孽擒僧去\ 西洋龙子捉鼍回]{黑河妖孽擒僧去\\西洋龙子捉鼍回}
\chapter[法身元运逢车力\ 心正妖邪度脊关]{法身元运逢车力\\心正妖邪度脊关}
\chapter[三清观大圣留名\ 车迟国猴王显法]{三清观大圣留名\\车迟国猴王显法}
\chapter[外道弄强欺正法\ 心猿显圣灭诸邪]{外道弄强欺正法\\心猿显圣灭诸邪}

第四十六回 外道弄强欺正法 心猿显圣灭诸邪

话说那国王见孙行者有呼龙使圣之法即将关文用了宝印便要递与唐僧放行西路。那三个道士慌得拜倒在金銮殿上启奏那皇帝即下龙位御手忙搀道:“国师今日行此大礼何也?”道士说:“陛下我等至此匡扶社稷保国安民苦历二十年来今日这和尚弄法力抓了功去败了我们声名陛下以一场之雨就恕杀人之罪可不轻了我等也?望陛下且留住他的关文让我兄弟与他再赌一赌看是何如。”那国王着实昏乱东说向东西说向西真个收了关文道:“国师你怎么与他赌?”虎力大仙道:“我与他赌坐禅。”国王道:“国师差矣那和尚乃禅教出身必然先会禅机才敢奉旨求经你怎与他赌此?”大仙道:“我这坐禅比常不同有一异名教做云梯显圣。”国王道:“何为云梯显圣?”大仙道:“要一百张桌子五十张作一禅台一张一张迭将起去不许手攀而上亦不用梯凳而登各驾一朵云头上台坐下约定几个时辰不动。”国王见此有些难处就便传旨问道:“那和尚我国师要与你赌云梯显圣坐禅那个会么?”行者闻言沉吟不答。八戒道:“哥哥怎么不言语?”行者道:“兄弟实不瞒你说若是踢天弄井搅海翻江担山赶月换斗移星诸般巧事我都干得;就是砍头剁脑剖腹剜心异样腾那却也不怕。但说坐禅我就输了我那里有这坐性?你就把我锁在铁柱子上我也要上下爬蹅莫想坐得住。”三藏忽的开言道:“我会坐禅。”行者欢喜道:“却好却好!

可坐得多少时?”三藏道:“我幼年遇方上禅僧讲道那性命根本上定性存神在死生关里也坐二三个年头。”行者道:“师父若坐二三年我们就不取经罢。多也不上二三个时辰就下来了。”三藏道:“徒弟呀却是不能上去。”行者道:“你上前答应我送你上去。”那长老果然合掌当胸道:“贫僧会坐禅。”国王教传旨立禅台。国家有倒山之力不消半个时辰就设起两座台在金銮殿左右。

那虎力大仙下殿立于阶心将身一纵踏一朵席云径上西边台上坐下。行者拔一根毫毛变做假象陪着八戒沙僧立于下面他却作五色祥云把唐僧撮起空中径至东边台上坐下。他又敛祥光变作一个蟭蟟虫飞在八戒耳朵边道:“兄弟仔细看着师父再莫与老孙替身说话。”那呆子笑道:“理会得!

理会得!”却说那鹿力大仙在绣墩上坐看多时他两个在高台上不分胜负这道士就助他师兄一功:将脑后短拔了一根捻着一团弹将上去径至唐僧头上变作一个大臭虫咬住长老。那长老先前觉痒然后觉疼。原来坐禅的不许动手动手算输一时间疼痛难禁他缩着头就着衣襟擦痒。八戒道:“不好了!师父羊儿风了。”沙僧道:“不是是头风了。”

行者听见道:“我师父乃志诚君子他说会坐禅断然会坐说不会只是不会。君子家岂有谬乎?你两个休言等我上去看看。”好行者嘤的一声飞在唐僧头上只见有豆粒大小一个臭虫叮他师父慌忙用手捻下替师父挠挠摸摸。那长老不疼不痒端坐上面。行者暗想道:“和尚头光虱子也安不得一个如何有此臭虫?想是那道士弄的玄虚害我师父。哈哈!枉自也不见输赢等老孙去弄他一弄!”这行者飞将去金殿兽头上落下摇身一变变作一条七寸长的蜈蚣径来道士鼻凹里叮了一下。那道士坐不稳一个筋斗翻将下去几乎丧了性命幸亏大小官员人多救起。国王大惊即着当驾太师领他往文华殿里梳洗去了。行者仍驾祥云将师父驮下阶前已是长老得胜。

那国王只教放行鹿力大仙又奏道:“陛下我师兄原有暗风疾因到了高处;冒了天风旧疾举故令和尚得胜。且留下他等我与他赌隔板猜枚。国王道:“怎么叫做隔板猜枚?”鹿力道:“贫道有隔板知物之法;看那和尚可能彀。他若猜得过我让他出去;猜不着凭陛下问拟罪名雪我昆仲之恨不污了二十年保国之恩也。”真个那国王十分昏乱依此谗言。即传旨将一朱红漆的柜子命内官抬到宫殿教娘娘放上件宝贝。

须臾抬出放在白玉阶前教僧道:“你两家各赌法力猜那柜中是何宝贝。”三藏道:“徒弟柜中之物如何得知?”行者敛祥光还变作蟭蟟虫钉在唐僧头上道:“师父放心等我去看看来。”好大圣轻轻飞到柜上爬在那柜脚之下见有一条板缝儿。他钻将进去见一个红漆丹盘内放一套宫衣乃是山河社稷袄乾坤地理裙。用手拿起来抖乱了咬破舌尖上一口血哨喷将去叫声“变”!即变作一件破烂流丢一口钟临行又撒上一泡臊溺却还从板缝里钻出来飞在唐僧耳朵上道:“师父你只猜是破烂流丢一口钟。”三藏道:“他教猜宝贝哩流丢是件甚宝贝?”行者道:“莫管他只猜着便是。”唐僧进前一步正要猜那鹿力大仙道:“我先猜那柜里是山河社稷袄乾坤地理裙。”唐僧道:“不是不是柜里是件破烂流丢一口钟。”国王道:“这和尚无礼!敢笑我国中无宝猜甚么流丢一口钟!”

教:“拿了!”那两班校尉就要动手慌得唐僧合掌高呼:“陛下且赦贫僧一时待打开柜看。端的是宝贫僧领罪;如不是宝却不屈了贫僧也?”国王教打开看。当驾官即开了捧出丹盘来看果然是件破烂流丢一口钟。国王大怒道:“是谁放上此物?”龙座后面闪上三宫皇后道:“我主是梓童亲手放的山河社稷袄乾坤地理裙却不知怎么变成此物。”国王道:“御妻请退寡人知之。宫中所用之物无非是缎绢绫罗那有此甚么流丢?”教:“抬上柜来等朕亲藏一宝贝再试如何。”

那皇帝即转后宫把御花园里仙桃树上结得一个大桃子有碗来大小摘下放在柜内又抬下叫猜。唐僧道:“徒弟啊又来猜了。”行者道:“放心等我再去看看。”又嘤的一声飞将去还从板缝儿钻进去见是一个桃子正合他意即现了原身坐在柜里将桃子一顿口啃得干干净净连两边腮凹儿都啃净了将核儿安在里面。仍变蟭蟟虫飞将出去钉在唐僧耳朵上道:“师父只猜是个桃核子。”长老道:“徒弟啊休要弄我。先前不是口快几乎拿去典刑。这番须猜宝贝方好桃核子是甚宝贝?”行者道:“休怕只管赢他便了。”三藏正要开言听得那羊力大仙道:“贫道先猜是一颗仙桃。”三藏猜道:“不是桃是个光桃核子。”那国王喝道:“是朕放的仙桃如何是核?三国师猜着了。”三藏道:“陛下打开来看就是。”当驾官又抬上去打开捧出丹盘果然是一个核子皮肉俱无。国王见了心惊道:

“国师休与他赌斗了让他去罢。寡人亲手藏的仙桃如今只是一核子是甚人吃了?想是有鬼神暗助他也。”八戒听说与沙僧微微冷笑道:“还不知他是会吃桃子的积年哩!”

正话间只见那虎力大仙从文华殿梳洗了走上殿前:“陛下这和尚有搬运抵物之术抬上柜来我破他术法与他再猜。”国王道:“国师还要猜甚?”虎力道:“术法只抵得物件却抵不得人身。将这道童藏在里面管教他抵换不得。”这小童果藏在柜里掩上柜盖抬将下去教:“那和尚再猜这三番是甚宝贝。”三藏道:“又来了!”行者道:“等我再去看看。”嘤的又飞去钻入里面见是一个小童儿。好大圣他却有见识果然是腾那天下少似这伶俐世间稀!他就摇身一变变作个老道士一般容貌进柜里叫声“徒弟。”童儿道:“师父你从那里来的?”行者道:“我使遁法来的。”童儿道:“你来有么教诲?”行者道:“那和尚看见你进柜来了他若猜个道童却不又输了?是特来和你计较计较剃了头我们猜和尚罢。”童儿道:“但凭师父处治只要我们赢他便了。若是再输与他不但低了声名又恐朝廷不敬重了。”行者道:“说得是。我儿过来赢了他我重重赏你。”将金箍棒就变作一把剃头刀搂抱着那童儿口里叫道:“乖乖忍着疼莫放声等我与你剃头。”须臾剃下来窝作一团塞在那柜脚纥络里收了刀儿摸着他的光头道:“我儿头便象个和尚只是衣裳不趁。脱下来我与你变一变。”那道童穿的一领葱白色云头花绢绣锦沿边的鹤氅真个脱下来被行者吹一口仙气叫“变!”即变做一件土黄色的直裰儿与他穿了。却又拔下两根毫毛变作一个木鱼儿递在他手里道:

“徒弟须听着但叫道童千万莫出去;若叫和尚你就与我顶开柜盖敲着木鱼念一卷佛经钻出来方得成功也。”童儿道:

“我只会念《三官经》、《北斗经》、《消灾经》不会念佛家经。”行者道:“你可会念佛?”童儿道:“阿弥陀佛那个不会念?”行者道:“也罢也罢就念佛省得我又教你。切记着我去也。”还变蟭蟟虫钻出去飞在唐僧耳轮边道:“师父你只猜是个和尚。”三藏道:“这番他准赢了。”行者道:“你怎么定得?”三藏道:“经上有云佛、法、僧三宝。和尚却也是一宝。”正说处只见那虎力大仙道:“陛下第三番是个道童。”只管叫他那里肯出来。三藏合掌道:“是个和尚。”八戒尽力高叫道:“柜里是个和尚!”那童儿忽的顶开柜盖敲着木鱼念着佛钻出来。喜得那两班文武齐声喝采:唬得那三个道士拑口无言。国王道:

“这和尚是有鬼神辅佐!怎么道士入柜就变做和尚?纵有待诏跟进去也只剃得头便了如何衣服也能趁体口里又会念佛?国师啊!让他去罢!”

虎力大仙道:“陛下左右是棋逢对手将遇良材。贫道将锺南山幼时学的武艺索性与他赌一赌。”国王道:“有甚么武艺?”虎力道:“弟兄三个都有些神通。会砍下头来又能安上;

剖腹剜心还再长完;滚油锅里又能洗澡。”国王大惊道:“此三事都是寻死之路!”虎力道:“我等有此法力才敢出此朗言断要与他赌个才休。”那国王叫道:“东土的和尚我国师不肯放你还要与你赌砍头剖腹下滚油锅洗澡哩。”行者正变作蟭蟟虫往来报事忽听此言即收了毫毛现出本相哈哈大笑道:“造化!造化!买卖上门了!”八戒道:“这三件都是丧性命的事怎么说买卖上门?”行者道:“你还不知我的本事。”八戒道:“哥哥你只象这等变化腾那也彀了怎么还有这等本事?”

行者道:“我啊砍下头来能说话剁了臂膊打得人。扎去腿脚会走路剖腹还平妙绝伦。就似人家包匾食一捻一个就囫囵。

油锅洗澡更容易只当温汤涤垢尘。”八戒沙僧闻言呵呵大笑。行者上前道:“陛下小和尚会砍头。”国王道:“你怎么会砍头?”行者道:“我当年在寺里修行曾遇着一个方上禅和子教我一个砍头法不知好也不好如今且试试新。”国王笑道:“那和尚年幼不知事砍头那里好试新?头乃六阳之砍下即便死矣。”虎力道:“陛下正要他如此方才出得我们之气。”那昏君信他言语即传旨教设杀场。

一声传旨即有羽林军三千摆列朝门之外。国王教:“和尚先去砍头。”行者欣然应道:“我先去!我先去!”拱着手高呼道:“国师恕大胆占先了。”拽回头往外就走。唐僧一把扯住道:“徒弟呀仔细些那里不是耍处。”行者道:“怕他怎的!撒了手等我去来。”那大圣径至杀场里面被刽子手挝住了捆做一团按在那土墩高处只听喊一声“开刀!”飕的把个头砍将下来又被刽子手一脚踢了去好似滚西瓜一般滚有三四十步远近。行者腔子中更不出血只听得肚里叫声:“头来!”慌得鹿力大仙见有这般手段即念咒语教本坊土地神祇:“将人头扯住待我赢了和尚奏了国王与你把小祠堂盖作大庙宇泥塑像改作正金身。”原来那些土地神祇因他有五雷法也服他使唤暗中真个把行者头按住了。行者又叫声:“头来!”那头一似生根莫想得动。行者心焦捻着拳挣了一挣将捆的绳子就皆挣断喝声:“长!”飕的腔子内长出一个头来。唬得那刽子手个个心惊;羽林军人人胆战。那监斩官急走入朝奏道:

“万岁那小和尚砍了头又长出一颗来了。”八戒冷笑道:“沙僧那知哥哥还有这般手段。”沙僧道:“他有七十二般变化就有七十二个头哩。”说不了行者走来叫声“师父。”三藏大喜道:“徒弟辛苦么?”行者道:“不辛苦倒好耍子。”八戒道:“哥哥可用刀疮药么?”行者道:“你是摸摸看可有刀痕?”那呆子伸手一摸就笑得呆呆睁睁道:“妙哉!妙哉!却也长得完全截疤儿也没些儿!”

兄弟们正都欢喜又听得国王叫领关文:“赦你无罪!快去!快去!”行者道:“关文虽领必须国师也赴曹砍砍头也当试新去来。”国王道:“大国师那和尚也不肯放你哩。你与他赌胜且莫唬了寡人。”虎力也只得去被几个刽子手也捆翻在地幌一幌把头砍下一脚也踢将去滚了有三十余步他腔子里也不出血也叫一声:“头来!”行者即忙拔下一根毫毛吹口仙气叫“变!”变作一条黄犬跑入场中把那道士头一口衔来径跑到御水河边丢下不题。却说那道士连叫三声人头不到怎似行者的手段长不出来腔子中骨都都红光迸出可怜空有唤雨呼风法怎比长生果正仙?须臾倒在尘埃众人观看乃是一只无头的黄毛虎。那监斩官又来奏:“万岁大国师砍下头来不能长出死在尘埃是一只无头的黄毛虎。”国王闻奏大惊失色目不转睛看那两个道士。鹿力起身道:“我师兄已是命到禄绝了如何是只黄虎!这都是那和尚惫懒使的掩样法儿将我师兄变作畜类!我今定不饶他定要与他赌那剖腹剜心!”

国王听说方才定性回神又叫:“那和尚二国师还要与你赌哩。”行者道:“小和尚久不吃烟火食前日西来忽遇斋公家劝饭多吃了几个馍馍这几日腹中作痛想是生虫正欲借陛下之刀剖开肚皮拿出脏腑洗净脾胃方好上西天见佛。”

国王听说教:“拿他赴曹。”那许多人搀的搀扯的扯。行者展脱手道:“不用人搀自家走去。但一件不许缚手我好用手洗刷脏腑。”国王传旨教:“莫绑他手。”行者摇摇摆摆径至杀场将身靠着大桩解开衣带露出肚腹。那刽子手将一条绳套在他膊项上一条绳札住他腿足把一口牛耳短刀幌一幌着肚皮下一割搠个窟窿。这行者双手爬开肚腹拿出肠脏来一条条理彀多时依然安在里面照旧盘曲捻着肚皮吹口仙气叫“长!”依然长合。国王大惊将他那关文捧在手中道:“圣僧莫误西行与你关文去罢。”行者笑道:“关文小可也请二国师剖剖剜剜何如?”国王对鹿力说:“这事不与寡人相干是你要与他做对头的请去请去。”鹿力道:“宽心料我决不输与他。”你看他也象孙大圣摇摇摆摆径入杀场被刽子手套上绳将牛耳短刀唿喇的一声割开肚腹他也拿出肝肠用手理弄。行者即拔一根毫毛吹口仙气叫“变!”即变作一只饿鹰展开翅爪飕的把他五脏心肝尽情抓去不知飞向何方受用。这道士弄做一个空腔破肚淋漓鬼少脏无肠浪荡魂。那刽子手蹬倒大桩拖尸来看呀!原来是一只白毛角鹿!

慌得那监斩官又来奏道:“二国师晦气正剖腹时被一只饿鹰将脏腑肝肠都刁去了。死在那里原身是个白毛角鹿也。”

国王害怕道:“怎么是个角鹿?”那羊力大仙又奏道:“我师兄既死如何得现兽形?这都是那和尚弄术法坐害我等。等我与师兄报仇者。”国王道:“你有甚么法力赢他?”羊力道:“我与他赌下滚油锅洗澡。”国王便教取一口大锅满着香油教他两个赌去。行者道:“多承下顾小和尚一向不曾洗澡这两日皮肤燥痒好歹荡荡去。”那当驾官果安下油锅架起干柴燃着烈火将油烧滚教和尚先下去。”行者合掌道:“不知文洗武洗?”国王道:“文洗如何?武洗如何?”行者道:“文洗不脱衣服似这般叉着手下去打个滚就起来不许污坏了衣服若有一点油腻算输。武洗要取一张衣架一条手巾脱了衣服跳将下去任意翻筋斗竖蜻蜓当耍子洗也。”国王对羊力说:“你要与他文洗武洗?”羊力道:“文洗恐他衣服是药炼过的隔油武洗罢。”行者又上前道:“恕大胆屡次占先了。”你看他脱了布直裰褪了虎皮裙将身一纵跳在锅内翻波斗浪就似负水一般顽耍。八戒见了咬着指头对沙僧道:“我们也错看了这猴子了!平时间劖言讪语斗他耍子怎知他有这般真实本事!”

他两个唧唧哝哝夸奖不尽。行者望见心疑道:“那呆子笑我哩!正是巧者多劳拙者闲老孙这般舞弄他倒自在。等我作成他捆一绳看他可怕。”正洗浴打个水花淬在油锅底上变作个枣核钉儿再也不起来了。那监斩官近前又奏:“万岁小和尚被滚油烹死了。”国王大喜教捞上骨骸来看。刽子手将一把铁笊篱在油锅里捞原来那笊篱眼稀行者变得钉小往往来来从眼孔漏下去了那里捞得着!又奏道:“和尚身微骨嫩俱札化了。”国王教:“拿三个和尚下去!”两边校尉见八戒面凶先揪翻把背心捆了慌得三藏高叫:“陛下赦贫僧一时。

我那个徒弟自从归教历历有功今日冲撞国师死在油锅之内奈何先死者为神我贫僧怎敢贪生!正是天下官员也管着天下百姓陛下若教臣死臣岂敢不死?只望宽恩赐我半盏凉浆水饭三张纸马容到油锅边烧此一陌纸也表我师徒一念那时再领罪也。”国王闻言道:“也是那中华人多有义气。”

命取些浆饭、黄钱与他。果然取了递与唐僧。唐僧教沙和尚同去行至阶下有几个校尉把八戒揪着耳朵拉在锅边三藏对锅祝曰:“徒弟孙悟空!自从受戒拜禅林护我西来恩爱深。指望同时成大道何期今日你归阴!生前只为求经意死后还存念佛心。万里英魂须等候幽冥做鬼上雷音!”八戒听见道:“师父不是这般祝了。沙和尚你替我奠浆饭等我祷。”那呆子捆在地下气呼呼的道:“闯祸的泼猴子无知的弼马温!

该死的泼猴子油烹的弼马温!猴儿了帐马温断根!”

孙行者在油锅底上听得那呆子乱骂忍不住现了本相赤淋淋的站在油锅底道:“馕糟的夯货!你骂那个哩!”唐僧见了道:“徒弟唬杀我也!”沙僧道:“大哥干净推佯死惯了!”慌得那两班文武上前来奏道:“万岁那和尚不曾死又打油锅里钻出来了。”监斩官恐怕虚诳朝廷却又奏道:“死是死了只是日期犯凶小和尚来显魂哩。”行者闻言大怒跳出锅来揩了油腻穿上衣服掣出棒挝过监斩官着头一下打做了肉团道:“我显甚么魂哩!”唬得多官连忙解了八戒跪地哀告:“恕罪!恕罪!”国王走下龙座。行者上殿扯住道:“陛下不要走且教你三国师也下下油锅去。”那皇帝战战兢兢道:“三国师你救朕之命快下锅去莫教和尚打我。”

羊力下殿照依行者脱了衣服跳下油锅也那般支吾洗浴。行者放了国王近油锅边叫烧火的添柴却伸手探了一把呀!那滚油都冰冷心中暗想道:“我洗时滚热他洗时却冷。我晓得了这不知是那个龙王在此护持他哩。”急纵身跳在空中念声“唵”字咒语把那北海龙王唤来:“我把你这个带角的蚯蚓有鳞的泥鳅!你怎么助道士冷龙护住锅底教他显圣赢我!”唬得那龙王喏喏连声道:“敖顺不敢相助。大圣原来不知这个孽畜苦修行了一场脱得本壳却只是五雷法真受其余都躧了旁门难归仙道。这个是他在小茅山学来的大开剥。那两个已是大圣破了他法现了本相这一个也是他自己炼的冷龙只好哄瞒世俗之人耍子怎瞒得大圣!小龙如今收了他冷龙管教他骨碎皮焦显什么手段。”行者道:“趁早收了免打!”那龙王化一阵旋风到油锅边将冷龙捉下海去不题。

行者下来与三藏、八戒、沙僧立在殿前见那道士在滚油锅里打挣爬不出来滑了一跌霎时间骨脱皮焦肉烂。监斩官又来奏道:“万岁三国师煠化了也。”那国王满眼垂泪手扑着御案放声大哭道:“人身难得果然难不遇真传莫炼丹。空有驱神咒水术却无延寿保生丸。圆明混怎涅槃徒用心机命不安。早觉这般轻折挫何如秘食稳居山!”这正是:点金炼汞成何济唤雨呼风总是空!毕竟不知师徒们怎的维持且听下回分解。

\chapter[圣僧夜阻通天水\ 金木垂慈救小童]{圣僧夜阻通天水\\金木垂慈救小童}
\chapter[魔弄寒风飘大雪\ 僧思拜佛履层冰]{魔弄寒风飘大雪\\僧思拜佛履层冰}
\chapter[三藏有灾沉水宅\ 观音救难现鱼篮]{三藏有灾沉水宅\\观音救难现鱼篮}
\chapter[情乱性从因爱欲\ 神昏心动遇魔头]{情乱性从因爱欲\\神昏心动遇魔头}
\chapter[心猿空用千般计\ 水火无功难炼魔]{心猿空用千般计\\水火无功难炼魔}
\chapter[悟空大闹金【山兜】洞\ 如来暗示主人公]{悟空大闹金【山兜】洞\\如来暗示主人公}
\chapter[禅主吞餐怀鬼孕\ 黄婆运水解邪胎]{禅主吞餐怀鬼孕\\黄婆运水解邪胎}

第五十三回 禅主吞餐怀鬼孕 黄婆运水解邪胎

德行要修八百阴功须积三千。均平物我与亲冤始合西天本愿。魔兕刀兵不怯空劳水火无愆。老君降伏却朝天笑把青牛牵转。话说那大路旁叫唤者谁?乃金皘山山神土地捧着紫金钵盂叫道:“圣僧啊这钵盂饭是孙大圣向好处化来的。

因你等不听良言误入妖魔之手致令大圣劳苦万端今日方救得出。且来吃了饭再去走路莫孤负孙大圣一片恭孝之心也。”三藏道:“徒弟万分亏你!言谢不尽!早知不出圈痕那有此杀身之害。”行者道:“不瞒师父说只因你不信我的圈子却教你受别人的圈子。多少苦楚可叹!可叹!”八戒道:“怎么又有个圈子。”行者道:“都是你这孽嘴孽舌的夯货弄师父遭此一场大难!着老孙翻天覆地请天兵水火与佛祖丹砂尽被他使一个白森森的圈子套去。如来暗示了罗汉对老孙说出那妖的根原才请老君来收伏却是个青牛作怪。”三藏闻言感激不尽道:“贤徒今番经此下次定然听你吩咐。”遂此四人分吃那饭那饭热气腾腾的。行者道:“这饭多时了却怎么还热?”土地跪下道:“是小神知大圣功完才自热来伺候。”须臾饭毕收拾了钵盂辞了土地山神。

那师父才攀鞍上马过了高山。正是涤虑洗心皈正觉餐风宿水向西行。行彀多时又值早春天气听了些“紫燕呢喃黄鹂睍睆。紫燕呢喃香嘴困黄鹂襕睆巧音频。满地落红如布锦遍山翠似堆茵。岭上青梅结豆崖前古柏留云。野润烟光淡沙暄日色曛。几处园林花放蕊阳回大地柳芽新。正行处忽遇一道小河澄澄清水湛湛寒波。唐长老勒过马观看远见河那边有柳阴垂碧微露着茅屋几椽。行者遥指那厢道:

“那里人家一定是摆渡的。”三藏道:“我见那厢也似这般却不见船只未敢开言。”八戒旋下行李厉声高叫道:“摆渡的!

撑船过来!”连叫几遍只见那柳阴里面咿咿哑哑的撑出一只船儿。不多时相近这岸。师徒们仔细看了那船儿真个是:

短棹分波轻桡泛浪。瞰堂油漆彩艎板满平仓。船头上铁缆盘窝船后边舵楼明亮。虽然是一苇之航也不亚泛湖浮海。纵无锦缆牙樯实有松桩桂楫。固不如万里神舟真可渡一河之隔。往来只在两崖边出入不离古渡口。那船儿须臾顶岸有梢子叫云:“过河的这里去。”三藏纵马近前看处那梢子怎生模样:头裹锦绒帕足踏皂丝鞋。身穿百纳绵裆袄腰束千针裙布衫。手腕皮粗筋力硬眼花眉皱面容衰。声音娇细如莺啭近观乃是老裙钗。行者近于船边道:“你是摆渡的?”那妇人道:

“是。”行者道:“梢公如何不在却着梢婆撑船?”妇人微笑不答用手拖上跳板。沙和尚将行李挑上去行者扶着师父上跳然后顺过船来八戒牵上白马收了跳板。那妇人撑开船摇动桨顷刻间过了河。

身登西岸长老教沙僧解开包取几文钱钞与他。妇人更不争多寡将缆拴在傍水的桩上笑嘻嘻径入庄屋里去了。三藏见那水清一时口渴便着八戒:“取钵盂舀些水来我吃。”

那呆子道:“我也正要些儿吃哩。”即取钵盂舀了一钵递与师父。师父吃了有一少半还剩了多半呆子接来一气饮干却伏侍三藏上马。师徒们找路西行不上半个时辰那长老在马上呻吟道:“腹痛!”八戒随后道:“我也有些腹痛。”沙僧道:“想是吃冷水了?”说未毕师父声唤道:“疼的紧!”八戒也道:“疼得紧!”他两个疼痛难禁渐渐肚子大了。用手摸时似有血团肉块不住的骨冗骨冗乱动。三藏正不稳便忽然见那路旁有一村舍树梢头挑着两个草把。行者道:“师父好了那厢是个卖酒的人家。我们且去化他些热汤与你吃就问可有卖药的讨贴药与你治治腹痛。”三藏闻言甚喜却打白马不一时到了村舍门口下马。但只见那门儿外有一个老婆婆端坐在草墩上绩麻。行者上前打个问讯道:“婆婆贫僧是东土大唐来的我师父乃唐朝御弟。因为过河吃了河水觉肚腹疼痛。”那婆婆喜哈哈的道:“你们在那边河里吃水来?”行者道:“是在此东边清水河吃的。”那婆婆欣欣的笑道:“好耍子!好耍子!你都进来我与你说。”

行者即搀唐僧沙僧即扶八戒两人声声唤唤腆着肚子一个个只疼得面黄眉皱入草舍坐下行者只叫:“婆婆是必烧些热汤与我师父我们谢你。”那婆婆且不烧汤笑唏唏跑走后边叫道:“你们来看!你们来看!”那里面蹼烤蹼踏的又走出两三个半老不老的妇人都来望着唐僧洒笑。行者大怒喝了一声把牙一嗟唬得那一家子跌跌蹡蹡往后就走。行者上前扯住那老婆子道:“快早烧汤我饶了你!”那婆子战兢兢的道:“爷爷呀我烧汤也不济事也治不得他两个肚疼。你放了我等我说。”行者放了他他说:“我这里乃是西梁女国。我们这一国尽是女人更无男子故此见了你们欢喜。你师父吃的那水不好了那条河唤做子母河我那国王城外还有一座迎阳馆驿驿门外有一个照胎泉。我这里人但得年登二十岁以上方敢去吃那河里水。吃水之后便觉腹痛有胎。至三日之后到那迎阳馆照胎水边照去。若照得有了双影便就降生孩儿。你师吃了子母河水以此成了胎气也不日要生孩子热汤怎么治得?”三藏闻言大惊失色道:“徒弟啊!似此怎了?”八戒扭腰撒胯的哼道:“爷爷呀!要生孩子我们却是男身!那里开得产门?如何脱得出来。”行者笑道:“古人云瓜熟自落若到那个时节一定从胁下裂个窟窿钻出来也。”八戒见说战兢兢忍不得疼痛道:“罢了罢了!死了死了!”沙僧笑道:“二哥莫扭莫扭!只怕错了养儿肠弄做个胎前病。”那呆子越慌了眼中噙泪。扯着行者道:“哥哥!你问这婆婆看那里有手轻的稳婆预先寻下几个这半会一阵阵的动荡得紧想是摧阵疼。

快了!快了!”沙僧又笑道:“二哥既知摧阵疼不要扭动只恐挤破浆泡耳。”三藏哼着道:“婆婆啊你这里可有医家?教我徒弟去买一贴堕胎药吃了打下胎来罢。”那婆子道:“就有药也不济事。只是我们这正南街上有一座解阳山山中有一个破儿洞洞里有一眼落胎泉。须得那井里水吃一口方才解了胎气。

却如今取不得水了向年来了一个道人称名如意真仙把那破儿洞改作聚仙庵护住落胎泉水不肯善赐与人。但欲求水者须要花红表礼羊酒果盘志诚奉献只拜求得他一碗儿水哩。你们这行脚僧怎么得许多钱财买办?但只可挨命待时而生产罢了。”行者闻得此言满心欢喜道:“婆婆你这里到那解阳山有几多路程?”婆婆道:“有三十里。”行者道:“好了!好了!师父放心待老孙取些水来你吃。”好大圣吩咐沙僧道:

“你好仔细看着师父若这家子无礼侵哄师父你拿出旧时手段来装吓虎唬他等我取水去。”沙僧依命只见那婆子端出一个大瓦钵来递与行者道:“拿这钵头儿去是必多取些来与我们留着用急。”行者真个接了瓦钵出草舍纵云而去。那婆子才望空礼拜道:“爷爷呀!这和尚会驾云!”才进去叫出那几个妇人来对唐僧磕头礼拜都称为罗汉菩萨一壁厢烧汤办饭供奉唐僧不题。

却说那孙大圣筋斗云起少顷间见一座山头阻住云角即按云光睁睛看处好山!但见那:幽花摆锦野草铺蓝。涧水相连落溪云一样闲。重重谷壑藤萝密远远峰峦树木蘩。鸟啼雁过鹿饮猿攀。翠岱如屏嶂青崖似髻鬟。尘埃滚滚真难到泉石涓涓不厌看。每见仙童采药去常逢樵了负薪还。果然不亚天台景胜似三峰西华山!这大圣正然观看那山不尽又只见背阴处有一所庄院忽闻得犬吠之声。大圣下山径至庄所却也好个去处看那:小桥通活水茅舍倚青山。村犬汪篱落幽人自往还。

不时来至门见一个老道人盘坐在绿茵之上大圣放下瓦钵近前道问讯那道人欠身还礼道:“那方来者?至小庵有何勾当?”行者道:“贫僧乃东土大唐钦差西天取经者。因我师父误饮了子母河之水如今腹疼肿胀难禁。问及土人说是结成胎气无方可治。访得解阳山破儿洞有落胎泉可以消得胎气故此特来拜见如意真仙求些泉水搭救师父累烦老道指引指引。”那道人笑道:“此间就是破儿洞今改为聚仙庵了。我却不是别人即是如意真仙老爷的大徒弟。你叫做甚么名字?

待我好与你通报。”行者道:“我是唐三藏法师的大徒弟贱名孙悟空。”那道人问曰:“你的花红酒礼都在那里?”行者道:

“我是个过路的挂搭僧不曾办得来。”道人笑道:“你好痴呀!

我老师父护住山泉并不曾白送与人。你回去办将礼来我好通报不然请回莫想莫想!”行者道:“人情大似圣旨你去说我老孙的名字他必然做个人情或者连井都送我也。”

那道人闻此言只得进去通报却见那真仙抚琴只待他琴终方才说道:“师父外面有个和尚口称是唐三藏大徒弟孙悟空欲求落胎泉水救他师父。”那真仙不听说便罢一听得说个悟空名字却就怒从心上起恶向胆边生急起身下了琴床脱了素服换上道衣取一把如意钩子跳出庵门叫道:

“孙悟空何在?”行者转头观见那真仙打扮:头戴星冠飞彩艳身穿金缕法衣红。足下云鞋堆锦绣腰间宝带绕玲珑。一双纳锦凌波袜半露裙襕闪绣绒。手拿如意金钩子鐏利杆长若蟒龙。凤眼光明眉菂竖钢牙尖利口翻红。额下髯飘如烈火鬓边赤短蓬松。形容恶似温元帅争奈衣冠不一同。行者见了合掌作礼道:“贫僧便是孙悟空。”那先生笑道:“你真个是孙悟空却是假名托姓者?”行者道:“你看先生说话常言道君子行不更名坐不改姓。我便是悟空岂有假托之理?”先生道:

“你可认得我么?”行者道:“我因归正释门秉诚僧教这一向登山涉水把我那幼时的朋友也都疏失未及拜访少识尊颜。

适间问道子母河西乡人家言及先生乃如意真仙故此知之。”

那先生道:“你走你的路我修我的真你来访我怎的?”行者道:“因我师父误饮了子母河水腹疼成胎特来仙府拜求一碗落胎泉水救解师难也。”那先生怒目道:“你师父可是唐三藏么?”行者道:“正是正是。”先生咬牙恨道:“你们可曾会着一个圣婴大王么?”行者道:“他是号山枯松涧火云洞红孩儿妖怪的绰号真仙问他怎的?”先生道:“是我之舍侄我乃牛魔王的兄弟。前者家兄处有信来报我称说唐三藏的大徒弟孙悟空惫懒将他害了。我这里正没处寻你报仇你倒来寻我还要甚么水哩!”行者陪笑道:“先生差了你令兄也曾与我做朋友幼年间也曾拜七弟兄但只是不知先生尊府有失拜望。如今令侄得了好处现随着观音菩萨做了善财童子我等尚且不如怎么反怪我也?”先生喝道:“这泼猢狲!还弄巧舌!我舍侄还是自在为王好还是与人为奴好?不得无礼!吃我这一钩!”大圣使铁棒架住道:“先生莫说打的话且与些泉水去也。”那先生骂道:“泼猢狲!不知死活!如若三合敌得我与你水去;敌不去只把你剁为肉酱方与我侄子报仇。”大圣骂道:“我把你不识起倒的孽障!既要打走上来看棍!”那先生如意钩劈手相还。二人在聚仙庵好杀:圣僧误食成胎水行者来寻如意仙。那晓真仙原是怪倚强护住落胎泉。及至相逢讲仇隙争持决不遂如然。言来语去成僝僽意恶情凶要报冤。这一个因师伤命来求水那一个为侄亡身不与泉。如意钩强如蝎毒金箍棒狠似龙巅。当胸乱刺施威猛着脚斜钩展妙玄。阴手棍丢伤处重过肩钩起近头鞭。锁腰一棍鹰持雀压顶三钩蜋捕蝉。往往来来争胜败返返复复两回还。钩挛棒打无前后不见输赢在那边。那先生与大圣战经十数合敌不得大圣。这大圣越加猛烈一条棒似滚滚流星着头乱打先生败了筋力倒拖着如意钩往山上走了。

大圣不去赶他却来庵内寻水那个道人早把庵门关了。

大圣拿着瓦钵赶至门前尽力气一脚踢破庵门闯将进去见那道人伏在井栏上被大圣喝了一声举棒要打那道人往后跑了。却才寻出吊桶来正自打水又被那先生赶到前边使如意钩子把大圣钩着脚一跌跌了个嘴哏地。大圣爬起来使铁棒就打他却闪在旁边执着钩子道:“看你可取得我的水去!”大圣骂道:“你上来!你上来!我把你这个孽障直打杀你!”那先生也不上前拒敌只是禁住了不许大圣打水。大圣见他不动却使左手轮着铁棒右手使吊桶将索子才突鲁鲁的放下。他又来使钩。大圣一只手撑持不得又被他一钩钩着脚扯了个躘踵连井索通跌下井去了。大圣道:“这厮却是无礼!”爬起来双手轮棒没头没脸的打将上去。那先生依然走了不敢迎敌。大圣又要去取水奈何没有吊桶又恐怕来钩扯心中暗暗想道:“且去叫个帮手来!”

好大圣拨转云头径至村舍门叫一声:“沙和尚。”那里边三藏忍痛呻吟猪八戒哼声不绝听得叫唤二人欢喜道:

“沙僧啊悟空来也。”沙僧连忙出门接着道:“大哥取水来了?”大圣进门对唐僧备言前事三藏滴泪道:“徒弟啊似此怎了?”大圣道:“我来叫沙兄弟与我同去到那庵边等老孙和那厮敌斗教沙僧乘便取水来救你。”三藏道:“你两个没病的都去了丢下我两个有病的教谁伏侍?”那个老婆婆在旁道:

“老罗汉只管放心不须要你徒弟我家自然看顾伏侍你。你们早间到时我等实有爱怜之意却才见这位菩萨云来雾去方知你是罗汉菩萨。我家决不敢复害你。”行者咄的一声道:“汝等女流之辈敢伤那个?”老婆子笑道:“爷爷呀还是你们有造化来到我家!若到第二家你们也不得囫囵了!”八戒哼哼的道:“不得囫囵是怎么的?”婆婆道:“我一家儿四五口都是有几岁年纪的把那风月事尽皆休了故此不肯伤你。若还到第二家老小众大那年小之人那个肯放过你去!就要与你交合。假如不从就要害你性命把你们身上肉都割了去做香袋儿哩。”八戒道:“若这等我决无伤。他们都是香喷喷的好做香袋;我是个臊猪就割了肉去也是臊的故此可以无伤。”行者笑道:“你不要说嘴省些力气好生产也。”那婆婆道:“不必迟疑快求水去。”行者道:“你家可有吊桶?借个使使。”那婆子即往后边取出一个吊桶又窝了一条索子递与沙僧。沙僧道:

“带两条索子去恐一时井深要用。”沙僧接了桶索即随大圣出了村舍一同驾云而去。那消半个时辰却到解阳山界按下云头径至庵外。大圣吩咐沙僧道:“你将桶索拿了且在一边躲着等老孙出头索战。你待我两人交战正浓之时你乘机进去取水就走。”沙僧谨依言命。

孙大圣掣了铁棒近门高叫:“开门!开门!”那守门的看见急入里通报道:“师父那孙悟空又来了也。”那先生心中大怒道:“这泼猴老大无状!一向闻他有些手段果然今日方知他那条棒真是难敌。”道人道:“师父他的手段虽高你亦不亚与他正是个对手。”先生道:“前面两回被他赢了。”道人道:

“前两回虽赢不过是一猛之性;后面两次打水之时被师父钩他两跌却不是相比肩也?先既无奈而去今又复来必然是三藏胎成身重埋怨得紧不得已而来也决有慢他师之心。管取我师决胜无疑。”真仙闻言喜孜孜满怀春意笑盈盈一阵威风挺如意钩子走出门来喝道:“泼猢狲!你又来作甚?”大圣道:“我来只是取水”。真仙道:“泉水乃吾家之井凭是帝王宰相也须表礼羊酒来求方才仅与些须。况你又是我的仇人擅敢白手来取?”大圣道“真个不与?”真仙道:“不与不与!”大圣骂道:“泼孽障!既不与水看棍!”丢一个架子抢个满怀不容说着头便打。那真仙侧身躲过使钩子急架相还。这一场比前更胜好杀:金箍棒如意钩二人奋怒各怀仇。飞砂走石乾坤暗播土扬尘日月愁。大圣救师来取水妖仙为侄不容求。

两家齐努力一处赌安休。咬牙争胜负切齿定刚柔。添机见越抖擞喷云嗳雾鬼神愁。朴朴兵兵钩棒响喊声哮吼振山丘。

狂风滚滚催林木杀气纷纷过斗牛。大圣愈争愈喜悦真仙越打越绸缪。有心有意相争战不定存亡不罢休。他两个在庵门外交手跳跳舞舞的斗到山坡之下恨苦相持不题。

却说那沙和尚提着吊桶闯进门去只见那道人在井边挡住道:“你是甚人敢来取水!”沙僧放下吊桶取出降妖宝杖不对话着头便打。那道人躲闪不及把左臂膊打折道人倒在地下挣命。沙僧骂道:“我要打杀你这孽畜怎奈你是个人身!

我还怜你饶你去罢!让我打水!”那道人叫天叫地的爬到后面去了。沙僧却才将吊桶向井中满满的打了一吊桶水走出庵门驾起云雾望着行者喊道:“大哥我已取了水去也!饶他罢!饶他罢!”大圣听得方才使铁棒支住钩子道:“你听老孙说我本待斩尽杀绝争奈你不曾犯法二来看你令兄牛魔王的情上。先头来我被钩了两下未得水去。才然来我是个调虎离山计哄你出来争战却着我师弟取水去了。老孙若肯拿出本事来打你莫说你是一个甚么如意真仙就是再有几个也打死了。正是打死不如放生且饶你教你活几年耳已后再有取水者切不可勒掯他。”那妖仙不识好歹演一演就来钩脚被大圣闪过钩头赶上前喝声:“休走!”那妖仙措手不及推了一个蹼辣挣扎不起。大圣夺过如意钩来折为两段总拿着又一抉抉作四段掷之于地道:“泼孽畜!再敢无礼么?”那妖仙战战兢兢忍辱无言这大圣笑呵呵驾云而起。有诗为证诗曰:真铅若炼须真水真水调和真汞干。真汞真铅无母气灵砂灵药是仙丹。婴儿枉结成胎象土母施功不费难。推倒旁门宗正教心君得意笑容还。大圣纵着祥光赶上沙僧得了真水喜喜欢欢回于本处按下云头径来村舍只见猪八戒腆着肚子倚在门枋上哼哩。行者悄悄上前道:“呆子几时占房的?”呆子慌了道:“哥哥莫取笑可曾有水来么?”行者还要耍他沙僧随后就到笑道:“水来了!水来了!”三藏忍痛欠身道:“徒弟啊累了你们也!”那婆婆却也欢喜几口儿都出礼拜道:“菩萨呀却是难得!难得!”即忙取个花磁盏子舀了半盏儿递与三藏道:“老师父细细的吃只消一口就解了胎气。”八戒道:“我不用盏子连吊桶等我喝了罢。”那婆子道:

“老爷爷唬杀人罢了!若吃了这吊桶水好道连肠子肚子都化尽了!”吓得呆子不敢胡为也只吃了半盏。那里有顿饭之时他两个腹中绞痛只听毂辘毂辘三五阵肠鸣。肠鸣之后那呆子忍不住大小便齐流唐僧也忍不住要往静处解手。行者道:

“师父啊切莫出风地里去。怕人子一时冒了风弄做个产后之疾。”那婆婆即取两个净桶来教他两个方便。须臾间各行了几遍才觉住了疼痛渐渐的销了肿胀化了那血团肉块。那婆婆家又煎些白米粥与他补虚八戒道:“婆婆我的身子实落不用补虚。且烧些汤水与我洗个澡却好吃粥。”沙僧道:

“哥哥洗不得澡坐月子的人弄了水浆致病。”八戒道:“我又不曾大生左右只是个小产怕他怎的?洗洗儿干净。”真个那婆子烧些汤与他两个净了手脚。唐僧才吃两盏儿粥汤八戒就吃了十数碗还只要添。行者笑道:“夯货!少吃些!莫弄做个沙包肚不象模样。”八戒道:“没事!没事!我又不是母猪怕他做甚?”那家子真个又去收拾煮饭。

老婆婆对唐僧道:“老师父把这水赐了我罢。”行者道:

“呆子不吃水了?”八戒道:“我的肚腹也不疼了胎气想是已行散了洒然无事又吃水何为?”行者道:“既是他两个都好了将水送你家罢。”那婆婆谢了行者将余剩之水装于瓦罐之中埋在后边地下对众老小道:“这罐水彀我的棺材本也!”众老小无不欢喜整顿斋饭调开桌凳唐僧们吃了斋。消消停停将息了一宿。次日天明师徒们谢了婆婆家出离村舍。唐三藏攀鞍上马。沙和尚挑着行囊。孙大圣前边引路猪八戒拢了缰绳这里才是洗净口孽身干净销化凡胎体自然。

毕竟不知到国界中还有甚么理会且听下回分解。

\chapter[法性西来逢女国\ 心猿定计脱烟花]{法性西来逢女国\\心猿定计脱烟花}
\chapter[色邪淫戏唐三藏\ 性正修持不坏身]{色邪淫戏唐三藏\\性正修持不坏身}
\chapter[神狂诛草寇\ 道昧放心猿]{神狂诛草寇\\道昧放心猿}
\chapter[真行者落伽山诉苦\ 假猴王水帘洞誊文]{真行者落伽山诉苦\\假猴王水帘洞誊文}
\chapter[二心搅乱大乾坤\ 一体难修真寂灭]{二心搅乱大乾坤\\一体难修真寂灭}
\chapter[唐三藏路阻火焰山\ 孙行者一调芭蕉扇]{唐三藏路阻火焰山\\孙行者一调芭蕉扇}
\chapter[牛魔王罢战赴华筵\ 孙行者二调芭蕉扇]{牛魔王罢战赴华筵\\孙行者二调芭蕉扇}
\chapter[猪八戒助力破魔王\ 孙行者三调芭蕉扇]{猪八戒助力破魔王\\孙行者三调芭蕉扇}
\chapter[涤垢洗心惟扫塔\ 缚魔归正乃修身]{涤垢洗心惟扫塔\\缚魔归正乃修身}
\chapter[二僧荡怪闹龙宫\ 群圣除邪获宝贝]{二僧荡怪闹龙宫\\群圣除邪获宝贝}
\chapter[荆棘岭悟能努力\ 木仙庵三藏谈诗]{荆棘岭悟能努力\\木仙庵三藏谈诗}
\chapter[妖邪假设小雷音\ 四众皆遭大厄难]{妖邪假设小雷音\\四众皆遭大厄难}
\chapter[诸神遭毒手\ 弥勒缚妖魔]{诸神遭毒手\\弥勒缚妖魔}
\chapter[拯救驼罗禅性稳\ 脱离秽污道心清]{拯救驼罗禅性稳\\脱离秽污道心清}
\chapter[朱紫国唐僧论前世\ 孙行者施为三折肱]{朱紫国唐僧论前世\\孙行者施为三折肱}
\chapter[心主夜间修药物\ 君王筵上论妖邪]{心主夜间修药物\\君王筵上论妖邪}
\chapter[妖魔宝放烟沙火\ 悟空计盗紫金铃]{妖魔宝放烟沙火\\悟空计盗紫金铃}
\chapter[行者假名降怪犼\ 观音现像伏妖王]{行者假名降怪犼\\观音现像伏妖王}
\chapter[盘丝洞七情迷本\ 濯垢泉八戒忘形]{盘丝洞七情迷本\\濯垢泉八戒忘形}
\chapter[情因旧恨生灾毒\ 心主遭魔幸破光]{情因旧恨生灾毒\\心主遭魔幸破光}
\chapter[长庚传报魔头狠\ 行者施为变化能]{长庚传报魔头狠\\行者施为变化能}
\chapter[心猿钻透阴阳窍\ 魔王还归大道真]{心猿钻透阴阳窍\\魔王还归大道真}
\chapter[心神居舍魔归性\ 木母同降怪体真]{心神居舍魔归性\\木母同降怪体真}
\chapter[群魔欺本性\ 一体拜真如]{群魔欺本性\\一体拜真如}
\chapter[比丘怜子遣阴神\ 金殿识魔谈道德]{比丘怜子遣阴神\\金殿识魔谈道德}
\chapter[寻洞擒妖逢老寿\ 当朝正主救婴儿]{寻洞擒妖逢老寿\\当朝正主救婴儿}
\chapter[姹女育阳求配偶\ 心猿护主识妖邪]{姹女育阳求配偶\\心猿护主识妖邪}
\chapter[镇海寺心猿知怪\ 黑松林三众寻师]{镇海寺心猿知怪\\黑松林三众寻师}
\chapter[姹女求阳\ 元神护道]{姹女求阳\\元神护道}
\chapter[心猿识得丹头\ 姹女还归本性]{心猿识得丹头\\姹女还归本性}
\chapter[难灭伽持圆大觉\ 法王成正体天然]{难灭伽持圆大觉\\法王成正体天然}
\chapter[心猿妒木母\ 魔主计吞禅]{心猿妒木母\\魔主计吞禅}
\chapter[木母助威征怪物\ 金公施法灭妖邪]{木母助威征怪物\\金公施法灭妖邪}
\chapter[凤仙郡冒天止雨\ 孙大圣劝善施霖]{凤仙郡冒天止雨\\孙大圣劝善施霖}
\chapter[禅到玉华施法会\ 心猿木土授门人]{禅到玉华施法会\\心猿木土授门人}
\chapter[黄狮精虚设钉钯宴\ 金木土计闹豹头山]{黄狮精虚设钉钯宴\\金木土计闹豹头山}
\chapter[师狮授受同归一\ 盗道躔禅静九灵]{师狮授受同归一\\盗道躔禅静九灵}
\chapter[金平府元夜观灯\ 玄英洞唐僧供状]{金平府元夜观灯\\玄英洞唐僧供状}
\chapter[三僧大战青龙山\ 四星挟捉犀牛怪]{三僧大战青龙山\\四星挟捉犀牛怪}
\chapter[给孤园问古谈因\ 天竺国朝王遇偶]{给孤园问古谈因\\天竺国朝王遇偶}
\chapter[四僧宴乐御花园\ 一怪空怀情欲喜]{四僧宴乐御花园\\一怪空怀情欲喜}
\chapter[假合形骸擒玉兔\ 真阴归正会灵元]{假合形骸擒玉兔\\真阴归正会灵元}
\chapter[寇员外喜待高僧\ 唐长老不贪富惠]{寇员外喜待高僧\\唐长老不贪富惠}
\chapter[金酬外护遭魔毒\ 圣显幽魂救本原]{金酬外护遭魔毒\\圣显幽魂救本原}
\chapter[猿熟马驯方脱壳\ 功成行满见真如]{猿熟马驯方脱壳\\功成行满见真如}
\chapter[九九数完魔刬尽\ 三三行满道归根]{九九数完魔刬尽\\三三行满道归根}
\chapter[径回东土\ 五圣成真]{径回东土\\五圣成真}













\backmatter

后记
注释
西游故事设定集
宇宙周演示意图
五行八卦示意图
四大部洲示意图
天宫分布示意图
地府分布示意图
灵山分布示意图
周天种类图
神魔关系图谱
取经路线及八十一难手册

\end{document}