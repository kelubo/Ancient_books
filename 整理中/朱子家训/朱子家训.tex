% 朱子家训
% 朱子家训.tex

\documentclass[12pt,UTF8]{ctexbook}

% 设置纸张信息。
\usepackage[a4paper,twoside]{geometry}
\geometry{
	left=25mm,
	right=25mm,
	bottom=25.4mm,
	bindingoffset=10mm
}

% 设置字体,并解决显示难检字问题。
\xeCJKsetup{AutoFallBack=true}
\setCJKmainfont{SimSun}[BoldFont=SimHei, ItalicFont=KaiTi, FallBack=SimSun-ExtB]

% 目录 chapter 级别加点(.)。
\usepackage{titletoc}
\titlecontents{chapter}[0pt]{\vspace{3mm}\bf\addvspace{2pt}\filright}{\contentspush{\thecontentslabel\hspace{0.8em}}}{}{\titlerule*[8pt]{.}\contentspage}

% 设置 part 和 chapter 标题格式。
\ctexset{
	chapter/name={},
	chapter/number={}
}

% 设置古文原文格式。
\newenvironment{yuanwen}{\bfseries\zihao{4}}

% 设置署名格式。
\newenvironment{shuming}{\hfill\bfseries\zihao{4}}

% 注脚每页重新编号,避免编号过大。
\usepackage[perpage]{footmisc}

\title{\heiti\zihao{0} 朱子家训}
\author{朱用纯}
\date{清}

\begin{document}

\maketitle
\tableofcontents

\frontmatter
\chapter{前言、序言}

\mainmatter

% 增加空行
~\\

% 增加字间间隔,适用于三字经、诗文等。
 \qquad  

\chapter{朱子家训}

\begin{yuanwen}
黎明即起,洒扫庭除\footnote{庭院。},要内外整洁;

既昏便息,关锁门户,必亲自检点。
\end{yuanwen}

每天早晨黎明就要起床,先用水来洒湿庭堂内外的地面然后扫地,使庭堂内外整洁;到了黄昏便要休息并亲自查看一下要关锁的门户。

\begin{yuanwen}
一粥一饭,当思来处不易,半丝半缕,恒念物力维艰。
\end{yuanwen}

对于一顿粥或一顿饭,我们应当想着来之不易;对于衣服的半根丝或半条线,我们也要常念着这些物资的产生是很艰难的。

\begin{yuanwen}
宜未雨而绸缪,毋临渴而掘井。
\end{yuanwen}

凡事先要准备,像没到下雨的时候,要先把房子修补完善,不要“临时抱佛脚”,像到了口渴的时候,才来掘井。

\begin{yuanwen}
自奉必须俭约,宴客切勿流连。
\end{yuanwen}

自己生活上必须节约,聚会在一起吃饭切勿流连忘返。

\begin{yuanwen}
器qì具jù质zhì而ér洁jié
,瓦wǎ缶fǒu胜shèng金jīn玉yù
;饮yǐn食shí约yuē而ér精jīng
,园yuán蔬shū愈yù珍zhēn馐xiū
。

器具质而洁,瓦缶胜金玉;
饮食约而精,园蔬愈珍馐。
\end{yuanwen}

【注释】瓦缶(fǒu):瓦制的器具。珍馐(xiū):珍奇精美的食品。
【译文】餐具质朴而干净,虽是用泥土做的瓦器,也比金玉制的好;食品节约而精美,虽是园里种的蔬菜,也胜于山珍海味。
\begin{yuanwen}
	
\end{yuanwen}
勿wù营yíng华huá屋wū
,勿wù谋móu良liáng田tián
。
【译文】不要营造华丽的房屋,不要图买良好的田园。
\begin{yuanwen}
	
\end{yuanwen}
三sān姑gū六liù婆pó
,实shí淫yín盗dào之zhī媒méi
;婢bì美měi妾qiè娇jiāo
,非fēi闺guī房fáng之zhī福fú
。
【译文】社会上不正派的女人,都是*淫和盗窃的媒介;美丽的婢女和娇艳的姬妾,不是家庭的幸福。
\begin{yuanwen}
	
\end{yuanwen}
童tóng仆pú勿wù用yòng俊jun4美měi
,妻qī妾qiè切qiē忌jì艳yàn装zhuāng
。
【译文】家僮、奴仆,不可雇用英俊美貌的,妻、妾切不可有艳丽的妆饰。

勿营华屋,勿谋良田。
三姑六婆,实淫盗之媒;
婢美妾娇,非闺房之福。
童仆勿用俊美,妻妾切忌艳妆。
祖宗虽远,祭祀不可不诚;
子孙虽愚,经书不可不读。
居身务期俭朴,教子要有义方。
莫贪意外之财,莫饮过量之酒。
与肩挑贸易,毋占便宜;
见穷苦亲邻,须加温恤。
刻薄成家,理无久享;伦常乖舛,立见消亡。
兄弟叔侄,须分多润寡;
长幼内外,宜法肃辞严。
听妇言乖骨肉,岂是丈夫?
重资财薄父母,不成人子。
嫁女择佳婿,毋索重聘;
娶媳求淑女,勿计厚奁。
见富贵而生谄容者最可耻,
遇贫穷而作骄态者贱莫甚。
居家戒争讼,讼则终凶;
处世戒多言,言多必失。
勿恃势力而凌逼孤寡,毋贪口腹而恣杀生禽。
乖僻自是,悔误必多;
颓惰自甘,家道难成。
狎昵恶少,久必受其累;
屈志老成,急则可相依。
轻听发言,安知非人之谮诉,当忍耐三思;
因事相争,焉知非我之不是,需平心暗想。
施惠无念,受恩莫忘。
凡事当留馀地,得意不宜再往。
人有喜庆,不可生嫉妒心;
人有祸患,不可生喜幸心。
善欲人见,不是真善;恶恐人知,便是大恶。
见色而起淫心,报在妻女;
匿怨而用暗箭,祸延子孙。
家门和顺,虽饔飧不继,亦有馀欢;
国课早完,即囊橐无馀,自得至乐。
读书志在圣贤,为官心存君国。
守分安命,顺时听天。
为人若此,庶乎近焉。
\begin{yuanwen}
	
\end{yuanwen}
祖zǔ宗zōng虽suī远yuǎn
,祭jì祀sì不bú可kě不bú诚chéng
;子zǐ孙sūn虽suī愚yú
,经jīng书shū不bú可kě不bú读dú
。
【译文】祖宗虽然离我们年代久远了,祭祀却仍要虔诚;子孙即使愚笨,教育也是不容怠慢的。
\begin{yuanwen}
	
\end{yuanwen}
居jū身shēn务wù期qī质zhì朴pǔ
,教jiāo子zǐ要yào有yǒu义yì方fāng
。
【注释】义方:做人的正道。
【译文】自己生活节俭,以做人的正道来教育子孙。
\begin{yuanwen}
	
\end{yuanwen}
勿wù贪tān意yì外wài之zhī财cái
,勿wù饮yǐn过guò量liàng之zhī酒jiǔ
。
【译文】不要贪不属于你的财,不要喝过量的酒。
\begin{yuanwen}
	
\end{yuanwen}
与yǔ肩jiān挑tiāo贸mào易yì
,毋wú占zhàn便biàn宜yí
;见jiàn贫pín苦kǔ亲qīn邻lín
,须xū加jiā温wēn恤xù
。
【译文】和做小生意的挑贩们交易,不要占他们的便宜,看到穷苦的亲戚或邻居,要关心他们,并且要给他们有金钱或其它的援助。
\begin{yuanwen}
	
\end{yuanwen}
刻kè薄báo成chéng家jiā
,理lǐ无wú久jiǔ享xiǎng
;伦lún常cháng乖guāi舛chuǎn
,立lì见jiàn消xiāo亡wáng
。
【注释】乖舛(chuǎn):违背。
【译文】对人刻薄而发家的,绝没有长久享受的道理。行事违背伦常的人,很快就会消灭。
\begin{yuanwen}
	
\end{yuanwen}
兄xiōng弟dì叔shū侄zhí
,需xū分fèn多duō润rùn寡guǎ
;长zhǎng幼yòu内nèi外wài
,宜yí法fǎ肃sù辞cí严yán
。
【译文】兄弟叔侄之间要互相帮助,富有的要资助贫穷的;一个家庭要有严正的规矩,长辈对晚辈言辞应庄重。
\begin{yuanwen}
	
\end{yuanwen}
听tīng妇fù言yán
,乖guāi骨gǔ肉ròu
,岂qǐ是shì丈zhàng夫fū
;重zhòng资zī财cái
,薄báo父fù母mǔ
,不bú成chéng人rén子zǐ
。
【译文】听信妇人挑拨,而伤了骨肉之情,那里配做一个大丈夫呢?看重钱财,而薄待父母,不是为人子女的道理。
\begin{yuanwen}
	
\end{yuanwen}
嫁jià女nǚ择zé佳jiā婿xù
,毋wú索suǒ重zhòng聘pìn
;娶qǔ媳xí求qiú淑shū女nǚ
,勿wù计jì厚hòu奁lián
。
【注释】 厚奁(lián):丰厚的嫁妆。
【译文】嫁女儿,要为她选择贤良的夫婿,不要索取贵重的聘礼;娶媳妇,须求贤淑的女子,不要贪图丰厚的嫁妆。
\begin{yuanwen}
	
\end{yuanwen}
见jiàn富fù贵guì而ér生shēng谄chǎn容róng者zhě
,最zuì可kě耻chǐ
;遇yù贫pín穷qióng而ér作zuò骄jiāo态tài者zhě
,贱jiàn莫mò甚shèn
。
【译文】看到富贵的人,便做出巴结讨好的样子,是最可耻的,遇着贫穷的人,便作出骄傲的态度,是鄙贱不过的。
\begin{yuanwen}
	
\end{yuanwen}
居jū家jiā戒jiè争zhēng讼sòng
,讼sòng则zé终zhōng凶xiōng
;处chù世shì戒jiè多duō言yán
,言yán多duō必bì失shī
。
【译文】居家过日子,禁止争斗诉讼,一旦争斗诉讼,无论胜败,结果都不吉祥。处世不可多说话,言多必失。
【评说】 争斗诉讼,总要伤财耗时,甚至破家荡产,即使赢了,也得不偿失。有了矛盾应尽量采取调解或和解的方法。
\begin{yuanwen}
	
\end{yuanwen}
勿wù恃shì势shì力lì而ér凌líng逼bī孤gū寡guǎ
,毋wú贪tān口kǒu腹fù而ér恣zì杀shā生shēng禽qín
。
【译文】不可用势力来欺凌压迫孤儿寡妇,不要贪口腹之欲而任意地宰杀牛羊鸡鸭等动物。

\begin{yuanwen}
	
\end{yuanwen}乖guāi僻pì自zì是shì
,悔huǐ误wù必bì多duō
;颓tuí惰duò自zì甘gān
,家jiā道dào难nán成chéng
。
【译文】性格古怪,自以为是的人,必会因常常做错事而懊悔;颓废懒惰,沉溺不悟,是难成家立业的。
\begin{yuanwen}
	
\end{yuanwen}狎xiá昵nì恶è少shǎo
,久jiǔ必bì受shòu其qí累lèi
;屈qū志zhì老lǎo成chéng
,急jí则zé可kě相xiàng依yī
。
【注释】狎昵(xiá nì):过分亲近。
【译文】亲近不良的少年,日子久了,必然会受牵累;恭敬自谦,虚心地与那些阅历多而善于处事的人交往,遇到急难的时候,就可以受到他的指导或帮助。
\begin{yuanwen}
	
\end{yuanwen}轻qīng听tīng发fā言yán
,安ān知zhī非fēi人rén之zhī谮zèn诉sù
,当dāng忍rěn耐nài三sān思sī
;
因yīn事shì相xiàng争zhēng
,焉yān知zhī非fēi我wǒ之zhī不bú是shì
,需xū平píng心xīn暗àn想xiǎng
。
【注释】 谮(zèn)诉:诬蔑人的坏话。
【译文】他人来说长道短,不可轻信,要再三思考。因为怎知道他不是来说人坏话呢?因事相争,要冷静反省自己,因为怎知道不是我的过错?
\begin{yuanwen}
	
\end{yuanwen}施shī惠huì无wú念niàn
,受shòu恩ēn莫mò忘wàng
。
【译文】对人施了恩惠,不要记在心里,受了他人的恩惠,一定要常记在心。
【评说】常记他人之恩,以感恩之心看待周围的人及所处的环境,则人间即是天堂。以忘恩负义之心看待周围的人事,则人间即是地狱。
、
\begin{yuanwen}
	
\end{yuanwen}
凡fán事shì当dāng留liú馀yú地dì
,得dé意yì不bú宜yí再zài往wǎng
。
【译文】无论做什么事,当留有余地;得意以后,就要知足,不应该再进一步。
\begin{yuanwen}
	
\end{yuanwen}人rén有yǒu喜xǐ庆qìng
,不bú可kě生shēng妒dù忌jì心xīn
;人rén有yǒu祸huò患huàn
,不bú可kě生shēng喜xǐ幸xìng心xīn
。
【译文】他人有了喜庆的事情,不可有妒忌之心;他人有了祸患,不可有幸灾乐祸之心。
\begin{yuanwen}
	
\end{yuanwen}善shàn欲yù人rén见jiàn
,不bú是shì真zhēn善shàn
;恶è恐kǒng人rén知zhī
,便biàn是shì大dà恶è
。
【译文】做了好事,而想他人看见,就不是真正的善人。做了坏事,而怕他人知道,就是真的恶人。
\begin{yuanwen}
	
\end{yuanwen}见jiàn色sè而ér起qǐ淫yín心xīn
,报bào在zài妻qī女nǚ
;匿nì怨yuàn而ér用yòng暗àn箭jiàn
,祸huò延yán子zǐ孙sūn
。
【注释】 匿(nì)怨:对人怀恨在心,而面上不表现出来。
【译文】看到美貌的女性而起邪心的,将来报应,会在自己的妻子儿女身上;怀怨在心而暗中伤害人的,将会替自己的子孙留下祸根。
\begin{yuanwen}
	
\end{yuanwen}家jiā门mén和hé顺shùn
,虽suī饔yōng飧sūn不bú继jì
,亦yì有yǒu馀yú欢huān
;国guó课kè早zǎo完wán
,即jí囊náng橐tuó无wú馀yú
,自zì得dé至zhì乐lè
。
【注释】 饔(yōng)飧(sūn):饔,早饭。飧,晚饭。国课:国家的赋税。囊(náng)橐(tuó):口袋。
【译文】家里和气平安,虽缺衣少食,也觉得快乐;尽快缴完赋税,即使口袋所剩无余也自得其乐。
\begin{yuanwen}
	
\end{yuanwen}读dú书shū志zhì在zài圣shèng贤xián
,非fēi徒tú科kē第dì
;为wéi官guān心xīn存cún君jun1国guó
,岂qǐ计jì身shēn家jiā
。
【译文】读圣贤书,目的在学圣贤的行为,不只为了科举及第;做一个官吏,要有忠君爱国的思想,怎么可以考虑自己和家人的享受?
\begin{yuanwen}
	
\end{yuanwen}守shǒu分fèn安ān命mìng
,顺shùn时shí听tīng天tiān
。
【译文】我们守住本分,努力工作生活,上天自有安排。
\begin{yuanwen}
	
\end{yuanwen}为wéi人rén若ruò此cǐ
,庶shù乎hū近jìn焉yān
。
【译文】如果能够这样做人,那就差不多和圣贤做人的道理相合了。

\backmatter

\end{document}