% 笠翁对韵
% 笠翁对韵.tex

\documentclass[12pt,UTF8]{ctexbook}

% 设置纸张信息。
\usepackage[a4paper,twoside]{geometry}
\geometry{
	left=25mm,
	right=25mm,
	bottom=25.4mm,
	bindingoffset=10mm
}

% 设置字体,并解决显示难检字问题。
\xeCJKsetup{AutoFallBack=true}
\setCJKmainfont{SimSun}[BoldFont=SimHei, ItalicFont=KaiTi, FallBack=SimSun-ExtB]

% 目录 chapter 级别加点(.)。
\usepackage{titletoc}
\titlecontents{chapter}[0pt]{\vspace{3mm}\bf\addvspace{2pt}\filright}{\contentspush{\thecontentslabel\hspace{0.8em}}}{}{\titlerule*[8pt]{.}\contentspage}

% 设置 part 和 chapter 标题格式。
\ctexset{
	part/name= {},
	part/number={},
	chapter/name={,},
	chapter/number={\chinese{chapter}}
}

% 设置古文原文格式。
\newenvironment{yuanwen}{\bfseries\zihao{4}}

% 设置署名格式。
\newenvironment{shuming}{\hfill\bfseries\zihao{4}}

% 注脚每页重新编号,避免编号过大。
\usepackage[perpage]{footmisc}

\title{\heiti\zihao{0} 笠翁对韵}
\author{李渔}
\date{}

\begin{document}

\maketitle
\tableofcontents

\frontmatter
\chapter{前言、序言}

\mainmatter

% 增加空行
~\\

% 增加字间间隔,适用于三字经、诗文等。
 \qquad  

\part{卷上}

\chapter{东}

\begin{yuanwen}
天对地,雨对风。大陆对长空\footnote{宽广高远的天空。}。山花\footnote{山间野花。}对海树\footnote{成长在海边的树。},赤日\footnote{红日,烈日。}对苍穹\footnote{青天。}。雷隐隐\footnote{雷声不分明的样子。},雾蒙蒙\footnote{雾迷茫的样子。}。日下对天中。风高\footnote{风大。}秋月白,雨霁\footnote{jì,雨后或雪后天转晴。}晚霞红。牛女\footnote{牵牛、织女二星。}二星河\footnote{银河。}左右,参商\footnote{参星,在西方。商星,在东方。这里比喻彼此相隔,不得相见。}两曜\footnote{yào}斗西东\footnote{参和商是二十八宿中的两宿。商即辰,也即是心宿。参宿在西方,心宿居东方,古人往往把亲友久别难逢比为参商。斗,指二十八宿之一的斗宿,不是北斗。两曜,古人把日、月、五星称七曜,曜就是星。} 。十月塞边,飒飒\footnote{形容风吹动树木枝叶等的声音。sà}寒霜惊戍旅\footnote{守卫边疆的将士。};三冬江上,漫漫\footnote{广远无际。}朔\footnote{shu\`o,北方。}雪冷渔翁。
\end{yuanwen}

\begin{yuanwen}
河\footnote{黄河。}对汉\footnote{汉水。由于河可以借指银河,汉也可借指银河。},绿对红。雨伯对雷公\footnote{雨伯、雷公是古代神话中的雨神和雷神。雨伯原称雨师,为了属对工整,这里把师改作伯。}。烟楼\footnote{耸立于烟云中之高楼。}对雪洞\footnote{被雪封住的山洞。},月殿\footnote{月宫。}对天宫\footnote{上帝或诸神在天上的住所。}。云叆叇\footnote{ài dài,浓云蔽日之状。},日曈曚\footnote{tóng méng,太阳将出天色微明的样子。}。蜡屐\footnote{古人穿的一种底下有齿的木鞋,以蜡涂抹其上,叫蜡屐。}对渔篷。过天星\footnote{指流星。}似箭,吐魄月\footnote{魄,又作霸,月球被自身遮掩的阴影部分。古人对月的圆缺道理不理解,以为月里有只蟾蜍,是由它反复吞吐造成的。吐魄月就是刚被吐出的月,指新月,所以说它如弓。}如弓。驿旅\footnote{古代官府设立的招待往来官员的旅舍。}客逢梅子雨\footnote{即梅雨、黄梅雨。中国南部五月至七月所下的雨,因正值梅子成熟的时节,故称为梅雨。},池亭人挹藕花风\footnote{荷花香气阵阵吹来,人们在亭台上饮酒。挹,yì,酌酒。}。茅店村前,皓月\footnote{月光茫茫的样子。}坠林鸡唱韵;板桥路上,青霜锁道马行踪\footnote{这一联是从晚唐温庭筠《商山早行》中“鸡声茅店月,人迹板桥霜”两句诗隐括出来的。}。
\end{yuanwen}

\begin{yuanwen}
山对海,华\footnote{西岳华山。}对嵩\footnote{中岳嵩山。}。四岳\footnote{传说尧时分掌四时、方岳的官。四岳又释指东岳泰山、西岳华山、南岳衡山、北岳恒山。}对三公\footnote{古代天子以下最大的三个官员,各代的职称并不一致。三公又释为星名。}。宫花对禁柳\footnote{古代皇帝居住的城苑禁止百姓出入,所以称禁宫。禁柳即宫廷中的柳树。},塞雁对江龙。清暑殿\footnote{相传三国时吴有避暑宫,夏日清凉不热。},广寒宫\footnote{神话里称月亮中的宫殿为广寒宫。}。拾翠\footnote{原指拾找像翡翠一样的羽毛,后来把青年妇女采集鲜花野草也称作拾翠。}对题红\footnote{刘斧《青琐高议》载:唐僖宗时士人于祐,偶然中从御沟流水上拾到一片红叶,上面题有两句诗:“流水何太急,深宫尽日闲。殷勤谢红叶,好去到人间。”于祐和了两句:“曾闻叶上题红怨,叶上题诗寄阿谁?”放在上游,红叶随水又流入宫中。后于祐娶得宫中韩夫人为妻,谈及此事,其妻倍感惊异,原来当年题诗红叶的就是她。于是她又题了一首诗:“一联佳句随流水,十载幽思满素怀。今日却成鸾凤友,方知红叶是良媒。”}。庄周梦化蝶,吕望兆飞熊\footnote{吕望,即太公望,又称姜太公。传说周文王一夜梦见飞熊进帐,经人占卜,说是将得到贤人的吉兆。第二天出猎,果然遇到姜太公。}。北牖\footnote{北窗。牖,y\v{o}u,窗户。}当风停夏扇,南帘曝日\footnote{曝,pù,晒。曝日即晒太阳。}省\footnote{shěng}冬烘\footnote{原意是指人头脑不清,这里借来同上句的“夏扇”对仗,就是冬天的火炉的意思。}。鹤舞楼头,玉笛弄残仙子月\footnote{唐李白诗:“黄鹤楼头吹玉笛,江城五月落梅花。”《齐谐记》:“仙人子安曾驾鹤经过黄鹤楼。”楼旧址在武昌黄鹤矶上,为古时游览胜地。};凤翔台上,紫箫吹断美人风\footnote{《列仙传》载:秦穆公有女名弄玉,好道。时有人名萧史,善吹箫作鸾凤鸣。穆公把女嫁给萧史,并为他们筑了一所凤凰台。萧史教弄玉以箫吹凤鸣声,凤凰聚止其屋。一日,萧史乘龙,弄玉跨凤,双双升仙而去。}。
\end{yuanwen}

\chapter{东}

\begin{yuanwen}
晨对午,夏对冬。下饷\footnote{下午饭。这里指下午。xi\v{a}ng}对高舂\footnote{薄暮,傍晚。chōng}。青春\footnote{这里指春天。}对白昼,古柏对苍松。垂钓客\footnote{垂竿钓鱼的人。},荷\footnote{hè,担着,扛着。}锄翁。仙鹤对神龙。凤冠珠闪烁,螭带\footnote{雕有龙形的玉带。螭,chī,传说中一种没有角的龙。}玉玲珑。三元\footnote{封建科举考试,乡试第一称解元,会试第一称会元,殿试第一称状元,连续考得三个第一,就是所谓连中三元,三元及第。}及第才千顷\footnote{形容人才学之广。},一品\footnote{古代宰相为一品官爵。}当朝禄\footnote{古代官吏的薪俸。}万钟\footnote{古代称粮的容积单位,每钟盛六斛四斗,万钟极言其多。}。花萼 楼\footnote{花萼(è)楼全称花萼相辉楼,是唐代长安城中著名的建筑。唐玄宗和自己的弟兄常在此设宴饮酒。}间,仙李盘根调国脉\footnote{语出杜甫诗《冬日洛城北谒玄元皇帝庙》:“仙李盘根大,猗兰奕叶光。”。唐朝皇族姓李,杜甫用这句诗比喻皇族子孙繁衍,江山永固。调脉,本指中医诊脉治病。调国脉,是说治理国家,左右国家的命运。},沉香亭\footnote{沉香亭,唐禁苑中的一座亭台。}畔,娇杨\footnote{指杨贵妃。}擅宠\footnote{即专宠,排挤掉别人,使皇帝只对她一个人欢心。}起边风\footnote{唐明皇早年宠爱杨贵妃,日夜同她饮酒作乐,不理朝政。他曾命人在沉香亭旁遍植牡丹,花开时同杨妃到亭上饮酒赏花。后来,安禄山从渔阳起兵叛乱,唐王朝自此走上了下坡路。“起边风”即指安禄山的叛乱。}。
\end{yuanwen}

\begin{yuanwen}
清对淡,薄\footnote{b\'o,淡。}对浓。暮鼓对晨钟\footnote{本指寺院僧众撞钟击鼓,此指言论警策,发人深省。}。山茶对石菊,烟锁对云封。金菡萏\footnote{hàn dàn},玉芙蓉\footnote{菡萏、芙蓉:荷花的别称。}。绿绮\footnote{相传是汉末蔡邕的琴名。qǐ}对青锋\footnote{剑名。}。早汤\footnote{早上起来喝的醒酒汤。}先宿酒\footnote{隔夜仍使人醉而不醒的酒力。},晚食继朝饔\footnote{zh\=ao yōng,早饭。}。唐库金钱能化蝶\footnote{《杜阳杂编》里说:唐穆宗时,殿前种千叶牡丹,开放时香气袭人,穆宗夜宴,有无数黄白蝴蝶飞集花间,天明即飞去。人们张网捕捉数百,天明都变成了金玉,后来打开宝橱,发现皆库中金银所化。},延津宝剑会成龙\footnote{传说晋代张华和雷焕在丰城地下挖出一对极为珍贵的宝剑,每人拿了一把。后来雷焕的儿子佩着剑路过延平津的时候,宝剑忽然跃入水中,变成了一条龙潜水而去。}。巫峡浪传\footnote{犹如空传,意思是宋玉讲的神女不过是个寓言而已,并无其事。},云雨荒唐神女庙\footnote{宋玉《高唐赋》,说楚国先王曾游高唐之观,梦中见一神女,神女临行时说她是巫山之女,“旦为朝云,暮为行雨,朝朝暮暮,阳台之下”。王为立庙,号朝云庙。后人多以巫山神女故事歌咏爱情。};岱宗\footnote{即泰山,古人以它为群山之首,所以称它为宗。杜甫《望岳》诗:“岱宗夫如何?齐鲁青未了。”后半句也是从杜诗变化出来的。杜甫七律《望岳》的原句是:“西岳危棱竦处尊,诸峰罗立如儿孙。”不过这里描写的是西岳华山,而不是东岳泰山。}遥望,儿孙罗列丈人峰\footnote{山峰名。在泰山上,因形状像老人,所以称为丈人峰。}。
\end{yuanwen}

\begin{yuanwen}
繁对简,叠对重\footnote{ch\'ong}。意懒对心慵\footnote{yōng,懒。}。仙翁\footnote{称男性神仙,仙人。}对释伴\footnote{犹如说道侣,同修一道的伙伴。},道范\footnote{敬称他人的容颜,风范。道家的典范。}对儒宗\footnote{儒者的宗师。汉以后亦泛指为读书人所宗仰的学者。}。花灼灼\footnote{zhuó,耀眼,光明。},草茸茸\footnote{草初生的样子。}。浪蝶对狂蜂。数竿君子竹\footnote{古人认为,竹劲节虚心,有君子之德。},五树大夫松\footnote{《史记》记载,秦始皇登泰山,遇到暴风雨,躲在一棵松树下避雨,于是封为“五大夫”松。}。高皇\footnote{汉高祖刘邦。}灭项\footnote{项羽。}凭三杰\footnote{指西汉初期的张良、萧何、韩信。},虞帝承尧殛\footnote{jí}四凶\footnote{古史传说,唐尧年老时把帝位让给虞舜,舜即位后,流放了四个尧舜时代恶名昭彰的部族首领。}。内苑佳人,满地风光愁不尽 ;边关过客,连天烟草憾无穷。
\end{yuanwen}

\chapter{江}

\begin{yuanwen}
奇对偶,只对双。大海对长江。金盘对玉盏,宝烛对银釭\footnote{银白色的灯盏、烛台。釭,gāng。}。朱漆槛\footnote{ji\`an},碧纱窗。舞调对歌腔。兴汉推马武\footnote{马武是汉光武帝的将军,在建立东汉王朝的斗争中起过一定的作用。},谏夏著龙逄\footnote{龙逄即关龙逄,传说是夏桀王的大臣。他见夏桀无道,淫侈暴虐,曾强力谏争,结果被夏桀处死。逄,páng。}。四收列国群王伏\footnote{北宋初大将曹彬,他曾同潘美等将帅一道,伐灭了后蜀、南汉、南唐及北汉等五代时的地方割据政权,帮助宋太祖统一了天下。},三筑高城众敌降\footnote{初唐张仁愿,中宗朝人,曾统领朔方军与突厥族的侵扰进行斗争,使突厥不敢过山牧马。建了三座受降城以威镇北敌,从此边境安宁。}。跨凤登台,潇洒仙姬秦弄玉\footnote{弄玉故事,详见一东“凤翔”二句注。};斩蛇当道,英雄天子汉刘邦\footnote{《史记·高祖本纪》记载,刘邦初起,酒醉夜行,先行者报告说有长蛇拦路,刘邦上前杀死长蛇,路遂通。后有一老太婆在斩蛇处夜哭,人们询问,她说是自己的儿子是白帝子变化为蛇,被赤帝子杀害了。}。
\end{yuanwen}

\begin{yuanwen}
颜对貌,像对庞\footnote{面庞。}。步辇\footnote{古代皇帝乘坐的人力拉的车。辇,niǎn。}对徒杠\footnote{只能步行通过的桥。}。停针对搁杼\footnote{zhù,放下梭子,与停针可以成对。},意懒对心降\footnote{xiáng,安稳、平和。心降就是心里安稳、平和。}。灯闪闪,月幢幢\footnote{chuáng,朦胧的样子。}。揽辔\footnote{控制马匹缰绳。pèi}对飞舡\footnote{xiāng,船只。}。柳堤驰骏马 ,花院吠村尨\footnote{即村狗。尨,máng。}。酒量微酡\footnote{tuó,饮酒后脸红的样子。}琼\footnote{qióng,美玉。}杏颊\footnote{ji\'a},香尘没\footnote{mò}印玉莲双\footnote{晋石崇豪富骄奢,多蓄婢妾,布香尘于地,令诸姬行其上,以试鞋底之大小。玉莲,比喻女人的脚。}。诗写丹枫,韩女幽怀流御水\footnote{见一东韵“题红”注。};泪弹斑竹,舜妃遗憾积湡\footnote{yú}江\footnote{古代神话传说,帝舜的两个妃子娥皇和女英,居住在洞庭之山,舜南巡死于苍梧之野,二妃尽日啼哭,泪洒竹上,竹尽斑,这就是今天的湘妃竹。湡江,水名。}。
\end{yuanwen}

\chapter{支}

\begin{yuanwen}
泉对石,干对枝。吹竹对弹丝\footnote{弹奏琴瑟一类的乐器。}。山亭对水榭\footnote{水上架台,台上建屋,可供人游憩。},鹦鹉对鸬鹚\footnote{lú cí,一种善于捕鱼的水鸟。}。五色笔\footnote{相传南朝梁江淹,年轻时梦见晋代学者和诗人郭璞赠给他五色笔,于是才思大进,写了许多优秀诗文。晚年,又梦见郭璞讨回了五色笔,从此才情顿减,人称“江郎才尽”。后以五色笔比喻文才。},十香词\footnote{辽道宗后萧氏,小字观音,才貌双绝,后以谏猎见疏,作《同心词》自明。耶律乙辛诬后与伶人私通,假造《十香词》为证,帝竟赐后自尽。}。泼墨\footnote{泼墨是绘画术语,意思是大量用墨渲染。}对传卮\footnote{zhī,古代盛酒的器具。}。神奇韩幹\footnote{gàn}画\footnote{韩幹是唐代著名画家,善写人物,尤工于鞍马。传说建中初年,有人牵患有足疾的马就诊。其马毛色骨相似韩幹所画的马,为真马所无。遂牵此马绕市,巧遇韩幹,幹亦惊疑。返家后,视其所画马本,脚有一点黑缺,方知是马画通灵。},雄浑李陵诗\footnote{李陵,西汉名将李广之孙,武帝天汉二年,率步卒五千与匈奴十万骑决战,终因缺少援军,战败投降。李陵在匈奴遇到出使被扣留的苏武,后苏武南还,李陵设酒送别。其赠别苏武之诗雄浑豪爽,十分感人。}。几处花街新夺锦\footnote{唐武则天驾临龙门,诏令群臣赋“明堂火珠”诗,诗先成者赐锦袍。东方虬诗先成,拜锦未坐,宋之问亦成,但写得比东方虬好。武后令夺东方虬锦袍赏给宋之问,此即所谓夺锦。},有人香径淡凝脂。万里烽烟,战士边关争保塞;一犁膏雨\footnote{甘霖。},农夫村外尽乘时\footnote{利用有利时机。}。
\end{yuanwen}

\begin{yuanwen}
葅\footnote{zū,古代酷刑,将人剁成肉酱。}对醢\footnote{hǎi,肉酱。},赋对诗。点漆对描脂。璠簪\footnote{fán zān,美玉制成的簪。}对珠履\footnote{用珠装饰的鞋。相传战国时楚公子春申君,为了向人夸富,让他和门客都穿珠履。lǚ},剑客对琴师。沽酒价\footnote{西晋阮咸每以百钱挂杖头,至酒市沽酒。},买山资\footnote{晋僧人支道林到深公那里去买邱山,深公曰:“未闻巢(父)、(许)由买山而隐(巢父、许由,尧时隐士)。”}。国色对仙姿。晚霞明似锦,春雨细如丝。柳绊长堤千万树,花横野寺两三枝。紫盖黄旗,天象预占\footnote{zhān}江左地\footnote{三国末年吴主孙皓时,有术士说“庚子之年,紫盖黄旗,当入于洛”,孙皓以为平晋。不料相反,庚子之年恰恰是他被俘入洛阳的一年。};青袍白马,童谣终应\footnote{yìng}寿阳儿\footnote{相传南朝梁武帝时,先是大同中有童谣曰“青袍白马寿阳儿”。不久,寿阳的侯景发动叛乱,叛军中尽青袍白马,终于亡梁。}。
\end{yuanwen}

\begin{yuanwen}
箴\footnote{zhēn,古代一种以规劝、告诫为内容的文体。}对讚\footnote{zàn,通赞,颂扬﹑称美。},缶\footnote{fǒu}对卮\footnote{zhī}。萤照对蚕丝。轻裾\footnote{形容人在走动或舞蹈时衣襟飘扬的样子。jū}对长袖,瑞草\footnote{相传不常见的草,见则为祥兆,故称为瑞草。如蓂荚、灵芝之类。}对灵芝。流涕策\footnote{古时大臣们呈给皇上的谏议书。西汉贾谊在写给汉文帝的《治安策》中有“可为痛哭,可为流涕,可为长太息”之句,因称流涕策。},断肠诗\footnote{宋代女诗人朱淑贞,相传其对婚姻不满,故诗词多幽愤哀伤情调,后人辑有《断肠诗集》、《断肠词集》传世。}。喉舌\footnote{泛指说话的器官,比喻要害之地。}对腰肢\footnote{腰身,身段。}。云中\footnote{汉代北方有云中郡,在今山西北部及内蒙古一部分。}熊虎将\footnote{指西汉名将魏尚,相传他做云中守时,匈奴远避,不敢近边。},天上凤凰儿\footnote{汉民歌《陇西行》有“天上何所有?历历种白榆……凤鸣何啾啾,一母将九雏”的诗句。后来多用为赞美别人儿子的话。}。禹庙千年垂橘柚\footnote{语出杜甫诗《禹庙》:“禹庙空寺里,秋风落日斜。荒庭垂橘柚,古屋画龙蛇。”},尧阶三尺覆茅茨\footnote{古书记载,帝尧生活简朴,他的居室土阶三尺,茅茨不剪,采椽不斫。茨,cí,苫房。茅茨,用茅草苫房。}。湘竹含烟,腰下轻纱笼\footnote{lǒng}玳瑁\footnote{dài mào};海棠经雨,脸边青泪湿胭脂\footnote{轻纱笼罩着腰身,好象烟雾环绕着的竹枝;脸边流下泪水,犹如雨点滴在海棠花上。}。
\end{yuanwen}

\begin{yuanwen}
争对让,望对思\footnote{望可解作盼望,思解作思念,成对;望又可解作怨恨,思也可解作怨恨,也成对。}。野葛对山栀\footnote{zhī,植物名。夏开白花,有香气。果实椭圆,色黄,可入药,亦可做染料。或称为栀子。}。仙风\footnote{神仙的风致。形容人的潇洒。}对道骨\footnote{修道者的气质。},天造\footnote{自然生成,对人为而言。}对人为。专诸剑\footnote{专诸,古代勇士名。《左传》载,春秋时,吴公子光为夺取王位,收买专诸为刺客,把匕首藏在鱼腹中,借进献食品的机会刺死了吴王僚。},博浪椎\footnote{汉代的张良,为了给被灭掉的韩国报仇,从仓海君那里请到一位大力士,携带六十公斤的大铁椎,在博浪沙地方狙击秦始皇,误中副车,未果。}。经纬对干支\footnote{天干地支的简称,用来表示年、月、日的方法。}。位尊民物主,德重帝王师。望切不妨人去远,心忙无奈马行迟。金屋闭来,赋乞茂陵题柱笔\footnote{汉武帝幼时,他的姑母馆陶长公主打算把自己的女儿阿娇许给他,就问:“儿欲得妇,阿娇好否?”帝曰:“若得阿娇,当以金屋贮之。”陈阿娇与汉武帝结婚后,颇得宠爱。但陈皇后嫉妒心很强,因自己未育而嫉妒卫夫人,后遭贬独居长门宫,心情悲愤。她听说司马相如很会写文章,就奉黄金百两让相如为她写一篇《长门赋》,抒写她的孤独寂寞之感和对武帝的思念。司马相如曾居住在茂陵,故称他的才思为茂陵题柱笔。题柱,司马相如初西去长安,过成都升仙桥,题柱曰:“不乘高车驷马,不过此桥。”};玉楼成后,记须昌谷负囊词\footnote{唐诗人李贺家乡濒临昌谷川,因之他的诗集称《昌谷集》,后人也称他李昌谷。相传李贺出行,常让小童背一锦囊,每得佳句,就记下投入囊中。后梦神人曰:“上帝白玉楼成,命君作记。”不久诗人就死了。}。
\end{yuanwen}

\chapter{微}

\begin{yuanwen}
贤对圣,是对非。觉奥对参\footnote{cān}微\footnote{觉奥、参微,都是弄懂深奥微小的道理的意思,多用于教学或宗教方面。}。鱼书\footnote{汉乐府《饮马长城窟行》:“客从远方来,遗我双鲤鱼。呼儿烹鲤鱼,中有尺素书。”因之后来称书信为鱼书。}对雁字\footnote{苏武出使匈奴被拘留。汉王朝向匈奴讨还苏武,匈奴推说苏武已死。苏武的随行人员给汉使者出个主意,让他对匈奴单于说:汉天子在上林苑射得一雁,雁脚上绑着苏武的信件,说明他在某某地方。匈奴只好放了苏武。由此后来书信也称雁书、雁字。},草舍对柴扉。鸡晓唱,雉\footnote{zhì}朝\footnote{zhāo}飞\footnote{乐府古题有《雉朝飞》。}。红瘦对绿肥\footnote{语出宋李清照《如梦令》:“知否,知否?应是绿肥红瘦。”}。举杯邀月饮\footnote{语出李白《月下独酌》:“花间一壶酒,独酌无相亲。举杯邀明月,对影成三人。”},骑马踏花归。黄盖能成赤壁捷\footnote{黄盖,孙权手下大将,以“苦肉计”诈降曹军,成就赤壁之火攻。},陈平善解白登危\footnote{汉高祖刘邦讨伐反叛韩王信,被匈奴困于白登,七天没有粮食,形势十分危急。据说靠陈平的奇计,方才解围。}。太白书堂,瀑泉垂地三千丈\footnote{语出李白《望庐山瀑布》:“飞流直下三千尺,疑是银河落九天。”};孔明祀庙,老柏参天四十围\footnote{语出杜甫《古柏行》:“孔明庙前有老柏,柯如青铜根如石。霜皮溜雨四十围,黛色参天二千尺。”}。
\end{yuanwen}

\begin{yuanwen}
戈对甲,幄\footnote{wò}对帏。荡荡对巍巍。严滩\footnote{即子陵滩。见东韵“垂钓客”注。}对邵圃\footnote{邵平,秦时为东陵侯。秦亡,种瓜于长安,瓜美,人称东陵瓜。},靖菊\footnote{晋诗人陶潜,性爱菊,“采菊东篱下,悠然见南山”是他的名句。陶死后,谥号为靖节先生,故称靖菊。}对夷薇\footnote{商代末年,孤竹君的两个儿子伯夷和叔齐在周文王处养老。文王死,武王起兵伐纣。伯夷和叔齐坚决反对,阻止不成,则隐居首阳山,采薇而食,意不餐周粟,终竟饿死。}。占鸿渐\footnote{《周易·渐》:“渐,女归吉。”爻辞中有“鸿渐于干”“鸿渐于磐”等话,意思是谁占得“鸿渐”一卦,嫁女是吉利的。},采凤飞\footnote{春秋时陈厉公太子陈完,逃亡到齐国,齐懿公打算把女儿许给他,占得一卦,其辞有“凤凰于飞,和鸣锵锵”的话,被认为是吉兆。后代以鸾凤比喻配偶,是这里出典。}。虎榜对龙旗。心中罗锦绣,口内吐珠玑。宽宏豁达高皇量\footnote{史称刘邦宽宏豁达,心胸开阔。高皇指汉高祖刘邦。},叱咤喑\footnote{y\=in}哑霸王威\footnote{叱咤、喑哑都是形容人发怒的声音。楚霸王豪气盖世,所以说霸王威。}。灭项兴刘,狡兔尽时走狗死\footnote{韩信帮助刘邦灭掉项羽,被封为楚王,有人告他谋反,刘邦逮捕了他,他说:“果若人言:狡兔死,走狗烹;飞鸟尽,良弓藏;敌国破,谋臣亡。天下已定,我固当烹。”走狗,春秋时越王勾践复国后,范蠡功成身退,留书给文种:“飞鸟尽,良弓藏;狡兔死,走狗烹。越王为人,长颈鸟喙,可与共患难,不可与共乐。子何不去?”文种后称病不上朝,然遭人谗言,言其意欲作乱,越王便赐剑给文种,文种自杀而亡。};连吴拒魏,貔貅\footnote{传说中的一种猛兽,这里借指勇猛的将士。pí}屯处卧龙归\footnote{诸葛亮雄才大略,居南阳,时人送给他的雅号叫“卧龙先生”,后为蜀相。}。
\end{yuanwen}

\begin{yuanwen}
衰对盛,密对稀。祭服对朝衣。鸡窗\footnote{晋宋处宗有一只极为宠爱的长鸣鸡,一直关在窗户边。后来鸡说人话,与处宗谈论,使处宗言谈技巧大增。后用于代指书房。}对雁塔\footnote{唐朝新科进士于皇帝赐宴后,须前往洛阳慈恩塔题写姓名。后比喻科举中试,金榜题名。},秋榜\footnote{秋试(乡试)后所发的榜。亦借指秋试。}对春闱\footnote{明、清会试都在春季,故名。}。乌衣巷\footnote{六朝时金陵一个居住区,位于今南京市东南。东晋时王导、谢安等贵族多居此,故世称王谢子弟为乌衣郎。},燕子矶\footnote{地名。位于江苏省南京市北的观音山上。前临长江,形如飞燕,故名。}。久别对初归。天姿真窈yǎo 窕tiǎo\footnote{形容女子摇曳多姿的样子。},圣德实光辉。蟠pán 桃紫阙quē 来金母\footnote{班固《汉武故事》说神人西王母来见汉武帝,拿出五个桃子,送给武帝两个,即所谓蟠桃。金母,即西王母,按五行学说,西方属金,故称金母。},岭荔红尘进玉妃\footnote{岭荔:史载唐代杨贵妃喜食荔枝,玄宗命人自岭南限七日快马送至长安。杜牧诗有“长安回望绣成堆,山顶千门次第开。一骑红尘妃子笑,无人知是荔枝来”的句子。}。霸王军营,亚父丹心撞玉斗dǒu\footnote{在秦末农民大起义中,刘邦率兵攻入函谷关,占了秦都咸阳。项羽随后赶到,打算同刘邦决战。刘邦势小,只好到项羽驻军的鸿门去陪罪。项羽宴请刘邦,项羽的谋士范增几次示意杀害刘邦都没有成功。刘邦走后,范增把刘邦赠送的玉斗摔在地上,用剑击破,说:“竖子不足与谋也。”发泄他对项羽的不满。这就是有名的鸿门宴。亚父,范增年高望重,被项羽尊称为亚父。丹心,指范增对项羽的一片忠心。};长安酒市,谪仙狂兴换银龟\footnote{传说李白初到长安,拿出所作的《蜀道难》给当时的名诗人贺知章看,贺十分赞赏,称之为“谪仙”,于是解下金龟换酒,与之畅饮尽日。传说李白也曾以银龟换酒。这都表示诗人们的轻视富贵、狂放不羁。金龟、银龟,唐代官员们的佩饰,用以表示官职的级别。谪,封建时代特指贬官。}。
\end{yuanwen}

\chapter{鱼}

\begin{yuanwen}
羹\footnote{gēng,用肉、菜等芶芡煮成的浓汤。}对饭,柳对榆。短袖对长裾。鸡冠对凤尾,芍药对芙蕖\footnote{qú}。周有若\footnote{有若,字子有,孔子弟子。他是东周春秋鲁国人,故称周有若。},汉相如\footnote{西汉著名辞赋家司马相如。}。王屋对匡庐\footnote{庐山。}。月明山寺远,风细水亭虚。壮士腰间三尺剑\footnote{史称汉高祖刘邦手提三尺剑起兵,因而后人常把三尺剑作为有志男儿的象征。},男儿腹内五车书\footnote{相传战国时学者惠施很有学问,“其书五车”,后来用以称人的博学。}。疏影暗香,和靖孤山梅蕊放\footnote{宋林逋性恬淡好古,好作诗,隐居西湖孤山,终身不仕,不娶,以植梅养鹤为乐,世称梅妻鹤子。诗风淡远,多写隐居生活和淡泊心境,卒谥和靖先生。他写的《梅花》诗有“疏影横斜水清浅,暗香浮动月黄昏”的句子,一向为人称道。};轻阴清昼,渊明旧宅柳条舒\footnote{陶渊明写的《五柳先生传》,头几句是:“先生不知何许人也,亦不详其姓字,宅边有五柳树,因以为号焉。”他的诗写自己住宅的环境,有“方宅十余亩,草屋八九间;榆柳荫后檐,桃李罗堂前”的句子。}。
\end{yuanwen}

\begin{yuanwen}
吾对汝,尔对余。选授\footnote{量才授官。}对升除\footnote{即除去旧职就新职,由皇帝授予。}。书箱对药柜,耒耜\footnote{lěi sì,翻土所用的农具。}对耰锄\footnote{用来平整田土或击碎土块的农具。yōu}。参虽鲁\footnote{参,曾参,孔子的弟子。孔子曾说:“柴也愚,参也鲁。”鲁,迟钝。},回不愚\footnote{回,颜回,孔子弟子颜渊的名。孔子说过:“吾与回言终日,不违,如愚。退而省其私,亦足以发,回也不愚。”}。阀阅\footnote{古代官吏们的功劳、阅历。}对阎闾\footnote{lǘ,大门楼,引申为高贵的社会地位。(代指平民?)}。诸侯千乘国\footnote{西周制度,诸侯国大者千乘。乘是战车的计量单位,一车四马叫一乘。},命妇\footnote{受有封号的妇女称命妇。}七香车\footnote{七香车,用多种香料涂抹的极为华贵的车。jū}。穿云采药闻仙子\footnote{《幽明录》载,东汉时刘晨、阮肇,入天台山采药迷路,遇两仙女。},踏雪寻梅策蹇驴\footnote{策,马鞭,这里是赶着的意思。蹇(jiǎn)驴,瘸驴。相传唐代诗人孟浩然曾骑骞驴于灞上踏雪寻梅,抒其幽兴。}。玉兔金乌,二气精灵为日月\footnote{古代神话,说月中有玉兔捣药,日中有三只脚的乌鸦,因以玉兔代月,以金乌代日。古人又认为,宇宙中存在着相互斗争的阴阳二气,天地万物都是由它变化而成,日月则是二气的精华。};洛龟河马,五行生克在图书\footnote{传说伏羲时,黄河出龙马,背负图,称河图;夏禹治水时,神龟从洛水出现,背负书,称洛书。又说龟背上有九组不同点数组成的图画,禹因而排列其次第,乃成治理天下的九种大法,称为洛书。伏羲根据它们画成了八卦。汉孔安国谓河图即八卦。五行即金、木、水、火、土,古人认为它们是构成世界的五种元素。}。
\end{yuanwen}

\begin{yuanwen}
欹\footnote{qī,倾斜。}对正,密对疏。囊橐\footnote{盛物的袋子。大称囊,小称橐tuó。或称有底面的叫囊,无底面的叫橐。}对苞 苴\footnote{包裹。自上包之叫苞,自下垫之叫苴jū。}。罗浮对壶峤\footnote{《初学记》云,罗浮二山随风雨而合离,壶桥二山逐波涛而下山。},水曲对山纡\footnote{山坳。yū}。骖cān 鹤驾,待鸾luán 舆yú\footnote{鹤驾、鸾舆,都是宗教传说中仙人所乘的车乘,由鹤和鸾凤驾着在空中飞行。骖,在这里是驾驶的意思。}。桀jié 溺对长沮\footnote{桀溺、长沮:二人为春秋时隐士。也有人说,长和桀都是身材高大的样子,溺和沮都是污泥。长沮和桀溺就是两个身上沾泥的高个子,并不是人名。}。搏虎卞\footnote{biàn}庄子\footnote{卞庄子,鲁人,古代名勇士。传说他看到二虎争一牛,欲刺虎,管竖子劝说道:“两只老虎共食一牛,一定会因为肉味甘美而相互搏斗起来。两虎相斗,大者必伤,小者必死。到那时候您跟在受伤老虎的后面刺杀老虎,就能一举得到刺杀两头老虎的美名。”卞庄子听从劝告,一次刺死两只虎。故有搏双虎之名。},当\footnote{dǎng}熊冯婕妤\footnote{婕妤yú,古代宫廷中女官名。冯婕妤侍汉元帝观虎圈,有熊出,众惊走,冯独挡之,帝深嘉其勇也。}。南阳高士吟梁父\footnote{诸葛亮原来隐居南阳,亲自种田,并且特别喜欢唱古曲《梁父吟》。},西蜀才人赋子虚\footnote{西蜀才人指司马相如,他写的《子虚赋》,受到汉武帝极大赞赏,叹不同时。}。三径风光,白石黄花供杖履\footnote{语出陶渊明《归去来兮辞》:“三径就荒,松菊犹存。”};五湖\footnote{即太湖,古今著名风景区。}烟景,青山绿水在樵渔。
\end{yuanwen}

\chapter{虞}

\begin{yuanwen}
红对白,有对无。布谷对提壶\footnote{鸟名。}。毛锥\footnote{即毛笔。}对羽扇,天阙对皇都。谢蝴蝶\footnote{宋谢逸有蝴蝶诗百首,人呼为“谢蝴蝶”。},郑鹧鸪\footnote{zhè gū,唐郑谷写的《鹧鸪》诗,有“雨昏青草湖边过,花落黄陵庙里啼”一联,诗家许为最得神韵,所以被称为郑鹧鸪。}。蹈海\footnote{战国时,秦兵围困赵都邯郸,魏王派客将军辛垣衍去劝说赵王,让他尊奉秦昭王为帝,秦兵自退。这事被围困在城中的齐国将士鲁仲连知道,当面批驳了辛垣衍的错误观点,说如果秦真的为帝,自己“有蹈东海而死耳,吾不忍为之民也”。}对归湖\footnote{春秋时范蠡帮助越王勾践灭吴后,功成身退,改名换姓,乘扁舟浮于五湖(即太湖)。}。花肥春雨润,竹瘦晚风疏。麦饭豆麋终创汉\footnote{汉光武帝刘秀初起兵,在饶阳地方遇到困难,将军冯异在滹沱河为他烧麦饭,在芜娄亭为他煮粥,使他度过难关,终于创立了东汉王朝。糜,粥。},莼羹鲈脍竟归吴\footnote{莼chún,莼菜,多年生水草,可做汤吃。莼羹gēng:一种用野菜煮成的汤。鲈脍lú kuài,鲈鱼切成的丝。晋时张翰,由于厌倦官场生活,见秋风起,思念起故乡吴地的莼羹、鲈鱼脍,当即弃官而去。}。琴调轻弹,杨柳月中潜去听;酒旗斜挂,杏花村里共来沽。
\end{yuanwen}

\begin{yuanwen}
罗对绮qǐ,茗对蔬。柏秀对松枯。中元\footnote{农历七月十五日,道教以之为中元节。}对上巳sì\footnote{农历三月三日,古人称上巳节。},返璧\footnote{战国时,赵国有和氏璧,秦王托言以十五城易之,实际是强行索取。赵使蔺相如奉璧入秦,秦不给城,相如诈说璧有微瑕,请原璧归赵。}对还珠\footnote{相传古代合浦郡不产谷物,只有海中盛产珍珠。许多太守到任后尽力搜刮,宝珠竟然迁往它处。后孟尝君为合浦太守,清廉自奉,宝珠又回来了。}。云梦泽\footnote{古代大泽名,在楚(今湖南洞庭湖一带),方九百里,后逐渐干涸,只剩下了洞庭湖。},洞庭湖。玉烛对冰壶\footnote{盛冰的玉壶。用以比喻人的清白,心地纯洁。}。苍头犀角带,绿鬓bìn 象牙梳。松阴白鹤声相应\footnote{《易经》中有“鸣鹤在阴,其子和之”的句子。},镜里青鸾luán影不孤\footnote{《异苑》载,罽(jì)宾国王买得一只鸾鸟,多年不鸣。夫人说:“听人说鸾鸟找到同类就鸣,何不让它照镜子试一试。”鸾鸟发现镜子里的影像,高声悲鸣,向天空奋力一飞,就死掉了。}。竹户\footnote{竹们。}半开,对牖\footnote{y\v{o}u,窗户。}不知人在否;柴门深闭,停车还有客来无。
\end{yuanwen}

\begin{yuanwen}
宾对主,婢对奴。宝鸭对金凫\footnote{金凫fú,原为动物名,或称为野鸭。这里宝鸭和金凫都是指古代用来焚香的器具。}。升堂对入室\footnote{古代居室建筑,室外有堂。一次孔子评价他的弟子子路,说:“由也,升堂矣,未入于室也。”意思是他已经有了一定的造诣。但还不够理想。},鼓瑟对投壶\footnote{上古宴会时的一种游戏。宾主依次将矢投入壶中,多者为胜,少者罚饮。}。觇合璧,颂联珠\footnote{古代迷信说法,日月合璧,五星联珠,是太平的征兆。觇,chān,观测。}。提瓮\footnote{汉人鲍宣的妻子桓少君喜欢打扮,鲍宣说:“这和我们的家境很不相称。”少君乃去服饰,著布衣,常提瓮出汲,并修妇道。瓮,wèng,瓦罐。}对当垆\footnote{卖酒。垆,lú,放置酒器的土台,这里借指酒店。}。仰高红日近\footnote{史载晋元帝太子明帝幼时聪明,其父帝抱以临朝。恰逢有长安使者至,元帝问他:“日与长安孰近乎?”对曰:“长安近,不闻人从日边来。”次日日薄西山宴群臣,帝夸于众,明帝又以为日近。帝问其说,对曰:“举头见日(按:日指他的父亲晋元帝,这是古代崇拜皇帝的说法),不见长安。”众大奇之。},望远白云孤\footnote{狄仁杰客外忆亲曰:“白云飞处为亲所在。”}。歆向秘书窥二酉\footnote{刘向、刘歆父子,都是西汉末年著名的学者,曾经多年整理皇家图书,对先秦典籍的整理、流传起了很大作用,刘歆继父业,整理六艺群书,编成《七略》。对经籍目录学有卓越贡献,为中国目录学之始。二酉,即大、小酉山,在湖南沅陵县西北。古代传说,秦时曾有人于此读书,留书千卷于山中。窥二酉,意思是读了许多古代的秘密藏书。},机云芳誉动三吴\footnote{陆机、陆云兄弟,都是西晋初年著名的文学家。吴亡后,与弟陆云至洛阳,为晋太常张华所器重,文名大噪,时称二陆。晋吴郡华亭(今江苏省松江县)人。三吴是二陆的家乡。}。祖饯jiàn 三杯,老去常斟花下酒;荒田五亩,归来独荷月中锄。
\end{yuanwen}

\begin{yuanwen}
君对父,魏对吴。北岳对西湖。菜蔬对茶荈chuǎn\footnote{粗茶。},苣jù 藤\footnote{芝麻。}对菖chāng 蒲pú\footnote{植物名。习俗在端午节取叶插于檐下。}。梅花数\footnote{古占法。相传为宋代邵雍所作。附会人事,以断吉凶。},竹叶符\footnote{即竹使符。汉代分与郡国守相的信符,右留京师,左留郡国。以竹箭五枚刻字制成。}。廷议\footnote{古时在朝廷之上、皇帝面前论辩国事称廷议。}对山呼\footnote{《汉书·武帝纪》载,汉武帝登中岳嵩山,曾听到群山多次呼喊“万岁”。}。两都班固赋\footnote{班固是东汉著名史学家、文学家,他曾写了《汉书》。《两都赋》是他辞赋中的代表作。},八阵孔明图\footnote{《三国志》载,孔明曾演八阵图,其遗址甚多,都在四川。八阵,古代作战阵法。}。田庆紫荆堂下茂\footnote{《续齐谐记》载,京兆田真、田庆、田广三兄弟商议分居,准备把堂前一棵紫荆树也截为三段。第二天树就枯死了,兄弟大惊,说:树木同株,听说将分就死掉了,难道人还不如树吗?决定不再分居,紫荆树又活了。},王裒póu青柏墓前枯\footnote{王裒,晋人,其父被文帝杀死,裒攀墓柏号哭,柏忽枯。这是迷信说法。}。出塞中郎,羝dī 有乳时归汉室\footnote{中郎,指苏武。汉苏武以中郎将身份出使匈奴,被扣留,匈奴使牧羝羊,告诉他:“羝乳乃得归。”羝,公羊。乳,生羔。};质秦太子,马生角日返燕都\footnote{据《燕丹子》载,战国末年,燕太子丹为质于秦,秦国对他很无礼,于是思归故乡。向秦王恳请,秦王说:“乌鸦白头,马生角,一定放你回去。”太子丹仰天而叹,乌鸦果然白了头,低头落泪;马就生出了角。秦王不得不放他回来。后用以比喻极不可能实现的事情。}。
\end{yuanwen}

\chapter{齐}

\begin{yuanwen}
鸾luán 对凤,犬对鸡。塞北对关西。长生对益智,老幼对旄倪\footnote{老人和小孩。旄,máo,通“耄”,老人。倪,小儿。}。颁竹策\footnote{皇帝给诸侯王颁发的委任状,以竹简为之。},剪桐圭\footnote{圭,古代帝王诸侯举行礼仪时所用的玉器,上尖下方,代表官阶。相传周成王同他的小弟弟叔虞开玩笑,用桐叶剪成圭形,赠给他说,封你为侯。大臣进来贺喜,成王说:这是开玩笑。大臣说:天子无戏言。最后只好把叔虞封于唐。}。剥枣\footnote{pū,剥,同扑,打。}对蒸梨。绵腰如弱柳,嫩手似柔荑\footnote{《诗经·卫风·硕人》写卫庄公夫人之美,说“手如柔荑,肤如凝脂”。荑,tí:初生的茅芽,色白且柔嫩,用以比喻女子的手细白柔美。}。狡兔能穿三穴隐\footnote{战国时,齐公子孟尝君出谋划策,谋求安稳的地位,说,狡兔有三窟,国君也应当如此。意思是多方采取措施,寻找几条出路。},鹪jiāo 鹩liáo 权借一枝栖\footnote{鹪鹩,一种食小虫的极小的鸟,又名“巧妇鸟”。《庄子》上说:“鹪鹩栖树,不过一枝。”意思是容易满足。}。甪lù 里先生,策杖垂绅扶少主\footnote{汉初,商山有四个隐士,名东园公、绮里季、夏黄公、甪里先生,因为年老须发皆白,所以称四皓。相传高祖刘邦没能聘请他们出来,后高祖立吕后子惠帝为太子,继又欲以赵王如意易之。吕后用张良计,请四皓辅佐太子,帝见之曰“幸烦公等善为调护”,遂不见废。};於wū 陵仲zhòng 子,辟纑lú 织履lǚ 赖贤妻\footnote{於陵仲子,即陈仲子,战国时齐国的隐士。因居于於陵,故号於陵子。《孟子》上记载他“身织屦,妻辟 ”。织屦即织草鞋。辟 ,原为剥麻,染麻。辟 指将分练过的麻搓成线。麻是古代纺织原料之一。 ,布缕,引申为织布。楚王欲以为相,不就,与妻逃去,为人灌园,妻子辟 织履。}。
\end{yuanwen}

\begin{yuanwen}
鸣对吠,泛对栖。燕语对莺啼。珊瑚对玛瑙,琥珀对玻璃。绛县老\footnote{即绛县老人。《左传》记载,晋绛县一位老人,不知道自己究竟多大年纪,只知道出生那年初一是甲子日。人们去问师旷,师旷说,他已经七十三岁了。},伯州犁\footnote{春秋时晋国大夫伯宗之子伯嚭,因其父被杀,奔楚,为太宰。}。测蠡\footnote{蠡,lǐ,贝壳做的瓢。管窥天,蠡测海,喻见小也,自不量力。}对燃犀\footnote{烛照明察。相传燃烧犀角可以照妖,晋温峤路过渚矶,人们说水下有怪物,温峤用点燃的犀角照之,果然见到许多奇形异状的精灵。夜梦人曰:“幽明道别,何苦相逼。”这是迷信传说。后比喻洞察事理或奸邪。}。榆槐堪作荫\footnote{yìn},桃李自成蹊\footnote{《史记·李将军传赞》:“谚曰:‘桃李不言,下自成蹊。’此言虽小,可以喻大也。”比喻一个人如果有高德美才,不用自我声张,自然得到人们的敬爱。蹊,小路。}。投巫救女西门豹\footnote{战国魏文侯时,邺地三老、廷掾,与巫祝勾结,假托河伯欲娶妻,每年强选少女,投入河中,愚弄人民并榨取钱财。后西门豹为邺令,在河伯娶妇时,托言所选女子不美,要巫祝、三老去与河伯商量,另行选送,便将其投入河中,因而制止了利用迷信虐害人民的恶行。},赁浣逢妻百里奚\footnote{赁,lìn,本意为租借,这里指雇用。浣,洗。《风俗通》载,春秋时百里奚为秦相,赁一浣妇,歌曰:“百里奚,五羊皮,忆别时,烹伏雌,舂黄 ,烦扊扅,今日富贵忘我为?”问她是谁,原来是被百里奚抛弃在故乡的妻子。}。阙里门墙,陋 巷规模原不陋\footnote{阙里,孔子居住的里巷名。陋巷,孔子弟子颜渊所居,狭小的巷子。引申为狭窄简陋的住处。孔子曾夸奖颜渊:“一箪食,一瓢饮,在陋巷。人不堪其忧,回也不改其乐。”后来唐刘禹锡作《陋室铭》说:“君子居之,何陋之有?”意思是,只要有德者居住,陋巷也不简陋。};隋堤基址,迷楼踪迹亦全迷\footnote{隋炀帝为游江都,开凿了大运河,在两岸栽种杨柳,堤长一千三百余里,称隋堤。迷楼,传说也是隋炀帝所建,用以寻欢作乐的地方。两句的意思是:隋堤也好,迷宫也罢,都成了历史的残迹,当年的迷宫如今真的迷失荒草中了。}。
\end{yuanwen}

\begin{yuanwen}
越对赵,楚对齐。柳岸对桃溪\footnote{指桃源。}。纱窗\footnote{蒙纱的窗户。}对绣户\footnote{雕绘华美的门户。多指妇女居室。},画阁\footnote{彩绘华丽的楼阁。}对香闺\footnote{指青年女子的内室。}。修月斧\footnote{传说唐代有人登嵩山,看见有人卧在道旁,问他为什么在道旁酣睡。那人回答说:“月亮由七宝合成,要由八万二千户人经常修理,我是其中的一个。”说着拿出身边的斧凿。},上天梯。蝃蝀\footnote{d\`i dōng,古时称虹为蝃蝀。}对虹霓。行乐游春圃\footnote{春日的园圃。},工谀\footnote{yú}病夏畦\footnote{于炎夏中耕田,比喻勤苦工作。xī}。李广不封空射虎\footnote{《史记·李将军传》:西汉李广守北平,出猎,见草中石以为虎,射之,箭没石中,以为奇。李广一生战功卓著,却不得封侯。},魏明得立为存麑\footnote{ní。魏明帝曹叡小时候随父射猎,文帝射死母鹿,让明帝去射小鹿。明帝不肯,说:“陛下已杀其母,臣不忍复杀其子。”同时流下了眼泪。文帝于是决心让他继承王位。}。按辔\footnote{勒住马。pèi}徐行,细柳\footnote{汉代周亚夫为将军时,屯兵于细柳,军纪森严,天子欲入军营,亦须依军令行事。}功成劳王敬;闻声稍卧 ,临泾\footnote{西汉赤玼守原州,虏不过临泾,人常道其名以吓唬小儿,使之不敢啼哭。}名震止儿啼。
\end{yuanwen}

\chapter{佳}

\begin{yuanwen}
门对户,陌对街。枝叶对根荄gāi\footnote{植物的根。}。斗鸡\footnote{古时让鸡与鸡相搏斗的一种游戏。}对挥麈zhǔ\footnote{晋代人们清谈时,常挥麈以为谈助,后称谈论为挥麈。麈,古书上指鹿一类的动物,其尾可做拂尘,即“麈尾”。},凤髻\footnote{古代女子的一种发型。jì}对鸾luán 钗chāi [3] 。登楚岫xiù [4] ,渡秦淮[5] 。子犯对夫差[6] 。石鼎龙头缩[7] ,银筝雁翅排[8] 。百年诗礼延余庆[9],万里风云入壮怀[10] 。能辨明伦,死矣野哉悲季路[11] ;不由径\footnote{小路。}窦\footnote{孔、洞。},生乎愚也有高柴[12]。
\end{yuanwen}

[3] 鸾钗:鸾形的钗子。

[4] 楚岫:楚地山峦。

[5] 秦淮:河名。流经南京,是南京市名胜之一。

[6] 子犯:即狐偃,字子犯,春秋晋人。为晋文公舅,故亦称为舅犯。夫差:差,为压韵可读chā。春秋时的吴王,因父阖闾为越王勾践所败,故败困勾践于会稽,以报父仇,并率精兵北会诸侯于黄池,与晋争霸,勾践乘虚而入,遂灭吴,夫差自刭而死,在位二十三年。

[7] 石鼎:陶制的烹茶用具。龙头:当指石鼎上的龙头形装饰。

[8] 银筝:用银装饰的筝或用银字表示音调高低的筝。雁翅:当指古筝上的琴码。

[9] 诗礼:旧时常用来称读书讲究礼教的人家。余庆:指留给子孙后辈的德泽。

[10] 壮怀:豪壮的胸怀。

[11] 季路:姓仲,名由,字子路,一字季路。孔子弟子,性好勇、事亲孝。

[12] 高柴:孔子门人。遇卫难不径不窦(既不走小路,又不走孔道,不知变通)。

[13] 鸡人:职官名。于天将亮时,报时以警醒百官。虎旅:勇猛善战的军队。

[14] 六市、三街:街市。亦作三街六巷。


[16] 皎皎、皑皑:洁白的样子。

[17] 东阁:东向的小门。

[18] 越绝:《越绝书》,历史小说。记载春秋末年与战国初期吴越争霸的历史故事。

[19] 齐谐:《齐谐》,志怪书名。

[20] 三槐:宋代兵部侍郎王佑,多阴德,手植三槐于庭,自言子孙必有为三公的。其子旦后果为相,世称为三槐王氏,子孙因建三槐堂。

[21] 山斋:山中居室。

[22] 漏箭:古代漏壶中用作计时指针的箭。

[23] 翠袖:青绿色衣袖。泛指女子的装束。

[24] 荆钗:用荆木做的发钗。代指与丈夫同甘共苦的贤惠的妻子。

[25] 竹径:竹林中的小径。籁:本指从孔窍中所发出的声音,后泛指一切的声音。

[26] 筛:洒、落。

[27] 携囊:李贺系囊贮诗。

[28] 荷锄:晋人刘伶,好酒。荷锄自随曰:“醉死便可埋我。”

[29] 乡关:故乡。

[30] 水泊:湖泽。

[31] 玉陛:帝王宫殿的台阶。瑶阶:玉砌的台阶。亦用为石阶的美称。

[32] 虎兕:虎与犀牛。比喻凶恶残暴的人。

[33] 马融堂上帐:马融字季长,茂陵(今陕西省兴平县东北)人,东汉学者。从学者常千数,注《孝经》、《论语》、《诗》、《易》、《尚书》、三《礼》等。马融堂前教授生徒,后设绛纱帐,置女乐。

[34] 羊侃水中斋:南朝梁羊侃,好奢侈,到衡州游玩时,结舟为斋,亭馆皆备,日事游宴。

[35] 黉宫:古代学校名。拾芥:捡取地上的草芥。比喻取之极易。

[36] 岱:泰山。畤:古代祭天地五帝之处。燔柴:烧柴,祭天之礼。

[37] 锦缆:锦制的精美的缆绳。碧落:天空。

[38] 华灯:雕饰华美而光辉灿烂的灯。遗簪:指失落的簪子。香街:指繁华的街道。





\begin{yuanwen}
冠对履,袜对鞋。海角对天涯。鸡人对虎旅[13] ,六市对三街[14] 。陈俎豆\footnote{相传孔子年幼时,经常自己摆放俎豆,模仿大人们行礼。俎zǔ豆,古代祭祀、宴飨时,用来盛祭品的两种礼器。亦泛指各种礼器。},戏堆埋\footnote{《列女传·母仪》载,孟子幼时,居近墓,习堆埋;移舍于市,又习贸易事;移学宫旁,乃习礼让。}。皎jiǎo 皎对皑ái皑[16] 。贤相聚东阁[17] ,良朋集小斋zhāi。梦里山川书越绝[18] ,枕边风月记齐谐[19] 。三径萧疏,彭泽 高风怡五柳;六朝华贵,琅琊佳气种三槐[20]。
\end{yuanwen}

\begin{yuanwen}
勤对俭,巧对乖。水榭对山斋[21] 。冰桃对雪藕,漏箭[22] 对更牌\footnote{古代用来打更报时的竹签。}。寒翠袖[23],贵荆钗[24]。慷慨对诙谐 。竹径风声籁[25],花溪月影筛[26]。携囊[27]佳韵随时贮zhù ,荷hè锄[28] 沉酣hān 到处埋。江海孤踪,雪浪风涛惊旅梦;乡关[29]万里,烟峦云树切归怀。
\end{yuanwen}

\begin{yuanwen}
杞对梓,桧guì对楷jiē \footnote{杞、梓、桧和楷,均为树名。}。水泊pō [30] 对山崖。舞裙对歌袖,玉陛bì 对瑶阶[31]。风入袂mèi ,月盈怀。虎兕sì[32] 对狼豺。马融堂上帐[33] ,羊侃水中斋[34] 。北面黉hóng 宫宜拾芥jiè[35] ,东巡岱dài畤zhì 定燔fán柴[36] 。锦缆春江,横笛洞箫通碧落[37] ;华灯夜月,遗簪zān 堕翠遍香街[38] 。
\end{yuanwen}


\chapter{灰}

\begin{yuanwen}
春对夏,喜对哀。大手\footnote{犹高手。指工于文辞的名家。}对长才\footnote{优异的才能。}。风清[2] 对月朗,地阔对天开。游阆làng 苑[3] ,醉蓬莱[4] 。七政对三台[5] 。青龙壶老杖[6] ,白燕玉人钗[7] 。香风十里望仙阁[8] ,明月一天思子台\footnote{汉武帝逼死了被诬陷的太子刘据,后来帝知其冤,作望思台和思子宫以表哀思。}。玉橘冰桃[10] ,王母几因求道降;莲舟藜lí 杖[11] ,真人原为读书来。
\end{yuanwen}

\begin{yuanwen}
朝zhāo对暮,去对来。庶矣\footnote{人口众多。}对康哉[12] 。马肝对鸡肋[13] ,杏眼对桃腮。佳兴适,好怀开。朔雪[14] 对春雷。云移鳷zh\=i鹊观guàn [15] ,日晒凤凰台[16] 。河边淑气[17] 迎芳草,林下轻风待落梅[18] 。柳媚 花明,燕语莺声浑是笑;松号柏舞,猿啼鹤唳总成哀。
\end{yuanwen}

\begin{yuanwen}
忠对信,博\footnote{广博。}对赅gāi\footnote{完备。}。忖cǔn 度duó 对疑猜[19] 。香消对烛暗[20] ,鹊喜对蛩qióng [21] 哀。金花报[22] ,玉镜台[23] 。倒d\v{a}o斝jiǎ 对衔杯[24] 。岩巅横老树,石磴[25] 覆苍苔。雪满山中高士卧[26] ,月明林下美人来[27]\footnote{出自明高启《咏梅九首》}。绿柳沿堤,皆因苏子[28] 来时种zhòng ;碧桃满观guàn ,尽是刘郎[29] 去后栽。
\end{yuanwen}


[2] 风清:风轻柔而凉爽。

[3] 阆苑:阆风之苑,神话传说的仙人居地。

[4] 蓬莱:神话传说中的海上仙山之一。

[5] 七政:日、月和金、木、水、火、土五星。三台:古有灵台、时台、囿台,合称三台。

[6] 青龙壶老杖:《后汉书·费长房传》载,东汉费长房从壶公学仙,辞归,壶公给他一竹杖,说:骑之可以到家,长房到家后把杖投入葛陂,杖化为龙。

[7] 白燕玉人钗:汉武帝升平元年,建招灵阁,有女神留玉钗与帝,后化为玉燕升天。

[8] 望仙阁:南朝陈后主建。


[10] 玉橘冰桃:《汉武外传》载,王母降汉武宫中,享帝以玉橘、冰桃、雪藕。

[11] 莲舟藜杖:据《拾遗记》,传说太乙真人坐莲舟,燃藜杖,降天禄阁,照刘向校书。

[12] 康哉:《尚书·益稷》:“﹝皋陶﹞乃赓载歌曰:‘元首明哉,股肱良哉,庶事康哉。’”歌词称颂君明臣良,诸事安宁。后遂以“康哉”为歌颂太平之词。

[13] 马肝:马肝味劣,比喻卑微琐碎的事。鸡肋:与鸡的肋骨一样无味。比喻没有味道或少有实惠。

[14] 朔雪:北方的雪。

[15] 鹊观:古代道观名。 鹊:鸟纲雀形目鸣禽类。

[16] 凤凰台:在江苏南京市。

[17] 淑气:温和怡人的气息。

[18] 落梅:汉应劭《风俗通》:五月有落梅风,江淮以为信风。

[19] 忖度:思量、考虑。疑猜:猜疑。古典诗词戏曲中为和韵脚常将一个词中的两个字颠倒使用。

[20] 香消:比喻女子死去。烛暗:人死去的通称。

[21] 蛩:蟋蟀的别名。古人认为蟋蟀的叫声凄凉哀伤。

[22] 金花报:古代状元及第时寄家信报喜,称为金花报。

[23] 玉镜台:东晋温峤娶其姑之女,以玉镜台为聘。

[24] 斝:古代青铜制的酒器,圆口,三足。衔杯:口含酒杯。多指饮酒。

[25] 石磴:以石头铺砌成的台阶。

[26] 高士卧:《后汉书·袁安传》载,袁安遇雪天在家高卧不出,人以为贤,举为孝廉。

[27] 美人来:隋赵师雄游罗浮山,日暮见一美人邀共饮,雄不觉醉卧。醒来在梅花树下,翠羽嘈唧其上,月落参横,惆怅不已。

[28] 苏子:苏轼守杭州,令西湖沿堤种桃柳,人号苏公堤,简称苏堤。

[29] 刘郎:语出刘禹锡《元和十一年自朗州召至京,戏赠看花诸君子》诗:“紫陌红尘拂面来,无人不道看花回。玄都观里桃千树,尽是刘郎去后栽。”





\chapter{真}

\begin{yuanwen}
莲对菊,凤对麟。浊富\footnote{不义而富。与“清贫”相对。}对清贫\footnote{生活清寒贫苦。}。渔庄\footnote{渔村。}对佛舍\footnote{寺院房舍,佛堂。},松盖\footnote{谓乔松枝叶茂密,状如伞盖。}对花茵\footnote{花儿开得茂盛,看起来像毯子一样。}。萝月叟sǒu\footnote{月下走在藤萝盘绕的山路上的老人。萝月,萝藤间的月色。},葛天民\footnote{传说中的上古帝王,其治世不言而信,不化而行,是远古社会理想化的政治领袖人物。古人认为是理想中的自然、淳朴之世。}。国宝对家珍[6]。草迎金埒liè [7]马,花醉玉楼[8]人。巢燕三春尝唤友[9] ,塞鸿八月始来宾[10] 。古往今来,谁见泰山曾作砺[11] ;天长地久,人传沧海几扬尘[12]。
\end{yuanwen}

\begin{yuanwen}
兄对弟,吏对民。父子对君臣。勾丁对甫甲[13] ,赴卯对同寅[14] 。折桂客[15] ,簪花人[16] 。四皓对三仁[17] 。王乔云外舃xì [18] ,郭泰雨中巾[19] 。人交好友求三益[20] ,士有贤妻备五伦[21] 。文教南宣 ,武帝平蛮开百越[22] ;义旗西指,韩侯扶汉卷三秦[23] 。
\end{yuanwen}

\begin{yuanwen}
申对午,侃kǎn 对訚yín [24] 。阿ē 魏对茵陈[25] 。楚兰对湘芷zhǐ [26] ,碧柳对青筠[27] 。花馥fù 馥,叶蓁zhēn 蓁[28] 。粉颈对朱唇。曹公奸似鬼[29] ,尧帝智如神[30] 。南阮ruǎn 才郎差chā北富[31] ,东邻丑女效西颦pín [32] 。色艳北堂,草号hào 忘忧[33] 忧甚事?香浓南国,花名含笑[34]笑何人?
\end{yuanwen}

[6] 家珍:家中的珍贵物品。

[7] 金埒:埒即勒,马具。

[8] 玉楼:华丽的楼。

[9] 巢燕三春尝唤友:语出《诗经·小雅·伐木》:“伐木丁丁,鸟鸣嘤嘤,出自幽谷,迁于乔木。嘤其鸣矣,求其友声。”

[10] 塞鸿八月始来宾:塞北的鸿雁直到八月才会飞到南方去做客。称之为宾,因为塞北才是雁的家乡,经过中原好象客人一样。

[11] 泰山曾作砺:汉代封功臣、皇帝封爵的誓词有“黄河如带,泰山若砺。国以永宁,爰及苗裔”的话,意思是遥远无期,不可能出现的情况。砺,磨刀石。

[12] 沧海几扬尘:犹言沧海桑田。《神仙传》载,仙人麻姑在蔡经家见到王远,说自己曾见东海三为桑田,目前东海水又浅,大约要变成陆地。王远叹息说:圣人都说海中将要扬起尘土了。

[13] 勾丁:即征兵。甫甲:即补甲,补充兵员。

[14] 赴卯:古代官府把检查出勤情况叫做点卯(因为卯时日出,开始工作),赴卯犹如今天说上班。同寅:同僚。

[15] 折桂客:晋都诜举贤,对策最优,自己夸口说:“犹桂林之一枝,昆山之片玉。”后因以考试得中为折桂。

[16] 簪花人:古代殿试得中,则赏令簪花,以显其荣。

[17] 四皓:商山四皓的简称,汉初商山的四个隐士。三仁:殷商末年,有微子、箕子、比干三个贤人。三人劝谏纣王,不被采纳,纣王的庶兄微子逃往国外,叔父箕子装疯做奴隶,比干因进谏而被杀,俱以仁德见称于世。孔子评价他们说“殷有三仁”。

[18] 王乔云外舃:《后汉书》载,汉人王乔做叶县县令,有神术,每月两次朝见皇帝。皇帝对他来去这么迅速感动惊异,叫人暗地观察。有人报告,王乔每次来朝,只见有一对凫雁飞来。人们用网捕捉这双飞雁,却只捉得了一只鞋。舃,鞋。

[19] 郭泰雨中巾:汉代郭泰是个有名望的人物,一次遇雨,头巾折起一角,人们以为他是有意这样做的,很雅观,于是效之,故意把头巾折起一角,称为“宗林(郭泰字)巾”。

[20] 三益:语出《论语·季氏》:孔子曰:“益者三友,损者三友。友直、友谅、友多闻,益矣。”三益指直、谅、多闻。

[21] 五伦:古代指君臣﹑父子、兄弟﹑夫妻﹑朋友之间的五种伦理体系。

[22] 文教南宣,武帝平蛮开百越:汉武帝时,统一南方百越之地,议立南海、苍梧等九郡。文教,文明、教化。南宣,推广到南方。百越,古代散居南方各地越族的总称,居住两广、海南岛一带。如汉时有闽越、瓯越、南越、骆越等。其文化特征为断发、纹身、契臂、巢居、使舟及铸铜鼓等。亦作百粤。

[23] 义旗西指,韩侯扶汉卷三秦:在刘邦和项羽争夺天下的斗争中,韩信作为刘邦的将领,曾南北转战,立下了很大功劳。在他刚刚被举用的时候,曾劝说刘邦,略定三秦。刘邦听从他的意见,尽得关中之地,为楚汉之争的胜利打下了基础。韩侯,即韩信。三秦,战国时秦的国土,在今陕西。秦亡后,项羽把关中地分为三份,封秦降将章邯为雍王于咸阳以西,司马欣为塞王于咸阳以东,董翳为翟王于上郡,合称为三秦。

[24] 侃:和乐的样子。訚:态度庄重的样子。

[25] 阿魏、茵陈:两味中药名。

[26] 兰、芷:都是香草,产在古代楚国。湘江在楚国境内,因称芷为湘芷。屈原的诗歌中经常提到这两种香草,用它比喻品行高洁的人物。

[27] 筠:竹。

[28] 蓁蓁:茂盛的样子。

[29] 曹公奸似鬼:三国时曹操奸伪,人称奸鬼。

[30] 尧帝智如神:《史记》上说,帝尧十分聪明,“其智如神”。

[31] 南阮才郎差北富:晋洛阳阮氏家族中的阮籍和阮咸叔侄居道南,家贫而多才;其他阮姓宗族居道北,家富。七月七日,北阮晒衣服,光彩夺目。阮咸也以竹杆把大布裤衩挑了出来。人问其故,他说:“未能免俗,聊复尔耳。”

[32] 东邻丑女效西颦:《庄子》里的一则寓言说,美女西施因胸口痛,经常抚胸口皱眉。东邻丑女也学西施的样子,在人前故意卖弄,却引得人们更加讨厌她。颦,皱眉。

[33] 忘忧:萱草也名忘忧草。

[34] 含笑:花名。

\chapter{文}

\begin{yuanwen}
忧对喜,戚对欣。二典对三坟\footnote{二典指《尚书》中的《尧典》、《舜典》两篇。三坟,指三皇伏羲、神农、黄帝之坟,亦指三皇所著之书。此与二典相对,当指三皇所著之书。}。佛经对仙语,夏耨nòu\footnote{古代锄草的器具。这里当为动词,意为“锄草”,与“耘”相对。}对春耘\footnote{锄草。}。烹早韭,剪春芹。暮雨对朝zhāo 云[3] 。竹间斜白接[4] ,花下醉红裙。掌握灵符五岳箓lù [5] ,腰悬宝剑七星纹[6] 。金锁未开\footnote{宫门的金锁还没有打开,指早朝时间还未到。},上相趋听宫漏[7]永;珠帘半卷,群僚仰对御炉[8] 薰。
\end{yuanwen}

\begin{yuanwen}
词对赋,懒对勤。类聚对群分[9] 。鸾箫对凤笛,带草对香芸[10] 。燕许笔\footnote{唐张说封为燕国公,苏颋(tǐng)封为许国公,二人以文章名世,时人称“燕许大手笔”。},韩柳文[12] 。旧话对新闻。赫hè赫周南仲zhòng [13] ,翩翩晋右军[14] 。六国说shuì 成苏子贵[15] ,两京收复郭公勋[16] 。汉阙què 陈书,侃kǎn 侃忠言推贾谊[17] ;唐廷对策,岩岩直谏有刘蕡fén [18] 。
\end{yuanwen}

\begin{yuanwen}
言对笑,绩对勋。鹿豕shǐ 对羊羵fén [19] 。星冠对月扇[20] ,把袂mèi 对书裙[21] 。汤事葛[22] ,说yuè兴殷[23] 。萝月对松云。西池青鸟使[24] ,北塞黑鸦军[25] 。文武成康为一代[26] ,魏吴蜀汉定三分[27] 。桂苑秋宵,明月三杯邀曲客[28] ;松亭夏日,薰风一曲奏桐君[29] 。
\end{yuanwen}

[3] 暮雨对朝云:据传楚襄王和宋玉一起游览云梦台时,宋玉对楚襄王说:“以前先王,也就是楚怀王曾经游览此地,玩累了便睡着了,梦见一位美丽动人的女子,她说是巫山之女,愿意献出自己的枕头席子给楚王享用。楚王知道弦外有音,非常高兴,立即宠幸那位巫山美女。巫山女临别之时告诉楚怀王:“妾在巫山之阳,高丘之阻。旦为朝云,暮为行雨,朝朝暮暮,阳台之下。”

[4] 竹间斜白接:晋山简为人狂放,做襄阳太守时,经常骑马出游,衣冠颠倒。当时有首民谣说:“山公时一醉,迳造高阳池。日暮倒载归,酩酊无所知。复能乘骏马,倒着白接篱。”白接,即白接篱,当时一种帽子。

[5] 掌握灵符五岳箓:道教传说,修炼到一定程度的道士,可以握三山五岳灵符,统领鬼神。箓,道士画的驱避邪魔的符号、帖子。

[6] 七星纹:宝剑上嵌饰的北斗图案。

[7] 宫漏:即铜壶滴漏,古代宫中计时的用具。

[8] 御炉:御用的香炉。

[9] 类聚对群分:《周易·系辞上》:“方以类聚,物以群分。”

[10] 带草对香芸:相传东汉末年郑玄(康成)曾在不其城东南山中教授,所居山下生一种草,叶长尺余,十分坚韧,人们叫它作“康成书带”。香芸,芸香一类的香草,俗称七里香。有特异香气,能去蚤虱,辟蠹奇验,古来藏书家多用以防蠹。

[12] 韩柳文:唐柳宗元、韩愈,文章绝代。

[13] 周南仲:南仲是周宣王时的大将,他曾率兵击败侵犯周国的少数民族玁狁。

[14] 翩翩:风流潇洒的样子。晋右军:即晋王羲之,著名书法家。他曾做过右军将军,所以人们称他为王右军。

[15] 六国说成苏子贵:战国时,苏秦以合纵术说服了山东六国诸侯,佩六国相印,为总约长。

[16] 两京收复郭公勋:唐郭子仪率兵平息“安史之乱”,收复了长安、洛阳两京,后以功封为汾阳王。

[17] 汉阙陈书,侃侃忠言推贾谊:西汉贾谊是个卓有远见的政治家,他曾上疏汉文帝,直切地指出汉王朝的危机,建议及早采取措施补救。侃侃,形容说话理直气壮,不慌不忙。

[18] 唐廷对策,岩岩直谏有刘 :唐文宗二年,举贤良方正百余人,在皇帝面前对策。进士刘 慷慨直言,切中时弊。但由于考官惧怕宦官的势力,不敢录取。同时对策的河南府参军李邰上疏,宁可把自己的官职让给刘 。后来因宦官的陷害,刘 终竟被贬死。刘 获得了许多正直的知识分子的同情,例如诗人李商隐就有《哭刘 》诗。岩岩,威严。

[19] 鹿豕:鹿和猪。比喻山野无知之物。羊 :相传春秋时鲁大夫季康子掘井,挖到一只瓦缸,里面有一只羊,问孔子,孔子说它是土之怪,叫 羊。

[20] 星冠:道士的帽子。月扇:团扇。形如满月,故称。

[21] 把袂:比喻把臂或握手。袂,衣袖。书裙:晋羊欣年十三,王羲之爱其才。昼卧,王羲之书其白练裙,羊欣视为珍宝,揣摩学习,因此书法遂大进。后以书裙称誉别人的书法,或指文人间的相互雅赏爱慕。

[22] 汤事葛:语出《孟子》。汤,成汤,商朝的第一个王。葛,汤时小国。传说葛伯不祀鬼神,汤曾帮助他祭祀。

[23] 说兴殷:说,傅说,商代人。传说他是奴隶,为人筑墙,后来商王武丁发现了他的才干,举以为三公。

[24] 西池青鸟使:《汉武内传》载,仙人西王母临降人间之前,先有青鸟飞来通报,后来诗词中多以青鸟为传达爱情信息的使者。西池,传说西王母住在西方昆仑山的瑶池。

[25] 北塞黑鸦军:唐李克用统领的守塞军队都穿黑色衣甲,号黑鸦军。

[26] 文武成康为一代:文、武、成、康,西周初的四个王,史称是承平之世。

[27] 魏吴蜀汉定三分:汉代以后魏、蜀、吴三国鼎立。

[28] 桂苑:栽有桂树的林园。曲客:指酒友。曲,造酒的媒质。

[29] 松亭:松间之亭。桐君:古琴名。因桐木可作琴,故以桐君为琴的代称。薰风:传说帝舜得五弦琴,作《南薰之歌》。





\chapter{元}

\begin{yuanwen}
卑对长zhǎng ,季\footnote{弟弟。}对昆\footnote{兄长。}。永巷\footnote{汉代拘禁妃嫔宫女的地方。}对长门\footnote{汉宫名,汉武帝时皇后阿娇失宠后居住的地方。}。山亭对水阁\footnote{靠近水的楼阁。},旅舍\footnote{旅馆。}对军屯\footnote{指驻屯的军队。}。杨子渡\footnote{古津渡名,在江苏江都县南。},谢公墩\footnote{山名,在江苏江宁县城北(古代金陵),晋谢安尝居半山,曾登临,故名。}。德重对年尊\footnote{年纪大。}。承乾对出震,叠坎对重坤\footnote{乾、坤、坎、震,《周易》的四个卦名。乾为龙,所以继位为君称承乾。震为雷声,有发号施令的意思,所以出震是皇帝发号令。}。志士报君思犬马\footnote{有志之人都愿意像犬马那样效忠君王。},仁王养老察鸡豚\footnote{战国思想家孟轲阐述他的仁政思想,说如果王者施仁政,“鸡豚狗彘之畜,无失其时,七十者可以食肉矣”。豚,泛指猪。}。远水平沙,有客泛舟桃叶渡\footnote{在江苏南京市内秦淮河、青溪合流处。据说晋王献之有妾名桃叶,桃叶渡江,以歌送之曰“桃叶复桃叶,渡江不用楫”之语。};斜风细雨,何人携榼kē\footnote{古盛酒器皿。}杏花村\footnote{在金陵。唐杜牧《清明》诗:“借问酒家何处有?牧童遥指杏花村。”后因以杏花村指卖酒之处。}。
\end{yuanwen}

\begin{yuanwen}
君对相,祖对孙。夕照\footnote{傍晚的阳光。}对朝曛xūn\footnote{本指日落时的余光。这里指早晨的昏暗的阳光。}。兰台\footnote{这里指汉代皇家贮藏图书的府库,又称兰台寺。}对桂殿\footnote{对寺观殿宇的美称。},海岛对山村。碑堕泪\footnote{晋羊祜为荆州都督,与东吴相对抗,甚有建树。羊祜死,襄阳民为之罢巿巷哭,为他在岘山建碑立庙,看见碑的人,莫不坠泪,因而称堕泪碑。},赋招魂\footnote{楚辞有《招魂赋》一篇,有人以为是屈原为招怀王之魂而作,有的以为是宋玉哀师屈原之死而作。还有说是屈原自招其魂。}。报怨对怀恩。陵埋金吐气\footnote{旧传秦始皇南巡,有望气者说,五百年后,金陵当有天子出。始皇于是埋金于金陵镇山以镇压之,故称金陵。},田种玉生根\footnote{《搜神记》载,杨伯雍家住无终山,山上无水,伯雍担水置路旁,供行人取饮。三年后,有一人饮水,送给他一斗石子,让他种。几年后,石子上生出了玉石。后其地称玉田。}。相府珠帘垂白昼,边城画角\footnote{古管乐器,传自西羌。形如竹筒,本细末大,以竹木或皮革等制成,因表面有彩绘,故称。发声哀厉高亢,古时军中多用以警昏晓,振士气,肃军容。帝王出巡,亦用以报警戒严。}动黄昏。枫叶半山,秋去烟霞堪倚杖;梨花满地,夜来风雨不开门\footnote{唐刘方平《春怨》诗:“寂寞空庭春欲晓,梨花满地不开门。”}。
\end{yuanwen}

\chapter{寒}

\begin{yuanwen}
家对国,治对安。地主\footnote{指住在本地的人。}对天官\footnote{官名。《周礼》分设六官,以天官冢宰居首,总御百官。}。坎男对离女\footnote{坎和离都是《周易》卦名,古人解释说坎为中男,离为中女。},周诰gào 对殷盘\footnote{《尚书》中属于西周的文献有《洛诰》、《康诰》诸篇,属于殷商的文献有《盘庚》上、中、下三篇。}。三三\footnote{农历三月三日,古人称上巳节。}暖,九九\footnote{农历九月九日,古人称重阳节。}寒。杜撰\footnote{凭空捏造之事,所谓不经之谈。}对包弹\footnote{宋包拯为御史中丞,弹劾不避权贵,人谓之包弹。}。古壁蛩qióng声\footnote{蟋蟀的鸣声。}匝zā\footnote{环绕。},闲亭鹤影单。燕出帘边春寂寂,莺闻枕上漏\footnote{古代计时器,铜制有孔,可以滴水或漏沙,有刻度标志以计时间。简称“漏”。}珊珊\footnote{形容衣裙玉珮的声音。}。池柳烟飘,日夕郎归青锁闼tà\footnote{翰林直宿的地方,门上刻画有青色连锁花纹,因称青锁闼。闼,门。};砌\footnote{台阶。}花雨过,月明人倚玉阑干。
\end{yuanwen}

\begin{yuanwen}
肥对瘦,窄对宽。黄犬\footnote{指晋陆机的黄耳犬。曾为陆机长途传递书信。}对青鸾\footnote{古代传说中凤凰一类的神鸟。}。指环对腰带,洗钵\footnote{即洗钵泉,今位于山东济南李清照纪念堂院内西北隅,为不规则泉池。}对投竿\footnote{投钓竿于水,即垂钓。}。诛佞nìng 剑\footnote{汉朱云忠直敢谏。成帝的老师安昌侯张禹,在朝廷甚有地位,然毫无作为。朱云对成帝说:“臣愿求赐上方宝剑,断佞臣一人,以厉其余。”上问为谁,曰张禹。帝怒令斩之,云攀殿槛,槛折以免。或请易槛,上不许,存之以旌忠臣。},进贤冠\footnote{文官戴的一种帽子。}。画栋\footnote{有彩绘装饰的栋梁。}对雕栏。双垂白玉箸\footnote{道家得道,临终有白玉气出鼻孔,双垂如双玉箸。},九转紫金丹\footnote{古代术士把朱砂烧成水银,又把水银炼成丹药,叫做还丹。九转,形容经过许多步骤。}。陕右棠高怀召sh\`ao伯\footnote{召虎是周宣王时的一位大臣,人们称他为召伯。他很有政绩,传说他的住处有一棵甘棠树,他走后,人们对这棵树加意保护,并且作了一首叫《甘棠》的诗歌,以资纪念。陕右,即关中地区。},河南花满忆潘安\footnote{河南疑当作河阳,潘安为河阳令,满县皆栽桃花,人曰花县。}。陌上芳春,弱柳当风\footnote{正对着风。}披彩线;池中清晓\footnote{清晨,天刚亮的时候。},碧荷承露\footnote{承接甘露。}捧珠盘。
\end{yuanwen}

\begin{yuanwen}
行对卧,听对看。鹿洞\footnote{指白鹿洞。宋朱熹讲学处。}对鱼滩。蛟腾对豹变\footnote{语出《周易·革卦》:“君子豹变。”意思是君子的变化像豹一样,越来越有文采。喻润色事业,或迁喜去恶。},虎踞对龙蟠\footnote{诸葛亮论金陵的地形,说:“钟阜龙蟠,石城虎踞。”}。风凛lǐn 凛\footnote{寒冷的样子。},雪漫漫\footnote{空间广远的样子。}。手辣对心酸。莺yīng 莺对燕燕[27] ,小小\footnote{南朝齐时,钱塘妓女苏小小,亦名简简。}对端端。蓝水远从千涧落,玉山高并两峰寒\footnote{是杜甫《九日兰田崔氏庄》一诗的腹联。}。至圣不凡,嬉戏六龄陈俎zǔ豆[30] ;老莱大孝,承欢七衮gǔn舞斑斓[31] 。
\end{yuanwen}

[27] 莺莺、燕燕:钱塘范十二郎有二女,曰莺莺燕燕,为富民陆氏妾。

[30] 至圣不凡,嬉戏六龄陈俎豆:《史记·孔子世家》载:“孔子为儿嬉戏,常陈俎豆,设礼容。”

[31] 老莱大孝,承欢七衮舞斑斓:老莱子,传说中的古孝子,父母年迈,无以为欢,他虽也年纪很大,但仍穿上花花绿绿的幼儿服装,在父母面前嬉笑,引逗双亲开心。

\chapter{删}

\begin{yuanwen}
林对坞wù\footnote{四面高,中间凹下的地方。},岭对峦luán 。昼永\footnote{白昼漫长。}对春闲。谋深对望重,任大\footnote{责任重大。}对投艰\footnote{赋予重任。}。裙袅niǎo袅\footnote{随风摆动的样子。},佩\footnote{古代女子头上或身上的佩饰。}珊珊。守塞对当关\footnote{把守关隘。}。密云千里合,新月一钩弯。叔宝君臣皆纵逸\footnote{南朝陈后主,名叔宝,历史上有名的荒淫皇帝。他经常召集江总、孔范等十个文人在一起饮宴,称为“狎客”,让张贵人等八名妃嫔与之交错而坐,整日纵情声色。},重华父母是嚚yín 顽\footnote{重华是帝舜的名。相传他的父亲瞽叟和弟弟象品行都很坏,曾多次设阴谋准备把他害死。嚚顽,愚蠢而顽固。瞽,瞎。}。名动帝畿jī\footnote{我国古代称靠近国都的地方。这里同句中的“日下”都指都城。},西蜀三苏\footnote{指宋著名文学家苏洵和他的儿子苏轼、苏辙。他们都是四川眉山人,名震一时,人称三苏。}来日下;壮游京洛,东吴二陆起云间\footnote{二陆指晋文学家陆机、陆云兄弟,大有才名,人称二陆。他们在东吴亡后,都来到洛阳从政。据说一次陆云遇到荀隐,互相自我介绍,陆说:“云间陆士龙。”荀说:“日下荀鸣鹤。”云间,江苏松江县之古称。壮游,谓怀抱壮志而远游。}。
\end{yuanwen}

\begin{yuanwen}
临\footnote{临摹。}对仿,吝对悭qiān\footnote{吝啬。}。讨逆\footnote{讨伐坏人。}对平蛮\footnote{旧指南方少数民族。}。忠肝对义胆,雾发fà 对云鬟huán\footnote{都是指女子浓密秀美的头发。}。埋笔冢zhǒng\footnote{陈、隋间僧人智永是著名的书法家,相传他写字用笔积十八瓮,后埋成一墓,号曰“退笔冢”。},烂柯山。月貌\footnote{形容女子容貌美丽。}对天颜。龙潜终得跃\footnote{《周易·乾卦》:“初九,潜龙勿用。”“九四,或跃在渊。”比喻人或事物由小到大、由弱到强的发展过程。},鸟倦亦知还\footnote{语出晋陶渊明《归去来兮辞》:“云无心以出岫,鸟倦飞而知还。”}。陇树\footnote{陇山一带的树木。泛指边塞之树。}飞来鹦鹉绿,池筠yún 密处鹧zhè 鸪gū 斑。秋露横江,苏子月明游赤壁\footnote{元丰四年,苏轼曾月夜泛舟赤壁,作《前赤壁赋》,赋中有“少焉,月出于东山之上,徘徊于斗牛之间。白露横江,水光接天”等语。};冻云迷岭,韩公雪拥过蓝关\footnote{唐文学家韩愈,以上《谏迎佛骨表》触怒宪宗,被贬为潮州刺史,行程中至蓝关遇雪,写了一首《左迁至蓝关示侄孙湘》,“云横秦岭家何在,雪拥蓝关马不前”是诗中名句。}。
\end{yuanwen}

\part{卷下}

\setcounter{chapter}{0}

\chapter{先}

\begin{yuanwen}
寒对暑,日对年。蹴cù 踘jū [1] 对秋千。丹山对碧水,淡雨对覃tán烟[2] 。歌宛转[3],貌婵娟[4]。雪鼓对云笺jiān [5] 。荒芦栖南雁,疏柳噪秋蝉。洗耳尚逢高士笑[6] ,折腰肯受小儿怜[7] 。郭泰泛舟,折角半垂梅子雨[8] ;山涛骑马,接䍠lí 倒着zhuó杏花天[9] 。
\end{yuanwen}
[1] 蹴踘:我国古代的一种足球运动。

[2] 覃烟:袅袅直升空中的饮烟或横浮低空的烟雾。覃,长。

[3] 宛转:声音委婉而动听。

[4] 婵娟:体态柔弱的样子。

[5] 云笺:唐韦陟(zhì)用五采笺写信,由他人代笔,自己签名。由于他写的“陟”字像五朵云,因而后来人们称书信为五云笺或云笺。

[6] 洗耳尚逢高士笑:传说帝尧时,箕山有高人隐士曰巢父、许由,尧同许由商量,准备把帝位传给他。巢父听到了,以为玷污了他的耳朵,就跑到池中去洗耳。池水主人怒曰:“何污我水!”这个故事说帝尧、许由、巢父、池水主人,一个比一个更高洁。

[7] 折腰肯受小儿怜:陶渊明为彭泽令。一次,郡督邮来视察。县吏向陶渊明建议,应穿上官服迎见。陶渊明气愤地说:“吾不能为五斗米折腰,拳拳事乡里小儿!”于是弃官而去。作《归去来兮辞》。

[8] 郭泰泛舟,折角半垂梅子雨:见真韵“郭泰”句注。

[9] 山涛骑马,接 倒着杏花天:见文韵“竹间”句注。
\begin{yuanwen}
轻对重,肥对坚[10] 。碧玉对青钱[11] 。郊寒对岛瘦[12] ,酒圣对诗仙[13] 。依玉树[14] ,步金莲[15] 。凿井对耕田[16]。杜甫清宵立[17] ,边韶sháo 白昼眠[18] 。豪饮客吞波底月,酣hān 游人醉水中天[19] 。斗草青郊[20] ,几行宝马嘶金勒[21] ;看花紫陌[22] ,千里香车拥翠钿diàn [23] 。
\end{yuanwen}


[10] 肥对坚:肥,肥马。坚,坚车。

[11] 碧玉:南朝宋汝南王妾,甚受宠爱,后代引为娇怜的爱人的代称。青钱:唐张鷟(zhuó)甚有才名,时人称之为“青钱学士”,意思是他的文章万选万中,万无一失。

[12] 郊寒对岛瘦:郊指孟郊,岛指贾岛,唐代的两个诗人。孟郊的诗内容清苦,失之寒,贾岛的诗风格瘦峭,失之瘦,后人于是有“郊寒岛瘦”的评价。

[13] 酒圣:晋刘伶旷达放饮,又曾作《酒德颂》,后人因称之为酒圣。

[14] 依玉树:唐崔宗之,美容仪,饮酒时更见风度。杜甫诗《饮中八仙歌》说:“宗之潇洒美少年,举觞白眼望青天,皎如玉树临风前。”

[15] 步金莲:南齐东昏侯宠爱潘妃,以金为莲花贴地,令潘妃行其上,叫“步步生莲花”。后以金莲指女子纤足。

[16] 凿井、耕田:传说尧帝游于康衢,有一老人击壤而歌曰:“日出而作,日入而息,凿井而饮,耕田而食,帝力于我何有哉!”

[17] 杜甫清宵立:杜甫诗有“思家步月清宵立”句。

[18] 边韶白昼眠:汉儒边韶,字孝先,性放达,开帐授徒,常昼眠,弟子编歌嘲之曰:“边孝先,腹便便。夜读书,昼贪眠。”

[19] 豪饮客吞波底月,酣游人醉水中天:杜甫《饮中八仙歌》有“左相日兴费万钱,饮如长鲸吸百川”;“知章骑马似乘船,眼花落井水底眠”等语,形容醉人们的情态。

[20] 斗草:也称“斗百草”。一种古代游戏。竞采花草,比赛多寡优劣,常于端午行之。青郊:指春天的郊野。

[21] 金勒:金饰的带嚼口的马络头。

[22] 紫陌:指京师郊野的道路。

[23] 翠钿:妇女用宝石金银雕饰的首饰,这里即代指妇女。钿,为压韵可读tián。

\begin{yuanwen}
吟对咏\footnote{均指有一定节奏的阅读。},授对传。乐矣对凄然。风鹏对雪雁,董杏对周莲[24] 。春九十[25] ,岁三千[26] 。钟鼓对管弦。入山逢宰相[27] ,无事即神仙。霞映武陵桃淡淡,烟荒隋堤柳绵绵[28] 。七碗月团,啜chuò 罢清风生腋下;三杯云液[29],饮余红雨晕yùn 腮边。
\end{yuanwen}
[24] 董杏:《神仙传》中载,三国东吴董奉为人治病不取报酬,病重的为他栽五棵杏,轻者栽一棵,数年后共得十万余株,郁然成林。周莲:宋儒周敦颐喜爱莲花,曾写《爱莲说》一篇,盛赞此花出污泥而不染的高洁品质。

[25] 春九十:春光九十,意思是春光将尽。

[26] 岁三千:极言年寿之长。传说汉武帝时,东郊献短人东方朔,谓帝曰:“王母蟠桃,三千岁一熟,此儿已三偷之矣。”

[27] 入山逢宰相:南朝梁陶宏景隐山中,武帝常问之以国事,时人称之“山中宰相”。

[28] 烟荒隋堤柳绵绵:隋炀帝自板渚引河达淮,岸上悉种柳。见齐韵“隋堤”注。

[29] 云液:酒的美称。
\begin{yuanwen}
中对外,后对先。树下对花前。玉柱对金屋[30] ,叠嶂对平川[31] 。孙子策[32] ,祖生鞭[33] 。盛席对华筵y\'an[34]。解醉知茶力,消愁识酒权[35] 。丝剪芰jì荷开冻沼zhǎo [36] ,锦妆凫fú 雁泛温泉[37] 。帝女衔石,海中遗魄为精卫[38] ;蜀王叫月,枝上游魂化杜鹃[39] 。
\end{yuanwen}
[30] 玉柱:石柱的美称。金屋:华美之屋。

[31] 叠嶂:重迭的山峰。平川:广阔平坦之地。

[32] 孙子策:孙子指春秋战国时吴国孙武,著名军事家,著有《孙子》十三篇传世。

[33] 祖生鞭:东晋祖逖与朋友刘琨同寝,他们立志收复中原,每天闻鸡鸣就起床舞剑。一次祖逖先醒,闻鸡鸣,逖蹴琨曰:“此非恶声也。”琨恐曰:“祖生先吾着鞭。”意思是比自己行动得快。

[34] 华筵:丰盛的筵席。

[35] 解醉知茶力,消愁识酒权:茶力、酒权互文,即茶和酒的功效。

[36] 丝剪芰荷开冻沼:传说中隋炀帝的故事,说他曾命人用锦绢剪为荷花,遍插池苑,从中游乐。芰,古书上指菱。

[37] 锦妆凫雁泛温泉:唐玄宗的故事。相传玄宗扩建华清宫汤池,规模宏丽,汤池内以玉莲为喷泉,又缝锦绣为凫雁,放于水中,自己乘小舟从中游嬉,极尽奢欲。

[38] 帝女衔石,海中遗魄为精卫:上古神话,赤帝有女名女娃,游于东海,溺而不返,魂魄变成一种鸟,名叫精卫,常常衔木石填海中。

[39] 蜀王叫月,枝上游魂化杜鹃:上古神话传说,蜀王名杜宇,在蜀治水,自以德薄,让位给大臣鳖冷,自己隐居山林,死后化为杜鹃鸟,夜夜悲啼,啼则吐血。


\chapter{萧}

\begin{yuanwen}
琴对管,斧对瓢。水怪对花妖。秋声对春色,白缣jiān 对红绡xiāo [1] 。臣五代[2] ,事三朝[3] 。斗 柄 对弓腰[4] 。醉客歌金缕lǚ [5] ,佳人品玉箫。风定落花闲不扫,霜余残叶湿难烧。千载兴周,尚父一竿投渭水[6] ;百年霸越,钱王万弩射江潮[7]。
\end{yuanwen}
[1] 缣:丝绢,这里指细绢。绡:生丝,又指用生丝织的东西,这里指绸子。

[2] 臣五代:指五代时冯道,他曾历事后唐、后晋、后辽、后汉、后周,对丧君亡国毫不介意,并自号“长乐老”。旧时代拿他做没气节的典型。

[3] 事三朝:沈约事南朝宋、齐、梁三朝。

[4] 斗柄:北斗七星中排成柄状的三星。弓腰:舞女反身将腰弯如弓形,叫做弓腰。

[5] 金缕:词牌《贺新郎》的别名,或说指唐女诗人杜秋娘所作《金缕衣》。

[6] 千载兴周,尚父一竿投渭水:西周初,吕望曾隐居在渭水垂钓,后被周文王聘请为太师,辅佐武王灭殷。被周武王尊为尚父。

[7] 百年霸越,钱王万弩射江潮:传说五代时钱 为吴越王,做御潮铁柱于江中,未成而潮水大至。吴越王命以万弩射之,潮水乃退。筑土一升者,赏钱一升,名之曰钱塘。



\begin{yuanwen}
荣\footnote{茂。}对悴\footnote{枯。},夕对朝。露地\footnote{佛教语。喻三界(欲界、色界、无色界)的烦恼俱尽,处于没有覆蔽的地方。}对云霄。商彝对周鼎\footnote{指商周二代的青铜器。},殷濩huò\footnote{传说是商汤王的舞乐。}对虞韶\footnote{传说帝舜时乐名。虞即指帝舜虞氏。}。樊素口,小蛮腰\footnote{樊素、小蛮都是白居易的歌伎。白有“樱桃樊素口,杨柳小蛮腰”的诗句。}。六诏\footnote{“诏”是唐代我国西南少数民族对王的称呼,时有蒙嶲(xī)、越析、浪穹、澄睒(shān)、施浪、蒙舍诸诏,合称六诏。其地在今云南及四川西南部。}对三苗\footnote{传说尧、舜时代居住在西南的我国少数民族。}。朝天车\footnote{指大臣们登朝拜见皇帝所用车乘。}奕yì奕\footnote{有次序的样子。},出塞马萧萧\footnote{杜甫《后出塞》诗有“马鸣风萧萧”之句。萧萧,马嘶声或风声。}。公子幽兰重泛舸gě\footnote{屈原《九歌》:“沅有芷兮澧(lǐ)有兰,思公子兮未敢言。”舸,大船。泛舸即乘船游览。},王孙芳草正联镳biāo\footnote{刘安《招隐士》:“王孙游兮不归,春草生兮萋萋。”镳,马辔头。联镳,意思是并马而行。}。潘岳高怀,曾向秋天吟蟋蟀\footnote{潘岳是西晋诗人,曾写有《蟋蟀赋》。};王维清兴,尝于雪夜画芭蕉\footnote{唐王维诗、画、书都有很高造诣。据说他的山水画随意写来,不分四时,曾画雪中芭蕉。}。
\end{yuanwen}

\begin{yuanwen}
耕对读,牧对樵。琥珀对琼瑶\footnote{美玉。}。兔毫\footnote{笔名,这里指毛笔。}对鸿爪zhǎo\footnote{指鸿雁在泥土上留下的脚印,比喻人生的阅历。},桂楫jí 对兰桡ráo\footnote{楫和桡都是划船撑船的工具。桂是桂树,兰指木兰。用桂和木兰制成的楫和桡,言其贵重华美。}。鱼潜藻,鹿藏蕉\footnote{《列子·周穆王》:郑人有薪者,遇鹿而毙之,藏诸泥中,覆之以蕉,俄而失其处,遂以为梦,顺途而道其事。傍闻者取之,归告室人曰:薪者梦得鹿,不知其处,我今得之,彼真在梦中矣。}。水远对山遥。湘灵\footnote{湘灵,尧女娥皇女英,哭舜于苍梧之野,死之为湘江之神。}能鼓瑟,嬴女解吹箫\footnote{即弄玉的故事。秦王族姓嬴,故称弄玉为嬴女。见江韵“跨凤”句注。}。雪点寒梅横小院,风吹弱柳覆平桥。月牖通宵,绛蜡罢时光不减\footnote{由于月光透窗而入,即使灭掉红烛,室内仍很明亮。绛蜡:即红烛。};风帘当昼,雕盘停后篆难消\footnote{篆,指袅袅上升的香烟好像篆字一样。二句意思是,因为风帘遮掩门户,尽管雕盘中的薰香不再点燃,室内的香气也很难消失。}。
\end{yuanwen}

\chapter{肴}

\begin{yuanwen}
《诗》对《礼》,卦对爻。燕引对莺调tiáo\footnote{引和调都是歌曲,这里指燕和莺动听的鸣声。}。晨钟对暮鼓\footnote{见上卷冬韵“暮鼓”句注。},野馔zhuàn 对山肴\footnote{馔、肴是饭菜的统称。野馔、山肴指淡素的饭食。}。雉zhì 方乳\footnote{汉鲁恭为中军令,很有政绩,蝗不入境。河南尹闻之,使人往看。见野鸡伏于桑下,儿童不捕,惊问,儿童说:“野鸡在孵卵,不要伤害它。”雉,野鸡。},鹊始巢\footnote{语出《礼记·月令》:“雁北乡,鹊始巢,雉雊,鸡乳。”}。猛虎对神獒áo\footnote{传说能听懂人语的犬叫獒。}。疏星浮荇xìng 叶,皓月上松梢。为邦自古推瑚琏liǎn\footnote{《论语》载,一次孔子弟子子贡问老师:“我是怎样一个人?”孔子说:“你是能成器的。”又问:“我是怎样的器?”孔子说:“你是瑚琏。”瑚琏,古代宗庙盛黍稷的器皿,是祭祀的贵重礼器,比喻子贡会成为治国的人材。为邦,治理国家。},从政于今愧斗dǒu筲shāo\footnote{《论语》载,一次子贡问,当今做官的人怎么样,孔子说:“噫,斗筲之人,何足算也!”斗筲之人,即德薄才疏的人。斗和筲,都是古代用竹子做成的容量很小的容器。}。管鲍bào 相知,能交忘形胶漆友\footnote{春秋时,管仲和鲍叔牙交情非常好,患难与共,旧时代常以管鲍为朋友间的楷模。管仲,春秋初年政治家。经鲍叔牙推荐,被齐桓公任为上卿。相知,即相友好。胶漆,形容难解难分,关系极为密切。};蔺lìn 廉有隙,终为刎wěn 颈\footnote{指发誓同死的交情。}死生交。
\end{yuanwen}

\begin{yuanwen}
歌对舞,笑对嘲。耳语\footnote{凑近耳朵小声说话。}对神交\footnote{彼此慕名而没有见过面的交谊。}。焉鸟对亥hài 豕shǐ\footnote{古文之讹。焉和鸟,亥和豕,字形相近,往往造成讹误。焉鸟:谓字形相近而易讹。},獭tǎ髓suǐ\footnote{水獭,旧传水獭的髓是很好的滋补品,服食能益神智;相传水獭的骨髓与玉屑、琥珀屑相和,可以灭瘢痕。}对鸾胶\footnote{传说海上有凤麟洲,多仙人,以凤喙麟角合煎作膏,名续弦胶,能续弓弩断弦。}。宜久敬,莫轻抛。一气\footnote{犹云同气,指有血缘关系的亲属,多喻兄弟。}对同胞。祭zhài 遵甘布被\footnote{祭遵是东汉光武帝的将军。《后汉书·祭遵传》:遵为人克己奉公,凡皇帝的赏赐一律分给士卒,家无私财,穿皮裤,盖布被,夫人裳不加缘,因而受到皇帝的敬重。},张禄念绨tí袍\footnote{战国时,范睢和须贾同事魏王,须贾出于嫉妒,唆使魏相治范睢几至于死。后范睢逃到秦国,改名张禄,为秦相。后须贾使秦,范睢故意穿了一身破衣服去见须贾。贾不知其为秦相,说“范叔何一寒至此”,以己绨袍赠之。不久,须贾终于知道范睢原来就是秦相张禄,吓得赶忙登门请罪。范睢说:“根据你旧日对我的态度,本当把你处死。但你送我一件袍子,看来还有点情谊,可以饶你一命。”绨,光滑厚实的丝织品。}。花径风来逢客访\footnote{语出杜甫《客至》:“花径不曾缘客扫,蓬门今始为君开。”},柴扉月到有僧敲。夜雨园中,一颗不雕王子柰nài\footnote{《二十四孝》载:晋人王祥至孝,后母不慈,命其看护后园柰树,柰落则鞭之。祥抱树大哭,感动上天,柰一颗不落。柰,落叶小乔木,花白色,果小,是苹果的一种。};秋风江上,三重曾卷杜公茅\footnote{杜公指杜甫。杜甫居成都时,一次大风吹坏了草堂,他曾为此写作了《茅屋为秋风所破歌》,中有“八月秋高风怒号,卷我屋上三重茅”之句。}。
\end{yuanwen}

\begin{yuanwen}
衙\footnote{旧时官舍之称。}对舍\footnote{居住的房子。},廪lǐn\footnote{粮仓。}对庖páo\footnote{厨房。}。玉磬\footnote{古代的一种用玉或石制成的打击乐器。}对金铙náo\footnote{一种用金属制成的打击乐器。}。竹林\footnote{晋时嵇康与阮籍等七人为友,蔑视礼教,狂放不羁,经常聚在竹林中啸饮清谈,时人号为“竹林七贤”。}对梅岭\footnote{英州司寇种梅三十株于大庾岭,故庾岭多梅。},起凤对腾蛟\footnote{都是形容文采的超拔。}。鲛ji\=ao绡xiāo\footnote{古代神话,南海外有鲛人,住在水中,善织绩,常出卖绡,眼能泣泪成珠。鲛绡,鲛人所织的细绢。鲛,就是鲨鱼。绡,生丝,又指用生丝织的东西。}帐,兽锦\footnote{绣有麟、豹一类野兽花纹的锦缎。}袍。露果对风梢。扬州输橘柚,荆土贡菁j\=ing茅\footnote{《尚书》有《禹贡》篇,记述九州的山川土宜,提出扬州要贡赋桔柚,荆州要贡献菁茅。菁茅,一种草类,古人用以扎神像,灌酒其上,表示神饮,叫祼。}。断蛇埋地称孙叔\footnote{孙叔敖,战国时楚国令尹,幼时见两个头的蛇,杀而埋之,回家后对母亲哭诉。母问其故,他说:“人们说遇到两头蛇的人一定会死,今天我遇到了。为了不至于让更多的人见而致死,我已杀死并且埋掉了它。”母亲说:“我儿做了好事,必有善报。”后来孙叔敖果然做了楚国的令尹。},渡蚁作桥识宋郊\footnote{迷信传说,宋郊为士人时,所居堂前有蚁穴为雨水冲毁,他编竹为桥让蚂蚁爬到了干处,据说因为有此阴德,后为状元。}。好梦难成,蛩qióng 响阶前偏唧唧;良朋远到,鸡声窗外正嘐jiāo嘐。
\end{yuanwen}

\chapter{豪}

\begin{yuanwen}
茭ji\=ao对茨c\'i,荻dí 对蒿hāo\footnote{茭、茨、荻、蒿,都是指蒿草。}。山麓\footnote{山脚下。}对江皋g\=ao\footnote{江边的高地。}。莺簧\footnote{黄莺啼叫的声音美如笙簧。}对蝶板\footnote{蝴蝶的双翅忽开忽合好象乐器中的板。},麦浪\footnote{风吹麦田,麦子像波浪般起伏的样子。}对桃涛\footnote{春二三月,桃花盛开之时,河中春汛,称为桃花汛。}。骐q\'i骥jì 足\footnote{良马。骐骥足,比喻人有才干。},凤凰毛\footnote{凤毛麟角,喻稀有的优秀人才。}。美誉对嘉褒。文人窥蠹dù简\footnote{指被虫蛀蚀的书籍。蠹,蛀书虫。},学士书兔毫\footnote{用兔毛制成的笔。泛指毛笔。}。马援南征载薏yì苡yǐ\footnote{马援是东汉的将军,他南征交趾时,曾携带数车薏苡,以防治瘴疠。薏苡,多年生草本植物,即中药苡仁。},张骞qiān 西使进葡萄\footnote{汉武帝时,张骞曾两次出使西域,使汉族和少数民族、中国和外国的文化得以交流。从西域引进葡萄。}。辩口悬河,万语千言常亹wěi 亹;词源倒峡,连篇累lěi 牍自滔tāo滔\footnote{都形容人善于谈吐。亹亹,原意是勤奋的样子,这里是言不绝口的意思。词源倒峡:谓诗文雄健有力,气势豪迈。}。
\end{yuanwen}

\begin{yuanwen}
梅对杏,李对桃。棫yù 朴pò\footnote{棫、朴,两种灌木名,据说可点燃祭天神。《诗经·大雅》中有《棫朴》篇。棫,白桵。朴,桴木。意谓棫朴丛生,根枝茂密,共同附着。喻贤人众多,国家蕃兴。}对旌j\=ing旄máo\footnote{指旗帜。}。酒仙对诗史\footnote{杜甫有《饮中八仙歌》,称李白、贺知章、李琎、张旭等八人为酒仙。诗史:杜甫的许多诗,较为真实地记述了当时的社会状况,被人称为“诗史”。
},德泽对恩膏\footnote{泽和膏都是指及时的好雨,因而被比作恩德。}。悬一榻\footnote{后汉徐稚,字孺子,家贫,有德行,当时陈蕃为豫章太守,不接待宾客,只特设一榻待徐稚,徐来则放下,徐走后即悬起。},梦三刀\footnote{迷信传说,晋王浚夜梦梁上悬三把刀,后又增加一把,醒来问别人是何吉凶。解者曰:三刀是州字,又加一把是“益”的意思,是益州,所以您要做益州刺史了。后果守益州。}。拙zhuō 逸对贵劳。玉堂花烛绕,金殿\footnote{金饰的殿堂,指帝王的宫殿。}月轮高\footnote{指月亮。}。孤山看鹤盘云下\footnote{宋林逋,隐西湖孤山,常养两鹤,纵之则飞入云霄,盘旋久之乃下。},蜀道闻猿向月号h\'ao\footnote{古代四川多猿,所以民歌有“巴东三峡巫峡长,猿啼三声泪沾裳”的说法。}。万事从人,有花有酒应自乐;百年皆客,一丘一壑hè 尽吾豪\footnote{这是一种消极的人生观,认为人生百年不过如客人一样暂住世间,应放浪山水之间,尽其豪情。}。
\end{yuanwen}

\begin{yuanwen}
台对省shěng ,署对曹\footnote{台、省、署、曹,都是古时官府的名称。}。分袂mèi\footnote{古时把离别称作分袂。袂,袖子。}对同袍\footnote{最早出自《诗经·秦风·无衣》:“岂曰无衣?与子同袍。王于兴师,修我戈矛,与子同仇。”后来多为军人用以互称。后亦用来泛指朋友、同僚、同学等。}。鸣琴对击剑,返辙\footnote{晋阮籍由于当时政治昏暗,心情苦闷,常酒醉后乘车出游,遇到绝路就痛哭而回。}对回艚\footnote{艚,就是船。晋王献之曾在雪夜乘船去访问他的老朋友戴逵,走到半路,忽然命令船只返回。人们问什么缘故,他说自己是“乘兴而来,兴尽而返”。}。良借箸zhù\footnote{楚汉战争中,汉高祖听信郦生的话,准备把诸将分封于各地为侯王。张良认为这是错误的,就在酒宴前,借席上箸一一陈说道理。箸,筷子。},操提刀\footnote{传说匈奴使者要拜谒曹操,曹操自以为相貌不扬,恐为耻笑,于是让崔琰装扮成魏王,曹操自己装扮成卫士,提刀立旁。朝见后,让人问使者对魏王的印象。使者曰,魏王相貌亦复平常,但床头捉刀人(指曹操)乃真英雄。}。香茶对醇chún 醪láo\footnote{味厚的美酒。}。滴泉归海大,篑kuì 土积山高\footnote{都是说积少成多的意思。篑,古代盛土的筐子。}。石室客来煎雀舌\footnote{一种名茶。},画堂宾至饮羊羔\footnote{美酒名。}。被谪贾生,湘水凄凉吟《 鵩fú鸟》\footnote{汉贾谊被黜为长沙王太傅,内心悲苦,一日有猫头鹰进宅,人皆以为不祥,他就写了一篇《 鵩鸟赋》抒发情怀。鵩,一种猫头鹰类的鸟。};遭谗屈子,江潭憔悴著《离骚》\footnote{战国时期楚国大夫、爱国诗人屈原,由于佞臣毁谤,遭到楚王贬谪,曾在湘江一带流浪,《史记·屈原贾生列传》:“披发行吟泽畔,颜色憔悴,形容枯槁。”后投汩罗江而死。《离骚》是他写作的长诗。}。
\end{yuanwen}

\chapter{歌}

\begin{yuanwen}
微对巨,少对多。直干\footnote{挺直的树干。}对平柯kē\footnote{柯,树枝。平柯犹言横枝。}。蜂媒\footnote{比喻为男女双方居间撮合或传递消息的人。}对蝶使\footnote{比喻男女双方情爱的媒介。},雨笠\footnote{遮雨的笠帽。}对烟蓑\footnote{蓑衣。}。眉淡扫\footnote{描画。},面微酡tuó 。妙舞对清歌。轻衫裁夏葛\footnote{指夏天穿的葛衣。},薄袂mèi\footnote{衣袖。}剪春罗\footnote{适于春季穿的绫罗。}。将相兼行唐李靖\footnote{李靖,唐初著名军事家。他曾在建立唐王朝的斗争中屡立战功,后又平突厥之叛,三定朔方,被封为卫国公。将相兼行是说他才兼文武。},霸王杂用汉萧何\footnote{楚汉战争中,萧何辅佐汉高祖定三秦,后为汉相,制作律令,对汉王朝的建立和巩固卓有贡献。霸王杂用,是说“王道”和“霸道”两用。儒家称以力假仁者为霸,以德行仁政者为王。}。月本阴精,岂有羿妻曾窃药\footnote{古代神话传说,有穷国君后羿从西王母那里得到了长生药,其妻嫦娥窃之服用后飞升到月宫。本联意为,月本是阴气的精华,哪里有嫦娥飞升的事呢?};星为夜宿,浪传织女漫投梭\footnote{古代神话说,织女是天帝的孙女,整夜在那里织布。本联意为,世传牛郎织女隔天以梭相投。这种说法也是荒诞虚无的事。夜宿,夜间的星宿。浪传,胡传,乱传。}。
\end{yuanwen}

\begin{yuanwen}
慈对善,虐对苛kē 。缥缈对婆pó 娑suō\footnote{树木或人的身躯摇曳多姿的样子。}。长杨\footnote{汉宫殿名。}对细柳\footnote{周亚夫曾屯军细柳。},嫩蕊对寒莎suō\footnote{秋天的莎草。}。追风马\footnote{《淮南子》中有“以兔之走,使犬如马则逮日归(追)风”的说法,后常以追风形容马跑得快。},挽日戈\footnote{据《淮南子·览冥训》载,楚国的鲁阳公与韩国人作战,战到天晚未分胜负,他举起戈来向太阳下令,太阳从西方退了回来,他继续战斗。}。玉液\footnote{古人服食的用玉屑调成的药酒。}对金波\footnote{太阳照在水面或宫殿上反射回来的光线。}。紫诏衔丹凤\footnote{《晋书·石季龙载记》说,当时诏书以五色纸衔木凤之口,后世遂称皇帝诏令为凤诏。又解衔丹凤:古人书信用泥封,泥上盖印,皇帝诏书则用紫泥,称为紫泥诏或紫诏,常以龙凤为图饰。},黄庭换白鹅\footnote{晋书法家王羲之喜欢山阴道士养的鹅,于是为道士写了一卷《黄庭经》做为交换条件。}。画阁江城梅作调diào\footnote{这是对李白“黄鹤楼中吹玉笛,江城五月落梅花”两句诗的概括。梅作调,古代笛曲名有《梅花落》。},兰舟野渡竹为歌\footnote{此指歌咏民俗风土人情的《竹枝词》。}。门外雪飞,错认空中飘柳絮xù\footnote{晋才女谢道韫,有才辩,一次降雪,他的叔父谢安问子侄们:“大雪纷纭何所似?”谢朗说:“撒盐空中差可拟。”谢道韫说:“未若柳絮因风起。”谢安十分赞赏。};岩边瀑响,误疑天半落银河\footnote{语出李白《观庐山瀑布》:“飞流直下三千尺,疑是银河落九天。”}。
\end{yuanwen}

\begin{yuanwen}
松对竹,荇x\`ing\footnote{多年生草本植物。}对荷。薜b\`i荔\footnote{南方的一种蔓生植物。}对藤萝。梯云\footnote{登上云端。}对步月\footnote{在月光下散步。},樵唱对渔歌。升鼎雉zhì\footnote{传说殷王武丁时祭祀太庙,有野鸡飞落鼎耳上而鸣,古人认为是一种祥瑞。},听经鹅\footnote{据《太平御览》载,净影寺僧人慧远养的一只鹅,每到寺内讲经时,就随众僧人一同进入堂内伏听,讲经结束,便鸣叫着离开。}。北海\footnote{后汉孔融曾为北海太守,时人称之为北海,好宴客。他是当时著名的文人。}对东坡\footnote{宋代诗人苏轼,在黄冈东坡筑室,号东坡居士。}。吴郎哀废宅\footnote{吴郎指唐代吴融,他曾写有《废宅》诗:“风飘碧瓦雨摧垣,却有邻人与锁门。”感叹朝廷衰败,官员门宅都荒芜、旷废的景象。},邵子乐行窝\footnote{宋经学家邵雍隐居不仕,居洛阳三十年,筑“安乐窝”以居,自称安乐先生。}。丽lí 水良金皆待冶,昆山美玉总须磨\footnote{旧传金生丽水,玉出昆仑。}。雨过皇州,琉璃色灿华清瓦;风来帝苑yuàn ,荷芰jì 香飘太液波\footnote{描写风雨中帝都景象。太液,即太液池,西汉时在长安掘成的人造湖。华清,即华清宫,在金陵,六朝陈时所建。}。
\end{yuanwen}

\begin{yuanwen}
笼对槛jiàn ,巢对窝。及第\footnote{指科举考试考中,特指考中进士,明清两代只用于殿试前三名。}对登科\footnote{科举时代应考人被录取。}。冰清对玉润\footnote{晋乐广、卫玠翁婿俱有名,时人称乐广为冰清,其婿卫玠为玉润,喻人品高洁。},地利对人和\footnote{语出《孟子·公孙丑下》:天时不如地利,地利不如人和。}。韩擒虎\footnote{隋朝大将,屡立战功,渡江平陈战役就是由他统帅的。},荣驾鹅\footnote{春秋时鲁昭公之大臣。}。青女\footnote{传说中的霜神。}对素娥\footnote{即嫦娥,月色白,故又称素娥。李商隐诗:“青女素娥俱耐冷,月中霜里斗婵娟。”}。破头朱泚cǐ 笏hù\footnote{唐德宗时,京师兵变,德宗出逃,太尉朱泚欲窃位,司农卿段秀实执象笏击破其头,卒遭所害。笏,古代大臣登朝所持用以记事的手板。},折齿谢鲲kūn 梭\footnote{《晋书·谢鲲传》:“邻家高氏女有美色,鲲尝挑之,女投梭,折其两齿。”}。留客酒杯应恨少,动人诗句不须多。绿野凝烟,但听村前双牧笛;沧江积雪,惟看滩上一渔蓑\footnote{唐柳宗元《江雪》诗:“孤舟蓑笠翁,独钓寒江雪。”}。
\end{yuanwen}

\chapter{麻}

\begin{yuanwen}
清对浊,美对嘉。鄙吝\footnote{形容心胸狭窄。}对矜夸\footnote{夸耀。}。花须\footnote{花蕊伸展如须。}对柳眼\footnote{柳叶如眉眼。},屋角对檐牙\footnote{檐际翘出如牙的部分。}。志和宅\footnote{唐诗人张志和,肃宗朝命待诏翰林,授左金吾卫录事参军,后遭贬黜,遂不复仕。浪迹江湖,言以太虚(天)为庐,明月为伴,自号烟波钓徒。},博望槎chá\footnote{博望,即张骞,因奉使西域有功封博望侯。《荆楚岁时记》:“汉武帝令张骞使大夏,寻河源。乘槎经月,而至一处,见城郭和州府,室内有一女织,又见一丈夫牵牛饮河。骞问曰:‘此是何处?’答曰:‘可问严君平。’织女取榰机石与骞而还。”始知已到牛郎、织女星。槎,木筏。}。秋实对春华\footnote{即春华秋实,古人比喻文采与德行。}。乾炉烹白雪,坤鼎炼丹砂\footnote{都是道教说法。乾炉指男,坤鼎指女。}。深宵望冷沙场月,边塞听残野戍笳jiā 。满院松风,钟声隐隐为僧舍[8];半窗花月,锡\footnote{僧人所持杖称锡。}影依依\footnote{隐隐约约的样子。}是道家。
\end{yuanwen}

\begin{yuanwen}
雷对电,雾对霞。蚁阵\footnote{蚂蚁排阵而战。引申为争强斗胜。}对蜂衙\footnote{蜂早晚定时的聚集,如下属参谒长官于衙中,故称为蜂衙。}。寄梅对怀橘\footnote{东汉陆绩幼时拜见袁术,将桌上的橘子藏在怀中准备带回家给自己母亲吃。},酿酒对烹茶。宜男草\footnote{即萱草,古人以为孕妇佩之可生男。},益母花\footnote{中药名。}。杨柳对蒹葭\footnote{即芦苇。}。班姬辞帝辇\footnote{汉成帝游后苑,命班婕妤同辇,班婕妤说:“古代圣贤之君,都有名臣在旁;只有末代皇帝才亲近女色。”成帝听了很钦佩。},蔡琰yǎn 泣胡笳\footnote{蔡琰,即蔡文姬,蔡邕女,汉末著名才女,早寡,汉末被虏入胡,在南匈奴生活了十二年,后被曹操赎回。传说她曾写了《胡笳十八拍》,历述她的不幸遭遇。}。舞榭歌楼千万尺,竹篱\footnote{用竹编的篱笆。}茅舍\footnote{茅屋,草屋。}两三家。珊枕\footnote{即珊瑚枕。}半床,月明时梦飞塞外;银筝一奏,花落处人在天涯。
\end{yuanwen}

\begin{yuanwen}
圆对缺,正对斜。笑语对咨嗟jiē\footnote{文言叹词,叹息。}。沈腰\footnote{南朝梁文学家沈约,字休文,体弱多病,腰肢纤弱。}对潘鬓bìn\footnote{晋文学家潘岳,由于屡遭不幸,身体早衰,在《秋兴赋》中,他曾自伤两鬓早白,说自己三十二岁“始见二毛”。},孟笋\footnote{孟宗母病中喜吃笋,因时节正值冬季,无笋可取,宗入竹林悲泣哀叹,笋竟为之而生。后人遂用来形容人子事亲尽孝,至诚感天,并将之列入“二十四孝”中。}对卢茶\footnote{唐代诗人卢仝好茶成癖,诗风浪漫,曾作《走笔谢孟谏议寄新茶》。}。百舌鸟\footnote{鸟名,又名乌鸫(dōng)。益鸟,喙尖,毛色黑黄相杂,鸣声圆滑。},两头蛇\footnote{见肴韵“断蛇”句注。}。帝里\footnote{犹言帝乡,指上帝所居之处。}对仙家。尧仁敷率土,舜德被bèi 流沙\footnote{都是对尧舜的称颂。敷率土,是说遍及所有的地方。被,遍及。流沙,古人指中国以西极远的地区。}。桥上授书曾纳履\footnote{传说张良年轻时曾遇到一位坐在下邳圯(桥)上的老人,命他到桥下去取失落的鞋,张良恭恭敬敬地做了这件事,老人很高兴,说孺子可教也,就授予他三卷兵书,并说自己就是黄石公。纳履,穿鞋。},壁间题句已笼纱\footnote{唐代王播少孤贫,客居扬州惠招寺木兰院,随僧斋食,为诸僧所不礼。后播显贵重游旧地,见昔日在该寺壁上所题诗句,僧已用碧纱盖护,因题曰:“上堂已散各西东,惭愧阇梨饭后钟。三十年来尘扑面,如今始得碧纱笼。”}。远塞迢tiáo 迢,露碛q\`i\footnote{水中堆沙。}风沙何可极;长沙渺miǎo 渺,雪涛烟浪信无涯。
\end{yuanwen}

\begin{yuanwen}
疏对密,朴对华。义鹘gǔ\footnote{鹰类鸷禽。鸷,凶猛的鸟。}对慈鸦\footnote{古人传说乌鸦是孝鸟,老鸟不能取食时,小鸟能反哺其母,因称慈鸦。}。鹤群对雁阵,白苎zhù\footnote{一种麻类,皮可为纺织原料。}对黄麻\footnote{此指黄麻纸,唐时以黄麻纸写诏书。}。读三到\footnote{古人经验,读书要眼到、口到、心到。},吟八叉\footnote{唐诗人温庭筠才思敏捷,传说他八叉其手而诗成,人呼之为温八叉。}。肃静对喧哗。围棋兼把钓,沉李并浮瓜\footnote{古人消暑,往往置水果于冷水中,故有沉李浮瓜之说。}。羽客片时能煮石\footnote{羽客,即仙人。道教说仙人能煮白石为饭。},狐禅千劫似蒸沙\footnote{佛教说法,狐禅毫无意义,犹如蒸沙土,虽历尽千劫,不能成饭。佛经云,狐禅如蒸沙,千劫不能成饭。}。党尉粗豪,金帐笼lǒng香斟美酒;陶生清逸,银铛chēng融雪啜chuò 团茶\footnote{《事文类聚》载,宋学士陶毂得党太尉家姬。一次烹雪茶,陶问姬曰:“党家有此味否?”姬曰:“彼但知坐销金帐里,共饮羊羔美酒,浅斟低唱而已。”铛,平底锅。}。
\end{yuanwen}

\chapter{阳}

\begin{yuanwen}
台对阁,沼对塘。朝雨对夕阳。游人对隐士,谢女\footnote{指晋代才女谢道韫,人称咏絮高才。}对秋娘\footnote{即杜秋娘,唐宗室李锜(qí)妾,能诗。}。三寸舌\footnote{指能说善辩。史载战国时毛遂以三寸之舌,强于百万之师。},九回肠\footnote{形容人心情郁闷。}。玉液对琼浆\footnote{都是道教服食的药饵。}。秦皇照胆镜\footnote{传说秦始皇有照胆镜,能透视人的内脏,发现有人胆张心动,就意味着要暗害他,当即杀掉。},徐肇zhào 返魂香\footnote{《十洲记》载,西海申未洲上有大树,叶香闻数百里,煎制成膏,名返生香,死尸在地,闻之可活。又释徐肇遇苏德音,授以返魂香,燃之,能起上世亡魂。}。青萍\footnote{宝剑名。}夜啸芙蓉匣,黄卷\footnote{用绢书写的书籍,此指道书或佛经。}时摊薜bì 荔床。元亨利贞,天地一机成化育\footnote{元亨利贞,是《周易·乾卦》中的一句。古人解释说:“元者善之长也,亨者嘉之会也,利者义之和也,贞者事之干也。”称为四德。二句的意思是,由于天地有此四德,才化生了万物。};仁义礼智,圣贤千古立纲常。
\end{yuanwen}

\begin{yuanwen}
红对白,绿对黄。昼永对更长。龙飞对凤舞,锦缆对牙樯qiáng\footnote{用锦缎做缆绳,以象牙为樯橹。樯,桅杆。}。云弁biàn 使\footnote{指蜻蜓。},雪衣娘\footnote{白鹦鹉。},故国对他乡。雄文能徙鳄\footnote{潮州有鳄鱼为害,韩愈做刺史,作《祭鳄鱼文》驱之,传说鳄鱼就迁到了它地。},艳曲为求凰\footnote{汉时成都卓王孙有女文君新寡,司马相如爱上了她,作《凤求凰》曲以挑之,文君于是同他私奔。}。九日高峰惊落帽\footnote{晋孟嘉为桓温之参军,九月九日游龙山,群僚毕集,有风将孟嘉帽子吹落而不觉。孙盛作文嘲笑,他即时作答,四座皆服。},暮春曲水喜流觞\footnote{晋永和上巳日农历三月初三,王羲之、王献之、谢安、孙绰诸人曾在山阴兰亭集会,于水边嬉游采兰,曲水流觞,饮酒赋诗以娱,以消除不祥,称为修禊。王羲之有《兰亭集序》记此事,文中有“暮春之初”“引以为流觞曲水”等语。}。僧占名山,云绕茂林藏古殿;客栖胜地\footnote{著名的景色宜人的地方。},风飘落叶响空廊。
\end{yuanwen}

\begin{yuanwen}
衰对壮,弱对强。艳饰\footnote{犹言浓妆打扮。}对新妆\footnote{指女子刚修饰好的仪容,或指女子新颖别致的打扮修饰。}。御龙\footnote{驾御龙。传说夏时刘累曾为孔甲养龙,因赐姓为御龙氏。}对司马\footnote{官名,也是姓。},破竹\footnote{比喻做事顺利。}对穿杨\footnote{《战国策·西周策》载,楚将养由基善射,百步之内,可穿杨叶。}。读班马\footnote{班固作《汉书》,司马迁作《史记》。},识求羊\footnote{据《三辅决录·逃名》载,西汉末,蒋诩解官归桂林后,于竹林中开三条小径,惟故人求仲、羊仲从之游,不与俗人往还。}。水色\footnote{水面呈现的色泽。}对山光\footnote{山的景色。}。仙棋藏绿橘\footnote{神话故事,巴邛人家有橘树,一年忽长三枚,果实大如斗,剖之有二叟对弈。},客枕梦黄粱。池草入诗因有梦\footnote{传说南朝宋诗人谢灵运一次生病,因梦见族弟惠连而得“池塘生春草,园柳变鸣禽”之佳句。},海棠带恨为无香\footnote{《冷斋夜话》载,宋彭渊林曰:吾生平五恨。一恨鱼多骨,二恨橘多酸,三恨菜性淡,四恨海棠无香,五恨曾子固不能诗。曾子固,曾巩,字子固,古文“唐宋八大家”之一。}。风起画堂\footnote{泛指华丽的堂舍。},帘箔\footnote{帘子。多以竹、苇编成。}影翻青荇沼;月斜金井\footnote{井栏上有雕饰的井。一般用以指宫庭园林里的井。},辘轳\footnote{井上的汲水器。}声度碧梧墙。
\end{yuanwen}

\begin{yuanwen}
臣对子,帝对王。日月对风霜。乌台\footnote{《汉书·朱博传》载,时御史府中列柏树,常有野乌数千栖息其上,后因称御史府(台)为乌台。}对紫府\footnote{道家称仙人居所。},雪牖y\v{o}u\footnote{雪窗。}对云房\footnote{僧、道或隐者所居之室。}。香山社\footnote{唐白居易于洛阳与胡杲(gǎo)、吉皎等八位老人结为九老会,因结于香山,故称为香山九老社。},昼锦堂\footnote{北宋韩琦封魏国公,在做武康节度使时,于故乡相州修了一所殿堂,取名昼锦堂以致其荣,致仕退老其中。文学家欧阳修曾写有《昼锦堂记》,详述其事。}。蔀bù 屋\footnote{指草屋。}对岩廊\footnote{高大的宫殿。}。芬椒涂内壁\footnote{汉代皇后所居宫室,以椒和泥涂内壁,取其香和多子之意,称椒房。},文杏饰高粱\footnote{旧题司马相如《长门赋》:“饰文杏以为梁。”后以杏梁指建筑华美。}。贫女幸分东壁影\footnote{《战国策》载寓言故事,齐女与邻妇共烛而绩,妇辞之,女曰:“我贫无烛。一室之中,多不为暗,少不为明,何惜东壁余光。”邻妇觉得有理,就留下了她。唐李白诗:“愿假东壁辉,余光照贫女。”},幽人高卧北窗凉\footnote{晋代陶潜《与子俨等疏》:“常言五六月中,北窗下卧,遇凉风暂至,自谓是羲皇上人。”意思是说他自己夏日卧北窗下,每当凉风吹来,就好像回到了无忧无虑的太古时代一样。}。绣阁\footnote{古代女子的闺房。}探春,丽日半笼lǒng 青镜\footnote{即青铜镜。}色;水亭醉夏,薰风常透碧筒\footnote{三国魏郑悫(què)取荷茎通之以盛酒,名曰碧筒杯。}香。
\end{yuanwen}

\chapter{庚}

\begin{yuanwen}
形对貌,色对声。夏邑对周京。江云对涧树,玉磬\footnote{古代石制乐器名。}对银筝\footnote{用银装饰的筝或用银字表示音调高低的筝。}。人老老\footnote{语出《孟子》:“老吾老,以及人之老;幼吾幼,以及人之幼。”意思是,尊敬自己的老人,从而也尊敬别人的老人;爱自己的孩子,从而也爱别人的孩子。人老老,即尊敬别人的老人。第一个“老”字作动词用。},我卿卿\footnote{卿是对人的尊称,也是对妻子的昵称。西晋大臣王戎妻呼戎曰卿。戎曰:“奈何卿我?”妻曰:“我不卿卿,谁当卿卿?”(意为我不称你为卿,还有谁称你为卿呢)?故后以“卿卿我我”作为夫妻恩爱之典。}。晓燕对春莺。玄霜舂玉杵\footnote{唐裴銏《传奇》中讲一个故事,下第秀才裴航,遇到仙人云翘夫人,赠诗一首曰:“一饮琼浆百感生,玄霜捣尽见云英。蓝桥便是神仙窟,何必崎岖上玉京。”后裴生经蓝桥驿,果遇一妪揖之求饮,妪使云英持瓯浆,令饮之。因诗合,欲娶云英,妪命裴购玉杵并捣药,果得玉杵。聘之,俱仙去。玄霜,传说中的仙药。},白露贮zhù 金茎\footnote{汉武帝好神仙之术,史载他曾作铜柱,上有铜仙人擎玉盘,承接夜露,据说以此露和玉屑饮之可长生。杜甫诗:“蓬莱宫阙对南山,承露金茎霄汉间。”又魏明帝亦作承露金茎,高十一丈。}。贾gǔ客君山秋弄笛\footnote{《博异志》载,有商人吕乡筠,善吹笛,一次泊舟君山附近,遇到一位老人,合上天神乐、仙乐和自己欣赏的三支仙笛,吹奏数声,湖上风波大作。},仙人缑gōu 岭夜吹笙\footnote{传说周灵王太子晋好吹笙,作《凤凰鸣》,遇浮邱公,接上蒿山。后于七月七日乘白鹤过缑氏山头,拱手谢别时人而去。缑岭,山名,在河南。}。帝业独兴,尽道汉高能用将\footnote{史载汉高祖刘邦善于用人,因而取得天下。汉高帝问韩信带兵几何?信曰:“多多益善。”帝曰:“卿何为我擒耶?”曰:“陛下不善将兵,而善将将。”};父书空读,谁言赵括善知兵\footnote{赵奢是战国时赵之名将。奢死,赵王令以其子赵括代廉颇为将。蔺相如说,赵括只能读其父的兵书,没有实际经验。赵王不听,使其率兵与秦交战。结果赵括中箭死,几十万军队都投降秦国,被秦人活埋了。}。
\end{yuanwen}

\begin{yuanwen}
功对业,性对情。月上对云行。乘龙\footnote{唐杜甫《李监宅》:“门阑多喜气,女婿近乘龙。”因称女婿为乘龙快婿。}对附骥jì\footnote{靠别人的力量使自己得以发展,喻附于先辈或名人之后。},阆làng 苑对蓬瀛yíng [11] 。春秋笔[12] ,月旦评\footnote{汉末河南许劭(shào)与其兄许靖俱有高名,好在一起甄别、评论当地人物,每月变换一次,农历初一发布公告,人们称之为“月旦评”。后称品评人物为月旦评或月旦。}。东作对西成[14] 。隋珠光照乘\footnote{传说一次隋侯出行,遇断蛇于路,隋侯命人给蛇敷药包扎,后蛇衔径寸之珠报偿隋侯,因称隋侯珠。光照乘,是说把这种宝珠挂在车上可以照明前后。},和璧价连城。三箭三人唐将勇[16] ,一琴一鹤赵公清[17] 。汉帝求贤,诏访严滩逢故旧[18] ;宋廷优老,年尊洛社重耆q\'i英[19] 。
\end{yuanwen}

\begin{yuanwen}
昏\footnote{黄昏。}对旦\footnote{早上。},晦\footnote{昏暗不明。}对明。久雨对新晴。蓼liǎo 湾\footnote{长有蓼草的水湾。}对花港,竹友对梅兄。黄石叟sǒu\footnote{即汉初张良所遇仙人黄石公,曾赠给张良兵书。},丹丘生\footnote{道教传说中的仙人。丹丘,神话中的神仙之地,昼夜长明。}。犬吠对鸡鸣。暮山云外断,新水月中平。半榻清风宜午梦,一犁好雨趁春耕。王旦登庸,误我十年迟作相\footnote{《宋史·王旦传》载,宋相王旦柄权十八年,死后,王钦若继为宰相。王钦若语人曰:“子明(即王旦)迟我十年作宰相。”登庸,选拔任用。};刘蕡fén 不第,愧他多士早成名[24]。
\end{yuanwen}

[11] 阆苑:传说中的仙境,在昆仑山上。蓬瀛:即蓬莱山,传说东海中的仙山。

[12] 春秋笔:旧说孔子作《春秋》,寓褒贬于字里行间,后称此种笔法为春秋笔。

[14] 东作、西成:《尚书·尧典》中有“平秩东作”“平秩西成”的话,“东作”是开始耕作,“西成”是收获之意。

[16] 三箭三人唐将勇:唐将薛仁贵东征与九姓突厥交战,三箭毙三人,威震军中。当时有歌谣曰:“将军三箭定天山,壮士长歌入汉关。”

[17] 一琴一鹤赵公清:宋赵汴治成都,匹马入蜀,以一琴一鹤相随,为政清廉简易。

[18] 汉帝求贤,诏访严滩逢故旧:即汉光武帝与严子陵的故事。见微韵“严滩”注。光武帝与严子陵友善,即位命访之,陵在富春江披蓑钓泽中,载以至朝,帝以故人礼敬之。尝以同寝,陵以足加腹。太史奏曰,有客星犯主座。

[19] 宋廷优老,年尊洛社重耆英:宋相文彦博,致仕后在洛阳同富弼、司马光等十三人,饮酒赋诗相乐,谓之耆英会。耆,年老。耆英,高年硕德的人。

[24] 刘 不第,愧他多士早成名:见文韵“唐廷”注。

\chapter{青}

\begin{yuanwen}
庚对甲,己对丁。魏阙\footnote{高大的城阙。魏,通巍,形容高大。}对彤庭\footnote{指帝王宫殿。}。梅妻对鹤子\footnote{宋林逋隐居西湖孤山,以梅鹤自娱。逋不娶,无子,时人说林“梅妻鹤子”。},珠箔对银屏\footnote{语出唐白居易《长恨歌》:“珠箔银屏迤逦开。”}。鸳浴沼,鹭飞汀t\=ing。鸿雁对鹡jí 鸰líng\footnote{鸟名,生活在水边,食小虫,喜欢群飞。}。人间寿者相\footnote{旧时迷信,讲论骨相。寿者相,就是看上去长寿的相貌。},天上老人星\footnote{《史记·天官书》载天上有南极老人星,主寿。}。八月好修攀桂斧\footnote{神话传说,汉人吴刚,因学仙有过,罚他砍月中桂树,桂树高五百尺,砍后伤口复合,所以吴刚要永远砍下去。旧时以科举登第为攀桂,考试一般定在八月,称“秋闱”。},三春须系护花铃\footnote{明代宁王爱花,尝作护花铃,蜂、鸟至则牵铃惊之。}。江阁凭临,一水净连天际\footnote{天边。}碧;石栏闲倚,群山秀向雨余青。
\end{yuanwen}

\begin{yuanwen}
危对乱,泰对宁。纳陛\footnote{原意是深入殿堂的台阶,这里是登上台阶的意思。}对趋庭\footnote{快步走过庭院。《论语》记载:孔子的儿子孔鲤,一次趋庭而过,被孔子叫住,问他学诗学礼的情况。以后就把见父亲叫趋庭。}。金盘对玉箸,泛梗\footnote{《说苑》中的一则寓言,孟尝君入秦,客止之。见有木梗人谓土偶人曰:“今将大雨,子必沮坏。”答曰:“我沮,乃反吾真耳。今子,东园之桃也。刻子以为梗,雨至必浮,子泛泛不知所至矣。”孟尝君乃止。后遂以泛梗比喻到处漂流,无处安身。梗,这里指木偶。}对浮萍。群玉圃pǔ\footnote{传说仙人西王母居住在群玉山的瑶圃。},众芳亭。旧典\footnote{旧时的制度、法则。}对新型。骑牛闲读史\footnote{隋末李密好学,常将《汉书》一帙挂于牛角之上,骑牛读书。},牧豕shǐ 自横经\footnote{汉公孙宏,年少时生活清贫,为人放猪,但自己勤奋学习,常带经卷读。年五十后位至丞相。}。秋首田中禾颖yǐng\footnote{带芒的谷穗。}重,春馀园内菜花馨\footnote{弥漫很远的香气。}。旅次\footnote{旅途中小住的地方。也指旅途中暂作停留。}凄凉,塞月江风皆惨淡;筵y\'an前欢笑,燕歌赵舞独娉pīng 婷\footnote{古代燕、赵多出歌伎,其人善歌舞。娉婷:舞姿优美的样子。}。
\end{yuanwen}

\chapter{蒸}

\begin{yuanwen}
萍\footnote{水生植物。}对蓼liǎo\footnote{一年生或多年生草本植物。},茭jiǎo\footnote{茭白,菰的花径一种菌侵入后,刺激其细胞增生而成的肥大嫩茎,可作蔬菜。}对菱\footnote{一年生水生草本植物,果实有硬壳,有角,称“菱”或“菱角”,可食。}。雁弋\footnote{一种尾上带绳子的箭。雁弋即射雁的这种箭。}对鱼罾zēng\footnote{一种用竹竿或木棍做的方形鱼网。}。齐纨wán 对鲁绮qǐ ,蜀锦jǐn 对吴绫\footnote{纨、绮、锦、绫都是名贵的丝织品;齐、鲁、蜀、吴是上述四种织品的产地。}。星渐没,日初升。九聘对三征\footnote{聘和征都是王朝或官府聘请的意思。九聘,多次聘请。三征,朝廷三次征召。}。萧何曾作吏\footnote{史载萧何曾做沛郡的主吏椽,是管人事的小官。},贾岛昔为僧\footnote{唐诗人贾岛曾为僧人,法名无本。韩愈赏其诗才,令其还俗,劝其读书,后登进士,官长江主簿。}。贤人视履循规矩\footnote{《尔雅·释言》:“履,礼也。”注:“礼可以履行也。”所以说视履成规矩。},大匠挥斤校准绳\footnote{《庄子》中的一则寓言说,郢人在鼻子尖上涂一点白土,一位石匠把父子抡得呼呼响,一下子就把泥点砍掉了,对鼻子丝毫无损。大匠,技术高超的匠人。斤,斧子的一种。}。野渡春风,人喜乘潮移\footnote{这里指划船。}酒舫\footnote{舫:船,画舫(装饰华美专供旅游用的船);酒舫,载酒或卖酒的船。};江天暮雨,客愁隔岸对渔灯。
\end{yuanwen}

\begin{yuanwen}
谈对吐,谓对称。冉rǎn 闵mǐn 对颜曾zēng\footnote{冉有、闵子骞、颜渊、曾参都是孔子的高足弟子。}。侯嬴\footnote{战国时魏人,初为大梁(今河南开封)夷门的守门小吏,慷慨任侠,帮助信陵君窃符救赵,最后以身殉之。王维《夷门歌》专咏此事。}对伯嚭pǐ\footnote{即太宰嚭,春秋时楚伯州犁之孙,吴国奸臣。他受越王贿赂,劝吴王同越讲和。勾践灭吴,以伯嚭对其主不忠,杀之。},祖逖tì\footnote{东晋时爱国将领。见先韵“祖生鞭”注。}对孙登\footnote{晋初隐士。}。抛白纻zhù\footnote{宋裴思谦登第,以红笺数十幅入平康赋诗。王元之有诗云:“利市襕衫抛白纻,风流名字写红笺。”白纻,白苎麻织成的衣服。白纻襕衫,唐举子之服。},宴红绫[15] 。胜友对良朋。争名如逐鹿[16] ,谋利似趋蝇[17]。仁杰姨惭cán 周不仕\footnote{唐狄仁杰为武后相,其姨卢氏有子,杰欲官之,姨曰:“姨止一子,不欲令事后周女主。”仁杰大惭而归。周,武则天的国号。},王陵母识汉方兴\footnote{王陵事汉,其母在楚,知汉必兴,嘱善事之。项羽令母召陵,母遂自刎。}。句写穷愁,浣花寄迹传工部[20] ;诗吟变乱,凝碧伤心叹右丞\footnote{王维官尚书右丞相,后人称之为王右丞。安史之乱陷身贼中,被迫为给事中。传说安禄山宴于凝碧宫,令乐人作乐,维闻而伤之,作七绝一首云:“万户伤心生野烟,百僚何日更朝天。秋槐叶落空宫里,凝碧池头奏管弦。”}。
\end{yuanwen}

[15] 宴红绫:唐御膳以红绫饼为重。昭宗时放进士榜,得裴格等二十八人,会宴曲江,命御厨烧作红绫饼二十八枚赐之。

[16] 逐鹿:逐鹿中原,原指在战场上争夺政权。后来又有“未知鹿死谁手”的话,比喻胜负难定,这里即用此意。

[17] 趋蝇:追赶苍蝇。古有“蝇头微利”的说法,“趋蝇”是说十分不值得。

[20] 句写穷愁,浣花寄迹传工部:这是写杜甫的事,杜拾遗曾为检校员外郎,后人称之为杜工部。晚年流落蜀中,寓居成都西郊浣花溪旁之浣花村草堂。

\chapter{尤}

\begin{yuanwen}
荣对辱,喜对忧。缱qiǎn 绻qu\v{a}n对绸chóu 缪móu\footnote{缱绻、绸缪,都是形容感情亲密、情意缠绵的样子。}。吴娃\footnote{吴地的姑娘。娃,少女。}对越女\footnote{古代越国多出美女,西施其尤著者。后因以泛指越地美女。},野马\footnote{《庄子·逍遥游》中说:“野马也,尘埃也,生物之以息相吹也。”野马说的是早春大地上蒸腾的水蒸气。}对沙鸥\footnote{指栖息沙洲的鸥一类的水鸟。}。茶解渴,酒消愁。白眼对苍头\footnote{在秦末农民大起义中,有一支义军的士卒以青巾裹头,称苍头军。后世苍头多指老年仆人。}。马迁修《史记》,孔子作《春秋》。莘sh\=en野耕夫闲举耜sì\footnote{此句疑用伊尹故事。《吕氏春秋》说:有侁(shēn)氏女子得婴儿于空桑之中,名伊尹,长而贤,商汤王准备聘请他,有侁氏不肯,汤于是聘有侁氏女,以伊尹为陪嫁奴隶取了去,后以为相,国大治。有侁氏即有莘氏。},渭滨渔父晚垂钩\footnote{指商代末年姜尚的故事。见萧韵“千载”注。}。龙马游河,羲帝因图而画卦[7] ;神龟出洛,禹王取法以明畴chóu\footnote{上古传说,夏禹曾参照洛水神龟献出的宝书,制定了“洪范九畴”。}。
\end{yuanwen}

\begin{yuanwen}
冠对履,舄xì\footnote{鞋。}对裘。院小对庭幽。面墙\footnote{《论语》记述孔子的话说:“人而不为《周南》《如南》,其犹正墙面而立也与?”后来“面墙”就成了思路闭塞的代用语。}对膝地\footnote{两膝着地。},错智对良筹\footnote{错指西汉政治家晁错,他在文帝时曾为太子家令。太子家令是主管太子府内庶务的官员,相当于太子府的总管,很有谋略,多智,大家称他为“智囊”。良指张良。良筹是说张良的高明策略。又解为汉初张良借箸筹画政事。}。孤嶂耸sǒng ,大江流。芳泽\footnote{泽本是妇女用的脂粉,或说内衣,后芳泽即转为女性的代称。}对圆丘\footnote{是古代天子祭祀天神的地方,也写作圜丘。}。花潭来越唱,柳屿起吴讴ōu\footnote{吴地的民歌。}。莺懒燕忙三月雨,蛩摧蝉退一天秋。钟子听琴,荒径入林山寂寂\footnote{上古故事,俞伯牙善于弹琴,钟子期善解琴,闻伯牙鼓“高山流水”曲,遂相知好。子期死,伯牙碎琴不复鼓,谓无知音也。};谪仙捉月,洪涛接岸水悠悠\footnote{古代民间传说,诗人李白特别喜爱明月,在采石矶,一次酒醉,看到江心倒映的月影,就前去扑捉,结果溺水而死。谪:封建时代特指贬官。}。
\end{yuanwen}

\begin{yuanwen}
鱼对鸟,鹡jí 对鸠jiū 。翠馆对红楼[16] 。七贤\footnote{晋嵇康与阮籍、山涛、向秀、阮咸、王戎、刘伶友好,常宴集于竹林之下,号为竹林七贤。}对三友\footnote{以三种事物为友,如松、竹、梅;琴、酒、诗;梅、石、竹等。},爱日\footnote{珍惜时间。}对悲秋\footnote{看到秋天草木凋零而感到伤悲。}。虎类狗[19] ,蚁如牛\footnote{晋殷浩患耳疾,听见床下蚂蚁动,以为是牛斗之声。}。列辟[21] 对诸侯。陈唱临春乐[22] ,隋歌清夜游[23] 。空中事业麒麟阁\footnote{汉宣帝时,为了表彰功臣,将霍光、苏武等画在麒麟阁上,共十一人。“空中事业”,是说功名富贵本来是虚幻的,这是作者的消极思想。},地下文章鹦鹉洲\footnote{三国时才士祢衡,因反对曹操,被排挤到荆州,后被刘表部下黄祖(忌其才)杀害。他曾写过《白鹦鹉赋》,因此人们把它被害之处称之为鹦鹉洲。“地下文章”是说该人已死去。}。旷野平原,猎士马蹄轻似箭;斜风细雨,牧童牛背稳如舟。
\end{yuanwen}

[7] 龙马游河,羲帝因图而画卦:见鱼韵“洛龟”注。


[16] 翠馆:犹青楼,妓院。红楼:犹青楼。妓女所居。

[19] 虎类狗:东汉马援在《戒兄子严敦书》中,告诫他们说,学龙伯高,不成犹为谨慎之士,所谓刻鹄不成尚类鹜;学习豪侠好义的杜季良,不成刚为天下轻薄子,所谓画虎不成反类狗。

[21] 列辟:诸王侯。

[22] 陈唱临春乐:南朝陈后主荒淫,修结绮、临春、望仙阁,与张丽华、江总、孔贵嫔诸人日夜游戏、歌唱,其中以《玉树后庭花》《临春乐》为最有名。

[23] 隋歌清夜游:传说隋炀帝夏夜宴游,放萤火虫照明,歌清夜之曲;冬日剪彩为花。

\chapter{侵}

\begin{yuanwen}
歌对曲,啸\footnote{撮口作声,打口哨。}对吟\footnote{声调抑扬地念。}。往古对来今。山头对水面,远浦对遥岑cén\footnote{远处陡峭的小山崖。}。勤三上\footnote{古人经验,认为善读者有“三上”之功,即枕上、途上、厕上。},惜寸阴\footnote{东晋大将陶侃致力收复中原,朝夕运甓(pì,砖),常勉励大家说:“大禹惜寸阴,吾人当惜分阴。”寸阴,很短的时光。}。茂树对平林\footnote{平原上的林木。}。卞和三献玉\footnote{见庚韵“和璧”句注。卞和即和氏。},杨震四知金。青皇\footnote{又称东皇、青帝。东方为春,古人所谓司春之神,故代指春天。}风暖吹芳草,白帝城高急暮砧zh\=en\footnote{唐杜甫《秋兴八首》诗:“寒衣处处催刀尺,白帝城高急暮砧。”白帝城在四川重庆市奉节县,三国刘备殁于此。砧,捣衣石,这里指砧杵之声。}。绣虎\footnote{曹子建有奇才,七步成诗,人称绣虎之手。}雕龙\footnote{南朝梁刘勰《文心雕龙》论古今文章的体裁和创作,有很高价值。},才子窗前挥彩笔;描鸾刺凤,佳人帘下度金针。
\end{yuanwen}

\begin{yuanwen}
登对眺tiào\footnote{望,往远处看。},涉\footnote{到,经历。}对临。瑞雪\footnote{应时的好雪。}对甘霖\footnote{久旱后下的雨;及时雨。}。主欢对民乐,交浅对言深\footnote{战国时范睢说秦王,有“交疏”“言深”等语。交浅犹交疏,是说人与人的关系很一般,没有深交。言深,是讲了至关重要的意见。}。耻三战\footnote{传说春秋时鲁国将军曹刿,曾三次兵败于齐。后来齐桓公和鲁庄公盟于柯,曹刿用匕首逼迫齐桓公,终于索回了失去的国土。},乐七擒\footnote{传说孔明征南蛮,曾对其首领孟获七擒七纵,使孟获受到感化,最后归顺。}。顾曲[13] 对知音。大车行槛槛[14] ,驷sì 马骤骎qīn 骎[15] 。紫电青虹腾剑气[16] ,高山流水识琴心[17] 。屈子怀君,极浦吟风悲泽畔[18] ;王郎忆友,扁舟卧雪访山阴[19]。
\end{yuanwen}

[13] 顾曲:《三国志·周瑜传》载,三国吴周瑜善审音律,曲有阙误,瑜必知之,知之必顾,故时人谣曰:“曲有误,周郎顾。”唐李瑞诗:“欲得周郎顾,时时误拂弦。”

[14] 大车行槛槛:语出《诗经·王风·大车》:“大车槛槛。”大车是上古载重的牛车。槛槛,车声。

[15] 驷马骤骎骎:语出《诗经·小雅·四牡》:“驾彼四骆,载骤骎骎。”驷马,上古一车四马。骤,奔驰。骎骎,马跑得很快的样子。

[16] 紫电青虹:形容宝剑的光华。

[17] 高山流水识琴心:这是关于钟子期、俞伯牙的故事。参见尤韵“钟子”注。据说一次伯牙弹琴,钟子期评论说,此曲“志在高山”;再弹,又评论说,此曲“志在流水”。琴心,琴曲的内容、主题。

[18] 屈子怀君,极浦吟风悲泽畔:见豪韵“遭谗”注。极浦,犹言远浦,远方的水滨。

[19] 王郎忆友,扁舟卧雪访山阴:见豪韵“回艚”注。


\chapter{覃}

\begin{yuanwen}
宫对阙què\footnote{皇帝居处,借指朝廷。},座对龛kān\footnote{供奉佛像、神位等的小阁子。}。水北对天南。蜃shèn 楼\footnote{海洋上由空气折射而成的幻影,古人以为是蜃(一种大蛤蜊)气所化,称蜃楼。}对蚁郡\footnote{唐李公佐《南柯太守传》写汉豪士淳于棼酒醉后梦游大槐安国,被招为附马,守南柯郡。醒后发现,原来槐安国和南柯郡是一群蚂蚁的窝巢。},伟论\footnote{高明超卓的言论。}对高谈\footnote{侃侃而谈,大发议论。}。遴lín 杞梓\footnote{比喻选拔人才。遴,谨慎选择;杞、梓,两种木质优良的树,古人以喻优秀人材。},树楩pián 楠\footnote{比喻培养人才。树,种植。楩,木名,即黄楩树。楩、楠是两种木质优良的树,生在南方。}。得一\footnote{“一”是个哲学概念。《老子》中有“昔之得一者,天得一以清,地得一以宁,神得一以灵,谷得一以盈,万物得一以生,侯王得一以为天下正”的话。}对函三\footnote{《易纬乾凿度》说:“《易》一名而含三义:所谓易也,变易也,不易也。”意思是:《周易》的“易”字含三方面意义:简易、变易和不变。}。八宝珊瑚枕,双珠玳瑁簪\footnote{这是汉乐府《有所思》中的一句。玳瑁,一种海龟,其甲可制作工艺品。}。萧王待士心惟赤\footnote{汉光武帝初起时,曾被更始帝刘玄封为萧王。他在镇压铜马、高湖等起义军时,收降许多人,并将首领封为列侯,以收买人心。所以当时有人说:“萧王推赤心置人腹中,安得不投死乎!”},卢相欺君面独蓝\footnote{唐卢杞长得特别丑陋,史称“鬼貌蓝色”,代宗时为相,迫害忠良,盘剥百姓,干了许多坏事,人曰“蓝面鬼”。}。贾岛诗狂,手拟nǐ 敲门行处想\footnote{唐诗人贾岛,一次在驴背上得“鸟宿池边树,僧敲月下门”两句诗,开始想用“推”字,后改“敲”,仍觉未妥,不觉冲撞京兆尹韩愈。韩愈问明原因,想了一会,认为“敲”字好。这就是“推敲”一语的由来。};张颠草圣,头能濡墨写时酣\footnote{唐张旭,善草书,好酒,每次大醉,则呼叫狂走,或把墨水浇到头上,然后写字,时人称他为“张颠”。}。
\end{yuanwen}

\begin{yuanwen}
闻对见,解\footnote{明白。}对谙ān\footnote{了解,熟悉。}。三橘对双柑\footnote{唐冯贽《云仙杂记》卷二引《高隐外书》:“晋戴颙,春日携双柑斗酒,人问何之,曰:“往听黄鹂声。此俗耳针砭,诗肠鼓吹,汝知之乎?”}。黄童对白叟,静女\footnote{《诗经》篇名。静女指仪态端方的少女。}对奇男。秋七七\footnote{七七是传说中的人名,姓殷。鹤林寺杜鹃花为天下第一。周宝谓殷七七曰:“闻君能顷刻开花,今方重九,花能开乎?”七七曰:“诺。”即于掌中作幻术使花开。夜间一女子曰:“妾为上帝司此花,不久即归阆苑。”此七七即代指杜鹃花。},径三三\footnote{陶渊明咏菊,“冶冶溶溶三径色,风风雨雨九秋时。”此“径三三”即代指菊花。}。海色对山岚\footnote{山中的雾气。}。鸾声何哕huì 哕\footnote{《诗经·小雅·庭燎》有“君子至止,鸾声哕哕”二句。鸾,车铃。哕,乐声。},虎视正眈dān 眈\footnote{《周易·颐卦》中的一句。眈眈,注视的样子。}。仪封疆吏知尼父\footnote{仪是春秋时卫国的地名。尼父即孔子。《论语》记载,孔子到卫国去,仪邑主管边境的“封人”要求见孔子,见过之后对孔子的学生说:“你们不要为流亡而苦恼,上天将让孔子制礼作乐。”},函谷关人识老聃dān\footnote{传说函谷关的令尹善天文,一次登楼四望,于东方见紫色云气,高兴地说:一定有圣人经过此地。后老子骑青牛过关。杜甫诗“东来紫气满函关”即用此典。聃,老子名李聃。}。江相归池,止水自盟真是止\footnote{《宋史·万里传》载,南宋末年,江万里为相,他听说元军已得襄樊,就在自家后园凿个池塘,题名“止水”。后元军至城破,万里遂投池自杀。};吴公作宰,贪泉虽饮亦何贪\footnote{《晋书·吴隐之传》载,晋吴隐之清廉,他到广州为刺史,州城附近有泉名“贪泉”,人们说,谁饮此水都会起贪心。吴隐之故意饮了贪泉水,并作诗一首说:“古人云此水,一歃怀千金。试使夷齐饮,终当不易心。”到郡后更加廉洁自守。歃,用嘴吸取。}。
\end{yuanwen}

\chapter{盐}

\begin{yuanwen}
宽对猛\footnote{《左传》载(郑)大夫子产临终前对他的儿子说:“我死,子必为政。惟有德者能以宽服民,其次莫如猛。”宽,指仁厚。猛,指严厉。},冷对炎。清直\footnote{清廉正直。}对尊严。云头\footnote{云彩上面。}对雨脚\footnote{随云飘行、长垂及地的雨丝。},鹤发\footnote{是说人发白如鹤羽,指老人。}对龙髯rán\footnote{龙的胡须。传说黄帝在鼎湖乘龙而升天,小臣扯龙髯而上,结果扯断了龙须。}。风台谏\footnote{风即讽,讽谏。台,台省。谏,谏臣。古谏官所居官署称讽台。},肃堂廉\footnote{肃堂即官署。廉,阶陛之侧隅。此指廉正。}。保泰对鸣谦\footnote{泰和谦是《周易》的两个卦名。保泰,意为保持安康。鸣谦是谦卦的一句爻辞,意思是以谦虚的品德为人所知。},五湖归范蠡lǐ\footnote{字少伯,佐越王勾践破吴后载西施归五湖,自号陶朱公。},三径\footnote{归隐者的家园。晋陶潜《归去来辞》:“三径就荒,松竹犹存。”}隐陶潜。一剑成功堪佩印\footnote{战国时苏秦曾佩一剑说六国,后为纵约长,佩六国相印。},百钱满卦便垂帘\footnote{汉严君平隐居成都,以卖卜自给,每日得百钱,即闭户垂帘而授《老子》。}。浊酒停杯,容我半酣愁际饮\footnote{语出杜甫诗《登高》:“艰难苦恨繁霜鬓,潦倒新亭浊酒杯。”};好花傍座,看他微笑悟时拈niān\footnote{佛教故事,传说在灵山会上,释迦牟尼拿出一朵花,众人都不解其意,唯独迦叶尊者露出笑颜,表示对佛的旨意有所领悟。后遂以拈花微笑表示心心相印、两心相通。拈,用手指轻轻拿着。}。
\end{yuanwen}

\begin{yuanwen}
连对断,减对添。淡泊\footnote{对于名利淡漠,不看重。}对安恬tián\footnote{淡泊,不追求名利。},回头对极目,水底对山尖。腰袅niǎo 袅\footnote{形容女子腰肢柔软。},手纤纤\footnote{形容手指细而长。}。凤卜对鸾占\footnote{凤卜、鸾占意同,见微韵“采凤飞”注。}。开田多种粟,煮海尽成盐。居同九世张公艺\footnote{唐人张公艺,九世同居。高宗祭泰山,幸其第,问何以能此,公书百“忍”字以进之。},恩给jǐ 千人范仲淹\footnote{宋范仲淹居官后,于姑苏城郊买良田千亩,建立“义庄”,以收养贫困的亲族。}。箫弄凤来,秦女有缘能跨羽\footnote{见东韵“凤翔”注。};鼎成龙去,轩臣\footnote{轩辕皇帝的大臣。}无计得攀髯rán\footnote{传说轩辕皇帝铸鼎成,龙降,骑之上升。其臣攀龙髯欲随之升天,未得。}。
\end{yuanwen}

\begin{yuanwen}
人对己,爱对嫌。举止\footnote{指姿态和风度。}对观瞻\footnote{显露于外的形象。}。四知对三语\footnote{四知见侵韵“杨震”注。三语:据《晋书》载,一次王戎问老子、孔子之道于阮瞻,阮瞻曰:“将无同。”意思是“大约差不多”。王戎听了很满意,就聘其为掾(署员),时人称阮瞻为“三语掾”。},义正对辞严。勤雪案,课风檐\footnote{雪案、风檐,形容读书条件很艰苦。勤和课指学习。}。漏箭对书笺。文繁归獭tǎ 祭\footnote{早春刚刚解冻,水獭把鱼衔出水面,排列在冰上,古人以为这是獭在祭祀,称为獭祭鱼。唐诗人李商隐作诗爱用典故,经常把翻阅的书排在一旁,书册左右麟次,时人也就称他为獭祭鱼。},体艳别香奁lián\footnote{体艳即艳体诗,指爱情或色情诗。唐诗人韩偓喜欢写这类诗,诗集名《香奁集》,时人号为“香奁体”。香奁,妇女梳妆用的匣子。}。昨夜题诗更g\=eng一字\footnote{唐僧齐己作《早梅》诗,曰:“前村深雪里,昨夜数枝开。”许丁卯改为“一枝开”,时人称为“一字师”。},早春来燕卷重帘。诗以史名,愁里悲歌怀杜甫\footnote{见豪韵“诗史”注。史名,杜甫感痛时事,发之为诗,人称为“诗史”。};笔经人索,梦中显晦老江淹\footnote{见支韵“五色笔”注。}。
\end{yuanwen}

\chapter{咸}

\begin{yuanwen}
栽对植,薙tì 对芟shān\footnote{薙:除去野草。芟:割草。薙、芟都是斩除野草的意思。}。二伯\footnote{西周时主掌国事的两个大臣,所谓“自陕以东,周公主之;自陕以西,召公主之”。}对三监\footnote{武王灭殷后,封纣子武庚于商都,派自己的三个弟弟管叔、蔡叔和霍叔监督,称三监。}。朝臣对国老\footnote{指国之重臣。},职事对官衔。鹿麌yǔ 麌\footnote{鹿成群结队的样子。},兔毚chán 毚\footnote{狡猾。}。启牍对开缄\footnote{启牍和开缄都是拆开信件的意思。}。绿杨莺睍xiàn睆huǎn\footnote{即莺啼的声音。},红杏燕呢喃\footnote{燕子叫声。}。半篱白酒娱陶令\footnote{陶令,即陶渊明。因为他曾为彭泽令,故称。},一枕黄粱度吕岩\footnote{见阳韵“客枕”注。原故事中的吕翁和卢生,后人附会成八仙中的钟离权度化吕洞宾(吕岩),所以这里说“度吕岩”。}。九夏\footnote{夏季。}炎飙biāo\footnote{热风。飙,狂风。},长日\footnote{指整天、终日。}风亭\footnote{亭子。}留客骑;三冬寒冽,漫天雪浪驻征帆。
\end{yuanwen}

\begin{yuanwen}
梧对杞,柏对杉。夏濩huò 对韶咸\footnote{见萧韵“殷濩”句注。}。涧瀍chán 对溱qín 洧wěi\footnote{古代四条河流。},巩洛对崤xiáo 函\footnote{巩,古地名,洛水流经其旁。崤,崤山,山名,又叫“崤陵”,其西有函谷关,故称崤函。巩、洛、崤、函均在今河南省。}。藏书洞\footnote{指传说中的二酉山,四川酉阳县翠屏山麓的小酉山石穴中,有书千卷,相传秦人读书于此,称为“二酉藏书洞”。},避诏岩\footnote{指汉初“四皓”所隐的商山,“四皓”(详见齐韵“甪里”注),高帝召之不至,故称其隐居的岩洞为“避诏岩”。}。脱俗对超凡。贤人羞献媚,正士嫉工谗。霸越谋臣推少伯\footnote{少伯,越国大夫范蠡的字。见虞韵“归湖”注。},佐唐藩将重 浑瑊jiān\footnote{浑瑊,唐王朝少数民族的著名将领,曾从李光弼、郭子仪平“安史之乱”,以功为太常卿。德宗出逃奉天,浑瑊率家人子弟从,与朱泚(cǐ)拒战,全城倚重,德宗得以保全。}。邺yè 下狂生,羯jié 鼓三挝zhuā 羞锦袄\footnote{狂生指祢衡。传说曹操欲辱祢衡,命他为鼓吏,击鼓为客人助酒兴。他不仅毫无惧色,反而脱掉衣服,敲起慷慨昂扬的“渔阳三挝”,以回敬曹操。渔阳三挝,传说中古代的鼓曲名。锦袄,代指曹操。挝,这里指敲鼓。}。江州司马,琵琶一曲湿青衫\footnote{唐诗人白居易曾谪为江州司马,一次到浔阳江边送客,遇到一位流落为商人妇的琵琶女,为他弹奏了一曲,引起了他强烈的共鸣,为之流下了泪水。故作长诗《琵琶行》。其中最后两句是:“座中泪下谁最多,江州司马湿青衫。”}。
\end{yuanwen}

仁厚贤能的人对巴结别人感到羞耻,正直善良的人厌恶在背后说人坏话。

\begin{yuanwen}
袍对笏hù\footnote{古代大臣上朝拿着的手板,用玉、象牙或竹片制成,上面可以记事。},履对衫。匹马对孤帆。琢磨对雕镂,刻划对镌juān 镵chán\footnote{都是刻削的意思。}。星北拱\footnote{星指北极星,拱是拱托、环绕的意思。古人认为群星都围绕北极星而分布。},日西衔。卮zhī 漏\footnote{卮,古代一种盛酒器。古语有“川源而不能实漏卮”的话,意为漏洞虽小,如不堵塞则后患无穷。}对鼎馋\footnote{孔子的祖先正考父为宋大夫,其家有鼎名馋鼎。馋,吃。}。江边生桂若\footnote{杜若,香草名。},海外树都d\=u咸\footnote{传说中生于海外的神木。}。但得恢恢存利刃\footnote{《庄子·养生主》中的一则寓言,说宋国有个庖丁,善于解牛,他的刀用了十九年,解过数千头牛,还好像新磨的一样。因为牛的关节之间是有缝隙的,而刀刃却很薄,让薄薄的刀刃通过有缝隙的关节,自然“恢恢乎其于游刃必有余地”。恢恢,宽绰的样子。},何须咄du\=o咄达空函\footnote{晋殷浩得到桓温将推荐他作尚书令的消息,非常高兴,准备回信,又怕言语不周,把信取出放进几十次,结果却寄出了空信封。后桓温将免职,他整日用手在空中乱划,连呼“咄咄怪事”。咄咄,表示惊讶的语气。}。彩凤知音,乐典后夔kuí\footnote{后夔,即夔,传说是舜的乐官,他奏起乐来,百兽起舞,凤凰也飞来。}须九奏\footnote{奏乐九曲。};金人守口,圣如尼父亦三缄\footnote{尼父即孔子。相传孔子入周太庙,见有铸金人,三缄其口,背后有铭文:“古之慎言人也。”三缄,封闭多层。两句的意思是,圣达如孔子,也要学习金人那样守口如瓶,讲话谨慎。}。
\end{yuanwen}

\backmatter

\end{document}