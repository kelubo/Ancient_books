%-*- coding: UTF-8 -*-
% 道德经
% 道德经.tex

\documentclass[a4paper,12pt,UTF8,twoside]{ctexbook}

% 设置纸张信息。
\RequirePackage[a4paper]{geometry}
\geometry{
	%textwidth=138mm,
	%textheight=215mm,
	%left=27mm,
	%right=27mm,
	%top=25.4mm, 
	%bottom=25.4mm,
	%headheight=2.17cm,
	%headsep=4mm,
	%footskip=12mm,
	%heightrounded,
	inner=1in,
	outer=1.25in
}

% 设置字体,并解决显示难检字问题。
\xeCJKsetup{AutoFallBack=true}
\setCJKmainfont{SimSun}[BoldFont=SimHei, ItalicFont=KaiTi, FallBack=SimSun-ExtB]

% 目录 chapter 级别加点(.)。
\usepackage{titletoc}
\titlecontents{chapter}[0pt]{\vspace{3mm}\bf\addvspace{2pt}\filright}{\contentspush{\thecontentslabel\hspace{0.8em}}}{}{\titlerule*[8pt]{.}\contentspage}

% 设置 part 和 chapter 标题格式。
\ctexset{
	part/name={},
	part/number={},
	chapter/name={第,章},
	chapter/number={\chinese{chapter}}
}

% 设置古文原文格式。
\newenvironment{yuanwen}{\bfseries\zihao{4}}

\title{\heiti\zihao{0} 道德经}
\author{老子}
\date{}

\begin{document}

\maketitle

\tableofcontents
	
\frontmatter
\chapter{前言}

《老子》是中国古代著名经典之一,与《庄子》如双峰并峙,是先秦道家学派的代表性著作,对中国传统文化的形成和发展产生了重大的影响。

老子其人

老子其人,古来颇有争议。

《史记·老子韩非列传》说:
老子者,楚苦县厉乡曲仁里人也。姓李氏,名耳,字聃。周守藏室之史也。孔子适周,将问礼于老子。老子曰:“子所言者,其人与骨皆已朽矣,独其言在耳。且君子得其时则驾,不得其时则蓬累而行。吾闻之,良贾深藏,若虚;君子盛德,容貌若愚。去子之骄气与多欲,态色与淫志,是皆无益于子之身,吾所以告子若是而已。”孔子去,谓弟子曰:“乌吾知其能飞,鱼吾知其能游,兽吾知其能走。走者可以为罔,游者可以为纶,飞者可以为赠。至于龙,吾不能知其乘风云面上天。吾今日见老子,其犹龙邪?”老子修道德,其学以自隐无名为务。居周久之,见周之衰,乃遂去。至关,关令尹喜曰:“子将隐矣,强为我著书。”于是老子乃著书上下篇,言道德之意五千余言而去,莫知其所终。或曰:老莱子亦楚人也,著书十五篇,言道家之用,与孔子同时。云盖老子百有六十余岁,或言二百余岁,以其修道而养寿也。自孔子死之后百二十九年,而史记周太史儋见秦献公曰:“始秦与周合,合五百岁而离,离七十岁而霸王者出焉。”或曰儋即老子,或日非也,世莫知其然否。老子,隐君子也。老子之子名宗,宗为魏将,封于段干。宗子注,注子宫,官玄孙假,假仕于汉孝文帝。而假之子解为胶西王印太傅,因家于齐焉。世之学老子者则绌儒学,儒学亦绌老子。“道不同,不相为谋”,岂谓是邪?李耳无为自化,清静自正。

上文中出现了周守藏室之史李耳(聃)、老莱子、周太史儋三人,都称老子,都是隐君子,孰是孰非,汉代司马迁已经糊涂了,搞不清楚,难以评判,只好诸说并存。至于说老子活了一百六十到一百岁,即就是“修道而养寿”也纯属神话,难以置信。正因为如此,关于老子及其著作《老子》的说法很多:
一说认为,老子就是教诲孔子的李耳(老聘),当在孔子之前的春秋后期,从《战国策》《礼记·曾子问》《庄子》《《荀子》《韩非子》《吕氏春秋》等典籍的引述可知。主此说者如马叙伦《老子校诂》、郭沫若《青铜时代·老聃、关尹、环渊》任继愈《老子新译》等。郭沫若指出《老子》书是出于战国环渊之手。
一说认为,老子是战国时代的人,老子》是战国时代的书。主此说者如清代汪中《老子考异》、梁启超《评胡适之中国哲学史大纲》冯友兰《中国哲学史》罗根泽《诸子考索·老子及(老子)书的问题》范文澜中国通史》等。
一说认为,老聃当在杨朱、宋之后,成书在秦汉之间。主此说者如顾颉刚《从(吕氏春秋)推测老子之成书年代》《《秦汉的方士与儒生》等。
一说认为,老聘是中国古代传说中的博大真人,《老子》成书于庄周、宋之后,可能出于詹何之手。主此说者如钱穆《先秦诸子系年·老子杂辨》、庄老通辨》等。
以上诸家,众说纷纭,各有道理,令人深思。如果折中而言,笔者认为,撰写《老子》的作者是战国时期周太史儋,老子》成书于《论语》《墨子》《孟子》之后,庄子》苟子》《韩非子》之前。
在诸侯争霸、社会动荡、矛盾尖锐、民不聊生的春秋战国时期,活跃着许多思想流派,号称“百家争鸣”其中有一些哲人,精通事理、练达人生,博通古今,见识卓越,洞察祸福之机,深明成败之道,对国家、社会、历史、人事有着深刻的分析和理解,他们认识到生命的珍贵和人生的价值,认为只有远离现实社会,不受外物诱惑,才能维护和保全自己的清白品德、高洁人格和纯朴天性。于是,他们或主动或被迫地逃避尘世生活,有的躬耕垄亩,自食其力;有的寄情山水,友风子雨;有的甚至出入庙堂,却大隐于市……总之,对现行制度和当朝权贵采取一种回避的、不合作的、甚至批判的态度,这就是历史上隐逸之士产生的由来。尧时的许由、巢父,周时的伯夷、叔齐就是早期的代表,所以,早在《周易》中就反映出隐逸思想(比如《遁卦》)。《论语》中多次记载楚狂接舆、长沮、桀溺、荷茶丈人等隐者,对热衷入世进取的孔子师徒讽刺、挖苦奚落乃至斥责。钱穆认为,斥责孔子“四体不勤,五谷不分”的隐者荷茶丈人就是老菜子,并非没有道理。(见《先秦诸子系年·老子杂辨》所以,后来庄子说
古之所谓隐士者,非伏其身而弗见也,非闭其言而不出也,非藏其知而不发也,时命大谬也。当时命而大行乎天下,则返一无迹;不当时命而大穷乎天下,则深根宁极而待,此存身之道也。(《庄子·缮性》)隐逸之士既然避世存身,韬讳自保,寄情山水,崇尚自然,因此,探索天道自然的法则和规律,并以之反观比照污浊的现实社会,就能够比较冷静地认识和批判社会弊病,揭示矛盾根源,并以天道自然为法式,提出疗治社会的主张,进而挑战传统,否定君权,从而,在此基础上逐步形成了以贵生存身为核心的道家思想理论体系。

任何一种思想学说,都是由粗到精,由疏到密,有一个生成发展、继往开来的历史过程,道家思想也不例外。既然如此,根据学术思想发展的社会环境和历史脉络,考察老子其人其书,是完全可行的。
(-)杨朱、老子、庄子是先秦道家思想发展史上的重要人物,代表了三个不同的阶段。杨朱最早对隐逸思想进行了初步总结,提出“贵生”,“为己”,“全性保真,不以物累形”,主张“损一毫利天下,不与也;悉天下奉一身,不取也”可以说是道家思想学说的先躯。老子则进一步明确提出“道”的哲学观念,以自然天道反观社会人道,主张贵身爱身,贵柔不争,俭啬谦下,绝圣弃智,小国寡民,无为无不为,可以说是对杨朱学说的继承和发展。后来庄子主张保身、全生、养亲、尽年,坚持恬淡、寂寞、虚静、无为,向往齐同、物化、坐忘、全真的境界,期盼“同与禽兽居,族与万物并”的“至德之世”,即原始自然的纯朴生活,从而把道家学说推向理论的高峰。
如果说杨朱主要是从自身命运立论,公开提出“一毛不拔”“公身公物”的叛逆观点,强调个体意识反对传统观念,锋芒毕露,惊世骇俗;老子则在杨朱学说的基础上,高举天道,正言若反,扑朔迷离,玄妙空灵,借为圣人立言达到无为而治,在有君的前提下做无君的文章,倡导“小国寡民”的理想社会,以辩证思维构建自己的理论体系,显然比杨朱学说严密精致,睿智高明。如果说老子还置身于现实社会,其思想意识还带有世俗的智慧、全身的谋略和人间的理想,具有愤世嫉俗的激情和忧国忧民的胸怀:那么庄子则是在老子理论的引导下,对现实社会更有忧患意识和恐惧心理,对人生命运的认识更为清醒理智,冷峻深刻因此,庄子试图完全游离于现实社会之外,超脱尘世虚无混沌,遗世独立,回归自然,向往“上与造物者游而下与外生死、无终始者为友”实现真正意义上的道遥游,以求得精神世界的彻底解脱,从而,把道家理论提升到一个更高的精神层面,集道家思想学说之大成。
因此,老子学说上承杨朱,下启庄子,是道家学说的一个非常重要的发展阶段,所以,老子必定晚于杨朱,早于庄子。《孟子》以前,并无《老子》之文;《庄子)以后,典籍已经大量引用《老子》那么《老子》就不会晚至秦汉时期,这是显而易见的。
(二)礼记·曾子问》《庄子》《《列子》等典籍,都曾记载孔子向老子问礼,又说到老子对杨朱的教诲也就是说,孔子与杨朱都曾先后师从老子,都是老子的晚辈。假如果真如此,说苑·政理篇》所记杨朱见梁王言治天下如运诸掌,梁之称王是从惠王始(见《史记·六国表》,而惠王元年为公元前369年,已经晚于孔子离世一百余年,其间孔子怎能问礼、杨朱怎能受教呢?难道老子是长生不老的神仙么?显然有悖于常理。
历史如此扭曲,恐怕与《老子》的论述是有关系的。孔子曾说:“大道之行也,天下为公。选贤与能,讲信修睦。”又说:“今大道既隐,天下为家。各亲其亲,各子其子,货力为己;大人世及以为礼,城郭沟池以为固,礼义以为纪。”(《礼记·礼运》)孔子的大道显然与老子的大道是不同的,但是名称却相同,于是老子借此提出:“大道废,有仁义。”(《老子·十八章》)“故失道而后德,失德而后仁,失仁而后义,失义而后礼。夫礼者,忠信之薄,而乱之首。”(《三十八章》)这就把他宣扬的道、德,凌驾于儒家提倡的仁、义、礼之上。既然老子提出的道、德在先,而孔子论述的仁义、礼在后,那么,老子自然成为早于孔子的圣人了从而,在有意无意之间为后世道家门徒们留下想象和发挥的巨大空间。对此,顾颉刚有精辟的分析:
老子为什么会成为孔子的老师?我以为这不是讹传的谣言,乃是有计划的宣传。老子这个学派大约当时有些势力,但起得后了,总敌不过儒家。他们想,如果自己的祖师能和儒家的祖师发生了师弟的关系,至少能耸动外人的视听,争得一点学术的领导权。于是他们造出了一件故事,说孔子当年到周朝时曾向老子请教过,但他的道力不商,而且有些骄矜之气,便给老子痛骂了一顿。他知道自己的根柢差得多,羞愧得说不出话。回得家来,只有对老子仰幕赞叹。借了孔子的嘴来判定了老、孔的高下,显见他们的门徒之间也是这等比例,道家的身价就
所言
朝其摇毕”银片
可提高。想不到他们这种宣传不但如了愿,竟至超过了预期,而使儒家承认为事实;又不但如此,而使儒家也增加了一段故事,说孔子曾向老子问过许多礼制,把老子也儒家化了。可怜的是《老子》里既有“礼者,忠信之薄而乱之首”的话,《礼记》中又有老聃答孔子问庙主、问葬礼的话,逼得他竟成了二重人格,自己打自己的嘴巴!他们这个工作成功了,索性再进一步,使出手段来拉拢黄帝。他们把本学派里的货色尽量向黄帝身上装,结果,装得黄帝也像了老子,而后道家里以老子为“太祖高皇帝”,黄帝为“肇祖原皇帝”,其学派的开创时代乃直顶到有史之始了。至于发踪指示的杨朱,早被一脚踢开,学术系统从此弄乱。《汉书·艺文志》所列道家著作,有《黄帝四经》《黄帝铭》等篇,注云“起六国时,与《老子》相似也”。这就是黄帝与老子合作的成绩,而“黄、老”一名也从此打不破了!(《秦汉的方士与儒生》)
道家门徒既然让老子攀扯上了早在史初的人文之祖黄帝,从此黄、老并称,高高供起,冥冥之中不仅“发踪指示”的杨朱、而且连儒家祖师的孔子,都只能登门求教、顶礼膜拜了!如此荒唐错乱的记载,怎能相信呢?(三)老子》三章曰:“不尚贤,使民不争。”十九章曰:“绝圣弃智,民利百倍;绝仁弃义,民复孝慈;绝巧

弃利,盗贼无有。”“尚贤”是墨家主张,“圣、智、仁、义”是儒家思想,既然《老子》一概反对,其书必在孔子、墨子之后。
(四)论语》《墨子》《孟子》从未提及老子,而《孟子》对杨朱却口诛笔伐,激烈地抨击和批判,认为“杨朱、墨翟之言盈天下。天下之言,不归杨,则归墨。杨氏为我,是无君也。墨氏兼爱,是无父也。无君无父,是禽兽也。……杨朱之道不息,孔子之道不著”(《孟子·滕文公下》。显然,孟子认为杨朱学说主张无君,大逆不道,视为洪水猛兽,必欲铲除而后快,但是,对于继承杨朱学说,进而提出“绝圣弃智”“绝仁弃义”“绝巧弃利”,大骂统治者是“盗夸”的老子却不着一字,未曾提及,这绝非是孟子的疏忽大意。只能说明,孟子(约前 372年~前289年)以前,老子或老子的著作尚未出现,或者虽同时而稍晚,未曾引起人们注意。
(五バ老子》的用语、句式和思想,说明《老子》是一部战国时代的个人著作。《老子》中常用“王侯”“侯王”、“王公”“万乘之君”“取天下”等战国语词就是《老子》成书的时代标志。《老子》中的惯用句式!如“夫唯……,是以……”“以其……,故能……”之类;老子》前后思想的连贯一致,自称“吾”、“我”等都能够证明《老子》是由一人撰写而成。今本有的内容稍有重复,也许是出于后人增补编篡,但是并不影响全书的思路和表达。
前
织武攀多。我
(六)从老子的籍贯看来,应当是战国时人,不是春秋时人。苦县本属陈国,后楚国灭陈国,苦县即属楚。《史记·陈杞世家》曰:“二十四年(前 478 年)楚惠王复国,以兵北伐,杀陈湣公,遂灭陈而有之。”既然《史记》说老子为楚苦县人,无疑当在楚灭陈之后,不在春秋,而在战国,老子》成书更在其后。
(七)老子可能就是周太史儋,老子》就是太史儋所著。“聃”即“儋”“聃”与“儋”音同字通,名称很相近;聃为周守藏室之史,儋为周太史,身份颇相似;老聘至关也罢,太史儋见秦献公也罢,均须西出关,方位也相同,是后来司马迁因传说不明而造成了错乱,其实二者似为一人。秦献公于前384年~前362年在位,此时太史儋见秦献公,与前面的推断在年代上也大致吻合。(参见罗根泽《再论老子及《老子》书的问题》
《老子》其文
《老子》一书共八十一章,多为韵文,分道经和德经两部分,所以又称《道德经》。传统的顺序是道经在前,德经在后,而1973 年发掘长沙马王堆三号汉墓中出土的《老子》帛书,是德经在前,道经在后,可能是古本的顺序。历史上为《老子》作注者甚多,最早的注解是《韩非子·解老》,后来重要的有晋王弼的《老子注》和假托西汉河上公的《老子章句》等。收入《诸子集成》的是王弼《老子注》和清魏源《老子本义》,今人高

亨《老子注泽》陈鼓应《老子注译及评介》等,都可供参考。
《老子》五千言,篇幅不长,论述精辟,意义丰富,思想深邃。其内容重在详尽论述作为宇宙本体、万物之源和运动规律的天道,并将这种天道用以关照人道,指导治国(包括砭时、议兵)和修身(包括养生),直面现实社会,涉及到宇宙、自然、社会、人生的各个方面。尽管《老子》的行文隐讳曲折,正言若反,扑朔迷离,飘忽不定,但是其思想学说始终如一,贯彻到底,用朴素的辩证思维构建起独特的理论体系。
(-)论道:
《老子》的一、四、六、十一、十四、二十一、二十五、三十二、三十四、三十五、四十、四十一、四十二、五十-章,共十四章,重在论道。其内容主要是:
1.道,浑沌而成,先天地生,是效法自然而形成的宇宙本体和规律法则。道一而生天地阴阳二气,阴阳交合而生成和谐之气,阴、阳、和三气生成万物,因此,道是“天地之母”,“万物之宗”,是一切事物产生的动力和最后的归宿。
2.道,无状无象,是人的视、听、触、味等感官知觉无法直接触及感知的,但又用之不尽,确实存在。道虽然恍惚迷离,质朴幽深,但是,通过万物来体现,依靠万物而存在,因此,可以通过万物变化感觉道,通过观察体验认识道,逐步了解其“象”、“物”“精”“信”进而把握道的特征和规律。
胎言
兴积武灌午·和中
3.道,浑然一体,独立存在,虽然无始无终,无影无踪,但是,超越时空,无处不在,周而复始,对立转化,影响决定着自然、社会和人生的命运。道,以循环的方式运动,以柔弱的姿态运用。天下万物都生于万物之母的“有”,而“有”则生于天地之始的“无(即道)”。因此,“有”,给人便利:“无”发挥作用。从而,引导、支配着自然和社会的变化运动。
4.道,空虚不盈,清静无为,永远存在,顺应自然,实际上又无所不为,永不穷尽,养育万物,除旧更新。万物依靠道而生,道养育万物而成。虽然如此,道无私无欲,不推辞,不占有,不主字,不自大;既可称为小,又可称为大。因为不自大,反而成就了道的伟大。
5.“道常无名”“道隐无名”,质朴纯厚,玄妙幽深,是人们根据自己的认识,勉强称之为“道”勉强称之为“大”。而“天道无亲”,“天地不仁”,“善贷且成”“常与善人”,公正平等,“莫之命而常自然”,无为无不为。
由此可知,老子的道,是一个非常重要、非常复杂的哲学观念,指的是浑然一体的宇宙本体,永恒存在的天地万物之源,运动不息而对立转化的规律和法则。老子正是以这种虚无的天道取代了商周以来的天命观,从而论证和构建了自己的宇宙观。
正如庄子后来论道说:
“夫道,有情,有信,无为,无形,可传而不可受,可得而不可见。自本,自根,未有天地,自古以固存。神

鬼,神帝,生天生地。在太极之上,而不为高;在六极之下,而不为深;先天地生,而不为久;长于上古,而不为老。”《《庄子·大宗师》)
应该看到,老子的道又是一个矛盾的混合体。道既然是宇宙本体、万物之源,又是运动规律和法则,就包含着不可克服的矛盾。因为,天地万物的物质运动是客观的永恒的,由此才能体现和总结出客观的运动规律和法则;既然运动规律和法则是建立在天地万物物质运动的基础之上的,那么,运动规律和法则就不可能先于天地万物的物质运动而存在,更不可能由此产生天地万物,成为宇宙本体。而老子将宇宙本体、万物之源与运动规律和法则并列乃至混同,就意味着否定了客观世界的物质本原,那么运动规律和法则就脱离天地万物的物质运动而独立存在,成为一种不知所出的神秘力量。所以,老子的道论虽然具有反对天命论的进步意义,也带有神秘唯心主义的局限性。
(二)治国:
《老子》的二、三、五、十七、十九、二十三、二十七、二十九、三十六、三十七、三十九、四十三、四十八、四十九、五十六、五十七、五十八、六十、六十一、六十四、六十五、六十六、七十二、七十三、七十九、八十章,共二十六章,重在治国。另外,十八、三十八、五十三、七十四、七十五、七十七章,共六章,重在砭时。三十、三十一、六十八、六十九章,共四章,重在议兵。其主要内容是:
历
粥武餐升·积D
1.统治者必须效法天道,无私无欲,公正公平,善待百姓,善待万物,不要弃人弃物,人为造成亲疏、利害、贵贱的差别,这样,才能真正得到善良和诚信。同时,还要认识到贵贱、高下的辩证关系,守道不争,谦下卑弱,去甚、去奢、去泰,称孤、道寡,言下、身后,才能无为而无不为,得到百姓拥戴,始终处于不败之地。
2.治国的关键在于清静无为,少私寡欲,慎行贵言,顺应自然,不要肆意妄为,扰民害民,让百姓自化、自正、自富、自朴,甚至让百姓感觉不到统治者的存在,才是最好的侯王,因为“多言数穷”,为者败之。所以,必须禁绝抛弃世俗的圣智、仁义、巧利的诱惑,排除人为的忌讳、利器、技巧、法令的干扰,不要压迫欺诈百姓,更不能作威作福,为所欲为,否则,就会招致更大的反抗,引起天下大乱。
3.统治者要让百姓无知无欲,“见素抱朴”,削弱心志,强健筋骨,抛弃现实社会的一切文明成果,恢复到质朴纯厚的原始状态,实现小国寡民,天下才能大治。要认识到福祸倚伏、对立转化的道理,物壮则老物极必反,周而复始,其事好还,因此,张歙、强弱、兴废、与取,都在不断变化,要从反面入手,得出正面的结果;难易、大小、有无、治乱,都是相对的,要由小到大,由易到难,为之未有,治之未乱,才可以成功。
4.老子憎恨统治者“损不足以奉有余”“以死惧之”的罪恶行为,指出百姓饥荒的根本原因在于统治者“食税之多”,认为仁、义、礼、智之类都是在道、德沦

丧之后的产物,既不可信,又不可用。国家昏乱,田园荒芜,而统治者自己却美服餍食,穷奢极欲,真是一伙背离大道、寡廉鲜耻的强盗头子!
5.老子认为战争对于双方都会带来极大的灾难其事好还,物壮则老,“师之所处,荆棘生焉。大军之后,必有凶年”,因此,要以不争之德对待战争,即就是不得已而战,取得胜利也不能骄傲自得,炫耀逞强:所以,他反对发动战争,反对主动进攻,更反对狂妄轻敌,主张防御应战,认为哀兵必胜。
显然,老子是以天道反观和指导人道,要求统治者守道不争,谦下卑弱,无为贵言,无私无欲,公平待民,不分贵贱,绝弃仁义巧智、舟舆甲兵之类当代文明,恢复质朴纯厚的原始状态,实现小国寡民,无为而治,所以,反对压榨剥削,反对发动战争,由此建立自己的社会观。
任何文明进步事物的出现,都可能推动社会的前进,但是在发挥其正面积极作用的同时,又存在负面的消极影响,古代如此,今天也难免。问题在于怎样理性的认识处理,发挥其积极作用,克服其消极影响而绝不能因噎废食,毁弃文明,回到原始状态。虽然今人看来,老子出于对当时社会弊病的极度愤怒和极端失望,提出了抛弃文明、崇尚复古的主张,显得偏颇过激,脱离现实,甚至逆历史而动,但是,就其对社会现实认识分析的尖锐深刻程度而言,老子却是言前人之未能言和未敢言,确实惊世骇俗,振聋发聩,具有震
南
积基增界。51平
撼人心的力量,其时代的进步意义不言而喻。
(三)修身:
《老子》中七、八、十、十三、十五、十六、二十、二十二、二十四、二十六、二十八、三十三、四十五、四十七、五十二、五十四、五十五、五十九、六十二、六十三、六十七、七十、七十一、七十六、七十八、八十一章,共二十六章,重在修身。另外,九、十二、四十四、四十六、五十章,共五章,重在养生。其主要内容是:
1.行道者修身要把守护灵魂和坚守大道紧密结合在一起,达到专气致柔的婴儿状态,关闭感官,纯洁心灵,知其雄,守其雌,永远保持质朴纯厚的品德,真正进入空虚无欲、清静无为的境界。因此,行道者异于常人,小心谨慎,犹豫踌躇,严肃庄重,温和可亲,虚怀若谷,浑朴纯正,寂寞恬淡,飘逸不定。因为理解自己的人很少,所以,只能“被褐而怀玉”。
2.为道者既要有知人之智,胜人之力,更要有自知之明,自胜之强,明白四达,知而不知。社会上宠之得失,辱之得失,都是因名利之类的身外之物而造成的后果,都会带来祸患,而行道之人无私无欲,清静无为,知足不辱,知止不殆,所以无论是得宠或是受辱都感到惊恐不安。只有贵身爱身,抛弃私欲,才能远离祸患,承担大任。
3.行道者以“慈”“俭”“不敢为天下先”为三宝,就是把慈爱公平、俭啬收敛、谦下不争作为人生的法则,即“治人事天,莫如啬”。因此,从来“不积”“不自

生”后其身以求身先,外其身以求身存,竭尽全力帮助他人,以求自我满足,所以,像天之道“利而不害”样,人之道是“为而不争”。
4.事物总是互相依存,相反相成,“曲则全”是普遍法则,因此,圣人不自见,不自是,不自伐,不自矜从没有“余食赘行”之类自我炫耀的多余行为,一切顺应自然。因为物壮则老,物极必反,所以,有道之人要像水一样,处于下位,柔弱自守,清静无为;慈爱真诚,滋养万物;以柔胜刚,以弱胜强;受垢才能为社稷主,受难才能为天下王。正面的语言如同反话一样。
5.物欲的满盈,声色的诱惑,奢华的奉养,必定给自身造成灾难和短命,因此,修身养生者必须“见素抱朴”,清心寡欲,俭啬收敛,功成身退,才能长保平安。如果过分看重名利财货,贪得无厌,不知满足,必然带来巨大的危害。所以,只有知足,才能不受辱;只有知止,才能不危险,这是修身养生者的妙道要诀。
老子就是这样将天道用来指导修身养生,要求人们坚持以慈爱、俭音、不争三宝为准则,空虚无欲,清静无为,质朴淳厚,知雄守雌,小心谨慎,虚怀若谷,恬淡安宁,被褐怀玉,谦下收敛,贵柔戒刚,知足不辱,知止不殆,委曲求全,功成身退,以确立自己的人生观。
显然,老子》的论道、治国、修身思想,是对杨朱“贵生”“为我”“全性保真,不以物累形”等学说的继承和发展。如果说杨朱侧重于摆脱寿、名、位、货的约束,强调个体生命的价值,是自我意识的可贵觉醒;主
辟店
渐武推二·现中八
张从心而动,从性而游,还留恋着现实社会的当世之乐,进而提出“损一毫利天下,不与也;悉天下奉一身,不取也”那样极端、直率、偏执、露骨的政治主张。那么,老子则更具有韬晦自保的忧患意识,“知其雄,守其雌”,“曲则全”,向往“小国寡民”,以求避世全身,所以其思想更为深沉、彻底、激愤、坚定,其学说更为含蓄、隐讳、迷离、精致,充满了辩证思维,是对杨朱学说的理论升华和提高。正因为老子具有这样的深邃思想和辩证认识,所以他对黑暗社会的认识更为深刻,对严酷现实的批判更为尖锐,反映了一代哲人的社会良知和理论勇气,不仅在那个时代大放异彩,而且造成了深远的历史影响。
在理论分析上,老子论述问题,正言若反,委婉曲折,很少直言不讳,直奔主题。他充分认识事物产生和发展的两极,及其内在的变化规律,以迂回的思维方式和表述方法,从反面(传统的正面)立论,而达到自己的正面(传统的反面)目的;为圣人立论,而达到为己的目的。比如《老子》中“夫唯弗居,是以不去”“为无为,则无不治”:“以其无私,故能成其私”“夫唯不争,故无尤”:“以其终不自为大,故能成其大”;“是以圣人不为大,故能成其大”;“以其不争,故天下莫能与之争”等表述,表面上都顺应了传统的主流道德价值观(即弗居、无为、无私、不自为大、不争),实质上则在更高的层面上达到了贵生、为己、避世、全身的目的(即不去、无不治、成其私、无尤、成其大、莫能与之

争),雄辩有力,令人信服。这样,表面上符合传统的价值观念,不致遭到统治者的攻击封杀,实质上则建立起自己反传统的宇宙观、社会观和人生观。这是老子对杨朱学说社会命运的经验总结,充分表现了老子的辩证思维和政治智慧,是比杨朱更为高明、更为睿智的贵生为己主义。
特别值得注意的是,老子学说本质上与杨朱一样,具有明显的否定君权的反传统倾向,触及到统治者的根本利益,显然与当时社会的主流观念相矛盾相冲突,因此孟子才咒骂杨朱是“禽兽”,大肆攻击,极力封杀;但是,老子却巧妙地应对了这个敏感的社会政治问题。他以玄虚抽象的辩证思维和正言若反的表述方式,来装饰其思想观念,钝化其价值取向,表面上为侯王统治者着想考虑,符合传统观念,实际上在巧妙隐晦的表述中使其反传统的思想合“理”合“法”化,从而模糊了社会的视线,得到各方的认可。他反复论述“有”与“无”的关系,并归之于无所不在、无比玄妙的道,提出“贵以贱为本,高以下为基”:“太上,不知有之”“是以圣人欲上民,必以言下之;欲先民,必以身后之。是以圣人处上而民不重,处前而民不害。是以天下乐推而不厌”。可见,老子并没有否定侯王,而是在肯定侯王,只是认为最好的候王,百姓“不知有之’而已;老子也没有否定有为,只是认为“无为”,才能“无不为”,那么,作为侯王又何必繁令苛政、劳民伤财、强行“有为”呢?显然,老子是在有君的旗号下大
前
新善激华中和1
做无君的文章,在有为的命题中大作无为的论证。这样,他一方面向侯王反复赞美和申明“清静无为”的行为准则,使现实的人道能够归顺天道,从而防止统治者因胡作非为而带来祸害灾难,否则,就会“轻则失根,躁则失君”;另一方面自己又因此而避免了“无君”的罪名,取得了“无君”的实效。既然如此,还有谁会对老子学说怀疑谴责呢?还有什么理由阻止老子思想的传播流行呢?
三《老子》解读
《老子》自问世以来,注者蜂起,众说纷纭。由于《老子》表述含蓄隐讳,正言若反,以辩证思维揭示了普遍存在于自然、社会、人事的矛盾对立转化规律,因此,触动了不同的学科领域,产生了广泛的社会联想。所以,老子》一书,有说是权谋之书,有说是兵法之书,有说是气功之书;而后来的道教甚至将《老子》列为《道藏》诸经之首,又成为宗教之书,并且各有注本、专书广为传播。
其实,老子生活在战国,与百家诸子一样,关注的都是天下、国家、社会、民生的诸多现实问题。不同的是,他提出了道的哲学观念,借助天道,统辖人道,在杨朱理论的基础上,进一步论述阐发慈爱贵柔,俭啬收敛,谦下不争,反对圣智仁义,主张无为而治,以达到贵生为我、韬晦自保、否定传统、顺应自然的目的,建立了自己独特的道家理论体系。因此,老子》虽然

论述规律,并非权谋之书;老子》明确反对战争,并非兵法之书:老子》讲解修身之道,并非气功之书;《老子》完全否定天命,更不是宗教之书。但是,从阐述矛盾对立转化的客观规律来说,老子》又与上述诸多领域所论述的问题密切联系。我们认为,从表述的内容、构建的理论来看,老子》在本质上是先秦道家的-部代表性著作。至于《老子》的文句和思想被其他学科领域引用发挥,那是另外的问题了。
既然如此,我们必须把《老子》放在战国时期特定的历史环境中去认识考察,切实从《老子》文本出发研究问题,理解意义,既要实事求是地肯定其思想成果:又要认真分析其时代局限,进而汲取有益的思想营养,弘扬优秀的传统文化。
对于《老子》的研究,历代学者特别是当代学者已经取得了巨大的成绩,今天我们考察分析《老子》的任何问题,都离不开前人奠定的学术基础,都必须认真学习先辈的研究成果。因此,考虑到本书的定位和性质,提出以下两点:
(-)尊重原文,关照全书
《老子》一书含蓄隐晦,正言若反,思想深邃,蕴涵丰富,必须认真研读原文,仔细分析,方能理解意义。因此,本书采用通行版本,在每一章正文前标注内容主旨,正文后分别列出“注释”“译文”两个栏目,简明扼要地诠释字词,准确翻译,解读思想,不作烦琐考证。同时,将本章与全书各章有关的论述,互相联系
印证,避免主观臆断,随意发挥,以求客观地把握和反映老子思想学说的系统性。
(二)博采众说,间出已意:
充分吸取前人研究成果,广泛联系历史背景,深入考察诸子学说,准确理解文句含义,真正贯通和完整理解《老子》的思想内容。在此基础上,如果发现疑难问题,不必迷信先贤,盲从成说,只要认真分析,有理有据,即可另行解说,以求通达。读者翻阅时,自行对比分析,即能理解会意。如有不妥之处,欢迎批评指正。
	
	老子,姓李名耳,字聃,一字或曰谥伯阳。华夏族, 楚国苦县厉乡曲仁里\footnote{今河南省鹿邑县太清宫镇。}人,约生活于前571年至471年之间。是我国古代伟大的哲学家和思想家、道家学派创始人,被唐朝帝王追认为李姓始祖。老子故里鹿邑县亦因老子先后由苦县更名为真源县、卫真县、鹿邑县,并在鹿邑县境内留下许多与老子息息相关的珍贵文物。老子乃世界文化名人,世界百位历史名人之一,存世有《道德经》,其作品的精华是朴素的辩证法,主张无为而治,其学说对中国哲学发展具有深刻影响。在道教中,老子被尊为道教始祖。老子与后世的庄子并称老庄。
	
	《老子》,又称《道德真经》、《道德经》、《五千言》、《老子五千文》,是中国古代先秦诸子分家前的一部著作,为其时诸子所共仰,传说是春秋时期的老子李耳\footnote{似是作者、注释者、传抄者的集合体。}所撰写,是道家哲学思想的重要来源。道德经分上下两篇,原文上篇《德经》、下篇《道经》,不分章,后改为《道经》37章在前,第38章之后为《德经》,并分为81章。是中国历史上首部完整的哲学著作。 
	
	1973年,长沙马王堆汉墓出土了一批古书,其中就包括《道德经》,分为甲乙两个版本。经整理复原之后发现,该版本与当下流行的传世版本,存在这一些差异。这些差异大多只在只字片语之间,但意义却有千差万别之远。
	
\mainmatter

\part{道经}

\chapter{论道}

本书,重在论道。本章是道的总论,也是全书的总纲。

道,是老子提出的一个重要哲学观念,是贯穿于全书的一条思想纽带。老子认为,道体非常玄妙幽深、蕴涵非常宽泛丰富,人们对道并非生而知之,而是后天逐步进行探索、认识,才能有所了解、感悟,因此是可以阐述和说解的。但是,人们的探索是渐进的,认识是主观的,阐述是非系统的,说解是有局限的,与作为客观本体的道的玄妙幽深和丰富内涵还有相当距离,并不等于道所具有的全部内涵、外延、情态和性状,要想全面彻底地掌握道的真知,还需要一个长期不断的探索过程,所以说,“道可道,非常道”。

同样,既然道本无名,道是由人们勉强命名的,那么,所命之名只是仅就道的某一特征为理据,或大或逝,或远或反,都不足以完全概括道的内涵、外延情态和性状,所以说“名可名,非常名”。

\begin{yuanwen}
道\footnote{名词,指的是宇宙的本原和实质,引申为原理、原则、真理、规律等。},可道\footnote{动词。指解说、表述的意思,犹言“说得出”。}也,非恒\footnote{一般的,普通的。}道也。名\footnote{名词,指“道”的形态。},可名\footnote{动词,说明的意思。}也,非恒名也。 “无”,名天地之始;“有”,名万物之母\footnote{母体,根源。}。 故,常“无”,欲以观其妙;常“有”,欲以观其徼\footnote{ji\`ao}。 此两者,同出而异名,同谓之玄。玄之又玄,眾妙之门。
\end{yuanwen}


\begin{yuanwen}
【通行本】

道\footnote{名词,指浑然一体的宇宙本体、永恒存在的天地万物之源、运动不息而又对立转化的规律和法则。因此,又称为“一”。《三十九章》曰:“昔之得一者---天得一以清,地得一以宁,神得一以灵,谷得一以盈,万物得一以生,侯王得一以为天下正。”《四十二章》曰:“道生一,一生二,二生三,三生万物。”}可道\footnote{动词,阐述,解说。},非常道\footnote{指浑然一体、永恒存在、运动不息的大道。}。名\footnote{名词,道之名。}可名\footnote{动词,命名,称谓。},非常名\footnote{常名,指浑然一体、永恒存在、运动不息的道之名。《二十五章》曰:“有物混成,先天地生。寂兮兮,独立而不改,周行而不殆,可以为天地母。吾不知其名,强字之曰‘道’,强为之名曰‘大’。大曰‘逝’,逝曰‘远’,远日‘反’。”}。

无\footnote{指道。《三十二章》曰:“道常无名,朴。”},名天地之始\footnote{天地的本初。}。有\footnote{指由道而产生的万物。《三十二章》曰:“始制有名。”},名万物之母\footnote{万物的本原,即无名之道是天地的本初,天地混沌初开,然后有万物的产生,才能制名,而道正是天下初始和万物产生的源头和动力,即母体。《四十章》曰:“天下万物生于‘有’,‘有’生于‘无’。”}。

故常无,欲\footnote{将。}以观其妙\footnote{微妙。};常有,欲以观其徼\footnote{jiào,边际。}。

此两者,同出而异名,同谓之玄\footnote{玄妙幽深。}。玄之又玄,众妙之门\footnote{天地万物变化的总源头。}。
\end{yuanwen}
	
道是可以阐述解说的,但是并非完全等同于浑然一体、永恒存在、而又运动不息的那个大道。道名也是可以命名的,但是并非完全等同于浑然一体、永恒存在、运动不息的道之名。

无,称天地的初始;有,称万物的本原。

因此,从常无中,将以观察道的微妙;从常有中,将以观察道的边际。

这无、有二者,同出于道而名称不同,都可谓玄妙幽深。玄妙而又玄妙,正是天地万物变化的总源头。
	
\chapter{美善}

本章讲述了相反相成、互相转化的道理,重在治国。

美-恶、善-不善、有-无、难-易、长-短、高-下、音-声、前-后等,都是相反相成的概念,离开前者则后者不存在,离开后者则前者不成立,在互相对立中互相依赖,互相补充;同时二者的关系又不是绝对的,比较而言,可以转化,这是来于自然的重要启示,是道的永恒规律。圣人正是掌握了这个规律,因此,“处无为之事,行不言之教”一切顺应自然的发展,而不加入自己的意志和私欲。只有“不为始”、“弗有”、“弗恃”、“弗居”,才能得到“不去”的结果。这种“功成而弗居”的不争思想,有利于治国。
	
\begin{yuanwen}
【通行本】

天下皆知美之为美,斯\footnote{则,就。}恶(è)\footnote{丑陋,与美相反。}已\footnote{表肯定的语气词,相当于“了”。};皆知善之为善,斯不善已。

有无相生\footnote{互相依存。生,存。},难易相成,长短相形,高下相倾,音声相和(hè),前后相随,恒也。

是以圣人处无为之事,行不言之教;万物作而弗始,生而弗有,为而弗恃,功成而弗居。夫(fú)唯弗居,是以不去。
\end{yuanwen}
	
\chapter{无为}
	
\begin{yuanwen}
不尚贤,使民不争;不贵难得之货,使民不为盗;不见(xiàn)可欲,使民心不乱。是以圣人之治,虚其心,实其腹;弱其志,强其骨。常使民无知无欲,使夫(fú)智者不敢为也。为无为,则无不治。
\end{yuanwen}
	
\chapter{道沖}
\begin{yuanwen}
道冲而用之或不盈,渊兮似万物之宗。锉其锐,解其纷,和其光,同其尘。湛兮似或存,吾不知谁之子,象帝之先。
\end{yuanwen}
	
	
\chapter{守中}
	
\begin{yuanwen}
天地不仁,以万物为刍(chú)狗;圣人不仁,以百姓为刍狗。天地之间,其犹橐龠(tuó	yuè)乎?虚而不屈,动而俞出。多言数(shuò)穷,不如守中。
\end{yuanwen}	
	
	
	
	
\chapter{谷神}
\begin{yuanwen}
谷神不死,是谓玄牝(pìn),玄牝之门,是谓天地之根。绵绵若存,用之不勤。
\end{yuanwen}
	
	
	
	
	
\chapter{无私}
\begin{yuanwen}
天长地久。天地所以能长且久者,以其不自生,故能长生。是以圣人后其身而身先,外其身而身存。以其无私,故能成其私。
\end{yuanwen}	
	
	
	
	
\chapter{上善}

\begin{yuanwen}
上善若水。水善利万物而不争,处众人之所恶(wù),故几(jī)于道。居善地,心善渊,与善仁,言善信,政善治,事善能,动善时。夫唯不争,故无尤。
\end{yuanwen}
	
	
\chapter{持盈}
	
\begin{yuanwen}
持而盈之,不如其已。揣(chuǎi)而锐之,不可长保。金玉满堂,莫之能守。富贵而骄,自遗(yí)其咎。功遂身退,天之道也。
\end{yuanwen}	

\chapter{玄德}
	
\begin{yuanwen}
载(zài)营魄抱一,能无离乎?专气致柔,能如婴儿乎?涤除玄鉴,能无疵乎?爱民治国,能无为乎?天门开阖(hé),能为雌乎?明白四达,能无知乎?生之、畜(xù)之,生而不有,为而不恃,长(zhǎng)而不宰,是谓玄德。
\end{yuanwen}	
	
\chapter{利用}

\begin{yuanwen}
三十辐共一毂(gǔ),当其无,有车之用。埏埴(shān zhí)以为器,当其无,有器之用。凿户牖(yǒu)以为室,当其无,有室之用。故有之以为利,无之以为用。
\end{yuanwen}
	
\chapter{为腹}

\begin{yuanwen}
五色令人目盲,五音令人耳聋,五味令人口爽,驰骋畋(tián)猎令人心发狂,难得之货令人行妨。是以圣人为腹不为目,故去彼取此。
\end{yuanwen}
	
\chapter{宠贵}

\begin{yuanwen}
宠辱若惊,贵大患若身。何谓宠辱若惊?宠为上,辱为下,得之若惊,失之若惊,是谓宠辱若惊。何谓贵大患若身?吾所以有大患者,为吾有身,及吾无身,吾有何患!故贵以身为天下,若可寄天下;爱以身为天下,若可托天下。
\end{yuanwen}
	
\chapter{道纪}

\begin{yuanwen}
视之不见名曰夷,听之不闻名曰希,搏之不得名曰微。此三者不可致诘(jié),故混(hùn)而为一。其上不皦(jiǎo),其下不昧。绳绳(mǐn mǐn )兮不可名,复归于无物,是谓无状之状,无物之象。是谓惚恍。迎之不见其首,随之不见其后。执古之道,以御今之有,能知古始,是谓道纪。
\end{yuanwen}
	
\chapter{保盈}

\begin{yuanwen}
古之善为道者,微妙玄通,深不可识。夫唯不可识,故强(qiǎng)为之容。豫兮若冬涉川,犹兮若畏四邻,俨兮其若容,涣兮其若凌释,敦兮其若朴,旷兮其若谷,混兮其若浊。孰能浊以静之徐清?孰能安以久动之徐生?保此道者不欲盈,夫唯不盈,故能蔽而新成。
\end{yuanwen}
	
	
	
\chapter{虚静}

\begin{yuanwen}
致虚极,守静笃(dǔ),万物并作,吾以观复。夫物芸芸,各归其根。归根曰静,静曰复命。复命曰常,知常曰明,不知常,妄作,凶。知常容,容乃公,公乃全,全乃天,天乃道,道乃久,没(mò)身不殆。
\end{yuanwen}
	
	
\chapter{太上}

\begin{yuanwen}
太上,不知有之。其次,亲而誉之。其次,畏之。其次,侮之。信不足焉,有不信焉。悠兮其贵言。功成事遂,百姓皆谓:“我自然。”
\end{yuanwen}
	
\chapter{大道}

\begin{yuanwen}
大道废,有仁义;慧智出,有大伪;六亲不和,有孝慈;国家昏乱,有忠臣。
\end{yuanwen}

	
	
	
\chapter{三绝}

\begin{yuanwen}
绝圣弃智,民利百倍;绝仁弃义,民复孝慈;绝巧弃利,盗贼无有。此三者,以为文不足,故令有所属,见(xiàn)素抱朴,少私寡欲,绝学无忧。
\end{yuanwen}
	
	
	
\chapter{绝学}

\begin{yuanwen}
唯之与阿(ē),相去几何?美之与恶,相去若何?人之所畏,不可不畏。荒兮,其未央哉!众人熙熙,如享太牢,如春登台。我独泊兮,其未兆。沌沌兮,如婴儿之未孩。傫傫(lěi)兮,若无所归。众人皆有余,而我独若遗。我愚人之心也哉!俗人昭昭,我独昏昏;俗人察察,我独闷闷。澹(dàn)兮其若海,飂(liù)兮若无止。众人皆有以,而我独顽且鄙。我独异于人,而贵食(sì)母。
\end{yuanwen}
	
	
	
\chapter{孔德}

\begin{yuanwen}
孔德之容,惟道是从。道之为物,惟恍惟惚。惚兮恍兮,其中有象;恍兮惚兮,其中有物。窈(yǎo)兮冥兮,其中有精;其精甚真,其中有信。自今及古,其名不去,以阅众甫。吾何以知众甫之状哉?以此。
\end{yuanwen}
	
	
\chapter{全归}

\begin{yuanwen}
曲则全,枉则直,洼则盈,敝则新,少则得,多则惑。是以圣人抱一为天下式。不自见xiàn),故明,不自是,故彰,不自伐,故有功,不自矜,故长。夫唯不争,故天下莫能与之争。古之所谓曲则全者,岂虚言哉?诚全而归之。
\end{yuanwen}	
	
\chapter{自然}
	
\begin{yuanwen}
希言自然。故飘风不终朝(zhāo),骤雨不终日。孰为此者?天地。天地尚不能久,而况人乎?故从事于道者,同于道,德者,同于德,失者,同于失。同于道者,道亦乐得之;同于德者,德亦乐得之;同于失者,失亦乐得之。信不足焉,有不信焉!
\end{yuanwen}

	
	
\chapter{跂跨}

\begin{yuanwen}
企者不立,跨者不行,自见(xiàn)者不明,自是者不彰,自伐者无功,自矜者不长。其在道也,曰:“余食赘(zhuì)行,物或恶(wù)之。”故有道者不处(chǔ)。
\end{yuanwen}	

	
	
	
\chapter{混成}

\begin{yuanwen}
有物混(hùn)成,先天地生。寂兮寥兮,独立而不改,周行而不殆,可以为天地母。吾不知其名,强字之曰道,强(qiǎng)为之名曰大。大曰逝,逝曰远,远曰反。故道大,天大,地大,人亦大。域中有四大,而人居其一焉。人法地,地法天,天法道,道法自然。
\end{yuanwen}
	

\chapter{重静}

\begin{yuanwen}
重为轻根,静为躁君。是以君子终日行不离辎(zī)重。虽有荣观(guàn),燕处超然,奈何万乘(shèng)之主,而以身轻天下?轻则失根,躁则失君。
\end{yuanwen}


\chapter{要妙}

\begin{yuanwen}
善行,无辙迹,善言,无瑕谪(xiá zhé),善数(shǔ),不用筹策,善闭,无关楗(jiàn)而不可开,善结,无绳约而不可解。是以圣人常善救人,故无弃人;常善救物,故无弃物,是谓袭明。故善人者,不善人之师;不善人者,善人之资。不贵其师,不爱其资,虽智大迷,是谓要妙。
\end{yuanwen}	

	
\chapter{常德}

\begin{yuanwen}
知其雄,守其雌,为天下谿。为天下谿,常德不离,复归于婴儿。知其白,守其黑,为天下式。为天下式,常德不忒(tè),复归于无极。知其荣,守其辱,为天下谷。为天下谷,常德乃足,复归于朴。朴散则为器,圣人用之则为官长(zhǎng),故大制不割。
\end{yuanwen}
	

\chapter{神器}

\begin{yuanwen}
将欲取天下而为之,吾见其不得已。天下神器,不可为也,不可执也。为者败之,执者失之。是以圣人无为,故无败;无执,故无失。夫物或行或随,或歔(xū)或吹,或强或羸(léi),或载或隳(huī)。是以圣人去甚,去奢,去泰。
\end{yuanwen}

\chapter{兵强}

\begin{yuanwen}
以道佐人主者,不以兵强天下,其事好(hào)还。师之所处,荆棘生焉。大军之后,必有凶年。善有果而已,不敢以取强。果而勿矜,果而勿伐,果而勿骄,果而不得已,果而勿强。物壮则老,是谓不道,不道早已。
\end{yuanwen}

	
\chapter{佳兵}

\begin{yuanwen}
夫兵者,不祥之器。物或恶(wù)之,故有道者不处(chǔ)。君子居则贵左,用兵则贵右。兵者,不祥之器,非君子之器。不得已而用之,恬淡为上,胜而不美。而美之者,是乐(yào)杀人。夫乐(yào)杀人者,则不可得志于天下矣。吉事尚左,凶事尚右。偏将军居左,上将军居右,言以丧(sāng)礼处之。杀人之众,以悲哀泣之,战胜,以丧礼处之。
\end{yuanwen}

	
	
\chapter{无名}

\begin{yuanwen}
道常无名,朴。虽小,天下莫能臣。侯王若能守之,万物将自宾。天地相合,以降甘露,民莫之令而自均。始制有名,名亦既有,夫亦将知止。知止可以不殆。譬道之在天下,犹川谷之于江海。
\end{yuanwen}
	
	
\chapter{明强}

\begin{yuanwen}
知人者智,自知者明。胜人者有力,自胜者强。知足者富,强行者有志,不失其所者久,死而不亡者寿。
\end{yuanwen}


	
\chapter{大道}

\begin{yuanwen}
大道汜兮,其可左右。万物恃之以生而不辞,功成而不有。衣养万物而不为主,常无欲,可名于小;万物归焉而不为主,可名为大。以其终不自为大,故能成其大。
\end{yuanwen}

	
	
\chapter{大象}

\begin{yuanwen}
执大象,天下往;往而不害,安平太。乐(yuè)与饵,过客止。道之出口,淡乎其无味,视之不足见(jiàn),听之不足闻,用之不足既\footnote{text}。
\end{yuanwen}

	
	
\chapter{微明}

\begin{yuanwen}
将欲歙(xī)之,必固张之;将欲弱之,必固强之;将欲废之,必固兴之;将欲夺之,必固与之,是谓微明\footnote{text}。柔弱胜刚强。鱼不可脱于渊\footnote{text},国之利器不可以示人。
\end{yuanwen}

	
\chapter{静正}
	
\begin{yuanwen}
道常无为而无不为。侯王若能守之,万物将自化。化而欲作,吾将镇之以无名之朴。镇之以无名之朴,夫将不欲。不欲以静,天下将自定\footnote{text}。
\end{yuanwen}
	
\part{德经}
	
\chapter{上德}
	
	\begin{yuanwen}
	上德不德,是以有德。下德不失德,是以无德。上德无为,而无以为也。上仁为之,而无以为也。上义为之,而有以为也。上礼为之,而莫之应也,则攘\footnote{r\v{a}ng}臂而扔之。故失道而后德,失德而后仁,失仁而后义,失义而后礼。夫礼者,忠信之薄\footnote{b\'o}也,而乱之首也。前识者,道之华也,而愚之首也。是以大丈夫居其厚,而不居其薄,居其实,而不居其华。故去彼取此。
	\end{yuanwen}
	
\begin{yuanwen}
【通行本】

上德不德\footnote{text},是以有德。下德不失德,是以无德。上德无为而无以为,下德无为而有以为\footnote{text}。上仁为之而无以为,上义为之而有以为,上礼为之而莫之应,则攘臂而扔之\footnote{text}。故失道而后德,失德而后仁,失仁而后义,失义而后礼。夫礼者,忠信之薄,而乱之首\footnote{text}。前识者\footnote{text},道之华,而愚之始。是以大丈夫处其厚,不居其薄;处其实,不居其华。故去彼取此。
\end{yuanwen}
		
\chapter{得一}
	
\begin{yuanwen}


	昔之得一者,天得一以清,地得一以宁,神得一以灵,谷得一以盈,侯王得一而以为天下正。其致之也,谓天毋已清,将恐裂。谓地毋已宁,将恐发\footnote{f\`ei,“发”通“废”。},谓神毋已灵,将恐歇。谓谷毋已盈,将恐竭。谓侯王毋已贵以高,将恐蹶\footnote{ju\'e}。故必贵而以贱为本,必高矣而以下为基。夫是以侯王自谓孤、寡、不穀\footnote{谷gǔ},此其贱之本与?非也?故致数\footnote{shu\`o}誉无誉。是故不欲琭\footnote{l\`u}琭若玉,硌硌若石。
	\end{yuanwen}
	
\begin{yuanwen}
【通行本】

昔之得一者,天得一以清\footnote{text},地得一以宁,神得一以灵\footnote{text},谷得一以盈,万物得一以生\footnote{text},侯王得一以为天下正。其致之也,天无以清\footnote{text},将恐裂;地无以宁,将恐废;神无以灵,将恐歇;谷无以盈,将恐竭;万物无以生,将恐灭\footnote{text};侯王无以正,将恐蹶。故贵以贱为本,高以下为基。是以侯王自称孤、寡、不穀。此非以贱为本邪(yé)?非乎?故至誉无誉。是故不欲琭琭如玉,珞(luò)珞如石。
\end{yuanwen}
	
	
	
	

\chapter{勤用}
	
	\begin{yuanwen}
	反也者,道之动也。弱也者,道之用也。天下之物生于有,有生于无。
	\end{yuanwen}
	
\begin{yuanwen}
【通行本】

反者,道之动\footnote{text};弱者,道之用\footnote{text}。天下万物生于有\footnote{text},有生于无\footnote{text}。
\end{yuanwen}
	
\chapter{闻道}
	 
	\begin{yuanwen}
	上士闻道,勤能行之。中士闻道,若存若亡。下士闻道,大笑之。弗笑,不足以为道。是以建言有之曰:明道如昧,进道如退,夷道如颣\footnote{l\`ei}。上德如谷,大白如辱,广德如不足。建德如偷,质真如渝。大方无隅,大器免成,大音希声,大象无形,道褒无名。夫唯道,善始且善成。
	\end{yuanwen}
	
\begin{yuanwen}
【通行本】

上士闻道\footnote{text},勤而行之;中士闻道,若存若亡\footnote{text};下士闻道,大笑之。不笑,不足以为道。故建言有之\footnote{text}:明道若昧,进道若退,夷道若颣\footnote{text}。上德若谷,大白若辱\footnote{text},广德若不足,建德若偷\footnote{text},质真若渝\footnote{text}。大方无隅\footnote{text},大器晚成,大音希声\footnote{text},大象无形,道隐无名。夫唯道,善贷且成\footnote{text}。
	\end{yuanwen}
	
\chapter{冲和}
	
\begin{yuanwen}
	道生一,一生二,二生三,三生万物。万物负阴而抱阳,中气以为和。人之所恶(wù),唯孤、寡、不穀(谷gǔ),而王公以自名也。物或损之而益,益之而损。故人之所教(jiào),亦议而教人。故强梁者不得其死,我将以为学父。
	\end{yuanwen}
	
\begin{yuanwen}
【通行本】

道生一\footnote{text},一生二\footnote{text},二生三,三生万物\footnote{text}。万物负阴而抱阳\footnote{text},冲气以为和\footnote{text}。人之所恶(wù),唯孤、寡、不穀(gǔ),而王公以为称(chēng)。故物,或损之而益,或益之而损。人之所教(jiào),我亦教之。强梁者不得其死,吾将以为教父\footnote{text}。
\end{yuanwen}
	
	
\chapter{至柔}
	
\begin{yuanwen}
天下之至柔\footnote{text},驰骋天下之至坚\footnote{text}。无有入无间\footnote{text},吾是以知无为之有益。不言之教,无为之益,天下希及之。	
\end{yuanwen}
	
	
\chapter{名身}

\begin{yuanwen}
【通行本】

名与身孰亲?身与货孰多\footnote{text}?得与亡孰病\footnote{text}?甚爱必大费\footnote{text},多藏必厚亡\footnote{text}。故知足不辱\footnote{text},知止不殆,可以长久。
\end{yuanwen}

\begin{yuanwen}
名与身孰亲?身与货孰多?得与亡孰病? 是故甚爱必大费,多藏必厚亡。知足不辱,知止不殆,可以长久。
\end{yuanwen}
	
	
	
\chapter{清静}

\begin{yuanwen}
大成若缺\footnote{text},其用不弊\footnote{text}。大盈若冲,其用不穷\footnote{text}。大直若屈\footnote{text},大巧若拙,大辩若讷\footnote{text}。静胜躁,寒胜热\footnote{text}。清静为天下正\footnote{text}。
\end{yuanwen}
	
	
	
	
\chapter{知足}

\begin{yuanwen}
天下有道,却走马以粪\footnote{text};天下无道,戎马生于郊\footnote{text}。祸莫大于不知足,咎莫大于欲得。故知足之足,常足矣\footnote{text}。
\end{yuanwen}
	
	
	
	
\chapter{户}

\begin{yuanwen}
不出户,知天下;不窥\footnote{text}牖,见天道\footnote{text}。其出弥远,其知弥少。是以圣人不行而知,不见而明,不为而成。
\end{yuanwen}
	


\chapter{日损}

\begin{yuanwen}
为学日益\footnote{text},为道日损\footnote{text}。损之又损,以至于无为。无为而无不为。取天下常以无事\footnote{text},及其有事\footnote{text},不足以取天下。
\end{yuanwen}	

	

\chapter{浑心}

\begin{yuanwen}
圣人常无心\footnote{有版本为“无常心”},以百姓之心为心。善者,吾善之;不善者,吾亦善之,德\footnote{text}善。信者,吾信之;不信者,吾亦信之,德信。圣人在天下,歙歙(xīxī)焉,为天下浑其心\footnote{text}。百姓皆注其耳目\footnote{text},圣人皆孩之\footnote{text}。
\end{yuanwen}
	
	
	
\chapter{摄生}

\begin{yuanwen}
出生入死\footnote{text},生之徒\footnote{text},十有三\footnote{text},死之徒\footnote{text},十有三。人之生,动之于死地,亦十有三。夫何故?以其生生之厚\footnote{text}。盖闻善摄生者\footnote{text},陆行不遇兕(sì)虎\footnote{text},入军不被(pī)甲兵\footnote{text}。兕无所投其角\footnote{text},虎无所措其爪(	zhǎo),兵无所容其刃。夫何故?以其无死地\footnote{text}。
\end{yuanwen}
	
	
	
\chapter{尊贵}

\begin{yuanwen}
道生之,德畜(xù)之,物形之,势成之\footnote{text}。是以万物莫不尊道而贵德。道之尊,德之贵,夫莫之命而常自然\footnote{text}。故道生之,德畜之。长之育之、亭之毒之\footnote{text}、养之覆之\footnote{text}。生而不有,为而不恃,长(zhǎng)而不宰,是谓玄德。
\end{yuanwen}


\chapter{有始}
	
\begin{yuanwen}
天下有始\footnote{text},以为天下母\footnote{text}。既得其母,以知其子\footnote{text};既知其子,复守其母,没(mò)身不殆。塞(sè)其兑,闭其门\footnote{text},终身不勤\footnote{text}。开其兑,济其事\footnote{text},终身不救。见(jiàn)小曰明\footnote{text},守柔曰强\footnote{text}。用其光,复归其明,无遗身殃\footnote{text},是为袭常\footnote{text}。
\end{yuanwen}
	
\chapter{}

\begin{yuanwen}
使我介然有知\footnote{text},行于大道,唯施是畏\footnote{text}。大道甚夷\footnote{text},而人好径\footnote{text}。朝(cháo)甚除\footnote{text},田甚芜,仓甚虚。服文彩,带利剑,厌饮食\footnote{text},财货有余,是为盗竽\footnote{text}。非道也哉!
\end{yuanwen}	

	
\chapter{}

\begin{yuanwen}
善建者不拔\footnote{text},善抱者不脱\footnote{text},子孙以祭祀不辍\footnote{text}。修之于身,其德乃真;修之于家,其德乃余;修之于乡,其德乃长(zhǎng)\footnote{text};修之于邦\footnote{text},其德乃丰;修之于天下,其德乃普。故以身观身,以家观家,以乡观乡\footnote{text},以邦观邦,以天下观天下。吾何以知天下然哉?以此\footnote{text}。
\end{yuanwen}	

	
\chapter{}

\begin{yuanwen}
含德之厚,比于赤子。毒虫不螫(shì)\footnote{text},猛兽不据\footnote{text},攫(jué)鸟不搏\footnote{text}。骨弱筋柔而握固。未知牝牡之合而朘作\footnote{text},精之至也。终日号而不嗄(shà)\footnote{text},和之至也\footnote{text}。知和曰常\footnote{text},知常曰明,益生曰祥\footnote{text},心使气曰强\footnote{text}。物壮则老\footnote{text},谓之不道,不道早已\footnote{text}。
\end{yuanwen}	

	
\chapter{}

\begin{yuanwen}
知(zhì)者不言,言者不知(zhì)\footnote{text}。塞(sè)其兑,闭其门\footnote{text},挫其锐\footnote{text};解其纷,和其光,同其尘\footnote{text},是谓玄同\footnote{text}。故不可得而亲,不可得而疏;不可得而利,不可得而害;不可得而贵,不可得而贱\footnote{text},故为天下贵。
\end{yuanwen}	

	
\chapter{}	

\begin{yuanwen}
以正治国\footnote{text},以奇用兵\footnote{text},以无事取天下。吾何以知其然哉?以此\footnote{text}:天下多忌讳\footnote{text},而民弥贫;人多利器\footnote{text},国家滋昏;人多伎(jì)巧\footnote{text},奇物滋起\footnote{text};法令滋彰,盗贼多有\footnote{text}。故圣人云:“我无为,而民自化\footnote{text};我好静,而民自正;我无事,而民自富;我无欲,而民自朴。”
\end{yuanwen}	

	
\chapter{}

\begin{yuanwen}
其政闷闷\footnote{text},其民淳淳\footnote{text};其政察察\footnote{text},其民缺缺\footnote{text}。祸兮,福之所倚;福兮,祸之所伏\footnote{text}。孰知其极:其无正也\footnote{text}。正复为奇\footnote{text},善复为妖\footnote{text}。人之迷,其日固久\footnote{text}。是以圣人方而不割\footnote{text},廉而不刿(guì)\footnote{text},直而不肆\footnote{text},光而不耀\footnote{text}。
\end{yuanwen}	

	
\chapter{}	

\begin{yuanwen}
治人事天\footnote{text},莫若啬(sè)。夫唯啬,是谓早服\footnote{text}。早服谓之重(chóng)积德,重(chóng)积德,则无不克,无不克,则莫知其极\footnote{text},莫知其极,可以有国。有国之母\footnote{text},可以长久。是谓深根固柢(dǐ),长生久视\footnote{text}之道。
\end{yuanwen}	

	
\chapter{}	

\begin{yuanwen}
治大国,若烹小鲜\footnote{text}。以道莅(lì)天下\footnote{text},其鬼不神\footnote{text}。非其鬼不神\footnote{text},其神不伤人;非其神不伤人,圣人亦不伤人。夫两不相伤\footnote{text},故德交归焉。
\end{yuanwen}	

	
\chapter{}	

\begin{yuanwen}
大邦者下流\footnote{text}。天下之牝,天下之交也\footnote{text}。牝常以静胜牡,以静为下。故大邦以下小邦,则取小邦;小邦以下大邦,则取大邦\footnote{text}。故或下以取,或下而取\footnote{text}。大邦不过欲兼畜(xù)人\footnote{text},小邦不过欲入事人,夫两者各得所欲,大者宜为下。
\end{yuanwen}
	
\chapter{}	
\begin{yuanwen}
道者,万物之奥\footnote{text},善人之宝,不善人之所保\footnote{text}。美言可以市尊,美行可以加人\footnote{text}。人之不善,何弃之有?故立天子,置三公\footnote{text},虽有拱璧以先驷马\footnote{text},不如坐进此道\footnote{text}。古之所以贵此道者何?不曰:求以得。有罪以免邪(yé)?故为天下贵。
\end{yuanwen}	

	
\chapter{}	

\begin{yuanwen}
为无为,事无事,味无味\footnote{text}。大小多少\footnote{text},报怨以德\footnote{text}。图难于其易,为大于其细。天下难事,必作于易,天下大事,必作于细。是以圣人终不为大\footnote{text},故能成其大。夫轻诺必寡信,多易必多难。是以圣人犹难之,故终无难矣。
\end{yuanwen}

	
\chapter{}	

\begin{yuanwen}
其安易持\footnote{text},其未兆易谋\footnote{text},其脆易泮(pàn)\footnote{text},其微易散。为之于未有,治之于未乱。合抱之木,生于毫末\footnote{text};九层之台,起于累土\footnote{text};千里之行,始于足下。为者败之,执者失之。是以圣人无为,故无败;无执,故无失\footnote{text}。民之从事,常于几成而败之\footnote{text}。慎终如始,则无败事。是以圣人欲不欲,不贵难得之货。学不学,复众人之所过。以辅万物之自然,而不敢为\footnote{text}。
\end{yuanwen}

	
\chapter{}	

\begin{yuanwen}
古之善为道者,非以明民\footnote{text},将以愚之\footnote{text}。民之难治,以其智多\footnote{text}。故以智治国,国之贼\footnote{text};不以智治国,国之福。知此两者\footnote{text},亦稽(jī)式\footnote{text}。常知稽式,是谓玄德。玄德深矣,远矣,与物反矣\footnote{text},然后乃至大顺\footnote{text}。
\end{yuanwen}

	
\chapter{}	

\begin{yuanwen}
江海所以能为百谷王者\footnote{text},以其善下之\footnote{text},故能为百谷王。是以圣人欲上民\footnote{text},必以言下之;欲先民,必以身后之。是以圣人处上而民不重\footnote{text},处前而民不害,是以天下乐推而不厌。以其不争,故天下莫能与之争。
\end{yuanwen}

	
\chapter{}	

\begin{yuanwen}
天下皆谓我“道”大\footnote{text},似不肖(xiào)\footnote{text}。夫唯大,故似不肖。若肖,久矣其细也夫\footnote{text}。我有三宝\footnote{text},持而保之:一曰慈,二曰俭\footnote{text},三曰不敢为天下先。慈,故能勇\footnote{text};俭,故能广\footnote{text};不敢为天下先,故能成器长(zhǎng)\footnote{text}。今舍慈且勇\footnote{text},舍俭且广,舍后且先,死矣!夫慈,以战则胜\footnote{text},以守则固。天将救之,以慈卫之。
\end{yuanwen}


\chapter{}	

\begin{yuanwen}
善为士者\footnote{text},不武,善战者,不怒,善胜敌者,不与\footnote{text},善用人者,为之下。是谓不争之德,是谓用人之力,是谓配天\footnote{text},古之极也。
\end{yuanwen}


\chapter{}	

\begin{yuanwen}

\end{yuanwen}
用兵有言:“吾不敢为主,而为客\footnote{text};不敢进寸,而退尺。”是谓行(xíng)无行(háng)\footnote{text},攘(rǎng)无臂\footnote{text},扔无敌\footnote{text},执无兵。祸莫大于轻敌,轻敌几丧吾宝。故抗兵相若\footnote{text},哀者胜矣\footnote{text}。

\chapter{}	

\begin{yuanwen}
吾言甚易知,甚易行。天下莫能知,莫能行。言有宗\footnote{text},事有君\footnote{text}。夫唯无知,是以不我知\footnote{text}。知我者希,则我者贵\footnote{text},是以圣人被(pī,“被”同“披”)褐而怀玉\footnote{text}。
\end{yuanwen}


\chapter{}	

\begin{yuanwen}
知不知\footnote{text},尚矣;不知知,病也。圣人不病\footnote{text},以其病病\footnote{text}。夫唯病病,是以不病。
\end{yuanwen}

	
\chapter{}	

\begin{yuanwen}
民不畏威\footnote{text},则大威至\footnote{text}。无狎(xiá)其所居\footnote{text},无厌(yà,“厌”同“压”)其所生\footnote{text}。夫唯不厌(yà,“厌”同“压”)\footnote{text},是以不厌(yàn)。是以圣人自知不自见(xiàn)\footnote{text};自爱不自贵。故去彼取此\footnote{text}。
\end{yuanwen}

	
\chapter{}	

\begin{yuanwen}
勇于敢则杀,勇于不敢则活\footnote{text}。此两者,或利或害\footnote{text}。天之所恶(wù),孰知其故?是以圣人犹难之\footnote{text}。天之道\footnote{text},不争而善胜,不言而善应,不召而自来,繟(chǎn)然而善谋\footnote{text}。天网恢恢\footnote{text},疏而不失\footnote{text}。
\end{yuanwen}

	
\chapter{}	

\begin{yuanwen}
民不畏死,奈何以死惧之?若使民常畏死,而为奇者\footnote{text},吾得执而杀之\footnote{text},孰敢?常有司杀者\footnote{text}杀,夫代司杀者杀\footnote{text},是谓代大匠斲(zhuó)。夫代大匠斲者,希有不伤其手矣。
\end{yuanwen}

	
\chapter{}	

\begin{yuanwen}
民之饥,以其上食税之多,是以饥。民之难治,以其上之有为\footnote{text},是以难治。民之轻死,以其上求生之厚\footnote{text},是以轻死。夫唯无以生为者\footnote{text},是贤于贵生\footnote{text}。
\end{yuanwen}

	
\chapter{}	

\begin{yuanwen}	
人之生也柔弱,其死也坚强。草木\footnote{text}之生也柔脆\footnote{text},其死也枯槁。故坚强者死之徒,柔弱者生之徒。是以兵强则灭,木强则折\footnote{text}。强大处下,柔弱处上。
\end{yuanwen}
		
\chapter{}	

\begin{yuanwen}	
天之道,其犹张弓与?高者抑之,下者举之;有余者损之\footnote{text},不足者补之。天之道,损有余而补不足。人之道\footnote{text}则不然,损不足以奉有余。孰能有余以奉天下?唯有道者。是以圣人为而不恃,功成而不处,其不欲见(xiàn)贤\footnote{text}。
\end{yuanwen}
		
\chapter{}

\begin{yuanwen}	
天下莫柔弱于水,而攻坚强者莫之能胜,以其无以易之\footnote{text}。弱之胜强,柔之胜刚,天下莫不知,莫能行。是以圣人云:“受国之垢\footnote{text},是谓社稷主;受国不祥\footnote{text},是为天下王。”正言若反\footnote{text}。
\end{yuanwen}
		
\chapter{}	

\begin{yuanwen}	
和大怨,必有余怨,安可以为善?是以圣人执左契\footnote{text},而不责\footnote{text}于人。有德司契,无德司彻\footnote{text}。天道无亲\footnote{text},常与善人。
\end{yuanwen}
		
\chapter{}	

\begin{yuanwen}	
小国寡民\footnote{text}。使有什伯(bǎi)之器而不用\footnote{text},使民重(zhòng)死而不远徙(xí)\footnote{text}。虽有舟舆\footnote{text},无所乘之;虽有甲兵,无所陈之\footnote{text}。使人复结绳\footnote{text}而用之。甘其食,美其服,安其居,乐其俗\footnote{text}。邻国相望,鸡犬之声相闻,民至老死,不相往来。
\end{yuanwen}
		
\chapter{}
	
\begin{yuanwen}
信言不美\footnote{text},美言不信。善者不辩\footnote{text},辩者不善。知(zhì)者不博,博者不知(zhì)。圣人不积\footnote{text},既以为人,己愈有\footnote{text};既以与人,己愈多\footnote{text}。天之道,利而不害\footnote{text}。圣人之道\footnote{text},为而不争。
\end{yuanwen}
	
\end{document}