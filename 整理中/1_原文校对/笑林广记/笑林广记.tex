% 笑林广记
% 笑林广记.tex

\documentclass[12pt,UTF8]{ctexbook}

% 设置纸张信息。
\usepackage[a4paper,twoside]{geometry}
\geometry{
	left=25mm,
	right=25mm,
	bottom=25.4mm,
	bindingoffset=10mm
}

% 设置字体,并解决显示难检字问题。
\xeCJKsetup{AutoFallBack=true}
\setCJKmainfont{SimSun}[BoldFont=SimHei, ItalicFont=KaiTi, FallBack=SimSun-ExtB]

% 目录 chapter 级别加点(.)。
\usepackage{titletoc}
\titlecontents{chapter}[0pt]{\vspace{3mm}\bf\addvspace{2pt}\filright}{\contentspush{\thecontentslabel\hspace{0.8em}}}{}{\titlerule*[8pt]{.}\contentspage}

% 设置 part 和 chapter 标题格式。
\ctexset{
	part/name= {,卷},
	part/number={\chinese{part}},
	chapter/name={},
	chapter/number={}
}

% 设置古文原文格式。
\newenvironment{yuanwen}{\bfseries\zihao{4}}

% 设置署名格式。
\newenvironment{shuming}{\hfill\bfseries\zihao{4}}

% 注脚每页重新编号,避免编号过大。
\usepackage[perpage]{footmisc}

\title{\heiti\zihao{0} 笑林广记}
\author{游戏主人\ 程世爵}
\date{清}

\begin{document}

\maketitle
\tableofcontents

\frontmatter
\chapter{前言、序言}

《笑林广记》是清代程世爵撰文言笑话集,又名《增广笑林广记》《增广笑林广记全集》,不分卷,成书于清光绪二十五年(1899)。 [1]
《笑林广记》是集中国古代民间传统笑话之大成者,清代署名“游戏主人”收集而成。全书分12部,所收笑话或直刺现实,或隐讽世情,有的格调高雅,妙趣横生,充分表现出了劳动人民的机敏和幽默,表现了他们对世间邪恶习气作风的嘲讽。 [2]

文学体裁
笑话集

书前有题为“光绪二十有五年岁次己亥仲夏”的作者自序,其中云:“爰自杜门谢客,假余岁月宽闲,闭户著书,读彼光阴迅速,抒胸中所记忆,必教尽相穷形。佐抗底成文章,原属耳闻目见。倘或逢人说鬼,对客解颐,有时拍案叫奇,供余适口,使敝庐顿作为安乐窝,鼓大块尽成欢笑场。岂非一时快意事哉!”其著书的目的是“以讥刺劝讽有关名教者”。全书意在嘲讽当时社会之各种病态世相,发泄作者对官场世相之种种愤懑之情;作者胸有块垒,行之于笔,诙谐幽默,文字也舒缓得多。虽与游戏主人《笑林广记》同名,而内容实异,全书不少小说叙事性强,有较完整的故事情节,人物性格鲜明。出现在作者笔下的小偷、妓女、嫖客、庸医、县令、巡抚、警察、夫人,无不栩栩如生,构成了一幅晚清社会的风俗画。 [1]

版本信息

现存主要版本有清光绪二十五年(1899)刊本;清石印本;民国广益书局铅印本,藏上海辞书出版社。台湾新兴书局“笔记小说大观”影印清石印本,1996年齐鲁书社排印清光绪二十五年(1899)刊本。 [1]

\mainmatter

\part{}

\chapter{古艳部}

\section{升官}

一官升职,谓其妻曰:“我的官职比前更大了。”

妻曰:“官大,不知此物亦大否?”

官曰:“自然。”

及行事,妻怪其藐小如故,官曰:“大了许多,汝自不觉耳。”

妻曰:“如何不觉?”官曰:“难道老爷升了官职,奶奶还照旧不成?少不得我的大,你的也大了。”

比职
甲乙两同年初中。甲选馆职,乙授县令。甲一日乃骄语之曰:“吾位列清华,身依宸禁,与年兄做有司者,资格悬殊。他不具论,即选拜客用大字帖儿,身份体面,何啻天渊。”乙曰:“你帖上能用几字,岂如我告示中的字,不更大许多?晓谕通衢,百姓无不凛遵恪守,年兄却无用处。”甲曰:“然则金瓜黄盖,显赫炫耀,兄可有否?”乙曰:“弟牌棍清道,列满街衢,何止多兄数倍?”甲曰:“太史图章,名标上苑,年兄能无羡慕乎?”乙曰:“弟有朝廷印信,生杀之权,惟吾操纵,视年兄身居冷曹,图章私刻,谁来怕你?”甲不觉词遁,乃曰:“总之,翰林声价值千金。”乙笑曰:“吾坐堂时,百姓口称青天爷爷,岂仅千金而已耶?”

发利市
一官新到任,祭仪门毕,有未烬纸钱在地,官即取一锡锭藏好。门子禀曰:“老爷,这是纸钱,要他何用?”官曰:“我知道,且等我发个利市看。”

贪官
有农夫种茄不活,求计于老圃。老圃曰:“此不难,每茄树下埋钱一文即活。”问其何故,答曰:“有钱者生,无钱者死。”

有理
一官最贪。一日,拘两造对鞫,原告馈以五十金,被告闻知,加倍贿托。及审时,不问情由,抽签竟打原告。原告将手作五数势曰:“小的是有理的。”官亦以手覆曰:“奴才,你讲有理。”又以手一仰曰:“他比你更有理哩。”

取金
一官出朱票,取赤金二锭,铺户送讫,当堂领价。官问:“价值几何?”铺家曰:“平价该若干,今系老爷取用,只领半价可也。”官顾左右曰:“这等,发一锭还他。”发金后,铺户仍候领价。官曰:“价已发过了。”铺家曰:“并未曾发。”官怒曰:“刁奴才,你说只领半价,故发一锭还你,抵了一半价钱。本县不曾亏了你,如何胡缠?快撵出去!”

胡涂
一青盲人涉讼,自诉眼瞎。官曰:“你明明一双清白眼,如何诈瞎?”答曰:“老爷看小人是清白的,小人看老爷却是糊堡得紧。”

不明
一官断事不明,惟好酒怠政,贪财酷民。百姓怨恨,乃作诗以诮之云:“黑漆皮灯笼,半天萤火虫。粉墙画白虎,黄纸写乌龙。茄子敲泥磬,冬瓜撞木钟。唯知钱与酒,不管正和公。”

启奏
一官被妻踏破纱帽,怒奏曰:“臣启陛下,臣妻罗皂,昨日相争,踏破臣的纱帽。”上传旨云:“卿须忍耐。皇后有些惫赖,与朕一言不合,平天冠打得粉碎,你的纱帽只算得个卵袋。”

偷牛
有失牛而讼于官者,官问曰:“几时偷去的?”答曰:“老爷,明日没有的。”吏在傍不觉失笑,官怒曰:“想就是你偷了!”吏洒两袖口:“任凭老爷搜。”

避暑
官值暑月,欲觅避凉之地。同僚纷议,或曰某山幽雅,或曰某寺清闲。一老人进曰:“山寺虽好,总不如此座公厅,最是凉快。”官曰:“何以见得?”答曰:“别处多有日头,独此处有天无日。”

石碑
一官素有清名,考察任满,父老与之立德政碑告成。官命打轿往观之,先于公厂坐下。少顷,左右禀曰:“请老爷看石(肏)碑(屄)。”

强盗脚
乡民初次入城﹔见有木桶悬于城上,问人曰:“此中何物?”应者曰:“强盗头。”及至县前,见无数木匣钉于谯楼之上,皆前官既去而所留遗爱之靴。乡民不知,乃点首曰:“城上挂的强盗头,此处一定是强盗脚了。”

属牛
一官遇生辰,吏典闻其属鼠,乃醵黄金铸一鼠为寿。官甚喜,曰:“汝等可知奶奶生日,亦在目下乎?”众吏曰:“不知,请问其属?”’官曰:“小我一岁,丑年生的。”

同僚
有妻妾各居者,一日,妾欲谒妻,谋之于夫:“当如何写帖?”夫曰:“该用‘寅弟’二字。”妾问:“其义何居?”夫曰:“同僚写帖,皆用此称呼,做官府之例耳。”妾曰:“我辈并无官职,如何亦写此帖?”夫曰:“官职虽无,同僚(屪)总是一样。”

家属
官坐堂,众役中有撒一响屁,官即叫:“拿来!”隶禀曰:“老爷,屁是一阵风,吹散没影踪,叫小的如何拿得?”官怒云:“为何徇情卖放,定要拿到。”皂无奈,只得取干屎回销:“禀老爷,正犯是走了,拿得家属在此。”

州同
一人最好古董,有持文王鼎求售者,以百金买之。又一人持一夜壶至,铜色斑驳陆离,云是武王时物,亦索重价。曰:“铜色虽好,只是肚里臭甚。”答曰:“腹中虽臭,难道不是个州同。”

衙官隐语
衙官聚会,各问何职。一官曰:“随常茶饭掇将来,盖义取现成(县丞)也”一官曰:“滚汤锅里下文书,乃煮(主)簿也。”一官曰:“乡下蛮子租粪窖。”问者不解,答曰:“典屎(史)。”

详梦
一作吏典者,有媳妇最善详梦。适三考已满,将往谒选。夜得一梦,呼媳详之。媳问:“何梦?”公曰:“梦见把许多册籍,放在锅内熬煮,不知主何吉凶?”媳曰:“初选一定是个主簿。”隔数日,公曰:“我又得一梦,梦见你我二人皆裸体而立,身子却是相背的,何也?”媳曰:“恭喜一转,就是县(现)丞(成)。”

太监观风
镇守大监观风,出“后生可畏焉”为题,众皆掩口而笑。珰问其故,教官禀曰:“诸生以题目太难,求减得一字也好。”珰笑曰:“既如此,除了‘后’字,只做‘生可畏焉’罢。”

常礼
内相见人撒尿,喜甚,唤他过来一看。其人脱裤,见此物尚在撺动,内相拍掌大喜曰:“我的乖儿,见我公公,只消常礼儿罢了。”

念劾本
一辽东武职,素不识字。一日被论,使人念劾本云:“所当革任回卫者也。”因痛哭曰:“‘革任回卫’还是小事,这‘者也’二字,怎么当得起!”

武弁夜巡
一武弁夜巡,有犯夜者,自称书生会课归迟。武弁曰:“既是书生,且考你一考。”生请题,武弁思之不得,喝曰:“造化了你,今夜幸而没有题目。”

垛子助阵
一武官出征将败,忽有神兵助阵,反大胜。官叩头请神姓名,神曰:“我是垛子。”官曰:“小将何德,敢劳垛子尊神见救?”答曰:“感汝平昔在教场,从不曾有一箭伤我。”

进士第
一介弟横行于乡,怨家骂曰:“兄登黄甲,与汝何干,而豪横若此?”答曰:“你不见匾额上面写着‘进士第(弟)’么?”

及第
一举子往京赴试,仆挑行李随后。行到旷野,忽狂风大作,将担上头巾吹下。仆大叫曰:“落地了!”主人心下不悦,嘱曰:“今后莫说落地,只说及第。”仆领之﹔将行李拴好,曰:“如今恁你走上天去,再也不会及第了。”

嘲武举诗
头戴银雀顶,脚踏粉底皂。也去参主考,也来谒孔庙。颜渊喟然叹,夫子莞尔笑。子路愠见曰:“这般呆狗醮,我若行三军,都去喂马料。”

封君
有市井获封者,初见县官,甚局蹐,坚辞上坐。官曰:“叨为令郎同年,论理还该侍坐。”封君乃张目问曰:“你也是属狗的么?”

老父
一市井受封,初见县官,以其齿尊,称之曰:“老先。”其人含怒而归,子问其故,曰:“官欺我太甚。彼该称我老先生才是,乃作歇后语,叫甚么老先,明系轻薄。我回称,也不曾失了便宜。”子询何以称呼,答曰:“我本应称他老父母,今亦缩住后韵,只叫他声老父。”

公子封君
有公子兼封君者,父对子,乃欣羡不已。讶问其故,曰:“你的爷既胜过我的爷,你的儿又胜过我的儿。”

送父上学
一人问:“公子与封君孰乐?”答曰:“做封君虽乐,齿已衰矣,惟公子年少最乐。”其人急趋而去,追问其故,答曰:“买了书,好送家父去上学。”

纳粟诗
赠纳粟诗曰:“革车买得截的高,周子窗前满腹包。有朝若遇高曾祖,焕乎其有没分毫。”

考监
一监生过国学门,闻祭酒方盛怒两生而治之,问门上人者:“然则打欤?罚欤?镦锁欤?”答曰:“出题考文。”生即咈然曰:“咦,罪不至此。”

坐监
一监生妻,屡劝其夫读书,因假寓于寺中。素无书箱,乃唤脚夫以罗担挑书先往。脚夫中途疲甚,身坐担上。适生至,闻傍人语所坐《通鉴》,因怒责脚夫。夫谢罪曰:“小人因为不识字,一时坐了鉴(监),弗怪弗怪。”

不往京
一监生娶妾,号曰京姐,妻妒甚。夫诣妾,必告曰:“京里去。”一日,欲往京去,妻曰:“且在此关上纳了纱着。”既行事讫,妻曰:“汝今何不往京!”生曰:“绒也没有一些在肚里,京里去做甚么!”

咬飞边
贫子途遇监生,忽然抱住兜耳一口。生惊问其故,答曰:“我穷苦极矣,见了大锭银子,如何不咬些飞边用用。”

入场
监生应付入场方出,一故人相遇揖之,并揖路傍猪屎。生问:“此臭物,揖之何为?”答曰:“他臭便臭,也从大肠(场)里出来的。”

书低
一生赁僧房读书,每日游玩,午后归房,呼童取书来。童持《文选》,视之,曰:“低。”持《汉书》,视之,曰:“低。”又持《史记》,视之,曰:“低。”僧大诧曰:“此三书,熟其一,足称饱学,俱云低,何也?”生曰:“我要睡,取书作枕头耳。”

监生娘娘
监生至城隍庙,傍有监生案。塑监生娘娘像。归谓妻曰:“原来我们监生恁般尊贵,连你的像,早已都塑在城隍庙里了。”

监生自大
城里监生与乡下监生,各要争大。城里者耻之曰:“我们见多识广,你乡里人孤陋寡闻。”两人争辩不已,因往大街同行,各见所长。到一大第门首,匾上“大中丞”三字,城里监生倒看指谓曰:“这岂不是丞中大?乃一征验。”又到一宅,匾额是“大理卿”,乡下监生以“卿”字认作“乡”字,忙亦倒念指之曰:“这是乡里大了。”两人各不见高下。又来一寺门首,上题“大士阁”,彼此平心和议曰:”原来阁(各)士(自)大。”

打丁
一人往妓馆打丁毕,妓牵之索谢,答曰:“我生员也,奉祖制免丁。”俄焉又一人至,亦如之。妓曰:“为何?”答曰:“我监生也。”妓曰:“监生便怎么?”其人曰:“岂不知监生从来是白丁。”

王监生
一监生姓王,加纳知县到任。初落学,青衿呈书,得“牵牛”章。讲诵之际,忽问:“那王见之是何人?”答曰:。‘此王诵之之兄也。”又问:“那王曰然是何人?”答曰:“此王曰叟之弟也。”曰:“妙得紧。且喜我王氏一门,都在书上。”

自不识
有监生,穿大衣,带圆帽,于着衣镜中自照,得意甚。指谓妻曰:“你看镜中是何人?”妻曰:“臭乌龟!亏你做了监生,连自(字)多不识。”

监生拜父
一人援例入监,吩咐家人备帖拜老相公。仆曰:“父子如何用帖,恐被人谈论。”生曰:“不然。今日进身之始,他客俱拜,焉有亲父不拜之理?”仆问:“用何称呼?”生沉吟曰:“写个眷侍教生罢。”父见,怒责之。生曰:“称呼斟酌切当,你自不解。父子一本至亲,故下一‘眷’字。‘侍’者,父坐子立也。‘教’者,从幼延师教训。生者,父母生我也。”父怒转盛,责其不通,生谓仆曰:“想是嫌我太妄了,你去另换过晚生帖儿来罢。”

半字不值
一监生妻谓其孤陋寡闻,使劝读书。问:“读书有甚好处?”妻曰:“一字值千金,如何无益?”生答曰:“难道我此身,半个字也不值?”

借药撵
一监生临终,谓妻曰:“我一生挣得这副衣冠,死后必为我殡殓。”妻诺。既死,穿衣套靴讫,惟圆帽左右欹侧难带。妻哭曰:“我的天,一顶帽子也无福带。”生复转魂,张目谓妻曰:“必要带的。”妻曰:“非不欲带,恨枕不稳耳。”生曰:“对门某医生家药撵槽,借来好做枕。”

斋戒库
一监生姓齐,家资甚富,但不识字。一日,府尊出票,取鸡二只,兔一只。皂亦不识票中字,央齐监生看。生曰:“讨鸡二只,免一只。”皂只买一鸡回话。太守怒曰:“票上取鸡二只,兔一只,为何只缴一鸡?”皂以监生事禀,太守遂拘监生来问。时太守适有公干,暂将监生收入斋戒库内候究。生入库,见碑上“斋戒”二字,认做他父亲“齐成”姓名,张目惊诧,呜咽不止。人问何故,答曰:“先人灵座,何人设建在此?睹物伤情,焉得不哭。”

附例
一秀才畏考援例,堂试之日,至晚不能成篇。乃大书卷面曰:“惟其如此,所以如此。若要如此,何苦如此。”官见而笑曰:“写得此四句出,毕竟还是个附例。”

酸臭
小虎谓老虎曰:“今日出山,搏得一人,食之滋味甚异,上半截酸,下半截臭,究竟不知是何等人。”老虎曰:“此必是秀才纳监者。”

仿制字
一生见有投制生帖者,深叹“制”字新奇。偶致一远札,遂效之。仆致书回,生问:“见书有何话说?”仆曰,“当面启看,便问:‘老相公无恙?’又问:‘老安人好否?’予曰:‘俱安。’乃沉吟半响,带笑而入,纔发回书。”生大喜曰:“人不可不学,只一字用得着当,便一家俱问,到添下许多殷勤。”

春生帖
一财主不通文墨,谓友曰:“某人甚是欠通,清早来拜我,就写晚生帖。”傍一监生曰:“这到还差不远。好像这两日秋天拜客,竟有写春(眷)生帖子的哩。”

借牛
有走柬借牛于富翁者,翁方对客,讳不识字,伪启缄视之。对来使曰:“知道了,少刻我自来也。”

哭麟
孔子见死麟,哭之不置。弟子谋所以慰之者,乃编钱挂牛体,告曰:“鳞已活矣。”孔子观之曰:“这明明是一只村牛,不过多得几个钱耳。”

江心赋
有富翁同友远出,泊舟江中。偶散步上岸,见壁间题“江心赋”三字,错认“赋”字为“贼”字,惊欲走匿。友问故,指曰:“此处有贼。”友曰:“赋也,非贼也。”其人曰:“赋(富)便赋了,终是有些贼形。”

吃乳饼
富翁与人论及童子多肖乳母,为吃其乳,气相感也。其人谓富翁曰:“若是如此,想来足下从幼是吃乳饼大的。”

不愿富
一鬼托生时,冥王判作富人。鬼曰:“不愿富也。但求一生衣食不缺,无是无非,烧清香,吃苦茶,安闲过日足矣。”冥王曰:“要银子便再与你几万,这样安闲清福,却不许你享。”

姜字塔
一富翁问“姜”字如何写,对以草字头,次一字,次田字,又一字,又田字,又一字。其人写草、壹、田、壹、田、壹,写讫玩之,骂曰:“天杀的,如何诳我!分明作耍我造成一座宝塔了。”

医银入肚
一富翁含银于口,误吞入腹,痛甚,延医治之。医曰:“不难,先买纸牌一副,烧灰咽之,再用艾丸灸脐,其银自出。”翁询其故,医曰:“外面用火烧,里面有强盗打劫,那怕你的银子不出来!”

田主见鸡
一富人有余田数亩、租与张三者种,每亩索鸡一只。张三将鸡藏于背后,田主遂作吟哦之声曰:“此田不与张三种。”张三忙将鸡献出,田主又吟曰:“不与张三却与谁?”张三曰:“初间不与我,后又与我,何也?”田主曰:“初乃无稽(鸡)之谈,后乃见机(鸡)而作也。”

讲解
有姓李者暴富而骄,或嘲之云:一童读《百家姓》首句,求师解释。师曰:“赵是精赵的赵字(吴俗谓人呆为赵),钱是有铜钱的钱字·孙是小猢狲的孙字,李是姓张姓李的李字。”童又问:“倒转亦可讲得否?”师曰:“也得。”童曰:“如何讲?”师曰:“不过姓李的小猢狲,有了几个臭铜钱饯,一时就精赵起来。”

训子
富翁子不识字,人劝以延师训之。先学“一”字是一画,次“二”字二画,次“三”字三画。其子便欣然投笔,告父曰:“儿已都晓字义,何用师为?”父喜之,乃谢去。一日,父欲招万姓者饮,命子晨起治状,至午不见写成。父往询之,子患曰:“姓亦多矣,如何偏姓万。自早至今,才得五百画着哩!」

\part{}

腐流部

辞朝
一教官辞朝见象,低徊留之不忍去。人问其故,答曰:“我想祭丁的猪羊,有这般肥大便好。”

上任
岁贡选教职,初上任,其妻进衙,不觉放声大哭。夫惊问之,妻曰:“我巴得你到今日,只道出了学门,谁知反进了学门。”

争脏
祭丁过,两广文争一猪大脏,各执其脏之一头。一广文稍强,尽掣得其脏,争者止两手撸得脏中油一捧而已。因曰:“予虽不得大葬(脏),君无尤(油)焉。”

厮打
教官子与县丞子厮打,教官子屡负,归而哭诉其母。母曰:“彼家终日吃肉,故恁般强健会打。你家终日吃腐,力气衰微,如何敌得他过?”教官曰:“这般我儿不要忙,等祭过了丁,再与他报复便了。”

钻刺
鼠与黄蜂拜为兄弟,邀一秀才做盟证,秀才不得已往,列为第三人。一友问曰:“兄何居乎鼠辈之下?”答曰:“他两个一会钻,一会刺,我只得让他罢了。”

证孔子
两道学先生议论不合,各自诧真道学而互诋为假,久之不决。乃请正于孔子,孔子下阶,鞠躬致敬而言曰:“吾道甚大,何必相同。二位老先生皆真正道学,丘素所钦仰,岂有伪哉。”两人各大喜而退。弟子曰:“夫子何谀之甚也!”孔子曰:“此辈人哄得他动身就勾了,惹他怎么?”

放肆
道学先生嫁女出门,至半夜,尚在厅前徘徊踱索。仆云:“相公,夜深请睡罢。”先生顿足怒云:“你不晓得,小畜生此时正在那里放肆了!”

贽礼
广文到任,门人以钱五十为贽者,题刺曰:“谨具贽仪五十文,门人某百顿首拜。”师书其帖而返之,曰:“减去五十拜,补足一百文何如?”门人答曰:“情愿一百五十拜,免了这五十文又何如?”

不养子
一士夫子孙繁衍,而同侪有无子者,乃骄语之曰:“尔没力量,儿子也养不出一个。像我这等子孙多,何等热闹。”同侪答曰:“其子尔力也,其孙非尔力也。”

借粮
孔子在陈绝粮,命颜子往回回国借之,以其名与国号相同,冀有情熟。比往通讫,大怒曰:“汝孔子要攘夷狄,怪俺回回,平日又骂俺回之为人也择(贼)乎!”粮断不与。颜子怏怏而归。子贡请往,自称平昔极奉承,常曰:“赐也何敢望回回。”群回大喜,以白粮一担,先令携去,许以陆续运付。子贡归,述之夫子,孔子攒眉曰:“粮便骗了一担,只是文理不通。”

廪粮
粮长收粮在仓廪内,耗鼠甚多,潜伺之,见黄鼠群食其中。开仓掩捕,黄鼠有护身屁,连放数个。里长大怒曰:“这样放屁畜生,也被他吃了粮去。”

脱科
其年乡试,一县脱科。诸生请堪舆来看风水,以泥塑圣像卵小,不相称故耳。遂唤妆佛匠改造。圣人大喝曰:“这班不通文理的畜生,你们自不读书,干我卵甚事!”

黉门
三秀才往妓家设东叙饮,内一秀才曰:“兄治何经?”曰:“通《诗经》。”复问其次,曰:“通《书经》。”因戏问妓曰:“汝通何经?”曰:“妾通月经。”众皆大笑。妓曰:“列位相公休笑我,你们做秀才,都从这红门中出来的。”

野味
甲乙二士应试,甲曰:“我梦一木冲天,何如?”乙曰:“一木冲天,乃‘未’字也,恐非佳兆。”因言己“梦一雉贴天而飞,此必文门之象,稳中无疑矣。”甲摇首曰:“咦,野(也)味(未)。”

僧士诘辩
秀才诘问和尚曰:“你们经典内‘南无’二字,只应念本音,为何念作那摩?”僧亦回问云:“相公,《四书》上‘于戏’二字,为何亦读作呜呼?如今相公若读于戏,小僧就念南无。相公若是呜呼,小僧自然要那摩。”

杨相公
一人问曰:“相公尊姓?”曰:“姓杨。”其人曰:“既是羊,为甚无角?”士怒曰:“呆狗入出的!”那人错会其意,曰:“嗄!”

头场
玉帝生日,群仙毕贺。东方朔后至,见寿星傍惶门外,问之,曰:“有告示贴出,不放我进。”又问:“何故贴出?”答曰:“怪我头长(场)”

后场
宾主二人同睡,客索夜壶。主人说:“在床下,未曾倒得。”只好棚过头一场,后场断断再来不得了。

识气
一瞎子双目不明,善能闻香识气。有秀才拿一《西厢》本与他闻,曰:“《西厢记》。”问:“何以知之?”答曰:“有些脂粉气。”又拿《三国志》与他闻,曰:“《三国志》。”又问:“何以知之。”答曰:“有些刀兵气。”秀才以为奇异,却将自做的文字与他闻,瞎子曰:“此是你的佳作。”问:“你怎知?”答曰:“有些屁气。”

蛀帽
有盛大、盛二者,所戴毡帽,合放一处。一被虫蛀,兄弟二人互相推竞,各认其不蛀者夺之。适一士经过,以其读书人明理,请彼决之。士执蛀帽反复细看,乃睨盛大曰:“此汝帽也!”问:“何以见得?”士曰:“岂不闻《大学》注解云:‘宣(先)着(蛀),盛大之貌(帽)’。”

无一物
穷人往各寺院,窃取神物灵心,止有土地庙未取。及去挖开,见空空如也。乃骇叹曰:“看他巾便带了一顶,原来腹中毫无一物!”

带巾人
一和尚撒尿,玩弄自己阳物。偶有带巾人走来,戏曰:“你师徒两个,在此讲甚么?”和尚曰:“看他头有几多大,要折顶方巾与他带带。”

穷秀才
有初死见冥王者,王谓其生前受用太过,判来生去做一秀才,与以五子。鬼吏禀曰:“此人罪重,不应如此善遣。”王笑曰:“正惟罪重,我要处他一个穷秀才,把他许多儿子,活活累杀他罢了。”

颂屁
一士死见冥王,自称饱学,博古通今。王偶撒一屁,士即进词云:“伏惟大王高耸金臀,洪宣宝屁,依稀乎丝竹之声,仿佛乎麝兰之气。臣立下风,不胜馨香之味。”王喜,命赐宴,准与阳寿一纪,至期自来报到,不消鬼卒勾引。士过十二年,复诣阴司,谓门上曰:“烦到大王处通禀,说十年前做放屁文章的秀才又来了。”

出学门
儒学碑亭新完,一士携妓往视,见碑下负重,戏谓妓曰:“汝父在此,为何不拜?”妓即下拜云:“我你爷,看你这等蹭蹬,何时得出学门!”

抄祭文
东家丧妻母,往祭,托馆师撰文。乃按古本误抄祭妻父者与之,为识者看出,主人怪而责之。馆师曰:“此文是古本刊定的,如何得错?只怕倒是他家错死了人,这便不关我事。”

行房
一秀士新娶,夜分就寝,问于新妇曰:“吾欲云雨,不知娘子尊意允否?”新人曰:“官人从心所欲。”士曰:“既蒙俯允,请娘子展股开肱,学生无礼又无礼矣。”及举事,新妇曰:“痛哉,痛哉!”秀才曰:“徐徐而进之,浑身通泰矣。”

做不出
租户连年欠租,每推田瘦做不出米来。士怒曰:“明年待我自种,看是如何?”租户曰:“凭相公拼着命去种,到底是做不出的。”

凑不起
一士子赴试,艰于构思。诸生随牌俱出。接考者候久,甲仆问乙仆曰:“不知作文一篇,约有多少字?”乙曰:“想来不过五六百。”甲曰:“五六百字,难道胸中便没有了,此时还不出来?”乙曰:“五六百字虽有在肚里,只是一时凑不起来耳。”

四等亲家
两秀才同时四等,于受责时曾识一面。后联姻,会亲日相见。男亲家曰:“尊容曾在何处会过来?”女亲家曰:“便是有些面善,一时想不起。”各沉吟间,忽然同悟,男亲家点头曰:“嗄。”女亲家亦点头曰:“嗄。”

七等割屪
一士考末等,自觉惭愧,且虑其妻之姗己也。乃架一说诳妻曰:“从前宗师止于六等,今番遇着这个瘟官,好不利害,又增出一等,你道可恶不可恶?”妻曰:“七等如何?”对曰:“六等不过去前程,考七等者,竟要阉割。”妻大惊曰:“这等,你考在何处?”夫曰:“还亏我争气,考在六等,幸而免割。”

腹内全无
一秀才将试,日夜懮郁不已。妻乃慰之曰:“看你作文,如此之难,好似奴生产一般。”夫曰:“还是你每生子容易。”妻曰:“怎见得?”夫曰:“你是有在肚里的,我是没在肚里的。”

不完卷
一生不完卷,考置四等,受朴。对友曰:“我只缺得半篇。”友云:“还好。若做完,看了定要打杀。”

求签
一士岁考求签,通陈曰:“考在六等求上上,四等下下。”庙祝曰:“相公差矣,四等止杖责,如何反是下下?”士曰:“非汝所知。六等黜退,极是干净。若是四等,看了我的文字,决被打杀。”

梦入泮
府取童生,祈梦:“道考可望入泮否?”神问曰:“汝祖父是科下否?”曰:“不是。”又问:“家中富饶否?”曰:“无得。”神笑曰:“既是这等,你做甚么梦!”

谒孔庙
有以银钱夤缘入泮者,拜谒孔庙,孔子下席答之。士曰:“今日是夫子弟子礼,应坐受。”孔子曰:“岂敢。你是我孔方兄的弟子,断不受拜。”

狗头师
馆师岁暮买舟回家,舟子问曰:“相公贵庚?”答曰:“属狗的,开年已是五十岁了。”舟人曰:“我也属狗,为何贵贱不等?”又问:“那一月生的?”答曰:“正月。”舟子大悟曰:“是了,是了,怪不得!我十二月生,是个狗尾,所以摇了这一世。相公正月生,是个狗头,所以教(叫)了这一世。”

狗坐馆
一人惯会说谎,对亲家云:“舍间有三宝:一牛每日能行千里,一鸡每更止啼一声,又一狗善能读书。”亲家骇云:“有此异事,来日必要登堂求看。”其人归与妻述之,“一时说了谎,怎生回护?”妻曰:“不妨,我自有处。”次日,亲家来访,内云:“早上往北京去了。”问:“几时回?”答曰:“七八日就来的。”又问:“为何能快?”曰:“骑了自家牛去。”问:“宅上还有报更鸡?”适值亭中午鸡啼,即指曰:“只此便是,不但夜里报更,日间生客来也报的。”又问:“读书狗请借一观。”答曰:“不瞒亲家说,只为家寒,出外坐馆去了。”

讲书
一先生讲书,至“康子馈药”,徒问:“是煎药是丸药?”先生向主人夸奖曰:“非令郎美质不能问,非学生博学不能答。上节‘乡人傩”,傩的自然是丸药。下节又是煎药,不是用炉火,如何就‘厩焚’起来!”

师赞徒
馆师欲为固馆计,每赞学生聪明。东家不信,命当面对课。师曰:“蟹。”学生对曰:“伞。”师赞之不已。东翁不解,师曰:“我有隐意,蟹乃横行之物,令郎对‘伞’,有独立之意,岂不绝妙。”东翁又命对两字课,师曰:“割稻。”学生对曰:“行房。”师又赞不已。东家大怒,师曰:“此对也有隐意,我出‘割稻’者,乃积谷防饥。他对‘行房’者,乃养儿待老。”

请先生
一师惯谋人馆,被冥王访知·着夜叉拿来。师躲在门内不出·鬼卒设计哄骗曰:“你快出来·有一好馆请你。”师闻有馆,即便趋出,被夜叉擒住。先生曰:“看你这鬼头鬼脑,原不像个请先生的。”

骂先生
一人见稳婆姿色美,欲诱之,乃假妆妇人将产,请来收生,稳婆摸着此物。大惊曰:“我收生多年矣,有头先生者,名为顺生﹔脚先生者,名为倒生﹔手先生者,名为横生。这个鸡巴先生,实是不曾见过。”

没坐性
夫妻夜卧,妇握夫阳具曰:“是人皆有表号,独此物无一美称,可赠他一号。”夫曰:“假者名为角先生,则真者当去一角字,竟呼为先生可也。”妇曰:“既是先生,有馆在此,请他来坐。”云雨既毕。次早,妻以鸡子酒啖夫。夫笑曰:“我知你谢先生也,且问你先生何如?”妻曰:“先生尽好,只是嫌他略罢软,没坐性些。”

兄弟延师
有兄弟两人,共延一师,分班供给。每交班,必互嫌师瘦,怪供给之不丰。于是兄弟相约,师轮至日,即秤斤两,以为交班肥瘦之验。一日,弟将交师于兄,乃令师饱餐而去。既上秤,师偶撒一屁,乃咎之曰:“秤上买卖,岂可轻易撒出!说不得,原替我吃了下去。”

读破句
庸师惯读破句,又念白字。一日训徒,教《大学序》,念云:“大学之,书古之,大学所以教人之。”主人知觉,怒而逐之。复被一荫官延请入幕,官不识律令,每事询之馆师。一日,巡捕拿一盗钟者至,官问:“何以治之?”师曰:“夫子之道(盗)忠(钟),恕而已矣。”官遂释放。又一日,获一盗席者至,官又问,师曰:“朝闻道夕(席),死可矣。”官即将盗席者立毙杖下。适冥王私行,察访得实,即命鬼判拿来,痛骂曰:“不通的畜生!你骗人馆谷,误人子弟,其罪不小,摘往轮回去变猪狗。”师再三哀告曰:“做猪狗固不敢辞,但猪要判生南方,狗乞做一母狗。”王问何故,答曰:“南方之(猪),强与北方之。”又问:“母狗为何?”答曰:“《曲礼》云:‘临财母苟(狗)得,临难母苟免。’”

退束修
一师学浅,善读别字。主人恶之,与师约,每读一别字,除修一分。至岁终,退除将尽,止余银三分,封送之。师怒曰:“是何言兴,是何言兴(与)!”主人曰:“如今再扣二分,存银一分矣。”东家母在傍曰:“一年辛若,半除也罢。”先生近前作谢曰:“夫人不言,言必有中。”主人曰:“恰好连这一分,干净拿进去。”

赤壁赋
庸师惯读别字。一夜,与徒讲论前后《赤壁)两赋,竟念“赋”字为“贼”字。适有偷儿潜伺窗外,师乃朗诵大言曰:“这前面《赤(作拆字)壁贼》呀。”贼大惊,因思前而既觉,不若往房后穿逾而入。时已夜深,师讲完,往后房就寝。既上床,复与徒论及后面《赤壁赋》,亦如前读。偷儿在外叹息曰:“我前后行藏,悉被此人识破。人家请了这样先生,看家狗都不消养得了!”

于戏左读
有蒙训者,首教《大学》,至“于戏前王不忘”句,竟如字读之。主曰:“误矣,宜读作呜呼。”师从之。至冬间,读《论语》注“傩虽古礼而近于戏”,乃读作鸣呼。主人曰:“又误矣,此乃于戏也。”师大怒,诉其友曰:“这东家甚难理会,只‘于戏’两字,从年头直与我拗到年尾。”

中酒
一师设教,徒问:“大学之道如何讲?”师佯醉曰:“汝偏拣醉时来问我。”归与妻言之,妻曰:“‘大学’是书名,‘之道’是书中之道理。”师颔之。明日,谓其徒曰:“汝辈无知,昨日乘醉便来问我。今日我醒,偏不来问,何也?汝昨日所问何义?”对以“大学之道”。师如妻言释之。弟子又问:“‘在明明德’如何?”师遽捧额曰:“且住,我还中酒在此。”

教法
主人怪师不善教,师曰:“汝欲我与令郎俱死耶?”主人不解,师曰:“我教法已尽矣,只除非要我钻在令郎肚里去,我便闷杀,令郎便胀杀。”

浇其妻妾
人家请一馆师,书房逼近内室。一日课徒,读“譬如四时之错行”句,注曰:“错,犹迭也。”东家母听见,嗔其有意戏狎,诉于主人。主人不通书解,怒欲逐之。师曰:“书义如此,汝自不解耳,我何罪焉?”遂迁馆于厅楼,以避啰皂。一日,东家妻妾游于楼下,师欲小便不得,乃从壁间溺之。不意淋在妻妾头上,复诉于主人。主因思前次孟浪怪他,今番定须考证书中有何出典。乃左右翻释,忽大悟曰:“原来在此,不然,几被汝等所误矣。”问:“有何凭据?”主曰:“施施从外来,骄(浇)其妻妾。”

书生意气
主人问先生曰:“为何讲书再不明白?”师曰:“兄是相知的,我胸中若有不讲出来,天诛地灭!”又问:“既讲不出,也该坐定些?”答云:“只为家下不足,故不得不走。”主人云:“既如此,为甚供给略淡泊,就要见过?”先生毅然变色曰:“若这点意气没了,还像个先生哩!”

梦周公
一师昼寝,而不容学生磕睡。学生诘之,师谬言曰:“我乃梦周公也。”明昼,其徒亦效之,师以戒方击醒曰:“汝何得如此?”徒曰:“亦往见周公耳。”师曰:“周公何语?”答曰:“周公说,昨日并不曾见尊师。”

猫逐鼠
一猫捕鼠,鼠甚迫,无处躲避,急匿在竹轿杠中。猫顾之叹云:“看你管(馆)便进得好,这几个节如何过得去!”

问馆
乞儿制一新竹筒,众丐沽酒称贺。每饮毕,辄呼曰:“庆新管酒干。”一师正在觅馆,偶经过闻之,误听以为庆新馆也,急向前揖之曰:“列位既有了新馆,把这旧馆让与学生罢!”

闲荡
一女将下教场点兵,中军官以马肾伸长不雅,各将竹管一个,预套阳物于内。及女将至,一马跳跃,脱去竹筒,阳物翘然挂于腹下。女将究问,中军禀曰:“那件东西,凡有管的,都在管里。这个失了管(馆)的,所以在此闲荡。”

改对
训蒙先生出两字课与学生对曰:“马嘶。”一徒对曰:“鹏奋。”师曰:“好,不须改得。”徒揖而退。又一徒曰:“牛屎。”师叱曰:“狗屁!”徒亦揖而欲行,师止之曰:“你对也不曾对好,如何便走?”徒曰:“我对的是牛屎,先生改的是狗屁。”

挞徒
馆中二徒,一聪俊,一呆笨。师出夜课,适庭中栽有梅树,即指曰:“老梅。”一徒见盆内种柏,应声曰:“小柏。”师曰:“善。”又命一徒“可对好些”,徒曰:“阿爹。”师以其对得胡说,怒挞其首。徒哭曰:“他小柏(伯)不打,倒来打阿爹。”

蜈蚣咬
上江人出外坐馆,每兴举,辄以手铳代之,以竹筒盛接。其精日久气腥,为蜈蚣潜啖。一日,其兴复发,正作事,忽被蜈蚣箝住阳物,师恐甚。岁暮归家,摸着其妻阴户多毛,乃大声惊诧曰:“光光竹筒,尚有蜈蚣,蓬蓬松松,岂无蛇虫!”

我不如
一先生出外坐馆,离家日久,偶见狗练,叹曰:“我不如也。”

掘荷花
一师出外就馆,虑其妻与人私通,乃以妻之牝户上,画荷花一朵,以为记号。年终解馆归,验之已落,无复有痕迹矣。因大怒,欲责治之。妻曰:“汝自差了,是物可画,为何独拣了荷花?岂不晓得荷花下面有的是藕,那须来往的人,不管好歹,那个也来掘掘,这个也来掘掘,都被他们掘干净了,与我何干!”

灒粪
师在田间散步,见乡人挑粪灌菜。师讶曰:“菜是人吃的,如何泼此秽物在上?”乡人曰:“相公只会看书,不晓我农家的事。菜若不用粪浇,便成苦菜矣。”一日,东家以苦菜膳师,师问:“今日为何菜味甚苦?”馆僮曰:“因相公嫌龌龊,故将不浇粪的菜请相公。”师曰:“既如此,粪味可盐,拿些来待我灒灒吃罢。”

咬饼
一蒙师见徒手持一饼,戏之曰:“我咬个月湾与你看?”既咬一口,又曰:“我再咬个定胜与你看?”徒不舍,乃以手掩之,误咬其指。乃呵曰:“没事,没事,今日不要你念书了。家中若问你,只说是狗夺饼吃,咬伤的。”

想船家
教书先生解馆归,妻偶谈及“喷啑鼻子痒,有人背地讲”。夫曰:“我在学堂内,也常常打啑的。”妻曰:“就是我在家想你了。”及开年,仍赴东家馆。别妻登舟,船家被初出太阳搐鼻,连打数啑。师频足曰:“不好了,我才出得门,这婆娘就在那里看想船家了!”

叔叔
师向主人极口赞扬其子沉潜聪慧,识字通透,堪为令郎伴读。主曰:“甚好。”师归谓其子曰:“明岁带你就学,我已在东翁前夸奖,只是你秉性痴呆,一字不识。”因写“被”、‘饭”、“父”三字,令其熟记,以备问对。及到馆后,主人连试数字,无一知者。师曰:“小儿怕生,待我写来,自然会识。”随写“被”字问之,子竟茫然。师曰:“你床上盖的是甚么?”答曰:“草荐。”师又写“饭”字与认,亦不答。曰:“你家中吃的是甚么?”曰:“麦粞。”又写“父”字与识,子曰:“不知。”师忿怒曰:“你娘在家,同何人睡的?”答曰:“叔叔。”

是我
一师值清明放学,率徒郊外踏青。师在前行,偶撒一屁,徒曰:“先生,清明鬼叫了。”先生曰:“放狗屁!”少顷,大雨倾盆,田间一瓦,为水淹没,仅露其背。徒又指谓先生曰:“这像是个乌龟。”师曰:“是瓦(我)。”

问藕
上路先生携子出外,吃着鲜藕,乃问父曰:“爹,来个沙东西,竖搭起竟似烟囱,横搭着好像泥笼,捏搭手里似把湾弓,嚼搭口里醒松醒松,已介甜水浓浓,咽搭落去蜘蛛丝绊住子喉咙,从来勿曾见过?”其父怒曰:“呆奴,呆奴!个就是南货店里包东包西的大(土)叶个根结么。”

卵脟皮
一师挈子赴馆,至中途,见卖汤圆者,指问其父曰:“爹,此是何物?”父怒其不争气,回曰:“卵子。”及到馆,主家设酒款待,菜中有用腐皮做浇头者。子拍掌大笑曰:“他家卵子,竟不值得拿来请人,好笑一派都用着卵脟皮了。”

屎在口头
学生问先生曰:“屎字如何写?”师一时忘却,不能回答,沉吟片晌曰:“咦,方才在口头,如何再说不出。”

村牛
一士善于联句,偶同友人闲步,见有病马二匹卧于城下。友即指而问曰:“闻兄捷才,素善作对,今日欲面领教。”士曰:“愿闻。”友出题曰:“城北两只病马。”士即对曰:“江南一个村牛。”

瘟牛
经学先生出一课与学生对曰:“隔河并马。”学生误认“并”字为“病”字,即应声曰:“过江瘟牛。”

善对
有游湖者,见岸上有儿马厥物伸出,因同行中一友善对。乃出对曰:“游湖客偶睹马屌。”友即回对曰:“过江人惯肏牛屄。”

个人个妻
一上路先生向人问:“原来吴下朋友的老妈官,个人是一个哥喇。”

歪诗
一士好做歪诗。偶到一寺前,见山门上塑赵玄坛喝虎像,士即诗兴勃发,遂吟曰:“玄坛菩萨怒,脚下踏个虎(座)。傍立一判官,嘴上一脸垩。”及到里面,见殿宇巍峨,随又续题曰:“宝殿雄哉大(度),大佛归中坐。文殊骑狮子,普贤骑白兔。”僧出见曰:“相公诗才敏妙,但韵脚欠妥。小僧回奉一首何如?”士曰:“甚好。”僧念曰:“出在山门路,撞着一瓶醋。诗又不成诗,只当放个破(破声,屁也)。”

歇后诗
一采桑妇,姿色美丽,遇一狂士调之,问:“娘子尊姓?”女曰:“姓徐。”士作诗一首戏之曰:“娘子尊姓徐,桑篮手内携。一阵狂风起,吹见那张”,下韵“屄”,因字义村俗,故作歇后语也。女知被嘲,还问:“官人尊姓?”答曰:“小生姓陆。”女亦回嘲云:“官人本姓陆,诗书不肯读。令正在家里,好与别人”,下“笃”字,亦作缩脚韵。士听之,乃大怒,交相讼之于官。值官升任,将要谢事,当堂作诗以绝之曰:“我今任已满,闲事都不管。两造俱赶出,不要咬我”,缩下“卵”字。

咏钟诗
有四人自负能诗。一日,同游寺中,见殿角悬钟一口,各人诗兴勃然,遂联句一首。其一曰:“寺里一口钟。”次韵云:“本质原是铜。”三曰:“覆转像只碗。”四曰:“敲来嗡嗡嗡。”吟毕,互相赞美不置口,以为诗才敏捷,无出其右。“但天地造化之气,已泄尽无遗,定夺我辈寿算矣。”四人懮疑,相聚环泣。忽有老人自外至,询问何事,众告以故。老者曰:“寿数固无碍,但各要患病四十九日。”众问何病,答曰:“了膀骨痛!”

老童生
老虎出山而回,呼肚饥。群虎曰:“今日固不遇一人乎?”对曰:“遇而不食。”问其故,曰:“始遇一和尚,因臊气不食。次遇一秀才,因酸气不食。最后一童生来,亦不曾食。”问:“童生何以不食?”曰:“怕咬伤了牙齿。”

认拐杖
县官考童生,至晚忽闻鼓角喧闹。问之,门子禀曰:“童生拿差了拐杖,在那里争认。”

拔须
童生拔须赶考,对镜恨曰:“你一日不放我进去,我一日不放你出来!”

未冠
童生有老而未冠者,试官问之,以“孤寒无网”对。官曰:“只你嘴上胡须剃下来,亦勾结网矣。”对曰:“童生也想要如此,只是新冠是桩喜事,不好带得白网巾。」

\part{}

术业部

医官
医人买得医官札付者,冠带而坐于店中。过者骇曰:“此何店,而有官在内?”傍人答曰:“此医官之店。”

冥王访名医
冥王遣鬼卒访阳间名医,命之曰:“门前无冤鬼者即是。”鬼卒领旨,来到阳世,每过医门,冤鬼毕集。最后至一家,见门首独鬼彷徨,曰:“此可以当名医矣。”问之,乃昨日新竖药牌者。

拾柩
一医生医死人,主家愤甚,呼群仆毒打。医跪求至再,主曰:“私打可免,官法难饶。”即命送官惩治。医畏罪,哀告曰:“愿雇人抬往殡殓。”主人许之。医苦家贫,无力雇募,家有二子,夫妻四人共来抬柩。至中途,医生叹曰:“为人切莫学行医。”妻咎夫曰:“为你行医害老妻。”幼子云:“头重脚轻抬不起。”长子曰:“爹爹,以后医人拣瘦的。”

医人
有送医士出门,犬适拦门而吠,主人喝之即止。医赞其能解人意,主曰:“虽则畜生,倒也还会依(医)人。”

好郎中
一人向医家买春药吃了。行至半路,药性发作,此物翘然直竖。乃以手捧住赞曰:“好郎中,好郎中,好郎中!”

谢郎中
有害赤眼者,百方治之不效。或教以用尿除头去尾,抹之即好,如言用后果愈。一日小便,手握阳具而言曰:“亏你医好我眼,欲折顶巾你戴,你头忽大忽小,做件衣你穿,你身时长时短。”人问为何自言自语,答曰:“我在此打点谢郎中。”

哭郎中
一人有一妻二妾,死后,妻妾绕尸而哭。妻抚其首,曰:“我的郎头呀!”次捏其足,曰:“我的郎脚呀!”又次者无可哭附,只得握其阳物曰:“我的郎中呀!”

屪子郎中
一士人往花园游玩,见篱边蔷蔽甚开,娇媚可人。近前攀折,被蔷薇刺破手指,出血不止。偶遇一牧童,言曰:“血不止,可将热尿淋之即好。”士依其言,血果即止。遂作口号以赞之曰:“今朝散步入园中,窥见蔷薇满树红。双手摘时遭一刺,血流不止手鲜红。牧童传把热尿淋,果然灭迹就无踪。莫道人间无妙药,屪子也会做郎中。”

迷妇药
一方士专卖迷妇人药,妇着在身,自来与人私合。一日,有轻浪子弟来买药,适方士他出,其妻取药付之。子弟就以药弹其身上。随妇至房,妇只得与伊交合。方士归,妻以其事告之。方士怒云:“谁教你就他?”妻曰:“我若不从,显得你的药便不灵了。”

跳蚤药
一人卖跳蚤药,招牌上写出“卖上好蚤药”。问:“何以用法?”答曰:“捉住跳蚤,以药涂其嘴,即死矣。”

医乳
人家请医看乳癖,医将好奶玩弄不已。主骇问何意,答曰:“我在此仔细斟酌,必要医得与他一样纔好。”

医屁
一人患病,医家看脉云:“吃了药,腹中定响,当走大便,不然,定撒些屁。”少顷,坐中忽闻屁声,医曰:“如何?”客应云:“是小弟撒的。”医曰:“也好。”

医按院
一按台患病,接医诊视,医惊持畏缩,错看了手背。按院大怒,责而逐之。医曰:“你打便打得好,只是你脉息俱无了。”

愿脚踢
樵夫担柴,误触医士。医怒,欲挥拳。樵夫曰:“宁受脚踢,勿动尊手。”傍人误之,樵者曰:“脚踢未必就死,经了他手,定然难活。”

锯箭竿
一人往观武场,飞箭误中其身,迎外科治之。医曰:“易事耳。”遂用小锯截其外竿,即索谢辞去。问:“内截如何?”答曰:“此是内科的事。”

怨算命
或见医者,问以生意何如,答曰:“不要说起,都被算命先生误了,嘱我有病人家不要去走。”

包殡殓
有医死人儿,许以袖归殡殓,其家恐见欺,命仆随之。至一桥上,忽取儿尸掷之河内。仆怒曰:“如何抛了我家小舍?”医曰:“非也。”因举左袖曰:“你家的在这里。”

屄打弹
一尼欲心甚炽,以萝卜代阳,大肆抽送,畅所欲为。不料用力太猛,折其半截在内。挖之不出,渐至肿胀。延医看视,医将两手阴傍按捺,良久突出,刚打在医人脸上。医者叹曰:“我也医千医万,从未见屄会打弹。”

送药
一医迁居,谓四邻曰:“向来打搅,无物可做别敬,每位奉药一帖。”邻舍辞以无病,医曰:“但吃了我的药,自然会生起病来。”

补药
一医止宿病家,夜半屎急不便,乃出于一箱格中,闭之。晨起,主人请用药,偶欲抽视此格,医坚执不许。主人问:“是何药?”答曰:“我自吃的补药在内。”

药户
一乡人与城里人同行,见一妓女,乡人问:“是谁家宅眷?”城里人曰:“此药户也。”乡人曰:“原来就是开药店的家婆。”

屄样
有生平未近女色者,不知阴物是何样范。向人问之,人曰:“就像一只眼睛竖起便是。”此人牢记在心。一日,嫖兴忽发,不知妓馆何在,遂向街头闲撞。见一眼科招牌,上画眼样数只,偶然横放,以为此必妓家也。进内道其来意,医士大怒,叱而逐之。其人曰:“既不是妓馆,为何摆这许多屄样在外面。”

取名
有贩卖药材者,离家数载,其妻已生下四子。一日夫归,问众子何来,妻曰:“为你出外多年,我朝暮思君,结想成胎,故命名俱暗藏深意:长是你乍离家室,宿舟沙畔,故名宿砂﹔次是你远乡作客,我在家志念,故名远志﹔三是料你置货完备,合当归家,故唤当归﹔四是连年盼你不到,今该返回故乡,故唤茴香。”夫闻之,大笑曰:“依你这等说来,我再在外几年,家里竟开得一片山药铺了。”

索谢
一贫士患腹泻,请医调治,谓医曰:“家贫不能馈药金,医好之日,奉请一醉。”医从之。服药而愈,恐医索谢,诈言腹泻未止。一日,医者伺其大便,随往验之。见撒出者俱是干粪,因怒指而示之曰:“撤了这样好粪,如何还不请我?”

包活
一医药死人儿,主家诟之曰:“汝好好殡殓我儿罢了,否则讼之于官。”医许以带归处置,因匿儿于药箱中。中途又遇一家邀去,启箱用药,误露儿尸。主家惊问,对曰:“这是别人医杀了,我带去包活的。”

退热
有小儿患身热,请医服药而死,父请医家咎之。医不信,自往验视,抚儿尸谓其父曰:“你太欺心,不过要我与他退热,今身上幸已冰凉的了,倒反来责备我。”

疆蚕
一医久无生理,忽有求药者至,开箱取药,中多蛀虫。人问:“此是何物?”曰:“疆蚕。”又问:“疆蚕如何是活的?”答曰:“吃了我的药,怕他不活?”

看脉
有医坏人者,罚牵麦十担。牵毕,放归。次日,有叩门者曰:“请先生看脉。”医应曰:“晓得了。你先去淘净在那里,我就来牵也。”

医女接客
医生、妓女、偷儿三人,死见冥王,王问生前技术。医士曰:“小人行医,人有疾病,能起死回生。”王怒曰:“我每常差鬼卒勾提罪人,你反与我把持抗衡,可发往油锅受罪。”次问妓女,妓曰:“接客。人没妻室者,与他解渴应急。”王曰:“方便孤身,延寿一纪。”再问偷儿,答曰:“做贼。人家晒浪衣服,散放银钱,我去替他收拾些。”王曰:“与人分劳代力,也加寿十年,发转阳世。”医士急忙哀告曰:“大王若如此判断,只求放我还阳。家中尚有一子一女,子叫他去做贼,女就叫他接客便了。”

大方打幼科
大方脉采住小儿科痛打,傍人劝曰:“你两个同道中,何苦如此。”大方脉曰:“列位有所不知,这厮可恶得紧。我医的大人俱变成孩子与他医,谁想他医的孩子,一个也不放大来与我医。”

幼科
富家延二医,一大方,一幼科。客至,问:“二位何人?”主人曰:“皆名医。”又问:“那一科?”主人曰:“这是大方,这个便是小儿。”

小儿窠
小儿科之妻,乃大方脉之女,每每互相讥诮。一夜行房,妇执阳物问夫曰:“此是何物?”夫曰:“大方脉。”夫亦指牝户问,妇曰:“这是小儿窠。”

小犬窠
有人畜一金丝小犬,爱同珍宝,恐其天寒冻坏,内外各用小棉褥铺成一窠,使其好睡。不意此犬一日竟卧于儿篮内,主人见之,大笑曰:“这畜生好作怪,既不走内窠,又不往外窠,倒钻进小儿窠(科)里去了。”

骂
一医看病,许以无事。病家费去多金,竟不起,因恨甚,遣仆往骂。少顷归,问:“曾骂否?”曰:“不曾。”问:“何以不骂?”仆答曰:“要骂要打的人,多得紧在那里,叫我如何挨挤得上?”

赔
一医医死人儿,主家欲举讼,愿以己子赔之。一日,医死人仆,家止一仆,又以赔之。夜间又有叩门者云:“娘娘产里病,烦看。”医私谓其妻曰:“淘气!那家想必又看中意你了。”

吃白药
有终日吃药而不谢医者,医甚憾之。一日,此人问医曰:“猫生病,吃甚药?”曰:“吃乌药。”“然则狗生病,吃何药?”曰:“吃白药。”

游水
一医生医坏人,为彼家所缚,夜半逃脱,赴水遁归。见其子方读《脉诀》,遽谓曰:“我儿读书尚缓,还是学游水要紧。”

地师
一风水新婚初夜,子摸着新人鼻梁曰:“此是发龙之所。”又摸其两乳曰:“喜得龙虎俱全。”再摸至肚上曰:“好一块平沙。”摸至腰下曰:“好个金井护穴。”及上妻身,问:“汝来何事?”地师曰:“阴地皆由做成,我把罗星来塞水口。”其父隔壁听见,放声大笑曰:“既有这等好穴,何不将我老骨头埋在里面,荫些好子孙出来。”

风水
一风水父子同室。其子与媳欲合,乃从头摸起曰:“密密层层一座山。”至乳则曰:“两峰高耸实非凡。”至肚则曰:“中间好块平阳地。”至阴户则曰:“正穴原来在此间。”父听见,乃高叫曰:“我儿有如此好地,千万留来把我先埋葬在里面。”

阴阳先生
昔一人患膀胱偏坠之症,请医调治。医曰:“外肾左边属阳,右边属阴,今偏于一边,却是阴阳不和之故耳。”其人问曰:“既是左属阳,右属阴,不知中间危坐者唤作何名?”医笑曰:“此是看阴阳的先生。”

阴阳生
从来人堕水淹死,飘浮水面,覆者是男,仰者是女。一日,有尸从河内侧身汆来者。人见之,皆道:“奇怪!若是女,一定仰,而男则覆转。今此人侧起,男女未知孰是。”傍一人曰:“此必是个阴阳生耳。”

法家
无赖子怒一富翁,思所以倾其家而不得。闻有茅山道士法力最高,往诉恳之。道士曰:“我使天兵阴诛此翁。”答:“其子孙仍富,吾不甘也。”曰:“然则,吾纵天火焚其室庐。”答曰:“其田土犹存,吾不甘也。”道士曰:“汝仇深至此乎!吾有一至宝,赐汝持去,朝夕供奉拜求,彼家自然立耗矣。”其人喜甚,请而观之。封缄甚密,启视,则纸做成笔一枝也。问:“此物有何神通?”道士曰:“你不知我法家作用耳。这纸笔上,不知破了多少人家矣。”

相相
有善相者,扯一人要相。其人曰:“我倒相着你了。”相者笑云:“你相我何如?”答曰:“我相你决是相不着的。”

卜孕
一人善卜,又喜诙谐。有以孕之男女来问者,卜讫,拱手恭喜曰:“是个夹卵的。”其人喜甚,谓为男孕无疑矣。及产,却是一女,因往咎之。卜者曰:“维男有卵,维女夹之。有夹卵之物者,非女子而何?”

不着
街市失火,延烧百余户。有星相二家欲移物以避,旁人止之曰:“汝两家包管不着,空费搬移。”星相曰:“火已到矣,如何说这太平话?”曰:“你们从来是不着的,难道今日反会着起来!”

写真
有写真者,绝无生意。或劝他将自己夫妻画一幅行乐贴出,人见方知。画者乃依计而行。一日,丈人来望,因问:“此女是谁?”答云:“就是令爱。”又问:“他为甚与这面生人同坐?”

胡须像
一画士写真既就,谓主人曰:“请执途人而问之,试看肖否?”主人从之,初见一人问曰:“那一处最像?”其人曰:“方巾最像。”次见一人,又问曰:“那一处最像?”其人曰:“衣服最像。”及见第三人,画士嘱之曰:“方巾、衣服都有人说过,不劳再讲,只问形体何如?”其人踌躇半晌,曰:“胡须最像。”

讳输棋
有自负棋高,与人角,连负三局。次日,人问之曰:“昨日较棋几局?”答曰:“三局。”又问:“胜负何如?”曰:“第一局我不曾赢,第二局他不曾输,第三局我本等要和,他不肯罢了。”

好棋
一人以好棋破产,因而为小偷,被人缚住。有相识者,见而问之,答云:“彼请我下棋,嗔我棋好,遂相困耳。”客曰:“岂有此理?”其人答曰:“从来棋高一着,缚手缚脚。”

银匠偷
一人生子,虑其难养,请一星家算命。星士曰:“关煞倒也没得,大来运限俱好,只是四柱中犯点贼星,不成正局。”那人曰:“不妨,只要养得大,就叫他学做银匠。”星士曰:“为何?”答曰:“做了银匠,那日不偷几分养家活口。”

利心重
银匠开铺三日,绝无一人进门。至暮,有以碎银二钱来倾者,乃落其半,倾作对充与之。其人大怒,谓其利心太重。银匠曰:“天下人的利心,再没有轻过如我的。开了三日店,止落得一钱,难道自己吃了饭,三分一日,你就不要还了?”

有进益
一翁有三婿,长裁缝,次银匠,惟第三者不学手艺,终日闲游。翁责之曰:“做裁缝的,要落几尺就是几尺。做银匠的,要落几钱就是几钱。独汝游手好闲,有何结局?”三婿曰:“不妨。待我打一把铁窍,窍开人家库门,要取论千论百,也是易事,稀罕他几尺几钱!”翁曰:“这等说,竟是贼了。”婿曰:“他们两个,整日落人家东西,难道不是贼?”

裁缝
时年大旱,太守命法官祈雨。雨不至,太守怒,欲治之,法官禀云:“小道本事平常,不如其裁缝最好。”太守曰:“何以见得?”答曰:“他要落几尺就是几尺。”

不下剪
缝匠裁衣,反复量,久不肯下剪。徒弟问其故,答曰:“有了他的,便没有了我的。有了我的,又没有了他的。”

要尺
一裁缝上厕坑,以尺挥墙上,便完忘记而去。随有一满洲人登厕,偶见尺,将腰刀挂在上面。少顷,裁缝转来取尺,见有满人,畏而不前,观望良久。满人曰:“蛮子,你要甚么?”答曰:“小的要尺。”满人曰:“咱囚攮的,屙也没有屙完,你就要吃(尺)!”

木匠
一匠人装门闩,误装门外,主人骂为“瞎贼”。匠答曰:“你便瞎贼!”主怒曰,“我如何倒瞎?”匠曰:“你若有眼,便不来请我这样匠人。”

含毛
一人破家与一妓相处数年,临别,妓女赠得阴毛数根,珍藏帽中,时为把玩。一日忽失去,遍寻不得。偶踱至街头,遇一皮匠口含猪鬃缝鞋,其人骂而夺之曰:“我用尽银钱,只落得这两根毛,如何偷来倒插在你口里面?”

待诏
一待诏初学剃头,每刀伤一处,则以一指掩之。已而伤多,不胜其掩,乃曰:“原来剃头甚难,须得千手观音来才好。”

蓖头
蓖头者被贼偷窃。次日,至主顾家做生活,主人见其戚容,问其故。答曰:“一生辛苦所积,昨夜被盗。仔细想来,只当替贼蓖了一世头耳。”主人怒而逐之。他日另换一人,问曰:“某人原是府上主顾,如何不用?”主人为述前言,其人曰:“这样不会讲话的,只好出来弄卵。”

头嫩
一待诏替人剃头,才举手,便所伤甚多。乃停刀辞主人曰:“此头尚嫩,下不得刀。且过几时,姑俟其老再剃罢。”

取耳
一待诏为人看耳,其人痛极,问曰:“左耳还取否?”曰:“方完,次及左矣。”其人曰:“我只道就是这样取过去了。”

同行
有善刻图书者,偶于市中唤人修脚。脚已脱矣,修者正欲举刀,见彼袖中取出一袱,内裹图书刀数把。修者不知,以为剔脚刀也,遂绝然而去。追问其故。则曰:“同行中朋友,也来戏弄我。”

偷肉
厨子往一富家治酒,窃肉一大块,藏于帽内。适为主人窥见,有意作耍他拜揖,好使帽内肉跌下地来。乃曰:“厨司务,劳动你,我作揖奉谢。”厨子亦知主人已觉,恐跌出不好看相,急跪下曰:“相公若拜揖,小人竟下跪。”

船家
一人睡倒,戏语人曰:“我好像一只船,头似船头,脚似船尾,肚腹似船舱。”又指阳物曰:“这个岂不像撑船的?”人曰:“那里有这等垂头丧气的家长。”答曰:“你不晓得,摇船的时节,从来是软腊塔的,一到讨船钱时,便硬挣得不象样了。”

稍公
稍公死,阎王判他变作阴户。稍公不服,曰:“是物皆可做,为何独变阴物?”阎王曰:“单取你开也会开,摆也会摆,又善摇,又善摆。”

水手
船家与妻同睡,夫摸着其妻阴户,问曰:“此是何物?”妻曰:“是船舱。”妻亦握夫阳具,问是何物,答曰:“客货。”妻曰:“既有客货,何不装入舱里来?”夫遂与云雨,而两卵在外。妻以手摸曰:“索性一并装入也罢。”夫曰:“这两个是水手,要在后面看舵的。”

卖淡酒
一家做酒﹔颇卖不去,以为家有耗神。请一先生烧椿退送,口念曰:“先除鹭鸶,后去青鸾。”主人曰:“此二鸟你退送他怎的?”先生曰:“你不知,都吃亏这两只禽鸟会下水,遣退了他,包你就卖得去!”

三名斩
朝廷新开一例,凡物有两名者充军,三名者斩。茄子自觉双名,躲在水中。水问曰:“你来为何?”茄曰:“避朝廷新例。因说我有两名,一名茄子,一名落苏。”水曰:“若是这等,我该斩了:一名水,二名汤,又有那天灾人祸的放了几粒米,把我来当酒卖。”

酒娘
人问:“何为叫做酒娘?”答曰:“糯米加酒药成浆便是。”又问:“既有酒娘,为甚没有酒爷?”答曰:“放水下去,就是酒爷。”其人曰:“若如此说,你家的酒,是爷多娘少的了。”

走作
一店中酿方熟,适有带巾者过,揖入使尝之。尝毕曰:“竟有些像我。”店主知其秀才也,谢去之。少焉,一女子过,又使尝之,女子亦曰:“像我。”店主曰:“方才秀才官人说‘像我’,是酸意了,你也说‘像我’,此是为何?”女子曰:“无他,只是有些走作。”

着醋
有卖酸酒者,客上店谓主人曰:“肴只腐菜足矣,酒须要好的。”少顷,店主问曰:“菜中可要着醋?”客曰:“醋滴菜心甚好。”又问曰:“腐内可要放些醋?”客曰:“醋烹豆腐也好。”再问曰:“酒内可要着醋否?”客讶曰:“酒中如何着得醋?”店主攒眉曰:“怎么处?已着下去了。”

酸酒
一酒家招牌上写:“酒每斤八厘,醋每斤一分。”两人入店沽酒,而酒甚酸。一人咂舌攒眉曰:“如何有此酸酒,莫不把醋错拿了来?”友人忙捏其腿曰:“呆子,快莫做声,你看牌面上写着醋比酒更贵着哩!”

炙坛
有以酸酒饮客者,个个攒眉,委吞不下。一人嘲之曰:“此酒我有易他良法,使他不酸。”主人曰:“请教。”客曰:“只将酒坛覆转向天,底上用艾火连炙七次,明日拿起,自然不酸。”主曰:“岂不倾去漏干了?”客曰:“这等酸酒,不倾去要他做甚!」

\part{}

形体部

嘲胡卖契
胡子家贫揭债,特把髭须质戤。只因无计谋生,情愿央中借贷,上连鼻孔、人中,下至喉咙为界,计开四址分明,两鬓篷松在外,根根真正胡须,并无阴毛杂带。若还过期不赎,听作猪综变卖。年分月日开填,居间借重卵袋。

呵冻笔
一人见春意一册,曰:“此非春画,乃夏画也。不然,何以赤身露体?”又一人曰:“亦非夏画,乃冬画也。”问曰:“何故?”答曰:“你不见每幅上,个个胡子在那里呵冻笔。”

揪肾毛
一人对胡子曰:“我昨晚梦见你做了官,旗伞执事,吆喝齐声,好不威阔。”胡子大喜。其人又云:“我梦里骂了你、你就呼皂隶来打我,被我将你胡须一把揪住。”胡子云:“骂了官长,自然该打。后来毕竟如何?”其人曰:“也就醒了,醒来一只手还揪住一把卵毛,紧紧不放。”

观相
一相士苦无生意,拉住人相。那人曰:“不要相。”相者强之再三,只得解裤出具,谓曰:“此物倒求一观。”相者端视良久,乃作赞词云:“看你生在一脐之下,长于两膀之间,软柔柔而向东向西,硬棚棚而矗上矗下,遇妻妾而无礼,应子孙而有功。一生梗直,两子送终。日后还有二十年好运。”问他有何好处,曰:“生得一脸好胡须。”

愁穷
有胡子愁穷,一友谑之曰:“据兄家事,不下二千金,何以过愁若此?”胡者曰:“二千金何在?”友曰:“兄面上现有千七百了,难道令正处便没有须私房?”

胡瘌杀
或看审囚回,人问之,答曰:“今年重囚五人,俱有色认:一痴子,一颠子,一瞎子,一胡子,一瘌痢。”问如何审了,答曰:“只胡子与瘌痢吃亏,其余免死。”又问何故,曰:“只听见问官说痴弗杀,颠弗杀,一眼弗杀,胡子搭瘌杀。”

直缝横缝
北方极寒之地,一妇倚墙撒尿,溺未完而尿已冻,连阴毛结于石上。呼其夫至,以口呵之。夫近视而胡者也,呵之不化,连气亦结成冰,须毛互冻而不解。乃命家僮凿开,吩咐曰:“看仔细子下凿,连着直缝的是毛,连着横缝的是须。”

被剃
贫妇裸体而卧,偷儿入其家,绝无一物可取。因思贼无空讨,见其阴户多毛,遂剃之而去。妇醒大骇,以告其夫。夫大叫曰:“世情这等恶薄,家中的毛尚且剃了去,以后连腮胡子竟在街上走不得了!”

抛猫
道士、和尚、胡子三人过江,忽遇狂风大作,舟将颠覆。僧、道慌甚,急把经卷掠入江中,求神救护。而胡子无可掷得,惟将胡须逐根拔下,投于江内。僧、道问曰:“你拔胡须何用?”其人曰:“我在此抛毛(锚)。”

胡子改屄
裁缝、皮匠、妓女三人,同席行令,各要道本行四句,贯串叶韵。缝匠曰:“失去一背挂,拾得一披风。改了一背挂,落下两袖桶。”皮匠曰:“失去一双鞋,拾得一双靴。改了一双鞋,落下两桶皮。”妓者曰:“失去一张屄,拾得一胡子。改了一张屄,落他一口齿。”

不斟酒
一家宴客,坐中一大胡子,酒僮畏缩不前,杯中空如也。主举杯朝拱数次,胡子愠曰:“安得有酒?”主骂僮为何不斟,僮曰:“这位相公没有嘴的。”胡子忿极,揭须以示,曰:这不是嘴,还是你娘的屄不成?”

吃白面
一僧人、一经纪、一妓女同途,陡遇大雪,遂往古庙避之。三人议曰:“今日我等在此,各将大雪为题,要插入自家本色。”和尚曰:“片片片,碎剪鹅毛空中旋。落在我山门上,好似一座白玉殿。”经纪曰:“片片片,碎剪鹅毛空中旋。落在我匾担上,好似一把白玉剑。”妓女曰:“片片片,碎剪鹅毛空中旋。落在我屄毛上,好似胡子吃白面。”

通谱
有一人须长过腹,人见之,无不赞为美髯。偶一日,遇见风鉴先生,请他一相。相者曰:“可惜尊髯短了些。”其人曰:“我之须已过腹,人尽赞羡,为何反嫌其短?”相者曰:“若再长得寸许,便好与下边通谱了。”

联宗
胡须与眉毛曰:“当今世情浇薄,必要帮手相助,我已与鬓毛连矣。看来眼前高贵,惟二位我们俱在头面,联了甚好。”眉曰:“承不弃微末,但我根基浅薄,何不往下路孔家前门,一带茂林,旗杆底下,联的更好。”

一般胡
两人聚论:“《论语》一书,皆讲胡子。开章就说:‘不亦悦乎’,‘不亦乐乎’,‘不亦君子乎’,这三个都是好胡﹔‘为人谋而不忠乎’,‘与朋友交而不信乎’,‘传不习乎’,这三个是不好胡﹔‘君子者乎’,‘色壮者乎’,这两个胡一好一不好。”或问:“使乎,使乎。”答曰:“上面的胡与下面的胡,总是一般。”

稀胡子
一稀胡子要相面,相士云:“尊相虽不大富,亦不至贫。”胡者云:“何以见得?”相士曰:“看公之须,比上不足,比下有余。”

出须药
一光脸自觉无须,非丈夫气,持银往医肆,求买出须药。适医生他出,医妻忽传一方云:“可将尿脬一个打气,每日放嘴边滚撞,自然就长出来。”医归,问出何典,妻曰:“医者,意也。我前日初嫁你时,一根也没得,被你的脬撞过不多几时,即长出恁一脸胡须来。”

问有猫
一妇患病,卧于楼上,延医治之。医适买鱼归,途遇邀之而去,遂置鱼于楼下。登楼诊脉,忽想起楼下之鱼,恐被猫儿偷食,因问:“下面有猫(毛)否?”母在傍曰:“我儿要病好,先生问你,可老实说了罢。”妇答曰:“多是不多,略略有几根儿。”

骂须少
胡子行路,一孩戏之曰:“胡子迎风走,只见胡子不见口。”胡子忿甚﹔揭须露口,指而骂曰:“这不是口,倒是你娘的屄不成!”小儿被骂,归而哭诉于母。母慰之曰:“我儿,他骂别人,不是骂你。你娘的此物上,却不多几根,随他骂去罢。

胡答嘲
颜回、子路、伯鱼三人私议曰:“夫子惟胡,故开口不脱‘乎’字。”颜子曰:“他对我说:‘回也,其庶乎。’”子路曰:“他对我说:‘由也,诲汝知之乎?”伯鱼曰:“我家尊对我也说:‘汝为周南、召南矣乎。’”孔子在屏后闻之,出责伯鱼曰:“回是个短命,由是个不得其死的,说我胡也罢了。你是我的儿子,如何也来说我老子?”

光屁股
有上司面胡者,与光脸属吏同饭。上台须间偶带米糁,门子跪下禀曰:“老爷龙须上一颗明珠。”官乃拂去。属吏回衙,责备门子:“你看上台门子何等伶俐!汝辈愚蠢,不堪重用。”一日,两官又聚会吃面,属吏方举箸动口,有未缩进之面挂在唇角。门子急跪下曰:“小的禀事。”问禀何事,答曰:“爷好张光净屁股,多了一条蛔虫挂在外面。”

亲爷
有妻甫受孕而夫出外经商者,一去十载,子已年长,不曾识面。及父归家,突入妻房,其子骤见,乃大喊曰:“一个面生胡子,大胆闯入母亲房里来了!”其母曰:“我儿勿做声,这胡子正是你的亲爷。”

无须狗
一税官瞽目者,恐人骗他,凡货船过关,必要逐一摸验,方得放心。一日,有贩羊者至,规例羊有税,狗无税,尽将羊角锯去,充狗过关。官用手摸着项下胡须,乃大怒曰:“这些奴才,明来骗我。明明是一船羊,狗是何曾出须的!”

没须屁股
一公领孙溪中洗澡,孙拿得一虾,或前跳,或却走。孙问公曰:“前赶后退,后赶前行,不知何处是头,何处是尾?”公答曰:“有须的是头,没须的是屁股。”

拔须去黑
一翁须白,令姬妾拔之。妾见白者甚多,拔之将不胜其拔,乃将黑者尽去。拔讫,翁引镜自照,遂大骇,因咎其妾。妾曰:“难道少的倒不拔,倒去拔多的?”

白须
老妓年近六旬,尚倚门接客。一人打钉,见其阴毛斑白,谓曰:“该用乌须药了。”妓问:“染药宜在何时?”答曰:“搽了过夜。”妓摇首曰:“老实对你说,没有这一夜闲工夫,由他白去罢了。”

黄须
一人须黄,每于妻前自夸:“黄须无弱汉,一生不受人欺。”一日出外,被殴而归,妻引前言笑之。答曰:“那晓得那人的须,竟是通红的。”

老面皮
或问:“世间何物最硬?”曰:“石头与钢铁。”其人曰:“石可碎,铁可錾,安得为硬?以弟看来,惟兄面上髭须最硬,铁石总不如也。”问其故,答曰:“看老兄这副厚脸皮,竟被他钻门了出来。”那有须者回嘲曰:“足下面皮更老,这等硬须还钻不透!”

胖子行房
夫妇两人身躯肥胖,每行房,辄被肚皮碍事,不能畅意。一娃子云:“我倒传你个法儿,须从屁股后面弄进去甚好。”夫妇依他,果然快极。次日,见娃子问曰:“你昨教我的法儿,是那里学来的?”答曰:“我不是学别人的,常见公狗、母狗是那般干。”

皂隶干法
一官夫妇体肥,每次行房,两下肚皮碍住,从无畅举时节。一日,官正坐堂,见一皂隶伟胖异常,料其交感必有良法。审事毕,唤至后堂询曰:“汝腹甚大,行房时用何法,而能使两物凑合,不为肚腹所碍乎?”隶曰:“小的每到交合之际,命妻子斜坐一大椅上,将两足架开,自己站起行事,彼此紧凑,便无阻隔之患。”官点首命出。至夜,果依法而行。奶奶不觉乐极,问:“是谁传授的?”官曰:“皂隶。”奶奶一面将臀耸凑身作颠簸之状,曰:“好皂隶,真爽利!来日赏他两担老白米。”

截长
夫问妻曰:“此物还是长的好,短的好?”妻实喜长,而故应之曰:“短的好。”夫曰:“这等我的太长,不如截去一段。”持刀便砍。妻发急,止之曰:“虽则长了些,却是父母生就的遗体,一毫也动不得。”

长卵叹气
一官到任,出票要唤兄弟三人,一胖子、一长子、一矮子备用,异姓者不许进见。一家有兄弟四人,仅有一胖三矮,私相计议曰:“四人之中,胖矮俱有,单少一长人,只得将二矮缝一长裤,两人接起充作长人,便觉全备。”如计行之。官见大喜,簪花劳酒。三人一时荣宠,下矮压得受苦,在内光哓哓,大有怨词。官听见,问:“下面甚响?”众慌禀曰:“这是长卵叹气。”|Qī|shu|ωang|

矮子看灯
矮子看灯,适一人小便,竟往腿下钻过。观见厥物,赞曰:“好盏绣球灯,为何不点烛?”其人溺完,将尿滴在矮子头上,以手摸曰:“不好,快回去,大点雨打下来了!”

亲嘴
一矮子新婚,上床连亲百余嘴。妇问其故,答曰:“我下去了,还有半日不得上来哩。”

扇坠
有持大扇者,遇矮子,戏以扇置其头曰:“欲借兄权作扇坠耳。”矮子大怒,骂曰:“肏娘贼!若拿我做扇坠,我就兜心一脚踢杀你!”

搁浅
矮人乘舟出游,因搁浅,自起撑之,失手坠水,水没过项。矮人起而怒曰:“偏我搁浅搁在深处。”

瞎叙盟
三瞎子相聚结盟,叙齿以分长幼。一人曰:“不必论年,只看那个先瞎者,便让他做大哥。”一人曰:“我是周岁上不见起的,该轮着我居长。”其次曰:“我是百日内坏眼的,还该我来做老大。”第三者曰:“不要说起,我竟从娘胎里就是瞎的了。”两人曰:“那有此事?”答曰:“不然,为何从小人就骂我瞎屄里肏出来的!”

瞽笑
一瞽者与众人同坐,众人有所见而笑,瞽者亦笑。众问之曰:“汝何所见而笑?”瞽者曰:“列位所笑,定然不差,难道是骗我的?”

被打
二瞽者同行,曰:“世上惟瞽者最好。有眼人终日奔忙,农家更甚,怎如得我们心上清闲。”众农夫窃听之,乃伪为官过,谓其失于回避,以锄把各打一顿而呵之去。随复窃听之,一瞽者曰:“毕竟是瞽者好,若是有眼人,打了还要问罪哩!”

吃螺蛳
有盲子暑月食螺蛳,失手堕一螺肉在地。低头寻摸,误捡鸡屎放在口里,向人曰:“好热天气,东西才落下地,怎就这等臭得快!”

响不远
盲子夫妇同睡,妻暗约一人与之交合。夫问曰:“何处作欢响?”妻云:“想是间壁,不要管他。”少顷,又响,瞽者曰:“蹊跷,此响光景不远。”

独眼
兄弟二人,同往河中洗浴。兄之阳物被水蛇咬住,扯之不脱,弟持刀欲砍。兄曰:“仔细看了下刀。两眼的是蛇头,独眼的是屪子。”

兄弟认匾
兄弟三人皆近视,同拜一客。堂上悬“遗清堂”一匾,伯曰:“主人原来患此病,不然,何以取‘遗精室’也。”仔细看良久,曰:“非也。想主人好道,故名‘道情堂’耳。”二人争论不已,以季弟目力更好,使辨之。乃张目眈视半晌,曰:“汝两人皆妄,上面安得有匾!”

金漆盒
一近视出门,见街头牛屎一大堆,认为路人遗下的盒子。随用双手去捧,见其烂湿,乃叹曰:“好个盒子,只可惜漆水未干。”

问路
一近视迷路,见道傍石上栖歇一鸦,疑是人也,遂再三诘之。少顷,鸦飞去,其人曰:“我问你不答应,你的帽子被风吹去了,我也不对你说!”

噀面
一乡人携鹅入市,近视见之,以为卖布者,连呼“买布”。乡人不应,急上前拗住鹅尾,逼而视之。鹅忽撒屎,适喷其面。近视怒曰:“不卖就罢,值得这等发急,就噀(喷)起人来!”

乌云接日
近视者赴宴,对席一胡子吃火朱柿,即起别主人曰:“路远告辞。”主曰:“天色甚早。”答云:“恐天下雨,那边乌云接日头哩。”

鼻影作枣
近视者拜客,主人留坐待茶。茶果吃完,视茶内鼻影,以为橄榄也,捞摸不已。久之忿极,辄用指撮起,尽力一咬,指破血出。近视乃仔细认之,曰:“啐!我只道是橄榄,却原来是一个红枣。”

虾酱
一乡人挑粪经过,近视唤曰:“拿虾酱来。”乡人不知,急挑而走。近视赶上,将手握粪一把,于鼻上闻之,乃骂道:“臭已臭了,什么奇货,还要这等行情!”

疑蛋
一近视见鱼,疑为鸭蛋,握之而腹瘪。讶曰:“如何小鸭出得恁快,蛋壳竟瘪下去了。”

拾蚂蚁
近视者行路,见蚂蚁摆阵,疏密成行,疑是一物,因掬而取之。撮之不起,乃叹息曰:“可惜一条好线,毁烂得蹙蹙断了。”

检银包
有近视新岁出门,拾一爆竹,错认他人遗失银包也,且喜新年发财,遂密藏袖内。至夜,乃就灯启视,药线误被火燃,立时作响。方在吃惊,傍一聋子抚其背曰:“可惜一个花棒槌,无缘无故,如何就是这样散了。”

近趣眼
妻指牝户谓夫曰:“此物你最爱的,何不取一美号赠他?”夫曰﹔“爱其有趣,就名为趣眼。”妇又指后庭曰:“你有时也用着他,也该取一美号。”夫曰﹔“他与趣眼相近,就叫他做近趣(觑)眼罢了。”

白果眼
一女年幼而许嫁一大汉者,姻期将近。母虑其初婚之夜不能承受,“莫若先将鸡子稍用油润,与你先期开破,省得临时吃苦。”女含之。不意油滑突入牝中,不能得出,遂夹蛋过门。夫据腹良久,牝口阻塞难进,乃大叫曰:“媒人误我,娶一石女矣!”母不信,向媳曰:“姑媳无碍,把我看看何如?”及看毕,乃骂其子曰:“畜生,亏你枉做半世人,一只白果眼也不认得!”

漂白眼
一漂白眼与赤鼻头相遇,谓赤鼻者曰:“足下想开染坊,大费本钱,鼻头都染得通红。”赤鼻答曰:“不敢也,只浅色而已。怎如得尊目,漂白得有趣。”

聋耳
一医者耳聋,至一家看病女人。问:“莲心吃得否?”医者曰:“面觔发病,是吃不得的。”病女曰:“是莲肉。”医者曰:“就是盐肉,也要少吃些。”病女曰:“先生耳朵是聋的。”医曰:“若是里股是红的,只怕要生横痃,倒要脱开来,待我看看好用药。”

呵欠
一耳聋人探友,犬见之吠声不绝,其人茫然不觉。入见主人,揖毕告曰:“府上尊犬,想是昨夜不曾睡来。”主问:“何以见得:”答曰:“见了小弟,只是打呵欠。”

火症
一聋子望客,雨中见狗吠不止,乃叹曰:“此犬犯了火症,枯渴得紧,只管开口接水吃哩。”

讳聋哑
聋哑二人,各欲自讳。一日,聋见哑者,恳其唱曲。哑者知其聋也,乃以嘴唇开合,而手拍板作按节状。聋者侧听良久,见其唇住,即大赞曰:“妙绝,妙绝!许久不听佳音,今番一发更进了。”

麻屄
一客与妓密甚,临别谓妓曰:“恩爱情深,愿得一表记,睹物如见卿面矣。”妓赠以香囊、汗巾,俱不要。问曰:“所爱何物?”答曰:“欲得卿阴上之肉一块耳。”妓曰:“可。然须问过母亲来。”鸨儿曰:“放屁!一个孤老割一块,千百个孤老割了千百块,养成一张麻屄,那个还来要你!”

屁股麻
俗云:“脚麻以草柴贴眉心,即止。”一人遍贴额上。人问:“为何?”答曰:“我屁股通麻了。”

麻卵袋
文宗岁试唱名,吏善读别字,第一名郁进徒,错唤曰“都退后”,诸生闻之,皆山崩往后而退。次名潘传采,又错唤“番转来”,诸生又跑上前。宗师大怒,逐之。第三名林卯伐,上前谢曰:“多谢大宗师,若不斥逐此人,则生员必唤做麻卵袋了。”

麻子咬卵
粜芝麻者,见一秀才经过,问:“相公要买麻子否?”士答曰:“我读书人,要麻子来咬卵!”

赤鼻
一官经过,有赤鼻者在傍,皂隶喝曰:“老爷专要拿吃酒的,还不快走!”其人无处躲闪,只得将鼻子塞进人家板缝中。官已过,里面人看见骂曰:“这人不达时务,外面多少毛厕,如何倒向人家屋里来撒尿!”

齆鼻狗
黄鼠狼遇狗追逐,即撒屁以触其鼻。有雄鼠觅食田间,被一犬逐之,鼠狼连放数屁,逐之愈甚。乃竭力跑脱,至穴诉之雌鼠。雌鼠曰:“汝防身屁何在?”曰:“连撒数屁,全然不理。”雌鼠曰:“我知道了,决然是个齆鼻狗。”

齆鼻请酒
甲乙俱齆鼻。甲设席不能治柬,画秤、尺、笤帚各一件。乙见之,便意会曰:“秤(请)尺(吃)帚(酒)。”乙答柬,画蜈蚣一条,斧一把。甲见之,点头曰:“蜈(无)蚣(功)斧(夫)。”

臭嘴
或行酒令,俱要就人身上,说一必不然之事。一人云:“鼻孔亏得向下,若朝上,雨落在内怎么好?”一人云:“脚板亏得在前,若在后,被人踏住怎么好?”一人云:“妇人阴物亏得直生,若横生,菠箕背米菠边嵌进怎么好?”一人云:“屁眼亏得在臀,若在面,臭气触人怎么好?”主令者曰:“此句该罚。屁眼尽有生在面上的,不信,眼前这老兄尊嘴,如何便怎般臭极!”

鼻耐性
人患口臭,一友问曰:“别人也罢,亏你自家鼻头如何过了?”旁人代答曰:“做了他的鼻头,随你臭极,也只索耐性跟他。”

蒜治口臭
一口臭者问人曰:“治口臭有良方乎?”答曰:“吃大蒜极好。”问者讶其臭,曰:“大蒜虽臭,还臭得正路。”

臭瘌痢
北地产梨甚佳。北人至南,索梨食不得,南人因进萝卜,曰:“此敝乡土产之梨也。”北人曰:“此物吃下,转气就臭,味又带辣,只该唤他做臭辣梨。”

残疾婿
一家有三婿,俱带残疾。长是瘌痢,次淌鼻脓,又次患疯癫。翁一日请客,三婿在坐,恐其各露本相,观瞻不雅,嘱咐俱要收敛。三人唯唯。至中席,各人忍耐不住,长婿曰:“适从山上来,撞见一鹿,生得甚怪。”众问何状,瘌痢头疮痒甚,用拳满首击曰:“这边一个角,那边一个角,满头生了无数角。”其次鼻涕长流,正无计揩抹,随应声曰:“若我见了,拽起弓来,棚的一箭,”急将右手作挽弓状,鼻间一拂,涕尽拭去。三癞子浑身发痒难禁,忙将身背牵耸曰:“你倒胆大,还要射他!把我见了,几乎吓杀,几乎吓杀。”

歪屄
一婢女乃壬午生,而与陈五之人私通者。一日算命,说知生辰。星家排定四柱,开言曰:“娘子是壬午养的。”此女认作说他是陈五养活的,遂曰:“你只算命,莫管闲事。”星家复言:“我是有名铁嘴,莫怪我讲。你这壬午命犯桃花,一生孤苦,身充贱役,性情惫赖,后运还要落薄。”婢益疑讦其阴私,遂怒骂曰:“瞎贼,不要你算了!”星士亦怒曰:“这个歪屄,恁般可恶!”女曰:“我相与一陈五,就被他认破。今他说我歪屄,莫非此物原有些异样?”乃跷起一足于凳上,解裤视之,不料果然带偏。因叹服曰:“真神仙也!不然,为何一张歪屄,也被他看出?”

鸽舌
有涩舌者,俗云鸽口是也。来到市中买桐油,向店主曰:“我要买桐桐桐……”,“油”字再说不出口。店主取笑曰:“你这人倒会打铜鼓的,何不再敲通铜锣与我听?”鸽者怒曰:“你不要当当当面来腾腾腾倒刮刮刮削我。”

过桥啑
一乡人自城中归,谓其妻曰:“我在城里打了无数喷啑。”妻曰:“皆我在家想你之故。”他日挑粪过危桥,复连打数啑,几乎失足。乃骂曰:“骚花娘,就是思量我,也须看甚么所在!”

大耳
一妓苦阴毛太多,为嫖客所厌,呼待诏剃之。呼者虑其不来,诈言剃面。既至,妓谓曰:“唤你剃面,乃剃小面,非大面也。”即解出阴物示之。待诏剃毕,谓妓曰:“小面既剃,小耳亦不可不取,待我拿出消息来。”即解裤出具,投入阴中。忽大诧曰:“不意小小一张面孔,竟有这只大耳朵。”

歪头
有素患痿阳之症,娶得新妇到家。初夜行房,苦于厥物不举,舞弄既久,终不能入。妇怒曰:“直恁没用,头都东倒西歪,还想硬挣甚么!”夫乃诡辞以应曰:“你不晓得,我此物生来原是个歪头,少不得弄他进去哩。”

争坐
眼与眉毛曰:“我有许多用处,你一无所能,反坐在我的上位。”眉曰:“我原没用,只是没我在上,看你还像个人哩!”

直背
一瞎子,一矮子,一驼子,吃酒争座,各曰:“说得大话的便坐头一位。”瞎子曰:“我目中无人,该我坐。”矮子曰:“我不比常(长)人,该我坐。”驼子曰:“不要争,算来你们都是直背(侄辈),自然该让我坐。”

驼叔
有驼子赴席,泰然上座。众客既齐,自觉不安,复趋下谦逊。众客曰:“驼叔请上座,直背(侄辈)怎敢。”

善屁
有善屁者,往铁匠铺打铁搭,方讲价,连撒十余屁。匠曰:“汝屁直恁多,若能连撒百个,我当白送一把铁搭与你。”其人便放百个,匠只得打成送之。临出门,又撒数十屁,乃谓匠曰:“算不得许多。这几个小屁,乞我几只钯头钉罢。”

祖师殿
祖师殿中忽闻屁臭,众人互推不认,乃推祖师曰:“汝为正祖,受十方香火,如何撒屁?”祖师惊起辩曰:“尚有四将,何独推我?”四将亦辩曰:“尚有龟、蛇。”蛇曰:“我肚小撒不出,定是这个乌龟!”
一说祖师辩曰:“尚有四将。”四将互相推卸。关圣傍立关平曰:“撒屁的定然脸红。”关圣大怒曰:“你是我的儿子,也来冤屈我!”

认屁
一女善屁,新婚随嫁一妪一婢,嘱以认屁遮羞。临拜堂,忽撒一屁,顾妪曰:“这个老妈无体面!”少顷,又撒一屁,顾婢曰:“这个丫头恁可恶!”随后又二屁,左右顾而妪婢俱不在,无可说得,乃曰:“这张屁股没正经。”

屁婢
一婢偶于主人前撒了一屁,主怒,欲挞之。见其臀甚白,不觉动火,非但免责,且与之狎。明日,主在书房,忽闻叩门声,启户视之,乃昨婢也。问来为何,答曰:“我适才又撒一屁矣。”

錾头
数人同舟,有撒屁者,众疑一童子,共錾其头。童子哭曰:“阿弥陀佛。别人打我也罢了,亏那撒屁的乌龟,担得这只手起,也来打我!”

路上屁
昔有三人行令,要上山见一古人,下山又见一古人,半路见一物件,后句要总结前后二句。一人曰:“上山遇见狄青,下山遇见李白,路上拾得一瓶酒,不知是清酒是白酒。”一人曰:“上山遇见樊哙,下山遇见赵盾,路上拾得一把剑,不知是快剑是钝剑。”一人云:“上山遇见林放,下山遇见贾岛,路上拾得一个屁,不知是放的屁、岛的屁。”

贼屁
穿窬躲在人家床底,忽撒一屁甚响。夫骂妻,妻云:“你撒了屁,倒来冤屈我!”争闹不已。贼无奈,只得出来招认曰:“这屁其实是贼放的。”

吃屁
酒席间有人撒屁者,众人互相推卸。内一人曰:“列位请各饮一杯,待小弟说了罢。”众饮讫,其人曰:“此屁实系小弟撒的。”众人不服,曰:“为何你撒了屁,倒要我们众人吃!”

棹面响
一人方陪客,偶撒一屁。自觉愧甚,欲掩饰之,乃假将指头擦桌面作响声。客曰:“还是第一声像得紧。”

田鸡叫
甲乙两亲家姆会亲,乙偶撒一屁,甲问曰:“亲家姆,甚响?”乙恐不雅,答曰:“田鸡叫。”甲曰:“为甚能臭?”乙曰:“死的呀。”又问:“适才会叫,如何是死的?”乙曰:“叫了就死的。”

不嘿
各行酒令,要嘿饮。席中有撒屁者,令官曰:“不嘿,罚一杯。”其人曰:“是屁响。”令官曰:“又不嘿,再罚一杯。”举坐为之大笑。令官曰:“通座皆不嘿,各罚一杯。”

怕冷
或问:“世间何物不怕冷?”曰:“鼻涕,天寒即出。”又问:“何物最怕冷?”曰:“屁,才离窟臀,又向鼻孔里钻进。”

大乳
一妇人两乳极大,每用抹胸束之。一日,忘紧抹胸,偶出见人。人怪而问曰:“令郎是几时生的?”妇曰:“还不曾产育。”人问曰:“既不是令郎,你胸前袋的是甚么?”

抓背
老翁续娶一妪,其子夜往窃听,但闻连呼“快活”,频叫“爽利”。子大喜曰:“吾父高年,尚有如此精力,此寿征也。”再细察之,乃是命妪抓背。

善生虱
有善生虱者,自言一年止生十二个虱。诘其故,曰:“我身上的虱,真真一月(捏)一个。”

赞阳物
一人客于他乡,见土著者问曰:“贵地之人好大阳物?”土著者甚喜,答曰:“果然,但不知尊客如何知道?”其人曰:“我在贵处嫖了几晚,觉得此处的阴物比别处更宽,所以知道。”

家当
一妇有姿色,而穷人欲谋娶之,恐其不许,乃贿托媒人极言其家事富饶。妇许之,及过门,见四壁萧然,家无长物,知堕计中。辄大哭不止,怨恨媒人。穷人以阳物托出,丰伟异常,放在桌上连敲数下,仍收起曰:“不是我夸口说,别人本钱放在家里,我的家当带在身边。如娘子不愿,任从请回。”妇忙掩面试泪曰:“谁说你甚么来。”

肚肠
有未嫁者,父方小解,亵物为女所见。问母曰:“那是甚么东西?”母不便显言,答曰:“挂出的肚肠。”女既嫁归宁,母愁婿家贫,劝之久住,谓其夫家柴米不足也。女曰:“人家穷便穷,喜得肚肠还好,就忍些饥饿也情愿。”

巨卵
一人死后,冥王罚变为驴。其人哀恳,得许复原形,放其还魂。因行急,犹有驴卵未变。既醒,欲再往换,仍复原体。其妻力止之曰:“胡阎王不是好讲话的,只得做我不着,挨些苦罢。”

小卵
一人命妻做鞋而小,怒曰:“你当小不小,偏小在鞋子上面!”妻亦怒曰:“你当大不大,偏大在这只脚上!”

贵相
有家人妇,得宠爱于主人者,同伴私问其状,答曰:“贵相真是不同。”问何故,答曰:“卵袋都是绵团丝软的。”

当卵
一妇榄权甚,夫所求不如意,乃以带系其阳于后而诳妻曰:“适因其用甚急,与你索不肯,已将此物当银一两与之矣。”妻摸之,果不见,乃急取银二两付夫,令速回赎,嘱曰:“若典中有当绝长大的,宁可加贴些银子,换上一根回来。你那怪小东西,弃绝了也罢。”

倭刺
甲乙两妇对坐,各问夫具之大小及伎俩如何,因不便明言,乃各比一物。甲曰:“我家的是铙碗盛小菜。”乙问其故,甲曰,“小便不小,只是数目不多,极好不过四碟。”乙曰:“这等还好,不像我家的物事,竟是一把倭刺。”甲问其故,乙曰:“又小又快。”

快刀
新郎初次行房,妇欣然就之,绝不推拒。至事毕之后,反高声叫曰:“有强盗,有强盗!”新郎曰:“我乃丈夫,如何说是强盗。”新妇曰:“既不是强盗,为何带把刀来?”夫曰:“刀在那里?”妇指其物曰:“这不是刀?”新郎曰:“此乃阳物,何认为刀?”新妇曰:“若不是刀,为何这等快极!”

瘪东西
一老人娶幼妇,云雨间对妇云:“愿你养一个儿子。”妇曰:“儿子倒养不出,只好养个团鱼。”夫骇问其故,答曰:“像你这样瘪东西,如何养的不是团鱼?”

硬中证
有病偏坠者,左肾以家私不均事告于肚皮。左肾自觉强良占脬太多,用厚礼结纳于阳具,诉状中求其做一硬中证。及临审,左肾抗辨力甚,而阳具缩首,不出一语。肚皮责阳物曰:“你向日直恁跳梁,今日何顿软弱,还不从直讲来?”答曰:“见本主子脱硬挣,我只得缩了。」

\part{}

殊禀部

善忘
一人持刀往园砍竹,偶腹急,乃置刀于地,就园中出恭。忽抬头曰:“家中想要竹用,此处倒有许多好竹,惜未带得刀来。”解毕,见刀在地,喜曰:“天随人愿,不知那个遗失这刀在此。”方择竹要斫,见所遗粪,便骂曰:“是谁狗肏的,阿此脓血,几乎屣了我的脚。”须臾抵家,徘徊门外曰:“此何人居?”妻适见,知其又忘也,骂之。其人怅然曰:“娘子颇有些面善,不曾得罪,如何开口便骂?”

恍惚
三人同卧,一人觉腿痒甚,睡梦恍惚,竟将第二人腿上竭力抓爬,痒终不减,抓之愈甚,遂至出血。第二人手摸湿处,认为第三人遗溺也,促之起。第三人起溺,而隔壁乃酒家,槯酒声滴沥不止,以为己溺未完,竟站至天明。

作揖
两亲家相遇于途,一性急,一性缓。性缓者,长揖至地,口中谢曰:“新年拜节奉扰,元宵观灯又奉扰,端午看龙舟,中秋玩月,重阳赏菊,节节奉扰,未曾报答,愧不可言。”及说毕而起,已半晌矣。性急者苦其太烦,早先避去。性缓者视之不见,问人曰:“敝亲家是几时去的?”人曰:“看灯之后,就不见了,已去大半年矣!”

爇衣
一最性急、一最性缓,冬日围炉聚饮。性急者衣坠炉中,为火所燃,性缓者见之从容谓曰:“适有一事,见之已久,欲言恐君性急,不言又恐不利于君,然则言之是耶,不言是耶?”性急者问以何事,曰:“火烧君裳。”其人遽曳衣而起,怒曰:“既然如此,何不早说!”性缓者曰:“外人道君性急,不料果然。”

卖弄
一亲家新置一床,穷工极丽,自思:“如此好床,不使亲家一见,枉自埋没。”乃假装有病,偃卧床中,好使亲家来望。那边亲家做得新裤一条,亦欲卖弄,闻病欣然往探。既至,以一足架起,故将衣服撩开,使裤现出在外,方问曰:“亲翁所染何症,而清减至此?”病者曰:“小弟的贱恙,却像与亲翁的心病一般。”

品茶
乡下亲家进城探望,城里亲家待以松罗泉水茶。乡人连声赞曰:“好,好。”亲翁以为彼能格物,因问曰:“亲家说好,还是茶叶好,还是水好?”乡人答曰:“热得有趣。”

出像
乡下亲家到城里亲家书房中,将文章揭看,摇首不已。亲家说:“亲翁无有得意的么?”答云:“正是。看了半日,并没有一张佛像在上面。”

刚执
有父子性刚,平素不肯让人。一日,父留客饭,命子入城买肉。子买讫,将出城门,值一人对面而来,各不相让,遂挺立良久。父寻至见之,谓子曰:“汝快持肉回去,待我与他对立看。”

应急
主人性急,仆有过犯,连呼:“家法!”不至,跑躁愈甚。家人曰:“相公莫恼,请先打两个巴掌,应一应急着。”

掇桶
一人留友夜饮,其人蹩额坚辞。友究其故,曰:“实不相瞒,贱荆性情最悍,尚有杩子桶未倒,若归迟,则受累不浅矣。”其人攘臂而言曰:“大丈夫岂有此理!把我便──”其妻忽出,大喝曰:“把你便怎么?”其人即双膝跪下曰:“把我便掇了就走!”

正夫纲
众怕婆者,各受其妻惨毒,纠合十人歃血盟誓,互为声援。正在酬神饮酒,不想众妇闻知,一齐打至盟所。九人飞跑惊窜,惟一人危坐不动。众皆私相佩服曰:“何物乃尔,该让他做大哥。”少顷妇散,察之,已惊死矣。

请下操
一武弁惧内,面带伤痕。同僚谓曰:“以登坛发令之人,受制于一女子,何以为颜?”弁曰:“积弱所致,一时整顿不起。”同僚曰:“刀剑士卒,皆可以助兄威。候其咆哮时,先令军士披挂,枪戟林立,站于两傍,然后与之相拒。彼摄于军威,敢不降服!”弁从之。及队伍既设,弓矢既张,其妻见之,大喝一声曰:“汝装此模样,将欲何为?”弁闻之,不觉胆落,急下跪曰:“并无他意,请奶奶赴教场下操。”

虎势
有被妻殴,往诉其友,其友教之曰:“兄平昔懦弱惯了,须放些虎势出来。”友妻从屏后闻之,喝曰:“做虎势便怎么?”友惊跪曰:“我若做虎势,你就是李存孝。”

访类
有惧内者,欲访其类,拜十弟兄。城中已得九人,尚缺一个,因出城访之。见一人掇马桶出,众齐声曰:“此必是我辈也。”相见道相访之意,其人摇手曰:“我在城外做第一个倒不好,反来你城中做第十个。”

吐绿痰
两惧内者,皆以积懮成疾,一吐红痰,一吐绿痰。因赴医家疗治,医者曰:“红痰从肺出,犹可医,绿痰从胆出,不可医,归治后事可也。”其人问由胆出之故,对曰:“惊碎了胆,故吐绿痰,胆既破了,如何医得!”

理旧恨
一怕婆者,婆既死,见婆像悬于柩侧,因理旧恨,以拳拟之。忽风吹轴动,忙缩手大惊曰:“我是取笑作耍。”

敕书
一官置妾,畏妻,不得自由,怒曰:“我只得奏一本去。”乃以黄袱裹绫历一册,从外擎回,谓妻曰:“敕旨在此。”妻颇畏惧。一日夫出,私启视之,见“正月大,二月小”,喜云:“原来皇帝也有大小。”看“三月大,四月小”:“到分得均匀”。至五月大、六月大、七月大、八月数月小,乃大怒云:“有这样不公道的皇帝,凉爽天气,竟被他占了受用,如何反把热天都派与我!”

吃梦中醋
一惧内者,忽于梦中失笑。妻摇醒曰:“汝梦见何事,而得意若此?”夫不能瞒,乃曰:“梦娶一妾。”妻大怒,罚跪床下,起寻家法杖之。夫曰:“梦幻虚情,如何认作实事?”妻曰:“别样梦许你做,这样梦却不许你做的。”夫曰:“以后不做就是了。”妻曰:“你在梦里做,我如何得知?”夫曰:“既然如此,待我夜夜醒到天明,再不敢睡就是了。”

葡萄架倒
有一吏惧内,一日被妻挝碎面皮。明日上堂,太守见而问之,吏权词以对曰:“晚上乘凉,被葡萄架倒下,故此刮破了。”太守不信,曰:“这一定是你妻子挝碎的,快差皂隶拿来。”不意奶奶在后堂潜听,大怒抢出堂外。太守慌谓吏曰:“你且暂退,我内衙葡萄架也要倒了。”

捶碎夜壶
有病其妻之吃醋,而相诉于友,谓:“凡买一婢,即不能容,必至别卖而后已。”一友曰:“贱荆更甚,岂但婢不能容,并不许置一美仆,必至逐去而后已。”傍又一友曰:“两位老兄,劝你罢,像你老嫂还算贤慧。只看我房下,不但不容婢仆,且不许擅买夜壶,必至捶碎而后已。”

手硬
有相士对人谈相云:“男手如枪,女手如姜,一生吃不了米饭,穿不了衣裳。”一人喜曰:“若是这等说,我房下是个有造化的。”人问:“何以见得?”答曰:“昨晚在床上,嫌我不能尽兴,被他打了一掌,今日还是辣渍渍的。”

呆郎
一婿有呆名,舅指门前杨竽问曰:“此物何用?”婿曰:“这树大起来,车轮也做得。”舅喜曰:“人言婿呆,皆妄也。”及至厨下,见研酱擂盆,婿又曰:“这盆大起来,石臼也做得。”适岳母撒一屁,婿即应声曰:“这屁大起来,霹雳也做得。”

痴婿
人家有两婿,小者痴呆,不识一字。妻曰:“娣夫读书,我爹爹敬他,你目不识丁,我面上甚不争气。来日我兄弟完姻,诸亲聚会,识认几字,也好在人前卖嘴。我家土库前,写‘此处不许撒尿’六字,你可牢记,人或问起,亦可对答,便不敢欺你了。”呆子唯诺。至日,行至墙边,即指曰:“此处不许撒尿。”岳丈喜曰:“贤婿识字大好。”良久,舅姆出来相见,裙上有销金飞带,绣“长命富贵,金玉满堂”八字,坠于裙之中间。呆子一见,忙指向众人曰:“此处不许撒尿。”

呆子
一呆子性极痴,有日同妻至岳家拜门,设席待之。席上有生柿水果,呆子取来,连皮就吃。其妻在内窥见,只叫得“苦呀”。呆子听得,忙答曰:“苦到不苦,惹得满口涩得紧着哩。”

赞马
一杭人有三婿,第三者甚呆。一日,丈人新买一马,命三婿题赞,要形容马之快疾,出口成文,不拘雅俗。长婿曰:“水面搁金针,丈人骑马到山阴。骑去又骑来,金针还未沉。”岳丈赞好。次及二婿曰:“火上放鹅毛,丈人骑马到余姚。骑去又骑来,鹅毛尚未焦。”再次轮到三婿,呆子沉吟半晌,苦无搜索。忽丈母撒一响屁,呆子曰:“有了。丈母撒个屁,丈人骑马到诸稽。骑去又骑来,孔门犹未闭。”

搠穿肚
一呆婿新婚,平素见人说男女交姤,而未得其详。初夜据妇股往来摩拟久之,偶插入牝中,遂大惊,拔户披衣而出,躲匿他处。越数日,昏夜潜至巷口,问人曰:“可闻得某家新妇,搠穿了肚皮没事么?”

携冻水
一呆婿至妻家留饭,偶吃冻水美味,乃以纸裹数块,纳之腰间带归。谓妻曰:“汝父家有佳味,我特携来啖汝。”索之腰中,已消溶矣。惊曰:“奇!如何撒出了一脬尿,竟自逃走了。”

莫说是我
夫妇正行房事,忽丈母闯入,夫即仓皇躲避,嘱其妻曰:“丈母若问,千万莫说是我。”

不道是你
新郎愚蠢,连朝不动,新人只得与他亲斗一嘴。其夫大怒,往诉岳母,母曰:“不要恼他,或者不道是你啰。”

只说是我
一丈人昼寝,以被蒙头。婿过床前,忽以手伸入被中,潜解其裤。丈人大惊,乃揭被视之,乃其婿也,诃责不已。丈母来劝曰:“你莫怪他,他不曾看得分明,只认是我了。”

丈母不该
女婿见丈人拜揖,遂将屁股一挖。丈人大怒,婿云:“我只道是丈母啰。”隔了一夜,丈人将婿责之曰:“畜生,我昨晚整整思量了一夜,就是丈母,你也不该。”

痴人生女
有痴人娶妻,久而不知交合。妻不得已,乃抱之使上,导之使入。及阳精欲泄,忽叫曰:“我要撒尿。”妻曰:“不妨,就撒在里面。”痴人从之。后生一女,问妻曰:“此从何来?”妻曰:“不记撒尿之事乎?”夫乃大悟,寻复悔之,因咎其妻曰:“撒尿生女,撒屎一定生男,当初何不早说。”

胡涂花面
痴人无子,遍访生儿之法。一人戏之曰:“先将阳物画作人形,然后做事,定然成胎。”痴人依法而行,事毕仍视其物,则满面胡涂矣。因自叹曰:“儿子有便有了,只是生下的,必定一个花脸了。”

事发觉
一人奔走仓惶,友问:“何故而急骤若此?”答曰:“我十八年前干差了一事,今日发觉。”问:“毕竟何事?”乃曰:“小女出嫁。”

父各爨
有父子同赴席,父上坐,而子遥就对席者。同席疑之,问:“上席是令尊否?”曰:“虽是家父,然各爨久矣。”

烧令尊
一人远出,嘱其子曰:“有人问你令尊,可对以家父有事出外,请进拜茶。”又以甚呆恐忘也,书纸付之。子置袖中,时时取看。至第三日,无人来问,以纸无用,付之灯火。第四日,忽有客至,问:“令尊呢?”觅袖中纸不得,因对曰:“没了。”客惊曰:“几时没的?”答曰:“昨夜已烧过了。”

子守店
有呆子者,父出门,令其守店。忽有买货者至,问:“尊翁有么?”答曰:“无。”又问:“尊堂有么?”亦曰:“无。”父归知之,责其子曰:“尊翁我也,尊堂汝母也,何得言无!”子懊怒曰:“谁知你夫妇两人,都是要卖的!”

活脱话
父戒子曰:“凡人说话,放活脱些,不可一句说煞。”子问:“如何活脱?”时适有邻家来借物件。父指而教之曰:“比如这家来借东西,看人打发,不可竟说多有,不可竟说多无,也有家里有的,也有家里无的,这便活脱了。”子记之。他日,有客到门问:“令尊在家否?”答曰:“我也不好说多,也不好说少,其实也有在家的,也有不在家的。”

母猪肉
有卖母猪肉者,嘱其子讳之。已而买肉者至,子即谓曰:“我家并非母猪肉。”其人觉之,不买而去。父曰:“我已吩咐过,如何反先说起!”怒而挞之。少顷,又一买者至,问曰:“此肉皮厚,莫非母猪肉乎?”子曰:“何如!难道这句话,也是我先说起的?”

望孙出气
一不肖子常殴其父,父抱孙不离手,爱惜愈甚。人间之曰:“令郎不孝,你却钟爱令孙,何也?”答曰:“不为别的,要抱他大来,好替我出气。”

买酱醋
祖付孙钱二文,买酱油、醋。孙去而复回,问曰:“那个钱买酱油?那个钱买醋?”祖曰:“一个钱酱油,一个钱醋,随分买,何消问得?”去移时,又复转问曰:“那个碗盛酱油?那个碗盛醋?”祖怒其痴呆,责之。适子进门,问以何故,祖告之。子遂自去其帽,揪发乱打,父曰:“你敢是疯了?”子曰:“我不疯,你打得我的儿子,我难道打不得你的儿子?”

劈柴
父子同劈一柴,父执柯,误伤子指。子骂曰:“老乌龟,汝眼瞎耶?”孙在傍见祖被骂,意甚不平,遂曰:“狗屄出的,父亲可是骂得的么?”

悟到
一富家儿不爱读书,父禁之书馆。一日,父潜伺窥其动静,见其子开卷吟哦,忽大声曰:“我知之矣。”父意其有所得,乃喜而问曰:“我儿理会了么?”子曰:“书不可不看。我一向只道书是写成的,原来是刊板印就的。”

藏锄
夫在田中耦耕,妻唤吃饭,夫乃高声应曰:“待我藏好锄头,便来也!”乃归,妻戒夫曰:“藏锄宜密。你既高声,岂不被人偷去?”因促之往看,锄果失矣。因急归,低声附其妻耳云:“锄已被人偷去了。”

较岁
一人新育女,有以两岁儿来议亲者,其人怒曰:“何得欺我!吾女一岁,他子两岁,若吾女十岁,渠儿二十岁矣,安得许此老婿!”妻谓夫曰:“汝算差矣!吾女今年虽一岁,等到明年此时,便与彼儿同庚,如何不许?”

拾簪
一人在枕边拾得一簪,喜出望外。诉之于友,友曰:“此不是兄的,定是尊嫂的,何喜之有?”其人答曰:“便是不是弟的,又不是房下的,所以造化。”

认鞋
一妇夜与邻人有私,夫适归,邻入逾窗而出。夫攫得一鞋,骂妻不已。因枕鞋而卧,谓妻曰:“且待大明,认出此鞋,与汝算帐!”妻乘其睡熟,以夫鞋易去之。夫晨起复骂,妻使认鞋。见是自己的,乃大悔曰:“我错怪你了,原来昨夜跳窗的倒是我。”

搽药
一呆子之妇,阴内生疮痒甚,请医治之。医知其夫之呆也,乃曰:“药须我亲搽,方知疮之深浅。”夫曰:“悉听。”医乃以药置龟头,与妇行事。夫在旁观之,乃曰:“若无这点药在上面,我就疑心到底。”

记酒
有觞客者,其妻每出酒一壶,即将锅煤画于脸上记数。主人索酒不已,童子曰:“少吃几壶罢,家主婆脸上,看看有些不好看了。”

狠干
苏人遇一友云:“昨日兄为何如此高兴,在家狠干。”友云:“并不曾。”其人曰:“我在府上亲听甚久,还要赖么?”友曰:“骗兄非人,我昨日实实不在家里。”

奸睡
奸夫闻亲夫归,急欲潜遁,妇令其静卧在床。夫至,问:“床上何人?”妻答云:“快莫做声,隔壁王大爷被老娘打出来,权避在此。”夫大笑云:“这死乌龟,老婆值得恁怕!”

杀妻
夫妻相骂,夫恨曰:“臭娼根,我明日做了皇帝,就杀了你。”妇日夜懮泣不止,邻女解之曰:“那有此事,不要听他。”妇曰:“我家这个臭乌龟倒从不说谎的,自养的儿女,前年说要卖,当真的旧年都卖去了。”

盗牛
有盗牛被枷者,亲友问曰:“汝犯何罪至此。”盗牛者曰:“偶在街上走过,见地下有条草绳,以为没用,误拾而归,故连此祸。”遇者曰:“误拾草绳,有何罪犯?”盗牛者曰:“因绳上还有一物。”人问:“何物?”对曰:“是一只小小耕牛。”

籴米
有持银入市籴米,失叉袋于途,归谓妻曰:“今日市中闹甚,没得好叉袋也。”妻曰:“你的莫非也没了?”答曰:“随你好汉便怎么?”妻惊问:“银子何在?”答曰:“这倒没事,我紧紧拴好在叉袋角上。”

在行
有行路者,对人门缝撒尿,为其家妇人看见,骂之不已。撒尿者曰:“我还是个童男,不消骂得。”妇曰:“头多褪了一大截,还说甚么童男!”邻人笑曰:“这一句话,却不该是娘子说的。”妇曰:“他明明欺我不在行,如何不指破他?”

呆算
一人家费纯用纹银,或劝以倾销八九色杂用,当有便宜。其人取元宝一锭,托熔八成。或素知其呆也,止倾四十两付之,而利其余。其人问:“元宝五十两,为何反倾四十?”答曰,“五八得四十。”其人遽曰:“吾为公误矣,用此等银反无便益。”

代打
有应受官责者,以银三钱,雇邻人代往。其人得银,欣然愿替。既见官,官喝打三十。方受数杖,痛极,因私出所得银,尽贿行杖者,得稍从轻。其人出谢前人曰:“蒙公赐银救我性命,不然,几乎打杀。”

七月儿
有怀孕七个月即产一儿者,其夫恐养不大,遇人即问。一日,与友谈及此事,友曰:“这个月无妨,我家祖亦是七个月出世的。”其人错愕问曰:“若是这等说,令祖后来毕竟养得大否?”

卵生翼
兄谓弟曰:“卵袋若生翅膀,见有好妇人便可飞去。”弟曰:“使勿得,别人家个卵也要飞来个。”

试试看
新妇与新郎无缘,临睡即踢打,不容近身。郎诉之父,父曰:“毕竟你有不是处,所以如此。”子云:“若不信,今晚你去睡一夜试试看。”

靠父膳
一人廿岁生子,其子专靠父膳,不能自立。一日算命云:“父寿八十,儿寿六十二。”其子大哭曰:“这两年叫我如何过得去!”

觅凳脚
乡间坐凳,多以现成树丫叉为脚者。一脚偶坏,主人命仆往山中觅取。仆持斧出,竟日空回,主人责之,答曰:“丫叉尽有,都是朝上生,没有向下生的。”

访麦价
一人命仆往枫桥打听麦价,仆至桥,闻有呼“吃扯面”者,以为不要钱的,连吃三碗径走。卖面者索钱不得,批其颏九下。急归谓主人曰:“麦价打听不出,面价吾已晓矣。”主问:“如何?”’答曰:“扯面每碗要三个耳光。”

锤
一人睡在床上,仰面背痛,覆卧肚痛,侧困腰痛,坐起臀痛,百医无效。或劝其翻床,及翻动,见褥底铁秤锤一个,垫在下面。

懒活
有人极懒者,卧而懒起,家人唤之吃饭,复懒应。良久,度其必饥,乃哀恳之。徐曰:“懒吃得。”家人曰:“不吃便死,如何使得?”复摇首漫应曰:“我亦懒活矣。”

白鼻猫
一人素性最懒,终日偃卧不起。每日三餐,亦懒于动口,恹恹绝粒,竟至饿毙。冥王以其生前性懒,罚去轮回变猫。懒者曰:“身上毛片,愿求大王赏一全体黑身,单单留一白鼻,感恩实多。”王问何故,答曰:“我做猫躲在黑地里,鼠见我白鼻,认作是块米糕,贪想愉吃,潜到嘴边,一口咬住,岂不省了无数气力。”

露水桌
一人偶见露水桌子,因以指戏写“谋篡”字样,被一仇家见之,夺桌就走,往府首告。及官坐堂,露水以为日色曝干,字迹减去。官问何事,其人无可说得,慌禀曰:“小人有桌子一堂,特把这张来看样,不知老爷要买否?”

衣软
一乡人穿新浆布衣入城,因出门甚早,衣为露水讽湿。及至城中,怪其顿软。事毕出城,衣为日色曝干,又硬如故。归谓妻曰:“莫说乡下人进城再硬不起来,连乡下人的衣服见了城里人的衣服,都会绵软起来。”

椅桌受用
乡民入城赴席,见椅桌多悬桌围坐褥。归谓人曰:“莫说城里人受用,连城里的椅桌都是极受用的。”人问其故,答曰:“桌子穿了绣花裙,椅子都是穿销金背心的。”

咸蛋
甲乙两乡人入城,偶吃腌蛋,甲骇曰:“同一蛋也,此味独何以咸?”乙曰:“我知之矣,决定是腌鸭哺的。”

看戏
有演《琵琶记》而找《关公斩貂蝉》者,乡人见之泣曰:“好个孝顺媳妇,辛苦了一生,竟被那红脸蛮子害了。”

演戏
有演《琵琶记》者,找戏是《荆钗·逼嫁》,忽有人叹曰:“戏不可不看,极是长学问的。今日方知蔡伯喈的母亲,就是王十朋的丈母。”

怯盗
一痴人闻盗入门,急写“各有内外”四字,贴于堂上。闻盗已登堂,又写“此路不通”四字,贴于内室。闻盗复至,乃逃入厕中。盗踪迹及之,乃掩厕门咳嗽曰:“有人在此。”

复跌
一人偶扑地,方爬起复跌。乃曰:“啐!早知还有只一跌,便不走起来也罢了。”

缓踱
一人善踱,行步甚迟。日将晡矣,巡夜者于城外见之,问以何往,曰:“欲至府前。”巡夜者即指犯夜,擒捉送官。其人辩曰:“天色甚早,何为犯夜?”曰:“你如此踱法,踱至府前,极早也是二更了。”

出辔头
有酷好乘马者,被人所欺,以五十金买驽马一匹。不堪鞭策,乃雇舟载马,而身跨其上。既行里许,嫌其迟慢,谓舟人曰:“我买酒请你,与我快些摇,我要出辔头哩。”

铺兵
铺司递紧急公文,官恐其迟,拨一马骑之。其人赶马而行,人问其“如此急事,何不乘马?”答曰:“六只脚走,岂不快如四只。”

米
一妇人与人私通,正在房中行事,丈夫叩门。妇即将此人装入米袋内,立于门背后。丈夫入见,问曰:“叉袋里是甚么?”妇人着忙,不能对答。其人从叉袋中应声曰:“米。”

鹅变鸭
有卖鹅者,因要出恭,置鹅在地。登厕后,一人以鸭换去。其人解毕,出视叹曰:“奇哉!才一时不见,如何便饿得恁般黑瘦了。”

帽当扇
有暑月带毡帽而出者,歇大树下乘凉,即脱帽以当扇。扇讫,谓人曰:“今日若不带此帽出来,几乎热杀。”

买海蛳
一人见卖海蛳者,唤住要买,问:“几多钱一斤?”卖者笑曰:“从来海蛳是量的。”其人喝曰:“这难道不晓得!问你几多钱一尺?”

浼匠迁居
一人极好静,而所居介于铜,铁两匠之间,朝夕聒耳,甚苦之,常曰:“此两家若有迁居之日,我宁可作东款谢。”一日,二匠并至曰:“我等欲迁矣,足下素许东道,特来叩领。”其人大喜,遂盛款之。席间问之曰:“汝两家迁往何处?”答曰:“他搬在我屋里,我即搬在他屋里。”

混堂嗽口
有人在混堂洗浴,掬水人口而嗽之。众各攒眉相向,恶其不洁。此人贮水于手曰:“诸公不要愁,待我嗽完之后,吐出外面去。”

何往
一人赋性呆蠢,不通文墨。途遇一友,友问曰:“兄何往?”此人茫然不答,乃记“何往”二字以问人。人知其呆,故为戏之曰:“此恶语骂兄耳。”其人含怒而别。次日,复遇前友问:“兄何往?”此人遽愤然曰:“我是不何往,你倒要何往哩!”

呆执
一人问大辟,临刑,对刽子手曰:“铜刀借一把来动手,我一生服何首乌的。”

信阴阳
有平素酷信阴阳,一日被墙压倒。家人欲亟救,其人伸出头来曰:“且慢,待我忍着,你去问问阴阳,今日可动得土否?”

丑汉看
一妇人在门首,被人注目而看,妇人大骂不已。邻妪劝曰:“你又不在内室,凭他看看何妨?”妇曰:“我若把好面孔看看也罢,被这样呆脸看了,岂不苦毒。”

爇翁腿
一老翁冬夜醉卧,置脚炉于被中,误爇其腿。早起骂乡邻曰:“悉老人家多吃了几杯酒,睡着了,便自不知。你们这班后生,竟不来唤醒一声,难道烧人臭也不晓得!”

合着靴
有兄弟共买一靴,兄日着以拜客赴宴。弟不甘服,亦每夜穿之,环行室中,直至达旦。俄而靴敝,兄再议合买,弟曰:“我要睡矣。”

教象棋
两人对奕象棋,傍观者教不至口。其一大怒,挥拳击之,痛极却步。右手么脸,左手遥指曰:“还不叉士!”

发换糖
一呆子见有以发换糖者,谬谓凡物皆可换也。晨起,袖中藏发一料以往,遇酒肆即入饱餐。餐毕,以发与之。肆佣皆笑,其人怒曰:“他人俱当钱用,到我偏用不得耶!”争辩良久,肆佣因揪发乱打。其人徐理发曰:“整料的与他偏不要,反在我头上来乱抢。」

\part{}

闺风部

洞房佳偶
一佳人新嫁,合欢之夜,佳人以对挑之曰:“君乃读书之辈,奴出一对,请君对之。如答得来,方许云雨,不然则不从也。”新郎曰:“愿闻。”女曰:“柳色黄金嫩,梨花白雪香,你爱不爱?”新郎对曰:“洞里乾坤大,壶中日月长,你怕不怕?”

拜堂产儿
有新妇拜堂,即产下一儿,婆愧甚,急取藏之。新妇曰:“早知婆婆这等爱惜,快叫人把家中阿大、阿二都领了来罢。”

抢婚
有婚家女富男贫,男家虑其新婚,率领众人抢亲,误背小姨以出。女家人急呼曰:“抢差了!”小姨在背上曰:“不差,不差!快走上些,莫信他哄你哩。”

两坦
有一女择配,适两家并求,东家郎丑而富,西家郎美而贫。父母问其欲适谁家。女曰:“两坦。”问其故,答曰:“我爱在东家吃饭,西家去眠。”

两尽
夫劝新妇解衣。妇曰:“母戒我勿解,母命不可违﹔夫劝我解,夫命又不可违﹔奈何?”正沉吟间,夫迫之,妇曰:“我知之矣!只脱去下截,做个两尽其情罢。”

问嫂
一女未嫁者,私问其嫂曰:“此事颇乐否?”嫂曰:“有甚乐处,只为周公之礼,制定夫妇耳。”及女出嫁后归宁,一见其嫂,即笑骂曰:“好个说谎精。”

没良心
一妓倚门而立,见有客过,拉人打钉,适对门楼上,姑嫂二人推窗见之,姑问嫂:“扯他何事?”嫂曰:“要他行房。”须臾事毕,妓取厘戥夹剪付之,姑曰:“彼欲何为?”嫂曰:“行过了房,要他出银子。”姑叹曰:“好没良心,如何反要他出。”

呼不好
一新妇初夜,新郎不甚在行,将阳物放进而不动。女呻吟曰:“哎哟,不好,胀痛!”夫曰:“拿出罢?”女又呻吟曰:“哎哟,不好,空痛!”夫曰:“进又胀痛,出又空痛,汝欲怎么?”女曰:“你且拿进拿出间看。”

谢周公
一女初嫁,哭问嫂曰:“此礼何人所制?”嫂曰:“周公。”女将周公大骂不已。及满月归宁,问嫂曰:“周公何在?”嫂云:“他是古人,寻他做甚?”女曰:“我要制双鞋谢他。”

死结
新人初夜上床,使性不止。喜娘隔壁劝曰:“此乃人伦大事,个个如此,不要害羞。”新人曰:“你不晓得,裤子衣带,偏生今夜打了死结。”

亲嘴
一女初嫁,次早新郎背立,女扳其嘴,连亲数下,郎大怒曰:“如何不识羞耻?”妇应曰:“其实一时认错了,不知是你,莫怪,莫怪。”

出气
一女未嫁,父母索重聘。既嫁初夜,婿怪岳家争论财礼,因恨曰:“汝父母直恁无情,我只拿你出气。”乃大干一次。少倾又曰:“汝兄嫂亦甚可恶,也把你来发泄。”又狠弄一番。两度之后,精力疲倦,不觉睡去。女复摇醒曰:“我那兄弟虽小,日常多嘴多舌,倒是极蛮惫的。”

通奸
一女与人通奸,父母知而责之。女子赖说:“都是那天杀的强奸我,非我本意。”父母曰:“你缘何不叫喊起来?”女曰:“我的娘呀,喊是要喊。你想那时,我的舌头,被他噙紧在口里,叫我如何喊得出。”

用枕
有女嫁于异乡,归宁,母问:“风土相同否?”答曰:“别事都一样,只有用枕不同。吾乡把来垫头,彼处垫在腰下的。”

掮脚
新人初夜,郎以手摸其头而甚得意,摸其乳腹俱欢喜,及摸下体,不见两足,惊骇问之,则已掮起半日矣。

新人哭
幼女出嫁,喜娘归。主母问:“姑娘连日动静何如?”答曰:“头夜听得姑娘哭,想是面生害怕。第二晚不想官人哭。”母骇问:“为何?”云:“姑娘扳痛了屁股。第三夜随嫁丫头又大哭。”母曰:“更奇怪。”喜娘曰:“我曾问来,他说这样一个好姑娘,口口声声只叫要死。”

舌头甜
新婚夜,送亲席散。次日,厨司捡点桌面,不见一顶糖人,各处查问。新人忽大笑不止,喜娘在傍问:“笑甚么?”女答曰:“怪不得昨夜一个人舌头是甜津津的。”

起半身
一夫妇新婚,睡至晌午不起。母嫌其贪睡,遣婢潜往探之。婢覆曰:“官人、娘子,大家才起得一半了。”母问何故,婢曰:“官人起了上半身,娘子只起得下半身着哩。”

大话
一女出嫁坐床,掌礼撤帐云:“撤帐东,官人屪子好撞钟。”女忙接口云:“弗怕。”喜嫔曰:“新娘子不宜如此口快。”新妇曰:“不是我也不说,才得进门,可恶他就把这大话来吓我。”

正好
新妇出嫁,坐床撤帐,掌礼念云:“夫妇双双喜气扬,官人屪子硬如枪。”伴送婆应曰:“忒硬过了!”新妇接口曰:“弗要说,正好。”

鹰啄
一母生一子一女,而女尤钟爱。及遣嫁后,思念不已。谓子曰:“人家再不要养女儿,养得这般长成,就如被饿鹰轻轻一爪便抓去了。”子曰:“阿姆阿姆,他们如今正在那里啄着哩。”

半处子
有寡妇嫁人而索重聘。媒曰:“再醮与初婚不同,谁肯出次高价。”妇曰:“我还是处子,未曾破身。”媒曰:“眼见嫁过人,今做孤孀,那个肯信?”妇曰:“实不相瞒,先夫阳具渺小,故外面半截,虽则重婚,里边其实是个处子。”

纳茄
一妇昼寝不醒,一人戏将茄子纳入牝中而去。妇觉,见茄在内,知为人所欺,乃大骂不止。邻妪谓曰:“其事甚丑,娘子省口些罢。”妇曰:“不是这等说,此番塞了茄儿不骂,日后冬瓜、葫芦便一起来了。”

嗔儿
夫妻将举事,因碍两子在旁,未知熟睡不曾。乃各唤一声以试之。两子闻而不应,知其欲为此事也。及云雨大作,其母乐极,大呼叫死。一子忽大笑,母惭而挞之。又一子曰:“打得好,打得好,娘死了不哭,倒反笑起来。”

冻杀
夫妇乘子熟睡,任意交感。事毕,问其妻“爽利么?”连问数语,妻碍口不答。子在脚后云:“娘快些说了罢,我已冻杀在这里了。”

软萝卜
姑嫂二人纺织,偶见萝卜一蓝,姑曰:“蓝中萝卜,变成男子阳物,便好。”嫂曰:“软的更妙。”姑曰:“为何倒要软的?”嫂曰:“软的硬起来,一蓝便是两蓝。”

捉虼蚤
妻好云雨,每怪其夫好睡,伺夫合眼,即翻身以扰之。夫问:“何以不睡?”曰:“虼蚤叮人故耳。”夫会其意,旋与之交。妻愿既遂,乃安眠至晓。夫执其物而叹曰:“我与他相处─生,竟不知他有这种本事。”妻曰:“甚么本事?”夫曰:“会捉虼蚤。”

贼干
贼至卧室,见一婢裸体熟睡,即与交合。婢大叫“有贼”,贼狠干不歇。,婢遂低声悄问曰:“贼哥,你几时来的?”

饭米
贫人正与妻合,妻云:“饭米都没了,有甚高兴?”夫物顿痿。妻复云:“虽如此说,坛内收拾起来,还勾明后日吃哩。”

擂棰
开腐店者,夫妇云雨,妻嫌其物渺小。夫潜往外,取研石膏擂棰,暗暗塞进。妻曰:“你在那里吃了什么来,此物顿然大了!天气和暖,为何冻得他恁冰冷?”

咎夫
一妇临产,腹中痛甚,乃咎其夫曰:“都是你作怪,带累我如此。”怨詈不止。夫呵之曰:“娘子,省得你埋怨,总是此物不好,莫若阉割了,绝此祸根!”遂持刀欲割。妻大呼曰:“活冤家!我痛得死去还魂,这刻才好些,你又来催命了。”

取名
一妇临产创甚,与夫誓曰:“以后不许近身,宁可一世无儿,再不干那营生矣。”夫曰:“谨依尊命。”及生一女,夫妻相议命名,妻曰:“唤做招弟罢。”

不怕死
一妇生育甚难,因咎丈夫曰:“皆你平素作孽,害我今日受苦。”夫甚不过意,遂相戒:“从今各自分床,不可再干此事。”妻然之。弥月后,夜间忽闻启户声。夫问:“是谁?”妻应曰:“那个不怕死的又来了。”

寡欲
一贫家生子极多,艰于衣食。夫咎妻曰:“多男多累,谁教你多男?”妻曰:“寡欲多子,谁教你寡欲!”

多男
一人连举数子,医士谀之曰:“寡欲多男子。兄少年老成,过于保养之故。何不乘此强壮,快活快活。”妻在屏后应曰:“先生说得极是。我也生育得不耐烦,觉得苦极了。”

问儿
一人从外归,私问儿曰:“母亲曾往何处去来?”答曰:“间壁。”问:“做何事?”儿曰:“想是同外公吃蟹。”又问:“何以知之?”儿曰:“只听见说:‘拍开来,缩缩脚。’娘又叫道:‘勿要慌,我个亲爷。’”

祈神
一人痿阳,具牲礼祷神。巫者祝曰:“世阳世阳,顾得卵硬如枪。”病者曰:“何敢望此?”妻从屏后呼曰:“费了大钱大陌,也得如此!”

下半截
一人欲事过度,惫甚,夫妇相约:“下次云雨,止放半截。”及行事,妻掬夫腰尽纳之。夫责以前约,妻曰:“我原讲过是下半截。”

嘴不准
妇人见男子鼻大,戏之曰:“你鼻大物也大。”男子见妇人嘴小,亦戏曰:“你嘴小阴亦小。”两人兴动,遂为云雨。不意男之物甚细,而女之阴甚大,妇曰:“原来你的鼻不准。”男曰:“原来你的嘴也不准。”

讼奸
有妇诉官云:“往井间汲水,被人从后淫污。”官曰:“汝那时何不立起?”答曰:“若立起,恐脱了出来耳。”

栗爆响
妇握夫两卵,问是何物。夫曰:“栗子。”夫亦指妻牝户,问是何物。妻曰:“火炉。既是你有栗子,何不放在炉内,煨他一煨?”夫曰:“可。”少顷,妇撒一屁,儿在傍叫曰:“爹爹,栗子熟矣,在炉内爆响了!”

铁箍
夫妇同饭,妻问曰:“韭蒜有何好处,汝喜吃他?”夫曰:“食之,此物如铁棒一般的。”妻亦连食不已,夫曰:“汝吃何用?”妻曰:“我吃了像铁箍一般的。”

两来船
一人遇两来船,手托在窗槛外,夹伤一指。归诉于妻,妻骇然嘱曰:“今后遇两来船,切记不可解小便。”

醉饱行房
一人好于酒后渔色,或戒之曰:“醉饱莫行房,五脏皆反复,此药石语也,如何犯之?”其人曰:“不妨。行过之后,再行一次,依旧掉转来,只当不曾反复。”

命运不好
一妇有淫行,每嫁一夫,辄有外遇,夫觉即被遣。三年之内,连更十夫。人问曰:“汝何故而偃蹇至此?”妇曰:“生来命运不好,嫁着的就要做乌龟。”

邻人看
一妇诉其夫曰:“邻某常常看我。”夫曰:“睬他做甚?”妇曰:“我今日对你说,你不在意,下次被他看上了,却不关我事。”

丝瓜换韭
妻令夫买丝瓜,夫立门外候之,有卖韭者至,劝之使买。夫曰:“要买丝瓜耳。”卖者曰:“丝瓜痿阳,韭菜兴阳,如何兴阳的不买,倒去买痿阳的?”妻闻之,高声唤曰:“丝瓜等不来,就买了韭菜罢。”

后园种韭
有客方饭,偶谈“丝瓜痿阳,不如韭菜兴阳”。已而主人呼酒不至,以问儿,儿曰:“娘往园里去了。”问:“何为?”答曰:“拔去丝瓜种韭菜。”

脚淘
夫妻反目,分头而睡。夜半,妻欲动而难以启口,乃摸夫脚问曰:“这是甚物?”夫曰:“脚。”妻曰:“既是脚,可放在脚淘里去。”

怕冷
幼女见两狗相牵,问母曰:“好好两只狗,为何联拢在一处?”母曰:“想是怕冷。”女摇头曰:“不是,不是。”母曰:“怎见得不是?”女曰:“前日大热天气,你和爹爹也是这样,难道都是怕冷不成?”

稳生男
问:“如何方稳生男?”给之者曰:“连二卵纳入,无不成胎矣。”夜则如其言,纳左则右出,纳右则左复出。恚曰:“便生出儿子来,也是个强种!”

龌龊
夫狎龙阳归,妻辄作呕吐状,谓其满身屎臭,不容近身。至夜同宿,夫故离开以试之。妻渐次捱近,久之,遂以牝户靠阳,将有凑合之意。夫曰:“此物龌龊,近之何为?”妻曰:“正为龌龊,要把阴水洗他一洗。”

浆硬
─人衣软,令其妻浆硬些。妻用浆浆好,随扯夫阳具,也浆一浆。夫骇问,答曰:“浆浆硬好用。”

老鼠数钱
夫妻同卧,妻指阳物曰:“此何物?”答曰:“老鼠。”妻曰:“既是老鼠,何不放他进窠去。”遂交合有声,儿在傍闻之,呼其母问曰:“阿妈,老鼠才进窠,如何便数起铜钱来?”

邻人问
妇谓夫曰:“脚盆内潮浴,还是脚盆好过,浴的好过?”夫曰:“消息子取耳,还是耳好过,消息好过?”语毕,云雨。邻人问曰:“消息落在脚盆里,那个好过?”

忌叫死
两夫妇度岁,夫于除夕戒妻曰:“往日行房,每到快活处,必定叫死。明日是新年,大家忌说死字,但说我要活。”妻然之。及次日行房,妻乐极,仍叫如前。夫怪其忌犯,妻曰:“不妨。像这种死法,那怕一年死到头!”

再醮
有再醮者,初夜交合,进而不觉也。问夫:“进去否?”曰:“进去矣。”妇遂颦蹩曰:“如此,我有些疼。”

扇尸
夫死,妻以扇将尸扇之不已。邻入问曰:“天寒何必如此?”妇拭泪答曰:“拙夫临终吩咐:‘你若要嫁人,须待我肉冷。’”

不不
两妇对门而居,甲问乙曰:“生过几胎了?”乙曰:“未曾破体。”甲曰:“难道你家大爷是不的么?”乙摇头曰:“不,不。”

愿杀
妻妾相争,夫实爱妾,而故叱之曰:“不如杀了你,省得啕气。”妾仰入房,夫持刀赶入。妻以为果杀,尾而视之,见二人方在云雨。妻大怒曰:“若是这等杀法,倒不如先杀了我罢!”

心在这里
有置妾者,与妻行乐,妻曰:“你身在这里,心自在那里。”夫曰:“若然,待我身在那里,心在这里何如?”

公直老人
妻妾争风,夫又倦于房事,乃曰:“我若就那个,只说我偏爱。今夜待我仰卧在床,看你们造化,凭他此物向谁,就去与他干事。”妻妾依言,各将阳物摸弄,一时兴起,竖若桅杆。夫大笑曰:“你两个扶持他起来,做了公直老人,不肯询私,我也没法。”

他大我大
一家娶妾,年纪过长于妻。有卖婆见礼,问:“那位是大?”妾应云:“大是他大,大是我大。”

罚真咒
一人欲往妾处,诈称:“我要出恭,去去就来。”妻不许,夫即赌咒云:“若他往做狗。”妻将索系其足放去。夫解索,转缚狗脚上,竟往妾房。妻见去久不至,收索到床边,起摸着狗背,乃大骇云:“这死乌龟,我还道是骗我,却原来倒罚了真咒。”

浇蜡师
人家有一妻一妾,前后半夜分认。上半夜至妻房,妻腾身跨上夫肚行事,夫问:“何为?”曰:“此倒浇蜡烛也。”其妾早在门外窃闻之矣。下半夜乃同妾睡,恣意欢娱,妾快甚,不觉失声曰:“我死也!”妻亦在外潜听之矣。次早量米造饭,妻曰:“今日当减一人饭米?”妾曰:“为何?”妻曰:“昨晚死了一个人。”妾亦微笑曰:“依我看来,今日还该添一人才是。”妻问何故,答曰:“闻得有个浇蜡烛的师父在此。”

谢媳
一翁扒灰,事毕,揖其媳曰:“多谢娘子美情。”媳曰:“爹爹休得如此客气,自己家里,那里谢得许多。”

毛病
一翁偷媳,媳不从,而诉于姑。姑曰:“这个老乌龟,像了他的爷老子,都有这个毛病。”

拿访
一人作客在外,见乡亲问曰:“我家父在家好么?”乡亲曰:“好是好,前日按院访拿十二个扒灰老,尊翁躲在毛厕里,几乎吓杀。”

卖古董
一翁素卖古董为业,屡欲偷觑其媳,媳诉于婆。一日,妪代媳卧,翁往摸之,姬乃夹紧以自掩。翁认为媳,极口赞誉,以为远出婆上。妪骂曰:“臭老贼,一件旧东西也不识,卖甚古董!”

换床
一翁欲偷媳,媳与姑说明,姑云:“今夜你躲过,我自有处。”乃往卧媳床,而灭火以待之。夜深翁果至,认为媳妇,云雨极欢。既毕,妪骂曰:“老杀才,今夜换得一张床,如何就这等高兴!”

雷击
有客外者,见故乡人至,问:“家乡有甚新闻?”曰:“某日一个霹雳,打死十余人,都是扒灰老。”其人惊问曰:“家父可无恙乎?”答曰:“令尊倒幸免,令祖却在数内,一同归天了。”

偷弟媳
一官到任,众里老参见。官下令曰:“凡偷媳妇者站过西边,不偷者站在东边。”内有一老人慌忙走到西首,忽又跑过东来。官问曰:“这是何说?”老人跪告曰:“未曾蒙老爷吩咐,不知偷弟媳妇的,该立在何处?”

老娶
一老人欲娶,妈妈见他须发尽白,不肯嫁他。老者贿嘱媒人曰:“还他夜夜有事,如一夜落空,愿责五下。”妈许之。过门初晚,勉干一度,次夜就不能动弹。妈将老儿推倒,责过五板,老者伏地不起。妈问何故,老者陪笑曰:“求妈妈索性打上整百,往后一起好算帐。”

使搭头
翁与妪行房,妪耻其宽,以手向臀后捏紧。翁亦苦阳痿,以两指衬贴,导之使进。妪曰:“老儿,你缘何在那里使搭头?”翁曰:“老娘,强如你在背地打后手。”

破开晒
翁、妪相对曝日,妪兴发动,拉翁行房,翁以天寒不举对。妪曰:“请各解其物晒之,热则举矣。”翁曰:“然。”遂解裤向日。少顷妪曰:“我的热了,快来。”翁曰:“我的还未。”妪曰:“一般晒法,为何冷热不均?”翁曰:“你是破开晒的,我是囫囵晒的,如何赶得上?”

忽举
有痿阳者,一夜忽举,心中甚喜,及扒上妻腹,仍痿如初。妻问:“何为?”答曰:“我想要里床去睡,借你肚子上来过路。”

许愿
老翁素苦阳痿,偶见猪羊交感,不觉动兴。夜归与妻同卧,触着日间所见,阳事突举,急与妻行事。恐其半途痿弃,遂摩拟日间形状,口念:“一个猪,一个羊。”妻曰:“老贼囚,来不得罢了,如何这般大愿,直得就许出来。”

上路来
一老翁勉力行房,阳痿不能进。舞弄既久,不觉鼻涕横流,因叹曰:“我说为何这等干涩,原来打从上路出来了。”

折不受
老年人娶妾,其物已痿,因急欲举子,云雨时嘱其妾曰:“请受,请受。”妾曰:“你干净折子,教我受什么!”

米粒
老年人行房,勉力交媾。妇云:“再进得一米粒也好。”老儿大怒曰:“我若有意留了一米粒,做我的倒头羹饭!”

日进
老年娶妾,欲结其欢心,说某处有田地若干,房屋若干。妾曰:“这都不在我心上。从来说家财万贯,不如肏进分文的好。”

喷嚏
老夫妇正在交合,妻忽打一喷嚏,此物脱出,乃大怒吵闹。次早,邻妇问曰:“你老夫妇,为何昨夜不睦?”答曰:“不要说起,老贼近来一发改变得不好,嚏也打不得一个。”

咬牙
有姑媳孀居,姑曰:“做寡妇,须要咬紧了牙根过日子。”未几,姑与人私,媳以前言责之。姑张口示媳曰:“你看,也得我有牙齿方好咬。”

藏年
一人娶一老妻,坐床时,见面多皱纹,因问曰:“汝有多少年纪?”妇曰:“四十五六。”夫曰:“婚书上写三十八岁,依我看来还不止四十五六,可实对我说。”曰:“实五十四岁矣。”夫再三诘之,只以前言对。上床后更不过,心乃巧生一计,曰:“我要起来盖盐瓮,不然被老鼠吃去矣。”妇曰:“倒好笑,我活了六十八岁,并不闻老鼠会偷盐吃。”

谢金口
夫妇皆年老者,元旦行房,相约各说吉利语。妻执夫阳物曰:“愿你自今日以后,愈老愈健。”夫随摸妻阴户曰:“多谢你的金口。”

挣命
僧、尼二人庙中避雨,至晚同宿。僧摸尼牝户问:“此事何物?”尼曰:“是口棺材。”尼摸僧阳具问:“此是何物?”僧曰:“是个死和尚。”尼曰:“既如此,我把棺材布施他装了。”僧遂以阳物投入阴中,抽提跃跳。尼曰:“你说是个死和尚,如何会动?”僧笑曰:“他在里头挣命哩。”

娶头婚
一人谋娶妇,虑其物小,恐贻笑大方,必欲得一处子。或教之曰:“初夜但以卵示之,若不识者,真闺女矣。”其人依言,转谕媒妁,如有破绽,当即发还。媒曰:“可。”及娶一妇,上床解物询之,妇以卵对。乃大怒,知非处子也,遂遣之。再娶一妇,问如前,妇曰:“鸡巴?”其人诧曰:“此物的表号都已晓得,一发不真。”又遣之。最后娶一年少者,仍试如前,答曰:“不知。”此人大喜,以为真处子无疑矣,因握其物指示曰:“此名为卵。”女摇头曰:“不是。我也曾见过许多,不信世间有这般细卵。”

咏物
两夫妇稍通文墨,一生琴瑟调和。及至暮年,精力衰耗,不能畅举,乃对物伤情,各咏一词以志感。妻先咏其牝户曰:“红焰焰,黑焰焰,嫩如出甑馒头解条线。自从嫁过你家来,日也,夜也,如今就像破门扇,东一片,西一片。”夫亦咏麈柄曰:“光溜溜,赤溜溜,硬如檀木匾担挑得豆。自从娶你进门来,朝也凑,暮也凑,如今好似葛布袖,扯便长,不扯皱。」

\part{}

世讳部

开路神
金刚遇开路神,羡之曰:“你我一般长大,我怎如你着好吃好。”开路神曰:“阿哥不知,我只图得些口腹耳。若论穿着,全然不济,剥去一层遮羞皮,浑身都是篾片了。”

焦面鬼
一帮闲汉途遇人家出丧,前面焦面鬼王,以为大老官人也,礼拜甚恭。少顷,大雨如注,而鬼身上纸衣被雨濯去。闲汉曰:“白日见鬼,我只道是大老官,却原来也是个篾片。”

咽糠
一闲汉咽糠而出,忽遇大老官留家早饭,答曰:“适间用狗肉过饱,饭是吃不下了,有酒倒饮几杯。”既饮忽吐,而糠出焉。主见,惊问曰:“你说吃了狗肉,为何吐此?”其人睨视良久,曰:“咦,我自吃的狗肉,想必狗曾吃糠来。”

望烟囱
富儿才当饮啖,闲汉毕集。因问曰:“我这里每到饭熟,列位便来,就一刻也不差,却是何故?”诸闲汉曰:“遥望烟囱内烟出,即知做饭,熄则熟矣,如何得错?”富儿曰:“我明日买个行灶来煮,且看你们望甚么?”众曰:“你煨了行灶,我等也不来了。”

老白相
荒岁闲汉无处活口,值官府于玄妙观施粥,闲汉私议曰:“我等平昔鲜衣美食,今往吃,必贻人笑。”俄延久之,无奈腹中饿甚,曰:“姑待众饥民吃过,尾其后可也。”远望人散而往,则粥已尽矣,乃以指拉食釜杓间余粥。道士见而问之,答曰:“我等原是捞(老)白相耳。”

借脑子
苏州人极奉承大老官,平日常谓主人曰:“要小子替死,亦所甘心。”一日主病,医曰:“病入膏肓,非药石所能治疗,必得生人脑髓配药,方可救得。”遍索无有,忽省悟曰:“某人平日常自谓肯替死,岂吝惜一脑乎?”即呼之至,告以故。乃大惊曰:“阿呀,使勿得,吾里苏州人,从来无脑子个。”

呵脬
一帮闲,见大老官生得面方耳圆,遂赞不置口。其人曰:“你又在此呵卵脬了?”

曲蟮
帮闲者自夸技能曰:“我件件俱精,天下无比。”一人曰:“只有一物最像。”问:“是何物?”答曰:“曲蟮。”问:“何以像他?”曰:“杀之无血,剐之无肉,要长就长,要短就短,又会唱曲,又会呵脬。”

件件熟
帮闲人除夜与妻同饭,忽然笑曰:“我想一生止受用得一个‘熟’字。你看大老官,那个不熟?私窠小娘,那个不熟?游船上,那个不熟?戏子歌童,那个不熟?萧管唱曲的朋友,那个不熟?”话未毕,妻忽大恸。其人问故,曰:“天杀的!你既件件皆热,如何我这件过年布衫,偏不替我赎。”

活千年
一门客谓贵人曰:“昨夜梦公活了一千年。”贵人曰:“梦生得死,莫非不祥么。”其人遽转口曰:“啐!我说差了,正是梦公死了一千年。”

屁香
有奉贵人者,贵人偶撒一屁,即曰:“那里伽楠香?”贵人惭曰﹔“我闻屁乃谷气,以臭为正。今反香,恐非吉兆。”其人即以手招气嗅之曰:“如今有点臭了。”

撞席
老鼠与獭结交。鼠先请獭,獭答席,邀鼠过河,暂往觅食。忽一猫见之欲捕,鼠慌曰:“请我的倒不见,吃我的到来了。”

泥高壁
燕子衔泥做窠,搬取蚯蚓上面土。蚓愤极曰:“你要泥高顶壁,为何把我来悔气?”燕子云:“我专怪你呵人家卵脬。”

嫖院吏
一吏假扮举人,往院嫖妓。妓以言戏之曰:“我今夜身上来,不得奉陪。”吏曰:“申上来我就驳回去。”妓曰:“不是这等说,行房龌龊。”吏曰:“刑房龌龊,我兵房是干干净净的。”曰:“是月经。”吏曰:“我从幼习的是详文、招稿,不管你甚么《易经》、《诗经》。”妓曰:“相公差矣,是流经。”吏曰:“刘洪他是都吏,你拿来吓我,难道就怕了不成?”

换班
一皂隶妻性多淫,夫昼夜防范。一日该班,将妻阴户左傍画一皂看守,并为记认。妻复与人干事,擦去前皂,奸夫仓卒仍画一皂形于右边而去。及夫落班归家,验之已非原笔,因怒曰:“我前记在左边的,缘何移在右边了?”妻曰:“亏你做衙门多年,难道不要轮流换班的么?”

争坐
鼻与眉争坐位,鼻曰:“一切香臭,皆我先知,我之功大矣。汝属无用之物,何功之有,辄敢位居我上?”眉曰:“是则然矣,假如鼻头坐上位,世上有此理否?”

软硬
屪子与鼻子争论,屪子云:“我能生男育女,有功人世,你有何德能,辄敢居我上位?”鼻曰:“我居五岳之中,能知气味,汝何敢轻觑我?”二物争之不决,告诉于口。口曰:“我劝你们和了罢。”鼻倔强不肯。口怒曰:“屪子还有软的时节,你做鼻头,倒是这等硬挣。”

婢子
有婢生子,既长,或问其号。子谦逊久之,乃曰:“贱号小梅。”问:“尊公原号何梅?”答曰:“非也,乃家母名腊梅耳。”

尿壶骂
一仆人之使,俗言鼻里。鼻也,出倾夜壶。归告主人曰:“阿爹,方才尿鳖骂我,又骂阿爹。”主人曰:“胡说!尿鳖如何会骂人?”小使曰:“起初骂了我鼻,后连声骂曰:‘鼻鼻鼻,鼻鼻鼻。’岂不把阿爹都骂在里头了?”

对戏
戏子出门,嘱其妻曰:“同伴来,可拿出戏鼓,教他对对戏眼。”妻误听,以为脱出屁股,教他对屁眼。同伴至,乃以后庭与之。伴问云:“你家主公比我做法如何?”妇云:“好是好,只是急撮戏文,板还要上紧些。”

屁股痛
麻苍蝇与青苍蝇结为兄弟,青蝇引麻蝇到一酒席上。麻蝇恣意饮啖,被小厮拿住,将竹签插入屁股,递灯草与他使棍。半日才得脱身,遇着青蝇泣诉曰:“承你挚带,吃倒尽有,只是屁股痛得紧。”

龙阳娶
一龙阳新娶,才上床,即攀妇臀欲干。妇曰:“差了。”答曰:“我从小学来的,如何得差?”妇曰:“我从小学来,却不是这等的,如何不差?”

撒精
一人患㾀病,医曰:“必须用少男之精,配药服之,方可还原。”乃令人持器往觅。途遇一美童,告以故。童令以器置地,遂解裤,向臀后撒之。求者曰:“精出在前,为何取之以后?”童曰:“你不知,出处不如聚处。”

臀凑
一龙阳新婚之夜,以臀凑其妻。妻摸之,讶曰:“你如何没有的?”龙阳亦摸其妻,讶曰:“你如何也没有的?”

袭职
龙阳生子,人谓之曰:“汝已为人父矣,难道还做这等事?”龙阳指其子曰:“深欲告致,只恨袭职的还小,再过十余年,使当急流勇退矣。”

兑车
两童以后庭相易,俗云兑车是也。一童甚黠,先戏其臀,甫完事,即赖之而走。被弄者赶至其家,且哭且叫曰:“要还我,要还我!”其母不知何事,出劝曰:“学生不要哭,他赖了你甚么,待我替他还你罢。”

挤进
一少年落夜船,有人挨至身边,将阳物插入臀窟内。少年骇问:“为何?”答云:“人多,挤了进去。”又问:“为何只管动?”答曰:“这却是我不是,在此擦痒哩。”

夫夫
有与小官契厚者,及长,为之娶妻。讲过通家不避。一日,闯入房中,适亲家母在,问女曰:“何亲?”女答曰:“夫夫。”

倒做龟
龙阳毕姻后,日就外宿。妻走母家,诉曰:“我不愿随他了。”母惊问故,答曰:“我是好人家儿女,为甚么倒去与他做乌龟。”

老了叫
有龙阳年纪过大者,偶撒一屁,狎客为之叩齿。众问其故,答曰:“你们不听见老了叫么?”

寿板
有好男风者,夜深投宿饭店,适与一无须老翁同宿。暗中以为少童也,调之。此翁素有臀风,欣然乐就。极欢之际,因许之以制衣打簪,俱云不愿。问所欲何物,答曰:“愿得一副好寿板。”

小娘
牝狗与牛交而生男,及长,人问其爷娘何在,指牛曰:“此爷也。”指狗曰:“此娘也。”其人讶曰:“这等一个大老官,如何配恁个小娘?”

好睡纳鞋
妓好睡,每至日高不醒。有闯寡门者,窃一酒壶而去。他日客至,又复鼾睡如初,客去方醒。检点衣物,失去绣鞋一只,及下床,忽于阴中坠出。盖客笑其善睡,戏将此鞋纳之而去也。鸨儿急曰:“仔细再寻一寻,前日不见的酒壶,只怕也还在里面。”

羡妓阴物
嫖客自妓馆归,妻问曰:“这些娼妇,经过千万人,此物定宽,有甚好处,而朝夕恋他?”夫曰:“不知甚么缘故,但是名妓,越接得客多,此物越好。”妻曰:“原来如此,这也何难,为甚不早说?”

豁拳
嫖客与妓密甚,相约同死。既设酖酒二瓯,妓让客先饮。客饮毕,因促妓,妓伸拳曰:“我的量窄,与你豁了这杯罢。”

嫌口阔
一少年嫖妓,嫌妓口阔,因述俗语云:“口阔屄儿大。”妓即撮口骂曰:“小猢狲。”

梦里梦
妓与客久别复会,各道相思。妓云:“我无夜不梦见你同食,同眠,同游戏,乃是积想所致。”客曰:“我亦梦之。”妓问曰:“梦怎的?”曰:“我梦见你,不梦见我。”

年倒缩
一商人嫖妓,问其青春几何。妓曰:“十八。”越数年,商人生意折本,仍过其家。妓忘之。问其年,则曰:“十七。”又过数年,入其家问之,则曰:“十六。”商人忽涕泣不止,妓问何故,曰:“你的年纪,倒与我的本钱一般,渐渐的少了。想到此处,能不令人伤心。”

子嫖父帮
有子好嫖而饿其父者,父谓之曰:“与其用他人闲闻,何不带挈我入席,我既得食,汝亦省钱,岂不两便?但不可说破耳。”子从之。父在妓家,诸事极善帮衬体贴。妓问曰:“何处得此帮客,大异常人。”子曰:“不好说得。他家媳妇与我有些私情,是我养活也,所以这般体贴。”明日,妓述此语于翁,翁曰:“虽则如此,他家母亲也与我有些勾搭,只当儿子一般,不得不体贴他。”

父多一次
子好游妓馆,父责之曰:“不成器的畜生,我到娼家,十次倒有九次见你。”子曰:“这等说来,你还多我一次,反来骂我?”

醉敲门
光棍醉敲妓门,妓知其乏钞,闭而不纳,辞以有客,实无客也。光棍破门而进,妓灭灯仰卧于床。光棍摸着其足,与男人无异,乃笑曰:“他不拒我,果然是有客。”

缠住
一螃蟹与田鸡结为兄弟,各要赌跳过涧,先过者居长。田鸡溜便早跳过来。螃蟹方行,忽被一女子撞见,用草捆住。田鸡见他不来,回转唤云:“缘何还不过来?”蟹曰:“不然几时来了,只因被这歪刺骨缠住在此,所以耽迟来不得。”

龟渡
有一士欲过河,苦无渡船。忽见有一大龟,士曰:“乌龟哥,烦你渡我过去,我吟诗谢你。”龟曰:“先吟后渡。”士曰:“莫被你哄,先吟两句,渡后再吟两句,何如?”龟曰:“使得。”士吟曰:“身穿九宫八卦,四游龙王也怕。”龟喜甚,即渡士过河。士续曰:“我是衣冠中人,不与乌龟答话。”

骨血
妓接一西客,临去,欲暖其心,伪云:“有三个月身孕,是你的骨血,须来一看。”客信之,如期果至。妓计困,乃以小白犬一只置儿篮内,蒙被而诳客曰:“儿生矣,熟睡不可搅动他。”客启视狗身,乃大喜,抚犬曰:“果是咱亲骨血,在娘胎里就穿上羊皮袄子了。”

妻当稍
一人好赌,日夜不归。已破家,止剩一妻,乃以出稍。不几掷,复输去。因请再饶一掷,赢家曰:“讲绝了稍做妻,如何又饶?”答曰:“其中有一缘故,房下还是室女,作少了价钱,饶一掷不为过。”赢家曰:“那有此理?”曰:“你若不信,只看我自做亲以来,何曾有一夜在家里?”

取头
好赌者,家私输尽,不能过活,取绳上吊。忽见一鬼在梁上云:“快拿头来。”此人曰:“也亏你开得这口,我输到这般地位,还来问我要头!”

捉头
按君访察,匡章、陈仲子及齐人,俱被捉。匡自信孝子,陈清客,俱不请托。惟齐人有一妻一妾,馈送显者求解。显者为见按君,按君述三人罪状,都是败坏风俗的头目,所以访之。显者曰:“匡章出妻屏子,仲子离母避兄,老公祖捉得极当。那齐人是叫化子的头,也捉他做甚么?”

白日鬼
法师上坛,焰口施食。天将明矣,正要安寝,又见一班披枷带锁、折手断脚的饿鬼索食。师问:“阳世作何生理,受此果报?”众云:“皆是拐骗子,做中保、镶局害人的。”又问:“夜间为何不来同领法食?”答曰:“我们一班,都是白日鬼。”

公子头
一人生平惯做分头,扣克人家银钱。死后阎王痛恨,发在黑暗地狱内受罪。进狱时即云:“列位在此,不见天日,何不出一公分,开个天窗?”

穿窬
一士人夜读,见偷儿穴墙有声,时炉内滚汤正沸,提汤潜伺穴口。及墙既穿,偷儿先以脚进,士遂擒住其两腿,徐以滚汤淋之。贼哀告求释,士从容谓曰:“多也不敢奉承,只尽此一壶罢。”

新雷公
雷公欲诛忤逆子,子执其手曰:“且慢击。我且问你还是新雷公,还是旧雷公?”雷公曰:“何谓?”其人曰:“若是新雷公,我竟该打死。若是旧雷公,我父忤逆我祖,你一向在那里去了?”

叫城门
一人最好唱曲。探亲回迟,城门已闭,因叫:“开门!”管门者曰:“你唱一曲我听,便放你进来。”此人曰:“唱便唱,只是我唱,你要答应。”管门曰:“依你。”其人先说白云:“叫周仓!”城上应曰:“嗄。”“关爷爷在城外了,还不快迎!”复应曰:“嗄。”其人曰:“你既晓得关出你爷在城外,就该开门,如何还敢要我唱曲?”

老鳏
苏州老鳏,人问:“有了令郎么?”答云:“提起小儿,其实心酸。前面妻祖与妻父定亲,说得来垂成了,被一个天杀的用计矗退了,致使妻父不曾娶得妻母,妻母不曾养得贱内,至今小儿沓然。”

抵偿
老虎欲吃猢狲,狲诳曰:“我身小,不足以供大嚼。前山有一巨兽,堪可饱餐,当引导前去。”同至山前,一角鹿见之,疑欲啖己,乃大喝云:“你这小猢狲,许我拿十二张虎皮送我,今只拿一张来,还有十一张呢?”虎惊遁,骂曰:“不信这小猢狲如此可恶,倒要拐我抵销旧帐!”

不利语
一翁无子,三婿同居,新造厅房一所。其长婿饮归,敲门不应,大骂:“牢门为何关得恁早!”翁怒,呼第二婿诉曰:“我此屋费过千金,不是容易挣的,出此不利之语,甚觉可恶。”次婿曰:“此房若卖也,只好值五百金罢了。”翁愈怒,又呼第三婿述之。三婿云:“就是五百金,劝阿伯卖了也罢,若然一场天火。连屁也不值。”

吹叭喇
乐人夜归,路见偷儿挖一壁洞,戏将叭喇插入吹起。内惊觉追赶,遇贼问云:“你曾见吹叭喇的么?”

戒狗肉
乞儿戒吃狗肉,众丐劝曰:“不必。”曰:“我不食之久矣。”众曰:“你便戒他,他却不戒你。”

病烂腿
一乞儿病腿烂,仰卧市中,狗见之欲餂。乞儿曰:“畜生,少不得是你口里食,何须这般性急?”

吃荇叶
清客贫甚,晨起无米,煮荇叶食之而出。少顷,赴富儿席,饮空心酒过多,遂大哕,而荇叶出焉。恐人嘲笑,乃指而言曰:“好古怪,早上吃白滚汤时,用不多几个莲心,如何一会子小荷叶出得恁快?”

书手
一人嫖院,饮酒过深,上床即鼾睡不醒,妓恐次日难索嫖钱,因而抚弄其阳。客既醒,问曰:“汝是何人?”妓曰:“李云卿的粗手。”其人曰:“理刑厅的书手,为何在此弄我的卵?”

滑吏
有快手,妻颇美。邻吏每欲调之不得,乃壁间凿一孔,俟其夫出,将阳物穿过而诱之。偶为快手瞧见,一把捏住不放。吏赞曰:“好快手。”吏以唾涂阳具,尽力一拔,遂缩回。快手亦赞曰:“好滑吏。”

做牌
有叩吏门者,妻曰:“出去了。你可是要做牌的么?留大些一个东道在我房里,任凭你要搁就搁,要捺就捺,要牒就牒,要销就销,要抽就抽,无有个做不来的。”

作仆
有投靠作仆者,自言:“一生不会横撑船,不肯缩退走,见饭就住的。”主人喜而纳之。一日,使捻河泥,辞曰:“说过不会横撑船。”又使其插秧,曰:“说过不会缩退走。”主人愤甚,伺其饭,辄连进不止,乃以“见饭就住”语责之。其人张口向主人曰:“请看喉咙内曾见饭否?”

戏改杜诗
有老妓年逾耳顺,犹强施膏沐,以媚少年。恐露白发,伪作良家妆束,以冠覆之。俗眼不辨,竟有为其所惑者。有名士于席间谈及,戏改杜诗一首,以嘲之云:“老去千秋强不宽,兴来今夜尽君欢。羞将短发还桃鬓,笑学良家也带冠。阴水似从千涧落,金莲高耸两峰寒。明年此际知谁在,醉抱鸡巴仔细看。”一时绝倒。亦凡主页

\part{}

僧道部

追度牒
一乡官游寺,问和尚:“吃荤否?”曰:“不甚吃,但逢饮酒时,略用些。”曰:“然则汝又饮酒乎?”曰:“不甚吃,但逢家岳妻舅来,略陪些。”乡官怒曰:“汝又有妻,全不像出家人的戒行,明日当对县官说,追你度牒。”僧曰﹔“不劳费心,三年前贼情事发,早已追去了。”

掠缘簿
和尚做功德回,遇虎,惧甚,以铙钹一片击之。复至,再投一片,亦如之。乃以经卷掠去,虎急走归穴。穴中母虎问故,答曰:“适遇一和尚无礼,只扰得他两片薄脆,就掠一本缘簿过来,不得不跑。

鬼王撒尿
大族出丧,路逢大雨,女眷人等,避于路傍檐下。和尚没处存身,暂躲开路神腹内。少顷,一僧从神腰里伸头探望,看雨住否。诸女眷惊曰:“我们回避,开路神要撒尿哩。”

发往酆都
有素不信佛事者,死后坐罪甚重。乃倾其冥资,延请僧鬼作功果,遍觅不得。问人曰:“此间固无僧乎?”曰:“来是来得多,都发往酆都了。”

开荤
师父夜谓沙弥曰:“今宵可干一素了。”沙弥曰:“何为素了?”僧曰:“不用唾者是也。”已而沙弥痛甚,叫曰:“师父,熬不得,快些开了荤罢。”

鸦噪
一士借僧房读书,忽闻鸦噪,连连叩齿。徒问:“相公为何?”答曰:“鸦噪。”徒曰:“我们丫燥,不是这等解法,是拓嚵吐的。”

忏悔
孝子忏悔亡父,僧诵普庵咒,至“南无佛佗耶”句,孝子喜曰:“正愁我爷难过奈何桥,多承佗过了。”乃出金劳之。僧曰:“若肯从重布施,连你娘等我也佗了过去吧”。

追荐
一僧追荐亡人,需银三钱,包送西方。有妇超度其夫者,送以低银。僧遂念往东方。妇不悦,以低银对,即算补之,改念西方。妇哭曰:“我的天,只为几分银子,累你跑到东又跑到西,好不苦呀。”

屁脬
一僧患大气脬,请医治之。医曰:“此症他人患之便可医,惟你出家人最难治。”问何以故,答曰:“这个大脬内,都是徒弟们的屁在里面。”

阳硬
或问和尚曰:“汝辈出家人,修炼参禅,夜间独宿,此物还硬否?”和尚曰:“幸喜一月止硬三次。”曰:“若如此大好?”和尚曰:“只是一件不妙,一硬就是十日。”

哭响屁
一人以幼子命犯孤宿,乃送出家,僧设酒款待。子偶撒一屁甚响,父不觉大恸。僧曰:“撒屁乃是常事,何以发悲?”父曰:“想我小儿此后要撒这个响屁,再不能勾了。”

闻香袋
一僧每进房,辄闭门口呼“亲肉心肝”不置。众徒俟其出,启钥瞷之,无他物,帷席下一香囊耳。众疑此有来历,乃去香,实以鸡粪。僧既归,仍闭门取香囊,且嗅且唤曰:“亲肉心肝呀,你怎么这等,莫非撒了一屁么?”|Qī|shu|ωang|

游方
头虱为足虱邀饮,值其人行房事,致被阻,观望久之方到。问:“何来迟?”曰:“不要说起。行至黑松林,遇一和尚甚奇,初时软弱郎当,有似怯病和尚﹔已而昂藏坚挺,竟似少林和尚﹔及其出入不休,好像当家和尚﹔忽然呕吐垂首,又像中酒和尚。”下虱曰:“究竟是甚和尚?”曰:“临了背着袱包就走,还是个游方和尚。”

桩粪
有买粪于寺者,道人索倍价。乡人讶之,道人曰:“此粪与他处不同,尽是师父们桩实落的,泡开来一担便有两担。”

僧赞僧
一秀才小便,和尚见之,大赞曰:“相公必然高中,生问:“何以知之?”僧曰:“适见龟头有痣。相书曰:‘龟头有痣终须发’,故以知之。”生曰:“你将来山门大兴,妙不可言。”僧问:“何以见得?”答曰:“若要佛法兴,除非僧赞僧。”

上下光
师号光明,徒号明光。客问:“贤师徒法号,如何分别?”徒答曰:“上头光是家师,下头光即是小僧。”

卖字
一妇游虎丘,手持素扇。山上有卖字者,每字索钱一文,妇止带有十八文求写。卖字者题曰:“美貌一佳人,胭脂点嘴唇。好像观音样,少净瓶。”子持扇,为馆师见之,问:“此扇何来?”子述以故。师曰:“被他取笑了。”因取十七文,看他如何写法。卖者即书云:“聪明一秀才,文章滚出来。一日宗师到,直呆。”生取扇含怒下山,途遇一僧,询知其故。僧曰:“待小僧去难他。”遂携十六文以往,写者题曰:“伶俐一和尚,好像如来样。睡到五更头,硬(音上)。”僧曰:“尾韵不雅,补钱四文,求你换过。”卖字曰:“既写,如何抹去?不若与你添上罢。”援笔写曰:“硬到大天亮。”

见和尚
有三人同行,途遇穿一破裤者。一友曰:“这好像猎户张豝。”一人曰:“不然,还似渔翁撒网。”又一人曰:“都不确,依我看来,好像一座多年破庙。”问:“为何?”答曰:“前也看见和尚,后也看见和尚。”

没骨头
秀才、道士、和尚三人,同船过渡。舟人解缆稍迟,众怒骂曰:“狗骨头,如何这等怠慢!”舟人忍气渡众下船,撑到河中,停篙问曰:“你们适才骂我狗骨头,汝秀才是甚骨头,讲得有理,饶汝性命,不然推下水去!”士曰:“我读书人攀龙附风,自然是龙骨头。”次问道士,乃曰:“我们出家人,仙风道骨,自然是神仙骨头。”和尚无可说得,乃慌哀告曰:“乞求饶恕,我这秃子,从来是没骨头的。”

和尚下爬
有浸苎麻于河埠者,被人窃去。适一妇人蹲倒涤衣,阴毛甚长,浸入河内,濯毕,带水而归。失苎者跟视水迹,疑是此妇偷去,骂詈不止。妇分辨不脱,怒将阴毛剪下,以火焚之。值邻家方在寻鸡声唤,忽闻隔壁毛臭,亦冤是他盗吃了。两边喊骂,受屈愈深。妇思多因此物遗祸,将刀连阴户挖出,抛在街心。值两公差拘提人犯回来,踹着此物,仔细端详,骇曰:“又是一桩人命了。怎么和尚的下爬,被人割落在这里。”

杜徐
一僧赴宴而归,人问:“坐第几席?”答曰:“首席是姓杜的,次席是姓徐的,杜徐之下,就是贫僧了。”

大家伙
一僧欲宿妓,苦无嫖钱,乃窃米一升而往。妓用大升量折,止存五合,嫌少不纳。僧复往窃升米与之,方许行事。僧愤恨,乃以头顶妓阴户。妓曰:“差了。”僧曰:“你把大家伙处我,我亦把大家伙弄你。”

小僧头
一僧宿娼,娼遽扳其头以就阴。僧曰:“非也,此小僧头耳。”娼意其嫌小,应曰:“尽勾了。”

倒挂
一士问僧云:“你看我腹中是甚么?”僧曰:“相公自然满腹文章在内。”士曰:“非也。”曰:“然则是五脏六腑乎?”士曰:“亦非也。”僧问何物,曰:“一肚皮和尚。若不信,现有一光头,挂出在里面。”

天报
老僧往后园出恭,误被笋尖搠入臀眼,乃唤疼不止。小沙弥见之,合掌云:“阿弥陀佛,天报。”

祭器
僧临终,嘱其徒曰:“享祀不须他物,只将你窟臀供座上足矣。”徒如命。方在祭献,听见有人叩门,忙应曰:“待我收拾了祭器就来。”

僧浴
僧见道家洗浴,先请师太,次师公,后师父,挨次而行,毫不紊乱。因感慨自叹曰:“独我僧家全无规矩,老和尚不曾下去,小和尚先脱得精光了。”

头眼
一僧与人对奕,因夺角不能成眼,躁甚头痒。乃手摩头顶而沉吟曰:“这个所在,有得一个眼便好。”

问秃
一秀才问僧人曰:“秃字如何写?”僧曰:“不过秀才的尾靶湾过来就是了。”

九思
一秀士每日往寺中听讲法,师问曰:“请教何谓‘君子有九思’?”士答曰:“都在人身上:头是三法司,耳是按察司,目是验封司,鼻是通政司,口是萦膳司,肚是尚宝司,手是提举司,足是行人司。”僧问:“还有一司?”生以手指阳物曰:“在这里。”僧问:“何司?”答曰:“僧纲司。”

当真取笑
和尚途行,一小厮叫曰:“和尚和尚,光头浪荡。”僧怒云:“一个筋头,翻在你娘肚上。”妇怒曰:“我家小厮,不过作耍,为何出此粗言?”僧曰:“娘娘,难道小僧当真,何须着急?

宿娼
一僧嫖院,以手摸妓前后,忽大叫曰:“奇哉,奇哉!前面的竟像尼姑,后面的宛似徒弟。”

僧道争儿
有僧道共偷一孀妇,有孕。及生子,僧道各争是他骨血,久之不决。子长,人问之,答曰:“我是和尚生的。”道士怒曰:“怎见得?”子曰:“我在娘胎里,只见和尚钻进钻出,并不曾见你道士。”

道士狗养
猪栏内忽产下一狗,事属甚奇。邻里环聚议曰:“道是(士)狗养的,又是猪的种,道是曰猪养的,又是狗的种。”

屄壳
一道士与妇人私,正行事,忽闻其夫叩门,道士慌甚,乃弃头上冠子在床而去。夫既登床,摸着道冠问曰:“此是何物?”妇急应曰:“此是我褪下的屄壳。”

入观
有无妻者,每放手铳,则以瓦罐贮精。久之精满,携出倾泼,乃对罐哭曰:“我的儿呀,只为你没娘,所以送你在罐里。”

跳墙
一和尚偷妇人,为女夫追逐,既跳墙,复倒坠。见地下有光头痕,遂捏拳印指痕在上,如冠子样,曰:“不怕道士不来承认。”

驱蚊
一道士自夸法术高强,撇得好驱蚊符。或请得以贴室中,至夜蚊虫愈多。往咎道士,道士曰:“吾试往观之。”见所贴符曰:“原来用得不如法耳。”问:“如何用法?”曰:“每夜赶好蚊虫,须贴在帐子里面。”

谢符
一道士过王府基,为鬼所迷,赖行人救之,扶以归。道士曰:“感君相救,无物可酬,有避邪符一道,聊以奉谢。”

祈雨
官命道士祈雨,久而不下,怪其身体不洁,亵渎神明,以致如此。乃尽拘小道,禁之狱中,令其无可掏摸。越数日,狱卒禀曰:“老道士祈雨,小道士求晴,如何得有雨下?”官问何故,狱卒曰:“他在狱念道:‘但愿一世不下雨,省得我们夜夜去熬疼。’”

养汉尼
有尼姑同一妓者,死见阎工。王问妓曰:“汝前世作何生理?”妓曰:“养汉接客。”王判云:“养汉接人,方便孤身,发还阳世,早去超生。”问尼姑:“你是何人?”答曰:“吃素念佛。”王亦判云:“吃素念经,佛口蛇心,一百竹片,打断脊筋。”尼哀告曰:“不瞒大王说,小妇人名虽是个尼姑,其实背地里养汉,做私窠子的。”

七字课
一学生聪颖,对答如流。师出两字课曰:“月明。”徒即对曰:“日出。”又云:“和尚。”答曰:“尼姑。”师曰:“青山。”徒曰:“白水。”又出一字曰:“去。”徒即应声曰:“来。”师又合串总念云:“月明和尚青山去。”徒亦答念对云:“日出尼姑白水来。”

几世修
一尼到一施主人家化缘,暑天见主人睡在醉翁椅上,露出阳物甚伟。进对主家婆曰:“娘娘,你几世上修来的,如此享用。”主婆曰:“阿弥陀佛,说这样话。”尼曰:“这还说不修哩。」

\part{}

贪吝部

开当
有慕开典铺者,谋之人曰:“需本几何?”曰:“大典万金,小者亦须千计。”其人大骇而去。更请一人问之,曰:“百金开一钱当亦可。”又辞去。最后一人曰:“开典如何要本钱,只须店柜一张,当票数纸足矣。”此人乃欣然。择期开典,至日,有持物来当者,验收讫,填空票计之。当者索银,答曰:“省得称来称去,费坏许多手脚,待你取赎时,只将利银来交便了。”

请神
一吝者,家有祷事,命道士请神,乃通城请两京神道。主人曰:“如何请这远的?”道士答曰:“近处都晓得你的情性,说请他,他也不信。”

好放债
一人好放债,家已贫矣,止余斗粟,仍谋煮粥放之。人问“如何起利?”答曰:“讨饭。”

大东道
好善者曰:“闻当日佛好慈悲,曾割肉喂鹰,投崖喂虎。我欲效之,但鹰在天上,虎在山中,身上有肉,不能使啖,夏天蚊子甚多,不如舍身斋了蚊罢。”乃不挂帐,以血饲蚊。佛欲试其虔诚,变一虎啖之。其人大叫曰:“小意思吃些则可,若认真这样大东道,如何当得起!”

打半死
一人性最贪,富者语之曰:“我白送你一千银子,你与我打死了罢。”其人沉吟良久,曰:“只打半死,与我五百两何如?”

命穷
乡下亲家新制佳酿,城里亲家慕而访之,冀其留饮。适亲家他往,亲母命子款待,权为荒榻留宿。其亲母卧房止隔一壁,亲家因未得好酒到口,方在懊闷。值亲母桶上撤尿,恐声响不雅,努力将臀夹紧,徐徐滴沥而下。亲家听见,私自喜曰:“原来才在里面滤酒哩,想明早得尝其味矣。”亲母闻言,不觉失笑,下边松动,尿声急大。亲家拍掌叹息曰:“真是命穷,可惜滤酒榨袋,又撑破了。”

兄弟种田
有兄弟合种田者,禾既熟。议分。兄谓弟曰:“我取上截,你取下截。”弟讶其不平,兄曰:“不难,待明年你取上,我取下可也。”至次年,弟催兄下谷种,兄曰:“我今年意欲种芋头哩。”

合伙做酒
甲乙谋合本做酒,甲谓乙曰:“汝出米,我出水。”乙曰:“米若我的,如何算帐?”甲曰:“我决不亏心。到酒熟时,只逼还我这些水罢了,其余多是你的。”

翻脸
穷人暑月无帐,复惜蚊烟费,忍热拥被而卧,蚊囋其面。邻家有一鬼脸,借而带之。蚊口不能入,谓曰:“汝不过省得一文钱耳,如何便翻了脸?”

画像
一人要写行乐图,连纸笔颜料,共送银二分。画者乃用水墨于荆川纸上,画出一背像。其人怒曰:“写真全在容颜,如何写背?”画者曰:“我劝你莫把面孔见人罢。”

许日子
一人性极吝啬,从无请客之事。家僮偶持碗一篮,往河边洗涤,或问曰:“你家今日莫非宴客耶?”僮曰:“要我家主人请客,除非那世里去!”主人知而骂曰:“谁要你轻易许下他日子!”

醵金
有人遇喜事,一友封分金一星往贺,乃密书对内云:“现五分,赊五分。”己而此友亦有贺分,其人仍以一星之敬答之。乃以空封往,内书云:“退五分,赊五分。”

携灯
有夜饮者,仆携灯往候,主曰:“少时天便明,何用灯为?”仆乃归。至天明,仆复往接,主责曰:“汝大不晓事,今日反不带灯来,少顷就是黄昏,叫我如何回去?”

不留客
客远来久坐,主家鸡鸭满庭,乃辞以家中乏物,不敢留饭。客即借刀,欲杀己所乘马治餐。主曰:“公如何回去?”客曰:“凭公于鸡鸭中,告借一只,我骑去便了。”

不留饭
一客坐至晌午,主绝无留饭之意。适闻鸡声,客谓主曰:“昼鸡啼矣。”主曰:“此客鸡不准。”客曰:“我肚饥是准的。”

射虎
一人为虎衔去,其子执弓逐之,引满欲射。父从虎口遥谓其子曰:“我儿须是兜脚射来,不要伤坏了虎皮,没人肯出价钱。”

吃人
一人远出回家,对妻云:“我到燕子矶,蚊虫大如鸡。后过三山硖,蚊虫大如鸭。昨在上新河,蚊虫大如鹅。”妻云:“呆子,为甚不带几只回来吃。”夫笑曰:“他不吃我就勾了,你还敢想去吃他!”

悭吝
一人性最悭吝,忽感痨瘵之疾,医生诊视云:“脉气虚弱,宜用人参培补。”病者惊视曰:“力量绵薄,惟有委命听天可也。”医士曰:“参既不用,须以熟地代之,其价颇贱。”病者摇首曰:“费亦太过,愿死而已。”医知其吝啬,乃诈言曰:“别有一方,用干狗屎调黑糖一二文服之,亦可以补元神。”病者跃然起问曰:“不知狗屎一味,可以秃用否?”

卖粉孩
一人做粉孩儿出卖,生意甚好,谓妻曰:“此后只做束手的,粉可稍省。”果卖去。又曰:“此后做坐倒的,当更省。”仍卖去。乃曰:“如今做垂头而卧者,不更省乎!”及做就,妻提起看曰:“省则省矣,只是看看不像个人了。”

独管裤
一人谋做裤而吝布,连唤裁缝,俱以费布辞去。最后一缝匠云:“只须三尺足矣。”其人大喜,买布与之。乃缝一脚管,令穿两足在内。其人曰:“迫甚,如何行得?”缝匠曰:“你脱煞要省,自然一步也行不开的。”

莫想出头
一人性吝者,买布一丈,命裁缝要做马衣一件,裤一条,袜一双,余布还要做顶包巾。匠每以布少辞去。落后一裁缝曰:“我做只消八尺,倒与你省却两尺,何如?”其人大喜。缝者竟做成一长袋,将此人从脚套至头顶,口用绳收紧。其人曰:“气闷极矣。”匠曰:“撞着你这悭吝鬼,自然是气闷的。省是省了,要想出头,却难哩。”

一毛不拔
一猴死见冥王,求转人身。王曰:“既欲做人,须将身上毛尽行拔去。”即唤夜叉动手。方拔一根,猴不胜痛楚,王笑曰:“畜生,看你一毛不拔,如何做人!”

因小失大
有造方便觅利者,遥见一人撩衣,知必小解,恐其往所对邻厕,乃伪为出恭,而先踞其上。小解者果赴己厕。其人不觉,偶撒一屁,带下粪来,乃大悔恨,曰:“何苦因小失大。”

七德
一家延师,供馔甚薄。一日,宾主同坐,见篱边一鸡,指问主人曰:“鸡有几德?”主曰:“五德。”师曰:“以我看来,鸡有七德。”问:“为何多了二德?”答曰:“我便吃得,你却舍不得。”

粪鸡
东家供师甚薄,久不买荤。一日,粪缸内淹死一鸡,烹以为馔。师食而疑之,问其徒,徒以实告,师愤甚。少顷,主人进馆,师忙执笤帚二把,塞其口中,逼使尽食。东家曰:“笤帚如何吃得?”师曰:“你既不肯吃笤帚,如何倒叫先生吃粪鸡(箕)。”

恶神
一神道险恶,赛者必用生人祭祷。有酬愿者,苦乏人献,特于供桌中挖一孔,藏身在桌下,而伸头于桌面。俟神举箸,头忽缩下。神大怒,骂曰:“这班小鬼都是贼,才得举箸,如何嗄饭就一些没有了。”

下饭
二子午餐,问父用何物下饭,父曰:“古人望梅止渴,可将壁上挂的腌鱼望一望,吃一口,这就是下饭了。”二子依法行之。忽小者叫云:“阿哥多看了一眼。”父曰:“咸杀了他。”

吃榧伤心
有担榧子在街卖者,一人连吃不止。卖者曰:“你买不买,如何只管吃?”答曰:“此物最能养脾。”卖者曰:“你虽养脾,我却伤心。”

一味足矣
一先生开馆,东家设宴相待,以其初到加礼,乃宰一鹅奉款。饮至酒阑,先生谓东翁曰:“学生取扰的日子正长,以后饮馔·毋须从俭,庶得相安。”因指盘中鹅曰:“日日只此一味足矣,其余不必罗列。”

卖肉忌赊
有为儿孙作马牛者,临终之日,呼诸子而问曰:“我死后,汝辈当如何殡殓?”长子曰:“仰体大人惜费之心,不敢从厚,缟衣布衾,二寸之棺,一寸之椁,墓道仅以土封。”翁攒眉良久,责其多费。次子曰:“衣衾棺椁,俱不敢用,但具蒿荐一条,送于郊外,谓之火葬而已。”翁犹疾其过奢。三子嘿喻父意,乃诡词以应曰:“吾父爱子之心,无所不至,既经殚力于生前,并惜捐躯于死后?不若以大人遗体,三股均分,暂作一日之屠儿,以享百年之遗泽,何等不好?”翁乃大笑曰:“吾儿此语,适获我心。”复戒之曰:“对门王三老,惯赖肉钱,断断不可赊。”

咬嚼不过
一人死后,转床殡殓,诸亲及众妇绕灵而哭。只见孝帏裂碎,到处飞扬,皆称怪象。特往关魂问之,乃曰:“无他,只是当众人咬嚼不过耳。”

醮酒
有性吝者,父子在途,每日沽酒一文,虑其易竭,乃约用箸头醮尝之。其子连醮二次,父责之曰:“如何吃这般急酒!”

吞杯
一人好饮,偶赴席,见桌上杯小,遂作呜咽之状。主人惊问其故,曰:“睹物伤情耳。先君去世之日,并无疾病,因友人招饮。亦似府上酒杯一般,误吞入口,咽死了的。今日复见此杯,焉得不哭?”

好酒
父子扛酒一坛,路滑跌翻。其父大怒,子乃伏地痛饮,抬头谓父曰:“决些来么,难道你还要等甚菜?”

恋席
客人恋席,不肯起身。主人偶见树上一大鸟,对客曰:“此席坐久,盘中肴尽,待我砍倒此树,捉下鸟来,烹与执事侑酒,何如?”客曰:“只恐树倒鸟飞矣。”主云:“此是呆鸟,他死也不肯动身的。”

恋酒
一人肩挑磁壶,各处货卖。行至山间,遇着一虎,咆哮而来。其人怆甚,忙将一壶掷去,其虎不退。再投一壶,虎又不退。投之将尽,止存一壶,乃高声大喊曰:“畜生,畜生!你若去,也只是这一壶。你就不去,也只是这一壶了!”

四脏
一人贪饮过度,妻子私相谋议曰:“屡劝不听,宜以险事动之。”一日,大饮而哕,子密袖猪膈置哕中,指以谓曰:“凡人具五脏,今出一脏矣,何以生耶?”父熟视曰:“唐三藏尚活世,况我有四脏乎!”

寡酒
一人以寡酒劝客,客曰:“不如拿把刀来杀了我罢。”主愕然,问曰:“劝酒无非好意,何出此言?”客曰:“其实当你寡不过了。”

白伺候
夜游神见门神夜立,怜而问之曰:“汝长大乃尔,如何做人门客,早晚伺候,受此苦辛?”门神曰:“出于无奈耳。”曰:“然则有饭吃否?”答:“若要他饭吃时,又不要我上门了。”

梦戏酌
一人梦赴戏酌,方定席,为妻惊醒,乃骂其妻。妻曰:“不要骂,趁早睡去,戏文还未半本哩。”

梦美酒
一好饮者,梦得美酒。将热而饮之,忽被惊醒,乃大悔曰:“早知如此,恨不冷吃。”

截酒杯
使僮斟酒不满,客举杯细视良久,曰:“此杯太深,当截去一段。”主曰:“为何?”客曰:“上半段盛不得酒,要他何用?”

切薄肉
主有留客定饭,仅用切肉一碗,既嚣且少。乃作诗以诮之,曰:“君家之刀利且锋,君家之手轻且松。切来片片如纸同,周围披转无二重。推窗忽遇微小风,顿然吹入五云中。忙忙令人觅其踪,已过巫山十二峰。”

满盘多是
客见坐上无肴,乃作意谢主人,称其大费。主人曰:“一些菜也没有,何云大费?”客曰:“满盘都是。”主人曰:“菜在那里?”客指盘中曰:“这不是菜,难道到是肉不成?”

滑字
一家延师,供膳菲薄。时值天雨,馆僮携午膳至,肉甚少,师以其来迟,欲责之。僮曰:“天雨路滑故也。”师曰:“汝可写滑字我看,如写得出,便饶你打。”僮曰:“一点儿,一点儿,又是斜披一点儿,其余都是骨了。”

不见肉
一母命子携萝卜一篮,往河边洗涤。久之不归,母往寻之,但存萝卜。知儿失足堕河,淹死水中,因大哭曰:“我的肉,我的肉,但见萝卜不见肉。”

和头多
有请客者,盘飧少而和头多,因嘲之曰:“府上的食品,忒煞富贵相了。”主问:“何以见得?”曰:“葱蒜萝卜,都用鱼肉片子来拌的。少刻鱼肉上来,一定是龙肝凤髓做和头了。”

盛骨头
一家请客,骨多肉少。客曰:“府上的碗想是偷来的?”主人骇曰:“何出此言?”客曰:“我只听见人家骂说:‘偷我的碗,拿去盛骨头。’”

收骨头
馆僮怪主人每食必尽,只留光骨于碗,乃对天祝曰:“愿相公活一百岁,小的活一百零一岁。”主问其故,答曰:“小人多活一岁,好收拾相公的骨头。”

涂嘴
或有宴会,座中客贪馋不已,肴使既尽。馆僮愤怒而不敢言,乃以锅煤涂满嘴上,站立傍侧。众人见而讶之,问其嘴间何物。答曰:“相公们只顾自己吃罢了,别人的嘴管他则甚。”

索烛
有与善啖者同席,见盘中且尽,呼主翁拿烛来。主曰:“得无太早乎?”曰:“我桌上已一些不见了。”

借水
一家请客,失分一箸。上菜之后,众客朝拱举箸,其人独抻手而观。徐向主人曰:“求赐清水一碗。”主问曰:“何处用之?”答曰:“洗干净了指头,好拈菜吃。”

善求
有作客异乡者,每入席,辄狂啖不已。同席之人甚恶之,因问曰:“贵处每逢月食,如何护法?”答曰:“官府穿公服群聚,率军校侍兵击鼓为对,俟其吐出始散。”其人亦问同席者曰:“贵乡同否?”答曰:“敝处不然,只是善求。”问:“如何求法?”曰:“合掌了手,对黑月说道:‘阿弥陀佛,脱煞凶了,求你省可吃些,剩点与人看看罢。’”

好啖
甲好啖,手不停箸,问乙曰:“兄如何箸也不动?”乙还问曰:“兄如何动也不住?”

同席不认
有客馋甚,每人座,辄餮饕不已。一日,与之同席,自言曾会过一次,友曰:“并未谋面,想是老兄错认了。”及上菜后,啖者低头大嚼,双箸不停。彼人大悟,曰:“是了,会便会过一次,因兄只顾吃菜,终席不曾抬头,所以认不得尊容,莫怪莫怪。”

喜属犬
一酒客讶同席者饮啖太猛,问其年,以属犬对。客曰:“早是犬,若属虎的,连我也都吃下肚了。”

问肉
一人与瞽者同席,先上东坡肉一碗,瞽者举箸即拑而啖之。同席者恶甚。少焉复来捞取,盘中已空如也。问曰:“肉有几块?”其人愤然答曰:“九块。”瞽者曰:“你到吃了八块么。”

吃黄雀
两人共席而饮,碗内有黄雀四只,一人贪食其三,谓同席者曰:“兄何不用?”其人曰:“索性放在兄腹中,省得他们拆了对?”

啖馄饨
一妻病,夫问曰:“想甚吃否?”妻曰:“除非好肉馄饨,想吃一二只。”夫为治一盂,意欲与妻同享,方往取箸回,而妻已染指啖尽,止余其一。夫曰:“何不并啖此枚?”妻攒眉曰:“我若吃得下此只,不害这病了。”

罚变蟹
一人见冥王,自陈一生吃素,要求个好轮回。王曰:“我那里查考,须剖腹验之。”既剖,但见一肚馋涎。因曰:“罚你去变一只蟹,依旧吐出了罢。”

不吃素
一人遇饿虎,将遭啖。其人哀恳曰:“圈有肥猪,愿将代己。”虎许之,随至其家。唤妇取猪喂虎,妇不舍曰:“所有豆腐颇多,亦堪一饱。”夫曰:“罢么,你看这样一个狠主客,可是肯吃素的么?”

酒煮滚汤
有以淡酒宴客者,客尝之,极赞府上烹调之美。主曰:“粗肴未曾上桌,何以见得?”答曰:“不必论其它,只这一味酒煮白滚汤,就妙起了。”

淡酒
有人宴客用淡酒者,客向主人索刀。主问曰:“要他何用?”曰:“欲杀此壶。”又问:“壶何可杀?”答曰:“杀了他,解解水气。”

淡水
河鱼与海鱼攀亲,河鱼屡往,备扰海错。因语海鱼:“亲家,何不到小去处下顾一顾?”海鱼许焉。河鱼归曰:“海头太太至矣。”遣手下择深港迎之。海鱼甫至港口便返,河鱼追问其故,答曰:“我吃不惯贵处这样淡水。”

索米
一家请客,酒甚淡。客曰:“肴馔只此足矣,倒是米求得一撮出来。”主曰:“要他何用?”答曰:“此酒想是不曾下得米,倒要放几颗。”

酒死
一人请客,客方举杯,即放声大哭。主人慌问曰:“临饮何故而悲?”答曰:“我生平最爱的是酒,今酒已死矣,因此而哭。”主笑曰:“酒如何得死?”客曰:“既不曾死,如何没有一些酒气?”

送君代酒
一客访客,主人不留饮食,起送出门,谓客曰:“古语云:‘远送当三杯’,待我送君里许。”恐客留滞,急拽其袖而行。客曰:“求从容些,量浅,吃不得这般急酒。」

\part{}
贪窭部

好古董
一富人酷嗜古董,而不辨真假。或伪以虞舜所造漆碗。周公挞伯禽之杖,与孔子杏坛所坐之席求售,各以千金得之。囊资既空,乃左执虞舜之碗,右持周公之杖,身披孔子之席,而行乞于市,曰:“求赐太公九府钱一文。”

不奉富
千金子骄语人曰:“我富甚,汝何得不奉承?”贫者曰:“汝自多金子,我何与而奉汝耶?”富者曰:“倘分一半与汝何如?”答曰:“汝五百,我五百,我汝等耳,何奉焉?”又曰:“悉以相送,难道犹不奉我?”答曰:“汝失千金,而我得之,汝又当趋奉我矣。”

穷十万
富翁谓贫人曰:“我家富十万矣。”贫人曰:“我亦有十万之蓄,何足为奇。”富翁惊问曰:“汝之十万何在?”贫者曰:“你平素有了不肯用,我要用没得用,与我何异?”

止一物
穷汉闻邻家喊捉贼,忙将阳物插妻牝内。妻曰:“贼至有何高兴?”答曰:“止此一物,藏好了,怕他怎么?”

失火
一穷人正在欢饮,或报以家中失火。其人即将衣帽一整,仍坐云:“不妨,家当尽在身上矣。”或曰:“令正却如何?”答曰:“他怕没人照管?”

夹被
暑月有拥夹被卧者,或问其故,答曰:“阿哟,绵被脱热。”

金银锭
贫子持金银锭行于街市,顾锭叹曰:“若得你硬起来,我就好过日子了。”傍人待答曰:“要我硬却不能勾,除非你硬了凑我。”

妻掇茶
客至乏人,大声讨茶,妻无奈,只得自送茶出。夫装鼾摚,乃大喝云:“你家男个那里去了?”

唤茶
一家客至,其夫唤茶不已。妇曰:“终年不买茶叶,茶从何来?”夫曰:“白滚水也罢。”妻曰:“柴没一根,冷水怎得热?”夫骂曰:“狗淫妇!难道枕头里就没有几根稻草?”妻回骂曰:“臭忘八!那些砖头石块,难道是烧得着的!”

留茶
有留客吃茶者,苦无茶叶,往邻家借之。久而不至,汤滚则溢,以冷水加之。既久,釜且满矣,而茶叶终不得。妻谓夫曰:“茶是吃不成了,不如留他洗个浴罢。”

怕狗
客至乏仆,暗借邻家小厮掇茶。至客堂后,逡巡不前,其人厉声曰:“为何不至?”僮曰:“我怕你家这只凶狗。”

食粥
一人家贫,每日省米吃粥。怕人耻笑,嘱子讳之,人前只说吃饭。一日,父同友人讲话,等久不进,子往唤曰:“进来吃饭。”父曰:“今日手段快,缘何煮得恁早?”子曰:“早到不早,今日又熬了些清汤。”

鞋袜讦讼
一人鞋袜俱破,鞋归咎于袜,袜又归咎于鞋,交相讼之于官。官不能决,乃拘脚跟证之。脚跟曰:“小的一向逐出在外,何由得知?”

被屑挂须
贫家盖蒿荐,幼儿不知讳,父挞而戒之曰:“后有问者,但云盖被。”一日父见客,而须上带荐草,儿从后呼曰:“爹爹,且除去面上被屑着?”

吃糟饼
一人家贫而不善饮,每出啖糟饼二枚,便有酣意。适遇友人问曰:“尔晨饮耶?”答曰:“非也,吃糟饼耳。”归以语妻,妻曰:“呆子,便说吃酒,也妆些体面。”夫颔之。及出,仍遇此友,问如前,以吃酒对。友诘之:“酒热吃乎?冷吃乎?”答曰:“是熯的。”友笑曰:“仍是糟饼。”既归,而妻知之,咎曰:“汝如何说熯,须云热饮。”夫曰:“我知道了。”再遇此友,不待问即夸云:“我今番的酒,是热吃的?”友问曰:“你吃几何?”其人伸手曰:“两个。”

烧黄熟
清客见东翁烧黄熟香,辄掩鼻不闻,以其贱而不屑用也。主人曰:“黄熟虽不佳,还强似府上烧人言、木屑。”清客大诧曰:“我舍下何曾烧这两件?”主人曰:“蚊烟是甚么做的?”

拉银会
有人拉友作会,友固拒之不得,乃曰:“汝若要我与会,除是跪我。”其人即下跪,乃许之。傍观者曰:“些须会银,左右要还他的,如此自屈,吾甚不取。”答曰:“我不折本的,他日讨会钱,跪还我的日子正多哩。”

兑会钱
一人对客,忽转身曰:“兄请坐,我去兑还一主会银,就来奉陪。”才进即出,客问:“何不兑银?”其人笑曰:“我曾算来,他是痴的,所以把会银与我。我若还他,也是痴的了。”

剩石沙
一穷人留客吃饭,其妻因饭少,以鹅卵石衬于添饭之下。及添饭既尽,而石出焉。主人见之愧甚,乃责仆曰:“瞎眼奴才,淘米的时节,眼睛生在那里?这样大石沙,都不拿来拣出。”

饭粘扇
一人不见了扇子,,骂曰:“拿我的扇子,去做羹饭!”傍人曰:“扇子如何做得羹饭?”其人曰:“你不晓得,我的扇子,糊掇许多饭粘在上面。”

没屪
穷人好妆体面,偶出访友,乏人跟随,令妻男妆以代仆。及至友家,闲谈至暮,遂留宿焉。因铺陈未备,主伴主,而仆伴仆,各睡一处。穷人解衣上床,下身无裤,次日起身后,主人叹曰:“好笑这朋友,穷得裤子也无,只穿一件单布麻裙。”仆在傍曰:“这还算好,不像他管家,竟穷得屪子都精光。”

破衣
一人衣多破孔,或戏之曰:“君衣好像棋盘,一路一路的。”其人笑曰:“不敢欺,再着着,还要打结哩。”

借服
有居服制而欲赴喜筵者,借得他人一羊皮袄,素冠而往。人知其有服也,因问:“尊服是何人的?”其人见友问及,以为讥诮其所穿之衣,乃遽视己身作色而言曰:“是我自家的,问他怎么?”

连三拐
一人三餐无食,夫妻枵腹上床。妻嗟叹不已,夫曰:“我今夜连要打三个拐,以当三餐。”妻从之。次早起来,头晕眼花,站脚不住,谓妻曰:“此事妙极,不惟可以当饭,且可当酒。”

酒瓮盛米
一穷人积米三四瓮,自谓极富。一日,与同伴行市中,闻路人语曰:“今岁收米不多,止得三千余石。”穷人谓其伴曰:“你听这人说谎,不信他一分人家,有这许多酒瓮。”

遇偷
偷儿入贫家,遍摸无一物,乃唾地开门而去。贫者床上见之,唤曰:“贼,有慢了,可为我关好了门去。”偷儿曰:“你这样人,亏你还叫我贼!我且问你,你的门关他做甚么?”

被贼
穿窬入一贫家,其家止蓄米一瓮,置卧床前。偷儿解裙布地,方取瓮倾米,床上人窃窥之,潜抽其裙去,急呼“有贼”。贼应声曰:“真个有贼,刚才一条裙在此,转眼就被贼屄养的偷去了。”

羞见贼
穿窬往窃一家,见主人向外而睡,忽转朝里。贼疑其素有相识,欲遁去。其人大呼曰:“来不妨,因我家乏物可敬,无颜见你啰。”

望包荒
贫士素好铺张,偷儿夜袭之,空如也,唾骂而去。贫士摸床头数钱,追赠之,嘱曰:“君此来,虽极怠慢,然在人前尚望包荒。”

借债
有持券借债者,主人曰:“券倒不须写,只画一幅行乐图来。”借者问其故,答曰:“怕我日后讨债时,便不是这副面孔耳。”

变爷
一贫人生前负债极多,死见冥王。王命鬼判查其履历,乃惯赖人债者,来世罚去变成犬马,以偿前欠。贫者禀曰:“犬马之报,所偿有限,除非变了他们的亲爷,方可还得。”王问何故,答曰:“做了他家的爷,尽力去挣,挣得论千论万,少不得都是他们的。”

梦还债
欠债者谓讨债者曰:“我命不久矣,昨夜梦见身死。”讨者曰:“阴阳相反,梦死反得生也。”欠债者曰:“还有一梦。”问曰:“何梦?”曰:“梦见还了你的债。”

说出来
一人为讨债者所逼,乃发急曰:“你定要我说出来么!”讨债者疑其发己心病,嘿然而去。如此数次。一日发狠曰:“由你说出来也罢,我不怕你。”其人又曰:“真个要说出来?”曰:“真要你说。”曰:“不还了!”

坐椅子
一家索债人多,椅凳俱坐满,更有坐槛上者。主人私谓坐槛者云:“足下明日来早些。”那人意其先完己事,乃大喜,遂扬言以散众人。次早黎明即往,叩其相约之意。答曰:“昨日有亵坐槛,甚是不安,今日早来,可占把交椅。”

扛欠户
有欠债屡索不还者,主人怒,命仆辈潜伺其出,扛之以归。至中途,仆暂歇息,其人曰:“快走罢,歇在这里,又被别人扛去,不关我事。”

拘债精
冥王命拘蔡青,鬼卒误听,以为勾债精也,遂摄一欠债者到案。王询之,知其谬,命鬼卒放回。债精曰:“其实不愿回去。阳间无处藏身,正要借此处一躲。”

摆海干
一人专好放生,龙王感之,命夜叉赠一宝钱,嘱曰:“此钱名为摆海干,教他把此钱在海中一摆,海水即干,任将金银宝贝拿去。”夜叉使命付讫。其人日日将钱去摆,遂成大富。后把此钱失去,贪心未足,只将空手海上去摆。一日,撞着夜叉,夜又曰:“你手内钱都没了,还有何脸面,在此摆甚么?」

\part{}
讥刺部

搬是非
寺中塑三教像,先儒,次释,后道。道士见之,即移老君于中。僧见,又移释迦于中。士见,仍移孔子于中。三圣自相谓曰:“我们原是好好的,却被这些小人搬来搬去搬坏了。”

丈人
有以岳丈之力得魁选行者,或为语嘲之曰:“孔门弟子入试,临揭晓,闻报子张第九。众曰:‘他一貌堂堂,果有好处。’又报子路第十三,众曰:‘这粗人到也中得高,还亏他这阵气魄好。’又报颜渊第十二,众曰:‘他学问最好,屈了他些。’又报公冶长第五,大家骇曰:‘那人平时不见怎的,为何倒中在前?’一人曰:‘他全亏有人扶持,所以高掇。’问:‘谁扶持他?’曰:‘丈人。’”

大爷
一人牵牛而行,喝人让路,不听,乃云:“看你家爷来。”一人回视曰:“难道我家有这样一个大爷?”

接风送程
一人往苏州娶得一妾,唤名苏娘。后又往杭州娶了一妾,就取名杭娘。其妻立下规矩:每到苏、杭身边去,必要投批挂号,先与他干讫一度,方许前行,名为送程。及轮该自晚,与夫交合,又名为接风。其夫苦于奔命,愿请独宿。一日,妻兴忽发,乃劝夫往苏、杭去。夫笑曰:“我苏、杭到也要去,只是当你接风、送程不起。”

苏杭同席
苏、杭人同席,杭人单吃枣子,而苏人单食橄榄。杭问苏曰,“橄榄有何好处,而兄爱吃他?”曰:“回味最佳。”杭人曰:“等得你回味好,我已甜过半日了。”

狗衔锭
狗衔一银锭而飞走,人以肉喂他不放,又以衣罩去,复甩脱。人谓狗曰:“畜生,你直恁不舍,既不爱吃,复不好穿,死命要这银子何用?”

不停当
有开当者,本钱甚少。初开之月,招牌写一“当”字。未几,本钱发尽,赎者不来,乃于“当”字之上,写一“停”字,言停当也。及后赎者再来,本钱复至,又于“停”字之上,加一“不”字。人见之曰:“我看你这典铺中,实实有些不停当了。”

和事
一夫妇反目,夜晚上床,夫以手摸其阴,妻推开曰:“手是日间打我的,不要来。”夫与亲嘴,又推开曰:“口是日间骂我的,不要来。”及将阳物插入阴户中,妇不之拒。夫问曰:“口与手,你甚怪他,独此物不拒,何也?”妇曰:“他不曾得罪我。往常争闹了,全亏他做和事老人,自然由他出入。”

朝奉
徽人狎妓,卖弄才学,临行事,待要说一成语切题。乃舒妓两股,以其阴对己之阳曰:“此丹凤朝阳也。”妓亦以徽人之阳对己之阴,徽人问曰:“此何故事?”妓曰:“这叫做卵袋朝奉。”

十只脚
关吏缺课,凡空身人过关,亦要纳税,若生十只脚者免。初一人过关无钞,曰:“我浙江龙游人也。龙是四脚,牛是四脚,人两脚,岂非十脚?”许之。又一人求免税曰:“我乃蟹客也。蟹八脚,我两脚,岂非十脚?”亦免之。末后一徽商过关,竟不纳税。关吏怒欲责之,答曰:“小的虽是两脚,其实身上之脚还有八只。”官问:“那里?”答曰:“小的徽人,叫做徽獭猫。猫是四脚,獭又四脚,小的两脚,岂不共是十只脚?”

亲家公
有见少妇抱小儿于怀,乃讨便宜曰:“好个乖儿子。”妇知其轻薄,接口曰:“既好,你把女儿送他做妻子罢。”其人答曰:“若如此,你要叫我亲──家公了。”

中人
玉帝修凌霄殿,偶乏钱粮,欲将广寒宫典与下界人皇。因思中人亦得一皇帝便好,乃请灶君皇帝下界议价。既见朝,朝中人讶之曰:“天庭所遣中人,何黑如此?”灶君笑曰:“天下中人,那有是白做的!”

媒人
有懮贫者,或教之曰:“只求媒人足矣。”其人曰:“媒安能疗贫乎?”答曰:“随你穷人家,经了媒人口,就都发迹了!”

表号
一富翁不通文墨,有借马者柬云:“偶欲他出,告假骏足一乘。”翁大怒曰:“我便是一双足,如何借得?”傍友代解曰:“所谓骏足者,马之称号也。”翁乃大笑曰:“不信畜生也有表号。”

精童
有好外者,往候一友。友知其性,呼曰:“唤精童具茶。”已而献茶者,乃一奇丑童子也。其人曰:“似此何名精童?”友白:“正惟一些人(音银)气也无得。”

相称
一俗汉造一精室,室中罗列古玩书画,无一不备。客至,问曰:“此中若有不相称者,幸指教,当去之。”客曰:“件件俱精,只有一物可去。”主人问:“是何物?”客曰:“就是足下。”

看扇
有借佳扇观者,其人珍惜,以绵紬衫衬之。扇主看其袖色不堪,谓曰:“倒是光手拿着罢。”

性不饮
一人以酒一瓶、腐一块,献利市神。祭毕,见狗在傍,速命童子收之。童方携酒入内,腐已为狗所啖。主怒曰:“奴才!你当收不收,只应先收了豆腐。岂不晓得狗是从来不吃酒的!”

担鬼人
钟馗专好吃鬼,其妹送他寿礼,帖上写云:“酒一坛,鬼两个,送与哥哥做点剁。哥哥若嫌礼物少,连挑担的是三个。”钟馗看毕,命左右将三个鬼俱送庖人烹之。担上鬼谓挑担鬼曰:“我们死是本等,你却何苦来挑这担子?”

鬼脸
阎王差鬼卒拘三人到案,先问第一个:“你生前作何勾当?”答去:“缝连补缀。”王曰:“你迎新弃旧,该押送油锅。”又问第二个:“你作何生理?”答曰:“做花卖。”王曰:“你节外生枝,发在油锅。”再问第三个,答曰:“糊鬼脸。”王曰:“都押到油锅去。”其人不服,曰:“我糊鬼脸,替大王张威壮势,如何同犯此罪?”王曰,“我怪你见钱多的,便把好脸儿与他,那钱少的,就将歹脸来欺他。”

牙虫
有患牙疼者,无法可治。医者云:“内有巨虫一条,如桑蚕样,须捉出此虫,方可断根。”问:“如何就有恁大?”医曰:“自幼在牙(衙)门里吃大,最能伤人。”

狗肚一句
新官到任,吏跪献鲫鱼一尾,其味佳美,大异寻常。官食后,每思再得,差役遍觅无有。仍向前吏索之,吏禀曰:“此鱼非市中所买。昨偶宰一狗,从狗肚中得者,以为异品,故敢上献。”官曰:“难道只有此鲫了?”吏曰:“狗肚里焉得有第二鲫(句)。”

吃粮披甲
一耗鼠在阴沟内钻出,近视者睨视良久,曰:“咦!一个穿貂裘的大老官。”鼠见人随缩入。少刻,又一大龟从洞内扒出,近视曰:“你行穿貂袄的主儿才得进去,又差出个披甲兵儿来了。”

卵穿嘴上
一女无故而腹中受孕,父母严诘其故,女曰:“并无外遇,止有某日偶遇某人对面而来,嘴上撞了一下,遂尔成胎。此外别无他事。”父沉吟良久,忽悟曰:“嗄,我晓得了,这人的卵袋,竟穿在嘴上的。”

风流不成
有嫖客钱尽,鸨儿置酒饯之。忽雨下,嫖客叹曰:“雨落天留客,天留人不留。”鸨念其撒钱,勉留一宿。次日下雪复留。至第三日风起,嫖客复冀其留,仍前唱叹。鸨儿曰:“今番官人没钱,风留(流)不成。”

好乌龟
时值大比,一人夤缘科举一名,命卜者占龟,颇得佳象,稳许今科公捷。其人大喜,将龟壳谨带随身。至期点名入场,主试出题,旨解茫然,终日不成一字。因抚龟叹息曰:“不信这样一个好乌龟,如何竟不会做文字!”

通谱
有人欲狎一处女,先举其物询之曰:“此是何物,汝知之否?”女曰:“那是一张。”因“卵”字不便出口,故作歇后语也。又问曰:“这等,你腰下的何物?”女曰:“也是一张。”男曰:“你也一张,我也一张,可见这两件东西都是姓张的了,五百年前共一家,何不使他通一通谱?”女许之,遂解裤相狎。事毕后,女叹曰:“谱便通了,只是这个门户渐渐的大起来,收敛不得,却怎么好?”

联宗
眉毛一日忽欲与腋毛联宗,腋毛不肯,曰:“我也在人手下,如何与你联得?有一好去处,引你去联可也。”问:“何处?”曰:“下边新竖旗杆的。”

定亲
一人登厕,隔厕先有一女在焉,偶失净纸,因言:“若有知趣的给我,愿为之妇。”其人闻之,即以自所用者,从壁隙中递与。女净讫径去。其人叹曰:“亲事虽定了一头,这一屁股债,如何干净?”

有钱夸口
一人迷路,遇一哑子,问之不答,惟以手作钱样,示以得钱,方肯指引。此人喻其意,即以数钱与之,哑子乃开口指明去路。其人问曰:“为甚无钱装哑?”哑曰:“如今世界,有了钱,便会说话耳!”

古今三绝
一家门首,来往人屙溺,秽气难闻。因拒之不得,乃画一龟于墙上,题云:“在此溺尿者,即是此物。”一恶少见之,问曰:“此是谁的手笔?”画者任之,恶少曰:“宋徽宗、赵子昂与吾兄三人,共垂不朽矣。”画者询其故,答曰:“宋徽宗的鹰,赵子昂的马,兄这样乌龟,可称古今三绝。”

白蚁蛀
有客在外,而主人潜入吃饭者。既出,客谓曰:“宅上好座厅房,可惜许多梁柱,都被白蚁蛀坏了。”主人四顾曰:“并无此物。”客曰:“他在里面吃,外边人如何知道。”

乌须药
婢少艾,而主人苍老,屡次偷之不从。主人怒曰:“不受人抬举!你这般做作,我自有法处你。”婢问何法,主人曰:“熬得你阴毛尽白,方许嫁人。”婢曰:“不妨,我自有乌须药。”

吃烟
人有送夜羹饭甫毕,已将酒肉啖尽。正在化纸将完,而群狗环集,其人曰:“列位来迟了一步,并无一物请你,都来吃些烟罢。”

烟户
嫖客爱洁之极,妓女百般清趣,尚多憎嫌。妓将阴户透香,嫖客临事闻嗅被中,乃大骇云:“原来是个吃烟的烟户。”

烦恼
或问:“樊迟之名谁取?”曰:“孔子取的。”问:“樊哙之名谁取?”曰:“汉祖取的。”又曰:“烦恼之名谁取?”曰:”这是他自取的。”

嘉兴人
下虱请上虱宴饮,上虱行至脐下,见肾倒挂,乃大惊而回。一日,下虱复遇上虱,叙述“前次奉请,何以见却?”上虱曰:“那日知兄府上为了人命,心绪欠宁,故不好取扰。”下虱曰:“并无其事。”上虱曰:“吊死一嘉兴人在你门首,如何讳赖?”下虱曰:“那见是嘉兴人?”答曰:“他身边现带着两个臭鸭蛋。”

猫逐鼠
昔有一猫擒鼠,赶入瓶内,猫不舍,犹在瓶边守候。鼠畏甚,不敢出。猫忽打一喷嚏,鼠在瓶中曰:“大吉利。”猫曰:“不相干,凭你奉承得我好,只是要吃你哩!”

祝寿
猫与耗鼠庆生,安坐洞口,鼠不敢出。忽在内打一喷嚏,猫祝曰:“寿年千岁!”群鼠曰:“他如此恭敬,何妨一见?”鼠曰:“他何尝真心来祝寿啰,骗我出去,正要狠嚼我哩。”

心狠
一人戏将数珠挂猫项间,群鼠私相贺曰:“猫老官已持斋念佛,定然不吃我们的了。”遂欢跃于庭,猫一见,连哺数个。众鼠奔走,背地语曰:“吾等以他念佛心慈了,原来是假意修行。”一答曰:“你不知,如今世上修行念佛的,比寻常人的心肠更狠十倍。”

嘲恶毒
蜂与蛇结盟,蜂云:“我欲同你上江一游。”蛇曰:“可,你须伏在我背间。”行到江中,蛇已无力,或沉或浮。蜂疑蛇害己,将尾刺钉紧在蛇背上。蛇负疼骂曰:“人说我的口毒,谁知你的肚里更毒!”

骂无礼
有数小厮同下池塘浴水,被一小蛇将屪子咬了一口。小厮忿怒,将池塘戽干,果见小蛇,乃大骂曰:“这小畜生太无礼,咬我屪子就是你!”

讥人弄乖
凤凰寿,百鸟朝贺,惟蝙蝠不至。凤责之曰:“汝居吾下,何踞傲乎?”蝠曰:“吾有足,属于兽,贺汝何用?”一日,麒麟生诞,蝠亦不至,麟亦责之。蝠曰:“吾有翼,属于禽,何以贺欤?”麟、凤相会,语及蝙蝠之事,互相慨叹曰:“如今世上恶薄,偏生此等不禽不兽之徒,真个无奈他何!”

素毒
人问:“羊肉与鹅肉,如何这般毒得紧?”或答曰:“生平吃素的。”

嘲姓倪
旧有放手铳诗一首,嘲姓倪者,录之以供一笑。诗曰:“独坐书斋手作妻,此情不与外人知。若将左手换右手,便是停妻再娶妻。一撸一撸复一撸,浑身骚痒骨头迷。点点滴滴落在地,子子孙孙都姓倪(泥)。”

白嚼
三人同坐,偶谈及家内耗鼠可恶。一曰:“舍间饮食,落放不得,转眼被他窃去。”一云:“家下衣服书籍,散去不得,时常被他侵损。”又一曰:“独有寒家老鼠不偷食咬衣,终夜咨咨叫到天明。”此二人曰:“这是何故?”答曰:“专靠一味白嚼。”

嚼蛆
有善说笑话者,人嘲之曰:“我家有一狗,落在粪坑中,三年零六个月还不曾死。”其人曰:“既然如此,他吃些甚么?”答曰:“单靠嚼蛆。”
笑话一担
秀才年将七十,忽生一子,因有年纪而生,即名年纪。未几,又生一子,似可读书者,命名学问。次年,又生一子,笑曰:“如此老年,还要生儿,真笑话也。”因名曰笑话。三人年长无事,俱命入山打柴。及归,夫问曰:“三子之柴孰多?”妻曰:“年纪有了一把,学问一些也无,笑话到有一担。”

听笑话
一妇与邻人私,谓妇曰:“我常要过来会你,碍汝夫在家,奈何?”妇曰:“壁间挖乙孔,你将此物伸过,如他不在,我好通信。”一日,夫在家正讲笑话,突见壁间之物,夫诘之,妇无可答,乃慌应曰:“是听笑话的。”

引避
有势利者,每出,逢冠盖,必引避。同行者问其故,答曰:“舍亲。”如此屡屡,同行者厌之。偶逢一乞丐,亦效其引避,曰:“舍亲。”问:“为何有此令亲。”曰:“但是好的,都被你认去了。”

取笑
甲乙同行,甲望见显者冠盖,谓乙曰:“此吾好友,见必下车,我当引避。”不意竟避入显者之家,显者既入门,诧曰:“是何白撞,匿我门内!”呼童挞而逐之。乙问曰:“既是好友,何见殴辱?”答曰:“他从来是这般与我取笑惯的。”

吃橄榄
乡人入城赴酌,腰席内有橄榄焉。乡人取啖,涩而无味,因问同席者曰:“此是何物?”同席者以其村气,鄙之曰:“俗。”乡人以为“俗”是名,遂牢记之。归谓人曰:“我今日在城尝一奇物,叫名‘俗’。”众未信,其人乃张口呵气曰:“你们不信,现今满口都是俗气哩。”

避首席
有病疯疾者,延医调冶,医辞不肯用药。病者曰:“我亦自知难医,但要服些生痰动气的药,改作痨、膨二症。”医曰:“疯、痨、膨、膈,同是不起之症,缘何要改?”病者曰:“我闻得疯、痨、膨、膈,乃是阎罗王的上客。我生平怕做首席,所以要挪在第二、第三。”

瓦窑
一人连生数女,招友人饮宴。友作诗一首,戏赠之云:“去岁相招因弄瓦,今年弄瓦又相招。弄去弄来都弄瓦,令正原来是瓦窑。”

嘲周姓
浙中盐化地方,有查、祝、董、许四大族,簪缨世冑,科甲连绵。后有周姓者,偶发两榜,其居乡豪横,欲与四大姓并驾齐驱。里人因作诗嘲之曰:“查祝董许周,鼋鼍蛟龙鳅,江淮河海沟,虎豹犀象猴。”

嘲滑稽客
一人留客午饭,其客已啖尽一碗,不见添饭。客欲主人知之,乃佯言曰:“某家有住房一所要卖。”故将碗口向主人曰:“椽子也有这样大。”主人见碗内无饭,急呼童使添之。因问客曰:“他要价值几何?”客曰:“如今有了饭吃,不卖了。”

认族
有王姓者,平素最好联谱,每遇姓相似者,不曰寒宗,就说敝族。偶遇一汪姓者,指为友曰:“这是舍侄。”友曰:“汪姓何为是盛族?”其人曰:“他是水窠路里王家。”遇一匡姓者,亦认是侄孙。人曰:“匡与王,一发差得远了。”答曰:“他是㰙墙内王家。”又指一全姓,亦云:“是舍弟。”“一发甚么相干?”其人曰:“他从幼在大人家做蔑片的王家。”又指姓毛者是寒族,友大笑其荒唐,曰:“你不知,他本是我王家一派,只因生了一个尾靶,弄得毛头毛脑了。”人问:“王与黄同音,为何反不是一家?”答同:“如何不是?那是廿一都田头八家兄。」

\part{}

谬误部

见皇帝
一人从京师回,自夸曾见皇帝。或问:“皇帝门景如何?”答曰:“四柱牌坊,金书‘皇帝世家’。大门内匾,金书‘天子第’。两边对联是:‘日月光天德,山河壮帝居。’”又问:“皇帝如何装束?”曰:“头带玉纱帽,身穿金海青。”问者曰:“明明说谎,穿了金子打的海青,如何拜揖?”其人曰:“呸!你真是个冒失鬼,皇帝肯与那个作揖的?”

僭称呼
一家父子僮仆,专说大话,每每以朝廷名色自呼。一日,友人来望,其父出外,遇其长子,曰:“父王驾出了。”问及令堂,次子又云:“娘娘在后花园饮宴。”友见说话僭分,含怒而去。途遇其父,乃述其子之言告之。父曰:“是谁说的?”仆在后云:“这是太子与庶子说的。”其友愈恼,扭仆便打。其父忙劝曰:“卿家弗恼,看寡人面上。”

看镜
有出外生理者,妻要捎买梳子,嘱其带回。夫问其状,妻指新月示之。夫货毕,忽忆妻语,因看月轮正满,遂依样买了镜子一面带归。妻照之骂曰:“梳子不买,如何反取了一妾回来?”两下争闹。母闻之往劝,忽见镜,照云:“我儿有心费钱,如何讨恁个年老婆儿?”互相埋怨,遂至讦讼。官差往拘之,差见镜,慌云:“才得出牌,如何就出添差来捉违限?”及审,置镜于案,官照见大怒云:“夫妻不和事,何必央请乡官来讲份上!”

高才
一官偶有书义未解,问吏曰:“此处有高才否?”吏误认以为裁缝姓高也,应曰:“有。”即唤进,官问曰:“‘贫而无谄’,如何?”答曰:“裙而无襉,折起来。”又问:“‘富而无骄’,如何?”答曰:“裤若无腰,做上去。”官怒喝曰:“唗!”裁缝曰:“极是容易,若是皱了,小人有熨斗,取来烫汤。”

谢赏
一官坐堂,偶撒一屁,自说“爽利”二字。众吏不知,误听以为“赏吏”,冀得欢心,争跪禀曰:“谢老爷赏。”

不识货
有徽人开典而不识货者,一人以单皮鼓一面来当,喝云:“皮锣一面,当银五分。”有以笙来当者,云:“斑竹酒壶一把,当银三分。”有当笛者,云:“丝绢火筒一根,当银一分。”后有持了事帕来当者,喝云:“虎狸斑汉巾一条,当银二分。”小郎曰:“这物要他何用?”答云:“若还不赎,留他来抹抹嘴也好。”

外太公
有教小儿以“大”字者,次日写“太”字问之,儿仍曰:“大字。”因教之曰:“中多一点,乃太公的太字也。”明日写“犬”字问之,儿曰:“太公的太字。”师曰:“今番点在外,如何还是太字?”儿即应曰:“这样说,便是外太公了。”

床榻
有卖床榻者,一日夫出,命妇守店。一人来买床,价少,银水又低,争值良久,勉强售之。次日,复宋买榻,妇曰:“这人不知好歹,昨日床上讨尽我便宜,今日榻上又想要讨我的便宜了。”

房事
一丈母命婿以房典银,既成交,而房价未足。因作书促之云:“家岳母房事悬望至紧,刻不可缓,早晚望公垂慈一处,以济其急。至感,至感。”

卖粪
一家有粪一窖,招人货卖,索钱一千,买者还五百。主人怒曰:“有如此贱粪,难道是狗撒的?”乡人曰:“又不曾吃了你的,何须这等发急。”

出丑
有屠牛者,过宰猪者之家,其子欲讳“宰猪”二字,回云:“家尊出亥去了。”屠牛者归,对子述之,称赞不已。子亦领悟,次日屠猪至,其子亦回云:“家父往外出丑去了。”问:“几时归?”答曰:“出尽丑自然回来了。”

整嫂裙
一嫂前行而裙夹于臀缝内者,叔从后拽整之。嫂顾见,疑其调戏也,遂大怒。叔躬身曰:“嫂嫂请息怒,待愚叔依旧与你塞进去,你再夹紧何如?”

戏嫂臂
兄患病献神,嫂收祭物,叔将嫂臂暗掐一把。嫂怒云:“看你肥肉吃得几块!”兄在床上听见,叫声:“兄弟没正经,你嫂嫂要留来结识人头的,大家省口出客罢。”

淫病
一人不通文墨,向友问曰:“三点水的‘淫’字如何解?”友曰:“淫乃妇人之大病。”其人颔之。一日,此人之妻忽抱病颇剧,出遇友人问曰:“令正病体何如?”其人曰:“不要说起,贱内这两日,着实一发淫得紧哩。”

利市
一人元旦出门云:“头一日必得利市方妙。”遂于桌上写一“吉”字。不意连走数家,求一茶不得。将“吉”字倒看良久,曰:“原来写了‘口干’字,自然没得吃了。”再顺看曰,“吾论来,竟该有十一家替我润口。”

健讼
一生好健讼。一日,妻在坑厕上撒尿,见月色照在妻豚,乃大怒,遂以月照妻豚事,讼之于官。县令不解其意,挂牌拘审。生以实情诉禀,求父师伸冤。官怒曰:“月照你妻的豚就来告理,倘日晒你妻的屄,你待要怎么?”

官话
有兄弟经商,学得一二官话。将到家,兄往隔河出恭,命弟先往见其父。父问曰:“汝兄何在?”弟曰:“撒屎。”父惊曰:“在何处杀死的?”答曰:“河南。”父方悲恸而兄已至,父遂骂其次子:“何得妄言如是?”曰:“我自打官话耳。”父曰:“这样官话,只好吓你亲爷罢了。”

掌嘴
一乡人进城,偶与人竞,被打耳光子数下。赴县叫喊,官问:“何事?”曰:“小人被人打了许多乳光。”官不信,连问,只以乳光对。官大怒,呼皂隶掌嘴。方被掌,乡人遽以指示官,正是这个样子。

乳广
一乡人涉讼,官受其贿,临审复掌嘴数下。乡人不忿,作官话曰:“老牙,你要人觜我就人觜,要铜团就铜团,要尾就尾,为何临了来又歹我的乳广?”

官物
一大气脬过关,关吏见之,指其夹带漏税。其人辩曰:“小的是疝气病。”吏曰:“既是扇子柄,难道不要起税的么?”曰:“疼的疝气病。”吏曰:“藤扎扇子柄,一发要报税了。”其人曰:“老爷,不是,是疼的大气脬。”吏怒曰:“铜的大剃刀,岂该容汝漏税?责打二十,以正其罪!”此人被打出来,偶为尿急,对人家门首撒之。门内妇人大骂,其人曰:“娘子休骂,我这官物,比众不同,才在衙门里纳过税,娘子就请看何妨。”

初上路
一人初上北路,才骑牲口踏镫,掉落一鞋。其人因作官话大声曰:“阿呀,掌鞭的,我的鞋。”赶鞭的以为唤他做爷,答云:“爷不敢。”其人愈发急,大呼曰:“我的鞋,我的鞋!”掌鞭的不会其意,亦连声响应曰:“爷,小的怎么敢?”其人只得仍作乡语,怒骂曰:“搠杀那娘,我一只鞋子脱掉了!”

闹一闹
一杭人妇,催轿往西湖游玩,贪恋湖上风景,不觉归迟。时已将暮,怕关城门,心中着急,乃对轿夫言曰:“轿夫阿哥,天色晚了,我多把银钱打发,你与我尽力闹一闹。早行进到里头去,不但是我好,连你们也落得自在快活些。”

摸一把
妇人门首买菜,问:“几个钱一把?”卖者说:“实价三个钱两把。”妇还两个钱三把,卖者云:“不指望我来摸娘娘一把,娘娘倒想要摸我一把,讨我这样便宜。”

苏空头
一人初往苏州,或教之曰:“吴人惯扯空头,若去买货,他讨二两,只好还一两。就是与人讲话,他说两句,也只好听一句。”其人至苏,先以买货之法,行之果验。后遇一人,问其姓,答曰:“姓陆。”其人曰:“定是三老官了。”又问:“住房几间?”曰:“五间。”其人曰:“原来是两间一披。”又问:“宅上还有何人?”曰:“只房下一个。”其人背曰:“原还是与人合的。”

连偷骂
吴人有灌园者,被邻居窃去蔬果,乃大骂曰:“入娘贼,春天偷了我婶(笋),夏天又来偷我妹(梅)子,到冬来还要偷我个老婆(萝卜)。”

晾杩桶
苏州人家晒晾两杩桶在外,瞽者不知,误撒小解。其姑喝骂,嫂忙问曰:“这肏娘贼个脓血,滴来你个里面,还是撒来我个里头。”姑回云:“我搭你两边都有点个。”

鸟出来
一家养子瞒人,邻翁问其妇曰,“娘子恭喜,添了令郎。”妇曰:“并无此事,要便是你鸟出来的。”

轧棉花
姑嫂二人地上轧棉花,嫂问姑:“轧得几何?”姑曰:“尽力轧得两腿酸麻,轧个戎(绒)勿出。”

庆生
松江有妪诞辰,子侄辈商所以庆生者。一曰:“叫伙戏子与渠汤汤,好弗热闹。”一曰:“个非阿娘所好,弗如寻几个和尚,与渠笃笃倒好。”

贺寿
贺友寿者,其友先期躲生,锁门而出。一日,路上遇见,此人惯作歇后语,因对友曰:“前兄寿日,弟拉了许多丧门吊客,替你生灾作贺,谁料你家入地无门,竟是披枷带(锁)了。”

寿气
一老翁寿诞,亲友醵分,设宴公祝,正行令,各人要带说“寿”字。而壶中酒忽竭,主人大怒,客曰:“为何动寿气(器)?”一客云:“欠检点,该罚。”少顷,又一人唱寿曲,傍一人曰:“合差了寿板。”合席皆曰:“一发该罚。”

譬字令
众客饮酒,要譬字《四书》一句为令,说不出者,罚一巨觥。首令曰:“譬如为山。”次曰“譬如行远必自迩”,以及“譬之宫墙”等句。落后一人无可说得,乃曰:“能近取譬。”众哗然曰:“不如式该罚。如何譬字说在下面?”其人曰:“屁原该在下,诸兄都从上来,不说自倒出了,反来罚我?”

不知令
饮酒行令,座客有茫然者。一友戏曰:“不知令,无以为君子也。”其人诘曰:“不知命,为何改作令字?”答曰:“《中庸》注云:‘命犹令也。’”

令官不举
夫妻二人对饮,妻劝夫行令。夫曰:“无色盆奈何?”妻指腰间曰:“色盆在此,要你行色令,非行酒令也。”夫曰:“可。”遂解裤出具就之,但苦其物之不硬。妻大叫曰:“令官不举,该罚一杯。”

十恶不赦
乡人夤缘进学,与父兄叔伯暑天同走,惟新生撑伞。人问何故,答曰:“入学不晒(十恶不赦)。”

馄饨
苏州人有卖馄饨者,夫偶出,令其妻守店,姿色甚美。一人来买馄饨,因贪看想慕出神,叫曰:“娘子,我要买你饨(臀)。”妇应曰:“你为何脱落子馄(魂)啰?”

茶屑
一妇人向山客买茶叶,客问曰:“娘子还是要细的,要粗的?”妇曰:“粗细倒也都用得着,只不要屑(泄)。”

卖糖
一糖担歇在人家门首敲锣,妇喝曰:“快请出去,只管在此甚么?出个小的儿来,又要害我淘气。”

食蔗
一家请客,摆列水果。家主母取甘蔗食之,连声叫淡。厨司曰:“娘娘想是梢(骚)了。”

秤人
天赦日秤人,婆先将媳上秤,婆云:“娘子,你放在大花星上正好。”次秤婆,媳云:“看婆婆不出,到(倒)梢(骚)了。”

蚬子
两人相遇,各问所生子女几何。一曰:“五女。”一曰:“一子。”生女者曰:“一子是险子。”生子者怒曰:“我是蚬子,强如你养了许多肉蚌。”

出甑馒头
一女人暑天卖馒头,一人进店取一个,拍开一闻,以其荤者,仍合拢不买而去。店主母大骂曰:“掰开屄个天杀的!我家这样初出笼的馒头,香喷喷,粉白肥嫩,不差甚么,你也用得过。为甚走进来拍开一条大缝,嗅了一嗅,竟自去了。”

绵在凳
一女买绵子,正在讲价,卖者欲出小恭,踌躇不决。女云:“你放在此,难道我偷了不成?”其人曰:“既如此,大娘绵(眠)在凳上,待我撒出了来。”

撒屁秤
一人问邻妇借秤,妇回云:“我家这管撒屁秤,是用不得的。”其人曰:“娘子,你在前另有不撒屁的,求借我用一用。”

猫乞食
一猫向妇人求食,叫唤不止。妇喝曰:“只管叫甚么,除非割下这张屄来与你吃。”邻汉听得曰:“娘子,你若当真,我就去买碎鱼来换。”

底下硬
一人夜膳后,先在板凳上去睡,翻身说:“底下硬得紧。”妻在灶前听见,回言曰:“不要忙,收拾过碗盏就来了。”

手氏
一人年逾四旬始议婚,自惭太晚,饰言续弦。及娶后,妻察其动静,似为未曾婚者。乃问其前妻何氏,夫骤然不及思,遽答曰:“手氏。”

两夫
丈夫欲娶妾,妻曰:“一夫配一妇耳,娶妾见于何典?”夫曰:“孟子云:‘齐人有一妻一妾。’又曰:‘妾妇之道。’妾自古有之矣。”妻曰:“若这等说,我亦当再招一夫。”夫曰:“何故?”妻曰:“岂不闻《大学》上云:‘河南程氏两夫’。《孟子》中亦有‘大丈夫’、‘小丈夫’。”

日饼
中秋出卖月饼,招牌上错写日饼。一人指曰:“月字写成白字了。”其人曰:“我倒信你骗,白字还有一撇哩!”

禁溺
墙脚下恐人撒尿,画一乌龟于壁上,且批其后曰:“撒尿者即是此物。”一人不知那里,仍去屙溺。其人骂曰:“瞎了眼睛,也不看看。”撒尿者曰:“不知老爹在此。”

墙龟
墙上画一乌龟,专禁人屙尿。一人竟撒,主家喝曰:“你看!”其人云:“原来乌龟在此看我撒尿。”

说大话
主人谓仆曰:“汝出外,须说几句大话,装我体面。”仆领之。值有言“三清殿大”者,仆曰:“只与我家租房一般。”有言“龙衣船大”者,曰:“只与我家帐船一般。”有言“牯牛腹大”者,曰:“只与我家主人肚皮一般。”

挣大口
两人好大言。一人唇说:“敝乡有一大人,头顶天,脚踏地。”一人曰:“敝乡有一人更大,上嘴唇触天,下嘴唇着地。”其人问曰:“他身子藏在那里?”答曰:“我只见他挣得一张大口。”

天话
一人说:“昨日某处,天上跌下一个人来,长十丈,大二丈。”或问之曰:“亦能说话否?”答曰:“也讲几句。”曰:“讲甚么话?”曰:“讲天话。”

慌鼓
一说谎者曰:“敝处某寺中有一鼓,大几十围,声闻百里。”傍又一人曰:“敝地有一牛,头在江南,尾在江北,足重有万余斤,岂不是奇事?”众人不信。其人曰:“若没有这只大牛,如何得这张大皮,慢得这面大鼓?”

大浴盆
好说谎者对人曰:“敝处某寺有一脚盆,可使千万人同浴。”闻者不信。傍一人曰:“此是常事,何足为奇?敝地一新闻,说来才觉诧异。”人问:“何事?”曰:“某寺有一竹林,不及三年,遂长有几百万丈,如今顶着天公长不上去,又从天上长下来。岂不是奇事?”众人皆谓诳言。其人曰:“若没有这等长竹,叫他把甚么篾子,箍他那只大脚盆?”

两企慕
山东人慕南方大桥,不辞远道来看。中途遇一苏州人,亦闻山东萝卜最大,前往观之。两人各诉企慕之意。苏人曰:“既如此,弟只消备述与兄听,何必远道跋涉?”因言:“去年六月初三,一人自桥上失足堕河,至今年六月初三,还未曾到水,你说高也不高?”山东人曰:“多承指教。足下要看敝处萝卜,也不消去得,明年此时,自然长过你们苏州来了。”

误听
一人过桥,贴边而走,傍人谓曰:“看仔细,不要踏了空。”其人误听说他偷了葱,因而大怒,争辨不已。复转诉一人,其人曰:“你们又来好笑,我素不相认,怎么冤我盗了钟?”互相厮打,三人扭结到官。官问三人情事,拍案恚曰:“朝廷设立衙门,叫我南面坐,尔等反叫我朝了东!”制签就打。官民争闹,惊动后堂。适奶奶在屏后窃听,闻之柳眉倒竖,抢出堂来,拍案吵闹曰:“我不曾干下歹事,为何通同众百姓要我嫁老公!”

招弗得
松江人无子,一友问:“尊嫂曾养否?”其人答曰:“房下养(同痒)是常常养呢,只是孽(入看)深招(抓看)勿得。”

手木笃
松江妇寒天淘米,似手冷插入腰内。主母疑其偷米,喝曰:“做甚么?”妇答云:“手木(摸)笃(音屄,手冷也)。”

圆谎
有人惯会说谎,其仆每代为圆之,一日,对人说:“我家一井,昨被大风吹往隔壁人家去了。”众以为从古所无,仆圆之曰:“确有其事。我家的井,贴近邻家篱笆,昨晚风大,把篱笆吹过井这边来,却像井吹在邻家去了。”一日,又对人说:“有人射下二雁,头上顶碗粉汤。”众又惊诧之,仆圆曰:“此事亦有。我主人在天井内吃粉汤,忽有一雁堕下,雁头正跌在碗内,岂不是雁头顶着粉汤。”一日,又对人说:“寒家有顶漫天帐,把天地遮得沿沿的,一些空隙也没有。”仆乃攒眉曰:“主人脱煞扯这漫天谎,叫我如何遮掩得来。」

\backmatter

\end{document}