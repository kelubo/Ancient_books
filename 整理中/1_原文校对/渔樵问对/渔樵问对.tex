% 渔樵问对
% 渔樵问对.tex

\documentclass[12pt,UTF8]{ctexbook}

% 设置纸张信息。
\usepackage[a4paper,twoside]{geometry}
\geometry{
	left=25mm,
	right=25mm,
	bottom=25.4mm,
	bindingoffset=10mm
}

% 设置字体,并解决显示难检字问题。
\xeCJKsetup{AutoFallBack=true}
\setCJKmainfont{SimSun}[BoldFont=SimHei, ItalicFont=KaiTi, FallBack=SimSun-ExtB]

% 目录 chapter 级别加点(.)。
\usepackage{titletoc}
\titlecontents{chapter}[0pt]{\vspace{3mm}\bf\addvspace{2pt}\filright}{\contentspush{\thecontentslabel\hspace{0.8em}}}{}{\titlerule*[8pt]{.}\contentspage}

% 设置 part 和 chapter 标题格式。
\ctexset{
	chapter/name={},
	chapter/number={}
}

% 图片相关设置。
\usepackage{graphicx}
\graphicspath{{Images/}}

% 设置古文原文格式。
\newenvironment{yuanwen}{\bfseries\zihao{4}}

% 设置署名格式。
\newenvironment{shuming}{\hfill\bfseries\zihao{4}}

% 注脚每页重新编号,避免编号过大。
\usepackage[perpage]{footmisc}

\title{\heiti\zihao{0} 渔樵问对}
\author{邵雍}
\date{北宋}

\begin{document}

\maketitle
\tableofcontents

\frontmatter
\chapter{前言}

渔樵问对,通过渔樵对话来消解古今兴亡等厚重话题。作者邵雍学贯易理,儒道兼通,毕生致力于将天与人统一于一心,从而试图把儒家的人本与道家的天道贯通起来。北宋儒家五子之一。

《渔樵问对》着力论述天地万物、阴阳化育和生命道德的奥妙和哲理。这本书通过樵子问、渔父答的方式,将天地、万物、人事、社会归之于易理,并加以诠释,目的是让樵者明白“天地之道备于人,万物之道备于身,众妙之道备于神,天下之能事毕矣”的道理。

《渔樵问对》中的主角是渔父,所有的玄理都出自渔父之口,在书中,渔父已经成了“道”的化身。

\mainmatter

\chapter{渔樵问对}

\begin{yuanwen}
渔者垂钓于伊水之上。樵者过之,弛担息肩,坐于磐石之上,而问于渔者,曰:“鱼可钩取乎?”

曰:“然。”

曰:“钩非饵可乎?”

曰:“否。”

曰:“非钩也,饵也。鱼利食而见害,人利鱼而蒙利,其利同也,其害异也。敢问何故?”

渔者曰:“子樵者也,与吾异治,安得侵吾事乎?然亦可以为子试言之。彼之利,犹此之利也;彼之害,亦犹此之害也。子知其小,未知其大。鱼之利食,吾亦利乎食也;鱼之害食,吾亦害乎食也。子知鱼终日得食为利,又安知鱼终日不得食为害?如是,则食之害也重,而钩之害也轻。子知吾终日得鱼为利,又安知吾终日不得鱼不为害也?如是,则吾之害也重,鱼之害也轻。以鱼之一身,当人之食,是鱼之害多矣;以人之一身,当鱼之一食,则人之害亦多矣。又安知钓乎大江大海,则无易地之患焉?鱼利乎水,人利乎陆,水与陆异,其利一也;鱼害乎饵,人害乎财,饵与财异,其害一也。又何必分乎彼此哉!子之言,体也,独不知用尔。”
\end{yuanwen}

渔者垂钓于伊水之边。有一樵者路过,放下柴担休息,坐在大石头上,问渔者:“能钓到鱼吗?”

答:“能。”

问:“鱼钩上不放鱼饵能钓到吗?”

答:“不能。”

问:“钓到鱼不是鱼钩而是鱼饵,可见鱼因吃食而受害,人因吃鱼而受利,都是因吃其利一样,而结果不一样。请问这是为什么?”

渔者说:“你是打柴的,与我工作不一样,又怎么能知道我的事呢?然而我可以给你解释一下。你口中的利就像你看到我钓到了鱼,你口中的害就像你看到鱼丢失了性命。你只知道眼前事件的利害关系,并没有看到影响眼前事件的利害关系是多方面的。鱼的利和我的利是一样的,鱼的害和我的害也是一样的。你只知其一,未知其二。鱼受利于食,我也受利于食,鱼受害于食,我也受害于食。你只知鱼终日有食吃而为利,又怎知鱼若终日无食吃而为害呢?如此,食物的害处太重了,而钓鱼的害处却轻了。你只知我终日钓到鱼而为利,又怎知我若终日钓不到鱼而为害呢?如此,我受到害太重了,而鱼受到的害却轻了。若以鱼为本,人吃了鱼,则鱼受到了伤害;若以人为本,以鱼为食,人无食吃则人受到了伤害。更何况在大江大海里钓鱼,又是多么的危险?鱼生活在水里,人生活在陆地,水与陆地不同,其利益一样。鱼受害于饵,人受害于财,饵与财不同,其害处一样,又何必分彼此呢!你说的,只是事物的本质,而不知事物的变化。”

\begin{yuanwen}

\end{yuanwen}\begin{yuanwen}

\end{yuanwen}\begin{yuanwen}

\end{yuanwen}\begin{yuanwen}

\end{yuanwen}\begin{yuanwen}

\end{yuanwen}\begin{yuanwen}

\end{yuanwen}\begin{yuanwen}

\end{yuanwen}\begin{yuanwen}

\end{yuanwen}\begin{yuanwen}

\end{yuanwen}\begin{yuanwen}

\end{yuanwen}\begin{yuanwen}

\end{yuanwen}\begin{yuanwen}

\end{yuanwen}\begin{yuanwen}

\end{yuanwen}\begin{yuanwen}

\end{yuanwen}\begin{yuanwen}

\end{yuanwen}\begin{yuanwen}

\end{yuanwen}\begin{yuanwen}

\end{yuanwen}\begin{yuanwen}

\end{yuanwen}\begin{yuanwen}

\end{yuanwen}\begin{yuanwen}

\end{yuanwen}\begin{yuanwen}

\end{yuanwen}\begin{yuanwen}

\end{yuanwen}\begin{yuanwen}

\end{yuanwen}\begin{yuanwen}

\end{yuanwen}\begin{yuanwen}

\end{yuanwen}\begin{yuanwen}

\end{yuanwen}\begin{yuanwen}

\end{yuanwen}\begin{yuanwen}

\end{yuanwen}\begin{yuanwen}

\end{yuanwen}\begin{yuanwen}

\end{yuanwen}\begin{yuanwen}

\end{yuanwen}\begin{yuanwen}

\end{yuanwen}\begin{yuanwen}

\end{yuanwen}\begin{yuanwen}

\end{yuanwen}\begin{yuanwen}

\end{yuanwen}\begin{yuanwen}

\end{yuanwen}\begin{yuanwen}

\end{yuanwen}\begin{yuanwen}

\end{yuanwen}\begin{yuanwen}

\end{yuanwen}\begin{yuanwen}

\end{yuanwen}\begin{yuanwen}

\end{yuanwen}\begin{yuanwen}

\end{yuanwen}\begin{yuanwen}

\end{yuanwen}

樵者又问曰:“鱼可生食乎?”
曰:“烹之可也。”
曰:“必吾薪济子之鱼乎?”
曰:“然。“
曰:“吾知有用乎子矣。”
曰:“然则子知子之薪,能济吾之鱼,不知子之薪所以能济吾之鱼也。薪之能济鱼久矣,不待子而后知。苟世未知火之能用薪,则子之薪虽积丘山,独且奈何哉?”
樵者曰:“愿闻其方。”
曰:“火生于动,水生于静。动静之相生,水火之相息。水火,用也;草木,体也。用生于利,体生于害。利害见乎情,体用隐乎性。一性一情,圣人能成。子之薪犹吾之鱼,微火则皆为腐臭败坏,而无所用矣,又安能养人七尺之躯哉?”
樵者曰:“火之功大于薪,固已知之矣。敢问善灼物,何必待薪而后传?”
曰:“薪,火之体也。火,薪之用也。火无体,待薪然后为体;薪无用,待火然后为用。是故凡有体之物,皆可焚之矣。”
曰:“水有体乎?”
曰:“然。”
曰:“火能焚水乎?“
曰:“火之性,能迎而不能随,故灭。水之体,能随而不能迎,故热。是故有温泉而无寒火,相息之谓也。”
曰:“火之道生于用,亦有体乎?”
曰:“火以用为本,以体为末,故动。水以体为本,以用为末,故静。是火亦有体,水亦有用也。故能相济又能相息,非独水火则然,天下之事皆然,在乎用之何如尔。”
樵者曰:“用可得闻乎?”
曰:“可以意得者,物之性也。可以言传者,物之情也。可以象求者,物之形也。可以数取者,物之体也。用也者,妙万物为言者也,可以意得,而不可以言传。”
曰:“不可以言传,则子恶得而知之乎?”
曰:“吾所以得而知之者,固不能言传,非独吾不能传之以言,圣人亦不能传之以言也。”
曰:“圣人既不能传之以言,则六经非言也耶?”
曰:“时然后言,何言之有?”
樵者赞曰:“天地之道备于人,万物之道备于身,众妙之道备于神,天下之能事毕矣,又何思何虑!吾而今而后,知事心践形之为大。不及子之门,则几至于殆矣。”
乃析薪烹鱼而食之,饫而论《易》。
渔者与樵者游于伊水之上。渔者叹曰:“熙熙乎万物之多,而未始有杂。吾知游乎天地之间,万物皆可以无心而致之矣。非子则孰与归焉?”
樵者曰:“敢问无心致天地万物之方?”
渔者曰:“无心者,无意之谓也。无意之意,不我物也。不我物,然后定能物物。”
曰:“何谓我,何谓物?”
曰:‘以我徇物,则我亦物也;以物徇我,则物亦我也。我物皆致,意由是明。天地亦万物也,何天地之有焉?万物亦天地也,何万物之有焉?万物亦我也,何万物之有焉?我亦万物也,何我之有焉?何物不我?何我不物?如是则可以宰天地,可以司鬼神,而况于人乎?况于物乎?“
樵者问渔者曰:“天何依?”
曰:“依乎地。”
曰:“地何附?”
曰:“附乎天。”
曰:“然则天地何依何附?”
曰:“自相依附。天依形,地附气。其形也有涯,其气也无涯。有无之相生,形气之相息。终则有始,终始之间,其天地之所存乎?天以用为本,以体为末;地以体为本,以用为末。利用出入之谓神,名体有无之谓圣。唯神与圣,能参乎天地者也。小人则日用而不知,故有害生实丧之患也。夫名也者,实之客也;利也者,害之主也。名生于不足,利丧于有余。害生于有余,实丧于不足。此理之常也。养身者必以利,贪夫则以身殉,故有害生焉。立身必以名,众人则以身殉名,故有实丧焉。窃人之财谓之盗,其始取之也,唯恐其不多也。及其败露也,唯恐其多矣。夫贿之与赃,一物而两名者,利与害故也。窃人之美谓之徼,其始取之也,唯恐其不多也。及其败露,唯恐其多矣。夫誉与毁,一事而两名者,名与实故也。凡言朝者,萃名之地也;市者,聚利之地也。能不以争处乎其间,虽一日九迁,一货十倍,何害生实丧之有耶?是知争也者取利之端也,让也者趋名之本也。利至则害生,名兴则实丧。利至名兴,而无害生实丧之患,唯有德者能之。天依地,地会天,岂相远哉!”
渔者谓樵者曰:“天下将治,则人必尚行也;天下将乱,则人必尚言也。尚行,则笃实之风行焉;尚言,则诡谲之风行焉。天下将治,则人必尚义也;天下将乱,则人必尚利也。尚义,则谦让之风行焉;尚利,则攘夺之风行焉。三王,尚行者也;五霸,尚言者也。尚行者必入于义也,尚言者必入于利也。义利之相去,一何如是之远耶?是知言之于口,不若行之于身,行之于身,不若尽之于心。言之于口,人得而闻之,行之于身,人得而见之,尽之于心,神得而知之。人之聪明犹不可欺,况神之聪明乎?是知无愧于口,不若无愧于身,无愧于身,不若无愧于心。无口过易,无身过难,无身过易,无心过难。既无心过,何难之有!吁,安得无心过之人,与之语心哉!”
渔者谓樵者曰:“子知观天地万物之道乎?”
樵者曰:“未也。愿闻其方。”
渔者曰:“夫所以谓之观物者,非以目观之也,非观之以目,而观之以心也;非观之以心,而观之以理也。天下之物,莫不有理焉,莫不有性焉,莫不有命焉。所以谓之理者,穷之而后可知也;所以谓之性者,尽之而后可知也;所似谓之命者,至之而后可知也。此三知也,天下之真知也,虽圣人无以过之也。而过之者,非所以谓之圣人也。夫鉴之所以能为明者,谓其能不隐万物之形也。虽然鉴之能不隐万物之形,未若水之能一万物之形也。虽然水之能一万物之形,又未若圣人之能一万物情也。圣人之所以能一万物之情者,谓其圣人之能反观也。所以谓之反观者,不以我观物也。不以我观物者,以物观物之谓也。又安有我于其间哉?是知我亦人也,人亦我也。我与人皆物也。此所以能用天下之目为己之目,其目无所不观矣。用天下之耳为己之耳,其耳无所不听矣。用天下之口为己之口,其口无所不言矣。用天下之心为己之心,其心无所不谋矣。天下之观,其于见也,不亦广乎?天下之听,其于闻也,不亦远乎?天下之言,其于论也,不亦高乎?天下之谋,其于乐也,不亦大乎?夫其见至广,其闻至远,其论至高,其乐至大,能为至广、至远、至高、至大之事,而中无一为焉,岂不谓至神至圣者乎?非唯吾谓之至神至圣者乎,而天下谓之至神至圣者乎。非唯一时之天下渭之至神至圣者乎,而千万世之天下谓之至神圣者乎。过此以往,未之或知也已。”
樵者问渔者曰:“子以何道而得鱼?”
曰:“吾以六物具而得鱼。”
曰:“六物具也,岂由天乎?”
曰:“具六物而得鱼者,人也。具六物而所以得鱼者,非人也。”
樵者未达,请问其方。
渔者曰:“六物者,竿也,纶也,浮也,沉也,钩也,饵也。一不具,则鱼不可得。然而六物具而不得鱼者,非人也。六物具而不得鱼者有焉,未有六物不具而得鱼者也。是知具六物者,人也。得鱼与不得鱼,天也。六物不具而不得鱼者,非天也,人也。”
樵者曰:“人有祷鬼神而求福者,福可祷而求耶?求之而可得耶?敢问其所以。”
曰:“语善恶者,人也;福祸者,天也。天道福善而祸淫,鬼神岂能违天乎?自作之咎,固难逃已。天降之灾,禳之奚益?修德积善,君子常分。安有余事于其间哉!”
樵者曰:“有为善而遇祸,有为恶而获福者,何也?”
渔者曰:“有幸与不幸也。幸不幸,命也;当不当,份也。一命一份,人其逃乎?”
曰:“何谓份?何谓命?”
曰:“小人之遇福,非份也,有命也;当祸,份也,非命也。君子之遇祸,非份也,有命也;当福,份也,非命也。”
渔者谓樵者曰:“人之所谓亲,莫如父子也;人之所渭疏,莫如路人也。利害在心,则父子过路人远矣。父子之道,天性也。利害犹或夺之,况非天性者乎?夫利害之移人,如是之深也,可不慎乎?路人之相逢则过之,固无相害之心焉,无利害在前故也。有利害在前,则路人与父子,又奚择焉?路人之能相交以义,又何况父子之亲乎?夫义者,让之本也;利者,争之端也。让则有仁,争则有害。仁与害,何相去之远也!尧、舜亦人也。桀、纣亦人也,人与人同而仁与害异尔,仁因义而起,害因利而生。利不以义,则臣弑其君者有焉,子弑其父者有焉。岂若路人之相逢,一目而交袂于中逵者哉!”
樵者谓渔者曰:“吾尝负薪矣,举百斤而无伤吾之身,加十斤则遂伤吾之身,敢问何故?”
渔者曰:“樵则吾不知之矣。以吾之事观之,则易地皆然。吾尝钓而得大鱼,与吾交战。欲弃之,则不能舍,欲取之,则未能胜。终日而后获,几有没溺之患矣。非直有身伤之患耶?鱼与薪则异也,其贪而为伤则一也。百斤,力分之内者也,十斤,力分之外者也。力分之外,虽一毫犹且为害,而况十斤乎!吾之贪鱼亦何以异子之贪薪乎!”
樵者叹曰:“吾而今而后,知量力而动者,智矣哉!”
樵者谓渔者曰:“子可谓知《易》之道矣。吾也问:《易》有太极,太极何物也?”
曰:“无为之本也。”
曰:“太极生两仪,两仪,天地之谓乎?”
曰:“两仪,天地之祖也,非止为天地而已也。太极分而为二,先得一为一,后得一为二。一二谓两仪。”
曰:“两仪生四象,四象何物也?”
曰:“大象谓阴阳刚柔。有阴阳然后可以生天,有刚柔然后可以生地。立功之本,于斯为极。”
曰:“四象生八卦,八卦何谓也?”
曰:“谓乾、坤、离、坎、兑、艮、震、巽之谓也。迭相盛衰终始于其间矣。因而重之,则六十四卦由是而生也,而《易》之道始备矣。”
樵者问渔者曰:“复何以见天地之心乎?”
曰:“先阳已尽,后阳始生,则天地始生之际。中则当日月始周之际,末则当星辰始终之际。万物死生,寒暑代谢,昼夜变迁,非此无以见之。当天地穷极之所必变,变则通,通则久,故《象》言‘先王以至日闭关,商旅不行,后不省方’,顺天故也。”
樵者谓渔者曰:“无妄,灾也。敢问何故?”
曰:“妄则欺他,得之必有祸,斯有妄也,顺天而动,有祸及者,非祸也,灾也。犹农有思丰而不勤稼稿者,其荒也,不亦祸乎?农有勤稼穑而复败诸水旱者,其荒也,不亦灾乎?故《象》言‘先王以茂对时育万物’,贵不妄也。”
樵者问曰:“姤,何也?”
曰:“姤,遇也。柔遇刚也,与夬正反。夬始逼壮,姤始遇壮,阴始遇阳,故称姤焉。观其姤,天地之心,亦可见矣。圣人以德化及此,罔有不昌。故《象》言‘施命诰四方’,履霜之慎,其在此也。”
渔者谓樵者曰:“春为阳始,夏为阳极,秋为阴始,冬为阴极。阳始则温,阳极则热;阴始则凉,阴极则寒。温则生物,热则长物,凉则收物,寒则杀物。皆一气别而为四焉。其生万物也亦然。”
樵者问渔者曰:“人之所以能灵于万物者,何以知其然耶?”
渔者对曰:“谓其目能收万物之色,耳能收万物之声,鼻能收万物之气,口能收万物之味。声色气味者,万物之体也。目耳口鼻者,万人之用也。体无定用,惟变是用。用无定体,惟化是体。体用交而人物之道于是乎备矣。然则人亦物也,圣亦人也。有一物之物,有十物之物,有百物之物,有千物之物,有万物之物,有亿物之物,有兆物之物。生一一之物,当兆物之物者,岂非人乎!有一人之人,有十人之人,有百人之人,有千人之人,有万人之人,有亿人之人,有兆人之人。当兆人之人者,岂非圣乎!是知人也者,物之至者也。圣也者,人之至者也。物之至者始得谓之物之物也。人之至者始得谓之人之人也。夫物之物者,至物之谓也。人之人者,至人之谓也。以一至物而当一至人,则非圣人而何?人谓之不圣,则吾不信也。何哉?谓其能以一心观万心,一身观万身,一物观万物,一世观万世者焉。又谓其能以心代天意,口代天言,手代天工,身代天事者焉。又谓其能以上识天时,下尽地理,中尽物情,通照人事者焉。又谓其能以弥纶天地,出入造化,进退今古,表里人物者焉。噫!圣人者,非世世而效圣焉。吾不得而目见之也。虽然吾不得而目见之,察其心,观其迹,探其体,潜其用,虽亿万千年亦可以理知之也。人或告我曰:‘天地之外,别有天地万物,异乎此天地万物。’则吾不得而知之也。非唯吾不得而知之也,圣人亦不得而知之也。凡言知者,谓其心得而知之也。言言者,谓其口得而言之也。既心尚不得而知之,口又恶得而言之乎?以不可得知而知之,是谓妄知也。以不可得言而言之,是谓妄言也。吾又安能从妄人而行妄知妄言者乎!”
渔者谓樵者曰:“仲尼有言曰:殷因于夏礼,所损益可知也;周因于殷礼,所损益可知也。其或继周者,虽百世可知也。夫如是,则何止于百世而已哉!亿千万世,皆可得而知之也。人皆知仲尼之为仲尼,不知仲尼之所以为仲尼,不欲知仲尼之所以为仲尼则已,如其必欲知仲尼之所以为仲尼,则舍天地将奚之焉?人皆知天地之为天地,不知天地之所以为天地。不欲知天地之所以为天地则已,如其必欲知天地之所以为天地,则舍动静将奚之焉?夫一动一静者,天地至妙者欤?夫一动一静之间者,天地人至妙者欤?是知仲尼之所以能尽三才之道者,谓其行无辙迹也。故有言曰:‘予欲无言’,又曰:‘天何言哉!四时行焉,百物生焉。’其此之谓与?”
渔者谓樵者曰:“大哉!权之与变乎?非圣人无以尽之。变然后知天地之消长,权然后知天下之轻重。消长,时也;轻重,事也。时有否泰,事有损益。圣人不知随时否泰之道,奚由知变之所为乎?圣人不知随时损益之道,奚由知权之所为乎?运消长者,变也;处轻重者,权也。是知权之与变,圣人之一道耳。”
樵者问渔者曰:“人谓死而有知,有诸?”
曰:“有之。”
曰:“何以知其然?”
曰:“以人知之。”
曰:“何者谓之人?”
曰:“目耳鼻口心胆脾肾之气全,谓之人。心之灵曰神,胆之灵曰魄,脾之灵曰魂,肾之灵曰精。心之神发乎目,则谓之视;肾之精发乎耳,则谓之听;脾之魂发乎鼻,则谓之臭;胆之魄发乎口,则谓之言。八者具备,然后谓之人。夫人也者,天地万物之秀气也。然而亦有不中者,各求其类也。若全得人类,则谓之曰全人之人。夫全类者,天地万物之中气也,谓之曰全德之人也。全德之人者,人之人者也。夫人之人者,仁人之谓也。唯全人,然后能当之。人之生也,谓其气行,人之死也,谓其形返。气行则神魂交,形返则精魄存。神魂行于天,精魄返于地。行于天,则谓之曰阳行;返于地,则谓之曰阴返。阳行则昼见而夜伏者也,阴返则夜见而昼伏者也。是故知日者月之形也,月者日之影也。阳者阴之形也,阴者阳之影也。人者鬼之形也,鬼者人之影也。人谓鬼无形而无知者,吾不信也。”
樵者问渔者曰:“小人可绝乎?”
曰: “不可。君子禀阳正气而生,小人禀阴邪气而生。无阴则阳不成,无小人则君子亦不成,唯以盛衰乎其间也。阳六分,则阴四分;阴六分,则阳四分。阳阴相半,则各五分矣。由是知君子小人之时有盛衰也。治世则君子六分。君子六分,则小人四分,小人固不能胜君子矣。乱世则反是,君君,臣臣,父父,子子,兄兄,弟弟,夫夫,妇妇,谓各安其分也。君不君,臣不臣,父不父,子不子,兄不兄,弟不弟,夫不夫,妇不妇,谓各失其分也。此则由世治世乱使之然也。君子常行胜言,小人常言胜行。故世治则笃实之士多,世乱则缘饰之士众。笃实鲜不成事,缘饰鲜不败事。成多国兴,败多国亡。家亦由是而兴亡也。夫兴家与兴国之人,与亡国亡家之人,相去一何远哉!”
樵者问渔者曰:“人所谓才者,有利焉,有害焉者,何也?”
渔者曰:“才一也,利害二也。有才之正者,有才之不正者。才之正者,利乎人而及乎身者也;才之不正者,利乎身而害乎人者也。”
曰:“不正,则安得谓之才?”
曰:“人所不能而能之,安得不谓之才?圣人所以异乎才之难者,谓其能成天下之事而归之正者寡也。若不能归之以正,才则才矣,难乎语其仁也。譬犹药疗疾也,毒药亦有时而用也,可一而不可再也,疾愈则速已,不已则杀人矣。平药则常日而用之可也,重疾非所以能治也。能驱重疾而无害人之毒者,古今人所谓良药也。《易》曰:‘大君有命,开国承家,小人勿用。’如是,则小人亦有时而用之。时平治定,用之则否。《诗》云:‘它山之石,可以攻玉。’其小人之才乎!”
樵者谓渔者曰:“国家之兴亡,与夫才之邪正,则固得闻命矣。然则何不择其人而用之?”
渔者曰:“择臣者,君也;择君者,臣也。贤愚各从其类而为。奈何有尧舜之君,必有尧舜之臣;有桀纣之君,而必有桀纣之臣。尧舜之臣,生乎桀纣之世,桀纣之臣,生于尧舜之世,必非其所用也。虽欲为祸为福,其能行乎?夫上之所好,下必好之。其若影响,岂待驱率而然耶?上好义,则下必好义,而不义者远矣;上好利,下必好利,而不利者远矣。好利者众,则天下日削矣;好义者众,则天下日盛矣。日盛则昌,日削则亡。盛之与削,昌之与亡,岂其远乎?在上之所好耳。夫治世何尝无小人,乱世何尝无君子,不用则善恶何由而行也。”
樵者曰:“善人常寡,而不善人常众;治世常少,乱世常多,何以知其然耶?”
曰:“观之于物,何物不然?譬诸五谷,耘之而不苗者有矣。蓬莠不耘而犹生,耘之而求其尽也,亦未如之何矣。由是知君子小人之道,有自来矣。君子见善则喜之,见不善则远之;小人见善则疾之,见不善则喜之。善恶各从其类也。君子见善则就之,见不善则违之;小人见善则违之,见不善则就之。君子见义则迁,见利则止;小人见义则止,见利则迁。迁义则利人,迁利则害人。利人与害人,相去一何远耶?家与国一也,其兴也,君子常多而小人常鲜;其亡也,小人常多而君子常鲜。君子多而去之者,小人也;小人多而去之者,君子也。君子好生,小人好杀。好生则世治,好杀则世乱。君子好义,小人好利。治世则好义,乱世则好利。其理一也。”
钓者谈已,樵者曰:“吾闻古有伏羲,今日如睹其面焉。”拜而谢之,及旦而去。


全文翻译


樵者又问:“鱼能生吃吗?”
答:“煮熟之后可以吃。”
问:“那必然用我的柴煮你的鱼了?”
答:“当然。”
问:“那我知道了,我的柴因你的鱼而发生了变化。”
答:“你知道你的柴能煮我的鱼,可你不知道你的柴为什么能煮我的鱼。用柴煮鱼的方法早就有了,在你之前人们就知道,可世人却不知道柴的作用是火。如果没有火,你的柴就是堆积如山又有何用呢。”
樵者:“愿意听你说其中的道理。”
渔者:“火生于动,水生于静。动静相生,水火相息。水火为用,草木为体。用生于利,体生于害。利与害表现在感情上,体与用隐藏于性情中。一明一暗,只有圣人才懂柴与火的道理。就像我的鱼,没有火烧煮直到腐臭烂掉,也不能吃,又怎能养人身体呢?”
樵者问:“火的功能大于柴,我已经知道了。那为什么易燃物还要柴引燃呢?”
答:“柴是火的本体,火是柴的作用。火本无体,通过柴燃烧后才有体。柴本无作用,待火烧起后才为有用。因此,凡是有体的物体,都可以燃烧。”
问:“水有体吗?”
答:“有。”
问:“水能燃烧?”
答:“火的性质,遇水后能与之对立而不能与之相随,所以灭了。水的性质,遇火后能与之相随而不能与之相对立,所以热了。因此有热水而无凉火,是因为水火相息的原因。”
问:“火的功能来于用,它有体吗?
答:“火以用为始,以体为终,所以火是动的。水以体为始,以用为终,所以水是静的。因此,火有体,水有用,二者既相济又相息。不止水火,天下的事物都如此,就在于你如何应用。”
问:“如何应用呢?”
答:“通过意识得到的,是事物的本性;通过语言传授的,是事物的外在表现;通过眼睛观察的,是事物的形状;;通过数量计算的,是事物的多少。如何应用,阐述万物的奥妙,只可意会,而不能言传。”
问:“不可以言传,你又如何知道的?”
答:“我之所以知道,我就不是言传得到的,并非我一人不能言传,圣人也不能用语言来传授。”
问:“圣人都不能用语言来传授,那六经不是语言传授的?”
答:“那是后人编的,圣人又说了什么?”
樵者闻听,赞叹说:“天地的道理具备于人,万物的道理具备于身,变化的道理具备于神,天下的各种道理都具备了,还有什么可思虑的!我从今天开始,才知道事物的变化如此之大,还没有入门,真是白活了。”
于是,樵者解开柴生火煮鱼。二人吃饱了后而论《易》。
渔、樵二人游玩于伊水之上。渔者感叹说:“世上万物之多,纷杂繁乱。我知道游戏于天地之间,万物都以无心来了解。并非像你熟悉的那样简单。”
问:“请问如何以无心来了解万物?”
答:“无心就是无意,无意就是不把我与物分开,然后物物相通。”
问:“什么是我?什么是物?”
答:“以万物为标准,则我也是物。以我为标准,则万物也是我。我与物一样,则道理简单明了。天地也是万物,万物也是天地;我也是万物,万物也是我;我与万物之间可以相互转换。如此可以主宰天地,号令鬼神。更何况于人?何况于物?”
问:“天依靠什么?”
答:“天依靠于地。”
问:“地依赖于什么?”
答:“地依赖于天。”
问:“那天地又依附于什么?”
答:“相互依附。天依靠于地形,地依赖于天气。其地形有边涯,其天气无边际。有与无相生,形与气相息。天与地就存在于终始之间。天以它的作用为主,形体为次;地以它的形体为主,作用为次。作用的表现称作神,形体的有无称作圣。只有神和圣,才能领悟天地的变化。平民百姓天天应用而不明白,所以有灾害产生利益丧失。名誉是次要的,利益才是害人的主体。名誉产生于不知足,利益丧失于有余。危害产生于有余,实际丧失于不知足。这些都是常理。生活于世必须有物质,故贪婪的人时时寻找利益,因此有危害产生。想出人头地必须出名,故世人都争强好胜,因此有东西丧失。窃人财物称之为盗。偷盗之时,唯恐东西偷的少,等到败露后,又恐东西多定罪大。受贿与收贿,都是一种物品,可却是两种名称,是因为利与害的不同。窃人物品时存在侥幸心理,偷时嫌少,逮时嫌多。名誉的兴与毁,虽然是一件事,可却有两种结果,是因为得到或丧失的不同,大机关事业单位,是出名的地方;集贸市场,是聚利的地方,能不以争名夺利的心态居其中,虽然一日官升三级,获利百倍,又怎能伤害得了你呢?因此争名,是夺利的开始。礼让,才是取名的根本。利益到来则危害产生,名扬天下则实物丧失。利益到来又名扬天下,而且无祸害相随,只有重德者才能达到。天依靠于地,地依赖于天,其中的含义多么深远!”
渔者说:“天下将要治理的时候,人民必然崇尚行动;天下将要叛乱的时候,人民必然崇尚言论。崇尚行动,则诚实之风盛行;崇尚言论,则诡诈之风盛行。天下将要治理的时候,人民必然崇尚仁义;天下将要叛乱的时候,人民必然崇尚利益。崇尚仁义,则谦虚之风盛行;崇尚利益,则争夺之风盛行。三王时代,人民崇尚行动;五霸时代,人民崇尚言论。崇尚行动必注重于仁义,崇尚言论必注重于利益。仁义与利益相比,相差的有多么远?所以言出于口,不如行之于身,行之于身,不如尽之于心。言论出于口,人得以听到;行动在于身体,人得以见到;尽职于心,神得以知道。人的聪明不可以欺骗,更何况神的聪明?因此无愧于口,不如无愧于身,无愧于身,不如无愧于心。无愧于身比无愧于口难;无愧于心比无愧于身难。如果内心都无过错,还有什么灾难!唉!那里找无心过的人,与之交心谈畅!”
渔者问:“你知道观察天地万物的道理吗?”
樵者:“不知道。愿听你讲。”
渔者说:“所谓观物,并非以眼观物;而是以心观物,再进一步说以理观物。天下万物的存在,都有它的道理、本性和命运。所以以理观物,研究以后可以知道;以本性观物,观察以后可以知道;以命观物,推算以后可以知道。此三知,才是天下的真知,就连圣人也无法超过。超出此三知,也就不能称为圣人。鉴别万物而能成为明白的人,是因为能不隐瞒万物的形状;虽然能鉴别而不隐瞒万物的形状,但不如水能化成万物的形状;虽然水能化成万物的形状,又不如圣人能模仿万物的性情。圣人之所以能模仿万物的性情,在于圣人能反观其物。所谓反观其物,就是不以我观物。不以我观物,而是以物观物。既然以物观物,我又怎么会在俩物之间呢?因此我也是人,人也是我,我与人都是物。这样才能用天下人的目为我目,则无所不见;用天下人的耳为我耳,则无所不闻;用天下人的口为我口,则无所不言;用天下人的心为我心,则无所不谋。如此观天下,所见多么广阔!所闻多么深远!所论多么精辟!所谋多么详密!如此所见至广,所闻至远,所论至精,所谋至密,其中无一不明,岂不是至神至圣?并非我一人称为至神至圣,而是天下的人都称之为至神至圣。并非一时天下人称之为至神至圣,几千万年以后天下人仍称之为至神至圣。长此以往,都是如此。”
樵者问渔者:“你如何钓到鱼?”
答:“我用六种物具钓到鱼。”
问:“六物具备,就能钓到鱼吗?”
答:“六物具备而钓上鱼,是人力所为。六物具备而钓不上鱼,非人力所为。”
樵者不明白,请问其中的道理。
渔者说:“六物,鱼杆、鱼线、鱼漂、鱼坠、鱼钩、鱼饵。有一样不具备,则钓不上鱼。然而有六物具备而钓不上鱼的时候,这不是人的原因。有六物具备而钓不上鱼的时候,但没有六物不具备而钓上鱼的时候。因此具备六物,是人力。钓上钓不上鱼,是天意。六物不具备而钓不上鱼,不是天意是人力。”
樵者问:“有人祈祷鬼神而求福,福可以求到吗?”
答:“言行善恶,是人的因素;福与祸,是天的结果。天的规律福善祸灾,鬼神岂能违背?自己做的坏事,岂能逃避。上天降下的灾祸,祈祷又有什么用?修德积善,是君子的本分。这样做就不会有灾祸来找!”
问:“有行善的而遇祸,有行恶的而获福。为什么?”
答:“这是有幸与不幸之分。幸与不幸,是命。遇与不遇,是分。命与分,人怎么能逃避?”
问:“什么是分?什么是命?”
答:“坏人遇福,不是分是命,遇祸是分不是命。好人遇祸,是命不是分,遇福是分不是命。”
渔者对樵者说:“人与人的亲情,莫过于父子;人与人的疏远莫过于路人,如果利与害在心里,父子之间就会像路人一样远,父子之间的亲情,属于天性,利与害都能夺掉,更何况不是天性的,利与害祸人,如此之深,不能不谨慎!路人相遇一过了之,并无相害之心,是因为没有利与害的关系。若有利与害的关系,路人与路人、父与子之间又如何选择呢?路人若能以义相交,又何况父子之亲呢!所谓义,是谦让之本。而利益是争夺之端。谦让则有仁义,争夺则有危害。仁义与危害相去甚远。尧、舜是人,桀、纣也是人。人与人同,而仁义与危害却不同。仁慈因义气而起,危害因利益而生。利益不会因义气而争夺,否则不会有臣杀君、子杀父之事。路人相逢,也不可能因一眼而情投意合。”
樵者问渔者:“我经常扛柴,扛一百斤也伤不了我,再加十斤就伤了我,为什么?”
渔者答:“扛柴我不清楚。以我钓鱼之事论之,其理一样。我经常钓到大鱼,与我较量。欲弃之,不舍得,欲钓取,又不容易。很长时间才能钓上来,有好几次溺水的危险。这不也是伤身的忧患?钓鱼与扛柴虽不一样,但因贪而受伤则无两样。一百斤,力所能及,再加十斤,则在你力所之外。力所之外,加一毫都是有害,何况十斤!我贪鱼,又何异于你贪柴呢?”
樵者感叹道:“从今以后,我知道做事量力而行才是有智慧的。”
樵者问:“你是知易理的人。请问易有太极,太极是何物?”
答:“无为之本。”
问:“太极生两仪,两仪是天地的称呼吗?”
答:“两仪,天地之祖,并非单指天地。太极一分为二,先得到的一为一,后得到的一为二,一与二叫做两仪。”
问:“两仪生四象,四象为何物?”
答:“四象就是阴阳刚柔。阴阳可以生天,刚柔可以生地。一切事物的根本,于此为极点。”
问:“四象生八卦。八卦是什么?”
答:“八卦就是乾、坤、离、坎、兑、艮、震、巽。是事物发展终始、盛衰的表现。两两相重,则六十四卦生出,易学之道就具备了。”
樵者问渔者:“如何见到天地的本性?”
答:“先阳耗尽,后阳出生。则天地开始出现,变化到中期日月开始周行,变化到末期星辰显现。万物死生,寒暑代谢,昼夜变迁,事物以此相变。当天地运行到终了必然变化,变则通,通则久。所以《易》中象言‘先王到最后一日闭关,哪儿也不去’,是顺天行所故。”
问:“无妄(卦名),属于灾,是什么原因?”
答:“妄是欺骗,得之必有祸,因此称妄。顺天意而行动,有祸秧及也不叫祸而叫灾。就像农民想着丰收而不去护理庄稼,其结果荒芜,不是祸是什么?农民勤劳治理庄稼而遭水涝或干旱,其结果荒芜,不是灾是什么?所以《易》中象言‘先王以诚对万物’,贵于不欺骗。”
问:“姤(gòu)(卦名),是什么?”
答:“姤是相遇。以柔遇刚。与夬(guài)卦相反。夬始强壮,姤由弱遇壮,由阴遇阳。故称为姤。观姤,天地的本性由此可见。圣人以德比喻,没有不明白的。所以《易》中象言‘姤施命于天下,就像走在霜雪之上,小心谨慎’,就在于此。”
渔者接着说:“春天是阳气的开始,夏天是阳气的极限;秋天是阴气的开始,冬天是阴气的极限。阳气开始则天气温暖,阳气极限则天气暑热;阴气开始则天气凉爽,阴气极限则天气寒冷。温暖产生万物,暑热成长万物;凉爽收藏万物,寒冷肃杀万物。皆是一气四种表现。其生万物也如此。”
樵着问渔者:“人为万物之灵,是如何表现的?”
渔者回答:“人的目能收万物之色,耳能收万物之声,鼻能收万物之气,口能收万物之味。声色气味,万物之本,目耳鼻口,人人皆用。物体本无作用,通过变化来表现作用;作用也并不是表现在一个物体上,而是不同的物体有不同的作用。由于物体和作用相交,则人和物的变化规律就具备了。然而人也是物,圣人也是人。有一物、百物、千物、万物、亿物、兆物。身为一物,就可以征兆万物,只有人。有一人、百人、千人、万人、亿人、兆人。生为一人,而能征兆他人,只有圣人。因此知道人是物的至尊;圣人是人的至尊。物的至尊为物中之物,人的至尊为人中之人。所以物的至极为至物,人的至极为至人。以一物知万物、以一人知万人,不是圣人是什么?人不是万物之灵,我不信。为什么?因为人能以一心观万心,以一身观万身,以一物观万物,以一世观万世;又能以心代天意,以口代天言,以手代天工,以身代天事;又能上识天时,下晓地理,中尽物情,通照人事;又能弥纶天地,出入造化,进退古今,表里人物。唉!圣人并非世世可见,我虽不能亲眼见到,但我观察其心迹,探访其行踪,研究其作用,虽经亿万年也能以理知道。有人告诉我说,天地之外,还有另外的天地万物,和此天地万物不一样。而我不得而知。并非我不得而知,连圣人也不得而知。凡说知道的,其实内心并不知道。而说出来的,也只是说说而已。既然内心都不明白,嘴又能说出什么?心里不知道而说知道的,叫做妄知。嘴说不清而又要说的,叫做妄言。我又怎么能相信妄人的妄言和妄知呢?”
渔者对樵者说:“仲尼说的好:‘殷继承于夏礼,所遇的损益便可知道;周继承于殷礼,所遇的损益也可知道。其次继承周礼的,虽经百世也可知道。’如此,何止百世而已!亿千万世,都可以知道。人都知道仲尼叫仲尼,却不知道仲尼为什么叫仲尼。不想知道仲尼为什么叫仲尼则已,若想知道仲尼为什么叫仲尼,则舍弃天地会怎么样?人都知道天地为天地,却不知道天地为什么为天地,不想知道天地为什么为天地则已,若想知道天地为什么为天地,则舍弃动静会怎么样?一动一静,天地至妙,一动一静之间,天地人至妙。因此仲尼之所以能尽三才之道,是因其行没有辙迹。所以有人说:‘仲尼什么也没说。’又说:‘天什么也没说!但四时运行,百物出生。’这些你知道吗?”
渔者接着说:“大事中:权力与变化谁重要?并非圣人不能讲清楚。变化过后可知天地的消长,掌权之后可知天下的轻重。消长是时间的表现,轻重是事物的表现。时间有亨通与闭塞,事物有损耗与收益。圣人若不知随时间亨通与闭塞之道,又怎知变化之所为呢?圣人若不知随时间损耗与收益之道,又怎知权力之所为呢?运用消长的是变化,处置轻重的是权力。因此权力与变化,是圣人的修行之一。”
樵者问渔者:“人死后有灵魂存在,有这种事么?”
答:“有。”
问:“如何才能知道?”
答:“以人为知。”
问:“什么样的叫人?”
答:“目耳鼻口心胆脾肾之气全的叫人。心之灵称神,胆之灵称魄,脾之灵称魂,肾之灵称精。(中医认为:心之灵称神,肝之灵称魂,脾之灵称意,肺之灵称魄,肾之灵称精。这里有不同的见解,不知原文有误?还是有何深意?——译者注)心之神表现在目,称为视;肾之精表现在耳,称为听;脾之魂表现在鼻,称为臭;胆之魄表现在口,称为言。八者具备,才可称之为人。人,禀天地万物之秀气而生。然而也有缺少某一方面的人,各归其类。如果各方面都齐全的人,则称为全人。全人得万物中的中和之气,则称为全德之人。全德之人,为人中之人。人中之人,则是仁人之称。只有全人,才能得到仁人之称。人之生,在于气行。人之死,则是形体返还。气行则神魂交,形返则精魄存。神魂行于天,精魄返于地。行于天,称之为阳行,返于地,称之为阴返。阳行于白天而夜间潜伏,阴返于夜间而白天潜伏。所以知道太阳是月亮的形状,月亮是太阳的影子,阳者是阴者的形状,阴者是阳者的影子,人是鬼的形状,鬼是人的影子。有人说,鬼无形而不可知,我不相信。”
樵者问渔者:“小人能灭绝吗?”
答:“不能。君子禀阳正气而生,小人禀阴邪气而生。无阴则阳不生,无小人则君子不生,只有盛衰的不同。阳六分,则阴四分;阴六分,则阳四分。阴阳各半,则各占五分。由此而知,君子与小人各有盛衰之时。太平盛世时期,君子占六分,小人占四分,小人不能战胜君子。君臣、父子、兄弟、夫妇各安其道。世间纷乱时期正相反。君不君、臣不臣、父不父、子不子、兄不兄、弟不弟、夫不夫、妇不妇则失其道。这是由治世或乱世所造成的。君子常以身作则胜过空话连篇,小人常空话连篇胜过实际行动。所以盛世时期诚实的人多,乱世时期奸诈的人多。诚实容易成事,奸诈容易败事,成事则国兴,败事则国亡。一个家庭也如此。兴家、兴国之人,与亡国、亡家之人,相差的是多么的远!”
樵者问:“人有才,有的有益,有的有害,为什么?”
答:“才为一,益与害为二、有才正、才不正之分、才正,益于身而无害,才不正,益于身而害人。”
问:“才不正,又如何成为才呢?”
答:“人所不能做的你能做到,能不成为才吗?圣人所以怜惜成才难,是因为能成天下事而又正派的人很少。若不正派,虽然有才,也难称有仁义。比如吃药治病,毒药也有用的时候,可不能一而再再而三的用。病愈则速停,不停则是杀人了。平常药日常皆可用,但遇重病则没有疗效。能驱除重病而又不害人的毒药,古今都称为良药。《易》说:‘开国立家,用君子不用小人。’如此,小人也有有用的时候。安邦治国,则不要用小人。《诗》说:‘它山之石,可以攻玉。’就是借用小人之才。”
问:“国家兴亡,与人才的正邪,各有其命。哪为什么不择人而用呢?”
答:“择臣者,是君王的事,择君者,是臣民的事,贤愚各从其类。世上有尧、舜之君,必有尧、舜之臣;有桀、纣之君,必有桀、纣之臣。尧舜之臣,生于桀、纣之世,则不会成为桀纣之臣。生于尧舜之世并非是他的所为,他想要为祸为福,可不是想干就能干的。上边所好的下边必效仿。君王的影响,还用驱赶去执行吗?上好义,则下必好义,而不义的人则远离;上好利,则下必好利,而不好利的人则远离,好利者多,则天下日渐消亡;好义者众,则天下日渐兴旺。日盛则昌,日消则亡。昌盛与消亡,难道不远吗?都是在上好恶影响的。治国安民之时何尝无小人,乱世之际又何尝无君子,没有君子和小人,善恶又如何区分呢?”
樵者问:“善人常少,不善人常多;盛世时代短,乱世时期长。如何鉴别呢?”
答:“观察事物。什么事物不能表现出来?比如五谷,耕种之后有长不出来的,而逢野生物不用耕种就能长出来,耕种之后想要全部收获,是不可能的!由此而知君子与小人之道,也是自然而生。君子见善事则欢喜,见不善事则远离;小人见善事则痛苦,见不善事则欢喜。善恶各从其类。君子见善事则去做,见不善事则阻止;小人见善事则阻止,见不善事则去做;君子见义则迁,见利则止;小人见义则止,见利则迁。迁义则益人,迁利则害人;益人与害人,相去有多远?家与国一样兴旺则君子常多,小人常少;消亡则小人常多君子常少。君子多小人躲避,小人多君子躲避。君子好生,小人好杀。好生则治国安民,好杀则祸国殃民。君子好义,小人好利。治国安民则好义,祸国殃民则好利。其道理是一样的。”
渔者说完,樵者感慨万分:“我听说上古有伏羲,今日好像一睹其面。”对渔者再三拜谢,相别而去。

\backmatter

\end{document}