% 吴子
% 吴子.tex

\documentclass[a4paper,12pt,UTF8,twoside]{ctexbook}

% 设置纸张信息。
\RequirePackage[a4paper]{geometry}
\geometry{
	%textwidth=138mm,
	%textheight=215mm,
	%left=27mm,
	%right=27mm,
	%top=25.4mm, 
	%bottom=25.4mm,
	%headheight=2.17cm,
	%headsep=4mm,
	%footskip=12mm,
	%heightrounded,
	inner=1in,
	outer=1.25in
}

% 设置字体,并解决显示难检字问题。
\xeCJKsetup{AutoFallBack=true}
\setCJKmainfont{SimSun}[BoldFont=SimHei, ItalicFont=KaiTi, FallBack=SimSun-ExtB]

% 目录 chapter 级别加点(.)。
\usepackage{titletoc}
\titlecontents{chapter}[0pt]{\vspace{3mm}\bf\addvspace{2pt}\filright}{\contentspush{\thecontentslabel\hspace{0.8em}}}{}{\titlerule*[8pt]{.}\contentspage}

% 设置 part 和 chapter 标题格式。
\ctexset{
	part/name= {},
	part/number={},
	chapter/name={第,篇},
	chapter/number={\chinese{chapter}}
}

% 设置古文原文格式。
\newenvironment{yuanwen}{\bfseries\zihao{4}}

% 设置署名格式。
\newenvironment{shuming}{\hfill\bfseries\zihao{4}}

\title{\heiti\zihao{0} 吴子}
\author{吴起}
\date{战国}

\begin{document}

\maketitle
\tableofcontents

\frontmatter
\chapter{前言}

《吴子》又称《吴子兵法》、《吴起兵法》,是一部兵法著作。提出以治为胜,赏罚严明,主张在军队实行“进有重赏,退有重刑”,做到“令行禁止,严不可犯”。提出“用兵之法,教戒为先”,主张通过严格的军事训练,使士卒掌握各种作战本领,提高整个军队的战斗力。强调“简募良材”,根据士卒体力、技能等条件的不同,合理分工和编组,实现军队的优化组合。

《吴子》相传为战国时吴起所著。《韩非·五蠢篇》:“境内皆言兵,藏孙、吴之书者家有之。”《史记·孙子吴起列传》:“世俗所称师旅,皆道《孙子 十三篇》,《吴起兵法》世多有。”《汉书·艺文志》著录有《吴起》四十八篇。可见《吴起兵法》在战国末年和汉初还在流传。但1972年,山东临沂银雀山汉墓出土竹简中,兵书很多,却未见该书\footnote{参见《临沂银雀山汉墓发掘简报》。}。据郭沫若同志考证:“但可惜这书\footnote{指《吴起兵法》。}是亡了”。“故今存《吴子》实可断言为伪。以笔调觇之,大率西汉中叶时人之所依托。\footnote{见郭著《青铜时代·述吴起》第三节。}”。书中提到的“笳\footnote{见《应变 第五》。}”、“鞍\footnote{见《治兵 第三》}”等都是吴起时代还没有的东西,特别是“骑三千匹\footnote{见《励士 第六》。}”这样大规模地使用骑兵部队更是那时所没有的。

《隋书·经藉志》著录:《吴起兵法》一卷。按《隋书》为唐初魏征所撰。而魏征在其《群书治要》中所摘录的《吴子》的《图国》、《论将》、《治兵》、《励土》四篇内容,大体与今本《吴子》相同。这样,原书亡佚时间至少在唐初以前了。

今本《吴子》分:《图国》、《料敌》、《治兵》、《论将》、《应变》、《励土》六篇。虽非吴起原著,但就其内容来看,仍不失为一部较有价值的兵书,不管是谁所著,都是前人留下的宝贵遗产。

宋神宗年间(公元1078-1085),将其列入《武经七书》,颁行武学,为将校所必读,颇受重视。

现有英、日、法、俄等文字译本。

\mainmatter

\part{卷上}

\chapter{图国}

图国,即筹划治理国家。本篇大体是论述治国、治军、兴兵作战、亲民用贤等国家大计。

\begin{yuanwen}
吴起儒服,以兵机见魏文侯。

文侯曰:“寡人不好军旅之事。”

起曰:“臣以见占隐,以往察来,主君何言与心违?今君四时,使斩离皮革,掩以朱漆,画以丹青,炼(烁)以犀象,冬日衣之则不温,夏日衣之则不凉。为长戟二丈四尺,短戟一丈二尺。革车掩户,缦轮笼毂,观之于目则不丽,乘之于国(田)则不轻,不识主君安用此也?若以备进战退守,而不求能用者,譬犹伏鸡之搏狸,乳犬之犯虎,虽有斗心,随之死矣。昔承桑氏之君,修德废武,以灭其国。有扈氏之君,恃众好勇,以丧其社稷。明主鉴兹,必内修文德,外治武备。故当敌而不进,无逮于义矣。僵尸而哀之,无逮于仁矣。”

于是文侯身自布席,夫人捧觞,醮吴起于庙,立为大将崇(守)西河。与诸侯大战七十六,全胜六十四,馀则钧解。辟土四面,拓地千里,皆起之功也。
\end{yuanwen}

吴起穿着儒生的服装,以兵法进见魏文侯。
文侯说:“我不爱好军事。”
吴起说:“我从表面现象推测您的意图,从您过去的言行观察您将来的抱负,您为什么要言不由衷呢?现在您一年到头杀兽剥皮,在皮革上涂以红漆,给以色彩,烫上犀牛和大象的图案。[若用来做衣服,]冬天穿着不暖和,夏天穿着不凉快。制造的长戟达二丈四尺,短戟达一丈二尺。用皮革把重车护起来,车轮车毂也加以覆盖,这看在眼里并不华丽,坐去打猎也不轻便,不知您要这些东西做什么?如果说您准备用来作战,却又不去寻求会使用它们的人。这就好象孵雏的母鸡去和野猫搏斗,吃奶的小狗去进犯老底,虽有战斗的决心,随之而来的必然是死亡。从前承桑氏的国君,只许文德,废驰武备,因而亡国。有扈氏的国君仗着兵多,恃勇好战,[不修文德,]也丧失了国家。贤明的君主有鉴于此,必须对内修明文德,对外做好战备。所以,面对敌人而不敢进战,这说不上是义;看着阵亡将士的尸体而悲伤,这说不上是仁。”
于是文侯亲自设席,夫人捧酒,宴请吴起于祖庙,任命他为大将,主持西河防务。后来,吴起与各诸侯国大战七十六次,全胜六十四次,其余十二次也来分胜负。魏国向四面扩张领土达千里,都是吴起的功绩!

\begin{yuanwen}
吴子曰:“昔之图国家者,必先教百姓而亲万民。有四不和:不和于国,不可以出军;不和于军,不可以出陈;不和于陈,不可以进战;不和于战,不可以决胜。是以道之主,将用其民,先和而造大事。不敢信其私谋,必告于祖庙,启于元龟,参之天时,吉乃后举。民知爱其命,惜其死,若此之至,而与之临难,则士以进死为荣,退生为辱矣。”
\end{yuanwen}

吴起说:“从前谋求治好国家的君主,必先教育‘百姓’,亲近。‘万民’。在四种不协调的情况下,不宜行动:国内意志不统一,不可以出兵;军队内部不团结,不可以上阵;临战阵势不整齐,不可以进战,战十行动不协调,不可能取得胜利。因此,英明的君主,准备用他的民众去作战的时候,必先搞好团结然后才进行战争。虽然如此,他还不敢自信其谋划的正确,必须祭告祖庙,占卜凶吉,参看天时,得到吉兆然后行动。让民众知道国君爱护他们的生命,怜惜他们的死亡,做到这样周到的地步,然后再率领他们去打仗,他们就会以尽力效死为光荣,以后退偷生为耻辱了。”

\begin{yuanwen}
吴子曰:“夫道者,所以反本复始;义者,所以行事立功;谋者,所以违害就利;要者,所以保业守成。若行不合道,举不合义,而处大居贵,患必及之。是以圣人绥之以道,理之以义,动之以礼,抚之以仁。此四德者,修之则兴,废之则衰。故成汤讨桀而夏民喜说,周武伐纣而殷人不非;举顺天人,故能然矣。”
\end{yuanwen}

吴子说:“‘道’是用来恢复人们善良的天性的,‘义’是用来建功立业的。‘谋’是用来趋利避害的。‘要’是用来巩固、保全事业成果的。如果行为不合于‘道’,举动不合于‘义’,而掌握大权,分居要职,必定祸患无穷。所以,‘圣人’用‘道’来安抚天下,用‘义’来治理国家,用‘礼’来动员民众,用‘仁’来抚慰民众。这四项美德发扬起来国家就兴盛,废弃了国家就衰亡。所以,商汤讨伐夏桀夏民很高兴,周武王讨伐殷纣殷人却不反对。这是由于他们进行的战争,顺手天理,合乎人情,所以才能这样。”

\begin{yuanwen}
吴子曰;“凡制国治军,必教之以礼,励之以义,使有耻也。夫人有耻,在大,足以战;在小,跳以守矣。然战胜易,守胜难。故曰:‘天下战国,五胜者祸,四胜者弊,三胜者霸,二胜者王,一胜者帝。’是以数胜得天下者稀,以亡者众。”
\end{yuanwen}

关于说:“凡治理国家和军队,必须用礼来教育人们,用义来勉励人们,使人们鼓起勇气。人们有了勇气,力量强大就能出战,力量弱小也能竖守。然而取得胜利比较容易,巩固胜利却很困难。所以说,天下从事战争的国家,五战五胜的,会招来祸患;四战四胜的,会国力疲弊;三战三胜的,可以称霸;二战二胜的,可以称王;一战一胜的,可以成就帝业。因此,靠多次战争的胜利而取得天下的少,由此而亡国的却很多。”

\begin{yuanwen}
吴子曰:“凡兵之所起者有五:一曰争名,二曰争利,三曰积德恶,四曰内乱,五曰因饥。其名又有五:一曰义兵,二曰强兵,三曰刚兵,四曰暴兵,五曰逆兵。禁暴救乱曰义,恃众以伐曰强,因怒兴师曰刚,弃礼贪利曰暴,国乱人疲,举事动众曰逆。五者之数,各有其道:义必以礼服,强必以谦服,刚必以辞服,暴必以诈服,逆必以权服。”
\end{yuanwen}

吴子说:“战争的起因有五种:一是争名,二是争利,三是积仇,四是内乱,五是饥荒。用兵的性质也有五种:一是义兵,二是强兵,三是刚兵,四是暴兵,五是逆兵。禁暴除乱,拯救危难的叫义兵,仗恃兵多,征伐别国的叫强兵,因怒兴兵的叫刚兵,背理贪利的叫暴兵,不顾国乱氏疲,兴师动众的叫逆兵。对付这五种不同性质的用兵,各有不同的方法,对义兵必须用道理折服它,对强兵必须用谦让悦服它,对刚兵必须用言辞说服它,对暴兵必须用计谋制服它,对逆兵必须用威力压服它。”

\begin{yuanwen}
武侯问曰:“愿闻治兵、料人、固国之道。”

起对曰:“古之明王,必谨君臣之礼,饰上下之仪,安集吏民,顺俗而教,简募良材,以备不虞。昔齐桓募士五万,以霸诸侯。晋文召为前行四万,以获其志。秦缪置陷陈三万,以服邻敌。故强国之君,必料其民。民有胆勇气力者,聚为一卒。乐以进战效力以显其忠勇者,聚为一卒。能逾高超远轻足善走者,聚为一卒。王臣失位而欲见功于上者,聚为一卒。弃城去守,欲除其丑者,聚为一卒。此五者军之练锐也。有此三千人,内出可以决围,外入可以屠城矣。”
\end{yuanwen}

武侯对吴起说“我想知道关于治理军队、统计人口、巩固国家的方法。”
吴起回答说:“古时贤明的国君,必严守君臣间的礼节,讲究上下间的法度,使吏民各得其所,按习俗进行教育,选募能干的人,以防不测。从前齐桓公招募勇士五万,赖以称霸诸侯。晋文公招集勇士四万作为前锋,以得志于天下,泰穆公建立冲锋陷阵的部队三万,用以制服邻近的敌国。所以,发备图强的君主,必须查清人口,把勇敢强壮的人,编为一队。把乐意效命显示忠勇的人,编为一队。把能攀高跳远、轻快善走的人,编为一队。把因罪罢官而想立功报效的人,编为一队。把曾弃守城邑而想洗刷耻辱的人,编为一队。这五种编队都是军队中的精锐部队。如果有这样三十人,由内出击可以突破敌人的包围,由外进攻,可以摧毁敌人的城邑。”

\begin{yuanwen}
武侯曰:“愿闻陈必定,守必固,战必胜之道。”

起对曰:“立见且可,岂直闻乎!君能使贤者居上,不肖者处下,则陈已定矣。民安其田宅,亲其有司,则守已固矣。百姓皆是吾君而非邻国,则战已胜矣。”
\end{yuanwen}

武侯说:“我想知道如何能使阵必定、守必固、战必胜的方法。”
吴起答:“立即看到成效都可以,岂只是知道而已!您能将有才德的人加以重用,没有才德的人不予重用,那末阵就已稳定了。民众安居乐业,亲敬官吏,那末守备就已巩固了。百姓都拥护自己的国君,而反对敌国。那末战争就已胜利了。”

\begin{yuanwen}
武侯尝谋事,群臣莫能及,罢朝而有喜色。

起进曰:“昔楚庄王尝谋事,群臣莫能及,罢朝而有忧色。申公问曰:‘君有忧色,何也?’曰:‘寡人闻之,也不绝圣,国不乏贤,能得其师者五,能得其友者霸。今寡人不才,而群臣莫及者,楚国其殆矣。’此楚庄王之所忧,而君说之,臣窃惧矣。”于是武侯有惭色。
\end{yuanwen}

武侯曾经和群臣商议国事,群臣的见解都不如他,他退朝以后面有喜色。吴起进谏说:“从前楚庄王曾经和群臣商议国事,群臣都不及他,他退朝后面有忧色。申公问他:‘您为什么面有忧色呢?’楚庄王说:‘我听说世上不会没有圣人,国家不会缺少贤人,能得到他们做老师的,可以称王,得到他们做朋友的,可以称霸。现在我没有才能,而群臣还不如我,楚国真危险了。’这是楚庄王所忧虑的事,您却反而喜悦,我私下深感忧惧。”于是武侯表示很惭愧。

\chapter{料敌}

料敌,即判断敌情。全篇主要是分析六国形势,治军、选士以及判断敌情和因敌致胜的方法。

\begin{yuanwen}
武侯谓吴起曰:“今秦胁吾西,楚带吾南,赵冲吾北,齐临吾东,燕绝吾后,韩据吾前。六国兵四守,势甚不便,忧此奈何?”

起对曰:“夫安国家之道,先戒为宝。今君已戒,祸其远矣。臣请论六国之俗:夫齐陈重而不坚,秦陈散而自斗,楚陈整而不久,燕陈守而不走,三晋陈治而不用。”
\end{yuanwen}

武侯对吴起说:“今秦国威胁着我西部,楚国围绕着我南部,赵国面对着我北部,齐国紧逼着我东部,燕国阻绝着我的后面,韩国据守在我的前面,六国军队四面包围着我们,形势非常不利,我对此很忧虑,该怎么办呢?”
吴起答:“保障国家安全的方法,先有戒备是最重要的。现在您已经有了戒备,离祸患就远了。请允许我分析一下六国军阵的情况,齐国阵势庞大但不坚固,泰国阵势分散但能各自为战,楚国阵势严整但不能持久,燕国阵势长于防守但不善于机动,韩、赵阵势整齐但不顶用。”

\begin{yuanwen}
夫齐性刚,其国富,君臣骄奢而简于细民,其政宽而禄不均,一陈两心,前重后轻,故重而不坚。击此之道,必三分之,列其左右,胁而从之,其陈可坏。

秦性强,其地险,其政严,其赏罚信,其人不让,皆有斗心,故散而自战。击此之道,必先示之以利而引去之,士贪于得而离其将,乘乖猎散,设伏投机,其将可取。

楚性弱,其地广,其政骚,其民疲,故整而不久。击此之道,袭乱其屯,先夺其气,轻进速退,弊而劳之,勿与战争,其军可败。

燕性悫,其民慎,好勇义,寡诈谋,故守而不走。击此之道,触而迫之,陵而远之,驰而后之,则上疑而下惧,谨我车骑必避之路,其将可虏。

三晋者,中国也,其性和,其政平,其民疲于战,习于兵,轻其将,薄其禄,士无死志,故治而不用。击此之道,阻陈而压之,众来则拒之,去则追之,以倦其师。此其势也。
\end{yuanwen}

“齐国人性情刚强,国家富足,君臣骄奢,忽视民众利益,政令松驰而待遇不均,一阵之中人心不齐,兵力布署前重后轻,所以阵势庞大但不坚固。打它的方法,必须把我军区分为三部,各以一部侧击其左右两翼,另以一部乘势从正面进攻,它的阵势就可以攻破了。泰人性情强悍,地形险要,政令严格,赏罚严明,士卒临阵争先恐后,斗志旺盛,所以能在分散的阵势中各自为战。打它的方法首先以利诱它,当其士卒因争利而脱离其将领掌握时,就来其混乱打击其零散的部队,并设置伏兵,饲机取胜,它的将领就可以擒获。楚国人性情柔弱,领土广大,政令紊乱,民力疲困,所以阵势虽然严整但不能持久,打它的方法,要袭扰它的驻地,先挫折它的士气,然后突然进击,突然撤退,使其疲于应付,不要和它决战,这样就可打败它的军队。燕国人性情诚实,行动谨慎,好男尚义,缺少诈谍,所以善于固守而不善于机动。打它的方法,是一接触就压迫它,打一下就撤走,并奔袭它的后方,这样,就会使它上下疑惧,再将我车骑埋伏在敌人撤退必经的道路上,它的将领就可被我俘虏。韩赵是中原的国家“,其民性温顺,其政令平和,其民众疲于残祸,久经战争,轻视其将帅,不满其待遇,士无死忠,所以,阵势虽然整齐但不中用,打它的方法,用坚强的阵势迫近它,敌众来攻就阻击它,敌人退却就追击它,这样来疲惫它的军队。这是六国的大概形势。”

\begin{yuanwen}
“然则一军之中必有虎贲之士,力轻扛鼎,足轻戎马,搴旗取将,必有能者。若此之等,选而别之,爱而贵之,是谓军命。其有工用五兵材力健疾,志在吞敌者,必加其爵列,可以决胜。厚其父母妻子,劝赏畏罚。此坚阵之士,可与持久。能审料此;或以击倍。”

武侯曰“善!”
\end{yuanwen}

“既然这样,那么我全军之中,就必定有“虎贲”之士,其力气之大可以轻易举鼎,行动轻捷能够追及战马。在战斗中,夺取敌旗,斩杀敌将,必须这样有能力的人。这样的人才,必须选拔出来,爱护并重用他们,他们就是军队的精华。凡有善于使用各种兵器、身强力壮、动作敏捷、志在杀敌的,一定要加官晋爵,这样就可以用他们来决战决肚。优待其父母妻子,用奖赏鼓励他们,用惩罚警诫他们,使他们成为加强阵势的骨干,用以进行持久战斗。若能清楚地了解这些问题,就可以打败成倍的敌人了。”
武候说:“很好。”

\begin{yuanwen}
吴子曰:“凡料敌,有不卜而与之战者八:一曰疾风大寒,早兴寤迁,刊木济水,不惮艰难;二曰盛夏炎热,晏兴无间,行驱饥渴,务于取远;三曰师既淹久,粮食无有,百姓怨怒,妖祥数起,上不能止;四曰军资既渴,薪刍既寡,天多阴雨,欲掠无所;五曰徒众不多,水地不利,人马疾疫,四邻不至;六曰道远日暮,士众劳惧,倦而未食,解甲而息;七曰将薄吏轻,士卒不固,三军数惊,师徒无助;八曰陈而未定,舍而未毕,行孤涉险,半隐半出。诸如此者,击之勿疑。有不占而避之者六:一曰土地广大,人民富众;二曰上爱萁下,惠施流布;三曰赏信刑察,发必得时,四曰陈功居列,任贤使能;五曰师徒之众,兵甲之精;六曰四邻之助,大国之援。凡此不如敌人,避之勿疑,所谓见可而进,知难而退也。”
\end{yuanwen}

吴起说:“判断敌情,不必占卜就可与其交战的,有八种情况。一是在大风严寒中,昼夜行军,伐木渡河,不顾部队艰难的。二是在盛夏炎热,出发很迟,途中不休息,行军急速,又饥又渴,只顾赶往远地的。三是出兵已久,粮食用尽,百姓怨怒,谣言屡起,将冲不能制止的。四是军资耗尽,柴草不多,阴雨连绵,无处可掠夺的。五是兵力不多,水土不服,人马多病,四邻援军未到的。六是路运日暮,部队疲劳恐惧,困倦未食,解甲休息的。七是将吏无威信,军心不稳定。三军屡次惊慌,而又孤主无援的。八是部署未定,宿营未毕,翻山越险只过了一半的。遇到这类情况,都应迟速进击,不要迟妄乏。
“不必占卜而应避免和敌人作战的情况有六种。一是土地广大,人口众多而且留足的。二是上爱其下,恩惠普及的。三是赏罚严明,行动及时的。四是论功叙位,任用贤能的。五是军队众多,装备精良的。六是有四邻帮助,大国支援的。凡是这些条件都不如敌人时,就应避免和它作战而不必迟疑,这就是所谓见可而进,知难而退。”

\begin{yuanwen}
武侯问曰:“吾欲观敌之外以知其内,察其进以知其止,以定胜负,可得闻乎?”

起对曰:“敌人之来,荡荡无虑,旌旗烦乱,人马数顾,一可击十,必使无措。诸侯未会,君臣未和,沟垒未成,禁令未施,三军匈匈,欲前不能,欲去不敢,以半击倍,百战不殆。”
\end{yuanwen}

武候问:“我想通过观察敌人的外部表现来了解它的内部情况,从观察敌人的行动来了解它的真实意图,从而判定胜负,你可以[把这个要领]说给我听听吗?”
吴起答:“敌人来时行动散漫而无顾虑,旗帜纷乱不整,人马东张西望,这样以一击十,就可使敌人惊慌失措。故人各路军队尚未会师,君臣意见不和,工事未完成,禁令未实施,三军吵吵嚷嚷,想前进不能前进,想后退不能后退,在这种情况下以半击倍,可以百战不败。”

\begin{yuanwen}
武侯问敌必可击之道,起对曰:“用兵必须审敌虚实而趋其危。敌人远来新至,行列未定,可击;既食未设备;可击;奔走,可击;勤劳,可击;未得地利,可击;失时不从,可击;旌旗乱动,可击;涉长道,后行未息,可击;涉水半渡,可击;险道狭路,可击;陈数移动,可击;将离士卒,可击;心怖,可击。凡若此者,选锐冲之,分兵继之,急击勿疑。”
\end{yuanwen}

武侯问敌人在什么情况下,我军可以打击它呢?
吴起答:“用兵必须查明敌人的虚实而冲击它的弱点。敌人远来新到,部署未定,可打。刚吃完饭,还未戒备,可打慌乱奔走的,可打。疲劳的,可打没有占据有利地形的,可打。天候季节对敌不利的,可打。部队混乱的,可打。经长途行军,其后队尚未得到休息的,可打。涉水半渡的,可打。通过险道隘路的,可打。阵势频繁移动的,可打。将帅脱离部队的,可打。军心恐怖的,可打。凡是遇着上述情况,就应先派精锐的部队冲向敌人,并继续派遣兵力接应它,必须要迅速进击,不可迟疑。”

\chapter{治兵}

治兵,即治理军队的意思。全篇共八节,主要讲的是进军、作战、训练、编组、指挥等问题。

\begin{yuanwen}
武侯问曰:“进兵之道何先?”

起对曰:“先明四轻、二重、一信。”

曰:“何谓也?”

对曰:“使地轻马,马轻车,车轻人,人轻战。明知阴阳,则地轻马;刍秣以时,则马轻车;膏锏有余,则车轻人;锋锐甲坚,则人轻战;进有重赏,退有重刑,行之以信,令制远,此胜之主也。”

武侯问:“进兵的方法什么是首要的?”
吴起答:“首先要懂得四轻、二重、一信。”
武侯又问:“这话怎么讲呢?”
吴起说:“[四轻]就是地形便于驰马,马便于驾车,车便于载人,人便于战斗。了解地形的险易,[善于利用]地形,就便于驰马。饲养适时,马就便于驾车。车轴经常保持润滑,车就便于载人。武器锋利,皑甲坚固,人就便于战斗。[二重]就是近战有重贫,后退有重刑。[一信]就是赏罚必信。确能做到这些,就掌握了胜利的主要条件。

武侯问曰:“兵何以为胜?”

起对曰:“以治为胜。”

又问曰:“不在众寡?”

对曰“:“若法令不明,赏罚不信,金之不止,鼓之不进,虽有百万何益于用?所谓治者,居则有礼,动则有威,进不可当,退不可追,前却有节,左右应麾,虽绝成陈,虽散成行。与之安,与之危,其众可合不可离,可用而不可疲,投之所往,天下莫当。名曰父子之兵。”

武侯问:“军队靠什么打胜仗?”
吴起答:“治理好军队就能打胜仗。”
又问:“不在于兵力多少吗?”
吴起答:“如果法令不严明,赏罚无信用,鸣金不停止,擂鼓不前进,虽有百万之众,又有什么用处?所谓治理好,就是平时守礼法,战时有威势,前进时锐不可挡,后退时速不可追,前进后退有节制,左右移动听指挥,虽被隔断仍能保持各自的阵形,虽被冲散仍能恢复行列。上下之间同安乐、共患难,这种军队,能团结一致而不会离散,能连续作战而不会疲惫,无论用它指向哪里,谁也不能阻挡。这叫父子兵。”

吴子曰:“凡行军之道,无犯进止之节,无失饮食之适,无绝人马之力。此三者,所以任其上令。任其上令,则治之所由生也。若进止不度,饮食不适,马疲人倦而不解舍,所以不任其上令。上令既废,以居则乱,以战则败。”

吴子说:“一般用兵作战的原则,不要打乱前进和停止的节奏不要耽误适时供给饮食;不要耗尽人马的体力。这三项是为了使军队保持充分的体力,能胜任上级付予的使命。使军队能胜任其上级付予的使命,就是治军的基础。如果前进和停止没有节奏;饮食不能适时供给,人马疲惫不得休息,军队就不能胜任其上级所付予的使命,上级的命令就不能实施,驻守必然混乱,作战必定失败。”

吴子曰:“凡兵战之场,立尸之地,必死则生,幸生则死。其善将者,如坐漏船之中,伏烧屋之下,使智者不及谋,勇者不及怒,受敌可也。故曰,用兵之害,犹豫最大;三军之灾,生于狐疑。”

吴子说:“凡两军交战的场所,都是流血牺牲的地方。抱必死决心就会闯出生路,侥幸偷生就会遭到灭亡。所以,善于指挥作战的将领,要使部队就象坐在漏船上,伏在烧屋之下那样;急迫地采取行动。[因为在这种紧急情况下,]即使机智的人,也来不及去周密谋划,勇敢的人也来不及去振奋军威,只能当机立断,奋力拼搏,[才可保全自己,打败敌人。]因此说,用兵的害处,犹豫最大,全军失利,多半产生于迟疑。”

吴子曰:“夫人当死其所不能,败其所不便。故用兵之法,教戒为先。一人学战,教成十人。十人学战,教成百人。百人学战,教成千人。千人学战,教成万人。万人学战,教成三军。以近待远,以佚待劳,以饱待饥。圆而方之,坐而起之,行而止之,左而右之,前而后之,分而合之,结而解之,每变皆习,乃用授其兵。是谓将事。”

吴子说:“士卒在战斗中往往死于没有技能,败于不熟悉战法。所以用兵的方法。首先是训练。一人学会战斗本领了,可以教会十人。十人学会,可以教会百人。百人学会,可以教会千人。千人学会,可以教会万人。万人学会,可以教会全军。[在战法上,]以近待远,以逸待劳,以饱待饥。[在阵法上,]圆阵变方阵,坐降变立阵,前进变停止,向左变向右,向前变向后,分散变集结,集始变分散。各种变化都熟悉了,才授以兵器。这些都是将领应该他的事情。”

吴子曰:“教战之令,短者持矛戟,长者持弓弩,强者持旌旗,勇者持金鼓,弱者给厮养,智者为谋主。乡里相比,什伍相保,一鼓整兵,二鼓习陈,三鼓趋食,四鼓严办,五鼓就行。闻鼓声合,然后举旗。”

吴子说:“教战的法则,身体矮的拿矛栽,身体高的用弓努,强壮的杜大旗,勇敢的操金鼓,体弱的担任饲养,聪明的出谋划策同乡同里的编在一起,同什同伍的互相联保。[军队行动的信号:]打一通鼓,整理兵器。打两通鼓,练习列阵。打三通鼓,迅速就餐。打四通鼓,整装待发。打五通鼓,站队整列。鼓声齐鸣,然后举旗[指挥军队行动]。”

武侯问曰:“三军进止,岂有道乎?”

起对曰:“无当天灶,无当龙差别。天灶者,大谷之口;龙头者,大山之端。必左青龙,右白虎,前朱雀,后玄武,招摇在上,从事于下。将战之时,审候风所从
来。风顺致呼而从之,风逆坚以待之。”

武侯问道:“军队前进、停止,有一定的原则吗?”
吴起答:“不要在‘天灶’扎营,不要在‘龙头’上驻兵。所谓天灶,就是大山谷的口子。所谓龙头,就是大山的顶端。军队指挥,必须左军用青龙旗,右军用白虎旗,前军用朱雀旗,后军用玄武旗,中军用招摇旗在高处指挥,军队在其指挥下行动。临战时,还要观察风向,顺风时就呼噪乘势进击,逆风时就坚阵固守,等待变化。”

琥侯问曰:“凡畜卒骑,岂有方乎?”

起对曰:“夫马,必安其处所,适其水草,节其饥饱。冬则温烧,夏则凉庑。刻剔毛魆;谨落四下。戢其耳目,无令惊骇。习其驰逐,闲其进止。人马相亲,然后可使。车骑之具,鞍、勒、衔、辔,必令完坚。凡马不伤于末,必伤于始;不伤于饥,必伤于饱。日暮道远,必数上下;宁劳于人,慎勿劳马;常令有余,备敌覆我。能明此者,横行天下。”
\end{yuanwen}

武侯问:“驯养军马,有什么方法呢?”
吴起答:“军马,饲养处所要安适,水草要喂得适当,饥饱要有节制。冬天要保持马厩的温暖,夏天要注意马棚的凉爽。经常剪刷鬃毛。细心铲蹄钉掌,让它熟悉各种声音和颜色,使其不致惊骇。练习奔驰追逐,熟悉前进、停止的动作,做到人马相亲,然后才能使用。挽马和乘马的装具,如马鞍、笼头、嚼子、缰‘绳等物,必使其完整坚固。凡马匹不是伤于使用完了时,就是伤于使用开始时。不伤于过饥,就伤于过饱。当天色已晚路程遥远时,就须使乘马与步行交替进行。宁可人疲劳些,不要使马太劳累。要经常保持马有余力,以防敌之袭击。能够懂得这些道理的,就能天下无敌。”

\part{卷下}

\chapter{论将}

论将,就是论述对将领的要求和判断。全篇共五节,前三节论述对己方将领的要求,后两节论述对敌方将领的判断。

\begin{yuanwen}
吴子曰:“夫总文武者,军之将也。兼刚柔者,兵之事也。凡人论将,常观于勇。勇之于将,乃数分之一耳。夫勇者必轻合,轻合而不知利;未可也。故将之所慎者五:一曰理,二曰备,三曰果,四曰戒,五曰约。理者,治众如治寡。备者,出门如见敌。果者,临敌不怀生。戒者,虽克如始战。约者,法令省而不烦。受命而不辞敌,破而后言返,将之礼也。故师出之日,有死之荣,无生之辱。”

吴子说:“文武兼备的人,才可以胜任将领。能刚柔并用,才可以统军作战。一般人对于将领的评价,往往是只看他的勇敢,其实勇敢对于将领来说,只是应该具备的若干条件之一。单凭勇敢,必定会轻率应战,轻率应战而不考虑利害是不可取的。所以,将领应当注重的有五件事:一是理,二是备,三是果,四是戒,五是约。理,是说治理众多的军队如象治理少数军队一样地有条理。备,是说部队出动就象面对敌人一样地有戒备。果,是说临阵对敌不考虑个人的死生。戒,是说虽然打了胜仗还是如同初战时那样慎重。约,是说法令简明而不烦琐。受领任务决不推诿,打败了敌人才考虑田师,这是将领应遵守的规则。所以自出征那一天起,将领使应下定决心,宁可光荣战死,绝不忍辱偷生。”

吴子曰:“凡兵有四机:一曰气机,二曰地机,三曰事机,四曰力机。三军之众,百万之师,张设轻众,在于一人,是谓气机。路狭道险,名山大寨,士夫所守,千夫不过,是谓地机。善行间谍,轻兵往来,分散其众,使其君臣相怨,上下相咎,是谓事机。车坚管辖,舟利橹辑,士习战陈,马闲驰逐,是谓力机。知此四者,乃可为将。然其威、德、仁、勇,必足以率下安众,怖敌决疑,施令而下不敢犯,所在寇不敢敌。得之国强,去之国亡,是谓良将。”

吴子说:“用兵有四个关键:一是掌握士气,二是利用地形,三是运用计谋,四是充实力量。三军之众,百万之师,掌握士气的盛衰,在于将领一人,这是掌握士气的关键。利用狭路险道,名山要塞十人防守,千人也不能通过,这是利用地形的关键。善于使用间谍离间敌人,派遣轻装部队,反复骚扰敌人,以分散其兵力,使其君臣互相埋怨,上下互相责难,这是运用计谋的关键。战车的轮轴插销要做得坚固,船只的橹、桨要做得适用,士卒要熟习战阵,马匹要熟练驰骋,这就是充实力量的关键。懂得这四个关键,才可以为将。而且他的威信、品德、仁爱、勇敢,都必须足以表率全军,安抚士众,威慑敌军,决断疑难。发布的命令,部属不敢违犯,所到的地方,敌人不敢抵抗。得到[这样的将领]国家就强盛,失去他,国家就危亡。这就叫做良将。

吴子曰:“夫鼙鼓金铎,所以威耳;旌旗麾帜,所以威目;禁令刑罚,所以威心。耳威于声,不可不清;目威于色,不可不明;心威于刑,不可不严。三者不立,虽有其国,必败于敌。故曰:将之所麾,莫不从移;将之所指,莫不前死。”

吴子说:鼙鼓金铎,是用来指挥军队的听觉号令。旌旗麾帜,是用来指挥军队的视觉号令。禁令刑罚,是用未约束全军的法纪。斗朵听命于声音,所以声音不可不清楚。眼睛听命于颜色,所以颜色不可不鲜明。军心受拘束于刑罚,所以,刑罚不可不严格。三者如果不确立,虽有国家必败于敌。所以说,将领所发布的命令,部队没有不依令而行的。将领所指向的地方,部队没有不拼死向前的。”

吴子曰:“凡战之要,必先战其将而察其才,因形用权,则不劳而功举。其将愚而信人,可诈而诱;贪而忽名,可货而赂;轻变无谋,可劳而困,上富而骄,下贫而怨,可离而间,进退多疑,其众无依,可震而走;士轻其将而有归志,塞易开险,可邀而取;进道易,退道难,可来而前,进道险,退道易,可薄而击;居军
下湿,水无所通,霖雨数至,可灌而沉;居军荒泽,草楚幽秽,风飚数至,可焚而灭,停久不移,将士懈怠,其军不备,可潜而袭。”

吴子说:“一般说作战最重要的是,首先探知敌将是谁,并充分了解他的才能。根据敌人情况,采取权变的方法,不费多大力气,就可取得成功。敌将愚昧而轻信于人,可用欺骗的手段来引诱他。敌将贪利而不顾名誉,可用财物收买他。轻率变更计划而无深谍远虑的,可以疲困他。上级富裕而骄横,下级贫穷而怨愤的,可以离间它。选退犹豫不决,部队无所适从的,可震憾吓跑它。士卒藐视其将领而急欲田家的,就堵塞平坦道路,佯开险阻道路,用拦击消灭它。敌人进路平易,退路艰难,可引诱它前来予以消灭。敌人进路艰难,退路平易,可以迫近攻击它。敌人处于低洼潮湿的地方,水道不通,大雨连绵,可以灌水淹没它。敌军处于荒芜的沼泽地,草木丛生,常有狂风,可用火攻消灭它。敌军久住一地而不移动,官兵懈怠,戒备疏忽,可以偷袭它。

武侯问曰:“两军相望,不知其将,我欲相对之,其术如何?”

起对曰:“令贱而勇者,将轻锐以尝之,务于北,无务于得。观敌之来,一坐一起,其政以理。其追北佯为不及,见其利佯为不知。如此将者,名为智将,勿与战也。若其众权哗,旌旗烦乱,其卒自行自止,其兵或纵或横,其追北恐不及,见利恐不得,此为愚将,虽众可获。”
\end{yuanwen}

武侯问:“两军对阵,不知敌将的才能,想要查明,用什么方法?”
吴起答:“令勇敢的下级军官,率领轻锐部队去试攻敌人。务必败退,不要求胜,以观察敌人前来的行动。如果敌人每次前进和停止,指挥都有条不紊,追击假装追不上,见到战利品装做没看见,象这样的将领是有智谋的,不要和他交战。如果敌人喧哗吵闹,旗帜纷乱,士卒自由行动,兵器横七竖八,追击惟恐追不上,见利惟恐得不到,这是愚昧的将领,敌军虽多也可以把他擒获。”

\chapter{应变}

应变,即应付各种情况的变化。全篇共十节,都是讲的应付各种情况的战法。 

\begin{yuanwen}
武侯问曰:“车坚马良,将勇兵强,卒遇敌人,乱而失行,则如之何?”

起对曰:“凡战之法,昼以旌旗幡麾为节,夜以金鼓笳笛为节。麾左而左,麾右而右,鼓之则进,金之则止,一吹而行,再吹而聚,不从令者诛。三军服威,士卒用命,则战无强敌,攻无坚陈矣。”
\end{yuanwen}

武侯问:“战车坚固,马匹驯良,将领勇敢,士卒强壮,突然遭遇敌人,乱得不成行列,该怎么办?”
吴起答:“一般作战的方法,白天用旌旗幡麾来指挥,夜间用金鼓笳笛来指挥。指挥向左就向左,指挥向右就向右。擂鼓就前进,鸣金就停止。第一次吹笳笛就出动,第二次吹笳苗就会合,不听号令的就杀。三军畏服威严,士卒听从命令,这样,就没有打不败的强敌,没有攻不破的坚阵。”

\begin{yuanwen}
武侯问曰:“若敌众我寡,为之奈何?”

起对曰:“避之于易,邀之于阨。故曰以一击十,莫善于阨;以士击百,莫善于险,以千击万,莫善于阻。今有少年卒起,击金鸣鼓于阨路,虽有大众,莫不惊动。故曰:‘用众者务易,用少者务隘。”
\end{yuanwen}

武侯问:“如果敌众我寡,怎么办呢?”
吴起答:“在平坦地形上避免和它作战,而要在险要地形上截击它,所以说,以一击十,最好是利用狭窄隘路;以十击百,最好是利用险要地形;以千击万,最好是利用阻绝地带。如果用少数兵力,突然出击,在狭隘道路上击鼓鸣金,敌人虽多,也莫不惊慌骚动。所以说,使用众多兵力,务必选择平坦地形;使用少数兵力,务必选择险要地形。”

\begin{yuanwen}
武侯问曰:“有师甚众,既武且勇,背大险阻,右山左水;深沟高垒,守以强弩;退如山移,进如风雨;粮食又多,难与长守。”

对曰:“大哉问乎!非此车骑之力,圣人之谋也。能备千乘万骑,兼之徒步,分为五军,各军一衢。夫五军五衢,敌人必惑,莫之所加。敌人若坚守,以固其兵,急行间谍,以观其虑。彼听吾说,解之而去;不听吾说,斩使焚书。分为五战,战胜勿追,不胜疾归。如是佯北,安行疾斗,一结其前,一绝其后,两军衔枚,或左或右,而袭其处,五军交至,必有其利。此击强之道也。”
\end{yuanwen}

武侯问:“敌人很多,既有良好训练,又很勇敢,背靠高山,前临险要,右依山,左靠水;深沟高全,强弩守备,后退稳如山移,前进急如风雨,粮食又很充足,很难与它长.久相持,应该怎么办呢?”吴起答:“您提的问题很大啊!这不能单靠车骑的力量,而要靠高明的计谋才能取胜的。如能准备战车十辆,骑兵万人,加上步兵,区分为五支军队,每支军队担任一个方向,五支军队分为五个方向,敌人必然发生迷惑,不知我将要打它哪里。如果敌人坚强防守,以巩固它的军队,我就立刻派出军使去摸清它的意图。假如故人听我劝说而撤兵,我也撤兵离开。如不听劝告,反而杀我使节,烧我的书信,就五路进攻。战胜不要追击,不胜就迅速撤回。如果要假装败退,引诱敌人,就应以一军稳妥地行动,急剧地战斗,其它四军,一军牵制敌人前方,一军断绝敌人后路,另以两军秘密行动,从左右两侧,袭击敌人据守的地方。五军合击,公然形成有利态势,这就是打击强敌的方法。”

\begin{yuanwen}
武侯问曰:“敌近而薄我,欲去无路;我众甚惧,为之奈何?”

对曰:“为此之术,若我众彼寡,分而乘之;彼众我寡,以方从之;从之无息,虽众可服。”
\end{yuanwen}

武侯问道:“敌人接近,迫我交战,我想摆脱它而没有去路,军心很恐惧,应该怎么办呢?”
吴起答:“解决这一问题的方法,如果我众敌寡,可以分兵包围敌人,如果敌众我寡,可以集令兵力袭击敌人,不断地袭击它,敌人虽多也可制服。”

\begin{yuanwen}
武侯问曰:“若遇敌于溪谷之间,谤多险阻,彼众我寡,为之奈何?”

起对曰:“遇诸丘陵、林谷、深山、大泽,疾行亟去,勿得从容。若高山深谷,卒然相遇,必选鼓噪而乘之。进弓与弩,且射且虏。审察其政,乱则击之勿疑。”
\end{yuanwen}

武侯问:“如在溪谷之间和敌人遭遇,两旁都是险峻地形,而且敌众我寡,应该怎么办呢?”
吴起答:“遇到丘陵、森林、谷地、深山、大泽等不利地形,都应迅速通过,不要迟缓。如果在高山深谷突然与敌遭遇,必先击鼓呐喊,乘势冲乱敌人,再把弓弩手挺进到前面,一面戒备,一面考虑计谋,并观察敌人的阵势是否混乱,如发现敌军混乱,就毫不迟疑地全力发起进攻。”

\begin{yuanwen}
武侯问曰:“左右高山,地甚狭迫,卒遇敌人,击之不敢,去之不得,为之奈何?”

起对曰:“此谓谷战,虽众不用。募吾材士,与敌相当,轻足利兵,以为前行,分车列骑,隐于四旁,相去数里,无见其兵,敌必坚陈,进退不敢。于是出旌列旆,行出山外营之。敌人必惧,车骑挑之,勿令得休。此谷战之法也。”
\end{yuanwen}

武侯问:“左右是高山,地形很狭窄,突然与敌遭遇,既不敢进攻,又不能退走,应该怎么办呢?”
吴起答:“这叫谷地战,兵力虽多也用不上,应挑选精锐士卒与敌对抗,用轻捷善走的士卒持锐利的武器作为前锋,而把车骑分散隐蔽在四周,与前锋距离几里,不要暴露自己的兵力,这样敌人必然坚守阵地,不敢前进,也不敢后退。这时,[我以一部兵力]张列旗帜,走出山外,迷惑扰乱敌人,敌人必然恐惧,然后再用车骑向敌挑战,使其不得休息。这就是谷地战的方法。”

\begin{yuanwen}
武侯问曰:“吾与敌相遇大水之泽,倾轮没辕,水薄车骑,舟楫不设,进退不得,为之奈何?”

起对曰:“此谓水战,无用车骑,且留其傍。登高四望,必得水情。知其广狭,尽其浅深,乃可为奇以胜之。敌若绝水,半渡而薄之。”
\end{yuanwen}

武侯问道:“我与敌相遇于大水汇聚的地方,水势倾陷了车轮,淹没了车辕,车骑都有被洪水吞没的危险,又没有准备船只,前进后退都困难,应该怎么办呢?”
吴起答:“这叫水战,车骑无法使用,暂且把它留在岸边。登高观察四方,一定要弄清水情,了解水面的宽窄,查明水的深浅,才可以出奇制胜。敌人如果渡水而来,就乘其半渡打击它。”

\begin{yuanwen}
武侯问曰:“天久连雨,马陷车止,四面受敌,三军惊骇,为之奈何?”

起对曰:“凡用车者,阴湿则停,阳燥则起,贵高贱下,驰其强车,若进若止,必从其道。敌人若起,必逐其迹。”
\end{yuanwen}

武侯问道:“阴雨连绵,车马难行,四面受敌,全军惶恐,应该怎么办?”
吴起答:“凡是用战车作战的,阴雨泥泞就停止,天晴地干就行动,要选择高处避开低处行动。要使战车迅速行驶,不论前进或停止,都必须利用道路。如果有敌人战车行动,就可以沿着它的车迹行动。”

\begin{yuanwen}
武侯问曰:“暴寇卒来,掠吾田野,取吾牛羊,则如之何?”

起对曰:“暴寇之来,必虑其强,善守勿应。彼将暮去,其装必重,其心必恐,还退务速,必有不属,追而击之,其兵可覆。
\end{yuanwen}

武侯问道:“强暴的敌人,突然到来,掠夺我的庄稼,抢劫我的牛羊,该怎么办呢?”
吴起答:“强暴的敌人前来,必须考虑它的强大,应严加防守,不要应战,待敌人傍晚撤走时,它的装载必然沉重,心理必然恐惧,退走力求迅速,必有互不联系的地方。这时进行追击,就可歼灭它。”

\begin{yuanwen}
吴子曰:“凡攻敌围城之道,城邑既破,各入其宫。御其禄秩,收其器物。军之所至,无刊其木、发其屋、取其栗、杀其六畜、燔其积聚,示民无残心。其有请降,许而安之。”
\end{yuanwen}

吴起说:“一般围攻敌城的原则,是城邑既被攻破,就分别进驻它的官府,控制和使用其原来的官吏,没收它的器材物资。军队所到之处,不准砍伐树木、毁坏房屋、擅取粮食、宰杀牲畜、焚烧仓库,以表明对民众无残暴之心。如有请降的,应允许并安抚他们。”

\chapter{励士}

励士,就是鼓励将士立功。全篇是讲述论功行赏,崇礼有功,以勉励全体将士,从而使全军争相建功。

\begin{yuanwen}
武侯问曰:“严刑明赏,足以胜乎?”

起对曰:“严明之事,臣不能悉,虽然,非所恃也。夫发号布令而人乐闻,兴师动众而人乐战,交兵接刃而人乐死,此三者,人主之所恃也。”
\end{yuanwen}

武侯问道:“赏罚严明就足以打胜仗了吗?”

吴起答:“赏罚严明这件事,我不能详尽地说明,虽然这很重要,但不能完全依靠它。发号施令,人们乐于听从,出兵打仗,人们乐于参战,冲锋陷阵,人们乐于效死。这三点,才是君主所应该依靠的。”

\begin{yuanwen}
武侯曰:“致之奈何?”

对曰:“君举有功而进飨之,无功而励之。”
\end{yuanwen}

武侯说:“怎样才能做到呢?”

吴起答:“您选拔有功人员,举行盛大宴会款待他们,这对无功的人也是一种勉励。”

\begin{yuanwen}
于是武侯设坐庙廷,为三行飨士大夫。上功坐前行,肴席,兼重器上牢;次功坐中行,肴席,器差减;无功坐后行,肴席无重器。飨毕而出,又颁赐有功者父母妻子于庙门外,亦以功为差。有死事之家,岁被使者劳赐其父母,著不忘于心。行之三年,秦人兴师,临于西河,魏士闻之,不待吏令,介胄而奋击之者以万数。
\end{yuanwen}

于是武侯设席于祖庙,分三排坐位宴请士大夫。立上等功的坐前排,用上等酒席和珍贵餐具,猪、牛、羊三杜俱全。二等功的坐中排,酒席、餐具较为差些。没有功的坐后排,只有酒席,没有贵重餐·具。宴后出来,又在庙门外赏赐有功人员的父母妻子,也按功劳大小而分差列。对于死难将士的家属,每年派人慰问、赏赐他们的父母,表示心里没有志记他们。这个办法实行了三年之后,泰国出兵到达魏国的西河边境,魏国的士卒听到这一消息,不待官吏的命令,就自动穿戴盔甲奋勇抗敌的数以万计。

\begin{yuanwen}
武侯召吴起而谓曰:“子前日之教行矣。”

起对曰:“臣闻人有短长,气有盛衰。君试发无功者五万人,臣请率以当之脱其不胜,取笑于诸侯,失权于天下矣。今使一死贼伏于旷野,千人追之,莫不枭视狼顾。何者?忌其暴起害己也。是以一人投命,足惧千夫。今臣以五万之众为一死贼,率以讨之,固难敌矣。”
\end{yuanwen}

于是武侯召见吴起说:“您以前教我的办法,现在见到成效了。”

吴起说:“我听说人有短处有长处,士气也有盛有衰。您不妨试派五万名没有立过功的人,让我率领去抵挡泰军,如果不胜,就会被诸侯讥笑,丧失权威于天下了。[但这是不会发生的。所以我敢去尝试。]譬如现在有一个犯了死罪的盗贼,隐伏在荒郊旷野,派一千人去追捕他,没有一个不瞻前顾后的。这是为什么呢?是怕他突然跳出来伤害了自己。所以一个人拼命,足使千人畏惧。现在我这五万人都象那个盗贼一样,率领他们去征讨敌人,敌人就很难抵挡了。”

\begin{yuanwen}
于是武侯从之,兼车五百乘,骑三千匹,而破秦五十万众,此励土之功也。

先战一日,吴起令三军曰;“诸吏士当从,受驰车、骑与徒,若车不得车,骑不得骑,徒不得徒,虽破军,皆无功。”故战之日,其令不烦而威镇天下。
\end{yuanwen}

于是武侯采纳了吴起的意见,并加派战车五百辆,战马三十匹,大破泰军五十万人。这就是激励士气的效果。

在作战的前一天,吴起命令三军说:“众吏士应当听从命令去和敌人战斗,无论车兵、骑兵和步兵,如果车兵不能缴获敌人的战车,骑兵不能俘获敌人的骑兵,步兵不能俘获敌人的步兵,即使打败敌人,都不算有功。”所以作战的那天,他的号令不多,却战果辉煌,威震天下。

\end{document}