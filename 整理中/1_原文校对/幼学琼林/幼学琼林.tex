% 幼学琼林
% 幼学琼林.tex

\documentclass[12pt,UTF8]{ctexbook}

% 设置纸张信息。
\usepackage[a4paper,twoside]{geometry}
\geometry{
	left=25mm,
	right=25mm,
	bottom=25.4mm,
	bindingoffset=10mm
}

% 设置字体,并解决显示难检字问题。
\xeCJKsetup{AutoFallBack=true}
\setCJKmainfont{SimSun}[BoldFont=SimHei, ItalicFont=KaiTi, FallBack=SimSun-ExtB]

% 目录 chapter 级别加点(.)。
\usepackage{titletoc}
\titlecontents{chapter}[0pt]{\vspace{3mm}\bf\addvspace{2pt}\filright}{\contentspush{\thecontentslabel\hspace{0.8em}}}{}{\titlerule*[8pt]{.}\contentspage}

% 设置 part 和 chapter 标题格式。
\ctexset{
	part/name={卷,},
	part/number={\chinese{part}},
	chapter/name={},
	chapter/number={}
}

% 设置古文原文格式。
\newenvironment{yuanwen}{\bfseries\zihao{4}}

% 设置署名格式。
\newenvironment{shuming}{\hfill\bfseries\zihao{4}}

% 注脚每页重新编号,避免编号过大。
\usepackage[perpage]{footmisc}

\title{\heiti\zihao{0} 幼学琼林}
\author{程登吉}
\date{}

\begin{document}

\maketitle
\tableofcontents

\frontmatter
\chapter{前言}



\mainmatter

\part{}

\chapter{天文}

混沌初开,乾坤始奠。气之轻清上浮者为天,气之重浊下凝者为地。
日月五星,谓之七政;天地与人,谓之三才。日为众阳之宗,月乃太阴之象。
虹名螮蝀,乃天地之淫气;月里蟾蜍,是月魄之精光。
风欲起而石燕飞,天将雨而商羊舞。
旋风名为羊角,闪电号曰雷鞭。
青女乃霜之神,素娥即月之号。
雷部至捷之鬼曰律令,雷部推车之女曰阿香。
云师系是丰隆,雪神乃是滕六。
歘火、谢仙,俱掌雷火;飞廉、箕伯,悉是风神。
列缺乃电之神,望舒是月之御。
甘霖、甘澍,仅指时雨;玄穹、彼苍,悉称上天。
雪花飞六出,先兆丰年;日上已三竿,乃云时晏。
蜀犬吠日,比人所见甚稀;吴牛喘月,笑人畏惧过甚。
望切者,若云霓之望;恩深者,如雨露之恩。
参商二星,其出没不相见;牛女两宿,惟七夕一相逢。
后羿妻,奔月宫而为嫦娥;傅说死,其精神托于箕尾。
披星戴月,谓早夜之奔驰;沐雨栉风,谓风尘之劳苦。

事非有意,譬如云出无心;恩可遍施,乃曰阳春有脚。
馈物致敬,曰敢效献曝之忱;托人转移,曰全赖回天之力。
感救死之恩,曰再造;诵再生之德,曰二天。
势易尽者若冰山,事相悬者如天壤。
晨星谓贤人廖落,雷同谓言语相符。
心多过虑,何异杞人忧天;事不量力,不殊夸父追日。
如夏日之可畏,是谓赵盾;如冬日之可爱,是谓赵衰。
齐妇含冤,三年不雨;邹衍下狱,六月飞霜。
父仇不共戴天,子道须当爱日。
盛世黎民,嬉游于光天化日之下;太平天子,上召夫景星庆云之祥。
夏时大禹在位,上天雨金;春秋孝经既成,赤虹化玉。
箕好风,毕好雨,比庶人愿欲不同;风从虎,云从龙,比君臣会合不偶。
雨旸时若,系是休徵;天地交泰,称斯盛世。

\chapter{地舆}

黄帝画野,始分都邑;夏禹治水,初奠山川。
宇宙之江山不改,古今之称谓各殊。
北京原属幽燕,金台是其异号;南京原为建业,金陵又是别名。
浙江是武林之区,原为越国;江西是豫章之地,又曰吴皋。
福建省属闽中,湖广地名三楚。
东鲁西鲁,即山东山西之分;东粤西粤,乃广东广西之域。
河南在华夏之中,故曰中州;陕西即长安之地,原为秦境。
四川为西蜀,云南为古滇。
贵州省近蛮方,自古名为黔地。
东岳泰山,西岳华山,南岳衡山,北岳恒山,中岳嵩山,此为天下之五岳。
饶州之鄱阳,岳州之青草,润州之丹阳,鄂州之洞庭,苏州之太湖,此为天下之五湖。
金城汤池,谓城池之巩固;砺山带河,乃封建之誓盟。
帝都曰京师,故乡曰梓里。
蓬莱弱水,惟飞仙可渡;方壶员峤,乃仙子所居。
沧海桑田,谓世事之多变;河清海晏,兆天下之升平。
水神曰冯夷,又曰阳侯,火神曰祝融,又曰回禄。
海神曰海若,海眼曰尾闾。

望人包容曰海涵,谢人恩泽曰河润。
无系累者曰江湖散人,负豪气者曰湖海之士。
问舍求田,原无大志;掀天揭地,方是奇才。
凭空起事,谓之平地风波;独立不移,谓之中流砥柱。
黑子、弹丸,漫言至小之邑;咽喉、右臂,皆言要害之区。
独立难持,曰一木焉能支大厦;英雄自恃,曰丸泥亦可封函关。
事先败而后成,曰失之东隅,收之桑榆;事将成而终止,曰为山九仞,功亏一篑。
以蠡测海,喻人之见小;精卫衔石,比人之徒劳。
跋涉谓行路艰难,康庄谓道路平坦。
硗地曰不毛之地,美田曰膏腴之田。
得物无所用,曰如获石田;为学已大成,曰诞登道岸。
淄渑之滋味可辨,泾渭之清浊当分。
泌水乐饥,隐居不仕;东山高卧,谢职求安。
圣人出则黄河清,太守廉则越石见。
美俗曰仁里,恶俗曰互乡。
里名胜母,曾子不入;邑号朝歌,墨翟回车。
击壤而歌,尧帝黎民之自得;让畔而耕,文王百姓之相推。
费长房有缩地之方,秦始皇有鞭石之法。

尧有九年之水患,汤有七年之旱灾。
商鞅不仁而阡陌开,夏桀无道而伊洛竭。
道不拾遗,由在上有善政;海不扬波,知中国有圣人。

\chapter{岁时}

爆竹一声除旧,桃符万户更新。
履端是初一元日,人日是初七灵辰。
元日献君以《椒花颂》,为祝遐龄;元日饮人以屠苏酒,可除疠疫。
新岁曰王春,去年曰客岁。
火树银花合,谓元宵灯火之辉煌;星桥铁锁开,调元夕金吾之不禁。
二月朔为中和节,三月三为上巳辰。
冬至百六是清明,立春五戊为春社。
寒食节是清明前一日,初伏日是夏至第三庚。
四月乃是麦秋,端午却为蒲节。
六月六日,节名天贶;五月五日,序号天中。
端阳竞渡,吊屈原之溺水;重九登高,效桓景之避灾。
五戊鸡豚宴社,处处饮治聋之酒;七夕牛女渡河,家家穿乞巧之针。
中秋月朗,明皇亲游于月殿;九日风高,孟嘉帽落于龙山。
秦人岁终祭神曰腊,故至今以十二月为腊;始皇当年御讳曰政,故至今读正月为征。
东方之神曰太皞,乘震而司春,甲乙属木,木则旺于春,其色青,故春帝曰青帝。
南方之神曰祝融,居离而司夏,丙丁属火,火则旺于夏,其色赤,故夏帝曰赤帝。
西方之神曰蓐收,当兑而司秋,庚辛属金,金则旺于秋,其色白,故秋帝曰白帝。

北方之神曰玄冥,乘坎而司冬,壬癸属水,水则旺于冬,其色黑,故冬帝曰黑帝。
中央戊己属土,其色黄,故中央帝曰黄帝。
夏至一阴生,是以天时渐短;冬至一阳生,是以日晷初长。
冬至到而葭灰飞,立秋至而梧叶落。
上弦谓月圆其半,系初八、九;下弦谓月缺其半,系廿二、三。
月光都尽谓之晦,三十日之名;月光复苏谓之朔,初一日之号;月与日对谓之望,十五日之称。
初一是死魄,初二旁死魄,初三哉生明,十六始生魄。
翌日、诘朝,言皆明日;榖旦、吉旦,悉是良辰。
片晌即谓片时,日曛乃云日暮。
畴昔、曩者,俱前日之谓;黎明、昧爽,皆将曙之时。
月有三浣:初旬十日为上浣,中旬十日为中浣,下旬十日为下浣;学足三馀:夜者日之馀,冬者岁之馀,雨者睛之馀。
以术愚人,曰朝三暮四;为学求益,曰日就月将。
焚膏继晷,日夜辛勤;俾昼作夜,晨昏颠倒。
自愧无成,曰虚延岁月;与人共语,曰少叙寒暄。
可憎者,人情冷暖;可厌者,世态炎凉。
周末无寒年,因东周之懦弱;秦亡无燠岁,由嬴氏之凶残。
泰阶星平曰泰平,时序调和曰玉烛。

岁歉曰饥馑之岁,年丰曰大有之年。
唐德宗之饥年,醉人为瑞;梁惠王之凶岁,野莩堪怜。
丰年玉,荒年谷,言人品之可珍;薪如桂,食如玉,言薪米之腾贵。
春祈秋报,农夫之常规;夜寐夙兴,吾人之勤事。
韶华不再,吾辈须当惜阴;日月其除,志士正宜待旦。

\chapter{朝廷}

三皇为皇,五帝为帝。
以德行仁者王,以力假仁者霸。
天子天下之主,诸侯一国之君。
官天下,乃以位让贤;家天下,是以位传子。
陛下尊称天子,殿下尊重宗藩。
皇帝即位曰龙飞,人臣觐君曰虎拜。
皇帝之言,谓之纶音;皇后之命,乃称懿旨。
椒房是皇后所居,枫宸乃人君所莅。
天子尊崇,故称元首;臣邻辅翼,故曰股肱。
龙之种,麟之角,俱誉宗藩;君之储,国之贰,首称太子。
帝子爰立青宫,帝印乃是玉玺。
宗室之派,演于天潢;帝胄之谱,名为玉牒。
前星耀彩,共祝太子以千秋;嵩岳效灵,三呼天子以万岁。
神器大宝,皆言帝位;妃嫔媵嫱,总是宫娥。
姜后脱簪而待罪,世称哲后;马后练服以鸣俭,共仰贤妃。
唐放勋德配昊天,遂动华封之三祝;汉太子恩覃少海,乃兴乐府之四歌。

帝王有出震向离之象,大臣有补天浴日之功。
三公上应三台,郎官上应列宿。
宰相位居台铉,吏部职掌铨衡。
吏部天官大冢宰,户部地官大司徒。
礼部春官大宗伯,兵部夏官大司马。
刑部秋官大司寇,工部冬官大司空。
都宪中丞,都御史之号;内翰学士,翰林院之称。
天使誉称行人,司成尊称祭酒。
称都堂曰大抚台,称巡按曰大柱史。
方伯、藩侯,左右布政之号;宪台、廉宪,提刑按察之称。
宗师称为大文衡,副使称为大宪副。
郡侯、邦伯,知府名尊;郡丞、贰侯,同知誉美。
郡宰、别驾,乃称通判;司理、豸史,赞美推官。
刺史、州牧,乃知州之两号;豸史、台谏,即知县之以称。
乡宦曰乡绅,农官曰田畯。
钧座、台座,皆称仕宦;帐下、麾下,并美武官。
秩官既分九品,命妇亦有七阶。

一品曰夫人,二品亦夫人,三品曰淑人,四品曰恭人,五品曰宜人,六品曰安人,七品曰孺人。
妇人受封曰金花诰,状元报捷曰紫泥封。
唐玄宗以金瓯覆宰相之名,宋真宗以美珠箝谏臣之口。
金马玉堂,羡翰林之声价;朱幡皂盖,仰郡守之威仪。
台辅曰紫阁名公,知府曰黄堂太守。
府尹之禄二千石,太守之马五花骢。
代天巡狩,赞称巡按;指日高升,预贺官僚。
初到任曰下车,告致仕曰解组。
藩垣屏翰,方伯犹古诸侯之国;墨绶铜章,令尹即古子男之帮。
太监掌阉门之禁令,故曰阉宦;朝臣皆缙笏于绅间,故曰缙绅。
萧曹相汉高,曾为刀笔吏;汲黯相汉武,真是社稷臣。
召伯布文王之政,尝合甘棠之下,后人思其遗爱,不忍伐其树;孔明有王佐之才,尝隐草庐之中,先主慕其令名,乃三顾其庐。
鱼头参政,鲁宗道秉性骨鲠;伴食宰相,卢怀慎居位无能。
王德用,人称黑王相公;赵清献,世号铁面御史。
汉刘宽责民,蒲鞭示辱;项仲山洁己,饮马投钱。
李善感直言不讳,竟称鸣凤朝阳;汉张纲弹劾无私,直斥豺狼当道。
民爱邓侯之政,挽之不留;人言谢令之贪,推之不去。

廉范守蜀郡,民歌五裤;张堪守渔阳,麦穗两歧。
鲁恭为中牟令,桑下有驯雉之异;郭汲为并州守,儿童有竹马之迎。
鲜于子骏,宁非一路福星;司马温公,真是万家生佛。
鸾凤不栖枳棘,羡仇香之为主簿;河阳遍种桃花,乃潘岳之为县官。
刘昆宰江陵,昔日反风灭火;龚遂守渤海,令民卖刀买牛。
此皆德政可歌,是以令名攸著。

\chapter{武职}

韩柳欧苏,固文人之最著;起翦颇牧,乃武将之多奇。
范仲淹胸中具数万甲兵,楚项羽江东有八千子弟。
孙膑吴起,将略堪夸;穰苴尉缭,兵机莫测。
姜太公有《六韬》,黄石公有《三略》。
韩信将兵,多多益善;毛遂讥众,碌碌无奇。
大将曰干城,武士曰武弁。
都督称为大镇国,总兵称为大总戎。
都阃即是都司,参戎即是参将。
千户有户侯之仰,百户有百宰之称。
以车为户曰辕门,显揭战功曰露布。
下杀上谓之弑,上伐下谓之征。
交锋为对垒,求和曰求成。
战胜而回,谓之凯旋;战败而走,谓之奔北。
为君泄恨曰敌忾;为国救难曰勤王。
胆破心寒,比敌人慑服之状;风声鹤唳,惊士卒败北之魂。
汉冯异当论功,独立大树下,不夸己绩;汉文帝尝劳军,亲幸细柳营,按辔徐行。
苻坚自夸将广,投鞭可以断流;毛遂自荐才奇,处囊便当脱颖。

羞与哙等伍,韩信降作淮阴;无面见江东,项羽羞归故里。
韩信受胯下之辱,张良有进履之谦。
卫青为牧猪之奴,樊哙为屠狗之辈。
求士莫求全,毋以二卵弃干城之将;用人如用木,毋以寸朽弃连抱之材。
总之君子之身,可大可小;丈夫之志,能屈能伸。
自古英雄,难以枚举;欲详将略,须读《武经》。

\chapter{祖孙父子}

何谓五伦?君臣、父子、兄弟、夫妇、朋友。
何谓九族?高、曾、祖、考、己身、子、孙、曾、玄。
始祖曰鼻祖,远孙曰耳孙。
父子创造,曰肯构肯堂;父子俱贤,曰是父是子。
祖称王父,父曰严君。
父母俱存,谓之椿萱并茂;子孙发达,谓之兰桂腾芳。
桥木高而仰,似父之道;梓木低而俯,如子之卑。
不痴不聋,不作阿家阿翁;得亲顺亲,方可为人为子。
盖父愆,名为干蛊;育义子,乃曰螟蛉。
生子当如孙仲谋,曹操羡孙权之语;生子须如李亚子,朱温叹存勖之词。
菽水承欢,贫士养亲之乐;义方是训,父亲教子之严。
绍箕裘,子承父业;恢先绪,子振家声。
具庆下,父母俱存;重庆下,祖父俱在。
燕翼贻谋,乃称裕后之祖;克绳祖武,是称象贤之孙。
称人有令子,曰鳞趾呈祥;称宦有贤郎,曰凤毛济美。
弑父自立,隋杨广之天性何存;杀子媚君,齐易牙之人心何在。
分甘以娱目,王羲之弄孙自乐;问安惟点颔,郭子仪厥孙最多。

和丸教子,仲郢母之贤;戏彩娱亲,老莱子之孝。
毛义捧檄,为亲之存;伯俞泣杖,因母之老。
慈母望子,倚门倚闾;游子思亲,陟岵陟屺。
爱无差等,曰兄子如邻子;分有相同,曰吾翁即若翁。
长男为主器,令子可克家。
子光前曰充闾,子过父曰跨灶。
宁馨英畏,皆是羡人之儿;国器掌珠,悉是称人之子。
可爱者子孙之多,若螽斯之蛰蛰;堪羡者后人之盛,如瓜瓞之绵绵。

\part{}

\chapter{兄弟}

天下无不是底父母,世间最难得者兄弟。
须贻同气之光,无伤手足之雅。
玉昆金友,羡兄弟之俱贤;伯埙仲篪,谓声气之相应。
兄弟既翕,谓之花萼相辉;兄弟联芳,谓之棠棣竞秀。
患难相顾,似鹡鸰之在原;手足分离,如雁行之折翼。
元方季方俱盛德,祖太丘称为难弟难兄;宋郊宋祁俱中元,当时人号为大宋小宋。
荀氏兄弟,得八龙之佳誉;河东伯仲,有三凤之美名。
东征破斧,周公大义灭亲;遇贼争死,赵孝以身代弟。
煮豆燃萁,谓其相害;斗粟尺布,讥其不容。
兄弟阋墙,谓兄弟之斗狠;天生羽翼,谓兄弟之相亲。
姜家大被以同眠,宋君灼艾而分痛。
田氏分财,忽瘁庭前之荆树;夷齐让国,共采首阳之蕨薇。
虽曰安宁之日,不如友生;其实凡今之人,莫如兄弟。

\chapter{夫妇}

孤阴则不生,独阳则不长,故天地配以阴阳;男以女为室,女以男为家,故人生偶以夫妇。
阴阳和而后雨泽降,夫妇和而后家道成。
夫谓妻曰拙荆,又曰内子;妻称夫曰藁砧,又曰良人。
贺人娶妻,曰荣偕伉俪;留物与妻,曰归遗细君。
受室即是娶妻,纳宠谓人娶妾。正妻谓之嫡,众妾谓之庶。
称人妻曰尊夫人,称人妾曰如夫人。
结发系是初婚,续弦乃是再娶。
妇人重婚曰再醮,男子无偶曰鳏居。
如鼓瑟琴,夫妻好合之谓;琴瑟不调,夫妇反目之词。
牝鸡司晨,比妇人之主事;河东狮吼,讥男子之畏妻。
杀妻求将,吴起何其忍心;蒸藜出妻,曾子善全孝道。
张敞为妻画眉,媚态可哂;董氏为夫封发,贞节堪夸。
冀郤缺夫妻,相敬如宾;陈仲子夫妇,灌园食力。
不弃槽糠,宋弘回光武之语;举案齐眉,梁鸿配孟光之贤。
苏蕙织回文,乐昌分破镜,是夫妇之生离;张瞻炊臼梦,庄子鼓盆歌,是夫妇之死别。
鲍宣之妻,提瓮出汲,雅得顺从之道;齐御之妻,窥御激夫,可称内助之贤。
可怪者买臣之妻,因贫求去,不思覆水难收;可丑者相如之妻,夤夜私奔,但识丝桐有意。
要知身修而后家齐,夫义自然妇顺。

\chapter{叔侄}

曰诸父,曰亚父,皆叔父之辈;曰犹子,曰比儿,俱侄儿之称。
阿大中郎,道韫雅称叔父;吾家龙文,杨素比美侄儿。
乌衣诸郎君,江东称王谢之子弟;吾家千里驹,符坚羡苻朗为侄儿。
竹林叔侄之称,兰玉子侄之誉。
存侄弃儿,悲伯道之无后;视叔犹父,羡公绰之居官。
卢迈无儿,以侄而主身之后;张范遇贼,以子而代侄之生。

\chapter{师生}

马融设绛帐,前授生徒,后列女乐;孔子居杏坛,贤人七十,弟子三千。
称教馆曰设帐,又回振铎;谦教馆曰糊口,又曰舌耕。
师曰西宾,师席曰函丈;学曰家塾,学俸曰束修。
桃李在公门,称人弟子之多;苜蓿长阑干,奉师饮食之薄。
冰生于水而寒于水,比学生过于先生;青出于蓝而胜于蓝,谓弟子优于师傅。
未得及门,曰宫墙外望;称得秘授,曰衣钵真传。
人称杨震为关西夫子,世称贺循为当世儒宗。
负笈千里,苏章从师之殷;立雪程门,游杨敬师之至。
弟子称师之善教,曰如坐春风之中;学业感师之造成,曰仰沾时雨之化。

\chapter{朋友宾主}

取善辅仁,皆资朋友;往来交际,迭为主宾。
尔我同心,曰金兰;朋友相资,曰丽泽。
东家曰东主,师傅曰西宾。
父所交游,尊为父执;己所共事,谓之同袍。
心志相孚为莫逆,老幼相交曰忘年。
刎颈交,相如与廉颇;总角好,孙策与周瑜。
胶漆相投,陈重之与雷义;鸡黍之约,元伯之与巨卿。
与善人交,如入芝兰之室,久而不闻其香;与恶人交,如入鲍鱼之肆,久而不闻其臭。
肝胆相照,斯为腹心之友;意气不孚,谓之口头之交。
彼此不合,谓之参商;尔我相仇,如同冰炭。
民之失德,乾糇以愆;他山之石,可以攻玉。
落月屋梁,相思颜色;暮云春树,想望丰仪。
王阳在位,贡禹弹冠以待荐;杜伯非罪,左儒宁死不徇君。
分首判袂,叙别之辞;拥彗扫门,迎迓之敬。
陆凯折梅逢驿使,聊寄江南一枝春;王维折柳赠行人,遂唱《阳关三叠》曲。
频来无忌,乃云入慕之宾;不请自来,谓之不速之客。
醴酒不设,楚王戊待士之意怠;投辖于井,汉陈遵留客之心诚。

蔡邕倒屣以迎宾,周公握发而待士。
陈蕃器重徐稚,下榻相延;孔子道遇程生,倾盖而语。
伯牙绝弦失子期,更无知音之辈;管宁割席拒华歆,调非同志之人。
分金多与,鲍叔独知管仲之贫;绨袍垂爱,须贾深怜范叔之窘。
要知主宾联以情,须尽东南之美;朋友合以义,当展切偲之诚。

\chapter{婚姻}

良缘由夙缔,佳偶自天成。
蹇修与柯人,皆是煤妁之号;冰人与掌判,悉是传言之人。
礼须六礼之周,好合二姓之好。
女嫁曰于归,男婚日完娶。
婚姻论财,夷虏之道;同姓不婚,周礼则然。
女家受聘礼,谓之许缨;新妇谒祖先,谓之庙见。
文定纳采,皆为行聘之名;女嫁男婚,谓了子平之愿。
聘仪日雁币,卜妻曰凤占。
成婚之日曰星期,传命之人曰月老。
下采即是纳币,合卺系是交杯。
执巾栉,奉箕帚,皆女家自谦之词;娴姆训,习《内则》,皆男家称女之说。
绿窗是贫女之室,红楼是富女之居。
桃夭谓婚姻之及时,摽梅谓婚期之已过。
御沟题叶,于祐始得宫娥;绣幕牵丝,元振幸获美女。
汉武与景帝论妇,欲将金屋贮娇;韦固与月老论婚,始知赤绳系足。
朱陈一村而结好,秦晋两国以联姻。
蓝田种玉,雍伯之缘;宝窗选婚,林甫之女。

架鹊桥以渡河,牛女相会;射雀屏而中目,唐高得妻。
至若礼重亲迎,所以正人伦之始;《诗》首好逑,所以崇王化之原。

\chapter{女子}

男子禀乾之刚,女子配坤之顺。
贤后称女中尧舜,烈女称女中丈夫。
曰闺秀,曰淑媛,皆称贤女;曰阃范,曰懿德,并美佳人。
妇主中馈,烹治饮食之名;女子归宁,回家省亲之谓。
何谓三从,从父从夫从子;何谓四德,妇德妇言妇工妇容。
周家母仪,太王有周姜,王季有太妊,文王有太姒;三代亡国,夏桀以妺喜,商纣以妲己,周幽以褒姒。
兰蕙质,柳絮才,皆女人之美誉;冰雪心,柏舟操,悉孀妇之清声。
女貌娇娆,谓之尤物;妇容妖媚,实可倾城。
潘妃步朵朵莲花,小蛮腰纤纤杨柳。
张丽华发光可鉴,吴绛仙秀色可餐。
丽娟气馥如兰,呵气结成香雾;太真泪红于血,滴时更结红冰。
孟光力大,石臼可擎;飞燕身轻,掌上可舞。
至若缇萦上书而救父,卢氏冒刃而卫姑,此女之孝者。
侃母截发以延宾,村媪杀鸡而谢客,此女之贤者。
韩玖英恐贼秽而自投于秽,陈仲妻恐陨德而宁陨于崖,此女之烈者。
王凝妻被牵,断臂投地;曹令女誓志,引刀割鼻,此女之节者。
曹大家续完《汉》帙,徐惠妃援笔成文,此女之才者。

戴女之练裳竹笥,孟光之荆钗裙布,此女之贫者。
柳氏秃妃之发,郭氏绝夫之嗣,此女之妒者。
贾女偷韩寿之香,齐女致袄庙之毁,此女之淫者。
东施效颦而可厌,无盐刻画以难堪,此女之丑者。
自古贞淫各异,人生妍丑不齐。
是故生菩萨、九子母、鸠盘荼,谓妇态之更变可畏;钱树子、一点红、无廉耻,谓青楼之妓女殊名。
此固不列于人群,亦可附之以博笑。

\chapter{外戚}

帝女乃公侯主婚,故有公主之称;帝婿非正驾之车,乃是附马之职。
郡主县君,皆宗女之谓;仪宾国宾,皆宗婿之称。
旧好曰通家,好亲日懿戚。
冰清玉润,丈人女婿同荣;泰水泰山,岳母岳父两号。
新婿曰娇客,贵婿日乘龙。
赘婚曰馆甥,贤婿曰快婿。
凡属东床,俱称半子。
女子号门楣,唐贵妃有光于父母;外甥称宅相,晋魏舒期报于母家。
共叙旧姻,曰原有瓜葛之亲;自谦劣戚,曰忝在葭莩之末。
大乔小乔,皆姨夫之号;连襟连袂,亦姨夫之称。
蒹葭依玉树,自谦借戚属之光;茑萝施乔松,自幸得依附之所。

\chapter{老幼寿诞}

不凡之子,必异其生;大德之人,必得其寿。
称人生日,曰初度之辰;贺人逢旬,曰生申令旦。
三朝洗儿,曰汤饼之会;周岁试婴,曰晬盘之期。
男生辰曰悬弧令旦,女生辰曰设帨佳辰。
贺人生子,曰嵩岳降神;自谦生女,曰缓急非益。
生子曰弄璋,生女曰弄瓦。
梦熊梦罴,男子之兆;梦虺梦蛇,女子之祥。
梦兰叶吉兆,郑燕姞生穆公之奇;英物试啼声,晋温峤闻声知桓公之异。
姜嫄生稷,履大人之迹而有娠;简狄生契,吞玄鸟之卵而叶孕。
鳞吐玉书,天生孔子之瑞;王燕投怀,梦孕张说之奇。
弗陵太子,怀胎十四月而始生;老子道君,在孕八十一年而始诞。
晚年得子,调之老蚌生珠;暮岁登科,正是龙头属老。
贺男寿曰南极星辉,贺女寿曰中天婺焕。
松柏节操,美其寿元之耐久;桑榆晚景,自谦老景之无多。
矍铄称人康健,聩眊自谦衰颓。
黄发儿齿,有寿之征;龙钟潦倒,年高之状。
日月逾迈,徒自伤悲;春秋几何,问人寿算。

称少年曰春秋鼎盛,羡高年曰齿德俱尊。
行年五十,当知四十九年之非;在世百年,那有三万六千日之乐。
百岁曰上寿,八十曰中寿,六十曰下寿;八十日耋,九十曰耄,百岁曰期颐。
童子十岁就外傅,十三舞勺,成童舞象;老者六十杖于乡,七十杖于国,八十杖于朝。
后生固为可畏,而高年尤是当尊。

\chapter{身体}

百体皆血肉之躯,五官有贵贱之别。
	尧眉分八彩,舜目有重瞳。
	耳有三漏,大禹之奇形;臂有四肘,成汤之异体。
	文王龙颜而虎眉,汉高斗胸而隆准。
	孔圣之顶若芋,文王之胸四乳。
	周公反握,作兴周之相;重耳骈胁,为霸晋之君。
	此皆古圣之英姿,不凡之贵品。
	至若发肤不可毁伤,曾子常以守身为大;待人须当量大,师德贵于唾面自干。
	谗口中伤,金可铄而骨可销;虐政诛求,敲其肤而吸其髓。
	受人牵制曰掣肘,不知羞愧曰厚颜。
	好生议论,曰摇唇鼓舌;共话衷肠,曰促膝谈心。
	怒发冲冠,蔺相如之英气勃勃;炙手可热,唐崔铉之贵势炎炎。
	貌虽瘦而天下肥,唐玄宗之自谓;口有蜜而腹有剑,李林甫之为人。
	赵子龙一身都是胆,周灵王初生便有须。
	来俊臣注醋于囚鼻,法外行凶;严子陵加足于帝腹,忘其尊贵。
	久不屈兹膝,郭子仪尊居宰相;不为米折腰,陶渊明不拜吏胥。
	断送老头皮,杨璞得妻送之诗;新剥鸡头肉,明皇爱贵妃之乳。

纤指如春笋,媚眼若秋波。
肩曰玉楼,眼名银海;泪曰玉箸,顶曰珠庭。
歇担曰息肩,不服曰强项。
丁谓与人拂须,何其谄也;彭乐截肠决战,不亦勇乎。
剜肉医疮,权济目前之急;伤胸扪足,计安众士之心。
汉张良蹑足附耳,东方朔洗髓伐毛。
尹维伦,契丹称为黑面大王;傅尧俞,宋后称为金玉君子。
土木形骸,不自妆饰;铁石心肠,秉性坚刚。
叙会晤曰得挹芝眉,叙契阔曰久违颜范。
请女客曰奉迓金莲,邀亲友曰敢攀玉趾。
侏儒谓人身矮,魁梧称人貌奇。
龙章凤姿,廊庙之彦;獐头鼠目,草野之夫。
恐惧过甚,曰畏首畏尾;感佩不忘,曰刻骨铭心。
貌丑曰不扬,貌美曰冠玉。
足跛曰蹒跚,耳聋曰重听。
欺欺艾艾,口讷之称;喋喋便便,言多之状。
可嘉者小心翼翼,可鄙者大言不惭。
腰细曰柳腰,身小曰鸡肋。

笑人齿缺,曰狗窦大开;讥人不决,曰鼠首僨事。
口中雌黄,言事而多改移;皮里春秋,胸中自有褒贬。
唇亡齿寒,谓彼此之失依;足上首下,谓尊卑之颠倒。
所为得意,曰吐气扬眉;待人诚心,曰推心置腹。
心荒曰灵台乱,醉倒曰玉山颓。
睡曰黑甜,卧曰偃息。
口尚乳臭,调世人年少无知;三折其肱,谓医士老成谙练。
西子捧心,愈见增妍;丑妇效颦,弄巧反拙。
慧眼始知道骨,肉眼不识贤人。
婢膝奴颜,谄容可厌;胁肩谄笑,媚态难堪。
忠臣披肝,为君之药;妇人长舌,为厉之阶。
事遂心曰如愿,事可愧曰汗颜。
人多言曰饶舌,物堪食曰可口。
泽及枯骨,西伯之深仁;灼艾分痛,宋祖之友爱。
唐太宗为臣疗病,亲剪其须;颜杲卿骂贼不辍,贼断其舌。
不较横逆,曰置之度外;洞悉虏情,曰已入掌中。
马良有白眉,独出乎众;阮籍作青眼,厚待乎人。
咬牙封雍齿,计安众将之心;含泪斩丁公,法正叛臣之罪。

掷果盈车,潘安仁美姿可爱;投石满载,张孟阳丑态堪憎。
事之可怪,妇人生须;事所骇闻,男人诞子。
求物济用,谓燃眉之急;悔事无成,曰噬脐何及。
情不相关,如秦越人之视肥瘠;事当探本,如善医者只论精神。
无功食禄,谓之尸位素餐;谫劣无能,谓之行尸走肉。
老当益壮,宁知白首之心;穷且益坚,不坠青云之志。
一息尚存,此志不容少懈;十手所指,此心安可自欺。

\chapter{衣服}

冠称元服,衣曰身章。
曰弁曰冔曰冕,皆冠之号;日履日舄曰屣,悉鞋之名。
上公命服有九锡,士人初冠有三加。
簪缨缙绅,仕宦之称;章甫缝掖,儒者之服。
布衣即白丁之谓,青衿乃生员之称。
葛屦履霜,诮俭啬之过甚;绿衣黄里,讥贵贱之失伦。
上服曰衣,下服曰裳;衣前曰襟,衣后曰裾。
敝衣曰褴褛,美服曰华裾。
襁褓乃小儿之衣,弁髦亦小儿之饰。
左衽是夷狄之服,短后是武夫之衣。
尊卑失序,如冠履倒置;富贵不归,如锦衣夜行。
狐裘三十年,俭称晏子;锦幛四十里,富羡石崇。
孟尝君珠履三千客,牛僧孺金钗十二行。
千金之裘,非一狐之腋;绮罗之辈,非养蚕之人。
贵者重裀叠褥,贫者裋褐不完。
卜子夏甚贫,鹑衣百结;公孙弘甚俭,布被十年。
南州冠冕,德操称庞统之迈众;三河领袖,崔浩羡裴骏之超群。

虞舜制衣裳,所以命有德;昭侯藏敝裤,所以待有功。
唐文宗袖经三浣,晋文公衣不重裘。
衣履不敝,不肯更为,世称尧帝;衣不经新,何由得故,妇劝桓冲。
王氏之眉贴花钿,被韦固之剑所刺;贵妃之乳服诃子,为禄山之爪所伤。
姜氏翕和,兄弟每宵同大被;王章未遇,夫妻寒夜卧牛衣。
绶带轻裘,羊叔子乃斯文主将;葛巾野服,陶渊明真陆地神仙。
服之不衷,身之灾也;緼袍不耻,志独超欤。

\part{}

\chapter{人事}

《大学》首重夫明新,小于莫先于应对。
其容固宜有度,出言尤贵有章。
智欲圆而行欲方,胆欲大而心欲小。
阁下足下,并称人之辞;不佞鲰生,皆自谦之语。
恕罪曰原宥,惶恐曰主臣。
大春元、大殿选、大会状,举人之称不一;大秋元、大经元、大三元,士人之誉多殊。
大掾史,推美吏员;大柱石,尊称乡宦。
贺入学曰云程发轫,贺新冠曰元服初荣。
贺人荣归,谓之锦旋;作商得财,谓之稇载。
谦送礼曰献芹,不受馈曰反璧。
谢人厚礼曰厚贶,自谦利薄曰菲仪。
送行之礼,谓之赆仪;拜见之赀,名曰贽敬。
贺寿仪曰祝敬,吊死礼曰奠仪。
请人远归曰洗尘,携酒进行曰祖饯。
犒仆夫,谓之旌使;演戏文,谓之俳优。
谢人寄书,曰辱承华翰;谢人致问,曰多蒙寄声。
望人寄信,曰早赐玉音;谢人许物,曰已获金诺。

具名帖,曰投刺;发书函,曰开缄。
思暮久曰极切瞻韩,想望殷曰久怀慕蔺。
相识未真,曰半面之识;不期而会,曰邂逅之缘。
登龙门得参名士,瞻山斗仰望高贤。
一日三秋,言思暮之甚切;渴尘万斛,言想望之久殷。
暌违教命,乃云鄙吝复萌;来往无凭,则曰萍踪靡定。
虞舜幕唐尧,见尧于羹,见尧于墙;门人学孔圣,孔步亦步,孔趋亦趋。
曾经会晤,曰向获承颜接辞;谢人指教,曰深蒙耳提面命。
求人涵容,曰望包荒;求人吹嘘,曰望汲引。
求人荐引,曰幸为先容;求人改文,曰望赐郢斫。
借重鼎言,是托人言事;望移玉趾,是浼人亲行。
多蒙推毂,谢人引荐之辞;望作领袖,托人倡首之说。
言辞不爽,谓之金石语;乡党公论,谓之月旦评。
逢人说项斯,表扬善行;名下无虚士,果是贤人。
党恶为非曰朋奸,尽财赌博曰孤注。
徒了事,曰但求塞责;戒明察,曰不可苛求。
方命是逆人之言,执拗是执己之性。
曰觊觎,曰睥睨,总是私心之窥望;曰倥偬,曰旁午,皆言人事之纷纭。

小过必察,谓之吹毛求疵;乘患相攻,谓之落井下石。
欲心难厌如溪壑,财物易尽若漏卮。
望开茅塞,是求人之教导;多豪药石,是谢人之箴规。
劳规芳躅,皆善行之可慕;格言至言,悉嘉言之可听。
无言曰缄默,息怒曰霁威。
包拯寡色笑,人比其笑为黄河清;商鞅最凶残,常见论囚而渭水赤。
仇深曰切齿,人笑曰解颐。
人微笑曰莞尔,掩口笑曰胡卢。
大笑回绝倒,众笑曰哄堂。
留位待贤,谓之虚左;官僚共署,谓之同寅。
人失信曰爽约,又曰食言;人忘誓曰寒盟,又曰反汗。
铭心镂骨,感德难忘;结草衔环,知恩必报。
自惹其灾,谓之解衣抱火;幸离其害,真如脱网就渊。
两不相入,谓之枘凿;两不相投,谓之冰炭。
彼此不合曰龃龉,欲进不前曰趦趄。
落落不合之词,区区自谦之语。
竣者,作事已毕之谓;醵者,敛财饮食之名。
赞襄其事,谓之玉成;分裂难完,谓之瓦解。

事有低昂曰轩轾,力相上下曰颉颃。
凭空起事曰作俑,仍踵前弊曰效尤。
手口共作曰拮据,不暇修容曰鞅掌。
手足并行曰匍匐,俯首而思曰低徊。
明珠投暗,大屈才能;入室操戈,自相鱼肉。
求教于愚人,是问道于盲;枉道以干主,是衒玉求售。
智谋之士,所见略同;仁人之言,其利甚溥。
班门弄斧,不知分量;岑楼齐末,不识高卑。
势延莫遏,谓之滋蔓难图;包藏祸心,谓之人心叵测。
作舍道旁,议论多而难成;一国三公,权柄分而不一。
事有奇缘,曰三生有幸;事皆拂意,曰一事无成。
酒色是耽,如以双斧代孤树;力量不胜,如以寸胶澄黄河。
兼听则明,偏听则暗,此魏征之对太宗;众怒难犯,专欲难成,此于产之讽子孔。
欲逞所长,谓之心烦技痒;绝无情欲,谓之槁木死灰。
座上有江南,语言须谨;往来无白丁,交接皆贤。
将近好处,曰渐入佳境;无端倨傲,曰旁若无人。
借事宽役曰告假,将钱嘱托曰夤缘。
事有大利,曰奇货可居;事宜鉴前,曰覆车当戒。

外彼为此曰左袒;处事而可曰摸棱。
敌甚易摧,曰发蒙振落;志在必胜,曰破釜沉舟。
曲突徙薪无恩泽,不念豫防之力大;焦头烂额为上客,徒知救急之功宏。
贼人曰梁上君子,强梗曰化外顽民。
木屑竹头,皆为有用之物;牛溲马渤,可备药石之资。
五经扫地,祝钦明自亵斯文;一木撑天,晋王敦未可擅动。
题凤题午,讥友讥亲之隐词;破麦破梨,见夫见子之奇梦。
毛遂片言九鼎,人重其言;季市一诺千金,人服其信。
岳飞背涅精忠报国,杨震惟以清白传家。
下强上弱,曰尾大不掉;上权下夺,曰太阿倒持。
当今之世,不但君择臣,臣亦择君;受命之主,不独创业难,守成亦不易。
生平所为皆可对人言,司马光之自信;运用之妙惟存乎一心,岳武穆之论兵。
不修边幅,谓人不饰仪容;不立崖岸,谓人天性和乐。
蕞尔幺么,言其甚小;卤莽灭裂,言其不精。
误处皆缘不学,强作乃成自然。
求事速成曰躐等,过于礼貌曰足恭。
假忠厚者谓之乡愿,出人群者谓之巨擘。
孟浪由于轻浮,精详出于暇豫。

为善则流芳百世,为恶则遗臭万年。
过多曰稔恶,罪满曰贯盈。
尝见冶容诲淫,须知慢藏诲盗。
管中窥豹,所见不多;坐井观天,知识不广。
无势可乘,英雄无用武之地;有道则见,君子有展采之恩。
求名利达,曰捷足先得;慰士迟滞,曰大器晚成。
不知通变,曰徒读父书;自作聪明,曰徒执己见。
浅见曰肤见,俗言曰俚言。
识时务者为俊杰,昧先几者非明哲。
村夫不识一丁,愚者岂无一得。
拔去一丁,谓除一害;又生一秦,是增一仇。
戒轻言,曰恐属垣有耳;戒轻敌,曰勿谓秦无人。
同恶相帮,谓之助桀为虐;贪心无厌,谓之得陇望蜀。
当知器满则倾,须知物极必反。
喜嬉戏名为好弄,好笑谑谓之诙谐。
谗口交加,市中可信有虎;众奸鼓衅,聚蚊可以成雷。
萋非成锦,谓谮人之酿祸;含沙射影,言鬼蜮之害人。
针砭所以治病;鸩毒必至杀人。

李义府阴柔害物,人谓之笑里藏刀;李林甫奸诡谄人,世谓之口蜜腹剑。
代人作事,曰代庖;与人设谋,曰借箸。
见事极真,曰明若观火;对敌易胜,曰势若摧枯。
汉武内多欲而外施仁义,廉颇先国难而后私仇。
卧榻之侧,岂容他人鼾睡,宋太祖之语;一统之世,真是胡越一家,唐太宗之时。
至若景泰以吕易嬴,是嬴亡于庄襄之手:弱晋以牛易马,是马灭于怀愍之时。
中宗亲为点筹于韦后,秽播千秋;明皇赐洗儿钱于贵妃,臭遗万代。
非类相从,不如鹑鹊;父子同牝,谓之聚麀。
以下淫上谓之烝,野合奸伦谓之乱。
从来淑慝殊途,惟在后人法戒;欺世情浊异品,全赖吾辈激扬。

\chapter{饮食}

甘脆肥脓,命曰腐肠之药;羹藜含糗,难语太牢之滋。
御食曰珍馐,白米曰玉粒。
好酒曰青州从事,次酒曰平原督邮。
鲁酒茅柴,皆为薄酒;龙团雀舌,尽是香茗。
待人礼衰,曰醴酒不设;款客甚薄,曰脱粟相留。
竹叶青,状元红,俱为美酒;葡萄绿,珍珠红,悉是香醪。
五斗解酲,刘伶独溺于酒;两腋生风,卢仝偏嗜乎茶。
茶曰酪奴,又曰瑞草;米曰白粲,又曰长腰。
太羹玄酒,亦可荐馨;尘饭涂羹,焉能充饿。
酒系杜康所造,腐乃淮南所为。
僧谓鱼曰水梭花,僧谓鸡曰穿篱菜。
临渊羡鱼,不如退而结网;扬汤止沸,不如去火抽薪。
羔酒自劳,田家之乐;含哺鼓腹,盛世之风。
人贪食曰徒餔啜,食不敬曰嗟来食。
多食不厌,谓之饕餮之徒;见食垂涎,谓有欲炙之色。
未获同食,曰向隅;谢人赐食,曰饱德。
安步可以当车,晚食可以当肉。

饮食贫难曰半菽不饱,厚恩图报曰每饭不忘。
谢扰人曰兵厨之扰,谦待薄曰草具之陈。
白饭青刍,待仆马之厚;炊金爨玉,谢款客之隆。
家贫待客,但知抹月披风;冬月邀宾,乃曰敲冰煮茗。
君侧元臣,若作酒醴之麴蘖;朝中冢宰,若作和羹之盐梅。
宰肉甚均,陈平见重于父老;戛釜示尽,邱嫂心厌乎汉高。
毕卓为吏部而盗酒,逸兴太豪;越王爱士卒而投醪,战气百倍。
惩羹吹齑,谓人惩前警后;酒囊饭袋,谓人少学多餐。
隐逸之士,漱石枕流;沉湎之夫,藉糟枕麴。
昏庸桀纣,胡为酒池肉林;苦学仲淹,惟有断齑画粥。

\chapter{宫室}

洪荒之世,野处穴居;有巢以后,上栋下宇。
竹苞松茂,谓制度之得宜;鸟革翚飞,调创造之尽善。
朝廷曰紫宸,禁门曰青琐。
宰相职掌丝纶,内居黄阁;百官具陈章疏,敷奏丹墀。
木天署,学土所居;紫薇省,中书所莅。
金马玉堂,翰林院宇;柏台乌府,御史衙门。
布政司称为藩府,按察司系是臬司。
潘岳种挑于满县,人称花县;于贱鸣琴以治邑,故曰琴堂。
谭府是仕宦之家,衡门乃隐逸之宅。
贺人有喜,曰门阑蔼瑞;谢人过访,曰蓬荜生辉。
美奂美轮,《礼》称屋宇之高华;肯构肯堂,《书》言父子之同志。
土木方兴曰经始,创造已毕曰落成。
楼高可以摘星,屋小仅堪容膝。寇莱公庭除之外,只可栽花;李文靖厅事之前,仅容旋马。
恭贺屋成曰燕贺,自谦屋小曰蜗庐。民家名曰闾阎,贵族称为阀阅。
朱门乃富豪之第,白屋是布衣之家。
客舍曰逆旅,馆驿曰邮亭。
书室曰芸窗,朝廷曰魏阙。

成均辟雍,皆国学之号;黉宫胶序,乃乡学之称。
笑人善忘,曰徙宅忘妻;讥人不谨,曰开门揖盗。
何楼所市,皆滥恶之物;垄断独登,讥专利之人。
荜门圭窦,系贫土之居;瓮牖绳枢,皆窭人之室。
宋寇准真是北门锁钥,檀道济不愧万里长城。

\chapter{器用}

一人之所需,百工斯为备。
但用则各适其用,而名则每异其名。
管城子、中书君,悉为笔号;石虚中、即墨侯,皆为砚称。
墨为松使者,纸号楮先生。
纸曰剡藤,又曰玉版;墨曰陈玄,又曰龙脐。
共笔砚,同富之谓;付衣钵,传道之称。
笃志业儒,曰磨穿铁砚;弃文就武,曰安用毛锥。
剑有干将镆邪之名,扇有仁风便面之号。
何谓箑,亦扇之名;何谓籁,有声之谓。
小舟名蚱蜢,巨舰曰艨艟。
金根是皇后之车,菱花乃妇人之镜。
银凿落原是酒器,玉参差乃是箫名。
刻舟求剑,固而不通;胶柱鼓瑟,拘而不化。
斗筲言其器小,梁栋谓是大材。
铅刀无一割之利,强弓有六石之名。
杖以鸠名,因鸠喉之不噎;钥同鱼样,取鱼目之常醒。
兜鍪系是头盔,叵罗乃为酒器。

短剑名匕首,毡毯曰氍毹。
琴名绿绮焦桐,弓号乌号繁弱。
香炉曰宝鸭,烛台曰烛奴。
龙涎鸡舌,悉是香茗;鹢首鸭头,别为船号。
寿光客,是妆台无尘之镜;长明公,是梵堂不灭之灯。
桔槔是田家之水车,袯襫是农夫之雨具。
乌金,炭之美誉;忘归,矢之别名。
夜可击,朝可炊,军中刁斗;《云汉》热,《北风》寒,刘褒画图。
勉人发愤,曰猛着祖鞭;求人宥罪,曰幸开汤网。
拔帜立帜,韩信之计甚奇;楚弓楚得,楚王所见未大。
董安于性援,常佩弦以自急;西门豹性急,常佩韦以自宽。
汉孟敏尝堕甑不顾,知其无益;宋太祖谓犯法有剑,正欲立成。
王衍清谈,常持麈尾;横渠讲《易》,每拥皋比。
尾生抱桥而死,固执不通;楚妃守符而亡,贞信可录。
温桥昔燃犀,照见水族之鬼怪;秦政有方镜,照见世人之邪心。
车载斗量之人,不可胜数;南金东箭之品,实是堪奇。
传檄可定,极言敌之易破;迎刃而解,甚言事之易为。
以铜为鉴,可整衣冠;以古为鉴,可知兴替。

\chapter{珍宝}

山川之精英,每泄为至宝;乾坤之瑞气,恒结为奇珍。
故玉足以庇嘉谷,明珠可以御火灾。
鱼目岂可混珠,碔砆焉能乱玉。
黄金生于丽水,白银出自朱提。
曰孔方、曰家兄,俱为钱号;日青蚨,曰鹅眼,亦是钱名。
可贵者明月夜光之珠,可珍者璠玙琬琰之玉。
宋人以燕石为玉,什袭缇巾之中;楚王以璞玉为石,两刖卞和之足。
惠王之珠,光能照乘;和氏之壁,价重连城。
鲛人泣泪成珠,宋人削玉为楮。
贤乃国家之宝,儒为席上之珍。
王者聘贤,束帛加壁;真儒抱道,怀瑾握瑜。
雍伯多缘,种玉于蓝田而得美妇;太公奇遇,钓璜于渭水而遇文王。
剖腹藏珠,爱财而不爱命;缠头作锦,助舞而更助娇。
孟尝廉洁,克俾合浦还珠;相如忠勇,能使秦廷归璧。
玉钗作燕飞,汉宫之异事;金钱成蝶舞,唐库之奇传。
广钱固可以通神,营利乃为鬼所笑。
以小致大,谓之抛砖引玉;不知所贵,谓之买椟还珠。

贤否罹害,如玉石俱焚;贪得无厌,虽辎珠必算。
崔烈以钱买官,人皆恶其铜臭;秦嫂不敢视叔,自言畏其多金。
熊衮父亡,天乃雨钱助葬;仲儒家窘,天乃雨金济贫。
汉杨震畏四知而辞金,唐太宗因惩贪而赐绢。
晋鲁褒作《钱神论》,尝以钱为孔方兄;王夷甫口不言钱,乃谓钱为阿堵物。
然而床头金尽,壮士无颜;囊内钱空,阮郎羞涩。
但匹夫不可怀壁,人生孰不爱财。

\chapter{贫富}

命之修短有数,人之富贵在天。
惟君子安贫,达人知命。
贯朽粟陈,称羡财多之谓;紫标黄榜,封记钱库之名。
贪爱钱物,谓之钱愚;好置由宅,谓之地癖。
守钱虏,讥蓄财而不散;落魄夫,谓失业之无依。
贫者地无立锥,富者田连阡陌。
室如悬磬,言其甚窘;家无儋石,谓其极贫。
无米曰在陈,守死曰待毙。富足曰殷实,命蹇曰数奇。
苏涸鲋,乃济人之急;呼庚癸,是乞人之粮。
家徒壁立,司马相如之贫;扊扅为炊,秦百里奚之苦。
鹄形菜色,皆穷民饥饿之形;炊骨爨骸,谓军中乏粮之惨。
饿死留君臣之义,伯夷叔齐;资财敌王公之富,陶朱倚顿。
石崇杀妓以侑酒,恃富行凶;何曾一食费万钱,奢侈过甚。
二月卖新丝,五月粜新谷,真是剜肉医疮;三年耕而有一年之食,九年耕而有三年之食,庶几遇荒有备。
贫士之肠习黎苋,富人之口厌膏梁。
石崇以蜡代薪,王恺以饴沃釜。
范丹土灶生蛙,破甑生尘;曾子捉襟见肘,纳履决踵,贫不胜言。

子路衣敝缊袍,与轻裘立;韦庄数米而饮,称薪而爨,俭有可鄙。
总之饱德之士,不愿膏梁;闻誉之施,奚图文绣。

\chapter{疾病死丧}

福寿康宁,固人之所同欲;死亡疾病,亦人所不能无。
惟智者能调,达人自玉。
问人病曰贵体违和,自谓疾曰偶沾微恙。
罹病者,甚为造化小儿所苦;患病者,岂是实沈台骀为灾。
病不可疗,曰膏肓;平安无事,曰无恙。
采薪之忧,谦言抱病;河鱼之患,系是腹疾。
可以勿药,喜其病安;厥疾勿瘳,言其病笃。
疟不病君子,病君子正为疟耳;卜所以决疑,既不疑复何卜哉?
谢安梦鸡而疾不起,因太岁之在酉;楚王吞蛭而疾乃痊,因厚德之及人。
将属纩,将易篑,皆言人之将死;作古人,登鬼箓,皆言人之已亡。
亲死则丁忧,居丧则读《礼》。
在床谓之尸,在棺谓之柩。
报孝书曰讣,慰孝子曰唁。
往吊曰匍匐,庐墓日倚庐。
寝苫枕块,哀父母之在土;节哀顺变,劝孝子之惜身。
男子死曰寿终正寝,女人死曰寿终内寝。
天子死曰崩,诸侯死曰薨,大夫死曰卒,土人死曰不禄,庶人死曰死,童子死曰殇。

自谦父死曰孤子,母死曰哀子,父母俱死曰孤哀子;自言父死曰失怙,母死曰失恃,父母俱死曰失怙恃。
父死何谓考,考者成也,已成事业也;母死何谓妣,妣者媲也,克媲父美也。
百日内曰泣血,百日外曰稽颡。
期年曰小祥,两期曰大祥。
不缉曰斩衰,缉之曰齐衰,论丧之有轻重;九月为大功,五月为小功,言服之有等伦。
三月之服曰缌麻,三年将满曰禫礼。
孙承祖服,嫡孙杖期;长子已死,嫡孙承重。
死者之器曰明器,待以神明之道;孝子之枚曰哀杖,为扶哀痛之躯。
父之节在外,故杖取乎竹;母之节在内,故杖取乎桐。
以财物助丧家,谓之赙;以车马助丧家,谓之赗;以衣殓死者之身,谓之禭;以玉实死者之口,谓之琀。
送丧曰执绋,出柩曰驾輀。
吉地曰牛眠地,筑坟曰马鬣封。
墓前石人,原名翁仲;柩前功布,今日铭旌。
挽歌始于田横,墓志创于傅奕。
生坟曰寿藏,死墓曰佳城。
坟曰夜台,圹曰窀穸。
已葬曰瘗玉,致祭曰束刍。
春祭曰禴,夏祭曰禘,秋祭曰尝,冬祭曰烝。

饮杯棬而抱痛,母之口泽如存;读父书以增伤,父之手泽未泯。
子羔悲亲而泣血,子夏哭子而丧明。
王裒哀父之死,门人因废《蓼莪》诗;王修哭母之亡,邻里遂停桑柘杜。
树欲静而风不息,子欲养而亲不在。
皋鱼增感,与其椎牛而祭墓,不如鸡豚之逮存,曾子兴思。
故为人子者,当思木本水源,须重慎终追远。

\part{}

\chapter{文事}

多才之士,才储八斗;博学之德,学富五车。
《三坟》《五典》,乃三皇五帝之书;《八索》《九丘》,是八泽九州之志。
《书经》载上古唐虞三代之事,故曰尚书;《易经》乃姬周文王周公所系,故曰《周易》。
二戴曾删《礼记》,故曰《戴礼》;二毛曾注《诗经》,故曰《毛诗》。
孔子作《春秋》,因获麟而绝笔,故曰《麟经》。
荣于华衮,乃《春秋》一字之褒;严于斧钺,乃《春秋》一字之贬。
缣缃黄卷,总谓经书;雁帛鸾笺,通称简札。
锦心绣口,李太白之文章;铁画银钩,王羲之之字法。
雕虫小技,自谦文学之卑;倚马可待,羡人作文之速。
称人近来进德,曰士别三日,当刮目相看;羡人学业精通,曰面壁九年,始有此神悟。
五凤楼手,称文字之精奇;七步奇才,羡天才之敏捷。
誉才高,曰今之班马;羡诗工,曰压倒元白。
汉晁错多智,景帝号为智囊;高仁裕多诗,时人谓之诗窖。
骚客即是诗人,誉髦乃称美士。
自古诗称李杜,至今字仰钟王。
白雪阳春,是难和难赓之韵;青钱万选,乃屡试屡中之文。
惊神泣鬼,皆言词赋之雄豪;遏云绕梁,原是歌音之嘹喨。

涉猎不精,是多学之弊;咿唔呫毕,皆读书之声。
连篇累牍,总说多文;寸楮尺素,通称简札。
以物求文,谓之润笔之资;因文得钱,乃曰稽古之力。
文章全美,曰文不加点;文章奇异,曰机杼一家。
应试无文,谓之曳白;书成绣梓,谓之杀青。
袜线之才,自谦才短;记问之学,自愧学肤。
裁诗曰推敲,旷学曰作辍。
文章浮薄,何殊月露风云;典籍储藏,皆在兰台石室。
秦始皇无道,焚书坑儒;唐太完好文,开科取士。
花样不同,乃谓文章之异;潦草塞责,不求辞语之精。
邪说曰异端,又曰左道;读书曰肄业,又曰藏修。
作文曰染翰操觚,从师曰执经问难。
求作文,曰乞挥如椽笔;羡高文,曰才是大方家。
竞尚佳章,曰洛阳纸贵;不嫌问难,曰明镜不疲。
称人书架曰邺架,称人嗜学曰书淫。
白居易生七月,便识之无二字;唐李贺才七岁,作高轩过一篇。
开卷有益,宋太宗之要语;不学无术,汉霍光之为人。
汉刘向校书于天禄,太乙燃藜;赵匡胤代位于后周,陶谷出诏。

江淹梦笔生花,文思大进;扬雄梦吐白凤,词赋愈奇。
李守素通姓氏之学,敬宗名为人物志;虞世南晰古今之理,太宗号为行秘书。
茹古含今,皆言学博;咀英嚼华,总曰文新。
文望尊隆,韩退之若泰山北斗;涵养纯粹,程明道如良玉精金。
李白才离,咳唾随风生珠玉;孙绰词丽,诗赋掷地作金声。

\chapter{科第}

士人入学曰游泮,又曰采芹;士人登科曰释褐,又曰得隽。
宾兴即大比之年,贤书乃试录之号。
鹿鸣宴,款文榜之贤;鹰扬宴,待武科之士。
文章入式,有朱衣以点头;经术既明,取青紫如拾芥。
其家初中,谓之破天荒;士人超拔,谓之出头地。
中状元,曰独占鳌头;中解元,曰名魁虎榜。
琼林赐宴,宋太宗之伊始;临轩问策,宋神宗之开端。
同榜之人,皆是同年;取中之官,谓之座主。
应试见遗,谓之龙门点额;进士及第,谓之雁塔题名。
贺登科,曰荣膺鹗荐;入贡院,曰鏖战棘闱。
金殿唱名曰传胪,乡会放榜曰撤棘。
攀仙桂,步青云,皆言荣发;孙山外,红勒帛,总是无名。
英雄入吾彀,唐太宗喜得佳士;桃李属春官,刘禹锡贺得门生。
薪,采也,槱,积也,美文王作人之诗,故考士谓之薪槱之典;汇,类也,征,进也,是连类同进之象,故进贤谓之汇征之途。
赚了英雄,慰人下第;傍人门户,怜士无依。
虽然有志者事竟成,伫看荣华之日;成丹者火候到,何惜烹炼之功。

\chapter{制作}

上古结绳记事,苍颉制字代绳。
龙马负图,伏羲因画八卦;洛龟呈瑞,大禹因列九畴。
历日是神农所为,甲子乃大桡所作。
算数作于隶首,律吕造自伶伦。
甲胄舟车,系轩辕之创始;权量衡度,亦轩辕之立规。
伏羲氏造网罟,教佃渔以赡民用;唐太宗造册籍,编里甲以税田粮。
兴贸易,制耒耜,皆由炎帝;造琴瑟,教嫁娶,乃是伏羲。
冠冕衣裳,至黄帝而始备;桑麻蚕绩,自元妃而始兴。
神农尝百草,医药有方;后稷播百谷,粒食攸赖。
燧人氏钻木取火,烹饪初兴;有巢氏构木为巢,宫室始创。
夏禹欲通神祗,因铸镛钟于郊庙;汉明尊崇佛教,始立寺观于中朝。
周公作指南车,罗盘是其遗制;钱乐作浑天仪,历家始有所宗。
育王得疾,因造无量宝塔;秦政防胡,特筑万里长城。
叔孙通制立朝仪,魏曹丕秩序官品。周公独制礼乐,萧何造立律条。
尧帝作围棋,以教丹朱;武王作象棋,以象战斗。
文章取士,兴于赵宋;应制以诗,起于李唐。
梨园子弟,乃唐明皇作始;《资治通鉴》,乃司马光所编。

笔乃蒙恬所造,纸乃蔡伦所为。
凡今人之利用,皆古圣之前民。

\chapter{技艺}

医士业岐轩之术,称曰国手;地师习青乌之书,号曰堪舆。
卢医扁鹊,古之名医;郑虔崔白,古之名画。
晋郭璞得《青囊经》,故善卜筮地理;孙思邈得龙宫方,能医虎口龙鳞。
善卜者,是君平、詹尹之流;善相者,即唐举、子卿之亚。
推命之士即星士,绘图之士曰丹青。
大风鉴,相士之称;大工师,木匠之誉。
若王良,若造父,皆善御之人;东方朔,淳于髡,系滑稽之辈。
称善卜卦者,曰今之鬼谷;称善记怪者,曰古之董狐。
称诹日之人曰太史,称书算之人曰掌文。
掷骰者,喝雉呼卢;善射者,穿杨贯虱。
樗蒱之戏,乃云双陆;橘中之乐,是说围棋。
陈平作傀儡,解汉高白登之围;孔明造木牛,辅刘备运粮之计。
公输子削木鸢,飞天至三日而不下;张僧繇画壁龙,点睛则雷电而飞腾。
然奇技似无益于人,而百艺则有济于用。

\chapter{讼狱}

世人惟不平则鸣,圣人以无讼为贵。
上有恤刑之主,桁杨雨润;下无冤枉之民,肺石风清。
虽囹圄便是福堂,而画地亦可为狱。
与人构讼,曰鼠牙雀角之争;罪人诉冤,有抢地吁天之惨。
狴犴猛犬而能守,故狱门画狴犴之形;棘木外刺而里直,故听讼在棘木之下。
乡亭之系有岸,朝廷之系有狱,谁敢作奸犯科;死者不可复生,刑者不可复续,上当原情定罪。
囹圄是周狱,羑里是商牢。
桎梏之设,乃拘罪人之具;缧绁之中,岂无贤者之冤。
两争不放,谓之鹬蚌相持;无辜牵连,谓之池鱼受害。
请公入瓮,周兴自作其孽;下车泣罪,夏禹深痛其民。
好讼曰健讼,挂告曰株连。
为人息讼,谓之释纷;被人栽冤,谓之嫁祸。
徒配曰城旦,遣戍是问军。三尺乃朝廷之法,三木是罪人之刑。
古之五刑,墨、劓、剕、宫、大辟;今之律例,笞、杖、死罪、徒、流。
上古时削木为吏,今日之淳风安在;唐太宗纵囚归狱,古人之诚信可嘉。
花落讼庭间,草生囹圄静,歌何易治民之间;吏从冰上立,人在镜中行,颂卢奂折狱之清。
可见治乱之药石,刑罚为重;兴平之梁肉,德教为先。

\chapter{释道鬼神}

如来释迦,即是牟尼,原系成佛之祖;老聃李耳,即是道君,乃是道教之宗。
鹫岭、祗园,皆属佛国;交梨、火枣,尽是仙丹。
沙门称释,始于晋道安;中国有佛,始于汉明帝。
籛铿即是彭祖,八百高年;许逊原宰旌阳,一家超举。
波罗犹云彼岸,紫府即是仙宫。
曰上方、曰梵刹,总是佛场;曰真宇、曰蕊珠,皆称仙境。
伊蒲馔可以斋僧,青精饭亦堪供佛。
香积厨,僧家所备;仙麟脯,仙子所餐。
佛图澄显神通,咒莲生钵;葛仙翁作戏术,吐饭成蜂。
达摩一苇渡江,栾巴噀酒灭火。
吴猛画江成路,麻姑掷米成珠。
飞锡挂锡,谓僧人之行止;导引胎息,谓道士之修持。
和尚拜礼曰和南,道士拜礼曰稽首。
曰圆寂、曰荼毗,皆言和尚之死;曰羽化、曰尸解,悉言道士之亡。
女道曰巫,男道曰觋,自古攸分;男僧曰僧,女僧曰尼,从来有别。
羽客黄冠,皆称道士;上人比丘,并美僧人。
檀越、檀那,僧家称施主;烧丹、炼汞,道士学神仙。

和尚自谦,谓之空桑子;道士诵经,谓之步虚声。
菩者普也,萨者济也,尊称神祇,故有菩萨之誉;水行龙力大,陆行象力大,负荷佛法,故有龙象之称。
儒家谓之世,释家谓之劫,道家谓之尘,俱谓俗缘之未脱;儒家曰精一,释家曰三昧,道家曰贞一,总言奥义之无穷。
达摩死后,手携只履西归;王乔朝君,舄化双凫下降。
辟谷绝粒,神仙能服气炼形;不灭不生,释氏惟明心见性。
梁高僧谈经入妙,可使岩石点头,天花坠地;张虚靖炼丹既成,能令龙虎并伏,鸡犬俱升。
藏世界于一粟,佛法何其大;贮乾坤于一壶,道法何其玄。
妄诞之言,载鬼一车;高明之家,鬼瞰其室。
《无鬼论》作于晋之阮瞻,《搜神记》撰于晋之干宝。
颜子渊、卜子商,死为地下修文郎;韩擒虎、寇莱公,死为阴司阎罗王。
至若土谷之神曰社稷,干旱之鬼曰旱魃。
魑魅魍魉,山川之祟;神荼郁垒,啖鬼之神。
仕途偃蹇,鬼神为之揶揄;心地光明,吉神自为之呵护。

\chapter{鸟兽}

麟为毛虫之长,虎乃兽中之王。
麟凤龟龙,谓之四灵;犬豕与鸡,谓之三物。
騄駬骅骝,良马之号;太牢大武,乃牛之称。
羊曰柔毛,又曰长髯主簿;豕名刚鬣,又曰乌喙将军。
鹅名舒雁,鸭号家凫。
鸡有五德,故称之为德禽;雁性随阳,因名之曰阳鸟。
家狸、乌圆,乃猫之誉;韩卢楚犷,皆犬之名。
麒麟驺虞,皆好仁之兽;螟螣蟊贼,皆害苗之虫。
无肠公子,螃蟹之名;绿衣使者,鹦鹉之号。
狐假虎威,谓借势而为恶;养虎贻患,谓留祸之在身。
犹豫多疑,喻人之不决;狼狈相倚,比人之颠连。
胜负未分,不知鹿死谁手;基业易主,正如燕入他家。
雁到南方,先至为主,后至为宾;雉名陈宝,得雄则王,得雌则霸。
刻鹄类鹜,为学初成;画虎类犬,弄巧成拙。
美恶不称,谓之狗尾续貂;贪图不足,谓之蛇欲吞象。
祸去祸又至,曰前门拒虎,后门进狼;除凶不畏凶,曰不入虎穴,焉得虎子。
鄙众趋利,曰群蚁附膻;谦己爱儿,曰老牛舐犊。

无中生有,曰画蛇添足;进退两难,曰羝羊触藩。
杯中蛇影,自起猜疑;塞翁失马,难分祸福。
龙驹凤雏,晋闵鸿夸吴中陆士龙之异;伏龙凤雏,司马徽称孔明庞士元之奇。
吕后断戚夫人手足,号曰人彘;胡人腌契丹王尸骸,谓之帝羓。
人之狠恶,同于梼杌;人之凶暴,类于穷奇。
王猛见桓温,扪虱而谈当世之务;宁戚遇齐桓,扣角而取卿相之荣。
楚王轼怒蛙,以昆虫之敢死;丙吉问牛喘,恐阴阳之失时。
以十人而制千虎,比言事之难胜;走韩卢而搏蹇兔,喻言敌之易摧。
兄弟如鹡鸰之相亲,夫妇如鸾凤之配偶。
有势莫能为,曰虽鞭之长,不及马腹;制小不用大,曰割鸡之小,焉用牛刀。
鸟食母者曰枭,兽食父者曰獍。
苛政猛于虎,壮士气如虹。
腰缠十万贯,骑鹤上扬州,谓仙人而兼富贵;盲人骑瞎马,夜半临深池,是险语之逼人闻。
黔驴之技,技止此耳;鼯鼠之技,技亦穷乎。
强兼并者曰鲸吞,为小贼者曰狗盗。
养恶人如养虎,当饱其肉,不饱则噬;养恶人如养鹰,饥之则附,饱之则飏。
随珠弹雀,谓得少而失多;投鼠忌器,恐因甲而害乙。
事多曰猬集,利小曰蝇头。

心惑似狐疑,人喜如雀跃。
爱屋及乌,谓因此而惜彼;轻鸡爱鹜,谓舍此而图他。
唆恶为非,曰教猱升木;受恩不报,曰得鱼忘筌。
倚势害人,真似城狐社鼠;空存无用,何殊陶犬瓦鸡。
势弱难敌,谓之螳臂当辙;人生易死,乃曰蜉蝣在世。
小难制大,如越鸡难伏鹄卵;贱反轻贵,似学鸴鸠反笑大鹏。
小人不知君子之心,曰燕雀焉知鸿鹄志;君子不受小人之侮,曰虎豹岂受犬羊欺。
跖犬吠尧,吠非其主;鸠居鹊巢,安享其成。缘木求鱼,极言难得;按图索骥,甚言失真。
恶人借势,曰如虎负嵎;穷人无归,曰如鱼失水。
九尾狐,讥陈彭年素性谄而又奸;独眼龙,夸李克用一目眇而有勇。
指鹿为马,秦赵高之欺主;叱石成羊,黄初平之得仙。
卞庄勇能擒两虎,高骈一矢贯双雕。
司马懿畏蜀如虎,诸葛亮辅汉如龙。
鹪鹩巢林,不过一枝;鼹鼠饮河,不过满腹。
人弃甚易,曰孤雏腐鼠;文名共抑,曰起凤腾蛟。
为公乎,为私乎,惠帝问虾蟆;欲左左,欲右右,汤德及禽兽。
鱼游于釜中,虽生不久;燕巢于幕上,栖身不安。
妄自称奇,谓之辽东豕;其见甚小,譬如井底蛙。

父恶子贤,谓是犁牛之子;父谦子拙,谓是豚犬之儿。
出人群而独异,如鹤立鸡群;非配偶以相从,如雉求牡匹。
天上石麟,夸小儿之迈众;人中骐骥,比君子之超凡。
怡堂燕雀,不知后灾;瓮里醯鸡,安有广见。
马牛襟裾,骂人不识礼仪;沐猴而冠,笑人见不恢宏。
羊质虎皮,讥其有文无实;守株待兔,言其守拙无能。
恶人如虎生翼,势必择人而食;志士如鹰在笼,自是凌霄有志。
鲋鱼困涸辙,难待西江水,比人之甚窘;蛟龙得云雨,终非池中物,比人大有为。
执牛耳,谓人主盟;附骥尾,望人引带。
鸿雁哀鸣,比小民之失所;狡兔三窟,诮贪人之巧营。
风马牛势不相及,常山蛇首尾相应。
百足之虫,死而不僵,以其扶之者众;千岁之龟,死而留甲,因其卜之者灵。
大丈夫宁为鸡口,毋为牛后;士君子岂甘雌伏,定要雄飞。
毋局促如辕下驹,毋委靡如牛马走。
猩猩能言,不离走兽;鹦鹉能言,不离飞鸟。
人惟有礼,庶可免相鼠之刺;若徒能言,夫何异禽兽之心。

\chapter{花木}

植物非一,故有万卉之名;谷种甚多,故有百谷之号。
如茨如梁,谓禾稼之蕃;惟夭惟乔,谓草木之茂。
莲乃花中君子,海棠花内神仙。
国色天香,乃牡丹之富贵;冰肌玉骨,乃梅萼之清奇。
兰为王者之香,菊同隐逸之士。
竹称君子,松号大夫。
萱草可忘忧,屈轶能指佞。
筼筜,竹之别号;木樨,桂之别名。
明日黄花,过时之物;岁寒松柏,有节之称。
樗栎乃无用之散材,楩楠胜大用之良木。
玉版,笋之异号;蹲鸱,芋之别名。
瓜田李下,事避嫌疑;秋菊春桃,时来尚早。
南枝先,北枝后,庾岭之梅;朔而生,望而落,尧阶蓂荚。
苾蒭背阴向阳,比僧人之有德;木槿朝开暮落,比荣华之不长。
芒刺在背,言恐惧不安;薰莸异气,犹贤否有别。
桃李不言,下自成蹊;道旁苦李,为人所弃。
老人娶少妇,曰枯杨生稊;国家进多贤,曰拔茅连茹。

蒲柳之姿,未秋先槁;姜桂之性,愈老愈辛。
王者之兵,势如破竹;七雄之国,地若瓜分。
苻坚望阵,疑草木皆是晋兵;索靖知亡,叹铜驼会在荆棘。
王祜知子必贵,手植三槐;窦钧五子齐荣,人称五桂。
鉏麑触槐,不忍贼民之主;越王尝蓼,必欲复吴之仇。
修母画荻以教子,谁不称贤;廉颇负荆以请罪,善能悔过。
弥子瑕常恃宠,将馀桃以啖君;秦商鞅欲行令,使徙木以立信。
王戎卖李钻核,不胜鄙吝;成王剪桐封弟,因无戏言。
齐景公以二桃杀三士,杨再思谓莲花似六郎。
倒啖蔗,渐入佳境;蒸哀梨,大失本真。
煮豆燃萁,比兄残弟;砍竹遮笋,弃旧怜新。
元素致江陵之柑,吴刚伐月中之桂。
捐资济贫,当效尧夫之助麦;以物申敬,聊效野人之献芹。
冒雨剪韭,郭林宗款友情殷;踏雪寻梅,孟浩然自娱兴雅。
商太戊能修德,详桑自死;寇莱公有深仁,枯竹复生。
王母蟠桃,三千年开花,三千年结子,故人借以祝寿诞;上古大椿,八千岁为春,八千岁为秋,故人托以比严君。
去稂莠,正以植嘉禾;沃枝叶,不如培根本。
世路之蓁芜当剔,人心之茅塞须开。

\end{document}