% 墨子
% 墨子.tex

\documentclass[12pt,UTF8]{ctexbook}

% 设置纸张信息。
\usepackage[a4paper,twoside]{geometry}
\geometry{
	left=25mm,
	right=25mm,
	bottom=25.4mm,
	bindingoffset=10mm
}

% 设置字体,并解决显示难检字问题。
\xeCJKsetup{AutoFallBack=true}
\setCJKmainfont{SimSun}[BoldFont=SimHei, ItalicFont=KaiTi, FallBack=SimSun-ExtB]

% 目录 chapter 级别加点(.)。
\usepackage{titletoc}
\titlecontents{chapter}[0pt]{\vspace{3mm}\bf\addvspace{2pt}\filright}{\contentspush{\thecontentslabel\hspace{0.8em}}}{}{\titlerule*[8pt]{.}\contentspage}

% 设置 part 和 chapter 标题格式。
\ctexset{
	part/name= {第,卷},
	part/number={\chinese{part}},
	chapter/name={第,篇},
	chapter/number={\chinese{chapter}}
}

% 设置 chapter 标题格式(古代小说,标题分两行)。
\usepackage{varwidth}
\ctexset{
	chapter/name={第,回},
	chapter/titleformat= \chaptertitleformat
}
\newcommand\chaptertitleformat[1]{
	\begin{varwidth}
		[t]{.7\linewidth}#1
	\end{varwidth}
}

% 图片相关设置。
\usepackage{graphicx}
\graphicspath{{Images/}}

% 设置古文原文格式。
\newenvironment{yuanwen}{\bfseries\zihao{4}}

% 设置署名格式。
\newenvironment{shuming}{\hfill\bfseries\zihao{4}}

% 注脚每页重新编号,避免编号过大。
\usepackage[perpage]{footmisc}

\title{\heiti\zihao{0} 墨子}
\author{}
\date{}

\begin{document}

\maketitle
\tableofcontents

\frontmatter
\chapter{前言}

《墨子》是中国文化中的一部奇书,也是一部寂寞的书。

鲁迅先生说:伟大也要有人懂。而伟大的《墨子》却在中国文化传统中,沉默了两千年,长时间在黑暗中的沉默,不仅影响了对其深层思想的诠释,甚至影响了对其浅层语言的理解,而且,也限制乃至取消了她对中华文化建构的发言权,墨子的思想与精神只好潜伏在中华文化的潜流之中,或沉默,或偶尔嗫嚅着发出微弱的声音。
然而,历史是公平的,一部真正伟大的作品可以暂时寂寞,但她不会永远寂寞,她终究会迎来发言的机会,而且,这一发言必然是黄钟大吕,天下耸动。转机来自于传统文化的变革,西学东渐的历程与新文化运动的勃兴,为古老的中国文化打开了新的视野,新的目光触及到了黑暗中的《墨子》,才惊讶地发现,她原本就焕发着夺目的光彩。
在清末,有一批认识了西方的学者对墨子作出了新的判断。邹伯奇提出了“西学源出墨学”的说法,他认为西方的天文、历法、算学等,都导源于《墨子》,并曾经依墨子的理论做过小孔成像的实验,制造过望远镜与我国历史上最早的照相机。张自牧在论说了墨家科技成就后说
“墨子为西学鼻祖”。王闿运认为《墨子》是西方宗教的源头,如佛家之释迦牟尼、基督教之耶稣都无官位俸禄而被奉为圣师,当受惠于墨学。郭嵩焘认为耶稣视人如己的教义正是墨家兼爱的意思。黄遵宪则从五个方面来论述这一命题:即西方的人权源于墨子的尚同;西方的独尊上帝源于墨子的尊天明鬼;西方的平等博爱源于墨子的兼爱;西学物理发达,源于《墨经》;西学长于器械制造,源于墨学备攻乃至于墨子造纸鸢之术。甚至得出“至于今日,而地球万国行墨之道者,十居其七”的结论……我们并不否认这些说法有“数人之齿而以为富”《墨子·公孟》的心理,但也要承认他们显然拥有了新的目光,并发现了墨子的价值。
在戊戌变法到五四时期,学人逐渐抛开了前者的夜郎心理,但对墨子的推崇却有增无减。《民报》创刊号卷首列古今中外四大伟人肖像,以墨子与黄帝、卢梭、华盛顿并列,被尊为“世界第一平等、博爱主义大家”。梁启超针对当时的国情,提出“今欲救之,厥惟墨学”的口号。爱国志士易白沙说:“周秦诸子之学,差可益于国人而无余毒者,殆莫过于墨子矣。其学勇于救国,赴汤蹈火,死不旋踵,精于制器,善于治守,以寡少之众,保弱小之邦,虽大国莫能破焉。”谭嗣同更为墨子精神的实践者,他不仅深念高望,私怀墨子摩顶放踵之志”而且能舍生赴死,慷慨就义,甘愿成为变革中不可避免的牺牲…
中国历史与中国文化崭新的一页,是伴随着墨子的被重新“发现”而缓缓打开的。

墨子《墨子》与墨家
历史总会给人留下种种的遗憾:对于墨子这样一个伟大的人,我们直到现在却依然所知甚少,甚至连他最基本的姓氏也难以确定。如元代伊世珍在《瑯环记》中引用《贾子说林》,称墨子并不姓墨,而是姓翟,因其母亲分娩前曾梦有乌鸦入室,醒来就生下了墨子,故取名为“乌”;清代周亮工《因树屋书影》卷十亦持此论;钱穆则认为墨子之所以叫墨子,是因为他是受了墨刑的囚徒;而胡怀琛与卫聚贤由认为他是印度人或阿拉伯人。这些奇怪的说法都表明了一个事实,那就是墨子的生平资料太少,其真实的面貌已经被湮没在茫茫的历史沙尘之中。而我们对于墨子的论述,只能依据学术界大体认可的说法来介绍:一般而言,人们认定,墨子姓墨名翟,当为春秋时鲁国人(亦有学者坚持其为宋国人或楚国人),即今山东滕州市。乃是宋襄公之兄的长子目夷的后代,此人因封于目夷,故名目夷子,而且夷原为商朝所建的同姓小方国:即在今滕州市内。
墨子的生卒年也是一个研究界莫衷一是的问题《史记·孟子苟卿列传》说“或曰并孔子时,或曰在其后”可见司马迁也已经不能知道墨子确切的生卒年了。而据学者的研究,大致可以推定墨子生于公元前 480年左右,卒于公元前390年左右。大约相当于孔子逝世后,孟子出生前的时代。
墨子的身份据其《贵义》中的记载,可知地位当较低
微《墨子·鲁问》《韩非子·外储说左上》里均曾记载墨子造车辖的事,后者甚至载其制木鸢,能在天上飞一天:由此可知,他也许曾从事过手工业,而且是一个能工巧匠。
当然,他的一生行事虽然没有明确的文献记载,,但我们从《墨子》一书中便可以看到大概。他与孔子一样,以救世解纷为已任,立说授徒,周游列国。平生足迹所至,曾向北到达齐国,向西到达卫国,多次游历楚国,到过郢都,到过鲁阳。亦曾劝阻鲁阳文君的攻郑,说服公输盘的谋宋等。而且,也多次推荐自己的弟子去做官,以此来推行自己的思想。
《墨子》一书,据《汉书·艺文志》记载,共有七十一篇,然而,现存的《墨子》已然不全了,只剩五十三篇。其中,有八篇有目无文,另有十篇既无目,亦无文。不过,可以知道均当为城守各篇的内容,
据先秦诸子的成书惯例,我们可以推测水墨子》一书也并非墨子一人所作。但是,具体哪些篇目是墨子所作,学术界还颇有歧见,但大体上,比较通达的是任继愈的看法,即从《尚贤》到《非儒》的十一组二十四篇当是墨子当年系统讲解自己的学说,后为弟子记录整理而成的;而《耕柱》、《贵义》等五篇则相当于墨子的语录,都可以当作墨子的著述来看。不过,像墨辩六篇、《亲士》等七篇及城守各篇则或为墨子弟子整理,或为墨家后学记录,如果都
看作墨子的作品也未尝不可。
据《韩非子·显学》记载:“世之显学,儒墨也。儒之
所至,孔丘也;墨之所至,墨翟也。”以此可知,在那个百家争鸣的辉煌时代,墨子所创立的墨家学派声势之浩大,超法逸道而直与儒家相抗衡。
《淮南子·要略》中说,墨子曾学儒者之业,授孔子之术,可见其最初是曾师孔学儒的,但是他对于孔子所主张的繁文缛节极为不满,故另为立说,从而走上了与儒学针锋相对的道路,他在产生之初就有与儒家争衡的意味而且,迅速崛起也当与其时之生产条件与社会关系有关正是在这样的基础上,他的影响越来越大,弟子也日益增多,从而形成了显赫的墨家学派。
《吕氏春秋·情欲》记载,儒家的弟子与墨家的弟子“从属弥众,弟子弥丰,充满天下”,又其《诚廉》中说“孔墨布衣之士也。万乘之主,千乘之君,不能与之争士也”这说明此时的墨家势力很大,然而,墨子弟子的情况却很少见于载籍,孙诒让的《墨学传授考》用尽心力,在《墨子》一书及先秦典籍中才钩沉出三十余人,即使据《墨子》书中所说,也不过有“臣之弟子禽滑鳌等三百余人”的记录:就这一点也可看出,墨家在那个时期其实已经开始衰微了。而在司马迁的《史记》中,不但把孔子列入世家,而且为孔子的弟子单独写了列传,可墨子本人也只有寥寥二十四个字,遑论墨家弟子。此后,墨家的宗教色彩越来越浓重,作为当世之显学的墨家,在秦代焚书坑儒的文化摧残之后,便也宣告衰落,而且,到了西汉,儒家复兴,墨家却未能东山再起。从这时起,一代之显学便成为了千古之绝学,进入了漫长的黑暗之中。

《墨子》的主要内容
墨子》一书的内容极为庞杂,大体可以分为三个部分
首先是体现墨家核心思想的文字:即《尚贤》以下十三篇专题论文中的十大主张,此前的《亲士》七篇所反映的思想都可以在其中找到更详尽与全面的论述,而《耕柱》《贵义》等五篇虽零散,但所论也不出这些主题的龙罩。
这十大主张又大致可以分为四类
一是伦理思想,也是墨子学说的理论基础,即兼爱:墨子认为,当时的整个社会之所以有这么多的问题,如人与人之间的相互残害、家与家之间的相互掠夺、国与国之间的相互攻伐,乃至于君臣间的不忠诚、父子间的不慈孝、兄弟间的不和睦……其最为核心的原因就在于人与人之间没有一种无差等的爱,如果人们能够做到兼爱:那么就会“强不执弱,众不劫寡,富不侮贫,贵不敖贱,诈不欺愚,凡天下祸篡怨恨,可使毋起”,从而达到天下的大治。而如果世人都兼爱了,就会互利互惠,并因此而达到非攻;因兼爱天下百姓而讲节用、节葬和非乐,并用天志说来限制人的浪费;以明鬼为推行兼爱的手段,并打破天命论对于兼爱的阻碍。
其实,儒家也是讲“爱”的,所谓“仁者爱人”即为此意,但儒家的爱是以“亲亲”为基础的,是有差别的,而墨子的兼爱却是无差等的爱,是所有的人之间互相平等的
爱。虽然,也许我们会觉得这种理论空想成分过多,但是,却也不得不承认,爱,永远是人类烟水苍茫的历史长河中熠熠闪烁的粼粼波光。

二是政治思想:即尚贤、尚同、非攻。一个社会的政治状况虽然受生产力状况与社会发展状况的制约,但是,统治者与各级当政者的个人品质及特点也无疑是其中最为重要的因素。所以,一种政治体制,其最为核心的政治活动便是官吏的选拔。而墨子所认定的“为政之本”就是尚贤,他的尚贤极为彻底,打破了封建社会的等级观念,唯贤是举。仅此而言,其思想之高远与宏达已远远超出同时代的思想家。更何况墨子在此篇及后边的《尚同》篇中也隐约表达出帝王也当由此途径而出的意思,这更是石破天惊之论,有人把他当作西方民主政治的前源也不足为怪。当然,此后墨家学派之所以从显学而变为绝学,这也是其重要原因之一,因为这从根本上危及了统治者的地位。
尚同则是要讨论下级对上级的服从。墨子认为一里之人要统一于里长,一乡之人要统一于乡长,一国之人要统一于国君,而天下之人要统一于天子,正是在这样的政治幻想中,墨子把全天下组织成了一个纲举目张、有条不紊的系统。只要能够达到以上级的是非为是非,就会统一而不会产生混乱,这一主张也反映出墨家理想化而又简单化的大同愿望。当然,墨子也考虑到了这种主张理论上的漏洞,所以,要联系他的尚贤论与天志论来理解:从兼爱观念出发,在政治思想上,墨子还极力主张非
攻。我们知道,在墨子生活的春秋战国时期,也恰是中国历史上战争最为频繁的时期,而墨子不仅从他的理论基石--兼爱出发,也从当时的社会现实出发,充满愤怒地论述了攻国之不义,并以层层深入的比喻来论证“窃钩者诛,窃国者侯”的荒谬。不过,我们还应当看到,墨子并非迂腐的说教者,他对春秋战国时期的现实极为清醒,他知道只凭借道德上的良好愿望与自律幻想是不可能阻止战争的,所以,与他非攻相辅而行的还有他卓越的军事主张。
三是经济思想:即节用、节葬、非乐,
其实,如果可以脱略主张的具体内容而只抽象看待的话,墨子的主张中,最有永恒意义并在每个时代都有可行性的便是节用。这其实也是他经济思想的核心。而就墨子所处的时代而言,节用的主张亦更显得重要,当时的生产力水平比较低下,人类所能创造出来的生活物资较少,提倡节约在某种程度上就相当于在创造价值,基于此,墨子认为,人类所有的消费,都应该以满足最为基本的自然需求为限,如食能果腹,衣可御寒,杜绝-切无益实用的消费。其实,这也是针对儒家的各种繁琐规定而发的。
节葬算是节用的一个分支,不过,儒家厚葬久丧之礼过于不切实用,但却流风所及,遍被士林,所以墨子将此单独提出详为论列。儒家的厚葬久丧在墨子看来,是完全没有必要的浪费。所以,墨子针锋相对提出节葬的主张,对于保存当时社会的生产力、增进社会财富而言,是极有意义的。而且,相对于儒家的主张,墨子所说的“衣食者,人之生利也,然且犹尚有节;葬埋者,人之死利也夫何独无节于此乎”,显得如此剀切而通达。
非乐其实是节用的外化。当然,从其行文中可以看出,墨子并非不能欣赏音乐的美,他的这一主张其实有很深远的考虑,那就是在当时的社会生产力条件下,王公大人对于声乐之美的追求,只会造成“亏夺民衣食之财”的后果。这不但是当时社会物质生产极端匮乏下的一种无奈之举,也是墨子对于当时社会的两极分化的一种批判。因为,统治者在衣食无忧的情况下沉酒声色,但这种行为却是以民众的牺牲为代价的。
四是宗教思想:即天志、明鬼、非命。
如果说兼爱是墨子从人世间筛选出来的理论基石和核心的话,那么,天志观则是墨子思想的原动力,是逻辑起点。他认为,上天是有意志的,而其意志主要表现为“天欲义而恶不义”和“天之爱天下之百姓”。其之尚同、兼爱、非攻等篇的推理无不以此为起点,而尚贤、节用、节葬也都通过圣王而间接源于此,
明鬼的论点也体现出墨子以唯心主义的外壳来装饰其改造社会的良苦用心,他不过想借此来整顿社会秩序。

他天真而且很可爱地设想,如果所有的人都能相信鬼神可以施福降灾、赏善罚恶,从而为全社会产生一种共同的约束力,就能达到天下大治。但他根本没有想到,这个说法本身已经暴露了他对于鬼神存在的怀疑。
非命的观点是在与儒家的争辩及社会生活的实践中
提出的:儒家的“生死有命,富贵在天”对于广大的民众而言是一针麻醉剂,也是墨子所说的“繁饰有命以教众愚朴之人”的阴谋;同时,天命思想在社会生活中也体现出其消极的特点,对于人类的创造性有深深的损伤。而墨子在社会生活中是一个态度积极的人,他认为,所有的事情,之所以做得好,是因为个人的努力,只有每个人都尽力了,社会才会发展。在后边的《鲁问》中,记载了墨子与其弟子彭轻生子的一段对话,就可以看出墨子对于人类自己努力的自信,这也正是人类能以自在的状态生存在这个世界上所必须的强烈自信。
其次是《墨经》所包含的与社会科学乃至于自然科学有关的知识。这一部分内容十分复杂,仅以谭戒甫的《墨经分类译注》为纲,即可分出十二种学科门类,何况此书并未包括《大取》《小取》二篇。《墨经》代表了先秦时代在各个学科所取得的成就,有许多成就的取得令人极为惊讶。如其在自然科学上所取得的成就,杨向奎曾评价说:“-部墨经,无论在自然科学哪一方面,都超过整个希腊,至少等于整个希腊。”
第三类是其军事思想。墨子十大思想中最为主要的是“兼爱”和“非攻”,但是,墨子并非当时以为礼乐便可安国的腐儒。对于当时的社会状况,墨子是极为清醒的,他清楚地知道,反对攻伐,仅仅靠道义的感召与理论的说服远远不够,正如鲁迅所说:“一首诗吓不走孙传芳”,所以,一个和平主义者,也要有坚强的力量来作为和平的保障乃至于砝码。因此,墨子》自《备城门》以下,全是有关军
事的内容,这些篇目从某种程度上可以看作是一部杰出且实用的“墨子兵法”
三对《墨子》的研究
从墨家以清新嘹亮的声音加入百家争鸣的大合唱时,就开始有人对其进行了研究。如孟子指责墨子兼爱的主张是“无父”,故诋之为“禽兽”,但也承认其“摩顶放踵利天下”的行为;苟子批评墨子“敝于用而不知文”;庄子在其《天下》篇中,论述墨子“意则是,其行则非”,然而,也充满同情地说墨子“真天下之好也,将求之不得也,虽枯槁不舍也,才士也夫”!汉代司马迁父子、王充、班固等人也均对墨子发表了意见。
直到墨子之后大致五百年,西晋学者鲁胜曾对《墨子》中的《墨辩》四篇进行了注释,此书是中国历史上可知的最早的《墨子》注本,虽然,此书现已佚,但其序还保存在《晋书》中。此后又经过了四五百年,在唐代产生了乐台的注本,但也早已失传。
唐代以继儒家道统为已任的韩愈曾写过《读墨子》-文,其文竟有“孔子必用墨子,墨子必用孔子,不相用不足为孔墨”之语,其实是有深远的考虑的,清代学术大师俞樾有“乃唐以来,韩昌黎外无一人能知墨子者”之语,亦得其实。
不过,对《墨子》真正意义上的研究,是从清代开始的。清代初期,傅山做《墨子大取篇释》,虽仅对《墨子》中的《大取》一篇进行训释,但却成为清代墨学复兴的第-箭阳光。此后,文学家汪中曾用六年时间校注《墨子》,当有所获,可惜其书却未能流传,而据其所流传下来的《墨子序》与《墨子后序》可以看出,他“不但为墨子辨千古之枉曲,而且把儒墨显学并称的历史首先指示出来,一扫千年来异端的诬蔑”侯外庐语)。几乎同时的毕沅在几千年的历史中,第一次对《墨子》全文进行了认真而富有成效的注释与读解工作,其十六卷的《墨子注》也成为《墨子》整理史上承前启后的力作。到了清代后期,终于产生了《墨子》整理史上空前的巨著:孙诒让的《墨子间诂》。此书以毕沅的《墨子注》为蓝本,以清代四十余家研究墨子的著作为参照,详为推考,以数十年功力,成此两千年墨学研究的集大成之作。梁启超在《中国近三百年学术史》中评价说:“大抵毕注仅据善本雠正,略释古训;苏氏始大胆刊正错简;仲容(即孙诒让)则诸法并用,识胆两皆绝伦,故能成此不朽之作。……其《附录》及《后语》,考订流别,精密闲括,尤为向来读子书者所未有。盖自此书出,然后《墨子》人人可读。现代墨学复活,全由此书导之。古今注《墨子》者固莫能过此书,而仲容一生著述,亦此书为第一也。”
据统计,清代大致有六十种墨学研究专著,而现代的三十年就产生了大约一百种,数量激增,研究的质量也很高。就拿全书整理本而言,就出现了两部极有特点的全注本。一是张纯一的《墨子集解》,此书为作者积十数年之功写成的,他吸收了孙诒让《墨子间话》未及收入的成果及其产生后问世的成果,并能参以已意,时有新说,于
句意、段意和篇意有通达的解说与发挥,虽然校勘粗略但材料宏富,解说尤详。二是吴毓江的《墨子校注》,此书最大的功绩在于校勘,作者积二十年之功,对于现存的古代《墨子》版本,几已网罗无遗,共用一种唐本、十四种明本、两种清本,此外还多从类书与古注中搜集引文以作比证,进行了孙诒让、王念孙诸人所未措意的文字校勘问题,全书后也有丰富的附录资料。而岑仲勉的《墨子城守各篇简注》则生面别开,以《孙子兵法》为背景来评价墨子城守各篇的军事价值,并能结合后世器具实物、古代兵书与兵图、古代战例来解释书中的各种器物。
建国以后,墨子的研究更是蓬勃开展,据统计约有二百种研究专著问世。其中,在《墨经》的整理方面有谭戒甫的《墨辩发微》与《墨经分类译注》、高亨的《墨经校诠》等。而在《墨子》全书文本的训释上,王焕镳耗十年心血而成百万言的《墨子集诂》成为《墨子》笺注史上又一部集大成的巨著,其书除去了《墨辨》六篇与城守各篇,仅释所余之三十六篇,以孙诒让《墨子间诂》为底本,并参照其所能搜罗到的诸家意见,择善而从,间出己意,不但是《墨子》整理史上引书最多的一家,而且,在校释上也多有发明。
本书的整理情况四
本书为《墨子》的一个选本。但由于《墨子》一书的特殊性,此选本在某种程度上亦有全本之功。如《墨子》卷-从《亲士》到《三辩》的七篇及《耕柱》《贵义》等五篇
都全部入选,不做删节。而中间从《尚贤》到《非儒》的十一组中,每组均当有三篇文章,内容基本相同,甚至措辞与事例都极相近,故被学者认为是“墨分为三”后弟子传述不同的结果。因此,这十一组文章,仅选其最为完整明晰的一篇。当然,也有个别例外,如《非攻》,本来《非攻中》论述更为全面,但《非攻上》之行文简洁严谨,层层设喻,是最典型的墨子文章,故舍“中”而选“上”;再如《节葬》、《明鬼》、《非乐》《非儒》四组都各仅存一篇,故无可选择。《墨经》内容庞杂,涉及了许多专业知识,且研究界也歧见迭出,故仅选其与光学有关的八条,以见一斑而已。城守各篇亦多涉及防守的方法与器械,疑晦难明处很多,故仅因《公输》之云梯而选《备梯》一篇以尝鼎一脔。
本书以王焕镳的《墨子集诂》为底本,因其书为集解性质,故所收极为丰富,一册在手,众善毕集,可以参照诸家,择善而从。而且,王之按断亦多精义,许多前人莫衷一是的问题,他都有别具手眼的考论。当然,也有个别地方似未得当,则参酌吴毓江《墨子校注》或张纯-《墨子集解》甚至孙诒让《墨子间诂》正之,偶尔也参以已意。此外,谭家健的《墨子选译》严谨而得当,故亦有所取资。关于正文,一般情况,以底本为主,不作改动,以尊重原貌。
有两种情况则在正文上直接改正而不作说明:-、王焕镳《墨子集诂》乃以孙诒让《墨子间诂》为底本,若孙本印误,王仍存其旧,仅在注中说明者,如此则改之;二、孙本又以毕沅《墨子注》为底本,毕本有不少无意之舛误,孙
本与王本仍之而未改者,亦改之。另有一种情况,即正文文字明显有误,研究者亦多指出者,如有确切的版本依据,则改动正文,并于注中说明。如果有研究者对一些文字有新的看法,虽近真却无版本依据者,不改动原本,只在注中说明当改为某某,译文也以校改后的文字为准。
关于注释。本书的注释尽量简明,一般通过译文可以了解的字词,便不再作注;一般性常识,如墨子常常引及的古代贤君与暴君的事例,在第一次出现时加注,其后则不再加注,个别前详后略,以助阅读
故本书之注约有五种:一是难字需注音者,二是难理解的字词与文化常识性的内容,三是难理解的语句需串释者,四是通假字,五是校改说明。
关于译文。为忠于原文,本书译文以直译为主,同时也尽量做到晓畅通达。而且,一些字词并未设注,实在译文中已有所体现;而个别语句极为复杂者,译文仍以直译为之,难解之处则在注中说明,以助理解,并使注与译可以交相为用。
此外,全书正文对话与引用极多,层次过于繁复,为避淆乱,兹从王焕镳书不用引号标识,译文亦从此。其实,全书在各段对话与引用前均有相应提示语,不用引号亦可明白,如用引号,反徒生滋扰而已
最后,此书的完成,除前文所及外,还借鉴了墨学研究界众多学者的研究成果,如任继愈《墨子》杨俊光《墨
子新论》、邢兆 良《墨子评传》、苏凤捷、程梅花《平民理想--〈墨子〉与中国文化》郑杰文《20 世纪墨学研究史》等;而且亦幸得张廷银老师的帮助与指导,在此深表谢忱。

春秋战国是中国古代社会从宗法贵族制向官僚地主制过渡的大变动时 
代。作为这种社会大变动的表现和结果,在当时涌现出许多思想主张互不相 
同的学派。其中影响最大的有二:一为孔子开创的儒家,一为墨子开创的墨 
家。它们在战国时期并称为当世的显学。《韩非子·显学》篇说:“世之显 
学,儒墨也。” 
墨子名翟,鲁国人。生卒年具体不详。但从历史文献来看,我们可以断 
定,墨家的产生当在儒家之后。据《淮南子·要略》之说,墨子原为儒门弟 
子,后因不满儒家学说而另创一对立的学派: 
墨子学儒者之业,受孔子之术,以为其礼烦扰而不说,厚葬靡 
财而贫民,(久)服伤生而害事,故背周道而用夏政。 
由此看来,墨家学说乃是墨子对儒家学说进行反思和批判的产物。从我 
们今天所能见到的《墨子》一书确实不难看到,墨家学派对儒家从周代贵族 
社会继承下来的礼乐等文化形式进行了大量的攻击,如《墨子》书的《节葬》、 
《节用》、《非乐》、《非儒》等,都可以说是直接针对儒家学说而发。因 
此,《淮南子》的论断并非无稽之谈。当然,应当指出的是,《淮南子》的 
作者把墨家学说的兴起归之为夏政的复活,则有失于简单。诚然,在《墨子》 
书中,夏禹被塑造成一位与儒家所宣传的礼乐文化背道而驰的古代圣王。但 
是,这个形象并不是古代历史的客观反映,而主要是墨家理想的象征。墨家 
要借助夏禹来压服儒家所声称的祖师爷文王、周公。事实上,在《墨子》书 
中,夏禹、商汤、文王都是被列为古代圣王的人物,并不是相互对立的。因 
此,与《汉书·艺文志》的百家出于王官说一样,《淮南子》的墨家“用夏 
政”说也是一种想当然的皮相之见。 
墨家作为一个与儒家对立的新生学术政治团体而出现在儒家声势浩大 
之时,它不仅站稳了脚跟,而且获得了与儒家平分秋色、甚至后来居上的地 
位,这用复古说恐怕是无法解释的。墨家与儒家并称为显学。所谓显学,包 
括两个方面的含义,一是队伍壮观,声威显赫,一是仕途通达,君主信任。 
而要做到这两点,它就必须让统治者和被统治者都可以从它的学说中看到对 
自己有利的东西。而要做到这一点,关键在于和历史前进的方向达到一致。 
墨家之所以能够在战国前期异军突起,其原因即是它比儒家更能抓住战国初 
期社会发展的新形势,提出了一些儒家所没有提出的社会学说和政治方案, 
从而引起了当时自君主到庶民等阶层的强烈兴趣。 
儒家诞生的春秋时代,官僚地主制与宗法贵族制两种新旧社会制度的交 
替尚处于一种潜在的温和状态。宗法贵族集团仍处于社会的统治地位。因此, 
作为新生社会力量的代表,孔子的思想虽然已超越宗法贵族时代而进入到官 
僚地主时代,但孔子的新思想却依然披着贵族社会旧文化的外衣。他希望通 
过对贵族文化进行输血式的改造而促成社会制度的变革。这样一来,孔子的 
思想便不可避免地带有温情主义和维新主义的色彩。一方面,孔子虽然主张 
贤人政治和平民参政,但他并不想冲击贵族阶级的既得利益,仍希望“贵可 
以守其业”;另一方面,孔子虽然以新的社会理想对神、礼、德等贵族文化 
的核心内容进行了超越和改造,但他并没有提出一套全新的政治方案,他所 
追求的有道之世乃是一个十分模糊的概念。而且,由于孔子把道德价值强调 
到独一无二的至高地位,将理想的实现寄希望于执政者的道德自律,这就使 

他的思想又带有强烈的理想主义色彩。孔子去世以后,新旧社会制度交替的 
潜在状态突然被打破。在先后左右着春秋时代的政治局势的齐、晋两个大国, 
代表新兴的官僚地主社会方向的田氏与韩、赵、魏三家分别取代原先由周天 
子所分封的齐、晋诸侯而建立了新的政权。在春秋时代尚处于统治地位的传 
统宗法贵族文化终于退出了历史舞台。明、清之际的著名学者顾炎武在《日 
知录》中对孔子死后百余年间的这种历史剧变曾有一精彩概括: 
自《左传》之终以至此(指周显王三十年),凡一百三十三年, 
史文阙佚,考古者为之茫昧。如春秋时犹尊礼重信,而七国则绝不 
言礼与信矣。春秋时犹宗周王,而七国则绝不言王矣。春秋时犹严 
祭祀、重聘享,而七国则无其事矣。春秋时犹宴会赋诗,而七国则 
不闻矣。春秋时犹有赴告策书,而七国则无有矣。邦无定交,士无 
定主,此皆变于一百三十三年之间。史之阙文,而后人可以意推者 
也,不待始皇之并天下,而文武之道尽矣。(《日知录集释》卷十 
三) 
总之,进入战国时代以后,周代贵族社会的各种制度全部被破坏。在这 
种新的历史背景下,孔子那种以宗法贵族文化的旧瓶装官僚地主社会之新酒 
的维新做法,无疑是落后而跟不上形势了。因此,结合贵族社会行将灭亡这 
种新的历史环境而对儒家学说进行反思和改造,又成为智识阶层所面临的一 
项历史使命。墨家学说即由此应运而生。 
从历史的眼光来看,就今天所见到的《墨子》而言,墨家学说比儒家更 
能符合战国时代社会发展的内容,至少表现在如下几个方面: 
首先,墨家学派明确提出了兼爱、尚贤的平民政治理论,把孔子提出的 
爱人和举贤思想推向了一个更新的高度,从而在理论上彻底打破了贵族阶级 
以亲亲为原则的血缘贵贱论。儒家创始人孔子为了给平民阶级争取更多的政 
治权力,在春秋末期提出了爱人(仁者爱人)和举贤的思想。这对于贵族阶 
级的血缘贵贱论来说,无疑是一个具有新的时代精神的创举。但是,由于当 
时宗法贵族势力仍然处于统治地位,平民阶级大规模地向贵族阶级争取平等 
权力的运动刚刚形成气候,因此,在孔子所具有的政治思想中,还不可能产 
生彻底打破贵族特权的认识。他所做的只能是要求贵族阶级将其特权向平民 
开放。要求贵族统治集团将普通民众当作与自己同样的人看待的爱人思想与 
要求贵族阶级从普通民众中选拔贤能参政的举贤思想所体现的即是这样一种 
历史特点。在主张爱人和举贤的同时,孔子并不反对对贵族阶级的利益予以 
照顾,部分地保留其特权,讲究“故旧不遗”。而且,平民阶级要获得“举 
贤”的机会,在事实上还必须经过一个贵族化的过程,先掌握以诗、书、礼、 
乐为代表的贵族文化。 
然而,在墨子的时代,平民阶级争取与贵族阶级平等的政治权力的斗争 
已经基本取得胜利,孔子所提出的爱人和举贤思想完全成为现实。因此,作 
为新时代的平民思想家,墨子必然要提出比孔子更为激进的平民革命思想。 
于是,孔子的爱人和举贤便被兼爱和尚贤所取代。墨子的兼爱与孔子的爱人 
之区别在于:爱人并不否定亲亲;而兼爱则实际上取消了亲亲,主张将他人 
与自己的亲人一样看待。《墨子·兼爱下》云: 
姑尝本原若众害之所自生。此胡自生?此自爱人利人生与?即 
必曰非然也,必曰从恶人贼人生。分名乎天下恶人而贼人者兼与别 
与?即必曰别也。然即之交别者,果生天下之大害者与!是故别非 

也。子墨子曰:“非人者,必有以易之,若非人而无以易之,譬之 
犹以水救火也。”其说将必无可焉。是故子墨子曰:“兼以易别。” 
然即兼之可以易别之故何也?曰:藉为人之国,若为其国,夫谁独 
举其国,以攻人之国者哉?为彼者由为己也。为人之都,若为其都, 
夫谁独举其都,以伐人之都者哉?为彼犹为己也。为人之家,若为 
其家,夫谁独举其家,以乱人之家者哉?为彼犹为己也。然即国都 
不相攻伐,人家不相乱贼,此天下之害与?天下之利与?即必曰天 
下之利也。 
在墨子的认识中,兼与别相对而言。所谓别,从政治的角度而言,无疑 
指的是周代社会以血缘和种姓为依据而确立的各种等级关系,也即周公制定 
的那种礼乐制度,而兼则是要废除这种礼乐等级制度,消除这种嫡庶亲疏观 
念。应该说,这即是兼爱的历史本质。 
血缘亲疏关系被彻底抛弃以后,贵族阶级凭藉出身而高处显贵地位的世 
官制度已完全没有存在的理由。因此,在墨子的思想中,一视同仁地从全体 
国民中选举贤能便成为唯一的仕官途径。只有贤能才是唯一有资格入仕为官 
和受人尊敬的人。而且,在墨家学派这里,贤能之士的入仕为官已不再需要 
经过贵族化的修养准备。如《墨子·尚贤上》云:“故古者圣王之为政,列 
德而尚贤。虽在农与工肆之人,有能则举之。高予之爵,重予之禄,任之以 
事,断之以令。”主张直接从农夫与工商小民中选贤任职,并举例云: 
故古者尧举舜于服泽之阳,授之政,天下平。禹举益于阴方之 
中,授之政,九州成。汤举伊尹于庖厨之中,授之政,其谋得。文 
王举闳夭、泰颠于罝罔之中,授之政,西土服。故当是时,虽在于 
厚禄尊位之臣,莫不敬惧而施;虽在农与工肆之人,莫不竞劝而尚 
意。 
舜、益、伊尹、闳夭、泰颠等人本来都是从事各种卑贱职业的体力劳动者, 
但却被尧、禹、汤、文王等最高统治者分别拔举到国家最高行政长官的地位。 
这就是墨子所理解的尚贤。 
与孔子那种学而优则仕的举贤思想相比,墨子这种从劳力者中选拔劳心 
者的观点无疑更为受到普通民众的欢迎。 
其次,墨家学派明确提出了尚同和如何成为天子的问题,把孔子“为东 
周”的梦想提到了改朝换代的高度,为建立一个取代周王朝的新的统一的中 
央集权王朝提供了舆论基础。 
在孔子所生活的春秋时代,周王朝虽然已经日薄西山、奄奄一息,但周 
天子在名义上仍然是诸侯们所承认的天下共主。在激烈的争霸斗争中,尊王 
一直是霸主们争取霸主地位的手段。为了让其他诸侯国承认自己的霸主地 
位,霸主们总要做出一些维护周王朝和周天子之体面的行动。所以,孔子虽 
然认识到应当创立一套超越周代贵族文化的地主官僚文化,但却还没有形成 
建立新王朝的明确的革命思想。他在周游列国的时候,还没有和诸侯们讨论 
如何为王、为天子的问题。 
到墨子时,随着周王朝地位的日趋下降,实力雄厚的诸候已不再满足于 
做诸侯之长,而是希望取代周天子而成为诸候之王。争夺霸主地位也不再成 
为他们的目标。以此为背景,墨家学派便提出了王、天子这样一些孔子尚未 
提出的时代主题。例如:《墨子·亲士》云: 
圣人者,事无辞也,物无违也,故能为天下器。是故江河之水, 

非一源之水也;千镒之裘,非一狐之白也。夫恶有同方取不取同 
而已者乎?盖非兼王之道也!是故天地不昭昭,大水不潦潦,大 
火不燎燎,王德不尧尧者。 
从王道的高度来研究政治,是《墨子》一书的特点。又如《墨子·尚贤 
中》云: 
今王公大人欲王天下、正诸侯,夫无德义,将何以哉?其说将 
必挟震威强。今王公大人将焉取挟震威强哉?倾者民之死也!民 
生为甚欲,死为甚憎。所欲不得,而所憎屡至。自古及今,未有 
尝能有以此王天下、正诸侯者也。今大人欲王天下、正诸侯,将 
欲使意得乎天下,名成乎后世,故不察尚贤为政之本也?此圣人 
之厚行也。 
如此明目张胆地用王天下、正诸侯来游说当世王公大人,恐怕是春秋时代的 
孔子所不敢做的。 
从《墨子》全书来看,墨家学派所提出的王天下、正诸侯的途径主要包 
括兼爱、尚贤、非攻、尚同等等。这基本都是从广大被统治者的利益着眼的。 
对当时的君主来说,这种理想主义的政治显然是无法付之实践的。但是,对 
当时这些纷纷自称为王的君主们来说,墨家的这种王天下、正诸侯的说法无 
疑比儒家的尊王、复礼之论悦耳动听得多。 
第三,墨家学派明确提出一种功利主义的政治哲学,这比儒家的道德政 
治更为符合统治者选拔人才的心理和任用人才的原则。例如,《墨子·亲士》 
篇云: 
故虽有贤君,不爱无功之臣;虽有慈父,不爱无益之子。是故 
不胜其任处而其位,非此位之人也;不胜其爵而处其禄,非此禄 
之主也。 
墨子说国君不会喜欢无功之臣,慈父不会喜欢无益之子,其目的固然在于攻 
击贵族集团的尸位素餐,但与此同时,他在这里实际上提出了一种具有划时 
代意义的人才思想。 
如前所述,在周代贵族社会,贵族阶级步入仕途依靠的是自己的出身地 
位。孔子为了为平民阶级的入仕创造理论条件,便提出了以道德为标准来选 
择人才的思想。这在当时无疑有着巨大的历史意义。到墨子的时代,新一代 
的君主已基本接受从各阶层中选拔人才的观念,但是,在当时激烈的竞争斗 
争中,君主们所需要的并不是孔子所推崇的道德君子,而是能为国家建功立 
业的谋臣策士。因此,墨子以功代替孔子的德来评价人才,自然更符合最高 
统治者的口味。 
与这种功利主义的人才观相应,墨家学派的思想也普遍具有功利主义色 
彩。墨家在论述其各种观点的时候,往往要从功利的角度论证其必要性。他 
们认为人民的本性无常,只对于他们有利的人、事感兴趣。《墨子·七患》 
云: 
故时年岁善,则民仁且良;时年岁凶,则民吝且恶。夫民何常 
此之有!为者疾,食者众,则岁无丰。 
既然民性随生活环境而变迁,没有恒性,那么,孔子所主张的道德教化政治 
也就无从附丽了。墨子还明确指出,道德的力量是有限的,只能从属于物质 
力量。《七患》云: 
故仓无备粟,不可以待凶饥;库无备兵,虽有义不能征无义; 

城郭不备全,不可以自守;心无备虑,不可以应卒。 
因此,他认为,一个国家最重要的东西即是备、兵、城等物质基础: 
故备者,国之宝也;兵者,国之爪也;城者,所以自守也: 
此三者,国之具也。 
与儒家的那种重义轻利的道德高调相比,这种现实主义的态度与战国初期的 
君主对政治的认识自然一致得多。 
在墨家的政治思想中,以这种功利主义和现实主义的认识为基础,他们 
还提出了两个在战国时代具有重大影响的政治术语——“法”和“术”。在 
《尚贤中》一篇,墨子指出,治理国家必须讲究法术,说:“既曰若法,未 
知所以行之术,则事犹若未成。”法指的是治理国家必须坚持的基本原则, 
而术则是贯彻这种原则行之有效的方法。在本篇中,法和术实际上具体指应 
当任用贤才和如何使用贤才二者。他认为若能做到这二者,统治者即可收到 
“美善在上,而所怨谤在下;宁乐在君,忧戚在臣”的效果。这种重法明术、 
尊君卑臣的政治观点,无疑直接开启了战国时代法家思想的先河。 
第四,墨家还具体提出了非攻、节用、节葬、非乐等政治主张,对当时 
统治者贪得无厌的掠夺战争和穷奢极侈的享乐生活进行了广泛而尖锐的批 
判。这无疑喊出了处身于无休无止的战争之中并担负无穷无尽的租赋徭役的 
人们的心声。 
总而言之,从上述四个方面来看,墨家学说可以说是从战国时代的政治 
形势出发,站在普通民众利益上提出的一套系统的政治理论。当然,与此同 
时,我们还应当进一步看到,由于墨家学派的成员大都是一些出身下层阶级 
的“贱民”,他们本身并不具备多少政治经济地位,因此,当他们通过游说 
当时君主来推行自己政治主张的时候,他们就不得不借助于一些超人间社会 
的力量来维护自己的权威地位。由此出发,他们对上天和鬼神进行了大量宣 
传,把自己的主张说成是上天、鬼神的愿望。这就使墨家学说具有浓厚的迷 
信色彩。 
从散文艺术的角度而言,《墨子》在先秦哲理散文中以质朴无文著称, 
这一特点极为显著。墨家主张尚质,反对尚文。因此,他们著书立说所采用 
的是当时的口语,而不是儒家那种经过修辞的“文言”或“雅言”。这使他 
们的文章有一种平易近人、娓娓道来的风格特色,但因此也给后人的阅读和 
理解带来了一定的困难。我们在对《墨子》进行白话翻译的时候,对这一种 
困难感触尤深。特别是《墨子》中《备城门》以下诸篇,有许多当时的战略 
术语,古来注者向无确解,我们虽勉强为之转译,但恐怕未必符合作者的原 
意。至于《经上》、《经下》、《经说上》、《经说下》诸篇,乃是墨子及 
其弟子对逻辑学、数学、物理学等专门知识的研究成果的总结,文字记录具 
有隐微难懂、言此意彼之特点,若强为之翻译,只能弄巧反拙,因此,我们 
只录了原文。读者若对这部分有兴趣,可参考谭戒甫的《墨辨发微》与《墨 
经分类译注》。 
《墨子》一书,据《汉书·艺文志》记载,有七十一篇,现存五十三篇。 
古人对此书的整理研究工作始于清代。较为著名的成果有毕沅的《墨子注》 
与孙诒让的《墨子闲诂》等。我们即以二书为主而旁采其他诸家之说。本书 
的译注工作由多人合作完成。吴龙辉负责《亲士》至《明鬼》篇,及《备城 
门》篇,过常宝负责《非乐》至《非儒》篇,张宗奇负责《大取》至《公输》 
篇,黄兴涛负责《备高临》以下诸篇。由于我们水平所限,缺点错误定会不 

少,则有待于方家与读者的教正了。 
吴龙辉 一九九二年六月 


序

这套“先秦诸子今译”丛书,从时间上看,正赶上由《资治通鉴》白话 
本出版而激起的古文今译热潮。既是“潮”,那就该归为“显学”,这个名 
称总是不大入耳的。而且,在有的人看来,将典雅古奥的国粹糟塌成浅俗不 
堪的白话,无异于挖掘祖坟,粗鄙无道。只是这潮仍不可阻遏地热起来了, 
起码说,还有许多读者喜爱这种“下里巴人”的东西。我常想,既然老祖宗 
的东西各个时代都有人注,而且注得好的都成了大师,拿不大准的就多多益 
善地收罗先人的话充数,号称什么经“注我”;甚至自己不“注”一字,尽 
得风流,达到了“大美不言”的化境,不但免遭物议,反为同行相与乐道。 
那么,今天我们译成大白话,不妨也可以冒充成一种注罢。当然,大潮一起, 
免不了泛些泥沙残渣,恰如这套丛书免不了多有注译上的错讹一样;但潮落 
之后,大浪淘沙,或者会有精妙之作显露出来。 
先秦诸子的时代,在我国历史上是读书人人格相对独立,思想最活跃、 
少束缚的时代,也是一个异彩纷呈、硕果累累、最为辉煌璀璨的时代。可以 
说,这个时代奠定了中国文化的基础。组成我们民族文化核心的儒、道、释 
三大思想宝库,就有两家半(因为佛教也中国化了)兴起于先秦。可惜自那 
以后,中国历史上就再没有重现过同样令人激动和向往的“百家争鸣”的自 
由壮观的局面。先有暴君秦始皇因惧惮思想的伟力而“坑儒”,继以汉武帝 
为了“役心”的需要,采纳最长于给同类致命一击的董仲舒“罢黜百家,独 
尊儒术”的建议,百家终竟只尚一家,儒家变成了儒教。更可怕的是其后近 
两千年,儒教与封建政体结合,形成政教合一的形态,大大方便了统治者“动 
口”不行就“动手”,思想“教育”不奏效就施虐于肉体。于是,创造被扼 
杀了,“万马齐喑”成为不争之实。今天,欣逢大力提倡“思想再解放一点” 
的盛世,我们着手先秦诸子白话今译的工作,也是奢望以绵薄(精神的东西 
毕竟不如物质的来得直捷快当,此之谓“绵”;学养太浅,无能传其精髓达 
其要义,此之谓“薄”)之力,让更多的人了解我们祖国曾有过的光辉时代, 
让更多的人歆享我们祖先创造的精神文明,让更多的人汲取菁华、走出蒙昧, 
为中华的复兴增添一分力量! 
一个时期,反传统文化成为时尚。有的人动辄对传统文化大加挞伐,仿 
佛民国初闹革命,以为只要“咔嚓”一声将辫子剪掉,耳濡目染、浸淫五腑 
的封建污秽也随之而去。类似的“战斗”,从来没有成功过。“五四”时力 
倡“打倒孔家店”,现在不但没倒,香火还甚于从前。还有人辩护说那样做 
是为了“矫枉过正”,这不禁使人回忆起物质生产一“过正”就诞生“大跃 
进”的教训,我想精神文化的建设也不会例外。当然,我们并不认为“传统” 
就是十全十美的(持此谬论者也大有人在)。只是,既然“传”诸后代而成 
为“统”,那就有它的合理性和它的生命力。传统文化固然与具体的时代和 
政治有千丝万缕的联系,我们甚至无法弄清是它在规定政治,还是政治常常 
要利用它,但是,传统文化绝不等同于它们,它是更趋于永恒的东西(如果 
不是伪文化)。一个时代结束了,一种具体的政治体制被更进步的取代了, 
几千年生生不息的传统文化精神可以增添新鲜血液,可以芟除与生俱来或在 
时间长河中衍生的赘物,但绝对无法结束它和取代它!退后一步说吧,来不 
及了解对象就挞伐所结出的果子,一定也与来不及了解对象就歌颂同样苦 
涩。这,也是我们译注先秦诸子的一个原因。 

这套丛书,承蒙著名学者启功、郭预衡两位老先生的关心,我们深感荣 
幸。北师大中文系郭英德先生和北京图书馆吴龙辉博士对本丛书的组织编译 
做了大量工作,没有他们的努力,这套丛书的出版是不可能的。丛书最后由 
我审定,由于学力不逮,时间紧迫,加之译注者水平不一,错漏之处在所难 
免。可以说,这套丛书如果还有可取之处,应该归功于学界前辈的指导和学 
养我只能望其项背的诸位先生的辛勤劳动;而它的所有不足,则应归咎于我 
的才疏学浅,力不胜任。 
《先秦诸子今译丛书》主编 李双 
1992 年 8 月 21 日于北京 


\mainmatter

\chapter{一  亲士}

“亲士”的意思是说要重视人才,这与墨子“尚贤”的主张是一致的,即认为一个国家兴盛与否的关键在于是否能够任用贤才。以此为开首第一篇,也可见其重视程度,这也无疑表现出了墨子宏通与长远的战略眼光。

文章首先把贤士的作用提到了一个极高的位置,然后通过晋文公、齐桓公与越王勾践的例子以及夏桀与商纣的反例来证明用贤的重要。接下来,作者还认为,国君要用贤,一定要律己严而待人宽,只有这样,才会有更多的贤人为国所用。此外,作者还极为深刻地指出,士因其能力的突出而遭受杀身之祸的事例太多了,所以警诫帝王一定要善待贤士,凡是人才,都有一定的个性,难于驾驭,但正因如此,帝王才更要尊重他们,只有这样,才能成就帝王的大业。

\begin{yuanwen}
入国\footnote{“入”疑“乂”之形误,乂国即治国。}而不存\footnote{恤问,即关心的意思。}其士,则亡国矣。见贤而不急,则缓其君矣。非贤无急,非士无与虑国。缓贤忘士,而能以其国存者,未曾有也。昔者文公\footnote{指晋文公重耳,他曾被迫流亡于外十九年,后来回国即位。他在位期间,重用贤才,终于使晋国强大起来,成为春秋五霸之一。}出走而正天下;桓公\footnote{指齐桓公,他未做国君前,他的哥哥齐襄公昏庸无道,而被迫出奔莒国,襄公死后他被迎回即位。此后他重用管仲,也成为春秋五霸之一。}去国而霸诸侯;越王勾践\footnote{越国国君,曾被吴王夫差打败,于是卧薪尝胆,励精图治,终于在范蠡与文种等贤臣的帮助下消灭吴国,报仇雪恨,并成为春秋五霸之一。}遇吴王之丑,而尚摄\footnote{同“慑”。}中国之贤君。三子之能达名成功于天下也,皆于其国抑而\footnote{同“尔”}大丑也。太上无败,其次败而有以成,此之谓用民。
\end{yuanwen}

治理国家却不关心那里的贤士,就会有亡国的危险。见到贤人却不马上任用,他们就会怠慢君主。没有比任用贤士更急迫的事了,如果没有贤士也就没人谋划国家大事。怠慢贤士、轻视人才,而能使国家长治久安,是从来没有过的。从前,晋文公被迫出逃却能够匡正天下;齐桓公流亡国外却能称霸诸侯;越王勾践遭受到败于吴王的耻辱,却还能威慑中原各国的贤君。这三个人能成功扬名于天下,都是因为他们在自己的国家能够忍受极大的屈辱。所以说,最好是不失败,其次则是败了却还有办法成功,这才叫善于用人。

\begin{yuanwen}
吾闻之曰:“非无安居也,我无安心也;非无足财也,我无足心也。”是故君子自难而易彼,众人自易而难彼。君子进不败其志,内究其情\footnote{text};虽杂庸民,终无怨心,彼有自信者也。是故为其所难者,必得其所欲焉;未闻为其所欲,而免其所恶者也。
\end{yuanwen}

\begin{yuanwen}
是故偪\footnote{text}臣伤君\footnote{text},谄\footnote{text}下伤上。君必有弗弗之臣\footnote{text},上必有詻詻之下\footnote{text},分议者延延\footnote{text},而支苟者詻詻\footnote{text},焉可以长生保国。\footnote{text}
臣下重其爵位而不言,近臣则喑\footnote{text},远臣则唫\footnote{text},怨结于民心。谄谀\footnote{text}在侧,善议障塞,则国危矣。桀\footnote{text}纣不以其无天下之士邪\footnote{text}?杀其身而丧天下。故曰:“归国宝\footnote{text},不若献贤而进士。”
\end{yuanwen}

\begin{yuanwen}
今有五锥,此其铦\footnote{text},铦者必先挫。有五刀\footnote{text},此其错\footnote{text},错者必先靡\footnote{text}。是以甘井近竭\footnote{text},招木近伐\footnote{text},灵龟近灼\footnote{text},神蛇近暴\footnote{text}。是故比干之殪\footnote{text},其抗也\footnote{text};孟贲之杀\footnote{text},其勇也;西施之沉\footnote{text},其美也;吴起之裂\footnote{text},其事也。故彼人者,寡不死其所长,故曰:太盛难守也。
\end{yuanwen}

\begin{yuanwen}
故虽有贤君,不爱无功之臣;虽有慈父,不爱无益之子。是故不胜其任而处其位,非此位之人也;不胜其爵而处其禄,非此禄之主也。良弓难张,然可以及高入深;良马难乘,然可以任重致远;良才难令,然可以致君见尊。是故江河不恶小谷之满已也,故能大。圣人者,事无辞也,物无违也,故能为天下器。是故江河之水,非一源之水也;千镒之裘\footnote{text},非一狐之白也\footnote{text}。夫恶有同方不取而取同已者乎?盖非兼王之道也!是故天地不昭昭,大水不潦潦,大火不燎燎,王德不尧尧者,乃千人之长也。其直如矢,其平如砥,不足以覆万物。是故溪陕者速涸\footnote{text},逝浅者速竭,墝埆\footnote{text}者其地不育\footnote{text}。王者淳泽,不出宫中,则不能流国矣。 
\end{yuanwen}


(3)内:依俞樾校,当作“■”(即“退”)。(4)逼:同“嬖”。(5)弗:通“拂”。(6) 

詻(è)詻:同“谔谔”。(7)延延:通“炎炎”。(8)支苟:疑“交苛”二字形误。(9)错:同“厝”, 

磨刀石。(10) 埆(qiāoquè):土地坚硬而瘠薄。 
[白话] 
治国而不优待贤士,国家就会灭亡。见到贤士而不急于任用,他们就会 
怠慢君主。没有比用贤更急迫的了,若没有贤士,就没有人和自己谋划国事。 
怠慢遗弃贤士而能使国家长治久安的,还不曾有过。 
从前,晋文公被迫逃亡在外,后为天下盟主;齐桓公被迫离开国家,后 
来称霸诸侯;越王勾践被吴王战败受辱,终成威慑中原诸国的贤君。这三君 

所以能成功扬名于天下,是因为他们都能忍辱负耻,以图复仇。最上的是不 
遭失败,其次是失败而有办法成功,这才叫善于使用士民。 
我曾听说:“我不是没有安定的住处,而是自己没有安定之心;不是没 
有丰足的财产,而是怀着无法满足的心。”所以君子严以律己,宽以待人。 
而一般人则宽以律己,严以待人。君子仕进顺利时不改变他的素志,不得志 
时心情也一样;即使杂处于庸众之中,也终究没有怨尤之心。他们是有着自 
信的人。所以说,凡事能从难处做起,就一定能达到自己的愿望,但却没有 
听说只做自己所想的事情,而能免于所厌恶之后果的。所以倖臣与谗佞之辈 
往往伤害君主。君主必须有敢于矫正君主过失的臣僚,上面必须有直言极谏 
的下属,分辩议事的人争论锋起,互相责难的人互不退让,这才可以长养民 
生,保卫国土。 
如果臣下只以爵禄为重,不对国事发表意见,近臣缄默不言,远臣闭口 
暗叹,怨恨就郁结于民心了。谄谀阿奉之人围在身边,好的建议被他们阻障 
难进,那国家就危险了。桀、纣不正是因为他们不重视天下之士吗?结果身 
被杀而失天下。所以说:赠送国宝,不如推荐贤士。 
比如现在有五把锥子,一把最锋利,那么这一把必先折断。有五把刀, 
一把磨得最快,那么这一把必先损坏。所以甜的水井最易用干,高的树木最 
易被伐,灵验的宝龟最先被火灼占卦,神异的蛇最先被曝晒求雨。所以,比 
干之死,是因为他抗直;孟贲被杀,是因为他逞勇;西施被沉江,是因为长 
得美丽;吴起被车裂,是因为他有大功。这些人很少不是死于他们的所长。 
所以说:太盛了就难以持久。 
因此,即使有贤君,他也不爱无功之臣;即使有慈父,他也不爱无益之 
子。所以,凡是不能胜任其事而占据这一位置的,他就不应居于此位;凡是 
不胜任其爵而享受这一俸禄的,他就不当享有此禄。良弓不容易张开,但可 
以射得高没得深;良马不容易乘坐,但可以载得重行得远;好的人才不容易 
驾驭,但可以使国君受人尊重。所以,长江黄河不嫌小溪灌注它里面,才能 
让水量增大。圣人勇于任事,又能接受他人的意见,所以能成为治理天下的 
英才。所以长江黄河里的水,不是从同一水源流下的;价值千金的狐白裘, 
不是从一只狐狸腋下集成的。哪里有与自己相同的意见才采纳,与自己不同 
的意见就不采纳的道理呢?这不是统一天下之道。所以大地不昭昭为明(而 
美丑皆收),大水不潦潦为大(而川泽皆纳),大火不燎燎为盛(而草木皆 
容),王德不尧尧为高(而贵贱皆亲),才能做千万人的首领。 
象箭一样直,象磨刀石一样平,那就不能覆盖万物了。所以狭隘的溪流 
干得快,平浅的川泽枯得早,坚薄的土地不长五谷。做王的人深恩厚泽不出 
宫中,就不能流遍全国。 


\chapter{二  修身(1)}

二  修身(1)

君子战虽有陈(2),而勇为本焉;丧虽有礼,而哀为本焉;士虽有学,而 
行为本焉。是故置本不安者,无务丰末;近者不亲,无务求远;亲戚不附, 
无务外交;事无终始,无务多业;举物而暗,无务博闻。 
是故先王之治天下也,必察迩来远,君子察迩而迩修者也。见不修行见 
毁而反之身者也,此以怨省而行修矣。谮慝之言,无入之耳;批扞之声,无 
出之口;杀伤人之孩(3),无存之心,虽有诋讦之民,无所依矣。 
故君子力事日强,愿欲日逾,设壮日盛。君子之道也:贫则见廉,富则 
见义,生则见爱,死则见哀;四行者不可虚假,反之身者也。藏于心者,无 
以竭爱;动于身者,无以竭恭;出于口者,无以竭驯。畅之四支,接之肌肤, 
华发隳颠,而犹弗舍者,其唯圣人乎! 
志不强者智不达;言不信者行不果;据财不能以分人者,不足与友;守 
道不笃,遍物不博,辩是非不察者(4),不足与游。本不固者末必几,雄而不 
修者,其后必惰,原浊者流不清,行不信者名必惰。名不徒生而誉不自长。 
功成名遂,名誉不可虚假,反之身者也。务言而缓行,虽辩必不听。多力而 
伐功,虽劳必不图。慧者心辩而不繁说,多力而不伐功,此以名誉扬天下。 
言无务为多而务为智,无务为文而务为察。故彼智无察,在身而情(5),反其 
路者也。善无主于心者不留,行莫辩于身者不立;名不可简而成也,誉不可 
巧而立也,君子以身戴行者也(6)。思利寻焉,忘名忽焉,可以为士于天下者, 
未尝有也。 


[注释] 

(1)本篇主要讨论品行修养与君子人格问题,强调品行是为人治国的根本,君子必须以品德修养 

为重。篇中提出。“君子之道”应包括‘贫则见廉,富则见义,生则见爱,死则见哀’以及明察是非、 

讲究信用、注重实际等内容。(2)陈:同“阵”。(3)孩:毕沆云:“当读如根荄之荄。”(4)辩:同“辨”。 

(5)彼:借为“非”。情:为“惰”之形讹。(6)戴:同“载”。 
[白话] 
君子作战虽用阵势,但必以勇敢为本;办丧事虽讲礼仪,但必以哀痛为 
本;做官虽讲才识,但必以德行为本。所以立本不牢的,就不必讲究枝节的 
繁盛;身边的人不能亲近,就不必讲究招徕远方之民;亲戚不能使之归附, 
就不必讲究结纳外人;做一件事情有始无终,就不必谈起从事多种事业;举 
一件事物尚且弄不明白,就不必追求广见博闻。 
所以先王治理天下,必定要明察左右而招徕远人。君子能明察左右,左 
右之人也就能修养自己的品行了。君子不能修养自己的品行而受人诋毁,那 
就应当自我反省,因而怨少而品德日修。谗害诽谤之言不入于耳,攻击他人 
之语不出于口,伤害人的念头不存于心,这样,即使遇有好诋毁、攻击的人, 
也就无从施展了。 
所以君子本身的力量一天比一天加强,志向一无比一天远大,庄敬的品 
行一天比一天完善。君子之道(应包括如下方面):贫穷时表现出廉洁,富 
足时表现出恩义,对生者表示出慈爱,对死者表示出哀痛。这四种品行不是 
可以装出来的,而是必须自身具备的。凡是存在于内心的,是无穷的慈爱; 
举止于身体的,是无比的谦恭;谈说于嘴上的,是无比的雅驯。(让上述四 
种品行)畅达于四肢和肌肤,直到白发秃顶之时仍不肯舍弃,大概只有圣人 

吧! 
意志不坚强的,智慧一定不高;说话不讲信用的,行动一定不果敢;拥 
有财富而不肯分给人的,不值得和他交友;守道不坚定,阅历事物不广博, 
辨别是非不清楚的,不值得和他交游。根本不牢的,枝节必危。光勇敢而不 
注重品行修养的,后必懒惰。源头浊的流不清,行为无信的人名声必受损害, 
声誉不会无故产生和自己增长。功成了必然名就,名誉不可虚假,必须反求 
诸己。专说而行动迟缓,虽然会说,但没人听信。出力多而自夸功劳,虽劳 
苦而不可取。聪明人心里明白而不多说,努力作事而不夸说自己的功劳,因 
此名誉扬于天下。说话不图繁多而讲究富有智慧,不图文采而讲究明白。所 
以既无智慧又不能审察,加上自身又懒惰,则必背离正道而行了。善不从本 
心生出就不能保留,行不由本身审辨就不能树立,名望不会由苟简而成,声 
誉不会因诈伪而立,君子是言行合一的。以图利为重,忽视立名,(这样) 
而可以成为天下贤士的人,还不曾有过。 


\chapter{三  所染(1)}

三  所染(1)

子墨子言见染丝者而叹曰:染于苍则苍,染于黄则黄。所入者变,其色 
亦变;五入必而已则为五色矣。故染不可不慎也! 
非独染丝然也,国亦有染。舜染于许由、伯阳,禹染于皋陶、伯益,汤 
染于伊尹、仲虺,武王染于太公、周公。此四王者所染当,故王天下,立为 
天下,功名蔽天地。举天下之仁义显人,必称此四王者。 
夏桀染于干辛、推哆(2),殷纣染于崇侯、恶来,厉王染于厉公长父、荣 
夷终,幽王染于傅公夷、蔡公穀。此四王者所染不当,故国残身死,为天下 
僇。举天下不义辱人,必称此四王者。 
齐桓染于管仲、鲍叔,晋文染于舅犯、高偃,楚庄染于孙叔、沈尹,吴 
阖闾染于伍员、文义,越勾践染于范蠡、大夫种。此五君者所染当,故霸诸 
侯,功名传于后世。 
范吉射染于长柳朔、王胜,中行寅染于藉秦、高强,吴夫差染于王孙雒、 
太宰嚭,智伯摇染于智国、张武,中山尚染于魏义、偃长,宋康染于唐鞅、 
佃不礼。此六君者所染不当,故国家残亡,身为刑戮,宗庙破灭,绝无后类, 
君臣离散,民人流亡。举天下之贪暴苛扰者,必称此六君也。 
凡君之所以安者何也?以其行理也。行理性于染当。故善为君者,劳于 
论人而佚于治官(3)。不能为君者,伤形费神,愁心劳意;然国逾危,身逾辱。 
此六君者,非不重其国、爱其身也,以不知要故也。不知要者,所染不当也。 
非独国有染也,士亦有染。其友皆好仁义,淳谨畏令,则家日益,身日 
安,名日荣,处官得其理矣,则段干木、禽子、傅说之徒是也。其友皆好矜 
奋,创作比周,则家日损,身日危,名日辱,处官失其理矣,则子西、易牙、 
竖刀之徒是也。诗曰:“必择所堪(4)”。必谨所堪者,此之谓也。 


[注释] 

(1)本篇以染丝为喻,说明天子、诸侯、大夫、士必须正确选择自己的亲信和朋友,以取得良好 

的熏陶和积极的影响。影响的好坏不同关系着事业的成败、国家的兴亡,国君对此必须谨慎。(2)推哆 

(chǐ):桀臣。(3)佚:同“逸”。(4)堪:当读为“湛”,浸染之意。 
[白话] 
墨子说,他曾见人染丝而感叹说:“(丝)染了青颜料就变成青色,染 
了黄颜料就变成黄色。染料不同,丝的颜色也跟着变化。经过五次之后,就 
变为五种颜色了。所以染这件事是不可不谨慎的。” 
不仅染丝如此,国家也有“染”。舜被许由、伯阳所染,禹被皋陶、伯 
益所染,汤被伊尹、仲虺所染,武王被太公、周公所染。这四位君王因为所 
染得当,所以能称王于天下,立为天子,功盖四方,名扬天下,凡是提起天 
下著名的仁义之人,必定要称这四王。 
夏桀被干辛、推哆所染,殷纣被崇侯、恶来所染,周厉王被厉公长父、 
荣夷终所染,周幽王被傅公夷、蔡公穀所染。这四位君王因为所染不当,结 
果身死国亡,遗羞于天下。凡是提起天下不义可耻之人,必定要称这四王。 
齐桓公被管仲、鲍叔牙所染,晋文公被舅犯、高偃所染,楚庄王被孙叔 
敖、沈尹茎所染,吴王阖闾被伍员、文义所染,越王句践被范蠡、文种所染。 
这五位君主因为所染得当,所以能称霸诸侯,功名传于后世。 
范吉射被长柳朔、王胜所染,中行寅被籍秦、高强所染,吴王夫差被王 

孙雒、太宰嚭所染,知伯摇被知国、张武所染,中山尚被魏义、偃长所染, 
宋康王被唐鞅、佃不礼所染。这六位君主因为所染不当,所以国破家亡,身 
受刑戮,宗庙毁灭,子孙灭绝,君臣离散,百姓逃亡。凡是提起天下贪暴苛 
刻的人,必定称这六君。 
大凡人君之所以能够安定,是什么原因呢?是因为他们行事合理。而行 
事合理源于所染得当。所以善于做国君的,用心致力于选拔人才。不善于做 
国君的,劳神伤身,用尽心思,然而国家更危险,自己更受屈辱。上述这六 
位国君,并非不重视他们的国家、爱惜他们的身体,而是因为他们不知道治 
国要领的缘故。所谓不知道治国要领,即是所染不得当。 
不仅国家有染,士也有“染”。一个人所交的朋友都爱好仁义,都淳朴 
谨慎,慑于法纪,那么他的家道就日益兴盛,身体日益平安,名声日益光耀, 
居官治政也合于正道了,如段干木、禽子、傅说等人即属此类(朋友)。一 
个人所交的朋友若都不安分守己,结党营私,那么他的家道就日益衰落,身 
体日益危险,名声日益降低,居官治政也不得其道,如子西、易牙、竖刀等 
人即属此类(朋友)。《诗》上说:“选好染料。”所谓选好染料,正是这 
个意思。 



\chapter{四  法仪(1)}
四  法仪(1)

子墨子曰:天下从事者,不可以无法仪;无法仪而其事能成者,无有也。 
虽至士之为将相者,皆有法。虽至百工从事者,亦皆有法。百工为方以矩, 
为圆以规,直以绳,正以县(2)。无巧工、不巧工,皆以此五者为法。巧者能 
中之,不巧者虽不能中,放依以从事(3),犹逾己。故百工从事,皆有法所度。 
今大者治天下,其次治大国,而无法所度,此不若百工辩也(4),然则奚 
以为治法而可?当皆法其父母,奚若?天下之为父母者众,而仁者寡。若皆 
法其父母,此法不仁也。法不仁,不可以为法。当皆法其学,奚若?天下之 
为学者众,而仁者寡。若皆法其学,此法不仁也。法不仁,不可以为法。当 
皆法其君,奚若?天下之为君者众,而仁者寡。若皆法其君,此法不仁也。 
法不仁,不可以为法。故父母、学、君三者,莫可以为治法。 
然则奚以为治法而可?故曰:莫若法天。天之行广而无私,其施厚而不 
德,其明久而不衰,故圣王法之。既以天为法,动作有为,必度于天。天之 
所欲则为之,天所不欲则止。然而天何欲何恶者也?天必欲人之相爱相利, 
而不欲人之相恶相贼也。奚以知天之欲人之相爱相利,而不欲人之相恶相贼 
也?以其兼而爱之,兼而利之也。奚以知天兼而爱之、兼而利之也?以其兼 
而有之、兼而食之也。 
今天下无大小国,皆天之邑也。人无幼长贵贱,皆天之臣也。此以莫不 
■羊(5),豢犬猪,洁为酒醴粢盛,以敬事天。此不为兼而有之、兼而食之邪? 
天苟兼而有食之,夫奚说以不欲人之相爱相利也?故曰:爱人利人者,天必 
福之;恶人贼人者,天必祸之。曰:杀不辜者,得不祥焉。夫奚说人为其相 
杀而天与祸乎?是以知天欲人相爱相利,而不欲人相恶相贼也。 
昔之圣王禹汤文武,兼爱天下之百姓,率以尊天事鬼。其利人多,故天 
福之,使立为天子,天下诸侯,皆宾事之。暴王桀纣幽厉,兼恶天下之百姓, 
率以诟天侮鬼。其贼人多,故天祸之,使遂失其国家,身死为僇于天下。后 
世子孙毁之,至今不息。故为不善以得祸者,桀纣幽厉是也。爱人利人以得 
福者,禹汤文武是也。爱人利人以得福者,有矣!恶人贼人以得祸者,亦有 
矣! 


[注释] 

(1)法仪即法度、准则之意。墨子认为,天子、诸侯治理天下、国家必须以天为法,以天意为归。 

而所谓天意,实即就是墨家学派所主张的兼爱兼利原则。篇中以古代圣王和暴君为正反两方面的例子, 

指出“爱人利人”即可得福,“恶人贼人”必然招祸。(2)县:即“悬”的本字。(3)放:通“仿”。 

(4)辩:通“辨”。(5)■:同“刍”。 
[白话] 
墨子说:天底下办事的人,不能没有法则;没有法则而能把事情做好, 
是从来没有的事。即使士人作了将相,他也必须有法度。即使从事于各种行 
业的工匠,也都有法度。工匠们用矩划成方形,用圆规划圆形,用绳墨划成 
直线,用悬锤定好偏正,(用水平器制好平面)。不论是巧匠还是一般工匠, 
都要以这五者为法则。巧匠能切合五者的标准,一般工匠虽做不到这样水平, 
但仿效五者去做,还是要胜过自身的能力。所以工匠们制造物件时,都有法 
则可循。 
现在大的如治天下,其次如治大国,却没有法则,这是不如工匠们能明 

辨事理。那么,用什么作为治理国家的法则才行呢?假若以自己的父母为法 
则何如?天下做父母的很多,但仁爱的少。倘若人人都以自己的父母为法则, 
这实为效法不仁。效法不仁,这自然是不可以的。假若以自己从学的师长为 
法何如?天下做师长的很多,但仁爱的少。倘若人人都以自己的师长为法则, 
这实为效法不仁。效法不仁,这自然是不可以的。假若以自己的国君为法则 
何如?天下做国君的很多,但仁爱的少。倘若人人都以自己的国君为法则, 
这实为效法不仁。效法不仁,这自然是不可以的。所以父母、师长和国君三 
者,都不可以作为治理国家的法则。 
那么用什么作为治理国家的法则才行呢?最好是以天为法则。天的运行 
广大无私,它的恩施深厚而不自居,它的光耀永远不衰,所以圣王以它为法 
则。既然以天为法则,行动作事就必须依天而行。天所希望的就去做,天所 
不希望的就应停止。那么天希望什么不希望什么呢?天肯定希望人相爱相 
利,而不希望人相互厌恶和残害。怎么知道天希望人相爱相利,而不希望人 
相互厌恶和残害呢?这是因为天对人是全爱和全利的缘故。怎么知道天对人 
是全爱和全利呢?因为人类都为天所有,天全部供给他们吃的。 
现在天下不论大国小国,都是天的国家。人不论长幼贵贱,都是天的臣 
民。因此人无不喂牛羊、养猪狗,洁净地准备好酒食祭品,用来诚敬事天。 
这难道不是全部地拥有和供给人食物?天既然全部地拥有和供给人食物,为 
何能说天不要人相爱相利呢?所以说:“爱人利人的人,天必定给他降福; 
相互厌恶和残害人的人,天必定给他降祸。所以说:杀害无辜的人,会得到 
不祥后果。为何说人若相互残杀,天就降祸于他呢?这是因为知道天希望人 
相爱相利,而不希望人相互厌恶和残害。” 
以前的圣王禹、汤、周文王、周武王,对天下百姓全都爱护,带领他们 
崇敬上天,侍奉鬼神。他们给人带来的利益多,所以天降福给他们,使他们 
立为天子。天下的诸侯,都恭敬地服事他们。暴虐的君王桀、纣、周幽王、 
周厉王,对于天下的百姓全部厌恶、憎恨,带领他们咒骂上天,侮辱鬼神。 
他们残害的人多,所以天降祸给他们,使他们丧失了国家,身死还要受辱于 
天下。后代子孙责骂他们,至今不休。所以做坏事而得祸的,桀、纣、周幽 
王、周厉王即是这类;爱人利人而得福的,禹、汤、周文王、周武王即是这 
类。爱人利人而得福的是有的,厌恶人残害人而得祸的,也是有的! 


\chapter{五  七患(1)}
五  七患(1)

子墨子曰:国有七患。七患者何?城郭沟池不可守而治宫室,一患也; 
边国至境(2),四邻莫救,二患也;先尽民力无用之功,赏赐无能之人,民力 
尽于无用,财宝虚于待客,三患也;仕者持禄,游者爱佼(3),君修法讨臣, 
臣慑而不敢拂,四患也;君自以为圣智而不问事,自以为安强而无守备,四 
邻谋之不知戒,五患也;所信者不忠,所忠者不信,六患也;畜种菽粟不足 
以食之,大臣不足以事之,赏赐不能喜,诛罚不能威,七患也。 
以七患居国,必无社稷;以七患守城,敌至国倾。七患之所当,国必有 
殃。 
凡五谷者,民之所仰也,君之所以为养也。故民无仰,则君无养;民无 
食,则不可事。故食不可不务也,地不可不力也,用不可不节也。五谷尽收, 
则五味尽御于主,不尽收则不尽御。一谷不收谓之馑,二谷不收谓之旱,三 
谷不收谓之凶,四谷不收谓之馈(4),五谷不收谓之饥。 
岁馑,则仕者大夫以下皆损禄五分之一;旱,则损五分之二;凶,则损 
五分之三;馈,则损五分之四;饥,则尽无禄,禀食而已矣。故凶饥存乎国, 
人君彻鼎食五分之五(5),大夫彻县(6),士不入学,君朝之衣不革制;诸侯 
之客,四邻之使,雍食而不盛(7);彻骖騑,涂不芸(8),马不食粟,婢妾不 
衣帛,此告不足之至也。 
今有负其子而汲者,队其子于井中(9),其毋必从而道之。今岁凶,民饥, 
道饿,重其子此疚于队,其可无察邪!故时年岁善,则民仁且良;时年岁凶, 
则民吝且恶。夫民何常此之有!为者疾,食者众,则岁无丰。 
故曰:财不足则反之时,食不足则反之用。故先民以时生财,固本而用 
财,则财足。故虽上世之圣王,岂能使五谷常收而旱水不至哉!然而无冻饿 
之民者,何也?其力时急而自养俭也。故《夏书》曰:“禹七年水。”《殷 
书》曰:“汤五年旱。”此其离凶饿甚矣(10),然而民不冻饿者,何也?其 
生财密,其用之节也。故仓无备粟,不可以待凶饥;库无备兵,虽有义不能 
征无义;城郭不备全,不可以自守;心无备虑,不可以应卒(11),是若庆忌 
无去之心,不能轻出。 
夫桀无待汤之备,故放;纣无待武之备,故杀。桀纣贵为天子,富有天 
下,然而皆灭亡于百里之君者,何也?有富贵而不为备也。故备者,国之重 
也。食者,国之宝也;兵者,国之爪也;城者,所以自守也;此三者,国之 
具也。 
故曰:以其极赏,以赐无功;虚其府库,以备车马、衣裘、奇怪;苦其 
役徒,以治宫室观乐;死又厚为棺椁,多为衣裘。生时治台榭,死又修坟墓。 
故民苦于外,府库单于内(12),上不厌其乐(13),下不堪其苦。故国离寇敌 
则伤,民见凶饥则亡,此皆备不具之罪也。且夫食者,圣人之所宝也。故《周 
书》曰:“国无三年之食者,国非其国也;家无三年之食者,子非其子也。” 
此之谓国备。 


[注释] 

(1)本篇首先分析了给国家造成危亡的七种祸患,然后指出国家防治祸患的根本在于增加生产和 

节省财用,并对当时统治者竭尽民力和府库之财以追求享乐生活的做法提出了严正警告。(2)边:“敌” 

字之误。(3)佼:通“交”。(4)馈:通“匮”,缺乏。(5)五分之五:疑作“五分之三”。(6)县:通 

“悬”,此指钟磬等悬挂的乐器。(7)雍:当作“饔”,指早餐和晚餐。(8)涂:通“途”。(9)队:通 

“坠”。(10)离:通“罹”,遭受。(11)卒:通“猝”。(12)单:通“殚”。(13)厌:通“餍”,满 

足。 
[白话] 
墨子说:国家有七种祸患。这七种祸患是什么呢?内外城池壕沟不足守 
御而去修造宫室,这是第一种祸患;敌兵压境,四面邻国都不愿来救援,这 
是第二种祸患;把民力耗尽在无用的事情上,赏赐没有才能的人,(结果) 
民力因做无用的事情而耗尽,财宝因款待宾客而用空,这是第三种祸患;做 
官的人只求保住俸禄,游学未仕的人只顾结交党类,国君修订法律以诛戮臣 
下,臣下畏惧而不敢违拂君命,这是第四种祸患;国君自以为神圣而聪明, 
而不过问国事,自以为安稳而强盛,而不作防御准备,四面邻国在图谋攻打 
他,而尚不知戒备,这是第五种祸患;所信任的人不忠实,而忠实的人不被 
信任,这是第六种祸患;家畜和粮食不够吃,大臣对于国事不胜使令,赏赐 
不能使人欢喜,责罚不能使人畏惧,这是第七种祸患。 
治国若存在这七种祸患,必定亡国;守城若存在这七种祸患,国都必定 
倾毁。七种祸患存在于哪个国家,哪个国家必有祸殃。 
五谷是人民所仰赖以生活的东西,也是国君用以养活自己和民众的。所 
以如果人民失去仰赖,国君也就没有供养;人民一旦没有吃的,就不可使役 
了。所以粮食不能不加紧生产,田地不能不尽力耕作,财用不可不节约使用。 
五谷全部丰收,国君就可兼进五味。若不全都丰收,国君就不能尽其享受。 
一谷无收叫做馑,二谷无收叫做旱,三谷不收叫做凶,四谷不收叫做匮,五 
谷不收叫做饥。 
遇到馑年,做官的自大夫以下都减去俸禄的五分之一;旱年,减去俸禄 
的五分之二;凶年,减去俸禄的五分之三;匮年,减去俸禄的五分之四;饥 
年,免去全部俸禄,只供给饭吃。所以一个国家遇到凶饥,国君撤掉鼎食的 
五分之三,大夫不听音乐,读书人不上学而去种地,国君的朝服不制新的; 
诸侯的客人、邻国的使者,来时饮食都不丰盛,驷马撤掉左右两匹,道路不 
加修理,马不吃粮食,婢妾不穿丝绸,这都是告诉国家已十分困乏了。 
现在假如有一人背着孩子到井边汲水,把孩子掉到井里,那么这位母亲 
必定设法把孩子从井中救出。现在遇到饥年,路上有饿死的人,这种惨痛比 
孩子掉入井中更为严重,能忽视这种局面吗?年成好的时候,老百姓就仁慈 
驯良;年成遇到凶灾,老百姓就吝啬凶恶;民众的性情哪有一定呢!生产的 
人少,吃饭的人多,就不可能有丰年。 
所以说:财用不足就注重农时,粮食不足就注意节约。因此,古代贤人 
按农时生产财富,搞好农业基础,节省开支,财用自然就充足。所以,即使 
前世的圣王,岂能使五谷永远丰收,水旱之灾不至呢!但(他们那时)却从 
无受冻挨饿之民,这是为何呢?这时因为他们努力按农时耕种而自奉俭朴。 
《夏书》说:“禹时有七年水灾。”《殷书》说:“汤时有五年旱灾。”那 
时遭受的凶荒够大的了,然而老百姓却没有受冻挨饿,这是何故呢?因为他 
们生产的财用多,而使用很节俭。所以,粮仓中没有预备粮,就不能防备凶 
年饥荒;兵库中没有武器,即使自己有义也不能去讨伐无义;内外城池若不 
完备,不可以自行防守;心中没有戒备之心,就不能应付突然的变故。这就 
好像庆忌没有逐走要离之意,就不可轻出致死。 
桀没有防御汤的准备,因此被汤放逐;纣没有防御周武王的准备,因此 

被杀。桀和纣虽贵为天子,富有天下,然而都被方圆百里的小国之君所灭, 
这是为何呢?是因为他们虽然富贵,却不做好防备。所以防备是国家最重要 
的事情。粮食是国家的宝物,兵器是国家爪牙,城郭是用来自我守卫的:这 
三者是维持国家的工具。 
所以说:拿最高的奖赏赐给无功之人;耗尽国库中的贮藏,用以置备车 
马、衣裘和稀奇古怪之物;使役卒和奴隶受尽苦难,去建造宫室和观赏游乐 
之所;死后又做厚重的棺椁,制很多衣服。活着时修造台榭,死后又修造坟 
墓。因此,老百姓在外受苦,内边的国库耗尽,上面的君主不满足其享受, 
下面的民众不堪忍受其苦难。所以,国家一遇敌寇就受损伤,人民一遭凶饥 
就死亡,这都是平时不做好防备的罪过。再说,粮食也是圣人所宝贵的。《周 
书》说:“国家若不预备三年的粮食,国家就不可能成其为这一君主的国家 
了;家庭若不预备三年的粮食,子女就不能做这一家的子女了。”这就叫做 
“国备”(国家的根本贮备)。 

\chapter{六  辞过(1)}
六  辞过(1)

子墨子曰:“古之民,未知为宫室时,就陵阜而居,穴而处,下润湿伤 
民,故圣王作为宫室。为宫室之法,曰室高足以辟润湿,边足以圉风寒,上 
足以待雪霜雨露,宫墙之高,足以别男女之礼,谨此则止(2)。凡费财劳力, 
不加利者,不为也。役(3),修其城郭,则民劳而不伤,以其常正(4),收其 
租税,则民费而不病。民所苦者非此也,苦于厚作敛于百姓。是故圣王作为 
宫室,便于生,不以为观乐也;作为衣服带履便于身,不以为辟怪也。故节 
于身,诲于民,是以天下之民可得而治,财用可得而足。 
当今之主,其为宫室,则与此异矣。必厚作敛于百姓,暴夺民衣食之财, 
以为宫室,台榭曲直之望,青黄刻镂之饰。为宫室若此,故左右皆法象之, 
是以其财不足以待凶饥、振孤寡,故国贫而民难治也。君欲实天下之治,而 
恶其乱也,当为宫室,不可不节。 
古之民,未知为衣服时,衣皮带茭,冬则不轻而温,夏则不轻而凊。圣 
王以为不中人之情,故作诲妇人,治丝麻,棞布绢,以为民衣。为衣服之法, 
冬则练帛之中,足以为轻且暖;夏则 绤之中,足以为轻且凊,谨此则止。 
故圣人之为衣服,适身体,和肌肤,而足矣。非荣耳目而观愚民也。当是之 
时,坚车良马不知贵也,刻镂文采,不知喜也,何则?其所道之然。故民衣 
食之财,家足以待旱水凶饥者,何也?得其所以自养之情,而不感于外也, 
是以其民俭而易治,其君用财节而易赡也。府库实满,足以待不然;兵革不 
顿,士民不劳,足以征不服。故霸王之业,可行于天下矣。 
当今之主,其为衣服,则与此异矣,冬则轻煗(5),夏则轻凊,皆已具矣, 
必厚作敛于百姓,暴夺民衣食之财,以为锦绣文采靡曼之衣,铸金以为钩, 
珠玉以为珮。女工作文采,男工作刻镂,以为身服,此非云益煗之情也。单 
财劳力(6),毕归之于无用也,以此观之,其为衣服非为身体,皆为观好,是 
以其民淫僻而难治,其君奢侈而难谏也,夫以奢侈之君,御好淫僻之民,欲 
国无乱,不可得也。君实欲天下之治而恶其乱,当为衣服不可不节。 
古之民未知为饮食时,素食而分处,故圣人作,诲男耕稼树艺,以为民 
食。其为食也,足以增气充虚,强体养腹而已矣。故其用财节,其自养俭, 
民富国治。今则不然,厚作敛于百姓,以为美食刍豢,蒸炙鱼鳖,大国累百 
器,小国累十器,前方丈,目不能遍视,手不能遍操,口不能遍味,冬则冻 
冰,夏则饰 (7),人君为饮食如此,故左右象之,是以富贵者奢侈,孤寡者 
冻馁,虽欲无乱,不可得也。君实欲天下治而恶其乱,当为食饮不可不节。 
古之民未知为舟车时,重任不移,远道不至,故圣王作为舟车,不便民 
之事。其为舟车也,全固轻利,可以任重致远,其为用财少,而为利多,是 
以民乐而利之。故法令不急而行,民不劳而上足用,故民归之。当今之主, 
其为舟车,与此异矣,全固轻利皆已具,必厚作敛于百姓,以饰舟车。饰车 
以文采,饰舟以刻镂。女子废其纺织而修文采,故民寒;男子离其耕稼而修 
刻镂,故民饥。人君为舟车若此,故左右象之,是以其民饥寒并至,故为奸 
邪。奸邪多则刑罚深,刑罚深则国乱。君实欲天下之治而恶其乱,当为舟车 
不可不节。 
凡回于天地之间,包于四海之内,天壤之情,阴阳之和,莫不有也,虽 
至圣不能更也。何以知其然?圣人有传:天地也,则曰上下;四时也,则曰 
阴阳;人情也,则曰男女;禽兽也,则曰牝牡雌雄也。真天壤之情,虽有先 

王不能更也。虽上世至圣,必蓄私,不以伤行,故民无怨。宫无拘女,故天 
下无寡夫。内无拘女,外无寡夫,故天下之民众。当今之君,其蓄私也,大 
国拘女累千,小国累百,是以天下之男多寡无妻,女多拘无夫,男女失时, 
故民少。君实欲民之众而恶其寡,当蓄私不可不节。 
凡此五者,圣人之所俭节也,小人之所淫佚也。俭节则昌,淫佚则亡, 
此五者不可不节。夫妇节而天地和,风雨节而五谷熟,衣服节而肌肤和。 


[注释] 

(1)本篇主要通过宫室、衣服、饮食、舟车、蓄私的古今对照,批判当时统治者的奢侈生活。主 

旨与《节用》篇基本相同。篇题所谓辞过,即要求时君改掉这五方面的过失。(2)谨:通“仅”。(3) 

役:上当有“以其常”三字。(4)正:通“征”。(5)煗:同“暖”。(6)单:通“殚”。(7)饰:“馂” 

的误字。 
[白话] 
墨子说:“上古的人民不知道作宫室之时,靠近山陵居住,住在洞穴里, 
地下潮湿,伤害人民,所以圣王开始营造宫室。营造宫室的法则是:地基的 
高度足以避湿润,四边足以御风寒,屋顶足以防备霜雪雨露,宫墙的高度足 
以分隔内外,使男女有别——仅此而已。凡属劳民伤财而不增加益处的事, 
是不会做的。(照常规)分派劳役,修治城郭,那么民众就虽劳苦而不至伤 
害;照常规征收租税,那么民众虽破费而不至困苦。因为民众所疾苦并不是 
这些,而是苦于对老百姓横征暴敛。所以圣王开始制造宫室,只为方便生活, 
并不是为了观赏之乐;开始创制衣服带履,只为便利身体,而不是为了奇怪 
的装束。所以,(圣王)自身节俭,(以身作则地)教导百姓,因而天下的 
民众得以治理,财用得以充足。 
现在的君主,修造宫室却与此不同:他们必定要向百姓横征暴敛,强夺 
民众的衣食之资用来营造宫室,(在宫室上)修造台榭曲折的景观,讲究颜 
色雕刻的装饰。营造宫室如此(铺张),身边的人都效法这种做法,因此财 
用(被浪费)而不能应付凶年饥馑,振恤孤寡之人,所以国家穷困而人民无 
法治理。国君若是真希望天下得到治理,而不愿其混乱,那么,营造宫室就 
不可不节俭。 
上古的人民不知道做衣服的时候,穿着兽皮,围着草索,冬天不轻便又 
不温暖,夏天不轻便又不凉爽。圣王认为这样不符合人情,所以开始教女子 
治丝麻、织布匹,以它作人的衣服。制造衣服的法则是:冬天穿生丝麻制的 
中衣,只求其轻便而温暖,夏天穿葛制的中衣,只求其轻便而凉爽,仅此而 
已。所以圣人制作衣服只图身体合适、肌肤舒适就够了,并不是夸耀耳目、 
炫动愚民。当这时候,坚车良马没有人知道贵重,雕刻文采没有人知道欣赏, 
为什么呢?这是(君主)教导的结果。所以民众的衣服之财,家家都足以防 
患水旱凶饥,为什么呢?因为他们懂得自我供养的情实,不被外界所诱惑, 
所以民众俭朴而容易治理,国君用财有节制而容易富足。国库充实,足以应 
付非常的变故:兵甲不坏,士民不劳,足以证伐不顺之臣,所以可实现霸王 
事业于天下。 
现在的君主,他们制造衣服却与此不同:冬天(的衣服)轻便而暖和, 
夏天(的衣服)轻便而凉爽,这都已经具备了,他们还一定要向百姓横征暴 
敛,强夺民众的衣食之资,用来做锦绣文彩华丽的衣服,拿黄金作成衣带钩, 
拿珠玉作成佩饰,女工作文采,男工作雕刻,用来穿在身上。这并非真的为 

了温暖。耗尽钱财费了民力,都是为了无用之事,由此看来,他们作衣服, 
不是为身体,而是为好看。因此民众邪僻而难以治理,国君奢侈而难以进谏。 
以奢侈的国君统治邪僻的民众,希望国家不乱,是不可能的。国君若真希望 
天下治理好而厌恶混乱,作衣服时就不可不节俭。 
上古的人民不知道制作饮食时,只吃素食而各自分居,所以圣人起来教 
勇子耕稼栽种,以供人作粮食。作饮食的原则是,只求补气益虚、强身饱腹 
就够了。所以他们用财节省,自养俭朴,(因而)民众富足,国家安定。现 
在却不是这样,向老百姓厚敛钱财,用来享受美味牛羊,蒸烤鱼鳖,大国之 
君集有上百样的菜,小国之君也有上十样的菜,摆在前面一丈见方,眼不能 
全看到,手不能全捡取到,嘴也不能全尝到,冬天结冻,夏天臭烂,国君这 
样讲究饮食,左右大臣都效法他。因此富贵的人奢侈,孤寡的人冻饿。这样 
一来,即使不希望国家混乱,也是不可能的。国君若真希望天下治理好而厌 
恶其混乱,饮食就不可不节省。 
上古的人民不知道制造舟车时,重的东西搬不动,远的地方去不了,所 
以圣王开始制造舟车,用以便利民事。他们作舟车只求坚固轻便,可以运重 
物、行远路,费用花的少,而利益很大,所以民众乐于使用。所以法令不用 
催促而可行使,民众不用劳苦而财用充足,所以民众归顺他了。现在的君主 
制造舟车则与此不同。舟车已经坚固轻利了,他们还要向百姓横征暴敛,用 
以装饰舟车。在车上画以文彩,在舟上加以雕刻。女子废弃纺织而去描绘文 
彩,所以民众受寒;男子脱离耕稼而去从事雕刻,所以民众挨饿。国君这样 
制造舟车,左右大臣跟着仿效,所以民众饥寒交迫,不得已而作奸邪之事。 
奸邪之事一多,刑罚必然繁重。刑罚一繁重,国家就乱了。国君如果真的希 
望天下治理好而厌恶混乱,制造舟车就不可不节省。 
凡周回于天地之间,包裹于四海之内的,天地之情,阴阳之和,一切都 
具备了,即使至圣也不能更动。何以知道这样呢?圣人传下的书说:天地称 
作上下,四时称作阴阳,人类分为男女,禽兽分为牝牡雌雄。这是真正的天 
地之情,即使有先世贤王也不能更动。即使上代至圣,一定都养有私人侍妾, 
但不伤害品行,所以民众无怨。宫中没有拘禁的女子,所以天下没有鳏夫。 
内无拘禁之妇,外无鳏夫,因而天下人民众多。现在的国君养侍妾,大国拘 
禁女子数千,小国数百,所以天下男子大多没有妻子,女子多遭拘禁而没有 
丈夫。男女婚姻失时,所以百姓减少。国君如果真想人民增多而厌恶减少, 
养侍妾就不可不节制。 
以上所说的五者,都是圣人所节俭而小人所奢侈淫佚的。节俭的就昌盛, 
淫佚的就灭亡,这五者不可不节制。夫妇之事有节制,天地就和顺;风雨调 
节,五谷就丰收;衣服有节制,身体肌肤就安适。 

\chapter{七  三辩(1)}
\chapter{八  尚贤(1)上}
\chapter{九  尚贤中  }             
\chapter{十  尚贤下}
\chapter{十一  尚同(1)上}
\chapter{十二  尚同中}
\chapter{十三  尚同下}
\chapter{十四  兼爱(1)上}
\chapter{十五  兼爱中}
\chapter{十六  兼爱下}
\chapter{十七  非攻(1)上}
\chapter{十八  非攻中}
\chapter{十九  非攻下}
\chapter{二十  节用(1)上}
\chapter{二十一  节用中}
\chapter{二十二  节葬(1)下}
\chapter{二十三  天志(1)上}
\chapter{二十四  天志中}
\chapter{二十五  天志下}
\chapter{二十六  明鬼(1)下}
\chapter{二十七  非乐(1)}
\chapter{二十八  非命(1)上}
\chapter{二十九  非命(1)中}
\chapter{三十  非命(1)下}
\chapter{三十一  非儒(1)下}
\chapter{三十二  大取(1)}
\chapter{三十三  小取(1)}
\chapter{三十四  耕柱(1)}
\chapter{三十五  贵义(1)}
\chapter{三十六  公孟(1)}
\chapter{三十七  鲁问(1)}
\chapter{三十八  公输(1)}
\chapter{三十九  备城门(1)}
\chapter{四十  备高临(1)}
\chapter{四十一  备梯(1)}
\chapter{四十二  备水(1)}
\chapter{四十三  备突(1)}
\chapter{四十四  备穴(1)}
\chapter{四十五  备蛾傅(1)}
\chapter{四十六  迎敌祠(1)}
\chapter{四十七  旗帜(1)}
\chapter{四十八  号令(1)      }   
\chapter{四十九  杂守(1)     }     
\chapter{附录: 一、经上}

附录: 一、经上

故,所得而后成也。 
止,以久也。 
体,分于兼也。 
必,不已也。 
知,材也。 
平,同高也。 
虑,求也。 
同长,以正相尽也。 
知,接也。 
中,同长也。 
■,明也。 
厚,有所大也。 
仁,体爱也。 
日中,正南也。 
义,利也。 
直,参也。 
礼,敬也。 
圜,一中同长也。 
行,为也。 
方,柱隅四讙也。 
实,荣也。 
倍,为二也。 
忠,以为利而强低也。 
端,体之无序而最前者也。 
孝,利亲也。 
有间,中也。 
信,言合于意也。 
间,不及旁也。 
佴,自作也。 
,间虚也。 
■,作嗛也。 
盈,莫不有也。 
廉,作非也。 
坚白,不相外也。 
令,不为所作也。 
撄,相得也。 
任,士损己而益所为也。 
似,有以相撄,有不相撄也。 
勇,志之所以敢也。 
次,无间而不撄撄也。 
力,刑之所以奋也。 

法,所若而然也。 
生,刑与知处也。 
佴,所然也。 
卧,知无知也。 
说,所以明也。 
梦,卧而以为然也。 
攸,不可两不可也。 
平,知无欲恶也。 
辩,争彼也。辩胜,当也。 
利,所得而喜也。 
为,穷知而县于欲也。 
害,所得而恶也。 
已,成、亡。 
治,求得也。 
使,谓、故。 
誉,明美也。 
名,达、类、私。 
诽,明恶也。 
谓,移、举、加。 
举,拟实也。 
知,闻、说、亲、名、实、合、为。 
言,出举也。 
闻,传亲。 
且,言然也。 
见,体、尽。 
君,臣、萌通约也。 
合,正、宜、必。 
功,利名也。 
欲正,权利;且恶正,权害。 
赏,上报下之功也。 
为,存、亡、易、荡、治、化。 
罪,犯禁也。 
同,重、体、合、类。 
罚,上报下之罪也。 
异,二、不体、不合、不类。 
同、异而俱于之一也。 
同、异交得放有、无。 
久,弥异时也。 
宇,弥异所也。 
闻,耳之聪也。 
穷,或有前不容尺也。 
循所闻而得其意,心之察也。 
尽,莫不然也。 
言,口之利也。 

始,当时也。 
执所言而意得见,心之辩也。 
化,征易也。 
诺,不一利用。 
损,偏去也。 
服,执誽、音利。 
巧转,则求其故。 
大益。 
儇,■秪。 
法同,则观其同。 
库,易也。 
法异,则观其宜。 
动,或从也。 
止,因以别道。 
读此书旁行,正无非。 


\chapter{二、经下}

二、经下

止,类以行人。说在同。 
所存与者,于存与孰存?驷异说。 
推类之难。说在之大小。 
五行毋常胜。说在宜。 
物尽同名:二与斗,爱,食与招,白与视,丽与,夫与履。 
一,偏弃之,谓而固是也。说在因。 
不可偏去而二。说在见与俱、一与二、广与修。 
无“欲、恶之为益、损”也。说在宜。 
不能而不害。说在害。 
损而不害。说在余。 
异类不吡。说在量。 
知而不以五路。说在久。 
偏去莫加少。说在故。 
必热。说在顿。 
假,必悖。说在不然。 
知其所以不知。说在以名取。 
物之所以然,与所以知之,与所以使人知之,不必同。说在病。 
无,不必待有。说在所谓。 
疑。说在逢、循、遇、过。 
擢,虑不疑。说在有、无。 
合与一,或复否。说在拒。 
且然,不可正,而不用害工。说在宜欧。 
物,一体也。说在俱一、惟是。 
均之,绝、不。说在所均。 
字,或徙。说在长宇、久。 
尧之义也,生于今而处于古,而异时。说在所义。 

二临鉴而立,景到。多而若少。说在寡区。 
狗,犬也。而杀狗非杀犬也,可。说在重。 
鉴位,景一小而易,一大则正。说在中之外内。 
使,殷、美。说在使。 
鉴团景一。 
不坚白。说在。 
荆之大,其沈,浅也。说在具。 
无久与宇坚白。说在因。 
以槛为抟,于“以为”,无知也。说在意。 
在诸其所然、未者然。说在于是推之。 
意未可知。说在可用过仵。 
景不徙。说在改为。 
一,少于二而多于五。说在建住。 
景二。说在重。 
非半弗■,则不动。说在端。 
景到,在午有端与景长。说在端。 
可无也,有之而不可去。说在尝然。 
景迎日。说在抟。 
正而不可担,说在抟。 
景之小、大。说在地正、远近。 
宇进无近。说在敷。 
天,而必正。说在得。 
行循以久。说在先后。 
贞而不挠。说在胜。 
一法者之相与也尽,若方之相合也。说在方。 
契与枝板。说在薄。 
狂举,不可以知异。说在有不可。 
牛马之非牛,与可之同。说在兼。 
倚者不可正。说在剃。 
循此循此,与彼此同。说在异。 
推之必往。说在废材。 
唱和同患。说在功。 
买无贵。说在仮其贾。 
闻所不知若所知,则两知之。说在告。 
贾宜则售。说在尽。 
以言为尽悖,悖。说在其言。 
无说而惧。说在弗心。 
唯吾谓非名也,则不可。说在仮。 
或,过名也。说在实。 
无穷不害兼。说在盈否知。 
知之、否之足用也,谆。说在无以也。 
不知其数而知其尽也。说在明者。 
谓辩无胜,必不当。说在辩。 
不知其所处,不害爱之。说在丧子者。 

无不让也,不可。说在始。 
仁、义之为内、外也,内。说在仵颜。 
于一,有知焉,有不知焉。说在存。 
学之,益也。说在诽者。 
有指于二,而不可逃。说在以二絫。 
诽之可否,不以众寡。说在可非。 
所知而弗能指。说在春也、逃臣、狗犬、贵者。 
非诽者谆。说在弗非。 
知狗,而自谓不知犬,过也。说在重。 
物甚不甚。说在若是。 
通意后对。说在不知其谁谓也。 
取下以求上也。说在泽。 
是是与是同。说在不州。 


\chapter{三、经说上}

三、经说上

故:小故,有之不必然,无之必不然。体也,若有端。大故,有之必无 
然,若见之成见也。 
体:若二之一、尺之端也。 
知材:知也者,所以知也,而必知,若明。 
虑:虑也者,以其知有求也,而不必得之,若睨。 
知:知也者,以其知过物而能貌之,若见。 
■:■也者,以其知论物,而其知之也著,若明。 
仁:爱己者,非为用己也,不若爱马,著若明。 
义:志以天下为芬,而能能利之,不必用。 
礼:贵者公,贱者名,而俱有敬僈焉。等,异论也。 
行:所为不善名,行也。所为善名,巧也,若为盗。 
实:其志气之见也,使人如己,不若金声玉服。 
忠:不利弱子亥。足将入,止容。 
孝:以亲为芬,而能能利亲,不必得。 
信:不以其言之当也,使人视城得金。 
佴:与人遇,人众,■。 
■:为是为是之台彼也,弗为也。 
廉:己惟为之,知其■也。 
所令:非身弗行。 
任:为身之所恶,以成人所急。 
勇:以其敢于是也命之,不以其不敢于彼也害之。 
力:重之谓。下、与;重,奋也。 
生:楹之生,商不可必也。 
平:惔然。 
利:得是而喜,则是利也。其害也,非是也。 
害:得是而恶,则是害也。其利也,非是也。 
治:吾事治矣,人有治,南北。 
誉之,必其行也。其言之忻,使人督之。 

诽:必其行也。其言之忻。 
举:告以文名,举彼实也。 
故言也者,诸口能之出民者也。民若画俿也。言也谓言,犹石致也。 
且:自前曰且,自后曰己,方然亦且。若石者也。 
君:以若名者也。 
功:不待时,若衣裘。 
赏:上报下之功也。 
罪:不在禁,惟害无罪,殆姑。上报下之功也。 
罚:上报下之罪也。 
同:二人而俱见是楹也,若事君。 
久:古今旦莫。 
宇:东西家南北。 
穷:或不容尺,有穷;莫不容尺,无穷也。 
尽:但止动。 
始:时或有久,或无久。始当无久。 
化:若蛙为鹑。 
损:偏去也者,兼之体也。其体或去或存,谓其存者损。 
儇:昫民也。 
库:区穴若,斯貌常。 
动:偏祭从者,户枢免瑟。 
止:无久之不止,当牛非马,若矢过楹。有久之不止,当马非马,若人 
过梁。 
必:谓台执者也,若弟兄。一然者,一不然者,必不必也,是非必也。 
同:捷与狂之同长也。 
心中自是往相若也。 
厚:惟无所大。 
圜:规写支也。 
方:矩见支也。 
倍:二尺与尺,但去一。 
端:是无同也。 
有间:谓夹之者也。 
间:谓夹者也。尺,前于区穴。而后于端,不夹于端与区内。及:及非 
齐之,及也。 
■:间虚也者,两木之间,谓其无木者也。 
盈:无盈无厚。于尺,无所往而不得,得二。坚异处不相盈,相非,是 
相外也。 
撄:尺与尺俱不尽,端与端俱尽。尺与或尽或不尽。坚白之撄相尽,体 
撄不相尽。端。 
仳:两有端而后可。 
次:无厚而后可。 
法:意、规、员三也,俱可以为法。 
佴:然也者,民若法也。 
彼:凡牛,枢非牛,两也,无以非也。 
辩:或谓之牛,谓之非牛,是争彼也,是不俱当。不俱当,必或不当, 

不若当犬。 
为:欲■其指,智不知其害,是智之罪也。若智之慎文也,无遗于其害 
也,而犹欲养之,则离之。是犹食脯也,骚之利害,未可知也,欲而骚,是 
不以所疑止所欲也。墙外之利害,未可知也,趋之而得力,则弗趋也,是以 
所疑止所欲也。观“为,穷知而县于欲”之理,养脯而非■也,养指而非愚 
也,所为与不所与为相疑也,非谋也。 
已:为衣,成也。治病,亡也。 
使:令,谓谓也,不必成;湿,故也,必待所为之成也。 
名:物,达也,有实必待文多也。命之马,类也,若实也者,必以是名 
也。命之臧,私也,是名也,止于是实也。声出口,俱有名,若姓宇洒。 
谓:狗犬,命也。狗犬,举也。叱狗,加也。 
知:传受之,闻也;方不障,说也;身观焉,亲也。所以谓,名也;所 
谓,实也;名实耦,合也;志行,为也。 
闻:或告之,传也;身观焉,亲也。 
见:时者,体也;二者,尽也。 
古:兵立反中,志工,正也;臧之为,宜也;非彼,必不有,必也。圣 
者用而勿必,必去者可勿疑。 
仗者两而勿偏。 
为:早台,存也;病,亡也;买鬻,易也;霄尽,荡也;顺长,治也; 
蛙买,化也。 
同:二名一实,重同也;不外于兼,体同也;俱处于室,合同也;有以 
同,类同也。 
异:二必异,二也;不连属,不体也;不同所,不合也;不有同,不类 
也。 
同异交得:于福家良,恕有无也;比度,多少也;免■还园,去就也; 
鸟折用桐,坚柔也;剑尤早,生死也;处室子子母,长少也;两绝胜,白黑 
也;中央,旁也;论行行行学实,是非也;难宿,成未也;兄弟,俱适也; 
身处志往,存亡也;霍,为姓故也;贾宜,贵贱也。 
诺:超、城、员、止也。相从、相去、先知、是、可,五色。长短、前 
后、轻重援,执服难成。言务成之,九则求执之。 
法:法取同,观巧。传法,取此择彼,问故观宜。以人之有黑者有不黑 
者也,止黑人;与以有爱于人有不爱于人,心爱人是孰宜? 
心:彼举然者,以为此其然也,则举不然者而问之。若圣人有非而不非。 
正:五诺,皆人于知有说;过五诺,若负,无直无说;用五诺,若自然 
矣。 


\chapter{四、经说下}

四、经说下

止:彼以此其然也,说是其然也;我以此其不然也,疑是其然也。 
□:谓四足兽,与生鸟与,物尽与,大小也。此然是必然,则俱。 
为麋同名,俱斗,不俱二,二与斗也。包、肝、肺、子,爱也。橘、茅, 
食与招也。白马多白,视马不多视,白与视也。为丽不必丽,不必丽与暴也。 
为非以人是不为非、若为夫勇不为夫,为屦以买衣为屦,夫与屦也。 
二与一亡,不与一在,偏去未。有文实也,而后谓之;无文实也,则无 

谓也。不若敷与美:谓是,则是固美也;谓也,则是非美;无谓,则无报也。 
见不见,离一二,不相盈,广修坚白。 
举不重不与箴,非力之任也;为握者之■(觭)倍,非智之任也。若耳 
目异。 
木与夜孰长?智与粟孰多?爵、亲、行、贾,四者孰贵?麋与霍孰高? 
麋与霍孰霍?■与瑟孰瑟? 
偏:俱一无变。 
假:假必非也而后假。狗,假霍也,犹氏霍也。 
物:或伤之,然也;见之,智也;告之,使智也。 
疑:逢为务则士,为牛庐者夏寒,逢也。举之则轻,废之则重,非有力 
也;沛从削,非巧也若石羽,循也。斗者之敝也,饮酒,若以日中,是不可 
智也,愚也。智与?以己为然也与?愚也。 
俱:俱一,若牛马四足;惟是,当牛马。数牛数马,则牛马二;数牛马, 
则牛马一。若数指,指五而五一。 
长宇:徙而有处宇,宇南北,在旦有在莫。宇徙久。 
无坚得白,必相盈也。 
在:尧善治,自今在诸古也。自古在之今,则尧不能治也。 
景:光至,景亡;若在,尽古息。 
景:二光夹一光,一光者景也。 
景:光之人煦若射。下者之人也高,高者之人也下。足敝下光,故成景 
于上;首敝上光,故成景于下。在远近有端,与于光,故景障内也。 
景:日之光反烛人,则景在日与人之间。 
景:木柂,景短大。木正,景长小。大小于木,则景大于木。非独小也, 
远近。 
临:正鉴,景寡、貌能、白黑、远近柂正,异于光。鉴、景当俱,就、 
去尒当俱,俱用北。鉴者之臭,于鉴无所不鉴。景之臭无数,而必过正。故 
同处其体俱,然鉴分。 
鉴:中之内,鉴者近中,则所鉴大,景亦大;远中,则所鉴小,景亦小。 
而必正,起于中,缘正而长其直也。中之外,鉴者近中,则所鉴大,景亦大; 
远中,则所鉴小,景亦小。而必易,合于中,而长其直也。 
鉴:鉴者近,则所鉴大,景亦大;其远,所鉴小,景亦小。而必正。景 
过正,故招。 
负:衡木,加重焉而不挠,极胜重也。右校交绳,无加焉而挠,极不胜 
重也。不胜重也。衡,加重于其一旁,必捶,权重相若也。相衡,则本短标 
长。两加焉重相若,则标必下,标得权也。 
挈:有力也;引,无力也。不正所挈之止于施也,绳制挈之也,若以锥 
刺之。挈,长重者下,短轻者上,上者愈得,下下者愈亡。绳直权重相若, 
则正矣。收,上者愈丧,下者愈得;上者权重尽,则遂。 
挈:两轮高,两轮为輲,车梯也。重其前,弦其前,载弦其前,载弦其 
轱,而县重于其前。是梯,挈且挈则行。凡重,上弗挚,下弗收,旁弗劾, 
则下直;扡,或害之也。流梯者不得流直也。今也废尺于平地,重,不下, 
无■也。若夫绳之引轱也,是犹自舟中引横也。倚:倍、拒、坚、■,倚焉 
则不正。 
谁:并石、累石,耳夹寝者,法也。方石去地尺,关石于其下,县丝于 

其上,使适至方石。不下,柱也。胶丝去石,挈也。丝绝,引也,未变而名 
易,收也。 
买:刀、籴相为贵。刀轻、则籴不贵;刀重,则籴不易。王刀无变,籴 
有变。岁变籴,则岁变刀,若鬻子。 
贾:尽也者,尽去其以不售也。其所以不售去,则售。正贾也宜不宜, 
正欲不欲,若败邦鬻室嫁子。 
无:子在军,不必其死生;闻战,亦不必其生。前也不惧,今也惧。 
或:知是之非此也,有知是之不在此也,然而谓此南北,过而以已为然。 
始也谓此南方,故今也谓此南方。 
智:论之非智,无以也。 
谓:“所谓非同也,则异也。同则或谓之狗,其或谓之犬也;异则或谓 
之牛,牛或谓之马也。俱无胜。”是不辩也。辩也者,或谓之是,或谓之非, 
当者胜也。 
无:让者酒,未让始也,不可让也。 
于:石,一也;坚、白,二也,而在石。故有智焉,有不智焉,可。 
有指:子智是,有智是吾所先举,重。则子智是,而不智吾所先举也, 
是一。谓“有智焉,有不智焉”,可。若智之,则当指之智告我,则我智之, 
兼指之以二也。衡指之,参直之也。若曰“必独指吾所举,毋举吾所不举”, 
则者固不能独指。所欲相不传,意若未校。且其所智是也,所不智是也,则 
是智是之不智也,恶得为一?谓而“有智焉,有不智焉”。 
所:春也,其执固不可指也;逃臣,不智其处;狗犬,不智其名也;遗 
者,巧弗能两也。 
智:智狗重智犬,则过;不重,则不过。 
通:问者曰:“子知驘乎?”应之曰:“驘,何谓也?”彼曰:“施。” 
则智之。若不问驘何谓,径应以弗智,则过。且应,必应问之时。若应长, 
应有深浅、大常中;在兵人长。 
所:室堂,所存也。其子,存者也。据在者而问室堂,恶可存也?主室 
堂而问存者,孰存也?是一主存者以问所存,一主所存以问存者。 
五合,水、土、火、火离,然火铄金,火多也。金靡炭,金多也。合之 
府水,木离木。若(识)麋与鱼之数,惟所利。 
无:欲恶伤生损寿,说以少连,是谁爱也?尝多粟,或者欲不有能伤也, 
若酒之于人也。且■人利人,爱也,则唯■,弗治也。 
损:饱者去余,适足,不害。能害,饱,若伤麋之无脾也。且有损而后 
益者,若疟病之之于疟也。 
智:以目见;而目以火见,而火不见。惟以五路智久,不当以目见,若 
以火见。 
火:谓火热也,非以火之热。 
我有若视,曰智。杂所智与所不智而问之,则必曰:“是所智也,是所 
不智也。”取、去,俱能之,是两智之也。 
无:若无焉,则有之而后无;无天陷,则无之而无。 
擢疑,无谓也。臧也今死,而春也得文,文死也可,且犹是也。 
且然,必然;且已,必已,且用工而后已者,必用工而后已。 
均:发均县轻重而发绝,不均也。均,其绝也莫绝。 
尧霍,或以名视人,或以实视人。举友富商也,是以名视人也;指是臛 

也,是以实视人也。尧之义也,是声也于今,所义之实处于古。若殆于城门 
与于臧也。 
狗:狗,犬也。谓之杀犬,可。若两■。 
使:令,使也。我使我,我不使,亦使我;殿戈亦使,毁不美,亦使殿。 
荆沈,荆之贝也。则沈浅非荆浅也,若易五之一。 
以楹之抟也,见之,其于意也不易,先智。意,相也。若楹轻于秋,其 
于意也洋然。 
段、椎、锥,俱事于履,可用也。成绘屦过椎,与成椎过绘屦同,过仵 
也。 
一:五,有一焉;一,有五焉;十,二焉。 
非■半,进前取也,前,则中无为半,犹端也。前后取,则端中也。■ 
必半,毋与非半;不可■也。 
可无也,已给,则当给,不可无也。久有穷而穷。 
正丸,无所处而不中县,抟也。 
伛宇不可偏举,字也。进行者,先敷近,后敷远。 
行者行者,必先近而后远。远近,修也;先后,久也。民行修,必以久 
也。 
一方尽类,俱有法而异。或木或石,不害其方之相合也。尽类犹方也。 
物俱然。 
牛狂与马惟异,以牛有齿,马有尾,说牛之非马也。不可。是俱有,不 
偏有,偏无有。曰之与马不类,用牛有角、马无角,是类不同也。若举牛有 
角、马无角,以是为类之不同也,是狂举也,犹牛有齿、马有尾。 
或不非牛而非牛也,则或非牛或牛而牛也可。故曰:牛马非牛也未可, 
牛马牛也未可。则或可或不可,而曰“牛马牛也未可”亦不可。且牛不二, 
马不二,而牛马二。则牛不非牛,马不非马,而牛马非牛非马,无难。 
彼:正名者彼、此,彼此,可。彼彼止于彼,此此止于此,彼此,不可, 
彼且此也,彼此亦可。彼此止于彼此,若是而彼此也,则彼亦且此此也。 
唱无过,无所周,若稗。和无过,使也,不得已。唱而不和,是不学也; 
智少而不学,必寡。和而不唱,是不教也;智而不教,功适息。使人夺人衣, 
罪或轻或重;使人予人酒,或厚或薄。 
闻在外者所不知也,或曰:“在室者之色,若是其色。”是所不智若所 
智也。犹白若黑也,谁胜?是若其色也,若白者必白。今也智其色之若白也, 
故智其白也。夫名,以所明正所不智,不以所不智疑所明。若以尺度所不智 
长。外,亲智也;室中,说智也。 
以悖,不可也。出入之言可,是不悖,则是有可也。之人之言不可,以 
当,必不审。惟:谓是霍,可,而犹之非夫霍也。谓彼是是也,不可。谓者 
毋惟乎其谓。彼犹惟乎其谓,则吾谓不行;彼若不惟其谓,则不行也。 
无:“南者有穷则可尽,无穷则不可尽。有穷、无穷未可智,则可尽、 
不可尽,不可尽,未可智。人之盈之否未可智,而必人之可尽、不可尽亦未 
可智,而必人之可尽爱也,悖。”人若不盈先穷,则人有穷也,尽有穷无难, 
盈无穷,则无穷尽也,尽有穷无难。 
不二智其数,恶知爱民之尽文也?或者遗乎其问也?尽问人,则尽爱其 
所问。若不智其数,而智爱之尽文也,无难。 
仁:仁,爱也;义,利也。爱、利,此也;所爱、所利,彼也。爱、利 

不相为内、外,所爱、利亦不相为外内。其为仁内也,义外也,举爱与所利 
也,是狂举也。若左目出,或目入。 
学也以为不知学之无益也,故告之也。是使智学之无益也,是教也。以 
学为无益也,教,悖。 
论诽:诽之可不可。以理之可诽,虽多诽,其诽是也;其理不可非,虽 
少诽,非也。今也谓多诽者不可,是犹以长论短。 
不诽,非已之诽也。不非诽。非可非也,不可非也。是不非诽也。 
物甚长甚短,莫长于是,莫短于是,是之是也非是也者,莫甚于是。 
取高下,以善不善为度。不若山泽,处于善于处上。下所请,上也。 
不是:是,则是,且是焉。今是文于是,而不于是,故是不文是不文, 
则是而不文焉。今是不文于是,而文与是,故文与是不文同说也。 


\end{document}