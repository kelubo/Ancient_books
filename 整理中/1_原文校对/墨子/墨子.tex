% 墨子
% 墨子.tex

\documentclass[12pt,UTF8]{ctexbook}

% 设置纸张信息。
\usepackage[a4paper,twoside]{geometry}
\geometry{
	left=25mm,
	right=25mm,
	bottom=25.4mm,
	bindingoffset=10mm
}

% 设置字体,并解决显示难检字问题。
\xeCJKsetup{AutoFallBack=true}
\setCJKmainfont{SimSun}[BoldFont=SimHei, ItalicFont=KaiTi, FallBack=SimSun-ExtB]

% 目录 chapter 级别加点(.)。
\usepackage{titletoc}
\titlecontents{chapter}[0pt]{\vspace{3mm}\bf\addvspace{2pt}\filright}{\contentspush{\thecontentslabel\hspace{0.8em}}}{}{\titlerule*[8pt]{.}\contentspage}

% 设置 part 和 chapter 标题格式。
\ctexset{
	part/name= {第,卷},
	part/number={\chinese{part}},
	chapter/name={第,篇},
	chapter/number={\chinese{chapter}}
}

% 图片相关设置。
\usepackage{graphicx}
\graphicspath{{Images/}}

% 设置古文原文格式。
\newenvironment{yuanwen}{\bfseries\zihao{4}}

% 设置署名格式。
\newenvironment{shuming}{\hfill\bfseries\zihao{4}}

% 注脚每页重新编号,避免编号过大。
\usepackage[perpage]{footmisc}

\title{\heiti\zihao{0} 墨子}
\author{}
\date{}

\begin{document}

\maketitle
\tableofcontents

\frontmatter
\chapter{前言}

《墨子》是中国文化中的一部奇书,也是一部寂寞的书。

鲁迅先生说:伟大也要有人懂。而伟大的《墨子》却在中国文化传统中,沉默了两千年,长时间在黑暗中的沉默,不仅影响了对其深层思想的诠释,甚至影响了对其浅层语言的理解,而且,也限制乃至取消了她对中华文化建构的发言权,墨子的思想与精神只好潜伏在中华文化的潜流之中,或沉默,或偶尔嗫嚅着发出微弱的声音。
然而,历史是公平的,一部真正伟大的作品可以暂时寂寞,但她不会永远寂寞,她终究会迎来发言的机会,而且,这一发言必然是黄钟大吕,天下耸动。转机来自于传统文化的变革,西学东渐的历程与新文化运动的勃兴,为古老的中国文化打开了新的视野,新的目光触及到了黑暗中的《墨子》,才惊讶地发现,她原本就焕发着夺目的光彩。
在清末,有一批认识了西方的学者对墨子作出了新的判断。邹伯奇提出了“西学源出墨学”的说法,他认为西方的天文、历法、算学等,都导源于《墨子》,并曾经依墨子的理论做过小孔成像的实验,制造过望远镜与我国历史上最早的照相机。张自牧在论说了墨家科技成就后说
“墨子为西学鼻祖”。王闿运认为《墨子》是西方宗教的源头,如佛家之释迦牟尼、基督教之耶稣都无官位俸禄而被奉为圣师,当受惠于墨学。郭嵩焘认为耶稣视人如己的教义正是墨家兼爱的意思。黄遵宪则从五个方面来论述这一命题:即西方的人权源于墨子的尚同;西方的独尊上帝源于墨子的尊天明鬼;西方的平等博爱源于墨子的兼爱;西学物理发达,源于《墨经》;西学长于器械制造,源于墨学备攻乃至于墨子造纸鸢之术。甚至得出“至于今日,而地球万国行墨之道者,十居其七”的结论……我们并不否认这些说法有“数人之齿而以为富”《墨子·公孟》的心理,但也要承认他们显然拥有了新的目光,并发现了墨子的价值。
在戊戌变法到五四时期,学人逐渐抛开了前者的夜郎心理,但对墨子的推崇却有增无减。《民报》创刊号卷首列古今中外四大伟人肖像,以墨子与黄帝、卢梭、华盛顿并列,被尊为“世界第一平等、博爱主义大家”。梁启超针对当时的国情,提出“今欲救之,厥惟墨学”的口号。爱国志士易白沙说:“周秦诸子之学,差可益于国人而无余毒者,殆莫过于墨子矣。其学勇于救国,赴汤蹈火,死不旋踵,精于制器,善于治守,以寡少之众,保弱小之邦,虽大国莫能破焉。”谭嗣同更为墨子精神的实践者,他不仅深念高望,私怀墨子摩顶放踵之志”而且能舍生赴死,慷慨就义,甘愿成为变革中不可避免的牺牲…
中国历史与中国文化崭新的一页,是伴随着墨子的被重新“发现”而缓缓打开的。

墨子《墨子》与墨家
历史总会给人留下种种的遗憾:对于墨子这样一个伟大的人,我们直到现在却依然所知甚少,甚至连他最基本的姓氏也难以确定。如元代伊世珍在《瑯环记》中引用《贾子说林》,称墨子并不姓墨,而是姓翟,因其母亲分娩前曾梦有乌鸦入室,醒来就生下了墨子,故取名为“乌”;清代周亮工《因树屋书影》卷十亦持此论;钱穆则认为墨子之所以叫墨子,是因为他是受了墨刑的囚徒;而胡怀琛与卫聚贤由认为他是印度人或阿拉伯人。这些奇怪的说法都表明了一个事实,那就是墨子的生平资料太少,其真实的面貌已经被湮没在茫茫的历史沙尘之中。而我们对于墨子的论述,只能依据学术界大体认可的说法来介绍:一般而言,人们认定,墨子姓墨名翟,当为春秋时鲁国人(亦有学者坚持其为宋国人或楚国人),即今山东滕州市。乃是宋襄公之兄的长子目夷的后代,此人因封于目夷,故名目夷子,而且夷原为商朝所建的同姓小方国:即在今滕州市内。
墨子的生卒年也是一个研究界莫衷一是的问题《史记·孟子苟卿列传》说“或曰并孔子时,或曰在其后”可见司马迁也已经不能知道墨子确切的生卒年了。而据学者的研究,大致可以推定墨子生于公元前 480年左右,卒于公元前390年左右。大约相当于孔子逝世后,孟子出生前的时代。
墨子的身份据其《贵义》中的记载,可知地位当较低
微《墨子·鲁问》《韩非子·外储说左上》里均曾记载墨子造车辖的事,后者甚至载其制木鸢,能在天上飞一天:由此可知,他也许曾从事过手工业,而且是一个能工巧匠。
当然,他的一生行事虽然没有明确的文献记载,,但我们从《墨子》一书中便可以看到大概。他与孔子一样,以救世解纷为已任,立说授徒,周游列国。平生足迹所至,曾向北到达齐国,向西到达卫国,多次游历楚国,到过郢都,到过鲁阳。亦曾劝阻鲁阳文君的攻郑,说服公输盘的谋宋等。而且,也多次推荐自己的弟子去做官,以此来推行自己的思想。
《墨子》一书,据《汉书·艺文志》记载,共有七十一篇,然而,现存的《墨子》已然不全了,只剩五十三篇。其中,有八篇有目无文,另有十篇既无目,亦无文。不过,可以知道均当为城守各篇的内容,
据先秦诸子的成书惯例,我们可以推测水墨子》一书也并非墨子一人所作。但是,具体哪些篇目是墨子所作,学术界还颇有歧见,但大体上,比较通达的是任继愈的看法,即从《尚贤》到《非儒》的十一组二十四篇当是墨子当年系统讲解自己的学说,后为弟子记录整理而成的;而《耕柱》、《贵义》等五篇则相当于墨子的语录,都可以当作墨子的著述来看。不过,像墨辩六篇、《亲士》等七篇及城守各篇则或为墨子弟子整理,或为墨家后学记录,如果都
看作墨子的作品也未尝不可。
据《韩非子·显学》记载:“世之显学,儒墨也。儒之
所至,孔丘也;墨之所至,墨翟也。”以此可知,在那个百家争鸣的辉煌时代,墨子所创立的墨家学派声势之浩大,超法逸道而直与儒家相抗衡。
《淮南子·要略》中说,墨子曾学儒者之业,授孔子之术,可见其最初是曾师孔学儒的,但是他对于孔子所主张的繁文缛节极为不满,故另为立说,从而走上了与儒学针锋相对的道路,他在产生之初就有与儒家争衡的意味而且,迅速崛起也当与其时之生产条件与社会关系有关正是在这样的基础上,他的影响越来越大,弟子也日益增多,从而形成了显赫的墨家学派。
《吕氏春秋·情欲》记载,儒家的弟子与墨家的弟子“从属弥众,弟子弥丰,充满天下”,又其《诚廉》中说“孔墨布衣之士也。万乘之主,千乘之君,不能与之争士也”这说明此时的墨家势力很大,然而,墨子弟子的情况却很少见于载籍,孙诒让的《墨学传授考》用尽心力,在《墨子》一书及先秦典籍中才钩沉出三十余人,即使据《墨子》书中所说,也不过有“臣之弟子禽滑鳌等三百余人”的记录:就这一点也可看出,墨家在那个时期其实已经开始衰微了。而在司马迁的《史记》中,不但把孔子列入世家,而且为孔子的弟子单独写了列传,可墨子本人也只有寥寥二十四个字,遑论墨家弟子。此后,墨家的宗教色彩越来越浓重,作为当世之显学的墨家,在秦代焚书坑儒的文化摧残之后,便也宣告衰落,而且,到了西汉,儒家复兴,墨家却未能东山再起。从这时起,一代之显学便成为了千古之绝学,进入了漫长的黑暗之中。

《墨子》的主要内容
墨子》一书的内容极为庞杂,大体可以分为三个部分
首先是体现墨家核心思想的文字:即《尚贤》以下十三篇专题论文中的十大主张,此前的《亲士》七篇所反映的思想都可以在其中找到更详尽与全面的论述,而《耕柱》《贵义》等五篇虽零散,但所论也不出这些主题的龙罩。
这十大主张又大致可以分为四类
一是伦理思想,也是墨子学说的理论基础,即兼爱:墨子认为,当时的整个社会之所以有这么多的问题,如人与人之间的相互残害、家与家之间的相互掠夺、国与国之间的相互攻伐,乃至于君臣间的不忠诚、父子间的不慈孝、兄弟间的不和睦……其最为核心的原因就在于人与人之间没有一种无差等的爱,如果人们能够做到兼爱:那么就会“强不执弱,众不劫寡,富不侮贫,贵不敖贱,诈不欺愚,凡天下祸篡怨恨,可使毋起”,从而达到天下的大治。而如果世人都兼爱了,就会互利互惠,并因此而达到非攻;因兼爱天下百姓而讲节用、节葬和非乐,并用天志说来限制人的浪费;以明鬼为推行兼爱的手段,并打破天命论对于兼爱的阻碍。
其实,儒家也是讲“爱”的,所谓“仁者爱人”即为此意,但儒家的爱是以“亲亲”为基础的,是有差别的,而墨子的兼爱却是无差等的爱,是所有的人之间互相平等的
爱。虽然,也许我们会觉得这种理论空想成分过多,但是,却也不得不承认,爱,永远是人类烟水苍茫的历史长河中熠熠闪烁的粼粼波光。

二是政治思想:即尚贤、尚同、非攻。一个社会的政治状况虽然受生产力状况与社会发展状况的制约,但是,统治者与各级当政者的个人品质及特点也无疑是其中最为重要的因素。所以,一种政治体制,其最为核心的政治活动便是官吏的选拔。而墨子所认定的“为政之本”就是尚贤,他的尚贤极为彻底,打破了封建社会的等级观念,唯贤是举。仅此而言,其思想之高远与宏达已远远超出同时代的思想家。更何况墨子在此篇及后边的《尚同》篇中也隐约表达出帝王也当由此途径而出的意思,这更是石破天惊之论,有人把他当作西方民主政治的前源也不足为怪。当然,此后墨家学派之所以从显学而变为绝学,这也是其重要原因之一,因为这从根本上危及了统治者的地位。
尚同则是要讨论下级对上级的服从。墨子认为一里之人要统一于里长,一乡之人要统一于乡长,一国之人要统一于国君,而天下之人要统一于天子,正是在这样的政治幻想中,墨子把全天下组织成了一个纲举目张、有条不紊的系统。只要能够达到以上级的是非为是非,就会统一而不会产生混乱,这一主张也反映出墨家理想化而又简单化的大同愿望。当然,墨子也考虑到了这种主张理论上的漏洞,所以,要联系他的尚贤论与天志论来理解:从兼爱观念出发,在政治思想上,墨子还极力主张非
攻。我们知道,在墨子生活的春秋战国时期,也恰是中国历史上战争最为频繁的时期,而墨子不仅从他的理论基石--兼爱出发,也从当时的社会现实出发,充满愤怒地论述了攻国之不义,并以层层深入的比喻来论证“窃钩者诛,窃国者侯”的荒谬。不过,我们还应当看到,墨子并非迂腐的说教者,他对春秋战国时期的现实极为清醒,他知道只凭借道德上的良好愿望与自律幻想是不可能阻止战争的,所以,与他非攻相辅而行的还有他卓越的军事主张。
三是经济思想:即节用、节葬、非乐,
其实,如果可以脱略主张的具体内容而只抽象看待的话,墨子的主张中,最有永恒意义并在每个时代都有可行性的便是节用。这其实也是他经济思想的核心。而就墨子所处的时代而言,节用的主张亦更显得重要,当时的生产力水平比较低下,人类所能创造出来的生活物资较少,提倡节约在某种程度上就相当于在创造价值,基于此,墨子认为,人类所有的消费,都应该以满足最为基本的自然需求为限,如食能果腹,衣可御寒,杜绝-切无益实用的消费。其实,这也是针对儒家的各种繁琐规定而发的。
节葬算是节用的一个分支,不过,儒家厚葬久丧之礼过于不切实用,但却流风所及,遍被士林,所以墨子将此单独提出详为论列。儒家的厚葬久丧在墨子看来,是完全没有必要的浪费。所以,墨子针锋相对提出节葬的主张,对于保存当时社会的生产力、增进社会财富而言,是极有意义的。而且,相对于儒家的主张,墨子所说的“衣食者,人之生利也,然且犹尚有节;葬埋者,人之死利也夫何独无节于此乎”,显得如此剀切而通达。
非乐其实是节用的外化。当然,从其行文中可以看出,墨子并非不能欣赏音乐的美,他的这一主张其实有很深远的考虑,那就是在当时的社会生产力条件下,王公大人对于声乐之美的追求,只会造成“亏夺民衣食之财”的后果。这不但是当时社会物质生产极端匮乏下的一种无奈之举,也是墨子对于当时社会的两极分化的一种批判。因为,统治者在衣食无忧的情况下沉酒声色,但这种行为却是以民众的牺牲为代价的。
四是宗教思想:即天志、明鬼、非命。
如果说兼爱是墨子从人世间筛选出来的理论基石和核心的话,那么,天志观则是墨子思想的原动力,是逻辑起点。他认为,上天是有意志的,而其意志主要表现为“天欲义而恶不义”和“天之爱天下之百姓”。其之尚同、兼爱、非攻等篇的推理无不以此为起点,而尚贤、节用、节葬也都通过圣王而间接源于此,
明鬼的论点也体现出墨子以唯心主义的外壳来装饰其改造社会的良苦用心,他不过想借此来整顿社会秩序。

他天真而且很可爱地设想,如果所有的人都能相信鬼神可以施福降灾、赏善罚恶,从而为全社会产生一种共同的约束力,就能达到天下大治。但他根本没有想到,这个说法本身已经暴露了他对于鬼神存在的怀疑。
非命的观点是在与儒家的争辩及社会生活的实践中
提出的:儒家的“生死有命,富贵在天”对于广大的民众而言是一针麻醉剂,也是墨子所说的“繁饰有命以教众愚朴之人”的阴谋;同时,天命思想在社会生活中也体现出其消极的特点,对于人类的创造性有深深的损伤。而墨子在社会生活中是一个态度积极的人,他认为,所有的事情,之所以做得好,是因为个人的努力,只有每个人都尽力了,社会才会发展。在后边的《鲁问》中,记载了墨子与其弟子彭轻生子的一段对话,就可以看出墨子对于人类自己努力的自信,这也正是人类能以自在的状态生存在这个世界上所必须的强烈自信。
其次是《墨经》所包含的与社会科学乃至于自然科学有关的知识。这一部分内容十分复杂,仅以谭戒甫的《墨经分类译注》为纲,即可分出十二种学科门类,何况此书并未包括《大取》《小取》二篇。《墨经》代表了先秦时代在各个学科所取得的成就,有许多成就的取得令人极为惊讶。如其在自然科学上所取得的成就,杨向奎曾评价说:“-部墨经,无论在自然科学哪一方面,都超过整个希腊,至少等于整个希腊。”
第三类是其军事思想。墨子十大思想中最为主要的是“兼爱”和“非攻”,但是,墨子并非当时以为礼乐便可安国的腐儒。对于当时的社会状况,墨子是极为清醒的,他清楚地知道,反对攻伐,仅仅靠道义的感召与理论的说服远远不够,正如鲁迅所说:“一首诗吓不走孙传芳”,所以,一个和平主义者,也要有坚强的力量来作为和平的保障乃至于砝码。因此,墨子》自《备城门》以下,全是有关军
事的内容,这些篇目从某种程度上可以看作是一部杰出且实用的“墨子兵法”
三对《墨子》的研究
从墨家以清新嘹亮的声音加入百家争鸣的大合唱时,就开始有人对其进行了研究。如孟子指责墨子兼爱的主张是“无父”,故诋之为“禽兽”,但也承认其“摩顶放踵利天下”的行为;苟子批评墨子“敝于用而不知文”;庄子在其《天下》篇中,论述墨子“意则是,其行则非”,然而,也充满同情地说墨子“真天下之好也,将求之不得也,虽枯槁不舍也,才士也夫”!汉代司马迁父子、王充、班固等人也均对墨子发表了意见。
直到墨子之后大致五百年,西晋学者鲁胜曾对《墨子》中的《墨辩》四篇进行了注释,此书是中国历史上可知的最早的《墨子》注本,虽然,此书现已佚,但其序还保存在《晋书》中。此后又经过了四五百年,在唐代产生了乐台的注本,但也早已失传。
唐代以继儒家道统为已任的韩愈曾写过《读墨子》-文,其文竟有“孔子必用墨子,墨子必用孔子,不相用不足为孔墨”之语,其实是有深远的考虑的,清代学术大师俞樾有“乃唐以来,韩昌黎外无一人能知墨子者”之语,亦得其实。
不过,对《墨子》真正意义上的研究,是从清代开始的。清代初期,傅山做《墨子大取篇释》,虽仅对《墨子》中的《大取》一篇进行训释,但却成为清代墨学复兴的第-箭阳光。此后,文学家汪中曾用六年时间校注《墨子》,当有所获,可惜其书却未能流传,而据其所流传下来的《墨子序》与《墨子后序》可以看出,他“不但为墨子辨千古之枉曲,而且把儒墨显学并称的历史首先指示出来,一扫千年来异端的诬蔑”侯外庐语)。几乎同时的毕沅在几千年的历史中,第一次对《墨子》全文进行了认真而富有成效的注释与读解工作,其十六卷的《墨子注》也成为《墨子》整理史上承前启后的力作。到了清代后期,终于产生了《墨子》整理史上空前的巨著:孙诒让的《墨子间诂》。此书以毕沅的《墨子注》为蓝本,以清代四十余家研究墨子的著作为参照,详为推考,以数十年功力,成此两千年墨学研究的集大成之作。梁启超在《中国近三百年学术史》中评价说:“大抵毕注仅据善本雠正,略释古训;苏氏始大胆刊正错简;仲容(即孙诒让)则诸法并用,识胆两皆绝伦,故能成此不朽之作。……其《附录》及《后语》,考订流别,精密闲括,尤为向来读子书者所未有。盖自此书出,然后《墨子》人人可读。现代墨学复活,全由此书导之。古今注《墨子》者固莫能过此书,而仲容一生著述,亦此书为第一也。”
据统计,清代大致有六十种墨学研究专著,而现代的三十年就产生了大约一百种,数量激增,研究的质量也很高。就拿全书整理本而言,就出现了两部极有特点的全注本。一是张纯一的《墨子集解》,此书为作者积十数年之功写成的,他吸收了孙诒让《墨子间话》未及收入的成果及其产生后问世的成果,并能参以已意,时有新说,于
句意、段意和篇意有通达的解说与发挥,虽然校勘粗略但材料宏富,解说尤详。二是吴毓江的《墨子校注》,此书最大的功绩在于校勘,作者积二十年之功,对于现存的古代《墨子》版本,几已网罗无遗,共用一种唐本、十四种明本、两种清本,此外还多从类书与古注中搜集引文以作比证,进行了孙诒让、王念孙诸人所未措意的文字校勘问题,全书后也有丰富的附录资料。而岑仲勉的《墨子城守各篇简注》则生面别开,以《孙子兵法》为背景来评价墨子城守各篇的军事价值,并能结合后世器具实物、古代兵书与兵图、古代战例来解释书中的各种器物。
建国以后,墨子的研究更是蓬勃开展,据统计约有二百种研究专著问世。其中,在《墨经》的整理方面有谭戒甫的《墨辩发微》与《墨经分类译注》、高亨的《墨经校诠》等。而在《墨子》全书文本的训释上,王焕镳耗十年心血而成百万言的《墨子集诂》成为《墨子》笺注史上又一部集大成的巨著,其书除去了《墨辨》六篇与城守各篇,仅释所余之三十六篇,以孙诒让《墨子间诂》为底本,并参照其所能搜罗到的诸家意见,择善而从,间出己意,不但是《墨子》整理史上引书最多的一家,而且,在校释上也多有发明。
本书的整理情况四
本书为《墨子》的一个选本。但由于《墨子》一书的特殊性,此选本在某种程度上亦有全本之功。如《墨子》卷-从《亲士》到《三辩》的七篇及《耕柱》《贵义》等五篇
都全部入选,不做删节。而中间从《尚贤》到《非儒》的十一组中,每组均当有三篇文章,内容基本相同,甚至措辞与事例都极相近,故被学者认为是“墨分为三”后弟子传述不同的结果。因此,这十一组文章,仅选其最为完整明晰的一篇。当然,也有个别例外,如《非攻》,本来《非攻中》论述更为全面,但《非攻上》之行文简洁严谨,层层设喻,是最典型的墨子文章,故舍“中”而选“上”;再如《节葬》、《明鬼》、《非乐》《非儒》四组都各仅存一篇,故无可选择。《墨经》内容庞杂,涉及了许多专业知识,且研究界也歧见迭出,故仅选其与光学有关的八条,以见一斑而已。城守各篇亦多涉及防守的方法与器械,疑晦难明处很多,故仅因《公输》之云梯而选《备梯》一篇以尝鼎一脔。
本书以王焕镳的《墨子集诂》为底本,因其书为集解性质,故所收极为丰富,一册在手,众善毕集,可以参照诸家,择善而从。而且,王之按断亦多精义,许多前人莫衷一是的问题,他都有别具手眼的考论。当然,也有个别地方似未得当,则参酌吴毓江《墨子校注》或张纯-《墨子集解》甚至孙诒让《墨子间诂》正之,偶尔也参以已意。此外,谭家健的《墨子选译》严谨而得当,故亦有所取资。关于正文,一般情况,以底本为主,不作改动,以尊重原貌。
有两种情况则在正文上直接改正而不作说明:-、王焕镳《墨子集诂》乃以孙诒让《墨子间诂》为底本,若孙本印误,王仍存其旧,仅在注中说明者,如此则改之;二、孙本又以毕沅《墨子注》为底本,毕本有不少无意之舛误,孙
本与王本仍之而未改者,亦改之。另有一种情况,即正文文字明显有误,研究者亦多指出者,如有确切的版本依据,则改动正文,并于注中说明。如果有研究者对一些文字有新的看法,虽近真却无版本依据者,不改动原本,只在注中说明当改为某某,译文也以校改后的文字为准。
关于注释。本书的注释尽量简明,一般通过译文可以了解的字词,便不再作注;一般性常识,如墨子常常引及的古代贤君与暴君的事例,在第一次出现时加注,其后则不再加注,个别前详后略,以助阅读
故本书之注约有五种:一是难字需注音者,二是难理解的字词与文化常识性的内容,三是难理解的语句需串释者,四是通假字,五是校改说明。
关于译文。为忠于原文,本书译文以直译为主,同时也尽量做到晓畅通达。而且,一些字词并未设注,实在译文中已有所体现;而个别语句极为复杂者,译文仍以直译为之,难解之处则在注中说明,以助理解,并使注与译可以交相为用。
此外,全书正文对话与引用极多,层次过于繁复,为避淆乱,兹从王焕镳书不用引号标识,译文亦从此。其实,全书在各段对话与引用前均有相应提示语,不用引号亦可明白,如用引号,反徒生滋扰而已
最后,此书的完成,除前文所及外,还借鉴了墨学研究界众多学者的研究成果,如任继愈《墨子》杨俊光《墨
子新论》、邢兆 良《墨子评传》、苏凤捷、程梅花《平民理想--〈墨子〉与中国文化》郑杰文《20 世纪墨学研究史》等;而且亦幸得张廷银老师的帮助与指导,在此深表谢忱。

春秋战国是中国古代社会从宗法贵族制向官僚地主制过渡的大变动时 
代。作为这种社会大变动的表现和结果,在当时涌现出许多思想主张互不相 
同的学派。其中影响最大的有二:一为孔子开创的儒家,一为墨子开创的墨 
家。它们在战国时期并称为当世的显学。《韩非子·显学》篇说:“世之显 
学,儒墨也。” 
墨子名翟,鲁国人。生卒年具体不详。但从历史文献来看,我们可以断 
定,墨家的产生当在儒家之后。据《淮南子·要略》之说,墨子原为儒门弟 
子,后因不满儒家学说而另创一对立的学派: 
墨子学儒者之业,受孔子之术,以为其礼烦扰而不说,厚葬靡 
财而贫民,(久)服伤生而害事,故背周道而用夏政。 
由此看来,墨家学说乃是墨子对儒家学说进行反思和批判的产物。从我 
们今天所能见到的《墨子》一书确实不难看到,墨家学派对儒家从周代贵族 
社会继承下来的礼乐等文化形式进行了大量的攻击,如《墨子》书的《节葬》、 
《节用》、《非乐》、《非儒》等,都可以说是直接针对儒家学说而发。因 
此,《淮南子》的论断并非无稽之谈。当然,应当指出的是,《淮南子》的 
作者把墨家学说的兴起归之为夏政的复活,则有失于简单。诚然,在《墨子》 
书中,夏禹被塑造成一位与儒家所宣传的礼乐文化背道而驰的古代圣王。但 
是,这个形象并不是古代历史的客观反映,而主要是墨家理想的象征。墨家 
要借助夏禹来压服儒家所声称的祖师爷文王、周公。事实上,在《墨子》书 
中,夏禹、商汤、文王都是被列为古代圣王的人物,并不是相互对立的。因 
此,与《汉书·艺文志》的百家出于王官说一样,《淮南子》的墨家“用夏 
政”说也是一种想当然的皮相之见。 
墨家作为一个与儒家对立的新生学术政治团体而出现在儒家声势浩大 
之时,它不仅站稳了脚跟,而且获得了与儒家平分秋色、甚至后来居上的地 
位,这用复古说恐怕是无法解释的。墨家与儒家并称为显学。所谓显学,包 
括两个方面的含义,一是队伍壮观,声威显赫,一是仕途通达,君主信任。 
而要做到这两点,它就必须让统治者和被统治者都可以从它的学说中看到对 
自己有利的东西。而要做到这一点,关键在于和历史前进的方向达到一致。 
墨家之所以能够在战国前期异军突起,其原因即是它比儒家更能抓住战国初 
期社会发展的新形势,提出了一些儒家所没有提出的社会学说和政治方案, 
从而引起了当时自君主到庶民等阶层的强烈兴趣。 
儒家诞生的春秋时代,官僚地主制与宗法贵族制两种新旧社会制度的交 
替尚处于一种潜在的温和状态。宗法贵族集团仍处于社会的统治地位。因此, 
作为新生社会力量的代表,孔子的思想虽然已超越宗法贵族时代而进入到官 
僚地主时代,但孔子的新思想却依然披着贵族社会旧文化的外衣。他希望通 
过对贵族文化进行输血式的改造而促成社会制度的变革。这样一来,孔子的 
思想便不可避免地带有温情主义和维新主义的色彩。一方面,孔子虽然主张 
贤人政治和平民参政,但他并不想冲击贵族阶级的既得利益,仍希望“贵可 
以守其业”;另一方面,孔子虽然以新的社会理想对神、礼、德等贵族文化 
的核心内容进行了超越和改造,但他并没有提出一套全新的政治方案,他所 
追求的有道之世乃是一个十分模糊的概念。而且,由于孔子把道德价值强调 
到独一无二的至高地位,将理想的实现寄希望于执政者的道德自律,这就使 

他的思想又带有强烈的理想主义色彩。孔子去世以后,新旧社会制度交替的 
潜在状态突然被打破。在先后左右着春秋时代的政治局势的齐、晋两个大国, 
代表新兴的官僚地主社会方向的田氏与韩、赵、魏三家分别取代原先由周天 
子所分封的齐、晋诸侯而建立了新的政权。在春秋时代尚处于统治地位的传 
统宗法贵族文化终于退出了历史舞台。明、清之际的著名学者顾炎武在《日 
知录》中对孔子死后百余年间的这种历史剧变曾有一精彩概括: 
自《左传》之终以至此(指周显王三十年),凡一百三十三年, 
史文阙佚,考古者为之茫昧。如春秋时犹尊礼重信,而七国则绝不 
言礼与信矣。春秋时犹宗周王,而七国则绝不言王矣。春秋时犹严 
祭祀、重聘享,而七国则无其事矣。春秋时犹宴会赋诗,而七国则 
不闻矣。春秋时犹有赴告策书,而七国则无有矣。邦无定交,士无 
定主,此皆变于一百三十三年之间。史之阙文,而后人可以意推者 
也,不待始皇之并天下,而文武之道尽矣。(《日知录集释》卷十 
三) 
总之,进入战国时代以后,周代贵族社会的各种制度全部被破坏。在这 
种新的历史背景下,孔子那种以宗法贵族文化的旧瓶装官僚地主社会之新酒 
的维新做法,无疑是落后而跟不上形势了。因此,结合贵族社会行将灭亡这 
种新的历史环境而对儒家学说进行反思和改造,又成为智识阶层所面临的一 
项历史使命。墨家学说即由此应运而生。 
从历史的眼光来看,就今天所见到的《墨子》而言,墨家学说比儒家更 
能符合战国时代社会发展的内容,至少表现在如下几个方面: 
首先,墨家学派明确提出了兼爱、尚贤的平民政治理论,把孔子提出的 
爱人和举贤思想推向了一个更新的高度,从而在理论上彻底打破了贵族阶级 
以亲亲为原则的血缘贵贱论。儒家创始人孔子为了给平民阶级争取更多的政 
治权力,在春秋末期提出了爱人(仁者爱人)和举贤的思想。这对于贵族阶 
级的血缘贵贱论来说,无疑是一个具有新的时代精神的创举。但是,由于当 
时宗法贵族势力仍然处于统治地位,平民阶级大规模地向贵族阶级争取平等 
权力的运动刚刚形成气候,因此,在孔子所具有的政治思想中,还不可能产 
生彻底打破贵族特权的认识。他所做的只能是要求贵族阶级将其特权向平民 
开放。要求贵族统治集团将普通民众当作与自己同样的人看待的爱人思想与 
要求贵族阶级从普通民众中选拔贤能参政的举贤思想所体现的即是这样一种 
历史特点。在主张爱人和举贤的同时,孔子并不反对对贵族阶级的利益予以 
照顾,部分地保留其特权,讲究“故旧不遗”。而且,平民阶级要获得“举 
贤”的机会,在事实上还必须经过一个贵族化的过程,先掌握以诗、书、礼、 
乐为代表的贵族文化。 
然而,在墨子的时代,平民阶级争取与贵族阶级平等的政治权力的斗争 
已经基本取得胜利,孔子所提出的爱人和举贤思想完全成为现实。因此,作 
为新时代的平民思想家,墨子必然要提出比孔子更为激进的平民革命思想。 
于是,孔子的爱人和举贤便被兼爱和尚贤所取代。墨子的兼爱与孔子的爱人 
之区别在于:爱人并不否定亲亲;而兼爱则实际上取消了亲亲,主张将他人 
与自己的亲人一样看待。《墨子·兼爱下》云: 
姑尝本原若众害之所自生。此胡自生?此自爱人利人生与?即 
必曰非然也,必曰从恶人贼人生。分名乎天下恶人而贼人者兼与别 
与?即必曰别也。然即之交别者,果生天下之大害者与!是故别非 

也。子墨子曰:“非人者,必有以易之,若非人而无以易之,譬之 
犹以水救火也。”其说将必无可焉。是故子墨子曰:“兼以易别。” 
然即兼之可以易别之故何也?曰:藉为人之国,若为其国,夫谁独 
举其国,以攻人之国者哉?为彼者由为己也。为人之都,若为其都, 
夫谁独举其都,以伐人之都者哉?为彼犹为己也。为人之家,若为 
其家,夫谁独举其家,以乱人之家者哉?为彼犹为己也。然即国都 
不相攻伐,人家不相乱贼,此天下之害与?天下之利与?即必曰天 
下之利也。 
在墨子的认识中,兼与别相对而言。所谓别,从政治的角度而言,无疑 
指的是周代社会以血缘和种姓为依据而确立的各种等级关系,也即周公制定 
的那种礼乐制度,而兼则是要废除这种礼乐等级制度,消除这种嫡庶亲疏观 
念。应该说,这即是兼爱的历史本质。 
血缘亲疏关系被彻底抛弃以后,贵族阶级凭藉出身而高处显贵地位的世 
官制度已完全没有存在的理由。因此,在墨子的思想中,一视同仁地从全体 
国民中选举贤能便成为唯一的仕官途径。只有贤能才是唯一有资格入仕为官 
和受人尊敬的人。而且,在墨家学派这里,贤能之士的入仕为官已不再需要 
经过贵族化的修养准备。如《墨子·尚贤上》云:“故古者圣王之为政,列 
德而尚贤。虽在农与工肆之人,有能则举之。高予之爵,重予之禄,任之以 
事,断之以令。”主张直接从农夫与工商小民中选贤任职,并举例云: 
故古者尧举舜于服泽之阳,授之政,天下平。禹举益于阴方之 
中,授之政,九州成。汤举伊尹于庖厨之中,授之政,其谋得。文 
王举闳夭、泰颠于罝罔之中,授之政,西土服。故当是时,虽在于 
厚禄尊位之臣,莫不敬惧而施;虽在农与工肆之人,莫不竞劝而尚 
意。 
舜、益、伊尹、闳夭、泰颠等人本来都是从事各种卑贱职业的体力劳动者, 
但却被尧、禹、汤、文王等最高统治者分别拔举到国家最高行政长官的地位。 
这就是墨子所理解的尚贤。 
与孔子那种学而优则仕的举贤思想相比,墨子这种从劳力者中选拔劳心 
者的观点无疑更为受到普通民众的欢迎。 
其次,墨家学派明确提出了尚同和如何成为天子的问题,把孔子“为东 
周”的梦想提到了改朝换代的高度,为建立一个取代周王朝的新的统一的中 
央集权王朝提供了舆论基础。 
在孔子所生活的春秋时代,周王朝虽然已经日薄西山、奄奄一息,但周 
天子在名义上仍然是诸侯们所承认的天下共主。在激烈的争霸斗争中,尊王 
一直是霸主们争取霸主地位的手段。为了让其他诸侯国承认自己的霸主地 
位,霸主们总要做出一些维护周王朝和周天子之体面的行动。所以,孔子虽 
然认识到应当创立一套超越周代贵族文化的地主官僚文化,但却还没有形成 
建立新王朝的明确的革命思想。他在周游列国的时候,还没有和诸侯们讨论 
如何为王、为天子的问题。 
到墨子时,随着周王朝地位的日趋下降,实力雄厚的诸候已不再满足于 
做诸侯之长,而是希望取代周天子而成为诸候之王。争夺霸主地位也不再成 
为他们的目标。以此为背景,墨家学派便提出了王、天子这样一些孔子尚未 
提出的时代主题。例如:《墨子·亲士》云: 
圣人者,事无辞也,物无违也,故能为天下器。是故江河之水, 

非一源之水也;千镒之裘,非一狐之白也。夫恶有同方取不取同 
而已者乎?盖非兼王之道也!是故天地不昭昭,大水不潦潦,大 
火不燎燎,王德不尧尧者。 
从王道的高度来研究政治,是《墨子》一书的特点。又如《墨子·尚贤 
中》云: 
今王公大人欲王天下、正诸侯,夫无德义,将何以哉?其说将 
必挟震威强。今王公大人将焉取挟震威强哉?倾者民之死也!民 
生为甚欲,死为甚憎。所欲不得,而所憎屡至。自古及今,未有 
尝能有以此王天下、正诸侯者也。今大人欲王天下、正诸侯,将 
欲使意得乎天下,名成乎后世,故不察尚贤为政之本也?此圣人 
之厚行也。 
如此明目张胆地用王天下、正诸侯来游说当世王公大人,恐怕是春秋时代的 
孔子所不敢做的。 
从《墨子》全书来看,墨家学派所提出的王天下、正诸侯的途径主要包 
括兼爱、尚贤、非攻、尚同等等。这基本都是从广大被统治者的利益着眼的。 
对当时的君主来说,这种理想主义的政治显然是无法付之实践的。但是,对 
当时这些纷纷自称为王的君主们来说,墨家的这种王天下、正诸侯的说法无 
疑比儒家的尊王、复礼之论悦耳动听得多。 
第三,墨家学派明确提出一种功利主义的政治哲学,这比儒家的道德政 
治更为符合统治者选拔人才的心理和任用人才的原则。例如,《墨子·亲士》 
篇云: 
故虽有贤君,不爱无功之臣;虽有慈父,不爱无益之子。是故 
不胜其任处而其位,非此位之人也;不胜其爵而处其禄,非此禄 
之主也。 
墨子说国君不会喜欢无功之臣,慈父不会喜欢无益之子,其目的固然在于攻 
击贵族集团的尸位素餐,但与此同时,他在这里实际上提出了一种具有划时 
代意义的人才思想。 
如前所述,在周代贵族社会,贵族阶级步入仕途依靠的是自己的出身地 
位。孔子为了为平民阶级的入仕创造理论条件,便提出了以道德为标准来选 
择人才的思想。这在当时无疑有着巨大的历史意义。到墨子的时代,新一代 
的君主已基本接受从各阶层中选拔人才的观念,但是,在当时激烈的竞争斗 
争中,君主们所需要的并不是孔子所推崇的道德君子,而是能为国家建功立 
业的谋臣策士。因此,墨子以功代替孔子的德来评价人才,自然更符合最高 
统治者的口味。 
与这种功利主义的人才观相应,墨家学派的思想也普遍具有功利主义色 
彩。墨家在论述其各种观点的时候,往往要从功利的角度论证其必要性。他 
们认为人民的本性无常,只对于他们有利的人、事感兴趣。《墨子·七患》 
云: 
故时年岁善,则民仁且良;时年岁凶,则民吝且恶。夫民何常 
此之有!为者疾,食者众,则岁无丰。 
既然民性随生活环境而变迁,没有恒性,那么,孔子所主张的道德教化政治 
也就无从附丽了。墨子还明确指出,道德的力量是有限的,只能从属于物质 
力量。《七患》云: 
故仓无备粟,不可以待凶饥;库无备兵,虽有义不能征无义; 

城郭不备全,不可以自守;心无备虑,不可以应卒。 
因此,他认为,一个国家最重要的东西即是备、兵、城等物质基础: 
故备者,国之宝也;兵者,国之爪也;城者,所以自守也: 
此三者,国之具也。 
与儒家的那种重义轻利的道德高调相比,这种现实主义的态度与战国初期的 
君主对政治的认识自然一致得多。 
在墨家的政治思想中,以这种功利主义和现实主义的认识为基础,他们 
还提出了两个在战国时代具有重大影响的政治术语——“法”和“术”。在 
《尚贤中》一篇,墨子指出,治理国家必须讲究法术,说:“既曰若法,未 
知所以行之术,则事犹若未成。”法指的是治理国家必须坚持的基本原则, 
而术则是贯彻这种原则行之有效的方法。在本篇中,法和术实际上具体指应 
当任用贤才和如何使用贤才二者。他认为若能做到这二者,统治者即可收到 
“美善在上,而所怨谤在下;宁乐在君,忧戚在臣”的效果。这种重法明术、 
尊君卑臣的政治观点,无疑直接开启了战国时代法家思想的先河。 
第四,墨家还具体提出了非攻、节用、节葬、非乐等政治主张,对当时 
统治者贪得无厌的掠夺战争和穷奢极侈的享乐生活进行了广泛而尖锐的批 
判。这无疑喊出了处身于无休无止的战争之中并担负无穷无尽的租赋徭役的 
人们的心声。 
总而言之,从上述四个方面来看,墨家学说可以说是从战国时代的政治 
形势出发,站在普通民众利益上提出的一套系统的政治理论。当然,与此同 
时,我们还应当进一步看到,由于墨家学派的成员大都是一些出身下层阶级 
的“贱民”,他们本身并不具备多少政治经济地位,因此,当他们通过游说 
当时君主来推行自己政治主张的时候,他们就不得不借助于一些超人间社会 
的力量来维护自己的权威地位。由此出发,他们对上天和鬼神进行了大量宣 
传,把自己的主张说成是上天、鬼神的愿望。这就使墨家学说具有浓厚的迷 
信色彩。 
从散文艺术的角度而言,《墨子》在先秦哲理散文中以质朴无文著称, 
这一特点极为显著。墨家主张尚质,反对尚文。因此,他们著书立说所采用 
的是当时的口语,而不是儒家那种经过修辞的“文言”或“雅言”。这使他 
们的文章有一种平易近人、娓娓道来的风格特色,但因此也给后人的阅读和 
理解带来了一定的困难。我们在对《墨子》进行白话翻译的时候,对这一种 
困难感触尤深。特别是《墨子》中《备城门》以下诸篇,有许多当时的战略 
术语,古来注者向无确解,我们虽勉强为之转译,但恐怕未必符合作者的原 
意。至于《经上》、《经下》、《经说上》、《经说下》诸篇,乃是墨子及 
其弟子对逻辑学、数学、物理学等专门知识的研究成果的总结,文字记录具 
有隐微难懂、言此意彼之特点,若强为之翻译,只能弄巧反拙,因此,我们 
只录了原文。读者若对这部分有兴趣,可参考谭戒甫的《墨辨发微》与《墨 
经分类译注》。 
《墨子》一书,据《汉书·艺文志》记载,有七十一篇,现存五十三篇。 
古人对此书的整理研究工作始于清代。较为著名的成果有毕沅的《墨子注》 
与孙诒让的《墨子闲诂》等。我们即以二书为主而旁采其他诸家之说。本书 
的译注工作由多人合作完成。吴龙辉负责《亲士》至《明鬼》篇,及《备城 
门》篇,过常宝负责《非乐》至《非儒》篇,张宗奇负责《大取》至《公输》 
篇,黄兴涛负责《备高临》以下诸篇。由于我们水平所限,缺点错误定会不 

少,则有待于方家与读者的教正了。 
吴龙辉 一九九二年六月 


序

这套“先秦诸子今译”丛书,从时间上看,正赶上由《资治通鉴》白话 
本出版而激起的古文今译热潮。既是“潮”,那就该归为“显学”,这个名 
称总是不大入耳的。而且,在有的人看来,将典雅古奥的国粹糟塌成浅俗不 
堪的白话,无异于挖掘祖坟,粗鄙无道。只是这潮仍不可阻遏地热起来了, 
起码说,还有许多读者喜爱这种“下里巴人”的东西。我常想,既然老祖宗 
的东西各个时代都有人注,而且注得好的都成了大师,拿不大准的就多多益 
善地收罗先人的话充数,号称什么经“注我”;甚至自己不“注”一字,尽 
得风流,达到了“大美不言”的化境,不但免遭物议,反为同行相与乐道。 
那么,今天我们译成大白话,不妨也可以冒充成一种注罢。当然,大潮一起, 
免不了泛些泥沙残渣,恰如这套丛书免不了多有注译上的错讹一样;但潮落 
之后,大浪淘沙,或者会有精妙之作显露出来。 
先秦诸子的时代,在我国历史上是读书人人格相对独立,思想最活跃、 
少束缚的时代,也是一个异彩纷呈、硕果累累、最为辉煌璀璨的时代。可以 
说,这个时代奠定了中国文化的基础。组成我们民族文化核心的儒、道、释 
三大思想宝库,就有两家半(因为佛教也中国化了)兴起于先秦。可惜自那 
以后,中国历史上就再没有重现过同样令人激动和向往的“百家争鸣”的自 
由壮观的局面。先有暴君秦始皇因惧惮思想的伟力而“坑儒”,继以汉武帝 
为了“役心”的需要,采纳最长于给同类致命一击的董仲舒“罢黜百家,独 
尊儒术”的建议,百家终竟只尚一家,儒家变成了儒教。更可怕的是其后近 
两千年,儒教与封建政体结合,形成政教合一的形态,大大方便了统治者“动 
口”不行就“动手”,思想“教育”不奏效就施虐于肉体。于是,创造被扼 
杀了,“万马齐喑”成为不争之实。今天,欣逢大力提倡“思想再解放一点” 
的盛世,我们着手先秦诸子白话今译的工作,也是奢望以绵薄(精神的东西 
毕竟不如物质的来得直捷快当,此之谓“绵”;学养太浅,无能传其精髓达 
其要义,此之谓“薄”)之力,让更多的人了解我们祖国曾有过的光辉时代, 
让更多的人歆享我们祖先创造的精神文明,让更多的人汲取菁华、走出蒙昧, 
为中华的复兴增添一分力量! 
一个时期,反传统文化成为时尚。有的人动辄对传统文化大加挞伐,仿 
佛民国初闹革命,以为只要“咔嚓”一声将辫子剪掉,耳濡目染、浸淫五腑 
的封建污秽也随之而去。类似的“战斗”,从来没有成功过。“五四”时力 
倡“打倒孔家店”,现在不但没倒,香火还甚于从前。还有人辩护说那样做 
是为了“矫枉过正”,这不禁使人回忆起物质生产一“过正”就诞生“大跃 
进”的教训,我想精神文化的建设也不会例外。当然,我们并不认为“传统” 
就是十全十美的(持此谬论者也大有人在)。只是,既然“传”诸后代而成 
为“统”,那就有它的合理性和它的生命力。传统文化固然与具体的时代和 
政治有千丝万缕的联系,我们甚至无法弄清是它在规定政治,还是政治常常 
要利用它,但是,传统文化绝不等同于它们,它是更趋于永恒的东西(如果 
不是伪文化)。一个时代结束了,一种具体的政治体制被更进步的取代了, 
几千年生生不息的传统文化精神可以增添新鲜血液,可以芟除与生俱来或在 
时间长河中衍生的赘物,但绝对无法结束它和取代它!退后一步说吧,来不 
及了解对象就挞伐所结出的果子,一定也与来不及了解对象就歌颂同样苦 
涩。这,也是我们译注先秦诸子的一个原因。 

这套丛书,承蒙著名学者启功、郭预衡两位老先生的关心,我们深感荣 
幸。北师大中文系郭英德先生和北京图书馆吴龙辉博士对本丛书的组织编译 
做了大量工作,没有他们的努力,这套丛书的出版是不可能的。丛书最后由 
我审定,由于学力不逮,时间紧迫,加之译注者水平不一,错漏之处在所难 
免。可以说,这套丛书如果还有可取之处,应该归功于学界前辈的指导和学 
养我只能望其项背的诸位先生的辛勤劳动;而它的所有不足,则应归咎于我 
的才疏学浅,力不胜任。 
《先秦诸子今译丛书》主编 李双 
1992 年 8 月 21 日于北京 


\mainmatter

\chapter{一  亲士}

“亲士”的意思是说要重视人才,这与墨子“尚贤”的主张是一致的,即认为一个国家兴盛与否的关键在于是否能够任用贤才。以此为开首第一篇,也可见其重视程度,这也无疑表现出了墨子宏通与长远的战略眼光。

文章首先把贤士的作用提到了一个极高的位置,然后通过晋文公、齐桓公与越王勾践的例子以及夏桀与商纣的反例来证明用贤的重要。接下来,作者还认为,国君要用贤,一定要律己严而待人宽,只有这样,才会有更多的贤人为国所用。此外,作者还极为深刻地指出,士因其能力的突出而遭受杀身之祸的事例太多了,所以警诫帝王一定要善待贤士,凡是人才,都有一定的个性,难于驾驭,但正因如此,帝王才更要尊重他们,只有这样,才能成就帝王的大业。

\begin{yuanwen}
入国\footnote{“入”疑“乂”之形误,乂国即治国。}而不存\footnote{恤问,即关心的意思。}其士,则亡国矣。见贤而不急,则缓其君矣。非贤无急,非士无与虑国。缓贤忘士,而能以其国存者,未曾有也。昔者文公\footnote{指晋文公重耳,他曾被迫流亡于外十九年,后来回国即位。他在位期间,重用贤才,终于使晋国强大起来,成为春秋五霸之一。}出走而正天下;桓公\footnote{指齐桓公,他未做国君前,他的哥哥齐襄公昏庸无道,而被迫出奔莒国,襄公死后他被迎回即位。此后他重用管仲,也成为春秋五霸之一。}去国而霸诸侯;越王勾践\footnote{越国国君,曾被吴王夫差打败,于是卧薪尝胆,励精图治,终于在范蠡与文种等贤臣的帮助下消灭吴国,报仇雪恨,并成为春秋五霸之一。}遇吴王之丑,而尚摄\footnote{同“慑”。}中国之贤君。三子之能达名成功于天下也,皆于其国抑而\footnote{同“尔”}大丑也。太上无败,其次败而有以成,此之谓用民。
\end{yuanwen}

治理国家却不关心那里的贤士,就会有亡国的危险。见到贤人却不马上任用,他们就会怠慢君主。没有比任用贤士更急迫的事了,如果没有贤士也就没人谋划国家大事。怠慢贤士、轻视人才,而能使国家长治久安,是从来没有过的。从前,晋文公被迫出逃却能够匡正天下;齐桓公流亡国外却能称霸诸侯;越王勾践遭受到败于吴王的耻辱,却还能威慑中原各国的贤君。这三个人能成功扬名于天下,都是因为他们在自己的国家能够忍受极大的屈辱。所以说,最好是不失败,其次则是败了却还有办法成功,这才叫善于用人。

\begin{yuanwen}
吾闻之曰:“非无安居也,我无安心也;非无足财也,我无足心也。”是故君子自难而易彼,众人自易而难彼。君子进不败其志,内究其情\footnote{text};虽杂庸民,终无怨心,彼有自信者也。是故为其所难者,必得其所欲焉;未闻为其所欲,而免其所恶者也。
\end{yuanwen}

\begin{yuanwen}
是故偪\footnote{text}臣伤君\footnote{text},谄\footnote{text}下伤上。君必有弗弗之臣\footnote{text},上必有詻詻之下\footnote{text},分议者延延\footnote{text},而支苟者詻詻\footnote{text},焉可以长生保国。\footnote{text}
臣下重其爵位而不言,近臣则喑\footnote{text},远臣则唫\footnote{text},怨结于民心。谄谀\footnote{text}在侧,善议障塞,则国危矣。桀\footnote{text}纣不以其无天下之士邪\footnote{text}?杀其身而丧天下。故曰:“归国宝\footnote{text},不若献贤而进士。”
\end{yuanwen}

\begin{yuanwen}
今有五锥,此其铦\footnote{text},铦者必先挫。有五刀\footnote{text},此其错\footnote{text},错者必先靡\footnote{text}。是以甘井近竭\footnote{text},招木近伐\footnote{text},灵龟近灼\footnote{text},神蛇近暴\footnote{text}。是故比干之殪\footnote{text},其抗也\footnote{text};孟贲之杀\footnote{text},其勇也;西施之沉\footnote{text},其美也;吴起之裂\footnote{text},其事也。故彼人者,寡不死其所长,故曰:太盛难守也。
\end{yuanwen}

\begin{yuanwen}
故虽有贤君,不爱无功之臣;虽有慈父,不爱无益之子。是故不胜其任而处其位,非此位之人也;不胜其爵而处其禄,非此禄之主也。良弓难张,然可以及高入深;良马难乘,然可以任重致远;良才难令,然可以致君见尊。是故江河不恶小谷之满已也,故能大。圣人者,事无辞也,物无违也,故能为天下器。是故江河之水,非一源之水也;千镒之裘\footnote{text},非一狐之白也\footnote{text}。夫恶有同方不取而取同已者乎?盖非兼王之道也!是故天地不昭昭,大水不潦潦,大火不燎燎,王德不尧尧者,乃千人之长也。其直如矢,其平如砥,不足以覆万物。是故溪陕者速涸\footnote{text},逝浅者速竭,墝埆\footnote{text}者其地不育\footnote{text}。王者淳泽,不出宫中,则不能流国矣。 
\end{yuanwen}


(3)内:依俞樾校,当作“■”(即“退”)。(4)逼:同“嬖”。(5)弗:通“拂”。(6) 

詻(è)詻:同“谔谔”。(7)延延:通“炎炎”。(8)支苟:疑“交苛”二字形误。(9)错:同“厝”, 

磨刀石。(10) 埆(qiāoquè):土地坚硬而瘠薄。 
[白话] 
治国而不优待贤士,国家就会灭亡。见到贤士而不急于任用,他们就会 
怠慢君主。没有比用贤更急迫的了,若没有贤士,就没有人和自己谋划国事。 
怠慢遗弃贤士而能使国家长治久安的,还不曾有过。 
从前,晋文公被迫逃亡在外,后为天下盟主;齐桓公被迫离开国家,后 
来称霸诸侯;越王勾践被吴王战败受辱,终成威慑中原诸国的贤君。这三君 

所以能成功扬名于天下,是因为他们都能忍辱负耻,以图复仇。最上的是不 
遭失败,其次是失败而有办法成功,这才叫善于使用士民。 
我曾听说:“我不是没有安定的住处,而是自己没有安定之心;不是没 
有丰足的财产,而是怀着无法满足的心。”所以君子严以律己,宽以待人。 
而一般人则宽以律己,严以待人。君子仕进顺利时不改变他的素志,不得志 
时心情也一样;即使杂处于庸众之中,也终究没有怨尤之心。他们是有着自 
信的人。所以说,凡事能从难处做起,就一定能达到自己的愿望,但却没有 
听说只做自己所想的事情,而能免于所厌恶之后果的。所以倖臣与谗佞之辈 
往往伤害君主。君主必须有敢于矫正君主过失的臣僚,上面必须有直言极谏 
的下属,分辩议事的人争论锋起,互相责难的人互不退让,这才可以长养民 
生,保卫国土。 
如果臣下只以爵禄为重,不对国事发表意见,近臣缄默不言,远臣闭口 
暗叹,怨恨就郁结于民心了。谄谀阿奉之人围在身边,好的建议被他们阻障 
难进,那国家就危险了。桀、纣不正是因为他们不重视天下之士吗?结果身 
被杀而失天下。所以说:赠送国宝,不如推荐贤士。 
比如现在有五把锥子,一把最锋利,那么这一把必先折断。有五把刀, 
一把磨得最快,那么这一把必先损坏。所以甜的水井最易用干,高的树木最 
易被伐,灵验的宝龟最先被火灼占卦,神异的蛇最先被曝晒求雨。所以,比 
干之死,是因为他抗直;孟贲被杀,是因为他逞勇;西施被沉江,是因为长 
得美丽;吴起被车裂,是因为他有大功。这些人很少不是死于他们的所长。 
所以说:太盛了就难以持久。 
因此,即使有贤君,他也不爱无功之臣;即使有慈父,他也不爱无益之 
子。所以,凡是不能胜任其事而占据这一位置的,他就不应居于此位;凡是 
不胜任其爵而享受这一俸禄的,他就不当享有此禄。良弓不容易张开,但可 
以射得高没得深;良马不容易乘坐,但可以载得重行得远;好的人才不容易 
驾驭,但可以使国君受人尊重。所以,长江黄河不嫌小溪灌注它里面,才能 
让水量增大。圣人勇于任事,又能接受他人的意见,所以能成为治理天下的 
英才。所以长江黄河里的水,不是从同一水源流下的;价值千金的狐白裘, 
不是从一只狐狸腋下集成的。哪里有与自己相同的意见才采纳,与自己不同 
的意见就不采纳的道理呢?这不是统一天下之道。所以大地不昭昭为明(而 
美丑皆收),大水不潦潦为大(而川泽皆纳),大火不燎燎为盛(而草木皆 
容),王德不尧尧为高(而贵贱皆亲),才能做千万人的首领。 
象箭一样直,象磨刀石一样平,那就不能覆盖万物了。所以狭隘的溪流 
干得快,平浅的川泽枯得早,坚薄的土地不长五谷。做王的人深恩厚泽不出 
宫中,就不能流遍全国。 


\chapter{二  修身}

本篇承上篇脉络讨论了一个人怎样才能成为贤士的问题,也就是“修身”的问题。所以,“修身”已经不仅是君子的个人修养,其实也关系到一个国家的治乱兴衰。

作者首先指出,君子务本,而这个根本就是修身,而且,他强调了“反之身”的修养方法。至于修身都包括什么内容,墨子也提出了很多原则,这些原则直至今天也仍有借鉴意义:如“谮慝之言,无入之耳;批扦之声,无出之口”、“贫则见廉,富则见义”、“务言而缓行,虽辩必不听;多力而伐功,虽劳必不图”等在谈论根本的时候,作者也顺笔讽刺了儒家的礼。在作者看来,丧礼中最根本的应该是“哀”而不是“礼”,如果对于死者没有哀思,再多的繁文缛节也没有用。这也可以看出墨子的通达。

\begin{yuanwen}
君子战虽有陈\footnote{text},而勇为本焉;丧虽有礼,而哀为本焉;士虽有学\footnote{text},而行为本焉。是故置本不安者\footnote{text},无务丰末;近者不亲,无务来远;亲戚不附,无务外交;事无终始,无务多业;举物而闇\footnote{text},无务博闻。
\end{yuanwen}

\begin{yuanwen}
是故先王之治天下也,必察迩来远。君子察迩修身也,修身见毁而反之身者也。此以怨省而行修矣。谮慝之言\footnote{text},无入之耳;批扞之声\footnote{text},无出之口;杀伤人之孩\footnote{text},无存之心。虽有诋讦之民\footnote{text},无所依矣。故君子力事日强,愿欲日逾\footnote{text},设壮日盛\footnote{text}。
\end{yuanwen}

\begin{yuanwen}
君子之道也,贫则见廉,富则见义,生则见爱,死则见哀;四行者不可虚假,反之身者也。藏于心者,无以竭爱;动于身者,无以竭恭;出于口者,无以竭驯\footnote{text}。畅之四支\footnote{text},接之肌肤,华发隳颠\footnote{text},而犹弗舍者,其唯圣人乎!
\end{yuanwen}

\begin{yuanwen}
志不强者智不达,言不信者行不果。据财不能以分人者,不足与友;守道不笃,遍物不博\footnote{text},辩是非不察者\footnote{text},不足与游。本不固者末必几\footnote{text},雄而不修者\footnote{text},其后必惰\footnote{text}。原浊者流不清,行不信者名必秏\footnote{text}。名不徒生而誉不自长,功成名遂。名誉不可虚假,反之身者也。务言而缓行,虽辩必不听。多力而伐功,虽劳必不图\footnote{text}。慧者心辩而不繁说,多力而不伐功,此以名誉扬天下。言无务为多而务为智,无务为文而务为察。故彼智无察\footnote{text},在身而情\footnote{text},反其路者也。善无主于心者不留,行莫辩于身者不立;名不可简而成也,誉不可巧而立也。君子以身戴行者也\footnote{text}。思利寻焉,忘名忽焉,可以为士于天下者,未尝有也。 
\end{yuanwen}

(1)本篇主要讨论品行修养与君子人格问题,强调品行是为人治国的根本,君子必须以品德修养 

为重。篇中提出。“君子之道”应包括‘贫则见廉,富则见义,生则见爱,死则见哀’以及明察是非、 

讲究信用、注重实际等内容。(2)陈:同“阵”。(3)孩:毕沆云:“当读如根荄之荄。”(4)辩:同“辨”。 

(5)彼:借为“非”。情:为“惰”之形讹。(6)戴:同“载”。 
[白话] 
君子作战虽用阵势,但必以勇敢为本;办丧事虽讲礼仪,但必以哀痛为 
本;做官虽讲才识,但必以德行为本。所以立本不牢的,就不必讲究枝节的 
繁盛;身边的人不能亲近,就不必讲究招徕远方之民;亲戚不能使之归附, 
就不必讲究结纳外人;做一件事情有始无终,就不必谈起从事多种事业;举 
一件事物尚且弄不明白,就不必追求广见博闻。 
所以先王治理天下,必定要明察左右而招徕远人。君子能明察左右,左 
右之人也就能修养自己的品行了。君子不能修养自己的品行而受人诋毁,那 
就应当自我反省,因而怨少而品德日修。谗害诽谤之言不入于耳,攻击他人 
之语不出于口,伤害人的念头不存于心,这样,即使遇有好诋毁、攻击的人, 
也就无从施展了。 
所以君子本身的力量一天比一天加强,志向一无比一天远大,庄敬的品 
行一天比一天完善。君子之道(应包括如下方面):贫穷时表现出廉洁,富 
足时表现出恩义,对生者表示出慈爱,对死者表示出哀痛。这四种品行不是 
可以装出来的,而是必须自身具备的。凡是存在于内心的,是无穷的慈爱; 
举止于身体的,是无比的谦恭;谈说于嘴上的,是无比的雅驯。(让上述四 
种品行)畅达于四肢和肌肤,直到白发秃顶之时仍不肯舍弃,大概只有圣人 

吧! 
意志不坚强的,智慧一定不高;说话不讲信用的,行动一定不果敢;拥 
有财富而不肯分给人的,不值得和他交友;守道不坚定,阅历事物不广博, 
辨别是非不清楚的,不值得和他交游。根本不牢的,枝节必危。光勇敢而不 
注重品行修养的,后必懒惰。源头浊的流不清,行为无信的人名声必受损害, 
声誉不会无故产生和自己增长。功成了必然名就,名誉不可虚假,必须反求 
诸己。专说而行动迟缓,虽然会说,但没人听信。出力多而自夸功劳,虽劳 
苦而不可取。聪明人心里明白而不多说,努力作事而不夸说自己的功劳,因 
此名誉扬于天下。说话不图繁多而讲究富有智慧,不图文采而讲究明白。所 
以既无智慧又不能审察,加上自身又懒惰,则必背离正道而行了。善不从本 
心生出就不能保留,行不由本身审辨就不能树立,名望不会由苟简而成,声 
誉不会因诈伪而立,君子是言行合一的。以图利为重,忽视立名,(这样) 
而可以成为天下贤士的人,还不曾有过。 


\chapter{三  所染}

\begin{yuanwen}
子墨子见染丝者而叹曰\footnote{text}:染于苍则苍,染于黄则黄。所入者变,其色亦变。五入必\footnote{text},而已则为五色矣。故染不可不慎也!
\end{yuanwen}

\begin{yuanwen}
非独染丝然也,国亦有染。舜染于许由、伯阳\footnote{text},禹染于皋陶、伯益\footnote{text},汤染于伊尹、仲虺\footnote{text},武王染于太公、周公\footnote{text}。此四王者,所染当,故王天下,立为天下,功名蔽天地。举天下之仁义显人,必称此四王者。夏桀染于干辛、推哆\footnote{text},殷纣染于崇侯、恶来\footnote{text},厉王染于厉公长父、荣
夷终\footnote{text},幽王染于傅公夷、祭公敦\footnote{text}。此四王者,所染不当,故国残身死,为天下僇\footnote{text}。举天下不义辱人,必称此四王者。
\end{yuanwen}

\begin{yuanwen}
齐桓染于管仲、鲍叔\footnote{text},晋文染于舅犯、郭偃\footnote{text},楚庄染于孙叔、沈尹\footnote{text},吴阖闾染于伍员、文义\footnote{text},越勾践染于范蠡、大夫种\footnote{text}。此五君者所染当,故霸诸侯,功名传于后世。范吉射染于长柳朔、王胜\footnote{text},中行寅染于藉秦、高彊\footnote{text},吴夫差染于王孙雒、太宰嚭\footnote{text},智伯摇染于智国、张武\footnote{text},中山尚染于魏义、偃长\footnote{text},宋康染于唐鞅、佃不礼\footnote{text}。此六君者所染不当,故国家残亡,身为刑戮,宗庙破灭,绝无后类,君臣离散,民人流亡。举天下之贪暴苛扰者,必称此六君也。凡君之所以安者,何也?以其行理也,行理生于染当。故善为君者,劳于论人,而佚于治官\footnote{text}。不能为君者,伤形费神,愁心劳意,然国逾危,身逾辱。此六君者,非不重其国爱其身也,以不知要故也。不知要者,所染不当也。
\end{yuanwen}

\begin{yuanwen}
非独国有染也,士亦有染。其友皆好仁义,淳谨畏令,则家日益、身日安、名日荣,处官得其理矣。则段干木、禽子、傅说之徒是也\footnote{text}。其友皆好矜奋\footnote{text},创作比周\footnote{text},则家日损、身日危、名日辱,处官失其理矣。则子西、易牙、竖刁之徒是也\footnote{text}。诗曰:“必择所堪\footnote{text},必谨所堪”者\footnote{text},此之谓也。
\end{yuanwen}

(1)本篇以染丝为喻,说明天子、诸侯、大夫、士必须正确选择自己的亲信和朋友,以取得良好 

的熏陶和积极的影响。影响的好坏不同关系着事业的成败、国家的兴亡,国君对此必须谨慎。(2)推哆 

(chǐ):桀臣。(3)佚:同“逸”。(4)堪:当读为“湛”,浸染之意。 
[白话] 
墨子说,他曾见人染丝而感叹说:“(丝)染了青颜料就变成青色,染 
了黄颜料就变成黄色。染料不同,丝的颜色也跟着变化。经过五次之后,就 
变为五种颜色了。所以染这件事是不可不谨慎的。” 
不仅染丝如此,国家也有“染”。舜被许由、伯阳所染,禹被皋陶、伯 
益所染,汤被伊尹、仲虺所染,武王被太公、周公所染。这四位君王因为所 
染得当,所以能称王于天下,立为天子,功盖四方,名扬天下,凡是提起天 
下著名的仁义之人,必定要称这四王。 
夏桀被干辛、推哆所染,殷纣被崇侯、恶来所染,周厉王被厉公长父、 
荣夷终所染,周幽王被傅公夷、蔡公穀所染。这四位君王因为所染不当,结 
果身死国亡,遗羞于天下。凡是提起天下不义可耻之人,必定要称这四王。 
齐桓公被管仲、鲍叔牙所染,晋文公被舅犯、高偃所染,楚庄王被孙叔 
敖、沈尹茎所染,吴王阖闾被伍员、文义所染,越王句践被范蠡、文种所染。 
这五位君主因为所染得当,所以能称霸诸侯,功名传于后世。 
范吉射被长柳朔、王胜所染,中行寅被籍秦、高强所染,吴王夫差被王 

孙雒、太宰嚭所染,知伯摇被知国、张武所染,中山尚被魏义、偃长所染, 
宋康王被唐鞅、佃不礼所染。这六位君主因为所染不当,所以国破家亡,身 
受刑戮,宗庙毁灭,子孙灭绝,君臣离散,百姓逃亡。凡是提起天下贪暴苛 
刻的人,必定称这六君。 
大凡人君之所以能够安定,是什么原因呢?是因为他们行事合理。而行 
事合理源于所染得当。所以善于做国君的,用心致力于选拔人才。不善于做 
国君的,劳神伤身,用尽心思,然而国家更危险,自己更受屈辱。上述这六 
位国君,并非不重视他们的国家、爱惜他们的身体,而是因为他们不知道治 
国要领的缘故。所谓不知道治国要领,即是所染不得当。 
不仅国家有染,士也有“染”。一个人所交的朋友都爱好仁义,都淳朴 
谨慎,慑于法纪,那么他的家道就日益兴盛,身体日益平安,名声日益光耀, 
居官治政也合于正道了,如段干木、禽子、傅说等人即属此类(朋友)。一 
个人所交的朋友若都不安分守己,结党营私,那么他的家道就日益衰落,身 
体日益危险,名声日益降低,居官治政也不得其道,如子西、易牙、竖刀等 
人即属此类(朋友)。《诗》上说:“选好染料。”所谓选好染料,正是这 
个意思。 

\chapter{四  法仪}

\begin{yuanwen}
子墨子曰:天下从事者,不可以无法仪。无法仪而其事能成者,无有也。虽至士之为将相者,皆有法。虽至百工从事者\footnote{text},亦皆有法。百工为方以矩,为圆以规,直以绳,正以县\footnote{text}。无巧工不巧工,皆以此五者为法\footnote{text}。巧者能中之,不巧者虽不能中,放依以从事\footnote{text},犹逾己。故百工从事,皆有法所度\footnote{text}。今大者治天下,其次治大国,而无法所度,此不若百工辩也。
\end{yuanwen}

\begin{yuanwen}
然则奚以为治法而可?当皆法其父母奚若\footnote{text}?天下之为父母者众,而仁者寡。若皆法其父母,此法不仁也。法不仁,不可以为法。当皆法其学奚若\footnote{text}?天下之为学者众,而仁者寡。若皆法其学,此法不仁也。法不仁,不可以为法。当皆法其君奚若?天下之为君者众,而仁者寡。若皆法其君,此法不仁也。法不仁,不可以为法。故父母、学、君三者,莫可以为治法。
\end{yuanwen}

\begin{yuanwen}
然则奚以为治法而可?故曰:莫若法天。天之行广而无私\footnote{text},其施厚而不德,其明久而不衰,故圣王法之。既以天为法,动作有为,必度于天。天之所欲则为之,天所不欲则止。然而天何欲何恶者也?天必欲人之相爱相利,而不欲人之相恶相贼也。奚以知天之欲人之相爱相利,而不欲人之相恶相贼也?以其兼而爱之,兼而利之也。奚以知天兼而爱之、兼而利之也?以其兼而有之、兼而食之也。今天下无大小国,皆天之邑也。人无幼长贵贱,皆天之臣也。此以莫不犓牛羊\footnote{text},豢犬猪\footnote{text},絜为酒醴粢盛\footnote{text},以敬事天,此不为兼而有之、兼而食之邪?天苟兼而有食之,夫奚说以不欲人之相爱相利也?故曰:“爱人利人者,天必福之;恶人贼人者,天必祸之。”曰:“杀不辜者,得不祥焉。”夫奚说人为其相杀而天与祸乎\footnote{text}?是以知天欲人相爱相利,而不欲人相恶相贼也。
\end{yuanwen}

\begin{yuanwen}
昔之圣王禹汤文武\footnote{text},兼爱天下之百姓,率以尊天事鬼,其利人多,故天福之,使立为天子,天下诸侯,皆宾事之。暴王桀纣幽厉\footnote{text},兼恶天下之百姓,率以诟天侮鬼。其贼人多,故天祸之,使遂失其国家,身死为僇于天下。后世子孙毁之,至今不息。故为不善以得祸者,桀纣幽厉是也。爱人利人以得福者,禹汤文武是也。爱人利人以得福者有矣,恶人贼人以得祸者亦有矣!
\end{yuanwen}

(1)法仪即法度、准则之意。墨子认为,天子、诸侯治理天下、国家必须以天为法,以天意为归。 

而所谓天意,实即就是墨家学派所主张的兼爱兼利原则。篇中以古代圣王和暴君为正反两方面的例子, 

指出“爱人利人”即可得福,“恶人贼人”必然招祸。(2)县:即“悬”的本字。(3)放:通“仿”。 

(4)辩:通“辨”。(5)■:同“刍”。 
[白话] 
墨子说:天底下办事的人,不能没有法则;没有法则而能把事情做好, 
是从来没有的事。即使士人作了将相,他也必须有法度。即使从事于各种行 
业的工匠,也都有法度。工匠们用矩划成方形,用圆规划圆形,用绳墨划成 
直线,用悬锤定好偏正,(用水平器制好平面)。不论是巧匠还是一般工匠, 
都要以这五者为法则。巧匠能切合五者的标准,一般工匠虽做不到这样水平, 
但仿效五者去做,还是要胜过自身的能力。所以工匠们制造物件时,都有法 
则可循。 
现在大的如治天下,其次如治大国,却没有法则,这是不如工匠们能明 

辨事理。那么,用什么作为治理国家的法则才行呢?假若以自己的父母为法 
则何如?天下做父母的很多,但仁爱的少。倘若人人都以自己的父母为法则, 
这实为效法不仁。效法不仁,这自然是不可以的。假若以自己从学的师长为 
法何如?天下做师长的很多,但仁爱的少。倘若人人都以自己的师长为法则, 
这实为效法不仁。效法不仁,这自然是不可以的。假若以自己的国君为法则 
何如?天下做国君的很多,但仁爱的少。倘若人人都以自己的国君为法则, 
这实为效法不仁。效法不仁,这自然是不可以的。所以父母、师长和国君三 
者,都不可以作为治理国家的法则。 
那么用什么作为治理国家的法则才行呢?最好是以天为法则。天的运行 
广大无私,它的恩施深厚而不自居,它的光耀永远不衰,所以圣王以它为法 
则。既然以天为法则,行动作事就必须依天而行。天所希望的就去做,天所 
不希望的就应停止。那么天希望什么不希望什么呢?天肯定希望人相爱相 
利,而不希望人相互厌恶和残害。怎么知道天希望人相爱相利,而不希望人 
相互厌恶和残害呢?这是因为天对人是全爱和全利的缘故。怎么知道天对人 
是全爱和全利呢?因为人类都为天所有,天全部供给他们吃的。 
现在天下不论大国小国,都是天的国家。人不论长幼贵贱,都是天的臣 
民。因此人无不喂牛羊、养猪狗,洁净地准备好酒食祭品,用来诚敬事天。 
这难道不是全部地拥有和供给人食物?天既然全部地拥有和供给人食物,为 
何能说天不要人相爱相利呢?所以说:“爱人利人的人,天必定给他降福; 
相互厌恶和残害人的人,天必定给他降祸。所以说:杀害无辜的人,会得到 
不祥后果。为何说人若相互残杀,天就降祸于他呢?这是因为知道天希望人 
相爱相利,而不希望人相互厌恶和残害。” 
以前的圣王禹、汤、周文王、周武王,对天下百姓全都爱护,带领他们 
崇敬上天,侍奉鬼神。他们给人带来的利益多,所以天降福给他们,使他们 
立为天子。天下的诸侯,都恭敬地服事他们。暴虐的君王桀、纣、周幽王、 
周厉王,对于天下的百姓全部厌恶、憎恨,带领他们咒骂上天,侮辱鬼神。 
他们残害的人多,所以天降祸给他们,使他们丧失了国家,身死还要受辱于 
天下。后代子孙责骂他们,至今不休。所以做坏事而得祸的,桀、纣、周幽 
王、周厉王即是这类;爱人利人而得福的,禹、汤、周文王、周武王即是这 
类。爱人利人而得福的是有的,厌恶人残害人而得祸的,也是有的! 


\chapter{五  七患}

\begin{yuanwen}
子墨子曰:国有七患。七患者何?城郭沟池不可守而治宫室,一患也; 
边国至境(2),四邻莫救,二患也;先尽民力无用之功,赏赐无能之人,民力 
尽于无用,财宝虚于待客,三患也;仕者持禄,游者爱佼(3),君修法讨臣, 
臣慑而不敢拂,四患也;君自以为圣智而不问事,自以为安强而无守备,四 
邻谋之不知戒,五患也;所信者不忠,所忠者不信,六患也;畜种菽粟不足 
以食之,大臣不足以事之,赏赐不能喜,诛罚不能威,七患也。 
以七患居国,必无社稷;以七患守城,敌至国倾。七患之所当,国必有 
殃。 
凡五谷者,民之所仰也,君之所以为养也。故民无仰,则君无养;民无 
食,则不可事。故食不可不务也,地不可不力也,用不可不节也。五谷尽收, 
则五味尽御于主,不尽收则不尽御。一谷不收谓之馑,二谷不收谓之旱,三 
谷不收谓之凶,四谷不收谓之馈(4),五谷不收谓之饥。 
岁馑,则仕者大夫以下皆损禄五分之一;旱,则损五分之二;凶,则损 
五分之三;馈,则损五分之四;饥,则尽无禄,禀食而已矣。故凶饥存乎国, 
人君彻鼎食五分之五(5),大夫彻县(6),士不入学,君朝之衣不革制;诸侯 
之客,四邻之使,雍食而不盛(7);彻骖騑,涂不芸(8),马不食粟,婢妾不 
衣帛,此告不足之至也。 
今有负其子而汲者,队其子于井中(9),其毋必从而道之。今岁凶,民饥, 
道饿,重其子此疚于队,其可无察邪!故时年岁善,则民仁且良;时年岁凶, 
则民吝且恶。夫民何常此之有!为者疾,食者众,则岁无丰。 
故曰:财不足则反之时,食不足则反之用。故先民以时生财,固本而用 
财,则财足。故虽上世之圣王,岂能使五谷常收而旱水不至哉!然而无冻饿 
之民者,何也?其力时急而自养俭也。故《夏书》曰:“禹七年水。”《殷 
书》曰:“汤五年旱。”此其离凶饿甚矣(10),然而民不冻饿者,何也?其 
生财密,其用之节也。故仓无备粟,不可以待凶饥;库无备兵,虽有义不能 
征无义;城郭不备全,不可以自守;心无备虑,不可以应卒(11),是若庆忌 
无去之心,不能轻出。 
夫桀无待汤之备,故放;纣无待武之备,故杀。桀纣贵为天子,富有天 
下,然而皆灭亡于百里之君者,何也?有富贵而不为备也。故备者,国之重 
也。食者,国之宝也;兵者,国之爪也;城者,所以自守也;此三者,国之 
具也。 
故曰:以其极赏,以赐无功;虚其府库,以备车马、衣裘、奇怪;苦其 
役徒,以治宫室观乐;死又厚为棺椁,多为衣裘。生时治台榭,死又修坟墓。 
故民苦于外,府库单于内(12),上不厌其乐(13),下不堪其苦。故国离寇敌 
则伤,民见凶饥则亡,此皆备不具之罪也。且夫食者,圣人之所宝也。故《周 
书》曰:“国无三年之食者,国非其国也;家无三年之食者,子非其子也。” 
此之谓国备。 
\end{yuanwen}

\begin{yuanwen}
	
\end{yuanwen}\begin{yuanwen}
	
\end{yuanwen}\begin{yuanwen}
	
\end{yuanwen}\begin{yuanwen}
	
\end{yuanwen}\begin{yuanwen}
	
\end{yuanwen}\begin{yuanwen}
	
\end{yuanwen}\begin{yuanwen}
	
\end{yuanwen}\begin{yuanwen}
	
\end{yuanwen}\begin{yuanwen}
	
\end{yuanwen}\begin{yuanwen}
	
\end{yuanwen}\begin{yuanwen}
	
\end{yuanwen}\begin{yuanwen}
	
\end{yuanwen}\begin{yuanwen}
	
\end{yuanwen}\begin{yuanwen}
	
\end{yuanwen}\begin{yuanwen}
	
\end{yuanwen}\begin{yuanwen}
	
\end{yuanwen}\begin{yuanwen}
	
\end{yuanwen}\begin{yuanwen}
	
\end{yuanwen}\begin{yuanwen}
	
\end{yuanwen}\begin{yuanwen}
	
\end{yuanwen}\begin{yuanwen}
	
\end{yuanwen}\begin{yuanwen}
	
\end{yuanwen}\begin{yuanwen}
	
\end{yuanwen}\begin{yuanwen}
	
\end{yuanwen}\begin{yuanwen}
	
\end{yuanwen}\begin{yuanwen}
	
\end{yuanwen}\begin{yuanwen}
	
\end{yuanwen}\begin{yuanwen}
	
\end{yuanwen}\begin{yuanwen}
	
\end{yuanwen}\begin{yuanwen}
	
\end{yuanwen}\begin{yuanwen}
	
\end{yuanwen}\begin{yuanwen}
	
\end{yuanwen}\begin{yuanwen}
	
\end{yuanwen}\begin{yuanwen}
	
\end{yuanwen}\begin{yuanwen}
	
\end{yuanwen}\begin{yuanwen}
	
\end{yuanwen}\begin{yuanwen}
	
\end{yuanwen}\begin{yuanwen}
	
\end{yuanwen}\begin{yuanwen}
	
\end{yuanwen}\begin{yuanwen}
	
\end{yuanwen}\begin{yuanwen}
	
\end{yuanwen}\begin{yuanwen}
	
\end{yuanwen}\begin{yuanwen}
	
\end{yuanwen}\begin{yuanwen}
	
\end{yuanwen}\begin{yuanwen}
	
\end{yuanwen}\begin{yuanwen}
	
\end{yuanwen}\begin{yuanwen}
	
\end{yuanwen}\begin{yuanwen}
	
\end{yuanwen}\begin{yuanwen}
	
\end{yuanwen}\begin{yuanwen}
	
\end{yuanwen}\begin{yuanwen}
	
\end{yuanwen}\begin{yuanwen}
	
\end{yuanwen}

(1)本篇首先分析了给国家造成危亡的七种祸患,然后指出国家防治祸患的根本在于增加生产和 

节省财用,并对当时统治者竭尽民力和府库之财以追求享乐生活的做法提出了严正警告。(2)边:“敌” 

字之误。(3)佼:通“交”。(4)馈:通“匮”,缺乏。(5)五分之五:疑作“五分之三”。(6)县:通 

“悬”,此指钟磬等悬挂的乐器。(7)雍:当作“饔”,指早餐和晚餐。(8)涂:通“途”。(9)队:通 

“坠”。(10)离:通“罹”,遭受。(11)卒:通“猝”。(12)单:通“殚”。(13)厌:通“餍”,满 

足。 
[白话] 
墨子说:国家有七种祸患。这七种祸患是什么呢?内外城池壕沟不足守 
御而去修造宫室,这是第一种祸患;敌兵压境,四面邻国都不愿来救援,这 
是第二种祸患;把民力耗尽在无用的事情上,赏赐没有才能的人,(结果) 
民力因做无用的事情而耗尽,财宝因款待宾客而用空,这是第三种祸患;做 
官的人只求保住俸禄,游学未仕的人只顾结交党类,国君修订法律以诛戮臣 
下,臣下畏惧而不敢违拂君命,这是第四种祸患;国君自以为神圣而聪明, 
而不过问国事,自以为安稳而强盛,而不作防御准备,四面邻国在图谋攻打 
他,而尚不知戒备,这是第五种祸患;所信任的人不忠实,而忠实的人不被 
信任,这是第六种祸患;家畜和粮食不够吃,大臣对于国事不胜使令,赏赐 
不能使人欢喜,责罚不能使人畏惧,这是第七种祸患。 
治国若存在这七种祸患,必定亡国;守城若存在这七种祸患,国都必定 
倾毁。七种祸患存在于哪个国家,哪个国家必有祸殃。 
五谷是人民所仰赖以生活的东西,也是国君用以养活自己和民众的。所 
以如果人民失去仰赖,国君也就没有供养;人民一旦没有吃的,就不可使役 
了。所以粮食不能不加紧生产,田地不能不尽力耕作,财用不可不节约使用。 
五谷全部丰收,国君就可兼进五味。若不全都丰收,国君就不能尽其享受。 
一谷无收叫做馑,二谷无收叫做旱,三谷不收叫做凶,四谷不收叫做匮,五 
谷不收叫做饥。 
遇到馑年,做官的自大夫以下都减去俸禄的五分之一;旱年,减去俸禄 
的五分之二;凶年,减去俸禄的五分之三;匮年,减去俸禄的五分之四;饥 
年,免去全部俸禄,只供给饭吃。所以一个国家遇到凶饥,国君撤掉鼎食的 
五分之三,大夫不听音乐,读书人不上学而去种地,国君的朝服不制新的; 
诸侯的客人、邻国的使者,来时饮食都不丰盛,驷马撤掉左右两匹,道路不 
加修理,马不吃粮食,婢妾不穿丝绸,这都是告诉国家已十分困乏了。 
现在假如有一人背着孩子到井边汲水,把孩子掉到井里,那么这位母亲 
必定设法把孩子从井中救出。现在遇到饥年,路上有饿死的人,这种惨痛比 
孩子掉入井中更为严重,能忽视这种局面吗?年成好的时候,老百姓就仁慈 
驯良;年成遇到凶灾,老百姓就吝啬凶恶;民众的性情哪有一定呢!生产的 
人少,吃饭的人多,就不可能有丰年。 
所以说:财用不足就注重农时,粮食不足就注意节约。因此,古代贤人 
按农时生产财富,搞好农业基础,节省开支,财用自然就充足。所以,即使 
前世的圣王,岂能使五谷永远丰收,水旱之灾不至呢!但(他们那时)却从 
无受冻挨饿之民,这是为何呢?这时因为他们努力按农时耕种而自奉俭朴。 
《夏书》说:“禹时有七年水灾。”《殷书》说:“汤时有五年旱灾。”那 
时遭受的凶荒够大的了,然而老百姓却没有受冻挨饿,这是何故呢?因为他 
们生产的财用多,而使用很节俭。所以,粮仓中没有预备粮,就不能防备凶 
年饥荒;兵库中没有武器,即使自己有义也不能去讨伐无义;内外城池若不 
完备,不可以自行防守;心中没有戒备之心,就不能应付突然的变故。这就 
好像庆忌没有逐走要离之意,就不可轻出致死。 
桀没有防御汤的准备,因此被汤放逐;纣没有防御周武王的准备,因此 

被杀。桀和纣虽贵为天子,富有天下,然而都被方圆百里的小国之君所灭, 
这是为何呢?是因为他们虽然富贵,却不做好防备。所以防备是国家最重要 
的事情。粮食是国家的宝物,兵器是国家爪牙,城郭是用来自我守卫的:这 
三者是维持国家的工具。 
所以说:拿最高的奖赏赐给无功之人;耗尽国库中的贮藏,用以置备车 
马、衣裘和稀奇古怪之物;使役卒和奴隶受尽苦难,去建造宫室和观赏游乐 
之所;死后又做厚重的棺椁,制很多衣服。活着时修造台榭,死后又修造坟 
墓。因此,老百姓在外受苦,内边的国库耗尽,上面的君主不满足其享受, 
下面的民众不堪忍受其苦难。所以,国家一遇敌寇就受损伤,人民一遭凶饥 
就死亡,这都是平时不做好防备的罪过。再说,粮食也是圣人所宝贵的。《周 
书》说:“国家若不预备三年的粮食,国家就不可能成其为这一君主的国家 
了;家庭若不预备三年的粮食,子女就不能做这一家的子女了。”这就叫做 
“国备”(国家的根本贮备)。 

\chapter{六  辞过(1)}
六  辞过(1)

子墨子曰:“古之民,未知为宫室时,就陵阜而居,穴而处,下润湿伤 
民,故圣王作为宫室。为宫室之法,曰室高足以辟润湿,边足以圉风寒,上 
足以待雪霜雨露,宫墙之高,足以别男女之礼,谨此则止(2)。凡费财劳力, 
不加利者,不为也。役(3),修其城郭,则民劳而不伤,以其常正(4),收其 
租税,则民费而不病。民所苦者非此也,苦于厚作敛于百姓。是故圣王作为 
宫室,便于生,不以为观乐也;作为衣服带履便于身,不以为辟怪也。故节 
于身,诲于民,是以天下之民可得而治,财用可得而足。 
当今之主,其为宫室,则与此异矣。必厚作敛于百姓,暴夺民衣食之财, 
以为宫室,台榭曲直之望,青黄刻镂之饰。为宫室若此,故左右皆法象之, 
是以其财不足以待凶饥、振孤寡,故国贫而民难治也。君欲实天下之治,而 
恶其乱也,当为宫室,不可不节。 
古之民,未知为衣服时,衣皮带茭,冬则不轻而温,夏则不轻而凊。圣 
王以为不中人之情,故作诲妇人,治丝麻,棞布绢,以为民衣。为衣服之法, 
冬则练帛之中,足以为轻且暖;夏则 绤之中,足以为轻且凊,谨此则止。 
故圣人之为衣服,适身体,和肌肤,而足矣。非荣耳目而观愚民也。当是之 
时,坚车良马不知贵也,刻镂文采,不知喜也,何则?其所道之然。故民衣 
食之财,家足以待旱水凶饥者,何也?得其所以自养之情,而不感于外也, 
是以其民俭而易治,其君用财节而易赡也。府库实满,足以待不然;兵革不 
顿,士民不劳,足以征不服。故霸王之业,可行于天下矣。 
当今之主,其为衣服,则与此异矣,冬则轻煗(5),夏则轻凊,皆已具矣, 
必厚作敛于百姓,暴夺民衣食之财,以为锦绣文采靡曼之衣,铸金以为钩, 
珠玉以为珮。女工作文采,男工作刻镂,以为身服,此非云益煗之情也。单 
财劳力(6),毕归之于无用也,以此观之,其为衣服非为身体,皆为观好,是 
以其民淫僻而难治,其君奢侈而难谏也,夫以奢侈之君,御好淫僻之民,欲 
国无乱,不可得也。君实欲天下之治而恶其乱,当为衣服不可不节。 
古之民未知为饮食时,素食而分处,故圣人作,诲男耕稼树艺,以为民 
食。其为食也,足以增气充虚,强体养腹而已矣。故其用财节,其自养俭, 
民富国治。今则不然,厚作敛于百姓,以为美食刍豢,蒸炙鱼鳖,大国累百 
器,小国累十器,前方丈,目不能遍视,手不能遍操,口不能遍味,冬则冻 
冰,夏则饰 (7),人君为饮食如此,故左右象之,是以富贵者奢侈,孤寡者 
冻馁,虽欲无乱,不可得也。君实欲天下治而恶其乱,当为食饮不可不节。 
古之民未知为舟车时,重任不移,远道不至,故圣王作为舟车,不便民 
之事。其为舟车也,全固轻利,可以任重致远,其为用财少,而为利多,是 
以民乐而利之。故法令不急而行,民不劳而上足用,故民归之。当今之主, 
其为舟车,与此异矣,全固轻利皆已具,必厚作敛于百姓,以饰舟车。饰车 
以文采,饰舟以刻镂。女子废其纺织而修文采,故民寒;男子离其耕稼而修 
刻镂,故民饥。人君为舟车若此,故左右象之,是以其民饥寒并至,故为奸 
邪。奸邪多则刑罚深,刑罚深则国乱。君实欲天下之治而恶其乱,当为舟车 
不可不节。 
凡回于天地之间,包于四海之内,天壤之情,阴阳之和,莫不有也,虽 
至圣不能更也。何以知其然?圣人有传:天地也,则曰上下;四时也,则曰 
阴阳;人情也,则曰男女;禽兽也,则曰牝牡雌雄也。真天壤之情,虽有先 

王不能更也。虽上世至圣,必蓄私,不以伤行,故民无怨。宫无拘女,故天 
下无寡夫。内无拘女,外无寡夫,故天下之民众。当今之君,其蓄私也,大 
国拘女累千,小国累百,是以天下之男多寡无妻,女多拘无夫,男女失时, 
故民少。君实欲民之众而恶其寡,当蓄私不可不节。 
凡此五者,圣人之所俭节也,小人之所淫佚也。俭节则昌,淫佚则亡, 
此五者不可不节。夫妇节而天地和,风雨节而五谷熟,衣服节而肌肤和。 


[注释] 

(1)本篇主要通过宫室、衣服、饮食、舟车、蓄私的古今对照,批判当时统治者的奢侈生活。主 

旨与《节用》篇基本相同。篇题所谓辞过,即要求时君改掉这五方面的过失。(2)谨:通“仅”。(3) 

役:上当有“以其常”三字。(4)正:通“征”。(5)煗:同“暖”。(6)单:通“殚”。(7)饰:“馂” 

的误字。 
[白话] 
墨子说:“上古的人民不知道作宫室之时,靠近山陵居住,住在洞穴里, 
地下潮湿,伤害人民,所以圣王开始营造宫室。营造宫室的法则是:地基的 
高度足以避湿润,四边足以御风寒,屋顶足以防备霜雪雨露,宫墙的高度足 
以分隔内外,使男女有别——仅此而已。凡属劳民伤财而不增加益处的事, 
是不会做的。(照常规)分派劳役,修治城郭,那么民众就虽劳苦而不至伤 
害;照常规征收租税,那么民众虽破费而不至困苦。因为民众所疾苦并不是 
这些,而是苦于对老百姓横征暴敛。所以圣王开始制造宫室,只为方便生活, 
并不是为了观赏之乐;开始创制衣服带履,只为便利身体,而不是为了奇怪 
的装束。所以,(圣王)自身节俭,(以身作则地)教导百姓,因而天下的 
民众得以治理,财用得以充足。 
现在的君主,修造宫室却与此不同:他们必定要向百姓横征暴敛,强夺 
民众的衣食之资用来营造宫室,(在宫室上)修造台榭曲折的景观,讲究颜 
色雕刻的装饰。营造宫室如此(铺张),身边的人都效法这种做法,因此财 
用(被浪费)而不能应付凶年饥馑,振恤孤寡之人,所以国家穷困而人民无 
法治理。国君若是真希望天下得到治理,而不愿其混乱,那么,营造宫室就 
不可不节俭。 
上古的人民不知道做衣服的时候,穿着兽皮,围着草索,冬天不轻便又 
不温暖,夏天不轻便又不凉爽。圣王认为这样不符合人情,所以开始教女子 
治丝麻、织布匹,以它作人的衣服。制造衣服的法则是:冬天穿生丝麻制的 
中衣,只求其轻便而温暖,夏天穿葛制的中衣,只求其轻便而凉爽,仅此而 
已。所以圣人制作衣服只图身体合适、肌肤舒适就够了,并不是夸耀耳目、 
炫动愚民。当这时候,坚车良马没有人知道贵重,雕刻文采没有人知道欣赏, 
为什么呢?这是(君主)教导的结果。所以民众的衣服之财,家家都足以防 
患水旱凶饥,为什么呢?因为他们懂得自我供养的情实,不被外界所诱惑, 
所以民众俭朴而容易治理,国君用财有节制而容易富足。国库充实,足以应 
付非常的变故:兵甲不坏,士民不劳,足以证伐不顺之臣,所以可实现霸王 
事业于天下。 
现在的君主,他们制造衣服却与此不同:冬天(的衣服)轻便而暖和, 
夏天(的衣服)轻便而凉爽,这都已经具备了,他们还一定要向百姓横征暴 
敛,强夺民众的衣食之资,用来做锦绣文彩华丽的衣服,拿黄金作成衣带钩, 
拿珠玉作成佩饰,女工作文采,男工作雕刻,用来穿在身上。这并非真的为 

了温暖。耗尽钱财费了民力,都是为了无用之事,由此看来,他们作衣服, 
不是为身体,而是为好看。因此民众邪僻而难以治理,国君奢侈而难以进谏。 
以奢侈的国君统治邪僻的民众,希望国家不乱,是不可能的。国君若真希望 
天下治理好而厌恶混乱,作衣服时就不可不节俭。 
上古的人民不知道制作饮食时,只吃素食而各自分居,所以圣人起来教 
勇子耕稼栽种,以供人作粮食。作饮食的原则是,只求补气益虚、强身饱腹 
就够了。所以他们用财节省,自养俭朴,(因而)民众富足,国家安定。现 
在却不是这样,向老百姓厚敛钱财,用来享受美味牛羊,蒸烤鱼鳖,大国之 
君集有上百样的菜,小国之君也有上十样的菜,摆在前面一丈见方,眼不能 
全看到,手不能全捡取到,嘴也不能全尝到,冬天结冻,夏天臭烂,国君这 
样讲究饮食,左右大臣都效法他。因此富贵的人奢侈,孤寡的人冻饿。这样 
一来,即使不希望国家混乱,也是不可能的。国君若真希望天下治理好而厌 
恶其混乱,饮食就不可不节省。 
上古的人民不知道制造舟车时,重的东西搬不动,远的地方去不了,所 
以圣王开始制造舟车,用以便利民事。他们作舟车只求坚固轻便,可以运重 
物、行远路,费用花的少,而利益很大,所以民众乐于使用。所以法令不用 
催促而可行使,民众不用劳苦而财用充足,所以民众归顺他了。现在的君主 
制造舟车则与此不同。舟车已经坚固轻利了,他们还要向百姓横征暴敛,用 
以装饰舟车。在车上画以文彩,在舟上加以雕刻。女子废弃纺织而去描绘文 
彩,所以民众受寒;男子脱离耕稼而去从事雕刻,所以民众挨饿。国君这样 
制造舟车,左右大臣跟着仿效,所以民众饥寒交迫,不得已而作奸邪之事。 
奸邪之事一多,刑罚必然繁重。刑罚一繁重,国家就乱了。国君如果真的希 
望天下治理好而厌恶混乱,制造舟车就不可不节省。 
凡周回于天地之间,包裹于四海之内的,天地之情,阴阳之和,一切都 
具备了,即使至圣也不能更动。何以知道这样呢?圣人传下的书说:天地称 
作上下,四时称作阴阳,人类分为男女,禽兽分为牝牡雌雄。这是真正的天 
地之情,即使有先世贤王也不能更动。即使上代至圣,一定都养有私人侍妾, 
但不伤害品行,所以民众无怨。宫中没有拘禁的女子,所以天下没有鳏夫。 
内无拘禁之妇,外无鳏夫,因而天下人民众多。现在的国君养侍妾,大国拘 
禁女子数千,小国数百,所以天下男子大多没有妻子,女子多遭拘禁而没有 
丈夫。男女婚姻失时,所以百姓减少。国君如果真想人民增多而厌恶减少, 
养侍妾就不可不节制。 
以上所说的五者,都是圣人所节俭而小人所奢侈淫佚的。节俭的就昌盛, 
淫佚的就灭亡,这五者不可不节制。夫妇之事有节制,天地就和顺;风雨调 
节,五谷就丰收;衣服有节制,身体肌肤就安适。 

\chapter{七  三辩(1)}

程繁问于子墨子曰:“夫子曰:‘圣王不为乐。’昔诸侯倦于听治,息 
于钟鼓之乐;士大夫倦于听治,息于竽瑟之乐;农夫春耕、夏耘、秋敛、冬 
藏,息于聆缶之乐(2)。今夫子曰:‘圣王不为乐’,此譬之犹马驾而不税, 
弓张而不弛,无乃非有血气者之所不能至邪!” 
子墨子曰:“昔者尧舜有茅茨者,且以为礼,且以为乐。汤放桀于大水, 
环天下自立以为王,事成功立,无大后患,因先王之乐,又自作乐,命曰《護》, 
又修《九招》。武王胜殷杀纣,环天下自立以为王,事成功立,无大后患, 
因先王之乐,又自作乐,命曰《象》。周成王因先王之乐,又自作乐,命曰 
《驺虞》。周成王之治天下也,不若武王;武王之治天下也,不若成汤;成 
汤之治天下也,不若尧舜。故其乐逾繁者,其治逾寡。自此观之,乐非所以 
治天下也。” 
程繁曰:“子曰:‘圣王无乐。’此亦乐已,若之何其谓圣王无乐也?” 
子墨子曰:“圣王之命也,多寡之,食之利也。以知饥而食之者,智也。因 
为无智矣。今圣有乐而少,此亦无也。” 


[注释] 

(1)本篇通过墨子与程繁对音乐的讨论,强调圣人治理天下重在事功,反对追求音乐享受。这对 

批判当时统治者的享乐生活有现实意义。(2)聆:通“铃”。 
[白话] 
程繁问墨子说:“先生曾经说过:‘圣王不作音乐。’以前的诸侯治国 
太劳累了,就以听钟鼓之乐的方式进行休息;士大夫工作太累了,就以听竽 
瑟之乐的方式进行休息;农夫春天耕种、夏天除草、秋天收获、冬天贮藏, 
也要借听瓦盆土缶之乐的方式休息,现在先生说:‘圣王不作音乐。’这好 
比马套上车后就不再卸下,弓拉开后不再放松,这恐怕不是有血气的人所能 
做到的吧!” 
墨子说:“以前尧舜只有茅草盖的屋子,所谓礼乐不过如此。后来汤把 
桀放逐到大水,统一天下,自立为王,事成功立,没有大的后患,于是就承 
袭先王之乐而自作新乐,取各为《護》,又修《九招》之乐。周武王战胜殷 
朝,杀死纣王,统一天下,自立为王,没有了大的后患,于是袭先王之乐而 
自作新乐,取名为《驺虞》。周成王治理天下不如武王;周武王治理天下不 
如成汤;成汤治理天下不如尧舜。所以音乐逾繁杂的国王,他的治绩就逾少。 
由此看来,音乐不是用来治理天下的。” 
程繁说:“先生说:‘圣王没有音乐。’但这些就是音乐,怎么能说圣 
王没有音乐呢?”墨子说:“圣王的教令:凡是太盛的东西就减损它。饮食 
于人有利,若因知道饥而吃的就算是智慧,也就无所谓智慧了。现在圣王虽 
然有乐,但却很少,这也等于没有音乐。” 

\chapter{八  尚贤上}

子墨子言曰:“今者王公大人为政于国家者,皆欲国家之富,人民之众, 
刑政之治。然而不得富而得贫,不得众而得寡,不得治而得乱,则是本失其 
所欲,得其所恶。是其故何也?”子墨子言曰:“是在王公大人为政于国家 
者,不能以尚贤事能为政也。是故国有贤良之士众,则国家之治厚;贤良之 
士寡,则国家之治薄。故大人之务,将在于众贤而已。” 
曰:“然则众贤之术将奈何哉?”子墨子言曰:“譬若欲众其国之善射 
御之士者,必将富之、贵之、敬之、誉之,然后国之善射御之士,将可得而 
众也。况又有贤良之士,厚乎德行,辩乎言谈,博乎道术者乎!此固国家之 
珍而社稷之佐也,亦必且富之、贵之、敬之、誉之、然后国之良士,亦将可 
得而众也。”是故古者圣王之为政也,言曰:“不义不富,不义不贵,不义 
不亲,不义不近。”是以国之富贵人闻之,皆退而谋曰:“始我所恃者,富 
贵也。今上举义不辟贫贱(2),然则我不可不为义。”亲者闻之,亦退而谋曰: 
“始我所恃者,亲也。今上举义不辟疏,然则我不可不为义。”近者闻之, 
亦退而谋曰:“始我所恃者,近也。今上举义不辟远,然则我不可不为义。” 
远者闻之,亦退而谋曰:“我始以远为无恃,今上举义不辟远,然则我不可 
不为义。”逮至远鄙郊外之臣、门庭庶子(3)、国中之众、四鄙之萌人闻之(4), 
皆竞为义。是其故何也?曰:上之所以使下者,一物也;下之所以事上者, 
一术也。譬之富者,有高墙深宫,墙立既,谨上为凿一门。有盗人入,阖其 
自入而求之,盗其无自出。是其故何也?则上得要也。 
故古者圣王之为政,列德而尚贤。虽在农与工肆之人,有能则举之。高 
予之爵,重予之禄,任之以事,断予之令。曰:爵位不高,则民弗敬;蓄禄 
不厚,则民不信;政令不断,则民不畏。举三者授之贤者,非为贤赐也,欲 
其事之成。故当是时,以德就列,以官服事,以劳殿赏(5),量功而分禄。故 
官无常贵而民无终贱。有能则举之,无能则下之。举公义,辟私怨,此若言 
之谓也。 
故古者尧举舜于服泽之阳,授之政,天下平。禹举益于阴方之中,授之 
政,九州成。汤举伊尹于庖厨之中,授之政,其谋得。文王举闳夭、泰颠于 
罝罔之中(6),授之政,西土服。故当是时,虽在于厚禄尊位之臣,莫不敬惧 
而施(7);虽在农与工肆之人,莫不竞劝而尚意。故士者,所以为辅相承嗣也。 
故得士则谋不困,体不劳,名立而功成,美章而恶不生(8),则由得士也。是 
故子墨子言曰:“得意,贤士不可不举;不得意,贤士不可不举。尚欲祖述 
尧舜禹汤之道,将不可以不尚贤。夫尚贤者,政之本也。” 


[注释] 

(1)本篇主要探讨尚贤与政治的关系,墨子提出尚贤“为政之本”,主张统治者打破血统界限, 

从各阶层中选拔真才实学之人,给他们地位和权力,同时将那些尸位素餐的贵族老爷统统撤免。这对 

当时广大平民阶级争取政治权力的斗争无疑有着现实意义和理论指导意义。《尚贤》分上、中、下三 

篇,内容一致而文字繁简不同,可能是墨家后学中流传的三种不同记录本子。(2)辟:通“避”。(3) 

庶子:此指诸侯之同族与卿大夫之子。(4)萌人:民人。(5)殿:定。(6)罝(jǖ居):捕兽的网。(7) 

施:上疑脱“不”字。(8)章:通“彰”。 
[白话] 
墨子说:现在王公大人治理国家,都希望国家富强,人民众多,刑政治 

理,然而结果却国家不得富强而得贫困,人口不得众多而得减少,刑政不得 
治理而得混乱,完全失去所希望的,而得到所厌恶的,这是什么原因呢? 
墨子说:这是因为王公大人治理国家不能做到尊贤使能。在一个国家中, 
如果贤良之士多,那么国家的治绩就大;如果贤良之士少,那么国家的治绩 
就小。所以王公大人的急务,将是如何使贤人增多。 
那么,使贤人增多的方法是什么呢?墨子说:譬如要使一个国家的善于 
射御之人增多,就必须使他们富裕,使他们显贵,尊敬他们,赞誉他们,这 
之后国家善于射御的人就可以增多了。何况还有贤良之士,德行醇厚,言谈 
辩给,道术宏博的人呢!他们确实是国家的珍宝、社稷的良佐呀!也必须使 
他们富裕,使他们显贵,尊敬他们,赞誉他们,这之后国家的良士也就可以 
增多了。所以古时圣王为政,说道:“不义的人不使富裕,不义的人不使显 
贵,不义的人不使相亲,不义的人不使接近。”所以国中富贵的人听到了, 
都退下来商议说:“当初我所依靠的是富贵,现在上面举义而不避贫贱,那 
我不可不为义。”有亲的人听到了,也退回来商议说:“当初我所倚仗的是 
与上有亲,现在上面举义而不避疏远,那我不可不为义。”相近的人听到了, 
也退回来商议说:“当初我所倚仗的是与上相近,现在上面举义而不避远人, 
那我不可不为义。”远处的人听了,也退回来商议说:“当初我以为与上面 
太疏远而无所倚仗,现在上面举义而不避远,那我不可不为义。”一直到边 
鄙郊外的臣僚,宫庭宿卫人员。国都的民众,四野的农民听到,都争先为义, 
这是什么原故呢?这是因为君上用来支使臣下的是一件事,臣下用来侍奉君 
上的也是同一条道。这好比富人有高墙深宫,墙已经立好了,仅只在上面开 
一个门,有强盗进来了,关掉他进入的那张门来捉拿,强盗就无从出去了。 
这是什么原因呢?这是在上面的得其要领。 
所以古时圣王为政,任德尊贤,即使是从事农业或手工、经商的人,有 
能力的就选拔他,给他高爵,给他厚禄,给他任务,给他权力。即是说,如 
果爵位不高,民众对他就不会敬重;俸禄不厚,民众对他就不信任;如果权 
力不大,民众对他就不畏惧。拿这三种东西给贤人,并不是对贤人予以赏赐, 
而是要把事情办成。所以在这时,根据德行任官,根据官职授权,根据功劳 
定赏。衡量各人功劳而分予禄位,所以做官的不会永远富贵,而民众不会永 
远贫贱。有能力的就举用他,没有能力的就罢黜他。举公义,避私怨,说的 
即这个意思。 
所以古时尧把舜从服泽之阳拔举出来,授予他政事,结果天下大治;禹 
把益从阴方之中拔举出来,授予他政事,结果天下统一;汤把伊尹从庖厨之 
中拔举出来,授予他政事,结果计谋得行;文王把闳夭、泰颠从狩猎者中拔 
举出来,授予他政事,结果西土大服。在这些时候,即使处在厚禄尊位的大 
臣,没有不敬惧而不邪的,即使处在农业与手工、经商地位的,没有不争相 
勉励而崇尚道德的。所以贤士是用来作为辅佐和接替的人选的。因此,得到 
了士,计谋就不会困乏,身体也不会劳苦,名立而功成,美的更加彰著,恶 
的不会产生。这都是因为得到贤士。所以墨子说道:“得意之时不可不举用 
贤士,不得意之时也不可不举用贤士。如果想继承尧舜禹汤的大道,就不可 
不尚贤。尚贤是政治的根本所在。” 

\chapter{九  尚贤中  }

子墨子言曰:“今王公大人之君人民、主社稷、治国家,欲修保而勿失, 
故不察尚贤为政之本也!何以知尚贤之为政本也?曰:自贵且智者为政乎愚 
且贱者则治,自愚贱者为政乎贵且智者则乱。是以知尚贤之为政本也。 
故古者圣王甚尊尚贤而任使能,不党父兄,不偏富贵,不嬖颜色(1)。贤 
者举而上之,富而贵之,以为官长;不肖者抑而废之,贫而贱之,以为徒役。 
是以民皆劝其赏,畏其罚,相率而为贤者,以贤者众而不肖者寡,此谓进贤。 
然后圣人听其言,迹其行,察其所能而慎予官,此谓事能。故可使治国者使 
治国,可使长官者使长官,可使治邑者使治邑。凡所使治国家、官府、邑里, 
此皆国之贤者也。 
贤者之治国也,蚤朝晏退(2),听狱治政,是以国家治而刑法正。贤者之 
长官也,夜寝夙兴,收敛关市、山林、泽梁之利,以实官府,是以官府实而 
财不散。贤者之治邑也,蚤出莫入(3),耕稼树艺、聚菽粟,是以菽粟多而民 
足乎食。故国家治则刑法正,官府实则万民富。上有以洁为酒醴粢盛以祭祀 
天、鬼,外有以为皮币,与四邻诸侯交接,内有以食饥息劳,将养其万民, 
外有以怀天下之贤人。是故上者天鬼富之,外者诸侯与之,内者万民亲之, 
贤人归之。以此谋事则得,举事则成,入守则固,出诛则强。故唯昔三代圣 
王尧舜禹汤文武之所以王天下,正诸侯者,此亦其法已。 
既曰若法,未知所以行之术,则事犹若未成。是以必为置三本。何谓三 
本?曰:爵位不高,则民不敬也;蓄禄不厚,则民不信也;政令不断,则民 
不畏也。故古圣王高予之爵,重予之禄,任之以事,断予之令。夫岂为其臣 
赐哉?欲其事之成也。《诗》曰:“告女忧恤,诲女予爵,孰能执热,鲜不 
用濯?”则此语古者国君诸侯之不可以不执善承嗣辅佐也。譬之犹执热之有 
濯也,将休其手焉。古者圣王唯毋得贤人而使之,般爵以贵之(4),裂地以封 
之,终身不厌。贤人唯毋得明君而事之,竭四肢之力,以任君之事,终身不 
倦。若有美善则归之上。是以美善在上,而所怨谤在下;宁乐在君,忧戚在 
臣。故古者圣王之为政若此。 
今王公大人亦欲效人,以尚贤使能为政,高予之爵而禄不从也。夫高爵 
而无禄,民不信也,曰:“此非中实爱我也,假藉而用我也。”夫假藉之, 
民将岂能亲其上哉?故先王言曰:“贪于政者,不能分人以事;厚于货者, 
不能分人以禄。”事则不与,禄则不分,请问天下之贤人将何自至乎王公大 
人之侧哉?若苟贤者不至乎王公大人之侧,则此不肖者在左右也。不肖者在 
左右,则其所誉不当贤,而所罚不当暴。王公大人尊此(5),以为政乎国家, 
则赏亦必不当贤,而罚亦必不当暴。若苟赏不当贤而罚不当暴,则是为贤者 
不劝,而为暴者不沮矣。是以入则不慈孝父母,出则不长弟乡里(6)。居处无 
节,出入无度,男女无别。使治官府则盗窃,守城则倍畔,君有难则不死, 
出亡则不从。使断狱则不中,分财则不均。与谋事不得,举事不成,入守不 
固,出诛不强。故虽昔者三代暴王桀纣幽厉之所以失措其国家,倾覆其社稷 
者,已此故也。何则?皆以明小物而不明大物也。 
今王公大人有一衣裳不能制也,必藉良工;有一牛羊不能杀也,必藉良 
宰。故当若之二物者,王公大人未知以尚贤使能为政也。逮至其国家之乱, 
社稷之危,则不知使能以治之。亲戚则使之,无故富贵、面目佼好则使之(7)。 
夫无故富贵、面目佼好则使之,岂必智且有慧哉?若使之治国家,则此使不 

智慧者治国家也。国家之乱,既可得而知已。 
且夫王公大人有所爱其色而使,其心不察其知,而与其爱。是故不能治 
百人者,使处乎千人之官;不能治千人者,使处乎万人之官,此其故何也? 
曰:处若官者,爵高而禄厚,故爱其色而使之焉!夫不能治千人者,使处乎 
万人之官,则此官什倍也。夫治之法将日至者也,日以治之,日不什修,知 
以治之,知不什益。而予官什倍,则此治一而弃其九矣。虽日夜相接,以治 
若官,官犹若不治。此其故何也?则王公大人不明乎以尚贤使能为政也。故 
以尚贤使能为政而治者,夫若言之谓也;以下贤为政而乱者,若吾言之谓也; 
以下贤为政而乱者,若吾言之谓也。今王公大人中实将欲治其国家,欲修保 
而勿失,故不察尚贤为政之本也? 
且以尚贤为政之本者,亦岂独子墨子之言哉?此圣王之道,先王之书, 
距年之言也。传曰:“求圣君哲人,以裨辅而身。”《汤誓》曰:“聿求元 
圣,与之戮力同心,以治天下。”则此言圣之不失以尚贤使能为政也。 
故古者圣王唯能审以尚贤使能为政,无异物杂焉,天下皆得其利。古者 
舜耕历山,陶河濒,渔雷泽。尧得之服泽之阳,举以为天子,与接天下之政, 
治天下之民。伊挚,有莘氏女之私臣,亲为庖人。汤得之,举以为己相,与 
接天下之政,治天下之民。傅说被褐带索,庸筑乎傅岩(8)。武丁得之,举以 
为三公,与接天下之政,治天下之民。此何故始贱卒而贵,始贫卒而富?则 
王公大人明乎以尚贤使能为政,是以民无饥而不得食,寒而不得衣,劳而不 
得息,乱而不得治者。 
故古圣王以审以尚贤使能为政,而取法于天。虽天亦不辩贫富、贵贱、 
远迩、亲疏,贤者举而尚之,不肖者抑而废之。 
然则富贵为贤以得其赏者谁也?曰:若昔者三代圣王尧舜禹汤文武者是 
也。所以得其赏何也?曰:其为政乎天下也,兼而爱之,从而利之;又率天 
下之万民,以尚尊天事鬼,爱利万民。是故天、鬼赏之,立为天子,以为民 
父母。万民从而誉之“圣王”,至今不已。则此富贵为贤以得其赏者也。 
然则富贵为暴以得其罚者谁也?曰:若昔者三代暴王桀纣幽厉者是也。 
何以知其然也?曰:其为政乎天下也,兼而憎之,从而贼之,又率天下之民 
以诟天侮鬼,贼傲万民。是故天、鬼罚之,使身死而为刑戮,子孙离散,室 
家丧灭,绝无后世。万民从而非之曰“暴王”,至今不已。则此富贵为暴而 
以得其罚者也。 
然则亲而不善以得其罚者谁也?曰:若昔者伯鲧,帝之元子,废帝之德 
庸,既乃刑之于羽之郊,乃热照无有及也,帝亦不爱。则此亲而不善以得其 
罚者也。 
然则天之所使能者谁也?曰:若昔者禹、稷、皋陶是也。何以知其然也? 
先王之书《吕刑》道之,曰:“皇帝清问下民,有辞有苗。曰:‘群后之肆 
在下,明明不常,鳏寡不盖(9)。德威维威,德明维明’。乃名三后,恤功于 
民:伯夷降典,哲民维刑;禹平水土,主名山川;稷隆播种,农殖嘉谷。三 
后成功,维假于民(10)。”则此言三圣人者,谨其言,慎其行,精其思虑; 
索天下之隐事遗利,以上事天,则天乡其德(11);下施之万民,万民被其利, 
终身无已。故先王之言曰:“此道也,大用之天下则不窕,小用则不困,修 
用之则万民被其利,终身无已。” 
《周颂》道之曰:“圣人之德,若天之高,若地之普,其有昭于天下也; 
若地之固,若山之承,下坼不崩;若日之光,若月之明,与天地同常。”则 

此言圣人之德章明博大,埴固以修久也。故圣人之德,盖总乎天地者也。 
今王公大人欲王天下、正诸侯,夫无德义,将何以哉?其说将必挟震威 
强。今王公大人将焉取挟震威强哉?倾者民之死也!民生为甚欲,死为甚憎。 
所欲不得,而所憎屡至。自古及今,未有尝能有以此王天下、正诸侯者也。 
今大人欲王天下、正诸侯,将欲使意得乎天下,名成乎后世,故不察尚贤为 
政之本也(12)?此圣人之厚行也。 


[注释] 

(1)嬖(bì 
):宠爱。(2)蚤:通“早”。(3)莫:通“暮”。(4)般:“颁”之假借字。(5)尊: 

通“遵”。(6)长弟:即“长悌”。(7)佼:通“姣”。(8)庸:通“佣”。(9)盖:“葢”之隶变,假 

借为“害”。(10)假:通“嘏”,受福之意。(11)乡:通“享”。(12)故:与“胡”同。 
[白话] 
墨子说:现在王公大人统治人民,主持社稷,治理国家,希望永久保持 
而不失,却怎么看不到崇尚贤能是为政的根本呢!从何知道崇尚贤能是为政 
的根本呢?答道:由高贵而聪明的人去治理愚蠢而低贱的人,那么,国家便 
能治理好;由愚蠢而低贱的人去治理高贵而聪明的人,那么,国家就会混乱。 
因此知道崇尚贤能是为政的根本。 
所以古时的圣王很尊崇贤人而任用能人,不偏党父兄,不偏护富贵,不 
爱宠美色。凡是贤人便选拔上来使其处于高位,给他富贵,让他做官长;凡 
是不肖之人便免去职位,使之贫贱,让他做奴仆。于是人民相互劝赏而畏罚, 
争相做贤人,所以贤人多而不肖的人少,这便叫进贤。之后圣人听贤人的言 
语,考察他的行为,察看他的能力而谨慎地给他官职,这便叫事能。因此, 
可以让他治国的,就让他治国;可以让他居官的,就让他居官;可以让他治 
县的,就让他治县。凡是派去治理国家、官府、邑里的,都是国家的贤人。 
贤人治理国家,早上朝而晚退朝,审听刑狱,处理政务,所以国家有治 
而刑法严正;贤人长官,晚寝早起,征收关、市、山林、川泽的税利,以充 
实官家府库,所以国库充实而财用不散;贤人治理都邑,早出晚归,翻耕种 
植,多聚豆粟,所以粮食多而人民食用充足,因此国家有治而刑法严正,官 
府充实而万民富足。上能洁治酒食,去祭祀上帝鬼神,外能制造皮币,与四 
邻诸侯交往,内可以使饥者得食,劳者得息,外可以招徕天下的贤人。所以 
上则天帝鬼神给他赐富,外则诸侯与他结交,内则万民亲附,外则贤人归顺。 
因此谋事有得,做事能成,自守坚固,出征强大。所以从前三代圣王尧、舜、 
禹、汤、文、武用以统一天下,出长诸侯的法则,即在于此。 
既然有这样的法则,但如果不知道用以推行这一法则的方法,那么事情 
仍然没有办成。所以要立下三项根本(措施)。什么叫三个根本呢?答道: 
爵位不高,人民不尊敬他;俸禄不厚,人民不信服他;权力不大,人民不惧 
怕他。所以古代圣王给他高的爵位,厚的俸禄,实际的任务,决断的权力。 
这难道是给臣下以赏赐吗?为的是要把事情办成呀!《诗经》说:“告诉你 
忧人之忧,教给你安排爵位,谁能拿了火热的东西,而不用冷水洗手呢?” 
这是说古代的国君诸侯不可不亲善那些继承人和辅佐大臣,就如同拿了热的 
东西后要用冷水洗濯一样,以使自己的手得到休息。古时的圣王得到贤人而 
使用他,颁赐爵位使他显贵,分割土地作他封邑,终身都不厌弃。至于贤人 
得事奉明君,也必竭尽全力来担任国君的工作,终身不倦。如果有了美好的 
功德,就归之国君。所以功德归上而怨恨诽谤归于臣下;安宁喜乐归于国君, 

而忧愁归于臣下。古代圣王为政大概如此。 
现在王公大人也想效法古人为政,尊敬贤者,任用能者,给他们高的爵 
位,但俸禄却不随着增加。爵位高而没有相应的俸禄,人民不会相信的,说: 
“这不是真正的爱我,不过是假借虚名来使用我罢了。”象这样既然被假借 
利用,人民怎能亲附君上呢?所以先王说:“贪于权位的,不能把政事分给 
别人;重视财货的,不能把俸禄分给别人。”政事既不让人参与,俸禄又不 
分给别人,请问天底下的贤人,怎么会到王公大人的旁边来呢?如果贤人不 
来到王公大人的旁边,那就有不肖的人在左右了。不肖的人在左右,则他们 
所称赞的不会是真贤,所惩罚的也不会是真暴。王公大人遵从这些人以治理 
国家,那么所赏的也一定不会是真贤,所罚的也一定不会是真暴。如果所赏 
非贤,所罚非暴,那么做贤人的得不到勉励,而作恶的人也得不到阻止了。 
所以在家不知道孝顺父母,出外不懂得敬重乡里。居处没有节制,出入没有 
限度,男女没有区别,使他治理官府就会偷窃,使他守城就会背叛,君上有 
难不肯献身,出亡不肯追随。使他判案则不当,分财则不均,和他谋事不得 
当,让他办事无所成,让他防守不坚固,让他征伐不坚强。所以象从前三代 
暴君桀、纣、幽、厉等所以损失其国家,倾覆其社稷,就是这个原故。为什 
么呢?他们都只明了小事而不明了大事。 
现在的王公大人,有一件衣裳不能制作,必定要借助好的工匠,有一只 
牛羊不能宰杀,必定要借助好的屠夫,所以遇着上面这两种事情,王公大人 
也未尝不知道以尚贤使能为重,而一到国家丧乱,社稷倾危,就不知道尚贤 
使能以治理它。凡是亲戚就任用他,凡是无缘无故得到富贵的,面目生得美 
丽的就任用他。那些无缘无故得到富贵的,面目生得美丽的就任用,难道这 
些人都很有智慧吗?如果使他们治理国家,那是使不聪明的人治理国家呀! 
国家的混乱也就可以知道的了。 
再说王公大人因爱一个人的美貌而任用他,心中并不察知他的智慧而给 
他以宠爱,所以不能治理百人的,竟让他做一千个人的官;不能治理千人的, 
竟让他做一万个人的官。这是为什么呢?回答说:做这种官的人,爵位高而 
俸禄厚,只因爱其美色而给他这个职位。不能治理一百人的,让他做一千人 
的官;不能治理一千人的,让他做一万人的官,这是授予的官职超过其能力 
的十倍了。治理国家的原则是,每天都必须去治理。一天的时间不能延长十 
倍,而其治事的智能也不能增加十倍,那么,这样一来,他就只能治理其中 
的一份而放弃其他九份了。即使日夜不停地治理官事,官事仍然治不好。这 
是什么原因呢?是王公大人不明白尚贤使能的缘故呀!所以,因尚贤使能为 
政而大治的,所说的就是上面这样的话。因下贤不使能为政而混乱的,就象 
我所说的一样。现在的王公大人,心中真正想治理国家,为什么不去体察尚 
贤为政这些根本呢? 
再说以尚贤使能作为政治的根本,又岂止是墨子这样说的呢?这原是圣 
王的道理,先王的书,老年人的话。传记说:“求圣君和哲人,以辅助你身。” 
《汤誓》说:“求到大圣,和他戮力同心,以治天下。”这些都说明圣人不 
放弃以尚贤使能治理国家。 
所以古时圣王只因能以尚贤使能治理政事,没有其他事情掺杂在内,因 
此天下都得其好处。古时舜在历山耕地,在河滨制陶器,在雷泽捕鱼,尧帝 
在服泽之阳找到他,选拔他作天子,让他掌管天下的政事,治理天下的人民。 
伊尹本是有莘氏的陪嫁私臣,身为厨子,汤得到他,任用他为宰相,让他掌 

管天下的政事,治理天下的人民。傅说身穿粗布衣,围着绳索,在傅岩受佣 
筑墙,武丁得到他,任用他为三公,让他掌管天下的政事,治理天下的人民, 
他们为什么始贱终贵,始贫终富呢?是因为王公大人懂得以尚贤使能治理国 
政。所以人民没有饥不得食,寒不得衣,劳不得息,乱不得治的。 
所以古时的圣王能审慎地以尚贤使能治理国政,而取法于天。只有天不 
分贫富贵贱,远近亲疏,凡贤人就选拔而重用他,不肖的人就抑制而废弃他。 
既然这样,那么,那些富贵而行仁政的人,又有哪些人得到上天的赏赐呢? 
回答说:象从前的圣王尧、舜、禹、汤、文、武等都是。他们又怎样得到赏 
赐呢?回答说:他们治理天下,能够相爱互利,又率领天下万民崇尚尊天事 
鬼,爱利人民。所以天地鬼神赏赐他们,立他们为天子,做人民的父母,人 
民从而称赞他们为“圣王”,至今不息。这就是富贵事贤而得到赏赐的。 
那么富贵行暴而得到惩罚的又有哪些人呢?回答说:象从前三代的暴君 
桀、纣、幽、厉就是。怎么知道呢?回答说:他们统治天下,互相仇恨和残 
害,又率领天下的人民咒骂上天,侮慢鬼神,残害万民。所以上天鬼神给他 
们惩罚,使他们本身被刑戮,子孙离散,家室毁灭,没有后代,万民从而毁 
骂他们为“暴王”,至今不息。这就是富贵行暴得到惩罚的。 
那么,亲近的人行为不善,而得到惩罚的又有谁呢?回答说:象从前的 
伯鳏,是帝颛顼的长子,败坏了帝的功德,不久就被诛杀于羽郊,那是日月 
所照不及之处,帝也不爱他。这就是亲近的人行为不善而得到惩罚的。 
那么,天所使用贤能的有谁呢?回答说:象从前禹、稷、皋陶就是。怎 
么知道这样呢?先王之书《吕刑》说过:“尧帝询问人民所患,人民都回答 
有苗为害。帝尧说:‘各位君主以及在下执事之人,凡是有德之人即可显用, 
鳏寡之人也没有关系。出于崇高品德的威严才是真正的威严,出于崇高品德 
的明察才是真正的明察。’于是命令伯夷、禹、稷三君,忧虑勤劳民事:伯 
益制定典礼,使人民效法哲人;禹平治水土,制定山川的名称;稷教民播种, 
让人民努力耕种粮食。这三君的成功,使人民大受其福。”这说的是三位圣 
人,谨言慎行,精心考虑,去求索天下没有被发现的事物和被遗忘的利益。 
以此上奉于天,天即享用其德;以此下施于万民,万民即蒙受其利,终身不 
止。所以先王的话说:“这种道,用到治天下这大处来说就不会缺损;用到 
小处来说,也不会困塞。长久用它,则万民受其利,终身不止。”《周颂》 
曾说过:“圣人的德行,象天一样高,象地一样广,光照于天下;象日一样 
光明,象月一样明朗,象天地一样长久。”这说的是圣人的德行彰明博大, 
坚牢而长久。所以圣人之德,总合天地之美德。 
现在的王公大人要统治天下,为诸侯之长,没有德义,那依靠什么呢? 
他们说必用威力和强权。现在王公大人将会从使用威力强权中得到什么呢? 
它必然把人民引上倾毁死亡之路。人民对生都十分爱惜,对死都十分憎恨。 
他们得不到自己所希求的,而常常得到所厌恶的。从古到今,绝对没有以这 
种方式统一天下、称霸诸侯的。现在王公大人想统一天下,称霸诸侯,将要 
使自己得志于天下,成名于后世,为什么不看到尚贤这一为政之根本呢?这 
是圣人崇高的品行所在。 

\chapter{十  尚贤下}

尚贤下

子墨子言曰:天下之王公大人皆欲其国家之富也,人民之众也,刑法之 
治也。然而不识以尚贤为政其国家百姓,王公大人本失尚贤为政之本也。若 
苟王公大人本失尚贤为政之本也,则不能毋举物示之乎? 
今若有一诸侯于此,为政其国家也,曰:“凡我国能射御之士,我将赏 
贵之;不能射御之士,我将罪贱之。”问于若国之士,孰喜孰惧?我以为必 
能射御之士喜,不能射御之士惧。我赏因而诱之矣(1),曰:“凡我国之忠信 
之士,我将赏贵之;不忠信之士,我将罪贱之。”问于若国之士,孰喜孰惧? 
我以为必忠信之士喜,不忠不信之士惧。今惟毋以尚贤为政其国家百姓,使 
国为善者劝,为暴者沮(2)。大以为政于天下,使天下之为善者劝,为暴者沮。 
然昔吾所以贵尧舜禹汤文武之道者,何故以哉?以其唯毋临众发政而治民, 
使天下之为善者可而劝也,为暴者可而沮也。然则此尚贤者也,与尧舜禹汤 
文武之道同矣。 
而今天下之士君子,居处言语皆尚贤;逮至其临众发政而治民,莫知尚 
贤而使能。我以此知天下之士君子,明于小而不明于大也。何以知其然乎? 
今王公大人有一牛羊之财不能杀,必索良宰;有一衣裳之财不能制,必索良 
工。当王公大人之于此也,虽有骨肉之亲、无故富贵、面目美好者,实知其 
不能也,不使之也。是何故?恐其败财也。当王公大人之于此也,则不失尚 
贤而使能。王公大人有一罢马不能治(3),必索良医;有一危弓不能张,必索 
良工。当王公大人之于此也,虽有骨肉之亲、无故富贵、面目美好者,实知 
其不能也,必不使。是何故?恐其败财也。当王公大人之于此也,则不失尚 
贤而使能。逮至其国家则不然,王公大人骨肉之亲、无故富贵、面目美好者 
则举之。则王公大人之亲其国家也,不若亲其一危弓、罢马、衣裳、牛羊之 
财与?我以此知天下之士君子,皆明于小而不明于大也。此譬犹喑者而使为 
行人,聋者而使为乐师。是故古之圣王之治天下也,其所富,其所贵,未必 
王公大人骨肉之亲、无故富贵、面目美好者也。是故昔者舜耕于历山,陶于 
河濒,渔于雷泽,灰于常阳。尧得之服泽之阳,立为天子。使接天下之政, 
而治天下之民。昔伊尹为莘氏女师仆,使为庖人。汤得而举之,立为三公, 
使接天下之政,治天下之民。昔者傅说居北海之洲,圜土之上(4),衣褐带索, 
庸筑于傅岩之城。武丁得而举之,立为三公,使之接天下之政,而治天下之 
民。是故昔者尧之举舜也,汤之举伊尹也,武丁之举傅说也,岂以为骨肉之 
亲、无故富贵、面目美好者哉?惟法其言,用其谋,行其道,上可而利天, 
中可而利鬼,下可而利人,是故推而上之。 
古者圣王既审尚贤,欲以为政,故书之竹帛,琢之槃盂,传以遗后世子 
孙。于先王之书《吕刑》之书然:王曰:“於!来!有国有士,告女讼刑。 
在今而安百姓,女何择言人?何敬不刑?何度不及?”能择人而敬为刑,尧 
舜禹汤文武之道可及也。是何也?则以尚贤及之。于先王之书、竖年之言然, 
曰:“晞夫圣武知人(5),以屏辅而耳。”此言先王之治天下也,必选择贤者, 
以为其群属辅佐。 
曰:今也天下之士君子,皆欲富贵而恶贫贱,曰然女何为而得富贵而辟 
贫贱?莫若为贤,为贤之道将奈何?曰:有力者疾以助人,有财者勉以分人, 
有道者劝以教人。若此,则饥者得食,寒者得衣,乱者得治。若饥则得食, 
寒则得衣,乱则得治,此安生生。 

今王公夫人,其所富,其所贵,皆王公大人骨肉之亲、无故富贵、面目 
美好者也。今王公大人骨肉之亲、无故富贵、面目美好者,焉故必知哉?若 
不知,使治其国家,则其国家之乱,可得而知也。 
今天下之士君子皆欲富贵而恶贫贱,然女何为而得富贵而辟贫贱哉? 
曰:莫若为王公大人骨肉之亲、无故富贵、面目美好者。王公大人骨肉之亲、 
无故富贵、面目美好者,此非可学能者也。使不知辩,德行之厚,若禹汤文 
武,不加得也;王公大人骨肉之亲,躄喑聋暴为桀纣,不加失也。是故以赏 
不当贤,罚不当暴。其所赏者,已无故矣;其所罚者,亦无罪。是以使百姓 
皆攸心解体(6),沮以为善;垂其股肱之力(7),而不相劳来也;腐臭余财, 
而不相分资也;隐慝良道,而不相教诲也。若此则饥者不得食,寒者不得衣, 
乱者不得治。 
推而上之以(8),是故昔者尧有舜,舜有禹,禹有皋陶,汤有小臣,武王 
有闳夭、泰颠、南宫括、散宜生,而天下和,庶民阜。是以近者安之,远者 
归之。日月之所照,舟车之所及,雨露之所渐,粒食之所养,得此莫不劝誉。 
且今天下之王公大人士君子,中实将欲为仁义,求为上士,上欲中圣王之道, 
下欲中国家百姓之利,故尚贤之为说,而不可不察此者也。尚贤者,天、鬼、 
百姓之利而政事之本也。 


[注释] 

(1)赏:当作“尝”。(2)沮:止。(3)罢:同“疲”。(4)圜(yuán)土:狱。(5)晞:通“希”。 

(6)攸:疑为“散”字之误。(7)垂:“堕”之借字。(8)推而上之以:此句为衍文。 
[白话] 
墨子说:天下的王公大人都希望自己的国家富足,人民众多,政治安定。 
但却不知道以尚贤作为对国家百姓为政的原则。王公大人从来就不知道尚贤 
是政治的根本。如果王公大人从来不知道尚贤这一治理政事的根本,我们就 
不能举出事例来开导他吗? 
现在假定这里有一个诸侯,在他的国家治理政事,说道:“凡是我国能 
射箭和驾车的人,我都将奖赏和尊贵他;不能射箭和驾车的人,我都将治罪 
和贱视他。”试问这个国家的人士,谁高兴谁害怕呢?我认为必定是善于射 
箭驾车的人高兴,不善于射箭驾车的人害怕。我曾顺着前一假设进一步申说: 
“凡是我国忠信之人,我都将奖赏和尊贵他;不忠不信的人,我都将治罪和 
贱视他。”试问这个国家的人士,谁高兴谁害怕呢?我认为必定是忠信的人 
高兴,不忠不信的人害怕。现在对自己的国家人民采取尚贤政治,使一国为 
善的人受到勉励,行暴的人受到阻止,大之行使政治于天下,使天下为善的 
人受到勉励,行暴的人受到阻止。我以前所以看重尧、舜、禹、汤、文、武 
之道,是什么缘故呢?因为他面对民众发布政令以治理人民,使天下为善的 
人可以受到勉励,行暴的人可以受到阻止。这就是尚贤,它和尧、舜、禹、 
汤、文、武之道是相同的。 
而今天下的士君子,平时言谈都知道尚贤,而一到他们面对民众发布政 
令以治理人民,就不知道尚贤使能了。我由此知道天下的士君子,只懂得小 
道理而不懂得大道理。怎么知道这样呢?现在的王公大人有一只牛羊不会 
杀,一定去找好的屠夫;有一件衣裳不会做,一定去找好的工匠。当王公大 
人在此之时,虽然有骨肉之亲,和无缘无故得到富贵者,以及面貌美丽的人, 
如果确实知道他们没有能力,就不会让他去做。为什么呢?因为担心损失自 

己的财物。当王公大人在此之时,尚不失为一个尚贤使能的人。王公大人有 
一匹病马不能治,一定要找好的兽医,有一张坏弓拉不开,一定要找好的工 
匠,当王公大人在此之时,虽然有骨肉之亲,和无缘无故得到富贵者,以及 
面貌美丽的人,如果确实知道他们没有能力,就不会使他去做。为什么呢? 
因为担心损失自己的财物。当王公大人在此之时,尚不失为一个尚贤使能的 
人。但一到他治理国家就不这样了。王公大人的骨肉之亲,无缘无故富贵以 
及面貌美丽的人,就举用他。如此看来,则王公大人爱他自己的国家,还不 
如爱他的一张坏弓、一匹病马、一件衣裳、一只牛羊 ?我因此知道天下的 
士君子只看到小处,没有看到大处。这就好像一个哑巴去充当外交人员,一 
个聋子去充当乐师一样。所以古代圣王治理天下,他所富所贵的,未必是王 
公大人的骨肉之亲,和无故富贵者,以及面貌美丽的人。 
所以,从前舜在历山下耕田,在河滨制陶器,在雷泽捕鱼,在常阳烧制 
石灰。尧在服泽之地得到他,立他为天子,让他接管天下的政事,治理天下 
的人民。从前伊尹是有莘氏女的私臣,让他作厨师,汤得到并举用他,立他 
为三公,使他接管天下的政事,治理天下的人民。从前傅说住在北海之洲的 
牢狱之中,穿着粗布衣,围着绳索,像佣人一样在傅岩筑城,武丁得到并举 
用他,立他为三公,使他接管天下的政事,治理天下的人民。由此看来,从 
前尧举用舜,汤举用伊尹,武丁举用傅说,难道是因为他们是骨肉之亲、无 
缘无故富贵者以及面貌美丽的人吗?那只是仿照他们的话去做,采用他们的 
谋略,实行他们的主张,从而上可以有利于天,中可以有利于鬼,下可有利 
于人,所以把他们选拔上去。 
古时的圣王既已明白了尚贤的道理,想以此为政,所以把它写在竹帛、 
雕在槃盂上,相传而遗留给后世子孙。在先王留下的书《吕刑》中这样记载: 
王说:“呵!来!有国家有领土的人,告诉你们用刑之道。在现今你们要安 
抚百姓,你们除了贤人,还有什么可选择的呢?除了刑罚,还有什么可慎重 
的呢?还有什么考虑,不能达到呢?”能选择人而敬重叫作刑,尧、舜、禹、 
汤、文、武之道就可以达到了。这是什么原因呢?因为可以通过尚贤而达到。 
在先王之书、老人的话中这样说到:“寻求圣人、武人、智人,来辅佐你自 
身。”这是说先王治理天下,一定要选择贤能的人,做他的僚属辅佐。 
现在天下的士君子,都希望富贵而厌恶贫贱。试问,你怎么做才能得到 
富贵而避免贫贱呢?最好是做贤人。那做贤人的道理又是怎样的呢?回答 
说:有力气的赶快助人,有钱财的努力分人,有道的人勉力教人。如此,饿 
的人就可以得到食,冷的人就可以得到衣,混乱的就可以得到治理。如果饿 
的人可以得到食,冷的人可以得到衣,混乱的可以得到治理,这就可以使人 
各安其生。 
现在的王公大人,他所富的所贵的,都是王公大人们的骨肉之亲、无缘 
无故富贵以及面貌美丽的人,这样的人怎能一定聪明呢?如果不聪明,让他 
治理国家,那么国家的混乱也就可想而知了。 
现在天下的士君子,都希望富贵而厌恶贫贱,可是你要怎样才能得到富 
贵而避免贫贱呢?(他们必定)说:最好是做王公大人的骨肉之亲、无缘无 
故富贵者以及面貌美丽的人。然而王公大人的骨肉之亲、无缘无故富贵者以 
及面貌美丽的人,却并不是学得到的。假使不知分辨的话,即使德行醇厚如 
禹、汤、文、武,也不会得到任用;而王公大人的骨肉之亲,即使是跛、哑、 
聋、瞎,乃至暴虐如桀纣,也不会加以抛弃。因此,赏的不会是贤人,罚的 

不会是暴人。他所赏的人是没有功的,所罚的也是没有罪的。所以使百姓人 
心涣散,阻止他们向善:怠惰他们的肢体,而不相互勉励帮助;使多余的财 
物腐臭变质,而不相互资助;隐藏自己好的学问,而不相互教导。如此,饥 
饿的人就不会得食,寒冷的人就不会得衣,混乱的状况就不会得到治理。 
所以从前尧有舜,舜有禹,禹有皋陶,汤有伊尹,武王有闳夭、泰颠、 
南宫括、散宜生,从而天下太平,人民富足。因此,近的人安于其居,远的 
人前来归附。凡是日月所照、舟车所至、雨露所滋润、谷食所养活的人们, 
得到这些贤人,无不相互劝勉和鼓励。假如现今天下的王公大人及士君子, 
心中真想行仁义,求做上士,上则想适合圣王之道,下则想符合国家与百姓 
之利,那就不可不认真考虑尚贤这一说法了。(总之),尚贤是天帝、鬼神、 
百姓的利益所在,也是政事的根本。 

\chapter{十一  尚同上}

子墨子言曰:古者民始生,未有刑政之时,盖其语,人异义。是以一人 
则一义,二人则二义,十人则十义。其人兹众(2),其所谓义者亦兹众。是以 
人是其义,以非人之义,故交相非也。是以内者父子兄弟作怨恶离散,不能 
相和合;天下之百姓,皆以水火毒药相亏害。至有余力,不能以相劳;腐■ 
余财,不以相分;隐匿良道,不以相教。天下之乱。若禽兽然。 
夫明虖天下之所以乱者(3),生于无政长,是故选天下之贤可者,立以为 
天子。天子立,以其力为未足,又选择天下之贤可者,置立之以为三公。天 
子、三公既以立,以天下为博大,远国异土之民,是非利害之辩,不可一二 
而明知,故画分万国,立诸侯国君。诸侯国君既已立,以其力为未足,又选 
择其国之贤可者,置立之以为正长(4)。 
正长既已具,天子发政于天下之百姓,言曰:“闻善而不善,皆以告其 
上。上之所是,必皆是之;所非,必皆非之。上有过则规谏之,下有善则傍 
荐之。上同而不下比者,此上之所赏而下之所誉也。意若闻善而不善,不以 
告其上;上之所是弗能是,上之所非弗能非;上有过弗规谏,下有善弗傍荐; 
下比不能上同者,此上之所罚而百姓所毁也。”上以此为赏罚,明察以审信。 
是故里长者,里之仁人也。里长发政里之百姓,言曰:“闻善而不善, 
必以告其乡长。乡长之所是,必皆是之;乡长之所非,必皆非之。去若不善 
言,学乡长之善言;去若不善行,学乡长之善行。”则乡何说以乱哉?察乡 
之所治者何也?乡长唯能壹同乡之义,是以乡治也。 
乡长者,乡之仁人也。乡长发政乡之百姓,言曰:“闻善而不善者,必 
以告国君。国君之所是,必皆是之;国君之所非,必皆非之。去若不善言, 
学国君之善言;去若不善行,学国君之善行。”则国何说以乱哉?察国之所 
以治者何也?国君唯能壹同国之义,是以国治也。 
国君者,国之仁人也。国君发政国之百姓,言曰:“闻善而不善,必以 
告天子。天子之所是,皆是之;天子之所非,皆非之。去若不善言,学天子 
之善言;去若不善行,学天子之善行。”则天下何说以乱哉?察天下之所以 
治者何也?天子唯能壹同天下之义,是以天下治也。 
天下之百姓皆上同于天子,而不上同于天,则灾犹未去也。今若天飘风 
苦雨,溱溱而至者,此天之所以罚百姓之不上同于天者也。是故子墨子言曰: 
“古者圣王为五刑,请以治其民(5)。譬若丝缕之有纪,网罟之有纲,所连收 
天下之百姓不尚同其上者也。” 


[注释] 

(1)尚同即上同,也即人们的意见应当统一于上级,并最终统一于天。这是墨子针对当时国家混 

乱而提出的政治纲领。墨子认为,天下混乱是由于没有符合天意的好的首领,因此主张选择“仁人”、 

“贤者”担任各级领导。这种思想与尚贤说在本质上基本一致,都是对当时贵族统治的批判。本篇分 

上、中、下三篇。(2)兹:通“滋”。(3)虖:通“乎”。(4)正长:即“政长”。(5)请:诚。 
[白话] 
墨子说:古时人类刚刚诞生,还没有刑法政治的时候,人们用言语表达 
的意见,也因人而异。所以一人就有一种意见,两人就有两种意见,十人就 
有十种意见。人越多,他们不同的意见也就越多。每个人都以为自己的意见 
对而别人的意见错,因而相互攻击。所以在家庭内父子兄弟常因意见不同而 

相互怨恨,使得家人离散而不能和睦相处。天下的百姓,都用水火毒药相互 
残害,以致有余力的人不能帮助别人;有余财者宁愿让它腐烂,也不分给别 
人;有好的道理也自己隐藏起来,不肯教给别人,以致天下混乱,有如禽兽 
一般。 
明白了天下所以大乱的原因,是由于没有行政长官,所以(人们)就选 
择贤能的人,立之为天子。立了天子之后,认为他的力量还不够,因而又选 
择天下贤能的人,把他们立为三公。天子、三公已立,又认为天下地域广大, 
他们对于远方异邦的人民以及是非利害的辨别,还不能一一了解,所以又把 
天下划为万国,然后设立诸侯国君。诸侯国君已立,又认为他们的力量还不 
够,又在他们国内选择一些贤能的人,把他们立为行政长官。 
行政长官已经设立之后,天子就向天下的百姓发布政令,说道:“你们 
听到善和不善,都要报告给上面。上面认为是对的,大家都必须认为对;上 
面认为是错的,大家都必须认为错。上面有过失,就应该规谏,下面有好人 
好事,就应当广泛地推荐给国君。是非与上面一致,而不与下面勾结,这是 
上面所赞赏,下面所称誉的。假如听到善与不善,却不向上面报告;上面认 
为对的,也不认为对,上面认为错的,也不认为错;上面有过失不能规谏, 
下面有好人好事不能广泛地向上面推荐;与下面勾结而不与上面一致,这是 
上面所要惩罚,也是百姓所要非议的。”上面根据这些方面来行使赏罚,就 
必然十分审慎、可靠。 
所以里长就是这一里内的仁人。里长发布政令于里中的百姓,说道:“听 
到善和不善,必须报告给乡长。乡长认为对的,大家都必须认为对;乡长认 
为错的,大家都必须认为错。去掉你们不好的话,学习乡长的好话;去掉你 
们不好的行为,学习乡长的好行为。”那么,乡里怎么会说混乱呢?我们考 
察这一乡得到治理的原因是什么呢?是由于乡长能够统一全乡的意见,所以 
乡内就治理好了。” 
乡长是这一乡的仁人。乡长发布政令于乡中百姓,说道:“听到善和不 
善,必须把它报告给国君。国君认为是对的,大家都必须认为对;国君认为 
是错的,大家都必须认为错。去掉你们不好的话,学习国君的好话;去掉你 
们不好的行为,学习国君的好行为。”那么,还怎么能说国内会混乱呢?我 
们考察一国得到治理的原因是什么呢?是因为国君能统一国中的意见。所以 
国内就治理好了。 
国君是这一国的仁人。国君发布政令于国中百姓,说道:“听到善和不 
善,必须报告给天子。天子认为是对的,大家都必须认为对;天子认为是错 
的,大家都必须认为错。去掉你们不好的话,学习天子的好话,去掉你们不 
好的行为,学习天子的好行为。”那么,还怎么能说天下会乱呢?我们考察 
天下治理得好的原因是什么呢?是因为天子能够统一天下的意见,所以天下 
就治理好了。 
天下的老百姓都知道与天子一致,而不知道与天一致,那么灾祸还不能 
彻底除去。现在假如天刮大风下久雨,频频而至,这就是上天对那些不与上 
天一致的百姓的惩罚。所以墨子说:“古时圣王制定五种刑法,确实用它来 
治理人民,就好比丝线有纪(丝头的总束)、网罟有纲一样,是用来收紧那 
些不与上面意见一致的老百姓的。” 

\chapter{十二  尚同中}

子墨子曰:方今之时,复古之民始生,未有正长之时,盖其语曰,天下 
之人异义,是以一人一义,十人十义,百人百义。其人数兹众,其所谓义者 
亦兹众。是以人是其义,而非人之义,故相交非也。内之父子兄弟作怨雠, 
皆有离散之心,不能相和合。至乎舍余力,不以相劳;隐匿良道,不以相教; 
腐■余财,不以相分。天下之乱也,至如禽兽然。无君臣上下长幼之节、父 
子兄弟之礼,是以天下乱焉。明乎民之无正长以一同天下之义,而天下乱也, 
是故选择天下贤良、圣知、辩慧之人,立为天子,使从事乎一同天下之义。 
天子既以立矣,以为唯其耳目之请(1),不能独一同天下之义,是故选择天下 
赞阅贤良、圣知、辩慧之人,置以为三公,与从事乎一同天下之义。天子三 
公既已立矣,以为天下博大,山林远土之民,不可得而一也。是故靡分天下, 
设以为万诸侯国君,使从事乎一同其国之义。国君既已立矣,又以为唯其耳 
目之请,不能一同其国之义,是故择其国之贤者,置以为左右将军大夫,以 
至乎乡里之长,与从事乎一同其国之义。天子、诸侯之君、民之正长,既已 
定矣,天子为发政施教,曰:“凡闻见善者,必以告其上;闻见不善者,亦 
必以告其上。上之所是,亦必是之;上之所非,亦必非之。己有善,傍荐之; 
上有过,规谏之。尚同义其上,而毋有下比之心。上得则赏之,万民闻则誉 
之。意若闻见善,不以告其上;闻见不善,亦不以告其上。上之所是不能是, 
上之所非不能非。已有善,不能傍荐之;上有过,不能规谏之。下比而非其 
上者,上得则诛罚之,万民闻则非毁之。”故古者圣王之为刑政赏誉也,甚 
明察以审信。是以举天下之人,皆欲得上之赏誉而畏上之毁罚。 
是故里长顺天子政而一同其里之义。里长既同其里之义,率其里之万民 
以尚同乎乡长,曰:“凡里之万民,皆尚同乎乡长而不敢下比,乡长之所是, 
必亦是之;乡长之所非,必亦非之。去而不善言(2),学乡长之善言;去而不 
善行,学乡长之善行。”乡长固乡之贤者也。举乡人以法乡长,夫乡何说而 
不治哉?察乡长之所以治乡者,何故之以也?曰唯以其能一同其乡之义,是 
以乡治。 
乡长治其乡而乡既已治矣,有率其乡万民(3),以尚同乎国君,曰:“凡 
乡之万民,皆上同乎国君而不敢下比。国君之所是,必亦是之;国君之所非, 
必亦非之。去而不善言,学国君之善言;去而不善行,学国君之善行。”国 
君固国之贤者也,举国人以法国君,夫国何说而不治哉?察国君之所以治国 
而国治者,何故之以也?曰:唯以其能一同其国之义,是以国治。 
国君治其国而国既已治矣,有率其国之万民以尚同乎天子,曰:“凡国 
之万民,上同乎天子而不敢下比。天子之所是,必亦是之;天子之所非,必 
亦非之。去而不善言,学天子之善言;去而不善行,学天子之善行。”天子 
者,固天下之仁人也,举天下之万民以法天子,夫天下何说而不治哉?察天 
子之所以治天下者,何故之以也?曰:唯以其能一同天下之义,是以天下治。 
夫既尚同乎天子,而未上同乎天者,则天灾将犹未止也。故当若天降寒 
热不节,雪霜雨露不时,五谷不孰,六畜不遂,疾灾戾疫,飘风苦雨,荐臻 
而至者(4),此天之降罚也,将以罚下人之不尚同乎天者也。 
故古者圣王明天、鬼之所欲,而辟天、鬼之所憎,以求兴天下之害(5), 
是以率天下之万民,齐戒沐浴(6),洁为酒醴粢盛,以祭祀天、鬼。其事鬼神 
也,酒醴粢盛不敢不蠲洁(7),牺牲不敢不腯肥,珪璧币帛不敢不中度量,春 

秋祭祀不敢失时几,听狱不敢不中,分财不敢不均,居处不敢怠慢。曰:其 
为正长若此,是故上者天、鬼有厚乎其为正长也,下者万民有便利乎其为政 
长也。天、鬼之所深厚而能强从事焉,则天、鬼之福可得也。万民之所便利 
而能强从事焉,则万民之亲可得也。其为政若此,是以谋事得,举事成,入 
守固,出诛胜者,何故之以也?曰:唯以尚同为政者也。故古者圣王之为政 
若此。 
今天下之人曰:“方今之时,天下之正长犹未废乎天下也,而天下之所 
以乱者,何故之以也?”子墨子曰:“方今之时之以正长,则本与古者异矣。 
譬之若有苗之以五刑然。昔者圣王制为五刑以治天下,逮至有苗之制五刑, 
以乱天下,则此岂刑不善哉?用刑则不善也。是以先王之书《吕刑》之道曰: 
‘苗民否用练(8),折则刑,唯作五杀之刑,曰法。’则此言善用刑者以治民, 
不善用刑者以为五杀。则此岂刑不善哉?用刑则不善,故遂以为五杀。是以 
先王之书《术令》之道曰:‘唯口出好兴戎。’则此言善用口者出好,不善 
用口者以为谗贼寇戎,则此岂口不善哉?用口则不善也,故遂以为谗贼寇 
戎。” 
故古者之置正长也,将以治民也。譬之若丝缕之有纪,而网罟之有纲也。 
将以运役天下淫暴而一同其义也。是以先王之书、相年之道曰:“夫建国设 
都,乃作后王君公,否用泰也。轻大夫师长,否用佚也。维辩使治天均(9)。” 
则此语古者上帝鬼神之建设国都立正长也,非高其爵,厚其禄,富贵佚而错 
之也(10)。将此为万民兴利除害,富贵贫寡,安危治乱也。故古者圣王之为 
若此。 
今王公大人之为刑政则反此:政以为便譬、宗於父兄故旧,以为左右, 
置以为正长。民知上置正长之非正以治民也,是以皆比周隐匿,而莫肯尚同 
其上。是故上下不同义。若苟上下不同义,赏誉不足以劝善,而刑罚不足以 
沮暴。何以知其然也? 
曰:上唯毋立而为政乎国家,为民正长,曰:“人可赏,吾将赏之。” 
若苟上下不同义,上之所赏,则众之所非。曰人众与处,于众得非,则是虽 
使得上之赏,未足以劝乎!上唯毋立而为政乎国家,为民正长,曰:“人可 
罚,吾将罚之。”若苟上下不同义,上之所罚,则众之所誉。曰人众与处, 
于众得誉,则是虽使得上之罚,未足以沮乎!若立而为政乎国家,为民正长, 
赏誉不足以劝善,而刑罚不沮暴,则是不与乡吾本言“民始生未有正长之时” 
同乎(11)?若有正长与无正长之时同,则此非所以治民一众之道。 
故古者圣王唯而审以尚同,以为正长,是故上下情请为通。上有隐事遗 
利,下得而利之;下有蓄怨积害,上得而除之。是以数千万里之外,有为善 
者,其室人未遍知,乡里未遍闻,天子得而赏之;数千万里之外,有为不善 
者,其室人未遍知,乡里未遍闻,天子得而罚之。是以举天下之人,皆恐惧 
振动惕栗,不敢为淫暴,曰:“天子之视听也神!”先王之言曰:“非神也。 
夫唯能使人之耳目助己视听,使人之吻助己言谈,使人之心助己思虑,使人 
之股肱助己动作。”助己视听者众,则其所闻见者远矣;助之言谈者众,则 
其德音之所抚循者博矣,助之思虑者众,则其谈谋度速得矣;助之动作者众, 
即其举事速成矣。故古者圣人之所以济事成功,垂名于后世者,无他故异物 
焉,曰:唯能以尚同为政者也。 
是以先王之书《周颂》之道之曰:“载来见辟王,聿求厥章。”则此语 
古者国君诸侯之以春秋来朝聘天子之廷,受天子之严教,退而治国,政之所 

加,莫敢不宾。当此之时,本无有敢纷天子之教者。《诗》曰:“我马维骆, 
六辔沃若,载驰载驱,周爰咨度。”又曰:“我马维骐,六辔若丝,载驰载 
驱,周爰咨谋。”即此语也。古者国君诸侯之闻见善与不善也,皆驰驱以告 
天子。是以赏当贤,罚当暴,不杀不辜,不失有罪,则此尚同之功也。是故 
子墨子曰:“今天下之王公大人士君子,请将欲富其国家(12),众其人民, 
治其刑狱,定其社稷,当若尚同之不可不察,此之本也。” 


[注释] 

(1)请:通“情”。(2)而:通“尔”。(3)有:通“又”。(4)荐臻:联绵词,重沓之意。(5)此 

句当为:“以求兴天下之利,除天下之害。”(6)齐:通“斋”。(7)蠲:通“涓”。(8)练:与“灵”、 

“命”一声之转。(9)辩:通“辨”。(10)错:通“措”。(11)乡:通“向”。(12)请:诚。 
[白话] 
墨子说:从现在回头考察古代人类刚刚诞生,还没有行政长官的时候, 
他们的说法是:“天下各人的意见不一样。”所以一人有一种意见,十人有 
十种意见,百人有百种意见。人数越多,意见也就越多。所以每人都认为自 
己的意见对,而认为别人的意见错,因而相互攻击。在家内父子兄弟相互怨 
恨,都有离散之心,不能和睦相处。以致有余力的不愿意帮助别人;把好的 
道理隐藏起来,不愿意指教别人;让多余的财物腐烂,也不愿意分给别人, 
因此天下混乱,如同禽兽一般,没有君臣上下长幼的区别,没有父子兄弟之 
间的礼节,因此天下大乱。明白了没有行政长官来统一天下的意见,天下就 
会大乱,所以人们就选择天下贤良、聪明而口才好的人,推举他立为天子, 
使他从事于统一天下的意见。天子已立,认为仅仅依靠自己耳闻目见的情况, 
不能独自统一天下的意见,所以又选择考察天下贤良、聪明而口才好的人, 
推举他为三公,参与从事统一天下的意见。天子、三公已经立定了,又因天 
下地域太广,远方山野的人民,不可能统一,所以划分天下,设立了数以万 
计的诸侯国君,让他们从事于统一他们各国的意见。国君既已立定了,又因 
但靠他一人的耳目所及,尚不能统一一国的意见,所以又在他们国内选择一 
些贤人,立为国君左右的将军、大夫,以及远至乡里之长,让他们参加从事 
统一国内的意见。天子、诸侯国君、人民的行政长官既已立定,天子就发布 
政令,说:“凡听到或看到善,必须报告给上面;凡听到或看到不善,也必 
须报告给上面。上面认为是对的,必须也认为对;上面认为是错的,也必须 
认为错。自己有好的计谋,就广泛地献给上面;上面有过失,就加以规谏。 
与上面意见一致,而不要有与下面勾结的私心。这样,上面得知就会赏赐他, 
万民听见了就会赞美他。假如听到或看到善,而不报告给上面;凡听到或看 
到不善,也不报告给上面。上面认为对的,不肯说对,上面认为错的,不肯 
说错。自己有好的计谋,不能广泛地献给上面;上面有过失,也不能予以规 
谏。与下面勾结而非毁上面。凡此等人,上面得知就要诛罚他,万民听见了 
就要非议他。”所以古时圣王制定刑法赏誉,都非常明察、可靠。因此凡是 
天下的人民,都希望得到上面的赏赐赞扬,而害怕上面的非毁与惩罚。 
所以里长顺从天子的政令,使他这一里内意见一致。里内意见一致了, 
又率领里内的人民向上与乡长意见一致,说:“凡里内的人民,都应该上同 
于乡长,而不敢与下面勾结。乡长认为是对的,大家都必须认为对;乡长认 
为错的,大家也都必须认为错。去掉你们不好的话,学习乡长的好话;去掉 
你们不好的行为,学习乡长的好行为。”乡长本是乡内的贤人。如果全乡人 

都能效法乡长,还能说乡内会治不好吗?考察之所以能把乡内治好,是什么 
缘故呢?回答说:只因为他能使全乡意见一致,所以乡内就治理好了。 
乡长治理他的乡,而乡内已经治理好了,又率领他乡内的万民,以上同 
于国君,说:“凡是乡内的万民,都应上同于国君,而不可与下面勾结。国 
君认为是对的,大家也必须认为对;国君认为错的,大家也必须认为错。去 
掉你们不好的话,学习国君的好话;去掉你们不好的行为,学习国君的好行 
为。”国君本是一国之中的贤人,如果国中所有的人都能效法国君,那么还 
能说这一国会治不好吗?考察国君所以能把国内治好,是什么缘故呢?回答 
说:“只因为他能统一全国的意见,所以国内就治理好了。” 
国君治理他本国,而国内已治理好了,又率领他国内的万民,以上同于 
天子,说:“凡是国内的万民,都应上同于天子,而不可与下面勾结。天子 
认为是对的,大家也必须认为对;天子认为错的,大家也必须认为错。去掉 
你们不好的话,学习天子的好话;去掉你们不好的行为,学习天子的好行为。” 
天子本是天下最仁爱的人,如果全天下的万民都能效法天子,那么还能说天 
下会治理不好吗?考察天子所以能把天下治理好,是什么缘故呢?回答说: 
“只因为他能统一天下的意见,所以天下就治理好了。 
已经做到上同于天子,而还不能上同于天,那么天灾还会不止。假如遇 
到气候的寒热不调,雪霜雨露降得不是时候,五谷不熟,六畜不蕃,疾疫流 
行,暴风久雨等等,一再来临,这就是上天降下的惩罚,用以惩诫那些不愿 
上同于天的世人。 
所以古时的圣王知道天帝鬼神喜欢什么,从而能避免天帝鬼神所憎恶的 
东西,以求(兴天下之利,除)天下之害,所以率领天下的万民,斋戒沐浴, 
预备了洁净而丰盛的酒饭,用来祭祀天帝鬼神。他们对鬼神的奉祀,酒饭不 
敢不洁净丰盛;牺牲不敢不肥壮硕大;珪璧币帛不敢不合乎大小标准;春秋 
二季的祭祀,不敢错过时间;审理狱讼,不敢不公正;分配财物,不敢不均 
匀;待人处事不敢怠慢礼节。这是说:他象这样当行政长官,在上的天帝鬼 
神优厚地看待他,在下的万民也便利他。天帝鬼神优厚地看待他,而他能努 
力办事,那么他就可以得到天帝鬼神的福了;万民便利他,而他能努力办事, 
那么他就可以得到万民的爱戴了。他以此治理政事,所以谋事得计,作事成 
功,守御坚固,出战胜利。这是什么缘故呢?回答说:只因为他在治理政事 
上能统一意见。所以古代圣王治理政事是这样的。 
现在天下的人说:“在今天,存在于普天之下的各种行政长官并未废除, 
而造成天下混乱的原因在哪里呢?”墨子说:“现在天下的行政长官,根本 
就和古代不同,就好像有苗族制订五刑那样。古代的圣王制定五刑,用来治 
理天下;等到有苗族制定五刑,却用来扰乱天下。这难道就是刑法不好吗? 
是刑法使用得不好。所以先王的书《吕刑》上这样记载:‘苗民不服从政令, 
就加之以刑。他们作了五种意在杀戮的刑罚,也叫作法。’这说的是善于用 
刑罚可以治理人民,不善用刑罚就变成五杀了。这难道是刑法不好吗?是刑 
法使用得不好,所以就变成了五杀。所以先王的书《术令》(即《说命》) 
记载说:‘人之口,可以产生好事,也可以产生战争。’这说的就是善用口 
的,可以产生好事;不善用口的,就可以产生谗贼战争。这难道是口不好吗? 
是由于不善用口,所以就变成谗贼战争。” 
所以古时候设置行政长官,是用来治理人民的。就好像丝线有纪(线头)、 
网罟有纲一样,他们是用来收服天下淫暴之徒,并使之与上面协同一致的。 

所以先王的书、老年人的话说过:“建国设都,设立天子诸侯,不是让他骄 
奢淫佚的;而设卿大夫师长,也不是叫他们放纵逸乐的,乃是让他们分授职 
责,按公平之天道治理(人民)。”这说的就是古时天帝鬼神建设国都,设 
置官长,并不是为了提高他们的爵位,增加他们的俸禄,使他过富贵淫佚的 
生活,而是让他给万民兴利除害,使贫者富,使民少者众,使危者安,使乱 
者治。所以古代圣王的作为是这样的。 
现在的王公大人行使政事却与此相反:将宠幸的弄臣、宗亲父兄或世交 
故旧,安置在左右,都置立为行政长官。于是人民知道天子设立行政长官并 
不是为了治理人民,所以大家都结党营私,隐瞒良道,不肯与上面意见一致。 
因此,上面与下面对于事理的看法发生偏差。假如上面与下面意见不一致, 
那么赞赏不能勉励人向善,而刑罚也不能阻止暴行。怎么知道是这样呢?回 
答说:假定处在上位、管理着国家、作为人民行政长官的人说:“这个人可 
以赏,我将赏他。”如果上面和下面意见不一致,上面所赏的人,正是大家 
所非议的人,说我们众人与他相处,众人都认为他不好。那么,这人即使得 
到上面的赏,也就不能起劝勉作用了!假定处在上位,管理着国家,作为人 
民行政长官的人说:“这个人可以罚,我将要罚他。”如果上面和下面意见 
不一致,上面所罚的人,正是大家所赞誉的人,说我们众人与他相处,众人 
都赞誉他好。那么,这人即使得到惩罚,也不能阻止不善了!假定处在上位、 
管理着国家、作为人民行政长官的人赞赏不能劝善,而刑罚又不能止暴,那 
不是与我前面说过的“人民刚产生,没有长官之时”的情况一样了吗?如果 
有行政长官与没有行政长官的时候一样,那么这就不是用来治理人民、统一 
民众的办法。所以古代的圣王,因为能够审慎地统一民众的意见,立为行政 
长官,所以上下之情就沟通了。上面若有尚被隐蔽而遗置的利益,下面的人 
能够随时开发他,使他得到好处;下面若有蓄积的怨和害,上面也能够随时 
除掉他。所以远在数千或数万里之外,如果有人做了好事,他的家人还未完 
全知道,他的乡人也未完全听到,天子就已知道并赏赐他;远在数千或数万 
里之外,如果有人做了坏事,他的家人还未完全知道,他的乡人也未完全听 
到,天子就已知道并惩罚了他。所以所有天下的人,十分害怕和震动战栗, 
不敢做淫暴的事。说:“天子的视听如神。”先王说过这样的话:“不是神, 
只是能够使他人的耳目帮助自己视听;使他人的唇吻帮助自己言谈,使他人 
的心帮助自己思考,使他人的四肢帮助自己动作。”帮助他视听的人多,那 
么他的所见所闻就广大了;帮助他言谈的人多,那么他的声音所安抚范围就 
广阔了;帮助他思考的人多,那么计划很快就能实行了;帮助他动作的人多, 
那么他所做的事情很快就能成功了。所以古代的圣人能够把事情办成功、名 
垂后世,没有别的其他原因,只是能够以上同的原则来行使政事。 
所以先王的书《周颂》上曾说过:“始来见君王,寻求(车服礼仪等) 
文章制度。”这说的是古代的诸侯国君在每年的春秋二季,到天子的朝廷来 
朝聘,接受天子严厉的教令,然后回去治理他们的国家,因此政令所到之处, 
没有人敢不服。当这个时候,根本没有人敢变乱天子的教令,《诗经》上说: 
“我的马是黑色鬃毛的白马,六条马缰绳柔美光滑,在路上或快或慢地跑, 
在所到之处普遍地询访查问。”又说:“我的马是青黑色毛片的,六条马缰 
绳象丝一般光滑,在路上或快或慢地跑,在所到之处普遍地询问谋划。”说 
的就是这个意思。古代的国君诸侯听见或看到好与坏的事情,都跑去报告天 
子。所以赏的正好是贤人,罚的正好是暴人,不杀害无辜,也不放过有罪, 

这就是上同带来的功效。所以墨子说:“现在天下的王公大人士君子,如果 
真想使他们的国家富有,人民众多,刑政治理,国家安定,就不可不考察上 
同,因为这是为政的根本。” 


\chapter{十三  尚同下}

子墨子言曰:“知者之事,必计国家百姓所以治者而为之,必计国家百 
姓之所以乱者而辟之(1)。”然计国家百姓之所以治者,何也?上之为政,得 
下之情则治,不得下之情则乱。何以知其然也?上之为政,得下之情,则是 
明于民之善非也。若苟明于民之善非也,则得善人而赏之,得暴人而罚之也。 
善人赏而暴人罚,则国必治。上之为政也,不得下之情,则是不明于民之善 
非也,若苟不明于民之善非,则是不得善人而赏之,不得暴人而罚之。善人 
不赏而暴人不罚,为政若此,国众必乱。故赏不得下之情,而不可不察者也。 
然计得下之情,将奈何可?故子墨子曰:“唯能以尚同一义为政,然后 
可矣!”何以知尚同一义之可而为政于天下也?然胡不审稽古之治为政之说 
乎?古者天之始生民,未有正长也,百姓为人。若苟百姓为人,是一人一义, 
十人十义,百人百义,千人千义。逮至人之众,不可胜计也;则其所谓义者, 
亦不可胜计。此皆是其义,而非人之义,是以厚者有斗,而薄者有争。是故 
天下之欲同一天下之义也,是故选择贤者,立为天子。天子以其知力为未足 
独治天下,是以选择其次,立为三公。三公又以其知力为未足独左右天子也, 
是以分国建诸侯。诸侯又以其知力为未足独治其四境之内也,是以选择其次, 
立为卿之宰。卿之宰又以其知力为未足独左右其君也,是以选择其次,立而 
为乡长、家君。是故古者天子之立三公、诸侯、卿之宰、乡长、家君,非特 
富贵游佚而择之也(2),将使助治乱刑政也。故古者建国设都,乃立后王君公, 
奉以卿士师长,此非欲用说也(3),唯辩而使助治天明也。 
今此何为人上而不能治其下?为人下而不能事其上?则是上下相贼也。 
何故以然?则义不同也。若苟义不同者有党,上以若人为善,将赏之,若人 
唯使得上之赏而辟百姓之毁(4);是以为善者必未可使劝,见有赏也。上以若 
人为暴,将罚之,若人唯使得上之罚,而怀百姓之誉;是以为暴者必未可使 
沮,见有罚也。故计上之赏誉,不足以劝善,计其毁罚,不足以沮暴。此何 
故以然?则义不同也。 
然则欲同一天下之义,将奈何可?故子墨子言曰:然胡不赏使家君,试 
用家君发宪布令其家?曰:“若见爱利家者,必以告;若见恶贼家者,亦必 
以告。”若见爱利家以告,亦犹爱利家者也,上得且赏之,众闻则誉之;若 
见恶贼家不以告,亦犹恶贼家者也,上得且罚之,众闻则非之。是以遍若家 
之人,皆欲得其长上之赏誉,辟其毁罚。是以善言之,不善言之;家君得善 
人而赏之,得暴人而罚之。善人之赏,而暴人之罚,则家必治矣。然计若家 
之所以治者,何也?唯以尚同一义为政故也。 
家既已治,国之道尽此已邪?则未也。国之为家数也甚多,此皆是其家, 
而非人之家,是以厚者有乱,而薄者有争。故又使家君总其家之义,以尚同 
于国君,国君亦为发宪布令于国之众,曰:“若见爱利国者,必以告;若见 
恶贼国者,亦必以告。”若见爱利国以告者,亦犹爱利国者也,上得且赏之, 
众闻则誉之;若见恶贼国不以告者,亦犹恶贼国者也,上得且罚之,众闻则 
非之。是以遍若国之人,皆欲得其长上之赏誉,避其毁罚。是以民见善者言 
之,见不善者言之;国君得善人而赏之,得暴人而罚之。善人赏而暴人罚, 
则国必治矣。然计若国之所以治者何也?唯能以尚同一义为政故也。 
国既已治矣,天下之道尽此已邪?则未也。天下之为国数也甚多,此皆 
是其国,而非人之国,是以厚者有战,而薄者有争。故又使国君选其国之义, 

以尚同于天子。天子亦为发宪布令于天下之众,曰:“若见爱利天下者,必 
以告;若见恶贼天下者,亦以告。”若见爱利天下以告者,亦犹爱利天下者 
也,上得则赏之,众闻则誉之;若见恶贼天下不以告者,亦犹恶贼天下者也, 
上得且罚之,众闻则非之。是以遍天下之人,皆欲得其长上之赏誉,避其毁 
罚,是以见善、不善者告之。天子得善人而赏之,得暴人而罚之,善人赏而 
暴人罚,天下必治矣。然计天下之所以治者,何也?唯而以尚同一义为政故 
也(5)。 
天下既已治,天子又总天下之义,以尚同于天。故当尚同之为说也,尚 
用之天子,可以治天下矣;中用之诸侯,可而治其国矣;小用之家君,可而 
治其家矣。是故大用之治天下不窕(6),小用之治一国一家而不横者,若道之 
谓也。故曰治天下之国,若治一家;使天下之民,若使一夫。意独子墨子有 
此而先王无此?其有邪,则亦然也。圣王皆以尚同为政,故天下治。何以知 
其然也?于先王之书也《大誓》之言然,曰:“小人见奸巧,乃闻不言也, 
发罪钧。”此言见淫辟不以告者,其罪亦犹淫辟者也。 
故古之圣王治天下也,其所差论以自左右羽翼者皆良,外为之人,助之 
视听者众。故与人谋事,先人得之;与人举事,先人成之;光誉令闻,先人 
发之。唯信身而从事,故利若此。古者有语焉,曰:“一目之视也,不若二 
目之视也;一耳之听也,不若二耳之听也;一手之操也,不若二手之强也。” 
夫唯能信身而从事,故利若此。是故古之圣王之治天下也,千里之外,有贤 
人焉,其乡里之人皆未之均闻见也,圣王得而赏之。千里之内,有暴人焉, 
其乡里未之均闻见也,圣王得而罚之。故唯毋以圣王为聪耳明目与?岂能一 
视而通见千里之外哉?一听而通闻千里之外哉?圣王不往而视也,不就而听 
也,然而使天下之为寇乱盗贼者,周流天下无所重足者,何也?其以尚同为 
政善也。 
是故子墨子曰:“凡使民尚同者,爱民不疾,民无可使,曰:必疾爱而 
使之,致信而持之,富贵以道其前(7),明罚以率其后。为政若此,唯欲毋与 
我同,将不可得也。” 
是以子墨子曰:“今天下王公大人士君子,中情将欲为仁义,求为上士, 
上欲中圣王之道,下欲中国家百姓之利,故当尚同之说而不可不察。尚同, 
为政之本而治要也。” 


[注释] 

(1)“辟”:通“避”。(2)“择”为“怿”字之误。(3)“说”通“悦”。(4)“辟”上疑脱“不” 

字。(5)“而”通“能”。(6)窕:不满。(7)“道”通“导”。 
[白话] 
墨子说道:“智者做事,必须考虑国家百姓所以治理的原因而行事,也 
必须考虑国家百姓所以混乱的根源而事先回避。”然而考虑国家百姓因之治 
理的原因是什么呢?居上位的人施政,能得到下面的实情则治理,不能得到 
下面的实情则混乱。怎么知道是这样呢?居上位的施政,得到了下边实情, 
这就对百姓的善否很清楚。假若清楚百姓的善否,那么得到善人就奖赏他, 
得到暴人就惩罚他。善人受赏而暴人受罚,那么国家就必然治理。如果居上 
位的施政,不能得知下面的实情,这就是对百姓的善否不清楚。假若不清楚 
百姓的善否,这就不能得到善人而赏赐他,不能得到暴人而惩罚他。善人得 
不到赏赐而暴人得不到惩罚,象这样施政,国家民众就必定混乱。所以赏(罚) 

若得不到下面的实情,是不可不考察其后果的。 
然而考虑应该怎么样才可以获知下情呢?所以墨子说:“只有能用向上 
统一意见施政,这以后就可以了。”怎么知道向上统一意见,就可以在天下 
施政呢?这为什么不审察古代施政时的情况呢?古代上天开始生育下民,还 
没有行政长官的时候,百姓人各为主。如果百姓人各为主,这就一人有一个 
道理,十人有十个道理,百人有百个道理,千人有千人道理。及至人数多得 
不可胜数,那么他们所谓的道理也就多得不可胜数。这样人都认为自己的道 
理正确,而认为别人的道理不正确,因此严重的发生斗殴,轻微的发生争吵。 
所以上天希望统一天下的道理,因此就选择贤人立为天子。天子认为他的智 
慧能力不足单独治理天下,所以选择次于他的贤人立为三公。三公又认为自 
己的智慧能力不足单独辅佐天子,所以分封建立诸侯;诸侯又认为自己的智 
慧能力不足单独治理他国家的四境之内,因此又选择次于他的贤人,立为卿 
与宰;卿、宰又认为自己的智慧能力不足以单独辅佐他的君主,因此选择次 
于他的贤人,立为乡长、家君。所以古时天子设立三公、诸侯、卿、宰、乡 
长,家君,不只是让他们富贵游乐而选择他们,而是将使他们协助自己治理 
刑政。所以古时建国立都,就设立了帝王君主,又辅佐以卿士师长,这不是 
想用来取悦自己喜欢的人,只是分授职责,使他们助天明治。 
现在为什么居人之上的人不能治理他的下属,居人之下的人不能事奉他 
的上级?这就是上下相互残害。什么原因会这样?就是各人的道理不同。假 
若道理不同的人双方有所偏私,上面认为这人为善,将赏赐他。这人虽然得 
到了上面的赏赐,却免不了百姓的非议,因此,为善的人未必因此而得到勉 
励,虽然人们看到有赏赐。上面认为这人行暴,将惩罚他,此人虽得到了上 
司的惩罚,却怀有百姓的赞誉,因此,行暴的人未必可使停止,虽然人们看 
到了惩罚。所以计议上面的赏赐赞誉,不足以勉励向善,计议上面的非毁惩 
罚,不足以阻止暴行。这是什么原故使之如此呢?就是各人道理不同。 
既然如此,那么想统一天下各人的道理,将怎么办呢?所以墨子说道: 
为何不试着使家君对他的下属发布政令说:“你们见到爱护和有利于家族的, 
必须把它报告给我,你们见到憎恨和危害家族的也必须把它报告给我。你们 
见到爱护和有利于家族的报告给我,也和爱护和有利家族一样,上面得知了 
将赏赐他,大家听到了将赞誉他。你们见到了憎害家族不拿来报告,也和憎 
害家族的一样,上面得知了将惩罚他,大家听到了将非议他。”以此遍告这 
全家的人。人们都希望得到长上的赏赐赞誉,而避免非议惩罚。所以,见了 
好的来报告,见了不好的也来报告。家君得到善人而赏赐他,得到暴人而惩 
罚他。善人得赏而暴人得罚,那么家族就会治理好。然而计议这一家治理得 
好的原因是什么呢?只是能以向上统一道理的原则治政之故。 
家已经治好了,治国的办法全都在此了吗?那还没有。国家之中的家数 
很多,它们都认为自己的家对而别人的家不对,所以严重的就发生动乱,轻 
微的就发生争执。所以又使家君总其家族的道理,用以上同于国君。国君也 
对国中民众发布政令说:“你们看到爱护和有利于国家的必定拿它来报告, 
你们看到憎恶和残害国家的也必定拿它来报告。你们看到爱护和有利于国家 
的把它上报了,也和爱护和有利国家的一样。上面得悉了将予以赏赐,大家 
听到了将予以赞誉。你们看到了憎恶和残害国家的不拿来上报,也和憎恶和 
残害国家的一样。上面得悉了将予以惩罚,大家听到了将予以非议。”以此 
遍告这一国的人。人们都希望得到长上的赏赐赞誉,避免他的非议惩罚,所 

以人民见到好的来报告,见到不好的也来报告。国君得到善人予以赏赐,得 
到暴人而予以惩罚。善人得赏而暴人得罚,那么国家必然治理好。然而计议 
这一国治理好的原因是什么呢?只是能以向上统一道理的原则治政之故。 
国家已经治理了,治理天下的办法尽在这里了吗?那还没有。天下国家 
为数很多,这些国家都认为自己的国家对而别人的国家不对,所以严重的就 
发生动乱,轻微的就发生争执。因此又使国君总同各国的意见,用来上同于 
天子。天子也对天下民众发布政令说:“你们看到爱护和有利于天下的必定 
拿它来报告,你们看到憎恶和残害天下的也必定拿它来报告。你们看到爱护 
和有利于天下而拿来报告的,也和爱护和有利于天下的一样。上面得悉了将 
予以赏赐,大家听到了将予以赞誉。你们看到了憎恶和残害天下的而不拿来 
上报的,也和憎恶和残害天下的一样。上面得悉了将予以惩罚,大家听到了 
将予以非毁。”以此遍告天下的人。人们都希望得到长上的赏赐赞誉,避免 
他的非毁惩罚,所以看到好的来报告,看到不好的也来报告。天子得到善人 
予以赏赐,得到暴人而予以惩罚。天下必定治理了。然而计议天下治理好的 
原因是什么呢?只是能以向上统一道理的原则治政的缘故。 
天下已经治理了,天子又总同天下的道理,用来上同于天。所以尚同作 
为一种主张,它上而用之于天子,可以用来治理天下;中而用之于诸侯,可 
以用来治理他的国家;小而用之于家长,可以用来治理他的家族。所以大用 
之治理天下不会不足,小用之治理一国一家而不会横阻,说的就是(尚同) 
这个道理。所以说:治理天下之国,如治一家,使今天下之民如使一人。抑 
或只有墨子有这个主张,而先王没有这个呢?则先王也是这样的。圣王都用 
尚同的原则治政,所以天下治理。从何知道这样呢?在先王的书《大誓》这 
样说过:“小人看到奸巧之事,知而不言的,他的罪行与奸巧者均等。”这 
说的就是看到淫僻之事不拿来报告的,他的罪行也和淫僻者的一样。 
所以古时的圣王治理天下,他所选择作为自己左右辅佐的人,都是贤良。 
在外边做事的人,帮助他察看和听闻的人很多。所以(他)和大家一起谋划 
事情,要比别人先考虑周到;和大家一起办事,要比别人先成功,(他的) 
荣誉和美好的名声要比别人先传扬出去。唯其以诚信从事,所以有这样多的 
利益。古时有这样的话,说:“一只眼睛所看到的,不如两只眼睛所看到的; 
一只耳朵听到的,不如两只耳朵听到的;一只手操拿,不如两只手强。”惟 
其以诚信从事,所以如此有利。所以古代圣王治理天下,千里之外的地方有 
个贤人,那一乡里的人还未全都听到或见到,圣王已经得悉而予以赏赐了。 
千里之外的地方有一个暴人,那一乡里的人还未全部听到或见到,圣王已经 
得悉而予以惩罚了。所以认为圣王是耳聪目明吧?难道张眼一望就到达千里 
之外吗?倾耳一听就到达千里之外吗?圣王不会亲自前去看,不会靠近去 
听。然而使天下从事寇乱盗贼的人走遍天下无处容足的原因,是什么呢?那 
是以尚同原则治政的好处。 
所以墨子说:“凡是使百姓尚同的,如果爱民不深,百姓就不可使令。 
即是说:必须切实爱护他们,以诚信之心拥有他们。用富贵引导于前,用严 
明的惩罚督率于后。象这样施政,即使要想人民不与我一致,也将办不到。” 
所以墨子说:“现在天下的王公大人、士君子们,如果心中确实将行仁 
义,追求做上士,上要符合圣王之道,下要符合国家百姓之利,因此对尚同 
这一主张不可不予以审察。尚同是施政的根本和统治的关键。” 


\chapter{十四  兼爱上}

圣人以治天下为事者也,必知乱之所自起,焉能治之;不知乱之所自起, 
则不能治。譬之如医之攻人之疾者然:必知疾之所自起,焉能攻之;不知疾 
之所自起,则弗能攻。治乱者何独不然?必知乱之所自起,焉能治之;不知 
乱之所自起,则弗能治。圣人以治天下为事者也,不可不察乱之所自起。 
当察乱何自起(2)?起不相爱。臣子之不孝君父,所谓乱也。子自爱,不 
爱父,故亏父而自利;弟自爱,不爱兄,故亏兄而自利;臣自爱,不爱君, 
故亏君而自利,此所谓乱也。虽父之不慈子,兄之不慈弟,君之不慈臣,此 
亦天下之所谓乱也。父自爱也,不爱子,故亏子而自利;兄自爱也,不爱弟, 
故亏弟而自利;君自爱也,不爱臣,故亏臣而自利。是何也?皆起不相爱。 
虽至天下之为盗贼者亦然:盗爱其室,不爱其异室,故窃异室以利其室。 
贼爱其身,不爱人,故贼人以利其身。此何也?皆起不相爱。虽至大夫之相 
乱家,诸侯之相攻国者亦然:大夫各爱其家,不爱异家,故乱异家以利其家。 
诸侯各爱其国,不爱异国,故攻异国以利其国。天下之乱物,具此而已矣。 
察此何自起?皆起不相爱。 
若使天下兼相爱,爱人若爱其身,犹有不孝者乎?视父兄与君若其身, 
恶施不孝(3)?犹有不慈者乎?视弟子与臣若其身,恶施不慈?故不孝不慈亡 
有(4)。犹有盗贼乎?故视人之室若其室,谁窃?视人身若其身,谁贼?故盗 
贼亡有。犹有大夫之相乱家,诸侯之相攻国者乎?视人家若其家,谁乱?视 
人国若其国,谁攻?故大夫之相乱家,诸侯之相攻国者亡有。若使天下兼相 
爱,国与国不相攻,家与家不相乱,盗贼无有,君臣父子皆能孝慈,若此, 
则天下治。 
故圣人以治天下为事者,恶得不禁恶而劝爱。故天下兼相爱则治,交相 
恶则乱。故子墨子曰:“不可以不劝爱人者,此也。” 


[注释] 

(1)兼爱是墨家学派最有代表性的理论之一。所谓兼爱,其本质是要求人们爱人如己,彼此之间 

不要存在血缘与等级差别的观念。墨子认为,不相爱是当时社会混乱最大的原因,只有通过“兼相爱, 

交相利”才能达到社会安定的状态。这种理论具有反抗贵族等级观念的进步意义,但同时也带有强烈 

的理想色彩。(2)当:读为“尝”。(3)恶(wū):何。(4)亡:通“无”。 
[白话] 
圣人是以治理天下为职业的人,必须知道混乱从哪里产生,才能对它进 
行治理。如果不知道混乱从哪里产生,就不能进行治理。这就好像医生给人 
治病一样,必须知道疾病产生的根源,才能进行医治。如果不知道疾病产生 
的根源,就不能医治。治理混乱又何尝不是这样,必须知道混乱产生的根源, 
才能进行治理。如果不知道混乱产生的根源,就不能治理。圣人是以治理天 
下为职业的人,不可不考察混乱产生的根源。 
试考察混乱从哪里产生呢?起于人与人不相爱。臣与子不孝敬君和父, 
就是所谓乱。儿子爱自己而不爱父亲,因而损害父亲以自利;弟弟爱自己而 
不爱兄长,因而损害兄长以自利;臣下爱自己而不爱君上,因而损害君上以 
自利,这就是所谓混乱。反过来,即使父亲不慈爱儿子,兄长不慈爱弟弟, 
君上不慈爱臣下,这也是天下的所谓混乱。父亲爱自己而不爱儿子,所以损 
害儿子以自利;兄长爱自己而不爱弟弟,所以损害弟弟以自利;君上爱自己 

而不爱臣下,所以损害臣下以自利。这是为什么呢?都是起于不相爱。 
即使在天底下做盗贼的人,也是这样。盗贼只爱自己的家,不爱别人的 
家,所以盗窃别人的家以利自己的家;盗贼只爱自身,不爱别人,所以残害 
别人以利自己。这是什么原因呢?都起于不相爱。 
即使大夫相互侵扰家族,诸侯相互攻伐封国,也是这样。大夫各自爱他 
自己的家族,不爱别人的家族,所以侵扰别人的家族以利他自己的家族;诸 
侯各自爱他自己的国家,不爱别人的国家,所以攻伐别人的国家以利他自己 
的国家。天下的乱事,全部都具备在这里了。细察它从哪里产生呢?都起于 
不相爱。 
假若天下都能相亲相爱,爱别人就象爱自己,还能有不孝的吗?看待父 
亲、兄弟和君上象自己一样,怎么会做出不孝的事呢?还会有不慈爱的吗? 
看待弟弟、儿子与臣下象自己一样,怎么会做出不慈的事呢?所以不孝不慈 
都没有了。还有盗贼吗?看待别人的家象自己的家一样,谁会盗窃?看待别 
人就象自己一样,谁会害人?所以盗贼没有了。还有大夫相互侵扰家族,诸 
侯相互攻伐封国吗?看待别人的家族就象自己的家族,谁会侵犯?看待别人 
的封国就象自己的封国,谁会攻伐?所以大夫相互侵扰家族,诸侯相互攻伐 
封国,都没有了。假若天下的人都相亲相爱,国家与国家不相互攻伐,家族 
与家族不相互侵扰,盗贼没有了,君臣父子间都能孝敬慈爱,象这样,天下 
也就治理了。 
所以圣人既然是以治理天下为职业的人,怎么能不禁止相互仇恨而鼓励 
相爱呢?因此天下的人相亲相爱就会治理好,相互憎恶则会混乱。所以墨子 
说:“不能不鼓励爱别人”,道理就在此。 

\chapter{十五  兼爱中}

 子墨子言曰:“仁人之所以为事者,必兴天下之利,除去天下之害,以 
此为事者也。”然则天下之利何也?天下之害何也?子墨子言曰:“今若国 
之与国之相攻,家之与家之相篡,人之与人之相贼,君臣不惠忠,父子不慈 
孝,兄弟不和调,此则天下之害也。” 
然则崇此害亦何用生哉(1)?以不相爱生邪?子墨子言:“以不相爱生。” 
今诸侯独知爱其国,不爱人之国,是以不惮举其国,以攻人之国。今家主独 
知爱其家,而不爱人之家,是以不惮举其家,以篡人之家。今人独知爱其身, 
不爱人之身,是以不惮举其身,以贼人之身。是故诸侯不相爱,则必野战; 
家主不相爱,则必相篡;人与人不相爱,则必相贼;君臣不相爱,则不惠忠; 
父子不相爱,则不慈孝;兄弟不相爱,则不和调。天下之人皆不相爱,强必 
执弱,富必侮贫,贵必敖贱(2),诈必欺愚。凡天下祸篡怨恨,其所以起者, 
以不相爱生也。是以行者非之(3)。 
既以非之,何以易之?子墨子言曰:“以兼相爱、交相利之法易之。” 
然则兼相爱、交相利之法将奈何哉?子墨子言:视人之国,若视其国;视人 
之家,若视其家;视人之身,若视其身。是故诸侯相爱,则不野战;家主相 
爱,则不相篡;人与人相爱,则不相贼;君臣相爱,则惠忠;父子相爱,则 
慈孝;兄弟相爱,则和调。天下之人皆相爱,强不执弱,众不劫寡,富不侮 
贫,贵不敖贱,诈不欺愚。凡天下祸篡怨恨,可使毋起者,以相爱生也。是 
以仁者誉之。 
然而今天下之士君子曰:“然!乃若兼则善矣;虽然,天下之难物于故 
也(4)。”子墨子言曰:“天下之士君子,特不识其利、辩其故也。今若夫攻 
城野战,杀身为名,此天下百姓之所皆难也。若君说之(5),则士众能为之。 
况于兼相爱、交相利,则与此异!夫爱人者,人必从而爱之;利人者,人必 
从而利之;恶人者,人必从而恶之;害人者,人必从而害之。此何难之有? 
特上弗以为政、士不以为行故也。”昔者晋文公好士之恶衣,故文公之臣, 
皆牂羊之裘(6),韦以带剑(7),练帛之冠,入以见于君,出以践于朝。是其 
故何也?君说之,故臣为之也。昔者楚灵王好士细要(8),故灵王之臣,皆以 
一饭为节,胁息然后带,扶墙然后起。比期年,朝有黧黑之色。是其故何也? 
君说之,故臣能之也。昔越王句践好士之勇,教驯其臣,和合之,焚舟失火, 
试其士曰:“越国之宝尽在此!”越王亲自鼓其士而进之,士闻鼓音,破碎 
乱行(9),蹈火而死者,左右百人有余,越王击金而退之。是故子墨子言曰: 
“乃若夫少食、恶衣、杀人而为名,此天下百姓之所皆难也。若苟君说之, 
则众能为之;况兼相爱、交相利,与此异矣!夫爱人者,人亦从而爱之;利 
人者,人亦从而利之;恶人者,人亦从而恶之;害人者,人亦从而害之。此 
何难之有焉?特士不以为政而士不以为行故也(10)。 
然而今天下之士君子曰:“然!乃若兼则善矣;虽然,不可行之物也。 
譬若挈太山越河、济也。”子墨子言:“是非其譬也。夫挈太山而越河、济, 
可谓毕劫有力矣。自古及今,未有能行之者也;况乎兼相爱、交相利,则与 
此异,古者圣王行之。”何以知其然?古者禹治天下,西为西河渔窦,以泄 
渠、孙、皇之水。北为防、原、派,注后之邸(11)、嘑池之窦,洒为底柱(12), 
凿为龙门,以利燕代胡貉与西河之民。东方漏之陆(13),防孟诸之泽,洒为 
九浍,以楗东土之水,以利冀州之民。南为江、汉、淮、汝,东流之注五湖 

之处,以利荆楚、干、越与南夷之民。此言禹之事,吾今行兼矣。昔者文王 
之治西土,若日若月,乍光于四方,于西土。不为大国侮小国,不为众庶侮 
鳏寡,不为暴势夺穑人黍稷狗彘。天屑临文王慈,是以老而无子者,有所得 
终其寿;连独无兄弟者(14),有所杂于生人之间;少失其父母者,有所放依 
而长。此文王之事,则吾今行兼矣。昔者武王将事泰山,隧传曰(15):“泰 
山,有道曾孙周王有事。大事既获,仁人尚作,以祗商(16)、夏、蛮夷丑貉。 
虽有周亲,不若仁人。万方有罪,维予一人。”此言武王之事,吾今行兼矣。 
是故子墨子言曰:“今天下之君子,忠实欲天下之富,而恶其贫;欲天 
下之治,而恶其乱,当兼相爱、交相利。此圣王之法,天下之治道也,不可 
不务为也。” 


[注释] 

(1)“崇”为“察”字之误。(2)“敖”通“傲”。(3)“行”为“仁”字之误。(4)“于”为“迂” 

之假借字。(5)“说”通“悦”。(6)牂羊:母羊。(7)韦:熟牛皮。(8)细要:细腰。(9)“碎”疑为“阵” 

字之误。(10)“士”为“上”之误。(11)“后”为“召”之误。(12)“底”为“厎”之误。(13)“之” 

为“大”之误。(14)“连”为“矜”之假借字。(15)“隧”疑为“遂”字之误。(16)祗:拯救。 
[白话] 
墨子说:“仁人处理事务的原则,一定是为天下兴利除害,以此原则来 
处理事务。”既然如此,那么天下的利是什么,而天下的害又是什么呢?墨 
子说:“现在如国与国之间相互攻伐,家族与家族之间相互掠夺,人与人之 
间相互残害,君臣之间不相互施惠、效忠,父子之间不相互慈爱、孝敬,兄 
弟之间不相互融洽、协调,这就都是天下之害。” 
既然如此,那么考察这些公害又是因何产生的呢?是因不相爱产生的 
吗?墨子说:“是因不相爱产生的。”现在的诸侯只知道爱自己的国家,不 
爱别人的国家,所以毫无忌惮地发动他自己国家的力量,去攻伐别人的国家。 
现在的家族宗主只知道爱自己的家族,而不爱别人的家族,因而毫无忌惮地 
发动他自己家族的力量,去掠夺别人的家族。现在的人只知道爱自己,而不 
爱别人,因而毫无忌惮地运用全身的力量去残害别人。所以诸侯不相爱,就 
必然发生野战;家族宗主不相爱,就必然相互掠夺;人与人不相爱,就必然 
相互残害;君与臣不相爱,就必然不相互施惠、效忠;父与子不相爱,就必 
然不相互慈爱、孝敬;兄与弟不相爱,就必然不相互融洽、协调。天下的人 
都不相爱,强大的就必然控制弱小的,富足的就必然欺侮贫困的,尊贵的就 
必然傲视卑贱的,狡猾的就必然欺骗愚笨的。举凡天下祸患、掠夺、埋怨、 
愤恨产生的原因,都是因不相爱而产生的。所以仁者认为它不对。 
既已认为不相爱不对,那用什么去改变它呢?墨子说道:“用人们全都 
相爱、交互得利的方法去改变它。”既然这样,那么人们全都相爱、交互得 
利应该怎样做呢?墨子说道:“看待别人国家就象自己的国家,看待别人的 
家族就象自己的家族,看待别人之身就象自己之身。”所以诸侯之间相爱, 
就不会发生野战;家族宗主之间相爱,就不会发生掠夺;人与人之间相爱就 
不会相互残害;君臣之间相爱,就会相互施惠、效忠;父子之间相爱,就会 
相互慈爱、孝敬;兄弟之间相爱,就会相互融洽、协调。天下的人都相爱, 
强大者就不会控制弱小者,人多者就不会强迫人少者,富足者就不会欺侮贫 
困者,尊贵者就不会傲视卑贱者,狡诈者就不会欺骗愚笨者。举凡天下的祸 
患、掠夺、埋怨、愤恨可以不使它产生的原因,是因为相爱而生产的。所以 

仁者称赞它。 
然而现在天下的士君子们说:“对!兼爱固然是好的。即使如此,它也 
是天下一件难办而迂阔的事。”墨子说道:“天下的士君子们,只是不能辨 
明兼爱的益处、辨明兼爱的原故。现在例如攻城野战,为成名而杀身,这都 
是天下的百姓难于做到的事。但假如君主喜欢,那么士众就能做到。而兼相 
爱、交相利与之相比,则是完全不同的(好事)。凡是爱别人的人,别人也 
随即爱他;有利于别人的人,别人也随即有利于他;憎恶别人的人,别人也 
随即憎恶他;损害别人的人,别人随即损害他。实行这种兼爱有什么困难呢? 
只是由于居上位的人不用它行之于政,士人不用它实之于行的缘故。”从前 
晋文公喜欢士人穿不好的衣服,所以文公的臣下都穿着母羊皮缝的裘,围着 
牛皮带来挂佩剑,头戴熟绢作的帽子,(这身打扮)进可以参见君上,出可 
以往来朝廷。这是什么缘故呢?因为君主喜欢这样,所以臣下就这样做。从 
前楚灵王喜欢细腰之人,所以灵王的臣下就吃一顿饭来节食,收着气然后才 
系上腰带,扶着墙然后才站得起来。等到一年,朝廷之臣都(饥瘦得)面有 
深黑之色。这是什么缘故呢?因为君主喜欢这样,所以臣下能做到这样。从 
前越王句践喜爱士兵勇猛,训练他的臣下时,先把他们集合起来,(然后) 
放火烧船,考验他的将士说:“越国的财宝全在这船里。”越王亲自擂鼓, 
让将士前进。将士听到鼓声,(争先恐后),打乱了队伍,蹈火而死的人, 
近臣达一百人有余。越王于是鸣金让他们退下。所以墨子说道:“象少吃饭、 
穿坏衣、杀身成名,这都是天下百姓难于做到的事。假如君主喜欢它,那么 
士众就能做到。何况兼相爱、交相利是与此不同的(好事)。爱别人的人, 
别人也随即爱他;有利于别人的人,别人也随即有利于他;憎恶别人的人, 
别人也随即憎恶他;损害别人的人,别人也随即损害他。这种兼爱有什么难 
实行的呢?只是居上位的人不用它行之于政,而士人不用它实之于行的缘 
故。” 
然而现在天下的士君子们说:“对!兼爱固然是好的。即使如此,也不 
可能行之于事,就象要举起泰山越过黄河、济水一样。”墨子说道:“这比 
方不对。举起泰山而越过黄河、济水,可以说是强劲有力的了,但自古及今, 
没有人能做得到。而兼相爱,交相利与此相比则是完全不同的(可行之事)。 
古时的圣王曾做到过。”怎么知道是这样呢?古时大禹治理天下,西边疏通 
了西河、渔窦,用来排泄渠水、孙水和皇水;北边疏通防水、原水、泒水, 
使之注入召之邸和滹沱河,在黄河中的厎柱山分流,凿开龙门以有利于燕、 
代、胡、貉与西河地区的人民。东边穿泄大陆的迂水,拦入孟诸泽,分为九 
条河,以此限制东土的洪水,用来利于冀州的人民。南边疏通长江、汉水、 
淮河、汝水,使之东流入海,以此灌注五湖之地,以利于荆楚、吴越和南夷 
的人民。这是大禹的事迹,我们现在要用这种精神来实行兼爱。从前周文王 
治理西土(指岐周),象太阳象月亮一样,射出的光辉照耀四方和西周大地。 
他不倚仗大国而欺侮小国,不倚仗人多而欺侮鳏寡孤独,不倚仗强暴势力而 
掠夺农夫的粮食牲畜。上天眷顾文王的慈爱,所以年老无子的人得以寿终, 
孤苦无兄弟的人可以安聚于人们中间,幼小无父母的人有所依靠而长大成 
人。这是文王的事迹,我们现在应当用这种精神实行兼爱。从前武王将祭祀 
泰山,于是陈述说:“泰山!有道曾孙周王有祭事。现在(伐纣的)大事已 
成功,(太公、周、召)一批仁人起而相助,用以拯救商夏遗民及四方少数 
民族。即使是至亲,也不如仁人。万方之人有罪,由我一人承当。”这是说 

周武王的事迹,我们现在应当用这种精神实行兼爱。 
所以墨子说道:“现在天下的君子,(如果)内心确实希望天下富足, 
而厌恶其贫穷;希望天下治理好,而厌恶其混乱,那就应当全都相爱、交互 
得利。这是圣王的常法,天下的治道,不可不努力去做。” 

\chapter{十六  兼爱下}
\chapter{十七  非攻(1)上}
\chapter{十八  非攻中}
\chapter{十九  非攻下}
\chapter{二十  节用(1)上}
\chapter{二十一  节用中}
\chapter{二十二  节葬(1)下}
\chapter{二十三  天志(1)上}
\chapter{二十四  天志中}
\chapter{二十五  天志下}

子墨子言曰:“天下之所以乱者,其说将何哉?则是天下士君子,皆明 
于小而不明于大。”何以知其明于小不明于大也?以其不明于天之意也。何 
以知其不明于天之意也?以处人之家者知之。今人处若家得罪,将犹有异家 
所以避逃之者;然且父以戒子,兄以戒弟,曰:“戒之!慎之!处人之家, 
不戒不慎之,而有处人之国者乎?”今人处若国得罪,将犹有异国所以避逃 
之者矣;然且父以戒子,兄以戒弟,曰:“戒之!慎之!处人之国者,不可 
不戒慎也。”今人皆处天下而事天,得罪于天,将无所以避逃之者矣;然而 
莫知以相极戒也(1)。吾以此知大物则不知者也。 
是故子墨子言曰:“戒之慎之,必为天之所欲,而去天之所恶。”曰天 
之所欲者,何也?所恶者,何也?天欲义而恶其不义者也。何以知其然也? 
曰:义者,正也。何以知义之为正也?天下有义则治,无义则乱,我以此知 
义之为正也。然而正者,无自下正上者,必自上正下。是故庶人不得次己而 
为正(2),有士正之;士不得次己而为正,有大夫正之;大夫不得次己而为正, 
有诸侯正之;诸侯不得次己而为正,有三公正之;三公不得次己而为正,有 
天子正之;天子不得次己而为政,有天正之。今天下之士君子,皆明于天子 
之正天下也,而不明于天之正天子也。是故古者圣人明以此说人,曰:“天 
子有善,天能赏之;天子有过,天能罚之。”天子赏罚不当,听狱不中,天 
下疾病祸福,霜露不时,天子必且■豢其牛羊犬彘,洁为粢盛酒醴,以祷祠 
祈福于天,我未尝闻天之祷祈福于天子也。吾以此知天之重且贵于天子也。 
是故义者,不自愚且贱者出,必自贵且知者出。曰:谁为知?天为知。然则 
义果自天出也。今天下之士君子之欲为义者,则不可不顺天之意矣! 
曰:顺天之意何若?曰:兼爱天下之人。何以知兼爱天下之人也?以兼 
而食之也。何以知其兼而食之也?自古及今,无有远灵孤夷之国(3),皆■豢 
其牛羊犬彘,洁为粢盛酒醴,以敬祭祀上帝、山川、鬼神,以此知兼而食之 
也。苟兼而食焉,必兼而爱之。譬之若楚、越之君:今是楚王食于楚之四境 
之内,故爱楚之人;越王食于越,故爱越之人。今天兼天下而食焉,我以此 
知其兼爱天下之人也。 
且天之爱百姓也,不尽物而止矣。今天下之国,粒食之民,杀一不辜者, 
必有一不详。曰:“谁杀不辜?”曰:“人也。”“孰予之不辜?”曰:“天 
也。”若天之中实不爱此民也,何故而人有杀不辜、而天予之不祥哉?且天 
之爱百姓厚矣,天之爱百姓别矣,既可得而知也。何以知天之爱百姓也?吾 
以贤者之必赏善罚暴也。何以知贤者之必赏善罚暴也?吾以昔者三代之圣王 
知之。故昔也三代之圣王,尧、舜、禹、汤、文、武之兼爱之天下也。从而 
利之,移其百姓之意焉,率以敬上帝、山川、鬼神。天以为从其所爱而爱之, 
从其所利而利之,于是加其赏焉,使之处上位,立为天子以法也,名之曰圣 
人。以此知其赏善之证。是故昔也三代之暴王,桀、纣、幽、厉之兼恶天下 
也,从而贼之,移其百姓之意焉,率以诟侮上帝、山川、鬼神。天以为不从 
其所爱而恶之,不从其所利而贼之,于是加其罚焉。使之父子离散,国家灭 
亡,抎失社稷(4),忧以及其身。是以天下之庶民,属而毁之。业万世子孙继 
嗣,毁之贲,不之废也,名之曰失王。以此知其罚暴之证。今天下之士君子 
欲为义者,则不可不顺天之意矣。 
曰:顺天之意者,兼也;反天之意者,别也。兼之为道也,义正;别之 

为道也,力正。曰:“义正者,何若?”曰:大不攻小也,强不侮弱也,众 
不贼寡也,诈不欺愚也,贵不傲贱也,富不骄贫也,壮不夺老也。是以天下 
之庶国,莫以水火、毒药、兵刃以相害也。若事上利天,中利鬼,下利人, 
三利而无所不利,是谓天德。故凡从事此者,圣知也,仁义也,忠惠也,慈 
孝也,是故聚敛天下之善名而加之。是其故何也?则顺天之意也。曰:“力 
正者,何若?”曰:大则攻小也,强则侮弱也,众则贼寡也,诈则欺愚也, 
贵则傲贱也,富则骄贫也,壮则夺老也。是以天下之庶国,方以水火、毒药、 
兵刃以相贼害也。若事上不利天,中不利鬼,下不利人,三不利而无所利, 
是谓之贼。故凡从事此者,寇乱也,盗贼也,不仁不义,不忠不惠,不慈不 
孝,是故聚敛天下之恶名而加之。是其故何也?则反天之意也。 
故子墨子置立天之以为仪法,若轮人之有规,匠人之有矩也。今轮人以 
规,匠人以矩,以此知方圜之别矣。是故子墨子置立天之,以为仪法,吾以 
此知天下之士君子之去义,远也!何以知天下之士君子之去义远也?今知氏 
大国之君宽者然曰:“吾处大国而不攻小国,吾何以为大哉?”是以差论蚤 
牙之士,比列其舟车之卒,以攻罚无罪之国,入其沟境,刈其禾稼,斩其树 
木,残其城郭,以御其沟池,焚烧其祖庙,攘杀其牺牷。民之格者,则刭拔 
之(5),不格者,则系操而归(6),丈夫以为仆圉、胥靡,妇人以为舂酋。则 
夫好攻伐之君,不知此为不仁义,以告四邻诸侯曰:“吾攻国覆军,杀将若 
干人矣。”其邻国之君,亦不知此为不仁义也,有具其皮币,发其綛处(7), 
使人飨贺焉。则夫好攻伐之君,有重不知此为不仁不义也,有书之竹帛,藏 
之府库,为人后子者,必且欲顺其先君之行,曰:“何不当发吾府库,视吾 
先君之法美?”必不曰“文、武之为正者,若此矣”,曰“吾攻国覆军,杀 
将若干人矣。”则夫好攻伐之君,不知此为不仁不义也。其邻国之君,不知 
此为不仁不义也。是以攻伐世世而不已者。此吾所谓大物则不知也。 
所谓小物则知之者,何若?今有人于此,入人之场园,取人之桃李瓜姜 
者,上得且罚之,众闻则非之。是何也?曰:不与其劳,获其实,已非其有 
所取之故。而况有逾于人之墙垣,抯格人之子女者乎!与角人之府库,窃人 
之金玉蚤累者乎(8)!与逾人之栏牢,窃人之牛马者乎!而况有杀一不辜人乎! 
今王公大人之为政也,自杀一不辜人者,逾人之墙垣,抯格人之子女者,与 
角人之府库,窃人之金玉蚤累者,与逾人之栏牢,窃人之牛马者,与入人之 
场园,窃人之桃李瓜姜者,今王公大人之加罚此也;虽古之尧、舜、禹、汤、 
文、武之为政,亦无以异此矣。今天下之诸侯,将犹皆侵凌攻伐兼并(9),此 
为杀一不辜人者,数千万矣!此为逾人之墙垣,格人之子女者,与角人府库, 
窃人金玉蚤累者,数千万矣!逾人之栏牢,窃人之牛马者,与入人之场园, 
窃人之桃李瓜姜者,数千万矣!而自曰:“义也!” 
故子墨子言曰:“是蕡我者(10),则岂有以异是蕡黑白、甘苦之辩者哉! 
今有人于此,少而示之黑,谓之黑;多示之黑,谓白。必曰:‘吾目乱,不 
知黑白之别。’今有人于此,能少尝之甘,谓甘;多尝,谓苦。必曰:‘吾 
口乱,不知其甘苦之味。’今王公大人之政也,或杀人,其国家禁之。此蚤 
越有能多杀其邻国之人(11),因以为文义。此岂有异蕡黑白、甘苦之别者哉!” 
故子墨子置天之以为仪法。非独子墨子以天之志为法也,于先王之书《大 
夏》之道之然:“帝谓文王,予怀明德,毋大声以色,毋长夏以革,不识不 
知,顺帝之则。”此诰文王之以天志为法也(12),而顺帝之则也。且今天下 
之士君子,中实将欲为仁义,求为上士,上欲中圣王之道,下欲中国家百姓 

之利者,当天之志而不可不察也。天之志者,义之经也。 


[注释] 

(1)“极”即“儆”,“敬”,通“警”。(2)“次”即“恣”,下同。(3)“远灵孤夷”应为“远 

夷蘦孤”,“蘦”通“零”。(4)抎:坠落。(5)“拔”为“杀”字之误。(6)“操”为“累”之误。(7) 

“綛”为“总”之误。(8)“蚤”为“布”字之误。(9)“凌”通“陵”。(10)“蕡”,“紊”之假借 

字。“我”为“义”字之误。(11)“蚤越”当为“斧钺”。(12)“诰”为“语”字之误。 
[白话] 
墨子说道:“天下混乱的原因,其原因是什么呢?就是天下的士君子, 
都只明白小道理而不明白大道理。”从何知道他们只明白小道理而不明白大 
道理呢?从他们不明白天意就可知道。从何知道他们不明白天意呢?从他们 
处身家族的情况可以知道。假如现在(有人)在家族中得了罪,他还有别的 
家族可以逃避,然而父亲以此告诫儿子,兄长以此告诫弟弟,说:“警戒呀! 
谨慎呀!处身家族中不警戒不谨慎,还能处身于别人的国里么?”假如现在 
(有人)在国中得了罪,还有别国可以逃避,然而父亲以此告诫儿子,兄长 
以此告诫弟弟,说:“警戒呀!谨慎呀!处身国中不可不警戒谨慎呀!”现 
在的人都处身天下,侍奉上天,如果得罪了上天,将没有地方可以逃避了。 
然而没有人知道以此互相警戒。我因此知道他们对大事情不知道。 
所以墨子说道:“警戒呀!谨慎呀!一定要做天所希望的,除去天所厌 
恶的。”天所希望的是什么呢?所厌恶的是什么呢?天希望义而厌恶不义。 
从何知道是这样呢?因为义即是正。 
因何知道义即是正呢?天下有义就治理,无义就混乱,我因此知道义就 
是正。然而所谓正,不能自下正上,必须从上正下。所以庶民百姓不得肆意 
去从事,有士来匡正他;士不得肆意去做。有大夫来匡正他;大夫不得肆意 
去做,有诸侯去匡正他;诸侯不得肆意去做,有三公来匡正他;三公不得肆 
意去做,有天子匡正他;天子不得肆意去做,有上天匡正他。现在天下的士 
君子对于天子匡正天下都很明白,但对上天匡正天子却不明白。所以古代的 
圣人明白地将此道理告诉人们,说:“天子有善,天能赏他;天子有过,天 
能罚他。”若天子赏罚不当,刑罚不公,天就会降下疾病灾祸,霜露失时。 
这时天子必须要喂养牛羊猪狗,洁净地整备粢盛酒醴,去向上天祭祀,祷告, 
求福。但我从来就不曾听说过上天向天子祷告和求福的。我由此知道天比天 
子尊贵、庄重。所以义不从愚蠢而卑贱的人中产生,必定从尊贵而聪明的人 
中产生。那么谁是尊贵的?天是尊贵的。谁是聪明的?天是聪明的。既然如 
此,那么义果真是从上天产生出来的了。现在天下的士君子希望行义的话, 
那么就不可不顺从天意。 
顺从天意应怎样做呢?回答说:兼爱天下的人怎么知道是兼爱天下的人 
呢?因为天对人民的祭祀全都享用。怎么知道天兼而食之呢?自古及今,无 
论如何遥远偏僻的国家,都喂养牛羊狗猪,洁净地整备酒醴粢盛,用以祭祀 
山川、上帝、鬼神,由此知道上天对人民兼而食之。假如兼而食之,必定会 
兼而爱之,就好象楚、越的国君一样。现在楚王在楚国四境之内享用食物, 
所以爱楚国的人。越王在越国享用食物,所以爱越国的人。现在天对天下兼 
而享用,我因此知道它爱天下的人。 
而且上天爱护百姓,不仅此而己。现在天下所有的国家,凡是吃米粮的 
人民,杀了无辜的人,必定得到一种不祥,杀无辜的是谁呢?回答说:“是 

人。”给他不祥的是谁呢?“是天”。假若上天内心确实不爱护这些百姓, 
那为什么在人杀了无辜之后,天要给他以不祥呢?并且上天爱护百姓是很厚 
重的,上天爱护百姓是很普遍的,这已经可以知道了。怎么知道上天爱护百 
姓呢?我从贤者必定要赏善罚暴得知。怎么知道贤者必然赏善罚暴呢?我从 
从前三代圣王的事迹知道这个。从前三代的圣王尧、舜、禹、汤、文王、武 
王兼爱天下,从而造福人民,改移百姓的心意,率领他们敬奉上帝、山川、 
鬼神。上天因为他们爱自己所爱的人,利自己所利的人,于是加重他们的赏 
赐,使他们居于上位,立为天子,(后世)以为表率,称之为圣人。从这可 
知赏善的证据。从前三代的暴君,如桀、纣、幽王、厉王等,对天下人全都 
憎恶,残害他们,改移百姓的心意,率领他们侮慢上帝、山川、鬼神,天因 
为他们不跟从自己的所爱,反而憎恶他们,不跟从自己的所利,反而残害他 
们,于是对他们加以惩罚,使他们父子离散,国家灭亡,丧失社稷,忧及本 
身。而天下的百姓也都非毁他们,到了子孙万世以后,仍然受人们的唾骂, 
称他们为暴君,这就是罚暴的明证了。现今天下的士君子,若要行事合乎义, 
就不可不顺从天意。 
顺从天意,就是“兼”;违反天意,就是“别”。兼的道理,就是义政; 
别的道理,就是力政。如果问道:“义政是什么样呢?”回答说:大的不攻 
打小的,强的不欺侮弱的,多的不残害少的,狡诈的不欺骗愚笨的,尊贵的 
不傲视卑贱的,富足的不傲慢贫困的,年壮的不掠夺年老的。所以天下众国, 
不以水火、毒药、刀兵相互杀害。这种事上利于天,中利于鬼,下利于人。 
三者有利,就无所不利,叫作天德。所以凡从事于此的,就是圣智、仁义、 
忠惠、慈孝,所以聚集天下的好名声加到他身上。这是什么缘故呢?就是顺 
从天意。问道:“力政是什么样呢?”回答说:大的攻打小的强的欺侮弱的, 
多的残害少的,狡诈的欺骗愚笨的,尊贵的傲视卑贱的,富裕的傲慢贫困的, 
年壮的掠夺年老的,所以天下众国,一齐拿着水火、毒药、刀兵来相互残害。 
这种事上不利于天,中不利于鬼,下不利于人,三者不利就无所得利,所以 
称之为(天)贼。凡从事于这些事的,就是寇乱、盗贼、不仁不义、不忠不 
惠、不慈不孝,所以聚集天下的恶名加在他们头上。这是什么缘故呢?就是 
违反了天意。 
所以墨子设立天志以为仪法,就象轮匠有圆规,木匠有方尺一样,现在 
轮人使用圆规,木匠使用方尺,以之知道方与圆的区别。所以墨子设立天志 
以为仪法,我因此而知道天下的士君子离义还很远。怎么知道天下的士君子 
离义还很远呢?现在大国的君主自得地说:“我们处于大国地位而不攻打小 
国,我怎能成为大国呢?”因此差遣他们的爪牙,排列他们的舟车队伍,用 
以攻伐无罪的国家。进入他们的国境,割掉他们的庄稼,砍伐他们的树木, 
毁坏他们的城郭,以及填没他们的沟池,焚烧他们的祖庙,屠杀他们的牲口。 
人民抵抗的,就杀掉;不抵抗的就捆缚回去,男人用作奴仆,马夫,女从用 
作舂米、掌酒的家奴。那些喜好攻伐的君主,不知道这是不仁不义,还以此 
通告四邻的国君说:“我攻下别国,覆灭他们的军队,杀了将领多少人。” 
他邻国的君主,也不知道这是不仁不义,又准备皮币,拿出仓库的积藏派人 
去犒劳庆赏。那些喜好攻伐的君主又绝对不知道这是不仁不义,又把它写在 
简帛上,藏在府库中,作为后世子孙的,必定将要顺从他们先君的志行,说 
道:“为什么不打开我们的府库,看看我们先君留下的法则呢?”(那上面) 
必定不会写着“文王、武王的政绩象这样”,而必定写着“我攻下敌国,覆 

灭他们的军队,杀了将领若干人”。那些喜好攻伐的君主不知道这是不仁不 
义;他的邻国君主,也不知道这是不仁不义,因此攻伐代代不止。这就是我 
所说的(士君子)对于大事全不明白的缘故。 
所谓小事则知道,又怎么样呢?比如现在这里有一个人,他进入别人的 
果场菜园偷窃人家的桃子、李子、瓜菜和生姜,上面抓住了将会惩罚他,大 
众听到了就指责他。这是什么原因呢?是因为他不参与种植之劳,却获得了 
果实,取到了不属于自己的东西的缘故。何况还有翻越别人的围墙,去抓取 
别人子女的呢!与角穿人家的府库,偷窃人家的金玉布帛的呢!与翻越人家 
的牛栏马圈,盗取人家牛马的呢!何况还有杀掉一个无罪的人呢!当今的王 
公大人执掌政治,对于从杀掉一个无罪的人,翻越人家的围墙抓取别人的子 
女,与角穿别人的府库而偷取人家的金玉布帛,与翻越别人的牛栏马牢而盗 
取牛马的,与进入人家的果场菜园而偷取桃李瓜果的,现在的王公大人对这 
些所判的罪,即使古代的圣王如尧、舜、禹、汤、文王、武王等治政,也不 
会与此不同。现在天下的诸侯,大概还全都在相互侵犯、攻伐、兼并,这与 
杀死一个无辜的人相比,(罪过)已是几千万倍了。这与翻越别人的围墙而 
抓取别人的子女相比,与角穿人家的府库而窃取金玉布帛相比,(罪过)也 
已数千万倍了。与翻越别人的牛栏马圈而偷窃别人的牛马相比,与进入人家 
的果场菜园而窃取人家的桃、李、瓜、姜相比,(罪过)已数千万倍了!然 
而他们自己却说:“这是义呀!” 
所以墨子说道:“这是混乱我的说法。它和把黑白甘苦混淆在一起有什 
么区别呢!假如现在这里有一个人,少许给他看一点黑色,他说是黑的,多 
给他看些黑色,他却说白的,结果他必然会说:‘我的眼睛昏乱,不知道黑 
白的分别。’假如现在这里有一个人,少许给他尝点甜味,他说是甜的;多 
多给他尝些甜味,他说是苦的。结果他必然会说:‘我的口味乱了,我不知 
道甜和苦的味道。’现在的王公大人施政,若有人杀人,他的国家必然禁止。 
如果有人拿兵器多多杀掉邻国的人,却说这是义。这难道与混淆黑白、甘苦 
的做法有什么区别吗!” 
所以墨子设立天志,作为法度标准。不仅墨子以天志为法度,就是先王 
的书《大夏》(即《诗·大雅》)中这样说过:“上帝对文王说:我思念有 
光明德行的人,他不大显露声色,也不崇尚侈大与变革,不识不知,顺从天 
帝的法则。”这是告诫周文王以天志为法度,顺从天帝的法则。所以当今天 
下的士君子,如果心中确实希望实行仁义,追求做上层士,上希望符合圣王 
之道,下希望符合国家百姓的利益,对天志就不可不详察。天志就是义的原 
则。 

\chapter{二十六  明鬼(1)下}
\chapter{二十七  非乐(1)}
\chapter{二十八  非命(1)上}
\chapter{二十九  非命(1)中}
\chapter{三十  非命(1)下}
\chapter{三十一  非儒(1)下}
\chapter{三十二  大取(1)}
\chapter{三十三  小取(1)}
\chapter{三十四  耕柱(1)}
\chapter{三十五  贵义}

子墨子曰:“万事莫贵于义。今谓人曰:‘予子冠履,而断子之手足, 
子为之乎?’必不为。何故?则冠履不若手足之贵也。又曰:‘予子天下, 
而杀子之身,子为之乎?’必不为。何故?则天下不若身之贵也。争一言以 
相杀,是贵义于其身也。故曰:万事莫贵于义也。” 
子墨子自鲁即齐,过故人,谓子墨子曰:“今天下莫为义,子独自苦而 
为义,子不若已。”子墨子曰:“今有人于此,有子十人,一人耕而九人处, 
则耕者不可以不益急矣。何故?则食者众而耕者寡也。今天下莫为义,则子 
如劝我者也(2),何故止我?” 
子墨子南游于楚,见楚献惠王,献惠王以老辞,使穆贺见子墨子。子墨 
子说穆贺,穆贺大说(3),谓子墨子曰:“子之言,则成善矣(4)!而君王, 
天下之大王也,毋乃曰‘贱人之所为’,而不用乎?”子墨子曰:“唯其可 
行。譬若药然,草之本,天子食之,以顺其疾,岂曰‘一草之本’而不食哉? 
今农夫入其税于大人,大人为酒醴粢盛,以祭上帝鬼神,岂曰‘贱人之所为’, 
而不享哉?故虽贱人也,上比之农,下比之药,曾不若一草之本乎?且主君 
亦尝闻汤之说乎?昔者汤将往见伊尹,令彭氏之子御,彭氏之子半道而问曰: 
‘君将何之?’汤曰:‘将往见伊尹。’彭氏之子曰:‘伊尹,天下之贱人 
也。若君欲见之,亦令召问焉,彼受赐矣。’汤曰:‘非女所知也(5)。今有 
药此,食之则耳加聪,目加明,则吾必说而强食之。今夫伊尹之于我国也, 
譬之良医善药也。而子不欲我见伊尹,是子不欲吾善也。’因下彭氏之子, 
不使御。彼苟然,然后可也。” 
子墨子曰:“凡言凡动,利于天、鬼、百姓者为之;凡言凡动,害于天、 
鬼、百姓者舍之。凡言凡动,合于三代圣王尧、舜、禹、汤、文、武者为之; 
凡言凡动,合于三代暴王桀、纣、幽、厉者舍之。” 
子墨子曰:“言足以迁行者,常之;不足以迁行者,勿常。不足以迁行 
而常之,是荡口也。” 
子墨子曰:“必去六辟(6)。默则思,言则诲,动则事,使三者代御,必 
为圣人。” 
“必去喜,去怒,去乐,去悲,去爱,而用仁义。手足口鼻耳,从事于 
义,必为圣人。” 
子墨子谓二三子曰:“为义而不能,必无排其道。譬若匠人之斫而不能, 
无排其绳。” 
子墨子曰:“世之君子,使之为一犬一彘之宰,不能则辞之;使为一国 
之相,不能而为之。岂不悖哉!” 
子墨子曰:“今瞽曰:‘钜者白也(7),黔者黑也。’虽明目者无以易之。 
兼白黑,使瞽取焉,不能知也。故我曰瞽不知白黑者,非以其名也,以其取 
也。今天下之君子之名仁也,虽禹、汤无以易之。兼仁与不仁,而使天下之 
君子取焉,不能知也。故我曰天下之君子不知仁者,非以其名也,亦以其取 
也。” 
子墨子曰:“今士之用身,不若商人之用一布之慎也(8)。商人用一布布 
(9),不敢继苟而雠焉(10),必择良者。今士之用身则不然,意之所欲则为之, 
厚者入刑罚,薄者被毁丑,则士之用身,不若商人之用一布之慎也。” 
子墨子曰:“世之君子欲其义之成,而助之修其身则愠,是犹欲其墙之 

成,而人助之筑则愠也。岂不悖哉!” 
子墨子曰:“古之圣王,欲传其道于后世,是故书之竹帛,镂之金石, 
传遗后世子孙,欲后世子孙法之也。今闻先王之遗而不为(11),是废先王之 
传也。” 
子墨子南游使卫,关中载书甚多(12),弦唐子见而怪之,曰:“吾夫子 
教公尚过曰:‘揣曲直而已。’今夫子载书甚多,何有也?”子墨子曰:“昔 
者周公旦朝读书百篇,夕见漆十士(13),故周公旦佐相天子,其修至于今。 
翟上无君上之事,下无耕农之难,吾安敢废此?翟闻之:‘同归之物,信有 
误者。’然而民听不钧(14),是以书多也。今若过之心者,数逆于精微。同 
归之物,既已知其要矣,是以不教以书也。而子何怪焉?” 
子墨子谓公良桓子曰:“卫,小国也,处于齐、晋之间,犹贫家之处于 
富家之间也。贫家而学富家之衣食多用,则速亡必矣。今简子之家(15),饰 
车数百乘,马食菽粟者数百匹,妇人衣文绣者数百人,吾取饰车食马之费 
(16),与绣衣之财,以畜士,必千人有余。若有患难,则使百人处于前,数 
百于后,与妇人数百人处前后,孰安?吾以为不若畜士之安也。” 
子墨子仕人于卫,所仕者至而反。子墨子曰:“何故反(17)?”对曰: 
“与我言而不当。曰‘待女以千盆’,授我五百盆,故去之也。”子墨子曰: 
“授子过千盆,则子去之乎?”对曰:“不去。”子墨子曰:“然则非为其 
不审也(18),为其寡也。” 
子墨子曰:“世俗之君子,视义士不若负粟者。今有人于此,负粟息于 
路侧,欲起而不能,君子见之,无长少贵贱,必起之。何故也?曰:义也。 
今为义之君子,奉承先王之道以语之,纵不说而行,又从而非毁之,则是世 
俗之君子之视义士也,不若视负粟者也。” 
子墨子曰:“商人之四方(19),市贾信徙(20),虽有关梁之难,盗贼之 
危,必为之。今士坐而言义,无关梁之难,盗贼之危,此为信徙,不可胜计, 
然而不为,则士之计利,不若商人之察也。” 
子墨子北之齐,遇日者(21)。日者曰:“帝以今日杀黑龙于北方,而先 
生之色黑,不可以北。”子墨子不听,遂北,至淄水,不遂而反焉。日者曰: 
“我谓先生不可以北。”子墨子曰:“南之人不得北,北之人不得南,其色 
有黑者,有白者,何故皆不遂也?且帝以甲乙杀青龙于东方,以丙丁杀赤龙 
于南方,以庚辛杀白龙于西方,以壬癸杀黑龙于北方,若用子之言,则是禁 
天下之行者也。是围心而虚天下也,子之言不可用也。” 
子墨子曰:“吾言足用矣,舍言革思者,是犹舍获而攗粟也。以其言非 
吾言者,是犹以卵投石也,尽天下之卵,其石犹是也,不可毁也。” 


[注释] 

(1)本篇各段以语录体形式记述了墨子的一些言论,主要说的还是“义”的问题。墨子提出,万 

事没有比义更珍贵的了,人们的一切言论行动,都要从事于义。墨子自己就能够自苦行义。他批评世 

俗君子,能嘴上道说仁义,实际上却不能实行。(2)如:宜。(3)说:通“悦”。(4)成:通“诚”,确 

实。(5)女:通“汝”。(6)辟:通“僻”,邪僻。(7)钜:疑“银”字之误。(8)布:古代钱币。(9) 

后一“布”字当作“市”,购物之意。(10)继:疑“纵”字之误;雠:通“售”,以钱买物。(11)遗: 

“道”字之误。(12)关中:指车上横阑之内,即车中。(13)漆:“七”之借音字。(14)钧:通“均”。 

(15)简:阅。(16)吾:“若”字之误。(17)反:通“返”。(18)审:疑为“当”字之误。(19)之:往。 

(20)贾:通“价”;信:“倍”字之误。(21)日者:古时候根据天象变化预测吉凶的人。 

[白话] 
墨子说:“万事没有比义更珍贵的了。假如现在对别人说:‘给你帽子 
和鞋,但是要砍断你的手、脚,你干这件事吗?’那人一定不干。为什么呢? 
因为帽、鞋不如手、脚珍贵。又说:‘给你天下,但要杀死你,你干这件事 
吗?’那人一定不干。为什么呢?因为天下不如自身珍贵。因争辩一句话而 
互相残杀,是因为把义看得比自身珍贵。所以说:万事没有比义更珍贵的了。” 
墨子从鲁国到齐国,探望了老朋友。朋友对墨子说:“现在天下没有人 
行义,你何必独自苦行为义,不如就此停止。”墨子说:“现在这里有一人, 
他有十个儿子,但只有一个儿子耕种,其他九个都闲着,耕种的这一个不能 
不更加紧张啊。为什么呢?因为吃饭的人多而耕种的人少。现在天下没有人 
行义,你应该勉励我行义,为什么还制止我呢?” 
墨子南游到了楚国,去见楚惠王,惠王借口自己年老推辞了,派穆贺会 
见墨子。墨子劝说穆贺,穆贺非常高兴,对墨子说:“你的主张确实好啊, 
但君王是天下的大王,恐怕会认为这是一个普通百姓的主张而不加采用吧!” 
墨子答道:“只要它能行之有效就行了,比如药,是一把草根,天子吃了它, 
用以治愈自己的疾病,难道会认为是一把草根而不吃吗?现在农民缴纳租税 
给贵族,贵族大人们酿美酒、造祭品,用来祭祀上帝、鬼神,难道会认为这 
是普通百姓做的而不享用吗?所以虽然是普通百姓,从上把他比于农民,从 
下把他比于药,难道还不如一把草根吗?况且惠王也曾听说过商汤的传说 
吧?过去商汤去见伊尹,叫彭氏的儿子给自己驾车。彭氏之子半路上问商汤 
说:‘您要到哪儿去呢?’商汤答道:‘我将去见伊尹。’彭氏之子说:‘伊 
尹,只不过是天下的一位普通百姓。如果您一定要见他,只要下令召见而问 
他,这在他已蒙受恩遇了!’商汤说:‘这不是你所知道的。如果现在这里 
有一种药,吃了它,耳朵会更加灵敏,眼睛会更加明亮,那么我一定会喜欢 
而努力吃药。现在伊尹对于我国,就好象良医好药,而你却不想让我见伊尹, 
这是你不想让我好啊!’于是叫彭氏的儿子下去,不让他驾车了。如果惠王 
能象商汤这样,以后就可以采纳普通百姓的主张了。” 
墨子说:“一切言论一切行动,有利于天、鬼神、百姓的,就去做;一 
切言论一切行动,有害于天、鬼神、百姓的,就舍弃。一切言论一切行动, 
合乎三代圣王尧、舜、禹、商汤、周文王、周武王的,就去做;合乎三代暴 
君夏桀、商纣、周幽王、周厉王的,就舍弃。” 
墨子说:“言论足以付之行动的,就推崇它;不足以付之行动的,就不 
要推崇。不足以付之行动,却要推崇它,就是空言妄语了。” 
墨子说:“一定要去掉六种邪僻,沉默之时能思索,出言能教导人,行 
动能从事义。使这三者交替进行,一定能成为圣人。”(墨子说:)“一定 
要去掉喜,去掉怒,去掉乐,去掉悲,去掉爱,以仁义作为一切言行的准则。 
手、脚、口、鼻、耳,都用来从事义,一定会成为圣人。” 
墨子对几个弟子说:“行义而不能胜任之时,一定不可归罪于学说、主 
张本身。好象木匠劈木材不能劈好,不可归罪于墨线一样。” 
墨子说:“世上的君子,使他作为宰杀一狗一猪的屠夫,如果干不了就 
推辞;使他作一国的国相,干不了却照样去作,这难道不荒谬吗?” 
墨子说:“现在有一个盲人说:‘银是白的,黔是黑的。’即使是眼睛 
明亮的人也不能更改它。把白的和黑的东西放在一块儿,让盲人分辨,他就 
不能知道了。所以我说:盲人不知白黑,不是因为他不能称说白黑的名称, 

而是因为他无法择取。现在天下的君子称说‘仁’的名,即使禹、汤也无法 
更改它。把符合仁和不符合仁的事物混杂在一起,让天下的君子择取,他们 
就不知道了。所以我说:天下的君子,不知道‘仁’,不是因为他不能称说 
仁的名,而是因为他无法择取。” 
墨子说:“现在士以身处世,不如商人使用一钱慎重。商人用一钱购买 
东西,不敢任意马虎地购买,一定选择好的。现在士使用自己的身体却不是 
这样,随心所欲地胡作非为。过错严重的陷入刑罚,过错轻的蒙受非议羞耻。 
士以身处世,不如商人使用一钱慎重。” 
墨子说:“当代的君子,想实现他的道义,而帮助他修养身心却怨恨。 
这就象要筑成墙,而别人帮助他却怨恨一样,难道不荒谬吗?” 
墨子说:“古时候的圣王,想把自己的学说传给后代,因此写在竹、帛 
上,刻在金、石上,传留给后代子孙,要后代子孙效法它。现在听到了先王 
的学说却不去实行,这是废弃先王所传的学说了。” 
墨子南游到卫国去,车中装载的书很多。弦唐子见了很奇怪,问道:“老 
师您曾教导公尚过说:‘书不过用来衡量是非曲直罢了。’现在您装载这么 
多书,有什么用处呢?”墨子说:“过去周公旦早晨读一百篇书,晚上见七 
十士。所以周公旦辅助天子,他的美善传到了今天。我上没有承担国君授予 
的职事,下没有耕种的艰难,我如何敢抛弃这些书!我听说过:天下万事万 
物殊途同归,流传的时候确实会出现差错。但是由于人们听到的不能一致, 
书就多起来了。现在象公尚过那样的人,心对于事理已达到了洞察精微。对 
于殊途同归的天下事物,已知道切要合理之处,因此就不用书教育了。你为 
什么要奇怪呢?” 
墨子对公良桓子说:“卫国是一个小国,处在齐国晋国之间,就象穷家 
处在富家之间一样。穷家如果学富家的穿衣、吃饭、多花费,那么穷家一定 
很快就破败了。现在看看您的家族,以文彩装饰的车子有数百辆,吃菽、粟 
的马有数百匹,穿文绣的妇人有数百人。如果把装饰车辆、养马的费用和做 
绣花衣裳的钱财用来养士,一定可以养一千人还有余。如果遇到危难,就命 
令几百人在前面,几百人在后面,这与几百个妇人站在前后,那一个安全呢? 
我以为不如养士安全。” 
墨子使人到卫国做官,去做官的人到卫国后却回来了。墨子问他:“为 
什么回来呢?”那人回答说:“卫国与我说话不合。说:‘给你千盆的俸禄’, 
却实际给了我五百盆,所以我离开了卫国。”墨子又问:“给你的俸禄超过 
千盆,你还离开吗?”那人答道:“不离开。”墨子说:“既然这样,那么 
你不是因为卫国说话与你不合,而是因为俸禄少。” 
墨子说:“世俗的君子,看待行义之人还不如一个背粟的人。现在这里 
有一个人背着粟,在路边休息,想站起来却起不来。君子见了,不管他是少、 
长、贵、贱,一定帮助他站起来。为什么呢?说:这是义。现在行义的君子, 
承受先王的学说来告诉世俗的君子,世俗的君子,即使不喜欢不实行行义之 
士的言论也罢,却又加以非议、诋毁。这就是世俗的君子看待行义之士,还 
不如一个背粟的人了。” 
墨子说:“商人到四方去,买卖的价钱相差一倍或数倍,即使有通过关 
卡那种艰难,碰见盗贼那种危险,也一定去做买卖。现在士坐着道说义,没 
有关卡的艰难,没有盗贼的危险,即使这样还不实行。那么士人计算利益, 
不如商人明察了。” 

墨子往北到齐国去,遇到一个占卦先生。占卦先生说:“历史上的今天, 
黄帝在北方杀死了黑龙,你的脸色黑,不能向北去。”墨子不听,竟继续向 
北走。到淄水边,没有渡河返了回来。占卦先生说:“我对你说过不能向北 
走。”墨子说:“淄水之南的人不能渡淄水北去,淄水之北的人也不能渡淄 
水南行,他们的脸色有黑的有白的,为什么都不能渡呢?况且黄帝甲乙日在 
东方杀死了青龙,丙丁日在南方杀死了赤龙,庚辛日在西方杀死了白龙,壬 
癸日在北方杀死了黑龙,假如实行你的办法,这是禁止天下所有的人来往了。 
这也是困蔽人心,使天下如同虚无人迹一样。所以你的言论不能用。” 
墨子说:“我的言论足够用了!舍弃我的学说、主张而另外思虑,这就 
象放弃收获而去拾别人遗留的谷穗一样。用别人的言论否定我的言论,这就 
象用鸡蛋去碰石头一样。用尽天下的鸡蛋,石头还是这个样子,并不能毁坏 
它。” 

\chapter{三十六  公孟(1)}
\chapter{三十七  鲁问}

鲁君谓子墨子曰:“吾恐齐之攻我也,可救乎?”子墨子曰:“可。昔 
者,三代之圣王禹、汤、文、武,百里之诸侯也,说忠行义,取天下;三代 
之暴王桀、纣、幽、厉,雠怨行暴,失天下。吾愿主君之上者尊天事鬼,下 
者爱利百姓,厚为皮币,卑辞令,亟遍礼四邻诸侯,驱国而以事齐,患可救 
也。非此,顾无可为者。” 
齐将伐鲁,子墨子谓项子牛曰:“伐鲁,齐之大过也。昔者,吴王东伐 
越,栖诸会稽;西伐楚,葆昭王于随(2);北伐齐,取国子以归于吴。诸侯报 
其雠,百姓苦其劳,而弗为用。是以国为虚戾,身为刑戮也。昔者智伯伐范 
氏与中行氏,兼三晋之地。诸侯报其雠,百姓苦其劳,而弗为用。是以国为 
虚戾,身为刑戮,用是也。故大国之攻小国也,是交相贼也,过必反于国。” 
子墨子见齐大王曰:“今有刀于此,试之人头,倅然断之,可谓利乎?” 
大王曰:“利。”子墨子曰:“多试之人头,倅然断之,可谓利乎?”大王 
曰:“利。”子墨子曰:“刀则利矣,孰将受其不祥?”大王曰:“刀受其 
利,试者受其不祥。”子墨子曰:“并国覆军,贼敖百姓(3),就将受其不祥?” 
大王俯仰而思之,曰:“我受其不祥。” 
鲁阳文君将攻郑,子墨子闻而止之,谓阳文君曰:“今使鲁四境之内, 
大都攻其小都,大家伐其小家,杀其人民,取其牛马、狗豕、布帛、米粟、 
货财,则何若?”鲁阳文君曰:“鲁四境之内,皆寡人之臣也。今大都攻其 
小都,大家伐其小家,夺之货财,则寡人必将厚罚之。”子墨子曰:“夫天 
之兼有天下也,亦犹君之有四境之内也。今举兵将以攻郑,天诛其不至乎?” 
鲁阳文君曰:“先生何止我攻郑也?我攻郑,顺于天之志。郑人三世杀其父 
(4),天加诛焉,使三年不全,我将助天诛也。”子墨子曰:“郑人三世杀其 
父,而天加诛焉,使三年不全,天诛足矣。今又举兵,将以攻郑,曰吾攻郑 
也,顺于天之志。譬有人于此,其子强梁不材(5),故其父笞之,其邻家之父, 
举木而击之,曰:吾击之也,顺于其父之志。则岂不悖哉!” 
子墨子谓鲁阳文君曰:“攻其邻国,杀其民人,取其牛马、粟米、货财, 
则书之于竹帛,镂之于金石,以为铭于钟鼎,传遗后世子孙,曰:‘莫若我 
多!’今贱人也,亦攻其邻家,杀其人民,取其狗豕、食粮、衣裘,亦书之 
竹帛,以为铭于席豆,以遗后世子孙,曰:‘莫若我多!’其可乎?”鲁阳 
文君曰:“然。吾以子之言观之,则天下之所谓可者,未必然也。” 
子墨子为鲁阳文君曰(6):“世俗之君子,皆知小物,而不知大物。今有 
人于此,窃一犬一彘,则谓之不仁,窃一国一都,则以为义。譬犹小视白谓 
之白,大视白则谓之黑。是故世俗之君子,知小物而不知大物者,此若言之 
谓也。” 
鲁阳文君语子墨子曰:“楚之南,有啖人之国者桥,其国之长子生,则 
鲜而食之(7),谓之宜弟,美则以遗其君,君喜则赏其父。岂不恶俗哉?”子 
墨子曰:“虽中国之俗,亦犹是也。杀其父而赏其子,何以异食其子而赏其 
父者哉?苟不用仁义,何以非夷人食其子也?” 
鲁君之嬖人死,鲁君为之诔,鲁人因说而用之(8)。子墨子闻之曰:“诔 
者,道死人之志也。今因说而用之,是犹以来首从服也。”(9) 
鲁阳文君谓子墨子曰:“有语我以忠臣者,令之俯则俯,令之仰则仰, 
处则静,呼则应,可谓忠臣乎?”子墨子曰:“令之俯则俯,令之仰则仰, 

是似景也(10);处则静,呼则应,是似响也。君将何得于景与响哉?若以翟 
之所谓忠臣者,上有过,则微之以谏(11);己有善,则访之上,而无敢以告。 
外匡其邪,而入其善。尚同而无下比,是以美善在上,而怨雠在下;安乐在 
上,而忧戚在臣。此翟之所谓忠臣者也。” 
鲁君谓子墨子曰:“我有二子,一人者好学,一人者好分人财,孰以为 
太子而可?”子墨子曰:“未可知也。或所为赏与为是也。钓者之恭,非为 
鱼赐也;饵鼠以虫,非爱之也。吾愿主君之合其志功而观焉。” 
鲁人有因子墨子而学其子者(12),其子战而死,其父让子墨子(13)。子 
墨子曰:“子欲学子之子,今学成矣,战而死,而子愠,而犹欲粜籴,雠则 
愠也。岂不费哉(14)!” 
鲁之南鄙人有吴虑者,冬陶夏耕,自比于舜。子墨子闻而见之。吴虑谓 
子墨子:“义耳义耳,焉用言之哉?”子墨子曰:“子之所谓义者,亦有力 
以劳人,有财以分人乎?”吴虑曰:“有。”子墨子曰:“翟尝计之矣。翟 
虑耕而食天下之人矣。盛,然后当一农之耕,分诸天下,不能人得一升粟。 
籍而以为得一升粟(15),其不能饱天下之饥者,既可睹矣。翟虑织而衣天下 
之人矣,盛,然后当一妇人之织,分诸天下,不能人得尺布。籍而以为得尺 
布,其不能暖天下之寒者,既可睹矣。翟虑被坚执锐(16),救诸侯之患,盛, 
然后当一夫之战,一夫之战,其不御三军,既可睹矣。翟以为不若诵先王之 
道,而求其说,通圣人之言,而察其辞,上说王公大人,次匹夫徒步之士。 
王公大人用吾言,国必治;匹夫徒步之士用吾言,行必修。故翟以为虽不耕 
而食饥,不织而衣寒,功贤于耕而食之、织而衣之者也。故翟以为虽不耕织 
乎,而功贤于耕织也。”吴虑谓子墨子曰:“义耳义耳,焉用言之哉?”子 
墨子曰:“籍设而天下不知耕,教人耕,与不教人耕而独耕者,其功孰多?” 
吴虑曰:“教人耕者,其功多。”子墨子曰:“籍设而攻不义之国,鼓而使 
众进战,与不鼓而使众进战而独进战者,其功孰多?”吴虑曰:“鼓而进众 
者,其功多。”子墨子曰:“天下匹夫徒步之士少知义,而教天下以义者, 
功亦多,何故弗言也?若得鼓而进于义,则吾义岂不益进哉!” 
子墨子游公尚过于越。公尚过说越王,越王大说,谓公尚过曰:“先生 
苟能使子墨子于越而教寡人,请裂故吴之地(17),方五百里,以封子墨子。” 
公尚过许诺。遂为公尚过束车五十乘,以迎子墨子于鲁。曰:“吾以夫子之 
道说越王,越王大说,谓过曰:‘苟能使子墨子至于越而教寡人,请裂故吴 
之地,方五百里,以封子。’”子墨子谓公尚过曰:“子观越王之志何若? 
意越王将听吾言,用吾道,则翟将往,量腹而食,度身而衣,自比于群臣(18), 
奚能以封为哉!抑越不听吾言,不用吾道,而吾往焉,则是我以义粜也。钧 
之粜,亦于中国耳,何必于越哉!” 
子墨子游,魏越曰:“既得见四方之君,子则将先语(19)?”子墨子曰: 
“凡入国,必择务而从事焉。国家昏乱,则语之尚贤、尚同;国家贫,则语 
之节用、节葬;国家憙音湛湎(20),则语之非乐、非命;国家淫僻无礼,则 
语之尊天事鬼;国家务夺侵凌,即语之兼爱、非攻。故曰:择务而从事焉。” 
子墨子出曹公子而于宋。三年而反,睹子墨子曰:“始吾游于子之门, 
短褐之衣,藜藿之羹,朝得之,则夕弗得祭祀鬼神。今而以夫子之教,家厚 
于始也。有家厚,谨祭祀鬼神。然而人徒多死,六畜不蕃,身湛于病,吾未 
知夫子之道之可用也。”子墨子曰:“不然。夫鬼神之所欲于人者多:欲人 
之处高爵禄,则以让贤也;多财,则以分贫也。夫鬼神,岂唯擢季拑肺之为 

欲哉?(21)今子处高爵禄而不以让贤,一不祥也;多财而不以分贫,二不祥 
也。今子事鬼神,唯祭而已矣,而曰‘病何自至哉’,是犹百门而闭一门焉, 
曰‘盗何从入’。若是而求福于有怪之鬼,岂可哉?” 
鲁祝以一豚祭(22),而求百福于鬼神。子墨子闻之曰:“是不可。今施 
人薄而望人厚,则人唯恐其有赐于己也。今以一豚祭,而求百福于鬼神,唯 
恐其以牛羊祀也。古者圣王事鬼神,祭而已矣。今以豚祭而求百福,则其富 
不如其贫也。” 
彭轻生子曰:“往者可知,来者不可知。”子墨子曰:“籍设而亲在百 
里之外,则遇难焉,期以一日也,及之则生,不及则死。今有固车良马于此, 
又有奴马四隅之轮于此(23),使子择焉,子将何乘?”对曰:“乘良马固车, 
可以速至。”子墨子曰:“焉在矣来(24)!” 
孟山誉王子闾曰:“昔白公之祸,执王子闾,斧钺钩要(25),直兵当心, 
谓之曰:‘为王则生,不为王则死!’王子闾曰:‘何其侮我也!杀我亲, 
而喜我以楚国(26)。我得天下而不义,不为也,又况于楚国乎?’遂而不为。 
王子闾岂不仁哉?”子墨子曰:“难则难矣,然而未仁也。若以王为无道, 
则何故不受而治也?若以白公为不义,何故不受王,诛白公然而反王?故曰: 
难则难矣,然而未仁也。” 
子墨子使胜绰事项子牛。项子牛三侵鲁地,而胜绰三从。子墨子闻之, 
使高孙子请而退之,曰:“我使绰也,将以济骄而正嬖也(27)。今绰也禄厚 
而谲夫子,夫子三侵鲁而绰三从,是鼓鞭于马靳也(28)。翟闻之,言义而弗 
行,是犯明也。绰非弗之知也,禄胜义也。” 
昔者楚人与越人舟战于江,楚人顺流而进,迎流而退,见利而进,见不 
利则其退难。越人迎流而进,顺流而退,见利而进,见不利则其退速。越人 
因此若势,亟败楚人(29)。公输子自鲁南游楚,焉始为舟战之器,作为钩强 
之备(30),退者钩之,进者强之,量其钩强之长,而制为之兵。楚之兵节(31), 
越之兵不节,楚人因此若势,亟败越人。公输子善其巧,以语子墨子曰:“我 
舟战有钩强,不知子之义亦有钩强乎?”子墨子曰:“我义之钩强,贤于子 
舟战之钩强。我钩强我(32),钩之以爱,揣之以恭(33)。弗钩以爱则不亲, 
弗揣以恭则速狎,狎而不亲则速离。故交相爱,交相恭,犹若相利也。今子 
钩而止人,人亦钩而止子,子强而距人,人亦强而距子,交相钩,交相强, 
犹若相害也。故我义之钩强,贤子舟战之钩强。” 
公输子削竹木以为鹊,成而飞之,三日不下。公输子自以为至巧。子墨 
子谓公输子曰:“子之为鹊也,不如匠之为车辖,须臾刘三寸之木(34),而 
任五十石之重。故所为功,利于人谓之巧,不利于人谓之拙。” 
公输子谓子墨子曰:“吾未得见之时,我欲得宋。自我得见之后,予我 
宋而不义,我不为。”子墨子曰:“翟之未得见之时也,子欲得宋,自翟得 
见子之后,予子宋而不义,子弗为,是我予子宋也。子务为义,翟又将予子 
天下。” 


[注释] 

(1)本篇各段记载了墨子与诸侯、弟子等人的一些谈话,其中比较重要的内容,有墨子提出的游 

说诸侯,“必择务而从事”的原则;文中多处申明“兼爱”、“非攻”的主张;也有几处专门申说“义” 

的重要性。所有这些内容,体现出墨子向往国家富强、天下安宁、人民安居乐业的理想。(2)葆:通“保”。 

(3)敖:古“杀”字。(4)三世:数代,言其多。(5)强梁:凶暴,强横。(6)为:通“谓”。(7)鲜:“解” 

字之形误。(8)这二句当作:“鲁人为之诔,鲁君因说而用之。”说:通“悦”。(9)来:即氂,牦牛。 

(10)景:通“影”。(11)微:伺察。(12)学:读作“敩”,教的意思。(13)让:责备。(14)费:为“悖” 

之借字。(15)籍:通“藉”,假使。(16)被:通“披”。(17)裂:分。(18)比:列。(19)先:“奚” 

之讹。(20)憙:同“喜”。(21)擢:“攫”之形误,攫:用手取;季:“黍”之形误;拑:“拑”之 

形误,抯:取。(22)祝:司祭人。(23)奴马:驽马。(24)此句应作“焉在不知来”。(25)要:古“腰” 

字。(26)喜:“嬉”之假借字,作弄。(27)济:止;嬖:同“僻”。(28)靳:马当胸的皮带,这里代 

指马胸。(29)亟:屡次。(30)钩强:即钩、镶,古兵器。(31)节:义同“适”。(32)后一个“我”字, 

为“义”之假借字。(33)揣:推拒之意。(34)刘“斵”之形误。 
[白话] 
鲁国国君对墨子说:“我害怕齐国攻打我国,可以解救吗?”墨子说: 
“可以。从前三代的圣王禹、汤、文、武,只不过是百里见方土地的首领, 
喜欢忠诚,实行仁义,终于取得了天下;三代的暴王桀、纣、幽、厉,把怨 
者当作仇人,实行暴政,最终失去了天下。我希望君主您对上尊重上天、敬 
事鬼神,对下爱护、有利于百姓,准备丰厚的皮毛、钱币,辞令要谦恭,赶 
快礼交遍四邻的诸侯,驱使一国的人民,抵御齐国的侵略,这样,祸患就可 
以解救。不这样,看来就毫无其他办法了。” 
齐国将要攻打鲁国,墨子对项子牛说:“攻伐鲁国,是齐国的大错。从 
前吴王夫差向东攻打越国,越王勾践困居在会稽;向西攻打楚国,楚国人在 
随地保卫楚昭王;向北攻打齐国,俘虏齐将押回吴国。后来诸侯来报仇,百 
姓苦于疲惫,不肯为吴王效力,因此国家灭亡了,吴王自身也成为刑戮之人。 
从前智伯攻伐范氏与中行氏的封地,兼有三晋卿的土地。诸侯来报仇,百姓 
苦于疲惫而不肯效力,国家灭亡了,他自己也成为刑戮之人,也由于这个缘 
故。所以大国攻打小国,是互相残害,灾祸必定反及于本国。” 
墨子对齐太公说:“现在这里有一把刀,试着用它来砍人头,一下子就 
砍断了,可以说是锋利吧?”太公说:“锋利。”墨子又说:“试着用它砍 
好多个人的头,一下子就砍断了,可以说是锋利吧?”太公说:“锋利。” 
墨子说:“刀确实锋利,谁将遭受那种不幸呢?”太公说:“刀承受它的锋 
利,试验的人遭受他的不幸。”墨子说:“兼并别国领土,覆灭它的军队, 
残杀它的百姓,谁将会遭受不幸呢?”太公头低下又抬起,思索了一会儿, 
答道:“我将遭受不幸。” 
鲁阳文君将要攻打郑国,墨子听到了就阻止他,对鲁阳文君说:“现在 
让鲁四境之内的大都攻打小都,大家族攻打小家族,杀害人民,掠取牛、马、 
狗、猪、布、帛、米、粟、货、财,那怎么办?”鲁阳文君说:“鲁四境之 
内都是我的臣民。现在大都攻打小都,大家族攻打小家族,掠夺他们的货、 
财,那么我将重重惩罚攻打的人。”墨子说:“上天兼有天下,也就象您具 
有鲁四境之内一样。现在您举兵将要攻打郑国,上天的诛伐难道就不会到来 
吗?”鲁阳文君说:“先生为什么阻止我进攻郑国呢?我进攻郑国,是顺应 
了上天的意志。郑国人数代残杀他们的君主,上天降给他们惩罚,使三年不 
顺利。我将要帮助上天加以诛伐。”墨子说:“郑国人数代残杀他们的君主, 
上天已经给了惩罚,使它三年不顺利,上天的诛伐已经够了!现在您又举兵 
将要攻打郑国,说:‘我进攻郑国,是顺应上天的意志。’好比这里有一个 
人,他的儿子凶暴、强横,不成器,所以他父亲鞭打他。邻居家的父亲,也 
举起木棒击打他,说:‘我打他,是顺应了他父亲的意志。’这难道还不荒 
谬吗!”是! 

墨子对鲁阳文君说:“进攻邻国,杀害它的人民,掠取它的牛、马、粟、 
米、货、财,把这些事书写在竹、帛上,镂刻在金、石上,铭记在钟、鼎上, 
传给后世子孙,说:‘战果没有人比我多!’现在下贱的人,也进攻他的邻 
家,杀害邻家的人口,掠取邻家的狗、猪、食、粮、衣服、被子,也书写在 
竹、帛上,铭记在席子、食器上,传给后世子孙,说:‘战果没有人比我多!’ 
难道可以吗?”鲁阳文君说:“对。我用您的言论观察,那么天下人所说的 
可以的事,就不一定正确了。” 
墨子对鲁阳文君说:“世俗的君子,知道小事却不知道大事。现在这里 
有一个人,假如偷了人家的一只狗一只猪,就被称作不仁;如果窃取了一个 
国家一个都城,就被称作义。这就如同看一小点白说是白,看一大片白则说 
是黑。因此,世俗的君子只知道小事却不知道大事的情况,如同这句话所讲 
的。” 
鲁阳文君告诉墨子说:“楚国的南面有一个吃人的国家,名叫‘桥’, 
在这个国家里,长子出生了,就被杀死吃掉,叫做‘宜弟’。味美就献给国 
君,国君喜欢了就奖赏他的父亲。这难道不是恶俗吗?”墨子说:“即使中 
国的风俗也象这样,父亲因攻战而死,就奖赏他的儿子,这与吃儿子奖赏他 
的父亲有何不同呢?如果不实行仁义,凭什么去指责夷人吃他们的儿子 
呢?” 
鲁国国君的爱妾死了,鲁国人阿谀国君,为她写了诔文,鲁国国君看了 
很高兴,就采用了。墨子听到这件事,说:“诔文,说明死人的心志。现在 
因为高兴采用了它,这就象用牦牛的头来做衣服一样。” 
鲁阳文君对墨子说:“有人把‘忠臣’的样子告诉我:叫他低下头就低 
下头,叫他抬起来就抬起来;日常居住很平静,呼叫他才答应,这可以叫做 
忠臣吗?”墨子答道:“叫他低下头就低下头,叫他抬起来就抬起来,这好 
象影子;日常居住很平常,呼叫他才答应,这就好象回声,你将从象影子和 
回声那样的臣子那里得到什么呢?我所说的忠臣却象这样:国君有过错,则 
伺察机会加以劝谏;自己有好的见解,则上告国君,不敢告诉别人。匡正国 
君的偏邪,使他纳入正道,崇尚同一,不在下面结党营私。因此,美善存在 
于上级,怨仇存在于下面,安乐归于国君,忧戚归于臣下。这才是我所说的 
忠臣。” 
鲁国国君对墨子说:“我有两个儿子,一个爱好学习,一个喜欢将财物 
分给人家,谁可以作为太子?”墨子答道:“这还不能知道。二子也许是为 
着赏赐和名誉而这样做的。钓鱼人躬着身子,并不是对鱼表示恭敬;用虫子 
作为捕鼠的诱饵,并不是喜爱老鼠。我希望主君把他们的动机和效果结合起 
来进行观察。” 
鲁国有一人因与墨子有关系,而使墨子教学他的儿子。他儿子战死了, 
父亲就责备墨子。墨子说:“你要让我教你的儿子,现在学成了,因战而死, 
你却怨恨我;这就象卖出买进粮食,粮食卖出去了却怨恨一样,难道不荒谬 
吗!” 
鲁国的南郊有一个叫吴虑的人,冬天制陶夏天耕作,拿自己与舜相比。 
墨子听说了就去见他。吴虑对墨子说:“义啊义啊,责在切实之行,何必空 
言!”墨子说:“你所谓的义,也有以力量给人效劳,以财物分配给人的方 
面吗?”吴虑回答说:“有。”墨子说:“我曾经思考过:我想自己耕作给 
天下人饭吃,十分努力,这才相当于一个农民的耕作,把收获分配给天下人, 

每一个人得不到一升粟。假设一个人能得一升粟,这不足以喂饱天下饥饿的 
人,是显而易见的。我想自己纺织给天下的人衣服穿,十分努力,这才相当 
于一名妇人的纺织,把布匹分配给天下人,每一个人得不到一尺布。假设一 
个人能得一尺布,这不足以温暖天下寒冷的人,是显而易见的。我想身披坚 
固的铠甲,手执锐利的武器,解救诸侯的患难,十分努力,这才相当于一位 
战士作战。一位战士的作战,不能抵挡三军的进攻,是显而易见的。我认为 
不如诵读与研究先王的学说,通晓与考察圣人的言辞,在上劝说王公大人, 
在下劝说平民百姓。王公大人采用了我的学说,国家一定能得到治理;平民 
百姓采用了我的学说,品行必有修养。所以我认为即使不耕作,这样也可以 
给饥饿的人饭吃,不纺织也可以给寒冷的人衣服穿,功劳胜过耕作了才给人 
饭吃、纺织了才给人衣穿的人。所以,我认为即使不耕作、不纺织,而功劳 
胜过耕作与纺织。”吴虑对墨子说:“义啊义啊,贵在切实之行,何必空言!” 
墨子问道:“假设天下的人不知道耕作,教人耕作的人与不教人耕作却独自 
耕作的人,他们功劳谁的多?”吴虑答道:“教人耕作的人功劳多。”墨子 
又问:“假设进攻不义的国家,击鼓使大家作战的人与不击鼓使大家作战、 
却独自作战的人。他们的功劳谁的多?”吴虑答道:“击鼓使大家作战的人 
功劳多。”墨子说:“天下平民百姓少有人知道仁义,用仁义教天下人的人 
功劳也多,为什么不劝说呢?假若我能鼓动大家达到仁义的要求,那么,我 
的仁义岂不是更加发扬光大了吗!” 
墨子使公尚过前往越国出仕做官。公尚过劝说越王。越王非常高兴,对 
公尚过说:“先生假如能让墨子到越国教导我,我愿意分出过去吴国的地方 
五百里封给墨子。”公尚过答应了。于是给公尚过套了五十辆车,到鲁国迎 
取墨子。公尚过对墨子说:“我用老师的学说劝说越王,越王非常高兴,对 
我说:‘假如你能让墨子到越国教导我,我愿意分出来过去吴国的地方五百 
里封给墨子。’”墨子对公尚过说:“你观察越王的心志怎么样?假如越王 
将听我的言论,采纳我的学说,那么我将前往。或者越国不听我的言论,不 
采纳我的学说,如果我去了,那是我把‘义’出卖了。同样是出卖‘义’, 
在中原国家好了,何必跑到越国呢!” 
墨子出外游历,魏越问他:“如果能见各地的诸侯,您将说什么呢?” 
墨子说:“到了一个国家,选择最重要的事情进行劝导:假如一个国家昏乱, 
就告诉他们尚贤尚同的道理;假如一个国家贫穷,就告诉他们节用节葬;假 
如一个国家喜好声乐、沉迷于酒,就告诉他们非乐非命的好处;假如一个国 
家荒淫、怪僻、不讲究礼节,就告诉他们尊天事鬼;假如一个国家以欺侮、 
掠夺、侵略、凌辱别国为事,就告诉他们兼爱、非攻的益处。所以说‘选择 
最重要的事情进行劝导。’” 
墨子让曹公子到宋国做国,三年后返了回来,见了墨子说:“开始我在 
您门下学习的时候,穿着粗布短衣,吃着野菜一类粗劣的食物,早晨吃了, 
晚上可能就没有了,不能够祭祀鬼神。现在因为你的教育培养,家比当初富 
多了。家富有了,就谨慎祭祀鬼神。象这样反而家里人多死亡,六畜不繁盛 
众多,自身困于病患之中。我还不知道老师的学说是不是可以用。”墨子说: 
“不对。鬼神希望人的东西很多:希望人处高官厚禄时可以让贤,财物多了 
可以分给穷人。鬼神难道仅仅是想取食祭品吗?现在你处在高官厚禄的位置 
上却不让贤,这是第一种不吉祥;财物多不分给穷人,这是第二种不吉祥。 
现在你侍奉鬼神,只有祭祀罢了,却说:病从那里来?这就象百门只闭了一 

门一样,却问:盗贼从哪里进来?象这样向对你有责怪的鬼神求福,难道可 
以吗?” 
鲁国的司祭人用一头小猪祭祀,向鬼神求百福。墨子听到了说:“这不 
行。现在施给人的少,希望人的多,那么,别人就只怕你有东西赐给他们了。 
现在用一头小猪祭祀,向鬼神求百福,鬼神就只怕你用牛羊祭祀了。从前圣 
王侍奉鬼神,祭祀罢了。现在用小猪祭祀向鬼神求百福,与其祭品丰富,还 
不如贫乏的好。” 
彭轻生子说:“过去的事情可以知道,未来的事情不可以知道。”墨子 
说:“假设你的父母亲在百里之外的地方,即将遇到灾难,以一日的期限, 
到达那里他们就活下来了,不到就死了。现在有坚固的车子和骏马在这里, 
同时这里又有驽马和四方形轮子的车,使你选择,你将选择哪一种呢?”彭 
轻生子回答说:“乘坐骏马拉的坚固的车子,可以很快到达。”墨子说:“怎 
能断言未来的事不可知呢?” 
孟山赞扬王子闾说:“从前白公在楚国作乱,抓住了王子闾,用斧钺钩 
着他的腰,用直兵器对着他的心窝,对他说:‘做楚王就让你活,不做楚王 
就让你死。’王子闾回答道:‘怎么这样侮辱我呢!杀害我的亲人,却用给 
予楚国来作弄我。用不义得到天下,我都不做;又何况一个楚国呢?’他终 
究不做楚王。王子闾难道还不仁吗?”墨子说:“王子闾守节不屈,难是够 
难的了,但还没有达到仁。如果他认为楚王昏聩无道,那么为什么不接受王 
位治理国家呢?如果他认为白公不义,为什么不接受王位,诛杀了白公再把 
王位交还惠王呢?所以说:难是够难的了,但还没有达到仁。” 
墨子让弟子胜绰去项子牛那里做官。项子牛三次入侵鲁国的领土,胜绰 
三次都跟从了。墨子听到了这件事,派高孙子请项子牛辞退胜绰,高孙子转 
告墨子的话说:“我派胜绰,将以他阻止骄气,纠正邪僻。现在胜绰得了厚 
禄,却欺骗您,您三次入侵鲁国,胜绰三次跟从,这是在战马的当胸鼓鞭。 
我听说:‘口称仁义却不实行,这是明知故犯。’胜绰不是不知道,他把俸 
禄看得比仁义还重罢了。” 
从前楚国人与越国人在长江上进行船战,楚国人顺流而进,逆流而退; 
见有利就进攻,见不利想要退却,这就难了。越国人逆流而进,顺流而退; 
见有利就进攻,见不利想要退却,就能很快退却。越国人凭着这种水势,屡 
次打败楚国人。公输盘从鲁国南游到了楚国,于是开始制造船战用的武器, 
他造了钩、镶的设备,敌船后退就用钩钩住它,敌船进攻就用镶推拒它。计 
算钩与镶的长度,制造了合适的兵器。楚国人的兵器适用,越国人的兵器不 
适用。楚国人凭着这种优势,又屡次打败了越国人。公输盘夸赞他制造的钩、 
镶的灵巧,告诉墨子说:“我船战有自己制造的钩、镶,不知道您的义是不 
是也有钩、镶?”墨子回答说:“我义的钩、镶,胜过你船战的钩、镶。我 
以‘义’为钩、镶,以爱钩,以恭敬推拒。不用爱钩就不会亲,不用恭敬推 
拒就容易轻慢,轻慢不亲近就会很快离散。所以,互相爱,互相恭敬,如此 
互相利。现在你用钩来阻止别人,别人也会用钩来阻止你;你用镶来推拒人, 
人也会用镶来推拒你。互相钩,互相推拒,如此互相残害。所以,我义的钩、 
镶,胜过你船战的钩、镶。” 
公输盘削竹、木做成鹊,做成了就让它飞起来,三天不从天上落下来。 
公输盘自己认为很精巧。墨子对公输盘说:“你做的鹊,不如匠人做的车轴 
上的销子,一会儿削成一块三寸的木头,可以担当五十石重的东西。所以, 

平常所做的事,有利于人,可称作精巧;不利于人,就叫作拙劣了。” 
公输盘对墨子说:“我没有见到你的时候,我想得到宋国。自从我见了 
你之后,给我宋国,假如是不义的,我不会接受。”墨子说:“我没有见你 
的时候,你想得到宋国。自从我见了你之后,给你宋国,假如是不义的,你 
不会接受,这是我把宋国送给你了。你努力维护义,我又将送给你天下。” 

\chapter{三十八  公输}

公输盘为楚造云梯之械,成,将以攻宋。子墨子闻之,起于齐,行十日 
十夜而至于郢,见公输盘。 
公输盘曰:“夫子何命焉为?”子墨子曰:“北方有侮臣者,愿借子杀 
之。”公输盘不说。子墨子曰:“请献十金。”公输盘曰:“吾义固不杀人。” 
子墨子起,再拜曰:“请说之。吾从北方闻子为梯,将以攻宋。宋何罪之有? 
荆国有余于地,而不足于民,杀所不足,而争所有余,不可谓智。宋无罪而 
攻之,不可谓仁。知而不争,不可谓忠。争而不得,不可谓强。义不杀少而 
杀众,不可谓知类。”公输盘服。子墨子曰:“然,乎不已乎(2)?”公输盘 
曰:“不可,吾既已言之王矣。”子墨子曰:“胡不见我于王?”公输盘曰: 
“诺。” 
子墨子见王,曰:“今有人于此,舍其文轩(3),邻有敝舆,而欲窃之; 
舍其锦绣,邻有短褐,而欲窃之;舍其粱肉,邻有糠糟,而欲窃之。此为何 
若人?”王曰:“必为窃疾矣。”子墨子曰:“荆之地,方五千里,宋之地, 
方五百里,此犹文轩之与敝舆也;荆有云梦,犀兕麋鹿满之,江汉之鱼鳖鼋 
鼍为天下富,宋所为无雉兔狐狸者也,此犹粱肉之与糠糟也;荆有长松、文 
梓、楩、枬、楠、豫章,宋无长木,此犹锦绣之与短褐也。臣以三事之攻宋 
也,为与此同类。臣见大王之必伤义而不得。”王曰:“善哉!虽然,公输 
盘为我为云梯,必取宋。” 
于是见公输盘。子墨子解带为城,以牒为械,公输盘九设攻城之机变, 
子墨子九距之(4)。公输盘之攻械尽,子墨子之守圉有余(5)。公输盘诎(6), 
而曰:“吾知所以距子矣,吾不言。”子墨子亦曰:“吾知子之所以距我, 
吾不言。”楚王问其故,子墨子曰:“公输子之意,不过欲杀臣,杀臣,宋 
莫能守,可攻也。然臣之弟子禽滑厘等三百人,已持臣守圉之器,在宋城上 
而待楚寇矣。虽杀臣,不能绝也。”楚王曰:“善哉!吾请无攻宋矣。” 
子墨子归,过宋。天雨,庇其闾中,守闾者不内也(7)。故曰:“治于神 
者,众人不知其功;争于明者,众人知之。” 


[注释] 

(1)本篇记述公输盘制造云梯,准备帮助楚国进攻宋国,墨子从齐国起身,到楚国制止公输盘、 

楚王准备进攻宋国的故事。全文生动地表现了墨子“兼爱”、“非攻”的主张,从故事中,我们也可 

以看到墨子不辞辛苦维护正义的品格和机智、果敢的才能。(2)第一个“乎”为“胡”之误,胡:何。 

(3)文轩:彩车。(4)距:通“拒”。(5)圉:御。(6)诎:屈。(7)内:通“纳”。 
[白话] 
公输盘为楚国造了云梯那种器械,造成后,将用它攻打宋国。墨子听说 
了,就从齐国起身,行走了十天十夜才到楚国国都郢,会见公输盘。 
公输盘说:“您将对我有什么吩咐呢?”墨子说:“北方有一个欺侮我 
的人,愿借助你杀了他。”公输盘不高兴。墨子说:“我愿意献给你十镒黄 
金。”公输盘说:“我奉行义,决不杀人。” 
墨子站起来,再一次对公输盘行了拜礼,说:“请向你说说这义。我在 
北方听说你造云梯,将用它攻打宋国。宋国有什么罪呢?楚国有多余的土地, 
人口却不足。现在牺牲不足的人口,掠夺有余的土地,不能认为是智慧。宋 
国没有罪却攻打它,不能说是仁。知道这些,不去争辩,不能称作忠。争辩 

却没有结果,不能算是强。你奉行义,不去杀那一个人,却去杀害众多的百 
姓,不可说是明智之辈。”公输盘服了他的话。 
墨子又问他:“那么,为什么不取消进攻宋国这件事呢?”公输盘说: 
“不能。我已经对楚王说了。”墨子说:“为什么不向楚王引见我呢?”公 
输盘说:“行。” 
墨子见了楚王,说:“现在这里有一个人,舍弃他的华丽的丝织品,邻 
居有一件粗布的短衣,却打算去偷;舍弃他的美食佳肴,邻居只有糟糠,却 
打算去偷。这是怎么样的一个人呢?”楚王回答说:“这人一定患了偷窃病。” 
墨子说:“楚国的地方,方圆五千里;宋国的地方,方圆五百里,这就 
象彩车与破车相比。楚国有云梦大泽,犀、兕、麋鹿充满其中,长江、汉水 
中的鱼、鳖、鼋、鼍富甲天下;宋国却连野鸡、兔子、狐狸、都没有,这就 
象美食佳肴与糟糠相比。楚国有巨松、梓树、楠、樟等名贵木材;宋国连棵 
大树都没有,这就象华丽的丝织品与粗布短衣相比。从这三方面的事情看, 
我认为楚国进攻宋国,与有偷窃病的人同一种类型。我认为大王您如果这样 
做,一定会伤害了道义,却不能据有宋国。” 
楚王说:“好啊!即使这么说,公输盘已经给我造了云梯,一定要攻取 
宋国。” 
于是又叫来公输盘见面。墨子解下腰带,围作一座城的样子,用小木片 
作为守备的器械。公输盘九次陈设攻城用的机巧多变的器械,墨子九次抵拒 
了他的进攻。公输盘攻战用的器械用尽了,墨子的守御战术还有余。公输盘 
受挫了,却说:“我知道用什么办法对付你了,但我不说。”楚王问原因。 
墨子回答说:“公输盘的意思,不过是杀了我。杀了我,宋国没有人能防守 
了,就可以进攻。但是,我的弟子禽滑厘等二百人,已经手持我守御用的器 
械,在宋国的都城上等待楚国侵略军呢。即使杀了我,守御的人却是杀不尽 
的。”楚王说:“好啊!我不攻打宋国了。” 
墨子从楚国归来,经过宋国,天下着雨,他到闾门去避雨,守闾门的人 
却不接纳他。所以说:“运用神机的人,众人不知道他的功劳;而于明处争 
辩不休的人,众人却知道他。” 

\chapter{三十九  备城门}

禽滑厘问于子墨子曰:“由圣人之言,凤鸟之不出,诸侯畔殷周之国(2), 
甲兵方起于天下,大攻小,强执弱,吾欲守小国,为之奈何?”子墨子曰: 
“何攻之守?”禽滑厘对曰:“今之世常所以攻者:临、钩、冲、梯、堙、 
水、穴、突、空洞、蚁傅、轒辒、轩车,敢问守此十二者奈何?”子墨子曰: 
“我城池修,守器具,推粟足(3),上下相亲,又得四邻诸侯之救,此所以持 
也。且守者虽善(4),则犹若不可以守也。若君用之守者,又必能乎守者,不 
能而君用之,则犹若不可以守也。然则守者必善而君尊用之,然后可以守也。” 
凡守围城之法,厚以高;壕池深以广;楼撕揗(5),守备缮利;薪食足以支三 
月以上;人众以选;吏民和;大臣有功劳于上者,多主信以义,万民乐之无 
穷;不然,父母坟墓在焉;不然,山林草泽之饶足利;不然,地形之难攻而 
易守也;不然,则有深怨于适而有大功于上;不然,则赏明可信而罚严足畏 
也。此十四者具,则民亦不宜上矣,然后城可守。十四者无一,则虽善者不 
能守矣。 
故凡守城之法,备城门为县门,沉机长二丈,广八尺,为之两相如;门 
扇数合相接三寸。施土扇上,无过二寸。堑中深丈五,广比扇,堑长以力为 
度,堑之末为之县,可容一人所。客至,诸门户皆令凿而慕孔。孔之,各为 
二慕二,一凿而系绳,长四尺。城四面四隅皆为高磨■,使重室子居其上候 
适,视其态状与其进左右所移处,失候斩。 
适人为穴而来,我亟使穴师选本,迎而穴之,为之且内弩以应之(6)。 
民室杵木瓦石(7),可以盖城之备者(8),尽上之。不从令者斩。 
昔筑(9),七尺一居属,五步一垒(10)。五筑有锑。长斧,柄长八尺。十 
步一长镰,柄长八尺。十步一斗(11),长椎,柄长六尺,头长尺,斧其两端。 
三步一大铤(12),前长尺,蚤长五寸。两铤交之,置如平,不如平不利,兑 
其两末。 
穴队若冲队,必审如攻队之广狭,而令邪穿其穴,令其广必夷客队。 
疏束树木,令足以为柴抟,毋前面树,长丈七尺一,以为外面,以柴抟 
从横施之,外面以强涂,毋令土漏。令其广厚,能任三丈五尺之城以上,以 
柴木土稍杜之,以急为故。前面之长短,豫蚤接之,令能任涂,足以为堞, 
善涂其外,令毋可烧拔也。 
大城丈五为闺门,广四尺。为郭门,郭门在外,为衡,以两木当门,凿 
其木维敷上堞。为斩县梁,■穿断城,以板桥邪穿外,以板次之,倚杀如城 
报(13)。城内有傅壤(14),因以内壤为外(15)。凿其间,深丈五尺,室以樵, 
可烧之以待适。令耳属城,为再重楼,下凿城外堞,内深丈五,广丈二。楼 
若令耳,皆令有力者主敌,善射者主发,佐皆广矢(16)。 
治裾(17)。诸延堞高六尺,部广四尺,皆为兵弩简格。 
转射机(18),机长六尺,貍一尺。两材合而为之辒,辒长二尺,中凿夫 
之为道臂,臂长至桓。二十步一,令善射之者(19),佐一人,皆勿离。 
城上百步一楼,楼四植,植皆为通舄,下高丈,上九尺,广、丧各丈六 
尺(20),皆为宁。三十步一突,九尺,广十尺,高八尺,凿广三尺,表二尺, 
为宁。城上为攒火,夫长以城高下为度,置火其末。城上九尺一弩、一戟、 
一椎、一斧、一艾,皆积参石、蒺藜。 
渠长丈六尺,夫长丈二尺(21),臂长六尺,其貍者三尺,树渠毋傅堞五 

寸。藉莫长八尺,广七尺,其木也广五尺,中藉苴为之桥,索其端;适攻, 
令一人下上之,勿离。 
城上二十步一藉车,当队者不用此数。城上三十步一■灶(22)。 
持水者必以布麻斗、革盆,十步一。柄长八尺,斗大容二斗以上到三斗。 
敝裕、新布长六尺,中拙柄,长丈,十步一,必以大绳为箭(23)。城上十步 
一鈂。水缻,容三石以上,小大相杂。盆、蠡各二财。 
为卒干饭,人二斗,以备阴雨,面使积燥处(24)。令使守为城内堞外行 
餐。置器备,杀沙砾、铁(25)。皆为坯斗。令陶者为薄缻,大容一斗以上至 
二斗,即用取,三秘合束(26)。坚为斗城上隔。栈高丈二,剡其一末。为闺 
门,闺门两扇,令可以各自闭也。 
救闉池者,以火与争,鼓橐,冯埴外内,以柴为燔。灵丁,三丈一,火 
耳施之(27)。十步一人,居柴,内弩;弩半,为狗犀者环之。墙七步而一。 
救车火(28),为烟矢射火城门上,凿扇上为栈,涂之,持水麻斗、革盆救之。 
门扇薄植,皆凿半尺一寸,一涿弋,弋长二寸,见一寸相去七寸,厚涂之以 
备火。城门上所凿以救门火者,各一垂水,火三石以上(29),小大相杂。门 
植关必环锢,以锢金若铁鍱之。门关再重,鍱之以铁,必坚。梳关,关二尺, 
梳关一苋,封以守印,时令人行貌封,及视关入桓浅深。门者皆无得挟斧、 
斤、凿、锯、椎。 
城上二步一渠,渠立程(30),丈三尺,冠长十丈,辟长六尺。二步一荅, 
广九尺,袤十二尺。二步置连梃,长斧、长椎各一物;枪二十枚,周置二步 
中。二步一木弩,必射五十步以上。及多为矢,节毋以竹箭(31),楛、赵、 
■、榆,可。盖求齐铁夫(32),播以射■及栊枞。二步积石,石重千钧以上 
者,五百枚。毋百以亢,疾犁、壁皆可善方。二步积苙(33),大一围,长丈, 
二十枚。五步一罂,盛水。有奚,奚蠡大容一斗。五步积狗尸五百枚,狗尸 
长三尺,丧以弟(34),瓮其端,坚约弋。十步积抟,大二围以上,长八尺者 
二十枚。二十五步一灶,灶有铁鐕容石以上者一,戒以为汤。及持沙,毋下 
千石。三十步置坐候楼,楼出于堞四尺,广三尺,广四尺,板周三面,密傅 
之,夏盖其上。五十步一藉车,藉车必为铁纂。五十步一井屏,周垣之,高 
八尺。五十步一方,方尚必为关籥守之。五十步积薪,毋下三百石,善蒙涂, 
毋令外火能伤也。百步一栊枞,起地高五丈;三层,下广前面八尺,后十三 
尺,其上称议衰杀之。百步一木楼,楼广前面九尺,高七尺,楼■居■(35), 
出城十二尺。百步一井,井十瓮,以木为系连。水器容四斗到六斗者百。百 
步一积杂秆,大二围以上者五十枚,百步为橹,橹广四尺,高八尺,为冲术。 
百步为幽■,广三尺高四尺者千(36)。二百步一立楼,城中广二丈五尺二, 
长二丈,出枢五尺。城上广三步到四步,乃可以为使斗。俾倪广三尺,高二 
尺五寸。陛高二尺五,广长各三尺,远广各六尺(37)。城上四隅童异,高五 
尺,四尉舍焉。 
城上七尺一渠,长丈五尺,貍三尺,去堞五寸;夫长丈二尺,臂长六尺 
半植一凿,内后长五寸(38)。夫两凿,渠夫前端下堞四寸而适。凿渠、凿坎, 
覆以瓦,冬日以马夫寒,皆待命,若以瓦为坎。 
城上千步一表,长丈,弃水者操表摇之。五十步一厕,与下同圂。之厕 
者不得操。城上三十步一藉车,当队者不用。城上五十步一道陛,高二尺五 
寸,长十步。城上五十步一楼■,■勇勇必重(39)。土楼百步一,外门发楼, 
左右渠之。为楼加藉幕,栈上出之以救外。城上皆毋得有室,若也可依匿者, 

尽除去之。城下州道内百步一积薪,毋下三千石以上,善涂之。城上十人一 
什长,属一吏士(40)、一帛尉。 
百步一亭,高垣丈四尺,厚四尺,为闺门两扇,令各可以自闭,亭一尉, 
尉必取有重厚忠信可任事者。二舍共一井爨,灰、康、秕、杯、马矢,皆谨 
收藏之。 
城上之备:渠谵(41)、藉车、行栈、行楼、到(42)、颉皋、连梃、长斧、 
长椎、长兹、距、飞冲、县□、批屈。楼五十步一,堞下为爵穴,三尺而一 
为薪皋,二围,长四尺半,必有洁(43)。瓦石重二升以上(44),上城上。沙, 
五十步一积。灶置铁鐟焉,与沙同处。木大二围,长丈二尺以上,善耿其本, 
名曰长从,五十步三十。木桥长三丈,毋下五十。复使卒急为垒壁,以盖瓦 
复之。用瓦木罂,容十升以上者,五十步而十,盛水且用之。五十二者十步 
而二(45)。 
城下里中家人,各葆其左右前后,如城上。城小人众,葆离乡老弱国中 
及他大城。寇至,度必攻,主人先削城编,唯勿烧。寇在城下,时换吏卒署, 
而毋换其养,养毋得上城。寇在城下,收诸盆瓮耕,积之城下,百步一积, 
积五百。城门内不得有室,为周官桓吏(46),四尺为倪。行栈内闭,二关一 
堞。 
除城场外,去池百步,墙垣树木小大俱坏伐,除去之。寇所从来,若昵 
道、傒近(47)若城场,皆为扈楼,立竹箭天中(48)。 
守堂下为大楼,高临城,堂下周散道;中应客,客待见。时召三老在葆 
宫中者,与计事得先(49)。行德计谋合,乃入葆。葆入守,无行城,无离舍。 
诸守者审知卑城浅池,而错守焉。晨暮卒歌以为度,用人少易守。 
守法:五十步丈夫十人、丁女二十人、老小十人,计之五十步四十人。 
城下楼卒,率一步一人,二十步二十人。城小大以此率之,乃足以守圉。 
客冯面而蛾傅之(50),主人则先之知,主人利,客适(51)。客攻以遂, 
十万物之众,攻无过四队者,上术广五百步,中术三百步,下术五十步。诸 
不尽百五步者,主人利而客病。广五百步之队,丈夫千人,丁女子二知人, 
老小千人,凡四千人,而足以应之,此守术之数也。使老小不事者,守于城 
上不当术者。 
城持出必为明填(52),令吏民皆智知之(53)。从一人百人以上,持出不 
操填章,从人非其故人乃其稹章也(54),千人之将以上止之,勿令得行。行 
及吏卒从之,皆斩,具以闻于上。此守城之重禁之(55)。夫奸之所生也,不 
可不审也。 
城上为爵穴,下堞三尺,广其外,五步一。爵穴大容苴(56),高者六尺, 
下者三尺,疏数自适为之。 
塞外堑(57),去格七尺,为县梁。城■陕不可堑者勿堑。城上三十步一 
聋灶(58)。 
人擅苣,长五节。寇在城下,闻鼓音,燔苣,复鼓,内苣爵穴中,照外。 
诸藉车皆铁什。藉车之柱长丈七尺,其貍者四尺;夫长三丈以上至三丈 
五尺,马颊长二尺八寸,试藉车之力而为之困,失四分之三在上(59)。藉车, 
夫长三尺(60),四二三在上(61),马颊在三分中。马颊长二尺八寸,夫长二 
十四尺,以下不用。治困以大车轮。藉车桓长丈二尺半。诸藉车皆铁什,复 
车者在之(62)。 
寇■池来(63),为作水甬,深四尺,坚慕貍之(64)。十尺一,覆以瓦而 

待令。以木大围长二尺四分而早凿之(65),置炭火其中合慕之,而以藉车投 
之。 
为疾犁投,长二尺五寸,大二围以上。 
涿弋,弋长七寸,弋间六寸,剡其末。 
狗走,广七寸,长尺八寸,蚤长四寸,犬耳施之(66)。 
子墨子曰:“守城之法,必数城中之木,十人之所举为十挈,五人之所 
举为五挈,凡轻重以挈为人数。为薪樵挈,壮者有挈,弱者有挈,皆称其任。 
凡挈轻重所为,吏人各得其任(67)。”城中无食则为大杀。 
去城门五步大堑之,高地三丈(68),下地至(69),施贼其中(70),上为 
发梁,而机巧之,比传薪土(71),使可道行,旁有沟垒,毋可逾越,而出佻 
且比(72),适人遂入,引机发梁,适人可禽。适人恐惧而有疑心,因而离。 


[注释] 

(1)《备城门》是墨子研究城池攻防战术的主要篇章之一。春秋战国时期,由于诸侯割据分裂, 

互相兼并,战争频繁,给人口带来极大的杀伤。因此,研究战争中的战略战术是当时社会的迫切需要, 

特别是当时的中小国家,面临大国的进攻和威胁,首先考虑的便是如何利用高城深池,在劣势的情况 

下以抵御敌人的进攻,保护自己的城邑和国家。即使是大国,有时也面临同样的问题。《备城门》便 

是墨子针对当时实际需要而做出的研究成果之一。(2)畔:通“叛”。(3)“推”应作“樵”。(4)“且 

守者虽善”下应加“而君不用之”五字。(5)“撕揗”应作“■修”。(6)“且”应作“具”。(7)“杵” 

应作“材”。(8)“盖”应作“益”。(9)“昔”应作“皆”。(10)垒:“■”。(11)“斗”应作“斫”。 

(12)“铤”应作“ ”。(13)“报”应作“势”。(14)(15)“壤”应作“堞”。(16)“广”应作“厉”。 

(17)治裾,即“作薄”。(18)转射机,一种能旋转的机弩。(19)“者”后应加“主之”二字。(20)“丧” 

应为“袤”之误。(21)“夫”为“矢”字之误。(22)“■”应作“垄”。(23)此句末详。(24)“面” 

应作“而”。(25)杀:散。(26)三秘:即为“垒施”之误。(27)“火耳”应作“犬牙”。(28)“车” 

应作“熏”。(29)“火”应作“容”。(30)“程”应作“桯”。(31)“节”应作“即”。“竹箭”前 

“以”字应移至“楛”前。(32)“夫”应作“矢”。(33)“苙”应作“苣”。(34)“弟”应作“茅”。 

(35)“■”应作“■”。“■”应作“坫”。(36)“干”应作“十”。(37)“远”应作“道”。(38) 

“后”应作“经”,“长”字疑衍。(39)“■勇勇必”应作“楼撕必再”。(40)“一”当作“十”。 

(41)“谵”应作“襜”。(42)“到”应作“斫”。(43)“洁”应作“絜”。(44)“升”应作“斤”。 

(45)“十二”应作“斗以上”。(46)“官”应作“宫”。“桓”应作“植”。(47)“傒近“应作“近 

傒”。(48)“天”应作“水”。(49)“先”应作“失”。(50)“面”字衍。(51)“适”应作“病”。 

(52)“持”应作“将”。(53)“智”应作“习”。(54)“稹”应作“填”。(55)“之”应作“也”。 

(56)“苴”应作“苣”。(57)“塞”应作“穿”。(58)“聋”应作“垄”。(59)“失”应作“夫”。 

(60)“尺”应作“丈”。(61)“二”应作“之”。(62)复:当为“后”。(63)“■”应作“■”。(64) 

“慕”应作“幂”。(65)“分”应作“寸”。“早”应作“中”。(66)“耳”应作“牙”。(67)“吏” 

应作“使”。(68)“三”字疑衍;“丈”后应加“五尺”二字。(69)“下地至”后应加“泉三尺”三 

字。(70)“贼”应作“栈”。(71)“传”应作“傅”。(72)“比”应作“北”。 
[白话] 
禽滑厘问墨子说:“从圣人的说法来看,现在凤鸟没有出现,诸侯背叛 
王朝,天下兵争方起,大国攻打小国,强国攻打弱国。我想为小国防守,应 
怎么做呢?”墨子说:“防御什么方式的进攻呢?”禽滑厘回答说:“现在 
世上常用的进攻方法有:筑山临攻、钩梯爬城、冲车攻城、云梯攻城、填塞 
城沟、决水淹城、隧道攻城、穿突城墙、城墙打洞、如蚁一般密集爬城、使 
用蒙上牛皮的四轮车、使用高耸的轩车。请问防守这十二种攻城方式应怎么 

办?”墨子说:“我方城池修固,守城器具备好,柴禾粮草充足,上下相亲, 
又取得四邻诸侯的救助,这就是用来长久守御的条件。而且,守城的人虽有 
本事,而国君不信任他,那么仍然不可防守。如果国君用来防守的人,一定 
是有能力防守的人;如果他没有能力而国君信任他,也是不能防守的。既然 
如此,那么守城的人必须有能力,而国君又信任他,这才可以防守得住。” 
凡守城的方法共有:(城墙)厚而高,濠沟深而宽,修好望敌之楼,防 
守器械精良,粮食柴草足以支持三月以上,防守的人多而经过挑选,官吏和 
民众相互和睦,为国家建立功劳的大臣多,国君讲信义,万民安乐无穷。或 
者,父母的坟墓就在这里;或者,具备富饶的山林草泽;或者,地形难攻易 
守;或者,(守者)对敌人有深仇大恨而对君主有大功;或者,奖赏明确可 
信,惩罚严厉可怕。这十四个条件具备,那么民众就不会怀疑君主,这以后 
城池才可以守住。这十四者一个没有,那么即使防守的人善于防守也守不住。 
所以凡守城的方法:在城门上准备好悬门和左右悬门的机关。悬门长二 
丈,宽八尺,两扇相同,两扇间有三寸重叠衔接;门扇上涂上泥土,不要厚 
于二寸。濠沟有一丈五尺深,宽度相当于门扇的长度,长短以人才为度,濠 
沟边修一管理悬门的房间,大概可以容纳一人。敌兵到了,各门都叫人凿开 
两个洞;一个洞系上绳子,绳长四尺。城墙的四边和四角都建高高的望敌楼, 
使贵家子在楼上了望敌人,观察敌人的势态,进(退)及左右移动的地方。 
失职者处以斩刑。 
敌兵打隧道来进攻,我方立即派穴地之师确定穴地之处,迎头穴地以待, 
准备好短弩射击敌人。 
民家的木材瓦石,凡可用来增加城池守备的,全部上缴。不服从命令的 
处斩。 
准备各种筑城工具:每七尺一把锄头,五步一筐,五筑有一铁锄,一柄 
长八尺的斧头,十步一把长镰刀,柄长八尺。十步一斫,一长锥,柄长六尺, 
头长一尺,用斧削其两端。三步一短矛,长一尺,刀尖五寸。两矛尖交叉安 
上,放得很平,不平不方便,两头要锋利。 
用打隧道的方法来抵御敌方的隧道进攻,我方所打隧道必要恰好相当敌 
方隧道的宽狭,使它斜穿敌方隧道,使之可以填平敌方的隧道。 
把木柴捆扎起来,使之能成为一捆捆的柴抟,穿前面树连起来,长一丈 
七尺一,作为外面,把柴抟横放在内面,外面涂上粘土,不要让土脱漏。柴 
抟推积的宽度和厚度,要足以充当三丈五尺高的城墙的屏障,用柴抟、树木、 
泥土来加固城墙,越坚固越好。柴抟前面的长短,要预先弄好,以便涂上泥 
土,可充城堞之用,妥善涂好外面,使敌方无法烧掉或拔掉。 
大城,要在一丈五尺之外做闺门,宽四尺。做一郭门,在闺门之外。做 
好两根横木,以关闭郭门。横木上凿孔,穿上绳子,牵到城堞上。做好悬梁, 
用木板做成,让它从城坎处向外斜着伸出。悬梁的斜度,符合城墙的形势。 
城墙内修傅堞,作为外堞的辅助。在其中凿穴,深一丈五尺,放柴草于内, 
可以焚烧御敌。连着城墙修筑令耳,令耳是两重的楼房。在城墙外堞下凿穴, 
深一丈五尺,宽一丈二尺。城楼与令耳,都派有勇力的人负责杀敌,善于射 
箭的人放箭,辅佐的人勇敢善射。 
编造樊篱,与城堞相连,高六尺,各宽四尺,都设置兵弩弓箭,格杀敌 
人。 
转射机,机身长六尺,埋入土中一尺。用两根木头合为车辒,辒长二尺, 

在中间凿之为道,插入横臂,臂长至趺足。二十步放一机,令善射的人主之, 
派一人辅助,都不要离开。 
城上百步筑一楼,楼有四根柱子,柱子安在基石上。下面高一丈,上面 
高九尺,长宽都一丈六尺,都安上窗户。三十步一个突门,长九尺,宽十尺, 
高八尺,凿一窗,宽三尺,长二尺。城上设置火捽,火捽长短以城墙的高下 
为度,置火于末端。城上每九尺置一弩、一戟、一椎、一斧、一镰。各处都 
贮备礌石、蒺藜。 
(用以防守的)渠长一丈六尺,箭长一丈二尺,臂长六尺,埋在地下三 
尺,竖立渠柱不要附着城堞,要离开五寸。藉幕长八尺,宽七尺。它的木架 
宽五尺。在藉幕中部,设立一桥,桥端系上绳索,以便牵拉上下。敌方来攻, 
派一人上下牵拉,不得离开。 
城上隔二十步安置一藉车,当攻隧道时不按此数,城上每三十步设置一 
垄灶。持水的必须用布麻斗、皮盆,十步一件。斗柄长八尺,斗的大小可以 
容纳两斗到三斗水。旧布、新布长六尺;麻斗中间安上柄,长一丈,每十步 
放一件。必以粗大的绳子为箭。城上隔十步有一支鈂。水缸要能装三尺以上, 
大小相杂。盆、蠡各二具。 
做好士卒的干粮,每人二斗,以防备阴雨天,而使之积贮于干燥处。派 
遣士卒为守卫内外城堞的人送餐。设置器备,撒放沙砾、铁屑。各处都准备 
好粗制陶斗。使陶工做小罐,大小为装一斗至二斗水,用时即取,垒着捆在 
一起。坚固地做好斗城上的隔栈,高一丈二尺,削其一端。造好闺门,闺门 
由两扇组成,使之可以各自关闭。 
抢救敌方填濠沟,用火攻与之争夺,鼓动风箱,在墙内外堆着木柴,以 
之焚烧。隔三丈安一个灵丁,犬牙交错地安放。每十步有一人管理柴抟和弩 
箭,弩边用狗犀环绕。 
抢救薰火,若敌人用燃着烟火的箭射到城门上,我方要凿门扇,安上木 
栈,涂上泥,拿盛水的麻斗、皮盆救火。门扇上安木桩的地方都凿上一寸深 
的穴,每穴安一根木桩,木桩长两寸,有一寸露在外面。木桩间隔七寸,厚 
厚地涂上泥巴来防火。城门下凿下救火的地方,各备一缸水,装三石以上, 
大小相杂。城门的直木和横栓,一定要完好坚固,用坚韧的钢铁包裹着。门 
的横栓要上下两根,用铁包裹,必须坚固。门楗长两尺,锁一把,加上封条, 
盖上守印。经常派人察看封条的情况,并视察门楗插入的深浅。守门的人都 
不得挟带斧、凿、锯和椎子。 
城上每两步设立一渠柱。渠是立着的木柱。丈三尺,顶长十丈,臂长六 
尺。每两步设立一排竹答,宽九尺,长一丈二尺。每两步设立连梃,长斧、 
长椎各一件,枪二十支,在两步范围内分开放置。每两步设一弓弩,射程在 
五十步以上。多做些箭。如果没有竹箭,楛木、挑木、柘木、榆木也可以做 
箭杆。再求齐铁为箭头。弩箭分布城上,用以射敌人的冲梯和栊枞。每二步 
堆积石头,石头重达半钧以上的,共五十块。如果没有石头可用来抗击敌人, 
蒺藜、砖瓦也可用作好的防备工具。每二步堆积火炬,大一围,长一丈,共 
二十根。每五步一坛子盛水。坛子旁有葫芦瓢,葫芦瓢可盛一斗水。每五步 
堆积狗尸五百条。狗尸长三尺,用茅草覆盖,削其尖端,牢牢捆好。每十步 
堆积柴抟,大二围以上,长八尺,共二十捆。每二十五步修一座灶,灶上有 
铁甑一个,可盛水一石以上,准备着烧热水。还要储备沙石,不下一千石。 
每三十步共建一座候楼。楼伸出女墙四尺,宽三尺,下面宽四尺,三面围上 

木板,密密涂泥,夏天盖住上面。每五十步一个藉车,藉车必用铁作车轴。 
每五十步一座厕所,周围的围墙,高八尺。每五十步一房,房上必须安置门 
柱和铁锁,以便住守。每五十步堆积柴木,不下于三百尺,好好用泥土封盖, 
使城外放的火不能燃烧它。每百步建一木楼,楼宽前面九尺,高七尺,楼窗 
安在城墙上,伸出城墙外十二尺。每百步挖一口井,每井安排十瓮,用木制 
造提水的桔槔。贮水器可容纳四斗到六斗水,共一百个。每百步堆积一堆禾 
秆,大于二围以上的五十捆。每百步树立一块木盾牌,宽四尺,高八尺。做 
好冲锋的大路。每百步要开暗沟,宽三尺高四尺的十条。每两百步建一座立 
楼,宽二丈五尺,其中五尺伸到女墙外。城墙上宽三步到四步,才可以使士 
兵活动战斗。俾倪(之墙)宽三尺,高二尺五寸。阶陛高二尺五,宽广各三 
尺,路宽六尺。城上四角为重楼,高五尺,四个尉官驻扎于此。 
城上每七尺建一渠柱,长一丈五尺,埋在地下三尺,离开城堞五寸;露 
在外者长一丈二尺,臂长六尺。在中部凿一孔,内径长五寸。外露部分凿两 
孔,渠柱顶端比女墙低四寸为宜。凿渠、凿坎,以瓦覆盖,冬天以马草覆盖。 
都待命而行,或以瓦为坎。 
城上每千步立一表,长一丈。要向城下颂倒废水的人,拿表摇动。每五 
十步一厕所,与城下的厕所同一粪坑。上厕所的人不准手拿武器。城上三十 
步一藉车,当攻打隧道时不按此数。城上五十步一道台阶,高二尺五寸,长 
十步。城上五十步一楼,楼必多层。每百步一座土楼,外面安上悬门,左右 
开渠。建楼加上藉幕,有栈道出城以救外面。城墙上都不能盖房屋,或其他 
可隐匿的处所,若有则必须全部拆除。城下道路每百步堆积柴薪,不少于三 
千石以上,用泥土好好涂上。城上每十人任命一名什长,管理十名士卒。 
每百步一座亭。墙高一丈四尺,厚四尺,做两扇闺门,使两扇门可以各 
自开关。每亭一尉,尉必须选稳重忠信能胜任的人担任。两舍共一井灶。灰、 
糠、秕谷、谷皮、马尿都要小心收藏。 
城上的守备工具:渠答、籍车、行栈、行楼、斫、桔槔、连梃、长斧、 
长椎、长锄、钩钜、飞冲、悬(梁)、批屈。楼五十步一座,城堞下挖掘“爵 
穴”。每三尺设立一个桔槔,大二围,长四尺半,必须有挈。瓦石重二斤以 
上,搬上城。沙土,五十步一堆。灶上安放铁甑,和沙堆放在一起。木大二 
围,长一丈二尺以上。把它的底部好好连在一起,叫作“长从”。每五十步 
放三十个。木桥长三丈,不下五十个。再派士卒急造垒壁,以瓦覆盖起来。 
用陶制或木制的坛子,能装十升以上的,每十步放十个,盛水时将使用它们。 
能盛水五斗的十步放两个。 
城墙下里巷中的人家,各保卫其左右前后,像城上一样。如果城小人众, 
就保护老弱离乡到国中的其他大城去。敌人来了,估计他们必定进攻,主方 
必定先拆除城外附属物,只是不烧毁。敌人在城下,我方应不时更换吏卒防 
守,但不要更换给养人员,给养人员不能上城。敌方在城下,我方收集盆、 
罐,堆在城下,百步一堆,堆五百堆。城门内不可有房子,只筑周宫派吏驻 
守。四尺为倪,行栈内閈,二关一堞。 
清除城外离开护城河百步内的墙垣,大小树木都伐毁,除掉。敌人从来 
之处,如便道、近道,或城场,都修建扈楼,并在水中插上竹箭。守官堂下 
造大楼,高可临视全城。堂下四往有路。在堂中应客,客等待接见。不时召 
见在有保护的室中的三老,与之计议事之得失。行事有得,计谋相合,就回 
入保护之屋。保者入屋,不要逃城,不要离开房屋。各个担负守卫的人要详 

知卑城浅池,而措意防守。早晚士卒歌以为度,用人少有变换。 
守卫之法:每五十步男子十人,成年女子二十人,老小十人,共计五十 
步四十人。城下守楼士卒,一步一人,计二十步二十人。按城的大小以此为 
标准,才足以守御。 
如果敌人附城如蛾进攻,主人预先知道,主人有利,进攻者不利。如果 
敌方以队进攻,十万之众,进攻不会超过四队,最大的排路五百步,中等三 
百步,下等五十步。各种不到百五十步宽的,主人有利而客方不利。防御宽 
五百步的队伍,需男子一千人,成年女子二千人,老小千人,共四千人,就 
足以应付,这是防守道路之数。使老小不能任事,在城上不当路的地方防守。 
城中将军出城,必须持“明填”,要使官民都了解“明填”。将军率一 
百人以上出城,不带“明填”,或不是由本人持有“明填”,千夫长以上的 
官可以制止他,不让他通行。如果出行或吏卒放纵他出行,都要杀头,要把 
具体情况报告给上级。这是守城的重大禁令。奸细往往出在这里,不能不详 
细考察。 
城上建爵穴,在城堞下三尺之处,它的外口要大,每五步建一穴。爵穴 
的大小能放得下火炬,高的有六尺,低的有三尺,它的密度可视情况而定。 
于城外挖壕沟,离“杜格”(栅栏)七尺远,沟上作吊桥。城外狭窄不 
能作壕沟处可以不挖。 
城上每三十步建一个垄灶。 
守城的人都备有火炬,有五个竹节长。敌人到了城下,听到鼓声后,点 
燃火炬;再次听到鼓声后,将火炬放入爵穴中,照亮城外。 
各种藉车皆用铁。藉车的柱子长一丈七尺,埋于地下部分长四尺;车座 
长三丈至三丈五尺,马颊长二尺八寸,根据所测试的籍车的力度而制作车困, 
车座四分之三在地面上。籍车,车座长三丈,四分之三在地面上,马颊在地 
面以上部分正中间。马颊长二尺八寸,车座长二丈四尺,更短的不用。用大 
车轮作车困。藉车的车桓长一丈二尺半。各种藉车都用铁包裹,后面的车辅 
助它。 
敌人填塞护城河来进攻,我方就制作水甬,深为四尺,封固,埋于地下。 
每十尺一个,盖上瓦待命而用。用围长二尺四寸的木头,凿空中间,把炭火 
放进去再封上,然后用藉车投向敌军。制“疾犁投”,长为二尺五寸,粗两 
围以上。门上钉小木桩,长七寸,木桩间距为六寸,末端削尖。“狗走”宽 
为七寸,长一尺八寸,钩长四寸,犬牙交错地安设。 
墨子说:“守城的方法,一定要计算城中的木头,十人所能举起的便是 
十挈,五人所能举起的便是五挈,挈的轻重与守城人数相符。把木柴捆成挈, 
强壮的人用重挈,弱小的人用轻挈,与他们的力量相称。总之,挈的轻重要 
使每个人各自能胜任。”城中缺乏粮食就要大大减轻挈的重量。在离城门五 
步远的地方挖掘大壕沟,地势高的地方挖一丈五尺深,地势低的地方挖到有 
地下水之后再向下挖三尺即止。在壕沟上架设栈板,栈板上设置悬梁,装置 
机关,栈板表面铺上草木泥土,使人可以行走,两旁有沟墙不能翻越。然后 
派兵出城挑战,并假装战败逃回,引诱敌人走的栈板,开动悬梁的机关,敌 
人便可以擒到。若敌人恐惧生疑,就会因此撤离。 

\chapter{四十  备高临}

禽子再拜再拜曰:“敢问适人积土为高,以临吾城,薪土俱上,以为羊 
黔,蒙橹俱前,遂属之城,兵弩俱上,为之奈何?” 
子墨子曰:子问羊黔之守邪?羊黔者,将之拙者也,足以劳卒,不足以 
害城。守为台城,以临羊黔,左右出巨,各二十尺,行城三十尺,强弩之(2), 
技机藉之,奇器口口之,然则羊黔之攻败矣。 
备临以连弩之车,材大方一方一尺(3),长称城之薄厚。两轴三轮,轮居 
筐中,重下上筐。左右旁二植,左右有衡植,衡植左右皆圜内,内径四寸。 
左右缚弩皆于植(4),以弦钩弦,至于大弦。弩臂前后与筐齐,筐高八尺,弩 
轴去下筐三尺五寸。连弩机郭同铜(5),一石三十钧。引弦鹿长奴(6)。筐大 
三围半,左右有钩距,方三寸,轮厚尺二寸,钩距臂博尺四寸,厚七寸,长 
六尺。横臂齐筐外,蚤尺五寸,有距,搏六寸,厚三寸,长如筐有仪,有诎 
胜,可上下,为武重一石,以材大围五寸。矢长十尺,以绳□□矢端,如如 
戈射(7),以磨■卷收(8)。矢高弩臂三尺,用弩无数,出人六十枚(9),用小 
矢无留。十人主此车。 
遂具寇,为高楼以射道(10),城上以荅罗矢。 


[注释] 

(1)《备高临》是墨子研究城池防守战术的篇章之一。主要阐述如何对付敌人采用居高临下攻城 

方法的战术。(2)“强弩”后应加“射”字。(3)“尺”字前疑重“方一”二字。(4)“缚”字应作“缚”。 

(5)“同”字应作“用”。(6)“长奴”应作“卢收”。(7)“戈”应作“弋”。(8)“磨”应作“磿”。 

(9)“人”应作“入”。(10)“道”应作“适”。 
[白话] 
禽滑厘一再谦拜后说:“请问:如果敌人堆积土古筑成高台,对我城造 
成居高临下之势,木头土石一齐上,构筑成名叫羊黔的土山,兵士以大盾牌 
做掩护从高台土山上一齐攻来,一下子就接近了我方的城头,刀箭齐用,该 
怎么对付呢?” 
墨子先生回答说:你问的是对付“羊黔”进攻的防守办法么?“羊黔” 
这种攻城办法,是带兵打仗者的蠢办法,只会将自己的士兵弄得疲劳不堪, 
不足以给守城一方造成危胁。守城的一方只要在城头上继续加高做所谓“台 
城”,依然对羊黔保持居高临下之势,台城左右用大木编连起来,两旁各横 
出二十尺。这种临时做成的台城又叫行城,高度为三十尺。在上面用强劲的 
弓箭射击敌人,凭借“技机”和精妙的武器对付敌人,这样一来,用羊黔之 
法进攻就失败了。 
对付筑台居高临下的进攻,还可以使用一种连弩车。造这种车的木材, 
要大小一尺见方,长度与城墙厚度相等。两根车轴,三个轮子,轮子装在车 
箱当中,车箱上下两个,左右各做两根立柱,还有横梁两根,横梁的左右两 
头都是圆榫头,榫头直径四寸,把有把的箭都捆在左右两边的柱子上,弓弦 
相钩,连到大弦上。弓把前后与车箱齐平,车箱高度为八尺,弓轴距下面的 
车箱三尺五寸。连弩的“机括”用铜做成,重一百五十斤。用辘轳收引弓弦。 
车箱周长为三围半,左右两边装有“钩距”,“钩距”三寸见方,车轮厚一 
尺二寸,钩距臂宽一尺四寸,厚七寸,长六尺。横臂与车箱外缘齐平,臂端 
一尺五寸的地方装有叫做“距”的横柄,柄宽六寸,厚三寸,长度与车箱相 

应。还装有一种瞄准仪,有出入时可以上下伸缩调整。再用大小一围五寸的 
木料做一个弩床,床重一百二十斤。箭长十尺,用绳子栓住箭尾,就象用细 
丝绳系住射空中飞鸟用的箭一样,以便将箭收回,不过这里是用辘轳卷收。 
箭高出弩臂三尺,用箭没有固定数,但至少要保证出入有六十枚,小箭就不 
必收回了。象这样的连弩车,十人掌管使用一辆。 
为了成功地抵距敌人的进攻,筑了高楼射击敌人,还得在城上用草编成 
厚厚的遮掩物来遮挡和收取敌方射来的箭。 

\chapter{四十一  备梯}

禽滑厘子事子墨子三年,手足胼胝,面目黧黑,役身给使,不敢问欲。 
子墨子其哀之(2),乃管酒块脯(3),寄于大山,昧葇坐之(4),以樵禽子。禽 
子再拜而叹。 
子墨子曰:“亦何欲乎?”禽子再拜再拜曰:“敢问守道?” 
子墨子曰:“姑亡,姑亡。古有其术者,内不亲民,外不约治,以少间 
众,以弱轻强,身死国亡,为天下笑。子其慎之,恐为身姜。” 
禽子再拜顿首,愿遂问守道。曰:“敢问客众而勇,烟资吾池(5),军卒 
并进,云梯既施,攻备已具,武士又多,争上吾城,为之奈何?” 
子墨子曰:问云梯之守邪?云梯者重器也,其动移甚难。守为行城,杂 
楼相见,以环其中。以适广陕为度,环中藉幕,毋广其处。行城之法,高城 
二十尺,上加堞,广十尺,左右出巨各二十尺,高、广如行城之法。 
为爵穴、煇鼠,施荅其外,机、冲、钱(6)、城,广与队等,杂其间以镌 
剑,持冲十人,执剑五人,皆以有力者。令案目者视适,以鼓发之,夹而射 
之,重而射(7),披机藉之(8),城上繁下矢、石、沙、炭以雨之(9),薪火、 
水汤以济之,审赏行罚,以静为故,从之以急,毋使生虑。若此,则云梯之 
攻败矣。 
守为行堞,堞高六尺而一等,施剑其面,以机发之,冲至则去之。不至 
则施之。 
爵穴,三尺而一。 
蒺藜投必遂而立,以车推引之。 
裾城外(10),去城十尺,裾厚十尺。伐裾(11),小大尽本断之,以十尺 
为传(12),离而深埋之,坚筑,毋使可拔。 
二十步一杀,杀有一鬲,鬲厚十尺。杀有两门,门广五尺。裾门一,施 
浅埋,弗筑,令易拔。城希裾门而直桀(13)。 
县火,四尺一钩樴。五步一灶,灶门有炉炭。令适人尽入,■火烧门, 
县火次之。出载而立,其广终队。两载之间一火,皆立而待鼓而然火,即具 
发之。适人除火而复攻,县火复下,适人甚病,故引兵而去,则令我死士左 
右出穴门击遗师,令贲士、主将皆听城鼓之音而出,又听城鼓之音而入。因 
素出兵施伏,夜半城上四面鼓噪,适人必或,有此必破军杀将。以白衣为服, 
以号相得,若此,则云梯之攻败矣。 


[注释] 

(1)《备梯》是墨子研究城池防守战术的篇章之一。主要讲如何对付敌人以云梯攻城的战术方法。 

在此篇中,墨子指出守城的战术方法固然重要,但更重要的还是外交战略。(2)“其”应作“甚”。(3) 

“块”应作“槐”。(4)“■”应作“茅”。(5)“烟资”应作“堙茨”。(6)“钱”应作“栈”。(7) 

“射”字后疑漏一“之”字。(8)“披”应作“技”。(9)“炭”应作“灰”。(10)“裾城外”三字前 

疑漏一“置”字。(11)“伐裾”后疑漏“之法”。(12)“传”应作“断”。(13)“城”后漏一“上” 

字。 
[白话] 
禽滑厘事奉墨子三年,手脚都起了老茧,脸晒得黑黑的,干仆役的活听 
墨子使唤,却不敢问自己想要问的事。墨子先生对此感到十分怜悯,于是备 
了酒和干肉,来到泰山,垫些茅草坐在上面,用酒菜酬劳禽滑厘。禽子行了 

再拜礼之后,叹了口气。 
墨子问他:“你有什么要问的吗?”禽滑厘又行了两次再拜礼,说道: 
“请问守城的方法。” 
墨子回答说:“先不要问,先不要问。古代也曾有懂得守城方法的人, 
但对内不亲抚百姓,对外不缔结和平,自己兵力少却疏远兵力多的国家,自 
己力量弱却轻视强大的国家,结果送命亡国,被天下人耻笑。你对此可要慎 
重对待,弄不好,懂得了守城的办法反为身累。” 
禽滑厘行再拜礼后又伏地叩头行稽首礼,希望能弄清防守的办法,说: 
“我还是冒昧地问问,如果攻城一方兵士众多又勇敢,堵塞了我方护城河, 
军士一齐进攻,攻城的云梯架起来了,进攻的武器已安排好,勇敢的士兵蜂 
涌而至,争先恐后爬上我方城墙,该如何对付呢?” 
墨子回答说:你问的是对付云梯的防卫办法吗?云梯是笨重的攻城器 
械,移动十分困难。守城一方可以在城墙上筑起“行城”和“杂楼”,将自 
己环绕起来。行城和杂楼之间要保持适当的距离,两者之间的部分要拉上防 
护用的遮幕,因此距度不宜过宽。筑行城的方法是:行城高出原城墙二十尺, 
上面加上锯齿状的矫墙,这种矫墙称作“堞”,宽十尺,左右两边所编大木 
横出各二十尺,高度和宽度与行城相应。 
矫墙下部开名叫“爵穴”、“■鼠”的小孔,孔外用东西遮挡起来。供 
投掷的技机,抵挡冲撞的冲撞车,供出外救援用的行栈,临时用的行城等器 
械,其排列的宽度应与敌人进攻的广度相等。各器械之间挟进持镌和持剑的 
士兵,其中掌冲车的十人,拿剑的五人,都应挑选力大的军士担任。用视力 
最好的兵士观察敌人,用鼓声发出抗击号令,或两边向敌人夹射,或重点集 
射一处,或借助技机向敌人掷械,从城上雨点般地将箭、砂石、灰土倾泄给 
城下之敌,加上往下投掷火把、倾倒滚烫的开水,同时赏罚严明,处事镇静, 
但又要当机立断,不致发生其他变故。象这样防守,云梯攻法就得被打败了。 
守城一方在“行城”上筑起临时用的矫墙“堞”,一律高六尺,在墙外 
安装剑,用机械发射,敌方的冲撞器来了就撤发射机,没来就使用它。 
矫墙下部开的名叫“爵穴”的小洞,每三尺一个。“蒺藜投”一定要针 
对敌方进攻的范围摆放,用车推下城墙然后又用车再拉上来,以便反复使用。 
在城外十尺远的地方安置断树,这称之为“置裾”。裾的厚度为十尺。 
采伐断树“裾”的方法是,无论大小,一律连根拔起,锯成十尺一段,间隔 
一段距离深埋于地中,一定要埋牢实,不能让它被拔出来。 
城墙上每隔二十步设置一个“杀”,备有一个储放投掷物的“鬲”,鬲 
厚十尺。“杀”安有两个门,门宽五尺。裾也可设有门,不过要浅埋才成, 
不要埋牢实,要让它能容易被拔出来。城上对着裾门的地方放置“桀石”, 
以供投掷。 
城上悬挂有火具,叫悬火,每隔四尺设置一个挂火具的钩樴。五步设一 
口灶,灶门备有炉炭。让敌人全部进入就放火烧门,接着投掷悬火。排出的 
作战器具,根据敌人的进攻范围相应摆放。两个作战器械之间设置一个悬火, 
由一个兵士掌执,等待出去的鼓声。鼓声一响就点悬火,敌人接近随即投放。 
敌人如将悬火打灭,就再次投放不绝。如此反复多次,敌人必定疲惫不堪, 
因此就会领兵而去。敌人一旦退出,就命令敢死队从左右出穴门追击溃逃之 
敌,但勇士和主将务必依照城上的鼓声从城内出去或退入城内。再趁着反击 
时布置埋伏,半夜三更时城上四面击鼓呐喊,敌人必定惊疑失措,伏兵乘机 

能攻破敌军军营,擒杀敌军首领。不过要用白衣做军服,凭口令相互联络。 
如此一来,用云梯的攻城方法就失败了。 

\chapter{四十二  备水}

城内堑外周道,广八步。备水谨度四旁高下。城地中徧下(2),令耳其内 
(3),及下地,地深穿之,令漏泉。置则瓦井中,视外水深丈以上,凿城内水 
耳(4)。 
并船以为十临,临三十人,人擅弩,计四有方,必善以船为轒辒。二十 
船为一队,选材士有力者三十人共船,其二十人擅有方,剑甲鞮瞀,十人人 
擅苗。 
先养材士,为异舍食其父母妻子以为质,视水可决,以临轒辒,决外堤, 
城上为射■,疾佐之(5)。 


[注释] 

(1)《备水》是墨子研究城池防守战术的篇章之一。主要讲如何防备敌人以水攻城的战术方法。 

(2)“地中”应作“中地”;“徧”应作“偏”。(3)“耳”应作“巨”,即“渠”之省。(4)同(3)。 

(5)“■”应作“机”。 
[白话] 
城内壕堑外设周道,宽八步。防备敌人以水灌城,必须要仔细地审视四 
周的地势情况。城中地势低的地方,要下令开挖渠道,至于地势更低的地方, 
则命令深挖成井,使其能互相贯通,以便引水泄漏。在井中置放“则瓦”, 
测量水位高低。如发现城外水深已有一丈以上,就开凿城内的水渠。 
每两只船连在一起为“一临”,将船共组成“十临”,每一临备三十人, 
人人都擅长射箭,每十人中四个还须带有锄头。必须善于用这种船作冲毁敌 
方堤防的“轒辒”(撞车)。每二十只船为一队,挑选勇武力大的兵士三十 
人共一条联合船,其中二十人每人备有一把锄头,穿戴盔甲皮靴,其余十人 
手拿长矛,人人擅使。 
当然预先供养勇武之人,另供给房子,安排供养他们的父母、妻子儿女, 
作为人质。发现可以决开水堤时,用两只船并联组成的“轒辒”冲决外堤, 
同时城上赶紧用射击机向敌人放箭,以掩护决堤的船队。 

\chapter{四十三  备突}

城百步一突门,突门各为窑灶,窦入门四五尺,为其门上瓦屋,毋令水 
潦能入门中。吏主塞突门,用车两轮,以木束之,涂其上,维置突门内,使 
度门广狭,令之入门中四五尺。置窑灶,门旁为橐,充灶伏柴艾(2),寇即入, 
下轮而塞之,鼓橐而熏之。 


[注释] 

(1)《备突》是墨子研究城池防守战术的篇章之一。主要讲如何防备敌人从城墙“突门”攻入的 

战术方法。(2)“伏”应作“状”。 
[白话] 
城墙内每百步设置一个“突门”,各个“突门”内都砌有一个瓦窑形的 
灶。灶砌在门内四五尺处。突门上装盖瓦可让雨水流入门内,安排一军吏掌 
管堵塞突门,方法是:用木头捆住两个车轮,上面涂上泥巴,用绳索将其悬 
挂在突门内,根据门的宽窄,使车轮挂在门中四五尺处。设置窑灶,门旁再 
安装上皮风箱,灶中堆满柴禾艾叶。敌人攻进来时,就放下车轮堵塞住,点 
燃灶里的柴火,鼓动风箱,烟薰火烤来犯之敌。 

\chapter{四十四  备穴}

禽子再拜再拜曰:“敢问古人有善攻者,穴土而入,缚柱施火,以坏吾 
城,城坏,或中人为之奈何(2)?” 
子墨子曰:问穴土之守邪?备穴者城内为高楼,以谨候望适人。适人为 
变筑垣聚土非常者,若彭有水浊非常者,此穴土也。急堑城内,穴其土直之。 
穿井城内,五步一井,傅城足。高地,丈五尺,下地,得泉三尺而止。令陶 
者为罂,容四十斗以上,固顺之以薄■革(3),置井中,使聪耳者伏罂而听之, 
审知穴之所在,凿穴迎之。 
令陶者为月明(4),长二尺五寸六围,中判之,合而施之穴中,偃一,覆 
一。柱之外善周涂,其傅柱者勿烧。柱者勿烧。柱善涂其窦际,勿令泄。两 
旁皆如此,与穴俱前。下迫地,置康若灰其中,勿满。灰康长五窦(5),左右 
俱杂,相如也。穴内口为灶令如窑,令容七八员艾,左右窦皆如此,灶用四 
橐。穴且遇,以颉皋冲之,疾鼓橐熏之,必令明习橐事者,勿令离灶口。连 
版,以穴高下,广陕为度,令穴者与版俱前,凿其版令容矛,参分其疏数, 
令可以救窦。穴则遇,以版当之,以矛救窦,勿令塞窦;窦则塞,引版而却, 
过一窦而塞之,凿其窦,通其烟,烟通,疾鼓橐以熏之。从穴内听穴之左右, 
急绝其前,勿令得行。若集客穴,塞之以柴,涂,令无可烧版也。然则穴土 
之攻败矣。 
寇至吾城,急非常也,谨备穴。穴疑有,应寇,急穴。穴未得,慎毋追。 
凡杀以穴攻者,二十步一置穴,穴高十尺,凿十尺,凿如前,步下三尺, 
十步拥穴,左右横行,高广各十尺。 
杀,俚两罂,深平城,置板其上,■板以井听。五步一密,用■若松为 
穴户,户穴有两蒺藜(6),皆长极其户,户为环,垒石外■(7),高七尺,加 
堞其上。勿为陛与石,以县陛上下出入,具炉橐,橐以牛皮,炉有两缻,以 
桥鼓之,百十(8)每亦熏四十什(9),然炭杜之,满炉而盖之,毋令气出。适 
人疾近五百穴(10),穴高若下,不至吾穴,即以伯凿而求通之(11)。穴中与 
适人遇,则皆圉而毋逐,且战北,以须炉火之然也,即去而入壅穴。 
杀,有鼠■(12),为之户及关籥独顺(13),得往来行其中。穴垒之中各 
一狗,狗吠即有人也。 
斩艾与柴长尺,乃置窑灶中,先垒窑壁,迎穴为连(14)。凿井傅城足, 
三丈一,视外之广陕而为凿井,慎勿失。城卑穴高从穴难。凿井城上(15), 
为三四井,内新■井中(16),伏而听之。审之知穴之所在(17),穴而迎之。 
穴且遇,为颉皋,必以坚材为夫,以利斧施之,命有力者三人用颉皋冲之, 
灌以不洁十余石。趣伏此井中,置艾其上,七分(18),盆盖井口,毋令烟上 
泄,旁其橐口(19),疾鼓之。 
以车轮辒(20)。束樵,染麻索涂中以束之。铁锁,县正当寇穴口。铁锁 
长三丈,端环,一端钩。 
■穴高七尺五寸,广柱间也尺(21),二尺一柱,柱下傅舄,二柱共一员 
十一(22)。两柱同质,横员士(23)。柱大二围半,必固其员士(24),无柱与 
柱交者。 
穴二窑,皆为穴月屋(25),为置吏、舍人各一人,必置水。塞穴门,以 
车两走为蒀,涂其上,以穴高下广陕为度,令人穴中四五尺,维置之。当穴 
者客争伏门,转而塞之。为窑容三员艾者,令其突入伏(26)。伏傅突一旁, 

以二橐守之勿离。穴矛以铁,长四尺半,大如铁服说,即刃之二矛。内去窦 
尺(27),邪凿之,上穴当心,其矛长七尺。穴中为环利率,穴二。 
凿井城上(28),俟其身井且通(29),居版上,而凿其一偏,已而移版, 
凿一徧。颉皋为两夫,而旁貍其植,而数钩其两端(30)。诸作穴者五十人, 
男女相半。五十人。攻内为传士之口(31),受六参,约枲绳以牛其下(32), 
可提而与投(33)。已则穴七人守退垒之中,为大庑一,藏穴具其中。难穴, 
取城外池唇木月散之什(34),斩其穴(35),深到泉,难近穴,为铁 ,金与 
扶林长四尺,财自足。客即穴,亦穴而应之。 
为铁钩钜长四尺者,财自足,穴彻,以钩客穴者。为短矛、短驽、 矢, 
财自足,穴彻以斗。以金剑为难,长五尺,为銎、木杘;杘有虑枚,以左客 
穴。 
戒持罂,容三十斗以上,貍穴中,丈一(36),以听穴者声。 
为穴,高八尺,广(37),善为傅置。具全、牛交槀皮及■(38),卫穴二, 
盖陈靃及艾(39),穴彻熏之(40)。 
斧金为斫(41),杘长三尺,卫穴四。为垒,卫穴四十,属四。为斤、斧、 
锯、凿、■(42),财自足。为铁校,卫穴四。 
为中橹,高十丈半(43),广四尺。为横穴八橹盖(44)。具槀枲,财自足, 
以烛穴中。 
盖持■(45),客即熏,以救目。救目分方■(46)穴,以益盛■置穴中(47), 
文盆毋少四斗(48)。即熏,以自临上及以沺目(49)。” 


[注释] 

(1)《备穴》是墨子研究城池防守战术的重要篇章之一。主要讲述如何防备敌人用打隧道来攻城 

的战术方法。(2)“或”应作“城”。(3)“顺”应作“幎”。(4)“月明”应作“瓦窦”。(5)“五” 

应作“亘”。(6)“穴”应作“内”。(7)“■”应作“埻”。(8)“百”字前漏一“重”字;“十”应 

作“斤”。(9)“每亦熏”应作“毋下重”;“什”应作“斤”。(10)“五百”应作“吾”。(11)“伯” 

应作“倚”。(12)“■”应作“窜”。(13)“独顺”应作“绳幎”。(14)“连”字后疑漏一“版”字。 

(15)“上”应作“下”。(16)“■”应作“甀”。(17)“审”后疑漏一“之”字。(18)“分”应作“八 

员”。(19)“其”应作“立”。(20)“辒”前疑漏一“为”字。(21)“也”应作“七”。(22)“员十 

一”应作“负土”。(23)“员士”应作“负土”。(24)同(23)。(25)“月屋”应作“门上瓦屋”。(26) 

“伏”后疑掉一“尺”字。(27)“内”应作“穴”。(28)“上”应作“下”。(29)“身”应作“穿”。 

(30)“数”应作“敷”。(31)“内”应作“穴”;“士”应作“土”;“口”应作“具”。(32)“牛” 

应作“绊”。(33)“与”应作“举”。(34)“月”应作“瓦”;“什”应作“外”。(35)“穴”应作 

“内”。(36)“丈一”前疑掉一“三”字。(37)“广”字后漏了“八尺”二字。(38)“全”应作“炉”; 

“交”应作“皮”;“槀”应作“橐”;“皮及■”应作“及瓦缶”。(39)“盖”应作“僧”。(40) 

“熏”前疑漏一“以”字。(41)“金”前疑漏一“以”字。(42)“鑺”应作“”。(43)“丈”应作 

“尺”。(44)“八”应作“大”;“盖”应作“蒀”。(45)“盖”应作“益”;“■”应作“醯”。 

(46)“■”应作“凿”。(47)“益”应作“盆”;“■”应作“醯”。(48)“文”应作“大”。(49) 

“自”应作“目”;“■”应作“醯”;“沺”应作“酒”。 
[白话] 
禽滑厘行了两次再拜礼之后说:“请问古代有善于攻城的人,挖地下隧 
道到城墙下,绕隧道里的支柱放火,隧道塌顶,以这种方法塌毁城墙,城墙 
毁坏,城中人该如何对付呢?” 
墨子回答说:你问的是对付用打隧道来攻城的防守方法吗?对付打隧道 

的攻城方法是要在城内修建高楼,用来密切观察敌情。敌方有变,修筑掩体 
墙而积聚土石就不同于一般情形,如果四周有平常不同的浑浊泥水,这便是 
敌人在挖隧道。应赶快在城内对着敌人隧道方向挖沟和隧道以防范它。假若 
还不能准确判断敌人挖隧道的方位,就在城内挖井,每隔五步挖一井,要靠 
近城墙墙基。地势高的地方掘深五尺,地势低的地方,打到出水,有三尺深 
就够了。命令陶匠烧制肚大口小的坛子,大小能容纳四十斗以上,用薄皮革 
蒙紧坛口放入井内,派听觉灵敏的人伏在坛口上静听传自地下的声音,确切 
地弄清楚敌方隧道的方位,然后挖隧道与之相抗。 
命令陶器匠烧制瓦管,每根长二尺五寸,大六围,从中剖开为两块,合 
起来安装在隧道里,两块上下合成圆柱。在圆柱的外面妥善地用泥涂好缝隙, 
那些已合成圆柱的暂不要点火。圆柱连成长管后还要在接口处四面涂泥,以 
免漏烟。在隧道的两边都安装这样的瓦管,随隧道而延伸,瓦管还要紧贴地 
面。在瓦管中备有糠头和炭末,不能装满堵死,但要沿管道一路装去,不可 
中断,同时要调放均匀。隧道口砌灶,须使灶的形状与烧陶器的窑差不多, 
使它能装下七八把艾草团,两边的瓦管道口都是如此。灶装备四个皮风箱。 
敌我双方隧道将要相接时,就用叫“颉皋”的武器冲破士层,立即鼓励风箱, 
以烟熏敌。一定要派能极其熟练地操作风箱者掌用风箱,不能让其离开灶口。 
拼连木板,以隧道的高度和宽度为标准,命令打隧道的兵士带着拼板向前, 
在拼板上打有孔眼,大小能使长矛通过,还须疏密相间,以便能用它抢救敌 
人对瓦管的破坏。敌我双方隧道一旦打通,就用拼板阻挡敌人,从孔中伸出 
长矛抢救瓦管,不让管道堵塞;假若管道被堵,就拉着拼板退后一节瓦管阻 
挡敌人,凿开被堵的瓦管,使烟气畅通,烟一通,就加速鼓动风箱熏敌。同 
时在隧道内还要注意仔细审听左右两边传来的声响,一旦发现情况,立即设 
法断绝阻挡敌人向前。如若冲到敌方隧道,就用涂了泥的木柴堵住敌人,不 
让敌人烧我拼板。这样一来,敌人用打隧道的方法来攻城就失败了。 
如果敌人已兵临城下,军情紧急已非同寻常,要谨防敌人挖隧道攻城。 
敌人一有挖隧道的迹象,就应赶紧打隧道来与之相对抗。如敌人隧道的方位 
还没有确切弄清,我方隧道就要慎挖,不要盲目往前打。 
凡是对付以隧道进攻的敌人,须每隔二十步挖一隧道,隧道高十尺、宽 
十尺。向前开挖时,每步向下低三尺,深入十步就向两边横向开挖支道,名 
叫“拥穴”,高和宽也分别为十尺。 
打败敌人隧道攻城,埋下两个坛子,深度使坛口与地面平齐,坛口上盖 
上木板,用来听取地下传来的声响。察听敌人打隧道用的井,每五步挖一个。 
每条隧道口都装有用枱和松两种木材做成的隧道门,门内安上两个蒺藜,蒺 
藜的长度与门的高度一致。门上装上铁环。门外用石头垒成围墙,高七尺, 
围墙上加砌锯齿状矫墙,围墙内不要修建阶梯和垒石块,用吊梯上下出入。 
筑炉灶和装设风箱,风箱用牛皮制做,用杠杆鼓动。炉中置备有煤,重四十 
斤;用燃烧着的木炭给煤助燃,装满炉灶就盖好盖子,不让烟气外泄。敌人 
隧道快要同我接通时,根据敌方隧道与我隧道上下方位的情况,分别向上或 
向下开挖以求开通。在隧道中与敌人相遇,只抵抗而不要驱赶敌人,甚至还 
可以假装败退,等待炉火燃烧,炉火一燃,就立即离开敌人进入壅穴中。 
在隧道中建构“■穴”,设置门和关锁的机关,要让狗能够在其中往来 
行走。每个隧道中放一条狗,狗叫就说明有人。 
在炉灶中,放置砍切成一尺一段的艾草与木柴。先垒石砌成灶壁,对着 

隧道把木板拼接好。 
打井要紧靠城墙的基础部位。每三丈远掘一口井。要根据地形的宽窄打 
井,谨慎不可大意。城基深而洞口位置高,那么隧道开凿就很困难。在城墙 
下掘井三四口,把蒙了皮的坛子装入井内,将耳朵贴在坛口静听地下传来的 
声响。确切弄清了敌人隧道的方位后,就从城内打隧道与之相对。敌我两方 
隧道快要接通时,一定要用坚硬的材料做成冲杆,以成颉皋,冲杆上装有锋 
刺的斧头,命令三个勇猛有力的人使用颉皋冲击敌方隧道土层,一旦打通, 
就将十几担糠、屎之类不干不净的东西灌入敌方隧道。把人员紧急埋伏在隧 
道井中,在上面堆放七八捆艾草,用大盆盖上井口,不要叫烟火上冒旁出, 
在旁边装有风箱,迅疾地向敌方隧道鼓烟。 
用车轮扎成“轒辒”,用木头连成一体,将麻索浸湿涂上泥捆扎车轮。 
用铁链将轒辒悬挂在敌人进攻的隧道口。铁链长三丈,一端结成环,一端安 
上挂钩。 
■穴高七尺五寸,支柱与支柱之间的横向宽度为七尺,纵向长度为二尺。 
支柱下面垫上垫块,两个支柱上端共一个顶板,名叫“负土”。下面也都一 
样垫上垫块,顶板“负土”要横着安放。支柱二围半粗,一定要将顶板装牢, 
柱与柱不要相交。 
每条隧道口设两个灶,灶上都要盖上瓦顶,安排小吏和帮人各一人掌管, 
一定要备足了水。阻塞隧道口的方法,是采用两个车轮扎成轒辒涂上泥,按 
照隧道的宽窄情形,让它在隧道中四五尺处的地方用绳索悬挂起来。当攻打 
隧道的敌人抢入我方伏门时,就转动悬挂轒辒的辘轳,放下轒辒堵住敌人。 
砌筑能容下三大团艾草的炉灶,使敌方的突击队员进入我方伏击圈,我方隐 
伏在突门一边,守住风箱不可离开。隧道中使用的短矛用铁铸造,长四尺半, 
大小与“铁服说”相同,“铁服说”就是古代兵器中的“酋矛”和“夷矛” 
两种矛。在离隧道口一尺处掘进时,要倾斜着,向下打到地心,所用矛长七 
尺。隧道中装置环索供上下牵引,每条隧道安装两种这样的环索。 
在城下掘井,等到井快要穿通时,就站到版上,向旁边斜凿,凿完就移 
动坐版开凿另一边。颉皋做成两端,旁边栽立柱,把钩子安装在两头。打隧 
道者每队五十人,男女各半。打隧道要用能装土石六竹箕的传运工具,用绳 
子兜住底部,可以提起将土倒出去。隧道工程中止后,每条隧道里凿有供休 
养的洞垒,由七人在其中守护。建大屋一间,专藏打隧道的各种工具。为阻 
止敌人打隧道,先捡取护城河边的木石瓦砾撒散在城墙外,城内开挖壕沟, 
深度打到冒出地下水。在快要接通敌方隧道时阻击敌人,要制造铁斧,斧头 
连同斧柄共长四尺,不过不必多造,只要够用即可。如果敌人挖隧道,我就 
以相对的隧道应战敌人。 
还要制作长四尺的铁钩巨,也不要多造,够用即可。隧道与敌方接通时, 
用这种武器钩打敌方兵士。短矛、短戟、短弓、短箭也不必多造,够用就行 
了。敌我隧道一接通,打隧道的人就可以拿它们与敌人战斗。在隧道里还可 
以使用一种叫“金剑”的武器,因在隧道中使用,只长五尺,要有装柄的孔 
眼,用木做柄,木柄上手握的部位刻上浅槽与齿纹。 
再备制一些容量在三十斗以上的大瓦坛,埋放在井洞中,每三丈一个, 
用来听取敌人挖隧道的声音。 
挖掘隧道,高宽各八尺,妥善立好支柱,安置好炉灶、牛皮风箱以及瓦 
钵等物,每条隧道都备有两套。灶中装满藿香、艾草等,敌我隧道一通,立 

刻烧烟薰敌。 
用金属制作斧子,木柄三尺长,每条隧道备上四把。制备盛土用的竹笼, 
每条隧道要备四十个,锄头之类四把。配备斧头、锯子、凿子、大锄等工具, 
数量只求够用不需多。配备大剪刀,每个隧道四把。制做中等大小的盾牌, 
高十尺半,宽四尺。制备横放在隧道中阻敌的大拼板。再预备禾杆,麻梗, 
不必贪多,够用即可,用来照明。还要配备一种名叫“醯”的酒,敌人攻来 
用烟薰,这种酒用来救护自己兵士的眼睛。眼睛的薰伤一解除,我方兵士就 
赶紧向各方开挖隧道。用盆装上“醯”放置在隧道里,大盆不要少于四斗的 
容量,假如烟薰了,就低头看盆中的“醯”酒,以便保护好眼睛。 

\chapter{四十五  备蛾傅(1)}

 禽子再拜再拜曰:“敢问适人强弱(2),遂以傅城,后上先断,以为法程; 
斩城为基,掘下为室。前上不止,后射既疾,为之奈何?” 
子墨子曰:子问蛾傅之守邪?蛾傅者,将之忿者也。守为行临射之,校 
机藉之,擢之,太氾迫之(3),烧荅覆之,沙石雨之,然则蛾傅之攻败矣。 
备蛾傅为县脾,以木板厚二寸,前后三尺,旁广五尺,高五尺,百折为 
下磨车(4),转径尺六寸(5),令一人操二丈四方(6),刃其两端,居县脾中, 
以铁璅,敷县二脾上衡,为之机,令有力四人下上之,弗离。施县脾,大数 
二十步一,攻队所在六步一。 
为絫荅,广从丈各二尺(7),以木为上衡,以麻索大遍之(8),染其索涂 
中,为铁鏁,钩其两端之县。客则蛾傅城,烧荅以覆之连■、抄大皆救之(9)。 
以车两走,轴间广大,以圉犯之,■其两端以束轮(10),遍遍涂其上,室中 
以榆若蒸(11),以棘为旁,命曰火捽,一曰传汤,以当队。客则乘队,烧传 
汤,斩维而下之,令勇士随而击之,以为勇士前行,城上辄塞坏城。 
城下足为下说镵杙(12),长五尺,大圉半以上(13),皆剡其末,为五行, 
行间广三尺,貍三尺,大耳树之(14)。为连殳,长五尺,大十尺。梃长二尺, 
大六寸,索长二尺。椎,柄长六尺,首长尺五寸。斧,柄长六尺,刃必利, 
皆■其一后(15)。荅广丈二尺,□□丈六尺,垂前衡四寸,两端接尺相覆, 
勿令鱼鳞三,著其后行中央(16)木绳一(17),长二丈六尺。荅楼不会者以牒 
塞,数暴干,荅为格,令风上下。堞恶疑坏者,先貍木十尺一枚一(18),节 
坏(19),■植,以押虑卢薄于木(20),卢薄表八尺(21),广七寸,经尺一(22), 
数施一击而下之,为上下釫而■之。 
经一。钧(23)、禾楼(24)、罗石(25)。县荅植内,毋植外。 
杜格(26),貍四尺,高者十丈,木长短相杂,兑其上,而外内厚涂之。 
为前行行栈,县荅。隅为楼,楼必曲里(27)。土五步一,毋其二十畾(28)。 
爵穴十尺一,下堞三尺,广其外。转■城上,楼及散与池,革盆。若转(29), 
攻卒击其后,煖失(30),治。车革火。…… 
凡杀蛾傅而攻者之法,置薄城外,去城十尺,薄厚十尺,伐操之法(31), 
大小尽木断之,以十尺为断,离而深貍坚筑之,毋使可拔。二十步一杀,有 
■(32),厚十尺。杀有两门,门广五步(33),薄门板梯貍之,勿筑,令易拔。 
城上希薄门而置捣(34)。 
县火。四尺一椅(35),五步一灶,灶门有炉炭。传令敌人尽入,车火烧 
门(36),县火次之,出载而立,其广终队,两载之间一火,皆立而待鼓音而 
然,即俱发之。敌人辟火而复攻,县火复下,敌人甚病。敌引哭而榆(37), 
则令吾死士左右出穴门击遗师,令贲土、主将皆听城鼓之音而出,又听城鼓 
之音而入。因素出兵将施伏,夜半而城上四面鼓噪,敌人必或,破军杀将。 
以白衣为服,以号相得。 


[注释] 

(1)《备蛾傅》是墨子研究城池防守战术的篇章之一。主要阐明如何对付敌军凭借人多势众,驱 

赶兵士象蚂蚁般强行爬城进行硬攻的战术防守方法。(2)“弱”应作“梁”。(3)“太氾”应作“火汤”。 

(4)“磨”应作“磿”。(5)“转”应作“轮”。(6)“方”应作“矛”。(7)“丈各”应作“各丈”。 

(8)“大遍”应作“编”。(9)“抄大”应作“沙灰”。(10)“■”应作“融”。(11)“室”应作“窒”。 

(12)“说”应作“锐”。(13)“圉”应作“围”。(14)“大耳”应作“犬牙”。(15)“皆■其一后” 

此句难以索解。(16)“行”应作“衡”。(17)“木”应作“大”。(18)“一”疑衍,误在此。(19)“节” 

应作“即”。(20)疑“虑”衍文。(21)“表”应作“袤”。(22)“经尺一”应作“径一尺”。(23)疑 

“经一”为衍文;“钧”应作“钩”。(24)“禾”应作“木”。(25)“罗”应作“■”。(26)“杜” 

应作“柞”。(27)“曲里”应作“再重”。(28)“其”应作“下”。(29)“转”应作“傅”。(30)“煖” 

应作“缓”。(31)“操”应作“薄”。(32)“■”应作“鬲”。(33)“步”应作“尺”。(34)“捣” 

应作“楬”。(35)“椅”应作“樴”。(36)“车”应作“熏”。(37)“哭”应作“师”;“榆”应作 

“逃”。 
[白话] 
禽滑厘行了两次再拜礼,然后说:“请问,如果敌兵强悍,以致攀爬我 
方城墙,对后上者实行当场斩首,作为军法,同时在城下挖壕沟,筑土山, 
在城下掘隧道。前面敌兵攀爬不止,后面的弓箭又一个劲猛射,这种情况该 
如何对付呢?” 
墨子回答说:你问的是对付敌人像蚂蚁一样爬城的防守战法吗?依仗人 
多势众、驱赶士兵像蚂蚁般强行攻城,这不过是敌将恼怒发急之下不理智的 
举措罢了,守城一方只须要加筑临时的城垛,居高临下向爬城的敌人射击, 
用“技机”投掷攻敌,并拔掉敌方爬城的器具,用火把、滚烫开水倾倒制服 
城下的敌兵,用点燃的名叫“荅”的战具从城上下罩敌人,沙石象雨点般向 
敌人头上打,这样一来,像蚂蚁般攀爬城墙的强行攻城法就失败了。 
为防备敌人像蚂蚁般爬城强攻,可制做悬滑车,这种车用二寸厚的木板 
做成,前后各三尺宽,两旁宽五尺,高五尺,还要制造能够上下滑动的悬滑 
车箱,所用辘轳的轮子直径为一尺六寸,派一个士兵拿一支长矛站在车箱中, 
矛长二丈四尺,两端都制成刃口。用铁链套住悬滑车上部的横梁,装上辘轳, 
派四个强壮有力的兵士转动辘轳使人同车箱急速地上升或下降,不要停留。 
这种悬滑车在一段地段每隔二十步置一架,在敌人所攻击的区域,每六步一 
架。 
制做“■荅”,长和宽各一丈二尺,上面的横梁以木制成,用大麻绳系 
住,麻绳要在泥水中浸泡;装备铁链,钩住两头的吊环。如果敌人像蚂蚁般 
爬城硬攻,就点燃“荅”从上往下罩敌人。此外,“连■”、沙灰等物都可 
供解救攻城用。 
配备两个车轮,让两轮轴之间的距离长一些,用“圉”固定,并将两头 
熔合使两轮束成一体,到处涂上泥,在里面塞满榆树枝叶和麻梗等易燃物, 
两边布装荆棘,这个装置被称之为“火捽”,又叫“传汤”,用来布置在敌 
人的主攻区域。假如敌人结队登城,就点燃“传汤”,砍断上面的吊绳让它 
滚下去,并命令勇士以汤开路反击敌人。城墙一有破坏应立即派人迅速填塞 
抢修。 
在城外墙根埋植锋利的木桩,长五尺,大一围半以上,末端都要削尖, 
共埋五行,行距三尺,深埋三尺,要犬牙交错般地埋栽。造“连殳”,长五 
尺,宽十尺。“梃”长二尺,宽六寸,绳索长二尺。椎,柄六尺,头部长一 
尺五寸。斧子,柄长六尺,斧口一定要锋利。“答”,宽一丈二尺,长一丈 
六尺,悬挂在前面横梁四寸处。两头衔接的地方要相互搭连一尺左右,但不 
要象鱼鳞那样交错。在后横梁的中间系上一根大绳,长二丈六尺。如果有不 
密合的地方就用版片填塞,要多曝晒,使其干燥,“荅”要制成格栅,能使 
空气流通。城上矫墙不太坚固恐怕倒塌的地方,要预先埋植木桩,每十尺一 

枚。如果城墙倒塌了,就树起木桩,在木桩上压上横木,横木长八尺,宽七 
寸,侧高一尺,一锤又一锤地将木桩打下去,然后用马钉钉牢。 
钩、木楼、■石都要备好。“荅”要悬挂在柱子靠里的一面,不要悬挂 
在柱子靠外的一面。 
再布置“柞格”,埋入四尺,露出地面高的以十尺为限,木头长短相间, 
交错置布,头部削尖,四周涂上厚厚的泥。 
制做行栈,悬挂烧荅。在城角建楼,楼要多层。备土,五步一堆,每堆 
不少于二十笼。 
打爵穴,每十尺一个,开在矫墙下部三尺处,外面的口要稍大一些。转 
■城上,须备设行楼,杀,水池和盛水用的皮盆。假如敌兵爬城,而担负攻 
击任务的士兵不能及时出击,贻误战机按军法处置。 
还要用火烤烟薰…… 
大凡防阻敌人爬城硬攻,一般方法是在城外设置木桩做成的屏障,屏障 
离城墙十尺,高十尺,采伐用以做屏障的木桩,办法是树木不分大小连根拔 
起,锯成十尺长一段,互相间隔一段距离深埋紧筑,不能让人能拔出来。 
每二十步设置一个“杀”,杀中设“鬲”,鬲厚十尺。杀装有两门,门 
宽五尺。木桩屏障设“杀”的门户要浅埋,不要筑牢实;以便它易于拔出。 
城上对着木桩屏障处,相应设置“捣”。 
在城上悬挂火具,称为悬火。每隔四尺装一个悬挂火具的钩樴,五步建 
一口灶,灶门堆放炉炭。等敌人全部进入后下令熏火、烧门,接着向下抛扔 
悬火。摆放作战器具,宽度与敌人进攻的范围相应。每两作战器具之间设一 
悬火,派一士兵站立在旁边,等待出击的鼓声,鼓声一响就点火,一齐往下 
投。敌人若避开悬火再度进攻,悬火也就再次往下投,如此反复,敌人必定 
被弄得疲惫不堪,终会引兵逃走。敌人一退逃,就命令我方敢死队从左右出 
穴门追击溃敌,但要严令勇士和将帅按照城上的鼓声行事,从城内出去或退 
回城内都应如此。趁着多次出击时还可设下埋伏,半夜三更城墙上四周击鼓 
呐喊,敌兵一定惊疑不定,伏兵便可乘机攻破敌营,擒杀敌军首领。不过要 
以白衣作军服,凭口号互相联络。 


\chapter{四十六  迎敌祠}

 敌以东方来,迎之东坛,坛高八尺,堂密八;年八十者八人,主祭;青 
旗、青神长八尺者八,弩八,八发而止;将服必青,其牲以鸡。敌以南方来, 
迎之南坛,坛高七尺,堂密七;年七十者七人,主祭;赤旗、赤神长七尺者 
七,弩七,七发而止;将服必赤,其牲以狗。敌以西方来,迎之西坛,坛高 
九尺,堂密九;年九十者九人,主祭;白旗、素神长九尺者九,弩九,九发 
而止;将服必白,其牲以羊。敌以北方来,迎之北坛,坛高六尺,堂密六; 
年六十者六人,主祭;墨旗、黑神长六尺者六,弩六,六发而止;将服必黑, 
其牲以彘。从外宅诸名大祠,灵巫或祷焉,给祷牲。 
凡望气,有大将气,有小将气,有往气,有来气,有败气,能得明此者 
可知成败、吉凶。举巫、医、卜有所长,具药,宫之(2),善为舍。巫必近公 
社,必敬神之。巫、卜以请守(3),守独智巫、卜望气之请而已。其出入为流 
言,惊骇恐吏民,谨微察之,断罪不赦。 
牧贤大夫及有方技者若工(4),弟之。举屠、酤者置厨给事,弟之。 
凡守城之法,县师受事,出葆,循沟防,筑荐通涂,修城,百官共财, 
百工即事,司马视城修卒伍。设守门,二人掌右阉,二人掌左阉,四人掌闭, 
百甲坐之。 
城上步一甲、一戟,其赞三人。五步有五长,十步有什长,百步有百长, 
旁有大率,中有大将,皆有司吏卒长。城上当阶,有司守之。移中中处,泽 
急而奏之。士皆有职。 
城之外,矢之所遝,坏其墙,无以为客菌。三十里之内,薪蒸、水皆入 
内(5)。狗、彘、豚、鸡食其肉,敛其骸以为醢,腹病者以起。 
城之内,薪蒸庐室,矢之所遝,皆为之涂菌。令命昏纬狗纂马,■纬。 
静夜闻鼓声而噪,所以阉客之气也,所以固民之意也,故时噪则民不疾矣。 
祝、史乃告于四望、山川、社稷,先于戎,乃退。公素服誓于太庙,曰: 
“其人为不道,不修义详,唯乃是王(6),曰:‘予必怀亡尔社稷,灭尔百姓。’ 
二参子尚夜自厦(7),以勤寡人,和心比力兼左右,各死而守。”既誓,公乃 
退食。舍于中太庙之右,祝、史舍于社。百官具御,乃斗(8),鼓于门,右置 
旗,左置旌于隅练名。射参发,告胜,五兵咸备,乃下,出挨(9),升望我郊。 
乃命鼓,俄升,役司马射自门右,蓬矢射之,茅参发(10),弓弩继之;校自 
门左,先以挥,木石继之。祝、史、宗人告社,覆之以甑。 


[注释] 

(1)《迎敌祠》是墨子探讨城池防守方法的篇章之一。主要讲述迎敌前的各种祭祀规则,对巫师 

卜师的态度,誓师形式以及各级官吏、将士的职守和有关布防问题。(2)“宫”字后疑掉一“养”字。 

(3)“请”字后疑漏一“报”字。(4)“牧”应作“收”。(5)“水”应作“木”,前疑有一“材”字。 

(6)“乃”应作“力”;“王”应作“上”。(7)“厦”应作“厉”。(8)“斗”应作“升”。(9)“挨” 

应作“俟”。(10)“茅”应作“矛”。 
[白话] 
敌人从东方来,就在东方的祭坛上迎祭神灵,坛高八尺,宽深也各八尺; 
由八个年龄八十岁的人主持祭青旗的仪式,安排八尺高的八位东方神,八个 
弓箭手,每个弓箭手射出八支箭;将领的服装必是青色,用鸡作祭品。敌人 
从南方来,就在南方的祭坛上迎祀神灵,坛高七尺,宽深也各六尺;安排七 

个年龄七十的人主持祭赤旗的仪式;准备七尺高的南方赤神七尊,弓箭手七 
个,每人发射七支箭;将领的军服一定要赤色,用狗作祭品。敌人从西方来, 
就在西边的祭坛迎祭神坛高九尺,宽深也各为九尺;九个年龄九十岁的人主 
持祭白旗的仪式;九尺高的西方白神九尊,九个弓箭手每人发射九支箭;将 
领的军服一定要白色的,用羊作祭品。敌人从北方来,就在北方的祭坛上迎 
祭神灵,祭坛高六尺,宽深各为六尺;由六位年龄六十岁的人主持祭黑旗的 
仪式;高六尺的北方黑神六尊,六个弓箭手每人各发六支箭;将领的军服一 
律黑色,用猪作祭品。从外面所有有名的大祠堂起,灵验的巫师有的在那里 
祈祷神灵,要供给他们祭品。 
凡占望气,有大将气,有小将气,有往气,来气、败气等种类区别,能 
懂得这些“气”别内容的人可预知成功、失败,吉利和凶险。找出所有有专 
长的巫师、医师和占卜的人,根据他们的特长,配备有关药物,供给住房, 
妥善安排宿住。巫师住的地方一定要靠近祭土地神的地方,一定要将其作为 
神灵来敬重。巫师和卜师将实情报告给守将,只能让守城主将知道其占望的 
结果,不要让其他人知道;如果巫师卜师出入制造传播流言,弄得官民惊恐 
不安,要谨慎地暗中侦察,处罚这些传言的巫师卜师,罪不容赦。 
将贤大夫和有专长的种种技师集中起来,给予相应的第等。挑选屠夫, 
酿酒人安排到厨房供职,也要给予职务等级。 
一般守城的法规,县师负责视察堡垒,巡视河沟城防,阻塞敌人的道路, 
修缮城墙。所有大小官吏要供应战争所需的粮饷钱款,一切有手艺的人都要 
各施所长。司马根据城防情况布派兵士守门,二人掌管城门右边门扇,二人 
掌左边门扇,四人共同掌管开关城门的职责,百名兵士带甲坐守城门。 
城墙上每一步派一个带甲的兵士,一个握戟的兵士,另加三个帮手。每 
五步派备一个伍长,每十步安排一名什长,百步委任一名佰长。在城的四面, 
分别派有一个大帅;城的中央有大将指挥。这样逐级都有首领和各自的职责。 
在上城墙的阶梯处,派专职官兵把守。将文书簿籍转移到合适的地方,选取 
紧急重要的部分上报。军士也都有各自的职守。 
在城外箭能射到的地方,要把墙统统推倒,以免被敌人利用来作为防御 
工事。三十里以内,所有柴草树木一律运进城内。狗,猪,鸡,吃掉肉,将 
其骨头收集起来制成酱,肠胃有病的可以用它治病。在城内,凡是城外箭能 
射到的地方,一切柴草堆和房屋都要抹上一层泥。黄昏之后,命令城内人拴 
住狗,套住马,务必拴套牢实。夜深人静之时一听到鼓声就一齐呐喊,用来 
压制敌人的气焰,同时也可以稳定自己的民心,不致使老百姓惊扰不安了。 
太祝和太史官在战前要祭告四周的山川和宗庙,然后才退出。诸侯穿着 
白祭服在太庙誓师。誓词说:“某人干不合道义的事情,不修仁义,唯力是 
尚,还声言‘我一定要灭掉你的国家,消灭你的百姓万民’。我的几位大臣 
尚自我勉励,勤力辅助我,率领左右部下齐心协力,誓死保守国土。”誓师 
结束,诸侯才退下用餐。他临时要住在中太庙的右边房舍中,太祝和太史临 
时住在社庙。其它百官各奉其职,于是上庙,在庙门击鼓,门的右边插上旗, 
左边插上旌,门的左右角布置铭识,兵士们发射三箭,祈祷胜利,各军兵都 
一应齐备。仪式结束后下太庙,出外等候登上城门台观望城郊情景。接着命 
令击鼓,一会儿登上门台,役司马从门的右边向天地四方发射用蓬蒿制成的 
箭,拿矛的兵士则用矛向空中刺三下,接着弓箭手向空发射;军校从门的左 
边先进行一种叫“挥”的制胜巫术,然后木头擂石齐下。太祝、太史,礼官 

向社庙祭告,然后把祭品用作饭的陶器甑盖起来。 

\chapter{四十七  旗帜}

守城之法,木为苍旗,火为赤旗,薪樵为黄旗,石为白旗,水为黑旗, 
食为菌旗,死士为仓英之旗,竟士为雩旗(2),多卒为双兔之旗,五尺童子为 
童旗,女子为(梯末)(妹妹)之旗,弩为狗旗,戟为■旗(3),剑盾为羽旗, 
车为龙旗,骑为鸟旗。凡所求索旗名不在书者,皆以其形名为旗。城上举旗, 
备具之官致财物,之足而下旗(4)。 
凡守城之法:石有积,樵薪有积,■茅有积,雚苇有积(5),木有积,炭 
有积,沙有积,松柏有积,蓬艾有积,麻脂有积,金铁有积(6),粟米有积; 
井灶有处,重质有居;五兵各有旗;节各有辨;法令各有贞;轻重分数各有 
请;主慎道路者有经。 
亭尉各为帜,竿长二丈五,帛长丈五、广半幅者大(7)。寇傅攻前池外廉 
城上当队鼓三,举一帜;到水中周,鼓四,举二帜;到藩,鼓五,举三帜, 
到冯垣,鼓六,举四帜,到女垣;鼓七,举五帜;到大城,鼓八,举六帜; 
乘大城半以上,鼓无休。夜以火,如此数。寇却解,辄部帜如进数,而无鼓。 
城为隆(8),长五十尺,四面四门将,门长四十尺,其次三十尺,其次二 
十五尺,其次二十尺,其次十五尺,高无下四十五尺(9)。 
城上吏卒置之背(10),卒于头上;城下吏、卒置之肩,左军于左肩,中 
军置之胸(11),各一。鼓,中军一三(12),每鼓三、十击之,诸有鼓之吏, 
谨以次应之;当应鼓而不应,不当应而应鼓,主者斩。 
道广三十步,于城下夹阶者各二,其井,置铁■(13)。于道之外为屏, 
三十步而为之圜,高丈。为民圂,垣高十二尺以上。巷术周道者,必为之门, 
门二人守之;非有信符,勿行,不从令者斩。 
城中吏卒男女,皆葕异衣章微,令男女可知。 
诸守牲格者(14),三出却适,守以令召赐食前,予大旗,署百户邑。若 
他人财物,建旗其署,令皆明白知之,曰某子旗。牲格内广二十五步(15), 
外广十步,表以地形为度(16)。 
靳卒中教(17),解前后、左右,卒劳者更休之。 


[注释] 

(1)《旗帜》是墨子研究城池防守战术的篇章之一。主要说明守城时用旗帜联络的种种方法。(2) 

“雩”应作“虎”。(3)“■”应作“旌”。(4)“之”应作“物”。(5)“雚”应作“萑”。(6)“铁” 

应作“钱”。(7)“大”应作“六”。(8)“城”后疑漏一“将”字;“隆”应作“绛帜”。(9)“十五 

尺”前疑衍误一“四”字。(10)“城上吏卒置之背”句前脱落“城中吏卒民男女,皆葕异衣章微,令 

男女可知”一段,此段中“葕”应作“辨”;“微”应作“徽”。(11)“中军置之胸”句前脱落“右 

军于右肩”五字。(12)“三”疑衍误在此。(13)“■”应作“罐”。(14)“牲”应作“柞”。(15)同 

(14)。(16)“表”应作“袤”。(17)“靳”应作“勒”。 
[白话] 
守城时用旗帜联络的方法是这样的。需要木材时就用青色旗,需要烟火 
时就用赤色旗,需要柴草时就挂黄旗,需要石头时就用白色旗,需要水时就 
悬黑色旗,需要食物时就扬绘有食菌的旗,需要调集敢死队时就打出画有苍 
鹰的旗帜,需调集战斗力最强的战士时就挂出虎旗,征调多余的兵士时挂双 
兔旗,征调五童子时挂童旗,征调女子时挂姊妹旗,需求弓箭时挂龙旗,需 
要战马时挂鸟旗,凡是需要征调的物质而旗帜又没有现存符号的,就按所需 

物质的形状名称含义挂上相应的旗帜。城上悬挂旗帜是军需官在联络征集调 
用财物,一旦满足军需,应随即把旗帜降下来。 
按守城的法则,必须积聚有充分的擂石、柴薪、茅草、芦苇、木材、炭、 
沙、松柏、蓬艾、麻杆、油脂、铜铁和粮食;水井、炊灶都有适当的地方, 
敌方重要的抵押品如人质有居处。各军兵种都有各自的旗号;调兵遣将各有 
符节;法令有定条定例,轻重等级各根据实际情况而定。主持巡查道路的官 
吏也有各自划定的区域。 
各个亭尉都有自己的旗帜,旗竿长二丈五尺,帛长一丈五尺,宽半幅, 
共六面旗帜。当敌人进攻到护城河外边时,与敌人进攻方位相对的守军就击 
三下鼓,并悬挂一面旗;当敌人进攻到护城河中洲处时就击五下鼓,挂上三 
面旗;当敌人进入到城外的第一道矫墙时,就击鼓六下,挂上四面旗;当敌 
人深入到大城墙下时,就击八下鼓,挂上六面旗;当敌人爬上城墙的一半时, 
就擂鼓不停。如果是在夜晚,就以举火把来代替挂旗,举火把的数目与白天 
挂旗的数目一样。假如敌人是由进攻而退却,悬挂旗的数目同敌进攻时的数 
目不变,但不击鼓。 
城中大将悬挂绛旗,高五十尺,东西南北四门守将的旗帜各高四十尺, 
其次一等的是三十尺,再次一等的高二十五尺,依次为二十尺,十五尺。不 
过将旗的高度没有低于十五尺的了。 
城里的军官、士兵、男女百姓都通过衣服上的徽章区别开来。城上小吏 
的徽章戴在衣背上,士兵的徽章戴在帽子上或头巾上;城下的小吏徽章都戴 
在衣肩上,士兵也是如此。左军的徽章都戴在左肩上,右军的徽章戴在右肩, 
中军的徽章戴在衣胸前,一人一个微章。中军有一个号鼓,每次击鼓三至十 
下,其余有鼓的官吏击鼓应答中军鼓号要仔细按等第进行。应当击鼓回应时 
却没有按时击鼓,不当击鼓回应时却胡乱击鼓,要处斩其主管人。 
修建道路要三十步宽,城下夹阶的大道各有两口井,井边设置铁罐。在 
道路外边筑起屏障,每三十步就砌成一个圆圈,高一丈。建修公共厕所,墙 
高十二尺以上。城中与大道相连通的大街小巷都一定要装设上门,每个门派 
两人把守,没有通行凭证不许通行,不服从命令的处斩。 
各个据守柞格的兵将,三次出战击退敌兵的,守将主帅便传令他们到官 
署领赏食物,授予他们大旗,并赐给百户的城邑或财物,把赏赐的大旗竖在 
他们的营署中,使人们都知道他们立有战功,此种大旗称为“某人旗”。柞 
格内宽二十五步,外宽十步,长度根据地形决定。要按照教令统领兵士,懂 
得以军事号令指挥前进后退,向左向右,还要让疲劳的士兵能轮番地休整。 

\chapter{四十八  号令}

安国之道,道任地始,地得其任则功成,地不得其任则劳而无功。人亦 
如此,备不先具者无以安主,吏卒民多心不一者,皆在其将长,诸行赏罚及 
有治者,必出于王公。数使人行劳赐守边城关塞、备蛮夷之劳苦者,举其守 
卒之财用有余、不足,地形之当守边者,其器备常多者。边县邑视其树木恶 
则少用,田不辟,少食,无大屋草盖,少用桑。多财,民好食。为内牒(2), 
内行栈,置器备其上,城上吏、卒、养,皆为舍道内,各当其隔部。养什二 
人,为符者曰养吏一人,辨护诸门。门者及有守禁者皆无令无事者得稽留止 
其旁,不从令者戮。敌人但至,千丈之城,必郭迎之,主人利。不尽千丈者 
勿迎也,视敌之居曲众少而应之,此守城之大体也。其不在此中者,皆心术 
与人事参之。凡守城者以亟伤敌为上,其延日持久以待救之至,不明于守者 
也(3),不能此(4),乃能守城。 
守城之法,敌去邑百里以上,城将如今尽召五官及百长,以富人重室之 
亲,舍之官府,谨令信人守卫之,谨密为故。 
及傅城,守城将营无下三百人。四面四门之将,必选择之有功劳之臣及 
死事之后重者,从卒各百人。门将并守他门,他门之上,必夹为高楼,使善 
射者居焉。女郭、冯垣一人。一人守之,使重室子。 
五十步一击(5)。因城中里为八部,部一吏,吏各从四人,以行冲术及里 
中。里中父老小不举守之事及会计者(6),分里以为四部,部一长,以苛往来 
不以时行、行而不他异者,以得其奸。吏从卒四人以上有分者(7),大将必与 
为信符;大将使人行守操信符,信不合及号不相应者,伯长以上辄止之,以 
闻大将。当止不止及从吏卒纵之,皆斩。诸有罪自死罪以上,皆遝父母、妻 
子同产。 
诸男女有守于城上者(8),什六弩、四兵。丁女子、老少,人一矛。 
卒有惊事,中军疾击鼓者三,城上道路、里中巷街,皆无得行,行者斩。 
女子到大军,令行者男子行左,女子行右,无并行。皆就其守,不从令者斩。 
离守者三日而一徇,而所以备奸也。里正与皆守宿里门(9),吏行其部,至里 
门,正与开门内吏,与行父老之守及穷巷幽间无人之处。奸民之所谋为外心, 
罪车裂。正与父老及吏主部者,不得,皆斩;得之,除,又赏之黄金,人二 
镒。大将使使人行守(10),长夜五循行,短夜三循行。四面之吏亦皆自行其 
守,如大将之行,不从令者斩。 
诸灶必为屏,火突高出屋四尺。慎无敢失火,失火者斩其端,失火以为 
事者车裂(11)。伍人不得,斩;得之,除。救火者无敢喧哗,及离守绝巷救 
火者斩。其正及父老有守此巷中部吏,皆得救之,部吏亟令人谒之大将,大 
将使信人将左右救之,部吏失不言者斩。诸女子有死罪及坐失火皆无有所失, 
逮其以火为乱事者如法。 
围城之重禁,敌人卒而至,严令吏命无敢喧嚣(12)、三最(13)、进行、 
相视坐泣、流涕若视、举手相探、相指、相呼、相麾、相踵、相投、相击、 
相靡以身及衣、讼驳言语。及非令也而视敌动移者,斩。伍人不得,斩;得 
之,除。伍人逾城归敌,伍人不得,斩;与伯归敌,队吏斩;与吏归敌,队 
将斩。归敌者父母、妻子同产,皆车裂。先觉之,除。当术需敌。离地,斩。 
伍人不得,斩;得之,除。 
其疾斗却敌于术,敌下终不能复上,疾斗者队二人,赐上奉。而胜围, 

城周里以上,封城将三十里地为关内侯,辅将如令赐上卿,丞及吏比于丞者, 
赐爵五大夫,官吏、豪杰与计坚守者,十人及城上吏比五官者(14),皆赐公 
乘。男子有守者爵,人二级,女子赐钱五千,男女老小先分守者(15),人赐 
钱千,复之三岁,无有所与,不租税(16)。此所以劝吏民坚守胜围也。 
吏卒侍大门中者,曹无过二人。勇敢为前行,伍坐,令各知其左右前后。 
擅离署,戮。门尉昼三阅之,莫,鼓击门闭一阅,守时令人参之,上逋者名。 
铺食皆于署,不得外食。守必谨微察视谒者、执盾、中涓及妇人侍前者志意、 
颜色、使令、言语之请。及上饮食,必令人尝,皆非请也(17),击而请故(18)。 
守有所不说谒者、执盾、中涓及妇人侍前者,守曰断之、冲之若缚之,不如 
令及后缚者,皆断。必时素诫之。诸门下朝夕立若坐,各令以年少长相次, 
旦夕就位,先佑有功有能,其余皆以次立。五日,官各上喜戏、居处不庄、 
好侵侮人者一。 
诸人士外使者来,必令有以执。将出而还若行县,必使信人先戒舍,室 
乃出迎,门守(19),乃入舍。为人下者常司上之,随而行,松上不随下。必 
须□□随。 
客卒守主人,及以为守卫,主人亦守客卒。城中戍卒,其邑或以下寇, 
谨备之,数录其署,同邑者弗令共所守。与阶门吏为符,符合入,劳;符不 
合,牧(20),守言(21)。若城上者,衣服,他不如令者。 
宿鼓在守大门中。莫令骑若使者操节闭城者,皆以执毚。昏鼓,鼓十, 
诸门亭皆闭之。行者断,必击问行故(22),乃行其罪。晨见,掌文鼓,纵行 
者,诸城门吏各入请籥,开门已,辄复上籥。有符节不用此令。寇至,楼鼓 
五,有周鼓,杂小鼓乃应之。小鼓五后众军,断。命必足畏,赏必足利,令 
必行,令出辄人随,省其可行、不行。号,夕有号,失号,断。为守备程而 
署之曰某程,置署街街衢阶若门(23),令往来者皆视而放。诸吏卒民有谋杀 
伤其将长者,与谋反同罪,有能捕告,赐黄金二十斤,谨罪。非其分职而擅 
取之,若非其所当治而擅治为之,断。诸吏卒民非其部界而擅入他部界,辄 
收以属都司空若侯,侯以闻守,不收而擅纵之,断。能捕得谋反、卖城、逾 
城敌者一人(24)。以令为除死罪二人,城旦四人。反城事父母去者(25),去 
者之父母妻子。 
悉举民室材木、瓦若蔺石数,署长短小大。当举不举,吏有罪。诸卒民 
居城上者各葆其左右(26),左右有罪而不智也,其次伍有罪。若能身捕罪人 
若告之吏,皆构之。若非伍而先知他伍之罪,皆倍其构赏。 
城外令任,城内守任。令、丞、尉亡得入当,满十人以上,令、丞、尉 
夺爵各二级;百人以上,令、丞、尉免,以卒戍。诸取当者,必取寇虏听之。 
募民欲财物粟米以贸易凡器者(27),卒以贾予(28)。邑人知识、昆弟有 
罪,虽不在县中而欲为赎,若以粟米、钱金、布帛、他财物免出者,令许之。 
传言者十步一人,稽留言及乏传者,断。诸可以便事者,亟以疏传言守。吏 
卒民欲言事者,亟为传言请之吏,稽留不言诸者(29),断。县各上其县中豪 
杰若谋士、居大夫重厚,口数多少。 
官府城下吏、卒、民家前后左右相传保火(30)。火发自燔,燔曼延燔人, 
断。诸以众强凌弱少及强奸人妇女,以喧哗者,皆断。诸以众强凌弱少及强 
奸人妇女,以喧哗者,皆断。 
诸城门若亭,谨侯视往来行者符(31)。符传疑若无符,皆诣县廷言,请 
问其所使;其有符传者,善舍官府。其有知识、兄弟欲见之,为召,勿令里 

巷中(32)。三老、守闾令厉缮夫为荅(33)。若他以事者、微者,不得入里中。 
三老不得入家人。传令里中有以羽(34),羽者三所差(35),家人各令其官中 
(36),失令若稽留令者,断。家有守者治食。吏、卒、民无符节而擅入里巷、 
官府,吏、三老、守闾者失苛止。皆断。 
诸盗守器械、财物及相盗者,直一钱以上,皆断。吏、卒、民各自大书 
于杰,著之其署同(37),守案其署,擅入者,断。城上日壹废席蓐(38),令 
相错发。有匿不言人所挟藏在禁中者,断。 
吏、卒民死者,辄召其人,与次司空葬之,勿令得坐泣。伤甚者令归治 
病家善养,予医给药,赐酒日二升、肉二斤,令吏数行闾,视病有瘳,辄造 
事上。诈为自贼伤以辟事者,族之。事已,守使吏身行死伤家,临户而悲哀 
之。 
寇去事已,塞祷。守以令益邑中豪杰力斗诸有功者,必身行死伤者家以 
吊哀之,身见死事之后。城围罢,主亟发使者往劳,举有功及死伤者数使爵 
禄,守身尊宠,明白贵之,令其怨结于敌。 
城上卒若吏各保其左右。若欲以城为外谋者,父母、妻子、同产皆断。 
左右知不捕告,皆与同罪。城下里中家人皆相葆,若城上之数。有能捕告之 
者,封之以千家之邑;若非其左右及他伍捕告者,封之二千家之邑。 
城禁:使(39)、卒、民不欲寇微职(40)、和旌者,断。不从令者,断。 
非擅出令者(41),断。失令者,断。倚戟县下城,上下不与众等者,断。无 
应而妄喧呼者,断。总失者(42),断。誉客内毁者,断。离署而聚语者,断。 
闻城鼓声而伍后上署者,断。人自大书版,著之其署隔,守必自谋其先后(43), 
非其署而妄入之者,断。离署左右,共入他署,左右不捕,挟私书,行请谒 
及为行书者,释守事而治私家事,卒民相盗家室、婴儿,皆断,无赦;人举 
而藉之。无符节而横行军中者,断。客在城下,因数易其署而无易其养。誉 
敌:少以为众,乱以为治,敌攻拙以为巧者,断。客、主人无得相与言及相 
藉,客射以书,无得誉(44),外示内以善,无得应,不从令者,皆断。禁无 
得举矢书若以书射寇,犯令者父母、妻子皆断,身枭城上。有能捕告之者, 
赏之黄金二十斤。非时而行者,唯守及操太守之节而使者。 
守入临城,必谨问父老、吏大夫、请有怨仇雠不相解者(45),召其人, 
明白为之解之。守必自异其人而藉之,孤之,有以私怨害城若吏事者,父母、 
妻子皆断。其以城为外谋者,三族。有能得若捕告者,以其所守邑小大封之, 
守还授印,尊宠官之,令吏大夫及卒民皆明知之。豪杰之外多交诸侯者,常 
请之,令上通知之,善属之,所居之吏上数选具之,令无得擅出入,连质之。 
术乡长者、父老、豪杰之亲戚父母、妻子,必尊宠之,若贫人食不能自给食 
者(46),上食之。及勇士父母、亲戚、妻子,皆时酒肉(47),必敬之,舍之 
必近太守。守楼临质宫而善周,必密涂楼,令下无见上,上见下,下无知上 
有人无人。 
守之所亲,举吏贞廉、忠信、无害、可任事者,其饮食酒肉勿禁,钱金、 
布帛、财物各自守之,慎勿相盗。葆宫之墙必三重,墙之垣,守者皆累瓦釜 
墙上。门有吏,主者门里,筦闭,必须太守之节。葆卫必取戍卒有重厚者。 
请择吏之忠信者、无害可任事者(48)。 
令将卫,自筑十尺之垣,周还墙,门、闺者非令卫司马门(49)。望气者 
舍必近太守,巫舍必近公社,必敬神之。巫祝史与望气者必以善言告民,以 
请上报守,守独知其请而已。无与望气妄为不善言惊恐民(50),断弗赦。 

度食不足,食民各自占家五种石升数(51),为期,其在莼害(52),吏与 
杂訾。期尽匿不占,占不悉,令吏卒■得,皆断。有能捕告,赐什三。收粟 
米、布帛、钱金,出内畜产,皆为平直其贾,与主券人书之。事已,皆各以 
其贾倍偿之。又用其贾贵贱、多少赐爵,欲为吏者许之,其不欲为吏而欲以 
受赐赏爵禄,若赎出亲戚、所知罪人者,以令许之。其受构赏者令葆宫见, 
以与其亲。欲以复佐上者,皆倍其爵赏,某县某里某子家食口二人,积粟六 
百石,某里某子家食口十人,积粟百石。出粟米有期日,过期不出者王公有 
之,有能得若告之,赏之什三。慎无令民知吾粟米多少。 
守入城,先以侯为始(53),得辄宫养之,勿令知吾守卫之备。侯者为异 
宫(54),父母妻子皆同其宫,赐衣食酒肉,信吏善待之。侯来若复(55),就 
间。守宫三难(56),外环隅为之楼,内环为楼,楼入葆宫丈五尺为复道。葆 
不得有室,三日一发席蓐,略视之,布茅宫中,厚三尺以上。发侯(57),必 
使乡邑忠信、善重士,有亲戚、妻子,厚奉资之。必重发侯(58),为养其亲 
若妻子,为异舍,无与员同所,给食之酒肉。遣他侯(59),奉资之如前侯(60), 
反,相参审信,厚赐之,侯三发三信(61),重赐之,不欲受赐而欲为吏者, 
许之二百石之吏。守珮授之印。其不欲为吏而欲受构赏,禄皆如前(62)。有 
能入深至主国者,问之审信,赏之倍他侯(63)。其不欲受赏而欲为吏者,许 
之三百石之吏者。扞士受赏赐者,守必身自致之其亲之其亲之所(64),见其 
见守之任(65)。其次复以佐上者(66),其构赏、爵禄、罪人倍之(67)。 
出候无过十里,居高便所树表,表三人守之,比至城者三表,与城上烽 
燧相望,昼则举烽,夜则举火。闻寇所从来,审知寇形必攻,论小城不自守 
通者,尽葆其老弱、粟米、畜产遣卒候者无过五十人,客至堞,去之,慎无 
厌建(68)。候者曹无过三百人,日暮出之,为微职。空队、要塞之人所往来 
者(69),令可口迹者无下里三人(70),平而迹(71);各立其表,城上应之。 
候出越陈表,遮坐郭门之外内,立其表,令卒之半居门内,令其少多无可知 
也。即有惊,见寇越陈去(72),城上以麾指之,迹坐击正期(73),以战备从 
麾所指。望见寇,举一垂;入竟,举二垂;狎郭,举三垂;入郭,举四垂; 
狎城;举五垂。夜以火,皆如此。 
去郭百步,墙垣、树木小大尽伐除之。外空井尽窒之(74),无令可得汲 
也。外空窒尽发之(75),木尽伐之。诸可以攻城者尽内城中,令其人各有以 
记之,事以,各以其记取之。事为之券(76),书其枚数。当遂材木不能尽内, 
即烧之,无令客得而用之。 
人自大书版,著之其署忠(77)。有司出其所治,则从淫之法,其罪射。 
务色谩正,淫嚣不静,当路尼众舍事后就,逾时不宁,其罪射。喧嚣骇众, 
其罪杀。非上不谏,次主凶言,其罪杀。无敢有乐器、弊骐军中,有则其罪 
射。非有司之令,无敢有车驰、人趋,有则其罪射。无敢散牛马军中,有则 
其罪射。饮食不时,其罪射。无敢歌哭于军中,有则其罪射。令各执罚尽杀, 
有司见有罪而不诛,同罚,若或逃之,亦杀。凡将率斗其众失法,杀。凡有 
司不使去卒、吏民闻誓令(78),代之服罪。凡戮人于市,死上目行(79)。 
谒者侍令门外,为二曹,夹门坐,铺食更(80),无空。门下谒者一长(81), 
守数令入中,视其亡者,以督门尉与其官长,及亡者入中报。四人夹令门内 
坐,二人夹散门外坐。客见,持兵立前,铺食更(82),上侍者名。 
守室下高楼候者(83),望见乘车若骑卒道外来者,及城中非常者,辄言 
之守。守以须城上候城门及邑吏来告其事者以验之,楼下人受候者言,以报 

守。 
中涓二人,夹散门内坐,门常闭,铺食更(84);中涓一长者。环守宫之 
术衢,置屯道,各垣其两旁,高丈,为埤■,立初鸡足置(85),夹挟视葆食。 
而札书得必谨案视参食者(86),节不法(87),正请之(88)。屯陈、垣外术衢 
街皆楼(89),高临里中,楼一鼓,聋灶;即有物故,鼓,吏至而止夜以火指 
鼓所。 
城下五十步一厕,厕与上同圂,请有罪过而可无断者(90),令杼厕利之 
(91)。 


[注释] 

(1)《号令》是墨子研究城池防守方法的重要篇章之一。全篇带有综合性质。但主要讲述种种军 

纪、法规、禁令、人员布防和处置的种种具体原则和方法。(2)“牒”应作“堞”。(3)“明”前衍出 

一“不”字。(4)“不”应作“必”。(5)“击”应作“隔”。(6)“父老”后衍一“小”字。(7)“者” 

前疑脱“守”字。(8)“女”应作“子”。(9)“与”后脱“父老”二字。(10)“使”应作“信”。(11) 

“事”前脱一“乱”字。(12)“命”应作“民”。(13)“最”应作“聚”。(14)“十”应作“士”。 

(15)“先”应作“无”。(16)此句有争议,按颜住“无有所与,不租税”为多余,“复”字即有“免 

除赋税”之意。(17)“皆”应作“若”。(18)“击”应作“系”。(19)“门”应作“闻”。(20)“牧” 

应作“收”。(21)“守言”应作“言守”。(22)同(18)。(23)“街”应作“术”。(24)“敌”前脱一 

“归”字。(25)“事”应作“弃”。(26)“卒民”前脱字“吏”。(27)“欲”字后脱“以”字;“粟 

米”衍出一个“以”字。(28)“卒以”应作“以平”。(29)“诸”应作“请”。(30)“家”应作“皆”。 

(31)“侯”应作“候”。(32)“里巷”前脱一“入”字。(33)“厉缮夫”应作“缮厉矢”。(34)“有” 

应作“者”。(35)“者”应作“在”;“三”后脱一“老”字;“所”后衍一“差”字。(36)“官” 

应作“家”。(37)“同”应作“隔”。(38)“废”应作“发”。(39)“使”应作“吏”。(40)“不欲” 

应作“下效”。(41)“非擅”应作“擅非”。(42)“总”应作“纵”。(43)“谋”应作“课”。(44) 

“誉”作“举”。(45)“请”应作“诸”。(46)“人”后衍出一“食”字。(47)“酒肉”前漏脱一“赐” 

字。(48)“请”应作“谨”。(49)“非”应作“并”。(50)“无”应作“巫”。(51)“食”应作“令”。 

(52)“莼害”应作“薄者”。(53)“侯”应作“候”。(54)(55)同(53)。(56)“难”应作“杂”。 

(57)(58)(59)(60)(61)同(53)。(62)“禄”前疑脱一“爵”字。(63)同(53)。(64)“其亲之”为衍文 

在此。(65)“见”应作“令”。(66)“次”应作“欲”。(67)“罪人”前脱“赎出”二字。(68)“建” 

应作“逮”。(69)“之人”应作“人之”。(70)“口”应作“以”。(71)“平”后脱一“明”字。(72) 

“去”应作“表”。(73)“迹”应作“遮”;“击”后脱一“鼓”字。(74)“空”应作“宅”字。(75) 

同(74)。(76)“事”应作“吏”。(77)“忠”应作“中”。(78)“去”应作“士”。(79)“上目行” 

应作“三日徇”。(80)“铺”应作“鋪”。(81)“长”后疑脱一“者”字。(82)同(80)。(83)“室” 

应作“堂”。(84)同(80)。(85)“初”应作“勿”。(86)“食”应作“验”。(87)“节”应作“即”。 

(88)“请”应作“诘”。(89)“楼”前疑脱一“为”字。(90)“请”应作“诸”。(91)“杼”应作“抒”; 

“利”应作“罚”。 
[白话] 
建立国家的途径从利用地理条件开始,地理条件能够获得利用就能成 
功,地理条件不能获得利用就会劳而无功。人也是这样,不预先作好准备就 
无法安定国主,官吏、士兵和百姓不能同心同德,责任在于将领和官长;所 
有的赏赐和处罚,都应以王公的名义来确定。必须多次派遣使臣慰劳赏赐镇 
守边城、边关和边塞防备蛮夷而又劳苦的将士,并报告哪些镇守将帅的军费 
是有余还是不足,哪些地形应该派兵据守以及武器装备经常保持充足的将 
帅。对于边境地区的州县城市,根据那里树木生长不好就要少用木材,土地 

没有开垦就要节约粮食,没有大屋和草屋的地方就要少砍桑树。经济富裕, 
老百姓讲究吃喝。城内要构筑矮墙和行栈,城墙上要装置武器装备,守城的 
头目、士兵、炊事人员都要在城内各自的所属营区驻扎每十个人一个炊事员, 
掌管符信凭证的养吏一人,监察守护各城门,不允许无公事的人在守门人以 
及担任警察任务的人旁边逗留,不听从命令的人可以杀掉。每当敌人攻来, 
城邑在千丈以上的大城,一定要在城市郊区迎战敌人,守城一方才有利;城 
邑不够千丈的中小城市,不要出城迎敌,但要根据敌人的多少灵活应战,这 
些都是防守城池的大体原则。以上没有提到的,就根据心术智谋和人事策划 
参照处理。所有守城的一方都应以迅速歼灭敌人为上策,如果拖延持久,等 
到敌人的援兵到来,这是不懂得守城的方法。能懂得这些道理才能守城。 
守城的方法还有:敌人在离城百里之外的时候,守城将领就要把所有的 
官吏、小军官以及富人、贵戚的亲眷全部集中起来住到官府,谨慎地派可靠 
的部下保卫他们,越谨慎机密越好。 
等到敌人开始爬城墙强攻的时候,守城将领所在的兵营不得少于五百 
人,东西南北四个城门的将领一定要选择立过军功,以及为君王和国事效过 
死力而获得荣誉和官职的人担任,每人带兵一百人。每一方城门的将领如果 
兼守其他城门,就必须在另一城门上建立起高楼,派善于射箭的士卒守在那 
里,城上矮墙、冯垣一个一个排列起士兵守护着。让贵家子弟来守。 
每五十步建置一个贮藏兵器的“隔”,按照城中街巷分为八部,每部设 
置一个头目,每个头目带领四人,在城中要道和街巷中巡逻。街巷中老年人、 
少年人等没有参与守城的人和管理财物出入的人,按街巷分为四部,每部设 
一首领,让他们盘查来往行人中那些不按规定时间来往或有异常举动的人, 
以便及时发现和捉拿奸细。带士兵四人以上的头目去执行守城任务,大将一 
定要给予信符作为凭证;大将派人巡查守卫情况之时,拿有大将给的信符, 
对信符不合及口号不相应的人,伯和长以上官吏就一律把这种人扣押起来, 
并报告大将。应当扣押而不扣押,以及头目或士兵把人放跑了的,一律斩首。 
凡是触犯刑律犯有死罪以上的人,他们的父母、妻子儿女和兄弟都要抓起来。 
在城上防守的男子,每十人中,六人拿弓箭,其余四人拿其他兵器;参 
加防卫的女子、老人和少年每人执一矛。 
突然间有紧急事情,中军赶快击鼓三次,城上道路、城内街巷都要禁止 
通行,擅自通行的人要杀掉,女子参与大军行动时,男子走左边,女子走右 
边,不许并排一起行走。所有军民都要坚守各自的岗位,不听从命令的要杀 
掉。对擅自离开防守岗位的要三天查询一次,以防止作弊。街坊里正和居民 
中的年长的人都要守护各街巷进出口,部吏巡行到他的划分的地方,到进出 
口,里正开门接待部吏,陪同巡查各居民父老所守的岗位和小巷中偏僻无人 
的地方。生有外心、图谋通敌的奸民,处以车裂刑法杀掉。街坊里正和负责 
守护街巷的居民以及负责这一地方的部吏,没有预先发觉和抓获图谋通敌的 
人,一律处以死刑,如果能及时发现和抓获,免罪之外,每人还得到赏金四 
十八两。大将派亲信巡查每一个防守区域,夜长时每晚巡查五次,夜短时每 
晚巡查三次。防守四方的将领都要像大将一样巡查各自的区域,不执行命令 
的斩首。 
所有炉灶一定要砌上防火的屏围,烟囱要比屋顶高出四尺,小心慎重不 
要失火,第一次失火的人要杀掉,故意失火捣乱的人,用车裂的刑法处死, 
邻居不举报或不抓住纵火的人也要杀掉;如果能抓住就免于处罚。救火的人 

不许大声喊叫,如果故意大声喊叫以及擅自离开防守岗位去街巷救火的人, 
也要杀掉。失火地区的里正和居民,以及防守这一地方的部吏都要救火,部 
吏迅速派人报告大将,大将派遣亲信率领部下去救火。部吏隐瞒不向大将报 
告,也要杀掉。女子犯有死罪,因失火犯罪但并没有损害别人,以纵火捣乱 
罪论处。 
城邑被敌人围困,最重的禁令是:敌人突然来到,要严厉禁止官吏和百 
姓大声喊叫,不准三人以上聚集一堆,或两人以上一起奔跑、相视哭泣、对 
面流泪、打手势探问、互相指手划脚、互相呼唤、你拉我扯、互相斗殴撕打、 
互相争辩,以及擅自察看敌人动静,否则一律处以死刑。同在一起的人不能 
及时制止和报告,斩首;能及时报告和制止的,免罪。同伴中有人翻越城墙 
投敌,同伴没有及时抓住,斩首;伯长叛变投敌,队吏要斩首;队吏叛变投 
敌,队将要斩首。叛变投敌的人,他的父母、妻子、儿女、兄弟都要处以车 
裂死刑。如果事先发觉而未投敌的,免罪。因害怕敌人而临阵脱逃的,斩首; 
同在一起的人不能发现制止的,斩首;能及时发现和制止的,免罪。 
迅速战斗击溃了敌人,并使敌人败退后不能再次组织进攻的队伍,每队 
选出二名勇猛杀敌的士兵,给予最高的奖赏。而打败敌人,冲破敌人围城的 
队伍,使敌人离开城邑一里以上,封守城将为关内侯,赏赐土地三十里;副 
将按规定赐给上卿的官职,丞、吏以及原来官职相当于丞的人赐给五大夫的 
官爵,其他官吏、豪杰参与谋划坚守城邑的、士人和城上那些原来官职相当 
于五官的,都赐给公乘官位。参与守城的男子赐给爵位,每人升二级,女子 
赏钱五千,其余不分男女老少参与防守的,不赐予官爵每人赏钱一千,免除 
三年赋税。这些都是用以鼓励官吏和百姓坚守城池,打败敌人解除围困的措 
施。 
守卫守城主将官署大门的头目和士兵,每班岗不要多于两人,卫兵中勇 
敢的在前行,根据队、伍排列,让他们知道各自的左右前后是谁。擅自离开 
官署的人,杀掉。门尉每天白天点名三次,晚上击鼓关门后再点名一次,守 
将随时派人检查巡察,记上擅自离开岗位人的姓名。早晚两餐都在官署吃饭, 
不许在外面吃饭。守将一定要谨慎、细致地暗中观察侍从中的谒者、执盾、 
中涓以及料理日常生活的妇人等人的思想、心理、脸色、动作和言语的情况。 
每次端上饮食,一定要先叫人尝一尝再吃。若有异常情况,就立即抓起来予 
以盘问。守城主将对身边侍从中的谒者、执盾、中涓及料理日常生活的妇人 
有不满意的,就可下令杀掉,殴打或者捆绑他们,其他侍从不执行命令的或 
行动迟缓的,都要给予处罚。这些务必时时告诫他们。所有官署门前负责早 
晚警卫的人员,有的站有的坐,分别以年龄大小为次序,早晚值勤时,有功 
劳和能耐的,居先站上位或坐上座,其余则按次序站坐。官长每隔五天,将 
那些嬉戏不庄重,喜欢侵犯欺侮别人的卫兵的情况分别予以上报。 
所有人士、外来使者入城,一定要拿出凭证。将领外出归来和巡行回来, 
一定要先派人告知其家属,家属才出来迎接,再向守城主将报告后才返回自 
家。作为下级要经常体察上级,上级须去哪儿,都要跟随一起去。下级须跟 
从上级,上级却不必跟从下级。 
外来士卒为主人防守及为主人担任守卫,主人也要防备外来士卒。担负 
城中防卫任务的外来兵卒,假如他们原来所在城邑已被敌人攻陷,尤其要戎 
备他们,要反复核查他们的名册,不要让同属一个城的人共同防守一处地方。 
城上掌查台阶的守卫军吏要严格检查凭证,凭证相合才能进入,并慰劳之; 

凭证不合者,就将其扣留,并报知守城主将。 
晚上时,大鼓设置在主将的大门之内。在黄昏时,派出骑兵和使者拿着 
命符去传令关闭城门,使者必须手执令牌。黄昏时刻以鼓为号令,击鼓十下, 
所有城门路亭一律关闭,不让通行,对要通行者要先抓起来问明要通行的原 
因后再按罪行事。早晨时,打响大鼓放行,所有管城门的官吏自官署拿出钥 
匙,开完门后再交还钥匙。有特别符节凭信的人不在此禁之列。敌人前来进 
攻,城楼上击鼓五次,又向四周击鼓,有小鼓应和,表示各营队已响应城鼓。 
小鼓响了五下之后才集合的,斩首。有令必行,同时,号令一发出,立即派 
人随着省察号令可行与否。口号要注意,夜晚有联络的口号,口号不合的, 
处斩。制定戒严章程题上标题就称“某某章程”,在街道,大路台阶和城门 
上张帖公布,使往来行人都能看到从而照章行事。所有那些谋杀和伤害自己 
上级的官兵和百姓,一律按谋反罪处置;若能捉拿到谋杀长官之人者,赏金 
二十斤,并可免除处罚。越出职权范围擅自乱拿乱取,和滥用职权办非法之 
事的,砍头。一切擅自闯入其它区域的官吏、士兵和百姓,都要由所在的都 
司空和侯将其拘留,由侯报告守将;不将其拘留而擅自放人的,杀头。能捉 
拿一个谋反、出卖本城军政机密或越墙投敌的人,给予特权凭证,将来可以 
赦免两次死罪或判城旦罪四次。翻越城墙抛弃父母离开的,该人的父母、妻 
子、儿女…… 
全数查报百姓家的木材、砖瓦、石头等物的数目,登记其长短和大小。 
应查报而没有查报的,官吏问罪。所有居住在城里的官吏、士兵和百姓,要 
同他们的邻居结成联保联防,邻居犯罪而不知也有罪。如能亲自捉拿住犯罪 
人或将其报告给官府,都予以奖赏。不是联保联防内部的却了解到该联保组 
的犯罪活动而报告给有关官吏,都加倍给予奖赏。 
“令”负责城外守卫任务,守城主将担任城内的防守职责。令、丞、尉 
等官,其部下有人逃跑,如果抓回俘虏的人数与逃兵数相当,那么功罪可以 
两消;逃兵数超过所俘敌兵数十个的,令、丞、尉各减爵位两级;逃兵数超 
过所俘敌兵数一百的,令、丞、尉就须被撤职罢官,充作兵士,担负防守。 
抓来抵挡罪过的一定要是从敌军抓来的俘虏才算数。 
征募百姓的财物和粟米的,如百姓想交换种种器具,可按平价予以交换。 
城里居民的朋友或相识、兄弟有罪的,即便他们不在本城内但想用粟米物财 
赎罪出去的,法令都许可。上下传话的人员如此安排:每隔十步派一人,滞 
留或失职没传达到话的,要杀头。凡是可以便利办的事情应赶紧用书面向守 
城主将报告。官吏、兵士和百姓有要向上进言的,紧急通过传言人报知,官 
吏滞流或不代为传达的,要问杀罪。各县的豪杰、谋士、在家居住的大夫官 
员及人品忠厚的百姓人数,各县都要统计上报。 
官府、城下官吏、士兵和百姓都要参加左邻右舍的火灾联防。失火烧了 
自家或漫延到了别人的家,都要判罪。凡是仗势以强凌弱和强奸妇女的,喧 
哗打闹的,或擅自跑上城墙以及不按规定着装的,都一律交官府定罪惩罚。 
在各个城门和路亭,都要严格检查往来行人的凭证。凭证有问题和没有 
凭证的,要送到县廷,查问他们系谁派来。往来人中有凭证的妥善安排其住 
在官府,他们想要会见兄弟朋友,就替他们传呼召来,不能让他们自己进入 
城中街巷。如果他们想见城中三老、守闾等有身份的人,可以让三老、守闾 
先委托家中仆役代替应召来官舍相见。其他有事的人及职位低下者都不得擅 
自进入街巷之中。三老不能进入一般民众家里。须向街巷传令就用羽书,羽 

书收在三老家中。向一般民众传令就直接传到他们家去,失职没有传送或延 
迟命令的,要砍头。三老家中有看家的备办吃的。对于官吏、兵士和百姓擅 
自进入里巷和官府而没有凭证的,如有关官吏、三老以及守门者没有及时盘 
问和制止,都要定罪。 
所有偷盗守城器械、财物以及私人财物的,价值在一钱以上就要判罪。 
官员、兵士和百姓要将自己姓名写在帖上并张贴在各自办事的墙头上,守城 
主将视察时如发现有擅自进入别人办事处时,要问罪。城上每天都换发垫铺, 
垫铺规定可以彼此互相交换使用。若有知道他人私藏禁品却隐瞒不报者,也 
要判罪。 
若官员、兵士和百姓战死了,要赶紧召来死者家属,同司空一道将死者 
埋葬,不得久坐哭泣。受伤很重的让回家疗养,妥善照料,供医送药,每天 
赏其两升酒,两斤肉,并经常派官员前往探慰,如病情好转,就赶紧归队效 
力。假若是自己故意致伤欺骗官府以求逃避战斗的,罪连三族。战死者埋葬 
以后,守城主将要派官员亲自到死者家中,表示悲伤和哀悼。 
当敌人退走,战争结束后,全城举行赛神仪式,守城主将下令奖赏城中 
豪杰拼死战斗的所有有功之人,论功行赏,并亲自到死伤者家中慰问家人, 
哀悼死者,接见为守城而牺牲的遗属。城邑解围之后,守城主将应迅速派使 
者前往一线慰劳将士,守城主将本人要以身示范,敬重和爱护他们,使人人 
懂得尊重他们,从而使其对敌人结下仇恨。 
城上兵士和官吏也组成联保联防。有人在城内替敌人出谋划策,其父母、 
妻子、儿女、兄弟都要杀头。左临右舍知情不捉不报者,同犯罪人一样判罪。 
在城内的街巷居民也都要如此,奖惩一如城上。能够捉拿罪犯并向上报告的 
人,封给他一千家的食邑,如果举报捉拿之人不属罪犯联保联防组的,就封 
给他二千家的食邑。 
守城的禁令:官吏、兵士和百姓仿效制作敌人的服饰的和军门旗帜的, 
杀;不服从军令的,杀;擅发号令的,杀;延误军令的,杀;靠着战戟悬身 
下城,上城下城不与众人配合的,杀;不是响应号令而胡叫乱喊的,杀;放 
走罪犯遗失公物的,杀;长他人威风灭我志气的,杀;擅离职守,聚众瞎谈 
的,杀;听到城墙鼓声却在应鼓击过五次之后才赶往办事地点的,杀;每个 
人都要将自己的姓名写在板上,挂在各自的办事处墙头,守城主将必须亲自 
验查他们所到先后,对不在某办事点却擅自进入的,杀;带领手下人离开自 
己的办事处进入别人的办事处,而该处办事人员不予捉拿;挟拿私人书信, 
替人请托成私的;弃城防事去干私事的;偷取他人妻子婴儿的,统统杀头, 
不予赦免。被偷取的妻子经人举报按法籍没。没有凭证却在军中乱窜的,杀; 
故意美化敌人:敌人兵将少而说成多,军纪混乱却说整肃,敌人进攻办法愚 
蠢却说巧妙的,杀;主人不得与陌生人交谈并借东西给他;敌人用箭射来书 
信,不得去捡拿;敌人向城内故示伪善,不得有人表示响应,不从禁令的, 
杀;禁令规定不得捡拿敌人射来的信物,城内也不得将书信射给敌人,触犯 
这条禁令的,父母,妻儿都要杀头,尸体还要挂城示众。抓获并报告有人向 
敌人射信或捡取敌人信物情况的人,赏金二十斤。只有守城主将和他发给了 
凭证干公差的人,才能在禁止通行的时间行走。 
守城主将守城,务必谨慎查询城中父老、官吏和大夫,以及互相有仇怨 
并无法消除的人,召见他们双方,讲明道理和利害,消除前嫌,一致对外; 
守城主将同时定要将他们的名字专门记下,不让其居住在一起或安排在一起 

共事。如果因私仇私怨而妨碍守城公务的,父母、妻子和儿女统统杀掉。那 
些身在城内却为城外敌军出谋划策的,灭三族。对于那些事先发觉或捉拿罪 
人上报的,赏封他同该城邑一样大小的城邑,守城主将还要授他官印,给他 
尊宠的官职,并广喻人知,要经常召请那些与诸侯有广泛结交的豪杰之士, 
使上级官吏都认识他们,妥善存恤他们,所在地方官要经常安排宴请他们, 
叫他们不得擅自出入并取他们作为人质。乡镇中的长老、父老、豪杰之士的 
亲戚、妻儿一定要给予尊重和爱护。假若他们属贫苦人,难以维持生活,官 
长要给予吃的。对于那些勇士的父母、亲戚、妻子、儿女,要经常赐给酒肉, 
敬重他们,将他们的住宿安排在靠近守城主将官署的地方。守城主将的官署 
楼居高临下对着人质居住的房舍,要周密防卫,楼务必密密地涂上泥,使得 
署楼上看得清署楼下,而署楼下却看不见楼上,不知道楼上是否有人。 
守城主将身边的人:选用在主将身边工作的官员一定要正派廉洁,忠诚 
可靠,正直无私,并且有能力承担事务的人。不要限制他们的饮食酒肉,金 
钱、布匹等财物各自保管,谨防盗窃。葆宫的围墙一定要修三道,在围墙的 
外垣上守卫应堆上破瓦烂锅之类的东西。城门设主管官员,负责城门和里巷 
的门,开锁和上锁都必须有守城主将所给的凭证。葆宫的守卫一定要选拔忠 
厚的卫兵担当,官吏也须挑选忠诚可靠、公正而又力能胜任的人。 
象令、将一级的官长要自行护卫,在官署和住处四周要环绕十尺高的围 
墙,守上大门和闺门,卫兵要一并守卫司马门。 
供占望吉凶的巫师卜师居住的地方务必要靠近守城主将的住所,巫师所 
住一定要靠近神社,必须将神灵当神灵敬重。他们务必将吉利的话告诉全城 
百姓,把占得的实际情形报告给守城主将,让守城主将一人知道就够了。如 
若巫师和卜师胡编不吉利的话使百姓惊恐不安,就杀无赦。 
估计到粮食不足,就让百姓自己估算能缴纳用作军粮的五谷数量,规定 
缴纳日期,登簿记帐、官吏偿付相当价格的钱物。若过了期限还隐藏不缴。 
或者还未全部交清,就派官员和兵士暗中搜求,如果搜出隐粮不缴者,给予 
判罪。有能知情举报的,官府赏给所藏粮食的十分之三。征收好的粟米、布 
帛、金钱、牲畜,都要公正估价,给主人开具征收证明,写清征收的数量和 
价值,战事完结后,一律按原价值双倍偿付。想作官的,还可根据应征财物 
当时的价格和数量赐给官做;不愿做官的人,依法还可准允其接受爵位,或 
赎出犯罪的亲戚,朋友。那些接受赏赐的人,让他们进葆宫接受接见,表示 
亲信爱护,能偿付征收品的财物再度捐献帮助官长的,就加倍赐予爵禄。缴 
纳单的格式如下:某县某里某人家里人口两个,存积粟米六百担;或某里某 
人人口十个,积存粟米百担。缴纳粟米财物有确定的日期,过期不纳的没收 
为王公所有。有知隐藏不交实情上报给官府的,将查出隐粮的十分之三赏给 
他。要小心谨慎,不可让百姓弄清我军存积多少粮食。 
守城主将一入城,就要开始挑选侦探。物色到充当侦探之人就把他接到 
宫里养起来,但万不可让他了解我方守卫的设施装备。侦探要互相隔离居住, 
他们的父母、妻儿同他们本人住在一块,赐给衣服、食物、酒肉,派人好好 
招待他们。侦探回来交差,要接受问询。守城主将的住房有三层,在外围墙 
的四角筑楼,内围墙也建楼,楼与葆宫相接一丈五尺修成上下复道。葆宫不 
砌内室。每隔三天发放一次垫席垫草,大略检查一下,把茅草铺在宫中,厚 
三尺以上。派侦探出城,一定要派乡镇中忠实可靠的厚重之士,其家中必须 
有父母妻儿。侦探出城要供给足够的钱。一定要反复地派遣侦探,安排供养 

好他们的家人。对于侦探要隔离居住,不要与众人同住一屋,同时供给他们 
好吃的食物。派遣别处的侦探,所给予的钱物须与前一个侦探相同。侦探回 
来后,对前后二人提供的情报参照核实,如果确实可信,要优厚地奖赐他们。 
如果三次派出侦察,所获情报无出入,都确实可信,就加重奖赏他。不愿受 
赏而愿做官的,给予二百石的官阶,守城主将授给官印。不愿做官而愿受赏 
的,爵禄同前一样。能够有能力深入敌人国都去探察情报的,如果确系确实 
可信,对于该侦探的赏赐要加倍,若他不愿受赏而愿做官,赐三百石的官阶。 
对于那些保卫城池功劳卓著的勇士,守城主将一定要亲自将赏赐品送往勇士 
父母住的地方,叫他们亲睹主将对他的恩宠。对那些把赏赐再度捐献给国家 
辅助长官的,所给奖赏、爵禄或赎出罪人的数量分别加倍。 
要派出警戒兵,但不要超出十里之外,在地势较高而又方便的地方树立 
标志,派三人看守。从最远的地方到城邑共树立三处标志,同城上烽火遥遥 
相望。白天就烧烟,晚上就点火。弄清了敌人来的方向和时间后,周密分析 
敌我形势可战与否,若考虑到城小难以守住交通要道,就要将老人小孩、粟 
米、牲畜等全部护送进城。一次派出警戒兵不要超过五十人。当敌兵攻到外 
城短墙地段时,警戒兵就马上撤入城中,不要滞留城外。警戒兵总数不必超 
过三百人,天黑时派他们出城,戴上军徽标记。要派人到行人经常路过的道 
路和重要关塞去察看路上所留下的踪迹,每人都树立向城上报点情况的标 
志,而城上对他们会作出相应的反应。出城侦察的警戒兵用标记向城内报告 
情况,城内的警戒兵坐守在郭门内外,也树立联络标记,命令兵士一半在郭 
门内,一半在郭门外,使敌人无法知晓人数多少。一旦有紧急军情,见敌兵 
越过田表,城上就以旗号指挥警戒兵,击鼓,整旗,预备战斗,一切都按城 
上指挥行事。看得见敌军,就点一堆烽烟;敌军进入我方境界,就点两堆烽 
烟;当敌军接近外城时就点三堆烽烟;一旦敌军进入外城内,就点燃四堆烽 
烟;敌军接近大城墙凡点五堆烽烟。夜晚时就点烽火,报告敌情的烽火数目 
与烽烟相同。 
在离外城一百步之内,所有墙和树木,不分高低大小全部拆除或砍掉。 
城外的井也要全部填塞,使敌人无法打水。城外的空屋子全部拆毁,树木尽 
伐。一切可以用作攻城的东西都运进城内,令人登记在册。战事结束后,再 
按所记数目各自领取。官员要给他们发收条,写清件数。那些不能全部运进 
城的当路木材,就就地烧掉,不致落入敌军之手供其使用。 
每个人都要将自己的姓名写好,贴在办事处所。官员公布处罚条规:凡 
纵淫欲的,用箭射穿他的耳朵;蛮骄无理欺凌正派人,吵吵闹闹不休不止, 
在道路上有意阻碍过往行人,分派工作拖拖拉拉,不按时就班,又不请假, 
也用箭射穿他的耳朵。狂呼乱叫惊忧百姓,那是死罪。不向上官进谏却背后 
非议,任意发表不利言论,论罪该杀。军伍中不准奏乐下棋、违令者判罚用 
箭穿耳。没有上级命令,不准驾车奔跑,犯罪则以箭穿耳。违者判罪以箭穿 
耳。也不准放纵牛马,违者判罪以箭穿耳。有不按时饮食的,判以箭穿耳, 
不准在军中唱歌、号哭,违令判以箭穿耳。传令各级官吏切实执行刑罚条规, 
该杀的一律杀掉,有罪却不处罚,官吏连同罪犯一起处罚。如果让罪犯逃走, 
就杀掉放走罪犯之人。凡是不能使兵士按规定作战的将官,都要杀头。如若 
官吏没有使兵士和百姓知晓军中禁令,一旦有人犯法,官吏代为服罪。凡是 
有人因犯罪在街上被斩首,就要陈尸示众三天。 
安排两排士兵守卫在守城主将的门外,让他们夹门而坐,早晚用餐时轮 

班接换,不能有空缺。门卫设一头领,守城主将应经常派他检查逃离的士兵, 
以此督促门尉和官长,并报告逃离者的姓名,安排四个士兵分两边夹守城主 
将门内坐,二人夹散门外坐,有人来见主将,卫兵应立即拿起武器迎上前去 
盘查。早晚开饭时换人接替,报告卫兵的姓名。 
守城主将堂下或高楼中都安排有观察情况的人,他们一望见有乘车和骑 
兵从道外到来,以及城中异常情况,立即报告给主将知道,守城主将等候城 
门上观察兵和县邑官吏的报告互相参验,了解情况。主将楼下的人将楼上观 
察员的话传报给主将本人。 
负责传话给守城主将的侍从称“中涓”,中涓有两名,夹着散门内坐, 
平时关着门,早晚开饭时轮换。中涓中要有一位年长之人。 
环绕守城主将宫室的大道要修筑夹道,在两边分别筑起墙,墙有一丈高; 
设置观察台,但不要象安鸡脚驾那样,以便监视葆舍,每收到信件文书都一 
定要谨慎地参证其他情报,如有不合军法之处就要询问或予以修正。在夹道、 
墙外大路、街道都要建有高楼,居高临下立在城巷中。楼上备有一鼓和垄灶, 
如有事故就击鼓,等官吏赶到时才停止。夜晚用火光指示事故地点。 
城下每五十步建一个厕所,上下厕所共用一个茅坑,安排所有有过失却 
又不必杀头的人去打扫,以示惩罚。 

\chapter{四十九  杂守}

杂守(1)

禽子问曰:“客众而勇,轻意见威,以骇主人;薪土俱上,以为羊坽(2), 
积土为高,以临民(3),蒙橹俱前,遂属之城,兵弩俱上,为之奈何?” 
子墨子曰:子问羊坽之守邪(4)?羊坽者(5),攻之拙者也,足以劳卒, 
不足以害城。羊坽之政(6),远攻则远害,近城则近害(7),不至城(8)。矢石 
无休,左右趣射,兰为柱后(9),□望以固。厉吾锐卒,慎无使顾,守者重下, 
攻者轻去。养勇高奋,民心百倍,多执数少(10),卒乃不怠。 
作士不休(11),不能禁御,遂属之城,以御云梯之法应之。凡待烟(12) 
冲、云梯、临之法,必应城以御之(13),曰不足,则以木椁之。左百步,右 
百步,繁下矢、石、沙、炭(14),以雨之,薪火、水汤以济之。选厉锐卒, 
慎无使顾,审赏行罚,以静为故,从之以急,无使生虑。恚■高愤(15),民 
心百倍,多执数赏,卒乃不怠。冲、临、梯皆以冲冲之。 
渠长丈五尺,其埋者三尺,矢长丈二尺(16)。渠广丈六尺,其弟丈二尺 
(17),渠之垂者四尺。树渠无傅叶五寸(18),梯渠十丈一梯,渠、荅大数, 
里二百五十八(19),渠、荅百二十九。诸外道可要塞以难寇,其甚害者为筑 
三亭,亭三隅,织女之,令能相救。诸距阜、山林、沟渎、丘陵、阡陌、郭 
门若阎术,可要塞及为微职,可以迹知往来者少多即所伏藏之处。 
葆民,先举城中官府、民宅、室署,大小调处,葆者或欲从兄弟、知识 
者许之。外宅粟米、畜产、财物诸可以佐城者,送入城中,事即急,则使积 
门内。民献粟米、布帛、金钱、牛马、畜产,皆为置平贾,与主券书之。 
使人各得其所长,天下事当;钧其分职,天下事得;皆其所喜,天下事 
备;强弱有数,天下事具矣。 
筑邮亭者圜之,高三丈以上,令侍杀(20)。为辟梯,梯两臂,长三尺, 
连门三尺(21),报以绳连之。椠再杂(22),为县梁。聋灶(23),亭一鼓。寇 
烽、惊烽、乱烽,传火以次应之,至主国止,其事急者引而上下之。烽火以 
举,辄五鼓传,又以火属之,言寇所从来者少多,旦弇还去来属次(24),烽 
勿罢。望见寇,举一烽;入境,举二烽;射妻,举三烽一蓝;郭会,举四烽 
二蓝;城会,举五烽五蓝;夜以火,如此数。守烽者事急。 
候无过五十,寇至叶(25),随去之,唯弇逮(26)。日暮出之,令皆为微 
职。距阜、山林皆令可以迹,平明而迹,无(27),迹各立其表(28),下城之 
应(29)。候出置田表,斥坐郭内外,立旗帜,卒半在内,令多少无可知。即 
有惊,举孔表(30),见寇,举牧表(31)。城上以麾指之,斥步鼓整旗,旗以 
备战从麾所指(32)。田者男子以战备从斥,女子亟走入。即见放(33),到(34), 
传到城止。守表者三人,更立捶表而望,守数令骑若吏行旁视,有以知为所 
为(35)。其曹一鼓。望见寇,鼓,传到城止。 
斗食,终岁三十六石;参食,终岁二十四石;四食,终岁十八石;五食, 
终岁十四石四斗;六食,终岁十二石。斗食食五升,参食食参升小半,四食 
食二升半,五食食二升,六食食一升大半,日再食。救死之时,日二升者二 
十日,日三升者三十日,日四升者四十日,如是而民免于九十日之约矣。 
寇近,亟收诸杂乡金器若铜铁及他可以左守事者(36)。先举县官室居、 
官府不急者,材之大小长短及凡数,即急先发。寇薄,发屋,伐木,虽有请 
谒,勿听。入柴,勿积鱼鳞簪,当队,令易取也。材木不能尽入者,燔之, 
无令寇得用之。积木,各以长短、大小、恶美形相从。城四面外各积其内, 

诸木大者皆以为关鼻,乃积聚之。 
城守,司马以上父母、昆弟、妻子有质在主所,乃可以坚守。署都司空, 
大城四人,候二人,县候面一,亭尉、次司空、亭一人。吏侍守所者财足廉 
信,父母、昆弟、妻子有在葆宫中者,乃得为侍吏。诸吏必有质,乃得任事。 
守大门者二人,夹门而立,令行者趣其外。各四戟,夹门立,而其人坐其下。 
吏日五阅之,上逋者名。 
池外廉有要有害,必为疑人,令往来行夜者射之,谋其疏者(37)。墙外 
水中为竹箭,箭尺广二步,箭下于水五寸,杂长短,前外廉三行,外外乡, 
内亦内乡。三十步一弩庐,庐广十尺,袤丈二尺。 
队有急,极发其近者往佐,其次袭其处。 
守节:出入使,主节必疏书,署其情,令若其事,而须其还报以剑验之 
(38)。节出:使所出门者,辄言节出时操者名。 
百步一队。 
閤通守舍,相错穿室。治复道,为筑墉,墉善其上。 
取疏:令民家有三年畜蔬食,以备湛旱、岁不为。常令边县豫种畜芫、 
芸(39)、乌喙、袾叶(40),外宅沟井可填塞,不可,置此其中安则示以危, 
危示以安。 
寇至,诸门户令皆凿而类窍之,各为二类,一凿而属绳,绳长四尺,大 
如指。寇至,先杀牛、羊、鸡、狗、乌(41)、雁,收其皮革、筋、角、脂、 
脑、羽。彘皆剥之。吏橝桐■(42),为铁錍,厚简为衡枉(43)。事急,卒不 
可远,令掘外宅林。谋多少(44),若治城□为击,三隅之。重五斤已上,诸 
林木(45),渥水中,无过一茷。涂茅屋若积薪者,厚五寸已上。吏各举其步 
界中财物可以左守备者上。 
有谗人,有利人,有恶人,有善人,有长人,有谋士,有勇士,有巧士, 
有使士,有内人者,外人者,有善人者,有善门人者,守必察其所以然者, 
应名乃内之。民相恶若议吏,吏所解,皆札书藏之,以须告之至以参验之(46)。 
睨者小五尺,不可卒者,为署吏,令给事官府若舍。 
蔺石、厉矢诸材器用皆谨部,各有积分数。为解车以枱(47),城矣以轺 
车(48),轮轱广十尺,辕长丈,为三辐,广三尺。为板箱,长与辕等,高四 
尺,善盖上治,令可载矢。 
子墨子曰:凡不守者有五:城大人少,一不守也;城小人众,二不守也; 
人众食寡,三不守也;市去城远,四不守也;畜积在外,富人在虚,五不守 
也。率万家而城方三里。 


[注释] 

(1)《杂守》是墨子研究城池防守战术的篇章之一。主要说明前文所述各种具体防守战术之外的 

其它方法和注意事项。比较复杂,但也具有综论性质。(2)“坽”应作“坽”。(3)“民”前疑脱一“吾” 

字。(4)(5)同(2)。(6)“坽”应作“坽”;“政”应作“攻”。(7)“城”应作“攻”。(8)“不至城” 

前疑脱“害”。(9)“兰”应作“蔺”。(10)“少”应作“赏”。(11)“士”应作“土”。(12)“烟” 

应作“堙”。(13)“应”应作“广”。(14)“炭”应作“灰”。(15)“恚”应作“恙”;“■”应作 

“恿”。(16)“矢”应伯“夫”。(17)“弟”应作“梯”。(18)“叶”应作“堞”。(19)“二百五十 

八”后脱一“步”字。(20)“侍”应作“倚”。(21)“门”应作“版”。(22)“椠”应作“堑”。(23) 

“聋”应作“垄”。(24)“旦”应作“毋”;“还”应作“逮”。(25)“叶”应作“堞”。(26)“唯” 

应作“无”。(27)“无”前疑脱“迹者”;“无”后疑脱“下里三人”。(28)“迹”衍误在此。(29) 

“下城”应作“城上”;“之应”应作“应之”。(30)“孔”应作“外”。(31)“牧”应作“次表”。 

(32)“旗”衍误在此;“备战”应为“战备”。(33)“放”应作“冠”。(34)“到”应作“鼓”。(35) 

“为”应作“其”。(36)“杂”应作“离”。(37)“谋”应作“诛”。(38)“剑”应作“参”。(39) 

“芸”应作“芒”。(40)“袾”应作“椒”。(41)“乌”应作“凫”。(42)“橝”应作“粟”。(43) 

“枉”应作“柱”。(44)“谋”应作“课”。(45)“林”应作“材”。(46)“告”字后脱一“者”字。 

(47)“解”应作“轺”。(48)“城矣”应作“盛矢”。 
[白话] 
禽滑厘问道:“敌人人多势众而勇猛,骄豪显威,威吓守方;木头土石 
一起用上,筑成名叫‘羊坽’的土山,堆积土石筑成高台,对我方构成居高 
临下之势,敌兵以大盾牌为掩护从高台猛攻下来,一下子就接近了我方城头, 
刀箭齐上,这时候该怎么对付呢?” 
墨子先生回答说:你问的是对付“羊坽”进攻的防守办法吗?羊坽这种 
攻城方法是进攻的蠢法子,只会导致进攻一方士兵的疲劳,不足以构成对守 
城一方的危害。敌人用羊坽进攻,远攻就以远攻的办法对付它,近攻就以近 
攻的方法对抗它,不会对守城一方造成危害。箭和擂石不停地从左右西边急 
速地发射,擂石接后,□望以固。激励精兵,谨慎而又不产生顾虑,守城的 
兵士个个敬重打退敌人的人,攻击敌人的兵士鄙视离开战斗岗位的人,培养 
兵士高昂的士气,民心百倍加强,多捉拿敌人就多奖赏,这样兵士就不会懈 
怠。 
假若敌兵不断筑土堆造成高台以便爬攻城墙,不能遭到有效地阻挡,一 
下子就接近了我方城头,这时我方就用防御云梯攻城的办法予以对付。对于 
敌人填塞护城河,冲车攻城、云梯爬城加筑不够高厚或时间来不及,就用木 
材加高加固,木椁尺寸为左边百步,右边百步。用弓箭、石头、沙子、土灰 
象雨点一样频繁地往下攻击敌兵,又用火把、开水助战,再挑选激励兵士, 
增强锐气,千万注意不要使士兵有所顾虑。赏罚要分明,以镇静为上但又须 
当机立断,不使发生变故。培养高昂的士气,使民心百倍增强,多抓俘虏多 
给奖赏,兵士不致懈怠。冲车、高临、云梯都可以用冲机撞击它们。 
渠柱长一丈五尺,埋三尺在地下,上端长一丈二尺。渠宽一丈六尺,梯 
长一丈二尺,渠下垂部分四尺。将渠树立时不要靠在矫墙上,要离开五寸; 
梯渠十丈一梯,渠和荅大约是一里二百五十八步,渠、荅共一百二十九具。 
城外各种交通路口,可以筑起要塞阻挡敌人,在极为要害的地方可筑三个了 
望亭,三亭的位置按织女三星构成三角形,使三个亭之间可以互相救援。在 
各种大土山、山林、河沟、丘陵田野、城郭门户和里门要道,可以筑要塞立 
标志,以此了解敌情,根据敌人留下的踪迹推知往来人数多寡和敌兵埋伏的 
地方。 
疏散民众,先取城中官府、民房、内室、外厅,按大小分派居住,被疏 
散的人准许兄弟朋友住在一起,外面的粮食、牲畜等所有可以帮助守城的财 
物,统统都送入城里,如情况紧急,就堆在城门内。对于百姓所缴纳的粮食、 
布匹、金钱、牛马牲畜,都一律要公平核价,给予收据,写清数量价值。 
让人们各尽所能,天下的事情就能办妥;各负其责,职责均衡,天下的 
事情就办得合理;分派的工作都是各人所爱,天下的事情就完备了;强弱有 
定数,天下的事情就没有遗漏了。 
建造供守望敌人用的邮亭要做成圆形的,三丈高以上,顶部呈斜尖形状。 
设置双柱梯子,宽三尺,每级梯板相距三尺,将梯板和双柱用绳子扎起来。 

修濠沟要修成内外两圈,架上悬梁。再安置垄灶,每个亭子备一鼓。报告敌 
人来进攻时点燃的烽火,情况十分紧急时的烽火,混战的烽火,情况不一, 
要依次传火,直至传到国都为止。假如军情紧急异常,还要上下牵引烽火。 
烽火点燃后,就先用鼓击五次传板,接着以烽火报告敌人的来向和人数的多 
少,切不可淹滞误事。敌人来了又去,去了又来,烽火不要熄灭。刚望得见 
敌兵,燃一堆烽烟;敌人已入境,烧两堆烽烟;敌人距离外城只一箭之地了, 
烧三堆烽烟再加烧一个大柴筐;敌人都聚集在外城,烧四堆加烧两个大柴筐; 
敌人若聚集到城墙下,则烧五堆烽烟加上五个大柴筐。夜晚时就用烽火代替 
烽烟,数目同上数相同。 
派出警戒兵时,每次不要超过五十名,若敌人到达外面矮墙,应赶紧离 
开入城去,不要滞留。天黑派兵出城,务必佩戴徽章标志。一切可以探察敌 
人踪迹的地方如大土山,山林等地,天亮时都要派人探察,要探察的地段, 
每里路派出者不能少于三人,他们各自都要树立标志向城上报告,城上看到 
标记则作出相应的反应。警戒兵出城立田表,城内警戒兵令其坐在郭内外, 
竖起旗帜,城内的警戒兵一半在郭内,使警戒兵的数目外人无法得知。一旦 
有紧急情况,就举“外表”,看得见敌人就举“次表”。城上用旗号指挥, 
警戒兵击鼓竖旗、预备战斗,都要按城上的指挥行动。在城外田野里劳动的 
男子应跟随警戒兵一起作战,女人便赶紧入城。如果见到敌人就赶紧击鼓, 
直到传到城上为止。守联络标志的三个人,还要立烽火烽烟标志和观望别的 
地方的标志。守城的主将要不断地派出骑兵和官吏到处巡视,了解他们的行 
动。守标志的警戒兵备有一鼓,望见敌人,依次出鼓报告,直到传到城上时 
为止。 
计每天吃一斗粮,一年则吃三十六担;计每天吃三分之二斗,一年则吃 
二十四担四斗;计每天六分之二斗,则一年吃十八担;计每天吃五分之二斗, 
则一年吃十四担四斗;计每天吃六分之二斗,则一年吃十二担;计每天吃一 
斗,则每餐吃五升;计每天吃三分之二斗,则每餐吃三升又一小半升;计每 
天吃四分之二斗,则每餐吃二升半;计每天吃五分之二斗,则每餐吃二升; 
计每天吃六分之二斗,则每餐吃一升加大半升;每日吃两餐。粮食十分紧缺 
的时期,每人每天按二升吃二十天,每天三升吃三十天,每天四升的吃四十 
天,照这样推算和实施,每人只要节约九十天,就有一个老百姓不致饿死。 
如敌兵逼近,就加紧收集偏远地区的金器,铜铁及其它可以用来帮助守 
城用的物品,先调查登记县中官吏、官府中不急需用的物品、木材大小、长 
短及总数,赶紧先发送进城。敌人一接近,就摧毁房舍,砍伐树木,即使有 
人求情也不能依从。运进城里的柴草,不要象鱼鳞一样一片压一片地堆放, 
要堆到当路的地方,以便于拿取。不能全数运进城的木材就就地烧掉,不要 
让其落入敌手。堆放木材,分别按长短、大小、好坏和曲直堆放。城外四面 
运来的财物仍各按四面堆放在城内,所有大木头都要凿好孔穴,以便搬运到 
一起。 
守卫城池的官吏,职位在司马以上的,父母、兄弟、妻子和儿女有人质 
留在主帅府,才可以坚守。任命都司空、大城四人;候二人,县候,城四面 
各有一人。亭尉,次司空,每亭一人。在守城主将衙署中任职的官吏,要选 
择有才能足以任事。廉洁而诚实、父母兄弟妻子儿女有在葆宫中的人,才能 
担任侍吏。所有官吏都一定要留有人质,才能让他承担任务。守卫城防大门 
的二个卫士,夹门站着,促行人快步走开,每个城门有四把戟、夹门放着, 

卫兵坐在戟下面。头目每天巡检五次,报告逃离卫兵的姓名。 
壕池外边岸上的要害之处,如果发现有可疑之人,则命令往来巡夜的士 
兵向其射箭,对疏忽大意者,斩首。城外的水中插上竹箭,插竹箭的地方宽 
一丈二尺,箭插入水中要比水面低五寸以上,长短错杂,前排外边三行,外 
边的竹箭尖向外斜,内边的竹箭尖向内斜。每隔三十步修座房子,收藏弓箭, 
房子宽十尺,长一丈二尺。 
哪一个部队有紧急情况,就立即派就近的其它部队前去增援,又拨出次 
近的部队去接替防务。 
守城主将发出的符节凭证:凡是出入的使者,掌管凭证的官吏一定要书 
写记录在案,记载其详情,等他回报时予以验证。凭证发出:使者凭证出门, 
无论从某门经过,一律要向上报告凭证出门的时间和拿凭证人的姓名。 
每一百步远布置一支队。 
主将衙门的边门与守城主将的房舍相通,旁门互相交错穿插;修建上下 
复道,筑好墙,在墙上垒放破瓦等物。 
要贮存蔬菜食物:使百姓家贮存的蔬菜粮食供够三年吃,用来防备水旱 
天灾和没有收成的年景。要经常让边远县预种一些芫华、莽草、乌头、椒叶 
等毒性植物,外宅的水沟水井可以填掉,不能填掉的就将上述毒性植物投入 
其中。在和平安定的时期,要向百姓说明战争存在的危险,战乱期间则要向 
百姓讲明从杀敌中求取和平安定。 
敌人打来时,所有的门户都要凿上两种孔,其中一种孔是用来穿绳子用 
的。绳子长四尺,指头大小。敌人打来了,就先杀掉牛、羊、鸡、狗、凫、 
雁等家畜家禽,并收集这些牲畜的皮革,筋骨、角、油脂、脑、羽毛。猪都 
要剥下皮。官吏们选取槚木,桐木,栗木制成铁錍,厚的木料就选做横柱。 
如情况紧急仓猝之间无法从远地弄来,就命令就地取材、挖掘外宅的林木, 
按修缮城墙和攻敌所需的三倍量征收。将重五斤以上的木材浸入水中,数量 
不可超过一排。用泥涂抹房屋顶和堆积的柴草,泥巴所涂的厚度要有五寸以 
上。各级地方官吏都要调查和征收所辖地区内可用以辅助打仗的财物上交。 
世上有谗间之人,有好利之人,有恶人有善人,有具有专长的人,有谋 
士,有勇士,有巧士,有使士,有能容人者,有不能容人者,有善于待人的 
人,有善于守门的人,守城主将务必要考察他们为何具备那种品性或特长, 
名符其实的便接纳使用。百姓们彼此仇恨或对官吏提出控告、及其被告的辩 
护,都要一起记录在案并收存起来,以待控告人到来时用以参考验证。那些 
身高仅五尺不能当兵的人,就让其在官府中当差或者让他们在官府和个人家 
里服务尽责。 
所有防守用的军事器材如擂石,锋利的箭等,都要小心部署,并且分别 
要有存放的固定数目。用枱木制造轺车装载弓箭,车辕长一丈,有轮子三个, 
轮与轮之间宽六尺。拼造车箱,车箱长度和车辕一样长,高度为四尺,要好 
好地给车箱加上盖子,并把车箱的里面修治整齐,使它能够多装弓箭。 
墨子说:不便防守的情形有五种:城太大而守城人数少,这是第一种不 
便防守的情形;城太小而城内军民却太多,这是第二种不便防守的情形;人 
多而粮食少,这是第三种不便防守的情形;集市离城太远,这是第四种不便 
防守的情形;储备屯积的守城物质在城外,富裕的百姓也不在城中,这是第 
五种不便防守的情形。大概说起来,城中居民一万家,城邑方圆三里,这种 
情形可以坚守。 

\chapter{附录: 一、经上}

附录: 一、经上

故,所得而后成也。 
止,以久也。 
体,分于兼也。 
必,不已也。 
知,材也。 
平,同高也。 
虑,求也。 
同长,以正相尽也。 
知,接也。 
中,同长也。 
■,明也。 
厚,有所大也。 
仁,体爱也。 
日中,正南也。 
义,利也。 
直,参也。 
礼,敬也。 
圜,一中同长也。 
行,为也。 
方,柱隅四讙也。 
实,荣也。 
倍,为二也。 
忠,以为利而强低也。 
端,体之无序而最前者也。 
孝,利亲也。 
有间,中也。 
信,言合于意也。 
间,不及旁也。 
佴,自作也。 
,间虚也。 
■,作嗛也。 
盈,莫不有也。 
廉,作非也。 
坚白,不相外也。 
令,不为所作也。 
撄,相得也。 
任,士损己而益所为也。 
似,有以相撄,有不相撄也。 
勇,志之所以敢也。 
次,无间而不撄撄也。 
力,刑之所以奋也。 

法,所若而然也。 
生,刑与知处也。 
佴,所然也。 
卧,知无知也。 
说,所以明也。 
梦,卧而以为然也。 
攸,不可两不可也。 
平,知无欲恶也。 
辩,争彼也。辩胜,当也。 
利,所得而喜也。 
为,穷知而县于欲也。 
害,所得而恶也。 
已,成、亡。 
治,求得也。 
使,谓、故。 
誉,明美也。 
名,达、类、私。 
诽,明恶也。 
谓,移、举、加。 
举,拟实也。 
知,闻、说、亲、名、实、合、为。 
言,出举也。 
闻,传亲。 
且,言然也。 
见,体、尽。 
君,臣、萌通约也。 
合,正、宜、必。 
功,利名也。 
欲正,权利;且恶正,权害。 
赏,上报下之功也。 
为,存、亡、易、荡、治、化。 
罪,犯禁也。 
同,重、体、合、类。 
罚,上报下之罪也。 
异,二、不体、不合、不类。 
同、异而俱于之一也。 
同、异交得放有、无。 
久,弥异时也。 
宇,弥异所也。 
闻,耳之聪也。 
穷,或有前不容尺也。 
循所闻而得其意,心之察也。 
尽,莫不然也。 
言,口之利也。 

始,当时也。 
执所言而意得见,心之辩也。 
化,征易也。 
诺,不一利用。 
损,偏去也。 
服,执誽、音利。 
巧转,则求其故。 
大益。 
儇,■秪。 
法同,则观其同。 
库,易也。 
法异,则观其宜。 
动,或从也。 
止,因以别道。 
读此书旁行,正无非。 


\chapter{二、经下}

二、经下

止,类以行人。说在同。 
所存与者,于存与孰存?驷异说。 
推类之难。说在之大小。 
五行毋常胜。说在宜。 
物尽同名:二与斗,爱,食与招,白与视,丽与,夫与履。 
一,偏弃之,谓而固是也。说在因。 
不可偏去而二。说在见与俱、一与二、广与修。 
无“欲、恶之为益、损”也。说在宜。 
不能而不害。说在害。 
损而不害。说在余。 
异类不吡。说在量。 
知而不以五路。说在久。 
偏去莫加少。说在故。 
必热。说在顿。 
假,必悖。说在不然。 
知其所以不知。说在以名取。 
物之所以然,与所以知之,与所以使人知之,不必同。说在病。 
无,不必待有。说在所谓。 
疑。说在逢、循、遇、过。 
擢,虑不疑。说在有、无。 
合与一,或复否。说在拒。 
且然,不可正,而不用害工。说在宜欧。 
物,一体也。说在俱一、惟是。 
均之,绝、不。说在所均。 
字,或徙。说在长宇、久。 
尧之义也,生于今而处于古,而异时。说在所义。 

二临鉴而立,景到。多而若少。说在寡区。 
狗,犬也。而杀狗非杀犬也,可。说在重。 
鉴位,景一小而易,一大则正。说在中之外内。 
使,殷、美。说在使。 
鉴团景一。 
不坚白。说在。 
荆之大,其沈,浅也。说在具。 
无久与宇坚白。说在因。 
以槛为抟,于“以为”,无知也。说在意。 
在诸其所然、未者然。说在于是推之。 
意未可知。说在可用过仵。 
景不徙。说在改为。 
一,少于二而多于五。说在建住。 
景二。说在重。 
非半弗■,则不动。说在端。 
景到,在午有端与景长。说在端。 
可无也,有之而不可去。说在尝然。 
景迎日。说在抟。 
正而不可担,说在抟。 
景之小、大。说在地正、远近。 
宇进无近。说在敷。 
天,而必正。说在得。 
行循以久。说在先后。 
贞而不挠。说在胜。 
一法者之相与也尽,若方之相合也。说在方。 
契与枝板。说在薄。 
狂举,不可以知异。说在有不可。 
牛马之非牛,与可之同。说在兼。 
倚者不可正。说在剃。 
循此循此,与彼此同。说在异。 
推之必往。说在废材。 
唱和同患。说在功。 
买无贵。说在仮其贾。 
闻所不知若所知,则两知之。说在告。 
贾宜则售。说在尽。 
以言为尽悖,悖。说在其言。 
无说而惧。说在弗心。 
唯吾谓非名也,则不可。说在仮。 
或,过名也。说在实。 
无穷不害兼。说在盈否知。 
知之、否之足用也,谆。说在无以也。 
不知其数而知其尽也。说在明者。 
谓辩无胜,必不当。说在辩。 
不知其所处,不害爱之。说在丧子者。 

无不让也,不可。说在始。 
仁、义之为内、外也,内。说在仵颜。 
于一,有知焉,有不知焉。说在存。 
学之,益也。说在诽者。 
有指于二,而不可逃。说在以二絫。 
诽之可否,不以众寡。说在可非。 
所知而弗能指。说在春也、逃臣、狗犬、贵者。 
非诽者谆。说在弗非。 
知狗,而自谓不知犬,过也。说在重。 
物甚不甚。说在若是。 
通意后对。说在不知其谁谓也。 
取下以求上也。说在泽。 
是是与是同。说在不州。 


\chapter{三、经说上}

三、经说上

故:小故,有之不必然,无之必不然。体也,若有端。大故,有之必无 
然,若见之成见也。 
体:若二之一、尺之端也。 
知材:知也者,所以知也,而必知,若明。 
虑:虑也者,以其知有求也,而不必得之,若睨。 
知:知也者,以其知过物而能貌之,若见。 
■:■也者,以其知论物,而其知之也著,若明。 
仁:爱己者,非为用己也,不若爱马,著若明。 
义:志以天下为芬,而能能利之,不必用。 
礼:贵者公,贱者名,而俱有敬僈焉。等,异论也。 
行:所为不善名,行也。所为善名,巧也,若为盗。 
实:其志气之见也,使人如己,不若金声玉服。 
忠:不利弱子亥。足将入,止容。 
孝:以亲为芬,而能能利亲,不必得。 
信:不以其言之当也,使人视城得金。 
佴:与人遇,人众,■。 
■:为是为是之台彼也,弗为也。 
廉:己惟为之,知其■也。 
所令:非身弗行。 
任:为身之所恶,以成人所急。 
勇:以其敢于是也命之,不以其不敢于彼也害之。 
力:重之谓。下、与;重,奋也。 
生:楹之生,商不可必也。 
平:惔然。 
利:得是而喜,则是利也。其害也,非是也。 
害:得是而恶,则是害也。其利也,非是也。 
治:吾事治矣,人有治,南北。 
誉之,必其行也。其言之忻,使人督之。 

诽:必其行也。其言之忻。 
举:告以文名,举彼实也。 
故言也者,诸口能之出民者也。民若画俿也。言也谓言,犹石致也。 
且:自前曰且,自后曰己,方然亦且。若石者也。 
君:以若名者也。 
功:不待时,若衣裘。 
赏:上报下之功也。 
罪:不在禁,惟害无罪,殆姑。上报下之功也。 
罚:上报下之罪也。 
同:二人而俱见是楹也,若事君。 
久:古今旦莫。 
宇:东西家南北。 
穷:或不容尺,有穷;莫不容尺,无穷也。 
尽:但止动。 
始:时或有久,或无久。始当无久。 
化:若蛙为鹑。 
损:偏去也者,兼之体也。其体或去或存,谓其存者损。 
儇:昫民也。 
库:区穴若,斯貌常。 
动:偏祭从者,户枢免瑟。 
止:无久之不止,当牛非马,若矢过楹。有久之不止,当马非马,若人 
过梁。 
必:谓台执者也,若弟兄。一然者,一不然者,必不必也,是非必也。 
同:捷与狂之同长也。 
心中自是往相若也。 
厚:惟无所大。 
圜:规写支也。 
方:矩见支也。 
倍:二尺与尺,但去一。 
端:是无同也。 
有间:谓夹之者也。 
间:谓夹者也。尺,前于区穴。而后于端,不夹于端与区内。及:及非 
齐之,及也。 
■:间虚也者,两木之间,谓其无木者也。 
盈:无盈无厚。于尺,无所往而不得,得二。坚异处不相盈,相非,是 
相外也。 
撄:尺与尺俱不尽,端与端俱尽。尺与或尽或不尽。坚白之撄相尽,体 
撄不相尽。端。 
仳:两有端而后可。 
次:无厚而后可。 
法:意、规、员三也,俱可以为法。 
佴:然也者,民若法也。 
彼:凡牛,枢非牛,两也,无以非也。 
辩:或谓之牛,谓之非牛,是争彼也,是不俱当。不俱当,必或不当, 

不若当犬。 
为:欲■其指,智不知其害,是智之罪也。若智之慎文也,无遗于其害 
也,而犹欲养之,则离之。是犹食脯也,骚之利害,未可知也,欲而骚,是 
不以所疑止所欲也。墙外之利害,未可知也,趋之而得力,则弗趋也,是以 
所疑止所欲也。观“为,穷知而县于欲”之理,养脯而非■也,养指而非愚 
也,所为与不所与为相疑也,非谋也。 
已:为衣,成也。治病,亡也。 
使:令,谓谓也,不必成;湿,故也,必待所为之成也。 
名:物,达也,有实必待文多也。命之马,类也,若实也者,必以是名 
也。命之臧,私也,是名也,止于是实也。声出口,俱有名,若姓宇洒。 
谓:狗犬,命也。狗犬,举也。叱狗,加也。 
知:传受之,闻也;方不障,说也;身观焉,亲也。所以谓,名也;所 
谓,实也;名实耦,合也;志行,为也。 
闻:或告之,传也;身观焉,亲也。 
见:时者,体也;二者,尽也。 
古:兵立反中,志工,正也;臧之为,宜也;非彼,必不有,必也。圣 
者用而勿必,必去者可勿疑。 
仗者两而勿偏。 
为:早台,存也;病,亡也;买鬻,易也;霄尽,荡也;顺长,治也; 
蛙买,化也。 
同:二名一实,重同也;不外于兼,体同也;俱处于室,合同也;有以 
同,类同也。 
异:二必异,二也;不连属,不体也;不同所,不合也;不有同,不类 
也。 
同异交得:于福家良,恕有无也;比度,多少也;免■还园,去就也; 
鸟折用桐,坚柔也;剑尤早,生死也;处室子子母,长少也;两绝胜,白黑 
也;中央,旁也;论行行行学实,是非也;难宿,成未也;兄弟,俱适也; 
身处志往,存亡也;霍,为姓故也;贾宜,贵贱也。 
诺:超、城、员、止也。相从、相去、先知、是、可,五色。长短、前 
后、轻重援,执服难成。言务成之,九则求执之。 
法:法取同,观巧。传法,取此择彼,问故观宜。以人之有黑者有不黑 
者也,止黑人;与以有爱于人有不爱于人,心爱人是孰宜? 
心:彼举然者,以为此其然也,则举不然者而问之。若圣人有非而不非。 
正:五诺,皆人于知有说;过五诺,若负,无直无说;用五诺,若自然 
矣。 


\chapter{四、经说下}

四、经说下

止:彼以此其然也,说是其然也;我以此其不然也,疑是其然也。 
□:谓四足兽,与生鸟与,物尽与,大小也。此然是必然,则俱。 
为麋同名,俱斗,不俱二,二与斗也。包、肝、肺、子,爱也。橘、茅, 
食与招也。白马多白,视马不多视,白与视也。为丽不必丽,不必丽与暴也。 
为非以人是不为非、若为夫勇不为夫,为屦以买衣为屦,夫与屦也。 
二与一亡,不与一在,偏去未。有文实也,而后谓之;无文实也,则无 

谓也。不若敷与美:谓是,则是固美也;谓也,则是非美;无谓,则无报也。 
见不见,离一二,不相盈,广修坚白。 
举不重不与箴,非力之任也;为握者之■(觭)倍,非智之任也。若耳 
目异。 
木与夜孰长?智与粟孰多?爵、亲、行、贾,四者孰贵?麋与霍孰高? 
麋与霍孰霍?■与瑟孰瑟? 
偏:俱一无变。 
假:假必非也而后假。狗,假霍也,犹氏霍也。 
物:或伤之,然也;见之,智也;告之,使智也。 
疑:逢为务则士,为牛庐者夏寒,逢也。举之则轻,废之则重,非有力 
也;沛从削,非巧也若石羽,循也。斗者之敝也,饮酒,若以日中,是不可 
智也,愚也。智与?以己为然也与?愚也。 
俱:俱一,若牛马四足;惟是,当牛马。数牛数马,则牛马二;数牛马, 
则牛马一。若数指,指五而五一。 
长宇:徙而有处宇,宇南北,在旦有在莫。宇徙久。 
无坚得白,必相盈也。 
在:尧善治,自今在诸古也。自古在之今,则尧不能治也。 
景:光至,景亡;若在,尽古息。 
景:二光夹一光,一光者景也。 
景:光之人煦若射。下者之人也高,高者之人也下。足敝下光,故成景 
于上;首敝上光,故成景于下。在远近有端,与于光,故景障内也。 
景:日之光反烛人,则景在日与人之间。 
景:木柂,景短大。木正,景长小。大小于木,则景大于木。非独小也, 
远近。 
临:正鉴,景寡、貌能、白黑、远近柂正,异于光。鉴、景当俱,就、 
去尒当俱,俱用北。鉴者之臭,于鉴无所不鉴。景之臭无数,而必过正。故 
同处其体俱,然鉴分。 
鉴:中之内,鉴者近中,则所鉴大,景亦大;远中,则所鉴小,景亦小。 
而必正,起于中,缘正而长其直也。中之外,鉴者近中,则所鉴大,景亦大; 
远中,则所鉴小,景亦小。而必易,合于中,而长其直也。 
鉴:鉴者近,则所鉴大,景亦大;其远,所鉴小,景亦小。而必正。景 
过正,故招。 
负:衡木,加重焉而不挠,极胜重也。右校交绳,无加焉而挠,极不胜 
重也。不胜重也。衡,加重于其一旁,必捶,权重相若也。相衡,则本短标 
长。两加焉重相若,则标必下,标得权也。 
挈:有力也;引,无力也。不正所挈之止于施也,绳制挈之也,若以锥 
刺之。挈,长重者下,短轻者上,上者愈得,下下者愈亡。绳直权重相若, 
则正矣。收,上者愈丧,下者愈得;上者权重尽,则遂。 
挈:两轮高,两轮为輲,车梯也。重其前,弦其前,载弦其前,载弦其 
轱,而县重于其前。是梯,挈且挈则行。凡重,上弗挚,下弗收,旁弗劾, 
则下直;扡,或害之也。流梯者不得流直也。今也废尺于平地,重,不下, 
无■也。若夫绳之引轱也,是犹自舟中引横也。倚:倍、拒、坚、■,倚焉 
则不正。 
谁:并石、累石,耳夹寝者,法也。方石去地尺,关石于其下,县丝于 

其上,使适至方石。不下,柱也。胶丝去石,挈也。丝绝,引也,未变而名 
易,收也。 
买:刀、籴相为贵。刀轻、则籴不贵;刀重,则籴不易。王刀无变,籴 
有变。岁变籴,则岁变刀,若鬻子。 
贾:尽也者,尽去其以不售也。其所以不售去,则售。正贾也宜不宜, 
正欲不欲,若败邦鬻室嫁子。 
无:子在军,不必其死生;闻战,亦不必其生。前也不惧,今也惧。 
或:知是之非此也,有知是之不在此也,然而谓此南北,过而以已为然。 
始也谓此南方,故今也谓此南方。 
智:论之非智,无以也。 
谓:“所谓非同也,则异也。同则或谓之狗,其或谓之犬也;异则或谓 
之牛,牛或谓之马也。俱无胜。”是不辩也。辩也者,或谓之是,或谓之非, 
当者胜也。 
无:让者酒,未让始也,不可让也。 
于:石,一也;坚、白,二也,而在石。故有智焉,有不智焉,可。 
有指:子智是,有智是吾所先举,重。则子智是,而不智吾所先举也, 
是一。谓“有智焉,有不智焉”,可。若智之,则当指之智告我,则我智之, 
兼指之以二也。衡指之,参直之也。若曰“必独指吾所举,毋举吾所不举”, 
则者固不能独指。所欲相不传,意若未校。且其所智是也,所不智是也,则 
是智是之不智也,恶得为一?谓而“有智焉,有不智焉”。 
所:春也,其执固不可指也;逃臣,不智其处;狗犬,不智其名也;遗 
者,巧弗能两也。 
智:智狗重智犬,则过;不重,则不过。 
通:问者曰:“子知驘乎?”应之曰:“驘,何谓也?”彼曰:“施。” 
则智之。若不问驘何谓,径应以弗智,则过。且应,必应问之时。若应长, 
应有深浅、大常中;在兵人长。 
所:室堂,所存也。其子,存者也。据在者而问室堂,恶可存也?主室 
堂而问存者,孰存也?是一主存者以问所存,一主所存以问存者。 
五合,水、土、火、火离,然火铄金,火多也。金靡炭,金多也。合之 
府水,木离木。若(识)麋与鱼之数,惟所利。 
无:欲恶伤生损寿,说以少连,是谁爱也?尝多粟,或者欲不有能伤也, 
若酒之于人也。且■人利人,爱也,则唯■,弗治也。 
损:饱者去余,适足,不害。能害,饱,若伤麋之无脾也。且有损而后 
益者,若疟病之之于疟也。 
智:以目见;而目以火见,而火不见。惟以五路智久,不当以目见,若 
以火见。 
火:谓火热也,非以火之热。 
我有若视,曰智。杂所智与所不智而问之,则必曰:“是所智也,是所 
不智也。”取、去,俱能之,是两智之也。 
无:若无焉,则有之而后无;无天陷,则无之而无。 
擢疑,无谓也。臧也今死,而春也得文,文死也可,且犹是也。 
且然,必然;且已,必已,且用工而后已者,必用工而后已。 
均:发均县轻重而发绝,不均也。均,其绝也莫绝。 
尧霍,或以名视人,或以实视人。举友富商也,是以名视人也;指是臛 

也,是以实视人也。尧之义也,是声也于今,所义之实处于古。若殆于城门 
与于臧也。 
狗:狗,犬也。谓之杀犬,可。若两■。 
使:令,使也。我使我,我不使,亦使我;殿戈亦使,毁不美,亦使殿。 
荆沈,荆之贝也。则沈浅非荆浅也,若易五之一。 
以楹之抟也,见之,其于意也不易,先智。意,相也。若楹轻于秋,其 
于意也洋然。 
段、椎、锥,俱事于履,可用也。成绘屦过椎,与成椎过绘屦同,过仵 
也。 
一:五,有一焉;一,有五焉;十,二焉。 
非■半,进前取也,前,则中无为半,犹端也。前后取,则端中也。■ 
必半,毋与非半;不可■也。 
可无也,已给,则当给,不可无也。久有穷而穷。 
正丸,无所处而不中县,抟也。 
伛宇不可偏举,字也。进行者,先敷近,后敷远。 
行者行者,必先近而后远。远近,修也;先后,久也。民行修,必以久 
也。 
一方尽类,俱有法而异。或木或石,不害其方之相合也。尽类犹方也。 
物俱然。 
牛狂与马惟异,以牛有齿,马有尾,说牛之非马也。不可。是俱有,不 
偏有,偏无有。曰之与马不类,用牛有角、马无角,是类不同也。若举牛有 
角、马无角,以是为类之不同也,是狂举也,犹牛有齿、马有尾。 
或不非牛而非牛也,则或非牛或牛而牛也可。故曰:牛马非牛也未可, 
牛马牛也未可。则或可或不可,而曰“牛马牛也未可”亦不可。且牛不二, 
马不二,而牛马二。则牛不非牛,马不非马,而牛马非牛非马,无难。 
彼:正名者彼、此,彼此,可。彼彼止于彼,此此止于此,彼此,不可, 
彼且此也,彼此亦可。彼此止于彼此,若是而彼此也,则彼亦且此此也。 
唱无过,无所周,若稗。和无过,使也,不得已。唱而不和,是不学也; 
智少而不学,必寡。和而不唱,是不教也;智而不教,功适息。使人夺人衣, 
罪或轻或重;使人予人酒,或厚或薄。 
闻在外者所不知也,或曰:“在室者之色,若是其色。”是所不智若所 
智也。犹白若黑也,谁胜?是若其色也,若白者必白。今也智其色之若白也, 
故智其白也。夫名,以所明正所不智,不以所不智疑所明。若以尺度所不智 
长。外,亲智也;室中,说智也。 
以悖,不可也。出入之言可,是不悖,则是有可也。之人之言不可,以 
当,必不审。惟:谓是霍,可,而犹之非夫霍也。谓彼是是也,不可。谓者 
毋惟乎其谓。彼犹惟乎其谓,则吾谓不行;彼若不惟其谓,则不行也。 
无:“南者有穷则可尽,无穷则不可尽。有穷、无穷未可智,则可尽、 
不可尽,不可尽,未可智。人之盈之否未可智,而必人之可尽、不可尽亦未 
可智,而必人之可尽爱也,悖。”人若不盈先穷,则人有穷也,尽有穷无难, 
盈无穷,则无穷尽也,尽有穷无难。 
不二智其数,恶知爱民之尽文也?或者遗乎其问也?尽问人,则尽爱其 
所问。若不智其数,而智爱之尽文也,无难。 
仁:仁,爱也;义,利也。爱、利,此也;所爱、所利,彼也。爱、利 

不相为内、外,所爱、利亦不相为外内。其为仁内也,义外也,举爱与所利 
也,是狂举也。若左目出,或目入。 
学也以为不知学之无益也,故告之也。是使智学之无益也,是教也。以 
学为无益也,教,悖。 
论诽:诽之可不可。以理之可诽,虽多诽,其诽是也;其理不可非,虽 
少诽,非也。今也谓多诽者不可,是犹以长论短。 
不诽,非已之诽也。不非诽。非可非也,不可非也。是不非诽也。 
物甚长甚短,莫长于是,莫短于是,是之是也非是也者,莫甚于是。 
取高下,以善不善为度。不若山泽,处于善于处上。下所请,上也。 
不是:是,则是,且是焉。今是文于是,而不于是,故是不文是不文, 
则是而不文焉。今是不文于是,而文与是,故文与是不文同说也。 


\end{document}