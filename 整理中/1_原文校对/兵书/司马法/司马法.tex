% 司马法
% 司马法.tex

\documentclass[a4paper,12pt,UTF8,twoside]{ctexbook}

% 设置纸张信息。
\RequirePackage[a4paper]{geometry}
\geometry{
	%textwidth=138mm,
	%textheight=215mm,
	%left=27mm,
	%right=27mm,
	%top=25.4mm, 
	%bottom=25.4mm,
	%headheight=2.17cm,
	%headsep=4mm,
	%footskip=12mm,
	%heightrounded,
	inner=1in,
	outer=1.25in
}

% 设置字体,并解决显示难检字问题。
\xeCJKsetup{AutoFallBack=true}
\setCJKmainfont{SimSun}[BoldFont=SimHei, ItalicFont=KaiTi, FallBack=SimSun-ExtB]

% 目录 chapter 级别加点(.)。
\usepackage{titletoc}
\titlecontents{chapter}[0pt]{\vspace{3mm}\bf\addvspace{2pt}\filright}{\contentspush{\thecontentslabel\hspace{0.8em}}}{}{\titlerule*[8pt]{.}\contentspage}

% 设置 part 和 chapter 标题格式。
\ctexset{
	chapter/name={第,},
	chapter/number={\chinese{chapter}}
}

% 注脚每页重新编号,避免编号过大。
\usepackage[perpage]{footmisc}

% 设置古文原文格式。
\newenvironment{yuanwen}{\bfseries\zihao{4}}

% 设置署名格式。
\newenvironment{shuming}{\hfill\bfseries\zihao{4}}

\title{\heiti\zihao{0} 司马法}
\author{司马穰苴}
\date{春秋}

\begin{document}

\maketitle
\tableofcontents

\frontmatter
\chapter{前言、序言}

《司马法》,亦名《司马兵法》《司马穰苴兵法》,是中国古代的一部著名兵书,宋代将其列为《武经七书》之一,成为武学策试的必读之书。
现在一般认为,《司马法》是春秋时期齐国的田穰苴所作,但从《史记·司马穰苴列传》记载的情况来看,《司马法》一书当不是司马穰苴一人所著。
《司马法》一书的军事思想极其丰富,涉及的内容也极为广泛,涵盖了军事的各个方面,如用兵原则、作战计谋、将士关系、军队治理等,又包括出师礼仪、兵器、徽章、赏罚等方面的要求。同时,书中还重视战争中精神力量与物质力量之间的转化以及轻与重之间辩证统一关系。具体来说,《司马法》一书所蕴含的军事思想价值主要体现在以下四个方面:
一、为正义方可出战的战争观
作者将战争作为一种政治手段,因此作者在书中根据战争的目的,把战争分为正义与非正义两大类,认为只有正义的战争才是可以进行的,非正义的战争是应该制止的。作者认为战争应该是仁义的,其目的是平天下之乱。因此,作者在首篇《仁本》中就进行了论述:“是故杀人安人,杀之可也;攻其国,爱其民,攻之可也;以战止战,虽战可也。”司马穰苴认为兴兵作战要“以仁为本”,发动战争必须符合“仁”的要求。正义战争的目的是“讨不义”、“诛有罪”。
在强调战争正义性的同时,司马穰苴还强调发动战争应以保护人民的利益不受损害为前提,“战道,不违时,不历民病,所以爱吾民也;不加丧,不因凶,所以爱夫其民也;冬夏不兴师,所以兼爱民也。”对敌国俘虏要给予优待,保护敌国人民的利益,尽量减少战争的破坏。
司马穰苴还告诫统治者,必须抓好战备,一定要慎重的对待战争,“天下虽安,忘战必危”,不能因为天下太平了,就忘记战争,而要把更多的精力放在战备上,以应付有可能发生的战争,“天下既平,天子大恺,春嵬秋狝。诸侯春振旅,秋治兵,所以不忘战也”。在强调战备的重要性时,司马穰苴还告诫统治者不能轻易发动战争,认为“国虽大,好战必亡”,不可因战争破坏国家的财富、人民正常的生产生活。
二、以法治军的思想
《司马法》认为在治军上,必须坚持以法治军。作者认为要将治国的“法”与治军的“法”区分开来,治理国家与治理军队是有着显著的区别的,不能将二者的治理原则混为一谈。“治国尚礼,治军尚法”,“国容不入军,军容不入国”,如果混淆了二者的界限,就会导致社会混乱,导致军队的战斗力下降,“军容入国则民德废,国容入军则民德弱”。司马穰苴认为以法治军的首要问题是严明赏罚,明确规定军法、纪律以及赏罚的标准,并提出了具体的方案。
司马穰苴十分强调将帅在军队中的重要性,并对杰出的将领提出了具体的要求。他认为一个优秀的将领应该具备“仁、义、智、勇、信”五条标准,要做到德才兼备,智勇双全,以身作则,身先士卒。
三、辩证的作战指导思想
《司马法》的战争指导思想是建立在朴素的军事辩证法思想基础之上的。司马穰苴强调对战争要全面考察,提出了“顺天、阜财、怿众、利地、右兵”的战争“五虑”。
司马穰苴认为军队的将帅要根据战场上的形势变化及时地进行战略战术的调整。将帅应随时洞悉敌方情况变化。要善于从众寡、轻重、治乱、进退、难易、先后、小惧与大惧等各种对立统一的关系中分析敌我双方的情况,决定作战策略。司马穰苴认为战场上的形势可以实现相互转化的,将帅要善于转变力量对比,取得优势。同时,在战争中要做到“击其微静,避其强静;击其倦劳,避其闲窕;击其倦劳,避其闲窕;击其大惧,避其小惧”。
四、讲究军械之利
“甲以重固,兵以轻胜”,“凡马车坚,甲兵利,轻乃重”。司马穰苴认为要达到最佳的战争效果,还需要充分利用各种武器资源,对各种武器进行有效地协调整合,通过长、短、轻、重、锐、钝等武器的相杂,使之各发挥其特长而相互弥补其不足。“兵不杂则不利,长兵以卫,短兵以守。太长则难犯,太短则不及。太轻则锐,锐则易乱。太重则钝,钝则不济。”“凡五兵五当,长以卫短,短以救长。迭战则久,皆战则强。”《司马法》关于兵器运用深刻的论述是非常值得我们重视的。
总之,《司马法》一书集中了春秋中期以前有关战争的各种情况,较多地保存了各种有关战争的资料,具有重要的军事价值和史料价值。
本书在写作过程中,参考了一些前人的研究成果,在此致谢。由于编者学识有限,书中难免存在不足之处,请广大读者批评指正。
黄向阳

\mainmatter

\chapter{仁本}

是故杀人安人,杀之可也;攻其国,爱其民,攻之可也;以战止战,虽战可也。

【题解】
司马穰苴在这一篇中讨论了战争的目的和治理国家与军队的方法,即“以仁为本”的治军、治国之道和“以战止战”的战争目的。

中国古代谈仁的多是儒家,兵家很少谈仁,因为战争本身就是残酷的,并且战场的特殊性也难以让人把战争与仁慈联系起来。但司马穰苴从战争的目的性出发,即战争只是一种手段,战争本身不是目的这一观点出发,论述了战争的目的是止战,战争本身是消灭战争的手段。同时作者认为,战争不应该只追求在战场上击败敌人,还必须从心理上击败敌人;对敌人不应只是以武力征服,还要从心理上征服,所以在战争中就是对敌人也必须做到“仁”、“义”、“礼”、“信”,只有这样才能真正完整地战胜敌人。

司马穰苴还认识到战争与政治之间的辩证关系,“正不获意则权,权出于战”。因此,司马穰苴认为出征作战要以仁爱为根本,杀人攻国是为了仁爱,是为了止息战争。“是故杀人安人,杀之可也;攻其国,爱其民,攻之可也;以战止战,虽战可也。”阐述了为何而战、战争与国家政治的关系、战争如何才能取胜以及出征交战的一些步骤,集中呈现了春秋时期和春秋以前战争的方式及特点。

\begin{yuanwen}
古者,以仁\footnote{博爱,人与人相互亲爱。“仁”是中国古代一种含义极广的道德观念,其核心指人与人相互亲爱。孔子以之作为最高的道德标准。}为本\footnote{根本。},以义\footnote{正义,合宜的道德、行为或道理。这里引申为合乎礼法规定的行为规范。}治之之谓正\footnote{常例,常法。}。正不获意\footnote{满足自己的意愿。}则权\footnote{这里指权变,即灵活应付随时变化的情况。}。权出于战,不出于中人\footnote{中,即中和。人,通“仁”,仁爱的意思。}。是故杀人安\footnote{安抚。}人,杀之可也\footnote{通过杀个别人来安抚保护大多数人,那么杀人是可以的。};攻其国,爱其民,攻之可也\footnote{如果攻打一个国家,而爱惜这个国家的人民,那么攻打这个国家是可以的。};以\footnote{凭借,依靠。}战止战\footnote{指实现和平。},虽战可也。故仁见亲\footnote{仁,仁爱,这里指仁爱的人。见,被。亲,亲近。},义见说\footnote{义,正义,这里指有正义感的人。说,同“悦”,喜爱。},智见恃\footnote{智,智慧,这里指有智慧的人。恃,信赖。},勇见(方/身)\footnote{勇,勇敢,这里指勇敢的人。方,通“仿”,效仿。},信见信\footnote{第一个“信”是指信用,指有信用的人。第二个“信”是指信服。}。内得爱\footnote{爱戴,拥护。}焉,所以守\footnote{守卫国家。}也;外得威\footnote{威惧,威慑。}焉,所以战也。
\end{yuanwen}

古时候治理国家和军队的人,都是坚持以仁爱为根本,坚持用合于道德、时宜和大众需求的方法来治理国家和军队,这种方法我们称之为常法。如果常法无法使自己满意,人们就会选择权变之道来实现自己的目的。权变之道是从战争中演变出来的,而不是来自于中和与仁爱。因此如果杀一个人可以安抚天下所有人的话,就可以将这个人杀掉。如果攻打一个国家,而爱惜这个国家的人民,那么也是可以攻打这个国家的。如果通过战争能够达到平息天下战争的目的,那么发动战争也是可以的。所以一个仁爱的人,人们就会亲近他;一个有正义感的人,人们就会喜爱他;一个有智慧的人,人们就会依靠他;一个勇敢的人,人们就会效仿他;一个有信誉的人,人们就会依赖他。在国内能够得到民众的爱戴与拥护,这就是保卫国家所以依赖的保障;其他国家及他们的军队威慑于你的威严而不愿与你交战,这就是战争所以凭借的保障。

\begin{yuanwen}
战道\footnote{作战的原则。},不违时\footnote{时令,这里指农时。},不历民病\footnote{指侵犯人民的利益。},所以\footnote{这样做的原因。}爱吾民也;不加丧\footnote{丧事。},不因凶\footnote{灾害。},所以爱夫其民\footnote{关心爱护敌国的民众。}也;冬夏不兴师\footnote{发动战争。},所以兼爱\footnote{这里指既爱护自己国家的民众,也爱护其他国家的民众。}其民也。故国虽大,好\footnote{喜欢,以……为乐。}战必亡\footnote{必定灭亡。};天下虽安\footnote{和平,安宁。},忘战必危。天下既平\footnote{平定。},天下大恺\footnote{假借为“凯”,军队胜利后所奏之乐。},春(蒐)\footnote{sōu,春天打猎。}秋(狝)\footnote{xiǎn,秋天打猎。},诸侯春振旅\footnote{整治军队。古时在春天农闲季节教民习战。},秋治兵\footnote{在秋季进行军事演习。},所以不忘战也。
\end{yuanwen}

行军打仗的原则是不违背农时,不侵害民众,不加重人民的负担,这样做的原因是因为爱护本国的国民;不在其他国家举行国丧的时候攻击他,不在这个国家遭受自然灾害的年份攻击他,这样做的原因是因为爱护这个国家的民众;冬天和夏天不发动战争,这样做既是为了爱护本国的民众,也是为了爱护其他国家的民众,这就是兼爱。因此国家虽然很强大,但如果好战,喜欢经常发动战争的话,这个国家就一定会灭亡;同时天下虽然和平,但如果忘了战争,忘记了战备,这样的国家也是很危险的。天下既然已经平定,战争已经不再发生,天子自然就要班师回朝,举行盛大的庆祝庆典。但就是在太平的时候,也要注意训练军队,春天和秋天可以通过田猎的方式来训练军队;诸侯国也在春天和秋天的农闲季节组织民众进行军事训练,以提高他们的战斗能力,以防不测,这些都是为战争做准备。

\begin{yuanwen}
古者,逐奔\footnote{追逐逃跑的敌人。奔,逃跑,逃亡。《左传·庄公十一年》:“大奔曰败。”}不过百步,纵绥\footnote{纵,同“从”,追赶。绥(suí),临阵退军,向后撤。}不过三舍\footnote{一舍为三十里,三舍共九十里。},是以明其礼\footnote{表明我们的礼让。}也。不穷不能\footnote{穷,穷追不舍。不能,失去战斗力的军队。},而哀怜伤病\footnote{同情敌方的伤病士兵。},是以明其仁\footnote{表明我们的仁爱。}也。成列而鼓\footnote{成列,排列好军队。鼓,击鼓进军。}是以明其信\footnote{表明我们的诚信。}也。争义\footnote{道义。}不争利\footnote{追逐小的利益。},是以明其义\footnote{表明我们的正义。}也。又能舍服\footnote{饶恕投降的敌人。舍,放过,饶恕。服,投降的敌人。},是以明其勇\footnote{表明我们的勇敢。}也。知终\footnote{结局,结果。}知始\footnote{开始,发端。},是以明其智\footnote{表明我们的智慧。}也。六德\footnote{指前面提到的“仁”、“义”、“礼”、“智”、“信”、“勇”六种道德。}以时合教\footnote{将民众聚合起来进行教化。合,聚合。教,教民。},以为民纪\footnote{管理民众。}之道也,自古之政也。
\end{yuanwen}

古代在战场上追逐败逃的敌兵不会超过一百步,追踪退却的敌兵不超过九十里,这样做的原因是为了表明自己的礼让。对那些已经丧失了战斗能力的人不要穷追不舍,对那些已经在战争中受伤、患病了的人,我们不应该加以杀戮,而应该同情他们并治好他们的伤,这样做是为了表示我们的仁爱。在战场上,在双方军队都排列好军队的阵形,做好战争的准备以后再发动进攻,这样做是为了表明我们的诚信。恪守大义而不去急于追逐小利,这是为了表示我们崇高的正义。对于已经降服的敌人,我们不加以杀戮,而是将其赦免,这样做是为了表明我们的胆略。时时抓住事物发展的趋势,自始至终掌握事物发展的进程,这样做是为了展示我们的智慧。坚持用礼、仁、信、义、勇、智这六种道德来教化我们的民众,作为管理民众、统率民众的根本方法,这是古代的圣贤君主们就定下的治理法则。

\begin{yuanwen}
先王\footnote{古时的圣贤明君。}之治,顺\footnote{顺应。}天之道,设\footnote{合乎。}地之宜\footnote{地利。},官(民/司)之德[5],而正名治物[6],立国[7]辨职[8],以爵分禄[9],诸侯说怀[10],海外[11]来服[12],狱弭[13]而兵寝[14],圣德之治也。
\end{yuanwen}

[5]官民之德:让品德优秀的人担任官职。
[6]正名治物:正名,辨正名分,使名实相符。治物,管理相关事务。
[7]立国:指建立诸侯国。
[8]辩职:指明确公、侯、伯、子、男等职。
[9]以爵分禄:根据爵位的高低来享受俸禄。
[10]说怀:说,通“悦”。说怀,心悦诚服。
[11]海外:中原地区以外的小国。
[12]服:臣服,归顺。
[13]狱弭:消除官司。
[14]兵寝:止息战争。
【译文】
先王治理国家,上顺应天意,下合乎地利,对于贤德的人委之以职务,明定天下百官的职位,让他们各司其职,在各自的职位上发挥自己的管理职能。根据国家的需要在各地设立诸侯国,明确诸侯国的等级秩序,按照爵位的高低来发放俸禄。使天下的诸侯都心悦诚服,就连海外的其他国家也来归顺,民众间的官司没有了,战争也消失了,这就是古代圣王用仁德治理国家的典范。

\begin{yuanwen}
其次,贤王[1]制礼乐法度,乃作五刑[2],兴甲兵以讨不义[3]。巡狩[4]省方[5],会诸侯,考不同[6]。其有失命[7]、乱常[8]、背德[9]、逆天[10]之时,而危[11]有功之君[12],(遍/徧)告[13]于诸侯,彰明[14]有罪。乃告于皇天[15]上帝[16]、日月星辰[17],祷[18]于后土[19]、四海神(祗)[20]、山川[21]、冢社[22],乃造于先王[23]。然后冢宰[24]征师[25]于诸侯曰:“某国为不道[26],征之,以某年月日,师至于某国,会天于正刑[27]”。冢宰与百官布令于军曰:“入罪人之地,无暴[28](圣)神祗[29],无行田猎[30],无毁土功[31],无燔墙屋[32],无伐[33]林木,无取六畜[34]、禾黍[35]、器械,见其老幼,奉(而/归)勿伤[36]。虽遇壮者,不校勿敌[37],敌若伤之,医药归之[38]。”既诛有罪,王及诸侯修正[39]其国,举贤[40]立明[41],正复厥职[42]。
\end{yuanwen}

[1]贤王:圣明之君。
[2]五刑:我国古代的五种刑罚,通常指墨、劓、、宫、大辟,也指笞、杖、徒、流、死。
[3]兴甲兵以讨不义:用武力来讨伐不义的人。
[4]巡狩:指天子巡视诸侯国。
[5]省方:视察四方。
[6]考不同:考核诸侯有没有违反礼制的行为。
[7]失命:违背命令。
[8]乱常:违背伦理纲常。
[9]背德:背离道德标准。
[10]逆天:违背天意。
[11]危:危害。
[12]有功之君:有功德的君主。
[13]遍告:通告。
[14]彰明:列举。
[15]皇天:古时人们认为至高无上的神。
[16]上帝:被人格化的至高无上的神。
[17]日月生辰:指天上的神灵。
[18]祷(dǎo):祈祷,祈神求福。
[19]后土:土地神。
[20]四海神:四方的群神。
[21]山川:指山神和水神。
[22]冢社:大社。
[23]造于先王:请示先王的意思。
[24]冢宰:冢,大的,地位高的。宰,古代官吏的通称。
[25]征师:向诸侯国征集军队。
[26]不道:违背常道。
[27]正刑:明正刑罚。
[28]暴:侮辱。
[29]神(qí):天曰神,地曰,这里指四海之神。
[30]田猎:狩猎,捕捉野生鸟兽。
[31]土功:治水土的工程。
[32]燔(fán):烧毁。
[33]伐:砍伐。
[34]六畜:指牛、马、羊、猪、鸡、狗。
[35]禾黍:指粮食作物。
[36]见其老幼,奉而勿伤:奉,奉送。这里指将老人小孩奉送回家而不加以伤害。
[37]不校勿敌:校(jiào),通“较”,对抗。不校勿敌的意思是如果强壮的人不对抗,就不要将他看作敌人。
[38]医药归之:医、药这里均做动词,治疗的意思。指将敌人的伤治好以后放他回家。
[39]修正:整顿治理。
[40]举贤:选举贤能。
[41]立明:设立明君。
[42]正复厥职:厥(jué),代词,他的。恢复他们上上下下的职务。
【译文】
其次,那些贤德的君主制定国家的礼乐法度,制定墨、劓、、宫、大辟等五种刑罚来惩治罪犯,发动战争以讨伐那些不义的人。亲自视察天下,了解各地的善恶之情;召见天下的诸侯,考察他们的行为,看他们是否有违背礼乐制度的行为。如果发现有诸侯玩忽职守,违背礼乐制度,违背伦理道德纲常,不顺应天时,又还想危害有功德的君主,一旦发现这样的人,就要通告天下诸侯,列举他们的罪行,将这些罪行公布于天下,然后设坛祭祀,将他们的罪行告之于皇天上苍,日月星辰,向大地上的后土四海、神山川、冢社祈祷,然后再向死去的先王们祭告。在祷告天地先祖的基础上,再由国家的冢宰向各诸侯国调集军队,发布文告说:“现在某一个国家违背天下常道,行不义之事,我们应该前往征讨他,现决定于某年某月某日在某一个国家汇合大家的军队,等待天子的正义判决以后,我们就去攻打这个国家。”在军队集合以后,冢宰和百官一起向军队发布号令,约定战争中的行为规范。命令道:“进入了犯罪人所管辖的地区以后,不要侮辱他们国家的神明,不要在这里打猎,不要毁坏这个地方的土木工程,不要焚烧他们的房屋,不要乱砍伐他们的树木,不要掠夺民众的马、牛、羊、狗、猪、鸡以及田野上的粮食和民众家中的财物。看见老人和小孩,要将他们送回家而不得加以伤害。遇上年轻力壮的人,只要他不抵抗,不攻击我们,我们就不应该将他视为敌人。敌人中间如果有负伤的,我们应该医治好他们的伤病,再将他们放回去。”在处罚了有罪的人以后,天子和诸侯还要想方设法治理好那个国家,要在这个国家选举贤能的人才,再在这个国家设立一个明君,恢复这个国家的官阶秩序,给这个国家的民众一个安居乐业的环境。

\begin{yuanwen}

王霸[1]之所以治[2]诸侯者六[3]:以土地形[4]诸侯,以政令平[5]诸侯,以礼信亲[6]诸侯,以(材/礼)力[7]说[8]诸侯,以谋人[9]维[10]诸侯,以兵革[11]服[12]诸侯。同患(共/同)利以合[13]诸侯,比小事大[14]以和[15]诸候。

【注释】
[1]王霸:这里指统治天下的天子。
[2]治:安定。
[3]六:六种方法。
[4]形:规范。
[5]平:平定。
[6]亲:亲近。
[7]材力:有能力的人。
[8]说:游说。
[9]谋人:深谋远虑的人。
[10]维:维持,辅助。
[11]兵革:指军事、武力。
[12]服:使动用法,使……臣服。
[13]合:联合。
[14]比小事大:亲近小国,侍奉大国。
[15]和:协调,统一。
【译文】
古代的天子治理国家,安定天下诸侯,靠的是以下六种方法:用土地分封天下诸侯,以政事法令安定天下诸侯,以礼义和诚信让天下的诸侯亲近他,用有能力的人士去游说安抚天下的诸侯,用有智谋的人去辅佐诸侯,用武力威慑诸侯。天子与诸侯们同患难,共享福,这就样可以掌控天下诸侯。大国亲近小国,小国侍奉大国,这样天下就能太平,各国就能和平相处。
\end{yuanwen}

\begin{yuanwen}

会之以发禁[1]者九:凭[2]弱犯寡则眚之[3],贼[4]贤害民则伐[5]之,暴内陵外则坛之[6],野荒民散则削之[7],负固不服则侵之[8],贼杀其亲则正之[9],放弑其君则残之[10],犯令陵政则杜之[11],外内乱[12]、禽兽行[13],则灭之。

【注释】
[1]发禁:发,颁布。禁,禁令。
[2]凭:欺凌。
[3]眚(shěng):通“省”,削减的意思。
[4]贼:迫害,伤害。
[5]伐:讨伐,征伐。
[6]坛:清除的意思。
[7]削:削除,减少。
[8]负固不服则侵之:指依仗险要的地势拥兵自重,不听天子号令,则用兵讨伐他。
[9]贼杀其亲则正之:残杀亲人,就治他的罪。
[10]放弑其君则残之:放,放逐。弑,杀害。放逐或者杀害君王的,就诛杀他。
[11]犯令陵政则杜之:犯令,违反法令。陵政,破坏政令。杜,关门,封闭。
[12]外内乱:内外淫乱。
[13]禽兽行:行为与禽兽一样。
【译文】
天子在集合诸侯的时候颁布九项法度禁令:侵略欺负弱小国家的,天子就削减这个国家的土地和民众;迫害贤良、残暴民众的,天子就会合诸侯去讨伐他;在国内实施暴政,同时又侵略其他国家的,天子就废除这个国家的君主,另立新的贤明的君主;诸侯治国无方,导致田野荒芜、人民流离失所的,天子就贬低他的官爵;仗着地势险固,不服从天子的命令的,天子就出兵讨伐他;迫害自己的亲友,天子就治他的罪;放逐或者杀害本国的君主的,天子就诛杀他的同党、毁掉他的家园;违反国家的政策法令,不执行天子的政令的,就断绝他与邻国的交往;内外淫乱,行为与禽兽一般的,就毁掉他们的家园,毁掉他们的国家。
\end{yuanwen}

【评析】
司马穰苴主张进行“义战”,同时要“慎战”和“备战”。对此,司马穰苴进行了深刻的阐述,警告政治家们不要好战,指出“故国虽大,好战必亡”,同时也劝告那些爱好和平的政治家们不要忘记战争,提出了“天下虽安,忘战必危”的观点。这一观点是针对当前各诸侯国的政治与军事发展情况而言的,有的国家好战,有的国家却一味地求安,躲避战争。对于战争要予以高度重视,认真积极做好各项准备,同时也要防止以战为乐思想的产生。作者的观点对走极端的政治家们是一个很好的警告。
战争只是一种手段,本身不是目的,所以克劳塞维茨在其著名的《战争论》中提出,战争是流血的政治,政治是不流血的战争。战争,在军事家眼中也许就是目的,但在政治家的眼中,战争不再是目的,而是手段。战胜敌人、征服敌人是军事家的目的,但并不是政治家的目的;对于政治家而言,目的还是在于和平与共处,因为无论什么样的战争,无论是胜还是败,战争的双方都会因此遭受重大损失。所以作者在两千多年前就提出了“以战止战”的观点,这一观点就是在今天也具有极其重要的价值。


【战例一】穰苴治军不战而退晋、燕军队

齐景公时,晋国进犯齐国的阿和甄两个地区,燕国也入侵齐国的黄河南岸地区,齐国军队不敌两国的的军队,丢失了这几个地方。
齐景公为此非常忧虑,晏婴就向齐景公推荐了田穰苴,说:“穰苴虽然是田氏的庶出子孙,但他是个有才有德的人,他的文德可以使部下亲附,他的武略可使敌人畏惧,在此国家危难之际,你不妨用一用他。”
景公于是召见田穰苴,同他讨论军事,对穰苴的军事水平大加赞赏,便任命他为将军,率兵抵御燕晋两国的军队。穰苴说:“我出身卑贱,您把我从民间提拔上来,给予我很高的地位,由于我没有任何战功,士兵并不会亲附我,百姓也不会信任我,我的资望太浅,又缺乏威信,要统率军队恐怕有点为难,希望得到您的宠臣、国内有威望的人来监察军队,帮助我治理和管理好军队。”
齐景公答应了他的要求,便派自己的亲信庄贾到穰苴的军队做监军。穰苴与庄贾约定第二天正午在军门外相会。第二天,穰苴先到达军营,树立日表,打开滴漏,等待庄贾。庄贾一向傲慢自大,喜欢摆架子,认为是率领自己的军队而由自己来当监军,于是不大着急。亲戚僚属为他送别,留下宴饮。直到正午庄贾仍未来。穰苴便放倒日表,截断滴漏,先行整顿军队,反复说明各项军规军纪。
到了傍晚,庄贾才到达军营。穰苴问:“为什么迟到?”庄贾道歉说:“因为送的大夫们和亲戚们太多,我应酬他们,所以耽搁迟到了。”穰苴说:“将领从接受任命之日起就不顾家庭,从亲临军营申明号令起就不顾亲戚,从拿起鼓槌指挥作战起就不顾个人安危。现在敌国深入我地,举国骚动,士兵暴露于境内,国君睡不好觉,吃不好饭,百姓之命皆系于您一身,还谈什么相送呢!”穰苴召来军正问道:“按照军法,按期不到者应如何处置?”军正说:“应当斩首。”庄贾害怕了,急派人向齐景公报告,求齐景公救命。但派出的使者还没有回来,庄贾已被斩首示众于三军。
三军士兵皆震惊战栗。过了好一会儿,景公派使者持节来赦免庄贾,车子闯入营垒之中。穰苴说:“将在军中,国君的命令可以不必完全照办。”问军正说:“闯入营垒依法当如何处置?”军正说:“应当斩首。”使者大惊失色。穰苴说:“国君的使者不可以杀。”便斩了驾车的驭手,砍断车子的左辅,杀死左边的马,示众于三军。派使者回报,然后开拔。
司马穰苴对士兵安营扎寨、打井砌灶、饮水吃饭、看病抓药等都亲自过问,以示关怀。把将军的粮食全部拿来与士兵共享,本人与士兵平分粮食,尤其照顾那些身体瘦弱者。三天之后集合待发,病弱的人都要求前往,奋勇争先要去作战。晋国的军队听说了这些,便撤兵而去。燕国的军队也听说了,于是也渡河而溃散。司马穰苴于是乘胜追击,收复了境内失去的国土,然后率师而归。进入国都之前放下武器,解除规定,盟誓之后才敢进城。景公与众大夫迎之于郊,依礼慰劳军队完毕,然后才返回休息。
穰苴能够在不利的形势下打赢这场战争,并且达到不战而退敌人之兵的目的,一方面是树立了自己的权威,另一方面是以自己对士兵的仁爱,调动了士兵的战斗积极性,使士兵都充满昂扬的士气。燕晋两国军队见司马穰苴的军队士气旺盛,而且士卒都有死战之心,于是只好撤兵而回了。


【战例二】诸葛亮平定南疆叛乱

诸葛亮在《隆中对》中向刘备提议向西南发展,以汉中为据点来图谋大业时,曾提出了“南抚夷越”的方针,刘备在建立蜀汉政权以后,也确实实施了一些安抚措施。但由于南方边疆地区的复杂政治情况和力量对比的关系,一些南方的豪强地主及少数民族头领依然经常发动叛乱。
公元223年,益州的雍闿起兵叛乱,在东吴的支持下,雍闿利诱永昌郡少数民族首领孟获、越巂(xī,今写作越西,在四川省)郡的夷族首领高定、牂柯郡的朱褒一起叛蜀。
南中的叛乱对当时的蜀汉政权构成了相当大的威胁,当时刘备刚因打了败仗而忧愤攻心,病死在白帝城,后主新继位,政权还不是特别巩固,而强敌魏吴又对蜀汉虎视眈眈,在这种情况下,诸葛亮没有马上出兵平定南中的叛乱,而是“抚而不讨”。先将工作重点放在与吴国关系的修补上,改善了与吴国的关系,在既缓解了蜀汉的外部压力,也切断了叛军的外援以后,诸葛亮开始了征讨南中的军事行动。
公元225年,诸葛亮开始平定南中的行动。首先,诸葛亮实施了招安,但遭到了雍闿的拒绝。在和平手段无法解决的情况下,诸葛亮亲率大军南征。在征讨前,诸葛亮确定了攻心为上的作战策略。
其次,诸葛亮抓住时机,歼灭了高定、朱褒的叛军,收复了越雋郡和牂柯郡,从心理上给孟获及南中的叛军一种震慑。
在和孟获直接进行军事对抗时,诸葛亮仔细分析了孟获的个人情况。首先,孟获在南中少数民族中影响力非常高,号召力很强,所以不能像对待高定和朱褒那样予以消灭;其次,孟获为人豪爽,但缺乏政治头脑,只是因为仇视汉人而为人所用,从而发动叛乱。针对这种情况,诸葛亮制定了“攻心为上”的政策,从而演绎了一出七擒孟获的经典战例。
第一擒:双方在益州城大战,诸葛亮在战场上生擒孟获,孟获不服,诸葛亮释放孟获。
第二擒:诸葛亮用计使孟获的副将擒了孟获交给诸葛亮,孟获却还是不服,诸葛亮便又放了他。
第三擒:诸葛亮利用孟优诈降的机会,在孟获来劫营时擒了孟获。
第四擒:诸葛亮利用孟获急于抓住自己的心理,诱出孟获将其抓住。
第五擒:带来洞主杨峰因感恩诸葛亮,将孟获生擒交给诸葛亮。
第六擒:诸葛亮打败木鹿大王的怪兽部队,生擒孟获。
第七擒:诸葛亮以火攻大破乌戈国藤甲兵,生擒孟获。
七擒孟获以后,孟获对诸葛亮心服口服,发誓再不起叛乱之心。诸葛亮平定南中叛乱以后,马上从南中撤军,并委任孟获为御史中丞,掌管南中地区。
诸葛亮平定南中后迅速撤军,目的就是为了展示自己的仁义之师的形象。一是七擒孟获展示了自己的作战计谋和军队的强大战斗力,从心理上震慑了南疆各民族。二是迅速撤军,缓解了当地的民族矛盾。三是帮助开发南中地区经济,促进地区经济的发展,使南中地区百姓对蜀汉政权心存感恩。正是因为师行仁义,诸葛亮为蜀汉建立了一个稳固的大后方。


\chapter{天子之义}

古之教民,必立贵贱之伦经,使不相陵。德义不相,材技不相掩,勇力不相犯,故力同而意和也。

【题解】
本篇的首句是“天子之义”,所以用天子之义作为本篇的篇名。这一篇作者主要探讨的是如何治军,特别是从国君行为准则的角度探讨了如何治理军队。作者提出天子之义,即天子的行为准则,必须取法于天地,借鉴于古代的圣贤君主,使自己的行为有所依凭。在规范君主的行为依据时,作者同时还强调了“教民”的重要性,认为“士不先教,不可用”的主张,从君民两个方面论述了国家治理的法则。
在谈到治军与治国时,作者提出了“国容不入军,军容不入国”的著名论断,主张军队和政治的礼仪制度不能相互逾越,必须保持各自的独立性。在论述奖惩善恶的问题上,作者提倡奖赏和惩罚都要及时进行,通过奖赏或者惩罚使“民速得为善之利”,“速规为不善之害”。同时还主张赏罚要有度,主张“大捷不赏”和“大败不诛”,以使上下都不夸功,上下都能主动承担错误,以培养将士的谦让之风。



\begin{yuanwen}
天子之义[1],必纯[2]取法[3]天地而观于先圣[4]。士庶[5]之义,必奉[6]于父母,而正[7]于君长[8]。故虽有明君[9],士不先教[10],不可用也。
\end{yuanwen}

【注释】
[1]天子之义:天子治国的准则。
[2]纯:专一不杂。
[3]法:效仿。
[4]先圣:古代的圣贤君主。
[5]士庶:贵族和平民。士,一指贵族,一指古代四民之一,即农工商以外学道艺、习武勇的人,或称“士民”,以区别于“庶民”。庶指平民,百姓。
[6]奉:敬奉,侍奉。
[7]正:纠正,改正,匡正。
[8]君长:君主,统治者。
[9]明君:开明圣贤的君主。
[10]教:教化,训练。
【译文】
天子治国的根本准则,就是一定要效法天地自然的运行规则,以先世圣王的治国方式为榜样。广大民众为人处世的根本准则,就是一定要奉养父母,同时还要勇于匡正君主的过失。所以即使有贤明的君主,如果事先不对民众进行很好的教育,这些民众就不会为君主所用。

\begin{yuanwen}

古之教民,必立贵贱之伦经[1],使不相陵[2]。德义不相逾[3],材技不相掩[4],勇力不相犯[5],故力同而意和[6]也。




【注释】
[1]伦经:伦理道德规范。
[2]陵:欺侮,欺压。
[3]德义不相:有德义的人不得以官职的升迁超越对方为目的。,同“逾”。超过,胜过。
[4]材技不相掩:有才能和技术的人不能互相埋没,指不以自己的技艺去压制别人的技艺。掩,遮掩,遮蔽。
[5]勇力不相犯:有勇气力量的人相互之间不得斗殴。犯,侵犯,这里的意思是指相互斗殴,私下斗殴。
[6]力同而意和:同心同德,齐心合力。
【译文】
古代的圣贤君主教化自己的民众,一定是先建立一套贵与贱、上与下的伦理道德规范,从而使贵与贱、上与下相互之间不侵犯。使那些有德义的人不以官职超越对方为目的,有才智技术的人不会相互埋没,有勇气力量的人不会相互间私下斗殴,做到了这一点,国家就能整体团结一心、同心同德、齐心合力了。
\end{yuanwen}

\begin{yuanwen}

古者,国容[1]不入[2]军,军容[3]不入国,故德义不相逾。上贵[4]不伐之士[5],不伐之士,上之器[6]也。苟[7]不伐则无求,无求则不争[8]。国中之听,必得其情;军旅之听,必得其宜[9],故材技不相掩[10]。从命为士上赏[11],犯命为士上戮[12],故勇力不相犯[13]。既致教其民,然后谨选[14]而使之。事极修[15],则百官给[16]矣。教极省[17],则民兴良[18]矣,习惯成,则民体俗[19]矣,教化之至也。




【注释】
[1]国容:国家朝庭的威仪,仪容。《礼记·曲礼》中记载了古代国中之人的仪容要求:“天子穆穆,诸侯皇皇,大夫济济,士子跄跄,庶人僬僬。”
[2]入:适用,应用。
[3]军容:军中将士的礼仪。
[4]贵:重视,看重。
[5]不伐之士:不自我炫耀的人。伐,炫耀。
[6]上之器:人才中的杰出之士。
[7]苟:假若,如果。
[8]争:争抢。
[9]宜:适宜。
[10]不相掩:不被埋没。
[11]从命为士上赏:听从命令,服从调度的人就给予奖赏。
[12]犯命为士上戮:违抗命令,不服从调度的人就予以处罚。戮(lù),惩罚。
[13]犯:触犯命令。
[14]谨选:谨慎选拔,任用官职。
[15]事极修:事情都处理得相当好了。
[16]给:称职。
[17]教极省:教化的内容简单明了。
[18]兴良:去恶向善,积极上进。
[19]体俗:按照好的习俗办事。
【译文】
古时候,朝庭中的礼仪规范不能用于军队,军队中的仪礼规范也不能用于朝庭,因此德和义不能相互超越。君主敬重那些不自夸功勋的人,那些不自夸功勋的人,是一个国家难得的人才。一个人才如果不自夸功勋,那么他就不会去争抢功名利禄。在朝庭中处理事务,必须掌握一个国家民众的实际情况;在军队中处理事务,必须掌握军队的实际情况,根据士卒的特点加以任用,这样那些有一技之长的人就不会被埋没。对于那些服从命令的将士要予以重赏,对于那些违抗命令,不服从指挥的士卒就要予以惩罚。赏罚分明,那些有勇力的人才不会触犯命令。对待民众,要先对他们进行教化,然后从中选出优良的人士予以任用。如果各级各类官员都能够恪尽职守,那么事情就会做得相当圆满。如果教化的内容都能够简单明了,让百姓容易接受,积极上进,那么民众中就会形成一种良好的风气,这种习惯一旦形成,那么民众就会按照习俗办事,于是在社会上就会形成一种良好的社会风气,这也就是教化趋于完美的结果。
\end{yuanwen}

\begin{yuanwen}

古者,逐奔[1]不远,纵绥[2]不及,不远则难诱[3],不及则难陷[4]。以礼为固[5],以仁为胜[6],既胜之后,其教可复[7],是以君子贵[8]之也。




【注释】
[1]逐奔:追逐逃跑的敌人。奔,逃跑的敌人。
[2]纵绥:跟踪撤退的敌人。纵,跟踪。绥,撤退的敌人。
[3]诱:被诱惑,被诱歼。
[4]陷:被敌人设的陷阱所害。
[5]以礼为固:依靠礼仪来整治军队,使军队团结一心,不可动摇。
[6]以仁为胜:用仁爱克敌制胜。
[7]复:反复。
[8]贵:以……为贵。
【译文】
古时候指挥作战的人,在追逐逃跑的敌兵时,不会追逐很远的距离;追踪主动向后撤退的敌兵,总是要保持一定的距离。追得不是很远就不会被敌人所诱骗,保持一定的距离就不会被敌军设置的陷阱所陷害。在整治军队时,要用礼节来整治军队,这样军队就会团结一心,牢不可破;同时还要用仁德治理天下,这样在用兵打仗时,就能打败敌人,获得胜利。战争获得胜利后,这种用来教化士卒和治理国家的方法,要反复经常地使用,那些君子们重视这些方法的原因,也就是因为这些方法在治理国家和军队时是非常有效的。
\end{yuanwen}

\begin{yuanwen}

有虞氏[1]戒[2]于国中,欲民体其命[3]也。夏后氏[4]誓于军中,欲民先成[5]其虑也。殷[6]誓于军门[7]之外,欲民先意[8]以(待/行)事[9]也。周[10]将交刃[11]而誓之,以致民志[12]也。




【注释】
[1]有虞氏:舜时期的一个部落,在山西南、河南西北一带活动,都城在蒲坂(今山西省运城市永济县)。
[2]戒:告诉,公告。
[3]体其命:理解君王下达的命令。
[4]夏后氏:中国古代的部落之一,禹是这个部落的首领。禹受舜禅,后禹将帝位传给儿子启,启因此建立夏王朝,称夏后氏。亦称“夏氏”、“夏后”。
[5]成:准备。
[6]殷:殷朝。盘庚迁都到今安阳附近,因其当时称为殷都,始有殷商之说。
[7]军门:军队营垒出入之门。
[8]意:意会,领会。
[9]待事:随时做好战斗的准备。
[10]周:周朝。周武王灭商后建立的政权。以公元前770年周平王迁都洛邑为标志,周朝划分为西周和东周两个时期。
[11]交刃:战场交战。
[12]民志:民众的志向,这里是指兵士必死的决心。
【译文】
上古时代的有虞氏在将要进行战争时,一定会告诫国内民众,让国内民众了解他下达的命令。夏后氏在将要进行战争时,一定会在军中发布誓约,让全军将士事先做好战斗的准备。殷在将要进行战争时,在军门外定下誓约,想让全体将士了解作战意图,以做好战争的准备。周代在即将交战的时候举行誓约,以此来鼓舞士气,表达全军将士必死的意志。
\end{yuanwen}

\begin{yuanwen}

夏后氏正[1]其德也,未用兵之刃[2],故其兵不杂[3]。殷义[4]也,始用兵之刃矣。周力[5]也,尽[6]用兵之刃矣。



【注释】
[1]正:匡正。
[2]未用兵之刃:没有直接使用武力。
[3]杂:众多,这里指武器式样众多。
[4]义:正义。
[5]力:武力。
[6]尽:完全。
【译文】
夏后氏以他的仁德取得天下,没有经历过大规模的战争,所以他的兵器结构比较简单,武器装备比较单一。殷以仁义取得天下,但进行战争开始使用兵器了。周朝凭借武力赢得天下,是完全通过战争和使用兵器来得到天下的。
\end{yuanwen}

\begin{yuanwen}

夏赏[1]于朝,贵善[2]也。殷戮于市[3],威不善[4]也。周赏于朝,戮于市,劝[5]君子惧小人[6]也。三王彰其德一[7]也。



【注释】
[1]赏:奖赏,赏赐。
[2]贵善:重视那些有善行的人。
[3]戮于市:在闹市区执行死刑。
[4]威不善:威慑那些想做坏事的人。
[5]劝:劝勉,勉励。
[6]惧小人:使小人感到恐惧。
[7]一:一样,一致。
【译文】
夏后氏在朝廷上奖赏有功的人,是为了鼓励人们向善。殷代在集市上当众惩罚犯罪的人,是为了威慑天下那些不向善的人。周代在朝廷上奖赏那些有功勋的人,同时在集市上惩罚那些做恶的人,既是为了劝勉有德的君子,也是为了让那些小人感到恐惧。夏商周这三位圣贤的君主在奖善罚恶上虽然采取的方法是不一样的,但他们的奖善罚恶的目的是一样的,都是为了引导人们向善。
\end{yuanwen}

\begin{yuanwen}

兵不杂则不利[1],长兵[2]以卫,短[3]兵以守。太长则难犯[4],太短则不及[5]。太轻则锐[6],锐则易乱[7]。太重则钝[8],钝则不济[9]。



【注释】
[1]兵不杂则不利:意思是战场上的兵器一定要多样化,在交战时,如果不搭配使用这些兵器,就发挥不了它们最大的作用。杂,掺杂搭配。利,锋利,这里指战斗力强大。
[2]长兵:长兵器,指枪矛等兵器。
[3]短兵:短兵器,指刀剑等兵器。
[4]难犯:难以操控,难以使用。犯,使用。《孙子兵法·九地篇》:“犯三军之众,若使一人。”
[5]不及:指杀伤不到敌人。
[6]锐:行动灵活。
[7]乱:混乱。
[8]钝:行动迟钝。
[9]济:取得胜利。
【译文】
兵器不能实现多样化,其在战争中就不能有效地发挥效力。在实战中,长兵器是用来掩护短兵器的,短兵器是用来在近距离战斗中防守用的。如果兵器太长会因为使用不灵活而很难有效的伤害敌人,而兵器太短就会杀伤不到敌人。使用太轻的兵器,使用起来相当灵活,但行动过于灵活容易导致混乱。兵器过重,又会导致行动迟钝,行动迟钝会使自己在战场上失去战机,从而很难取得战争的胜利。
\end{yuanwen}

\begin{yuanwen}

戎车[1],夏后氏曰钩车[2],先正[3]也;殷曰寅车[4],先疾[5]也;周曰元戎[6],先良[7]也。旗[8],夏后氏玄[9],首人之(执/孰)[10]也;殷白,天之义也[11];周黄,地之道[12]也。章[13],夏后氏以日月[14],尚明[15]也;殷以虎,尚威[16]也;周以龙,尚文[17]也。



【注释】
[1]戎车:兵车,战车。春秋时期之前,一辆战车通常由四匹马来拉,每一辆车上有兵士三人,左边的执弓射箭,右边的持矛杀敌,中间的负责驾车。
[2]钩车:古代兵车名。车前栏杆弯曲。
[3]先正:首先注重的是车的稳定性。
[4]寅车:殷朝时一种行动迅速、便于进攻的战车。
[5]先疾:首先注重的是车的速度性。
[6]元戎:制造精良的大型兵车。
[7]先良:首先注重的是战车的装备精良。
[8]:旗帜的总称,是古代军队作战时用于指挥、联络的主要工具。
[9]玄:赤黑色,黑中带红。古人以不同的颜色来配五行,以五行配五德,以五德终始来说明朝代的更替。
[10]首人之执:象征手执人头,抓其头发。
[11]天之义:即代天做事。
[12]地之道:像大地一样厚重。
[13]章:士卒佩戴在身上的标记,上面有用作标识之物的纹饰图案。用于标明士卒所隶属的队伍和在军阵中的位置等。
[14]日月:这里指以太阳和月亮作为徽章上的图案。
[15]尚明:崇尚光明。
[16]尚威:崇尚威严。
[17]尚文:崇尚文采。
【译文】
战车,在夏后氏时代叫钩车,是将战车的稳重性放在第一位;在殷朝的时候叫寅车,是将战车的速度放在第一位;周朝的时候叫元戎,是将战车的结构与装备的精良放在第一位。军旗,夏后氏用的是黑色的军旗,象征着在战场上抓住敌人头发,手执人头那样威猛;商代的军旅是白的,象征着自己代天从事,像上天一样覆盖万物;周朝的军旗是黄色的,象征着自己有如大地一样的厚德载物。在部队的徽章使用上,夏后氏用日月作为部队的徽章,象征着自己崇尚光明;殷商用虎作为徽章的标志,象征着自己的威仪;周朝用龙作为部队徽章的标志,象征着自己对文采的重视。
\end{yuanwen}

\begin{yuanwen}

师多务威[1]则民诎[2],少威则民[3]不胜[4]。上使民不得其义[5],百姓[6]不得其叙[7],技用不得其利[8],牛马不得其任[9],有司[10]陵[11]之,此谓多威。多威则民诎。上不尊[12]德而任诈(慝/匿)[13],不尊道[14]而任勇力[15],不贵用命[16]而贵犯命[17],不贵善行而贵暴行,陵之有司[18],此谓少威,少威则民不胜。


【注释】
[1]师多务威:军队中将领过于威严。
[2]诎:同“屈”,压抑,受约束。
[3]民:此处指士卒。
[4]不胜:不听从指挥,很难取得胜利。
[5]义:通“宜”,适宜,适度。
[6]百姓:庶民,众人。
[7]叙:同“序”,次第,秩序。
[8]利:专长。
[9]任:合理的使用。
[10]有司:官吏。
[11]凌:欺凌,欺压。
[12]尊:尊重并任用。
[13]慝:有邪念。
[14]尊道:尊重那些有道义的人。
[15]任勇力:恃勇逞强。
[16]用命:听从指挥的人。
[17]犯命:违抗命令的人。
[18]凌之有司:这里指秩序发生混乱。
【译文】
在军旅中,如果将领过于威严,士卒就会有压抑感;如果将领没有威严,则士卒就不会听从将领的指挥从而战胜敌人。如果君主使用民众不得其度,那么民众就会因此丧失行事的次序,有技能的人就不能发挥专长,牛马就不能得到充分的使用,为官者又横加凌辱,这就是所谓的过于威严。过于威严,则老百姓就会有压抑感。如果君主不重用那些有德行的人而任用那些奸诈的小人,不重用有道义的人而重用那些有勇力的人,不重用听从命令的人而重用那些违抗命令的人,不重用有善行的人而重用那些有暴行的人,就会因此导致民众欺凌官吏,这就是缺少威严的表现,如果官吏缺少威严,就不能驱使士卒在战场上战胜敌人。
\end{yuanwen}

\begin{yuanwen}

军旅以舒[1]为主,舒则民力足[2]。虽交兵致刃[3],徒不趋[4],车不驰[5],逐奔不逾列[6],是以不乱。军旅之固[7],不失行列之政[8],不绝[9]人马之力,迟速[10]不过诫命[11]。



【注释】
[1]舒:舒缓。
[2]力足:精力充足。
[3]交兵致刃:指战场上激烈地交锋。
[4]徒不趋:步兵不快速奔跑。趋,快步走。
[5]车不驰:战车不奔驰。
[6]逐奔不列:追逐逃跑的敌人不打乱行军的队列。列,队列,行列。
[7]固:稳固。
[8]行列之政:指军队行列的部署。
[9]绝:用尽。
[10]迟速:行动的快慢。
[11]诫命:将帅的警戒命令。
【译文】
在军队行动中要以舒缓为主,行事舒缓然后士卒的精力就会充足。在交战中,虽然出现了短兵相接的场面,但步兵不乱跑,战车不随意奔驰,追赶逃跑的敌人时也不会超越行列,这样在战斗中就能保持良好的阵形,而不会发生混乱,从而使部队永远保持良好的战斗力。军队要想稳固强大,关键在于不失去行列的秩序,不将人马的精力消耗殆尽,行动的快慢都要听从将帅的命令。
\end{yuanwen}

\begin{yuanwen}

古者,国容[1]不入[2]军,军容不入国。军容入国,则民德废[3];国容入军,则民德弱[4]。故在国言文而语温[5],在朝恭以逊[6],修己以待人[7],不召[8]不至,不问不言,难进易退[9];在军抗而立[10],在行遂而果[11],介者不拜[12],兵车不式[13],城不上趋[14],危事不齿[15]。故礼与法,表里[16]也;文与武,左右也。



【注释】
[1]容:礼仪。
[2]入:用于。
[3]民德废:人民的礼让之风就会废弃。
[4]民德弱:士卒士气低落。
[5]语温:语气温和,温文尔雅。
[6]恭以逊:谦逊有礼。
[7]修己以待人:严于律己,宽以待人。
[8]召:召见。
[9]难进易退:上朝时礼节繁琐复杂,退朝时礼节简单。清朱墉《武经七书汇解》:“难进者,三揖而进也;易退者,一辞而退也。”
[10]在军抗而立:在军队中昂然而立。抗,抗拒的意思,指在军队中敢于抗拒君主的不正确命令、不正确行为。
[11]在行遂而果:在战斗中,行动要果断。
[12]介者不拜:身披铠甲的人不行跪拜之礼。
[13]式:通“轼”,车舆前面扶手的横木,古人在车上抚式行俯首之礼,以表示敬意。
[14]趋:小步急走。
[15]危事不齿:遇到危险时,不论老幼少长,都可以挺身而出。不齿,不按照年龄。
[16]表里:指礼节与法令。
【译文】
古时候,朝廷的礼仪不用于军队,军队的礼节也不用于朝廷。如果将军队的礼节用于朝堂之上,那么民众常规的道德规范就会废除。如果国家常用的仪礼规范用于军队之中,那么民众的道德仪礼习俗就会遭到削弱。因此在朝廷中说话要温文尔雅,语气要温和,在朝廷上参拜君主时恭敬而谦逊,要加强自身的修养,以良好的德行来对待他人,如果君主不召见自己就不要主动的去见君主,如果君主不向自己提出问题就不要随便乱说,觐见君主时一定要注意觐见的礼节,虽然礼节繁多也不能舍弃其中的任何一项,在拜退君主时,则要注意不要有太多的礼节。在军队中则要注意军队的特殊情况,对于君主的命令也不能完全执行,特别是对于君主不正确的命令要敢于违抗。在战阵中,行动要迅速果断。如果已经穿戴上了铠甲,见到了君主和上级就不用跪拜,已经登上了兵车的将士就不必行军礼了,将士登上了城墙就不能允许奔跑,遇到了危险情况不论年龄大小都可以挺身而出。这就是治理国家和军队的礼节与法令的关系,这种关系就相当于表里关系,在朝廷上讲求的是礼节,在军队中奉行的是法令。文武的关系就如同左右关系,在朝廷要注意文采,但在军队中则必须尚武。
\end{yuanwen}

\begin{yuanwen}

古之贤王,明[1]民之德,尽[2]民之善,故无废[3]德,无简民[4]。赏无所生,罚无所试[5]。有虞氏不赏不罚,而民可用,至德[6]也。夏赏而不罚,至教[7]也。殷罚而不赏,至威[8]也。周以赏罚,德衰[9]也。赏不逾时[10],欲民速得为善之利也。罚不迁列[11],欲民速(睹/规)[12]为不善之害也。



【注释】
[1]明:使……彰显。
[2]尽:鼓励。
[3]废:违反。
[4]简民:不遵守法令之人。
[5]赏无所生,罚无所试:大家都做好事,因此就根本不用奖赏;大家都不做坏事,因此就根本用不着惩罚。
[6]至德:最高的德治。
[7]至教:最高水平的教化。
[8]至威:最高的威严。
[9]德衰:道德衰微。
[10]赏不时:奖赏不拖延时间。
[11]罚不迁列:惩罚时不要拖延时间。
[12]睹:看见,知道,了解。
【译文】
古代圣明的君主,在治理国家时,对于民众中良好的品德一定会大加赞扬,不会有丝毫的隐瞒,鼓励民众尽可能的去发扬善行,也没有丝毫加以隐瞒的。因此在这样的时候,没有任何违反道德的事情发生,也不会有违犯法令的人,因为大家都在做好事,所以奖赏也没有用了;因为大家都不做坏事,所以惩罚也就没有什么用了。有虞氏治理自己的部落就是没有奖赏也没有惩罚,而民众都听从他的命令,这就是以德治国的最高典范。夏后氏治理国家时,只有奖赏而没有惩罚,这是教化的最高表现。殷商治理国家时,只惩罚有恶行的人,而没有任何奖赏,这是以威严治理国家的最高表现。周王在治理国家时,对民众既有奖赏又有惩罚,这是民众首先风气开始衰败的表现。奖赏有功的人不可拖延时间,目的就是要让民众能够很快的看到做好事能够得到的好处,以鼓励人们争相向善。惩罚也不能拖延时间,目的就是为了让民众看到作恶事的危害和所要受到的惩罚,以惩戒那些想作恶的人。
\end{yuanwen}

\begin{yuanwen}

大捷[1]不赏,上下皆不伐善[2]。上苟不伐善,则不骄矣;下苟不伐善,必亡等[3]矣。上下不伐善若此,让之至[4]也。大败不诛,上下皆以[5]不善[6]在己,上苟以不善在己,必悔其过,下苟以不善在己,必远其罪[7]。上下分恶[8]若此,让[9]之至也。

【注释】
[1]捷:全胜。
[2]伐善:自我夸耀,自我标榜。
[3]等:等级差别。
[4]至:最高表现。
[5]以:将。
[6]不善:错误。
[7]远其罪:指不会再犯类似的错误。
[8]分恶:共同承担错误。
[9]让:谦让。
【译文】
军队在战斗中获得全胜的时候,不要对军队进行封赏,这样军队中从将帅到士卒都不会炫耀自己的功劳。将帅不炫耀自己的功劳,就不会骄傲;士兵们不炫耀自己的功劳,就不会有等级差别了。军队中上上下下都能够做到不炫耀自己的功劳,这就是礼让的最高表现了。在战争中遭到重大挫折的时候,从上到下都不要治罪,不要将罪责归到某个人或者某一部分人身上,这样从上到下都认为错误是自己造成的,上下都能在反思战争失败的原因时反省自己。将帅反省自己的失职,就一定能改正自己的错误,提高自己的指挥水平。士兵反思自己的错误,就一定不会再犯类似的错误,提高自己在战争中的作战能力。从上到下都能主动承担战争失败的责任,这是相互谦让的最好表现。
\end{yuanwen}

\begin{yuanwen}

古者戍军[1],三年不兴[2],(睹/覩)[3]民之劳[4]也。上下相报[5]若此,和之至[6]也。得意[7]则恺歌[8],示喜也。偃伯[9]灵台[10],答[11]民之劳,示休[12]也。

【注释】
[1]戍军:服兵役。
[2]兴:重新征招。
[3]睹:见,体察。
[4]劳:劳苦。
[5]上下相报:相互报答。
[6]和之至:非常的和谐。
[7]得意:战争胜利。
[8]恺歌:即凯歌,军队胜利归来所唱的歌。
[9]偃伯:止息争霸战争。偃,止息的意思。伯,通“霸”,指争霸战争。
[10]灵台:这里指周文王姬伯高筑灵台。
[11]答:报答。
[12]示休:休养生息。
【译文】
古时候,如果是已经守卫过边疆的士兵,就三年之内不再征调他们去服兵役,这是君主体谅百姓的劳苦。如果上下都能如此相互回报,那么这个社会就会相当的和谐。军队在取得胜利的时候就要高唱凯歌而回,用来表示胜利的喜悦。战争结束后,高筑灵台,在灵台上答谢人民的苦劳,用来显示从此将不再发动战争,让老百姓休养生息。
\end{yuanwen}

 【评析】
对一个国家而言,要使自己的军队在战场上百战百胜,就必须形成一种良好的军民关系,君主要善于调动军队的积极性,因此适度的奖惩措施是很有必要的。对于那些有功于国家、有功于社稷的人就一定要予以奖赏,予以重用,而对于那些不服从命令,并没有任何实际才干的人,则不能予以重用,这样才能使那些真正有才能的人为国出力。选拔出良将以后,君主还要坚持用人不疑的原则,要注意到军队治理的特殊性,不能用朝廷的礼节规范来要求军队,只有这样,军队才能有强大的战斗力,才能所向披靡,战无不胜。
司马穰苴对建军、治军的问题进行了深入的阐述,他认为国家要崇尚礼仪,军队要崇尚法令,二者互为表里,互为补充,并行不悖。军队的法令不能用于国家的治理,国家的礼仪也不能用于军队的管理,否则,必然会发生混乱。
司马穰苴还主张作战时要“徒不趋,车不驰,逐奔不逾列”、“逐奔不远,纵绥不及”、“军旅以舒为主”,可以看出司马穰苴认为作战时保持阵形的稳定是十分重要的。
文中还列举了车战时代方阵作战的基本特点和方阵作战时武器的运用情况,而这一点恰恰是其他军事理论著作中很少提到的。


【战例一】周亚夫严明军令

 “国容不入军,军容不入国”,这是司马穰苴的主张,在这一点上,汉代的周亚夫做得相当好。
汉代的时候,北边的匈奴经常南下侵犯中原地区。公元前158年,匈奴再一次进犯汉朝北部边境,在加强边境防卫的同时,为了保卫都城长安不受匈奴的侵犯,文帝调派了三路军队到长安附近担负起守卫都城的任务。这三支分别由刘礼带兵驻守在灞上,祝兹侯徐厉带兵驻守在棘门,河内太守周亚夫带兵驻守在细柳。
面对紧张的战势,为鼓舞士气,文帝亲自犒劳慰问三路军队。文帝先到了灞上刘礼的军队,再到棘门徐厉的军队中。在刘礼和徐厉的军营中,守卫的士兵因为见到是皇帝的马车,因此在没有做任何通报的情况下就让皇帝的马车进入军营。因为文帝尉劳军队事先没有通报,所以刘礼和徐厉直到文帝到了军营中以后才知道消息,于是慌慌张张地出来迎接文帝。为了弥补迎接不周的罪责,刘礼和徐厉在文帝离开时都亲自率领全军送到军营门口。
当文帝来到周亚夫的营寨时,奉命在前面为天子出巡开道的士兵被军营门口守卫的士兵拦住了,当告知是文帝来犒劳军队时,守卫都尉告诉对方:“周将军曾经下令,在军队中,只能听将军的命令,就是天子的诏令也不能遵从。”
文帝到达军营以后,看到这种情况,只好派使者拿自己的符节进去通报,在见到文帝符节以后,周亚夫才命令打开寨门迎接文帝。这时,守营的士兵依然严肃地告诉文帝的随从:“将军命令在军营之中不许车马急驰。”车夫只好牢牢地控制着缰绳,不敢让马走得太快。文帝来到军中大帐前时,周亚夫一身戎装,出来迎接,手持兵器向文帝行拱手礼:“甲胄之士不宜行跪拜之礼,请陛下允许臣下以军中之礼拜见。”文帝听了,非常感动,起身扶着车前的横木向将士们行军礼。
劳军完毕,离开周亚夫的军营时,文帝颇有感触地对那些因为周亚夫和文帝两人的行动而惊讶不已的群臣说:“像周亚夫这样的将军才是真将军啊!驻扎在灞上和棘门的军队,简直是儿戏一般,随便允许外人在军营中行走,如果敌人来偷袭他们,就是将军也恐怕难逃被俘虏的命运,但像周亚夫这样的军队,敌人是无法偷袭的。”从周亚夫军营劳军回来后好长时间,文帝对周亚夫的治军之道都赞叹不已。
一个多月以后,汉军逼退了匈奴的军队。文帝在让三路大军撤军以后,将周亚夫提升为中尉,掌管京城的兵权,负责京师的警卫。


【战例二】韩信计诱龙且出击

司马穰苴主张“逐奔不远,纵绥不及,不远则难诱,不及则难陷。”这是从自身的立于不败之地而言的,那么在战场上面对占有地利且坚守不出的敌军,则就要反其道而行之,引诱敌人出击而歼灭他们。韩信计诱龙且出击就是这样一个经典战例。
公元前204年九月,项羽亲率大军东征彭越,刘邦为了扩大自己的势力范围,便趁机向东扩张,为了拿下齐国,刘邦采用了说降与攻战两种方法。在派说客郦食其前往齐国劝降的同时,刘邦也命令韩信率兵东进,准备攻打齐国。
齐王田广因为考虑到自身力量不足以对抗汉王的军队,再加上田广本人缺乏深谋远虑,战争准备不充分,因此在郦食其前来劝降的时候就答应了汉王的要求,背叛楚国,投降了汉国。齐王投降以后,韩信就打算不再向齐国进军。这时齐国的说客蒯通在齐王投降以后投奔韩信,对韩信说:“虽然汉王暗中派人劝降了齐王,可是汉王同时也命令你进攻齐国,现在没有汉王的命令你怎么就能够停止军事行动呢?而且郦食其仅凭三寸不烂之舌就说降了齐,获得七十多座城池。可是将军您带领数万人马征战一年,才攻下赵国五十多座城池。人家外人会怎么看呢?会觉得你一个堂堂大将军还不及一个白面书生。”蒯通的一席话使韩信决心继续实施军事进攻,下令大军渡河向齐地进发。
齐王田广在听从郦食其劝降之后,认为已经是天下太平了,因此对已经兵临齐境的韩信根本没有做任何的防备,对韩信的军事行动也没有任何的侦察,而是天天同郦食其饮酒作乐。韩信率领大军突袭齐国,大败齐军,直逼齐国都城临淄时,齐王才知道韩信的军事行动。齐王见汉王一方面遣郦食其劝降,一方面派韩信率兵攻击,齐王认为汉王在骗自己,于是一怒之下将郦食其在油锅中煮了。然后匆忙领兵逃跑到高密,并向项羽求援。项羽便派龙且率领大军二十万,与齐王会合,齐、楚联军准备迎战韩信。
龙且手下的谋士在分析了敌我的形势以后,向龙且献计说:“汉军远道而来,采用的是长途奔袭的战术,他们需要的是速战速决,因此汉军现在的战斗力是相当强大的。而且齐军在本土作战,他们牵挂家室,很容易因为担心家人而溃散。因此在这种情况下,最好的办法是先不与汉军进行直接的战争,而是进行坚守。命令军士挖好濠沟,筑好堡垒,再让齐王出来做齐军的说服工作,让齐国人知道齐王还在,而且楚王发了救兵,他们还有复国的希望,自己的家园还会重建起来,这样他们就会纷纷起兵反攻韩信。如果汉军陷入四面受敌的境地,又断绝了给养,因此一定会不战自退。”可是刚愎自用的龙且却不以为然,一方面他对韩信的军事才能并不认同,另外一方面龙且带兵救齐的目的也是想趁机抢占齐国的地盘。因此龙且对手下说:“我此次带大军前来的目的不仅仅是为了救齐国,如果不战而胜汉军,又如何显示出我的功劳呢?如果我能击败韩信,不仅可以树立楚军的威风,还可以得到齐国大半的疆土。”正是由于龙且的刚愎自用导致了自己的失败。几天以后,两军在潍河两岸摆开阵势,龙且在河东,韩信在河西,准备交战。
韩信知道自己的部队是劳师远袭,必须速战速决,因此在知道龙且没有用谋士坚守阵地的计划时,就知道自己有了胜算的把握。为了破敌,韩信仔细观察了战场地形,根据潍河的水势特点,决定用水战破敌。他秘密派人连夜装满一万多个沙袋,将潍水上游堵起来,这样下游河水变浅了。第二天上午,韩信亲自率军渡过潍河攻击龙且。龙且见状,毫不示弱,也是亲自率领大军迎敌。但双方刚一开战,韩信就佯装败退撤兵。刚愎自用的龙且不知是计,以为汉军劳师远袭,已经没有了作战能力,便得意地说:“我早知道韩信胆子小。”于是,命令大军渡河追赶,要一举消灭韩信。当齐、楚联军刚刚冲到河心时,韩信暗令埋伏在上游的汉军扒开沙袋,飞奔而下的大水将正在渡河的齐、楚联军截为两段,被大水卷走的士兵不计其数。韩信回兵冲杀过去,一举全歼了已过河的齐、楚联军,齐王逃跑,龙且战死,留在东岸尚未渡河的齐、楚联军见主帅已死,彻底丧失了战斗力,纷纷落荒而逃。于是汉军一举占领了齐国全境。
这一场战争对韩信来说本是极为不利的,如果龙且听从谋士的劝说,那么韩信就只有撤兵这一条路可走,但正是因为龙且过于自负,对战场的形势又缺乏深入的了解,一味的盲目追击,从而导致了自己的失败。


\chapter{定爵}

凡战,定爵位,著功罪,收游士,申教诏,询厥众,求厥技,方虑极物,变嫌推疑,养力索巧,因心之动。

【题解】
本篇是论述战前的准备工作,内容涵盖军政各方面事务。首先,作者强调了如何运用赏罚的手段来调动将士的战争积极性。要先做到“定爵位,著功罪,收游士,申教诏,询厥众,求厥技”,在交战前制定规章制度,广收人才。其次作者对战争中天时环境的运用、钱财等物质资源的准备、士卒士气的调整、地形地势的利用、兵器的组织等五个方面论述了战争前的准备工作。最后作者论述了“七政”、“四守”等一些具体的问题,就如何进行战前的准备工作做了详细的论述。最后作者对作战的原则、战争中规章制度的制定与完善以及执行等进行了阐述。
在这一篇中,作者提出了“居国惠以信,在军广以武,刃上果以敏”的治国治军原则,作者还强调“服正成耻,约法省罚”、“不令而行”,要使全军勇往直前,奋勇杀敌。提倡要注意吸纳优秀人才用于战争的准备;在作战时,将帅要以身作则,冲锋在前,关心爱护士卒,与士卒共苦,上下一心,这样才能使全军士气高昂。这些无论是对过去还是现在都具有重要的意义。

\begin{yuanwen}

凡战,定爵位[1],著功罪[2],收游士[3],申教诏[4],询厥众[5],求厥技[6],方虑极物[7],变嫌推疑[8],养力索巧[9],因心之动[10]。



【注释】
[1]定爵位:确定军职爵位。指在作战前,要预先制定好与军功相对应的各级爵位。当时各国有不同的爵位等级。
[2]著功罪:著,著明,公布。指公布奖惩条例。
[3]收游士:收,集中。将那些游说之士集中起来,为我所用。指用游士去为自己的战争进行游说。
[4]申教诏:申,申明。指将军队的教诫条令明示。
[5]询厥众:厥(jué),代词,相当于“其”。指向人民大众征询意见。
[6]求厥技:多方寻求有一定才能的人。
[7]方虑极物:多方考虑,认真的探究事物各方面的情况。
[8]变嫌推疑:将各种复杂的情况进行推演,以解决可能遇到的问题。相当于现代战争中的沙盘推演。
[9]养力索巧:让部队养精蓄锐,寻找战胜敌人的技巧。
[10]因心之动:因,根据,依据。意思是根据民心的动向而做出决策,根据民众的意愿来采取行动。
【译文】
作战前首先应该确定好在军队中执掌各级事务的官职爵位,公布奖惩的标准,广泛召集四方的游士,为自己出谋划策,向士卒公布军队中的教诫条令,广泛征集众人关于作战的建议,搜寻利用各种有技能的人才。比较各种谋略,探索事物的来龙去脉,辨别和推理各种疑难问题,从而使自己的决策尽可能科学。让部队养精蓄锐,研究最佳的作战技巧,最后根据人心所向作出战争的行动决策。
\end{yuanwen}

\begin{yuanwen}

凡战,固众[1],相利[2],治乱[3],进止[4],服正[5],成耻[6],约法[7],省罚[8],小罪乃杀[9],小罪胜[10],大罪因[11]。




【注释】
[1]固众:稳固军心。
[2]相利:明辩利害关系。
[3]治乱:整治紊乱的秩序。
[4]进止:士兵的前进和停止。
[5]服正:遵从服膺正义。
[6]成耻:激发士兵的廉耻观。
[7]约法:精简法令。
[8]省:减少刑罚。
[9]杀:同“煞”,制止,止住。
[10]胜:得逞,指没有得到惩罚。
[11]因:随之而来。
【译文】
在作战中,要注意稳定军心,要给士兵讲明战争的利害关系,要整治紊乱的秩序,使士卒都能听从统一的调度指挥,在统一的号令下进退,要服膺正义,要采取种种措施激发士卒的廉耻感,要精简法令,减少刑罚,对于那些小的犯罪就要及时地煞住,如果小的犯罪不能煞住,那大的犯罪就会随之而来。
\end{yuanwen}

\begin{yuanwen}

顺天[1]、阜财[2]、怿[3]众、利地[4]、右[5]兵,是谓五虑。顺天奉[6]时,阜财因敌[7]。怿众勉若[8]。利地,守隘险阻[9]。右兵,弓矢御[10]、殳[11]矛守、戈戟助[12]。凡五兵[13]五当[14],长以卫短,短以救长。迭[15]战则久,皆[16]战则强。见物与侔[17],是谓两之[18]。



【注释】
[1]顺天:顺应天时。
[2]阜财:广聚资财。阜,丰富、富有的意思。
[3]怿:取悦,令人高兴的意思。
[4]利地:利用地形的优势。
[5]右:古时以右方的位置为尊,故右指上位。这里指重视的意思。
[6]奉:根据。
[7]阜财因敌:广聚钱财,要利用敌人的物资来补充自己的军力。
[8]若:顺从。
[9]守隘险阻:控制住险隘的地形,争取主动权。
[10]御:抵御。
[11]殳:一种木制的古代兵器。
[12]助:辅助作战。
[13]五兵:指矛、戟、殳、戈、弓矢五种兵器。
[14]五当:五种不同的用途。
[15]迭:轮番。
[16]皆:一齐。
[17]侔:匹配,等齐。
[18]两之:两相平衡。
【译文】
顺应天时的变化、广泛地征集财富、取悦民众、利用险地、注重武器,这就是作战时必须要考虑的五个问题。顺应天时,就是要根据天时的变化来调整自己的行为。在战争中广聚钱财,就是要关于利用敌人的物资来补充自己的军力。取悦民众,就是要顺从民众的意向。利用险地,就是要守住关隘和各种险要的地形。注重兵器的运用,就是要在战争中充分地运用各种兵器。弓箭用来抵御,殳和矛用来防守,戈和戟用来进行辅助作战。五种兵器的运用都要各得其所,在战争中要善于运用长兵器来补救短兵器的不足,也要运用短兵器来补救长兵器的不足。只有将长短兵器交替使用,各取其长,军队的战斗力才会增强。如果发现敌军使用了新式的武器,就要马上仿制出来,这样才能在战场上保持双方力量的均衡。
\end{yuanwen}

\begin{yuanwen}

主固勉若[1],视[2]敌而举。将心,心也;众心,心也[3]。马、牛、车、兵[4]、佚饱[5],力[6]也。教惟豫[7],战惟节[8]。将军,身也;卒[9],支[10]也;伍[11],指姆也。



【注释】
[1]主固勉若:主将要不断地顺应众人的意愿,巩固军心。
[2]视:根据。
[3]将心,心也;众心,心也:这里指上下应该共同一心,齐心协力的意思。众心,指普通士兵的心愿。
[4]兵:兵器。
[5]佚饱:佚(yì),通“逸”,安逸。安逸和吃饱。
[6]力:战斗力。
[7]教惟豫:豫,通“预”,预先,事先。指平时训练士兵时就是要士兵做好战时的准备。
[8]战惟节:节,符节,古代使者所持以作凭证。这里指战场上调度的统一性,要统一听从指挥。
[9]卒:古代军队的一级编制单位,通常每一百人为一卒。
[10]支:人的四肢。
[11]伍:古代军队最低的一级编制单位,通常每五人为一伍。
【译文】
主将应该在战场上不断地顺应将士们的意志,巩固军心,勉励士兵,同时还要根据敌情的变化来采取相应的对策。在战争过程中,将帅和士卒要团结一心,要充分养好马牛这些战时的运输力量,要保养好战争时的战车和武器,让士兵充分地休息好、吃饱,这样部队才会有强大的战斗力。在平时的训练中要使士兵充分地考虑到战时的状态,这样才会有逼真的训练效果,在战争时,要求士兵必须听从指挥调度,这样才能发挥出强大的战斗力。在战场上,将帅与士兵就如同一个身体,将帅就是身体的躯干,百人以上的战斗队伍就是身体的四肢,五人为伍的小分队,就是人的五指,五指的使力全靠拇指的作用(队长的作用)。
\end{yuanwen}

\begin{yuanwen}

凡战,智[1]也。斗[2],勇也。陈[3],巧[4]也。用其所欲[5],行其所能[6],废其不欲[7]不能。于敌反是[8]。



【注释】
[1]智:智谋韬略。
[2]斗:近敌战斗。
[3]陈:通“阵”,指摆战阵。
[4]巧:灵活巧妙。
[5]欲:意愿。
[6]行其所能:做自己能做到的事。
[7]不欲:不想做。
[8]反是:反其道而行之。
【译文】
指挥作战,要运用智谋韬略。近距离战斗,靠的是勇敢。布阵的时候要讲求阵法的巧妙。指挥军队作战,一定要有自己的思想,要根据自己的能力来采取行动,切忌做自己不想做也不能做成的事。而对于敌人言,则要反过来,要诱导敌人做他们不想做的事,做他们力所不能及的事。
\end{yuanwen}

\begin{yuanwen}

凡战,有天[1],有财[2],有善[3]。时日不迁[4],龟胜微行[5],是谓有天。众有有[6],因生美[7],是谓有财。人习陈(列/利)[8],极物以豫[9],是谓有善。



【注释】
[1]天:指天时。
[2]有财:有雄厚的钱财。
[3]善:完备。
[4]迁:错过,过去。
[5]龟胜微行:龟胜,商周时期人们用火烧龟骨出现的裂纹来判断吉凶,龟胜表示用龟骨占卜得到的是吉兆。微行,指秘密行动。这句话的意思是指纵然通过占卜得到了吉兆也要秘密行事,不可太过张扬。
[6]众有有:指人民生活富足。
[7]因生美:指民众富足,国家强盛。
[8]陈列:战阵和队列。
[9]极物以豫:充分准备好各种作战物资,预先做好战争的准备。极,充分,最大程度。豫,通“预”,预先。
【译文】
凡是行军作战,必须要同时具备三个因素才有可能取得战争的胜利,即要注意天时,要注意资财的准备,要注意战争准备的充分。这三者缺一不可。如果碰上了战争的好时机,就千万不要错过,同时要注意战争的保密性,就是在占卜中得到了好的兆头,也不能得意忘形,这就是顺应天时。民众都相当富足,在经济生活满足的情况下,人们就会产生对美德的需求,这就是战争中的有财。兵士都能够熟悉各种战法的运用,能够充分积聚战争所需要的物资,这就是有善
\end{yuanwen}

\begin{yuanwen}

人勉[1]及任,是谓乐人[2]。大军以固[3],多力[4]以烦[5],堪物简治[6],见物应卒[7],是谓行豫[8]。轻车[9]轻徒[10],弓矢固御[11],是谓大军。密静多内力[12],是谓固陈。因[13]是进退,是谓多力[14]。上暇[15]人教[16],是谓烦陈[17]。然有以职[18],是谓堪物。因是辨物[19],是谓简治[20]。


【注释】
[1]勉:勉励。
[2]乐人:乐于战斗的人。
[3]固:坚固,巩固。
[4]多力:战斗力强大。
[5]烦:训练有素。
[6]堪物简治:选拔能够胜任的人来管理各种事物。清朱墉《武经七书汇解》:“堪物者,事物虽繁多,堪为职主者也。简治者,简用之以致治也。”
[7]见物应卒:仔细观察事物以应对各种可能发生的问题。
[8]行豫:行事前的安排。
[9]轻车:指兵车快捷。
[10]轻徒:指步兵精锐快捷。
[11]固御:防守坚固。
[12]密静多内力:指作战计划周密,军心安定,战斗力强盛。
[13]因:依托,凭借。
[14]多力:这里的多力是指巩固阵势,增强战斗力。多,在这里做动词用。
[15]暇:从容不迫。
[16]人教:训练士卒。
[17]烦陈:陈,通“阵”。反复练习阵法,使士卒熟悉阵法。清朱墉《武经七书汇解》:“频繁于阵,教而又教,使之熟也。”
[18]职:专职。
[19]因是辨物:根据事情的轻重缓急来处理。
[20]简治:简明的管理。
【译文】
人人都能够相互勉励,完成各自应该完成的任务,这就是乐人。军队战斗力强大而且阵营坚固,兵力充实而且兵士们都相当熟悉战法,选拔那些能够胜任管理的人去管理各种事物,仔细观察事物以应对各种可能发生的问题,这就是为战争做准备。战场上用于冲锋的战车行动迅速,步兵迅疾精锐,兵士们武器精良,阵地防守坚固,这就是一支强大的军队。作战计划周密,军心安定,战斗力强盛,这就是坚固的阵地。依托坚固的阵势,进退有序,这就是强大的战斗力。将帅心情轻松,士兵熟悉战法,不断地加以演练,这就是熟悉阵法。部队各个部门都能很好地根据自己的职位管理好相关的工作,根据事物的发展情况来选择适当的管理人才,就是良好的管理。
\end{yuanwen}

\begin{yuanwen}

称众[1],因地[2],因敌令陈[3];攻战守[4],进退止[5],前后序[6],车徒因[7],是谓战参[8]。

【注释】
[1]称众:衡量双方实力。
[2]因地:根据地理条件。
[3]陈:排兵布阵。
[4]攻战守:这是三种基本作战样式。
[5]进退止:指前进、后退、停止进攻。
[6]序:有秩序。
[7]因:配合协调。
[8]战参:战争过程中的仔细检验。
【译文】
衡量敌我兵力的多少,比较敌我所占有的地势的险要,根据敌人的情况摆出合适的阵势。让士兵掌握进攻、战斗和防守三样战法,注意战争的前进、后退、停止间的配合,注意在军队中前后次序的调整,战车和步兵互相配合,整个阵法结构严谨,这就是战争中所需要考虑的事情。

\end{yuanwen}

\begin{yuanwen}

不服、不信、不和、怠[1]、疑[2]、厌[3]、慑[4]、枝[5]、拄[6]、诎[7]、顿[8]、肆[9]、崩[10]、缓[11],是谓战患[12]。



【注释】
[1]怠:玩忽职守。
[2]疑:相互猜疑。
[3]厌:厌倦战争。
[4]慑:害怕战争。
[5]枝:军心涣散。
[6]拄:不听从命令。
[7]诎:通“屈”,委屈。
[8]顿:疲惫困顿。
[9]肆:肆无忌惮。
[10]崩:分崩离析。《孙子兵法·地形篇》:“大吏怒而不服,遇敌怼而自战,将不知其能,曰崩。”
[11]缓:军纪松弛。
[12]战患:战争的祸害,有害于战争的因素。
【译文】
如果部队中出现了下级不相信上级彼此不信任的情况,各部队间不和谐,不能相互配合,士卒中出现了玩忽职守的情况,相互猜疑,厌烦战争,害怕战斗,军心涣散,不服从指挥,委屈不伸,疲惫困顿,肆无忌惮,任其所为,分崩离析,放任自流,这些情况就是战争的祸害,需要尽可能避免的。
\end{yuanwen}

\begin{yuanwen}

骄傲、慑慑[1]、吟旷[2]、虞惧[3]、事悔[4],是谓毁折[5]。



【注释】
[1]慑慑:心理恐惧。
[2]吟:士兵的呻吟声。(huáng),呼叫。
[3]虞惧:忧虑恐惧。
[4]事悔:军令经常更改。
[5]毁折:毁灭和夭折。
【译文】
部队中骄傲情绪高涨,恐惧心理也达到了极点,士兵相互指责,不能很好的团结一致,上下都感觉到忧虑恐惧,军令经常变动,这些情况都会导致军队的毁灭和夭折。
\end{yuanwen}

\begin{yuanwen}

大小[1]、坚柔[2]、参伍[3]、众寡[4]、凡两[5],是谓战权[6]。



【注释】
[1]大小:声势浩大和渺小。
[2]坚柔:刚柔。
[3]参伍:三人为参,五人为伍,这里指部队组织可以变化不一。《通典》:“凡立军,一人曰独,二人曰比,三人曰参,比参曰伍,五人为列。”
[4]众寡:指兵力的多少。
[5]两:事物的两个方面。
[6]战权:战争的权变之道。
【译文】
军队的声势可浩大也可以缩小,战略战术可以刚强也可以具有弹性,战斗队伍的编制既可以是参编也可以是伍编,投入战斗的兵力可以多也可以少,选择哪一种情况来进行战争,需要根据战争过程的利害得失来进行选择,这就是作战的权变。
\end{yuanwen}

\begin{yuanwen}

凡战,间远[1],观迩[2],因时,因财,贵信[3],恶疑[4]。作兵义[5],作事时[6],使人惠[7],见敌静[8],见乱暇[9],见危难无忘[10]其众。




【注释】
[1]间远:对距离远处的敌人要使用间谍的方法。间,作动词用,使用间谍刺探情报。
[2]观迩:近距离的敌人用直接观察的方法。
[3]贵信:士兵之间充满相互信任。
[4]恶疑:士兵之间反对相互猜疑。
[5]作兵义:兴兵作战要合乎正义。
[6]作事时:做事要把握住时机。《六韬·龙韬·军势》:“善者见利不失,遇时不疑。失利后时,反受其殃。”
[7]惠:给以恩惠。
[8]见敌静:遇到敌情要沉着冷静。
[9]见乱暇:遇到混乱要从容不迫。
[10]忘:抛弃,丢下。
【译文】
在战争过程中,对于位置还比较远的敌人可以使用离间的战术,对于距离比较近的敌人就要直接观察敌情,掌握敌军的动向。在用兵作战中,要注意天时、财力的配合,要使将士之间相互信任,尽可能减少将士间的相互猜疑。起兵要注意起兵的正当性,选择战机要注意时机的准确性,任用他人要给以恩惠,遇到敌人的时候要注意冷静,面对混乱的时候要从容不迫,遇到危难的时候不要抛下自己的士兵。
\end{yuanwen}

\begin{yuanwen}

居[1]国惠以信[2],在军广以武[3],刃[4]上果以敏[5]。居国和[6],在军法[7],刃上察[8]。居国见好[9],在军见方[10],刃上见信[11]。



【注释】
[1]居:治理。
[2]惠以信:给老百姓实惠,并且讲诚信。
[3]广以武:对待士卒心胸要宽广,同时还要注意威严。广,宽广,宽容。
[4]刃:兵刃相接,指近距离发生战争时。
[5]果以敏:果断坚决。
[6]和:和谐。
[7]法:法令。
[8]察:仔细审察。
[9]见好:被人们爱戴。见,“被”的意思。
[10]见方:被民众拥戴。
[11]见信:为广大兵士所相信。
【译文】
治理国家要广施恩惠,讲求信用,治理军队既宽厚又有威严,面对交锋时刻要坚决善断。治理国家一定要和睦安定,统率军队要法令严明,临阵对敌要察明情势。治理国家一定要被民众爱戴,统率军队一定要被士卒敬重,临阵交战一定要被大家信赖。
\end{yuanwen}

\begin{yuanwen}

凡陈[1],行惟疏[2],战惟密[3],兵惟杂[4]。人教厚[5],静乃治[6]。威利章[7],相守义[8],则人勉[9]。虑多成则人服[10],时中服[11]厥次治。物[12]既章,目乃明[13]。虑既定[14],心乃强[15]。进退无疑[16],见敌无谋,听诛[17]。无诳其名[18],无变其旗[19]。




【注释】
[1]陈:布置军阵。
[2]行惟疏:军队行列间要注意间隔。
[3]战惟密:战斗的时候要注意战斗阵形的结合紧密。
[4]杂:混合配置使用。
[5]人教厚:士兵训练有素,沉着冷静。
[6]治:森严整齐。
[7]威利章:军令威严而明确。章,同“彰”,彰显,显著。
[8]守义:恪守信义。
[9]人勉:人们相互勉励。
[10]虑多成则人服:战斗中的权谋能够多次成功,士兵们对将帅的指挥就会心悦诚服。
[11]时中服:人人心悦诚服。
[12]物:这里指旗帜。
[13]明:明了,明白。
[14]虑既定:作战计划一经确定。
[15]心乃强:信心就会增强。
[16]进退无疑:拿不定进退的主意。
[17]听诛:听受惩罚。
[18]无诳其名:不要随意使用金鼓,使下级产生迷惑。
[19]无变其旗:不要随便改变旗号,使下级产生错觉。
【译文】
布置军阵的时候,要注意队列之间有一定的间隔,以便于各种兵器的配合使用,但同时要注意在战争中队伍要紧密配合,不能给敌人留下可乘的缝隙,在战斗中,兵器要注意长短搭配。士兵训练要严格有素,气度要沉着冷静,只有保持冷静,在战斗中士卒们才能保持整齐的队伍。军令明确且有威严,将士们都能恪守信义,则人人都会在战斗中勉力向前,共同战斗。战斗中的权谋能够多次成功,士兵们就对将帅的指挥心悦诚服,事情就能依次处理好。军队中的旗帜鲜明,兵士们就能看得清楚明了,作战计划一旦确定下来,信心就会增强。对于那些在战斗中拿不定进退的主意、遇到敌人没有一点谋略的人,要予以惩罚。在作战的时候,士兵们不能乱使用金鼓,从而使下级产生迷惑。将帅也不能随便改变旗帜的颜色,从而使下级产生错觉,不知道该采取什么样的行动。
\end{yuanwen}

\begin{yuanwen}

凡事善则长[1],因古[2]则行。誓[3]作章,人乃强,灭历祥[4]。灭厉之道[5]:一曰义[6],被之以信,临之以强[7],成基一天下之形[8],人莫不(说/就)[9],是谓兼用其人[10]。一曰权[11]。成其溢[12],夺其好[13],我自其外,使自其内。




【注释】
[1]凡事善则长:做事能够从善如流就能够保持长久。
[2]因古:沿袭旧时的制度。
[3]誓:战斗誓词。
[4]灭历祥:厉,厉鬼。祥,妖祥。消灭蛊惑军心的奇谈怪论,以稳定军心。这里指消灭一切敌人。
[5]道:方法。
[6]义:道义。
[7]被之以信,临之以强:要用诚信感动敌人,以强大的气势压倒敌人。
[8]成基一天下之形:形成天下统一的局面。
[9]说:同“悦”,喜悦。
[10]兼用其人:使敌人为我所用。
[11]权:权谋。
[12]成其溢:助长敌人骄傲自满的情绪。
[13]夺其好:改变敌人的兴趣爱好。
【译文】
处世为事能够从善如流就能保持长久,沿袭古代圣王的法度办事就能顺利办成。在战斗之前,作战的誓词鲜明且有鼓动力,就能振奋人心,军队的战斗力就会大大的增强,就能歼灭一切有可能的来犯之敌。消灭来犯之敌的方法一是道义,也就是要用诚信感动敌人,以强大的气势压倒敌人,使敌人不敢与我军抗衡,形成天下一统的大局面,使天下人人都喜悦并来归顺我,这样也就能够争取敌国为我所用。消灭敌人的第二个办法就是权谋,也就是通过运用手段,使敌人滋生骄傲自满的情绪,改变敌人的喜好,然后我军从外部向敌军发动进攻,间谍在内部进行策应,做到内应外合,从而击败敌人。
\end{yuanwen}

\begin{yuanwen}

一曰人[1],二曰正[2],三曰辞[3],四曰巧[4],五曰火,六曰水,七曰兵[5],是谓七政[6]。荣、利、耻、死[7],是谓四守[8]。容色积威[9],不过改意[10]。凡此道也。


【注释】
[1]人:指网罗人才并使用他。
[2]正:正义。
[3]辞:言辞。
[4]巧:用兵技巧。
[5]兵:武器装备,引申为军队建设。
[6]七政:指上述人、正、辞、巧、火、水、兵七项军政大事。
[7]死:刑戮。
[8]四守:使人们遵守法纪的四种方法。
[9]积威:威严。
[10]不过改意:不要强行超过自己的本意。
【译文】
治理国家,一是要知人善用,二是要坚持正气,以正率下,三是要注意辞令通畅,四是要注意技巧方法,五是要慎用火攻,六是要兴修水利,七是要加强军队建设,这是治理国家的七项重要政策。荣誉、利禄、耻辱、刑罚,这是治理军队的四种奖罚手段。在军队中,无论是劝人从善还是增长自己的威势,目的都是为了使别人改过从善,这就是治军的方法。
\end{yuanwen}

\begin{yuanwen}

唯仁有亲[1]。有仁无信[2],反败厥身[3]。人人[4],正正[5],辞辞[6],火火[7]。


【注释】
[1]亲:亲近。
[2]有仁无信:只讲仁爱不讲信义。
[3]反败厥身:自己会遭到灾祸。
[4]人人:知人善任。第一个人作动词用。
[5]正正:修正他人。
[6]辞辞:言辞必须义正辞严。
[7]火火:慎用火攻。
【译文】
只有主张仁爱的君主才能得到民众的亲近。如果只在口头上谈仁爱而没有实际的行动,在行动中不讲信义,这样的君主最终必然会身败名裂。任人要做到知人善任,正人必须先正己,辞令必须义正辞严,战场上用火攻必须使用适宜。
\end{yuanwen}

\begin{yuanwen}

凡战之道[1],既作其气[2],因[3]发其政[4]。假之以色[5],道之以辞[6]。因惧而戒[7],因欲而事[8],蹈敌制地[9],以职[10]命之,是谓战法。



【注释】
[1]道:原则。
[2]既作其气:士气已经高涨。作,这里是鼓舞的意思。
[3]因:随即。
[4]发其政:公布刑罚的政令。
[5]假之以色:用和颜悦色的态度开导士卒。假,借,用。色,面部表情。
[6]道之以辞:用言辞开导士卒。
[7]戒:劝诫。
[8]因欲而事:根据士卒的欲望加以使用他们。
[9]蹈敌制地:跟踪敌人并及时占据有利地形。蹈,跟踪。制,占据,控制。
[10]职:不同职位。
【译文】
作战的原则是,一旦士气已经被鼓舞起来,接着就要颁布奖赏惩罚的规定,以使士卒都有所约束。对待士卒要和颜悦色,要用诚恳的言辞开导士卒。利用士卒害怕的心理去禁止和劝诫他们的行为,利用他们对名利的欲望而驱使他们去做事。跟踪敌人,抢占有利的地形,并按照将士的职位来分配任务,守住胜利的果实,这就是作战的方法。
\end{yuanwen}

\begin{yuanwen}

凡人之形[1],由众之求[2],试以名行[3],必善行之。若行不行,身以将之[4]。若行而行,因使勿忘[5]。三乃成章[6],人生之宜,谓之法。




【注释】
[1]形:通“型”,指规章、制度、准则。
[2]由众之求:来源于民众的愿望。
[3]试以名行:要从实践中来检验名与行是否相符。
[4]身以将之:将领亲自带头实践,以身作则。
[5]因使勿忘:使士卒记住这些规章制度。
[6]三乃成章:多次反复以后成为规章制度。
【译文】
凡是要求人们执行的规章制度,一定要来源于民众的愿望和要求。在执行的过程中,还要注意检验他们名实是否相符。并将实践中证明好的规则继续实施下去。对那些可以做到而没有做到的,将帅应该以身示范,亲自带头做到,从而引导士卒也一并做到。如果要求做到的都已经做到了,就要求士卒不要忘记这些行为准则,并在多次的反复实践中将这些准则形成规章制度,符合民众要求的规章制度就是治理军队的“法”。

\end{yuanwen}

\begin{yuanwen}

凡治乱之道[1],一曰仁[2],二曰信[3],三曰直[4],四曰一[5],五曰义[6],六曰变[7],七曰专[8]。


【注释】
[1]治乱之道:治理乱世的方法。
[2]仁:仁义,仁爱。
[3]信:信义,诚信。
[4]直:正直。
[5]一:统一。
[6]义:道义。
[7]变:权变,变通之道。
[8]专:政令专一,也可以理解为中央的权威。
【译文】
治理乱世的方法,一是要对人民仁爱,二是要对人民讲信义,三是管理者要正直,不能偏私,四是言行要统一,五是要崇尚道义,六是要注意权变,要根据实际情况加以变化,七是要注意中央政令的权威性。
\end{yuanwen}

\begin{yuanwen}

立法[1],一曰受[2],二曰法[3],三曰立[4],四曰疾[5],五曰御其服[6],六曰等其色[7],七曰百官宜无淫服[8]。




【注释】
[1]立法:指立法的原则。
[2]受:可以使人接受。
[3]法:严明的法令。
[4]立:有法必依,指执法的严肃性。
[5]疾:执法的时效性。
[6]御其服:战争中的着装要求。
[7]等其色:用旗帜和服装的颜色来区别士卒的等级。
[8]淫服:指不按规定着装。
【译文】
立法的原则,一是要使法令可以为人所接纳,二是法令要严明,三是执法要严,有法必依,四是要注意法令的时效性,不能拖延法令的实施,五是要规定好服饰制度,六是要根据等级的差异来确定旗帜衣服的颜色,七是各级官吏要根据自己官阶的大小来着装,不能随便乱穿服饰。
\end{yuanwen}

\begin{yuanwen}

凡军,使法在己曰专[1],与下畏法[2]曰法。军无小听[3],战无小利[4],日成[5]行微[6],曰道。



【注释】
[1]专:专制独断。
[2]畏法:畏惧受法律制约。
[3]小听:小道消息。
[4]小利:眼前利益。
[5]成:确定。
[6]行微:行动隐秘。
【译文】
在军队中,如果执行法令完全是由将帅说了算,这就叫做专制。如果上下级都能统一一致,以法为惧,严守法令,这才叫做法。在军队中不听信道听途说的小道消息,作战中不许贪图小便宜,这样的军队在战斗中就能取得胜利。作战方针一旦确定下来,在行动中就要注意隐秘性,这就是治军的方法。
\end{yuanwen}

\begin{yuanwen}

凡战,正不行则事专[1],不服[2]则法,不相信则一[3]。若怠[4]则动[5]之,若疑[6]则变之,若人不信[7]上,则行其不复[8]。自古之政也。


【注释】
[1]正不行则事专:用正常的手段行不通,就采取专制的措施。
[2]服:服从。
[3]不相信则一:互相不信任就要形成统一的认识。一,指统一思想认识。
[4]怠:军心懈怠。
[5]动:鼓舞,鼓励。
[6]疑:疑虑。
[7]信:相信,信任。
[8]复:改变,反复。
【译文】
作战时,如果采用常规的方法不能控制军队,就要用专制的手段来控制军队。如果士卒不听从将帅的命令,就要用严明的法令来制裁他们。如果将领与士兵不能相互信任,就要加强沟通,使上下都能形成一个统一的认识。如果士卒有懈怠的情况,就要加以鼓励;如果士卒心存疑虑,就要设法消除这种疑虑的心理。如果下级不能相信上级,那么上级发布的命令也必须执行而不能更改,这就是自古以来治理军队的基本准则。
\end{yuanwen}

【评析】
战争不能打无准备之仗,要取得战争的胜利,战前的准备必须是相当充分的。有勇无谋,不进行任何的战争准备就想一举击溃敌人是不切实际的空想。作者认为要“顺天、阜财、怿众、利地、右兵”,一定要做到顺应气候条件,发展国家的经济,争取民众的支持,巧妙利用地形,同时要改进武器装备。另外还要习练各种阵法,这样在作战时才会有秩序地进攻。
因此在进行战争前,国家和军队都必须进行充分的准备,从人的准备到物资的准备,从士兵的训练到武器装备的提升,从将帅的选拔到管理能力的提升都是战前备战的重要内容。纵观世界战争,没有进行认真的备战而取得战争胜利几乎是不可能的。


【战例一】项羽赏罚不均而失天下

楚霸王项羽是一位杰出的军事家,在他的一生中打了无数胜仗,虽然最后兵败垓下,但仍不失为一代杰出的军事家。但我们在考察项羽之所以不能得天下时,我们不得不注意项羽在用人上的失误,特别是在激励部下时的失误。
项羽在对待士卒时温和慈爱,言语温和,如果士卒生病,项羽会心疼地流泪,还会将自己的饮食分给他。从这一方面来说,项羽是特别得到了士卒的感恩和爱戴的。但项羽却不能很好地对待那些有军功的人,不能很好地按照论功行赏的原则来对立下军功的人进行奖赏,从而无法使自己的手下为自己卖命。
据《史记》记载,项羽在对待那些已经立下战功,应该加封爵位的将领时,常把刻好的大印放在手里玩磨而舍不得给奖赏的人,以致有的大印都被他磨得失去了棱角。拖延奖赏时间,使得将士对项羽有很大怨言,以致在战场上都不想再奋勇杀敌了,也严重影响了军队的士气。
通观项羽对待士卒和对待立了军功的将领的态度,我们可以发现项羽在用人上还是存在很多失误的,因此韩信将项羽对待手下的仁义称为妇人之仁。特别是在灭秦以后,项羽在封赏功臣时,违背了义帝的约定,将自己的亲信分封为王,而将首先攻破秦都城的刘邦分封到偏远的蜀地。因此诸侯们愤愤不平,特别是封秦朝的降将章邯、司马欣和董翳为王,更是激起了天下的不满。而刘邦听从韩信的建议,反项羽之道而为之,任用天下英勇善战的人才,将城邑分封给有功之臣,于是留住了很多人才,将帅士卒也都愿意为刘邦拼死奋战,从而为汉朝的建立奠定了雄厚的根基。


【战例二】曹操重赏,杨素重罚

在丰厚的赏赐激励之下,必定会有勇猛的战士出现,曹操深刻认识到了这一点。东汉末年,每当曹操打了胜仗,攻破敌人城镇的时候,就会将所缴获来的贵重财物,按照所建军功大小,全部都奖给有功将士。曹操对作战勇敢功劳显著的,即使赏赐千金他也不吝惜;但对那些作战不勇敢又没有立功而妄想获得奖赏的人,他连一分一毫也不给。因为曹操懂得用奖赏来激励将士,所以全军上下作战时都奋勇争先,于是曹操每次率军攻战都会取得很大的胜利。
和曹操的重赏相反,杨素则是重罚。隋朝时期,大将杨素治军非常严厉。如果有士卒违犯了军令,他就会对违犯军令的士卒立即处斩,绝不饶恕。每当要对敌交战时,他就寻找犯有过失的士卒而立即杀掉,被杀之人多时一次上百人,少时也不下数十人。鲜血流遍帐前,杨素却谈笑风生,不当一回事。杨素的这种做法,警戒了士卒不敢随便违犯军令。到了与敌交战之时,杨素先令三百人攻击敌人,若能攻陷敌阵也就罢了,如果不能攻破敌阵而活着回来的就都杀掉。然后再派三百人去进攻,仍不能攻破敌阵却活着回来的照旧杀掉。因此将士们都非常恐惧,为了不被杀掉,将士们都抱定必死之心去奋勇杀敌,所以每战都能获胜。


\chapter{严位}

人有胜心,惟敌之视。人有畏心,惟畏之视。两心交定,两利若一。两为之职,惟权视之。

【题解】
这篇文章虽然从题意上来说是论述关于士卒在阵列中的位置的,但从全篇的内容来看,是一篇全方位论述如何进行战争组织的文章。在本篇中,作者就如何进行队列的排列提出了意见,“立进俯,坐进跪。畏则密,危则坐。”提出了如何在战斗中进行队列的排列,“振马躁,徒甲畏亦密之,跪坐、坐伏,则膝行而宽誓之。起、躁,鼓而进,则以铎止之。”阐明了各兵种在队列中的位置和战斗中的纪律等问题。
作者在战术运用上还提出了如何根据自身和敌人实力的情况来安排不同的战法,特别是在“相轻重”的问题上进行了详细的论述,提出了“以轻行轻则危,以重行重则无功,以轻行重则败,以重行轻则战,故战相为轻重”的观点。作者强调要等待有利战机,打击敌人的弱点,要做到“击其微静,避其强静;击其疲劳,避其闲窕;击其大惧,避其小惧”。这些观点是战争辩证法的很好体现。

\begin{yuanwen}
	
	凡战之道,位[1]欲严,政欲栗[2],力欲窕[3],气欲闲[4],心欲一[5]。
	凡战之道,等[6]道义,立卒伍[7],定行列[8],正纵横[9],察[10]名实。
	
	
	
	【注释】
	[1]位:位置。
	[2]政欲栗:政令、号令要使人感到畏惧。政,政令,号令。栗,通“”,畏惧的意思。
	[3]窕:通“佻”,这里指士卒行动敏捷。
	[4]闲:安闲,安静。这里指沉着冷静。
	[5]一:统一,齐心协力。
	[6]等:等级。
	[7]立卒伍:建立军队的各级编制。
	[8]行列:队列阵形。
	[9]正:端正。
	[10]察:检察,弄清。
	【译文】
	用兵作战的规则是,士卒在行伍中的位置必须严格整齐,不能混乱,军队的政令要使人感到畏惧,士卒在战斗中的行动必须敏捷,战斗中的气势必须旺盛,而军队上下必须意志统一,齐心协力。
	在战争过程中,要按照道义的准则将军队分成不同的等级,建立起卒伍等军队的基层编制,设立相应的管理职位。然后编制部队在战斗中的队列阵形,使部队纵横都要成形,并且检察队列中士卒的名号与实际情况是否相符。
\end{yuanwen}

\begin{yuanwen}


立进俯[1],坐进跪[2]。畏[3]则密,危则坐[4]。远者视之则不畏,迩者勿视[5]则不散[6]。位下左右[7],下甲坐,誓徐行之[8],位逮徒甲[9],筹以轻重[10]。振马躁[11],徒甲畏亦密之,跪坐[12]、坐伏[13],则膝行[14]而宽[15]誓之。起,躁[16],鼓而进,则以铎[17]止之。衔枚[18],誓,糗[19],坐,膝行而推[20]之。执戮禁顾[21],噪以先之。若畏太甚,则勿戮杀,示以颜色[22],告之以所生[23],循省其职[24]。




【注释】
[1]立进俯:士卒采用站立方式前进时要弯腰前进。
[2]坐进跪:士卒采用蹲坐式前进时应该跪姿前进。
[3]畏:畏惧。
[4]坐:坐姿。
[5]迩者勿视:当敌军突然临近我军时,要求士卒当没有看见一样。
[6]散:溃散。
[7]下左右:士卒行列之位。
[8]誓徐行之:宣誓完毕以后,队伍徐徐前进。
[9]位逮徒甲:确定每个甲士和步卒在队列中的位置。
[10]筹以轻重:充分考虑到兵器使用的轻重缓急。
[11]振马躁:振动马匹,让马匹嘶喊起来。
[12]跪坐:由跪姿变为坐姿。
[13]坐伏:由坐姿变为卧姿。
[14]膝行:用膝盖行走。
[15]宽:宽和。
[16]躁:一呼而起。
[17]铎:大铃,形如铙而有舌,古代宣布政教法令时用的。
[18]衔枚:古代军队秘密行动时,让兵士口中横衔着枚(像筷子的东西),防止说话,以免敌人发觉。
[19]糗:干粮,这里指用餐。
[20]推之:缓慢挪动。
[21]执戮禁顾:在战场上要用杀戮来严禁士卒顾盼不前。
[22]颜色:和颜悦色。
[23]告之以所生:告诉士卒杀敌求生之道。
[24]循省其职:竭力去完成任务。
【译文】
在采用站立式的队列前进时,士卒要弯腰低头前进。在采用蹲坐式前进的时候,士卒要双膝跪地前进。如果士卒中出现了畏惧的情绪,就要在战斗中编制较为密集的队列,以达到稳定军心的作用。如果遇见危险的情况,就要采用蹲坐的方式。在敌我距离还比较远的时候,如果让士卒看清楚敌人,士卒有了心理准备就不会惧怕敌人。如果敌人迫在眼前,则要求士卒当没有看见一样,按照自己的准备来迎接敌人,这样军心就不会涣散了。在战斗队列中,士卒在阵势中的位置,按左右行列划分成排。甲士在相对偏后的位置,都采用跪坐的方式。誓师仪式结束以后,部队缓缓的向前开进。卒长下达命令,确定每个甲士和士卒的位置,强调士卒在战斗中注意轻重武器的搭配运用。开进时,为了营造出强大的声势,要振动马匹,让马匹嘶鸣起来,以壮声势。如果士卒们感到害怕,就要采用密集的阵形前进,同时可以通过改变士卒行进中的姿态来减轻他们的压力,如让跪着前进的士兵坐下来,让坐着前进的士兵伏下来,同时用宽厚的言辞来安慰他们。到了进攻的时候,就要一呼而起,击鼓前进。如果要停止进攻了,就要鸣金收兵。在前进中,为了不让士兵们发出声音,可以让他们的嘴中衔枚而行,在誓师的时候以及到了要吃干粮的时候,都要让士卒跪着前进。在战场上要用来严禁士卒顾盼不前,将帅要高喊着口号鼓励他们奋勇前进。如果士卒在战场上不敢杀人,就要用和悦的脸色示意,告诉他在战场上杀敌求生的方法,使每一个士兵都能努力冲锋,尽到一个做战士的职责。
\end{yuanwen}

\begin{yuanwen}

凡三军,人[1]戒分日[2];人禁不息[3],不可以分食[4]。方[5]其疑惑,可师[6]可服[7]。


【注释】
[1]人:一说为“卒”。
[2]分日:半日。
[3]禁:禁令。
[4]食:就餐,吃饭
[5]方:处于。
[6]师:出兵攻击。
[7]服:征服敌人。
【译文】
三军之中,对小分队下达的命令,半天内就要执行,对个别人员下达的命令,要立即执行,不执行命令,就不允许就餐。敌人正处于困惑的时候,可以发兵攻击他们,这样敌人才可以被征服。
\end{yuanwen}

\begin{yuanwen}

凡战,以力久[1],以气胜[2]。以固久[3],以危胜[4]。本心固[5],新气胜[6]。以甲固[7],以兵胜[8]。凡车以密固[9],徒以坐固[10],甲以重[11]固,兵以轻胜[12]。



【注释】
[1]以力久:凭借士卒的体力强盛才能持久。
[2]以气胜:凭借士卒高昂的斗志才能取得战争的胜利。
[3]以固久:依靠坚固的阵地才能坚持长久。
[4]以危胜:凭借险要的地势取得战争的胜利。《孙子兵法·九地篇》:“投之亡地然后存,陷之死地而后生。”其说与“以危胜”相同。
[5]本心固:求战之心坚定。
[6]新气胜:士气蓬勃向上就能取胜。
[7]以甲固:凭借精良的装甲进行坚固的防守。
[8]以兵胜:凭借兵器的优良来取得战争的胜利。
[9]车以密固:在车阵作战时,密集队形则阵形坚固。
[10]徒以坐固:步兵用坐阵防守就能形成坚固阵势。
[11]重:厚重。
[12]兵以轻胜:兵器胜敌在于其兵器的锐利。
【译文】
与敌人作战,士兵的气力是能否坚持久战的关键因素,士兵的士气是取得胜利的重要因素。我军的阵地坚固防守才能持久,当我军处于险地的时候就能够激励士兵以死相战,获得战争的胜利。士卒们的求战之心坚定,防御就能稳固,不会出现士卒逃跑的现象。我军求战的士气高涨就能取得战争的胜利。士卒用甲胄保护自己,用兵器杀伤敌人以取得胜利。在战阵中,兵车密集则阵地巩固,轻装步兵们以坐姿坚持防守就比较牢固。甲胄厚重才能坚固,才能抵御敌人的攻击,兵器胜敌在于其兵器的锐利。
\end{yuanwen}

\begin{yuanwen}

人有胜心[1],惟敌之视[2]。人有畏心[3],惟畏之视[4]。两心交定[5],两利若一。两为之职[6],惟权视之[7]。



【注释】
[1]胜心:求胜之心。
[2]惟敌之视:根据敌人的虚实情况决定是否出击。
[3]畏心:畏惧之心。
[4]惟畏之视:考察畏惧的是谁。
[5]两心交定:把求胜之心和畏惧之心都要考察清楚。
[6]职:把握,掌握。
[7]惟权视之:根据实际情况加以运用。
【译文】
士卒都有求胜之心,这时就需要根据敌人的虚实来决定是否可以乘虚攻击他们,如果敌人防守严密,我军士兵虽然求胜心切,也要注意不可冒进。如果士兵有畏惧之心,就要看他们畏惧的对象,如果是畏惧自己的主帅就能取得战争的胜利,如果是畏惧敌人就不能取胜。士兵的求胜之心和畏惧之心都要考察清楚,将二者的有利条件充分利用起来,这样才能取得战争的胜利。而对这二者的把握,全靠将帅在战场对实际情况的掌握。
\end{yuanwen}

\begin{yuanwen}

凡战,以轻行轻[1]则危,以重行重[2]则无功[3],以轻行重[4]则败,以重行轻[5]则战[6],故战相为轻重[7]。
舍[8]谨[9]甲兵,行慎[10]行列,战谨进止[11]。


【注释】
[1]轻:小部队。
[2]重:大部队。
[3]无功:劳而无功。
[4]以轻行重:以小部队对付敌人的大部队。
[5]以重行轻:以大部队对付敌人的小部队。
[6]战:战胜。
[7]为轻重:使用不同的兵力。
[8]舍:军队宿营。
[9]谨:谨慎。
[10]慎:谨慎,慎重。
[11]进止:前进与停止的节奏。
【译文】
在战斗中,以小部队攻击敌人的小部队就可能出现危险,以大部队攻击敌人的大部队很难取得成功。用小部队攻击敌人的大部队就要失败。用大部队攻击敌人的小部队就能取得胜利。因此在战争中双方兵力实力的大小决定了战争的胜败。
军队宿营时也要注意甲胄不离身,武器不离手,以防备敌人的偷袭。在行军时,要注意行列的整齐,在猝然遇敌的时候也能通过变换队列来迎战。在作战的时候,要注意前进和停止的节奏,防止队伍出现混乱。
\end{yuanwen}

\begin{yuanwen}

凡战,敬则慊[1],率[2]则服。上烦轻[3],上暇重[4]。奏鼓轻[5]舒鼓重[6]。服肤轻[7],服美重[8]。
凡马车坚,甲兵利,轻乃[9]重。

【注释】
[1]慊:快心,满意。
[2]率:做表率。
[3]轻:行事草率。
[4]重:行事稳重。
[5]轻:疾速前进。
[6]舒鼓重:鼓声舒缓前进速度就慢下来。
[7]轻:行动敏捷。
[8]重:行动迟缓。
[9]乃:于是。
【译文】
在作战中,将帅谨慎行事就能使士卒感到满足,将帅以身作则就能使士卒归服。而将帅如果心情急躁就会轻易行事,从而可能使军队陷入危险之中。将帅如果从容决断就能行事稳重,使军队能够保持稳定。在进行战斗时,如果擂鼓声急促就是命令士卒急速前进,鼓声轻缓就是要求士兵缓慢前进。士兵身穿较薄的衣服行动就会比较迅速,身穿厚重的衣服行动就会比较迟缓。
战场上,兵强马壮,兵车坚固,铠甲武器装备精良,这样的队伍即使是小部队,它的战斗力也可以和大部队一样强大。
\end{yuanwen}

\begin{yuanwen}

上同[1]无获,上专[2]多死,上生[3]多疑,上死[4]不胜。
凡人,死爱[5],死怒[6],死威[7],死义[8],死利[9]。
凡战,教约人轻死[10],道约人死正[11]。




【注释】
[1]上同:随声附和。
[2]专:专断。
[3]生:贪生怕死。
[4]死:拼死作战,这里指一味蛮干。
[5]死爱:为感恩拼死。
[6]死怒:因为激怒而拼死。
[7]死威:因为受到威逼而拼死。
[8]死义:为了道义拼死。
[9]死利:在利益的诱惑下拼死。
[10]轻死:不怕死。
[11]死正:为正义而战死。
【译文】
在军队中,将帅如果喜欢下属随声附和自己,就会一事无成。如果将帅喜欢独断专行,不听别人的劝谏,那么就会对那些不听自己命令的士兵大加杀戮,就会有很多人因此获死。将帅如果贪生怕死,指挥作战的时候就会疑心重生,行动迟疑难决。将帅如果只知道拼死作战,不知道战术的运用,同样也难取得战争的胜利。
士兵在战场上拼死而战的原因有很多,有为感恩而拼死一战的,有因为被人激怒而拼死一战的,有因为将帅的威严而拼死一战的,有因为道义而拼死一战的,有因为在利益的诱惑下而拼死一战的。
在战场上要充分调动士兵的战斗积极性,就需要用法令去约束士兵,使他们不怕死,敢于在战场上拼杀。用道义约束士兵,使士兵明白正义之所在,为正义而战。
\end{yuanwen}

\begin{yuanwen}

凡战,若[1]胜,若否[2]。若天[3],若人[4]。
凡战,三军之戒,无过三日;一卒[5]之警,无过分日;一人之禁,无过瞬息[6]。



【注释】
[1]若:或者。
[2]否:失败。
[3]若天:顺应天时。
[4]若人:顺应民心。
[5]卒:周代的军队编制之一,一卒一百人。
[6]瞬息:立刻。
【译文】
在与敌人交战时,或胜利,或失败。胜利和失败都在于是否顺应天时,顺应人心。
在指挥作战时,对军队下达的警戒命令必须在三天内执行,对一百人的军队下达的警示命令必须在半天内执行,对一个士兵下达的禁止命令必须立即执行。
\end{yuanwen}

\begin{yuanwen}

凡大善用本[1],其次用末[2]。执略守微[3],本末惟权[4]。战也。
凡胜,三军一人,胜[5]。



【注释】
[1]本:根本,这里指战争中以谋略取胜。
[2]末:这里指用进攻取胜。
[3]执略守微:略,指整个战局。微,指局部情况,各个细小的环节。
[4]本末惟权:根据形势的变化进行本末取舍。
[5]三军一人,胜:指军队团结得像一个人,就一定能取得战争的胜利。
【译文】
战争取得胜利的根本方法就是凭借谋略,其次才是在战场上拼杀。要取得战争的胜利,就必须把握好整个战局,抓住战场上的任何细微的变化,配合自己的谋略来取得战争的胜利。在战场上是伐谋还是伐战,需要根据战场的形势来做判断,这就是在用心作战必须注意的问题。
那些能够取得胜利的军队,全军将士高度团结,齐心合力,在战场上如同一个人一样勇猛直前,从而取得战争的胜利。
\end{yuanwen}

\begin{yuanwen}

凡鼓[1],鼓旌旗[2],鼓车,鼓马[3],鼓徒[4],鼓兵[5],鼓首[6],鼓足[7],七鼓兼齐[8]。



【注释】
[1]鼓:战鼓,这里指击鼓。
[2]鼓旌旗:指用鼓声指挥旌旗。
[3]鼓马:指挥骑兵。
[4]鼓徒:指挥步兵。
[5]鼓兵:指挥兵器使用。
[6]鼓首:《武经七书直解》:“使首四顾,左顾左,右顾右,前顾前,后顾后。”
[7]鼓足:指挥起坐行动。
[8]齐:齐全,齐备。
【译文】
在用鼓声指挥作战时,要用不同的鼓声来指挥战场上士兵的进退。鼓声有指挥军旗的,有指挥战车的,有指挥骑兵的,有指挥步兵的,有指挥兵器使用的,有指挥队列调整的,有指挥起坐进退的,这七种鼓声在战场上必须齐备,也必须要让所有的士兵熟悉。
\end{yuanwen}

\begin{yuanwen}

凡战,既固勿重[1]。重进勿尽[2],几尽[3]危。
凡战,非陈之难,使人可陈[4]难;非使可陈难,使人可用[5]难。非知之难,行之难[6]。



【注释】
[1]重:持重,指过于自大。
[2]重进勿尽:进攻时不要将全部的实力用光。
[3]尽:全力投入。
[4]可陈:掌握阵法。
[5]可用:在战争中熟练地运用兵法。
[6]非知之难,行之难:不是演习阵法困难,而是在实践中实际运用阵法困难。
【译文】
在与敌人交战时,即使自己的军队特别强大,也不要过于持重。即使兵力充实,在攻击敌人进也不要将所有的兵力全部用上去,如果将兵力全部投入到战场上,而没有留下预备部队,就可能因为战场形势的变化而出现危险。
作战,难的不是如何布置阵法,而是如何让士卒都熟悉阵法。难的也不是让士卒如何掌握阵法,而是如何让士卒在实战中能很好地运用阵法,不是演习阵法困难,而是在实践中实际运用阵法困难。
\end{yuanwen}

\begin{yuanwen}

人方有性[1],性州异[2],教成俗[3],俗州异[4],道化俗[5]。



【注释】
[1]性:禀性。
[2]性州异:人的性格随所居住州的不同而不同。
[3]教成俗:教化使人养成良好的习俗。
[4]俗州异:习俗在各州都有差异。
[5]道化俗:通过道德教化来改变风俗。
【译文】
每个人都有自己的性格,这种性格会因为每个人所居住的环境的不同而不同。针对性格的不同,我们用教化的手段使人们都养成良好的习俗。民众的习俗在各地也是不一样的,可以通过道德的教化来改变它,使之在全国形成一种统一的习俗。
\end{yuanwen}

\begin{yuanwen}

凡众寡[1],既胜若否[2]。兵不告[3]利,甲不告坚,车不告固,马不告良[4],众不自多[5],未获道[6]。



【注释】
[1]众寡:指兵力的对比。
[2]若否:像没有打胜仗一样。
[3]告:要求,力求。
[4]良:优良。
[5]自多:增加,增多,扩充军队。
[6]未获道:未真正掌握用兵的规律。
【译文】
无论兵力的多少,即使打了胜仗以后也要像没有打胜仗一样,要防止骄兵必败。如果平时不做好战备,不力求使兵器锋利,不力求使甲胄坚固,不力求使战车坚固,不力求使战马精良,不力求使民众增多。那就意味着没有从本质上掌握用兵的道理。
\end{yuanwen}

\begin{yuanwen}

凡战,胜则与众分善[1]。若将复战[2],则重赏罚。若使不胜,取过在己[3]。复战,则誓以居前[4],无复先术[5]。胜否勿反[6],是谓正则[7]。



【注释】
[1]分善:分享战功。
[2]复战:再一次的投入战争。
[3]取过在己:自己承担全部责任。
[4]居前:冲锋在前。
[5]无复先术:不再使用前面在实践中证明无效的战术。
[6]胜负勿反:不论胜负与否。
[7]正则:正确的作战指导原则。
【译文】
将帅在战争结束以后,胜利了就要和士兵分享胜利的喜悦,如果还要继续进行战争,就要加大奖罚的力度。如果没有取得战争的胜利,将帅要勇于承担过失,如果要再进行战争,就要举行誓师仪式激励士卒,并冲锋在前,而且不重复上次失败时的战术。不论胜利与否,都要遵守这个法则,这就是正确的作战法则。
\end{yuanwen}

\begin{yuanwen}

凡民,以仁救[1],以义战[2],以智决[3],以勇斗[4],以信专[5],以利劝[6],以功胜[7]。故心中[8]仁,行中义[9],堪物智[10]也,堪大勇[11]也,堪久信[12]也。
让以[13]和,人以洽[14],自予以不循[15],争贤以为人[16],说[17]其心,效[18]其力。

【注释】
[1]以仁救:以仁爱解救士卒的危难。
[2]以义战:用道义激励士卒作战。
[3]以智决:用智慧来决断士卒的是非。
[4]以勇斗:以勇气率领士卒战斗。
[5]以信专:用威信使士卒听从指挥。
[6]以利劝:用利益鼓励士卒作战。
[7]以功胜:用功爵鼓舞士卒取得胜利。
[8]中:合乎。
[9]义:义的规范。
[10]堪物智:辨别事物的是非,靠的是智慧。
[11]堪大勇:战胜敌人,靠的是勇敢。
[12]堪久信:长久的赢得大家的拥护,靠的是诚信。
[13]以:连词。
[14]洽:融洽。
[15]自予以不循:自己主动承担过错。
[16]争贤以为人:有了荣誉,主动让给别人。
[17]说:通“悦”,使高兴。
[18]效:效命。
【译文】
对待士卒,要用仁爱的心态去帮他们渡过危难,用道义去激励他们奋勇作战,用智慧帮助他们去决断是非,以自己的勇气率领他们进行战斗,以自己的威信使他们绝对服从自己的指挥,用物质利益去激励他们在战场上拼杀,用功爵鼓励士兵去求取战争的胜利。将帅的思想要合乎仁的观念,自己的行为必须合乎义的规范,用自己的智慧去考察事物的发展变化,辨别是非。用自己的勇敢带领士兵战胜强敌,争取战争的胜利,将帅必须明白,要取得战争的胜利,要长期得到士卒的拥护,靠的是自己的诚信。
如果人们都能谦让,待人温和,那么人们之间的关系就会很和谐。有了错误,自己勇于承担错误,有了荣誉,主动让给别人,这样,士卒们就会心悦诚服的跟着将帅,在战场上奋勇拼杀,甘愿效死力。

\end{yuanwen}

\begin{yuanwen}

凡战,击其微静[1],避其强静[2];击其倦劳,避其闲窕[3];击其大惧[4],避其小惧[5],自古之政也。


【注释】
[1]微静:兵力微弱却强做镇静。
[2]强静:兵力强大沉着冷静的敌人。
[3]闲窕:休整良好。
[4]大惧:特别恐惧。
[5]小惧:行事谨慎,有所戒备。
【译文】
在与敌人作战时,要进攻那些兵力微弱却故作镇静的敌人,而要避开那些兵力强盛而且沉着冷静的敌人。要进攻疲劳的敌人,而避开那些休养得很好的敌人。要进攻那些特别有恐惧感的敌人,避开那些在战场上行事谨慎的敌人。这些是自古以来指挥作战都必须遵循的一些基本原则。
\end{yuanwen}

【评析】
战争不仅仅是简单的实力的较量,也是科学管理的较量,在战争中,虽然强大的军队具有一定的优势,但强胜弱并不是必然规律,以弱胜强的战例可谓是层出不穷。弱能胜强,并不完全是强敌的骄傲自满,而是弱势一方善于进行战争资源的组织,通过提升自己的管理水平,科学的调度军队,寻求有利的战机,从而达到战胜敌人的目的。
作者在本篇中还提到了将帅的个人修养对战争胜负的影响。作者认为将帅要“以仁救,以义战,以智决,以勇斗,以信专,以利劝,以功胜。”要懂得爱护士卒、团结士卒。“心中仁,行中义”,将帅一定要虚心,做事合乎正义。“让以和,人以洽,自予以不循,争贤以为人,说其心,效其力。”把过错给自己,而把荣誉给将士,这样士卒才会在战场上拼死杀敌,从而取得战争的胜利。


【战例一】曹刿以鼓声控制军队士气

古人击鼓进军,鸣金收兵,因此鼓声在战场的作用是相当大的,鼓声运用得当,同样可以改变战场的局势。
据《左传》记载,鲁国的曹刿就上演了一出以鼓声来引导战场局势的好戏。
鲁庄公十年春,齐国军队攻打鲁国,鲁庄公打算迎战,曹刿求见鲁庄公,打算向鲁庄公进献破敌的良策。他的同乡对他说:“这些事是那些高官们考虑的,你又何必掺和进去呢?”曹刿说:“现在的一些高官们不能很好的考虑问题,他们眼光短浅,不能深谋远虑,因此很难打胜这场战争。”
于是曹刿进宫廷去见鲁庄公。曹刿问鲁庄公:“您凭什么跟齐国打仗?”庄公说:“衣食是使人生活安定的东西,我不敢独自占有,一定拿来分给别人。”曹刿说:“这种小恩小惠不能遍及百姓,老百姓是不会听从您的。”庄公说:“祭祀用的牛羊、玉帛之类,我从来不敢虚报数目,一定要做到诚实可信。”曹刿说:“这点诚意难以使人信服,神是不会保佑您的。”庄公说:“大大小小的案件,虽然不能件件都了解得清楚,但一定要处理得合情合理。”曹刿说:“这是(对人民)尽本职的事,可以凭这一点去打仗。作战时请允许我跟您去。”
鲁庄公和曹刿同坐一辆战车。在长勺和齐军作战。战争一开始,鲁庄公便要击鼓进军,曹刿说:“现在还不行。”等齐军击过三通战鼓后,曹刿说:“可以击鼓进军了。”此时齐军在三通鼓声以后,士气已经开始有所低落,而鲁国军队进攻的鼓声才起,因此鲁国军队的士气一下子高涨起来,最终齐军大败。看到齐军败退,鲁庄公便要下令追击,曹刿拦住鲁庄公说:“现在还不能追击。”
曹刿在察看了齐军的车印,又登上车前横木瞭望齐军撤退时的队形以后,便对鲁庄公说:“可以追击了。”
战争结束以后,鲁庄公向曹刿询问战争之所以能够取胜的原因。曹刿答道:“打仗靠的是勇气。头通鼓能振作士兵们的勇气,二通鼓时勇气减弱,到三通鼓时勇气已经消失了。齐国三通鼓声过后,他们的勇气已经消失了,而我方才擂第一通鼓,我方是凭借高涨的士气打败齐军的。在追击齐军时,我之所以小心谨慎,是因为齐国是一个大国,他们强大,我们弱小,要防止他们的撤退是一种诱敌之计,也要防止他们在路上有埋伏。通过观察,我发现他们的车印混乱,军旗也倒下了,不是有埋伏的样子,因此才下令追击他们。”


【战例二】李牧示弱击败匈奴

战国末期,赵国将领李牧常年驻守代郡、雁门郡边境地区,这些地区与匈奴靠得很近,因此经常受到匈奴的袭击。
李牧到此地任职后,他根据代郡、雁门郡地区的实际状况设置官吏,要求收取的租税全部送到他的府署,充作养兵的军费。每天都宰杀几头牛改善部队伙食,训练士卒骑马射箭,让士卒小心地把守烽火台,经常派出间谍侦察敌情,平时给战士丰厚待遇。
李牧对手下将士规定:“假如匈奴入侵边境进行抢掠,大家应立即退入城堡营寨坚守,不许迎战,倘若有人胆敢捕捉匈奴兵,一律问斩绝不宽赦。”
因此,匈奴每次犯境抢掠,他的部队都退入城堡营寨,不同匈奴兵交战,任凭匈奴兵抢掠。像这样过了好多年,边境上没有什么伤亡和损失。但匈奴人却认为李牧这是胆小没本事,就连赵国边境上的士兵也认为自己的将军是胆小怕事。赵王因此而责备了李牧,但李牧依然不同匈奴直接交战。于是赵王把李牧撤了回来,另派他人去那里取代李牧为将。
新上任的将领到职一年多以来,每当匈奴兵来侵犯,他都命令部队出战,由于匈奴兵剽悍勇猛,因此汉军经常受挫失利,损失伤亡很多,导致边境地区无法正常耕种和放牧。赵王没有办法,只好又去请李牧担任边帅。但是李牧这回推说有病,闭门不出。赵王不得不强行命令李牧去镇守边疆。
李牧说:“如果一定让我去,必须允许我像以前那样处理守边事务,这样我才敢接受任命。”赵王一口答应了。李牧于是前往北部边境,到达后仍按原来那样约束部队。匈奴兵来犯,李牧仍然不予应战,匈奴还是认为李牧胆怯不敢出战。
守边士卒每天都得到赏赐,却不让他们出战,士兵们于是纷纷向李牧请求与匈奴决一死战。李牧于是开始了做好战前的准备,挑选一千三百辆战车,精选一万三千匹战马,挑选获重金奖赏的勇士五万人,会拉弓射箭的射手十万人,然后把士兵全部组织起来不间断地进行强化军事训练、学习战法。李牧让老百姓赶着成群的牲畜出城放牧,满山遍野都是放牧的人。
匈奴见此情景,就出动人马进行抢掠,李牧假装战败,数千人被俘虏。匈奴首领单于听到这个消息后,立即亲率大军入侵赵国边境劫掠。李牧早已布设好了兵阵,指挥赵军展开左右两翼包抄猛击匈奴军,把他们打得大败,歼灭匈奴骑兵十余万人,单于仓皇败走。其后十余年间,匈奴都不敢再侵犯赵国边境。


\chapter{用众}

凡战之道,用寡固,用众治。寡利烦,众利正。用众进止,用寡进退。

【题解】
本篇题为用众,也就是论述了如何根据自己部队的战斗力来进行战争。作者论述了在我众敌寡、敌众我寡的情况下,如何根据敌人的情况来取得战争的胜利。也论述了如何根据战场形势的变化来调整士卒的气势,使之勇于决战。
在这一篇中,作者提出了“众以合寡,则运裹而阙之”,“凡从奔勿息,敌人或止于路则虑之”、“众寡以观其变。进退以观其固,危而观其惧,静而观其怠,动而观其疑,袭而观其治。击其疑,加其卒,致其屈”等一系列战场上出奇制胜的原则,这些原则在后世的军事战争中已经得到了广泛的实践。

\begin{yuanwen}
凡战之道,用寡固\footnote{作战兵力弱小时,宜坚守阵地,做好营阵的稳固。固,稳固,坚固。},用众治\footnote{作战兵力强大时,要整肃好军队。}。寡利烦\footnote{兵力弱小就要讲求阵势的变化。},众利正\footnote{兵力强大要采取正规的作战方法。}。用众进止\footnote{兵力强大时,军队要懂前进和停止。},用寡进退\footnote{兵力弱小时,军队要懂得前进和后退。}。众以合寡\footnote{以强大的兵力迎战弱小的兵力。},则远裹而阙\footnote{较远的距离包围并留下空缺。}之,若\footnote{或者。}分而迭击\footnote{轮番攻击。}。寡以待\footnote{迎战。}众,若\footnote{如果。}众疑之,则自用之。擅利\footnote{占据在有利地形。},则释旗\footnote{将军旗倒下。}迎而反之。敌若众,则相众而受裹\footnote{被包围。}。敌若寡若畏\footnote{军队数量少且行动谨慎。},则避之\footnote{避开敌人。}开之。
\end{yuanwen}

用兵作战,如果自己的兵力弱小,难以抵敌强大的敌人,就要凭借坚固的阵地进行防守。如果自己的兵力强大,就要组织好自己的部队,防止因为组织不善而削弱部队的战斗力。兵力弱小的时候就要注意采取灵活多变的方法来应对敌人,争取出奇制胜。兵力强大的时候就要采取正规作战的方法和敌人进行正面的作战。兵力强大的时候要注意训练士兵懂得前进和停止,兵力弱小的时候要注意训练士兵在战场上懂得前进和撤退。以强大的兵力迎战弱小的兵力时,可以以较大的包围圈将敌人包围起来,然后故意留下一两个缺口,在敌人突围的时候将他们歼灭,或者采用轮番进攻的方法不断的冲击敌人,消耗敌人的力量,在敌人疲劳以后一举歼灭他们。如果以弱小的部队迎战强大的敌人,就要采用谋略来战胜敌人,或者通过离间的手段使敌人相互猜疑,从而趁虚战胜敌人。如果敌人已经占据了有利的地形,也要注意采用计谋来战胜敌人,可以通过假装放下战旗制造要逃跑的假象来引诱敌人追击,在敌人没有占据有利地形的情况下再进行反击。如果敌人的兵力过于强大,就要审视敌我力量的对比变化,要做好被敌人包围以后的战争准备。如果敌军兵力弱小而且行动比较谨慎,就要先避开敌人,然后再寻找机会消灭敌人。

\begin{yuanwen}
凡战,背风\footnote{背对风向。}背高\footnote{依靠高地。},右高\footnote{右侧临近高地。}左险\footnote{左边依附险阻。},历沛\footnote{历,快速通过。沛,多水草的沼泽地。}历圮\footnote{难以行走的道路。《孙子兵法·九地篇》:“行山林、险阻、沮泽,凡难行之道者,为圮地。”},兼舍环龟\footnote{宿营的时候要在中间高、四周低的地方宿营。}。
\end{yuanwen}

与敌人交战,要注意与环境的影响,在布阵时要注意背对风向,背靠高地,右边要临近高地,左边要依附于险要的地势。如果行军中遇到多水草的沼泽地或难以行走的道路,要迅速通过,尽量减少停留的时间,同时还要注意四面设防,以防止敌人的伏击。

\begin{yuanwen}
凡战,设[1]而观其作[2],视敌而举[3]。待则循而勿鼓[4],待众之作[5]。攻则屯[6]而伺之。



【注释】
[1]设:已经布置好战阵。
[2]作:行动反应。
[3]举:攻击。
[4]鼓:发起进攻。
[5]待众之作:等待敌人下一步的行动。
[6]屯:聚集兵力。
【译文】
在与敌人交战时,如果阵势已经布置好,就要先观察敌人的动向,再了解清楚敌军的兵力虚实情况,然后再决定是否发动攻击。敌人如果布置好了阵地来防备我军,就要注意不要顺着敌人的意图发动攻击,而是要坚守以了解敌人的行动。如果敌人主动发动攻击,我军就要集中兵力伺机打击敌人最薄弱的部分。
\end{yuanwen}

\begin{yuanwen}

凡战,众寡[1]以观其变。进退以观其固[2],危而观其惧[3],静而观其怠[4],动而观其疑[5],袭而观其治[6]。击其疑,加其卒[7],致其屈。袭其规[8],因其不避[9],阻其图[10],夺其虑[11],乘其惧。



【注释】
[1]众寡:这里指出兵的多少。
[2]固:稳固。
[3]惧:恐惧,恐慌。
[4]怠:懈怠。
[5]观其疑:观察敌人是否疑惑。
[6]治:政治不乱。
[7]卒:同“猝”。仓促,急速。
[8]规:部署。
[9]不避:冒险开进。
[10]图:意图。
[11]虑:谋虑,策略。
【译文】
在与敌人交战时,用数量不等的兵力去试探敌人,观察他的反应。可以通过部队的进退来观察敌人阵势的牢固程度,通过逼近敌人来观察敌人是否恐慌。在两军对峙时,可以通过自己按兵不动来观察敌人是否有懈怠之心;可以通过部队的运动来观察敌人是否有疑惑的表现;也可以用少量的兵力袭击敌人来观察敌人的队列布置是整肃还是混乱。在了解清楚敌人的情况以后,就要攻击他们薄弱的地方,在敌人仓猝迎战的时候就要抓紧进攻,从而使敌人陷入困境。在敌人没有任何防备的时候加大进攻的力度,以阻止敌人权谋的实施,利用敌人冒险开进犯的错误,粉碎敌人的战略企图。使他们无法思考新的应对策略,在敌人还在疑虑害怕的时候就击败敌人。
\end{yuanwen}


\begin{yuanwen}

凡从奔勿息[1],敌人或止于路则虑[2]之。
凡近敌都[3],必有进路。退,必有反[4]虑。
凡战,先则弊[5],后则慑[6],息则怠[7],不息亦弊,息久亦反其慑[8]。
书亲绝[9],是谓绝顾之虑[10]。选良[11]次兵,是谓益[12]人之强。弃任[13]节食,是谓开[14]人之意。自古之政也。


【注释】
[1]息:停息。
[2]虑:思考。
[3]敌都:敌方的城邑。
[4]反:后撤,返回。
[5]先则弊:出兵过早使士卒容易疲惫。
[6]后则慑:出兵过迟容易使士兵出现畏惧心理。
[7]息则怠:休息过久会使士兵产生懈怠之心。
[8]息久亦反其慑:指在对敌时如果休息时间过长,就会使士卒产生恐惧和过于谨慎的心理。
[9]书亲绝:禁止士兵与家人通信。
[10]绝顾之虑:断绝将士思恋家人的后顾之忧。
[11]选良:挑选良将。
[12]益:增强。
[13]弃任:丢弃笨重的设备。
[14]开:激发。
【译文】
在追击溃逃的敌人时,一定不要停止追击。敌人如果在逃跑途中停下来歇息,就要小心行事,认真思考敌人停下来的理由,防备敌人的埋伏。
如果逼近了敌人的都城,军队必须考虑好进攻的路线,同时还要考虑到返回的路线。如果没有考虑好进攻的路线就不要仓皇出击。如果不考虑退路,就有可能遭到敌人的埋伏。
在与敌人交战时,要注意出击的时间,如果出击过早士卒就容易疲劳,如果出兵过迟,则可能因为士卒不熟悉战场的情况而产生恐惧的情绪。如果在行军和战斗的过程中让士卒停下来休息,有可能使士卒产生懈怠之心。但如果不休息,又会使士卒因为体力消耗过大而降低战斗力。但如果休息时间过长,也会使士卒产生恐惧和过于谨慎的心理。
在战前,禁止士兵和家人通信,以断绝他们的后顾之忧。挑选良将,配备精良的兵器,以此来增强部队的战斗力。在战斗中丢弃一些不大实用的笨重的装备,减少随身携带的粮食,以激发士卒奋勇杀敌、誓死作战的勇气和决心。这些是自古以来作战的基本法则。
\end{yuanwen}

【评析】
司马穰苴在本篇中阐述了兵多与兵少不同的作战特点,强调要“设而观其作,视敌而举。待则循而勿鼓,待众之作。攻则屯而伺之”,要善于观察敌情。司马穰苴认为如果处于优势地位,就可以堂堂正正地与敌人交战;如果出于劣势地位,就要对敌人采取虚张声势的办法来迷惑敌人。在对敌作战过程中,还要充分利用有利的地形条件,“背风背高,右高左险,历沛历圮,兼舍环龟”。另外司马穰苴还要求在作战时,要禁止士卒与亲人通信,以免影响士气,“书亲绝,是谓绝顾之虑”,这样才可以稳定军心,保持全军高昂的作战士气。


【战例一】官渡之战

汉朝末年,各地军阀经过混战以后,北方地区基本上形成了曹操与袁绍对峙的局面,袁占有青、幽、冀、并四州之地。曹操把汉献帝挟持到许昌以后,形成了“挟天子以令诸侯”的局面,取得政治上的优势,在消灭袁术以后,曹操控制了兖、豫、徐州,黄河以南以及淮、汉以北大部地区,从而与袁绍形成沿黄河下游南北对峙的局面。袁绍因为兵力胜于曹操,自然不甘屈居于曹操之下,便想与曹操一决雌雄。公元199年六月,袁绍挑选精兵十万,战马万匹,南下进攻许昌,拉开了官渡决战的序幕。
在兵力对比上来说,袁绍的兵力要比曹操强大得多,这一点曹操的部将也因此认为难以战胜袁绍。但曹操在分析了情况以后,认为可以与袁绍进行战斗。
首先,针对袁绍军队的弱点,曹操集中精兵与袁绍对决。曹操认为袁绍志大才疏,胆略不足,刻薄寡恩,刚愎自用,兵多而指挥不明,将骄而政令不一,于是决定以所能集中的数万兵力抗击袁绍的进攻。在防守上,曹操没有展开全面防守,而是实施了重点防守,引诱袁绍在局部地区进行大决战。
公元200年正月,袁绍发布讨曹檄文,进军黎阳,企图渡河与曹军主力决战。袁绍首先派颜良进攻白马的东郡太守刘延,企图夺取黄河南岸要点,以保障主力渡河。四月,曹操为争取主动,求得初战的胜利,亲自率兵北上解救白马之围。此时谋士荀攸认为袁绍兵多,建议声东击西,分散其兵力,先引兵至延津,伪装渡河攻袁绍后方,使袁绍分兵向西,然后遣轻骑迅速袭击进攻白马的袁军,攻其不备,定可击败颜良。曹操采纳了这一建议,袁绍果然分兵延津。曹操乃乘机率轻骑前行,派张辽、关羽为前锋,急趋白马。关羽迅速迫近颜良军队,颜良仓促应战被斩杀,袁军溃败。
但这一战并没有改变战场的局势,而且在白马之战以后,曹操在撤退过程中,袁绍发兵追击曹操,当时曹操只有骑兵六百,驻扎于南阪下,而追击的袁军多达五六千骑,还有步兵在后跟进。曹操令士卒解鞍放马,并故意将辎重丢弃道旁。袁绍军一见果然中计,纷纷争抢财物。曹操突然发起攻击,终于击败袁军,杀了文丑,顺利退回官渡。
其次在战场对峙阶段,曹操在将少兵微的情况下坚持守阵以寻求战机。公元200年八月,袁绍军与曹操军队在官渡扎营对抗,双方在战斗中互有胜负,在守卫战中,曹军还制作了一种抛石装置的霹雳车,发石击毁了袁军所筑的楼橹。在僵持战中,双方坚持了三个月。
再次,曹操针对袁绍军远来粮草运输困难的特点,决定截断袁绍军的粮草供应,使袁绍军无法再战。曹操先后派徐晃、史涣截击袁绍的粮草队,焚烧了数千辆粮车。特别是听从来降的袁绍谋士许攸的建议,亲自率五千精兵奇袭乌巢,烧毁其全部粮草,从而导致前线的袁军军心动摇,内部分裂。曹操乘势出击,大败袁军。袁绍仓皇带八百骑退回河北,曹军先后歼灭和坑杀袁军七万余人。
官渡之战是袁曹双方力量转变、当时中国北部由分裂走向统一的一次关键性战役,对于三国历史的发展有着极其重要的影响。
导致官渡一战曹操胜利的因素是多方面的,但从中也不难看出曹操在指挥用兵上的高明之处:
第一,曹操根据敌强我弱的情况,没有采取以硬碰硬的政策,而是采用先进行坚固防守,再伺机寻找战机的策略。
第二,曹操善于采纳手下的意见。这一点可以说是官渡之战成败的关键。在要与曹操进行决战前,先后有田丰和沮授向袁绍提出了正确的建议,但袁绍都没有予以采纳。而曹操却充分采纳了谋士们的建议,特别是许攸的建议。在这一点上,曹操表现出知人善任、信任手下的特点,对降将许攸没有丝毫怀疑。
第三,运用奇战。针对袁绍军队强大、己方弱小的特点,没有进行正面的碰撞,而是通过奇袭袁绍军队粮草的谋略来动摇袁绍军队的军心,使其不战自乱,从而导致全军溃败。


【战例二】渭曲之战

西魏大统三年(537年),西魏大丞相宇文泰与东魏高欢进行交战,宇文泰运用奇战,以正面突击和两翼夹击的综合战术,成功打败了东魏高欢的进犯。
西魏大统三年,东魏将领高欢率领军队渡过黄河后,直逼西魏华州。由于高欢兵力众多,因此华州刺史王罴采取了防御的策略,全州上下防守非常严密。高欢见无法攻克华州,只好让军队渡过洛水,在许原之西驻扎下来。西魏大丞相宇文泰奉命率军抗击高欢军队。
宇文泰率领军队到达渭水南面时,所征的诸州兵马还没有聚齐,西魏诸将领以敌众我寡为理由,请求先让高欢继续西进,观察形势再商量对策。宇文泰反驳说:“高欢如果挺进咸阳城,军心将会产生动荡,民心也会不稳,不如现在乘其刚到立足未稳,趁势打击他。”
随后,宇文泰命令士卒立即在渭水之上赶造浮桥,同时命令士兵携带三天口粮,迅速渡过渭水,并让运载军需物资的部队由渭水南岸沿河向西进军。十月初一,宇文泰率军到达沙苑,距高欢军营仅六十余里。高欢知道后立即率军赶来。骑兵侦察员向宇文泰报告了这一情况,宇文泰和众将商议对策。骠骑大将军李弼说:“敌方兵力多,我方兵力少,在开阔之地列阵同敌开战对我方军队非常不利。从这里向东十里,在渭水一拐弯处,地形有利于我们,可以抢先占领此险处以等待敌军。”
于是宇文泰率军到达渭水拐弯处,选择好作战地形,背靠渭水自东向西列阵,以李弼所部为右翼,以赵贵所部为左翼,命令士兵把武器放倒在芦苇之中,听到鼓声就立即奋起出击。
黄昏时分,高欢率领大军到达,看到西魏军队兵力较少,于是就命令部队快速进攻,没想到由于兵力众多,士卒争相前进,部队一下子就乱了阵形。在这时,宇文泰突然擂响战鼓,躲藏在芦苇丛中的西魏士兵闻声都骤然奋起出击。骠骑大将军于谨等将率主力与高欢军正面交锋,李弼、赵贵率部从左、右两翼出击,把敌军分割为两段,正面突击并且两翼横击,在这种情势下,高欢军队被打得大败而逃。


\backmatter

\end{document}