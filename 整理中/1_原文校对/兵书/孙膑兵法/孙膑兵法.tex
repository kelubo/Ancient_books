%-*- coding: UTF-8 -*-
% 孙子兵法
% 孙子兵法.tex

\documentclass[12pt,UTF8]{ctexbook}

% 设置纸张信息。
\usepackage[a4paper,twoside]{geometry}
\geometry{
	left=25mm,
	right=25mm,
	bottom=25.4mm,
	bindingoffset=10mm
}

% 设置字体,并解决显示难检字问题。
\xeCJKsetup{AutoFallBack=true}
\setCJKmainfont{SimSun}[BoldFont=SimHei, ItalicFont=KaiTi, FallBack=SimSun-ExtB]

% 目录 chapter 级别加点(.)。
\usepackage{titletoc}
\titlecontents{chapter}[0pt]{\vspace{3mm}\bf\addvspace{2pt}\filright}{\contentspush{\thecontentslabel\hspace{0.8em}}}{}{\titlerule*[8pt]{.}\contentspage}

% 设置 part 和 chapter 标题格式。
\ctexset{
	chapter/name={第,篇},
	chapter/number={\chinese{chapter}}
}

% 设置古文原文格式。
\newenvironment{yuanwen}{\bfseries\zihao{4}}

% 设置署名格式。
\newenvironment{shuming}{\hfill\bfseries\zihao{4}}

% 注脚每页重新编号,避免编号过大。
\usepackage[perpage]{footmisc}

\title{\heiti\zihao{0} 孙子兵法}
\author{孙武}
\date{}

\begin{document}
	
\maketitle
\tableofcontents

\frontmatter
\chapter{前言}

《孙膑兵法》是1972年在山东临沂银雀山一号汉墓发现的失传两千多年的珍贵秘籍。它的发现不但为我国兵书宝库增添了极其珍贵的资料,更重要的是由于银雀山一号汉墓同时出土了《孙子兵法》(《吴孙子》)和《孙膑兵法》(《齐孙子》),使聚讼千年的有关孙膑的若干疑问,如:有无孙膑其人、有无孙膑其书、《孙子兵法》(《吴孙子》)和《孙膑兵法》(《齐孙子》)是否为一书等,涣然冰释。

据《史记·孙子吴起列传》记载,孙膑是孙武的后代子孙,曾和庞涓一起学习兵法。后被庞涓骗到魏国,施以酷刑,“断其两足而黥之”。幸而在齐国使臣的帮助下秘密回到齐国,由齐将田忌进于威王。“威王问兵法”,孙膑陈述了富国强兵等主张,并回答了威王提出的有关用兵决胜的问题,“遂以为师”。《史记·太史公自序》中也讲:“孙子膑脚而论兵法。”《汉书·艺文志》记载有“《齐孙子》八十九篇”,颜师古注云:“即孙膑。”并将它与孙武所著的《吴孙子》相提并论。又《吕氏春秋·不二》篇高诱注云:“孙膑,楚人,为齐臣,作谋八十九篇。”

自《孙膑兵法》问世以后,即在社会上广为流传。《吕氏春秋·不二》里说:“老聃贵柔,孔子贵仁,墨翟贵廉关尹贵清,子列子贵虚,阳生贵己,孙膑贵势。”该书把孙膑同老子、孔子、墨子、关尹等伟大的思想家相提并论,可以看出孙膑及其学说的地位和在当时的影响。而且《吕氏春秋》是在评述政治思想流派时特别提到孙膑的“贵势”思想的,是把军事上的“贵势”应用为政治上的贵势,这说明,孙膑及《孙膑兵法》不但在军事领域有重大的影响,而且在政治上也有一定的影响。司马迁在《孙子吴起列传》中也说:“孙膑以此名显天下,世传其兵法。”这都表明,孙膑本人和他的兵法在战国后期和秦汉之际的社会上是影响很大的。然而自《隋书·经籍志》以下却不见有关《齐孙子》(《孙膑兵法》的记载,只是在不少古籍中仍保存了一些《孙膑兵法》的佚文。在两汉魏晋时期,该书也多次被人征引,受到许多兵家学者的重视和赞誉,影响很大。尤其是孙膑在桂陵之战中创造的“围魏救赵”、攻其所必救的战法(见《孙膑兵法·擒庞涓》),已成为中外军事史上避实击虚的著名战例,被历代军事家和军队统帅们所效法和运用。毛泽东在《抗日游击战争的战略问题》中对这一战例也作过高度的评价。此外,孙膑在马陵之战中采用的“增兵减灶”、制造假象、诱敌就范的战法(见《孙膑兵法·陈忌问垒》),也成为历史上设伏牙敌的重要典范。值得庆贺的是,1972年在山东临沂银雀山一号汉墓中发现了失传两千多年的《孙膑兵法》,使人们才得以重见久已失传的《孙膑兵法》的部分篇章。

大家知道,孙武的《孙子兵法》是我国现存最早的兵书和中国古代军事理论的代表性著作,被世界公认为“兵学圣典”。后世学习、研究、运用《孙子兵法》的人很多,孙膑就是其中的佼佼者。作为孙武的后裔孙膑在军事上的天赋和才能得益于家学渊源。他不但在实际指挥作战中用兵如神,功勋卓著,成为一代名将,而且在军事理论上也作出了突出贡献。他所撰写的《孙膑兵法》在体系和风格上与《孙子兵法》一脉相承、互相辉映,是继《孙子兵法》之后“孙子学派”的又一部重要著作,在我国军事学说史上久负盛名,影响深远。

临沂银雀山一号汉墓发现的《孙膑兵法》,现存竹简四百四十余枚,简长27.5厘米,3道编绳。经过整理,现存十六篇:《擒庞涓》《见威王》《威王问》《陈忌问 垒》《选卒》、《月 战》、《八阵》、《地 葆》、《势 备》
《兵情》、《行选》、《杀士》、《延气》、《官一》《五教法》.(强兵》,其中除《见威王》《兵情》、《强兵》三篇篇名为整理者根据内容所加外,其余均为原书篇题。(擒庞涓》记载了孙膑在“围魏救赵”的桂陵之战中为齐国谋划、指挥并取得胜利的经过,所记情节与《史记》《战国策》大体相同。《见威王》记述了孙膑初见威王时所阐述的战争的胜负与国家存亡的关系,表现了孙膑的战争观。《威王问》记述了孙膑与齐威王、田忌关于用兵的对话。《陈忌问垒》是田忌和孙膑关于设垒布防问题的对话。《选卒》是讲精选士卒的问题。《八阵》讲的是在战场上根据敌情、地形选用的“八阵之宜”。

势备》是讲军队力量的配备和使用。《行选》是讲如何在战争中权衡利弊,以支持长时期的作战。《延气》是讲如何使士卒保持旺盛的战斗士气等。总之,这部失传两千多年的兵法内容十分丰富,不仅继承了《孙子兵法》中的许多思想因素,而且在不少方面对《孙子兵法》的理论还作了进一步的发展,在战争观、军队建设和作战指导上都提出了一些新的有价值的观点和原则。
孙膑生活在战国中期,此时各国在政治、经济上都发生了很大的变革,战争日趋频繁和残酷,社会秩序也日趋动荡和不安。面对当时的社会实际,孙膑兵法》的内容也有了自己鲜明的时代特色,甚至有些理论作了进一步的发展。如该书指出,战争不仅是关系到国家生死存亡的大事,而且是除暴乱、禁争夺、实现并巩固国家统一的重要手段。它针对战国中期七雄割据争立的现实,驳斥了齐人“政教”“散粮”“静”等理论,鲜明地提出了“战胜而强立,故天下服”的观点。这一观点完全符合当时诸雄争霸、全国趋向统一的客观要求,具有进步意义。与“强兵”的观点密切相联的,孙膑同时还强调“富国”,他认为富国是“强兵之急者也”。主张发展生产,增强经济实力,为强化军事、强化国防提供可靠的物质基础。这无疑是战国中期变法革新思想的反映。在继承《孙子兵法““攻而必取者,攻其所不守”观点的同时,孙膑也提出了“必攻不守”的积极进攻战略,并将这一原则放在高于一切的地位。这是我国古代战争史上一次划时代的变革,它使大规模的机动作战成为可能。同时,“必攻不守的观点也更加深刻地体现了战争的主要目的,即保存自己、消灭敌人。这对指导战争具有较为普遍的意义。在该书中孙膑特别强调“知道”他从战争实践中总结出了某些规律性的东西,并极端重视对战争规律的把握和研究,他把这些规律性的东西称之为“道”他明确指出“知道,胜”,“不知道,不胜”“以绝胜败安危者,道也”。他要求“安万乘国、广万乘王、全万乘之民命者,唯知道”,他认为作为战争的指挥者必须正确认识战争的规律,必须全面掌握天文、地理、民心、敌情、阵法这些与战争胜负直接相关的重要因素,“上知天之道,下知地之理,内得其民之心,外知敌之情,阵则知八阵之经”。孙膑认为,只有“知其道者,兵有功,
主有名”,才能赢得战争的胜利,扬名天下。这种按客观规律办事的观点,是对《孙子兵法》中有关思想的进一步发展,构成了我国古代军事学说的优秀传统。
此外,孙膑兵法》还有一个显著特点就是非常重视“势”。书中多次讲到了“势”的使用和发挥,孙膑强调在一定客观条件的基础上,要主动地造成有利于己、不利于敌的态势,争取作战的主动权,从而达到克敌制胜的目的。孙膑把“势”看作是改变力量对比、转化敌我势态、战胜强敌众敌的重要手段。他认为,战争双方的力量对比和各种利弊因素,既是客观的,又是可变的,在一定的条件下是可以相互转化的,“积疏相变”“众寡相变”指挥战争的人要善于“便势利地”因势利导,就能扼制敌军的优势,改变我军的劣势,打破敌我的均势,能动地夺取战争的胜利。例如在“我强敌弱,我众敌寡”的情况下,要善于“赞师”隐蔽实力,示敌以弱,诱敌出战,聚而牙之;在“敌众我寡,敌强我弱”的情况下,要善于“让威”“避而骄之”用“告之不敢,示之不能”的手段“以骄其意”“以惰其志”,给敌人造成错觉,使其变众为寡,孤立无援,后发制人;在势均力敌时,要善于佯败诱敌,然后“并卒而击之”;要善于利用地形,“料敌计险”“居生击死”使自己处于生地,置敌人于死地。要根据不同的地形情况和各兵种的不同特点,“易则多其车,险则多其骑,厄则多其弩”,因情用兵,扬长避短。对于凭险固守之敌,要用“攻其所不救”的战法,“使离其固”“施伏设援,击其移庶”,调动敌人脱离险固,这样才能战胜敌人,等等。“势”的变化是无穷尽的,通过造“势”可以使敌人“分离而不相救”,“受敌而不相知”“沟深垒高不得以为固”“甲坚兵利不得以为纬”“士有勇力不得以为强”。这些都是孙膑立“势”思想的具体体现。
此外在书中,孙膑还特别强调阵、势、变、权这四个环节。他用剑来比喻“阵”,用弓弩来比喻“势”,用舟车来比喻“变”用长柄器来比喻“权”,只有掌握好这四个关键的战术环节,才能“破强敌,取猛将”。孙膑的这些战术思想,在诸多方面丰富、发展了《孙子兵法》中关于“兵因敌而制胜”的理论。在书中,孙膑也提出了以人为贵的治军原则,并强调在战争实践中充分发挥人的主观能动性。他认为:“于天地之间,,莫贵于人。”要求将帅具备“五德”,即智、信、仁、勇、严等品质,要“得主专制”拥有指挥战争的全权)“知道”(懂得作战的规律)、“得众”(取得士兵的信赖)“左右和’(上下之间要团结)等,这样才能担当起指挥战争的重任。此外他还强调“兵之胜在于篡卒(即选拔士卒)其勇在于制”,要求对士兵严格挑选,严格训练,使之'素听”、“素信”,赏罚分明,令行禁止,这样才有战斗力。所有这些思想,都是《孙膑兵法》对我国古代军事理论宝库的突出贡献,在长期的战争实践中发挥着重要的指导作用。
今年春三月,中华书局约我对《孙子兵法》孙膑
兵法》进行新的注释。在注释《孙子兵法》过程中,我们参考和吸收了以往一些同行师友的研究成果,如李零先生的《吴孙子发微》中华书局的《孙子兵法新注)以及文物出版社的《银雀山汉墓竹简(壹)·孙子兵法》(1985 年版)等。在《孙膑兵法》的注释过程中,我们参考文物出版社出版的《银雀山汉墓竹简(壹)·孙膑兵法》(1985年版)、李均明先生的《孙膑兵法译注》张震泽先生的《孙膑兵法校理》邓泽宗先生的《孙膑兵法注译》等,在此向他们表示诚挚的感谢!

    \mainmatter

孙膑兵法
作者:战国·孙膑
· 上编
· 凡例
· 擒庞涓
· [见威王]
· 威王问
· 陈忌问垒
· 篡卒
· 月战
· 八阵
· 地葆
· 势备
· [兵情]
· 行篡
· 杀士
· 延气
· 官一
· [强兵]
·下编
· 十阵
· 十问
· 略甲
· 客主人分
· 善者
· 五名五恭
· [兵失]
· 将义
· [将德]
· 将败
· [将失]
· [雄牝城]
· [五度九夺]
· [积疏]
· 奇正

凡例
一、本书分上下两编。上编前四篇记孙膑擒庞涓事迹以及孙膑与齐威王、田忌的问答。其它各篇篇首都称“孙子曰”,但内容书体都与银雀山汉墓所出孙武兵法佚篇不相类,所以可以肯定是孙膑兵法。下编各篇没有提到孙子,今据内容、文例及书体定为孙膑兵法。由于竹简残断散乱,而孙膑兵法又早巳亡佚,无从核对,整理工作中肯定会有错误。本书中可能有一些本来不属于孙膑兵法的内容搀杂在内,请读者指正。
二、每篇释文前标出篇题。凡由编者补加的篇题,外加[]号以示区别。
三、一篇中所收各简,凡文字相连的,或其间虽有缺字、缺简,但确知其属于同一段文字的,释文都连成一段写。简文提行分段时,释文也分段。
四、有的简虽然可以确定属于某篇,但不能确定它在篇中的位置。有的简很象是属于某篇的,但又不能十分肯定。释文把这些简分别附于各篇之末,加三个 * 号与成段释文隔开。这类简除去彼此文字相连的以外,每简释文都提行写。
五、不能辨识的字以及由于竹简残断而缺去的字用□号表示,但字数超过五个或字数无法确定时(包括中间缺整简的情况),则用……号表示。与……号相连约□号一般省去。
六、根据上下文补出的缺文或简文原来脱字,外加[]号。
七、简文原来的各种标号,释文一律略去,另加标点符号。
八、如果简文中引语的开头或结尾处正在残缺部分,释文就只标下引号或上引号。
九、为了便利广大读者,简文中的异体字和假借字多改写为通行字,如“{月豊}”改作“体”,“宭”改作“窘”,“亓”改作“其”,“{亻啇}”“適”改作“敌”,“陳”改作“阵”,“侍”改作“待”,“埶”改作“势”,“請”改作“情”,“兑”改作“锐”等。
—— 银雀山汉墓竹简整理小组

擒庞涓
本篇记述孙膑在“围魏救赵”之战中,用避实击虚、“攻其必救”等办法,在桂陵大破魏军,俘获庞涓。这是孙膑运用他的军事思想取得胜利的一个著名战例。
—— 银雀山汉墓竹简整理小组
擒庞涓(1)
昔者,梁君将攻邯郸(2),使将军庞涓、带甲(3)八万至于茬丘(4)。齐君(5)闻之,使将军忌子(6)、带甲八万至……竞。庞子攻卫(7)□□□,将军忌[子]……卫□□,救与……曰:“若不救卫,将何为?”孙子曰:“请南攻平陵(8)。平陵,其城小而县大,人众甲兵盛,东阳战邑(9),难攻也。吾将示之疑。吾攻平陵,南有宋(10),北有卫,当途有市丘(11),是吾粮途绝也。吾将示之不知事。”于是徒舍而走平陵(12)。……陵,忌子召孙子而问曰:“事将何为?”孙子曰:“都大夫孰为不识事(13)?”曰:“齐城、高唐(14)。”孙子曰:“请取所……二大夫□以□□□减□□都横卷四达环涂(15)□横卷所□阵也。环涂{车皮}甲(16)之所处也。吾未甲劲,本甲不断(17)。环涂击柀(18)其后,二大夫可杀也(19)。”于是段齐城、高唐为两(20),直将蚁附(21)平陵。挟{艹世}(22)环涂夹击其后,齐城、高唐当术而大败(23)。将军忌子召孙子问曰:“吾攻乎陵不得而亡齐城、高唐,当术而厥(24)。事将何为?”孙子曰:“请遣轻车西驰梁郊(25),以怒其气。分卒而从之,示之寡(26)。”于是为之。庞子果弃其辎重(27),兼趣舍(28)而至。孙子弗息而击之桂陵(29),而擒庞涓(30)。故曰,孙子之所以为者尽矣(31)。
四百六(32)
* * *
……子曰:“吾……
……孙子曰:“毋待三日……”
(1) 此是篇题,写在本篇第一简简背。庞涓,战国时人,早年曾与孙膑同学兵法,后被魏惠王任为将军。简文中庞涓又称庞子。
(2) 梁君,指魏国国君惠王 (公元前三六九-前三一九年在位)。魏国在惠王时迁都大梁(今河南开封),故魏又称梁。邯郸,赵国国都,今河北邯郸。
(3) 带甲,穿有铠甲的士卒,此处泛指军队。
(4) 茬丘,地名,其地末详。
(5) 齐君,指齐威王(公元前三五六-前三二○年在位)。
(6) 忌子,即田忌,齐国的将军,曾荐孙膑于齐威王。
(7) 卫,国名,原建都朝歌(今河南淇县),春秋时迁都帝丘(今河南濮阳)。
(8) 平陵,地名。据下文“吾攻平陵,南有宋,北有卫”,则此平陵应在宋、卫之间。
(9) 东阳,地区名。战邑,指平陵。意谓平陵是东阳地区军事上的重要城邑。
(10) 宋,国名,原建都商丘(今河南商丘),战国初期迁都彭城(今江苏徐州)。
(11) 市丘,地名,在魏国。
(12) 徙舍,拔营。走,急趋。
(13) 都,齐国称大城邑为都。都大夫,治理“都”的长官。这里似指那些率领自己都邑军队跟从田忌参加战争的都大夫。孰,谁。
(14) 齐城、高唐,齐国的两个都邑。齐城,疑即齐都临淄,在今山东临淄。高唐,在今山东高唐、禹城之间。
(15) 环涂,下文屡见,疑是魏军驻地或将领之名。一说“环涂”即“环途”,迂回的意思。
(16) {车皮},疑借为彼此之彼。
(17) 末甲,前锋部队。本甲,后续部队。
(18) 柀,疑借为破。
(19) 孙膑的意思似是要牺牲“不识事”的二大夫,使魏军产生齐军软弱无能的错觉。
(20) 段,借为断。意谓把齐城、高唐二大夫带领的军队分成两部。
(21) 蚁附,指攻城,形容军士攻城时攀登城墙,如蚂蚁附壁而上。
(22) 挟{艹世},疑是魏军驻地或将领之名。一说借为浃渫,形容军队相连不断。
(23) 术,道路。意谓齐城和高唐二大夫的军队在行军的道路上大败。
(24) 厥,借为蹶(jue绝),摔倒,败。
(25) 请派遣轻快的战车向西直趋魏国国都大梁城郊。
(26) 以上两句意谓把队伍分散,让敌人觉得我方兵力单薄。
(27) 辎(zi资)重,军用物资器材。
(28) 趣,行进。舍,止息。趣舍,指行军。“兼趣舍”就是急行军,昼夜不停。
(29) 弗息,不停息。桂陵,地名,在今山东菏泽东北。
(30) 《史记·魏世家》记魏惠王十八年(据《竹书纪年》当为十七年,公元前三五三年)齐、魏桂陵之战,没有提到庞涓;记后十三年(据《竹书纪中》当为后十二年)的马陵之战时,说庞涓被杀,太子申被虏(《史记·孙子吴起列传》所记略同,但谓庞涓自杀)。简文记庞涓于桂陵之役被擒,与《史记》所记不同。
(31) 尽,终极。意思是称赞孙膑的作为尽善尽美。
(32) 此数字为本篇字数总计。

[见威王]
此篇题为编者所加。本篇记孙膑初见齐威王时,陈述自己对战争的看法。孙膑认为只有通过战争才能禁止争夺,对春秋战国以来儒家所鼓吹的以“仁义”去战的说教,作了有力的批判。
—— 银雀山汉墓竹简整理小组
孙子见威王,曰:“夫兵者,非士恒势也(1)。此先王之傅道也(2)。战胜,则所以在亡国而继绝世也(3)。战不胜,则所以削地面危社稷(4)也。是故兵者不可不察。然夫乐兵(5)者亡,而利胜(6)者辱。兵非所乐也,而胜非所利也。事备(7)而后动。故城小而守固者,有委(8)也;卒寡而兵强者,有义也。夫守而无委,战而无义,天下无能以固且强者。尧有天下之时,黜王命而弗行者七,夷(9)有二,中国(10)四,……素佚而致利也(11)。战胜而强立,效天下服矣。昔者,神戎战斧遂(12);黄帝战蜀禄(13);尧伐共工(14);舜伐劂□□而并三苗(15),……管;汤放桀(16);武王伐纣(17);帝奄(18)反,故周公浅之(19)。故曰,德不若五帝(20),而能不及三王(21),智不若周公,曰我将欲责(22)仁义,式(23)礼乐,垂衣裳(24),以禁争夺。此尧舜非弗欲也,不可得,效举兵绳(25)之。”
(1) 士,借为恃。意谓军事上没有永恒不变的有利形势可以依赖。
(2) 傅,借为敷,布,施。意谓这是先王所传布的道理。一说“傅”为“传”宇之误。
(3) 在,存。孙膑这句话的意思是说战争的胜负关系到国家的存亡,与孔丘复辟奴隶制的反动纲领“兴灭国,继绝世”的含义不同。
(4) 社,士神。稷(ji既),谷神。古代以社稷代表国家。
(5) 乐兵,好战。
(6) 利胜,贪图胜利。
(7) 事备,做好战争的准备。
(8) 委,委积,即物资储备。
(9) 夷,指古代我国东方地区的部族。
(10) 中国,指中原地区。
(11) 此句上文残缺,原文大概是说帝王不能无所作为而致利。佚,同逸,安闲。
(12) 神戎,即神农。斧遂,或作补遂。《战国策·秦策》:“昔者神农伐补遂。”
(13) 蜀禄,即涿鹿,地名。《战国策·秦策》:“黄帝伐涿鹿而禽蚩(chi痴)尤。”
(14) 共工,传说中的部落首领。
(15) 并,借为屏,屏除,放逐。传说舜曾征伐过南方部落三苗。
(16) 汤,商朝开国国君。桀,夏朝最后的国君。放,流放。
(17) 武王指周武王,周王朝的建立者。纣,即商纣王,商朝最后一个王。
(18) 帝,疑是商宇之误。奄,商的同盟国,在今山东曲阜东。
(19) 周公,周武王弟。武王死,子成王年幼,周公辅政。浅,借为践,毁、灭之意。据《史记·周本纪》记载,周灭商后,被封的纣王之子武庚又联合奄、徐等国叛周,被周公征服。
(20) 关于五帝,历来说法不一。据《史记》,指黄帝、颛顼(zhuanxu专恤)、帝喾(ku酷)、尧、舜。简文似以神农为五帝之一。
(21) 三王,指夏、商、周三代开国的君主,即夏禹、商汤、周文王和周武王。
(22) 责,借为积。
(23) 式,用。
(24) 譬喻雍容礼让,不进行战争。
(25) 绳,纠正。意谓以战争解决问题。

威王问
本篇记述孙膑与齐威王、田忌关于用兵的问答。前一部分就敌我兵力对比的不同情况,提出不同的作战方法。后一部分主要指出用兵最重要的是“必攻不守”。
—— 银雀山汉墓竹简整理小组
威王问(1)
齐威王问用兵孙子(2),曰:“两军相当,两将相望(3),皆坚而固,莫敢先举(4),为之奈何?”孙子答曰:“以轻卒尝(5)之,贱而勇者将(6)之,期于北(7),毋期于得(8)。为之微阵以触其侧(9)。是谓大得。”威王曰:“用众用寡有道乎?”孙子曰:“有”。威王曰:“我强敌弱,我众敌寡,用之奈何?”孙子再拜曰:“明王之问。夫众且强,犹问用之,则安国之道也。命(10)之曰赞师。毁卒乱行(11),以顺其志,则必战矣。”威王曰:“敌众我寡,敌强我弱,用之奈何?”孙子曰:“命曰让威。必臧其尾,令之能归(12)。长兵(13)在前,短兵(14)在□,为之流弩,以助其急者(15)。□□毋动,以待敌能(16)。”威王曰:“我出敌出,未知众少,用之奈何?”孙子[曰](17):“命曰……威王曰:“击穷寇奈何?”孙子[曰]……可以待生计矣。”威王曰:“击均(18)奈何?”孙子曰:“营而离之(19),我并卒(20)而击之,毋令敌知之。然而不离(21),按而止(22)。毋击疑。”威王曰:“以一击十,有道乎?”孙子曰:“有。攻其无备,出其不意(23)。”威王曰:“地平卒齐(24),合(25)而北者,何也?”孙子曰:“其阵无锋也。”威王曰:“令民素听(26),奈何?”孙子曰:“素信(27)。”威王曰:“善哉!言兵势不穷(28)。”
田忌问孙子曰:“患兵者何也?困敌者何也?壁延不得者何也?失天者何也?失地者何也?失人者何也?请问此六者有道乎?”孙子曰:“有。患兵者地也,困敌者险也。故曰,三里{氵籍}洳将患军(29)……涉将留大甲(30)。故曰,患兵者地也,困敌者险也,壁延不得者{洰虫}寒(31)也,……奈何?”(32)孙子曰:“鼓而坐之(33),十而揄之(34)。”田忌曰:“行阵已定,动而令士必听,奈何?”孙子曰:“严而示之利(35)。”田忌曰:“赏罚者,兵之急者(36)耶?”孙子曰:“非。夫赏者,所以喜众,令士忘死也。罚者,所以正乱(37),令民畏上(38)也。可以益胜(39),非其急者也。”田忌曰:“权、势、谋、诈,兵之急者耶?”孙子曰:“非也。夫权者,所以聚众也。势者,所以令士必斗也。谋者,所以令敌无备也。诈者,所以困敌也。可以益胜,非其急者也。”田忌忿然作色:“此六者,皆善者(40)所用,而子大夫(41)曰非其急者也。然则其急者何也?”孙子曰:“料敌计险(42),必察远近,……将之道也。必攻不守(43),兵之急者也。……骨也。”田忌问孙子曰:“张军(44)毋战有道?”孙子曰:“有。倅险增垒(45),诤戒(46)毋动,毋可□□毋可怒。”田忌曰:“敌众且武,必战有道乎?”孙子曰:“有。埤垒广志(47),严正辑众(48),避而骄之,引而劳之,攻其无备,出其不意,必以为久(49)。”田忌问孙子曰:“锥行者何也?雁行者何也(50)?篡卒(51)为士者何也?劲弩趋发(52)者何也?飘风之阵者何也?众卒(53)者何也?”孙子曰:“锥行者,所以冲坚毁锐也。雁行者,所以触侧应□[也]。篡卒力士者,所以绝阵取将(54)也。劲弩趋发者,所以甘战持久也。飘风之阵者,所以回□□□[也]。众卒者,所以分功有胜也。”孙子曰:“明主、知道(55)之将,不以众卒几(56)功。”孙子出而弟子问曰:“威王、田忌臣主之问何如?”孙子曰:“威王问九,田忌问七(57),几(58)知兵矣,而未达于道(59)也。吾闻素信者昌,立义……用兵无备者伤,穷兵(60)者亡。齐三世其忧矣(61)。”
* * *
……善则敌为之备矣。”孙子曰……
……孙子曰:“八阵已陈……
……孙子……
……险成,险成敌将为正,出为三阵,……
……倍人也,按而止之,盈而待之,然而不□……
……无备者困于地,不□者……
……士死□而傅……
(1) 此是篇题,写在本篇第一简简背。
(2) 齐威王问用兵的道理于孙膑。
(3) 相望,对峙。
(4) 先举,先采取行动。
(5) 尝,试探。
(6) 将,率领。
(7) 期,预期。北,败北。
(8) 得,得胜。
(9) 微,隐蔽的。意谓以一部分隐蔽的兵力袭击敌军的侧面。
(10) 命,名。
(11) 卒,古代军队组织的一种单位。行,指队列。意谓故意使阵列显得混乱,以诱惑敌人。
(12) 臧,疑借为藏。意谓隐蔽好后面的部队,以便撤退。
(13) 长兵,长柄兵器,如戈矛。
(14) 短兵,短柄兵器,如刀剑。
(15) 弩,用机械发箭的弓。流弩,即机动的弩兵。意谓在危急的时候,以机动的弩兵救应。
(16) 《通典》卷一百五十九引《孙子》佚文:“敌鼓噪不进,以观吾能。”“能”宇用法与此相近。
(17) “曰”宇原简写脱,据文义补。
(18) 击均,攻击势均力敌的敌人。
(19) 营,迷惑。离,分离。意谓迷惑敌人,使之分散兵力。
(20) 并卒,集中兵力。
(21) 不离,谓敌人不分散兵力。
(22) 指我方按兵不动。
(23) 此二句见于《孙子·计》。
(24) 平,平敞。齐,严整。此句意谓地形和士卒条件都很好,却打败仗。
(25) 合,交战。
(26) 素,平时,一贯。听,听从命令。
(27) 信,守信用。
(28) 一说此句应读作:“善哉言!兵势不穷……”此简与下一简之间尚有缺简。
(29) {氵籍}洳,即沮洳(ju ru巨入),沼泽泥泞地区。意谓周围若有三里沼泽泥泞地带,则将为军队的患害。
(30) 大甲,疑指全副武装、铠甲坚厚的兵卒。
(31) {洰虫}寒,疑借为渠幰,即渠{巾詹},亦称渠答,张在城上防矢石的设备。一说渠答就是蒺藜。关于蒺藜,参看《陈忌问垒》注(4)。
(32) 此处下引号与前一上引号无关。 “……壁延不得者{洰虫}寒也……”是孙膑的话,“……奈何?”应是田忌的话,其间有脱简。
(33) 鼓,击鼓。古代用鼓指挥进攻。坐,疑借为挫。此句可能是说用进攻来挫败敌人。
(34) 揄,引。疑此句意谓以多种办法引诱敌人。
(35) 意谓要有严明的法纪,又要有奖励。
(36) 急者,最要紧的事情。
(37) 正乱,整饬军纪。
(38) 畏上,敬畏上级。
(39) 益胜,有助于取胜。
(40) 善者,指善战者。
(41) 子大夫,敬称,此处指孙膑。
(42) 分析敌情,审察地形。
(43) 指以进攻为主,而不是以防御为主的战略。
(44) 张军,即陈兵。
(45) 倅,借为萃,居止的意思。意谓凭据险要,增高壁垒。
(46) 诤(zheng证),借为静。戒,戒备。意谓加强戒备,按兵不动。
(47) 埤,同卑。广志,发扬士气。意谓修筑低垒,表示无所畏惧,以激励士气。
(48) 正,疑借为政。辑,团结。意谓严明法令,以团结士卒。
(49) 意谓必须持久。
(50) 锥行、雁行,皆阵名,参看《十阵》。
(51) 篡,借为选。选卒,经过挑选的善战的士卒。
(52) 劲弩,强弩。趋发,利箭。
(53) 众卒,与选卒相对,指一般士卒。
(54) 绝阵取将,破敌阵、擒敌将。
(55) 道,法则,规律。
(56) 几,这里作指望讲。
(57) 九和七疑指威王与田忌所问问题的数目。据上文,威王所问有“两军相当……”、“我强敌弱……”、“敌众我寡……”、“我出敌出……”、“击穷寇”、“击均”、“以一击十”、“地平卒齐……”、“令民素听”等九个问题,田忌所问有“患兵者何也……”、“……奈何”、“行阵已定……”、“兵之急者”、“张军毋战”、“敌众且武必战”、“锥行者何也……”等七个问题,与此处所说的数字正相符合。
(58) 几,这里作接近讲。
(59) 未达于道,意谓还没能达到掌握战争规律的境地。
(60) 穷兵,指穷兵黩(du读)武。
(61) 齐国在威王、宣王时,国势很强,至湣王末年为燕国所败之后,国势遂衰。自威王至湣王,恰为三世。由此看来,孙膑兵法有可能是孙膑后学在湣王以后写定的。

陈忌问垒
本篇记田忌与孙膑之间的问答。简文残缺,现存部分主要说明在未能构筑壁垒时,如何组织配备各种兵力来进行作战。
—— 银雀山汉墓竹简整理小组
陈忌问垒(1)
田忌问孙子曰:“吾卒……不禁,为之奈何?”孙子曰:“明将之问也。此者人之所过而不急也。此□之所以疾……志也。”田忌曰:“可得闻乎?”曰:“可。用此者,所以应猝窘处隘塞死地之中也(2)。是吾所以取庞□而擒太子申也(3)。”田忌曰:“善。事已往而形不见。”孙子日:“蒺藜者,所以当沟池也(4)。车者,所以当垒[也]。□□[者],所以当堞(5)也。发(6)者,所以当埤堄也(7)。长兵次之,所以救其隋也(8)。鏦(9)次之者,所以为长兵□也。短兵次之者,所以难其归而檄其衰也(10)。弩次之者,所以当投机也(11)。中央无人,故盈之以……卒已定,乃具其法。制曰:以弩次蒺藜,然后以其法射之。垒上弩戟分(12)。法曰:见使{世木}来言而动……去守五里置候(13),令相见也。高则方之,下则圆之(14)。夜则举鼓,昼则举旗。”
* * *
……田忌问孙子曰:“子言晋邦(15)之将荀息(16)、孙轸(17)之于兵也,未……(18)
……无以军恐不守。”忌子曰:“善。”田忌问孙子曰:“子言晋邦之将荀息、孙[轸]……
……也,劲将之阵也。”孙子曰:“士卒……
……田忌曰:“善。独行之将也。……
……言而后中。”田忌请问……
……人。”田忌请问兵情奈何?……
……见弗取。”田忌服问孙……
……橐□□□焉。”孙子曰:“兵之……
……应之。”孙子曰:“伍……
……孙子曰:……
……见之。”孙子……
……以也。”孙……
……将战书柧(19),所以哀正也。诛□规旗,所以严后也。善为阵者,必□□贤……
……明之吴越,言之于齐。曰知孙氏之道者,必合于天地。孙氏者(20)……
……求其道,国故长久。”孙子……
……问知道奈何。”孙子……
……而先知胜不胜之谓知道。□战而知其所
……所以知敌,所以曰智,故兵无……
(1) 此是篇题,写在本篇第一简简背。陈忌即田忌,陈、田二字古代音近通用。
(2) 应猝,应付突然发生的事变。这句的意思可能是说:这种方法是用来应付处于隘塞死地之中的紧急情况的。
(3) “庞”下所缺之字当为“子”或“涓”字。太子申,魏惠王的长子,参看《擒庞涓》注(30)。
(4) 蒺藜,古代用木或金属制成的带刺的障碍物,布在地面以阻碍敌军前进。因与蒺藜果实形似,故名蒺藜。池,护城河。此句意谓蒺藜的作用相当于沟池。
(5) 堞(die碟),城墙上的矮墙。
(6) 发,疑借为瞂(fa伐),即盾。
(7) 埤堄(bi ni闭逆),城墙上有孔的矮墙。意谓瞂的作用相当于埤堄。
(8) 隋,疑借为隳(hui灰),危也。
(9) 鏦(cong葱),小矛。
(10) 徼,通“邀”,截击。衰,疲惫。意谓截断敌军的归路,阻击疲惫的敌人。
(11) 投机,抛石机。意谓弩的作用相当于抛石机。
(12) 分,半。意谓壁垒上弩和戟各占一半。
(13) 候,即斥候。意谓距守望之处五里设置哨所。
(14) 方和圆疑指哨所的外形。
(15) 晋邦,晋国。
(16) 荀息,春秋时晋国名将。
(17) 孙轸(zhen诊),《汉书·艺文志》兵形势家有《孙轸》五篇、图二卷。疑孙轸即先轸,春秋时晋国名将。
(18) 自此以下各简,字体与本篇前面的简文相似,其中有的简似不属于本篇,但由于残缺过甚,不能单独成篇,姑附于此。
(19) 柧,或作觚(gu姑),古人在上面写字用的多棱的木条。
(20) 这里大概是把孙武、孙膑的军事理论作为一家的学说看待。“明之吴越”,是说孙武运用此种军事理论于吴越。“言之于齐”,是说孙膑以此种军事理论言之于齐威王。由于兼包两个孙子而言,所以称“孙氏”,不称“孙子”。

篡卒
本篇论述关系战争胜负的一些重要因素。
—— 银雀山汉墓竹简整理小组
篡卒(1)
孙子曰:兵之胜在于篡卒(2),其勇在于制(3),其巧在于势(4),其利在于信(5),其德在于道(6),其富在于亟归(7),其强在于休民(8),其伤在于数战(9)。孙子曰:德行者,兵之厚积也(10)。信者,兵[之](11)明赏也。恶战者,兵之王器也(12)。取众者,胜□□□也。孙子曰:恒胜有五:得主专制,胜(13)。知道,胜。得众,胜。左右和,胜。量敌计险,胜。孙子曰:恒不胜有五:御将,不胜(14)。不知道,不胜。乖将,不胜(15)。不用间(16),不胜。不得众,不胜。孙子曰:胜在尽□,明赏,选卒,乘敌之□。是谓泰武之葆。孙子曰:不得主弗将也……
* * *
……令,一曰信,二曰忠,三曰敢。安(17)忠?忠王。安信?信赏。安敢?敢去不善。不忠于王,不敢用其兵。不信于赏,百姓弗德。不敢去不善,百姓弗畏。
二百卅五
(1) 此是篇题,写在本篇第一简简背。篡借为选,篡卒即选卒,参看《威王问》注(51)。
(2) 军队打胜仗在于选用强卒。
(3) 士卒作战勇敢在于军法严明。
(4) 军队作战机动灵活,在于利用形势。
(5) 利,锐。意谓军队战斗力强,在于将帅言而有信。一说“利”即利害之利,此句意谓将帅有信,为军队的利之所在。
(6) 军队具有好的素质,在于将帅明白用兵的道理。
(7) 军用不绌,在于速战速决。亟(ji及),急。
(8) 军队战斗力强,在于养精蓄锐。
(9) 军队战斗力挫伤,在于频繁作战。
(10) 厚积,丰富的储备。意谓德行是军队的凭藉。
(11) “之”字原简写脱,据文义补。
(12) 恶(wu勿),厌恶。恶战,不好战。王器,王者之器。意谓不好战才是用兵的根本。
(13) 将帅得到君主信任,有指挥作战的全权,可以胜利。
(14) 御,驾驭,控制。意谓将帅受君主牵制,不能自主,就不能胜利。
(15) 乖,离异。意谓将帅不和,不能胜利o
(16) 间,间谍。
(17) 安,疑问代词,相当于现代语的“哪里”。

月战
本篇谈到战争胜败与日、月、星的关系。古人认为月主阴,象征刑杀,所以用兵宜在月盛之时。篇中虽然涉及此类流行于当时的迷信说法,但篇首指出“间于天地之间,莫贵于人”,特别强调了人的作用。
—— 银雀山汉墓竹简整理小组
月战(1)
孙子曰:间于天地之间(2),莫贵于人。战□□□□不单。天时、地利、人和,三者不得,虽胜有殃。是以必付与而□战,不得巳而后战。故抚时而战,不复使其众。无方而战者小胜以付磿者也。孙子曰:十战而六胜,以星也。十战而七胜,以日者也。十战而八胜,以月者也。十战而九胜,月有……[十战]而十胜,将善而生过者也(3)。一单……
* * *
……所不胜者也五,五者有所壹,不胜。故战之道,有多杀人而不得(4)将卒者,有得将卒而不得舍者,有得舍而不得将军者,有覆军杀将者。故得其道,则虽欲生不可得也。
八十
(1) 此是篇题,写在本篇第一简简背。
(2) 犹言介于天地之间。
(3) 过,疑借为祸。古代军事家多认为屡次打胜仗并不一定是好事,如《吴子·图国》说:“天下战国,五胜者祸,四胜者弊,三胜者霸,二胜者王,一胜者帝。”
(4) 得,疑是俘获之意。

八阵
本篇前一段说明所谓“王者之将”应具备的条件。后一段论述用“八阵”作战,要根据敌情和地形确定战法,配备兵力。
—— 银雀山汉墓竹简整理小组
八阵(1)
孙子曰:智不足,将兵,自恃也。勇不足,将兵,自广也。不知道,数战不足,将兵,幸也。夫安万乘国(2),广万乘王,全万乘之民命者,唯知道。知道者,上知天之道,下知地之理,内得其民之心,外知敌之情,阵则知八阵之经,见胜而战,弗见而诤(3),此王者之将也。
孙子曰:用八阵战者,因地之利,用八阵之宜。用阵三分,诲阵有锋,诲锋有后(4),皆待令而动。斗一,守二(5)。以一侵敌,以二收。敌弱以(6)乱,先其选卒以乘之(7)。敌强以治(8),先其下卒(9)以诱之。车骑与(10)战者,分以为三,一在于右,一在于左,一在于后。易(11)则多其车,险则多其骑,厄(12)则多其弩。险易必知生地、死地,居生击死(13)。
二百一十四 八阵
(1) 此是篇题,写在本篇第一简简背。古人讲布阵之法多称“八阵”。“八阵”不是指八种不同的阵。
(2) 万乘国,指可以出兵车万乘的大国。
(3) 诤(zheng证),借为静。意谓没有取胜的把握就按兵不动。
(4) 诲,疑借为每。锋,先锋部队。后,后续部队。
(5) 意谓以三分之一的兵力与敌交战,以三分之二的兵力等待时机。
(6) 以,犹言“而”。下文“敌强以治”同。
(7) 乘,凌犯。意谓先以精兵攻击敌人。
(8) 治,严整。意谓敌人战斗力强,阵容严整。
(9) 下卒,战斗力弱的士卒。
(10) 与,参与。
(11) 易,地形平坦。
(12) 厄(e饿),指两边高峻的狭窄的地形。
(13) 生、死,指生地、死地。

地葆
本篇从军事上论述各种地形的优劣。篇题原写在篇末。
—— 银雀山汉墓竹简整理小组
孙子曰:凡地之道,阳为表,阴为里(1),直者为纲,术(2)者为纪。纪纲则得,阵乃不惑。直者毛产(3),术者半死。凡战地也,日其精也,八风(4)将来,必勿忘也。绝水(5)、迎陵(6)、逆流(7)、居杀地(8)、迎众树(9)者,钧举也,五者皆不胜。南阵之山,生山也。东阵之山,死山也。东注之水,生水也。北注之水,死水。不流,死水也。五地之胜(10)曰:山胜陵,陵胜阜,阜胜陈丘,陈丘胜林平地。五草之胜曰:藩、棘、椐、茅、莎。五壤之胜:青胜黄,黄胜黑,黑胜赤,赤胜白,白胜青。五地之败(11)曰:谿、川、泽、斥。五地之杀(12)曰:天井、天宛、天离、天隙、天柖(13)。五墓(14),杀地也,勿居也,勿□也。春毋降,秋毋登。军与阵皆毋政前右,右周毋左周(15)。
地葆 二百
(1) 阳,疑指高亢明敞的地形。阴,疑指低洼幽暗的地形。
(2) 术,疑借为屈。
(3) 毛和产都有生长的意思,“毛产”与下文“半死”相对。
(4) 八风,八方之风。古人认为风的方向、大小、疾徐都与战争胜负相关。
(5) 绝水,渡水。
(6) 迎陵,面向高陵。
(7) 逆流,军阵处于河流下游。
(8) 杀地,极不利的地形。
(9) 迎众树,面向树林。
(10) 五种地形的优劣。
(11) 五地之败,五种败地。此下简文仅列举四地,疑漏抄一字。
(12) 五地之杀,五种杀地。
(13) 《孙子·行军》言险地种类有天井、天牢、天罗、天隙、天陷五类。天井,指四边高中间低洼之地。天离即《孙子》天罗(离、罗二宇古代音近通用),指草木茂密如罗网之地。天隙,指出道少而狭的地形。天宛,疑与《孙子》天牢相当。《孙子》“天陷”,银雀山竹简本《孙子兵法》作“天{尧召}”,本篇“天柖”当为“天{尧召}”的异文。
(14) 五墓,疑即指天井、天宛等五种杀地。
(15) 周,周匝环绕。左周、右周,疑指山陵高地在军阵的左侧或右侧。古兵书多认为军阵右背山陵为有利。

势备
本篇以剑、弓弩、舟车、长兵为比喻,说明阵、势、变、权四者在军事上的重要作用。
—— 银雀山汉墓竹简整理小组
势备(1)
孙子曰:夫陷齿戴角,前爪后距(2),喜而合,怒而斗,天之道也,不可止也。故无天兵者(3)自为备,圣人之事也。黄帝作(4)剑,以阵象(5)之。羿(6)作弓弩,以势象之。禹(7)作舟车,以变象之。汤、武(8)作长兵,以权象之。凡此四者,兵之用也。何以知剑之为阵也?旦暮服(9)之,未必用也。故曰,阵而不战,剑之为阵也。剑无锋,虽孟贲(10)[之勇]不敢□□□。阵无锋,非孟贲之勇也敢将而进者,不知兵之至也。剑无首铤(11),虽巧士不能进□□。阵无后,非巧士敢将而进者,不知兵之情者。故有锋有后,相信不动,敌人必走(12)。无锋无后,……劵不道。何以知弓弩之为势也?发于肩膺之间,杀人百步之外,不识其所道至(13)。故曰,弓弩也。何以[知舟车]之为变也?高则……何以知长兵之[为](14)权也?击非高下非……卢毁肩,故曰,长兵权也。凡此四……中之近……也,视之近,中之远。权者,昼多旗,夜多鼓,所以送战也。凡此四者,兵之用也。□皆以为用,而莫彻(15)其道。……功。凡兵之道四:曰阵,曰势,曰变,曰权。察此四者,所以破强敌,取猛将也(16)。
* * *
……之有锋者,选阵□也。爵……
……得四者生,失四者死……
(1) 此是篇题,写在本篇第一简简背。
(2) 陷,借为含。“含齿戴角、前爪后距”,指有牙、角、爪、距的禽兽。
(3) 天兵,指自然赋予动物的武器,如齿、角、爪、距等。无天兵者,指人。
(4) 作,创造,发明。
(5) 象,象征。
(6) 羿(yi意),后羿,夏代有穷国的君主。
(7) 禹,夏朝的建立者。
(8) 汤、武,指商汤和周武王。
(9) 服,佩带。
(10) 孟贲(ben奔),古代著名的勇士。
(11) 首铤(ting挺),剑的把柄。
(12) 走,败走。
(13) 道,由。意谓不知从何而来。
(14) “为”字原简写脱,据文义补。
(15) 彻,通达,明白。
(16) 自“……功。凡兵之道四”至此为一残简。这一简的位置也有可能在上文“凡此四……”与……中之近”之间。

[兵情]
此篇题为编者所加。本篇以矢、弩、发者分别比喻士卒、将帅和君主,认为只有三方面都合乎要求,才能胜敌。此篇字体与《势备》篇相同,文章思路也近似,有可能就是《势备》篇的后半。
—— 银雀山汉墓竹简整理小组
孙子曰:若欲知兵之情,弩矢其法也。矢,卒也。弩,将也。发者,主也(1)。矢,金在前,羽在后(2),故犀而善走(3)。前……今治卒则后重而前轻,阵之则辨,趣之敌则不听(4),人治卒不法矢也。弩者,将也。弩张柄(5)不正,偏强偏弱而不和,其两洋之送矢也不壹(6),矢虽轻重得,前后适,犹不中[招也]……将之用心不和……得,犹不胜敌也。矢轻重得,前[后]适,而弩张正,其送矢壹,发者非也,犹不中招也(7)。卒轻重得,前……兵……犹不胜敌也(8)。故曰,弩之中彀(9)合于四,兵有功……将也,卒也,□也。故曰,兵胜敌也,不异于弩之中招也。此兵之道也。
* * *
……所循以成道也。知其道者,兵有功,主有名。
(1) 发者,指发射的人。主,君主。
(2) 金,箭簇。羽,箭羽。
(3) 犀,犀利。走,疾行。
(4) 辨,同办。以上两句意谓使之列阵,虽能办到,但使其进攻敌人,则不听命。
(5) 柄,指弩臂。
(6) 洋,疑借为翔。两翔,两翼。此句意谓由于弩臂不正,弩弓两翼发矢的力量就不一致。
(7) 招,箭靶。犹,仍然。这几句的意思是说:弩和箭都合标准,但发射的人有错误,仍不能射中箭靶。
(8) 本句残缺,大意似谓将与卒都合标准,君主不能善用,也不能胜敌。
(9) 彀(gou够),箭靶。

行篡
本篇论述如何使士卒和百姓在战争中为统治者尽力。篇中有“篡贤取良”语,故以“行篡”为篇名。
—— 银雀山汉墓竹简整理小组
行篡(1)
孙子曰:用兵移民之道,权衡也(2)。权衡,所以篡贤取良也。阴阳,所以聚众合敌也(3)。正衡再纍……既忠,是谓不穷。称乡县衡(4),虽(5)其宜也。私公之财壹也。夫民有不足于寿而有余于货者(6),有不足于货而有余于寿者(7),唯明王、圣人知之,故能留之。死者不毒(8),夺者不愠(9)。此无穷……民皆尽力,近者弗(10)则远者无能。货多则辨(11),辨则民不德其上。货少则□,□则天下以为尊。然则为民赇也,吾所以为赇也(12),此兵之久也。用兵之……
(1) 此是篇题,写在本篇第一简简背。篡借为选。下文“篡贤取良”即“选贤取良”。
(2) 此句意谓:用兵和使民,如同用天平称东西一样。
(3) 聚众,集结兵力。合敌,同敌人交战。
(4) 称,举。乡,同向。县,同悬。衡,天平。称向,定方向。悬衡,衡量轻重利弊。
(5) 虽,疑借为唯。
(6) 指富有而贪生的人。
(7) 指因贫困而轻生的人。
(8) 毒,痛恨。
(9) 愠(yun韵),抱怨。
(10) 弗字下疑脱漏一宇。
(11) 辨,疑借为便,安逸。
(12) 赇,此处疑指积聚财富。

杀士
“杀士”意谓善于用兵的将帅能使士卒为之效死。《尉缭子·兵令下》说:“古之善用兵者,能杀卒之半,其次杀其十三,其下杀其十一”,与本篇主旨相近。
—— 银雀山汉墓竹简整理小组
杀士(1)
孙子曰:明爵禄而……
* * *
……杀士则士……
……知之。知士可信,毋令人离之。必胜乃战,毋令人知之。当战毋忘旁毋……
……必审而行之,士死……
(1) 此是篇题,写在本篇第一简简背。

延气
本篇列举激气、利气、厉气、断气、延气五事,反复强调激励士气、鼓舞斗志的重要性。
—— 银雀山汉墓竹简整理小组
延气(1)
孙子曰:合军聚众,[务在激气](2)。复徒(3)合军,务在治兵利气(4)。临境近敌,务在厉气(5)。战日有期,务在断气(6)。今日将饯,务在延气(7)。……以威三军之士,所以激气也。将军令……其令,所以利气也。将军乃……短衣絜裘(8),以劝(9)士志,所以厉气也。将军令,令军人人为三日粮,国人家为……[所以]断气也。将军召将卫人者而告之曰:“饮食毋……[所]以延气……也。
延气
* * *
……营也。以易营之,众而贵武,敌必败。气不利则拙,拙则不及,不及则失利,失利……
……气不厉则慑,慑则众□,众……
……而弗救,身死家残。将军召使而勉之,击……
(1) 此是篇题,写在本篇第一简简背。
(2) 激气,激发士气。
(3) 徙,拔营。复徙,疑指进发。
(4) 治兵,整治士卒。利气,使士中有锐气。
(5) 厉气,即励气,意调鼓励士卒的斗志。
(6) 断气,使士卒果断,有决心。
(7) 延气,疑指使士卒有持续作战的精神准备。
(8) 絜(jie结),疑借为褐(he合)。褐裘,疑即裘褐,粗衣。
(9) 劝,勉励。

官一
本篇篇首有“立官则以身宜”一语,放以“官”字为篇题。篇后所附残简文字均与本篇重复,可见此篇原有两本。“宫一”之“一”,疑指两本中的第一种本子。篇后所附残简疑为“官二”,因残缺过甚,姑附于《官一》篇后。由于本篇文字艰涩费解,简文次序不易确定。释文中一部份简文前加*号,表示这些简目前的排列顺序不一定正确。本篇内容主要论述各种军事措施及阵法的作用或适用的场合。
—— 银雀山汉墓竹简整理小组
官一(1)
孙子曰:凡处卒利阵体甲兵者(2),立官则以身宜,贱令以采章(3),乘削以伦物,序行以□□,制卒以州闾,授正以乡曲(4),辨疑以旌舆,申令以金鼓(5),齐兵以从迹,庵结以人 * 雄,邋军以索阵(6),茭肄以囚逆,陈师以危□,射战以云阵,御裹以羸渭(7),取喙以阖 * 燧,即败以包□,奔救以皮傅,燥战以错行。用□以正□,用轻以正散,攻兼用行城,* □地□□用方,迎陵而阵用刲,险□□□用圜,交易武退用兵,□□阵临用方 * 翼,泛战接厝用喙逢,囚险解谷以□远,草驵沙荼以阳削,战胜而阵以奋国,而…… * 为畏以山胠(8),秦怫以逶迤,便罢以雁行,险厄以杂管,还退以蓬错,绕山林以曲次,袭国邑以水则,辩夜退以明简,夜警以传节(9),厝入内寇以棺士,遇短兵以必舆,火输积以车,阵刃以锥行,阵少卒以合杂。合杂,所以御裹也。脩行连削,所以结阵也。云折重杂,所权趮也。猋凡振陈(10),所以乘疑也。隐匿谋诈,所以钓战也(11)。龙隋陈伏,所以山斗也。□□乖举,所以厌津也。□□□卒,所以□□也。不意侍卒,所以昧战也。遏沟□陈,所以合少也。疏削明旗,所以疑敌也。剽阵{车差}车,所以从遗也。椎下移师,所以备强也。浮沮而翼,所以燧斗也。禅{衤舌}括{艹瀪}避,所以莠{聂木}也。简练剽便(12),所以逆喙也。坚阵敦□,所以攻槥也。揆断藩薄,所以眩疑也。伪遗小亡,所以聭敌也(13)。重害,所以茭□也。顺明到声,所以夜军也。佰奉离积,所以利胜也。刚者,所以御劫也。更者,所以过□也。□者,所以御□也。……者,所以厌□也。胡退□入,所以解困也。
* * *
……令以金……
……云阵,御裹……
……胠,秦怫以逶迤,便罢……
……夜退以明简,夜警……
……舆,火输积以车,阵……
……龙隋阵……
……也。简练□便,所以逆……
……断藩薄,所以眩……
……所以聭敌也。重害,所……
……奉离积,所以利……
(1) 此是篇题,写在本篇第一简简背。
(2) 处卒,疑指选择有利地形驻军。利阵,疑谓使其阵坚利。体甲兵,疑指统帅军队。
(3) 贱,疑借为践,实行。采章,指彩色的旗帜、车服等物。
(4) 州闾,州里。州里、乡曲,古代地方基层行政单位。正,长。以上两句意谓按地方行政组织编制士卒,任命官长。
(5) 舆,疑借为旟(yu余),古代绘有鸟纹的旗。金,指金属军乐器。以上两句意谓军中以旗帜、金鼓指挥行动,士卒不会有疑虑。
(6) 索阵,与下文之囚逆、云陈、羸(lei雷)渭、皮傅、错行等,疑皆阵名。
(7) 御,抵御。裹,包围。
(8) 山胠(qu驱),与下文之逶迤、杂管、篷错、曲次等,疑皆阵名。
(9) 传(zhuan撰),符信。节,符节。意谓夜间巡逻以传节为凭证。
(10) 猋凡振陈,疑当读为飙风振尘。
(11) 钓战,引诱敌人出战。
(12) 简练,训练选拔。剽(piao漂)便,指骁勇敏捷的士卒。
(13) 聭,西汉前期文字多用作“恥”,此处疑借为“饵”。意谓故意丢失一些财物引诱敌军。

[强兵]

此篇题为编者所加。本篇从内容、文例和字体看,不象是孙膑兵法的本文,但它记述了齐威王与孙膑之间关于富国强兵的问答,内容重要,故列为上编之末。估计本篇可能是后人抄附在《孙膑兵法》书后的。
—— 银雀山汉墓竹简整理小组
威王问孙子曰:“……齐士教寡人强兵者,皆不同道。……[有]教寡人以政教者,有教寡人以……[有教]寡人以散粮者,有教寡人以静者,……之教□□行之教奚……[孙子曰]:“……皆非强兵之急者也。”威[王]……孙子曰:“富国。”威王日:“富国……厚,威王、宣王以胜诸侯(1),至于……
* * *
……将胜之,此齐之所以大败燕(2)……
……众乃知之,此齐之所以大败楚人(3)反……
……大败赵(4)……
……人于齧桑而擒氾皋也(5)。
……擒唐□也(6)。
……擒□瞏……
(1) 《史记·孟子荀卿列传》:“齐威王、宣王用孙子(膑)、田忌之徒,而诸侯东面朝齐”,可参考。
(2) 齐败燕,当指公元前三一四年齐宣王伐燕事。
(3) 齐败楚,疑指齐与韩、魏等国伐楚取重丘之战(参看注(6))。事在公元前三○一年齐湣(min民)王初立时。
(4) 据《竹书纪年》,魏惠王后元十年(齐威王三十二年,公元前三二五年)齐败赵于平邑,俘赵将韩举。
(5) “人”上一字尚余残画,似是“宋”字。据史书记载,齐湣王十五年宋为齐所灭。此处所记可能是灭宋以前的某次战役。齧(nie镍)桑,今江苏沛县。
(6) “唐□”疑即唐昧。《史记·楚世家》记怀王二十八年(公元前三○一年)“齐、韩、魏共攻楚,杀楚将唐昧,取我重丘而去。”唐昧,他书或作唐蔑。如果“唐□”确系唐昧,则此简与上文“大败楚人”一简所记当为一事。

十阵
本篇论述十种阵法的特点和作用。
—— 银雀山汉墓竹简整理小组
十阵(1)
凡阵有十:有方阵,有圆阵,有疏阵(2),有数阵(3),有锥行之阵(4),有雁行之阵(5),有钩行之阵(6),有玄襄之阵(7),有火阵,有水阵。此皆有所利。方阵者,所以剸(8)也。圆阵者,所以槫(9)也。疏阵者,所以{口犬}也。数阵者,为不可掇(10)。锥行之阵者,所以决绝(11)也。雁行之阵者,所以接射(12)也。钩行之阵者,所以变质易虑也(13)。玄{羽襄}之阵(14)者,所以疑众难故也。火阵者,所以拔也。水阵者,所以伥固也。
方阵之法,必薄中厚方(15),居阵在后。中之薄也,将以{口犬}也。重□其□,将以剸也。居阵在后,所以……
[圆阵之法]……(16)
[疏阵之法],其甲寡而人之少也,是故坚之。武者在旌旗,是人者在兵(17)。故必疏钜间(18),多其旌旗羽旄,砥刃以为旁。疏而不可蹙(19),数而不可军(20)者,在于慎。车毋驰,徒人毋趋(21)。凡疏阵之法,在为数丑(22),或进或退,或击或{豕页}(23),或与之佂,或要其衰(24)。然则疏可以取锐矣(25)。
数阵之法,毋疏钜间,戚而行首积刃而信之,前后相保,变□□□,甲恐则坐(26),以声坐□,往者弗送,来者弗止,或击其迂,或辱其锐(27),笲之而无间,{车反}山而退。然则数不可掇也。
锥行之阵,卑(28)之若剑,末不锐则不入(29),刃不薄则不剸,本(30)不厚则不可以列阵。是故末必锐,刃必薄,本必鸿(31)。然则锥行之阵可以决绝矣。
[雁行之阵],……中,此谓雁阵之任(32)。前列著{有雍}(33),后列若貍(34),三……阙罗而自存,此之谓雁阵之任。
钩行之阵,前列必方,左右之和(35)必钩。三声(36)既全,五彩(37)必具,辨吾号声(38),知五旗。无前无后,无……
玄{羽襄}之阵,必多旌旗羽旄,鼓{羽非}{羽非}庄,甲乱则坐,车乱则行,已治者□,榼榼啐啐(39),若从天下,若从地出,徒来面不屈(40),终日不拙。此之谓玄{羽襄}之阵。
火战之法(41),沟垒已成,重为沟堑,五步积薪,必均疏数,从役有数,令之为属枇,必轻必利,风辟……火既自覆,与之战弗克,坐行而北。火战之法,下面衍以{艹外},三军之士无所出泄(42)。若此,则可火也。陵猋蒋{艹外},薪荛(43)既积,营窟未谨(44)。如此者,可火也。以火乱之,以矢雨之,鼓譟敦兵(45),以势助之。火战之法。
水战之法,必众其徒而寡其车,令之为钩楷蓯柤贰辑□绛皆具。进则必遂,退则不蹙,方蹙从流,以敌之人为招(46)。水战之法,便舟以为旗,驰舟以为使,敌往则遂,敌来则蹙,推攘因慎而饬之,移而革之,阵而□(47)之,规(48)而离之。故兵有误车有御徒,必察其众少,击舟{豕页}(49),示民徒来。水战之法也。
七百八十七
(1) 此是篇题,写在本篇第一简简背。
(2) 疏,稀疏。
(3) 数,密集。
(4) 锥行之阵,前尖如锥的阵形。
(5) 雁行之阵,横列展开的阵形。
(6) 钩行之阵,左右翼弯曲如钩的阵形。
(7) 玄襄之陈,掘后文所述当是一种疑阵。
(8) 剸(zhuan专),截断。
(9) 槫(tuan团),借为团,结聚。
(10) 掇(duo多),疑借为剟(duo多),割取。
(11) 决绝,突破而切断之。
(12) 接射,疑指用弓矢交战。
(13) 虑,计谋,图谋,指作战的方针、计划。此句之意疑谓钩行之阵宜在改变作战计划时使用。
(14) 玄{羽襄}之阵,即玄襄之阵。
(15) 方,疑借为旁。薄中厚旁,意谓方阵中心人少,周围人多。
(16) 据上文,此处当有论圆阵的简文。“圆阵之法”四字据本篇文例增补。
(17) 是,疑借为示。以上二句意谓用旌旗和兵器以显示威武。
(18) 钜,借为距。疏距间,加大阵列的间隔距离。
(19) 蹙(cu促),迫促。
(20) 军,包围。
(21) 徒人,步卒。趋,疾走。
(22) 丑,类,群。数丑,几个小群,指几个小型的战斗单位。
(23) {豕页},意义不详。银雀山所出其他竹简中或用作刚毅之毅,疑即《说文》毅字异体。
(24) 要,通“邀”。参看《陈忌问垒》注(10)。
(25) 意谓疏阵可以用来袭取敌人的精锐部队。
(26) 坐,指军阵稳定不动。
(27) 辱,借为衂(nu女去声),挫折。
(28) 卑,借为譬。
(29) 未,指剑端。不入,不能突破。
(30) 本,指剑身。
(31) 鸿,大。
(32) 任,作用。
(33) {有雍},疑借为{豸雍}(yong雍),兽类,形似猿。
(34) 貍,野猫。
(35) 左右之和,指军阵的左右两翼。
(36) 三声,指军中金鼓笳铎的声音。
(37) 五彩,指各种颜色的军旗。
(38) 号声,号令之声。
(39) 榼榼(ke磕)啐啐(zu卒),疑指士卒鼓譟之声。
(40) 徒,步兵。屈,穷尽。“徒来”之语见《孙子·行军》:“尘高而锐者,车来也。卑而广者,徒来也。”
(41) 此节文字分前后两段,自此以下至“坐行而北”为一段,说明防御火攻的方法。“火战之法,下而衍以{艹外}”以下为另一段,说明火攻敌军的方法。下文“水战之法”也分两段,前一段似说明防御敌人自水上进攻之法,后一段似说明自水上进攻敌人之法。
(42) 无所出泄,无处逃脱。
(43) 薪荛(rao饶),柴草。
(44) 营地整治不周密。
(45) 敦,劝勉。意谓鸣鼓喧譟,以激励士卒的斗志。
(46) 招,箭靶。
(47) 此字有残损,可能是“歹”宇,也可能是“支”宇或“丈”宇。
(48) 规,疑借为窥。
(49) 津,渡口。

十问
本篇用问答形式就敌我双方力量对比的各种不同情况,提出不同的击敌方法。本篇共十组问答,除第一组和第十组的位置可以肯定以外,其它各组的前后顺序是整理者编次的。
—— 银雀山汉墓竹简整理小组
十问(1)
兵问曰:交和而舍(2),粮食均足,人兵敌衡(3),客主(4)两惧。敌人圆阵以胥(5),因以为固,击[之奈何?曰]:击此者,三军之众分而为四五,或傅(6)而佯北,而示之惧。彼见我惧,则遂分而不顾。因以乱毁其固。驷鼓同举,五遂(7)俱傅。五遂俱至,三军同利。此击圆之道也。
交和而舍,敌富我贫,敌众我少,敌强我弱,其来有方,击之奈何?日:击此者,□阵而□(8)之,规而离之,合而佯北,杀将其后,匆令知之。此击方之道也。
交和而舍,敌人既众以强,劲捷以刚,锐阵以胥,击之奈何?击此者,必三而离之,一者延而衡(9),二者□□□□□恐而下惑,下上既乱,三军大北。此击锐之道也。
交和而舍,敌既众以强,延阵以衡,我阵而待之,人少不能,击之奈何?击此者,必将三分我兵,练我死士,二者延阵张翼,一者材士练兵(10),期其中极(11)。此杀将击衡之道也。
交和而舍,我人兵则众,车骑则少,敌人十倍,击之奈何?击此者,当保险带隘(12),慎避广易(13)。故易则利车,险则利徒。此击车之道也。
交和而舍,我车骑则众,人兵则少,敌人十倍,击之奈何?击此者,慎避险阻,决而导之,抵诸易(14)。敌虽十倍,便我车骑,三军可击。此击徒人(15)之道也。
交和而舍,粮食不属(16),人兵不足俧(17),绝根而攻,敌人十倍,击之奈何?曰:击此者,敌人既□而守阻,我……反而害其虚。此击争□之道也。
交和而舍,敌将勇而难惧,兵强人众自固,三军之士皆勇而无虑,其将则威,其兵则武,而理强梁偼(18),诸侯莫之或待(19)。击之奈何?曰:击此者,告之不敢,示之不能,坐拙而待之,以骄其意,以惰其志,使敌弗识,因击其不□,攻其不御,压其骀(20),攻其疑。彼既贵既武,三军徙舍,前后不相睹,故中而击之,若有徒与。此击强众之道也。
交和而舍,敌人保山而带阻,我远则不接,近则无所(21),击之奈何?击此者,彼敛阻移□□□□□则危之,攻其所必救(22),使离其固,以揆其虑(23),施伏设援,击其移庶(24)。此击保固之道也。
交和而舍,客主两阵,敌人形箕(25),计敌所愿,欲我陷覆,击之奈何?击此者,渴者不饮,饥者不食,三分用其二,期于中极,彼既□□,村士练兵,击其两翼,□彼□喜□□三军大北。此击箕之道也。
七百一十九
(1) 此是篇题,写在本篇第一简简背。
(2) 和,军队左右垒门。舍,扎营。意谓两军相对,准备交战。
(3) 敌,相当。意谓双方人力和武器相当。
(4) 客指进攻的一方,主指守御的一方。
(5) 胥,等待。
(6) 傅,借为薄,迫近,接触。
(7) 遂,借为队。
(8) 参看《十阵》注(47)
(9) 延而衡,与下文“延阵以衡”同意,指把军阵延长,横着摆开。
(10) 材士,材力之士。练兵,精选的士卒。
(11) 中极,要害。意谓务期攻敌要害。下文“期于中极”与此同意。
(12) 意谓凭据险阻隘塞之地,恃以为固。
(13) 意谓要避开平敞开阔的地形。
(14) 抵,挤,推。意谓把敌人压迫到平坦的地带。
(15) 徒人,步卒。
(16) 属,连续。意谓粮食接济不上。
(17) 俧,疑借为恃。
(18) 理强梁偼,疑当读为“吏强粮接”,吏指军吏。
(19) 待,抵御。意谓其它诸侯国都不能抵御。
(20) 骀(tai台),疑借为怠。
(21) 以上二句意谓我离敌太远则打不到敌人,离敌过近则无立足之地。
(22) “攻其所必救”之语见《孙子,虚实》。
(23) 揆(kui葵),揣度。意谓揣度敌人的行动意图。
(24) 庶,众。移庶,移动中的敌众。
(25) 意谓敌人把军队布置成簸箕形的阵势。

略甲
本篇简文残缺,无法看出主要内容。除首简以外,各简次序不能确定,释文一律提行,不连写。

本篇字体与《十阵》、《十问》相近,不易区分,现将可能属于这三篇的残简一并附于本篇之后。
—— 银雀山汉墓竹简整理小组
略甲(1)
略甲之法,敌之人方阵□□无……
……欲击之,其势不可,夫若此者,下之……
……以国章,欲战若狂,夫若此者,少阵……
……反,夫若此者,以众卒从之,篡(2)中因之,必将……
……篡卒因之,必……
* * *
……左右旁伐以相趋,此谓{钅畟}钩击。
……之气不藏于心,三军之众□循之知不……
……将分□军以脩□□□□寡而民……
……威□□其难将之□也。分其众,乱其……
……阵不厉,故列不……
……远揄之,敌倦以远……
……治,孤其将,荡其心,击……
……其将勇,其卒众……
……彼大众将之……
……卒之道……
(1) 此是篇题,写在本篇第一简简背。
(2) 篡,借为选。

客主人分
本篇指出,作战时人众、粮多、武器精良都不足恃,只有掌握战争规律,明了敌我双方情况,善于利用有利形势和良好地形,才是取得胜利的保证。
—— 银雀山汉墓竹简整理小组
客主人分(1)
兵有客之分,有主人之分。客之分众,主人之分少。客倍主人半,然可敌也(2)。负……定者也(3)。客者,后定者也,主人安地抚势以胥(4)。夫客犯隘逾险而至,夫犯隘……退敢刎颈,进不敢拒敌,其故何也?势不便,地不利也。势便地利则民自……自退。所谓善战者,便势利地者也。带甲数十万,民有余粮弗得食也,有余……居兵多而用兵少也,居者有余而用者不足。带甲数十万,千千而出,千千而□之……万万以遗我。所谓善战者,善翦断之,如□会捝者也。能分人之兵,能按人之兵,则锱[铢]而有余(5)。不能分人之兵,不能按人之兵,则数倍而不足。众者胜乎?则投算而战耳(6)。富者胜乎?则量粟而战耳(7)。兵利甲坚者胜乎?则胜易知矣(8)。故富未居安也,贫未居危也;众未居胜也,少[未居败也]。以决胜败安危者,道也。敌人众,能使之分离而不相救也,受敌者不得相……以为固(9),甲坚兵利不得以为强,士有勇力不得以卫其将,则胜有道矣。故明主、知道之将必先□,可有功于未战之前,故不失;可有之(10)功于已战之后,故兵出而有功,入而不伤,则明于兵者也。
五百一十四
* * *
……焉。为人客则先人作……
……兵曰:主人逆客于境……
……客好事则……
……使劳,三军之士可使毕失其志,则胜可得而据也。是以按左抶右(11),右败而左弗能救;按右抶左,左败而右弗能救。是以兵坐而不起,避而不用,近者少而不足用,远者疏而不能……
(1) 此是篇题,写在本篇第一简简背。客,指战争中攻入他人境内的一方。主人,指在自己土地上防守的一方。分,份量,比例。
(2) 敌,匹敌。意谓主人兵力只有客方的一半,然而可以与之匹敌。《汉书·陈汤传》:“又兵法曰:客倍而主人半,然后敌。”
(3) 此句残缺,原文疑当作:“主人者,先定者也。”先定,指先作好部署。
(4) 意谓凭据良好地形,利用有利形势,严阵以待。
(5) 《淮南子·兵略》:“故能分人之兵,疑人之心,则锱(zi资)铢有余;不能分人之兵,疑人之心,则数倍不足。”简文“锱”字残存“金”旁,“铢”字全缺,今据《淮南子》补。锱、铢都是古代两以下的重量单位,比喻份量极小。
(6) 算,古代计数用的算筹。意谓如果人多既能取得胜利,那只要数数算筹就可以决定胜负了。
(7) 意谓如果财富雄厚就能取得胜利,那只要量一量粮食的多少就可以决定胜负了。
(8) 意谓如果武器装备精良就能取得胜利,那么胜负也就太容易知道了。
(9) “以为固”上约缺八字,据《善者》篇的类似文字,以上两句可补足为,“受敌者不得相[知,沟深垒高不得]以为固。”受敌,受攻击。
(10) “之”字疑是衍文。
(11) 抶(chi翅),击。按左抶右,意谓牵制敌人之左翼,而攻击其右翼。

善者
本篇指出善战者在作战时能使自己处于主动而陷敌于被动。
—— 银雀山汉墓竹简整理小组
善者(1)
善者,敌人军□人众,能使分离而不相救也,受敌(2)而不如知(3)也。故沟深(4)垒高不得以为固,车坚兵利不得以为威,士有勇力而不得以为强。故善者制险量阻(5),敦三军,利屈伸,敌人众能使寡,积粮盈军能使饥,安处不动能使劳,得天下能使离,三军和能使柴(6)。故兵有四路、五动:进,路也;退,路也;左,路也;右,路也。进,动也;退,动也;左,动也;右,动也;默然而处,亦动也。善者四路必彻(7),五动必工(8)。故进不可迎于前(9),退不可绝于后(10),左右不可陷于阻,默[然而处],□□于敌之人。故使敌四路必穷,五动必忧。进则傅(11)于前,退则绝于后,左右则陷于阻,默然而处,军不免于患。善者能使敌卷甲趋远(12),倍道兼行(13),倦病而不得息,饥渴而不得食。
以此薄敌,战必不胜矣(14)。我饱食而待其饥也,安处以待其劳也,正静以待其动也。故民见进而不见退,蹈白刃而不还踵(15)。
二百□□□
(1) 此是篇题,写在本篇第一简简背。善者,指善战者。
(2) 受敌,受攻击。
(3) 不相知,互不知情。
(4) 简文“沟深”二字只残存“水”旁,据文义释。
(5) 意谓善战者能审察地形,利用险阻。
(6) 柴,陋俗为訾(zi子),怨恨。
(7) 彻,通达。
(8) 工,巧,善。
(9) 意谓进军时敌人不能阻挡前进。
(10) 意谓退军时敌人不能切断退路。
(11) 傅,借为薄。薄,迫。
(12) 卷甲,卷起铠甲。趋远,向远方急进。
(13) 一天走两天的路。
(14) “战必不胜”是指敌方说的。
(15) 不还踵(zhong肿),犹言不旋踵。意谓冒锋刃而不后退。

五名五恭
《五名》、《五恭》原为两段,标题分别写在段末,今据文例、字体合为一篇。《五名》论述用不同方法对付五种不同的敌军。《五恭》论述军队进入敌方境内时,“恭”“暴”两种手段要交替使用。
—— 银雀山汉墓竹简整理小组
兵有五名:一日威强,二日轩骄(1),三曰刚至(2),四曰{目力}忌(3),五曰重柔(4)。夫威强之兵,则屈软而待之(5);轩骄之兵,则恭敬而久之;刚至之兵,则诱而取之;{目鸟}忌之兵,则薄其前,譟其旁,深沟高垒而难其粮;重柔之兵,则譟而恐之,振而捅之,出则击之,不出则回(6)之。
五名
兵有五恭、五暴。何谓五恭?入境而恭,军失其常。再举而恭,军无所粮(7)。三举而恭,军失其事(8)。四举而恭,军无食。五举而恭,军不及事。入境而暴,谓之客。再举而暴,谓之华。三举而暴,主人惧。四举而暴,卒士见诈(9)。五举而暴,兵必大耗。故五恭、五暴,必使相错也(10)。
五恭 二百五十六(11)
(1) 轩骄,疑是高傲或骄悍之意。
(2) 刚至,“至”疑借为{忄至}(zhi质)。刚俊,刚愎(bi必)自用。
(3) {目力}忌,下文作{目鸟}忌。{目鸟},{目力}皆从目声,疑当读为冒。冒,贪。忌,疑忌。
(4) 重柔,极其软弱。
(5) 意谓用示弱的办法对付强敌。
(6) 回,围。
(7) 军队征集不到粮草。
(8) 失其事,误事。
(9) 见诈,受骗。
(10) 相错,交替使用。
(11) 据篇末所记字数,本篇除《五名》、《五恭》外,还应有一段,但在整理过程中没有发现。

[兵失]

此篇题为编者所加。本篇分析了作战失利的各种因素,提出军队要行“起道”的主张。
—— 银雀山汉墓竹简整理小组
欲以敌国之民之所不安,正俗所……难敌国兵之所长,耗兵也。欲强多(1)国之所寡,以应敌国之所多,速屈(2)之兵也。备固,不能难敌之器用(3),陵兵(4)也。器用不利,敌之备固,挫兵也。兵不……明者也。善阵,知背向(5),知地形,而兵数困,不明于国胜、兵胜者也。民……兵不能昌大功,不知会(6)者也。兵失民,不知过者也。兵用力多功少,不知时者也。兵不能胜大患,不能合民心者也。兵多悔,信疑者也。兵不能见福祸于未形,不知备者也。兵见善而怠(7),时至而疑(8),去非而弗能居(9),止道也。贪而廉,龙而敬(10),弱而强,柔而[刚],起道也(11)。行止道者,天地弗能兴也。行起道者,天地……
* * *
……之兵也。欲以国……
……内疲之兵也。多费不固……
……见敌难服,兵尚淫天地……
……而兵强国……
……兵不能……
(1) 强多,勉强增加。
(2) 屈,竭尽。
(3) 意谓设防坚固,但抵挡不住敌人进攻的器械。
(4) 陵兵,被欺凌的军队。
(5) 背向,指行军布阵时的所向或所背。如《孙子·军争》:“故用兵之法,高陵勿向,背丘勿逆”,同书《行军》:“平陆处易而右背高”,《司马法·用众》:“凡战,背风背高,右高左险”之类。
(6) 金,时机。
(7) 见善而怠,见到有利条件而怠惰不前。
(8) 面临良好战机而犹豫不决。
(9) 抛弃错误,但又不能照正确的去做。
(10)《六韬·文韬·明传》有一段类似的话,“龙而敬”作“恭而敬”(参看注(10))。龙、恭二字古通用,但此处上下文为“贪而廉”“弱而强”,而字前后二字义正相反,恭、敬二字义重,疑有误。一说“龙”借为“宠”。
(11)《六韬·文韬·明传》:“见善而怠,时至而疑,知非而处。此三者,道之所止也。柔而静,恭而敬,强而弱,忍而刚。此四者,道之所起也。”文字与本篇相近。但本篇的“止道”“起道”,从下文“行止道”“行起道”二语来看,似是两种道的名称。疑“止道”指停滞、灭亡之道,“起道”指兴旺、胜利之道。

将义

本篇提出将帅必须具备义、仁等品质。从文中可以看出,要求义是为了立威严,使士卒效死;要求仁是为了克敌立功。可见此所谓义、仁与儒家所说的仁义不同。《孙子·计》说:“将者,智、信、仁、勇、严也”,可参考。
—— 银雀山汉墓竹简整理小组
义将(1)
将者不可以不义,不义则不严,不严则不威,不威则卒弗死(2)。故义者,兵之首也。将者不可以不仁,不仁则军不克,军不克则军无功。故仁者,兵之腹也。将者不可以无德,无德则无力,无力则三军之利不得。故德者,兵之手也。将者不可以不信,不信则令不行,令不行则军不槫,军不槫则无名(3)。故信者,兵之足也。将者不可以不智胜,不智胜(4)……则军无□。故决(5)者,兵之尾也。
将义
(1) 此是篇题,写在本篇第一简简背。篇末亦有篇题,作“将义”。从文义看,以作“将义”为是。
(2) 卒弗死,士卒不肯效死。
(3) 名,功绩。
(4) 简文胜字及其下重文号疑是抄书者多写的,原文当作:“不可以不智,不智……”。一说“不智胜”当读为“不知胜,不知胜即不智。
(5) 决,果断。

[将德]

此篇题为编者所加。篇中提出了不轻敌、赏罚及时等将帅应具备的品德。本篇除最未一简外,次序都不能确定,释文中各简都提行,不连写。
—— 银雀山汉墓竹简整理小组
……赤子,爱之若狡(1)童,敬之若严师,用之若土芥(2),将军……
……不失,将军之智也。不轻寡(3),不劫于敌(4),慎终若始(5),将军……
……而不御,君令不入军门,将军之恒也。入军……
……将不两生,军不两存,将军之……
……将军之惠也。赏不逾日,罚不还面(6),不维其人,不何……
……外辰,此将军之德也。
(1) 狡,年少而美好。
(2) 芥,草芥。土芥比喻轻微无价值的东西。此数句意谓将帅之于士卒,平时须爱护,敬重,该用的时候又要舍得用。
(3) 不因敌人数量少而轻视它。
(4) 劫,迫。意谓不为强大的敌人所吓倒。
(5) 《老子》六十四章:“慎终如始,则无败事”,可参考。
(6) 还面,转脸。

上一页 回目录 下一页

将败

本篇列举将帅品质上的种种缺点,这些缺点都会导致战争失败。
—— 银雀山汉墓竹简整理小组
将败(1)
将败:一曰不能而自能。二曰骄。三曰贪于位。四曰贪于财。[五曰]□。六曰轻。七曰迟。八曰寡勇。九曰勇而弱。十曰寡信。十一[曰]……十四曰寡决。十五曰缓。十六曰怠。十七曰□。十八曰贼(2)。十九曰自私。廿曰自乱。多败者多失。
(1) 此是篇题,单独写在一简上。
(2) 贼,残暴。

[将失]

此篇题为编者所加。篇中分析了造成将帅作战失利的种种情况。此篇内容与《将败》篇相关,文例、字体也相同,可能本为一篇。疑《将败》篇末“多败者多失”下即紧接本篇文字。
—— 银雀山汉墓竹简整理小组
将失:一曰,失所以往来(1),可败也。二曰,收乱民而还用之,止北卒而还斗之(2),无资而有资(3),可败也。三曰,是非争,谋事辩讼(4),可败也。四曰,令不行,众不壹,可败也。五曰,下不服,众不为用,可败也。六曰,民苦其师,可败也。七曰,师老(5),可败也。八曰,师怀(6),可败也。九曰,兵遁,可败也。十曰,兵□不□,可败也。十一曰,军数惊,可败也。十二曰,兵道足陷,众苦,可败也。十三曰,军事险固,众劳(7),可败也。十四[曰],□□□备,可败也。十五曰,日暮路远,众有至气(8),可败也。十六曰,……可败也。十七[曰],……众恐,可败也。十八曰,令数变,众偷(9),可败也。十九曰,军淮,众不能其将吏(10),可败也。廿曰,多幸(11),众怠,可败也。廿一曰,多疑,众疑,可败也。廿二曰,恶闻其过,可败也。廿三曰,与不能(12),可败也。廿四曰,暴露伤志(13),可败也。廿五曰,期战心分(14),可败也。廿六曰,恃人之伤气(15),可败也。廿七曰,事伤人,恃伏诈(16),可败也。廿八曰,军舆无□,[可败也。廿九曰],□下卒,众之心恶,可败也。卅曰,不能以成阵,出于夹道(17),可败也。卅一曰,兵之前行后行之兵,不参齐于阵前,可败也。卅二曰,战而忧前者后虚,忧后者前虚,忧左者右虚,忧右者左虚。战而有忧,可败也。
(1) 意谓军队行动茫无目的。
(2) 以上两句意谓收用乱民和败卒来打仗。
(3) 本无实力而自以为有实力。
(4) 以上两句的意思是说:在是非问题上总是争执;在谋划大事时,总是辩论争吵,不能作出决定。
(5) 士卒长期出征在外,不得休息。
(6) 士卒有所挂念。
(7) 以修筑军事要塞为事,使士卒劳苦。
(8) 至,疑借为{忄至},怨恨。
(9) 偷,苟且敷衍。
(10) 淮,疑借为乖,不和。众不能其将吏,意谓士卒与将吏的关系不好。
(11) 幸,偏爱。
(12) 与,亲近,交往。不能,无能之辈。一说与借为举,意谓举用无能之人。
(13) 士卒暴露于野外,伤其心志。
(14) 临战之前军心涣散。
(15) 恃,凭借。意谓所凭借的是敌人的斗志消沉。
(16) 做的是伤害人的事,靠的是阴谋诡诈的手段。
(17) 夹,疑借为狭。

[雄牝城]

此篇题为编者所加。篇中主要论述难攻的雄城和易攻的牧城在地形上的特点。
—— 银雀山汉墓竹简整理小组
城在渒泽(1)之中,无亢山名谷(2),而有付丘(3)于其四方者,雄城也,不可攻也。军食流水,[生水也,不可攻]也。城前名谷,背亢山,雄城也,不可攻也。城中高外下者,雄城也,不可攻也。城中有付丘者,雄城也,不可攻也。营军趣舍(4),毋回名水(5),伤气弱志(6),可击也。城背名谷,无亢山其左右,虚城也,可击也。□尽烧(7)者,死壤也,可击也。军食泛水(8)者,死水也,可击也。城在发泽(9)中,无名谷付丘者,牝城(10)也,可击也。城在亢山间,无名谷付丘者,牝城也,可击也。城前亢山,背名谷,前高后下者,牝城也,可击也。
(1) 渒泽,小泽。
(2) 亢,高。名,大。
(3) 付丘,疑即负丘,两层的丘。
(4) 营军,安营。趣舍,行军。
(5) 回,环绕。名水,指大江大河。
(6) 伤气,损伤士气。以上几句之意,疑谓行军安营不要绕着大河走,否则会沮丧士气。
(7) 烧,疑借为硗(qiao敲),坚硬贫瘠的土地。
(8) 泛水,积水,与流水相对。
(9) 发,疑借为沛。沛泽,大泽。
(10) 牝(pin聘),雌。牝城与雄城相对。

[五度九夺]

此篇题为编者所加。篇中指出作战时针对自己一方的不利条件应该避免什么,以及为了挫败敌军应当争夺什么。
—— 银雀山汉墓竹简整理小组
……矣。救者至,又重败之。故兵之大数(1),五十里不相救也。况近□□□□□数百里(2),此程(3)兵之极也。故兵(4)曰:积(5)弗如,勿与持久。众弗如,勿与接和(6)。□[弗如,勿与□□。□□弗如,勿]与□长。习(7)弗如,毋当其所长。五度(8)既明,兵乃横行。故兵……趋敌数。一曰取粮。二曰取水。三曰取津(9)。四曰取途。五曰取险。六曰取易。七曰[取□。八曰取□。九]曰取其所读(10)贵。凡九夺,所以趋敌也。
四百二字
(1) 大数,大要。
(2) 此句有缺文,据文义,原文似当为:“况近者数里,远者数百里”。
(3) 程,衡量。
(4) 兵,指古兵法。
(5) 积,委积,指粮草。
(6) 接和,与交和同意,两军对垒。
(7) 习,训练。
(8) 五度,指上文所说“积弗如,勿与持久”等五事。
(9) 津,渡口。
(10) 读,借为独。

[积疏]

此篇题为编者所加。本篇主要阐述积疏、盈虚、径行、疾徐、众寡、佚劳六对矛盾的相互关系。此篇字体与《五度九夺》篇相同,可能本为一篇。
—— 银雀山汉墓竹简整理小组
……[积]胜疏,盈胜虚,径胜行(1),疾胜徐,众胜寡,佚胜劳。积故积之(2),疏故疏之,盈故盈之,虚[故虚之,径故径]之,行故行之,疾故疾之,[徐故徐之,众故众]之,寡故寡之,佚故佚之,劳故劳之。积疏相为变(3),盈虚[相为变,径行相为]变,疾徐相为变,众寡相[为变,佚劳相]为变。毋以积当积(4),毋以疏当疏,毋以盈当盈,毋以虚当虚,毋以疾当疾,毋以徐当徐,毋以众当众,毋以寡当寡,毋以佚当佚,毋以劳当劳。积疏相当(5),盈虚相[当,径行相当,疾徐相当,众寡]相当,佚劳相当。敌积故可疏(6),盈故可虚,径故可行,疾[故可徐,众故可寡,佚故可劳]。……
(1) 径,小路,指捷径。行,大道。
(2) 集聚的就使它集聚。
(3) 集聚与分散互相变化。
(4) 不要用集聚对集聚。
(5) 集聚和分散相对。
(6) 犹言敌积故可疏之。

奇正
奇正是古代军事上常用的术语。奇和正相对。正指一般的、正常的,奇指特殊的、变化的。本篇阐述奇正的相互关系和变化,以及如何运用奇正的原则以克敌制胜。
—— 银雀山汉墓竹简整理小组
奇正(1)
天地之理,至则反,盈则败,□□(2)是也。代兴代废(3),四时是也。有胜有不胜,五行(4)是也。有生有死,万物是也。有能有不能,万生(5)是也。有所有余,有所不足,形势是也。故有形之徒,莫不可名(6)。有名之徒,莫不可胜(7)。故圣人以万物之胜胜万物(8),故其胜不屈(9)。战者,以形相胜者也。形莫不可以胜,而莫知其所以胜之形(10)。形胜之变,与天地相敝而不穷(11)。形胜,以楚越之竹书之而不足(12)。形者,皆以共胜胜者也(13)。以一形之胜胜万形,不可(14)。所以制形壹也,所以胜不可壹也(15)。故善战者,见敌之所长,则知其所短;见敌之所不足,则知其所有余。见胜如见日月。其错胜(16)也,如以水胜火。形以应形,正也;无形而制形,奇也(17)。奇正无穷,分也。分之以奇数(18),制之以五行,斗之以□□。分定则有形矣,形定则有名[矣]。……同不足以相胜也,故以异为奇。足以静为动奇,佚为劳奇,饱为饥奇,治为乱奇,众为寡奇。发而为正,其未发者奇也。奇发而不报,则胜矣。有余奇者,过胜者也。故一节痛,百节不用(19),同体也。前败而后不用,同形也。故战势,大阵□断,小阵□解。后不得乘前,前不得然(20)后。进者有道出,退者有道入。赏未行,罚未用,而民听令者,其令,民之所能行也。赏高罚下,而民不听其令者,其令,民之所不能行也。使民虽不利,进死而不旋踵,孟賁之所难也,而责之民,是使水逆流也。故战势,胜者益(21)之,败者代之,劳者息之,饥者食之。故民见□人而未见死,蹈白刃而不旋踵。故行水得其理,漂石折舟(22);用民得其性,则令行如流。 四百八十七
(1) 此是篇题,单独写在一简上。
(2) 此处所缺二字疑是“日月”或“阴阳”。
(3) 代,更替。
(4) 五行,指金、木、水、火、土。胜,指五行相克,如水胜火。
(5) 万生,各种生物。
(6) 有形体的事物,没有不可命名的。
(7) 有名称的事物,没有不可制服的。
(8) 以万物之胜胜万物,意谓用一物的特性克制另一物,以此驾驭万物。
(9) 屈,穷尽。
(10) 有形之物没有不可制服的,问题是不知道用什么去制服它。《孙子·虚实》说:“人皆知我所以胜之形,而莫知吾所以制胜之形”,可参考。
(11) 敝,尽。意谓万事万物相生相克的现象和天地共始终而无穷无尽。
(12) 楚和越都盛产竹。古人在竹简上写字。此句意谓万物相胜的现象是写不完的。
(13) 犹言皆以其胜相胜者也。
(14) 以一种事物去制胜万物,是不可能的。
(15) 以上两句的意思是说:用来制胜的原则是一样的,但用来制胜的事物是各种各样的。
(16) 错,同措,措置。错胜,犹言制胜。
(17) 以上两句意谓:用有形对付有形,是正;用无形制服有形,是奇。
(18) 《孙子·势》:“凡治众如治寡,分数是也”,梅尧臣注:“部伍奇正之分数,各有所统”,可参考。
(19) 节,骨节。意谓身上一处有病痛,全身就都不听使唤。
(20) 然,借为蹨(nian捻),践踏。
(21) 益,增。指增加兵力。
(22)《孙子·势》:“激水之疾,至于漂石者,势也。” 


\end{document}