% 海防要览
% 海防要览.tex

\documentclass[a4paper,12pt,UTF8,twoside]{ctexbook}

% 设置纸张信息。
\RequirePackage[a4paper]{geometry}
\geometry{
	%textwidth=138mm,
	%textheight=215mm,
	%left=27mm,
	%right=27mm,
	%top=25.4mm, 
	%bottom=25.4mm,
	%headheight=2.17cm,
	%headsep=4mm,
	%footskip=12mm,
	%heightrounded,
	inner=1in,
	outer=1.25in
}

% 设置字体,并解决显示难检字问题。
\xeCJKsetup{AutoFallBack=true}
\setCJKmainfont{SimSun}[BoldFont=SimHei, ItalicFont=KaiTi, FallBack=SimSun-ExtB]

% 目录 chapter 级别加点(.)。
\usepackage{titletoc}
\titlecontents{chapter}[0pt]{\vspace{3mm}\bf\addvspace{2pt}\filright}{\contentspush{\thecontentslabel\hspace{0.8em}}}{}{\titlerule*[8pt]{.}\contentspage}

% 设置 part 和 chapter 标题格式。
\ctexset{
	part/name={},
	part/number={},
	chapter/name={},
	chapter/number={}
}

% 注脚每页重新编号,避免编号过大。
\usepackage[perpage]{footmisc}

% 设置古文原文格式。
\newenvironment{yuanwen}{\bfseries\zihao{4}}

% 设置署名格式。
\newenvironment{shuming}{\hfill\bfseries\zihao{4}}

\title{\heiti\zihao{0} 海防要览}
\author{丁日昌}
\date{清}

\begin{document}

\maketitle
\tableofcontents

\frontmatter
\chapter{前言、序言}



\mainmatter

\part{海防要览卷上}

\chapter{海防条议}

\begin{yuanwen}

\end{yuanwen}
一练兵原奏称陆路之兵固宜益加训练外海水师尤当
益事精求各口岸固须设防然非有海洋重兵可迎勦可
截击可尾追彼即可随处登岸使我有防不胜防之苦等
语是所注意者在于要口设防不效从前零星散漫即兵
法所谓致人而不致于人之意查十余年来泰西凡三大

战一曰英法照士攻俄之战开衅之初英法即以重兵守
黑海口使俄不能出入其后俄卒求成于英法一曰花旗
南北之战开衅后北花旗即将所有兵船驶往南花旗各
海口全行堵塞俾不得乞援邻国购办战械南花旗卒致
歼灭一曰布法之战布人自闻法国动兵即将道国劲旅
先堵礼吴河口而法亦卒为布所困即如中外用武以来
兵非不多饷非不足然彼族不过数千人今曰扰可而粤
之全省疲于奔命矣明日扰闽而闽之全省疲于奔命矣
我则备多力分彼则择瑕而蹈是皆未练重兵屯扎

处处设防之弊故致此也外国之有战事也力与力相敌
则器精者胜器与器相等则先下辣手者胜故今日择要
练兵以备攻勦黾击之用尤不可须臾缓矣今以天下大
势言之法国占据安南之胥江及南天省既与我广西云
南贵州之边境毗连英国占据五印度既与我云南四川
之边境毗连成国染指新疆联络回部巳与甘肃陜西之
边境毗连其占据黑龙江以北者又且与我盛京等处边
境毗连至东南七省之逼近海泽为洋船所可朝发夕至
者又无论巳从古中外交涉急于陆者恒缓于水固未有

水陆交逼处处环伺如今曰之甚者也然以理与势揆之
凡外国陆地之与我毗连者不过得步进步志在蚕食而
不在鲸吞其水路之实逼处此者则动辄制我要害志在
鲸吞而不在蚕食故东北为最要东南与西北为次要西
南又次之此四要者若分缓急选练重兵水则首尾互应
陆则各自为战庶几渐息乎敌人觊觎之心或有可稍固
吾圉之一日也中国旗绿各营数非不多口粮太薄器械
太窳断难恃以制合年来虽有减兵增饷之议而饷仍薄
汛兵未裁终难化散为整彻底攺观臣在可苏时言将抚

标数营苞旨兵一律裁汰易以新勇𢸑去汛地改操洋𬬰洋
𪿫当时舆论颇碍撒泛之难经臣密奏以和议不可长恃
自强必须早计仰蒙圣恩特允照办迄今并未闻汛地
撤后稍有流弊若使各省均以勇易兵减额优饷分别练
为𪿫队𬬰队虽不增帑增费而十万劲兵固可星罗暮布
而其要则在于裁泛并营盖分泛则兵断不能练不练则
虽优饷减额而兵何自而精乎至于各省沿海水师但知
即精而所用乃艇船旧𪿫则仍以予敌也沿海渔人胥户

熟习风涛之险者其根祗较内地之兵为能耐劳次则挑
选水师之得力者易其船械勤其操演教以测量规算渐
练渐熟使其常以水为家而且当令沿海全洋统筹兼顾
不可稍分畛域何则风涛驰骤一息干里若分各省疆界
则彼此推诿寇盗终无殄灭之日故化散为整之法不特
陆师宜然而水师尤为切要日本弹丸小乌不过夜郎靡
英之伦而年来发愤自雄变更峨冠博带之旧习师法轮
船飞𪿫之新制其阴而有谋固属可虑其穷而无赖则更
可忧以北境之塞希伦地与俄而日俄之交固用李太国

开火车铁路而多借英国之债其国主尝见英使巴夏礼
与之潸谋密计而日英之交固用黎展远密杳台湾情形
资为𭎡臂腹心而日美之交固彼其低首下心沁沁𬀪𬀪
以求悦于各国者岂有他哉盖其觊觎台湾已寝食寤寐
之不忘中国倘弃之如遗固既从心所欲万一势出于战
则又交昵各国为之解铃说合不致能发而不能收此其
所以敢肆然无忌快志于一逞也臣任苏藩司时曾于议
覆修钧条内陈明曰本阴柔而有远志中国所买𬬰𪿫皆
彼国选余之物宜阳与之好而阴为之备其时李鸿章深

以臣言为然常即代为密陈今日本虽小有所偿然彼之
所费既不啻十倍此数况死于是役者复五六百人万一
他日复借端发难以数舶横亘于黄海黑水洋之间则洋
沪之气不通事事为之棘手而台湾之患犹其小而且缓
者也故今日驭远之法内则力图整顿不可徒托空言外
则虚与委蛇不必稍涉虚㤭不惟与泰西各国开诚布公
示之以信即曰本亦宜暂事羁縻使目前不致决裂候我
水陆各军均既精练自可潜消其窥伺之心万一不能彼
出于骄而我应之以正亦为薄海之所其谅此练兵之当

务速务实不可得过且过者也
一简器原奏称凡𪿫台及水𪿫台所需巨𪿫应如何购办
水陆各军所用洋𬬰应如何一律购用最精之器及以后
应如何自行铸造精益求精之处等因查外洋火器至今
日如此之精非惟唐宋元明之所未有抑亦尧舜禹汤之
所不及料总理衙门所称知效彼之长巳居于后然使并
无此器更何所恃诚为洞见症结之论惟火器一项不外
𪿫𬬰火箭等物有宜于攻者有宜于守者有攻与守并宜
者英国之大𪿫有曰阿勿斯郎有曰巴留西有曰安司脱

浪有曰回得活特法国大𪿫曰墨迭儿鲁士布国大陛曰
克虏伯美国大𪿫有曰巴勒得有曰四得卧得有曰布鲁
嘎斯有曰德里氏戛盖诸国之𪿫以阿勿斯耶德里氏嘎
为最大以克虏伯布鲁戛斯为最精大者吃子至六百磅
闻其铸造时内用生铁外套熟铁钉以螺丝既成之后多
用火药轰放使内外二层涨力匀透生铁与熟铁相切巳
紧然后以之施用阿勿斯郎在其本国购买巳须一万九
千圆买价既大即运费亦当不轻精者二十四磅之弹能
与百磅弹同其锐力从前𪿫后开门仅用左右双劈近则

用整块圆劈又用药演放千数百次腹内始加钢圈钢底
弹则加以铅壳比膛略大𪿫腹有螺旋三十二转必使弹
由腹中相荡相摩宛转而后出口此涨力之所以加大速
率之所以加快也但无论如何大𪿫其命中须在一里内
外过远则弹子本体之坠重力与空气之阻拦力皆足以
累之恐攻坚不能有劲矣至美国之格林𪿫管多放速有
同鱼贯蝉联布国之联珠𬬰两人肩负而行若中国之抬
𬬰一分杪可放数十次亦为陆战行营之所必不可少者
或欲击近则用马口铁盒实以群子以漆固之出口后亦

能四散扑人如风雨之驿至但须圆滑合膛方能得力其
欲越山越城而击不能望见之物则用十五寸径口以上
之么打𪿫昂其首而用高弧之度自上而下可以炸物焚
营南花旗𪿫台为北花旗所毁多受此种𪿫子之害至洋
𬬰一件外国不三十年而巳屡变其制初用火石引火枪
继用钢策引火枪最后以来福枪为第一等自南北花旗
交战北花旗始用林明敦枪南花旗始用果伦比枪自布
法交战法国始用筛师拨枪布国始用尼一根抢经一次
之战争则必增一番之惨酷造物至此亦几无以供其雕

镌其枪腹背有来福皆从后门进子循环迭放无坚不摧
然而机簧太繁用久则渐失挺力而不能尽如人意而且
铜捲子药购觅艰难子罄则𬬰为废物故只能用于临阵
操演只可仍用旧𬬰其抵御马队则用能炸之火箭倘两
军相接我占顺风则用喷筒毒烟以迷敌目使其洋𬬰不
能施放器椷既利则又在平时心定与手热总之机器及
前敌之军械必须精于腹地各省庶得以重驭轻之法至
于𪿫台则宜建于地险水曲敌船必宛转而后能驭行者
之处方能使敌船多受数𪿫又可从前面后面为通行之

围若台设于水路径直之地则敌船瞬息即过岂能𪿫𪿫
中其要害北海惟大沽之水道最曲大江自镇江以下惟
金山前水势迥环均可建筑𪿫台焦山四面受敌似不如
也造台之法极内一层须用灰墙外墙则用砖石不如用
三合土其厚总须在二十尺以外高低则视地势之低昂
及水路之中偏护墙必须成交角而不可成正角斜至五
股之一勾敌𪿫若来自可斜拂而过不致显与为抵其𪿫
台及火药仓上必设太平盖以御自上而下之𪿫子下必
设高阳堆以御横扫之𪿫子其最下层之他隧必须加筑

留互四面俱通沟外之小𪿫台大沙堆亦必须迤逦照应
百使敌用陆兵闯入尚可侧轰横截然而北花旗之铁甲
船为南花旗𪿫台之𪿫所于伤者仅三只为水雷所轰沈
者十余只盖专用𪿫台而无水椿水雷浮坝等物阻于前
则𪿫台断不能得力而敌船之游驶可以自如而无忌若
台中大𪿫则自六百磅以至二十四磅之𪿫无不可用惟
放𪿫地步愈宽则愈可转移愈密则愈受敌弹此在位置
者先事之绸缪与临时之变通耳外国寓兵于工即寓工
于士故制造与行兵穊可归于一贯中国两离之则两缺

此其所以不能以格致为自强之本也若夫机器局之设
必须在煤木麇集五金易采之处光为便易江西之鄱阳
湖边有数大岛山阻水环敌船所不能入而南赣汀建之
水亦司乘涨而至上达楚蜀而下逮皖吴于此建一大机
器厂气易通而科易集臣上年曾以此事商之曾国藩李
鸿章皆以为可只以无事而止今机器之役事方经始有
进境而无止境若精华全在涧滨势同孤注万一彼族变
生不测先下辣手岂不深尝经营是则欲制器又必先觅
制器之地为尤切而且要者矣

一造船原奏称创立外海水师应如何添购各兵船及铁
甲船水𪿫台应用若干船只该船吃水最深各海口何处
宜于驻洎如何抵御如何攻破逐一详议等因查外国前
十余年新闻纸即有云中国自唐虞用木船荡桨至今数
千年仍是用木船荡桨可谓永远守执古法等语盖所以
调之者微矣易曰穷则变变则通战国杜挚有言曰利不
百不变法工不干不易器盖及今而能变则尚有可通之
日及今而不变则再无可变之时外国之铁甲船有数等
其最上者中用样木与黄松木外加极韧而有大凹凸力

全无炭质之熟铁板五层每层约厚四刀层层用螺丝钉
嵌凡遇船中吃力之处则铁板加层加厚盖铁甲数层相
合者𪿫子难穿独层厚铁者𪿫子易穿也铁板之下必用
坚木以为之垫有厚十二寸者有厚八寸者方可稍减敌
弹震动之力而又嵌铁弹于木垫之内使不穿透其最大
者机器力重有一千五百疋马力吃水太深中国口岸内
恐无此深水之港难以购用今年英国驶乘换仁刁之銕
甲船约八百疋马力者用之于中国洋面最为合式若如
曰本所购之铁甲船本系木质不过上面蒙以三四寸之

铁仅有二百八十疋马力船下吃水之处亦至无铁若以
两枝半栀之结实头板船乘风撞之自必震动松裂非真
铁甲船也购买之价视船之精粗大小厚薄新旧及马力
多寡机器锅炉之灵便结宝往往有贱于兵轮船者大约
上与中之铁甲船价总在十万磅以外二十万磅以内每
百磅又须加船渔杂费十二圆半若托洋人辗转购求必
致误买木质之蒙铁者不如选派熟习船务结实可靠之
委员分往外国船厂托其制造一面带同中国制船𮩽船
之人前往认真学习俟其造成中国工人亦可习焉而化

大约英美法丹各国船厂每厂各宜定造一船成后再行
考较优劣贵贱以为委员之赏罚方不致虚縻巨款现在
英国有大小炼甲船五十四号法国六十二号俄国二十
四号美国四十六号其间以木船旧质外蒙铁甲藉为虚
声者亦属不少中国洋面延袤最宽目前大小铁甲船栭
少须十号将来自能创造极少须三十号方敷防守海口
以及游历五大洲保护中国商人至停泊铁甲船之处固
须水深然海底必须硬泥之质庶能受锚若软泥质则起
锚艰难沙质则𬮁易走动石质及蛤壳质则不能受锚中

国极好锚地以香港为最盖上有重山迥护可以避风而
下则水深二三十拓不致过浅过深今巳归之英人抑无
庸议北海辽东之老铁山前后以及搭途上长子凫等处
海面全是泥质水深二三十拓不等直隶辽东二海大风
不越十二时虽无山势挡阻亦属无妨此间似可泊铁甲
船二三号距大沾南𪿫台之南高墩约八里以外海底泥
质此间似可泊铁甲船三四号东北海有此数船首尾相
应则津沽山海关鸭绿江之门户可固惟十月水漠以后
须将各船移泊烟台以资活动烟台港外有崆峒列岛可

以遮护风力海底亦是泥质似可泊中小铁甲船一二号
杨子江口崇明山之南面水深二三十拓不等惟海底软
泥居多中亦有泥沙相合者可以抛锥此间似可泊铁甲
船二三号上以通津沪之气下以保太平洋万里之安台
湾北面距日本之九修乌为直隶一苇可杭似宜泊铁甲
船二三号以固东南枢细但台湾东北海面水势为吕宋
诸山所束缚波涛最险不如泊于澎湖渔翁二乌之间抑
或鸡笼港等处亦易运煤锚地亦尚稳妥广东虎门水非
不深而海底不平且一旦与诸国有事即不能驶出香港

与东北洋诸铁甲船相联络照应资首尾之互击似只可
泊铁甲船一二号以为自固之用其铁甲船攻破𪿫台之
法在八百丈以内者可用八寸径以上之螺丝𪿫配以实
心尖弹专指台角一处层放迭击不可忽东忽西俟有倾
圯之形然后自上而下递击递低其台墙自必渐裂渐离
卸矣其十五寸以上之么打𪿫炸弹则用以仰攻台中之
火药仓太平盖使其延烧台兵自无站足之地而船中人
抽配陆兵为常行垒以逼之敌人接济一绝有不涣然瓦
解者乎其铁甲船目卫之法尚遇两岸有林木之处船桅

\backmatter

\end{document}