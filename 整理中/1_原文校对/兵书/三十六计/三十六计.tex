% 三十六计
% 三十六计.tex

\documentclass[12pt,UTF8]{ctexbook}

% 设置纸张信息。
\usepackage[a4paper,twoside]{geometry}
\geometry{
	left=25mm,
	right=25mm,
	bottom=25.4mm,
	bindingoffset=10mm
}

% 设置字体,并解决显示难检字问题。
\xeCJKsetup{AutoFallBack=true}
\setCJKmainfont{SimSun}[BoldFont=SimHei, ItalicFont=KaiTi, FallBack=SimSun-ExtB]

% 目录 chapter 级别加点(.)。
\usepackage{titletoc}
\titlecontents{chapter}[0pt]{\vspace{3mm}\bf\addvspace{2pt}\filright}{\contentspush{\thecontentslabel\hspace{0.8em}}}{}{\titlerule*[8pt]{.}\contentspage}

% 设置 part 和 chapter 标题格式。
\ctexset{
	chapter/name={第,计},
	chapter/number={\chinese{chapter}}
}

% 图片相关设置。
\usepackage{graphicx}
\graphicspath{{Images/}}

% 设置古文原文格式。
\newenvironment{yuanwen}{\bfseries\zihao{4}}

% 设置署名格式。
\newenvironment{shuming}{\hfill\zihao{4}}

% 注脚每页重新编号,避免编号过大。
\usepackage[perpage]{footmisc}

\title{\heiti\zihao{0} 三十六计}
\author{}
\date{}

\begin{document}

\maketitle
\tableofcontents

\frontmatter

\chapter{前言}

《三十六计》是我国古代兵家计谋的总结和军事谋略学的宝贵遗产,是根据我国古代卓越的军事思想和丰富的斗争经验总结而成的智谋书,它精炼概括了中国古代谋略的精华,是当代中外智谋学家颇为关注的重要著作之一。本书以“三十六计”为纲目,总领古今经典谋略,史论结合,充分发挥案例对计谋含义的形象诠释作用,是计谋学图书的一大创新。

“三十六计”一语,出自《南齐书·王敬则传》,《传》云:“檀公(道济)三十六策,走是上计,汝父子唯应急走耳。”原意是王敬则用来讽刺东昏侯父子,败局已定,无可挽回,唯有退却,方是上策。宋代惠洪《冷斋夜话》里说:“渊材曰:三十六计,走为上计。”此句也作“三十六着,走为上着”,如《水浒》第二回里说:“我儿,三十六着,走为上着,只恐没处走。”

“三十六”本为虚数,不必过于拘泥,只是借太阳六六之数,表示诡计多端而已。在我国古代的书籍里,以“三十六”命名的事物很多:如三十六雨、三十六段、三十六苑、三十六宫、三十六峰,三十六兽,三十六禽,三十六国,三十六郡等。职业三十六行,一变就是七十二行;连孔子三千学生中,后人却以七十二位来作“贤人”之数。这和明、清两代描写地方景物时,一定要凑足“八景”、“十景”;罗织别人的罪名一定要列成“十大罪状”是一样的。到了明末清初,就有好事者采集群书,根据我国古代卓越的军事思想和丰富的斗争经验编撰成《三十六计》。但此书成于何时?为何人所撰?目前已难确考。

据原书“序言”说,在全书“三十六计”中,每六计为一套,共分六套:第一套为“胜战计”,第二套为“敌战计”,第三套为“攻战计”,第四套为“混战计”,第五套为“并战计”,第六套为“败战计”。前三套是处于优胜时所用之计,后三套是处于劣势时所用之计。每套又各包含六计,总共三十六计。其计名,有的来源于历史典故,如“围魏救赵”、“假道伐虢”等;有的来源于古代军事术语,如“以逸待劳”、“声东击西”等;有的来源于古代诗人的诗句,如“李代桃僵”、“擒贼擒王”等;有的借用众所周知的成语定名,如“金蝉脱壳”、“指桑骂槐”等;还有出自其他方面的。这些计名易懂易记,便于在群众中流传。各计所含内容,多属古代兵家诡谲之谋,可以说它是采集兵家之“诡道”,专讲军事谋略的一本兵书。每计中的文字虽然不多,但非常精炼,其中每计中的尾语,均依据《易经》中的阴阳变化之理及古代兵家刚柔、奇正、攻防、彼己、虚实、主客、劳逸、顺逆、高下、死生等对立关系相互转化的思想推演而成,使每一计都含有朴素的军事辩证法的色彩。我国古代军事家,如孙武、孙膑、韩信、李靖等,都精通《易经》,并用于军事,以“易”演兵。原“按语”中多引征宋代以前的战例和孙武、吴起、尉缭子等兵家的精辟语句,为理解《三十六计》的正文起到了很好的辅助作用。

现传《三十六计》较早的版本为1941年成都兴华印刷厂土纸翻印本,封面书“三十六计”,旁注“秘本兵法”,并说明原书是手抄本。该书于1941年在邠州(今陕西省彬县)某书摊发现,抄本前部为“养生之谈”,而末尾附抄“三十六计”。1961年,收藏者在《光明日报》撰文加以介绍后,又将这土纸本赠给了中国人民解放军政治学院。此后便出现了各种翻印和传抄的版本。《三十六计》原本不分卷,发现时已有残缺,跋语不完。作者不详,亦无年代可考,有人疑为晚明或清初人所撰。但是从书中若干所用的计目名称看,也不能完全排除清代后期的可能。从全书内容看,可以肯定的是:该作者详熟《周易》,崇尚谋略,喜读兵书,了解古战例。

鉴于《三十六计》“原文”部分有的文字难于理解,有些成语典故的出处也不为大众所知,为广大读者能进一步了解《三十六计》的本义,我们首先对每一计的内容进行了“题解”,在“题解”中尽量对每一计的计名来源作了介绍;其次对“原文”中较难理解的词句尽可能地加以注释,然后对“原文”、“按语”进行了白话文的翻译;在“原文”和“按语”之后我们还增加了“点评”,并参考了有关书籍针对每一计的内容附列了一些历史上有关的生动事例,以供读者阅读参考。

由于作者水平有限,在注释和翻译中,难免还存在这样或那样的错误,敬请读者批评指正。

\begin{shuming}
王志萍、禹谦
\end{shuming}

\begin{shuming}
2015年10月
\end{shuming}

\chapter{序}

“三十六”本为虚数,不必过于拘泥,只是借太阳“六六”之数,表示诡计多端而已。“原序”解语重“数”不重“理”,盖“理”明则“术”自明,而“数”则在言外。如果只知“数”而不知“数”中有“术”,则“术”多不应。此外,三十六计中也特别强调“阴阳调和”,反映了我国古代朴素的辩证方法。

\begin{yuanwen}
用兵如孙子\footnote{指吴孙子孙武。孙武,字长卿,春秋时期齐国乐安(今山东惠民,一说博兴,或说广饶)人,著名军事家。曾以其著作《孙子兵法》十三篇见吴王阖闾,被受任为将。领兵打仗,战无不胜,与伍子胥率吴军破楚,五战五捷,率兵三万打败六十万楚国大军,攻入楚国郢都。北威齐晋,南服越人,显名诸侯。《孙子兵法》为后世兵法家所推崇,被誉为“兵学圣典”,孙武也被后人尊称为孙子、兵圣、百世兵家之师、东方兵学的鼻祖。《史记》有《孙子吴起列传》。},策谋三十六\footnote{指三十六个计谋。}。

六六三十六,数中有术,术中有数\footnote{“数”即气数,指客观规律;“术”是指方法、手段、计谋、策略。指主观意识。“数”产生“术”,“数”决定“术”。“数”、“术”也合称为“数术”,亦作“术数”,在古代它是一种专门之学,即用种种方术观察自然界现象,来推测人和国家的气数和命运。李零先生认为:“‘数术’一词大概与‘象数’的概念有关。‘象’是形于外者,指表象或象征;‘数’是涵于内者,指数理关系和逻辑关系。它既包括研究实际天象历数的天文历算之学,也包括用各种神秘方法因象求义、见数推理的占卜之术。虽然按现代人的理解,占卜和天文历算完全是两类东西,但在古人的理解中,它却是属于同一体系,因为在他们看来,前者和后者都是沟通天、人的技术手段。”}。阴阳燮理,机在其中\footnote{“阴”、“阳”是我国古代哲学文化中的重要内容,这一学说是把宇宙万物作为对立的统一体来看待,表现出我国古代哲学朴素的辩证思想。在兵法中,兵阴阳家依据《易经》中的阴阳变化之理及古代兵家刚柔、奇正、攻防、彼己、虚实、主客、劳逸、顺逆、高下、死生等相互对立、相互转化的关系来指挥作战,并形成了一种颇具特色的“兵阴阳家”理论。在本书中的多数“计”中都使用了这种理论,使整个书中都含有朴素的军事辩证法的色彩。燮,调和,谐和。指阴阳调和。机,机变。此句的意思是说:只要“阴阳”和谐,机变也就在其中了。“机”不可预设,必须根据实际情况而定,即所谓随机而变。《兵法圆机·二机》里说:“势之维系处为机,事之转变处为机,物之紧切处为机,时之凑合处为机。”燮(xiè)}。机不可设,设则不中\footnote{意谓机遇是不可事先来设计的,如果事先想象有机遇,往往是不会实现的。中,适于,合于。}。
\end{yuanwen}

【按语】解语重数不重理。盖理,术语自明;而数,则在言外。若徒知术之为术,而不知术中有数,则数多不应。且诡谋权术,原在事理之中,人情之内。倘事出不经则诡异立见,诧事惑俗而机谋泄矣。或曰,在三十六计中,每六计成为一套,第一套为胜战计,第二套为敌战计,第三套为攻战计,第四套为混战计,第五套为并战计,第六套为败战计。

【译文】用兵打仗就应该像孙武一样,要多想些计谋。六六三十六,“数”中有“术”,“术”中有“数”。按照阴阳变化的道理去思考,机遇就在其中。但机遇是不可事先来设计的,如果事先设计了机遇,往往是不会实现的。

【按语译文】本书的解释着重在运用计谋的客观规律上,而不是讲述运用计谋的具体方法。具体的方法,在战术策略中已经明白;而客观规律则是在言语之外。如果只知道战术策略,却不知道其中的客观规律,则所使用的方法大多不能取得成果。况且诡谋权术原本就蕴含在事理之中和人情之内。如果计谋违背了事理人情,那么奇怪的现象就会出现,从而引起诧异和迷惑,这样,计谋就会被泄露。也有人说,在三十六计中,每六计成为一套:第一套为稳操胜券的战计,第二套是势均力敌的战计,第三套为主动进攻的战计,第四套是战势混乱的战计,第五套为联盟作战的战计,第六套是不利情况下的战计。

\mainmatter

\chapter{瞒天过海}

“瞒天过海”,本来是比喻用伪装来哄骗对方,背地里偷偷地行动。用在兵法上是一种“示假隐真”的疑兵之计,用来作战役的伪装,以期达到“出其不意”的战斗成果。“瞒天过海”之“天”喻指皇帝,本义指瞒着皇帝,让他平稳过海。语出《永乐大典·薛仁贵征辽事略》。云:

唐太宗贞观十七年(643),御驾亲征,领三十万大军来安定东土。一天,浩荡大军东进来到大海边上,太宗看见眼前只是一片白浪排空,茫茫无穷,即向众随臣询问过海之计。臣下面面相觑,无一回应。忽传有一个近居海上的豪民请求见驾,并称三十万大军过海所需军粮此家业已独备。太宗大喜,便率百官随这豪民来到海边。只见万户皆用一彩幕遮围,十分严密。豪民老人引帝入室,室内更是绣幔彩锦,茵褥铺地。百官进酒,宴饮甚乐。不久,风声四起,波响如雷,杯盏倾侧,人身摇动,良久不止。太宗警惊,忙令近臣揭开彩幕察看,不看则已,一看愕然。满目皆一片清清海水横无际涯,哪里是什么在豪民家作客,大军竟然已航行在大海之上了!原来这豪民是由新招壮士薛仁贵扮成,这“瞒天过海”的计策就是他策划的。

“瞒天过海”这种计谋在某种程度上含有欺骗性在内,但其动机、性质、目的是决不可以与“欺上瞒下”之类混为一谈。这一计的兵法运用,常常是着眼于人们在观察处理世事中,由于对某些事情的习见不疑而自觉不自觉地产生了疏漏和松懈,因此乘虚而示假隐真,掩盖某种军事行动,使用伪装的手段,把握时机,出奇制胜。

\begin{yuanwen}
备周则意怠\footnote{防备周密,就会使人斗志松懈,削弱战斗力。},常见则不疑。阴在阳之内,不在阳之对\footnote{“阴”、“阳”是我国古代哲学文化的重要内容,这一学说是把宇宙万物作为对立的统一体来看待,表现出我国古代哲学朴素的辨证思想。在本计中,“阴”、“阳”是指兵阴阳家依据《易经》中阴阳变化之理所形成的一种颇具特色的“兵阴阳家”理论。“阴”指机密、隐蔽;“阳”,指公开、暴露。“阴在阳之内,不在阳之对”就是说“阴(秘密的)”往往包含在“阳(公开的)”里,而不是在“阳”的对立面。这是一种朴素的军事辩证法理论。}。太阳,太阴\footnote{太:极,极大。阳:这里指公开的。阴:这里指隐蔽的。此句意谓非常公开的事情里往往蕴藏着非常的秘密。它们是对立统一的关系。}。
\end{yuanwen}

【按语】阴谋作为,不能于背时秘处行之。夜半行窃,僻巷杀人,愚俗之行,非谋士之所为也。如:开皇九年,大举伐陈。先是弼请缘江防人,每交代之际,必集历阳,大列旗帜,营幕蔽野。陈人以为大兵至,悉发国中士马,既而知防人交代,其众复散。后以为常,不复设备。及若弼以大军济江,陈人弗之觉也,因袭南徐州,拔之。

【译文】防备十分周到的,就容易斗志松懈,麻痹轻敌;平时看惯了的,往往就不再会产生怀疑。“阴(秘密的)”往往包含在“阳(公开的)”里,而不是在“阳”的对立面。非常公开的事情里往往蕴藏着非常的秘密。

【按语译文】施行秘密的谋略,不能在不适合的时候和秘密的地方进行。夜半行窃、僻巷杀人都是愚蠢、鄙俗的人干的,智谋之士是不干这种事情的。如开皇九年,隋兵讨伐陈国。在此之前,隋将贺若弼统兵驻防江岸,每次换防的时候,一定要把军队集中到历阳,大张旗鼓,到处都支满了营帐。陈国的军队以为隋兵要大举进攻了,就调发全国的兵马准备迎战,后来才发现隋军只是在换防,所以陈国的军队就又解散了。以后习以为常,陈国的军队就不再戒备了。等到隋将贺若弼率兵渡江时,陈国的军队居然没有察觉到,因此隋军攻下了陈国的南徐州(今江苏省镇江市一带)。

\chapter{围魏救赵}

“围魏救赵”是古代以少胜多的著名战例。事见《史记·孙子吴起列传》,讲的是战国时期齐国与魏国的桂陵之战。公元前354年,魏惠王欲报丢失中山的旧恨,便派大将庞涓前去攻打中山。中山原本是东周时期魏国北邻的小国,后被魏国收服,后来赵国乘魏国国丧伺机将中山强占了。魏将庞涓认为中山不过弹丸之地,距离赵国又很近,不如直接攻打赵国的都城邯郸,这样既解旧恨又可得到邯郸,可谓一举双得。魏王从之,即拨五百战车,以庞涓为将,直奔赵国,包围了赵国都城邯郸。赵王急难中只好求救于齐国,并许诺解围后以中山相赠。齐威王应允,令田忌为将,并起用从魏国救得的孙膑为军师领兵出发。

孙膑曾与庞涓同学,对用兵之法谙熟精通。庞涓自觉能力不及孙膑,恐其贤于己,遂以毒刑将孙膑致残,断孙膑两足并在他脸上刺字,企图使孙膑不能行走,又羞于见人。后来孙膑装疯,幸得齐使者救助,逃到齐国。这是一段关于庞涓与孙膑的旧事。

再说田忌与孙膑率兵进入魏、赵交界之地时,田忌想直逼赵国邯郸,孙膑制止说:“排解争斗,不能直接参与搏击,平息纠纷要抓住要害,乘虚取势,双方因受到制约才能自然分开。现在魏国精兵倾国而出,若我直攻魏国,那庞涓必回师解救,这样一来邯郸之围一定会自解。我们再从中途伏击庞涓归路,其军必败。”田忌依计而行。果然,魏军离开邯郸,归途中又陷伏击,与齐战于桂陵,魏部卒长途疲惫,溃不成军,庞涓勉强收拾残部,退回大梁。齐师大胜,赵国之围遂解。这便是历史上有名的“围魏救赵”的故事。又后十三年,齐魏之军再度相交于战场,庞涓又复陷于孙膑的伏击,庞涓自知智穷兵败,遂自刎。孙膑以此名显天下。

\begin{yuanwen}
共敌不如分敌\footnote{此句意谓攻打集中的敌人不如设法分散它而后再打。共,集中。分,分散。},敌阳不如敌阴\footnote{此句意谓打击气势旺盛的敌人不如打击气势不旺盛的敌人。敌,这里用作动词,攻打。这里的“阳”和“阴”也是兵阴阳家的军事理论。敌阳,敌军的阳刚部分。敌阴,敌人的阴弱部分。《唐太宗李卫公问对·卷中》里说:“后用则阴,先用则阳。尽敌阳节,盈吾阴节而夺之,此兵家阴阳之妙也。”意思是说后出兵就会显得虚弱,先出兵就会显得气盛。使敌军气势衰竭,使我军士气强盛来夺取敌人,这是兵家在阴阳方面的妙用。《百战奇谋·后战》:“侯其气衰而击之,则胜。”意思是说等待敌人的士气衰落之后再对他发起进攻,就可以取得胜利。}。
\end{yuanwen}

【按语】治兵如治水,锐者避其锋,如导疏;弱者塞其虚,如筑堰。故当齐救赵时,孙子谓田忌曰:“夫解杂乱纠纷者不控拳,救斗者不搏击。批亢捣虚,形格势禁,则自为解耳。”

【译文】进攻兵力集中的部队,不如攻打兵力分散的部队;打击气势旺盛的敌人,不如打击气势衰弱的敌人。

【按语译文】用兵作战就如同治理洪水一样,对于来势凶猛的敌人,要避开他的锋芒,就像治理洪水要导流一样;对于来势虚弱的敌人,就要堵住他歼灭他,这就像治理洪水要修筑堤坝一样。所以当齐国派田忌援救赵国之时,孙膑对田忌说:“要想解开纷繁杂乱的结绳,不能用拳头去捶打;要解开打架斗殴者,只能动口劝说,不能动手参加。对敌人应避实就虚,使对方的形势受挫而不能发展,赵都之围自然也就解决了。”

\begin{yuanwen}
	
\end{yuanwen}\begin{yuanwen}
	
\end{yuanwen}\begin{yuanwen}
	
\end{yuanwen}\begin{yuanwen}
	
\end{yuanwen}\begin{yuanwen}
	
\end{yuanwen}\begin{yuanwen}
	
\end{yuanwen}\begin{yuanwen}
	
\end{yuanwen}\begin{yuanwen}
	
\end{yuanwen}\begin{yuanwen}
	
\end{yuanwen}\begin{yuanwen}
	
\end{yuanwen}\begin{yuanwen}
	
\end{yuanwen}\begin{yuanwen}
	
\end{yuanwen}\begin{yuanwen}
	
\end{yuanwen}\begin{yuanwen}
	
\end{yuanwen}\begin{yuanwen}
	
\end{yuanwen}\begin{yuanwen}
	
\end{yuanwen}\begin{yuanwen}
	
\end{yuanwen}\begin{yuanwen}
	
\end{yuanwen}\begin{yuanwen}
	
\end{yuanwen}\begin{yuanwen}
	
\end{yuanwen}\begin{yuanwen}
	
\end{yuanwen}\begin{yuanwen}
	
\end{yuanwen}\begin{yuanwen}
	
\end{yuanwen}\begin{yuanwen}
	
\end{yuanwen}\begin{yuanwen}
	
\end{yuanwen}\begin{yuanwen}
	
\end{yuanwen}\begin{yuanwen}
	
\end{yuanwen}\begin{yuanwen}
	
\end{yuanwen}\begin{yuanwen}
	
\end{yuanwen}\begin{yuanwen}
	
\end{yuanwen}\begin{yuanwen}
	
\end{yuanwen}\begin{yuanwen}
	
\end{yuanwen}\begin{yuanwen}
	
\end{yuanwen}\begin{yuanwen}
	
\end{yuanwen}\begin{yuanwen}
	
\end{yuanwen}\begin{yuanwen}
	
\end{yuanwen}\begin{yuanwen}
	
\end{yuanwen}\begin{yuanwen}
	
\end{yuanwen}\begin{yuanwen}
	
\end{yuanwen}\begin{yuanwen}
	
\end{yuanwen}\begin{yuanwen}
	
\end{yuanwen}\begin{yuanwen}
	
\end{yuanwen}\begin{yuanwen}
	
\end{yuanwen}\begin{yuanwen}
	
\end{yuanwen}\begin{yuanwen}
	
\end{yuanwen}\begin{yuanwen}
	
\end{yuanwen}\begin{yuanwen}
	
\end{yuanwen}\begin{yuanwen}
	
\end{yuanwen}\begin{yuanwen}
	
\end{yuanwen}\begin{yuanwen}
	
\end{yuanwen}\begin{yuanwen}
	
\end{yuanwen}\begin{yuanwen}
	
\end{yuanwen}\begin{yuanwen}
	
\end{yuanwen}\begin{yuanwen}
	
\end{yuanwen}\begin{yuanwen}
	
\end{yuanwen}\begin{yuanwen}
	
\end{yuanwen}\begin{yuanwen}
	
\end{yuanwen}\begin{yuanwen}
	
\end{yuanwen}\begin{yuanwen}
	
\end{yuanwen}\begin{yuanwen}
	
\end{yuanwen}\begin{yuanwen}
	
\end{yuanwen}\begin{yuanwen}
	
\end{yuanwen}\begin{yuanwen}
	
\end{yuanwen}\begin{yuanwen}
	
\end{yuanwen}\begin{yuanwen}
	
\end{yuanwen}\begin{yuanwen}
	
\end{yuanwen}\begin{yuanwen}
	
\end{yuanwen}\begin{yuanwen}
	
\end{yuanwen}\begin{yuanwen}
	
\end{yuanwen}\begin{yuanwen}
	
\end{yuanwen}\begin{yuanwen}
	
\end{yuanwen}\begin{yuanwen}
	
\end{yuanwen}\begin{yuanwen}
	
\end{yuanwen}\begin{yuanwen}
	
\end{yuanwen}\begin{yuanwen}
	
\end{yuanwen}\begin{yuanwen}
	
\end{yuanwen}\begin{yuanwen}
	
\end{yuanwen}\begin{yuanwen}
	
\end{yuanwen}\begin{yuanwen}
	
\end{yuanwen}\begin{yuanwen}
	
\end{yuanwen}\begin{yuanwen}
	
\end{yuanwen}\begin{yuanwen}
	
\end{yuanwen}\begin{yuanwen}
	
\end{yuanwen}\begin{yuanwen}
	
\end{yuanwen}\begin{yuanwen}
	
\end{yuanwen}\begin{yuanwen}
	
\end{yuanwen}\begin{yuanwen}
	
\end{yuanwen}\begin{yuanwen}
	
\end{yuanwen}\begin{yuanwen}
	
\end{yuanwen}\begin{yuanwen}
	
\end{yuanwen}\begin{yuanwen}
	
\end{yuanwen}\begin{yuanwen}
	
\end{yuanwen}\begin{yuanwen}
	
\end{yuanwen}\begin{yuanwen}
	
\end{yuanwen}\begin{yuanwen}
	
\end{yuanwen}\begin{yuanwen}
	
\end{yuanwen}\begin{yuanwen}
	
\end{yuanwen}\begin{yuanwen}
	
\end{yuanwen}\begin{yuanwen}
	
\end{yuanwen}\begin{yuanwen}
	
\end{yuanwen}\begin{yuanwen}
	
\end{yuanwen}\begin{yuanwen}
	
\end{yuanwen}\begin{yuanwen}
	
\end{yuanwen}\begin{yuanwen}
	
\end{yuanwen}\begin{yuanwen}
	
\end{yuanwen}\begin{yuanwen}
	
\end{yuanwen}\begin{yuanwen}
	
\end{yuanwen}\begin{yuanwen}
	
\end{yuanwen}\begin{yuanwen}
	
\end{yuanwen}\begin{yuanwen}
	
\end{yuanwen}\begin{yuanwen}
	
\end{yuanwen}\begin{yuanwen}
	
\end{yuanwen}\begin{yuanwen}
	
\end{yuanwen}\begin{yuanwen}
	
\end{yuanwen}\begin{yuanwen}
	
\end{yuanwen}\begin{yuanwen}
	
\end{yuanwen}\begin{yuanwen}
	
\end{yuanwen}\begin{yuanwen}
	
\end{yuanwen}\begin{yuanwen}
	
\end{yuanwen}\begin{yuanwen}
	
\end{yuanwen}\begin{yuanwen}
	
\end{yuanwen}\begin{yuanwen}
	
\end{yuanwen}\begin{yuanwen}
	
\end{yuanwen}\begin{yuanwen}
	
\end{yuanwen}\begin{yuanwen}
	
\end{yuanwen}\begin{yuanwen}
	
\end{yuanwen}\begin{yuanwen}
	
\end{yuanwen}\begin{yuanwen}
	
\end{yuanwen}\begin{yuanwen}
	
\end{yuanwen}










第三计 借刀杀人


“借刀杀人”,比喻自己不出面,借别人的手去害人。用在军事上,就是利用矛盾、反间、离间等谋略,巧妙地借用别国的力量来击败敌人的谋略,而自己却不会受到损失。这个“刀”不一定是“人”,也可以是一种事物或一种势力。

此计多是封建官僚之间尔虞我诈、相互利用的一种政治权术。在军事上,主要体现了善于利用第三者的力量,善于利用或者制造敌人内部的矛盾,以达到取胜的目的。此计在使用上也没有固定的形式,要因地、因事、因时、因势来“借刀”。古时有名的战例有:曹操借孙权杀关羽、郑桓公借刀诛郐臣等。

此计是根据《周易》六十四卦中的《损卦》推演而得的。《损卦》里说:“损下益上,其道上行。”此卦认为,“损”、“益”不可截然划分,二者可以相辅相成,充满了辩证思想。





敌已明,友未定 (1) ,引友杀敌,不自出力,以《损》推演 (2) 。

【按语】敌象已露,而另一势力更张,将有所为,便应借此力以毁敌人。如:郑桓公将欲袭郐,先向郐之豪杰、良臣、辨智、果敢之士尽书姓名,择郐之良田赂之,为官爵之名而书之,因为设坛场郭门之处而埋之,衅之以鸡猪,若盟状。郐君以为内难也,而尽杀其良臣。桓公袭郐,遂取之(《韩非子·内储说下》)。

诸葛亮之和吴拒魏,及关羽围樊、襄,曹欲徙都,懿及蒋济说曹曰:“刘备、孙权外亲内疏,关羽得志,权必不愿也。可遣人劝蹑其后,许割江南以封权,则樊围自释。”曹从之,羽遂见擒(《长短经·格形》)。





【注释】


(1) 友未定:“友”指军事上的结盟者,即除敌、我两方之外的一时可以同盟而借力的人或国家。“敌已明,友未定”,意思是说敌人已经明确,但盟友尚持徘徊、观望的态度,其主意不明不定。

(2) 以《损》推演:《损》即指《易经·损卦》,《损卦》云:“损:有孚,元吉,无咎,可贞,利有攸往。”孚,信用。元,大。贞,正。意谓只要有诚心,就会有大的吉利,没有错失,合于正道,这样行事就可一切如意。《彖》辞曰:“损:损下益上,其道上行。”意谓《损卦》是损减下面而增益上面,它是上面推行的道理,指“损”与“益”的转化关系。借用盟友的力量去打击敌人,势必要使盟友受到损失,但盟友的损失正可以换得自己的利益。推演,推断,演绎。





【译文】


敌方的情况已经明确,而盟友的态度还不稳定,要诱导盟友去消灭敌人,自己不用出实力,这就是根据《损卦》“损下益上”的道理推演出来的谋略。

【按语译文】敌人已经显露出来,另一股势力正在发展,而且将有所行动,这时便应借此势力来摧毁敌人。比如,郑桓公将要袭击郐国,先将郐国的豪杰、良臣、辨智、英勇果敢之士的名单列出,公开张贴布告,说要选择郐国的良田赠送给他们,封给他们各种名称的官爵也写清楚,并在城郊设起祭坛,把名单埋在地下,用公鸡、公猪作祭品,装作盟誓的样子。郐国国君就会以为国内这些豪杰、良臣要勾结郑国,便按照郑桓公公布的名单把他们一个个杀掉了。桓公看到郐国豪杰、良臣都已除尽,便马上攻打郐国,并占领了郐国。(见《韩非子·内储说下》)

又如诸葛亮与吴国结盟,共同抗拒魏国,当关羽围攻魏国属地襄阳、樊城时,曹操想迁都,司马懿和蒋济劝说曹操说:“刘备、孙权表面上是亲密,但内心里却隔阂很深。关羽得志,孙权内心是不甘愿的。因此,可以派人跟随孙权身后劝说他,并答应割让江南的土地封给他,这样,樊城的包围就会自然解开。”曹操听从了司马懿和蒋济的计策,关羽终于兵败麦城,束手被擒了。(见《长短经·格形》)





第四计 以逸待劳


“以逸待劳”,意思是指作战时采取守势,养精蓄锐,等待来进攻的敌人疲劳之后再出击。语出《孙子·军争篇》,原作“以佚待劳”,佚(yì),同“逸”。《军争篇》曰:“故善用兵者,避其锐气,击其惰归,此治气者也。以治待乱,以静待哗,此治心者也。以近待远,以佚待劳,以饱待饥,此治力者也。”意思是说,善于用兵的将领可以避开敌人的锐气,趁其士气低落衰竭时就发起猛攻,这就是正确掌握士气的方法。用治理严整的军队来对付治理混乱的军队,用军心镇定的军队来对付军心躁动的军队,这就是正确掌握军心的方法。在自己较近的战场上等待远道而来的敌人,在自己部队得到充分休息的状态下等待疲惫不堪的敌人,在自己部队吃饱肚子的情况下等待饥饿的敌人,这就是正确掌握军力的方法。又,《孙子·虚实篇》曰:“凡先处战地而待敌者佚(同‘逸’),后处战地而趋战者劳。故善战者,致人而不致于人。”原意是说,凡是先到战场等待敌人的,就从容、主动,后到达战场的只能仓促应战,一定会疲劳、被动。所以,善于指挥作战的人,总是调动敌人,而决不会被敌人调动。





困敌之势 (1) ,不以战 (2) ;损刚益柔 (3) 。

【按语】此即致敌之法也。兵书云:“凡先处战地而待敌者佚,后处战地而趋战者劳。故善战者,致人而不致于人。”(《孙子·虚实篇》)兵书论敌,此为论势,则其旨非择地以待敌,而在以简驭繁,以不变应变,以小变应大变,以不动应动,以小动应大动,以枢应环也。如:管仲寓军令于内政,实而备之(《史记·管晏列传》);孙膑于马陵道伏击庞涓(《史记·孙子吴起列传》);李牧守雁门,久而不战,而实备之,战而大破匈奴(《史记·廉颇蔺相如列传》)。





【注释】


(1) 困敌之势:迫使敌人处于被围困的境地。

(2) 不以战:不用进攻的方法。以,用。

(3) 损刚益柔:语出《易经·损卦》。《损卦》中说:“损刚益柔有时,损益盈虚,与时偕行。”意思是说损减刚强、补益弱柔有一定的时机,损盈益虚,要随着时机一起进行。这里是讲“损”、“益”是相互联系、相互转化的关系。“损刚益柔”是根据此卦象讲述“刚柔相推,而主变化”的普遍道理和法则。“刚”、“柔”也是两个相对的事物现象,在一定的条件下相对的两方有可相互转化。此计正是根据“损”卦的道理,以“刚”喻敌,以“柔”喻己。可以根据刚柔相互转化的原理,实行积极防御,逐渐地消耗、疲惫对方,使它由强变弱;而我因势利导,又可使自己变被动为主动,不用直接进攻的方法,同样可以制胜。





【译文】


迫使敌人处于困难的局面,不一定要战争直接进攻的手段;可以采取“损刚益柔”相互转化的原理来使敌人由强转弱。

【按语译文】这就是“以柔制刚”的攻敌办法。兵书上说:“凡是先到战场等待敌人者,他就显得安逸而有精力;后赶到阵地仓促应战者,必然显得疲劳困顿。因此,善于用兵作战的人,总是能够调动敌人而不被敌人所调动。”兵书上讲的是战争中的敌我形势,这里讲的是如何挖掘主动权,并非仅仅指选好地形来等待敌人,而是泛指以简便控制繁难、以不变应万变、以小变应大变、以静制动、以小变动对待大的变动,抓住关键来控制周围局面。例如春秋时齐国的管仲寓军事于内政,寓兵于民,这是“实而备之”的施政大战略(做到有备无患,以逸待劳);孙膑在马陵道伏击庞涓(是守株待兔,以逸待劳);战国时赵国大将李牧戍守北方的雁门关,用了很长时间来备战却不主动挑起战争,他是在“以实备之”,一旦奋起,便一战而大破匈奴。





第五计 趁火打劫


“趁火打劫”,趁,乘机。劫,抢劫。原意是趁别人家里失火,一片混乱、无暇自顾的时候,去抢人家的财物。比喻乘人之危,谋取私利;也比喻乘别人处在危难时刻从中捞取一把。这是个贬义成语。

此计用在军事上,指的是当敌方遇到困难或危难的时候,就要乘此机会进兵出击,制服对手。《孙子·计篇》云“乱而取之”,唐朝杜牧解释《孙子》此句说:“敌有昏乱,可以乘而取之。”讲的就是这个道理。





敌之害大 (1) ,就势取利 (2) ,刚决柔也 (3) 。

【按语】敌害在内,则劫其地;敌害在外,则劫其民;内外交害,则劫其国。如:越王乘吴国内蟹稻不遗种而谋攻之,后卒乘吴北会诸侯于黄池之际,国内空虚,因而捣之,大获全胜。(《国语·吴语·越语下》)





【注释】


(1) 敌之害大:指敌人遭遇到危厄的处境。害,指困难,危机。

(2) 就势取利:意谓乘着敌人柔弱之时,冲击敌人从而取胜。就,依照。势,时机。

(3) 刚决柔也:这里指强大者乘势出兵征服弱小者。语出《易经·夬卦》。《夬卦》的《彖》辞说:“夬,决也。刚决柔也。”决,冲决、冲开、去掉的意思。此卦为异卦相叠(乾下兑上)。上卦为兑,兑为泽;下卦为乾,乾为天。因乾卦为六十四卦的第一卦,乾为天,是大吉大利,吉利的贞卜,所以此卦的本义是力争上游,刚健不屈。所谓“刚决柔”,就是下乾这个阳刚之卦在冲决上兑这个阴柔的卦。此计是以“刚”喻己,以“柔”喻敌,言乘敌之危,就势出兵征服弱小者。





【译文】


在敌方处于大的困难、危机时,就乘机出兵夺取胜利。这就是“以刚克柔”的方法。

【按语译文】敌人遭遇内乱,就乘机占领其土地;敌人遭受外患,就乘机掠夺其民财;敌人内忧外患,就乘机占领其国家。例如:越王勾践乘吴国国内遭遇大旱灾,连螃蟹和稻谷的种子都没有的困境下,想乘机进攻吴国。后来终于等到吴王夫差北上到黄池与诸侯会盟的时候,国内空虚,因而大举进攻吴国,不费吹灰之力,大获全胜。





第六计 声东击西


“声东击西”,声,指声张。意思是说表面上声张去攻打东边,实际上却去攻打西边。在军事上是给对方制造错觉而出奇制胜的一种战术。语本于《淮南子·兵略》。《兵略》云:“故用兵之道,示之以柔而迎之以刚,示之以弱而乘之以强,为之以歙(收合)而应之以张,将欲西而示之以东。”意思是说用兵的策略是向敌人显示我军似乎很柔弱,实际上却用刚强的力量来迎击他;向敌人做出收敛的势态,实际上却大张旗鼓地进攻;将要向西进攻,却摆出一副向东的假象。引诱敌人作出错误判断,然后乘机歼敌。

“声东击西”之计,是运用“坤下兑上”之卦象的象理,比喻“敌志乱萃”而造成了错综复杂、危机四伏的处境,我则要抓住敌人这一不能自控的混乱之势,机动灵活地运用时东时西、似打似离、不攻而示它以攻、欲攻而又示之以不攻等战术,进一步造成敌人的错觉,出其不意地一举夺胜。

“声东击西”之计早已被历代军事家熟知,原“按语”中通过两个战例,来提醒使用此计的人必须考虑对手的情况:敌方指挥可以扰乱者,用此计必胜;如果对方指挥官头脑冷静,识破计谋,此计就不可能发挥效力了。黄巾军中了朱隽佯攻西南方之计,遂丢失宛城(今河南南阳)。而周亚夫处变不惊,识破敌方计谋。吴军佯攻东南角,周亚夫下令加强了西北方向的防守,当吴军主力进攻西北角时,周亚夫早有准备,吴军无功而返。





敌志乱萃 (1) ,不虞 (2) ,坤下兑上之象 (3) ,利其不自主而取之。

【按语】西汉,七国反,周亚夫坚壁不战。吴兵奔壁之东南陬,亚夫便备西北。已而吴王精兵果攻西北,遂不得入。(《汉书·周勃传》附)此敌志不乱,能自主也。汉末,朱隽围黄巾于宛,张围结垒,起土山以临城内,鸣鼓攻其西南,黄巾悉众赴之。隽自将精兵五千,掩其东北,遂乘虚而入。此敌志乱萃,不虞也。然则声东击西之策,须视敌志乱否为定。乱,则胜;不乱,将自取败亡,险策也。





【注释】


(1) 敌志乱萃:语出《易经·萃卦》中的《象》辞。《象》辞曰:“乃乱乃萃,其志乱也。”乱,散乱。萃,聚会,聚集。意思是说一会儿散乱,一会儿聚集,这里指敌人的意志已经混乱。

(2) 不虞:意想不到,未预料到。《孙子·九地篇》:“兵之情主速,乘人之不及,由不虞之道攻其所不戒也。”

(3) 坤下兑上:语出《易经·萃卦》。《象》曰:“泽上于地,《萃》。”泽在地上(水在地上横流),是《萃卦》。《萃卦》为异卦相叠(坤下兑上)。上卦为兑,兑为泽;下卦为坤,坤为地。有泽水淹及大地、洪水横流之象。这里用来比喻有意外的变乱。





【译文】


敌人的意志已经混乱,一会儿散乱,一会儿聚集,又不能判明和应付突然事变的发生,这是指挥员失去分析判断情况的能力的一种征象。这时就要利用敌人失去控制力(不自主)的时机将其消灭。

【按语译文】西汉景帝时,七国叛乱。周亚夫固守城堡,拒不出战。吴王刘濞带兵佯攻城堡的东南角,周亚夫识破了敌人声东击西的计谋,在西北方向加强防备,不久吴王刘濞的精兵果然进攻西北面,但没有成功。这是统帅的神志不乱,能够镇定自若、沉着应战的结果。西汉末年,朱隽包围了宛城(今河南南阳)的黄巾军,他在城外堆起一座土山观察城内的形势,当他鸣鼓进攻西南角时,黄巾军立即集结到西南面进行抵抗。在这种情况下,朱隽亲自率领五千精兵,突然袭击城的东北角,于是乘虚而攻入。这就是敌人神志慌乱,不能正确预料和判断战场形势的结果。然而运用“声东击西”的计策,必须观察敌人是否真能被迷惑后才能决定用与不用。敌人慌乱无主时,运用这一计谋就能取胜;敌人不慌乱而有准备时,运用这一计谋将自取灭亡。因此这是一个十分冒险的计谋。





第七计 无中生有


所谓“无中生有”,原来是道家对事物的朴素的辩证看法。道家认为,天下万物生于有,有生于无。语本《老子》四十章“天下万物生于有,有生于无”。后用来形容本无其事,凭空捏造。是个贬义成语。在军事上,“无中生有”的这个“无”,指的是“假”、“虚”;这个“有”,指的是“真”、“实”。“无中生有”,就是真真假假,虚虚实实,真中有假,假中有真。虚实互变,扰乱敌人,使敌方造成判断上的错觉和行动上的失误。

此计的关键在于真、假要有变化,虚、实必须结合,以各种假象掩盖真相,造成敌人的错觉,给予不意的攻击。若一假到底,易被敌人发觉,难以制敌。先假后真,先虚后实,“无”中必须生“有”。指挥者必须抓住敌人已被迷惑的有利时机,迅速地以“真”、以“实”、以“有”出奇制胜,攻击敌方,在敌人头脑还来不及清醒时,即被击溃。





诳也 (1) ,非诳也,实其所诳也 (2) 。少阴、太阴、太阳 (3) 。

【按语】无而示有,诳也。诳不可久而易觉,故无不可以终无。无中生有,则由诳而真,由虚而实矣,无不可以败敌,生有则败敌矣,如:令狐潮围雍丘,张巡缚蒿为人千余,披黑衣,夜缒城下;潮兵争射之,得箭数十万。其后复夜缒人,潮兵笑,不设备,乃以死士五百砍潮营,焚垒幕,追奔十余里。(《新唐书·张巡传》、《战略考·唐》)。





【注释】


(1) 诳(kuáng):欺诈,诳骗。指各种欺骗活动。

(2) 实:实在,真实。

(3) 少阴、太阴、太阳:指少阴渐变为太阴,太阴至极而转化为太阳。讲阴阳转化之理—“阳变阴来”、“阴极阳生”。这里指欺敌活动的发展过程,即由虚假转化为真实,造成敌人再次上当受骗,达到出其不意。这里的“阴”指假象;这里的“阳”指真相。此句意谓用大大小小的假象去掩护真相。





【译文】


运用假象欺骗对方,但并非全部都是假的,而是让对方把真相当成假象。这就是要巧妙地用阴阳转化之理,“由阴变阳”、“由虚变实”、“由真变假”等。

【按语译文】没有而装作有,就叫诳骗。诳骗不能长久使用,因为长久使用容易被对方发觉,所以没有的不能长久没有。“无中生有”,就是把诳骗变成事实,由空虚变成实在。什么都没有是不可以战败敌人的,只有人为地制造出“实在”,才可以战败敌人。例如:唐朝叛将令狐潮围困雍丘城,张巡用草扎束成一千多个假人,并给它们穿上黑色的衣服,乘黑夜用绳索把草人从城墙上吊下去;令狐潮的士兵争先恐后地用箭射击,因此张巡获得几十万支箭。后来,张巡在黑夜把真人吊下城去,令狐潮的兵士看着发笑,以为又是草人,也没有再提防。于是张巡乘机用五百名敢死队员突然冲进令狐潮的军营,烧毁营垒帐篷,并且把令狐潮的军队一直追杀了十多里远。





第八计 暗度陈仓


“暗度陈仓”,指正面迷惑敌人,而从侧翼进行突然袭击;亦比喻暗中进行活动。原话为“明修栈道,暗度陈仓”。“栈道”,是指在悬崖峭壁间傍山凿石架木而成的道路。“陈仓”,古县名,在今陕西宝鸡市东,古代为通向汉中的交通孔道。汉将刘邦将从汉中出兵攻打项羽的部队时,听取了张良、韩信的计策,表面上派兵修复栈道,迷惑对方,实际上暗中绕道奔袭陈仓,并取得大捷。《古今杂剧·韩元帅暗度陈仓》里说:“樊哙明修栈道,俺可暗度陈仓古道。这楚兵不知是智,必然排兵在栈道守把。俺往陈仓古道抄截,杀他个措手不及也。”后来因称用明显的行动迷惑对方、使人不备的策略为“明修栈道,暗度陈仓”。此计是汉代大将军韩信创造,“暗度陈仓”也是古代战争史上的著名成功战例。





示之以动 (1) ,利其静而有主 (2) ,“《益》动而巽” (3) 。

【按语】奇出于正,无正则不能出奇。不明修栈道,则不能暗度陈仓。昔邓艾屯白水之北,姜维遣廖化屯白水之南,而结营焉。艾谓诸将曰:“维令卒还,吾军少,法当来渡而不作桥,此维使化持我,令不得还。必自东袭取洮城矣。”艾即夜潜军,径到洮城。维果来渡。而艾先至,据城,得以不破。此则是姜维不善用暗度陈仓之计,而邓艾察知其声东击西之谋也。





【注释】


(1) 示:给人看。动:这里指军事上的正面佯攻等迷惑敌方的军事行动。

(2) 利其静而有主:此句意谓利用敌人平静时做出作战的主张。静,平静。主,主张。

(3) 《益》动而巽:益,增强。“巽”为风,无孔不入。语出《易经·益卦》。此卦为异卦相叠(震下巽上)。上卦为巽,巽为风;下卦为震,震为雷。意即风雷激荡,其势愈增,故卦名为《益》。与《损卦》之义互相对立。《益卦》的《彖》辞里说:“《益》动而巽,日进无疆。”这是说《益卦》下震为雷为动,上巽为风为顺,那么,动而合理,是天生地长,好处无穷。在军事上是指增强和发挥作战的机动性。





【译文】


故意暴露行动,利用敌方平静时做出作战的主张,暗中迂回到敌人的后边偷袭,这就《易经·益卦》所说的能乘虚而入,出奇制胜。

【按语译文】出奇制胜的用兵之法来源于正常的用兵原则,假若没有正常的用兵原则也就没有出奇制胜的用兵之法了。推而言之,如果不去佯修栈道,也就不能暗度陈仓了。三国时邓艾驻军在白水的北岸,姜维则派遣廖化在白水的南岸安营扎寨。邓艾对他的几位将领说:“姜维把他的军队撤回去了。我们的部队人数少,按常理他应该不等架桥就急速过江来进攻我们。而现在我看他们不急来架桥,这肯定是姜维利用廖化想把我们拖住,使我们不得返回。姜维他自己必定率领大部队向东袭取洮城。”于是邓艾连夜急速带领部队从暗中小路直回洮城。果不出他所料,姜维正在那里渡河。由于邓艾领兵抢先一步赶到,并全力拒守洮城,洮城才没有被姜维攻破。这就是姜维不善于运用“暗度陈仓”之计,而邓艾则识破了他“声东击西”的计谋。





第九计 隔岸观火


“隔岸观火”的本义是从河这边观看对岸失火。比喻在别人出现危难时,袖手旁观,待其自毙,以便从中取利。语出唐乾康《投谒齐己》,云:“隔岸红尘忙似火,当轩青嶂冷如冰。”义同“坐山观虎斗”、“黄鹤楼上看翻船”。

如果用在战争中,若敌方内部分裂,矛盾激化,相互倾轧,势不两立,这时切切不可操之过急,免得反而促成他们暂时联手来对付我方。正确的方法是静止不动,让他们互相残杀,力量削弱,甚至自行瓦解。

原“按语”里认为“隔岸观火”与《孙子·火攻篇》里孙子言“慎动之理”意相吻合。在《火攻篇》的后段里,孙子强调战争是利益的争夺,如果打了胜仗而无实际利益,这是没有用的。所以,“非利不动,非得(指取胜)不用,非危不战,主不可以怒而兴师,将不可以愠(指怨愤、恼怒)而致战。合于利而动,不合于利而止”。所以说一定要慎用兵,戒轻战,战必以利为目的。此计正是运用了《豫》卦顺时以动的哲理,说坐观敌人的内部恶变,我不急于采取攻逼手段,顺其变,“坐山观虎斗”,最后让敌人自残自杀,时机一到而我即坐收其利,一举成功。





阳乖序乱 (1) ,阴以待逆 (2) 。暴戾恣睢 (3) ,其势自毙 (4) 。顺以动《豫》,《豫》顺以动 (5) 。

【按语】乖气浮张,逼则受击,退则远之,则乱自起。昔袁尚、袁熙奔辽东,尚有数千骑。初,辽东太守公孙康恃远不服。及曹操破乌丸,或说曹逐征之,尚兄弟可擒也。操曰:“吾方使斩送尚、熙首来,不烦兵矣。”九月,操引兵自柳城还,康即斩尚、熙,传其首。诸将问其故,操曰:“彼素畏尚等,吾急之,则并力;缓之,则相图,其势然也。”或曰:此兵书火攻之道也。按兵书《火攻篇》前段言火攻之法,后段言慎动之理,与“隔岸观火”之意亦相吻合。





【注释】


(1) 阳:这里指公开的。乖:违背,不和谐,不协调。《骈字分笺》中说:“反和为乖。”“阳乖序乱”,就是指敌方内部明显地表现出多方面的混乱。

(2) 阴:这里指隐蔽的、暗中的。逆:叛逆。“阴以待逆”,就是指暗中观察和等待敌方的内部发生叛逆。

(3) 戾:凶暴,猛烈。“暴戾”就是残暴凶狠的意思。恣睢:放纵,任意胡为。

(4) 自毙:自取灭亡。

(5) 顺以动《豫》,《豫》顺以动:语出《易经·豫卦》。本卦为异卦相叠(坤下震上)。本卦的下卦为坤为地,上卦为震为雷。坤者顺,震者动。顺其性而动者,莫不得其所。“顺以动《豫》,《豫》顺以动”为彖辞,讲顺物性而动的道理。《豫卦》的《彖》辞说:“《豫》,刚应而志行,顺以动,《豫》。《豫》顺以动,故天地如之,而况‘建侯行师’乎?天地以顺动,故日月不过而四时不忒。圣人以顺动,则刑法清而民服,《豫》之时义大矣!”意思是说《豫卦》(有一个阳爻,是刚;上下五个阴爻,是柔)刚得柔相应(如君得臣民相应),而君的意志得行。(《豫卦》坤下是顺,震上是动)顺着自然而动,是《豫卦》。《豫卦》顺着自然而动,所以天地也像它,天地按照顺着自然而动,所以日月的运行没有偏差,四时的循环没有差错。圣人顺着自然而动,就会刑法清明,人民服从。





【译文】


敌方内部公开地表现出多方面混乱时,我方应暗中观察和等待其内部发生叛逆。待其残暴凶狠、任意胡为时,敌方就会自取灭亡。这就是《豫卦》所讲的“顺以动《豫》,《豫》顺以动”的道理。

【按语译文】当敌方内部公开地表现出多方面混乱嚣张时,若直接进逼,必然会遭到敌人的拼命反击;若是退避得远远的,敌人内部必然会发生混乱。从前,袁尚、袁熙被曹操击败后率领数千残兵逃到辽东。起初,辽东太守公孙康仗着离曹操遥远,并不臣服曹操。后来曹操击败乌桓,有人建议曹操立刻乘胜去征服公孙康,就能擒获袁氏兄弟。然而曹操说:“叫公孙康自动杀掉袁尚、袁熙,把首级亲自送来吧!”到九月间,当曹操率领大军从柳城归来时,公孙康果然杀了袁氏兄弟,并把他们的首级送上。众将领向曹操请教其中道理,曹操说:“公孙康向来惧怕袁尚二兄弟,如果我从外部急追紧攻,他们必然会联合起来对付我;如果我慢慢地远远回避,他们必然会互相残杀,这种形势是客观必然的。”有人说,这是孙子兵书《火攻篇》所讲的道理。《孙子·火攻篇》的前面部分是阐述火攻的方法,后面部分是阐述用兵慎重的道理,这与“隔岸观火”计谋意思恰好吻合。





第十计 笑里藏刀


“笑里藏刀”,形容对人表面上表现得和蔼可亲,而内心里却阴险毒辣。语本于《旧唐书·李义府传》:“义府貌状温恭,与人语必嬉怡微笑,而褊忌阴贼。既处权要,欲人附己,微忤意者,辄加倾陷。故时人言义府笑中有刀。”意思是说义府貌似温柔,与人说话时一定要面带微笑,但是一旦与他的意见不一,就必加陷害。所以当时人们称他为“笑中刀”。后来人们就以“笑里藏刀”来指那种口蜜腹剑、两面三刀、“口里喊哥哥,手里摸家伙”的做法。

此计用在军事上,是指应用政治外交上的伪装手段,欺骗麻痹对方,来掩盖己方的军事行动。这是一种表面友善、使人疏于防备,而暗藏杀机的谋略。





信而安之 (1) ,阴以图之 (2) ,备而后动,勿使有变 (3) 。刚中柔外也 (4) 。

【按语】兵书云:“辞卑而益备者,进也……无约而请和者,谋也。”故凡敌人之巧言令色,皆杀机之外露也。宋曹玮知渭州,号令明肃,西夏人惮之。一日,玮方对客弈棋,会有叛卒数千,亡奔夏境。堠骑报至,诸将相顾失色,公言笑如平时。徐谓骑曰:“吾命也,汝勿显言。”西夏人闻之,以为袭己,尽杀之。此临机应变之用也。若勾践之事夫差,则意使其久而安之矣。





【注释】


(1) 信:这里是使动用法,即使敌人相信。安:这里是使动用法,即使敌人安定,不生疑心。

(2) 阴:暗地里。图:图谋。

(3) 勿:不要。“勿使有变”就是不要使敌人发生意外的变化。

(4) 刚:刚毅。柔:柔顺。“刚中柔外”,意谓表面柔顺,实质强硬尖利。语出《易经·兑卦》。《易经·兑卦》里说:“兑,说也。刚中而柔外。”意思是说:“兑”是喜悦的意思。要内中刚毅而外表柔和。





【译文】


表面上要做到使敌方信任我方而丧失警惕,暗地里我方则秘密策划消灭敌人的办法。做好充分准备以后才能行动,不要使敌人发生意外的变化。这就是表面上柔和而内中却刚毅的道理。

【按语译文】兵书上说:“敌人的态度表现得谦卑并在暗中加紧准备,这是敌人要向我方发起进攻的征候……没有具体条约文字而请求媾和的,一定是另有阴谋。”所以,凡是敌人花言巧语、满脸堆笑的,这都是暗藏杀机的外在表露。宋朝的曹玮镇守渭州,纪律严明,西夏人为此都很害怕他。有一天,曹玮正在与客人下棋,突然有几千名士兵叛变,逃奔到西夏的境内去了。当边防的侦察员骑马迅速来报告时,曹玮下面的将领都大惊失色,而曹玮却依然谈笑自如,好像什么事都没发生。他不慌不忙地对侦察人员说:“这都是按照我的命令去行事,你们千万不要声张。”西夏人听说后,以为这些叛军是宋营派来杀他们的奸细,于是便把他们全杀掉了。这就是“随机应变”计谋的运用。就像越王勾践要报复吴国一样,越王勾践卧薪尝胆,百恭百顺地侍候夫差,竟然能使夫差长期安逸而失去戒备。这也是勾践成功地运用了“笑里藏刀”之计的结果。





第十一计 李代桃僵


“李代桃僵”,“僵(jiāng)”,是枯死的意思,这里指李树代替桃树而死。原比喻兄弟互相爱护,互相帮助。后转用来比喻互相顶替或代人受过。语出《乐府诗集·鸡鸣篇》。《鸡鸣篇》云:“桃生露井上,李树生桃旁。虫来啮桃根,李树代桃僵。树木身相代,兄弟还相忘?”此诗的原意是以李树代替桃树受虫蛀,用来比喻兄弟要像桃树、李树相互帮助,相互友爱。

此计用在军事上,是指在双方势均力敌或者敌优我劣的情况下,用最小的损失换得最大的胜利,即用小的代价换取大的胜利。也就是大家在象棋比赛中的“舍车保帅”的战术。

两军对峙,敌优我劣或势均力敌的情况是经常发生的,如果指挥者指导正确,就可变劣势为优势。原“按语”中举的“田忌赛马”的故事就是大家所熟知的。孙膑帮助田忌在其马总体上不如齐王之马的情况下,仍能以二比一获胜。但是,运用此法也不可生搬硬套。春秋时齐魏桂陵之战,魏军左军最强,中军次之,右军最弱。齐将田忌准备按孙膑赛马之法作战,孙膑却认为不可。他说,这次作战不是争个二胜一负,而应大量消灭敌人。田忌于是用下军对付敌人最强的左军,以中军对付与敌人力量相近的中军,以自己力量最强的左军迅速消灭敌人最弱的右军。齐军虽有局部失利,但敌方左军、中军已被钳制住,右军很快败退。田忌迅速指挥自己的上军乘胜与中军合力,力克敌方中军,得手后,三军合击,一起攻破敌方最强的左军。这样,齐军在全局上形成了优势,终于取胜。





势必有损 (1) ,损阴以益阳 (2) 。

【按语】我敌之情,各有长短。战争之事,难得全胜。而胜负之决,即在长短之相较;而长短之相较,乃有以短胜长之秘诀。如以下驷敌上驷,以上驷敌中驷,以中驷敌下驷之类,则诚兵家独具之诡谋,非常理之可测也。





【注释】


(1) 势:指势态,局势。损:减少,损失。

(2) 阴:这里此指某些细微的、局部的事物。阳:这里指整体意义的、全局性的事物。益:补充,增加。“损阴以益阳”用在军事谋略上,就是要敢于以某种损失为代价来换取最终的胜利。





【译文】


如果形势必然有所损失,那就应该放弃局部的利益去换取整体的利益。

【按语译文】敌我双方的情况,各自存在优势和劣势。在战争中,很难取得全面胜利。而双方的胜败,就在于双方力量中的长处与短处、优势与劣势的较量。在长处与短处、优势和劣势的较量中,也存在着劣势战胜优势的秘诀。比如田忌赛马,就是用自己的下等马对对方的上等马,用自己的上等马对对方的中等马,用自己的中等马对对方的下等马之类的巧妙办法,这就是军事家独具的谋策,并不是用常理可以推断的。





第十二计 顺手牵羊


“顺手牵羊”,原来的意思是说,在路上看到一只羊,就顺便把别人的羊牵回自己家的意思。“顺”就是顺便的意思,比喻伺机取利,得到意外的收获。后来比喻趁势将敌人捉住,或乘机利用别人。现在多比喻趁对方不注意的时候,顺手拿走人家的东西或顺势做某件事情。

语出《礼记·曲礼上》,云:“效马效羊者右牵之。”郑玄注:“用右手便。效,犹呈见。”本意谓进献的马羊就用右手随便牵着。后世因以“顺手牵羊”比喻顺便行事,毫不费力。

古人对“顺手牵羊”之计十分重视。《草庐经略·游兵》中说:“伺敌之隙,乘间取利。”《登坛必究·叙战》中说:“见利宜疾,未利则止。取利乘时,间不容息。先之一刻则大过,后之一刻则失时也。”《鬼谷子·谋篇》中说:“察其天地,伺其空隙。”《李卫公问对·卷中》说:“伺隙捣虚。”上述兵法中虽未出现“顺手牵羊”的字样,但却是对“顺手牵羊”之计的最好说明。

用在军事上,“顺手牵羊”取胜的例子不胜枚举。在战争中就是要看准敌人出现的漏洞,及时抓住其薄弱点,乘虚而入,获取胜利。古人云:“善战者,见利不失,遇时不疑。”意思是说要把握战机,乘隙争利。





微隙在所必乘 (1) ,微利在所必得 (2) 。少阴,少阳 (3) 。

【按语】大军动处,其隙甚多,乘间取利,不必以战。胜固可用,败亦可用。





【注释】


(1) 微:小。隙:疏忽,漏洞。这里指敌方的某些漏洞、疏忽。乘:这里是“抓住”的意思。

(2) 得:得到,获取。

(3) 少阴:小的损失,这里指敌方小的疏漏。少阳:小的利益,这里指我方小的利益。此句意谓我方要善于捕捉时机,伺隙捣虚,变敌方小的疏漏为我方小的利益。





【译文】


(敌方)细微的疏忽也必须抓住利用,无论多么微小的利益我方也必须得到。变敌方小的疏忽为我方小的利益。

【按语译文】(敌人的)大部队在调动的过程中一定会有很多漏洞,要利用敌人的疏忽来获取利益,不一定要通过正规的作战方法。这个方法,在胜利的形势下本来就可以使用,在失败的形势下也可以使用。





第十三计 打草惊蛇


蛇一般都隐藏在草里,只要草被打扰,蛇就会惊跑,这就是“打草惊蛇”的本义。后来比喻某些事情相似,如果甲受到惩处,就会使乙感到惊慌。再后来就用“打草惊蛇”来比喻因做事不密,反使对方得以警戒预防。

语出宋代郑文宝《南唐近事·类说二一》。《南唐近事》里说:唐代当涂县令王鲁贪赃敛财,搜刮老百姓的民脂民膏。一天,当地百姓联名告发他手下的一个人受贿。王鲁见了状子,十分恐慌,生怕自己的不法行径也被揭露出来,便不由自主地在状子上批了八个字:“汝虽打草,吾已惊蛇。”后来,人们就把这八个字简化为“打草惊蛇”这个成语。

“打草惊蛇”作为谋略,是指敌方兵力没有暴露、行踪诡秘、意向不明时,万万不可轻敌冒进,暴露我军的意图,应当查清敌方主力配置、运动状况再说。





疑以叩实 (1) ,察而后动 (2) ;复者,阴之媒也 (3) 。

【按语】敌力不露,阴谋深沉,未可轻进,应遍探其锋。兵书云:“军旁有险阻、潢井、葭苇、山林、翳荟者,必谨复索之,此伏奸之所藏也。”(《孙子兵法·行军篇》)





【注释】


(1) 叩:问,查究。“疑以叩实”就是发现了疑点就应当询问清楚。

(2) 察:察看。

(3) 复者,阴之媒也:意谓反复叩实查究,是发现隐藏之敌的重要手段。复:反复。阴:此指某些隐藏着的、暂时尚不明显或未暴露的事物、情况。媒:媒介。





【译文】


对于有疑问的要予以询问核实,察看清楚后方可行动。反复察看询问,是发现隐情的重要手段。

【按语译文】当敌人的情况没有暴露出来,而且将其阴谋深藏不露,此时万万不可轻视敌人而贸然前进,应该采用各种方式从不同的侧面探明其锋芒所在。《孙子兵法》里说:“行军的路旁如有重险关隘、湖沼、芦苇、灌木茂盛的地方,必须谨慎地反复进行搜索,因为这些地方都是敌人可能隐匿伏兵和奸细的地方。”





第十四计 借尸还魂


“借尸还魂”,是指迷信的人认为人死后,灵魂还可以附在别人的身体上而复活。用来比喻某些已经死亡的事物又假托别的名义或借着另一种形式得以出现。

此计名出自中国古代的一个民间故事。从前,一个名叫李玄的人,拜太上老君为师,学长生不老之术。一天,他随太上老君魂游太空,留下凡胎肉体叫徒弟看守,说七日便回。到了第六天,徒弟忽然得到母亲病危的消息,匆忙将李玄的尸体焚化后离去。李玄的灵魂回来后无尸可投,不得不借路边一个刚死的乞丐的尸体还了魂,成了人形。这样,李玄就变成了蓬头垢面、跛脚秃头的“铁拐李”了。

“借尸还魂”作为一种计谋,指的是已经衰落或死亡的事物借另一种形式重新出现。用在军事上,引申指处于被动或面临失败的局面时,善于利用一切有利条件,扭转局势,争取主动,实现原先的意图。这些都可视为“借尸还魂”。

在历史上,尤其是在改朝换代的时候,都喜欢推出亡国之君的后代,打着死者的旗号来号召天下,用这种“借尸还魂”的方法,达到夺取天下的目的。





有用者,不可借 (1) ;不能用者,求借 (2) 。借不能用者而用之,匪我求童蒙,童蒙求我 (3) 。

【按语】换代之际,纷立亡国之后者,固借尸还魂之意也。凡一切寄兵权于人,而代其攻守者,皆此用也。





【注释】


(1) 有用者:这里是指“有作为”的意思。借:这里是“驾驭”的意思。意为世间有许多看上去很有用处的东西,但不可以去驾驭而为己用。

(2) 不能用者,求借:此句意与上句相对而言。意谓有些事物看上去没什么用途,但有时也还可以借助它而为己发挥作用。在兵法上,是说兵家要善于抓住一切机会,使无用的东西变为有用的东西。

(3) 匪我求童蒙,童蒙求我:语出《易经·蒙卦》。本卦是异卦相叠(下坎上艮)。上卦为艮为山,下卦为坎为水为险。山下有险,草木丛生,故说“蒙”。这是《蒙卦》的卦象。这里的“匪”同“非”,不是。童蒙,指幼小无知的孩儿,喻童子弱昧,必求师教诲以强立,故曰“童蒙”。此句意谓不是我求助于愚昧之人,而是愚昧之人有求于我。





【译文】


凡有作为的,就难以驾驭,不可利用;凡无作为的,必然会求助于我以自立。驾驭无作为的人来为我所用,这就是《蒙卦》所说的:不是我求助于愚昧之人,而是愚昧之人求助于我。

【按语译文】每当改朝换代的时候,有人会纷纷拥立亡国之君的后代,这是打着死者的旗号来“借尸还魂”的计策。凡是把一切兵力都寄托给别人,并让别人代替进攻或防御的,也都属于这一计谋的运用。





第十五计 调虎离山


“调虎离山”的意思是用计策把老虎调离原来对它有利的地形或据点,以便乘机进攻它。比喻为了便于行事,想法子引诱敌人离开原来的地方。

这在军事上是一种调动敌人的谋略。老虎是山中之王,因此若想打虎,必须要先把老虎从山中引诱出来,所谓“虎落平阳被犬欺”,因为老虎离开了山区后,威风尽失。“调虎离山”用在战略上,就是说当强敌一旦离开其根据地或有利的地形,就失去了它的优势。如果此时给予攻击,就会取得胜利。

《孙子兵法》中早就指出:不顾双方条件强硬进攻是下等策略,是会失败的。敌人既然已占据了有利的地势,又做好了应战的准备,就不能去与他争斗。应该巧妙地用小利去引诱敌人,把敌人调离坚固的防地,引诱到对我军有利的战区,我方就可以变被动为主动。利用天时、地利和人和的条件,一定可以击败敌人。汉末虞诩智骗羌人的故事就是个好例证。他故意说等待援兵,松懈了敌人的斗志,分散了他们的兵力;而他却日夜兼程行军,充分利用了时间;还用增加灶的数量来伪装,让敌人误以为援军已到,不敢轻举妄动。目的是扰乱敌人的意图。这样就充分发挥了己方的主动性,牵住了敌方的鼻子,以己方的意图随意调动敌方,终于取得了平羌的胜利。

这个计策的内涵是说在战场上若遇强敌,就要善用谋略,用假象使敌人离开有利的地形,丧失他的优势,使他寸步难行,并由主动变为被动,而我则出其不意地变被动为主动,从而取得战争的胜利。





待天以困之 (1) ,用人以诱之 (2) ,往蹇来连 (3) 。

【按语】兵书曰:“下政攻城。”若攻坚,则自取败亡矣。敌既得地利,则不可争其地。且敌有主而势大:有主,则非利不来趋;势大,则非天人合用不能胜。汉末,羌率众数千,遮虞诩于陈仓、崤谷。诩即停军不进,而宣言上书请兵,须到乃发。羌闻之,乃分抄旁县。诩因其兵散,日夜进道,兼行百余里,令军士各作两灶,日倍增之,羌不敢逼,遂大破之。兵到乃发者,利诱之也;日夜兼进者,用天时以困之也;倍增其灶者,惑之以人事也。(《后汉书·虞诩传》、《战略考·东汉》)





【注释】


(1) 待:等待。这里是利用或依靠的意思。天:指天时、天候、自然条件。困:围困。此句意谓利用天然的条件去围困敌人。

(2) 用人以诱之:用人为的假象去诱惑敌人。

(3) 往蹇来连:语出《易经·蹇卦》。本卦为异卦相叠(艮下坎上)。上卦为坎为水,下卦为艮为山。山上有水流,山石多险,水流曲折,言行道之不容易,这是《蹇卦》的卦象。《彖》曰:“《蹇》,难也,险在前也。见险而能止,知矣哉。”意思是说,《蹇》卦是难,前面有危险。看到危险能停止前进,这是有智慧的表现。蹇,困难。连,负车,即商人出门时拉的小车。比喻步履艰难。这句意谓:往来皆难,路途困难重重。

【译文】

利用天然的条件去围困敌人,用人为的方法去诱骗敌人。如果前进有危险,就引诱敌人过来。

【按语译文】《孙子兵法》中说:“最下的策略是围攻城邑。”如果强行围攻坚实的城邑,就是自取灭亡。既然敌人已经占据了有利的地形,就不应该去强攻争地。况且敌人早有准备,力量处于优势。敌人占据了有利的地形,是因为他认为有利可图,没有利益他是不会轻易来进攻的;敌人处于优势,如果不利用天时地利等有利条件就无法取胜。东汉末年,羌人首领统率数千兵马,在陈仓、崤谷中阻挡虞诩行军。虞诩就此驻军停止前进,并扬言向朝廷请求援兵,等待援兵到来后再进军。羌人听到这一消息后,便信以为真,就分散开去附近县城抢掠财物去了。虞诩便利用羌兵分散开的时机,下令日夜兼程向前进军,每日疾行百余里。虞诩又命令军士驻军做饭时做两个灶,如此则灶的数量每日增加一倍,羌人以为诩军兵力大增,不敢再追击他们,结果虞诩大破羌兵。虞诩宣称要等待援军来到后再向前进军,就是用利诱的方法将羌人调开了;他命令昼夜急行军,就是要利用天时地利的有利条件来使敌人处于困难之地;加倍修灶,就是人为地制造援军陆续赶到的假象来迷惑敌人。





第十六计 欲擒故纵


“欲擒故纵”,“擒”,捉。“故”,故意。“纵”,放。意思是故意先放开他,使他放松戒备,充分暴露,然后再把他捉住。比喻为了更好地控制,故意暂且放松一步。

打仗的目的是为了消灭敌人,夺取地盘。如果逼得敌人太急,“穷寇”就会狗急跳墙,垂死挣扎,反而会使己方损兵失地,这是不可取的。放敌人一马,不是放虎归山,目的在于让敌人斗志逐渐懈怠,体力、物力逐渐消耗,最后己方寻找机会,全歼敌军,达到消灭敌人的目的。

诸葛亮七擒七纵,决非感情用事,他的最终目的是在政治上利用孟获的影响,稳住南方。在地盘上,每次都乘机扩大疆土。在军事谋略上,释放敌人主帅不属常例。通常情况下,抓住了敌人是不可以轻易放掉的,以免后患。而诸葛亮审时度势,采用攻心之计,对孟获七擒七纵,主动权始终操在自己的手上,最终达到目的。这说明诸葛亮深谋远虑,能随机应变,巧用兵法,是个难得的军事奇才,也是军事史上“欲擒故纵”的一个绝妙战例。





逼则反兵 (1) ,走则减势 (2) 。紧随勿迫,累其气力 (3) ,消其斗志,散而后擒,兵不血刃 (4) 。《需》,有孚,光 (5) 。

【按语】所谓纵者,非放之也,随之,而稍松之耳。“穷寇勿追”,亦即此意。盖不追者,非不随也,不追之而已。武侯之七纵七擒,即纵而蹑之,故展转推进,至于不毛之地。武侯之七纵,其意在拓地,在借孟获以服诸蛮,非兵法也。若论战,则擒者不可复纵。





【注释】


(1) 逼则反兵:逼迫敌人太紧,他们就会拼死反扑。逼,逼迫。反,反扑。

(2) 走则减势:逃跑了就会减弱气势。走,跑。

(3) 累:拖累。

(4) 兵不血刃:意谓兵器上不沾血。即指不费一兵一卒。血刃,血染刀刃。

(5) 《需》,有孚,光:语出《易经·需卦》。《需》,卦名。本卦为异卦相叠(乾下坎上)。需的下卦为乾为天,上卦为坎为水,是降雨在即之象。也象征着一种危险存在着(因为“坎”有险义),必得去突破它,但突破危险又要善于等待。需,等待。《易经·需卦》卦辞曰:“《需》,有孚,光亨。”孚,俘获。光亨,通顺。《需卦》意谓要善于等待,要有耐性,就会有收获,就会大顺利。





【译文】


逼迫敌人太紧,他们就会拼死反扑;逃跑了就会减弱气势。紧紧跟随而不要逼迫敌人,以拖累敌人的体力,削弱敌人的斗志,待敌人溃不成军时再擒获他。这样不费一兵一卒就可以取得胜利。这就是从《易经·需卦》“有孚,光”(有收获,大顺利)里悟出的道理。

【按语译文】所谓“纵”者,并不是说要将敌人放走,而是要一直跟随他,只是稍微放松一点而已。“穷寇勿追”,就是这个意思。所谓“不追”者,并不是说不跟随,只是说不要把敌人追得太紧之意。诸葛亮施用七擒七纵的计谋追孟获,就是采用放了又追获的方法。所以要辗转推进,直到不毛之地。诸葛亮七次放了孟获,其用意在于扩展土地,是要利用孟获的地位使南方蛮族全部服从。严格地讲,这不属于兵法的范畴。如果从战争角度来说,已经逮住的敌人就不能轻易放走他。





第十七计 抛砖引玉


“抛砖引玉”,本义是抛出砖,引来玉。比喻用粗浅的、不成熟的意见或作品引出高明的、成熟的意见或作品。语出《景德传灯录·赵州东院从谂禅师》。《景德传灯录》云:“大众晚参,师云:‘今夜答话去也,有解问者出来。’时有一僧便出礼拜,师云:‘比来抛砖引玉,却引得个墼(jī,未烧的砖坯)子。’”相传唐代高僧从谂禅师对徒弟们的参禅要求极为严格,每个人必须集中精力,精心打坐,达到一种不受外界干扰、身心不动的佳境。一次晚上参禅的时候,从谂禅师有意试探徒弟们的定力,说:“今夜解答问题,有需要解答者出来。”其他的人都聚精会神地盘腿打坐,不为所动。唯有一个小僧走出来,回答禅师。从谂禅师看了看他,说了一句:“刚才本来想抛砖引玉,却引来一块连砖都不如的土坯!”

此计用在军事上,其意是指:利用没有价值的东西来换取珍贵而有价值的东西。有以小易大、以贱易贵的意思。“砖”和“玉”,就是一种形象的比喻。“砖”指的是小利,是诱饵;“玉”指的是作战的目的,即大利。“抛砖”是为了达到目的的手段,“引玉”才是要得到的目的。就好比钓鱼需用钓饵,先让鱼儿尝到一点甜头,它才会上钩;敌人占了一点便宜,才会误入圈套,吃大亏。这些都是用相类似的手段去迷惑、诱骗对方,使其懵懂上当,中我圈套,然后乘机获得利益。





类以诱之 (1) ,击蒙也 (2) 。

【按语】诱敌之法甚多,最妙之法,不在疑似之间,而在类同,以固其惑。以旌旗金鼓诱敌者,疑似也;以老弱粮草诱敌者,则类同也。如:楚伐绞,军其南门,屈瑕曰:“绞小而轻,轻则寡谋,请勿捍采樵者以诱之。”从之,绞人获利。明日,绞人争出,驱楚役徙于山中。楚人坐守其北门,而伏诸山下,大败之,为城下之盟而还。又如孙膑减灶而诱杀庞涓。(《史记·孙子吴起列传》)





【注释】


(1) 类以诱之:用某种类似的方法去诱惑他。

(2) 击蒙:语出《易经·蒙卦》。曰:“上九:击蒙,不利为寇,利御寇。”意思是说:攻击愚蒙之人或昏乱之国,作为侵略是不利的,作为抵御侵略是有利的。





【译文】


用某种类似的方法去诱惑他,然后攻击愚蒙之人或昏乱之国。

【按语译文】引诱敌人的方法有很多,但最妙的方法,不是用似是而非的方法,而是利用类同的东西,加强敌人的迷惑。用旌旗招展、击鼓鸣锣的方法去引诱敌人,就属于似是而非的一类;以老弱残兵、遗弃粮草的方法去引诱敌人,就属于用类同的一类。例如,春秋时代楚国征伐绞国,两军在绞国都城的南门相对峙,僵持不下。楚国大将屈瑕向楚武王献策说:“绞国弱小,国人轻浮。轻浮则谋略少,请派遣一些樵夫去引诱绞军。”楚武王听从了他的计策,果然,绞人获得了不少楚国的樵夫。第二天,楚军又如法炮制,绞国人争相出城,追赶楚国樵夫,樵夫们往山上逃。此时楚军主力列阵于绞国的北门外,另设伏兵于山下,这时趁机发起突袭,大败绞军,迫使绞国签订了城下之盟。此外,如孙膑减灶而诱杀庞涓的战例也属于这类计略。





第十八计 擒贼擒王


“擒贼擒王”的“擒”就是捉住的意思,多指作战时要先捉住敌人的首领。比喻做事必须先抓住要害。语出杜甫《前出塞》诗之六,诗云:“挽弓当挽强,用箭当用长。射人先射马,擒贼先擒王。”

此计认为攻打敌军主力,捉住敌人首领,这样就能瓦解敌人的整体力量。敌军一旦失去指挥,就会不战而溃。

俗话说“打蛇要打七寸”,也是因为“七寸”之处是蛇的心脏所在的位置。打坏了蛇的心脏,蛇自然不能存活。同理,做任何事情都要抓住要害和关键,这样就会取得事半功倍的效果。

此计用在军事上,是指作战时要先打垮敌军主力、擒获敌军的首领、使敌军彻底瓦解的谋略。捕杀敌军首领或者摧毁敌人的首脑机关,敌方陷于混乱,便可以彻底击溃它。指挥员不能满足于小的胜利,要通观全局,扩大战果,以得全胜。如果错过时机,放走了敌军主力或敌方首领,就好比放虎归山,后患无穷。





摧其坚 (1) ,夺其魁 (2) ,以解其体 (3) 。龙战于野,其道穷也 (4) 。

【按语】攻胜,则利不胜取。取小遗大,卒之利、将之累、帅之害、功之亏也。全胜而不摧坚擒王,是纵虎归山也。擒王之法,不可图辨旌旗,而当察其阵中之首动。昔张巡与尹子奇战,直冲敌营,至子奇麾下,营中大乱,斩贼将五十余人,杀士卒五千余人。巡欲射子奇而不识,剡蒿为矢。中者喜,谓巡矢尽,走白子奇,乃得其状。使霁云射之,中其左目,几获之,子奇乃收军退还。(《新唐书·张巡传》、《战略考·唐》)





【注释】


(1) 摧:摧垮,攻克。坚:坚固,坚实。这里指军队的主力。

(2) 魁:首领。

(3) 解其体:瓦解他的整体力量。

(4) 龙战于野,其道穷也:语出《易经·坤卦》。本卦是同卦相叠(坤下坤上),为纯阴之卦。引本卦上六《象传》曰:“龙战于野,其道穷也。”意思是说强龙在田野大地之上争斗,它的“道”已穷尽。传说“龙”本来是在大海里或在天空云雨中才能施展威力,如果陷在田野里搏斗,便一筹莫展,难以挣脱失败的结局。在这里比喻敌人面临绝境。





【译文】


摧毁敌人的主力,擒获敌人的首领,就可以瓦解他的整体力量。就好比苍龙出海到陆地上作战,便一筹莫展,没有好的办法。

【按语译文】战争取得胜利,其利取不胜取。如果满足于小的胜利而丢掉了获取大的胜利的时机,那是士兵的胜利、将军的累赘、主帅的祸害,甚至前功尽弃。如果取得全部胜利而不摧毁敌人的主力、擒获敌人的首领,就如同放虎归山,后患无穷。擒获敌人首领的方法,不能单从旌旗上去辨认,而要察看敌军的阵地上是谁在首先行动。唐肃宗时,张巡和尹子奇交战,张巡指挥的部队一直冲击到敌营的帅旗下面。当时敌营大乱,张巡指挥斩杀将领五十余人,斩杀士兵五千余人。可是,当张巡想用箭射死敌人首领尹子奇时,他却不认识尹子奇。张巡便命士兵用削尖的蒿杆当箭射敌。被射中的敌人发现是蒿杆后很高兴,认为张巡的箭已经射完,急忙跑去禀告尹子奇,于是张巡看清楚了尹子奇容貌。张巡便立即命令南霁云放箭射尹子奇,射中了尹子奇的左眼,差点儿俘虏了他。在这种情况下尹子奇只好收兵回营。





第十九计 釜底抽薪


“釜底抽薪”,“釜”,古代的一种锅。“薪”,柴。本义是把柴火从锅底抽掉。比喻从根本上解决问题。

在古代文献里,有关“釜底抽薪”的记载很多。如《吕氏春秋·数尽》里说:“夫以汤止沸,沸愈不止,去其火则止矣。”意思是说用水来制止沸腾的水,沸腾的水会更加沸腾而停不下来,把锅底的火抽去以后沸腾的水才停止下来。汉代董卓《上何进书》里也云:“臣闻扬汤止沸,莫若去薪。”北齐魏收《为侯景叛移梁朝文》里也说:“抽薪止沸,剪草除根。”古人还说:“故以汤止沸,沸乃不止,诚知其本,则去火而已矣。”这个比喻很通俗,道理也说得非常清楚。水烧开了,再兑水进去是不能马上让水的温度降下来的。因为水的沸腾是靠柴火在釜底烧的原因,要想解决问题,根本的办法是把火从釜底抽掉,水温自然就降了下来。

“釜底抽薪”用在军事上,关键在于抓住“主要矛盾”。对强敌不可用强攻的办法,而应该避其锋芒,削减敌人的气势,再去乘机取胜。在战场上,指挥员要准确判断,抓住那些影响战争全局的关键点,乘机攻击敌人的弱点。比如,如能乘机夺得敌人的粮草辎重,敌军就会不战自乱。三国时的“官渡之战”就是一个有名的战例。





不敌其力 (1) ,而消其势 (2) ,兑下乾上之象 (3) 。

【按语】水沸者,力也,火之力也,阳中之阳也,锐不可当;薪者,火之魄也,即力之势也,阴中之阴也,近而无害,故力不可当而势犹可消。《尉缭子》曰:“气实则斗,气夺则走。”而夺气之法,则在攻心。昔吴汉为大司马,有寇夜攻汉营,军中惊扰,汉坚卧不动。军中闻汉不动,有顷乃定。乃选精兵反击,大破之,此即不直当其力而扑消其势也。宋薛长儒为汉、湖、滑三州通判,驻汉州。州兵数百叛,开营门,谋杀知州、兵马监押,烧营以为乱。有来告者,知州、监押皆不敢出。长儒挺身出营,谕之曰:“汝辈皆有父母妻子,何故做此?叛者立于左,胁从者立于右!”于是,不与谋者数百人立于右,独主谋者十三人突门而出,散于诸村野,寻捕获。时谓非长儒,则一城涂炭矣!此即攻心夺气之用也。或曰:敌与敌对,捣强敌之虚以败其将成之功也。





【注释】


(1) 敌:用作动词,攻打。力:指最坚强的部位。

(2) 消:削弱。势:气势。

(3) 兑下乾上之象:《易经》六十四卦中的《履卦》为“兑下乾上”,上卦为乾为天.下卦为兑为泽。《易经·履卦》:“彖曰:柔履刚也。”兑为阴卦,为柔;乾为阳卦,为刚。兑在下,从循环关系和规律上说,下必冲上,于是出现“柔克刚”之象。此计正是运用此象推理衍生出来的,比喻“以柔克刚”可以战胜强敌。





【译文】


不要直接攻打敌人的主力,只是削弱他的气势,采用“以柔克刚”的办法来制服它。

【按语译文】水的沸腾,是靠力量,是火使它产生这股强大的力量,火的力量是“阳中之阳”,其锐气是无法抵挡的;柴草,是产生火力的根本,是强大力量产生的源泉。这是“阴中之阴”,靠近柴草是没有害处的,凶猛的火力虽然不能抵挡,却可以削弱产生这种力量的源泉。《尉缭子》说:“有勇气时就发起进攻,没勇气时就回避逃跑。”而瓦解敌人气势的办法就是采取攻心战术。东汉时,吴汉被任命为大司马,有一次敌人夜袭军营,营内士兵顿时惊慌措乱,唯独吴汉依然卧床不动。官兵听说吴汉依然卧床不动,不一会儿,军营中情绪就稳定下来。这时,吴汉便挑选精锐勇士连夜反击,把敌人打得大败。这就是不直接抵挡敌人的主力而削弱其锋锐势力的策略。北宋薛长儒做汉、湖、滑三州通判时,驻军在汉州。有数百名守卫的士兵叛变,他们打开营门,图谋杀害知州和兵马监押,烧毁营房,进行叛乱。有人前来禀报,知州、兵马监押吓得都不敢露面。长儒挺身出营来,劝告叛兵说:“你们都有父母妻子,为什么要干这些事情?凡是叛变者站在左边,胁从者站在右边。”于是没有参加谋反的数百人站到了右边,只有主谋的十三个人冲出营门逃跑,分散躲藏到野外的村庄里,但不久又都被捕获归案。当时人们说,若不是有薛长儒在,全城人就要遭殃了。这里用的就是“攻心夺气”的计谋。有人说:当两军对垒时,必须捣毁敌人的虚弱之处,来破坏它即将取得的成功。





第二十计 浑水摸鱼


“浑水摸鱼”,原意是把水搅混,乘鱼儿晕头转向时,乘机摸鱼,可以得到意外的收获。比喻乘混乱的时候从中捞取利益。

此计用于军事,是指当敌人混乱无主时,乘机夺取胜利的谋略。在浑浊的水中,鱼儿辨不清方向,在复杂的战争中,弱小的一方经常会动摇不定,这就有可乘之机。更多的时候,这个可乘之机不能只靠等待,而应主动去制造。即我方应主动去把水搅浑,使情况复杂起来,然后借机行事。

局面混乱不定,一定存在着多种互相冲突的力量,那些弱小的力量这时都在考虑,到底要依靠哪一边,一时难以确定。古代兵书《六韬》中列举了敌军的衰弱征状:全军多次受惊,兵士军心不稳,发牢骚,说泄气话,传递小道消息,谣言不断,不怕法令,不尊重将领……这时,可以说是水已浑了,就应该乘机捞鱼,取得胜利。运用此计的关键,是指挥员一定要正确分析形势,发挥主观能动性,千方百计把水搅浑,主动权就牢牢掌握在自己的手中了。

此计运用《随卦》的象理,是说打仗时要善于抓住敌方的可乘之隙,而我借机行事,便可乱中取利。





乘其阴乱 (1) ,利其弱而无主 (2) 。《随》,以向晦入宴息 (3) 。

【按语】动荡之际,数力冲撞,弱者依违无主,散蔽而不察,我随而取之。《六韬》曰:“三军数惊,士卒不齐,相恐以敌强,相语以不利,耳目相属,妖言不止,众口相惑,不畏法令,不重其将,此弱征也。”是鱼,混战之际,择此而取之。如:刘备之得荆州,取西川,皆此计也。





【注释】


(1) 乘其阴乱:意谓乘敌人内部发生混乱。阴,这里指“内部”。

(2) 利:利用。主:主见。

(3) 《随》,以向晦入宴息:语出《易经·随卦》。《随》,卦名。本卦为异卦相叠(震下兑上)。本卦上卦为兑为泽,下卦为震为雷。言雷入泽中,大地寒凝,万物蛰伏,故如象名“随”。随,顺从。《随卦》的《彖》辞说:“泽中有雷,《随》,君子以向晦入宴息。”《彖》辞的意思是说“震下兑上,雷下泽上,泽中有雷,是《随卦》。君子因此在晚上就入室休息”。晦,太阳下山,天黑了。向晦,就是随着太阳下山而天暗下来。宴息,休息。意谓人要随从天时的变化去作息,到了夜晚就要进入寝室休息。





【译文】


乘敌人内部发生混乱,利用它力量虚弱而没有主见时,使他顺从我。就像人要随从天时的变化去作息,到了夜晚就要进入寝室休息一样。

【按语译文】在动荡不安时,就会有多种力量在互相冲突,其中弱者犹豫不决,毫无主见,他们就会分散躲避而察觉不到,此时我方即可把他们争取过来。《六韬·兵征》里说:“全军多次惊慌,队伍混乱,又因把敌人估计过强而产生畏惧心理,交头接耳地说些不利的话,并且相互挤眉弄眼,谣言不断,蛊惑人心,不怕军令,不尊重将帅,所有这些都是怯弱的征兆啊。”这就像无所适从的“鱼”,应该在混战之时乘机捕捉。比如刘备得荆州、取西川,用的都是这一计谋。





第二十一计 金蝉脱壳


“金蝉脱壳”的本意是:寒蝉(知了)在变为成虫时,就会脱去原来的外壳而走,只留下蝉壳还挂在枝头。比喻只留下表面的假象,实际上已经脱身而逃。比喻用计脱身,使人不能及时发觉。

此计用于军事,是指通过伪装来摆脱敌人,或撤退或转移,以实现我方转移目标的谋略。撤退或转移,绝不是惊慌失措,消极逃跑,而是保留形式,抽走内容,稳住对方,使自己脱离险境,从而用巧妙分兵转移的机会出击敌人。

“金蝉脱壳”是摆脱敌人、转移部队的一种分身术。但这种调动要神不知,鬼不觉,极其隐蔽。因此,一定要把假象造得有逼真的效果。转移时,依然要旗帜招展,战鼓隆隆,好像仍然保持着原来的阵势,这样可以使敌军不敢动,友军不怀疑。檀道济在被敌人围困时,竟然能带着武装士兵,自己穿着显眼的白色服装,坐在车上,不慌不忙地向外围进发。敌军见此,以为檀道济设有伏兵,不敢逼近,让檀道济安然脱离围困。檀道济此计,就是在千钧一发之际,设法留下伪装的假象,以掩人耳目,然后暗中逃走,这种巧妙的脱身方法就是“金蝉脱壳”。





存其形 (1) ,完其势 (2) ;友不疑 (3) ,敌不动 (4) 。巽而止,《蛊》 (5) 。

【按语】共友击敌,坐观其势。倘另有一敌,则须去而存势。则金蝉脱壳者,非徒走也,盖为分身之法也。故大军转动,而旌旗金鼓俨然原阵,使敌不敢动,友不生疑。待己摧他敌而返,而友敌始知,或犹且不知。然则金蝉脱壳者,在对敌之际,而抽精锐以袭别阵也。如:诸葛亮病卒于军,司马懿追焉。姜维令仪反击鸣鼓,若向懿者,懿退,于是仪结营而去。檀道济被围,乃命军士悉甲,身白服乘舆徐出外围。魏惧有伏,不敢逼,乃归。(《南史·檀道济传》、《广名将传》)





【注释】


(1) 存其形:意谓保持阵地已有的战斗阵形。

(2) 完其势:意谓进一步完备继续战斗的各种态势。

(3) 疑:怀疑。

(4) 动:这里指“进犯”。

(5) 巽而止,《蛊》:语出《易经·蛊卦》。《蛊》,卦名。本卦为异卦相叠(巽下艮上)。本卦上卦为艮为山为刚,为阳卦;巽为风为柔,为阴势。故“蛊”的卦象是“刚上柔下”。《彖》曰:“《蛊》,刚上而柔下,巽而止,《蛊》。”意谓高山在上,风行于山下,谦逊而沉静,事可顺当。又,艮在上卦,为静;巽为下卦,为谦逊,故说“谦虚沉静”,“弘大通泰”是天下大治之象。此计引本卦《彖》辞“巽而止,《蛊》”,“蛊”,意谓事顺。其意是我暗中谨慎地实行主力转移,稳住敌人,乘敌不怀疑之际脱离险境,就可安然躲过战乱之危。





【译文】


保持阵地已有的战斗阵形,进一步完备继续战斗的各种态势;使友军不产生怀疑,敌人也不敢贸然进犯。在敌人迷惑不解时,谨慎地实行主力转移,事情就会顺利。

【按语译文】同盟军联合抗击敌人时,首先要静观各方的态势。如果还存在另一股敌人,就要离开这里,但要保持原来阵容的气势。所谓“金蝉脱壳”,并不是一走了之,大盖它是分身的法术。所以,当我军转移主力时,依然要像在原阵地一样,旌旗招展,金鼓喧天,使敌人不敢轻举妄动,友军也对我不生疑心。等到已经消灭了别处的敌人返回来时,友军和敌军才会发觉,或者仍然没有发觉。所谓“金蝉脱壳”就是在对敌作战时,暗中抽调精锐的主力去袭击别处敌人的军阵。例如:三国时诸葛亮北伐,病死军中,魏将司马懿乘机追击。当蜀将姜维命令杨仪把旌旗指向魏军、指向司马懿时,司马懿怕中计而退兵,因此杨仪也才收兵回营。南北朝时檀道济被敌人所困,便命令军士全副武装,自己则穿一身白色服装,悠闲地坐在车子上,缓缓地走出敌人的包围。北魏的军队怕有埋伏,不敢逼近,檀道济才得以收兵回营。





第二十二计 关门捉贼


“关门捉贼”,从字面上理解,是一种围困并歼灭敌人的计谋。是指对弱小的敌军要采取四面包围、聚而歼之的谋略。“关门”就是使敌人出逃的希望破灭,让其无路可退,这是“攻心”战略。一旦把门关起来,敌人的心理将会受到致命的创伤。此外,“关门捉贼”,不仅仅是怕敌人逃走,而是怕它逃走之后被别人所利用。如果门关不紧,敌人脱逃,千万不可轻易追赶,防止中了敌人的诱兵之计。

这里所说的“贼”,是指那些善于偷袭的小部队,它的特点是轻小诡秘,出没不定,行踪难测。它的数量不多,破坏性很大,常会乘我方不备,侵扰我军。所以,对这种“贼”,不可放其逃跑,而要断他的后路,聚而歼之。当然,此计运用得好,决不只限于“小贼”,甚至可以围歼敌主力部队。如果指挥员能统观全局,因势用计,因情变通,捉到的也可能不是小贼,而是敌军的主力部队。这和“关门打狗”、“瓮中捉鳖”有异曲同工之妙。





小敌困之 (1) 。《剥》,不利有攸往 (2) 。

【按语】捉贼而必关门,非恐其逸也,恐其逸而为他人所得也。且逸者不可复追,恐其诱也。贼者,奇兵也,游兵也,所以劳我者也。《吴子》曰:“今使一死贼伏于旷野,千人追之,莫不枭视狼顾。何者?恐其暴起而害己也。是以一人投命,足惧千夫。”追贼者,贼有脱逃之机,势必死斗;若断其去路,则成擒矣。故小敌必困之,不能,则放之可也。





【注释】


(1) 小敌困之:对弱小或者数量较少的敌人,要设法去围困他。

(2) 《剥》,不利有攸往:语出《易经·剥卦》。《剥》,卦名。本卦异卦相叠(坤下艮上),上卦为艮为山,下卦为坤为地。意即广阔无边的大地在吞没山,故名曰“剥”。剥,“落”的意思,万物零落之象。卦辞:“《剥》,不利有彼往”,意谓:《剥卦》说,不利于直追远赶。





【译文】


对弱小或者数量较少的敌人,要设法去围困他。不利于急追远赶。

【按语译文】捉贼时必须要关闭大门,这并不是怕他逃走,是怕他逃走后被别人所得到而利用他。况且,对于已经逃走的贼不可再去追赶,是怕上了贼的诱兵之计。所谓“贼”,就是指那些性情狡猾、善于奇袭、神出鬼没、专门引我疲于奔命的敌人。《吴子兵法》里说:“现在让一个亡命之徒隐藏到广阔的野外,尽管让千百人去追捕他,追捕者没有一个不左顾右盼,顾虑重重。这是为什么呢?是怕贼突然跳出来伤害自己。所以,只要有一个不怕死的,就足以让一千个人害怕。”追击贼的方法:只要贼有脱逃的机会,他就必然会拼死格斗;如果截断他逃脱的道路,贼就必然会被捉住。因此,对付小股敌人,就必须围困他,如果办不到,就暂时让他逃走也是可以的。





第二十三计 远交近攻


“远交近攻”,意思是说和那些远离自己的国家交朋友,攻击或占领与自己邻近的国家。语出《战国策·秦策》。这是战国时期范雎为秦国制定的一条外交策略。后也指待人处世的一种手段。

《秦策》云:“范雎曰:王不如远交而近攻,得寸,则王之寸;得尺,亦王之尺也。”意思是说:大王不如和那些远离自己的国家交朋友,攻击与自己邻近的国家。只要得到一寸土地,你就可以称王一寸土地,得到一尺土地,你就可以称王一尺土地。这是范雎说服秦王的一句名言。

“远交近攻”是分化瓦解敌方联盟、各个击破、结交远离自己的国家而先攻打邻国的谋略。“远交”只是一种外交权术,并不是真的同遥远的国家交好,而是为了避免树敌过多而采用的外交诱骗。当实现军事目标的企图受到地理条件的限制时,应先攻取就近的敌人,而不能越过近敌去打远离自己的敌人。为了防止敌方结盟,要千方百计去分化敌人,各个击破。消灭了近敌之后,“远交”的国家就又成为新的攻击对象。





形禁势格 (1) ,利从近取,害以远隔 (2) 。上火下泽 (3) 。

【按语】混战之局,纵横捭阖之中,各自取利。远不可攻,而可以利相结;近者交之,反使变生肘腑。范雎之谋,为地理之定则,其理甚明。(《战国策·秦策》、《战略考·战国》)





【注释】


(1) 禁:禁止。格:阻碍。“形禁势格”,意谓受到地势的限制和阻碍。

(2) 利从近取,害以远隔:意谓攻取较近的敌人有利,攻取远隔的敌人有害。

(3) 上火下泽:语出《易经·睽卦》。《睽》,卦名。本卦为异卦相叠(兑下离上)。上卦为离为火,下卦为兑为泽。上离下泽,是水火相克,水火相克则又可相生,循环无穷。睽,乖离,乖异。即矛盾。本卦《象》辞:“上火下泽,《睽》。”意谓兑下离上,泽下火上。火焰往上冒,池水往下淌,两相矛盾。此计运用“上火下泽”相互乖离的道理,说明采取“远交近攻”的不同做法,使敌相互矛盾,而我正好各个击破。





【译文】


如果受到地势的限制和阻碍,攻取较近的敌人有利,攻取远隔的敌人有害。这就是从《睽卦》里“上火下泽”中悟出的道理。

【按语译文】在混战的局面中,各种势力也陷于联合与分裂的频繁变换之中,都是为了自己争夺利益。远处不要去进攻,可以用利益与其友好相交;但如果与邻近国家相交好,反而会使变乱发生在自己身边。战国时范雎的谋略(见《战国策·秦策》),就是以地理位置的远近作为结交或攻打的准则,其道理是显而易见的。





第二十四计 假道伐虢


“假道伐虢”,事见《左传·僖公二年》和《左传·僖公五年》。假道,是“借路”的意思。虢,春秋时诸侯国名。本义是指晋国向虞国借道去讨伐虢国。书中说:晋国想吞并邻近的两个小国虞国和虢国,这两个国家之间关系不错。晋如袭虞,虢会出兵救援;晋若攻虢,虞也会出兵相助。大臣荀息向晋献公献上一计,他说:“要想攻占这两个国家,必须要离间他们,使他们互不支持。虞国的国君贪得无厌,我们可以投其所好。”他建议晋献公拿出心爱的屈产的良马和垂棘产的玉璧两件宝物送给虞公。献公哪里舍得,荀息说:“大王放心,只不过让他暂时保管罢了,等灭了虞国,一切都还会回到你的手中。”献公依计而行。虞公得到良马和美璧,高兴得嘴都合不拢。虞公得了晋国的好处,只得答应借道给晋国。晋大军借用虞国的道路去攻打虢国,很快就取得了胜利。晋国的军队回国时,把劫夺的财产分了许多送给虞公。虞公更是大喜过望。这时晋军大将里克装病,称不能带兵回国,请求暂时把部队驻扎在虞国京城附近。虞公毫不怀疑。几天之后,晋献公亲率大军前去,虞公出城相迎。献公约虞公前去打猎。不一会儿,只见京城中起火。虞公赶到城外时,京城已被晋军里应外合强占了。就这样,晋国又轻而易举地灭了虞国。

此计的关键在于“假道”,以“假道”为借口,隐蔽了“假道”的真正意图。后来就以“假道伐虢”来形容“借彼图此,别有用心”的伎俩。





两大之间 (1) ,敌胁以从 (2) ,我假以势 (3) 。《困》,有言不信 (4) 。

【按语】假地用兵之举,非巧言可诳,必其势不受一方之胁从,则将受双方之夹击。如此境况之际,敌必迫之以威,我则诳之以不害,利其幸存之心,速得全势。彼将不能自阵,故不战而灭之矣。如:晋侯假道于虞以伐虢。晋灭虢,虢公丑奔京师。师还,袭虞灭之。(《左传·僖公二年》、《左传·僖公五年》)





【注释】


(1) 两大之间:这里指位于两个大国之间。

(2) 胁:胁迫。从:屈从,服从。

(3) 假:借。势:态势。句意为:处在我与敌两个大国之中的小国,敌方若胁迫小国屈从于他时,我则要借机去援救,造成一种有利于我的军事态势。

(4) 《困》,有言不信:语出《易经·困卦》。《困》,卦名。本卦为异卦相叠(坎下兑上),上卦为兑为泽,为阴;下卦为坎为水,为阳。卦象表明,本该容纳于泽中的水,现在离开泽而向下渗透,以致泽无水而受困,水离开泽流散无归也自困,故名为“困”。困,困乏。卦辞曰:“《困》,有言不信。”《周易姚氏学·困》解释为:“处困之不见信于人,故有言不信。”意思是:“处于困难的境地,不肯轻易听信别人的空话。”此计应用此卦理,是说处在两个大国之间的小国,面临着受人胁迫的境地时,我若说援救他,他在困顿中会相信吗?





【译文】


处于两个大国之间的小国,当受到敌方用武力胁迫小国屈从于他时,我方应以援助的姿态,把力量渗透进去。对于处在困顿中的国家,只有口头许诺而无实际的援助是难以取得他的信任的。

【按语译文】借道行军的举动,不是仅靠花言巧语所能蒙骗取得的,必须是这个中间势力处于不受一方势力的威胁,而是处于双方势力的夹击中。在这种情况下,敌方必然会用武力逼迫他屈服,我方则用不使他受害为诱饵,利用他侥幸图存的心理,迅速控制局势。他势必不能守住阵地,这样不用经过实战就能把他消灭了。例如:春秋时晋国向虞国借道攻打虢国。(见《左传·僖公二年》)晋国消灭了虢国以后,虢公丑逃奔到周朝的京师洛阳。晋军灭了虢国后在回师的途中,再度借道虞国,趁其失去戒心而最后又消灭了虞国。(见《左传·僖公五年》)





第二十五计 偷梁换柱


“偷梁换柱”,语出宋代罗泌《路史·发挥》。《发挥》里说:桀、纣能“倒拽九牛,抚梁易柱”,本义是强调桀、纣力大无穷,可以拽倒九牛,换梁易柱。后作“偷梁换柱”,比喻以玩弄偷换的办法,暗中改变事物的本质和内容,以假代真,以劣代优,以达到蒙混欺骗的目的。

古代作战,军队里的“梁”和“柱”就是精兵主力之所在。一间房的梁、柱被调换,那么房子就会倒塌;一支军队的主力如果被调换,那么这支军队就要垮掉。在对阵时,暗中调动走敌人的主力,分散和削弱敌人的力量,就叫做“偷梁换柱”。此计亦指不断改变自己的主力布置,有效地控制和蒙骗敌人,借以扩大自己的实力,变劣势为优势。

此计与“偷天换日”、“偷龙换凤”、“调包计”都是同样的意思。用在军事上,“偷梁换柱”也是一种军事谋略。以次要的换主要的,以假的换真的,以坏的换好的。这种“偷梁换柱”的方法,就是暗中将对方“好”的东西换掉,其目的是削弱对方的战斗能力。此计中也包含尔虞我诈、乘机控制别人的权术,所以也往往用于政治谋略和外交谋略上。不过,从军事谋略上去理解本计,或“偷梁”,或“换柱”,重点都放在对敌军“频更其阵”上,促使敌人变换阵容,然后伺机攻其弱点。这种调动敌人的谋略,往往也能收到很好的效果。





频更其阵 (1) ,抽其劲旅 (2) ,待其自败,而后乘之 (3) ,曳其轮也 (4) 。

【按语】阵有纵横,天衡为梁,地轴为柱。梁柱以精兵为之,故观其阵,则知其精兵之所在。共战他敌时,频更其阵,暗中抽换其精兵,或竟代其为梁柱,势成阵塌,遂兼其兵。并此敌以击他敌之首策也。





【注释】


(1) 频:频繁,多次。更:更换,变化。因为本计被列入“并战计”中,所以有人认为句中的几个“其”字,均指盟友、盟军言之。是一种如何争取、控制和兼并友军的计策。

(2) 抽:抽掉。

(3) 乘:控制,驾驭。

(4) 曳其轮也:语出《易经·既济卦》。《既济》,卦名,本卦为异卦相叠(离下坎上)。上卦为坎为水,下卦为离为火。水处火上,水势压倒火势,救火之事,大告成功,故卦名“既济”。既,已经;济,成功。本卦初九《象》辞曰:“曳其轮,义无咎也。”曳,拖住。意为拖住了车轮,应该是无害的。





【译文】


多次变动他的阵容,暗中抽换他的主力,等待他自趋失败,然后趁机控制或吞并他。这就像拖住了大车的轮子,大车就不能运行了一样。

【按语译文】阵势有纵有横,阵中有“天横”,首尾相对,是阵的“大梁”;“地轴”在阵中央,是阵的“支柱”。梁和柱的位置都是部署主力部队的地方。因此,察看敌人的阵势,就能知道他的精兵主力所在。如果与友军联合抗敌作战,应设法多次变动友军的阵容,暗中更换它的主力,或者派自己的部队去代替它的梁柱,这样就会使它的阵地无法由它自己控制,这时就立即吞并友军的部队。这就是兼并控制这一股敌人而再去攻击另一股敌人的首要良策。





第二十六计 指桑骂槐


“指桑骂槐”,谓指的是桑树,骂的是槐树,比喻表面上骂这个人,实际上是骂那个人。与指着张三骂李四、指东骂西是一个意思。实际也是“杀鸡儆猴”的方法,即抓住个别坏典型,从严处理,就可以震慑其他人。

《三十六计》中将它演绎成间接训诫部下,以使其敬服的谋略。将帅们为了在下级面前树立自己的威严,多使用“杀鸡儆猴”、“敲山震虎”的权术。有时也指势力强大者要使弱小者屈服,又不想暴露痕迹,于是就委婉地提出警告,是一种间接的指责方法。对于比较强大的对手也可以旁敲侧击地威慑他。如春秋时期,齐相管仲为了降服鲁国和宋国,就运用了此计。他先攻下弱小的遂国,鲁国畏惧,立即谢罪求和;宋见齐、鲁联盟,也只得认输求和。管仲“敲山震虎”,不用大的损失就使鲁、宋两国臣服。





大凌小者 (1) ,警以诱之 (2) 。刚中而应,行险而顺 (3) 。

【按语】率数未服者以对敌,若策之不行,而利诱之,又反启其疑。于是故为自误,责他人之失,以暗警之。警之者,反诱之也,此盖以刚险驱之也。或曰:此遣将之法也。





【注释】


(1) 大凌小者:意谓强大者欺凌控制弱小者。凌,欺凌,欺压。

(2) 警以诱之:意谓用警戒的办法去诱导他。诱,引诱。

(3) 刚中而应,行险而顺:语出《易经·师卦》。《师》,卦名。本卦为异卦相叠(坎下坤上)。本卦下卦为坎为水,上卦为坤为地,水流地下,随势而行。这正如军旅之象,故名为“师”。本卦《彖》辞说:“刚中而应,行险而顺,以此毒天下,而民从之。”毒,“治”的意思。这句话的意思是说:刚健中正而上下相应,行于险地而顺利,这样来治理天下,人们都会听从他的。以此卦象的道理督治天下,百姓就会服从。这是吉祥之象。





【译文】


强大者欺凌控制弱小者,用警戒的办法去诱导他。主帅刚强中正就会上下相应,行于险地也会顺利。

【按语译文】率领一个还没有顺服你的军队去对敌作战,如果你的策略不能执行,这时若用利益来引诱他们,反而会引起他们的怀疑。在这种情况下,你可以故意制造事端,借此惩罚别人发生的过错,借以暗示警告那些不服从命令的人。所谓“警告”,就是从另一个角度来诱导制服他们,这是一种以刚猛险诈的手段驱使他们服从管制的方法。也有人说:这也是一种调兵遣将的方法。





第二十七计 假痴不癫


“假痴不癫”,假,伪装。痴,痴呆,无知。癫,疯癫,癫狂。形容外表看似愚钝,而心里却十分清醒。“假痴不癫”就是假装痴呆,而内心里却特别清醒。此计不论作为政治谋略还是军事谋略,都算高招。

该计在军事上指的是,虽然自己具有相当强大的实力,但故意不露锋芒,显得软弱可欺,痴呆无知,用来麻痹敌人,其实内心非常冷静,然后伺机给敌人以措手不及的打击。

用在政治上,就是韬晦之术。在形势不利于自己的时候,表面上装疯卖傻,给人以碌碌无为的印象,隐藏自己的才能,掩盖内心的抱负,以免引起别人的警觉。其实自己内心非常清楚,等待时机一到,就大显身手来实现自己的抱负。三国时期,曹操与刘备青梅煮酒论英雄的故事,就是个典型的例证。刘备早已有夺取天下的抱负,只是当时力量太弱,根本无法与曹操抗衡,而且还处在曹操控制之下。刘备装作每日只是饮酒种菜,不问世事。一日曹操请他喝酒,席上曹操问刘备谁是天下英雄,刘备列了几个名字,都被曹操否定了。忽然,曹操说道:“天下的英雄,只有我和你两个人!”一句话说得刘备惊慌失措,生怕曹操了解自己的政治抱负,吓得手中的筷子掉在地下。幸好此时一声炸雷,刘备急忙遮掩,说自己被雷声吓掉了筷子。曹操见状,大笑不止,认为刘备连打雷都害怕,成不了大事,对刘备放松了警觉。后来刘备摆脱了曹操的控制,终于在中国历史上干出了一番事业。





宁伪作不知不为 (1) ,不伪作假知妄为 (2) 。静不露机,云雷屯也 (3) 。

【按语】假作不知而实知,假作不为而实不可为,或将有所为。司马懿之假病昏以诛曹爽,受巾帼、假请命以老蜀兵,所以成功;姜维九伐中原,明知不可为而妄为之,则似痴矣,所以破灭。兵书曰:“故善战者之胜也,无智名,无勇功。”当其机未发时,静屯似痴;若假癫,则不但露机,且乱动而群疑。故假痴者胜,假癫者败。或曰:“假痴可以对敌,并可以用兵。”宋代,南俗尚鬼。狄青征侬智高时,大兵始出桂林之南,因佯祝曰:“胜负无以为据。”乃取百钱自持,与神约:“果大捷,则投此钱尽钱面也。”左右谏止:“倘不如意,恐沮师。”青不听。万众方耸视,已而挥手一掷,百钱皆面。于是举兵欢呼,声震林野,青亦大喜;顾左右,取百钉来,即随钱疏密,布地而贴钉之,加以青纱笼护,手自封焉。曰:“俟凯旋,当酬神取钱。”其后平邕州还师,如言取钱,幕府士大夫共祝视,乃两面钱也。(《战略考·宋》)





【注释】


(1) 宁伪作不知不为:宁可假装无知而不行动。伪,假装。不为,不行动。

(2) 不伪作假知妄为:不可以假装知道而去轻举妄动。妄为,轻举妄动。

(3) 静不露机,云雷屯也:语出《易经·屯卦》。《屯》,卦名。本卦为异卦相叠(震下坎上),震为雷,坎为雨,此卦象为雷雨并作,环境险恶,为事困难。屯,难也。《屯卦》的《象》辞又说:“云雷,屯。”坎为雨,又为云,震为雷。这是说云行于上,雷动于下,云在上有压抑雷之象征,这是屯卦之卦象。此计运用屯卦的象理,是说在军事上,有时为了以退求进,必须假痴不癫,老成持重,以达后发制人。这就如同云势压住雷动一样,一旦爆发攻击,便出其不意而获胜。





【译文】


宁可假装不知道而不采取行动,也不可假装知道而轻举妄动。要冷静沉着,藏而不露。这是从《屯卦》象辞“云雷,屯,君主以经纶”一语中悟出的道理。

【按语译文】假装不知而实际上清晰明了,假装不做而实际上是时机不成熟不能做,或是将等待条件具备、时机成熟时才行动。三国时,司马懿假装病,借口神志不清杀死曹爽;他又接到诸葛亮“馈赠”的妇女首饰,他并不为此侮辱而恼怒,而是上表假装请战,却坚壁不出,借以疲劳蜀军,因此获得成功。姜维九次率兵讨伐中原,明知这样做不行而偏要轻举妄动,那就是真的痴了,所以他理所当然地遭到了失败。《孙子兵法》上说:“善于用兵而取得胜利的人,并不显示自己的智谋和争着出名,也不炫耀自己的勇敢与战功。”当他们的计谋还没有机会实行时,他们会像《屯卦》所说的那样,沉着冷静得像个呆子。如果假装癫狂,不仅会暴露战机,而且会因胡乱行动而引起三军猜疑。所以,假装成呆子的必取胜,假装成癫狂的必失败。有人说:“假装成呆子的既可以用来对付敌人,又可以用来治军。”在宋代,南方少数民族有迷信鬼神的风俗。狄青率军征伐侬智高时,大军刚到桂林以南,狄青便假装拜神说:“天神啊!这次打仗胜负难料呀!”说着就拿了一百个铜钱向神许愿:“如果能取胜,请让丢在地上的这些钱都正面朝天吧!”左右随从部将劝他说:“这样做不行啊,倘若这些钱不能都是正面朝天,恐怕会影响士气!”狄青不听从劝说。在千万人的注视之下,他大手一抛,一百个铜钱落在地上,全是面朝上!这时全军举手欢呼,声音响彻山林和旷野,狄青也很兴奋。他命令左右随从取来一百根钉子,根据铜钱散落的疏密,在原地用钉子钉牢,并亲手用青纱盖上来保护,说:“等到取得胜利凯旋时,一定来酬谢神灵,并取回铜钱。”后来,狄青率军平定了邕州凯旋时,按照原先说的那样去取那些钱。他的士兵们都虔诚地蹲在周围祷告观看,却见那些铜钱原来两面都是一样的图案。





第二十八计 上屋抽梯


“上屋抽梯”,原意是诱惑人用梯子爬上屋顶,然后将梯子抽走,使其陷入进退两难的地步,最后屋顶上的人只能束手就擒。比喻以利益诱使对方深入困境后,再彻底地予以消灭,不留后路。和“过桥拆板”、“过河拆桥”、“上树拔梯”的意思相同。

“上屋抽梯”既可用之于敌,也可用之于我。最早见于《孙子兵法》。《孙子兵法·九地》中说:“帅与之期,如登高而去其梯;帅与之深入诸侯之地,而发其机。”这句话的意思是说将帅与士卒一同前往作战,置于有进无退之地,就像登高后抽掉梯子一样,迫使士兵同敌人决一死战。将帅率众深入诸侯国境,要像射出的箭矢一样,使他们只能一往直前,不可返回。

此计用在军事上,是指利用小利引诱敌人,然后截断敌人的后路,以便将敌围歼的谋略。这种诱敌之计,自有其高明之处。敌人一般不是那么容易上当的,所以,你应该先给它安放好“梯子”,也就是故意给以方便。等敌人“上了楼”,也就是进入已布好的“口袋”之后即可拆掉“梯子”,围歼敌人。





假之以便 (1) ,唆之使前 (2) ,断其援应 (3) ,陷之死地 (4) 。遇毒,位不当也 (5) 。

【按语】唆者,利使之也。利使之而不先为之便,或犹且不行。故抽梯之局,须先置梯,或示之以梯。如:慕容垂、姚苌诸人怂秦苻坚侵晋,以乘机自起。(《晋书·苻坚传》)





【注释】


(1) 假:借给。便:方便。

(2) 唆:唆使。这里是指用利去引诱敌人。

(3) 援应:后援接应。

(4) 陷:掉进,坠入。

(5) 遇毒,位不当也:语出《易经·噬嗑卦》。《噬嗑》,卦名。本卦为异卦相叠(震下离上)。上卦为离为火,下卦为震为雷,是既打雷,又闪电,威严得很。又离为阴卦,震为阳卦,是阴阳相济,刚柔相交,用来比喻人要恩威并用,严明结合,故名为“噬嗑”。噬嗑,意为“咀嚼”。本卦六三《象》辞曰:“遇毒,位不当也。”本是说“遇毒”,地位不相称(六三是阴爻,三是阳位,阴处阳位,故不相称),是位不当。此计运用此理,是说敌人受我之唆使,犹如贪食抢吃,只怪他自己见利而受骗,才陷于死地。





【译文】


借给别人一些方便,唆使他们前进,然后切断他们的后援,陷他们于死地。这是从《噬嗑卦》象辞“遇毒,位不当也”一语中悟出的道理。

【按语译文】所谓唆使,就是用利益去驱使引诱敌人。如果只用利益去驱使引诱敌人而不给以方便,那么有的敌人就会犹豫不前。所以,要采用“上屋抽梯”之计,就必须先给敌人设置好梯子,或让敌人知道有方便上屋的梯子。前秦权臣慕容垂、姚苌皆怀二心,他们怂恿苻坚攻晋,苻坚为之心动,大军倾巢而出,结果大败于淝水。慕容垂、姚苌乘机而起,称帝立国。





第二十九计 树上开花


“树上开花”,是指树上本来没有花,但可以用彩色的绸子剪成花朵粘在树上,做得和真花一样,不仔细去看,让人真假难辨。“树上开花”是由“铁树开花”转化而来,原意为很难开花的铁树竟然开起花来了,比喻极难实现的事情。《三十六计》里把它作为制造声势以慑服敌人的一种计谋。铁树也开了花,变不可能为可能,所以能够制服敌人。

用在军事上,是指借别人的声势以壮大自己的军威,以慑服敌人的一种计谋。弱小的部队要善于借助各种因素,改变外部形态,使自己的阵容显得充实强大,就像鸿雁长上了羽毛丰满的翅膀一样。用假花冒充真花,取得以假乱真的效果,巧布迷魂阵,虚张声势,也可以慑服甚至击败敌人。





借局布势,力小势大 (1) 。鸿渐于阿,其羽可用为仪也 (2) 。

【按语】此树本无花,而树则可以有花,剪彩贴之,不细察者不易觉。使花与树交相辉映,而成玲珑全局也。此盖布精兵于友军之阵,完其势以威敌也。





【注释】


(1) 借局布势,力小势大:此句意谓借助某种局面(或手段)布置成有利的阵势,虽然兵力弱小,但阵势却显得很强大。

(2) 鸿渐于阿,其羽可用为仪:语出《易经·渐卦》。《渐》,卦名。本卦为异卦相叠(艮下巽上)。上卦为巽为木,下卦为艮为山。卦象为木植长于山上,不断生长,比喻人要培养自己的品德,进而影响他人。渐,渐进。本卦上九说:“鸿渐于阿,其羽可用为仪,吉利。”此处“阿”字有版本作“陆”。阿,山陵。此句是说鸿雁飞到山头,它的羽毛可以作为文舞的道具,这是吉利之兆。





【译文】


借助某种局面(或手段)布置成有利的阵势,虽然兵力弱小,但阵势却显出强大的样子。鸿雁飞到山头,它的羽毛可以作为文舞的道具,这是吉利之兆。

【按语译文】这棵树上本来没有生长花朵,然而可以人为地使它有花,把彩色绸绢剪成花朵粘在枝上,不仔细察看的人就不容易发觉。让美丽的假花和树交相辉映,就可造成一棵精巧逼真的完美树花。这就是指精锐的兵力布置到友军的阵地上,形成一个完整的阵势以震慑敌人。





第三十计 反客为主


“反客为主”的本义是客人反过来变为主人。一般用来比喻由被动的地位变为主动的地位。与“喧宾夺主”的意思相近。在日常生活中,主客之势常常发生变化,有的变客为主,有的变主为客。关键在于要变被动为主动,争取掌握主动权。

“反客为主”用在军事上,是指在战争中,要努力变被动为主动,争取掌握战争主动权的谋略。尽量想办法钻空子,插脚进去,控制它的首脑机关或者要害部位,抓住有利时机,兼并或者控制他人。在战争中,发动进攻者是“客”,迎战者是“主”。敌人大兵压境,远道而来,我方不妨主动转攻为守,利用有利地形狙击敌人,最终取得胜利。它是一种换位法,或者说是夺位法。

按语称此计为“渐进之阴谋”,既是“阴谋”,就必须“渐进”,只有慢慢地把主动权掌握到自己手中来,才能奏效。





乘隙插足 (1) ,扼其主机 (2) ,渐之进也 (3) 。

【按语】为人驱使者为奴,为人尊处者为客,不能立足者为暂客,能立足者为久客,客久而不能主事者为贱客,能主事则可渐握机要,而为主矣。故反客为主之局:第一步须争客位,第二步须乘隙,第三步须插足,第四足须握机,第五乃成为主。为主,则并人之军矣,此渐进之阴谋也。如李渊书尊李密,密卒以败;汉高视势未敌项羽之先,卑事项羽,使其见信,而渐以侵其势,至垓下一役,一举亡之。(《隋书·李密》、《史记·汉高祖》)





【注释】


(1) 隙:空隙。

(2) 扼:扼制。主机:指要害之处。

(3) 渐之进也:语出《易经·渐卦》。《渐》,卦名,本卦为异卦相叠(艮下巽上)。上卦为巽为木,下卦为艮为山。卦象为木植长于山上,不断生长,比喻人要培养自己的品德,进而影响他人。渐,循序渐进。本卦《彖》辞曰:“《渐》之进也。”意为“渐”就是“渐进”的意思。





【译文】


乘着有漏洞、空隙就赶紧插足进去,扼住它的关键、要害的部分,循序渐进地达到自己的目的。

【按语译文】受他人驱使的人是奴仆;受他人尊重的人是贵客;到人家做客,不能站稳脚跟的是暂时的客人;能够长久立足的是长久的客人;虽然能站住脚跟长期当客人,但不能主事的是地位卑下的客人;能主事并且可以渐渐掌握其机要的人就可以变成主人。所以“反客为主”的演变过程为:第一步须争得客位;第二步须善于发现有利机会;第三步须乘机插足进去;第四步须掌握他的机要;第五部就可以变成主人。做了主人后,就可以兼并别人的军队为自己所有,这是一个循序渐进的阴谋。隋末李渊打天下时,在尚未站稳脚跟之前,曾致书尊崇称霸一方的李密,李密对李渊因此疏于防备,终于被李渊打败。汉高祖刘邦评估自己尚难与项羽抗衡之前,便谦卑地侍奉项羽,取得项羽的信任,然后慢慢地侵吞他的势力,等到垓下一役,一举消灭了项羽。





第三十一计 美人计


“美人计”,就是用美女来诱惑对方,使其沉溺于享乐,失去战斗的意志,继而一举消灭敌人的策略。古代的战例有:西施绝色媚夫差、纣王女色亡国、孙权赔了夫人又折兵。“美人计”最早见于《六韬·文伐》,《文伐》里说:“厚赂珠玉,娱以美人”,“进美女淫声以惑之”。意思是说,用丰厚的珠玉和美女进献给敌国的君主,来讨取他的欢心,诱使敌国的君主过荒淫无度的生活,使其心智迷乱,然后乘有利时机把他消灭掉。这一计策除利用“美人”之外,还可以理解为通过敌人可以信赖的人或事来左右敌人,使敌方斗志涣散,意志消退,由强变弱,从而一举战胜敌军。

原“按语”中把侍奉或讨好强敌的方法分成三种:一种是用献土地的方法,这势必增强了敌人的力量,像六国争相以地事秦一样,并没有什么好结果。一种是用金钱珠宝、绫罗绸缎去讨好敌人,这必然增加了敌人的财富,像宋朝侍奉辽、金那样,也不会有什么成效。独有用美人计这一种才见成效,这样可以消磨涣散敌军将帅的意志,同时也可以增加其部下对他们的怨恨情绪。春秋时期,越王勾践用美女西施和贵重珠宝来取悦吴王夫差,让他贪图享受,丧失警惕,后来越国终于打败了吴国。

此计运用《易经·渐卦》的象理,是说利用敌人自身的弱点,顺势以对,使其自颓自损。





兵强者,攻其将;将智者,伐其情 (1) 。将弱兵颓 (2) ,其势自萎 (3) 。利用御寇,顺相保也 (4) 。

【按语】兵强将智,不可以敌,势必事之。事之以土地,以增其势,如六国之事秦,策之最下者也。事之以币帛,以增其富,如宋之事辽金,策之下者也。惟事之以美人,以佚其志,以弱其体,以增其下之怨。如勾践以西施、重宝取悦夫差,乃可转败为胜。





【注释】


(1) 兵强者,攻其将;将智者,伐其情:此句意谓对兵力强大的敌人,就攻击他的将帅;对明智的将领,就打击他的情绪。

(2) 颓:颓废消沉。

(3) 萎:萎缩,衰退。

(4) 利用御寇,顺相保也:语出《易经·渐卦》。《渐》,卦名,本卦为异卦相叠(艮下巽上)。上卦为巽为木,下卦为艮为山。卦象为木植长于山上,不断生长,比喻人要培养自己的品德,进而影响他人。渐,渐进。本卦九三《象》辞:“利御寇,顺相保也。”是说有利于抵御敌人,能顺利地保卫自己。





【译文】


对于兵力强大的敌人,就攻击他的将帅;对于有智慧的将帅,就打击他的情绪。等到将帅情绪低落、兵士斗志颓废时,敌人的气势自然会萎缩衰退。利用这种方法来抵御敌人,就可以顺利地保存自己。

【按语译文】对于实力强大而将帅又明智的敌人,不能与他们为敌,势必要暂时侍奉、顺从他们。用割让土地的办法来讨好他们,以增强他们的势力,这就如同齐、楚、韩、燕、魏、赵六国侍奉秦国一样,这是最下的策略。用金钱和绸缎来侍奉他们,使他们的财富增加,这就如同宋朝侍奉辽、金一样,这也是下等策略。只有用美女来侍奉他们,来消磨他们的意志,削弱他们的体力,并且以此来增加其部下对他们的抱怨。这就如同越王勾践以美女西施和国内的贵重宝物来取悦吴王夫差一样,这样才能转败为胜。





第三十二计 空城计


“空城计”是一种心理战术,泛指掩饰力量空虚、骗过对方的策略。在己方无力守城的情况下,故意向敌人暴露我城内空虚,这就是所谓“虚者虚之”。这样敌方就会产生怀疑,怕城内有埋伏,就更会犹豫不前,这就是所谓“疑中生疑”。但“空城计”是悬而又悬的“险策”。使用此计的关键,就是要清楚地了解并掌握敌方将帅的心理状况和性格特征。

历史上以“空城计”退兵得胜者不少。最早的是春秋战国时期郑国以“空城计”智退楚军的历史记载,但其影响远远不如罗贯中《三国演义》第九十五回《马谡拒谏失街亭,武侯弹琴退仲达》中诸葛亮巧设空城计、吓退司马懿大军的故事。诸葛亮屯兵于阳平,把部队都派去攻打魏军了,只有少数老弱残兵留在城中。忽然听到魏军大都督司马懿率十五万大军来攻城。诸葛亮临危不惧,传令打开城门,还派人去城门口洒扫。诸葛亮自己则登上城楼,端坐弹琴,态度从容,琴声不乱。司马懿来到城门前,见此情形,心生疑窦,怕城中有伏兵,因此不敢贸然前进,便下令退兵。此故事在民间盛为流传。

虚虚实实,兵无常势,变化无穷。在使用“空城计”时,一定要充分掌握对方主帅的心理和性格特征,切切不可轻易出此险招。诸葛亮之所以敢使用空城计解围,就是他充分了解司马懿谨慎多疑的性格特点,才敢出此险策。此计在多数情况下只能当做缓兵之计,还得防止敌人卷土重来。所以还必须有实力与敌方对抗。解救危局,还是要凭真才实力。





虚者虚之,疑中生疑 (1) ;刚柔之际 (2) ,奇而复奇。

【按语】虚虚实实,兵无常势。虚而示虚,诸葛而后,不乏其人。如吐蕃陷瓜州,王君焕死,河西汹惧。以张守珪为瓜州刺史,领余众,方复筑州城。版榦裁立,敌又暴至。略无守御之具,城中相顾失色,莫有斗志。守珪曰:“彼众我寡,又疮痍之后,不可以矢石相持,须以权道制之。”乃于城上置酒作乐,以会将士。敌疑城中有备,不敢攻而退。又如齐祖珽为北徐州刺史,至州,会有陈寇,百姓多反,珽不关城门。守陴者,皆令下城,静坐街巷,禁断行人,鸡犬不乱鸣吠。贼无所见闻,不测所以,或疑人走城空,不设警备。珽复令大叫,鼓噪聒天,贼大惊,顿时走散。





【注释】


(1) 虚者虚之,疑中生疑:第一个“虚”为名词,意为“空虚”,第二个“虚”为动词,使动,意为“使它空虚”。全句意谓:空虚的就让它空虚,使他在疑惑中更加产生疑惑。

(2) 刚柔之际:语出《易经·解卦》。《解》,卦名。本卦为异卦相叠(坎下震上)。上卦为震为雷,下卦为坎为雨。雷雨交加,荡涤宇内,万象更新,万物萌生,故卦名为“解”。解,解除。本卦初六《象》辞曰:“刚柔之际,义无咎也。”意思是说使刚与柔相互交会,没有灾难。此计运用《解卦》象理,是说敌我交战,运用此计可产生奇妙而又奇妙的功效。





【译文】


空虚的就让它空虚,使他在疑惑中更加产生疑惑。用这种刚与柔相互交会的方法对付刚强的敌人,这是奇法中的奇法。

【按语译文】虚中有实,实中有虚,用兵没有固定不变的模式。本来兵力就空虚,却要把空虚的样子显示给人看,自从诸葛亮以后,运用此计的人为数不少。比如唐玄宗时,吐蕃人攻陷了瓜州,守将王君焕战死,河西的老百姓十分惊慌。这时朝廷任命张守珪为瓜州刺史,他率领一部分民众,重新去修复城墙。筑墙夹板两端的木桩刚刚立好,吐蕃人又来突袭。当时没有防御的武器,城里的人面面相视,不知所措,丧失了战斗的勇气。这时张守珪对大家说:“敌众我寡,战乱的创伤还没有平复,不能用弓箭、雷石等武器去硬抗,必须用智谋来制服敌人。”于是他在城墙上面设置了酒席大宴,和众将领、士兵们作乐。吐蕃人因此怀疑城中有埋伏,不敢进攻,反而撤退了。又比如,北齐祖珽被任命为徐州刺史,刚到任就遇南陈军侵入,当地的老百姓大多数也起来造反。祖珽下令不关城门,并让守城的士兵都到城内去静坐在大街小巷,街道上禁止行人通行。全城顿时寂然无声,鸡鸣狗叫声不乱。陈军也没有探听到城中的情况,也摸不清城中到底是怎么一回事,因此怀疑是人员已都撤离,只剩一座空城,也就没有什么警备。就在敌人疑惑不定之时,祖珽命令城中士兵突然大喊大叫,同时锣鼓喧天,陈军大惊,一会儿就纷纷逃散了。





第三十三计 反间计


所谓“反间”,本来是指利用敌人派来的间谍给敌人反馈回虚假的情报,就是将敌方的间谍反为我所用。后专指用计使敌人内部不团结。在战争中,敌我双方使用间谍是十分常见的现象。

《孙子兵法》就特别强调间谍的作用,认为将帅打仗必须事先了解敌方的情况。要想准确地掌握敌方的情况,不可靠鬼神,不可靠经验,“必取于人,知敌之情者也”。这里的“人”,就是“间谍”。《孙子兵法》里专门有一篇《用间篇》,在《用间篇》中孙武主要论述了使用间谍侦察敌情在作战中的重要意义。文中提出间谍有五种:乡间、内间、反间、死间、生间。如果五种间谍同时起作用,使敌人摸不清我军的行动规律,这就是使用间谍神妙的道理,也是君主克敌制胜的法宝。

唐代杜牧在解释“反间计”时说:“敌有间来窥我,我必先知之,或厚赂诱之,反为我用;或佯为不觉,示以伪情而纵之,则敌人之间,反为我用也。”意思是说如果敌人的间谍前来窥探我军情况,我军必须要事先知道,或者用贿赂来引诱他,使他反为我用;或者假装没有发觉,告诉间谍一些假情报,然后放了他,这样一来,敌人的间谍就能被我利用了。





疑中之疑 (1) 。比之自内,不自失也 (2) 。

【按语】间者,使敌自相疑忌也;反间者,因敌之间而间之也。如燕昭王薨,惠王自为太子时,不快于乐毅。田单乃纵反间曰:“乐毅与燕王有隙,畏诛,欲连兵王齐。齐人未附,故且缓攻即墨,以待其事。齐人唯恐他将来,即墨残矣。”惠王闻之,即使骑劫代将,毅遂奔赵。又如周瑜利用曹操间谍,以间其将,亦疑中之疑之局也。





【注释】


(1) 疑中之疑:在疑阵中再布疑阵。

(2) 比之自内,不自失也:语出《易经·比卦》。《比》,卦名。本卦为异卦相叠(坤下坎上)。本卦上卦为坎为水,下卦为坤为地,水附托于大地,大地容纳着水,此为相依相赖,故名“比”。比,亲比,亲密相依。本卦六二《象》辞曰:“比之自内,不自失也。”意思是说亲近他从内部做起,自己没有失误。此计运用《比卦》象理,是说在布下重重的疑阵之后,能使来自敌方的间谍去误传假情报,这样就不会因有内奸而遭受损失。





【译文】


在疑阵中再布置疑阵,(使敌方安插在我方的间谍因搞不清真实情况而去传递假情报),这样依靠敌人的“内部人”(反间),就不会导致自己的失败。

【按语译文】所谓间谍,就是使敌人内部互相怀疑和猜忌的计谋;所谓反间,就是利用敌人派来的间谍转而离间敌人的计谋。如战国时燕昭王死后,其子惠王在做太子时就和大将乐毅有私仇。齐国大将田单于是乘机用反间计,故意派人到燕国散布谣言说:“乐毅与燕惠王有私仇,怕惠王杀他,所以他想联合齐国军队,称王于齐国。因为齐人还没有投降他,所以他才暂缓攻下即墨,目的是为了等待时机,成就大事。齐国人唯恐燕国改派别的大将来取代乐毅,如果那样,即墨城早就陷落了!”燕惠王听信了谣言,于是就派骑劫为大将,取代了乐毅,乐毅被迫逃亡到赵国。又如三国时东吴大将周瑜曾利用曹操派来的间谍进行反间活动,来挑拨离间他的将领,使曹操斩杀了大将蔡瑁、张允,这同样也是疑局中再设疑局的谋略。





第三十四计 苦肉计


“苦肉计”就是故意伤害自己,利用血和泪来换取敌人的信任,再行反间颠覆敌人的谋略。它的特点,在于利用“人不自害”的常理,做出必要的牺牲,达到欺骗敌人的目的。这种迷惑敌人的手法,是违背人们分析判断事物正常的思维习惯的,因此人们一般不容易看透它的本质。不按“人之常性”行事,就如同水中看侧影一样,使对方得出与事物本质相反的结论。这就是“苦肉计”成功的奥秘。

其实此计是一种特殊作法的“离间计”。运用此计,“自害”是真,但要达到“他害”的目的。己方要造成内部矛盾激化的假象,再派人装作受到迫害,借机钻到敌人心脏中去进行间谍活动。

“苦肉计”的用法有多种多样,目的和形式也不尽相同。“周瑜打黄盖,一个愿打,一个愿挨”,就是其中的一种。两人事先商量好了,假戏真做,自家人打自家人,骗过曹操,诈降成功,打败了曹操八十三万兵马。这种谋略,在近代和现代的间谍战中仍不少见。





人不自害,受害必真 (1) 。假真真假,间以得行 (2) 。童蒙之吉,顺以巽也 (3) 。

【按语】间者,使敌人相疑也;反间者,因敌人之疑,而实其疑也。苦肉计者,盖假作自间以间人也。凡遣与己有隙者以诱敌人,约为响应,或约为共力者,皆苦肉计之类也。如郑武公伐胡而先以女妻胡君,并戮关其思(《韩非子·说难》);韩信下齐而骊生遭烹。





【注释】


(1) 人不自害,受害为真:(正常情况下)人不会自我伤害,一旦自我伤害,别人必然会认为是真情。自害,自我伤害。真,真情。

(2) 假真真假,间以得行:(利用这种常理)我则以假作真,以真作假,那么离间计就可实行了。

(3) 童蒙之吉,顺以巽也:语出《易经·蒙卦》。本卦是异卦相叠(下坎上艮)。本卦上卦为艮为山,下卦为坎为水为险。山下有险,草木丛生,故称“蒙”。这是《蒙卦》的卦象。本卦六五《象》辞曰:“童蒙之吉,顺以巽也。”本意是说幼稚蒙昧之人所以吉利,是因为他柔顺服从。顺以巽,《周易浅述》云:“舍己从人,顺也;降志下求,巽也。”本计用《蒙卦》象理,是说采用这种办法欺骗敌人,就能顺着他的弱点而达到自己的目的。





【译文】


人们一般不会自我伤害,如果真的自我伤害,别人必然会认为是真实情况。我以假作真,以真作假,那么离间计就可实行了。抓住敌人“幼稚朴素”的心理进行欺弄,就能顺着他的弱点达到自己的目的。

【按语译文】使用间谍,就是使敌人内部互相猜疑;使用反间计,就是利用敌人的猜疑心理,使他真正自己怀疑自己。所谓“苦肉计”,就是装作自己内部有矛盾,愿当敌人的间谍,实际是打入敌方内部离间敌人,或乘机进行间谍活动。凡是派遣“与自己有矛盾的人”去诱骗敌人,约定里应外合,或约定协作行动的,都属于“苦肉计”一类。像春秋时期的郑武公想征伐胡国,就先把自己的女儿嫁给胡君,又杀了主张征伐胡国的大臣关其思,使胡国对郑武公毫无戒心,郑国因此消灭了胡国。汉朝时,刘邦派郦食其去劝齐王降汉,使齐王松懈了战备,韩信乘机攻齐,齐王因而烹杀郦食其,郦食其被动地演出了“苦肉计”。





第三十五计 连环计


“连环计”,就是计上加计,计计相连,环环相扣。先用一条计谋使敌人自己钳制自己,再以另一个计谋加以攻击,如此计中生计,使敌人内部互相矛盾,以达到击败敌人的目的。

此计正文的意思是如果敌方力量强大,就不要硬拼,要用计谋使其自相钳制,借以削弱敌方的战斗力。巧妙地运用“连环计”,就像有天神相助一般。三国时,庞统怂恿曹操把战舰用铁链勾连起来,然后纵火焚烧,使之无法逃脱,这就是“连环计”。

原“按语”中列举了庞统和毕再遇两个战例,说明“连环计”是一计累敌,一计攻敌,两计扣用。但是有时也并不见得要看用计的数量,而要重视用计的质量。“使敌自累”之法,可以看做是战略上使敌人自己牵制自己,为我军集中优势、各个击破创造有利条件。这也是“连环计”在谋略思想上的反映。清人揭暄《兵经百言》云:“大凡用计者,非一计之可孤行,必有数计以勷(辅助)之也……故善用兵者,行计务实施,运巧必防损,立谋虑中变。”意思是说,大凡使用计策的人,并不是一计孤行,一定要有好几条计策相互辅助使用。所以善于作战用兵的人,在使用计策时追求的是实用有效果;在使用技巧时,一定要防止对自己有损害;在确定计谋时,一定要考虑到情况的变化,要相应地在计中套计,这样才会使对方防不胜防。





将多兵众,不可以敌,使其自累,以杀其势 (1) 。在师中吉,承天宠也 (2) 。

【按语】庞统使曹操战舰勾连,而后纵火焚之,使不得脱。则连环计者,其结在使敌自累,而后图之。盖一计累敌,一计攻敌,两计扣用,以摧强势也。如宋毕再遇尝引敌与战,且前且却,至于数四。视日已晚,乃以香料煮黑豆,布地上。复前搏战,佯败走。敌乘胜追逐。其马已饥,闻豆香,乃就食,鞭之不前。遇率师反攻,遂大胜。皆连环之计也。(《历代名将用兵方略·宋》)





【注释】


(1) 使其自累,以杀其势:使他们自相钳制,以削弱他们的势力。自累,指自己束缚自己、牵累自己。

(2) 在师中吉,承天宠也:语出《易经·师卦》。《师》,卦名。本卦为异卦相叠(坎下坤上)。本卦下卦为坎为水,上卦为坤为地,水流地下,随势而行。这正如军旅之象,故名为“师”。本卦九二《象》辞曰:“在师中吉,承天宠也。”是说主帅在军中指挥,吉利,因为得到上天的宠爱。此计运用《师卦》象理,是说将帅巧妙地运用此计,就如同有上天保佑一样。





【译文】


敌人将多兵广,不可以和他们硬拼,应设法使他们自相钳制,以削弱他们的势力。主帅在军中指挥,用兵得法,就会像有天神保佑一样。

【按语译文】庞统怂恿曹操用铁链把大小船只统统连接在一起,然后派人纵火焚烧,使之无法逃散。所谓连环计,就是设法使敌人互相牵制,然后再去进攻敌人的计谋。前一计是使敌人自己束缚自己,后一计是配合进攻敌人,两种计谋一环扣一环,灵活运用,可以摧毁强敌的势力。例如,宋代抗金将领毕再遇,曾设计引诱敌人来战。他时进时退,如此三番五次地引诱敌人。见天色已晚,就把用香料煮过的黑豆撒在阵地上,又前去与敌人搏斗。不多时,又假装败走。于是敌人乘胜追击,战马又累又饿,嗅到遍地豆子的香味,只顾争着吃,任凭鞭子抽打也不肯往前跑。这时,毕再遇率领大军进行反攻,大获全胜。这用的也是连环计。





第三十六计 走为上


“走为上”,是指敌我力量悬殊的形势下,为了保存实力,采取有计划的主动撤退,避开强敌,寻找战机,以退为进。这在谋略中也应是一种上策。这句话出自《南齐书·王敬则传》:“檀公三十六策,走为上计。”所谓“三十六策,走为上计”,并不是说“走为上”是“三十六计”中最明智的计策,而是如“按语”中所说的,当情况非常危险的时候,那就应该采取投降、媾和与撤退三条谋略。如果采取投降的手段就是彻底的失败,采取媾和的手段就是失败了一半,采取撤退的手段就不等于失败。没有失败,就会有转胜的机会。相对“投降”和“媾和”的谋略来说,“撤退”(走)就是“上策”了。

但何时走?怎样走?这要随机应变,并非简单地一走了之。撤退绝不是消极逃跑,撤退的目的一是避免与敌人主力决战,二就是为了更进一步的前进。按语中讲的毕再遇用缚羊击鼓蒙蔽金人、从容撤走的故事,就显出毕再遇运用“走为上计”的高超本领。“走”得恰到好处,会让人觉得是聪明的做法。古代有名的战例如:曹操的鸡肋战术、檀道济走避魏军。





全师避敌 (1) 。左次无咎,未失常也 (2) 。

【按语】敌势全胜,我不能战,则必降、必和、必走。降则全败,和则半败,走则未败。未败者,胜之转机也。如宋毕再遇与金人对垒,度金兵至者日众,难与争锋。一夕拔营去,留旗帜于营,豫缚生羊悬之,置其前二足于鼓上,羊不堪倒悬,则足击鼓有声,金人不觉为空营。相持数日,乃觉,欲追之,则已远矣。(《战略考·南宋》)可谓善走者矣!





【注释】


(1) 全师避敌:全军退却,避开强敌。

(2) 左次无咎,未失常也:语出《易经·师卦》。《师》,卦名。本卦为异卦相叠(坎下坤上)。本卦下卦为坎为水,上卦为坤为地,水流地下,随势而行。这正如军旅之象,故名为“师”。本卦六四《象》辞曰:“左次无咎,未失常也。”是说军队在左边扎营,没有危险,也没有违背行军常道。





【译文】


为了保全军事实力,全师退却以避强敌。退在左边扎营,既不会有危险,也没有违背行军常道。

【按语译文】敌人的兵力处在绝对优势的情况下,我方不能与之死拼硬打,那就应该采取投降、媾和与撤退三条谋略。如果采取投降的手段就是彻底的失败,采取媾和的手段就是失败了一半,采取撤退的手段就不等于失败。没有失败,就会有转胜的机会。例如,宋代毕再遇和金兵对抗,因为金兵强大,而且每天来的援兵都很多,难以和金兵抗衡。于是他便在一天傍晚把队伍全部撤走了,只留下旗帜飘扬在营房前,并预先把羊捆吊起来,把羊的前腿放在鼓面上。羊受不了被倒悬,两腿乱蹬,就把鼓敲得咚咚作响。金兵根本没有察觉毕再遇把队伍全部撤走了。就这样相持了好几天,金兵才发觉情况异常,想追赶时,宋兵已经远走高飞了。这可称得上是“走为上”的优秀战例。

\backmatter

\end{document}