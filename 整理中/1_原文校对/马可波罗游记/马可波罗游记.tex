% 马可波罗游记
% 马可波罗游记.tex

\documentclass[12pt,UTF8]{ctexbook}

% 设置纸张信息。
\usepackage[a4paper,twoside]{geometry}
\geometry{
	left=25mm,
	right=25mm,
	bottom=25.4mm,
	bindingoffset=10mm
}

% 设置字体,并解决显示难检字问题。
\xeCJKsetup{AutoFallBack=true}
\setCJKmainfont{SimSun}[BoldFont=SimHei, ItalicFont=KaiTi, FallBack=SimSun-ExtB]

% 目录 chapter 级别加点(.)。
\usepackage{titletoc}
\titlecontents{chapter}[0pt]{\vspace{3mm}\bf\addvspace{2pt}\filright}{\contentspush{\thecontentslabel\hspace{0.8em}}}{}{\titlerule*[8pt]{.}\contentspage}

% 设置 part 和 chapter 标题格式。
\ctexset{
	part/name= {第,卷},
	part/number={\chinese{part}},
	chapter/name={第,篇},
	chapter/number={\chinese{chapter}}
}

% 设置古文原文格式。
\newenvironment{yuanwen}{\bfseries\zihao{4}}

% 设置署名格式。
\newenvironment{shuming}{\hfill\bfseries\zihao{4}}

% 注脚每页重新编号,避免编号过大。
\usepackage[perpage]{footmisc}

\title{\heiti\zihao{0} The Travels of Marco Polo \\马可·波罗游记}
\author{[意大利]\ 马可·波罗}
\date{1298--1299年}

\begin{document}

\maketitle
\tableofcontents

\frontmatter
\chapter{前言}

《马可·波罗游记》是公元十三世纪意大利商人马可·波罗记述他经行地中海、欧亚大陆和游历中国的长篇游记。

马可·波罗是第一个游历中国及亚洲各国的意大利旅行家。他依据在中国十七年的见闻,讲述了令西方世界震惊的一个美丽的神话。这部游记有“世界一大奇书”之称,是人类史上西方人感知东方的第一部著作,它向整个欧洲打开了神秘的东方之门。

作品别名:《东方见闻录》、《马可·波罗行纪》、《寰宇记》

马可·波罗是中西交通史上最早的海陆兼程旅行家。他的著作,在世界史、亚洲史、中西交通史、中意关系史和地理学史等诸方面,都有着重要的历史价值。

《马可·波罗游记》介绍了东方宽广富饶的土地和国家,引起了欧洲人对于东方的向往。学术界的一些有识之士,更以它所提供的最新知识,来丰富自己的头脑和充实自己的著作。比如,1375年(明太祖洪武八年)的西班牙喀塔兰大地图,便是冲破传统观念,摈弃宗教谬说,以《马可·波罗游记》为主要参考书制成的,图中的印度、中亚和远东部分都取材于这部著作,成为中世纪最有价值的地图,也是早期的《世界地图》,以后的地图多以此为依据。

《马可·波罗游记》对十五世纪左右欧洲航海事业的发展,起了巨大的促进作用。例如著名的葡萄牙航海家亨利、意大利航海家哥伦布,都津津有味地读过马可·波罗的书。哥伦布小时读了《马可·波罗游记》,非常羡慕中国、印度的文明富裕,特别是对书中所载日本盛产黄金,“其数无限”、“地铺金砖”,更是向往已极……哥伦布写的旅行记序文中,也谈了他到中国去的动机,是要去找《马可·波罗游记》所写的大汗国……

哥伦布死后不到五十年,欧洲人开始了全球性探访,追本溯源,不能不说是受《马可·波罗游记》的影响。以一部书而引起中西交通史上如此多的探访,是罕见的。

《马可·波罗行纪》共分四卷,第一卷记载了马可·波罗诸人东游沿途见闻,直至上都止。第二卷记载了蒙古大汗忽必烈及其宫殿,都城,朝廷,政府,节庆,游猎等事;自大都南行至杭州,福州,泉州及东地沿岸及诸海诸洲等事;第三卷记载日本、越南、东印度、南印度、印度洋沿岸及诸岛屿,非洲东部,第四卷记君临亚洲之成吉思汗后裔诸鞑靼宗王的战争和亚洲北部。每卷分章,每章叙述一地的情况或一件史事,共有229章。书中记述的国家,城市的地名达100多个,而这些地方的情况,综合起来,有山川地形,物产,气候,商贾贸易,居民,宗教信仰,风俗习惯等,及至国家的琐闻佚事,朝章国故,也时时夹见其中。

马可波罗的这本书是一部关于亚洲的游记,它记录了中亚,西亚,东南亚等地区的许多国家的情况,而其重点部分则是关于中国的叙述,马可波罗在中国停留的时间最长,他的足迹所至,遍及西北,华北,西南和华东等地区。他在《游记》中以大量的篇章,热情洋溢的语言,记述了中国无穷无尽的财富,巨大的商业城市,不错的交通设施,以及华丽的宫殿建筑。以叙述中国为主的《游记》第二卷共82章,在全书中分量很大。在这卷中有很多篇幅是关于忽必烈和北京的描述。

书的开头,对当时人们十分惊奇的事物作了介绍 [2]:“皇帝、国王、公爵、侯爵、伯爵、骑士和市民们,以及其他所有的人们,不论是谁,如果你们希望了解人类各种族的不同,了解世界各地区的差异,请读一读或听人念这本书吧!你们将发现,在这本书中,正如梅塞·马可·波罗所叙述的那样,我们条理分明地记下了东方各大地区——大亚美尼亚、波斯、鞑靼地方、印度以及其他许多国家——的所有伟大而又奇特的事物。马可·波罗是威尼斯市民,聪明而又高贵,被称为‘百万先生’。他亲眼目睹了这些事情。……所有读或听人念这本书的人,都应置信不疑,因为这里所记叙的一切都是真实的。的确,自上帝用他的手创造了我们的祖先亚当以来,直到今在,从未有过任何人,基督教徒或异教徒,鞑靼人或印度人,以及其他种族的人,像这位海塞·马可那样,知道并考察过世界各地如此众多、如此伟大的奇闻轶事。”

马可叙述的故事,确实和这一介绍所说的那样激动人心。他讲到了带有花园和人造湖的大汗宫廷,装载银挽具和宝石的大象。他还讲到了各条大道,高于周围地面,易于排水;大运河上,商人船只每年川流不息;各个港口,停泊着比欧洲人所知道的还要大的船只,并谈到了生产香料、丝绸、生姜、糖、樟脑、棉花、盐、藏红花、檀香木和瓷器的一些地方。马可还描写了他护送中国公主到波斯去时,访问和听说过的所有寓言般的国度——新加坡、爪哇、苏门答腊、锡兰、印度、索科特拉岛、马达加斯加、阿拉伯半岛、桑给巴尔和阿比西尼亚。

书中的一切仿佛离奇古怪,言过其实,因此,人们给他起了个别号叫“百万先生”,因为“他开口闭口总是说百万这个、百万那个”。其实,他向16世纪中叶的欧洲人提供了有关中国最为全面可靠的资料。这本书题名为《世界见闻录》并非偶然。实际上,这部著作使西方人对世界的了解范围突然扩大了一倍。马可·波罗正如两个世纪后的哥伦布一样,为同时代人开辟了崭新的天地。的确,正是他所描写的有关中国和香料群岛的迷人景象,召唤着伟大的探险者们,在穆斯林封锁陆上道路之后,直接寻找一条海上航线,继续前进。

马可·波罗小时候,他的父亲和叔叔到东方经商,来到元大都(北京)并朝见过蒙古帝国的忽必烈大汗,还带回了大汗给罗马教皇的信。

1271年,马可·波罗17岁时,父亲和叔叔拿着教皇的复信和礼品,带领马可·波罗与十几位旅伴一起向东方进发了。他们从威尼斯进入地中海,然后横渡黑海,经过两河流域来到中东古城巴格达,改走陆路。这是一条充满艰难险阻的路,是让最有雄心的旅行家也望而却步的路。他们从霍尔木兹向东,越过荒凉恐怖的伊朗沙漠(今卢特沙漠),跨过险峻寒冷的帕米尔高原,一路上跋山涉水,克服了疾病、饥渴的困扰,躲开了强盗、猛兽的侵袭,终于来到了中国新疆。

一到这里,马可·波罗的眼睛便被吸引住了。美丽繁华的喀什、盛产美玉的和田,还有处处花香扑鼻的果园,马可他们继续向东,穿过塔克拉玛干沙漠,来到古城敦煌,瞻仰了举世闻名的佛像雕刻和壁画。接着,他们经玉门关见到了万里长城。最后穿过河西走廊,终于到达了上都——元朝的北部都城。这时已是1275年的夏天,距他们离开祖国已经过了四个寒暑了!

马可·波罗的父亲和叔叔向忽必烈大汗呈上了教皇的信件和礼物,并向大汗介绍了马可·波罗。大汗留他们在元朝当官任职。聪明的马可·波罗很快就学会了蒙古语和汉语。他借奉大汗之命巡视各地的机会,走遍了中国的山山水水,中国的辽阔与富有让他惊呆了。他先后到过新疆、甘肃、内蒙古、山西、陕西、四川、云南、山东、江苏、浙江、福建以及北京等地,还出使过越南、缅甸、苏门答腊。他每到一处,总要详细地考察当地的风俗、地理、人情。在回到大都后,又详细地向忽必烈大汗进行了汇报。在《马可·波罗游记》中,他盛赞了中国的繁盛昌明;发达的工商业、繁华热闹的市集、华美廉价的丝绸锦缎、宠伟壮观的都城、完善方便的驿道交通、普遍流通的纸币等等。书中的内容,使每一个读过这本书的人都无限神往。17年很快就过去了,马可·波罗越来越想家。

1292年春天,马可·波罗和父亲、叔叔受忽必烈大汗委托,护送一位蒙古公主到波斯成婚。他们趁机向大汗提出回国的请求。大汗答应他们,在完成使命后,可以转路回国。1295年末,他们三人终于回到了阔别二十四载的亲人身边。他们从中国回来的消息迅速传遍了整个威尼斯,他们的见闻引起了人们的极大兴趣。他们从东方带回的无数奇珍异宝,一夜之间使他们成了威尼斯的巨富。

1298年,马可·波罗参加了威尼斯与热那亚的战争,9月7日不幸被俘。在狱中他遇到了作家鲁斯蒂谦,于是便有了马可·波罗口述、鲁斯蒂谦记录的《马可·波罗游记》。

艺术特色
《马可·波罗游记》有很多不同的抄本流传至今,但都不是原稿,尽管可以肯定,笔录的原稿是用法文写成的,可现存抄本却差不多囊括了西欧的所有语种,其中最重要的是那些法文、意大利方言和拉丁文抄本。现存的抄本都不完全,因此,学术界一直致力于推断哪些材料是原稿中就有的,哪些是后来的抄录者擅自增添的。在辨别哪些是马可·波罗的原话,哪些是抄录者添加的细节方面,学术界也同样花费了很大的气力。中文史料在一定程度上对解决这些问题是有所帮助的。其次,还有一个棘手的问题,即如何把马可·波罗所引用的人名和地名与当今的名称对应起来,表音法不规则,加之用不同体系的字母直译.以及其它一些变化(例如君士坦丁堡改名为伊斯坦布尔),这些都是造成这个问题的原因。由于原文中存在这些问题,所以,对马可·波罗所叙述的故事的分析和评价,只能说是一种尝试。

然而,有一点与其它方面相比,争议要少得多,即《马可·波罗游记》在文学上的地位。毫无疑问,马可·波罗的叙述理所当然地是中世纪和文艺复兴时期历史游记文学的主要作品之一。蒙默思的杰弗里所作的《不列颠国王史》,擂安维尔的《圣路易史》,傅华萨的《闻见录》,曼德维尔的《旅行记》和理查德·哈克卢特的《航海记》,同马可·波罗的《游记》一起形成了这一文学传统大厦的上部结构。所有这些作品以及其它一些相似的作品具有某些共同的特点,即它们中的多数都反映了时代的观念(与现代观念是不同的);将事实同想象混在一起;某种文化上的坦率态度;以及对超自然现相当普遍的轻信。因而,现代的读者对马可·波罗所讲述的事情在某种程度上持怀疑态度是合情合理和无可非议的。

这部游记是专为对东方了解甚少的西方读者而作的,这一点是它的优越之处。所以,波罗的最大兴趣是把西方应该了解的东方作为某种有趣的事物来描述。他的动机基本上是商业性的。然而,由于波罗只同君主们来往,他所获得的信息便带有局限性。而且,他的判断在很大程度上受到商业和宗教因家的左右,很显然,他对社会政治问题毫不关心,因为他的兴趣在于贸易和商品,而非思想观念。由于他对发现一条通往东方的安全航路非常热心,以至使他所叙述的亲眼所见事物的可信程度受到了影响。尽管如此,《马可·波罗游记》叙述的大部分内容,仍旧表明他不但对西方中世纪的观念,‘而且也对东方同代的情况有罕见的洞察力。

作品争议

从《马可·波罗行纪》一书问世以来,700年来关于他的争议就没有停止过,一直不断有人怀疑:他是否到过中国?《游记》是否伪作?并形成了马可·波罗学的两种观点:怀疑论者和肯定论者。
早在马可·波罗活着的时候,由于书中充满了人所未知的奇闻异事,《行纪》遭到人们的怀疑和讽刺。关心他的朋友甚至在他临终前劝他把书中背离事实的叙述删掉。之后,随着地理大发现,欧洲人对东方的知识越来越丰富,《游记》中讲的许多事物逐渐被证实,不再被目为荒诞不经的神话了。但还有人对《行纪》的真实性发生怀疑。
直到19世纪初,学术界开始有人站在学者的角度批判此书,并质疑马可·波罗。德国学者徐而曼是最早提出马可-波罗根本没有到过中国的论证,认为所谓他在元朝17年的历史完全是荒诞的捏造,为游记而编排拙劣的教会传奇故事,是为了传教士和商人的利益,借以激发感化蒙古人的热情以便到中国通商而创作的。并且说,波罗一家最远不过到达大布哈里亚(Bucharia)境内,关于蒙古帝国的情况是从曾到过该地的商人们口中听来的;关于印度、波斯、阿拉伯及埃塞俄比亚的叙述则抄自阿拉伯著作。1965年,德国汉史学家福赫伯则列举了许多疑点,如扬州做官、襄阳献炮等疑点加以印证。1979年,美国学者J.W.Haeger(海格尔)翻检《马可·波罗行纪》全文,撰成《马可·波罗到过中国吗—从内证中看到问题》一文提出质疑。1982年,英国《泰晤士报》发表了英国学者C.Clunas(克雷格·克鲁纳斯)《探险家的足迹》的一文,提出四条疑问对波罗到过中国一说提出质疑。与此同时,中国国内学者也有不少人质疑马可·波罗。不过都是写些短文或在其他论文中附带提及。1995年,英国学者Frances Wood(吴芳思)博士经过多年研究,把所有的疑问写成了一本书《马可·波罗到过中国吗?》,从而成为“怀疑论者”的代表。
国内“肯定论者”以杨志玖先生为代表。他从40年代起就不断地同国内外的“怀疑论者”进行论战。国外许多学者也认为或承认马可·波罗曾到过中国以及《马可·波罗行纪》的真实性。其代表性人物是德国的傅海波、英国的亨利·玉尔和法国的伯希和。在19世纪90年代,英国的马可·波罗研究专家亨利·玉尔在他的《马可·波罗游记—导言》中一一列举了《马可·波罗游记》中存在的缺陷和失误。他认为《马可·波罗游记》中对中国的记载有多处缺陷,如根本没有记载任何关于长城、茶叶、用鹭鸶捕鱼、人工孵卵、印刷书籍、中国汉字及其它奇技巧术和怪异风俗等等,还有许多不确定的地方,如中国的地名多用鞑靼语或波斯语、记载成吉思汗死事及其子孙世系关系失误、攻陷襄阳城等等。但是他没有怀疑过马可·波罗到过中国这一事实。德国的马可·波罗研究专家傅海波曾经说过,“不管怎样,在没有举出确凿证据证明马可·波罗的书只是一部世界地理志,其中有关中国的几章是取自其它的、也许是波斯的资料(他用了一些波斯词汇)以前,我们只好作善意解释,假定(姑且认为)他还是到过中国。”法国的东方学家伯希和虽然花费了很多时间为《马可·波罗游记》作了大量的注释,但对马可·波罗书中的疏失也是表示谅解的,基本承认马可·波罗到过中国。
质疑与辩驳
面对质疑,值得奇怪的是中国史学家们的态度耐人寻味。国学大师钱穆的回答妙趣横生,或许代表了他们普遍的想法和观点。他说他“宁愿”相信他真的到过中国,因为他对马可-波罗怀有一种“温情的敬意”。因此真正对上述质疑“奋起”进行说明和批驳的,也仅有杨志玖教授等寥寥数人。双方利用报纸、电视、著作、学术研讨会等各种手法进行了上百年激烈争辩。双方质疑与辩驳集中为四点。其他质疑和辩驳都是在以下几点基础上的扩展:
疑点一
自称深受大汗信任,还担任过官职。但所有东方史籍并没有一条关于马可·波罗的记载。
疑马者:在中国古代浩如烟海的史籍中,无数学者查阅数十年,没有找到一件可供考证的关于记载马可·波罗的史料。
挺马者:为反驳史书中没有关于马可·波罗记载的疑问,杨教授皓首穷经,终于在永乐大典残片《站赤》中找到一条记载“兀鲁得、阿必失和火者取道马二八往阿鲁浑大王位下”与《马可·波罗游记》中记载一致,从而可以证明他是到过中国的。
(但是这是所有史料中仅有的一条孤证。并且《马可·波罗游记》说蒙古忽必烈汗因他识海路,让他带领这个使团,把蒙古的公主护送去波斯完婚,而且这条孤证中并没有提到马可·波罗的名字。)
疑马者:“然而用明朝的史书证实元朝的人物明显让人难以信服,更重要的是:这条记载中和其他史书一样没有提到马可·波罗的名字,只能说明此事与他的叙述一致,不过是他讲述尽人皆知的故事(如远征日本和王著行刺)罢了,而不能证明马可-波罗与此事有任何联系,更不能作为他到过中国的直接证据。”(注:北京青年报2004-08-13)
挺马者:杨志玖先生几乎在他的所有批驳性论文中都提到这条“确凿证据”,用它来作为批驳“怀疑论者”的致命武器。这段公文虽然一个字都没有提到马可·波罗,但至少能够说明《马可·波罗游记》所记载的关于他们随从波斯使臣离华回国的内容有着一致的地方。学者们根据这条材料后来还推断出马可·波罗他们由中国泉州从海道回国的具体时间在1291年初。对于这一条材料,学术界一致认为这是已知在汉文文献中发现的惟一有关马可·波罗的间接记录。这也是国内外“肯定论”学者惟一感到欣慰的地方,至少可以用这条材料来抵挡一下“怀疑论”学者的穷追猛打了。
疑马者:马可·波罗自称在中国17年深受忽必烈器重,但没有任何一本元朝史书能找到哪怕一条可供考证的记录。包括他自称扬州做官三年,扬州地方志里同样无从考稽。关于马可·波罗自称在扬州做总管三年的谎言,史书和扬州地方志都没有记载。
挺马者:马可·波罗其时也许只是一个管理盐务的小官,因为他在游记中写到了产盐区长卢、海门和真州,关于盐务的小官是不会记入史籍的。
疑马者:马可·波罗是色目人,色目人作为元朝的贵族阶级“二等人”,他也自称在扬州地方上担任总管。即使不担任要职,可扬州地方志中明确记载了元代大小官员,包括外国人的详尽名单,仍然没有找到他的记录。
挺马者:后人将马可·波罗的原话“奉大汗命‘居住’扬州三年”误抄成了“奉大汗命‘治理’扬州三年”造成了误会。
疑马者:仅仅“居住”扬州为何要“奉大汗命”?他“居住”扬州三年做什么?既然不是做官,那他住在那只可能是两个原因:其一,他喜爱扬州;其二,是肩负了元廷的特殊使命。既然如此,为什么他三年对繁华扬州的印象仅仅是除了出产马饰外“没有什么值得一提的”?
挺马者:作为一个商人,他对马饰有特别兴趣,至于其他,是属于记载疏忽。
疑马者:把自己待在扬州的目的也忘了?
疑点二
典型中国特色的事物在书中只字未提
疑马者:而同一时期的波斯商人的游记,以及1792年英国马噶尔尼访华回国后游记,都有记载各种中国奇特事物。
1、关于长城
挺马者:没有提到长城,是因为元长城已经年久失修破败不堪,况且元长城土木结构并非明长城砖石结构那样引人注目,没有引起他的足够重视。
疑马者:金人修建的金长城(也叫金界壕)受战乱损坏并不严重。如果马可·波罗真的游遍中国,必然要数次经过长城,不可能视而不见。而同时代的元名臣张德辉曾记载“北上漠北途中,有长城颓址,望之绵延不绝”;王恽写道:“恒州西南十里外有北界壕,尚宛然也”。
2、关于茶叶
挺马者:没有提到茶叶是因为蒙古人不喜饮茶,因此马可·波罗对此也无印象。
疑马者:游牧民族以奶肉为主食,从来没有不嗜茶的蒙古人。所以中原历代王朝都把茶叶视为牵制少数民族的重要战略物资,严禁擅自出口给少数民族,以至明太祖还为此杀了一个驸马。而忽必烈自己也于1268年开始榷买蜀茶,1275年逐渐榷江南各地之茶,1276年设立常湖等处茶园都提司“采摘茶芽,以供内府”;而8~9世纪西域商人苏来曼所写的《中国印度见闻录》则明确提到了茶。中原人好茶也是天性,马可·波罗如果真的在中国生活过十几年还做过官,不可能没有接待过汉人,难道这些汉人都是喝咖啡的?
3、关于汉字与印刷术
挺马者:关于汉字书法和印刷术,马可·波罗不认识汉字,故对中国汉字书法和印刷术不会做记载。
疑马者:十几年在中国,不认识汉字,还没见过汉字吗?所有汉字出版物整齐划一、一模一样的字体看不出来?而且当马可·波罗写书的时候,欧洲处于手抄书的年代,他当时费劲巴力的找人抄书出版,为什么不用印刷术这种方便、准确而廉价的方式出书?而比他早30年到蒙古的法国传教士鲁不鲁乞《东游记》却记载了中国的书法和印刷术。
4、其他中华民俗事物
挺马者:对于中医、筷子、缠足、鸬鹚等等,17年来,马可·波罗只用刀叉,没见过筷子;他是城里人,而不是渔民等等。
疑马者:其他外国人如14世纪英国旅行家曼德维尔在《爵士游记》中、1862年退役的英国亨利·裕尔上校在《中国和通向中国之路》里,对这些事物都有提及。况且,就算真的没见过这些事情,老马一个番邦人,十几年在中原都不会水土不服、不生病的?难道没有一个中医给他看过病?
挺马者:《马可·波罗行纪》也许有过记载,但它成书后经过无数人传抄,也许是传抄中的失误,或者原稿散失。
疑马者:为什么不印刷?
疑点三
捏造了一系列史实。
1、襄阳献炮
疑马者:马可·波罗自称蒙军久攻襄阳不下,于是他献出了威力巨大的抛石机,迫使襄阳守将出降。事实是1273年蒙军攻襄阳时,他还在来中国的路上,而献抛石机的自然也不是他,而是波斯的亦思马因和阿老瓦丁,《元史·方伎传》:“亦思马因,回回氏,西域旭烈人也。善造炮。……天历三年以疾卒。”白纸黑字,写得十分清楚。
挺马者:亦思马因会不会就是马可·波罗?
疑马者:“亦思马因”,这是一个典型的穆斯林名字,作为天主教徒的马克·波罗不会用这个名字。而且《元史》上说得很清楚,亦思马因在元世祖至元十一年,亦即公元1274年就去世了(“以疾卒”)。亦思马因的儿子布伯这个名字听起来倒是与“马可”相近,但此人也不能与马可·波罗画等号,因为他卒于天历三年。“天历”是元文宗图贴睦耳的年号,天历三年是公元1330年,马可·波罗则卒于公元1324年。更能说明他们之间没有任何关系的一点证据是:马可·波罗死在了欧洲,而布伯却卒于中国。
挺马者:这是后人传抄《游记》时随意添加的。因为马可·波罗的手稿已经流失了。
疑马者:为什么不印刷?
2、宗教问题
疑马者:《游记》中记载了镇江的基督教堂和一些“可疑的”、“战战兢兢、躲躲闪闪”的基督徒。而元朝迫害基督徒找不到任何一本史料证明。
挺马者:他们其实是摩尼教徒,因为该教被认为是邪教,不敢公开活动,这一记载翔实生动,没到过中国是不可能掌握这样的第一手资料的,这正说明了马可·波罗到过中国。
疑马者:元朝把摩尼教看作邪教还是元末的事,蒙元初期政府对各种宗教采取优容的态度,对各教一视同仁。对各教的分歧提倡以辩论的方式分优劣,更不采取灭教这样极端的迫害措施。忽必烈曾说:“世上常奉预言人,我都致敬礼。”因此即使是马可·波罗时代的摩尼教徒在一个宽松的宗教氛围中,根本用不着“战战兢兢、躲躲闪闪”。
3、其他
疑马者:蒙古大汗铁木真与忽必烈的名号,早已随着蒙古铁骑远征世界的脚步,在欧洲家喻户晓了,谁谁都能把蒙古大帝国的各种轶事说个三三两两。而《游记》中道听途说、尽人皆知的特大新闻和错误比比皆是,李松寿之乱的时间整整推后十年;把成吉思汗的病死说成是膝上中箭而死;将传说中的非洲的祭司王约翰嫁接成为忽必烈外孙阔里吉思的祖父,而记载的脱脱被那海打败的事居然发生在他回国之后,
挺马者:书中所有的猜测、臆断和错误都是在传抄中后人的过失。
疑马者:老问题,为什么要手抄?为什么不印刷?
疑点四
《马可·波罗游记》中记载的中国地名用的都是波斯词汇。
疑马者:马可·波罗自称懂蒙古语和汉语,虽然是鲁斯蒂谦用法文写成此书,但书中关于中原地区的地名却仍然只用了波斯词汇,我们注意到,当时来往的商人们以波斯人居多,可以证明游记内容是听来的。
挺马者:马可·波罗是色目人。他在元朝打交道也是色目贵族。他本人也是属于色目贵族。所以他不知道中国的很多名称的汉文叫法,只知道波斯语叫法。
疑马者:可笑到不值一驳。大汗的圣旨、元廷的公文,江南的汉民,难道都用波斯文?
疑点五
马可·波罗对元帝国内部的重大事件的认知一塌糊涂
1、1268年-1306年,蒙古帝国西亚领地爆发海都之乱,而一路穿越了战斗最激烈的交战区来到元廷的马可波罗,在《游记》中对于这场持续了38年、惊动了忽必烈汗御驾亲征、直到大汗去世都还没有结束的叛乱,竟然只字未提。
2、忽必烈在太子真金薨逝之后,并没有立即指定新的继承人,直到1293年,大汗才册立真金的三儿子铁穆耳为皇太孙(即元成宗)。马可波罗于1292年离开中国,但在《游记》却一口咬定铁穆耳是真金仅有的儿子、蒙古帝国惟一的皇位候选人。他是怎么在元成宗被册立的前一年就知道这件事的?
3、《元史》记载忽必烈汗的四位妻子共生育了16个儿女,但《游记》却说大汗仅儿子就有22人。
4、马可·波罗自称旅居中国多年,游历中国大部,十分了解中国,但他却对元朝的行省制度毫无了解。只是单纯地把北方称为“契丹省”,把江南称为“蛮子省”(甚至“契丹”、“蛮子”这两个称呼也是使用的波斯语)。在具体描述中,马可·波罗完全无法理解中国层级分明的地方行政区划,将“某城市”、“某地区”称为“某王国”的错误比比皆是。
疑点六
其他
1、书中几乎很少提到马可·波罗的父亲和叔父,也从未提到过他们的生意,没有提到过在中国符合他们身份的任何经商活动,恰恰说明他们没有到过中国,所以经商也无从谈起。
2、马可·波罗回国时没有携带任何中国特有的东西,威尼斯珍宝馆收藏的马可·波罗罐,其实是十四世纪的德化白瓷,与他毫无关系,而他带回的一些宝石倒是波斯的特产。
3、书中的叙述描写充满了夸张失实的情节、信口妄说的逸事,其中许多地方即使现今看来也是非常夸张而令人吃惊的。《游记》中对中原的描述极尽夸张之能事,无论是描述人口、城市还是财富,马可·波罗动辄使用“百万”这个词,以至于人们送他“百万先生”的外号。以至于如今,在马可·波罗的家乡,“马可·波罗”已经成为一句谚语,意为“大忽悠”。
4、1999年美国组成一个科学考察队,他们使用现代交通工具重走了一遍马可波罗来中国的路线,然而旅程同样十分艰辛。考察结束后,10位考察队员和22位提供后援的专家们一致认为,马可·波罗通过这条路来中国“简直是难以想象的”。全程网上直播,一万名对马可·波罗深信不疑的网民看过直播后举行投票,65%的观众投票认为他根本没有到过中国。
是否到过中国
马可·波罗与他的故事《马可·波罗行纪》,早就已经家喻户晓、妇孺皆知了。国内外“肯定论者”之所以肯定或承认马可·波罗到过中国,主要基于两个方面的理由:一是人们对马可·波罗与《马可·波罗游记》的善意解释。另一方面的原因是《马可·波罗游记》所记载的某些内容若非亲身经历是不可能知道得那样详细具体的。许多学者认为《马可·波罗游记》的内容都是在重述一些尽人皆知的故事,比如元朝的远征日本、王著叛乱、襄阳回回炮、波斯使臣护送阔阔真公主等。但是,《马可·波罗行纪》所记载的某些内容却使学者们很惊奇。比如《马可·波罗游记》关于杭州的记载说,杭州当时称行在,是世界上最美的城市,商业兴隆,有12种行业,每种行业有12000户。城中有一个大湖(即西湖),周围达30英里,风景优美。这些记载在《乾道临安志》和《梦梁录》等古籍中得到了印证。其它的如苏州的桥很多,杭州的人多,还有卢沟桥等等。《马可·波罗游记》的记载都相当地详细、具体,这些材料在当时的历史背景下是不可能从道听途说中得到的,
然后,国际马可·波罗学却形成了两种相互对立的学派,即肯定马可·波罗到过中国的“肯定论者”和怀疑马可·波罗到过中国的“怀疑论者”。两方激烈争辩。其实这种争辩意义不大。提起哥伦布,可能几乎所有的人都知道他发现了美洲。不过,这只是我们后人的想法。就哥伦布本人来说,他至死都认为他发现的不是美洲,而是印度,所以,他坚持称他航海中于美洲附近所见到的第一片陆地为“西印度群岛”。哥伦布不是个骗子,马可·波罗也不是有意要撒谎,他可能像克鲁纳斯所认为的那样,只到过中亚的某些国家,而他则把这些国家当成了中国。
解决马可·波罗问题的出路在哪里呢?关键的一点就是不能只躺在《马可·波罗行纪》上去研究所谓的“马可·波罗学”。国内外学者们都承认《马可·波罗行纪》在开拓东西方交流方面做出了巨大的贡献。那么我们又何必去计较马可·波罗是谁呢?其实,“马可·波罗”这个名字不一定是指某个特定的人,而是指当时的一批东西方交通的开拓者。“马可·波罗”就是他们的代名词,《马可·波罗行纪》就是他们当时历险经历的总结。
历史贡献
播报
编辑
马可·波罗的中国之行及其游记,在中世纪时期的欧洲被认为是神话,被当作“天方夜谭”。但《马可·波罗游记》却大大丰富了欧洲人的地理知识,打破了宗教的谬论和传统的“天圆地方”说;同时《马可·波罗游记》对15世纪欧洲的航海事业起到了巨大的推动作用。意大利的哥伦布、葡萄牙的达·伽马、鄂本笃,英国的卡勃特、安东尼·詹金森和约翰逊、马丁·罗比歇等众多的航海家、旅行家、探险家读了《马可·波罗游记》以后,纷纷东来,寻访中国,打破了中世纪西方神权统治的禁锢,大大促进了中西交通和文化交流。因此,可以说,马可·波罗和他的《马可·波罗游记》给欧洲开辟了一个新时代。
同时,在《马可·波罗行纪》以前,更准确地说是在13世纪以前,中西方在政治、经济、文化等方面的交流都是通过中亚这座桥梁间接地联系着。在这种中西交往中,中国一直是以积极的态度,努力去了解和认识中国以外的地方,特别是西方文明世界。最早可以追述到周穆王西巡。尽管周穆王西巡的故事充满了荒诞和神话色彩,但至少反映了中国人已开始去了解和认识西方,西汉武帝时期张骞通西域之后,一条从中国经中亚抵达欧洲的“丝绸之路”出现了,中国对西方世界有了更进一步的认识和了解。唐朝是中国封建社会的鼎盛时期,经济、文化等都达到了空前的繁荣,一大批西方的商人来到中国,中国对西方世界的认识更深入了。但直到13世纪以前,中西交往只停留在以贸易为主的经济联系上,缺乏直接的接触和了解。而欧洲对中国的认识,在13世纪以前,一直停留在道听途说的间接接触上,他们对中国的认识和了解非常肤浅。因而欧洲人对东方世界充满了神秘和好奇的心理。《马可·波罗游记》对东方世界进行了夸大甚至神话般的描述,更激起了欧洲人对东方世界的好奇心。这又有意或者无意地促进了中西方之间的直接交往。从此,中西方之间直接的政治、经济、文化的交流的新时代开始了。马可·波罗是一个时代的象征。
《马可·波罗行纪》直接或间接地开辟了中西方直接联系和接触的新时代,也给中世纪的欧洲带来了新世纪的曙光。事实已经证实,《马可·波罗行纪》给这个世界带来了巨大的影响,其积极的作用是不可抹杀的。
后世影响
播报
编辑
《马可·波罗游记》是世界学术名著之一,是历史和地理的重要典籍。它在世界史、中西交通史等许多方面都有重要的历史价值。本书沟通了东西方文化的交流,向西欧介绍了东方辽阔的土地、众多的国家和富庶的中国,引起了欧洲人民对东方的向往,给十三、十四世纪欧洲的知识界、工商界、航海界带来了新的知识。《马可·波罗游记》的流传,对15世纪末欧洲航海事业的发展起了促进作用。 [5]
马可波罗的游记在13世纪末年问世后,一般人为其新奇可喜所动争相传阅和翻印,成为当时很受欢迎的读物,被称为“世界一大奇书”,其影响是巨大的。它打开了中古时代欧洲人的地理视野,在他们面前展示了一片宽阔而富饶的土地,国家和文明,引起了他们对于东方的向往,也有助于欧洲人冲洗了中世纪的黑暗,走向近代文明。学术界的一些有识之士,更以它所提供的最新知识,来丰富自己的头脑和充实自己的著作。如1375年的西班牙喀塔兰大地图,便是冲破传统观念,摈弃宗教谬说,以马可·波罗的游记为主要参考书制成的。图中的印度、中亚和远东部分都是取材于《马可·波罗行纪》这部著作,成为中世纪有很高科学价值的地图,以后地图多以此为依据。
它让西方人了解了“东方”,对东方充满向往;也为资本主义扩张提供了理想上的对象。
名家评说
播报
编辑
哥伦布:马可·波罗的书引起了我对东方神秘的向往……在我的航行中,很多次是按《马可·波罗游记》里说的去做的。 [6]
不过对于哥伦布的说法,也有人认为《游记》恰恰就是哥伦布为什么找不到印度和中国的原因。
作者简介
播报
编辑
马可·波罗(Marco Polo,1254年—1324年),世界著名的旅行家、商人。他生于意大利威尼斯一个商人家庭,也是旅行世家。
马可波罗的祖父名叫安得利亚波罗,他有三个儿子,大儿子叫老马可波罗,是马可波罗的伯父,二儿子叫尼哥罗波罗,是马可波罗的父亲,三儿子名叫马窦波罗,是马可波罗的叔叔,他的父亲和叔叔都是威尼斯商人。
马可·波罗17岁时跟随父亲和叔叔,途径中东,历时四年多来到中国,在中国游历了17年。回国后口授一本《马可·波罗行纪》,记述了他在东方最富有的国家——中国的见闻,激起了欧洲人对东方的热烈向往,对以后新航路的开辟产生了巨大的影响。同时,西方地理学家还根据书中的描述,绘制了早期的“世界地图”。 [7]

\mainmatter

\chapter{引言}

欲知世界各地之真相,可取此书读之。君等将在其中得见所志大亚美尼亚(Grande Arménie)、波斯(Perse)、鞑靼(Tartarie)、印度(Inde)及其他不少州区之伟大奇迹,且其叙述秩序井然,明了易解:凡此诸事,皆是威尼斯贤而贵的市民马可·波罗君所目睹者,间有非彼目睹者,则闻之于确实可信之人。所以吾人之所征引,所见者著明所见,所闻者著明所闻,庶使本书确实,毫无虚伪。有聆是书或读是书者,应信其真。盖书中所记皆实,缘自上帝创造吾人始祖亚当(Adam)以来,历代之人探知世界各地及其伟大奇迹者,无有如马可·波罗君所知之广也。故彼以为,若不将其实在见闻之事笔之于书,使他人未尝闻见者获知之,其事诚为不幸。余更有言者,凡此诸事,皆彼居留各国垂二十六年之见闻。迨其禁锢于热那亚(Gênes)狱中之时,乃求其同狱者皮撒(Pise)城人鲁思梯谦(Rusticien)诠次之,时在基督降生后之1298年云。

\chapter{波罗弟兄二人自君士坦丁堡往游世界}

马可君之父尼古剌(Nicolas),同尼古剌之弟玛窦(Matteo),自威尼斯城负贩商货,而至君士坦丁堡。兹二人乃华胄,谨慎而贤明。基督降生后之1260年,实在博丹(Baudoin)为君士坦丁堡皇帝之时。此兄弟二人商议后,决定赴黑海营商,于是购买珍宝,自君士坦丁堡出发,遵海而抵苏达克(Soudak)。

\chapter{002 波罗弟兄二人之离苏达克}

他们到了克里米亚以后,商量不如仍往前进,于是从苏达克首途。骑行多日,遂抵一个鞑靼君主驻所。此鞑靼君主名称别儿哥汗(Barkakhan),其主要汗牙有二,一名撒莱,一名不里阿耳(Bolghar)。别儿哥颇喜他们弟兄二人之来,待遇优渥。他们以所赍珍宝悉献于别儿哥,别儿哥乐受之,颇爱其物,乃偿以两倍以上之价。

他们留居汗牙一年后,别儿哥同东鞑靼君主旭烈兀(Houlagou)之大战发生。彼此战斗很烈,末了西鞑靼君主败衄。

双方死亡之人不少。因有此次战事,凡经行道路之人,皆有被俘之虞。波罗弟兄二人所遵之来途,危险尤大。若往前进,倒可安然无事。他们既不能后退,于是前行。

他们从不里阿耳首途,行抵一城,名称兀迦克(Oukak),是为别儿哥所领国土之尽境。他们渡伏尔加大河,经行沙漠十有七日,沿途不见城市村庄,仅见鞑靼人的畜皮帐幕同牧于田野之牲畜。

\chapter{003 波罗弟兄二人经过沙漠而抵不花剌城}

他们经过此沙漠以后,抵一城,名不花剌(Boukhara)。城大而富庶,在一亦名不花剌(Boukharie)之州中。其王名称八剌(Borak)。此城是波斯全境最要之城。他们抵此城时,既不能进,又不能退,遂留居此不花剌城三年。

他们居留此城时,有东鞑靼君主旭烈兀遣往朝见世界一切鞑靼共主的大汗之使臣过此。使臣看见此威尼斯城的弟兄二人,颇以为异。因为他们在此国中,从未见过拉丁人。遂语此二人曰:“君等若信我言,将必享大名而跻高位。”他们答云,愿从其言。使臣复曰:“大汗从未见过拉丁人,极愿见之。君等如偕我辈往谒大汗,富贵可致。且随我辈行,沿途亦安宁也。”





004 波罗弟兄二人从使臣言往朝大汗

波罗弟兄二人遂预备行装,随从使臣首途。先向北行,继向东北行,骑行足一年,始抵大汗所。他们在道见过不少奇异事物,兹略。盖马可亦曾亲见此种事物,后在本书中别有详细之叙述也。





005 波罗弟兄二人抵大汗所

他弟兄二人抵大汗所以后,颇受优礼。大汗颇喜其至,垂询之事甚夥。先询诸皇帝如何治理国土,如何断决狱讼,如何从事战争,如何处理庶务。复次询及诸国王、宗王及其他男爵。





006 大汗询及基督教徒及罗马教皇

已而大汗详询关于教皇、教会及罗马诸事,并及拉丁人之一切风俗。此弟兄二人贤智而博学,皆率直依次对答。盖彼等熟知鞑靼语言也。





007 大汗命波罗弟兄二人使教皇所

全世界同不少国土的鞑靼皇帝忽必烈汗,聆悉波罗弟兄二人所言拉丁人一切事情以后,甚喜。自想命他们为使臣,遣往教皇所(Apostolle)。于是力请他们同其男爵一人为使臣,同奉使往。他们答言,愿奉大汗之命如奉本主之命无异。由是大汗命人召其男爵一人名豁哈塔勒(Cogatal)一人来前。命他预备行装,偕此弟兄二人往使教皇所。豁哈塔勒答言,必竭全力而行主命。

已而大汗命人用鞑靼语作书,交此弟兄二人及此男爵,命他们赍呈教皇,并命他们面致其应达之词。此类书信之内容,大致命教皇遣送熟知我辈基督教律,通晓七种艺术者百人来。此等人须知辩论,并用推论,对于偶像教徒及其他共语之人,明白证明基督教为最优之教,他教皆为伪教。如能证明此事,他(指大汗)同其所属臣民,将为基督教徒,并为教会之臣仆。此外并命他们将耶路撒冷(Jérusalem)救世主墓上之灯油携还。

大汗命他三个使臣,鞑靼男爵、尼古剌·波罗、玛窦·波罗三人,赍呈教皇书的内容如此。





008 大汗以金牌赐波罗弟兄二人

大汗畀以使命以后,又赐彼等以金牌。其上有文曰,使臣三人所过之地,必须供应其所需之物,如马匹及供保护的人役之类。使臣三人预备一切行装既毕,遂辞大汗首途。

彼等骑行不知有若干日,鞑靼男爵得病不能前进,留止于一城中,病愈甚。波罗弟兄二人乃将他留在此城养病,别之西行。所过之地皆受人敬礼。凡有所需,悉见供应,皆金牌之力也。

如是骑行多日,抵于亚美尼亚之剌牙思(Layas),计在途有三年矣。因为气候不时,或遇风雪,或遇暴雨,兼因沿途河水漫溢,所以耽搁如是之久。





009 波罗弟兄二人之抵阿迦城

他们从剌牙思首途,抵于阿迦(Acre),时在1269年之4月。及至,闻教皇已死,他们遂往见驻在埃及(Egypte)全国之教廷大使梯博(Thibaud de Plaisance)。既见,告以奉使来此之意。大使闻之,既惊且喜,以此事为基督教界之大福大荣。

于是大使答波罗弟兄曰,君辈既知教皇已死,则应等待后任教皇之即位,然后履行君辈之使命。

他们见大使所言属实,遂语之曰,此后迄于教皇即位以前,我们拟还威尼斯省视家庭。乃自阿迦首途,抵奈格勒朋(Negrepont)。复由奈格勒朋登舟,而抵威尼斯。既抵威尼斯,尼古剌君闻其妻死,遗一子,名马可(Marco),年十五岁。此人即是本书所言之马可·波罗。弟兄留居威尼斯二年,等待教皇之即位。





010 波罗弟兄二人携带尼古剌子马可往朝大汗

他们弟兄二人等候许久,教皇尚未选出。于是互相商量,以为回去复命大汗时,未免太迟。于是他们携带马可,从威尼斯出发,径赴阿迦,见着那个大使,告以这种情形。并请他允许他们往耶路撒冷去取圣墓灯油,俾能复命于大汗。

大使许之。他们遂自阿迦赴耶路撒冷,取了圣墓灯油,重还阿迦。复见大使,语之曰:“教皇既未选出,我们想回到大汗所,因为我们耽搁时间业已过久了。”大使答曰:“君等既想归去,我亦乐从。”于是命人作书致大汗,证明此弟兄二人业已奉命来此。惟无教皇,故其使命未达。





011 波罗弟兄二人携带马可从阿迦首途

他们弟兄二人得到大使的书信以后,从阿迦首途,拟往复命大汗。行到剌牙思,不久听说大使梯博业已当选为教皇,号格里高利十世。他们大喜,会大使遣使者至剌牙思,告此弟兄二人云,奉教皇命,不必再往前进,可立回阿迦谒见教皇。于是亚美尼亚国王以海舶一艘,载此弟兄二人赴阿迦。





012 波罗弟兄二人还谒教皇格里高利十世

他们到了阿迦以后,卑礼晋谒教皇。教皇以礼待之,并为祝福。嗣命宣教士二人往谒大汗,履行职务。此二人皆为当时最有学识之人。一名尼古勒(Nicolede Vicence),一名吉岳木(Guillaume de Tripoli)。教皇付以特许状及致大汗书。他们四人接到书状以后,教皇赐福毕,遂携带尼古剌君之子马可,辞别教皇,从阿迦至剌牙思。

他们到了剌牙思以后,适闻巴比伦(Babylone)算端(soudan、sultan)奔多达里(Bondokdari)统领回教大兵侵入亚美尼亚,大肆蹂躏,行人大有被杀或被俘之虞。此二宣教士惧甚,不敢前进,乃以所有书状交给尼古剌、玛窦二君,与之告别,回投圣堂卫护会长(Maitre du Tempie)所。





013 尼古剌玛窦马可三人赴大汗所

他们弟兄二人携带马可首途,骑行久之,经冬及夏,抵大汗所。时大汗所驻之城曰上都,大而且富。至若他们来往途中所见所闻,后在本书中详细叙述,兹不赘言。他们归程已费时三年有半,因为气候不时,同天气严寒,所以耽搁如是之久。大汗听说他的使臣尼古剌·波罗同玛窦·波罗二人归来,命别的使臣迎之于四十日程之外。他们来去并受沿途敬礼,凡有所需,悉皆供应。





014 尼古剌玛窦马可觐见大汗

他们弟兄二人携带马可到此大城以后,遂赴宫廷觐见君主。时其左右侍臣甚众,他们三人跪见,执礼甚卑。大汗命他们起立,待遇优渥,询问他们安好及别后之事。

他们答复沿途无恙,于是呈递其所赍之教皇书状。大汗甚喜。已而进呈圣墓灯油,大汗亦甚欢欣。及见马可在侧,询为何人,其父尼古剌答曰:“是为我子,汗之臣仆。”大汗曰:“他来甚好。”

此后之事毋庸细说。读者只需知道大汗宫中大宴以庆其至,宫中诸人皆礼款之,他们偕诸侍臣留居朝中。





015 大汗遣马可出使

尼古剌君之子马可,嗣后熟习鞑靼的风俗语言,以及他们的书法,同他们的战术,精练至不可思议。他人甚聪明,凡事皆能理会,大汗欲重用之。所以大汗见他学问精进、仪态端方之时,命他奉使至一程途距离有六个月之地。

马可慎重执行他的使命,因为他从前屡见使臣出使世界各地,归时仅知报告其奉使之事,大汗常责他们说:“我很喜欢知道各地的人情风俗,乃汝辈皆一无所知。”大汗既喜闻异事,所以马可在往来途中注意各地之事,以便好归向大汗言之。





016 马可之出使归来

马可奉使归来,谒见大汗,详细报告其奉使之事。言其如何处理一切,复次详述其奉使中之见闻。大汗及其左右闻之咸惊异不已,皆说此青年人将必为博识大才之人。自是以后,人遂称之曰“马可·波罗阁下”(Messire Marc Poi),故嗣后在本书中常以此号名之。

其后马可·波罗仕于大汗所垂十七年,常奉使往来于各地。他人既聪明,又能揣知大汗之一切嗜好,于是他颇习知大汗乐闻之事。每次奉使归来,报告详明。所以大汗颇宠爱之。凡有大命,常派之前往远地,他每次皆能尽职。所以大汗尤宠之,待遇优渥,置之左右,致有侍臣数人颇妒其宠。

马可·波罗阁下因是习知世界各地之事尤力。尤专事访询,以备向大汗陈述。





017 尼古剌玛窦马可之求大汗放还本国

他们弟兄二人同马可留在大汗所的时间,前此已经说过。后来他们想归本国,数请于大汗,并委婉致辞。然大汗爱之切,欲置之左右,不许其归。

会东鞑靼君主阿鲁浑之妃卜鲁罕(Bolgana)死,遗命非其族人不得袭其位为阿鲁浑妃。因是阿鲁浑遣派贵人曰兀剌台(Oulatai)、曰阿卜思哈(Apousca)、曰火者(Coja)三人,携带侍从甚盛,往大汗所,请赐故妃卜鲁罕之族女为阿鲁浑妃。

三人至大汗所,陈明来意。大汗待之优渥,召卜鲁罕族女名阔阔真(Cogatra)者来前。此女年十七岁,颇娇丽,大汗以示三使者,三使者喜,愿奉之归国。

会马可阁下出使自印度还,以其沿途所闻之事,所经之海,陈述于大汗前。三使者见尼古剌、玛窦、马可皆是拉丁人,而聪明过人,拟携之同行。缘其计划拟取海路,恐陆道跋涉非女子所宜,加以此辈拉丁人历涉印度海诸地,熟悉道路情形,尤愿携之同往。

他们于是请求大汗遣派此三拉丁人同行,盖彼等将循海道也。大汗宠爱此三拉丁人甚切,前已说过。兹不得已割爱,许他们偕使者三人护送赐妃前往。





018 波罗弟兄同马可别大汗西还

大汗见他们弟兄二人同马可阁下将行,乃召此三人来前,赐以金牌两面,许其驰驿,受沿途供应。并付以信札,命彼等转致教皇、法兰西国王、英吉利国王、西班牙国王及其他基督教国之国王。复命备船十三艘,每艘具四桅,可张十二帆。关于此类船舶者,后再叙述,因言之甚长也。

船舶预备以后,使者三人、赐妃、波罗弟兄同马可阁下,遂拜别大汗,携带不少随从及大汗所赐之两年粮食,登船出发。航行有三月,抵南方之一岛,其名曰爪哇(Java)。岛上奇物甚众,后再详细言之。已而从此岛解维,航行印度海十八月,抵其应至之地。他们所见异物不少,后此言之。

他们到了目的地后,听说阿鲁浑已死。所以将其护送之妃交于其子合赞。他们入海之时,除水手不计外,共有六百人,几尽死亡,惟八人得免。此是实情,非誓言也。他们见君临其国者,是乞合都(Chiato),乃以护送之妃付之,并完成他们的一切使命。他们弟兄二人同马可既将大汗护送此妃的使命执行,于是告别,重复首途。他们临行前,阔阔真赐以金牌四面,两面是鹰牌,一面是虎牌,一面是净面牌,上有文云:“此三使者沿途所过之地,应致敬礼,如我亲临,必须供应马匹,及一切费用,与夫护卫人役。”于是他们所过之地,所得供应甚丰,卫骑常有二百。

他们骑行多日,始达特烈比宗德(Trébizonde),已而抵君士坦丁堡,复由此经奈格勒朋而归威尼斯,时在基督降生后之1295年也。

以上皆是引言,以后则为马可阁下所见种种事物之记录。





019 小亚美尼亚

亚美尼亚确有两处,一名大亚美尼亚,一名小亚美尼亚。小亚美尼亚国王善治其国,而臣属于鞑靼。国中有城堡不少,百物丰饶,兼为大猎禽兽之地。惟地颇不洁,而不适于健康。昔日其国贵人以好勇尚武著名。然在今日,贫贱可怜,勇气毫无,只善饮酒。其国海岸有一城,名剌牙思,商业茂盛,内地所有香料、丝绸、黄金及其他货物,皆辐辏于此。威尼斯、热那亚与夫其他各国之商人,皆来此售卖其国出产,而购其所需之物。凡商人或他种人之欲赴内地者,皆自此城发足。





020 突厥蛮州

突厥蛮州(Turcomanie)之人,凡有三种。一种是崇拜摩诃末之突厥蛮,其人粗野,自有其语言,居于山中及牧场丰富之地。盖此辈以牧畜为生,其地产良马,名曰突儿罕(Turquans)。别二种人是亚美尼亚人及希腊人,与突厥蛮杂居城堡中,为商贾或工匠。盖彼等制造世界最精美之毛毡,兼制极美极富之各色丝绸,所制甚多。又制其他布匹亦夥。其要城曰科尼亚(Konieh)、曰西瓦思(Sivas)、曰凯撒里亚(Cfisarée),此外尚有其他城市,及主教驻所不少,言之甚长,未便在此处叙述。其人隶属鞑靼,为其藩臣。

现在既述此州毕,请言大亚美尼亚。

021 大亚美尼亚

大亚美尼亚是一大州,其境始于一城,名曰阿儿赞干(Arzingan),世界最良之毛织物出产于此。境内有最美之浴场同最良之喷泉。居民是亚美尼亚人,臣于鞑靼。境内有城堡不少,其最贵重者,首数阿儿赞干。此城有其大主教。其次二城,一名阿儿疾隆(Arziron),一名阿儿疾利(Arziri),其地构成一极大之国。每届夏日,东方鞑靼全军驻夏于此,缘境内牧地甚良,可以放牧也。惟冬季酷寒,彼等不居其地,所以一届冬季,即徙居天暖有良好牧地之所,君辈应知诺亚(Noé)避洪水之大舟,即在此大亚美尼亚境内一高山之上。其南境迄于东方,与摩苏尔(Mossoul)国相接,摩苏尔国居民是雅各派(Jacobites)及景教派(Nestoriens)之基督教徒,后此别有说明。其北境与格鲁吉亚(Géorgie)人相接。此格鲁吉亚人,后章言之。其与格鲁吉亚人接境之处,有一泉,喷油甚多,同时竟可盛满百船。然其油不可食,只供燃烧,并为骆驼涂身诊治癣疥之用。人自极远之地来此以取此油,盖其地全境附近之地仅燃此油也。

兹置大亚美尼亚不言,请言格鲁吉亚。

022 格鲁吉亚及其诸王

格鲁吉亚(Gérgie)有一国王,名称大卫蔑里(David Melic),法兰西语犹言大卫国王(Roi David),臣属鞑靼。古昔国王诞生,右臂皆有一鹰痕为记。国人皆美,勇敢善射,战斗殊烈。信奉希腊派之基督教,蓄短发如书记生。亚历山大(Alexandre)西征未能通过之地,即是此州。盖其道路狭险,一方滨海,一方傍大山,不能通战骑。其道长逾四里由(Lieue,译者按:每里由约合华里十里),所以少数人守之可御重兵。亚历山大曾建一垒,极其坚固,俾此地之人不能来侵,名此垒曰铁门。亚历山大之书所言困鞑靼人于两山之间,即此地也。惟其人实非鞑靼,乃为一种名称库蛮(Comans)之民族,与夫其他众多部落。盖当此时代,尚无鞑靼也。

其地多城堡,产丝甚富,制种种金锦丝绸,极丽。产世界最良之秃鹫。百物丰饶,人民以工商为业。全州皆山,山道甚狭,凭险以守,所以鞑靼从来未能完全臣服此地。

此地有一修道院,世人名之曰圣烈庸纳儿(Saint Léonard)。有一奇迹,兹为君等述之。礼拜堂附近山下有一大湖,全年大小鱼皆无。惟至斋节(caréme)之第一日,世人取世界最美之鱼于此湖中。全斋节内,止于复活节前之土曜日,产鱼不绝。至此日后,以至下一斋节,不复有鱼。每年如此,诚为灵异。前此所言滨山之海,名称岐剌失兰(Gelachelan),广约七百哩(milles),与他海相距有十二日程。额弗剌特大河注入此海。别有数河亦然。海之周围皆山,近来不久,有热那亚商人运船置此海中,以供航行。有丝名曰岐里(ghell),即从此来。

既言大亚美尼亚之北境,今请言东境与南境间之其他边地。





023 亚美尼亚东南界之摩苏尔国

别一边界,东南之间,有摩苏尔(Mossoul)国。国甚大,人有数种。兹为说明如下:

其一种是崇拜摩诃末之阿拉伯人(Arabes)。其他与之有别,是聂思脱里派(Nestoriens)同雅各派(Jacobites)之基督教徒。他们有一总主教,名曰阿脱里克(Atolic)。此总主教任命大主教、道院长以及其他一切司教,遣派至各地,至印度,自巴格达至于契丹,如同罗马教皇派遣人员至拉丁诸国者无异。君等应知此地之一切基督教徒为数甚夥,皆是雅各派同聂思脱里派,与罗马教皇教会所统治者有别。盖其对于几种信条尚在误解之中也。此地之一切金锦同丝绸名曰摩苏尔纱(Musseline)。有许多名曰摩苏尔商(Mossolins)之商人,从此国输出香料、布匹、金锦丝绸无算。

尚有别种人名曰库尔德人(Kurdes),居住此国山中,或奉基督教,或奉回教,皆意欲劫掠商人者也。

兹置摩苏尔不言,请述巴格达大城。





024 巴格达大城及其陷落

巴格达(Bagdad)是一大城,世界一切回教徒之哈里发(Caliphe)居焉,同罗马之为基督教教皇之驻所者无异。有一极大河流通过此城,由此河可至印度海。此海距巴格达十八日程,所以有极多商人运载货物往来河上,至一城名曰怯失(Kise),由此人印度海。河上巴格达、怯失之间尚有一大城,名曰弼斯哕(Bassora)。树林围绕,出产世上最良之海枣。巴格达城纺织丝绸金锦,种类甚多,是为纳石失(Nasich)、紫锦同不少别种奇丽织物。此城乃是其地最贵最大之城。

纪元1255年时,东鞑靼君主名称旭烈兀者,是今大汗之弟,曾率大军进攻巴格达,夺据之。是役奇难,盖巴格达城中有骑兵十万,步兵尚未计焉。取此城时,见有哈里发藏宝之一塔,满藏金银宝物,任何别地宝藏从无此藏之富。旭烈兀见财宝之多,不胜惊异。命人召哈里发至,而语之曰:“哈里发,汝可告我聚积多金之理,欲聚此财何用。汝不知我为汝敌率大军而夺汝之遗业欤?曷不散此财以赐战士武人而保汝身兼保汝城。”

哈里发默然不知所答。于是此君主语之曰:“哈里发,汝既爱财宝之甚,我欲以此财宝供汝食,俾之属汝。”语毕,将哈里发闭置于藏宝塔中,禁人给与饮食。复语之曰:“哈里发,汝既爱之切,今汝可尽量食汝之财宝,任汝所欲,盖汝不复有他物可食也。”

由是哈里发困顿塔中四日,以至于死。所以为彼之计,与其如是困顿而死,不如先以财宝散给臣民以防其国也。自是以后,巴格达及别地不复再有哈里发。

兹请言上帝在巴格达城对于基督教徒所为之一极大灵奇。





025 巴格达之移山灵迹

巴格达、摩苏尔之间有一哈里发,1225年前后时驻在巴格达。深恨基督教徒,日夜思维,如何能使其国之基督教徒改从其教,抑尽杀之。常与其教之长老同谋进行之策。盖诸人亦皆敌视基督教徒,而世界上之一切回教徒对于基督教徒意见甚深,乃事实也。

有日此哈里发与诸长老在我辈之福音书中发现一文曰,一基督教徒之信心虽如芥子大,而其力可以移山。此诚实事也。此辈见此文之后,遂大欢欣,盖此为强迫一切基督教徒改教或尽杀其人之良策。如是哈里发同时召集境内之一切基督教徒,其数甚众,及诸人至,乃以福音书此段文字命其读之。读既毕,哈里发询此文是否实言。诸基督教徒答曰尽实。哈里发曰:“汝辈既以为实,汝辈人数既如是之众,其中当不乏有此少量信心之人,可选此人出,移动汝辈共见之山(并手指邻近之一山以示诸人)。否则我将尽杀汝辈。欲免死者,必须改从吾人圣教,而成为回教徒。兹限十日,到期如其事未成,或汝辈尽死,或尽改从回教。”语毕遣之出,俾其思量移山之法。





026 基督教徒闻哈里发之言大惧

基督教徒既闻哈里发之言,大惧。然他们处此情况之中,完全属望造物之主,盼其解免此种大难。所有贤明之人于是聚议。其中有不少长老主教,然除向众善所自来之天主,祈发慈悲,拯救彼等,不遭此残忍哈里发之毒手外,别无他法。

由是男女悉皆祈祷,八日八夜。至第八夜,有一主教,极善良之基督教徒也,见一神灵告语,谓天之圣神命彼令一独眼靴工祈祷天主。天主悲悯,必因靴工之清德而如其愿。

兹请言此靴工为何种人。君辈应知此人正直纯洁,持斋,不犯何种罪恶,逐日必赴礼拜堂聆听弥撒(messe),并以其工资之一部贡献天主。至其仅存一眼之缘由,则如下说。某日有一妇人嘱彼缝制一靴。此妇腿足皆丽,出其足以量靴之尺寸。靴工心动,已而大悔。其人数闻福音书中之言,外眼有过,累及良心,应于犯过之前,立时将眼抉出头外。于是待此妇去后,取缝靴之锥剌其一眼。由是仅存一眼。

君辈观此事,足见此人正直清洁,品行优良。





027 主教见独眼靴工

前所言之神灵,主教数见之。于是以其事详告诸基督教徒。诸人乃召此靴工来前,及至,求其祈祷。并言天主曾许,彼所祷者,将如其愿。靴工闻言,谢以无此德行,不敢为之。诸人委婉祈请,靴工始应。





028 靴工之祈祷移山

至限期之末日,一切基督教徒黎明即起。男女老少十万余人,群赴礼拜堂。聆弥撒毕,进赴此山附近之平原,以十字架前导,大声歌唱,流涕而行。及至平原,见哈里发率其回教军队以待,俟不愿改教之时,尽将此辈处死。盖回教徒决未思及上帝施此恩惠于基督教徒也。基督教徒畏甚,然盼望天主耶稣基督之心未已。

至是,靴工受主教赐福毕,跪于十字架前,引手向天,致此祷词曰:“万能的天主,请发神圣慈悲,惠汝人民,俾其不死。俾汝之教理不致推翻,不致减削,不致为人所蔑视。我虽不足为祈祷请求之人,然汝之权能慈悲并大,当必聆悉汝罪恶充满的奴仆之祈祷也。”

靴工向施与一切恩惠的天主致祷词毕,于哈里发一切回教徒及其他诸人众目共睹之下,忽见此山从地而起,自移向哈里发前此所指之处。哈里发及回教徒见之惊诧,由是回教徒改从基督教者为数甚众。甚至哈里发亦奉圣父圣子圣神之名,接受洗礼,成为基督教徒。然其事秘,外人鲜知。迨至此哈里发死后,人见其项上悬有一小十字架,始获知之。因是回教徒将他别葬他处,不葬于其他诸哈里发之列。诸基督教徒见此伟大神圣灵迹,皆大欢喜。归后作大庆贺,以谢其造物主之恩。

其事之经过,诚如君等所闻,是为一种极大灵迹。回教徒之恨基督教徒,君等勿以为异。缘彼等所奉者非同一教法也。

我今述巴格达之事毕,尚可述其事业风俗。然因我所述之大事同灵迹已甚冗长,如再增益他事,则君等将有琐细之讥矣。

所以我今接言帖必力思(Tauris)贵城。





029 帖必力思城

帖必力思是一大而名贵之城,位在一名曰伊剌克(Irak)大州之中。其州别有城堡数处,然以帖必力思最为名贵。故为君等叙述此城。

帖必力思之人,实以工商为业。缘其制作种种金丝织物,方法各别,价高而奇丽也。此城位置适宜,印度、巴格达、摩苏尔、格儿墨昔儿(Guermessir)及其他不少地方之商货,皆辐辏于此。拉丁商人数人,尤其是热那亚商人,亦至其城购货,并经营他种商业。盖城中尚有宝石不少,商人于此大获其利。

居民贫苦,杂有种种阶级之人。其中有亚美尼亚人、聂思脱里派人、雅各派人、格鲁吉亚人、波斯人,并有性恶而崇拜摩诃末名称帖兀力思(Taurizi)之土人。城之四围,绕以可供娱乐之美丽园林,内产数种良果,果大而味美。

今置帖必力思不言,请言波斯大州。





030 波斯大州

波斯古为著名强盛大国,今已为鞑靼所破毁。境内有城名曰撒巴(Saba),昔日崇拜耶稣基督之三王发迹于此。死后葬此城中。三墓壮丽,各墓上有一方屋,保存完好。三屋相接,三王遗体尚全,须发仍存。一王名札思帕儿(Jaspar),一王名墨勒觉儿(Melchior),一王名巴勒塔咱儿(Balthazar)。马可·波罗阁下久询此三王之事于此城民,无人能以其事告之,仅言昔有三王死葬于此。然在距此三日程之地,获闻下说。兹请为君等述之。其地有一堡,名曰哈剌阿塔毕里思丹(Gala Ataperistan),法兰西语犹言“拜火之堡”。此名于此堡颇宜,盖此地之人崇拜火光,兹请为君等说明其故。

相传昔日此国有三王,闻有一预言人降生,偕往顶礼。三王各携供品,一携黄金,一携供香,一携没药。欲以此测度此预言人为天神,为人王,抑为医师。盖若受金则为人王,受香则为天神,受没药则为医师也。

及至此婴儿诞生之处,三王年最幼者先人谒,见此婴儿与己年相若。年壮者继人,亦见婴儿与己年相若。较长者后人,所见婴儿年岁亦与己同。三王会聚,共言所见,各言所见不同,遂大惊诧。三王共人,则见婴儿实在年岁,质言之,诞生后之十三日也。乃共顶礼,献其金、香、没药,婴儿尽受之。旋赐三王以封闭之匣一具,诸王遂就归途。





031 三王之归

三王骑行数日后,欲启示婴儿所给之物。发匣视之,仅石一块。三王见之惊诧,互询婴儿给物之意何居。其意义实如下说:盖三王献其供物之时,婴儿尽取三物,由是足见婴儿为天神,为人王,并为医师。以石给之者,乃欲三王之信心坚如此石也。乃三王不解此意,投石井中。石甫下,忽有烈火自天下降此井。

三王见此灵异,既惊且悔,乃知其意既大且善,不应投石井中。乃取此火,奉还其国,置一华美礼拜堂中,继续焚烧,崇拜如同天神。凡有供物,皆用此火烧熟。设若火熄,则往附近信仰同教之他城求火,奉归其礼拜堂中,此地人民拜火之原因如此。常往十日程途之地以求此火。

此地之人所告马可·波罗阁下之言如此,力证其事如是经过。其一王是撒巴城人,别一王是阿瓦(Ava)城人,第三王是今尚崇拜火教之同堡之人。

我辈既述此故事毕,请接言波斯诸州及其特点。





032 波斯之八国及其名称

波斯是一极大之国,境内有八国,兹为君等尽举其名如下:波斯境界开始之第一国,名曰可疾云(Casvin);第二国稍南,名曰曲儿忒斯单(Curdistan);第三国名曰罗耳(Lor);第四国名曰薛勒斯单(Cielstan);第五国名曰伊思塔尼惕(Istanit);第六国名曰泄剌失(Serasy);第七国名曰孙思哈剌(Sonscara);第八国名曰秃讷哈因(Tunocain);是为波斯门户。自北往南,行程皆经诸国,仅有一国在外。此国即是秃讷哈因,境在“独树”(Arbre seul)附近。

在此波斯国中,颇有不少良马,中有运赴印度贩卖者。盖其马价值甚贵,一马约值“秃儿城的里物”(livres-tournois)二百枚。视其优劣,价有贵于此数者,亦有贱于此数者。国亦有驴,是为世界最美之驴。一头价值银马克(marc)三十,盖其躯大而健走。其国之人运马至于怯失及忽鲁模思两城,此两城在印度海沿岸,有商人在此购马转贩印度。

此国中有不少残忍好杀之人,每日必有若干人被杀。若不在东鞑靼君主统治之下,商人受害,将必甚重。虽有此种统治,尚不免有加害商人之举。若商人武装不足,则人尽被杀,物尽被掠。盖有时商人防备不严,悉被杀害也。





033 耶思德大城

耶思德是一最良、最名贵,并且可以注意之城。商业茂盛,居民制作丝织物名曰耶思的(yazdi)。由商人运赴各地,贩卖谋利。居民崇拜摩诃末。

若离此城远行,骑行平原亘七日,仅有三处可以住宿。时常经过美林,其中极易走马,亦易豢鹰猎取鹧鸪、鹌鹑及其他飞鸟。所以商人经行此地者行猎娱乐,其地亦有极美之野驴。

在平原中行逾七日,抵一最美之国,名曰起儿漫。





034 起儿漫国

起儿漫是波斯境内之一国。昔日国王世袭,自经鞑靼侵略以后,世袭之制遂废。鞑靼遣其乐意之王治之。此国出产名曰突厥玉(turquoises)之宝石甚多,产于山中,采自某种岩石之内。亦有不少钢及“翁苔尼克”(ondanique)之矿脉。居民善制骑士军装,如马圈、马鞍、靴剌、剑、弓、菔等物,手艺甚巧,皆适于用。妇女善于女红,善为各色剌绣,绣成鸟、兽、树、花及其他装饰。并为贵人绣帐幕,其妙不可思议。亦绣椅垫、枕、被及其他诸物。

起儿漫山中有世界最良之鹰,比较游鹰为小,胸、尾及两股间并为红色,其飞迅捷,捕捉时无有飞鸟能免。

自此起儿漫城骑行七日,道上城村及美丽居宅不绝。所以旅行甚乐,亦可携鹰行猎,其愉快不可言状。在此平原中行过七日,抵一山甚大,登至山巅,则见大坡。下坡须经两日,沿途见有不少果实。昔日此地居宅不少,今则寂无一宅。然见有人牧其牲畜。自起儿漫城至此坡,冬季酷寒,几莫能御。





035 哈马底城及其残破

骑行整二日,到一大平原,首见一城,名曰哈马底(Camadi)。昔甚壮丽,自经鞑靼数次残破以后,今日已非昔比。此平原位在一极热地带之中,首至之州名曰别斡巴儿勒(Beobarles),此地出产海枣、天堂果及其他寒带所无之种种果实。平原之中有一种鸟,名曰黑鹧鸪(franco-lins),与别地所产者异。盖其羽毛黑白错杂,而喙爪皆朱色也。兽类亦异,请先言牛。牛身大,色自如雪,蹄小而扁平,地热使然。角短而巨,其端不锐,两肩之中有圆峰,高有两帕麦(palmes)。世界悦目之兽,无过于此。载物之时,跪地受之,与骆驼同,载物讫则起立,虽重亦然,盖其力甚强也。又有羊,高如驴,尾大而宽,有重三十磅者,身美肥,肉味佳。

此平原中有城村数处,环以土筑高墙,可御盗贼。其地盗贼甚夥,名曰哈剌兀纳(Caraonas),缘彼等之母是印度人,父是鞑靼人,故以此为名。君等应知此辈哈剌兀纳欲出抄掠之时,则念咒语,天忽阴暗,对面几不见人,阴暗亘七日。此辈熟悉地形,阴暗之中可以并骑而驰。聚众有时至万人左右。由是所到之处,尽据城村以外之地,尽俘男女牲畜,杀其老弱,卖其壮丁妇女于他国,无能免者。所以大为其地患,使之几成荒原。

此辈恶人之王,名曰那古答儿(Nogodar)。察合台汗者,大汗之弟,亦那古答儿之从父也。那古答儿曾率所部万骑,往投察合台汗廷。当其留居汗廷之时,曾经谋叛。会其从父远在大亚美尼亚境内,他率凶勇之士骑无数,进躏巴达哈伤(Badakchan),复躏名曰帕筛底儿(Pachai-Dir)之别州,又躏名曰阿里斡剌客失木儿(Ariora-Kechemour)之别州。顾道路险狭,丧失士骑不少。他占领上述诸州以后,侵入印度,至一名曰答里瓦儿(Dalivar)之州之尽境。据此城,复据其国。时此国之王名称阿思丁莎勒檀(Asedin Soldan),一富强之君主也。那古答儿率军据其国后,遂无足畏者,乃与附近之一切鞑靼相争战。

我既述此种恶人及其历史毕,尚有为君等告者。马可·波罗阁下在阴暗之中,曾为若辈所擒。赖天之佑,得脱走,人一名哥那撒勒迷(Conosalmi)之村中。然同伴尽没,仅有七人获免。

即述此事毕,请进而别言他事。

036 又下坡至忽鲁模思城

此平原向南延展,足有五日程。已而又见一坡,长二十哩,道路不靖,盗贼恶人充斥。抵此坡下,又见一平原,甚丽,名曰福鲁模思(Formose)平原。广二日程,内有美丽川流。出产海枣及其他果物不少,并有种种美鸟无数,皆为吾辈国中所未见者。骑行二日,抵于大洋,海边有一城,名曰忽鲁模思(Ormus)。城有港,商人以海舶运载香料、宝石、皮毛、丝绸、金锦与夫象牙暨其他货物数种,自印度来此,售于他商,转贩世界各地。此城商业极其繁盛,盖为国之都城。所属城村不少。国王名称鲁墨耽阿合马(Ruomedam Ahomet)。阳光甚烈,天时酷热。城在陆上,外国商人殁于此者,国王尽取其资财。

此地用香料酿海枣酒,甚佳。初饮此酒者,必暴泻,然再饮之,则颇有益,使人体胖。其地之人惟于有病时食肉与面包,无病食之则致疾。其习食之物,乃为海枣、咸鱼、枸橼、玉葱。其人欲保健康,所以用玉葱代肉。其船舶极劣,常见沉没,盖国无铁钉,用线缝系船舶所致。取“印度胡桃”(椰子)树皮捣之成线,如同马鬣,即以此线缝船,海水浸之不烂,然不能御风暴。船上有一桅、一帆、一舵,无甲板。装货时,则以皮革覆之,复以贩售印度之马置于革上。既无铁作钉,乃以木钉钉其船。用上述之线缝系船板,所以乘此船者危险堪虞,沉没之数甚多。盖在此印度海中,有时风暴极大也。

其人色黑,崇拜摩诃末。其地天时酷热,居民不居城中,而居城外园林。园林之间,水泉不少。虽然如是,若无下述之法,仍不能抵御:此热:

夏季数有热风,自沙漠来至平原。其热度之大,不知防御者遭之必死。所以居民一觉热风之至,即入水中,仅露其首,俟风过再出。

每年11月播种小麦、大麦及其他诸麦,次年3月收获。除海枣延存至5月外,别无青色植物,盖因热大,植物俱干也。

船虽不坚,然有时不致破损者,盖有鱼油涂之。居民有死者,则持大眼,盖悲泣亘四年也。在此期内,亲友邻人会聚,举行丧礼,大号大哭,至少每日一次。

兹置此地不言,至关于印度者,后再述之。今往北行,从别一道复至起儿漫城,盖赴别地者,不能不经过起儿漫也。

君等应知忽鲁模思国王鲁墨耽阿合马是起儿漫国王之藩臣。

从忽鲁模思还起儿漫之途中,路见天然浴泉不少。地为平原,城市甚众,果实亦多,其价甚贱。面包甚苦,非习食者不能食,缘其水甚苦也。上述之浴泉可治癣疥及其他数种疾病。

兹请在本书言北行所过诸地。





037 经行一疲劳而荒寂之道途

离此起儿漫城以后,必须经行至少七日程之困难路途。前三日路上无水,虽有若无,盖所见之水味苦色绿,奇咸不可饮。饮此水一滴者,在道必洞泻十次。其水道所含盐质,类皆如此。行人既不敢饮,亦不敢食。盖食者常致泻痢也。所以行人必先预赍饮水,以供三日之需。然牲畜渴甚,不得不饮此咸水,故偶有泻痢而致毙者。三日之中,不见民居,尽是沙漠干旱,亦无野兽痕迹,盖其不能在其中求食也。

经行此三日沙漠以后,见一清流,流行地下,沿流地面,有穴可以见之。水量甚大,行人困于沙漠者,必息于此饮水,并以饮其牲畜。

于是又人别一沙漠,亘四日程。景况与前一沙漠完全相同。惟有野雁,斯为异耳。逾此四日程之沙漠,遂出起儿漫国界,而至一城,名曰忽必南(Cobinam)。





038 忽必南城及其出品

忽必南是一大城,居民崇拜摩诃末。出产铁、钢、翁答尼克甚夥,而制造钢镜极巨丽。其地制造眼药(toutie),治眼疾之良药也。并作矿滓(espodie),其法如下:掘地为长坑,置火灶于其中,上置铁格,坑中上升之烟与液,粘于格上,是为眼药。火熔余物则成为滓。

兹置此城不言,请接言前途之地。





039 亘延八日程之沙漠

离此忽必南城以后,见一沙漠,亘延八日。完全干旱,绝无果木。水亦苦恶,行人必须携带饮食。牲畜渴甚,不得不饮此恶水。八日后,行抵一州,名曰秃讷哈因(Tunocain)。境内有环以城墙之城村不少,是为北方波斯之边界。其地有一极大平原,吾人名曰“枯树”(Arbre sec)之“太阳树”(Arbre Sol)在焉。兹请言其状。树高大,树皮一部分绿色,一部分白色。出产子囊,如同栗树,惟子囊中空。树色黄如黄杨,甚坚。除一面六哩外生有树木外,周围百哩之地别无其他树木。土人言亚历山大进攻大留士(Darius),即战于此。城村百物丰饶,缘其地气候适宜,不甚热,亦不甚寒。居民尽奉摩诃末之教,形貌甚美,女子尤其美甚。

离开此地以后,吾人将言一名称木剌夷(Mulette)之地,即“山老”习与其哈昔新(Hasisins)居留之所也。





040 山老

木剌夷(Mulette)是山老昔日习居之地,法兰西语犹言地神。兹请将马可·波罗阁下所闻此地数人所述其地之历史,为君等述之。此老在其本地语言中,名称曰阿剌丁(Alaeddin)。他在两山之间,山谷之内,建一大园,美丽无比。中有世界之一切果物,又有世人从来未见之壮丽宫殿,以金为饰,镶嵌百物,有管流通酒、乳、蜜、水。世界最美妇女充满其中,善知乐、舞、歌唱,见之者莫不眩迷。山老使其党视此为天堂,所以布置一如摩诃末所言之天堂。内有美园、酒、乳、蜜、水,与夫美女,充满其中。凡服从山老者得享其乐。所以诸人皆信其为天堂。

只有欲为其哈昔新者,始能人是园,他人皆不能人。园口有一堡,其坚固之极,全世界人皆难夺据。人人此园者,须经此堡。山老宫内蓄有本地十二岁之幼童,皆自愿为武士,山老授以摩诃末所言上述天堂之说。诸童信之,一如回教徒之信彼。已而使此辈十人,或六人,或四人同人此园。其人园之法如下:先以一种饮料饮之,饮后醉卧,使人畀置园中,及其醒时,则已在园中矣。





041 山老训练哈昔新之法

彼等在园中醒时,见此美境,真以为处天堂中。妇女日日供其娱乐,此辈青年适意之极,愿终于是不复出矣。

山老有一宫廷,彼常绐其左右朴质之人,使之信其为一大预言人此辈竟信之。若彼欲遣其哈昔新赴某地,则以上述之饮料,饮现居园中之若干人,乘其醉卧,命人畀来宫中。此辈醒后,见已身不在天堂,而在宫内,惊诧失意。山老命之来前,此辈乃跪伏于其所信为真正预言人之前。山老询其何自来。答曰,来自天堂。天堂之状,诚如摩诃末教法所言。由是未见天堂之人闻其语者,急欲一往见之。

若彼欲剌杀某大贵人,则语此辈曰:“往杀某人,归后,将命我之天神导汝辈至天堂。脱死于彼,则将命我之天神领汝辈重还天堂中。”

其诳之之法如是。此辈望归天堂之切,虽冒万死,必奉行其命。山老用此法命此辈杀其所欲杀之人。诸国君主畏甚,乃纳币以求和好。





042 山老之灭

基督教诞生后1252年,东鞑靼君主旭烈兀闻此老之大恶,欲灭之。乃选一将,命率一大军进围此堡。堡甚坚,围之三年而不能克。设若彼等有粮可食,彼等殆永不能克之。然三年之后,粮食欠缺,遂尽作俘虏。山老及其部众并被屠杀。嗣后不复有其他山老,盖其恶贯已盈矣。





043 撒普儿干

离此堡后,骑行所过,或是美丽平原,或是饶沃流域。中有极美之草原、良好之牧场,果实不少,百物丰富,军队颇愿留驻于此。其地亘延六日程,颇有绕以墙垣之城村。居民崇拜摩诃末。有时见有沙漠地带,长约六十哩,或不及六十哩,其中点水毫无,行人必须载水而行。

骑行六日后,抵一城,名曰撒普儿干(Sapourgan)。百物皆富,尤出世界最良之甜瓜。居民切瓜作条,在太阳下曝干,既干食之,其甜如蜜。全境售此以作商货。其地颇有猎兽飞禽。

今对此城不复有可言者,请言别一名曰巴里黑(Balkh)之城。





044 巴里黑城

巴里黑是一名贵大城,昔日尤形重要。然历经鞑靼人及他种人之残破,昔之美丽宫殿以及大理石之房屋,已不复存。据城人云,亚历山大取大留士女为妻,即在此城。居民崇拜摩诃末。东鞑靼君主所辖之地止于此城。是为波斯与东境及东北境分界之处。

兹置此地不言,请言别一名曰哈纳(Khana)之地。

离开上述之城后,向东方与东北方之间骑行十二日,不见人烟。盖居民因避兵与匪之害,皆移居山寨也。其地有鹫与猎兽不少,狮子亦众。行人不能在此处得食,须赍此十二日内必需之物而行。





045 盐山

此十二日行毕后,抵一堡,名曰塔亦寒(Taican)。有一大市场出售小麦。其地风景甚美。南方诸山甚高,皆由盐构成。全境周围三十余日程地方之人,皆来此取盐。是为世界最佳之盐,其质硬,须用大铁锄始能取之。其量之多,可供全世界人之需,至于世界末日。

从此城行,仍向东及东北间骑行三日,经过甚美之地。广有果实,民居不少,葡萄及其他贱价之物甚多。居民悍恶而好杀人,嗜饮酒,善饮,饮辄致醉,其酒煮饮。头缠一绳,长有十掌,绕于头上。善猎,能取野兽无算。仅以猎兽之皮制衣做靴。各知制皮以作衣靴。

骑行此三日后,至一城,名曰讫瑟摩(Casem)。其他具有墙垣之城村尽在山中。有一河流尚大,流经此城。其地出产豪猪不少,其躯甚大。猎人携犬往猎时,数豪相聚互守,以脊剌剌犬,使之数处负有重伤。

此讫瑟摩城管理一州,亦名讫瑟摩,居民自有其语言。农民偕其牲畜居于山中,在地下掘室,颇巨丽。掘窨甚深,以居。掘之甚易,盖山为土质也。

经过此讫瑟摩城以后,骑行三日,不见人烟,不得饮食,所以行人必须携带所需之物。行三日毕,至一州,名曰巴达哈伤。兹请为君等述其沿革。





046 巴达哈伤州

巴达哈伤(Badakchan)一州之地,人民崇拜摩诃末,自有其语言,是为一大国。君位世袭,王族皆是亚历山大与波斯大国君主大留士女之后裔。回教语言名如是诸王曰竹勒哈儿年(Zulcarniens),法兰西语犹言亚历山大。盖因追忆亚历山大大王而有斯称也。

此州出产巴剌思红宝石(rabis balais),此宝石甚美,而价甚贵。采之于若干山岩中,掘大隧以采之,与采银矿之法同。仅在一名尸弃尼蛮(Sygniman)之山中发现此物。国王只许官采,他人不得至此山采发,否则杀其人而没其资财。任何人不许将此物运往国外,所采宝石尽属国王,或以之贡于他国,或以之赠与他国国王。所以此红宝石甚稀,而其价值甚贵。盖若任人采取,则此宝石将充满于世界,不足重矣。采取之少,防守之严,其故在此。

同一境内别有一山,出产瑟瑟(azur、lapis-lazuli),其莹泽为世界最。产于矿脉中,与银矿同。他山复有银矿不少,所以此州最富。然其气候亦最寒。兼产良马,善于奔驰,蹄下不钉蹄铁,而能驰骋山中及崎岖之地。此地诸山又出产大鹰(faucon sacre)同郎奈鹰(faucon lanier),猎兽飞禽为数亦夥。又产良好小麦及无壳大麦,出产芝麻油、胡桃不少,惟无橄榄油。

国中有狭隘甚多,难攻而易守。建城村于高山上,形式险要。居民善射,善猎。布价甚贵,多衣兽皮。但贵妇人及贵人则衣布。以棉布作裤,需布有至百寻者。如是表示其腰宽大,男子颇乐衣此也。

述此国之事既毕,兹请言南方相距此地十日程之诸国。





047 帕筛州

巴达哈伤南方相距十日程之地,有一州,名称帕筛(Pashai)。居民自有其语言,崇拜偶像。面褐色,颇知魔术。男子耳戴金银环,以珍珠宝石饰之。其人颇狡猾,其风习则有节制,食肉米,气候甚热。

兹不再言此州,请言其东相距七日程之客失迷儿(Kachmir)州。





048 客失迷儿州

客失迷儿亦是一州,居民是偶像教徒,自有其语言。熟知禁咒,其奇不可思议。缘彼等能使其偶像发言,能用巫术变更天时,使之黑暗。其事之奇,非亲睹者,无人能信其有此事。此地乃是偶像教发生之源。

仍向此方向陆续前行,则可抵于印度海。

其人色褐而体瘦,其妇女虽褐色,而貌甚美。其食物为肉、乳及米。气候温和,不甚热,亦不甚寒。有环以墙垣之城村不少,亦有林木、旷野及天然险隘。居民不畏外侵,自主其国,自有国王治理。其俗有隐士居住隐所,节其饮食,持身极严,不犯其法所禁之一切过失,所以其门弟子视之如同圣人。此国之人享年甚高,国内有寺院不少。从吾辈地域输入之珊瑚,售价之贵,过于他国。

兹置此地及其附属地带不言,盖若更从(同一方向)前行,则将进入印度。顾关于印度之事,吾人在归途述之。由是吾人重回巴达哈伤,否则将不能继续吾人之旅行矣。





049 巴达哈伤大河

从巴达哈伤首途,骑行十二日,向东及东北溯一河流而上。此河流所经之地,隶属巴达哈伤君主之弟。境内有环以墙垣之城村及散布各处之房屋不少。居民信奉摩诃末,勇于战斗。行此十二日毕,抵一大州,宽广皆有三日程,其名曰哇罕(Wakhan)。居民信奉摩诃末,自有其语言。善战斗,有一君主名曰那奈(None),法兰西语犹言伯爵也。其人称藩于巴达哈伤君主。

境内颇有种种野兽。离此小国以后,向东北骑行三日,所过之地皆在山中。登之极高,致使人视其为世界最高之地。既至其巅,见一高原,中有一河。风景甚美。世界最良之牧场也。瘦马牧于是,十日可肥。其中饶有种种水禽,同野生绵羊。羊躯甚大,角长有六掌。牧人削此角作食盘,且有用作羊群夜宿之藩篱者。此高原名称帕米尔(Pamir),骑行其上,亘十二日,不见草木人烟,仅见荒原,所以行人必须携带其所需之物。

其地甚高,而且甚寒,行人不见飞鸟。寒冷既剧,燃火无光。所感之热不及他处,烤煮食物亦不易熟。

今请言东方及东北方更远之地。继续山行亘四十日,见有溪涧甚多,亦有不少沙漠。沿途不见人烟草木,所以行人必须携带其所需之物。

此地名曰博洛尔(Belor)。居民居住高山之上,信奉偶像,风俗蛮野,仅以猎兽为生,衣兽皮,诚恶种也。

今请离去此地,接言可失合儿州。





050 可失合儿国

可失合儿(Kachgar)昔是一国,今日隶属大汗。居民信奉摩诃末境内有环以墙垣之城村不少,然最大而最丽者,即是可失合儿本城。此国亦在东方及东北方之间,居民为工匠商贾。有甚美之园林,有葡萄园,有大产业,出产棉花甚饶。有不少商人由此地出发,经行世界贸易商货。居民甚吝啬窘苦,饮食甚劣。此地有不少聂思脱里派之基督教徒,有其本教教堂。国人自有其语言,地广五日程。

兹置此地不言,请言撒麻耳干(Samarkand)。





051 撒麻耳干大城

撒麻耳干是一名贵大城。居民是基督教徒同回教徒,臣属大汗之侄海都(Kaidou)。然大汗与其侄交恶。城在西北方,兹请为君等叙述此城之一大灵迹。

距今未久,大汗之族兄察合台(Djagatai,君临此地及其他诸地者)皈依基督教。国内基督教徒见其主奉行其教,因之大欢。遂在此城建一大礼拜堂,奉祀圣若望巴迪思忒(Saint Jean Baptiste),即以此圣名名其礼拜堂。有一美石原属回教徒,建堂人取之以承堂中上承堂顶之柱。会察合台死,诸回教徒颇欲将现在基督教礼拜堂中之柱石索还。遂互议曰,用善言抑用武力收回此石,此其时矣。缘彼等人数十倍于基督教徒,其力足以为此也。乃群赴基督教徒之礼拜堂前,语基督教徒,言欲必得其石。基督教徒答言,石固属彼等,然愿以金易之。回教徒言,世上无论何物不足以易,由是彼此争持甚烈。其主闻声,询得其故,乃命基督教徒能用金偿则偿之,否则退还此石,限期三日执行。

回教徒无论如何不愿以石易金,彼等并知此石若去,礼拜堂必陷。由是基督教徒怒极不知所为,遂祷告耶稣基督,求其庇佑,俾主持圣者若望巴迪思忒之名不在本堂毁坠。限其既届,某日黎明,忽见其石移出柱下。时柱离地高有三掌,悬空不坠,与有基础时同。回教徒虽得其石,然皆丧气而去。是为此大灵迹之经过。其柱现仍悬空如故,以迄天主不欲之时。





052 鸭儿看州

鸭儿看(Yarkend)乃是一州,广五日程。居民遵守摩诃末教法,然亦有聂思脱里派(Nestoriens)同雅各派(Jacobites)之基督教徒。并属大汗之侄,即前此所言之同一君主是已。居民百物丰饶,然无足言者,所以置之,请言别一名曰忽炭(Khotan)之州。





053 忽炭州

忽炭一州处东方及东北方之间,广八日程。臣属大汗,居民崇拜摩诃末。境内有环以墙垣之城村不少。然最名贵者是忽炭城,国之都也,故其国亦名忽炭。百物丰饶,产棉甚富,居民植有葡萄园及林园,而不尚武。

兹从此地发足,请言别一名曰培因(Pein)之州。





054 培因州(播仙)

培因州,广五日程,处东方及东北方之间。居民崇拜摩诃末,臣属大汗。境内有环以墙垣之城村不少。最名贵者是培因城,国之都也,有河流经行城下。河中产碧玉(jaspe)及玉髓(chalcédoine)甚丰。(译者按:后文有注,而本文于此处作注,而无注,比对附注,此处实是注,则此下应有脱文。)

君等应知,前述自可失合儿迄于此地之诸州,与夫行将说明前途之诸州,并属大突厥。





055 车尔成州

车尔成是大突厥之一州,处东方及东北方间。居民崇拜摩诃末。有环以墙垣之城村不少。国之都城亦名车尔成,境内河流中有碧玉及玉髓,取以贩售契丹,可获大利。全州之地满布沙砾,自培因达此之道途亦然。所以水多苦恶,然有数处有甘水可饮。军队通过其境时,居民挈其妻儿牲畜逃往沙漠中,彼等习知有水可以生存之处。行后风掩其迹,追者莫知其逃亡之所。

自车尔成首途后,在沙漠中骑行五日,仅见苦水。然更往前行,有一地有甘水可饮。

此地既无他事足述,吾人仍往前行,请述一名曰罗不(Lop)之州。行上述之五日毕,抵一城,名曰罗不。此城在人广大沙漠之处。所以行人于人沙漠之前,必在此城停息。





056 罗不城

罗不(Lop)是一大城,在名曰罗不沙漠之边境,处东方及东北方间。此城臣属大汗,居民崇拜摩诃末。前此已言凡行人渡此沙漠者,必息于此城一星期,以解人畜之渴。已而预备一月之粮秣,出此城后,进入沙漠。

此沙漠甚长,骑行垂一年,尚不能自此端达彼端。狭窄之处,须时一月,方能渡过。沿途尽是沙山沙谷,无食可觅。然若骑行一日一夜,则见有甘水,足供五十人或百人暨其牲畜之饮。甘水为数虽不多,然全沙漠中可见此类之水。质言之,渡沙漠之时,至少有二十八处得此甘水,然其量甚寡。别有四处,其水苦恶。

沙漠中无食可觅,故禽兽绝迹。然有一奇事,请为君等述之。行人夜中骑行渡沙漠时,设有一人或因寝息,或因他故落后,迨至重行,欲觅其同伴时,则闻鬼语,类其同伴之声。有时鬼呼其名,数次使其失道。由是丧命者为数已多。甚至日间亦闻鬼言,有时闻乐声,其中鼓声尤显。渡漠情形困难如此。

兹置此罗不大沙漠不言,请言出漠后所见之诸州。





057 唐古忒州

在此沙漠中行三十日毕,抵一城,名曰沙州。此城隶属大汗。全州名唐古忒(Tangout)。居民多是偶像教徒,然亦稍有聂思脱里派之基督教徒若干,并有回教徒。其偶像教徒自有其语言。城在东方及东北方间。居民恃土产之麦为食。境内有庙寺不少,其中满布种种偶像,居民虔诚大礼供奉。例如凡有子女者,为偶像畜养一羊。年终或偶像节庆之日,畜羊者挈其子女携羊至偶像前礼拜。拜后,烤煮羊肉使熟,复礼奉之于偶像前陈之。礼拜祈祷,求神降福于其子女。据云,偶像食肉。供奉既毕,取肉还家,延亲属共食。食后谨藏余骨于匣中。

君等应知世界之一切偶像教徒皆有焚尸之俗。焚前,死者之亲属在丧柩经过之道中,建一木屋,覆以金锦绸绢。柩过此屋时,屋中人呈献酒肉及其他食物于尸前,盖以死者在彼世享受如同生时。迨至焚尸之所,亲属等先行预备纸扎之人、马、骆驼、钱币,与尸共焚。据云,死者在彼世因此得有奴婢、牲畜、钱财等若所焚之数。柩行时,鸣一切乐器。

其焚尸也,必须请星者选择吉日。未至其日,停尸于家,有时停至六月之久。

其停尸也,方法如下:先制一匣,匣壁厚有一掌,接合甚密,施以绘画,置樟脑香料不少于匣中,以避臭气。旋以美丽布帛覆于尸上。停丧之时,每日必陈食于柩前桌上,使死者之魂饮食。陈食之时,与常人食时相等。其尤怪者,卜人有时谓不宜从门出丧,必须破墙而出。此地之一切偶像教徒焚尸之法皆如是也。

兹置此事不言,请言此沙漠西北极端之别一城。





058 哈密州

哈密(Camul)州昔是一国,境内有环以墙垣之城村不少,然其要城即是哈密。此州处两沙漠间,一面是罗不大沙漠,别一面是一广三日程之小沙漠。居民皆是偶像教徒,自有其语言。土产果实不少,居民恃以为生。其人爱娱乐,只知弹唱歌舞。设有一外人寄宿其家,主人甚喜,即命其妻厚为款待,自己避往他所,至外人去后始归。外人寄宿者,即有主人妻做伴,居留久暂惟意所欲,主人不以为耻,反以为荣。妇女类皆美丽,全州之中皆使其夫作龟(cornards),其事非伪也。

蒙哥汗在位辖有此州之时,闻此风习,命人禁绝,犯者严惩。居民奉命忧甚,共醵重币以献,请许保其祖宗遗风。且谓赖有此俗,偶像降福,否则彼等不能生存。蒙哥汗乃曰:“汝等既欲耻辱,保之可也。”于是放任如故,至今尚保存此恶俗也。

兹置哈密不言,请言西北方与北方间之别一州。此州隶于大汗,名曰许许塔剌(Chiuchiutala)。





059 欣斤塔剌思州

欣斤塔剌思(Chingtalas)州,亦在沙漠边地,处西北方与北方间,广十六日程。隶属大汗。境内有环以墙垣之城村不少。居民有三种,曰偶像教徒、曰回教徒、曰若干聂思脱里派之基督教徒。此州北边有一山,内藏良钢与翁苔里克(ondanique)之矿脉。君等应知此山并有一种矿脉,其矿可制火鼠(salamandre)。

须知此火鼠非兽,如我辈国人之所云,实为采自地中之物。其法如下:

由其性质,此物非兽无疑,盖凡动物皆为四元素所结合,不能御火也。马可·波罗有一突厥伴侣名称苏儿非哈儿(Surfiar),广有学识,为大汗尽职于此地者三年。采取火鼠以献大汗。据称,掘此山中,得此矿脉。取此物碎之,其中有丝,如同毛线。曝之使干,既干,置之铁臼中。已而洗之,尽去其土,仅余类似羊毛之线,织之为布。布成,色不甚白。置于火中炼之,取出毛白如雪。每次布污,即置火中使其色白。

上所言关于火鼠之事皆实,土人之言亦复如此。其言有异者,则妄言也。君等应知大汗曾将一极美之火浣布献之罗马教皇,以供包裹耶稣基督圣骸之用。

兹置此州不言,请言东北方与东方间之其他诸地。





060 肃州

从前述之州首途,在东北方及东方间骑行十日:道中毫无民居,虽有亦等若无有,所以在本书中无足记者。

行此十日毕,抵一别州,名曰肃州(Suctur)。境内有环以墙垣之城村不少,而其要城即名肃州。居民是基督教徒或偶像教徒,并臣属大汗。

前此所言之三州,并属一大州,即唐古忒也。

如是诸州之山中并产大黄甚富,商人来此购买,贩售世界,居民恃土产果实为活。

兹置此事不言,请言别一城,其城名曰甘州。





061 甘州城

甘州(Campicion)是一大城,即在唐古忒境内,盖为唐古忒全州之都会,故其城最大而最尊。居民是偶像教徒、回教徒及基督教徒。基督教徒在此城中有壮丽教堂三所。偶像教徒依俗有庙宇甚多,内奉偶像不少,最大者高有十步,余像较小,有木雕者,有泥塑者,有石刻者,制作皆佳,外傅以金,诸像周围有数像极大,其势似向诸像作礼。

关于偶像教徒者,前此尚未尽言,兹请为君等述之。

其遵守偶像教徒之僧人,生活较之他人正直。彼等禁止淫逸,然不视之为大罪,但对于犯男色者罚以死罪。彼等有一教会日历,与我辈同。每月有五日谨守斋戒,不杀生,不食肉,节食甚于他日。

其地之人娶妻致有三十。否则视其资力,娶妻之数惟意所欲。然第一妻之地位为最尊。诸妻中有不善者得出之,别娶一人。男子得娶从姊妹,或其父已纳之妇女为妻,然从不娶其生母。总之,其人生活如同禽兽。

玛窦阁下及马可·波罗曾奉命留居此城垂一年。

兹置此事不言,请言北方诸州,盖吾人将从此方向继续旅行六十日也。





062 亦集乃城

从此甘州城首途,若骑行十六日,可抵一城,名曰亦集乃(Edzina)。城在北方沙漠边界,属唐古忒州。居民是偶像教徒。颇有骆驼牲畜,恃农业牧畜为生。盖其人不为商贾也。其地产鹰甚众。行人宜在此城预备四十日粮,盖离此亦集乃城后,北行即人沙漠。行四十日,冬季酷寒,路绝人烟,亦无草木。惟在夏季始见有人。其中亦见野兽,缘有若干处所有小松林也。行此四十日沙漠毕,抵一北方之州,请为君等言之。





063 哈剌和林城

哈剌和林(Karakorum)城延袤三哩,是为昔日鞑靼人离其本地以后所据之第一城。兹请为君等详述鞑靼人发展其势力之经过。

昔日鞑靼人确居北方,距主儿扯(Ciorcia)人之地不远。其地是大平原,无城无堡。然有良好牧地,巨大河流,多数水道。地广而风景美丽,且无君长,然每年纳贡赋于一大君。其方言名之曰王罕(Wang-khan),法兰西语犹言长老约翰(Prêtre-Jean)也。世传权力甚大之长老约翰,即指此人,所纳之贡赋,每牲畜十头缴纳一头,此外他物亦十分取一。

迨其人繁殖既众,长老约翰恐其为患,欲以之散处数地,于是命其臣一人执行。鞑靼人闻之忧甚,遂群聚不散,自此地出发,渡一沙漠,徙于其北别一地方。地远不受长老约翰之害,由是离叛,不复献纳贡赋有若干时。





064 成吉思之为鞑靼第一汗

基督诞生后1187年时,鞑靼人推选一大勇大智大有手腕之人为王,其名曰成吉思汗。散处诸地之鞑靼人,闻其当选,悉皆归心,奉之为主。而彼亦能善治其部。

鞑靼归之甚众,成吉思见部众已多,乃大积戈矛及其他兵器,率之侵略此地带内八州之地。占据其地以后,不扰居民,亦不损其财物,仅留部将数人统率一部分之部众镇守,尽驱余众侵略他州。于是得地甚众。侵地内之居民,见其能为之免战士之扰,而毫未受害,于是皆乐而归顺,为之效忠。

迨至其聚集其地全境之人时,遂欲进而侵略世界一大部分之地。基督诞生后1200年,遣使往长老约翰所,言欲娶其女为妻。长老约翰闻言恚甚,语使者曰:“汝主缘何如此无礼,敢求娶吾女为妻。彼应知彼为我之奴仆。可归告汝主,我宁焚杀吾女,而不畀之为妻。论理我当处汝主死,以为叛逆不忠者戒。”语毕,立命使者行,不许再至其前。

使者闻言,疾驰归报,尽述长老约翰之词,一无所隐。





065 成吉思集军进攻长老约翰

成吉思闻听长老约翰辱己之言,心腹膨胀,愤懑几至于裂,盖其人意气甚高也。已而厉声呼曰:“不报此从来未受之大辱,枉为部主。”呼声甚高,左右尽闻。

于是集其一切军队,一切臣民,大为前此未闻未见之战备。遣人往告长老约翰善治防守。长老约翰闻成吉思汗率大军来攻之确讯,以为戏言。尚谓成吉思非战士,有何能为。但亦召集一切臣民,征发一切兵力,大筹战备。俾成吉思汗至,得俘而杀之。其所集军队内有种种外国之人,其数之众,得谓为世界最大不可思议之举。

彼此如此备战,所以我言之甚长。成吉思汗率其全军至一美丽而甚大之平原,其名曰天德(Tenduc),隶属长老约翰,结营于此,其人之众,虽成吉思汗本人亦不知其数。及闻长老约翰将至,心中大欢。盖此地广大,适于战争,所以极愿其来,乃留此以待。

暂置成吉思汗及其军队不言,请言长老约翰及其军队。





066 长老约翰进击成吉思

史载长老约翰闻成吉思及其军来攻之时,即率其众出发,兼程进至此天德平原,结营距成吉思营二十哩。彼此两军休息二日,以养士气,俾能剧战。

两军在此天德平原结营之情形,如上所述。一日,成吉思汗召基督教及回教之星者来前,命卜战之胜败,胜者为本军,抑为长老约翰军。回教星者卜之,不能言其实。基督教星者则明示其吉凶,命人持二杖至,中劈之为两半,分置二处,不许人触之。名此一半杖曰成吉思汗,彼一半杖曰长老约翰。谓今可注目视之,将见胜利谁属。脱有牛杖自就彼半杖而覆于其上者,则为胜军。

成吉思汗答言极愿视此,命立为之。由是基督教之星者口诵圣诗集中之诗一篇,作其法术,于众目睽视之下,忽见名成吉思汗之半杖,未经何人手触,自就名长老约翰之半杖而覆于其上。成吉思汗见之大喜,顾后来战事果如基督教徒所卜。由是厚礼基督教徒,视其为能言真理之人。





067 成吉思汗与长老约翰之战

两军休息二日后,遂进战,战甚剧烈,是为世人从来未见之大战。双方死亡甚夥,最后成吉思汗胜敌,长老约翰殁于阵中。是日以后,成吉思汗逐渐侵略其全土。此战以后,成吉思汗君临者六年。在此时间之中,侵略州郡城堡,为数甚众。至第六年终,进围一名哈剌图(Calatuy)要塞之时,膝上中流矢死。世人惜之,因其为人勇智也。

鞑靼人有其第一君长名曰成吉思汗之事实,暨其战胜长老约翰之情形,既已备述于前。兹请言嗣后君临之人,与夫鞑靼人之风习。





068 成吉思汗后之嗣君及鞑靼人之风习

君等须知此第一君主成吉思汗之后,首先继承大位者,是贵由汗(Cuy-khan)。第三君主是拔都汗(Batuy-khan)。第四君主是阿剌忽汗(Alacou-khan)。第五君主是蒙哥汗(Mangoukhan)。第六君主是忽必烈汗(Koubilai-khan),即现时(1298年)在位之君主也。其权较强于前此之五君,盖合此五人之权,尚不足与之抗衡。更有进者,虽将全世界之基督教同回教帝王联合,其力及其事业亦不及此忽必烈汗之大。此汗为世界一切鞑靼之君主,统治东方西方之鞑靼。缘鞑靼皆是其臣民。此大权我将在本书中为君等切实言之。

君等并应知一切大汗及彼等第一君主之一切后裔,皆应葬于一名阿勒台(Altai)之山中。无论君主死于何地,皆须运葬于其中,虽地远在百日程外,亦须运其遗骸葬于此山。

尚有一不可思议之事,须为君等述者。运载遗体归葬之时,运载遗体之人在道见人辄杀,杀时语之云:“往事汝主于彼世。”盖彼等确信凡被杀者皆往事其主于彼世。对于马匹亦然,盖君主死时,彼等杀其所乘良马,俾其在彼世乘骑。蒙哥汗死时,在道杀所见之人二万有余,其事非虚也。

吾人既开始叙述鞑靼,请再续言他事。鞑靼冬居平原,气候温和而水草丰肥足以畜牧之地。夏居冷地,地在山中或山谷之内,有水林牧场之处。其房屋用竿结成,上覆以绳,其形圆,行时携带与俱,交结其竿,使其房屋轻便,易于携带。每次编结其屋之时,门皆向南。彼等有车,上覆黑毡甚密,雨水不透。驾以牛驼,载妻儿于其中。妇女为其夫做一切应作之事,如买卖及家务之事皆属之。盖男子仅为打猎、练鹰,作适于贵人之一切武事也。

彼等以肉乳猎物为食,凡肉皆食,马、犬、鼠、田鼠(pharaons)之肉,皆所不弃,盖其平原窟中有鼠甚众也。彼等饮马乳。鞑靼人无论如何不私他人之妻,盖其视此事为恶行也。妇女对其夫驯良忠顺,为其分内应为诸事。

婚姻之法如下:各人之力如足赡养,可娶妻至于百数。然视第一妻为最驯良。赠聘金于其妻,或妻之父母。待等所生之子,较他人为众,盖其妻多如上所述也。鞑靼可娶其从兄妹,父死可娶其父之妻,惟不娶牛母耳。娶者为长子,他子则否,兄弟死亦娶兄弟之妻。婚时大行婚礼。





069 鞑靼人之神道

君等须知其信仰如下所云:彼等有神,名称纳赤该(Nacigay),谓是地神,而保佑其子女、牲畜、田麦者,大受礼敬。各置一神于家,用毡同布制作神像,并制神妻神子之像,位神妻于神左,神子之像全与神同。食时取肥肉涂神及神妻神子之口,已而取肉羹散之家门外,谓神及神之家属由是得食。

鞑靼人饮马乳,其色类白葡萄酒,而其味佳,其名曰忽迷思(Koumiss)。衣金锦及丝绢,其里用貂鼠、银鼠、灰鼠狐之皮制之。其甲胄皆美,而价甚巨。其兵器有弓箭、剑、骨朵,然常用弓,缘其人善射,世无与比。背负熟皮甲,坚甚,其人为良武士,勇于战斗,能为他人所不能为。数作一月行,不携粮秣,只饮马乳,只食以弓猎得之兽肉。马牧于原,盖其性驯良,无需以大麦、燕麦、草料供其食也。此种鞑靼人能耐劳苦,食少,而能侵略国土,世人无能及之,是以今日为世界一大部分之主人。其军队编制甚善,说如下方。

君等应知一鞑靼君主之作战,若率万骑,则命一人长十人,一人长百人,一人长千人,一人长万人,俾其本人只将十人,而彼十人亦各将十人,以次类推。将士服从,统率极易。此外彼等名十万人为一秃黑(tuc),万人为一土绵(toman),千人为一敏黑(ming),百人为一忽思(guz),十人为一温(on)。行军时常有二百骑前行,距大军二日程巡逻。后队及两翼亦有巡逻者。四面皆有防守,不易为敌所袭。远征时,不负甲胄,仅各携二皮囊,以置所饮之乳,一煮肉之土釜,一避雨之小帐。设须急行,则急驰十日,不携粮,不举火,而吸马血,破马脉以口吸之,及饱则裹其创。彼等亦有干乳如饼,携之与俱,欲食时,则置之水中溶而饮之。

其作战胜敌之法如下:此辈不以退走为耻,盖退走时回首发矢射敌,射极准,敌人大受伤。马受训练,往回疾驰,惟意所欲,虽犬亦不能如其迅捷,则其退走战亦不弱于相接战。盖退走时向追者发矢甚多,追者自以为胜,不虞及此也。及见敌骑死伤,则皆回骑,大呼进击破敌。

盖彼等极骁勇耐劳,敌人见其奔逃而自以为获胜时,实不自知为败亡之征,而鞑靼将乘势回击也。其用此法取胜之例不少。

前所言者,乃真正鞑靼之生活及风习,然今日则甚衰微矣。盖其居留契丹者染有偶像教之积习,自弃其信仰。而居留东方者则采用回教徒之风习也。

其治理狱讼之法如下:有窃一微物者,杖七下,或十七,或二十七,或三十七,或四十七,而止于一百零七,视其罪大小而异。有时被杖至死者。设有盗马一骑或其他重要物品者,则为死罪,处以腰斩之刑。然应附带言及者,其罪可以买赎,偿窃物之九倍则免。各君主或他人之畜养牲畜,如马、牛、骆驼及其他大牲畜,在畜身上做一记号,任其牧于野中,不用人看守。各主之畜混牧一处,赖有记号,可以辨识,牧后各归其主。小牲畜则命牧人守之,其躯大而且肥。

彼等尚有另一风习,设有女未嫁而死,而他人亦有子未娶而死者,两家父母大行婚仪,举行冥婚。婚约立后焚之,谓其子女在彼世获知其已婚配。已而两家父母互称姻戚,与子女在生时婚姻者无别。彼此互赠礼物,写于纸上焚之。谓死者在彼世获有诸物。

鞑靼人之风习既已叙述于前。至若君临一切鞑靼的大汗及其宫廷之事,将在本书中随时言之,盖其事亦奇也。兹请接述前文初人平原时之事迹。





070 哈剌和林平原及鞑靼人之种种风习

若从哈剌和林同前文所述鞑靼诸主埋葬遗骸之阿勒台山首途,北行四十日,抵一高原,名曰巴儿忽(Bargou)平原。居民名称蔑克里惕(Mékrites),是为一种蛮野部族,恃其牲畜为活。风习与鞑靼人同,隶属大汗。其人无麦无酒,夏日猎取鸟兽甚夥,然冬日严寒则无所得。

又从此大平原骑行四十日,抵于海洋。其处有山,山中有隼(fauconpelerin)做巢。此外山中无男无女,无鸟无兽,仅有一种飞鸟名曰巴儿格儿剌黑(barguerlac),供隼之食。此鸟大如鹧鸪,爪如鹦鹉,尾如燕,飞甚捷。盖因严寒,故无动物居处其间。大汗欲得此做巢之隼时,则遣人取之。此海诸岛亦产海青(gerfaut)。地在极北之处,中午可见北极之星。海青甚多,君主欲得之者,可以取之不尽。君等切勿以为取得海青之基督教徒以海青贡献大汗,其实乃贡献东方君主者也。

此北方诸州迄于地尽大海之处,既已备述于前,兹将言往谒大汗沿途所经之其他诸州,所以吾人重返本书业已叙述之甘州。





071 额里湫国

从前此已言之甘州首途,骑行五日,夜间多闻鬼声。行此五日毕,东向有国,名曰额里湫(Erginul)。臣属大汗,隶唐古忒州。时此州内有数国,居民是聂思脱里派之基督教徒,或偶像教徒,或崇拜摩诃末之教徒。

此国之中,多有城市,其要城名曰凉州。从此城向东南行,可至契丹之地,在此方道上见有一城,名称申州(Singuy)。所辖城村甚夥,亦属唐古忒,隶于大汗。居民是偶像教徒同回教徒,然亦有基督教徒。地产野牛,身大如象,其形甚美,盖牛毛被覆全身,仅露其脊,毛长四掌,呈黑白色,其美竟至不可思议。牛幼时即畜养之,所以为数颇众。用以负载,并命作其他诸事,且用以耕种,缘其力大,耕地倍于他畜也。

此地有世界最良之麝香,请言其出产之法如下:此地有一种野兽,形如羚羊,蹄尾类羚羊,毛类鹿而较粗,头无角,口有四牙,上下各二,长三指,薄而不厚,上牙下垂,下牙上峙。兽形甚美。取麝之法如下:捕得此兽以后,割其脐下之血袋。袋处皮肉之间,连皮割下,其中之血即是麝香。其味甚浓,此地所产此兽无算。

居民是商贾工匠,出产小麦甚饶。地广二十六日程。中产野鸡,大倍吾人之雉,尾长十掌。别有其他种种禽鸟,羽毛具有各色,其形甚丽。信仰偶像之人,体肥,鼻小,头发黑色,微有须,而无髯。女子除头发外,遍身无毛,色白而美。居民淫逸,娶妻甚多,盖其教与其俗皆无此禁。女虽微贱,第若美丽,国之大贵人不惜与之为婚,并赠女之父母以多金。

兹从此地前行,请言东方之别一州。





072 额里哈牙国

如从凉州首途,东进,骑行八日,至一州,名曰额里哈牙(Egrigaia)。隶属唐古忒,境内有城堡不少,主要之城名哈剌善(Calachan)。居民是偶像教徒,然有聂思脱里派之基督教堂三所。其人臣属大汗城中制造驼毛毡不少,是为世界最丽之毡,亦有白毡,为世界最良之毡盖以白骆驼毛制之也。所制甚多,商人以之运售契丹及世界各地。今从此州东行,将言一名天德(Tenduc)之州,由是进入昔属长老约翰之地。





073 天德州及长老约翰之后裔

天德是向东之一州,境内有环以墙垣之城村不少,主要之城名曰天德。隶属大汗,与长老约翰之一切后裔隶属大汗者同。此州国王出于长老约翰之血统,名称阔里吉思(George),受地于大汗,然所受者非长老约翰旧据之全土,仅其一部分而已。然我应为君等言者,此长老约翰族之国王皆尚主,或娶大汗之女,或娶皇族公主为妻。

此州有石可制琉璃(azur)其质极细,所产不少。州人并用驼毛制毡甚多,各色皆有。并恃畜牧务农为生,亦微作工商。治此州者是基督教徒,然亦有偶像教徒及回教徒不少。此种持有治权之基督教徒,构成一种阶级,名曰阿儿浑(Argon),犹言伽思木勒(Gasmoul)也。其人较之其他异教之人形貌为美,知识为优,因是有权,而善为商贾。

君等应知昔日长老约翰统治鞑靼时,即定都于此天德城中。今其后裔尚居于是,盖前此已言此阔里吉思国王出其血统,其实为长老约翰以后之第六君主也。

此地即吾人所称峨格(Gog)同马峨格(Magog)之地。其人则自称曰汪格(Ung)同木豁勒(Mugul)。盖在此州中原有二种人,先鞑靼人居住其地,汪格人是土著,木豁勒人则为鞑靼,所以鞑靼人常自称曰木豁勒,而不名曰鞑靼。

由此州东向骑行七日,则抵契丹(Cathay)之地。此七日中,见有城堡不少,居民崇拜摩诃末,然亦有偶像教徒及聂思脱里派之基督教徒。以商工为业,制造金锦,其名曰纳石失(nasich)、毛里新(molisins)、纳克(naques)。并织其他种种绸绢,盖如我国之有种种丝织毛织等物,此辈亦有金锦同种种绸绢也。

其人皆属大汗,其地有一城,名曰申达州(Suydatuy、Syndatny)。居民多以制造君主臣下之武装为业。此州有一山,中有银矿甚佳,采量不少,其名曰伊的非儿(ydifir)。居民多游猎养鸟。

兹从此州首途,远行三日。三日后,至一城,名曰察罕脑儿(Tchagan-nor)。中有大宫一所,属于大汗。周围有湖川甚多,内有天鹅,故大汗极愿居此。其地亦有种种禽鸟不少,周围平原颇有白鹤、鹧鸪、野鸡等禽,所以君主极愿居此以求畋猎之乐,在此驯养鹰隼海青,是即其乐为之艺也。

此地有鹤五种,一种躯甚大,身黑如乌。第二种全白,其翼甚美,其圆眼上呈金色,此鹤为诸类中之最大者。第三种与我辈地方所产者同。第四种较小,耳旁有长羽甚美,下垂作红黑色。第五种甚大,全身灰色,头呈红黑色。此城附近有一山谷,君主建数小屋于其中,畜养鹧鸪无数,命数人守之,大汗至时,取之惟意所欲。

兹吾人更向北方及东北方远行三日。





074 上都城

从上述之城首途,向北方及东北方间骑行三日,终抵一城,名曰上都,现在在位大汗之所建也。内有一大理石宫殿,甚美,其房舍内皆涂金,绘种种鸟兽花木,工巧之极,技术之佳,见之足以娱人心目。

此宫有墙垣环之,广袤十六哩,内有泉渠川流草原甚多。亦见有种种野兽,惟无猛兽,是盖君主用以供给笼中海青、鹰隼之食者也。海青之数二百有余,鹰隼之数尚未计焉。汗每周亲往视笼中之禽,有时骑一马,置一豹于鞍后。若见欲捕之兽,则遣豹往取,取得之后,以供笼中禽鸟之食,汗盖以此为乐也。

此草原中尚有别一宫殿,纯以竹茎结之,内涂以金,装饰颇为工巧。宫顶之茎,上涂以漆,涂之甚密,雨水不能腐之。茎粗三掌,长十或十五掌,逐节断之。此宫盖用此种竹茎结成。竹之为用不仅此也,尚可做屋顶及其他不少功用。此宫建筑之善,结成或拆卸,为时甚短,可以完全拆成散片,运之他所,惟汗所命。给成时则用丝绳二百余系之。

汗在此草原中,或居大理石宫,或居竹宫,每年三阅月,即六月、七月、八月是已。居此三月者,盖其地天时不甚炎热而颇清凉也。迨至每年八月二十八日,则离此他适。君等应知汗有一大马群,马皆牝马,其色纯白,无他杂色,为数逾万。汗与其族皆饮此类牝马之乳,他人不得饮之。惟有一部落,因前此立有战功,大汗奖之,许饮此马乳,与皇族同。此部落人名称曰火里牙惕(Horiad)。

此种牝马经行某地,贵人见之者,不论其地位如何高贵,须让马行。否则绕道半日程以避之。盖无人敢近此马,见之宜行大礼。每年八月二十八日,大汗离此地时,尽取此类牝马之乳,洒之地上。缘其星者及偶像教徒曾有言曰,每年八月二十八日,宜洒乳于地,俾地上空中之神灵得享,而保佑大汗及其妻女财产,以及国内臣民,与夫牲畜、马匹、谷麦等物。洒乳以后,大汗始行。

有一异事,前此遗忘,今须为君等述之者。大汗每年居留此地之三月中,有时天时不正,则有随从之巫师星者,谙练巫术,足以驱除宫上之一切风云暴雨。此类巫师名称脱孛惕(Tebet)及客失木儿(Quesimour),是为两种不同之人,并是偶像教徒。盖其所为者尽属魔法,乃此辈诳人谓是神功。此辈尚有别一风习,设有一人犯罪,依法处决者,取其尸体熟而食之,然善终之尸体则不食。

尚有别一异事为此二种人所能为者,亦请为君等述之。大汗在其都城大宫之内,坐于席前。席高八肘,位于廷中。其饮盏相距至少有十步之远,内盛酒或其他良好饮料。此辈巫师巫术之精,大汗欲饮酒时,致能做术使饮盏自就汗前,不用人力。此事常见之,见之者不只万人,此乃实事,毫无伪言。我国术人明悉巫术者,将告君等此事洵可为之也。

偶像之节庆既届,此辈巫师往告大汗曰:“我辈某偶像节庆之期已届(言时举其名)。陛下深知若无祭享,此偶像将使天时不正,损害吾人财产。所以请赐黑首之羊若干以享之,并请颁给沉香、檀香及他物若干(此辈任意索取各物),以备奉祀我辈偶像,俾其默佑我辈之一切财物。”

于是大汗命左右诸臣如数付之。诸巫师得之以后,遂往享其偶像。大燃灯火,焚数种香,熟祭肉,置于偶像前。已而散之于各处,谓其偶像可以取之,惟意所欲。其庆贺之法概如是也。各偶像各有其名,各有其节庆之日,一如我辈圣者每年有其纪念之日也。

此辈亦有广大寺院,其大如一小城。每寺之中有僧二千余人,衣服较常人为简。须发皆剃。其中有娶妻而有多子者。

尚有别种教师名称先生(sensin),守其教戒,节食苦修,终身仅食糠,浇以热水,此外不食他物,仅饮水,日日持斋,是盖为一种过度苦行生活也。此辈亦有其大偶像,为数不少。然偶亦拜火,及其他不属本派之偶像。不娶妻室。其衣黑色而兼蓝色,卧于席上。其生活之苦竟至不可思议。其偶像皆女形,质言之,其名皆属女名也。

兹置此事不言,请为君等叙述“诸汗之大汗”之伟迹异事,是为鞑靼人之大君,其名曰忽必烈,极尊极强之君主也。





075 大汗忽必烈之伟业

现在君临之大汗,名称忽必烈汗,今特述其伟业,及其朝廷一切可以注意之事实,并其如何保持土地、治理人民之方法。

今首先在本书欲言者,乃为现在(1298年)名称忽必烈汗的大汗之一切丰功异绩。忽必烈汗,犹言诸君主之大君主,或皇帝。彼实有权被此名号,盖其为人类元祖亚当(Adam)以来迄于今日世上从来未见广有人民、土地、财货之强大君主。我将于本书切实言之,俾世人皆知我言尽实,皆知其为世上从来未有如此强大之君主。君等将在本书得悉其故。





076 大汗征讨诸父乃颜之大战

应知此忽必烈汗为成吉思汗之直系后人,世界一切鞑靼之最高君主,序在第六,前已言之。基督诞生后1256年时,彼始以睿智英武而得国。其为人也,公正而有条理,初即位时,诸弟与诸宗族与之争位,然彼以英武得之。且论权力与夫道理,彼为帝系之直接继承人,应得国也。

自其即位以后,迄于现在基督诞生后之1298年,在位已有四十二年,其年龄约有八十五岁,则其即位时已有四十三岁矣。未即位前数临戎阵,作战甚勇。但自为君以后,仅有一次参加战争。事在基督诞生后1286年时,兹请为君等叙述此战之缘由。

时有一鞑靼大君主名称乃颜(Nayan),乃此忽必烈汗之诸父。年事正幼,统治国土州郡甚多。自恃为君,国土甚大,幼年骄傲,盖其战士有三十万骑也。然在名分上彼实为其侄大汗忽必烈之臣,理应属之。

然彼自恃权重,不欲为大汗之臣,反欲夺取其国,遂遣使臣往约别一鞑靼君主海都(Kaidon)。海都者,乃颜之族而忽必烈之侄也。势颇强盛,亦怨大汗而不尽臣节。乃颜语之云:“我今聚全力往攻大汗,请亦举兵夹攻,而夺其国。”

海都闻讯大喜,以为时机已至,乃答之曰,行将举兵以应,于是集兵有十万骑。

兹请言闻悉此种叛事之大汗。





077 大汗进讨乃颜

大汗闻悉此事之时,洞知彼等背理谋叛,立即筹备征讨,盖其为人英明,凡事皆不足使之惊异。并有言曰,若不讨诛此叛逆不忠之鞑靼二王,将永不居此大位。

筹备战事秘密迅速,十日或十二日间,除其近臣以外,无人能悉其事者。征集骑兵三十六万,步兵十万,所征士卒如此之少者,盖仅征集手边队伍。余军无数,曾奉命散戍各州各地,非短期中所能调集。脱将一切兵力集中,其数无限,殆未能言之,虽言之亦无人信之。而此三十六万人仅为其养鹰人及左右之猎户也。

迨其征集此少数军队以后,命其星者卜战之吉凶,星者卜后告之曰,可以大胆出兵,将必克敌获胜,大汗闻之甚喜。遂率军行,骑行二十日,抵一大原野。乃颜率其全军四十万骑屯驻其中。大汗士卒薄晓倏然进击。他人皆未虞其至。缘大汗曾遣谍把守诸路,往来之人悉被俘掳。乃颜不意其至,部众大惊。大汗军抵战场之时,乃颜适与其妻共卧帐中。忽必烈汗预知其宠爱此妇,常与同寝,故特秘密进军,薄晓击之。





078 大汗讨伐叛王乃颜之战

比曙,汗及全军至一阜上,乃颜及其众安然卓帐于此,以为无人能来此加害彼等。其自恃安宁不设防卫之理,盖因其不知大汗之至。缘诸道业被大汗遣人防守,无人来报。且自恃处此野地远距大汗有三十日程,不虞大汗率其全军疾行二十日而至也。

大汗既至阜上,坐大木楼,四象承之,楼上树立旗帜,其高各处皆见。其众皆合三万人成列,各骑兵后多有一人执矛相随,步兵全队皆如是列阵,由是全地满布士卒,大汗备战之法如此。

乃颜及其众见之大惊,立即列阵备战,当两军列阵之时,种种乐器之声及歌声群起,缘鞑靼人作战以前,各人习为歌唱,弹两弦乐器,其声颇可悦耳。弹唱久之,迄于鸣鼓之时,两军战争乃起,盖不闻其主大鼓声不敢进战也。

当诸军列阵弹唱以后,大汗鼓鸣之时,乃颜亦鸣鼓,由是双方部众执弓弩、骨朵、刀、矛而战,其迅捷可谓奇观。人见双方发矢蔽天,有如暴雨。人见双方骑卒坠马而死者为数甚众,陈尸满地。死伤之中,各处大声遍起,有如雷震,盖此战殊烈,见人辄杀也。

是战也,为现代从未见之剧战,从未见疆场之上战土、骑兵有如是之众者。盖双方之众有七十六万骑,可云多矣,而步卒之多尚未计焉。混战自晨至于日中,然上帝与道理皆以胜利属大汗。乃颜败创,其众不敌大汗部众之强,失气败走。乃颜及其诸臣悉被擒获,并其兵器执送大汗之前。乃颜为一受洗之基督教徒,旗帜之上以十字架为徽志,然此毫无裨于彼。盖其与诸祖并受地于大汗,既为大汗之臣,不应背主而谋叛也。





079 (一)大汗之诛乃颜

大汗知乃颜被擒,甚喜。命立处死,勿使人见,盖虑其为同族,恐见之悯而宥其死也。遂将其密裹于一毡中,往来拖曳,以至于死。盖大汗不欲天空、土地、太阳见帝族之血,故处死之法如此。

大汗讨平此乱以后,乃颜所领诸州之臣民,悉皆宣誓尽忠于大汗。先是隶于乃颜之州有四,一名主儿扯(Ciorcia),二名高丽(Cauly),三名不剌思豁勒(Brascol),四名西斤州(Sighingiu),合此四州为一极大领土。

乃颜所领四州之民为偶像教徒及回教徒,然其中亦有若干基督教徒。大汗讨灭乃颜以后,此四州之种种人民遂揶揄基督教徒及乃颜旗帜上之十字架,讥其不能持久,其语若曰:“乃颜既奉基督教而崇拜十字架,汝辈天主之十字架援助乃颜,如是而已。”此语喧传,致为大汗所闻。

大汗闻知以后,严责揶揄基督教徒之人,而语基督教徒曰:“汝等应自慰也,十字架未助乃颜,盖有其大理存焉。若为善物,其所行应当如是。乃颜叛主不忠,应当受罚。汝辈天主之十字架不助之为逆,甚是。”

大汗发言声音甚高,各人皆闻。基督教徒答曰:“大汗之言诚是。我辈之十字架不欲援助罪人。其不助乃颜谋逆做乱者,盖其不欲助之为恶也。”自是以后,遂无有人讥讽基督教徒。缘其已闻大汗对于基督教徒所言乃颜旗上之十字架未助乃颜之理也。





079 (二)大汗对于基督教徒犹太教徒回教徒佛教徒节庆付与之荣誉及其不为基督教徒之理由

大汗得胜以后,盛陈卤簿,凯旋人其名称汗八里(Cambaluc)之都城,时在11月之中也。驻跸此城讫于2月杪,或3月吾人复活节届之时,应知此节为吾人重要节庆之一。大汗届时召大都之一切基督教徒来前,并欲彼等携内容四种福音之《圣经》俱来。数命人焚香,大礼敬奉此经,本人并虔诚与经接吻,并欲在场之一切高官大臣举行同一敬礼。彼对于基督教徒主要节庆,若复活节、诞生节等节,常遵例为之。对于回教徒、犹太教徒、偶像教徒之主要节庆,执礼亦同。脱有人询其故,则答之曰:“全世界所崇奉之预言人有四,基督教徒谓其天主是耶稣基督,回教徒谓是摩诃末,犹太教徒谓是摩西(Mosie),偶像教徒谓其第一神是释迦牟尼(Cakya Mouni)。我对于兹四人,皆致敬礼,由是其中在天居高位而最真实者受我崇奉,求其默佑。”然大汗有时露其承认基督教为最真最良之教之意。盖彼曾云,凡非完善之事,此教决不令人为之。大汗不欲基督教徒执十字架于前,盖因此十字架曾受耻辱,而将一完善伟大之人如基督者处死也。

或曰,彼既以基督教为最良,缘何不皈依此教,而为基督教徒欤?

曰,其理由如下:尼古剌、玛窦阁下二人常以基督教理语大汗,大汗曾遣之为使臣,往使教皇所。并告之日:“汝辈欲我为基督教徒,特未解我心。此国之基督教徒蠢无所知,庸碌无用。至若偶像教徒则能为所欲为。我坐于席前时,置于中庭之盏满盛酒浆者,不经人手接触,可以自来就我饮。天时不正时,此辈可以使之正。所为灵异甚多,汝辈谅已知之。其偶像能言,预告彼等所询之事。脱我皈依基督之教,而成为基督教徒,则不识此教之臣民语我曰,汗因何理由受洗而信奉基督教,汗曾见有何种灵异何种效能欤?汝等应知此处之偶像教徒断言其能为灵异,乃由其偶像之神圣与威权而能为之。脱以此语见询,我将无以作答。此种偶像教徒既借其咒语、学识能为种种灵异,我若铸此大错,此辈不难将我处死。汝等奉命往谒教皇时,可求其遣派汝教中有学识者百人来此,俾其能面责此种教徒行为之非。并告之曰,彼等亦能为之,特不欲为者,盖因此为魔术耳。脱能如是驳击偶像教徒,使此辈法术不能在彼等之前施行,复经吾人身亲目击,吾人行将禁止其教,放逐其人,而受洗礼。我受洗以后,我之一切高官大臣暨一切服从彼等之人必将效法,由是此国之基督教徒将较汝辈国中为多矣。”

教皇若曾派遣可能宣传吾辈宗教之人,大汗必已为基督教徒,盖其颇有此意,此事之无可疑者也。





080 大汗还汗八里城

大汗讨灭乃颜以后,还其汗八里都城,大行庆赏。别一鞑靼君主名海都者,闻乃颜败亡之讯,甚痛,遂止兵,盖其恐陷乃颜覆辙也。

大汗仅为一次亲征,前已言之,即此一役而已。盖其他一切诸役,皆遣其诸子或其诸臣代往,仅有此役不欲他人代行,缘此叛逆乃颜傲甚,事实重大而危险也。

兹置此事不言,请复言大汗之伟业。其血统及其年龄,前已言之。兹欲述者,奖赏诸臣战功之事。其为百夫长有功者升千夫长,千夫长升万夫长,皆依其旧职及战功而行赏。此外赐以美丽银器及美丽甲胄,加给牌符,并赐金银、珍珠、宝石、马匹。赐与之多,竟至不可思议。盖将士为其主尽力,从未见有如是日之战者也。

牌符之式如下,百夫长银符,千夫长金符或镀金符,万夫长狮头金符,兹请言其重量及其意义如下:

百夫长及千夫长之牌符各重一百二十钱(gros),万夫长之狮首符亦重一百二十钱,诸符并勒文于其上曰:“长生天气力里,大汗福荫里,不从命者罪至死。”

凡持此种牌符者,皆有特权在其封地内为其所应为诸事。其有十万人之大藩主,或一大军之统帅,牌符重逾三百钱。其上勒文如前所述,文下勒一狮形,狮下勒日月形,再下勒此符付与之特权。符之背面则勒命令。凡持此贵重牌符者,每骑行时,头上应覆一盖,其名曰伞,以一长矛承之,表示其为显贵之意。每坐时,则应坐于一银座上。

有时给海青符于此诸大藩主。持有此符者,权势如大汗亲临。持此符之人欲遣使至某地,得取其地之良马及他物,惟意所欲。

兹置此事不言,请言大汗之体貌风仪。





081 大汗之体貌风仪

君主的君主名称忽必烈的大汗之体貌如下:不长不短,中等身材,筋肉四肢配置适宜,面上朱白分明,眼黑,鼻正。有妇四人为正妇,此四妇诞生之长子,于父死后依礼应承袭帝位。此四妇名称皇后,然各人别有他名。四妇各有宫廷甚广,各处至少有美丽侍女三百,并有勇武侍臣甚众,及其他男女不少,由是每处合有万人。

大汗每次欲与此四妇之一人共寝时,召之至其室内,有时亦亲往就之。尚有妃嫔不少,兹请为君等叙其选择之法。

鞑靼有一部落名称弘吉剌(Ungrat),其人甚美。每年由此部贡献室女百人于大汗。命宫中老妇与之共处,共寝一床,试其气息之良恶,肢体是否健全。体貌美善健全者,命之轮番侍主。六人一班,三日三夜一易。君主内寝之事,悉由此种侍女司之,君主惟意所欲。三日三夜期满,另由其他侍女六人更番人侍。全年如是。概用三日三夜六人轮番入侍之法。





082 大汗之诸子

此四妇为大汗生男二十二人,最长者名称成吉思(Gengis)。盖追忆鞑靼第一君主成吉思汗而取此名也。此大汗长子成吉思应于父死后袭帝位,乃先死。遗一子,名铁木耳(Timour),应在其祖死后继承大汗位,缘其为大汗长子之子也。此铁木耳贤明英武,业在不少机会中证明。

并应知者,大汗别有二十五子,乃诸女友所出,皆为勇良武人,各为大藩主。四正妻所生之子,中有七人为大州大国之王,皆能善治其国,盖彼等皆贤明英勇。缘其父大汗为最贤明英武之人,兼为将兵之最大统帅,治国之最良君主,为一切鞑靼诸部落最勇之士卒所不能及。

大汗及其妻子既已备述于前,兹请言其朝廷宫殿。





083 大汗之宫廷

应知大汗居其名曰汗八里之契丹都城,每年三阅月,即12月、1月、2月是已。在此城中有其大宫殿,其式如下:

周围有一大方墙,宽广各有一哩。质言之,周围共有四哩。此墙广大,高有十步,周围白色,有女墙。此墙四角各有大宫一所,甚富丽,贮藏君主之战具于其中,如弓、菔、弦、矢、鞍、辔及一切军中必需之物是已。四角四宫之间,复各有一宫,其形相类。由是围墙共有八宫甚大,其中满贮大汗战具。但每宫仅贮战具一种,此宫满贮战弓,彼宫则满贮马辔,由是每宫各贮战具一种。

此墙南面辟五门,中间一门除战时兵马甲仗由此而出外,从来不开。中门两旁各辟二门,共为五门。中门最大,行人皆由两旁较小之四门出入。此四门并不相接,两门在墙之两角,面南向,余二门在大门之两侧。如是布置,确使此大门居南墙之中。

此墙之内,围墙南部中,广延一哩,别有一墙,其长度逾于宽度。此墙周围亦有八宫,与外墙八宫相类,其中亦贮君主战具。南面亦辟五门,与外墙同,亦于每角各辟一门。此二墙之中央,为君主大宫所在,其布置之法如下:

君等应知此宫之大,向所未见。宫上无楼,建于平地。惟台基高出地面十掌。宫顶甚高,宫墙及房壁满涂金银,并绘龙、兽、鸟、骑士形像及其他数物于其上。屋顶之天花板,亦除金银及绘画外别无他物。

大殿宽广,足容六千人聚食而有余,房屋之多,可谓奇观。此宫壮丽富赡,世人布置之良,诚无逾于此者。顶上之瓦,皆红黄绿蓝及其他诸色。上涂以釉,光泽灿烂,犹如水晶,致使远处亦见此宫光辉。应知其顶坚固,可以久存不坏。

上述两墙之间,有一极美草原,中植种种美丽果树。不少兽类,若鹿、獐、山羊、松鼠,繁殖其中。带麝之兽为数不少,其形甚美,而种类甚多,所以除往来行人所经之道外,别无余地。

由此角至彼角,有一湖甚美,大汗置种种鱼类于其中,其数甚多,取之惟意所欲。且有一河流由此出入,出入之处间以铜铁格子,俾鱼类不能随河水出入。

北方距皇宫一箭之地,有一山丘,人力所筑。高百步,周围约一哩。山顶平,满植树木,树叶不落,四季常青。汗闻某地有美树,则遣人取之,连根带土拔起,植此山中,不论树之大小。树大则命象负而来,由是世界最美之树皆聚于此。君主并命人以琉璃矿石满盖此山。其色甚碧,由是不特树绿,其山亦绿,竟成一色。故人称此山曰绿山,此名诚不虚也。

山顶有一大殿,甚壮丽,内外皆绿,致使山树宫殿构成一色,美丽堪娱。凡见之者莫不欢欣。大汗筑此美景以为赏心娱乐之用。

马可·波罗在此本及诸古本中,仅言皇宫,则所指者:(一)宽广各一哩之外墙,或今紫禁城之故址。(二)此墙之内别一南北较长之第二道城墙,亦即今日宫殿之南半部。(三)两墙中央之正殿。此外对于下章所言宽广各六哩之外墙,概未之及焉。

剌木学本叙述较有次第,自外墙及于中央,此外别有若干细情不见于诸原本者,兹录其文如下,用见北京初建时之遗迹:

“大汗常在名曰汗八里之大城中,每年居留三月……此城在契丹州之东北端,其大宫殿之所在也。宫与新城相接,在此城之南部,其式如下:

“先有一方墙,宽广各八哩。其外绕以深壕,各方中辟一门,往来之人由此出入。墙内四面皆有空地,广一哩,军队驻焉。空地之后复有一方墙,宽广各六哩,南北各辟三门,中门最大,常关闭,仅大汗出入时一为开辟而已。余二门较小,在大门之两侧,常开以供公共出入之用。

“此内墙四角及中央,各建一壮丽城楼。由是全墙周围共有八楼,贮大汗战具于其中。每楼仅贮战具一种,若一楼贮藏鞍辔及其他构成骑兵战具之类,别一楼贮藏弓菔弦矢及弓兵所用其他战具之类,第三楼贮藏甲胄及其他熟皮所制战具之类,其他诸楼由此类推。

“此第二方墙之内,有一第三城墙,甚厚,高有十步,女墙皆白色。墙方,周围有四哩,每方各有一哩,此第三墙辟六门,布置与第二城墙同。

“亦有城楼甚大,位置与第二墙之城楼相同,亦贮大汗之战具于中。

“第二、第三两墙之间,有树木草原甚丽。内有种种兽类,若鹿、麝、獐、山羊、松鼠等兽,繁殖其中两墙之间皆满。此种草原草甚茂盛,盖经行之道路铺石,高出平地至少有二肘(三尺)也。所以雨后泥水不留于道,皆下注草中,草原因是肥沃茂盛。此周围四哩墙垣之内,即为大汗宫殿所在。其宫之大,素所未见。盖其与上述城墙相接,南北仅留臣民士卒往来之路。宫中无楼,然其顶甚高。宫基高出地面十掌,四围环以大理石墙,厚有两步。其宫矗立于此墙中,墙在宫外,构成平台。其上行人外间可见,墙有外廊,石栏缘之。

“内殿及诸室墙壁刻画涂金,代表龙、鸟、战士、种种兽类、有名战事之形像。天花板之刻画亦只见有金饰绘画,别无他物。

“宫之四方各有一大理石级,从平地达于环绕宫殿之大理石墙上。朝贺之殿极其宽广,足容多人聚食。宫中房室甚众,可谓奇观,布置之善,人工之巧,无逾此者。屋顶为红绿蓝紫等色,结构之坚,可以延存多年。窗上玻璃明亮有如水晶。

“宫后(宫北)有大宫殿,为君主库藏之所,置金银、宝石、珍珠及其金银器具于中,妃嫔即居于此,惟在此处始能为所欲为,盖此处不许他人出入也。

“宫墙(四哩之墙)之外(之西),与大汗宫殿并立,别有一宫,与前宫同,大汗长子成吉思居焉。臣下朝谒之礼,与见其父同,盖其父死后由彼承袭大位也。

“大汗宫殿附近,北方一箭之地,城墙之中(皇城之中),有一丘陵,人力所筑,高百步……名曰绿山……

“更北城中(二十四哩城墙之中)有一大坑,深广。即以其土建筑上述之丘陵,掘后成坑。有一小渠贯注流水于其中,布置与一鱼池无异,诸兽皆来此饮水。此渠由上言丘陵附近旁之一水道流出,注入别一坑中。其坑亦宽广,处大汗宫及其子成吉思之宫间,其土亦曾供筑丘之用。

“后一坑中畜鱼种类甚多,以供御食,大汗取之惟意所欲。渠水由别端(南端)外流,其两端间以铜铁格子,畜鱼不能外出,其间亦见有天鹅及其他水禽。两宫之间有桥,通行水上……”(剌木学本第二卷第六章22页)

布莱慈奈德(Bretschneider)博士所撰《北京考古记》,裒辑中国史料不少,颇有足以参证马可·波罗之说者,兹广录之,以考旧迹。





084 (一)大汗太子之宫

尚应知者,大汗为其将来承袭帝位之子建一别宫,形式大小完全与皇宫无异,俾大汗死后内廷一切礼仪习惯可以延存。此王已受帝国印玺一方,然权力未备,大汗在生之时仍是大汗为主君也。

大汗及其子之宫殿,既已叙述于前,兹欲言者,其宫殿所在之契丹大城,及其营建之原因而已,此城名曰汗八里。

古昔此地必有一名贵之城名称汗八里,汗八里此言“君主城”也。大汗曾闻星者言,此城将来必背国谋叛,因是于旧城之旁,建筑此汗八里城。中间仅隔一水,新城营建以后,命旧城之人徙居新城之中。

此城之广袤,说如下方:周围有二十四哩,其形正方,由是每方各有六哩。环以土墙,墙根厚十步,然愈高愈削,墙头仅厚三步,遍筑女墙,女墙色白,墙高十步。全城有十二门,各门之上有一大宫,颇壮丽。四面各有三门五宫,盖每角亦各有一宫,壮丽相等。宫中有殿广大,其中贮藏守城者之兵仗。街道甚直,以此端可见彼端,盖其布置,使此门可由街道远望彼门也。

城中有壮丽宫殿,复有美丽邸舍甚多。城之中央有一极大宫殿,中悬大钟一口,夜间若鸣钟三下,则禁止人行。鸣钟以后,除为育儿之妇女或病人之需要外,无人敢通行道中。纵许行者,亦须携灯火而出。每城门命千人执兵把守。把守者,非有所畏也,盖因君主驻跸于此,礼应如是,且不欲盗贼损害城中一物也。既言其城,请言其人,以及朝廷之布置,并其他诸事。





084 (二)汗八里城之谋叛及其主谋人之处死

下所言者,皆实事也。有一会议,正式任命十二人组合成之,职司处分土地官爵及一切他物,惟意所欲。中有一人是回教徒,名称阿合马(Ahmed),为人较狡黠而有才能,权任甚重,颇得大汗宠任。大汗宠之甚切,任其为所欲为,但至阿合马死后,始知其曾用魔术蛊惑君主,致使言听计从,任其为所欲为。

此人管理政府一切官司,任命一切官吏,宣布一切裁判,其所厌恶之人而彼欲除之者,不问事之曲直,辄进谗言于大汗曰:“某人对于陛下不敬,罪应处死。”大汗则答之曰:“汝意所乐,为之可也。”于是阿合马立杀其人,其权力由是无限,大汗宠眷亦无限,无人敢与之抗言。是以官位权力无论大小,莫不畏之。凡有人受谗因蒙大罪而欲自解者,绝不能提出其自解之法,盖无人敢庇之而与阿合马抗,由是枉死者为数甚众。

不仅此也,凡有美妇而为彼所欲者,无一人得免。妇未婚,则娶以为妻。已婚,则强之从己。如闻某家有美女,则遣其党徒语其父曰:“汝有女如是,曷不嫁之伯罗(bailo)阿合马(盖人称其为伯罗,犹之吾人之称副王也),则彼将授汝以高官显职,荣任三年。”女父若以女献,阿合马则言于汗曰:“某官缺人,或某官行将任满,某人可以铨选。”大汗辄答之云:“汝以为是,为之可也。”女父遂立受显职,由是或因他人盼得高官显职,或因他人畏其权势,阿合马尽得美妇为其妻妾。彼有子二十五人,皆任显要,其中有若干子因父荫,而淫纵亦如其父,所行无耻无义。此阿合马聚积多金,盖欲任显职或他官者,必须以重赂贿之也。

彼执行此无限权势,垂二十二年。迄后国人,质言之契丹人,因其妻女或本身蒙大辱或受奇害者,忍无可忍,乃相谋杀之而叛政府。其中有一契丹人名陈箸(Tchen-tchou)者,身为千户,母及妻女并为阿合马所辱。愤恨已极,遂与别一契丹人身为万户名称王箸(Wang-tchou)者同谋杀之。决定在大汗驻跸汗八里三个月满,驾幸上都驻跸三月之时举事。时皇太子成吉思亦离都城往驻他所,仅阿合马留守都城,有事则由阿合马遣人往上都请旨。

王箸、陈箸同谋以后,遂以其谋通知国中之契丹要人。诸人皆赞成其谋,并转告其他不少城市友人,定期举事,以信火为号,见信火起,凡有须之人悉屠杀之。盖契丹人当然无须,仅鞑靼人、回教徒及基督教徒有须也。契丹人之厌恶大汗政府者,盖因其所任之长官是鞑靼人,而多为回教徒,待遇契丹人如同奴隶也。复次大汗之得契丹地,不由世袭之权,而由兵力,因是疑忌土人,而任命忠于本朝之鞑靼人、回教徒或基督教徒治理,虽为契丹国之外人,在所不计也。

迨至约定之日,王箸、陈箸夜入皇宫,王箸据帝座,燃不少灯火于前。遣其党一人赴旧城,矫传令旨,伪称皇太子已归,召阿合马立入宫。阿合马闻之大异,然畏皇太子甚,仓卒遽行,入城门,鞑靼统将统一万二千人守备大都名火果台(Cogotai)者,询之曰:“夜深何往?”答曰:“成吉思已至,将往谒之。”火果台曰:“皇太子秘入都城,缘何我毫无所闻。”遂与偕行,并率领一部分人护从。阿合马入宫,见灯光大明,以为据宝座之王箸是皇太子,进前跪谒。陈箸俟其跪,举刀断其首。火果台在宫门见状,呼曰:“中奸计。”立张弓发矢,射杀王箸。同时命所部擒陈箸,布告城中,不许居民外出,有至街市者,立即杀之。契丹人见其谋泄,主谋者一死一擒,不敢出外,不能举信号通知其他诸城。火果台立遣使者驰奏大汗。大汗立命严搜叛人,捕同谋者悉杀之。翌日黎明,火果台搜查诸契丹人,同谋罪重者多伏诛,其他诸城所为亦同。

大汗还汗八里后,欲知此次叛事之原因,已而得阿合马罪状,始知其父子做恶多端,如前所述。阿合马本人及其七子(盖诸子非尽恶也)娶妻妾无算,强取者尚未计焉。大汗命人没收旧城中阿合马所积之一切货财,徙之新城,尽入帝库,至是始发现其数甚巨。并命发墓剖棺,戮阿合马尸,置之街市,纵犬食之。诸子之为恶者,生剥其皮。

经此事变以后,大汗始知回教对于异教之人纵使本教之人犯罪,甚至杀人亦所不辞。既见此教使阿合马父子纵为奸恶,遂痛恶之,所以召诸回教徒来前,对于其教命为之事,多严禁之。例如命其娶妻从鞑靼俗,杀牲遵鞑靼法,不许再用断喉之法,只许破腹取脏,皆此类也。





085 名曰怯薛丹之禁卫一万二千骑

应知大汗之禁卫,命贵人为之,数有一万二千骑,名称怯薛丹(Quesitan),法兰西语犹言“忠于君主之骑士”也。设禁卫者,并非对人有所疑惧,特表示其尊严而已。此一万二千人四将领之,每将各将三千人。而此三千人卫守宫内三昼夜,饮食亦在宫中。三昼夜满,离宫而去,由别一三千人卫守,时日亦同,期满复易他人。由是大汗常有名称怯薛丹之禁卫三千骑更番宿卫。此一万二千人轮番守卫各有定日。周而复始,终年如此。

大汗开任何大朝会之时,其列席之法如下:大汗之席位置最高,坐于殿北,面南向。其第一妻坐其左。右方较低之处,诸皇子侄及亲属之座在焉。皇族等座更低,其坐处头与大汗之足平,其下诸大臣列坐于他席。妇女座位亦同,盖皇子侄及其他亲属之诸妻,坐于左方较低之处,诸大臣骑尉之妻坐处更低。各人席次皆由君主指定,务使诸席布置,大汗皆得见之,人数虽众,布置亦如此也。殿外往来者四万余人,缘有不少入贡献方物于君主,而此种人盖为贡献异物之外国人也。

大汗所坐殿内,有一处置一精金大瓮,内足容酒一桶(untonneau communal)。大瓮之四角,各列一小瓮,满盛精贵之香料。注大瓮之酒于小瓮,然后用精金大勺取酒。其勺之大,盛酒足供十人之饮。取酒后,以此大勺连同带柄之金盏二,置于两人间,使各人得用盏于勺中取酒。妇女取酒之法亦同。应知此种勺盏价值甚巨,大汗所藏勺盏及其他金银器皿数量之多,非亲见者未能信也。

并应知者,献饮食于大汗之人,有大臣数人,皆用金绢巾蒙其口鼻,俾其气息不触大汗饮食之物。大汗饮时,众乐皆作,乐器无数。大汗持盏时,诸臣及列席诸人皆跪,大汗每次饮时,各人执礼皆如上述。

至若食物,不必言之,盖君等应思及其物之丰饶。诸臣皆聚食于是,其妻偕其他妇女亦聚食于是。食毕撤席,有无数幻人艺人来殿中,向大汗及其他列席之人献技。其技之巧,足使众人欢笑。诸事皆毕,列席之人各还其邸。

剌木学本第二卷第九章增入之文如下:

(一)但在昼间,未番上之怯薛歹(Quésitaux)不得离开宫中。惟奉大汗使命,或因本人家事,而经怯薛长许可者,始能放行。设若有重大理由,如父兄及其他亲属之丧,抑非立归必有重大损害之类,则应请求大汗许可。然在夜中,此九千人可以还家。

(二)君等勿以为人人皆可坐于席上。尚有官吏,甚至有贵人不少,无席可列,应坐于殿中毡上而食。复有无数人在殿外,此种人盖来自各州贡献远地异物者。其中间有土地被没收之若干藩主冀将土地发还者,此辈于朝会及皇子结婚之日常临殿外。

(三)殿中有一器,制作甚富丽,形似方柜,宽广各三步,刻饰金色动物甚丽。柜中空,置精金大瓮一具,盛酒满,量足一桶。柜之四角置四小瓮,一盛马乳,一盛驼乳,其他则盛种种饮料。柜中亦置大汗之一切饮盏,有金质者甚丽,名曰勺(vernique),容量甚大,满盛酒浆,足供八人或十人之饮。列席者每二人前置一勺,满盛酒浆,并置一盏,形如金杯而有柄。

(四)此外命臣下数人接待入朝之外国人,告以礼节,位置席次。此辈常在殿中往来,俾会食者不致有所缺,设有欲酒乳肉及其他食物者,则立命仆役持来。

每殿门,尤其大汗所在处之殿门,有大汉二人持杖列于左右,勿使入者足触其阈。设有触者,立剥其衣,必纳金以赎。若不剥衣,则杖其人。顾外国人得不明此禁,如是命臣下数人介之入,预警告之,盖视触阈为凶兆,故设此禁也。但出殿时,会食之人容有醉者,罚之则不如入门之严。

(五)大汗每次饮时,侍者献盏后,退三步,跪伏于地,诸臣及其他在场之人亦然。乐器齐奏,其数无算,饮毕乐止,会食者始起立。大汗每次饮时,执礼皆如是也。





086 每年大汗之诞节

应知每年鞑靼人皆庆贺其诞生之日。大汗生于阳历9月即阴历八月二十八日。是日大行庆贺,每年之大节庆,除后述年终举行之节庆外,全年节庆之重大无有过之者也。

大汗于其庆寿之日,衣其最美之金锦衣。同日至少有男爵骑尉一万二千人,衣同色之衣,与大汗同。所同者盖为颜色,非言其所衣之金锦与大汗衣价相等也。各人并系一金带,此种衣服皆出汗赐,上缀珍珠宝石甚多,价值金别桑(besant)确有万数。此衣不止一袭,盖大汗以上述之衣颁给其一万二千男爵骑尉,每年有十三次也。每次大汗与彼等服同色之衣,每次各易其色,足见其事之盛,世界之君主殆无有能及之者也。

庆寿之日,世界之一切鞑靼人及一切州区皆大献贡品于大汗。此种贡品皆有定额,并有他人献进厚礼以求恩赏。大汗选任男爵十二人,视其应颁赏之数而为赏赐。是日也,一切偶像教、回教、基督教之教徒,及其他种种人,各向其天主燃灯焚香,大事祈祷礼赞,为其主祝福求寿。大汗寿诞之日,庆祝之法盖如此也。

此事言之既详,兹请为君等一述年终举行名曰白节之节庆。

剌木学本第二卷第十一章较有异文,兹转录如下:

(一)有男爵骑尉二万人,所服之衣与大汗同式同色……此外各人别受一羚羊皮带,上饰金银丝甚奇,又受有靴一双。

(二)此种衣服有若干袭,上缀宝石珍珠,其价有逾金别桑一千者。其男爵忠诚可恃而得近大汗者,名称怯薛丹(Quiecitan)。

(三)此种衣服专在大庆贺时服之,鞑靼人每年大节视阴历十三月之数共举行十三次,其衣服之盛,各人俨如国王。此种衣服诸





087 年终大汗举行之庆节

其新年确始于阳历2月,届时大汗及其一切臣属复举行一种节庆,兹述其情形如下:

是日依俗大汗及其一切臣民皆衣白袍,至使男女老少衣皆白色,盖其似以白衣为吉服,所以元旦服之,俾此新年全年获福。是日臣属大汗的一切州郡国土之人,大献金银、珍珠、宝石、布帛,俾其君主全年获有财富欢乐。臣民互相馈赠白色之物,互相抱吻,大事庆祝,俾使全年纳福。

应知是日国中数处入贡极富丽之白马十万余匹。是日诸象共有五千头,身披锦衣甚美,背上各负美匣二,其中满盛白节宫廷所用之一切金银器皿甲胄。并有无数骆驼身披锦衣,负载是日所需之物,皆列行于大汗前,是为世界最美之奇观。

尚有言者,节庆之日黎明,席案未列以前,一切国王、藩主,一切公侯伯男骑尉,一切星者、哲人、医师、打捕鹰人,以及附近诸地之其他不少官吏,皆至大殿朝贺君主。其不能入殿者,位于殿外君主可见之处。其行列则皇子侄及皇族在前,后为诸国王、公爵,其后则为其他诸人,各按其等次而就位。

各人就位以后,其间之最贤者一人起立,大声呼曰:“鞠躬拜。”呼毕,诸人跪拜,首触于地,祝赞其主事之如神。如是跪拜四次,礼毕,至一坛前。坛上置一朱牌,上写大汗名,牌前置一美丽金炉,焚香,诸人大礼参拜毕,各归原位。

诸礼皆毕后,遂以前述贡献之物上呈大汗,其物颇美而价值甚贵。大汗遍视诸物毕,然后将一切席案排列,各人案序就位,进食如前所述。食毕,诸艺人来前做术以娱观众。诸事毕后,诸人各归其邸。

此年初之白节,既已备述如前,兹请言大汗之一豪举,即前此言诸。





088 大汗命人行猎

大汗居其都城之三个月中,质言之阳历12月、1月、2月中,在四围相距约四十日程之地,猎户应行猎捕鸟,以所获之鸟与大兽献于大汗。大兽中有牝鹿、花鹿、牡鹿、狮子,及其他种种大野兽,其数居猎物之强半。其人献兽之先,应剖腹取脏,然后以车运赴汗所。行程有需二三十日者,而其数颇众也。其远道未能献肉者,则献其皮革,以供君主制造军装之用。

此事既已言毕,兹请叙述大汗驯养其游猎时所用之猛兽。





089 豢养以备捕猎之狮豹山猫

尚应知者,大汗豢有豹子,以供行猎捕取野兽之用。又有山猫(loupscerviers)甚夥,颇善猎捕。更有狮子数头,其躯较巴比伦(Babylonie)之狮子为大,毛色甚丽,缘其全身皆有黑朱白色斑纹也,此则豢养以供捕取野猪、熊鹿、野驴及其他大猛兽之用。此种狮子猎取猛兽,颇可悦目。用狮行猎之时,以车载狮,每狮辅以小犬一头,别有雕类无数,用以捕取狼、狐、花鹿、牡鹿等兽,所获甚多。惟猎狼者躯甚大而力甚强,凡狼遇之者无能免也。

此事既已备述于前,兹请一言大汗豢养无数大犬之法。





090 管理猎犬之两兄弟

大汗有两男爵,是亲兄弟,一名伯颜(Bayan),一名明安(Mingam)。人称此二人曰古尼赤(Cunici),此言管理番犬之人也。弟兄两人各统万人,每万人衣皆同色,此万人衣一色,彼万人衣又一色,此万人衣朱色,彼万人衣蓝色,每从君主出猎时,即衣此衣,俾为人识。

每万人队有二千人,各有大犬一二头或二头以上,由是犬数甚众。大汗出猎时,其一男爵古尼赤将所部万人,携犬五千头,从右行。别一男爵古尼赤率所部从左行。相约并途行,中间留有围道,广二日程。围中禽兽无不被捕者。所以其猎同猎犬、猎人之举动,颇可观。君主偕诸男爵骑行旷野行猎时,可见此种大犬无数,驰逐于熊、鹿或他兽之后,左右奔驰,其状极堪娱目也。

管理猎犬者及其状况,既已备述如前,兹请言君主于别三月在他处行猎之事。

君主驻跸于其都城,逾阳历12月、1月、2月共三阅月后,阳历3月初即从都城首途南下,至于海洋,其距离有二日程。行时携打捕鹰人万人,海青五百头,鹰鹞及他种飞禽甚众,亦有苍鹰(autours),皆备沿诸河流行猎之用。然君等切勿以为所携禽鸟皆聚于一处,可以随意分配各所。每所分配禽鸟一二百,或二百以上,为数不等,此种打捕鹰人以其行猎所获多献大汗。

君主携其海青及其他禽鸟行猎之时,如上所述。此外尚有万人,以供守卫,其人名称脱思高儿(Toscaors),此言守卫之人也。以两人为一队,警卫各处,散布之地甚广。





091 大汗之行猎

各人有一小笛及一头巾,以备唤鸟持鸟之用,俾君主放鸟之时,放鸟人勿须随之。盖前此所言散布各处之人,守卫周密,鸟飞之处不用追随,鸟须救助时,此辈立能赴之也。

君主之鸟,爪上各悬一小牌,以便认识。诸男爵之鸟亦然,牌上勒鸟主同打捕鹰人之名,鸟如为人所得,立时归还其主,如不识其主,则持交一男爵名曰不剌儿忽赤(Boulargoutchi)者,此言保管无主之物者也。盖若有人拾得一马一剑一鸟或一别物而不识其主者,立以此物付此男爵保管之。如拾得者不立时交出,则由此男爵惩罚,失物者亦赴此男爵处求之,如有此物立时还付其人。

此男爵常位于众人易见之处,立其旌旗,俾拾物及失物者易见,而使凡失物皆得还原主。君主由此路径赴海洋,其地距其汗八里都城有二日程,沿途景物甚丽,世界赏心娱目之事无逾此者。

大汗坐木楼甚丽,四象承之。楼内布金锦,楼外覆狮皮。携最良之海青十二头。扈从备应对者有男爵数人。其他男爵则在周围骑随,时语之曰:“陛下,鹤过。”大汗闻言,立开楼门视之,取其最宠之海青放之。此鸟数捕物于大汗前,大汗在楼中卧床观之,甚乐。侍从之诸男爵亦然。故余敢言世界之人,娱乐之甚,能为之优,无有逾大汗者。

前行久之,抵于一地,名称火奇牙儿末敦(Cocciar Modun),其行帐及其诸子诸臣友诸妇之行帐在焉。都有万帐,皆甚富丽,其帐之如何布置,后此一言之。其用以设大朝会之帐甚广大,足容千人而有余。帐门南向,诸男爵骑尉班列于其中。西向有一帐,与此帐相接,大汗居焉。如欲召对某人时,则遣人导人此处。

大帐之后有一小室,乃大汗寝所。此外尚有别帐、别室,然不与大帐相接。此二帐及寝所布置之法如下:

每帐以三木柱承之,辅以梁木,饰以美丽狮皮。皮有黑白朱色斑纹,风雨不足毁之。此二大帐及寝所外,亦覆以斑纹狮皮。帐内则满布银鼠皮及貂皮,是为价值最贵最美丽之两种皮革。盖貂袍一袭值价金钱(Hivre d'or)二千,至少亦值金钱一千,鞑靼人名之曰“毛皮之王”。帐中皆以此两种毛皮覆之,布置之巧,颇悦心目。凡系帐之绳,皆是丝绳。总之,此二帐及寝所价值之巨,非一国王所能购置者也。

此种帐幕之周围,别有他帐亦美,或储大汗之兵器,或居扈从之人员。此外尚有他帐,鹰隼及主其事者居焉。由是此地帐幕之多,竞至不可思议。人员之众,及逐日由各地来此者之多,竟似大城一所。盖其地有医师、星者、打捕鹰人,及其他有裨于此周密人口之营业,而依俗各人皆携其家属俱往也。

大汗居此迄于(复活节)之第一夜。当其居此之时,除在周围湖川游猎外,别无他事。其他湖川甚多,风景甚美,饶有鹤、天鹅及种种禽鸟。周围之人亦时时行猎,逐日献种种猎物无算。丰饶之极,其乐无涯,未目击者决不信有此事也。

尚有一事须为君等言及者,此地周围二十日程距离之内,无人敢携鹰犬行猎。在大汗所有辖地之中,有兽四种,无人敢捕,即山兔、牡鹿、牝鹿、獐鹿(chevreuil)是已。此禁仅在阳历3月迄阳历10月之间有之。违禁者罚。顾其臣民忠顺,行于路者,虽见此种兽类卧地,亦不敢惊之。由是繁息甚众,地为之满。大汗取之惟意所欲。惟逾此阳历3月至10月期限之外,则解其禁,各人得随意捕之。

大汗居留此距海不远之地,自阳历3月迄于阳历5月半间,然后携其一切扈从之人,重循来道,还其契丹都城汗八里。





092 大汗猎后设大朝会

大汗归其都城汗八里后,留居宫中三日,于是设大朝会,偕诸后妃大事宴乐。然后从汗八里宫出发,赴上都,即前此所述有大草原及竹宫,并驯养海青之地也。大汗留居上都,始阳历5月初,迄阳历8月之28日。是日洒马乳如前所述。夫然后还其汗八里都城。在此都城于阳历9月中举行万寿节,嗣后历10月、11月、12月、次年1月、次年2月。于2月举行所谓白节之元旦节,亦如前此所述。至是向海洋畋猎,始阳历3月初,迄5月半。猎毕还居都城三日,偕诸后妃设大朝会,会毕复行,亦如前述。

由是全年如是分配,居汗八里都城大宫中六月,即阳历9月、10月、11月、12月、次年1月、次年2月是已。

已而赴海岸举行大猎者三月,即阳历3月、4月、5月是已。

猎后复返其汗八里宫中,留居三日。

其后赴其营建之上都,竹宫所在之地,历阳历6月、7月、8月。

最后复还其汗八里都城。

于是一年之中居其都城者六阅月,游猎者三阅月,居其竹宫避暑者三阅月。偶亦赴他处,惟意所欲。总之,其起居悉皆欢乐也。

此章在此本以前之法文本,如地学会之法文本及与法文本对照之拉丁文本中,并阙,惟其为马可·波罗出狱后增入其原口授本之文,似可勿庸怀疑者也。此章虽节述前此诸章之语,要在使读者明了忽必烈之起居。此事只有马可·波罗独能为之也(颇节本第一册311页)。





093 汗八里城之贸易发达户口繁盛

应知汗八里城内外人户繁多,有若干城门即有若干附郭。此十二大郭之中,人户较之城内更众。郭中所居者,有各地来往之外国人,或来入贡方物,或来售货宫中。所以城内外皆有华屋巨室,而数众之显贵邸舍,尚未计焉。

应知城内不许埋葬遗骸。脱死者是一偶像教徒,则移尸于城郭外,曾经指定一较远之处焚之。脱死者所信仰者为别教,则视其为基督教徒、回教徒或他教之人,亦运尸于郭外,曾经指定之远地殡葬。由是城内最适宜于卫生。

尚应知者,凡卖笑妇女,不居城内,皆居附郭。因附郭之中外国人甚众,所以此辈娼妓为数亦夥,计有二万有余,皆能以缠头自给,可以想见居民之众。外国巨价异物及百物之输入此城者,世界诸城无能与比。盖各人自各地携物而至,或以献君主,或以献宫廷,或以供此广大之城市,或以献众多之男爵骑尉,或以供屯驻附近之大军。百物输入之众,有如川流之不息。仅丝一项,每日人城者计有千车。用此丝制作不少金锦绸绢,及其他数种物品。附近之地无有亚麻质良于丝者,固有若干地域出产棉麻,然其数不足,而其价不及丝之多而贱,且亚麻及棉之质亦不如丝也。

此汗八里大城之周围,约有城市二百,位置远近不等。每城皆有商人来此买卖货物,盖此城为商业繁盛之城也。

此汗城之广大庄严,既已备述于前,兹请言大汗铸造货币之所,用以证明大汗之所为,诚有逾我之所言,及此书之所记者。盖我言之无论如何诚实,皆不足取信于人也。





094 大汗用树皮所造之纸币通行全国

在此汗八里城中,有大汗之造币局,观其制设,得谓大汗专有方士之点金术,缘其制造如下所言之一种货币也。此币用树皮作之,树即蚕食其叶做丝之桑树。此树甚众,诸地皆满。人取树干及外面粗皮间之白细皮,旋以此薄如纸之皮制成黑色,纸既造成,裁为下式。

幅最小之纸值秃儿城之钱(denier tournois)一枚,较大者值威尼斯城之银钱(gros vénitien)半枚,更大者值威尼斯城之银钱一枚。别有值威尼斯银钱五枚、六枚、十枚者。又有值金钱(besant d’or)一枚者,更有值二枚、四枚、五枚以至十枚者。此种纸币之上,钤盖君主印信,由是每年制造此种可能给付世界一切帑藏之纸币无数,而不费一钱。

既用上述之法制造此种纸币以后,用之以作一切给付。凡州郡国土及君主所辖之地莫不通行。臣民位置虽高,不敢拒绝使用,盖拒用者罪至死也。兹敢为君等言者,各人皆乐用此币,盖大汗国中商人所至之处,用此纸币以给费用,以购商物,以取其售物之售价,竟与纯金无别。其量甚轻,致使值十金钱者,其重不逾金钱一枚。

尚应知者,凡商人之携金银、宝石、皮革来自印度或他国而莅此城者,不敢售之他人,只能售之君主。有贤明能识宝货价值之男爵十二人专任此事。君主使之用此纸币偿其货价,商人皆乐受之,盖偿价甚优,可立时得价,且得用此纸币在所至之地易取所欲之物,加之此种纸币最轻便可以携带也。

由是君主每年购取贵重物品颇多,而其帑藏不竭,盖其用此不费一钱之纸币给付也。复次每年数命使者宣告城中,凡藏有金银、宝石、珍珠、皮革者,须送至造币局,将获善价,其臣民亦乐售之。盖他人给价不能有如是之优,售之者众,竟至不可思议。大汗用此法据有所属诸国之一切宝藏。

此种货币虽可持久,然亦有敝坏者,持有者可以倒换新币,仅纳费用百分之三。诸臣民有需金银、宝石、皮革用以制造首饰、器皿、衣服或其他贵重物品者,可赴造币局购买,惟意所欲,即以此种纸币给价。

大汗获有超过全世界一切宝藏的财货之方法,业已备述于前。君等闻之,必解其理。兹请言此城执行大权之诸大官吏。

剌木学本第二卷第十八章增补之文如下:

(一)译者按:此条原阙。

(二)“此薄树皮用水浸之,然后捣之成泥,制以为纸,与棉纸无异,惟其色纯黑。君主造纸既成,裁作长方形,其式大小不等。”

(三)“此种纸币制造之法极为严重,俨同纯金纯银,盖每张纸币之上,有不少专任此事之官吏署名盖章。此种程式完毕以后,诸宫之长复盖用朱色帝玺,至是纸币始取得一种正式价值,伪造者处极刑。”

(四)“所有军饷皆用此种货币给付,其价如同金银。”





095 执掌大权之十二男爵

应知大汗选任男爵十二人,指挥监察其国三十四区域中之要政。兹请述其执行之方法及其衙署。

应知此十二男爵同居于一极富丽之宫中,宫在汗八里城内。宫内有分设之房屋亭阁数所,各区域各有断事官一人、书记数人,并居此宫之内,各有其专署。此断事官及书记等承十二男爵之命,处理各该区域之一切事务。事之重大者,此十二男爵请命于君主决之。

然此十二男爵权力之大,致能自选此三十四区域之藩主。迨至选择其所视为堪任之人员以后,人告于君主,由君主核准,给以金牌,俾之授职。此十二男爵权势之大,亦能决定调度军队,调发必要之额数,遣赴其视为必要之处所。然此事应使君主知之。其名曰省(scieng),此言最高院所是已。其所居之宫亦名最高院所,是为大汗朝廷之最大卿相,盖其广有权力,可随意施惠于其所欲之人。此三十四区域之名称,后在本书中分别言之,今暂不言及。

兹置此事不言,请言大汗如何遣派使臣铺卒,及其如何有业已预备之马匹以供急行。

剌木学本之文大异,标题作“节制军队之十二男爵及管理普通政务之其他十二男爵”。其文曰:

“大汗选任强大男爵十二人,决定关于军事之一切问题,如遣调驻所,更迭主将,抑调动军队于认为必要之地,征发战时所需之军额等事是已。此外分别勇懦而为黜陟,勇者升,懦者降。设有千夫长不称职者,上述之诸男爵降之为百夫长;反之,设其人勇敢堪于任使,则升之为万夫长。惟此种黜陟常应使君主知之,所以彼等欲降某官时,必语君主日某人不称职,君主则日降其职。欲升某官时,亦语君主曰某千夫长称职,足任万夫长。君主则按职以牌符赐之,如前所述(见本书第80章),然后厚给赏赐‘俾能鼓励他人’。

“此十二男爵所组织之高等会议,名称曰台(thai)。此言最高院所,缘其上除大汗外,别无他官管辖也。

“除上述之男爵外,别有十二男爵执。司指挥三十四区域之一切政务。汗八里城为诸区域置有富丽宫殿一所,内有房室甚众,各区域有断事官一人、书记多人居此宫内,各有专室,承此十二男爵之命,处理本区域之一切事务。彼等有权选任一切区域之长官、法官,选择堪任之员以后,上呈大汗核准,视各人之官位赐以金银牌符。此种男爵并监察贡赋之征收及其使用分配,除关于军队之事务外,大汗之一切其他事务并隶属之。

“此高等会议组织之所,名称曰省。此言第二最高院所,盖其亦直隶大汗,不受他官管辖也。

“由是观之,此二院所名曰省、台者,直隶大汗,不隶他官。惟台,质言之,职司调度军事之院所,视为一切官署中最高贵之官署。”(剌木学本第二卷第十九章)

马可·波罗所志此管理军务之十二男爵,与《元史》卷八六《百官志》所载之枢密院相近。元代总政务者曰中书省,秉兵柄者曰枢密院,司黜陟者曰御史台。省秩正一品,院秩从一品,台秩从二品。枢密院掌天下兵甲机密之务,凡宫禁、宿卫、边庭、军翼、征讨、戍守、简阅、差遣、举功、转官、节制、调度,无不由之。





096 从汗八里遣赴各地之使臣铺卒

应知有不少道路从此汗八里城首途,通达不少州郡。此道通某州,彼道通别州,由是各道即以所通某州之名为名,此事颇为合理。如从汗八里首途,经行其所取之道时,行二十五哩,使臣即见有一驿,其名曰站(Iamb),一如吾人所称供给马匹之驿传也。每驿有一大而富丽之邸,使臣居宿于此,其房舍满布极富丽之卧榻,上陈绸被,凡使臣需要之物皆备。设一国王莅此,将见居宿颇适。

此种驿站中备马,每站有多至四百匹者。有若干站仅备二百匹,视各站之需要而为增减。盖大汗常欲站中存有余马若干,以备其所遣使臣不时之用。应知诸道之上,每二十五哩或三十哩,必有此种驿站一所,设备如上所述。由是诸要道之通诸州者,设备皆如此;赴大汗所辖之诸州者,经行之法如此。

设若使臣前赴远地而不见有房屋邸舍者,大汗亦在其处设置上述之驿站。惟稍异者,骑行之路程较长。盖上所述之驿站,彼此相距仅有二三十哩;至若此种远地之驿站,彼此相距则在三十五哩至四十五哩之间。所需马匹百物,悉皆设备,如同他驿,俾来往使臣不论来自何地者皆获供应。 【1】

是为最盛大之举,从未见有皇帝、国王、藩主之殷富有如此者。盖应知者,此种驿站备马逾三十万匹,特供大汗使臣之用,驿邸逾万所,供应如上述之富饶。其事之奇,其价之巨,非笔墨所能形容者也。 【2】

尚有一事,前此忘言,兹应补述。应知此一驿与彼一驿之间,无论在何道上,大汗皆命在每三哩地置一小铺,铺周围得有房屋四十所,递送大汗文书之步卒居焉。每人腰系一宽大腰带,全悬小铃,俾其行时铃声远闻。彼等竭力奔走一切道路,止于相距三哩之别铺,别铺闻铃声,立命别一铺卒系铃以待。奔者抵铺,接替者接取其所赉之物,暨铺书记所给之小文书一件,立从此铺奔至下三哩之铺。下铺亦有一接替之铺卒,辗转递送。由是每三哩一易铺卒,所以大汗有无数铺卒,日夜递送十日路程之文书消息。缘铺卒递送,日夜皆然,脱有必要时,百日路程之文书消息,十日夜可以递至,此诚伟举也。复次此种铺卒递送果实及其他异物于大汗,于一日间奔走十日程途之地。 【3】

大汗对于此种人不征赋税,反有赐给。尚有言者,上述诸铺别有人腰带亦系小铃,设有急须传递某州之消息,或某藩主背叛事,或其他急事于大汗者,其人于日间奔走二百五十至三百哩之远,夜间亦然。其法如下:其人于所在之驿站取轻捷之良马,疾驰至于马力将竭,别驿之人间铃声亦备良马铺卒以待;来骑抵站,接递者即接取其所赉之文书或他物,疾驰至于下站;下站亦有预备之良马铺卒接递,于是辗转接递,其行之速,竟至不可思议。

此种人颇受重视,头胸腹皆缠布带,否则不堪其劳。常持一海青符, 【4】 俾其奔驰之时,偶有马疲或其他障碍之时,得在道上见有骑者即驱之下,而取其马。此事无人敢拒之,由是此种铺卒常得良马以供奔驰。

上所言之马,驿站中数甚众。应知大汗对于此种马匹毫无所费,兹莆述其理由如下:大汗命人调查各站及邻城附近居民人数,俾知其能出马若干。所出之马给之站铺, 【5】 城乡供给驿马之法悉皆如此。惟在远道及荒地驿站,则由大汗供给马匹。一使臣驿站之事,既已详细诚实叙述如前,兹请言大汗每年两次施惠于其人民之事。

剌木学本第二卷第二十章之异文如下:

注释

【1】 “命人居住此等处所,耕种田亩,兼服站役,由是在其地建设不少大村;凡由大汗所辖国土入朝之使臣,及大汗派往之使臣,皆得安适便利。”

【2】 “或有疑及服役之人不能有如是之众,而人众不能得其食粮者,吾人将答之曰:一切佛教徒如同回教徒,皆视其力之能养赡,娶妻六人、八人、十人不等。所生子女甚多,且有不少人有子三十余人,能与其父共执兵器者,斯盖因妻妾之众有以致之。至若吾人国内,一人仅娶一妻,有时且无所出而致绝后,我辈人口单弱之理在此。至若食粮,彼等甚为丰足,盖其主要食物为米稷粟,尤以在鞑靼、契丹、蛮子境内为甚。其处田亩种植此三种谷食,每一容量(setier)足以收获百倍。此种民族不识面包,仅将其谷连同乳或肉煮食,其处小麦产额则不如是之丰,收获小麦者仅制成饼面而食。境内无荒地,牲畜繁殖无限,乡间每人自用之马至少有六八头。上述诸地人众食丰之理即在此也。”

【3】 “果实成熟之时,常见晨摘之果,于翌晚可以递送至距离十日程之上都城中,进奉大汗。”

“每站程内相距三哩即置一铺,每铺有一书手,记录铺卒到达之曰时,及所辖转递人出发之日时,一如驿站簿记之法。此外尚有监察人,每月亲至此种驿站,视察铺卒怠慢而处罚之。”

【4】 “彼等持海青符,示其必须急行,设有使者二人同在一地出发,共登二良骑后,即包头束腰,纵马疾驰。迨近一站,即吹角,俾站内人闻之,从速备马。到站即跃登彼骑,由是终日疾驰,迄于日晡,每日可行二百五十哩。设有大事,则须夜行,若无月光,站中人持炬火前导。惟使者夜行不速,盖持炬火者步行,不能如骑者之速也。凡使者疾驰而能耐疲劳者,辄被重视。”

【5】 “大汗命各城官府调查本城可以供应邻站马匹若干,乡村可供应若干,征发并以此为准。诸城互约各城供应之额(盖两站之间必有一城),诸城以应缴大汗之赋税养马。所以每人视其应纳之额等若一马或马之一部者,奉命供养邻站之马。”

“但应知者,诸城并不长年供养每站之马四百匹,仅供养应役之马约二百匹,其余二百匹则留牧地。应役之马一月期满,取牧地之马代役;役毕之马则赴牧地休养,各以半数互相更代。设在某地有一川一湖,步行或骑行之使臣、铺卒必须经过者,应由邻城预先供应船只三四。设其必须经行距离数日程之沙漠而不见民居者,则由最近之城供应马匹、食粮于使臣及其从人,止于沙漠彼端。然此城将受一种赔偿,至若距离大道甚远之驿站,其驿马之供应,一部分出自君主,一部分出自邻近之城村乡里。”





097 歉收及牲畜频亡时大汗之赈恤其民

应知大汗遣使臣周巡其国土州郡,调查其人民之谷麦是否因气候不时或疾风暴雨受有损害,抑有其他疫疠。





【1】 其受损害者,则蠲免本年赋税,并以谷麦赐之,俾有食粮、种子。是为大汗之一德政。冬季既届,又命人调查畜养牲畜者是否因死亡频繁或其他疫疠受有损害。其受损害者,亦蠲免本年赋税,并以牲畜赐之。 【2】 大汗每年赈恤其臣民之法如此。

剌木学本第二卷第二十一章补订之文如下:

注释

【1】 “雨水过度,暴风为灾,或因蝗灾虫害及其他灾害。”

【2】 “设若某州牲畜频亡,则以他州所缴什一税之牲畜赈恤之。”

“其意之所注,惟在赈恤其人民,俾能生存劳作富庶。”

“然大汗尚有别事而为吾人所不应遗漏者:若有雷震大小家畜畜群,不问其属于一人或数人,亦不问其数多寡,概免除其什一税三年。设有装载商货之船为雷所击,亦免除其一切差税。缘其视此种灾害如同凶兆。据云:天罚物主,大汗不欲取此种曾遭天怒之物也。”

《元史》云:“救荒之政,莫大于赈恤。元赈恤之名有二:曰蠲免者,免其差税,即《周官》大司徒所谓薄征者也;曰赈贷者,给以米粟,即《周官》大司徒所谓散利者也。然蠲免有以恩免者,有以灾免者。赈贷有以鳏寡孤独而赈者,有以水旱疫疠而赈者,有以京师人物繁凑而每岁赈粜者。若夫纳粟补官之令,亦救荒之一策也。其为制各不同。”(见《元史》卷九六《食货志》)

又云:“元初取民,未有定制。及世祖立法,一本于宽。其用之也,于宗戚则有岁赐,于凶荒则有赈恤。大率以亲亲爱民为重,而尤倦倦于农桑一事。可谓知理财之本者矣。世祖尝语中书省臣曰:‘凡赐与,虽有朕命,中书其斟酌之。’”(见《元史》卷九三《食货志》)





098 大汗命人沿途植树

并应知者:大汗曾命人在使臣及他人所经过之一切要道上种植大树,各树相距二三步,俾此种道旁皆有密接之极大树木,远处可以望见,俾行人日夜不至迷途。盖在荒道之上,沿途皆见此种大树,颇有利于行人也。所以一切通道之旁,视其必要,悉皆种植树木。

剌木学本第二卷第二十二章之文微异,其文云:

“大汗尚有别一制设,既有裨益,亦重观瞻,即沿大道两旁命人种植树木是已。务以将来树身能高大者为限。各树相距两步,由是行人易识道途。此事有裨于行人,且使行人愉快。所以在一切要道之旁,视地上所宜,为此种植。第若此种道路经过沙碛不毛之地,或岩石山岭,而不能种植树木者,则立标柱,以示路途;并任命官吏保持路途,使之不致损坏。其使大汗乐于种植树木者,且因巫师星者曾预言爱植树者必长寿也。”





099 契丹人所饮之酒

尚应知者:契丹地方之人大多数饮一种如下所述之酒,彼等酿造米酒,置不少好香料于其中,其味之佳,非其他诸酒所可及。盖其不仅味佳,而且色清爽目。其味极浓,较他酒为易醉。

兹置此事不言,请言别事。





100 用石作燃料

契丹全境之中,有一种黑石,采自山中,如同脉络,燃烧与薪无异。其火候且较薪为优,盖若夜间燃火,次晨不息。其质优良,致使全境不燃他物。所产木材固多,然不燃烧。盖石之火力足,而其价亦贱于木也。

剌木学本第二卷第二寸三章补订之文云:“此种石燃烧无火焰,仅在初燃时有之,与燃桴炭同。燃之以后,热度甚高……其地固不缺木材,然居民众多,私人火炉及公共浴场甚众,而木材不足用也。每人于每星期中至少浴三次,冬季且日日入浴。地位稍高或财能自给之人,家中皆置火炉,燃烧木材势必不足。至若黑石取之不尽,而价值亦甚贱也。”

中国始用石炭,两千年前业已见之。《前汉·地理志》日:“豫章郡出石,可燃为薪。”(颇节本第一册344页)

伊本拔秃塔所志与马可·波罗同,亦云:“中国及契丹居民所燃之炭,仅用一种特产之土。此土坚硬,与吾人国内所产之黏土同。置之火中,燃烧与炭无异,且热度较炭为高。及成灰烬,复溶之水中,取出晒干,可以复用一次。”





101 物价腾贵时大汗散麦赈恤其民

应知君主见其人民麦丰价贱之时,即在诸州聚积多量,藏于大仓之中,保存甚善,可存三四年而不朽。所藏者诸麦皆具,如小麦、大麦、粟、稻、稷及其他谷类,悉皆有之。一旦诸谷中有若干种价贵之时,君主视其所需之量,取此谷于诸仓中,以贱价粜之人民。如每石(mesure)售价一别桑(besant),则以同一价值粜四石于需要粮食之人。

大汗市粜之法如此,务使其民不受价值腾贵之害。举凡管辖之地,办理之法悉皆如此。盖其在各地聚积粮粟,一旦查明,必须和粜之时,各人皆得有其必需之粮也。

此种赈粜之法,在中国史中发源最古。其制汉已有之,然大盛于唐。阿拉伯人《古行纪》有云:“粮价腾贵之时,中国算端取必须之粮食于公仓中,贱价粜之于民,由是物价腾贵不能持久。粮食云者,乃指米、麦、粟及其他诸谷也。”(Reinaud书第一册39页)元时设置公仓,特委官吏调和物价,俾在荒年物价不能逾常,而丰年农民亦不致受贱价之害。丰年米价甚贱之时,官吏聚积粮食于仓,以备荒年之用;荒年米价腾贵之时,官吏则出粮食于仓以粜之。





102 (一)大汗之赈恤贫民

大汗在物价腾贵之时,赈粜百物于民一事,业已备述于前。兹欲言者,其赈恤汗八里城贫民之事。大汗在此城中,选择贫户,养之邸舍之中,每邸舍六户、八户、十户不等,由是所养贫民甚众。每年赈给每户麦粮,俾其能供全年之食,年年如此。此外凡欲逐日至宫廷领取散施者,每人得大热面包一块,从无被拒者。盖君主命令如是散给;由是每日领取赈物之人,数逾三万。是盖君主爱惜其贫民之大惠,所以人爱戴之,崇拜如同上帝。

其朝廷之事既已备述于前,兹从汗八里城发足,进入契丹境内,续言其中伟大富庶之事。





102 (二)汗八里城之星者

汗八里城诸基督教徒、回教徒及契丹人中,有星者、巫师约五千人,大汗亦赐全年衣食,与上述之贫户同。其人惟在城中执术,不为他业。

彼等有一种观象器,上注行星宫位,经行子午线之时间,与夫全年之凶点。各派之星者每年用其表器推测天体之运行,并定其各月之方位,由是决定气象之状况。更据行星之运行状态,预言各月之特有现象。例如某月雷始发声,并有风暴,某月地震,某月疾雷暴雨,某月疾病、死亡、战争、叛乱。彼等据其观象器之指示,预言事物如此进行,然亦常言上帝得任意增减之。记录每年之预言于一小册子中,其名曰“塔古音”(Tacuin),售价一钱(gros)。其预言较确者,则视其术较精,而其声誉较重。

设有某人欲经营一种大事业,或远行经商,抑因他事而欲知其事之成败者,则往求此星者之一人,而语之曰:“请检汝辈之书,一视天象,盖我将因某事而卜吉凶也。”星者答云:须先知其诞生之年月日时,始能做答。既得其人年月日时以后,遂以诞生时之天象,与其问卜时之天象,比较观之,夫然后预言其所谋之成败。

应知鞑靼人用十二生肖纪年:第一年为狮儿年,次年为牛儿年,三年为龙儿年,四年为狗儿年,其数止于十二。所以每询某人诞生之年时,其人则答以某儿年某日某夜某时某分。此种时刻曾由亲属笔之于册。计年之十二生肖既满,复用此十二生肖继续计之。

上第102(二)重章并出剌木学本第二卷第三章。





102 (三)契丹人之宗教关于灵魂之信仰及其若干风习

其人是偶像教徒,前已言之。各人置牌位一方于房壁高处,牌上写一名,代表最高天帝。每日焚香礼拜,合手向天,叩齿三次,求天保佑安宁。所祷之事只此。此牌位之下,地上供一偶像,名称纳的该(Natigai),奉之如同地上一切财产及一切收获之神,配以妻子,亦焚香侍奉,举首叩齿祷之。凡时和年丰、家人繁庶等事,皆向此神求之。

彼等信灵魂不死。以为某人死后,其魂即转入别一体中。视死者生前之善恶,其转生有优劣。质言之,穷人行善者,死后转入妇人腹中,来生成为贵人。三生人一贵妇腹中,生为贵人。嗣后愈升愈高,终成为神。反之,贵人之子行恶者,转生为贱人之子,终降为狗。

其人语言和善,互相礼敬。见面时貌现欢容。食时特别洁净。礼敬父母,若有子不孝敬父母者,有一特设之公共法庭惩之。

各种罪人拘捕后,投之狱,而缢杀之。但大汗于三年开狱,释放罪人一次。然被释者面烙火印,俾永远可以认识。

现在大汗禁止一切赌博及其他诈欺方法,盖此国之人嗜此较他国为甚。诏令禁止之词有云:“我既用兵力将汝曹征服,汝曹之财产义应属我。设汝辈赌博,则将以我之财产为赌注矣。”虽然如此,大汗从未使用其权擅夺人民产业。

尚不应遗漏者,大汗诸臣朝仪之整肃也。诸臣行近帝座,距离约有半哩时,各卑礼致敬,肃静无声。由是在场者不闻声息。既无呼唤之音,亦无高声谈话者。凡臣下莅朝时,皆持有一小唾壶,无人敢唾于地,欲唾时揭壶作礼而唾。彼等尚携有白皮之靴,其为君主召见之人,人殿时易此白靴,以旧靴付仆役,俾殿中金锦地衣不为旧靴所污。

上第102(三)章并出剌木学本第二卷第三十六章。

剌木学本之标题虽作“鞑靼人之宗教……”然本章之内容显指中国人之信仰,惟将纳的该神名掺入耳。





103 契丹州之开始及桑干河石桥

应知君主曾遣马可·波罗阁下奉使至西方诸州。彼曾志其经行之事于下:自汗八里城发足,西行亘四月程。所以我为君等述其在此道上往来见闻之事。

自从汗八里城发足以后,骑行十哩,抵一极大河流,名称普里桑干(Puilsangin、Pulisangan)。此河流叩齿海洋。商人利用河流运输商货者甚夥。河上有一美丽石桥,各处桥梁之美鲜有及之者。桥长三百步,宽逾八步,十骑可并行于上。下有桥拱二十四,桥脚二十四,建置甚佳,纯用极美之大理石为之。桥两旁皆有大理石栏,又有柱,狮腰承之。柱顶别有一狮。此种石狮巨丽,雕刻甚精。每隔一步有一石柱,其状皆同。两柱之间,建灰色大理石栏,俾行人不致落水。桥两面皆如此,颇壮观也。

兹述此美桥毕,请言其他新事。

剌木学本第二卷第二十七章增订之文如下:

“此普里桑干桥有二十四拱,承以桥脚二十五(内有桥台二),皆立基水中,用蛇纹石建筑,颇工巧。桥两旁各有一美丽栏杆,用大理石板及石柱结合,布置奇佳。登桥时桥路较桥顶为宽,两栏整齐,与用墨线规者无异。”

“桥口(两方)初有一柱甚高大,石龟承之,柱上下皆有一石狮。”

“上桥又见别一美柱,亦有石狮,与前柱距离一步有半。”

“此两柱间,用大理石板为栏,雕刻种种形状。石板两头嵌以石柱,全桥如此。此种石柱相距一步有半,柱上亦有石狮。既有此种大理石栏,行人颇难落水,此诚壮观,自入桥至出桥皆然也。”





104 涿州大城

从此石桥首途,西行二十哩,沿途皆见有美丽旅舍、美丽葡萄园、美丽园囿、美丽田亩及美丽水泉。行毕然后抵一大而美丽之城,名曰涿州(Giogiu)。内有偶像教徒之庙宇甚众,居民以工商为业,织造金锦丝绢及最美之罗,亦有不少旅舍以供行人顿止。

从此城首途,行一哩,即见两道分歧:一道向西,一道向东南。西道是通契丹之道,东南道是通蛮子地域之道。

遵第一道从契丹地域西行十日,沿途皆见有环以城垣之城村,及不少工商繁盛之聚落,与夫美丽田亩,暨美丽葡萄园,居民安乐。 【1】 惟其地无足言者,兹仅述一名太原府(Tainfu)之国。

注释

【1】 剌木学本第二卷第二十八章增订之文录下:

“从此地输酒入契丹境内,缘契丹境内不酿酒也。此处亦饶有桑树,其桑叶足使居民养蚕甚多。居民颇有礼貌,盖沿途城市密接,行人来往甚众,商货灌输甚多故也。”

“行上述十日之五日毕,即闻人言,有一城较太原府更为壮丽。城名阿黑八里(Achbaluch),自此达彼,皆属君主游猎禁地。除君主及诸宗王暨名列打捕人匠、总管府之人外,无人敢在其地猎捕。然在其地界外,只须身为贵人,可以随意行猎,顾大汗从前未至此地行猎。野兽繁殖甚众,尤以山兔为多,颇伤全境禾稼。大汗闻悉此事,遂率领全宫之人至此捕获野兽无数。”





105 太原府国

自涿州首途,行此十日毕,抵一国,名太原府。所至之都城甚壮丽,与国同名,工商颇盛,盖君主军队必要之武装多在此城制造也。其地种植不少最美之葡萄园,酿葡萄酒甚饶。契丹全境只有此地出产葡萄酒,亦种桑养蚕,产丝甚多。

自此太原府城,可至州中全境。向西骑行七日,沿途风景甚丽,见有不少城村,环以墙垣;其中商业及数种工业颇见繁盛,有大商数人自此地发足前往印度等地经商谋利。

行此七日毕,抵一城,名平阳府(Pianfu)。城大而甚重要,其中恃工商业为活之商人不少,亦产丝甚饶。

兹置此事不言,请言一名哈强府(Cacianfu)之大城。然欲述此城,须先言一名称该州(Caigiu)或太斤(Thaigin)之名贵堡塞。





106 该州或太斤堡

从平阳府发足,西向骑行二日程,则见名贵堡塞该州。昔为此地一国王所筑,王名黄金王。堡内有一宫,极壮丽,宫中有一大殿。昔日此地国王皆有绘像列于其中,像作金色,并其他美色,颇为娱目。诸像之成,乃由君临本地之王陆续为之。

兹请据此堡人之传说,一述此黄金王与长老约翰之一故事。

据说昔日此黄金王与长老约翰战,黄金王据险要,长老约翰既难进兵,亦不能加害此王,缘是甚怒。长老约翰时有幼年骑尉十七人,相率建议与长老约翰,愿生擒黄金王以献。长老约翰答言极愿彼等为此,事成必厚宠彼等。

诸骑尉别其主长老约翰以后,结成一种骑尉队伍,往投黄金王所。及见王,遂语之曰:“彼等来自外国,愿仕王所。”王慰而录用之,不虞其有恶意也。由是此种怀有异心之骑尉,遂为黄金王臣,竭尽臣职,王甚宠之,置之左右。

彼等留王所亘二年,所行所为,毫不微露叛意。一日随王出游,其他扈从之人甚少,盖王信任彼等,而不虞有他故也。迨渡一河后,河距堡约有一哩,时仅彼等与王相随,遂互议曰:“执行所谋,此其时矣。”于是皆拔剑胁王立随彼等行,否则杀之。王见状,既惊且惧,语诸人曰:“汝曹所言何事,欲余何往?”诸人答曰:“往吾主长老约翰所。”





107 长老约翰之如何待遇黄金王

黄金王闻言,忧郁几频于死,语诸人曰:“我既宠待汝曹,何不悯而释我,俾不致陷敌手。脱汝曹为此,则犯大恶而为不义矣。”诸人答言:“势必出此。”遂拘之至其主长老约翰所。

长老约翰见黄金王至,大喜,而语之曰:“汝既来此,将不获善待。”王不知所答。长老约翰立命人监守之,命其看守牲畜,然未加虐待,由是沦于牧畜之役矣。长老约翰怒此王甚,欲抑贱之,而表示其不足与彼相侔也。

如是看守牲畜垂两年。长老约翰招之来前,以礼待之,赐以华服,而语之曰:“王,今汝知否势不我敌?”王答曰:“固也。我始终皆知我力不足与君抗。”长老约翰乃曰:“我别无他求。自今而后,将以礼待,而送君归。”于是赠以马匹鞍辔,命人护送归其本国。嗣后黄金王遂称藩而奉长老约翰为主君。

兹置此黄金王故事不言,请言他事,以续本书。





108 哈剌木连大河及哈强府大城

离此堡后,向西骑行约二十哩,有一大河,名哈剌木连(Karamouren)。河身甚大,不能建桥以渡,盖此河流宽而深也。此河流入环绕世界全土之大洋,河上有城村数处,皆有城墙,其中商贾甚夥,河上商业繁盛;缘其地出产生姜及丝不少,禽鸟众至不可思议,野鸡三头仅值威尼斯银钱(gros)一枚。 【1】

渡此河后,向西骑行二日,抵一名贵城市,名称哈强府(Cacianfu)。居民皆是偶像教徒。兹应为君等申言者,契丹居民大致皆属偶像教徒也。 【2】 此城商业茂盛,织造种种金锦不少。

此外别无可述,兹请接言一名贵城市,此城是一国之都会,名曰京兆府(Quengianfu)。

注释

【1】 剌木学本第二卷第三十二章增订之文云:“此河附近之地,种植一种大竹,其数颇众,(其圆径)致达一尺至一尺有半者。”

【2】 “城中商业茂盛,艺业繁多。土产之中,饶有丝、姜、高良姜(galangal)、唇形科植物(lavande)及吾国未见之其他不少香料。”(剌木学本第三十三章)





109 京兆府城

离上述之哈强府城后,





【1】 西向骑行八日,沿途所见城村,皆有墙垣。工商发达,树木园林既美且众,田野桑树遍布,此即蚕食其叶而吐丝之树也。居民皆是偶像教徒,土产种种禽鸟不少,可供猎捕畜养之用。

骑行上述之八日程毕,抵一大城,即前述之京兆府(Quengianfu)是已。城甚壮丽,为京兆府国之都会。昔为一国,甚富强,有大王数人,富而英武。惟在今日,则由大汗子忙哥剌(Mangalay)镇守其地。大汗以此地封之,命为国王。此城工商繁盛,产丝多,居民以制种种金锦丝绢,城中且制一切武装。凡人生必需之物,城中皆有,价值甚贱。

城延至西,居民是偶像教徒。城外有王宫,即上述大汗子国王忙哥剌之居也。宫甚壮丽,在一大平原中,周围有川湖泉水不少,高大墙垣环之,周围约五哩。墙内即此王宫所在,其壮丽之甚,布置之佳,罕有与比。宫内有美丽殿室不少,皆以金绘饰。此忙哥剌善治其国,颇受人民爱戴,军队驻扎宫之四围,游猎为乐。

今从此国首途,请言一名关中(Cuncun)之州。州境全在山中,道路难行。

注释

【1】 剌木学本第二卷第三十四章之异文云:“离哈强府后,西向骑行七日,沿途陆续见有城村,皆有墙垣环之。商业茂盛,并见有园囿及耕种之田亩不少。全境桑树遍布,此树用以产丝。居民大多数是偶像教徒,然亦有基督教徒、突厥种人、聂思脱里教徒及若干回教徒。可在其地猎取不少野兽,并可捕取不少种类禽鸟。别又骑行七日程,抵一名贵大城,名称京兆府(Quenzanfu)。”剌木学本此段异文未可忽视。盖1556年之法文译本(65页)、Mailer之拉丁文译本(89页)、Bergeron本,并证明此文之非误。然则应承认马可·波罗之哈强府所指者乃二城,并在黄河西二日程。其一是前章之华州,其一城不属京兆府也(剌木学本第三三章)。

由是推之,势须将马可·波罗渡黄河处远徙于北。考宋、金时陕西东部仅有二府:一为京兆,一为延安。延安在平阳西,东距黄河二日程,与《行纪》所言之距离完全相符。又考地图,太原、西安间有两道:一为经行平阳、蒲州一道,一为经行延安一道。马可·波罗来去之时,似从此道去而循彼道归,由是哈强、京兆间两个七日程之记载始得其解。抑况野兽繁殖可供捕猎之处,只能在山地中也。但是哈强、延安音韵难以相对,殆马可·波罗后此改订其《行纪》时,漏言延安,误记哈强欤。





110 难于跋涉之关中州

离上述忙哥剌之宫室后,西行三日,沿途皆见有不少环墙之乡村及美丽平原。居民以工商为业,有丝甚饶。行此三日毕,见有高山深谷,地属关中州矣。其中有环墙之城村,居民是偶像教徒,恃地之所产及大林中之猎物以为生活。盖其地有不少森林,中有无数猛兽,若狮、熊、山猫及其他不少动物,土人捕取无数,获利甚大。由是逾山越谷,沿途见有不少环墙之城村、大森林及旅人顿止之大馆舍。

现从此州发足,将言别一地域,说详后方。





111 蛮子境内之阿黑八里大州

骑行逾关中诸山,行二十日,抵一蛮子之州,名阿黑八里(Acbalec)。州境全处平原中,辖有环墙之城村甚众,隶属大汗。居民是偶像教徒,恃工商为活。此地出产生姜甚多,输往契丹全境,此州之人恃此而获大利。彼等收获麦稻及其他诸谷,量多而价贱,缘土地肥沃,宜于一切种植也。

主要之城名称阿黑八里。

此平原广延二日程,风景甚丽,内有环墙之城村甚众。行此二日毕,则见不少高山深谷丰林。由此道西行二十日,见有环以墙垣之城村甚众。居民是偶像教徒,恃土之所出,及牲畜,与夫饶有之野兽猎物为活。亦有不少兽类产生麝香。

兹从此地发足,请依次历言其他诸地。





112 成都府

向西骑行山中,经过上述之二十日程毕,抵一平原,地属一州,名成都府(Sindufu、Sindafu),与蛮子边境为邻。此州都会是成都府,昔是强大城市,历载富强国王多人为主者垂二千年矣。然分地而治,说如下文:

此州昔有一王,死时遗三子,命在城中分地而治,各有一城。然三城皆在都会大城之内,由是此三子各为国王,各有城地,各有国土,皆甚强大。大汗取此三王之国而废其王。

有一大川,经此大城。川中多鱼,川流甚深,广半哩,长延至于海洋,其距离有八十日或百日程,其名曰江水(Quiansuy)。水上船舶甚众,未闻未见者,必不信其有之也。商人运载商货往来上下游,世界之人无有能想象其盛者。此川之宽,不类河流,竟似一海。

城内川上有一大桥,用石建筑,宽八步,长半哩。桥上两旁,列有大理石柱,上承桥顶。盖自此端达彼端,有一木制桥顶,甚坚,绘画颜色鲜明。桥上有房屋不少,商贾工匠列肆执艺于其中。但此类房屋皆以木构,朝构夕拆。桥上尚有大汗征税之所,每日税收不下精金千量。

居民皆是偶像教徒。出此城后,在一平原中,又骑行五日。见有城村甚众,皆有墙垣。其中纺织数种丝绢,居民以耕种为活。其地有野兽如狮、熊之类不少。

骑行此五日毕,然后抵一颇遭残害之州,名称土番(Tibet),后此述之。





113 土番州

行上述之五日程毕,人一极广森林,地属土番州矣。此州昔在蒙哥汗诸战中,曾受残破,所见城村,业已完全削毁。

其中颇有大竹,粗有三掌,高至十五步,每节长逾三掌。商贾旅人经行此地者,于夜间习伐此竹燃火,盖火燃之后,爆炸之声甚大,狮、熊及其他野兽闻之惊走,不敢近火。此州自经残破以后,不复有居民,遂致野兽繁殖。若无此竹燃火,爆炸作声,使野兽惊逃,则将无人敢经行其地。

兹请言此竹如何能发大声响之理。其地青竹甚多,行人伐之,燃其数茎,久之皮脱,直裂,爆炸作声,其声之巨,夜间十哩之地可闻。 【1】 若有人未预知其事而初闻其声者,颇易惊惶致死,然熟悉其事者不复惊惧。其实未习闻此声者,应取棉塞耳,复取所能有之衣服蒙其头面,初次如此,嗣后且屡为之,迄于习惯而后已。

马匹亦然。设其未曾习闻此声,初次闻之,即断其索勒,如是丧失牲口者,已有旅客数人。如欲保存其牲口者,势须系其四蹄,蒙其首与眼耳,然后可能驾驭,马匹数闻此声以后,始不复惊。我敢断言初闻此声者,必以为世上可怖之声,无有逾于此者矣。复次虽有此种预防之法,有时不能免狮、熊及其他野兽之为大害,盖其地野兽甚众也。

如是骑行二十日,不见人烟,行人势须携带一切食粮,从来不免遭遇此种可畏而为害之野兽,末后始见环墙之城村。 【2】 此类城民有一种婚俗,兹请为君等述之。

此地之人无有取室女为妻者,据称女子未经破身而习与男子共寝者,毫无足重。凡行人经过者,老妇携其室女献之外来行人,行人取之惟意所欲,事后还女于老妇,盖其俗不许女子共行人他适也。 【3】 所以行人经过一堡一村或一其他居宅者,可见献女二三十人,脱行人顿止于土人之家,尚有女来献。凡与某女共寝之人,必须以一环或一小物赠之,俾其婚时可以示人,证明其已与数男子共寝。凡室女在婚前皆应为此,必须获有此种赠物二十余事。其得赠物最多者,证其尤为人所喜爱,将被视为最优良之女子,尤易嫁人。然一旦结婚以后,伉俪之情甚笃,遂视污及他人妻之事为大侮辱。

其事足述,故为君等言之。我国青年应往其地以求室女,将必惟意所欲,而不费一钱应人之请也。

居民是偶像教徒,品行极恶,对于窃盗或其他恶行,绝不视为罪过。彼等且为世上最好揶揄之人,恃所猎之兽、牲畜所产之物及土地所产之果实为生。尚有不少兽类出产麝香, 【4】 土语名曰古德里(Gouderi)。此种恶人畜犬甚多,犬大而丽,由是饶有麝香。境内无纸币,而以盐为货币。衣服简陋,所衣者为兽皮及用大麻或粗毛所织之布。其人自有其语言,而自称曰土番人。此土番地构成一极大之州,后此将申言之。

剌木学本第二卷第三十七章增订之文如下:

注释

【1】 “一到夜间,行人结此种青竹为束,置于其幕若干距离之处,然后燃之。竹因热力皮脱而后炸裂,发声可畏,其声之巨,相距二哩之地可闻。”

【2】 “由是骑行此种荒野亘二十日,不见人烟食粮,仅在每三四日程之地,或者一得生活必需之物。行此多日程毕,始见若干堡镇,建于悬崖之上或山岭之巅,然后入一有民居种植之地,遂不复畏惧野兽矣。”

【3】 “由是商队至止结幕以后,有女待嫁之母,立携其女至幕,各求诸商选择其女共寝。由是较动人之幼女为商人所择,其他皆失意而归。被选择者与商人共处,迄于商人行时,至是还女于母。”

【4】 “此种产生麝香之兽甚众,其味散布全境,盖每月产麝一次。前次(第七十一章)已曾言及此种兽类,脐旁有一胞,满盛血,每月胞满血出,是为麝香。此种地带有此类动物甚众,麝味多处可以嗅觉。”





114 重言土番州

此土番州是一极大之州,居民自有其语言,并是偶像教徒,前已言之。地与蛮子及其他不少州郡相接,乐为盗贼,其境甚大,内有八国及环墙之城村甚众。有数地川湖中饶有金沙,其量之多,足以惊人。肉桂繁殖,珊瑚输入之地,即是此州。其价甚贵,盖居民乐以此物为其妻及其偶像之颈饰也。此州亦有种种金锦丝绢,并繁殖不少香料,概为吾国所未见者。

应知其地有最良之星者及最巧之魔术家,为诸州之所不及。其人常施魔术,作最大灵异,闻之见之足以惊人,所以我在本书不为君等言及。盖人将大为惊异,而不得何种良好印象也。

其人衣服简陋,前已述之。有无数番犬,身大如驴,善捕野兽。亦有其他猎犬数种。并有良鹰甚多,其飞甚疾,产自山中,训练以作猎禽之用。

关于此土番州之诸事,既已略述于前,将置此不言。请言别一名称建都(Caindu)之州。惟关涉土番者,君等应知其隶属大汗,一如本书随时著录之其他国土州郡之隶属东方君主阿鲁浑(Argoun)之子者无异。顾此东方君主以宗王及藩臣之资格,受地于大汗,则谓诸地并属大汗,亦无不可。自本州以后,凡将著录之其他诸州,虽未特别注明其隶属大汗者,君等亦须作是解也。

兹置此事不言,请言建都州。





115 建都州

建都是西向之一州,隶属一王。





【1】 居民是偶像教徒,臣属大汗。境内有环墙之城村不少。有一湖,内产珍珠,然大汗不许人采取。盖其中珍珠无数,若许人采取,珠价将贱,而不为人所贵矣。惟大汗自欲时,则命人采之,否则无人敢冒死往采。

此地有一山,内产一种突厥玉(turquoise),极美而量颇多,除大汗有命外,禁人采取。

此州有一种风俗而涉及其妻女者,兹为君等述之。设有一外人或任何人奸其妻女、其姊妹或其家之其他妇女者,居民不以为耻,反视与外人奸宿后之妇女为可贵。以为如是其神道偶像将必降福,所以居民情愿听其妇女与外人交。

设其见一外人觅求顿止之所,皆愿延之来家。外人至止以后,家主人命其家人善为款待,完全随客意所欲;嘱毕即离家而去,远避至其田野,待客去始归。客居其家有时亘三四日,与其妻女、姊妹或其他所爱之妇女交,客未去时,悬其帽或其他可见之标识于门,俾家主人知客在室未去。家主人见此标识,即不敢入家。此种风俗全州流行。

至其所用之货币,则有金条,按量计值,而无铸造之货币。其小货币 【2】 则用盐。取盐煮之,然后用模型范为块,每块约重半磅,每八十块值精金一萨觉(saggio),则萨觉是盐之一定分量。其通行之小货币如此。

境内有产麝之兽甚众,所以出产麝香甚多。其产珠之湖亦有鱼类不少。野兽若狮、熊、狼、鹿、山猫、羚羊以及种种飞禽之属,为数亦夥。其他无葡萄酒,然有一种小麦、稻米、香料所酿之酒,其味甚佳。此州丁香繁殖,亦有一种小树,其叶类月桂树叶,惟较狭长,花白而小,如同丁香。其地亦产生姜、肉桂甚饶,尚有其他香料,皆为吾国从来未见者,所以无须言及。

此州言之既详,但尚有言者:若自此建都骑行十日,沿途所见环墙之城村仍众,居民皆属同种,彼等可能猎取种种鸟兽。骑行此十日程毕,见一大河,名称不里郁思(Brius),建都州境止此。河中有金沙甚饶,两岸亦有肉桂树,此河流入海洋。

此河别无他事足述。兹置之不言,请言别一名称哈剌章(Carajan)之州。

剌木学本第二卷第三十八章增订之文如下:

注释

【1】 “然自经大汗征服后,遣官治之。我言其为西向之一州者,切勿以为此地属于西域,盖吾人来自东北方诸地,而此地在吾人所遵行程之西也……其都会亦名建都,位置距州北境不远。其地有一大咸湖,中有白珠甚众,然珠形不圆。”

【2】 “此国中有咸水,居民取盐于其中,置于小釜煮之,水沸一小时则成盐泥,范以为块,各值二钱(denier)。此种盐块上凸下平,置于距火不远之热砖上烤之,俾干硬,每块上盖用君主印记,其印仅官吏掌之,每八十盐块价值黄金一萨觉。第若商人运此货币至山中僻野之处,则每金一萨觉可值盐块六十、五十,甚至四十,视土人所居之远近而异。诸地距城较远而不能常售卖其黄金及麝香等物者,盐块价值愈重,纵得此价,采金人亦能获利,盖其在川湖可获多金也。”

“此种商人且赴山中及上言土番州之其他诸地,其地盐块亦通行,商人亦获大利。盖其地居民用此盐为食,视其为必需之物,城居之民则用碎块,而将整块作货币使用也。”





116 哈剌章州

渡此河后,立即进入哈剌章(Carajan)州。州甚大,境内致有七国,地延至西,居民是偶像教徒,而臣属大汗。汗之一子君临此地,其名曰也先帖木儿(Essentimour),是为一极大而富强之国王,为人贤明英武,善治其国。

从前述之河首途,西向行五日,见有环墙之城村甚众,是一出产良马之地。人民以畜牧耕种为生,自有其语言,颇难解。行此五日毕,抵一主城,是为国都,名称押赤(Jacin)。城大而名贵,商工甚众。人有数种,有回教徒、 【1】 偶像教徒及若干聂思脱里派之基督教徒。颇有米麦,然此地小麦不适卫生,不以为食,仅食米,并以之掺和香料酿成一种饮料,味良而色明。所用货币则以海中所出之白贝而用作狗颈圈者为之。八十贝值银一两,等若威尼斯城钱(gros)二枚,或二十四里物(livres)。银八两值金一两。

其地有盐井而取盐于其中,其地之人皆恃此盐为活;国王赖此盐收入甚巨。

居民不以与他人妻奸宿为异,只须妻同意可矣。

尚有一湖甚大,广有百哩,其中鱼类繁殖,鱼最大,诸类皆有,盖世界最良之鱼也。尚有为君等言者,此地之人食生肉,不问其为羊、牛、水牛、鸡之肉,或其他诸肉, 【2】 赴屠市取兽甫剖腹之生肝,归而脔切之,置于热水掺和香料之酌料中而食。其食其他一切生肉,悉皆类此。其食之易,与吾人之食熟食同。

兹记述此事毕,然尚有关于哈剌章州之若干事而须续言者。

注释

【1】 剌木学本作“回教徒,然偶像教徒最众”。

【2】 “脔切肉为细块,先置盐中腌之,然后用种种香料调合,是为贵人之食。至若贫民,则将脔切之肉置于蒜制之酌料中而食。其食之易,与吾人食熟食同。”(剌木学本第二卷第三十九章)





117 重言哈剌章州

从前述之押赤城首途后,西向骑行十日,至一大城,亦在哈剌章州中,其城即名哈剌章。居民是偶像教徒,而臣属大汗,大汗之别一子名忽哥赤(Cogacin)者为其国王。此地亦产金块甚饶,川湖及山中有之,块大逾常,产金之多,致于交易时每金一两值银六两。彼等亦用前述之海贝,然非本地所出,而来自印度。

此州出产毒蛇大蟒,其躯之大,足使见者恐怖;其形之丑,闻者惊异。兹请言其巨大之形。

其身长有至十步者,或有过之,或有不及;粗如巨靴,则巨有六掌矣。近头处有两腿, 【1】 无足而有爪,如同鹰、狮之爪。头甚大,其眼大逾一块大面包,其口之大,足吞一人全身。其形丑恶狞猛,人兽见之者,无不惊惧战栗。

捕之之法如下:应知此种大蟒日中避热,藏伏土内,夜出捕食诸兽,而饮水于川湖及泉中。其躯之重,夜出求食时,曳尾而行,在沙中成一,深坑,如曳一满盛酒浆之桶而行者无异。猎人取之之法,仅植猎具于其所过之道上,盖其逆知蟒必循旧道而归也。其法深植一木桩于地,桩上置一铁,形同剃刀,锋甚锐利,然后以沙掩之,俾蟒行时不见此机。蟒所经行之处,植此种桩铁数具,蟒归时行其上,剖腹至脐,立死。

猎人捕之之法如此, 【2】 捕得以后,取其腹胆售之,其价甚贵。盖此为一种极宝贵之药品,设有为疯狗所啮者,用此胆些许,量如一小钱(denier)重,饮之立愈。 【3】 设有妇女难产者,以相当之量治之,胎儿立下。此外凡有疾如癣疥或其他恶疾者,若以此胆些许治之,在一最短期间内,必可痊愈,所以其售价甚贵。

彼等亦售此蟒肉,盖其味佳,而人亦愿食之也。此种蟒蛇饥甚之时,偶亦至狮、熊或其他大野兽巢穴之中捕食其子,父母不能救,亦捕取大兽而食,兽亦不能自防。

此州亦产良马,躯大而美,贩售印度。然应知者,人抽取其尾筋二三条,俾其不能用尾击其骑者。尚应知者,其人骑马用长骑(montent long)之法,与法兰西人同; 【4】 其甲胄用熟皮为之,执矛盾弩,并以毒药傅其矢。 【5】 大汗未征服其地时,其人有一种恶事,请为君等述之。脱有人体态威严尊贵,或体貌完全无缺,而顿止于土人之家者,土人即毒杀之,或以他法杀之。其杀之者非为夺取其资,乃因其以为被害者之良魂良宠良识,完全留存于身死之家。由是在大汗侵略其地以前,杀人甚众;但在侵略以后,质言之,约有三十五年来,土人不复再犯此罪,而弃此恶行。盖大汗有禁,而土人畏威也。

既述此地毕,请于后章接言别地。

剌木学本第二卷第四十章增订之文如下:

注释

【1】 “上身近头处有两小腿,各具三爪,如同虎爪。眼大逾四钱(sous)之面包,颇光亮。牙长而锐。其躯最小者不过八步、六步或五步。”

【2】 “蟒死后,立有乌鸦聚噪,猎人闻声,知蟒已死,循声觅取蟒躯剥之。”

【3】 “一钱(denier)之量置酒中饮之。”

【4】 “其人骑马用长镫,如法兰西人习自吾人之法。兹言长镫者,盖鞑靼人及其他一切民族几尽用短镫,以便易于引弓,而于发矢时在马上起立也。”

【5】 “确闻此辈做恶者多藏毒药于身,俾事泄被捕时服之,免受拷问。服毒后,死甚速。然其君主知其人有此自毙之法,曾常备有狗矢,见罪人服毒时,立取狗矢强使吞之,俾将毒药吐出,由是对于此辈恶人有解毒之法矣。”(剌木学本注云:“Strabon书第三卷末称西班牙人常携带毒药而自杀。”)





118 金齿州

离大理府后,西向骑行五日,抵一州,名称匝儿丹丹(Zardandan,即金齿)。居民是偶像教徒,而臣属大汗。都会名称永昌(Vocian)。此地之人皆用金饰齿,别言之,每人齿上用金作套如齿形,套于齿上,上下齿皆然。男子悉如此,妇女则否。其俗男子尽武士,除战争、游猎、养鸟之外,不作他事。一切工作皆由妇女为之,辅以战争所获之俘奴而已。

妇女产子,洗后裹以襁褓,产妇立起工作,产妇之夫则抱子卧床四十日。卧床期间,受诸亲友贺。其行为如此者,据云妻任大劳,夫当代其受苦也。

彼等食一切肉,不问生熟,习以熟肉共米而食。饮一种酒,用米及香料酿造,味甚佳。其货币用金,然亦用海贝。其境周围五月程之地无银矿,故金一两值银五两。商人携多银至此易金而获大利。

其人无偶像,亦无庙宇,唯崇拜其族之元祖,而云:“吾辈皆彼所出。”

彼等无字母,亦无文字,斯亦不足为异。盖其地处蛮野之区,入境不易,遍布高山大林,颇难通行;空气不洁,外人之入境者,必有丧命之忧。土人缔约,取一木杖,或方或圆,中分为二,各刻画二三符记于上。每方各执一片,负债人偿还债务后,则将债权人手中所执之半片收回。

尚应言者,此押赤、大理、永昌三州无一医师,如有人患病,则召看守偶像之巫师至;病者告以所苦,诸巫师立响其乐器,而为歌舞,迨其中一人昏厥如死始止。此事表示鬼降其人之身,同伴巫师与之语,问病者所患何疾,其人答曰:“某神罚其病卧,盖其侮此神,而神不欢也。”其他诸巫师遂祝神曰:“请汝宥其过,而愈其疾,任汝取其血或他物以为报。”祝毕,静听卧地人附身之神做答,如答语为“此病者对于某神犯有某种恶行,神怒,不许宥之”,则犹言病者应死。

然若病者应愈,则答诸人,命献羊两三头,做饮料十种或十二种。其价甚贵,味甚佳,而置香料亦甚众;并限此种羊应有黑首,或神所欲之其他颜色,如是诸物应献某神,并应有巫师若干、妇女若干与俱。献诸物时,应为赞词歌颂,大燃灯焚香。病者若应愈,神之答复如此。病者亲属闻言,立奉命而行,其倒地之巫师遂起。

诸人立时献所索某色之羊,杀而洒其血于所指之处,然后在病人家熟其肉,延巫师、妇女如指定之数,祭祀此神。诸人齐至,预备已毕,遂开始歌舞,作乐器而祝神,取食物、饮料、肉、沉香及香灯甚众,并散饮食及肉于各处,如是历若干时,复见巫师中之一人倒地,口喷涎沫,诸巫师询此人曰:“神是否已宥病者?”有时答曰:“宥。”有时答曰:“否。”若答曰否,则尚应献神复欲之物,俾病者获宥。重献既毕,其人乃云:“病者获宥,其病将愈。”诸人得此答复,乃言神怒已息,如是欣然聚食,其晕厥于地者亦起,与诸人同食。诸人饮食毕,各归其家。至是病者立起,其病若失。

此民族之风俗及其恶习,既已叙述于前,兹请不复再言此州,接言其他诸州,依次切实述之于后。





119 大汗之侵略缅国及班加剌国

应知昔在匝儿丹丹州永昌国中有一大战,前忘言之。今在本书详细述其始末。

基督诞生后1272年时,大汗遣多军戍守此永昌及哈剌章等国,防备恶人之为害。时尚未遣皇子出镇其地。嗣后始命已故皇子某之子皇孙也先帖木儿为其地国王。所以当时缅(Mien)及班加剌(Bangala)之国王据有土地、财货、人民甚众;其势甚强,尚未臣属大汗。然其后不久,大汗即征服之,而取上述之两国。

此缅及班加剌之国王,闻大汗军至永昌,自云:彼为国主,势力较强,将尽歼大汗军,俾其不再遣军至此。

于是此王大聚其众,与夫兵械,得大象二千头。各象上负木楼,极坚固,楼中载战士十人或十二人以战。别有步骑六万,其军如是之众,具见其为强主,而此军足以鏖战也。

彼做此大筹备毕,不久即遣军出发,往敌鞑靼。沿途无事足述。行至大汗军顿止处三日程之地结营,俾其军队休息,时大汗军在匝儿丹丹国内永昌城中也。





120 大汗军将与缅国国王之战

鞑靼军统将名纳速剌丁(Nacireddin),闻知此国王确以其众至,而己所将众仅一万二千骑,初颇迟疑。已而自信雄武,善将兵,而习于战阵,遂激励其众,使用种种方法以自防。盖其所部为善战之武士也。于是鞑靼军一万二千骑,乘良骑,相率进至永昌平原而迎敌,在其地列阵以待。其为此者,因其用善策而有良将,恃此平原附近有一极大森林,树木遍布也。

兹暂置鞑靼不言,请言其敌。

缅国王休息其军毕,自其地率军出发,至于永昌平原,距鞑靼备战之处有一哩,整顿象楼,列战士于楼中,复列步骑备战。布置既毕,开始进军击敌。鞑靼见之,伪作毫不惊异之状,仍整列前进,及两军既接,甫欲交锋之时,鞑靼军马见敌军战象,大惊骇,遂退走,缅国王乘势率众进逐。





121 重言此战

鞑靼见之大恚怒,不知所为,盖其明见其在战前若不将马勒回,将必全军败没也。然其将知战略,一如早已预知者然,立命各人下骑,系马于附近森林树上,已而引弓发矢射象,鞑靼善射,无人能及,前进之象,未久死伤过半。敌军士卒射不如鞑靼之精,亦伤亡甚众。

时矢如雨下,象负伤者奔逃,践踏声大,俨若世界土地全陷,诸象逃散入林中,楼甲等一切战具尽毁。

鞑靼见象逃不敢再战,遂重登骑,进击其敌,持刀与骨朵与敌酷战。缅王军虽众,然非善战之士,亦未习于战,否则鞑靼军少,绝不能以少胜众也。

由是见刀与骨朵互下者有之,骑土战马被屠杀者有之,头足臂手斫断者有之,死伤卧地而永不能再起者无算。两军呼喊之声甚巨,脱有雷声而莫能闻。双方战斗奇烈,终由鞑靼获胜。

此战始于不利于缅王军队之时,迄于正午,缅王军不能复敌,遂溃而逃。鞑靼见敌败走,乘胜追逐,杀戮无算,见之诚可悯也。追击久之,始止。已而还至林中,捕取逃象,谋捕象,伐象藏伏处之大树。虽如此,若无缅军俘虏之助,仍不能得。盖象性较他兽为灵,俘虏识其性,教以捕之之法,得二百余头。自此战后,大汗始有多象。此国王败于鞑靼之策略,经过如此。





122 下一大坡

离前述之州后,不久至一大坡,亘两日有半,行人始终循此坡下行。在此距离全途之中,无事足述。仅见有一重要处所,昔为一大市集,附近之人皆于定日赴市。每星期开市三次,以其金易银。盖彼等有金甚饶,每精金一两易纯银五两,银价既高,所以各地商人携银来此易金,而获大利。至若携金来市之土人,无人知其居处。盖土人畏惧恶人,皆居僻地,不在通道之上;居宅在荒野处所,与人隔绝,使外人不能为患。土人不欲世人知其居处,从不许人随行。

行此二日有半,下坡讫,抵于一州,位置南方,与印度邻近,其名曰阿缅(Amien)。复自是骑行十五日。所经之地,路鲜行人,皆行丛薄中,其间有象、犀及其他野兽甚众,既无人烟,亦无居宅。所以吾人不复言此野地,缘其间无足述者。兹请述一故事。





123 上缅国之都城城有二塔一金塔一银塔

行人经行上述之荒地中,人烟断绝,必须携带食粮。骑行十五日毕,至此缅州,主要城市亦名阿缅(Amien)。城极大而名贵,是为国之都城,居民是偶像教徒,自有语言,臣属大汗。城中有一物,极富贵,请为君等述之。

昔日此城有一富强国王,弥留时,命在其墓上建二塔, 【1】 一金塔,一银塔,以石为之。其一上傅以金,有一指厚,全塔俨若金制;其一塔建筑与金塔同,上傅以银,全塔俨若银制。每塔高十步,其大与其高度相称,上部皆圆形,周围悬铃,金塔悬金铃,银塔悬银铃,风起作声。 【2】 国王为其生前光荣及死后英灵,特建此二塔,诚为世界最美观之物,太阳照之,光明灿烂,远处可见。

大汗征服其地之法如此:先是朝中有幻人术者甚众,大汗一日与之言,欲彼等前往征服缅州,将辅以良助及善将之人。语毕,命彼等做一切适于一军之筹备,遣将一人及士卒一队辅之。彼等遂行,至于缅州,全取其地,及见城中有此金、银二塔,甚为惊奇,请命于大汗,如何处置。大汗知其王建此为死后安灵之所,命彼等切勿毁坏,保存如故,由是世界之鞑靼无敢手触死者之物者。

此州有象及野牛甚众,余若美丽鹿獐及其他大兽亦复甚多。

既述此缅州毕,请言一名班加剌(Bangala)之别地。

剌木学本第二卷第四十四章增订之文如下:

注释

【1】 “二塔为三尖塔(pyramide)形,建于墓之两端,全用大理石建,高十步。”

【2】 “其墓亦然,一部分包金,一部分包银。”





124 班加剌州

班加剌者,向南之一州也。基督诞生后之1290年,马可·波罗阁下在大汗朝廷时,尚未征服,然已遣军在道。应知此州自有一种语言,居民是极恶偶像教徒,与印度(小印度)为近邻。其地颇多阉人,诸男爵所有之阉人,皆得之于此州。

其地有牛,身高如象,然不及象大。居民以肉、乳、米为粮,种植棉花,而棉之贸易颇盛,香料如莎草(souchet)、姜糖之属甚众。印度人来此求阉人及男女奴婢,诸奴婢盖在战争中得之于他州者也,售之印度商贾,转贩之于世界。

此地别无他事足述,所以离此而言别一名称交趾国(Canggu)之州。

“班加剌州在其南(缅南),近印度(小印度)边境,大汗征服其地,适在马可·波罗在朝之时。此国及其国王并强盛,如上所述,故久攻始臣服之。其州自有一种特别语言。”

“其人崇拜偶像,中有教师,教授魔术及偶像教仪式,其说通行于国内诸藩主中。”

“有不少印度人来此购买土产及阉人之为奴者,其数甚众。盖此辈为战争俘虏,获之即阉割,遂成阉人,顾诸藩主或男爵皆欲得阉人看管妇女,故商人来此购买,贩售他方,而获大利。”

“此州广三十日程,东尽处,抵一别州,名曰交趾国。”(剌木学本第二卷第四十五章)





125 交趾国州

交趾国(Cangigu)是东向日出处之一州,有国王,居民是偶像教徒,自有其语言,臣属大汗,每年入贡。其国王贪淫,致有妻三百人,如见国内有美妇,即娶以为妻。

此州有金甚饶,亦有香料甚众,然其地距海远广土产价值甚贱,产象多,亦有其他数种野兽及猎物不少。居民以肉、乳、米为粮,有酒,用米及香料酿之,味甚佳。其人多用针剌身,作狮、龙、鸟及其他各物形,文身以后,其色永远不灭。此种文身之事,或在面颈胸上为之,或在臂手上为之,或在腹上为之,或在全身上为之,以此为美,剌愈多者,其美更甚。

兹置此州不言,请言其东向日出处一名阿木(Amu)之州。





126 阿木州

阿木是东向日出处之一州,其民是偶像教徒,臣属大汗,以畜牧耕种为活,自有其语言。妇女腿、臂戴金银圈,价甚贵,男子亦然,其价较女子所戴者更贵。产马不少,多售之印度人而为一种极盛之贸易。其地有良土地,好牧场,故牛及水牛亦甚多,凡生活必需之物,悉皆丰饶。

则应知此阿木国后为交趾国,相距十五日程,交趾国后有班加剌国,相距三十日程。

今从此阿木州发足,东向日出处行,八日至一别州。





127 秃落蛮州

秃落蛮(Tholomau)是东向之一州,居民是偶像教徒,自有一种语言,臣属大汗。其人形色虽褐色而不白皙,然甚美,善战之士也。有环墙之城村甚众,并有高山天险。

人死焚尸,用小匣盛其余骸,携之至高山山腹大洞中悬之,俾人兽不能侵犯。

此地有金甚饶,然使用海贝,如前所述。上述诸州若班加剌、交趾国、阿木等州,亦习用海贝、黄金。其地商人甚富,而为大宗贸易。居民以肉、乳、米为粮,用米及最好香料酿酒饮之。

此外无足言者,兹置此州不言,请言东方别一名称叙州(Ciugui)之州。





128 叙州

叙州是东向之一州,自秃落蛮地发足,沿一河骑行十二日,沿途见有环墙之城村甚众,然无他事足供特别记录。沿河行此十二日毕,抵一城,名风古勒(Fungulo),城甚大而名贵,居民是偶像教徒,臣属大汗。恃商工为生,用某种树皮织布,甚丽,夏季衣之。彼等善战,而用纸币。自是以后,吾人遂在使用大汗纸币之地矣。

其地多虎,无人敢夜宿屋外,纵在夜间航行此河之上,若不远离河岸,诸虎即至舟中,搏人而食。此州之人若无一种良助,将无人敢行于道,盖虎数甚多,其躯大而性猛也。

幸而此地有一种犬,身大而猛,若两犬同行,其勇可拒猛虎,所以行人常携犬二头与俱,犬若见虎,即奋勇往搏,虎返击,犬亦善避,不为虎伤,常随虎吠,龅虎尾、虎腿或其他可能龀及之处。虎若无所作为者,然有时怒而搏犬,得则杀之。然犬颇知自防,最后虎闻犬吠逃走向一林中,倚一树下,俾犬不能龀其后。行人见虎逃,即引弓射虎(盖其人善射),虎贯矢而死,行人取虎之法如此。

其地产丝及其他商品甚众,赖有此河,运赴上下游各地。

已而沿此河骑行十二日,沿途见有城市甚众。居民是偶像教徒,臣属大汗,使用纸币,而业工商,其间颇有战士。骑行此十二日毕,抵于本书业已著录之成都府城。

自成都府城起行,骑行七十日,经行业已经过之诸州郡城村。七十日后,抵于前已著录之涿州。

再从涿州起行,复行四日,经过环墙之城村不少。居民商工茂盛,崇拜偶像,使用大汗纸币。行此四日毕,抵哈寒府(Cacanfu)城,城在南方,属契丹地域,后章言之。





129 哈寒府城

哈寒府是一贵城,居民是偶像教徒,人死焚其尸,使用纸币,恃工商为生,饶有丝,以织金锦丝罗,其额甚巨。此城领治一广大之地,所辖环墙之城村甚众。

兹从此城发足,南向骑行三日,抵一城,名强格路(Cianglu),后此述之。

此哈寒府若不证以剌木学本增订之文,颇难考订其方位,剌木学第二卷第四十三章之文曰:

“距涿州四日程,有巴章府(Pazanfu),位置于南方(涿州南),属契丹地域,还(向南)时经行此地域之别一部分,则见此城。此城居民崇拜偶像,人死焚其尸,城内尚有若干基督教徒,置有教堂一所。有一大河流经此城,转运不少商货至于汗八里城,盖有不少运河沟渠通都城也。”

波罗在前章引导读者复回涿州南两道分道处(见105章),第一道通西方及西南方诸省,业已叙述于前,兹言第二道,即东南通蛮子地域或江南之道,然其所言之方向,并不严格,未可以为准也。自是以后,波罗经行大平原中,此处所言四日不误,盖三十年前乘骡车由涿州赴正定者,即需此时间也。





130 强格路城

强格路(Cianglu)亦是向南之一大城,隶属大汗,而在契丹地域之中,使用纸币,崇拜偶像,人死焚其尸。应知此城制盐甚多,其法如下:

取一种极咸之土,聚之为丘,泼水于上,俾浸至底,然后取此出土之水,置于大铁锅中煮之,煮后俟其冷,结而成盐,粒细而色白,运贩于附近诸州,因获大利。

此外别无足述,前行又五日,抵一州,名强格里(Ciangli),后此述之。





131 强格里城

强格里是契丹向南之一城,隶属大汗,居民是偶像教徒,使用纸币。此城附近有一宽大之河,其运赴上下流之商货,有丝及香料不少,并有其他物产及贵重货品甚多。

兹从强格里城发足,请言南向距离六日程之别一城,其名曰中定府(Cundinfu)。

剌木学本第二卷第五十一章之异文云:“强格里距强格路五日程,沿途见有环墙之城村甚众,皆隶属大汗。其中商业茂盛,为大汗征收赋税,其额甚居。此强格里城中央有一宽而深之河流经过,河上运输有丝、香料及其他巨价货物不少。”





132 中定府城

自强格里城发足,向南骑行五日,沿途在在皆见有不少环墙之城村,外颇美观,内甚繁盛。居民是偶像教徒,人死而焚其尸,臣属大汗,使用纸币,执商工业,适于生活之百物悉皆丰饶。然沿途别无殊异之事足述,故下此即言中定府(Cundinfu)城。

应知中定府是一极大城市,昔日曾为国都,大汗曾用兵力征服。此城为此地一带最大之城,有商人无数经营大规模之商业,产丝之饶竟至不可思议。此外有园林,美丽堪娱心目,满园大果。应知此中定府城所辖巨富城市十有一所,商业茂盛,产丝过度而获利甚巨。

基督诞生后1273年时,大汗曾命其男爵一人,名李璮将军(Liytansangon),率军约八万骑,戍守此城及此州境。此将守境无几时,遂谋叛,并劝此州绅耆共叛大汗。于是彼等共推此李璮为主,而举叛旗。大汗闻讯,遣其男爵二人,一名阿术(Eguil),一名茫家台(Mangatay),率骑兵十万及步兵甚众,往讨。惟此次叛事极为严重,盖李璮与此州及附近从叛之人,数逾十万骑,且有步兵甚众也。虽然如是,李嬗与其党大败,讨叛之二男爵大胜。大汗闻之甚欢,命将诸谋叛首领悉加诛戮,其余胁从者悉加原宥。此二男爵遂将此次乱事之诸重要首领并处极刑,位置低微者悉皆赦免。自是以后,彼等遂忠于其主。

兹既述此乱事毕,请言更南之别一地,其名曰新州马头(Singuy-Matu)。





133 新州马头

离中定府后,南向骑行三日,沿途见有环墙之城村甚众,皆贵丽,工商业颇盛,有种种猎物,百物悉皆丰饶。

骑行此三日毕,抵一贵城名称新州马头,颇富丽,工商茂盛。居民是偶像教徒,为大汗臣民,使用纸币。有一河流,彼等因获大利,兹请言其故。

此河来自南方,流至此新州马头城,城民析此河流为二,半东流,半西流,使其一注蛮子之地,一注契丹之地。此城船舶之众,未闻未见者,绝不信其有之,此种船舶运载货物往契丹、蛮子之地,运载之多,竟至不可思议,及其归也,载货而来,由是此二河流来往货物之众可以惊人。

兹请接言更南之一州,其名曰临州(Linguy)。





134 临州城

从新州马头发足,南向骑行八日,沿途所经诸地,在在皆见有环墙之城村甚众,皆大而富丽,工商茂盛,人死焚其尸,臣属大汗,使用纸币行此八日毕,则见临州城,州名与城名同,盖国之都也。是为一富贵城,居民是善战之士,颇务工商,有带羽毛之猎物甚饶,凡适于生活之物,悉皆丰富。其城位置于上述之河上,河中有船舶甚众,船身大于前章所著录者,所载贵重货物甚多。

兹置此州此城不言,请言其他新事。





135 邳州城

离此临州城后,南向骑行三日,沿途皆见有环墙之城村,并富丽,尚属契丹境。居民是偶像教徒,人死焚其尸,臣属大汗,使用纸币,不用其他货币,有世界最良之鸟兽以供猎捕,凡适于人生之百物皆饶。

行此三日毕,抵邳州(Piguy),城大而富贵,工商业颇茂盛,产丝甚饶。此城在蛮子大州入境处,在此城见有商人甚众,运输其货物往蛮子境内及其他数个城市聚落;此城为大汗征收赋税,其额甚巨。

此外无足述者,故离此而去,接言更南之别一城,其名曰西州(Siguy)。





136 西州城

离邳州城后,向南骑行二日,经行美丽丰饶之地,其中颇有带羽毛之猎物。行此二日毕,遂抵西州城。城大而华富,营工商业,居民是偶像教徒,人死焚其尸,有纸币,而臣属大汗。此地一带有极广之田亩与美丽之平原,产小麦及其他谷类颇丰饶。惟此外别无他事足述,所以离此而言前途诸地。

离此西州城后,南向骑行三日,在在见有美地、美村、美聚落、美农舍,与夫垦植之田亩。其地饶有野味与小麦,并其他谷类。居民是偶像教徒,而臣属大汗。

行此三日毕,抵哈喇木连(Caramoran)大河,来自长老约翰之地。是为一极大河流,宽逾一哩,水甚深,大舟可航行于其上。水中有大鱼无数,河上有属于大汗之船舶,逾一万五千艘,盖于必要时运输军队赴印度海诸岛者也。缘此地距海仅有一日程,每舟平均足容水手二十人,可载马十五匹暨其骑者,与夫食粮、军械、甲胄。

此河两岸各有一城,此岸有一小城,彼岸亦有一城,隔岸相对。小城名海州(Cai-guy),对岸大城名淮安州(Coyganguy)。渡此河后,遂入蛮子大州境内。 【1】 兹请叙述大汗侵略此蛮子大州之事于后。

注释

【1】 剌木学本第二卷第五十四章增订之文云:“君等切勿以为吾人曾将契丹全境完全做有系统之说明,业已著录者实不及(应说明者)二十分之一。马可·波罗君经行此境之时,仅著录其沿途所见诸城,而置(此种种道上)道外及距离中之他城未言,盖若完全记录,势必成为一种冗长无味之工作也。”





137 大汗之侵略蛮子地域

蛮子(Mangi)大州





【1】 有一国王,名称法黑福儿(Faghfollr)甚强大,广有财货、人民、土地,世界君主除大汗外无有及之者。惟此国之人非战士,仅知沉湎于女色之中,而其国王尤甚,其所顾及者,惟诸妇女及赈恤其贫民而已。全境之中无马,其民未习战争武器,亦不谙兵术。此蛮子地域是一防守坚固之地,盖所有城市皆以水环之,水深,而宽有兰矢之远,仅有桥可通,脱其民为战士,将永不至于陷落,然其人非战士,遂致其地为人所得。

基督诞生后1268年时, 【2】 现今在位之大汗决定征服此国,命其男爵一人名伯颜丞相(Bayan Chincsan)者奉命前往,伯颜丞相,犹言百眼之伯颜也。先是,蛮子国王卜其国运,知其国只能亡于一百眼人之手,其心遂安。盖世上绝无百眼之人,缘其不知此人之名,因而自误。

此伯颜率领大汗之步骑甚众,挈船舶无数,运载步骑进至蛮子境中。全军行抵蛮子地界之时,即吾人现在所止之淮安州(Coyganguy),谕居民降,居民拒不纳款,伯颜弃之而去,进至一城,亦拒不降,又弃而去,率军前进,其为此者,盖知大汗别遣有大军在后也。

由是经五城,五城不战不降,皆未攻取,至第六城,始以兵攻陷之,已而复取一城,已而取一第三城,又取一第四城,陆续攻取城市十有二所。攻取诸城以后,进至国之都城,名曰行在(Quinsay),国王及其王后所居之处也。

国王见伯颜率如许大军至,既未习见此事,甚惧,遂率领其不少臣民,登千舟,逃入印度海洋诸岛之中,仅留王后镇守。王后以胜败事及敌军将领名询之星者,始知敌将即百眼之伯颜,遂知全国必亡于此人,于是举其全国一切城堡降于伯颜,不复防卫。是为一种最大侵略,盖世界诸国无与此国相侔,国王财货之众,竟至不可思议。兹请述其举动如下:

其国诸州小民之不能养其婴儿者,产后即弃,国王尽收养之。记录各儿出生时之十二生肖以及日曜,旋在数处命人乳哺之。如有富人无子者,请求国王赐给孤儿,其数惟意所欲。迨诸儿长大成人,国王为之婚配,赐资俾其存活,由是每年所养男女有二万人。

国王尚有别事足以著录者,当其骑而出,经行城市时,若见某家房舍过小,辄询其故,如答者谓物主过贫,无资使房屋高大,国王立出资,命将其屋扩大而美饰之,俾与他屋相等。设若房屋属于富人,则命其立时增高。职是之故,其蛮子都城之中,凡有房屋悉皆壮丽。别有巨大宫殿邸舍无数,尚未计焉。

执役于国王所者,男女仆役逾千人,衣饰皆富丽。国王治国至公平,境内不见有人为恶,城中安宁,夜不闭户,房屋及层楼满陈宝贵商货于其中,而不虞其有失。此国人之大富与大善,诚有未可言宣者也。

兹既述国王及其国毕,请言王后。王后至大汗所,大汗礼待之,然其夫国王则永不离去海岛,而殁于其中。兹置国王、王后不言,请回言蛮子大州及其风习,以续前记,按次述其端末,首言淮安州城,从此继述蛮子地域侵略之事。

剌木学本第二卷第五十五章增订之文如下:

注释

【1】 “蛮子地域为东方全境最开化而最富足之地,1269年顷,隶于一君主名范福儿(Fanfur)者,此范福儿为近百年所未见之富强君王,然其人平和而好善,自信以为世界之君主无有加害于彼者,缘其爱民之切,而有极大河流保护其国,遂不事军备,亦不鼓励其人民注意及此。”

【2】 “鞑靼君主大汗之性质,与国王范福儿迥乎不同,只知好战,侵略国土,崇尚武功,既得不少州郡国土以后,又决定侵略蛮子之地征集步骑甚众,组成一强大军队,命一名称丞相伯颜(Chinsam baian)者统之。丞相伯颜,此言百眼。”





138 淮安州城

淮安州(Coyganguy)是一甚大城市,在蛮子地界入境之处,居民是偶像教徒,焚死者之尸骸,臣属大汗。其城有船舶甚众,并在黄色大河之上,前已言之也。此城为府治所在,故有货物甚众,辐辏于此。缘此城位置此河之上,有不少城市运货来此,由此运往不少城市,惟意所欲。应知此城制盐甚多,供给其他四十城市之用,由是大汗收入之额甚巨。

兹述此城毕,请言别一名称宝应(Pauchin)之城。





139 宝应县城

离淮安州后,东南向沿堤骑行一日。此堤用美石建筑,在蛮子地界入境之处。此堤两岸皆水,故人其境只有此道可通。





【1】 行此一日毕,则抵宝应美城。居民是偶像教徒,人死焚其尸,臣属大汗,其货币为纸币,恃商工为活,有丝甚饶,用织金锦丝绢,种类多而且美,凡生活必需之物皆甚丰饶。

此外无足述者,请言别一名曰高邮(Cayu)之城。

注释

【1】 剌木学本第二卷第五十七章之异文云:“堤外两面湖泽甚广,水深可以行舟,除此堤外,别无他道可通其地,除非用舟船,如大汗统将率其全军进航之法也。”





140 高邮城

离宝应城东南骑行又一日,抵高邮城,城甚大。居民是偶像教徒,使用纸币,臣属大汗,恃工商为活,凡生活必需之物悉皆丰饶。产鱼过度,野味中之鸟兽亦夥。威尼斯城银钱(gros)一枚不难购得良雉三头。

兹从此地发足,继续前进,请言一别城,此城名称泰州(Tiguy)。





141 泰州城

从高邮城发足,向东南骑行一日,沿途在在皆见有村庄、农舍,与夫垦治之田亩,然后抵泰州,城不甚大,然百物皆丰。居民是偶像教徒,使用纸币,臣属大汗,恃商工为活,盖其地贸易繁盛,来自上述大河之船舶甚众,皆辐辏于此。应知其地左延向东方日出处,距海洋有三日程。自海至于此城,在在制盐甚夥,盖其地有最良之盐池也。

尚有一城,名称真州(Tingdy)。城甚大,出盐可供全州之食,大汗收入之巨,其数不可思议,非亲见者未能信也。居民是偶像教徒,使用纸币。

兹从此地发足,重返前述之泰州,请言别一名称扬州(Ianguy)之城。





142 扬州城

从泰州发足,向东南骑行一日,终抵扬州。城甚广大,所属二十七城,皆良城也。此扬州城颇强盛,大汗十二男爵之一人驻此城中,盖此城曾被选为十二行省治所之一也。应为君等言者,本书所言之马可·波罗阁下,曾奉大汗命,在此城治理亘三整年。居民是偶像教徒,使用纸币,恃工商为活。制造骑尉战士之武装甚多,盖在此城及其附近属地之中驻有君主之戍兵甚众也。

此外无足述者,后此请言西方之两大州,此两州亦在蛮子境内。兹请首述名称南京(Nanghin)之城。





143 南京城

南京是一大州,位置在西。居民是偶像教徒,使用纸币,臣属大汗,恃商工为活。有丝甚饶,以织极美金锦及种种绸绢。是为一富足之州,由是一切谷粮皆贱。境内有野味甚多,且有虎。有富裕之大商贾包办其所买卖商货之税额,君主获有收入甚巨。

此外无足述者,兹从此地发足,请言甚大之襄阳府(Saianfu)城。此城堪在本书著录,盖有关系此城之一大事必须叙述也。





144 襄阳府大城及其被城下炮机夺取之事

襄阳府是一极重要之大城,所辖富裕大城十有二所,并为一种繁盛工商业之中区。居民是偶像教徒,使用纸币,焚死者尸,臣属大汗。产丝多,而以制造美丽织物,亦有野味甚众。节而言之,凡一大城应有之物,此城皆饶有之。

现应知者,此城在蛮子地域降服以后,尚拒守者三年。大汗军队不断猛攻之,但只能围其一面,质言之,北面,盖其余三面皆有宽深之水环之,防守者赖以获得食粮及其他意欲之物。脱无下述之一事,余敢保其永远不能攻下。

大汗军队围攻此城三年而不能克,军中人颇愤怒。由是尼古剌波罗阁下,其弟玛窦·波罗阁下及尼古剌·波罗阁下之子马可·波罗阁下献议,谓能用一种器械可取此城,而迫其降。此种器械名曰茫贡诺(Mansonneau),形甚美,而甚可怖,发机投石于城中,石甚大,所击无不摧陷。

大汗及其左右诸男爵,与夫军中遣来报告此城不降之使臣,闻此建议,颇为惊异。盖此种地域中人,不知茫贡诺为何物,亦不识战机及投石机,而其军队向未习用此物,既未识之,亦从未见之,所以闻议甚喜。大汗乃命此二兄弟及马可阁下从速制造此机,大汗及其左右极愿亲睹之,因其为彼等从来未见之奇物也。

上述之三人立命人运来材木如其所欲之数,以供造机之用。彼等随从中有二人详悉一切制造之事,其一人是聂思脱里派之基督教徒,其一人是日耳曼之日耳曼人,亦一基督教徒也。于是此二人及上述之三人制造三机,皆甚壮丽。每机可发重逾三百磅之石,石飞甚远,同时可发六十石,彼此高射程度皆相若。诸机装置以后,大汗及其他观者皆甚欢欣,命彼等当面发射数石,发射之后,皆极惊赏其制作之巧。大汗立命运机至军中,以供围城之用。机至军中,装置以后,鞑靼未见此物一次,见之似甚惊奇。

此机装置以后,立即发石,每机各投一石于城中,发声甚巨,石落房屋之上,凡物悉被摧陷。此城中人从来未见未闻此物,见此大患,皆甚惊愕,互询其故,恐怖异常,因聚议,皆莫筹防御此大石之法。彼等信为一种巫术,情形窘迫,似只能束手待毙。聚议以后,皆主降附,遣使者往见主将,声明愿降附大汗,与州中其他诸城相同。大汗闻之甚喜,而许其降。于是此城遂下,待遇与其他诸城同。此皆尼古剌阁下、其弟玛窦阁下及其子马可阁下之功也。此功诚不为小,盖此城及此地在昔在今皆为良土,大汗可在其境中获得重大收入也。 【1】

兹既述赖有上述三人所造机械迫使此城降附之事毕,请言别一名曰新州(Singui)之城。

注释

【1】 颇节所用之主要本为G字本,其文简而不明,故吾人取472至473页注录C字本之文代之,此本与玉耳所选之地学会法文本颇相近。

剌木学本第二卷第六十二章之文虽较简略,然录之足供比对,兹录其文如下:

“尼古剌·波罗及玛窦·波罗所取之襄阳府城”

“襄阳府具有属于大城之一切优点,赖其形势坚固,虽在大汗侵略蛮子地域之后,尚拒守三年而不降附。其故在此,盖军队仅能近城之北面,其余三面皆有极大湖沼环之,保其粮道继续不断,而为围攻者势所不能及。大汗知之,极感不快,盖蛮子全境咸已降顺,只有此城固守不下也。时尼古剌、玛窦弟兄二人在朝,闻悉此事,立入谒,愿用西方之法,制造茫贡诺,可发重三百磅之石,足使围城中人死屋摧。”

“大汗闻此议甚喜,命其统率最良之铁匠、木匠执行。诸匠中有若干聂思脱里派之基督教徒,深谙工作。无何,诸匠依波罗弟兄之指导,制造茫贡诺三具,在大汗及全朝人之前试之,发射各重三百磅之石。”

“立将此种机械用舟载赴军中,及至,遂在襄阳城下装置,发第一石,坠势猛烈,一屋几尽摧毁。居民惊骇,有如雷从天降,乃决议投降。于是遣使者出城纳款,其归降条件与蛮子全境之归降条件同。”

“此役之奇捷,遂增波罗弟兄二人在大汗所及全朝之声望及信任。”





145 新州城

从襄阳城发足,向东南骑行十五哩,抵一城,名曰新州。城不甚大,然商业繁盛,舟船往来不绝。居民是偶像教徒,臣属大汗,使用纸币。并应知者,其城位在世界最大川流之上,其名曰江,宽有十哩,他处较狭,然其两端之长,逾百日行程。所以此城商业甚盛,盖世界各州之商货皆由此江往来,故甚富庶,而大汗赖之获有收入甚丰。

此江甚长,经过土地城市甚众,其运载之船舶货物财富,虽合基督教民之一切江流海洋运载之数,尚不逮焉。虽为一江,实类一海。马可·波罗阁下曾闻为大汗征收航税者言,每年溯江而上之船舶,至少有二十万艘,其循江而下者尚未计焉,可见其重要矣。沿此江流有大城四百,别有环以墙垣之城村不在数内,并有船舶停止。其船甚大,所载重量,核以吾人权量,每船足载一万一二千石(quintaux),其上可盖席篷。

此外无足述者,因是重行,请言一名瓜州(Caigui)之城。然有一事前此忘言,请追述之。应知上行之船舶,因江流甚急,须曳之而行,无缆则不能上。曳船之缆长三百步,用竹结之,其法如下:劈竹为长片,编结为缆,其长惟意所欲,如此编得之缆,较之用大麻编结者为坚。





146 镇江府城

镇江府(Chingianfu)是一蛮子城市,居民是偶像教徒,臣属大汗,使用纸币,恃商工为活。产丝多,以织数种金锦丝绢,所以见有富商大贾。野味及适于生活之百物皆饶。其地且有聂思脱里派基督教徒之礼拜堂两所,建于基督诞生后之1278年,兹请述其缘起。

是年耶稣诞生节,大汗任命其男爵一人名马薛里吉思(Mar-Sarghis)者,治理此城三年。其人是一聂思脱里派之基督教徒,当其在职三年中,建此两礼拜堂,存在至于今日,然在以前,此地无一礼拜堂也。

兹置此事不言,请先言一甚大之城,名曰镇巢军(Chingingui)。





147 镇巢军城

从镇江府城发足,东南向骑行三日,抵镇巢军,城甚大。居民是偶像教徒,使用纸币,臣属大汗,恃工商为活。丝及供猎捕之禽兽甚多,种种粮食皆饶,盖此地为一丰富之地也。

兹请言此城人所为一次恶行而受重惩之事 【1】 ,先是蛮子大州略定之时,军帅伯颜遣一队名称阿兰(Alains)之人往取此城。诸阿兰皆是基督教徒,取此城入据之,在城中见有美酒,饮之醉,酣睡如同猪豚,及夜,居民尽杀之,无能脱者。

伯颜闻其遣军被袭杀,别遣一将率一大军攻取此城,尽屠居民,无一免者,此城人民完全消灭之法如此。

兹置此事不言,请言别一名称苏州(Sugui)之城。

注释

【1】 剌木学本第二卷第六十六章之异文云:“居民甚贱恶,丞相伯颜平定蛮子地域之时,曾遣若干信奉基督教之阿兰,统率本军一队,往取此城。军至城下,未受抵抗,即入据之。此城有两城垣,诸阿兰既据外城,发现藏酒甚多,彼等颇饥疲,取酒饮之至醉。内城居民见其敌醉卧于地,乘隙尽屠之。丞相伯颜闻其军被屠,怒甚,遣他队往讨,取此城,尽屠居民,男女老少无一免者。”





148 苏州城

苏州是一颇名贵之大城,居民是偶像教徒,臣属大汗,恃商工为活。产丝甚饶,以织金锦及其他织物。其城甚大,周围有六十哩,人烟稠密,至不知其数。假若此城及蛮子境内之人皆是战士,将必尽略世界之余土,幸而非战士,仅为商贾与工于一切技艺之人。此城亦有文士;医师甚众。

此城有桥六千,皆用石建;桥甚高,其下可行船,甚至两船可以并行。此城附近山中饶有大黄,并有姜,其数之多,威尼斯钱(gros)一枚可购六十磅。此城统辖十六大城,并商业繁盛之良城也。此城名称苏州,法兰西语犹言“地”,而其邻近之一别城行在(Quinsay),则犹言“天”,因其繁华,故有是名。行在城后此言之。

兹从苏州发足,先至一城,名曰吴州(Vouguy),距苏州一日程,是一工商繁盛之富庶大城也。顾无他事足述,请离此而言别一名称吴兴(Vughin)之城。此吴兴尚为一大而富庶之城,居民是偶像教徒,臣属大汗,使用纸币,产丝及其他不少贵重货物甚饶,皆良商贾与良工匠也。

兹从此城发足,请言强安(Ciangan)城。应知此强安城甚大而富庶,居民是偶像教徒,臣属大汗,使用纸币,恃工商为活,织罗(taffetas)甚多,而种类不少。此外无足言者,请从此处发足前进,而言他城。兹请先言极名贵之行在城,蛮子之都会也。





149 (一)蛮子国都行在城

自强安城发足,骑行三日,经行一美丽地域,沿途见有环墙之城村甚众,由是抵极名贵之行在(Quinsay)城。行在云者,法兰西语犹言“天城”,前已言之也。既抵此处,请言其极灿烂华丽之状,盖其状实足言也,谓其为世界最富丽名贵之城,良非伪语。兹请续言此国王后致略地之伯颜书,请其将此书转呈大汗,俾悉此城大佳,请勿毁坏事。吾人今据此书之内容以及马可·波罗阁下之见闻述之。

书中首称此行在城甚大,周围广有百哩。内有一万二千石桥,桥甚高,一大舟可行其下。其桥之多,不足为异,盖此城完全建筑于水上,四围有水环之,因此遂建多桥以通往来。

书中并言此城有十二种职业,各业有一万二千户,每户至少有十人,中有若干户多至二十人、四十人不等。其人非尽主人,然亦有仆役不少,以供主人指使之用。诸人皆勤于作业,盖其地有不少城市,皆依此城供给也。

此书又言城中有商贾甚众,颇富足,贸易之巨,无人能言其数。应知此职业主人之为工场长者,与其妇女,皆不亲手操作,其起居清洁富丽,与诸国王无异。此国国王有命,本业只能由子承袭,不得因大利而执他业。

城中有一大湖,周围广有三十哩,沿湖有极美之宫殿,同壮丽之邸舍,并为城中贵人所有。亦有偶像教徒之庙宇甚多。湖之中央有二岛,各岛上有一壮丽宫室,形类帝宫。城中居民遇有大庆之事,则在此宫举行。中有银制器皿、乐器,举凡必要之物皆备,国王贮此以供人民之用。凡欲在此宫举行大庆者,皆任其为之。

在此城中并见有美丽邸舍不少,邸内有高大楼台,概用美石建造,城中有火灾时,移藏资财于其中,盖房屋用木建造,火灾时起也。

居民是偶像教徒,自经大汗经略以后,使用纸币。彼等食一切肉,基督教徒绝不食之狗肉及其他贱畜之肉亦食。自从大汗据有此城以后,于一万二千桥上,每桥命十人日夜看守,俾叛乱之事不致发生。此城有一山丘,丘上有一塔,塔上置一木板,每遇城中有火警或他警时,看守之人执棰击板,声大远处皆闻,人闻板声,即知城内有火警或乱事。

大汗在此城警戒甚严者,盖因其为蛮子地域都城,并因其殷富,而征收之商税甚巨,其额之巨,仅闻其说而未见其事者绝不信之。

城中街道皆以石铺地,蛮子地域之一切道路皆然,由是通行甚易,任往何处不致沾泥。蛮子地域多泥泞,设若道路不以石铺地,则步骑皆难跋涉,盖其地低而平,雨时颇多陷坑也。

尚应知者,此行在城中有浴所三千,水由诸泉供给,人民常乐浴其中,有时足容百余人同浴而有余。

海洋距此有二十五哩,在一名澉浦(Ganfu)城之附近。其地有船舶甚众,运载种种商货往来印度及其他外国,因是此城愈增价值。有一大川自此行在城流至此海港而人海,由是船舶往来,随意载货,此川流所过之地有城市不少。

大汗区分蛮子地域为九部,而为九国,每国遣一国王治之。诸国王皆臣属大汗,每年各以国中会计上之都城计院。行在城驻在之国王,所辖富庶大城一百四十。应知此广大蛮子地域共有富庶大城一千二百余所,其环墙之村庄及城市无数,尚未计焉。此一千二百城,大汗各置戍兵一队,最少者额有千人,有至一万、二万、三万人者,由是其人之众不可胜计。此种戍守之人皆契丹州人,善战之士也,然不人尽有马,步卒甚众,皆隶大汗军。

凡关涉此城之事,悉具广大规模。大汗每年征收种种赋税之巨,笔难尽述。其中财富之广,而大汗获利之大,闻此说而未见此事者,必不信其有之。

此地之人有下述之风习,若有胎儿产生,即志其出生之日时生肖,由是每人知其生辰。如有一人欲旅行时,则往询星者,告以生辰,卜其是否利于出行,星者偶若答以不宜,则罢其行,待至适宜之日。人信星者之说甚笃,缘星者精于其术,常作实言也。

人死焚其尸,设有死者,其亲友服大丧,衣麻,携数种乐器行于尸后,在偶像前作丧歌,及至焚尸之所,取纸制之马匹、甲胄、金锦等物并尸共焚之。据称死者在彼世获有诸物,所作之乐,及对偶像所唱之歌,死者在彼世亦得闻之,而偶像且往贺之也。

此城尚有出走的蛮子国王之宫殿,是为世界最大之宫,周围广有十哩,环以具有雉堞之高墙,内有世界最美丽而最堪娱乐之园囿,世界良果充满其中,并有喷泉及湖沼,湖中充满鱼类。中央有最壮丽之宫室,计有大而美之殿二十所,其中最大者,多人可以会食。全饰以金,其天花板及四壁,除金色外无他色,灿烂华丽,至堪娱目。

并应知者,此宫有房室千所,皆甚壮丽,皆饰以金及种种颜色。此城有大街一百六十条,每街有房屋一万,计共有房屋一百六十万所,壮丽宫室夹杂其中。城中仅有聂思脱里派基督教徒之礼拜堂一所。

尚有一事须言及者,此城市民及其他一切居民皆书其名、其妻名、其子女名、其奴婢名以及居住家内诸人之名于门上,牲畜之数亦开列焉。此家若有一人死则除其名,若有一儿生则增其名,由是城中人数,大汗皆得知之。蛮子、契丹两地皆如是也。

一切外国商贾之居留此种地域者,亦书其名及其别号,与夫人居之月日,暨离去之时期,大汗由是获知其境内来往之人数,此诚谨慎贤明之政也。

兹请言大汗在此城及其辖境所征收之赋税于后。





149 (二)补述行在(出剌木学本)

(一)离吴州后,连续骑行三日,沿途见有环墙城村,富庶聚落……行在城所供给之快乐,世界诸城无有及之者,人处其中,自信为置身天堂。马可·波罗阁下数至此城,曾留心其城之事,以其见闻,笔之于书,后此诸行,特为其节略而已。

据共同之说,此城周围有百哩,道路河渠颇宽展,此外有衢,列市其中,赴市之人甚众。

城之位置,一面有一甘水湖,水极澄清,一面有一甚大河流。河流之水流入不少河渠,河渠大小不一,流经城内诸坊,排除一切污秽,然后注入湖中,其水然后流向海洋,由是空气甚洁。赖此河渠与夫街道,行人可以通行城中各地。街渠宽广,车船甚易往来,运载居民必需之食粮。人谓城中有大小桥梁一万二千座,然建于大渠而正对大道之桥拱甚高,船舶航行其下,可以不必下桅,而车马仍可经行桥上,盖其坡度适宜也。就事实言,如果桥梁不多,势难往来各处。

(二)城与湖相对,围城有渠,长有四十哩,甚宽,乃由昔日此地诸国王开掘而成,以备容纳诸河流漫溢之水者也,平时则导上述河流之水于其中。此渠且供防守此城之用,掘渠之土,聚而成堤,围绕此城。

城中有大市十所,沿街小市无数,尚未计焉。大市方广每面各有半哩,大道通过其间。道宽四十步,自城此端达于彼端,经过桥梁甚众。此道每四哩必有大市一所,每市周围二哩,如上所述。市后与此大道并行,有一宽渠,邻市渠岸有石建大厦,乃印度等国商人挈其行李商货顿止之所,利其近市也。

每星期有三日为市集之日,有四五万人挈消费之百货来此贸易。由是种种食物甚丰,野味如獐鹿、花鹿、野兔、家兔,禽类如鹧鸪、野鸡、家鸡之属甚众,鸭、鹅之多,尤不可胜计,平时养之于湖上,其价甚贱,威尼斯城银钱一枚,可购鹅一对、鸭两对。复有屠场,屠宰大畜,如小牛、大牛、山羊之属,其肉乃供富人大官之食,至若下民,则食种种不洁之肉,毫无厌恶。

此种市场常有种种菜蔬果实,就中有大梨,每颗重至十磅,肉白如面,芬香可口。按季有黄桃、白桃,味皆甚佳。然此地不产葡萄,亦无葡萄酒,由他国输入干葡萄及葡萄酒,但土人习饮米酒,不喜饮葡萄酒。

每日从河之下流二十五哩之海洋,运来鱼类甚众,而湖中所产亦丰,时时皆见有渔人在湖中取鱼。湖鱼各种皆有,视季候而异,赖有城中排除之污秽,鱼甚丰肥。有见市中积鱼之多者,必以为难以脱售,其实只须数小时,鱼市即空,盖城人每餐皆食鱼肉也。

上述之十市场,周围建有高屋。屋之下层则为商店,售卖种种货物,其中亦有香料、首饰、珠宝。有若干商店仅售香味米酒,不断酿造,其价甚贱。

包围市场之街道甚多,中有若干街道置有冷水浴场不少,场中有男女仆役辅助男女浴人沐浴。其人幼时不分季候即习于冷水浴,据云,此事极适卫生。浴场之中亦有热水浴,以备外国人未习冷水浴者之用。土人每日早起非浴后不进食。

(三)其他街道,娼妓居焉。其数之多,未敢言也,不但在市场附近此辈例居之处见之,全城之中皆有。衣饰灿丽,香气逼人,仆妇甚众,房舍什物华美。此辈工于惑人,言辞应对皆适人意,外国人一旦涉足其所,即为所迷,所以归去以后,辄谓曾至天堂之城行在,极愿重返其地。

其他街道居有医士、星者,亦有工于写读之人,与夫其他营业之人,不可胜计,居所皆在市场周围。每市场对面有两大官署,乃副王任命之法官判断商人与本坊其他居民狱讼之所。此种法官每日必须监察附近:看守桥梁之人是否尽职,否则惩之。

上述自城此端达彼端之大道,两旁皆有房屋宫殿,与夫园囿。然在道旁,则为匠人之房屋。道上往来行人之众,无人能信有如许食粮可供;彼等之食,除非在市集之日,见买卖之人充满于中,车船运货络绎不绝,运来之货无不售者,始能信也。

兹取本城所食之胡椒以例之,由是可知平常消耗其他物品若肉、酒、香料之属之众。马可·波罗阁下曾闻大汗关吏言,行在城每日所食胡椒四十四担,而每担合二百二十三磅也……

(四)……居人面白形美,男妇皆然,多衣丝绸,盖行在全境产丝甚饶,而商贾由他州输入之数尤难胜计……

……此种商店富裕而重要之店主,皆不亲手操作,反貌若庄严,敦好礼仪,其妇女妻室亦然。妇女皆丽,育于婉娩柔顺之中,衣丝绸而戴珠宝,其价未能估计。其旧王虽命居民各人子承父业,第若致富以后,可以不必亲手操作,惟须雇用工人,执行祖业而已。其家装饰富丽,用巨资设备饰品、图画、古物,观之洵足乐也。

行在城之居民举止安静,盖其教育及其国王榜样使之如此。不知执武器,家中亦不贮藏有之。诸家之间,从无争论失和之事发生,纵在贸易制造之中,亦皆公平正直。男与男间,女与女间,亲切之极,致使同街居民俨与一家之人无异。

互相亲切之甚,致对于彼等妇女,毫无忌妒猜疑之心。待遇妇女亦甚尊敬,其对于已婚妇女出无耻之言者,则视同匪人。彼等待遇来共贸易之外人,亦甚亲切,款之于家,待遇周到,辅助劝导,尽其所能。反之,彼等对于士卒,以及大汗之戍兵,悉皆厌恶,盖以其国王及本地长官之败亡,皆缘此辈有以致之也……

(五)……湖中有两岛,各有宫一所,宫内有分建之殿阁甚众。脱有人欲举行婚礼,或设大宴会者,即赴一宫举行。其中器皿、布帛皆备,是皆城民公置,贮之宫中,以供公用者也。有时在此可见人众百群,或设宴会,或行婚礼,各在分建殿阁之中举行,秩序严整,各不相妨。

此外湖上有大小船只甚众,以供游乐。每舟容十人、十五人或二十人以上。舟长十五至二十步,底平宽,常保持其位置平稳。凡欲携其亲友游乐者,只须选择一舟可矣,舟中饶有桌椅及应接必需之一切器皿。舟顶用平板构成,操舟者在其上执篙撑舟湖底以行舟(盖湖深不过两步),拟赴何处,随意所欲。舟顶以下,与夫四壁,悬挂各色画图。两旁有窗可随意启闭,由是舟中席上之人,可观四面种种风景。地上之赏心乐事,诚无有过于此游湖之事者也,盖在舟中可瞩城中全景,无数宫殿、庙观、园囿、树木,一览无余。湖中并见其他游船,载游人往来,盖城民操作既毕,常携其妇女或娼妓乘舟游湖,或乘车游城。其车游亦有足言者,城民亦以此为游乐之举,与游湖同也。

首应知者,行在一切道路皆铺砖石,蛮子州中一切道途皆然,任赴何地,泥土不致沾足。惟大汗之邮使不能驰于铺石道上,只能在其旁土道之上奔驰。

上言通行全城之大道,两旁铺有砖石,各宽十步,中道则铺细沙,下有阴沟宣泄雨水,流于诸渠中,所以中道永远干燥。在此大道之上,常见长车往来,车有棚垫,足容六人。游城之男女日租此车以供游乐之用,是以时时见车无数,载诸城民行于中道,驰向园囿,然后由看守园囿之人招待至树下休息,城民偕其妇女如是游乐终日,及夜,始乘原车返家。

(六)行在居民风习,儿童诞生,其亲立即记录其生庚日时,然后由星者笔录其生肖。儿童既长,经营商业,或出外旅行,或举行婚姻,有持此纸向星者卜其吉凶。有时所卜甚准,人颇笃信之。此种星者要为巫师,一切公共市场中为数甚众。未经星者预卜绝不举行婚姻。

尚有别一风习,富贵人死,一切亲属男女,皆衣粗服,随遗体赴焚尸之所。行时作乐,高声祷告偶像,及至,掷不少纸绘之仆婢、马驼、金银、布帛于火焚之。彼等自信以为用此方法,死者在彼世可获人畜、金银、绸绢。焚尸既毕,复作乐,诸人皆唱言,死者灵魂将受偶像接待,重生彼世。

(七)此城每一街市建立石塔,遇有火灾,居民可藏物于其中(盖房屋多用木料建造,火灾常起)。此外大汗有命,诸桥之上,泰半遣人日夜看守。每桥十人,分为两班,夜间五人,日间五人,轮流看守。每桥置木梆一具,大锣一具,及日夜识时之沙漏一具。夜中第一时过,看守者中之一人击梆、锣一下,邻近诸户知为一时,二时以后则击二下,由是每逾一时多击一下,看守者终夜不眠。日出之后,重由第一时击起,每时加增,与夜间同。

有一部分看守之人巡行街市,视禁时以后是否尚有灯火,如有某家,灯火未熄,则留符记于门,翌晨传屋主于法官所讯之,若无词可借,则处罚。若在夜间禁时以后有人行街中,则加拘捕,翌晨送至法庭。日间若在街市见有残废穷苦不能工作之人,送至养济院中收容;此种养济院甚多,旧日国王所立,资产甚巨。其人疾愈以后,应使之有事可作。

若见一家发火,则击梆警告,由是其他诸桥之守夜人奔赴火场救火,将商人及其他被害人之物,或藏之上述之石塔中,或运至湖岛。纵在此情况中,任何城民皆不能离家外出进至火场。只见运物之人及救火之人往来其间,救火者其数至少有一二千人。

此种看守之人,尚须防备城中居民叛乱之事。大汗常屯有步兵、骑兵无数于此城中及其附近,并遣忠诚可恃之大藩主来此镇守。盖其视此州极为重要,既为都会,而其财富为世界其他诸城所不及也。

又在同一目的中,每距一哩之地,建立不少土丘,每丘之上置一木架,悬一大响板,一人持板,一人以木棰击板,响声远处可闻。有看守人永在此处看守,遇有火警,则击板以警众,盖若火警报告不速,全城一半将成灰烬。又如前述叛乱之事,警板一响,附近诸桥之看守人立执兵奔赴……

(八)……君等切勿以为蛮子诸城此种戍兵皆是鞑靼,要以契丹人为最众,盖鞑靼为骑士,其屯驻之地要在土地干燥平坦可以驰骋之所,不能屯驻于饶有池泽诸城也。至在潮湿之地,则命契丹人及蛮子地方堪服军役之人前往戍守。每年大汗选其臣民之能执兵者编入军队,命为士卒。其在蛮子州中征集之人,不戍本城,应往戍守远距二十日程之地,戍期四五年,然后调还。此法并适用于契丹人及蛮子地域之人也。

从诸城征收之赋税,大部分入大汗之库藏,用以养给此种戍兵。设有某城叛乱,即抽调邻近诸城戍兵前往平服,盖叛乱时起,若从契丹州调兵平乱,需时二月也。职是之故,行在城中常置戍兵三万,其他诸城或置步兵,或置骑兵,至少亦有千人。

(九)今请言一华丽宫殿,国王范福儿(Fanfur)之居也。其诸先王围以高墙,周有十哩,内分三部,中部有一大门,由此而入,余二部在其两旁。(东西)见一平台,上有高大殿阁,其顶皆用金碧画柱承之。正殿正对大门,漆式相同,金柱承之,天花板亦饰以金,墙壁则绘前王事迹。

每年偶像庆日,国王范福儿例在此殿设大朝会,大宴重臣高官及行在城之富商。诸殿足容万人列席,朝会延十日或十二日。其盛况可惊,与宴者皆服金衣绸衣,上饰宝石无数,富丽无比。

此殿之后有墙,中辟一门,为内宫门。入门有一大庭,绕以回廊,国王及王后诸室即在其中,装饰华丽,天花板亦然。逾庭入一廊,宽六步,其长抵于湖畔。此廊两旁各有十院,皆长方形,有游廊,每院有五十室,园圃称是,此处皆国王宫嫔千人所居。国王有时偕王后携带宫嫔游行湖上,巡幸庙宇,所乘之舟,上覆丝盖。

墙内余二部,有小林,有水泉,有果园,有兽囿,畜獐鹿、花鹿、野兔、家兔。国王携诸宫嫔游此两部,有驾车者,有乘马者,男子不许擅入。

有时携犬猎取上述之兽,宫嫔驰逐既疲,则入小林,尽去衣服,游泳水中,国王观之甚乐,泳毕皆还宫院。有时国王息于林中树下,命诸宫嫔进食。由是日亲女色,不识武器为何物,怯懦至于亡国,土地悉为大汗所得,蒙耻忍辱,如前所述也(本书第137章)。

以上乃我在此城时所闻行在某富商之言。其人年甚老,曾事国王范福儿,熟悉其生平诸事。既已目睹宫廷之旧状,乃携我往游。今为大汗任命副王之驻所,前殿尚保存如故,然后宫则已颓废,仅余遗迹,林园之围墙亦倾圮,不复见有树木兽畜……

(十)……大汗使臣征收年赋检括户口之时,马可阁下适在行在城中,曾检阅户口有一百六十秃满(toman)。每户等于一家,每秃满等于一万,则城中共有一百六十万家矣。人数虽有如是之众,仅有聂思脱里派之礼拜堂一所……

蛮子州中贫民无力抚养儿女者,多以儿女售之富人,冀其养育之易,生活之丰。





150 大汗每年取诸行在及其辖境之巨额赋税

行在城及其辖境构成蛮子地方九部之一,兹请言大汗每年在此部中所征之巨额课税。第一为盐课,收入甚巨。每年收入总数合金八十秃满,每秃满值金色干(sequin)七万,则八十秃满共合金色;干五百六十万,每金色干值一佛罗铃(florin)有奇,其合银之巨可知也。

述盐课毕,请言其他物品货物之课。应知此城及其辖境制糖甚多,蛮子地方其他八部,亦有制者,世界其他诸地制糖总额不及蛮子地方制糖之多,人言且不及其半。所纳糖课值百取三,对于其他商货以及一切制品亦然。木炭甚多,产丝奇饶,此种出产之课,值百取十。此种收入,合计之多,竟使人不能信此蛮子第九部之地,每年纳课如是之巨。

叙述此事之马可·波罗阁下,曾奉大汗命审察此蛮子第九部地之收入,除上述之盐课总额不计外,共达金二百一十秃满,值金色干一千四百七十万,收入之巨,向所未闻。

大汗在此第九部地所征课额,既如是之巨,其他八部收入之多,从可知也。然此部实为最大而获利最多之一部,大汗取之既多,故爱此地甚切,防守甚密,而以维持居民安宁。

兹从此地发足请言他城。 【1】

注释

【1】 剌木学本第二卷第六十九章,所记微异,兹录其文如下,以资参稽:

“大汗之收入”

“兹请略言大汗在此行在城及其所辖诸城所取之课税。此城同构成蛮子地方第九部之其他诸城,别言之,蛮子境内九国之一国,所纳之课,首为盐课,其额最巨,年入之额值金八十秃满,每秃满值八万金色干,每金色干值一金佛维罗铃有奇,则共值六百四十万色干矣。其故乃在此州位在海洋沿岸,由是饶有池泽,夏季海水蒸发,所取之盐足供蛮子其他五国之食。”

“其地制糖甚多,其课值百取三点三三,与其他诸物同,又如米酒及上述共有一万二千店肆之十二业之出产亦然。商人或输入货物至此城,或遵陆输出货物至他州,抑循海输出货物至外国者,亦纳课百分之三点三三。然远海之地如印度等国输入之货物,应纳课百分之十。一切土产若牲畜、果实、丝绸之类,亦纳什一之税于副王。”

“马可阁下曾审察其额,除上述之盐课不计外,君主年入共有二百一十秃满,每秃满值八万金色干,则共有一千六百八十万金色干矣。”





151 塔皮州城

自行在发足骑行一日,抵塔皮州(Tacpiguy),城甚壮丽富庶而隶属行在。居民臣属大汗而使用纸币。彼等是偶像教徒,而焚其死者尸,其法如前所述。恃工商及种种职业为活。凡生活必需之物,悉皆丰饶而价贱。

此外无足言者,所以前行言一别城,城名武州,距塔皮州有三日程。居民是偶像教徒,臣属大汗,使用纸币而隶属行在。彼等恃工商为活。

此外无足言者,因是仍前行。

距此两日程,有衢州(Giuguy)城甚壮丽。居民使用纸币,产丝多,而恃工商为活,食粮丰饶。此城隶属行在,有竹最粗长,为蛮子地方最,粗四掌,长十五尺。此外无足言者,因是仍前行。

自衢州发足,骑行四日,经行一最美之地,中有环墙之城村甚众,然后抵于强山,城甚壮丽,位在一丘陵上,将流赴海洋之河流,析而为二。此城亦在行在辖境之中。蛮子全境不见绵羊,惟多山羊与牛。居民是偶像教徒,而恃商业及种种技艺为活。臣属大汗而使用纸币。

此外无足言者,因是仍前行。

离强山后,骑行三日,抵信州城。居民是偶像教徒,臣属大汗而使用纸币。彼等恃工商为活。此城壮丽,乃此方向中行在所辖之末一城。至若吾人现在行抵之福州(Fugui)则为蛮子九部中之一部,与行在同也。

此外无足言者,请仍前行。





152 福州国

从行在国最后之信州(Cinguy)城发足,则入福州国境,





【1】 由是骑行六日,经行美丽城村,其间食粮及带毛带羽之野味甚饶。亦见有虎不少,虎躯大而甚强。产姜及高良姜过度,威尼斯城银钱一枚,可购好姜四磅。并见有一种果,形类洎夫蓝(safran),用以为食。应知其地居民,凡肉皆食,甚至人肉亦极愿食之,惟须其非病死者之肉耳。所以此辈寻觅被害者之尸而食其肉,颇以为美。

其赴战者,有一种风习,请为君等述之。此辈剃其额发,染以蓝色,如同剑刃。除队长外皆步行,手执矛,而为世界上最残忍之人。盖其辄寻人而杀,饮其血而食其肉。

兹置此不言,请言他事。上述之六日程行三日毕, 【2】 则见有城名格里府(Quelifu),城甚广大。居民臣属大汗,使用纸币,并是偶像教徒。城中有三石桥,世界最美之桥也。每桥长一哩,宽二十尺,皆用大理石建造,有柱甚美丽。

居民恃工商为活。产丝多,而有姜及高良姜甚饶。其妇女甚美。有一异事,足供叙录,其地母鸡无羽而有毛,与猫皮同。鸡色黑,产卵,与吾国之卵无异,宜于食。

此外无足言者,请言他事。 【3】 再行三日又十五哩,抵一别城,名称武干(Vuguen),制糖甚多。居民是偶像教徒而使用纸币。

此外无足言者,此后请言福州之名贵。

剌木学本之异文如下:

注释

【1】 “离行在国最后一城名称吉匝(Gieza-Cinguy)之城后,入崇迦(Concha)国境,其主要之城名曰福州。由此东南行六日,过山越谷……其地产姜、高良姜及他种香料甚饶,用一值威尼斯城银钱一枚之货币,可购生姜八十磅。尚有一种植物,其果与真正洎夫蓝之一切原质无别,有其色味,人甚重之,而用为一切食馔中之酌料,所以其价甚贵……其人作战时,垂发至肩,染面作蓝色,甚光耀……”(第七十五章)

【2】 “行此国六日至格陵府(Quelinfu),城甚广大。有三桥甚美,各长百余步,宽八步,用石建造,有大理石柱。此城妇女甚美,生活颇精究。其地产生丝甚多,用以织造种种绸绢,并纺棉作线,染后织为布,运销蛮子全境……闻人言,其地有一种母鸡,无羽,而有黑毛如猫毛,产卵,与吾国之卵无异,颇宜于食。其地有虎甚众,颇为行人患,非聚多人不能行。”(第七十六章)

【3】 “自建宁府出发,行三日,沿途常见有环墙之城村,居民是偶像教徒,饶有丝,商业贸盛。抵温敢(Unguem)城。此城制糖甚多,运至汗八里城,以充上供。温敢城未降顺大汗前,其居民不知制糖,仅知煮浆,冷后成黑渣。降顺大汗以后,时朝中有巴比伦(Babylonie,指埃及)地方之人,大汗遣之至此城,授民以制糖术,用一种树灰制造。”(第七十七章)





153 福州之名贵

应知此福州城,是楚伽(Chouka)国之都城,而此国亦为蛮子境九部之一部也。此城为工商辐辏之所。居民是偶像教徒而臣属大汗。大汗军戍此者甚众,缘此城习于叛变,故以重兵守之。





【1】


有一大河宽一哩,穿行此城。 【2】 此城制糖甚多,而珍珠、宝石之交易甚大,盖有印度船舶数艘,常载不少贵重货物而来也。此城附近有刺桐(Zayton)港在海上,该河流至此港。 【3】

在此(福州)见有足供娱乐之美丽园囿甚多。此城美丽,布置既佳,凡生活必需之物皆饶,而价甚贱。

此外无足言者,请仍前行。

地学会法文本有增订之文如下:

注释

【1】 “军队戍此者甚众,盖其境内城村屡有叛变之事,故大汗以数军戍之,由是若有叛变发生,福州之戍军立取叛城毁之。”

【2】 “此城建造不少船舶,以供航行此河之用。”

【3】 “有不少印度船舶来此,亦有商人赴印度诸岛贸易。尚须为君等言者,此城近海上之刺桐港,印度船舶运载不少货物赴此港者甚众。诸船离此港后,上溯前述之大河而至福州城。此城因此输入印度之贵重货物。”





154 刺桐城

离福州后,渡一河,在一甚美之地骑行五日,则抵刺桐(Zayton)城,城甚广大,隶属福州。此城臣属大汗。居民使用纸币而为偶像教徒。应知刺桐港即在此城,印度一切船舶运载香料及其他一切贵重货物咸莅此港。是亦为一切蛮子商人常至之港,由是商货宝石珍珠输入之多竟至不可思议,然后由此港转贩蛮子境内。我敢言亚历山大(Alexandrie)或他港运载胡椒一船赴诸基督教国,乃至此刺桐港者,则有船舶百余,所以大汗在此港征收税课,为额极巨。

凡输入之商货,包括宝石、珍珠及细货在内,大汗课额十分取一,胡椒值百取四十四,沉香、檀香及其他粗货值百取五十。

此处一切生活必需之食粮皆甚丰饶。并知此刺桐城附近有一别城,名称迪云州(Tiunguy),制造碗及瓷器,既多且美。除此港外,他港皆不制此物,购价甚贱。此迪云州城,特有一种语言。大汗在此崇迦(Concha)国中征收课税甚巨,且逾于行在国。

蛮子九国,吾人仅言其三,即行在、扬州、福州是已。其余六国虽亦足述,然叙录未免冗长,故止于此。

由前此之叙述,既使君等详知契丹、蛮子同其他不少地方之情形,于种族之别,贸易之物、金银,与夫所见之其他诸物,悉具是编。然吾人所欲言者,本书未尽。尚有印度人之事物,及举凡足供叙述之印度大事,至为奇异,确实非伪,吾人亦据波罗阁下之说笔之于书。盖其久居印度,对于风习及特点,知之甚审,我敢言无有一人闻见如彼之多也。

剌木学本此章较详,兹全录其文,以资参考。

“刺桐城港及亭州(Tingui)城”

“离漳州(Cangiu)后先渡一河,然后向东南行五日,见一美地,城市民居接连不断,一切食粮皆饶,其道经过山丘、平原同不少树林,林中有若干出产樟脑之树,是一野味极多之地。居民是偶像教徒,臣属大汗而隶漳州。行五日毕,则抵壮丽之城刺桐,此城有一名港在海洋上,乃不少船舶辐辏之所,诸船运载种种货物至此,然后分配于蛮子全境。所卸胡椒甚多,若以亚历山大运赴西方诸国者衡之,则彼数实微乎其微,盖其不及此港百分之一也。此城为世界最大良港之一,商人、商货聚积之多,几难信有其事。”

“大汗征收税课为额甚巨,凡商货皆值百抽十。顾商人细货须付船舶运费值货价百分之三十,胡椒百分之四十四,沉香、檀香同其他香料或商品百分之四十,则商人所缴副王之税课连同运费,合计值抵港货物之半价,然其余半价尚可获大利,致使商人仍欲载新货而重来。”

“居民是偶像教徒,而有食粮甚饶。其地堪娱乐,居民颇和善,乐于安逸。在此城中见有来自印度之旅客甚众,特为剌青而来(语见第一二六章),盖此处有人精于文身之术也。”

“抵于刺桐港之河流甚宽大,流甚急,为行在以来可以航行之一支流。其与主流分流处,亭州城在焉,此城除制造瓷质之碗盘外,别无他事足述。制瓷之法,先在石矿取一种土,暴之风雨太阳之下三四十年。其土在此时间中成为细土,然后可造上述器皿,上加以色,随意所欲,旋置窑中烧之。先人积土,只有子侄可用。此城之中瓷市甚多,威尼斯钱一枚,不难购取八盘。”

“崇迦(Concha)国是蛮子九州之一,大汗所征税额与行在国相等。今既述此国若干城市毕,其余诸国置之不言,盖波罗阁下在余国居留,皆不及居留行在、崇迦两国之久也。”

“尚应言者,蛮子全境各地有种种方言,犹之热那亚人(Génois)、米兰人(Milanais)、弗罗郎司人(Florentins)、阿普里人(Apuliens)各有一种语言,仅有本地之人能解,第蛮子全境仅有一种主要语言,一种文字也。”

“波罗阁下所欲言者吾人述之未尽,兹结束此第二卷,请述大印度、小印度、中印度之州郡城邑。波罗阁下曾奉大汗命亲莅其中若干城邑,而最后归国时曾偕其父叔送王妃于国王阿鲁浑也。由是彼有机会述其亲览之异事,而对于所闻可信之人之言,与夫航行于印度者的地图之所载,亦毫无遗漏焉。”

第二卷译后语

马可·波罗书四卷以此卷为最长,而难题亦以此卷为最多。此卷专记中国事,论理地名可从中国,其实有不然者。此书不过是大德年间一部撰述,在中国人视之,不能算为古本,但因传本太多,写法不一,其难一。波罗路线不明,如自涿州至西安,又自涿州至淮安,中间究竟经行何地,别无他书可以参考,其难二。沙氏个人考订,颇多附会穿凿,往往妄改原书地名,改行在为杭州府,倘有说也,写镇巢军作常州府、写塔皮州作绍兴府,未免过于武断,由是于地名错杂之中,更加紊乱,其难三。职是之故,译文于本卷之地名,经沙氏妄改者皆复其旧,大致不误者录其原名,稍涉疑义者写其对音,所以有该州、哈强府、阿木州、秃落蛮、哈寒府、强格路、强格里、中定府、新州马头、临州、西州、新州、塔皮州等无从比附之译名。此类译名之对音,未敢必其读法不误,缘此书涉及语言甚多,固有主张原本为法文本之说者,然其中有若干写法多从意大利文,故本卷译音大致从意大利语读法。复次,本书对于同一地名,著录之写法不一,如第一五四章,注甲之崇迦,第一五五章,又作楚迦,乃其一例,译文两录之,他皆仿此。此外译文,务求不失原文朴质风味,原文编次虽欠条理,且多复词叠句,然未敢稍加改窜,宁失之干燥,不愿钩章棘句而失其真。总之,本卷中之难题甚多,足供考据家之爬梳也。

1935年8月1日冯承钧识





155 首志印度述所见之异物并及人民之风俗

前述诸地,叙录既毕,此后请言印度及其异物,而首言商人所附以往来印度诸岛之船舶。

应知其船舶用枞木(sapin)制造,仅具一甲板。各有船房五六十所,商人皆处其中,颇宽适。船各有一舵,而具四桅,偶亦别具二桅,可以竖倒随意。 【1】 船用好铁钉结合,有二厚板叠加于上,不用松香,盖不知有其物也,然用麻及树油掺和涂壁,使之绝不透水。

每船舶上,至少应有水手二百人,盖船甚广大,足载胡椒五六千担。 【2】 无风之时,行船用橹,橹甚大,每具须用橹手四人操之。每大舶各曳二小船于后,每小船各有船夫四五十人,操棹而行,以助大舶。别有小船十数助理大舶事务,若抛锚、捕鱼等事而已。大舶张帆之时,诸小船相连,系于大舟之后而行。然具帆之二小舟,单行自动与大舶同。

此种船舶,每年修理一次,加厚板一层,其板刨光涂油,结合于原有船板之上,其单独行动张帆之二小船,修理之法亦同。应知此每年或必要时增加之板,只能在数年间为之,至船壁有六板厚时遂止。盖逾此限度以外,不复加板,业已厚有六板之船,不复航行大海,仅供沿岸航行之用,至其不能航行之时,然后卸之。

既述往来海洋及诸印度岛屿之船舶毕,请先言印度之异物。

剌木学本第三卷第一章之异文如下:

注释

【1】 “其商船用枞木、松木制造,诸船皆只具一甲板,上有船房,视船之大小,房数在六十所上下,每房有一船客,居甚安适。诸船皆有一坚舵,具四桅,张四帆,有时其中二桅可以随意竖倒。此外有若干最大船舶有内舱至十三所,互以厚板隔之,其用在防海险,如船身触礁或触饿鲸而海水透入之事。其事常见,盖夜行破浪之时,附近之鲸见水起白沫,以为有食可取,奋起触船,常将船身某处破裂也。至是水由破处浸入,流入船舱,水手发现船身破处,立将浸水舱中之货物徙于邻舱,盖诸舱之壁嵌隔甚坚,水不能透,然后修理破处,复将徙出货物运回舱中。”

【2】 “诸船舶之最大者,需用船员三百人或二百人或一百五十人,多少随其大小而异,足载胡椒五六千包。昔日船舶吨数常较今日为重,但因波浪激烈,曾将不少地方沙滩迁徙,尤其是在诸重要海港之中,吃水量浅,不足以容如是大舟,所以今日造船较小。”





156 日本国岛

日本国(Zipangu)是一岛,在东方大海中,距陆一千五百哩。其岛甚大,居民是偶像教徒,而自治其国。据有黄金,其数无限,盖其所属诸岛有金,而地距陆甚远,商人鲜至,所以金多无量,而不知何用。

此岛君主宫上有一伟大奇迹,请为君等言之。君主有一大宫,其顶皆用精金为之,与我辈礼拜堂用铅者相同,由是其价颇难估计。复次宫廷房室地铺金砖,以代石板,一切窗栊亦用精金,由是此宫之富无限,言之无人能信。

有红鹧鸪甚多而其味甚美。亦饶有宝石、珍珠,珠色如蔷薇,甚美而价甚巨,珠大而圆,与白珠之价等重。 【1】

忽必烈汗闻此岛广有财富,谋取之。因遣其男爵二人统率船舶、步骑甚众而往。兹二男爵谨慎勇敢,一名阿巴罕(Abacan)、一名范参真(Vonsainchin),率其部众自刺桐、行在两港登舟出发。既至,登陆,夺据一切平原、村庄,然未能攻下何种城堡,由是有下述之祸发生。

会北风大起,此岛沿岸少有海港,因是大受损害,风烈甚,大汗舰队不能抵御。诸帅见之以为船舶留此,势必全灭,于是登舟,张帆离去。航行不久,至一小岛,风浪漂流,欲避不能,船多破沉,其军多死,仅余三万人得免,避难于此岛中。 【2】

彼等既无食粮,不知所措,待死而已。然在绝望之中,见有若干船舶未遭难者疾驶返向本国,不来援救。盖统军之二男爵互相嫌忌,得脱走之男爵遂不欲回救其避难岛中之同僚。但风势不久便息,可以回舟援救,而彼不欲救之,径还本国。避难之岛,绝无人烟,除彼等外,不见他人。

剌木学本第三卷第二章增订之文如下:

注释

【1】 “岛人死者,或用土葬,或用火葬,土葬者习含此珠一粒于口而殓。”

【2】 “风暴怒起,船舶破沉者甚众,仅有攀附破船遗物者得免,避难于邻近之一岛中,岛距日本国岸四哩。其他诸舟距岸较远者,未曾受难,二男爵及诸统将若百户、千户、万户等并在其中,遂张帆还其本国。”





157 避难岛中之大汗军夺据敌城

留于岛中者有三万人,既无法得脱,待死而已。大岛之王闻敌兵一部避难岛中,另一部皆散走逃还本国,甚喜,遂聚集大岛一切船舶,已而进至小岛,环岛登岸。登岸以后,不留一人看守船舶,其谋至为不慎。鞑靼人较有策谋,见敌众登岸,乃伪作逃走之状,群登敌舟,舟中空无一人,登之甚易。

登舟以后,立即出发,航至大岛,登陆以后,执本岛君主旗帜,进向都城。城中戍守之众,未虞敌至,见本国旗帜,以为本国兵至,听其入城。彼等尽入城后,占据一切险要,尽驱城众出城,仅留美女,大汗军取城之法如此。

大岛之王及其军队,见都城、舰队尽失,大痛,然犹登余舟进至大岛沿岸,立集全军,近围都城,围之甚密,无人可以出入。城内之众守城七阅月,日夜谋以其事通知大汗,然交通既断,无法上闻也。及见不能再守,遂约免死求降,并许永不离去此岛。事在救世主诞生后之1279年也。

二男爵中之一人逃归者大汗断其首,嗣后并杀留居岛中之别一人,盖在战中,练达之将不能有此失也。

此次远征中尚有别一异事,前忘言之,兹请追述于此。初,大汗军在大岛登岸占据平原后,有一塔拒守不降,攻拔之,尽断守者之首,惟有八人不受刃伤。盖其臂上皮肉之间,巧嵌石块,以作护符,此石效力足使嵌之者可免铁伤。诸男爵闻其事,命杖杀其人,将各人之石取出而宝视之。

兹置此事不言,请回言本题。





158 (一)偶像之形式

应知契丹、蛮子之偶像与夫日本之偶像,悉皆相同。有牛头者,别有猪头、狗头或羊头及其他数种形貌者。又有若干四头者,别有三头者,两臂各有一头。更有四手者、十手者及千手者,千手之像,受人信奉较甚于他像。基督教徒曾询彼等,何以造作偶像形貌不同而不相类,彼等答曰,祖宗传之子孙,即已如此,而彼等留传于后人亦复如是,由是永远皆然。应知此种偶像作为悉属魔术,未便述之。所以置此偶像不言,请言他事。

尚应为君等言者,此岛(日本)及印度海其他诸岛之居民,俘一敌者,若敌不能用金赎还,俘敌之人则召集一切亲友,杀此人而聚食其肉,谓是为世上最美之肉。

兹置此事不言,请言他事。

应知此类岛屿所处之海,名称秦(Cin、Cim)海,犹言接触蛮子地方之海也。盖此类岛民语言称蛮子曰秦,故以名之。此海延至东方,据习于航行此海渔夫、水手之说,彼等时常往来水道之中,共有七千四百五十九岛,彼等除航海外不作他事,故熟知之。诸岛皆出产贵重芬芳之树木,如沉香木及其他良木之类,亦有调味香料种类甚多。例如制造胡椒,色白如雪,产额甚巨,即在此类岛屿也。由是其中一切富源,或为黄金、宝石,或为一切种类香料,多至不可思议,然诸岛距陆甚远,颇难到达,刺桐、行在船舶之赴诸岛者皆获大利。

来往行程需时一年,盖其以冬季往,以夏季归。缘在此海之中,年有信风二次,一送其往,一送其归。此二信风,前者亘延全冬,后者亘延全夏。君等应知其地距印度甚远,赴其地者需时甚长。此海虽名秦海,广大不下西方大海(指大西洋)。其在此处具此名,犹之在英吉利名海曰英吉利海,他处名海曰印度海,然此种种海,皆不失为西海之一部也。

此地为难至之贵地,马可·波罗阁下从未涉足其间。大汗与之毫无关系,诸岛对之不纳贡赋,不尽藩职。

所以吾人重返刺桐,是为小印度发航之所。





158 (二)海南湾及诸川流

从刺桐港发足向西,微偏西南行一千五百哩,经一名称海南(Cheinan)之海湾。其海岸延长二月程,船沿行其北部全境,其地一方面与蛮子州东南部连界,一方面与阿木(Amu)、秃落蛮(Toloman)及其他业经著录之诸州境界相接。湾内多有岛屿,泰半繁殖居民,岸边有金沙甚多,在诸川入海处拣之。亦产铜及他物,各岛以其产物贸易,此岛有者,他岛无之。岛民亦与陆地之人交易,出售金、铜及他种出产,而购人其所需之物。诸岛多半饶有谷食。此湾幅员之广,人民之众,似构成一新世界。





159 占巴大国

从刺桐出发向西、西南航行千五百哩,则抵一地,名称占巴(Ciampa、Cyamba),是为一极富之地,自有国王,并自有其语言。居民是偶像教徒,每年贡象于大汗,除象以外不贡他物,兹请述其贡象之故





【1】 。

基督诞生后1278年时,大汗遣其男爵一人名称撒合都(Sagatu)者,率领步骑甚众,往讨此占巴国王。此男爵对于此国及其国王作大军事行动。国王年老,所部军不如男爵之众,见此男爵残破其国,颇痛心。遂遣使臣往大汗所,而致辞曰:“我主占巴国王,以藩臣名义遣使入朝,国王年老,所治之国久已平和安宁。今愿称臣,每年贡象,其数惟君所欲。请赐怜悯,命君之男爵及所部之众不再残破我国,率众他适,以后国王奉君之命,代治此国。”

大汗闻言悯之,乃命其男爵率其部众离去此国,往侵他国。大汗命至,男爵及其部众遂行。此国王成为大汗藩臣之故如此,每年贡象二十头,乃国中最大而最美之象也。

兹置此事不言,请言占巴国王之若干特点。

应知此国之妇女 【2】 ,未经国王目见者,不得婚嫁,国王见而喜,则娶以为妻,不喜,则赐以嫁资,俾能婚嫁。并应知基督诞生后1280年时,马可·波罗阁下身在此国,是时国王有子女三百二十六人,其中能执兵者一百五十。

此国有象甚众,并见有大森林,林木黑色,名称乌木,以制箱匣。此外无足言者,此后接言他事。

剌木学本第三卷第六章增订之文如下:

注释

【1】 “国王称阿占巴勒(Accambale),年事甚老,无军可敌大汗军,逃避于安宁可守之城堡中,然平原中之城市民居,悉遭残破。国王见敌众残破其一切领土,遂遣使臣往谒大汗致辞,谓其年老,国家久安,请勿残破,命该男爵率部众他适,每年将进象及沉香,以充贡品。”

【2】 “此国室女美者,非进献国王后,不得婚嫁,若国王见喜,留之若干时,然后赐以金,俾其婚嫁……国王有子女三百二十五人,诸子多为勇武战士。此国产象甚众,而沉香亦甚多。”

斡朵里克书亦有异文足资转录:

“纳覃(一作Panthen)岛附近有一国名称占巴(Campe,即Champa),是一美丽之国,种种食粮、产物悉皆丰饶。我莅此国时,国王有子女二百人,盖王有妻数人,而妄甚众也。此王有象万头以供役使,命城人看守饲养之。此国有一异事,海中诸鱼辄聚于海岸,岸边只见有鱼,不见有水。每类轮流聚集岸边三日,期满则去,由是别种继至,停留之期亦同,迄于诸类皆至始止,每年如是。以询土人,土人答曰诸鱼来朝国王。我在此地曾见一龟,其大无比,较之巴杜(padoue)城圣马儿丁(Saint-Martin)之钟楼更大。此地男子死,则将其妻生殉,据云,妻宜随夫于彼世。”——戈尔迭本斡朵里克书187-188页。





160 爪哇大岛

自占巴首途向南航行千五百哩,抵一大半岛,名称爪哇(Jawa)。据此国水手言,此地为世界最大之岛。此岛周围确有五千哩,属一大王而不纳贡他国。居民是偶像教徒。此岛甚富,出产黑胡椒、肉豆蔻、高良姜、荜澄茄、丁香及其他种种香料,在此岛中见有船舶商贾甚众,运输货物往来,获取大利。大汗始终未能夺取此岛,盖因其距离甚远,而海上远征需费甚巨也。刺桐及蛮子之商人在此大获其利。





161 桑都儿岛及昆都儿岛

自爪哇首途,





【1】 向南航行七百哩,见有二岛,一大一小,一岛名桑都儿(Sandur),一岛名昆都儿(Condur)。此处无足言者, 【2】 请言更远之一地,其地名称苏哈惕(Soucat),在桑都儿岛外五百哩。 【3】 是一富庶良好之地,自有其国王。居民是偶像教徒,自有其语言。其地远僻,无人能来侵,故不纳贡赋于何国。设若有人能至其地,则大汗将尽征服之矣。 【4】

此地饶有吾人所用之苏木。黄金之多,出人想象之外,亦有象及不少野味。前述诸国用作货币之海贝,皆取之于此国也。

此地甚荒野,往者甚稀,此外无足言者。而且国王不欲人知其国之财富,亦不愿外人来此。

兹请接述朋丹(Pontain)岛。

剌木学本第三卷第八章增订之文如下:

注释

【1】 “自此爪哇(Giava)岛首途向西南南航行七百哩。”

【2】 “盖此二岛,未有民居。”

【3】 “隶属陆地,地大而富,名称Lochac。”

【4】 “此地有一种果,名称berci,大如柠檬,味佳可食。”





162 朋丹岛

尚应知者,自罗迦克(Lochac)首途,向南航行五百哩,抵一岛,名称朋丹,地甚荒野,一切树林满布香味之树。

此外无足言者,又在上言二岛间





【1】 航行六十哩,此六十哩中水深仅有四步,由是大船经过者必须起舵。

行此六十哩毕,又行三十哩,见一岛,是为一国,名麻里予儿(Maliur)。居民自有国王,并其特别语言。其城大而美,商业繁盛。有种种香料,此外一切食粮皆饶。

此外无足言者,请作更远之行。

注释

【1】 剌木学本第三卷第九章之异文云:“罗迦克州及彭覃(Pentam)岛间宽六十哩,水深多不过四步,所以航行之人必须起舵。向东南(应作西南)航行此六十哩后,复接行约三十哩,至一岛,岛为一国,都城名称麻剌予儿,故名麻剌予儿岛。”

波罗在此处叙述较为明了,“上言二岛”,盖一指罗迦克州,殆视之为半岛(则指马来半岛),一距宽六十哩水深四步之海峡(星加坡老峡),然则非今地图上之万丹(Bintang)岛,而为星加坡(Singapour)岛矣。





163 小爪哇岛

自麻里予儿岛首途,向西南航行九十哩,则抵小爪哇(Jauva la mineur)岛,虽以小名,其周围实逾二千哩也。兹请全述关于此岛之事。

应知此岛有八国八王。居民皆属偶像教徒,各国自有其语言。此岛有香料甚多。





【1】


兹请为君等叙述关系此八国中大半数国之事。然有一事先应知者:此岛偏在南方,北极星不复可见。 【2】

今请回言本题,首述八儿剌(Ferlec、Frelach)国。

应知回回教徒时常往来此国,曾将国人劝化,归信摩诃末(Mahomet)之教,然仅限于城居之人,盖山居之人生活如同禽兽,食人肉及一切、肉,并崇拜诸物也。此辈早起,崇拜其首见之物,终日皆然。

既述此八儿剌国毕,请言巴思马(Basma、Basman)国。

离八儿剌国后入巴思马国,亦一独立国也。居民自有其语言,生活如同禽兽,盖其不信何教,虽自称隶属大汗,缘地过远而不纳贡赋。第若大汗可畏之士卒能抵此地,不难将其征服,然偶亦进奉异物于朝。 【3】 国中多象,亦有犀牛,鲜有小于象者。此种独角兽,毛类水牛,蹄类象额中有一角,色白甚巨,不用角伤人,仅用舌,舌有长剌甚坚利,其首类野猪,常俯而向地,喜居湖沼及垦地附近。 【4】 此兽甚丑恶,人谓室女可以擒之,非事实也。亦见有猴甚众,计有数种。并见有苍鹰,其黑如乌,躯大可饲养也。

有携小人至吾国,而谓其产自印度者,盖伪言也,是即此岛所产之猴,兹请言其伪造之法。有一种猴身躯甚小,面貌与人无异。人捕之,全拔其毛,仅留颔毛、阴毛,已而听其干,剥而用洎夫蓝 【5】 及他物染制,俾其类人。然是为一种欺人之术,盖在全印度境中以及其他更较蛮野之地,从来未见如是之人也。

兹不复言此巴思马国,后此按次历言他国。

离此巴思马国后,即至一名须文答剌(Samudra)之国,此国亦在同岛之中。马可·波罗阁下曾因风浪不能前行,留居此国五月。在此亦不见有北极星及金牛宫星(bouvier)。居民亦自称隶属大汗。马可·波罗阁下既因风浪停留此国五月,船员登陆建筑木寨以居。 【6】 盖土人食人,恐其来侵也。其地鱼多,世界最良之鱼也。无小麦,而恃米为食,亦无酒。然饮一种酒,请言取酒之法如下:应知此地有一种树,土人欲取酒时,断一树枝,置一大钵于断枝下,一日一夜,枝浆流出,钵为之满。此酒味佳,有白色者,有朱色者。此树颇类小海枣树。土人断枝,仅限四枝,迨至诸枝不复出酒时,然后以水浇树根,及甫出嫩枝之处。土产椰子甚多,大如人首,鲜食甚佳,盖其味甜而肉白如乳也。肉内空处有浆,如同清鲜之水,然其味较美于酒及其他一切饮料。 【7】

既述此国毕,请言他国。

离此须文答剌后,入一别国,名称淡洋(Dagroian、Dangroian、Angrinan)。是一独立国。居民是偶像教徒,性甚野蛮,自称隶属大汗。今请言其一种恶俗。

若有一人有疾,即招巫师来,询其病能愈与否。巫者若言应愈,则听其愈。然若巫者预卜其病不愈,则招集多人处此种病人死。诸人以衣服堵病者口而死之。病者死后,熟其肉,死者诸亲属共食之。此辈吸其骨髓及其他脂肪罄尽,据云骨内若有余留之物,则将生虫,此虫则必饿死,由是死者负担虫死之责,故尽食之。食后聚其骨,盛于美匣之中,然后携往山中禽兽不能至之大洞中悬之。应知此辈若得俘虏,而此俘虏不能买赎时,立杀而食之。此俗极恶也。

既述此国毕,请言他国。

离此国后,入一别国,名称南巫里(Lambry、Lanbri)。居民自称隶属大汗而为偶像教徒。多有樟脑及其他种种香料,亦有苏木甚多。种植苏木,徒其出小茎时,拔而移种他处,听其生长三年,然后连根拔之。马可·波罗阁下曾将苏木子实携归威尼斯,种之不出,殆因天气过寒所致。

尚应知者,南巫里国有生尾之人,尾长至少有一掌而无毛。此种人居在山中,与野人无异,其尾巨如犬尾。此国多有犀牛,亦有不少野味。

既述此南巫里国毕,入一别国,名称班卒儿(Fansour)。国人是偶像教徒而自称隶属大汗。此班卒儿国出产世界最良之樟脑,名称班卒儿樟脑,质极细,其量值等黄金。无小麦,然有米,共乳肉而食。亦从树取酒,如前所述(见须文答剌条)。

尚有一异事,须为君等述者。其国有一种树,出产面粉,颇适于食。树巨而高,树皮极薄,皮内满盛面粉。马可·波罗阁下见此,曾言其数取此面,制成面包,其味甚佳。 【8】

此外无足言者,岛中八国,已将此方面之六国述讫,别有二国在岛之彼方,未能言之,盖马可·波罗阁下未至其地也。所以吾人叙述小爪哇岛事,仅止于此。兹请述二小岛,一名加威尼思波剌(Gavenispola)岛,一名捏古朗(Nécouran)岛。

剌木学本第三卷第十至第十七章增订之文如下:

注释

【1】 “其地饶有金银,一切香料,沉香、苏木、乌木等物,因道远而海行险,故未输入吾国,然运往蛮子、契丹诸州。”(十章)

【2】 “此八国马可阁下曾历六国,其所述者仅此六国,余二国无缘可见,故略。”(十章)

【3】 “有时船至其国,国人乘便进奉珍异之物入朝,以为贡品,就中有一种苍鹰。”(十二章)

【4】 “此种独角兽较象甚小……不用角伤人,仅用舌与蹄……其攻击之法,蹂敌于地,以舌裂之。喜处泥中,盖为粗野动物也。”(十二章)

【5】 “风干而后用樟脑及其他药物保存,用此法制造,使之完全具有小人形貌,然后用木匣盛之,售之商人,贩往世界各地。”(十二章)

【6】 “既须久留此岛,马可阁下遂偕众约二千人登陆居住海岸。土居野人常捕人而食,防其来袭,在所居之处掘一大沟,两端通海。在此沟上用木建筑堡塞数所,土产木料甚多,足以供用。如是防守,因得安居五月。岛人慑服,乃如约供给粮食及其他必要物品。”(十三章)

【7】 “此浆功效甚大,可治水肿病及脾肺之疾。若见断枝不复流浆,则用小沟导溪水浇树,赖有此法,水浆重流如前。有蔷薇色者,有白色者。”(十三章)

【8】 “外皮甚薄,先剥去之,内有木厚三指,木内有粉如玉米粉。此树甚大,须两人始能合抱之。取粉置水器中,用杖搅之,俾糠及其他不洁之物浮出水面,净粉沉于器底。去水留粉,然后适用。以制种种饼饵,形味如同大麦面包,马可阁下常食之,并且携归威尼斯。此树之木重如铁,投于水,即沉底;可直劈如竹,盖取髓以后,尚有厚三指之木也。土人削此木做短矛,从不作长矛,盖矛长量重不能执也。将其一端削尖,然后用火烤之,其坚可洞任何甲胄,较优于铁矛也。”(十六章)





164 加威尼思波剌岛及捏古朗岛

离前述之爪哇(小爪哇)岛及南巫里国后,若向北行一百五十哩,则见二岛,一名捏古朗(Nécouran),一名加威尼思波剌(Gavenispola)。居民无王无主,生活如同禽兽。其人裸体,男女皆无寸缕。并是偶像教徒。其林中只有贵重树木,出产檀香、椰子、丁香、苏木及其他数种香料。

此外无足言者,是以仍前行,请言一名案加马难(Angamanain)之岛。





165 案加马难岛

案加马难(Angamanain)是一大岛,居民无国王,并是偶像教徒,生活如同禽兽。此案加马难岛民皆有头类狗,牙眼亦然,其面纯类番犬。彼等颇有香料,然甚残猛,每捕异族之人,辄杀而食之。彼等食米与肉乳,亦有果实,然与我辈地方所产者不同。

此族堪在本书著录,故为君等言之。兹请为君等叙述名称锡兰(Seilan、Seylam)之岛。





166 锡兰岛

若从案加马难岛首途,向西航行约千哩,不见一物;然若向西南行,则抵锡兰(Seilan、Seylam)岛。由其面积言,是诚为世界最良之岛。应知其周围确有二千四百哩,然古昔面积更大,据富有经验之水手言,昔日此岛周围有三千哩。然因北风强烈,致使此岛一大部分陆沉。其面积不及昔日之大,理由如此。应知北风所吹之海岸甚低而平,船舶来自大海时,若不航近其处,不见陆地。

兹请言此岛足以注意之事。岛民有一国王名称桑德满(Sandemain),而不隶属何人。彼等是偶像教徒,裸体往来,仅掩盖其下体而已。无麦而有米,亦有芝麻作油。食肉乳,而饮前述之树酒。饶有苏木,世界最良者也。

兹请不言此事,请言世界最贵重之物。应知此岛所产之红宝石,他处无有,仅在此岛见之。岛内亦有蓝宝石、黄宝石、紫晶及其他种种宝石。岛中国王有一红宝石,为世界红宝石中之最大而最美者。兹请言其状:其长有一大掌,其巨如同人臂。是为世界最光辉之物,其红如火,毫无瑕疵,价值之大,颇难以货币计之。大汗曾遣使臣礼求,请将此宝石售出,请之甚切,致愿以任何城市易之。国王答言,此宝祖先传留,无论世界何物皆不足以易。

人民不习武备,皆是孱弱怯懦之人,然若需要士卒,则募别一国之回教徒为之。

尚应知者,锡兰岛中有一高山矗立,除用下一法外,难登其巅:此岛之人系大而巨之链数条于此山上,行人攀链以登。回教徒自称此山乃是元祖亚当(Adam)之墓。然偶像教徒,又断言是为世界第一偶像教徒葬身之所,其名曰释迦牟尼不儿罕(Sagamoni Borcam),据称是一大圣人。

据说其人是一富强国王之子,不染世俗浮华风习,不欲袭位为王。其父闻其不愿为王,不爱荣华,忧甚,曾以重大许诺饵之。然其子一无所欲,其父别无他子承袭王位,尤深忧痛。由是国王建一大宫以居其子,多置美丽侍女侍之。命诸美女日夜与其子游乐,歌舞以娱,俾之得染世俗浮华之习,然悉皆无效。

王子好学,从未出宫,从未见有死人及残废之人,盖其父不许旅人之微有残疾者入见其子也。一日王子骑而出游,见一死人,彼从未见此,颇以为异,询之侍从,知是死人。王子问曰凡人皆死欤,诸人答曰然。王子遂不复言,沉思仍前行。行若干时,见一老人鬻口中齿尽落,不能举步。王子又询此为何人,何以不行,侍者答攒其人老朽齿落而不能行。王子回宫自思,不能再居此恶世,应往求永远不死之造化者。

缘其既见此世之中老少皆死,遂于某夜秘密离宫,往大山中。在其地节欲习苦,俨若基督教徒。盖若其为基督教徒,则将共吾主耶稣成为大圣矣。

迨其死后,有人见之,畀往父所。国王见其爱子之尸,悲恸几至疯狂。命人范金做像,饰以宝石,令国人尽崇拜之。众人皆谓其成神,今日言尚如此,并言其曾死八十四次。第一次做人死,已而复生为牛,牛死为马,如是死八十四次,每次成一不同之兽畜,末次死后遂成为神,传说如此。彼等奉之为最大之神。据说此王子是偶像教徒亘古所无之第一偶像,其他偶像皆出于此,而此事在印度、锡兰岛中也。

尚应为君等言者,回教徒自远道来此巡礼,谓其是亚当。偶像教徒亦自远道来此巡礼,如同基督教徒之赴加里思(Galice)朝拜圣雅各(St.Jacques)者无异,据云确是王子,如前所述,而今尚存山中之牙、发与钵,确是圣者释迦牟尼之物。何说为是,只有上帝知之,考吾教之《圣经》,亚当墓不在斯处也。

嗣后大汗闻此山中有元祖亚当之墓,而其所遗之牙、发、供食之钵尚存,于是欲得之,乃于1284年遣使臣往。使臣循海遵陆,抵锡兰岛,入谒国王,求得齿二粒,甚巨;并得所遗之发及供食之钵,钵为绿色云斑石(porphyre)质,甚美。大汗使臣获有诸物后,欣然回国复命。及至大汗驻跸之汗八里城附近,命人请命于大汗,如何呈献诸物。大汗闻讯大喜,命人往迎亚当遗物。于是往迎并往致敬者人数甚众,大汗大礼庄严接受之。相传此钵颇有功效,置一人之食肉于其中,其肉足食五人。大汗曾面试之,果验。

大汗大耗费用取得此种遗物之事,君等已知之,土人传说关于王子遗物之沿革,君等亦已知之。

此外无足言者,所以离此请言马八儿(Maabar)州。





\chapter{陆地名称大印度之马八儿大州}

若离锡兰岛,向西航行可六十哩,则至名称大印度之马八儿大州,是为印度之良土而属大陆。

应知此州之中有国王五人是亲兄弟,行将依次言之。此州为世界最美而最名贵之州。

州之极端,五兄弟国王之一人君临其地,而称宋答儿班弟答瓦儿(Sonder Bandi Davar)。其国有珍珠,甚大而美,兹请言其采取之法。

应知锡兰岛与陆地之间有一海湾,沿湾之水,仅深十步至十二步,间有不逾两步者。采珠之人在四月至五月半间,乘舟至此湾中名称别贴剌儿(Betelar)之地。复由是在湾内航海六十哩。及至其地,抛锚停船,离大船而驾小舟。

应知彼等有商贾数人偕行,并应在4月至5月半间,雇用数人与俱。彼等纳什一之税于国王,并应视所得物额给二十分之一于咒镇大鱼之人,俾下水采珠之人不为大鱼所害。此种咒鱼之人名称婆罗门(Brahmans),其咒镇仅一日,盖在夜间则解其咒,使鱼得任意为患。并应知此种婆罗门亦知咒镇禽兽及一切具有灵魂之物。

诸人下小舟后,投身水中深处,水深四步至十二步不等,留存水中至于力尽之时,采取产珠之贝。此种贝形,如同牡蛎或海蟹一般。贝中有大小珠结于贝中肉内。采珠之法如此,所采甚多,因是其珠散布世界。应知此国国王所课珠税甚高,而获有一种极大收入。惟过5月半后,则不复见有产珠之贝。但距此至少有三百哩之地,亦产贝珠,然仅在9月至10月半间采之。

应知此马八儿全州之中,不见有一裁缝师或缝衣工人裁制衣服,缘居民皆裸体往来,仅以片布盖其下体,男女贫富皆然。国王亦若是,惟戴有若干物品,请为君等述之。

彼项上戴环,全饰宝石,如红宝石、蓝宝石、绿宝石及其他宝石之类,由是此环价值甚巨。胸前项下悬一丝线,串大珠一百零四颗与红宝石数粒。据说国王悬此一百零四珠与宝石之线串者,盖因每日应对其偶像祷颂一百零四次也。其教俗如此,国王祖先皆悬之,所以留传后人,俾其悬挂。

国王臂上亦戴三金环,全以重价珍珠、宝石为饰,腿上甚至脚趾亦然。因是国王所戴之黄金、珍珠、宝石价值连城。此事不足为异,盖其所藏甚多,兼为国中所出也。并应知者,凡珍珠重半量(demi-poids)者,不许携出国外,除非密带出境。国王欲一切珠宝属己,故有此禁,由是其所藏之多,言之无人能信。每年数次宣告国中,凡有重价珍珠、宝石,必须呈献国王,国王倍给其价,由是人亦愿献,国王尽收之,而偿各人之价。

尚应知者,此国王有妻不下五百人,每闻有一美女,即娶以为妇;此王曾有恶行,请为君等述之。其弟有一美妇,国王知之,强夺占为己有。其弟贤明,不与之争。此国王有子女甚众。

国王并有侍臣数人,随侍左右,与之并骑而出,彼等在国中权势甚重,名称“君主之忠臣”。国王若死,依俗应焚尸,焚时,诸侍臣皆自投入火而死。据云彼等既随侍于生前,应亦随侍于死后。国王既死,诸子无敢动其宝藏者。据云父王既然聚此宝藏,我辈亦应为相类之聚集。由是此国之中有一极大宝藏。

此国不养马,因是用其大部分财富以购马,兹请述其购取之法。应知怯失(Kais)、忽鲁模思(Ormuz)、祖法儿(Dhafar)、琐哈儿(Sohar)、阿丹(Aden)诸城之商人屯聚多马,其他数国数州之人亦然,由是运输入此国王及其他四兄弟之国。一马售价至少值金五百量(poids),合银百马克(marc)有余,而每年所售甚众也。国王每年购人二千余匹,其四兄弟之为国王者购马之数称是。每年购马如是之众者,盖因所购之马不到年终即死,彼等不知养马,而且国中无蹄铁工人也。售马之商人不愿失其每年售马之利,运马来时,不携蹄铁工人俱来,缘是每年获利甚巨。其马皆用船舶从海上运载而来。

此国有一习俗,请为君等言之。设有一人犯罪,被判死刑,其人若云愿为某神之牺牲而自杀,官辄许之。于是其亲友取其人置车上,给刀十二柄,游行全城,唱言曰:“此勇敢之人将为某神牺牲而自杀。”及至自杀之所,其人取一刀穿其臂,呼曰:“我为某神牺牲。”旋取第二刀穿别一臂,又取第三刀洞其腹,如是历取诸刀自剌而至于死。既死,诸亲属皆大欢喜,取其尸焚之。脱其人多妻,死后火葬时,诸妻亦皆自投于火而死。凡妇女之为此者,皆受人称赞。

彼等是偶像教徒,多崇拜牛,据云牛是有益之畜,彼等不食牛肉,亦不伤害之。但有某种阶级之人名果维(Govy)者愿食牛肉,然不敢杀之。牛之自死者或者因他故死者,彼等则食其肉。

应知土人皆用牛脂涂其居宅,无问贵贱,甚至国王大臣,仅坐于此。据说地是最尊荣之物,盖吾人皆是以土做成,而死后应归于土,由是敬土而不敢慢。应知此果维族有一特征,无论如何,不敢人圣多玛斯(Saint Thomas)之墓室,盖此圣者遗体现在此马八儿州之一城中也。虽用二三十人强执一果维人往,仍不能强其留在此耶稣宗徒葬身之处。盖因此族之祖先曾杀圣多玛斯,而此圣者之神力不许果维人莅此,其事后此言之。

应知此州除米之外不产他谷。尚有异事,即此州之人无论用何方法不能养马,屡试皆然。纵有良种与牝马配合,只产小驹,蹄曲而不能骑。

此国之人裸体而战,仅持一矛一盾,然其为战士而尚慈悲。彼等不杀禽兽,并不杀具有灵魂之物,至所食之畜,则由回教徒或非本教之人杀之。男女每日浴二次,不守此习者,视同无信心之人。犯罪者罚甚重,而禁饮酒,凡饮酒及航海者,不许为保证人,据说只有失望之人才作海行。彼等不以淫逸为罪恶。

应知其地有时奇热。每年只有6月、7月、8月三个月有雨。脱此三月无雨使气候清凉,则其干燥将不可耐。

此地颇有名称“相者”(physionomie)之术人,能知人之性情地位。脱有询之者,此辈立时可以答复,遇一鸟一兽,此辈可以解释其义,盖此国重视预兆,甚于他国也。设有一人出行闻一鸟鸣,如认为吉兆,则仍前行;如认为不吉,则或暂时停留,或遵来途而返。婴儿诞生,此辈记录其年月日时。盖行为皆遵迷信,而此辈颇谙魔术、巫术及他种妖法也。

此国及印度全境之鸟兽,种类颇异,与我国所产者完全不同,仅有鹑类(caille)相同,其他迥异。此国有鸟夜飞,名称蝙蝠,大如苍鹰(autour)。苍鹰色黑如乌,较吾国所产者为大,颇善猎捕。尚应知者,彼等用饭和肉并其他熟食以饲马,此国之马尽死之故如此。

男女偶像皆有侍女,乃信仰此偶像之父母所献。某寺庙之僧众对于偶像举行庆贺之时,则召集一切献女。诸女既至,在神前歌舞,已而献馔神前。久之撤馔,谓偶像食毕,乃自食之。每年如是数次,诸女迄于婚后始止。

马八儿州中此国之事既已备述如前,兹请暂不接述州中他国,盖其风俗应叙述者尚多也。





168 使徒圣多玛斯遗体及其灵异

圣多玛斯(Saint Thomas)教长之遗体,在此马八儿(Maabar)州中一人烟甚少之小城内,其地偏僻,商人至此城者甚稀。然基督教徒及回教徒常至此城巡礼,回教徒对之礼奉甚至,谓之为回教大预言人之一,而名之曰阿瓦连(Avarian),法兰西语犹言圣人也。基督教徒至此城巡礼者,在此圣者被害之处取土,使患四日热或三日热之病人服之,赖有天主及此圣者之佑,其疾立愈。基督诞生后1288年时,此城有一极大灵异,请为君等述之。

此地有一藩主,屯米甚多,皆屯于礼拜堂周围之诸房屋中。看守礼拜堂之基督教徒忧甚,盖诸房屋既尽屯积米粮,巡礼人不复有息宿之所,数请于藩主,请空屯米之屋,而藩主不从。某夜圣者见形,手持一杖置于藩主之口,而语之曰:“脱汝不空余屋,俾巡礼之人得以息宿,汝将不得善终。”

及曙,藩主畏死,立将所屯之米运出,并将圣者见形之事告人。基督教徒对此灵异大为庆幸,皆感天主及使徒圣多玛斯之恩。尚有其他大灵异屡屡发生,如疾病、残废及种种病苦之获痊愈之类,尤以对于基督教徒最为灵验。

看守礼拜堂之基督教士所言圣者死事,兹为君等述之。据说圣者昔在林中隐庐祷颂,周围孔雀甚多,盖他处孔雀之众无逾此地者也。此地有一偶像教徒,属于上述之果维(Govis)族者,持弓矢猎取圣者左右之孔雀,发矢射雀,误中圣者之身,圣者立死。圣者死前曾传道奴比亚(Nubie)之地,土人皈依耶稣基督之教者,为数甚众。

其地儿童产生,体色尽黑,色愈黑者,愈为人所重;产生以后,每星期中,人用芝麻油涂擦其身,因是色黑如同魔鬼。此辈之神亦黑色,魔鬼则为白色,故所绘圣像皆黑色也。

此辈奉牛如同圣物,其出战也,取野牛之毛系于马颈。若步战,则系毛于盾,或系毛于头发上,因是牛毛之价甚贵。出战之人无此牛毛则不自安,缘其人以为系有牛毛,战后必然安然无恙。

既述此马八儿州之要事毕,请离此他适,而言木夫梯里(Muftili)国如下文。





169 木夫梯里国

若从马八儿发足,北行约千里,则至木夫梯里(Muftili)国。此国昔属一王,惟自王死后,四十年间,由其王后治理,王后爱王甚切,不愿改嫁他人。王后在此四十年间,善治其国,尤甚于国王在世之时,盖其爱好法律正义平和,人皆爱戴也。

人民是偶像教徒,不纳贡赋于何国,食肉、米与乳。此国出产金刚石,采之之法如下:境内多有高山,冬降大雨,水自诸山流下,其声甚大,构成大溪。雨过山水流下之后,人往溪底寻求金刚石,所获甚多。及至夏季,日光甚烈,山中奇热,登山甚难,盖至是山无水也。人在此季登山者,可得金刚石无算。山中奇热,由是大蛇及其他毒虫颇众。人在山中见有世界最毒之蛇,往者屡为所食。

如是诸山尚有山谷,既深且大,无人能下。往取金刚石之人掷最瘦之肉块于谷中。山中颇有白鹫,以蛇为食,及见肉掷谷中,用爪攫取,飞上岩石食之。取金刚石之人伏于其处者,立即捕而取其所攫之肉,可见其上沾结谷中金刚石全满,盖深谷之中金刚石多至不可思议。然人不能降至谷底,且有蛇甚众,降者立被吞食。

尚有别法觅取金刚石。山中多有鹫巢,人往巢中鹫粪内觅之,亦可获取不少,盖鹫食人掷谷底之肉,粪石而出也。彼等捕鹫时,亦可剖腹求之,可得石无算,其石甚巨。携来吾国之石乃是选择之余,盖金刚石之佳者以及大石、大珠,皆献大汗及世界各国之君王,而彼等独据有世界之大宝也。

应知世界诸国除此木夫梯里国之外,皆不出产金刚石。此国亦制世界质最精良之硬布(bougran),其价甚巨。亦产世界最大之羊,生活必需之物悉皆丰饶。

此外无足言者,此后请言婆罗门所在之剌儿(Lar)州。





170 婆罗门所在之剌儿州

剌儿是延向西方之一州。若从圣多玛斯遗体所在之地发足,即可立入此州,世界之一切婆罗门皆从此州而来。

应知此种婆罗门乃是世界最优良诚实之商人,盖其无论如何不作伪言也。彼等不食肉,不饮酒,而持身正直;除与妻交外,不与其他妇女交;不窃他人物;法律欲其如此也。彼等皆挂一棉线于胸前肩后,俾为人识。

彼等有一富强国王,乐购巨价之宝石大珠,王遣此种婆罗门商人赴全世界求取所能得之一切珠宝而归。彼等以珠宝献王,王倍给其价,王用此法,遂有一极大宝藏。

此种婆罗门是偶像教徒,重视先兆及命数甚于他地。每星期中逐日有一特征。婆罗门早穿衣时必视其影,若见日下之影有必须之长度,则立订交易;脱其影不及必须之长度,则在此日不作何等交易。若在一旅舍订结交易时,见一蜘蛛行于墙上,所行之方向若视为吉,则立订交易;方向视为不吉则否。出门时若闻一人喷嚏,视若吉则行;视若凶兆则坐于地,过其认为必须之时始起。若在道上见一燕飞,视飞向吉则行;否则归。由是其迷信较之吾国异教徒为甚。彼等食少而大有节制,故得长年。彼等从不放血,亦不任人取滴血出。

有一种人名曰浊肌(Gioghid),亦属婆罗门,然构成一种祀神之教派。彼等寿甚长,有至一百五十至二百岁者。彼等食甚少,仅食良食,尤食肉、米及乳。此种浊肌尚服一种奇特饮料,合水银、硫黄而饮之。彼等以为饮此可以长寿,每月服二次,自童年时即如此也。

此教持身严肃为世界最。彼等裸体而敬奉牛。其人多系一牛像于额,其质或用黄金,或用黄铜,或用青铜。彼等焚牛粪成灰,用此灰做膏,涂擦身体。

其食也不用钵,亦不用盘,只用天堂果树(pommier de paradis)之叶,或其他大叶盛之,但须其叶为干叶而非青叶。据说色尚青者必有灵魂,用之必有罪过。彼等宁死而不愿违戒而用一犯过之物。或有询之者,缘何裸体而不顾羞耻,答曰:“吾辈裸体出生,而不欲此世何物,是以裸体。加之吾辈正直无过,吾辈不以阴茎犯过,吾辈可以之示人与其他肢体同。然汝辈犯淫罪,汝辈引以为羞,故掩蔽之。”

彼等不杀生,虽虱、蝇及任何生物亦然,据说此种生物皆有灵魂,杀之有罪。彼等不食颜色尚青之物,必俟其干。裸卧于地,上不用被,下不用褥。彼等不尽死亡,其事甚奇。终日持斋,仅日日饮水。其收录徒弟时,纳之彼等庙宇中,使之持同一生活。然后试之,召前述祀神之室女来,命诸女触之,吻之,抚摩之。设其阴茎不动则留,茎动则逐出。据说彼等不愿与一淫人共处也。

其为偶像教徒也,残忍不义,俨若鬼魔。据说彼等焚死者之尸者,盖若焚尸,则食尸之虫不生,脱任其生虫,虫终将缺食而死,由是死者之灵魂有大罪而受大罚。彼等焚尸之理由如此。

马八儿州人之事及其风习,今已言其多半,兹将叙述此马八儿州之他事,请言一名加异勒(Cail、Kayalpatnam)之城。





171 加异勒城

加异勒是一名贵大城,隶属阿恰儿(Aciar),五兄弟国王中之长兄也。凡船舶自西方,质言之,自怯失(Kais)、忽鲁模思(Ormuz)、阿丹(Aden)及阿拉伯全境,运载马匹及其他货物面来者,皆停泊于此。职是之故,附近诸地之人皆辐辏于此,而使此加异勒,城商业繁盛。

国王据有宝石甚众,身戴宝石不少。彼生活优裕,而善治其国。颇喜商贾及外国人,故人皆乐至此城。

国王有妻三百人,盖此国人男子妻愈多而声望愈重。

此马八儿州有国王五人是亲兄弟,前已言之,而此国王即是五兄弟之一人。彼等之母尚存。设若彼等失和,彼此争战时,其母即居中阻之,不听其斗。如仍欲斗,其母则手持一刀而语诸子曰:将割乳哺汝等之乳房,然后剖腹而死于汝等之前。因是数使诸子言归于好。但在母死之后,彼等恐将互相残害也。

兹置此国王不言,请言俱蓝(Coilum、Quilon)国。





172 俱蓝国

若从马八儿发足,西南行五百哩,则抵俱蓝国。居民是偶像教徒,基督教徒甚少。彼等自有其语言,自有其国王,而不纳贡赋于何国。

出产苏木(bresil)甚多,名称俱蓝苏木,盖以产地名也,其质甚细。产姜甚良,名称俱蓝姜,亦以产地名也。全国出产胡椒甚多,土人种植胡椒树,5月、6月、7月中采之。亦饶有蓝靛甚细,太阳极烈,草受曝而产蓝。盖此国热不可耐,若浸鸡蛋于溪水中,阳光曝之立熟。

蛮子、地中海东(Levant)、阿拉伯诸地之商人乘舟载货来此,获取大利。

此国饶有种种牲畜,与他国种类迥异。狮子尽黑色,其一例也。鹦鹉种类甚多,有身白如雪而爪喙红者,有朱色者,有蓝色者最为美观,有小者亦美,其他皆绿色。

亦有孔雀甚美,较吾人之孔雀为大,种类亦殊。其鸡最美而最良,种类亦异。其果实亦甚奇,斯皆因酷热使然。

彼等除米外无他谷。用椰糖造酒,颇易醉人。凡适于人身之物,悉皆丰饶,价值甚贱。其星者甚良,医师亦然。人皆黑色,妇孺亦然,尽裸体,仅以美丽之布一片遮其丑处。不以淫乱为罪过,可以从姊妹为妻,兄弟死可以妻其嫂娣。此俗遍及印度全境。

此外无足言者。吾人离此,请言一名戈马利(Comary)之地。





173 戈马利地方

戈马利是印度境内之一地,自苏门答剌至此,今不能见之北极星,可在是处微见之。如欲见之,应在海中前行至少三十哩,约可在一肘高度上见之。此地是一蛮野之地,有种种兽畜,尤有猿甚奇,不知者误识为人。尚有名称加特保罗(Gat Paul)之猿,一种可注意之种类也。





174 下里国

下里(Ely)是西向之一国,距戈马利约三百哩。居民是偶像教徒,自有国王而不纳贡赋于何国,彼等自有其语言。吾人至是进入较为熟识之地,行将确实叙述各国之民风、土产,而君等亦将闻之较审,盖吾人行抵更近之地也。

此州无港,然颇有大河,河口既宽且深,盖良河口也。土产胡椒、生姜及其他香料甚饶。其国王富于宝货,然无强兵。顾其国据有天险,无人可能侵入,所以有恃无恐。应知船舶之赴他处而抵此国海港者,国人辄尽夺船中所载之物,而语之曰:“汝曹欲赴他处,然汝辈之神导汝辈至此,由是汝曹之物应属吾曹。”彼等视此钞暴不为过失。第若船舶径来此国,则以礼待而保护之。此种恶习,遍及印度全境,脱有船舶赴他处而因风暴在途中被难者,辄受钞掠。

蛮子及他国船舶夏季来此,卸载货物六日或八日即行,盖此地除河口外无海港,质言之,仅有沙滩、沙礁可庇也。蛮子船舶有木锚甚大,置之沙滩,颇多危险。

其地颇有虎及其他猛兽甚恶,亦饶有披毛带羽之野味。

此外无足言者,请言马里八儿(Melibar)国。





175 马里八儿国

马里八儿是一大国,国境延向西方。居民自有语言,而为偶像教徒,彼等自有国王而不纳贡赋于何国。在此国中,看见北极星更为清晰,可在水平面二肘上见之。应知此马里八儿国及一名称胡茶辣(Guzarat)之别国,每年有盗船百余出海,钞掠船舶,全夏皆处海中,携带妇孺与俱。此种盗船每二三十船为一队,每距五六哩以一队守之,在海上致成一线,凡商船经过,无得脱者。盗船每见一帆,即举火或烟为信号,由是诸船皆集,群向来船,捕之而尽夺商人之物。然后释之而语之曰:“复往求利,将重为吾辈所得也。”然自是以后,诸商知自防,再赴海者必载大船,携兵器人员与俱,除有时遭难外,不复畏海盗也。

此国出产胡椒、生姜、肉桂、图儿比特(turbith)、椰子甚多。纺织古里布(calicot)甚精美。船舶自极东来者,载铜以代沙石。运售之货有金锦、绸缎、金银、丁香及其他细货香料,售后就地购买所欲之物而归。此国输出之粗货香料,泰半多运往蛮子大州,别一部分则由商船西运至阿丹,复由阿丹转运至埃及之亚历山大(Alexandrie),然其额不及运往极东者十分之一,此事颇可注意也。

既述马里八儿国毕,请接述胡茶辣国如下文。但应知者,如是诸国仅志都城,其他城堡甚众,言之冗长,故略。





176 胡茶辣国

胡茶辣是一大国,居民是偶像教徒,自有其语言。彼等有一国王而不纳贡赋于何国。国境延至西方,至是观北极星更审,盖其出现于约有六肘的高度之上也。

彼等是世界最大的海盗,有一恶俗,请为君等述之。彼等夺一商船时,强使商人饮一种名称罗望子(tamarin)之汁,俾其尽泻腹中之物,盖商人被擒时得将重价的珍珠、宝石吞于腹中,用此方法,海盗可以尽得其物也。

此胡茶辣州饶有胡椒、生姜、蓝靛,亦多有棉花。产棉之树高有六步,生长可达二十年。然若年岁如是之老,所产之棉则不适于纺织,只作他用。

此国制作种种皮革,如山羊、黄牛、水牛、野牛、犀牛及其他诸兽之皮是已。所制甚多,故每年运载皮革赴阿拉伯及他国之船舶,为数甚众。其国亦制最美之红皮,嵌极美之鸟兽于其中,用金银线巧缝之,其美不可思议,有值银六马克(marc)者。

此外无足言者,故于此后接言一名塔纳(Tana)之国。





177 塔纳国

塔纳是一大国,位置在西,面积与价值并大。居民是偶像教徒而自有其语言,自有国王而不纳贡赋于何国。国内不产胡椒或其他香料,然饶有乳香,其色褐,交易甚盛。制造皮革甚多,并纺织美丽毛布。

此国颇有海盗,与国王同谋钞掠商人。此种海盗与其国王约,得马则属国王,得他物则属彼等。国王无马而须运多马至印度,故其行为如此,凡船舶之赴此国者,莫不运马及其他不少货物。此种恶习颇足贻国王羞。

此外无足言者,此后请言一名坎巴夷替(Cambaet)之国。





178 坎巴夷替国

坎巴夷替是一大国,位置更西,居民是偶像教徒,而自有其语言,自有国王而不纳贡赋于何国。在此国中,所见北极星更明,盖愈向西行,星位更高也。此国商业繁盛,蓝靛甚佳,出产甚饶,纺织细毛布甚多,亦饶有棉花,输往不少地域。制作皮革甚佳,贸易亦盛。此国无海盗,居民皆良民,恃工商为活。

此外无足言者,兹请言别一名称须门那(Semenat)之国。





179 须门那国

须门那是更西之一大国。居民是偶像教徒,自有其国王及语言,不纳贡赋于何国,而恃工商为活。国人中无作海盗者,工业茂盛。彼等洵为残忍的偶像教徒。

此外无足言者,兹请言一名称克思马可兰(Kesmacoran)之别国。





180 克思马可兰国

克思马可兰乃是一国,自有国王及语言。居民是偶像教徒而恃工商为活,人多业商,而从海陆运输其商货于各地。食肉、米及乳。

此外无足言者。但往西行及往西北行,此克思马可兰国则是印度最末之一州。自马八儿(Maabar)迄于此州并属大印度境,而为世界之良土。此大印度经吾人叙述者,仅为沿海之城国,至若内地城国概未之及,盖言之殊冗长也。

兹从此地首途,请言尚属印度之若干岛屿。首言二岛,一名男岛,一名女岛。





181 独居男子之男岛及独居女子之女岛

若从此陆地之克思马可兰国首途,向南海行约五百哩,则抵二岛,一名男岛,一名女岛。两岛相距约三十哩。居民皆是曾经受洗之基督教徒,然保存旧约书之风习:妻受孕时,其夫不与接触;妻若生女,产后四十日亦不与接触。

名称男岛之岛,一切男子居处其中。每年第三月,诸男子尽赴女岛,居三月,是为每年之3月、4月、5月,在此三个月中与诸女欢处。逾三月,诸男重回本岛,其余九个月中,则为种植工作贸易等事。

此岛有龙涎香甚佳。居民食肉乳及米。彼等善渔,获鱼甚多,干之以供全年之食,余者售之来岛之商人。岛民无君主,服从一主教,而此主教隶属一大主教。此大主教居在别岛,其岛名称速可亦剌(Scoira),别详后章。彼等亦自有其语言。

彼等与诸妇所产之子女,女则属母,男则由母抚养至十四岁,然后遣归父所。此二岛之风习如此。诸妇除抚养子女、摘取本岛之果实外,不作他事,必须之物则由男子供给之。

此外无足言者,兹请言名称速可亦剌之别岛。





182 速可亦剌岛

从此二岛首途,南行约五百哩,则见速可亦剌岛。居民皆是已受洗礼之基督教徒,而有一大主教。彼等多收龙涎香,饶有棉布,并有其他货物,尤多大而良之咸鱼。彼等食米、肉及乳,不获何种谷类。人尽裸体,与其他印度人同。

岛中商业茂盛,盖各处船舶运载种种货物,来此售于岛民,在岛购买黄金,而获大利。凡船舶之赴阿丹(Aden)者皆泊此岛。

此大主教不属罗马教皇,而隶驻在巴格达(Bagdad)之聂思脱里派基督教徒之总主教。此总主教统辖此岛及其他数地之大主教,与我辈教皇同。

颇有海盗来此岛中,陈售其所掠之物,而售之极易,盖此岛之基督教徒明知物属回教徒或偶像教徒,乐为购取也。

并应知者,世界最良之巫师即在此岛。大主教固尽其所能禁止此辈做术,然此辈辄言祖宗业已如此,我辈特效祖宗所为耳。此辈巫术,请言一事以例之。如有船舶乘顺风张帆而行者,此辈能咒起逆风,使船舶退后。彼等咒起风云,惟意所欲,可使天气晴和,亦可使风暴大起。有其他巫术,不宜在本书著录也。

此外无足言者,请前行,述一名称马达伽思迦儿(Made-isgascar)之岛。





183 马达伽思迦儿岛

马达伽思迦儿是向南之一岛,距速可亦剌至少有千哩。居民是回教徒而崇拜摩诃末,人谓有四老人治理此岛。应知是岛伟美而为世界最大岛屿之一,盖其周围有四千哩也。居民恃工商为活。

我敢断言此岛象数之众,世界他州无能及者,后此叙述别一岛屿名僧祗拔儿(Zanquibar)者,情形亦同。缘此二岛,象业贸易之盛,竟至不可思议。

此岛除骆驼肉外不食他肉,逐日宰驼之多,未目击者必不信有此事。据谓是为世上最良而最卫生之肉,是以日日食之。

此岛紫檀树颇繁殖,致使林中无他木材,彼等多有龙涎香,盖其海中多鲸,而捕取者众;并多大头鲸,是为极大之鱼,饶有龙涎,与鲸鱼同。岛中有豹、熊、虎及其他野兽甚众。商人载大舟来此贸易而获大利者,为数不少。

应知此岛位置甚南,致使船舶不能在同一方向更作远行,而赴其他诸岛,只能止于此马达伽思迦儿岛及后此著录之僧祗拔儿岛。其故则在海流永向南流,其流之急,船舶更作远行者,不复能归。

马八儿船舶之莅此马达伽思迦儿岛及僧祗拔儿岛者,航行奇速,路程虽远,二十日可至。但在归途则需时三月,盖水向南流,归时须逆流而上。年中无论何季,海水常向南流,其流之急,洵不可思议。

人言位置更南之他岛,因海流阻碍船舶之归,故船舶皆不敢往。其地有巨鸟,每年一定季候中可以见之。然闻人言,此种巨鸟与我辈史籍著录者异,据曾至其岛身亲目击者告马可·波罗阁下之言,鸟形与鹫同,然其躯绝大,据说其翼广三十步,其羽长逾十二步。此鸟力大,能以爪搏象高飞,然后掷象于地,飞下食之。岛人名此鸟曰罗克(Rock),别无他名。未识此鸟诚为鹫首狮身之鸟(griffon),抑是别种相类大鸟。然我敢断言其形不类吾人传说半狮半鸟之形,其躯虽大,完全类鹫。

大汗曾遣使至此山中采访异闻,往者以其事归报。先是大汗遣使臣往,被久留岛中,此次遣使,亦为救前使归也。使臣归后,将此异岛之诸异闻,陈告大汗,并及此鸟。彼等并献野猪齿二枚,齿甚大,每枚重逾十四磅,则生长此齿之野猪,形体之大可知。据称其体之大如大水牛。其地亦有麒麟(girafe)、野驴甚众,奇形异状的野兽之多,竟至不可思议。

此外无足言者,请接述僧祗拔儿岛。





184 僧祗拔儿岛

僧祗拔儿(Zanquibar)是一大岛,周围约有两千哩。居民是偶像教徒,自有国王及语言,而不纳贡赋于何国。其人长大肥硕,然长与肥不相称。其长大类似巨人,其力强可载四人负载之物,可兼五人之食。体皆黑色,裸无衣服,仅遮其丑处而已。卷发黑如胡椒。口大,鼻端上曲,唇厚,眼大而红,俨同鬼魔,丑恶之甚,世上可怖之物,似无逾于此者。

此地产象甚多,其多竟成奇观。有狮黑色,与我辈狮种异,亦有熊、豹不少。羊色皆同,头黑而体全白,别无他种。亦多有麒麟,颇美观。

兹请言关于象之一事。应知牡象与牝象交时,掘地作大坑,牝象仰卧坑中,牡象卧其上,与男女交合无异,是盖因牝象丑处生在腹下也。

此岛之女子,是世上最丑陋之女子,其乳房大逾他处女子四倍。居民食米、肉、乳及海枣。彼等用海枣、米及若干好香料作酒,兼亦用糖。其地商业颇茂盛,商人及大舶来此者颇多。然岛中重要商品则为象牙,岛中饶有之。近海多鲸,故龙涎香亦甚饶。

尚应知者,彼等是良战士,勇于斗而不畏死。彼等无马,然乘驼、象而斗。象背置木楼,足容十人至十六人,人处其中持矛剑及石而斗;由是处象上者颇善斗,盖其有木楼也。人无甲胄,仅有盾与矛、剑,由是互相屠杀。当其率象而战之时,以酒饮象使之半醉,盖象饮酒后较傲勇,战时更为出力。

此外无足言者,是以此后将言阿巴西(Abbasie)大州,是为中印度。言此以前,请先概述关涉印度之事。

应知吾人所述印度诸岛,仅就其中最名贵之州国言之,盖能备述印度一切岛屿者,世无其人也。故我之叙述,仅及精华,至所遗之其他岛屿,尽隶上述诸州国也。据熟悉海行的水手所用之图籍,此大海中有已识之岛一万二千七百,而人不能至的未识之岛尚未计焉,此一万二千七百岛皆有人居。诸岛之中,有面积广大无限者,如君等前此之所闻。此海水手所言如此,彼等知之甚审,盖彼等日日只作航行也。

大印度境始马八儿迄克思马可兰,凡有大国十三,吾人仅述十国而遗其三,诸国尽在大陆。

小印度境始爪哇州迄木夫梯里国,凡八国,并在陆地。

应知此种国家尽在陆地,盖诸岛国为数甚多,不在此数之内,如前所述也。





185 陆地名称中印度之阿巴西大州

阿巴西(Abbasie)是一大州,君等应知其为中印度而属大陆。境内有六国国王,六国皆甚大,此六王中有基督教徒三人,回教徒三人,最大国王是基督教徒,余五王并隶属之。

此国之基督教徒面上并有三种记号,一自额达于鼻中,别二记在两颊。此种记号用铁烙于面,表示其已受洗,盖彼等受水洗后立烙此记,或表示其忠顺,或表示其洗礼之完成也。此国亦有犹太教徒,两颊各有记。至若回教徒之记号,仅自额达于鼻中。

国之大王驻于国之中央,诸回教徒居近阿丹(Aden Adel)。圣多玛斯曾在此州传教,俟其皈依后,乃赴马八儿州而殁于彼。其遗体即在彼处,前已言之也。

应知彼等是最良战士而乘马,盖国内多马也。彼等日与阿丹之算端(sultan)战,并与奴比亚(Nubie)人战,且与其他不少部落战,此诚有其必要也。兹请述一美事,事出基督降世之1288年。

此基督教国王而为阿巴西州之君主者,曾言欲赴耶路撒冷(Jérusalem)朝拜耶稣基督圣主之墓,诸男爵以道途危险,谏止之,劝其遣一主教或别一在教高级职员代往。国王从之,乃遣一持身如同圣者之主教某前往巡礼。此主教经行海陆而抵圣墓,礼之如一基督教徒之所应为,代其主呈献一极大供品。诸事既毕,遂就归途,而抵阿丹。阿丹算端闻其为基督教徒主教,兼是阿巴西大国王之使臣,拘之,询其是否为基督教徒,主教据实以对。于是算端命其改从回教,否则将使其大受耻辱。主教答言宁死而不背其造物主。

算端闻言甚恚,命人割其茎皮。人遂依回教俗割之,割毕算端语云:“轻其王故辱其使臣。”已而释之归。

主教受耻辱后,心中大悲痛。然私衷自慰,既为保持我辈救世主耶稣基督之戒律而受辱,于灵魂之救赎必有大功。

创愈后,自此循海遵陆而还抵阿巴西国王所。国王见之甚欢,大款待之,然后询以圣墓之事,主教据实以对,国王因是信奉愈切。主教述耶路撒冷之事毕,然后述阿丹算端轻其王而加辱于彼事。国王闻之既恚且痛,痛恼之深,几濒于死,终呼曰:“若不大复此仇,决不为王治国。”呼声之大,左右尽闻。

国王立命其全军步骑备战,并遣多数负木楼之战象至军中。诸事筹备既毕,遂率此重大军队出发,进向阿丹国境。算端闻此国王来侵,亦率其极众之军队进至国境最坚固之要道上,以阻敌军之人。国王率众至坚固要道时,回教徒已待于此矣。由是杀人流血之鏖战开始,盖双方皆残忍也。最后因我辈救世主耶稣基督之意,回教徒不能抵抗基督教徒,盖其作战不及基督教徒之优也。回教徒败走,死者无算。阿巴西国王率其全军攻入阿丹国内。回教徒屡在狭道上拒之,迄未成功,辄遭败亡。国王留驻月余,残破其敌人之国,每见回教徒即杀,毁其田亩,迨杀戮已众而其耻已雪,遂欲还国,盖其至是可载大誉而归。纵欲久留,亦不能再使敌人受创,盖因敌拒守险隘之地,道狭颇难攻人。由是国王自阿丹敌国率军出发,载荣誉欢心而还本国。国王及其主教所受之耻既雪,回教徒死伤之众,田亩毁坏之多,其事诚可惊也。此事颇为重大,盖基督教徒认为不应败于回教徒之手也。

兹既述此事毕,对于此州尚有言者。此州一切食粮皆甚富饶,居民食肉、米、乳及芝麻。多象,然不产于本地,而来自别印度之岛屿。亦多麒麟,产自此国。又见有熊、豹、狮子及其他种种异兽甚众。多有野驴,及最美观之母鸡,并有不少其他种类禽鸟。有鸵鸟,鲜有小于驴者,并有鹦鹉甚美,并颇有异猫及猴。

此阿巴西州中城村甚众,亦多有商人,盖其境内商业繁盛也。其地制造极美之硬布及其他棉布。

此外无足言者,是以后此接述阿丹州。





186 阿丹州

应知在此阿丹(Aden)州中,有一君主名称算端(sultan)。居民是回教徒,崇拜摩诃末,极恨基督教徒。国中有环以墙垣之城村甚众。

阿丹有海港,多有船舶自印度装载货物而抵于此。商人由此港用小船运载货物,航行七日,起货登岸,用骆驼运载,陆行三十日,抵尼罗(Nil)河,复由河运至亚历山大(Alexandrie)。由是亚历山大之回教徒用此阿丹一道输入胡椒及其他香料,盖供给亚历山大物品之道途,别无便利稳妥于此者也。

阿丹算端对于运输种种货物往来印度之船舶,征收赋税甚巨。对于输出货物亦征赋税,盖从阿丹运往印度之战马、常马及配以双鞍之巨马,为数甚众也。印度马价甚贵,贩马而往者获利甚厚,缘印度不养一马,前已言之也。

每一战马在印度售价可值银百马克(marc)有余。由是此阿丹算端对于其海港运输之一切货物征取一种重大收入,人谓其为世界最富君主之一。

开罗(Caire)算端前此攻取阿迦(Acre)城时,阿丹算端曾以骑士三万人、骆驼四万余匹往助,回教徒因获大益,基督教徒因受大害。其为此者,与其谓向埃及算端表示友好,勿宁谓恨基督教徒有以致之,缘彼等亦互相怨恨也。

兹置此阿丹算端不言,请言隶属算端之一城,城名爱舍儿(Escier),位在西北,自有一王。





187 爱舍儿城

爱舍儿(Escier)城甚大,位在阿丹港西北,相距四百哩。其王隶属阿丹算端,善治其地,辖有城堡数所。居民是回教徒。境内有一良港,由是自印度运载不少商货之船舶咸莅于此。饶有白色乳香,国主获利甚巨。土人只售之于国主,不敢售之于他人。国主每石(quintal)购价金镑十枚,而售价则为六十枚,因是获利甚巨。

所产海枣亦多。除米外不产他谷,所产之米且甚少,而由各处输入者多,输入者因获大利。饶有鱼类,就中有一种大鱼,产鱼之多,每威尼斯银钱一枚可购大鱼两尾。居民食肉、米、乳、鱼,无葡萄酒,然用糖、米、海枣酿酒,味甚佳。

应知其羊皆缺耳,生耳之处有一小角,是为美丽之小畜。

尚应言者,土产之一切牲畜,包括马、牛、骆驼在内,只食小鱼,不食他物。食物仅限于此,盖此地境内毫无青草,乃世界最干燥之地。牲畜所食之鱼甚小,每年3月、4月、5月捕取,所获奇多。然后干而藏之于家,以供牲畜全年之食。渔人且以活鱼饲牲畜,鱼出水时即以饲之。此外尚有他鱼,大而良,价甚贱,切之为块,曝干之,然后藏之于家,全年食之,如同饼饵。

此外无足言者,此后请言一名祖法儿(Dufar)之城。





188 祖法儿城

祖法儿是一名贵大城,在爱舍儿之西北,相距有五百哩。居民是回教徒,有一国主隶属阿丹算端。城在海上,有一良港,位置甚佳,颇有船舶往来印度。商人运输多数战马于印度而获大利。此城辖有不少城堡。

此地有白乳香甚多,兹请言其出产之法。境内有树木颇类小杉,人用刀剌破数处,乳香从剌处流出。有时不用刀剌而自流出,盖因其地酷热所致也。

此外无足言者,是以离此,请言哈剌图(Calatu)湾,并及哈剌图城。





189 哈剌图湾及哈剌图城

哈剌图(Calatu)是一大城,在一名哈剌图之海湾内。城在海岸,距祖法儿约六百哩,居其西北。居民是回教徒而隶忽鲁模思(Ormuz)。忽鲁模思国王每与别一势力更强之国王争战时,辄莅此哈剌图城,缘此城地势良而防守坚也。

不产谷食,而取之于他国,盖商人用船舶载谷而至也。海港大而良,船舶由印度运不少商货来此,然后由此城贩往环有墙垣的村城数处。亦从此港运输阿拉伯种良马至印度,其数甚众。应知此城及前此著录之其他诸城,每年运往诸岛之马匹多至不可思议,其故盖在诸岛之中不畜一马。此外马至诸岛后不久即死,缘诸岛之人不善养育马匹,以熟粮及其他诸物饲之,诚如前述,而且其地无蹄铁工人也。

此哈剌图城位在一湾口(Oman湾),若无哈剌图国王之许可,凡船舶皆不能出入。此哈剌图国王同时为忽鲁模思国王,并为起儿漫(Kerman)算端之藩臣。若畏其主起儿漫算端时,则赴哈剌图,而不容湾中停留船舶,因是起儿漫算端受害甚巨,盖其丧失印度等国商人人境之税课也。平时载货之商船莅此者甚众,所课税额甚高,最后起儿漫算端势须顺从忽鲁模思国王之意。此国王尚有堡寨一所,更较哈剌图城为强,控制湾口尤力。

此地人民以海枣、咸鱼为粮,所藏无算,然君主所食则较优也。

此外无足言者,吾人前行,请言前此业已叙述之忽鲁模思城。





190 前已叙述之忽鲁模思城

若自哈剌图城首途,在北方及东北方中间行三百哩,则至忽鲁模思(Ormuz)城,城在海边,一大而名贵之城也。其城有一蔑里(Melic),此言国王。居民臣属起儿漫(Kerman)算端,是回教徒,而辖有不少城堡。其地甚热,所以居宅皆置通风器以迎风。此种通风器置于风来处,使风入室而取凉,否则酷热,人不能耐。

此外别无所言,盖关于此忽鲁模思城并起儿漫之事,前已次第述之也。兹特因往来殊途,重回斯地,复再言及而已。

自是以后,吾人离此,将言大突厥国(Grande Turquie)。然尚有漏述之事,应补志于此。盖若从哈剌图城首途,在西方及西北方中间行五百哩,可抵怯失(Kais)城,然吾人无暇叙述此城,仅能在此做简单之记录,而接述大突厥如下文。





191 大突厥

大突厥(Grande Turquie)境内有一国王名称海都(Caidou)。其人是大汗侄,盖其为察合台(Djagatai)子,而察合台为大汗之亲兄也。彼是大君主而有城堡甚众,亦是鞑靼,与其部众同。部众皆善战之士,缘其常在战中也。此国王海都从未与其叔大汗和好,常与之战,曾与大汗军屡作大战。失和之故盖因海都父之略地应属于海都者,海都曾索之于大汗,就中有契丹(Cathay)、蛮子(Mangi)诸州之分地。大汗答曰愿以分地授之,但须大汗遣使召海都入朝时,海都即以藩臣之礼朝见。海都疑叔意不诚,拒不入朝,仅言任在何时服从大汗命令而已。

盖其数为叛乱,恐大汗杀之,故不敢至。由是叔侄失和,发生大战,国王海都军与大汗军大战已有数次。大汗在此海都国境沿边屯驻军队以备海都。然此不足防止海都侵入大汗境内,而海都常修武备与其敌人战斗也。

海都大王势力甚强,不难将十万骑以战,皆训练有素,勇于作战之师也。彼并有帝系藩主数人与俱,兹数人者并系出成吉思汗,首应获有分地,并是曾经侵略世界一大部分土地之人,前在本书中已言之矣。

君辈应知此大突厥地位在忽鲁模思(Ormuz)之西北,起于只浑(Djihon,即阿母河)河外,北抵大汗国境。

兹置此事不言,请言海都国王部众与大汗军之若干战事如下文。





192 海都国王攻击大汗军之数战

迨至基督降世后之1276年时,此海都国王同别一王即其从兄弟名也速答儿(Yesudar)者,大集部众,编成一军,进击大汗之藩主二人。兹二藩主是海都之亲侄,盖彼等是察合台之后裔,而察合台是曾受洗礼之基督教徒,并是大汗忽必烈(Koubilai)之亲兄也。二藩主一名只伯(Djibai),一名只班(Djiban)。

海都全军共有六万骑,海都率之进攻此二藩主,而此二藩主所将大军逾六万骑。战争甚烈,二藩主终败走,海都及其部众获胜。双方之众死者无算,然藩主兄弟二人赖骑捷,疾驰得脱走。于是海都国王欢然旋师本国,留两年,相安无事,不与大汗战。

然度此两年毕,海都国王征集重军,所部骑士甚众。彼知大汗子名那木罕(Nomogan)者时镇哈剌和林(Karakoroum),而长老约翰(Prêtre-Jean)孙阔里吉思(Georges)与之共同镇守,此二王亦有战骑甚众。海都国王预备既毕,即率师出国,疾行,沿途无抗者,抵于哈剌和林附近。时大汗子与新长老约翰已率大军驻此以待,盖彼等已闻报海都率重军来侵,故为种种筹备,俾不受何种侵袭。及闻海都国王及其部众行抵附近,彼等奋勇迎敌。行至相距海都国王十哩之地卓帐结营。其敌逾六万骑,所为亦同。双方预备既毕,各分其军为六队。双方之众各持剑盾、骨朵、弓矢及种种习用武器。应知鞑靼人之赴战也,每人例携弓一张、箭六十支,其中三十支是轻箭,镞小而锐,用以远射追敌;三十支是重箭,镞大而宽,用以破肤、穿臂、断敌弓弦,而使敌受大害。各人奉命携带如此,此外并持有骨朵、剑、矛,用以互相杀害。

两军备战既毕,开战之大角大鸣,每军有角一具,盖其俗大角未鸣时不许进战也。众军闻角鸣后,残忍激烈之血战开始,双方奋怒进击。双方死亡甚众,死者伤者遍地,马匹亦然。战中呼叱之声大起,雷霆之声不过是也。海都国王以身作则,大逞勇武以励士气。对方大汗子与长老约翰孙勇武亦不下于海都,常赴酣战之处驰突,以显武功而励将士。

我尚有何言欤?应知此战之久,为鞑靼人从来未经之酷战。各方奋勉,务求败敌,然皆不副所期,混战至于日暮,胜负未决。

战争至于日落之时,各人退还营帐。其未负伤者疲劳已甚,至于不能站立。伤者双方并众,各视伤之轻重而为呻吟。各人亟须休息,甚愿安度此夜而不欲战。及至黎明,海都国王闻谍报大汗遣来重军援助其子,自量久持无益,遂命退军,比曙,上马驰还本国。大汗子与长老约翰孙见海都国王率军而退,不事追逐,盖彼等亦甚疲劳,亟愿休息也。海都国王及其部众疾驰不停,至于大突厥国撒麻耳干(Samarkand)城,自是以后遂息战。





193 大汗言其侄海都为患事

应知大汗对于海都扰害其人民土地事颇愤恚。曾云,海都脱非宗室,脱非其侄,而为亲属关系所妨阻,彼将并其身与土地灭之,虽亲征亦非所惜。盖应知者,大汗脱欲之,海都势不能脱其叔之掌握,第大汗因其为宗属,释之不问。由是海都国王得脱其叔大汗之手。

兹置此事不言,后此请言国王海都一女之神力。





194 国王海都女之勇力

国王海都有一女名称阿吉牙尼惕(Agianit),鞑靼语犹言“光耀之月”。此女甚美,甚强勇,其父国中无人能以力胜之。

其父数欲为之择配,女辄不允,尝言有人在角力中能胜我者则嫁之,否则永不适人。其父许之,听其择嫁其所欲所喜之人(其俗如此)。女身高大,近类巨人。女尝致书诸国与人约,来较力者,胜我者则嫁之,否则输我百马。由是来较力之贵人子甚众,皆不敌,女遂获马万有余匹。

基督降生后1290年时,有一贵胄,乃一富强国王之子,勇侠而力甚健,闻此女角觚事,欲与之角,俾能胜之,如约娶以为妻。然欲之甚切,盖女姿容秀丽,仪态庄严,而彼亦是美男子,甚健强,在其父国中无人能敌也。

由是此王子携千马毅然莅此国,自度力强,胜女以后,并得千马,为注固甚大也。

国王海都及王后即女生母见而悦之,阴诫女无论如何必让王子胜,盖王子为贵胄,且为一大国王子,极愿以女妻之也。然女答曰,脱彼力能敌之,决不任其胜我,脱力不能敌,则愿如约为彼妻,不甘伪败以让之也。

及期,人皆集于国王海都宫内,国王及王后亦亲临。人众既集(盖来观角觚者人数甚众),女先出场,衣小绒袄,王子继出,衣锦袄,是诚美观也。二人既至角场,相抱互扑,各欲仆角力者于地,然久持而胜负不决。最后女仆王子于地。王子既仆,引为大耻大辱,起后即率其从者窜走,还其父国,彼自以从来无敌于国中,而竟为一女所败,耻莫大焉,所携千马亦委之而去。

国王海都及王后甚怒,盖彼等皆以王子是富人,兼是勇健美男子,意欲以女妻之,孰知不如所期。

今述王女之事如此。自是以后,其父远征辄携女与俱,盖扈从骑尉中使用武器者无及其女者也。有时女自父军中出突敌阵,手擒一敌人归献其父,其易如鹰之捕鸟,每战所为辄如是也。兹置此事不言,请言国王海都与东鞑靼君主阿八哈(Abaga)子阿鲁浑(Argoun)之一大战如下文。





195 阿八哈遣其子往敌国王海都

东方君主阿八哈所辖州郡,邻接国王海都之辖地者甚众,是即位置在太阳树(Arbre Sol)附近之地,此太阳树即亚历山大(Alexandre)书所称之“枯树”(Arbre sec),前已言之矣。阿八哈防备海都部众之侵扰,命其子阿鲁浑(Argoun)率领骑兵甚众,进驻枯树,达于只浑(Djihon)河之地。

阿鲁浑率军驻守于此。会国王海都大集部众,命其弟名八剌(Barac)者统率,八剌为人颇慎重,故以军属之。已而遣此军与其弟往攻阿鲁浑。

八剌率众出发,久行抵于只浑河,距阿鲁浑约十哩。阿鲁浑闻八剌来攻,立为种种预备,率军迎敌,已而卓帐于一营内。双方备战既毕,大角齐鸣,战争立启,彼此发矢蔽空,犹如雨下,人马死者甚众,遍地皆满。战争迄于八剌部众被阿鲁浑部众击败之时,彼等重渡河去,然阿鲁浑及其部众任意虐待溃兵。由是战争结果阿鲁浑胜而八剌败,八剌赖骑捷得脱走。

我既为君等言及阿鲁浑,兹置海都及其弟八剌不言,此后请言阿鲁浑,及其父死后如何得国之法。





196 阿鲁浑战后闻父死而往承袭义应属己之大位

阿鲁浑战胜海都弟八剌及其部众以后,越时未久,闻父死,甚悲痛。命其军就归途,往取义应属己之大位,但须行四十日始达。

会其叔名算端阿合马(Sultan Ahmed,盖其皈依回教,故有是称)者,闻兄阿八哈死,而其侄阿鲁浑在远不能即归,意谋得国。遂率领所部甚众,赴其兄阿八哈宫廷,攫取大位。自立以后,见宝藏充满,其为人也颇狡智,遂尽以宝藏散给诸藩主战士等,用以收揽人心。诸藩主及战士等既受重赏,皆颂其为良君,愿爱戴之而不愿事别主。惟嗣后彼有一恶行而不免众人之谴责者,即尽纳其兄阿八哈之诸妻一事是已。

彼夺据大权以后,越时未久,闻其侄阿鲁浑率大军归,彼遂乘时召集诸藩主部众,在一星期中派遣战骑甚众往拒阿鲁浑。彼自信不难取胜,故亲将以行,不虞有失也。





197 算端阿合马率军往敌义应承袭君位之侄

算端阿合马聚众六万骑,率以进讨,行十日,闻敌军已迫而其众与本军相等。阿合马结帐于一美丽大平原中,待阿鲁浑至,与之决战。筹备既毕,聚诸藩主骑尉战士,与议进取,盖其为人狡智,欲知人心之从违,因致如下之词曰:

“诸君应知我与兄阿八哈为同父子,我曾助之侵略现有之一切土地州郡,则我兄旧有之物,义应兄终弟及。阿鲁浑固为我兄之子,容有人主张其应袭父地。然我以为此意不公,缘我父终身治理此国,君等之所知也,父死义应由我终身治理;况父在生时,我应有国之半,而曾因柔弱上与欤。今我言如此,请君等共同防卫吾人之权利以拒阿鲁浑,俾国土乃属吾辈众人所有,盖我所欲者仅为荣誉,而一切土地州郡并权柄利益,概归君等得之也。此外别无他言,盖我知君等侠义贤明,爱好公道,必将为有利于众人之福利与光荣也。”

语毕遂默不复言。诸藩主骑尉闻言皆众口一词答曰:“有生之年,起不奉戴他主,将助其抵抗世界一切人类,尤愿抵抗阿鲁浑,请勿疑。可鲁浑生或死必执以献。”阿合马鼓励其众,而借识人心之法如此。

兹置阿合马及其部众不言,请言阿鲁浑及其军队。





198 阿鲁浑与诸将议往攻僭位之叔算端阿合马事

阿鲁浑既确知阿合马率众待于营中,因甚愤恚。然故作镇定之状,盖其不欲部众信其畏慑致乱军心。所以伪若无事者然,反示其无所畏以励士气。

由是召集诸藩主及最贤明之人甚众议于帐中(盖其结帐于一最美之地),致如下之词曰:

“兄弟友朋齐听我言。汝曹应知我父爱汝曹之切,待汝曹如同亲子弟。汝曹昔曾偕之数作大战,助之侵略所辖之全土,汝曹应知我为切爱汝曹之人之子,而我亦爱汝曹甚切。我既以实言告汝曹,论理汝曹应助我以讨僭夺吾曹之国之人。汝曹并知其不守吾曹教理,皈依回教而崇拜摩诃末。一回教徒君临鞑靼之国,其事非宜。据此种种理由,汝曹应增加勇气决心,俾免此辱。所以我祈汝曹各尽其力,勇战务求必胜,俾国属吾曹,不致沦于回教之徒。权利既属吾曹,罪恶既属敌人,各人应抱必胜之信心。此外我别无所言,汝曹各熟思之。”

阿鲁浑语毕遂默不复言。





199 诸藩主答阿鲁浑之词

诸藩主骑尉聆悉阿鲁浑之词以后,各人自励,宁死不让敌胜。众人如是沉思之时,其中一大藩主起而答曰:

“阿鲁浑殿下,吾曹皆知谕众之言皆是实言。是以我代表众人致此答词:吾曹有生之年决不奉戴他主,宁死而不愿败。抑况权利属吾曹,而罪恶属敌人,尤应自信此战必胜。用是请殿下从速率领我曹赴敌,而我祈同辈力战自效,俾扬名于世。”

藩主语毕,遂不复言。众人意皆与之同,只欲与敌战,故无继之发言者。翌日,阿鲁浑率其众早起出发,决与敌战。骑行至于敌人卓帐之平原中,距阿合马帐十哩结营。阿鲁浑结营毕,遣其亲信二人赴其叔所,致下述之词。





200 阿鲁浑遣使者至阿合马所

此二贤明之人并是高年之人,奉命以后,立与主别,登骑而行。彼等径赴阿合马营,在其帐前下马,会见藩主甚众。诸人皆识之,彼等亦识诸人。彼等见阿合马致敬毕,阿合马好颜厚待之,命其坐于前,未几,两使者中之一人起而致辞曰:

“阿合马殿下,君之侄阿鲁浑对于君之所为,惊异甚至:君既夺其封地,而又率军进讨,与之作殊死战,叔对其侄行为不宜如是也。所以彼命吾辈善言以请,彼既视君如叔如父,君应放弃此种企图,彼此罢战;彼言始终奉君如长如父,而愿君为彼之全土之主。君侄命我等口传之言如是。”

此藩主语至此,遂默不复言。





201 阿合马答阿鲁浑使者之词

算端阿合马聆使者代达其侄阿鲁浑之词毕,乃作下述之答词曰:

“使者阁下,吾侄所言,毫无根据,盖土地属我而不属彼,我与其父并得之也。可往告吾侄,我将使之为大诸侯,授以多地,待之如子,而使之为一人之下之最大藩主。如若不从,我必将其处死。我欲告吾侄之言如此,汝等别无其他条件或退让可图也。”

阿合马语至此,遂默不复言。使者聆算端之词毕,复问曰:“此外无他言欤?”答曰:“我在生时别无他言。”使者闻言立行,赴其主营帐,在帐前下马,入谒阿鲁浑转达其叔之言。阿鲁浑甚恚,大声发言,左右皆闻,其词曰:

“我叔有大过,而加我以大辱,我不报此仇,誓不复生此世,亦不复管理土地。”语毕告诸藩主骑尉曰:

“今已无复踌躇者,只须从速讨诛此种不义叛人,自明朝始,可进击而歼灭之。”于是终夜筹备战事。算端阿合马闻谍报阿鲁浑将于明朝进攻,亦备战,命其众奋勇进击。





202 阿鲁浑与阿合马之战

比及翌日,阿鲁浑部勒全军甚善,号令既毕,率之迎敌。算端阿合马所为亦同,亦部勒行列,不待阿鲁浑行抵其营,即率其众前进。行未久,即遇阿鲁浑及其所部军。两军既接,双方皆急欲战,冲突遂起。至是见飞矢蔽天如同雨下,战争酷烈,见骑士坠马仆地,闻仆地者及受致命伤者号痛悲泣之声。矢尽,执剑与骨朵以战,断手断臂者有之,丧躯丧首者有之,喧噪之声大如雷霆。

此一战也,双方死者甚众,而妇女之服丧号泣终身者颇多。是日阿鲁浑颇尽其职,大示勇武,以励士气,然其结果终不免于失利,其众不能御敌,皆溃走恐后。阿合马及其众追击,斩杀甚众,而阿鲁浑即在追逐中被擒。彼等擒获阿鲁浑后,不复再追溃众,欢欣还其营幕。阿合马综拆其侄,命人严加看守,已而归其后宫与诸美妇娱乐,盖其为人好声色也。命一大藩主代总全军,并嘱之看守其拆,缓缓归师,以免将卒疲劳。阿合马离军而命此藩主代总其军之经过如此,阿鲁浑既被擒,悲伤欲死。





203 阿鲁浑之被擒及遇救

会有一鞑靼大藩主,年事甚高,颇怜阿鲁浑,以为囚禁主人,既犯大恶,而又不义,遂谋救之。因即与其他诸藩主谋,而语之曰:“囚其委质之主,是为大恶,应救出而奉之为主,且彼于义应承大位也。”其他诸藩主视此藩主为最贤明之人,觉其所言盖属实情,遂共愿与之同谋。诸同谋者为不花(Buga,是为谋主)、宴只歹(Elcidai)、脱欢(Togan)、忒罕纳(Tegana)、塔哈(Taga)、梯牙儿乌剌台(Tiar Oulatai)、撒马合儿(Samagar)等,同谋后共赴阿鲁浑囚居之帐。入帐后,不花年最长,且为主谋,遂致辞曰:“阿鲁浑殿下,我曹拘禁殿卞,诚为有过。今特来改过,救殿下出此。请殿下为吾曹之主,且亦殿下义所当为也。”

不花语至此,遂默不复言。





204 阿鲁浑之得国

阿鲁浑闻不花言,以其嘲己,愤而答曰:

“汝之揶揄,诚犯大过,汝曹应奉我为主,而反加以锒铛,已为大恶,然尚以为未足欤?犯大恶而为大不义,汝曹应自知之,请他适,勿再嘲弄也。”

不花又致辞曰:“阿鲁浑殿下,我辈诚心为此,并非揶揄,愿誓以明此心。”诸藩主等遂共发誓,承认阿鲁浑为主。阿鲁浑亦对诸人誓,不复咎彼等擒己之旧恶,将厚待之如其父阿八哈之恩遇。誓毕,彼等解阿鲁浑之综而奉之为主。阿鲁浑立命向代总军队之藩主帐发矢,迄于其人死而后已。已而阿鲁浑即位,统率奉彼为主之人,时国人皆已服从矣。应知吾人所称藩主者,名称琐勒聃(Soldam),其人为次于阿合马之最大藩主。阿鲁浑得国之经过如上所述。





205 阿鲁浑杀其叔阿合马

阿鲁浑受众人推戴以后,即命进向宫廷。会阿合马在其最大宫内大宴,有使臣来报曰:“今有恶耗报闻,诚非所愿。诸藩主已杀君之爱友琐勒聃,已将阿鲁浑救出,奉之为主,彼等已向此处急进,而谋杀君,请速为计。”使者言至此遂默不复言。阿合马知使臣忠诚可恃,闻言之下,惊惧异常,不知所措,但其为人豪迈勇武,亟为镇定,而告使臣不得以此恶耗吐露于人。使臣许之。阿合马立与亲信可恃者上马,欲奔投埃及算端而逃死,除偕行者外,无人能知其赴何地也。

行六日,至一狭道,乃其所必经之道途,守关者识其为阿合马,见其逃,决捕之,缘阿合马之随从甚少也。守关者遂捕阿合马,阿合马乞怜请释,并许以重宝赂之。守关者爱阿鲁浑甚切,拒不允,且谓虽尽以世界宝藏赂之,亦不能阻其献俘于其主阿鲁浑。守关者因多发护卒挈阿合马赴宫廷,并严加监守,俾其不能遁逃。沿途不停,直抵宫廷,时阿鲁浑至已三日,正怒阿合马之得脱走也。





206 诸藩主之委质于阿鲁浑

守关者挈阿合马至,以献,阿鲁浑大喜,而语其叔将依法惩之。即时遂命引之去,杀而灭其尸。奉命执行者引阿合马至行刑之所,杀阿合马而投其尸于一无人能识之处。阿鲁浑与其叔阿合马争位之经过如此。





207 阿鲁浑之死

阿鲁浑既为前述诸事以后,遂赴主要宫殿,君临全国。各方藩主前隶阿八哈者皆来朝贺而尽臣职。至是阿鲁浑军权已固,遂命其子合赞(Gazan)率三万骑往枯树之地,防卫土地人民,以御敌侵。阿鲁浑得国之经过如此,时在耶稣基督降世后之1286年也。阿合马在位仅二年,阿鲁浑君临六年,得疾死,一说中毒死。





208 阿鲁浑死后乞合都之得国

阿鲁浑死后,其一叔即其父阿八哈之亲弟名乞合都(Kaikhatou)者,立时夺据大位,盖合赞远在枯树之地,不能与之争也。合赞闻其父死耗,甚痛,同时又闻其父之叔夺据大位之讯,甚怒。然恐敌侵,不敢遽离此地,曾云将俟机往复此仇,如其父之擒阿合马也。乞合都既得国,国人皆服从,惟合赞之党不奉命。乞合都颇好色,遂沉溺于女色之中。在位二年死,盖为人所毒杀也。





209 乞合都死后伯都之得国

乞合都死后,其诸父伯都(Baidou)是基督教徒,据有大位,事在基督降世后之1294年也。伯都既居君位,国人皆服从,惟合赞及其军不奉命。合赞闻乞合都死而伯都得国,甚愤恚,盖其未能及乞合都之生而报仇也。然有言曰,将对伯都报此仇,必使众人皆传其事。由是决定不再俟机,即兴兵往州白都。决定以后,与所部回师,谋复故国。伯都知合赞进兵,亦大集其众往敌,行十日结营,而待合赞军至。结营不及二日,合赞军至。是日残忍战争即见开始,然伯都不能久敌合赞,盖战争甫开之时,伯都部众多投合赞,倒戈而向伯都,所以伯都败,且被杀。合赞既胜,遂为全国之主。合赞既胜而杀伯都后,即赴宫廷即位,诸藩主皆对之委质称臣。基督降世后之1294年,合赞开始君临其国之经过如此。

此国自阿八哈迄合赞之史事如前所述。并应知者,侵略巴格达(Bagdad)之旭烈兀(Houlagou),乃是大汗忽必烈之弟,而前述诸人之共祖,缘其为阿八哈之父,阿八哈为阿鲁浑之父,而阿鲁浑为今日君临其国的合赞之父也。

东方鞑靼既已备述于前,兹请复还大突厥国。顾大突厥国及其国王海都前已言及,则可不复再述,请离此而述较北之州郡人民。





210 北方之国王宽彻

应知北方有一国王名称宽彻(Kauntchi),彼是鞑靼,而其臣民皆是鞑靼,彼等遵守极强暴之鞑靼法规,然守之如成吉思汗及其他真正鞑靼无异,兹请略述其事。

应知彼等有一毡制之神,名曰纳赤该(Nacigai),神有一妻,土人相传兹二神,质言之,纳赤该与其妇,是保佑其牲畜收获并一切土产之地神。彼等崇奉之,每有盛馔,必以油涂神口。彼等生活绝对如同禽兽。

其国王不隶何人,确为成吉思汗裔,质言之,属帝室而为大汗之近亲也。此国王无城无堡,与其人民居于广大平原之中,或处大谷高山之内。彼等食牲畜之乳与肉,而五谷类。国王统治人民甚众,然不与他族争战,而维持平和。饶有牲畜,如驼、马、牛、羊及其他动物。

其地多有白熊,熊长逾二十掌,亦多有大狐,全身黑色,并有野驴及貂甚众。用貂皮作裘,男袍一袭值千别桑(basant)。饶有灰鼠,并多夏生甚肥之土鼠(rats de Pharaon)。且饶有种种野兽,盖其生活于极荒野而无人居之区也。

更应知者,此国王辖有某地,马不能至,盖其地多湖泽水泉,多冰与泥,马不能行。此恶地广十三日程,每一日程设一驿站,以供往来使臣顿止之所。每站有犬四十头,犬大如驴,载使臣自此站达彼站,质言之,行一日程,兹请言其状。

应知在此旅行全程之中,冰泥阻止马行,盖在此十三日中行于两山间之大深谷内,冰泥沉陷马蹄也。职是之故,马不能前,有轮之车亦不能进。所以土人制无轮之橇,行于冰泥之上,俾其不致深陷于其中。每橇置一熊皮,使臣坐其上,用上述之大犬六头驾之,不用人驭,径至下站,安行冰泥之上,每站皆然。驿站之人,别乘一橇,用犬驾之,取捷道径赴下站。两橇既至,使臣又见有业已预备之犬橇,送之前行,至若原乘之橇则回后站。十三日行程之中,皆如是也。

更有言者,此十三日行程中,沿途山谷中居民皆为猎人,猎取价值贵重之罕见动物而获大利,是为貂、银鼠、灰鼠、黑狐及不少皮价甚贵之罕见动物。其人有猎具,猎物无得脱者。其地酷寒,土人居于土窟,而常处土窟之中。

此外无足言者,是以离此,请言一常年黑暗之地。





211 黑暗之州

此国境外偏北有一州名称黑暗,盖其地终年阴黑,五日月星光,常年如是,与吾辈之黄昏同。居民无君主,生活如同禽兽而不隶属于何人。

鞑靼人偶亦侵入其国如下所述。彼等欲确识归途,选牝马之有驹者乘之,人其地前,放驹境外,盖牝马较人易识路途,将重循来途回觅其驹也。由是鞑靼人留驹于境外,乘牝马入其地,尽盗其所见之物。饱载以后,任牝马重循来路往觅其驹,盖其常识归途也。

其地之人饶有贵重毛皮,盖其境内多有贵重之貂,如前所述,又有银鼠、北极兽(glouton)、灰鼠、黑狐,及其他不少贵重毛皮。人皆善猎,聚积此种毛皮,多至不可思议。居处边境之人,而认识光亮者,向此辈购买一切毛皮,盖此黑暗州人携之以售于光亮地界之人,而光亮地界之人首先购取而获大利。

其人身体魁伟,四肢相称,然颜色黯淡而无色。大斡罗思(Russie)境界一端与此州相接。此外无足言者,兹离此,请首言斡罗思州。





212 斡罗思州及其居民

斡罗思(Russie)是北方一广大之州。居民是基督教徒而从希腊教。有国王数人,而自有其语言。其人风仪淳朴,男女皆甚美,皮白而发呈金褐色。不纳贡赋于何国,仅纳贡于西鞑靼国王脱脱(Toktai)然其数甚微。此非业商之国,但有不少稀有之贵重毛皮,如貂、狐、银鼠、灰鼠、北极兽等毛皮之类,世界毛皮中之最美而最大者也。又有银矿不少,采银甚多。

此外无足言者,兹离斡罗思,请言大海,列述其沿岸诸州及其居民,首述君士坦丁堡。

然我将先言北方及西北方间之一州。应知此地有一州,名称瓦剌乞(Valachie),与斡罗思接境,自有其国王,居民是基督教徒及回教徒。彼等颇有贵重毛皮,由商人运售诸国,彼等恃工商为活。

此外无足言者,所以离去此国,而言他国。然尚有关系斡罗思之事前忘言之,应补述于此。须知斡罗思国酷寒为世界最,居民颇难御之。此州甚大,延至海洋。此海之中有若干岛屿,出产鹰鹞甚多,输往世界数地。尚有言者,自斡罗思至挪威(Norvège),里程不远,如非酷寒,旅行甚易,但因严寒之故,往来甚难。

今置此不言,请言前此欲言之大海。虽有不少商贾、旅客曾至其地,然尚有世人未识之处甚多,兹略为叙述,首言君士坦丁堡之海口与海峡。





213 黑海口

从西方入大海之海峡中,有一山名称发罗(Faro)。但言及大海以后,吾人颇悔将其笔之于书,盖世人熟识此海者为数甚众也。是故记述仅止于此,别言他事,请述西方鞑靼及其君主。





214 西鞑靼君主

西鞑靼第一君主即是赛因(Sain),强大国王也。此赛因国王曾略取斡罗思(Russie)、钦察(Kiptchak)、阿兰(Alains)、瓦剌乞(Valachie)、匈牙利(Hongrie)、撒耳柯思(Circassie)、克里米亚(Crime)、陶利德(Tauride)等州。如是诸州,侵略以前皆属钦察,然未统一,构成一国,所以其居民失其土地而散处各方,其尚留居者皆沦为国王赛因之奴。

国王赛因之后在位者是国王拔都(Batou),拔都之后是国王别儿哥(Barka),别儿哥之后是国王忙哥帖木儿(Mongou Timour),忙哥帖木儿之后是国王脱脱蒙哥(Toudai Mongou),最后是脱脱(Tokai),今日君临其国。

兹既列举西鞑靼之君主毕,后此请述东鞑靼君主旭烈兀(Houlagou)与西鞑靼君主别儿哥之一大战,并言战争之原因与夫战争之状况及结果。





215 旭烈兀别儿哥之战

基督降世后1260年时,东鞑靼国王旭烈兀与西鞑靼国王别儿哥发生一种大战。其故盖在彼此境界间有一州地,各欲攘为己有,自度势力强盛,皆相持不让。彼等皆作挑战之词,谓将往取此州,看何人敢抗。挑衅以后,各集战士,大筹从来未见之军备,各为其过度之努力,务期必胜。挑衅以后,未逾六月,各集兵三十万骑,一切习用战具皆备。备战既毕,东鞑靼君主旭烈兀率众出发。骑行多日,无事可述,久之,抵于铁门及里海间之一大平原。在此平原中结营,观其帐幕之富丽奢华,俨同一富豪营幕。旭烈兀谓待别儿哥及其众于此,应知其结营之地在两国边境之上。

兹置旭烈兀与其军不言。请言别儿哥与其部众。





216 别儿哥率军进攻旭烈兀

国王别儿哥筹备战事,调集全军既毕,闻旭烈兀率军进迫,遂不再待,亦率其军出发。骑行久之,进至敌人所驻之大平原,距旭烈兀营十里结营,其营帐之富丽,亦不下于旭烈兀营。我敢断言曾见此种金锦帐者,将必谓从来未见营帐之富丽有逾此者。别儿哥部众较多于旭烈兀军,盖其确有三十五万骑也。卓帐以后,休息二整日。别儿哥至是集众与议,而语之云:

“汝曹知我得国以后,爱汝曹如同亲子弟,汝辈多曾偕我屡经大战,吾人现有之土地,多由汝曹助我得之,汝曹又知我之所有亦属汝曹,既然如此,各人必须奋勉保存今兹以前未坠之名誉。汝曹知此强大国王旭烈兀非理进兵,彼既无理,而吾人有理,则各人应自信将来必操胜券,况且吾人兵多于敌,其事尤无可疑。盖彼等仅有三十万骑,而吾人则有三十五万骑,将士优良此亦不下于彼也。职是之故,具见吾人确操胜券。吾人远来此地,惟在作战,兹限战期于三日后,望汝曹努力为之,战争之日,务必奋勇进击,俾人皆畏我。现在除求汝曹各人预备及期奋勇作战外,别无他言。”

别儿哥言至此遂默不复言。兹暂置别儿哥军不言,请言旭烈兀军在别儿哥军进迫后,如何应战之事。





217 旭烈兀谕众之词

史载旭烈兀确闻别儿哥率领众军行抵其地之时,复又大集其优良将士而语之云:

“兄弟友朋,汝曹皆知我一生时皆赖汝曹之助,迄于今兹,汝曹在不少战中助我,每战必胜。吾人今抵此地与别儿哥大王战,固知其众与我军等,或且过之,但其数虽众,其战士不及我军之良,吾人将不难使之败亡。今闻谍报,三日后敌军将来进攻,吾闻此讯甚欢,所以请汝曹届期勇战犹昔。仅有一事汝曹不应忘者,则宁死于疆场保其令名,不可败于敌,应使敌人败亡也。”

旭烈兀语至此遂默不复言。此两大君主砥砺其众之词如此,其藩主等则做种种预备而待战期之至。





218 旭烈兀别儿哥之大战

预定之战期既届,旭烈兀黎明即起,命全军尽执武器,然后发令慎重进战。分其军为三十队,每队万骑,盖其军有三十万骑,前已言之矣。每队良将一人统之。布置既毕,命诸队进击敌军。其众立即奉命前行,进至两军营帐中间之地,静止以待敌至。

对方国王别儿哥亦偕其部众早起,命各执兵备战,分其军为三十五队,每队万骑,各以良将一人统之,与旭烈兀军部勒相同。预备既毕,别儿哥命诸队前行,行列甚整,进至距敌人半哩之地,稍停,复前进。

两军进至相距两箭之地,皆停止预备作战。战场在一平原中,最广大美丽,足容无数战骑驰突。此广大美丽平原恰为两军之所必须,盖从来未有一大战场能容战士如是之众者也。应知两军共有六十五万骑,旭烈兀、别儿哥并是世界最强大国王,尚应附带言及者,彼等谊属近亲,二人皆系出成吉思汗也。





219 重言旭烈兀别儿哥之战

两大国王及其部众对峙片刻,皆待战号,惟盼大角之鸣。来久战号起,战角鸣,两军即开始作战,皆引弓发矢射敌。双方发矢蔽空,不见天日。至是见死者仆地甚众,马匹亦然,盖发矢既多,死者无算也。

应知彼等菔中矢不尽,射击不止,由是死伤遍地。及发矢已尽,遂执剑与骨朵,彼此交斫。此战杀人流血之甚,观之可悯,有断手者,有断臂者,有断头者,人马仆地,其数之众至堪惨恻。从来战事,死亡无逾此役之多。呼噪之声大起,如闻雷震。满地伏尸流血,人不能进,势须踏尸前行。

战中死亡之众,如是役者,久未见之。死者之多,受致命伤而仆地者号泣之惨,诚不忍闻。妇女因之而寡者,子女因之而孤者,其数未可胜计。彼等战斗之烈,表示其仇怨之深。

国王旭烈兀勇武善战,于是日以身作则,以证其无愧于冠王冠而治国土。彼大逞武功,以励其众,凡友与敌见之者莫不惊异,盖其有类雷霆暴雨,非同凡人也。

旭烈兀在此战中之行为如此。





220 别儿哥之勇武

国王别儿哥亦甚勇武,善于作战。然在是日部众几尽死,勇亦徒然,伤者仆地之多,余众不复能抗。所以战至晚祷之时,国王别儿哥及其部众不能支持,势须奔逃。彼等疾驰,旭烈兀及其众追蹑其后,凡被追及者皆被杀。追杀之惨,观之诚可悯也。追杀久之,始收兵还其营帐,释其兵械,其受伤者,洗裹其伤。彼等疲劳之极,已不复能战,安寝终夜。次日黎明,旭烈兀命尽焚战亡者之尸,不分友与敌也。

诸事既毕,国王旭烈兀率其余众还国,盖虽战胜,亡损已多。然其敌亡损尤众,在此战中死亡之多,虽言其数,恐亦无人信也。旭烈兀在此役获胜之经过如此。

兹置旭烈兀及此战不言,请言西鞑靼之一战事,详情别见后章。





221 脱脱蒙哥取得西鞑靼君位事

应知西鞑靼君主忙哥帖木儿(Mangou Timour)死后,君权属一幼王秃剌不花(Tolobouga),然有脱脱蒙哥(Toudai Mongou)者,一强有力之人也,与别一鞑靼国王名那海(Nogai)者结合,共杀秃剌不花。脱脱蒙哥既得那海助,夺据君位,在位不久死。至是君权遂属脱脱(Toktai),其人甚贤勇,既得脱脱蒙哥之国,遂执有大权。会是时秃剌不花之二子,渐长成为可以执兵之人。

兹二弟兄,质言之,秃剌不花之二子,颇贤慎,携带护卫甚盛,往役脱脱衙。既至,二人跪谒脱脱,脱脱厚遇之,命之起。二人起立后,年较长者致如下之词曰:

“大王脱脱,兹请一述我辈来谒之故。我辈为秃剌不花子而父为脱脱蒙哥、那海二人所害,王之所知也。脱脱蒙哥已死,我辈无所言,然那海尚在,特来求我辈正主为我辈正杀父者罪。我辈来谒原因在此。”

此王子语毕,遂默不复言。





222 脱脱遣使至那海所质问秃剌不花死状

脱脱闻此童儿言,知为实情,乃答之云:“好友,汝求我治那海罪,我甚愿为之。我将召之至衙,按理治之。”脱脱于是遣使者二人至那海所,召之来衙,对秃剌不花之子服罪。使者语那海毕,那海揶揄之,答言不愿赴衙。使者得复,立行,还主所复命,告以那海决不来衙。脱脱闻言大恚怒,呼曰:“天若助我,必使那海来此对秃剌不花诸子服罪,抑使我率军往讨灭之。”呼声甚大,左右皆闻。由是立时别遣二使者以下述之词往告那海。





223 脱脱遣使至那海所

脱脱二使者奉命立行,骑行久之,至那海衙。人见,以礼谒之,那海厚遇使者。使者中之一人致辞曰:“大王,脱脱有谕,如王不赴衙向秃剌不花之子服罪,彼将尽率其众进讨,而使王之财产及王之身大受损害。请决从违,俾吾曹归报。”

那海闻使者转达脱脱之词毕,大恚怒,答使者曰:“使者请立时还告汝主,谓我不畏战争,如彼以兵来,我将不待其入境,而迎之于半道。”语毕遂默不复言。

使者闻那海言,遂不复留,立行,还其主所。既至,转达那海言,谓其不畏战争,将迎之于半道。

脱脱闻言,见战争不可复免,立遣使者四出赴诸辖地,征集部众,进讨国王那海。彼大筹军备,那海一方既知脱脱将以重军来讨,亦筹战备,然不及脱脱之大,缘其部众及兵力不及脱脱之强,但所部亦甚众也。





224 脱脱往讨那海

国王脱脱一切战备既毕,率众出发,应知其众逾二十万骑也。沿途无事可述,已而抵于广大美丽之赖儿吉(Nerghi)平原,脱脱结营于此以待那海,缘其知那海率其众来敌也。秃剌不花之二子亦率骑士一队至此,冀报父仇。

兹置脱脱及其众不言,请言那海及其部众。应知那海闻脱脱进兵之讯,立即率众出发,所部有十五万骑皆勇健之士,较优于脱脱所部之战士也。

脱脱抵此平原未及二日,那海率全军至,距敌十哩结营。结营以后,则见金锦美丽帐幕无数,俨若富强国王之营垒。脱脱营帐富丽亦同,然且过之,盖其帐幕奇富丽也。

两王既抵此赖儿吉平原,皆休息以待战日之至。





225 脱脱谕众之词

国王脱脱大会部众而致如下之词曰:

“我辈至此与国王那海及其军队战,而理在我方,盖应知者,怨恨之结,乃因那海不欲向秃剌不花诸子服罪也。彼既无理,则在此战之中吾人必胜而那海败亡。是以吾辈应勇战胜敌,我知汝曹皆勇士,务必灭敌而置之死地。”语毕遂默不复言。

国王那海亦会部众致如下之词曰:

“兄弟友朋齐听我言,汝曹皆知吾辈在诸大战中战胜敌人,诸敌且强于此敌也。况且理在我方,曲在彼方,汝曹尤应自信此战必胜,盖脱脱非我主,不能召我赴衙向他人服罪也。我今求汝曹各尽其职,俾世人皆知吾曹善战,而使吾曹与吾曹之后裔永为人所畏慑,此外别无他言也。”国王那海语至此遂不复言。

兹二国王励众既毕,翌日即预备作战。国王脱脱分其众为二十队,每队以良将一人统之。国王那海仅分其众为十五队,每队万骑,各以良将一人统之。兹二国王部勒既毕,双方进兵,彼此进至一箭之地,止而不进。越时未久,战角始鸣。战角鸣后,双方发矢,发矢之多,人马死伤坠地者甚众,到处皆闻呼叱呻吟之声。矢尽,两军之众各持剑与骨朵斫敌,由是杀人流血之混战遂启,互断身首手臂。至是则见骑士死伤仆地,呼噪之声、兵刃交接之声,其响有如雷霆,死亡之众,前此诸战久未见之。然脱脱军死较多于那海军,盖那海之众作战较优也。秃剌不花二子奋勇杀敌,冀复父仇,然皆徒劳,盖欲致国王那海于死地,其事甚难也。

此战残酷,有无数战士是最健全者,在此战中多遭杀害,此战以后,妇女因而寡居者为数不少。国王脱脱竭力鼓励其众保其令名,且以身作则,大逞勇武,驰突于敌中,视死若无事,所过之处,见人则杀。其作战之勇,友敌并受其害,盖敌人被其手杀者甚众,而友人受其鼓励亦作殊死战,因而阵亡也。





226 国王那海之勇武

国王那海作战之勇,两军中人无人能与比者,此战之誉,尽属于彼,良非伪言。彼驰突敌阵之中,勇如狮子之搏野兽。往来格杀,使敌人大受损害,每见敌众群集之处辄赴之,击散敌人如驱小畜。部众见其主之勇武,亦效之,奋勇杀敌,使敌大受损害。脱脱之众虽努力保其战誉,然徒劳而无功,盖势不敌也。损伤既重,不能久持,遂败逃。国王那海及其部众追逐,杀人无算。

那海获胜,诚如上述。此一役也,死者至少六万人,然国王脱脱暨秃剌不花之二子皆得脱走。

\chapter{结言}

关于鞑靼人、回教徒暨其风习,与夫世界其他诸国之事,兹已据所闻见述之如前。惟独遗黑海沿岸诸州,缘其地时有人往游,威尼斯人、热那亚人、皮撒人之航行此海者甚众,述之似乎累赘,人尽识之,故遗而不述。

至若吾人得以离开大汗宫廷之情形,业已在本书卷首言之,玛窦、尼古剌、马可阁下等因求大汗许可,所经之忧虑困难与夫得还本国之良好机缘,并具此章。吾人若无此良好机缘,殆恐永远难回本国。我以为吾人之得还,盖出天主之意,俾使吾人得以闻见之事传播于世人也。盖据本书卷首引言所云,世人不论为基督教徒或回教徒、鞑靼人或偶像教徒,经历世界之广,无有逾此威尼斯城之名贵市民尼古剌阁下之子马可阁下者也。恩宠的上帝,阿门(Deo Gratias.Amen)。

\backmatter

\end{document}