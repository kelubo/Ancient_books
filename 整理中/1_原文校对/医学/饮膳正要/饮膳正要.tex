% 饮膳正要
% 饮膳正要.tex

\documentclass[12pt,UTF8]{ctexbook}

% 设置纸张信息。
\usepackage[a4paper,twoside]{geometry}
\geometry{
	left=25mm,
	right=25mm,
	bottom=25.4mm,
	bindingoffset=10mm
}

% 设置字体,并解决显示难检字问题。
\xeCJKsetup{AutoFallBack=true}
\setCJKmainfont{SimSun}[BoldFont=SimHei, ItalicFont=KaiTi, FallBack=SimSun-ExtB]

% 目录 chapter 级别加点(.)。
\usepackage{titletoc}
\titlecontents{chapter}[0pt]{\vspace{3mm}\bf\addvspace{2pt}\filright}{\contentspush{\thecontentslabel\hspace{0.8em}}}{}{\titlerule*[8pt]{.}\contentspage}

% 设置 part 和 chapter 标题格式。
\ctexset{
	chapter/name={},
	chapter/number={}
}

% 图片相关设置。
\usepackage{graphicx}
\graphicspath{{Images/}}

% 设置古文原文格式。
\newenvironment{yuanwen}{\bfseries\zihao{4}}

% 设置署名格式。
\newenvironment{shuming}{\hfill\bfseries\zihao{4}}

% 注脚每页重新编号,避免编号过大。
\usepackage[perpage]{footmisc}

\title{\heiti\zihao{0} 饮膳正要}
\author{忽思慧}
\date{元}

\begin{document}

\maketitle
\tableofcontents

\frontmatter
\chapter{序言}

《饮膳正要》为元代营养学家兼医学家忽思慧(又译和斯辉)所撰。忽思慧,元代蒙古人,生卒年月不详,约为13--14世纪间人,在元仁宗延祐年间任宫廷饮膳太医,他兼通蒙汉两种医学,不但精研食疗养生和烹饪技艺,同时还博览诸家本草、名医方术,值元天历三年(至顺元年、公元1330年)著成《饮膳正要》一书,该书文图并茂,形式活泼,具有独到的见解和鲜明的民族特色,它是一部具有极高价值的古代营养食谱,是我国乃至世界上最完整的营养学专著。

全书共分为三卷,收载食物、饮物、料物和禽类等。书中论述了蒙、汉、回、藏等各族群众常用的食物及其营养价值。卷一载有三皇圣纪、养生保健、饮食避忌、聚珍异馔、妊娠食忌和乳母食忌。卷二主要记载了诸般汤煎、食物治病、食物相反、食物中毒的救治等。卷三分门别类选录了丰富的食疗营养天然食品,如米、谷、果、菜、鱼、禽、兽以及料物等,详细叙述了每种食品的性味、功能、适应病症以及有无毒性的内容。《饮膳正要》集饮食学、药物学、养生学、康复学以及心理学之大成,具有浓郁的民族特色,有较高的实用价值和学术价值。

\chapter{校注说明}

《饮膳正要》于元天历三年(至顺元年、公元1330年)成书并刻梓问世,迄今已近680年,此书早年曾流传到日本,据《经籍访古志》记载有称:《饮膳正要》为吉田安手抄本,卷末有成化乙未鼎新刊记等字。

《饮膳正要》于历代书目中均有记载,如《元史艺文志》、《万卷楼书目》、《医藏书目》、《百川书志》以及清代的《四库全书总目》、《铁琴铜剑楼藏书目录》、《皕宋楼藏书志》、《中国医籍考》等。

我国明清两代曾多次翻印《饮膳正要》,如景泰七年(公元1456年)代宗帝亲为作序,颁旨重刻;至清代官梓民刻亦多,流传甚广。但由于战火纷乱而未能保全以致散佚。现存主要版本有明经厂刊大字本仅存卷二(已残卷),1924年上海涵芬楼影印明景泰刊本(即《四部丛刊》本),1924年上海商务印书馆的《国学基本丛书》本(铅排本),《万有文库》本(铅排本),1982年内蒙古人民出版社出版了胡和禄翻译的蒙文本,1986年人民卫生出版社出版了刘玉书的此书点校本(铅排本)等。

现今仅存的4个古本均未经过校勘,书中错漏谬误不少;而现代的《饮膳正要》点校本(铅排本)中有的字词注释多引自《史记》、《汉书》、《尚书》、《说文》等,其中校文较为浑朴古奥;为了既保存古籍的原汁原味,又使原貌更加清新简洁,特进行了点校。现将此次点校中有关问题申述于此:

一、本次点校以卷帙完整,文字与插图清晰的上海涵芬楼影印明景泰刊本(即《四部丛刊》本)为底本,以《国学基本丛书》本、《万有文库》本和明经厂刊大字本为校本;原书中援引诸家之文,参以《素问》、《灵枢》、《难经》、《日华子本草》、《食医心镜》、《食疗本草》等书之佳善版本以为旁校;同时还参照了刘玉书的点校本等。

二、由于本书是据底本排印出版,原文不作任何改动。校勘、训诂统一编码。故所有校勘均以训释注于地脚;原书明显错误者,在校语中加以提示,如“本书误”等。

三、全书用新式标点符号断句。生僻字词、通假字、避讳字,稍加训释;怪僻字、异体字注音用直音法或汉语拼音法注音。

四、凡底本的文字出现讹错、脱漏、衍文、颠倒等情况均出校记说明。

五、凡底本与校本的字、词不一致者均作校记加以说明。

六、凡底本出现不规范字或一字多次出现字体不一者,如“矱”与“鸡”,“朴”与“朴”,“癬”与“斤”,“觔”与“脉”等,此类情况一般不予训释,也不予律齐。

七、对于既没有书籍校本校注,又无法分析论证、猜测推理的少数民族动作用语和物品用语,如渴忒、兀麻食、舍儿别、哈夫儿、脑瓦刺等,出以校注,以示存疑待考。

八、正文方剂中,诸本药序互异,无碍方义者,从底本,不出校。药量互异,悬殊小者,从底本,不出校;悬殊大者,出校并存。

九、由于作者所处时代的局限性和历史传统的影响,书中有些内容往往神化了某些药物或食物的效果,同时也夸大了某些药物或食物的副作用,甚至还有带有迷信色彩的内容(如原书载:服用琼玉膏可返老还童而成仙人等),为了忠实于原著,对于这些内容既没有删节,也没有加以说明校记,意在全面保持原著的原风原貌。

\chapter{御序}

\begin{yuanwen}
朕惟人物皆禀天地之气以生者也。然物又天地之所以养乎人者,苟用之失其所以养,则至于戕害者有矣。如布帛菽粟鸡豚之类,日用所不能无,其为养甚大也。然过则失中,不及则未至,其为戕害一也。其为养甚大者尚然,而况不为养而为害之物,焉可以不致其慎哉!此特其养口体者耳。若夫君子动息威仪,起居出入,皆当有其养焉,又所以养德也。尝观前元《饮膳正要》一书,其所以养口体、养德之要,无所不载,盖当时尚医所论著。其执艺事,以致忠爱,虽深于圣贤之道者不外是也。夫善莫大于取诸人,取诸人以为善,大舜所先肆\footnote{勤苦,极力。《文选·陆机·演连珠》:“轮匠肆目,不乏奚仲之妙;瞽史清耳,而无伶伦之察。”}朕嘉是书而用之,以资摄养之助,且锓诸梓,以广惠利于人,亦庶几乎,好生之仁。虽然生禀于天,非人之所能为,若或戕之,与立岩墙\footnote{将要倒塌的墙,借指危险之地。《孟子·尽心上》:“是故知命者不立乎岩墙之下。”朱熹集注:“岩墙,墙之将覆者。”}之下者同,有不由于人乎!故此非但摄养之助,而抑顺受其正之大助也。
\end{yuanwen}

\begin{shuming}
景泰七年四月初一日
\end{shuming}

~\\

\begin{yuanwen}
景泰一序,原书已佚,初版未获印入,殊为缺憾,嗣从瞿氏借得,今当重印,因以冠诸卷端,读者鉴之。
\end{yuanwen}

\begin{shuming}
元济再识
\end{shuming}

\chapter{虞序}

\begin{yuanwen}
臣闻古之君子善修其身者,动息节宣以养生,饮食衣服以养体,威仪行义以养德,是故周公之制礼也。天子之起居、衣服、饮食,各有其官,皆统于冢宰\footnote{古代官名,即后来所称之宰相。冢,zhǒng。},盖慎之至也。今上皇帝,天纵圣明,文思深远,御\footnote{御,治理、管理,《诗经·大雅·思齐》:“以御于家邦。”}延阁\footnote{帝王藏书之处。},阅图书,旦暮有恒,则尊养德性,以酬酢万几,得内圣外王之道焉。于是赵国公臣常普兰奚,以所领膳医臣忽思慧所撰《饮膳正要》以进。其言曰:昔世祖皇帝,食饮必稽于本草,动静必准乎法度,是以身跻上寿,贻子孙无疆之福焉。是书也,当时尚医之论著者云,噫!进书者可谓能执其艺事,以致其忠爱者矣。是书进上,中宫览焉。念祖宗卫生之戒,知臣下陈义之勤,思有以助圣上之诚身,而推其仁民之至意。命中政院使臣拜住刻梓而广传之。兹举也,盖欲推一人之安,而使天下之人举安;推一人之寿,而使天下之人皆寿。恩泽之浓,岂有加于此者哉!书之既成,大都留守臣金界奴传勅\footnote{chì,同敕,帝王的诏书、命令。}命臣集序其端云。臣集再拜稽首而言曰:臣闻《易》之《传》有云“大哉乾元,万物资始”,“至哉坤元,万物资生”;天地之大德,不过生生而已耳。今圣皇正统于上,乾道也;圣后顺承于中,坤道也。乾坤道备,于斯为盛,斯民斯物之生于斯时也,何其幸欤!愿飏言\footnote{高声朗朗地讲话。多用于臣下奏辞。《书·益稷》:“皋陶拜手稽首,飏言曰:‘念哉!’”孔传:“大言而疾曰飏。”飏,yánɡ 。}之,使天下后世有以知。夫高明博厚之可见如此,於戏\footnote{於戏,wūhū ,叹词,读呜乎,不能简写成于戏。}休哉。
\end{yuanwen}

\begin{shuming}
天历三年五月朔日谨序
\end{shuming}

\begin{shuming}
奎章阁侍书学士翰林直学士中奉
\end{shuming}

\begin{shuming}
大夫知制诰同修国史臣虞集譔
\end{shuming}

\chapter{自序}

\begin{yuanwen}
伏睹国朝,奄有四海,遐迩罔不宾贡。珍味奇品,咸萃内府,或风土有所未宜,或燥湿不能相济,倘司庖厨者,不能察其性味而概于进献,则食之恐不免于致疾。钦惟世祖皇帝圣明,按《周礼·天官》有师医\footnote{据《周礼·天官》内容应改为“医师”。}、食医、疾医、疡医,分职而治。行依典故,设掌饮膳太医四人。于本草内选无毒、无相反,可久食,补益药味,与饮食相宜,调和五味,及每日所造珍品,御膳必须精制。所职何人,所用何物,进酒之时,必用沉香木、沙金、水晶等盏。斟酌适中,执事务合称职。每日所用,詄\footnote{与标通,意标注。}注于历,以验后效。至于汤煎、琼玉、黄精、天门冬、苍术等膏,牛髓、枸杞等煎,诸珍异馔,咸得其宜。以此世祖皇帝圣寿延永无疾。恭惟皇帝陛下自登宝位,国事繁重,万机之暇,遵依祖宗定制,如补养调护之术,饮食百味之宜,进加日新,则圣躬万安矣。臣思慧自延祐年间选充饮膳之职,于兹有年,久叨天禄,退思无以补报,敢不竭尽忠诚,以答洪恩之万一。是以日有余闲,与赵国公臣普兰奚,将累朝亲侍进用奇珍异馔,汤膏煎造,及诸家本草,名医方术,并日所必用谷肉果菜,取其性味补益者,集成一书,名曰《饮膳正要》,分为三卷。本草有未收者,今即采摭附写。伏望陛下恕其狂妄,察其愚忠,以燕\footnote{古时通“宴”,安闲,安逸。《史纪·万石君传》:“虽燕居必冠。”}闲之际,鉴先圣之保摄,顺当时之气候,弃虚取实,期以获安,则圣寿跻于无疆,而四海咸蒙其德泽矣。谨献所述《饮膳正要》一集,以闻,伏乞圣览下情,不胜战眎激切屏营\footnote{惶恐的样子。《国语·吴语》:“王亲独行,屏营彷徨于山林之中。”屏,bīng。}之至。
\end{yuanwen}

\begin{shuming}
天历三年三月三日饮膳太医臣忽思慧进上
\end{shuming}

\begin{shuming}
中奉大夫太医院使臣耿允谦校正
\end{shuming}

\begin{shuming}
奎章阁都主管上事资政大夫大留守内宰隆祥总管
\end{shuming}

\begin{shuming}
提调织染杂造人匠都总管府事臣张金界奴校正
\end{shuming}

\begin{shuming}
资德大夫中政院使储政院使臣拜住校正
\end{shuming}

\begin{shuming}
集贤大学士银青荣禄大夫赵国公臣常普兰奚编集
\end{shuming}

\chapter{前言}

\begin{yuanwen}
天之所生,地之所养,天地合气,人以禀天地气生,并而为三才。三才者,天地人。人而有生,所重乎者心也。心为一身之主宰,万事之根本,故身安则心能应万变,主宰万事,非保养何以能安其身。保养之法,莫若守中,守中则无过与不及之病。调顺四时,节慎饮食,起居不妄,使以五味调和五脏。五脏和平则血气资荣,精神健爽,心志安定,===============================================================================================================================================================================================================================
诸邪自不能入,寒暑不能袭,人乃怡安。夫上古圣人治未病不治已病,
故重食轻货,盖有所取也。故云:食不厌精,鲙(1)不厌细。鱼馁(2)肉
败者,色恶者,臭恶者,失饪不时者,皆不可食。然虽食饮,非圣人口
腹之欲哉!盖以养气养体,不以有伤也。若食气相恶则伤精,若食味不
调则损形。形受五味以成体,是以圣人先用食禁以存性,后制药以防
命。盖以药性有大毒,有大毒者治病,十去其六;常毒治病,十去其
七;小毒治病,十去其八;无毒治病,十去其九。然后谷肉果菜,十养
一尽之,无使过之,是以
伤其正。虽饮食百味,要其精粹,审其有补益助养之宜,新陈之
异,温凉寒热之性,五味偏走之病。若滋味偏嗜,新陈不择,制造失
度,俱皆致疾。可者行之,不可者忌之。如妊妇不慎行,乳母不忌口,
则子受患。若贪爽口而忘避忌,则疾病潜生,而中(3)不悟,百年之身,
而忘于一时之味,其可惜哉!孙思邈曰:谓其医者,先晓病源,知其所
犯,先以食疗,不瘥,然后命药,十去其九。故善养生者,谨先行之。
摄生之法,岂不为有裕(4)矣。
【注释】
(1)鲙:(kuài,快)一作脍(脍),细切的肉或鱼。《诗经·小雅·
六月》:
“饮御诸友,糓鳖脍鲤。”
(2)馁:(nèi,内)指鱼腐烂,不新鲜。《论语·乡党》:
“鱼馁而肉
败。”
(3)中:再意也。
(4)裕:道、径也。《方言》曰:
“裕、猷,道也。”	
\end{yuanwen}

\mainmatter

\chapter{}





\begin{yuanwen}

\end{yuanwen}\begin{yuanwen}

\end{yuanwen}\begin{yuanwen}

\end{yuanwen}\begin{yuanwen}

\end{yuanwen}\begin{yuanwen}

\end{yuanwen}\begin{yuanwen}

\end{yuanwen}\begin{yuanwen}

\end{yuanwen}\begin{yuanwen}

\end{yuanwen}\begin{yuanwen}

\end{yuanwen}\begin{yuanwen}

\end{yuanwen}\begin{yuanwen}

\end{yuanwen}\begin{yuanwen}

\end{yuanwen}\begin{yuanwen}

\end{yuanwen}\begin{yuanwen}

\end{yuanwen}\begin{yuanwen}

\end{yuanwen}\begin{yuanwen}

\end{yuanwen}\begin{yuanwen}

\end{yuanwen}\begin{yuanwen}

\end{yuanwen}\begin{yuanwen}

\end{yuanwen}\begin{yuanwen}

\end{yuanwen}\begin{yuanwen}

\end{yuanwen}\begin{yuanwen}

\end{yuanwen}\begin{yuanwen}

\end{yuanwen}\begin{yuanwen}

\end{yuanwen}\begin{yuanwen}

\end{yuanwen}\begin{yuanwen}

\end{yuanwen}\begin{yuanwen}

\end{yuanwen}\begin{yuanwen}

\end{yuanwen}\begin{yuanwen}

\end{yuanwen}\begin{yuanwen}

\end{yuanwen}\begin{yuanwen}

\end{yuanwen}\begin{yuanwen}

\end{yuanwen}\begin{yuanwen}

\end{yuanwen}\begin{yuanwen}

\end{yuanwen}\begin{yuanwen}

\end{yuanwen}\begin{yuanwen}

\end{yuanwen}\begin{yuanwen}

\end{yuanwen}\begin{yuanwen}

\end{yuanwen}\begin{yuanwen}

\end{yuanwen}\begin{yuanwen}

\end{yuanwen}\begin{yuanwen}

\end{yuanwen}



卷 一
三皇圣纪
太昊伏羲氏 风姓之源,皇熊氏之后。生有圣德,继天而王,为万
世帝王之先。位在东方,以木德王,为苍精之君。都陈时,神龙(1)出于
荥河(2),则而画之为八卦。造书契,以代结绳(3)之政,立五常,定五
行,正君臣,明父子,别夫妇之义,制嫁娶之理。造屋舍,结网 (4),
以佃渔(5),服牛乘马,引重致远。取牺牲(6),供祭祀,故曰伏羲氏。
治天下一百一十年。
炎帝神农氏 姜姓之源,烈山氏之后。生有圣德,以火承木,位在
南方,以火德王,为赤精之君。时人民
茹草饮水,采树木之实,而食蠃蛖(7)之肉,多生疾病,乃求可食之
物,尝百草,种五谷,以养人民。日中为市。作陶冶(8),为斧斤(9),
造耒耜(10),教民耕稼,故曰神农。都曲阜。治天下一百二十年。
黄帝轩辕氏 姬姓之源,有熊国君少典之子。生而神灵,长而聪
明,成而登天。以土德王,为黄精之君,故曰黄帝。都涿鹿。受河图
(11),见日月星辰之象,始有星官之书。命大挠(12)探五行之情,占斗
罡(13)所建,
始作甲子;命容成(14)作历;命隶首(15)作筭(16)数;命伶伦(17)造律
吕(18);命岐伯定医方。为衣冠以表贵贱,治干戈,作舟车,分州野,
治天下一百年。
【注释】
(1)神龙:古以龙为神物,称龙为神龙。《史记·三皇记》:
“有娲氏
之女,为少典妃,感神龙而生炎帝。”
(2)荥河:疑为荥泽,古泽名《尚书·禹贡》。孔传:
“荥泽之水,已
成遏猪”,又“导禮水,东流为济,入于河,溢为荥。”
(3)结绳:上古无文字,结绳以记事。《易·系辞下》:
“上古结绳而
治,后世圣人易之以书契。”
(4)网 (gǔ,古):古代打猎捕鱼的工具。
(5)佃(tián,田)渔:猎兽和捕鱼。佃,通“畋”。
(6)牺牲:供祭祀用的纯色全体牲畜。
(7)蠃蛖(luó bàng,罗棒):蠃,同螺,《国语·吴语》:
“其民必
移就蒲蠃于东海之滨。”韦昭注:
“蠃,蚌蛤之属。”蛖,即蚌。《淮
南子·说林训》:
“蛖象之病,人之宝也。”高诱注:
“蛖,大
蛤。”《淮南子》:
“月死而蠃膲。”按,蛖与蚌同。
(8)陶冶:谓烧制陶器和冶炼金属,《荀子·王制》:
“故泽人足乎
木,山人足乎鱼,农夫不觔削、不陶冶而足械用,工贾不耕田而足菽
粟。”
(9)斧斤:泛指各种斧子。《孟子·梁惠王上》:
“斧斤以时入山林,
材木不可胜用也。”
(10)耒耜(leì sì,累四):古代耕地翻土的农具。耒是耒耜的柄,耜
是耒耜下端的起土部分。《礼记·月令》:
“(孟春之月)天子亲载耒
耜,措之于参保介之御间。”可泛指农具。
(11)河图:《周易》八卦。孔传:
“伏牺王天下,龙马出河,遂则其文
以画八卦,谓之河图。”
(12)大挠:传说为黄帝史官,始作甲子。《吕氏春秋·尊师》:
“黄帝
师大挠。”
(13)斗罡(gāng,刚):星名,即北斗七星的柄。晋代葛洪的《抱朴子
·地真》:
“左罡右魁。”《西游记》第四十四回:
“查罡布斗,香烟
馥郁透清霄。”
(14)容成:相传为黄帝大臣,发明历法。
(15)隶首:黄帝史官,始作算数。清代黄遵宪的《杂感》诗:
“隶首不
能算,知有几万年。”
(16)筭(suàn,算):亦作“笇”。古代计数的筹码。《说文·竹
部》:
“筭,长六寸,计历数者。从竹从弄,言常弄乃不误也。”
(17)伶伦:传说为黄帝时的乐官。古以为乐律的创始者。《吕氏春秋·
古乐》:
“昔黄帝令伶伦作为律。”
(18)律吕:古代校正乐律的器具。用竹管或金属管制成,共十二管,管
径相等,以管的长短来确定音的不同高度。从低音管算起,成奇数的六
个管叫做“律”
;成偶数的六个管叫做“吕”,合称“律吕”。后亦用
以指乐律或音律。
养生避忌
夫上古之人,其知道者,法于阴阳,和于术数(1),食饮有节,起居
有常,不妄作劳,故能而寿。今时之人不然也,起居无常,饮食不知忌
避,亦不慎节,多嗜欲,厚滋味,不能守中,不知持满,故半百衰者多
矣。夫安乐之道,在乎保养,保养之道,莫若守中,守中则无过与不及
之病。春秋冬夏,四时阴阳,生病起于过与,盖不适其性而强。故养生
者,既无过耗之弊,又能保守真元,何患乎外邪所中也。故善服药者,
不若善保养;不善保养,不若善服药。世有不善保养,又不能善
服药,仓卒病生,而归咎于神天乎!善摄生者,薄滋味,省思虑,
节嗜欲,戒喜怒,惜元气,简言语,轻得失,破忧阻,除妄想,远好
恶,收视听,勤内固,不劳神,不劳形,神形既安,病患何由而致也。
故善养性者,先饥而食,食勿令饱;先渴而饮,饮勿令过。食欲数而
少,不欲顿而多。盖饱中饥,饥中饱,饱则伤肺,饥则伤气。若食饱,
不得便卧,即生百病。
凡热食有汗,勿当风,发痉病,头痛,目涩,多睡。夜不可多食,
卧不可有邪风。
凡食讫温水漱口,令人无齿疾、口臭。汗出时,不可扇,生偏枯。
勿向西北大小便。勿忍大小便,令人成膝劳、冷痹痛。勿向星辰、日
月、神堂、庙宇大小便。夜行,勿歌唱大叫。一日之忌,暮勿饱食;一
月之忌,晦(2)勿大醉;一岁之忌,暮勿远行;终身之忌,勿燃灯房事。
服药千朝,不若独眠一宿。如本命日,及父母本命日,不食本命所属
肉。
凡人坐,必要端坐,使正其心;凡人立,必要正立,使直其身。立
不可久,立伤骨;坐不可久,坐伤血;行不可久,行伤筋;卧不可久,
卧伤气;视不可久,视伤神。食饱勿洗头,生风疾。如患目赤病,切忌
房事,不然令人生内障。沐浴勿当风,腠理百窍皆开,切忌邪风易入。
不可登高履险,奔走车马,气乱神惊,魂魄飞散。
大风、大雨,大寒、大热,不可出入妄为。口勿吹灯火,损气。
凡日光射,勿凝视,损人目。勿望远,极目观,损眼力。坐卧勿当
风、湿地。夜勿燃灯睡,魂魄不守。昼勿睡,损元气。食勿言,寝勿
语,恐伤气。
凡遇神堂、庙宇,勿得辄入。
凡遇风雨雷电,必须闭门,端坐焚香,恐有诸神过。怒不可暴,怒
生气疾、恶疮。远唾不如近唾,近唾不如不唾。虎豹皮不可近肉铺,损
人目。
避色如避箭,避风如避雠(3),莫吃空心(4)茶,少食申(5)后粥。
古人有云:入广(6)者,朝不可虚,暮不可实。然不独广,凡早皆忌
空腹。古人云:烂煮面,软煮肉,少饮酒,独自宿。古人平日起居而摄
养,今人待老而保生,盖无益。
凡夜卧,两手摩令热,揉眼,永无眼疾。凡夜卧,两手摩令热,摩
面,不生疮 (7)。一呵十搓,一搓十摩,久而行之,皱少颜多。凡清
旦,以热水洗目,平日无眼疾。凡清旦刷牙,不如夜刷牙,齿疾不生。
凡清旦盐刷牙,平日无齿疾。凡夜卧,被发梳百通,平日头风少。
凡夜卧,濯足而卧,四肢无冷疾。盛热来,不可冷水洗面,生目
疾。
凡枯木大树下,久阴湿地,不可久坐,恐阴气触人。立秋日,不可
澡浴,令人皮肤粗糙,因生白屑。常默,元气不伤;少思,慧烛内光;
不怒,百神安畅;不恼,心地清凉;乐不可极,欲不可纵。
【注释】
(1)术数:谓以种种方术,观察可注意的现象,来推测人的气数和命
运。也称“数术”。
(2)晦:农历每月的最后一日。《春秋·僖公十五年》:
“己卯晦,震
夷伯之庙。”杨伯峻注:
“己卯,九月三十日。”
(3)雠(chóu,仇):亦作“讐”。仇敌之意。宋代叶适的《齐云楼》
诗:
“项梁起雠秦,子弟奋投袂。”
(4)空心:意指空腹。
(5)申:指申时,下午3~5时。又指傍晚。
(6)广:通旷;意空旷,开阔。《老子》:
“旷兮其若谷,混兮其若
浊。”
(7) (gān,杆): 通筎,《通俗文》:
“面黧黑曰筎。”
妊娠食忌
上古圣人有胎教之法,古者妇人妊子,寝不侧,坐不边,立不跸
(1)。不食邪味,割(2)不正不食,席不正不坐,目不视邪色,耳不听淫
声,夜则令瞽(3)诵诗,道正事,如此则生子形容端正,才过人矣。故太
任生文王,聪明圣哲,闻一而知百,皆胎教之能也。圣人多感生,妊娠
故忌见丧孝、破体、残疾、贫穷之人;宜见贤良、喜庆、美丽之事。欲
子多智,观看鲤鱼、孔雀;欲子美丽,观看珍珠、美玉;欲子雄壮,观
看飞鹰、走犬。如此善恶犹感,况饮食不知避忌乎。
妊娠所忌:食兔肉,令子无声缺唇。食山羊肉,令子多疾。食鸡
子、干鱼,令子多疮。
食桑椹、鸭子,令子倒生。食雀肉,饮酒,令子心淫情乱,不顾羞
耻。食鸡肉、糯米,令子生寸白虫。食雀肉、豆酱,令子面生 黯。食
鳖肉,令子项短。食驴肉,令子延月。食冰浆,绝产。食骡肉,令子难
产。
【注释】
(1)跸(bì,毕):站立不正。汉代刘向的《列女传·周室三
母》:
“古者妇人妊子,寝不侧,坐不边,立不跸。”
(2)割:用刀分解牲畜的骨肉。《周礼·天官·内饔》:
“掌王及后世
子膳羞之割亨煎和之事。”郑玄注:
“割,肆解肉也。”
(3)瞽(gǔ,鼓):失明。《韩诗外传》卷五:
“两瞽相扶,不触墙
木,不陷井穽,则其幸也。”
乳母食忌
凡生子择于诸母,必求其年壮,无疾病,慈善,性质宽裕,温良详
雅,寡言者,使为乳母。子在于母资乳以养,亦大人之饮食也。善恶相
习,况乳食不遂母性。若子有病无病,亦在乳母之慎口。如饮食不知避
忌,倘不慎行,贪爽口而忘身适性致疾,使子受患,是母令子生病矣。
乳母杂忌:夏勿热暑乳,则子偏阳而多呕逆。冬勿寒冷乳,则子偏
阴而多咳痢。母不欲多怒,怒则气逆,乳之令子癫狂。母不欲醉,醉则
发阳,乳之令子身热腹满。母若吐时,则中虚,乳之令子虚羸。母有积
热,盖赤黄为热,乳之令子变黄不食。新房事劳伤,乳之令子瘦痒,交
胫不能行。母勿太饱乳之,母勿太饥乳之,母勿太寒乳之,母勿太热乳
之。子有泻痢、腹痛、夜啼疾,乳母忌食寒凉发病之物。子有积热、惊
风、疮疡,乳母忌食湿热、动风之物。子有疥癣、疮疾,乳母忌食鱼、
虾、鸡、马肉、发疮之物。子有癖、疳、瘦疾,乳母忌食生茄、黄瓜等
物。
凡初生儿时,以未啼之前,用黄连浸汁,调朱砂少许,微抹口内,
去胎热邪气,令疮疹稀少。凡初生儿时,用荆芥、黄连熬水,入野牙猪
(1)胆汁少许,洗儿。在后虽生斑疹、恶疮,终当稀少。凡小儿未生疮疹
时,用腊月兔头并毛骨,同水煎汤,洗儿,除热去毒,能令斑疹、诸疮
不生,虽有亦稀少。凡小儿未生斑疹时,以黑子母驴乳令饮之,及长不
生疮疹、诸毒。如生者,亦稀少。仍治小儿心热、风痫(2)。
【注释】
(1)野牙猪:即野猪。
(2)风痫:指癫痫因风而发者。
饮酒避忌
酒,味苦甘辛,大热,有毒。主行药势,杀百邪,去恶气,通血
脉,厚肠胃,润肌肤,消忧愁。少饮尤佳,多饮伤神损寿,易人本性,
其毒甚也。醉饮过度,丧生之源。
饮酒不欲使多,知其过多,速吐之为佳,不尔成痰疾。醉勿酩酊大
醉,即终身百病不除。酒,不可久饮,恐腐烂肠胃,渍髓,蒸筋。
醉不可当风卧,生风疾。醉不可向阳卧,令人发狂。醉不可令人
扇,生偏枯。醉不可露卧,生冷痹。醉而出汗当风,为漏风。醉不可卧
黍,生癞疾。醉不可强食、嗔怒,生痈疽。醉不可走马及跳踯,伤筋
骨。醉不可接房事,小者面生 、咳嗽,大者伤脏、澼、痔疾。醉不可
冷水洗面,生疮。醉,醒不可再投,损后又损。醉不可高呼、大怒,令
人生气疾。晦勿大醉,忌月空(1)。醉不可饮酪水,成噎病。醉不可便
卧,面生疮疖,内生积聚。大醉勿燃灯叫,恐魂魄飞扬不守。醉不可饮
冷浆水,失声成尸噎。
饮酒,酒浆照不见人影勿饮。醉不可忍小便,成癃闭、膝劳、冷
痹。空心饮酒,醉必呕吐。醉不可忍大便,生肠澼、痔。酒忌诸甜物。
酒醉不可食猪肉,生风。醉不可强举力,伤筋损力。饮酒时,大不可食
猪、羊脑,大损人,炼真之士尤宜忌。酒醉不可当风乘凉、露脚,多生
脚气。醉不可卧湿地,伤筋骨,生冷痹痛。醉不可澡浴,多生眼目之
疾。如患眼疾人,切忌醉酒、食蒜。
【注释】
(1)月空:丛辰名。月中阳辰。指寅辰戌月之壬日,辛卯未月之庚日,
申子辰月之丙日,己酉丒月之甲日。
聚珍异馔
马思荅吉(1)汤 补益,温中,顺气。
羊肉一脚子(2),卸成事件(3) 草果五个 官桂二钱 回回豆
子(4)半升,捣碎,去皮
上件,一同熬成汤,滤净,下熟回回豆子二合,香粳米一升,马思
荅吉一钱,盐少许,调和匀,下事件肉、芫荽叶。
大麦汤 温中下气,壮脾胃,止烦渴,破冷气,去腹胀。
羊肉一脚子,卸成事件 草果五个 大麦仁二升,滚水淘洗净,微
煮熟
上件,熬成汤,滤净,下大麦仁,熬熟,盐少许,调和令匀,下事
件肉。
八儿不汤 系西天(5)茶饭名。补中,下气,宽胸膈。
羊肉一脚子,卸成事件 草果五个 回回豆子半升,捣碎,去皮萝
卜二个
上件,一同熬成汤,滤净,汤内下羊肉,切如色数(6)大,熟萝卜切
如色数大,咱夫兰(7)一钱,姜黄二钱,胡椒二钱,哈昔呢(8)半钱,芫
荽叶、盐少许,调和匀,对香粳米干饭食之,入醋少许。
沙乞某儿(9)汤 补中,下气,和脾胃。
羊肉一脚子,卸成事件 草果五个 回回豆子半升,捣碎,去皮沙
乞某儿五个,系蔓菁
上件,一同熬成汤,滤净,下熟回回豆子二合,香粳米一升。熟沙
乞某儿切如色数大,下事件肉,盐少许,调和令匀。
苦豆汤 补下元,理腰膝,温中,顺气。
羊肉一脚子,卸成事件 草果五个 苦豆一两,系胡芦巴
上件,一同熬成汤,滤净,下河西兀麻食(10)或米心
(11)子,哈昔呢半钱,盐少许,调和。
木瓜汤 补中,顺气。治腰膝疼痛,脚气不仁。
羊肉一脚子,卸成事件 草果五个 回回豆子半升,捣碎,去皮
上件,一同熬成汤,滤净,下香粳米一升,熟回回豆子二合,肉弹
儿木瓜二斤,取汁,沙糖四两,盐少许,调和,或下事件肉。
鹿头汤 补益,止烦渴。治脚膝疼痛。
鹿头蹄一副,退洗净,卸作块
上件,用哈昔呢豆子大,研如泥,与鹿头蹄肉同拌匀,用回回小油
四两同炒,入滚水熬令软,下胡椒三钱,哈昔呢二钱,荜拨一钱,牛奶
子一盏,生姜汁一合,盐少许,调和。一法用鹿尾取汁,入姜末、盐,
同调和。
松黄(12)汤 补中,益气,壮筋骨。
羊肉一脚子,卸成事件 草果五个 回回豆子半升,捣碎,去皮
上件,同熬成汤,滤净,熟羊胸子一个,切作色数大,松黄汁二
合,生姜汁半合,一同下炒,葱、盐、醋、芫荽叶,调和匀。对经卷儿
食之。
粆(13)汤 补中,益气,健脾胃。
羊肉一脚子,卸成事件 草果五个 回回豆子半升,去皮
上件,同熬成汤,滤净,熟干羊胸子一个,切片,炒三升,白菜或
荨麻菜,一同下锅,盐调和匀。
大麦筭(14)子粉 补中,益气,健脾胃。
羊肉一脚子,卸成事件 草果五个 回回豆子半升,去皮
上件,同熬成汤,滤净,大麦粉三斤,豆粉一斤,同作粉。羊肉炒
细乞马(15),生姜汁二合,芫荽叶、盐、醋调和。
大麦片粉 补中,益气,健脾胃。
羊肉一脚子,卸成事件 草果五个 良姜二钱
上件,同熬成汤,滤净,下羊肝酱,取清汁,胡椒五钱,熟羊肉切
作甲叶,糟姜二两,瓜虀(16)一两,切如甲叶,盐、醋调和,或浑汁亦
可。
糯米粉搊粉 补中益气。
羊肉一脚子,卸成事件 草果五个 良姜二钱
上件,同熬成汤,滤净,用羊肝酱熬取清汁,下胡椒五钱,糯米粉
二斤,与豆粉一斤,同作搊粉,羊肉切细乞马,入盐、醋调和,浑汁亦
可。
河 羹 补中益气。
羊肉一脚子,卸成事件 草果五个
上件,同熬成汤,滤净,用羊肉切细乞马,陈皮五钱,去白,葱二
两,细切,料物二钱,盐、酱拌馅儿,皮用白面三斤,作河豚,小油炸
熟,下汤内,入盐调和,或清汁亦可。
阿菜汤 补中益气。
羊肉一脚子,卸成事件 草果五个 良姜二钱
上件,同熬成汤,滤净,下羊肝酱,同取清汁,入胡椒五钱。另羊
肉切片,羊尾子一个,羊舌一个,羊腰子一副,各切甲叶;蘑菇二两,
与白菜一同下,清汁、盐、醋调和。
鸡头粉雀舌 子 补中,益精气。
羊肉一脚子,卸成事件 草果五个回 回豆子半升,捣碎,去皮
上件,同熬成汤,滤净,用鸡头粉二斤,豆粉一斤,同和,切作食
其子,羊肉切细乞马,生姜汁一合炒,葱调和。
鸡头粉血粉 补中,益精气。
羊肉一脚子,卸成事件 草果五个回 回豆子半升,捣碎,去皮
上件,同熬成汤,滤净,用鸡头粉二斤,豆粉一斤,羊血和作搊
粉,羊肉切细乞马炒,葱、醋一同调和。
鸡头粉撧面 补中,益精气。
羊肉一脚子,卸成事件 草果五个回 回豆子半升,捣碎,去皮
上件,同熬成汤,滤净,用鸡头粉二斤,豆粉一斤,白面一斤,同
作面。羊肉切片儿乞马入炒,葱、醋一同调和。
鸡头粉搊粉 补中,益精气。
羊肉一脚子,卸成事件 草果五个良 姜二钱
上件,同熬成汤,滤净,用羊肝酱同取清汁,入胡椒一两,次用鸡
头粉二斤,豆粉一斤,同作搊粉,羊肉切细乞马,下盐、醋调和。
鸡头粉馄钝(17) 补中益气。
羊肉一脚子,卸成事件 草果五个回 回豆子半升,捣碎,去皮
上件,同熬成汤,滤净,用羊肉切作馅,下陈皮一钱,去白,生姜
一钱,细切,五味和匀,次用鸡头粉二斤,豆粉一斤,作枕头馄饨。汤
内下香粳米一升,回回豆子二合,生姜汁二合,木瓜汁一合,同炒,
葱、盐匀调和。
杂羹 补中益气。
羊肉一脚子,卸成事件 草果五个回 回豆子半升,捣碎,去皮
上件,同熬成汤,滤净,羊头洗净二个,羊肚、肺各二具,羊白血
双肠儿一副,并煮熟切,次用豆粉三斤,作粉,蘑菇半斤,杏泥半斤,
胡椒一两,入青菜、芫荽炒,葱、盐、醋调和。
荤素羹 补中益气。
羊肉一脚子,卸成事件 草果五个回 回豆子半升,捣碎,去皮
上件,同熬成汤,滤净,豆粉三斤,作片粉,精羊肉切条道乞马,
山药一斤,糟姜二块,瓜虀一块,乳饼一个,胡萝卜十个,蘑菇半斤,
生姜四两,各切,鸡子十个,打煎饼,切,用麻泥一斤,杏泥半斤,同
炒,葱、盐、醋调和。
珍珠粉 补中益气。
羊肉一脚子,卸成事件 草果五个回 回豆子半升,捣碎,去皮
上件,同熬成汤,滤净,羊肉切乞,马心、肝、肚、肺各一具,生
姜二两,糟姜四两,瓜虀一两,胡萝卜十个,山药一斤,乳饼一个,鸡
子十个,作煎饼,各切,次用麻泥一斤,同炒,葱、盐、醋调和。
黄汤 补中益气。
羊肉一脚子,卸成事件 草果五个回 回豆子半升,捣碎,去皮
上件,同熬成汤,滤净,下熟回回豆子二合,香粳米一升,胡萝卜
五个,切,用羊后脚肉丸肉弹儿,肋枝一个,切,寸金姜黄三钱,姜末
五钱,咱夫兰一钱,芫荽叶同盐、醋调和。
三下锅 补中益气。
羊肉一脚子,卸成事件 草果五个良 姜二钱
上件,同熬成汤,滤净,用羊后脚肉丸肉弹儿,丁头 子,羊肉指
甲匾食,胡椒一两,同盐、醋调和。
葵菜羹 顺气。治癃闭不通。性寒,不可多食。今与诸物同制造,
其性稍温。
羊肉一脚子,卸成事件 草果五个良 姜二钱
上件,同熬成汤,熟羊肚、肺各一具,切,蘑菇半斤,切,胡椒五
钱,白面一斤,拌鸡爪面,下葵菜炒,葱、盐、醋调和。
瓠子(18)汤 性寒。主消渴,利水道。
羊肉一脚子,卸成事件 草果五个
上件,同熬成汤,滤净,用瓠子六个,去皮,切掠,熟羊肉,切
片,生姜汁半合,白面二两,作面丝同炒,葱、盐、醋调和。
团鱼汤主伤中。益气,补不足。
羊肉一脚子,卸成事件 草果五个
上件,熬成汤,滤净,团鱼五六个,煮熟,去皮、骨,切作块,用
面二两,作面丝,生姜汁一合,胡椒一两,同炒,葱、盐、醋调和。
盏蒸 补中益气。
挦(19)羊背皮或羊肉三脚子,卸成事件 草果五个良 姜二钱陈皮
二钱,去白小椒二钱
上件,用杏泥一斤,松黄二合,生姜汁二合,同炒,葱、盐五味调
匀,入盏内蒸,令软熟,对经卷儿食之。
台苗羹 补中益气。
羊肉一脚子,卸成事件 草果五个良 姜二钱
上件,熬成汤,滤净,用羊肝下酱,取清汁,豆粉五斤,作粉,乳
饼一个,山药一斤,胡萝卜十个,羊尾子一个,羊肉等,各切细,入台
子菜、鮲菜、胡椒一两,盐、醋调和。
熊汤 治风痹不仁,脚气。
熊肉二脚子,煮熟,切块 草果三个
上件,用胡椒三钱,哈昔呢一钱,姜黄二钱,缩砂二钱,咱夫兰一
钱,葱、盐、酱一同调和。
鲤鱼汤治黄疸。止渴,安胎。有宿瘕者,不可食之。
大新鲤鱼十头,去鳞肚,洗净小椒末五钱
上件,用芫荽末五钱,葱二两,切,酒少许,盐一同淹(20),拌清
汁内,下鱼,次下胡椒末五钱,生姜末三钱,荜拨末三钱,盐、醋调
和。
炒狼汤古本草不载狼肉,今云性热,治虚弱。然食之末闻有毒。今
制造用料物以助其味,暖五脏,温中。
狼肉一脚子,卸成事件草果三个胡椒五钱哈昔呢一钱荜拨二钱缩砂
二钱姜黄二钱咱夫兰一钱
上件,熬成汤,用葱、酱、盐、醋一同调和。
围像 补益五脏。
羊肉一脚子,煮熟,切细羊尾子二个,熟,切细藕二枝蒲笋二斤黄
瓜五个生姜半斤乳饼二个糟姜四两瓜虀半斤鸡子十个,煎作饼蘑菇一斤
蔓菁菜、鮲菜各切条道
上件,用好肉汤,调麻泥二斤、姜末半斤,同炒。葱、盐、醋、调
和,对胡饼食之。
春盘面 补中益气。
白面六斤,切细面羊肉二脚子,煮熟,切条道乞马羊肚肺各一个,
煮熟,切鸡子五个,煎作饼,裁鏣生姜四两,切鮲黄半斤蘑菇四两台子
菜蓼牙(21)胭脂
上件,用清汁下胡椒一两,盐、醋调和。
皂羹面 补中益气。
白面六斤,切细面羊胸子二个,退洗净,煮熟,切如色数块
上件,用红麹(22)三钱,淹拌,熬令软,同入清汁内,下胡椒一
两,盐、醋调和。
山药面 补虚羸,益元气。
白面六斤鸡子十个,取白生姜汁二合豆粉四两
上件,用山药三斤,煮熟,研泥,同和面,羊肉二脚子,切丁头乞
马,用好肉汤下炒,葱、盐调和。
挂面 补中益气。
羊肉一脚子,切细乞马挂面六斤蘑菇半斤,洗净,切鸡子五个,煎
作饼糟姜一两,切瓜虀一两,切
上件,用清汁,下胡椒一两,盐、醋调和。
经带面 补中益气。
羊肉一脚子,炒焦肉乞马蘑菇半斤,洗净,切
上件,用清汁,下胡椒一两,盐、醋调和。
羊皮面 补中益气。
羊皮二个,挦洗净,煮软 羊舌二个,熟 羊腰子四个,熟,各切
如甲叶蘑菇一斤,洗净糟姜四两,各切如甲叶
上件,用好肉酽汤或清汁,下胡椒一两,盐、醋调和。
秃秃麻食 系手撇面。补中益气。
白面六斤,作秃秃麻食羊肉一脚子,炒焦肉乞马
上件,用好肉汤下炒,葱调和匀,下蒜酪、香菜末。
细水滑绢边水滑一同。补中益气。
白面六斤,作水滑羊肉二脚子,炒焦肉乞马鸡儿一个,熟,切丝蘑
菇半斤,洗净,切
上件,用清汁,下胡椒一两,盐、醋调和。
水龙 子 补中益气。
羊肉二脚子,熟,切作乞马 白面六斤,切作钱眼 子 鸡子十
个 山药一斤 糟姜四两 胡萝卜五个 瓜虀二两,各切细 三色弹儿
内一色肉弹儿,外二色粉,鸡子弹儿
上件,用清汁,下胡椒二两,盐、醋调和。
马乞 系手搓面。或糯米粉,鸡头粉亦可。补中益气。
白面六斤,作马乞羊肉二脚子,熟,切乞马
上件,用好肉汤炒,葱、醋、盐一同调和。
搠罗脱因 系畏兀儿茶饭。补中益气。
白面六斤,和,按作钱样羊肉二脚子,熟切 羊舌二个,熟切 山
药一斤 蘑菇半斤 胡萝卜五个 糟姜四两,切
上件,用好酽肉汤同下,炒,葱、醋调和。
乞马粥 补脾胃,益气力。
羊肉一脚子,卸成事件,熬成汤,滤净 粱米二升,淘洗净
上件,用精肉切碎乞马,先将米下汤内,次下乞马、米、葱、盐,
熬成粥,或下圆米,或折米,或渴米皆可。
汤粥 补脾胃,益肾气。
羊肉一脚子,卸成事件
上件,熬成汤,滤净,次下粱米三升,作粥熟,下米、葱、盐,或
下圆米、渴米、折米皆可。
粱米淡粥 补中益气。
粱米二升
上件,先将水滚过,澄清,滤净,次将米淘洗三至五遍,熬成粥,
或下圆米、渴米、折米皆可。
河西米(23)汤粥 补中益气。
羊肉一脚子,卸成事件河西米二升
上件,熬成汤,滤净,下河西米,淘洗净,次下细乞马、米、葱、
盐,同熬成粥,或不用乞马亦可。
撒速汤 系西天茶饭名。治元脏虚冷,腹内冷痛,腰脊酸疼。
羊肉二脚子,头蹄一副 草果四个 官桂三两 生姜半斤 哈昔呢
如回回豆子两个大
上件,用水一铁络,熬成汤,于石头锅内盛顿,下石榴子一斤,胡
椒二两,盐少许,炮石榴子用小油一杓,哈昔呢如豌豆一块,炒鹅黄色
微黑,汤末(24)子油去净,澄清,用甲香、甘松、哈昔呢、酥油烧烟薰
瓶,封贮任意。
炙羊心 治心气惊悸,郁结不乐。
羊心一个,带系桶咱夫兰三钱
上件,用玫瑰水一盏,浸取汁,入盐少许,签子签羊心,于火上
炙,将咱夫兰汁徐徐涂之,汁尽为度。食之安宁心气,令人多喜。
炙羊腰 治卒患腰眼疼痛者。
羊腰一对 咱夫兰一钱
上件,用玫瑰水一杓,浸取汁,入盐少许,签子签腰子火上炙。将
咱夫兰汁徐徐涂之,汁尽为度,食之。甚有效验。
攒鸡儿
肥鸡儿十个,挦洗净,熟切 攒生姜汁一合 葱二两,切姜末半
斤 小椒末四两 面二两,作面丝
上件,用煮鸡儿汤炒,葱、醋入姜汁调和。
炒鹌鹑
鹌鹑二十个,打(25)成事件 萝卜二个,切姜末四两 羊尾子一
个,各切如色数 面二两,作面丝
上件,用煮鹌鹑汤炒,葱、醋调和。
盘兔
兔儿二个,切作事件 萝卜二个,切羊尾子一个,切片 细料物二
钱
上件,用炒,葱、醋调和,下面丝二两,调和。
河西肺
羊肺一个 鮲六斤,取汁 面二斤,打糊酥油半斤 胡椒二两 生
姜汁二合
上件,用盐调和匀,灌肺,煮熟,用汁浇食之。
姜黄腱子
羊腱子一个,熟 羊肋枝二个,截作长块 豆粉一斤 白面一斤
咱夫兰二钱 栀子五钱
上件,用盐、料物调和,搽腱子,下小油炸。
鼓儿签子
羊肉五斤,切细 羊尾子一个,切细 鸡子十五个 生姜二钱 葱
二两,切 陈皮二钱,去白 料物三钱
上件,调和匀,入羊白肠内,煮熟切作鼓样,用豆粉一斤,白面一
斤,咱夫兰一钱,栀子三钱,取汁,同拌鼓儿签子,入小油炸。
带花羊头
羊头三个,熟切 羊腰四个 羊肚肺各一具,煮熟切,攒胭脂染
生姜四两 糟姜二两,各切 鸡子五个,作花样 萝卜三个,作花样
上件,用好肉汤炒,葱、盐、醋调和。
鱼弹儿
大鲤鱼十个,去皮、骨、头、尾 羊尾子二个,同剁为泥生姜一
两,切细 葱二两,切细 陈皮末三钱胡椒末一两哈昔呢二钱
上件,下盐,入鱼肉内拌匀,丸如弹儿,用小油炸。
芙蓉鸡
鸡儿十个,熟攒 羊肚、肺各一具,熟切生姜四两,切胡萝卜十
个,切 鸡子二十个,煎作饼,刻花样 赤根(26)芫荽打糁(27) 胭脂
栀子染 杏泥一斤
上件,用好肉汤炒,葱、醋调和。
肉饼儿
精羊肉十斤,去脂膜筋,捶为泥 哈昔呢三钱 胡椒二两 荜拨一
两芫 荽末一两
上件,用盐调和匀,捻饼,入小油炸。
盐肠
羊苦肠水洗净
上件,用盐拌匀,风干,入小油炸。
脑瓦剌(28)
熟羊胸子二个,切薄片鸡子二十个,熟
上件,用诸般生菜,一同卷饼。
姜黄鱼
鲤鱼十个,去皮鳞 白面二斤 豆粉一斤 芫荽末二两
上件,用盐、料物淹拌过搽鱼,入小油炸熟,用生姜二两,切丝。
芫荽叶,胭脂染,萝卜丝炒,葱调和。
攒雁
雁五个,煮熟,切攒 姜末半斤
上件,用好肉汤炒,葱、盐调和。
猪头姜豉
猪头二个,洗净,切成块 陈皮二钱,去白 良姜二钱 小椒二
钱 官桂二钱 草果五个 小油一斤 蜜半斤
上件,一同熬成,次下芥末炒,葱、醋、盐调和。
蒲黄瓜虀
净羊肉十斤,煮熟,切如瓜虀小椒一两蒲黄半斤
上件,用细料物一两,盐同拌匀。
攒羊头
羊头五个,煮熟攒姜末四两胡椒一两
上件,用好肉汤炒,葱、盐、醋调和。
攒牛蹄(马蹄、熊掌一同)
牛蹄一副,煮熟,攒 姜末二两
上件,用好肉汤同炒,葱、盐调和。
细乞思哥
羊肉一脚子,煮熟,切细 萝卜二个,熟,切细 羊尾子一个,熟
切 哈夫儿(29)二钱
上件,用好肉汤同炒,葱调和。
肝生
羊肝一个,水浸,切细丝 生姜四两,切细丝 萝卜二个,切细
丝 香菜 蓼子各二两,切细丝
上件,用盐、醋、芥末调和。
马肚盘
马肚肠一副,煮熟,切 芥末半斤
上件,将白血灌肠,刻花样,涩脾,和脂剁心子攒成炒,葱、盐、
醋、芥末调和。
炸胙(30)儿(系细项)
胙儿二个,卸成各一节 哈昔呢一钱 葱一两,切细
上件,用盐一同淹拌,少时,入小油炸熟。次用咱夫兰二钱,水浸
汁,下料物、芫荽末,同糁拌。
熬蹄儿
羊蹄五副,退洗净,煮软,切成块 姜末一两 料物五钱
上件,下面丝炒,葱、醋、盐调和。
熬羊胸子
羊胸子二个,退毛洗净,煮软,切作色数块 姜末二两 料物五钱
上件,用好肉汤,下面丝炒,葱、盐、醋调和。
鱼脍
新鲤鱼五个,去皮、骨、头、尾 生姜二两 萝卜二个 葱一两
香菜 蓼子各切如丝 胭脂打糁
上件,下芥末炒,葱、盐、醋调和。
红丝
羊血同白面依法煮熟 生姜四两 萝卜一个 香菜 蓼子各一两,
切细丝
上件,用盐、醋、芥末调和。
烧雁(烧鹚 (31)、烧鸭子等一同)
雁一个,去毛、肠、肚,净羊肚一个,退洗净,包雁 葱二两 芫
荽末一两
上件,用盐同调,入雁腹内烧之。
烧水札(32)
水札十个,挦洗净 芫荽末一两 葱十茎 料物五钱
上件,用盐同拌匀烧,或以肥面包水扎,就笼内蒸熟亦可。或以酥
油水和面包水扎,入炉内炉熟亦可。
柳蒸羊
羊一口,带毛
上件,于地上作炉,三尺深,周回以石,烧令通赤,用铁芭盛羊
上,用柳子盖覆,土封,以熟为度。
仓馒头
羊肉 羊脂 葱 生姜 陈皮各切细
上件,入料物、盐、酱,拌和为馅。
鹿奶肪馒头(或作仓馒头,或做皮薄馒头皆可)
鹿奶肪羊尾子各切如指甲片生姜陈皮各切细
上件,入料物、盐,拌和为馅。
茄子馒头
羊肉 羊脂 羊尾子 葱 陈皮各切细 嫩茄子去
上件,同肉作馅,却入茄子内蒸,下蒜酪、香菜末,食之。
剪花馒头
羊肉 羊脂 羊尾子 葱 陈皮各切细
上件,依法入料物、盐、酱拌馅包馒头,用剪子剪诸般花样,蒸,
用胭脂染花。
水晶角儿
羊肉 羊脂 羊尾子 葱 陈皮 生姜各切细
上件,入细料物、盐、酱拌匀,用豆粉作皮包之。
酥皮奄子
羊肉 羊脂 羊尾子 葱 陈皮 生姜各切细或下瓜哈孙系山丹根
上件,入料物、盐、酱拌匀,用小油、米粉与面,同和作皮。
撇列角儿
羊肉羊脂羊尾子新鮲各切细
上件,入料物、盐、酱拌匀,白面作皮,鏊上炮熟,次用酥油、
蜜,或以葫芦瓠子作馅亦可。
时萝角儿
羊肉 羊脂 羊尾子 葱 陈皮生姜各切细
上件,入料物、盐、酱拌匀,用白面、蜜与小油拌入锅内,滚水搅
熟作皮。
天花(33)包子(或作蟹黄亦可。藤花包子一同)
羊肉 羊脂 羊尾子 葱 陈皮 生姜各切细天花滚水烫熟,洗
净,切细
上件,入料物、盐、酱拌馅,白面作薄皮,蒸。
荷莲兜子
羊肉三脚子,切羊尾子二个,切 鸡头仁八两 松黄八两 八檐仁
(34)四两 蘑菇八两 杏泥一斤胡 桃仁八两 必思荅仁(35)四两 胭脂
一两 栀子四钱 小油二斤 生姜八两 豆粉四斤 山药三斤 鸡子三
十个 羊肚肺各二副 苦肠一副 葱四两 醋半瓶 芫荽叶
上件,用盐、酱、五味调和匀,豆粉作皮,入盏内蒸,用松黄汁浇
食。
黑子儿(36)烧饼
白面五斤 牛奶子二升 酥油一斤 黑子儿一两,微炒上件,用
盐、减少许,同和面作烧饼。
牛奶子烧饼
白面五斤 牛奶子二斤 酥油一斤 茴香一两,微炒
上件,用盐、减(37)少许,同和面作烧饼。
饼(经卷儿一同)
白面十斤 小油一斤 小椒一两,炒去汗茴香一两,炒
上件,隔宿用酵子、盐、减、温水,一同和面。次日入面接肥,再
和成面。每斤作二个,入笼内蒸。
颇儿必汤(即羊辟膝骨)主男女虚劳,寒中,羸瘦,阴气不足。利
血脉,益经气。
颇儿必三四十个,水洗净
上件,用水一铁络,同熬。四分中熬取一分,澄滤净,去油去滓,
再凝定。如欲食,任意多少。
米哈讷关列孙(38) 治五劳七伤,藏气虚冷。常服可补中益气。
羊后脚一个,去筋膜,切碎
上件,用净锅内干爁(39)熟。令盖封闭,不透气,后用净布绞纽取
汁。
【注释】
(1)马思荅吉:西域香料,形类地榭,极香。一云伊兰语mastaki,乳
香。
(2)一脚子:畜兽身体的四分之一部分。
(3)事件:鸟兽类的肢、体、内脏碎成块。宋代孟元老的《东京梦华录
·食店》:
“更有川饭店,则有插肉面……杂煎事件。”
(4)回回豆子:状如榛子肉,味极香美,可磨细和于面。
(5)西天:我国古代对印度的通称。印度古称天竺,因在中国之西,故
称。唐代皇甫曾的《锡杖歌送明楚上人归佛川》:
“上人远自西天至,
头陀行遍南朝寺。”
(6)色数:又称骰子、投子、色子,为赌具。也可于占卜、行酒令或作
游戏。多以兽骨制成小正方块形,相传为三国魏曹植创制。
(7)咱夫兰:状如红花同,云即阿魏根。原书又云“即是回回地面所产
红花”。
(8)哈昔呢:原书云即阿魏。蒙古语gajni阿魏,因产地chazni而得名。
(9)沙乞某儿:一写沙吉木儿,即蔓青根。大头菜。
(10)兀麻食:待考。
(11) :疑为“棋”或“碁”。或作形状云。
(12)松黄:松花粉。
(13)粆(shā,沙):沙糖。
(14)筭(suàn,算):古代计数的筹码。《说文·竹部》:
“筭,长六
寸,计历数者。从竹从弄,言常弄乃不误也。”
(15)乞马:肉细切状。
(16)虀(jī,机):细切后用盐酱等浸渍的蔬果,如腌菜、酱菜、果酱
之类;作调味用的姜、蒜、葱、韭等菜的碎末。
(17)钝:应为“饨”。
(18)瓠(hù,户)子:葫芦科一年生草本植物,果实长圆形,嫩时可以
吃。
(19)挦(xián,闲):拔取;扯、摘取。唐代贾岛的《原居即事言怀赠
孙员外》诗:
“镊挦白发断,兵阻尺书传。”
(20)淹:通腌。
(21)蓼牙:牙通芽。
(22)麹(qǔ,曲):亦作“麯”,简写“曲”。
(23)河西米:指黄河以西,产稷米。
(24)末:通“沫”。
(25)打:据文意应改为“切”。
(26)赤根:即波菜。
(27)糁(sǎn,散):杂,混和。《仪礼·大射》:
“参七十。”汉代
郑玄注:
“参读为糁。糁,杂也。”
(28)脑瓦剌:意不明,恐为译音。
(29)哈夫儿:意不明,恐为译音。
(30)胙(zuò,作):裴骃集解:
“胙,膰肉也。”《史记·周本
纪》:
“致文武胙于秦孝公。”
(31)鹚 :水鸟名。
(32)水札:水鸟名。
(33)天花:天花菜,蕈类植物。
(34)八檐仁:波斯语badam,亦写为巴旦杏仁、巴达杏仁、八檐杏仁。
(35)必思荅仁:波斯语pista。一种叶似山茶,果似银杏的果树的仁。
(36)黑子儿:黑芝麻。
(37)减:通“碱”。
(38)米哈讷关列孙:意不明,恐为译音。
(39)爁(lǎn,缆):烤炙之意。宋代林洪的《山家清供》:
“适有人
携双鸳至,得之燖以油爁,下酒酱香料燠熟。”
卷 二
诸般汤煎
桂浆 生津止渴,益气和中,祛湿逐饮。
生姜三斤,取汁 熟水二斗赤茯苓三两,去皮,为末桂三两,去
皮,为末 曲末半斤 杏仁一百个,汤洗,去皮、尖,生研为泥 大麦
糵半两,为末 白沙蜜三斤,炼净
上件,用药前,蜜水拌和匀,入磁罐内,油纸封口数重,泥固济,
冰窖内放三日方熟。绵滤冰浸,暑月饮之。
桂沉浆 祛湿逐饮,生津止渴,顺气。
紫苏叶一两,锉 沉香三钱,锉 乌梅一两,取肉 沙糖六两
上件四味,用水五六碗,熬至三碗,滤去滓,入桂浆一升,合和作
浆,饮之。
荔枝膏 生津,止渴,去烦。
乌梅半斤,取肉 桂十两,去皮,锉 沙糖二十六两麝香半钱,
研 生姜汁五两 熟蜜十四两
上件,用水一斗五升,熬至一半,滤去滓,下沙糖、生姜汁,再熬
去籸,澄定少时,入麝香搅匀,澄清如常,任意服。
梅子丸 生津止渴,解化酒毒,祛湿。
乌梅一两半,取肉 白梅一两半,取肉 干木瓜一两半紫苏叶一两
半 甘草一两,炙 檀香二钱 麝香一钱,研
上件为末,入麝香和匀,沙糖为丸如弹大。每服一丸,噙化。
五味子汤 代葡萄酒饮。生津止渴,暖精益气。
北五味一斤,净肉 紫苏叶六两人参四两,去芦,剉沙糖二斤
上件,用水二斗,熬至一斗,滤去滓,澄清,任意服之。
人参汤 代酒饮。顺气,开胸膈,止渴生津。
新罗参四两,去芦,剉 橘皮一两,去白 紫苏叶二两沙糖一斤
上件,用水二斗,熬至一斗,去滓,澄清,任意饮之。
仙术汤 去一切不正之气。温脾胃,进饮食,辟瘟疫,除寒湿。
苍术一斤,米泔浸三日,竹刀子切片,焙干,为末 茴香二两,
炒,为末 甘草二两,炒,为末 白面一斤,炒 干枣二升,焙干,为
末 盐四两,炒
上件,一同和匀。每日空心白汤点服。
杏霜汤 调顺肺气,利胸膈,治咳嗽。
粟米五升,炒,为面 杏仁二升,去皮、尖,麸炒,研 盐三两,
炒
上件拌匀,每日空心白汤调一钱。入酥少许尤佳。
山药汤 补虚益气,温中润肺。
山药一斤,煮熟 粟米半升,炒,为面 杏仁二斤,炒令过熟,去
皮、尖,切如米
上件,每日空心白汤调二钱,入酥油少许,山药任意。
四和汤 治腹内冷痛,脾胃不和。
白面一斤,炒芝麻一斤,炒茴香二两,炒盐一两,炒
上件,并为末。每日空心白汤点服。
枣姜汤 和脾胃,进饮食。
生姜一斤,切作片 枣三升,去核,炒 甘草二两,炒 盐二两,
炒
上件为末,一处拌匀。每日空心白汤点服。
茴香汤 治元藏虚弱,脐腹冷痛。
茴香一斤,炒 川楝子半斤陈皮半斤,去白甘草四两,炒 盐半
斤,炒
上件为细末,相和匀。每日空心白汤点服。
破气汤 治元藏虚弱,腹痛,胸膈闭闷。
杏仁一斤,去皮、尖,麸炒,别研 茴香四两,炒 良姜一两 荜
澄茄二两 陈皮二两,去白 桂花半斤 姜黄一两 木香一两 丁香一
两 甘草半斤 盐半斤
上件为细末。空心白汤点服。
白梅汤 治中热,五心烦躁,霍乱呕吐,干渴,津液不通。
白梅肉一斤 白檀四两 甘草四两 盐半斤
上件为细末。每服一钱,入生姜汁少许,白汤调下。
木瓜汤 治脚气,不仁,膝劳,冷痹,疼痛。
木瓜四个,蒸熟,去皮,研烂如泥 白沙蜜二斤,炼净
上件二味,调和匀,入净磁器内盛之。空心白汤点服。
橘皮醒酲汤 治酒醉不解,呕噫吞酸。
香橙皮一斤,去白 陈橘皮一斤,去白 檀香四两葛花半斤 绿豆
花半斤 人参二两,去芦 白豆蔻仁二两 盐六两,炒
上件为细末。每日空心白汤点服。
渴忒(1)饼儿 生津止渴,治嗽。
渴忒一两二钱 新罗参一两,去芦菖蒲一钱,各为细末白纳八三
两,研,系沙糖
上件,将渴忒用葡萄酒化成膏,和上项药末,令匀为剂,印作饼。
每用一饼,徐徐噙化。
官桂渴忒饼儿 生津,止寒嗽。
官桂二钱,为末渴忒一两二钱新罗参一两二钱,去芦,为末白纳八
三两,研
上件,将渴忒用玫瑰水化成膏,和药末为剂,用诃子油印作饼子。
每用一饼,徐徐噙化。
荅必纳(2)饼儿 清头目,利咽膈,生津止渴,治嗽。
荅必纳二钱为末,即草龙胆 新罗参一两二钱,去芦,为末白纳八
五两,研
上件,用赤赤哈纳(即北地酸角儿)熬成膏,和药末为剂,印作饼
儿,每用一饼,徐徐噙化。
橙香饼儿 宽中顺气,清利头目。
新橙皮一两,焙,去白沉香五钱 白檀五钱 缩砂五钱 白豆蔻仁
五钱 荜澄茄三钱 南硼砂三钱,别研龙脑二钱,别研麝香二钱,别研
上件为细末,甘草膏和剂印饼。每用一饼,徐徐噙化。
牛髓膏子 补精髓,壮筋骨,和血气,延年益寿。
黄精膏五两 地黄膏三两 天门冬膏一两 牛骨头内取油二两
上件,将黄精膏、地黄膏、天门冬膏与牛骨油一同不停用银匙搅,
令冷定和匀成膏。每日空心温酒调一匙头。
木瓜煎
木瓜十个,去皮,取汁,熬水尽白沙糖十斤,炼净
上件,一同再熬成煎。
香圆(3)煎
香圆二十个,去皮取肉 白沙糖十斤,炼净
上件,一同再熬成煎。
株子(4)煎
株子一百个,取净肉 白沙糖五斤,炼净
上件,同熬成煎。
紫苏煎
紫苏叶五斤 干木瓜五斤 白沙糖十斤,炼净
上件,一同熬成煎。
金橘煎
金橘五十个,去子,取皮 白沙糖三斤
上件,一同熬成煎。
樱桃煎
樱桃五十斤,取汁 白沙糖二十五斤
上件,同熬成煎。
桃煎
大桃一百个,去皮,切片取汁 白沙蜜二十斤,炼净上件,一同熬
成煎。
石榴浆
石榴子十斤,取汁 白沙糖十斤,炼净
上件,一同熬成煎。
小石榴煎
小石榴二斗,蒸熟去子,研为泥 白沙蜜十斤,炼净
上件,一同熬成煎。
五味子舍儿别(5)
新北五味十斤,去子,水浸取汁 白沙糖八斤,炼净
上件,一同熬成煎。
赤赤哈纳(即酸刺)
赤赤哈纳不以多少,水浸取汁
上件,用银石器内熬成膏。
松子油
松子不以多少,去皮,捣研为泥
上件,水绞取汁熬成,取浮清油,绵滤净,再熬澄清。
杏子油
杏子不以多少,连皮捣碎
上件,水煮熬,取浮油,绵滤净,再熬成油。
酥油
牛乳中取浮凝,熬而为酥。
醍醐油
取上等酥油,约重千斤之上者,煎熬过滤净,用大磁瓮贮之,冬月
取瓮中心不冻者,谓之醍醐。
马思哥油
取净牛奶子不停用阿赤(即打油木器也)打取浮凝者为马思哥油。
今亦云白酥油。
枸杞茶
枸杞五斗,水淘洗净,去浮麦,焙干,用白布筒净,去蒂萼、黑
色,选拣红熟者,先用雀舌茶展溲碾子,茶芽不用,次碾枸杞为细末。
每日空心用,匙头,入酥油搅匀,温酒调下,白汤亦可。忌与酪同食。
玉磨茶
上等紫笋五十斤,筛筒净,苏门炒米五十斤,筛筒净,一同拌和
匀,入玉磨内,磨之成茶。
金字茶
系江南湖州造,进末茶。
范殿帅茶
系江浙庆元路造,进茶芽,味色绝胜诸茶。
紫笋雀舌茶
选新嫩芽蒸过为紫笋。有先春、次春、探春,味皆不及紫笋雀舌。
女须儿
出直北地面。味温甘。
西番茶
出本土。味苦涩。煎用酥油。
川茶、藤茶、夸茶
皆出四川。
燕尾茶
出江浙、江西。
孩儿茶
出广南。
温桑茶
出黑峪。凡诸茶,味甘苦,微寒,无毒。去痰热,止渴,利小便,
消食下气,清神少睡。
清茶
先用水滚过滤净,下茶芽,少时煎成。
炒茶
用铁锅烧赤,以马思哥油、牛奶子、茶芽同炒成。
兰膏
玉磨末茶三匙头,面、酥油同搅成膏,沸汤点之。
酥签
金字末茶两匙头,入酥油同搅,沸汤点服。
建汤
玉磨末茶一匙,入碗内研匀,百沸汤点之。
香茶
白茶一袋 龙脑成片者三钱 百药煎半钱 麝香二钱上件同研细,
用香粳米熬成粥,和成剂,印作饼。
【注释】
(1)渴忒:意不明,恐为译音。
(2)荅必纳:译音,即草龙胆,颖为龙胆草。
(3)香圆:即香橼,芸香科柑橘类植物,佛手一类,有疏肝理气、宽中
化痰的作用。
(4)株子:果品,状似金橘。
(5)舍儿别:意不明,恐为饮料译音。
诸 水
泉水
甘平,无毒。治消渴,反胃,热痢。今西山有玉泉水,甘美味胜诸
泉。
井华水
甘平,无毒。主人九窍大惊出血,以水豿面即住。及洗人目翳。投
酒醋中,令不损败,平旦汲者是也。
邹店水
今内府御用之水,常于邹店取之。缘自至大初武宗皇帝幸柳林飞
放,请皇太后同往观焉。由是道经邹店,因渴思茶,遂命普兰奚国公金
界奴朵儿只煎造。公亲诣诸井选水,唯一井水,味颇清甘。汲取煎茶以
进,上称其茶味特异。内府常进之茶,味色两绝。乃命国公于井所建观
音堂,盖亭井上,以栏翼之,刻石纪其事。自后御用之水,日必取焉。
所造汤茶,比诸水殊胜,邻左有井,皆不及也。此水煎熬过,澄莹如
一。常较其分两与别水增重。
神仙服饵
铁瓮先生琼玉膏 此膏填精补髓,肠化为筋,万神具足,五脏盈
溢,髓血满,发白变黑,返老还童,行如奔马。日进数服,终日不食亦
不饥,开通强志,日诵万言,神识高迈,夜无梦想。人年二十七岁以
前,服此一料,可寿三百六十岁。四十五岁以前服者,可寿二百四十
岁。六十三岁以前服者,可寿一百二十岁。六十四岁以上服者,可寿百
岁。服之十剂,绝其欲,修阴功,成地仙矣。一料分五处,可救五人痈
疾,分十处,可救十人劳疾。修合之时,沐浴至心,勿轻示人。
新罗参二十四两,去芦 生地黄十六斤,汁 白茯苓四十九两,去
黑皮 白沙蜜十斤,炼净
上件,人参、茯苓为细末,蜜用生绢滤过,地黄取自然汁,捣时不
用铜铁器,取汁尽,去滓,用药一处拌和匀,入银石器或好磁器内,封
用净纸二三十重封闭。入汤内,以桑柴火煮三昼夜。取出,用蜡纸数重
包瓶口,入井口去火毒一伏时。取出再入旧汤内煮一日,出水气,取出
开封,取三匙作三盏,祭天地百神,焚香设拜,至诚端心。每日空心,
酒调一匙头。
地仙煎 治腰膝疼痛,一切腹内冷病。令人颜色悦泽,骨髓坚固,
行及奔马。
山药一斤 杏仁一升,汤泡,去皮、尖 生牛奶子二升
上件,将杏仁研细,入牛奶子、山药,拌绞取汁,用新磁瓶密封,
汤煮一日。每日空心,酒调一匙头。
金髓煎 延年益寿,填精补髓。久服发白变黑,返老还童。
枸杞不以多少,采红熟者
上用无灰酒浸之,冬六日,夏三日,于沙盆内研令烂细,然后以布
袋绞取汁,与前浸酒一同慢火熬成膏,于净磁器内封贮。重汤煮之,每
服一匙头,入酥油少许,温酒调下。
天门冬膏 去积聚,风痰,癫疾,三虫,伏尸,除瘟疫。轻身,益
气,令人不饥,延年不老。
天门冬不以多少,去皮,去根、须,洗净
上件捣碎,布绞取汁,澄清滤过,用磁器、沙锅或银器,慢火熬成
膏。每服一匙头,空心温酒调下。
服天门冬
道书《八帝经》:欲不畏寒,取天门冬、茯苓为末服之。每日顿
服,大寒时汗出,单衣。
《抱朴子》云:杜紫微服天门冬,御八十妾,有子一百四十人,日
行三百里。
《列仙子》云:赤松子食天门冬,齿落更生,细发复出。
《神仙传》:甘始者,太原人。服天门冬,在人间三百年。
《修真秘旨》:神仙服天门冬,一百日后怡泰和颜,羸劣者强。三
百日,身轻;三年,身走如飞。
服地黄
《抱朴子》云:楚文子服地黄八年,夜视有光,手上车弩。
服苍术
《抱朴子》云:南阳文氏,值乱逃于壶山,饥困,有人教之食术,
遂不饥。数年乃还乡里,颜色更少,气力转胜。
《药经》云:必欲长生,当服山精。是苍术也。
服茯苓
《抱朴子》云:任季子服茯苓一十八年,玉女从之,能隐彰,不食
谷,面生光。
孙真人《枕中记》:茯苓久服,百日百病除。二百日,夜昼二服
后,役使鬼神。四年后,玉女来侍。
服远志
《抱朴子》云:陵阳仲子服远志二十年,有子三十人,开书所见,
便记不忘。
服五加皮酒
东华真人《煮石经》:舜常登苍梧山,曰:厥金玉香草,即五加
也,服之延年。故云:宁得一把五加,不用金玉满车;宁得一斤地榆,
安用明月宝珠。昔鲁定公母,单服五加皮酒,以致长生。如张子声、杨
始建、王叔才、于世彦等,皆古人服五加皮酒而房室不绝,皆寿三百
岁,有子二三十人。世世有服五加皮酒而获年寿者甚众。
服桂
《抱朴子》云:赵他子服桂二十年,足下毛生,日行五百里,力举
千斤。
服松子
《列仙传》:偓佺食松子,能飞,行健,走如奔马。
《神仙传》:松子不以多少,研为膏,空心温酒调下一匙头,日三
服则不饥渴。久服日行五百里,身轻体健。
服松节酒
《神仙传》:治百节疼痛,久风虚,脚痹痛。松节酿酒,服之神
验。
服槐实
《神仙传》:槐实于牛胆中渍浸百日,阴干。每日吞一枚,十日身
轻,二十日白发再黑,百日通神。
服枸杞
《食疗》云:枸杞叶能令人筋骨壮,除风补益,去虚劳,益阳事。
春夏秋采叶,冬采子,可久食之。
服莲花
太清诸本草:七月七日采莲花七分,八月八日采莲根八分,九月九
日采莲子九分,阴干食之,令人不老。
服栗子
《食疗》云:如肾气虚弱,取生栗子不以多少,令风干之。每日空
心细嚼之三五个,徐徐咽之。
服黄精
神仙服黄精成地仙:昔临川有士人虐其婢,婢乃逃入山中。久之,
见野草枝叶可爱,即拔取食之,甚美。自是常食之,久而不饥,遂轻
健。夜息大木下,闻草动以为虎,惧而上木避之,及晓下平地,其身豁
然,凌空而去,或自一峰之顶,若飞鸟焉,数岁。其家采薪见之,告其
主,使捕之,不得。一日,遇绝壁下,以网三面围之,俄而腾上山顶。
其主异之,或曰:此婢安有仙风道骨?不过灵药服食。遂以酒馔五味香
美,置往来之路,观其食否,果来食之,遂不能远去,擒之。问以述其
故,所指食之草,即黄精也。谨按:黄精宽中益气,补五藏,调良肌
肉,充实骨髓,坚强筋骨,延年不老,颜色鲜明,发白再黑,齿落更
生。
神枕法
汉武帝东巡泰山下,见老翁锄于道,背上有白光高数尺。帝怪而问
之,有道术否?老翁对曰:臣昔年八十五时,衰老垂死,头白齿落,有
道士者,教臣服枣、饮水、绝谷,并作神枕法,中有三十二物。内二十
四物善,以当二十四气;其八物毒,以应八风。臣行转少,黑发更生,
堕齿复出,日行三百里。臣今年一百八十矣,不能弃世入山,顾恋子
孙,复还食谷,又已二十余年,犹得神枕之力,往不复老。武帝视老
翁,颜壮当如五十许人,验问其邻人,皆云信然。帝乃从授其方作枕,
而不能随其绝谷、饮水也。
神枕方:用五月五日,七月七日,取出林柏以为枕。长一尺二寸,
高四寸,空中容一斗二升,以柏心赤者为盖,厚二分,盖致之令密,又
使可开闭也。又钻盖上为三行,每行四十九孔,凡一百四十七孔,令容
粟大。用下项药:
芎穷 当归 白芷 辛荑 杜衡 白术 藁本 木兰 蜀椒 桂
干姜 防风 人参 桔梗 白薇 荆实 肉苁蓉 飞廉 柏实 薏苡
仁 款冬花 白衡 秦椒 麋芜
凡二十四物,以应二十四气。
乌头 附子 藜芦 皂角 菵草 矾石 半夏 细辛
八物毒者,以应八风。
上三十二物各一两,皆 咀。以毒药上安之,满枕中,用囊以衣
枕。百日而有光泽,一年体中诸疾一一皆愈而身尽香。四年白发变黑,
齿落重生,耳目聪明。神方验秘,不传非人也。武帝以问东方朔,答
云:昔女廉以此传玉青,玉青以传广成子,广成子以传黄帝。近者谷城
道士淳于公枕此药,枕百余岁而头发不白。夫病之来皆从阳脉起,今枕
药枕,风邪不得侵入矣。又虽以布囊衣枕,犹当复以帏囊重包之,须欲
卧时乃脱去之耳。诏赐老翁疋帛,老翁不受,曰:臣之于君,犹子之于
父也,子知道以上之于父,义不受赏。又臣非卖道者,以陛下好善,故
进此耳。帝止而更赐诸药。
服菖蒲
《神仙服食》:菖蒲寻九节者,窨(1)干百日,为末,日三服。久服
聪明耳目,延年益寿。
《抱朴子》云:韩聚服菖蒲十三年,身上生毛,日诵万言,冬袒不
寒。须得石上生者,一寸九节,紫花尤善。
服胡麻
《神仙服食》:胡麻,食之能除一切痼疾,久服长生。肥健人,延
年不老。
服五味
《抱朴子》:服五味十六年,面色如玉,入火不灼,入水不濡。
服藕实
《食医心镜》:藕实,味甘平,无毒。补中养气,清神,除百病。
久服令人止渴悦泽。
服莲子、莲蕊
《日华子》云:莲子并石莲去心,久食令人心喜,益气、止渴。治
腰痛,泄精,泻痢。
《日华子》云:莲花蕊,久服镇心益色,驻颜轻身。
服何首乌
《日华子》云:何首乌,味甘,无毒,久服壮筋骨,益精髓,黑髭
鬓,令人有子。
【注释】
(1)窨(yìn,印):窖藏;深藏。宋代张邦基的《墨庄漫录》卷
二:
“令众香蒸过,入磁器,有油者,地窖窨一月。”
四时所宜
春三月,此谓发陈,天地俱生,万物以荣,夜卧早起,广步于庭,
被发缓行,以使志生,生而勿杀,予而勿夺,赏而勿罚,此春气之应,
养生之道也。逆之则伤肝,夏为寒变,奉长者少。
春气温,宜食麦,以凉之,不可一于温也。禁温饮食及热衣服。
夏三月,此谓蕃秀,天地气交,万物华实,夜卧早起,无厌于日,
使志无怒,使华英成秀,使气得泄,若所爱在外,此夏气之应,养长之
道也。逆之则伤心,秋为痎疟,奉收者少,冬至重病。
夏气热,宜食菽,以寒之,不可一于热也。禁温饮食、饱食、湿
地、濡衣服。
秋三月,此谓容平,天气以急,地气以明,早卧早起,与鸡俱兴,
使志安宁,以缓秋形,收敛神气,使秋气平,无外其志,使肺气清,此
秋气之应,养收之道也。逆之则伤肺,冬为飧泄,奉藏者少。
秋气燥,宜食麻,以润其燥。禁寒饮食、寒衣服。
冬三月,此谓闭藏,水冰地坼,无扰乎阳,早卧晚起,必待日光,
使志若伏若匿,若有私意,若己有得,去寒就温,无泄皮肤,使气亟
夺,此冬气之应,养藏之道也。逆之则伤肾,春为痿厥,奉生者少。
冬气寒,宜食黍,以热性治其寒。禁热饮食,温炙衣服。
五味偏走
酸涩以收,多食则膀胱不利,为癃闭。苦燥以坚,多食则三焦闭
塞,为呕吐。辛味薰蒸,多食则上走于肺,荣卫不时而心洞,咸味涌
泄,多食则外注于脉,胃竭,咽燥而病渴。甘味弱劣,多食则胃柔缓而
虫过,故中满而心闷。
辛走气,气病勿多食辛。咸走血,血病勿多食咸。苦走骨,骨病勿
多食苦。甘走肉,肉病勿多食甘。酸走筋,筋病勿多食酸。
肝病禁食辛,宜食粳米、牛肉、葵枣之类。心病禁食咸,宜食小
豆、犬肉、李、韭之类。脾病禁食酸,宜食大豆、豕肉、栗、藿之类。
肺病禁食苦,宜食小麦、羊肉、杏、薤之类。肾病禁食甘,宜食黄黍、
鸡肉、桃、葱之类。
多食酸,肝气以津,脾气乃绝,则肉胝皱而唇揭。多食咸,骨气劳
短,肥气折,则脉凝泣而变色。多食甘,心气喘满,色黑,肾气不平,
则骨痛而发落。多食苦,则脾气不濡,胃气乃浓,则皮槁而毛拔。多食
辛,筋脉沮弛,精神乃央,则筋急而爪枯。
五谷为食,五果为助,五肉为益,五菜为充,气味合和而食之则补
精益气。虽然五味调和,食饮口嗜,皆不可多也。多者生疾,少者为
益。百味珍馔,日有慎节,是为上矣。
食疗诸病
生地黄鸡 治腰背疼痛,骨髓虚损,不能久立,身重气乏,盗汗,
少食,时复吐利。
生地黄半斤 饴糖五两 乌鸡一枚
上件三味,先将鸡去毛、肠,肚净,细切,地黄与糖相和匀,内鸡
腹中,以铜器中放之,复置甑中蒸,炊饭熟成,取食之。不用盐、醋,
唯食肉,尽却饮汁。
羊蜜膏 治虚劳,腰痛,咳嗽,肺痿,骨蒸。
熟羊脂五两 熟羊髓五两 白沙蜜五两,炼净 生姜汁一合 生地
黄汁五合
上件五味,先以羊脂煎令沸,次下羊髓又令沸,次下蜜、地黄、生
姜汁,不住手搅,微火熬数沸成膏。每日空心温酒调一匙头。或作羹
汤,或作粥食之亦可。
羊藏羹 治肾虚劳损,骨髓伤败。
羊肝、肚、肾、心、肺各一具,汤洗净 牛酥一两 胡椒一两 荜
拨一两 豉一合 陈皮二钱,去白 良姜二钱 草果两个 葱五茎
上件,先将羊肝等,慢火煮令熟,将汁滤净。和羊肝等并药一同入
羊肚内,缝合口,令绢袋盛之,再煮熟,入五味,旋旋(1)任意食之。
羊骨粥 治虚劳,腰膝无力。
羊骨一副,全者,捶碎 陈皮二钱,去白 良姜二钱 草果二个
生姜一两 盐少许
加水三斗,慢火熬成汁,滤出澄清,如常作粥,或作羹汤亦可。
羊脊骨粥 治下元久虚,腰肾伤败。
羊脊骨一具,全者,捶碎 肉苁蓉一两,洗,切作片 草果三个
荜拨二钱
上件,水熬成汁,滤去滓,入葱白、五味,作面羹食之。
白羊肾羹 治虚劳,阳道衰败,腰膝无力。
白羊肾二具,切作片 肉苁蓉一两,酒浸,切羊脂四两,切作片
胡椒二钱 陈皮一钱,去白荜拨二钱 草果二钱
上件相和,入葱白、盐、酱,煮作汤,入面食其子,如常作羹食
之。
猪肾粥 治肾虚劳损,腰膝无力,疼痛。
猪肾一对,去脂膜,切粳米三合 草果二钱 陈皮一钱,去白 缩
砂二钱
上件,先将猪肾、陈皮等煮成汁,滤去滓,入酒少许,次下米成
粥,空心食之。
枸杞羊肾粥 治阳气衰败,腰脚疼痛,五劳七伤。
枸杞叶一斤 羊肾一对,细切 葱白一茎 羊肉半斤,炒
右四味拌匀,入五味,煮成汁,下米熬成粥,空腹食之。
鹿肾羹 治肾虚耳聋。
鹿肾一对,去脂膜,切
上件于豆豉中,入粳米三合,煮粥或作羹,入五味,空心食之。
羊肉羹 治肾虚衰弱,腰脚无力。
羊肉半斤,细切萝卜一个,切作片 草果一钱 陈皮一钱,去白
良姜一钱 荜拨一钱 胡椒一钱 葱白三茎
上件,水熬成汁,入盐、酱熬汤,下面食其子,作羹食之。将汤澄
清,作粥食之亦可。
鹿蹄汤 治诸风、虚,腰脚疼痛,不能践地。鹿蹄四只 陈皮二
钱 草果二钱
上件,煮令烂熟,取肉,入五味,空腹食之。
鹿角酒 治卒患腰痛,暂转不得。
鹿角新者,长二三寸,烧令赤
上件,内酒中浸二宿,空心饮之立效。
黑牛髓煎 治肾虚弱,骨伤败,瘦弱无力。
黑牛髓半斤 生地黄汁半斤 白沙蜜半斤,炼去蜡
上件三味和匀,煎成膏,空心酒调服之。
狐肉汤 治虚弱,五脏邪气。
狐肉五斤,汤洗净 草果五个 缩砂二钱 葱一握 陈皮一钱,去
白 良姜二钱 哈昔呢一钱,即阿魏上件,水一斗,煮熟,去草果等,
次下胡椒二钱,姜黄一钱,醋、五味,调和匀,空心食之。
乌鸡汤 治虚弱,劳伤,心腹邪气。
乌雄鸡一只,洗净,切作块子陈皮一钱,去白 良姜一钱 胡椒二
钱 草果二个
上件,以葱、醋、酱相和,入瓶内,封口,令煮熟,空腹食。
醍醐酒 治虚弱,祛风湿。
醍醐一盏
上件,以酒一杯和匀,温饮之,效验。
山药饦 治诸虚,五劳七伤,心腹冷痛,骨髓伤败。
羊骨五七块,带肉 萝卜一枚,切作大片 葱白一茎 草果五个
陈皮一钱,去白 良姜一钱 胡椒二钱 缩砂二钱 山药二斤
上件同煮,取汁澄清,滤去籸,面二斤,山药二斤,煮熟,研泥,
搜(2)面作饦(3),入五味,空腹食之。
山药粥 治虚劳,骨蒸,久冷。
羊肉一斤,去脂膜,烂煮熟,研泥山药一斤,煮熟,研泥
上件,肉汤内下米三合,煮粥,空腹食之。
酸枣粥 治虚劳,心烦,不得睡卧。
酸枣仁一碗
上件用水,绞取汁,下米三合煮粥,空腹食之。
生地黄粥 治虚弱骨蒸,四肢无力,渐渐羸瘦,心烦不得睡卧。
生地黄汁一合酸枣仁水绞,取汁二盏
上件,水煮同熬数沸,次下米三合煮粥,空腹食之。
椒面羹 治脾胃虚弱,久患冷气,心腹结痛,呕吐不能下食。
川椒三钱,炒,为末白面四两
上件同和匀,入盐少许,于豆豉作面条,煮羹食之。
荜拨粥 治脾胃虚弱,心腹冷气痛,烦闷不能食。
荜拨一两胡椒一两桂五钱
上件三味为末。每用三钱,水三大碗,入豉半合,同煮令熟,去
滓,下米三合作粥,空腹食之。
良姜粥 治心腹冷痛,积聚,停饮。
高良姜半两,为末 粳米三合
上件,水三大碗,煎高良姜至二碗,去滓,下米煮粥,食之效验。
吴茱萸粥 治心腹冷气,冲胁肋痛。
吴茱萸半两,水洗,去涎,焙干,炒,为末
上件,以米三合,一同作粥,空腹食之。
牛肉脯 治脾胃久冷,不思饮食。
牛肉五斤,去脂膜,切作大片 胡椒五钱 荜拨五钱 陈皮二钱,
去白 草果二钱 缩砂二钱 良姜二钱
上件为细末,生姜汁五合,葱汁一合,盐四两,同肉拌匀,淹二
日,取出焙干,作脯,任意食之。
莲子粥 治心志不宁,补中强志,聪明耳目。
莲子一升,去心
上件煮熟,研如泥,与粳米三合,作粥,空腹食之。
鸡头粥 治精气不足,强志,明耳目。
鸡头实三合
上件煮熟,研如泥,与粳米一合,煮粥食之。
鸡头粉羹 治湿痹,腰膝痛。除暴疾,益精气,强心志,耳目聪
明。
鸡头磨成粉 羊脊骨一副,带肉,熬取汁
上件,用生姜汁一合,入五味调和,空心食之。
桃仁粥 治心腹痛,上气咳嗽,胸膈烦满,喘急。
桃仁三两,汤煮熟,去尖、皮,研
上件取汁,和粳米同煮粥,空腹食之。
生地黄粥 治虚劳,瘦弱,骨蒸,寒热往来,咳嗽,唾血。
生地黄汁二合
上件,煮白粥,临熟时入地黄汁,搅匀,空腹食之。
鲫鱼羹 治脾胃虚弱,泄痢,久不瘥者,食之立效。
大鲫鱼二斤 大蒜两块 胡椒二钱 小椒二钱 陈皮二钱 缩砂二
钱 荜拨二钱
上件,葱、酱、盐、料物、蒜,入鱼肚内,煎熟作羹,五味调和令
匀,空心食之。
炒黄面 治泄痢,肠胃不固。
白面一斤,炒令焦黄
上件,每日空心温水调一匙头。
乳饼面 治脾胃虚弱,赤白泄痢。
乳饼一个,切作豆子样
上件,用面拌煮熟,空腹食之。
炙黄鸡 治脾胃虚弱,下痢。
黄雌鸡一(4)只,挦净
上件以盐、酱、醋、茴香、小椒末同拌匀,刷鸡上,令炭火炙干
焦,空腹食之。
牛奶子煎荜拨法
贞观中,太宗苦于痢疾,众医不效,问左右能治愈者,当重赏。时
有术士进此方:用牛奶子煎荜拨,服之立瘥。
猯(5)肉羹 治水肿,浮气,腹胀,小便涩少。
猯肉一斤,细切 葱一握 草果三个
上件,用小椒、豆豉,同煮烂熟,入粳米一合作羹,五味调匀,空
腹食之。
黄雌鸡 治腹中水癖,水肿。
黄雌鸡一只,挦净 草果二钱 赤小豆一升
上件,同煮熟,空心食之。
青鸭羹 治十肿(6)水病不瘥。
青头鸭一只,退净 草果五个
上件,用赤小豆半升,入鸭腹内煮熟,五味调,空心食。
萝卜粥 治消渴,舌焦,口干,小便数。
大萝卜五个,煮熟,绞取汁
上件,用粳米三合,同水并汁,煮粥食之。
野鸡羹 治消渴,口干,小便频数。
野鸡一只,挦净
上件入五味,如常法作羹臛(7)食之。
鹁鸽羹 治消渴,饮水无度。
白鹁鸽一只,切作大片
上件,用土苏(8)一同煮熟,空腹食之。
鸡子黄 治小便不通。
鸡子黄一枚,生用
上件,服之不过三服,熟亦可食。
葵菜羹 治小便癃闭不通。
葵菜叶不以多少,洗择净
上件煮作羹,入五味,空腹食之。
鲤鱼汤 治消渴,水肿,黄胆,脚气。
大鲤鱼一头赤小豆一合陈皮二钱,去白小椒二钱草果二钱
上件,入五味,调和匀,煮熟,空腹食之。
马齿菜粥 治脚气,头面水肿,心腹胀满,小便淋涩。
马齿菜洗净,取汁
上件,和粳米同煮粥,空腹食之。
小麦粥 治消渴,口干。
小麦淘净,不以多少
上件以煮粥,或炊作饭,空腹食之。
驴头羹 治中风头眩,手足无力,筋骨烦痛,言语蹇涩。
乌驴头一枚,挦洗净 胡椒二钱 草果二钱
上件,煮令烂熟,入豆豉汁中,五味调和,空腹食之。
驴肉汤 治风狂,忧愁不乐。安心气。
乌驴肉不以多少,切
上件,于豆豉中,烂煮熟,入五味,空心食之。
狐肉羹 治惊风,癫痫,神情恍惚,言语错谬,歌笑无度。
狐肉不以多少及五脏
上件,如常法入五味,煮令烂熟,空心食之。
熊肉羹 治诸风,脚气,痹痛不仁,五缓筋急。
熊肉一斤
上件,于豆豉中,入五味、葱、酱,煮熟,空腹食之。
乌鸡酒 治中风,背强,舌直不得语,目睛不转,烦热。
乌雌鸡一只,挦洗净,去肠肚
上件,以酒五升,煮取酒二升,去滓。分作三服,相继服之。汁
尽,无时熬葱白、生姜粥投之,盖覆取汁。
羊肚羹 治诸中风。
羊肚一枚,洗净粳米二合葱白数茎豉半合 蜀椒去目,闭口者,炒
出汗,三十粒 生姜二钱半,细切
上件六味拌匀,入羊肚内烂煮熟,五味调和,空心食之。
葛粉羹 治中风,心脾风热,言语蹇涩,精神昏愤(9),挦手足不
遂。
葛根半斤,捣,取粉四两 荆芥穗一两 豉三合
上件三味,先以水煮荆芥、豉,六七沸,去滓,取汁,次将葛粉作
索面,于汁中煮熟,空腹食之。
荆芥粥 治中风,言语蹇涩,精神昏愦,口面 斜。
荆芥穗一两 薄荷叶一两 豉三合 白粟米三合
上件,以水四升,煮取三升,去滓,下米煮粥,空腹食之。
麻子粥 治中风,五脏风热,语言蹇涩,手足不遂,大肠滞涩。
冬麻子二两,炒,去皮,研 白粟米三合 薄荷叶一两 荆芥穗一
两
上件,水三升,煮薄荷、荆芥,去滓,取汁,入麻子仁同煮粥,空
腹食之。
恶实菜 即牛蒡子,又名鼠粘子。治中风,燥热,口干,手足不遂
及皮肤热疮。
恶实菜叶肥嫩者 酥油
上件,以汤煮恶实叶三五升,取出,以新水淘过,布绞取汁,入五
味,酥点食之。
乌驴皮汤 治中风,手足不遂,骨节烦疼,心燥,口眼面目 斜。
乌驴皮一张,挦洗净
上件,蒸熟,细切如条,于豉汁中,入五味,调和匀,煮过,空心
食之。
羊头脍 治中风,头眩,羸瘦,手足无力。
白羊头一枚,挦洗净
上件,蒸令烂熟,细切,以五味汁调和脍,空腹食之。
野猪臛 治久痔、野鸡病,下血不止,肛门肿满。
野猪肉二斤,细切
上件,煮令烂熟,入五味,空心食之。
獭肝羹 治久痔下血不止。
獭肝一副
上件,煮熟,入五味,空腹食之。
鲫鱼羹 治久痔,肠风,大便常有血。
大鲫鱼一头,新鲜者,洗净,切作片 小椒二钱,为末 草果一
钱,为末
上件,用葱三茎,煮熟,入五味,空腹食之。
【注释】
(1)旋旋:频频。唐代顾况的《焙茶坞》诗:
“旋旋续新烟,呼儿劈寒
木。”
(2)搜(shǎo,少)面:搜,即搅和,拌和。唐代费冠卿的《答萧建》
诗:
“搜泥如和面,拾橡半添穜。”搜面,和面意。
(3)饦,(tuō,托):饼。《方言》第十三:
“饼谓之。”
(4)一:原缺,据《丛书》本补上。
(5)猯(tuān,湍):兽名。猪獾。唐代李白的《大猎赋》:
“拳封
猯,肘巨狿。”
(6)肿:应为“种”。
(7)臛(huò,霍):肉羹。宋代刘敬叔的《异苑》卷三:
“吾被拘系,
方见烹臛。”
(8)土苏:芦菔,即萝卜。宋代陈达叟的《本心斋蔬食谱》:
“土酥,
芦菔也。亦名地酥。”
(9)愤:乱。《文选·班固〈幽通赋〉》:
“周贾荡而贡愤兮,齐生死
与祸福。”李善注引曹大家曰:
“贡,溃也。愤,乱也。”
服药食忌
但服药不可多食生芫荽及蒜,杂生菜、诸滑物、肥猪肉、犬肉、油
腻物,鱼脍腥膻等物及忌见丧尸、产妇、淹(1)秽之事。又不可食陈臭之
物。
有术勿食桃、李、雀肉、胡荽、蒜、青鱼等物。有黎芦勿食猩肉。
有巴豆勿食芦笋及野猪肉。有黄连、桔梗,勿食猪肉。有地黄勿食芜
荑。有半夏、菖蒲,勿食饴糖及羊肉。有细辛勿食生菜。有甘草勿食菘
菜、海藻。有牡丹勿食生胡荽。有商陆勿食犬肉。有常山勿食生葱、生
菜。有空青、朱砂,勿食血。
凡服药通忌:食血。有茯苓勿食醋。有鳖甲勿食苋菜。有天门冬勿
食鲤鱼。
凡久服药通忌:未不服药,又忌满日。正、五、九月忌巳日。二、
六、十月忌寅日。三、七、十一月忌亥日。四、八、十二月忌申日。
【注释】
(1)淹:腐败。《方言》第十三:
“澷、淹,败也。湿敝为澷,水敝为
淹。”郭璞注:
“皆谓水潦涝坏物也。”一云同腌(ɑ阿),意肮脏。
亦谓弄脏。元代王实甫的《西厢记》第五本第三折:
“枉腌了他金屋银
屏,枉污了他锦衾绣薩。”
食物利害
盖食物有利害者,可知而避之。
面有 (1)气,不可食。生料色臭,不可用。浆老而饭溲(2),不可
食。煮肉不变色,不可食。诸肉非宰杀者,勿食。诸肉臭败者,不可
食。诸脑,不可食。凡祭肉自动者,不可食。猪羊疫死者,不可食。曝
肉不干者,不可食。马肝、牛肝,皆不可食。兔合眼,不可食。烧肉,
不可用桑柴火。獐、鹿、麋,四月至七月勿食。二月内,勿食兔肉。诸
肉脯,忌米中贮之,有毒。鱼馁者,不可食。羊肝有孔者,不可食。诸
鸟自闭口者,勿食。蟹八月后可食,余月勿食。虾不可多食,无须及腹
下丹,煮之白者,皆不可食。腊月脯腊之属,或经雨漏所渍、虫鼠啮残
者,勿食。海味糟藏之属,或经湿热变损,日月过久者,勿食。六月、
七月,勿食雁。鲤鱼头,不可食,毒在脑中。诸肝青者,不可食。五月
勿食鹿,伤神。九月勿食犬肉,伤神。十月勿食熊肉,伤神。不时者,
不可食。诸果核未成者,不可食。诸果落地者,不可食。诸果虫伤者,
不可食。桃杏双仁者,不可食。莲子不去心,食之成霍乱。甜瓜双蒂
者,不可食。诸瓜沉水者,不可食。蘑菇勿多食,发病。榆仁不可多
食,令人瞑。菜着霜者,不可食。樱桃勿多食,令人发风。葱不可多
食,令人虚。芫荽勿多食,令人多忘。竹笋勿多食,发病。木耳色赤
者,不可食。三月勿食蒜,昏人目。二月勿食蓼,发病。九月勿食着霜
瓜。四月勿食胡荽,生狐臭。十月勿食椒,伤人心。五月勿食鮲,昏人
五脏。
【注释】
(1) (yǎn,奄):变质也。
(2)溲:用同“馊”。饭菜变质发出的一种酸臭味。
食物相反
盖食不欲杂,杂则或有所犯,知者分而避之。
马肉不可与仓米同食。马肉不可与苍耳、姜同食。猪肉不可与牛肉
同食。羊肝不可与椒同食,伤心。兔肉不可与姜同食,成霍乱。羊肝不
可与猪肉同食。牛肉不可与栗子同食。羊肚不可与小豆、梅子同食,伤
人。羊肉不可与鱼脍、酪同食。猪肉不可与芫荽同食,烂人肠。马奶子
不可与鱼脍同食,生癥瘕。鹿肉不可与鎨鱼同食。麋鹿不可与虾同食。
麋肉脂不可与梅、李同食。牛肝不可与鲇鱼同食,生风。牛肠不可与犬
肉同食。鸡肉不可与鱼汁同食,生癥瘕(1)。鹌鹑肉不可与猪肉同食,面
生黑。鹌鹑肉不可与菌子同食,发痔。野鸡不可与荞面同食,生虫。野
鸡不可与胡桃、蘑菇同食。野鸡卵不可与葱同食,生虫。雀肉不可与李
同食。鸡子不可与鳖肉同食。鸡子不可与生葱、蒜同食,损气。鸡肉不
可与兔肉同食,令人泄泻。野鸡不可与鲫鱼同食。鸭肉不可与鳖肉同
食。野鸡不可与猪肝同食。鲤鱼不可与犬肉同食。野鸡不可与鲇鱼同
食,食之令人生癫疾。鲫鱼不可与糖同食。鲫鱼不可与猪肉同食。黄鱼
不可与荞面同食。虾不可与猪肉同食,损精。虾不可与糖同食。虾不可
与鸡肉同食。大豆黄不可与猪肉同食。黍米不可与葵菜同食,发病。小
豆不可与鲤鱼同食。杨梅不可与生葱同食。柿、梨不可与蟹同食。李子
不可与鸡子同食。枣不可与蜜同食。李子、菱角不可与蜜同食。葵菜不
可与糖同食。生葱不可与蜜同食。莴苣不可与酪同食。竹笋不可与糖同
食。蓼不可与鱼脍同食。苋菜不可与鳖肉同食。韭不可与酒同食。苦苣
不可与蜜同食。薤不可与牛肉同食,生癥瘕。芥末不可与兔肉同食,生
疮。
【注释】
(1)癥瘕:指腹中包块。固定不移者为癥,游走不定者为瘕。
食物中毒
诸物品类,有根性本毒者,有无毒而食物成毒者,有杂合相畏、相
恶、相反成毒者,人不戒慎而食之,致伤腑脏和乱肠胃之气,或轻或
重,各随其毒而为害,随毒而解之。
如饮食后不知记何物毒,心烦满闷者,急煎苦参汁饮,令吐出。或
煮犀角汁饮之,或苦酒、好酒煮饮,皆良。
食菜物中毒,取鸡粪烧灰,水调服之。或甘草汁,或煮葛根汁饮
之。胡粉水调服亦可。食瓜过多,腹胀,食盐即消。食蘑菇、菌子毒,
地浆(1)解之。食菱角过多,腹胀满闷,可暖酒和姜饮之即消。食野山芋
毒,土浆解之。食瓠中毒,煮黍汁饮之即解。
食诸杂肉毒及马肝漏脯中毒者,烧猪骨灰调服,或芫荽汁饮之,或
生鮲汁亦可。食牛、羊肉中毒,煎甘草汁饮之。食马肉中毒,嚼杏仁即
消,或芦根汁及好酒皆可。食犬肉不消成?胀,口干,杏仁去皮、尖,
水煎饮之。
食鱼脍过多成虫瘕,大黄汁、陈皮末,同盐汤服之。食蟹中毒,饮
紫苏汁,或冬瓜汁,或生藕汁解之。干蒜汁、芦根汁亦可。食鱼中毒,
陈皮汁、芦根及大黄、大豆、朴硝汁皆可。食鸭子中毒,煮秫米汁解
之。食鸡子中毒,可饮醇酒,醋解之。
饮酒大醉不解,大豆汁、葛花、椹子、柑子皮汁皆可。
食牛肉中毒,猪脂炼油一两,每服一匙头,温水调下即解。食猪肉
中毒,饮大黄汁,或杏仁汁、朴硝汁,皆可解。
【注释】
(1)土浆:即地浆(土、泥浆)。
禽兽变异
禽兽形类,依本体生者,犹分其性质有毒无毒者,况异像变生,岂
无毒乎。倘不慎口,致生疾病,是不察矣。
兽歧尾,马蹄夜目,羊心有孔,肝有青黑,鹿豹文,羊肝有孔,黑
鸡白首,白马青蹄,羊独角,白羊黑头,黑羊白头,白鸟黄首,羊六
角,白马黑头,鸡有四距,爆肉不燥,马生角,牛肝叶孤,蟹有独螯,
鱼有眼睫,虾无须,肉入水动,肉经宿暖,鱼无肠、胆、腮,肉落地不
沾土,鱼目开合及腹下丹。
卷 三
米谷品
稻米 味甘、苦,平,无毒。主温中,令人多热,大便坚,不可多
食。即糯米也。(苏门者为上,酿酒者多用。)
粳米 味甘、苦,平,无毒。主益气,止烦,止泄,和胃气,长肌
肉。即今有数种。(香粳米,匾子米、雪里白,香子米。)香味尤胜。
诸粳米捣碎,取其圆净者,为圆米,亦作渴米。
粟米 味咸,微寒,无毒。主养肾气,去脾胃中热,益气。陈者
良,治胃中热,消渴,利小便,止痢。《唐本》注云:粟类多种,颗粒
细如粱米,捣细,取匀净者为浙米。
青粱米 味甘,微寒,无毒。主胃痹,中热,消渴。止泄痢,益气
补中,轻身延年。
白粱米 味甘,微寒,无毒。主除热,益气。
黄粱米 味甘,平,无毒。主益气和中,止泄。《唐本》注云:穗
大毛长,谷米俱粗于白粱。
黍米 味甘,平,无毒。主益气补中。多热,令人烦。久食昏人五
脏,令人好睡。肺病宜食。
丹黍米 味苦,微温,无毒。主咳逆,霍乱。止烦渴,除热。
稷米 味甘,无毒。主益气,补不足。关西谓之糜子米,亦谓郘
米。古者取其香可爱,故以供祭祀。
河西米 味甘,无毒。补中益气。颗粒硬于诸米。出本地。
绿豆 味甘寒,无毒。主丹毒,风疹,烦热。和五脏,行经脉。
白豆 味甘,平,无毒。调中,暖肠胃,助经脉。肾病宜食。
大豆 味甘,平,无毒。杀鬼气,止痛,逐水,除胃中热,下瘀
血,解诸药毒。作豆腐即寒而动气。
赤小豆 味甘、酸,平,无毒。主下水,排脓血,去热肿,止泻
痢,通小便。解小麦毒。
回回豆子 味甘,无毒。主消渴。勿与盐煮食之。出在回回地面,
苗似豆,今田野中处处有之。
青小豆 味甘寒,无毒。主热中,消渴。止下痢,去腹胀。产妇无
乳汁,烂煮三五升食之,即乳多。
豌豆 味甘,平,无毒。调顺荣卫,和中益气。
匾豆 味甘,微温。主和中。叶主霍乱吐下不止。
小麦 味甘,微寒,无毒。主除热,止烦躁,消渴,咽干。利小
便,养肝气,止痛。治唾血。
大麦 味咸,温、微寒,无毒。主消渴。除热,益气,调中。令人
多热,为五谷长。《药性论》云:能消化宿食,破冷气。
荞麦 味甘,平、寒,无毒。实肠胃,益气力。久食动风气,令人
头眩。和猪肉食之,患热风,脱人须眉。
白芝麻 味甘,大寒,无毒。治虚劳。滑肠胃,行风气,通血脉,
去头风,润肌肤。食后生豼(1)一合。与乳母食之,令子不生病。
胡麻 味甘,微寒。除一切痼疾。久服长肌肉,健人。油,利大
便,治胞衣不下。《修真秘旨》云神仙服胡麻法:久服面光泽,不饥,
三年水火不能害,行及奔马。
饧 味甘,微温,无毒。补虚乏,止渴,去血,健脾,治嗽。小儿
误吞钱,取一斤,渐渐尽食之即出。
蜜 味甘,平、微温,无毒。主心腹邪气,诸惊痫。补五脏不足,
气益中(2),止痛,解毒,明耳目,和百药,除众病。
曲 味甘,大暖。疗藏府中风气,调中益气,开胃消食,补虚冷。
陈久者良。
醋 味酸,温,无毒。消痈肿、散水气,杀邪毒,破血运,除癥块
坚积。醋有数种:酒醋、桃醋、麦醋、葡萄醋、枣醋、米醋为上,入药
用。
酱 味咸、酸,冷,无毒。除热止烦,杀百药、热汤火毒,杀一切
鱼、肉、菜蔬毒,豆酱主治胜面酱。陈久者尤良。
豉 味苦,寒,无毒。主伤寒,头痛,烦躁,满闷。
盐 味咸,温,无毒。主杀鬼蛊邪,疰毒伤寒,吐胸中痰癖,止心
腹卒痛。多食伤肺,令人咳嗽,失颜色。
酒 味苦、甘、辣,大热,有毒。主行药势,杀百邪,通血脉,厚
肠胃,润皮肤,消忧愁,多饮损寿伤神,易人本性。酒有数般,唯酝酿
以随其性。
虎骨酒 以酥炙虎骨捣碎,酿酒。治骨节疼痛,风疰,冷痹痛。
枸杞酒 以甘州枸杞依法酿酒。补虚弱,长肌肉,益精气,去冷
风,壮阳道。
地黄酒 以地黄绞汁酿酒。治虚弱,壮筋骨,通血脉。治腹内痛。
松节酒 仙方以五月五日采松节,剉碎,煮水酿酒。治冷风虚,骨
弱,脚不能履地。
茯苓酒 仙方,依法茯苓酿酒。治虚劳,壮筋骨,延年益寿。
松根酒 以松树下撅坑置瓮,取松根津液酿酒。治风。壮筋骨。
羊羔酒 依法作酒,大补益人。
五加皮酒 五加皮浸酒,或依法酿酒。治骨弱不能行走。久服壮筋
骨,延年不老。
腽肭脐酒 治肾虚弱,壮腰膝,大补益人。
小黄米酒 性热。不宜多饮,昏人五脏,烦热多睡。
葡萄酒 益气调中,耐饥强志。酒有数等,有西番者,有哈剌火
者,有平阳太原者,其味都不及哈剌火者。田地酒最佳。
阿剌吉酒 味甘、辣,大热,有大毒。主消冷坚积,去寒气。用好
酒蒸熬,取露成阿剌吉。
速儿麻酒 又名拨糟。味微甘、辣。主益气,止渴。多饮令人膨
胀、生痰。
【注释】
(1)豼(dàn,淡):食,吃。汉代荀悦的《汉纪·平帝纪》:
“莽之为
人……或云所谓鸱目虎喙豺声也,故能豼人,亦为人所豼。”
(2)气益中:据文义应改为“益中气”。
兽 品
牛肉 味甘,平,无毒。主消渴,止/
(1)泄,安中益气,补脾胃。
牛髓 补中,填精髓。
牛酥 凉,益心肺,止渴、嗽,润毛发,除肺痿、心热、吐血。
牛酪 味甘、酸,寒,无毒。主热毒,止消渴,除胸中虚热,身、
面热疮。
牛乳腐 微寒,润五脏,利大、小便,益十二经脉。微动气。
羊肉 味甘,大热,无毒。主暖中。头风,大风,汗出,虚劳,寒
冷,补中益气。
羊头 凉。治骨蒸,脑热,头眩,瘦病。
羊心 主治忧恚,膈气。
羊肝 性冷,疗肝气虚热,目赤、暗。
羊血 主治女人中风、血虚,产后血晕,闷欲绝者,生饮一升。
羊五脏 补人五脏。
羊肾 补肾虚,益精髓。
羊骨 热。治虚劳,寒中,羸瘦。
羊髓 味甘,温。主治男女伤中,阴气不足,利血脉,益经气。
羊脑 不可多食。
羊酪 治消渴。补虚乏。
黄羊 味甘,温,无毒。补中益气。治劳伤虚寒。其种类数等成
群,至于千数。
白黄羊 生于野草内。
黑尾黄羊 生于沙漠中。能走善卧,行走不成群。其脑不可食,髓
骨可食,能补益人。煮汤无味。
山羊 味甘,平,无毒。补益人,生山谷中。
(2) 味甘,平,无毒。补五劳七伤,温中益气。其肉稍腥。
马肉 味辛、苦,冷,有小毒。主热,下气,长筋骨,强腰膝,壮
健轻身。
马头骨 作枕令人少睡。
马肝 不可食。
马蹄 白者,治妇人漏下,白崩;赤者,治妇人赤崩。
白马茎 味咸、甘,无毒。主伤中,脉绝。强志,益气,长肌肉,
令人有子。能壮盛阴气。
马心 主喜忘。马肉内有生黑墨汁者,有毒,不可食。白马多有
之。
马乳 性冷,味甘。止渴,治热。有三等,一名升坚,一名晃禾
儿,一名窗元。以升坚为上。
野马肉 味甘,平,有毒。壮筋骨。与家马肉颇相似,其肉落地不
沾沙,然不宜多食。
象肉 味淡。不堪食,多食令人体重。胸前小横骨,令人能浮水。
身有百兽肉,皆有分段,惟鼻是本肉。
象牙 无毒。主诸铁及杂物入肉,刮取屑,细研和水敷(傅)疮上
即出。
驼肉 治诸风,下气,壮筋骨,润皮肤。疗一切顽麻风痹,肌肤紧
急,恶疮肿毒。
驼脂 在两峰内,有积聚者,酒服之良。
驼乳(系爱剌) 性温,味甘。补中益气,壮筋骨。令人不饥。
野驼 味甘,温平,无毒。治诸风。下气,壮筋骨,润皮肤。
驼峰 治虚劳风。有冷积者,用葡萄酒温调峰子油,服之良。好酒
亦可。
熊肉 味甘,无毒。主风痹,筋骨不仁。若腹中有积聚,寒热羸瘦
者,不可食之,终身不除。
熊白(3) 凉,无毒。治风。补虚损,杀劳虫。
熊掌 食之可御风寒。此是八(4)珍之数,古人最重之。十月勿食
之,损神。
驴肉 味甘,寒,无毒。治风狂,忧愁不乐。安心气,解心烦。
驴头肉 治多年消渴,煮食之良。乌驴者,尤佳。
驴脂 和乌梅作丸,治久疟。
野驴 性味同。比家驴鬃尾长,骨骼大。食之能治风眩。
麋肉 味甘,温,无毒。益气补中,治腰脚无力。不可与野鸡肉及
虾、生菜、梅、李果实同食,令人病。
麋脂 味辛,温,无毒。主痈肿恶疮,风痹,四肢拘缓。通血脉,
润泽皮肤。
麋皮 作靴能除脚气。
鹿肉 味甘,温,无毒。补中,强五脏,益气。
鹿髓 甘,温。主男女伤中,绝脉,筋急,咳逆,以酒服之。
鹿头 主消渴,夜梦见物。
鹿蹄 主脚膝疼痛。
鹿肾 主温中,补肾,安五脏,壮阳气。
鹿茸 味甘,微温,无毒。主漏下恶血,寒热惊痫。益气强志,补
虚羸,壮筋骨。
鹿角 微咸,无毒。主恶疮痈肿。逐邪气,除小腹血、急痛,腰脊
痛及留血在阴中。
獐肉 温。主补益五脏。《日华子》云:肉,无毒。八月至腊月食
之,胜羊肉;十二月以后至七月食之,动气。道家多食,言无禁忌也。
犬肉 味咸,温,无毒。安五脏,补绝伤,益阳道,补血脉,厚肠
胃,实下焦,填精髓。黄色犬肉尤佳。不与蒜同食,必顿损人。九月不
宜食之,令人损神。
犬四脚蹄 煮饮之,下乳汁。
猪肉 味苦,无毒。主闭血脉,弱筋骨,虚肥人。不可久食,动
风。患金疮者,尤甚。
猪肚 主补中益气,止渴。
猪肾 冷。和理肾气,通利膀胱。
猪四蹄 小寒。主伤挞,诸败疮。下乳。
野猪肉 味苦,无毒。主补肌肤,令人虚肥。雌者肉更美,冬月
食。橡子肉色赤,补人五脏,治肠风泻血,其肉味胜家猪。
江猪 味甘,平,无毒。然不宜多食,动风气,令人体重。
獭肉 味咸,平,无毒。治水气胀满。疗瘟疫病,诸热毒风,咳嗽
劳损。不可与兔同食。
獭肝 甘,有毒。治肠风下血及主疰病相染。
獭皮 饰领袖则尘垢不着。如风沙翳目,以袖拭之即出。又鱼刺鲠
喉中不出者,取獭爪爬项下即出。
虎肉 味咸、酸,平,无毒。主恶心欲呕。益气力。食之入山,虎
见则畏,辟三十六种魅。
虎眼睛 主疟疾,辟恶,止小儿热惊。
虎骨 主除邪恶气,杀鬼疰毒,止惊悸。主恶疮,鼠瘘。头骨尤
良。
豹肉 味酸,平,无毒。安五脏,补绝伤,壮筋骨,强志气。久食
令人猛,健忘,性粗疏,耐寒暑。正月勿食之,伤神。《唐本》注云:
车驾卤簿用豹尾,取其威重为可贵也。
土豹脑子 可治腰疼。
麅子 味甘,平,无毒。补益人。
麂肉 味甘,平,无毒。主五痔。多食能动人痼疾。
麝肉 无毒,性温。似獐肉而腥,食之不畏蛇毒。
狐肉 温,有小毒。《日华子》云:性暖,补虚劳。治恶疮疥。
犀牛肉 味甘,温,无毒。主诸兽蛇虫蛊毒,辟瘴气,食之入山不
迷其路。
犀角 味苦咸,微寒,无毒。主百毒蛊疰,邪鬼瘴气,杀钩吻、鸩
羽、蛇毒。疗伤寒、瘟疫。犀有数等:山犀、通天犀、辟尘犀、水犀、
镇帷犀。
狼肉 味咸,性热,无毒。主补益五脏,厚肠胃,填精髓。腹有冷
积者,宜食之。味胜狐、犬肉。
狼喉嗉皮 熟成皮条,勒头去头痛。
狼皮 熟作番皮,大暖。
狼尾 马胸堂前带之,辟邪,令马不惊。
狼牙 带之辟邪。
兔肉 味辛,平,无毒。补中益气。不宜多食,损阳事,绝血脉,
令人痿黄。不可与姜、橘同食,令人患卒心痛。妊娠不可食,令子缺
唇。二月不可食,伤神。
兔肝 主明目。
腊月兔头及皮毛 烧灰,酒调服之,治难产,胞衣不出,余血不
下。
塔剌不花(5)(一名土拨鼠)
味甘,无毒。主野鸡瘘疮,煮食之宜人。生山后草泽中。北人掘取
以食,虽肥,煮则无油,汤无味。多食难克化,微动气。
皮 作番皮,不湿透,甚暖。
头骨 去下颏肉,令齿全,治小儿无睡,悬之头边,即令得睡。
獾肉 味甘,平,无毒。治上气咳逆,水腹(6)不差。作羹食良。
野狸 味甘,平,无毒。主治鼠瘘,恶疮,头骨尤良。
黄鼠 味甘,平,无毒。多食发疮。
猴肉 味酸,无毒。主治诸风,劳疾。酿酒尤佳。
【注释】
(1)啘(yè,叶):干呕。滑寿本义:
“啘,干呕也。”《难经·十六
难》:
“假令得心脉,其外证面赤,口干,喜笑,其内证脐上有动气,
按之牢若痛,其病烦心,心痛,掌中热而/。”
(2) :有写羖(ɡǔ,古) ,黑色的公羊。亦泛指公羊。《史记
·秦本纪》:
“吾媵臣百里奚在焉,请以五羖羊皮赎之。”明代李时珍
的《本草纲目·兽一·羊》:
“牡羊曰羖,曰羝。”
(3)熊白:熊背上的脂肪。色白,故名。为珍贵美味。明代李时珍的
《本草纲目·兽二·熊》(释名)引陶弘景曰:
“脂即熊白,乃背上
肪,色白如玉,味甚美。寒月则有,夏月则无。”《北齐书·徐之才
传》:
“德正径造坐席,连索熊白。”
(4)八:原作为“入”,据《丛书》本改为“八”(从医理亦可改)。
(5)塔剌不花:一名土拨鼠,又名豚鼠。
(6)腹:从医理可改作“胀”。
禽 品
天鹅 味甘,性热,无毒。主补中益气。鹅有三、四等,金头鹅为
上,小金头鹅为次。有花鹅者,有一等鹅不能鸣者,飞则翎响,其肉微
腥,皆不及金头鹅。
鹅 味甘,平,无毒。利五脏,主消渴。孟诜云:肉性冷,不可多
食,亦发痼疾。《日华子》云:苍鹅性冷,有毒,食之发疮。白鹅无
毒,解五脏热,止渴。脂润皮肤,主治耳聋。鹅弹(1)补五脏,益气。有
痼疾者,不宜多食。
雁 味甘,平,无毒。主风挛拘急,偏枯,气不通利。益气,壮筋
骨,补劳瘦。
雁骨灰 和米泔洗头,长发。
雁膏 治耳聋,亦能长发。
雁脂 补虚羸,令人肥白。六月、七月勿食雁,令人伤神。
味甘,温,无毒。补中益气,食之甚有益人,炙食之味尤
美。然有数等,白 、黑头 、胡 ,其肉皆不同。
髓 味甘美。补精髓。
水札 味甘,平,无毒。补中益气。宜炙食之,甚美。
丹雄鸡 味甘,平,微温,无毒。主妇人崩中,漏下赤白。补虚,
温中,止血。
白雄鸡 味酸,无毒。主下气,疗狂邪,补中,安五脏。治消渴。
乌雄鸡 味甘酸,无毒。主补中,止痛,除心腹恶气。虚弱者,宜
食之。
乌雌鸡 味甘,温,无毒。主风寒湿痹,五缓六急,中恶,腹痛及
伤折骨疼。安胎血,疗乳难。
黄雌鸡 味酸,平,无毒。主伤中,消渴,小便数,不禁,肠澼,
泄痢。补五脏。先患骨热者,不可食。
鸡子 益气。多食令人有声。主产后痢,与小儿食之止痢。《日华
子》云:鸡子,镇心,安五脏。其白微寒,疗目赤热痛,除心下伏热,
止烦满、咳逆。
野鸡 味甘酸,微寒,有小毒。主补中益气,止泄痢。久食令人
瘦。九月至十一月食之,稍有益,他月即发五痔及诸疮。亦不可与胡桃
及菌子、木耳同食。
山鸡 味甘,温,有小毒。主五脏气喘不得息者,如食法服之。然
久食能发五痔,与荞麦面同食生虫。今辽阳有食鸡,味甚肥美;有角
鸡,味尤胜诸鸡肉。
鸭肉 味甘,冷,无毒。补内虚,消毒热,利水道及治小儿热惊
痫。
野鸭 味甘,微寒,无毒。补中益气,消食,和胃气。治水肿。绿
头者为上,尖尾者为次。
鸳鸯 味咸,平,有小毒。主治瘘疮。若夫妇不和者,作羹私与食
之,即相爱。
鸂鶒(2) 味甘,平,无毒。治惊邪。
鹁鸽 味咸,平,无毒。调精益气,解诸药毒。
鸠肉 味甘,平,无毒。安五藏,益气明目,疗痈肿,排脓血。
鸨肉 味甘,平,无毒。补益人。其肉粗,味美。
寒鸦 味酸、咸,平,无毒。主瘦病,止咳嗽,骨蒸羸弱者。
鹌鹑 味甘,温、平,无毒。益气,补五脏,实筋骨,耐寒暑,消
结热。酥煎食之,令人肥下焦。四月以前未可食。
雀肉 味甘,无毒,性热。壮阳道,令人有子。冬月者良。
蒿雀 味甘,温,无毒。食之益阳道,美于诸雀。
【注释】
(1)弹:禽鸟的蛋。元代杨蠫的《山居新话》:
“余家藏石子一块,色
青而质麄,大如鹅弹。”
(2)鸂鶒(xīchì,奚敕):水鸟名。形大于鸳鸯,而多紫色,好并游。
俗称紫鸳鸯。唐代温庭筠的《开成五年秋以抱疾郊野一百韵》:
“溟渚
藏鸂鶒,幽屏卧鹧鸪。”顾嗣立补注:
“《临海异物志》:鸂鶒,水
鸟,毛有五采色,食短狐,其中溪中无毒气。”
鱼 品
鲤鱼 味甘,寒,有毒。主咳逆上气,黄疸。止渴,安胎。治水
肿,脚气。天行病后不可食,有宿瘕者不可食。
鲫鱼 味甘,温、平,无毒。调中,益五脏。和莼菜作羹食良,患
肠风,痔瘘下血宜食之。
鲂鱼 甘,温、平,无毒。补益与鲫鱼同功。若作脍食,助脾胃。
不可与疳痢人食。
白鱼 味甘,平,无毒。开胃下食,去水气。久食发病。
黄鱼 味甘,有毒。发风动气,不可与荞面同食。
青鱼 味甘,平,无毒。南人作鲊。不可与芫荽、面酱同食。
鲇鱼 味甘,寒,有毒。勿多食,目赤、须赤者,不可食。
沙鱼 味甘咸,无毒。主心气鬼疰、蛊毒、吐血。
鳝鱼 味甘,平,无毒。主湿痹。天行病后,不可食。
鲍鱼 味腥臭,无毒。主坠蹶折,瘀血,痹在四肢不散者,及治妇
人崩血不止。
河 (1)鱼 味甘,温。主补虚,去湿气,治腰、脚、痔等疾。
石首鱼 味甘,无毒。开胃益气。干而味咸者,名为鲎。
阿八儿忽鱼 味甘,平,无毒。利五脏,肥美人,多食难克化,脂
黄肉粗,无鳞,骨止有脆骨。胞可作膘胶,甚粘。膘与酒化服之,消破
伤风。其鱼大者有一二丈长,一名鲟鱼,又名髒鱼。生辽阳东北海河
中。
乞里麻鱼 味甘,平,无毒。利五脏,肥美人。脂黄肉稍粗。脆亦
作膘。其鱼大者,有五六尺长,生辽阳东北海河中。
鳖肉 味甘,平,无毒。下气,除骨节间劳热、结实壅塞。
蟹 味咸,有毒。主胸中邪热结痛。通胃气,调经脉。
虾 味甘,有毒。多食损人。无须者,不可食。
螺 味甘,大寒,无毒。治肝气热。止渴,解酒毒。
蛤蜊 味甘,大寒,无毒。润五脏,止渴,平胃,解酒毒。
蹳 味苦,平,无毒。理胃气,实下焦。
蚌 冷,无毒。明目,止消渴,除烦,解热毒。
鲈鱼 平。补五脏,益筋骨,和肠胃。治水气。食之宜人。
【注释】
(1) (tún,豚):亦作“豚”、“豘”。小猪。亦泛指猪。《国语·
越语上》:
“生丈夫,二壶酒,一犬;生女子,二壶酒,一豚。”清代
姚鼐的《东梁山僧舍》诗:
“鹘盘树侧,豘出浪花巅。”
果 品
桃 味辛甘,无毒。利肺气,止咳逆上气,消心下坚积,除卒暴击
血,破癥瘕,通月水,止痛。桃仁,止心痛。
梨 味甘,寒,无毒。主热嗽,止渴,疏风,利小便,多食寒中。
柿 味甘,寒,无毒。通耳鼻气,补虚劳,肠澼不足,厚脾胃。
木瓜 味酸,温,无毒。主湿痹邪气,霍乱吐下,转筋不止。
梅实 味酸,平,无毒。主下气,除烦热,安心,止痢,住渴。
李子 味苦,平,无毒。主僵仆,瘀血,骨痛。除痼热,调中。
奈子 味苦,寒。多食令人腹胀,病人不可食。
石榴 味甘、酸,无毒。主咽渴。不可多食,损人肺。止漏精。
林檎 味甘、酸,温。不可多食,发热,涩气,令人好睡。
杏 味酸。不可多食,伤筋骨。杏仁有毒,主咳逆上气。
柑子 味甘,寒。去肠胃热,利小便,止渴。多食发痼疾。
橘子 味甘、酸,无毒,温。止呕,下气,利水道,去胸中瘕热。
橙子 味甘、酸,无毒。去恶心。多食伤肝气。皮甚香美。
栗 味咸,温,无毒。主益气,厚肠胃,补肾虚。
炒食,壅人气。
枣 味甘,无毒。主心腹邪气,安中养脾,助经脉,生津液。
樱桃 味甘,主调中,益脾气,令人好颜色。暗风人忌食。
葡萄 味甘,无毒。主筋骨湿痹。益气强志,令人肥健。
胡桃 味甘,无毒。食之令人肥健,润肌黑发。多食动风。
松子 味甘,温,无毒。治诸风头眩。散水气,润五脏,延年。
莲子 味甘,平,无毒。补中养神,益气,除百疾,轻身不老。
鸡头 味甘,平,无毒。主湿痹,腰膝痛。补中,除疾,益精气。
芰实 味甘,平,无毒。主安中,补五脏,轻身不饥。
荔枝 味甘,平,无毒。止渴生津,益人颜色。
龙眼 味甘,平,无毒。主五脏邪气。安志,厌食,除虫,去毒。
银杏 味甘、苦,无毒。炒食、煮食皆可,生食发病。
橄榄 味酸、甘,温,无毒。主消酒,开胃,下气,止渴。
杨梅 味酸甘,温,无毒。主祛痰,止呕,消食,下酒。
榛子 味甘,平,无毒。益气力,宽肠胃,健行,令人不饥。
榧子 味甘,无毒。主五痔,去三虫、蛊毒、鬼疰。
沙糖 味甘,寒,无毒。主心腹热胀。止渴,明目。(即甘蔗汁熬
成沙糖。)
甜瓜 味甘,寒,有毒。止渴,除烦热。多食发冷病,破腹。
西瓜 味甘,平,无毒。主消渴,治心烦。解酒毒。
酸枣 味酸、甘,平,无毒。主心腹寒热,邪结气聚。除烦。
海红 味酸、甘,平,无毒。治泄痢。
香圆(1) 味酸、甘,平,无毒。下气,开胸膈。
株子(2) 味酸、甘,平,无毒,性微寒。不可多食。
平波(3) 味甘,无毒。止渴生津。置衣服箧笥中,香气可爱。
八檐仁(4) 味甘,无毒。止咳下气,消心腹逆闷。(其果出回回田
地。)
必思荅(5) 味甘,无毒。调中顺气。(其果出回回田也。)
【注释】
(1)香圆:见“香圆煎”条注释。
(2)株子:见“株子煎”条注释。
(3)平波:即苹果,苹果又名频婆、天然子、柰,苹果为蔷薇科苹果属
植物的果实。
(4)八檐仁:见“荷莲兜子”条注释。
(5)必思荅:见“荷莲兜子”条注释
菜 品
葵菜 味甘,寒、平,无毒。为百菜主。治五脏六府寒热,羸瘦,
五癃。利小便,疗妇人乳难。
蔓菁 味苦,温,无毒。主利五脏,轻身,益气。蔓菁子明目。
芫荽 味辛,温,微毒。消谷,补五藏不足,通利小便。(一名胡
荽。)
芥 味辛,温,无毒。主除肾邪气,利九窍,明目,安中。
葱 味辛,温,无毒。主明目,补不足。治伤寒。发汗,去肿。
蒜 味辛,温,有毒。主散痈肿,除风邪,杀毒气。独颗者佳。
韭 味辛,温,无毒。安五脏,除胃热,下气,补虚。可以久食。
冬瓜 味甘,平、微寒,无毒。主益气,悦泽驻颜,令人不饥。
黄瓜 味甘,平、寒,有毒。动气发病,令人虚热。不可多食。
萝卜 味甘,温,无毒。主下气消谷,去痰癖。治渴。制面毒。
胡萝卜 味甘,平,无毒。主下气,调利肠胃。
天净菜 味苦,平,无毒。除面目黄,强志清神,利五脏。(即野
苦买。)
瓠 味苦,寒,有毒。主面目四肢浮肿,下水。多食令人吐。
菜瓜 味甘,寒,有毒。利肠胃,止烦渴。不可多食。(即稍
瓜。)
葫芦 味甘,平,无毒。主消水肿,益气。
蘑菇 味甘,寒,有毒。动气发病。不可多食。
菌子 味苦,寒,有毒。发五脏风,壅气,动脉痔,令人昏闷。
木耳 味苦,寒,有毒。利五脏,宣肠胃壅毒气。不可多食。
竹笋 味甘,无毒。主消渴。利水道,益气。多食发病。
蒲笋 味甘,无毒。补中益气,治血脉。
藕 味甘,平,无毒。主补中,养神,益气,除疾,消热渴,散
血。
山药 味甘,温,无毒。补中益气,治风眩。止腰痛,壮筋骨。
芋 味辛,平,有毒。宽肠胃,充肌肤,滑中。野芋不可食。
莴苣 味苦,冷,无毒。主利五脏,开胸膈,壅气,通血脉。
白菜 味甘,温,无毒。主通行肠胃,除胸中烦。解酒渴。
蓬蒿 味甘,平,无毒。主通利肠胃,安心气,消水饮。
茄子 味甘寒,有小毒。动风,发疮及痼疾。不可多食。
苋 味苦,寒,无毒。通九窍。苋子,益精。菜,不可与鳖同食。
芸台 味辛,温,无毒。主风热,丹肿,乳痈。
波錊(1) 味甘,冷,微毒。利五藏,通肠胃热,解酒毒。(即赤
根。)
莙薘(2) 味甘,寒,无毒。调中下气,去头风,利五藏。
香菜 味辛,平,无毒。与诸菜同食,气味香,辟腥。
蓼子 味辛,温,无毒。主明目,温中,耐风寒,下水气。
马齿 味酸,寒,无毒。主青盲,白!瞖去寒热,杀诸虫。
天花(3) 味甘,平,有毒。与蘑菇稍相似,未详其性。(生五台
山。)
回回葱 味辛,温,无毒。温中,消谷,下气,杀虫。久食发病。
甘露子 味甘,平,无毒。利五藏,下气,清神。(名滴露。)
榆仁 味辛,温,无毒。可作酱,甚香美。能助肺气,杀诸虫。
沙吉木儿(4) 味甘,平,无毒。温中,益气,去心腹冷痛。(即蔓
菁根。)
出莙薘儿 味甘,平,无毒。通经脉,下气,开胸膈。(即莙薘根
也。)
山丹根 味甘,平,无毒。主邪气腹胀。除诸疮肿。(一名百
合。)
海菜 味咸,寒,微腥,无毒。主瘿瘤。破气核、痈肿。勿多食。
蕨菜 味苦,寒,有毒。动气发病,不可多食。
薇菜 味甘,平,无毒。益气,润肌,清神,强志。
苦买菜 味苦,冷,无毒。治面目黄。强力,止困。可敷诸疮。
水芹 味甘,平,无毒。主养神益气,令人肥健,杀药毒。疗女人
赤沃。
【注释】
(1)波錊:亦作“菠眣”。即菠菜,又名“赤根”。宋代孙奕的《履斋
示儿编·字说·集字二》:
“《艺苑雌黄》云蔬品有颇陵者,昔人自颇
陵国将其子来,因以为名,今俗乃从草而为菠錊。”
(2)莙薘:波斯语gundar,甜菜的变种,嫩叶可作蔬菜。
(3)天花:见“天花包子”条注释。
(4)沙吉木儿:一写沙乞某儿,见“沙乞某儿汤”条注释。
料物性味
胡椒 味辛,温,无毒。主下气,除藏府风冷,去痰,杀肉毒。
小椒 味辛,热,有毒。主邪气咳逆。温中,下冷气,除湿痹。
良姜 味辛,温,无毒。主胃中冷逆,霍乱,腹痛。解酒毒。
茴香 味甘,温,无毒。主膀胱、肾经冷气。调中,止痛,住呕。
莳萝(1) 味辛,温,无毒。健脾开胃,温中,补水藏,杀鱼、肉
毒。
陈皮 味甘,平,无毒。止消渴,开胃气,下痰,破冷积。
草果 味辛,温,无毒。治心腹痛。止呕,补胃,下气。消酒毒。
桂 味甘、辛,大热,有毒。治心腹寒热,冷痰。利肝肺气。
姜黄 味辛、苦,寒,无毒。主心腹结积。下气破血,除风热。
荜拨 辛,温,无毒。温中下气,补腰脚痛,消食,除胃冷。
缩砂 味辛,温,无毒。主虚劳,冷泻,宿食不消。下气。
荜澄茄 味辛,温,无毒。消食下气,去心腹胀,令人能食。
甘草 味甘,平,无毒。和百药,解诸毒。
芫荽子 辛,温,无毒。消食。治五藏不足。杀鱼、肉毒。
干姜 味辛,温热,无毒。主胸膈咳逆。止腹痛,霍乱,胀满。
生姜 味辛,微温。主伤寒头痛,咳逆上气。止呕,清神。
五味子 味酸,温,无毒。益气,补精,温中,润肺,养藏,强
阴。
苦豆 味苦,温,无毒。主元藏虚冷,腹胁胀满,治膀胱疾。
红麹(2) 味甘,平,无毒。健脾,益气,温中。淹(3)鱼、肉内
用。
黑子儿(4) 味甘,平,无毒。开胃下气。烧饼内用,极香美。
马思荅吉(5) 味苦香,无毒。去邪恶气,温中利膈,顺气止痛,生
津解渴,令人口香。(生回回地面,云是极香种类。)
咱夫兰(6) 味甘,平,无毒。主心忧郁积,气闷不散,久食令人心
喜。(即是回回地面红花,未详是否。)
哈昔呢(7) 味辛,温,无毒。主杀诸虫,去臭气,破癥瘕,下恶除
邪,解蛊毒。(即阿魏。)
稳展 味辛,温,苦,无毒。主杀虫去臭。其味与阿魏同。又云,
即阿魏树根,腌羊肉香味甚美。
胭脂 味辛,温,无毒。主产后血运(8),心腹绞痛,可傅(9)游
肿。
栀子 味苦,寒,无毒。主五内邪气,疗目赤热。利小便。
蒲黄 味甘,平,无毒。治心腹寒热。利小便,止血疾。
回回青 味甘,寒,无毒。解诸药毒。可敷热毒疮肿。
【注释】
(1)莳萝:波斯语onload,又作孜然、慈勒。为一种多年生草本植物。
香料。
(2)红麴:亦作“麯”,简写“曲”。即红曲。
(3)淹:通腌。
(4)黑子儿:见“黑子儿烧饼”条注释。
(5)马思荅吉:见“马思荅吉汤”条注释。
(6)咱夫兰:见“八儿不汤”条注释。
(7)哈昔呢:一写“哈昔呢”,见“八儿不汤”条注释。
(8)运:通晕。眩晕、昏厥。《内经·灵枢·经脉》:
“五阴气俱绝,
则目系转,转则目运。”
(9)傅:同敷。涂搽。唐代元稹的《〈蟆子〉诗序》:
“故驥人成疮,
秋夏不愈,膏楸叶而傅之,则差。”
跋
《饮膳正要》三卷,元代忽思慧撰。前有天历三年常普兰奚进书
表,虞集奉敕序,盖元代饮膳太医官书也。明景泰间重刻于内府。此本
《皕宋楼藏书志》作元刊元印,余向见常熟瞿氏铁琴铜剑楼藏本,同出
一刻而楮印较逊。有景泰年序,知此为明本而非元本,特佚去景泰一序
耳。其书详于育婴,妊娠、饮膳卫生、食性宜忌。诸端虽未合于医学真
理,然可考见元人之俗尚。旧时民间传本极稀,近世藏目以钞本为多,
究不若此刻本之可信。余求之有年,十七年冬始觏之于东京静嘉文库,
因得借印流传,偿余夙昔之愿焉。
民国纪元十有九年十月海盐张元济

\backmatter

\end{document}