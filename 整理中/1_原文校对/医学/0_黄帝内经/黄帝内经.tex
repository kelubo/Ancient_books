% 黄帝内经
% 素问及灵枢概述等不适合分开存放的内容,置于此文件中。

\documentclass[12pt,UTF8]{ctexbook}

% 设置纸张信息。
\usepackage[a4paper,twoside]{geometry}
\geometry{
	left=25mm,
	right=25mm,
	bottom=25.4mm,
	bindingoffset=10mm
}

% 目录 chapter 级别加点(.)。
\usepackage{titletoc}
\titlecontents{chapter}[0pt]{\vspace{3mm}\bf\addvspace{2pt}\filright}{\contentspush{\thecontentslabel\hspace{0.8em}}}{}{\titlerule*[8pt]{.}\contentspage}

% 设置 part 和 chapter 标题格式。
\ctexset{
	part/name= {第,册},
	part/number={\chinese{part}},
	chapter/name={第,篇},
	chapter/number={\chinese{chapter}}
}

% 设置古文原文格式。
\newenvironment{yuanwen}{\bfseries\zihao{4}}

% 注脚每页重新编号,避免编号过大。
\usepackage[perpage]{footmisc}

\title{\heiti\zihao{0} 黄帝内经}
\date{}

\begin{document}
\maketitle

\tableofcontents

\frontmatter

\chapter{说明}
以明顧从德刻的《重广补注黄帝内经素问》为蓝本,参考清咸丰二年(1852)金山钱氏守山阁本和其校勘记,对全书进行了校勘。

《汉书·艺文志·方技略》载有医经、经方、神仙和房中四种中医典籍。其中医经有:《黄帝内经》十八卷,《外经》三十七卷;《扁鹊内经》九卷,《外经》十二卷;《白氏内经》三十八卷,《外经》三十六卷,《旁篇》二十五卷。

除《黄帝内经》外,其他医经均已亡佚。因此,《黄帝内经》便成了现存最早的中医经典了。所以称之为“经”,是因为它很重要。古人把具有一定法则、一般必须学习的重要书籍称之为“经”,如儒家的“六经”,老子的“道德经”以及浅显的“三字经”之类。所以称“内经”,并不是像吴昆《素问注》、王九达《内经合类》所称“五内阴阳之谓内”,也不像张介宾《类经》所说“内者,生命之道”,而仅仅是与“外”相对为言而已。这和“韩诗内传”、“韩诗外传”,“春秋内传”、“春秋外传”,《庄子》的《内篇》、《外篇》,《韩非子》的《内储》、《外储》之意相同,只是《黄帝外经》及扁鹊、白氏诸经均已不可得见罢了。

《黄帝内经》肯定不是黄帝所作,纯属后人伪托。这正如《淮南子·修务训》所指出的那样:“世俗之人多尊古而贱今,故为道者必托之于神农黄帝而后能入说。”冠以“黄帝”之名,意在溯源崇本,藉以说明我国医药文化发祥甚早。

《黄帝内经》究竟成于什么时代呢?宋林亿、高保衡等认为:“非大圣上智,孰能知之?战国之人何与焉?大哉《黄帝内经》十八卷,《针经》三卷,最出远古”;邵雍认为:“《素问》、《阴符》,七国时书也”;程颢认为:“《素问》书出战国之末”;司马光认为:“谓《素问》为真黄帝之书,则恐未可。黄帝亦治天下,岂终日坐明堂,但与歧伯论医药针灸耶?此周、汉之间医者依托以取重耳”;朱熹也认为:“至于战国之时,方术之士遂笔之于书以相传授,如列子之所引与夫《素问》、《握奇》之属……”;明方孝儒认为:“皆出战国、秦、汉之人”;方以智认为:“守其业而浸广之,《灵枢》、《素问》也,皆周末笔”。清魏荔彤认为:“轩岐之书类春秋战国人所为,而托于上古”。综上所说,将《黄帝内经》之成书定为战国时期是较为可信的,但也不能认为《素问》八十一篇、《灵枢》八十一篇尽出于战国。吕复对此发表过中肯的见解:“《内经素问》,世称黄帝岐伯问答之书,及观其旨意,殆非一时之言,其所撰述,亦非一人之手。刘向指为韩诸公子所著,程子谓出于战国之末,而其大略正如《礼记》之萃于汉儒而与孔子、子思之言并传也。”以上仅仅是从大多数学者的看法上加以归纳分析所做的判断。其实还可以从《黄帝内经》的哲学思想、理论体系、内容特点、先秦古韵等诸方面论证《黄帝内经》的绝大多数篇章(也即《黄帝内经》的主体部分)成于战国。限于篇幅,这方面的论述只好从略了。

对成于战国以后的少数篇章须做如下说明:《素问》的第七卷亡佚已久,唐王冰据其先师张公秘本而补入的《天元纪大论》、《五运行大论》、《六微旨大论》、《气交变大论》、《五常政大论》、《六元正纪大论》和《至真要大论》,实际上是另一部医书《阴阳大论》。以其用甲子纪年,便可断定必在东汉章帝元和二年(公元85 年)颁布四分历之后。以其曾被张仲景撰写《伤寒杂病论》时所引用,因此它一定得在张仲景之先。

《灵枢》中也有个别篇章晚出,如《阴阳系日月篇》有“寅者,正月之生阳也”句,故可断定成于汉武帝太初元年(公元前100 年)颁布太初历之后。

《素问》中的第七十二篇《刺法论》和第七十三篇《本病论》,在王冰次注《素问》时已是有目无文,宋刘温舒著《素问入式运气论奥》时却将该二篇作为《素问遗篇》陈列于后。因此可以认为这两篇当系唐宋间之伪作。

总而言之,《黄帝内经》非自一人一手,其笔之于书,应在战国,其个别篇章成于两汉。至于王冰之所补与刘温舒之所附不应视为《黄帝内经》文,但仅依惯例而仍其旧亦无不可。

《素问》之名最早见于张仲景《伤寒杂病论·自序》。他说:“撰用《素问》、《九卷》、《八十一难》、《阴阳大论》、《胎胪》、《药录》。”

迄今1700 多年,《素问》之名未曾改变。为什么叫《素问》?林亿、高保衡等人的“新校正”说:“所以名《素问》之义,全元起有说云:“素者本也,问者黄帝问岐伯也。方陈性情之源,五行之本,故曰《素问》。’元起虽有此解,义未甚明。按《乾凿度》云:“夫有形者生于无形,故有太易、有太初、有太始、有太素。太易者,未见气也;太初者,气之始也;太始者,形之始也;太素者,质之始也。’气形质具而苛瘵由是萌生。故黄帝问此太素质之始也。《素问》之名,义或由此。”人是具备气形质的生命体,难免会有小大不同的疾病发生,故以问答形式予以阐明,这就是《素问》本义。隋杨上善整理《内经》,迳称为《黄帝内经太素》是颇有见地的。

《灵枢》最早称《针经》。《灵枢》第一篇《九针十二原》就有“先立《针经》”之语,无疑等于自我介绍。后来又称为《九卷》(见张仲景《伤寒论》序),晋皇甫谧复又称之为《针经》。再后又有《九虚》(见《高丽史书》、《宋志》及林亿引文等)、《九灵》(见《隋志》、《唐志》、《宋志》等)、《黄帝针经》(见《七录》、《隋志》、《唐志》及新罗国、高丽国史书等)等名。《灵枢》一名,始见于王冰《素问》序及王冰的《素问》注语中。王冰在注《素问》时,曾两次引用“经脉为里,支而横者为络,络之别者为孙络”这句话,在《三部九候论》中引用时称“《灵枢》曰”,在《调经论》中引用时又称“《针经》曰”,是知《灵枢》即《针经》也。而其他《素问》注中所引《针经》者,皆为《灵枢》之文,则更证明了这一点。

《灵枢》名称的演变大略如此,至其命名之义则需具体分析。因其书主要研究针刺问题,故称《针经》;因其卷帙为九卷,故名曰《九卷》,并因此而有《九灵》、《九虚》等名。何以称《灵枢》呢?马翃说:“《灵枢》者,正以枢为门户阖辟所系,而灵乃至神至玄之称。此书之切,何以异是?”

张介宾则简言之说:“神灵之枢要,是谓《灵枢》。”这些说法都是以枢机之玄奥为依据的。

《素问》自战国时代成书到齐梁间全元起作《素问训解》时,一直保持九卷的旧制。只是到全元起注《素问》时,《素问》的第七卷已经亡佚了。

王冰认为是“惧非其人而时有所隐,故第七一卷师氏藏之”的缘故。王冰自谓“得先师张公秘本”,“因而撰注,用传不朽,兼旧藏之卷,合八十一篇二十四卷”。由于王冰补入了《天元纪大论》、《五运行大论》、《六微旨大论》、《气交变大论》、《五常政大论》、《六元正纪大论》和《至真要大论》等七篇大论,并将《素问》全文广为次注,所以才从原来的九卷大大地扩展为二十四卷了。从而成了至今行世的《黄帝内经素问》。当然世上还存在有元代胡氏“古林书堂”十二卷刊本和明代正统年间所刊五十卷《道藏》本,但其内容、篇目次第并无变动,一仍王冰之旧。

至于《灵枢》,虽有《九卷》、《九虚》、《九灵》和《针经》等几个传本系统,但隋唐以后却都亡佚了。宋臣林亿、高保衡等校正医书时亦因其残缺过甚而欲校不能。南宋史崧氏所献的《灵枢经》虽与王冰所引之《灵枢》及王唯一所引之《灵枢》在内容上均有所不同,但毕竟是现今行世的唯一版本。史崧之所以将《灵枢》改成二十四卷,也只是为了与王冰所注之《素问》卷数相同而别无深意。因为原本这两部书都是九卷,现在则都成二十四卷。

元代胡氏“古林书堂”刊本将《灵枢》并为十二卷亦是与其所刊《素问》十二卷本相匹配。至于明刊《道藏》本之《灵枢》只二十三卷而不是五十卷,则是因为《灵枢》较《素问》文字量少之故。

总之,王冰次注的二十四卷本《素问》是现存最早、又经北宋校正医书局校正的。史崧改编的二十四卷本《灵枢》也是现存最早和唯一行世的。

四、《黄帝内经》是我国战国时代以前的医学大成

如前所述,《黄帝内经》既非一时之作,亦非自一人之手,而是战国以前的许许多多的医学著作的总结。这不仅可以从《素问》、《灵枢》各八十一篇这一点得到证明,而且也可以从《黄帝内经》引用了大量的古文献及《素问》、《灵枢》互引、各篇互引等现象上得到证明。

《黄帝内经》所引的古文献大约有50 余种,其中既有书名而内容又基本保留者有《逆顺五体》、《禁服》、《脉度》、《本藏》、《外揣》、《五色》、《玉机》、《九针之论》、《热论》、《诊经》、《终始》、《经脉》、《天元纪》、《气交变》、《天元正纪》、《针经》等16 种;仅保存零星佚文者,有《刺法》、《本病》、《明堂》、《上经》、《下经》、《大要》、《脉法》、《脉要》等8 种;仅有书名者,有《揆度》、《奇恒》、《奇恒之势》、《比类》、《金匮》、《从容》、《五中》、《五过》、《四德》、《上下经》、《六十首》、《脉变》、《经脉上下篇》、《上下篇》、《针论》、《阴阳》、《阴阳传》、《阴阳之论》、《阴阳十二官相使》、《太始天元册》、《天元册》等29 种。至于用“经言”、“经论”、“论言”或“故曰……”、“所谓……”等方式引用古文献而无法知其书名者亦复不少。

正是由于上述情况,我们才说《黄帝内经》的成书是对我国上古医学的第一次总结,《黄帝内经》是仅存的战国以前医学的集大成之作。

五、《黄帝内经》的学术思想

《黄帝内经》接受了我国古代唯物的气一元论的哲学思想,将人看作整个物质世界的一部分,宇宙万物皆是由其原初物质“气”形成的。在“人与天地相参”、“与日月相应”的观念指导下,将人与自然紧密地联系在一起。

人的一切正常的生理活动和病理变化与整个自然界是息息相关的。为了进一步明确这一点,拟从以下几方面加以阐述:

(一)“气”是宇宙万物的本原

如同老子所说:“有物混成,先天地生。寂兮寥兮,独立而不改,周行而不殆,可以为天下母”,“道之为物,惟恍惟惚”,“其上不皎,其下不昧”,“视之不见名曰夷,听之不闻名曰希,搏之不得名曰微”,这都是在说构成世界的原初物质——形而上者的“道”。宋钘、尹文将这种原初物质称之为“气”。《黄帝内经》受这些学说的影响,也认为“气”是宇宙万物的本原。在天地未形成之先便有了气,充满太虚而运行不止,然后才生成宇宙万物。如《天元纪大论》:“臣積(稽)考《太始天元册》文曰:“太虚寥廓,肇基化元,万物资始,五运终天。布气真灵,珝统坤元,九星悬朗,七曜周旋。曰阴曰阳,曰柔曰刚,幽显既位,寒暑弛张,生生化化,品物咸章。’”这其实是揭示天体演化及生物发生等自然法则。在宇宙形成之先,就是太虚。太虚之中充满着本元之气,这些气便是天地万物化生的开始。由于气的运动,从此便有了星河、七曜,有了阴阳寒暑,有了万物。阴阳五行的运动,总统着大地的运动变化和万物的发生与发展。

(二)人与自然的关系

《黄帝内经》认为人与自然息息相关,是相参相应的。自然界的运动变化无时无刻不对人体发生影响。《素问·宝命全形论》说:“人以天地之气生,四时之法成”。这是说人和宇宙万物一样,是禀受天地之气而生、按照四时的法则而生长的,所以《素问·四气调神大论》说:“夫四时阴阳者,万物之根本也。所以圣人春夏养阳,秋冬养阴,以从其根,故与万物沉浮于生长之门。逆其根,则伐其本,坏其真矣。”《素问·阴阳应象大论》也说:“天有四时五行,以生长收藏,以生寒暑燥湿风;人有五脏化五气,以生喜怒悲忧恐。”人生天地之间,人必须要依赖天地阴阳二气的运动和滋养才能生存,正如《素问·六节藏象论》所说:“天食人以五气,地食人以五味。五气入鼻,藏于心肺,上使五色修明,音声能彰。五味入口,藏于肠胃,味有所藏,以养五脏气。气和而生,津液相成,神乃自生。”

人体的内环境必须与自然界这个外环境相协调、相一致。这就要求人对自然要有很强的适应性。比如《灵枢·五癃津液别》说:“天暑衣厚则腠理开,故汗出。……天寒则腠理闭,气湿不行,水下留于膀胱,则为溺与气。”

这明显是水液代谢方面对外环境的适应。人的脉象表现为春弦、夏洪、秋毛、冬石,同样是由于人体气血对春夏秋冬不同气候变化所做出的适应性反应,以此达到与外环境的协调统一。如果人们违背了春生夏长秋收冬藏的养生之道,就有可能产生病变。如《素问·四气调神大论》说:“逆春气,则少阳不生,肝气内变;逆夏气,则太阳不长,心气内洞;逆秋气,则太阳不收,肺气焦满;逆冬气,则少阴不藏,肾气独沉。”就是一日之内、日夜之间,人体也会随天阳之气的盛衰而相应变化。如果违反了客观规律,也会受到损害。如《素问·生气通天论》说:“故阳气者,一日而主外,平旦人气生,日中而阳气隆,日西而阳气已虚,气门乃闭。是故暮而收拒,无扰筋骨,无见雾露,反此三时,形乃困薄。”

人与自然这种相参相应的关系在《黄帝内经》中是随处可见的。无论是生理还是病理,无论是养生预防还是诊断与治疗,都离不开这种理论的指导。

(三)人是阴阳对立的统一体

人是阴阳对立的统一体,这在生命开始时已经决定了。《素问·生气通天论》说:“生之本,本于阴阳。”具有生命力的父母之精相媾,也就是阴阳二气相媾,形成了生命体。诚如《灵枢·决气》所说:“两神相搏,合而成形,常先身生,是谓精。”生命体形成之后,阴阳二气存在于其中,互为存在的条件。相互联系、相互资生、相互转化,又相互斗争。如《素问·阴阳应象大论》所说:“阴在内,阳之守也;阳在外,阴之使也。”《素问·生气通天论》说:“阴者,藏精而起亟也,阳者,卫外而为固也。”这两句话精辟地解释了人体阴阳的对立统一关系。

从人体的组织结构上看,《黄帝内经》把人体看成是各个层次的阴阳对立统一体。《素问·金匮真言论》说:“夫言人之阴阳,则外为阳,内为阴;言人身之阴阳,则背为阳,腹为阴;言人身之脏腑中阴阳,则脏者为阴,腑者为阳……故背为阳,阴中之阳,心也;背为阳,阳中之阴,肺也;腹为阴,阴中之阴,肾也;腹为阴,阴中之至阴,脾也。”《黄帝内经》还把每一脏、每一腑再分出阴阳。从而使每一层次,无论整体与局部、组织结构与生理功能都形成阴阳的对立统一,所以说人是阴阳的对立统一体。

(四)人体是肝心脾肺肾五大系统的协调统一体

《黄帝内经》所说的五脏,实际上是指以肝心脾肺肾为核心的五大系统。

以心为例:心居胸中,为阳中之太阳,通于夏气,主神明,主血脉,心合小肠,生血、荣色,其华在面,藏脉、舍神、开窍于舌、在志为喜。在谈心的生理、病理时,至少要从以上诸方面系统地加以考虑才不至于失之片面。因此可以每一脏都是一大系统,五大系统通过经络气血联系在一起,构成一个统一体。这五大系统又按五行生克制化规律相互协调、资生和抑制,在相对稳态的情况下,各系统按其固有的规律从事各种生命活动。

(五)《黄帝内经》的生命观

《黄帝内经》否定超自然、超物质的上帝的存在,认识到生命现象来源于生命体自身的矛盾运动。认为阴阳二气是万物的胎始。《素问·阴阳应象大论》说:“阴阳者,万物之能(读如胎)始也。”对整个生物界,则曰:天地气交,万物华实;又曰:天地合气,命之曰人。阴阳二气是永恒运动的,其基本方式就是升降出入。《素问·六微旨大论》说:“出入废,则神机化灭;升降息,则气立孤危。故非出入,则无以生长壮老已;非升降则无以生长化收藏。是以生降出入,无器不有。”《黄帝内经》把精看成是构成生命体的基本物质,也是生命的原动力。《灵枢·本神》说:“生之来谓之精,两精相搏谓之神”。在《灵枢·经脉》还描绘了胚胎生命的发展过程:“人始生,先成精,精成而脑髓生。骨为干,脉为营,筋为刚,肉为墙,皮肤坚而毛发长”。这种对生命物质属性和胚胎发育的认识是基本正确的。

(六)《黄帝内经》的形神统一观

《黄帝内经》对于形体与精神的辩证统一关系做出了的说明,指出精神统一于形体,精神是由形体产生出来的生命运动。如《灵枢·邪气脏腑病形》说:“十二经脉、三百六五络,其气血皆上于面而走空窍,其精阳气上走于目而为睛(视),其别气走于耳而为听,其宗气上出于鼻而为臭,其浊气出于胃走唇舌而为味。”这就将视听嗅味等感觉认为是由于气血津液注于各孔窍而产生的生理功能。对于高级神经中枢支配的思维活动也做出了唯物主义解释。《灵枢·本神》说:“故生之来谓之精,两精相搏谓之神,随神往来者谓之魂,并精出入者谓之魄。所以任物者谓之心,心之所忆谓之意,意之所存谓之志,因志而存变谓之思,因思而远慕谓之虑,因虑而处物谓之智。”

如此描写人的思维活动基本上是正确的。在先秦诸子中对神以及形神关系的认识,没有哪一家比《黄帝内经》的认识更清楚、更接近科学。关于形神必须统一、必须相得的论述颇多,如《灵枢·天年》:“神气舍心,魂魄毕具,乃成为人。”又《素问·上古天真论》:“形与神俱而尽终其天年”。如果形神不统一、不相得,人就得死。如《素问·汤液醪醴》:“形弊血尽……神不使也。”又《素问·逆调论》:“人身与志不相有,曰死。”《黄帝内经》这种形神统一观点对我国古代哲学是有很大贡献的。

六、《黄帝内经》的理论体系

历代医家用分类法对《黄帝内经》进行研究。其中分类最繁的是杨上善,分做18 类;最简的是沈又彭,分做4 卷。各家的认识较为一致的是脏象(包括经络)、病机、诊法和治则四大学说。这四大学说是《黄帝内经》理论体系的主要内容。现分述如下:

(一)脏象学说

脏象学说是研究人体脏腑组织和经络系统的生理功能、相互之间的联系以及在外的表象乃至与外环境的联系等等之学说。

脏象学说是以五脏六腑十二经脉为物质基础的。《灵枢·经水》说:“若夫八尺之士,皮肉在此,外可度量切循而得之。其死,可解剖而视之,其脏之坚脆,腑之大小,谷之多少,脉之长短,血之清浊,气之多少,十二经之多血少气,与其少血多气,与其皆血多气,与其皆少血气,皆有大数。”当然有关解剖学之内容还远不止此,但更重要的还是通过大量的医疗实践不断认识、反复论证而使此学说逐渐丰富起来的,最终达到了指导临床的高度。

《黄帝内经》充分认识到“有诸内必形诸外”的辩证法则,使脏象学说系统而完善。正如《灵枢·本脏》说:“视其外应以知其内脏,则知所病也。”

脏象学说主要包括脏腑、经络和精气神三部分。脏腑又由五脏、六腑和奇恒之腑组成。五脏,即肝、心、脾、肺、肾。《素问·五脏别论》指出:“所谓五脏者,藏精气而不泻也,故满而不能实。”《灵枢·本脏》说:“五脏者,所以藏精、神、血、气、魂、魄者也。”六腑,即胆、胃、大肠、小肠、膀胱和三焦。《素问·五脏别论》说:“六腑者,传化物而不藏,故实而不能满也。”奇恒之腑也属于腑,但又异于常。系指脑、髓、骨、脉、胆和女子胞。这里边胆即是大腑之一,又属于奇恒之腑。《素问·五脏别论》说:“脑、髓、骨、脉、胆、女子胞,此六者地气之所生也,皆藏于阴而象于地,故藏而不泻,名曰奇恒之腑。”脏腑虽因形态功能之不同而有所分,但它们之间却不是孤立的,而是相互合作、相互为用的。如《素问·五脏生成篇》说:“心之合脉也,其荣色也,其主肾也;肺之合皮也,其荣毛也,其主心也;肝之合筋也,其荣爪也,其主肺也;脾之合肉也,其荣唇也,其主肝也;肾之合骨也,其荣发也,其主脾也。”又如《灵枢·本输》说:“肺合大肠,大肠者,传导之腑。心合小肠,小肠者,受盛之腑。肝合胆,胆者,中精之腑。脾合胃,胃者,五谷之腑。肾合膀胱,膀胱者,津液之腑。三焦者,中渎之腑也,水道出焉,属膀胱,是孤之腑也。是六腑之所与合者。”

经络系统可以分经脉、络脉和腧穴三部分。《灵枢·本脏》说:“经脉者,所以行血气而营阴阳,濡筋骨,利关节者也。”经脉有正经十二:手太阴肺经、手阳明大肠经、足阳明胃经、足太阴脾经、手少阴心经、手太阳小肠经、足太阳膀胱经、足少阴肾经、手厥阴心包经、手少阳三焦经、足少阳胆经、足厥阴肝经。十二经脉首尾相联如环无端,经气流行其中周而复始。另有别于正经的奇经八脉:督脉、任脉、冲脉、带脉、阴跷脉、阳跷脉、阴维脉、阳维脉。(需要说明的是“奇经八脉”一名始于《难经·二十七难》)

经脉之间相交通联络的称络脉。其小者为孙络不计其数;其大者有十五,称十五络脉。《灵枢·经脉》言之甚详,这里仅摘其要:手太阴之别,名曰列缺;手少阴之别,名曰通里;手心主之别,名曰内关;手太阳之别,名曰支正;手阳明之别,名曰偏历;手少阳之别,名曰外关;足太阳之别,名曰飞阳;足少阳之别,名曰光明;足阳明之别,名曰丰隆;足太阴之别,名曰公孙;足少阴之别,名曰大钟;足厥阴之别,名曰蠡沟;任脉之别,名曰尾翳;督脉之别,名曰长强;脾之大络,名曰大包。

腧穴为经气游行出入之所,有如运输,是以名之。《黄帝内经》言腧穴者,首见《素问·气穴论》,再见于《素问·气府论》,两论皆言三百六十五穴。实际《气穴论》载穴三百四十二,《气府论》载穴三百八十六。

精气神为人身三宝。精,包括精、血、津、液;气,指宗气、荣气、卫气;神,指神、魂、魄、意、志。《灵枢·本脏》说:“人之血气精神者,所以奉身而周于性命者也。”精和气是构成人体的基本物质,气和神又是人体的复杂的功能,也可以认为气为精之御,精为神之宅,神为精气之用。

(二)病机学说

研究疾病发生、发展、转归及变化等等之内在机理的学说称病机学说。

《黄帝内经》所说“审察病机,无失气宜”“谨守病机,各司其属”(皆出自《素问·至真要大论》)皆为此学说之内容。现从病因、发病和病变三方面加以叙述:

1.病因:引起人发病的原因很多,《黄帝内经》将其归纳为二类。《素问·调经论》说:“夫邪之生也,或生于阴,或生于阳。其生于阳者,得之风雨寒暑;其生于阴者,得之饮食居处,阴阳喜怒”。风雨寒暑实为“六淫”的概括;阴阳喜怒乃“七情”的概括;饮食居处即“饮食劳倦”。可以认为这就是后世三因说之滥觞。

2.发病:正邪双方力量的对比,决定着疾病的发生与发展。《灵枢·百病始生》说:“风雨寒热,不得虚邪,不能独伤人。卒然逢疾风暴雨而不病者,盖无虚。故邪不能独伤人,此必因虚邪之风,与其身形,两虚相得,乃克其形。”这就是“正气存内,邪不可干”之意。《素问·上古天真论》所说“精神内守,病安从来”,《素问·评热病论》所说“邪之所凑,其气必虚”等,都论证了这一点。

3.病变:疾病的变化是复杂的,《黄帝内经》概括病变也是多方面的,有从阴阳来概括的,如《素问·阴阳应象大论》:“阳受风气,阴受湿气”;“阳病者上行极而下,阴病者下行极而上”:“阴胜则阳病,阳胜则阴病。阳胜则热,阴胜则寒”:“阳受之则入六腑,阴受之则入五腑”。《素问·宣明五气篇》亦有“邪入于阳则狂,邪入于阴则痹;搏阳则为癫,搏阴则为瘖;阳入之阴则静,阴出之阳则怒”。

用表里中外归纳的,如《素问·玉机真脏论》:“其气来实而强,此谓太过,病在外;其气来不实而微,此谓不及,病在中。”又如《素问·至真要大论》有“从内之外”、“从外之内”、“从内之外而盛于外”、“从外之内而盛于内”及“中外不相及”等病变规律。

用寒热归纳的,如《灵枢·刺节真邪》:“阳盛者则为热,阴盛者则为寒”。又如《素问·调经论》:“阳虚则外寒,阴虚则内热,阳盛则外热,阴盛则内寒”。

从虚实而论者,如《素问·通评虚实论》:“邪气盛则实,精气夺则虚。”又如《素问·调经论》:“气之所并为血虚,血之所并为气虚。”实指邪气盛,虚指正气衰。概括说来,有正虚而邪实者,有邪实而正不虚者,有正虚而无实邪者,有正不虚而邪不实者。

以上为病机学说之梗概。

(三)诊法学说

望闻问切四诊源于《黄帝内经》,如《素问·阴阳应象大论》说:“善诊者,察色按脉,先别阴阳,审清浊,而知部分;视喘息,听音声,而知所苦;观权衡规矩,而知病所主;按尺寸,观浮沉滑涩,而知病所生。以治无过,以诊则不失矣。”又如《灵枢·邪气脏腑病形》说:“见其色,知其病,命曰明;按其脉,知其病,命曰神;问其病,知其处,命曰工。”《黄帝内经》论诊法者甚多,谨按望闻问切之序列举如下:

1.望诊:包括观神色、察形态、辨舌苔。

观神色者如《灵枢·五色》:“五色各见其部,察其浮沉,以知浅深;察其泽天,以观成败;察其散抟,以知远近;视色上下,以知病处;积神于心,以知往今。”又如《灵枢·五阅五使》:“肺病者喘息鼻胀;肝病者,眦青;脾病者,唇黄;心病者,舌卷短,颧赤;肾病者,颧与颜黑。”又如《灵枢·五色》说:“赤色出两颧,大如母指者,病虽小愈,必卒死”。这些在临床上都是很有意义的。

察形态者,如《素问·经脉别论》:“诊病之道,观人勇怯、骨肉、皮肤,能知其情,以为诊法也。”这是察看人的骨肉皮肤而推断病情的例证。又如《素问·刺志论》说:“气实形实,气虚形虚,此其常也,反此者病。”

在临床上虚实是错综复杂的,只有知其常,才能达其变。

辨舌苔者,如《素问·热论》:伤寒五日,“口燥舌干而渴。”《素问·刺热论》:肺热病者,“舌上黄”。又如《灵枢》:“舌本烂、热不已者死。”其他如“舌本出血”、“舌本干”、“舌本强”、“舌卷”、“舌萎”等等不能一一列举。

2.闻诊:包括闻声和嗅气味。

闻声音者如《素问·阴阳应象大论》:“听音声而知所苦”,“脾在变动为哕”;又如《素问·刺热论》:“肝热病者,热争则狂言及惊。”再如《素问·调经论》:“神有余,则笑不休,神不足,则悲”。这些都是听患者的声音而诊断病情的。

其次是嗅气味,如《素问·金匮真言论》所说肝病其臭臊,心病其臭焦,脾病其臭香,肺病其臭腥,肾病其臭腐。

3.问诊:问讯患者的自觉症状以诊断病情是谓问诊。如《素问·三部九候论》说:“必审问其所始病,与今之所方病”,又如《素问·移精变气论》说:“闭户塞牖,系之病者,数问其情,以从其意”。又如《素问·疏五过论》:“凡欲诊病者,必问饮食居处,暴乐暴苦,始乐后苦”。

4.切诊:包括切脉与切肤。《黄帝内经》言切脉最详,实难备述,姑择其要:

(1)三部九候法:即分头手足三部,每部分天地人三候。详《素问·三部九候论》。

(2)人迎寸口脉法:即兼诊人迎和寸口两处之脉,互相比较。详见《灵枢·终始》、《四时气》、《禁服》、《五色》。

(3)调息法:即调医者之呼吸,诊病人之脉候。如《素问·平人气象论》:“常以不病调病人,医不病,故为病人平息以调之为法。人一呼脉一动,一吸脉一动,曰少气。人一呼脉三动,一吸脉三动,而躁、尺热,曰病温;尺不热、脉滑,曰病风;脉涩曰痹。人一呼脉四动以上,曰死;脉绝不至,曰死;乍疏乍数,曰死。”

(4)谓胃气脉:脉象之中有无胃气,至关重要,有胃气则生,无胃气则死。如《素问·平人气象论》说:“春胃微弦曰平;弦多胃少曰肝病;但弦无胃曰死”。“夏胃微钩曰平;钩多胃少曰心病;但钩无胃曰死”。“长夏胃微软弱曰平;弱多胃少曰脾病;但代无胃曰死”。“秋胃微毛曰平;毛多胃少曰肺病;但毛无胃曰死”。“冬胃微石曰平;石多胃少曰肾病;但石无胃曰死。”

(5)六纲脉:《黄帝内经》所载脉象很多,如浮、沉、迟、数、虚、实、滑、涩、长、短、弦、细、微、濡、软、弱、散、缓、牢、动、洪、伏、芤、革、促、结、代、大、小、急、坚、盛、躁、疾、搏、钩、毛、石、营、喘等等。但常以六脉为纲加以概括,如《灵枢·邪气脏腑病形》说:“调其脉之缓、急、大、小、滑、涩,而病变定矣。”

其次是切肤:肤泛指全身肌肤,按肌肤而协助诊断的内容很多,如“按而循之”、“按而弹之”等等。但论之最详细的是切尺肤。如《灵枢·论疾诊尺》说:“余欲无视色持脉,独诊其尺,以言其病,从外知内,为之奈何?”对曰:“审其尺之缓、急、大、小、滑、涩,肉之坚脆,而病形定矣。”因为脉象与尺肤有必然的联系,故诊病时亦可互相配合。故《灵枢·邪气脏腑病形》说:“脉急者,尺之皮肤亦急;脉缓者,尺之皮肤亦缓;脉小者,尺之皮肤亦减而少气;脉大者,尺之皮肤亦贲而起;脉滑者,尺之皮肤亦滑;脉涩者,尺之皮肤亦涩。凡此变者,有微有甚”。

(四)治则学说

研究治疗法则的学说称治则学说。《黄帝内经》对治疗法则是颇有研究的,至少可以从以下几方面加以概括:

1.防微杜渐:包括未病先防和已病防变。如《素问·上古天真论》说:“虚邪贼风,避之有时;恬淡虚无,真气从之;精神内守,病安从来”,“饮食有节,起居有常,不妄作劳,故能形与神俱,而尽终其天年,度百岁乃去”。他如“春夏养阳、秋冬养阴”等等皆言预防疾病。有病早治防其传变的如:《素问·阴阳应象大论》说:“故邪风之至,疾如风雨,故善治者治皮毛,其次治肌肤,其次治筋脉,其次治六腑,其次治五脏。治五脏者,半死半生也。”

2.因时、因地、因人制宜:因时制宜者,如《素问·六元正纪大论》:“司气以热,用热无犯;司气以寒,用寒无犯;司气以凉,用凉无犯;司气以温,用温无犯”。这是告诫医者用药勿犯四时寒热温凉之气。

因地制宜者,如“至高之地,冬气常在;至下之地,春气常在”(同上篇),在治疗时不可一概而论,必须加以区别。而《素问·异法方宜论》论述东南西北中“一病而治各不同”的因地制宜甚详,如东方之域,其治宜砭石;西方之域,治宜毒药;北方之域,治宜灸祔;南方之域,治宜微针;中央之域,治宜导引按偁。

因人制宜者,如《素问·五常政大论》:“能(读如耐)毒者,以厚药;不胜毒者,以薄药。”又如《素问·征四失论》:“不适贫富贵贱之居,坐之厚薄,形之寒温,不适饮食之宜,不别人之勇怯,不知比类,足以自乱,不足以自明,此治之三失也。”

3.标本先后:即因病之主次而先后施治。《素问·至真要大论》说:“夫标本之道,要而博,小而大,可以言一而知百病之害。言标与本,易而勿损,察本与标,气可令调”。有关标本先后施治的大法在《素问·标本病传论》中言之最详,兹不赘述。

4.治病求本:这是《黄帝内经》治则中最根本的一条。《素问·阴阳应象大论》说:“治病必求于本。”

5.因势利导:在治病求本的基础上巧妙地加以权变。如“因其轻而扬之,因其重而减之,因其衰而彰之”,“其高者,因而越之;其下者,引而竭之;中满者,泻之于内”,“其在皮者,汗而发之”。(皆出《素问·阴阳应象大论》)

6.协调阴阳:此为治疗之大法,故《素问·至真要大论》说:“谨察阴阳所在而调之,以平为期”,《素问·阴阳应象大论》说:“阳病治阴,阴病治阳”。

7.正治反治:正治亦称逆治,是与病情相逆的直折的治疗方法。比如“热者寒之,寒者热之,虚者补之,实者泻之”之类;反治也称从治,如“寒因寒用,热因热用,通因通用,塞因塞用”之类。故《素问·至真要大论》说:“微者逆之,甚者从之。逆者正治,从者反治,从少从多,观其事也。”

8.适事为度:无论扶正还是祛邪都应适度,对于虚实兼杂之症,尤当审慎。切记“无盛盛,无虚虚”,即使用补,也不能过。因为“久而增气,物化之常也,气增而久,夭之由也”(《素问·至真要大论》)。《素问·五常政大论》还说:“大毒治病,十去其六;常毒治病,十去其七;小毒治病,十去其八;无毒治病,十去其九。谷肉果菜,食养尽之,无使过之,伤其正也。”

9.病为本,工为标:《素问·汤液醪醴论》指出:“病为本,工为标。”这是说病是客观存在的,是本;医生认识治疗疾病,是标。医生必须以病人为根据,这样才能标本相得,治愈疾病。

10.辨证施治:《黄帝内经》虽未提出“辨证施治”一词,却有辨证施治之实。上述几点均含此意,而书中已有脏腑辨证、经络辨证、八纲辨证、六经辨证的内涵。

11.制方遣药:《黄帝内经》虽载方药无多,但其方药之理已具。《素问·至真要大论》说:“辛甘发散为阳,酸苦涌泄为阴,咸味涌泄为阴,淡味渗泄为阳。六者或收或散,或缓或急,或燥或润,或软或坚,以所利而行之,调其气,使其平也。”又有“主病之谓君,佐君之谓臣,应臣之谓使”,“君一臣二,制之小也;君一臣三佐五,制之中也;君一臣三佐九,制之大也”,“君一臣二,奇之制也;君二臣四,偶之制也;君二臣三,奇之制也;君二臣六,偶之制也。故曰:近者奇之,远者偶之,汗者不以奇,下者不以偶,补上治上制以缓,补下治下制以急,急则气味厚,缓则气味薄,适其至所,此之谓也”。如此等等,实难尽述。

12.针刺灸祔:《黄帝内经》言经络、腧穴、针刺、灸祔者甚多,不遑列举。单就补泻手法则有呼吸补泻(见《素问·离合真邪论》)、方员补泻(见《素问·八正神明论》及《灵枢·官能》)、深浅补泻(见《灵枢·终始》)、徐疾补泻(见《素问·针解篇》)和轻重补泻(见《灵枢·九针十二原》)等,这些手法一直被后世所沿用。

\mainmatter

\end{document}