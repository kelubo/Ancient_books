% 增广贤文
% 增广贤文.tex

\documentclass[12pt,UTF8]{ctexbook}

% 设置纸张信息。
\usepackage[a4paper,twoside]{geometry}
\geometry{
	left=25mm,
	right=25mm,
	bottom=25.4mm,
	bindingoffset=10mm
}

% 设置字体,并解决显示难检字问题。
\xeCJKsetup{AutoFallBack=true}
\setCJKmainfont{SimSun}[BoldFont=SimHei, ItalicFont=KaiTi, FallBack=SimSun-ExtB]

% 目录 chapter 级别加点(.)。
\usepackage{titletoc}
\titlecontents{chapter}[0pt]{\vspace{3mm}\bf\addvspace{2pt}\filright}{\contentspush{\thecontentslabel\hspace{0.8em}}}{}{\titlerule*[8pt]{.}\contentspage}

% 设置 part 和 chapter 标题格式。
\ctexset{
	part/name= {第,卷},
	part/number={\chinese{part}},
	chapter/name={第,篇},
	chapter/number={\chinese{chapter}}
}

% 图片相关设置。
\usepackage{graphicx}
\graphicspath{{Images/}}

% 设置古文原文格式。
\newenvironment{yuanwen}{\bfseries\zihao{4}}

% 设置署名格式。
\newenvironment{shuming}{\hfill\bfseries\zihao{4}}

% 注脚每页重新编号,避免编号过大。
\usepackage[perpage]{footmisc}

\title{\heiti\zihao{0} 增广贤文}
\author{佚名}
\date{明}

\begin{document}

\maketitle
\tableofcontents

\frontmatter

\chapter{前言}

《增广贤文》原名《昔时贤文》,亦称《古今贤文》,是中国明代时期编写的儿童启蒙书目。该书不知辑自何人,始于何时。书名最早见之于明朝万历年间汤显祖的戏曲《牡丹亭》。其中第七出《闺塾》有云:“【绕地游】(旦引贴捧书上)素装才罢,款步书堂下,对净几明窗潇洒。(贴)《昔氏贤文》,把人禁杀,恁时节则好教鹦哥唤茶。”据考证,此处《昔氏贤文》即《昔时贤文》。由此可推知,此书最迟成书于万历年间。作者一直未见任何书载,相传由明朝中叶的一个儒生编纂。后来,经过明、清两代文人的不断增补修订,这部书才成为现在的面貌,称为《增广昔时贤文》,通称《增广贤文》,或简称《增广》。

自清朝后期以来,这部书就风靡全国,影响极大,几乎家喻户晓,妇孺皆知。旧时人们说:“读了《增广》会说话,读了《幼学》走天下。”一些人即使没读过《增广贤文》,由于经常听他人口口相传,也能够说出其中的一些名言警句。

《增广贤文》里的语句十分精辟,言语浅白,含义深刻,通俗易懂;形式上,大多两两相对,音韵和谐,朗朗上口,一经成诵,便经久难忘。《增广贤文》的内容相当广泛,涉及立身安命、为人处世、礼仪道德、风物典故、天文地理、自然规律等多方面,体现了儒、释、道等多种思想,蕴含着丰厚的人生智慧。具体说来,主要有以下主题:

一、惜时劝学。古人很早就意识到时光如水,不舍昼夜,并有很多形象化的表达:例如“人生一世,草木一春”,“枯木逢春犹再发,人无两度再少年”,“光阴似箭,日月如梭”等,都是我们耳熟能详的句子。他们也劝人勤学,认为读书比积累黄金更重要,如“少壮不努力,老大徒伤悲”,“读书须用意,一字值千金”,“积金千两,不如明解经书”等。这些金句也都脍炙人口。

二、品格修养。修身是人生第一要务,本书涉及很多方面:如仁义无价,“钱财如粪土,仁义值千金”等;如自强不息,“人老心未老,身贫志不穷”等;如宽容谦让,“将相顶头堪走马,公侯肚里好撑船”,“亏人是祸,饶人是福”等;如诚信待人,“许人一物,千金不移”,“人而无信,不知其可也”等;如自省精神,“平生只会量人短,何不回头把自量”等。这些潜藏着无穷哲理的句子,融合了儒家仁义礼智信、温良恭俭让等传统文化美德的优秀价值观。

三、人际关系。主要包括亲子关系,如“儿孙自有儿孙福,莫为儿孙作马牛”;兄弟关系,如“兄弟相害,不如友生”;夫妻关系,如“夫妻相好合,琴瑟与笙簧”;朋友关系,如“相逢好似初相识,到老终无怨恨心”,“知音说与知音听,不是知音莫与弹”。这些语句对于如何处理好各种人际关系也有很强的指导意义。

四、批判社会黑暗丑陋。如写到司法不公、诉讼之难:“一字入公门,九牛拖不出”,“衙门八字开,有理无钱休进来”;揭露人性丑陋:“莫信直中直,须防仁不仁”,“有钱道真语,无钱语不真”;感叹世态炎凉:“有茶有肉多兄弟,急难何曾见一人”,“人情似纸张张薄,世事如棋局局新”。

五、古人的天道观。例如“顺天者存,逆天者亡”,“人间私语,天闻若雷。暗室亏心,神目如电”。古人认为,凡事遵循天道,才会得到好的结果。这固然受限于科学和时代的因素,却很有积极意义,比如引导人们向善向美、遵循客观规律等,我们今天仍然会用“天理难容”这样的成语。

此外,本书还有很多句子体现了古人居安思危的忧患意识、知足知止的生存智慧、物极必反的辩证思维方法,这些对于今天的我们为人处世同样有疗效。

然而,本书毕竟是封建社会、农业时代的产物,不可避免地带有鲜明的时代印痕和思想局限。如宿命论,“死生有命,富贵在天”,“万事皆先定,浮生空自忙”。如因果报应,“善有善报,恶有恶报”等。如男尊女卑的观念,“在家从父,出嫁从夫”,“有儿贫不久,无子富不长”,这种价值观明显已经过时。书中也有不少消极颓废的思想,如“月过十五光明少,人到中年万事休”,“今朝有酒今朝醉,明日愁来明日忧”;但求自保、不顾大局的观念,如“见事莫说,问事不知。闲事莫管,无事早归”。对这类内容,我们在今天应该批判地对待。

书中还有些内容十分相似,甚至部分重复。例如:“白发不随老人去,看来又是白头翁。”“记得少年骑竹马,看看又是白头翁。”“儿孙自有儿孙福,莫为儿孙作马牛。”“莫把真心空计较,儿孙自有儿孙福。”之所以出现这种情况,或许是这些话语在不同时代、不同地方流传,后加入书中而未加审读造成的。

《增广贤文》在结构上没有什么组织,每则之间没有多少联系,这使得本书显得有些散乱。这与编纂者没有精心整理有关,也或许是因为不同时代的人分散累积补缀造成的,当然,这也与它是格言警句的集锦有关。无论如何,这些经过大浪淘沙沉淀下来的经典语句,每一句都是智慧的结晶,是人类世代积累的经验的传递,是一种价值理念、道德观念、社会知识等的“再生产”。

《增广贤文》的内容来源十分广泛,大部分直接来自先秦典籍、诸子言论、笔记小说、诗词曲赋等,博采众长,名句荟萃。“有文言,有俗言,有直言,有婉言,有善恶言、勉戒言、在家出家言,复有仕宦治世言,隐逸出世言,士农工商,无一不备。”(何荣爵《重订增广序》)。通过对《增广贤文》每则格言源流的梳理考察,可以发现来源较多的除“四书五经”外,还有佛教典籍《五灯会元》、戏曲《琵琶记》等著作。在明末藏书家毛晋编辑的《六十种曲》里,也可以看到明代中期以前的很多戏曲杂剧已广泛征引《增广贤文》里的语句了,这说明彼时《增广贤文》已经相当流行。而在之后的“三言二拍”中,《增广贤文》里的语句就更为随处可见。通过对源流的探究,可以推断原文在流传过程中的一些用字讹误,如“黄金无假,阿魏无真”应为“黄芩无假,阿魏无真”。同时,源流的梳理也为理解原文提供了有益的帮助,因为有些语句如果脱离了上下文和原始语境,我们就无法准确理解其内在含义。

中华优秀传统文化具有非常强大的生命力与延续性。《增广贤文》作为古人宝贵人生经验的总结,具有价值引领、方法指导、动力激发等多重功能,可以给人以中华优秀传统文化的积极人生力量。因此,这部书被人们视为处世金箴、做人指南。几百年来,《增广贤文》流传至今并且仍然是许多人所喜爱的读物,其辩证看待问题的理性态度、得宠思辱的忧患意识、知足知止的乐观心态、讲仁义守诚信的价值理念,对于现代人的行为规范仍然具有灯塔般的指引作用。因此,这是一部人生之书,值得终生细品慢悟,学习借鉴。读《增广贤文》,有时像拜访了一位饱经沧桑的老者,因听其谆谆之言而醍醐灌顶;有时又像偶遇了一个鲜衣怒马的少年,因见其生机蓬勃而为之欢喜。

如前所述,本书对人性丑恶、官场腐败、世态炎凉等诸多问题的揭露与批判十分犀利。有人认为,作为一部蒙学读物,这些过多聚焦人情冷暖的句子对于儿童来说不够友好,过于世俗和早熟。然而,社会的阴暗面不会因我们的一时屏蔽而凭空消失,正如罗曼·罗兰所言:“世界上只有一种英雄主义,那就是在看清生活的真相后依然热爱生活。”《增广贤文》中的语句直面世事的复杂、人性的弱点,其对人性的深刻洞见,对世情的冷峻观察,正是这部书的深刻所在。或许,我们也不应低估孩子的辨别力,这是人本自具足的判断能力。

《增广贤文》的版本除通行本外,还有改本、重订本、新编本等多种。清朝咸丰年间,一位署名硕果山人(生平不详)的儒士,对它进行了一番修订补缀,并且按字数多少,以四言、五言、六言、七言、杂言的顺序分五个部分编排,末附对仗俚语57联,书名更易为《训蒙增广改本》,特别标明了童蒙教育的鹄的。清朝同治八年(1869),儒生周希陶(生卒、籍贯不详)对《昔时贤文》稍作删补,以平韵、上韵、去韵、入韵等四韵进行了新的编排,即成《重定增广》。

以国家图书馆所藏清末聚善堂本《改良增广贤文》为底本,同时参校了1912年上海昌文书局出版的《绘图明贤集》,以及1937年《全本增广贤文(绘图格言)》等版本。对于不同版本中的重要异文,在注释中加以说明。在条目划分时,不求形式的整齐,而求句意的完整,一个独立的句意即为一则条目,共三百多则条目。



《增广贤文》集结中国从古到今的各种格言、谚语。后来,经过明、清两代文人的不断增补,才改成现在这个模样,称《增广昔时贤文》,通称《增广贤文》。

《增广贤文》从表面上看似乎杂乱无章,但只要认真通读全书,不难发现有其内在的逻辑。该书对人性的认识以儒家荀子“性恶论”思想为前提,以冷峻的目光洞察社会人生。亲情被金钱污染,“贫居闹市无人问,富在深山有远亲”;友情只是一句谎言,“有酒有肉多兄弟,急难何曾见一人”;尊卑由金钱来决定,“不信但看筵中酒,杯杯先敬有钱人”;法律和正义为金钱所操纵,“衙门八字开,有理无钱莫进来”;人性被利益扭曲,“山中有直树,世上无直人”;世故导致人心叵测,“画虎画皮难画骨,知人知面不知心”;人言善恶难辨,“入山不怕伤人虎,只怕人情两面刀”。《增广贤文》把社会诸多方面的阴暗现象高度概括,冷冰冰地陈列在读者面前。

《增广贤文》绝大多数句子都来自经史子集、诗词曲赋、戏剧小说以及文人杂记,其思想观念都直接或间接地来自儒家佛家经典,从广义上来说,它是雅俗共赏的“经”的普及本。不需讲解就能读懂,通过读《增广贤文》同样能领会到经文的思想观念和人生智慧。

《增广贤文》的内容大致有这样几个方面:一是谈人及人际关系,二是谈命运,三是谈如何处世,四是表达对读书的看法。

在《增广贤文》描述的世界里,人是虚伪的,人们为了一己之私变化无常,嫌贫爱富,趋炎附势,从而使世界布满了陷阱和危机。文中有很多强调命运和报应的内容,认为人的一切都是命运安排的,人应行善,才会有好的际遇。《增广贤文》有大量篇幅叙述如何待人接物,这部分内容是全文的核心。文中对忍让多有描述,认为忍让是消除烦恼祸患的方法。在主张自我保护、谨慎忍让的同时,也强调人的主观能动性,认为这是做事的原则。文中也不乏劝人向善“害人之心不可有,防人之心不可无。”

《增广贤文》强调了读书的重要、孝义的可贵,这些观点体现了正统的儒家精神。但也正是由于这种庞杂,不同思想的人都可以从中看到自己认可的格言,使之具有了广泛的代表性。

《增广贤文》还是谚语的选集。

《增广贤文》以有韵的谚语和文献佳句选编而成,其内容十分广泛,从礼仪道德、典章制度到风物典故、天文地理,几乎无所不含,而又语句通顺,易懂。但中心是讲人生哲学、处世之道。其中一些谚语、俗语反映了中华民族千百年来形成的勤劳朴实、吃苦耐劳的优良传统,成为宝贵的精神财富,如“一年之计在于春,一日之计在于晨”;许多关于社会、人生方面的内容,经过人世沧桑的千锤百炼,成为警世喻人的格言,如“良药苦口利于病,忠言逆耳利于行”、“善有善报,恶有恶报”、“乐不可极,乐极生悲”等;一些谚语、俗语总结了千百年来人们同自然斗争的经验,成为简明生动哲理式的科学知识,如“近水知鱼性,近山识鸟音”、“近水楼台先得月,向阳花木早逢春”等。

一定的文化是一定的社会政治经济在观念形态上的反映,《增广贤文》也不例外。由于时代和历史的局限,必然打上那个时代的印记。不少内容反映了封建伦理和道德观念,甚至带有明显的封建迷信、宿命论的色彩;有的内容以个人为中心,反映了当时及时下人们普遍存在的自私自利、损人利己的思想;有的反映了在当时乃至当今社会制度下小市民阶层得过且过、畏缩苟安的心理和避祸厌世的消极人生哲学;有的在当时社会不失为对社会现象的正确反映,今天已失去了借鉴的意义;还有一些内容含义比较模糊,或者只有片面的真理性,如果不做正确的理解,就会变成错误的东西。这些都是不符合时代精神的,要在阅读时采取批判的态度,明察扬弃,批判继承,吸取其有营养的成分,古为今用。

人称“读了增广会说话,读了幼学走天下”,“增广”就是《增广贤文》,“幼学”则是《幼学琼林》。

\mainmatter

\chapter{上集}

\begin{yuanwen}
昔时\footnote{过去,从前。}贤文\footnote{贤达之人所作文章,或曰精悍优美的文字。},诲汝谆谆\footnote{《诗经·大雅·抑》:“诲尔谆谆,听我藐藐。”诲,huì,教导,劝说。汝,你。谆谆,zhūn,恳切、耐心的样子。}。集韵\footnote{按照韵文的形式采集编排。韵,韵语,韵文,如诗、词、曲、赋、对联等。}增广\footnote{增加见闻,广开视野。},多见多闻\footnote{多见多闻:多看别人行事,多听别人说话。此指读此书能带来增加见闻之效果。《论语·为政》:“子张学干禄。子曰:‘多闻阙疑,慎言其余,则寡尤;多见阙殆,慎行其余,则寡悔。言寡尤,行寡悔,禄在其中矣。’”}。观今宜鉴古\footnote{指以古为镜。《新唐书·魏徵传》载唐太宗语:“以铜为鉴,可正衣冠;以古为鉴,可知兴替;以人为鉴,可知得失。朕尝保此三鉴,内防己过。今魏徵逝,一鉴亡矣。”鉴,镜子。},无古不成今\footnote{没有过去就没有今天。据《庄子·知北游》,冉求问于仲尼曰:“未有天地可知邪?”仲尼曰:“可。古犹今也。”冉求失问而退。明日复见,曰:“昔者吾问‘未有天地可知乎?’夫子曰:‘可。古犹今也。’昔日吾昭然,今日吾昧然,敢问何谓也?”仲尼曰:“昔之昭然也,神者先受之;今之昧然也,且又为不神者求邪!无古无今,无始无终。未有子孙而有子孙可乎?”冉求未对。}。
\end{yuanwen}

从前贤达之人的文字,对你有恳切的教益。收集文质兼美的格言警句,可以帮助你扩充耳目、广博见闻、增加智慧。观察体悟今日之世事,应以古代的历史为借鉴,因为今天是由过去延续而来的,没有过去就没有现在。

\begin{yuanwen}
知己知彼\footnote{认识自己、了解他人。《孙子兵法·谋攻篇》:“知彼知己者,百战不殆;不知彼而知己,一胜一负;不知彼不知己,每战必殆。”},将心比心\footnote{以自己的感受衡量他人的感受。《朱子语类·传十章释治国平天下》:“问:‘前后左右何指?’曰:‘譬如交代官相似。前官之待我者既不善,吾毋以前官所以待我者待后官也。左右,如东邻西邻。以邻国为壑,是所恶于左而以交于右也。俗语所谓“将心比心”,如此,则各得其平矣。’”将,拿、用。比,比较,衡量。}。
\end{yuanwen}

人要认识自己,同时也要了解他人;要以自己的感受衡量别人的感受。

\begin{yuanwen}
酒逢知己饮,诗向会人吟。\footnote{《五灯会元·泐谭文准禅师》:“蓦拈拄杖,起身云:‘大众宝峰何似孔夫子?’良久曰:‘酒逢知己饮,诗向会人吟。’卓拄杖,下座。”知己,能够理解自己的人。会人,能够领悟的人。会,懂得,理解。吟,吟咏,吟诵。}
\end{yuanwen}

酒要与能够理解自己的人一起喝,诗要向真正懂诗的人吟咏。

\begin{yuanwen}
相识满天下,知心能几人?\footnote{《 五灯会元·云盖继鹏禅师》:“问:‘佛未出世时如何?’师曰:‘天。’曰:‘出世后如何?’师曰:‘地。’上堂:‘高不在绝顶,富不在福严。乐不在天堂,苦不在地狱。’良久曰:‘相识满天下,知心能几人?’”}
\end{yuanwen}

相识之人千千万万,满天下都是,但真正能够相知的又有几个人呢?

\begin{yuanwen}
相逢好似初相识\footnote{第一次见面互相认识。},到老终无怨恨心。
\end{yuanwen}

如果人与人每次相逢都能做到像第一次见面时那样互尊互敬,那么即使相交到终老也不会产生怨恨之心。

\begin{yuanwen}
近水知鱼性\footnote{鱼的生活习性。《诗经·小雅·鹤鸣》:“鱼潜在渊,或在于渚。……鱼在于渚,或潜在渊。”笺云:“此言鱼之性。寒则逃于渊,温则见于渚。”},近山识\footnote{辨识。}鸟音\footnote{鸟的鸣叫声。金代高公振《裴氏西园》:“竹阴疏处见潭影,人语定时闻鸟音。”}。
\end{yuanwen}

居于水边的人,可知晓各种鱼的生活习性;住在大山附近的人,能辨别各种鸟的鸣叫声。

\begin{yuanwen}
易涨易退山溪水,易反易覆小人\footnote{指识见浅狭或心口不一的人。}心。
\end{yuanwen}

容易涨也容易退的是山溪里的流水,容易变化无常的是小人的心态。

\begin{yuanwen}
运去金成铁,时来铁似金。\footnote{二句:宋邵康节《养心歌》:“得岁月,忘岁月;得欢悦,忘欢悦。万事乘除总在天,何必愁肠千万结?放心宽,莫胆窄,古今兴废言可彻。金谷繁华眼里尘,淮阴事业锋头血。陶潜篱畔菊花黄,范蠡湖边芦月白。临会上胆气雄,丹县里箫声绝。时来顽铁有光辉,运退黄金无艳色。逍遥且学圣贤心,到此方知滋味别。粗衣淡饭足家常,养得浮生一世拙。”运,运势。时来,时运来了。}
\end{yuanwen}

运势逝去时,金子也像废铁一样不值钱;时运来临时,废铁也像金子一样珍贵。

\begin{yuanwen}
读书须用意\footnote{用心。},一字值千金\footnote{《史记·吕不韦列传》:“是时诸侯多辩士,如荀卿之徒,著书布天下。吕不韦乃使其客人人著所闻,集论以
为八览、六论、十二纪,二十余万言。以为备天地万物古今之事,号曰《吕氏春秋》。布咸阳市门,悬千金其上,延诸侯游士宾客有能增损一字者予千金。”后来常用“一字千金”来形容文章或书籍价值极高。}。
\end{yuanwen}

读书一定要用心,一个字就会价值千金。

\begin{yuanwen}
逢人且说三分话,未可全抛一片心。\footnote{《全宋文》方大琮《与岩仲书》:“昔人出一言可见肝胆,近世则有逢人且说三分话之说。若司马氏教人自不妄语始,则其法严矣。不妄云者,直在其中,而疏率自无矣。”《五灯会元·育王怀琏禅师》:“上堂:‘太阳东升,烁破大千之暗。诸人若向明中立,犹是影响相驰。若向暗中立,也是藏头露影汉。到这里作么生吐露?’良久曰:‘逢人只可三分语,未可全抛一片心。参!’”逢,遇到。且,暂且。}
\end{yuanwen}

与人说话时只说三分,不可把自己内心的想法全部说出来。

\begin{yuanwen}
有意栽花花不发,无心插柳柳成(阴/荫1)。\footnote{元关汉卿《包待制智斩鲁斋郎》第二折:“(鲁斋郎引张千上)着意栽花花不发,等闲插柳柳成阴。小官鲁斋郎是也。”}
\end{yuanwen}

用心栽种的花没有开放,无意插下的柳条却长得枝叶繁茂形成了树荫。

\begin{yuanwen}
画虎画皮难画骨\footnote{指画虎外表易画,骨相难描。比喻人心难测。},知人知面不知心。
\end{yuanwen}

画老虎,画出它的皮毛容易,难的是画出它的骨相;认识人,了解他的外貌容易,难的是了解他的内心。

\begin{yuanwen}
钱财如粪土\footnote{指轻视钱财。《晋书·殷浩传》:“或问浩曰:‘将莅官而梦棺,将得财而梦粪,何也?’浩曰:‘官本臭腐,故将得官而梦尸。钱本粪土,故将得钱而梦秽。’时人以为名言。”},仁义\footnote{为传统道德的最高原则,与“礼、智、信”合称为五常。《礼记·曲礼上》:“道德仁义,非礼不成。”《韩非子·五蠹》:“故文王行仁义而王天下。”}值千金。
\end{yuanwen}

金钱和财物就像粪土一样,并没有多少价值;而仁德和道义像千锭金银一样,贵重而难得。

\begin{yuanwen}
流水下滩\footnote{江河中水浅多石而水流很急的地方。}非有意,白云出岫本无心\footnote{晋陶渊明《归去来辞》:“云无心以出岫,鸟倦飞而知还。”岫,xiù,峰峦、山谷。}。
\end{yuanwen}

流水向下面的河滩流动并非有意为之,白云自然地飘出山峰也本是无心之举。

\begin{yuanwen}
当时若不登高望,谁(信1/知)东流海(樣/洋1)深?
\end{yuanwen}

当时若不是登到高处去眺望,谁会相信东流的水所抵达的海洋是那样深广呢。

\begin{yuanwen}
路遥知马力,(事/日)久见人心。\footnote{《古尊宿语录》:“上堂,举兴化问克宾维那:‘汝不久为唱道之师?’克宾云:‘我不入这保社。’化云:‘你会了不入,不会了不入?’克宾云:‘我总不恁么。’化便打。遂罚钱五贯,设饡饭了,趁出院。后来却法嗣兴化。师云:‘还会么?路遥知马力,岁久见人心。’以拂子击禅床,下座。”}
\end{yuanwen}

路途遥远,才能够看出马的耐力;做事久了,才能够看出人的心性品质。

\begin{yuanwen}
两人一般心,(有/无)钱堪买金;一人一般心,(有/无)钱(难/堪)买针。\footnote{明代南戏《杀狗记》(全名《杨德贤妇杀狗劝夫》)第十九出:“俗谚说:‘家有一心,有钱买金;家有二心,无钱买针。’”一般,一样。堪,能够。}
\end{yuanwen}

两个人同心同德,才能够拥有买到黄金的钱财;一个人一个想法,意见不一,就会沦落到无钱买针的境地。

\begin{yuanwen}
相见易得好,久住难为人。\footnote{《全宋诗》释道颜《颂古二十首》:“将谓众生苦,更有苦众生。相见易得好,共住难为人。”久住,指在他人家里住得时间长。}
\end{yuanwen}

初次相见的时候,容易相处融洽;若是长久住在一起,则难免有诸多不方便,产生各种矛盾。

\begin{yuanwen}
马行无力皆因瘦,人不风流\footnote{风度仪态。}只为贫\footnote{缺乏钱资。}。
\end{yuanwen}

马跑起来没有力气,都是因为长得太瘦了;人不风流潇洒,只是因为太贫穷了。

\begin{yuanwen}
饶人不是痴汉,痴汉不会饶人。\footnote{元吴亮《忍经》:“谚曰:‘忍事敌灾星。’谚曰:‘凡事得忍且忍,饶人不是痴汉,痴汉不会饶人。’谚曰:‘得忍且忍,得诫且诫,小事成大。’”饶,饶恕、宽容。痴汉,此指没思想、没头脑的人。}
\end{yuanwen}

会饶恕别人的人不是痴汉,痴汉是不会饶恕别人的。

\begin{yuanwen}
是亲不是亲\footnote{《全元曲·包龙图智赚合同义字》:“(包待制云)这老儿好葫芦提。怎生婆婆说不是就不是?兀那李社长,端的他是亲不是亲?”},非亲却是亲\footnote{南宋戏文《张协状元》:“(末白)亚婆,且放心,它自记得买将归。(净)我命非亲却是亲。(末)你门得镜我无因。(净)自家骨肉尚如此。(合)何况区区陌路人。”}。
\end{yuanwen}

是亲人却不像亲人一样对待,不是亲人却像亲人一样亲近。

\begin{yuanwen}
美不美,乡中水;亲不亲,故乡人\footnote{元佚名《冻苏秦》第三折:“凭着我胸中豪气三千丈,笔下文才七步章。亲不亲,是乡党,若今番到举场,将万言书见帝王。”}。
\end{yuanwen}

无论是不是甜美,最爱的还是家乡的水。不管是不是亲近,眷恋的还是故乡的人。

\begin{yuanwen}
莺花犹怕春光老,岂可教人枉度春?
\end{yuanwen}

\begin{yuanwen}
相逢不饮空归去,洞口桃花也笑人。
\end{yuanwen}

\begin{yuanwen}
红粉佳人休使老,风流浪子莫教贫。
\end{yuanwen}

\begin{yuanwen}
在家不会迎宾客,出门方知少(主/故)人。
\end{yuanwen}

\begin{yuanwen}
黄芩(金)无假,阿魏无真。
\end{yuanwen}

\begin{yuanwen}
客来主不顾,自是无良宾。良宾(主/方)不顾,应恐是痴人。
\end{yuanwen}

\begin{yuanwen}
贫居闹市无人问,富在深山有远亲。
\end{yuanwen}

\begin{yuanwen}
谁人背后无人说,哪个人(认)前不说人?
\end{yuanwen}

\begin{yuanwen}
有钱道真语,无钱语不真。
\end{yuanwen}

\begin{yuanwen}
不信但看筵中酒,杯杯先敬有钱人。
\end{yuanwen}

\begin{yuanwen}
闹(市/里)(挣/有)钱,静处安身。
\end{yuanwen}

\begin{yuanwen}
来如风雨,去似微尘。
\end{yuanwen}

\begin{yuanwen}
长江后浪推前浪,世上新人(换/赶)旧人。
\end{yuanwen}

\begin{yuanwen}
近水楼台先得月,向阳花木(易/早)逢春。
\end{yuanwen}

\begin{yuanwen}
古人不见今时月,今月曾经照古人。
\end{yuanwen}

\begin{yuanwen}
先到为君,后到为臣。
\end{yuanwen}

\begin{yuanwen}
莫道君行早,更有早行人。
\end{yuanwen}

\begin{yuanwen}
莫信直中直,须防仁不仁。
\end{yuanwen}

\begin{yuanwen}
山中有直树,世上无直人。
\end{yuanwen}

\begin{yuanwen}
自恨枝无叶,莫怨太阳偏。
\end{yuanwen}

\begin{yuanwen}
(万般皆由/大家都是/一切都是)命,半点不由人。
\end{yuanwen}

\begin{yuanwen}
一年之计在于春,一日之计在于(晨/寅)。一家之计在于和,一生之计在于勤。
\end{yuanwen}

\begin{yuanwen}
责人之心责己,恕己之心恕人。
\end{yuanwen}

\begin{yuanwen}
守口如瓶,防意如城。
\end{yuanwen}

\begin{yuanwen}
宁可人负我,切莫我负人。
\end{yuanwen}

\begin{yuanwen}
再三须慎意,第一莫欺心。
\end{yuanwen}

\begin{yuanwen}
虎(生/身)犹可近,人(熟/毒)不(堪/可)亲。
\end{yuanwen}

\begin{yuanwen}
来说是非者,便是是非人。
\end{yuanwen}

\begin{yuanwen}
远水难救近火,远亲不如近邻。
\end{yuanwen}

\begin{yuanwen}
有(酒/茶)有(肉/酒)多兄弟,急难何曾见一人?
\end{yuanwen}

\begin{yuanwen}
人情似纸张张薄,世事如棋局局新。
\end{yuanwen}

\begin{yuanwen}
山中也有千年树,世上难逢百岁人。
\end{yuanwen}

\begin{yuanwen}
力微休负重,言轻莫劝人。
\end{yuanwen}

\begin{yuanwen}
无钱休入众,遭难莫寻亲。
\end{yuanwen}

\begin{yuanwen}
平生(不/莫)做皱眉事,世上应无切齿人。
\end{yuanwen}

\begin{yuanwen}
士(乃/者)国之宝,儒为席上珍。
\end{yuanwen}

\begin{yuanwen}
若要断酒法,醒眼看醉人。
\end{yuanwen}

\begin{yuanwen}
求人须求(英雄汉/大丈夫),济人须济急时无。
\end{yuanwen}

\begin{yuanwen}
渴时一滴如甘露,醉后添杯不如无。
\end{yuanwen}

\begin{yuanwen}
久住令人贱,频来亲也疏。
\end{yuanwen}

\begin{yuanwen}
酒中不语真君子,财上分明大丈夫。
\end{yuanwen}

\begin{yuanwen}
(贫贱之交不可忘,糟糠之妻不下堂。)
\end{yuanwen}

\begin{yuanwen}
出家如初,成佛有余。
\end{yuanwen}

\begin{yuanwen}
积金千两,不如明解经书。
\end{yuanwen}

\begin{yuanwen}
养子不教如养驴,养女不教如养猪。
\end{yuanwen}

\begin{yuanwen}
\end{yuanwen}\begin{yuanwen}
\end{yuanwen}\begin{yuanwen}
\end{yuanwen}\begin{yuanwen}
\end{yuanwen}\begin{yuanwen}
\end{yuanwen}\begin{yuanwen}
\end{yuanwen}\begin{yuanwen}
\end{yuanwen}\begin{yuanwen}
\end{yuanwen}\begin{yuanwen}
\end{yuanwen}\begin{yuanwen}
\end{yuanwen}\begin{yuanwen}
\end{yuanwen}\begin{yuanwen}
\end{yuanwen}\begin{yuanwen}
\end{yuanwen}\begin{yuanwen}
\end{yuanwen}\begin{yuanwen}
\end{yuanwen}\begin{yuanwen}
\end{yuanwen}\begin{yuanwen}
\end{yuanwen}\begin{yuanwen}
\end{yuanwen}\begin{yuanwen}
\end{yuanwen}\begin{yuanwen}
\end{yuanwen}\begin{yuanwen}
\end{yuanwen}\begin{yuanwen}
\end{yuanwen}\begin{yuanwen}
\end{yuanwen}\begin{yuanwen}
\end{yuanwen}\begin{yuanwen}
\end{yuanwen}\begin{yuanwen}
\end{yuanwen}\begin{yuanwen}
\end{yuanwen}\begin{yuanwen}
\end{yuanwen}\begin{yuanwen}
\end{yuanwen}\begin{yuanwen}
\end{yuanwen}\begin{yuanwen}
\end{yuanwen}\begin{yuanwen}
\end{yuanwen}\begin{yuanwen}
\end{yuanwen}\begin{yuanwen}
\end{yuanwen}\begin{yuanwen}
\end{yuanwen}\begin{yuanwen}
\end{yuanwen}\begin{yuanwen}
\end{yuanwen}\begin{yuanwen}
\end{yuanwen}\begin{yuanwen}
\end{yuanwen}\begin{yuanwen}
\end{yuanwen}\begin{yuanwen}
\end{yuanwen}\begin{yuanwen}
\end{yuanwen}\begin{yuanwen}
\end{yuanwen}\begin{yuanwen}
\end{yuanwen}\begin{yuanwen}
\end{yuanwen}\begin{yuanwen}
\end{yuanwen}\begin{yuanwen}
\end{yuanwen}\begin{yuanwen}
\end{yuanwen}\begin{yuanwen}
\end{yuanwen}\begin{yuanwen}
\end{yuanwen}\begin{yuanwen}
\end{yuanwen}\begin{yuanwen}
\end{yuanwen}\begin{yuanwen}
\end{yuanwen}



有田不耕仓廪虚,有书不读子孙愚。

仓廪虚兮岁月乏,子孙愚兮礼(仪/义)疏。

(听/同)君一席话,胜读十年书。

人不通古今,马牛(如/而)襟裾。

茫茫四海人无数,哪个男儿是丈夫?

(白/美)酒酿成缘好客,黄金散尽为(诗/收)书。

救人一命,胜造七级浮屠。

城门失火,殃及池鱼。

庭前生瑞草,好事不如无。

欲求生富贵,须下死工夫。

百年成之不足,一旦(坏/败)之有余。

人心似铁,官法如炉。

善化不足,恶化有余。

水至清则无鱼,人(至/太)察则无谋。

知者减半,(省/愚)者全无。

在家由父,出嫁从夫。
痴人畏妇,贤女敬夫。

是非终日有,不听自然无。

竹篱茅舍风光好,道院僧房终不如。

宁可正而不足,不可邪而有余。

宁可信其有,不可信其无。

命里有时终须有,命里无时莫强求。
道院迎仙客,书堂隐相儒。
庭栽栖凤竹,池养化龙鱼。
结交须胜己,似我不如无。
但看三五日,相见不如初。
人情似水分高下,世事如云任卷舒。
会说说都是,不会说无理。
磨刀恨不利,刀利伤人指。
求财恨不多,财多害自己。
知足常足,终身不辱;
知止常止,终身不耻。
有福伤财,无福伤己。
失之毫厘,谬以千里。
若登高必自卑,若涉远必自迩。
三思而行,再思可矣。
使口不如亲为,求人不如求己。
小时是兄弟,长大各乡里。
嫉财莫嫉食,怨生莫怨死。
人见白头嗔,我见白头喜。
多少少年郎,不到白头死。
墙有缝,壁有耳。
好事不出门,坏事传千里。
若要人不知,除非己莫为。

为人(不/莫)做亏心事,半夜敲门心不惊。

贼是小人,智过君子。
君子固穷,小人穷斯滥矣。
贫穷自在,富贵多忧。
不以我为德,反以我为仇。
宁向直中取,不可曲中求。
人无远虑,必有近忧。
知我者谓我心忧,不知我者谓我何求? [3]
晴天不肯去,直待雨淋头。
成事莫说,覆水难收。
是非只为多开口,烦恼皆因强出头。
忍得一时之气,免得百日之忧。
近来学得乌龟法,得缩头时且缩头。
惧法朝朝乐,欺心日日忧。
人生一世,草长一秋。
月过十五光明少,人到中年万事休。
儿孙自有儿孙福,莫为儿孙做马牛。
为人莫做千年计,三十河东四十西。
人生不满百,常怀千岁忧。
今朝有酒今朝醉,明日愁来明日忧。
路逢险处须回避,事到临头不自由。
人贫不语,水平不流。
一家养女百家求,一马不行百马忧。
有花方酌酒,无月不登楼。
三杯通大道,一醉解千愁。
深山毕竟藏猛虎,大海终须纳细流。
惜花须检点,爱月不梳头。
大抵选她肌骨好,不搽红粉也风流。
受恩深处宜先退,得意浓时便可休。
莫待是非来入耳,从前恩爱反为仇。
留得五湖明月在,不愁无处下金钩。
休别有鱼处,莫恋浅滩头。
去时终须去,再三留不住。
忍一句,息一怒,饶一着,退一步。
三十不豪,四十不富,五十将衰寻子助。
生不认魂,死不认尸。
一寸光阴一寸金,寸金难买寸光阴。
黑发不知勤学早,转眼便是白头翁。
父母恩深终有别,夫妻义重也分离。
人生似鸟同林宿,大难来时各自飞。
人善被人欺,马善被人骑。
人无横财不富,马无夜草不肥。
人恶人怕天不怕,人善人欺天不欺。
善恶到头终有报,只盼来早与来迟。
黄河尚有澄清日,岂能人无得运时?
得宠思辱,居安思危。
念念有如临敌日,心心常似过桥时。
英雄行险道,富贵似花枝。
人情莫道春光好,只怕秋来有冷时。
送君千里,终有一别。
但将冷眼观螃蟹,看你横行到几时。
见事莫说,问事不知。
闲事莫管,无事早归。
假缎染就真红色,也被旁人说是非。
善事可做,恶事莫为。
许人一物,千金不移。
龙生龙子,虎生虎儿。
龙游浅水遭虾戏,虎落平阳被犬欺。
一举首登龙虎榜,十年身到凤凰池。
十年寒窗无人问,一举成名天下知。
酒债寻常处处有,人生七十古来稀!
养儿防老,积谷防饥。
鸡豚狗彘之畜,无失其时,数口之家,可以无饥矣。
当家才知盐米贵,养子方知父母恩。
常将有日思无日,莫把无时当有时。
树欲静而风不止,子欲养而亲不待。
时来风送滕王阁,运去雷轰荐福碑。
入门休问荣枯事,且看容颜便得知。
官清司吏瘦,神灵庙祝肥。
息却雷霆之怒,罢却虎豹之威。
饶人算之本,输人算之机。
好言难得,恶语易施。
一言既出,驷马难追。
道吾好者是吾贼,道吾恶者是吾师。
路逢侠客须呈剑,不是才人莫献诗。
三人同行,必有我师。择其善者而从之,其不善者而改之。
欲昌和顺须为善,要振家声在读书。
少壮不努力,老大徒伤悲。
人有善愿,天必佑之。
莫饮卯时酒,昏昏醉到酉。
莫骂酉时妻,一夜受孤凄。
种麻得麻,种豆得豆。
天眼恢恢,疏而不漏。
做官莫向前,作客莫在后。
宁添一斗,莫添一口。
螳螂捕蝉,岂知黄雀在后?
不求金玉重重贵,但愿儿孙个个贤。
一日夫妻,百世姻缘。
百世修来同船渡,千世修来共枕眠。
杀人一万,自损三千。
伤人一语,利如刀割。
枯木逢春犹再发,人无两度再少年。
未晚先投宿,鸡鸣早看天。
将相顶头堪走马,公侯肚内好撑船。
富人思来年,穷人想眼前。
世上若要人情好,赊去物品莫取钱。
生死有命,富贵在天。
击石原有火,不击乃无烟。
人学始知道,不学亦徒然。
莫笑他人老,终须还到老。
和得邻里好,犹如拾片宝。
但能守本分,终身无烦恼。
大家做事寻常,小家做事慌张。
大家礼义教子弟,小家凶恶训儿郎。
君子爱财,取之有道。
贞妇爱色,纳之以礼。
善有善报,恶有恶报。
不是不报,时候未到。
万恶淫为首,百善孝当先。
人而无信,不知其可也。
一人道虚,千人传实。
凡事要好,须问三老。
若争小利,便失大道。
家中不和邻里欺,邻里不和说是非。
年年防饥,夜夜防盗。
学者是好,不学不好。
学者如禾如稻,不学如草如蒿。
遇饮酒时须防醉,得高歌处且高歌。
因风吹火,用力不多。
不因渔夫引,怎能见波涛?
无求到处人情好,不饮任他酒价高。
知事少时烦恼少,识人多处是非多。
世间好语书说尽,天下名山僧占多。
进山不怕伤人虎,只怕人情两面刀。
强中更有强中手,恶人须用恶人磨。
会使不在家富豪,风流不用衣着多。
光阴似箭,日月如梭。
天时不如地利,地利不如人和。
黄金未为贵,安乐值钱多。
为善最乐,作恶难逃。
羊有跪乳之恩,鸦有反哺之情。
孝顺还生孝顺子,忤逆还生忤逆儿。
不信但看檐前水,点点滴滴旧池窝。
隐恶扬善,执其两端。
妻贤夫祸少,子孝父心宽。
已覆之水,收之实难。
人生知足时常足,人老偷闲且是闲。
处处绿杨堪系马,家家有路通长安。
既坠釜甑,反顾无益。
见者易,学者难。莫将容易得,便作等闲看。
厌静还思喧,嫌喧又忆山。
自从心定后,无处不安然。
用心计较般般错,退后思量事事宽。
道路各别,养家一般。
由俭入奢易,从奢入俭难。
知音说与知音听,不是知音莫与谈。
点石化为金,人心犹未足。
信了赌,卖了屋。
他人观花,不涉你目。
他人碌碌,不涉你足。
谁人不爱子孙贤,谁人不爱千钟粟。
奈五行,不是这般题目。
莫把真心空计较,儿孙自有儿孙福。
书到用时方恨少,事非经过不知难。
天下无不是的父母,世上最难得者兄弟。
与人不和,劝人养鹅;与人不睦,劝人架屋。
但行好事,莫问前程。不交僧道,便是好人。
河狭水激,人急计生。

明知山有虎,莫向虎山行。

路不铲不平,事不为不成。
无钱方断酒,临老始读经。
点塔七层,不如暗处一灯。
堂上二老是活佛,何用灵山朝世尊。
万事劝人休瞒昧,举头三尺有神明。
但存方寸土,留与子孙耕。
灭却心头火,剔起佛前灯。
惺惺多不足,蒙蒙作公卿。
众星朗朗,不如孤月独明。
兄弟相害,不如友生。
合理可作,小利不争。
牡丹花好空入目,枣花虽小结实多。
欺老莫欺小,欺人心不明。
勤奋耕锄收地利,他时饱暖谢苍天。
得忍且忍,得耐且耐,不忍不耐,小事成灾。
相论逞英豪,家计渐渐退。
贤妇令夫贵,恶妇令夫败。
一人有庆,兆民咸赖。
人老心未老,人穷志莫穷。
人无千日好,花无百日红。
黄蜂一口针,橘子两边分。
世间痛恨事,最毒淫妇心。
杀人可恕,情理不容。
乍富不知新受用,乍贫难改旧家风。
座上客常满,杯中酒不空。
屋漏更遭连夜雨,行船又遇打头风。
笋因落箨方成竹,鱼为奔波始化龙。
记得少年骑竹马,转眼又是白头翁。
礼义生于富足,盗贼出于赌博。
天上众星皆拱北,世间无水不朝东。
士为知己者死,女为悦己者容。
色即是空,空即是色。
君子安贫,达人知命。
良药苦口利于病,忠言逆耳利于行。
顺天者昌,逆天者亡。
有缘千里来相会,无缘对面不相逢。
有福者昌,无福者亡。
人为财死,鸟为食亡。
夫妻相和好,琴瑟与笙簧。
红粉易妆娇态女,无钱难作好儿郎。
有子之人贫不久,无儿无女富不长。
善必寿老,恶必早亡。
爽口食多偏作病,快心事过恐遭殃。
富贵定要依本分,贫穷不必枉思量。
画水无风空作浪,绣花虽好不闻香。
贪他一斗米,失却半年粮。
争他一脚豚,反失一肘羊。
龙归晚洞云犹湿,麝过春山草木香。
平生只会说人短,何不回头把己量?
见善如不及,见恶如探汤。
人穷志短,马瘦毛长。
自家心里急,他人未知忙。
贫无达士将金赠,病有高人说药方。
触来莫与竞,事过心清凉。
秋来满山多秀色,春来无处不花香。
凡人不可貌相,海水不可斗量。
清清之水为土所防,济济之士为酒所伤。
蒿草之下或有兰香,茅茨之屋或有侯王。
无限朱门生饿殍,几多白屋出公卿。
酒里乾坤大,壶中日月长。
拂石坐来春衫冷,踏花归去马蹄香。
万事前身定,浮生空自忙。
叫月子规喉舌冷,宿花蝴蝶梦魂香。
一言不中,千言不用。
一人传虚,百人传实。
万金良药,不如无疾。
千里送鹅毛,礼轻情义重。
世事如明镜,前程暗似漆。
君子怀刑,小人怀惠。
架上碗儿轮流转,媳妇自有做婆时。
人生一世,如驹过隙。
良田万顷,日食一升。
大厦千间,夜眠八尺。
千经万典,孝义为先。
天上人间,方便第一。
一字入公门,九牛拔不出。
八字衙门向南开,有理无钱莫进来。
欲求天下事,须用世间财。
富从升合起,贫因不算来。
近河不得枉使水,近山不得枉烧柴。
家无读书子,官从何处来?
慈不掌兵,义不掌财。
一夫当关,万夫莫开。
万事不由人计较,一生都是命安排。
白云本是无心物,却被清风引出来。
慢行急行,逆取顺取。
命中只有如许财,丝毫不可有闪失。
人间私语,天闻若雷。
暗室亏心,神目如电。
一毫之恶,劝人莫作。一毫之善,与人方便。
亏人是祸,饶人是福,天眼恢恢,报应甚速。
圣贤言语,神钦鬼服。
人各有心,心各有见。
口说不如身逢,耳闻不如目见。
见人富贵生欢喜,莫把心头似火烧。
养兵千日,用在一时。
国清才子贵,家富小儿娇。
利刀割体疮犹使,恶语伤人恨不消。
公道世间唯白发,贵人头上不曾饶。
有才堪出众,无衣懒出门。
为官须作相,及第必争先。
苗从地发,树由枝分。
宅里燃火,烟气成云。
以直报怨,知恩报恩。
红颜今日虽欺我,白发他时不放君。
借问酒家何处有,牧童遥指杏花村。
父子和而家不退,兄弟和而家不分。
一片云间不相识,三千里外却逢君。
官有公法,民有私约。
平时不烧香,临时抱佛脚。
幸生太平无事日,恐防年老不多时。
国乱思良将,家贫思良妻。
池塘积水须防旱,田地深耕足养家。
根深不怕风摇动,树正何愁月影斜。
争得猫儿,失却牛脚。
愚者千虑,必有一得,智者千虑,必有一失。
始吾于人也,听其言而信其行。
今吾于人也,听其言而观其行。
哪个梳头无乱发,情人眼里出西施。
珠沉渊而川媚,玉韫石而山辉。
夕阳无限好,只恐不多时。
久旱逢甘霖,他乡遇故知;洞房花烛夜,金榜题名时。
惜花春起早,爱月夜眠迟。
掬水月在手,弄花香满衣。
桃红李白蔷薇紫,问着东君总不知。
教子教孙须教义,栽桑栽柘少栽花。
休念故乡生处好,受恩深处便为家。
学在一人之下,用在万人之上。
一日为师,终生为父。
忘恩负义,禽兽之徒。
劝君莫将油炒菜,留与儿孙夜读书。
书中自有千钟粟,书中自有颜如玉。
莫怨天来莫怨人,五行八字命生成。
莫怨自己穷,穷要穷得干净;莫羡他人富,富要富得清高。
别人骑马我骑驴,仔细思量我不如。
待我回头看,还有挑脚汉。
路上有饥人,家中有剩饭。
积德与儿孙,要广行方便。
作善鬼神钦,作恶遭天遣。
积钱积谷不如积德,买田买地不如买书。
一日春工十日粮,十日春工半年粮。
疏懒人没吃,勤俭粮满仓。
人亲财不亲,财利要分清。
十分伶俐使七分,常留三分与儿孙,
若要十分都使尽,远在儿孙近在身。
君子乐得做君子,小人枉自做小人。
好学者则庶民之子为公卿,不好学者则公卿之子为庶民。
惜钱莫教子,护短莫从师。
记得旧文章,便是新举子。
人在家中坐,祸从天上落。
但求心无愧,不怕有后灾。
只有和气去迎人,哪有相打得太平。
忠厚自有忠厚报,豪强一定受官刑。
人到公门正好修,留些阴德在后头。
为人何必争高下,一旦无命万事休。
山高不算高,人心比天高。
白水变酒卖,还嫌猪无糟。
贫寒休要怨,宝贵不须骄。
善恶随人作,祸福自己招。
奉劝君子,各宜守己。
只此呈示,万无一失。 [2]

\chapter{下集}

前人俗语,言浅理深。
补遗增广,集成书文。
世上无难事,只怕不专心。
成人不自在,自在不成人;
金凭火炼方知色,与人交财便知心。
乞丐无粮,懒惰而成。
勤俭为无价之宝,节粮乃众妙之门。
省事俭用,免得求人。
量大祸不在,机深祸亦深。
善为至宝深深用,心作良田世世耕。
群居防口,独坐防心。
体无病为富贵,身平安莫怨贫。
败家子弟挥金如土,贫家子弟积土成金。
富贵非关天地,祸福不是鬼神。
安分贫一时,本分终不贫。
不拜父母拜干亲,弟兄不和结外人。
人过留名,雁过留声。
择子莫择父,择亲莫择邻。
爱妻之心是主,爱子之心是亲。
事从根起,藕叶连心。
祸与福同门,利与害同城。
清酒红人脸,财帛动人心!
宁可荤口念佛,不可素口骂人。
有钱能说话,无钱话不灵。
岂能尽如人意?但求不愧吾心。
不说自己井绳短,反说他人箍井深。
恩爱多生病,无钱便觉贫。
只学斟酒意,莫学下棋心。
孝莫假意,转眼便为人父母。
善休望报,回头只看汝儿孙!
口开神气散,舌出是非生!
弹琴费指甲,说话费精神。
千贯买田,万贯结邻。
人言未必犹尽,听话只听三分。
隔壁岂无耳,窗外岂无人?
财可养生须注意,事不关己不劳心。
酒不护贤,色不护病;
财不护亲,气不护命!
一日不可无常业,安闲便易起邪心!
炎凉世态,富贵更甚于贫贱;
嫉妒人心,骨肉更甚于外人!
瓜熟蒂落,水到渠成。
人情送匹马,买卖不饶针!
过头饭好吃,过头话难听!
事多累了自己,田多养了众人。
怕事忍事不生事自然无事;
平心静心不欺心何等放心!
天子至尊不过于理,在理良心天下通行。
好话不在多说,有理不在高声!
一朝权在手,便把令来行。
甘草味甜人可食,巧言妄语不可听。
当场不论,过后枉然。
贫莫与富斗,富莫与官争!
官清难逃猾吏手,衙门少有念佛人!
家有千口,主事一人。
父子竭力山成玉,弟兄同心土变金。
当事者迷,旁观者清。
怪人不知理,知理不怪人。
未富先富终不富,未贫先贫终不贫。
少当少取,少输当赢!
饱暖思淫欲,饥寒起盗心!
蚊虫遭扇打,只因嘴伤人!
欲多伤神,财多累心!
布衣得暖真为福,千金平安即是春。
家贫出孝子,国乱显忠臣!
宁做太平犬,莫做离乱人!
人有几等,官有几品。
理不卫亲,法不为民。
自重者然后人重,人轻者便是自轻。
自身不谨,扰乱四邻。
快意事过非快意,自古败名因败事。
伤身事莫做,伤心话莫说。
小人肥口,君子肥身。
地不生无名之辈,天不生无路之人。
一苗露水一苗草,一朝天子一朝臣。
读未见书如逢良友,见已读书如逢故人。
福满须防有祸,凶多料必无争。
不怕三十而死,只怕死后无名。
但知江湖者,都是薄命人。
不怕方中打死人,只知方中无好人。
说长说短,宁说人长莫说短;
施恩施怨,宁施人恩莫施怨。
育林养虎,虎大伤人。
冤家抱头死,事要解交人。
卷帘归乳燕,开扇出苍蝇。
爱鼠常留饭,怜蛾灯罩纱。
人命在天,物命在人。
奸不通父母,贼不通地邻。
盗贼多出赌博,人命常出奸情。
治国信谗必杀忠臣,治家信谗必疏其亲。
治国不用佞臣,治家不用佞妇。
好臣一国之宝,好妇一家之珍。
稳的不滚,滚的不稳。
儿不嫌母丑,狗不嫌家贫。
君子千钱不计较,小人一钱恼人心。
人前显贵,闹里夺争。
要知江湖深,一个不做声。
知止自当出妄想,安贫须是禁奢心。
初入行业,三年事成;
初吃馒头,三年口生。
家无生活计,坐吃如山崩。
家有良田万顷,不如薄艺在身;
艺多不养家,食多嚼不赢。
命中只有八合米,走遍天下不满升。
使心用心,反害自身。
国家无空地,世上无闲人。
妙药难医怨逆病,混财不富穷命人。
耽误一年春,十年补不清;
人能处处能,草能处处生。
会打三班鼓,也要几个人。
人不走不亲,水不打不浑。
三贫三富不到老,十年兴败多少人!
买货买得真,折本折得轻;
不怕问到,只怕倒问。
人强不如货强,价高不如口便。
会买买怕人,会卖卖怕人。
只只船上有梢公,天子足下有贫亲。
既知莫望,不知莫向。
在一行,练一行;
穷莫失志,富莫癫狂。
天欲令其灭亡,必先让其疯狂。
梢长人胆大,梢短人心慌。
隔行莫贪利,久炼必成钢。
瓶花虽好艳,相看不耐长。
早起三光,迟起三慌。
未来休指望,过去莫思量;
时来遇好友,病去遇良方。
布得春风有夏雨,哈得秋风大家凉。
晴带雨伞,饱带饥粮。
满壶全不响,半壶响叮当。
久利之事莫为,众争之地莫往。
老医迷旧疾,朽药误良方;
该在水中死,不在岸上亡。
舍财不如少取,施药不如传方。
倒了城墙丑了县官,打了梅香丑了姑娘。
燕子不进愁门,耗子不钻空仓。
苍蝇不叮无缝蛋,谣言不找谨慎人。
一人舍死,万人难当。
人争一口气,佛争一炷香。
门为小人而设,锁乃君子之防。
舌咬只为揉,齿落皆因眶。
硬弩弦先断,钢刀刃自伤。
贼名难受,龟名难当。
好事他人未见讲,错处他偏说得长。
男子无志纯铁无钢,女子无志烂草无瓤。
生男欲得成龙犹恐成獐,生女欲得成凤犹恐成虎。
养男莫听狂言,养女莫叫离母。
男子失教必愚顽,女子失教定粗鲁。
生男莫教弓与弩,生女莫教歌与舞。
学成弓弩沙场灾,学成歌舞为人妾。
财交者密,财尽者疏。
婚姻论财,夫妻之道。
色娇者亲,色衰者疏。
少实胜虚,巧不如拙。
百战百胜不如无争,万言万中不如一默。
有钱不置怨逆产,冤家宜解不宜结。
近朱者赤,近墨者黑。
一个山头一只虎,恶龙难斗地头蛇。
出门看天色,进门看脸色。
商贾买卖如施舍,买卖公平如积德。
天生一人,地生一穴。
家无三年之积不成其家,国无九年之积不成其国。
男子有德便是才,女子无才便是德。
有钱难买子孙贤,女儿不请上门客。
男大当婚女大当嫁,不婚不嫁惹出笑话。
谦虚美德,过谦即诈。
自己跌倒自己爬,望人扶持都是假。
人不知己过,牛不知力大。
一家饱暖千家怨,一物不见赖千家。
当面论人惹恨最大,是与不是随他说吧!
谁人做得千年主,转眼流传八百家。
满载芝麻都漏了,还在水里捞油花!
皇帝坐北京,以理统天下。
五百年前共一家,不同祖宗也同华!
学堂大如官厅,人情大过王法。
找钱犹如针挑土,用钱犹如水推沙!
害人之心不可有,防人之心不可无!
不愁无路,就怕不做。
须向根头寻活计,莫从体面下功夫!
祸从口出,病从口入。
药补不如肉补,肉补不如养补。
思虑之害甚于酒色,日日劳力上床呼疾。
人怕不是福,人欺不是辱。
能言不是真君子,善处方为大丈夫!
为人莫犯法,犯法身无主。
姊妹同肝胆,弟兄同骨肉。
慈母多误子,悍妇必欺夫!
君子千里同舟,小人隔墙易宿。
文钱逼死英雄汉,财不归身恰是无。
妻子如衣服,弟兄似手足。
衣服补易新,手足断难续。
盗贼怨失主,不孝怨父母。
一时劝人以口,百世劝人以书。
我不如人我无其福,人不如我我常知足!
捡金不忘失金人,三两黄铜四两福。
因祸得福,求赌必输。
一言而让他人之祸,一忿而折平生之福。
天有不测风云,人有旦夕祸福。
不淫当斋,淡饱当肉。
缓步当车,无祸当福。
男无良友不知己之有过,女无明镜不知面之精粗。
事非亲做,不知难处。
十年易读举子,百年难淘江湖!
积钱不如积德,闲坐不如看书。
思量挑担苦,空手做是福。
时来易借银千两,运去难赊酒半壶。
天晴打过落雨铺,少时享过老来福。
与人方便自己方便,一家打墙两家好看。
当面留一线,过后好相见。
入门掠虎易,开口告人难。
手指要往内撇,家丑不可外传。
浪子出于祖无德,孝子出于前人贤。
货离乡贵,人离乡贱。
树挪死,人挪活。
在家千日好,出门处处难。
三员长者当官员,几个明人当知县?
明人自断,愚人官断。
人怕三见面,树怕一墨线。
村夫硬似铁,光棍软如棉。
不是撑船手,怎敢拿篙竿!
天下礼仪无穷,一人知识有限。
一人不得二人计,宋江难结万人缘。
家有三亩田,不离衙门前,乡间无强汉,衙门就饿饭。
人人依礼仪,天下不设官。
衙门钱,眼睛钱;
田禾钱,千万年。
诗书必读,不可做官。
为人莫当官,当官皆一般。
换了你我去,恐比他还贪。
官吏清廉如修行,书差方便如行善。
靠山吃山,种田吃田。
吃尽美味还是盐,穿尽绫罗还是棉。
一夫不耕,全家饿饭,一女不织,全家受寒。
金银到手非容易,用时方知来时难。
先讲断,后不乱,免得藕断丝不断。
听人劝,得一半。
不怕慢,只怕站。
逢快莫赶,逢贱莫懒。
谋事在人,成事在天!
长路人挑担,短路人赚钱。
宁卖现二,莫卖赊三。
赚钱往前算,折本往后算。
小小生意赚大钱,七十二行出状元。
自己无运至,却怨世界难。
胆大不如胆小,心宽甚如屋宽。
妻贤何愁家不富,子孙何须受祖田。
是儿不死,是财不散。
财来生我易,我去生财难。
十月滩头坐,一日下九滩。
结交一人难上难,得罪一人一时间。
借债经商,卖田还债;
赊钱起屋,卖屋还钱。
修起庙来鬼都老,拾得秤来姜卖完。
不嫖莫转,不赌莫看。
节食以去病,少食以延年。
豆腐多了是包水,梢公多了打烂船。
无口过是,无眼过难。
无身过易,无心过难。
不会凫水怨河湾,不会犁田怨枷担。
他马莫骑,他弓莫挽。
要知心腹事,但听口中言。
宁在人前全不会,莫在人前会不全。
事非亲见,切莫乱谈。
打人莫打脸,骂人莫骂短。
好言一句三冬暖,话不投机六月寒。
人上十口难盘,帐上万元难还。
放债如施,收债如讨。
告状讨钱,海底摸盐。
衙门深似海,弊病大如天。
银钱莫欺骗,牛马不好变。
好汉莫被人识破,看破不值半文钱。
狗咬对头人,雷打三世冤。
不卖香烧无剩钱,井水不打不满边。
事宽则园,太久则偏。
高人求低易,低人求高难。
有钱就是男子汉,无钱就是汉子难。
人上一百,手艺齐全。
难者不会,会者不难。
生就木头造就船,砍的没得车的圆。
心不得满,事不得全。
鸟飞不尽,话说不完。
人无喜色休开店,事不遂心莫怨天。
选婿莫选田园,选女莫选嫁奁。
红颜女子多薄命,福人出在丑人边。
人将礼义为先,树将花果为园。
临危许行善,过后心又变。
天意违可以人回,命早定可以心挽。
强盗口内出赦书,君子口中无戏言。
贵人语少,贫子话多。
快里须斟酌,耽误莫迟春。
读过古华佗,不如见症多。
东屋未补西屋破,前帐未还后又拖。
今年又说明年富,待到明年差不多。
志不同己,不必强合。
莫道坐中安乐少,须知世上苦情多。
本少利微强如坐,屋檐水也滴得多。
勤俭持家富,谦恭受益多。
细处不断粗处断,黄梅不落青梅落。
见钱起意便是贼,顺手牵羊乃为盗。
要做快活人,切莫寻烦恼。
要做长寿人,莫做短命事。
要做有后人,莫做无后事。
不经一事,不长一智。
宁可无钱使,不可无行止。
栽树要栽松柏--,结交要结君子。
秀才不出门,能知天下事。
钱多不经用,儿多不耐死。
弟兄争财家不穷不止,妻妾争风夫不死不止。
男人有志,妇人有势。
夫人死百将临门,将军死一卒不至。
天旱误甲子,人穷误口齿。
百岁无多日,光阴能几时?
父母养其身,自己立其志。
待有余而济人,终无济人之日;
待有闲而读书,终无读书之时。
此书传后世,句句必精读,其中礼和义,奉劝告世人。
勤奋读,苦发奋,走遍天涯如游刃。

【新增广贤文】
尊师以重道,爱众而亲仁。
钱财如粪土,仁义值千金。
作事须循天理,出言要顺人心。
处富贵地,要矜持贫贱的痛痒,当少壮时,须体念衰老的辛酸。
孝当竭力,非徒养身。
鸦有反哺之孝,羊知跪乳之恩。
打虎还要亲兄弟,出阵还须父子兵。
父子和而家不败,弟兄和而家不分。
知己知彼,将心比心。
责人之心责己,爱己之心爱人。
贪爱沉溺即苦海,利欲炽燃是火坑。
随时莫起趋时念,脱俗休存矫俗心。
昼夜惜阴,夜坐惜灯。读书须用意,一字值千金。
平生不作皱眉事,世上应无切齿人。
近水知鱼性,近山识鸟音。
路遥知马力,日久见人心。
饶人不是痴汉,痴汉不会饶人。
不说自己桶索短,但怨人家箍井深。
美不美,乡中水;亲不亲,故乡人。
割不断的亲,离不开的邻。
但行好事,莫问前程。
钝鸟先飞,大器晚成。
一年之计在于春,一日之计在于寅。
一家之计在于和,一生之计在于勤。
无病休嫌瘦,身安莫怨贫。
岂能尽如人意,但求无愧人心。
偏听则暗,兼听则明。
耳闻是虚,眼见是实。
毋施小惠而伤大体,毋借公论而快私情。
毋以己长而形人之短,毋因己拙而忌人之能。
平日不作亏心事,半夜敲门心不惊。
牡丹花好空入目,枣花虽小结实成。
汝惟不矜,天下莫与汝争能;汝惟不伐,天下莫与汝争功。
明不伤察,直不过矫。
仁能善断,清能有容。
不自是而露才,不轻试以幸功。
受享不逾分外,修持不减分中。
肝肠煦若春风,虽囊乏一文,还怜茕独;
气骨清如秋水,纵家徒四壁,终傲王公。
早把甘旨勤奉养,夕阳光阴不多时。
得宠思辱,居安思危。
成名每在穷苦日,败事多因得意时。
许人一物,千金不移。
一言既出,驷马难追。
博学而笃志,切问而近思。
惜钱休教子,护短莫从师。
须知孺子可教,勿谓童子何知。
静坐常思己过,闲谈莫论人非。
三人同行,必有我师,择其善者而从,其不善者改之。
狎昵恶少,久必受其累;屈志老成,急则可相依。
心口如一,童叟无欺。人有善念,天必佑之。
过则无惮改,独则毋自欺。道吾好者是吾贼,
道吾恶者是吉师。
学不尚行,马牛而襟裾。
结交须胜己,似我不如无。
同君一席话,胜读十年书。
水至清,则无鱼;人至察,则无徒。
宁可正而不足,不可斜而有余。
认真还自在,作假费功夫。
是非朝朝有,不听自然无。
聪明逞尽,惹祸招灾。
富从升合起,贫因不算来。
用人不宜刻,刻则思效者去;交友不宜滥,滥则贡谀者来。
乐不可极,乐极生哀;欲不可纵,纵欲成灾。
言顾行,行顾言。
不作风波于世上,但留清白在人间。
勿因群疑而阻独见,勿任己意而废人言。
自处超然,处人蔼然。得意淡然,失意泰然。
由俭入奢易,由奢入俭难。
枯木逢春犹再发,人无两度再少年。
儿孙胜于我,要钱做甚么;儿孙不如我,要钱做甚么。
谦恭待人,忠厚传家。
不学无术,读书便佳。
与治同道罔不兴,与乱同事罔不亡。
居身务期质朴,训子要有义方。
富若不教子,钱谷必消灭。
贵若不教子,衣冠受不长。
人无远虑,必有近忧。
勿临渴而掘井,宜未雨而绸缪。
酒虽痒性还乱性,水能载舟亦覆舟。
克己者,触事皆成药石;尤人者,启口即是戈矛。
儿孙自有儿孙福,莫与儿孙做牛马。
深山毕竟藏猛虎,大海终须纳细流。
休向君子诌媚,君子原无私惠;休与小人为仇,小人自我对头。
登高必自卑,若涉远必自迩。
磨刀恨不利,刀利伤人指;求财恨不多,财多终累己。
居视其所亲,达视其所举;富视其所不为,贫视其所不取。
知足常足,终身不辱;知止常止,终身不耻。
君子爱财,取之有道;小人放利,不顾天理。
悖入亦悖出,害人终害己。
身欲出樊笼外,心要在腔子里。
勿偏信而为奸所欺,勿自任而为气所使。
使口不如自走,求人不如求己。
处骨肉之变,宜从容不宜激烈;当家庭之衰,宜惕厉不宜委靡。
务下学而上达,毋舍近而趋远。
量入为出,凑少成多。
溪壑易填,人心难满。
用人与教人,二者却相反,用人取其长,教人责其短。
仕宦芳规清、慎、勤,饮食要诀缓、暖、软。
留心学到古人难,立脚怕随流俗转。
凡是自是,便少一是。
有短护短,更添一短。
好问则裕,自用则小。
勿营华屋,勿作营巧。
若争小可,便失大道。
但能依本分,终须无烦恼。
有言逆于汝心,必求诸道;有言逊于汝志,必求诸非道。
吃得亏,坐一堆;要得好,大做小。
志宜高而身宜下,胆欲大而心欲小。
学者如禾如稻,不学者如蒿如草。
唇亡齿必寒,教弛富难保。
书中结良友,千载奇逢;门内产贤郎,一家活宝。
狗不嫌家贫,儿不嫌母丑。
勿贪意外之财,勿饮过量之酒。
进步便思退步,着手先图放手。
责善勿过高,当思其可从。
攻恶勿太严,要使其可受。
和气致祥,乖气致戾。
玩人丧德,玩物丧志。
门内有君子,门外君子至;门内有小人,门外小人至。
趋炎虽暖,暖后更觉寒增;食蔗能甘,甘余更生苦趣。
家庭和睦,蔬食尽有余欢;骨肉乖违,珍馐亦减至味。
先学耐烦,切莫使气。
性躁心粗,一生不济。
得时莫夸能,不遇休妒世。
物盛则必衰,有隆还有替。
路径仄处,留一步与人行;滋味浓时,减三分让人嗜。
为人要学大莫学小,志气一卑污了,品格难乎其高;
持家要学小莫学大,门面一 弄阔了,后来难乎其继。
三十不立,四十见恶,五十相将寻死路。
见怪不怪,怪乃自败。
一正压百邪,少见必多怪。
君子之交淡以成,小人之交甘以坏。
爱人者,人恒爱。敬人者,人恒敬。
损友敬而远,益友亲而敬。
善与人交,久而能敬。
过则相规,言而有信。
木受绳则直,人受柬则圣。
良药苦口利于病,忠言逆耳利于行。
智生识,识生断。当断不断,反受其乱。
一毫之恶,劝人莫作;一毫之善,与人方便。
难合亦难分,易亲亦易散。
传家二字耕与读,防家二字盗与奸,
倾家二字淫与赌,守家二字勤与俭。
不汲汲于富贵,不戚戚于贫贱。
素位而行,不尤不怨。
先达之人可尊也,不可比媚。
权势之人可远也,不可侮慢。
善有善报,恶有恶报,若有不报,日子未到。
贤者不炫己之长,君子不夺人所好。
救既败之事,如驭临岩之马,休轻加一鞭;
图垂成之功,如挽上滩之舟,莫稍停一棹。
大事不糊涂,小事不渗漏。
内藏精明,外示浑厚。
恩宜先淡而浓,先浓后淡者,人忘其惠;
威宜自严而宽,先宽后严者,人怨其酷。
以积货财之心积学问,则盛德日新;
以爱妻子之心爱父母,则孝行自笃。
学须静,才须学。
非学无以广才,非静无以成学。
不患老而无成,只怕幼而不学。
富贵如刀兵戈矛,稍放纵便销膏靡骨而不知;
贫贱如针砭药石,一忧勤即砥节砺行而不觉。
不矜细行,终累大德。
亲戚不悦,无务外交;
事不终始,无务多业。
民为邦本,本固邦宁。
安居饱食,天下太平。
临难勿苟免,临财勿苟得。
谗言不可听,听之祸殃结。
君听臣遭诛,父听子遭灭,夫妇听之离,
兄弟听之别,朋友听之疏,亲戚听之绝。
性天澄澈,即饥餐渴饮,无非康济身肠;
心地沉迷,纵演偈谈玄,总是播弄精魄。
芝兰生于深林,不以无人而不芳;
君子修其道德,不为穷困而改节。
廉官可酌贪泉水,志士不受嗟来食。

\backmatter

\end{document}