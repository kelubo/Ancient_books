%-*- coding: UTF-8 -*-
% 道德经
% 道德经.tex

\documentclass[a4paper,12pt,UTF8,twoside]{ctexbook}

% 设置纸张信息。
\RequirePackage[a4paper]{geometry}
\geometry{
	%textwidth=138mm,
	%textheight=215mm,
	%left=27mm,
	%right=27mm,
	%top=25.4mm, 
	%bottom=25.4mm,
	%headheight=2.17cm,
	%headsep=4mm,
	%footskip=12mm,
	%heightrounded,
	inner=1in,
	outer=1.25in
}

% 设置字体,并解决显示难检字问题。
\xeCJKsetup{AutoFallBack=true}
\setCJKmainfont{SimSun}[BoldFont=SimHei, ItalicFont=KaiTi, FallBack=SimSun-ExtB]

% 目录 chapter 级别加点(.)。
\usepackage{titletoc}
\titlecontents{chapter}[0pt]{\vspace{3mm}\bf\addvspace{2pt}\filright}{\contentspush{\thecontentslabel\hspace{0.8em}}}{}{\titlerule*[8pt]{.}\contentspage}

% 设置 part 和 chapter 标题格式。
\ctexset{
	part/name={},
	part/number={},
	chapter/name={第,章},
	chapter/number={\chinese{chapter}}
}

% 设置古文原文格式。
\newenvironment{yuanwen}{\bfseries\zihao{4}}

\title{\heiti\zihao{0} 道德经}
\author{老子}
\date{}

\begin{document}

\maketitle

\tableofcontents
	
\frontmatter
\chapter{前言}
	
	老子,姓李名耳,字聃,一字或曰谥伯阳。华夏族, 楚国苦县厉乡曲仁里\footnote{今河南省鹿邑县太清宫镇。}人,约生活于前571年至471年之间。是我国古代伟大的哲学家和思想家、道家学派创始人,被唐朝帝王追认为李姓始祖。老子故里鹿邑县亦因老子先后由苦县更名为真源县、卫真县、鹿邑县,并在鹿邑县境内留下许多与老子息息相关的珍贵文物。老子乃世界文化名人,世界百位历史名人之一,存世有《道德经》,其作品的精华是朴素的辩证法,主张无为而治,其学说对中国哲学发展具有深刻影响。在道教中,老子被尊为道教始祖。老子与后世的庄子并称老庄。
	
	《老子》,又称《道德真经》、《道德经》、《五千言》、《老子五千文》,是中国古代先秦诸子分家前的一部著作,为其时诸子所共仰,传说是春秋时期的老子李耳\footnote{似是作者、注释者、传抄者的集合体。}所撰写,是道家哲学思想的重要来源。道德经分上下两篇,原文上篇《德经》、下篇《道经》,不分章,后改为《道经》37章在前,第38章之后为《德经》,并分为81章。是中国历史上首部完整的哲学著作。 
	
	1973年,长沙马王堆汉墓出土了一批古书,其中就包括《道德经》,分为甲乙两个版本。经整理复原之后发现,该版本与当下流行的传世版本,存在这一些差异。这些差异大多只在只字片语之间,但意义却有千差万别之远。
	
\mainmatter

\part{道经}

\chapter{论道}

\begin{yuanwen}
道\footnote{名词,指的是宇宙的本原和实质,引申为原理、原则、真理、规律等。},可道\footnote{动词。指解说、表述的意思,犹言“说得出”。}也,非恒\footnote{一般的,普通的。}道也。名\footnote{名词,指“道”的形态。},可名\footnote{动词,说明的意思。}也,非恒名也。 “无”,名天地之始;“有”,名万物之母\footnote{母体,根源。}。 故,常“无”,欲以观其妙;常“有”,欲以观其徼\footnote{ji\`ao}。 此两者,同出而异名,同谓之玄。玄之又玄,眾妙之门。

道可道\footnote{text},非常道。名可名,非常名\footnote{text}。无\footnote{text},名天地之始\footnote{text}。有,名万物之母\footnote{text}。故常无,欲以观其妙\footnote{text};常有,欲以观其徼\footnote{text}。此两者,同出而异名,同谓之玄\footnote{text}。玄之又玄,众妙之门\footnote{text}。
\end{yuanwen}
	
	
	
	
	
	“道”如果可以用言语来表述,那它就是常“道”(“道”是可以用言语来表述的,它并非一般的“道”);“名”如果可以用文辞去命名,那它就是常“名”(“名”也是可以说明的,它并非普通的“名”)。“无”可以用来表述天地浑沌未开之际的状况;而“有”,则是宇宙万物产生之本原的命名。因此,要常从“无”中去观察领悟“道”的奥妙;要常从“有”中去观察体会“道”的端倪。无与有这两者,来源相同而名称相异,都可以称之为玄妙、深远。它不是一般的玄妙、深奥,而是玄妙又玄妙、深远又深远,是宇宙天地万物之奥妙的总门(从“有名”的奥妙到达无形的奥妙,“道”是洞悉一切奥妙变化的门径)。
	

	
	
	④无名:指无形。
	
	⑤有名:指有形。
	
	⑦恒:经常。
	
	⑧眇(miao):通妙,微妙的意思。
	
	⑨徼(jiao):边际、边界。引申端倪的意思。
	
	⑩谓:称谓。此为“指称”。
	
	⑾玄:深黑色,玄妙深远的含义。
	
	⑿门:之门,一切奥妙变化的总门径,此用来比喻宇宙万物的唯一原“道”的门径。
	
	\chapter{美善}
	
	天下皆知美之为美,斯恶(è)已;皆知善之为善,斯不善已。故有无相生,难易相成,长短相较,高下相倾,音声相和(hè),前后相随。是以圣人处无为之事,行不言之教,万物作焉而不辞,生而不有,为而不恃,功成而弗居。夫(fú)唯弗居,是以不去。
	
	\chapter{无为}
	
	不尚贤,使民不争;不贵难得之货,使民不为盗;不见(xiàn)可欲,使民心不乱。是以圣人之治,虚其心,实其腹;弱其志,强其骨。常使民无知无欲,使夫(fú)智者不敢为也。为无为,则无不治。
	
	\chapter{道沖}
	道冲而用之或不盈,渊兮似万物之宗。挫其锐,解其纷,和其光,同其尘。湛兮似或存,吾不知谁之子,象帝之先。
	
	
	\chapter{守中}
	
	天地不仁,以万物为刍(chú)狗;圣人不仁,以百姓为刍狗。天地之间,其犹橐龠(tuó	yuè)乎?虚而不屈,动而愈出。多言数(shuò)穷,不如守中。
	
	
	
	
	\chapter{谷神}
	谷神不死,是谓玄牝(pìn),玄牝之门,是谓天地根。绵绵若存,用之不勤。
	
	
	
	
	
	\chapter{无私}
	天长地久。天地所以能长且久者,以其不自生,故能长生。是以圣人后其身而身先,外其身而身存。非以其无私邪(yé)?故能成其私。
	
	
	
	
	\chapter{上善}
	上善若水。水善利万物而不争,处众人之所恶(wù),故几(jī)于道。居善地,心善渊,与善仁,言善信,正善治,事善能,动善时。夫唯不争,故无尤。
	
	
	\chapter{持盈}
	
	持而盈之,不如其已。揣(chuǎi)而锐之,不可长保。金玉满堂,莫之能守。富贵而骄,自遗(yí)其咎。功成身退,天之道。	

	\chapter{玄德}
		
	载(zài)营魄抱一,能无离乎?专气致柔,能婴儿乎?涤除玄览,能无疵乎?爱民治国,能无知(zhì)乎?天门开阖(hé),能无雌乎?明白四达,能无为乎?生之、畜(xù)之,生而不有,为而不恃,长(zhǎng)而不宰,是谓玄德。
	
	\chapter{利用}
	
	三十辐共一毂(gǔ),当其无,有车之用。埏埴(shān zhí)以为器,当其无,有器之用。凿户牖(yǒu)以为室,当其无,有室之用。故有之以为利,无之以为用。
	
	\chapter{为腹}
	
	五色令人目盲,五音令人耳聋,五味令人口爽,驰骋畋(tián)猎令人心发狂,难得之货令人行妨。是以圣人为腹不为目,故去彼取此。
	
	\chapter{宠贵}
	
	宠辱若惊,贵大患若身。何谓宠辱若惊?宠为下,得之若惊,失之若惊,是谓宠辱若惊。何谓贵大患若身?吾所以有大患者,为吾有身,及吾无身,吾有何患!故贵以身为天下,若可寄天下;爱以身为天下,若可托天下。
	
	\chapter{道纪}
	视之不见名曰夷,听之不闻名曰希,搏之不得名曰微。此三者不可致诘(jié),故混(hùn)而为一。其上不皦(jiǎo皎),其下不昧。绳绳(mǐn mǐn )不可名,复归于无物,是谓无状之状,无物之象。是谓惚恍。迎之不见其首,随之不见其后。执古之道,以御今之有,能知古始,是谓道纪。
	
	
	
	\chapter{保盈}
	
	古之善为士者,微妙玄通,深不可识。夫唯不可识,故强(qiǎng)为之容。豫焉若冬涉川,犹兮若畏四邻,俨兮其若容,涣兮若冰之将释,敦兮其若朴,旷兮其若谷,混兮其若浊。孰能浊以静之徐清?孰能安以久动之徐生?保此道者不欲盈,夫唯不盈,故能蔽不新成。
	
	
	
	\chapter{虚静}
	
	致虚极,守静笃(dǔ),万物并作,吾以观复。夫物芸芸,各复归其根。归根曰静,是谓复命。复命曰常,知常曰明,不知常,妄作,凶。知常容,容乃公,公乃王(wàng),王(wàng)乃天,天乃道,道乃久,没(mò)身不殆。
	
	

	
	
	
	\chapter{太上}
	太上,下知有之。其次,亲而誉之。其次,畏之。其次,侮之。信不足焉,有不信焉。悠兮其贵言。功成事遂,百姓皆谓我自然。
	
	\chapter{大道}
	
	大道废,有仁义;慧智出,有大伪;六亲不和,有孝慈;国家昏乱,有忠臣。
	
	
	
	\chapter{三绝}
	
	绝圣弃智,民利百倍;绝仁弃义,民复孝慈;绝巧弃利,盗贼无有。此三者,以为文不足,故令有所属,见(xiàn)素抱朴,少私寡欲。
	
	
	
	
	\chapter{绝学}
	绝学无忧。唯之与阿(ē),相去几何?善之与恶,相去若何?人之所畏,不可不畏。荒兮其未央哉!众人熙熙,如享太牢,如春登台。我独泊兮其未兆,如婴儿之未孩。傫傫(lěi)兮若无所归。众人皆有余,而我独若遗。我愚人之心也哉!沌沌兮!俗人昭昭,我独昏昏;俗人察察,我独闷闷。澹(
	dàn)兮其若海,飂(liù)兮若无止。众人皆有以,而我独顽似鄙。我独异于人,而贵食(sì)母。
	
	
	
	
	\chapter{孔德}
	孔德之容,惟道是从。道之为物,惟恍惟惚。惚兮恍兮,其中有象;恍兮惚兮,其中有物。窈(yǎo)兮冥兮,其中有精;其精甚真,其中有信。自古及今,其名不去,以阅众甫。吾何以知众甫之状哉?以此。
	
	
	
	
	
	\chapter{全归}
	曲则全,枉则直,洼则盈,敝则新,少则得,多则惑。是以圣人抱一,为天下式。不自见xiàn)故明,不自是故彰,不自伐故有功,不自矜故长。夫唯不争,故天下莫能与之争。古之所谓曲则全者,岂虚言哉!诚全而归之。
	
	
	
	\chapter{自然}
	
	希言自然。故飘风不终朝(zhāo),骤雨不终日。孰为此者?天地。天地尚不能久,而况于人乎?故从事于道者,道者同于道,德者同于德,失者同于失。同于道者,道亦乐得之;同于德者,德亦乐得之;同于失者,失亦乐得之。信不足焉,有不信焉。
	
	
	
	\chapter{跂跨}
	
	企者不立,跨者不行,自见(xiàn)者不明,自是者不彰,自伐者无功,自矜者不长。其在道也,曰余食赘(zhuì)行。物或恶(wù)之,故有道者不处(chǔ)。
	
	
	
	\chapter{混成}
	
	有物混(hùn)成,先天地生。寂兮寥兮,独立不改,周行而不殆,可以为天下母。吾不知其名,字之曰道,强(qiǎng)为之名曰大。大曰逝,逝曰远,远曰反。故道大,天大,地大,王亦大。域中有四大,而王居其一焉。人法地,地法天,天法道,道法自然。
	
	
	

	\chapter{重静}
	重为轻根,静为躁君。是以圣人终日行不离辎(zī)重。虽有荣观(guàn),燕处超然,奈何万乘(shèng)之主,而以身轻天下?轻则失本,躁则失君。
	
	
	
	\chapter{要妙}
	
	善行无辙迹,善言无瑕谪(xiá zhé),善数(shǔ)不用筹策,善闭无关楗(jiàn)而不可开,善结无绳约而不可解。是以圣人常善救人,故无弃人;常善救物,故无弃物,是谓袭明。故善人者,不善人之师;不善人者,善人之资。不贵其师,不爱其资,虽智大迷,是谓要妙。	
	
	
	\chapter{常德}
	
	知其雄,守其雌,为天下溪。为天下溪,常德不离,复归于婴儿。知其白,守其黑,为天下式。为天下式,常德不忒(tè),复归于无极。知其荣,守其辱,为天下谷。为天下谷,常德乃足,复归于朴。朴散则为器,圣人用之则为官长(zhǎng)。故大制不割。	
	

	\chapter{神器}
		
	将欲取天下而为之,吾见其不得已。天下神器,不可为也。为者败之,执者失之。故物或行或随,或歔(xū)或吹,或强或羸(léi),或挫或隳(huī)。是以圣人去甚,去奢,去泰。
	
	
	

	\chapter{兵强}
	以道佐人主者,不以兵强天下,其事好(hào)还。师之所处,荆棘生焉。大军之后,必有凶年。善有果而已,不敢以取强。果而勿矜,果而勿伐,果而勿骄,果而不得已,果而勿强。物壮则老,是谓不道,不道早已。	
	
	
	\chapter{佳兵}
	夫佳兵者,不祥之器。物或恶(wù)之,故有道者不处(chǔ)。君子居则贵左,用兵则贵右。兵者,不祥之器,非君子之器。不得已而用之,恬淡为上,胜而不美。而美之者,是乐(yào)杀人。夫乐(yào)杀人者,则不可以得志于天下矣。吉事尚左,凶事尚右。偏将军居左,上将军居右,言以丧(	sāng)礼处之。杀人之众,以哀悲泣之,战胜,以丧礼处之。	
	
	
	
	\chapter{无名}
	道常无名,朴虽小,天下莫能臣也。侯王若能守之,万物将自宾。天地相合以降甘露,民莫之令而自均。始制有名,名亦既有,夫亦将知止。知止可以不殆。譬道之在天下,犹川谷之于江海。
	
	
	
	\chapter{明强}
	
	知人者智,自知者明。胜人者有力,自胜者强。知足者富,强行者有志,不失其所者久,死而不亡者寿。
	
	
	
	\chapter{大道}
	
	大道泛兮,其可左右。万物恃之而生而不辞,功成不名有,衣养万物而不为主,常无欲,可名于小;万物归焉而不为主,可名为大。以其终不自为大,故能成其大。	
	
	
	\chapter{大象}
	执大象,天下往;往而不害,安平太。乐(yuè)与饵,过客止。道之出口,淡乎其无味,视之不足见(jiàn),听之不足闻,用之不足既。
	
	
	
	
	\chapter{微明}
	将欲歙(xī)之,必固张之;将欲弱之,必固强之;将欲废之,必固兴之;将欲夺之,必固与之,是谓微明。柔弱胜刚强。鱼不可脱于渊,国之利器不可以示人。
	
	\chapter{静正}
	
	
	
	
	
	
	道常无为而无不为,侯王若能守之,万物将自化。化而欲作,吾将镇之以无名之朴。无名之朴,夫亦将无欲。不欲以静,天下将自定。	
	
	
	
	
	
	\part{德经}
	
	\chapter{上德}
	
	\begin{yuanwen}
	上德不德,是以有德。下德不失德,是以无德。上德无为,而无以为也。上仁为之,而无以为也。上义为之,而有以为也。上礼为之,而莫之应也,则攘\footnote{r\v{a}ng}臂而扔之。故失道而后德,失德而后仁,失仁而后义,失义而后礼。夫礼者,忠信之薄\footnote{b\'o}也,而乱之首也。前识者,道之华也,而愚之首也。是以大丈夫居其厚,而不居其薄,居其实,而不居其华。故去彼取此。
	\end{yuanwen}
	
	\begin{yuanwen}
	上德不德,是以有德。下德不失德,是以无德。上德无为而无以为,下德为之而有以为\footnote{text}。上仁为之而无以为,上义为之而有以为,上礼为之而莫之应,则攘臂而扔之。故失道而后德,失德而后仁,失仁而后义,失义而后礼。夫礼者,忠信之薄而乱之首。前识者,道之华而愚之始。是以大丈夫处其厚,不居其薄;处其实,不居其华。故去彼取此。
	\end{yuanwen}
	
		
	\chapter{得一}
	
	\begin{yuanwen}
	昔之得一者,天得一以清,地得一以宁,神得一以灵,谷得一以盈,侯王得一而以为天下正。其致之也,谓天毋已清,将恐裂。谓地毋已宁,将恐发\footnote{f\`ei,“发”通“废”。},谓神毋已灵,将恐歇。谓谷毋已盈,将恐竭。谓侯王毋已贵以高,将恐蹶\footnote{ju\'e}。故必贵而以贱为本,必高矣而以下为基。夫是以侯王自谓孤、寡、不穀\footnote{谷gǔ},此其贱之本与?非也?故致数\footnote{shu\`o}誉无誉。是故不欲琭\footnote{l\`u}琭若玉,硌硌若石。
	\end{yuanwen}
	
	\begin{yuanwen}
	昔之得一者,天得一以清,地得一以宁,神得一以灵,谷得一以盈,万物得一以生\footnote{text},侯王得一以为天下贞。其致之。天无以清将恐裂,地无以宁将恐发,神无以灵将恐歇,谷无以盈将恐竭,万物无以生将恐灭\footnote{text},侯王无以贵高将恐蹶。故贵以贱为本,高以下为基。是以侯王自谓孤寡不穀。此非以贱为本邪(yé)?非乎?故致数(shuò)舆(yù)无舆。不欲琭琭如玉,珞(luò)珞如石。
	\end{yuanwen}
	
	
	
	

	\chapter{勤用}
	
	\begin{yuanwen}
	反也者,道之动也。弱也者,道之用也。天下之物生于有,有生于无。
	\end{yuanwen}
	
	\begin{yuanwen}
	反者,道之动;弱者,道之用。天下万物生于有,有生于无。
	\end{yuanwen}
	
	\chapter{闻道}
	 
	\begin{yuanwen}
	上士闻道,勤能行之。中士闻道,若存若亡。下士闻道,大笑之。弗笑,不足以为道。是以建言有之曰:明道如昧,进道如退,夷道如颣\footnote{l\`ei}。上德如谷,大白如辱,广德如不足。建德如偷,质真如渝。大方无隅,大器免成,大音希声,大象无形,道褒无名。夫唯道,善始且善成。
	\end{yuanwen}
	
	\begin{yuanwen}
	上士闻道,勤而行之;中士闻道,若存若亡;下士闻道,大笑之,不笑不足以为道。故建言有之:明道若昧,进道若退,夷道若颣。上德若谷,大白若辱,广德若不足,建德若偷,质真若渝。大方无隅,大器晚成,大音希声,大象无形。道隐无名,夫唯道善贷且成。
	\end{yuanwen}
	
	\chapter{冲和}
	
	\begin{yuanwen}
	道生一,一生二,二生三,三生万物。万物负阴而抱阳,中气以为和。人之所恶(wù),唯孤、寡、不穀(谷gǔ),而王公以自名也。物或损之而益,益之而损。故人之所教(jiào),亦议而教人。故强梁者不得其死,我将以为学父。
	\end{yuanwen}
	
	\begin{yuanwen}
	道生一,一生二,二生三,三生万物。万物负阴而抱阳,冲气以为和。人之所恶(wù),唯孤寡不穀(谷gǔ),而王公以为称(chēng)。故物,或损之而益,或益之而损。人之所教(jiào),我亦教之。强梁者不得其死,吾将以为教父。
	\end{yuanwen}
	
	
\chapter{至柔}
	
\begin{yuanwen}
天下之至柔\footnote{text},驰骋天下之至坚\footnote{text}。无有入无间\footnote{text},吾是以知无为之有益。不言之教,无为之益,天下希及之。	
\end{yuanwen}
	
	
\chapter{名身}

\begin{yuanwen}
【通行本】

名与身孰亲?身与货孰多\footnote{text}?得与亡孰病\footnote{text}?甚爱必大费\footnote{text},多藏必厚亡\footnote{text}。故知足不辱\footnote{text},知止不殆,可以长久。
\end{yuanwen}

\begin{yuanwen}
名与身孰亲?身与货孰多?得与亡孰病? 是故甚爱必大费,多藏必厚亡。知足不辱,知止不殆,可以长久。
\end{yuanwen}
	
	
	
\chapter{清静}

\begin{yuanwen}
大成若缺\footnote{text},其用不弊\footnote{text}。大盈若冲,其用不穷\footnote{text}。大直若屈\footnote{text},大巧若拙,大辩若讷\footnote{text}。静胜躁,寒胜热\footnote{text}。清静为天下正\footnote{text}。
\end{yuanwen}
	
	
	
	
\chapter{知足}

\begin{yuanwen}
天下有道,却走马以粪\footnote{text};天下无道,戎马生于郊\footnote{text}。祸莫大于不知足,咎莫大于欲得。故知足之足,常足矣\footnote{text}。
\end{yuanwen}
	
	
	
	
\chapter{户}

\begin{yuanwen}
不出户,知天下;不窥\footnote{text}牖,见天道\footnote{text}。其出弥远,其知弥少。是以圣人不行而知,不见而明,不为而成。
\end{yuanwen}
	


\chapter{日损}

\begin{yuanwen}
为学日益\footnote{text},为道日损\footnote{text}。损之又损,以至于无为。无为而无不为。取天下常以无事\footnote{text},及其有事\footnote{text},不足以取天下。
\end{yuanwen}	

	

\chapter{浑心}

\begin{yuanwen}
圣人常无心\footnote{有版本为“无常心”},以百姓之心为心。善者,吾善之;不善者,吾亦善之,德\footnote{text}善。信者,吾信之;不信者,吾亦信之,德信。圣人在天下,歙歙(xīxī)焉,为天下浑其心\footnote{text}。百姓皆注其耳目\footnote{text},圣人皆孩之\footnote{text}。
\end{yuanwen}
	
	
	
\chapter{摄生}

\begin{yuanwen}
出生入死\footnote{text},生之徒\footnote{text},十有三\footnote{text},死之徒\footnote{text},十有三。人之生,动之于死地,亦十有三。夫何故?以其生生之厚\footnote{text}。盖闻善摄生者\footnote{text},陆行不遇兕(sì)虎\footnote{text},入军不被(pī)甲兵\footnote{text}。兕无所投其角\footnote{text},虎无所措其爪(	zhǎo),兵无所容其刃。夫何故?以其无死地\footnote{text}。
\end{yuanwen}
	
	
	
\chapter{尊贵}

\begin{yuanwen}
道生之,德畜(xù)之,物形之,势成之\footnote{text}。是以万物莫不尊道而贵德。道之尊,德之贵,夫莫之命而常自然\footnote{text}。故道生之,德畜之。长之育之、亭之毒之\footnote{text}、养之覆之\footnote{text}。生而不有,为而不恃,长(zhǎng)而不宰,是谓玄德。
\end{yuanwen}


\chapter{有始}
	
\begin{yuanwen}
天下有始\footnote{text},以为天下母\footnote{text}。既得其母,以知其子\footnote{text};既知其子,复守其母,没(mò)身不殆。塞(sè)其兑,闭其门\footnote{text},终身不勤\footnote{text}。开其兑,济其事\footnote{text},终身不救。见(jiàn)小曰明\footnote{text},守柔曰强\footnote{text}。用其光,复归其明,无遗身殃\footnote{text},是为袭常\footnote{text}。
\end{yuanwen}
	
\chapter{}

\begin{yuanwen}
使我介然有知\footnote{text},行于大道,唯施是畏\footnote{text}。大道甚夷\footnote{text},而人好径\footnote{text}。朝(cháo)甚除\footnote{text},田甚芜,仓甚虚。服文彩,带利剑,厌饮食\footnote{text},财货有余,是为盗竽\footnote{text}。非道也哉!
\end{yuanwen}	

	
\chapter{}

\begin{yuanwen}
善建者不拔\footnote{text},善抱者不脱\footnote{text},子孙以祭祀不辍\footnote{text}。修之于身,其德乃真;修之于家,其德乃余;修之于乡,其德乃长(zhǎng)\footnote{text};修之于邦\footnote{text},其德乃丰;修之于天下,其德乃普。故以身观身,以家观家,以乡观乡\footnote{text},以邦观邦,以天下观天下。吾何以知天下然哉?以此\footnote{text}。
\end{yuanwen}	

	
\chapter{}

\begin{yuanwen}
含德之厚,比于赤子。毒虫不螫(shì)\footnote{text},猛兽不据\footnote{text},攫(jué)鸟不搏\footnote{text}。骨弱筋柔而握固。未知牝牡之合而朘作\footnote{text},精之至也。终日号而不嗄(shà)\footnote{text},和之至也\footnote{text}。知和曰常\footnote{text},知常曰明,益生曰祥\footnote{text},心使气曰强\footnote{text}。物壮则老\footnote{text},谓之不道,不道早已\footnote{text}。
\end{yuanwen}	

	
\chapter{}

\begin{yuanwen}
知(zhì)者不言,言者不知(zhì)\footnote{text}。塞(sè)其兑,闭其门\footnote{text},挫其锐\footnote{text};解其纷,和其光,同其尘\footnote{text},是谓玄同\footnote{text}。故不可得而亲,不可得而疏;不可得而利,不可得而害;不可得而贵,不可得而贱\footnote{text},故为天下贵。
\end{yuanwen}	

	
\chapter{}	

\begin{yuanwen}
以正治国\footnote{text},以奇用兵\footnote{text},以无事取天下。吾何以知其然哉?以此\footnote{text}:天下多忌讳\footnote{text},而民弥贫;人多利器\footnote{text},国家滋昏;人多伎(jì)巧\footnote{text},奇物滋起\footnote{text};法令滋彰,盗贼多有\footnote{text}。故圣人云:“我无为,而民自化\footnote{text};我好静,而民自正;我无事,而民自富;我无欲,而民自朴。”
\end{yuanwen}	

	
\chapter{}

\begin{yuanwen}
其政闷闷\footnote{text},其民淳淳\footnote{text};其政察察\footnote{text},其民缺缺\footnote{text}。祸兮,福之所倚;福兮,祸之所伏\footnote{text}。孰知其极:其无正也\footnote{text}。正复为奇\footnote{text},善复为妖\footnote{text}。人之迷,其日固久\footnote{text}。是以圣人方而不割\footnote{text},廉而不刿(guì)\footnote{text},直而不肆\footnote{text},光而不耀\footnote{text}。
\end{yuanwen}	

	
\chapter{}	

\begin{yuanwen}
治人事天\footnote{text},莫若啬(sè)。夫唯啬,是谓早服\footnote{text}。早服谓之重(chóng)积德,重(chóng)积德,则无不克,无不克,则莫知其极\footnote{text},莫知其极,可以有国。有国之母\footnote{text},可以长久。是谓深根固柢(dǐ),长生久视\footnote{text}之道。
\end{yuanwen}	

	
\chapter{}	

\begin{yuanwen}
治大国,若烹小鲜\footnote{text}。以道莅(lì)天下\footnote{text},其鬼不神\footnote{text}。非其鬼不神\footnote{text},其神不伤人;非其神不伤人,圣人亦不伤人。夫两不相伤\footnote{text},故德交归焉。
\end{yuanwen}	

	
\chapter{}	

\begin{yuanwen}
大邦者下流\footnote{text}。天下之牝,天下之交也\footnote{text}。牝常以静胜牡,以静为下。故大邦以下小邦,则取小邦;小邦以下大邦,则取大邦\footnote{text}。故或下以取,或下而取\footnote{text}。大邦不过欲兼畜(xù)人\footnote{text},小邦不过欲入事人,夫两者各得所欲,大者宜为下。
\end{yuanwen}
	
\chapter{}	
\begin{yuanwen}
道者,万物之奥\footnote{text},善人之宝,不善人之所保\footnote{text}。美言可以市尊,美行可以加人\footnote{text}。人之不善,何弃之有?故立天子,置三公\footnote{text},虽有拱璧以先驷马\footnote{text},不如坐进此道\footnote{text}。古之所以贵此道者何?不曰:求以得。有罪以免邪(yé)?故为天下贵。
\end{yuanwen}	

	
\chapter{}	

\begin{yuanwen}
为无为,事无事,味无味\footnote{text}。大小多少\footnote{text},报怨以德\footnote{text}。图难于其易,为大于其细。天下难事,必作于易,天下大事,必作于细。是以圣人终不为大\footnote{text},故能成其大。夫轻诺必寡信,多易必多难。是以圣人犹难之,故终无难矣。
\end{yuanwen}

	
\chapter{}	

\begin{yuanwen}
其安易持\footnote{text},其未兆易谋\footnote{text},其脆易泮(pàn)\footnote{text},其微易散。为之于未有,治之于未乱。合抱之木,生于毫末\footnote{text};九层之台,起于累土\footnote{text};千里之行,始于足下。为者败之,执者失之。是以圣人无为,故无败;无执,故无失\footnote{text}。民之从事,常于几成而败之\footnote{text}。慎终如始,则无败事。是以圣人欲不欲,不贵难得之货。学不学,复众人之所过。以辅万物之自然,而不敢为\footnote{text}。
\end{yuanwen}

	
\chapter{}	

\begin{yuanwen}
古之善为道者,非以明民\footnote{text},将以愚之\footnote{text}。民之难治,以其智多\footnote{text}。故以智治国,国之贼\footnote{text};不以智治国,国之福。知此两者\footnote{text},亦稽(jī)式\footnote{text}。常知稽式,是谓玄德。玄德深矣,远矣,与物反矣\footnote{text},然后乃至大顺\footnote{text}。
\end{yuanwen}

	
\chapter{}	

\begin{yuanwen}
江海所以能为百谷王者\footnote{text},以其善下之\footnote{text},故能为百谷王。是以圣人欲上民\footnote{text},必以言下之;欲先民,必以身后之。是以圣人处上而民不重\footnote{text},处前而民不害,是以天下乐推而不厌。以其不争,故天下莫能与之争。
\end{yuanwen}

	
\chapter{}	

\begin{yuanwen}
天下皆谓我“道”大\footnote{text},似不肖(xiào)\footnote{text}。夫唯大,故似不肖。若肖,久矣其细也夫\footnote{text}。我有三宝\footnote{text},持而保之:一曰慈,二曰俭\footnote{text},三曰不敢为天下先。慈,故能勇\footnote{text};俭,故能广\footnote{text};不敢为天下先,故能成器长(zhǎng)\footnote{text}。今舍慈且勇\footnote{text},舍俭且广,舍后且先,死矣!夫慈,以战则胜\footnote{text},以守则固。天将救之,以慈卫之。
\end{yuanwen}


\chapter{}	

\begin{yuanwen}
善为士者\footnote{text},不武,善战者,不怒,善胜敌者,不与\footnote{text},善用人者,为之下。是谓不争之德,是谓用人之力,是谓配天\footnote{text},古之极也。
\end{yuanwen}


\chapter{}	

\begin{yuanwen}

\end{yuanwen}
用兵有言:“吾不敢为主,而为客\footnote{text};不敢进寸,而退尺。”是谓行(xíng)无行(háng)\footnote{text},攘(rǎng)无臂\footnote{text},扔无敌\footnote{text},执无兵。祸莫大于轻敌,轻敌几丧吾宝。故抗兵相若\footnote{text},哀者胜矣\footnote{text}。

\chapter{}	

\begin{yuanwen}
吾言甚易知,甚易行。天下莫能知,莫能行。言有宗\footnote{text},事有君\footnote{text}。夫唯无知,是以不我知\footnote{text}。知我者希,则我者贵\footnote{text},是以圣人被(pī,“被”同“披”)褐而怀玉\footnote{text}。
\end{yuanwen}


\chapter{}	

\begin{yuanwen}
知不知\footnote{text},尚矣;不知知,病也。圣人不病\footnote{text},以其病病\footnote{text}。夫唯病病,是以不病。
\end{yuanwen}

	
\chapter{}	

\begin{yuanwen}
民不畏威\footnote{text},则大威至\footnote{text}。无狎(xiá)其所居\footnote{text},无厌(yà,“厌”同“压”)其所生\footnote{text}。夫唯不厌(yà,“厌”同“压”)\footnote{text},是以不厌(yàn)。是以圣人自知不自见(xiàn)\footnote{text};自爱不自贵。故去彼取此\footnote{text}。
\end{yuanwen}

	
\chapter{}	

\begin{yuanwen}
勇于敢则杀,勇于不敢则活\footnote{text}。此两者,或利或害\footnote{text}。天之所恶(wù),孰知其故?是以圣人犹难之\footnote{text}。天之道\footnote{text},不争而善胜,不言而善应,不召而自来,繟(chǎn)然而善谋\footnote{text}。天网恢恢\footnote{text},疏而不失\footnote{text}。
\end{yuanwen}

	
\chapter{}	

\begin{yuanwen}
民不畏死,奈何以死惧之?若使民常畏死,而为奇者\footnote{text},吾得执而杀之\footnote{text},孰敢?常有司杀者\footnote{text}杀,夫代司杀者杀\footnote{text},是谓代大匠斲(zhuó)。夫代大匠斲者,希有不伤其手矣。
\end{yuanwen}

	
\chapter{}	

\begin{yuanwen}
民之饥,以其上食税之多,是以饥。民之难治,以其上之有为\footnote{text},是以难治。民之轻死,以其上求生之厚\footnote{text},是以轻死。夫唯无以生为者\footnote{text},是贤于贵生\footnote{text}。
\end{yuanwen}

	
\chapter{}	

\begin{yuanwen}	
人之生也柔弱,其死也坚强。草木\footnote{text}之生也柔脆\footnote{text},其死也枯槁。故坚强者死之徒,柔弱者生之徒。是以兵强则灭,木强则折\footnote{text}。强大处下,柔弱处上。
\end{yuanwen}
		
\chapter{}	

\begin{yuanwen}	
天之道,其犹张弓与?高者抑之,下者举之;有余者损之\footnote{text},不足者补之。天之道,损有余而补不足。人之道\footnote{text}则不然,损不足以奉有余。孰能有余以奉天下?唯有道者。是以圣人为而不恃,功成而不处,其不欲见(xiàn)贤\footnote{text}。
\end{yuanwen}
		
\chapter{}

\begin{yuanwen}	
天下莫柔弱于水,而攻坚强者莫之能胜,以其无以易之\footnote{text}。弱之胜强,柔之胜刚,天下莫不知,莫能行。是以圣人云:“受国之垢\footnote{text},是谓社稷主;受国不祥\footnote{text},是为天下王。”正言若反\footnote{text}。
\end{yuanwen}
		
\chapter{}	

\begin{yuanwen}	
和大怨,必有余怨,安可以为善?是以圣人执左契\footnote{text},而不责\footnote{text}于人。有德司契,无德司彻\footnote{text}。天道无亲\footnote{text},常与善人。
\end{yuanwen}
		
\chapter{}	

\begin{yuanwen}	
小国寡民\footnote{text}。使有什伯(bǎi)之器而不用\footnote{text},使民重(zhòng)死而不远徙(xí)\footnote{text}。虽有舟舆\footnote{text},无所乘之;虽有甲兵,无所陈之\footnote{text}。使人复结绳\footnote{text}而用之。甘其食,美其服,安其居,乐其俗\footnote{text}。邻国相望,鸡犬之声相闻,民至老死,不相往来。
\end{yuanwen}
		
\chapter{}
	
\begin{yuanwen}
信言不美\footnote{text},美言不信。善者不辩\footnote{text},辩者不善。知(zhì)者不博,博者不知(zhì)。圣人不积\footnote{text},既以为人,己愈有\footnote{text};既以与人,己愈多\footnote{text}。天之道,利而不害\footnote{text}。圣人之道\footnote{text},为而不争。
\end{yuanwen}
	
\end{document}