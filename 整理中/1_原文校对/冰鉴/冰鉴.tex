% 冰鉴
% 冰鉴.tex

\documentclass[12pt,UTF8]{ctexbook}

% 设置纸张信息。
\usepackage[a4paper,twoside]{geometry}
\geometry{
	left=25mm,
	right=25mm,
	bottom=25.4mm,
	bindingoffset=10mm
}

% 设置字体,并解决显示难检字问题。
\xeCJKsetup{AutoFallBack=true}
\setCJKmainfont{SimSun}[BoldFont=SimHei, ItalicFont=KaiTi, FallBack=SimSun-ExtB]

% 目录 chapter 级别加点(.)。
\usepackage{titletoc}
\titlecontents{chapter}[0pt]{\vspace{3mm}\bf\addvspace{2pt}\filright}{\contentspush{\thecontentslabel\hspace{0.8em}}}{}{\titlerule*[8pt]{.}\contentspage}

% 设置 part 和 chapter 标题格式。
\ctexset{
	chapter/name={},
	chapter/number={}
}

% 设置古文原文格式。
\newenvironment{yuanwen}{\bfseries\zihao{4}}

% 设置署名格式。
\newenvironment{shuming}{\hfill\bfseries\zihao{4}}

% 注脚每页重新编号,避免编号过大。
\usepackage[perpage]{footmisc}

\title{\heiti\zihao{0} 冰鉴}
\author{曾国藩}
\date{}

\begin{document}

\maketitle
\tableofcontents

\frontmatter
\chapter{前言}



\mainmatter

\chapter{神骨}

语云:“脱谷为糠,其髓斯存”,神之谓也。“山骞不崩,唯石为镇”,骨之谓也。一身精神,具乎两目;一身骨相,具乎面部。他家兼论形骸,文人先观神骨。开门见山,此为第一。

文人论神,有清浊之辨。清浊易辨,邪正难辨。欲辨邪正,先观动静;静若含珠,动若木发;静若无人,动若赴的,此为澄清到底。静若萤光,动若流水,尖巧而喜淫;静若半睡,动若鹿骇,别才而深思。一为败器,一为隐流,均之托迹于清,不可不辨。 

凡精神,抖擞处易见,断续处难见。断者出处断,续者闭处续。道家所谓“收拾入门”之说,不了处看其脱略,做了处看其针线。小心者,从其不了处看之,疏节阔目,若不经意,所谓脱略也。大胆者,从其做了处看之,慎重周密,无有苟且,所谓针线也。二者实看向内处,稍移外便落情态矣,情态易见。 

骨有九起:天庭骨隆起,枕骨强起,顶骨平起,佐串骨角起,太阳骨线起,眉骨伏犀起,鼻骨芽起,颧骨若不得而起,项骨平伏起。在头,以天庭骨、枕骨、太阳骨为主;在面,以眉骨、颧骨为主。五者备,柱石之器也;一则不穷;二则不贱;三则动履稍胜;四则贵矣。 

骨有色,面以青为贵,“少年公卿半青面”是也。紫次之,白斯下矣。骨有质,头以联者为贵。碎次之。

总之,头上无恶骨,面佳不如头佳。然大而缺天庭,终是贱品;圆而无串骨,半是孤僧;鼻骨犯眉,堂上不寿。颧骨与眼争,子嗣不立。此中贵贱,有毫厘千里之辨。 

\chapter{刚柔}

既识神骨,当辨刚柔。刚柔,则五行生克之数,名曰"先天种子",不足用补,有余用泄。消息与命相通 ,此其较然易见者。 

五行有合法,木合火,水合木,此顺而合。顺者多富,即贵亦在浮沉之间。金与火仇,有时合火,推之水土者皆然,此逆而合者,其贵非常。然所谓逆合者,金形带火则然,火形带金,则三十死矣;水形带土则然,土形带水,则孤寡终老矣;木形带金则然,金形带木,则刀剑随身矣。此外牵合,俱是杂格,不入文人正论。 

五行为外刚柔,内刚柔,则喜怒、跳伏、深浅者是也。喜高怒重,过目辄忘,近"粗"。伏亦不伉,跳亦不扬,近"蠢"。初念甚浅,转念甚深,近"奸 "。内奸者 ,功名可期 。粗蠢各半者,胜人以寿。纯奸能豁达,其人终成。纯粗无周密,半途必弃。观人所忽,十有九八矣。 

\chapter{容貌}

容以七尺为期,貌合两仪而论。胸腹手足,实接五行;耳目口鼻,全通四气。相顾相称,则福生;如背如凑,则林林总总,不足论也。 

容贵“整”,“整”非整齐之谓。短不豕蹲,长不茅立,肥不熊餐,瘦不鹊寒,所谓“整”也。背宜圆厚,腹宜突坦,手宜温软,曲若弯弓,足宜丰满,下宜藏蛋,所谓“整”也。五短多贵,两大不扬,负重高官,鼠行好利,此为定格。他如手长于身,身过于体,配以佳骨,定主封侯;罗纹满身,胸有秀骨,配以妙神,不拜相即鼎甲矣。

貌有清、古、奇、秀之别,总之须看科名星与阴骘纹为主。科名星,十三岁至三十九岁随时而见;阴骘纹,十九岁至四十六岁随时而见。二者全,大物也;得一亦贵。科名星见于印堂眉彩,时隐时见,或为钢针,或为小丸,尝有光气,酒后及发怒时易见。阴骘纹见于眼角,阴雨便见,如三叉样,假寐时最易见。得科名星者早荣,得阴骘纹者迟发。二者全无,前程莫问。阴骘纹见于喉间,又主生贵子;杂路不在此路。 

目者面之渊,不深则不清。鼻者面之山,不高则不灵。口阔而方禄千种,齿多而圆不家食。眼角入鬓,必掌刑名。顶见于面,终司钱谷:出贵征也。舌脱无官,橘皮不显。文人有伤左目,鹰鼻动便食人:此贱征也。

\chapter{情态}

容貌者,骨之余,常佐骨之不足。情态者,神之余,常佐神之不足。久注观人精神,乍见观人情态。大家举止,羞涩亦佳;小儿行藏,跳叫愈失。大旨亦辨清浊,细处兼论取舍。 

有弱态,有狂态,有疏懒态,有周旋态。飞鸟依人,情致婉转,此弱态也。不衫不履,旁若无人,此狂态也。坐止自如,问答随意,此疏懒态也。饰其中机,不苟言笑,察言观色,趋吉避凶,则周旋态也。皆根其情,不由矫枉。弱而不媚,狂而不哗,疏懒而真诚,周旋而健举,皆能成器;反之,败类也。大概亦得二三矣。 

前者恒态,又有时态。方有对谈,神忽他往;众方称言,此独冷笑;深险难近,不足与论情。言不必当,极口称是,未交此人,故意底毁;卑庸可耻,不足与论事。漫无可否,临事迟回;不甚关情,亦为堕泪。妇人之仁,不足与谈心。三者不必定人终身。反此以求,可以交天下士。 

\chapter{须眉}

“须眉男子”。未有须眉不具可称男子者。“少年两道眉,临老一付须。”此言眉主早成,须主晚运也 。然而紫面无须自贵,暴腮缺须亦荣:郭令公半部不全,霍膘骁一副寡脸。此等间逢,毕竟有须眉者,十之九也。 

眉尚彩,彩者,杪处反光也。贵人有三层彩,有一二层者。所谓“文明气象”,宜疏爽不宜凝滞。一望有乘风翔舞之势,上也;如泼墨者,最下。倒竖者,上也;下垂者,最下。长有起伏,短有神气;浓忌浮光,淡忌枯索。如剑者掌兵权,如帚者赴法场。个中亦有征范,不可不辨。但如压眼不利,散乱多忧,细而带媚,粗而无文,是最下乘。 

须有多寡,取其与眉相称。多者,宜清、宜疏、宜缩、宜参差不齐;少者,宜光、宜健、宜圆、宜有情照顾。卷如螺纹,聪明豁达;长如解索,风流荣显;劲如张戟,位高权重;亮若银条,早登廊庙,皆宦途大器。紫须剑眉,声音洪壮;蓬然虬乱,尝见耳后,配以神骨清奇,不千里封侯,亦十年拜相。他如“辅须先长终不利”、“人中不见一世穷”、“鼻毛接须多滞晦”、“短毙遮口饿终身”,此其显而可见者耳。

\chapter{声音}

人之声音,犹天地之气,轻清上浮,重浊下坠。始于丹田,发于喉,转于舌,辨于齿,出于唇,实与五音相配。取其自成一家,不必一一合调,闻声相思,其人斯在,宁必一见决英雄哉! 

声与音不同 。声主“张”,寻发处见;音主“敛”,寻歇处见。辨声之法,必辨喜怒哀乐;喜如折竹,怒如阴雷起地,哀如击薄冰,乐如雪舞风前,大概以“轻清”;为上。声雄者,如钟则贵,如锣则贱;声雌者,如雉鸣则贵,如蛙鸣则贱。远听声雄,近听悠扬,起若乘风,止如拍琴,上上 。“大言不张唇,细言不露齿”,上也,出而不返 ,荒郊牛鸣。急而不达,深夜鼠嚼;或字句相联,喋喋利口;或齿喉隔断,喈喈混谈:市井之夫,何足比较? 

音者,声之余也,与声相去不远,此则从细曲中见耳。贫贱者有声无音,尖巧者有音无声,所谓“禽无声,兽无音”是也。凡人说话,是声散在前后左右者是也。开谈多含情,话终有余响,不唯雅人,兼称国士;口阔无溢出,舌尖无窕音,不·唯实厚,兼获名高。 

\chapter{气色}

面部如命,气色如运。大命固宜整齐,小运亦当亨泰。是故光焰不发,珠玉与瓦砾同观;藻绘未扬,明光与布葛齐价。大者主一生祸福,小者亦三月吉凶。 

人以气为主,于内为精神,于外为气色。有终身之气色,“少淡、长明、壮艳、老素”是也。有一年之气色 ,“春青、夏红、秋黄、冬白“是也。有一月之气色 ,“朔后森发,望后隐跃”是也 。有一日之气色,“早青、昼满、晚停、暮静”是也。 

科名中人,以黄为主,此正色也。黄云盖顶,必掇大魁;黄翅入鬓,进身不远;印堂黄色,富贵逼人;明堂素净,明年及第。他如眼角霞鲜,决利小考;印堂垂紫,动获小利,红晕中分,定产佳儿;两颧红润,骨肉发迹。由此推之,足见一斑矣。

色忌青,忌白。青常见于眼底,白常见于眉端。然亦不同:心事忧劳,青如凝墨;祸生不测,青如浮烟;酒色惫倦,白如卧羊;灾晦催人,白如傅粉。又有青而带紫,金形遇之而飞扬,白而有光,土庚相当亦富贵,又不在此论也 。最不佳者:“太白夹日月,乌鸟集天庭,桃花散面颊 ,预尾守地阁。”有一于此,前程退落,祸患再三矣。 

\end{document}