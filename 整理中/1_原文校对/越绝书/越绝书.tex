% 越绝书
% 越绝书.tex

\documentclass[12pt,UTF8]{ctexbook}

% 设置纸张信息。
\usepackage[a4paper,twoside]{geometry}
\geometry{
	left=25mm,
	right=25mm,
	bottom=25.4mm,
	bindingoffset=10mm
}

% 设置字体,并解决显示难检字问题。
\xeCJKsetup{AutoFallBack=true}
\setCJKmainfont{SimSun}[BoldFont=SimHei, ItalicFont=KaiTi, FallBack=SimSun-ExtB]

% 目录 chapter 级别加点(.)。
\usepackage{titletoc}
\titlecontents{chapter}[0pt]{\vspace{3mm}\bf\addvspace{2pt}\filright}{\contentspush{\thecontentslabel\hspace{0.8em}}}{}{\titlerule*[8pt]{.}\contentspage}

% 设置 part 和 chapter 标题格式。
\ctexset{
	part/name= {越绝卷第,},
	part/number={\chinese{part}},
	chapter/name={第,篇},
	chapter/number={\chinese{chapter}}
}

% 设置古文原文格式。
\newenvironment{yuanwen}{\bfseries\zihao{4}}

% 设置署名格式。
\newenvironment{shuming}{\hfill\bfseries\zihao{4}}

% 注脚每页重新编号,避免编号过大。
\usepackage[perpage]{footmisc}

\title{\heiti\zihao{0} 越绝书}
\author{}
\date{}

\begin{document}

\maketitle
\tableofcontents

\frontmatter
\chapter{前言、序言}
《越绝书》十五卷,不著撰人名氏。书中《吴地传》称勾践徙琅琊,到建武二十八年,凡五百六十七年,则后汉初人也。书末《叙外传记》以廋词隐其姓名。其云以去为姓,得衣乃成,是袁字也。厥名有米,覆之以庚,是康字也。禹来东征,死葬其疆,是会稽人也。又云文词属定,自於邦贤,以口为姓,承之以天,是吴字也。楚相屈原,与之同名,是平字也。然则此书为会稽袁康所作,同郡吴平所定也。王充《论衡·按书篇》曰:东番邹伯奇,临淮袁太伯、袁文衡,会稽吴君高、周长生之辈,位虽不至公卿,诚能知之囊橐,文雅之英雄也。观伯奇之《元思》、太伯之《易童句》(案童疑作章),文术之《箴铭》,君高之《越纽录》,长生之《洞历》,刘子政、扬子云不能过也。所谓吴君高殆即平字,所谓《越纽录》殆即此书欤?杨慎《丹铅录》、胡侍《珍珠船》、田艺衡《留青日札》皆有是说。核其文义,一一吻合。《隋唐志》皆云子贡作,非其实矣。其文纵横曼衍,与《吴越春秋》相类,而博丽奥衍则过之。中如《计倪内经》军气之类,多杂术数家言。皆汉人专门之学,非后来所能依托也。此本与《吴越春秋》皆大德丙午绍兴路所刊。卷末一跋,诸本所无。惟申明复仇之义,不著姓名。详其词意,或南宋人所题耶?郑明选《秕言》引《文选·七命》注引《越绝书》:大翼一艘十丈,中翼九丈六尺,小翼九丈。又称王鏊《震泽长语》引《越绝书》,风起震方云云。谓今本皆无此语,疑更有全书,惜未之见。案《崇文总目》称《越绝书》旧有内记八、外传十七。今文题阙舛,裁二十篇。是此书在北宋之初已佚五篇。选注所引盖佚篇之文,王鏊所称亦他书所引佚篇之文。以为此本之外更有全书,则明选误矣。别有《续越绝书》二卷,上卷曰《内传本事》、《吴内传》、《德序记》、《子游内经外传》、《越绝后语》、《西施郑旦外传》;下卷曰《越外传》、《杂事别传》、《变越上别传》、《变越下经》、《内雅琴考序传后记》。朱彝尊《经义考》谓为钱伪撰,诡云得之石匣中。与彝尊友善,所言当实。今未见传本,其伪妄亦不待辨。以其续此书而作,又即托於撰此书之人,恐其幸而或传,久且乱真。又恐其或不能传,而好异者耳闻其说,且疑此书之真有续编,故附订其伪於此,释来者之惑焉。

\mainmatter

% 增加空行
~\\

% 增加字间间隔,适用于三字经、诗文等。
 \qquad  

\part{}

越绝外传本事第一

问曰:“何谓越绝?”“越者,国之氏也。”“何以言之?”“按春秋序齐鲁,皆以国为氏姓,是以明之。绝者,绝也。谓句践时也。当是之时,齐将伐鲁,孔子耻之,故子贡说齐以安鲁。子贡一出,乱齐,破吴,兴晋,疆越。其后贤者辩士,见夫子作春秋而略吴越,又见子贡与圣人相去不远,唇之与齿,表之与里,盖要其意,览史记而述其事也。”

问曰:“何不称越经书记,而言绝乎?”曰:“不也。绝者,绝也。句践之时,天子微弱,诸侯皆叛。于是句践抑疆扶弱,绝恶反之于善,取舍以道,沛归于宋,浮陵以付楚,临沂、开阳,复之于鲁。中国侵伐,因斯衰止。以其诚在于内,威发于外,越专其功,故曰越绝。故作此者,贵其内能自约,外能绝人也。贤者所述,不可断绝,故不为记明矣。”

问曰:“桓公九合诸侯,一匡天下,任用贤者,诛服疆楚,何不言齐绝乎?”曰:“桓公,中国。兵疆霸世之后,威凌诸侯,服疆楚,此正宜耳。夫越王句践,东垂海滨,夷狄文身,躬而自苦,任用贤臣,转死为生,以败为成。越伐疆吴,尊事周室,行霸琅邪,躬自省约,率道诸侯,贵其始微,终能以霸,故与越专其功而有之也。”

问曰:“然越专其功而有之,何不第一,而卒本吴太伯为?”曰:“小越而大吴。”“小越大吴奈何?”曰:“吴有子胥之教,霸世甚久。北陵齐、楚,诸侯莫敢叛者,乘,薛、许、邾、娄、莒旁毂趋走,越王句践属刍莝养马,诸侯从之,若果中之李。反邦七年,焦思苦身,克己自责,任用贤人。越伐疆吴,行霸诸侯,故不使越第一者,欲以贬大吴,显弱越之功也。”

问曰:“吴亡而越兴,在天与?在人乎?”“皆人也。夫差失道,越亦贤矣。湿易雨,饥易助。”曰:“何以知独在人乎?”“子贡与夫子坐,告夫子曰:‘太宰死。’夫子曰:‘不死也。’如是者再。子贡再拜而问:‘何以知之?’夫子曰:‘天生宰嚭者,欲以亡吴。吴今未亡,宰何病乎?’后人来言不死。圣人不妄言,是以明知越霸矣。”“何以言之?”曰:“种见蠡之时,相与谋道:‘东南有霸兆,不如往仕。’相要东游,入越而止。贤者不妄言,以是知之焉。”

问曰:“越绝谁所作?”“吴越贤者所作也。当此之时,见夫子删书作春秋,定王制,贤者嗟叹,决意览史记,成就其事。”

问曰:“作事欲以自着,今但言贤者,不言姓字何?”曰:“是人有大雅之才,直道一国之事,不见姓名,小之辞也。或以为子贡所作,当挟四方,不当独在吴越。其在吴越,亦有因矣。此时子贡为鲁使,或至齐,或至吴。其后道事以吴越为喻,国人承述,故直在吴越也。当是之时,有圣人教授六艺,删定五经,七十二子,养徒三千,讲习学问鲁之阙门。越绝,小艺之文,固不能布于四方,焉有诵述先圣贤者,所作未足自称,载列姓名,直斥以身者也?一说盖是子胥所作也。夫人情,泰而不作,穷则怨恨,怨恨则作,犹诗人失职怨恨,忧嗟作诗也。子胥怀忠,不忍君沈惑于谗,社稷之倾。绝命危邦,不顾长生,切切争谏,终不见听。忧至患致,怨恨作文。不侵不差,抽引本末。明己无过,终不遗力。诚能极智,不足以身当之,嫌于求誉,是以不着姓名,直斥以身者也。后人述而说之,仍稍成中外篇焉。”

问曰:“或经或传,或内或外,何谓?”曰:“经者,论其事,传者,道其意,外者,非一人所作,颇相覆载。或非其事,引类以讬意。说之者见夫子删诗、书,就经易,亦知小艺之复重。又各辩士所述,不可断绝。小道不通,偏有所期。明说者不专,故删定复重,以为中外篇。”

越绝荆平王内传第二

昔者,荆平王有臣伍子奢。奢得罪于王,且杀之,其二子出走,伍子尚奔吴,伍子胥奔郑。王召奢而问之,曰:“若召子,孰来也?”子奢对曰:“王问臣,对而畏死,不对不知子之心者。尚为人也,仁且智,来之必入,胥为人也,勇且智,来必不入。胥且奔吴邦,君王必早闭而晏开,胥将使边境有大忧。”

于是王即使使者召子尚于吴,曰:“子父有罪,子入,则免之,不入,则杀之。”子胥闻之,使人告子尚于吴:“吾闻荆平王召子,子必毋入。胥闻之,入者穷,出者报仇。入者皆死,是不智也。死而不报父之仇,是非勇也。”子尚对曰:“入则免父之死,不入则不仁。爱身之死,绝父之望,贤士不为也。意不同,谋不合,子其居,尚请入。”

荆平王复使使者召子胥于郑,曰:“子入,则免父死,不入,则杀之。”子胥介胄彀弓,出见使者,谢曰:“介胄之士,固不拜矣。请有道于使者:王以奢为无罪,赦而蓄之,其子又何适乎?”使者还报荆平王,王知子胥不入也,杀子奢而并杀子尚。

子胥闻之,即从横岭上大山,北望齐晋,谓其舍人曰:“去,此邦堂堂,被山带河,其民重移。”于是乃南奔吴。至江上,见渔者,曰:“来,渡我。”渔者知其非常人也,欲往渡之,恐人知之,歌而往过之,曰:“日昭昭,侵以施,与子期甫芦之碕。”子胥即从渔者之芦碕。日入,渔者复歌往,曰:“心中目施,子可渡河,何为不出?”船到即载,入船而伏。半江,而仰谓渔者曰:“子之姓为谁?还,得报子之厚德。”渔者曰:“纵荆邦之贼者,我也,报荆邦之仇者,子也。两而不仁,何相问姓名为?”子胥即解其剑,以与渔者,曰:“吾先人之剑,直百金,请以与子也。”渔者曰:“吾闻荆平王有令曰:‘得伍子胥者,购之千金。’今吾不欲得荆平王之千金,何以百金之剑为?”渔者渡于于斧之津,乃发其箪饭,清其壶浆而食,曰:“亟食而去,毋令追者及子也。”子胥曰:“诺。”子胥食已而去,顾谓渔者曰:“掩尔壶浆,无令之露。”渔者曰:“诺。”子胥行,即覆船,挟匕首自刎而死江水之中,明无泄也。

子胥遂行。至溧阳界中,见一女子击絮于濑水之中,子胥曰:“岂可得讬食乎?”女子曰:“诺。”即发箪饭,清其壶浆而食之。子胥食已而去,谓女子曰:“掩尔壶浆,毋令之露。”女子曰:“诺。”子胥行五步,还顾女子,自纵于濑水之中而死。

子胥遂行。至吴。徒跣被发,乞于吴市。三日,市正疑之,而道于阖庐曰:“市中有非常人,徒跣被发,乞于吴市三日矣。”阖庐曰:“吾闻荆平王杀其臣伍子奢而非其罪,其子子胥勇且智,彼必经诸侯之邦可以报其父仇者。”王者使召子胥。入,吴王下阶迎而唁,数之曰:“吾知子非恒人也,何素穷如此?”子胥跪而垂泣曰:“胥父无罪而平王杀之,而并其子尚。子胥遯逃出走,唯大王可以归骸骨者,惟大王哀之。”吴王曰:“诺。”上殿与语,三日三夜,语无复者。王乃号令邦中:“无贵贱长少,有不听子胥之教者,犹不听寡人也,罪至死,不赦。”

子胥居吴三年,大得吴众。阖庐将为之报仇,子胥曰:“不可。臣闻诸侯不为匹夫兴师。”于是止。其后荆将伐蔡,子胥言之阖庐,即使子胥救蔡而伐荆。十五战,十五胜。荆平王已死,子胥将卒六千,操鞭捶笞平王之墓而数之曰:“昔者吾先人无罪而子杀之,今此报子也。”

后,子昭王、臣司马子期、令尹子西归,相与计谋:“子胥不死,又不入荆,邦犹未得安,为之奈何?莫若求之而与之同邦乎?”昭王乃使使者报子胥于吴,曰:“昔者吾先人杀子之父,而非其罪也。寡人尚少,未有所识也。今子大夫报寡人也特甚,然寡人亦不敢怨子。今子大夫何不来归子故坟墓丘冢为?我邦虽小,与子同有之,民虽少,与子同使之。”子胥曰:“以此为名,名即章,以此为利,利即重矣。前为父报仇,后求其利,贤者不为也。父已死,子食其禄,非父之义也。”使者遂还,乃报荆昭王曰:“子胥不入荆邦,明矣。”

\part{}

越绝外传记吴地传第三

昔者,吴之先君太伯,周之世,武王封太伯于吴,到夫差,计二十六世,且千岁。阖庐之时,大霸,筑吴越城。城中有小城二。徙治胥山。后二世而至夫差,立二十三年,越王句践灭之。

阖庐宫,在高平里。

射台二,一在华池昌里,一在安阳里。

南城宫,在长乐里,东到春申君府。

秋冬治城中,春夏治姑胥之台。旦食于纽山,昼游于胥母,射于鸥陂,驰于游台,兴乐石城,走犬长洲。

吴王大霸,楚昭王、孔子时也。

吴大城,周四十七里二百一十步二尺。陆门八,其二有楼。水门八。南面十里四十二步五尺,西面七里百一十二步三尺,北面八里二百二十六步三尺,东面十一里七十九步一尺。阖庐所造也。吴郭周六十八里六十步。

吴小城,周十二里。其下广二丈七尺,高四丈七尺。门三,皆有楼,其二增水门二,其一有楼,一增柴路。

东宫周一里二百七十步。路西宫在长秋,周一里二十六步。秦始皇帝十一年,守宫者照燕失火,烧之。

伍子胥城,周九里二百七十步。

小城东西从武里,面从小城北。

邑中径从阊门到娄门,九里七十二步,陆道广二十三步,平门到蛇门,十里七十五步,陆道广三十三步。水道广二十八步。

吴古故陆道,出胥门,奏出土山,度灌邑,奏高颈,过犹山,奏太湖,随北顾以西,度阳下溪,过历山阳、龙尾西大决,通安湖。

吴古故水道,出平门,上郭池,入渎,出巢湖,上历地,过梅亭,入杨湖,出渔浦,入大江,奏广陵。

吴古故从由拳辟塞,度会夷,奏山阴。辟塞者,吴备候塞也。

居东城者,阖庐所游城也,去县二十里。

柴辟亭到语儿就李,吴侵以为战地。

百尺渎,奏江,吴以达粮。

千里庐虚者,阖庐以铸干将剑。欧冶僮女三百人。去县二里,南达江。

阊门外高颈山东桓石人,古者名“石公”,去县二十里。

阊门外郭中冢者,阖庐冰室也。

阖庐冢,在阊门外,名虎丘。下池广六十步,水深丈五尺。铜椁三重。澒池六尺。玉凫之流,扁诸之剑三千,方圆之口三千。时耗、鱼肠之剑在焉。十万人筑治之。取土临湖口。葬三日而白虎居上,故号为虎丘。

虎丘北莫格冢,古贤者避世冢,去县二十里。

被奏冢,邓大冢是也,去县四十里。

阖庐子女冢,在阊门外道北。下方池广四十八步,水深二丈五尺。池广六十步,水深丈五寸。隧出庙路以南,通姑胥门。并周六里。舞鹤吴市,杀生以送死。

余杭城者,襄王时神女所葬也。神多灵。

巫门外麋湖西城,越宋王城也。时与摇城王周宋君战于语招,杀周宋君。毋头骑归,至武里死亡,葬武里南城。午日死也。

巫门外冢者,阖庐冰室也。

巫门外大冢,吴王客齐孙武冢也,去县十里。善为兵法。

蛇门外塘波洋中世子塘者,故曰王世子造以为田。塘去县二十五里。

洋中塘,去县二十六里。

蛇门外大丘,吴王不审名冢也,去县十五里。

筑塘北山者,吴王不审名冢也,去县二十里。

巫门外欐溪椟中连乡大丘者,吴故神巫所葬也,去县十五里。

娄门外马亭溪上复城者,故越王余复君所治也,去县八十里。是时烈王归于越,所载襄王之后,不可继述。其事书之马亭溪。

娄门外鸿城者,故越王城也,去县百五十里。

娄门外鸡陂墟,故吴王所畜鸡处,使李保养之,去县二十里。

胥门外有九曲路,阖庐造以游姑胥之台,以望太湖中,窥百姓。去县三十里。

齐门,阖庐伐齐,大克,取齐王女为质子,为造齐门,置于水海虚。其台在车道左、水海右。去县七十里。齐女思其国死,葬虞西山。

吴北野禺栎东所舍大□者,吴王田也,去县八十里。

吴西野鹿陂者,吴王田也。今分为耦渎,胥卑虚,去县二十里。

吴北野胥主□者,吴王女胥主田也,去县八十里。

麋湖城者,阖庐所置麋也,去县五十里。

欐溪城者,阖庐所置船宫也。阖庐所造。

娄门外力士者,阖庐所造,以备外越。

巫欐城者,阖庐所置诸侯远客离城也,去县十五里。

由钟穷隆山者,古赤松子所取赤石脂也,去县二十里。子胥死,民思祭之。

莋碓山,故为鹤阜山,禹游天下,引湖中柯山置之鹤阜,更名莋碓。

放山者,在莋碓山南。以取长之莋碓山下,故有乡名莋邑。吴王恶其名,内郭中,名通陵乡。

莋碓山南有大石,古者名为“坠星”,去县二十里。

抚侯山者,故阖庐治以诸侯冢次,去县二十里。

吴东徐亭东西南北通溪者,越荆王所置,与麋湖相通也。

马安溪上干城者,越干王之城也,去县七十里。

巫门外冤山大冢,故越王王史冢也,去县二十里。

摇城者,吴王子居焉,后越摇王居之。稻田三百顷,在邑东南,肥饶,水绝。去县五十里。

胥女大冢,吴王不审名冢也,去县四十五里。

蒲姑大冢,吴王不审名冢也,去县三十里。

石城者,吴王阖庐所置美人离城也,去县七十里。

通江南陵,摇越所凿,以伐上舍君。去县五十里。

娄东十里坑者,古名长人坑,从海上来。去县十里。

海盐县,始为武原乡。

娄北武城,阖庐所以候外越也,去县三十里。今为乡也。

宿甲者,吴宿兵候外越也,去县百里,其东大冢,摇王冢也。

乌程、余杭、黝、歙、无湖、石城县以南,皆故大越徙民也。秦始皇帝刻石徙之。

乌伤县常山,古人所采药也,高且神。

齐乡,周十里二百一十步,其城六里三十步,墙高丈二尺,百七十步,竹格门三,其二有屋。

虞山者,巫咸所出也。虞故神出奇怪。去县百五里。

母陵道,阳朔三年太守周君造陵道语昭。郭周十里百一十步,墙高丈二尺。陵门四,皆有屋。水门二。

无锡城,周二里十九步,高二丈七尺,门一楼四。其郭周十一里百二十八步,墙一丈七尺,门皆有屋。

无锡历山,春申君时盛祠以牛,立无锡塘。去吴百二十里。

无锡湖者,春申君治以为陂,凿语昭渎以东到大田。田名胥卑。凿胥卑下以南注大湖,以写西野。去县三十五里。

无锡西龙尾陵道者,春申君初封吴所造也。属于无锡县。以奏吴北野胥主□。

曲阿,故为云阳县。

毗陵,故为延陵,吴季子所居。

毗陵县南城,故古淹君地也。东南大冢,淹君子女冢也。去县十八里。吴所葬。

毗陵上湖中冢者,延陵季子冢也,去县七十里。上湖通上洲。季子冢古名延陵墟。

蒸山南面夏驾大冢者,越王不审名冢,去县三十五里。

秦余杭山者,越王栖吴夫差山也,去县五十里。山有湖水,近太湖。

夫差冢,在犹亭西卑犹位。越王候干戈人一累土以葬之。近太湖七里。

三台者,太宰嚭、逢同妻子死所在也,去县十七里。

太湖,周三万六千顷。其千顷,乌程也。去县五十里。

无锡湖,周万五千顷。其一千三顷,毗陵上湖也。去县五十里。一名射贵湖。

尸湖,周二千二百顷,去县百七十里。

小湖,周千三百二十顷,去县百里。

耆湖,周六万五千顷,去县百二十里。

乘湖,周五百顷,去县五里。

犹湖,周三百二十顷,去县十七里。

语昭湖,周二百八十顷,去县五十里。

作湖,周百八十顷,聚鱼多物,去县五十五里。

昆湖,周七十六顷一亩,去县一百七十五里。一名隐湖。

湖王湖,当问之。

丹湖,当问之。

吴古故祠江汉于棠浦东,江南为方墙,以利朝夕水。古太伯君吴,到阖庐时绝。

胥女南小蜀山,春申君客卫公子冢也,去县三十五里。

白石山,故为胥女山,春申君初封吴,过,更名为白石。去县四十里。

今太守舍者,春申君所造,后殿屋以为桃夏宫。

今宫者,春申君子假君宫也。前殿屋盖地东西十七丈五尺,南北十五丈七尺。堂高四丈,十霤高丈八尺。殿屋盖地东西十五丈,南北十丈二尺七寸。户霤高丈二尺。库东乡屋南北四十丈八尺,上下户各二。南乡屋东西六十四丈四尺,上户四,下户三。西乡屋南北四十二丈九尺,上户三,下户二。凡百四十九丈一尺。檐高五丈二尺。霤高二丈九尺。周一里二百四十一步。春申君所造。

吴两仓,春申君所造。西仓名曰均输,东仓周一里八步。后烧。更始五年,太守李君治东仓为属县屋,不成。

吴市者,春申君所造,阙两城以为市。在湖里。

吴诸里大闬,春申君所造。

吴狱庭,周三里,春申君时造。

土山者,春申君时治以为贵人冢次,去县十六里。

楚门,春申君所造。楚人从之,故为楚门。

路丘大冢,春申君客冢。不立,以道终之。去县十里。

春申君,楚考烈王相也。烈王死,幽王立,封春申君于吴。三年,幽王征春申为楚令尹,春申君自使其子为假君治吴。十一年,幽王征假君与春申君,并杀之。二君治吴凡十四年。后十六年,秦始皇并楚,百越叛去,更名大越为山阴也。春申君姓黄,名歇。

巫门外罘罳者,春申君去吴,假君所思处也。去县二十三里。

寿春东凫陵亢者,古诸侯王所葬也。楚威王与越王无疆并。威王后烈王,子幽王,后怀王也。怀王子顷襄王也,秦始皇灭之。秦始皇造道陵南,可通陵道,到由拳塞,同起马塘,湛以为陂,治陵水道到钱唐,越地,通浙江。秦始皇发会稽适戍卒,治通陵高以南陵道,县相属。

秦始皇帝三十七年,坏诸侯郡县城。

太守府大殿者,秦始皇刻石所起也。到更始元年,太守许时烧。六年十二月乙卯凿官池,东西十五丈七尺,南北三十丈。

汉高帝封有功,刘贾为荆王,并有吴。贾筑吴市西城,名曰定错城,属小城,北到平门,丁将军筑治之。十一年,淮南王反,杀刘贾。后十年,高皇帝更封兄子濞为吴王,治广陵,并有吴。立二十一年,东渡之吴,十日还去。立四十二年,反。西到陈留县,还奔丹阳,从东欧。越王弟夷乌将军杀濞。东欧王为彭泽王,夷乌将军今为平都王。濞父字为仲。

匠门外信士里东广平地者,吴王濞时宗庙也。太公、高祖在西,孝文在东。去县五里。永光四年,孝元帝时,贡大夫请罢之。

桑里东今舍西者,故吴所畜牛、羊、豕、鸡也,名为牛宫。今以为园。

汉文帝前九年,会稽并故鄣郡。太守治故鄣,都尉治山阴。前十六年,太守治吴郡,都尉治钱唐。

汉孝景帝五年五月,会稽属汉。属汉者,始并事也。汉孝武帝元封元年,阳都侯归义,置由钟。由钟初立,去县五十里。

汉孝武元封二年,故鄣以为丹阳郡。

天汉五年四月,钱唐浙江岑石不见,到七年,岑石复见。

越王句践徙琅邪,凡二百四十年,楚考烈王并越于琅邪。后四十余年,秦并楚。复四十年,汉并秦。到今二百四十二年。句践徙琅邪到建武二十八年,凡五百六十七年。

\part{}

越绝吴内传第四

吴何以称人乎?夷狄之也。忧中邦奈何乎?伍子胥父诛于楚,子胥挟弓,身干阖庐。阖庐曰:“士之甚,勇之甚。”将为之报仇。子胥曰:“不可,诸侯不为匹夫报仇。臣闻事君犹事父也,亏君之行,报父之仇,不可。”于是止。

蔡昭公南朝楚,被羔裘,囊瓦求之,昭公不与。即拘昭公南郢,三年然后归之。昭公去,至河,用事曰:“天下谁能伐楚乎?寡人愿为前列!”楚闻之,使囊瓦兴师伐蔡。昭公闻子胥在吴,请救蔡。子胥于是报阖庐曰:“蔡公南朝,被羔裘,囊瓦求之,蔡公不与,拘蔡公三年,然后归之。蔡公至河,曰:‘天下谁能伐楚者乎?寡人愿为前列。’楚闻之,使囊瓦兴师伐蔡。蔡非有罪,楚为无道。君若有忧中国之事意者,时可矣。”阖庐于是使子胥兴师,救蔡而伐楚。楚王已死,子胥将卒六千人,操鞭笞平王之坟,曰:“昔者吾先君无罪,而子杀之,今此以报子也!”君舍君室,大夫舍大夫室,盖有妻楚王母者。

囊瓦者何?楚之相也。郢者何?楚王治处也。吴师何以称人?吴者,夷狄也,而救中邦,称人,贱之也。

越王句践欲伐吴王阖庐,范蠡谏曰:“不可。臣闻之,天贵持盈,持盈者,言不失阴阳、日月、星辰之纲纪。地贵定倾,定倾者,言地之长生,丘陵平均,无不得宜。故曰地贵定倾。人贵节事,节事者,言王者已下,公卿大夫,当调阴阳,和顺天下。事来应之,物来知之,天下莫不尽其忠信,从其政教,谓之节事。节事者,至事之要也。天道盈而不溢,盛而不骄者,言天生万物,以养天下。蠉飞蠕动,各得其性。春生夏长,秋收冬藏,不失其常。故曰天道盈而不溢,盛而不骄者也。地道施而不德,劳而不矜其功者也,言地生长五谷,持养万物,功盈德博,是所施而不德,劳而不矜其功者矣。言天地之施,大而不有功者也。人道不逆四时者,言王者以下,至于庶人,皆当和阴阳四时之变,顺之者有福,逆之者有殃。故曰人道不逆四时之谓也。因惛视动者,言存亡吉凶之应,善恶之叙,必有渐也。天道未作,不先为客者。”

范蠡值吴伍子胥教化,天下从之,未有死亡之失,故以天道未作,不先为客。言客者,去其国,入人国。地兆未发,不先动众,言王者以下,至于庶人,非暮春中夏之时,不可以种五谷、兴土利,国家不见死亡之失,不可伐也。故地兆未发,不先动众,此之谓也。

吴人败于就李,吴之战地。败者,言越之伐吴,未战,吴阖庐卒,败而去也。卒者,阖庐死也。天子称崩,诸侯称薨,大夫称卒,士称不禄。阖庐,诸侯也,不称薨而称卒者,何也?当此之时,上无明天子,下无贤方伯,诸侯力政,疆者为君。南夷与北狄交争,中国不绝如线矣。臣弑君,子弑父,天下莫能禁止。于是孔子作春秋,方据鲁以王。故诸侯死皆称卒,不称薨,避鲁之谥也。

晋公子重耳之时,天子微弱,诸侯力政,疆者为君。文公为所侵暴,失邦,奔于翟。三月得反国政,敬贤明法,率诸侯朝天子,于是诸侯皆从,天子乃尊。此所谓晋公子重耳反国定天下。

齐公子小白,亦反齐国而匡天下者。齐大夫无知,弑其君诸儿。其子二人出奔。公子纠奔鲁。鲁者,公子纠母之邦。小白奔莒,莒者,小白母之邦也。齐大臣鲍叔牙为报仇,杀无知,故兴师之鲁,聘公子纠以为君。鲁庄公不与。庄公,鲁君也,曰:“使齐以国事鲁,我与汝君。不以国事鲁,我不与汝君。”于是鲍叔牙还师之莒,取小白,立为齐君。小白反国,用管仲,九合诸侯,一匡天下,故为桓公。此之谓也。

尧有不慈之名。尧太子丹朱倨骄,怀禽兽之心,尧知不可用,退丹朱而以天下传舜。此之谓尧有不慈之名。

舜有不孝之行。舜亲父假母,母常杀舜。舜去,耕历山。三年大熟,身自外养,父母皆饥。舜父顽,母嚚,兄狂,弟敖。舜求为变心易志。舜为瞽瞍子也,瞽瞍欲杀舜,未尝可得。呼而使之,未尝不在侧。此舜有不孝之行。舜用其仇而王天下者,言舜父瞽瞍,用其后妻,常欲杀舜,舜不为失孝行,天下称之。尧闻其贤,遂以天下传之。此为王天下。仇者,舜后母也。

桓公召其贼而霸诸侯者,管仲臣于桓公兄公子纠,纠与桓争国,管仲张弓射桓公,中其带钩,桓公受之,赦其大罪,立为齐相。天下莫不向服慕义。是谓召其贼霸诸侯也。

夏启献牺于益。启者,禹之子。益与禹臣于舜,舜传之禹,荐益而封之百里。禹崩,启立,晓知王事,达于君臣之义。益死之后,启岁善牺牲以祠之。经曰:“夏启善牺于益。”此之谓也。

汤献牛荆之伯。之伯者,荆州之君也。汤行仁义,敬鬼神,天下皆一心归之。当是时,荆伯未从也,汤于是乃饰牺牛以事。荆伯乃媿然曰:“失事圣人礼。”乃委其诚心。此谓汤献牛荆之伯也。

越王句践反国六年,皆得士民之众,而欲伐吴。于是乃使之维甲。维甲者,治甲系断。修内矛赤鸡稽繇者也,越人谓“人铩”也。方舟航买仪尘者,越人往如江也。治须虑者,越人谓船为“须虑”。亟怒纷纷者,怒貌也,怒至。士击高文者,跃勇士也。习之于夷。夷,海也。宿之于莱。莱,野也。致之于单。单者,堵也。

舜之时,鲧不从令。尧遭帝喾之后乱,洪水滔天,尧使鲧治之,九年弗能治。尧七十年而得舜,舜明知人情,审于地形,知鲧不能治,数谏不去,尧殛之羽山。此之谓舜之时,鲧不从令也。

殷汤遭夏桀无道,残贼天下,于是汤用伊尹,行至圣之心。见桀无道虐行,故伐夏放桀,而王道兴跃。革乱补弊,移风易俗,改制作新,海内毕贡,天下承风。汤以文圣,此之谓也。

文王以务争者,纣为天下,残贼奢佚,不顾邦政。文王百里,见纣无道,诛杀无刑,赏赐不当,文王以圣事纣,天下皆尽诚知其贤圣,从之。此谓文王以务争也。纣以恶刑争,文王行至圣,以仁义争,此之谓也。

武王以礼信。文王死九年,天下八百诸侯,皆一旦会于孟津之上。不言同辞,不呼自来,尽知武王忠信,欲从武王,与之伐纣。当是时,比干、箕子、微子尚在,武王贤之,未敢伐也,还诸侯。归二年,纣贼比干,囚箕子,微子去之。刳妊妇,残朝涉。武王见贤臣已亡,乃朝天下,兴师伐纣,杀之。武王未下车,封比干之墓,发太仓之粟,以赡天下,封微子于宋。此武王以礼信也。

周公以盛德。武王封周公,使傅相成王。成王少,周公臣事之。当是之时,赏赐不加于无功,刑罚不加于无罪。天下家给人足,禾麦茂美。使人以时,说之以礼。上顺天地,泽及夷狄。于是管叔、蔡叔不知周公而谗之成王。周公乃辞位,出,巡狩于边一年。天暴风雨,日夜不休,五谷不生,树木尽偃。成王大恐,乃发金縢之柜,察周公之册,知周公有盛德。王乃夜迎周公,流涕而行。周公反国,天应之福。五谷皆生,树木皆起,天下皆实。此周公之盛德也。


\part{}

越绝计倪内经第五

昔者,越王句践既得反国,欲阴谋吴。乃召计倪而问焉,曰:“吾欲伐吴,恐弗能取。山林幽冥,不知利害所在。西则迫江,东则薄海,水属苍天,下不知所止。交错相过,波涛浚流,沈而复起,因复相还。浩浩之水,朝夕既有时,动作若惊骇,声音若雷霆。波涛援而起,船失不能救,未知命之所维。念楼船之苦,涕泣不可止。非不欲为也,时返不知所在,谋不成而息,恐为天下咎。以敌攻敌,未知谁负。大邦既已备,小邑既已保,五谷既已收。野无积庾,廪粮则不属,无所安取?恐津梁之不通,劳军纡吾粮道。吾闻先生明于时交,察于道理,恐动而无功,故问其道。”计倪对曰:“是固不可。兴师者必先蓄积食、钱、布、帛。不先蓄积,士卒数饥。饥则易伤,重迟不可战。战则耳目不聪明,耳不能听,视不能见,什部之不能使,退之不能解,进之不能行。饥馑不可以动,神气去而万里。伏弩而乳,郅头而皇皇。疆弩不彀,发不能当。旁军见弱,走之如犬逐羊。靡从部分,伏地而死,前顿后僵。与人同时而战,独受天之殃。未必天之罪也,亦在其将。王兴师以年数,恐一旦而亡。失邦无明,筋骨为野。”越王曰:“善。请问其方。吾闻先生明于治岁,万物尽长。欲闻其治术,可以为教常。子明以告我,寡人弗敢忘。”

计倪对曰:“人之生无几,必先忧积蓄,以备妖祥。凡人生或老或弱,或疆或怯,不早备生,不能相葬。王其审之。必先省赋敛,劝农桑。饥馑在问,或水或塘。因熟积以备四方。师出无时,未知所当。应变而动,随物常羊。卒然有师,彼日以弱,我日以疆。得世之和,擅世之阳,王无忽忘。慎无如会稽之饥,不可再更。王其审之。尝言息货,王不听,臣故退而不言,处于吴、楚、越之间,以鱼三邦之利,乃知天下之易反也。臣闻君自耕,夫人自织,此竭于庸力,而不断时与智也。时断则循,智断则备。知此二者,形于体万物之情,短长逆顺,可观而已。臣闻炎帝有天下,以传黄帝。黄帝于是上事天,下治地。故少昊治西方,蚩尤佐之,使主金。玄冥治北方,白辨佐之,使主水。太皞治东方,袁何佐之,使主木。祝融治南方,仆程佐之,使主火。后土治中央,后稷佐之,使主土。并有五方,以为纲纪。是以易地而辅,万物之常。王审用臣之议,大则可以王,小则可以霸,于何有哉?”

越王曰:“请问其要。”计倪对曰:“太阴三岁处金则穣,三岁处水则毁,三岁处木则康,三岁处火则旱。故散有时积,籴有时领,则决万物不过三岁而发矣。以智论之,以决断之,以道佐之。断长续短,一岁再倍,其次一倍,其次而反。水则资车,旱则资舟,物之理也。天下六岁一穣,六岁一康,凡十二岁一饥,是以民相离也。故圣人早知天地之反,为之预备。故汤之时,比七年旱而民不饥,禹之时,比九年水而民不流。其主能通习源流,以任贤使能,则转毂乎千里外,货可来也。不习,则百里之内,不可致也。人主所求,其价十倍,其所择者,则无价矣。夫人主利源流,非必身为之也。视民所不足,及其有余,为之命以利之,而来诸侯。守法度,任贤使能,偿其成事,传其验而已。如此,则邦富兵强而不衰矣。群臣无空恭之礼、淫佚之行,务有于道术。不习源流,又不任贤使能,谏者则诛,则邦贫兵弱。刑繁,则群臣多空恭之礼、淫佚之行矣。夫谀者反有德,忠者反有刑,去刑就德,人之情也,邦贫兵弱致乱,虽有圣臣,亦不谏也,务在谀主而已矣。今夫万民有明父母,亦如邦有明主。父母利源流,明其法术,以任贤子,徼成其事而已,则家富而不衰矣。不能利源流,又不任贤子,贤子有谏者憎之,如此者,不习于道术也。愈信其意而行其言,后虽有败,不自过也。夫父子之为亲也,非得不谏。谏而不听,家贫致乱,虽有圣子,亦不治也,务在于谀之而已。父子不和,兄弟不调,虽欲富也,必贫而日衰。”

越王曰:“善。子何年少,于物之长也?”计倪对曰:“人固不同。慧种生圣,痴种生狂。桂实生桂,桐实生桐。先生者未必能知,后生者未必不能明。是故圣主置臣不以少长,有道者进,无道者退。愚者日以退,圣者日以长,人主无私,赏者有功。”

越王曰:“善。论事若是,其审也。物有妖祥乎?”计倪对曰:“有。阴阳万物,各有纪纲。日月、星辰、刑德,变为吉凶,金木水火土更胜,月朔更建,莫主其常。顺之有德,逆之有殃。是故圣人能明其刑而处其乡,从其德而避其衡。凡举百事,必顺天地四时,参以阴阳。用之不审,举事有殃。人生不如卧之顷也,欲变天地之常,数发无道,故贫而命不长。是圣人并苞而阴行之,以感愚夫。众人容容,尽欲富贵,莫知其乡。”越王曰:“善,请问其方。”计倪对曰:“从寅至未,阳也。太阴在阳,岁德在阴,岁美在是。圣人动而应之,制其收发。常以太阴在阴而发,阴且尽之岁,亟卖六畜货财,以益收五谷,以应阳之至也。阳且尽之岁,亟发籴,以收田宅、牛马、积敛货财,聚棺木,以应阴之至也。此皆十倍者也。其次五倍。天有时而散,是故圣人反其刑,顺其衡,收聚而不散。”

越王曰:“善。今岁比熟,尚有贫乞者,何也?”计倪对曰:“是故不等,犹同母之人,异父之子,动作不同术,贫富故不等。如此者,积负于人,不能救其前后。志意侵下,作务日给,非有道术,又无上赐,贫乏故长久。”越王曰:“善。大夫佚同、若成,尝与孤议于会稽石室,孤非其言也。今大夫言独与孤比,请遂受教焉。”计倪曰:“籴石二十则伤农,九十则病末。农伤则草木不辟,末病则货不出。故籴高不过八十,下不过三十,农末俱利矣。故古之治邦者本之,货物官市开而至。”越王曰:“善。”计倪乃传其教而图之,曰:“审金木水火,别阴阳之明,用此不患无功。”越王曰:“善。从今以来,传之后世以为教。”

乃着其法,治牧江南,七年而禽吴也。甲货之户曰粢,为上物,贾七十。乙货之户曰黍,为中物,石六十。丙货之户曰赤豆,为下物,石五十。丁货之户曰稻粟,令为上种,石四十。成货之户曰麦,为中物,石三十。己货之户曰大豆,为下物,石二十。庚货之户曰穬,比疏食,故无贾。辛货之户曰□,比疏食,无贾。壬癸无货。

...
------------

\part{}

越绝请籴内传第六

昔者,越王句践与吴王夫差战,大败,保栖于会稽山上,乃使大夫种求行成于吴。吴许之。越王去会稽,入官于吴。三年,吴王归之。大夫种始谋曰:“昔者吴夫差不顾义而媿吾王。种观夫吴甚富而财有余,其刑繁法逆,民习于战守,莫不知也。其大臣好相伤,莫能信也。其德衰而民好负善。且夫吴王又喜安佚而不听谏,细诬而寡智,信谗谀而远士,数伤人而亟亡之,少明而不信人,希须臾之名而不顾后患。君王盍少求卜焉?”越王曰:“善。卜之道何若?”大夫种对曰:“君王卑身重礼,以素忠为信,以请籴于吴,天若弃之,吴必许诺。”

于是乃卑身重礼,以素忠为信,以请于吴。将与,申胥进谏曰:“不可。夫王与越也,接地邻境,道径通达,仇雠敌战之邦,三江环之,其民无所移,非吴有越,越必有吴。且夫君王兼利而弗取,输之粟与财,财去而凶来,凶来而民怨其上,是养寇而贫邦家也。与之不为德,不若止。且越王有智臣曰范蠡,勇而善谋,将修士卒,饰战具,以伺吾间也。胥闻之,夫越王之谋,非有忠素。请籴也,将以此试我,以此卜要君王,以求益亲,安君王之志。我君王不知省也而救之,是越之福也。”吴王曰:“我卑服越,有其社稷。句践既服为臣,为我驾舍,却行马前,诸侯莫不闻知。今以越之饥,吾与之食,我知句践必不敢。”申胥曰:“越无罪,吾君王急之,不遂绝其命,又听其言,此天之所反也。忠谏者逆,而谀谏者反亲。今狐雉之戏也,狐体卑而雉惧之。夫兽虫尚以诈相就,而况于人乎?”吴王曰:“越王句践有急,而寡人与之,其德章而未靡,句践其敢与诸侯反我乎?”申胥曰:“臣闻圣人有急,则不羞为人臣仆,而志气见人。今越王为吾浦伏约辞,服为臣下,其执礼过,吾君不知省也而已,故胜威之。臣闻狼子野心,仇雠之人,不可亲也。夫鼠忘壁,壁不忘鼠,今越人不忘吴矣!胥闻之,拂胜,则社稷固,谀胜,则社稷危。胥,先王之老臣,不忠不信,则不得为先王之老臣。君王胡不览观夫武王之伐纣也?今不出数年,鹿豕游于姑胥之台矣。”

太宰嚭从旁对曰:“武王非纣臣耶?率诸侯以杀其君,虽胜,可谓义乎?”申胥曰:“武王则已成名矣。”太宰嚭曰:“亲僇主成名,弗忍行。”申胥曰:“美恶相入,或甚美以亡,或甚恶以昌,故在前世矣。嚭何惑吾君王也?”太宰嚭曰:“申胥为人臣也,辨其君何必翙翙乎?”申胥曰:“太宰嚭面谀以求亲,乘吾君王,币帛以求,威诸侯以成富焉。今我以忠辨吾君王。譬浴婴儿,虽啼勿听,彼将有厚利。嚭无乃谀吾君王之欲,而不顾后患乎?”吴王曰:“嚭止。子无乃向寡人之欲乎?此非忠臣之道。”大宰嚭曰:“臣闻春日将至,百草从时。君王动大事,群臣竭力以佐谋。”

因逊遯之舍,使人微告申胥于吴王曰:“申胥进谏,外貌类亲,中情甚疏,类有外心。君王常亲睹其言也,胥则无父子之亲、君臣之施矣。”吴王曰:“夫申胥,先王之忠臣,天下之健士也。胥殆不然乎哉!子毋以事相差,毋以私相伤,以动寡人,此非子所能行也。”太宰嚭对曰:“臣闻父子之亲,张户别居,赠臣妾、马牛,其志加亲,若不与一钱,其志斯疏。父子之亲犹然,而况于士乎?且有知不竭,是不忠,竭而顾难,是不勇,下而令上,是无法。”

吴王乃听太宰嚭之言,果与粟。申胥逊遯之舍,叹曰:“于乎嗟!君王不图社稷之危,而听一日之说。弗对,以斥伤大臣,而王用之。不听辅弼之臣,而信谗谀容身之徒,是命短矣!以为不信。胥愿廓目于邦门,以观吴邦之大败也。越人之入,我王亲为禽哉!”

太宰嚭之交逢同,谓太宰嚭曰:“子难人申胥,请为卜焉。”因往见申胥,胥方与被离坐。申胥谓逢同曰:“子事太宰嚭,又不图邦权而惑吾君王,君王之不省也,而听众彘之言。君王忘邦,嚭之罪也。亡日不久也。”逢同出,造太宰嚭曰:“今日为子卜于申胥,胥诽谤其君不用胥,则无后。而君王觉而遇矣。”谓太宰嚭曰:“子勉事后矣。吴王之情在子乎?”太宰嚭曰:“智之所生,不在贵贱长少,此相与之道。”

逢同出见吴王,惭然有忧色。逢同垂泣不对。吴王曰:“夫嚭,我之忠臣,子为寡人游目长耳,将谁怨乎?”逢同对曰:“臣有患也。臣言而君行之,则无后忧。若君王弗行,臣言而死矣!”王曰:“子言,寡人听之。”逢同曰:“今日往见申胥,申胥与被离坐,其谋惭然,类欲有害我君王。今申胥进谏类忠,然中情至恶,内其身而心野狼。君王亲之不亲?逐之不逐?亲之乎?彼圣人也,将更然有怨心不已。逐之乎?彼贤人也,知能害我君王。杀之为乎?可杀之,亦必有以也。”吴王曰:“今图申胥,将何以?”逢同对曰:“君王兴兵伐齐,申胥必谏曰不可,王无听而伐齐,必大克,乃可图之。”

于是吴王欲伐齐。召申胥,对曰:“臣老矣,耳无闻,目无见,不可与谋。”吴王召太宰嚭而谋,嚭曰:“善哉,王兴师伐齐也。越在我犹疥癣,是无能为也。”吴王复召申胥而谋,申胥曰:“臣老矣,不可与谋。”吴王请申胥谋者三,对曰:“臣闻愚夫之言,圣主择焉。胥闻越王句践罢吴之年,宫有五灶,食不重味,省妻妾,不别所爱,妻操斗,身操概,自量而食,适饥不费,是人不死,必为国害!越王句践食不杀而餍,衣服纯素,不袀不玄,带剑以布,是人不死,必为大故。越王句践寝不安席,食不求饱,而善贵有道,是人不死,必为邦宝。越王句践衣弊而不衣新,行庆赏,不刑戮,是人不死,必成其名。越在我,犹心腹有积聚,不发则无伤,动作者有死亡。欲释齐,以越为忧。”吴王不听,果兴师伐齐,大克。还,以申胥为不忠,赐剑杀申胥,髡被离。

申胥且死,曰:“昔者桀杀关龙逢,纣杀王子比干。今吴杀臣,参桀纣而显吴邦之亡也。”王孙骆闻之,旦即不朝。王召骆而问之:“子何非寡人而旦不朝?”王孙骆对曰:“臣不敢有非,臣恐矣。”吴王曰:“子何恐?以吾杀胥为重乎?”王孙骆对曰:“君王气高,胥之下位而杀之,不与群臣谋之,臣是以恐矣。”王曰:“我非听子而杀胥,胥乃图谋寡人。”王孙骆曰:“臣闻君人者,必有敢言之臣,在上位者,必有敢言之士。如是,即虑日益进而智益生矣。胥,先王之老臣,不忠不信,不得为先王臣矣。”王意欲杀太宰嚭,王孙骆对曰:“不可。王若杀之,是杀二胥矣。”吴王近骆如故。

太宰嚭又曰:“图越,虽以我邦为事,王无忧。”王曰:“寡人属子邦,请早暮无时。”太宰嚭对曰:“臣闻驷马方驰,惊前者斩,其数必正。若是,越难成矣。”王曰:“子制之,断之。”

居三年,越兴师伐吴,至五湖。太宰嚭率徒谓之曰。谢战者五父。越王不忍,而欲许之。范蠡曰:“君王图之廊庙,失之中野,可乎?谋之七年,须臾弃之。王勿许,吴易兼也。”越王曰:“诺。”居军三月,吴自罢。太宰嚭遂亡,吴王率其有禄与贤良遯而去。越追之,至余杭山,禽夫差,杀太宰嚭。越王谓范蠡:“杀吴王。”蠡曰:“臣不敢杀主。”王曰:“刑之。”范蠡曰:“臣不敢刑主。”越王亲谓吴王曰:“昔者上苍以越赐吴,吴不受也。夫申胥无罪,杀之。进谗谀容身之徒,杀忠信之士。大过者三,以至灭亡,子知之乎?”吴王曰:“知之。”越王与之剑,使自图之。吴王乃旬日而自杀也。越王葬于卑犹之山,杀太宰嚭、逢同与其妻子。

...
------------

\part{}

越绝外传纪策考第七

昔者,吴王阖庐始得子胥之时,甘心以贤之,以为上客,曰:“圣人前知乎千岁,后睹万世。深问其国,世何昧昧,得无衰极?子其精焉,寡人垂意,听子之言。”子胥唯唯,不对。王曰:“子其明之。”子胥曰:“对而不明,恐获其咎。”王曰:“愿一言之,以试直士。夫仁者乐,知者好。诚。秉礼者探幽索隐。明告寡人。”子胥曰:“难乎言哉!邦其不长,王其图之。存无忘倾,安无忘亡。臣始入邦,伏见衰亡之证,当霸吴厄会之际,后王复空。”王曰:“何以言之?”子胥曰:“后必将失道。王食禽肉,坐而待死。佞谄之臣,将至不久。安危之兆,各有明纪。虹蜺牵牛,其异女,黄气在上,青黑于下。太岁八会,壬子数九。王相之气,自十一倍。死由无气,如法而止。太子无气,其异三世。日月光明,历南斗。吴越为邻,同俗并土,西州大江,东绝大海,两邦同城,相亚门户,忧在于斯,必将为咎。越有神山,难与为邻。愿王定之,毋泄臣言。”

吴使子胥救蔡,诛疆楚,笞平王墓,久而不去,意欲报楚。楚乃购之千金,众人莫能止之。有野人谓子胥曰:“止!吾是于斧掩壶浆之子,发箪饭于船中者。”子胥乃知是渔者也,引兵而还。故无往不复,何德不报。渔者一言,千金归焉,因是还去。

范蠡兴师战于就李,阖庐见中于飞矢,子胥还师,中媿于吴,被秦号年。至夫差复霸诸侯,兴师伐越,任用子胥。虽夫差骄奢,释越之围。子胥谏而诛。宰嚭谀心,卒以亡吴。夫差穷困,请为匹夫。范蠡不许,灭于五湖。子胥策于吴,可谓明乎!

昔者,吴王夫差兴师伐越,败兵就李。大风发狂,日夜不止。车败马失,骑士堕死。大船陵居,小船没水。吴王曰:“寡人昼卧,梦见井嬴溢大,与越争彗,越将扫我,军其凶乎?孰与师还?”此时越军大号,夫差恐越军入,惊骇。子胥曰:“王其勉之哉,越师败矣!臣闻井者,人所饮,溢者,食有余。越在南,火,吴在北,水。水制火,王何疑乎?风北来,助吴也。昔者武王伐纣时,彗星出而兴周。武王问,太公曰:‘臣闻以彗斗,倒之则胜。’胥闻灾异或吉或凶,物有相胜,此乃其证。愿大王急行,是越将凶,吴将昌也。”

子胥至直,不同邪曲。捐躯切谏,亏命为邦。爱君如躯,忧邦如家。是非不讳,直言不休。庶几正君,反以见疏。谗人间之,身且以诛。范蠡闻之,以为不通:“知数不用,知惧不去,岂谓智与?”胥闻,叹曰:“吾背楚荆,挟弓以去,义不止穷。吾前获功,后遇戮,非吾智衰,先遇阖庐,后遭夫差也。胥闻事君犹事父也,爱同也,严等也。太古以来,未尝见人君亏恩,为臣报仇也。臣获大誉,功名显着,胥知分数,终于不去。先君之功,且犹难忘,吾愿腐发弊齿,何去之有?蠡见其外,不知吾内。今虽屈冤,犹止死焉!”子贡曰:“胥执忠信,死贵于生,蠡审凶吉,去而有名,种留封侯,不知令终。二贤比德,种独不荣。”范蠡智能同均,于是之谓也。

伍子胥父子奢,为楚王大臣。为世子聘秦女,夫有色,王私悦之,欲自御焉。奢尽忠入谏,守朝不休,欲匡正之。而王拒之谏,策而问之,以奢乃害于君,绝世之臣。听谗邪之辞,系而囚之,待二子而死。尚孝而入,子胥勇而难欺。累世忠信,不遇其时,奢谏于楚,胥死于吴。诗云:“谗人罔极,交乱四国。”是之谓也。

太宰者,官号,嚭者,名也,伯州之孙。伯州为楚臣,以过诛,嚭以困奔于吴。是时吴王阖庐伐楚,悉召楚仇而近之。嚭为人览闻辩见,目达耳通,诸事无所不知。因其时自纳于吴,言伐楚之利。阖庐用之伐楚,令子胥、孙武与嚭将师入郢,有大功。还,吴王以嚭为太宰,位高权盛,专邦之枋。未久,阖庐卒,嚭见夫差内无柱石之坚,外无断割之势,谀心自纳,操独断之利,夫差终以从焉。而忠臣籥口,不得一言。嚭知往而不知来,夫差至死,悔不早诛。传曰:“见清知浊,见曲知直,人君选士,各象其德。”夫差浅短,以是与嚭专权,伍胥为之惑,是之谓也。

范蠡其始居楚也,生于宛橐,或伍户之虚。其为结僮之时,一痴一醒,时人尽以为狂。然独有圣贤之明,人莫可与语,以内视若盲,反听若聋。大夫种入其县,知有贤者,未睹所在,求邑中,不得其邑人,以为狂夫多贤士,众贱有君子,泛求之焉。得蠡而悦,乃从官属,问治之术。蠡修衣冠,有顷而出。进退揖让,君子之容。终日而语,疾陈霸王之道。志合意同,胡越相从。俱见霸兆出于东南,捐其官位,相要而往臣。小有所亏,大有所成。捐止于吴。或任子胥,二人以为胥在,无所关其辞。种曰:“今将安之?”蠡曰:“彼为我,何邦不可乎?”去吴之越,句践贤之。种躬正内,蠡治出外,内浊不烦,外无不得。臣主同心,遂霸越邦。种善图始,蠡能虑终。越承二贤,邦以安宁。始有灾变,蠡专其明,可谓贤焉,能屈能申。

\part{}

越绝外传记范伯第八

昔者,范蠡其始居楚,曰范伯。自谓衰贱,未尝世禄,故自菲薄。饮食则甘天下之无味,居则安天下之贱位。复被发佯狂,不与于世。谓大夫种曰:“三王则三皇之苖裔也,五伯乃五帝之末世也。天运历纪,千岁一至。黄帝之元,执辰破巳。霸王之气,见于地户。子胥以是挟弓干吴王。”于是要大夫种入吴。

此时冯同相与共戒之,伍子胥在,自与不能关其辞。蠡曰:“吴越二邦,同气共俗,地户之位,非吴则越。”乃入越。越王常与言尽日。大夫石买,居国有权,辩口,进曰:“炫女不贞,炫士不信。客历诸侯,渡河津,无因自致,殆非真贤。夫和氏之璧,求者不争贾,骐骥之才,不难阻险之路。□□□□之邦,历诸侯无所售,道听之徒,唯大王察之。”于是范蠡退而不言,游于楚越之间。大夫种进曰:“昔者市偷自炫于晋,晋用之而胜楚,伊尹负鼎入殷,遂佐汤取天下。有智之士,不在远近取也,谓之帝王求备者亡。易曰:‘有高世之材,必有负俗之累,有至智之明者,必破庶众之议。’成大功者不拘于俗,论大道者不合于众。唯大王察之。”

于是石买益疏。其后使将兵于外,遂为军士所杀。是时句践失众,栖于会稽之山,更用种、蠡之策,得以存。故虞舜曰:“以学乃时而行,此犹良药也。”王曰:“石买知往而不知来,其使寡人弃贤。”后遂师二人,竟以禽吴。

子贡曰:“荐一言,得及身,任一贤,得显名。”伤贤丧邦,蔽能有殃。负德忘恩,其反形伤。坏人之善毋后世,败人之成天诛行。故冤子胥僇死,由重谮子胥于吴,吴虚重之,无罪而诛。传曰:“宁失千金,毋失一人之心。”是之谓也。

越绝内传陈成恒第九

昔者,陈成恒相齐简公,欲为乱,惮齐邦鲍、晏,故徙其兵而伐鲁。鲁君忧也。孔子患之,乃召门人弟子而谓之曰:“诸侯有相伐者,尚耻之。今鲁,父母之邦也,丘墓存焉,今齐将伐之,可无一出乎?”颜渊辞出,孔子止之,子路辞出,孔子止之,子贡辞出,孔子遣之。

子贡行之齐,见陈成恒曰:“夫鲁,难伐之邦,而伐之,过矣。”陈成恒曰:“鲁之难伐,何也?”子贡曰:“其城薄以卑,池狭而浅,其君愚而不仁,其大臣伪而无用,其士民有恶闻甲兵之心,此不可与战。君不如伐吴。吴城高以厚,池广以深,甲坚以新,士选以饱,重器精弩在其中,又使明大夫守,此邦易也。君不如伐吴。”成恒忿然作色曰:“子之所难,人之所易也,子之所易,人之所难也。而以教恒,何也?”子贡对曰:“臣闻忧在内者攻疆,忧在外者攻弱。今君忧内。臣闻君三封而三不成者,大臣有不听者也。今君破鲁以广齐,堕鲁以尊臣,而君之功不与焉。是君上骄主心,下恣群臣,而求成大事,难矣。且夫上骄则犯,臣骄则争,是君上于主有却,下与大臣交争也。如此,则君立于齐,危于重卵矣。臣故曰不如伐吴。且夫吴明猛以毅而行其令,百姓习于战守,将明于法,齐之愚,为禽必矣。今君悉择四疆之中,出大臣以环之,黔首外死,大臣内空,是君上无疆臣之敌,下无黔首之士,孤立制齐者,君也。”陈恒曰:“善。虽然,吾兵已在鲁之城下,若去而之吴,大臣将有疑我之心,为之奈何?”子贡曰:“君按兵无伐,臣请见吴王,使之救鲁而伐齐,君因以兵迎之。”陈成恒许诺,乃行。

子贡南见吴王,谓吴王曰:“臣闻之,王者不绝世,而霸者不疆敌,千钧之重,加铢而移。今万乘之齐,私千乘之鲁,而与吴争疆,臣切为君恐,且夫救鲁,显名也,而伐齐,大利也。义在存亡鲁,勇在害疆齐而威申晋邦者,则王者不疑也。”吴王曰:“虽然,我常与越战,栖之会稽。夫越君,贤主也。苦身劳力,以夜接日,内饰其政,外事诸侯,必将有报我之心。子待吾伐越而还。”子贡曰:“不可。夫越之疆不下鲁,而吴之疆不过齐,君以伐越而还,即齐也亦私鲁矣。且夫伐小越而畏疆齐者不勇,见小利而忘大害者不智,两者臣无为君取焉。且臣闻之,仁人不困厄,以广其德,智者不弃时,以举其功,王者不绝世,以立其义。今君存越勿毁,亲四邻以仁,救暴困齐,威申晋邦以武,救鲁,毋绝周室,明诸侯以义。如此,则臣之所见,溢乎负海,必率九夷而朝,即王业成矣。且大吴畏小越如此,臣请东见越王,使之出锐师以从下吏,是君实空越,而名从诸侯以伐也。”吴王大悦,乃行子贡。

子贡东见越王,越王闻之,除道郊迎至县,身御子贡至舍而问曰:“此乃僻陋之邦,蛮夷之民也。大夫何索,居然而辱,乃至于此?”子贡曰:“吊君,故来。”越王句践稽首再拜,曰:“孤闻之,祸与福为邻,今大夫吊孤,孤之福也,敢遂闻其说。”子贡曰:“臣今见吴王,告以救鲁而伐齐。其心申,其志畏越,曰:‘尝与越战,栖于会稽山上。夫越君,贤主也。苦身劳力,以夜接日,内饰其政,外事诸侯,必将有报我之心。子待我伐越而听子。’且夫无报人之心而使人疑之者,拙也,有报人之心而使人知之者,殆也,事未发而闻者,危也。三者,举事之大忌。”越王句践稽首再拜,曰:“昔者,孤不幸少失先人,内不自量,与吴人战,军败身辱,遗先人耻。遯逃出走,上栖会稽山,下守溟海,唯鱼鳖是见。今大夫不辱而身见之,又出玉声以教孤,孤赖先人之赐,敢不奉教乎?”子贡曰:“臣闻之,明主任人不失其能,直士举贤不容于世。故临财分利则使仁,涉危拒难则使勇,用众治民则使贤,正天下、定诸侯则使圣人。臣窃练下吏之心,兵疆而不并弱,势在其上位而行恶令其下者,其君几乎?臣窃自练可以成功至王者,其唯臣几乎?今夫吴王有伐齐之志,君无惜重器,以喜其心,毋恶卑辞,以尊其礼,则伐齐必矣。彼战而不胜,则君之福也。彼战而胜,必以其余兵临晋。臣请北见晋君,令共攻之,弱吴必矣。其骑士、锐兵弊乎齐,重器、羽旄尽乎晋,则君制其敝,此灭吴必矣。”越王句践稽首再拜曰:“昔者吴王分其人民之众,以残伐吾邦,杀败吾民,屠吾百姓,夷吾宗庙,邦为空棘,身为鱼鳖饵。今孤之怨吴王,深于骨髓。而孤之事吴王,如子之畏父,弟之敬兄,蹋孤之外言也。大夫有赐,故孤敢以疑?”请遂言之:“孤身不安床席,口不甘厚味,目不视好色,耳不听钟鼓者,已三年矣。焦唇干嗌,苦心劳力,上事群臣,下养百姓。愿一与吴交天下之兵于中原之野,与吴王整襟交臂而奋,吴越之士,继迹连死,士民流离,肝脑涂地,此孤之大愿也。如此不可得也。今内自量吾国不足以伤吴,外事诸侯不能也。孤欲空邦家,措策力,变容貌,易名姓,执箕□,养牛马,以臣事之。孤虽要领不属,手足异处,四支布陈,为乡邑笑,孤之意出焉。大夫有赐,是存亡邦而兴死人也,孤赖先人之赐,敢不待命乎?”子贡曰:“夫吴王之为人也,贪功名而不知利害。”越王慥然避位曰:“在子。”子贡曰:“赐为君观夫吴王之为人,贤疆以恣下,下不能逆,数战伐,士卒不能忍。太宰嚭为人,智而愚,疆而弱,巧言利辞以内其身,善为伪诈以事其君,知前而不知后,顺君之过以安其私,是残国之吏,灭君之臣也。”越王大悦。

子贡去而行,越王送之金百镒、宝剑一、良马二,子贡不受,遂行。

至吴,报吴王曰:“敬以下吏之言告越王,越王大恐,乃惧曰:‘昔孤不幸,少失先人。内不自量,抵罪于县。军败身辱,遯逃出走,栖于会稽,邦为空棘,身为鱼鳖饵。赖大王之赐,使得奉俎豆而修祭祀。大王之赐,死且不忘,何谋敢虑?’其志甚恐,似将使使者来。”

子贡至五日,越使果至,曰:“东海役臣孤句践使使臣种,敢修下吏问于左右:昔孤不幸,少失先人,内不自量,抵罪于县。军败身辱,遯逃出走,栖于会稽。邦为空棘,身为鱼鳖饵。赖大王之赐,使得奉俎豆而修祭祀。大王之赐,死且不忘。今窃闻大王将兴大义,诛疆救弱,困暴齐而抚周室,故使越贱臣种以先人之藏器,甲二十领、屈卢之矛、步光之剑,以贺军吏。大王将遂大义,则弊邑虽小,悉择四疆之中,出卒三千,以从下吏,孤请自被坚执锐,以受矢石。”吴王大悦,乃召子贡而告之曰:“越使果来,请出卒三千,其君又从之,与寡人伐齐,可乎?”子贡曰:“不可。夫空人之邦,悉人之众,又从其君,不仁也。君受其币,许其师,而辞其君。”吴王许诺。

子贡去之晋,谓晋君曰:“臣闻之,虑不先定不可以应卒,兵不先辨不可以胜敌。今齐吴将战,胜则必以其兵临晋。”晋君大恐,曰:“为之奈何?”子贡曰:“修兵休卒以待吴,彼战而不胜,越乱之必矣。”晋君许诺。子贡去而之鲁。

吴王果兴九郡之兵,而与齐大战于艾陵,大败齐师,获七将,陈兵不归。果与晋人相遇黄池之上。吴晋争疆,晋人击之,大败吴师。越王闻之,涉江袭吴,去邦七里而军阵。吴王闻之,去晋从越。越王迎之,战于五湖。三战不胜,城门不守,遂围王宫,杀夫差而僇其相。伐吴三年,东乡而霸。故曰子贡一出,存鲁,乱齐,破吴,疆晋,霸越,是也。

...
------------

\part{}

越绝外传记地传第十

昔者,越之先君无余,乃禹之世,别封于越,以守禹冢。问天地之道,万物之纪,莫失其本。神农尝百草、水土甘苦,黄帝造衣裳,后稷产穑,制器械,人事备矣。畴粪桑麻,播种五谷,必以手足。大越海滨之民,独以鸟田,小大有差,进退有行,莫将自使,其故何也?曰:禹始也,忧民救水,到大越,上茅山,大会计,爵有德,封有功,更名茅山曰会稽。及其王也,巡狩大越,见耆老,纳诗书,审铨衡,平斗斛。因病亡死,葬会稽。苇椁桐棺,穿圹七尺,上无漏泄,下无即水。坛高三尺,土阶三等,延袤一亩。尚以为居之者乐,为之者苦,无以报民功,教民鸟田,一盛一衰。当禹之时,舜死苍梧,象为民田也。禹至此者,亦有因矣,亦覆釜也。覆釜者,州土也,填德也。禹美而告至焉。禹知时晏岁暮,年加申酉,求书其下,祠白马。禹井,井者法也。以为禹葬以法度,不烦人众。

无余初封大越,都秦余望南,千有余岁而至句践。句践徙治山北,引属东海,内、外越别封削焉。句践伐吴,霸关东,徙琅玡,起观台,台周七里,以望东海。死士八千人,戈船三百艘。居无几,躬求贤圣。孔子从弟子七十人,奉先王雅琴,治礼往奏。句践乃身被赐夷之甲,带步光之剑,杖物卢之矛,出死士三百人,为阵关下。孔子有顷姚稽到越。越王曰:“唯唯。夫子何以教之?”孔子对曰:“丘能述五帝三王之道,故奉雅琴至大王所。”句践喟然叹曰:“夫越性脆而愚,水行而山处,以船为车,以楫为马,往若飘风,去则难从,锐兵任死,越之常性也。夫子异则不可。”于是孔子辞,弟子莫能从乎。

越王夫镡以上至无余,久远,世不可纪也。夫镡子允常。允常子句践,大霸称王,徙琅玡,都也。句践子与夷,时霸。与夷子子翁,时霸。子翁子不扬,时霸。不扬子无疆,时霸,伐楚,威王灭无疆。无疆子之侯,窃自立为君长。之侯子尊,时君长。尊子亲,失众,楚伐之,走南山。亲以上至句践,凡八君,都琅玡二百二十四岁。无疆以上,霸,称王。之侯以下微弱,称君长。

句践小城,山阴城也。周二里二百二十三步,陆门四,水门一。今仓库是其宫台处也。周六百二十步,柱长三丈五尺三寸,霤高丈六尺。宫有百户,高丈二尺五寸。大城周二十里七十二步,不筑北面。而灭吴,徙治姑胥台。

山阴大城者,范蠡所筑治也,今传谓之蠡城。陆门三,水门三,决西北,亦有事。到始建国时,蠡城尽。

稷山者,句践斋戒台也。

龟山者,句践起怪游台也。东南司马门,因以照龟。又仰望天气,观天怪也。高四十六丈五尺二寸,周五百三十二步,今东武里。一曰怪山。怪山者,往古一夜自来,民怪之,故谓怪山。

驾台,周六百步,今安城里。

离台,周五百六十步,今淮阳里丘。

美人宫,周五百九十步,陆门二,水门一,今北坛利里丘土城,句践所习教美女西施、郑旦宫台也。女出于苎萝山,欲献于吴,自谓东垂僻陋,恐女朴鄙,故近大道居。去县五里。

乐野者,越之弋猎处,大乐,故谓乐野。其山上石室,句践所休谋也。去县七里。

中宿台马丘,周六百步,今高平里丘。

东郭外南小城者,句践冰室,去县三里。

句践之出入也,齐于稷山,往从田里,去从北郭门。照龟龟山,更驾台,驰于离丘,游于美人宫,兴乐中宿,过历马丘。射于乐野之衢,走犬若耶,休谋石室,食于冰厨。领功铨土,已作昌土台。藏其形,隐其情。一曰:冰室者,所以备膳羞也。

浦阳者,句践军败失众,懑于此。去县五十里。

夫山者,句践绝粮,困也。其山上大冢,句践庶子冢也。去县十五里。

句践与吴战于浙江之上,石买为将。耆老、壮长进谏曰:“夫石买,人与为怨,家与为仇,贪而好利,细人也,无长策。王而用之,国必不遂。”王不听,遂遣之。石买发,行至浙江上,斩杀无罪,欲专威服军中,动摇将率,独专其权。士众恐惧,人不自聊。兵法曰:“视民如婴儿,故可与赴深溪。”士众鱼烂而买不知,尚犹峻法隆刑。子胥独见可夺之证,变为奇谋,或北或南,夜举火击鼓,画陈诈兵,越师溃坠,政令不行,背叛乖离。还报其王,王杀买,谢其师,号声闻吴。吴王恐惧,子胥私喜:“越军败矣。胥闻之,狐之将杀,噆唇吸齿。今越句践其已败矣,君王安意,越易兼也。”使人入问之,越师请降,子胥不听。越栖于会稽之山,吴退而围之。句践喟然用种、蠡计,转死为霸。一人之身,吉凶更至。盛衰存亡,在于用臣。治道万端,要在得贤。越栖于会稽日,行成于吴,吴引兵而去。句践将降,西至浙江,待诏入吴,故有鸡鸣墟。其入辞曰:“亡臣孤句践,故将士众,入为臣虏。民可得使,地可得有。”吴王许之。子胥大怒,目若夜光,声若哮虎:“此越未战而服,天以赐吴,其逆天乎?臣唯君王急剬之。”吴王不听,遂许之浙江是也。

阳城里者,范蠡城也。西至水路,水门一,陆门二。

北阳里城,大夫种城也,取土西山以济之。径百九十四步。或为南安。

富阳里者,外越赐义也。处里门,美以练塘田。

安城里高库者,句践伐吴,禽夫差,以为胜兵,筑库高阁之。周二百三十步,今安城里。

故禹宗庙,在小城南门外大城内。禹稷在庙西,今南里。

独山大冢者,句践自治以为冢。徙琅玡,冢不成。去县九里。

麻林山,一名多山。句践欲伐吴,种麻以为弓弦,使齐人守之,越谓齐人“多”,故曰麻林多,以防吴。以山下田封功臣。去县一十二里。

会稽山上城者,句践与吴战,大败,栖其中。因以下为目鱼池,其利不租。

会稽山北城者,子胥浮兵以守城是也。

若耶大冢者,句践所徙葬先君夫镡冢也,去县二十五里。

葛山者,句践罢吴,种葛,使越女织治葛布,献于吴王夫差。去县七里。

姑中山者,越铜官之山也,越人谓之铜姑渎。长二百五十步,去县二十五里。

富中大塘者,句践治以为义田,为肥饶,谓之富中。去县二十里二十二步。

犬山者,句践罢吴,畜犬猎南山白鹿,欲得献吴,神不可得,故曰犬山。其高为犬亭。去县二十五里。

白鹿山,在犬山之南,去县二十九里。

鸡山、豕山者,句践以畜鸡豕,将伐吴,以食士也。鸡山在锡山南,去县五十里。豕山在民山西,去县六十三里。洹江以来属越。疑豕山在余暨界中。

练塘者,句践时采钖山为炭,称“炭聚”,载从炭渎至练塘,各因事名之。去县五十里。

木客大冢者,句践父允常冢也。初徙琅玡,使楼船卒二千八百人伐松柏以为桴,故曰木客。去县十五里。一曰句践伐善材,文刻献于吴,故曰木客。

官渎者,句践工官也。去县十四里。

苦竹城者,句践伐吴还,封范蠡子也。其僻居,径六十步。因为民治田,塘长千五百三十三步。其冢名土山。范蠡苦勤功笃,故封其子于是,去县十八里。

北郭外路南溪北城者,句践筑鼓钟宫也,去县七里。其邑为龚钱。

舟室者,句践船宫也,去县五十里。

民西大冢者,句践客秦伊善照龟者冢也,因名冢为秦伊山。

射浦者,句践教习兵处也。今射浦去县五里。射卒陈音死,葬民西,故曰陈音山。

种山者,句践所葬大夫种也。楼船卒二千人,钧足羡,葬之三蓬下。种将死,自策:“后有贤者,百年而至,置我三蓬,自章后世。”句践葬之,食传三贤。

巫里,句践所徙巫为一里,去县二十五里。其亭祠今为和公群社稷墟。

巫山者,越●,神巫之官也,死葬其上,去县十三里许。

六山者,句践铸铜,铸铜不烁,埋之东阪,其上马箠。句践遣使者取于南社,徙种六山,饰治为马箠,献之吴。去县三十五里。

江东中巫葬者,越神巫无杜子孙也。死,句践于中江而葬之。巫神,欲使覆祸吴人船。去县三十里。

石塘者,越所害军船也。塘广六十五步,长三百五十三步。去县四十里。

防坞者,越所以遏吴军也。去县四十里。

杭坞者,句践杭也。二百石长买卒七士人,度之会夷。去县四十里。

涂山者,禹所取妻之山也,去县五十里。

朱余者,越盐官也。越人谓盐曰“余”。去县三十五里。

句践已灭吴,使吴人筑吴塘,东西千步,名辟首。后因以为名曰塘。

独妇山者,句践将伐吴,徙寡妇致独山上,以为死士示,得专一也。去县四十里。后说之者,盖句践所以游军士也。

马嗥者,吴伐越,道逢大风,车败马失,骑士堕死,疋马啼嗥,事见吴史。

浙江南路西城者,范蠡敦兵城也。其陵固可守,故谓之固陵。所以然者,以其大船军所置也。

山阴古故陆道,出东郭,随直渎阳春亭。山阴故水道,出东郭,从郡阳春亭。去县五十里。

语儿乡,故越界,名曰就李。吴疆越地以为战地,至于柴辟亭。

女阳亭者,句践入官于吴,夫人从,道产女此亭,养于李乡,句践胜吴,更名女阳,更就李为语儿乡。

吴王夫差伐越,有其邦,句践服为臣。三年,吴王复还封句践于越,东西百里,北乡臣事吴,东为右,西为左。大越故界,浙江至就李,南姑末、写干。

觐乡北有武原。武原,今海盐。姑末,今大末。写干,今属豫章。

自无余初封于越以来,传闻越王子孙,在丹阳皋乡,更姓梅,梅里是也。

自秦以来,至秦元王不绝年。元王立二十年,平王立二十三年,惠文王立二十七年,武王立四年,昭襄王亦立五十六年,而灭周赧王,周绝于此。孝文王立一年,庄襄王更号太上皇帝,立三年,秦始皇帝立三十七年,号曰赵政,政,赵外孙,胡亥立二年,子婴立六月。秦元王至子婴,凡十王,百七十岁。汉高帝灭之,治咸阳,壹天下。

政使将魏舍、内史教攻韩,得韩王安。政使将王贲攻魏,得魏王歇。政使将王涉攻赵,得赵王尚。政使将王贲攻楚,得楚王成。政使将史敖攻燕,得燕王喜。政使将王涉攻齐,得齐王建。政更号为秦始皇帝,以其三十七年,东游之会稽,道度牛渚,奏东安,东安,今富春。丹阳,溧阳,鄣故,余杭轲亭南。东奏槿头,道度诸暨、大越。以正月甲戌到大越,留舍都亭。取钱塘浙江“岑石”。石长丈四尺,南北面广六尺,东面广四尺,西面广尺六寸,刻文立于越栋山上,其道九曲,去县二十一里。是时,徙大越民置余杭伊攻□故鄣。因徙天下有罪适吏民,置海南故大越处,以备东海外越。乃更名大越曰山阴。已去,奏诸暨、钱塘,因奏吴。上姑苏台,则治射防于宅亭、贾亭北。年至灵,不射,去,奏曲阿、句容,度牛渚,西到咸阳,崩。

...
------------

\part{}

越绝外传计倪第十一

昔者,越王句践近侵于疆吴,远媿于诸侯,兵革散空,国且灭亡,乃胁诸臣而与之盟:“吾欲伐吴,奈何有功?”群臣默然而无对。王曰:“夫主忧臣辱,主辱臣死,何大夫易见而难使也?”计倪官卑年少,其居在后,举首而起,曰:“殆哉!非大夫易见难使,是大王不能使臣也。”王曰:“何谓也?”计倪对曰:“夫官位财币,王之所轻,死者,是士之所重也。王爱所轻,责士所重,岂不艰哉?”王自揖,进计倪而问焉。

计倪对曰:“夫仁义者,治之门,士民者,君之根本也。闿门固根,莫如正身。正身之道,谨选左右。左右选,则孔主日益上,不选,则孔主日益下。二者贵质浸之渐也。愿君王公选于众,精炼左右,非君子至诚之士,无与居家。使邪僻之气无渐以生,仁义之行有阶,人知其能,官知其治。爵赏刑罚,一由君出,则臣下不敢毁誉以言,无功者不敢干治。故明主用人,不由所从,不问其先,说取一焉。是故周文、齐桓,躬于任贤,太公、管仲,明于知人。今则不然,臣故曰殆哉。”越王勃然曰:“孤闻齐威淫泆,九合诸侯,一匡天下,盖管仲之力也。寡人虽愚,唯在大夫。”计倪对曰:“齐威除管仲罪,大责任之,至易。此故南阳苍句。太公九十而不伐,磻溪之饿人也。圣主不计其辱,以为贤者。一乎仲,二乎仲,斯可致王,但霸何足道。桓称仲父,文称太公,计此二人,曾无跬步之劳、大呼之功,乃忘弓矢之怨,授以上卿。传曰:直能三公。今置臣而不尊,使贤而不用,譬如门户像设,倚而相欺,盖智士所耻,贤者所羞。君王察之。”越王曰:“诚者不能匿其辞,大夫既在,何须言哉!”计倪对曰:“臣闻智者不妄言,以成其劳,贤者始于难动,终于有成。传曰:‘易之谦逊对过问,抑威权势,利器不可示人。’言赏罚由君,此之谓也。故贤君用臣,略责于绝,施之职而成其功,远使,以效其诚。内告以匿,以知其信。与之讲事,以观其智。饮之以酒,以观其态。选士以备,不肖者无所置。”

越王大媿,乃坏池填堑,开仓谷,贷贫乏,乃使群臣身问疾病,躬视死丧,不厄穷僻,尊有德;与民同苦乐,激河泉井,示不独食。行之六年,士民一心,不谋同辞,不呼自来,皆欲伐吴。遂有大功而霸诸侯。孔子曰:“宽则得众。”此之谓也。

夫有勇见于外,必有仁于内。子胥战于就李,阖庐伤焉,军败而还。是时死伤者不可称数,所以然者,罢顿不得已。子胥内忧:“为人臣,上不能令主,下令百姓被兵刃之咎。”自责内伤,莫能知者。故身操死持伤及被兵者,莫不悉于子胥之手,垂涕啼哭,欲伐而死。三年自咎,不亲妻子,饥不饱食,寒不重彩,结心于越,欲复其仇。师事越公,录其述。印天之兆,牵牛南斗。赫赫斯怒,与天俱起。发令告民,归如父母。当胥之言,唯恐为后。师众同心,得天之中。

越乃兴师,与战西江。二国争疆,未知存亡。子胥知时变,为诈兵,为两翼,夜火相应。句践大恐,振旅服降。进兵围越会稽填山。子胥微策可谓神,守战数年,句践行成。子胥争谏,以是不容。宰嚭许之,引兵而还。夫差听嚭,不杀仇人。兴师十万,与不敌同。圣人讥之,是以春秋不差其文。故传曰:“子胥贤者,尚有就李之耻。”此之谓也。

哀哉!夫差不信伍子胥,而任太宰嚭,乃此祸晋之骊姬、亡周之褒姒,尽妖妍于图画,极凶悖于人理。倾城倾国,思昭示于后王,丽质冶容,宜求监于前史。古人云:“苦药利病,苦言利行。”伏念居安思危,日谨一日。易曰:“知进而不知退,知存而不知亡,知得而不知丧。”又曰:“进退存亡不失其正者,唯圣人乎!”由此而言,进有退之义,存有亡之几,得有丧之理。爱之如父母,仰之如日月,敬之如神明,畏之如雷霆,此其可以卜祚遐长,而祸乱不作也。

...
------------

\part{}

越绝外传记吴王占梦第十二昔者,吴王夫差之时,其民殷众,禾稼登熟,兵革坚利,其民习于斗战,阖庐□剬子胥之教,行有日,发有时。

道于姑胥之门,昼卧姑胥之台。觉寤而起,其心惆怅,如有所悔。即召太宰而占之,曰:“向者昼卧,梦入章明之宫。入门,见两□炊而不蒸;见两黑犬嗥以北,嗥以南;见两铧倚吾宫堂;见流水汤汤,越吾宫墙;见前园横索生树桐;见后房锻者扶挟鼓小震。子为寡人精占之,吉则言吉,凶则言凶,无谀寡人之心所从。”太宰嚭对曰:“善哉!大王兴师伐齐。夫章明者,伐齐克,天下显明也。见两□炊而不蒸者,大王圣气有余也。见两黑犬嗥以北,嗥以南,四夷已服,朝诸侯也。两铧倚吾宫堂,夹田夫也。见流水汤汤,越吾宫墙,献物已至,财有余也。见前园横索生树桐,乐府吹巧也。见后房锻者扶挟鼓小震者,宫女鼓乐也。”吴王大悦,而赐太宰嚭杂缯四十疋。

王心不已,召王孙骆而告之。对曰:“臣智浅能薄,无方术之事,不能占大王梦。臣知有东掖门亭长越公弟子公孙圣,为人幼而好学,长而□游,博闻疆识,通于方来之事,可占大王所梦。臣请召之。”吴王曰:“诺。”王孙骆移记,曰:“今日壬午,左校司马王孙骆,受教告东掖门亭长公孙圣:吴王昼卧,觉寤而心中惆怅也,如有悔。记到,车驰诣姑胥之台。”圣得记,发而读之,伏地而泣,有顷不起。

其妻大君从旁接而起之,曰:“何若子性之大也!希见人主,卒得急记,流涕不止。”公孙圣仰天叹曰:“呜呼,悲哉!此固非子之所能知也。今日壬午,时加南方,命属苍天,不可逃亡。伏地而泣者,不能自惜,但吴王。谀心而言,师道不明;正言直谏,身死无功。”大君曰:“汝疆食自爱,慎勿相忘。”伏地而书,既成篇,即与妻把臂而决,涕泣如雨。

上车不顾,遂至姑胥之台,谒见吴王。吴王劳曰:“越公弟子公孙圣也,寡人昼卧姑胥之台,梦入章明之宫。入门,见两□炊而不蒸;见两黑犬嗥以北,嗥以南;见两铧倚吾宫堂;见流水汤汤,越吾宫墙;见前园横索生树桐;见后房锻者扶挟鼓小震。子为寡人精占之,吉则言吉,凶则言凶,无谀寡人心所从。”公孙圣伏地,有顷而起,仰天叹曰:“悲哉!夫好船者溺,好骑者堕,君子各以所好为祸。谀谗申者,师道不明。正言切谏,身死无功。伏地而泣者,非自惜,因悲大王。夫章者,战不胜,走傽傽;明者,去昭昭,就冥冥。见两□炊而不蒸者,王且不得火食。见两黑犬嗥以北,嗥以南者,大王身死,魂魄惑也。见两铧倚吾宫堂者,越人入吴邦,伐宗庙,掘社稷也。见流水汤汤,越吾宫墙者,大王宫堂虚也。前园横索生树桐者,桐不为器用,但为甬,当与人俱葬。后房锻者鼓小震者,大息也。王毋自行,使臣下可矣。”太宰嚭、王孙骆惶怖,解冠帻,肉袒而谢。

吴王忿圣言不祥,乃使其身自受其殃。王乃使力士石番,以铁杖击圣,中断之为两头。

圣仰天叹曰:“苍天知冤乎!直言正谏,身死无功。令吾家无葬我,提我山中,后世为声响。”吴王使人提于秦余杭之山:“虎狼食其肉,野火烧其骨,东风至,飞扬汝灰,汝更能为声哉!”太宰嚭前再拜,曰:“逆言已灭,谗谀已亡,因酌行觞,时可以行矣。”吴王曰:“诺。”王孙骆为左校司马,太宰嚭为右校司马,王从骑三千,旌旗羽盖,自处中军。

伐齐大克。师兵三月不去,过伐晋。晋知其兵革之罢倦,粮食尽索,兴师击之,大败吴师。

涉江,流血浮尸者,不可胜数。吴王不忍,率其余兵,相将至秦余杭之山。

饥饿,足行乏粮,视瞻不明。据地饮水,持笼稻而餐之。顾谓左右曰:“此何名?”群臣对曰:“是笼稻也。”吴王曰:“悲哉!此公孙圣所言,王且不得火食。”太宰嚭曰:“秦余杭山西阪闲燕,可以休息,大王亟餐而去,尚有十数里耳。”吴王曰:“吾尝戮公孙圣于斯山,子试为寡人前呼之,即尚在耶,当有声响。”太宰嚭即上山三呼,圣三应。

吴王大怖,足行属腐,面如死灰色,曰:“公孙圣令寡人得邦,诚世世相事。”言未毕,越王追至。

兵三围吴,大夫种处中。范蠡数吴王曰:“王有过者五,宁知之乎?杀忠臣伍子胥、公孙圣。胥为人先知、忠信,中断之入江;圣正言直谏,身死无功。此非大过者二乎?夫齐无罪,空复伐之,使鬼神不血食,社稷废芜,父子离散,兄弟异居。此非大过者三乎?夫越王句践,虽东僻,亦得系于天皇之位,无罪,而王恒使其刍茎秩马,比于奴虏。此非大过者四乎?太宰嚭谗谀佞谄,断绝王世,听而用之。此非大过者五乎?”吴王曰:“今日闻命矣。”越王抚步光之剑,杖屈卢之矛,瞠目谓范蠡曰:“子何不早图之乎?”范蠡曰:“臣不敢杀主。臣存主若亡,今日逊敬,天报微功。”越王谓吴王曰:“世无千岁之人,死一耳。”范蠡左手持鼓,右手操枹而鼓之,曰:“上天苍苍,若存若亡。何须军士,断子之颈,挫子之骸,不亦缪乎?”吴王曰:“闻命矣。以三寸之帛,幎吾两目,使死者有知,吾惭见伍子胥、公孙圣,以为无知,吾耻生。”越王则解绶以幎其目,遂伏剑而死。

越王杀太宰嚭,戮其妻子,以其不忠信。断绝吴之世。


\part{}

越绝外传记宝剑第十三昔者,越王句践有宝剑五,闻于天下。客有能相剑者,名薛烛。

王召而问之,曰:“吾有宝剑五,请以示之。”薛烛对曰:“愚理不足以言,大王请,不得已。”乃召掌者,王使取毫曹。

薛烛对曰:“毫曹,非宝剑也。夫宝剑,五色并见,莫能相胜。毫曹已擅名矣,非宝剑也。”王曰:“取巨阙。”薛烛曰:“非宝剑也。宝剑者,金锡和铜而不离。今巨阙已离矣,非宝剑也。”王曰:“然巨阙初成之时,吾坐于露坛之上,宫人有四驾白鹿而过者,车奔鹿惊,吾引剑而指之,四驾上飞扬,不知其绝也。穿铜釜,绝铁□,胥中决如粢米,故曰巨阙。”王取纯钧,薛烛闻之,忽如败。

有顷,惧如悟。下阶而深惟,简衣而坐望之。手振拂扬,其华捽如芙蓉始出。

观其釽,烂如列星之行;观其光,浑浑如水之溢于塘;观其断,岩岩如琐石;观其才,焕焕如冰释。

“此所谓纯钧耶?”王曰:“是也。客有直之者,有市之乡二,骏马千疋,千户之都二,可乎?”薛烛对曰:“不可。当造此剑之时,赤堇之山,破而出锡;若耶之溪,涸而出铜;雨师扫洒,雷公击橐;蛟龙捧鑪,天帝装炭;太一下观,天精下之。欧冶乃因天之精神,悉其伎巧,造为大刑三、小刑二:一曰湛卢,二曰纯钧,三曰胜邪,四曰鱼肠,五曰巨阙。吴王阖庐之时,得其胜邪、鱼肠、湛卢。阖庐无道,子女死,杀生以送之。湛卢之剑,去之如水,行秦过楚,楚王卧而寤,得吴王湛卢之剑,将首魁漂而存焉。秦王闻而求之,不得,兴师击楚,曰:‘与我湛卢之剑,还师去汝。’楚王不与。时阖庐又以鱼肠之剑刺吴王僚,使披肠夷之甲三事。阖庐使专诸为奏炙鱼者,引剑而刺之,遂弑王僚。此其小试于敌邦,未见其大用于天下也。今赤堇之山已合,若耶溪深而不测。群神不下,欧冶子即死。虽复倾城量金,珠玉竭河,犹不能得此一物,有市之乡二、骏马千疋、千户之都二,何足言哉!”楚王召风胡子而问之曰:“寡人闻吴有干将,越有欧冶子,此二人甲世而生,天下未尝有。精诚上通天,下为烈士。寡人愿齎邦之重宝,皆以奉子,因吴王请此二人作铁剑,可乎?”风胡子曰:“善。”于是乃令风胡子之吴,见欧冶子、干将,使之作铁剑。

欧冶子、干将凿茨山,泄其溪,取铁英,作为铁剑三枚:一曰龙渊,二曰泰阿,三曰工布。

毕成,风胡子奏之楚王。楚王见此三剑之精神,大悦风胡子,问之曰:“此三剑何物所象?其名为何?”风胡子对曰:“一曰龙渊,二曰泰阿,三曰工布。”楚王曰:“何谓龙渊、泰阿、工布?”风胡子对曰:“欲知龙渊,观其状,如登高山,临深渊;欲知泰阿,观其釽,巍巍翼翼,如流水之波;欲知工布,釽从文起,至脊而止,如珠不可衽,文若流水不绝。”晋郑王闻而求之,不得,兴师围楚之城,三年不解。

仓谷粟索,库无兵革。左右群臣、贤士,莫能禁止。于是楚王闻之,引泰阿之剑,登城而麾之。

三军破败,士卒迷惑,流血千里,猛兽欧瞻,江水折扬,晋郑之头毕白。

楚王于是大悦,曰:“此剑威耶?寡人力耶?”风胡子对曰:“剑之威也,因大王之神。”楚王曰:“夫剑,铁耳,固能有精神若此乎?”风胡子对曰:“时各有使然。轩辕、神农、赫胥之时,以石为兵,断树木为宫室,死而龙臧。夫神圣主使然。至黄帝之时,以玉为兵,以伐树木为宫室,凿地。夫玉,亦神物也,又遇圣主使然,死而龙臧。禹穴之时,以铜为兵,以凿伊阙,通龙门,决江导河,东注于东海。天下通平,治为宫室,岂非圣主之力哉?当此之时,作铁兵,威服三军。天下闻之,莫敢不服。此亦铁兵之神,大王有圣德。”楚王曰:“寡人闻命矣。”


------------

\part{}

越绝内经九术第十四昔者,越王句践问大夫种曰:“吾欲伐吴,奈何能有功乎?”大夫种对曰:“伐吴有九术。”王曰:“何谓九术?”对曰:“一曰尊天地,事鬼神;二曰重财币,以遗其君;三曰贵籴粟槁,以空其邦;四曰遗之好美,以为劳其志;五曰遗之巧匠,使起宫室高台,尽其财,疲其力;六曰遗其谀臣,使之易伐;七曰疆其谏臣,使之自杀;八曰邦家富而备器;九曰坚厉甲兵,以承其弊。故曰九者勿患,戒口勿传,以取天下不难,况于吴乎?”越王曰:“善。”于是作为策楯,婴以白璧,镂以黄金,类龙蛇而行者。

乃使大夫种献之于吴,曰:“东海役臣孤句践,使者臣种,敢修下吏,问于左右。赖有天下之力,窃为小殿,有余财,再拜献之大王。”吴王大悦。

申胥谏曰:“不可。王勿受。昔桀起灵门,纣起鹿台,阴阳不和,五谷不时,天与之灾,邦国空虚,遂以之亡。大王受之,是后必有灾。”吴王不听,遂受之而起姑胥台。

三年聚材,五年乃成。高见二百里。行路之人,道死尸哭。越乃饰美女西施、郑旦,使大夫种献之于吴王,曰:“昔者,越王句践窃有天之遗西施、郑旦,越邦洿下贫穷,不敢当,使下臣种再拜献之大王。”吴王大悦。

申胥谏曰:“不可。王勿受。臣闻五色令人目不明,五音令人耳不聪。桀易汤而灭,纣易周文而亡。大王受之,后必有殃。胥闻越王句践昼书不倦,晦诵竟旦,聚死臣数万,是人不死,必得其愿。胥闻越王句践服诚行仁,听谏,进贤士,是人不死,必得其名。胥闻越王句践冬披毛裘,夏披絺绤,是人不死,必为利害。胥闻贤士,邦之宝也;美女,邦之咎也。夏亡于末喜,殷亡于妲己,周亡于褒姒。”吴王不听,遂受其女,以申胥为不忠而杀之。

越乃兴师伐吴,大败之于秦余杭山,灭吴,禽夫差,而戮太宰嚭与其妻子。

越绝外传记军气第十五夫圣人行兵,上与天合德,下与地合明,中与人合心。

义合乃动,见可乃取。小人则不然,以疆厌弱,取利于危,不知逆顺,快心于非。

故圣人独知气变之情,以明胜负之道。凡气有五色:青、黄、赤、白、黑。

色因有五变。人气变,军上有气,五色相连,与天相抵。此天应,不可攻,攻之无后。

其气盛者,攻之不胜。军上有赤色气者,径抵天,军有应于天,攻者其诛乃身。

军上有青气盛明,从□,其本广末锐而来者,此逆兵气也,为未可攻,衰去乃可攻。

青气在上,其谋未定;青气在右,将弱兵多;青气在后,将勇谷少,先大后小;青气在左,将少卒多,兵少军罢;青气在前,将暴,其军必来。

赤气在军上,将谋未定。其气本广末锐而来者,为逆兵气,衰去乃可攻。

赤气在右,将军勇而兵少,卒疆,必以杀降;赤气在后,将弱,卒疆,敌少,攻之杀将,其军可降;赤气在右,将勇,敌多,兵卒疆;赤气在前,将勇兵少,谷多卒少,谋不来。

黄气在军上,将谋未定。其本广末锐而来者,为逆兵气,衰去乃可攻。

黄气在右,将智而明,兵多卒疆,谷足而不可降;黄气在后,将智而勇,卒疆兵少,谷少;黄气在左,将弱卒少,兵少谷亡,攻之必伤;黄气在前,将勇智,卒多疆,谷足而有多为,不可攻也。

白气在军上,将贤智而明,卒威勇而疆。其气本广末锐而来者,为逆兵气,衰去乃可攻。

白气在右,将勇而卒疆,兵多谷亡;白气在后,将仁而明,卒少兵多,谷少军伤;白气在左,将勇而疆,卒多谷少,可降;白气在前,将弱卒亡,谷少,攻之可降。

黑气在军上,将谋未定。其气本广末锐而来者,为逆兵,去乃可攻。黑气在右,将弱卒少,兵亡,谷尽军伤,可不攻自降;黑气在后,将勇卒疆,兵少谷亡,攻之杀将,军亡;黑气在左,将智而勇,卒少兵少,攻之杀将,其军自降;黑气在前,将智而明,卒少谷尽,可不攻自降。

故明将知气变之形,气在军上,其谋未定;其在右而低者,欲为右伏兵之谋;其气在前而低者,欲为前伏阵也;其气在后而低者,欲为走兵阵也;其气阳者,欲为去兵;其气在左而低者,欲为左阵;其气间其军,欲有入邑。

右子胥相气取敌大数,其法如是。军无气,算于庙堂,以知疆弱。一、五、九,西向吉,东向败亡,无东;二、六、十,南向吉,北向败亡,无北;三、七、十一,东向吉,西向败亡,无西;四、八、十二,北向吉,南向败亡,无南。

此其用兵月日数,吉凶所避也。举兵无击太岁上物,卯也。始出各利,以其四时制日,是之谓也。

韩故治,今京兆郡,角、亢也。郑故治,角、亢也。燕故治,今上渔阳、右北平、辽东、莫郡,尾、箕也。

越故治,今大越山阴,南斗也。吴故治西江,都牛、须女也。齐故治临灾,今济北、平原、北海郡、灾川、辽东、城阳,虚、危也。

卫故治濮阳,今广阳、韩郡,营室、壁也。鲁故治太山、东温、周固水,今魏东,奎、娄也。

梁故治,今济阴、山阳、济北、东郡,毕也。晋故治,今代郡、常山、中山、河间、广平郡,觜也。

秦故治雍,今内史也,巴郡、汉中、陇西、定襄、太原、安邑,东井也。

周故治雒,今河南郡,柳、七星、张也。楚故治郢,今南郡、南阳、汝南、淮阳、六安、九江、庐江、豫章、长沙,翼、轸也。

赵故治邯郸,今辽东、陇西、北地、上郡、雁门、北郡、清河,参也。


------------

\part{}

越绝外传枕中第十六

昔者,越王句践问范子曰:“古之贤主、圣王之治,何左何右?何去何取?”范子对曰:“臣闻圣主之治,左道右术,去末取实。”越王曰:“何谓道?何谓术?何谓末?何谓实?”范子对曰:“道者,天地先生,不知老;曲成万物,不名巧。故谓之道。道生气,气生阴,阴生阳,阳生天地。天地立,然后有寒暑、燥湿、日月、星辰、四时,而万物备。术者,天意也。盛夏之时,万物遂长。圣人缘天心,助天喜,乐万物之长。故舜弹五弦之琴,歌南风之诗,而天下治。言其乐与天下同也。当是之时,颂声作。所谓末者,名也。故名过实,则百姓不附亲,贤士不为用。而外□诸侯,圣主不为也。所谓实者,谷□也,得人心,任贤士也。凡此四者,邦之宝也。”

越王曰:“寡人躬行节俭,下士求贤,不使名过实,此寡人所能行也。多贮谷,富百姓,此乃天时水旱,宁在一人耶?何以备之?”范子曰:“百里之神,千里之君。汤执其中和,举伊尹,收天下雄隽之士,练卒兵,率诸侯兵伐桀,为天下除残去贼,万民皆歌而归之。是所谓执其中和者。”越王曰:“善哉,中和所致也!寡人虽不及贤主、圣王,欲执其中和而行之。今诸侯之地,或多或少,疆弱不相当。兵革暴起,何以应之? ”范子曰:“知保人之身者,可以王天下;不知保人之身,失天下者也。”越王曰:“何谓保人之身?”范子曰:“天生万物而教之而生。人得谷即不死,谷能生人,能杀人。故谓人身。”

越王曰:“善哉。今寡人欲保谷,为之奈何?” 范子曰:“欲保,必亲于野,睹诸所多少为备。”越王曰:“所少,可得为因其贵贱,亦有应乎?”范子曰: “夫八谷贵贱之法,必察天之三表,即决矣。”越王曰:“请问三表。”范子曰:“水之势胜金,阴气蓄积大盛,水据金而死,故金中有水。如此者,岁大败,八谷皆贵。金之势胜木,阳气蓄积大盛,金据木而死,故木中有火。如此者,岁大美,八谷皆贱。金、木、水、火更相胜,此天之三表者也,不可不察。能知三表,可为邦宝。不知三表之君,千里之神,万里之君。故天下之君,发号施令,必顺于四时。四时不正,则阴阳不调,寒暑失常。如此,则岁恶,五谷不登。圣主施令,必审于四时,此至禁也。”越王曰:“此寡人所能行也。愿欲知图谷上下贵贱,欲与他货之内以自实,为之奈何? ”范子曰:“夫八谷之贱也,如宿谷之登,其明也。谛审察阴阳消息,观市之反覆,雌雄之相逐,天道乃毕。 ”

越王问范子曰:“何执而昌?何行而亡?”范子曰:“执其中则昌,行奢侈则亡。”越王曰:“寡人欲闻其说。”范子曰:“臣闻古之贤主、圣君,执中和而原其终始,即位安而万物定矣;不执其中和,不原其终始,即尊位倾,万物散。文武之业,桀纣之迹,可知矣。古者天子及至诸侯,自灭至亡,渐渍乎滋味之费,没溺于声色之类,牵孪于珍怪贵重之器,故其邦空虚。困其士民,以为须臾之乐,百姓皆有悲心,瓦解而倍畔者,桀纣是也。身死邦亡,为天下笑。此谓行奢侈而亡也。汤有七十里地。务执三表,可谓邦宝;不知三表,身死弃道。”

越王问范子曰:“春肃,夏寒,秋荣,冬泄,人治使然乎?将道也?”范子曰:“天道三千五百岁,一治一乱,终而复始,如环之无端,此天之常道也。四时易次,寒暑失常,治民然也。故天生万物之时,圣人命之曰春。春不生遂者,故天不重为春。春者,夏之父也。故春生之,夏长之,秋成而杀之,冬受而藏之。春肃而不生者,王德不究也;夏寒而不长者,臣下不奉主命也;秋顺而复荣者,百官刑不断也;冬温而泄者,发府库赏无功也。此所谓四时者,邦之禁也。”越王曰:“ 寒暑不时,治在于人,可知也。愿闻岁之美恶,谷之贵贱,何以纪之?”范子曰:“夫阴阳错缪,即为恶岁;人生失治,即为乱世。夫一乱一治,天道自然。八谷亦一贱一贵,极而复反。言乱三千岁,必有圣王也。八谷贵贱更相胜。故死凌生者,逆,大贵;生凌死者,顺,大贱。”越王曰:“善。”

越王问于范子曰:“寡人闻人失其魂魄者,死;得其魂魄者,生。物皆有之,将人也?”范子曰:“人有之,万物亦然。天地之间,人最为贵。物之生,谷为贵,以生人,与魂魄无异,可得豫知也。”越王曰:“ 其善恶可得闻乎?”范子曰:“欲知八谷之贵贱、上下、衰极,必察其魂魄,视其动静,观其所舍,万不失一。”问曰:“何谓魂魄?”对曰:“魂者,橐也;魄者,生气之源也。故神生者,出入无门,上下无根,见所而功自存,故名之曰神。神主生气之精,魂主死气之舍也。魄者主贱,魂者主贵,故当安静而不动。魂者,方盛夏而行,故万物得以自昌。神者,主气之精,主贵而云行,故方盛夏之时不行,即神气槁而不成物矣。故死凌生者,岁大败;生凌死者,岁大美。故观其魂魄,即知岁之善恶矣。”

越王问于范子曰:“寡人闻阴阳之治,不同力而功成,不同气而物生,可得而知乎?愿闻其说。”范子曰:“臣闻阴阳气不同处,万物生焉。冬三月之时,草木既死,万物各异藏,故阳气避之下藏,伏壮于内,使阴阳得成功于外。夏三月盛暑之时,万物遂长,阴气避之下藏,伏壮于内,然而万物亲而信之,是所谓也。阳者主生,万物方夏三月之时,大热不至,则万物不能成。阴气主杀,方冬三月之时,地不内藏,则根荄不成,即春无生。故一时失度,即四序为不行。”

越王曰:“善。寡人已闻阴阳之事,谷之贵贱,可得而知乎?”范子曰:“阳者主贵,阴者主贱。故当寒而不寒者,谷为之暴贵;当温而不温者,谷为之暴贱。譬犹形影、声响相闻,岂得不复哉!故曰秋冬贵阳气施于阴,阴极而复贵;春夏贱阴气施于阳,阳极而不复。”越王曰:“善哉!”以丹书帛,置之枕中,以为国宝。

越五日,困于吴,请于范子曰:“寡人守国无术,负于万物,几亡邦危社稷,为旁邦所议,无定足而立。欲捐躯出死,以报吴仇,为之奈何?”范子曰:“臣闻圣主为不可为之行,不恶人之谤己;为足举之德,不德人之称己。舜循之历山,而天下从风。使舜释其所循,而求天下之利,则恐不全其身。昔者神农之治天下,务利之而已矣,不望其报。不贪天下之财,而天下共富之。所以其智能自贵于人,而天下共尊之。故曰富贵者,天下所置,不可夺也。今王利地贪财,接兵血刃,僵尸流血,欲以显于世,不亦谬乎?”

越王曰:“上不逮于神农,下不及于尧舜,今子以至圣之道以说寡人,诚非吾所及也。且吾闻之也,父辱则子死,君辱则臣死。今寡人亲已辱于吴矣。欲行一切之变,以复吴仇,愿子更为寡人图之。”范子曰:“ 君辱则死,固其义也。立死。下士人而求成邦者,上圣之计也。且夫广天下,尊万乘之主,使百姓安其居、乐其业者,唯兵。兵之要在于人,人之要在于谷。故民众则主安,谷多则兵疆。王而备此二者,然后可以图之也。”越王曰:“吾欲富邦疆兵,地狭民少,奈何为之? ”范子曰:“夫阳动于上,以成天文,阴动于下,以成地理。审察开置之要,可以为富。凡欲先知天门开及地户闭,其术:天高五寸,减天寸六分以成地。谨司八谷,初见出于天者,是谓天门开,地户闭,阳气不得下入地户。故气转动而上下、阴阳俱绝,八谷不成,大贵必应其岁而起,此天变见符也。谨司八谷,初见入于地者,是谓地户闭。阴阳俱会,八谷大成,其岁大贱,来年大饥,此地变见瑞也。谨司八谷,初见半于人者,籴平,熟,无灾害。故天倡而见符,地应而见瑞。圣人上知天,下知地,中知人,此之谓天平地平,以此为天图。 ”

越王既已胜吴三日,反邦未至,息,自雄,问大夫种曰:“夫圣人之术,何以加于此乎?”大夫种曰: “不然。王德范子之所言,故天地之符应邦,以藏圣人之心矣。然而范子豫见之策,未肯为王言者也。”越王愀然而恐,面有忧色。请于范子,称曰:“寡人用夫子之计,幸得胜吴,尽夫子之力也。寡人闻夫子明于阴阳进退,豫知未形,推往引前,后知千岁,可得闻乎?寡人虚心垂意,听于下风。”范子曰:“夫阴阳进退,前后幽冥。未见未形,此持杀生之柄,而王制于四海,此邦之重宝也。王而毋泄此事,臣请为王言之。”越王曰:“夫子幸教寡人,愿与之自藏,至死不敢忘。”范子曰:“阴阳进退者,固天道自然,不足怪也。夫阴入浅者即岁善,阳入深者则岁恶。幽幽冥冥,豫知未形。故圣人见物不疑,是谓知时,固圣人所不传也。夫尧舜禹汤,皆有豫见之劳,虽有凶年而民不穷。”越王曰:“ 善。”以丹书帛,置之枕中,以为邦宝。

范子已告越王,立志入海,此谓天地之图也。
------------

\part{}

越绝外传春申君第十七昔者,楚考烈王相春申君吏李园。园女弟女环谓园曰:“我闻王老无嗣,可见我于春申君。我欲假于春申君。我得见于春申君,径得见于王矣。”园曰:“春申君,贵人也,千里之佐,吾何讬敢言?”女环曰:“即不见我,汝求谒于春申君:‘才人告,远道客,请归待之。’彼必问汝:‘汝家何等远道客者?’因对曰:‘园有女弟,鲁相闻之,使使者来求之园,才人使告园者。’彼必有问:‘汝女弟何能?’对曰:‘能鼓音。读书通一经。’故彼必见我。”园曰:“诺。”明日,辞春申君:“才人有远道客,请归待之。”春申君果问:“汝家何等远道客?”对曰:“园有女弟,鲁相闻之,使使求之。”春申君曰:“何能?”对曰:“能鼓音,读书通一经。”春申君曰:“可得见乎?明日,使待于离亭。”园曰:“诺。”既归,告女环曰:“吾辞于春申君,许我明日夕待于离亭。”女环曰:“园宜先供待之。”春申君到,园驰人呼女环到,黄昏,女环至,大纵酒。

女环鼓琴,曲未终,春申君大悦。留宿。明日,女环谓春申君曰:“妾闻王老无嗣,属邦于君。君外淫,不顾政事,使王闻之,君上负于王,使妾兄下负于夫人,为之奈何?无泄此口,君召而戒之。”春申君以告官属:“莫有闻淫女也。”皆曰:“诺。”与女环通,未终月,女环谓春申君曰:“妾闻王老无嗣,今怀君子一月矣,可见妾于王,幸产子男,君即王公也,而何为佐乎?君戒念之。”春申君曰:“诺。”五日而道之:“邦中有好女,中相,可属嗣者。”烈王曰:“诺。”即召之。

烈王悦,取之。十月产子男。十年,烈王死,幽王嗣立。女环使园相春申君。

相之三年,然后告园:“以吴封春申君,使备东边。”园曰:“诺。”即封春申君于吴。

幽王后怀王,使张仪诈杀之。怀王子顷襄王,秦始皇帝使王翦灭之。越绝德序外传记第十八昔者,越王句践困于会稽,叹曰:“我其不伯乎!”欲杀妻子,角战以死。

蠡对曰:“殆哉!王失计也,爱其所恶。且吴王贤不离,不肖不去,若卑辞以地让之,天若弃彼,彼必许。”句践晓焉,曰:“岂然哉!”遂听能以胜。

越王句践即得平吴,春祭三江,秋祭五湖。因以其时,为之立祠,垂之来世,传之万载。

邻邦乐德,以来取足。范蠡内视若盲,反听若聋,度天关,涉天机,后衽天人,前带神光。

当是时言之者,□其去甚微甚密,王已失之矣,然终难复见得。于是度兵徐州,致贡周室,元王以之中兴,号为州伯,以为专句践之功,非王室之力。

是时越行伯道,沛归于宋;浮陵以付楚;临沂、开阳,复之于鲁。中邦侵伐,因斯衰止。

以其诚行于内,威发于外,越专其功,故曰越绝是也。故传曰:“桓公迫于外子,能以觉悟。句践执于会稽,能因以伯。”尧舜虽圣,不能任狼致治。

管仲能知人,桓公能任贤,蠡善虑患,句践能行焉。臣主若斯,其不伯,得乎?

易曰:“君臣同心,其利断金。”此之谓也。吴越之事烦而文不喻,圣人略焉。

贤者垂意,深省厥辞,观斯智愚。夫差狂惑,贼杀子胥,句践至贤,种曷为诛?

范蠡恐惧,逃于五湖,盖有说乎?夫吴知子胥贤,犹昏然诛之。传曰:“人之将死,恶闻酒肉之味,邦之将亡,恶闻忠臣之气。”身死不为医,邦亡不为谋,还自遗灾,盖木土水火,不同气居,此之谓也。

种立休功,其后厥过自伐。句践知其仁也,不知其信。见种为吴通越,称:“君子不危穷,不灭服。”以忠告,句践非之,见乎颜色。

范蠡因心知意,策问其事,卜省其辞,吉耶凶耶?兆言其灾。夫子见利与害,去于五湖。

盖谓知其道,贵微而贱获。易曰:“知几其神乎?道以不害为左。”传曰:“知始无终,厥道必穷。”此之谓也。

子胥赐剑将自杀,叹曰:“嗟乎!众曲矫直,一人固不能独立。吾挟弓矢以逸郑楚之间,自以为可复吾见凌之仇,乃先王之功,想得报焉,自致于此。吾先得荣,后僇者,非智衰也,先遇明,后遭险,君之易移也已矣。坐不遇时,复何言哉。此吾命也,亡将安之?莫如早死,从吾先王于地下,盖吾之志也。”吴王将杀子胥,使冯同征之。

胥见冯同,知为吴王来也。泄言曰:“王不亲辅弼之臣而亲众豕之言,是吾命短也。高置吾头,必见越人入吴也,我王亲为禽哉!捐我深江,则亦已矣!”胥死之后,吴王闻,以为妖言,甚咎子胥。

王使人捐于大江口。勇士执之,乃有遗响,发愤驰腾,气若奔马。威凌万物,归神大海。

仿佛之间,音兆常在。后世称述,盖子胥,水仙也。子胥挟弓去楚,唯夫子独知其道。

事□世□有退,至今实之,实秘文之事。深述厥兆,征为其戒。齐人归女,其后亦重。

各受一篇,文辞不既,经传外章,辅发其类。故圣人见微知着,睹始知终。

由此观之,夫子不王可知也。恭承嘉惠,述畅往事。夫子作经,揽史记,愤懑不泄,兼道事后,览承传说。

厥意以为周道不敝,春秋不作。盖夫子作春秋,记元于鲁。大义立,微言属,五经六艺,为之检式。

垂意于越,以观枉直。陈其本末,抽其统纪,章决句断,各有终始。吴越之际,夫差弊矣,是之谓也。

故观乎太伯,能知圣贤之分;观乎荆平,能知信勇之变;观乎吴越,能知阴谋之虑;观乎计倪,能知阴阳消息之度;观乎请籴,能知□人之使敌邦贤不肖;观乎九术,能知取人之真,转祸之福;观乎兵法,能知却敌之路;观乎陈恒,能知古今相取之术;观乎德叙,能知忠直所死,狂●通拙。

经百八章,上下相明。齐桓兴盛,执操以同。管仲达于霸纪,范蠡审乎吉凶终始。

夫差不能□邦之治。察乎冯同、宰嚭,能知谄臣之所移,哀彼离德信不用。

内痛子胥忠谏邪君,反受其咎。夫差诛子胥,自此始亡之谓也。


\part{}

越绝篇叙外传记第十九

维先古九头之世,蒙水之际,兴败有数,承三继五。故曰众者传目,多者信德。自此之时,天下大服。三皇以后,以一治人。至于三王,争心生,兵革越,作肉刑。五胥因悉挟方气,历天汉。孔子感精,知后有疆秦丧其世,而汉兴也。赐权齐、晋、越,入吴。孔子推类,知后有苏秦也。权衡相动,衡五相发。道获麟,周尽证也,故作春秋以继周也。此时天地暴清,日月一明,弟子欣然,相与太平。孔子怀圣承弊,无尺土所有,一民所子,睹麟垂涕,伤民不得其所,非圣人孰能痛世若此。万代不灭,无能复述。故圣人没而微言绝。赐见春秋改文尚质,讥二名,兴素王,亦发愤记吴越,章句其篇,以喻后贤。赐之说也,鲁安,吴败,晋疆,越霸,世春秋二百余年,垂象后王。赐传吴越,□指于秦。圣人发一隅,辩士宣其辞,圣文绝于彼,辩士绝于此。故题其文,谓之越绝。

问曰:“越绝始于太伯,终于陈恒,何?”“论语曰:‘虽小道,必有可观者焉。’乃太伯审于始,知去上贤。太伯特不恨,让之至也。始于太伯,仁贤,明大吴也。仁能生勇,故次以荆平也,勇子胥忠、正、信、智以明也。智能生诈,故次以吴人也,善其务救蔡,勇其伐荆。其范蠡行为,持危救倾也,莫如循道顺天,富邦安民,故次计倪。富邦安民,故于自守,易以取,故次请籴也。一其愚,故乖其政也。请粟者求其福禄,必可获,故次以九术。顺天心,终和亲,即知其情。策于廊庙,以知疆弱。时至,伐必可克,故次兵法。兵,凶器也。动作不当,天与其殃。知此上事,乃可用兵。易之卜将,春秋无将,子谋父,臣杀主,天地所不容载。恶之甚深,故终于陈恒也。”

问曰:“易之卜将,春秋无将。今荆平何善乎?君无道,臣仇主,以次太伯,何?”曰:“非善荆平也,乃勇子胥也,臣不讨贼,子不复仇,非臣子也。故贤其冤于无道之楚,困不死也;善其以匹夫得一邦之众,并义复仇,倾诸侯也;非义不为,非义不死也。”

问曰:“子胥妻楚王母,无罪而死于吴。其行如是,何义乎?”曰:“孔子固贬之矣。贤其复仇,恶其妻楚王母也。然春秋之义,量功掩过也。贤之,亲亲也。”“子胥与吴何亲乎?”曰:“子胥以困干阖庐,阖庐勇之甚,将为复仇,名誉甚着。诗云:‘投我以桃,报之以李。’夫差下愚不移,终不可奈何。言不用,策不从,昭然知吴将亡也。受阖庐厚恩,不忍去而自存,欲着其谏之功也。故先吴败而杀也。死人且不负,而况面在乎?昔者管仲生,伯业兴。子胥死,伯名成。周公贵一概,不求备于一人。及外篇各有差叙,师不说。”

问曰:“子胥未贤耳。贤者所过化,子胥赐剑,欲无死,得乎?”“盲者不可示以文绣,聋者不可语以调声。瞽瞍不移,商均不化。汤系夏台,文王拘于殷。时人谓舜不孝,尧不慈,圣人不悦下愚,而况乎子胥?当困于楚,剧于吴,信不去耳,何拘之有?孔子贬之奈何?其报楚也,称子胥妻楚王母,及乎夷狄。贬之,言吴人也。”

问曰:“句践何德也?”曰:“伯德,贤君也。 ”“传曰:‘危人自安,君子弗为;夺人自与,伯夷不多。’行伪以胜,灭人以伯,其贤奈何?”曰:“是固伯道也。祺道厌驳,一善一恶。当时无天子,疆者为右,使句践无权,灭邦久矣。子胥信而得众道,范蠡善伪以胜。当明王天下太平,诸侯和亲,四夷乐德,款塞贡珍,屈膝请臣,子胥何由乃困于楚?范蠡不久乃为狂者?句践何当属莝养马?遭逢变乱,权以自存,不亦贤乎?行伯非贤,晋文之能因时顺宜,随而可之。故空社易为福,危民易为德,是之谓也。”

问曰:“子胥、范蠡何人也?”“子胥勇而智,正而信。范蠡智而明,皆贤人。”问曰:“子胥死,范蠡去,二人行违,皆称贤,何?”“论语曰:‘陈力就列,不能者止。’事君以道言耳。范蠡单身入越,主于伯,有所不合,故去也。”问曰:“不合何不死?”曰:“去止,事君之义也。义无死,胥死者,受恩深也。今蠡犹重也,不明甚矣。”问曰:“受恩死,死之善也。臣事君,犹妻事夫,何以去?”“论语曰:‘三日不朝,孔子行。’行者,去也。传曰:‘孔子去鲁,燔俎无肉;曾子去妻,藜蒸不熟。’微子去,比干死,孔子并称仁。行虽有异,其义同。”“死与生,败与成,其同奈何?”“论语曰:‘有杀身以成仁。’子胥重其信,范蠡贵其义。信从中出,义从外出。微子去者,痛殷道也。比干死者,忠于纣也。箕子亡者,正其纪也。皆忠信之至,相为表里耳。”问曰:“二子孰愈乎?”曰:“以为同耳。然子胥无为能自免于无道之楚,不忘旧功,灭身为主。合,即能以霸;不合,可去则去,可死则死。范蠡遭世不明,被发佯狂,无正不行,无主不止。色斯而举,不害于道。亿则屡中,货财殖聚。作诈成伯,不合乃去。三迁避位,名闻海内。去越入齐,老身西陶。仲子由楚,伤中而死。二子行有始终。子胥可谓兼人乎?”

问曰:“子胥伐楚宫,射其子,不杀,何也?” “弗及耳。楚世子奔逃云梦之山。子胥兵笞平王之墓,昭王遣大夫申包胥入秦请救。于斧渔子进谏子胥,子胥适会秦救至,因引兵还。越见其荣于无道之楚,兴兵伐吴。子胥以不得已,迎之就李。”问曰:“笞墓何名乎?”“子之复仇,臣之讨贼,至诚感天,矫枉过直。乳狗哺虎,不计祸福。大道不诛,诛首恶。子胥笞墓不究也。”

维子胥之述吴越也,因事类,以晓后世。着善为诚,讥恶为诚。句践以来,至乎更始之元,五百余年,吴越相复见于今。百岁一贤,犹为比肩。记陈厥说,略其有人。以去为姓,得衣乃成。厥名有米,覆之以庚。禹来东征,死葬其疆。不直自斥,讬类自明。写精露愚,略以事类,俟告后人。文属辞定,自于邦贤。邦贤以 □为姓,丞之以天。楚相屈原,与之同名。明于古今,德配颜渊。时莫能与,伏窜自容。年加申酉,怀道而终。友臣不施,犹夫子得麟。览睹厥意,嗟叹其文,于乎哀哉!温故知新,述畅子胥,以喻来今。经世历览,论者不得,莫能达焉。犹春秋锐精尧舜,垂意周文。配之天地,着于五经。齐德日月,比智阴阳。诗之伐柯,以己喻人。后生可畏,盖不在年。以□为姓,万事道也。丞之以天,德高明也。屈原同名,意相应也。百岁一贤,贤复生也。明于古今,知识宏也。德比颜渊,不可量也。时莫能用,籥□键精,深自诚也。犹子得麟,丘道穷也。姓有去,不能容也。得衣乃成,贤人衣之能章也。名有米,八政宝也。覆以庚,兵绝之也。于乎哀哉,莫肯与也。屈原隔界,放于南楚,自沉湘水,蠡所有也。

\backmatter

\end{document}