% 茶经
% 茶经.tex

\documentclass[12pt,UTF8]{ctexbook}

% 设置纸张信息。
\usepackage[a4paper,twoside]{geometry}
\geometry{
	left=25mm,
	right=25mm,
	bottom=25.4mm,
	bindingoffset=10mm
}

% 设置字体,并解决显示难检字问题。
\xeCJKsetup{AutoFallBack=true}
\setCJKmainfont{SimSun}[BoldFont=SimHei, ItalicFont=KaiTi, FallBack=SimSun-ExtB]

% 目录 chapter 级别加点(.)。
\usepackage{titletoc}
\titlecontents{chapter}[0pt]{\vspace{3mm}\bf\addvspace{2pt}\filright}{\contentspush{\thecontentslabel\hspace{0.8em}}}{}{\titlerule*[8pt]{.}\contentspage}

% 设置 part 和 chapter 标题格式。
\ctexset{
	part/name= {},
	part/number={},
	chapter/name={},
	chapter/number={},
	section/name={},
	section/number={}
}

% 设置古文原文格式。
\newenvironment{yuanwen}{\bfseries\zihao{4}}

% 设置署名格式。
\newenvironment{shuming}{\hfill\bfseries\zihao{4}}

% 注脚每页重新编号,避免编号过大。
\usepackage[perpage]{footmisc}

\title{\heiti\zihao{0} 茶经}
\author{陆羽}
\date{唐}

\begin{document}

\maketitle
\tableofcontents

\frontmatter

\chapter{前言}

中国是茶的故乡,茶文化的发源地。传说“茶之为饮,发乎神农”,有文字记载的饮茶历史也有数千年之久。如今全球已有五十多个国家种茶,一百六十多个国家与地区、二十多亿人在饮茶。可以说茶的发现与利用是中华民族对人类文明的一大重要贡献。

要想全面了解茶文化,首先要了解陆羽的《茶经》。

《茶经》是世界上第一部茶学专著,分上、中、下三卷,包括茶的本源、制茶器具、茶的采制、煮茶方法、历代茶事、茶叶产地等十章。内容丰富、详实。其中第七章“茶之事”,辑录了自上古神农氏到唐代中叶数千年间有关茶事的记录,系统而全面地介绍了我国古代茶的发展演变,尤具史料价值。在中国以及世界茶文化史上占有无可比拟的重要地位,是研究茶文化史不可或缺的重要典籍。

\section{陆羽其人}

陆羽(733--804),唐复州竟陵(今湖北天门)人。一名疾,字鸿渐,字季疵,居吴兴(今浙江湖州)号竟陵子,居上饶(即今江西上饶)号东冈子,于南越(今广东)称桑苎翁。世称陆处士、陆文学、陆三山人、东园先生等。自幼好学,性淡泊,闭门著书,不愿为官。安史之乱后,尽心于茶的研究。撰成《茶经》一书,对促进我国茶业的发展起了积极的推动作用。后人为了纪念陆羽在茶业上的功绩,祀他为“茶圣”。

陆羽在所写《陆文学自传》中称自己不知所生,三岁时被遗弃野外,竟陵龙盖寺(后改名为西塔寺)僧智积在水滨拾得而收养于寺。陆羽长大后以《周易》为自己占卦,得《蹇》之《渐》,卦曰“鸿渐于陆,其羽可用为仪”,因而用它们作为自己的名姓,姓陆名羽字鸿渐。\footnote{据《新唐书·隐逸传》记载:“不知所生,或言有僧得诸水滨,畜之。既长,以《易》自筮,得《蹇》之《渐》,曰‘鸿渐于陆,其羽可用为仪’。乃以陆为氏,名而字之。”}一说因智积俗姓陆,故以陆为姓\footnote{见《因话录》卷三}。

九岁时,陆羽开始学习撰写文章。师父智积想让他学佛,“示以佛书出世之业”,而陆羽一心向往儒学,智积屡劝不从,因而罚他做扫寺地、洁僧厕、践泥圬墙、负瓦施屋、牧牛等重务以示惩戒。在繁重劳动之余,陆羽仍然坚持学习。没有纸练习写字,就用竹枝在牛背上写。智积知道陆羽坚持学习的情况后,怕他看多了佛家之外的典籍,心去佛道日远,就将陆羽拘束在寺中,“芟剪榛莽”,并派门人之伯看管他。有时陆羽因诵读经书精神恍惚,被看管的人鞭打其背,直打到棍子断才住手。陆羽因此感叹:“岁月往矣,奈何不知书。”

陆羽不堪困辱,于天宝元年(742)逃寺而去,投靠当地戏班,弄木人、假吏、藏珠之戏,演戏为生,很快显现才华,著《谑谈》三篇,并任伶正。

唐玄宗天宝五载(746),复州人聚饮于沧浪之洲,陆羽为伶正之师,参加欢庆活动。当时河南府尹李齐物谪守竟陵,很欣赏陆羽,抚背赞叹,亲授诗集。此后,陆羽负书火门山(在今湖北天门北)邹夫子门下,受到了正规教育。

天宝十载(751),陆羽离开邹夫子,结束了五年的学习生涯。

天宝十一载(752),礼部郎中崔国辅被贬为竟陵司马,也很赏识陆羽,相与交游三年,品茶论水,诗词唱和,雅意高情一时所尚,有酬酢歌诗合集流传。临别时,崔国辅特地赠送陆羽白驴、乌犎牛各一头,以及文槐书套一枚。这些物品都是崔国辅所珍视的,由此可见二人交情之深。

李齐物的赏识及与崔国辅的交往,使陆羽得以跻身士流、闻名文坛。

天宝十四载(755),安史之乱爆发。唐肃宗至德元载(756),士人多渡江向南以避战祸,正在陕西游历的陆羽亦随流民渡江南行。

至德二载(757),陆羽至无锡,品惠山泉水,结识了时任无锡尉的皇甫冉。行至吴兴时,结识了皎然和尚,两人同为茶道中人,一见如故,遂结为“缁素忘年之交”,曾与之同居妙喜寺。乾元元年(758),陆羽寄居南京栖霞寺研究茶事,其间皇甫冉、皇甫曾兄弟数次来访。皇甫冉与陆羽离别之时,作《送陆鸿渐栖霞寺采茶》诗:“采茶非采菉,远远上层崖。布叶春风暖,盈筐白日斜。旧知山寺路,时宿野人家。借问王孙草,何时泛碗花?”

唐肃宗上元元年(760),陆羽“更隐苕溪,自称桑苎翁,阖门著书”。

唐代宗大历二年(767)至三年(768)间,陆羽常在常州义兴县(今江苏宜兴)君山一带访茶品泉,他建议常州刺史李栖筠上贡阳羡茶。

大历七年(772),颜真卿任湖州刺史,由此湖州文坛品茶吟诗等茶事活动达到了高潮。(大历八年(773)正月,颜真卿到湖州刺史任。)

大历八年(773)三月,陆羽应颜真卿之邀参与重修《韵海镜源》。(夏六月,陆羽应颜真卿约参加其主编的《韵海镜源》编撰工作。)

这年冬天十月,《韵海镜源》编纂完成,颜真卿在湖州东南建新亭以示纪念,因时在癸丑年、癸卯月、癸亥日竣工,陆羽名之为“三癸亭”。皎然作《奉和颜使君真卿与陆处士羽登妙喜寺三癸亭》以记之。

大历十二年(777),颜真卿离任湖州,陆羽也开始了新的游历。

建中三年(782),陆羽移居江西。

唐德宗贞元元年(785),陆羽到信州(今江西上饶)茶山,与孟郊交游唱和。

贞元二年(786)岁暮,到洪州(今江西南昌)玉芝观、庐山等地,与权德舆、戴叔伦等人交游唱和。

贞元五年(789)后,陆羽入广州刺史、岭南节度使李复(李齐物之子)幕府担任幕僚。

贞元九年(793),陆羽由岭南返回杭州,与灵隐寺道标、宝达禅师交游唱和,自此以后行迹不明。

贞元二十年(804),陆羽病死于湖州,终年72岁。葬于杼山,其墓与皎然砖塔相对。

陆羽在文学、史学、茶学以及地理、方志等方面都取得很大成就。时人权德舆称赞他“词艺卓异,为当时闻人”。据《陆文学自传》记载,陆羽著有《君臣契》三卷、《源解》三十卷、《江表四姓谱》八卷、《南北人物志》十卷、《吴兴历官记》三卷、《湖州刺史记》一卷、《茶经》三卷、《占梦》三卷,在当时曾产生一定影响,皇帝征召他为太子文学、太常寺太祝,皆不就职。

然而在其身后,影响至深、流传最广的是他所著《茶经》在茶文化学方面的卓越成就。“自从陆羽生人间,人间相学事春茶。”(梅尧臣《次韵和永叔尝新茶杂言》)陆羽在当时就被奉为茶神、茶仙。在《连句多暇赠陆三山人》诗中,耿湋即称陆羽:“一生为墨客,几世作茶仙。”李肇《唐国史补》记载唐后期时人们已经将陆羽作为茶神看待,《唐才子传》称陆羽《茶经》之后“天下益知饮茶矣”。陆羽及其《茶经》对茶业及茶文化的发生、发展起着不可磨灭的创始作用。

\section{成书过程}

陆羽被竟陵龙盖寺智积禅师收养后,幼年就在寺中为智积禅师煮茶,因其所煮茶适合智积禅师的口味,以致后来陆羽离开龙盖寺后,智积禅师不再喝其他人煮的茶\footnote{《纪异录》}。幼年这段经历对陆羽后来的茶事影响至深,它不仅培养了陆羽的煮茶技术,更激发了陆羽对茶的无穷兴趣。

关于《茶经》的成书时间,学界有上元二年(761)以前、广德二年(764)、大历十年(775)、建中元年(780)四种说法,各有依据,然而也各有失偏颇。

第一种说法是根据上元二年(761)陆羽《陆文学自传》一文中提及所列著作已有《茶经》三卷,由此说明《茶经》成书于陆羽撰写自传之前。日本学者布目潮沨先生考证《茶经·八之出》所列产茶州县地名均为758-761年所改,因此推断《茶经》初稿形成于761年以前。据成书于8世纪末的封演《封氏闻见记》卷六《饮茶》的记载内容表明,《茶经》在760年完成初稿之后就广为流传。

第二种说法根据《茶经·四之器》记载陆羽自制风炉,其足铭文有“圣唐灭胡明年铸”,“圣唐灭胡”即指763年平定安史之乱,陆羽于764年铸风炉,当然也可看作是对《茶经》的一次修订。

第三种说法是陆羽在大历八年(773)应颜真卿之邀担任《韵海镜源》编撰工作时,接触到大量新的文献资料,这有助于补充修订《茶经》中关于历史、文学、医药、人物等相关的资料。因此陆羽在完成《韵海镜源》编撰工作后,于大历十年(775)又对《茶经》进行第二次修订。

第四种说法,指的是《茶经》的正式刊行时间。主要依据是此后陆羽移居江西、湖南、广东等地,却未如在浙江湖州时那样将,所经历地区的茶产细致记入《茶经·八之出》的小注文中,说明未再修订,即使修订也未能刊刻行世。

《茶经》在完成初稿后的近二十年来,一直在不断修订、补充和完善,而且初稿和修订稿都曾在社会中广为流传,但最终修订完成不会晚于780年。

而在773年应邀参加《韵海镜源》的编撰工作成为陆羽修改《茶经》的新契机。陆羽在这一工作中能够接触大量的文献,有助于他在774年完成编纂工作后补充修改《茶经·七之事》中与茶有关的历史、医药、文学的文献记录,陆羽当凭借从中所获的大量文献资料对《茶经》部分内容尤其是《七之事》部分进行补充修改。

有人认为《茶经》约正式刊行于780年左右。这一推论有一定道理,因为此后陆羽曾较长时间定居江西,却未如在浙江湖州时那样,将所经历地区的茶产,细致记入《茶经·八之出》茶产地的小注文中。其后所经历的湖南、广东等地区也未有茶产地加入《茶经·八之出》。抑或陆羽曾又修改补充《茶经》内容,但却未能再流传于世。

\section{内容}

《茶经》上、中、下三卷十章,内容十分丰富。它总结了当时茶叶生产技术与经验,收集历代茶叶史料,记述作者实践调查。从现代学科分科的角度来说,《茶经》是茶叶文化的百科全书,涵盖了茶叶栽培、生产加工、药理、茶具、饮用、历史、文化、茶产区划等方面的内容。

卷上《一之源》言茶之本源、植物性状、名字称谓、种茶方式及茶饮的俭德之性;《二之具》叙采制茶叶的用具尺寸、质地与用法;《三之造》论采制茶叶的适宜季节、时间、天气状况,及对原料茶叶的选择、制茶的七道工序、成品茶叶的质量鉴别。

卷中《四之器》记煮饮茶的全部器具,计二十四组二十九种。全套茶具的组合使用体现着陆羽以“经”名茶的思想,风炉、鍑、夹、漉水囊、碗等器具的材质使用与形制设计,则具体体现出陆羽五行协谐的和谐思想、入世济世的儒家理想以及对社会安定和平的渴望。而陆羽在关注世事的同时,又满怀山林之志,是典型的中国传统人文情怀。

卷下《五之煮》介绍煮茶程序及注意事项,包括炙茶碾茶、宜火薪炭、宜茶之水、水沸程度、汤花之育、坐客碗数、乘热速饮等方面;《六之饮》强调茶饮的历史意义由来已久,区分除加盐之外不添加任何物料的单纯煮饮法与夹杂其他食物淹泡或煮饮的区别,认为真饮茶者只有排除克服饮茶所有的“九难”,才能领略茶饮的奥妙真谛;《七之事》详列历史人物的饮茶事、茶用、茶药方、茶诗文以及图经等文献对茶事的记载;《八之出》列举当时全国各地的茶产并品第其质量高下,而对于不甚了解地区的茶产,则诚实地谦称“未详”;《九之略》列举在野寺山园、瞰泉临涧诸种饮茶环境下种种可以省略不用的制茶、煮饮茶用具,最后又强调,“但城邑之中,王公之门,二十四器阙一,则茶废矣”,认为只有完整使用全套茶具,体味其中存在的思想轨范,茶道才能存而不废;《十之图》讲要用绢素书写全部《茶经》,张挂在平常可以看得见的地方,使其内容目击而存、烂熟于胸,这样《茶经》才真正完整了。

\section{历史价值}

陆羽《茶经》是世界上第一部关于茶的专门著作,在茶文化史上占有无可比拟的重要地位。《茶经》在《新唐书·艺文志·小说类》、《通志·艺文略·食货类》、《郡斋读书志·农家类》、《直斋书录解题·杂艺类》、《宋史·艺文志·农家类》等书中,都有著录。历来为《茶经》作序跋者很多,至今可见的有十七种之多。

作为世界上的第一部茶书,《茶经》被奉为茶文化的经典。唐末皮日休作《〈茶,取舍有时也。季疵始为三卷《茶经》,由是……命其煮饮之者,除痟而疠去,虽疾医之,不若也。其为利也,于人岂小哉!”宋欧阳修《集古录》:“后世言茶者必本陆鸿渐,盖为茶著书自其始也。”明陈文烛在《茶经序》中甚至以为:“人莫不饮食也,鲜能知味也。稷树艺五谷而天下知食,羽辨水煮茗而天下知饮,羽之功不在稷下,虽与稷并祠可也。”

对于古代中国绝大多数文人来说,修齐治平之外,没有绝对的理想;文章之外,没有可以称道的技能;道德、礼教之外,没有必须遵循的规范。

唐宋两朝是一个转折点,唐宋时代的社会、文化几乎各个方面都发生了重大的变革,六经注我,文人们的个体意识开始觉醒,文人们的精神世界开始变得更为丰富复杂,有些方面甚至出现了对立的状态。对于大多数文人个体来说,修齐治平的理想,文章的技能,道德、礼教的规范,是社会与传统之于他们的规范,是社会历史与文化传统赋予他们的价值观念和行为规范,过去很多人只有这些,或最多只表现出这些。而在唐宋变革之际,个体意识开始觉醒的文人,也同时开始向社会提供他们的价值观念和行为规范。

陆羽想要通过茶饮提供给社会的新的东西,是“精行俭德之人”茶饮行为的规范,这是他孤零的身世和遭逢乱世的经历之下所渴求的东西,他想通过茶叶、茶具、煮饮茶的程序等过程与方面的规范化程序,提倡某种在道德、礼教之外的行为规范,应当说这确实是中国古代社会所缺乏的。但中国古代文人内心深处在道德与礼教之外不受任何约束的传统,使得茶并未最终在文人士大夫中间形成新的行为规范。

同时,唐中期兴起讲求顿悟的禅宗,由于它不讲求苦苦的修行,因而在事实上缺乏对禅林僧众的一定的约束力。但任何一个庞大的社会团体,是一定要有某些具有强制性约束力的规范才能维系它在社会中的存在和发展的,为了做到这一点,唐宋之际,禅林清规应时出现,茶也趁此时机进入到禅林的律规之中。

在中国,社会文化根据自己的特性有选择地接受了陆羽《茶经》提供的茶艺文化的部分内容,茶的礼仪、程序部分最终大都进入到需要礼仪规范的宗教之中和一部分民俗当中,留在文人士大夫和众多茶叶消费者中间的,是茶的清雅、芬芳的享受,是精美器物的玩赏,是生命过程中体验与经历在茶中的印证与延伸,人们在其中更多的是享受自适,也是为了充分发挥茶的禀质,更多地享受茶饮茶艺的乐趣。

《茶经》是世界上第一部茶学专著,被奉为茶文化的经典,在茶文化史上有无可比拟的重要地位。《茶经》的出现,使茶学发展成为一门独立的学问。《新唐书·艺文志·小说类》《通志·艺文略·食货类》《郡斋读书志·农家类》《直斋书录题解·杂艺类》《宋史·艺文志·农家类》等书中都有著录。

唐皮日休《茶中杂咏并序》对陆羽和《茶经》推崇备至:“自周已降,及于国朝茶事,竟陵子陆季疵言之详矣。然季疵以前,称茗饮者,必浑以烹之,与夫瀹蔬而啜者无异也。季疵之始为经三卷,由是分其源,制其具,教其造,设其器,命其煮,俾饮之者,除痟而去疠,虽疾医之未若也。其为利也,于人岂小哉!余始得季疵书,以为备矣。”宋陈师道《〈茶经〉序》的评价亦堪称精到:“夫茶之著书自羽始,其用于世亦自羽始,羽诚有功于茶者也。上自宫省,下迨邑里,外及异域遐陬,宾祀燕享,预陈于前,山泽以成市,商贾以起家,又有功于人者也,可谓智矣。”今人王旭烽《瑞草之国》认为:“《茶经》是中国茶文化的标志性文本,使茶文化具备了经典的意义,完成了茶从粗放性走向艺术化的过程,使一种物质事象和精神事象结合成一种文化事象,……也为后世的茶文化活动提供了无限开放的可能性。”

《茶经》不仅是中国茶学,也是世界茶学的开山之作,对中国后世茶学影响至深。在《茶经》的影响下,茶学专著应运而生,较为著名的有五代毛文锡《茶谱》、宋蔡襄《茶录》、宋赵佶《大观茶论》、明朱权《茶谱》等。与《茶经》同名或相似的则有宋周绛《补茶经》、明真清《茶经外集》、明孙大绶《茶经外集》、明张谦德《茶经》以及清陆廷灿《续茶经》等,这些茶书在一定程度上可以说都是对陆羽《茶经》的补充和注解。陆羽《茶经》的问世,对中国茶文化的发展,具有划时代的意义,更是将茶推上了“国饮”的高度。

《茶经》不仅有功于中国茶叶及茶文化,也影响了世界其他国家和地区的茶叶与茶文化。日本的茶道、韩国的茶礼,以及近年在东南亚和欧美盛行的茶文化,无一不是在陆羽《茶经》的影响下而渐次发展起来的。茶叶成为世界三大无酒精饮料之一,陆羽《茶经》亦当推首功。

\section{版本}

《茶经》自问世以来,历代传抄刊刻不绝。据宋陈师道《〈茶经〉序》记载:“陆羽《茶经》,家书一卷,毕氏、王氏书三卷,张氏书四卷,内外书十有一卷。其文繁简不同。”据不完全统计,截至民国时期,《茶经》版本有六十多种,现存的自宋代至民国的版本约有五十余种。宋时磊《唐代茶史研究》中说:“《茶经》的版本流传可分为抄本和刊本,刊本按形式可分为丛书本、独立刊本、附刊本三种,按内容可分为初注本、无注本、增注本、增释本和删节本五种。”丛书本,现存最早的版本是南宋咸淳九年(1273)左圭的《百川学海》本,后又有《百家名书》本、《格致丛书》本、喻政《茶书》本、《重订欣赏编》本、《清媚合谱》本、《唐宋丛书》本、《五朝小说大观》本、《唐人百家小说》本、《稗史汇编》本、宛委山堂《说郛》本、《古今图书集成》本、《四库全书》本版本、《唐人说荟》本、《学津讨原》本、《植物名实图考》本、《湖北先正遗书》本、《丛书集成初编》本等。独立刊本如明嘉靖二十一年(1542)柯双华竟陵本、万历十六年(1588)程福生竹素园陈文烛校本、万历十六年(1588)孙大绶秋水斋刊本、明乐元声倚云阁刻本、明汤显祖《别本茶经》本、清雍正七年(1729)仪鸿堂本、民国西塔寺常乐刻《陆子茶经》本等。附刊本如清雍正十三年(1735)陆廷灿寿椿堂《续茶经》之《原本茶经》本、道光元年(1821)《天门县志》附刊《陆子茶经》本等。

另外,《茶经》也流传到日本、韩国、美国、意大利、法国、德国等国。据日本京都大学森鹿三教授《中国茶书的日本传入》一文中称,1191年7月荣西从宋国携带其手写的《茶经》入日本,这是《茶经》引入日本的开端。日本学者布目潮沨认为明代万历年间,日本开始出版《茶经》的单行本。日本元禄五年(1692)《文益书籍目录》录有春秋馆发行的《茶经》两册,这是最早的日语版《茶经》。韩国的版本有崔凡述《韩国之茶道》本、韩雄斌《茶文化之研究》资料本、徐延柱《茶经》本、金荣学《韩国茶文化》本、李圭正《茶经》刊本、金明培《茶经》刊本等。美国有威廉·乌克斯《茶叶全书》本、弗郎西斯·卡朋特《茶经》本,英国有《大百科全书》本,意大利有《陆羽〈茶经〉》本,另外法国和德国也有《茶经》译本。

唐代以来《茶经》刊行甚多,据不完全统计,历来相传的《茶经》版本约有六十余种。而现存至今的版本自宋代至民国约有五十余种。一部在传统四部分类中归类不明的著作——诸家书目分别有归于小说类、食货类、农家类、杂艺类者,千百年来在中国本土有六十多种版本刊行流传,在海外有日、韩、德、意、英等多种文字版本刊行,这不仅是出版史上的一个奇迹,也是文化史上的一个奇迹。一直以来,除了儒家经典与佛道经典外,没有什么著作能像《茶经》这样被翻刻重印了如此多次,从中我们既可见到茶业与茶文化的历史性繁荣,也可见到《茶经》的巨大影响。

原文以中国国家图书馆藏南宋咸淳刊百川学海本《茶经》为底本,参校明以来多种版本,但因本书所在书系的体例,不出校记,在原文上径改,其中少量有特殊意义的校勘,在注释中予以说明。

《茶经》版本繁多,本书以现存最早刻本南宋咸淳九年(1273)左圭的《百川学海》本为底本,并参酌《四库全书》本、《学津讨原》本等进行校勘。

本书在整理过程中,曾参考吴觉农《茶经述评》(农业出版社,1987)、鲍思陶纂注《茶典》(山东画报出版社,2004)、宋一明《茶经(外三种)译注》(上海古籍出版社,2009)、沈冬梅编著《茶经》(中华书局,2010)、朱自振、沈冬梅编著《中国古代茶书集成》(上海文化出版社,2010)、杨东甫主编《中国古代茶学全书》(广西师范大学出版社,2011)、于良子注释《茶经》(浙江古籍出版社,2011)、曹海英译注《茶经·续茶经》(北方文艺出版社,2014)、崇贤书院译《图解茶经·续茶经》(黄山书社,2015)、肖思学译注《茶经·续茶经》(团结出版社,2016)、郭孟良注译《茶经·续茶经》(中州古籍出版社,2017)、宋时磊著《唐代茶史研究》(中国社会科学出版社,2017)、文轩《茶经译注》(上海三联书店,2018)等专著,另参阅沈冬梅论文《陆羽〈茶经〉的历史影响与意义》(《形象史学研究》,人民出版社,2012)。

\mainmatter

\part{卷上}

\chapter{一之源}

“源”,本作“原”,指水流所从出。引申为事物的来源。《茶经》从茶源头写起,“茶者,南方之嘉木也”,明确指出茶的产地在中国南方,以及茶所具有的善和美的特性。“其巴山峡川,有两人合抱者”,这是最早记载中国野生茶树的资料。“上者生烂石,中者生砾壤,下者生黄土”,说明茶树生长的土壤环境。上等茶生在岩石充分风化的烂石土壤中,中等茶生在含有碎石子的黏性土壤中,下等茶生在土质松软的黄色黏土中。“野者上,园者次”,可以理解为生长在山野里的野生茶品质较好,茶园里人工培育的茶品质较差。陆羽曾有入山采摘野生茶、几宿不归的亲身经历。他的友人皇甫冉《送陆鸿渐栖霞寺采茶》及皇甫曾《送陆鸿渐山人采茶回》等诗中有“远远上层崖”、“时宿野人家”、“采摘知深处,烟霞羡独行”等诗句,“野者上,园者次”看法的形成,当与陆羽入山采摘野生茶的亲身经历有关。

“凡艺而不实,植而罕茂”,则论及茶树的不宜移植性。古人对茶树习性的认识仍有局限性,认为茶树只能从种子萌芽成株,不能移植,所以茶树过去一直有“艺茶必下种”、“移植不复生”的说法。所以在古代婚俗中,茶便成为坚贞不移、从一而终和婚后多子的象征,婚娶聘物必定有茶。

“法如种瓜,三岁可采”,北魏贾思勰《齐民要术·种瓜第十四》中详细讲了种瓜的方法,可与唐韩鄂《四时纂要》记载的“种茶法”对应来看。唐代距离北魏时间较近,唐代在种瓜和种茶时,在整地、挖坑、施肥及播种等方面与北魏确有相似之处,“三岁可采”的说法也基本正确。

“茶之为用,味至寒,为饮,最宜精行俭德之人”,叙述了茶的性味至寒,并特别强调茶“最宜精行俭德之人”,这与“君子比德于玉”有异曲同工之妙。把饮茶与人的品德修养相联系,认为饮茶人应品行端正、俭约,饮茶不是单纯满足人的生理需求,从而提升了饮茶的境界,并最终形成了博大精深的中国茶文化。

茶最初是因具有药用价值而进入人类生活,以后才慢慢发展成为一种保健饮料。除本章提到可治“热渴”、“凝闷”、“脑疼”、“目涩”、“四肢烦”和“百节不舒”等病症外,南朝梁任昉《述异记》认为茶还具有“能诵无忘”的功用,唐陈藏器《本草拾遗》记载茶具有“消食”、“去痰”、“少睡”、“利水道”、“益意思”、“轻身”等功用。此外,“醒酒”和“除烦去腻”的功用也常被提到。

本章结尾将选用茶叶的困难与选用人参相比。人参,得地之精灵,故有土精、地精之名。明李时珍《本草纲目》记载,人参有“明目”、“益智”、“消食”、“止渴”、“止烦躁”、“治头痛”、“去痰”、“能诵不忘”等几十种功用,这与茶的功用大体相同。本章开头将茶称为“嘉木”,结尾把茶与人参作比,一头一尾,遥相呼应,可见茶在陆羽心中的位置是何等重要。

\begin{yuanwen}
茶\footnote{植物名,山茶科,多年生深根常绿植物。有乔木型、半乔木型和灌木型之分。叶子长椭圆形,边缘有锯齿。秋末开花。种子棕褐色,有硬壳。嫩叶加工后即为可以饮用的茶叶。}者,南方\footnote{唐贞观元年(627)时分天下为十道,南方泛指山南道、淮南道、江南道、剑南道、岭南道所辖地区,基本与现今一般以秦岭山脉—淮河以南地区为南方相一致,包括四川、重庆、湖北、湖南、江西、安徽、江苏、上海、浙江、福建、广东、广西、贵州、云南(唐时为南诏国)诸省市区,以及陕西、河南两省的南部,皆为唐代的产茶区,亦是今日中国之产茶区。}之嘉木\footnote{美好的树木,优良树木。屈原《楚辞·九章·橘颂》:“后皇嘉树。”嘉,用同“佳”,美好。陆羽称茶为嘉木,北宋苏轼称茶为嘉叶,都是夸赞茶的美好。}也。一尺\footnote{古尺与今尺量度标准不同,唐尺有大尺和小尺之分,一般用大尺,传世或出土的唐代大尺一般都在三十厘米左右,比今尺略短一些。}、二尺廼\footnote{n\v{a}i}至数十尺\footnote{高数米乃至十多米的大茶树。在中国西南地区(云南、四川、贵州)发现了众多的野生大茶树,它们一般树高几米到十几米不等,最高的达二三十米,树龄多在一两千年以上。云南思茅地区澜沧拉祜族自治县“千年古茶树”树高11.8米;云南勐海县南糯山乡“南糯山茶树王”(当地称“千年茶树王”,现已枯死)树高5.45米。};其巴山\footnote{又称大巴山,广义的大巴山指绵延四川、重庆、甘肃、陕西、湖北边境山地的总称,狭义的大巴山,在汉江支流任河谷地以东,重庆、陕西、湖北三省市边境。}峡川\footnote{一指巫峡山,即重庆、湖北交界处的三峡;二指峡州,在三峡口,治所在今宜昌。故此处巴山峡川指重庆东部、湖北西部地区。},有两人合抱者,伐\footnote{砍下枝条。《诗经.周南》:伐其条枚。}而掇\footnote{du\'o,拾拣。}之。

其树如瓜芦\footnote{又名皋芦,分布于中国南方的一种叶似茶叶而味苦的树木。晋代就有南方人用皋芦煎煮饮用。宋唐慎微《证类本草》:“瓜芦木……一名皋芦,而叶大似茗,味苦涩,南人煮为饮,止渴,明目,除烦,不睡,消痰,和水当茗用之。”明李时珍《本草纲目》云:“皋芦,叶状如茗,而大如手掌,捼(ruó)碎泡饮,最苦而色浊,风味比茶不及远矣。”},叶如栀子\footnote{属茜草科,常绿灌木或小乔木,夏季开白花,有清香,叶对生,长椭圆形,近似茶叶。},花如白蔷薇\footnote{属蔷薇科,落叶灌木,枝茂多刺,高四五尺,夏初开花,花五瓣而大,花冠近似茶花。},实如栟榈\footnote{棕榈,属棕榈科。核果近球形,淡蓝黑色,有白粉,近似茶籽内实而稍小。
《说文》:“栟榈,棕也”。栟,b\=ing。},蒂\footnote{花或瓜果与枝茎相连的部分。}如丁香\footnote{一属常绿乔木,又名鸡舌香,丁子香。叶子长椭圆形,花淡红色,果实长球形。生在热带地区。花供药用,种子可榨丁香油,做芳香剂。种仁由两片形状似鸡舌的子叶抱合而成。一属落叶灌木或小乔木。叶卵圆形或肾脏形。花紫色或白色,春季开,有香味。花冠长筒状,果实略扁。多生在中国北方。},根如胡桃\footnote{属核桃科,深根植物,与茶树一样主根向土壤深处生长,根深常达二三米以上。}。{\footnotesize [瓜芦木,出广州\footnote{今属广东。三国吴永安七年(264)分交州置,治广信县(今广西梧州)。不久废。永安七年复置,治番禺(今属广东)。统辖十郡,南朝后辖境渐缩小。隋大业三年(607)改为南海郡。唐武德四年(621)复为广州,后为岭南道治所,天宝元年(742)改为南海郡,乾元元年(758)复为广州,乾宁二年(895)改为清海军。},似茶,至苦涩。栟榈,蒲葵\footnote{属棕榈科,常绿乔木,叶大,大部分掌状分裂,可做扇子,裂片长披针形,圆锥花序,生在叶腋间,花小,果实椭圆形,成熟时黑色。生长在热带和亚热带地区。}之属,其子似茶。胡桃与茶,根皆下孕\footnote{植物根系在土壤中往地下深处发育滋生。},兆\footnote{指核桃与茶树生长时根将土地撑裂,方始出土成长。《说文》解释为“灼龟坼(chè)也”,本意是龟裂,指古人占卜时烧灼甲骨呈现裂纹,这里作裂开解。}至瓦砾\footnote{破碎的砖头瓦片,引申为硬土层。},苗木上抽\footnote{向上萌发生长。
}。]}
\end{yuanwen}

茶,是南方地区的优良树种。树高一尺、二尺以至数十尺。在巴山峡川一带(今重庆东部、湖北西部地区),有的茶树树干粗到要两个人合抱,需要先将树枝砍下来,再采摘茶叶。

茶树的树形像瓜芦,叶子像栀子叶,花像白蔷薇,种子像棕榈籽,蒂像丁香蒂,根像胡桃树根。(瓜芦木,产自广州地区,叶子和茶相似,味道非常苦涩;棕榈,蒲葵类植物,种子很像茶籽;胡桃树和茶树,根部都向下生长得很深,碰到有碎砖烂瓦的硬土层时,苗木才开始向上萌发生长。)

\begin{yuanwen}
其字,或从草,或从木,或草木并。{\footnotesize [从草,当作“茶”,其字出《开元文字音义\footnote{字书名。唐开元二十三年(735)编辑的字书。共有三十卷,已佚。清代黄奭《汉学堂丛书经解·小学类》辑存一卷,汪黎庆《学术丛编·小学丛残》中亦有收录。此书中已收有“茶”字,说明在陆羽《茶经》写成之前二十五年,“茶”字已经被收录在官修字书当中。}》。从木,当作“𣗪”,其字出《本草\footnote{指唐高宗显庆四年(659)李勣、苏敬等人所撰的《新修本草》(今称《唐本草》),已佚。今存宋唐慎微《重修政和经史证类备用本草》引用。敦煌、日本有《新修本草》钞写本残卷,清傅云龙《籑喜庐丛书》之二中收有日本写本残卷,有上海群联出版社1955年影印本;敦煌文献分类录校丛刊《敦煌医药文书辑校》中录有敦煌写本残卷,有江苏古籍出版社1999年版本。}》。草木并,作“荼”,其字出《尔雅\footnote{中国最早的字书,共十九篇,为考证词义和古代名物的重要资料。古来相传为周公所撰,或谓孔子门徒解释六艺之作。实际应当是由秦汉间经师学者缀辑周汉诸书旧文,递相增益而成,非出于一时一手。《尔雅》既是中国古代的词典,也是儒家的经典之一,列入十三经之中。“尔”是近的意思,“雅”是正、雅言的意思,是某一时代官方规定的规范语言。“尔雅”就是近正,使语言接近官方规定的语言。}》。]}
\end{yuanwen}

茶字,有属草部的,有属木部的,有并属草、木两部的。(属草部的,应当写作“茶”,在《开元文字音义》中有收录;属木部的,应当写作“𣗪”,此字见于《本草》;并属草、木两部的,写作“荼”,此字见于《尔雅》。)

\begin{yuanwen}
其名,一曰茶,二曰檟\footnote{ji\v{a},本意是楸树,落叶乔木。又用作茶的别名。《尔雅》第十四篇《释木》:“槚,苦荼。”},三曰蔎\footnote{sh\`e,本为香草名。《玉篇》:“蔎,香草也”。},四曰茗\footnote{北宋徐铉注《说文》作为新附字补入,注为“茶芽也”。三国吴陆玑《毛诗草木鸟兽虫鱼疏》卷上:“椒树似茱萸……蜀人作茶,吴人作茗,皆合煮其叶以为香。”据此,则茗字作为茶名来自长江中下游,后代成为主要的茶名之一。},五曰荈\footnote{chu\v{a}n,西汉司马相如《凡将篇》以“荈诧”迭用代表茶名。三国时“茶荈”二字连用,《三国志·吴书·韦曜传》:“曜素饮酒不过三升,初见礼异时,常为裁减,或密赐茶荈以当酒。”西晋杜育《荈赋》以后,“荈”字成为历代主要的茶名之一,现代已经很少用。}。{\footnotesize [周公云:“檟,苦荼。”杨执戟\footnote{即杨雄。西汉人。著有《方言》等书。}云:“蜀西南人谓荼曰蔎。”郭弘农\footnote{即郭璞。晋时人。注释过《方言》、《尔雅》等字书。}云:“早取为荼,晚取为茗,或一曰荈耳。”]}
\end{yuanwen}

茶的名称:第一为茶,第二为槚,第三为蔎,第四为茗,第五为荈。(周公说:“槚,就是苦荼。”扬雄说:“蜀地西南的人称茶为蔎。”郭璞说:“早采摘的称为荼,晚采摘的称为茗,或称为荈。”)

\begin{yuanwen}
其地,上者生烂石,中者生栎壤{\footnotesize [栎字当从石为砾]},下者生黄土。凡艺\footnote{指种植技术。}而不实,植而罕茂。法如种瓜,三岁可采。野者上,园者次。阳崖阴林,紫者上,绿者次;笋者上,芽者次;叶卷上,叶舒次。阴山坡谷者,不款项堪采掇,性凝滞\footnote{凝滞,凝结不散。},结瘕\footnote{瘕,腹中肿块。《正字通》:“腹中肿块,坚者曰症,有物形曰瘕”。}疾。
\end{yuanwen}

\begin{yuanwen}
茶之为用,味至寒,为饮最宜。精行俭德之人,若热渴、凝闷、脑疼、目涩、四肢烦、百节不舒,聊四五啜,与醍醐\footnote{酥酪上凝聚的油,味甘美。}、甘露\footnote{即露水,古人说它是“天之津液”。}抗衡也。采不时,造不精,杂以卉莽\footnote{野草。},饮之成疾。
\end{yuanwen}

\begin{yuanwen}
茶为累也,亦犹人参。上者生上党\footnote{唐时郡名,治所在今山西长治市,长子、潞城一带。},中者生百济、新罗\footnote{唐时位于朝鲜半岛上的两个小国,百济在半岛西南部,新罗在半岛东南部。},下者生高丽\footnote{唐时周边小国之一,即今朝鲜。}。有生泽州、易州、幽州、檀州\footnote{皆为唐时州名。治所分别在今山西晋城、河北易县、北京市区北,北京市怀柔县一带。}者,为药无效,况非此者!设服荠苨\footnote{一种形似人参的野果。}使六疾\footnote{指人遇阴、阳、风、雨、晦、明得的多种疾病。}不瘳\footnote{痊愈。}。知人参为累,则茶累尽矣。
\end{yuanwen}

\chapter{二之具}

\begin{yuanwen}
籯\footnote{ying。竹制的箱、笼、篮子等盛物器具。}:一曰篮,一曰笼,一曰筥\footnote{ju。圆形的盛物竹器。}。以竹织之,受五升,或一斗、二斗、三斗者,茶人负以采茶也。{\footnotesize [籯,音盈,《汉书》所谓“黄金满籯,不如一经。\footnote{语出《汉书.韦贤传》。谓留给儿孙满箱黄金,不如留给他一本经书。}”颜师古\footnote{名籕。唐初经学家,曾注《汉书》。}云:“籯,竹器也,容四升耳。”]}

灶:无用突\footnote{突,烟囱。成语有“曲突徙薪”。}者。

釜:用唇口者。

甑\footnote{读音zeng。古代蒸炊器。今蒸笼。}:或木或瓦,匪腰而泥。篮以箅\footnote{读音bi。蒸笼中的竹屉。}之,篾\footnote{读音mie,长条细簿竹片,在此作从甑中取出箅的理耳。}以系之。始其蒸也,入乎箅;既其熟也,出乎箅。釜涸,注于甑中,{\footnotesize [甑,不带而泥之。]}又以彀木枝三亚者制之,{\footnotesize [亚字当作桠,木桠枝也。]}散所蒸芽笋并叶,畏流其膏。

杵臼:一名碓,惟恒用者为佳。

规:一曰模,一曰棬。以铁制之,或圆、或方、或花。

承:一曰台,一曰砧。以石为之。不然,以槐、桑木半埋地中,遣无所摇动。

襜\footnote{chan系在衣服前面的围裙。《尔雅.释物》:“衣蔽前谓之襜。}:一曰衣。以油绢或雨衫单服败者为之。以襜置承上,又以规置襜上,以造茶也。茶成,举而易之。

芘莉\footnote{竹制的盘子类器具。}:一曰赢子,一曰蒡莨\footnote{pang lang。笼、盘一类盛物器具。},以二小竹,长三尺,躯二尺五寸,柄五寸。以篾织方眼,如圃人箩,阔二尺,以列茶也。

棨\footnote{qi,穿茶饼用的锥刀。}:一曰锥刀。柄以坚木为之。用穿茶也。

扑:一曰鞭。以竹为之。穿茶以解茶也。 

焙:凿地深二尺,阔二尺五寸,长一丈。上作短墙,高二尺,泥之。

贯:削竹为之,长二尺五寸。以贯茶焙之。

棚:一曰栈。以木构于焙上,编木两层,高一尺,以焙茶也。茶之半干,升下棚;全干,升上棚。

穿:江东、淮南剖竹为之;巴川峡山,纫谷皮为之。江东以一斤为上穿,半斤为中穿,四、五两为小穿。峡中以一百二十斤为上穿,八十斤为中穿,四五十斤为小穿。穿,旧作钗钏之“钏”字,或作贯“串”。今则不然,如“磨、扇、弹、钻、缝”五字,文以平声书之,义以去声呼之,其字,以“穿”名之。

育:以木制之,以竹编之,以纸糊之。中有隔,上有覆,下有床,旁有门,掩一扇。中置一器,贮糖煨火,令火煴煴然\footnote{煴,yun,没有光焰的火。煴煴然,火热微弱的样子。颜师古说:“煴,聚火无焰者也。” }。江南梅雨时,焚之以火。{\footnotesize [育者,以其藏养为名。]}
\end{yuanwen}

\chapter{三之造}

\begin{yuanwen}
凡采茶,在二月、三月、四月之间。茶之笋者,竽烂石沃土,长四、五寸,若薇蕨始抽,凌露采焉。茶之芽者,发于丛薄之上,有三枝、四枝、五枝者,选其中枝颖拔者采焉。其日,有雨不采,晴有云不采;晴,采之、蒸之、捣之、焙之、穿之、封之、茶之干矣。

茶有千万状,卤莽而言,如胡人靴者,蹙缩然;{\footnotesize [京锥文也。]}犎牛臆者,廉襜然;{\footnotesize [犎,音朋,野牛也。]}浮云出山者,轮囷然;轻飙拂水也。又如新治地者,遇暴雨流潦之所经;此皆茶之精腴。有如竹箨者,枝干坚实,艰于蒸捣,故其形籭簁然;有如霜荷者,茎叶凋沮,易其状貌,故厥状委悴然;此皆茶之瘠老者也。

自采至于封,七经目。自胡靴至于霜荷,八等。或以光黑平正言佳者,斯鉴之下也。以皱黄坳垤言佳者,鉴之次也。若皆言佳及皆言不佳者,鉴之上也。何者?出膏者光,含膏者皱,宿制者则黑,日成者则黄;蒸压则平正,纵之则坳垤;此茶与草木叶一也。茶之否臧,存于口决。
\end{yuanwen}


注释:
1.若薇蕨始抽,凌露采焉:薇、蕨,都是野菜。《诗经.小雅》有“采薇”篇,《毛传》:“薇,菜也”。《诗经》又有“吉采其蕨”句,《诗义疏》说:“蕨,山菜也”。二者都在春季抽芽生长。凌,冒着。 
2.丛薄:灌木、杂草丛生的地方。《汉书注》:“灌木曰丛”。杨雄《甘草同赋注》:“草丛生曰薄”。 
3.京锥文也:京,高大。《诗经.皇矣》:“依其在京”。《毛传》:“京,大阜也”。锥,刀锥。文,同“纹”。全句意为:大钻子刻钻的花纹。 
4.臆者,廉[衤詹]然:臆,指牛胸肩部位的肉。廉,边侧。《说文》:“廉,仄也”。[衤詹],帷幕。全句意为:像牛胸肩的肉,像侧边的帷幕。 
5.轮[囗禾]:轮,车轮。[囗禾],圆顶的仓。《说文》:“[上竹下丽],竹器也”《集韵》说就是竹筛。 
6.竹箨:竹笋的外壳。箨,读音ruo。 
7.[上竹下丽][上竹下徙]两相通,读音亦同:si。皆为竹器。《说文》:“[上竹下丽],竹器也”《集韵》说就是竹筛。 
8.坳垤:土地低下处叫坳,小土堆叫垤。形容茶饼表面的凸凹不平。 
9.否臧:否,读音pi,贬,非议。臧,褒奖。《世说新语.德行第一》:“每与人言,未尝臧否人物。” 

注1:[火日皿]:左“火”旁,右“上日下皿”。下同。

\part{卷中}

\chapter{四之器}

\begin{yuanwen}
风炉(灰承) 筥 炭挝 火䇲 鍑 交床 夹纸囊 碾拂末 罗 合 则 水方 漉水囊 瓢 竹䇲 鹾簋揭 碗 熟 盂 畚 札 涤方 滓方 巾 具列 都篮

风炉[灰承]

风炉:以铜、铁铸之,如古鼎形。厚三分,缘阔九分,令六分虚中,致其污墁。凡三足,古文书二十一字:一足云:“坎上巽下离于中”;一足云:“体均五行去百疾”;一足云:“圣唐灭胡明年铸。”其三足之间,设三窗,底一窗以为通飙漏烬之所。上并古文书六字:一窗之上书“伊公”二字;一窗之上书“羹陆”二字;一窗之上书“氏茶”二字,所谓“伊公羹、陆氏茶”也。置滞(土旁)[土臬],于其内设三格:其一格有翟焉,翟者,火禽也,画一卦曰离;其一格有彪焉,彪者,风兽也,画一卦曰巽;其一格有鱼焉,鱼者,水虫也,画一卦曰坎。巽主风,离主火,坎主水,风能兴火,火能熟水,故备其三卦焉。其饰,以连葩、垂蔓、曲水、方文之类。其炉,或锻铁为之,或运泥为之.其灰承,作三足铁[木半]抬之。

[上竹下吕]

[上竹下吕]:以竹织之,高一尺二寸,径阔七寸。或用藤,作木楦如[上竹下吕]形织之。六出圆眼。其底盖若莉箧口①,铄之。

炭挝

炭挝:以铁六棱制之。长一尺,锐上丰中。执细头,系一小[钅展],以饰挝也。若今之河陇军人木吾也。或作[木追],或作斧,随其便也。

火[上竹下夹]

火[上竹下夹]:一名箸,若常用者,圆直一尺三寸。顶平截,无葱薹句[钅巢]之属。以铁或熟铜制之。

[钅复](音辅,或作釜,或作[鬲甫])

[钅复]:以生铁为之。今人有业冶者,所谓急铁,其铁以耕刀之趄炼而铸之。内抹土而外抹沙。土滑于内,易其摩涤;沙涩于外,吸其炎焰。方其耳,以令正也。广其缘,以务远也。长其脐,以守中也。脐长,则沸中;沸中,末易扬,则其味淳也。洪州以瓷为之,莱州以石为之。瓷与石皆雅器也,性非坚实,难可持久。用银为之,至洁,但涉于侈丽。稚则雅矣,洁亦洁矣,若用之恒,而卒归于铁也。

交床

交床:以十字交之,剜中令虚,以支[钅复]也。

夹

夹:以小青竹为之,长一尺二寸。令一寸有节,节以上剖之,以炙茶也。彼竹之筱,津润于火,假其香洁以益茶味。恐非林谷间莫之致。或用精铁、熟铜之类,取其久也。

纸囊

纸囊:以剡藤纸白厚者夹缝之,以贮所炙茶,使不泄其香也。

碾(拂末)

碾:以桔木为之,次以梨,桑、桐、柘为之。内圆而外方。内圆,备于运行也;外方,制其倾危也。内容堕而外无余木。堕,形如车轮,不辐而轴焉。长九寸,阔一寸七分。堕径三寸八分,中厚一寸,边厚半寸。轴中方而执圆。其拂未,以鸟羽制之。

罗、合

罗、合:罗末,以合贮之,以则置合中。用巨竹剖而屈之,以纱绢衣之。其合,以竹节为之,或屈杉以漆之。高三寸,盖一寸,底二才,口径四寸。

则

则:以海贝、蜗蛤之属,或以铜、铁,竹匕、策之类。则者,量也,准也,度也。凡煮水一升,用末方寸匕”,若好薄者减之,故云则也。

水方

水方:以稠(木旁)木(原注,音胄,木名也。]槐、楸、梓等合之,其里井外缝漆之。受一斗。

漉水囊

漉水囊:若常用者。其格,以生铜铸之,以备水湿无有苔秽、腥涩之意;以熟铜、苔秽;铁,腥涩也。林栖谷隐者,或用之竹木。木与竹非持久涉远之具,故用之生铜,其囊,织青竹以卷之,裁碧缣以缝之,细翠钿以缀之,又作油绿囊以贮之。圆径五寸,柄一寸五分。

瓢

瓢:一曰牺、杓,剖瓠为之,或刊木为之。晋舍人杜毓《舛(艹头)赋》云:“酌之以瓠”。瓠,瓢也,口阔,胚薄,柄短。永嘉中,余姚人虞洪入瀑布山采茗,遇一道士云:“吾,丹丘子,祈子他日瓯牺之余,乞相遗也。”牺,木杓也。今常用以梨木为之。

竹[上竹下夹]

竹[上竹下夹]:或以桃、柳、蒲葵木为之,或以柿心木为之。长一尺,银裹两头。

鹾簋揭

鹾簋:以瓷为之,圆径四寸,若合形。或瓶、或缶。贮盐花也。其揭,竹制,长四寸一分,阔九分。揭,策也。

熟盂

熟盂:以贮熟水。或瓷、或砂。受二升。

碗

碗:越州上,鼎州、婺州次;丘州上,寿州、洪州次。或者以邢州处越州上,殊为不然。若邢瓷类银,越瓷类玉,邢不如越一也;若邢瓷类雪,则越瓷类冰,邢不如越二也;邢瓷白而茶色丹,越瓷青而茶色绿,邢不如越三也。晋杜琉《[艹舛]赋》所谓:“器择陶拣,出自东瓯”。瓯,越州也,瓯越上。口唇不卷,底卷而浅,受半升以下。越州瓷、丘瓷皆青,青则益茶,茶作红白之色。邢州瓷白,茶色红;寿州瓷黄,茶色紫;洪州瓷褐,茶色黑;悉不宜茶。

畚

畚:以白蒲卷而编之,可贮碗十枚,或用[上竹下吕]。其纸[巾巴]以剡纸夹缝令方,亦十之也。

札

札:缉[木并]榈皮,以茱萸莫木夹而缚之,或截竹束而管之,若巨笔形。

涤方

涤方:以贮洗涤之余。用楸木合之,制如水方,受八升。

卷中 四之器
滓方

滓方:以集诸滓,制如涤方,处五升。

巾

巾:以拖(纟旁)布为之。长二尺,作二枚,互用之,以洁诸器。

具列

具列:或作床,或作架。或纯木、纯竹而制之;或木或竹……,黄黑可扃而漆者。长三尺,阔二尺,高六寸。具列者,悉敛诸器物,悉以陈列也。

都篮

都篮:以悉设诸器而名之,以竹蔑,内作三角方眼,外以双蔑阔者经之,以单蔑纤者缚之,递压双经,作方眼,使玲成。高一尺五寸,底阔一尺,高二寸,长二尺四寸,阔二尺。
\end{yuanwen}




[注释]
1.污漫:本为涂墙用的工具。这里指涂泥。 
2.坎上巽下离于中:坎、巽、离都是八卦的卦名,坎为水,巽为风,离为火。 
3.“盛唐灭胡明年铸”:盛唐灭胡,指唐平息安史之乱,时在唐广德元年(763),此鼎则铸于公元764年。 
4.伊公羹、陆氏茶:伊公,指商汤时的大尹伊挚。相传他善调汤昧,世称“伊公羹”。陆,即陆羽自己。“陆氏茶”。陆羽的茶具。 
5.[上封下牛]比(氵旁):读音die2 nie4。[上封下牛],贮藏。《广韵》:“滞,贮也,止也。”比(氵旁),土堆。《集韵》:“比(氵旁),小山也”。 
6.三足铁[上竹下丽]:[上竹下丽],通“盘”,盘子。 
7.莉(上竹下夹):用小竹蔑编成的长方形箱子。 
8.木吾:木棒。崔豹《古今注》:“木吾,樟也”。 
9.葱[艹壹]、句[上竹下徙],[艹壹],读音tan4。葱的籽实,长在葱的顶部,呈圆珠形。句,通“勾”,弯曲形。[上竹下徙],即“锁”的异体字。 
10.耕刀之趄:耕刀,即锄头、犁头。趄,读音ju,艰难行走之意,成语有“趑趄不前”,此引申为坏的、旧的。 
11.洪州:唐时州名。治所在今江西南昌一带。 
12.莱州:唐时州名。治所在今山东掖县一带。 
13.被竹之筱:筱,竹的一种,名小箭竹。 
14.判藤纸:产于唐时浙江剡县、用藤为原料制成的纸,洁白细致有韧性,为唐时包茶专用纸。 
15.竹匕:匕,读bi3.匙子。 
16.用末方寸匕:用竹匙挑起茶叶末一平方寸。陶弘景《名医别录》:“方寸匕者,作匕正方一寸,抄散取不落为度。” 
17.漉水囊:漉,读音ui,滤过。漉水囊,即滤水袋。 
18.杜毓:西晋时人,字方叔,曾任中书舍人等职。 
19.鹾簋:盐罐,鹾,读音cuo2,盐。《礼记?曲礼》:“盐曰咸鹾”。簋,读音gui1,古代盛食物的圆口竹器。 
20.越州、鼎州、婺州:越州,治所在今浙江省绍兴地区。唐时越窑主要在余姚,所产青瓷,极名贵。鼎州,冶所在今陕西省径阳三原一带。婺州,治所在今浙江省金华一带。 
21.岳州、寿州、洪州、邢州:皆唐时州郡名。治所分别在今湖南岳阳、安徽寿县、江西南昌、河北邢台一带。 
22.畚:读音ben3。即簸箕。 
23.滞(土旁)布,滞(土旁),读音shi1,粗绸。 
24.扃:读音jiong1,可关锁的门。 

\part{卷下}

\chapter{五之煮}

\begin{yuanwen}
凡炙茶,慎勿于风烬间炙,[火票]焰如钻,使凉炎不均。特以逼火,屡其翻正,候炮出培[土娄]状 蟆背(1),然后去火五寸。卷而舒,则本其始,又炙之。若火干者,以气熟止;日干者,以柔止。

其始,若茶之至嫩者,蒸罢热捣,叶烂而芽笋存焉。假以力者,持千钧杵亦不之烂,如漆科珠(2),壮士接之,不能驻其指。及就,则似无穰骨也。炙之,则其节若倪倪如婴儿之臂耳。既而,承热用纸囊贮之,精华之气无所散越,候寒末之。[原注:末之上者,其屑如细米;末之下者,其屑如菱角。]

其火,用炭,次用劲薪。[原注:谓桑、槐、桐、枥之类也。]其炭曾经燔炙为膻腻所及,及膏木、败器,不用之。[原注:膏木,谓柏、松、桧也。败器,谓朽废器也。]古人有劳薪之味(3),信哉!

其水,用山水上,江水中,井水下。[原注:《[上艹下舛]赋》所谓“水则岷方之注,挹彼清流(4)。”]其山水拣乳泉、石池漫流者上;其瀑涌湍漱,勿食之。久食,令人有颈疾。又水流于山谷者,澄浸不泄,自火天至霜郊以前(5),或潜龙蓄毒于其间,饮者可决之,以流其恶,使新泉涓涓然,酌之。其江水,取去人远者。井,取汲多者。

其沸,如鱼目(6),微有声,为一沸;缘边如涌泉连珠,为二沸;腾波鼓浪,为三沸,已上,水老,不可食也。初沸,则水合量,调之以盐味,谓弃其啜余,[原注:啜,尝也,市税反,又市悦反。]无乃[卤舀][卤监]而钟其一味乎,[原注:[卤舀],古暂反。[卤监],吐滥反。无味也。]第二沸,出水一瓢,以竹 环激汤心,则量末当中心而下。有顷,势若奔涛溅沫,以所出水止之,而育其华也。 

凡酌至诸碗,令沫饽均。[原注:字书并《本草》:“沫、饽,均茗沫也。”饽蒲笏反。]沫饽,汤之华也。华之薄者曰沫,厚者曰饽,轻细者曰花,花,如枣花漂漂然于环池之上;又如回潭曲渚青萍之始生;又如晴天爽朗,有浮云鳞然。其沫者,若绿钱浮于水湄(7);又如菊英堕于樽俎之中(8)。饽者,以滓煮之,及沸,则重华累沫,皤皤然若积雪耳(9)。《[上艹下舛]赋》所谓“焕如积雪,烨若春[莆方攵](10)”,有之。

第一煮沸水,弃其上有水膜如黑云母,饮之则其味不正。其第一者为隽永,[原注:徐县、全县二反。至美者曰隽永。隽,味也。永,长也。史长曰隽永,《汉书》蒯通著《隽永》二十篇也。]或留熟盂以贮之,以备育华救沸之用,诸第一与第二、第三碗次之,第四、第五碗外,非渴甚莫之饮。凡煮水一升,酌分五碗,[原注:碗数少至三,多至五;若人多至十,加两炉。]乘热连饮之。以重浊凝其下,精英浮其上。如冷,则精英随气而竭,饮啜不消亦然矣。

茶性俭(11),不宜广,广则其味黯澹。且如一满碗,啜半而味寡,况其广乎!其色缃也,其馨[上必下土右欠] 也,[原注:香至美曰[上必下土右欠]。[上必下土右欠] ,音备。]其味甘,[木贾] 也;不甘而苦,[上艹下舛] 也;啜苦咽甘,茶也。
\end{yuanwen}


[注释]
(1)炮出培[土娄]状虾蟆背:炮,烘烤。培[土娄],小土堆。[土娄]读lou。 蟆 背,有很多丘泡,不平滑,形容茶饼表面起泡如蛙背。
(2)如漆科珠:科,用斗称量。《说文》:“从禾,从斗。斗者,量也”。这句意为用漆斗量珍珠,滑溜难量。
(3)劳薪之昧:用旧车轮之类烧烤,食物会有异味。典出《晋书.荀勖传》。
(4)挹彼清流:挹,读yi,舀取。 
(5)自火天至霜郊:火天,酷暑时节。《诗经.七月》:“七月流火”。霜郊,秋末冬初霜降大地。二十四节气中,“霜降”在农历九月下旬。
(6)如鱼目:水初沸时,水面有许多小气泡,像鱼眼睛,故称鱼目。后人又称“蟹眼”。
(7)水湄:有水草的河边。《说文》:“湄,水草交为湄”。
(8)樽俎:樽是酒器,俎是砧板,这里指各种餐具。
(9)皤皤然:皤,读音po。皤皤,满头白发的样子。这里形容白色水沫。 
(10)烨若春[上艹下敷] :烨,读ye,光辉明亮。[上艹下敷],读fu,花。《集韵》:“[上艹下敷],花之通名。
(11)茶性俭:俭,俭朴无华。比喻茶叶中可溶于水的物质不多。

\chapter{六之饮}

\begin{yuanwen}
翼而飞,毛而走,[口去]而言(1),此三者俱生于天地间,饮啄以活,饮之时义远矣哉!至若救渴,饮之以浆;蠲忧忿(2),饮之以酒;荡昏寐,饮之以茶。

茶之为饮,发乎神农氏(3),闻于鲁周公(4),齐有晏婴(5),汉有杨雄、司马相如(6),吴有韦曜(7),晋有刘琨、张载、远祖纳、谢安、左思之徒(8),皆饮焉。滂时浸俗,盛于国朝,两都并荆俞[原注:俞,当作渝。巴渝也]间(9),以为比屋之饮。

饮有粗茶、散茶、末茶、饼茶者。乃斫、乃熬、乃炀、乃舂,贮于瓶缶之中,以汤沃焉,谓之[病字头下奄]茶(10)。或用葱、姜、枣、桔皮、茱萸、薄荷之等,煮之百沸,或扬令滑,或煮去沫,斯沟渠间弃水耳,而习俗不已。

于戏!天育有万物,皆有至妙,人之所工,但猎浅易。所庇者屋,屋精极;所著者衣,衣精极;所饱者饮食,食与酒皆精极之;[译者注:此处有脱文]茶有九难:一曰造,二曰别,三曰器,四曰火,五曰水,六曰炙,七曰末,八曰煮,九曰饮。阴采夜焙,非造也。嚼味嗅香,非别也。膻鼎腥瓯,非器也。膏薪庖炭,非火也。飞湍壅潦(11),非水也。非炙也。碧粉缥尘,非末也。操艰搅遽(12),非煮也。夏兴冬废,非饮也。

夫珍鲜馥烈者,其碗数三;次之者,碗数五。若座客数至五,行三碗;至七,行五碗;若六人以下,不约碗数,但阙一人而已,其隽永补所阙人。
\end{yuanwen}


[注释]
(1)[口去]而言:[口去],读音qi,张口。《集韵》:“启口谓之[口去]”。这里指开口会说话的人类。
(2)蠲忧忿:蠲,读音juan,免除。《史记.太史公自序》:“蠲除肉刑”。
(3)神农氏:传说中的上古三皇之一,教民稼穑,号神农,后世尊为炎帝。因有后人伪作的《神农本草》等书流传,其中提到茶,故云“发乎神农氏”。
(4)鲁周公:名姬旦,周文王之子,辅佐武王灭商,建西周王朝,“制礼作乐”,后世尊为周公,因封国在鲁,又称鲁周公。后人伪托周公作《尔雅》,讲到茶。
(5)晏婴:字仲,春秋之际大政治家,为齐国名相。相传著有《晏子春秋》,讲到他饮茶事。
(6)杨雄、司马相如:杨雄,见前注。司马相如(前178--前118),字子柳,蜀郡成都人。西汉著名文学家,著有《子虚赋》、《上林赋》等。 
(7)韦曜:字弘嗣,三国时人(220--280),在东吴历任中书仆射,太傅等要职。
(8)晋有刘琨、张载、远祖纳、谢安、左思之徒:刘琨(271--318),字越石,中山魏昌人(今河北无极县)。曾任西晋平北大将军等职。张载,字孟阳,安平人(今河北深县)。文学家,有《张孟阳集》传世。远祖纳,即陆讷(320?--395),字祖言,吴郡吴人(今江苏苏州)。东晋时任吏部尚书等职。陆羽与其同姓,故尊为远祖。谢安(319--385),字安石,陈国阳夏人(今河南太康县)。东晋名臣。历任太保、大都督等职。左思(250?--305?),字太冲,山东临淄人。著名文学家,代表作有《三都赋》、《咏史》诗等。
(9)两都并荆俞间:两都,长安和洛阳。荆州,治所在今湖北江陵。俞,当作渝。渝州。治所在今四川重庆一带。
(10)[病字头下奄]茶:[病字头下奄],读an,病。《博雅》:“病也”。 
(11)飞湍壅潦:飞湍,飞奔的急流。壅潦,停滞的积水。潦,雨后积水。
(12)操艰搅遽:操作艰难、慌乱。遽,读音ju,惶恐、窘急。

\chapter{七之事}

\begin{yuanwen}
三皇:炎帝神农氏。

周:鲁周公旦,齐相晏婴。

汉:仙人丹丘之子,黄山君,司马文园令相如,杨执戟雄。

吴:归命侯(1),韦太傅弘嗣。

晋:惠帝(2),刘司空琨,琨兄子兖州刺史演,张黄门孟阳(3),傅司隶咸(4),江洗马统(5),孙参军楚(6),左记室太冲,陆吴兴纳,纳兄子会稽内史俶,谢冠军安石,郭弘农璞,桓扬州温(7),杜舍人毓,武康小山寺释法瑶,沛国夏侯恺(8),余姚虞洪,北地傅巽,丹阳弘君举,乐安任育长(9),宣城秦精,敦煌单道开(10),剡县陈务妻,广陵老姥,河内山谦之。

后魏:琅邪王肃(11)。

宋:宋安王子鸾,鸾弟豫章王子尚(12),鲍昭妹令晖(13),八公山沙门谭济(14)。

齐:世祖武帝(15)。

梁:刘廷尉(16),陶先生弘景(17)。

皇朝:徐英公勣(18)。

《神农食经》(19):“荼茗久服,令人有力悦志”。

周公《尔雅》:“[木贾],苦荼”。

《广雅》云(20):“荆巴间采叶作饼,叶老者,饼成以米膏出之。欲煮茗饮,先炙令赤色,捣末,置瓷器中,以汤浇覆之,用葱、姜、桔子[上艹下毛]之。其饮醒酒,令人不眠。”

《晏子春秋》(21):“婴相齐景公时,食脱粟之饭,炙三戈、五卵茗菜而已。

司马相如《凡将篇》(22):“鸟喙,桔梗,芫华,款冬,贝母,木檗,蒌苓,X草,芍药,X桂,漏芦,蜚廉,萑菌,[上艹下舛]诧,白敛,白芷,菖蒲,芒消,莞椒,茱萸。”

《方言》:“蜀西南人谓荼曰[上艹下设]”。

《吴志.韦曜传》:“孙皓每飨宴,坐席无不悉以七胜为限,虽不尽入口,皆浇灌取尽。曜饮酒不过二升,皓初礼异,密赐茶[上艹下舛]以代酒。”

《晋中兴书》(23):陆纳为吴兴太守时,卫将军谢安尝欲诣纳,[原注:《晋书》以纳为吏部尚书。]纳兄子[亻叔]怪纳无所备,不敢问之,乃私蓄十数人馔。安既至,所设唯茶果而已。[亻叔]遂陈盛馔,珍羞必具。及安去,纳杖[亻叔]四十,云:‘汝既不能光益叔,奈何秽吾素业?’”

《晋书》:“桓温为扬州牧,性俭,每宴饮,唯下七奠[木半]茶果而已。”

《搜神记》(24):“夏侯恺因疾死,宗人字苟奴,察见鬼神,见恺来收马,并病其妻。著平上帻、单衣,入坐生时西壁大床,就人觅茶饮。”

刘琨“与兄子南兖州史演书”(25)云:“前得安州干姜一斤(26),桂一斤,黄岑一斤,皆所须也。吾体中溃[原注:溃,当作愦。]闷,常仰真茶,汝可致之。”

傅咸《司隶教》曰:“闻南方有蜀妪作茶粥卖,为廉事打破其器具,后又卖饼于市,而禁茶粥以因蜀妪何哉?”

《神异记》(27):余姚人虞洪,入山采茗,遇一道士,牵三青牛,引洪至瀑布山,曰:‘予,丹丘子也。闻子善具饮,常思见惠。山中有大茗,可以相给,祈子他日有瓯牺之余,乞相遗也’。因立奠祀。后常令家人入山,获大茗焉”。

左思《娇女诗》(28):“吾家有娇女,皎皎颇白皙。小字为纨素,口齿自清历。有姊字蕙芳,眉目灿如画。驰骛翔园林,果下皆生摘。贪华风雨中,倏忽数百适。心为荼[上艹下舛]剧,吹嘘对鼎[钅历]。”

张孟阳《登成都楼诗》(29)云:“借问扬子舍,想见长卿庐。程卓累千金,骄侈拟五侯。门有连骑客,翠带腰吴钩。鼎食随时进,百和妙且殊。披林采秋桔,临江钓春鱼。黑子过龙醢,吴馔逾蟹[虫胥]。芳荼冠六清,溢味播九区。人生苟安乐,兹土聊可娱。”

傅巽《七诲》:“蒲桃、宛柰,齐柿、燕栗,恒阳黄梨,巫山朱桔,南中荼子,西极石蜜。”

弘君举《食檄》:“寒温既毕,应下霜华之茗。三爵而终,应下诸蔗、木瓜、元李、杨梅、五味、橄榄、悬钩、葵羹各一杯。”

孙楚《歌》:“茱萸出芳树颠,鲤鱼出洛水泉。白盐出河东,美豉出鲁渊。姜桂茶[上艹下舛]出巴蜀,椒桔木兰出高。蓼苏出沟渠,精稗出中田。

华佗《食论》(30):“苦荼久食益意思。”
\end{yuanwen}


(1)归命侯:即孙皓。东吴亡国之君。公元280年,晋灭东吴,孙皓[口卸]壁投降,封“归命侯”。
(2)惠帝:晋惠帝司马衷,公元290--306年在位。
(3)张黄门孟阳:张载字孟阳,但未任过黄门侍郎。任黄门侍郎的是他的弟弟张协。
(4)傅司隶咸:傅咸(239--294),字长虞,北地泥阳人(今陕西铜川),官至司隶校尉,简称司隶。
(5)江洗马统:江统(?--310),字应元,陈留 县人(今河南杞县东)。曾任太子洗马。
(6)孙参军楚:孙楚(?--293),字子刑,太原中都人(今山西平遥县),曾任扶风的参军。
(7)桓扬州温:桓温(312--373),字元子,龙亢人(今安徽怀远县西)。曾任扬州牧等职。
(8)沛国夏侯恺:晋书无传。干宝《搜神记》中提到他。
(9)乐安任育长:任育长,生卒年不详,乐安人(今山东博兴一带)。名瞻,字育长,曾任天门太守待职。
(10)敦煌单道开:晋时著名道士,敦煌人。《晋书》有传。
(11)琅琊王肃:王肃(436--501)字恭懿,琅琊人(今山东临沂),北魏著名文士,曾任中书令待职。
(12)新安王子鸾、鸾弟豫章王子尚:刘子鸾、刘子尚,都是南北朝时宋孝武帝的儿子。一封新安王,一封豫章王。但子尚为兄,子鸾为弟。
(13)鲍昭妹令晖:鲍昭,即鲍昭(414--466)字明远,东海郡人(今江苏镇江),南朝著名诗人。其妹令晖,擅长词赋,钟嵘《诗品》说她:“歌待往往崭新清巧,拟古尤胜。”
(14)八公山沙门潭济:八公山,在今安徽寿县北。沙门,佛家指出家修行的人。潭济,即下文说的“覃济道人”。
(15)世祖武帝:南北朝时南齐的第二个皇帝,名肖[臣责],483--493在位。
(16)刘廷尉:刘孝绰(480--539),彭城人(今江苏徐州)。为梁昭明太子赏识,任太子仆兼延尉卿。
(17)陶先生弘景:陶弘景(456--536),字通明,秣陵人(今江苏宁县),有《神农本草经集注》传世。
(18)徐英公[责力]:徐世[责力](592--667),字懋功,启开国功臣,封英国公。
(19)神农食经:古书名,已佚。
(20)广雅:字书。三国时张辑撰,是对《尔雅》的补作。





[注释]
(21)宴子春秋:又称《晏子》,旧题齐晏婴撰,实为后人采晏子事辑成。成书约在汉初。此处陆羽引书有误。《晏子春秋》原为:“炙三戈五卵苔菜而矣”。不是“茗菜”。
(22)凡将篇:伪托司马相如作的字书。已佚。此处引文为后人所辑。X为脱漏字。
(23)晋中兴书:佚书。有清人辑存一卷。
(24)搜神记:东晋干宝著,计三十卷,为我国志怪小说之始。
(25)南兖州:晋时州名,治所在今江苏镇江市。
(26)安州:晋时州名。治所在今湖北安陆县一带。
(27)神异记:西晋王浮著。原书已佚。
(28)左思《娇女诗》:原诗五十六句,陆羽所引仅为有关茶的十二句。
(29)张孟阳《登成都楼诗》:张孟阳,见前注。原诗三十二句,陆羽仅录有关茶的十六句。
(30)华佗《食论》:华佗(约141--208),字元化。是东汉末著名医师。《三国志.魏书》有传.

卷下 七之事

壶居士《食忌》(31):“苦荼久食,羽化。与韭同食,令人体重。”

郭璞《尔雅注》云:“树小似栀子,冬生叶,可煮羹饮。今呼早取为荼,晚取为茗,或一曰[上艹下舛],蜀人名之苦荼”。

《世说》(32):“任瞻,字育长,少时有令名,自过江失志。既下饮,问人云:‘此为荼?为茗?’觉人有怪色,乃自申明云:‘向问饮为热为冷耳’。”

《续搜神记》(33):“晋武帝时,宣城市人秦精,常入武昌山采茗,遇一毛人,长丈余,引精至山下,示以丛茗而去。俄而复还,乃探怀中桔以遗精。精怖,负茗而归。”

《晋四王起事》(34):“惠帝蒙尘,还洛阳,黄门以瓦盂盛茶上至尊。”

《异[上艹下舛]》(35):“剡县陈务妻,少与二子寡居,好饮茶茗。以宅中有古冢,每饮,辄先祀之。儿子患之,曰:‘古冢何知?徒以劳意!’欲掘去之,母苦禁而止。其夜梦一人云:‘吾止此冢三百余年,卿二子恒欲见毁,赖相保护,又享吾佳茗,虽泉壤朽骨,岂忘翳桑之报(36)!’及晓,于庭中获钱十万,似久埋者,但贯新耳。母告二子惭之,从是祷馈愈甚。”

《广陵耆老传》:“晋元帝时,有老妪每旦独提一器茗,往市鬻之。市人竞买,自旦至夕,其器不减。所得钱散路旁孤贫乞人。人或异之。州法曹絷之狱中。至夜老妪执所鬻茗器从狱牖中飞出。”

《艺术传》(37):“敦煌人单道开,不畏寒暑,常服小石子,所服药有松、桂、蜜之气,所饮荼苏而已。”

释道该说《续名僧传》:“宋释法瑶,姓杨氏,河东人。元嘉中过江,遇沈台真君武康小山寺,年垂悬车。[原注:悬车,喻日入之候,指重老时也。《淮南子》(38)曰:“日至悲泉,爱息其马’,亦此意。]饭所饮荼。永明中,敕吴兴礼致上京,年七十九。”

宋《江氏家传》(39):“江统,字应,迁[民攵下心]怀太子洗马(40),尝上疏谏云:‘今西园卖醯(41)、面、蓝子、菜、茶之属,亏败国体’”。

《宋录》:“新安王子鸾、豫章王子尚,诣昙济道人于八公山。道人设荼茗,子尚味之,曰:‘此甘露也,何言荼茗?’”。
王微《杂诗》(42):“寂寂掩高阁,寥寥空广厦。待君竟不归,收领今就[木贾]。” 

鲍昭妹令晖著《香茗赋》。 

南齐世祖武皇帝《遗诏》(43):“我灵座上慎勿以牲为祭,但设饼果、茶饮、干饭、酒脯而已”。

梁刘孝绰《谢晋安王饷米等启》(44):“传诏李孟孙宣教旨,垂赐米、酒、瓜、笋、菹、脯、酢、茗八种。气[上艹下必]新城,味芳云松。江潭抽节,迈昌荇之珍。疆场擢翘,越葺精之美。羞非纯束野麋,[衣字中邑]似雪之驴;[鱼乍]异陶瓶河鲤,操如琼之粲。茗同食粲,酢类望柑。免千里宿春,省三月粮聚。小人怀惠,大懿难忘。”

陶弘景《杂录》:“苦荼,轻身换骨,昔旦丘子、黄山君服之。”

《后魏录》:“琅邪王肃,仕南朝,好茗饮、莼羹。及还北地,又好羊肉、酪浆。人或问之:‘茗何如酪?’肃曰:‘茗不堪与酪为奴’。”(45)

[注释]
(31)壶居士:道家臆造的真人之一,又称壶公。
(32)世说:即《世说新语》,南朝宋临川王刘义庆著,为我国志人小说之始。
(33)续搜神记:旧题陶潜著,实为后人伪托。
(34)晋四王起事:南朝卢[纟林]著。原书已佚。
(35)异苑:东晋末刘敬叔所撰。今存十卷。
(36)翳桑之报:翳桑,古地名。春秋时晋赵盾,曾在翳桑救了将要饿死的灵辄,后来晋灵公欲杀赵盾,灵辄扑杀恶犬,救出赵盾。后世称此事为“翳桑之报”。
(37)艺术传:即唐房玄龄所著《晋书.艺术列传》。
(38)淮南子:又名《淮南鸿烈》,为汉淮南王刘安及其门客所著。今存二十篇。
(39)江氏家传:南朝宋江饶著。已佚。
(40) 怀太子:晋惠帝之子,立为太子,元康元年(300年)为贾后害死,年仅二十一岁。
(41)醯:读xi,醋。陆德明《经典释文》:“醯,酢(醋)也。”
(42)王微《杂诗》:王微,南朝诗人。《杂诗》原二十八句,陆羽仅录四句。
(43)南齐世祖武帝《遗诏》:南朝齐武皇帝名肖[臣责]。《遗诏》写于齐永十一年(493年)。
(44)梁刘孝绰《谢晋安王饷米等启》:刘孝绰,见前注。他本名冉,孝绰是他的字。晋安王名肖纲,昭明太子卒后,继为皇太子。后登位称简文帝。 
(45)王肃事:王肃,本在南朝齐做官,后降北魏。北魏是北方少数民族鲜卑族拓跋部建立的政权,该民族习性喜食牛羊肉、鲜牛羊奶加工的酪浆。王肃为讨好新主子,所以当北魏高祖问他时,他贬低说茶还不配给酪浆作奴仆。这话传出后,北魏朝贵遂称茶为“酪奴”,并且在宴会时,“虽设茗饮,皆耻不复食”。[见《洛阳伽蓝记》]

卷下 七之事

《桐君录》(46):“西阳、武昌、庐江、晋陵好茗,(47)皆东人作清茗。茗有饽,饮之宜人。凡可饮之物,皆多取其叶,天门冬、拔葜取根,皆益人。又巴东别有真茗茶(48),煎饮令人不眠。俗中多煮檀叶并大皂李作荼,并冷。又南方有真瓜芦木、亦似茗,至苦涩,取为屑茶饮,亦可通夜不眠。煮盐人但资此饮,而交、广最重(49),客来先设,乃加以香[上艹下毛]辈。”

《坤元录》(50):“辰州溆浦县西北三百五十里无射山,云蛮俗当吉庆之时,亲族集会歌舞于山上。山多茶树。”

《括地图》(51):“临遂县东一百四十里有茶溪(52)。”

山谦之《吴兴记》(53):“乌程县西二十里有温泉山(54),出御[上艹下舛]。”

《夷陵图经》(55):“黄牛、荆门、女观、望州等山(56),茶茗出焉。”

《永嘉图经》:“永嘉县东三百里有白茶山”。(57)

《淮阴图经》:“山阳县南二十里有茶坡。”(58)

《茶陵图经》:“茶陵者,所谓陵谷生茶茗焉。”(59)

《本草.木部》(60):“茗——<苦茶>。味甘苦,微寒,无毒。主瘘疮,利小便,去痰渴热,令人少睡。秋采之苦,主下气消食。《注》云:‘春采之’。”

《本草.菜部》:“苦荼——一名荼,一名选,一名游冬,生益州川谷山陵道旁,凌冬不死。三月三日采干。《注》云:‘疑此即是今[木茶],一名茶,令人不眠。’《本草注》:‘按,《诗》云:“谁谓荼苦”(61),又云:“堇荼如饴(62)”,皆苦菜也,陶谓之苦茶,木类,非菜流。茗,春采谓之苦[木茶][上艹下亥][原注]:“迟途遐反]。’”

《枕中方》:“疗积年瘘,苦荼、蜈蚣并炙,令香熟,等分,捣筛,煮干草汤洗,以敷之。”

《孺子方》:“疗小儿无故惊蹶,以苦茶、葱须煮服之。”

[注释]
(46)桐君录:全名《桐君采药录》,已佚。
(47)西阳、武昌、庐江、晋陵:西阳、武昌、庐江、晋陵均为晋郡名,治所分别在今湖北黄冈、湖北武昌、安徽舒城、江苏常州一带。
(48)巴东:晋郡名。治所在今四川万县一带。
(49)交广:交州和广州。交州,在今广西合浦、北海市一带。
(50)坤元录:古地学书名,已佚。
(51)括地图:即《地括志》,诏肖德言等人著,已散佚,清人辑存一卷。
(52)临遂:晋时县名,今湖南衡东县。
(53)吴兴记:南朝宋山谦之著,共三卷。
(54)乌程县:县治所在今浙江湖州市。
(55)夷陵图经:夷陵,在今湖北宜昌地区,这是陆羽从方志中摘出自已加的书名。(下同)
(56)黄牛、荆门、女观、望州:黄牛山在今宜昌市向北八十里处。荆门山在今宜昌市东南三十里处。女观山在今宜都县西北。望州山在今宜昌市西。
(57)永嘉县:州治在今浙江温州市。
(58)山阳县:今称淮安县。
(59)茶陵:即今湖南茶陵县。
(60)本草、木部:《本草》即《唐新修本草》又称《唐本草》或《唐英本草》,因唐英国公徐[责力]任该书总监。下文《本草》同。
(61)谁谓荼苦:语出《诗经.谷风》:“谁谓荼苦,其甘如荠。”周秦时,荼作二解,一为茶,一为野菜。这里是野菜。
(62)堇荼如饴:语出《诗经.绵》:“周原[月无][月无],堇荼如饴”。荼也是野菜。

\chapter{八之出}

\begin{yuanwen}
山南\footnote{唐贞观十道之一。唐贞观元年,划全国为十道,道辖郡州,郡辖县。}:以峡州上\footnote{又称夷陵郡,治所在今湖北宜宾市。},[原注:峡州生远安、宜都、夷陵三县\footnote{即今湖北远安县、宜都县、宜昌市。}山谷。]襄州\footnote{今湖北襄樊市}、荆州\footnote{今湖北江陵县。}次,[原注:襄州生南漳县山谷(5),荆州生江陵县山谷。]衡州下\footnote{今湖南衡阳地区。},[原注:生衡山、茶陵二县山谷(7)。]金州、梁州又下(8)[原注:金州生西城、安康二县山谷(9)。梁州生褒城、金牛二县山谷(10)。]

淮南(11):以光州上(12),[原注:生光山县黄头港者,与峡州同。]义阳郡(13)、舒州次(14),[原注:生义阳县钟山者(15),与襄州同。舒州生太湖县潜山者(16),与荆州同。]寿州下(17),[原注:生盛唐县霍山者(18),与衡州同。]蕲州(19)、黄州又下(20)。[原注:蕲州生黄梅县山谷,黄州生麻城县山谷,并与金州、梁州同也。]

浙西(21):以湖州上(22),[原注:湖州生长城县(23)顾渚山谷(24),与峡州、光州同;若生山桑、儒师二寺、白茅山悬脚岭(25),与襄州、荆州、义阳郡同;生凤亭山伏翼阁、飞云曲水二寺(26)、啄木岭(27),与寿州同。生安吉、武康二县山谷,与金州、梁州同。]常州次(28),[原注:常州义兴县(29)生君山悬脚岭北峰下(30),与荆州、义阳君同;生圈岭善权寺(31)、石亭山,与舒州同。]宣州\footnote{又称宣城郡。今安徽宣城、当涂一带。}、杭州、睦州、歙州下(32),[原注:宣州生宣城县雅山(33),与蕲州同;太平县生上睦、临睦(34),与黄州同;杭州临安、于潜(35)二县生天目山(36),与舒州同。钱塘生天竺、灵隐二寺(37);睦州生桐庐县山谷;歙州生婺源山谷;与衡州同。]润州(38)、苏州又下(39)。[原注:润州江宁县生傲山(40),苏州长洲生洞庭山(41),与金州、蕲州、梁州同。]

剑南(42):以彭州上(43),[原注:生九陇县马鞍山至德寺、堋口(44),与襄州同。]绵州、蜀州次(45),[原注:绵州龙安县生松岭关(46),与荆州同,其西昌、昌明、神泉县西山者(47),并佳;有过松岭者,不堪采。蜀州青城县生八丈人山(48),与绵州同。青城县有散茶、末茶。]邛州次(49),雅州、泸州下(50),[原注:雅州百丈山、名山(51),泸州泸川者(52),与金州同也。]眉州(53)、汉州又下(54)。[原注:眉州丹棱县生铁山者,汉州绵竹县生竹山者(55),与润州同。]

浙东(56):以越州上(57),[原注:余姚县生瀑布泉岭曰仙茗,大者殊异,小者与襄州同。]明州(58)、婺州次(59),[原注:明州[贸阝]县生榆荚村(60),婺州东阳县东白山(61),与荆州同。]台州下(62),[原注:台州始丰县(63)生赤城者(64),与歙州同。]

黔中(65):生思州、播州、费州、夷州(66)。

江西(67):生鄂州、袁州、吉州(68)。

岭南(69):生福州、建州、韶州、象州(70)。[原注:福州生闽方山\footnote{在福建福州市闽江南岸。}山阴。]

其思、播、费、夷、鄂、袁、吉、福、建、、韶、象十一州未详,往往得之,其味极佳。
\end{yuanwen}

(7)衡山县:县治所在今衡阳朱亭镇对岸。
(8)金州、梁州:金州,今陕西安康一带;梁州,今陕西汉中一带。
(9)西城、安康:西城,今陕西安康市;安康,治所在今安康市城西五十里汉水西岸。
(10)褒城、金牛:褒城,今汉中褒城镇;金牛,今四川广元。
(11)淮南:唐贞观十道之一。
(12)光州:又称弋阳郡。今河南潢川、光山县一带。
(13)义阳郡:今河南信阳市及其边围。
(14)舒州:又名同安郡。今安徽太湖安庆一带。
(15)义阳县钟山:义阳县,今河南信阳。钟山,在信阳市东八十里。
(16)太湖县潜山:潜山,在安徽潜山县西北三十里。
(17)寿州:又名寿春郡。今安徽寿县一带。
(18)盛唐县霍山:盛唐县,今安徽六安县。霍山,在今霍山县境。
(19)蕲州:又名蕲州郡。今湖北蕲春一带。蕲,读qi。
(20)黄州:又名齐安郡。今湖北黄冈一带。
(21)浙西:唐贞观十道之一。
(22)湖州:又名吴兴郡。今浙江吴兴一带。
(23)长城县:今浙江长兴县。
(24)顾渚山:在长兴县西三十里。
(25)白茅山悬脚岭:在长兴县渚顾山东面。
(26)凤亭山:在长兴县西北四十里。伏翼阁、飞云寺、曲水寺,都是山里的寺院。
(27)啄木岭:在长兴县北六十里,山中多啄木鸟。
(28)常州:又名晋陵郡。今江苏常州市一带。
(29)义兴县:今江苏宜兴县。
(30)君山:在宜兴县南二十里。
(31)圈岭善权寺:善权,相传是尧时隐士。
(32)宣州、杭州、睦州、歙州:杭州,又名余杭郡。今浙江杭州、余杭一带。睦州,又称新定郡。今浙江建德、桐庐、淳安一带。歙州,又名新安郡。今安徽歙县、祁门一带。
(33)雅山:又称鸦山、鸭山、丫山。在宁国县北。
(34)目睦、临睦:太平县二乡名。
(35)于潜县:现已并入临安县。
(36)天目山:又名浮玉山。山脉横亘于浙江西、皖东南边境。
(37)钱塘生天竺、灵隐二寺:钱塘县,今浙江杭州市,灵隐寺在市西灵隐山下。天竺寺分上、中、下三寺。下天竺寺在灵隐飞来峰。
(38)润州:又称丹阳郡。今江苏镇江、丹阳一带。
(39)苏州:又称吴郡。今江苏苏州、吴县一带。
(40)江宁县傲山:江宁县在今南京市及江宁县。傲山在南京市郊。
(41)长州县洞庭山:长洲县在今苏州一带。洞庭山是太湖是的一些小岛。
(42)剑南:唐贞观十道之一。
(43)彭州:又叫[氵蒙]阳郡。今四川彭县一带。
(44)九陇县、马鞍山至德寺、堋口:九陇县,今彭县。马鞍山,即今至德山,在鼓城西。堋口,在鼓城西。
(45)锦州、蜀州:锦州,又称巴西郡,今四川绵阳、安县一带。蜀州,又称唐安郡,今四川崇庆、灌县一带。
(46)龙安县、松岭关:龙安县,今四川安县。松岭关,在今龙安县西五十里。
(47)西昌、昌明、神泉县、西山:西昌,在今四川安县东南花[上艹下亥]镇。昌明,在今四川江油县附近,神泉县,在安县南五十里。西山,岷山山脉之一部分。 
(48)青城县、丈人山:今四川灌县南四十里。因境内有青城山而得名。丈人山为青城山三十六峰之主峰。
(49)邛州:又称临邛郡。今四川邛峡、大邑一带。
(50)雅州、泸州:雅州又称卢山郡,今四川雅安一带。泸州,又称泸川郡,今四川泸州市及其周边。
(51)百丈山、名山:百丈山,在今四川名山县东四十里。名山,在名山县北。
(52)泸州县:今四川泸县。
(53)眉州:又名通义郡,今四川眉山、洪雅一带。
(54)汉州:又称德阳郡,今四川广汉、德阳一带。
(55)铁山、竹山:铁山,又名铁桶山,在四川丹陵县境内。竹山,即绵竹山,在四川绵竹县境内。
(56)浙东:浙江东道节度使方镇的简称。节度使驻地浙江绍光。
(57)越州:又称会稽郡。今浙江绍兴、嵊县一带。
(58)明州:又称余姚郡。今浙江宁波、奉化一带。
(59)婺州:又称东阳郡。今浙江金华、兰溪一带。
(60)[贸阝]县:今浙江宁波市东南的东钱湖畔。[贸阝],读音mao。
(61)东白山:在今浙江东阳县巍山镇北。
(62)台州:又名临海郡。今浙江临海,天台一带。
(63)始丰县:今浙江天台县。
(64)赤城:山名。天台山十景之一。
(65)黔中:唐开元十五道之一。
(66)思州、播州、费州、夷州:思州,又称宁夷郡。今贵州沿河一带。播州,又名播川郡,今贵州遵义一带。费州,又称涪川郡,今贵州思南、德江一带。夷州:又名义泉郡,今贵州风冈、绥阳一带。
(67)江西:江西团练观察使方镇的简称。观察使驻地在今江西南昌市。
(68)鄂州、袁州、吉州:鄂州,又称江夏郡。今湖北武昌、黄石一带。袁州,又名宜春郡。今江西吉安、宁冈一带。
(69)岭南:唐贞观十道之一。
(70)福州、建州、韶州、象州:福州,又名长乐郡。今福建福州、甫田一带。建州,又称建安郡。今福建建阳一带。韶州,又名始兴郡。今广东韶关、仁化一带。象州,又称象山郡。今广西象州县一带。

\chapter{九之略}

\begin{yuanwen}
其造具,若方春禁火\footnote{古时民间习俗。即在清明前一二日禁火三天,用冷食,叫“寒食节”。}之时,于野寺山园丛手而掇,乃蒸、乃舂,乃复以火干之,则[上启夂下木]、扑、焙、贯、棚、穿、育等七事皆废。

其煮器,若松间石上可坐,则具列废。用槁薪、鼎䥶之属,则风炉、灰承、炭挝、火䇲、交床等废。若瞰泉临涧,则水方、涤方、漉水囊废。若五人以下,茶可末而精者,则罗废。若援藟\footnote{lei,藤蔓。《广雅》:“藟,藤也”。}跻\footnote{ji,登、升。《释文》:“跻,升也。”}岩,引絙\footnote{geng,绳索。}入洞,于山口灸而末之,或纸包、盒贮,则碾、拂末等废。既瓢、碗、䇲、札、熟盂、鹾簋悉以一筥盛之,则都篮废。但城邑之中,王公之门,二十四器阙一,则茶废矣。
\end{yuanwen}



\chapter{十之图}

十之图:第十章,挂图。是指把《茶经》本文写在素绢上挂起来。《四库全书提要》说:“其曰图者,乃谓统上九类写绢素张之,非有别图。其类十,其文实九也”。

\begin{yuanwen}
以绢素或四幅、或六幅分布写之,陈诸座隅,则茶之源、之具、之造、之器、之煮、之饮、之事、之出、之略,目击而存\footnote{击,接触。此处作看见。俗语有“目击者”。},于是《茶经》之始终备焉。
\end{yuanwen}

\backmatter

\end{document}