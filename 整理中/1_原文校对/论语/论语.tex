% 论语
% 论语.tex

\documentclass[a4paper,12pt,UTF8,twoside]{ctexbook}

% 设置纸张信息。
\RequirePackage[a4paper]{geometry}
\geometry{
	%textwidth=138mm,
	%textheight=215mm,
	%left=27mm,
	%right=27mm,
	%top=25.4mm, 
	%bottom=25.4mm,
	%headheight=2.17cm,
	%headsep=4mm,
	%footskip=12mm,
	%heightrounded,
	inner=1in,
	outer=1.25in
}

% 设置字体,并解决显示难检字问题。
\xeCJKsetup{AutoFallBack=true}
\setCJKmainfont{SimSun}[BoldFont=SimHei, ItalicFont=KaiTi, FallBack=SimSun-ExtB]

% 目录 chapter 级别加点(.)。
\usepackage{titletoc}
\titlecontents{chapter}[0pt]{\vspace{3mm}\bf\addvspace{2pt}\filright}{\contentspush{\thecontentslabel\hspace{0.8em}}}{}{\titlerule*[8pt]{.}\contentspage}

% 设置 part 和 chapter 标题格式。
\ctexset{
	chapter/name={第,篇},
	chapter/number={\chinese{chapter}},
	section/name={},
	section/number={}
}

% 设置古文原文格式。
\newenvironment{yuanwen}{\bfseries\zihao{4}}

\title{\heiti\zihao{0} 论语}
\author{孔子弟子及再传弟子}
\date{春秋战国}

\begin{document}

\maketitle
\tableofcontents

\frontmatter

\mainmatter

\chapter{学而}

本篇内容涉及学习、为人和修养道德等方面,也有一些论政的语录。包括“学而时习”的学习方法,孝弟为本的仁学基础,不断反省的进德手段,节用爱人、使民以时的治国手段,先道德后文化的学习进程,“无友不如己者”的交友原则,过则能改的君子气度,“慎重追远”的行孝规定,“温良恭俭让”的行己作风,安贫乐道、敏行慎言的君子之德,推己及彼、举一反三的治学能力等。

\begin{yuanwen}
子\footnote{古代对于有地位、有学问的男子的尊称。有时也泛称男子。}曰\footnote{本书中“子曰”都是孔子所讲的话。}:“学而时\footnote{在一定的时候或者在适当的时候。}习\footnote{指演习礼乐,复习诗书。}之,不亦说\footnote{yu\`e,同“悦”,高兴、愉快的意思。}乎?有朋\footnote{指志同道合的人。}自远方来,不亦乐乎?人不知而不愠\footnote{y\`un,恼怒,怨恨。},不亦君子\footnote{此处指孔子理想中具有高尚人格的人。}乎?”
\end{yuanwen}

孔子说:“学了,然后按时实习,不也是很高兴的吗?有志同道合的人从远方来相会,不也是很快乐的吗?别人不了解自己,自己并不生气,不也是君子吗?”

\begin{yuanwen}
有子\footnote{孔子弟子。姓有,名若。《论语》中记载孔子弟子时一般称字,只对曾参和有若全部尊称为子,据此有很多人认为《论语》一书是曾参和有若的弟子记录而成的。}曰:“其为人也孝弟\footnote{t\`i,同“悌”,遵从兄长。},而好犯上者,鲜\footnote{xi\v{a}n,少。}矣!不好犯上,而好作乱者,未之有也。君子务\footnote{致力于。}本,本立而道生。孝弟也者,其为仁之本与\footnote{同“欤(y\'u)”,疑问语气词。}!”
\end{yuanwen}

有子说:“为人孝敬父母、尊敬兄长的,却喜欢冒犯上级,这种人很少。不喜欢冒犯上级,却喜欢造反作乱,这种人从来也没有过。君子致力于根本性工作,根本确立了,正道就随之产生。孝敬父母、尊敬兄长这些内容,大概就是施行‘仁’道的基础吧!”

\begin{yuanwen}
子曰:“巧言令色\footnote{好的脸色。这里指假装和善。},鲜矣仁!”
\end{yuanwen}

孔子说:“花言巧语、面貌伪善的人,仁德是很少的。”

\begin{yuanwen}
曾子\footnote{孔子弟子。姓曾,名参,字子舆。}曰:“吾日三省\footnote{x\v{i}ng。多次反省。古代汉语中动作性动词前加数字修饰成份的,一般表示动作的频率。而“三”、“九”等数字,一般表示次数多,不必落实为具体次数。}吾身:为人谋而不忠乎?与朋友交而不信乎?传\footnote{chu\'an,老师的传教。}不习乎?”
\end{yuanwen}

曾子说:“我每天多次自我反省:替别人谋划事情是否尽心竭力呢?与朋友交往是否诚实相待呢?老师传授的学业是否认真复习了?”

\begin{yuanwen}
子曰:“道\footnote{d\v{a}o,“导”的古体字,治理。}千乘\footnote{sh\`eng}\footnote{古代用四匹马拉的一辆兵车称为一乘。春秋战国时代,国力的强盛以该国所拥有的兵车的数量来计算。孔子生活之世,“千乘之国”已算不上是诸侯大国了,所以《论语》中有“千乘之国,摄乎大国之间”的话。}之国,敬事而信,节用而爱人,使民以时\footnote{按时,这里指不违背农时。}。”
\end{yuanwen}

孔子说:“治理拥有一千辆兵车的国家,就要严肃认真地对待工作,言而有信,节约用度,关爱百姓,不在农忙时节役使百姓。”

\begin{yuanwen}
子曰:“弟子入则孝,出则悌,谨而信,泛爱众,而亲仁。行有余力,则以学文。”
\end{yuanwen}

孔子说:“年轻人,在家就要孝顺父母,出门在外就要尊敬兄长,行为谨慎,言语有信,博爱众人,亲近仁者。这些都做到之后还有余力的话,就去学习文化。”

\begin{yuanwen}
子夏\footnote{孔子弟子。姓卜,名商,字子夏。孔子弟子中有所谓“四科十哲”之说,子夏长于“文学”。}曰:“贤贤易\footnote{轻视。}色\footnote{看重德行,轻视表面的姿态。},事父母,能竭其力,事君,能致\footnote{给予,献出。}其身,与朋友交,言而有信。虽曰未学,吾必谓之学矣。”
\end{yuanwen}

子夏说:“看重实际的德行,轻视表面的姿态。侍奉父母要竭尽全力,服务君主要奉献自身,与朋友交往要说话诚实有信。这样的人,虽说没有学习过,我也一定说他学习过了。”

\begin{yuanwen}
子曰:“君子不重则不威,学则不固\footnote{固执己见。}。主忠信\footnote{以下三句与前两句意思不相连贯,又见于其他篇内,疑是错简重出于此。},无友不如己者,过则勿惮改。”
\end{yuanwen}

孔子说:“称得上君子的人,如果不庄重就没有威严,知道学习就不会自以为是、顽固不化。恪守忠诚信实的道德要求,不与道德上不如自己的人交往,有了错误就不要怕改正。”

\begin{yuanwen}
曾子曰:“慎终\footnote{终与后边的远,分别指长辈丧亡之事和对于远祖的祭祀。},追远,民德归厚矣!”
\end{yuanwen}

曾子说:“恭敬慎重地办理父母的丧事,虔诚静穆地追祭历代的祖先,老百姓的道德就会趋向敦厚了。”

\begin{yuanwen}
子禽\footnote{陈亢,字子禽。从《子张》篇的记事来看,陈亢不是孔子的弟子,他对孔子的学说总是持怀疑的态度。}问于子贡\footnote{孔子弟子。姓端木,名赐。在“四科十哲”中属“言语”。}曰:“夫子\footnote{古人对于做过大夫的男子的敬称。孔子曾是鲁国的司寇(掌管刑狱的官员),所以他的学生称他为夫子,后来沿袭成对老师的称呼。在一定的场合下,又可以专指孔子。}至于是邦也,必闻其政。求之与?抑与之与?”

子贡曰:“夫子温、良、恭、俭\footnote{约束。}、让以得之。夫子之求之也,其诸\footnote{表示不肯定的推测语气。}异乎人之求之与!”
\end{yuanwen}

子禽问子贡说:“夫子每到一个国家,一定能够听到那个国家的政治状况,是求教得来的呢?还是人家主动告诉他的呢?”

子贡说:“先生温和、善良、恭敬、谨慎。谦让,是凭着这些德性得到的。先生求取的方法,大概不同于别人求取的办法吧!”

\begin{yuanwen}
子曰:“父在,观其\footnote{指代儿子。}志。父没\footnote{m\`o,死去。},观其行;三年无改于父之道,可谓孝矣。”
\end{yuanwen}

孔子说:“父亲在世的时候,要观察儿子的志向。父亲去世之后,要观察儿子的实际行动。如果能够多年不改变父亲传下来的正道的话,就可以说是尽孝了。”

\begin{yuanwen}
有子曰:“礼之用\footnote{施行。},和为贵。先王之道,斯\footnote{此,这。}为美,小大由之。有所不行,知和而和,不以礼节之,亦不可行也。”
\end{yuanwen}

有子说:“礼的施行,以和谐为美。前代君王的治道,最可贵的地方就在这里,大事小事都遵循这个道理。如果有行不通的地方,只是知道和谐为贵的道理而一味追求和谐,不懂得用礼来节制的道理的话,也是行不通的。”

\begin{yuanwen}
有子曰:“信近于义,言可复\footnote{因循,实践。}也。恭近于礼,远耻辱也。因\footnote{依靠,凭借。}不失其亲,亦可宗\footnote{尊重,推崇而效法。}也。”
\end{yuanwen}

有子说:“许下的诺言如果合乎义的话,这样的诺言就是可以遵循实践的。恭敬的样子如果合乎礼的话,就能够避开耻辱。依靠的人中不缺少关系深的,也就可靠了。”

\begin{yuanwen}
子曰:“君子食无求饱,居无求安,敏于事而慎于言,就\footnote{靠近。}有道而正\footnote{匡正。}焉。可谓好学也已。”
\end{yuanwen}

孔子说:“君子,吃饭不贪求满足,居住不贪求安逸,做事勤敏,说话谨慎,求教于有道德的人来端正自己,这样就可以说是好学的了。”

\begin{yuanwen}
子贡曰:“贫而无谄\footnote{ch\v{a}n,巴结,奉承。},富而无骄,何如?”

子曰:“可也。未若贫而乐,富而好礼者也。”

子贡曰:“《诗》云:‘如切如磋,如琢如磨\footnote{《诗经·卫风·淇奥》中的句子。切、磋、琢、磨都是制作器物时反复修治的动作,这里用来比喻治学、修身要精益求精。}’,其斯之谓与?”

子曰:“赐也,始可与言《诗》已矣!告诸往而知来者。”
\end{yuanwen}

子贡说:“贫穷却不谄媚,富有却不骄纵,人能做到这些怎么样?”

孔子说:“可以了。但是不如贫穷却能怡然自乐,富贵却能谦逊好礼。”

子贡说:“《诗经》里说:‘像制造器物一样,切割、磋治、雕琢、打磨’,大概就是说这类反复修治、精益求精的事吧!”

孔子说:“赐呀,可以和你讨论《诗经》了,告诉你一件事,就可以推知另一件事。”

\begin{yuanwen}
子曰:“不患人之不己知,患不知人也。”
\end{yuanwen}

孔子说:“不担心别人不了解自己,担心的是自己不了解别人。”

\chapter{为政}

本篇全都是孔子的语录。提及的人则有鲁国国君、鲁国大夫、孔子弟子等,据此可以了解孔子为众人师表的情况。

本篇论及为政、教化、学习、修养、孝道等方面的内容。孔子主张德政礼治:认为治政必须以教化百姓为首任,从政必须以学习为前提,对于有疑问之事采取谨慎的态度;国君要任用正直之人来辅政,当政者都要从修养自身做起,以使社会形成普遍的道德风气:友爱、孝悌、讲信用。还指出了教学科目的特点,概述了自己为学进德的经历,提倡学思并重的学习方法,反对研习具有极端倾向的学说。

对孝道的涵义做了集中阐释:能够按照礼的规定办事,无论是父母在世时的赡养义务,还是父母去世后的悼念程序,这样就是尽孝。不要违背礼的规定,不要让父母为自己担忧,父母亲所有的担心只出现在疾病这一非人力可以控制的范围内,这也是对父母孝顺的方式。孝敬父母突出在这个“敬”字上,这种感情史人类所特有的,要在日常与长者的交往中显示出来。虽然在孔子的时代,敬顺之情明显是受到等级制的影响而产生的,时至今日则完全可以用对于长者的尊敬来代替。

谈到考察人的品性要以行动为依据。

分析了君子的特点:多才多能、堪当重任、积极实践、言行一致。

总结了政治文化世代继承的特点。

\begin{yuanwen}
子曰:“为政以德,譬如北辰\footnote{北极星。《尔雅·释天》:“北极谓之北辰。”},居其所而众星共\footnote{通“拱”,环抱、环绕之意。这里是以北辰比喻统治者,以众星比喻被统治者。}之。”
\end{yuanwen}

孔子说:“当政者运用道德来治理国政,就好像北极星,安居其所,而其他众星井然有序地环绕着它。”

\begin{yuanwen}
子曰:“《诗》\footnote{《诗经》}三百\footnote{概举整数而言。《诗经》实有三百零五篇诗,连同有题无辞的六篇笙诗,共三百一十一篇。},一言以蔽之,曰:‘思无邪\footnote{《诗经·鲁颂·駉(ji\=ong)》中的句子,孔子借用来评价《诗经》各篇思想内容的纯正。}’。”
\end{yuanwen}

孔子说:“《诗》三百篇,用一句话来总括它,就是‘思想主旨纯正无邪’。”

\begin{yuanwen}
子曰:“道\footnote{同“导”,引导。}之以政\footnote{法制,禁令。},齐\footnote{整饬(ch\`i)。}之以刑,民免\footnote{逃避。}而无耻。道之以德,齐之以礼,有耻且格\footnote{至,来。}。”
\end{yuanwen}

孔子说:“用政令来训导百姓,用刑罚来整饬百姓,百姓只会尽量地避免获罪,却没有羞耻心;用道德来引导人民,用礼教来整饬人民,人民就会有羞耻心而且归顺。”

\begin{yuanwen}
子曰:“吾十有\footnote{通“又”。古人十五岁为入学之年,《礼记·王制》“立四教”。郑玄注引《尚书传》曰:“年十五始入小学,年十八入大学。”}五而志于学,三十而立\footnote{指立身行事。《论语》一书中多有以“礼”为立身行事基本原则的说法。},四十而不惑,五十而知天命\footnote{懂得天命不可抗拒而听天由命。},六十而耳顺,七十而从心所欲,不逾矩。”
\end{yuanwen}

孔子说:“我十五岁立志于学习;三十岁能依照礼仪的要求立足于世;四十岁不再感到困惑;五十岁能乐天知命;六十岁能听得进各种不同的意见;七十岁能随心所欲地行事,而又从不超出规矩。”

\begin{yuanwen}
孟懿子\footnote{鲁国大夫。姓仲孙,名何忌。“懿”是谥(sh\`i)号(死后所得的尊号)。}问孝。子曰:“无违。”

樊迟\footnote{孔子弟子。姓樊,名须,字子迟。}御,子告之曰:“孟孙问孝于我,我对曰,‘无违’。”

樊迟曰:“何谓也?”

子曰:“生,事之以礼;死,葬之以礼,祭之以礼。”
\end{yuanwen}

孟懿子文什么是孝。孔子说:“不要违背礼的规定。”

樊迟为孔子驾御马车,孔子告诉他说:“孟孙向我询问怎样才算是孝,我回答说,‘不要违背礼的规定’。”

樊迟说:“这话是什么意思?”

孔子说:“父母在世的时候,按照礼的要求来服侍他们;去世以后,按照礼的要求来安葬他们,按照礼的要求来祭祀他们。”

\begin{yuanwen}
孟武伯\footnote{孟懿子的儿子。姓仲孙,名彘。“武”是谥号。}问孝。子曰:“父母唯其\footnote{指代子女。}疾之忧。”
\end{yuanwen}

孟武伯问什么是孝。孔子说:“父母对于子女,只为他们的疾病担忧。”

\begin{yuanwen}
子游\footnote{孔子弟子。姓言,名偃,字子游,吴人。在“四科十哲”中属“文学”。}问孝。子曰:“今之孝者,是谓能养。至于\footnote{就连,就是。表示提起另一件事。}犬马,皆能有养。不敬,何以别乎?”
\end{yuanwen}

子游问什么是孝。孔子说:“如今所谓的孝,只是就能养活父母而言。说到狗、马这些动物,都能被人饲养。如果对父母没有敬顺的心意,用什么来区别孝顺和饲养呢?”

\begin{yuanwen}
子夏问孝。子曰:“色\footnote{指敬爱和悦的容色态度。}难。有事,弟子服其劳;有酒食,先生\footnote{年长者。}馔\footnote{zhu\`an,吃喝。}。曾\footnote{乃,竟。}是以为孝乎?”
\end{yuanwen}

子夏问什么是孝。孔子说:“保持敬爱和悦的容态最难。遇有事情,年轻人替长者们效劳;遇有酒食,让给长者享用,仅仅这样就算是孝了吗?”

\begin{yuanwen}
子曰:“吾与回\footnote{孔子弟子颜回。字子渊,鲁国人。在“四科十哲”中属“德行”,是孔子所喜爱的最聪慧最有修养的一个学生。}言,终日不违\footnote{不违拗。},如愚。退\footnote{指散学回去。}而省其私\footnote{独处。这里指独自钻研和自我实践。},亦足以发\footnote{发挥。},回也不愚。”
\end{yuanwen}

孔子说:“我给颜回讲学,他整天从不表示异议,像是一个愚笨的人。等回去之后,省察他的钻研和实践,又能发挥所学的内容,颜回并不愚笨啊!”

\begin{yuanwen}
子曰:“视其所以\footnote{作为,行动。},观其所由\footnote{经由,经历。},察其所安\footnote{习。},人焉\footnote{怎样。}廋\footnote{s\=ou,隐藏。}哉?人焉廋哉?”
\end{yuanwen}

孔子说:“注意看他的所作所为,观察他的一贯经历,考察他的秉性习惯,一个人怎么能隐藏得住呢?一个人怎么能隐藏得住呢?”

\begin{yuanwen}
子曰:“温故而知新,可以为师矣。”
\end{yuanwen}

孔子说:“温习旧的知识,而能在其中获得新的体会,这样的人可以做老师了。”

\begin{yuanwen}
子曰:“君子不器。”
\end{yuanwen}

孔子说:“君子不能像器皿一样只有单一的用途。”

\begin{yuanwen}
子贡问君子。子曰:“先行其言,而后从之。”
\end{yuanwen}

子贡问怎样才能算是君子。孔子说:“先实践所要说的话,然后再把话说出来。”

\begin{yuanwen}
子曰:“君子周\footnote{合。}而不比\footnote{齐同。},小人比而不周。”
\end{yuanwen}

孔子说:“君子团结而不勾结,小人勾结而不团结。”

\begin{yuanwen}
子曰:“学而不思则罔\footnote{w\v{a}ng,无知的样子。},思而不学则殆\footnote{d\`ai,疑惑。}。”
\end{yuanwen}

孔子说:“只是学习,却不思考,就会惘然无知。只是思考,却不学习,就会疑惑不解。”

\begin{yuanwen}
子曰:“攻\footnote{从事某事,进行某项工作。}乎异端\footnote{历来的注疏多释为错误的学说或危险思想,而与孔子本人的学说相对。实际上,汉以前的古书没有以“邪说”为“异端”的记载。另外,《论语》中“异”字凡八见,多数情况下可释为“不同的”,因此此处的“异”作“不同的”解为佳。端,顶头,极。所以“异端”应该相当于“我叩其两端而竭焉”中的“两端”,也就是“过犹不及”中的“过”与“不及”这两端。},斯害也已\footnote{语气词连用,表示肯定。}。”
\end{yuanwen}

孔子说:“攻治两极的学说,这是一种祸害啊!”

\begin{yuanwen}
子曰:“由\footnote{仲由,孔子弟子,字子路,卞(bi\`an)人(今山东人)。在“四科十哲”中属“政事”。},诲女\footnote{通“汝”,第二人称代词,你。}知\footnote{同“智”。}之乎?知之为知之,不知为不知,是知也。”
\end{yuanwen}

孔子说:“由,教导你的内容都知道了吧?知道就是知道,不知道就是不知道,这才是有智慧。”

\begin{yuanwen}
子张\footnote{颛孙师,孔子弟子,字子张。}学干\footnote{求。}禄\footnote{官俸。}。子曰:“多闻阙疑,慎言其余,则寡尤\footnote{过失。};多见阙殆,慎行其余,则寡悔。言寡尤,行寡悔,禄在其中矣。”
\end{yuanwen}

子张向孔子学习求仕的方法。孔子说:“多聆听,对于有疑问的地方保留不言,其余有把握的地方,谨慎的发表意见,这样就可以少犯错。多观察,对于有疑问的地方保留不言,其余有把握的地方,谨慎地采取行动,这样就可以少后悔。言语方面少犯错误,行动方面避免后悔,官职俸禄就在这里面了。”

\begin{yuanwen}
哀公\footnote{鲁国的国君。姓姬,名蒋,公元前494-前466年在位。“哀”是谥号。}问曰:“何为则民服?”

孔子对曰:“举直错\footnote{放置。}诸枉\footnote{邪曲不正。},则民服;举枉错诸直,则民不服。”
\end{yuanwen}

鲁哀公问道:“怎么做才能使人民服从呢?”

孔子回答说:“选用正直的人,让他们居于邪曲之人的上位,这样百姓就会服从了。如果选用邪曲之人,让他们居于正直之人的上位,百姓就不会服从。”

\begin{yuanwen}
季康子\footnote{季孙肥,鲁哀公时的正卿,是当时最有权力的政治人物。“康”是谥号。}问:“使民敬、忠以\footnote{连词,和。}劝\footnote{勤勉。},如之何?”

子曰:“临之以庄则敬;孝慈则忠;举善而教不能则劝。”
\end{yuanwen}

季康子问道:“要使人民敬顺、忠诚又勤勉,应该怎么做呢?”

孔子说:“当政者对待百姓庄重,百姓就会敬顺;对待父母孝顺,百姓就会忠诚;提拔好人,教导能力不足之人,百姓就会勤勉。”

\begin{yuanwen}
或\footnote{不定代词,有人。}谓孔子曰:“子奚\footnote{疑问词,为何。}不为政。”

子曰:“《书》云:‘孝乎惟孝,友于兄弟,施\footnote{延及。}于有政。’\footnote{这三句是《尚书》的佚文。}是亦为政,奚其为为政?”
\end{yuanwen}

有人对孔子说:“你为什么不从事政治?”

孔子说:“《尚书》说:‘孝敬父母,友爱兄弟,用这种风气去影响当政者。’这也是从事政治了,为什么一定要做官才算从事政治呢?”

\begin{yuanwen}
子曰:“人而\footnote{若。}无信,不知其可也。大车无輗\footnote{n\'i,车辕与驾辕的横木相衔接的活销。},小车无軏\footnote{yu\`e,车辕前端与车横衔接处的关键。},其何以行之哉?”
\end{yuanwen}

孔子说:“人如果没有信用,不知道那怎么可以。大车如果没有安装横木的輗,小车如果没有安装横木的軏,怎么能够行车呢?”

\begin{yuanwen}
子张问:“十世可知也?”

子曰:“殷因\footnote{承袭。}于夏礼,所损益可知也;周因于殷礼,所损益可知也;其或继周者,虽\footnote{即使。}百世可知也。”
\end{yuanwen}

子张问道:“今后十代的情况可以知道吗?”

孔子说:“殷代承袭夏代的礼仪制度,废除的和增加的是可以知道的。周代承袭殷代的礼仪制度,废除的和增加的是可以知道的。如果有继承周代统治的政权,即使有百代也是可以知道的。”

\begin{yuanwen}
子曰:“非其鬼\footnote{一般指死去的祖先而言。}而祭之,谄也。见义不为,无勇也。”
\end{yuanwen}

孔子说:“不是自己该祭祀的鬼神而去祭祀他,这是谄媚的行为。遇见正义的事却袖手旁观,这是没有胆量。”

\chapter{八佾}

本篇内容多与礼、乐有关,比较集中地反映了孔子的礼乐思想。

\begin{yuanwen}
孔子谓\footnote{说,用于评论人物。}季氏\footnote{text}:“八佾\footnote{text}舞于庭,是可忍也,孰不可忍也?”
\end{yuanwen}

孔子说:“。”

\begin{yuanwen}
三家\footnote{text}者以《雍》\footnote{text}彻\footnote{text}。

子曰:“‘相\footnote{text}维辟公\footnote{text},天子穆穆’\footnote{text},奚取于三家之堂?”
\end{yuanwen}

孔子说:“。”

\begin{yuanwen}
子曰:“人而不仁,如礼何?人而不仁,如乐何?”
\end{yuanwen}

孔子说:“。”

\begin{yuanwen}
林放\footnote{text}问礼之本。

子曰:“大哉问!礼,与其奢也,宁俭;丧,与其易\footnote{text}也,宁戚。”
\end{yuanwen}

孔子说:“。”

\begin{yuanwen}
子曰:“夷狄\footnote{text}之有君,不如诸夏\footnote{text}之亡\footnote{text}也。”
\end{yuanwen}

孔子说:“。”

\begin{yuanwen}
季氏旅\footnote{text}于泰山。

子谓冉有\footnote{text}曰:“女弗能救\footnote{text}与?”

对曰:“不能。”

子曰:“呜呼!曾\footnote{text}谓泰山不如林放\footnote{text}乎?”
\end{yuanwen}

孔子说:“。”

\begin{yuanwen}
子曰:“君子无所争,必也射\footnote{text}乎!揖让而升,下而饮。其争也君子。”
\end{yuanwen}

孔子说:“。”

\begin{yuanwen}
子夏问曰:“‘巧笑倩\footnote{text}兮,美目盼\footnote{text}兮,素以为绚\footnote{text}兮\footnote{text}。’何谓也?”

子曰:“绘事后素\footnote{text}。”

曰:“礼后乎?”

子曰:“起\footnote{text}予者商也,始可与言《诗》已矣!”
\end{yuanwen}

孔子说:“。”

\begin{yuanwen}
子曰:“夏礼,吾能言之,杞\footnote{text}不足征也;殷礼,吾能言之,宋\footnote{text}不足征也。文献\footnote{text}不足故也,足,则吾能征之矣。”
\end{yuanwen}

孔子说:“。”

\begin{yuanwen}
子曰:“禘\footnote{text},自既灌\footnote{text}而往者,吾不欲观之矣。”
\end{yuanwen}

孔子说:“。”

\begin{yuanwen}
或问禘之说。子曰:“不知也。知其说者之于天下也,其如示\footnote{text}诸斯乎!”指其掌。
\end{yuanwen}

孔子说:“。”

\begin{yuanwen}
祭如在\footnote{text},祭神如神在。子曰:“吾不与\footnote{text}祭,如不祭。”
\end{yuanwen}

孔子说:“。”

\begin{yuanwen}
王孙贾\footnote{text}问曰:“‘与其媚于奥\footnote{text},宁媚于灶\footnote{text}’,何谓也?”

子曰:“不然,获罪于天,无所祷也。”
\end{yuanwen}

孔子说:“。”

\begin{yuanwen}
子曰:“周监\footnote{text}于二代\footnote{text},郁郁乎文哉!吾从周。”
\end{yuanwen}

孔子说:“。”

\begin{yuanwen}
子入太庙\footnote{text},每事问。

或曰:“孰谓鄹\footnote{text}人之子知礼乎?入太庙,每事问。”

子闻之,曰:“是礼也。”
\end{yuanwen}

孔子说:“。”

\begin{yuanwen}
子曰:“射不主皮\footnote{text},为\footnote{text}力不同科\footnote{text},古之道也。”
\end{yuanwen}

孔子说:“。”

\begin{yuanwen}
子贡欲去告朔\footnote{text}之饩羊\footnote{text}。子曰:“赐也,尔爱其羊,我爱其礼。”
\end{yuanwen}

孔子说:“。”

\begin{yuanwen}
子曰:“事君尽礼,人以为谄也。”
\end{yuanwen}

孔子说:“。”

\begin{yuanwen}
定公\footnote{text}问:“君使臣,臣事君,如之何?”

孔子对曰:“君使臣以礼,臣事君以忠。”
\end{yuanwen}

孔子说:“。”

\begin{yuanwen}
子曰:“《关雎》\footnote{text}乐而不淫\footnote{text},哀而不伤。”
\end{yuanwen}

孔子说:“。”

\begin{yuanwen}
哀公问社\footnote{text}于宰我\footnote{text}。

宰我对曰:“夏后氏以松,殷人以柏,周人以栗,曰使民战栗。”

子闻之,曰:“成事不说,遂事不谏,既往不咎。”
\end{yuanwen}

孔子说:“。”

\begin{yuanwen}
子曰:“管仲\footnote{text}之器小哉!”

或曰:“管仲俭乎?”

曰:“管氏有三归\footnote{text},官事不摄\footnote{text},焉得俭?”

“然则管仲知礼乎?”

曰:“邦君树\footnote{text}塞\footnote{text}门,管氏亦树塞门;邦君为两君之好,有反坫\footnote{text},管氏亦有反坫。管氏而\footnote{text}知礼,孰不知礼?”
\end{yuanwen}

孔子说:“。”

\begin{yuanwen}
子语\footnote{text}鲁大师\footnote{text}乐,曰:“乐其可知也:始作,翕\footnote{text}如\footnote{text}也;从\footnote{text}之,纯\footnote{text}如也,皦\footnote{text}如也,绎\footnote{text}如也,以成。”
\end{yuanwen}

孔子说:“。”

\begin{yuanwen}
仪\footnote{text}封人\footnote{text}请见,曰:“君子之至于斯也,吾未尝不得见也。”

从者见之。出,曰:“二三子何患于丧乎?天下之无道也久矣,天将以夫子为木铎\footnote{text}。”
\end{yuanwen}

孔子说:“。”

\begin{yuanwen}
子谓《韶》\footnote{text}:“尽美矣,又尽善也。”

谓《武》\footnote{text}:“尽美矣,未尽善也。”
\end{yuanwen}

孔子说:“。”

\begin{yuanwen}
子曰:“居上不宽\footnote{text},为礼不敬,临丧\footnote{text}不哀,吾何以观之哉?”
\end{yuanwen}

孔子说:“。”

\chapter{里仁}

本篇大多论及道德修养的问题,包括仁、义、利、礼、孝、言、行、事君、交友等内容。

\begin{yuanwen}
子曰:“里\footnote{text}仁为美。择不处仁,焉得知\footnote{text}?”
\end{yuanwen}

孔子说:“。”

\begin{yuanwen}
子曰:“不仁者,不可以久处约\footnote{text},不可以长处乐。仁者安仁,知者利仁。”
\end{yuanwen}

\begin{yuanwen}
子曰:“唯仁者能好\footnote{text}人,能恶\footnote{text}人。”
\end{yuanwen}

孔子说:“。”

\begin{yuanwen}
子曰:“苟志于仁矣,无恶\footnote{text}也。”
\end{yuanwen}

孔子说:“。”

\begin{yuanwen}
子曰:“富与贵,是人之所欲也;不以其道得之,不处也。贫与贱,是人之所恶\footnote{text}也;不以其道得之,不去也。君子去仁,恶乎\footnote{text}成名?君子无终食之间违仁,造次\footnote{text}必于是,颠沛\footnote{text}必于是。”
\end{yuanwen}

孔子说:“。”

\begin{yuanwen}
子曰:“我未见好仁者、恶不仁者。好仁者,无以尚\footnote{超过。}之;恶不仁者,其为仁矣,不使不仁者加乎其身。有能一日用其力于仁矣乎?我未见力不足者。盖\footnote{大概。}有之矣,我未之见也。”
\end{yuanwen}

孔子说:“。”

\begin{yuanwen}
子曰:“人之过也,各于其党\footnote{text}。观过,斯知仁\footnote{text}矣。”
\end{yuanwen}

\begin{yuanwen}
子曰:“朝闻道,夕死可矣。”
\end{yuanwen}

\begin{yuanwen}
子曰:“士志于道,而耻恶衣恶食者,未足与议\footnote{text}也。”
\end{yuanwen}

\begin{yuanwen}
子曰:“君子之于天下也,无适\footnote{text}也,无莫\footnote{text}也,义之与比\footnote{text}。”
\end{yuanwen}

\begin{yuanwen}
子曰:“君子怀德,小人怀土;君子怀刑,小人怀惠\footnote{text}。”
\end{yuanwen}

\begin{yuanwen}
子曰:“放\footnote{text}于利而行,多怨。”
\end{yuanwen}

\begin{yuanwen}
子曰:“能以礼让为国乎?何有\footnote{text}!不能以礼让为国,如礼何?”
\end{yuanwen}

\begin{yuanwen}
子曰:“不患无位,患所以立\footnote{text}。不患莫己知,求为可知\footnote{text}也。”
\end{yuanwen}

\begin{yuanwen}
子曰:“参乎!吾道一以贯\footnote{text}之。”

曾子曰:“唯。”

子出。门人问曰:“何谓也?”

曾子曰:“夫子之道,忠\footnote{text}恕\footnote{text}而已矣!”
\end{yuanwen}

\begin{yuanwen}
子曰:“君子喻\footnote{text}于义,小人喻于利。”
\end{yuanwen}

\begin{yuanwen}
子曰:“见贤思齐焉,见不贤而内自省也。”
\end{yuanwen}

\begin{yuanwen}
子曰:“事父母几\footnote{text}谏。见志不从,又敬不违,劳\footnote{text}而不怨。”
\end{yuanwen}

\begin{yuanwen}
子曰:“父母在,不远游\footnote{text},游必有方\footnote{text}。”
\end{yuanwen}

\begin{yuanwen}
子曰:“三年无改于父之道,可谓孝矣。”\footnote{text}
\end{yuanwen}

\begin{yuanwen}
子曰:“父母之年,不可不知也。一则以喜,一则以惧。”
\end{yuanwen}

\begin{yuanwen}
子曰:“古者言之不出,耻躬\footnote{text}之不逮\footnote{text}也。”
\end{yuanwen}

\begin{yuanwen}
子曰:“以约\footnote{text}失之者,鲜矣!”
\end{yuanwen}

\begin{yuanwen}
子曰:“君子欲讷\footnote{text}于言,而敏于行。”
\end{yuanwen}

\begin{yuanwen}
子曰:“德不孤,必有邻。”
\end{yuanwen}

\begin{yuanwen}
子游曰:“事君数\footnote{text},斯辱矣;朋友数,斯疏矣。”
\end{yuanwen}


\chapter{公冶长}

\begin{yuanwen}
子谓公冶长\footnote{text}:“可妻\footnote{text}也。虽在缧绁\footnote{text}之中,非其罪也。”以其子\footnote{text}妻之。
\end{yuanwen}

\begin{yuanwen}
子谓南容\footnote{text}:“邦有道,不废;邦无道,免于刑戮\footnote{text}。”以其兄之子妻之。
\end{yuanwen}

\begin{yuanwen}
子谓子贱\footnote{text}:“君子哉若\footnote{text}人!鲁无君子者,斯焉取斯?”
\end{yuanwen}

\begin{yuanwen}
子贡问曰:“赐也何如?”

子曰:“女\footnote{text},器也。”

曰:“何器也?”

曰:“瑚琏\footnote{text}也。”
\end{yuanwen}

\begin{yuanwen}
或曰:“雍\footnote{text}也仁而不佞\footnote{text}。”

子曰:“焉用佞?御人以口给\footnote{text},屡憎于人。不知其仁\footnote{text},焉用佞?”
\end{yuanwen}

\begin{yuanwen}
子使漆雕开\footnote{text}仕。对曰:“吾斯之未能信。”子说。
\end{yuanwen}

\begin{yuanwen}
子曰:“道不行,乘桴\footnote{text}浮于海。从我者,其由与?”

子路闻之喜。

子曰:“由也,好勇过我,无所取材\footnote{text}。”
\end{yuanwen}

\begin{yuanwen}
孟武伯问:“子路仁乎?”

子曰:“不知也。”

又问。子曰:“由也,千乘之国,可使治其赋\footnote{text}也。不知其仁也。”

“求也何如?”

子曰:“求也,千室之邑\footnote{text},百乘之家\footnote{text},可使为之宰\footnote{text}也。不知其仁也。”

“赤\footnote{text}也何如?”

子曰:“赤也,束带\footnote{text}立于朝,可使与宾客言也。不知其仁也。”
\end{yuanwen}

\begin{yuanwen}
子谓子贡曰:“女\footnote{text}与回也孰愈\footnote{text}?”

对曰:“赐也何敢望\footnote{text}回?回也闻一以知十,赐也闻一以知二。”

子曰:“弗如也!吾与\footnote{text}女弗如也!”
\end{yuanwen}

\begin{yuanwen}
宰予昼寝。子曰:“朽木不可雕也,粪土之墙不可杇\footnote{text}也,于予与何诛\footnote{text}?”

子曰:“始吾于人也,听其言而信其行;今吾于人也,听其言而观其行。于予与改是。”
\end{yuanwen}

\begin{yuanwen}
子曰:“吾未见刚者!”

或对曰:“申枨\footnote{text}。”

子曰:“枨也欲\footnote{text},焉得刚?”
\end{yuanwen}

\begin{yuanwen}
子贡曰:“我不欲人之加诸我也,吾亦欲无加诸人。”

子曰:“赐也,非尔所及也。”
\end{yuanwen}

\begin{yuanwen}
子贡曰:“夫子之文章\footnote{text},可得而闻也;夫子之言性\footnote{text}与天道\footnote{text},不可得而闻也。”
\end{yuanwen}

\begin{yuanwen}
子路有\footnote{text}闻,未之能行,唯恐有闻。
\end{yuanwen}

\begin{yuanwen}
子贡问曰:“孔文子\footnote{text}何以谓之‘文’也?”

子曰:“敏\footnote{text}而好学,不耻下问,是以谓之‘文’也。”
\end{yuanwen}

\begin{yuanwen}
子谓子产\footnote{text}:“有君子之道四焉:其行己\footnote{text}也恭,其事上也敬,其养民也惠,其使民也义。”
\end{yuanwen}

\begin{yuanwen}
子曰:“晏平仲\footnote{text}善与人交,久而敬之\footnote{text}。”
\end{yuanwen}

\begin{yuanwen}
子曰:“臧文仲\footnote{text}居蔡\footnote{text},山\footnote{text}节\footnote{text}藻\footnote{text}棁\footnote{text},何如其知\footnote{text}也\footnote{text}?”
\end{yuanwen}

\begin{yuanwen}
子张问曰:“令尹\footnote{text}子文\footnote{text}三仕为令尹,无喜色;三已\footnote{text}之,无愠色。旧令尹之政,必以告新令尹。何如?”

子曰:“忠矣。”

曰:“仁矣乎?”

曰:“未知,焉得仁?”

“崔子\footnote{text}弑\footnote{text}齐君\footnote{text},陈文子\footnote{text}有马十乘\footnote{text},弃而违\footnote{text}之。至于他邦,则曰:‘犹吾大夫崔子也。’违之。之一邦,则又曰:‘犹吾大夫崔子也。’违之。何如?”

子曰:“清矣。”

曰:“仁矣乎?”

曰:“未知,焉得仁?”
\end{yuanwen}

\begin{yuanwen}
季文子\footnote{text}三思而后行。子闻之,曰:“再\footnote{text},斯可矣。”
\end{yuanwen}

\begin{yuanwen}
子曰:“宁武子\footnote{text},邦有道,则知\footnote{text};邦无道,则愚。其知可及也,其愚不可及也。”
\end{yuanwen}

\begin{yuanwen}
子在陈\footnote{text},曰:“归与!归与!吾党之小子狂\footnote{text}简\footnote{text},斐然\footnote{text}成章\footnote{text},不知所以裁\footnote{text}之!”
\end{yuanwen}

\begin{yuanwen}
子曰:“伯夷、叔齐\footnote{text}不念旧恶,怨是用\footnote{text}希\footnote{text}。”
\end{yuanwen}

\begin{yuanwen}
子曰:“孰谓微生高\footnote{text}直?或乞醯\footnote{text}焉,乞诸其邻而与之。”
\end{yuanwen}

\begin{yuanwen}
子曰:“巧言、令色、足恭\footnote{text},左丘明\footnote{text}耻之,丘亦耻之。匿怨\footnote{text}而友其人,左丘明耻之,丘亦耻之。”
\end{yuanwen}

\begin{yuanwen}
颜渊、季路侍\footnote{text}。子曰:“盍\footnote{text}各言尔\footnote{text}志?”

子路曰:“愿车马衣裘,与朋友共,敝之而无憾。”

颜渊曰:“愿无伐善\footnote{text},无施劳\footnote{text}。”

子路曰:“愿闻子之志。”

子曰:“老者安之,朋友信之,少者怀之。”
\end{yuanwen}

\begin{yuanwen}
子曰:“已矣乎!吾未见能见其过而内自讼者也。”
\end{yuanwen}

\begin{yuanwen}
子曰:“十室之邑,必有忠信如丘者焉,不如丘之好学也。”
\end{yuanwen}



\chapter{雍也}

\begin{yuanwen}
子曰:“雍也可使南面。”
\end{yuanwen}

\begin{yuanwen}
仲弓问子桑伯子,子曰:“可也简。”

仲弓曰:“居敬而行简,以临其民,不亦可乎?居简而行简,无乃大简乎?”

子曰:“雍之言然。”
\end{yuanwen}

\begin{yuanwen}
哀公问:“弟子孰为好学?”孔子对曰:“有颜回者好学,不迁怒,不贰过,不幸短命死矣,今也则亡,未闻好学者也。”
\end{yuanwen}

\begin{yuanwen}
子华使于齐,冉子为其母请粟,子曰:“与之釜。”

请益,曰:“与之庾。”冉子与之粟五秉。

子曰:“赤之适齐也,乘肥马,衣轻裘。吾闻之也,君子周急不继富。”
\end{yuanwen}

\begin{yuanwen}
原思为之宰,与之粟九百,辞。

子曰:“毋以与尔邻里乡党乎!”
\end{yuanwen}

\begin{yuanwen}
子谓仲弓曰:“犁牛之子骍且角,虽欲勿用,山川其舍诸?”
\end{yuanwen}

\begin{yuanwen}
子曰:“回也,其心三月不违仁,其余则日月至焉而已矣。”
\end{yuanwen}

\begin{yuanwen}
季康子问:“仲由可使从政也与?”

子曰:“由也果,于从政乎何有?”

曰:“赐也可使从政也与?”

曰:“赐也达,于从政乎何有?”

曰:“求也可使从政也与?”

曰:“求也艺,于从政乎何有?”
\end{yuanwen}

\begin{yuanwen}
季氏使闵子骞为费宰,闵子骞曰:“善为我辞焉。如有复我者,则吾必在汶上矣。”
\end{yuanwen}

\begin{yuanwen}
伯牛有疾,子问之,自牖执其手,曰:“亡之,命矣夫!斯人也而有斯疾也!斯人也而有斯疾也!”
\end{yuanwen}

\begin{yuanwen}
子曰:“贤哉回也!一箪食,一瓢饮,在陋巷,人不堪其忧,回也不改其乐。贤哉,回也!”
\end{yuanwen}

\begin{yuanwen}
冉求曰:“非不说子之道,力不足也。”

子曰:“力不足者,中道而废,今女画。”
\end{yuanwen}

\begin{yuanwen}
子谓子夏曰:“女为君子儒,毋为小人儒。”
\end{yuanwen}

\begin{yuanwen}
子游为武城宰,子曰:“女得人焉尔乎?”

曰:“有澹台灭明者,行不由径,非公事,未尝至于偃之室也。”
\end{yuanwen}

\begin{yuanwen}
子曰:“孟之反不伐,奔而殿,将入门,策其马曰:‘非敢后也,马不进也。’”
\end{yuanwen}

\begin{yuanwen}
子曰:“不有祝鮀之佞,而有宋朝之美,难乎免于今之世矣。”
\end{yuanwen}

\begin{yuanwen}
子曰:“谁能出不由户?何莫由斯道也?”
\end{yuanwen}

\begin{yuanwen}
子曰:“质胜文则野,文胜质则史。文质彬彬,然后君子。”
\end{yuanwen}

\begin{yuanwen}
子曰:“人之生也直,罔之生也幸而免。”
\end{yuanwen}

\begin{yuanwen}
子曰:“知之者不如好之者;好之者不如乐之者。”
\end{yuanwen}

\begin{yuanwen}
子曰:“中人以上,可以语上也;中人以下,不可以语上也。”
\end{yuanwen}

\begin{yuanwen}
樊迟问知,子曰:“务民之义,敬鬼神而远之,可谓知矣。”

问仁,曰:“仁者先难而后获,可谓仁矣。”
\end{yuanwen}

\begin{yuanwen}
子曰:“知者乐水,仁者乐山。知者动,仁者静。知者乐,仁者寿。”
\end{yuanwen}

\begin{yuanwen}
子曰:“齐一变至于鲁,鲁一变至于道。”
\end{yuanwen}

\begin{yuanwen}
子曰:“觚不觚,觚哉!觚哉!”
\end{yuanwen}

\begin{yuanwen}
宰我问曰:“仁者,虽告之曰:‘井有仁焉。’其从之也?”子曰:“何为其然也?君子可逝也,不可陷也;可欺也,不可罔也。”
\end{yuanwen}

\begin{yuanwen}
子曰:“君子博学于文,约之以礼,亦可以弗畔矣夫。”
\end{yuanwen}

\begin{yuanwen}
子见南子,子路不说,夫子矢之曰:“予所否者,天厌之!天厌之!”
\end{yuanwen}

\begin{yuanwen}
子曰:“中庸之为德也,其至矣乎!民鲜久矣。”
\end{yuanwen}

\begin{yuanwen}
子贡曰:“如有博施于民而能济众,何如?可谓仁乎?”子曰:“何事于仁,必也圣乎!尧、舜其犹病诸!夫仁者,己欲立而立人,己欲达而达人。能近取譬
,可谓仁之方也已。”
\end{yuanwen}

\begin{yuanwen}
\chapter{述而}
子曰:“述而不作,信而好古,窃比于我老彭。”
\end{yuanwen}

\begin{yuanwen}
子曰:“默而识之,学而不厌,诲人不倦,何有于我哉?”
\end{yuanwen}

\begin{yuanwen}
子曰:“德之不修,学之不讲,闻义不能徙,不善不能改,是吾忧也。”
\end{yuanwen}

\begin{yuanwen}
子之燕居,申申如也,夭夭如也。
\end{yuanwen}

\begin{yuanwen}
子曰:“甚矣,吾衰也!久矣,吾不复梦见周公。”
\end{yuanwen}

\begin{yuanwen}
子曰:“志于道,据于德,依于仁,游于艺。”
\end{yuanwen}

\begin{yuanwen}
子曰:“自行束脩以上,吾未尝无诲焉。”
\end{yuanwen}

\begin{yuanwen}
子曰:“不愤不启,不悱不发,举一隅不以三隅反,则不复也。”
\end{yuanwen}

\begin{yuanwen}
子食于有丧者之侧,未尝饱也。
\end{yuanwen}

\begin{yuanwen}
子于是日哭,则不歌。
\end{yuanwen}

\begin{yuanwen}
子谓颜渊曰:“用之则行,舍之则藏,惟我与尔有是夫!”子路曰:“子行三军,则谁与?”子曰:“暴虎冯河,死而无悔者,吾不与也。必也临事而惧,好谋而成者也。”
\end{yuanwen}

\begin{yuanwen}
子曰:“富而可求也,虽执鞭之士,吾亦为之。如不可求,从吾所好。”
\end{yuanwen}

\begin{yuanwen}
子之所慎:齐,战,疾。
\end{yuanwen}

\begin{yuanwen}
子在齐闻《韶》,三月不知肉味,曰:“不图为乐之至于斯也。”
\end{yuanwen}

\begin{yuanwen}
冉有曰:“夫子为卫君乎?”子贡曰:“诺,吾将问之。”入,曰:“伯夷、叔齐何人也?”曰:“古之贤人也。”曰:“怨乎?”曰:“求仁而得仁,又何怨?”出,曰:“夫子不为也。”
\end{yuanwen}

\begin{yuanwen}
子曰:“饭疏食饮水,曲肱而枕之,乐亦在其中矣。不义而富且贵,于我如浮云。”
\end{yuanwen}

\begin{yuanwen}
子曰:“加我数年,五十以学《易》,可以无大过矣。”
\end{yuanwen}

\begin{yuanwen}
子所雅言,《诗》、《书》、执礼,皆雅言也。
\end{yuanwen}

\begin{yuanwen}
叶公问孔子于子路,子路不对。子曰:“女奚不曰:其为人也,发愤忘食,乐以忘忧,不知老之将至云尔。”
\end{yuanwen}

\begin{yuanwen}
子曰:“我非生而知之者,好古,敏以求之者也。”
\end{yuanwen}

\begin{yuanwen}
子不语:怪、力、乱、神。
\end{yuanwen}

\begin{yuanwen}
子曰:“三人行,必有我师焉。择其善者而从之,其不善者而改之。”
\end{yuanwen}

\begin{yuanwen}
子曰:“天生德于予,桓魋其如予何?”
\end{yuanwen}

\begin{yuanwen}
子曰:“二三子以我为隐乎?吾无隐乎尔!吾无行而不与二三子者,是丘也。”
\end{yuanwen}

\begin{yuanwen}
子以四教:文,行,忠,信。
\end{yuanwen}

\begin{yuanwen}
子曰:“圣人,吾不得而见之矣;得见君子者,斯可矣。”子曰:“善人,吾不得而见之矣,得见有恒者斯可矣。亡而为有,虚而为盈,约而为泰,难乎有恒
乎。”
\end{yuanwen}

\begin{yuanwen}
子钓而不纲,弋不射宿。
\end{yuanwen}

\begin{yuanwen}
子曰:“盖有不知而作之者,我无是也。多闻,择其善者而从之;多见而识之,知之次也。”
\end{yuanwen}

\begin{yuanwen}
互乡难与言,童子见,门人惑。子曰:“与其进也,不与其退也,唯何甚?人洁己以进,与其洁也,不保其往也。”
\end{yuanwen}

\begin{yuanwen}
子曰:“仁远乎哉?我欲仁,斯仁至矣。”
\end{yuanwen}

\begin{yuanwen}
陈司败问:“昭公知礼乎?”孔子曰:“知礼。”孔子退,揖巫马期而进之,曰:“吾闻君子不党,君子亦党乎?君取于吴,为同姓,谓之吴孟子。君而知礼,
孰不知礼?”巫马期以告,子曰:“丘也幸,苟有过,人必知之。”
\end{yuanwen}

\begin{yuanwen}
子与人歌而善,必使反之,而后和之。
\end{yuanwen}

\begin{yuanwen}
子曰:“文,莫吾犹人也。躬行君子,则吾未之有得。”
\end{yuanwen}

\begin{yuanwen}
子曰:“若圣与仁,则吾岂敢?抑为之不厌,诲人不倦,则可谓云尔已矣。”公西华曰:“正唯弟子不能学也。”
\end{yuanwen}

\begin{yuanwen}
子疾病,子路请祷。子曰:“有诸?”子路对曰:“有之。《诔》曰:‘祷尔于上下神祇。’”子曰:“丘之祷久矣。”
\end{yuanwen}

\begin{yuanwen}
子曰:“奢则不孙,俭则固。与其不孙也,宁固。”
\end{yuanwen}

\begin{yuanwen}
子曰:“君子坦荡荡,小人长戚戚。”
\end{yuanwen}

\begin{yuanwen}
子温而厉,威而不猛,恭而安。
\end{yuanwen}


\chapter{泰伯}

\begin{yuanwen}
子曰:“泰伯,其可谓至德也已矣。三以天下让,民无得而称焉。”
\end{yuanwen}

\begin{yuanwen}
子曰:“恭而无礼则劳;慎而无礼则葸;勇而无礼则乱;直而无礼则绞。君子笃于亲,则民兴于仁;故旧不遗,则民不偷。”
\end{yuanwen}

\begin{yuanwen}
曾子有疾,召门弟子曰:“启予足,启予手。《诗》云:‘战战兢兢,如临深渊,如履薄冰。’而今而后,吾知免夫,小子!”
\end{yuanwen}

\begin{yuanwen}
曾子有疾,孟敬子问之。曾子言曰:“鸟之将死,其鸣也哀;人之将死,其言也善。君子所贵乎道者三:动容貌,斯远暴慢矣;正颜色,斯近信矣;出辞气,
斯远鄙倍矣。笾豆之事,则有司存。”
\end{yuanwen}

\begin{yuanwen}
曾子曰:“以能问于不能;以多问于寡;有若无,实若虚,犯而不校。昔者吾友尝从事于斯矣。”
\end{yuanwen}

\begin{yuanwen}
曾子曰:“可以托六尺之孤,可以寄百里之命,临大节而不可夺也。君子人与?君子人也。”
\end{yuanwen}

\begin{yuanwen}
曾子曰:“士不可以不弘毅,任重而道远。仁以为己任,不亦重乎?死而后已,不亦远乎?”
\end{yuanwen}

\begin{yuanwen}
子曰:“兴于《诗》,立于礼,成于乐。”
\end{yuanwen}

\begin{yuanwen}
子曰:“民可使由之,不可使知之。”
\end{yuanwen}

\begin{yuanwen}
子曰:“好勇疾贫,乱也。人而不仁,疾之已甚,乱也。”
\end{yuanwen}

\begin{yuanwen}
子曰:“如有周公之才之美,使骄且吝,其余不足观也已。”
\end{yuanwen}

\begin{yuanwen}
子曰:“三年学,不至于谷,不易得也。”
\end{yuanwen}

\begin{yuanwen}
子曰:“笃信好学,守死善道。危邦不入,乱邦不居。天下有道则见,无道则隐。邦有道,贫且贱焉,耻也;邦无道,富且贵焉,耻也。”
\end{yuanwen}

\begin{yuanwen}
子曰:“不在其位,不谋其政。”
\end{yuanwen}

\begin{yuanwen}
子曰:“师挚之始,《关雎》之乱,洋洋乎盈耳哉!”
\end{yuanwen}

\begin{yuanwen}
子曰:“狂而不直,侗而不愿,悾悾而不信,吾不知之矣。”
\end{yuanwen}

\begin{yuanwen}
子曰:“学如不及,犹恐失之。”
\end{yuanwen}

\begin{yuanwen}
子曰:“巍巍乎!舜、禹之有天下也而不与焉。”
\end{yuanwen}

\begin{yuanwen}
子曰:“大哉尧之为君也!巍巍乎,唯天为大,唯尧则之。荡荡乎,民无能名焉。巍巍乎其有成功也,焕乎其有文章!”
\end{yuanwen}

\begin{yuanwen}
舜有臣五人而天下治。武王曰:“予有乱臣十人。”孔子曰:“才难,不其然乎?唐虞之际,于斯为盛;有妇人焉,九人而已。三分天下有其二,以服事殷。周之德,其可谓至德也已矣。”
\end{yuanwen}

\begin{yuanwen}
子曰:“禹,吾无间然矣。菲饮食,而致孝乎鬼神;恶衣服,而致美乎黻冕;卑宫室,而尽力乎沟洫。禹,吾无间然矣!”
\end{yuanwen}

\begin{yuanwen}
\chapter{子罕}
\end{yuanwen}

\begin{yuanwen}
子罕言利与命与仁。
\end{yuanwen}

\begin{yuanwen}
达巷党人曰:“大哉孔子!博学而无所成名。”子闻之,谓门弟子曰:“吾何执?执御乎,执射乎?吾执御矣。”
\end{yuanwen}

\begin{yuanwen}
子曰:“麻冕,礼也;今也纯,俭,吾从众。拜下,礼也;今拜乎上,泰也;虽违众,吾从下。”
\end{yuanwen}

\begin{yuanwen}
子绝四:毋意、毋必、毋固、毋我。
\end{yuanwen}

\begin{yuanwen}
子畏于匡,曰:“文王既没,文不在兹乎?天之将丧斯文也,后死者不得与于斯文也;天之未丧斯文也,匡人其如予何?”
\end{yuanwen}

\begin{yuanwen}
太宰问于子贡曰:“夫子圣者与,何其多能也?”子贡曰:“固天纵之将圣,又多能也。”子闻之,曰:“太宰知我乎?吾少也贱,故多能鄙事。君子多乎哉?不多也。”
\end{yuanwen}

\begin{yuanwen}
牢曰:“子云:‘吾不试,故艺。’”
\end{yuanwen}

\begin{yuanwen}
子曰:“吾有知乎哉?无知也。有鄙夫问于我,空空如也。我叩其两端而竭焉。”
\end{yuanwen}

\begin{yuanwen}
子曰:“凤鸟不至,河不出图,吾已矣夫!”
\end{yuanwen}

\begin{yuanwen}
子见齐衰者、冕衣裳者与瞽者,见之,虽少,必作,过之必趋。
\end{yuanwen}

\begin{yuanwen}
颜渊喟然叹曰:“仰之弥高,钻之弥坚。瞻之在前,忽焉在后。夫子循循然善诱人,博我以文,约我以礼,欲罢不能。既竭吾才,如有所立卓尔,虽欲从之,末由也已。”
\end{yuanwen}

\begin{yuanwen}
子疾病,子路使门人为臣。病间,曰:“久矣哉,由之行诈也!无臣而为有臣,吾谁欺?欺天乎?且予与其死于臣之手也,无宁死于二三子之手乎!且予纵
\end{yuanwen}

\begin{yuanwen}
不得大葬,予死于道路乎?”
\end{yuanwen}

\begin{yuanwen}
子贡曰:“有美玉于斯,韫椟而藏诸?求善贾而沽诸?”子曰:“沽之哉,沽之哉!我待贾者也。”
\end{yuanwen}

\begin{yuanwen}
子欲居九夷。或曰:“陋,如之何?”子曰:“君子居之,何陋之有!”
\end{yuanwen}

\begin{yuanwen}
子曰:“吾自卫反鲁,然后乐正,《雅》、《颂》各得其所。”
\end{yuanwen}

\begin{yuanwen}
子曰:“出则事公卿,入则事父兄,丧事不敢不勉,不为酒困,何有于我哉?”
\end{yuanwen}

\begin{yuanwen}
子在川上曰:“逝者如斯夫!不舍昼夜。”
\end{yuanwen}

\begin{yuanwen}
子曰:“吾未见好德如好色者也。”
\end{yuanwen}

\begin{yuanwen}
子曰:“譬如为山,未成一篑,止,吾止也;譬如平地,虽覆一篑,进,吾往也。”
\end{yuanwen}

\begin{yuanwen}
子曰:“语之而不惰者,其回也与!”
\end{yuanwen}

\begin{yuanwen}
子谓颜渊,曰:“惜乎!吾见其进也,未见其止也。”
\end{yuanwen}

\begin{yuanwen}
子曰:“苗而不秀者有矣夫,秀而不实者有矣夫。”
\end{yuanwen}

\begin{yuanwen}
子曰:“后生可畏,焉知来者之不如今也?四十、五十而无闻焉,斯亦不足畏也已。”
\end{yuanwen}

\begin{yuanwen}
子曰:“法语之言,能无从乎?改之为贵。巽与之言,能无说乎?绎之为贵。说而不绎,从而不改,吾末如之何也已矣。”
\end{yuanwen}

\begin{yuanwen}
子曰:“主忠信。毋友不如己者,过,则勿惮改。”
\end{yuanwen}

\begin{yuanwen}
子曰:“三军可夺帅也,匹夫不可夺志也。”
\end{yuanwen}

\begin{yuanwen}
子曰:“衣敝缊袍,与衣狐貉者立而不耻者,其由也与!‘不忮不求,何用不臧?’”子路终身诵之,子曰:“是道也,何足以臧?”
\end{yuanwen}

\begin{yuanwen}
子曰:“岁寒,然后知松柏之后凋也。”
\end{yuanwen}

\begin{yuanwen}
子曰:“知者不惑,仁者不忧,勇者不惧。”
\end{yuanwen}

\begin{yuanwen}
子曰:“可与共学,未可与适道;可与适道,未可与立;可与立,未可与权。”
\end{yuanwen}

\begin{yuanwen}
“唐棣之华,偏其反而。岂不尔思?室是远尔。”子曰:“未之思也,夫何远之有。”
\end{yuanwen}


\chapter{乡党}

\begin{yuanwen}
孔子于乡党,恂恂如也,似不能言者;其在宗庙朝廷,便便言,唯谨尔。
\end{yuanwen}

\begin{yuanwen}
朝,与下大夫言,侃侃如也;与上大夫言,訚訚如也。君在,踧踖如也,与与如也。
\end{yuanwen}

\begin{yuanwen}
君召使摈,色勃如也,足躩如也。揖所与立,左右手,衣前后襜如也。趋进,翼如也。宾退,必复命曰:“宾不顾矣。”
\end{yuanwen}

\begin{yuanwen}
入公门,鞠躬如也,如不容。立不中门,行不履阈。过位,色勃如也,足躩如也,其言似不足者。摄齐升堂,鞠躬如也,屏气似不息者。出,降一等,逞颜
色,怡怡如也;没阶,趋进,翼如也;复其位,踧踖如也。
\end{yuanwen}

\begin{yuanwen}
执圭,鞠躬如也,如不胜。上如揖,下如授。勃如战色,足蹜蹜如有循。享礼,有容色。私觌,愉愉如也。
\end{yuanwen}

\begin{yuanwen}
君子不以绀緅饰,红紫不以为亵服。当暑,袗絺绤,必表而出之。缁衣羔裘,素衣麑裘,黄衣狐裘。亵裘长,短右袂。必有寝衣,长一身有半。狐貉之厚以居。去丧,无所不佩。非帷裳,必杀之。羔裘玄冠不以吊。吉月,必朝服而朝。
\end{yuanwen}

\begin{yuanwen}
齐,必有明衣,布。齐必变食,居必迁坐。
\end{yuanwen}

\begin{yuanwen}
食不厌精,脍不厌细。食饐而餲,鱼馁而肉败,不食;色恶,不食;臭恶,不食;失饪,不食;不时,不食;割不正,不食;不得其酱,不食。肉虽多,不使胜食气。唯酒无量,不及乱。沽酒市脯,不食。不撤姜食,不多食。
\end{yuanwen}

\begin{yuanwen}
祭于公,不宿肉。祭肉不出三日,出三日不食之矣。
\end{yuanwen}

\begin{yuanwen}
食不语,寝不言。
\end{yuanwen}

\begin{yuanwen}
虽疏食菜羹,瓜祭,必齐如也。
\end{yuanwen}

\begin{yuanwen}
席不正,不坐。
\end{yuanwen}

\begin{yuanwen}
乡人饮酒,杖者出,斯出矣。
\end{yuanwen}

\begin{yuanwen}
乡人傩,朝服而立于阼阶。
\end{yuanwen}

\begin{yuanwen}
问人于他邦,再拜而送之。
\end{yuanwen}

\begin{yuanwen}
康子馈药,拜而受之。曰:“丘未达,不敢尝。”
\end{yuanwen}

\begin{yuanwen}
厩焚,子退朝,曰:“伤人乎?”不问马。
\end{yuanwen}

\begin{yuanwen}
君赐食,必正席先尝之。君赐腥,必熟而荐之。君赐生,必畜之。侍食于君,君祭,先饭。
\end{yuanwen}

\begin{yuanwen}
疾,君视之,东首,加朝服,拖绅。
\end{yuanwen}

\begin{yuanwen}
君命召,不俟驾行矣。
\end{yuanwen}

\begin{yuanwen}
入太庙,每事问。
\end{yuanwen}

\begin{yuanwen}
朋友死,无所归,曰:“于我殡。”
\end{yuanwen}

\begin{yuanwen}
朋友之馈,虽车马,非祭肉,不拜。
\end{yuanwen}

\begin{yuanwen}
寝不尸,居不容。
\end{yuanwen}

\begin{yuanwen}
见齐衰者,虽狎,必变。见冕者与瞽者,虽亵,必以貌。凶服者式之,式负版者。有盛馔,必变色而作。迅雷风烈,必变。
\end{yuanwen}

\begin{yuanwen}
升车,必正立,执绥。车中不内顾,不疾言,不亲指。
\end{yuanwen}

\begin{yuanwen}
色斯举矣,翔而后集。曰:“山梁雌雉,时哉时哉!”子路共之,三嗅而作。
\end{yuanwen}

\chapter{先进}

\begin{yuanwen}
子曰:“先进于礼乐,野人也;后进于礼乐,君子也。如用之,则吾从先进。”
\end{yuanwen}

\begin{yuanwen}
子曰:“从我于陈、蔡者,皆不及门也。”
\end{yuanwen}

\begin{yuanwen}
德行:颜渊,闵子骞,冉伯牛,仲弓。言语:宰我,子贡。政事:冉有,季路。文学:子游,子夏。
\end{yuanwen}

\begin{yuanwen}
子曰:“回也非助我者也,于吾言无所不说。”
\end{yuanwen}

\begin{yuanwen}
子曰:“孝哉闵子骞!人不间于其父母昆弟之言。”
\end{yuanwen}

\begin{yuanwen}
南容三复白圭,孔子以其兄之子妻之。
\end{yuanwen}

\begin{yuanwen}
季康子问:“弟子孰为好学?”孔子对曰:“有颜回者好学,不幸短命死矣,今也则亡。”
\end{yuanwen}

\begin{yuanwen}
颜渊死,颜路请子之车以为之椁。子曰:“才不才,亦各言其子也。鲤也死,有棺而无椁,吾不徒行以为之椁。以吾从大夫之后,不可徒行也。”

\end{yuanwen}

\begin{yuanwen}
	颜渊死,子曰:“噫!天丧予!天丧予!”

颜渊死,子哭之恸,从者曰:“子恸矣!”曰:“有恸乎?非夫人之为恸而谁为?”
\end{yuanwen}

\begin{yuanwen}
颜渊死,门人欲厚葬之,子曰:“不可。”门人厚葬之,子曰:“回也视予犹父也,予不得视犹子也。非我也,夫二三子也!”
\end{yuanwen}

\begin{yuanwen}
季路问事鬼神,子曰:“未能事人,焉能事鬼?”,曰:“敢问死。”曰:“未知生,焉知死?”
\end{yuanwen}

\begin{yuanwen}
闵子侍侧,訚訚如也;子路,行行如也;冉有、子贡,侃侃如也。子乐。“若由也,不得其死然。”
\end{yuanwen}

\begin{yuanwen}
鲁人为长府,闵子骞曰:“仍旧贯如之何?何必改作?”子曰:“夫人不言,言必有中。”
\end{yuanwen}

\begin{yuanwen}
子曰:“由之瑟,奚为于丘之门?”门人不敬子路,子曰:“由也升堂矣,未入于室也。”
\end{yuanwen}

\begin{yuanwen}
子贡问:“师与商也孰贤?”子曰:“师也过,商也不及。”曰:“然则师愈与?”子曰:“过犹不及。”
\end{yuanwen}

\begin{yuanwen}
季氏富于周公,而求也为之聚敛而附益之。子曰:“非吾徒也,小子鸣鼓而攻之可也。”
\end{yuanwen}

\begin{yuanwen}
柴也愚,参也鲁,师也辟,由也喭。
\end{yuanwen}

\begin{yuanwen}
子曰:“回也其庶乎,屡空。赐不受命而货殖焉,亿则屡中。”
\end{yuanwen}

\begin{yuanwen}
子张问善人之道,子曰:“不践迹,亦不入于室。”
\end{yuanwen}

\begin{yuanwen}
子曰:“论笃是与,君子者乎,色庄者乎?”
\end{yuanwen}

\begin{yuanwen}
子路问:“闻斯行诸?”子曰:“有父兄在,如之何其闻斯行之?”冉有问:“闻斯行诸?”子曰:“闻斯行之。”公西华曰:“由也问:“闻斯行诸?”子曰:‘有父兄在’;求也问:‘闻斯行诸’。子曰‘闻斯行之’。赤也惑,敢问。”子曰:“求也退,故进之;由也兼人,故退之。”
\end{yuanwen}

\begin{yuanwen}
子畏于匡,颜渊后。子曰:“吾以女为死矣!”曰:“子在,回何敢死!”
\end{yuanwen}

\begin{yuanwen}
季子然问:“仲由、冉求可谓大臣与?”子曰:“吾以子为异之问,曾由与求之问。所谓大臣者,以道事君,不可则止。今由与求也,可谓具臣矣。”曰:“然则从之者与?”子曰:“弑父与君,亦不从也。”
\end{yuanwen}

\begin{yuanwen}
子路使子羔为费宰,子曰:“贼夫人之子。”子路曰:“有民人焉,有社稷焉,何必读书然后为学。”子曰:“是故恶夫佞者。”
\end{yuanwen}

\begin{yuanwen}
子路、曾皙、冉有、公西华侍坐,子曰:“以吾一日长乎尔,毋吾以也。居则曰‘不吾知也’如或知尔,则何以哉?”子路率尔而对曰:“千乘之国,摄乎大国之间,加之以师旅,因之以饥馑,由也为之,比及三年,可使有勇,且知方也。”夫子哂之。“求,尔何如?”对曰:“方六七十,如五六十,求也为之,比及三年,可使足民。如其礼乐,以俟君子。”“赤!尔何如?”对曰:“非曰能之,愿学焉。宗庙之事,如会同,端章甫,愿为小相焉。”“点,尔何如?”鼓瑟希,铿尔,舍瑟而作,对曰:“异乎三子者之撰。”子曰:“何伤乎?亦各言其志也。”曰:“暮春者
\end{yuanwen}

\chapter{颜渊}

\begin{yuanwen}
颜渊问仁,子曰:“克己复礼为仁。一日克己复礼,天下归仁焉。为仁由己,而由人乎哉?”颜渊曰:“请问其目?”子曰:“非礼勿视,非礼勿听,非礼勿言,非礼勿动。”颜渊曰:“回虽不敏,请事斯语矣。”
\end{yuanwen}

\begin{yuanwen}
仲弓问仁,子曰:“出门如见大宾,使民如承大祭。己所不欲,勿施于人。在邦无怨,在家无怨。”仲弓曰:“雍虽不敏,请事斯语矣。”
\end{yuanwen}

\begin{yuanwen}
司马牛问仁,子曰:“仁者,其言也讱。”曰:“其言也讱,斯谓之仁已乎?”子曰:“为之难,言之得无讱乎?”
\end{yuanwen}

\begin{yuanwen}
司马牛问君子,子曰:“君子不忧不惧。”曰:“不忧不惧,斯谓之君子已乎?”子曰:“内省不疚,夫何忧何惧?”
\end{yuanwen}

\begin{yuanwen}
司马牛忧曰:“人皆有兄弟,我独亡。”子夏曰:“商闻之矣:死生有命,富贵在天。君子敬而无失,与人恭而有礼,四海之内皆兄弟也。君子何患乎无兄弟也?”
\end{yuanwen}

\begin{yuanwen}
子张问明,子曰:“浸润之谮,肤受之愬,不行焉,可谓明也已矣;浸润之谮、肤受之愬不行焉,可谓远也已矣。”
\end{yuanwen}

\begin{yuanwen}
子贡问政,子曰:“足食,足兵,民信之矣。”子贡曰:“必不得已而去,于斯三者何先?”曰:“去兵。”子贡曰:“必不得已而去,于斯二者何先?”曰:“去食。自古皆有死,民无信不立。”
\end{yuanwen}

\begin{yuanwen}
棘子成曰:“君子质而已矣,何以文为?”子贡曰:“惜乎,夫子之说君子也!驷不及舌。文犹质也,质犹文也。虎豹之鞟犹犬羊之鞟。”
\end{yuanwen}

\begin{yuanwen}
哀公问于有若曰:“年饥,用不足,如之何?”有若对曰:“盍彻乎?”曰:“二,吾犹不足,如之何其彻也?”对曰:“百姓足,君孰与不足?百姓不足,君孰与足?”
\end{yuanwen}

\begin{yuanwen}
子张问崇德、辨惑,子曰:“主忠信,徙义,崇德也。爱之欲其生,恶之欲其死;既欲其生又欲其死,是惑也。‘诚不以富,亦祗以异。’”
\end{yuanwen}

\begin{yuanwen}
齐景公问政于孔子,孔子对曰:“君君,臣臣,父父,子子。”公曰:“善哉!信如君不君、臣不臣、父不父、子不子,虽有粟,吾得而食诸?”
\end{yuanwen}

\begin{yuanwen}
子曰:“片言可以折狱者,其由也与?”子路无宿诺。
\end{yuanwen}

\begin{yuanwen}
子曰:“听讼,吾犹人也。必也使无讼乎。”
\end{yuanwen}

\begin{yuanwen}
子张问政,子曰:“居之无倦,行之以忠。”
\end{yuanwen}

\begin{yuanwen}
子曰:“博学于文,约之以礼,亦可以弗畔矣夫。”
\end{yuanwen}

\begin{yuanwen}
子曰:“君子成人之美,不成人之恶;小人反是。”
\end{yuanwen}

\begin{yuanwen}
季康子问政于孔子,孔子对曰:“政者,正也。子帅以正,孰敢不正?”
\end{yuanwen}

\begin{yuanwen}
季康子患盗,问于孔子。孔子对曰:“苟子之不欲,虽赏之不窃。”
\end{yuanwen}

\begin{yuanwen}
季康子问政于孔子曰:“如杀无道以就有道,何如?”孔子对曰:“子为政,焉用杀?子欲善而民善矣。君子之德风,小人之德草,草上之风必偃。”
\end{yuanwen}

\begin{yuanwen}
子张问:“士何如斯可谓之达矣?”子曰:“何哉尔所谓达者?”子张对曰:“在邦必闻,在家必闻。”子曰:“是闻也,非达也。夫达也者,质直而好义,察言而观色,虑以下人。在邦必达,在家必达。夫闻也者,色取仁而行违,居之不疑。在邦必闻,在家必闻。”
\end{yuanwen}

\begin{yuanwen}
樊迟从游于舞雩之下,曰:“敢问崇德、修慝、辨惑。”子曰:“善哉问!先事后得,非崇德与?攻其恶,无攻人之恶,非修慝与?一朝之忿,忘其身,以及其亲,非惑与?”
\end{yuanwen}

\begin{yuanwen}
樊迟问仁,子曰:“爱人。”问知,子曰:“知人。”樊迟未达,子曰:“举直错诸枉,能使枉者直。”樊迟退,见子夏,曰:“乡也吾见于夫子而问知,子曰:‘举直错诸枉,能使枉者直’,何谓也?”子夏曰:“富哉言乎!舜有天下,选于众,举皋陶,不仁者远矣。汤有天下,选于众,举伊尹,不仁者远矣。”
\end{yuanwen}

\begin{yuanwen}
子贡问友,子曰:“忠告而善道之,不可则止,毋自辱焉。”
\end{yuanwen}

\begin{yuanwen}
曾子曰:“君子以文会友,以友辅仁。”
\end{yuanwen}

\chapter{子路}

\begin{yuanwen}
子路问政,子曰:“先之,劳之。”请益,曰:“无倦。”
\end{yuanwen}

\begin{yuanwen}
仲弓为季氏宰,问政,子曰:“先有司,赦小过,举贤才。”曰:“焉知贤才而举之?”子曰:“举尔所知。尔所不知,人其舍诸?”
\end{yuanwen}

\begin{yuanwen}
子路曰:“卫君待子而为政,子将奚先?”子曰:“必也正名乎!”子路曰:“有是哉,子之迂也!奚其正?”子曰:“野哉,由也!君子于其所不知,盖阙如也。名不正、则言不顺,言不顺则事不成,事不成则礼乐不兴,礼乐不兴则刑罚不中,刑罚不中则民无所措手足。故君子名之必可言也,言之必可行也。君子于其言,无所苟而已矣。”
\end{yuanwen}

\begin{yuanwen}
樊迟请学稼,子曰:“吾不如老农。”请学为圃,曰:“吾不如老圃。”樊迟出。子曰:“小人哉,樊须也!上好礼,则民莫敢不敬;上好义,则民莫敢不服;上好信,则民莫敢不用情。夫如是,则四方之民襁负其子而至矣,焉用稼?”
\end{yuanwen}

\begin{yuanwen}
子曰:“诵《诗》三百,授之以政,不达;使于四方,不能专对;虽多,亦奚以为?”
\end{yuanwen}

\begin{yuanwen}
子曰:“其身正,不令而行;其身不正,虽令不从。”
\end{yuanwen}

\begin{yuanwen}
子曰:“鲁卫之政,兄弟也。”
\end{yuanwen}

\begin{yuanwen}
子谓卫公子荆,“善居室。始有,曰:‘苟合矣。’少有,曰:‘苟完矣。’富有,曰:‘苟美矣。’”
\end{yuanwen}

\begin{yuanwen}
子适卫,冉有仆,子曰:“庶矣哉!”冉有曰:“既庶矣,又何加焉?”曰:“富之。”曰:“既富矣,又何加焉?”曰:“教之。”
\end{yuanwen}

\begin{yuanwen}
子曰:“苟有用我者,期月而已可也,三年有成。”
\end{yuanwen}

\begin{yuanwen}
子曰:“‘善人为邦百年,亦可以胜残去杀矣。’诚哉是言也!”
\end{yuanwen}

\begin{yuanwen}
子曰:“如有王者,必世而后仁。”
\end{yuanwen}

\begin{yuanwen}
子曰:“苟正其身矣,于从政乎何有?不能正其身,如正人何?”
\end{yuanwen}

\begin{yuanwen}
冉子退朝,子曰:“何晏也?”对曰:“有政。”子曰:“其事也。如有政,虽不吾以,吾其与闻之。”
\end{yuanwen}

\begin{yuanwen}
定公问:“一言而可以兴邦,有诸?”孔子对曰:“言不可以若是。其几也。人之言曰:‘为君难,为臣不易。’如知为君之难也,不几乎一言而兴邦乎?”曰:“一言而丧邦,有诸?”孔子对曰:“言不可以若是其几也。人之言曰:‘予无乐乎为君,唯其言而莫予违也。’如其善而莫之违也,不亦善乎?如不善而莫之违也,不几乎一言而丧邦乎?”
\end{yuanwen}

\begin{yuanwen}
叶公问政,子曰:“近者说,远者来。”
\end{yuanwen}

\begin{yuanwen}
子夏为莒父宰,问政,子曰:“无欲速,无见小利。欲速则不达,见小利则大事不成。”
\end{yuanwen}

\begin{yuanwen}
叶公语孔子曰:“吾党有直躬者,其父攘羊,而子证之。”孔子曰:“吾党之直者异于是。父为子隐,子为父隐,直在其中矣。”
\end{yuanwen}

\begin{yuanwen}
樊迟问仁,子曰:“居处恭,执事敬,与人忠。虽之夷狄,不可弃也。”
\end{yuanwen}

\begin{yuanwen}
子贡问曰:“何如斯可谓之士矣?”子曰:“行己有耻,使于四方不辱君命,可谓士矣。”曰:“敢问其次。”曰:“宗族称孝焉,乡党称弟焉。”曰:“敢问其次
。”曰:“言必信,行必果,踁踁然小人哉!抑亦可以为次矣。”曰:“今之从政者何如?”子曰:“噫!斗筲之人,何足算也!”
\end{yuanwen}

\begin{yuanwen}
子曰:“不得中行而与之,必也狂狷乎!狂者进取,狷者有所不为也。”
\end{yuanwen}

\begin{yuanwen}
子曰:“南人有言曰:‘人而无恒,不可以作巫医。’善夫!”“不恒其德,或承之羞。”子曰:“不占而已矣。”
\end{yuanwen}

\begin{yuanwen}
子曰:“君子和而不同,小人同而不和。”
\end{yuanwen}

\begin{yuanwen}
子贡问曰:“乡人皆好之,何如?”子曰:“未可也。”“乡人皆恶之,何如?”子曰:“未可也。不如乡人之善者好之,其不善者恶之。”
\end{yuanwen}

\begin{yuanwen}
子曰:“君子易事而难说也,说之不以道不说也,及其使人也器之;小人难事而易说也,说之虽不以道说也,及其使人也求备焉。”
\end{yuanwen}

\begin{yuanwen}
子曰:“君子泰而不骄,小人骄而不泰。”
\end{yuanwen}

\begin{yuanwen}
子曰:“刚、毅、木、讷近仁。”
\end{yuanwen}

\begin{yuanwen}
子路问曰:“何如斯可谓之士矣?”子曰:“切切偲偲,怡怡如也,可谓士矣。朋友切切偲偲,兄弟怡怡。”
\end{yuanwen}

\begin{yuanwen}
子曰:“善人教民七年,亦可以即戎矣。”
\end{yuanwen}

\begin{yuanwen}
子曰:“以不教民战,是谓弃之。”
\end{yuanwen}


\chapter{宪问}


\begin{yuanwen}
宪问耻,子曰:“邦有道,谷;邦无道,谷,耻也。”“克、伐、怨、欲不行焉,可以为仁矣?”子曰:“可以为难矣,仁则吾不知也。”
\end{yuanwen}

孔子说:“。”

\begin{yuanwen}
子曰:“士而怀居,不足以为士矣。”
\end{yuanwen}

孔子说:“。”

\begin{yuanwen}
子曰:“邦有道,危言危行;邦无道,危行言孙。”
\end{yuanwen}

孔子说:“。”

\begin{yuanwen}
子曰:“有德者必有言,有言者不必有德。仁者必有勇,勇者不必有仁。”
\end{yuanwen}

孔子说:“。”

\begin{yuanwen}
南宫适问于孔子曰:“羿善射,奡荡舟,俱不得其死然;禹、稷躬稼而有天下。”夫子不答。南宫适出,子曰:“君子哉若人!尚德哉若人!”
\end{yuanwen}

孔子说:“。”

\begin{yuanwen}
子曰:“君子而不仁者有矣夫,未有小人而仁者也。”
\end{yuanwen}

孔子说:“。”

\begin{yuanwen}
子曰:“爱之,能勿劳乎?忠焉,能勿诲乎?”
\end{yuanwen}

孔子说:“。”

\begin{yuanwen}
子曰:“为命,裨谌草创之,世叔讨论之,行人子羽修饰之,东里子产润色之。”
\end{yuanwen}

孔子说:“。”

\begin{yuanwen}
或问子产,子曰:“惠人也。”问子西,曰:“彼哉,彼哉!”问管仲,曰:“人也。夺伯氏骈邑三百,饭疏食,没齿无怨言。”
\end{yuanwen}

孔子说:“。”

\begin{yuanwen}
子曰:“贫而无怨难,富而无骄易。”
\end{yuanwen}

孔子说:“。”

\begin{yuanwen}
子曰:“孟公绰为赵、魏老则优,不可以为滕、薛大夫。”
\end{yuanwen}

孔子说:“。”

\begin{yuanwen}
子路问成人,子曰:“若臧武仲之知、公绰之不欲、卞庄子之勇、冉求之艺,文之以礼乐,亦可以为成人矣。”曰:“今之成人者何必然?见利思义,见危授命,久要不忘平生之言,亦可以为成人矣。”
\end{yuanwen}

孔子说:“。”

\begin{yuanwen}
子问公叔文子于公明贾曰:“信乎,夫子不言,不笑,不取乎?”公明贾对曰:“以告者过也。夫子时然后言,人不厌其言;乐然后笑,人不厌其笑;义然后取,人不厌其取。”子曰:“其然?岂其然乎?”
\end{yuanwen}

孔子说:“。”

\begin{yuanwen}
子曰:“臧武仲以防求为后于鲁,虽曰不要君,吾不信也。”
\end{yuanwen}

孔子说:“。”

\begin{yuanwen}
子曰:“晋文公谲而不正,齐桓公正而不谲。”
\end{yuanwen}

孔子说:“。”

\begin{yuanwen}
子路曰:“桓公杀公子纠,召忽死之,管仲不死,曰未仁乎?”子曰:“桓公九合诸侯不以兵车,管仲之力也。如其仁,如其仁!”
\end{yuanwen}

孔子说:“。”

\begin{yuanwen}
子贡曰:“管仲非仁者与?桓公杀公子纠,不能死,又相之。”子曰:“管仲相桓公霸诸侯,一匡天下,民到于今受其赐。微管仲,吾其被发左衽矣。岂若匹夫匹妇之为谅也,自经于沟渎而莫之知也。”
\end{yuanwen}

孔子说:“。”

\begin{yuanwen}
公叔文子之臣大夫僎与文子同升诸公,子闻之,曰:“可以为‘文’矣。”
\end{yuanwen}

\begin{yuanwen}
子言卫灵公之无道也,康子曰:“夫如是,奚而不丧?”孔子曰:“仲叔圉治宾客,祝鮀治宗庙,王孙贾治军旅,夫如是,奚其丧?”
\end{yuanwen}

孔子说:“。”

\begin{yuanwen}
子曰:“其言之不怍,则为之也难。”
\end{yuanwen}

孔子说:“。”

\begin{yuanwen}
陈成子弑简公,孔子沐浴而朝,告于哀公曰:“陈恒弑其君,请讨之。”公曰:“告夫三子。”,孔子曰:“以吾从大夫之后,不敢不告也,君曰‘告夫三子’者
!”之三子告,不可。孔子曰:“以吾从大夫之后,不敢不告也。”
\end{yuanwen}

孔子说:“。”

\begin{yuanwen}
子路问事君,子曰:“勿欺也,而犯之。”
\end{yuanwen}

孔子说:“。”

\begin{yuanwen}
子曰:“君子上达,小人下达。”
\end{yuanwen}

孔子说:“。”

\begin{yuanwen}
子曰:“古之学者为己,今之学者为人。”
\end{yuanwen}

孔子说:“。”

\begin{yuanwen}
蘧伯玉使人于孔子,孔子与之坐而问焉,曰:“夫子何为?”对曰:“夫子欲寡其过而未能也。”使者出,子曰:“使乎!使乎!”
\end{yuanwen}

孔子说:“。”

\begin{yuanwen}
子曰:“不在其位,不谋其政。”曾子曰:“君子思不出其位。”
\end{yuanwen}

孔子说:“。”

\begin{yuanwen}
子曰:“君子耻其言而过其行。”
\end{yuanwen}

孔子说:“。”

\begin{yuanwen}
子曰:“君子道者三,我无能焉:仁者不忧,知者不惑,勇者不惧。”子贡曰:“夫子自道也。”
\end{yuanwen}

孔子说:“。”

\begin{yuanwen}
子贡方人,子曰:“赐也贤乎哉?夫我则不暇。”
\end{yuanwen}

孔子说:“。”

\begin{yuanwen}
子曰:“不患人之不己知,患其不能也。”
\end{yuanwen}

孔子说:“。”

\begin{yuanwen}
子曰:“不逆诈,不亿不信,抑亦先觉者,是贤乎!”
\end{yuanwen}

孔子说:“。”

\begin{yuanwen}
微生亩谓孔子曰:“丘何为是栖栖者与?无乃为佞乎?”孔子曰:“非敢为佞也,疾固也。”
\end{yuanwen}

孔子说:“。”

\begin{yuanwen}
子曰:“骥不称其力,称其德也。”
\end{yuanwen}

孔子说:“。”

\begin{yuanwen}
或曰:“以德报怨,何如?”子曰:“何以报德?以直报怨,以德报德。”
\end{yuanwen}

孔子说:“。”

\begin{yuanwen}
子曰:“莫我知也夫!”子贡曰:“何为其莫知子也?”子曰:“不怨天,不尤人,下学而上达。知我者其天乎!”
\end{yuanwen}

孔子说:“。”

\begin{yuanwen}
公伯寮愬子路于季孙。子服景伯以告,曰:“夫子固有惑志于公伯寮,吾力犹能肆诸市朝。”子曰:“道之将行也与,命也;道之将废也与,命也。公伯寮其如命何?”
\end{yuanwen}

孔子说:“。”

\begin{yuanwen}
子曰:“贤者辟世,其次辟地,其次辟色,其次辟言。”子曰:“作者七人矣。”
\end{yuanwen}

孔子说:“。”

\begin{yuanwen}
子路宿于石门,晨门曰:“奚自?”子路曰:“自孔氏。”曰:“是知其不可而为之者与?”
\end{yuanwen}

孔子说:“。”

\begin{yuanwen}
子击磬于卫,有荷蒉而过孔氏之门者,曰:“有心哉,击磬乎!”既而曰:“鄙哉,硁硁乎!莫己知也,斯己而已矣。深则厉,浅则揭。”子曰:“果哉!末之难矣。”
\end{yuanwen}

孔子说:“。”

\begin{yuanwen}
子张曰:“《书》云,‘高宗谅阴,三年不言。’何谓也?”子曰:“何必高宗,古之人皆然。君薨,百官总己以听于冢宰三年。”
\end{yuanwen}

孔子说:“。”

\begin{yuanwen}
子曰:“上好礼,则民易使也。”
\end{yuanwen}

孔子说:“。”

\begin{yuanwen}
子路问君子,子曰:“修己以敬。”曰:“如斯而已乎?”曰:“修己以安人。”曰:“如斯而已乎?”曰:“修己以安百姓。修己以安百姓,尧、舜其犹病诸!”
\end{yuanwen}

孔子说:“。”

\begin{yuanwen}
原壤夷俟,子曰:“幼而不孙弟,长而无述焉,老而不死,是为贼!”以杖叩其胫。
\end{yuanwen}

孔子说:“。”

\begin{yuanwen}
阙党童子将命,或问之曰:“益者与?”子曰:“吾见其居于位也,见其与先生并行也。非求益者也,欲速成者也。”
\end{yuanwen}

孔子说:“。”


\chapter{卫灵公}



\begin{yuanwen}
卫灵公问陈于孔子,孔子对曰:“俎豆之事,则尝闻之矣;军旅之事,未之学也。”明日遂行。
\end{yuanwen}

孔子说:“。”

\begin{yuanwen}
在陈绝粮,从者病莫能兴。子路愠见曰:“君子亦有穷乎?”子曰:“君子固穷,小人穷斯滥矣。”
\end{yuanwen}

孔子说:“。”

\begin{yuanwen}
子曰:“赐也,女以予为多学而识之者与?”对曰:“然,非与?”曰:“非也,予一以贯之。”
\end{yuanwen}

孔子说:“。”

\begin{yuanwen}
子曰:“由,知德者鲜矣。”
\end{yuanwen}

孔子说:“。”

\begin{yuanwen}
子曰:“无为而治者其舜也与!夫何为哉?恭己正南面而已矣。”
\end{yuanwen}

孔子说:“。”

\begin{yuanwen}
子张问行,子曰:“言忠信,行笃敬,虽蛮貊之邦,行矣。言不忠信,行不笃敬,虽州里,行乎哉?立则见其参于前也,在舆则见其倚于衡也,夫然后行。”子张书诸绅。
\end{yuanwen}

孔子说:“。”

\begin{yuanwen}
子曰:“直哉史鱼!邦有道如矢,邦无道如矢。君子哉蘧伯玉!邦有道则仕,邦无道则可卷而怀之。”
\end{yuanwen}

孔子说:“。”

\begin{yuanwen}
子曰:“可与言而不与之言,失人;不可与言而与之言,失言。知者不失人亦不失言。”
\end{yuanwen}

孔子说:“。”

\begin{yuanwen}
子曰:“志士仁人无求生以害仁,有杀身以成仁。”
\end{yuanwen}

孔子说:“。”

\begin{yuanwen}
子贡问为仁,子曰:“工欲善其事,必先利其器。居是邦也,事其大夫之贤者,友其士之仁者。”
\end{yuanwen}

孔子说:“。”

\begin{yuanwen}
颜渊问为邦,子曰:“行夏之时,乘殷之辂,服周之冕,乐则《韶》、《舞》;放郑声,远佞人。郑声淫,佞人殆。”
\end{yuanwen}

孔子说:“。”

\begin{yuanwen}
子曰:“人无远虑,必有近忧。”
\end{yuanwen}

孔子说:“。”

\begin{yuanwen}
子曰:“已矣乎!吾未见好德如好色者也。”
\end{yuanwen}

孔子说:“。”

\begin{yuanwen}
子曰:“臧文仲其窃位者与!知柳下惠之贤而不与立也。”
\end{yuanwen}

孔子说:“。”

\begin{yuanwen}
子曰:“躬自厚而薄责于人,则远怨矣。”
\end{yuanwen}

孔子说:“。”

\begin{yuanwen}
子曰:“不曰‘如之何、如之何’者,吾末如之何也已矣。”
\end{yuanwen}

孔子说:“。”

\begin{yuanwen}
子曰:“群居终日,言不及义,好行小慧,难矣哉!”
\end{yuanwen}

孔子说:“。”

\begin{yuanwen}
子曰:“君子义以为质,礼以行之,孙以出之,信以成之。君子哉!”
\end{yuanwen}

孔子说:“。”

\begin{yuanwen}
子曰:“君子病无能焉,不病人之不己知也。”
\end{yuanwen}

孔子说:“。”

\begin{yuanwen}
子曰:“君子疾没世而名不称焉。”
\end{yuanwen}

孔子说:“。”

\begin{yuanwen}
子曰:“君子求诸己,小人求诸人。”
\end{yuanwen}

孔子说:“。”

\begin{yuanwen}
子曰:“君子矜而不争,群而不党。”
\end{yuanwen}

孔子说:“。”

\begin{yuanwen}
子曰:“君子不以言举人,不以人废言。”
\end{yuanwen}

孔子说:“。”

\begin{yuanwen}
子贡问曰:“有一言而可以终身行之者乎?”子曰:“其恕乎!己所不欲,勿施于人。”
\end{yuanwen}

孔子说:“。”

\begin{yuanwen}
子曰:“吾之于人也,谁毁谁誉?如有所誉者,其有所试矣。斯民也,三代之所以直道而行也。”
\end{yuanwen}

孔子说:“。”

\begin{yuanwen}
子曰:“吾犹及史之阙文也,有马者借人乘之,今亡矣夫!”
\end{yuanwen}

孔子说:“。”

\begin{yuanwen}
子曰:“巧言乱德,小不忍,则乱大谋。”
\end{yuanwen}

孔子说:“。”

\begin{yuanwen}
子曰:“众恶之,必察焉;众好之,必察焉。”
\end{yuanwen}

孔子说:“。”

\begin{yuanwen}
子曰:“人能弘道,非道弘人。”
\end{yuanwen}

孔子说:“。”

\begin{yuanwen}
子曰:“过而不改,是谓过矣。”
\end{yuanwen}

孔子说:“。”

\begin{yuanwen}
子曰:“吾尝终日不食、终夜不寝以思,无益,不如学也。”
\end{yuanwen}

孔子说:“。”

\begin{yuanwen}
子曰:“君子谋道不谋食。耕也馁在其中矣,学也禄在其中矣。君子忧道不忧贫。”
\end{yuanwen}

孔子说:“。”

\begin{yuanwen}
子曰:“知及之,仁不能守之,虽得之,必失之。知及之,仁能守之,不庄以涖之,则民不敬。知及之,仁能守之,庄以涖之,动之不以礼,未善也。”
\end{yuanwen}

孔子说:“。”

\begin{yuanwen}
子曰:“君子不可小知而可大受也,小人不可大受而可小知也。”
\end{yuanwen}

孔子说:“。”

\begin{yuanwen}
子曰:“民之于仁也,甚于水火。水火,吾见蹈而死者矣,未见蹈仁而死者也。”
\end{yuanwen}

孔子说:“。”

\begin{yuanwen}
子曰:“当仁不让于师。”
\end{yuanwen}

孔子说:“。”

\begin{yuanwen}
子曰:“君子贞而不谅。”
\end{yuanwen}

孔子说:“。”

\begin{yuanwen}
子曰:“事君,敬其事而后其食。”
\end{yuanwen}

孔子说:“。”

\begin{yuanwen}
子曰:“有教无类。”
\end{yuanwen}

孔子说:“。”

\begin{yuanwen}
子曰:“道不同,不相为谋。”
\end{yuanwen}

孔子说:“。”

\begin{yuanwen}
子曰:“辞达而已矣。”
\end{yuanwen}

孔子说:“。”

\begin{yuanwen}
师冕见,及阶,子曰:“阶也。”及席,子曰:“席也。”皆坐,子告之曰:“某在斯,某在斯。”师冕出。子张问曰:“与师言之道与?”子曰:“然,固相师之道也。”
\end{yuanwen}

孔子说:“。”

\chapter{季氏}

\begin{yuanwen}
季氏将伐颛臾,冉有、季路见于孔子,曰:“季氏将有事于颛臾。”孔子曰:“求,无乃尔是过与?夫颛臾,昔者先王以为东蒙主,且在邦域之中矣,是社稷之臣也。何以伐为?”冉有曰:“夫子欲之,吾二臣者皆不欲也。”孔子曰:“求,周任有言曰:‘陈力就列,不能者止。’危而不持,颠而不扶,则将焉用彼相矣?且尔言过矣,虎兕出于柙,龟玉毁于椟中,是谁之过与?”冉有曰:“今夫颛臾固而近于费,今不取,后世必为子孙忧。”孔子曰:“求,君子疾夫舍曰欲之而必为之辞。丘也闻,有国有家者,不患寡而患不均,不患贫而患不安。盖均无贫,和无寡,安无倾。夫如是,故远人不服则修文德以来之,既来之,则安之。今由与求也相夫子,远人不服而不能来也,邦分崩离析而不能守也,而谋动干戈于邦内。吾恐季孙之忧不在颛臾,而在萧墙之内也。”
\end{yuanwen}

孔子说:“。”

\begin{yuanwen}
孔子曰:“天下有道,则礼乐征伐自天子出;天下无道,则礼乐征伐自诸侯出。自诸侯出,盖十世希不失矣;自大夫出,五世希不失矣;陪臣执国命,三世希不失矣。天下有道,则政不在大夫;天下有道,则庶人不议。”
\end{yuanwen}

孔子说:“。”

\begin{yuanwen}
孔子曰:“禄之去公室五世矣,政逮于大夫四世矣,故夫三桓之子孙微矣。”
\end{yuanwen}

孔子说:“。”

\begin{yuanwen}
孔子曰:“益者三友,损者三友。友直、友谅、友多闻,益矣;友便辟、友善柔、友便佞,损矣。”
\end{yuanwen}

孔子说:“。”

\begin{yuanwen}
孔子曰:“益者三乐,损者三乐。乐节礼乐、乐道人之善、乐多贤友,益矣;乐骄乐、乐佚游、乐宴乐,损矣。”
\end{yuanwen}

孔子说:“。”

\begin{yuanwen}
孔子曰:“侍于君子有三愆:言未及之而言谓之躁,言及之而不言谓之隐,未见颜色而言谓之瞽。”
\end{yuanwen}

孔子说:“。”

\begin{yuanwen}
孔子曰:“君子有三戒:少之时,血气未定,戒之在色;及其壮也,血气方刚,戒之在斗;及其老也,血气既衰,戒之在得。”
\end{yuanwen}

孔子说:“。”

\begin{yuanwen}
孔子曰:“君子有三畏:畏天命,畏大人,畏圣人之言。小人不知天命而不畏也,狎大人,侮圣人之言。”
\end{yuanwen}

孔子说:“。”

\begin{yuanwen}
孔子曰:“生而知之者上也,学而知之者次也;困而学之又其次也。困而不学,民斯为下矣。”
\end{yuanwen}

孔子说:“。”

\begin{yuanwen}
孔子曰:“君子有九思:视思明,听思聪,色思温,貌思恭,言思忠,事思敬,疑思问,忿思难,见得思义。”
\end{yuanwen}

孔子说:“。”

\begin{yuanwen}
孔子曰:“见善如不及,见不善如探汤;吾见其人矣。吾闻其语矣。隐居以求其志,行义以达其道;吾闻其语矣,未见其人也。”
\end{yuanwen}

孔子说:“。”

\begin{yuanwen}
齐景公有马千驷,死之日,民无德而称焉;伯夷、叔齐饿于首阳之下,民到于今称之。其斯之谓与?”
\end{yuanwen}

孔子说:“。”

\begin{yuanwen}
陈亢问于伯鱼曰:“子亦有异闻乎?”对曰:“未也。尝独立,鲤趋而过庭,曰:‘学《诗》乎?’对曰:‘未也。’‘不学《诗》,无以言。’鲤退而学《诗》。他日,又独立,鲤趋而过庭,曰:‘学《礼》乎?’对曰:‘未也。’‘不学《礼》,无以立。’鲤退而学《礼》。闻斯二者。”陈亢退而喜曰:“问一得三,闻《诗》,闻《礼》,又闻君子之远其子也。”
\end{yuanwen}

孔子说:“。”

\begin{yuanwen}
邦君之妻,君称之曰夫人,夫人自称曰小童;邦人称之曰君夫人,称诸异邦曰寡小君;异邦人称之亦曰君夫人。
\end{yuanwen}

孔子说:“。”

\chapter{阳货}

\begin{yuanwen}
阳货欲见孔子,孔子不见,归孔子豚。孔子时其亡也而往拜之,遇诸涂。谓孔子曰:“来,予与尔言。”曰:“怀其宝而迷其邦,可谓仁乎?”曰:“不可。”“好从事而亟失时,可谓知乎?”曰:“不可!”“日月逝矣,岁不我与!”孔子曰:“诺,吾将仕矣。”
\end{yuanwen}

孔子说:“。”

\begin{yuanwen}
子曰:“性相近也,习相远也。”
\end{yuanwen}

孔子说:“。”

\begin{yuanwen}
子曰:“唯上知与下愚不移。”
\end{yuanwen}

孔子说:“。”

\begin{yuanwen}
子之武城,闻弦歌之声。夫子莞尔而笑,曰:“割鸡焉用牛刀?”子游对曰:“昔者偃也闻诸夫子曰:‘君子学道则爱人,小人学道则易使也。’”子曰:“二三
子,偃之言是也!前言戏之耳。”
\end{yuanwen}

孔子说:“。”

\begin{yuanwen}
公山弗扰以费畔,召,子欲往。子路不说,曰:“末之也已,何必公山氏之之也?”子曰:“夫召我者而岂徒哉?如有用我者,吾其为东周乎!”
\end{yuanwen}

孔子说:“。”

\begin{yuanwen}
子张问仁于孔子,孔子曰:“能行五者于天下为仁矣。”请问之,曰:“恭、宽、信、敏、惠。恭则不侮,宽则得众,信则人任焉,敏则有功,惠则足以使人。”
\end{yuanwen}

孔子说:“。”

\begin{yuanwen}
佛肸召,子欲往。子路曰:“昔者由也闻诸夫子曰。亲于其身为不善者,君子不入也。佛肸以中牟畔,子之往也,如之何?"子曰:“然。有是言也。不曰坚乎,磨而不磷?不曰白乎,涅而不缁。吾岂匏瓜也哉?焉能系而不食?”
\end{yuanwen}

孔子说:“。”

\begin{yuanwen}
子曰:“由也,女闻六言六蔽矣乎?”对曰:“未也。”“居!吾语女。好仁不好学,其蔽也愚;好知不好学,其蔽也荡;好信不好学,其蔽也贼;好直不好学,其蔽也绞;好勇不好学,其蔽也乱;好刚不好学,其蔽也狂。”
\end{yuanwen}

孔子说:“。”

\begin{yuanwen}
子曰:“小子何莫学夫诗!诗,可以兴,可以观,可以群,可以怨:迩之事父,远之事君.多识于鸟兽草木之名。”
\end{yuanwen}

孔子说:“。”

\begin{yuanwen}
子谓伯鱼曰:“女为《周南》、《召南》矣乎?人而不为《周南》、《召南》,其犹正墙面而立也与?”
\end{yuanwen}

孔子说:“。”

\begin{yuanwen}
子曰:“礼云礼云,玉帛云乎哉?乐云乐云,钟鼓云乎哉?”
\end{yuanwen}

孔子说:“。”

\begin{yuanwen}
子曰:“色厉而内荏,譬诸小人,其犹穿窬之盗也与?”
\end{yuanwen}

孔子说:“。”

\begin{yuanwen}
子曰:“乡愿,德之贼也。”
\end{yuanwen}

孔子说:“。”

\begin{yuanwen}
子曰:“道听而涂说,德之弃也。”
\end{yuanwen}

孔子说:“。”

\begin{yuanwen}
子曰:“鄙夫可与事君也与哉?其未得之也,患得之;既得之,患失之。苟患失之,无所不至矣。”
\end{yuanwen}

孔子说:“。”

\begin{yuanwen}
子曰:“古者民有三疾,今也或是之亡也。古之狂也肆,今之狂也荡;古之矜也廉,今之矜也忿戾;古之愚也直,今之愚也诈而已矣。”
\end{yuanwen}

孔子说:“。”

\begin{yuanwen}
子曰:“巧言令色,鲜矣仁。”
\end{yuanwen}

孔子说:“。”

\begin{yuanwen}
子曰:“恶紫之夺朱也,恶郑声之乱雅乐也,恶利口之覆邦家者。”
\end{yuanwen}

孔子说:“。”

\begin{yuanwen}
子曰:“予欲无言。”子贡曰:“子如不言,则小子何述焉?”子曰:“天何言哉?四时行焉,百物生焉,天何言哉?”
\end{yuanwen}

孔子说:“。”

\begin{yuanwen}
孺悲欲见孔子,孔子辞以疾。将命者出户,取瑟而歌,使之闻之。
\end{yuanwen}

孔子说:“。”

\begin{yuanwen}
宰我问:“三年之丧,期已久矣!君子三年不为礼,礼必坏;三年不为乐,乐必崩。旧谷既没,新谷既升,钻燧改火,期可已矣。”子曰:“食夫稻,衣夫锦,于女安乎?”曰:“安!”“女安则为之!夫君子之居丧,食旨不甘,闻乐不乐,居处不安,故不为也。今女安,则为之!”宰我出,子曰:“予之不仁也!子生三年,然后免于父母之怀。夫三年之丧,天下之通丧也,予也有三年之爱于其父母乎!”
\end{yuanwen}

孔子说:“。”

\begin{yuanwen}
子曰:“饱食终日,无所用心,难矣哉!不有博弈者乎?为之犹贤乎已。”
\end{yuanwen}

孔子说:“。”

\begin{yuanwen}
子路曰:“君子尚勇乎?”子曰:“君子义以为上。君子有勇而无义为乱,小人有勇而无义为盗。”
\end{yuanwen}

孔子说:“。”

\begin{yuanwen}
子贡曰:“君子亦有恶乎?”子曰:“有恶。恶称人之恶者,恶居下流而讪上者,恶勇而无礼者,恶果敢而窒者。”曰:“赐也亦有恶乎?”“恶徼以为知者,恶
不孙以为勇者,恶讦以为直者。”
\end{yuanwen}

孔子说:“。”

\begin{yuanwen}
子曰:“唯女子与小人为难养也,近之则不孙,远之则怨。”
\end{yuanwen}

孔子说:“。”

\begin{yuanwen}
子曰:“年四十而见恶焉,其终也已。”
\end{yuanwen}

孔子说:“。”


\chapter{微子}

\begin{yuanwen}
微子去之,箕子为之奴,比干谏而死。孔子曰:“殷有三仁焉。”
\end{yuanwen}

孔子说:“。”

\begin{yuanwen}
柳下惠为士师,三黜。人曰:“子未可以去乎?”曰:“直道而事人,焉往而不三黜?枉道而事人,何必去父母之邦?”
\end{yuanwen}

孔子说:“。”

\begin{yuanwen}
齐景公待孔子曰:“若季氏,则吾不能。”以季、孟之间待之,曰:“吾老矣,不能用也。”孔子行。
\end{yuanwen}

孔子说:“。”

\begin{yuanwen}
齐人归女乐,季桓子受之,三日不朝,孔子行。”
\end{yuanwen}

孔子说:“。”

\begin{yuanwen}
楚狂接舆歌而过孔子曰:“凤兮凤兮,何德之衰?往者不可谏,来者犹可追。已而已而,今之从政者殆而!”孔子下,欲与之言,趋而辟之,不得与之言。
\end{yuanwen}

孔子说:“。”

\begin{yuanwen}
长沮、桀溺耦而耕,孔子过之,使子路问津焉。长沮曰:“夫执舆者为谁?”子路曰:“为孔丘。”曰:“是鲁孔丘与?”曰:“是也。”曰:“是知津矣。”问于桀溺,桀溺曰:“子为谁?”曰:“为仲由。”曰:“是鲁孔丘之徒与?”对曰:“然。”曰:“滔滔者天下皆是也,而谁以易之?且而与其从辟人之士也,岂若从辟世之士哉?”耰而不辍。子路行以告,夫子怃然曰:“鸟兽不可与同群,吾非斯人之徒与而谁与?天下有道,丘不与易也。”
\end{yuanwen}

孔子说:“。”

\begin{yuanwen}
子路从而后,遇丈人,以杖荷蓧。子路问曰:“子见夫子乎?”丈人曰:“四体不勤,五谷不分,孰为夫子?”植其杖而芸,子路拱而立。止子路宿,杀鸡为黍而食之,见其二子焉。明日,子路行以告,子曰:“隐者也。”使子路反见之,至则行矣。子路曰:“不仕无义。长幼之节不可废也,君臣之义如之何其废之?欲洁其身而乱大伦。君子之仕也,行其义也,道之不行已知之矣。”
\end{yuanwen}

孔子说:“。”

\begin{yuanwen}
逸民:伯夷、叔齐、虞仲、夷逸、朱张、柳下惠、少连。子曰:“不降其志,不辱其身,伯夷、叔齐与!”谓:“柳下惠、少连降志辱身矣,言中伦,行中虑,其斯而已矣。”谓:“虞仲、夷逸隐居放言,身中清,废中权。我则异于是,无可无不可。”
\end{yuanwen}

孔子说:“。”

\begin{yuanwen}
太师挚适齐,亚饭干适楚,三饭缭适蔡,四饭缺适秦,鼓方叔入于河,播鼗武入于汉,少师阳、击磬襄入于海。
\end{yuanwen}

孔子说:“。”

\begin{yuanwen}
周公谓鲁公曰:“君子不施其亲,不使大臣怨乎不以,故旧无大故则不弃也,无求备于一人。”
\end{yuanwen}

孔子说:“。”

\begin{yuanwen}
周有八士:伯达、伯适、仲突、仲忽、叔夜、叔夏、季随、季騧。
\end{yuanwen}

孔子说:“。”

\chapter{子张}

\begin{yuanwen}
子张曰:“士见危致命,见得思义,祭思敬,丧思哀,其可已矣。”
\end{yuanwen}

孔子说:“。”

\begin{yuanwen}
子张曰:“执德不弘,信道不笃,焉能为有?焉能为亡?”
\end{yuanwen}

孔子说:“。”

\begin{yuanwen}
子夏之门人问交于子张,子张曰:“子夏云何?”对曰:“子夏曰:‘可者与之,其不可者拒之。’”子张曰:“异乎吾所闻。君子尊贤而容众,嘉善而矜不能。我之大贤与,于人何所不容?我之不贤与,人将拒我,如之何其拒人也?”
\end{yuanwen}

孔子说:“。”

\begin{yuanwen}
子夏曰:“虽小道必有可观者焉,致远恐泥,是以君子不为也。”
\end{yuanwen}

孔子说:“。”

\begin{yuanwen}
子夏曰:“日知其所亡,月无忘其所能,可谓好学也已矣。”
\end{yuanwen}

孔子说:“。”

\begin{yuanwen}
子夏曰:“博学而笃志,切问而近思,仁在其中矣。”
\end{yuanwen}

孔子说:“。”

\begin{yuanwen}
子夏曰:“百工居肆以成其事,君子学以致其道。”
\end{yuanwen}

孔子说:“。”

\begin{yuanwen}
子夏曰:“小人之过也必文。”
\end{yuanwen}

孔子说:“。”

\begin{yuanwen}
子夏曰:“君子有三变:望之俨然,即之也温,听其言也厉。”
\end{yuanwen}

孔子说:“。”

\begin{yuanwen}
子夏曰:“君子信而后劳其民,未信,则以为厉己也;信而后谏,未信,则以为谤己也。”
\end{yuanwen}

孔子说:“。”

\begin{yuanwen}
子夏曰:“大德不逾闲,小德出入可也。”
\end{yuanwen}

孔子说:“。”

\begin{yuanwen}
子游曰:“子夏之门人小子,当洒扫应对进退则可矣。抑末也,本之则无,如之何?”子夏闻之,曰:“噫,言游过矣!君子之道,孰先传焉?孰后倦焉?譬诸草木,区以别矣。君子之道焉可诬也?有始有卒者,其惟圣人乎!”
\end{yuanwen}

孔子说:“。”

\begin{yuanwen}
子夏曰:“仕而优则学,学而优则仕。”
\end{yuanwen}

孔子说:“。”

\begin{yuanwen}
子游曰:“丧致乎哀而止。”
\end{yuanwen}

孔子说:“。”

\begin{yuanwen}
子游曰:“吾友张也为难能也,然而未仁。”
\end{yuanwen}

孔子说:“。”

\begin{yuanwen}
曾子曰:“堂堂乎张也,难与并为仁矣。”
\end{yuanwen}

孔子说:“。”

\begin{yuanwen}
曾子曰:“吾闻诸夫子,人未有自致者也,必也亲丧乎!”
\end{yuanwen}

孔子说:“。”

\begin{yuanwen}
曾子曰:“吾闻诸夫子,孟庄子之孝也,其他可能也;其不改父之臣与父之政,是难能也。”
\end{yuanwen}

孔子说:“。”

\begin{yuanwen}
孟氏使阳肤为士师,问于曾子。曾子曰:“上失其道,民散久矣。如得其情,则哀矜而勿喜!”
\end{yuanwen}

孔子说:“。”

\begin{yuanwen}
子贡曰:“纣之不善,不如是之甚也。是以君子恶居下流,天下之恶皆归焉。”
\end{yuanwen}

孔子说:“。”

\begin{yuanwen}
子贡曰:“君子之过也,如日月之食焉。过也人皆见之,更也人皆仰之。”
\end{yuanwen}

孔子说:“。”

\begin{yuanwen}
卫公孙朝问于子贡曰:“仲尼焉学?”子贡曰:“文武之道未坠于地,在人。贤者识其大者,不贤者识其小者,莫不有文武之道焉,夫子焉不学?而亦何常师之有?”
\end{yuanwen}

孔子说:“。”

\begin{yuanwen}
叔孙武叔语大夫于朝曰:“子贡贤于仲尼。”子服景伯以告子贡,子贡曰:“譬之宫墙,赐之墙也及肩,窥见室家之好;夫子之墙数仞,不得其门而入,不见宗庙之美、百官之富。得其门者或寡矣,夫子之云不亦宜乎!”
\end{yuanwen}

孔子说:“。”

\begin{yuanwen}
叔孙武叔毁仲尼,子贡曰:“无以为也,仲尼不可毁也。他人之贤者,丘陵也,犹可逾也;仲尼,日月也,无得而逾焉。人虽欲自绝,其何伤于日月乎?多见其不知量也。”
\end{yuanwen}

孔子说:“。”

\begin{yuanwen}
陈子禽谓子贡曰:“子为恭也,仲尼岂贤于子乎?”

子贡曰:“君子一言以为知,一言以为不知,言不可不慎也。夫子之不可及也,犹天之不可阶而升也。夫子之得邦家者,所谓立之斯立,道之斯行,绥之斯来,动之斯和。其生也荣,其死也哀,如之何其可及也?”
\end{yuanwen}

孔子说:“。”

\chapter{尧曰}

\begin{yuanwen}
尧曰:“咨!尔舜!天之历数在尔躬,允执其中。四海困穷,天禄永终。”
\end{yuanwen}

孔子说:“。”

\begin{yuanwen}
舜亦以命禹。曰:“予小子履,敢用玄牡,敢昭告于皇皇后帝:有罪不敢赦,帝臣不蔽,简在帝心。朕躬有罪,无以万方;万方有罪,罪在朕躬。”周有大赉,善人是富。“虽有周亲,不如仁人。百姓有过,在予一人。”谨权量,审法度,修废官,四方之政行焉。兴灭国,继绝世,举逸民,天下之民归心焉。所重:民、食、丧、祭。宽则得众,信则民任焉,敏则有功,公则说。
\end{yuanwen}

孔子说:“。”

\begin{yuanwen}
子张问于孔子曰:“何如斯可以从政矣?”
\end{yuanwen}

孔子说:“。”

\begin{yuanwen}
子曰:“尊五美,屏四恶,斯可以从政矣。”
\end{yuanwen}

孔子说:“。”

\begin{yuanwen}
子张曰:“何谓五美?”
\end{yuanwen}

孔子说:“。”

\begin{yuanwen}
子曰:“君子惠而不费,劳而不怨,欲而不贪,泰而不骄,威而不猛。”
\end{yuanwen}

孔子说:“。”

\begin{yuanwen}
子张曰:“何谓惠而不费?”
\end{yuanwen}

孔子说:“。”

\begin{yuanwen}
子曰:“因民之所利而利之,斯不亦惠而不费乎?择可劳而劳之,又谁怨?欲仁而得仁,又焉贪?君子无众寡,无小大,无敢慢,斯不亦泰而不骄乎?君子正其衣冠,尊其瞻视,俨然人望而畏之,斯不亦威而不猛乎?”
\end{yuanwen}

孔子说:“。”

\begin{yuanwen}
子张曰:“何谓四恶?”
\end{yuanwen}

孔子说:“。”

\begin{yuanwen}
子曰:“不教而杀谓之虐;不戒视成谓之暴;慢令致期谓之贼;犹之与人也,出纳之吝谓之有司。”
\end{yuanwen}

孔子说:“。”

\begin{yuanwen}
孔子曰:“不知命,无以为君子也;不知礼,无以立也;不知言,无以知人也。”
\end{yuanwen}

孔子说:“。”

\end{document}