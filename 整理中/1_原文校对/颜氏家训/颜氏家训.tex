% 颜氏家训
% 颜氏家训.tex

\documentclass[12pt,UTF8]{ctexbook}

% 设置纸张信息。
\usepackage[a4paper,twoside]{geometry}
\geometry{
	left=25mm,
	right=25mm,
	bottom=25.4mm,
	bindingoffset=10mm
}

% 设置字体,并解决显示难检字问题。
\xeCJKsetup{AutoFallBack=true}
\setCJKmainfont{SimSun}[BoldFont=SimHei, ItalicFont=KaiTi, FallBack=SimSun-ExtB]

% 目录 chapter 级别加点(.)。
\usepackage{titletoc}
\titlecontents{chapter}[0pt]{\vspace{3mm}\bf\addvspace{2pt}\filright}{\contentspush{\thecontentslabel\hspace{0.8em}}}{}{\titlerule*[8pt]{.}\contentspage}

% 设置 part 和 chapter 标题格式。
\ctexset{
	chapter/name={},
	chapter/number={}
}

% 设置古文原文格式。
\newenvironment{yuanwen}{\bfseries\zihao{4}}

% 设置署名格式。
\newenvironment{shuming}{\hfill\bfseries\zihao{4}}

% 注脚每页重新编号,避免编号过大。
\usepackage[perpage]{footmisc}

\title{\heiti\zihao{0} 颜氏家训}
\author{颜之推}
\date{}

\begin{document}

\maketitle
\tableofcontents

\frontmatter
\chapter{前言、序言}

\mainmatter

% 增加空行
~\\

% 增加字间间隔,适用于三字经、诗文等。
 \qquad  

\chapter{序致篇}

\begin{yuanwen}

夫圣贤之书,教人诚孝,慎言检迹,立身扬名,亦已备矣。魏、晋已来,所著诸子,理重事复,递相模学,犹屋下架屋,床上施床耳。吾今所以复为此者,非敢轨物范世也,业以整齐门内,提撕子孙。夫同言而信,信其所亲;同命而行,行其所服。禁童子之暴谑,则师友之诫不如傅婢之指挥,止凡人之斗阋,则尧、舜之道不如寡妻之诲谕。吾望此书为汝曹之所信,犹贤于傅婢寡妻耳。

吾家风教,素为整密。昔在龆龀,便蒙诱诲;每从两兄,晓夕温凊,规行矩步,安辞定色,锵锵翼翼,若朝严君焉。赐以优言,问所好尚,励短引长,莫不恳笃。年始九岁,便丁荼蓼,家涂离散,百口索然。慈兄鞠养,苦辛备至;有仁无威,导示不切。虽读《礼》、《传》,微爱属文,颇为凡人之所陶染,肆欲轻言,不修边幅。年十八九,少知砥砺,习若自然,卒难洗荡。二十已后,大过稀焉;每常心共口敌,性与情竞,夜觉晓非,今悔昨失,自怜无教,以至于斯。追思平昔之指,铭肌镂骨,非徒古书之诫,经目过耳也。故留此二十篇,以为汝曹后车耳。
\end{yuanwen}

\chapter{教子篇}

\begin{yuanwen}

上智不教而成,下愚虽教无益,中庸之人,不教不知也。古者,圣王有胎教之法:怀子三月,出居别宫,目不邪视,耳不妄听,音声滋味,以礼节之。书之玉版,藏诸金匮。生子咳提,师保固明,孝仁礼义,导习之矣。凡庶纵不能尔,当及婴稚,识人颜色,知人喜怒,便加教诲,使为则为,使止则止。比及数岁,可省笞罚。父母威严而有慈,则子女畏慎而生孝矣。吾见世间,无教而有爱,每不能然;饮食运为,恣其所欲,宜诫翻奖,应呵反笑,至有识知,谓法当尔。骄慢已习,方复制之,捶挞至死而无威,忿怒日隆而增怨,逮于成长,终为败德。孔子云:“少成若天性,习惯如自然”是也。俗谚曰:“教妇初来,教儿婴孩。”诚哉斯语!  凡人不能教子女者,亦非欲陷其罪恶;但重于呵怒,伤其颜色,不忍楚挞惨其肌肤耳。当以疾病为谕,安得不用汤药针艾救之哉?又宜思勤督训者,可愿苛虐于骨肉乎?诚不得已也!  王大司马母魏夫人,性甚严正;王在湓城时,为三千人将,年逾四十,少不如意,犹捶挞之,故能成其勋业。梁元帝时,有一学士,聪敏有才,为父所宠,失于教义:一言之是,遍于行路,终年誉之;一行之非,掩藏文饰,冀其自改。年登婚宦,暴慢日滋,竟以言语不择,为周逖抽肠衅鼓云。  父子之严,不可以狎;骨肉之爱,不可以简。简则慈孝不接,狎则怠慢生焉。由命士以上,父子异宫,此不狎之道也;抑搔痒痛,悬衾箧枕,此不简之教也。或问曰:“陈亢喜闻君子之远其子,何谓也?”对曰:“有是也。盖君子之不亲教其子也。《诗》有讽刺之辞,《礼》有嫌疑之诫,《书》有悖乱之事,《春秋》有邪僻之讥,《易》有备物之象。皆非父子之可通言,故不亲授耳。”  齐武成帝子琅邪王,太子母弟也,生而聪慧,帝及后并笃爱之,衣服饮食,与东宫相准。帝每面称之曰:“此黠儿也,当有所成。”及太子即位,王居别宫,礼数优僭,不与诸王等;太后犹谓不足,常以为言。年十许岁,骄恣无节,器服玩好,必拟乘舆;常朝南殿,见典御进新冰,钩盾献早李,还索不得,遂大怒,訽曰:“至尊已有,我何意无?”不知分齐,率皆如此。识者多有叔段、州吁之讥。后嫌宰相,遂矫诏斩之,又惧有救,乃勒麾下军士,防守殿门;既无反心,受劳而罢,后竟坐此幽薨。  人之爱子,罕亦能均;自古及今,此弊多矣。贤俊者自可赏爱,顽鲁者亦当矜怜,有偏宠者,虽欲以厚之,更所以祸之。共叔之死,母实为之。赵王之戮,父实使之。刘表之倾宗覆族,袁绍之地裂兵亡,可为灵龟明鉴也。  齐朝有一士大夫,尝谓吾曰:“我有一儿,年已十七,颇晓书疏。教其鲜卑语及弹琵琶,稍欲通解,以此伏事公卿,无不宠爱,亦要事也。”吾时俛而不答。异哉,此人之教子也!若由此业,自致卿相,亦不愿汝曹为之。
\end{yuanwen}

\chapter{兄弟篇}

\begin{yuanwen}

夫有人民而后有夫妇,有夫妇而后有父子,有父子而后有兄弟:一家之亲,此三而已矣。自兹以往,至于九族,皆本于三亲焉,故于人伦为重者也,不可不笃。兄弟者,分形连气之人也,方其幼也,父母左提右挈,前襟后裾,食则同案,衣则传服,学则连业,游则共方,虽有悖乱之人,不能不相爱也。及其壮也,各妻其妻,各子其子,虽有笃厚之人,不能不少衰也。娣姒之比兄弟,则疏薄矣;今使疏薄之人,而节量亲厚之恩,犹方底而圆盖,必不合矣。惟友悌深至,不为旁人之所移者,免夫!  二亲既殁,兄弟相顾,当如形之与影,声之与响;爱先人之遗体,惜己身之分气,非兄弟何念哉?兄弟之际,异于他人,望深则易怨,地亲则易弭。譬犹居室,一穴则塞之,一隙则涂之,则无颓毁之虑;如雀鼠之不恤,风雨之不防,壁陷楹沦,无可救矣。仆妾之为雀鼠,妻子之为风雨,甚哉!  兄弟不睦,则子侄不爱;子侄不爱,则群从疏薄;群从疏薄,则僮仆为仇敌矣。如此,则行路皆踖其面而蹈其心,谁救之哉!人或交天下之士,皆有欢爱,而失敬于兄者,何其能多而不能少也!人或将数万之师,得其死力,而失恩于弟者,何其能疏而不能亲也!  娣姒者,多争之地也,使骨肉居之,亦不若各归四海,感霜露而相思,伫日月之相望也。况以行路之人,处多争之地,能无间者,鲜矣。所以然者,以其当公务而执私情,处重责而怀薄义也;若能恕己而行,换子而抚,则此患不生矣。  人之事兄,不可同于事父,何怨爱弟不及爱子乎?是反照而不明也。沛国刘琎,尝与兄连栋隔壁,呼之数声不应,良久方答;怪问之,乃曰:“向来未着衣帽故也。”以此事兄,可以免矣。  江陵王玄绍,弟孝英、子敏兄弟三人,特相友爱,所得甘旨新异,非共聚食,必不先尝,孜孜色貌,相见如不足者。及西台陷没,玄绍以形体魁梧,为兵所围,二弟争共抱持,各求代死,终不得解,遂并命尔。
\end{yuanwen}

\chapter{后娶篇}

\begin{yuanwen}

吉甫,贤父也,伯奇,孝子也。以贤父御孝子,合得终于天性,而后妻间之,伯奇遂放。曾参妇死,谓其子曰:“吾不及吉甫,汝不及伯奇。”王骏丧妻,亦谓人曰:“我不及曾参,子不如华、元。”并终身不娶,此等足以为诫。其后,假继惨虐孤遗,离间骨肉,伤心断肠者,何可胜数。慎之哉!慎之哉!  江左不讳庶孽,丧室之后,多以妾媵终家事;疥癣蚊虻,或未能免,限以大分,故稀斗之耻。河北鄙于侧出,不预人流,是以必须重娶,至于三四,母年有少于子者。后母之弟,与前妇之兄,衣服饮食,爰及婚宦,至于士庶贵贱之隔,俗以为常,身没之后,辞讼盈公门,谤辱彰道路,子诬母为妾,弟黜兄为佣,播扬先人之辞迹,暴露祖考之长短,以求直己者,往往而有,悲夫!自古奸臣佞妾,以一言陷人者众矣!况夫妇之义,晓夕移之,婢仆求容,助相说引,积年累月,安有孝子乎?此不可不畏。  凡庸之性,后夫多宠前夫之孤,后妻必虐前妻之子;非唯妇人怀嫉妒之情,丈夫有沉惑之僻,亦事势使之然也。前夫之孤,不敢与我子争家,提携鞠养,积习生爱,故宠之;前妻之子,每居己生之上,宦学婚嫁,莫不为防焉,故虐之。异姓宠则父母被怨,继亲虐则兄弟为仇,家有此者,皆门户之祸也。  思鲁等从舅殷外臣,博达之士也,有子基、谌,皆已成立,而再娶王氏。基每拜见后母,感慕呜咽,不能自持,家人莫忍仰视。王亦凄怆,不知所容,旬月求退,便以礼遣,此亦悔事也。
\end{yuanwen}

\chapter{治家篇}

\begin{yuanwen}

夫风化者,自上而行于下者也,自先而施于后者也。是以父不慈则子不孝,兄不友则弟不恭,夫不义则妇不顺矣。父慈而子逆,兄友而弟傲,夫义而妇陵,则天之凶民,乃刑戮之所摄,非训导之所移也。  笞怒废于家,则竖子之过立见;刑罚不中,则民无所措手足。治家之宽猛,亦犹国焉。  孔子曰:“奢则不孙,俭则固。与其不孙也,宁固。”又云:“如有周公之才之美,使骄且吝,其余不足观也已。”然则可俭而不可吝已。俭者,省约为礼之谓也;吝者,穷急不恤之谓也。今有施则奢,俭则吝;如能施而不奢,俭而不吝,可矣。  生民之本,要当稼稽而食,桑麻以衣。蔬果之畜,园场之所产;鸡豚之善,树圈之所生。复及栋宇器械,樵苏脂烛,莫非种殖之物也。至能守其业者,闭门而为生之具以足,但家无盐井耳。令北土风俗,率能躬俭节用,以赡衣食。江南奢侈,多不逮焉。  梁孝元世,有中书舍人,治家失度,而过严刻。妻妾遂共货刺客,伺醉而杀之。  世间名士,但务宽仁,至于饮食饷馈,僮仆减损,施惠然诺,妻子节量,狎侮宾客,侵耗乡党,此亦为家之巨蠹矣。  裴子野有疏亲故属饥寒不能自济者。皆收养之。家素清贫,时逢水旱,二石米为薄粥,仅得遍焉,躬自同之,常无厌色。邺下有一领军,贪积已甚,家童八百,誓满一千,朝夕每人肴膳,以十五钱为率,遇有客旅,更无以兼。后坐事伏法,籍其家产,麻鞋一屋,弊衣数库,其余财宝,不可胜言。南阳有人,为生奥博,性殊俭吝。冬至后女婿谒之,乃设一铜瓯酒,数脔獐肉,婿恨其单率,一举尽之,主人愕然,俯仰命益,如此者再,退而责其女曰:“某郎好酒,故汝常贫。”及其死后,诸子争财,兄遂杀弟。  妇主中馈,惟事酒食衣服之礼耳,国不可使预政,家不可使干蛊。如有聪明才智,识达古今,正当辅佐君子,助其不足。必无牝鸡晨鸣,以致祸也。  江东妇女,略无交游,其婚姻之家,或十数年间来相识者,惟以信命赠遗,致殷勤焉。邺下风俗,专以妇持门户,争讼曲直,造请逢迎,车乘填街衢,绮罗盈府寺,代子求官,为夫诉屈,此乃恒代之遗风平?南间贫素,皆事外饰,车乘衣服,必贵整齐,家人妻子,不免饥寒。河北人事,多由内政,绮罗金翠,不可废阙,羸马悴奴,仅充而已,倡和之礼,或尔汝之。  河北妇人,织任组训之事,黼黻 锦绣罗绮之工,大优于江东也。  太公曰:“养女太多,一费也。” 陈蕃曰:“盗不过五女之门。”女之为累,亦以深矣。然天生蕃民,先人传体,其如之何?世人多不举女,贼行骨肉,岂当如此而望福于天乎?吾有疏亲,家饶妓媵,诞育将及,便遣阍竖守之,体有不安,窥窗倚户,若生女者,辄持将去,母随号泣,使人不忍闻也。

妇人之性,率宠子婿而虐儿妇,宠婿则兄弟之怨生焉,虐妇则姊妹之谗行焉。然则女之行留,皆得罪于其家者,母实为之。至有谚曰:“落索阿姑餐。”此其相报也。家之常弊,可不诫哉!  婚姻素对,靖候成规。近世嫁娶,遂有卖女纳财,买妇输绢,比量父祖,计较锱铢,责多还少,市井无异。或猥婿在门,或傲妇擅室,贪荣求利,反招羞耻,可不慎欤?  借人典籍,皆须爱护,先有缺坏,就为科治,此亦士大夫百行之一也。济阳江禄,读书未竟,虽有急速,必待卷束整齐,然后得起,故无损败,人不厌其求假焉。或有狼藉几案,分散部帙,多为童幼婢妾之所点污。风雨虫鼠之所毁伤,实为累德。吾每读圣人之书,未尝不肃敬对之。其故纸有《五经》词义及贤达姓名,不敢秽用也。  吾家巫觋祷请,绝于言议;符书章酸,亦无祈焉。并汝曹所见也,勿为妖妄之费。
\end{yuanwen}

\chapter{风操篇}

\begin{yuanwen}

吾观《礼经》,圣人之教:箕帚匕箸,咳唾唯诺,执烛沃盥,皆有节文,亦为至矣。但既残缺,非复全书;其有所不载,及世事变改者,学达君子,自为节度,相承行之,故世号士大夫风操。而家门颇有不同,所见互称长短;然其阡陌,亦自可知。昔在江南,目能视而见之,耳能听而闻之;蓬生麻中,不劳翰墨。汝曹生于戎马之间,视听之所不晓,故聊记录,以传示子孙。
汝曹生于戎马之间,视听之所不晓,故聊记录,以传示子孙。
《礼》曰:“见似目瞿,闻名心瞿。”有所感触,侧怆心眼,若在从容平常之地,幸须申其情耳。必不可避,亦当忍之,犹如伯叔、兄弟,酷类先人,可得终身肠断与之绝耶?又“临文不讳,庙中不讳,君所无私讳”。盖知闻名须有消息,不必期於颠沛而走也。梁世谢举”,甚有声誉,闻讳必哭,为世所讥。又有臧逢世,臧严之子也,笃学修行,不坠门风,孝元经牧江州,遣往建昌督事,郡县民庶,竞修笺书,朝夕辐辏,几案盈积,书有称“严寒”者,必对之流涕,不省取记,多废公事,物情怨骇’,竟以不办而还。此并过事也。
近在扬都,有一士人讳审,而与沉氏交给周厚,沉与其书,名而不姓,此非人情也。
昔候霸之子孙,称其祖父曰家公;陈思王称其父为家父,母为家母;潘尼 称其祖曰家祖:古人之所行,令人之所笑也。今南北风俗,言其祖及二亲,无云人言,言已世父“,以次第称之,不云“家”者,以尊于父,不敢“家”也。凡言姑、姊妹、女子子,已嫁则以夫氏称之,在室则以次第称之,言礼成他族,不得云“家”也。子孙不得称“家”者,轻略之也。蔡邕书集呼其姑、姊为家姑、家姊,班固书集亦云家孙,今并不行也。
凡与人言,称彼祖父母、世父母;父母及长姑,皆加“尊”字,自叔父母已下,则加“贤”子,尊卑之差也。王羲之书,称彼之母与自称己母同,不云“尊”字,今所非也。
昔者,王侯自称孤、寡、不谷。自兹以降,虽孔子圣师,及闸人言皆称名也。后虽有臣、仆之称,行者盖亦寡焉。江南轻重,各有谓号,具诸《书仪》。北人多称名者,乃古之遗风。吾善其称名焉。
古人皆呼伯父、叔父,而今世多单呼伯、叔。从父兄弟姊妹已孤,而对其前呼其母为伯叔母,此未可避者也。兄弟之子已孤,与他人言,对孤者前呼为兄子。弟子,颇为不忍,北土人多呼为佳。案《尔雅》、《丧服经》、《左传》,侄虽名通男女,并是对姑之称,晋世以来,始呼叔侄。今呼为侄,于理为胜也。
古者,名以正体,字以表德,名终则讳之,字乃可以为孙氏。孔子弟子记事者,皆称仲尼;吕后微时,尝字高祖为季;至汉麦种,字其叔父曰丝;王丹与侯霸子语,字霸为君房。江南至今不讳字也。河北人士全不辨之,名亦呼为字,字固呼为字。尚书王元景兄弟,皆号名人,其父名云,字罗汉,一皆讳之,其馀不足怪也。
偏傍之书,死有归杀,子孙逃窜,莫肯在家;画瓦书符,作诸厌胜;丧出之日,门前然火,户外列灰,祓送家鬼,章断注连。凡如此比,不近有情,乃儒雅之罪人,弹议所当加也。
《礼经》:“父之遗书,母之杯圈,感其手口之泽,不忍读用。”政为常所讲习,讎校缮写,及偏如服用,有迹可思者耳。若寻常坟典,为生什物,安可悉废之乎?既不读用,无容散逸,惟当缄保,以留后世耳。
江南风俗,儿生一期,为制新衣,盥浴装饰,男则用弓矢纸笔,女则刀尺针缕,并加饮食之物,及珍宝服玩,置之儿前,观其发意所取,以验贪廉愚智,名之为试儿。亲表聚集,致宴享焉。自兹已后,二亲若在,每至此日,尝有酒食之事耳。无教之徒,虽已孤露,其日皆为供顿,酣畅声乐,不知有所感伤。梁孝元年少之时,每八月六日载诞之辰,常设斋讲;自阮修容薨殁之后,此事亦绝。
人有忧疾,则呼天地父母,自古而然。今世讳避,触途急切。而江东士庶,痛则称祢。祢是父之庙号,父在无容称庙,父殁何容辄呼?《苍颉篇》有倄字,《训诂》云:“痛而謼也,音羽罪反。”今北人痛则呼之。《声类》音于耒反,今南人痛或呼之。此二音随其乡俗,并可行也。
梁世被系劾者,子孙弟侄,皆诣阙三日,露跣陈谢;子孙有官,自陈解职。子则草屩粗衣,蓬头垢面,周章道路,要候执事,叩头流血,申诉冤枉。若配徒隶,诸子并立草庵于所署门,不敢宁宅,动经旬日,官司驱遣,然后始退。江南诸宪司弹人事,事虽不重,而以教义见辱者,或被轻系而身死狱户者,皆为怨仇,子孙三世不交通矣。到洽为御史中丞,初欲弹刘孝绰,其兄溉先与刘善,苦谏不得,乃诣刘涕泣告别而去。

四海之人,结为兄弟,亦何容易,必有志均义敌,令终如始者,方可议之。一尔之后,命子拜伏,呼为丈人,申父交之敬,身事彼亲,亦宜加礼。比见北人甚轻此节,行路相逢,便定昆季,望年观貌,不择是非,至有结父为兄、托子为弟者。
昔者,周公一沐三握发,一饭三吐餐,以接白屋之士,一日所见者七十余人。晋文公以沐辞竖头须,致有图反之诮。门不停宾,古所贵也。失教之家,阍寺无礼,或以主君寝食嗔怒,拒客未通,江南深以为耻。黄门侍郎裴之礼,号善为士大夫,有如此辈,对宾杖之;其门生僮仆,接于他人,折旋俯仰,辞色应对,莫不肃敬,与主无别也。
\end{yuanwen}

\chapter{慕贤篇}

\begin{yuanwen}

古人云:“千载一圣,犹旦暮也;五百年一贤,犹比髆心。”言圣贤之难得,疏阔如此。傥遭不世明达君子,安可不攀附景仰之乎?吾生于乱世,长于戎马,流离播越,闻见已多;所值名贤,未尝不心醉魂迷向慕之也。人在年少,神情未定,所与款狎,熏渍陶染,言笑举动,无心于学,潜移暗化,自然似之;何况操履艺能,较明易习者也?是以与善人居,如入芝兰之室,久而自芳也;与恶人居,如入鲍鱼之肆,久而自臭也。墨子悲于染丝,是之谓矣。君子必慎交游焉。孔子曰:“无友不如己者。”颜、闵之徒,何可世得!但优于我,便足贵之。
世人多蔽,贵耳贱目,重遥轻近。少长周旋,如有贤哲,每相狎侮,不加礼敬;他乡异县,微借风声,延颈企踵,甚于饥渴。校其长短,核其精麤,或彼不能如此矣。所以鲁人谓孔子为东家丘,昔虞国宫之奇,少长于君,君狎之,不纳其谏,以至亡国,不可不留心也。
用其言,弃其身,古人所耻。凡有一言一行,取于人者,皆显称之,不可窃人之美,以为己力;虽轻虽贱者,必归功焉。窃人之财,刑辟之所处;窃人之美,鬼神之所责。
梁孝元前在荆州,有丁觇者,洪亭民耳,颇善属文,殊工草隶;孝元书记,一皆使之。军府轻贱,多未之重,耻令子弟以为楷法,时云:“丁君十纸,不敌王褒数字。”吾雅爱其手迹,常所宝持。孝元尝遣典签惠编送文章示萧祭酒,祭酒问云:“君王比赐书翰,及写诗笔,殊为佳手,姓名为谁?那得都无声问?”编以实答。子云叹曰:“此人后生无比,遂不为世所称,亦是奇事。”于是闻者稍复刮目。稍仕至尚书仪曹郎,末为晋安王侍读,随王东下。及西台陷殁,简牍湮散,丁亦寻卒于扬州;前所轻者,后思一纸,不可得矣。
侯景初入建业,台门虽闭,公私草扰,各不自全。太子左卫率羊侃坐东掖门,部分经略,一宿皆办,遂得百余日抗拒凶逆。于时,城内四万许人,王公朝士,不下一百,便是恃侃一人安之,其相去如此。古人云:“巢父、许由,让于天下;市道小人,争一钱之利。”亦已悬矣。
齐文宣帝即位数年,便沈湎纵恣,略无纲纪;尚能委政尚书令杨遵彦,内外清谧,朝野晏如,各得其所,物无异议,终天保之朝。遵彦后为孝昭所戮,刑政于是衰矣。斛律明月齐朝折冲之臣,无罪被诛,将士解体,周人始有吞齐之志,关中至今誉之。此人用兵,岂止万夫之望而已哉!国之存亡,系其生死。
张延隽之为晋州行台左丞,匡维主将,镇抚疆埸,储积器用,爱活黎民,隐若敌国矣。群小不得行志,同力迁之;既代之后,公私扰乱,周师一举,此镇先平。齐亡之迹,启于是矣。
\end{yuanwen}

\chapter{勉学篇}

\begin{yuanwen}

自古明王圣帝,犹须勤学,况凡庶乎!此事遍于经史,吾亦不能郑重,聊举近世切要,以启寤汝耳。士大夫之弟,数岁已上,莫不被教,多者或至《礼》、《传》,少者不失《诗》、《论》。及至冠婚,体性梢定,因此天机,倍须训诱。有志向者,遂能磨砺,以就素业;无履立者,自兹堕慢,便为凡人。人生在世,会当有业,农民则计量耕稼,商贾则讨论货贿,工巧则致精器用,伎艺则沉思法术,武夫则惯习弓马,文士则讲议经书。多见士大夫耻涉农商,羞务工伎,射则不能穿札,笔则才记姓名,饱食醉酒,忽忽无事,以此销日,以此终年。或因家世馀绪,得一阶半级,便自为足,全忘修学,及有吉凶大事,议论得失,蒙然张口,如坐云雾,公私宴集,谈古赋诗,塞默低头,欠伸而已。有识旁观,代其入地。何惜数年勤学,长受一生愧辱哉!
梁朝全盛之时,贵游子弟,多无学术,至於谚曰:“上车不落则著作,体中何如则秘书。”无不熏衣剃面,傅粉施朱,驾长檐车,跟高齿履,坐棋子方褥,凭斑丝隐囊,列器玩于左右,从容出入,望若神仙,明经求第,则顾人答策,三九公宴,则假手赋诗,当尔之时,亦快士也。及离乱之后,朝市 迁革,铨衡选举,非复曩者之亲,当路秉权,不见昔时之党,求诸身而无所得,施之世而无所用,被揭而丧珠,失皮而露质,兀若枯木,泊若穷流,鹿独戎马之间,转死沟壑之际,当尔之时,诚驽材也。有学艺者,触地而安。自荒乱以来,诸见俘虏,虽百世小人,知读《论语》、《孝经》者,尚为人师;虽千载冠冕,不晓书记者,莫不耕田养马,以此现之,安可不自勉耶?若能常保数百卷书,千载终不为小人也。
有客难主人曰:“吾见强弩长戟,诛罪安民,以取公侯者有吴;文义习吏,匡时富国,以取卿相者有吴;学备古今,才兼文武,身无禄位,妻子饥寒者,不可胜数,安足贵学乎?”主人对曰:“夫命之穷达,犹金玉木石也;修以学艺,犹磨莹雕刻也。金玉之磨莹,自美其矿璞;木石之段块,自丑其雕刻。安可言木石之雕刻,乃胜金玉之矿璞哉?不得以有学之贫贱,比於无学之富贵也。且负甲为兵,咋笔为吏,身死名灭者如牛毛,角立杰出者如芝草;握素披黄,吟道咏德,苦辛无益者如日蚀,逸乐名利者如秋茶,岂得同年而语矣。且又闻之:生而知之者上,学而知之者次。所以学者,欲其多知明达耳。必有天才,拔群出类,为将则暗与孙武、吴起同术,执政则悬得管仲、子产之教,虽未读书,吾亦谓之学矣。今子即不能然,不师古之踪迹,犹蒙被而卧耳。”
人见邻里亲戚有佳快者,使子弟慕而学之,不知使学古人,何其蔽也哉?世人但知跨马被甲,长槊强弓,便云我能为将;不知明乎天道,辩乎地利,比量逆顺,鉴达兴亡之妙也。但知承上接下,积财聚谷,便云我能为相;不知敬鬼事神,移风易俗,调节阴阳,荐举贤圣之至也。但知私财不入,公事夙办,便云我能治民;不知诚己刑物,执辔如组,反风灭火,化鸱为凤之术也。但知抱令守律,早刑晚舍,便云我能平狱;不知同辕观罪,分剑追财,假言而好露,不问而情得之察也。表及农商工贾,廝役奴隶,钓鱼屠肉,饭牛牧羊,皆有先达,可为师表,博学求之,无不利於事也。
夫学者所以求益耳。见人读数十卷书,便自高大,凌忽长者,轻慢同列;人疾之如仇敌,恶之如鸱枭。如此以学自损,不如无学也。
人生小幼,精神专利,长成已后,思虑散逸,固须早教,勿失机也。吾七岁时,诵《灵光殿赋》,至於今日,十年一理,犹不遗忘。二十以外,所诵经书,一月废置,便至荒芜矣。然人有坎禀,失于盛年,犹当晚学,不可自弃。孔子 曰:“五十以学《易》,可以无大过矣。”魏武、袁遗,老而弥笃;此皆少学而至老不倦也。曾子十七乃学,名闻天下;荀卿五十始来游学,犹为硕儒;公孙弘四十余方读《春秋》,以此遂登丞相;朱云亦四十始学《易》、《论语》,皇甫谧二十始受《孝经》、《论语》,皆终成大儒:此并早迷而晚寤也。世人婚冠未学,便称迟暮,因循面墙,亦为愚耳。幼而学者,如日出之光;老而学者,如秉烛夜行,犹贤乎瞑目而无见者也。
学之兴废,随世轻重。汉时贤俊,皆以一经弘圣人之道,上明天时,下该 人事,用此致卿相者多矣。末俗已来不复尔,空守章句,但诵师言,施之世务,殆无一可。故士大夫子弟,皆以博涉为贵,不肯专儒。梁朝皇孙以下,总之年 ,必先入学,观其志尚,出身己后,便从文吏,略无卒业者。冠冕,而为上者,则有何胤、刘献、明山宾、周舍、朱异、周弘正、贺琛、贺革、萧子政、刘绥等,兼通文史,不徒讲说也。洛阳亦闻崔浩、张伟、刘芳,邺下又见邢子才:此四儒者,虽好经术,亦以才博擅名。如此诸贤,故为上品。以外率多田野间人,音辞鄙陋,风操蚩拙,相与专固,无所堪能。问一言辄酬数百,责其指归,或无要会。那下谚云:“博士买驴,书卷三纸,未有‘驴’字。”使汝以此为师,令人气塞。孔子曰:“学也,禄在其中矣。”今勤无益之事,恐非业也。夫圣人之书,所以设教,但明练经文,粗通注义,常使言行有得,亦足为人;何必“仲尼居”即须两纸疏义,燕寝、讲堂,亦复何在?以此得胜,宁有益乎?光阴可惜,譬诸逝水。当博览机要,以济功业,必能兼美,吾无间焉。
俗间儒士,不涉群书,经纬之外,义疏而已。吾初八邺,与博陵崔文彦交游,尝说《王粲集》中难郑玄《尚书》事,崔转为诸儒道之。始将发口,悬见排蹙,云:“文集只有诗赋、铭、诔,岂当论经书事乎?且先儒之中,未闻有王粲也。”崔笑而退,竟不以《粲集》示之。魏收之在议曹,与诸博士议宗庙事,引据《汉似》,博士笑曰:“未闻《汉书》得证经术。”收便忿怒,都不复言,取《韦玄成传》,掷之而起。博士一夜共披寻之,达明,乃来谢曰:“不谓玄成如此学也。”
邺平之后,见徒入关。思鲁尝谓吾曰:“朝无禄位,家无积财,当肆筋力,以申供养。每被课笃,勤劳经史,未知为子,可得安乎?”吾命之曰:“子当以养为心,父当以学为教。使汝弃学徇财,丰吾衣食,食之安得甘?衣之安得暖?若务先王之道,绍家世之业,藜羹褐,我自欲之。”
校订书籍,亦何容易,自扬雄、刘向,方称此职耳。观天下书未遍,不得妄下雌黄。或彼以为非,此以为是,或本同末异,或两文皆欠,不可偏信一隅也。
\end{yuanwen}

\chapter{文章篇}

\begin{yuanwen}

夫文章者,原出《五经》:诏命策檄,生于《书》者也;序述论议,生于《易》者也;歌咏赋颂,生于《诗》者也;祭祀哀诔,生于《礼》者也;书奏箴铭,生于《春秋》者也。朝廷宪章,军旅誓诰,敷显仁义,发明功德,牧民建国,施用多途。至于陶冶性灵,从容讽谏,入其滋味,亦乐事也。行有余力,则可习之。然而自古文人,多陷轻薄:屈原露才扬己,显暴君过;宋玉体貌容冶,见遇俳优;东方曼倩,滑稽不雅;司马长卿,窃赀无操;王褒过章《僮约》;扬雄德败《美新》;李陵降辱夷虏;刘歆反覆莽世;傅毅党附权门;班固盗窃父史;赵元叔抗竦过度;冯敬通浮华摈压;马季长佞媚获诮;蔡伯喈同恶受诛;吴质诋忤乡里;曹植悖慢犯法;杜笃乞假无厌;路粹隘狭已甚;陈琳实号粗疏;繁钦性无检格;刘桢屈强输作;王粲率躁见嫌;孔融、祢衡,诞傲致殒;杨修、丁廙,扇动取毙;阮籍无礼败俗,稽康凌物凶终,傅玄念斗免官,孙楚矜夸凌上,陆机犯顺履险,潘岳干没取危,颜延年负气摧黜,谢灵运空疏乱纪,王元长凶贼自诒,谢玄晖侮慢见及。凡此诸人,皆其翘秀者,不能悉记,大较如此。至於帝王,亦或未免。自昔天子而有才华者,唯汉武、魏太祖、文帝、明帝、宋孝武帝,皆负世议,非懿德之君也。自子游、子夏、荀况、孟轲、枚乘、贾谊、苏武、张衡、左思之传,有盛名而免过患者,时复闻之,但其损败居多耳。每尝思之,原其所积,文章之体,标举兴会,发引性灵,使人矜伐,故忽於持操,果於进取。今世文士,此患弥切,一事惬当,一句清巧,神厉九霄,志凌千载,自吟自赏,不觉更有傍人。加以砂砾所伤,惨於矛戟,讽刺之祸,速乎风尘。深宜防虑,以保元吉。
学问有利钝,文章有巧拙。钝学累功,不妨精熟;拙文研思,终归蚩鄙。但成学士,自足为人;必乏天才,勿强操笔。吾见世人,至无才思,自谓清华,流布丑拙,亦以众矣,江南号为“许痴符”。近在并州,有一士族,好为可笑诗赋,铫弊邢、魏诸公,众共嘲弄,虚相赞说,便击牛釃酒,招延声誉。其妻明鉴妇人也,泣而谏之,此人叹曰:“才华不为妻子所容,何况行路!”至死不觉。自见之谓明,此诚难也。
学为文章,先谋亲友,得其评裁,知可施行,然后出手,慎勿师心自任,取笑旁人也。自古执笔为文者,何可胜言。然至於宏丽精华,不过数十篇耳。但使不失体裁,辞意可观,便称才士。要须动俗盖世,亦俟河之清乎。
凡为文章,犹人乘骐骥,虽有逸气,当以衔勒制之,勿使流乱轨躅,放意填坑岸也。
文章当以理致为心旅,气调为筋骨,事义为皮肤,华而为冠冕。今世相承,趋末弃本,率多浮艳,辞与理竞,辞胜而理伏;事与才争,事繁而才损,放逸者流宕而忘归,穿凿者补缀而不足。
时俗如此,安能独违,但务去泰去甚耳。必有盛才重誉,改革体裁者,实吾所希。
古人之文,宏才逸气,体度风格,去今实远;但缉缀疏朴,未为密致耳。今世音律谐靡,章句偶对,讳避精详,贤於往昔多矣。宜以古之制裁为本,今之辞调为末,并须两存,不可偏弃也。
\end{yuanwen}

\chapter{名实篇}

\begin{yuanwen}

名之与实,犹形之与影也。德艺周厚,则名必善焉;容色姝丽,则影必美焉。今不修身而求令名于世者,犹貌甚恶而责妍影于镜也。上士忘名,中士立名,下士窃名。忘名者,体道合德,享鬼神之福祜,非所以求名也;立名者,修身慎行,惧荣观之不显,非所以让名也;窃名者,厚貌深奸,於浮华之虚称,非所以得名也。
吾见世人,清名登而金贝入,信誉显而然诺亏,不知后之矛戟,毁前之干橹也!虑子贱云:“诚于此者形於彼。”人之虚实真伪在乎心,无不见平迹,但察之未熟耳。一为察之所鉴,巧伪不如拙诚,承之以羞大矣。伯石让卿,王莽辞政,当于尔时,自以巧密,后人书之,留传万代,可为骨寒毛竖也。近有大贵,以孝著声,前后居丧,哀毁逾制,亦足以高於人矣;而尝於苫块之中,以巴豆涂脸,遂使成疮,表哭泣之过,左右童竖,不能掩之,益使外人谓其居处饮食皆为不信。以一伪丧百诚者,乃贪名不已故也!
有一士族,读书不过二三百卷,天才钝拙,而家世殷厚,雅自矜持,多以酒犊珍玩,交诸名士。甘其饵者,递共吹嘘,朝廷以为文华,亦尝出境聘。东莱王韩晋明笃好文学,疑彼制作,多非机杼,遂设宴言,面相讨试。竟日欢谐,辞人满席,属音赋韵,命笔为诗,彼造次即成,了非向韵,众客各自沉吟,遂无觉者。韩退叹曰:“果如所量。”
治点子弟文章,以为声价,大弊事也。一则不可常继,终露其情;二则学者有凭,益不精励。
邺下有一少年,出为襄国今,颇自勉笃,公事经怀,每加抚恤,以求声誉。凡遣兵役,握手送离,或齎梨枣饼饵,人人赠别,云:“上命相烦,情所不忍,道路饥渴,以此见思。”民庶称之,不容於口。及迁为泗州别驾,此费日广,不可常周。一有伪情,触涂难继,功绩遂损败矣。
\end{yuanwen}

\chapter{涉务篇}

\begin{yuanwen}

士君子之处世,贵能有益於物耳,不徒高谈虚论,左琴右书,以费人君禄位也!国之用材,大较不过六事:一则朝廷之臣,取其鉴达治体,经纶博雅;二则文史之臣,取其著述宪章,不忘前古;三则军旅之臣,取其断决有谋,强干习事;四则藩屏之臣,取其明练风俗,清白爱民;五则使命之臣,取其识变从宜,不辱君命;六则兴造之臣,取其程功节费,开略有术:此则皆勤学守行者所能办也。人性有长短,岂责具美于六涂哉?但当皆晓指趣,能守一职,便无愧耳。
吾见世中文学之士,品藻古今,若指诸掌,及有试用,多无所堪。居承平之世,不知有丧乱之祸;处庙堂之下,不知有战阵之急;保俸禄之资,不知有耕稼之苦;肆吏民之上,不知有劳役之勤:故难可以应世经务也。晋朝南渡,优借士族,故江南冠带有才干者,擢为令仆已下尚书郎、中书舍人已上,典掌机要。其馀文义之士。多迂诞浮华,不涉世务,纤微过失,又惜行捶楚,所以处於清高,盖护其短也。至於台阁令史,主书监帅,诸王签省,并晓习吏用,济办时须,纵有小人之态,皆可鞭枚肃督,故多见委使,盖用其长也。人每不自量,举世怨梁武帝父子爱小人而疏士大夫,此亦眼不能见其睫耳。
梁世士大夫,皆尚褒衣博带,大冠高履,出则车舆,入则扶持,郊郭之内,无乘马者。周弘正为宣城王所爱,给一果下马,常服御之,举朝以为放达。至乃尚书郎乘马,则纠劾之。及侯景之乱,肤脆骨柔,不堪行步,体羸气弱,不耐寒暑,坐死仓猝者,往往而然。建康今王复,性既儒雅,未尝乘骑,见马嘶贲陆梁。莫不震慑,乃谓人曰:“正是虎,何故名为马乎?”其风俗至此。
古人欲知稼穑之艰难,斯盖贵谷务本之道也。夫食为民天,民非食不生矣,三日不粒,父子不能相存。耕种之,休组之,对获之,载积之,打拂之,簸扬之,凡几涉手,而入仓廪,安可轻农事而贵末业哉?江南朝士,因晋中兴,南渡江,卒为羁旅,至今八九世,未有力田,悉资俸禄而食耳。假令有者,皆信僮仆为之,未尝目观起一拨土,耕一株苗;不知几月当下,几月当收,安识世间馀务乎?故治官则不了,营家则不办,皆优闲之过也。
\end{yuanwen}

\chapter{省事篇}

\begin{yuanwen}

铭金人云:“无多言,多言多败;无多事,多事多患。”至哉斯戒也!能走者夺其翼,善飞者减其指,有角者无上齿,丰后者无前足,盖天道不使物有兼焉也。古人云:“多为少善,不如执一;鼷鼠五能,不成伎术。”近世有两人,朗悟士也,性多营综,略无成名,经不足以待问,史不足以讨论,文章无可传於集录,书迹未堪以留爱玩,卜筮射六得三,医药治十差五,音乐在数十人下,弓矢在千百人中,天文、画绘、棋博、鲜卑语、胡书、煎胡桃油、炼锡为银,如此之类,略得梗概,皆不通熟。惜乎!以彼神明,若省其异端,当精妙也。
\end{yuanwen}

\chapter{止足篇}

\begin{yuanwen}

《礼》云:“欲不可纵,志不可满。”宇宙可臻其极,情性不知其穷,唯在少欲知止,为立涯限尔。先祖靖侯戒子侄曰:“汝家书生门户,世无富贵,自今仕宦不可过二千石,婚姻勿贪势家。”吾终身服膺,以为名言也。
天地鬼神之道,皆恶满盈,谦虚冲损,可以免害。人生衣趣以覆寒露,食趣以塞饥乏耳。形骸之内,尚不得奢靡,己身之外,而欲穷骄泰邪?周穆王、秦始皇、汉武帝富有四海,贵为天子,不知纪极,犹自败累,况士庶乎?常以二十口家,奴婢盛多不可出二十人,良田十顷,堂室才蔽风雨,车马仅代杖策,蓄财数万,以拟吉凶急速。不啻此者,以义散之;不至此者,勿非道求之。
\end{yuanwen}

\chapter{诫兵篇}

\begin{yuanwen}

颜氏之先,本乎邹、鲁,或分入齐,世以儒雅为业,遍在书记。仲尼门徒,升堂者七十有二,颜氏居八人焉。秦汉魏晋,下逮齐梁,未有用兵以取达者。春秋世颜高、颜鸣、颜息、颜羽之徒,皆一斗夫耳。齐有颜涿聚,赵有颜最,汉末有颜良,宋有颜延之,并处将军之任,竟以颠覆。汉郎颜驷,自称好武,更无事迹。颜忠以党楚王受诛,颜俊以据武威见杀,得姓已来,无清操者,唯此二人,皆罹祸败。顷世乱离,衣冠之士,虽无身手,或聚徒众,违弃素业,侥幸战功。吾既赢薄,你惟前代,故置心於此,子孙志之。孔子力翘门关,不以力闻,此圣证也。吾见今世士大夫,才有气干,便倚赖之,不能被甲执兵,以卫社稷,但微行险服,逞弄拳腕,大则陷危亡,小则贻耻辱,遂无免者。
国之兴亡,兵之胜败,博学所至,幸讨论之。入帷幄之中,参庙堂之上,不能为主尽规以谋社稷,君子所耻也。然而每见文士,颇读兵书,微有经略。若居承平之世,脾睨宫阃,幸灾乐祸,首为逆乱,诖误善良;如在兵革之时,构扇反覆,纵横说诱,不识存亡,强相扶戴:此皆陷身灭族之本也。诫之哉!诫之哉!
习五兵,便乘骑,正可称武夫尔。今世士大夫,但不读书,即称武夫儿,乃饭囊酒瓮也。
\end{yuanwen}

\chapter{养生篇}

\begin{yuanwen}

神仙之事,未可全诬;但性命在天,或难钟值。人生居世,触途牵絷;幼少之日,既有供养之勤;成立之年,便增妻孥之累。衣食资须,公私驱役;而望遁迹山林,超然尘滓,千万不遇一尔。加以金玉之费,炉器所须,益非贫士所办。学如牛毛,成如麟角。华山之下,白骨如莽,何有可遂之理?考之内教,纵使得仙,终当有死,不能出世,不愿汝曹专精於此。若其爱养神明,调护气息,慎节起卧,均适寒暄,禁忌食饮,将饵药物,遂其所禀,不为夭折者,吾无间然”。诸药饵法,不废世务也。庚肩吾常服槐实,年七十馀,目看细字,须发犹黑。邺中朝士,有单服杏仁、枸杞、黄精、白术、车前得益者甚多,不能—一说尔。吾尝患齿,摇动欲落,饮食热冷,皆苦疼痛。见《抱朴子》牢齿之法,早朝叩齿三百下为良;行之数日,即便平愈,今恒持之。此辈小术,无损於事,亦可修也。凡欲饵药,陶隐居《太清方》中总录甚备,但须精审,不可轻脱。近有王爱州在邺学服松脂不得节度,肠塞而死,为药所误者其多。
夫养生者先须虑祸,全身保性,有此生然后养之,勿徒养其无生也。单豹养於内而丧外,张毅养於外而丧内,前贤所戒也。稽康著《养身》之论,而以傲物受刑,石崇冀服饵之征,而以贪溺取祸,往事之所迷也。
夫生不可不惜,不可苟惜。涉险畏之途,干祸难之事,贪欲以伤生,谗慝而致死,此君子之所惜哉!行诚孝而见贼,履仁义而得罪,丧身以全家,泯躯而济国,君子不咎也。自乱离已来,吾见名臣贤士,临难求生,终为不救,徒取窘辱,令人愤懑。
\end{yuanwen}

\chapter{归心篇}

\begin{yuanwen}

内外两教,本为一体,渐积为异,深浅不同。内典初门,设五种禁,外典仁、义、礼、智、信,皆与之符。仁者,不杀之禁也;义者,不盗之禁也;礼者,不邪之禁也;智者,不酒之禁也;信者,不妄之禁也。至如畋狩军旅,燕享刑罚,因民之性,不可卒除,就为之节,使不淫滥尔。归周、孔而背释宗,何其迷也!
释三曰:“开辟已来,不善人多而善人少,何由悉责其精洁乎?见有名僧向行,异而不说;若睹凡僧流俗,便生非毁。且学者之不勤,岂教者之为过?俗僧之学经律,何异世人之学《诗》、《礼》?以《诗》、《礼》之教,格朝廷之人,略无全行者;以经律之禁,格出家之辈,而独责无犯哉?且阙行之臣,犹求禄位;毁禁之侣,何惭供养乎?其於戒行,自当有犯。一披法服,已堕僧数,岁中所计,斋讲诵持.比诸白衣,犹不啻山海也。
形体虽死,精神犹存。人生在世,望於后身似不相属;及其殁后,则与前身似犹老少朝夕耳。世有魂神,示现梦想,或降童妾,或感妻孥,求索饮食,征须福祜,亦为不少矣。今人贫贱疾苦,莫不怨尤前世不修功业。以此而论,安可不为之作地乎?夫有子孙,自是天地间一苍生耳,何预身事,而乃爱护,遗其基址。况於已之神爽,顿欲弃之哉?凡夫蒙蔽,不见未来,故言彼生与今非一体耳……
世有痴人,不识仁义,不知富贵并由天命。为子娶妇,恨其生资不足,倚作舅始之尊,蛇虺其性,毒口加诬,不识忌讳,骂辱妇之父母,却成教妇不孝己身,不顾他恨。但怜已之子女,不爱己之儿妇。如此之人,阴纪其过,鬼夺其算。慎不可与为邻,何况交结乎?避之哉!

\end{yuanwen}

\chapter{书证篇}

\begin{yuanwen}

太公《六韬》,有天陈、地陈、人陈、云鸟之陈。《论语》曰:“卫灵公问陈於孔子。”《左传》:“为鱼丽之陈。”俗本多作“阜”旁车乘之“车”。案诸陈队,并作陈、郑之“陈”。夫行陈之义,取於陈列耳,此“六书”为假借也。《苍》、《雅》及近世字书,皆无别字,唯王羲之《小学章》独“阜”旁作“车”。纵复俗行,不宜追改《六韬》、《论语》、《左传》也。
“也”是语已及助句之辞,文籍备有之矣。河北经传,悉略此字。其间字有不可得无者。至如“伯也执殳”,“於旅也语”,“回也屡空”心,“风,风也,教也”,及《诗传》云“不戢,我也;不傩,傩也”,“不多,多也”如斯之类,傥削此文,颇成废阙。《诗》言:“青青子衿”,《传》曰:“青衿,青领也,学子之服。”按古者斜领下连於衿,故谓领为衿,孙炎、郭璞注《尔雅》,曹大家注《列女传》,并云:“衿,交领也。”邺下《诗》本既无“也”字,群儒固谬说云:“青衿、青领,是衣两处之名,皆以青为饰。”用释“青青”二字,其失大矣。又有俗学,闻经、传中时须“也”字,辄以意加之,每不得所,益成可笑。
《后汉书》:“酷吏樊晔为天水太守,凉州为之歌曰:‘宁见乳虎穴,不入冀府寺,”而江南书本“穴”皆误作“六”,学士因循,迷而不寐。夫虎豹穴居,事之较者,所以班超云:“不探虎穴,安得虎子?”宁当论其六七耶?
客有难主人曰:“今之经典,子皆谓非,《说文》所言,子皆云是,然则许慎胜孔子乎?”主人拊掌大笑,应之曰:“今之经典,皆孔子手迹耶?”客曰:“今之《说文》,皆许慎手迹乎?”答曰:“许慎检以六文,贯以部分,使不得误,误则觉之。孔子存其义而不论其文也。先儒尚得改文从意,何况书写流传邓?必如《左传》止戈为武,反正为乏,虫为蛊,亥有二首六身之类,后人自不得辄改也,安敢以《说文》校其是非哉?且馀亦不专以《说文》为是也,其有援引经传,与今乖者,未之敢从。又相如《封禅书》曰:‘导一茎六穗于扈,牺双解共抵之兽,此导训择,光武诏云:‘非徒有豫养导择之劳,是也。而《说文》云:‘道是禾名。’引《封禅书》为证;无妨自当有禾名道,非相如所用也。‘禾一茎六穗于扈,’岂成文乎?纵使相如天才鄙拙,强为此语,则下句当云‘麟双角共抵之兽,’不得云牺也。吾尝笑许纯儒,不达文章之体,如此之流,不足凭信,大抵服其为书,隐括有条例,剖析穷根源,郑玄注书,往往引以为证;若不信其说,则冥冥不知一点一画,有何意焉。”
\end{yuanwen}

\chapter{音辞篇}

\begin{yuanwen}

夫九州之人,言语不同,生民已来,固常然矣。自春秋标齐言之传,离骚目楚词之经,此盖其较明之初也。后有扬雄着方言,其言大备。然皆考名物之同异,不显声读之是非也。逮郑玄注六经,高诱解吕览、淮南,许慎造说文,刘熹制释名,始有譬况假借以证音字耳。而古语与今殊别,其间轻重清浊,犹未可晓;加以内言外言、急言徐言、读若之类,益使人疑。孙叔言创尔雅音义,是汉末人独知反语。至于魏世,此事大行。高贵乡公不解反语,以为怪异。自兹厥后,音韵锋出,各有土风,递相非笑,指马之谕,未知孰是。共以帝王都邑,参校方俗,考核古今,为之折衷。搉而量之,独金陵与洛下耳。南方水土和柔,其音清举而切诣,失在浮浅,其辞多鄙俗。北方山川深厚,其音沈浊而(金化)钝,得其质直,其辞多古语。然冠冕君子,南方为优;闾里小人,北方为愈。易服而与之谈,南方士庶,数言可辩;隔垣而听其语,北方朝野,终日难分。而南染吴、越,北杂夷虏,皆有深弊,不可具论。其谬失轻微者,则南人以钱为涎,以石为射,以贱为羡,以是为舐;北人以庶为戍,以如为儒,以紫为姊,以洽为狎。如此之例,两失甚多。至邺已来,唯见崔子约、崔瞻叔侄,李祖仁、李蔚兄弟,颇事言词,少为切正。李季节着音韵决疑,时有错失;阳休之造切韵,殊为疏野。吾家儿女,虽在孩稚,便渐督正之;一言讹替,以为己罪矣。云为品物,未考书记者,不敢辄名,汝曹所知也。
古今言语,时俗不同;著述之人,楚、夏各异。苍颉训诂,反稗为逋卖,反娃为于乖;战国策音刎为免,穆天子传音谏为间;说文音戛为棘,读皿为猛;字林音看为口甘反,音伸为辛;韵集以成、仍、宏、登合成两韵,为、奇、益、石分作四章;李登声类以系音羿,刘昌宗周官音读乘若承;此例甚广,必须考校。前世反语,又多不切,徐仙民毛诗音反骤为在遘,左传音切椽为徒缘,不可依信,亦为众矣。今之学士,语亦不正;古独何人,必应随其伪僻乎?通俗文曰:“入室求曰搜。”反为兄侯。然则兄当音所荣反。今北俗通行此音,亦古语之不可用者。玙璠,鲁人宝玉,当音余烦,江南皆音藩屏之藩。岐山当音为奇,江南皆呼为神只之只。江陵陷没,此音被于关中,不知二者何所承案。以吾浅学,未之前闻也。
北人之音,多以举、莒为矩;唯李季节云:“齐桓公与管仲于台上谋伐莒,东郭牙望见桓公口开而不闭,故知所言者莒也。然则莒、矩必不同呼。”此为知音矣。
夫物体自有精麤,精麤谓之好恶;人心有所去取,去取谓之好恶。此音见于葛洪、徐邈。而河北学士读尚书云好生恶杀。是为一论物体,一就人情,殊不通矣。
甫者,男子之美称,古书多假借为父子;北人遂无一人呼为甫者,亦所未喻。唯管仲、范增之号,须依字读耳。
案:诸字书,焉者鸟名,或云语词,皆音于愆反。自葛洪要用字苑分焉字音训:若训何训安,当音于愆反,“于焉逍遥”,“于焉嘉客”,“焉用佞”,“焉得仁”之类是也;若送句及助词,当音矣愆反,“故称龙焉”,“故称血焉”,“有民人焉”,“有社稷焉”,“托始焉尔”,“晋、郑焉依”之类是也。江南至今行此分别,昭然易晓;而河北混同一音,虽依古读,不可行于今也。
邪者,未定之词。左传曰:“不知天之弃鲁邪?抑鲁君有罪于鬼神邪?”庄子云:“天邪地邪?”汉书云:“是邪非邪?”之类是也。而北人即呼为也,亦为误矣。难者曰:“系辞云:‘乾坤,易之门户邪?’此又为未定辞乎?”答曰:“何为不尔!上先标问,下方列德以折之耳。”
古人云:“膏粱难整。”以其为骄奢自足,不能克励也。吾见王侯外戚,语多不正,亦由内染贱保傅,外无良师友故耳。梁世有一侯,尝对元帝饮谑,自陈“痴钝”,乃成“飔段”,元帝答之云:“飔异凉风,段非干木。”谓“郢州”为“永州”,元帝启报简文,简文云:‘庚辰吴入,遂成司隶。”如此之类,举口皆然。元帝手教诸子侍读,以此为诫。
河北切攻字为古琮,与工、公、功三字不同,殊为僻也。比世有人名暹,自称为纤;名琨,自称为衮;名洸,自称为汪;名(素勺),自称为獡。非唯音韵舛错,亦使其儿孙避讳纷纭矣。
\end{yuanwen}

\chapter{杂艺篇}

\begin{yuanwen}

真草书迹,微须留意。江南谚云:“尺牍书疏,千里面目也”承晋宋馀俗,相与事之,故无顿狼狈者。吾幼承门业,加性爱重,所见法书亦多,而玩习功夫颇至,遂不能佳者,良由无分故也。然而此艺不须过精。夫巧者劳而智者忧,常为人所役使,更觉为累。韦仲将遗戒,深有以也。
王逸少风流才士,萧散名人,举世唯知其书,翻以能自蔽也。萧子云每叹曰:“吾著《齐书》,勒成一典,文章弘义,自谓可观,唯以笔迹得名,亦异事也。”王褒地胃清华,才学仇敏,后虽入关,亦被礼遇,犹以书工,崎岖碑碣<间,辛苦笔砚之役,尝悔恨曰:“假使吾不知书,可不至今日邪?”以此观之,慎勿以书自命。虽然,廝猥之人,以能书拔擢者多矣。故“道不同不相为谋”也。
梁氏秘阁散逸以来,吾见二王真草多矣,家中尝得十卷,方知陶隐居、阮交州、萧祭酒诸书,莫不得羲之之体,故是书之渊源。萧晚节所变,乃右军年少时法也。
晋宋以来,多能书者,故其时俗,递相染尚,所有部帙,楷正可观,不无俗字,非为大损。至梁天监之间,斯风未变。大同之末,讹替滋生,萧子云改易字体,邵陵王颁行伪字,朝野翕然,以为楷式,画虎不成,多所伤败。至为一字,唯见数点,或妄斟酌,逐便转移。尔后坟籍,略不可看。北朝丧乱之馀,书迹鄙陋,加以专辄造字,猥拙甚於江南,乃以“百”“念”为“忧”,“言”“反”为“变”,“不”“用”为“罢”,“追”“来”为“归”,”“更”“生”为“苏”,“先”“人”为“老”,如此非一,遍满经传。唯有姚元标工於楷隶,留心小学,后生师之者众,泊于齐末,秘书缮写,贤於往日多矣。
江南闾里间有《画书赋》,乃陶隐居弟子林道士所为。其人未甚识字,轻为轨则,托名贵师,世俗传信,后生颇为所误也。
画绘之工,亦为妙矣,自古名士,多或能之。吾家尝有梁元帝手画蝉雀白团扇及马图,亦难及也。武烈太子偏能写真,坐上宾客,随宜点染,即成数人,以问童孺,皆知姓名矣。萧贲、刘孝先、刘灵,并文学已外,复佳此法。玩阅古今,特可宝爱。若官未通显,每被公私使令,亦为猥役。吴县顾士端出身湘东王国待郎,后为镇南府刑狱参军,有子曰庭,西朝中书舍人,父子并有琴、书之艺,尤妙丹青,常被元帝所使,每怀羞恨。彭城刘岳,囊之子也,仕为骠骑府管记、平氏县令,才学快士,而画绝伦。后随武陵王入蜀,下牢之败,遂为陆护军画支江寺壁,与诸工巧杂处。向使三贤都不晓画,直运素业,岂见此耻乎?
孤矢之利,以威天下,先王所以现德择贤,亦济身之急务也。江南谓世之常射,以为“兵射”,冠冕儒生,多不习此。别有“博射”,弱弓长箭,施於准的,揖让升降,以行礼焉,防御寇难,了无所益,乱离之后,此术遂亡。河北文士,率晓“兵射”,非直葛洪一箭,已解追兵,三九宴集,常縻荣赐。虽然,要轻禽,截狡兽,不愿汝辈为之。
算术亦是六艺要事。自古儒士论天道。定律历者,智学通之。然可以兼明,不可以专业。江南此学殊少,唯范阳祖恒精之,位至南康太守。河北多晚此术。
医方之事,取妙极难,不劝汝曾以自命也。微解药性,小小和合,居家得以救急,亦为胜事,皇甫谧、殷仲堪则其人也。
《礼》曰:“君子无故不彻琴瑟。”古来名士,多所爱好。洎于梁初,衣冠子孙,不知琴者,号有所阙。大同以末,斯风顿尽。然而此乐音音雅致,有深味哉!今世曲解,虽变于古,犹足以畅神情也。唯不可令有称誉,见役勋贵,处之下坐,以取残杯冷炙之辱。戴安道犹遭之,况尔曹乎!

\end{yuanwen}

\chapter{终制篇}

\begin{yuanwen}

死者,人之常分,不可免也。吾年十九,值梁家丧乱,其间与白刃为伍者,亦常数辈,幸承馀福,得至於今。古人云:“五十不为夭。”吾已六十馀,故心坦然,不以残年为念。先有风气之疾,常疑奄然,聊书素怀,以为汝诫。
先君先夫人皆未还建邺旧山,旅葬江陵东郭。承圣末,已启求扬都,欲营迁靥,蒙诏赐银百两,已於扬州小郊北地烧砖。便值本朝沦没,流离如此。数十年间,绝於还望。今虽混一,家道馨穷,何由办此奉营资费?且扬都污毁,无复遗,还被下湿,未为得计。自咎自责,贯心刻髓。
孔子之葬亲也,云:“古者墓而不坟,丘东西南北之人也,不可以弗识也。”於是封之崇四尺。然则君子应世行道,亦有不守坟墓之时,况为事际所逼也。吾今羁旅,身若浮云,竟未知何乡是吾葬地,唯当气绝便埋之耳。汝曹宜以传业扬名为务,不可顾恋朽壤,以取湮没也。
\end{yuanwen}

\backmatter

\end{document}