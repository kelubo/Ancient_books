% 朱子家训
% 朱子家训.tex

\documentclass[12pt,UTF8]{ctexbook}

% 设置纸张信息。
\usepackage[a4paper,twoside]{geometry}
\geometry{
	left=25mm,
	right=25mm,
	bottom=25.4mm,
	bindingoffset=10mm
}

% 设置字体,并解决显示难检字问题。
\xeCJKsetup{AutoFallBack=true}
\setCJKmainfont{SimSun}[BoldFont=SimHei, ItalicFont=KaiTi, FallBack=SimSun-ExtB]

% 目录 chapter 级别加点(.)。
\usepackage{titletoc}
\titlecontents{chapter}[0pt]{\vspace{3mm}\bf\addvspace{2pt}\filright}{\contentspush{\thecontentslabel\hspace{0.8em}}}{}{\titlerule*[8pt]{.}\contentspage}

% 设置 part 和 chapter 标题格式。
\ctexset{
	chapter/name={},
	chapter/number={}
}

% 设置古文原文格式。
\newenvironment{yuanwen}{\bfseries\zihao{4}}

% 设置署名格式。
\newenvironment{shuming}{\hfill\bfseries\zihao{4}}

% 注脚每页重新编号,避免编号过大。
\usepackage[perpage]{footmisc}

\title{\heiti\zihao{0} 朱子家训}
\author{朱用纯}
\date{清}

\begin{document}

\maketitle
\tableofcontents

\frontmatter
\chapter{前言、序言}

\mainmatter

% 增加空行
~\\

% 增加字间间隔,适用于三字经、诗文等。
 \qquad  

\chapter{朱子家训}

\begin{yuanwen}
黎明即起,洒扫庭除\footnote{庭院。},要内外整洁;既昏便息,关锁门户,必亲自检点。
\end{yuanwen}

每天早晨黎明就要起床,先用水来洒湿庭堂内外的地面然后扫地,使庭堂内外整洁;到了黄昏便要休息并亲自查看一下要关锁的门户。

\begin{yuanwen}
一粥一饭,当思来处不易;半丝半缕,恒念物力维艰。
\end{yuanwen}

对于一顿粥或一顿饭,我们应当想着来之不易;对于衣服的半根丝或半条线,我们也要常念着这些物资的产生是很艰难的。

\begin{yuanwen}
宜未雨而绸缪,毋临渴而掘井。
\end{yuanwen}

凡事先要准备,像没到下雨的时候,要先把房子修补完善,不要“临时抱佛脚”,像到了口渴的时候,才来掘井。

\begin{yuanwen}
自奉必须俭约,宴客切勿流连。
\end{yuanwen}

自己生活上必须节约,聚会在一起吃饭切勿流连忘返。

\begin{yuanwen}
器具质而洁,瓦缶\footnote{瓦制的器具。缶,fǒu}胜金玉;饮食约而精,园蔬愈珍馐\footnote{珍奇精美的食品。馐,xiū}。
\end{yuanwen}

餐具质朴而干净,虽是用泥土做的瓦器,也比金玉制的好;食品节约而精美,虽是园里种的蔬菜,也胜于山珍海味。

\begin{yuanwen}
勿营华屋,勿谋良田。
\end{yuanwen}

不要营造华丽的房屋,不要图买良好的田园。

\begin{yuanwen}
三姑六婆,实淫盗之媒;婢美妾娇,非闺房之福。
\end{yuanwen}

社会上不正派的女人,都是淫和盗窃的媒介;美丽的婢女和娇艳的姬妾,不是家庭的幸福。

\begin{yuanwen}
(童仆/奴仆)勿用俊美,妻妾切忌艳妆。
\end{yuanwen}

家僮、奴仆,不可雇用英俊美貌的,妻、妾切不可有艳丽的妆饰。

\begin{yuanwen}
祖宗虽远,祭祀不可不诚;子孙虽愚,经书不可不读。
\end{yuanwen}

祖宗虽然离我们年代久远了,祭祀却仍要虔诚;子孙即使愚笨,教育也是不容怠慢的。

\begin{yuanwen}
居身务期(俭/质1)朴,教子要有义方\footnote{做人的正道。}。
\end{yuanwen}

自己生活节俭,以做人的正道来教育子孙。

\begin{yuanwen}
(莫/勿1)贪意外之财,(莫/勿1)饮过量之酒。
\end{yuanwen}

不要贪不属于你的财,不要喝过量的酒。

\begin{yuanwen}
与肩挑贸易,毋占便宜;见(穷/贫)苦亲邻,须加温恤。
\end{yuanwen}

和做小生意的挑贩们交易,不要占他们的便宜,看到穷苦的亲戚或邻居,要关心他们,并且要给他们有金钱或其它的援助。

\begin{yuanwen}
刻薄成家,理无久享;伦常乖舛\footnote{违背。舛,chuǎn},立见消亡。
\end{yuanwen}

对人刻薄而发家的,绝没有长久享受的道理。行事违背伦常的人,很快就会消灭。

\begin{yuanwen}
兄弟叔侄,(须1/需)分多润寡;
长幼内外,宜法肃辞严。
\end{yuanwen}

兄弟叔侄之间要互相帮助,富有的要资助贫穷的;一个家庭要有严正的规矩,长辈对晚辈言辞应庄重。

\begin{yuanwen}
听妇言,乖骨肉,岂是丈夫?重资财,薄父母,不成人子。
\end{yuanwen}

听信妇人挑拨,而伤了骨肉之情,那里配做一个大丈夫呢?看重钱财,而薄待父母,不是为人子女的道理。

\begin{yuanwen}
嫁女择佳婿,毋索重聘;娶媳求淑女,勿计厚奁\footnote{丰厚的嫁妆。奁,lián}。
\end{yuanwen}

嫁女儿,要为她选择贤良的夫婿,不要索取贵重的聘礼;娶媳妇,须求贤淑的女子,不要贪图丰厚的嫁妆。

\begin{yuanwen}
见富贵而生谄容者,最可耻;遇贫穷而作骄态者,贱莫甚。
\end{yuanwen}

看到富贵的人,便做出巴结讨好的样子,是最可耻的,遇着贫穷的人,便作出骄傲的态度,是鄙贱不过的。

\begin{yuanwen}
居家戒争讼,讼则终凶;处世戒多言,言多必失。
\end{yuanwen}

居家过日子,禁止争斗诉讼,一旦争斗诉讼,无论胜败,结果都不吉祥。处世不可多说话,言多必失。

【评说】 争斗诉讼,总要伤财耗时,甚至破家荡产,即使赢了,也得不偿失。有了矛盾应尽量采取调解或和解的方法。

\begin{yuanwen}
勿恃势力而凌逼孤寡,毋贪口腹而恣杀生禽。
\end{yuanwen}

不可用势力来欺凌压迫孤儿寡妇,不要贪口腹之欲而任意地宰杀牛羊鸡鸭等动物。

\begin{yuanwen}
乖僻自是,悔误必多;颓惰自甘,家道难成。
\end{yuanwen}

性格古怪,自以为是的人,必会因常常做错事而懊悔;颓废懒惰,沉溺不悟,是难成家立业的。

\begin{yuanwen}
狎昵\footnote{过分亲近。xiá nì}恶少,久必受其累;屈志老成,急则可相依。
\end{yuanwen}

亲近不良的少年,日子久了,必然会受牵累;恭敬自谦,虚心地与那些阅历多而善于处事的人交往,遇到急难的时候,就可以受到他的指导或帮助。

\begin{yuanwen}
轻听发言,安知非人之谮诉\footnote{诬蔑人的坏话。谮,zèn},当忍耐三思;因事相争,焉知非我之不是,需平心暗想。
\end{yuanwen}

他人来说长道短,不可轻信,要再三思考。因为怎知道他不是来说人坏话呢?因事相争,要冷静反省自己,因为怎知道不是我的过错?

\begin{yuanwen}
施惠(无/勿)念,受恩莫忘。
\end{yuanwen}

对人施了恩惠,不要记在心里,受了他人的恩惠,一定要常记在心。

【评说】常记他人之恩,以感恩之心看待周围的人及所处的环境,则人间即是天堂。以忘恩负义之心看待周围的人事,则人间即是地狱。

\begin{yuanwen}
凡事当留馀地,得意不宜再往。
\end{yuanwen}

无论做什么事,当留有余地;得意以后,就要知足,不应该再进一步。

\begin{yuanwen}
人有喜庆,不可生嫉妒心;人有祸患,不可生喜幸心。
\end{yuanwen}

他人有了喜庆的事情,不可有妒忌之心;他人有了祸患,不可有幸灾乐祸之心。

\begin{yuanwen}
善欲人见,不是真善;恶恐人知,便是大恶。
\end{yuanwen}

做了好事,而想他人看见,就不是真正的善人。做了坏事,而怕他人知道,就是真的恶人。

\begin{yuanwen}
见色而起淫心,报在妻女;匿怨\footnote{对人怀恨在心,而面上不表现出来。}而用暗箭,祸延子孙。
\end{yuanwen}

看到美貌的女性而起邪心的,将来报应,会在自己的妻子儿女身上;怀怨在心而暗中伤害人的,将会替自己的子孙留下祸根。

\begin{yuanwen}
家门和顺,虽饔飧\footnote{饔,yōng,早饭。飧,sūn,晚饭。}不继,亦有馀欢;国课\footnote{国家的赋税。}早完,即囊橐\footnote{口袋。橐,tuó}无馀,自得至乐。
\end{yuanwen}

家里和气平安,虽缺衣少食,也觉得快乐;尽快缴完赋税,即使口袋所剩无余也自得其乐。

\begin{yuanwen}
读书志在圣贤,非徒科第;为官心存君国,岂计身家。
\end{yuanwen}

读圣贤书,目的在学圣贤的行为,不只为了科举及第;做一个官吏,要有忠君爱国的思想,怎么可以考虑自己和家人的享受?

\begin{yuanwen}
守分安命,顺时听天。
\end{yuanwen}

我们守住本分,努力工作生活,上天自有安排。

\begin{yuanwen}
为人若此,庶乎近焉。
\end{yuanwen}

如果能够这样做人,那就差不多和圣贤做人的道理相合了。

\backmatter

\end{document}