% 九尾龟
% 九尾龟.tex

\documentclass[12pt,UTF8]{ctexbook}

% 设置纸张信息。
\usepackage[a4paper,twoside]{geometry}
\geometry{
	left=25mm,
	right=25mm,
	bottom=25.4mm,
	bindingoffset=10mm
}

% 设置字体,并解决显示难检字问题。
\xeCJKsetup{AutoFallBack=true}
\setCJKmainfont{SimSun}[BoldFont=SimHei, ItalicFont=KaiTi, FallBack=SimSun-ExtB]

% 目录 chapter 级别加点(.)。
\usepackage{titletoc}
\titlecontents{chapter}[0pt]{\vspace{3mm}\bf\addvspace{2pt}\filright}{\contentspush{\thecontentslabel\hspace{0.8em}}}{}{\titlerule*[8pt]{.}\contentspage}

% 设置 chapter 标题格式(古代小说,标题分两行)。
\usepackage{varwidth}
\ctexset{
	chapter/name={第,回},
	chapter/titleformat= \chaptertitleformat
}
\newcommand\chaptertitleformat[1]{
	\begin{varwidth}
		[t]{.7\linewidth}#1
	\end{varwidth}
}

% 设置古文原文格式。
\newenvironment{yuanwen}{\bfseries\zihao{4}}

% 设置署名格式。
\newenvironment{shuming}{\hfill\bfseries\zihao{4}}

% 注脚每页重新编号,避免编号过大。
\usepackage[perpage]{footmisc}

\title{\heiti\zihao{0} 九尾龟}
\author{}
\date{}

\begin{document}

\maketitle
\tableofcontents

\frontmatter
\chapter{前言、序言}

\mainmatter

\chapter[谈楔子演说九尾龟\ 访名花调查青阳地]{谈楔子演说九尾龟\\访名花调查青阳地}

龟有三足,亦有九尾。《尔雅》注云:“南方之龟有九尾,见之者得富贵。”古来麟、凤、龟、龙,列在四灵之内,那乌龟是何等宝贵的东西。降至如今,世风不古,竟把乌龟做了极卑鄙龌龊的混名:妇女或有外遇,群称其夫为“乌龟”。这是个什么讲究呢?大抵也有一个来历,诸公静听,待鄙人慢慢的说来。

从前管仲设女闾三百,以为兵士休宿之所,这便是妓女的滥觞。唐时官妓多隶教坊,设教坊司以管领女乐。那教坊中的人役,皆头裹绿巾,取其象形有似乌龟。列公试想:那乌龟一头两眼,不多是碧绿的么?还有取义的一说,是龟不能交,那雌龟善与蛇交,雄不能禁,因此大凡妇女不端,其夫便有乌龟之号。在下这部小说名叫“九尾龟”,是近来一个富贵达官的小影。这贵官帷薄不修,闹出许多笑话,倒便宜在下,编成了这一部《九尾龟》。

闲话少提,书归正传。且先将一个风流才子类弄登场,好为诸公解秽。正是:

\begin{quotation}
莫把酒杯浇块垒,且将绮梦说莺花。
\end{quotation}

且说这名士姓章,单名一个莹字,别号秋谷,江南应天府人氏,寄居苏州常熟县。生得白皙丰颐,长身玉立。论他的才调,便是胸罗星斗,倚马万言;论他的胸襟,便是海阔天空,山高月朗;论他的意气,便是蛟龙得雨,鹰隼盘空。这章秋谷有如此的才华意气,却又谈词爽朗,举止从容,真个是美玉良金,隋珠和璧,一望而知他日必为大器的了。

只是秋谷时运不济,十分偃蹇,十七岁便丁了外艰,三年服阕,便娶了亲。他夫人张氏,身材不长不短,面孔不瘦不肥,虽不是绝世佳人,恰也不十分丑怪,但是性情古执,风趣全无。若在别人,原也不至夫妻反目,无奈秋谷倚着自家万斛清才,一身侠骨,准备着要娶一个才貌双全的绝代名姝,方不辜负他自家才调,娶了这等一个平庸女子,叫他如何不气?气到无可如何之际,便动了个寻花问柳的念头,就借着他事,告禀了太夫人,定了行期,收拾行李,便登舟往苏州进发。

不一日到了苏州,在盘门外一个客栈名叫“佛照楼”的住下。那苏州自从日本通商以来,在盘门城外开了几条马路,设了两家纱厂,那城内仓桥滨的书寓,统通搬到城外来,大菜馆、戏馆、书场,处处俱有,一样的车水马龙,十分热闹。

秋谷落栈之后,歇息了一日,不免往书场、戏馆去涉猎涉猎。坐了几天马车,吃了两回大菜,觉得苏州马路的风景不过如此。与上海大不相同,虽然灯火繁华,却时时露出荒凉景象。日间欢场征逐,自有那一班朋友声应气求,到也并不寂寞,只是到了酒阑人散之时,客舍独居,孤灯相对,你道这样风流人物,怎生消受得来?

一日夜饭后并无应酬,信步出栈望马路走来。见那来往兜圈子的马车上坐的那些倌人,真是杨柳为眉,芙蓉如面。同着客人坐在一车的,更是佯嗔娇笑,慎态动人。只苦的自己初到苏州,并无熟识,只得走到一家书场名叫“余香阁”的,走了进去,拣张桌子泡茶坐下,细细的打量台上倌人。只见左首第三座上坐着一个倌人。年纪约十六七岁,珠光侧聚,珮响流葩,眉锁春山,目澄秋水,那粉颊上晕着两个酒涡,似笑非笑的低头敛手,坐在那里弄衣角儿。秋谷一眼看见,吃了一惊,那双眼睛就如被他勾了去的一般,登时神魂不定起来,便呆呆的看着他。

一会儿,那堂倌在傍凑趣,低低的问秋谷道:“这倌人名叫许宝琴,名气狠大,今年尚止十六岁,唱得好一口京调。老爷可要点他两出?”秋谷不答,只微微的点一点头。堂倌便如飞去取了粉牌过来,并拿一枝笔递给秋谷。秋谷提起笔来,写了两出《朱砂痣》、《琼林宴》的京戏,《卖花球》、《白兰花》的两支小调,顿时喊上台去。原来苏州规矩与上海不同,点戏是当台招呼的。

那倌人听有客人点戏,抬起头来,飘了秋谷一眼,又微笑一笑,只觉媚眼横波、红潮上颊,越显得光容绰约、丰彩飞扬,喜得秋谷色舞眉飞,十分得意。又见一个年轻大姐,手拿着银水烟袋,下来装烟,便问秋谷尊姓,随即应酬了几句,秋谷一一的回答了。

此时许宝琴抱着琵琶,弹了一套开片,背脸儿亢起娇声来,虽不是裂石穿云,却也引商刻羽。唱过一段《朱砂痣》,便把琵琶捺低一调,低低的唱那小调《白兰花》。唱到关情之处,星眸低漾,杏脸微红,把眼波只顾向秋谷溜来,台下看客齐声喝采,到把秋谷弄得不好意思起来。

一会宝琴唱完,对那大姐使一个眼色,那大姐便又下来装了几筒烟,说声:“对勿住,停歇请过来!”便扶着宝琴姗姗而去;临行之际,又向秋谷一笑,方才下楼去了。秋谷急叫堂倌算好了帐,立起身来跟下扶梯,许宝琴还未上轿。立在门口,见秋谷匆匆的下来,含笑招呼道:“章大少,啥勿一淘到倪搭去嗄!”秋谷答应道:“我正要去坐坐,你叫大姐同我去罢。”宝琴便叫那大姐道:“阿仙,格末倪先转去哉,耐同仔章大少要就来格虐。”阿仙答应一声,宝琴便上轿走了。

秋谷同着阿仙一路问答,慢慢的走过了甘棠桥。秋谷早看见了许宝琴的牌子,便进门登楼,相帮叫了一声:“客人上来!”宝琴早换了衣服,接到扶梯边,秋谷携了宝琴的手,同进房来。抬头一看,房间虽然不大,收拾得十分富丽。

秋谷便在炕上坐下。宝琴敬过瓜子,细细的打量秋谷。正是二月初天气,见他穿着一件白灰色灰鼠皮袍,玄色外国缎草上霜一宇襟坎肩,外罩天青贡缎洋灰鼠马褂,颜色配搭得十分匀衬。长眉凤目。白面丰颐,英爽之气,奕奕逼人,觉得眼中从未见过这样人物,不觉亲热起来,挨着秋谷身旁坐下,应酬了一回。秋谷看他言语之间尚觉有些羞涩,便知初入青楼,不是那林黛玉、翁梅倩一流人物;又见他低颦浅笑,顾盼生怜,不由心花大放,便向宝琴说道:“我今日虽然还是第一次来,竟要在这里请几个客,不知房间可空不空?”宝琴笑道:“只要大少肯照应倪,是再好勿有格事体,倪阿有啥倒勿肯格?”便回头叫房间里娘姨,交代一台菜下去。

秋谷叫拿笔砚过来,写好请客票,发去不多一刻,客人陆续到来。发过局票,秋谷叫起手巾,其时台面已经摆好,大家入座。其中恰有一位客人,是秋谷最敬重的朋友,双姓东方,单名一个瑶字,又号小松。生得仪容俊雅,眉目风流,素有璧人之目,同秋谷意气相投,时常会面的。当下到了席中,一眼先看见了许宝琴,山花宝髻,石竹罗衣,神彩惊鸿,珮环回雪,不觉呆了一呆;又见秋谷与他非常亲热,眉语目成,又如飞燕依人,夭桃初放,便大笑道:“秋谷说苏州地方并无相好,这位贵相知难道是天外飞来的不成?快快实说:是几时做起,为何瞒着我们,是何道理?”秋谷尚未开口,宝琴早已两颊通红,扭转身子,恰好与小松打个照面,更加不好意思,低下头去,口中咕噜道:“耐笃总是实梗瞎三话四,阿要无淘成,倪是要板面孔格。”秋谷听了好笑,便道:“这位方大少,天生的不老成,没有好话说的,你只当他放屁就是了。”又向小松道:“我向来作事从未瞒你,此处我实是今日第一回来,在余香阁点戏之后,钉梢回来的。你不信,只顾问房间里人便了。”那房间里娘姨阿彩、大姐阿仙,一齐说道:“方大少,勿要勿相信,轧实章大少是今朝做起格勒,倪阿肯骗耐嗄。”

小松听了,方才相信,想了一想,又摇摇头道:“我只不信。既然是今天做起,为甚你们先生的神气,倒像与章大少是老相好一样,是何道理?”小松说到此际,早被秋谷捏了一把,使个眼色,小松方才住口。秋谷悄悄埋怨他道:“你取笑也要看地方起的。我今天初次在此请客,你便如此胡言乱语,倘被他真个板起面孔来,你我岂不大家没趣?”小松笑道:“你不要来吓我,我是不怕的,你只好好的叫他转个局,我便不开口了,你肯不肯?”秋谷不觉大笑道:“原来你说了半天,是要割我的靴腰,何不早说,恰要绕着弯儿说呢?”便叫宝琴转过去坐在小松旁边。宝琴抬起头来,着实钉了秋谷一眼,也不言语。秋谷又催一遍,宝琴方才对着小松说道:“方大少,对勿住,倪间搭格规矩:一帮里客人勿做两个格。阿好谢谢耐,勿要扳倪格差头。倪情愿吃子一杯罚酒末哉。”说罢,便叫阿仙取出一只鸡缸杯来,斟了一杯热酒,立起身来,将杯照着小松,竟自吃干了。“小松倒也无可再言。停了一会,忽然笑道:”可恶可恶,我在堂子里头顽儿,总弄你这促掐鬼不过,你总要占个上风,究竟我同你是一样的人,难道我短了什么不成?“说着,又问宝琴道:”你看我们两人,倒底谁的风头好些?“宝琴听小松说得好笑,不免面红一笑,暗中又飞了秋谷一眼,早被对坐的客人名叫孔伯虚的看见,便笑道:”据我看来,秋翁与小翁二人正是工力愁敌,可算得瑜亮并生,一时无两。只是宝琴的意思有些看不上小翁,或是小翁的内才短些,比不上秋翁的精力,那我们外人就无从晓得了。“说得合席大笑起来。恰好各人的局陆续到了,彼此打断了话头。

酒过数巡,小松鼓起兴来,便要摆五十杯的庄。秋谷微笑道:“你这种的酒量也敢摆庄?待我来打坍你的。”于是攘臂而起,正与小松旗鼓相当。旁坐一个姓吴的劝道:“五十杯太多,留几杯等别人来打,你打了二十杯罢!”秋谷依了,便与小松五魁三元的叫了一阵。二十杯庄打完,秋谷自己也输了十五六杯,秋谷慢慢的喝了十杯,还有五杯,便折在一个大玻璃缸里,回过身来递与阿彩,叫他代饮。阿彩刚刚接过,早被宝琴劈手夺来,一口气咕嘟嘟的竟喝了一个干净,面上早红晕起来,放下杯子,那两只秋波水汪汪的更加了几分风韵。小松只顾与别人搳拳,竟不理会。秋谷却是留心的,见他杏眼微饧,桃腮带涩,心上觉得好生怜惜,只是说不出来,便低低的合他说道:“你何苦这样拼命的喝酒,喝醉了便怎样呢?”宝琴微笑不答,秋谷更是魂销。两人相视了好一会,小松的庄早已打完。小松除代酒外,自家也喝了三十余杯,觉得有些沉醉,从腰间掏出一个表来一看,早已指到十二点三刻了,便道:“时候不早了,我们散罢!好等你们两人细细的谈心。”上过干稀饭,各人都掏出两块洋钱放在桌上。秋谷也取出下脚四元,添菜两元,一齐放在台上。相帮进来收拾台面,把洋钱数了一数,七个客人共是十四块,一总二十块洋钱,便高叫一声:“多谢各位大少。”拿了洋钱出房去了。

看官且慢,你道此是什么规矩?原来姑苏书寓规条,大凡请客,须每位客人出台面洋两元,谓之“丢台面。”朋友请吃花酒,若非素日知己,不肯到场。因非但赔贴局钱,又要现丢台面,绝非上海请吃花酒,客人到了就算赏光的风俗。再加上海碰和一概二十元,苏州却无论长三幺二均是八元。以前上海青楼风俗,凡生客进门,倌人必唱京调或小曲一支,名为“堂唱”,恰须现钱开销。现在上海此例已除,姑苏却至今未改,这是苏、沪不同之处,在下预先一一申明,免得要受看官的指摘。

只说客人散后,只有秋谷未曾回去,就在那里借了一夜干铺。名说干铺,只怕明干暗湿也未可知,不在话下。

秋谷睡至晌午,方才起来,洗漱已毕,待要回栈,宝琴叫相帮到正元馆端了一碗一钱六分生炒鸡丝面来,让秋谷吃了;又亲自替秋谷梳了一条辫子,方才放他下楼,又叮嘱他晚上要来。秋谷一一答应了,自回栈去,仍就睡了。约至三下钟,方睡醒起来,随意吃些东西。正待出去,只见许宝琴家的阿仙笑嘻嘻的走进来,道:“章大少,阿是刚刚起来勒?倪先生到书场浪去哉,请耐去点戏。”秋谷也无可不可的,同了阿仙走到余香阁。

正待上楼,只见一顶倌人轿子停在门前,眼前觉得毫光一闪,走出一个倌人来,穿一件黑地银花外国缎灰鼠皮祆,下衬品蓝花缎裤子,玄色缎子弓鞋不到四寸,眉眼虽比许宝琴略逊,那一种的丰姿袅娜,骨格轻盈,却比许宝琴更加妩媚。秋谷立在扶梯边,一直等到他上了楼,目光尚有些定定的,被阿仙从后推了一把,道:“阿是看得头里向有点浑淘淘哉,快点上去哩!”秋谷被他一推,吓了一跳,不觉自己好笑,便走上扶梯,拣一个座位。刚刚坐下,堂倌早送了点戏牌过来,秋谷且不点戏,问着堂倌,那外国缎袄的叫甚名字。堂倌道:“他住在谈瀛里,名叫花云香,还是新近从上海来的,章老爷可要也点他两出?”秋谷要过笔来,便写了《二进宫》、《龙虎斗》、《探寒窑》、《铡美案》四出,都要花云香与许宝琴两人合唱。

堂倌喊了上去,花云香听得分明,回头一看,就是楼梯边的相遇人,不免低头一笑,随叫娘姨下来装烟。许宝琴却着实的钉了秋谷一眼。秋谷虽也看见,并不理会。花云香先了和弦,唱出一段《二进宫》,许宝琴随接唱下去,唱到末尾一句,两人一齐背过脸去,把琵琶放高一调,全用轮指合唱。那一声摇板却唱得顿挫抑扬,十分圆稳,秋谷喝一声采。随后又合唱了一出《铡美案》,许宝琴便先起身走了。只有花云香又独唱一出《探寒窑》,那喉咙愈唱愈高,愈高愈亮,唱到极高之后,一落千丈,就如银瓶落井一般,落到一半却又陡然提起,又如鹤唳入云,声声摇曳,真是珠喉遏月,逸响回风,只听得台下喝采之声轰然不绝。秋谷异常得意。花云香唱完之后,方才立起身来,正走秋谷面前经过,向秋谷点一点头,下楼去了。

秋谷见他走了,无精打采的付了帐,慢慢的下来。才到楼下,不防阿仙候在门口,便一把衣袖拉了秋谷,一直拉到甘棠桥,进门推他上楼。只见宝琴欲笑不笑,一付尴尬面孔,道:“章大少,耐倒有功夫到倪搭来坐坐,啥勿到花云香搭去嗄!”秋谷听了笑道:“你们这班人实在难说话得狠。叫了我来,又叫我到别处去,我就依着你的吩咐,到花家去。”说着,假做回身要走,早被阿仙一把拉住,说道:“耐阿要好意思格!花家里明朝去末哉,倪搭小场化,委屈耐点阿好?”宝琴接口说道:“耐放俚去嗫,看俚阿好意思走出去。”秋谷呵呵笑道:“你们不要我去,也就罢了,何必做出许多生意筋络来。”一面说,一面坐下。

宝琴问道:“阿要吃夜饭哉,就倪搭便饭,去叫仔两样菜阿好?”秋谷正待写菜去叫,只听楼下喊声“请客”。把请客条子递将上来一看,原来是小松请到如意里金黛玉家,上面写着:“容齐坐候入席”,秋谷便立起身来。阿仙便说道:“章大少,阿要带局去罢,省得来叫哉。”秋谷点头道:“也好。”因如意里与许家只隔一桥,便不用轿子,催许宝琴换好了出局衣裳,二人携手出门。

到了金黛玉家,问了房间,恰在楼下。小松早在房门口招呼,进房坐下,满房客人都与秋谷相识,不用套谈。小松见秋谷同着宝琴,便道:“你带局来,倒也简便,可还叫别人么?”秋谷因叫小松代写了一张花云香的局票,一同发去。

少时,大家入席,花云香早姗姗其来,进房含笑叫了一声,便坐在秋谷身后。秋谷不及应酬,便留心打量金黛玉的妆束,只见他:淡扫蛾眉,薄施脂粉,穿一件蜜色皮袄,衬一条妃色裤子。风鬟雾鬓,虽非倾国之姿;素口蛮腰,稳称芳菲之选。那边小松见了花云香,也打量了一会,忽嚷道:“不好了,又被你抢了一个去了!怎么我到处留心,总没有好的;你遇见的,总是好的呢?”秋谷道:“你为什么总是这样脾气?今天是你自己的主人,劝你少说两句罢!”说着,金黛玉起身斟了一巡酒,众客人的局也来了。花云香先唱了一出《取成都》,唱完了,对秋谷说声“献丑”,秋谷说声“辛苦”,便慢慢的谈起来。两人咬着耳朵不知讲些什么。许宝琴却看着冷笑。偶而秋谷回过身来同宝琴说话,宝琴却只是扭过身去,不肯理他。

秋谷正在没做理会处,小松斟了一大杯酒要与秋谷照杯,又笑道:“知己希逢,佳人难得,你快干了这一杯。”秋谷猛然听得,触起他的心事来,长叹一声,举杯一饮而尽,口中高吟道:“此时此景不沉醉,岂待三尺蓬蒿坟。”与小松彼此相对黯然。停了一回,小松方勉强笑道:“我们原是寻乐的,怎么倒寻起烦恼来呢?我与你还是喝酒罢。”秋谷也不回言,自己斟了一杯,又高吟道:“今日少年若长在。古之少年安在哉?”就又干了一杯。

花云香看见秋谷无故不乐,心中觉得十分难过,却又替他不得,便咬着秋谷耳朵道:“耐勿要煞死个吃酒哉,到倪搭去坐歇罢。耐坐仔我个轿子去阿好?”秋谷只点点头。花云香便叫自己的轿子来,亲手将秋谷扶在轿内,自己也立起身来,跟着走出,叫一部东洋车,傍着轿子同走。秋谷也不顾许宝琴,竟自到花家去了,连主人方小松都未招呼。正是:

名士风尘多涕泪,美人香草寄牢骚。

要知后事如何,且听下回分解。

\chapter[真抑塞粉墨登场\ 假从良姑苏遇旧]{真抑塞粉墨登场\\假从良姑苏遇旧}

只说方小松见秋谷不辞而别,也晓得他别有伤心,无不劝解,当下草草终席,小松便进城去了。秋谷自从坐着花云香的轿子,同到花家之后,便常在许、花二家走动,许宝琴虽只心中不悦,也无可如何。

开筵坐花,飞觞醉月,不觉已是一月有余。一日夜间,秋谷在花家吃过夜膳,想到二马路丹桂去看戏,便同着云香走出谈瀛里。那丹桂就在谈瀛里对门,不用轿子。走到戏园门口,案目认得秋谷,慌忙同了进去。苏州戏园没有厢楼,就在正桌坐下。

那时台上正在演那《翠屏山》,周凤林扮着潘巧云,虽然年纪大些,台容倒还不错。筱荣祥扮的杨雄,陈云仙扮的石秀,却也工力悉敌。末后陈云仙一路单刀,身眼手步,一丝不走,舞到妙处,就如一片电光,满身飞舞。秋谷见了高兴起来,忽然发一个奇想:自己想要粉墨登场,出一出胸中的郁勃之气。原来秋谷自幼投师习武,拳棒极精,等闲一二十人近他不得。打定主意,叫了案目过来,叫出开丹桂的老板郝尔铭走到座前。秋谷向来认得,便同他商议,要点一出《鸳鸯楼》,叫陈云仙扮武松,到那舞刀的一场,让秋谷自己登台试演,一场舞过,仍叫陈云仙上场。

郝尔铭听了也觉诧异,踌躇一会,方才答应道:“照例是没有这个规矩,不过既是章老爷高兴,云仙又是我的徒弟,不比外来的武生,不妨迁就。”秋谷大喜,便取出两张十元的钞票交给他说:“这就算点戏的钱,我既硬出了这个新鲜主意,自然要多出些钱。”郝尔铭随意谢了一声收下,便走了进去,早见挂出一面点戏牌来。

随后《翠屏山》唱完,便是《鸳鸯楼》出场,陈云仙仍扮武松,那脱靠的一场解数,筋斗跌扑,十分伶俐。此时秋谷早已走进戏房,打扮去了,花云香拦阻不住。

少时,陈云仙下去,只听得锣声一响,那板鼓的声音,打得犹如飘风疾雨一般,值场的掀开软帘,秋谷执刀在手,迅步登场。花云香见了,呆了一呆,觉得另换了一副英武的精神,绝非秋谷平时缓带轻裘的态度。只见他头紥玄缎包巾,上挽英雄结,身穿玄缎密扣紧身,四周用湖色缎镶嵌着灵芝如意,胸前白绒绳绕着双飞蝴蝶,腰紥月蓝带子约有四寸半阔,上钉着许多水钻,光华夺目,两边倒垂双扣,中间垂着湖色回须,下着黑绉纱兜裆叉裤,脚登玄缎挖嵌快靴,衬着这身装束,越显得狼腰猿臂,鹤势螂形。再加头上用一幅黑纱巾当头紧紥,紥得眼角眉梢高高吊起,那一派的英风锐气,直可辟易千人。加以秋谷出身贵介,天然台步从容,拳棒精通,自尔功夫圆稳。此时台上台下,眼睁睁的都看着秋谷一人。

秋谷左手擎刀,用一个怀中抱月的架式,右手向上一横,亮开门户,霍地把身子一蹲,“拍”的一声,起了一个飞腿,收回右腿,缴转左腿,旋过身来,就势用个金鸡独立,右手接过刀来,慢慢的舞起。初时还松,后来渐紧,起初还见人影,后来只见刀光,那一把刀护着全身,丝毫不漏,只看见一团白光在台上滚来滚去,却没有一些脚步声音。说时迟,那时快,猛然见刀光一散,使一个燕子街泥,这一个筋斗,直从戏台东边直扑到台角,约有八九尺,那手中的刀便在自己脚下反折过来,“呼”的一声,收了刀法,现出全身,面上不红,心头不跳,仍用怀中抱月,收住了刀。正待进去,忽听得喝采声中,有一个妇女的声音十分清脆,高叫一声:“好呀!”

秋谷诧异起来,回头一看,只见二排上坐着一个二十岁上下的女子,衣装娇艳,态度妖娆,面目有些相熟,好像那里见过的一样,一双莹莹的眼波,只注在秋谷身上。照例武松舞刀一场,便要进去,此时秋谷见他看得认真,故意卖弄精神。好个章秋谷,另使出一番解数,把腰刀插在背后,空手开了一个四门,忽然左右开弓,连扑两交筋斗。翻过身来,脚跟尚未着地,那一把明晃晃的刀早掣在手中。这路刀法,与前更是不同,风声飒飒,冷气飕飕,刀光映着灯光,异常精采。这一路刀舞有半刻余钟,方才收住。进场换了衣服,下得台来,并不见一些儿杀气威风,依然是一个风流才子,台上仍换了陈云仙上场接演。

那知这一路刀,虽然不打紧,却引出一个人的故事来,就是那喝采的女子。你道是谁?就是三年前盛名之下的大金月兰。

这金月兰自从十七岁梳栊之后,不到一年,便有一个杭州黄大军机的长孙公子名叫黄伯润的,看中了他,花了八千银子的身价将他娶去,做了一位现现成成的姨太太。这位黄公子年方二十,正妻亡过,尚未续弦,性情极是温和,眉目也还清秀。

家财巨万,门第清华。至于服食起居,更是一呼百诺,要一奉十。论起来,这金月兰也该自家知足,跟他过了一生,倘或生得一男半女,怕不是一位诰命夫人?岂非天外飞来的一段福分?

无奈上海这些做倌人的,骨相天生,万不能再做良家妇女。这班倌人,马夫、戏子是姘惯了,身体是散淡惯了,性情是放荡惯了,坐马车,游张园,吃大菜,看夜戏,天天如此,也觉得视为固然,行所无事。你叫他从良之后,怎生拘束得来?

再如良家妇女,看得“失节”二字是一件极重大的事情;倌人出身的,只当作家常便饭一样,并不是什么奇事。就是那一班情愿从良的妓女,偶然见了一个俊俏后生,便由不得背地里私通款曲,这不过如家常便饭之外,偏背了一顿点心,算不是毁名败节,却轻轻的把一顶绿头巾暗暗送与主人公戴在头上。这还算是好的,更有那一种倌人,自己或是讨人,不能作主,或是欠了债项,不得自由,便拣一个有钱的客人,预先灌了无数迷汤,发下千斤重誓,一定要嫁那客人,身价不是三千,就是五千。这班寿头码子的客人却也奇怪:平时亲戚通融,友朋借贷,就立刻翻转面皮,倒反说穷告苦,非但一毛不拔,而且还要从此断绝往来;独到了遇着这种倌人,却情情愿愿,伏伏贴贴的,捧着大把的银子去孝敬他,还不敢说一个“不”字,好似儿子见了父母一样。这班人具着卑鄙龌龊的面目,怀着势利狭窄的心肠,那面目比纯钢炼就的还厚,那心肠比煤炭烧枯的还焦。目不识丁,偏会看不起读书种子;骨头鄙贱,偏要摆着那富贵的规模。真个是“投畀豺虎,豺虎不食;投畀有北,有北不受”的东西。他自己丧尽良心,所以就有丧尽良心的倌人来收拾他。归根花了一注大钱,不上一年半载,得个方便,卷了值钱的衣饰,远走高飞。那时非但人财两空,连他自家的血本都丢在东洋大海去了。这便叫“倌人淴浴”。借了他人的财力,自己拔出火坑;及至出了火坑,却又负义忘恩,全不顾人情天理。终究报应循环,丝毫不爽。自家拐骗的邪财,迟早原被那戏子、马夫一齐骗去。如此得来如此去,依旧是一双空手,蓄积毫无,到了年长色衰,门前冷落,这便追悔也追悔不来了。

看官,你道上海的倌人可以娶得的么?

闲话少提,书归正传。只说金月兰嫁了黄公子之后,同到杭州,不上几时,便觉得十分拘束,渐渐的不惯起来,就撺掇黄公子,要赁房子住在上海。黄公子道:“你的意思无非拘束不惯,要去住在上海,好游园听戏,散散心情。但是上海地方不是可以长住得的,况且你更不比从前,做了良家妇女,就要诸事小心,就是住在上海,也不能时常出去。你既然嫁了我,便是我家的人,却要依着我家的规矩。别样事情我总可答应,这件事情是答应不来的,劝你不必起这念头罢。”

金月兰听了十分不悦,敢怒而不敢言,心中便有重落风尘之意。存了这条心念,便时时刻刻打算私逃。苦的是侯门如海,无计可施。好容易想着一个主意:那黄府的后进一带房屋,都是楼房,最后一进的后楼就靠着城河,城河内的船都停在黄府楼下,说话都听得见的。月兰便对公子说了,要搬到后楼去住,好看看往来船上的行人。黄公子梦里也想不到他要逃走,就应允了,任他搬去。月兰暗暗欢喜,拣了一个好日搬了上去。不多几时,买通了楼下一个船户,趁那夜黄公子不在房中,先把金银细软打了一个包袱,开了楼窗,在窗洞内吊将下去;然后自己也用一条汗巾,一头紧系窗搭,一头拴在自己腰间,又用两手紧紧扳住窗口,耐着惊吓,大着胆子,慢慢的在楼上坠下船来,连夜开船逃走,离了杭州,趁轮船到上海去了。

黄府直到明日午后,见月兰还不开门,方才疑惑。在门外大声叫唤,也不见有人答应。黄公子就晓得事情不妙,叫了两个家人打开了门,进去看时,那里有什么金月兰的影子?楼窗大开,箱笼抖乱。开箱看时,所有金珠首饰,值钱细软,都被他收拾一空。黄公子气得目瞪口呆,气了一会,也无可如何,只得取了月兰两张照片,并大略开了一个失单,已有万金开外,自己去拜钱塘县,托他上紧追拿,又请他发一角公文到上海缉访。一面写信知会华洋同知,将失单、照片一同寄去,叫包探认真探访。明知一时海阔天空,无从缉获,只好暂时放下,再作理会。因是为了此事,心中不乐,便也懒懒的坐在家中,有一月有余并未出去。屡次叫人到县里催过几趟,也并无影响。

忽一日,钱塘县差了一个家人,来黄府报知公子,黄公子方才晓得金月兰现在上海,依旧挂牌应局。自从黄公子将照片、失单寄到上海之后,那华洋同知翁延寿便派了两个有名的包探,仔细采访。你想上海的包探何等精细,金月兰又不会改头换面,不多几日,早被两个包探访了出来,立时协同巡捕,将金月兰人赃并获,解到公堂。会审官略略问了几句,道:“我这里也不难为你,只把你移县解回杭州,等你主人自己发落就是了。”就把金月兰移交上海县收禁起来。上海县登时发了一角咨文到钱塘县,叫他派差来申,将金月兰提回核办。钱塘县接了咨文,连忙叫人到黄府送信,请示办法。

黄公子听了,心中反又踌躇起来,暗想:月兰虽然可恶,既自己经逃走,便成覆水难收,若仍把他提到杭州追赃审问,岂不辱没了相府的门楣?况且耐着现在的凄凉,想到当初的恩爱,不觉心早软了一半。心中盘算了一回,打定主意,方对那差人道:“你回去上覆你们贵上。这金月兰虽是府中逃妾,但是张扬起来,未免声名不雅。据我看来,不必一定去办他逃走的罪名,只不许他再做生意,也就是了。

请你们贵上就回一角文书,人也不必去提,只叫他具一个以后不再为娼的切结,再切实在上海县存一个案,如金月兰再在苏、杭、沪三处卖娼,便要彻底重究。你照我的话去说就是了。“钱塘差人诺诺连声,回去说了。钱塘县就发一角公文到上海县,存了一个案,准了金月兰具结取保出去,把一场天大的官司,化得来无影无踪,烟销火灭。

谁知金月兰江山好改,本性难移,只不敢在上海、苏、杭再做生意。闻得人说天津地方富盛,阔客极多,林黛玉、张书玉二人在天津不到两年,都是服用豪奢,外场阔绰,就是手中私蓄,何止万金,那衣饰尚不在数内,金月兰便想也到天津,投奔黛玉。他们本是要好姊妹,那有不收留他的道理。便收拾了随身的金珠衣服,趁了招商局新裕轮船的房舱。不一日,到了天津紫竹林。

停船上岸,好容易问到侯家后东天保南班林黛玉的寓所。黛玉见了月兰,惊喜交集,便问他如何脱身出来?月兰将逃走被拿、取保释放情形细说一遍,后说到上海不能再做生意,特地到天津投奔他的话。黛玉喜道:“这里正为人少做不出生意,要想去上海请人。我想近来上海的一班人也没有什么色艺双佳、擒纵客人的手段,所以我也不敢荐人。如今你既来此,甚是凑巧,那生意料想做得起的。我便叫本家替你预备房间,但房内的铺设是要的,两房间的陈设,少也要四五百块钱,你可打算得出么?”月兰道:“我身旁现银虽然不多,却有几十两金条在此,约莫也有二三千块钱,料想没有什么不够,这倒不用打算的。”黛玉更是欢喜,忙叫本家进来,说明缘故,要他预备房间。

那女本家名叫阿毛,也是上海人,大姐出身,近来着实有些积蓄,所以到天津来开这爿南班堂子。此时听得金月兰要包他的房间,见月兰年纪尚轻,风头又好,也是高兴,便满口答应。月兰开了箱子,取出六十两金条来托他去换,正正换了三千多块钱。俗语:“有钱诸事办。”不上两日,把月兰的房间收拾得花团锦簇。当夜由黛玉的熟客,一个候补道姓钱的,替他摆了一个双台。

从此之后,果然车马盈门,和酒纷纷不绝。约有半年光景,开销之外多了二千开外的衣饰,三千余两的现银,月兰得意非常。

那晓得祸不单行,福无双至。恰值拳匪之乱,联军破了天津,林黛玉、金月兰等一齐狼狈南归。金月兰只逃得一个空身,那黄家卷出来的金珠也丢得干干净净。

到了上海住不两日,联军又进了北京,信息一日紧似一日,风声鹤唳,草木皆兵。

月兰是个惊弓之鸟,更加寝食不安,只得又逃到苏州暂时住下,再听消息,恰好与章秋谷同住佛照楼栈房。此时金月兰除了随身衣服、头上钗环之外,已是一无所有。

这一日偶然看戏,无心中遇着了秋谷。他从前在上海时,与秋谷虽然认识,一则记忆不真,二则也不知秋谷有这样的英雄本领,只觉得秋谷人才出众,气宇轩昂,那一把刀舞得来滚雪飞花,神出鬼没,不觉脱口而出,叫了一声:“好呀!”及至秋谷下台之后,走到月兰面前仔细一认,方才猛然记了起来,便对他笑道:“我瞧着就有点像你,只是有些模糊,原来到底是你。我们有二三年不见了,也不知那一阵风把你这红人儿吹到这苏州地面来了,只怕有什么事情罢?”原来秋谷虽是认得月兰,嫁与黄公子一节却并不晓得。

金月兰此番到得苏州,两手空空,连房饭钱也无从设法,又不敢再做生意,正在进退两难、哭笑不得之际,见了秋谷,好似见了前世亲人一般,一把拉住道:“阿呀!果然是二少,我的事情一言难尽,好在我就住在此地佛照楼,你停回到我栈里去细细的说罢。”秋谷喜道:“我也是寓在佛照楼,凑巧得狠,等回儿回栈再说也好。”说着,仍到花云香桌上坐下。花云香早看得明白,冷笑道:“章大少,恭喜耐,咦到仔一位贵相知哉。”秋谷道:“你不要只管疑心。我从前在上海时就认得他的,并没有什么交情。你放心就是了。”云香道:“倪末阿有啥勿放心格,本来耐章大少格相好,阿关得倪啥事,倪是勿好来管耐格啘。”秋谷见他满面怒容,醋意可掬,便不去分说,只笑了一笑,只顾看戏。

台上《杀嫂》做完,换了小喜顺的《珍珠衫》上来。秋谷急欲同着金月兰回栈,要问问他的情形,却碍着花云香不便。恰巧云香的相帮走了进来,手中拿着几张局票来催云香去出党差,秋谷趁势叫他去罢,云香只得略坐一坐,立起来道:“难倪去哉,倪倒勿做啥讨厌人,等唔笃去随便那哼末哉。”秋谷也不理会,等到他去了,急急的走到月兰面前,低低说道:“这戏也没有什么看头,我们先回去罢!”月兰会意,点一点头,起身先走。随后秋谷出来,到了栈中,跟到金月兰房中坐下,二人方才剪烛长谈。

月兰细细把数年事情一字不遗告诉了秋谷,说到那身世飘零之苦,不觉滴下泪来,秋谷也为之太息不止。正是:

襄王旧梦迷巫峡,子建新诗拟洛妃。

欲知后事,请听下回。

\chapter{第三回 余香阁初点满堂红 章秋谷重过谈瀛里}
\chapter{第四回 金月兰无端受气 方幼惲有意寻芳}
\chapter{第五回 陆兰芳游园逢土地 方幼惲摆酒闹金刚}
\chapter{第六回 留夜厢假装阔客 抢汇票硬捉瘟生}

\chapter{第七回 车走雷声香尘一瞬 酒酣奇气名士高吟}

\chapter{第八回 章秋谷意气结新知 方幼惲平康逢旧识}

\chapter{第九回 章秋谷苦口劝迷途 陆兰芳惊心怜薄命}

\chapter{第十回 兆贵里刘厚卿行令 吉升栈张书玉发标}

\chapter{第十一回 对酒当歌忽逢旧友 阳春白雪快和新诗}

\chapter{第十二回 翻花样偷天换日 吊膀子接木移花}

\chapter{第十三回 汪宏超花钱代审 金汉良拼命吹牛}

\chapter{第十四回 一监生录遗受气 两承差讨赏翻腔}

\chapter{第十五回 曲辫子坐轿出风头 红倌人有心敲竹杠}

\chapter{第十六回 论妍媸畅谈电气 谈嫖界痛骂官场}

\chapter{第十七回 吃花酒初遇假同知 讽官场怒嘲真令尹}

\chapter{第十八回 设机关流氓传电报 卖风情名妓访萧郎}

\chapter{第十九回 闯房间莽客怒生波 圆好梦良宵花解语}

\chapter{第二十回 王云生安排紥火囤 章秋谷踏破仙人跳}

\chapter{第二十一回 闹张园醋海起风潮 苦劝和金刚寻旧好}

\chapter{第二十二回 香车宝马陌上相逢 纸醉金迷花前旖旎}

\chapter{第二十三回 瘟富翁误堕迷途 名校书安心淴浴}

\chapter{第二十四回 邱公子狠心惩爱妾 林黛玉拼命闹华堂}

\chapter{第二十五回 恨无良闭户锁金刚 消妒意开笼放鹦鹉}

\chapter{第二十六回 说瘟生平心论嫁娶 评嫖客谈笑骂官商}

\chapter{第二十七回 林黛玉春宵引凤 王云生黑夜捉奸}

\chapter{第二十八回 吹大话满口牛屄 露真情一箱石块}

\chapter{第二十九回 写伏辩光棍无颜 听良言名花有主}

\chapter{第三十回 章秋谷乱叉麻雀 陆畹香暗印灵犀}

\chapter{第三十一回 西安坊名士讲嫖经 高升栈优伶夸大口}

\chapter{第三十二回 吊膀子小丑帮忙 掉枪花秋娘中计}

\chapter{第三十三回 姘戏子苦劝陆畹香 扳差头驳倒花筱舫}

\chapter{第三十四回 杀风景莽客醉飞觞 意缠绵良宵花解语}

\chapter{第三十五回 暗提调碰和叫局 现开销当面坍台}

\chapter{第三十六回 说大话满口吹牛 摆双台安心落局}

\chapter{第三十七回 真急色春宵圆好梦 假堂差黑夜渡陈仓}

\chapter{第三十八回 还带挡做成圈套 订白头再捉瘟生}

\chapter{第三十九回 陆兰芬雨后试新妆 方子衡花前申旧约}

\chapter{第四十回 蓝桥咫尺旧雨不来 芳草天涯王孙归去}

\chapter{第四十一回 骂瘟生西楼惊好梦 唱骊歌南浦黯销魂}

\chapter{第四十二回 吃大菜粲花生妙谑 错房间无意遇名姝}

\chapter{第四十三回 章秋谷痛骂无耻奴 王佩兰暗吃山西醋}

\chapter{第四十四回 有情人都成新眷属 懊恼记重仿玉台文}

\chapter{第四十五回 说官话小子无知 困春悉萧娘多病}

\chapter{第四十六回 争闲气怒掷缠头 恶跳槽气伤名妓}

\chapter{第四十七回 负心郎黄衫求作合 薄命女紫玉竟成姻}

\chapter{第四十八回 章秋谷惊散野鸳鸯 霍春荣排演花蝴蝶}

\chapter{第四十九回 方小松演说风流案 贝夫人看戏丽华园}

\chapter{第五十回 巧姻缘良夜渡银河 杀风景三更飞黑索}

\chapter{第五十一回 美优伶驳翻堂上官 懦太史不问河东吼}

\chapter{第五十二回 霍春荣利口受官刑 宋子英丧心施骗局}

\chapter{第五十三回 弱书生几成薄幸郎 老学究怒责亲生女}

\chapter{第五十四回 拍马屁流氓讨好 抱春愁侠客传书}

\chapter{第五十五回 一封书琴心通绿绮 百尺楼黑夜盗红绡}

\chapter{第五十六回 真大胆登门报信 假小心曲意邀欢}

\chapter{第五十七回 贡春树一棹载名花 章秋谷良宵圆好梦}

\chapter{第五十八回 驰宝马争看绿衣郎 博枭庐埋冤曲辫子}

\chapter{第五十九回 萧静园输钱重约赌 王云生设计报前仇}

\chapter{第六十回 吃大菜贵绅中计 游虎丘画舫嬉春}

\chapter{第六十一回 倒脱靴两番骗局 破机关一怒挥拳}

\chapter{第六十二回 讨局帐当场出丑 托微波名士多情}

\chapter{第六十三回 会审官左袒黑心妇 金月兰不认薄情郎}

\chapter{第六十四回 章秋谷有心试名妓 玉太史临老入花丛}

\chapter{第六十五回 老风流艳福难销 美少年名花独占}

\chapter{第六十六回 苦温柔太史多情 空缋绻秋娘薄幸}

\chapter{第六十七回 桃花人面惆怅刘郎 细雨斜风重寻关盼}

\chapter{第六十八回 花彩云有意骗痴郎 王太史两番逃爱宠}

\chapter{第六十九回 兆贵里翰林出丑 春申浦名士吟秋}

\chapter{第七十回 好良宵诗征出阁词 留学生弹打章秋谷}

\chapter{第七十一回 李子霄他乡逢旧友 辛修甫谈笑讽良朋}

\chapter{第七十二回 章秋谷名花成眷属 张书玉陌上遇萧郎}

\chapter{第七十三回 李子霄销魂春照夜 沈剥皮拼命死贪财}

\chapter{第七十四回 假病危瞒天造谎 打官司教士分家}

\chapter{第七十五回 撩云拨雨夜渡银河 辣手狠心朝施毒计}

\chapter{第七十六回 假温柔瘟生中计 真淴浴名妓私奔}

\chapter{第七十七回 楼空燕子神女成虹 帘卷西风檀郎懊恼}

\chapter{第七十八回 洪月娥有心讹曲辫 沈仲思同病劝瘟生}

\chapter{第七十九回 论嫖界新小说收场 结全书九尾龟出现}

\chapter{第八十回 通关节花钱遭巨骗 捐道员拜客出风头}

\chapter{第八十一回 演前文重见九尾龟 醒迷途续成新小说}

\chapter{第八十二回 送萧郎南浦赠将离 返故乡天涯留别恨}

\chapter{第八十三回 风凄繐帐泣凤悲麟 月冷空房鸾孤鹄寡}

\chapter{第八十四回 办交涉庸奴降秩 谄大官观察欺贫}

\chapter{第八十五回 负奇冤烈女骂奸雄 溅热血公堂飞白刃}

\chapter{第八十六回 归故里堂上奉慈亲 泛轻舟姑苏逢旧友}

\chapter{第八十七回 卖风情陌路遇萧郎 感华年高楼圆好梦}

\chapter{第八十八回 章秋谷意外得奇逢 贡春树开筵宴良友}

\chapter{第八十九回 闯房间流氓横索诈 惩无理名士怒挥拳}

\chapter{第九十回 银汉仙槎刘郎惆怅 秋风莼菜张翰归来}

\chapter{第九十一回 开花榜名妓占鳌头 掷金钱瘟生游北里}

\chapter{第九十二回 红倌人安心施巧计 曲辫子拼命害相思}

\chapter{第九十三回 花低月亚虚度春宵 凤去台空可怜良夜}

\chapter{第九十四回 陈海秋痛恨范彩霞 章秋谷重游安垲第}

\chapter{第九十五回 当冤桶观察开心 吊膀子张园受辱}
\chapter{第九十六回 借洋钱硬捉瘟生 呼将伯欣逢故友}

\chapter{第九十七回 莺飞草长望断萧郎 添酒回灯重开夜宴}

\chapter{第九十八回 范彩霞安心慢客 东尚仁叫局碰和}

\chapter{第九十九回 叉麻雀名士讲牌经 卖风情倌人吊膀子}

\chapter{第一百回 打茶围乌龟送礼 出奇谋嫖客施威}

\chapter{第一百零一回 扣局帐陈海秋发标 留夜厢范彩霞中计}

\chapter{第一百零二回 酒阑人散软语缠绵 送客留髡深情缱绻}
\chapter{第一百零三回 味莼园遇旧感前游 金小宝寻春逢浪子}
\chapter{第一百零四回 跳空槽滑头得志 翻醋罐名妓争风

\chapter{第一百零五回 祝小春得意占情郎 章秋谷正言讥浪子}
\chapter{第一百零六回 危崖勒马虚度清宵 宝镜孤鸾枉辜良夜}
\chapter{第一百零七回 游张园初看髦儿戏 访萧郎又遇意中人}
\chapter{第一百零八回 情切切密意慰檀郎 意绵绵深情回倩女}
\chapter{第一百零九回 梦巫山良宵圆好事 忆倾城名士苦相思}
\chapter{第一百一十回 传眉语喜遇秋娘 托微波暗通青鸟}

\chapter{第一百一十一回 赋高唐东墙窥宋玉 隔巫峰云雨恼襄王}

\chapter{第一百一十二回 度良宵名花开并蒂 歌白纻病渴过三秋}

\chapter{第一百一十三回 久安里旧雨续新欢 春申浦高朋宴良夜}

\chapter{第一百一十四回 弃尘寰烈妇捐躯 征挽联豪绅仗义}

\chapter{第一百一十五回 看马戏忽逢荡妇 闻狮吼惊散鸳鸯}

\chapter{第一百一十六回 谋补缺观察入都 受苞苴奸奴作弊}

\chapter{第一百一十七回 严选政部办吃虚惊 出奇兵名优施巧计}

\chapter{第一百一十八回 闹相公尚书中计 告病假巡抚归田}
\chapter{第一百一十九回 思淴浴名妓嫁衰翁 约空房家妈私爱妾}

\chapter{第一百二十回 王素秋看戏轧姘头 柳飞云当场施绝技}
\chapter{第一百二十一回 联美眷荡子迷香 破温柔滑头泼醋}

\chapter{第一百二十二回 闹茶楼扬慕陶受窘 抱不平章秋谷解围}

\chapter{第一百二十三回 大观园流氓争口舌 乐仁里名士见秋娘}

\chapter{第一百二十四回 王素秋家庭翻醋瓮 康已生中冓咏新台}

\chapter{第一百二十五回 闹花厅白昼敦伦 闯深闺黄昏惊梦}

\chapter{第一百二十六回 感风寒中丞卧病 乱人伦令子宣劳}

\chapter{第一百二十七回锡佳名注释九尾鱼 写牢骚演说烟花史}

\chapter{第一百二十八回 换桃符阳春回大地 喧爆竹风雪度残年}

\chapter{第一百二十九回 假漂帐嫖客行权 真索债倌人受骗}

\chapter{第一百三十回 享温柔误人销金窟 敲竹杠偏遇守财奴}

\chapter{第一百三十一回 聚家庭天伦全乐事  度残年骨肉庆团圆}

\chapter{第一百三十二回 设华筵良朋守岁 兜喜神名妓迎春}

\chapter{第一百三十三回 让房间安心慢客 受讥评当面坍台}

\chapter{第一百三十四回 忍恶气冤桶无颜 遭白眼瘟生致病}

\chapter{第一百三十五回 发电信开函惊老母 抱不平疗病出奇方}

\chapter{第一百三十六回 抱沉疴三宵占勿药 起乡心千里整归装}

\chapter{第一百三十七回 讲嫖经名士高谈 打茶围瘟生吃醋}

\chapter{第一百三十八回 洪素卿昧良施巧计 章秋谷谈笑破奸谋}

\chapter{第一百三十九回 闯房间痛骂滑头 驱恩客难为名妓}

\chapter{第一百四十回 感良朋深交铭肺腑 论时艰极目痛山河}
\chapter{第一百四十一回 恨天涯深闺挥别泪 折将离南浦送檀郎}

\chapter{第一百四十二回 出吴淞离怀随逝水 走津沽壮志破长风}

\chapter{第一百四十三回 金观察夜走宝华班 章秋谷重到侯家后}

\chapter{第一百四十四回 舞衫歌扇清夜无愁 大道青楼良宵载酒}

\chapter{第一百四十五回 走章台良宵开夜宴 入花丛蓦地遇无盐}

\chapter{第一百四十六回 论交涉清言讥俗吏 纵微辞谈笑说官场}

\chapter{第一百四十七回 演活剧刻意绘春情 儆淫风当场飞黑索}

\chapter{第一百四十八回 印深情软语留春 谐好事平康选梦

\chapter{第一百四十九回 遇秋娘一箭贯双雕 卖丰姿春风描倩影}

\chapter{第一百五十回 矢从良缠绵倾肺腑 悲身世老大感年华}
\chapter{第一百五十一回 两调头翡翠共移巢 三鼎足鸳鸯齐比翼}

\chapter{第一百五十二回 循旧例双美拥檀郎 闹相公新知结幽愫}

\chapter{第一百五十三回 中和园书生听戏 升平班观察开筵}
\chapter{第一百五十四回 吃大菜安心寻绮梦 走歧途着意访名姝}

\chapter{第一百五十五回 访天台三士入桃源 定花榜群芳登上第}

\chapter{第一百五十六回 饯长亭良朋悲远别 脱火坑名士作冰人}

\chapter{第一百五十七回 解腰缠豪情成义举 翻醋翁冷语试深心}

\chapter{第一百五十八回 逢醉鬼狭路动干戈 数前尘花丛谈掌故}

\chapter{第一百五十九回 范彩霞歇夏观盛里 陆丽娟独游味莼园}

\chapter{第一百六十回 吊膀子淫令得意 闹包厢戏馆争风}

\chapter{第一百六十一回 泼醋当场争口舌 单相思狭路劫伶人}
\chapter{第一百六十二回 杜春心严亲怜少子 困债台名妓叹穷途}

\chapter{第一百六十三回 逢旧待深宵谈秘戏 索新逋软语媚干娘}

\chapter{第一百六十四回 逼残年倌人借债 丧良心小子探囊

\chapter{第一百六十五回 逐香尘游春驰绮陌 骋飞车奋勇捉瘟生}

\chapter{第一百六十六回 巧机关深谋排陷阱 奇遇合豪客入牢笼}

\chapter{第一百六十七回 蓄深心连环施妙策 狙缠头反扑出奇丈}

\chapter{第一百六十八回 假缠绵爱语稳痴人 真懊恼芳心乖宿愿}

\chapter{第一百六十九回 阻观光无端婴小极 喜同心着意护檀郎}

\chapter{第一百七十回 发清言高论寄牢骚 访桃源良朋联伴侣}
\chapter{第一百七十一回 证心期三生传慧业 听眉语一晌醉风情}

\chapter{第一百七十二回 赋皇华小星随使节 开绮席大尉遇佳人}

\chapter{第一百七十三回 慰离悰倾心结幽愫 上手本屈膝拜红裙}

\chapter{第一百七十四回 暮夜金奸奴行重贿 美人计相国赠明珠}

\chapter{第一百七十五回 联中外名妓说英雄 闹平康宵有张虐焰}

\chapter{第一百七十六回 杀风景恶客试尊拳 弃尘寰佳人悲薄命}

\chapter{第一百七十七回 罡风无赖折柳摧花 眉语彷徨双心一抹}

\chapter{第一百七十八回 渡银河秋娘联旧好 谐凤侣名士结新欢}

\chapter{第一百七十九回 真阅历发明攻战术 正比例研究床第谈}

\chapter{第一百八十回 忆前尘同游钓鱼巷 怀旧事重访莫愁湖}
\chapter{第一百八十一回 吃花酒騃儒得意 入乡闱词客观光}

\chapter{第一百八十二回 闹新闻撞墙翻瓦罐 洒霜毫论史出奇文}

\chapter{第一百八十三回 传急电游子还乡 开花榜庸奴得贿}

\chapter{第一百八十四回 挥别泪红杏嫁东风 讶奇遇仙云吐华月}

\chapter{第一百八十五回 辛修甫良宵逢旧识 汤娟娘薄命堕风尘}

\chapter{第一百八十六回 证前因深情结遥誓 出奇计险语试倾城}

\chapter{第一百八十七回 甘同梦永夜听鸡声 困洪波长堤成漏泽}

\chapter{第一百八十八回 悯哀鸿仁人兴义举 泛明湖好景入诗囊}

\chapter{第一百八十九回 吞存款市侩昧良 萎慈萱北堂弃养}

\chapter{第一百九十回 章秋谷闭门守制 祁祖云挟忿兴谣}

\chapter{第一百九十一回 救灾黎大开赛珍会 放焰火普照不夜城}

\chapter{第一百九十二回 阻星期曲房惊好梦 行酒令东阁宴嘉宾}


第三回 余香阁初点满堂红 章秋谷重过谈瀛里





却说金月兰重提旧事,挥泪不已。秋谷劝了一回,又问他道:“你现在既到苏州,生意又不能做,总要想个法子才好,难道住在客栈一辈子不成?”月兰乘势说道:“现在我是一个落难的人,还有什么一定的主意?我的意思,只要拣一个中意的客人暂时同住,叫他认了我的开销,或者竟嫁了他。那从前的事,也是一时之错,追悔也追悔不来了。”说着眼圈儿又一红。秋谷见了,甚是可怜着他,便道:“你的主意虽好,只是急切之间,那里就寻得出什么中意的客人,这不又是一件难事么?”

月兰见他假做不知,绝不兜搭,心中暗暗着急,便把坐的椅子往前挪了一挪,挨着秋谷,低声说道:“我们既是认得一场,今日又恰好在此相遇,你总要替我打算打算,难不成你看着我落薄在此地么?”秋谷道:“你这样一个人,落薄是万万不会的,但请放心就是。你现在的意思,不过是要人认你的开销,那倒不妨。真到十分过不去的时候,我自然要同你想法。只是你要拣一个中意客人,是个难题目。我又不是你的肚子里蛔虫,我可知道你中意的是什么人呢?”月兰更加着急,皱了眉头,把秋谷的手紧紧拉住道:“你同我认得也不是一天了,我的脾气你也不是不晓得,虽然没有什么交情,我到了这个时候,你还要装着糊涂来取笑我么?”

秋谷是个聪明绝顶的人,又是粉阵花丛的老手,那有不领会他的意思?只为金月兰是个豪奢放荡的大名家,与四大金刚不相上下,你想他在黄中堂家尚且逃了出来,别人可是供给他得起的?所以心里徘徊,不肯爽爽快快的答应。此刻见金月兰发了急,方才说道:“你的意思,我岂有不知?只是我却也有我的心事。我们现在是要好的,万一将来一言不合,翻转面来,何苦为好成仇,弄到一场没趣?况且我的情形,你是向来知道的,不过是一个外场。你是中堂府里出来的人,怎能弄得到一块儿?你到自己仔细想想,不要一下子闹冒失了,收不回来。我看还是图个暂时的好。”

月兰听了秋谷一番说话,真个被他刺入心脾,无从分说,长叹一声道:“你的说话原也难怪。我如今若要赌神罚咒的分解,料想你也是不相信的,我也勉强不来,只好日后见我的心罢了。只是可怜我金月兰,当初时节,何等锋芒,差不多有点钱的客人,花了无数银钱,休想近着我的身体。不料我一时错了主意,自己在黄家走了出来,到了今日之下,就像做梦一般。我便自家迁就,别人也还有许多推托,今世那得还有出头,不如就……”月兰说到这里,良心发现,心上一酸,早呜呜咽咽的,那眼泪就如断线珍珠一般落了下来,点点滴滴的,秋谷手上也沾了几点。

秋谷见他如此,心中老大不忍,连忙偎着她粉面道:“你不要这等伤心,我答应就是了。”月兰趁势把纤腰一扭,和身倒在秋谷怀中,含着一包眼泪,欲言不语的道:“我命苦到这般田地,你还这样硬着心肠,怎的叫人不心上难过呢?”说着,又低头拭泪。那神情态度,犹如雨打桃花,风吹杨柳。正是:

三眠初起,春融楚国之腰;半面慵妆,香委甄家之髻。

那一阵阵的粉香兰气,更熏得人色授魂飞。秋谷见了,好生怜惜,无限关情。

心中想道:这样的上门生意,落得顺水推船,且图现在的风流,莫管将来的牵惹,难道我章秋谷这样一个人,就会上了他的当么?当下取出一块丝巾,为他拭干眼泪,又密密切切的劝慰了一番。此夜桥填乌鹊,春泛灵槎,玉漏三更,双星照影。杨柳怀中之玉,春意温存;胭脂颊上之痕,梨涡熨贴。真个是:

但能神女销魂夜,便是檀奴得意时。

且说秋谷一连三日不出栈门,花、许二家也来请过几次,秋谷虽随口答应,却只是不去。到得却情不过,勉强也去了两次。只天天与金月兰坐坐马车,吃吃大菜,有时去丹桂看戏,也只到十点多钟,便被金月兰拉着回来。

如此又是月余,秋谷动了思亲之念,对月兰说知,要回常熟。月兰要跟着到常熟去。秋谷不允,叫月兰先去上海等他。月兰那里肯依,道:“我现在打定主意,没有第二个念头。你到那里,我跟到那里,好好歹歹要同在一起,总然吃苦,也是情愿的。”秋谷被他缠死了,无可奈何,只得权时答应。雇了一只二号快船,搬下行李,算清栈帐,明日想要动身,却心中想道:我在青阳地住了多时,不曾出什么名,明日既要回去,定要花几个钱闹一个大大的名气,方不枉到此一场。必须如此如此,方才妥当。主意已定,便取出表来一看,恰才三点一刻,也不与月兰说知,立起身来,出了佛照楼,一直到余香阁来。

上了楼一看,只见坐得满满的。堂倌见了秋谷,赶紧走过来招呼,引到台前,好容易在头排排了一张椅子,请秋谷坐下,泡好了茶。秋谷举目看时,花云香、许宝琴二人都尚未到,台上只有十余人,暗想:今天已经不早,如何他二人还不见来?

一面转念,堂倌早送上点戏牌来。秋谷便问堂倌道:“今日为何人少?”堂倌陪笑道:“现在日长了,要到五点余钟方住,所以有些好的还没有来,若来齐,也有二十余人。”秋谷打量台上的椅位,正面十张,两旁每面八张,一共二十六把椅子,就对堂倌道:“你们这里台上通共二十六张椅子,我要照着椅子的人数,点一个满堂红。你快去叫人,不要迟误。”堂倌听了,屁滚尿流,诺诺连声的连忙走到柜上帐台说了,立刻叫人到各处书寓去催。

果然歇不多时,那些倌人陆续的来了,许宝琴也随后而来,只有花云香来得最迟。秋谷看他精神惨淡,宝髻惺忪,脂粉不施,蛾眉半蹙,那一种低徊宛转的神情,明露着十分幽怨。秋谷想:他那天临走之时本是满心醋意,后来一连半月不到他家走动,只听娘姨来请时说他有病,我则以为是他们请客的一句口头说话,今日看他这付神气,又像真有病的一般。一头思想,一面打量台上的倌人,竟有一半认得的。

堂倌早捧着笔砚粉牌在旁伺候,秋谷分付道:“许宝琴、花云香每人十出,其余一概每人两出,你随便配搭去写罢。”堂倌答应了下去,自去料理。

不多时,台上早挂出十几面牌来。秋谷看时,只见一半都是京戏,也有几支小调,一半便是梆子、昆腔。那班台上倌人听得有点满堂红的客人,未免众人的视线都聚在秋谷一人身上,大家脉脉含情。跟来的娘姨、大姐,早各人拿着银水烟袋,争先恐后的走下台来装烟应酬。有老有少,有村有俏,登时把一个章秋谷团团围住,就像一座肉屏风一般。秋谷面前一张台上的银水烟筒,排得满台都是。秋谷左顾右盼,如入山阴道上,应接不暇,不觉满心大乐。忙乱了一会,众人方才散去。台上花、许二人,已经唱了几折,接着别人唱下去。

秋谷此番原不过要闹个名头,并不是有心听曲,见花、许二人唱过,就在身旁摸出一卷钞票来,点点数目,叫堂倌过来交代道:“一共七十块钱的钞票,内中六十八块是点戏的钱,至于桌子的钱,今天并没有照会你们预定台子,你们也没有地方,多的两块钱,就算赏了你罢。”堂倌连声称谢,接了自去分派。

秋谷整顿衣服,要待立起走时,娘姨人等又早一哄而来,拥住秋谷,七张八嘴的要秋谷去坐坐。秋谷道:“我今日还有别事,一家也不能来,明日两点钟时,叫你们先生早些梳头,我放马车到门口来接,请你们多兜两个圈子何如?”众人还不肯放,你拉我扯的。秋谷洒脱众人的手,头也不回,一直走下楼来,也不回栈,径到谈瀛里花家来。

云香尚未回来,只有他的妹子花彩云在家,见秋谷进来,忙起身笑道:“阿呀!

贵人勿踏贱地,倪搭长远勿来哉啘,阿姊牵记得来!请宽仔马褂坐歇,对勿住,阿姊就要转格。“自己走过来替秋谷脱了马褂,挂上衣架,推他坐下。秋谷问道:”我才看见云香瘦了许多,头也不梳,好像有了病的样子。既然有病,为什么又要出去冒风?“彩云道:”格两日倪阿姊本来勿出来格呀,难末刚刚困好,书场浪来叫哉,说耐二少点子戏下来哉。耐二少爷面子,是勿能勿去格啘。“秋谷笑道:”言重之至,我早知云香有病,我决不来多事的。“

正说不了,早听楼梯上一阵脚声,云香掀着软帘走了进来,口中喘个不住,一屁股就坐在门口一张椅子上,面色也不狠好看。停了约有一杯茶的时候,方才渐渐的住了喘,回过面色来,向秋谷瞪了一眼,道:“谢谢耐格好作成,倪今朝头里向正有点发热,困也困哉,勿壳张耐来起花样,阿要诧异。”秋谷走到云香的面前深深一揖,道:“千不是,万不是,总是我的不是。但是你既然发热,何苦一定要出来?只要打发人招呼一声就是了,难道我好怪了你么?”云香冷笑一声道:“阿唷!

耐章二少爷来叫,阿敢勿去!倪无啥错处末,还要想扳倪个差头,禁得倪再要回报仔勿来,是人也杀得脱个哉!“秋谷道:”好奇怪!我何曾扳过你的错处,你倒要说个明白。“云香道:”请仔耐十几埭,耐定规勿来,还说勿曾扳差头!“秋谷道:”我另有应酬,分不开身,并不是怪你不来,难道这就算扳了你的错处么?“云香扳着面孔道:”自然哙,几年格老相好哉,阿肯勿应酬俚,惯脱仔到倪搭来格。“

把章秋谷说得无言可答。又见他娇嗔满面,情不自禁,自己扪心想想,实在有些对不起他,只得陪着小心殷勤相劝。又道:“你的病不打紧,只要多吃白糖,包管立时就好。”云香诧异道:“咦来瞎三话四哉,阿有啥人生仔病,吃点白糖就会好格?”

秋谷忍笑道:“你岂不知糖能解醋?你的毛病不是醋上来的么?”说得云香又觉好笑,又觉好气,把手狠狠在秋谷身上一推,道:“阿要热昏,啥人来理耐嗄!”秋谷也哈哈的笑了,当夜不表。

且说秋谷明日起来,便到许宝琴家去了一趟,又将各处局帐开销清楚,便回佛照楼来。见了月兰,问他昨夜住在什么地方,秋谷依实回答,月兰默然不语。秋谷觉得月兰也有几分醋意,便将别话打岔开了,随向月兰道:“今日一准要下船的,你先到船上招呼行李,我还到朋友人家走走,再下船来。”月兰依言,把随身的衣服铺盖叫娘姨收拾好了,发下船去,自己随后下船。

秋谷见月兰去了,忙忙的到甘棠桥边,叫一个素日相识的马夫名叫歪毛阿桂的,叫他代叫十四辆橡皮马车,立刻等着要兜圈子。阿桂呆了一呆,问:“要这许多马车何用?”秋谷道:“你不要多管闲事,快去叫来。”阿桂果然飞奔去了。不到一点钟时候,马车都已雇齐,齐齐整整停在甘棠桥下。秋谷便拣一部最新的橡皮车,两个马夫都穿着玄色丝绒水钻镶嵌的号衣,自己坐下,招呼那一众马夫跟着,先到如意堂去接陆韵仙、王二宝、金小宝,又到翠凤堂接小林黛玉、陈巧林等,许宝琴、花云香家是不必说,自然一定在内的了。原来秋谷安心闹标劲,所以把昨日在余香阁的所有倌人通通叫到,要做一个大跑马车的胜会。正是:

潘郎年少,香留陌上之尘;苏小风流,春压鞭丝之影。

后来究竟如何,请听下回分解。





第四回 金月兰无端受气 方幼惲有意寻芳





却说秋谷叫齐了那班倌人,两人合坐一车,独秋谷在后与花云香同坐。当下十四部马车,别人在前,秋谷压尾,头连尾接,就如一条游龙一般。马夫把马加上一鞭,各逞精神,那一群马车,便风驰电掣,滔滔滚滚,直向二马路一带兜转来。旁观的人,见十余部马车络绎而来,末后一部车上坐着秋谷,精神轩翥,丰度翩翩,香留荀令之裾,粉傅何郎之面,真似灵和疏柳,张绪当年。花云香与秋谷同坐一车,神彩惊鸿,珮环回雪。半偏云髻,梁家堕马之妆;斜倚香肩,赵后回风之体。又似海棠炤夜,芍药扶春。看的人个个目眩心迷,神惊色骇。再兼那前面坐的倌人,也都是骨格轻盈,丰姿婀娜,争娇斗艳,目送眉迎,把两边茶楼上的客人以及马路的行人都看得呆了,不觉齐声喝彩,啧啧叹羡。秋谷听在耳中,甚是舒畅,连兜了两三个圈子,便叫马夫把马车放到纱厂码头上船。

到了码头,秋谷跨下车来,随开发马夫,叫仍送他们回去,自己便要上船。只见一群倌人一齐下来,拥着秋谷,你一句我一言的说个不了。秋谷忙乱之中也听不仔细,大约是叫他下次早来的意思。秋谷只点头答应。只有花云香携着秋谷的手再三叮嘱,见秋谷匆匆要走,忍不住淌下泪来。秋谷也只好劝他几句,并说不多时就来的话,云香掩泪点头。秋谷也凄然不舍,狠着心撇开云香,跳上船去,立在船头,望着云香等上了马车,看不见了,方才无精打彩的进舱。

金月兰在船窗内望见一大群倌人围住秋谷,恋恋不舍,心中不大自然,却又不好发作。此刻见秋谷面上不甚高兴,倒要打起精神,殷殷勤勤的陪着他谈笑。秋谷倒底是个豪士,一会儿便不放在心上,吩咐船家开船,望常熟进发。

那常熟离苏州只有一日路程,本是苏州府属该管,在船上只住了一夜,明日上午却早到了。秋谷想月兰虽然跟来,万不能同着回去,只好自己先行上岸,到一个同窗朋友家中,与他商量,要替月兰另租房子。

那朋友姓史,字玉卿,狠有几处房产,家中颇是有钱,见秋谷与他商量,便道:“你要租房子,却来得凑巧,我对门一所房子,是楼上楼下十间水阁,房客前月才搬去的。我们至好,也不争论你的房租,竟是请你的贵相知搬进去就是了。”秋谷大喜致谢,又道:“既承吾兄如此关切,租金一定加倍奉上,只是没有动用器物,却一总要借你府上的了。”史玉卿也一口应允。秋谷便先付了二十元房租。史玉卿再三推不脱,只得收了,立刻叫人搬了一张花梨六柱藤床,并些桌椅梳头台等器皿、动用物件过去。好在人多手众,七手八脚,就登时铺设起来。秋谷再回船,叫船家把船放到水阁码头,打发月兰上岸,开销了船钱,船家自去,便同着月兰往楼上房间里来。

月兰见房子虽然不大,却甚是精致,也觉心中欢喜。月兰原带着一个娘姨,便打开铺盖,铺在大床上,挂好帐子。坐不多一刻,早见史家的家人送了一桌菜过来,还有一坛绍酒,向秋谷道:“家爷说,本要与章少爷接风,因自己不便过来,所以送一桌菜在此,要章少爷赏收。”秋谷道:“难为你老爷费心,想得周到,回去替我着实道谢。”封了一块钱赏他,秋谷饭后又到玉卿家,托他寻了一个厨子。当夜晚膳,也是史家送来。秋谷当晚且不回去,就在月兰那边往下。

月兰便一心一意的要嫁秋谷,那知秋谷心上却又不然,心中暗暗的打着算盘,想道:我当初顺口答应,以为他是收不住缰绳的野马,万不肯真心嫁人,不料他竟是认真起来,这便如何是好?又想了一会道:他此时一心嫁我,是恋着我貌美力强,也不是贪图什么别事。现在我的竭力应酬哄骗他,是趁着一团高兴,博个片刻风情,更不是生死难离的情分。不要说太夫人治家严肃,断断不肯答应娶一个妓女进门,就是瞒着太夫人,把他养在外边,一则不是长久之计;二则妓女水性杨花,只图枕席的欢娱,不顾丈夫的廉耻,自己是长要出门的,又不能处处带他同去,那时孤灯寂寞,长夜凄凉,难保不别生他念;三则既做良家妇女,便有良家妇女的规模,他这样一个飞扬荡佚的人,只看中堂府内尚且逃走出来,何况我一个中人之产,怎样供得他的挥霍、称得他的心情?万一再有卷逃等事,难道我还做第二个黄伯润么?存了这个念头,便觉万万娶他不得。但是他欢天喜地在苏州跟了出来,又不好无缘无故的叫他回去。他既想着一心嫁我的主意,料想也不肯好好开交,便又为难起来。踌躇一会,忽然得计道:“只消如此这般,叫他自己不愿起来,自然改了念头,也就罢了。”定了主意,方才睡去。

到了次日,秋谷将自己行李搬回家去,又叫了两个老年诚实的家人看守门户,私自吩咐:“无论何人,不许放进,并不许放金月兰主仆走出大门。”两人诺诺领命。秋谷又交代了月兰几句说话:“略停一二日就来看你,你须要定心住下,不可心焦。”交代过了,秋谷便自回去。

月兰等了两日,不见他来,以为必是家中有事耽搁住了。那知秋谷一去不来,直等到半月有余,还是绝无影响。问问那两个家人,又都是装聋做哑,假推不知。虽然饮食不缺,却是寂寞异常,无聊之极。月兰发起急来,要叫娘姨到秋谷家中去请,却被那两个看门的家人拦住,说:“少爷交代过的,一概闲人不许进门,你们也不许出去。”月兰气得发昏,与家人闹了一场。家人不去理会,只是守着门口不放出门。

要知金月兰是个有名荡妇,他此次安心要嫁秋谷,是贪图他貌美力强,要想和他夜夜并头,朝朝交颈,怎禁得秋谷冷淡了他半月有余,又把他关在这陌生地方,不许他出去消遣。这等情形,叫月兰如何忍耐得住?

看看已过了一月,秋谷依然不来,月兰度日如年,急得没法,方才后悔起来。想道:现在人还未到他家,尚且把我这般冷淡,将来到了他家之后,还不知要怎生打发,那里保得住久后的恩情?便暗暗的又想脱身之法。但是自己身无一文,就是脱身出来,作何计较?左思右想,没法儿,只得呆呆的等着秋谷。

直到了四十余日,秋谷方才来了。月兰见秋谷到来,好似黑夜里拾着了斗大明珠一般,一把拉住道:“你好,你好,去了一个多月,面都不见,却叫着家人来糟蹋我,可是该的么?你临走的时候,说一两天就来看我,那知今日望你不来,明日望你不来,差不多把我的眼睛要望穿了。我只认着你把我丢在这里,一世不来的了,你也还有来的日子么?”秋谷故意道:“那两个家人是我叫他们来看门的,怎么会得罪起你来?他们那里有这样的大胆?”月兰便把要叫娘姨来请、家人不许出门的话说知。秋谷故意把家人叫将进来,骂了几句,却暗暗的好笑。月兰又问他多时不来的缘故,可是家里少奶奶管束得凶,不许出来么?秋谷假作面上一红,口中支吾推托道:“我出来得日子久了,到得家里,就被事情缠住,天天想来看你,实在不得脱身,难道少奶奶管得住我么?若管得住,也不放我到苏州去了。”月兰道:“少奶奶向来原是相信你的,所以放你出来;现在不相信你了,自然就不肯放你出门了。”秋谷道:“不要胡说!我章秋谷可是惧内的么?”月兰鼻子里嗤的笑了一声,又把嘴一披道:“啊唷!还要海外!凭你如何解说,我也总不上当的了。”秋谷一笑,忙用别话岔开。冷眼看月兰相待的情形,已不似从前十分熨帖、万种缠绵的样子,心中暗暗得计。

到得晚间,月兰慢慢说起从前未嫁黄伯润之先,有两房间外国木器,铁床、藤椅、大菜台面、汤台一应俱全,寄在娘姨家里,现在既然嫁你,这些器具丢在上海也甚可惜,意思要先到上海一趟,去搬了回来,此处也好摆设,只是自家没有盘费去搬的话,婉婉转转的说了出来。心上还是忐忐忑忑的,恐怕秋谷不肯放他。那知秋谷心上虽然明白,外面只做不知,欣然答道:“我正愁此间的器具不够使用,既有两房间木器在上海,你去搬来甚好。你明日便可动身前去,盘费是小事,你约着要用多少洋钱,我给你就是了。”

月兰见秋谷一口允许,心中大喜。又盘算了一会,方才答道:“明日就走也好。但是我既到上海,总要去会会姊妹们的,我身上没有一件应时的衣饰,怎好意思见人?免不得要你花费。连着往来用度,恐怕也要几百块钱,不知你明日可来得及?”秋谷明和其故,微笑一笑,答道:“几百洋钱也不是什么大事,料想我还预备得来。但是衣服首饰,也只要略略置备些,场面过得去,不致坍台也就是了。”月兰更喜,把秋谷竭力奉承。

这一夜,翠倚红偎,香温玉软。颠狂凤女,春迷洞口之云;前度刘郎,夜捣蓝桥之杵,直到明日午间方起。秋谷便急到一处往来的庄上取了二百洋钱,又向银楼兑了一支珍珠镶嵌的押发。回到月兰处来,将洋钱、押发交与月兰道:“这支押发虽不甚好,也可勉强带得。至于衣服,上海衣庄现成的狠多,你到上海再买也还不迟。这二百洋钱,做来去的盘费,并买几件衣服,料也够了。到了上海,若没有甚事,便赶快些回来,不要十分耽搁。今日晚了,来不及开船。我叫人去雇好了船,你就今夜上船,明日一早好开。”月兰听一句,答应一句,偷眼看秋谷甚是高兴,止不住流出眼泪来;又怕秋谷看见根问,慌忙背过脸去,将巾拭干。

秋谷虽也看见,只作不知,叫了家人进来,叫立刻雇只快船,先到苏州;到了苏州,用小火轮拖至上海。家人答应去了。秋谷也一面留心金月兰的举动,见他尚有些依恋之意,暗中点头,知他天良尚未泯灭,究比林黛玉等较胜一筹,未免心中也有些惆怅。两人大家怀着鬼胎,却不能说出。日西时候,叫船家人回来,船已雇好,开了过来。秋谷便令家人替月兰收拾行李,料理上船,在船上吃了一顿晚膳,秋谷便仍住在船上,此夜比前更加欢畅。

天明后,秋谷起身上岸。月兰惺忪两鬓,携着秋谷的手,送到船头。秋谷立在岸上,看着月兰。月兰却含着两包眼泪,呆呆的也看着秋谷。眼睁睁的看船家拔篙起缆,一棒锣声,那船早顺流而去。秋谷不觉长叹一声,回进水阁,把器具一切还了玉卿,又将房子交代了,便自回去。

如今要把秋谷一边暂时按下。再提起两个曲辫子客人来,只为羡慕张书玉、陆兰芬四大金刚的名望,挟着重资到上海来结交他。但是眼孔不大,终久舍不得大注银钱,又是语言无味,面目可憎,行动举止不免有些寿头寿脑。你想这等的豪华名妓,那里看得上这种客人?到后来卒至花了一注大钱,受了几场闷气。正是:

人前输却三分丑,被底赢来一段骚。

后来幸而遇着章秋谷替他出场争回场面,劝他回去,他从此知难而退,不敢再到春申。

闲语休提,书归正传。且说常州东门内有一家著名乡宦,姓方名惲,是个翰林出身。散馆得了知县,论俸推升,做了几年贵州知府,便告了病回来。止生一子,名叫宝椿,别字幼惲。这方知府把他钟爱非常。到得渐渐长成,方知府替他娶了贝季瑰太史之妹为媳,便把家事交他掌管。

方幼惲出身纨袴,菽麦不辨,甘苦不知,却只爱奢华放荡;又是生性吝啬,等闲不肯破费一文。一向听亲友在上海回来,夸说上海如何热闹,马路如何平坦,倌人如何标致,心中便跃跃欲动。此番趁方知府将家事叫他独掌,便与方知府说明,要到上海去见见世面。方知府心中虽觉不甚喜欢,因是向来溺爱惯的,不忍拂他,只得允许,只再三叮嘱早早回来。这方幼惲便欢天喜地的择了行期,雇好了船,辞别了方知府竟往上海去了。正是:

岂有画堂登犬豕,从来名妓爱金钱。

未知方幼惲究竟如何,请听下回分解。





第五回 陆兰芳游园逢土地 方幼惲摆酒闹金刚





且说方幼惲到了上海,拣了石路上一处客栈,是他的本家一位方运判开的,名叫吉升栈,占一间大号官房住下。

这方幼惲初到上海,没有认得的亲友,叫家人帮着茶房铺好行李之后,便走到帐房中来,想和帐房先生谈谈。刚刚跨进帐房门口,见一个人手中拿着一篇帐单,直闯出来,几乎把幼惲撞了一个满怀。幼惲与那人同吃一惊,停住脚步,那人把幼惲认了一认,便大笑道:“原来是幼惲兄,几时到的?你是难得到上海来的呀!”

方幼惲定睛一看,不是别人,是他的表亲同乡,姓刘,号厚卿,颇有家财,专喜游荡,只是性情刻啬,也同方幼惲一般。平日方幼惲与他极是亲密,比时一见厚卿,便心中大喜,答道:“我是今天才到,你想必到此多时了。”厚卿道:“我也止到得十多日,不到半月。”幼惲道:“今日遇着了你狠好,我初到此地,一些没有头脑,你比我多到过几次,自然样样熟悉。我此番到此,是仰慕四大金刚的名气,要来见识见识怎样一个好法。你可认得他们么?厚卿笑道:”不瞒你老兄说,兄弟此来亦是为此。现在我做的倌人,就是四大金刚之一,名叫张书玉,应酬工夫再好没有。你今天到此,本要替你接风,晚上就请你到张书玉家吃饭何如?“幼惲听了大乐,便和厚卿同回房间。

坐了一会,厚卿道:“这栈里的饭菜恶劣非常,我们还是上馆子去罢。”同了幼惲走出吉升栈,望雅叙园来,拣了一个雅座坐下。堂倌送上烟茶,便来问菜。幼惲先要了红烧大肠、油爆肚;厚卿要了炒肉片、炸八块、鲫鱼汤,要了一壶京庄,又要了醉虾、拌腰片两个碟子。两人先对酌起来。一会,堂倌送上菜来,味儿甚好,吃毕算帐,却甚是便宜,止一千六百余文。两人走到柜上,厚卿会了帐,同到四马路来,在升平楼吃了一碗茶。徜徉一刻,已有三点余钟光景,厚卿便同幼惲回到栈房。

幼惲要坐马车到张园去,叫茶房去叫了一部橡皮马车来。二人上车坐下,马夫摇动鞭子,那马四蹄跑动,如飞而去。刘厚卿是司空见惯,不以为奇。方幼惲却从未坐过,觉得双轮一瞬,电闪星流,异常爽快。那马车望张园一路而来。这日却好是礼拜六,倌人来往的马车甚是热闹,方幼惲坐在车中,那头就如泼浪鼓一般,不住的东西摇晃,真是目迷五色,银海生花。

到了张园,在安垲第泡了一碗茶,坐下看时,倌人来得不多,疏疏落落的。方幼惲见来人尚少,要到别处去走走,被刘厚卿一把拉住,道:“少停一会,就有倌人到来,你且坐着,不要性急到各处去乱走。”方幼惲只得坐下。果然,不多时,粉白黛绿一群群联队而来,一个个都是飞燕新妆,惊鸿态度,身上的衣服不是绣花,就是外国缎,更有浑身镶嵌水钻,晶光晃耀的。

方幼惲正在看得有些头晕,只见一个倌人走到面前,朝着刘厚卿微笑点头,便款步向隔壁一张桌上坐下。方幼惲提起精神,细细的打量他。只见他穿一件蜜色素缎棉袄,下系品蓝绣花缎裙,露着一线湖色镶边的裤子,下着玄色弓鞋,一搦凌波,尖如削笋,看得方幼惲已是浑身发痒。再往头上看时,梳一个涵烟笼雾灵蛇髻,插一支珍珠紥就斜飞凤簪饰,虽是不多几件,而珠光宝气晔晔照人;薄施脂粉,淡扫蛾眉,虽无林下之风,大有萧疏之态。直把个方幼惲看得一双眼睛钉在那倌人身上,呆呆的出了神去,任凭刘厚卿与他说话,他耳中总未听见。

刘厚卿觉得诧异,回过头来,见他这般光景,不觉失声一笑。方把那方幼惲出窍的神魂重新提上身来,惊得一身冷汗。那倌人听得刘厚卿失笑,也回头一看,见方幼惲虽是衣装炫耀,却有些土头土脑的神情;又见他两只眼睛对着自家目不转瞬的呆看,被刘厚卿这一笑,惊得直立起来,失张落智的大有曲气,不觉樱唇半启,皓齿微呈,对着方幼惲嫣然微笑。这方幼惲的神魂,方才被刘厚卿一笑吓了回来,又被那倌人这一笑,把方幼惲的三魂七魄一齐飞出顶门,飘飘荡荡的不知散向何处,浑身骨节十分松快,却坐也不是,立也不是,满身的不得劲儿。刘厚卿在旁看着,甚是好笑。

幼惲好容易定了一回神,挣紥住了,回头低问厚卿那倌人叫甚名字。厚卿哈哈的笑道:“你两人对看了半天,难道还没有晓得名姓么?待我来同你两位做个媒人,见一个礼可好?”那倌人面上一红,瞟了厚卿一眼。厚卿便向那倌人道:“这位是方少大人,在常州第一个有名的富户。”回头又向幼惲道:“你道他是谁人?就是四大金刚坐第一把交椅的陆兰芬哟!你的眼力居然不错。”

方幼惲听得就是陆兰芬,心中更加大喜,以为陆兰芬是上海第一个名妓,尚且有情于我,何况别人?在兰芬心上却又是一个念头,想道:起先我看他是个寿头码子,所以对他一笑,并不是有心吊他的膀子;但他既是个有名的富户,料想总肯花几个钱,做妓女的钱财为重,不免折些志气,将机就计的去拉拢他。便放出手段来,那一双勾魂摄魄的媚眼,连飞了方幼惲几眼,又向他略略点头。方幼惲虽是门外汉,然而眼风总是看得出的,不觉乐得手舞足蹈。陆兰芬见他已经入彀,便算了茶钱,立起身来,向刘厚卿道:“倪先去哉。”又向方幼惲一笑道:“晏歇一淘请过来。”

临去之时,又似笑非笑的看了幼惲一眼,方才姗姗而去。

方幼惲直看他出了安垲第,方才要问刘厚卿陆兰芬住在那里,早见厚卿竖起一个大指头向着方幼惲道:“好运气!第一回看见就吊你的膀子。看你不出倒是个老手。”幼惲便问什么叫吊膀子。刘厚卿笑得打跌道:“你连吊膀子都不晓得么?”

便告诉了他原故,幼惲方始恍然大悟。于是两人出了大洋房,寻着马车坐下,径回原路。马夫照例在四马路兜了两个圈子。其时已是掌灯,厚卿叫马夫不必回栈,到新清和坊停车,叫他回栈到帐房去算帐。二人跳下车来,马夫驱车自去。

刘厚卿同着方幼惲走进清和坊巷,不多几家,便是张书玉的牌子。厚卿不让幼惲,竟自当先走进。幼惲暗暗诧异。走到扶梯,听得相帮高叫一声,也听不出叫的什么,倒把幼惲吓一了跳,立住了脚不敢上去。厚卿上了扶梯,连连招手,幼惲方才跟着上来。早见左首的一间房间,高高打起绣花门帘。张书玉满面春风立在门口,叫了一声:“刘大少!”厚卿一面招呼,一面跨进房去。幼惲跟进房门,厚卿让幼惲在炕上坐下。只见一个娘姨过来对幼惲道:“大少,宽宽马褂嗫。”幼惲慌忙立起身来,脱下马褂,娘姨便来接去,不防张书玉端着一盆西瓜子,要递与幼惲,口内问他尊姓。幼惲见张书玉前来应酬,连忙立起身来,恭恭敬敬的答应了一声:“我姓方。”双手去接书玉手中的盆子。书玉忍不住掩口要笑,那接着马褂的娘姨也笑起来。方幼惲自知错了,涨红了脸,把手往回一缩,书玉手中一个脱空,把一只高脚玻璃盆子跌在地下,打得粉碎。书玉倒吃一惊,惹得一房间的人都笑起来,刘厚卿也止不住要笑,却见方幼惲一张脸上涨得飞红,红中泛紫,紫中又泛出金酱色来,恐他恼羞变怒,连忙摇手止住众人道:“跌碎了个把盆子,什么大不了的事,你们也要这样的笑法!”众人才止住了笑。一个小大姐便来拾去碎玻璃,将地上的瓜子扫得干干净净。张书玉还在那里格格吱吱的笑个不住。刘厚卿急使个眼色,与幼惲说些闲话,天南地北的攀谈。

停了好一会,幼惲方才转过面色来。刘厚卿叫娘姨取过请客票,又拿了笔砚过来,请幼惲替他写票请客。幼惲替他写了五六张客票,请的是什么纱厂买办金咏南,轮船买办陈少东,又有什么招商局提调祝华封、电报局文案何令仪等,交与相帮发去。不多时相帮回来,说请客多到,一概就来。厚卿满心大喜,便靠在炕上,一面烧烟,一面与张书玉问答。

方幼惲此时已定了心,晓得张书玉也是金刚队中人物,便也仔细看他。只见张书玉家常穿一件湖色绉纱棉袄,妃色绉纱裤子,下穿品蓝素缎弓鞋,觉得走起路来,不甚稳当,想是装着高底的缘故;头上却是满头珠翠,灿烂有光。再打量他的眉目时,只见他浓眉大目,方面高颧,却漆黑的画着两道蛾眉,满满的搽着一面脂粉,乍看去竟是胭脂铅粉,同乌煤合成的面孔,辨不出什么妍媸;更且腰圆背厚,实大声洪,胭脂涂得血红,眉毛高高吊起,只觉得满面上杀气横飞,十分可怕,那里有什么如玉如花,分明是一副夜叉变相。方幼惲看了,想道:原来四大金刚的名气也不过如此,都是浪得虚名。怎么方才见过的陆兰芬,又相貌甚好呢?心中计算。知

厚卿所请的客人已陆续到来,大家一揖坐下,问起姓名,知是常州的富户,众人也就肃然起敬。厚卿便写起局票来,问到幼惲,晓得他上海并无相好。厚卿向幼惲道:“你此地没有熟人,就叫陆兰芬罢。”幼惲点头应允。

局票发去,客已到齐,厚卿叫起手巾,邀客入席。坐定之后,张书玉便执壶斟了一巡酒。陆兰芬却第一个来,走进房门,那几步路儿,就如春云出岫一般,被风冉冉吹将上来。走到身边,方扶着幼惲椅背款款坐下。众客多喝一声采。兰芬坐下之后,自拉胡琴,唱了一支小调。厚卿瞅着兰芬笑道:“你的胡琴有二三年不拉了,怎么今天破例起来?”兰芬一笑不语。

方幼惲见陆兰芬换了一件湖色绣花袄,下着玄色缎裙,梳妆雅淡,态度温厚,较之张书玉那种可怕的情形竟有天渊之隔;更是坐近身旁,口脂芬馥,吹气如兰;加以陆兰芬有心勾引,眉梢眼角卖弄风情,把一个未入柔乡、乍经色界的方幼惲,好似雪狮子向火──浑身融化,张大了口,急切再合不拢来。陆兰芬见他如此情形,更加合拍,便慢慢的一问一答,引起谈锋。二人只顾密切谈心起来,直至客人的局到齐,主人要搳通关,方才打断了话头。

陆兰芬却依旧坐着不去,早见兰芬的相帮拿进一搭局票。约有一二十张,来催他转局。兰芬嗔道:“啥格要紧嗄,倪还要坐歇去勒,耐回报俚转过来,嘤嘤喤喤,吵勿清爽。”相帮不敢多言。座客大家叹羡。陈少东先开口向兰芬打着强苏州白道:“阿唷!恩得来,一歇歇才舍勿脱个哉。”兰芬正色道:“陈老,倪搭耐一径客客气气,从来朆说过歇笑话格,耐勿要像煞有价事,勒浪瞎三话四。方大少还是第一转叫勒。”陈少东碰了这个顶子,不好意思起来,红了脸正待回答,厚卿急道:“兰芬说的倒是真话,方幼翁果然今朝第一次叫。少翁也不必动气,我们还是来搳拳罢!”陈少东也便趁势收科道:“我不过随口说了一句笑话,不料兰芬倒动起气来。我是本来没有动气。”兰芬见陈少东自己转弯,便也笑道:“倪是勿会动啥气格,陈老末也勿要扳倪个差头。”厚卿道:“好了好了,你们两家本来都没有动气,我来做个和事人罢!”随即取过酒壶斟了二杯,一杯递给少东,一杯递与兰芬。兰芬立起身来,笑道:“谢谢耐,勿敢当。”就接过酒杯,一饮而尽。陈少东也干了这一杯,便与厚卿搳拳。兰芬却咬着方幼惲的耳朵,悄悄问道:“耐今朝扰子刘大少末,也应该复复俚个东,停歇阿要就翻到倪搭去,请仔一台罢。”幼惲见合他吃酒,正中下怀,心中大喜,便向厚卿说了,托他代邀在座诸客,停会务必要赏光,翻台到陆兰芬家去。众人一齐应允。

只见兰芬的相帮又拿了十余张局票进来,兰芬皱着眉头对方幼惲道:“格个断命堂差末,厌烦得来!倪头脑子也痛格哉!”方幼惲道:“既是你有转局,你就去罢,只要去去就来,招呼台面就是了。”陆兰芬假意坐着尚不肯走。幼惲又连连催他,方才起身。先叫娘姨回去交代台面,却暗暗的把幼惲衣服扯了一把,口中照例说声“对勿住,停歇就请过来”的套话。出了房门,尚回头望着幼惲一笑,下楼而去。方幼惲被他这一拉,拉得心花怒开,无心饮酒。众客人同厚卿也因还有翻台,便多不肯尽量,大家随意饮了几杯,等菜将近上齐,就叫干稀饭来吃了,谢了主人,一同出门,同到四马路陆兰芬寓的洋房内来。

到得门口,方幼惲便让客人先走。厚卿大笑道:“啊唷!老兄怎的这般老实,你还没有晓得规矩么?上海堂子的规例,进门时主人在前,出门时主人方才在后。

你先走进去,不要混闹的你的怯排场。“幼惲被他排揎了这一阵,觉得不好意思,又羞又笑,方明白刚才张书玉家厚卿先走的道理。

到了楼上,兰芬尚未回来,房间台面已经预备,娘姨请进房中坐下,幼惲便向厚卿道:“此地的规矩,我是一毫不懂。你只好替我招呼招呼客人罢。”厚卿应允,便代客人写了局票,先行发去,又叫先起手巾。

不多时,兰芬已经回来,一进房门便含笑招呼,执壶斟酒,应酬得十分圆到,真是满场飞舞,八面张罗。这一台酒吃得十分酣畅,众客人尽醉方休。方幼惲被兰芬灌得沉迷不醒,睡在炕上犹如死狗一般。刘厚卿恰还清醒,见方幼惲醉到如此,料想不能回栈的了,便先自回去了。

兰芬见众人去了,时候已经不早,想把幼惲扶到床上去睡,那里叫得醒他?兰芬无奈,打发娘姨等出去,掩上房门,把炕上烟盘移去,自己也便侧身而睡;又取过一条绒毯,替幼惲盖好。幼惲直到五更方才酒醒,见兰芬睡在身旁,春色横眉,脂香扑鼻,真个是:

烟笼芍药,雨洗芙蓉。

欲知后事如何,且听下回分解。





第六回 留夜厢假装阔客 抢汇票硬捉瘟生





且说方幼惲酒醒之后,见陆兰芬睡在身旁,星眼朦胧,玉山颓倒,那一种娇媚之态,真教人心荡神飞。从来酒是色媒,不觉心旌大动,便坐起身来,想去唤他。兰芬早被惊醒,连忙也坐起来,低声问道:“耐故歇心浪那哼?刚刚叫耐勿应,倪吓得来!”幼惲见兰芬陪他坐起,睡眼含饧,桃腮微涩,低言悄语的问他,更是心中快活。便道:“我现在酒已醒了,只是口渴的狠。”兰芬忙道:“倪炖好仔开水来浪,倪去冲碗杏仁露来,耐解解酒阿好。”幼惲点头。兰芬便掀开绒毯,掠了一掠鬓发,下炕去,把莲子壶上炖现成的开水提了下来,取了一只玻璃杯,又取出一瓶杏仁露,冲入开水,对了一杯,自己放在口边尝了一尝,方走至榻床旁边,挨着幼惲肩头坐下,把玻璃杯送在幼惲口边。幼惲大醉初醒,口中奇苦,干渴非常,把那一杯杏仁茶不多几口吃个干净,就如醍糊灌顶一般。兰芬候他吃完,放下杯子,又问道:“耐阿要到床浪向去靠歇罢。”幼惲大喜,故意问道:“我睡在床上,你呢?”兰芬低头一笑,觉得有一种脉脉幽情,荡漾出来。

看官,你道陆兰芬是上海数一数二的名妓,平日间有等花了无数冤钱、近也不得一近的客人也是狠多,为什么今日见了方幼惲,就这般出奇的迁就起来?原来陆兰芬自张园见了方幼惲,听刘厚卿说他是个常州首富,便认定了他是个初出茅庐的脚色,有心要去笼络了他,敲他大注的银钱,好供自家的挥霍,所以第一台酒就留他住下。万想不到幼惲是个一钱如命的人,以致大失所望,所以后来终久弄得不欢而散。

闲话休提。且说方幼惲住在兰芬处,明日起来,止给了二十块钱的下脚。兰芬见他出手不大,不像有名富户的规模,心中未免有些不快,还只认自己骗工尚未到家,所以不肯拿出钱来,就一连几天不放幼惲回栈,把那擒纵客人的手段施展出来。这几日加倍殷勤,直把个方幼惲弄得神魂颠倒。

这一日,兰芬午后起来,坐在窗下梳头,幼惲就坐在梳头桌边呆呆的看他。兰芬梳完了头,对方幼惲道:“倪今朝要到亨达利去看点洋货,耐同仔去阿好?”幼惲此时心神已乱,不觉应允。兰芬大喜,随叫相帮去叫了一部马车来。兰芬与幼惲携手登车,径到亨达利洋行门口停车。

兰芬同着幼惲进去,先看了些表链、香水,不过二三十元;末后看了一对戒指,那戒面上镶的金刚钻竟有黄豆大小,光芒四射,要七百两银子。幼惲猛然听见,早吃了一惊。兰芬笑迷迷的把一对戒指套在手上,向方幼惲道:“方大少,耐看格对戒指那哼?”幼惲料着兰芬必要他出钱代买,心内就如十五个吊桶打水一般,七上八落的跳个不住,只好将就看了一看,胡乱称赞了两声,便想走开,被兰芬一把拉住,靠着他的肩头,附耳说道:“倪呒拨洋钱,耐替倪买仔罢。”方幼惲急得涨红了脸,答应不出来。兰芬见他面色来得诧异,便追着问道:“方大少,阿肯买拨倪介?”幼惲那里敢答应他。兰芬见此光景,不觉顿时掇转面孔,冷笑一声,便向亨达利的人说道:“物事倪先带得去,洋钱明朝送来。”洋行中人都是久仰大名,向来认得,那有什么不肯?答应了一声。陆兰芬便移步出来,也不招呼幼惲,径自上车坐下。幼惲老着面孔,只得也跨上马车。马夫问道:“还是一直回去,还是要到张园?”兰芬道:“倪勿到张园哉,一直转去罢。”马夫答应,把马车直赶回四马路来。

不消片刻,早到门前。兰芬径自下车进去。幼惲没法,也跟进去。上了楼,兰芬向方幼惲不依道:“方大少,耐是有名气格大客人啘!倪要耐买两只戒指末,一塌刮仔,不过七百两银子,也勿算啥格希奇事体。耐索性勿答应倒也罢哉,板起仔只面孔一声勿响,实梗架音,阿是有心坍坍倪格台?几百两银子格事体,耐方大少也勿造至于啘。”方幼惲被他说得满面通红,无言可答,恨不得有个地洞钻了进去,勉强说道:“并不是我不肯答应,实在我带来的银子不够数目,恐怕答应了付不出来。你休要认错了。如今我立刻写信回去,汇几千银子来替你付戒指的钱可好?”兰芬冷笑道:“谢谢耐格好心,只要少坍坍倪格台就好哉!倪穷末穷,七百两银子格事体,还出得起来里!看耐方大少自家心浪阿意得过?”

方幼惲被他逼得愈加局促,只得立刻要了纸笔,写封急信给他家中的帐房,叫他立刻汇二千银子。写完,叫相帮赶紧去送,信面上限着日期。兰芬方才有点笑容,道:“勿然是倪也无啥希奇。不过俚笃说起来,倒说耐方大少买一对戒指才舍勿得,勿要说倪坍勿落格个台,就是耐方大少面浪末,也无啥好看啘,方大少阿对?”幼惲刚刚被他发作了一场,那里还敢驳回,只好连连答应。

自此兰芬相待就冷落了许多,却也还敷衍着他。刘厚卿也来看过幼惲几次,只是幼惲已经迷惑,也不回栈,终日在兰芬那里,昏昏沉沉的过了几日。

那日幼惲还未起身,当差的拿了一封常州来信,并同着一个后马路厚大钱庄的伙计寻到兰芬来,原来是常州汇来的银子,要幼惲亲笔写个收条。娘姨叫醒了幼惲。兰芬正在好睡,便也惊醒。幼惲连忙起来,走到外间。家人送上来信,那钱庄伙计拿出一张即期本庄的票子来,共是二千规银。幼惲看完了信,无甚话说,便进房寻着笔砚,写了一个收条给那钱庄伙计,接了自去。进来再看兰芬,已披着衣服坐在床上,便问幼惲道:“啥格事体,实梗贼形怪气?”幼惲道:“是我家里汇来的银子。”兰芬又问银子放在何处?幼惲笑道:“不过是一张汇票,凭着票子去拿洋钱,那里来的现银。”兰芬道:“汇票是啥个样式介,拨倪看看哩!”幼惲正要炫耀于他,便在袋中取出,递与兰芬。兰芬看了半晌,半真半假的将一张银票向自家衣袋一塞,向幼惲道:“方大少,耐银子未汇得来哉,倪格戒指铜钱好去还脱仔哉啘。”幼惲见陆兰芬将一张银票轻轻的袋了进去,出其不意,急得满头是汗,急忙赶过来夺时,已经不及,满心烦恼,又不好意思认真,只得勉强按住心神,向兰芬道:“不要取笑,你把票子还了我,那戒指的钱我替你付就是了。”兰芬见他急得不可开交,嗤的一笑道:“阿唷!耐放得定点嗫,吓得来格付神气,阿要难为情!”又伸出手来把幼惲拉着,坐在床上,轻轻把手去摩他的心口,道:“阿唷!急得来!故歇心口里向还勒浪跳,阿要作孽?”这几句不痛不痒的话,说得方幼惲满面羞惭,满心难过,又不好认真发作,那一时的可笑可怜的情状,竟难以言语形容。

陆兰芬料他发作不出,心中暗自好笑,一面还在调侃他道:“方大少,刚刚阿是吓煞哉?头浪出仔几化格汗,倒拿倪别生能一跳,现在阿好仔点哉?”方幼惲被兰芬颠来倒去,就如三两岁的小孩一般玩之股掌,哭又哭不得,笑又笑不出来,赌气立起身来,一言不发,便要走出房去,早被一个娘姨劈胸搪住道:“方大少,到啥场化去?”幼惲不语,想要夺路走出,娘姨那里肯放?正在扭结固结之际,兰芬已着好衣服,赶下床来,一把衣角拉住,口中说道:“耐格人阿要无趣!说说笑话末,就说勿连牵哉,可煞作怪。”方幼惲方才本是满心愤恨,想要奔回栈去与刘厚卿商量一个主意,挖他的出来,所以娘姨留他,毫不瞻顾。不知怎么被陆兰芬拉了一把,又轻描淡写的说了几句,心头那一把三千丈高的无名业火也不知消到那里去了,身体便不觉软绵绵的,回过身来,被兰芬推他坐在椅上,反埋怨他道:“耐末总是实梗性急。倪又勿做啥强盗,阿好抢耐格铜钱,晏歇点倪自然要还耐格。耐放心末哉,勿要急坏了自家格身体,倒勿止格点铜钱。”幼惲听兰芬说仍旧还他,心中大喜,却勉强遮饰道:“我是偶然想起一件要事,所以要紧回栈,并不是为着票子。你既不叫我走,我就不走也好。”兰芬又去温存了一番。

幼惲虽然迷惑,却究竟后天的“色”字,抵不过先天的“财”字,到底二千银子的事情不是轻易,总有些失神落智的。兰芬口中虽说取笑,却只是哄和着他,不肯真拿出来还他。幼惲又不便只管催逼,只急得团团走转,坐立不宁。兰芬看破他的神气,只当并无此事一般。

幼惲勉强在兰芬处又住了一夜,却通晚不曾合眼,到了天明之后才朦胧睡去。八点余钟便又惊醒,就坐起身来兰芬问道:“要紧起来到啥场化去?”幼惲道:“我有正事要回栈房去一趟,下午就来的。”兰芬拉着他的手不放,道:“耐去仔就要来格口虐。”幼惲道:“自然就来。”兰芬道:“耐格人有点鬼头鬼脑,倪倒勿相信耐格闲话。”就在幼惲左手上勒下一个戒指来带在自家手上道:“耐去罢。耐要戒指末,自家来拿。”原来幼惲这个戒指,是他的母舅徐观察出使美国带来送他的,约来也值一千多块洋钱,现在又被兰芬探去,更加心痛,只得忍住了,穿衣起身。兰芬暗笑,也不留他,任幼惲一径回栈去了。

只说幼惲回至栈中,满心焦燥,便一直走到刘厚卿房里来。谁知锁着房门,人已不知何处去了。问他的家人,说是好几日没有回来。幼惲想他一定住在张书玉处,便也不回房,寻到新清和来。

走进客堂,还是静悄悄的;及至走上楼梯,并不见一个娘姨、大姐,张书玉的房门却是虚掩,一半开着。就蹑足进房,只见垂着湖色绉纱帐子,衣架上挂着厚卿常穿的一件漳缎马褂,知是刘厚卿在此。榻上睡着一个小大姐,听得幼惲脚步之声,方才惊醒,连忙坐起,擦着两眼,看不明白,只道是厚卿已经起来,口中说道:“刘大少,啥勿困歇起来介?”方幼惲道:“我不是刘大少,是来看刘大少的,快去请他起来。”小大姐又仔细看了一看,方知认错了人,忙笑道:“阿呀!看错仔眼睛哉,方大少啥能格早介?”一面下了榻床去揭开帐子,低低的叫了两声,把厚卿、书玉一齐惊醒,忙问何人。小大姐道:“方大少来哉,说请刘大少快早点起来,有闲话说勒。”

刘厚卿听幼惲一早寻到此间,谅必有甚要事,连忙起来穿好衣服,跨下床来,看幼惲的面孔笑道:“前两日我到兰芬处,看你们二人就如蛤蚧一般连得紧紧的,一刻也分不开来,怎么今日就这样的早起,可是当差不合,被他赶了出来么?”幼惲皱着眉头摇手道:“我正为一件事心上十分懊恼,要来寻你商量,你怎么开口就是取笑!”厚卿见他面色仓皇,也就不好再去笑他,只问道:“你有什么事情,清早赶到这里寻我?”幼惲恐被张书玉听见不好意思,移过椅子,附着厚卿的耳朵,低低的把兰芬抢去汇票、戒指的情节说了一遍。“所以来寻你想个法儿去问他要回,可有什么主意?”

厚卿听了不住的摇头,道:“这是你自家不好。汇票、戒指怎的落在他的手中?我看起来,要去问他拿回,只怕是办不到的了。”幼惲再三要他设法,厚卿道:“我只好替你到兰芬那里去问他一声,探探他的口气,至于一定要他拿出来还你,也是拿把不定的。”幼惲听了,略略放心。

厚卿问道:“你一早起来只怕没有吃点心,就在这里吃罢。”厚卿就叫去叫了两碗鸡丝面来,两人吃毕。张书玉蓬着头,正要下妆梳洗。幼惲看他剩粉残脂,熠然满面,那隔夜画眉的轻煤都一条一条、横七竖八的印在面上,比前更加可怕,暗想:这样一付面貌,怎也居然列在金刚之内?上海地方真是无奇不有的了。略坐一坐,便催厚卿前去。厚卿叫方幼惲在张书玉处宽坐一会等他回来,匆匆的穿了马褂出门而去。见了兰芬,说了一回闲话,便提起幼惲的汇票来。

兰芬告诉他道:“刘大少勿要说起。倪末当俚是个户头客人,勿壳张格位方大少着实有点踱头踱脑。倪前日仔到亨达利去买仔两只戒指,为仔倪自家呒拨洋钱,问仔俚一声,俚就跷起仔格面孔,一理勿理,难末倪也有点光火哉,埋怨仔俚两声。昨日仔俚屋里向汇仔洋钱来哉,倪为仔朆看见过歇汇票,问俚要得来看看,说仔一句笑话,俚加二勿对哉,面孔末涨得通红,头浪向汗末出仔几化,极得来要死要活。倪并勿是要抢俚格汇票嗄,为仔俚做出格副极形,有心叫俚难过难过。刘大少去耐想嗫,倪为仔呒拨洋钱问俚一声,就是耐刘大少末,也勿好意思勿答应倪啘。俚倒直头做得出格,阿要讨气!今朝对勿住刘大少,到倪搭来,托耐刘大少带声信拨俚:倪总勿见得要抢仔俚洋钱格,叫俚尽管放心。倪归搭呒拨啥格老虎勒浪,勿会吃脱仔俚格,叫俚自家只顾来拿末哉。”

厚卿尚未开谈,先被陆兰芬一大片话兜头罩住,竟是无可如何,不便再说,只得自家做个收场道:“他倒并不是不放心,也没有托我问你讨取,我不过自己问问罢了。”说着,更不久坐,回到新清和,见了幼惲,慌问事体如何,厚卿摇头道:“这事竟办不到。据我看来,你竟认个晦气,丢掉了一笔钱也就罢了,若一定要问他讨取,总要你仍旧回去,好好的哄着他,或者可以拿得回来。我是旁人,不好出头多事。”正是:

误入销金之窟,荡子堪怜;重寻照夜之屏,莺花无恙。

要知方幼惲到底如何,下回交代。





第七回 车走雷声香尘一瞬 酒酣奇气名士高吟





且说方幼惲听了厚卿言语,着急道:“我的口才本不如你,上海又是初到,你既不肯为力,我是更没有指望的了。”厚卿道:“并不是我不肯出力,实在现在上海堂子中的倌人十分歪撇,非但敲竹杠、砍斧头,不肯放松一点,你就是花了整千整万的银钱在他身上,不说一个好字。何况你的银票已经到了他的手中,要再去挖他的出来,是休想的了。不如歇了这个念头罢!”幼惲更加着急,厚卿道:“你着急也无用,还是慢慢的想法。”

忽听张书玉冷笑了一声,向厚卿说道:“倪堂子里向格人末才是勿好格,唔笃客人用脱仔洋钱也勿犯着,像煞耐刘大少勒倪面上,勿知用脱仔几化洋钱,耐倒自家摸摸良心,倪阿曾敲过歇耐啥格竹杠?”厚卿道:“我是说的别人,没有说你。

你既没有敲过我的竹杠,为什么要你这样多心?“书玉愈加不依道:”实梗说起来末,倪直头敲仔耐格竹杠哉啘,阿要热昏!“厚卿也打着苏白回答他道:”倪是昨日仔夜里向发仔一个大昏,直到今朝故歇辰光还勿曾转来格勒。“书玉听得厚卿取笑,便急了,连忙瞪他一眼,赶过来要拧厚卿的嘴,道:”你阿要瞎三话四哉,倪要拨生活耐吃格嗫!“厚卿哈哈的笑道:”我的生活,你昨天还没有晓得么?“书玉更加不好意思,红着脸,狠狠的把手在厚卿大腿上拧了两把,拧得厚卿叫声”阿唷坏“,直立起来。幼惲也觉好笑。书玉却才住手不拧,走了开去,口中还自咕噜着,自去梳头。

幼惲终是无精打采的纳闷。厚卿道:“你心中不快,倒要出去散散,我们还是在此吃过了饭,到张园去走走,还可解解你的气闷。”幼惲也无可不可的。

厚卿看表时,已是十二点三刻,便开一桌菜单,叫相帮到雅叙园去吃一样糟溜鱼片,一样溜鸡丁,一样炸丸子,一样粉蒸肉。并火腿蛤蜊汤,要两壶酒。不多一刻,菜已送来,便与幼惲对坐小酌。张书玉梳完了头,也来斟了两杯酒,坐在旁边。

幼惲叫他同坐,书玉推辞道:“倪吃饭还有一歇勒,方大少先请末哉。”幼惲本来量浅,又是喝的闷酒,不多几杯便觉有些醉意。厚卿见他面上已有酒意,也不劝他,便叫盛饭上来。两人吃完,又停一会,约有三点余钟。叫相帮去叫马车,因书玉也要同去,多叫了一部。

当下厚卿、幼惲同车,书玉独坐一车,向张园而来。进了园门,马夫照例加紧一鞭,如飞疾驶,至大洋房门口停下。厚卿、幼惲同下车来,书玉还未下车,只听马蹄声响,一部亨斯美自拉缰马车,风一般的跑来,也到安垲第停下。眼光一瞥,早跳下一个美少年,携着一个绝色倌人。那少年身穿湖色熟罗十行绵襔,外罩玄色漳缎马褂,生得细腰窄背,白面朱唇,气概非常,丰仪出众,眉目之间别有一种英爽之气,咄咄逼人。那倌人生得秋水为神,琼瑶作骨。凌波微步,何殊洛浦惊鸿;袅娜依人,不数汉家飞燕。姿容妍媚,举止大方,穿一件白缎子绣花夹袄,头上不多几件钗环。只在厚卿、幼惲眼前一闪,便先进安垲第去了。幼惲、厚卿觉得眼中从未见过这般人物,暗暗叹羡。张书玉更看得呆在一旁,直至厚卿同幼惲进去一会,回头不见书玉,厚卿复身出来寻他,方见书玉立在门旁,好似想着什么心事一般。

厚卿问他为什么还不进去,可是等什么人?书玉才被他提醒,忙道:“倪勿是等啥倌人,像煞唔笃还朆进去,所以勒浪看看。”遮掩过了。随同着厚卿走进大洋房,拣了一张桌子,泡茶坐下。

幼惲却想着刚刚马车上坐的美少年十分面熟,满腹想不出这个人来,便又留心看他,却却回过头来,见他同着那绝色倌人同坐在斜对一张桌上,真是和璧隋珠,珊瑚玉树,交枝合璞,掩映生辉。

正在细细打量,只见又走进一个倌人,朝着幼惲略略点了点头,却叫了厚卿一声。原来就是陆兰芬,竟不坐下,一直走了过去,忽回头见了那少年,兰芬登时满面堆欢,叫了一声“二少”。那少年也含笑招呼,招他坐下。兰芬便坐在那少年身旁一张椅上,那绝色倌人也招呼了兰芬一声,兰芬竟和那少年密切长谈起来。方幼惲这一气非同小可,又不好发作出来,眼睁睁的看着他。不到半点钟时,只见那少年立起身来,同着兰芬三人从右边转出,一面谈笑,一面慢慢的缓步往弹子房一带去了。

兰芬临去,头也不回一回,直把一个方幼惲气得口呆目瞪,无可如何。刘厚卿却被别个朋友邀在隔壁一张桌上谈心,不曾理会。张书玉也闲步往弹子房去了。只剩幼惲一人,无人可说,就如泥神土佛一般坐着。好容易刘厚卿走了回来,不见了张书玉,忙问书玉他们那里去了!幼惲回答不知。厚卿道:“天色已晚,是回去的时候了,书玉怎不见来?”便惠了茶钞,同幼惲出来,寻到老洋房照相处,都不见书玉的踪影。厚卿说声“奇怪”,回身要到弹子房去寻他。刚走到门口,劈面遇见方才少年同着兰芬出来。兰芬似欲招呼,早已擦肩过去。随后张书玉跟着出来,见了厚卿才立住了脚。厚卿对书玉道:“时候已经不早,快些回去罢。”张书玉一言不发,似乎有些不耐烦的意思,同厚卿走到前边。马车早已等了多时,三人登车回去。

兜了几个圈子,回到新清和来,相帮送上两张请客票头,一张是金咏南请到迎春坊花琴舫家,一张是祝华封请到兆贵里张月红家。金咏南的是七点钟,祝华封的是八点钟。厚卿便向幼惲道:“这两个既来请我,必定也要请你,想是票头发到陆兰芬那里去了,你就少停同我一淘去可好?”幼惲想来不错,便无别话。

厚卿因在嫖赌场中久了,已有了烟瘾,躺下炕去吃烟。幼惲和他对面躺着。张书玉却只是无情无绪,不来应酬。厚卿过好了烟瘾,又坐了一会,早有金咏南的催请票到来,便同着幼惲一同赴席。

到了花琴舫家,见客人已经到齐,金咏南连忙催摆台面。厚卿举眼看时,却只有一半认得,幼惲更只认得陈少东一人,不免一一寒温,请教名姓。金咏南便问:“厚卿、幼惲,你们叫什么人?”厚卿道:“我坐定是张书玉了,幼惲可是仍叫陆兰芬?”幼惲满肚子没得好气,连忙朝他摇头。厚卿向他使个眼色,幼惲不解其故,便不开口,也叫了陆兰芬。随着金咏南去发局票,厚卿乘空附着幼惲耳朵说道:“你在上海又没有做第二个倌人,况且兰芬与你又没翻面,场面上还是好好的,何苦再去叫个陌陌生生的人呢?”幼惲正待回答,那边主人已在邀客入席,便打断了话头。

坐定之后,客人的局已经到齐,只有张书玉、陆兰芬两人还不见来,叫人去催催,说是要转过来。幼惲也还罢了,厚卿却满心不自在起来。直等客人的局已经去了一半,方见陆兰芬进来,淡淡的招呼一声,便默然坐下,一言不发。幼惲也低着头不开口。大家看着诧异,晓得一定有些缘故,却见二人面色不好,倒不便去问他。

接着张书玉也来了,厚卿问他那里的转局,直到台面要散快才来?书玉冷笑道:“倪格生意就是勿好末,也总有几户客人,勿见得就做仔耐刘大少一干仔,问得阿要稀奇!”厚卿突然被张书玉顶了这几句,气得他面皮紫涨,竟说不出什么话来。

金咏南见此光景,虽明知是书玉的不好,却怕刘厚卿性子暴躁,张书玉的脾气又不是肯省事的人,生恐闹出事来,连忙分解道:“厚翁不要动气。书玉向来也不是这个样儿,想是今天堂唱多了些,未免有点不自在。你是有过相好的客人,总得要比别人体谅他些才好。”厚卿因主人极力劝说,不便发作,只得忍住。张书玉也知自己说话孟浪了些,只因看着刘厚卿是个刮皮客人,不甚放在心上,此刻见厚卿不语,自然不再开口,却止略坐一会,同着陆兰芬起身而去。厚卿、幼惲恨在心头,只得谢了主人,要到兆贵里去。金咏南知他二人另有应酬,便不留他。

到得张月红家,祝毕封因客齐久等,先已入席,见厚卿同幼惲来了,深致不安,便请一同坐下。随问厚卿、幼惲可是仍叫陆兰芬同张书玉。厚卿赌气换叫了一个公阳里的林佩珠,又替幼惲代叫了一个西鼎丰花宝玉。局票去不多时,两人先后来了。

席中大家欢呼畅饮,只有幼惲心中纳闷,没甚精神,并连叫来的局也不去理会。

却听得对过房间也有客人在内请客,甚是热闹,但并不搳拳,也不听见倌人唱曲,只在那里高谈阔论。有一个人的声音甚是熟落,只听得他抗声说道:“你道现在上海的新党,日本的留学生,一个个都是有志之士么?这是认得大错了。他们那班人,开口奴隶,闭口革命,实在他的本意是求为奴隶而不可得,又没有那夤缘钻刺的本钱,所以就把这一班奴隶当作不共戴天的仇人一般,今日骂,明日骂,指望要骂得他回心转意,去招致他们一班新党入幕当差,慢慢的得起法来,借此好脱去这一层穷骨。那知朝中这班大老,耳朵是聋的,眼睛是瞎的,心地是面糊蒙着的,面孔是牛皮做成的,就是拍着他的脸痛骂他一场,他也只是不见不闻,我行我素。

所谓‘笑骂由他笑骂,奴隶我自为之’,凭你怎样的大声疾呼,那里叫他得醒?也有万一碰着运气,逢时得济,遇着了贤明的督抚大臣,聘请他做个顾问官,居然的当差入幕起来。无夸这班新党中人,却又是一得到了优差优馆,便把从前革命自由的宗旨、强种流血的心肠,一齐丢入东洋大海,一个个仍旧改成奴隶性质,天天去奴颜婢膝起来。你道可笑不可笑?他们现在的宗旨,是开口闭口总说满人不好,非我族类,其心必异,固然不错。要晓得,满洲人虽是蒙古入关,究竟还是我们亚洲的同种。所以欲分满汉,先分中西。这班人就该帮扶同种,摈斥外人,方不背同类相扶的主义。不料他们非但不能如此,反去倚仗着外国人的势力,拼命的欺负同种的中国人。总之,这班人本是寒士出身,穷得淌屎,却又不中举人,不中进士,无计可施,以致变成了这等一个气质。说起来也甚可怜,那里有什么爱国的热诚,合群的团体?纵使有几个英雄杰士,伤心大局,蒿目时艰,要想力挽狂澜,主持全局,却又是手无寸柄,说也枉然。“说到这里,便长叹了一声。又有一人击节叹赏道:”你这话实在说得痛切!新党中间未尝没有通人志士,却被这班无耻小人借着新党的名目,到处招摇撞骗,无所不为,弄得坏的带累了好的,施展不来,真是可恨!“

听得方幼惲暗暗不住的点头。

原来方幼惲虽是个贵介子弟出身,从小十分聪颖,只是自恃天分,就不肯在书史上用心,只弄些雪月风花的学问。平时也看过几部新书,晓得些中外的大势,向来以新党自居。今天听见这一席议论,却是闻所未闻,不觉爽然自失。

又听见那人高吟道:

华夷相混合,宇宙一膻腥。

接着说道:“这是《花月痕》中韦痴珠的牢骚气派,我年纪虽不逮痴珠,然而天壤茫茫,置身荆棘,其遇合也就相等的了。”又听一人说道:“你是喝了几杯酒,故态复作,何物狂奴,悲歌击节。”却不听见那人回答,幼惲便静静的听他。停了一会,又听见高吟道:

回首当年万事休,元龙豪气尽销磨。

关山跃马秋横塞,风雨闻鸡夜渡河。

前路苍茫愁日暮,唾壶击缺任悲歌。

何须更忆繁华梦,搔首沉吟唤奈何。

念到末句,那声音就低了好些,只听一人大叫道:“好诗,好诗!沉郁苍凉,读之令人有身世悲凉之感,我当浮一大白,请窥全豹。便听得又吟道:

一夜西风动客愁,只余身世寄扁舟。

千秋事业怜青史,一代功名负黑头。

蜀国相如今贳酒,天涯王粲莫登楼。

匆匆归去真堪笑,惆怅题诗记玉钩。

梦醒扬州一惘然,可怜往事竟成烟。

桓温种柳只流涕,殷浩书空欲问天。

剩有闲情随逝水,拼将绮思逐华年。

输他绝塞从军客,万里秋风早着鞭。

飘泊谁怜屋上鸟,江湖落拓竟何如。

荒唐槐国三年梦,慷慨苏秦十上书。

纵有文章惊四海,更无涕泪哭穷途。

请缨投笔男儿事,夜半床头啸鹿庐。

幼惲听了,赞赏非常,此时再忍不住,便问娘姨:“对过房间是何人请客?”

娘姨道:“听见说是一格姓章格常熟客人。”幼惲便想私去窥探窥探他,到底是个何等样人,居然这样的见识高超,才华卓荦,因立起来向外便走。走到对房门口,隐在门帘外边,向房里看去,早吃了一惊。原来那向外坐着的主人,就是方才在张园相遇不知姓名的人,心中想道:果然外貌挺秀,内才也自不差。忽听得旁座一人赞道:“秋翁佳作,气韵沉雄,真与杜甫律诗颉颃千古。”正是:斋

伤心身世,悲闻宋玉之辞;极目河山,不断新亭之相。

要知究竟何人,下回交代。





第八回 章秋谷意气结新知 方幼惲平康逢旧识





却说方幼惲正在偷看那对过房间的客人,心中转念甚是面善,忽听得那人称呼他“秋翁”,方才兜的想起这人的姓名,不觉大悟,自己笑道:“我的记性怎的坏到这步田地,隔不多时,竟是想他不起,可不是笑话么?”连忙掀起门帘,进去招呼。

看官,你道那不知姓名的少年是谁?原来就是那风流才子、诗酒名家的章秋谷。

自从打发金月兰动身之后,在家中住得不多几日,总觉郁郁寡欢,加以秋谷才华绝世,丰采惊人,论文则援笔万言,论武则上马杀贼。惊心烽火,聊为梁父之吟;举目河山,尽有唐衢之恸。一身傲骨,四海无家,钟期之遇难逢,狂白之金欲尽,不免就牢蚤郁勃,变成个使酒的灌夫,骂人的刘四,竟有些信陵君醇酒妇人的气象起来。便觉在家无趣,重为沪上之游,也住在四马路吉升栈。到此虽不多几日,却着实结识了几个有名的人,一个叫做辛修甫,是个内阁中书,学问极其渊博。秋谷闻名往访,辛修甫与他谈得十分投合,果然名下无虚,一见如故。一个叫做王小屏,是个报馆的主笔,深通时务,兼擅西文。他从前看过秋谷一篇论说,甚是佩服;此次晓得秋谷来申,急急的到栈相访,成了倾盖之交。还有两个,一个叫葛怀民,是个举人;一个是大挑知县,叫吕仰正,却是辛修甫介绍与秋谷相知的。这几个人都是金石论心,芝兰合臭,俯视山海,高见风云,绝无时下少年酒食征逐的恶习。

秋谷自到上海,访他去年一个旧好倌人,名叫陈文仙,年止十七,花妍柳媚,玉润珠温。去年秋谷做他,甚是要好。这陈文仙气息沉静,居然像个闺阁大家,并无红倌人的一种时髦气派,今年从西安坊调到兆贵里来。秋谷除了访友,便到陈文仙处闲坐。文仙也从不叫他吃酒碰和,转是秋谷过意不去,替他绷绷场面。这一日,正是秋谷的主人,请的就是辛修甫等数人,并两个同栈居住的同乡,隔夜已经照会客人点好了菜。秋谷恰午后无事,便到陈文仙处,约他同坐马车到张园吃茶;又遇见了陆兰芬,谈了一会。秋谷因坐不住,便到弹子房去合人打了两盘弹子,方才同了兰芬、文仙出来。天色已是不早,因兰芬苦邀秋谷同文仙去坐坐,便又到兰芬处坐了一会。看看已有七点多钟,兰芬知有台面,不好留他,只叮嘱秋谷常来走走。

原来秋谷与兰芬只是淡淡的交情,并没有什么相好,只是兰芬向来敬重秋谷,所以见了面,不觉十分亲热,以致在张园相遇,引起方幼惲的气来。

只说秋谷同文仙回到院中,辛修甫已先来了,余客也便络绎而来。秋谷做了主人,殷勤对釂无不尽量。到得酒酣耳热之际,辛修甫偶然说起新党悖谬之处。从来酒在肚里,事在心头,早把章秋谷一肚皮的牢骚提了上来,便高谈阔论了一大篇,又痛饮了几大杯酒,方才吟出那四首感怀的七律来。座客一齐称叹。

秋谷连饮了数杯急酒,微觉有了醉意,忽见门帘一起,又走进一个客人高叫秋谷道:“老世兄,幸会幸会!你发得好议论,吟得好诗啊!”秋谷醉眼朦胧,急切认不出他是谁,立起来细看,方认得是小时同学的方幼惲,便笑道:“我的眼钝,几乎认不出来,幼惲兄好眼力。”方幼惲大笑道:“岂敢!你在张园和陆兰芬谈心的时候,我早就看见你了,觉得面熟,又一肚皮想不起你来。刚才若非有人叫了你一声‘秋翁’,只怕到明年也想不起的了。”秋谷也大笑,慌忙作揖,又请幼惲与众客一一相见,道:“不嫌残席,就请一同坐下,叙叙可好?”幼惲道:“我是一个姓祝的朋友请我在张月红处吃酒,恰恰遇见了你,岂非奇逢?你这边我不能久坐,还要过去应酬。你住在什么栈房,我明早过去奉看就是了。”秋谷连说:“不敢奉屈,现在暂寓吉升栈。”幼惲大喜道:“我也是寓吉升栈。既是同栈,更好相叙。

少停回栈,我们再谈罢。“秋谷留他不住。

幼惲仍旧过来,见花宝玉、林佩珠一齐走了,台面将散,刘厚卿看见嚷道:“你这半天走到那里去了?马褂也没有穿。”幼惲对他说了缘故,便同着厚卿谢了主人先走。两人又到花宝玉、林佩珠家去打了两个茶围。林佩珠出局,没有回来,花宝玉已经回院,应酬得甚是周到。幼惲看他相貌,眉目清扬,腰肢柔细,也算得花丛中一个出色人材。

幼惲为着自己心中不快,也无心久坐,拉着刘厚卿出来,路上埋怨他道:“我朝你摇手不叫陆兰芬,你偏要我仍旧叫他。你看他刚才的形状,口也不开,立起身来就往外走,惹气不惹气?”厚卿被他埋怨,倒也无言可答。幼惲又道:“我以前的银票、戒指被他抢去,不上紧去追他,为的是有过相好,不好意思。不料他钱物到手,顿时翻转面来。他既无情,我亦无义,如今我们就商量一个主意,去问他硬讨可好?”厚卿笑道:“这是你说痴话,他东西已经入手,你就去问他硬讨,他可肯拿出来么?”幼惲愈觉气忿道:“难道他不肯拿出来就罢了不成?我一个世家子弟,白白的受了他一场糟塌,还送了一大注钱,竟连个妓女都弄不过,这不是笑话么?”厚卿大笑道:“老弟,怎么看着你这样一个人,竟是一点不通世故。你的银票、戒指被他抢去,可有什么凭据么?这是打不得官司、告不得状的事,可有什么法儿!就是打了官司,那堂上的官儿也要审情度理。你们自然交情深厚,那银票、戒指才得到他的手中,现在你要硬追回来,难道好当他贼赃追取么?这样的事情都要经官,他吃了皇上的俸禄,那里管得了这些闲事!况且宦家子弟饮酒宿娼,自己先有一层不合,怎能再去告他?这里又是租界,不能违背章程,不比内地各处的娼寮,若真个十分可恶,便好打掉他的房间,叫他吃了惊吓。上海地方,是打闹娼家先就犯了捕房的规矩,就要拉到捕房里去。我们都是面子上人,可坍得起这个台么?

你想这事有甚法儿?“

幼惲先前怒气填胸,恨不得立刻把陆兰芬的房间打毁,方出这一口恶气,被厚卿一番话,说得顿口无言。想来想去,呆了多时,觉得这话果然不错,叹一口气道:“果然如此,我也只好认个晦气,只算自家病了一场,用几个买命的银钱罢了。

但是那一张票子被他抢去还是小事,那一个戒指是母舅徐观察美国回来送给我戴的。

我戴在手上,家父还时常叫我留心,千万不可失去。现在回去,倘然为不见了戒指,查问起来,可不是一件难事?你总要去想个妙法,将那戒指代我收回,感激非浅,那银票就送了他也罢。“厚卿摇头道:”我前天已经碰了他一个钉子,现在就去问他,想来万万无用。你不晓得我在他那里,被他一冷一热的话说得十分难过,我是再不去寻第二个钉子碰了。“幼惲见厚卿不肯答应,便急了道:”不论有用无用,托你务必要去一趟。“我本来也不认得什么陆兰芬、林黛玉,原是你的来头,难道我们的交情,这点点小事多应承不来么?”说罢,又连连作揖。厚卿无奈,应允道:“我去是去,然而收得回收不回,我是不管的,我总尽心竭力替你去干就是了。

“幼惲连连称谢,便催他:”此刻就去,我在栈房候你的回信可好?“厚卿知道推却不脱,只得同幼惲分路,幼惲自回栈去。

厚卿到兰芬院中,寻见了陆兰芬,婉婉转转的将来意说了一遍,又道:“幼惲现在的意思,情愿将二千银子不要,只望收回戒指,你的意思如何?若肯还他,便交给我带去也好?”兰芬听了冷笑道:“耐刘大少来说仔,论理是勿好勿依,不过俚格人忒嫌来得希奇。倪叫俚自家来拿,倪自然要拨俚格,啥格人影子也勿见,像煞倪是啥格强盗。倪倒也有点脾气格,俚耐自家勿来末,倪直头抢定还仔俚格哉。”

厚卿陪笑劝解道:“你也不要动气,他的心上并不怪你,你把戒指给我带去还他,我随后再叫他来陪你的礼可好?”兰芬又冷笑道:“戒指是勿错,倪探子俚一只勒浪,也勿知拨倪放到仔陆里去哉,现在一时无寻处。俚一定要倪还末,倪只好赔仔俚一只末哉。”一面说,一面伸出纤手来,两手共带着十余只金刚钻、红蓝宝石的戒指,耀眼争光,向刘厚卿道:“刘大少,耐拣仔一只罢。”厚卿见他伸出手来,吃了一惊,只见五光十色,光怪陆离,不觉目定口呆,停了一会,方才说道:“既不是他的原物,我怎好胡乱拿去?我回去对他说明,一定叫他自己来拿,好在我是旁人,也不能管你们的事。”兰芬道:“格末谢谢耐,对俚说声,叫俚明朝就来,倪还有闲话说勒。”

厚卿应了,自回吉升栈来,见了方幼惲,把手一拍道:“何如?我说是万万无用的。”幼惲忙问何如。厚卿把兰芬的话向幼惲说了,幼惲气得发昏,长叹一声,默然不语。厚卿也因张书玉忽然改了面孔,不知是为什么,也是闷闷不乐。

过了一夜,幼惲去看章秋谷。原来他住在纳字官房。相见之后,略叙几句寒温,秋谷见他似有不悦之意,便问他道:“幼惲兄,为着什么事情神气这般萧索?”幼惲意欲相告,又觉难以为情,只推头痛并没有什么心事,秋谷道:“我们两人道义相交,幼同笔砚,如有为难之事,尽可同我商量,或者是有可以为力之处,亦未可知。”‘幼惲听了,沉吟不语,欲言不言。秋谷再三问他,幼惲仍是不肯实说。秋谷心中不悦,拂袖而起道:“我再三请问你有何心事,原是一片热肠,想要替你排解,怎么你把我看作外人,半吞半吐的做那妇人女子的样儿,究竟是何意见。”幼惲见秋谷已有怒意,只得把初做兰芬甚是要好,后来为着一对戒指顿然翻面,抢去银票、戒指的前后情形细细说明,又道:“并不是把你当作外人,不肯相告,实是我在张园见兰芬待你甚是亲近,只道你和他也有什么瓜葛,所以不便说明。”秋谷道:“我与兰芬向来认得,却不曾有过交情,并连局也不曾过一个,这有什么嫌疑?”

幼惲乘便要秋谷去替他要回银物,又道:“昨日的光景,兰芬待你甚好,你如肯替我收回,料想兰芬也不好意思不听。”秋谷道:“我生平为人最爱管人闲事,时常骂那班坐观成败的鄙夫都是凉血动物,自家岂肯遇事退避,畏缩不前?但是天下无论什么事情,都有一个公理,不能专听一人的私见。我也要审情度理,方可替你出头。或者没有什么别故,自然可以替你收回。兰芬也不是那种专爱银钱的人,或是你们有了相好,其中另有别情,那我就不能过问了。”幼惲力辨并无别情。

秋谷听了心中疑惑,想起兰芬为人尚好,向来待客还算略有良心,何至如此?

想了一会,又问幼惲道:“他可晓得你有钱?”幼惲道:“我虽没有同他说过,却是第一天在张园见面的时候,刘厚卿朝他说的。”秋谷猛然拍手笑道:“是了,是了。”便问幼惲在兰芬身上除了那二千两钱之外,一共花过多少银钱,可曾替他办过什么衣裳首饰。幼惲道:“通共算来,那二千两票银不算外,只吃了三台酒,现还没有付钱,就是现付了二十块钱的下脚,也没有替他办甚衣饰,他又并没有向我开口,我也乐得省几个钱。”秋谷不待说完,哈哈大笑道:“算了罢,我的老哥!

你要省钱是要住在家里,为什么要走到上海这花钱的地方来?既然到了此间,上了场面,可就讲不起省钱的话了。你且坐着不要性急慌忙,听我替你讲这道理。“秋谷言无数句,说出一番道理来。幼惲听了,方才如梦初醒,连连点首。正是:

说破高唐之攀,顽石点头;忏除丝竹之情,现身说法。

未知章秋谷所说云何,请听下回交代。





第九回 章秋谷苦口劝迷途 陆兰芳惊心怜薄命





且说秋谷向幼惲道:“你想那陆兰芬是四大金刚中数一数二有名的人物,平时何等风头,真有好些大人先生的客人,花了整千整万的银钱近不到他的身体。你是个初到上海的人,向来又没有什么名气,通共在张园见过一面,摆了一台酒,却轻轻易易的留你住下,有了交情,就是平常的倌人也不到如此迁就。他是贪图你的什么?为着晓得你是有名富户,想要弄你一大注钱,先给你些甜头,不怕你不死心塌地的报效。这是他们擒拿客人的第一等利害工夫。你是个富家子弟,又没有到过此间,那里懂得这些诀窍,以为第一台酒就留你住了,又是个有名妓女,自然荣幸非常。殊不知既已入了他的圈套,便如飞蛾投火,高鸟惊弓,随你一等吝啬的人,也不得不倾筐倒箧。况且他既破格待你,你更该破格待他,非但应该私下送他些值钱的衣饰,或是多送他几百洋钱,替他排排场面,就是那下脚的洋钱也至少要再加一倍,难道他有名的第一个金刚,这样的排场,那般的声价,留你住了一夜,只值二十块钱不成?他们一班名妓,身分自高,不肯轻易向人开口。他初时指望你是个有钱的好客人,自然总肯花费,直等到过了几天,你仍旧一毛不拔,所以向你开场,要你买那一对戒指。你若答应了他,倒也罢了,却又土头土脑的不肯答应。他看透了你是个拼不得用钱的人,所以先把钱物骗到他手中,然后和你翻面,料想你这样的客人,做下去也没有什么好处,才下这一着绝户工夫。你还痴心妄想要去拿回!

他遇着你这种不知世故的人,他不敲你一下竹杠,他也不用做生意了。这些情景都是我身亲其境,阅历之谈,并不是说的空话。我向来性直,句句实言,你却不要见怪,把这一番话,认作我是有意讥诮之谈,那就辜负了我的好意了。“

这一席话,如雷震耳,如石惊天,把个方幼惲听得面上冷一会,热一会,冷了又热,热了又冷,听到后来,竟通身冰冷,满身汗下,立起来执着秋谷的手,道:“你这一番说话真是金石之谈,发人深省,指我迷途,我怎敢把你直言当作讥诮?

惟有自家懊悔而已。“秋谷大喜道:”幼惲兄真是聪明,不消几句话的工夫,已是心中明白,此后只要自己留心,不去上当就是了。“幼惲点头称是,想了一会,忽然又气愤起来,向秋谷道:”这陆兰芬十分可恶,竟把我当作傀儡一般,随他提弄。

我想上海妓女爱的是钱,有了钱财就有情义。我回去另汇几千银子出来,重做一个有名的妓女。料想上海地方甚大,名妓不独是陆兰芬一人,那时叫他在旁看着,心中难过,便算报了我的冤仇。你道如何?“

秋谷听了,甚是笑他痴气,不免又要劝解他一番,便道:“这话真是公子哥儿的脾气,一步也行不开来。依着你的主意是赌气跳槽,叫他在旁懊悔。即使果然如此,拼着自己的银钱去博别人的懊恼,试问于你有何好处?万一重做一个,仍与兰芬一般,或者比他更甚,可不是求荣反辱,你又怎的落场?现在你的心上虽然有些省悟,却还是半明不白的,将来一定要重入迷途。我索性把上海嫖界的情形,从头至尾演说出来,好等你死心塌地。古来教坊之盛起于唐时,多有走马王孙,坠鞭公子,貂裘夜走,桃叶朝迎;亦有一见倾心,终身互订,却又是红颜薄命,到后来免不了月缺花残。如那霍小玉、杜十娘之类,都是女子痴情,男儿薄幸,文人才子千古伤心。至现在上海的倌人情性却又不然,从没有一个妓女从良得个好好的收梢结果,不是不安于室,就是席卷私逃,只听见妓女负心,不听见客人薄幸。那杜十娘、霍小玉一般的事,非但眼中不曾看见,并连耳中也不曾听见过来。这是说妓女从良的了。至于逢场作戏,原是面上的应酬,流水行云,本来没有什么深情密意。倌人的心性爱的因是银钱,然而有了银钱就有情义,这句话却又未必。无论你在他面上花了一万八千,就是挥金如土的客人,他们背后也不说他一个好字,反说他是土老儿、曲辫子,这种客人不敲他的竹杠也没有日子的了。银钱花得越多,背后骂得更加利害,这是什么原故呢?他做着一个好户头客人,银钱撒漫,不消说心中是如意的了,却又怕同院的姊妹本家说他做了恩客,所以不肯背后说他。有钱的客人尚且如此,无钱可知;肯用钱的如此,不肯用钱可知。再说到堂子中近来的规矩,更是日趋日下,无从说起。从前都是倌人巴结客人,现在差不多要客人奉承妓女;以前都是客人要拣妓女的风头,现在差不多倌人要看客人的功架。偶然有几个初入勾栏的客人,不懂他们妓院中的规例,就要百般诽笑,甚至当面批评。你想,人家花了钱财,原是寻欢乐,博个快意,怎禁得倒是这般拘束起来,不是去寻开心,倒是自寻烦恼了。你道现在的嫖界还着得脚么?所以我劝你不要痴心。要晓得现在的上海非比从前,要想做个倌人,都要有嫖界的资格,不是门外汉可以误打误撞得的。你吃了陆兰芬如此的亏,还不自家猛省,倒要去再汇几千银子,去寻第二个陆兰芬,岂不是一误再误么?”

这番议论,比前一席话更加切当精微,尽情抉发,说得方幼惲连连叹服,又问道:“男女之情,无人不有,为什么上海这班妓女竟是太上忘情,难道他果然是个野兽山精,不知情爱的么?”秋谷哈哈笑道:“你的学问竟长进了一层了,但只知其一,不知其二。要想青楼妓女,朝张暮李,送旧迎新,他做的就是这行生意,叫他拿出什么情义来?古人欲于青楼中觅情种,已是大谬不然;你更要在上海倌人之内寻起情种来,岂非更是谬中之谬?那古来的霍王小女、杜氏名娼,都是千载一时、可遇而不可求的。你道现在上海倌人之内,千千万万可寻得出这样一个么?”

幼惲听了,虽然佩服他的议论,然而心上毕竟还有些疑惑,又向秋谷道:“如此说来,上海的堂子倌人没有一个好的,竟是足迹不入青楼的好。但是我前天在张园,看见你同陈文仙坐在一张桌上,喁喁私语,情意缠绵,就是那陆兰芬待你的情形,也是十分巴结。为什么他们待你又甚是见好,这是个什么原故呢?我就不懂得了。”秋谷狂笑道:“我好心相劝,你倒盘驳起我来。我原对你说,上海地方要做一个倌人,也要有嫖界中的资格,我就把嫖界的资格与你讲个明白。大凡古来妓女所重者,第一是银钱,第二是相貌,第三是才情。如今却又改了一番局,换了一派情形。近来上海倌人,第一是喜欢功架,第二才算着银钱,那相貌倒要算在第三。

至于‘才情’两字,不消说起是挂在瓢底的了。什么叫做功架呢?这‘功架’二字,就如人的功夫架子一般,总要行为豪爽,举止大方,谈吐从容,衫裳倜傥,这是功架的外扬。倌人做了这种客人,就是不甚用钱,场面上也十分光彩。再要说到功架的内场来,这是神而明之,存乎其人,可以意会而不可以言传的,只好说个大概给你听听。比如初做一个倌人,最怕做出那小家气相,动脚动手,不顾交情的深浅,一味歪缠,这是他们堂子里最犯忌的事情,免不得就要受他们的奚落。至于碰和吃酒,也要看个时候,不可一味听着他们的说话;或者那倌人生意闹忙,和酒不断,便不必去凑他们的热闹,只要不即不离的,每月总有几场和酒,也就是了;或者倌人生意并不见好,和酒稀疏,这却就要不等他们开口,自家请客碰和,绷绷他的场面。若是做了多时,已成熟客,倌人未免要留住夜,却万不可一留便住,总要多方推托,直至无可再推,方才下水。倌人们擒纵客人只靠一个色事。你越是转他的念头,他越是敲你的竹杠。客人们有了这一身功架,倌人就有通天本事,也无可如何。

总之,以我之假,应彼之假;我利彼钝,我逸彼劳,这方是老于嫖界的资格。若用了一点真情,一丝真意,就要上他们的当了。这几句话,便是功架的捷径、嫖界的指南。我从前曾经仿着“四书”做这‘功架’二字道:“功也者,功夫之谓也;架也者,架子之谓也。有工夫而无架子者,盖有之矣,未有无功夫而有架子者也。‘你把这几句揣摩纯熟,便有了一半工程。但是功架出于阅历,也不是一朝一夕的工夫,这是我章秋谷在嫖界中绝大的经济学问,所以歌场酒阵,整整混了三年,从不曾吃亏落后。幼惲兄以为何如?”

幼惲听了秋谷的第三篇议论,方才心下通明,笑道:“如此说来,你竟是个嫖界中的三折肱了。不料花柳场中,花钱取乐的地方,也有这许多道理!幸而我还沉溺未深,被你这切切实实的几场提醒,说得光致全无,不然,怕不闹个大大的笑话么?但是陆兰芬拿去那一只戒指是我母舅徐观察给我的,家严时常查问,不见了却有好些不便。我想另出几百块钱,托你想法子去赎他的回来可好?”秋谷笑道:“你既然言下悔悟,我怎肯袖手旁观?那银子虽然未见得拿得回来,这戒指在我身上,取了还你便了。”

幼惲虽被秋谷劝醒,却终是个慳吝的人,见秋谷肯替他到陆兰芬处去要回戒指,只喜得眼笑眉开,连忙立起身来,朝着秋谷深深一揖。秋谷慌忙拉住,笑道:“这点小事当得效劳,又算什么?”当下便拉了幼惲同到兰芬院中,幼惲觉得不好意思,不肯同去。秋谷道:“有我同着,尽去不妨,你难道怕他再要糟蹋你么?”竟扯了幼惲的衣袖向外便走。幼惲力弱,拗他不过,被秋谷一把拖着,好似鸡雏一般,一直走到马路上。幼惲着急道:“你放了手,我去就是了。你不怕马路上人笑么?”

秋谷方才放手。

到了兰芬院内,兰芬尚未起来。秋谷问知昨夜没有客人,便直走兰芬卧房坐下,叫幼惲去叫兰芬起来。幼惲摇手不肯,要叫娘姨去唤时。秋谷止住,自己掀开帐子,坐在床沿。看兰芬时,穿着一件湖色绉纱小袖紧身夹袄,盖着一条熟罗薄棉被,睡得正浓;星眸双合,杏脸微红,一缕漆黑的头发拖于枕畔,约有三尺七八寸长,香气扑人。秋谷便低低的叫了两声。兰芬已经惊醒,开眼见是秋谷,忙笑道:“阿唷!

二少,那哼今朝有工夫到倪搭来,耐是难得格客人啘!“一面说,一面坐起身来,挽了一挽头发,又披了一件玄色绉纱夹袄,斜盼着秋谷一笑。秋谷乖觉,便走了过来,在靠窗一张洋圈椅上坐下。幼惲却不开口,秋谷正要问他,陆兰芬已下床来,换好弓鞋,又问秋谷道:”二少,倪搭耐是勿大来格,阿是怪仔倪勒勿来介,今朝陆里一阵风拿耐格二少吹仔来哉?“秋谷笑道:”那里是什么风,倒是你的方大少同我来的。“兰芬还只认秋谷取笑,口中答应道:”倪陆里来啥格方大少,耐例说说看嗫。“不防回身过来,却却的与方幼惲打了一个照面。

原来兰芬下床之时,面向床里,所以不曾看见。当下兰芬吃了一惊,倒诧异起来,只得叫了一声:“方大少!”便回头问秋谷道:“唔笃阿是一淘来格?啥格勿声勿响,倒拿倪吓仔一跳。”秋谷笑道:“你说没有方大少,这不是方大少么?”

兰芬也笑了。幼惲见了兰芬,脸上不免有些赸赸的。

兰芬见他和秋谷同来,心中已瞧料了几分,略略应酬了幼惲几句,便一面梳头,与秋谷细细谈心。幼惲在旁看他眉敛春山,含烟如笑,目欺秋水,娇盼欲流,同秋谷谈得娓娓不倦,却并没有狎昵的话头。但觉两人眉目之间,若离若合,幼惲方相信秋谷的话,与兰芬果然没有交情。只听得秋谷同他说道:“现在的客人固然难做,现在的倌人更加难做。倒是那没有什么名气的人,不撑场面,还可支持,你们有了这个名气,撑着这个外场,要想从良,又拣不出个可嫁的人,生意虽然闹忙,日后终无结局,你也要自己留心才好。”兰芬拍手道:“划一,耐格闲话一点勿错。勿瞒耐说,要讨倪转去格人多得势来浪。倪为仔一生一世格事体,勿肯瞎来来,拣来拣去,总无拨对劲格客人。倪格做格个断命生意,也叫呒说法。”兰芬说到此处,忽咽住不说,神气黯然。秋谷也相对不语。

两人这一席长谈,兰芬已梳完头,秋谷对他招手,将兰芬招至后房,剩幼惲一人在外。不多一刻,便见秋谷先出来,随后兰芬走出,到床头边去拿了一个拜匣出来,身边摸出钥匙开了锁,取出一件东西。幼惲偷眼看时,原来是他的戒指,喜得心中乱跳,见兰芬将那戒指递与秋谷,秋谷接来,就带在手上。兰芬对秋谷道:“倪也并勿是要俚格戒指,为仔怕俚勿来,说戒指放勒倪搭,等俚自家来拿。倒说俚自家末勿来,叫仔俚格朋友来问倪要,倪拨俚要得光火起来哉,索性勿还拨俚。

今朝是耐二少爷来,勿好勿答应,勿然是随便啥人来要,倪定归勿拨俚格。“秋谷笑道:”承情之至,改日再谢。“便同了方幼惲出来。兰芬送到楼梯,叫秋谷常来走走,秋谷答应,回栈去了。正是:

红袖青衫相偎倚,佳人名士两倾心。

要知以后如何,请听下回交代。





第十回 兆贵里刘厚卿行令 吉升栈张书玉发标





且说秋谷回栈,把戒指交还了幼惲,又劝他早些回去。幼惲已经被他提醒,又因家中有信催归,当下也便应了,收拾行装径回常州去了。只有刘厚卿沉迷不改,又做了一个中尚仁里的时髦倌人,叫做洪笑梅。这洪笑梅面貌中平,身材却生得甚是长大,走到人前摇摇摆摆的,毫没有一丝婀娜的神情。自与厚卿落了相好,天天叫他吃酒碰和,还要叫他置办衣饰。厚卿是个钻在钱眼中过日的人,那里拚得这般挥霍?却为着张书玉待他冷淡,跳槽出来,要争这一口闲气,不得不熬住心痛,略略应酬。在洪笑梅虽把他看得并不在眼,刘厚卿却已着实出了一身臭汗。幼惲回去之时,想要与厚卿一同回去,厚卿不肯,依旧住下。

这几日工夫,刘厚卿在洪笑梅处约莫也花了五六百洋钱,曾在笑梅院中请秋谷吃过一台花酒。秋谷为他是幼惲至亲,自己又与他向来认得,不好推却,勉强应酬,却厌他是个胸无点墨、目不识丁的人,只略略的坐了一坐,便托故先走。

隔了数日,秋谷又因他先来应酬,只得在陈文仙处还他一席,坐中免不得仍是辛修甫等几个人。坐定之后,酒过几巡,秋谷便要行令,修甫道:“还是联句,还是飞觞?只不要搳拳摆庄,闹得头痛。”秋谷道:“联句虽好,只是座中恐有不能遵令的人,我想用个容易些的字面飞觞,这才雅俗共赏,你道如何?”修甫等大家称是。只见刘厚卿连忙嚷道:“章秋翁不要故意难我兄弟。我小时虽然读过几年书,这些年来都已还了先生的了,那里行得出什么酒令?我情愿先行受罚三杯,这酒令是不能遵的。”秋谷微笑道:“酒令严如军令,旁人不许阻挠,怎么令官刚才出令,你就先自喧哗,且先罚酒三杯再说。以后如再有人违令,取大杯来连罚十杯。”厚卿听了,把舌头伸了一伸,不敢再说,怕真要罚起大杯来。秋谷叫娘姨斟了三杯罚酒放在厚卿面前,逼他一气饮干。厚卿无奈,只得直着喉咙将三杯酒一齐灌下。

秋谷先饮了令杯,道:“我的意思,用‘风花雪月’四字飞觞。我们在坐恰好七人,从第一字起,各飞唐宋诗一句,飞至第七字为止,要依着次序,不许颠倒乱飞。各人饮门面杯一杯;说不出者罚五杯,再敬合席一杯,请旁人代说;说错一字者罚一杯;飞到本地风光或贴切本身者,大家公贺一杯。如今我是令官,就先从我飞起。”便又饮了一杯门面杯,先飞“风”字道:“风波不信菱枝弱。”大家赞好。

其次却轮着葛怀民了。怀民也干了门面杯,飞第二个“风”字道:“春风得意马蹄疾。”秋谷赞道:“吐属不凡,的是金马玉堂中人物,这是明年恭喜的预兆了。”

大家公贺一杯,合席饮了。第三轮到秋谷的同乡、一同来沪的何玉山,虽然没有什么才情,也还勉强来得。想了一会,飞了一句:“二月春风似剪刀。”秋谷笑道:“虽不甚切当,恰也总算亏他。”

待要过令时,早见王小屏立起来拦住,道:“且慢。”随取酒壶斟了三杯酒,放在秋谷面前道:“你且吃了罚酒再说。”秋谷呆了一呆,道:“为什么要罚起我来?就是说错了,也没有罚到令官的道理。”小屏道:“你且吃了,再和你说罚酒的缘故。”秋谷不肯。小屏道:“我若说得不是,吃还你加倍罚酒,何如?”秋谷一笑,把三杯罚酒折放在一个茶碗内,一饮而尽。小屏方才说道:“怀民说的是第二个‘风’字,第三个‘风’字还没有飞,如何就跳到第四个‘风’字去?他说错也还罢了,你这令官怎不检举出来,还要旁人来替你纠劾,难道要你这令官是摆样的么?”秋谷方才省悟,大笑道:“该罚,该罚!”连忙罚了何玉山一杯,要他再说一句。玉山想不出来,就连饮了五杯罚酒,又自己执壶敬合席的人各一杯。秋谷代飞了一句:“只愁风日损红芳。”方才轮着小屏。小屏随口飞一句:“飒飒东风细雨来。”又及修甫。

修甫正与一个叫来的倌人名叫谢兰荪在那里并肩携手,细细的讲话,秋谷叫他过令,道:“你们只顾谈心,连酒令也顾不得了。有心违令,要罚十杯。”修甫不答应道:“既要过令,你做令官的就要早些招呼,我不罗唣令官也就罢了,你反要罚起我的酒来,这不是有心罗织么?”秋谷道:“你们既把我举作令官,就要大家遵令,你这般倔强,要加倍罚你二十杯。”修甫愈加不服。吕仰正主张着罚了修甫五杯,修甫勉强饮了,就把令杯递与仰正,叫他接令。秋谷早劈手夺过令杯,道:“第五个‘风’字尚未飞出,便自过令,要罚七杯。”修甫无言可答,也觉好笑,只得又饮了五杯。谢兰荪因秋谷不许代酒,暗地里替他泼掉了两杯。原来修甫不会喝酒,不多几杯便要沉醉,吃了这十余杯急酒,已是头晕眼花,勉强撑住了,飞了一句:“山雨欲来风满楼。”秋谷还叫他是敷衍过令,再要罚他五杯,经大家劝住了。吕仰正便飞了一句:“年初十五最风流。”众人都赞他本地风光,合席贺了一杯。原来仰正叫来的局是个雏妓,叫做小媛媛,年止十五,玲珑第一,娇小无双,大家都赞他是个后来之秀,所以仰正就借了这个本地风光。

结末才轮到刘厚卿,厚卿一手接了酒杯,面涨通红,假作思索。秋谷将象箸敲着桌子催他,厚卿更加着急,急得咳嗽连声,还是秋谷看不过,向厚卿道:“一时想不出来,我就代飞一句可好?”厚卿就如逢了郊天大赦一般,忙道:“我实在荒了多年,竟一句也搜索不出,秋翁肯替我代说,兄弟认罚就是。”众人十分好笑,秋谷就飞了一句:“昨夜星辰昨夜风。”厚卿连吃了五杯,秋谷也陪了一杯。

正要从新起令,用“花”字飞觞,只见厚卿的家人走了进来,向厚卿道:“张书玉亲到栈里来寻少爷,说有要紧话说,叫小的立刻来请少爷回去,已经坐在房里等了半天,看他着急得了不得,也不知他有什么事情。”厚卿听得张书玉亲身到客栈寻他,还有要紧话说,觉得这句说话,耳中甜迷迷的钻了进去,料想他没有什么事情,不过为了几天不到他院中去,所以自己寻他。心中欢喜,面上便露出一副得意的神气来,立起身向秋谷道:“我回去走走就来,不知他来寻我有甚缘故,须要回栈问他一声。”秋谷却早料到书玉到栈寻他,必定不是什么好意,见厚卿十分高兴,不好当面说穿,便答道:“去去就来也好,我们在此专候。”厚卿连称不敢,告了失陪,穿上马褂,一直回栈而来。

到了自己的房间,抬头一看,只见书玉高高的坐在床上,却是怒容满面,同娘姨阿宝姐在那里咬着耳朵说话。见厚卿跨进房门,娘姨便含笑向书玉道:“先生勿要发极哉,刘大少来格哉。”有啥闲话末,同俚商量商量,料想刘大少也总要替耐想点法子格。“厚卿见书玉面有怒容,已是吃吓,又听得阿宝姐这等话头,虽摸不着头脑,知道事情不妙,老大着忙,又不好退回出去,只得进房坐下。正要开口,只听张书玉迎头问道:”刘大少,耐倒好格!倪就是有啥格推扳耐格地方,耐心浪勿舒齐末,也好朝倪说格啘,耐倒好意思跳槽,跳到仔洪笑梅搭去,倪搭人影子也勿见,还要瞎三话四,说勒倪搭用脱仔几化洋钱哉。耐倒自家摸摸良心,阿有介事?

勿要有仔天呒拨仔日头。现在外势才晓得耐刘大少用仔歹格洋钱拨倪哉,倪格新欠帐格店家,才来问倪收帐,逼得倪走头无路,人也急杀快。耐想半节里向阿有啥格洋钱还帐?勿还俚笃末,倪又坍勿落格个台。倪想想,也无拨啥格法子,横竖横竖格哉,倪归碗断命堂子饭也吃得勿要吃格哉。耐刘大少既然放仔格句闲话出去,叫倪做勿落生意末,倪索性拜托仔耐刘大少,一塌刮仔替倪开销仔罢,耐刘大少也勿在乎此格。

厚卿听他要他开销帐目,口气说得大了,早发极起来,勉强向张书玉道:“你这话从那里说起?非但我没有对人说过,并且待你也没有什么怠慢的地方,不过应酬场面多带了一个局,这就算是跳了槽么?倌人也不止做一个客人,客人也不见得做一个倌人,怎么你的店帐要我替你开销?难道你不认得我这个人,就欠的帐目都不要还么?你们想想可有这个道理?”书玉听了只冷笑一声,向阿宝姐道:“耐听听看,才勿关俚事,阿要推得干净!”又正色向厚卿道:“刘大少,耐勿要假痴假呆,倪向来格闲话说一句是一句,勿是啥格说仔搂白相。耐倒要替倪打算打算笃嗫!”

厚卿被他逼住,没有转身,已是十分惹气;又见张书玉声色利害,明知他不肯空回,只急得两足乱跳道:“这是什么说话!无缘无故的来寻起我来,叫我怎样的打算?我又没有用你的钱,没有欠你的帐,听凭你怎样便了。”书玉冷笑道:“上海滩浪有铜钱格人末也多煞,倪啥勒勿去寻着别人,独独寻着耐刘大少一干仔?耐自家想想,说出该号闲话来,阿对倪得住?”

厚卿听他说得没头没脑的,更加摸不着缘故,只是干着急,口中嚷道:“我倒底说了什么,你也要说个明白,不要半吞半吐,弄得人糊里糊涂。依着你的心上,要我怎样,你放着正经话不说,单单的同我转起大远的圈子来,我可知道你是个什么主意?”书玉道:“耐自家对别人说格闲话自家明白,倪也勿来替耐啥对格话头。

倪现在牌子拿脱仔,生意也勿做哉。娘姨笃格带挡,一千几百块,各处格店帐末,二千多点;一塌刮仔勿到五千洋钱。说起来是也呒啥希奇,就不过半中节里,一歇辰光要倪还起洋钱来,收末收勿着,借末无借处,叫倪身浪也勿会出啥洋钱。刘大少,倪一径待耐末也朆坏过歇良心,耐勿应该放倪格谣言,故歇弄得倪勿上勿落,格一杯酒是要挨拨耐吃格哉。“

厚卿听他盘子开得阔绰,心上没有了主意,虽然明知书玉有心敲他的竹杠,然而张书玉既然起了这个念头,料想不是三百、五百块钱可以打得倒他的,免不得要忍着心痛买个彼此相安;却不料他开口就要五千,早吃了一吓,心想就是一半,也要二千块钱。厚卿向来为人比幼惲更加刻啬,那里割舍得下?心中踌躇,方寸交战了一会,不觉恨起张书玉来,恨他无故生枝,硬敲他的竹杠。又被书玉说了一席不讲情理、一厢情愿的蛮话,心中更加了几分焦躁,那怒气竟按捺不住起来,便也变了面孔,冷笑道:“倌人敲客人的竹杠,也要客人情愿,方才显出交情。你说这样的蛮话,就是我情愿出钱,你也没有什么趣味。我在上海多年,倌人要客人的小货,我也见得甚多,却从未看见你这种泛蛮的人,真是第一遭儿,实在可笑!我还有正事在身,也没有工夫和你讲理,你请罢,我却先要失陪了。”说罢,立起身来就要往外走出。

那晓得张书玉性情本来悍泼,淫恶非常;又因厚卿跳槽到洪笑梅家,天天摆酒碰和的报效,眼睁睁看着大肥的鸭子,盖在锅里还被他飞了出去,已是气得不可开交。却没有想到他自己,那一天在张园看见了章秋谷,心荡神飞,恨不得立刻与他团成一块,把十分情意都用在章秋谷身上,去吊他的膀子。万不料章秋谷眼力高强,他这一副尊容那里看得上眼,所以凭着张书玉百般做作,搔头弄姿,抹巾障袖,只如没有看见一般,付之一笑,并不放在心上。张书玉却受了个老大没趣,又羞又气,他却还不死心,想慢慢的跟着,再去打动于他。刚刚走出弹子房,就遇见厚卿寻他,叫他一同回去。张书玉满肚皮没好气,只得上了马车一同回去,反怪着厚卿不该打断他吊膀子的心肠。看着厚卿的面目委琐,举止堪憎,越看越气,心中便二十四分厌恶他起来,便待他淡淡的,冷言冷语的讥诮。及至厚卿叫局,故意迟至台面将散,催了几遍方才到来,是有意叫他知难而退的意思。又不料厚卿跳到洪笑梅那里,居然的放开手段,银钱挥霍起来,懊悔前日不该做断了他,便要想个撒下瞒天大网,捞他一个罄尽的主意。同娘姨们商议了几日,才想出这一条计策来;预备先软后硬,要和厚卿大闹一场,万不肯空回白转。他明欺厚卿虽然滑溜,却是个无用怕事的人,就是事情决撒,也不怕他去告状经官。听见厚卿一场发作,正中下怀,只见他腮边起两朵红云,眉际横一团杀气,柳眉倒竖,杏眼圆睁,大声说道:“刘大少,耐勿要勒浪摆啥格松香架子,勿要说耐格种客人,就是比仔耐再要利害点,倪也勿见得吓杀仔人。耐开口闭口说倪敲耐格竹杠,倪就算是敲耐格竹杠末哉,老实说,倪格排客人勒倪身浪用格一千、八百,三千搭仔二千洋钱,也勿算啥事体。只有耐末一格铜钱才勿肯用,寒色搂抖极杀仔人,还要说倪敲仔耐格竹杠哉。倪自然总有道理勒,好敲耐格竹杠啘。耐今朝到底那哼?说一句闲话拨倪,勿要勒浪装啥格妈虎。”

厚卿正待要走,却被张书玉翻转面皮,不遗余力的数说了一顿,只气得浑身乱抖,一句话也回答不出来;停了半晌才说出一句话来道:“你这说话真是岂有此理!

难道世上没有王法的么?“一面说,一面仍想脱身走出,早被书玉抢上前劈胸揪住。

正是:

爱河滚滚,大家同在沉沦;情海茫茫,何苦自寻烦恼。

不知厚卿怎生打发书玉,且待下回交代。





第十一回 对酒当歌忽逢旧友 阳春白雪快和新诗





且说书玉抢步上前,把厚卿胸前衣服一把扭住道:“晓得耐刘大少是有财有势,倪也壳张格哉,上海县新衙门随时耐刘大少格便,耐勿要走嗫。”厚卿被他扭住,不由的心中乱跳,又急又气,嚷道:“你、你、你要怎、怎样?怎、怎么不、不、不问青红皂白,就动、动、动起手来?这、这、这样拉拉扯扯的,算、算、算什么样子!”书玉道:“耐勿理倪格闲话,要想走出去,倪自然只好动手哉啘。”厚卿着了急,把书玉用力一推,想要把他的手推开方好脱身。那知书玉力大非常,一把衣服紧紧的拉住,那里肯放!只是脚下跳着高底,立脚不稳。厚卿用力一推,来得势猛,竟是仰面一交。厚卿因衣服被他带住,也是一交,跌在书玉身上。那书玉吃了一交筋斗,愈加撒泼,高声喊道:“耐只顾打末哉,唔笃大家来看嗫!”

只一闹,把栈中茶房并隔壁房间的客人,都一齐拥到厚卿房门口来,却不知为着何事。阿宝姐见不是势头,连忙上前拉开厚卿,又把书玉扶起,劝书玉道:“先生勿要实梗嗫!有啥闲话末,好好里替刘大少说,刘大少也无啥勿肯格呀!”又向刘厚卿道:“刘大少勿要动气,倪先生末也是一时之火。耐是老相好哉,总要包涵俚点,大家好好里商量末哉。”书玉跌了一交,头发已经披下,更如枉死城内放出来的小鬼一般,愈加可怕;被阿宝姐扶了起来,也趁势住了口,却还咕噜着道:“俚耐要打末让俚去打末哉,倪索性拿格条性命交拨仔俚完毕。倪活勒世浪也呒拨啥格好处,拨别人家逼杀快。”

那厚卿被阿宝姐拉开,捺在椅上坐下,看看今天这般风势,料想不得好好开交,走又走不脱,回又回不去,心上就如热锅上的蚂蚁一般团团走转,想不出个脱身的法儿。忽想起章秋谷来,曾替方幼惲在陆兰芬处讨回戒指,在上海花柳场中颇颇的有些名气,大家都晓得这一个人,而且为人重义,风骨非常。若得他肯来劝解书玉,调处这件事情,真是十分把稳,便连忙叫了当差的来,分付他道:“你快快到南兆贵里陈文仙院中,飞请章老爷立刻就来,说我在栈中有要紧事情,无论如何务必请他就到,不可耽搁。”当差的答应了,忙忙到兆贵里去。

只说秋谷自刘厚卿回栈之后,对修甫等说道:“这个人虽是世家子弟,实在俗不可耐,满面上露着浮华之气,不是个可交的人。听见我要行令,便吓得屁滚尿流,这种人真是可笑!如今他既去了,我们这酒令却止剩了六人,况且这令极是浅近,实在无趣,我们改作即席联句罢。”修甫等一齐称善。

秋谷便先干了一杯,修甫等也干了,问娘姨要过纸笔,秋谷提起笔来正要写起句时,忽见门帘一起,又闯进一个人来。秋谷忙起身看时,那人向秋谷兜头一揖,道:“你好快活!在苏州闹了个大大的名儿,也不来招呼我一声,没有看见你们的盛会。现在又走到上海来,可被我寻着了。”秋谷连忙回揖。原来这个人与秋谷是总角之交,姓贡,号叫春树,是一个诗词名手,正与秋谷旗鼓相当,且又生得粉面欺何,素腰压沈,那神情意态一味的温柔抚媚,竟如美女一般,迥非秋谷那一种眉目清扬、神情英武的态度。秋谷与他诗文知己,互相推许。

这贡春树本是杭州人氏,幼年随着父亲,做过一任常州府同知。他父亲终于任所,身后略略有些宦囊,苏州还有几处房屋。贡春树因杭州地方没有什么宗支亲友,便不回原籍,就在常州府城居住。秋谷因曾祖以下坟墓俱在常州,每年春、秋二季,必到常州扫墓,便住在春树家中,诗酒盘桓,十分相得。此番贡春树打听得秋谷在苏州青阳地浪游曲院,用度豪华,便赶到苏州要与秋谷相会,不料秋谷已经回去了,扑了一空。春树在苏州住了两月,顺便收取房租。前日方幼惲自上海回去,路过苏州,恰好遇见了春树,与他说知备细。春树方晓得秋谷已到上海,便急急赶来,打算与秋谷商量一件事情,要秋谷替他出力,却忘记了问明方幼惲住在什么栈房,所以到了码头,只好先将行李发在三洋径桥长发栈去,自己却各处寻问。上灯之后,方才寻到吉升栈来,晓得秋谷在兆贵里请客,连忙径到陈文仙院中来寻秋谷。

当下秋谷问明了春树的行止,方知他特地到沪相访,故友相逢,心中大喜,便向春树道:“你来得正好,我在此间结了一班朋友,都是性命道义之交,我的朋友就是你的朋友一般,你且见过了这几位,再说别话。”春树便与修甫等拱手,彼此问了姓名。春树见修甫、仰正等意气惊人,行为豪爽,修甫等见春树仪容俊雅,谈吐风流。从来方以类聚,物以群分,不觉大家共相倾慕。修甫等便让春树上坐,春树不肯,修甫道:“春树兄今日才来,又是远客,我等忝为地主,岂有僭坐之理?”

春树推辞不得,方才坐下。

春树见台上有笔砚信笺,问秋谷道:“你们台上放着笔砚,想是行什么酒令,却被我这催租隶来败了你们的清兴。”秋谷微笑,将改令联句向他说了。春树大笑道:“席间联句是近来一班斗方名士的习气,你如何也学起他们来?好好的饮酒何等不妙,却做这等酸子的事情!我是第一个不遵令的。”秋谷一笑,答道:“我们的席中联句,是大家舒写性情,平章风月,却不是做了诗连忙去刻在新闻纸上的斗方名士可比。你既不以为然,我亦乐得藏拙,免得去搜索枯肠,但是你刚刚入席,就第一个违了我的酒令,却饶你不得,须要罚你十杯,若喝不了这许多,罚你即席赋诗自赎。”春树道:“要我做诗不难,我即席赋诗,你亦要立时和韵,方算得令官的公允。若只许州官放火,不许百姓点灯,我就要鼓噪了。”秋谷笑道:“依你,依你,但古人七步八叉,俱有成例,若构思迟了,就要加倍罚你二十杯,须要落笔如风,不许停顿,你可敢答应么?”春树毅然作色道:“这个何难?料想也未见得难我得倒。你且吃了令杯,看我立时挥洒何如?”秋谷道:“我做令官并无私曲,你若能文不加点,大家也要公贺三杯。”秋谷果然干了令杯。

春树要过一张八行信笺,也不思索,提起笔来,看他走笔如飞。秋谷等在旁看着,只见写得好一笔赵松雪的行楷,娟秀非常,写着《即席赋赠秋谷章君》一首七律道:

五陵公子正翩翩,裘马清狂佳客前。

太白豪情穷碧落,冬郎才调况青年。

诗肠对月原如水,剑气凌云快欲仙。

春树写到此处,正要奋笔直书结句,忽然一想,错了一个韵脚,便略略停了一停,要换个韵,却未免就停笔不下。秋谷早大笑道:“温八叉今竟如何,若再停一刻,便要倍罚二十杯了。”春树笑道:“你不要自恃做了令官作威作福,停会待我也做一回令官考你一考,看你这曹子建还能七步成章否?”秋谷道:“你不要与我斗口,且完了正文再说。”春树一面说,一面早把两句结句写了出来。众人看是:

我愧郊寒并岛瘦,闻君高论为开颜。

修甫等一齐赞好。秋谷笑道:“诗意甚佳,姑且免罚,但是揄扬太过,却要罚你一杯,我也陪你一杯。”春树也不推辞,欣然饮了,道:“你的令官已经卸任,待我这令官也来出个题目何如?”秋谷笑道:“任从尊意。”春树道:“我如今先要你原韵和出一首,非但不许停顿,而且还要击钵催诗。若鼓已绝而诗未成,也要罚你二十杯,众位以为何如?”修甫等齐和道:“秋翁向来诗才敏捷,真可倚马万言,想必不至受罚。我辈拭目以俟佳作便了。”

秋谷笑了一笑,随取过纸笔来。春树取一支象箸,在茶杯上“当”的打了一下,道:“鼓声已起,速速做来。”秋谷提笔便写,兔起鹘落,满纸淋漓,一笔草书比春树更加神速,不一刻早已写完。春树也自怪诧,暗想:怎地比自己更快?果然并生瑜、亮,自己较逊一等。大家看那诗时,只见写着也是一首七律,上写“奉和原韵”:

江南词客太翩翩,况在临安画阁前。

己分玉萧成隔世,漫将锦瑟误流年。

惭无叔宝风前度,应有瑶台月下仙。

拚把清樽同一醉,不须惆怅问朱颜。

众人看完道好。秋谷笑道:“我向来不爱和韵,今日被他逼住,无可如何,只得潦草塞责,诸兄怎还要谬赞起来,岂非违心之论?”仰正道:“我们知己相叙,不作套谈,秋谷为何总有一番谦逊,这要罚你一杯。”就斟了一杯酒送过来,秋谷倒也无言可答,只得受罚了一杯。

春树还有些心中不服,便又出令道:“我见《随园诗话》中有新婚诗,以‘阶乖骸埋’四字为韵,我想这四个韵脚虽然难用,也不至十二分艰难,我们在座各依韵和他一首。我却要自家僭妄,做个令官品评甲乙。”向秋谷道:“你可能遵我的令么?”秋谷道:“只要大家承认你做令官,独我一人,岂有不肯道令之理。”修甫等道:“树春兄此令甚好,我们大家遵令而行。”春树大喜,复向众人告罪,先饮了门面一杯,众人也多干了,便各各构思起来。那知看着虽不甚难,却也不甚容易,春树自家也在沉吟。

却是秋谷略一思索,取过纸来,早已一挥而就。众人惊异看时,只见写道:

十里珠帘开画靥,两行宫使列瑶阶。

仙裙簇蝶情初定,玉佩和鸾愿未乖。

慧质只应天上有,冰姿直与雪同骸。

明灯更照红绡色,莫令名花宝帐埋。

大家看了,哄然叫好。修甫道:“有此佳作在前,我等只好大家搁笔,不必再去苦思力索的了。”秋谷道:“我们诸位都是高才,怎么也这般谦逊起来?”修甫道:“并不是故意推辞,我同你讲这缘故,你就明白了。这四个韵脚本来难押,有《随园诗话》一首于前,又有你这一首于后,我们就是再做出来,也是拾人唾余,味同嚼蜡了。我们还是受罚一杯罢了。”就大家斟了一杯干了,又公贺了秋谷三杯。

修甫把秋谷这一首诗翻来覆去的看了几遍,赞叹不置。连贡春树暗中也是十分佩服,秋谷真是天赋清才,不同流俗,就也极意称扬。秋谷谦让不已。

正说之间,只见又闯进一个人,满头大汗。秋谷诧异,看时,原来就是刚才来请厚卿回去的家人,气喘吁吁,上气不接下气的向秋谷说道:“张书玉来了。家爷叫家人来请老爷立刻前去,有要话说呢。”秋谷更觉奇异,笑道:“张书玉是去寻你家少爷的,你家少爷同他有甚瓜葛,我却同他没有什么交情。他有话说,怎么你来寻起我来?你不要弄错了人罢!”那家人因厚卿被书玉糟蹋,不成局面,心中也是着急,又为厚卿吩咐他立刻去请秋谷,他果然并不停留,飞一般跑到兆贵里来。

跑得气喘,便夹七夹八的说了几句。此时被秋谷提醒,自家也觉好笑,定一定神,方才说道:“家人来得慌忙,说错了话,实是张书玉寻到栈中要与家爷拚命,家爷着急,才吩咐家人来请老爷的。”秋谷更加摸不着头脑,诧怪得了不得,修甫等大家也觉希奇。秋谷又问道:“张书玉好好的,为什么无缘无故要同你家少爷拚起命来?他既要拚命,又请我去做什么?你可慢慢的讲。”那家人方把书玉要厚卿开销店帐、动手揪扭的话说了出来。秋谷皱着眉头道:“这样的事情何必定来请我,难道我还能止住他不闹么?你去上复你家少爷,说我没有工夫管这闲事。”那家人见秋谷不去,便着了急,又道:“老爷的明见,家爷再三吩咐家人,说一定要请到老爷。老爷若是不去,家人回去销不得差。况且家爷这事全要仗着老爷调停,别人料想也是分解不来的。还求老爷的恩典,体恤家人罢!”说着,又打了一个千,恭恭敬敬直挺挺的站着伺候。秋谷听那家人说话例甚是伶俐,料推却不得,况也要去看看张书玉究竟做出什么悍泼情形,便点了一点头。那家人大喜。

秋谷又对修甫等道:“本欲与诸兄畅叙一宵,无奈又有别事,只得失陪,改日再行补叙的了。”众人齐称:“好说。”秋谷起身要走,陈文仙亲手替他披上马褂,又替他扭好,低问他:“今夜可还来?”秋谷摇头,便别了众人要走。春树一把拉住道:“且慢,我还有正经话有同你说呢!”就附着耳朵说了几句。秋谷皱皱眉道:“你又去闯出祸来,我可不能管了。”春树着急,又悄悄说了几句。秋谷道:“你同我回栈去,慢慢的商量罢。”春树便同秋谷同走出来。众人因主人已去,随意用过干稀饭,一哄而散。

看官且慢,那有秋谷做了主人,不等客人先散,自己先走的道理?殊不知秋谷是个豪士,落落难合的,同这班人都是道义之交,相交以神,不拘形迹,况且他们数人都敬重秋谷的才华文采,大家都是胸襟阔大的人,全不在这些小节。正是:斋

琼枝璧月,人争掷果之姿;斗酒百篇,光照生花之笔。

欲知秋谷如何劝解,只看下回便晓。





第十二回 翻花样偷天换日 吊膀子接木移花





不说章秋谷同着贡春树回栈,再说刘厚卿自从打发家人去请秋谷,略觉放心。

等了一会,还不见来,心中焦躁。偷眼看张书玉时,头发虽然挽起,那面上还是铁铮铮的杀气横飞,一双眼睛定定的斜睃着他,又有个要发作的意思。只看得厚卿坐立不安,背上如有芒刺,屁股如坐针毡,急得满屋子里团团打转,眼巴巴的只望秋谷到来,好央他劝解书玉。那知左等也不来,右等也不来。原来等人心焦,况且厚卿有事在心,更觉得时候长久,满口里乱骂那家人:“这个混帐东西,怎么这样没用,去请一个人也请不来!”忽听书玉冷笑道:“耐就是去请仔耐格朋友来,也无拨啥格说法啘。阿是朋友来仔末,倪就怕耐,勿敢替耐说话哉?”厚卿听了又羞又恨,欲待骂他几句,又怕书玉性情凶恶,索性借此大闹起来,客中甚是不好意思,只得忍住了气,不敢开口。那一种可笑可怜的情状,真是好看。

好容易等得外间脚步之声,约略是秋谷的声音来了,心中一块石头刚才落地。

果然不多时,那家人先抢步进来,回道:“章老爷来了。”厚卿大喜,忙走到门口。

家人便打起门帘,只见秋谷笑吟吟的进来,口中说道:“有累吾兄久等,心切不安。”

厚卿连称“不敢”,迎进房来坐下。秋谷道:“刚才盛价来说,你与书玉有些口角,但书玉同你向来要好,为什么淘气起来?或是你自家有不到之处也未可知。我倒要请教请教,你们到底是为什么缘故?”

先前秋谷进来,书玉本是坐在床上,低着头装做没有看见;及至秋谷开口,并不派着书玉不是,反说厚卿或是有些不到。这本是秋谷的口才,不劝自劝,料想书玉听了自然心中欢喜,方好乘便劝和。果然张书玉听得秋谷说话在行,不由的就有几分高兴,抬起头来打量秋谷的相貌时,心中早突然一跳,又喜又惊,原来就是张园相遇、眠思梦想、不得到手的心上人儿。此际书玉不由自主,连忙立起来叫了秋谷一声,登时把方才面上的那一团杀气威光,消化得干干净净,变作满面笑容,喜孜孜的在台旁坐下,便告诉秋谷道:“章大少,耐勿晓得倪格事体,倪说拨耐听仔,随便啥人也要心浪惹气格。格个刘大少,做仔倪一个多点月哉。自从俚到仔倪搭来,倪倒当俚好客人格,从来朆叫俚打啥格首饰,做啥格衣裳,碰和吃酒也随俚格便,洋钱是加二朆见歇。倒说归转仔,俚来叫倪格局,倪为仔转局过去晏仔点点,俚就此扳倪格差头,搭倪反子一泡,倪搭勿来哉,跳槽过去,另外做仔格洪笑梅,日日替俚碰和吃酒,做衣裳,打首饰。倪也勿去管俚,只当无介事,不过少做一个客人,算得好说闲说格哉。勿壳张俚勒浪外势,还要说倪格邱话,放倪格谣言,倒说俚勒浪倪搭白相仔勿到一个月,用脱仔论万洋钿哉。难末拨倪格排欠帐格店家、借债格户头听见仔,大家勿好哉,一淘到倪搭来,收帐格收帐,要债格要债,才问倪要洋钱。章大少,耐去想嗫,半节里倪陆里来啥格洋钱,勿还俚笃末倪又坍勿落台,逼得来倪急杀快。格件事体弄僵哉啘,倪想起来才是刘大少格勿好,勿放倪格谣言末,倪也勿造至于实梗样子。今朝倪实在弄勿落哉,跑到刘大少搭来,想问俚借点洋钱开销开销,等倪过仔节,收帐下来,更好还俚,也勿算敲俚格竹杠。俚耐洋钱末勿借,拿倪骂仔一泡勿算,还要动手打倪,推仔倪一交筋斗。章大少,耐想想看,世界路浪,阿有格号道理?请耐章大少替倪评评,倪是横竖呒啥念头转,今朝定规要俚拨倪一句闲话,随俚去拿倪那哼末哉。”口中说着,一面笑微微的向秋谷连丢几个眼风,又用金莲在桌子底下,勾住秋谷,那两只眼睛水汪汪的,把秋谷浑身上下钉住呆看,恨不得要立刻扑在秋谷怀中。

厚卿初时见秋谷进来坐定,刚刚开口,张书玉便是满面含春,撇去了先时凶狠形容,平添出一副温柔体态,厚卿心中暗想:秋谷果然名不虚传,怎么他才开口,张书玉便不似先前那般形状,出奇的柔顺起来?后来听张书玉向秋谷一番说话,句句说他不是,甚是气忿,待要开口辨白几句,却被秋谷对他连连摇手,厚卿只得默默无言。

好个张书玉,把一番话说得来婉转非常,遮掩得自己并没一些不是,秋谷暗暗点头称赞,到了紧要之处,也还飞他二个眼风。书玉觉得秋谷今日情态温存,绝不是前日在张园那一副待理不理的面孔,更是十分意满,那两旁面颊之上,早泛出点点桃花,隐隐的眉目之间,大含荡意。

秋谷听他说完了一席话,心中想道:我要驳倒他,叫他无言可答,有何难处?

但是书玉本是泼赖非常,厚卿又是十分无用,我一个旁人怎好管他闲事?不要弄得他恼羞变怒,依旧不讲情理起来,于自家面子岂不有碍?只是又有一件难处,书玉本来有心于我,前天在张园极意迁就,吊我的膀子,我却嫌他面貌不好,没有理会于他。如今自家要替厚卿调处劝解这件事情,不用说,拿得稳书玉是一说一听的。

既要曲意替他和解,自家却就免不得要领书玉的盛情。看着书玉那雄赳赳的神情,着实有些退避三舍,不觉的就为难起来。忽然眉头一皱,想出一条“接木移花”的计策,心中大喜道:“有了,有了!只消如此这般,这事便有二十四分拿手,不怕书玉再要装腔。”

正待开口,只听得厚卿接口道:“秋谷兄,你不要听他的说话,我并没有在外边放他什么谣言,这是他一厢情愿的主意,你须要替我分解分解才好。”书玉在旁冷笑,接口正要驳他,也被秋谷朝他摇头示意,书玉便不开口。秋谷向厚卿微笑道:“你有也罢,没有也罢,总之,书玉无缘无故不见得起你的花头。你们这班曲辫子的大少爷,专喜对着别人说你自己的阔劲,如何用钱,如何发标,乌烟瘴气,闹得一塌糊涂。在你们的心上,以为不如此装不出自家的幌子。那晓得嫖场的诀窍,世路的人情,非但装不来自家的场面,还出了个吹牛屄说大话的名头,从此别人看你不起,就如自己贴了招子,出卖曲辫子的招牌一般。书玉的说话固然不可全信,未免也有些过甚之谈,然而想情度理起来,你也不要推得干干净净。大约在人前说几句大话,说在书玉面上用了多少银钱,想去哄动人家来巴结你,也是有的。我从来未曾开口,早已洞察情形,你若再要在我面前遮掩支吾,不肯说出实话,那却你就怪我不得,不管你们的帐了。”

厚卿被他说着了真病,面上红了一阵,闭口无言。张书玉更是喜欢,五体投地。

秋谷却向书玉道:“你的意思我都晓得,自然总有个调停。你且到我的房间去略坐片时,你有什么说话,我再同你商量可好?”书玉巴不得秋谷说这一声,大喜应允,又向秋谷道:“章大少格说话,句句才说到倪格心浪。”回头将手指着厚卿道:“俚耐格闲话,搭耐章大少一样仔末,倪也勿要替俚反哉。”说着又斜盼着秋谷一笑,以目送情。厚卿看见,岂有不知?虽也不免有些醋意,但是看着秋谷样样较胜一筹,自己那里比他得上?况且又要秋谷替他调处,自然只好由他,只在腹中暗暗的叹着冷气。

秋谷随手立起来,向厚卿说道:“我去去就来回你的话,你可不要出去。”厚卿连连答应。书玉也不理厚卿,同了阿宝姐跟在秋谷后面就走。厚卿虽然心中不乐,也无可如何,只自家悔恨当初不该做他,如今弄得这般无趣。

只说书玉跟着秋谷一路走上楼来,心中暗喜。只说秋谷将他引到自己房间,必定有什么心腹的说话,却不晓得秋谷另有一番意思。秋谷在兆贵里同了贡春树回来,因为他与刘厚卿素不相识,便叫他在自己房中宽坐等候。春树正是等得不耐烦,反背着手在房中踱来踱去,忽见秋谷进来,背后还同着一个倌人,忙笑道:“你在那里有什么正经?去了半天,把我丢在这里,好不心焦。”

书玉跟着秋谷走进房间,见房内还有一个客人,心中觉得不甚自然;及至举目看时,那知不看犹可,一看早又吃了一惊。只见春树容华俊雅,骨格风流,粉面朱唇,细腰窄背,同秋谷立在一处,真是一对璧人,不分上下。但春树是一团的妩媚非常,秋谷是一派的英风流露,若要两人相并,还觉得秋谷胜些。书玉心中暗想:怎么相貌好的都聚在一处?为什么我在上海见了无数客人,没有一个比得上他们的呢?看看秋谷,又看看春树,把个书玉竟看呆了。秋谷招呼他坐下,方才觉得,未免不好意思,随便在窗口一张椅子上坐下了。秋谷却不向书玉说话,叫过春树来悄悄附耳说了几句。春树微笑,回头把书玉细细的上下打量一番,朝书玉微微一笑,又向秋谷摇头。秋谷顿然不悦道:“你不答应么?”春树点一点头。秋谷便道:“你不听我的说话,回来你有什么事情,可不必再来找我。”春树忙陪笑道:“你不要着急,我倒不是不答应,倒是怕你要吃……”春树说了半句又不说了,朝着书玉格格的笑。秋谷道:“吃什么?说下去,你说出不好的话来,可不要怪我粗鲁。”

春树听了,连忙将头项缩了一缩,舌头伸了一伸,说道:“罢罢,我不说了。谁不知你是个拳棒名家,我这几根鸡肋,那里当得起你的尊拳?”秋谷也一笑,便剪住了话头。

此时张书玉坐在旁边呆呆的看着他们两个,听得秋谷与春树互相问答,又看着他笑,心中早已十分明白。若在别人,说了这几句说话,书玉早已就板起面孔来,无奈书玉看着秋谷同春树两人,一个是玉树临风,一个是琼枝照月,恨不得取一碗清水过来,把这两个傅粉郎君一齐吞下肚去,爱还爱不过来,巴不得他们与他说笑。

看张书玉这一时的光景,就是叫他无论如何,他也断无不肯。

当下秋谷携着春树的手,向书玉道:“这是我的把弟贡春树,待我替你们做个媒人。”书玉低鬟一笑,不觉面上生红,把秋谷斜睃了一眼。秋谷对春树道:“你今夜就在他那里请一台酒可好?”春树道:“摆酒不难,只是时候已经不早,那里还请得着什么客人?况且我初到上海,也没人认得。”秋谷大笑道:“你这说话越说越呆,真真是个饭桶,叫你请客,无非开个堂簿的意思,以后便可往来,难道叫你认真请客么?”春树恍然,也自好笑。

书玉眉花眼笑的道:“贡大少要吃酒末,倪先转去预备起来阿好?”秋谷道:“你先回去也好,但是厚卿的事情,你究竟是什么一个主意,你不妨同我说明,可好看我的薄面,将就了结。”书玉道:“倪也勿是一定要俚那哼,为仔俚讨气勿过,倪有心要替俚拌拌嘴舌。既然耐章大少说仔末,随便章大少末哉,倪总呒拨啥勿肯格。”秋谷大喜,笑道:“你既听我的说话,也不必与他吵闹,料想你也不是一定希罕他的银钱,只要他以后晓得些轻重也就是了。现在总算我来替他讨个情,叫他拿出几百银子,罚他个不该乱放谣言,他此后料也无颜再在你家走动,你道如何?”

书玉道:“章大少格闲话,倪总无啥勿听。谢谢耐,要耐章大少费心,就是实梗末哉。”秋谷笑道:“这是我承你的情,看我得起,怎么你倒谢起我来?”说着,便连忙去厚卿那里,替他说了情形,又道:“我的意思,硬作主张,你竟是干干净净送他五百银子,从此一刀两断,他也勉勉强强的应了下来,你的意思怎样?”

厚卿听张书玉居然应允,心中虽是欢喜,却又舍不得五百银子,蝎蝎螫螫的说道:“怎么竟要五百银子?可好费秋翁的心,这数目少些?”秋谷不觉大怒道:“原来你这个人如此的不知好歹,怪不得张书玉要敲你的竹杠。照你这样说来,倒是我多事的不是。我也不管你们的闲事,我去回复他就是了。”秋谷说这几句话时声色俱厉。厚卿见秋谷发怒,已是吓慌,知道自己失言,十分懊悔;又见秋谷拂衣要走,更加着急,连忙拦住秋谷,连连作揖,赔了许多不是,秋谷方息了怒气。说定明日汇了银子,由秋谷经手付与书玉,又数说了厚卿几句,便回自己房间里来。

见春树与书玉二人谈得正是热闹,阿宝姐坐在一旁打盹。

秋谷进来,笑道:“时光不早,我们就到书玉院中去罢!”当下议定,夜深无处请客,单请秋谷一人。先打发书玉回去,二人随后慢慢的同到院中。书玉含笑相迎,房中台面已经摆好,秋谷等一到,就起手巾入席。秋谷见并无外人,便令书玉同吃,书玉不肯。秋谷道:“我们二人不比别客,你难道还要拘着院中规矩么?”

书玉一想不错,果然坐了。席间,与秋谷谈些旧事,秋谷酒落欢肠,已觉微醉。这一席酒虽止有三人,却低酌浅斟,吃得甚是爽快。书玉虽觉有些美中不足,然而看着春树的面貌娇柔,丰姿倜傥,也甚是喜欢。

秋谷饮到半酣,便要先走,被春树留住,悄悄谈了一会。秋谷道:“这样的好差使,为什么不去寻着别人,总只缠我一个,这是什么道理?”春树陪笑央求,又朝秋谷作揖,秋谷勉强点一点头道:“也只好碰你的运气便了。”春树大喜。书玉在旁,也不知他们说的什么,又不好问他,秋谷便先回栈去了。正是:斋

一双蝴蝶,可怜同命之虫;卅六鸳鸯,妒煞双飞之鸟。

欲知后事如何,且待下回分解。





第十三回 汪宏超花钱代审 金汉良拼命吹牛





且说秋谷回栈之后过了一夜,明日一早便会见了刘厚卿,问他银子可曾齐备,厚卿回称:“钞票已经现成。”便在枕头旁一个大皮包内取出一卷钞票,点了数目,双手交与秋谷。秋谷收了起来,因见厚卿瘟得利害,觉得他也甚可怜。

厚卿将钞票交代了秋谷,又连连致谢秋谷费心。秋谷便想再费一番唇舌,把刘厚卿劝醒转来,便他不至沉迷不醒,也算大家认得一场。便邀厚卿到自己房间坐下,将以前劝解方幼惲的几层说话,恳恳切切的功了厚卿一遍。又道:“你道张书玉同你吵闹,是要敲你的竹杠么?他是因为你土头土脑的不甚漂亮,又不肯爽爽快快的花钱,他心上不愿意你在他院中走动,所以平空把你冷淡起来,好等你从此不来的意思。你想上海堂子还有什么玩头?即如我章秋谷,老于嫖界的人,也要步步留心,不肯一丝大意。凭着你这样一个人,不知嫖界的情形,不懂院中的规矩,平空的走到上海,要去嫖起四大金刚的张书玉来,上海的金刚可是好嫖的么?像你这样没有功架、不肯花钱的客人,他眼睛角里也没有梢着你,你还要想去装呆做傻与他论交情。他不糟蹋你,倒糟蹋我么?”

厚卿虽是沉迷,倒底心上总还明白,听了秋谷这一番议论,把上海堂子的情形,倌人的性度,一齐抉发出来,无论再是下愚不移,听了这种激切的说话,也不由得毛骨悚然,通身汗下,便向秋谷道:“秋翁现身说法,真令顽石点头。怪不得方幼惲经你一番劝解,立时收拾归家。我如今回想起来,真真是个痴子,花了多少冤钱不算,还惹出许多气来,岂不是自寻苦吃?我在此间略停数日,便也要回到常州,从此看破他们的手段,不再去惹草拈花,省得辜负了秋翁的苦心劝解。”

秋谷起初劝解厚卿之时,还当他未必果能猛省,姑且把他提醒一番。今见厚卿居然言下大悟,心中爽快非常,哈哈大笑道:“果然厚卿兄甚是聪明,一说已经明白。我章秋谷浪游花柳,到处留情,未免也惹下了许多风流孽障。如今仗着这广长妙舌,居然劝得你们勒马回头,也是我一生快心之举了。”厚卿听了,感激万分,想秋谷这样的人,侠骨柔肠,真是世间难得,着实谢了几声。秋谷连忙止住,又说了几句闲话,拱手别了厚卿,便到别处寻人去了。天有正午,方才到栈,吃过了饭,想着厚卿的钞票还在身边尚未交出,本来想去问春树的信,就到新清和张书玉院中来。

出了栈房,信步慢慢的行走。新清和离吉升栈本来甚近,不用坐车。正走到大新街口,忽见对面一乘光彩辉煌的轿子,三个轿夫都着绉纱紧身小袄,绉纱兜裆马裤,抬着轿子飞一般的直撞过来。那轿子是用翠色洋蓝大呢做了四围的轿衣,通身用白绒线绣着折枝梅竹,中间还镶嵌着水钻,光华夺目。轿子四角边结着四个湖色流苏,两旁玻璃也衬着绣花软帘,垂着湖色绉纱黑线酒花的遮阳,瘦瘦的一付杭州香藤轿杠,杠上前后也结着四个小小的彩球。那轿子四周更用白铜打就的各色折枝花样,钉在轿上,耀眼争光,收拾得十分精致。秋谷暗想:好一乘讲究的轿子,谅来是什么红倌人坐的了,但是天气刚刚过午,为何出这样的早堂差?正在暗想,那乘轿子抬得飞快,已是擦肩过来。秋谷要看轿内坐的倌人面貌如何,便住了脚步,仔细往轿内看时,那知不是倌人,竟是坐的一个男子,扶手板也没有,端端正正??坐在轿中。秋谷大为诧异,看那男人时,穿着玄色外国缎马褂,鼻架金丝眼镜,衣裳甚是华丽,帽子上还钉着一块披霞,面上却满面烟色,青生生的甚是难看。獐头鼠目,缩头拱肩坐在轿中,眼睛四围乱转,得意洋洋的神气。秋谷见了这副怪状,忍不住哈哈大笑,心想:天下真有如此寿头码子,真是可笑!轿子刚刚过去,忽听得轿中那人叫了一声:“秋谷兄几时来的?”秋谷不及回答,轿子已折到四马路去了,秋谷听了他的声音,方才想起原来是这个人。

看官,你道这人是谁?原来是常州有名的冤桶瘟生,姓金,号汉良,是个乌龟的儿子。本不姓金,他父亲叫金幼川,因为自家无子,就把这乌龟的儿子抱养成人,便顶姓了金,承受了这金幼川的一分家产。

这金幼川也不是好好出身,本来一贫如洗,在一个徽州汪家管管帐目。可巧这汪家同一个姓申的举人争夺地基,大家告状,地方官判断不来,姓申的就赶到省中,在臬台衙门告了一状。臬台准了状词,提审起来。汪家虽有家财,却是向来胆小,极是怕见官员,又为自己没有功名,恐怕上堂出丑,便害怕起来,要叫这管帐的去顶名冒审。金幼川那里肯去,汪家急了,便许他若肯替代上堂,无论吃苦与不吃苦,总送他一万银子。这金幼川虽然怕打,却是漆黑的眼睛见了白花花的银子,由不得就答应了,跟着差人到了苏州。

不多两天,臬台挂牌提审,先问了原告的口供,再传被告上来。金幼川仗着胆子上堂跪下,臬台把他看了一看,用旗鼓在公案上一拍,问道:“你可就是汪宏超么?”金幼川战抖抖的答应了一声:“监生正是。”臬台又问道:“你这监生是在那一案报捐的,折色几成,可曾领到部照?从实进上来。”两旁吏役齐齐的吆喝一声。金幼川原不曾捐过监生,只道监生是个微末的功名,臬台不致追问,不料臬台认真盘驳起来,他如何回答得出?又被两旁差役喊了一声堂威,愈加慌得六神无主,竟说不出什么来。臬台又拍着惊堂道:“讲!”满堂人役又喊了一声,把个金幼川吓得呆了,一句话也挣不出来。臬台大怒道:“怎么本司问你的话,你竟不回答?

好大胆的奴才,掌嘴!“值刑皂隶轰然答应一声,赶上几个人来,不由分说,把金幼川拿住,一个捺住他的肩头,一个扳着他的脸面,把个嘴巴放得平平的。金幼川听得臬台叫打,已是魂飞天外,魄散九霄,就要喊也喊不出了。早被差役取过皮掌,照着金幼川的嘴巴,一五一十的打了四十,方才放他起来。那臬台堂上的刑法十分利害,这四十个嘴巴,直打得金幼川肿了半边的面孔,就如猴儿屁股一般,牙齿也打了两个下来,满口里喷出鲜血,只把他打得昏天黑地,连他自己的生年月日都一齐忘了,那里还说得出什么话来?臬台又拍案喝道:”看你这般光景,你这功名料想不是真的,本司也没有多大的工夫同你追究,只问你争夺基地的案情,你这欺贫倚富的奴才,为什么去争夺人家的基地?在本司这里好好的供上来,若有一字支吾,你可知道本司的刑法?“

金幼川被他打得昏了,也听不出臬台问的什么话来,只连连磕头道:“监生冤枉,求大公祖明镜高悬。”臬台冷笑道:“还敢自称监生?左右与我结实再打!”

金幼川急了,连碰响头道:“总是小人该死,求大人开恩。”臬台冷笑一声,又道:“本司看你这个样子,就不是安分良民,那强占人家的地方,自然也是有的,你还敢在本司这里称冤道屈么?”只这兜头一盖,把金幼川盖住了,不敢开口。臬台喝道:“快快的供上来!”金幼川只吓得心中乱跳,又不敢再叫冤屈。臬台见他并不开口,发起火来,大声喝道:“我把你这放肆的奴才,你在本司堂上,尚敢如此支吾,你平日的倚富欺人,可想而知的了。”一片声叫看大板伺候,皂隶吆喝一声,便要来揪金幼川下去。金幼川着了急,高声叫道:“求大人开恩饶打,小的愿招。”

臬台吩咐不要动手,等他实供。金幼川无奈,只得胡乱招了几句“不合恃富欺贫,谋占基地是实。”招房录了口供,叫他自家画供,呈上。臬台看了一遍,冷笑道:“本该把你这奴才重重惩办,以儆将来,姑念你在本司这里从实供招,饶你一顿板子,回去好生改过,学做良民,若再有什么案情犯到本司这里,哼哼,那里莫怪本司就不是这样的办法了,下去!”值堂的听臬台叫他下去,齐声吆喝。金幼川只得磕了几个头,走了下来,又羞又气。这里臬台又传了原告上来,将基地断归原告,叫他当堂具领,就此退堂。

原来这臬台也是寒士,科第出身。从前未遇之时,着实被本乡的富户欺凌讪笑,所以做官之后,存了一个偏心:凡是穷人与富户打到官司,到他台下,一定要偏袒穷人。金幼川哪里知道,冒冒失失的顶了汪宏超的名字上去,吃了这一场大亏。当下出了衙门,又羞又气,连夜回到常州。汪家见他果然吃苦,免不得要抚慰他一番,又当真给了他一万银子。这金幼川甚有心计,把这银子同人合股开了一家钱庄,自己辞了汪家出来,就在钱庄管事。不多几年,竟被他盘了一倍出来。

金幼川有了银子,就要摆起臭架子来,家里用了两个粗使的老妈子,买了两个丫头,叫他自己是老爷,老婆是太太,儿子是少爷。把这过继的儿子十分钟爱,延师教读,要想替他光大门闾。无奈这金汉良心地极是糊涂,资质更加愚鲁,整整的念了十五年书,连个之乎者也的虚字,也不曾掉得连牵。这先生明欺金幼川是个外行,不知黑白,对着他反称赞他令郎的学问。金幼川本来满腹草包,那里懂得什么学问,连先生都赞起他的儿子来,可想自家儿子的本事,是大到极处的了。就把他欢喜得手舞足蹈,无可不可,以为儿子指日就是大官,自己就是现现成成的一位老封君了,便拼命的把儿子恭维起来。他这令郎本是龟奴的儿子,自然就带些祖父家风,虽然别的事情一样不会,却偏偏生就一副说大话、吹牛屄的本领,凭你无影无踪的事,他偏会说得确实非常,有凭有据。至于生性的卑鄙,行为的刻薄,便是他的本色,在下也没有这些闲力来一桩一件的形容他。

只说这金幼川巴结了儿子十年,指望自己好做封君,享受他儿子的福气,不料他年纪已高,等他不及,一病死了。金幼川病死之后,他儿子非但不知哀痛,倒反高兴起来,把金幼川辛苦积来的家产随意花销。鸦片烟瘾甚大,每日要吸二两几钱。

同的一班朋友,都是不三不四的人,帮闲蔑片,都跟着他吃喝。正经朋友的面上,却是一文不肯花费,吝啬异常,所以人人都赶着他叫“瘟生冤桶”。他家产虽然不多,却最喜人赞他有钱,夸他豪富。他自己也一天到晚摇摇摆摆的只在街上闲闯,摆着不三不四的架子,打着半南半北的京腔,好像真是世家公子、百万财翁一般。

那一年联军进京,开了捐例,秦晋顺直甚是便宜。他忽然发起官兴来,到处托人替他捐了一个试用知县,加了三班银两,分发直隶。他捐了这个官十分高兴,登时就戴起水晶顶子,拖着一条花翎,每逢城内有什么婚丧喜事,他无论向来认得认不得,一概到场,为的是好摇摆他晶顶花翎的架子。也有几个通品乡绅,见他那种不中款式的样儿甚是可笑,便问他这五品顶戴可是知县上的加衔?他就大声答道:“兄弟这个顶戴,是五年之前山东开黄河口子,抚台奏保兄弟的虚衔。兄弟这个知县倒是在这五品顶戴上加捐的,所以他们这一班新捐知县的人,谁也没有兄弟这个面子。”那问的人几乎笑了出来,知道他是个初出茅庐的人,不好意思同他辨论,只好走了开去,告诉别人,个个把他当作笑谈。他却意气昂昂,毫不为怪。只是他笔下虽然不通,他却自道是个通品,说起话来,满口之乎者也的牵文掉宇,人家都不懂他说的什么。

适值联军已经退出北京,皇上回銮之后,举行乡试,恩科、正科并在一起,那中的额子就有二百余名。他又发了一个奇想,又要想去中起举人来。他本来底子是个监生,现在虽然捐了功名,尚未到省,照例可以入场。金汉良就在本县起了一角送考文书,结了几个同伴径往南京而来,在文德轿左近租了两间房屋住下。转瞬已是七月廿七,便要进场录遗。这金汉良穿了一身簇新的实地纱袍褂,浑身挂着玉器“叮玲当琅”的,又扣着平金的眼镜袋同扇袋,背后飘着两对荷包,而且挂着大长的忠孝带,头上戴着簇新的凉帽,翡翠翎管,拖着上好的花翎,挤进贡院,累得满头大汗。原来学院录遗,也有大员子弟的官卷,也有已经捐过功名的官监,照例多要带着顶戴入场,但都是随身衣服,头上带顶帽子,脚下穿双靴子,从没有像金汉良这样全身披挂的,好似进士谢恩、大员升见一般,大家多望着他好笑。正是:斋

傀儡登台,真个官场如戏;沐猴冠服,果然嫖界新闻。

不知后事如何,请听下回交代。





第十四回 一监生录遗受气 两承差讨赏翻腔





且说金汉良见一班录遗的监生大家都看着他笑,又有指指点点的谈论着他,金汉良那里想到是笑他自己,还认是自家身上衣服过于华丽,所以众人羡慕着他,倒反低下头来,看着自己的衣裳,甚是扬扬得意。

不多时,学台放炮开门,点起名来,那一班监生便一排一排的挨挤上去。点了一会,渐渐点到常州府来,先点武阳二县。金汉良挤在学台的公案旁边,听得点到他名字,他连忙赶到案前,接了卷子。学台见他穿着得袍褂齐整,靴帽时新,头上还拖着一枝花翎,腰中挂的玉器不住的乱响,已觉诧异。到得他缴验官照之时,学台看只有两张部照,没有加衔同翎枝的执照,却见他明明戴着水晶顶子,拖着花翎,心中诧异起来。又恐自己眼花看错,便把鼻上架的大圆老光眼镜往上撑了一撑,仔细再看时,金汉良见学台大人不住的看他,满心欢喜,只道学台有话问他,便朝着公案深深的请了一个安,口中恭恭敬敬的说了一声:“嗻。”引得两旁的承差吏役多笑起来。学台也觉得这个人大有痴气,便也不去盘问他顶戴的来历。好在学台衙门只管录遗,那有什么工夫来管你的闲事?只觉得这个人呆得可笑,却又不好笑出来,失了体制。学台把脸沉了一沉,承差便一齐喊道:“进去进去,接了卷子,还站在这里做什么?”金汉良正是一团高兴,等学台同他谈心,不料被承差赶了进来,讨了个大大没趣,只得走上甬道,一直进文场来,依着卷面上的字号坐了。却只有自己一人,同伴的都不见来。他是做大老官做惯的人,举目无亲的坐着,甚是纳闷。

坐了不多一会,他的烟瘾早已发作。烟具是预先带着,急急的拿了出来,苦的是没有榻床,又且四不住的吹进风来,勉强坐着,上了一口吃了,却是塞了几次,好容易吃完。金汉良平时过瘾,总要大口装烟,一顿要吃一两,这样不爽快的吃法,那里挡得住他的烟瘾?

正在没法,只见一个差官带着几个承差前来查号。原来外面已经封门,两边文场都有学院衙门的差官,同着各学的教官一同查察。那差官看看查到金汉良面前,金汉良一见这个差官,心中大喜,认得他就是同乡的胡养甫,向来晓得他是学院衙门的总书房,便连忙招呼他道:“养甫兄,幸会幸会。”胡总书听得有人叫他,回头看见了金汉良,平日彼此原是相识,便也同他拱手说了几句套谈。胡养甫道:“兄弟还有公事,不能奉陪,改日再叙罢!”便要一直查看过去。金汉良因文场内不能过瘾,心上慌忙,见了胡养甫来,正要托他想法,连忙说道:“养甫兄,且少停一刻,有一件事与你商量,可有什么安稳的地方,可以躺着过瘾的?托你想个法儿。”胡养甫听了,沉吟道:“里面都是关防地方,外人轻易不能进去,兄弟也担不起这个责任。只好我叫两个承差同你到花厅上去过瘾,那里头本有榻床,狠是便当。并好叫他们替你预备茶水,只要你酌量着酬劳他们几个钱就是了。”金汉良听说可以把他同到花厅过瘾,甚是喜欢,忙拱手道谢道:“费心费心,容当后报。至于酬劳,本是小事,兄弟格外从丰便了。”胡养甫谦逊了几句,随叫过两个承差来,向他们说道:“这金大老爷是个慷慨的人,你们领他到花厅上去,让他在炕上吃烟。

回来出了题目牌,你们就送到厅上,好好的预备茶水,伺候金大老爷,等回自有酬劳。“

那学院衙门的承差见钱如命,见金汉良衣服辉煌,又是养甫吩咐,大约总可赚他一注赏钱,就连连答应,领着金汉良到花厅上来,金老爷长,金老爷短,十分巴结,又去泡上好茶,摆出四盘点心。此时金汉良不顾别的,急忙将烟盘放在炕上,横下身去,取出打就的一罐子烟泡,装得满满的,约有三四两烟,装上签子,不问青红皂白,呼呼的先抽了二十来口,方才把他的烟瘾挡了回去,坐起身来,吃了些点心,承差已掮了一扇高脚牌来,牌上写着题目给他看过。

题目虽不甚难,金汉良那里做得出?想了一会,一句也没有做出来,只得翻出来带的书来,什么《宋明四书义》、《东莱博议》、《古文观止》等,看了多时,拣两个牛头不对马嘴的题目,东边抄两句,西边集几句,自己联上些半通不通的虚宇,勉强敷衍了两篇,急急的过了瘾,誊上卷子。时候已经午后,承差格外殷勤,去开出一桌饭来,四样鸡鱼肉鸭,滋味倒也不坏,另外还有一壶酒。金汉良用了心思,正是腹中饥饿,也不推辞,狼飧虎咽了一顿。吃完了,提笔再眷。

写到约有大半,只见两个承差手中拿着一搭收票进来。原来监生录遗,要把监照呈验,验过无误,打一个录遗戳子,候缴卷时,将原照还给本人。这班承差作弊,不于当日交还,于众人缴卷之前,叫众人在收票上注明姓名、籍贯,每人或是一元,或是五角,也要注明数目,仍将这收票交给录遗监生。隔了一日,照着注明的洋钱数目,拿着这张收票去学院衙门取回监照。这是承差舞弊贪财之处。学台明知关防衙门差役异常清苦,故意假作不知,不去禁止。论起理来,也就是驭下不严,辜负朝廷的恩典了。这且按下不表。

再说两个承差手中拿了收票进来,满面笑容的对金汉良说道:“金老爷的官照还没有交回,请在这收票上注明功名姓字,明日好叫人凭票取回,我们还要讨讨你金老爷的赏呢!”说着,笑嘻嘻的请了一个安。金汉良大模大样的点了一点头,接过收票,先写了姓名、籍贯,又注明了功名,写到那洋钱数目的地方,那承差目不转睛看着他写,写好了连忙接过去,看那照费时,只见端端正正的写着,却止一块洋钱。两个承差见了,你看着我我看着你呆了一时,还恐怕他忙中有错,或者写错了,亦未可知。一个承差便陪着笑,仍旧把那张收票放在他面前,说道:“收票上的数目,只怕金老爷写错了,我们靠山吃山,还要你老人家高升一点。”这番说话,在那两个承差也总算小心巴结的了。那知金汉良不知抬举,竟像学院衙门的承差应该伺候他的一般,登时放下面孔,正色说道:“这赏钱的数目,那会写错?本来我们应考的人那有什么赏号?这是我看你小心伺候,所以格外加恩,那里有写错的道理?难道你们还要争多嫌少么?”

两个承差听了,不觉心中大怒。暗想天下有这样不知好歹的死囚,翻转面皮冷笑一声道:“既然你你金老爷看得这一块钱十分郑重,我们虽是当个承差,倒还不至于这般小气,你就请不必花费,留着自己买稀饭吃罢。通共花了一块钱,什么大不了的事,还要说格外加恩!我们学院衙门的人,除了我们大人提拔,才算格外加恩。不是我瞧不起你金老爷,还摆不了这个架子!你自己想想,请你坐了花厅,点心茶水的伺候,还要开出饭来,闹得乌烟瘴气,这一块钱还不够做茶水钱呢!”金汉良听得承差出言不逊,也就大怒起来,高声说道:“学院大人叫你们当差,没有叫你们讹诈。你们勒索考生的银钱,还要辱骂斯文,真是岂有此理!我同你们到学台面前去讲,可是该应这样的么?”两个承差听他索性发作起来,更觉眼内生烟,鼻中出火,劈面朝他啐了一口唾沫,道:“摆你的什么臭架子!像你这样的考生,我们看见得狠多。这是什么地方,容得你这等放肆骂人?老实说,我们小心伺候,一者是胡老爷的吩咐,二者原是巴结你的银钱,点心酒饭,那一样不是钱买来的?

我们倒没有这样老脸去白叨别人的光,只算认一个晦气罢了。你白吃白喝了不算,还要装腔做势的在这里骂起人来!我们当了学院衙门的差,是来伺候你的么?“把个金汉良骂得闭口无言。

两个承差又道:“平常一张监照也要一块多钱,你坐了花厅,伺候你的点心茶酒没有看见你一个钱,倒反说我们讹诈,要同我们去见大人。我们倒底讹诈了你什么?你倒讹了我们两顿酒饭点心去了。你要去见大人,你只顾自己去见,我们候着就是了。我们还有公事,不得同你闲谈,这些考生都要像起你来,一毛不拔的,那我们就要喝西北风了。”说完了,便两人一同出去。一个承差还对他同伴说道:“这个人真是不开眼的东西,我们只当做个好事,给他吃了两顿罢了。”

金汉良明明听见,又气又恼,只好假作不知。心中暗想:虽被这两个承差骂了一场,究竟省了一注赏钱,吃了他们二顿饭点,算起来也还值得。便慢慢的抄完了二篇文字,默起圣谕来。他不知格式,把那一段圣谕直抄到底,竟有十二三行,他并不觉得,转得意扬扬的缴了卷子,出来逢人便说他文字如何好法,必定第一无疑。

别人听着好笑,也不去理他。那知发出案来,单单没有金汉良的名字,金汉良气得发昏,他还不晓得为着什么缘故,急忙去寻着了胡养甫,要他做个手脚把名字补出。

胡养甫见面不免埋怨他几句,道:“那承差原是想你的赏钱,所以出力巴结。

你不肯花钱,还要闹你的标劲,连我的面上也不好看相。那天交照的时候,若不是我在里头,你这几张官照就莫想拿回去了。不瞒你说,我还赔掉好几个钱呢!这都是小事,也不必说了。“金汉良被他埋怨,只得向他谢罪,又把来意说了一遍。胡养甫道:”你的卷子只要没有违式之处,过了几天自然会补出来,不必性急;若是违式被贴,那就难了。我且替你去查查,你在这里少待。“说罢立起身来,去了多时方才回来,皱着眉头,像是有些难处的一般。金汉良就吃了一惊,急问事情怎样,养甫道:”你的卷子是多抄了圣谕,违格贴出的。刚才我查着了你的卷子,竟把一段圣谕通通抄完,多写了七八行,照例不能补出。我看我竟另想法子,我却力不从心,实在对你不起。“金汉良方知是为多抄圣谕,以致被贴。又听胡养甫说不能设法,甚是着急,缠住了养甫,打恭作揖的央求。养甫被他恳求不过,道:”法子是有一个在此,只是我却不能替你赔钱,你自家去酌量而行。“汉良大喜问计。养甫道:”只有替你重换一本卷子,等你重新誊好,把你那一本坏卷换出来,我们在内里做些手脚,就可以挂牌补你名字。但是那班承差恨你入骨,一定要你二百块钱。

你若肯忍着心痛,我便替你包办下来。除了这个法子,没有第二条路。“金汉良听了,呆了一回,虽然舍不得二百块钱,究竟中举人的心重,发了一个狠,咬着牙齿答应了下来,当晚就把二百块钱悄悄去。隔不多两日,果然学院衙门前挂了一面粉牌出来,把金汉良的名字高高补出。金汉良欢喜,收拾进场。

转眼三场已过,金汉良也随众出来,也不知道他做的什么东西,在卷子上写些什么,做书的不曾见过他的场作,不能备载出来。

金汉良在南京耽搁了几日,便回到常州,安心等榜。以为这个举人是捏在荷包里的了,一味的大言不惭,还说他做梦看见天榜,他的名字高高的列在第三。听见的人,付之一笑。等到放榜之期,家里预先染了几千喜蛋,预备榜后送人。不料等了一天杳无影响,听见报子的锣声接二连三的在门口敲了过来,又敲了过去,偏偏的不到金汉良家。眼见得这个举人是没分的了,气得金汉良一佛出世,二佛升天,一天到晚饭也不吃,拍着桌子大骂房官瞎眼,主考糊涂。骂了一会也无可如何,恹恹的过了几天,也就丢过去了。只带着那一班下流社会的人,天天往那妓院烟灯开心作乐,往往的成日成夜并不回家。

隔了一年,忽然觉得常州玩得不畅,他也久闻四大金刚的名气,想到上海来见识见识,住在宝善街新鼎升栈。到了两日,就去寻着了一个书局中朋友,也是常州人,同他向来相识。金汉良央他带着往各处妓院中走动,陆兰芬处也去过两次。兰芬在外出局。没有见他。又到金小宝院中见了小宝,十分倾倒,当夜就要替他摆酒,拿出现钱来。堂子中的规矩,是现钱摆酒不能推却的。金小宝只得让他吃了一台。

四五日之间,也碰了两场和,吃了两台酒。金小宝看得了然,金汉良却一厢情愿,癞蛤蟆想吃起天鹅肉来。小宝却见他满身土气,牛屄倒吹得一塌糊涂,娘姨等人都在他背后指指点点的取笑,也觉得他假作痴呆,甚是讨厌。而且这金汉良打茶围没有时候,每每天未到午,他已经踱了进来;坐下了,又夹七夹八的不肯走。小宝满心不悦,却又不能回他,看他那啬钝的情形,料不是出钱的阔客,所以大家心里都在恨他。这一日才打十一点钟,小宝还未起身,金汉良已经来了,坐在小宝房中,娘姨把小宝叫将起来。正是:

承差讨赏,才闻狼虎之声;曲院寻欢,又惹莺花之笑。

不知小宝说些什么,请看下回便知分晓。





第十五回 曲辫子坐轿出风头 红倌人有心敲竹杠





且说金小宝被娘姨叫了起来,见了金汉良坐在房中,冷着面孔,冷笑道:“金大少耐倒直头来得早笃,区得倪呒拨客人。”金汉良还不晓得是骂他的说话,并不理会。坐了一回,一个小大姐进来向小宝道:“轿子搭得来哉,阿要请先生自家去看看?”汉良忙问谁的轿子?小宝没有睬他,便蓬着头走下楼来去看轿子,汉良也跟着下来。只见一乘金碧辉煌的轿子,停在客堂里面。原来小宝因轿子已经半旧,特地花了一百四十块钱糊出来的。这乘轿子,金汉良看了连连称赞,道:“好齐整的轿子,可是你坐的么?”小宝不应,只微微的点一点头。

汉良看小宝这乘轿子十分华丽,忽发一个痴想,要坐着他的轿子到马路上去出出风头。他的意思是要叫马路上的人,看他坐着红倌人的轿子,这倌人同他必定有些交情,想要夸耀路人的意思。便向小宝道:“你的轿子果然精致,可肯借给我坐一天,出去拜拜客么?”小宝听了大为诧异,答道:“倪格轿子,唔笃得勿好坐格啘。”旁边一个娘姨急在后拉了小宝一把,使个眼色,接口说道:“倪先生格顶轿子,自家朆坐歇格勒,第一转等金大少坐仔去末,再好勿有,让俚笃相帮也好问金大少讨点赏钱。”小宝听了微微一笑,便不开言。

汉良见小宝允了,大喜,连忙叫了抬轿的相帮说知原故。相帮们一齐好笑,却乐得弄他几个赏钱,就绰出轿子。汉良坐进轿去,小宝看着这般怪状,忍不住格格的只待要笑。相帮将轿子抬上肩头,问汉良抬到何处,金汉良便叫一直到新北门进城拜客,那轿子便如飞的直过四马路来。在路口无意之中遇见了秋谷,便在轿中叫了一声。及至轿子进城之后,相帮问他拜什么客人,他却又无客可拜,吩咐相帮抬出小东门,一径回去。相帮抬着他空走一回,真是可笑。暗想:从没有看见这样曲辫子的客人。路上的人见了,大家拍手笑他,金汉良毫不在意。一直抬着仍到金小宝院中来。

汉良出轿上楼,便问金小宝,“你的相帮抬我一趟,约莫要赏他几块钱,小宝却正色说道:”倪堂子里向格规矩,换仔轿子第一转坐出去,相帮笃才要问倪讨赏格,故歇耐金大少来替倪开销,真真请也请耐勿到。俚笃抬着仔耐金大少,是俚格运气来哉。“倪平常日脚末赏格几十洋钱,耐金大少多赏点末,顶好哉啘,随耐金大少自家格心浪。”金汉良被小宝一番话说得呆在一旁,不敢开口,不想小宝开出这个大盘子来。尚未回答得出,小宝又接口说道:“像耐金大少格牌子末,至少赏格四十洋钱,再多末也可以勿必格哉。”说着,便看金汉良的面色。汉良依然答应不出,小宝又道:“金大少身浪呒拨洋钱末,倪有来浪,倪替耐垫仔一垫罢。”不由分说,即在枕旁一个大大的皮包内取出一大卷钞票来。金汉良吃了一惊,暗想:他那里来的这许多钞票?偷眼看时,只见小宝将一卷钞票打开,却都是一百元一张的,汉良更加吃吓,估量那一卷足有一百多张。又见小宝仍把这一卷放入皮包,重新又取出一卷来,方才检着十元的钞票,检了四张交在娘姨手内,向他说道:“格个是金大少格赏钱,耐去交拨俚笃,叫俚笃上来谢声。”娘姨答应出去。不多时,带了三个抬轿的相帮上来,对金汉良谢了一声,便都下去。

金汉良满心懊恼,却说不出口来。好一会,才问小宝说道:“怎么我坐了一趟轿子,就要赏这许多?”小宝冷笑道:“格是耐金大少自家格场面啘。老实说,上海滩浪要出来白相,顾勿得啥铜钱。倪堂子里向加二才是铜钱格世界,倪为仔耐金大少是格体面客人,所以替耐装装场面,故歇耐舍勿得末,倪倒拿子出去,坍勿落格个台,就算仔倪格末哉。倪多末勿成功,四十块洋钱格东还作得起。金大少,耐勿要放勒心浪,倪倒也勿在乎此格。”金汉良听他话中有刺,看得他不值一文,羞得满面飞红。娘姨大姐等又在旁边冷言冷语的取笑,再坐不住,只得立起来要走。

小宝并不相送,随他下楼而去,这且不表。

再说秋谷走到书玉院中,春树与书玉刚刚起身,书玉正在梳洗。秋谷一见,便向书玉说了一声:“恭喜!我这媒人做得如何?”书玉瞟了秋谷一眼,低头而笑。

秋谷将厚卿的钞票交给书玉,书玉接了,称谢秋谷费心。春树便与秋谷长谈起来。

书玉在旁静听。只听秋谷道:“你的事情,我虽然已经答应,然而不能立刻就去,总要等我上海回去,方能径到苏州,大约不至误事就是了。但是你的朋友也不止我一人,难道竟没个有些热血的,偏偏将这样的好差使硬栽在我的身上,这不是无妄之灾么?”春树道:“我的朋友虽然甚多,那里有你这般的意气?他们这一班现在的朋友,平常时候倒也说义谈忠,十分要好,一到那有事之时,或是问他借钱,或是要他出力,他就缩起头来,躲得你远远的,影子也寻不着他。如今世上这朋友一伦,是可以不讲的了。你是近今有名的黄衫客古押衙,所以特地前来寻你,料想只有你还可以商量,别人那里担当得起?你务必要替我设个法儿。”秋谷大笑道:“言重之至,当不起,当不起!请你少灌两句米汤罢,怎么把我近今的一个人,去比起古时剑侠来,岂不是刻划无盐、唐突西子?”说得春树也笑起来。又问秋谷几时回去,秋谷笑道:“怎么你这般性急?我此次来沪有些正事,大约还要耽搁月余。

你若等不及,就去托别人如何?“春树忙分解道:”并不是我性急,只是我虽然走了,却实实的不放心,恐怕日子长了,弄出事来,我怎的对人得起?“秋谷道:”看你不出,倒是个多情种子。但是耽搁月余,料想还不至误你的事。“春树听了点头。

张书玉在傍,听他们一问一答说得热闹,却是没头没脑,一句也听不出来,忍不住在旁问道:“唔笃说仔半日,倪一句也听勿出,倒底啥格事体介?”秋谷、春树一齐笑而不答。书玉又问了两声,秋谷道:“不关我事,是你们的贡大少做的事情,你去问他就是了。”书玉果然走到春树身旁,低低的问他道:“倒底啥格事体?

替倪说嗫!“春树攒眉朝他摇头道:”此刻不便,停会再和你说。“书玉见他不说,也无可如何,口中咕噜了两声也就罢了,只在自己腹中猜想他们这个闷葫芦。

看官且住,不要说张书玉在那里猜想,就是看官料想也在腹中猜想。做书的在下心中虽然明白,却不好直说出来,要留着这个波澜,做那文章的曲折。看官们暂时掩卷平章,等到《九尾龟》后集出来,自然明白。并且在下这书,名目叫做《九尾龟》,原说是一个富贵达官的小影,怎么平铺直叙到了第十五回,还没有提起一字,只把那章秋谷一人颠来倒去说个不了,说的又都是苏州、上海的繁华,名妓金刚的小影,这与《九尾龟》的正文有什么干涉呢?须知在下这前半部小说,原名叫做《嫖界醒世小说》,不过把九尾龟做个提头,下半部方是《九尾龟》的正文。只因限于篇幅,所以把一部小说分做两段出来,并不是在下脱枝失节。

闲话休提,书归正传。且说秋谷同春树谈了一会,秋谷笑道:“我今日看见一桩笑话,真是奇谈。”就把在大新街遇见金汉良坐着倌人的轿子在四马路过去。

“他还在轿中招呼了我一声,天下竟有这样士气的人,你道可笑不可笑?”春树听了笑不可仰,张书玉也笑起来。春树道:“这个人本来是个出名的寿头码子,现在忽然跑到上海来出起风头来,正不知以后还要闹出多少笑话呢!我们只打点着耳朵听就是了。”

大家又笑了一会。春树问秋谷:“可有什么事情,我们去吃大菜可好?”秋谷点头,当下二人就同着张书玉到一品香去。吃完了大菜回来,已是家家上火。春树便要秋谷同他到有名的红倌人处多打几个茶围。秋谷微笑,拍着张书玉道:“他这不是个红倌人么?你还要另外去寻别人,真是岂有此理!”书玉被他说得一笑,回道:“倪是勿好格,耐勿要钝。”却把眼望着春树。春树便向秋谷道:“我要你同去打几个茶围,是不过去见识见识,并没有别的心肠,你就说出许多牵枝带叶的话来。”秋谷哈哈大笑,对着春树把手在自己面上捋了一捋,道:“算了罢,你不用和我支吾。”又向书玉道:“你只管放心,等他出去走走,有我这保镖的跟着他,包你没人抢夺。停回晚上我亲送他来此,如何?”书玉面上一红道:“耐末总无拨好闲话,阿要瞎三话四。”说着,忍不住也笑了。秋谷道:“我原是走你的心经,你倒不见我的情,还叫我没有好话,真是好人难做。”一面同了春树走出院中,顺便先到陆兰芬家。

兰芬却好在家,见了春树暗暗喝彩,那面貌竟与秋谷不相上下,只是秋谷丰采惊人,风华出众,比春树的一味柔弱,又觉较胜一筹。略坐一会,秋谷见兰芬房间甚忙,便起身辞去,又到金小宝院中来。

秋谷走进客堂,一眼就看见小宝那乘轿子,便指给春树道:“日间看见金汉良坐的就是这乘轿子,想必他做的是小宝,不知小宝待他何如?”一面说,走上楼梯,直到小宝房中。小宝与秋谷本来相识,便含笑相迎。刚刚坐下,秋谷猛然笑道:“我们今日特地到你这里烧香,快点起蜡烛来。”小宝虽也晓得秋谷定是取笑着他,却摸不清头路,呆呆的看着他。秋谷又笑道:“你这里新近到了一个土地客人,你岂不是个土地奶奶?我们是到土地庙来烧香的,你还不点起大蜡烛来么?”小宝方才明白说的是姓金的客人,便也笑道:“随便啥格闲话,到仔耐嘴里向末就变坏哉,格个客人唔笃阿认得俚介?”秋谷道:“非但认得,而且还看见他坐你的轿子。”

小宝笑道:“阿唷!信息倒灵笃啘!俚坐仔倪格轿子,倒来问起倪来,说相帮笃约摸要赏俚几化洋钱,拨倪敲仔一记小小里格竹杠,相帮笃倒弄仔四十洋钱。耐想格号人阿要讨气?倪上海滩浪住末住仔几年,客人也见得勿少哉,格种曲辫子,倪倒从来朆碰着过歇。”秋谷笑道:“这点小事算得什么。你还没有晓得他向来的历史呢!”就将金汉良以前所作所为极可笑的事情,一一的演说出来,把个金小宝笑得如花枝乱颤,伏在桌上气也喘不过来。

春树见小宝笑得红潮晕颊,俊眼流波,娇小玲珑,动人怜爱,比张书玉大是不同,便细细的看他。小宝住了笑,坐在榻上掠着鬓脚,也抬头打量二人。秋谷是素来认得,不必说了;看了春树,朱唇粉面,那相貌竟同大家闺秀一般,也觉脉脉无言,芳心自动。后来小宝与书玉二人,为着春树,几乎闹出绝大风潮,后文自有交代,此处一言表过不提。

且说秋谷又问小宝道:“这样的客人虽然可恶,你这一下竹杠也敲得太凶,留着他吃吃酒碰碰和,也是你的场面,为什么一定要吓得他不敢再来呢?”小宝笑道:“二少,耐朆晓得格当中格道理,倪告诉仔耐末就明白哉。俚耐一干仔,也替倪装勿啥出格场面,加仔格排常州客人格辫子,就是勿曲末也有点湾湾里格。倪拨俚吵勿清爽,闹得头脑子才痛格哉。格号客人勒倪房间里向摆酒碰和,勿要说替倪绷啥格场面,连搭仔倪格抬才拨俚坍完格哉。”秋谷听了,狂笑道:“骂得畅快,真是雕心镂肺之谈,也等那班曲辫子的客人听听,好叫他们知难而退,才晓得你们四大金刚的院中,不是他们可以轻易踏得进的。”说着,把春树肩头一拍,道:“你这个常州客人,可听见么?”春树不觉面上一红,道:“别人拿我们常州人取笑,也还罢了,怎么你也说起常州人来?”

小宝听得春树是常州人,甚觉不好意思,忙向贡春树陪笑道:“大少勿要生气。

倪说格是姓金格客人,耐勿要听章二少格闲话。“说罢,向春树嫣然一笑,笑得春树神志荡然,细细把小宝恣意看了一会,觉得他无处不好。正是:从脚看到头,风流往上流;从头看到脚,风流往下落。便向秋谷道:”我有一件事情却不明白,要来请问你,你可说得出这个道理么?上海的倌人声价,名妓平章,出于众口。那相貌好的红倌人不用说了,自然是有目共赏,众口交称,一登龙门,声价十倍。最可怪的是那一班自抬声价的倌人,相貌极是平常,酬应更无可取,偏会走着运气,无缘无故的红起来;又自然有那班瞎了眼睛的人当他是个名妓,倒去巴结着他,好像不是他去用钱,倒是倌人倒贴一般,你道诧异不诧异?这还说是烟花曲院,没有什么定评。我所最不解的是一样一个人,我看着他竟是越国西施,你看着却是东邻嫫母;或者你看着就是赵家飞燕,别人看着却竟是齐国无盐。同是一双眼睛,怎么眼中的妍媸好恶就这般的各别,还是真个是没有凭据的呢?还是依着那稗官小说,世间男女都是月下老人注定的前缘,所以分辨不清的呢?你向来自诩是个聪明绝顶的人,你且演说演说这个道理。“章秋谷言无数句,果然说出一篇闻所未闻的道理来。

正是:

一曲琵琶之恨,名士多情;十年歌舞之场,秋娘未老。

未知秋谷如何回答,且听下回。





第十六回 论妍媸畅谈电气 谈嫖界痛骂官场





且说秋谷听了春树问他的说话,嗤的笑了一声,道:“亏你平时还自命通人,怎么迷信起稗官野史家的话来,连这点道理都分解不出?你想月下老人有什么凭据,又有谁人见过?世界上的男女千千万万,婚姻配合那里捉摸得住?都要一个个注起册来,这月下老人如何有这许多手脚?再说起众人的公论来,同是一双眼睛,又同是一付面貌,怎么妍媸好恶截然不同,这究竟是个什么缘故呢?也不是什么偏见,也不是什么前缘,是男女身体之中各人天生的一股电气。大凡人的性情面目各有不同,那禀赋的电气也就不同。合着电气的,看他就是西子南威;合不着电气的,看他便是东施嫫母。那电气又怎的会合呢?将男女二人的电气比较起来,差不多的性质,所以那电气热度高的,便喜欢面有春气、温和柔媚的人;电气热度低的,便喜欢清洁俏俐、一团秋气的人:这是男女电气的大概了。还有那一种男女,初时两情相爱,电气原是相合的,后来忽然两下变心起来,这是各人的电气慢慢的改了性质。

就如人的气血一般,也有少年时本来强壮,到中年忽然无故衰疲;也有少年时本是衰颓,到中年忽地变成强壮。气血既然改变,电气也自然慢慢的不同。无论什么丑陋的人,他的身体之中自有他本来的电气,天下之大,总有同他合着电气的人,所以齐国无盐人人唾弃,齐宣王倒反将他立作正宫,这就是合着电气的证据。齐景公宠幸弥子瑕,初时十分相爱,后来弥子将近中年,景公见之,如有芒刺在背,这就是电气先后不同的证据。总之,电气相同,便一颦一笑俱觉生妍;电气不同,便一举一动也觉生厌。这是说各人眼界之中,另有一番境界,有时可以为凭,却又不能一定。在你看这个人是国色天香,笑着别人没有眼力,焉知别人看他不是个蛇神牛鬼,也在那里笑你的眼界不高。这又从何说起呢?至于上海的倌人声价,名妓品评,却不是这般讲究,另有一番可笑的情形。大约现在的嫖界,就是今日的官场,第一要讲究资格,第二就是讲究应酬,那‘色艺’两字竟可以不讲的了。资格熬炼得年深月久,声价一定会高;应酬习学得圆到随和,生意自然会好。就有一两个色艺俱佳的人,到了这种昏天黑地的地方,也不得不学些应酬,熬些资格,忍着一肚子的气,去同那猪狗一般的客人、夜叉一般的同辈勉强周旋,真正屈杀了许多女子。这才是佳人名士,同一伤心。“

秋谷说到此处,早不觉引起他的牢骚来,春树也默然相对,觉得大有天壤茫茫之感。回头看金小宝,呆坐在旁,听着秋谷说的,一字一句都打入自家心里,想起当年的情景,竟是流下泪来。再听秋谷说道:“最可恨的是这班瞎眼聋耳的客人,他也不晓得‘色艺’两字是个什么东西,只看见这个倌人声价高抬,他便道他一定是才貌双全的名妓,花了大把的银子去巴结他。那真正有些才貌没有名气的倌人,他正眼也不去看他一看。你想,还有什么公论么??小宝拭泪,向秋谷说道:”二少格闲话一点勿错,倪刚刚出来格辰光,勿懂啥格应酬,生意末呒拨,节浪向总归极煞快。看看别家格倌人面孔生得怕煞,生意倒好得野哚,碰和吃酒闹忙得来,格当中啥格道理,倪也解说勿出。直到过仔几年,生意也慢慢里好哉,名气也慢慢里出哉,到仔故歇辰光大家才晓得上海滩浪有倪格金小宝格名宇。倪人末还是从前格人,勿见得换仔一只面孔,想起倪归格辰光真真作孽。二少耐想上海滩浪格事体,阿有啥淘成?倪也不过是得过且过,混混哉罢。“秋谷点头称是,叹息不已。

春树道:“你这一番议论,真是绝后空前,未经人道,实在佩服得很。但是倌人的难处,你也说得切当不移。你又没有做过倌人,怎么这般明白?还是有人同你说过的呢?”秋谷微笑道:“我这般的苦口提撕,开你的见解,你反取笑起我来。

我章秋谷歌场酒阵,整整混了五年,难道这点阅历工夫都没有,定要像着你们遇事绝不经心、出口便谈市语的酒囊饭袋么?“春树笑道:”骂得结实。但是如今世上,像我一般的人在在皆是,而且未必如我一般,你何不一个个去寻着他们痛骂,却单在这里骂我一人?这就是你的不公之处。“秋谷道:”我原是借你一个骂着众人,也不是一定骂你。至于那些更不如你的人,是天生的没有意识、不生气血的畜生,那就无从骂起了。“春树道:”你一概骂在里头也是情愿,但是竟把他们比做畜生,未免过于挖苦。“秋谷道:”我把他们比做禽兽,还把他们的程度看得高了,觉得有些拟不于伦。你想羔羊跪乳、鼹鼠成群,虽是禽兽,也还都有孝义之心。他们这班混帐东西那里赶得上禽兽,你还嫌我过于挖苦么?“一席话说得贡春树咨嗟不已。

秋谷因辛修甫请春树在西安坊龙蟾珠家吃酒,要他作陪,略歇了一会,便辞了小宝,同春树到西安坊来。到了院中,辛修甫同了章秋谷等走进房间,龙蟾珠也来应酬了两声。春树看蟾珠淡扫双眉,轻施朱粉,穿一件素缎夹袄,面目之间颇有清气,便称赞了几句。到得写起局条,秋谷自然是陈文仙了;要叫春树去叫书玉,春树不肯,叫了金小宝。秋谷道:“你这个人,真是得陇望蜀。你还没有晓得他的脾气,将来若是被他晓得,必定要闹出笑话来。”春树看着秋谷,似信不信的摇头不语。正值相帮递上手巾,秋谷也没工夫再说闲话。

局条去了不多一刻,叫局的相帮未曾回转,金小宝早已姗姗而来。走进房门,香风已到,那几步路儿放出全付的身段来,走得十分圆稳。走到春树背后刚刚立住,觉得有些微微娇喘的样儿,一手掠着鬓发,一手扶着椅背,抬起一对秋波将座上的客人四围飞了一转。众人觉得金小宝这双俊眼如秋月光明,如宝珠闪烁,一顾一盼华彩非常。当下小宝笑容满面,一一招呼,又向秋谷应酬了几句方才坐下,回头向着春树低鬟微笑。春树大喜,待要和他说话时,小宝却又扭过头去装作不知,只低头敛手的弄手帕子,却时时飞出眼风暗中关照。合席人的眼光都注在他的身上,暗赞小宝的场面工夫真个是八面张罗,满场飞舞。秋谷更是击节叹赏,忽向小宝道:“我同你虽然认识多年,局却不曾叫过,今天我竟要借光转一个局,不知你赏光不赏光?”小宝笑道:“二少笑话哉!只怕耐勿肯照应倪啘,阿有啥倪倒勿肯格?”

随叫跟局的大姐把豆蔻盒子放在秋谷面前,随向春树说了一声:“对勿住!”便坐到秋谷背后来。秋谷同他谈谈说说,甚是投机。

小宝向来敬重秋谷,况且秋谷的神情意气身段都比春树较胜一等,小宝自然愈加亲热。在秋谷意中又另是一个念头。那一班现在有名的时髦倌人,个个都晓得章秋谷的名字,而且待他要好非常,却并没有什么邪念。大抵秋谷聪明绝世,意气如云,陈王八斗之才,李泌九仙之骨;又且花丛阅历已有数年,那班名妓金刚倾慕他的才华,想望他的丰彩,大家传说,到处承迎,秋谷却只是淡淡的交接,从没有迷恋过什么倌人,这也就算是他绝大的定力,真是庸中佼佼,铁中铮铮的了。一言表过不提。

只说秋谷与小宝谈了一会,陈文仙也走了进来。春树暗想:文仙见了小宝定要吃醋,要看秋谷怎样调停。谁知陈文仙醋意毫无,仍是笑盈盈的打起精神应酬秋谷,秋谷与小宝说得正是闹热,不甚理会于他,陈文仙也没有一毫怒意。春树暗暗希奇,想秋谷拿人的手段真是利害。正在暗想,仰正等所叫的局已是接踵而来,春树一个个看时,也有相貌好的,也有相貌平常的,却没有十分粗蠢的在里头。那些倌人看见秋谷、春树这样两个临风玉树的少年,未免有情,大家多要飞他两眼。小宝因堂差甚忙,相帮来催了几次,秋谷叫他快些前去,小宝尚在俄延,秋谷道:“我们不是曲辫子的客人,你尽管去罢。”小宝一笑,方才辞了秋谷,又向春树招呼了一声,斜扶着大姐金妹的肩头,好似风吹杨柳一般一步步的挨出门去。跨出房门,那眼波正与秋谷打个照面。恰好秋谷眼光一转,也飞到小宝那边,同小宝那一对水汪汪的秋波碰了一个针锋相对。小宝登时红潮晕颊,似笑非笑的斜睨了秋谷一眼,急急别转了头下楼去了。这里众人并未留心,不曾看见,只有陈文仙坐在秋谷背后看得分明,忍不住低叫一声:“好呀!”秋谷急回头示之以目,文仙会意,微笑不言。

秋谷因要早些回栈,还有分拨的事情,便先起身辞了主人,到陈文仙处坐了一会。文仙知他有事,也不留他,秋谷便回吉升栈来。

到了自己房间门首,只见隔壁一间福字官房已经有了客人,那说话的声音夹着些妇女的口气,一口杭州说话,清脆异常。秋谷心痒起来,且不进房,隐在隔壁房间门外,悄悄的在门帘缝里偷看时,只见房内床横头放着五六只皮箱,床上挂着一顶湖色绉纱的帐子,行装甚是辉煌。床上放着一??烟具,明晃晃的点着烟灯,那男人躺在床上吃烟,看不见他什么面貌。一个二十五六岁的女子坐在对面床沿,神情流动,意态鲜妍,眉目清扬,身材纤巧,穿一件杨妃色绉纱紧身夹袄,蜜色绉纱裤子,一双红缎弓鞋约有四寸。看着这身打扮,更觉动人,想是临睡卸妆,所以只穿着这一身小衣服,衬着这酥胸玉腕,粉颈香肩,越显得态度温存,丰姿妩媚。秋谷看了一回,觉得这女子风头甚好,竟和陈文仙差得不多;同苏州的许宝琴、花云香比较起来,却也不相上下。秋谷再要看时,只见那男人坐起来,“噗”的一声吹灭了烟灯,就走来关门。秋谷恐怕被他看见,急忙缩进自己房中。听见“呀”的一声,想是把门关了,秋谷回房,坐在灯下想了一回,也就睡了。

明早十点钟刚刚打过,秋谷起来,还未洗面,忽见茶房领进一个人来,灰布袍子,天青背心,脚下蹬着黑布快靴,手内拿着一张名片,向秋谷道:“家爷过来奉拜。”秋谷不知是什么人,接过名片看时,写着“王保建”三字。正在疑惑,客人已经进来,穿着一件银灰绉纱夹衫,玄色外国缎马褂,跨进房来,对着秋谷就是深深一揖。秋谷忙还礼让坐,家人送上茶来。秋谷问他来历,方晓得他号叫云生,安徽人氏,就是间壁房间的客人,是个浙江同知,向在杭州候补,此番同着如君到上海苏州游玩,因上海没有熟人,要结交几个相识。原来秋谷昨夜窥见的妙人,就是这王云生的姨太太。秋谷见他语言伶俐,应对圆融,觉得这个人也不甚讨厌,便随口也敷衍了他几句,送他出来,当时就过去回拜了一趟。王云生把秋谷十分巴结,秋谷却只是想着那女人的面貌,要想个法子见他一见,却又想不出什么主意来。

次日,王云生来请秋谷吃酒,在公阳里林桂芬家,秋谷欣然赴席。正是:

酒绿灯红之夜,别有深情;征歌选舞之场,忽逢局骗。

下文章秋谷识破仙人跳,张书玉大闹味莼园,倒脱一靴,两番骗局,康伯宣帷薄不修,留学生弹打章秋谷,这些情节都在下回交代,此时只好暂停演说,下回再续《九尾龟》的正文。不知王云生请秋谷赴席,后来究竟如何,请看后集分解。





第十七回 吃花酒初遇假同知 讽官场怒嘲真令尹





且说前集中章秋谷住在上海吉升栈内,无意中结识了王云生。那王云生把秋谷十分巴结,百倍恭维。秋谷觉得云生这人并不十分讨厌,且又极会凑趣奉承,便渐渐地与他莫逆起来。但秋谷那夜间隙偷窥,看见王云生的姨太太虽然年近三旬,却是生得娇媚非常,风头甚好。王云生住的房间,又与秋谷的房间只隔一重板壁,偏偏这位王姨太太行为放诞,举止风流,每常趁着王云生出去、秋谷在栈的时候,他偏要走到房门口来,合那带来的娘姨说长道短,卖弄风情;又常常到秋谷房间门口偷窥秋谷。这章秋谷是个脂粉丛中的老手,未免也要领领他的盛情,虽然言语未通,却已两心相印。正是:

高唐旧梦迷神女,巫峡新欢隔楚王。

闲语休提,书归正传。只说那一天王云生在公阳里林桂芬家摆酒,专请秋谷、春树二人。恰好春树正在秋谷栈中,两人不等他催请,便同到公阳里来,寻着了林桂芬的牌子,问了房间。相帮说在楼上,二人缓步登楼,王云生早迎出房门,笑容满面的招呼二人进去。秋谷当先,春树在后,进得房来,举眼一看,先有三四个面生客人坐在房内,秋谷一一招呼。那四位客人,一个姓宋,号叫伯容,自己说也在浙江候补,与王云生却是同寅。一个姓朱,号惠甫,是上海城内有名的富户。那两个是胞兄弟,一个叫施理仲,一个叫施务仲,也是安徽人氏,现在上海开着厚德钱庄,恰都是语言无味、目不识丁的人。秋谷觉得他们的谈吐甚是浊气,眼中便有些看不起他,随便坐下。林桂芬出来应酬了一遍,秋谷看他的相貌甚是平常,心中不解王云生为什么要做这样的倌人。

正在心内转念,忽见后房走出一个十七八岁的绝色大姐来,瓜子脸儿,长挑身材,穿一件湖色熟罗夹袄,玄色皱纱裤子,一双不到五寸的金莲,穿着宝蓝缎子白绒钱挑绣的鞋子,长眉掩鬓,笑靥承颧。流光欲活,眼含秋水之波;弱燕惊风,腰似汉宫之柳。秋谷见了,不觉吃了一惊,便打着苏州白赞道:“阿唷,电气灯来哉!”

那大姐听见有人赞他,方才抬起头来,恰恰与秋谷打了一个照面。见秋谷衫裳倜傥,举止安详,目光眉彩,奕奕照人,眼光也定了一定,微微的笑了一声。秋谷早立起身来,携着那大姐的手,问他叫什么名字?那大姐回头一笑,答道:“倪是呒拨名字格。”王云生在旁,代他说道:“他叫做阿媛,来得不多几时,上节是在中尚仁金寓的。秋翁,你看相貌如何?”秋谷笑道:“我在上海看见了无数的娘姨、大姐,却从没有遇见这样一个人,直是天上神仙,人间珠玉。”

阿媛听秋谷将他极口称扬,心内虽是十分欢喜,却被众人视线所逼,面上觉得不好意思起来。想要洒脱秋谷的手跑了开去,怎奈秋谷紧紧携住他的纤腕,细细的打量他,那里洒他得脱?阿媛面上更加红晕起来,只得低低向秋谷说道:“勿要实梗嗫,阿要难为情。”众人听了,轰然都乱叫起好来。秋谷一笑,放了阿媛的手,阿媛早一溜烟仍旧跑到后房去了。王云生还恐秋谷动气,向秋谷说道:“这孩子到底年轻,不懂顽笑,等我去叫他出来。”秋谷连忙止住,大笑道:“你做的地方我来割了你的靴腰,你不吃醋也就罢了,倒反帮起我的腔来,只怕你这个贤惠觉得过分了些。”说得众人哈哈大笑,云生也笑道:“我是好心照应,你倒取笑起来。”

说话之间,那阿媛又在后房跑将出来,也不言语,坐在床边一张凳上,眉眼之间,总觉得与秋谷有些关会,若离若合,脉脉含情。秋谷也默坐不语,暗中领略。王云生同那一班朋友都是粗人,那里看得出来?只有贡春树在旁看着含笑点头。直至又有客人,方才打断。

秋谷立起身来看时,只见门帘起处,早走进一个客人,年约三十余岁,衣裳甚是时新,深目高鼻,尖嘴寡腮,走进来似招呼非招呼的向秋谷点一点头,也不作揖,大模大样的便向炕上坐下。秋谷见他这傲慢的样儿,心中十分有气,不去理他。王云生过来张罗道:“这位邵大令是吴淞钓船委员,台甫是允甫二字。”秋谷不应,只从鼻子管里哼了一声。云生又向那邵允甫通了秋谷的姓名。略坐了一会,摆好台面,起过手巾,大家入席。

云生本来要让秋谷首座,只因邵允甫是个本省的候补官员,又与他认识不久,便虚让了他一声。那知他竟不推辞,居然得意扬扬的坐了首席,只向秋谷微笑,道声:“有僭。”秋谷见他进来的时候目中无人,已是可厌,又见他占了首席,那有好气答他?秋谷便勉勉强强的坐在邵允甫肩下,贡春树坐了第三,其余众客以次坐定。林桂芬斟了一巡酒,唱了一支京调,一支昆腔。

秋谷叫的陈文仙却第一个先到,便坐在秋谷身后,低问他为甚两日不来,可是身体有些不快。秋谷道:“我因前两日应酬多了,把正事搁了下来,这两日在栈中料理事情,没有工夫出去。”文仙点头,便拉着胡琴唱了一支小调,对秋谷道:“前日仔倪勒浪一品香出堂差,拨格断命客人灌仔几杯酒,格两日喉咙唱勿出哉。”

秋谷皱眉道:“你既然喉咙不好,何必一定要唱呢?”两人凭肩私语,情致缠绵。

不多一刻,春树叫的金小宝也来了,穿一身湖色缎子绣花的衣裤,越显得宜嗔宜喜,如花如玉。刚刚坐下,便问秋谷道:“二少,耐阿晓得张书玉要替倪翻腔?”

秋谷诧异道:“我又没有同你到书玉院中去过,怎么晓得你们的事情?春树为什么口多不开,没有同我提起?”回头便向春树道:“何如,我早晓得你们这件事情,迟早总有一个乱子。”春树觉得有些惭愧,俯首无言。金小宝又告诉秋谷道:“格个张书玉,实头勿要面皮,几转叫娘姨到倪搭来,要请贡大少过去。倪回报仔俚勿勒浪,俚就一直闯到仔格房间里来,刚刚拨俚撞着,拨倪翻转面孔来说仔一泡,难末格个张书玉恨伤仔倪,说倪抢仔俚格客人哉,要来替倪讲理性。二少,耐想想看,阿有格号道理?真真是上海滩浪少有出见格事体。”

秋谷正要回答,王云生做了主人,要搳一通关,便把秋谷话头打断。秋谷打起精神,搳了五拳,秋谷输了两杯,便一气饮干。王云生完了通关,邵允甫鼓起兴来便要摆庄。云生道:“不必一定摆庄,也搳了通关罢!”允甫依言。原来那邵允甫酒量极大,叫娘姨拿了几只大玻璃杯出来,那杯子一杯大约可盛十二两酒。邵允甫先从秋谷搳起,秋谷无奈,推辞不得,只得也同他搳了五拳,恰是秋谷输的,邵允甫便送过一大杯酒来,陈文仙伸手过来想要拿去代吃,早被邵允甫一手按住酒杯,道:“不准代酒,代者要罚十大杯。”文仙便缩住了手。秋谷赌气取过酒杯,一口气灌了下去。那知秋谷吃得太急,又是热酒,登时呛得咳嗽起来,吃了几口茶,方才慢慢平复。秋谷本来甚是鄙薄这位邵大老爷,又听他开口抚宪,闭口藩台,更是心中厌恶,忍不住向邵允甫笑道:“老公祖是个官场,兄弟恰有一个官场笑话。你们贵省湖南从前有一位抚台,是翰林出身,侍郎外放,性情蕴藉,极爱诙谐。有一次这抚台出省阅兵,阅到常德府属,恰好这常德府知府和抚台是同年同学,又是同乡,一向顽皮惯的。抚台阅过了兵,这位府尊就请他署中安息。抚台因同他是多年旧友,十分隆重,欣然答应,便到府署中来。吃过午饭,抚台换了便衣,同常德府到大堂闲走。忽见那大堂旁边竖着两块石碑,约有一丈多高,下面驼碑的乌龟雕得甚是工细,高大异常。抚台看了一会,忽向常德府笑道:”这个乌龟雕得工细非常,大约老兄一府之中,要推这乌龟第一的了。‘常德府也笑道:’回大帅的话,这外乌龟岂但是常德府第一,就是湖南合省也没有这样的大乌龟。依卑府看来,竟是湖南第一。‘说罢,彼此相视大笑。我看你老公祖气象巍巍,今天一定要把你推为第一,况且你公祖善于谋干,将来平地飞升,怕不是个抚台么?“那邵允甫本是个胸无点墨的人,那里听得出秋谷是骂他的说话,还当秋谷真是恭维着他,心中大乐,只喜得他手舞足蹈,眉开眼笑,向秋谷拱手谦让道:”承赞承赞,兄弟现在不过是一个小小的知县,那里一时就会升到抚台?也只好碰碰运气罢了。“

春树听了秋谷取笑他的说话,已是忍笑不住,又听邵允甫懵懵懂懂说了一番得意之言,再也熬忍不住,恰好正喝了一口酒在嘴里,只听“噗嗤”一声,把口中的酒一齐吐了出来,不及回头,喷了金小宝一头一脸、淋淋漓漓的,连衣裳也带湿了好些。春树越发觉得好笑,竟哈哈大笑起来。邵允甫同王云生等不知春树笑的什么,大家眼睁睁的看他。金小宝皱着眉头,取一方洋巾揩干头面,秋谷已叫人绞了一把手巾过来,亲手递与小宝,小宝接了,含笑说声“对勿住”。秋谷笑道:“好呀!

你同我闹起这个来了。“小宝一笑,用手巾把身上酒痕揩净,看春树时,还在那里狂笑不已。小宝推了春树一把,瞅他一眼道:”啥格好笑介,拨耐格一笑,笑脱仔倪一件衣裳,倪要问耐赔格。“春树方才住了笑,道:”件把衣裳什么了不得的事,我就立刻赔你一件何如?“便立时叫了相帮上来,要写张条子叫他到石路生大衣庄去拿,却被小宝一把拦住道:”耐格种人直头少有出见格,倪搭耐说说笑话,耐就当起真来哉。勿要说倪格件衣裳,就是随便啥格物事末,倪也呒拨格号道理啘。耐一定要赔倪格衣裳,是有心勒浪扳倪格差头哉!阿要忒嫌难为情仔点。“春树笑道:”原是你叫我赔的,我又不是你肚子里蛔虫,怎晓得你的意思呢?“小宝听了,轻轻举起手来,在春树背上打了一下。春树道:”你替我捶背,索性多捶两下,这样的棉花拳头捶得不痛不痒的,却是难受得狠。“小宝被他说得也笑起来。

坐了一会,金小宝因有转局,便先走了。秋谷又与陈文仙附耳说了几句,文仙约他当夜到他院中,秋谷应允,文仙便也走了。不多时,菜已上齐,上过干稀饭,客人各散。秋谷也要告辞,被王云生一把拉住,再三苦留。秋谷道:“实不相瞒,我今天要到兆贵里去,所以不能耽搁。”王云生道:“我晓得你要去应酬相好,但时候尚早,在此略坐何妨?”秋谷仍是不肯。阿媛在旁听了,瞪了秋谷一个白眼,口中说道:“王老勿要拉俚,俚耐是要到陈文仙搭去格,倪格号小地方阿肯赏光,洛里好委屈俚介。”说着又把秋谷衣袖一推,道:“耐豪燥点去嗫,别人家等耐勿来,要性急格啘。”秋谷哈哈一笑,回过身来坐在炕上,把阿媛拉着坐在身旁,问他道:“我就是到兆贵里去与你什么相干,要你这样着急?你既然把我留在此间,我今天就在院中借个干铺,你可肯陪我么?”阿媛听秋谷说得刻薄,登时满面生红,想要立起身来走进后房,又被秋谷拉住,只得说道:“耐到兆贵里去本来勿关倪事,倪好心叫耐豪燥点去,耐倒勿见倪格情,耐格人阿有良心?”秋谷笑道:“不要动气,就算我的不是何如?”阿媛道:“勿是耐错,到是倪错?”云生忽向秋谷道:“秋翁既然赏识阿媛,我把林桂芬荐与秋翁可好?”秋谷大喜,深喜云生为人随和,全无醋意,当夜秋谷就在林桂芬家摆了一个双台,直闹至四更方散。从此与王云生交谊又深了一层。有分教:

灵犀一点,暗传青鸟之书;彩凤双飞,不隔蓬山之路。

欲知后事如何,但听下回交代。





第十八回 设机关流氓传电报 卖风情名妓访萧郎





且说章秋谷与王云生二人同住栈中,十分莫逆,云生便要与秋谷换起帖来。秋谷道:“我向来没有换帖的朋友,你我既然要好,就不换帖也是一般。”云生便向秋谷道:“我们既是通家,小妾理当相见,就请到我房内,等他叩见。”秋谷一听,心中大喜。秋谷自从那夜一见之后,思思索索的一直想要设法见他,现在听得此言,真是求之不得,便换了衣服,同着王云生走进隔壁房中。

只见这位姨太太坐在靠窗一张桌上,斜倚香肩,双蛾半蹙,好像想什么心事一般,见云生同了秋谷进来,连忙立起。他每天见秋谷在门口往来出入,本来认得,不用招呼。云生叫他过来行礼,他连忙走近秋谷身旁,凌波微步,罗袜无尘,袅袅娜娜的好似风吹杨柳一般,望着秋谷磕下头去。秋谷连忙闪在一旁,还礼不及。云生便邀秋谷坐下。姨太太也坐下来,低着头一言不发,双颊微红。秋谷口中天南地北的同云生谈论,暗中细细的偷看着他。只见他穿一件春纱夹袄,系一条玄色缎裙,梳妆淡雅,骨格风华。那一双俊眼水汪汪的活泼非常,巧笑流波,瞳神欲活,左顾右盼,宛转关情。正是:

羞态矜持,秋剪横谈之影;欢痕融洽,春添眉妩之云。

秋谷看得十分畅满,那位姨太太也时时偷转秋波,暗中窥觑。秋谷坐了一会,不好意思再坐下去,起身辞出。云生同步出来。姨太太送到门边方才进去。主

自此,秋谷与云生居然竟是通家,有时云生不在栈中,姨太太见了秋谷也并不回避,彼此目成眉语,差不多要学那红拂私奔。幸而秋谷为人伉直,虽然倜傥风流不拘小节,却是性情阔大举止端方。以前同王云生没有什么瓜葛,所以胸中存着这个念头;现在既然是同他彼此通家,交情莫逆,便不免有些惭愧在心,轻易不肯孟浪从事。

忽一日,秋谷正在栈中刚刚起身,尚未洗脸,忽见王云生神色仓皇,满头是汗,手中拿着一封电报匆匆的走了进来。秋谷见他这样,不晓得什么事情,尚未开口,云生已进房坐下,向秋谷道:“我刚才接到一封急电,是安徽家母寄来,说内人病在垂危,叫我立时回去。但是我有一件为难的事要同你商量,不知你肯答应不肯答应?我此刻方寸已乱,一些也摆布不来,况且我今天晚上就要动身,这事情实在尴尬得狠。”说罢,立起来向秋谷深深打了一拱。秋谷急忙回礼,不知他要相托什么事情,便道:“原来令正病危,这自然该立时回去。此间如有什么不了之事,只要我力量做得到的,总可商量,你只顾请说。”

王云生听了,脸上露出十分感激的样子来,随把坐的椅子挪到床边,低声诉说。

原来他这位姨太太也是苏州人氏,妓女出身,名叫李双林,向在芜湖女戏馆中唱戏。

王云生路过芜湖,见他生得标致,用了一千二百银子,将他讨做二房。但是云生十分惧内,太夫人家教极严,虽然娶了双林,那里敢同他回去?所以一向住在浙江。

现在云生接到了这封电报,当天晚上就要上船,只得把双林暂时留在吉升栈中,要托秋谷代为照应,等他到了安徽再作道理。秋谷听了,慨然应允,云生感激非常,又略谈了几句,便连忙辞去。

直至七点余钟,云生方才回栈,将衣箱行李打叠起来,只带了一只衣箱、一个脚篮,其余箱笼一齐留在上海,先叫栈内轿夫把行李发下船去。那天刚刚是礼拜一,长江是招商轮船,恰恰正是江裕,又教家人同着先去招呼。云生自己又到秋谷房间内来作揖告别,就同着秋谷到自己房内坐定。双林红潮晕颊,故意立得远远的,倚着床后的栏杆。云生叫他过来,道:“我今天回去,论不定什么时候回来。你住在栈中如有什么事情,可请章老爷招呼一切。我与他就如自己兄弟一般,你自己须要小心为上。”双林靦靦觍觍的叫了秋谷一声,秋谷谦让不遑,只得含糊答应。秋谷要与云生送行,云生道:“秋翁厚意本不敢辞,但兄弟今天实在没有心绪,并且要早些上船,只好心领了罢。”说着便有匆匆要走的样子,叮嘱了双林几句,便移步出门。秋谷此时留心看双林的举动,只见他眉敛湘烟,眼含秋水,似有许多幽怨说不出来。当下送出门外,觉得眼圈儿一红,连忙背过脸去,袖回香雪,衣展春云,急急的回进房去。秋谷暗暗称赏,便一直送了云生上船,在轮船上又谈了一会方才别去。这里王云生自转安庆不提。

且说秋谷回到栈房过了几日,已是端阳将近。秋谷把一切局钱开销清楚,自己也到陈文仙家住了几天,天销了二十块钱的手巾。文仙劝他不要浪费,秋谷不肯听他。

到了端阳这一天,秋谷上午没有出去,忽见陈文仙明妆丽服,珠翠满头,打扮得婷婷袅袅的走将进来,背后跟着一个相帮,挑进一担物事。秋谷诧异起来,向文仙道:“你们的节盘已经担过,为什么要送第二回?”文仙含笑答道:“节盘末是相帮笃格孝敬,勿关倪事格。格是倪自家买仔送拨耐格,请耐赏赏倪格光。”说着,叫相帮一一搬将上来。秋谷大为诧异,看那送的礼时,只见是两只上好金腿,十篓白沙枇杷,一盒吕宋烟,一身外国纱衣料。又见相帮端过一只提篮,文仙道:“晓得耐客栈里向格菜勿好吃,倪自家烧仔几样菜,一淘带得来。”就自己去开了篮盖,一样一样的摆在台上。秋谷看时,见是一大盆鲥鱼,一盆白汁巴翅,又是一只整鸭,一碗鲍鱼。原来陈文仙晓得秋谷素来爱吃的品味,所以特地做了送他。

秋谷看了大为奇怪,向文仙笑道:“怎么你忽然这样的破费起来?真是意想不到,又不好辜负你的来意,只好照数全收,但是大大的破费你了。”便叫了家人进来,叫他收拾;又叫把送来的四样菜,送到双林那边与他过节。留文仙坐了一会,文仙恐院中有客,起身要走。秋谷取出二十块钱的钞票来交与当差的,叫他交给相帮作为轿钱送力,却被文仙一把拦住,道:“格个物事是倪自家格一点意思,俚笃送仔来随便赏点好哉,倪实梗搭耐说格闲话,总勿肯听倪一句格。”秋谷笑道:“我原晓得你的意思,不要我浪费银钱,但既是相帮送来,我给他二十块钱也是你的场面。我们要好放在心上,倒不必讲论什么银钱。”文仙不肯,道:“实梗说起来,是倪有心叫相帮来打耐格把式哉啘,耐勿要看仔堂子里向一塌刮仔才是坏人,倪倒并呒拨格号心思,耐勿要缠错哩!”秋谷听了只得收回,给了四块洋钱送力,两块洋钱轿钱,文仙方才欢喜。临行问秋谷几点钟来吃酒,秋谷道:“大约八九点钟,你须要让出房间才好。”文仙应允。

秋谷待文仙走后,出去应酬了一转,傍晚方才回来。尚未坐定,只见隔壁那位王姨太太的娘姨走来,向秋谷道:“姨太太叫我来请章老爷过去,说是有话面谈。

姨太太已经候了多时,请章老爷就去。“

秋谷听了,也不知什么事情,便立起身来走过隔壁。见双林满面春风的迎了出来,向秋谷道了一个万福,又谢他送菜的盛情。秋谷也谦让了几句,随便坐下。举眼看时,只见双林打扮得十分齐整,蛾眉挹翠,檀口含朱,媚态横妍,珠光侧聚,穿一件玄色花纱夹袄,衬一条湖色熟罗裤子,却把裤管高高吊起,露出一对尖尖瘦瘦的双翅,真是:

踏青有迹,一钩软玉之魂;落地无声,两瓣秋莲之影。

秋谷见他这一身打扮,已觉得有些心荡神摇,不能自主。暗想随:“怪道他见了客人不穿裙子,故意卖弄他一对金莲。”再往双林面上看时,只见他:盈盈欲语,羌巧笑以含情;怯怯回眸,欲通辞而未敢。那一双俊眼注着秋谷,半晌无言。秋谷此时看了双林的神景,止不住色胆如天,便起身走过这边,想要与他并坐。猛见门帘一起,那娘姨端着盖碗送上茶来,秋谷吃了一惊,连忙缩住了脚,却已经走到床边,禁不住红生满面。双林见了会意,急唤娘姨道:“你到我镜匣内,把那一瓶香水拿来,请章老爷看个样子,明天好请章老爷照着牌子代买两瓶。”娘姨应了一声,自到房后去取香水,秋谷方才心定。

双林对着秋谷微笑点头,又略略向他摇手,似乎叫他不要性急的样子。秋谷更是满心欢喜。不一刻,那娘姨已在后房把香水取来,双林立起来接着,就走到秋谷身旁,亲手将香水交与秋谷。秋谷伸手接时,双林微微一笑,背转身去,下面那一双凌波三寸的鞋尖,早有意无意的在秋谷脚上碰了一下。这一碰,越发把秋谷引得心痒难搔。双林回身坐下,一面手掠云鬟,一面向秋谷道:“费心代买两瓶香水,今天如晚间没有什么应酬,再请过来坐坐。”秋谷是个绝顶聪明的人,那有不领略的道理?答应了,移步出来。双林送到门口,眼波莹莹打了一个暗号,方才回身进去。那娘姨是个粗人,站在门旁眼睁睁的看着,一毫不懂。

秋谷回到自己房中,觉得心满意足,准备着夜间暗渡蓝桥。忽然回过心来,自家一想道:“不好不好,我章秋谷一生,自负品学兼优,虽然花柳陶情,却从不曾干过这钻穴逾墙的行止;况且王云生与我虽是新交,尚称莫逆。从来说‘朋友之妻不可欺,朋友之妾不可灭’。我难道这点定力都没有么?”想到此间,便把先前的高兴减了一半,有些问心自疚起来。忽又回念想道:“虽然如此,但是双林十分情况,专注在我一人,又不肯辜负了他的意思。”左思右想,那一缕情丝,把个顶天立地的章秋谷缠得定定的,休想展动分毫。以心问口、以口问心了好一会,跃然而起道:“倾国倾城,佳人难得。就是明知祸水,也只得姑且一行。”主意已定,便在行箧中抽出一本《渔洋诗稿》来,歪在床上看着。那知看了半天,一页也不曾翻动,连秋谷自己也不解看的是什么东西,只觉得心上扑扑的跳个不住,不知是忧是喜,好像有无数的酸甜苦辣一齐并上心来,觉得好笑。猛然又想起陈文仙约的话来,心中暗想:“我非但答应文仙吃酒,叫他腾出房间,而且还有几处应酬不能不去。”

便定一定神,掏出表来一看,已有七点余钟,想起辛修甫请他在西安坊吃酒,正是约的七点钟,便连忙立起身来,锁好了房门出去。

到得龙赡珠院中,主客一齐久候,见秋谷一到,立刻叫起手巾,相将入坐。秋谷虽在席上应酬,面上却无精打采,冷冷的不甚高兴。修甫见他这般形景,不由不疑惑起来,便问秋谷道:“你今天为着什么事情这个样子,只怕有什么心事罢?”

秋谷笑道:“你这一问问得奇怪,我好好的有什么心事,你忽然考察起我来?”修甫不好再问。

饮过数巡,忽听见秋谷口中微吟道:

谁将三足鸟,来向天上搁;安得后羿弓,射此一轮落。

修甫不觉笑道:“怪道你今天失神落智的样儿,原来你有了奇遇,所以不肯告诉别人。”秋谷无意之中因为心上想念双林,随口吟了几句《西厢记》中的口白,却被辛修甫猜破说了出来。秋谷也无从分辩,只得彼此一笑而罢。

这一席酒因在席诸人多要翻台,草草终席。秋谷又应酬了王小屏、贡春树两处花酒,方才同着春树、修甫等一班客人同到兆贵里来。走进陈文仙院内,尚未上楼,便听得陈文仙房中有人在那里高声吵闹,打着一口京腔,又夹着些娘姨大姐劝解之声,十分热闹。秋谷甚是诧异,估量不出那吵闹的是何等样人,到底为着何事。秋谷急于要问,急步登楼。到了客堂,听那吵闹之声依然未息。文仙同娘姨等吓得昏了,也不听见客人上来。秋谷邀众人暂在客堂坐下,仔细听时,有分教:

留云借月,果然别有深情;煮鹤焚琴,何处忽来伧父。

欲知后事,且待下回。





第十九回 闯房间莽客怒生波 圆好梦良宵花解语





且说章秋谷同了客人来到陈文仙院中,听得有人吵闹。秋谷在外听时,只听见大房间内的客人高声骂道:“我把你这班不知抬举的奴才,你不过是个婊子罢了。

咱们到你院中是照顾你的生意。你靠着谁的势头,竟把咱们糟蹋起来!房间里明明没有客人,你下着门帘不叫咱们进去,咱们是不给钱的么?你的客人那里去了?咱们倒要见见你这个客人是多大的来头,难道缩着脖子跑了,咱们就罢了不成?“秋谷不听犹可,一听这几句说话,不由的怒从心上起,恶向胆边生,霍地立起身来把纱马褂脱去,抢前一步闯进房来。

看官,你道这个吵闹的客人是什么来历,为何与文仙有意为难?原来这人姓金,名叫和甫,是个吴淞口炮台统领的儿子,平日间仗着他父亲的势耀,在外面无所不为。走到堂子里头,看中了这个倌人,立时立刻硬要摆酒住夜,却又是白吃白喝,一个钱也不肯拿出来。若有那个倌人得罪了他,他一定要带着一班流氓光棍寻事生非,把倌人的房间打一个落花流水。以此北里中人闻着金和甫的大名,一个个心惊心痛。

这金和甫二三月间在聚丰园看见陈文仙出局,一身香艳,满面春情,就如失了魂魄一般,一直跟到兆贵里。走进院中硬要摆酒,当夜就吃了一个双台。依着金和甫,就要在院中住宿。文仙急了,慌与娘姨商量,叫相帮假做叫局,叫到后马路董公馆去碰和,方得脱身逃去,在隔壁花小兰家暗听消息。这里金和甫一直等到一点多钟,不见文仙回院,等得他意懒心灰,娘姨等把他千哄百骗的说:“先生代客碰和,一时不能回院,少大人有心照应,隔日再来末哉。”好容易把他骗出门去。自此之后也一连来过几次,多亏娘姨宝珠姐知风识势,诸事在行,把他敷衍过去。金和甫也渐渐晓得他们的意思,含怒在心,只是宝珠姐等人当面十分巴结,扳不着他的错头。

到了端午晚间,金和甫有心寻事,带了一班不三不四的朋友,喝得醺醺大醉,闯到文仙院中。文仙出局未回,娘姨等晓得秋谷要来摆酒,又经文仙分付把大房间留着等他,宝珠姐就把门帘放下。刚刚回转身来,劈面撞着金和甫跟着一班流氓,一哄而上就要拥进房去。宝珠姐吃了一惊,连忙拦住和甫,陪着笑面,说道:“对勿住!金少大人,里向有客人勒浪,只好先请客堂间里坐歇,等客人去仔再调阿好?”

金和甫听说内房有客,无可如何,只得就在客堂坐下。那些无赖立的立,坐的坐,挨挨挤挤塞满一层。恰好文仙堂唱回来,见金和甫坐在客堂,无数短衣窄袖的人在旁拥护,心下大惊。明知今日金和甫安心寻衅,一定要打闹房间,然而既然如此,也是无可如何;又刚刚走到客堂,已被金和甫一眼看见,躲避不来,没奈何硬着头皮,双蛾紧蹙,勉勉强强的走进来,叫了一声:“金少大人!”便坐在旁边,低头不语。

和甫正要开口,忽然有一个带来的流氓,走过来在和甫耳边低低说了几句,和甫登时大怒,问宝珠姐道:“刚才你同我说里房现有客人,为什么我来了半天,不听见一些儿声气,分明房里没有客人。我也不管你们青红皂白,我自己闯进房间看看,若是没有客人,你休想安然无事。”说着,不由分说,跳起身来一拥进去,见果然没有客人,更加火上添油,把文仙同宝珠姐叫进房去,问他什么原故,把他不当客人。珠宝姐任是伶俐,到了此刻,也只是顿口无言。文仙被金和甫一惊一气,不觉粉面通红,蛾眉倒竖,索性横了心肠,便冷笑道:“金少大人,耐末勿是做倪一个倌人,倪末也弗是做耐一干仔。客人付仔现洋钱定倪格房间吃酒,倪接仔俚格洋钱,自然只好留拨俚哩。比方耐少大人定仔房间要来请客,拨别人抢仔房间去,耐少大人阿肯答应格?”金和甫听了怒不可遏,厉声喝道:“别人吃酒有了现钱,你们就留给房间。咱们是没有钱的么?你好好的把房间让给咱们,好多着呢!如若不然……”金和甫一面说着,一面早伸出一只巨灵般的手掌来,五个手指就如胡萝葡一般,把文仙的衣袖一把拉住,两眼圆睁,势将用武。文仙只吓得金莲倒退,脚步踉跄,几乎放出哭声来。

说时迟,那时快,只见门帘一起,一条人影噗的穿将进来,直穿到二人身旁方才立住,也不开口,轻轻的把左手往金和甫臂上一格,金和市不由得臂上酸麻,放了手连退几步,一个鹞子翻身跌下地去。文仙定一定神,方才看见进来的是秋谷,不觉滚下泪来。秋谷不及温存,挥手叫他:“快快躲开!这班人不要怕他,有我在此。”文仙听了,一愁一喜,愁的是恐怕秋谷受亏,喜的是秋谷既已到来,那班朋友辛修甫、王小屏等自然一同到此。修甫住在上海,本来结纳官场,在租界中着实有些手面,不怕金和甫再起风波。便连忙一溜烟,同着宝珠姐躲到隔壁去了。

这里众无赖见金和甫被秋谷一掌打翻,便大嚷起来,一拥上前,先把和甫扶起,乱嚷道:“你是个什么东西?好生大胆,竟敢打起我们少大人来!”秋谷微笑道:“不要说是少大人,就是老大人来,我姓章的也不是怕事的人物。你们这班奴才光棍,大胆的只管上来!”

金和甫从地上起来,跌得浑身生痛,气得眼中出火。鼻内生烟,倚仗人多势众,指挥一群无赖,揎拳掳袖的蜂拥而来。秋谷不慌不忙把两手往两边一分,把一班流氓就像倒骨牌的一般,“匹力拍六”,一齐跌倒。金和甫见此情形正在发躁,不防被秋谷当胸一把,揪住衣裳,擒了过来,就如一只小鸡一样,就势往地下一摔,摔得他“阿呀”一声。秋谷一脚把他踏定,骂道:“你这个撒泼的奴才,你占了房间也还罢了,还敢不三不四的骂人!我看你这个样儿,一定是外来流棍。你好好的替我滚了出去万事全体,若有一声不字,叫你进来有路,出去无门。”那金和甫被秋谷踏在地上,口中还硬挣道:“我是个统领少爷,你不可如此糟蹋。”秋谷哈哈笑道:“好一个营官公子,统领公郎,你供了家世出来,难道我就怕了你么?你的老子既在上海统领营兵,你就该凡事敛迹,保守他的官声才是。怎样你在外边这般胡闹,不怕上司得着风声,提参你的老子么?你今日遇见了我尚且如此横行,平日间在外的不法招摇可想而知的了。我就立刻写信到一营,把你的恶迹说个明白,再托各报馆上起报来,看你老子的统领可做得成做不成?”金和市被秋谷一脚踏在地下,踏得浑身骨节酸痛非常,还想着自己是统领的少爷,姑且吓他几句,或是吓退了,也未可知。现在听得秋谷话头利害,像是个大来历的人,已是着慌,又见秋谷人才轩爽,举止大方,一定是个宦家公子,知道今天脱不得身,却又不肯折了志气,出口告饶。

正在为难之际,恰好辛修甫等听得秋谷将他打倒,恐怕秋谷一时不分轻重,打出事来,大家联步进房。修甫一眼看去,就认得他是炮台统领金建屏的儿子金和甫──修甫与他同席几回,所以认得──便连忙上前拦住秋谷道:“此人与我素来相识,你且放他起来,大家坐下,有话慢慢的说。”秋谷的意思本来不要打他,不过警戒他的下回罢了,见修甫上前相劝,顺水推船,趁势把脚一松,回身坐下。金和甫也从地下扒了起来,满面羞惭,与修甫相见。刚刚坐下未及开言,修甫先拦住道:“你们今日的事情原是大家鲁莽。你既然把房间占去,不该出口伤人,以致这位章秋翁忍耐不住动起手来。你虽然跌了两交筋斗,幸而并未受伤。据我看来大家都有不是。俗语说得好,不打不成相识,你们二位从此打成相识,各不介怀,改日我在西安坊摆酒请你二人,与你们做个和事,你们以为何如?可肯听我旁人的劝解么?”

那金和甫本来是个外强中干的人,瞒着金建屏在外闲闯,惟恐被金建屏查了出来,巴不得有人替他和事,就满口答应道:“既是辛修翁的朋友,彼此多是相知,大家不知不罪,只是章秋翁也要释然才好。”秋谷微微一笑,答道:“金和翁言重了!

我拳脚无情,多多得罪,改天当得负荆。“金和甫连称不敢,面上生红,回身又与修甫说了几句”仰仗费心“的话,自觉坐身不住,拱手告辞。秋谷也不相留,任他带着众人,狐兔成群一哄而去。

金和甫既走之后,陈文仙方从后房走了出来。云髻半偏,花钿不整,眼含泪晕,颊褪红潮,含怨含颦的向秋谷道:“谢谢耐,帮仔倪格忙,格格断命杀千刀,格付架形,赛过是格长毛,人也杀得脱格!倪拨俚吓得来,主意才呒拨格哉,勿知拿俚那哼仔格好。区得耐刚刚跑来,拿俚赶仔出去,勿然是直头一塌糊涂哉!想起来,总是倪做仔格断命生意勿好,随便啥人才好出倪格花头,换仔倪是好好俚格人家人,俚阿敢碰倪一碰?”说着,牵了秋谷的手,泪流不已。秋谷也不觉凄然,安慰了好一会,文仙方才止住,拭干眼泪,走到镜台旁边,一面招呼相帮摆好台面,一面重施朱粉,再画蛾眉,收拾去满面啼妆,平添出一团春色。换好了衣服,移步上来斟了一巡酒。

这一席酒,因是秋谷把金和甫赶走,大家十分高兴,连房间里娘姨大姐也十分巴结,竭力招呼。文仙坐在秋谷身后,虽然不讲什么说话,他两人默默相对,眉目之间觉得有一种说不出来的情况流露出来。秋谷忽回头,见春树叫的金小宝刚刚走进,便问他张书玉的事情,可曾到院中去过,小宝道:“俚耐来是的来歇,不过倪听见说俚要勒浪张园里向等着仔倪,要坍坍倪格台,倪也勿见得怕仔俚勒勿到张园去,随便俚去那哼末哉!”春树笑道:“张书玉要同你吵闹,你只要请章二少保镖,还你无事。”小宝认他取笑,回道:“倪勒浪讲正经闲话,耐咿要来瞎三话四哉。”

春树笑着,把方才的事一一同他说了,又道:“他有了这样本事,你请他替你保镖,还怕什么张书玉么?”小宝听了,似信不信的看着秋谷,笑道:“倒看耐勿出,阿是真格介?”文仙又代说了一遍,小宝方才相信。那席上的倌人听了,大家凝视秋谷,眼波脉脉,俱有欣慕之情。正是:

银灯依约,香迷六曲之屏;宝篆温存,春满九华之帐。

欲知后事如何,下回交代。





第二十回 王云生安排紥火囤 章秋谷踏破仙人跳





且说当夜席散之后,客人谢过主人,一齐散去。秋谷略坐一会,又慰藉了陈文仙几句,便立起身来,也想回栈。文仙牵住秋谷的衣裳,不肯放他回去。秋谷因惦记双林约他晚间过去,一定不肯住在院中。文仙见留他不住,生起气来,放了手回身坐在床前,翠黛低颦,一言不发。秋谷回过身来,见文仙泪揾秋波,红生宝靥,那一付西子捧心的态度直令人动魄销魂,不觉怜惜起来,心上不知怎样的好,连忙笑道:“你不要我回去,我就不去,只望你不要生气,无论什么说话总可商量。”

文仙见秋谷应了不去,方才抬起头来,拭泪应道:“耐要去末只管去末哉,倪是勿好拉住仔耐格啘。倪就是千日勿好末,也有一日格好处,耐倒直头好意思格。”秋谷笑道:“不要说了,总是我的不是。”说着就走过去,与文仙并肩坐下。文仙一手推开秋谷,道:“勿要像熬有介事。倪间搭是小地方,勿要委屈仔耐。耐豪燥点到别人家去,勿要倪末拉住仔耐格章二少,叫别人家勒浪瞎等一泡,阿要罪过?”

秋谷对着宝珠姐等把舌头一伸,道:“阿唷!唔笃格先生凶得来,拿倪横伊勿好竖伊勿好,倒直头利害哚。舍勒刚刚金家里勒浪格辰光,勿拿点本事出来介。”几句话,说得宝珠姐同文仙都笑起来。文仙道:“倪是从来勿晓得凶别人格,耐自家勿好啘。”秋谷也一笑而罢。

坐谈一刻,相帮已开了稀饭上来,秋谷吃了半碗,文仙也略略点饥,相携就寝。

但见:罗帐四垂,华灯背影。锦帏不卷,珍簟新铺。宝靥偎霞,纤腰抱月。半含雀舌,春融檀口之酥;低照云鬟,暗度麝兰之气。卧后之清宵细细,凤女颠狂;枕边之私语轻轻,檀奴珍重。欢能解事,旖旎如云;侬本多情,温柔似水。正是:古

果然知己心无那,博得蛾眉死也甘。

且说秋谷初六一早醒来,听得自鸣钟“当当”的响了六下,那时五月天气不比冬间,天已大亮。秋谷惦记双林昨夜在栈内空等了一夜,想要回去看他,便坐起身来。回头再看陈文仙时,只见他杏眼朦胧,樱唇半绽,一缕漆黑的头发拖在枕边,膏沐之香中人肺腑,一只雪白的手腕搁在枕上,带着一付金镯,一付翡翠镯头,正在好睡,呼吸之间微微透出豆蔻香味,秋谷悄悄坐起,竟自不知。秋谷见了他这一付可爱的神情,不忍叫唤,恐怕惊醒了他,轻轻的跨下床去,穿好衣服。见宝珠姐睡在榻上,兀自呼声大作,秋谷觉得好笑,不去惊动他们,慢慢的开了房门,走出院中,竟自回栈。

栈内静悄悄的,一个也没有起来。秋谷一直走到自己房间门首,且不开门,先向隔壁一看,只见房门虚掩,露出一条微微的缝儿。秋谷暗想:他果然等了一夜,背地里不知要怎生埋怨呢!便轻轻的推开了半扇门,没有一毫声息,挨身进去。见双林尚还未睡,却坐在床边,开了箱子像似要寻什么衣裳,忽听得脚步之声,急回头见秋谷悄然走进,不觉大吃一惊,惟恐秋谷走到床横,看见箱子里的物件,连忙“硼”的一声,把箱盖盖上,那光景就像箱子里头有什么宝贝一般。随手抢过一把洋锁来,“咯蹬”的把箱子锁好,方才回过身来。

秋谷看双林如此张致,觉得有些疑惑起来,便低低问道:“你箱子里是什么东西,如此贵重?我又不是强盗,难道会抢了你的么?”一句话问得双林张口结舌,一时回答不出,面上竟红起来;定了一定,方才勉强遮饰道:“你不要瞎起疑心,我箱子里头并没有什么贵重的东西,就有什么罕物,给你看看也是不妨。我因等你一夜不来,心上好生懊恼,打算你是不来的了。刚才忽然见你走了进来,恐怕天色已明,有人看见不是玩的,所以我不觉害怕起来。你为什么昨夜不来?累得我吊胆提心,坐守了一夜。你自己想想,恋了别处的相好,哄骗别人,还要来瞎起疑心,你可过意得去么?”好个李双林,这一席说话得来宛转圆融,有情有理,竟被他遮掩过了。一面斜视秋谷,含笑微梁,欲言不语。

章秋谷听了双林这一番言语,虽然不去驳他,却觉得有些诧异,未免还有脱校失节的地方。心上虽如此想,面上却一丝不露,仍旧满面笑容的敷衍着他,又低低的告诉他昨夜不得回来的原故。双林未免还要撒娇撒痴,埋怨几句。秋谷竭意温存。

自此,章秋谷与李双林竟成眷属。窥中堂之韩令,贾午留香;感汉浦之郑郎,洛妃解珮。早不觉一连又是几天,秋谷同双林早把那娘姨买通一路,朝欢暮乐,夜去明来。

有一天,秋谷尚未起身,茶房已经起来扫地。双林着急叫醒秋谷,叫他速速回到自己房间,免得茶房知觉。秋谷被双林唤醒,冒冒失失的起来一看,房门外已经有人行动,出去不得,只好关着房门,乘空再行出去。秋谷见双林起来梳洗,枕旁遗下一串钥匙,秋谷随手取来看时,见那钥匙的形式十分古怪,秋谷便拿着钥匙,走到箱子旁边去配那锁门当作消遣。双林正在梳头,听见钥匙声响,急回头看时,见秋谷已将一把洋锁开在旁边,正要去揭开箱盖。双林大惊失色,三脚两步的急急跑来,将秋谷手中钥匙一把夺去,捺住箱盖仍旧锁上,方埋怨秋谷道:“外面有人行动,你还要翻箱倒笼的吵闹,不肯悄悄的安坐一回,万一被人看见,将来我家老爷晓得风声,追究起来如何了得?我劝你悄没声儿的守过一刻罢。”

秋谷见双林这样惊慌,抢去钥匙,锁好箱子,把前日的疑惑兜的又提上心来。

心中想道:“现在茶房等虽已起来,却是关着房门,那里一时就会被他们看见?就是怕我开箱吵闹,也用不着这等惊慌。明明是这箱子里头一定有什么秘密事务,所以一连两次都是如此张皇,这是不问可知的了。但是我与他既然有了交情,何必还要这般遮掩?真是诧异的事情。”心中盘算,外面假作不知,反笑向双林低低说道:“我们关着房门,料想断断无人闯进,你何必这样胆小?”双林道:“你说得好太平活儿!事情闹了出来,你是不怕,我还有性命么?”秋谷一笑不语。等了一刻,趁着房外无人,一溜烟溜回房去。心中疑虑思索,却想不出他到底是什么原故来,便想要设个调虎离山之计,把他调出栈外,要看看他的行李究竟是何等珍贵的东西。

前两日,秋谷请过双林逛了两次张园,秋谷也和他同去,却是两部马车,双林登车先走,秋谷少停一刻,然后登车。到了张园,两张桌子泡茶,所以去过两回,没有露出一毫形迹。隔了一日,秋谷便哄着双林道:“我前日在张园看见一个倌人,名叫洪菊香,那身材相貌竟和你生得一般无二,只有口音不同。若是你们二人站在一处,不要开口,竟是分辨不出的。你可要去看看么?”那李双林以前两次开箱,见秋谷毫不在意,面上更没有露出一点疑惑的情形,那里想得到秋谷是哄他的说话?

听见有个倌人的相貌与他长得一模一样,自然要去认认他究竟相貌如何,况又是秋谷一同前去,更觉放心,便欢欢喜喜的答应了。秋谷便立刻叫了两部马车来。秋谷向双林道:“我要先到兆贵里去一趟,看那洪菊香可曾前去。他是照例天天要到一趟张园的。你随后就来,不要耽搁。”说罢,便己登车先走。双林见秋谷先走,更自坦然无忌,随后上了马车,带着娘姨向张园去了。

不防秋谷关照马夫,止把马车放到麦家圈,略停一会,仍旧回到吉升栈来。见双林已经去了,心中大喜,便走到帐房,要了双林的房门钥匙,一直进去开了房门。

茶房虽然看见,因秋谷与云生往来甚密,云生走后又把姨太太托他招呼,那里有什么疑忌?任他开进房门。秋谷在自己身旁取出一把钥匙──原来秋谷两天之内,早暗暗画了锁门,将钥匙配好,就随带在身。在秋谷想起来,不过少年好事,喜欢闹玩意儿,要看看他箱内倒底装的什么,要这样的避人眼目,原不是什么歹心。当下开了锁,揭开箱盖看时,只见箱子里头不过几件半旧的平常衣服。翻开衣服,箱底并没有什么东西,只有被单裹着几大包挺硬的东西重得镇手。暗想:“这般呆气,带着现银子出来,所以怕人看见。”便提出一包打开再看时,那知不看犹可,这一看,把个章秋谷看得目定口呆。

看官,你道是什么东西这般郑重?哈哈,原来不是别的,是一包的砖头石块,大的小的,整的碎的,假充银子放在箱中。秋谷呆了一会,还疑惑他是防备盗贼的意思,替他原封不动的放好,索性再打开底下的箱子看个明白,五只箱子多是一般装着碎砖乱石,上面铺着几件衣裳,开到着底两只时,连一件衣服也没有了,一箱都是碎石,塞着许多败絮破棉。

秋谷到了此际方才恍然大悟,信王云生也不是什么浙江候补的官员,这李双林也不是什么芜湖戏馆的妓女,多是王云生的瞒天大谎,掉着那天字第一号的枪花,真个是仙人跳的都头,紥火囤的光棍。他见秋谷性情豪爽,用度奢华,故意赔着本钱,有心结识。王云生却假做了一封电报,立时立刻要回到安徽,把双林留在栈中托他照应,却叫双林暗地把秋谷勾搭上手。到得秋谷上钩之后,隔了十天半月,王云生与双林暗中约定,摹然闯了回来,将男女二人双双捉住。假意摆着架子,说着大话,哄吓别人要杀要打,再不就要送官。他们拿定章秋谷是场面中人,最怕的是出乖露丑,那时要求他息事,不要送官,怕不三千二千银子双手高高的捧出来,孝敬了他,还要叫你写张伏辩。到了这个时光,就是明晓得他是个仙人跳的流氓,中了他的诡计,也只好眼睁睁的看着他,说不出一个“不”字。你道利害不利害?凭你章秋谷这样一个聪明人物,平时何等精明,若不是为了两次开箱,生出一番疑忌,也几乎着了他的道儿,险不被他敲了一下大大的竹杠。

当下秋谷暗恨王云生、李双林做得好事,竟顽起仙人跳的勾当来。又想道:“我现在既然识破,随处可以留心,面上只当不知,暗中仍旧与他来往,试试他怎样的一个开场。就是被他们当场拿住,难道我章秋谷就怕这一班光棍么?”主意打定,便把箱子一只只通通装好,照着原排的部位,一毫不错。又把房门锁好,便跳上马车,叫马夫加紧一鞭,星飞电掣的赶到张园。正是:主

大海鲸鲲,不上金钩之饵;摩天鸾鹤,难惊高鸟之弓。

欲知后事如何,请听下回交代。





第二十一回 闹张园醋海起风潮 苦劝和金刚寻旧好





且说前回书中章秋谷几乎被骗,幸而识破机关。列公且住,这王云生到底是个何等样人,为什么不骗别人,单单要寻着秋谷,这是什么道理呢?其中也有一个缘故,诸君耐烦静听,待在下一一的演说出来,好待看官明白。

这主云生的原籍本是扬州,从小爱嫖爱赌。家中狠有点儿田产,父母死后不上几年,被他嫖赌得干干净净。无可奈何,便改了行业出去当差,央人荐到浙江一个候补知府公馆内当了几年跟班,居然也有了积蓄。后来这知府轮署了绍兴府,王云生跟到署中,作威作福无所不为,直闹到风声大了,地方绅士联名上控起来,上台准了状词,就把这知府当时撤任。知府恨极,便把王云生发到县里,打了二千板子,又把他监禁一年。期满出来,浙江住不得了,便挟着几年的积蓄,直到苏州,要想寻条门路,依旧跟官。寻了多时,门路不曾寻着,银钱用得一空,却在青阳地结识了一班朋友,多是流氓马夫一流人物。

这王云生绝了资斧,免不得跟了这班流氓拆梢度日。适值章秋谷游玩苏州,就住在佛照楼栈内,银钱挥霍,服御奢华,又见他临行之际在余香阁点了一个满堂红,不到两点钟时就用去了百元上下。隔了一天,又雇了十余部马车,在二马路兜到阊门,通通兜了一个圈子。王云生同着一班流氓,看在眼里,见秋谷这般撒漫,一定是个富家,便想要纠集众人敲他一下竹杠。一则见章秋谷气宇不凡,不敢冒昧;二则那一天,秋谷在丹桂戏园粉墨登台,那舞刀的一场解数,不但看戏的众人称道,就是本园的武小生陈云仙也是极口称扬,自叹不及。明晓得秋谷是个拳棒名家,若突然去拆起他的稍来,光棍不吃眼前亏,不要拆梢没有拆成。反被秋谷白打一顿。

有此两层畏缩,所以大家不敢开场。众人彼此商量了一会,想不着个计较出来,王云生便想出这个紥火囤的主意,包了一个城内摆碰和台子的私窠子,叫做李雪梅,替他改了名字,说知缘故,约定将来得彩三七均分。因王云生久在官场,颇请礼节,众人就推他做了老大,把李雪梅充了他的姨太太,大家凑出本钱,又拣两个略为漂亮些的当作家人。部署已定,方才雇船到常熟来。

那知秋谷回了常熟,正事甚忙,那有工夫闲走?好容易等得秋谷送了金月兰回到上海,不多几时,秋谷自家也到沪江,这王云生就跟到上海来,与秋谷同栈房住下,磨拳擦掌的想要大大的弄他一注银钱。他在苏州看了秋谷的豪华气脉,料定他是个百万财翁,那知章秋谷不过一个中人之产;全是外面的排场,又且阅历甚深,十分精细。

这王云生到了上海,候了半月有余,只指望秋谷见了双林,先来拜会。那知候了多时,秋谷的面也不曾见着,只得借着同栈为名,先去拜望,慢慢的亲热起来。

假说要和他换帖,其实是要叫双林出来相见,卖弄风骚,秋谷果然着了他的道儿。

王云生便假做一封电报,说是妻子病重,立刻要回到安徽,故意把双林留在栈中,托秋谷随时照应,好等他慢慢的上钩。他自己却并不当真回去,那一夜上船之后,打发了栈内的茶房回去,依旧把行李搬上岸来,在左近一个小栈房内暗暗住下,打听风声。双林用的娘姨也是他们一路,便悄悄的传送消息,知道秋谷早已上钩。

只因这王云生自己假充是浙江的候补官员,此番接了家中电报,赶回安庆,却是众目昭彰。大家晓得的事体,若过了三天五日突然走了回来,不但秋谷疑心,就是客栈中人在旁看见也不兔要心中疑惑,明是仙人跳的行为。况且他那一封电报又是假的,不敢出场,未免有些不妥之处,所以定要扣准日期,装做在安庆回来的样儿,方好遮掩众人的耳目。计算的安排的智出万全,要叫秋谷无从摆脱。万不料这两天之内,双林无意之中露出马脚,自己还全然不晓,却被秋谷做了提防,把他们多时的计算安排一朝化作了虚乌有,赔了应酬的本钱不算,还出了一个名声,上海地方从此无颜再到。在他们看起来,也就叫“周郎妙计高天下,赔了夫人又折兵”

了。

闲话休提,书归正传。且说章秋谷上了马车,一口气直到张国,马车在安垲第门口停下。秋谷因恐怕双林在张园等久要起疑心,急于进去,便一跃而下,正要进门,忽见门口拥着一班不三不四的马夫,多是纺绸短衫,纺绸裤子,窄袖高领,盘着油晃晃的一根大辫,脚下多是挖花鞋子,一个个揎拳掳袖,怒目横眉的,像似要与人寻事一般。秋谷看了这班人的行径,心中甚是骇怪,估量不出为的什么事情。

回过头来见草地上还有一群马夫,却三个一堆、五个一簇的往来闲走。秋谷虽然看见,不去管他,便一直进去。刚刚走到中间,耳中听见好像一个倌人的口声在那里与人相骂,却像金小宝的声音。秋谷想起前日小宝席间的说话,心中早已瞧料了几分,顺着那相骂的声音看去,只见张书玉不施脂粉,穿着一身半旧的衣裳,头上也没有一些首饰,双眉倒竖,杀气横飞的坐在那里,一言不发。又见金小宝立在当地,对着众人,指手画脚的不知说些什么。秋谷方才明白,定是张书玉因贡春树被金小宝平空夺去,吃起醋来,所以在张园等着小宝,要和他一决雌雄,争回嫖客。秋谷看了,心中想道:“刚才门外的那班马夫,一定是书玉约来帮助的了。但是金小宝没有防备,恐怕未免吃亏。”又四面看了一转,却不见春树的影儿,又恐被小宝、书玉二人看见,多要请他评起理来,无从偏袒,便把身子隐在一旁。

只听得金小宝道:“别人家格吃醋末放勒心浪,俚耐格吃醋,放勒面浪仔勿算,还要跑到归搭来,搭倪讲哈格理性,赛过恐怕呒拨人晓得,自家勒浪挂招牌,陪笃大家想想看,客人末勿止做一格倌人,倌人末勿止做一个客人,有本事末,伴牢仔客人勿要放俚出去。现在俚耐总说倪抢仔俚格客人哉,倪做仔生意,挂仔牌子,客人来来去去,只好随俚个便,倪阿好叫俚勿来格?就算是倪抢仔俚格客人末,也是客人自家情愿到倪搭来格,耐亦勿是俚格家主婆,阿好管牢仔俚介,做出格付极形来,阿要踉跄?”这几句不痛不痒尖刁刻薄的说话,张书玉听了气得面青唇白,半晌无言,一时竟回答不出什么来。停了一刻,方才跳起身来指着金小宝,大骂道:“耐格肏千人格烂污婊子,直头勿要面皮!倪搭格客人做得好好里格,平空拨耐引仔过去,还要背后说倪格邱话。耐要拉客人末,四马路浪几几化化格人勒浪,耐做仔野鸡,随便去拉格两格好哉啘。拉仔倪格客人去,还勒浪像煞有介事,勿要面孔格肏千人。”一席话把个金小宝骂得火星直冒,冷笑答道:“倪是烂污婊子,耐是好好里是人家人啘,倪归格辰光是花烟间里格出身,所以大家才勒浪叫倪老枪。耐去想嗫,倪花烟间里向出身格人末,阿要啥格面孔?自然马夫、戏子姘得一塌糊涂哉啘,耐格实梗一个规矩人,阿好搭倪说话?”说得旁人多大笑起来,秋谷也暗笑不已。

张书玉听小宝说得愈加刻薄,枭着了他的痛疮,越发无明业火按捺不住,霍地立起身向外便走,口中说道:“倪也无啥闲话替耐说,耐有本事末跑到外势来,倪大家说个明白,勿敢出来末,是只众生。”小宝微笑答道:“随便到啥地方,倪怕仔耐勿去末,上海滩浪,倪也勿要住哉!”一面立起来,跟着张书玉往外就走。

那知刚刚走出门前,张书玉对着一班马夫使个眼色,这些马夫大家会意,一拥而上,竟把一个金小宝围在当中。小宝见此情形,大惊失色,方才晓得张书玉有心算计,自己入了牢笼,今天免不了一场羞辱。只见张书玉对着金小宝冷笑道:“耐格烂污婊子,阿敢再凶?今朝勿拨点生活耐吃吃末,呒拨日脚格哉!”那些马夫听了,七手八脚的围着金小宝,正要动手。

小宝只急得红生粉面,汗透罗衣,正在窘急万分、分说不得之际,只见那些马夫忽然往旁边一卸,开了一条路出来。小宝大喜,举目看时,原来就是章秋谷,先前隐在一旁,恐怕被他们看见;后来听得书玉与小宝恶言相抵,大家翻了面皮,又见张书玉立起身来,金小宝随后出去,暗说:“不好,小宝跟他出去,定要吃亏。”

便连忙随后跟来。出了洋房门口,便看见一班马夫围着小宝,声势汹汹,小宝只急得粉黛霪霪,喘汗交下。秋谷见此光景,心中不忍,知道不得开交,便急急的走上一步,把两手往人丛插进,两下一分。那班马夫多是淘虚身体的人,那里禁得起秋谷的神力?被秋谷轻轻这一分,早一个个东倒西歪,让出一条大路。

秋谷见这班马夫如此无用,暗暗好笑,走进围中,向书玉、小宝二人说道:“你们有什么事情也要好好的讲说,为什么一言不合就这样胡闹起来,不怕打出祸来的么?你们聚了这许多的人,在此七乱八糟的吵闹,倘被巡捕听见赶了进来,大家不便。无论你们两下有什么委屈,有我在此承当,你们大家不许多说。”张书玉听了尚未开口,金小宝见秋谷进来排解,心中大喜,抢先说道:“倪今朝礼拜日到间搭来坐歇,勿壳张俚耐来起倪格花头,倪是从来朆搭别人吵过歇。二少,耐替倪评评格个理性看。”秋谷摇手道:“你们的事情我统通晓得。你也不许多言,书玉也不消生气,大家同我进来,有话好说。”说罢,一手携了小宝,一手携了书玉,拔步向内便走。

张书玉心中虽然怪着秋谷不该多事,待要发作几句时,无奈书玉一见章秋谷那一付玉树临风的骨格,一个身子就酥麻了半边,不由的怒气全消,春云上颊,伏伏贴贴的跟着秋谷举步进来。那班马夫原是张书玉约来的人,要想把金小宝羞辱一场,出出他的酸风醋气。不料突然走出一个章秋谷,分开了众人,同着书玉、小宝二人往内便走,那班人见张书玉一言不发,跟着他走进洋房,蛇无头而不行,大家只得一哄而散。

这里秋谷携着两人的纤手走了进来,拣一张桌子泡茶坐定,方才对着张书玉笑道:“你到底为了什么事情这样生气,我来替你们做个和事何如?”张书玉见秋谷开口问他,把先前的一腔怒气丢到东洋大海去了,只向秋谷似嗔不笑的道:“耐倒好格,阿对倪得起?”说着便低下头去,眼圈儿一红,似有无穷怨恨说不出来。秋谷明知其故,陪笑说道:“你们彼此不要相争,大家伤了和气,我叫他两边走走,不要冷落你一边可好?”书玉听了,抬起头来,低低的啐了秋谷一口,又把嘴一披道:“耐格人末,说说就呒拨好话出来哉,格号呒拨良心格众生,啥人来说俚介,故歇想起来,才是耐格勿好,耐勿该应……”书玉说到此际,说了半句咽住不说,却只呆呆的瞅着秋谷。瞅了半晌,方把一个指头向秋谷额上狠狠的推了一推,道:“倪也呒啥说头,耐自家去想罢!”

秋谷听了书上的话,回心一想,觉得自己果然有些对不起他的地方,便先向金小宝道:“你在此间没有什么事情,你先回去罢,以后或者你们席上相逢,大家不消提起,免得旁观不雅,坏了彼此的名声。”小宝受了这一场惊吓,云鬓蓬松,钗环撩乱,身上的一身外国纱衫裤也都有了皱痕,巴不得要立时回院,重新插带梳头,听了秋谷叫他先自回去,答应一声立起身来,叫了同来的一个小大姐一同出去。

这里秋谷着实的安慰了书玉一番,又说:“这件事情,与小宝无干,多是春树一人不好,做了相好,三三两两的没有良心,就是垃圾马车一般。你也不犯着为他生气。我明天一定把他拉到你的院中,凭你怎生处治便了。”书玉听了秋谷这一番心平气和的说话,方才敛怒成欢,转忧为喜,向秋谷笑道:“倪本来勿认得啥姓贡格客人,才是耐荐拨仔倪,弄得鸭屎臭。老实说,格号客人,倪做仔俚也勿见得绷得出啥格场面,不过情理浪讲勿过去末,倪总要搭俚说两声闲话,故歇俚耐勿高兴来未,倪也勿在乎此,只要耐二少有心照应,绷绷倪格场面,勿要坍倪格台好哉。”

说着,斜视而笑。

秋谷正要回答,忽想起双林尚在园中,不知可曾回去,怎么刚才不见他的影儿?

便不及和书玉说话,立起来向书玉道:“我还有些小事,要在这里寻一个人。你先回到院中,停会晚间我再来与你细谈。”书玉听了,俊眼含娇,眉尖微蹙,道:“倪闲话才说完哉,耐勿去末,倪也只好随耐格便,只要耐天理良心自家去想想看末哉。”秋谷连声“晚间决不负约,你只管放心”,一面说着,一面急往四下里寻觅双林,那里找他得着?

秋谷十分焦躁,正要上楼去找,先一抬头,只见双林倚在靠东的一带栏杆上面,看着秋谷微微含笑。秋谷大喜,急忙走上楼去问他:“何故不到楼下泡茶?累得我寻了一身大汗。”双林道:“我因楼下人多,又见有人吵闹,所以改在楼上。等了多时,方才见你来了,为什么又不上来?”正是:

摧花折柳,大兴醋海之波;倚玉偎香,双入桃源之洞。

欲知以后如何,下文分解。





第二十二回 香车宝马陌上相逢 纸醉金迷花前旖旎





且说秋谷向双林说道:“我先到兆贵里去了一趟,刚刚他们院中有客摆酒,菊香要应酬台面,料想今天不得出来。我出了兆贵里,跳上马车一直到此,听得他们相骂,两下几乎动起手来。我因张书玉、金小宝两人都是向来认得,恐怕他们闹出事来,所以把他们解劝回去,方才想着你尚在园中未曾回栈,急急的四边寻你,想不到忽然在楼上泡起茶来。”说着,双林因菊香不来,便要回栈。秋谷一同下来,马车已在门前伺候。秋谷与双林先后登车,但见夕照衡山,林梢倒影,一路滔滔滚滚的直望大马路泥城桥一带跑来。帽影鞭丝,马龙车水,在着那斜阳影里驰骤争先。

秋谷与双林两部马车,一前一后,紧紧跟着,一个是徐娘未老,春风三月之花;一个是张绪当年,汉苑灵和之柳。秋谷前面有几部倌人的马车,时时回过头来秋波送娇,瓠犀微露的对着秋谷脉脉含情。

秋谷正在心旷神怡,应接不暇之际,忽见对面飞也似的一般来了一部马车。两个马夫一齐穿着号衣,马车上的装饰也十分精致:杨妃色的车垫车围,倚着绣花靠枕。车上坐着一个倌人,翠羽明珰,烟鬟雾鬓。感飞仙于洛浦,神彩回风;拥宜主之罗衣,珮环照夜。珠光外露,宝气内含。虽不是什么国色天香,而顾盼之间婀娜多姿,丰神绝世。秋谷不觉目光定了一定,微吃一惊。暗想:“这个倌人甚是面熟,好似在那里见过的一般,却又不是金刚队中的人物。这一付身段煞是可人。看他眉目之间也不是什么倾国倾城的相貌,不过善于装饰,一天风韵随处撩人,就觉得比那天生丽质还要略胜一筹。”正在心中思想,忽又见那倌人欠起身来,一对秋波眼不转睛的注视秋谷,两下眼光一错。那马夫跑得电掣风驰,已离有一箭之地,猛听得那倌人巧始莺喉,高叫一声:“二少!”

秋谷听了,甚觉诧异,便立起身来,远远的应了一声,心中还在盘算,不知他究竟是谁。又见那倌人指挥马夫勒住僵绳,缓缓的回过车来,加上一鞭,跟在秋谷马车后面。秋谷见他来得切近,仔细看了一回,忽失声道:“你是黛玉啊!听说你先前嫁了邱八,甚是得意,为何又要出来?”

看官,你道那车上是谁?原来真是去年嫁人、坐第二把交椅的金刚林黛玉。当下黛玉含笑答道:“倪格闲话一时也说俚勿完,等歇倪到大菜间去搭耐说罢。”秋谷也因隔着马车谈心不便,点了一点头,便关照自己车上的马夫,叫双林的马车先回吉升栈去,自己的马车同着林黛玉一直到一品香来。

马车到了门前一齐停下,黛玉款步下车,一同上了楼梯,占了第六号房间,进去坐下。秋谷尚未开口,黛玉先向秋谷笑道:“耐格眼睛总算还好,倒还认得倪勒。”

原来秋谷从前与黛玉甚是要好,彼此无话不谈,不过秋谷醉翁之意并不在酒,所以他们两下虽然往来秘密,却没有什么交情,后来秋谷回去之后,再到申江,听见黛玉已经嫁了邱八,秋谷不禁怅然,未免有人面桃花之恨。现在旧好重逢,心上自然欢喜。当下秋谷答道:“我们相别不到一年,倒像过了好几十年的样子。你的面貌比先前瘦了好些,却觉得神彩飞扬,容光照耀,比从前更是不同。所以我觌面相逢,也没有想着是你。后来听了你的声气,方才记起你来。”说着,秋谷急于要问他在邱家为着何故重落风尘,几时到的上海,细细盘问。黛玉听秋谷问他,不觉触起去年的苦境,长叹一声道:“说起倪格事体来,真真作孽,倪今朝到仔上海,赛过是重投格人身。”说到此处,便滚下泪来,真如微风振箫,幽鸣欲泣。秋谷连忙安慰他几句,逼他快说。黛玉方才噙着珠泪,把初嫁邱八,以及近日下堂的情形,从头至尾一字一句的诉说出来。说到此间,做书的不得不暂停笔墨,把林黛玉嫁人复出的情节细细的铺叙一番,提清眉目,免得看官们无从捉摸,抱怨在下的头绪不清。

闲话休提,只说那邱八是个甚等样人物?原来他祖籍湖州,家财百万,浙江一省大家都晓得邱八公子的大名。从小儿父母双亡,家无兄弟,幸亏他一个嫡亲母舅把他抚养成人。到了娶亲之后,他母舅见邱八心地也还明白,便把那百万家财一齐交代,叫他自己支持门户。这邱八从小极是聪明,为人浑厚,举止大方。作事虽然精爽,却没有一毫吝刻的心肠;性情虽是豪华,却没有一点骄奢的习气。若有明师益友朝夕追随,把他成就起来,岂不是绝好的青年子弟?无奈无人管束,渐渐的自家放荡身心,就自然而然有那一班帮闲绰趣的朋友,掇臀放屁的把声色狗马来引动他。这邱八虽然质地聪明,却是个少年公子的心性,那里有什么定力把持,就不由的挟着重资,同了这一班朋友走到上海,任情的挥霍起来。在妓院中做着那天字第一号的瘟生,赌场中做那有一无二的冤桶,无论长三幺二,野鸡住家,以及广东堂子、外国妓院,各处的番摊牌九,甚至城隍庙内的地摊,他也要一处处的阅历过来,尝些滋味。不到两年,就把那百万家财销化了十分之四。虽然挥霍了数十万金,他自己却也长了十分见识,无论什么事情都瞒不过他。

自此之后,这邱八也不肯像从前一般憨嫖滥赌,收抬行李,回到湖州。每年之中,一定要到上海四次,春、夏。秋、冬每季一次。身边带着一万银子的钞票,纵情花柳,到处留名,要把这一万银子用得精光,方才立刻束装回去。若有朋友约他去到赌场玩耍,他也不推辞,却只带一千银子。进了赌场动起手来,他若赢了,就把身边所有的本利一齐滚上,庄家每每被他卷得精光,吃亏不小;若是风头不顺,他却又甚是调皮,输掉的身边带的一千银子,他就回转身来尘土不沾,拍腿就走,也不作翻本的念头。以此一班赌脚见了邱八进来,一个个攒眉蹙额,却又无可如何。

到了嫖界之中,他若看中了一个倌人,随意到院中走走,却只是随随便便的,不一定去转他的念头,就是吃酒碰和,也要他自己高兴,不肯附和着倌人。倘若倌人偶然开口,要他请客碰和绷绷场面,他就立刻翻转面皮,把局帐开销清楚,从此断了交情。有些倌人做得久了,摸着了他的脾气,从不轻易开口叫他吃酒、叫他碰和,他却又不等倌人开口,自家先就和酒连绵,十分报效,并且打首饰做衣裳,绝没有一毫吝啬。也有那些倌人不知道邱八的性情,想要敲他的竹杠,他非但不肯答应,把那倌人教训一场,还要立刻跳槽,当时叫局,给一个大大的没趣。就是住夜留厢,也要那倌人再三俯就方肯应酬,从不肯轻易自家开口。以此妓院中人见了邱八,十分巴结,处处小心,惟恐有些儿不到之处被他扳着了差头,他立时就要发挥,不顾倌人的场面。真是个赌博场中的大彼得,平康巷里的拿坡仑。

这一年邱八到了上海,正值林黛玉也在申江悬牌应客。黛玉是风月场中的老手,应酬队里的能员,况且盛名之下,自然枇杷门巷,车马纷纷。无奈黛玉的生意虽然甚好,却是浪费银钱奢华无度,做了两节,渐渐的支持不来,勉强各处移挪,略为敷衍。过节之后,各处店家因黛玉旧欠未清,大家不肯赊欠。刚刚过了中秋,正是起生意的时候,黛玉两手空空,借尽当绝,没有垫场,这生意如何做得下去?直把个林黛玉急得走头无路,进退两难。左思右想,只有淴浴的一个法子,却一时那里寻得出这样的一个主儿?

说也凑巧,却好邱八到了上海,住在鼎升栈内,已经耽搁了一月有余。因邱八在上海试办一家丝厂,那丝厂开创之初,未免事情忙碌,所以暂时不得回家。邱八这回到此,看中了范彩霞,就到东荟芳范彩霞院中,接二连三的碰和摆酒,不多几日,便有了交情。这范彩霞生得皓腕纤腰,长身玉立,蛾眉挹翠,凤目流波,也是上海滩上数一数二的有名人物,应酬圆转,丰格轻盈。但是神气之间觉得有些秋气,迥不如林黛玉的一团和蔼、八面春风。

半月之前,邱八在范彩霞家请客,有一个姓马的客人把黛玉叫到席上。黛玉素来认得邱八,况又久闻大名,极意应酬了邱八一回,暗想:范彩霞做着了这种客人,也是他交的花运甚好。邱八见了黛玉,虽是向来相识,恰见他回眸顾盼,卖弄风头,一到席间就唱一折昆腔《长生殿》里的《絮阁》。原来林黛玉的昆腔,上海颇颇的有名,轻易不肯就唱,真是穿云裂石之音,刻羽引宫之技。唱完之后,又把在席主客一个个的应酬转来,丝毫不漏。邱八着实赞了黛玉几句,心中也在暗想:“彩霞的应酬工夫虽然不错,若要比起林黛玉,未免较逊一筹。”心中便存了个要做黛玉的念头。两下都有些意思。

此番被林黛玉千思万想,想着了他,心中大喜;盘算了一会,就备了几色极丰盛的礼物,叫一个房间里娘姨名叫金秀的,教导了一番说话,带一个相帮挑着礼盒,又取了自己一张林黛玉的名片,又附着金秀的耳朵说了几句极密切的话。金秀点头会意,带了礼物一直送到鼎升栈来,在帐房内问明了邱八的房间是二十五号楼上官房。

却好邱八还未出去,正同他手下的一班朋友在那里谈论丝厂的事情,见金秀进来,笑迷迷的叫了一声:“八少!”相帮跟着进来呈上礼物,乃是鹿脯、燕窝、金腿、鱼翅四样。邱八见了甚觉奇异,看着金秀却又不认得他,疑惑他是新到范彩霞家,彩霞叫他来的,便道:“你想是新到他家,我所以不认得你,为什么无缘无故要送起礼来?”金秀含着笑,袋里取出黛玉的名片来放在桌上,口中说道:“倪先生特为叫倪过来,请请八少格安,格点点物事勿好算啥格礼。倪先生说,总是倪格意思,请八少留仔赏赏人,难末倪先生有两句闲话搭八少说,叫倪来请八少过去坐歇。倪搭末不过地方小点,勿得知八少阿肯赏倪格光?”邱八听得金秀一番说话来得十分圆转,心中自然欢喜,晓得林黛玉要吊他的膀子,特地叫娘姨过来请他。这邱八前回在席上见了黛玉,已是留情,更兼林黛玉也是个金刚队里的出色人员,又是这般的迁就着他,不觉心花怒开,十分得意,便向金秀道:“既是你先生这般要好,送来礼物,我自然一概全收,停回晚间再到你们院中请客。”便叫家人进来把送的礼收了进去,又朝着那家人使个眼色。不多一会,取出一卷红纸封的洋钱,也不知他多少,放在盘内。金秀是已经受了黛玉的教导,成竹在胸,急忙枪上一步,把那一封洋钱仍旧取出,放在邱八面前,陪笑说道:“笑话哉,倪送仔格点物事,八少还要赏啥格洋钱。倪来格辰光,先生再三再四交代倪格,叫倪勿许收八少格赏钱。八少有心照应末,等八少到倪搭来仔,再说末哉。倪先生实梗交代仔,倪要拿仔转去,是先生要搭倪反得一塌糊涂哉。倪先生说过歇格,说八少搭倪真心要好末,放勒心浪,勿在乎一定要绷啥格场面。八少,耐是格明白人,洛里一样事体瞒耐得过?耐阿好体贴倪点,叫倪转去少吃两句钝杠。”

说也奇怪,自有个茶花女的放诞风流,就有个收服他的亚猛;自有个莫立亚堆的奸巧诈伪,就有个侦缉他的呵尔晤斯。这也是新法格致家,心理学中的一种作用。

这邱八的性情向来极是尴尬,不知怎样听了金秀的两番说话觉得甜迷迷的,不知不觉在耳朵中钻了进去,不由的满面是笑,连连点头。这真是名妓的揣摸迷人的伎俩。

可惜那林黛玉终究不是格致专门,不懂心理学中他心通的妙用,后来终久弄得棋输一着,几乎九死一生,这也真是林黛玉一生哄骗客人的报应。

当下金秀同着相帮回去,见了黛玉,把邱八的情形说了一番。黛玉大喜,晓得有了几分意思。果然上灯之后,邱八已到院中。黛玉打起全付的精神,应酬得邱八甚是欢喜。当时写了请客票头叫相帮分头去发,就摆了一个双台面,黛玉坐在席间竭力巴结。不多一会,叫局的局条一起一起,陆续而来,顷刻之间已接了二十余张局票。黛玉叫娘姨回报,多要在王家库转过来,依然坐着不去,与邱八谈得甚是亲密,一时之间把邱八灌了无数迷汤。邱八被黛玉一番追魂摄魄的言语,说得心里觉得浑淘淘的,六神无主,竟把持不定起来。只见黛玉忽地起身,走到后房去了,过了一刻走了出来,却是换了一身衣服,连弓鞋裤子一齐更换,明妆丽服,光艳照人。

黛玉先前是穿一件湖色外国缎夹袄,杨妃色外国缎裤子,宝蓝弓鞋。现在进去,换了一件玄色织银夹袄,宝蓝织金裤子,玄色平金弓鞋,越显得明眸皓齿,粉颈香肩。

邱八见了,甚觉高兴,恨不得立刻把黛玉搂了过来团成一片,上上下下的把林黛玉看个不住。黛玉故意一手扶着椅背,用指尖掠着云鬓,俊眼四流,娇波欲笑,又把眉尖微蹙,跷起弓鞋,欠身下去,用手握着鞋尖捏了几捏,方才背转身来,退到原处坐下。那光景就是风飐蜻蜒,十分娇弱。黛玉坐在邱八背后;低垂云鬓,斜亸香肩。那眼光四面飘来,将到邱八面前,忽地回头斜坐,从背后转过秋波,大宽转的打了一个圈子,眼波澄澄正注到邱八面上。见邱八不转睛的看他,面红微笑,依旧低下头来。正是:

低颦浅笑,春添颊上之涡;宝枕银屏,花压双星之影。

欲知邱八与黛玉究竟如何,且待下回分解。





第二十三回 瘟富翁误堕迷途 名校书安心淴浴





且说林黛玉见邱八仔细看他,低低的朝着邱八笑道:“啥格好看介,阿是勿认得倪?”邱八笑道:“并不是不认得你,只为你一刻之间换了两身衣服,越觉娇媚动人,所以我留心打量一番,打算要替你画个小照。”黛玉听了把嘴一披道:“倪是勿好格,陆里赶得上范彩霞?耐勿要钝嗫!”邱八一笑,也学着苏白道:“阿唷,先生勿要客气,倪倒是真心闲话嗫!”说得一席客人通笑起来。黛玉故意把邱八瞟了一眼,道:“故歇末说得实梗好,只怕隔脱仔两日厌烦起来,倪搭请也请耐勿到。”

说话之间,黛玉又进去转了一转,又换了一身衣服。密色绣花缎袄,妃色绣花裤子,天青缎子弓鞋,将头上珠花一齐卸去,单戴着一只一条龙珍珠押发。脸上的脂粉洗得淡了些些,那粉颊之上略略晕起两个酒窝,觉得他淡抹浓妆,无一不好。

邱八虽然是个花丛老手,却从来没有经过这样风情,只乐得心窝上奇痒难熬,扒搔不着。黛玉见邱八已经入彀,越发的笑语殷勤,风生四座。

邱八忽然想着,问林黛玉道:“刚刚有好几张叫局的票头来叫你的局,你为什么不去应酬?台面虽然要紧,好去了再回来的呀!你不怕脱了局得罪客人么?”黛玉含笑道:“耐八少是难得到倪搭来格,耐肯赏仔倪格光,就是倪交仔运哉。格两上堂差勿去,得罪仔客人末,啥格希奇勿煞,倪刚刚关照下去,说倪今朝堂差勿出哉。”邱八听了,十分欢喜。那一班客人要拍邱八的马屁,好讨他的喜欢,大家极力称扬,恨不得把个林黛玉立时就抬上天去。依着他们的口气,差不多说得个邱八就是个再世的李药师,林黛玉便是个当今的张红拂。这一席酒直吃到十二点钟方才散席,客人陆续辞去。

黛玉见邱八贼忒嘻嘻的坐下,天南地北的扳谈,明知邱八心中巴不得要想住下,却做个欲擒故纵的法儿,立起身来,袅袅娜娜的走到邱八身旁,低声问道:“辰光勿早哉哩,耐阿要原到范彩霞搭去罢。倪是勿好留耐格,明朝说起来,大家难为情。”

说着,把身子一倒,直倒入邱八怀中,并倚香肩,低偎檀口,又问着邱八道:“八少,倪格说闲话阿对?”邱八此时已经心荡魂摇,六神无主,急切问张开大口,一时说不出话来。黛玉又逼他一句道:“勿然末勒浪倪搭,借仔一夜干铺罢,倪到后房去困,让耐一干仔舒舒齐齐阿好?故歇是深秋天气哉,勿要半夜里转去受仔风寒,倪倒担勿落格个干系。耐格身体又亏,勿是约约乎格。”邱八听了,觉得林黛玉说的话一句一句的打入心坎里来,十分熨贴,就是自己家中的妻子,那里有这样关心?

便含笑向黛玉道:“你特地叫娘姨过去把我请到院中,现在好意思推我出去么?

我就依着你的话儿,在你院中借个干铺,但你却不许避到后房。我们大家规规矩矩的可好?“黛玉道:”只要耐八少肯赏光,是再好勿有哉啘。耐八少说格闲话,随便那哼倪总呒啥勿肯格,只怕倪呒拨格号福气。“说着背脸低头,掩口而笑,邱八更觉魂消。

这一夜,邱八就在黛玉院中住下。黛玉把平生第一等迷人的伎俩施展出来,任是邱八的外交学问再好些儿,已不知不觉的把一块主权所及的地方,轻轻地输到林黛玉的势力圈内去了,施着那禁制的压力,渐渐的不得自由起来。这邱八住了一夜,被黛玉骗得骨软筋酥,给了五十块钱的下脚,又体己给了黛玉三百块钱。黛玉故意分毫不受,退还邱八道:“倪故歇呒拨啥格用场,等到倪有用场格辰光再问耐拿好哉,倪倒勿像格号倌人单敲客人格竹杠。既然大家要好末,也勿在乎格点洋钱,八少阿是?”邱八听他说得有理,也便收回,心上反觉过意不去,便问黛玉可要什么衣裳首饰?黛玉一口咬定不要,反说邱八不晓得他的脾气,当他是爱抄小货的倌人。

邱八听了,那里晓得黛玉存着一个要借他淴浴的念头,只认得黛玉同他恩到极处,所以不肯叫他浪费银钱。

隔了两日,黛玉关照相帮,说先生有病暂时不能出局,须要调理几时。就有什么客人来到院中,黛玉自己不去应酬,只叫娘姨回覆有病不能出来,却成日成夜的伴着邱八,和他寸步不离。邱八一举一动都是黛玉亲身服侍,不肯假手他人。那班娘姨、大姐的趋奉殷勤更不消说。邱八因他们连日辛苦,另外给了一百块钱。黛玉执意不许,叫娘姨仍旧退还,自己却向邱八说道:“倪出仔工钱用仔俚笃,生来该应服侍格,要赏啥格洋钱!倪也晓得耐格脾气,勿要说是一百毛毛洋钱,就是一千一万,耐也勿放勒心浪。不过倪人末吃仔格碗断命堂子饭,倒勿是格号坏人,要倪坏仔良心敲客人笃格竹杠,倪从来勿行格。”说得邱八更加欢喜,伏伏贴贴的住在院中。

又隔了几天,黛玉看准邱八的性情已是死心塌地,没有什么变卦的了,那一天夜饭之后,黛玉正陪着邱人说说笑笑,甚是高兴,忽然皱着双眉,看着邱八。看了半晌,长叹一声,那一对秋波便流下泪来,慌得邱八连忙追问。黛玉只是不答应他,尽管低头温泪,那一种可怜情态,真如雨打桃花,风欺杨柳,画也画不出来。邱八见他这样,十分心痛,便挨着黛玉一处坐了,低低的问他。黛玉一言不发,只把粉面偎着邱八脸儿,拉着他的手呜呜咽咽的,那眼中的泪就是如乱滚珍珠一般,扑籁籁的流个不住。凭着邱八怎样温存,怎样追问,只是漠漠无言,直把个邱八哭得急了,恨不得自己替他,拍着胸脯道:“无论你有天大的为难,总有我一人承认。料想也没有什么做不到的事情,你快快住了哭,和我说个明白。你可知你哭到这个样儿,叫我心上好生难过,替又替你不得,倘若哭坏了怎么好呢?”黛玉听邱八说到这句话儿,心上好生欢喜,方才停住了哭,拭了泪痕,抬起头来看着邱八,叹一口气道:“别人家看仔倪末像煞蛮开心,倪心浪说勿出格心事,赛过勒浪黄连树底下弹琴。”急得个邱八做足道:“急惊风撞着了你这慢郎中,我这样的问你,你还要说着闲话。”黛玉道:“倪格事体才是肐里肐搭格,说起来也叫作孽。”

黛玉便装点了一番说话,说自己的亏空约有二万开外,又不肯坏了良心,敲客人的竹杠,所以生意虽然甚好,总是不够开销,以致亏空愈拖愈重;前节又被客人漂了两笔局帐,各店帐开销不转,几乎坏了名头,生意做不下去。添枝带叶,细细的向邱八说了一遍。又道:“倪故歇想起来,做仔格个断命生意,总归呒拨收梢,倪倒是早点肯坏坏良心末,也勿造至于弄到实梗样式,故歇倒是上勿上,落勿落,要除脱仔牌子勿做生意末,倪坍勿起格个台,要做下去末,倪实在拖勿起格亏空。

八少,耐替倪想想看,叫倪阿有啥格法子?“

邱八听了,哈哈的笑道:“我道你是什么天大的事情,要急得这般模样。原来不过是为着一点儿亏空,也值得放在心上,这样的张皇,难道我姓邱的这点事儿都担当不起么?”黛玉道:“耐八少看仔格点亏空自然呒啥希奇,像倪陆里想得出啥法子?”邱八道:“你究竟有若干亏空,不妨对我说明,待我替你慢慢的想法。”

黛王朝着邱八看了一眼,面上做出一付感激的样儿,却又朝他摇手,道:“谢谢耐格好心,肯替倪想法,原是再好勿有格事体,不过倪无缘无故拿仔耐格洋钱,叫倪心浪陆里意得过,故歇倪想起来,随便那哼总归还是嫁仔人格好。不过倪要嫁起人来,比仔别个倌人加二烦难。倪勒浪上海滩浪总算有点名气,老实说推扳点格客人,倪也看俚勿上。再说起格排滑头码子格年轻客人,要讨倪转去格多煞来浪,格是加二勿连牵哉。格个嫁人是一生一世格正经事体,勿是勒浪弄白相,倪又勿比格排呒拨长心格倌人,嫁仔人再要出来做生意。倪要末勿嫁,嫁仔人末陆里再好出来,所以倪拣来拣去,总归呒拨中意格客人,像耐人少一样格客人,倪看得总算中意格哉,耐人少咿是格规规矩矩格人,陆里肯讨格倌人转去?八少耐去搭倪想嗫;倪看中仔客人末,客人笃勿肯要倪;客人看中倪末,偏生倪又勿肯嫁俚。说来说去,总归一格勿成功。倪格种人活勒世浪,真真叫作孽嗫!”说着把眼睛挤了一挤,觉得眼里酸酸的好像又要流下泪来。

邱八听了黛玉这一番说话,就如新莺巧啭,娇鸟弄晴,又似成衣的熨斗一般浑身熨贴,三万六千毛孔无一处不曾熨到,满身发起奇痒,从骨髓缝中透出一股说不出的快活来,向黛玉笑道:“你也太多虑了!你既然想要嫁人,何不早些与我商议?

只要你自己心中情愿,没有什么委屈的地方,我总可以替你设法。只怕你心中不愿嫁人,三心两意的打不定主见,我就无从提起了。“黛玉道:”倪末阿有啥勿愿意格?倪格碗断命饭也吃得勿要吃格哉。只怕耐八少看倪勿中,勿肯要倪,倪也呒啥念头转呢。“邱八道:”只要你拿定念头,不要到了将来自家懊悔,我岂有倒反推辞的道理?但有一件,我却有些不甚放心,你须要自己心中打算,免得懊悔嫌迟。“

黛玉问他还有那件事儿不甚放心,邱八道:“你们做了倌人,身体是散淡惯的,一嫁了人,便要依着良家的规矩,有许多不能自由的地方。你们堂子出身的人那里受得住这般的拘束?我们二人,现在的交情是再好没有的了,但是要讲到‘嫁’‘娶’二字,也甚是烦难,不是可以卤莽从事得的。万一你心中不愿,口是心非,那时我把你娶到家中,进退不得,岂不是为好成恶,耽误了你一生一世的事情?所以我也要预先同你说明,好等你自家筹划,不要勉强应承,这倒不是玩的。”

黛玉听了着急起来,便拉邱八的手道:“倪格闲话,一塌刮仔才搭耐说完哉。

耐再要说倪三心两意,耐倌人阿有良心?耐既然勿相信倪末,等倪罚格咒拨耐听听,省得耐吓杀仔人。“说着,便发誓道:”倪要说仔一句假话,呒拨真心末,叫倪活勿过今年格大年夜。“邱八听了,连忙按住黛玉的嘴,道:”我不过一句话儿,你也值得这样的着急,一定要发起誓来。“黛玉道:”耐开口闭口总说倪是坏人,叫倪阿要发极格!“

邱八此时觉得心满意足,畅快非常,也说不出什么话来,只看着黛玉嘻嘻的笑个不住。黛玉横波斜睨,星眼朦胧,也用一方白细手巾掩口匿笑。四体慵抬,玉山自倒,倚在邱八身上,好像没有一丝气力一般。邱八便问他倒底有多少亏空?黛玉便一一的细说出来,却止有一半真情,其余多是虚报,约有二万开外。若在别人听了这许多亏空,怕不先就吓得顿口无言,筋酥骨软。幸而邱八家中真有百万家财,听了黛玉这些亏空,不过口中答应一声,全不在他心上。当夜黛玉又把邱八灌了无数迷汤,说了许多刺骨锥心的说话,追魂摄魄的深情,任是邱八花丛阅历的惯家,也免不得被他迷得梦魂颠倒。

到了次日,邱八便请了他一个朋友来,名叫陆友恭的,却是个有名的堂子帮闲、青楼蔑片。请了他来,与黛玉讲论身价。黛玉却一口咬定不要丝毫身价,只要邱八替他还清亏空,此外不取分文;并说他拣来拣去,并不是为着邱八有钱,为的是拣中邱八的人物,所以情愿嫁他。邱八起先尚有些疑疑惑惑的,没有十分决定,及至听了黛玉这一番说话,觉得十分入耳,好似鱼吞香饵,蝶恋花心,被他钩得定定的,那里还计算什么将来?当下一口许定,先替他还清亏空,然后择日迎娶。林黛玉见邱八已经应允,便立刻叫相帮的出去,把门首那一块一尺余长、四寸余阔、金地黑字的书寓牌子探了进来。黛玉亲手接了,放在桌上,回过身来笑迷迷的走到邱八身旁,并肩坐下,向邱八道:“故歇倪探仔格块牌子下来,倪就是耐格人哉,难是随便啥人到倪搭来,倪也勿见格哉。”邱八见他做事爽快,自是欢喜。

隔了一天,邱八便去划了一张二万银子的期票,先交与黛玉,到期付银;又择了三日之后,迎娶黛玉进门。黛玉收了邱八这张银票,也不知他究竟还了许多亏空,自家留下若干,这却做书的人未曾看见,不便讲他。

只说邱八在新马路赁了一所五楼五底的洋房作为公馆,以为迎娶黛玉的地方。

那公馆内铺设得十分富丽,尽是红木、紫檀镶嵌螺甸的木器,夺目辉煌;又有两间大莱间,都是外国家生,装饰得更是雅洁,邱八在上海的应酬本来阔大,那班知己些的朋友公送了两班髦儿戏,闹热非常。到了吉期,一样的红裙披风,朝珠补褂,清香彩轿,顶马高灯,把个四大金刚的林黛玉抬到家中。新人出轿之后,喜娘扶着黛玉,独自一人参拜天地,然后向邱八见礼。邱八连忙朝着喜娘摇手,叫他不要叩头,只行常礼。于是喜娘扶着黛玉深深万福,邱八也微微的还了一躬,方才送入洞房,大家饮酒。正是:

楼上花枝之影,昨夜星辰;枕边钿合之盟,春宵苦短。

欲知黛玉嫁了邱八,究竟如何,下文交代。





第二十四回 邱公子狠心惩爱妾 林黛玉拼命闹华堂





且说林黛玉嫁了邱八之后,邱八看承黛玉甚是殷勤,又恐黛玉坐在家中气闷,天天同着黛玉坐了马车到张园去兜个圈子。上灯之后,便同到一品香去吃顿番菜,有时吃过大菜再到丹桂茶园去看看夜戏,以为常事。黛玉倒也并不寂寞,所以嫁了邱八将近半月有余,倒还没有寻事生非、借端吵闹。

光阴迅速,已经一月有余。邱八因在上海耽搁久了,便和黛玉商议,要退了房子同他回到湖州。黛玉心上虽然不愿,却也无可如何,只好暂时答应一同回去,到了湖州之后再行计较脱身的法儿。邱八便雇了一号大船,把公馆中一切新买的器具一齐装载上船。黛玉也带了一个娘姨、两个大姐,收拾登舟。

邱八到轮船局中,单雇了一号轮船拖带,不消一日,早到了湖州。大船直顶到邱八门口的水码头停下,早有许多当差的一哄上船,先见了主人,再叩见了这位新姨太太,便乱烘烘把行李搬上岸去。邱八向黛玉道:“你既然到此,却不比住在上海的时候,上岸之后见了我们内人,先要你委屈一遭,朝他行个全礼,好在他平素为人甚是贤惠,待你一定不差,你凡事看在我的面上退让一分,尽他一个面子,我终不肯叫你吃亏。你可肯听我一句说话么?”黛玉听了面上登时变色,半晌不应。

邱八见他不肯,又说了无数安慰解劝的说话。黛玉无奈,只得勉强应承。

进门之后,见了那位八少奶奶,忍气吞声行了一个全礼。少奶奶果然甚是和气,见林黛玉朝他叩下头去,满面堆下笑来,一把拉住,连说不要客气。黛玉已叩完了头起来,连忙叫他坐下,说了几句闲话,又叫人替他赶紧收拾房间。一会儿房间已经铺设齐整,少奶奶便携了黛玉的手一同过来。黛玉见房屋高大,铺设鲜明,比上海的房间收拾得更加富丽,略略觉得安心。少奶奶送了黛玉进房,又向他道:“你要什么,只管向我去取。我家事烦杂,恐怕有料理不到的地方。”当夜又送了一席菜摆在黛玉房内,算是替他暖房,请了邱八进来一同坐下。是夜,邱八依旧住在黛玉房中。

到了明日,众家亲友晓得邱八回家,又新在上海娶了一个妓女,大家陆续登堂,纷纷道喜。只为邱八是城中首富,没有一人不趋奉他,把邱八倒忙了好几天。接着就是本城绅士,大家请酒,忙得打发不开。有时通宵在外,竟不回家;有时在家中书房安歇,还要料理家事,清算田租,盘查各处的帐目。因邱八出门已久,那帐目就堆积了一大堆,忙得个发昏,那里有返归内室的工夫?不要说是林黛玉房内绝脚不来,就是正室夫人也难得和他一面。别人也还罢了,这林黛玉是个有名荡妇,熬得清水直流。依着黛玉的本心,原只要借着邱八淴一个浴,替他还清债务,好等他脱然无累的重落风尘,并不是真心要嫁。现在邱八已经落了他的圈套,花了二万多银子把他娶到家中,总算是达其目的,如愿以偿的了。黛玉到了此际也没有别的心肠,只是辗转思量要想一个脱身之计。但是邱八是个有名富户,家中仆婢如云,而且规矩极其严肃,黛玉平日之间不要说想脱身逃走,就是等闲要走出中门一步,也是艰难,倒弄得进退两难,展变不得。黛玉方才懊悔起来,左思右想没有法儿,只得慢慢的打鸡骂狗,借事生端,渐渐的露出不安于室的样子来。幸亏邱八的正室夫人甚是贤惠,不去与他计较,黛玉无从费气,无可奈何。

不觉又过了几天,邱八把两月中欠积的事情料理清楚,应酬也渐渐的少了,晓得黛玉已经久旷,便先到黛玉房中住了一夜,觉得黛玉待他冷冷的不甚应接,那神气之间也是十分萧索,默默无言。邱八大为诧异,便留意看他举动,却又不好意思问他。

到了午后,黛玉便向邱八道:“倪到仔间搭一格多月,人也几乎闷煞快,再要实梗样式下去,是实头要生病哉。倪明朝要到上海去住格两日,让倪去坐坐马车,吃吃大菜,等倪散散心看,勿然是坐勒屋里向,倪头脑子也涨格哉。耐阿肯同倪去?”

邱八听黛玉说得容易,倒好笑起来,便回报他道:“你从前住在上海是在堂子里头,况且又是自家身体,天天可以出门。现在你既已嫁人,便是良家妇女,理应守着规矩,轻易不可出门。就算现在你要到上海,我同你一同前去,也比不得当初你做着倌人,可以随心任意到处招摇。我先时原曾和你说过,恐怕你做过倌人,受不得人家的拘束。现在我娶你到家不到两月,你果然已经不惯起来,可不被我料着了么?”

黛玉听了,面红眉竖,不发一言,停了半晌方才冷笑道:“倪住勒浪上海格辰光,看见几化人家格太太。小姐,日日勒浪坐马车游张园,做仔人家人,勿相信大门才出勿得格哉。倪又勿到上海去轧啥格姘头,啥格希奇勿煞格事体,阿要像煞有价事?”

说着,又冷笑了一声。

邱八听黛玉出言生硬,忽然同他顶撞起来,从前那一付温柔婉转的神情不知消到那里去了,顿时换出一付铁铮铮的面色来,心中已有了七八分怒意。还只道黛玉是无心顶撞,勉强按住了怒气,又向他说道:“你坐在家里没有什么事情,气闷起来,原也怪你不得。只要你除了上海去的念头,凭你要想着法儿如何消遣,我总依你的话就是了。”黛玉听邱八的口风始终不肯放松,心中甚是着急,又见邱八并不翻腔,话风倒反有些迁就,越发胆大起来,把邱八也只当作寻常公子哥儿,易于打发,便又向邱八道:“倪上海是定规要去格,耐勿要勒浪扭结固结,耐勿肯同倪去末,倪自家一干仔去末哉。”邱八听了,再捺不住,那心上的火直冒到顶门上来,也冷笑道:“你说得好轻松说话!从来嫁鸡随鸡,嫁狗随狗;你既然嫁我,便要听我的指挥。你还当在上海做着倌人,凭着你的性儿胡闹,无人管束么?老实对你说声,我邱八不是个省事的人物,叫你自家见亮早早收篷;如若再要不知进退,随口胡言,那时间莫怪我反面无情,不留你的地步。”

黛玉见邱八反了面皮,心上一毫不怕,却自己心中想道:若不与他这一个决裂,那里撒手得开?这样蝎蝎螫螫的将就下来,何时得个了局?不如借着他翻脸的题目,索性和他大闹一场,且看他怎生应付,再作道理。想定主意,便也翻转面来,粉面通红,蛾眉倒竖,大声说道:“耐勿要缠错仔人!倪嫁末总算嫁拨仔耐,勿见得有啥格卖身文书。耐要管牢仔倪,叫倪一直勿要出去,今生今世耐做勿到格哉。老实搭耐说,倪上海末定规要去格,明朝倪一干仔动身,看耐阿有本事拉牢仔倪,随便耐去那哼,倪总勿见得怕仔耐格。”

邱八起初还认林黛玉真是看中了他的人物,一心一意的嫁他,并没有要他写什么婚书卖契。现在听了黛玉这一番说话,方才晓得黛玉是借他淴,骗得他的银钱到手,登时掉过头来,拿定邱八没有婚书,又没有借据,就是告到当官,那邱八也只好眼睁睁的看着他重落风尘,说不出一个“不”字,也算得诡计阴谋毒如蛇蝎了。

当下邱八听他说出这一番说话来,明知自己当初大意,没有婚书,拿不住他的把柄,这一气气得非同小可,顿口无言,一时呆在椅子上竟说不出什么说话。呆了半晌方才回过这一口气来,定一定神,跳起身来指着林黛玉的面孔,骂道:“我把你这良心丧尽的混帐东西!你把我当作瘟生,这是你的运气来了。你当初没有进我的门也还罢了,现在你既然进了我的大门,凭你如何,你休想移挪一步!你把我也当作那班曲辫子的客人,就如木偶一般,凭着你颠来倒去的闹玩意儿么?你口口声声想到上海,那里有什么事情?无非想到了上海,捉个空儿逃走出去,过了一年半载,等得我这里事情冷了,你却依然做起生意来。我劝你休要打错了念头,你既然嫁我,便是我的人,我不许你出去,看你有什么本事飞上了天!”

黛玉听了愈加着恼,也立起身来道:“耐勿许倪出去末,倪定规要去,看耐有本事那哼!开口闭口总说倪故歇嫁仔耐哉,倪嫁耐阿有啥格凭据?耐倒拿倪格婚书出来大家看看。老实搭耐说仔罢,嫁人呒拨婚书是勿好算数格。耐格一转末总算上仔倪格当哉,下转叫耐学学倪格乖,勿要再上仔别人家格当去,阿晓得?”一面说着,一面带着同来的娘姨往外就走,口中说道:“倪要少陪耐哉,倪格衣裳首饰,一塌刮子送拨仔耐阿好?倪也勿要哉。”

邱八被黛玉说得七窍生烟,三厂暴躁,回过念头一想:“当初果然上了他的恶当,不曾要得一张婚书,现在就是和他打到官司,两下都没有凭据,他只要绝口不招,也和他争执不得。花了二万开外的银子也还罢了,但是自己向来自负是个花柳惯家,从不曾着了别人的圈套花这冤枉的银钱;现在受了林黛玉这样的一个骗局,还仍旧被他走到上海,再落平康,非但坏了向来的名气,将来到了上海,怎样有脸见人?”心中正在万分懊悔,又见黛玉摇摇摆摆的一直往外就走,更是烈火飞腾,猛然间把心一横,想道:“他这样的奸刁十恶,难道我就看他走了不成?无论如何,拼着再花掉一注银钱,也没有什么不了的事。”主意已定,连忙追上前去。

黛玉刚刚跨出中堂,被邱八赶到后边,把黛玉的衣服一把揪住,用尽平生之力向内一拖,把个林黛玉拖得几乎跌倒。邱八拖住了黛玉,不等黛玉开口,一片声叫:“来人!”就有四五个家人听见,答应一声齐赶进来。见主人与黛玉这个样儿,都吓得不敢开口,垂手立在一旁。邱八气呼呼的指着黛玉道:“你们快把他捆起来!”

众家人听了,你看着我,我看着你,面面相觑,一个也不敢动手。黛玉听得邱八要叫人捆他,趁势撒起泼来,望着邱八一头撞去,把邱八撞了一个躘. 黛玉便滚倒在地,把头发技在背后,就像活鬼一般,反大哭起来。急得邱八朝着家人顿足,骂道:“你们这一班无用的奴才,怎么我叫你们捆他,你们大家不肯动手?明天你们替我一起儿滚蛋,我用不着你们这起混帐东西!”众家人立在旁边本来不敢动手,听得主人这般发急,没奈何上来几个,走到黛玉身旁正要动手,谁知林黛玉老奸巨滑,看见邱八认真翻起面来,不是头路,此刻自家身体还在别人手内,眼前不免吃亏,见众家人一拥上前,明知不好,连忙住了哭,在地上扒起身来,不等众人动手,一溜烟望自家房内就走。邱八见他仍旧缩回房内,冷笑一声,暂时叫住家人不要动手,自己跟着黛玉也走进来。

只见黛玉刚刚走到房内,一直抢至烟榻旁边,把榻上烟盘内的一个洋錾白银烟盒抢在手中,随手开了盒盖,把那一盒子装得满满的鸦片烟,望着自己的口内作势便倒。说时迟,那时快,早被旁边一个带来的娘姨从背后伸过一只手来劈手夺去,口中喊道:“大小姐,耐有啥格闲话末,好好里搭俚说末哉,年纪轻轻,啥格就要寻死路。”黛玉装作恨恨的样儿,向那娘姨道:“倪格号人身活勒世浪无啥趣势,还是死仔格好,耐勿要来多管嗫。”说着假作要夺那娘姨手中的烟盒。娘姨急得看着邱八,口中嚷道:“大小姐要吃生鸦片烟哉呀,唔笃大家来劝劝嗫。”黛玉一面在那里用力的要抢娘姨手中的烟盒,两人结做一堆;一面却偷眼看着邱八的面孔,指望他怕他寻死,心中不忍起来过来解劝,便算自己占了上风。那知道邱八绝不关。

也也不过来相劝,只望着黛玉和娘姨二人不住的冷笑。黛玉见了这般光景,明晓得那邱八已经看破机关,倒反弄得开交不得。

正在左右为难的时候,恰好那位八少奶奶听得他们吵闹,赶了过来。刚刚走进房门,见黛玉这般做作,认以为真,不免大吃一吓,连忙赶上前去,把娘姨手中的一只烟盒接了过来,随手就向门外一摔,只听得“当啷”一声,一个装烟的银盒子不知撩到那里去了。又把黛玉拖了过来,捺他坐下,口中劝道:“你们偶然斗口,也是人家常有的事情,有话也须好好的说,为什么这样的认起真来?”黛玉此时正是不得落场,万分惭愧,巴不得有人相劝,连忙借此坐下,泪流满面,默默无言。

忽听得邱八冷笑一声,指着那位少奶奶道:“你这个人真是十分多事,为什么要去劝他,你道他的寻死是当真的么?”正是:

画中爱宠,凄凉白纻之歌;镜里萧郎,辜负天魔之舞。

欲知邱八究竟肯放黛玉出来与否,请看下回。





第二十五回 恨无良闭户锁金刚 消妒意开笼放鹦鹉





且说邱八见他正室夫人进来相劝,便指着黛玉,把前后被骗的情形细细的告诉他:黛玉如何叫娘姨请他,摆了一个双台,当时就落了水;如何黛玉竭力奉承,把他哄得死心塌地,花了二万几千银子把他娶到家中;如何上了他的圈套,没有要他的婚书,现在他翻转面皮,一定要往上海。“因我不肯放他出去,他同我抢白了一场,竟自往外就走。我把他拉了转来,又要叫人把他捆住,他便打滚撒泼,寻死撞头。他的意思是要我怕他时常吵闹,放他出去,便好随心适意,安安稳稳的重落风尘。后来见我咬定口风不肯答应,他没有什么法子,只好寻死觅活的指望吓倒别人。

幸而遇着了我不怕什么风波,若是换了别人,怕不被他吓倒?你道他这样的心思可刻毒不刻毒!这样的混帐东西,凭他当真死了便罢,为什么你又多事起来?“

那位八少奶奶听了邱八这一番言语,方才如梦初醒,暗想:“堂子里头的倌人果然恶毒!”又恐黛玉当真的寻起死来,也是一条人命,便劝着邱八道:“虽然如此,倒底人命关天,不是顽的,况且我们这样人家,也不在乎这点儿银子。他既不肯跟你,勉强留他在此,料想也没有真心。依着我的意见,不如依着他的话,把他打发出去,省得他心中不愿,天天的寻事生非,何必费了自己的功夫,同他淘这般闲气!”

邱八听了,低头想了一会,道:“你的说话,虽是不差,但是你还没有晓得细情。我花了许多银子替他还债,倒也并不怪他;最可恨的是他把我当作瘟生看待,说的话都是虚无缥缈的,没有一句真情。我当初再三再四的问他,可是真心嫁我?

他一口咬定,不肯露出一点话风,哄得我满心欢喜,对着一班朋友说了许多大话,吹了无数牛屄。到了今日之下,依旧把他放到上海做起生意来,将来他们追问起来,叫我怎生回答,岂不是倒坏名声?不瞒你说,我自从出世以来,从没有受过这般恶气,现在他既然同我蛮缠,不讲情理,我也会些蛮派,把他关锁起来,不怕他生出翅膀飞上天去。就算他当真死了,这样害人不浅的东西,省得把他留在世上再害别人。你若是怕他死了,有他的父母兄弟来同我吵闹,告状经官,我只要拼得再花掉一注银钱,就买了他的一条性命。料想如今世上只要银钱作主,没有什么不了的事情。你凭着我怎样安排,不要来多管闲事。“说着,便喝叫众人一齐出去,单留黛玉一人在房。

邱八也立起来,指着黛玉的脸道:“你要寻死,凭你去上吊吞烟,快些死了,好等我预备官司。我拼着再花二万银子,买嘱你的尸亲,怕不是安安稳稳的闭口无言?你丢了一条性命,只当死了一只猫狗一般,看还是你的性命值钱,还是我的银子值钱!”一面说着便走出房去,就取了一把洋锁。“咯噔‘一声把房门锁上;又叫家人去叫了一个木匠来,在板壁中间开了一个尺余见方的壁洞,就像衙门内的转桶一般,好做传送食物的地方。另派二个家人交起板铺来,睡在中堂,看守房门,防他逃走。

只说黛玉听了邱八的话,心中暗暗吃惊;又见邱八气势淘淘,料想他已经气到极处,万万挽回不来;却又恐怕吃了现亏,不敢开口,眼睁睁的看他锁着房门走了出去,方才懊悔自己当初不应错了念头同他蛮闹,却已无可如何;又不肯当真自录死路,跳又跳不出去,走又走不来,只得坐在房中哭泣咒骂,头也不梳,脸也不洗,糟蹋得蓬头垢面,就如个腌攧花子一般,那里还有当初的丰致?真是:

慵梳宝髻,惺松堕马之妆;愁倚熏笼,寂寞惊鸿之影。银华不御,芳泽无加;珠泪琳琅,玉容惨淡。

一个邱八公子的府中,差不多变做了江采苹的宫院。黛玉被他锁在房中一连就是半月,虽是饮食不缺,却是懊闷异常。幸而黛玉还有几年花运,平空降了一个救星下来,你道那救星是谁?原来就是那位八少奶奶。

从来女子的性情,总不免有些娇妒。这位八少奶奶正在妙龄,又同邱八十分恩爱,平空的邱八娶了一个花枝般的宠妾,要与他分恩夺爱起来,那得不心怀妒意。

但是他平日为人温厚,性格和平,无论什么事情,不肯放在面上,所以黛玉进门之后,心上虽然不乐,面子上却做得甚是殷勤,不但讨了邱八的喜欢,还落得博一个贤惠的名气。现在见邱八把黛玉关锁起来,心中未免一愁一喜。喜的是眼前去了这样一个搔头弄姿、顾影自喜的妖姬,邱八心无二用,那夫妇间的恩爱登时就加了几分。正是:卧榻之旁,岂容他人酣睡?愁的是邱八虽然把他锁在房中,却是余情不断,时常叫家人仆妇走到那壁洞之前与他问答,探问他的意思,看他可有些儿悔悟;分明邱八的心上尚在系恋着他。万一将来回味思量,磨折几时,依时把他放出,他二人一个是风月名娼,一个是豪华公子,那时黛玉放出二十四分的工夫手段,怕不把邱八依然骗得个意服心输?到了这个时候,赛又赛他不过,赶又赶他不掉,岂不倒是一个后患?他想着这两层主意,心中便怀着鬼胎,天天解劝邱八道:“黛玉虽然可恶,然而也是妓女的常情,不算什么奇异。本来一个堂子出身的妓女,那里有什么良心?你把他当作好人,已经错了;现在你又把他锁了起来,他是个散淡惯了的人,那里受得起这般磨折?我们世代忠厚,从没有做过刻薄事情,万一他当真死了,你虽然没有逼他,总是你身上的孽障。不如看破些儿放他出去,听凭他去再做生意,或者重新嫁人。譬如当初没有嫁你,你也管不着他。况且你娶他的时候又没有什么媒证婚书,更是作不得准。难道你丢下了一个妓女,就算坏了你的名气么?”

劝来劝去,邱八先起那里肯听,连连摇头。当不得他被底温存,枕边旖旎;今日劝,明日劝,竟把个邱人劝得活动起来,便一口答应。八少奶奶大喜,还恐他要变卦,连忙叫人去开了房门,把黛玉叫将出来。

黛玉此时已经被邱八把十分性子磨去了九分,粉黛纵横,泪痕隐约,听得叫他出去,心中估量着一定是邱八回心,却想不到竟肯放他出去。当下将就换了一件衣服,淡扫蛾眉,走到邱八房中,叫了八少奶奶一声,又瞅了邱八一眼,粉头低垂,春山不展。邱八留意看他,只见他云鬓蓬松,芙蓉惨淡。瘦比经秋之燕,弱不禁风;娇如解语之花,含情欲涕。真个是暗呜如泣,幽怨可怜,大有伤心之色,早不觉心上怜惜起来。八少奶奶明知邱八的意思,不等他开口,先把自家劝解的话,向黛玉说了一遍,又说:“八少已经应允放你出门,你可快去把你随身带来的衣饰立时收拾,你要到上海,今天就可动身,省得又要耽搁一夜。”

黛玉忽然听见邱八答应放他,这一喜非同小可,好像那寒儒登第,枯木逢春,又好似刑部狱中的囚犯逢了郊天大赦一般,登时色舞眉飞,走将过来,朝着八少奶奶花摇柳颤的磕下头去,八少奶奶忙忙扶起。黛玉回过身来,见邱八一双眼睛只钉在他的身上,黛玉此时喜到极处,忘其所以,便无可不可的,朝着邱八也磕了一个头。邱八别转头去没有扶他,却不由的口中长叹一声,默默无语。八少奶奶怕他又要反悔,急急的催着黛玉收拾衣箱。黛玉嫁来的时候,自家止有六只衣箱,其余都是邱八替他置备,现在仍叫黛玉把原带来的衣箱带去。

黛玉莫莫的收拾了一会,带着同来的一个娘姨、两个大姐,辞别了邱八和八少奶奶便要出门。八少奶奶索性做个好人,早叫人替他雇了一乘轿子,一直送他到轮船码头。黛玉此时就是鲸鱼脱网,彩凤开笼,恨不得一步就跨出门去,忙忙登舆而去。

这里邱八见黛玉出门,心中不免有些恋恋,但一则已经答应,反悔不来;二则明知黛玉不是真心,留他无益,乐得听了他夫人的说话做个好人;三则自己把他关锁多时,不肯折着志气,反去留他。有此三层事理,所以邱八勉强放他出去,虽是心中不舍,也是无可如何。可笑那林黛玉骗了邱八二万余金替他还债,自以为是得计的了,不料偏偏遇着了这样的一个惫赖人物,非但吓诈不倒,反吃了一场大亏,几乎白送了一条性命,这也是林黛玉平时丧尽良心的报应。邱八这边按下不提且说黛玉出门之后,一直径到轮船码头,发下衣箱行李,写了一间上海房舱,不消一日工夫,早到上海。暂时落了客寓,不多几日,便看中了惠福里的一家房子,三楼三底,甚是宽大,当下付了房租,立时搬了过去,置备了些中西器具,登时铺设得焕然一新。他从前骗了邱八的二万银子,还债赎当止用得一万多些,其余的都暗地托人存在庄上。此番到了上海,犹如死里逃生的一般,觉得喜出望外,便自己到钱庄上去了一趟,把他那些存项取了一半回来,任情挥霍。依旧的珠围翠绕,罗绮辉煌,时常坐着马车到张园兜个圈子,回来的时候在大马路、四马路一带出出风头,却暂时不敢再做生意。听着那邱八的风声,只把惠福里的房子当作住家。早不知不觉的过了两节,打听得邱八已到过两趟上海,却把林黛玉的事绝口不提,就是那一班朋友也恐他要恼羞成怒,不便去追问于他。黛玉打听得实,放下了心,方才打算要再做生意,挂起牌子来。、接十天已黛玉坐着马车正要到张园去,刚刚马车跑到泥城桥方缺油之中遇着了章秋谷的马车、黛玉见秋谷坐在车中,气宇轩昂,衣裳倜傥,长眉秀目,光彩照人,不觉芳心微动。便横波凝睇以目送情。无奈两下的马车都跑得风卷云驰,倾刻之间那眼前就如电光一闪,两下早已跑开。黛玉直待马车跑过之后,方才猛然想起好像章秋谷的神情,姑且冒叫一声看他答应不答应,便立起来高叫一声。听得秋谷在前答应,方知真个是他。黛玉心中大喜,连忙叫马夫转过马车,跟着秋谷直到一品香来。当下把一年的境遇向秋谷细细说明,说到邱八把他关锁在房一节,黛玉不免还有些谈虎色变,毫发悚然。

当下二人促膝密坐谈了一回,秋谷便问黛玉究竟作何行止,黛玉道:“倪也无拨啥一定格主意,晏歇点耐阿好到倪搭来一埭,大家商量商量。”秋谷摇头道:“我今天有自己的事情,连几处台面都不能应酬,料想没有空儿。我看还是明天罢!”

黛玉点头答应,又告诉了他住处的门牌。不一时吃完大菜,已是掌灯,黛玉自回惠福里去,秋谷便一直到吉升栈来。

到了栈内,在自己房内略坐一刻,便走到双林房内来。双林早已回来,凝妆悄坐,低问秋谷为何此刻回来。秋谷把遇见黛玉之后,在大菜间谈了一点余钟,所以回来晚了。双林又问他今天可要出去。秋谷不答,只把头点了一点。双林睄了秋谷一眼,便不作声。秋谷心中暗笑,假作不知,略谈几句便起身出栈,径到新清和张书玉院中来。

书玉恰好在家,迎门相候,满面堆欢的叫了一声:“二少!”秋谷含笑招呼,跨进房来。书玉亲手替他宽了马褂,又叫他脱去长衫。秋谷因五月中旬天气已经燥热,便略略点头。书玉一并替他宽了下来,把一件罗纺长衫、单纱马褂交与旁边的娘姨,朝他使个眼色。那娘姨会意,便把两件衣服折叠起夺,开了衣橱,把秋谷的衣服放在橱内,取过一把锁来轻轻的锁好。秋谷见了,明知书玉的意思,并不开言,只是对着书玉微微而笑。书玉此时心花大放,乐不可支,极力的应酬秋谷。秋谷心上虽言不甚情愿,却已到了这步田地,就是坐怀不乱的柳下惠,也不得不随和起来。

夜分之后,书玉扫榻熏香,殷勤留宿。秋谷料想推辞不得,只得应承。

这一夜,章秋谷的神情,却是曾经沧海,难为洛浦之波;除却云英,不是蓝桥之路。在张书玉是当时相见,已销情女之魂;今日重逢,留得宓妃之枕。凤女之颠狂如许,赵后回风;擅奴之华彩非常,何郎无恙。

只说秋谷在书玉院内住了一宵,明日起来,照例开销了二十块钱下脚,书玉一定不肯。推了多时,见秋谷面上已经微含怒意,方才叫娘姨收了。秋谷便要起身,书玉千叮万嘱的叫他晚上一定要来。秋谷道:“这却不能一定。没有事情,自然来的;倘或有了正事,这却要耽搁一天的了。”书玉无奈,一直送下楼梯,走到屏门边方才立住,望着秋谷出了院中,一步懒一步的回上楼去。正是:斋

窥中堂之韩令,贾午留香;感汉浦之郑郎,洛妃解珮。

未知秋谷再到何处,请听下回交代。





第二十六回 说瘟生平心论嫁娶 评嫖客谈笑骂官商





却说章秋谷在张书玉院中住了一夜,将近午刻方才出来,走出新清和弄内,穿进迎春坊,径到金小宝院中来。

上了扶梯,走进房内,只见金小宝坐在当窗一张桌上,正在那里对镜梳头,鬟凤低垂,新妆未竟,地隔夜的胭脂映在脸上,晕出淡淡的红色,越觉得丰神绝世,妩媚天然。身上穿一件半新的湖色熟罗短袄,衬着粉红席法布紧身,胸前的钮扣一齐解散,微微的露出酥胸;内着湖色春纱兜肚;下身穿一条品蓝实地纱裤子;脚下拖着一双湖色缎子绣花拖鞋,双翘瘦削,就如玉笋一般,不盈四寸。手中正在那里调和花露,一阵阵的脂粉之香中人肺腑。眉弯秋月,颊晕朝霞,真是春意透酥胸,春色横眉黛。秋谷见了小宝这般风格,不由不暗暗称扬。又见贡春树坐在小宝旁边呆呆的看着,一言不发。

秋谷悄步进来。走到小宝背后。春树正在那里看得出神,全不觉得有人走进。

小宝本是对窗坐着,秋谷轻轻的掩至后边,连那同小宝梳头的娘姨都一毫不觉。金小宝正在对着镜子,细匀铅黄,忽然看见镜子中间添了一个朱唇粉面的美少年立在自家背后,笑容可掬的像要和他说话一般。金小宝出其不意,大吃一惊,吓得他满身香汗,直立起来,叫得一声“阿呀”,回头一看,见是章秋谷立在身后,方才定了心神,已经吓得花容失色,娇喘微微。重新坐下,向秋谷笑道:“耐末总是实梗,走进来响也勿响,人也拨耐吓煞快。人吓人,要吓杀人格嗫!”春树被小宝叫了一声“阿呀”,直头起来,也吓了一跳,抬头见是秋谷,急忙离座相迎,拱手称谢他昨日替小宝解围的好意。

秋谷笑道:“你为什么预先躲避,有心不到张园?你还没有看见昨日的势头,若不是我来解劝,恐怕小宝定要吃亏。从前我原曾向你说过几次,张书玉的性情十分惫赖,不是好说话的人。你住的一夜,又没有什么口角,无缘无故的忽然不去,冷淡起来,偏又被他晓得风声,你成日成夜钻在这里,差不多竟是和他断了交情,怪不得书玉吃起醋来,闹出这场笑话。幸而昨日遇着了我,小宝没有吃亏;万一我不到张园,无人解劝,小宝必定被他揪扭,吃了一场现亏。在千人百众的地方叫他受气坍台,你怎的对他得起?”一席话说得春树闭口无言,面上狠觉有些惭愧。小宝又在旁插口道:“二少格闲话倒的刮嗫,昨日仔勿是二少刚正跑来,拿格张书玉拉仔进去,是倪直头一塌糊涂格哉。”说着,便拉着秋谷的手,笑道:“谢谢耐替倪拉开仔格张书玉,总算倪朆坍台,倪也呒啥补报耐,只好屁股吃人参──后补格哉。”说着,小宝先格格的笑了。秋谷道:“你们真好良心,果然一张床上睡不出两样人来。”说到此处,小宝脸一红,把秋谷肩上打了一下。

秋谷又道:“昨天的事情,原是因你二人而起,我来是个旁人,不干我事。好意前来解劝,恐怕你要吃亏。那知你们二人一样心肠,把自己的事情都卸到旁人身上。一个预先不肯出来,一连忙走了回去,只叫我替你们顶缸,今天还要开我的玩笑,你们自己想想,可有良心么?”春树道:“我昨日实是有事进城,并不是有心躲避,直至晚上一点钟时候方才回到此间。不信,你问小宝便知真假。”秋谷道:“你们两人这样的开心,却苦着我这旁人调停劝解,费了我无数功夫。你自己不听我的言语,惹出事来你倒像没事的一般,可不是笑话么?”春树听了,果然回心一想有些过意不去的地方,连忙向他谢罪,秋谷也一笑无言。

金小宝坐在旁边听他说话,却不住的一双俊眼看着秋谷的脸儿,目不转睛的浑身上下只顾打量。秋谷回头看见,不觉笑道:“诧异得狠,你为着何事,看得这样认真?”小宝不答,又细细的看了一回,方向秋谷笑道:“耐一面孔格勿尴尬,定规是昨日勒浪张书玉搭出来啘。”秋谷被他一口道着,不觉微笑点头。小家又笑道:“耐前日仔末,叫倪‘土地奶奶’寻倪格开心,故歇倪也要叫耐‘金刚老爷’哉!”

说得一房间内的娘姨多笑起来。秋谷更狂笑道:“我倒不是什么金刚老爷。”拍着春树道:“你们这位贡大少爷,倒是个实缺的金刚奶奶。”春树笑道:“你们大家取笑,却无缘无故的把我带上,可和我什么相干呢?”大家说笑一回,随意坐下。

秋谷忽问小宝道:“你可晓得林黛玉如今又到了上海么?”小宝道:“倪是老早就晓得格哉。张园里向也看见歇俚几转。俚耐上年仔嫁仔邱八,一淘转去格,勿晓得俚为啥咿要出来?”秋谷就把黛玉嫁了邱八之后这些肐瘩事情,一段一节的对着小宝细讲,原原本本的直讲了一点余钟。恰好贡春树见秋谷到来,料想他没有吃饭,就到聚丰园叫几样菜,两壶京庄,一同摆了上来。上宝过来斟了一杯酒,便请秋谷上坐。贡春树坐在横头。小宝因秋谷是极熟的客人,便也不拘俗套,随意相陪。

秋谷一面饮酒,一面演说林黛玉嫁人复出的事情,把个金小宝听得津津有味。春树在旁听着,也嗟叹不已。

小宝道:“格是林黛玉自家勿好,朆看得清客人,马马虎虎格跟仔别人就走,自然弄勿好哉啘。”春树道:“妓女嫁人,嫁着了邱八这样人家,也算手中选一的了;为什么黛玉还要闹着出来?可见得堂子里头的人,果然一个个丧尽良心,怪不得邱八要这般着恼。幸而邱八毕竟是个好人,还肯开笼放鸟。若是我做了邱八,真把他要关禁终身,那里有这样便宜,好好的放他出去!”

金小宝听了春树这样活风,瞪了他一个白眼,冷笑道:“倪堂子里向倌人,生来阿有啥良心,就是客人到倪搭来末,也是客人笃自家情愿,勿见得客人勿来,倪去拉仔进来格。耐下转当心点,倪堂子里向才是坏人,耐勿要上仔倪格当。”说着,眉尖微竖,俊眼含瞋,薄有几分怒意。春树道:“我不过一句话儿,又不是有心说你,为什么要你这样留心,无端生气?”小宝道:“耐说倪堂子里向才是丧尽良心,还说勿是有心骂倪,阿要叫仔倪金小宝格名字,多骂两声?”春树见小宝一定说骂的是他,无从分辨,只得任他说了几声,含笑不语。

秋谷向春树道:“你刚才的话虽然不错,未免也太过了些,不可一概而论。据我看来,青楼妓女自然大半都是些无耻丧心之辈,然而替他们设身处地细细想来,却也怪他不得。为什么呢?你想,堂子里的倌人做的本来是迎新送旧的生涯,若不说着假话哄骗客人,那里有什么生意?没有生意岂不要倒贴开销,你叫他的良心如何好法?大凡一个好好的良家女子,无可奈何做到了这行生意,已是可怜,做客人的应当可怜他,爱惜他,不要扳他的错处,把他们当作个暂时消遣的名花好鸟一般,才是做客人的道理。所以花街柳巷,俗说叫做顽耍的地方,你想既是顽耍之地,原不过趁着一时高兴,博那片刻的风情。倌人相待殷勤固然最好,就是倌人看承不好,也没有什么希奇。上海的地方甚大,堂子极多,除了一处,还有别人,你就随意跳槽,他也不能禁止,更何苦去争风吃醋,处处认真,实做那‘瘟生’二字。总而言之,倌人看待客人,纯是一个‘假’字,客人看待倌人,也纯用一个‘假’字去应他,切不可把他当作真心,自寻烦恼。若要在酒阵歌场之内处处认真起来,就要如邱八一般,三十岁老娘倒绷孩儿,若不得要闹出一场笑话。你们以为何如?”金小宝听了,连连点头。

春树又道:“话虽如此,但邱八看承黛玉狠是不差,况且邱八预先问过黛玉,叫他自己商量,黛玉一口咬定,定要嫁他,邱八方肯娶他回去。娶到家中之后,黛玉不该又要出来。既然不肯嫁他,为什么要随口答应,叫他还债呢?这不是有心敲邱八的竹杠么?你为什么还要偏护着他,说他不错?”秋谷道:“你说的通是公子哥儿的痴话,全不是我的本心,我何曾偏护黛玉,说他不错?我的意思是说黛玉虽然丧尽良心,邱八也一半自己不好,平空的去问黛玉可肯嫁他。你想堂子里的倌人做的是什么生意,又做着了邱八这样的一个有名阔客,乐得顺水推船,哄得他一个死心塌地,方好骗他大注的银钱,那里有当面回报不肯嫁他之理?就是别个客人,也不能这样有心得罪,何况邱八是个浙江通省的富家。这一问,岂不是问得痴到极处么?还有你这般痴了,当真的同我辩驳起来,可不比邱八更痴一倍么?”春树听了,觉得果然是言言透澈,沁人心脾,便道:“如此说来,上海的倌人是万娶不得的了。”

秋谷道:“也不是这般说法。大凡天地生人,必有本来的性情,就是客人也有客人的脾气,倌人也有倌人的性情。倘或嫖客的性情同倌人不合,倌人的脾气与嫖客不投,就有石崇、王恺的家财,西子、太真的丰调,用了九牛二虎之力,也弄不到一块来。若勉强把他并到一堆,彼此的性情不合,一定要闹出笑话,没有好好的收场,岂不是一个为好成仇,一个求荣反辱?何苦要闹到这步田地,弄得两败俱伤呢?即如邱八与黛玉的交情原是十分要好,不过是大家一时鲁莽,没有仔细思量,草草的一个嫁了过来,一个娶了回去,到后来毕竟闹了一场笑柄,倒反大家结了冤仇。所以依我看来,花柳场中只可暂时取乐,就如行云流水一般,万万不可认真,免得后来烦恼。譬如一树名花,种在那水边篱落,临流照影,姿媚横生,你就天天的载酒看花,暂时领略,也未尝不妙,何苦一定要伤根动叶,把他移到家中?虽然锦帐雕栏,殷勤爱护,却是离开了他自己的托根之地,未免水土不宜,雨露不润,眼看着那一株可爱的名花不由的叶萎花落,渐渐的憔悴起来。这还算是好的,更有硬硬的折了一枝,把他供在花瓶之内,天天相对,爱惜非常,却过得不多几天,依然枯死。假使花能解语,你问他可是愿意的么?大抵上海的倌人,只好把他当作名花娇鸟一般,博个片时的欢乐,若定要将他娶到家中,就免不得要杀风景了。从古以来,煮鹤焚琴,蹂香躏王,煞是伤心,这就是这班妓女嫁人的小影……”说到此间,回过头来向金小宝打着苏白道:“先生,倪格闲话阿对?”金小宝正在听得出神,就如醍醐灌顶,草木当春,正在赞叹之际,忽听秋谷问他,连忙点头笑道:“二少格闲话,一句勿错,真真是格过来人哉!说出来格闲话,赛过勒倪心浪挖出来格。不过倪要说起来,讲勿出格当中格道理。”

春树又问秋谷道:“上海倌人的现形,你已经同我说过几番,大约也不过如此。

但是上海嫖客的情形,你没有和我讲过,究竟倌人做起客人来,情愿做那一种呢?“

秋谷道:“现在上海的客人,大约要分两种:一种是官场,一种是商界。论起来,自然是商界的客人好做,既肯花钱,又不闹什么嫖劲,倌人们看着银钱面上,也不得不敷衍他些。但是也有一样难处,那些商人平日之间寸铢积累,刻薄成家,看得那银钱十分郑重,你若要起他的钱来,比要他的命更加刻毒,万一浪费了他一文半钞,更是一生的切骨之仇。独独到了堂子里头,挥霍起来一日千金,绝无吝色,面子上装得甚是大方。谁知他花了银钱,暗中在那里心痛异常,恨不得想法儿仍旧拿回家去。真是哑子梦见妈──说不出的苦。所以那些呆商虽然在倌人身上略略花钱,却是见了倌人,自以为是花钱的客人,大模大样呼幺喝六的不算外,还要拉拉扯扯动手动脚的做出无数的丑态来,差不多要捞回他的本钱方才算数。倌人们虽是心上恨他,无奈自家做着生意,也只好勉强应酬。这是商界中人的现形了。再说官场客人来,更加可笑。无论什么龟奴皂隶出身,只要有了几千银子,遵例报捐,指省分发。到省之后,连他自己也忘了自家的本来面目,居然是一位候补老爷。有时被他撞着木钟,凑着运气,委了一个差使,就立刻花天酒地、驷马高车的阔起来。你想他们的出身本是卑微,又不是什么世家公子,更兼候补的时候只晓得磕头请安、大人卑职这一套仪注,余外的事情,都是昏天黑地,一事不知。这样的一班人物,那里晓得什么嫖界的情形?到了堂子里头,自然而然闹出许多笑话。他除了不肯花钱,还要对着倌人乱吹牛屄,混摆官派。这样的官场客人,你道可笑不可笑?总而言之,官场中人到了嫖界,真是那天字第一号的瘟生,世界之上有一无二的饭桶。到了堂子里头,也是懵懵懂懂的,那该应挑眼儿的地方,他却一毫不懂;偏是那不该挑眼之处,却会忽然撞着他的高兴,平空的发起标来。就是花了几个钱儿,也花得不伦不类的,全不着些腔板。那场面上的花钱,就如吃酒碰和等类,偏偏不肯花销,反说倌人敲他的竹杠;及至倌人私下放起差来,他却情情愿愿,一千八百、三百五百的双手奉送,去塞那无底的狗洞,全不见一些响声。若有朋友问起他来,他还赖得干干净净,不肯招承,好似那属员馈送上司一般。倌人若做着了这种客人,还有些儿贪取。就只有一件,官商两途的嫖客,大约寿头码子居多。一到了堂子里头,就把那倌人钉住,跟前跟后,一步不离,一双色眼贼忒嘻嘻,毛手毛脚的就如饿鬼一般。在旁人看起来,不晓得里头的缘故,不说那客人曲气,是个寿头,反说倌人烂污,做了恩客,所以倌人做着他们这样的客人,有了这样的贪图,便有那样的惹厌。

如今上海的堂子生意,也渐渐的不好做了。“又道:”他们这班做官的东西,真是饭桶,一个‘嫖’字都学不会,你想他还有什么用头?不是我说句笑话,这些堂子里倌人,若叫他去替他们做起官来,怕不到是个通省有名的能吏。官场如此,时事可知。那班穿靴戴帽的长官,倒不如个敷粉调脂的名妓,你道如今的官场还有什么交代?“说着长叹一声。

春树听了多时,等他说定了,便哈哈的笑道:“算了,算了,不用再往下说了。

你那里是讲论什么嫖界,竟是在这里骂人,不过是借着嫖界的名目,发你的牢骚罢了。“秋谷不觉也笑起来,道:”我是借他人之酒杯,浇自己之块垒,狂奴故态,何足为奇!难道他们这班无意识的畜生还不该骂么?“就高吟道:”少年努力纵谈笑,万事终伤不自保。“言下不觉怅然。春村听了,不由的也提起心事来。大家相对无言,觉得大有天壤茫茫之感。

秋谷坐了一会,。忽想起林黛玉约他前去,便立起身来,告辞出去,便一直到惠福里来。走进弄中,数清了门牌,见双扉紧掩,寂寂无人。秋谷轻轻的扣了两声,里边问:“是啥人?”秋谷道声:“是我。”只听得“呀”的一声,一个小大姐走来把门开了。秋谷问他:“大小姐可在家中?”小大姐回他尚未出去。秋谷便走进来,见这几间房子收拾得甚是精致。忽听得楼窗“呀”的开了一扇,黛玉探出身来。

正是:

十年一觉,扬州杜牧之狂;载酒看花,太白西川之痛。

欲知后事如何,下回分解。





第二十七回 林黛玉春宵引凤 王云生黑夜捉奸





且说秋谷走进天井,见黛玉在楼上探出半身,淡妆素服,丰艳动人,向着秋谷笑道:“楼浪坐嗫。”秋谷点一点头,走上楼去。黛玉一直迎到扶梯边来,携着秋谷的手,进房坐下。秋谷举目看时,只见一并三间房子;中间摆着客堂;上首一间是黛玉的卧房,一律是红木器具,铺设的华丽非常;下首一间挂着绝精致的东洋门帘,想是外国房间了。

坐定之后,黛玉亲手送上茶来,秋谷连忙立起身来,接了茶碗笑道:“阿唷!

对勿住先生,倪是勿敢当格。“黛玉横波一盼,黍谷春回,微微笑道:”耐搭倪客气起来哉。“便仍旧推他坐下,黛玉自己也趁势坐在秋谷身旁。秋谷问他还做生意不做,黛玉道:”倪自家呒拨主意,正要搭耐商量。倪心浪本来打算到仔下节再做生意,不过倪做起生意来,生意随便那哼好法,总归开销勿落,格当中勿知啥格讲究?二少耐替倪想想主意看。“秋谷道:”你的开销本来太大,平日间任情挥霍。

到了节上自然要开销不来。若要就是这样做个住家,眼前虽然尚可支持,久后终非了局。但是,你要现在再做生意,他却还有一件为难。那邱八虽然放你出来,总算是把你已经置于度外的了,万一他再到上海,听见你又落风尘,一时发狠,同你说话起来,虽不怕怎样,也是个累赘的事情。依我看来,你还是权时不必悬牌应局,包一个十三四岁的雏姬,叫他出局,你自己在院中酬应房间,既可节省开支,又一样好招罗生意。你道如何?“黛玉听了,点头称是。

说话之间,听得壁上的挂钟“当当”的敲了七下,早有娘姨进房点起自来火来。

黛玉料着秋谷没有吃饭,便叫相帮去宝丰楼天津馆内叫了几样菜来。秋谷因五月中天气已是燥热,不大吃酒,止饮了一杯便放下杯子。黛玉道:“耐勿吃热酒,倪搭有口力沙勒浪,阿要开一瓶来?”秋谷素来最爱口力沙同勃兰地两种洋酒,听说有口力沙,心中大喜,便叫快快开来。黛玉便自己走过外国房间去,取过一个酒瓶来,叫娘姨开了,替秋谷斟了一杯,黛玉自家侧坐相陪。

二人促坐谈心,浅斟低酌了一会。黛玉问秋谷可去看戏,秋谷点头道:“看戏也好。但是现在不知那一家戏园的戏好些?”黛玉道:“桂仙里花旦倒呒啥,倪看桂仙阿好?”秋谷点一点头。黛玉就催他吃饭,吃完之后,黛玉便去对镜晚妆,再画蛾眉,重施脂粉,换了一件湖色闪光外国纱衫,玄色纱裤子,头上也不带什么珠花,止带着一头风凉押发。只见他媚眼流波,盈盈欲笑,纤腰约素,款款随风,真个是清丽天然,丰姿绝俗。打扮已毕,恰好秋谷也立起身来,一同出去。秋谷自有包车,黛玉坐着轿子。

到了桂仙,案目连忙同到楼上,坐了一间二包,送上戏单来。秋谷看时,只见做花旦戏的小喜凤恰好排的《武十回》,正是他拿手的好戏。那时场上锣鼓喧天的正在那里做着《四杰村》,差不多说话都听不见。秋谷甚是厌烦,便问黛玉跟来的娘姨取过一个千里镜来,拿在手中四围照看,也没有看见什么熟人。好容易盼到做完了《四杰村》,又做了两出配戏,直到第五出上,方是小喜凤的《武十回》。手锣响处,小喜凤袅袅婷婷走将出来,那几步跷工,真如杨柳随风,春云出岫;戏台下的看客,早大家哄然叫起好来。秋谷仔细看时,只见他丰格轻盈,容光飞舞,宛然就像个小家碧玉一般。就是唱那两声,也是清越非常,余音不绝。秋谷甚是叹赏。

做到“挑帘”一段,小喜凤和那扮西门庆的小生目挑眉语,卖弄风骚。那双眼睛就如一对流星,在场上滚去,四面关情。到了吃紧之际,又像那吸铁石和铁针一般,吸铁石刚刚一动,早把铁针吸了过来,并在一处。小喜凤的眼光四面飘来,那小生扮的西门庆,就随着他的眼光满场乱转,那一种轻佻狂荡的情形,真做得体贴入微,形容尽妙,一时那里说得出来?只听得台下边喝彩之声,殷然雷动,秋谷也不觉喝一声彩。

不多一刻,《武十回》已经完了,小喜凤走进后场。秋谷叫黛玉早些回去,便同下楼来。秋谷意欲回栈。黛玉那里肯放?依然同到惠福里来,那时已将近十一点钟。

秋谷坐了一会,因回来的时候身上衣裳单薄,受了夜凉,腹中觉得有些隐隐的作痛,便叫黛玉去暖了一杯勃兰地来,赶赶腹中的凉气。黛玉忙叫娘姨温好了酒,又排上几只盆子来,却就是稀饭小菜,甚是精美。秋谷看时,见是一盆鸡松,一盆熏鱼,还有油鸡、南腿,以及糟蛋、乳腐之类,排了八盆。秋谷随意吃些,黛玉便和他并肩坐下,一手拿了

一只勃兰地的杯子,直送到秋谷口边。秋谷一口气“咕嘟嘟”的就于了一杯,觉得一般热气自喉间直达腹中,把风寒一齐赶尽,登时周身就松快起来,心中大喜。

黛玉便又斟上一杯,秋谷又饮了半杯,觉得已经微微的有些醉意,便停杯不饮。黛玉劝他再喝一杯,秋谷摇头不答,却把那吃剩的一杯残酒递在黛玉手中,微微含笑。

黛玉会意,接了杯子便就喝了一口,抬起头来看着秋谷。四日偷窥,两心互印,灵犀一点,暗暗关情。黛玉连喝了几口酒,已经红上脸来,媚眼横斜,春情荡漾,把一只纤手托着香腮,好像一个身体没有放处一般,坐立不安,和身融化。却又伸过一只手来,把秋谷的手拉住,用力揉搓,杏脸微饧,星眸半闭,那两边颊上透出点点桃花,晕着那淡淡的胭脂,十分精彩。秋谷留意看他,只见他鬓影惺忪,酒情撩乱,樱唇之内时时咽着香津,大有芍药含烟、海棠带露之致。

看官且住,那林黛玉虽是上海的有名人物,却并不是什么倾城倾国的姿容,既没有金小宝那样的纤浓,又没有陆兰芬这般的清丽,不过比起张书玉来较胜一筹,是个中人之质罢了。为什么在下要这般的极力揄扬,岂不要受看官的指摘么?列公请听,那林黛玉虽然相貌平常,却是个天生尤物,丰韵天然,那一步颦一笑的风头,一举一动的身段,真是姑苏第一,上海无双;更兼那一双媚眼,顾盼起来真可销荡子之魂,摄登徒之魄??这便是林黛玉出奇制胜第一等的工夫。看官们有老于嫖界认得黛玉的人,方晓得在下的说话不是无根之论。

闲话休提。只说章秋谷见黛玉这般光景,风月场中的老手那有不知?却装作不曾理会的样子。看黛玉时,看着秋谷的面孔像要说话,刚刚开口却又缩住了,一语不发。有时秋谷抬起头来,他却又低下头去。约有一刻多钟,娘姨早搬了稀饭上来。

秋谷吃了半碗,就不吃了。黛玉也随便吃了些儿,卸妆就寝。一个是刘郎再到,人面依然,一个是倩女还家,檀奴无恙,自然比旧不同。一育无话,不提。

明日秋谷与黛玉商量,借着黛玉的房间,请辛修甫等一班朋友欢聚一天。散席之后,黛玉还想留他,秋谷坚辞,定要回栈。黛玉苦留不住,只得由他。

秋谷回到栈中歇了一夜,早间起来,就见双林房中的娘姨请他过去。秋谷梳洗过了,便走过来,见双林靓妆相待,一见秋谷进来,问他为什么这样忙法,一连两夜没有回来。秋谷一笑不答。双林就取出一封王云生的信来叫秋谷看,说是云生在家里寄的。秋谷抽出信来看时,也没有什么要紧说话,就说他夫人病虽好了,一时不能脱身,恐怕要直到下月中旬,方能到此,一切事情暂托秋谷照应等语。

秋谷看了,明知是假,心中却暗暗好笑,自己想道:明是王云生等了多时,预备下手,所以故意发这一封信来,好叫我放心大胆的全不提防,主意倒也甚是恶毒。

我虽然大胆,这样冒险的事情,也要打算一个对付的法儿方好,心下盘算,面上不露丝毫,对着双林笑道:“他迟到下月方到,却便宜了我们多聚几天。”双林瞑了他一眼,劈手把秋谷手中的信夺了过来,道:“你说得倒狠是要好,只怕你口不应心,一连两夜住在外边,还要在我面前虚情假意,装着幌子。我倒不领你这个情。”

说着,微微的冷笑一声。秋谷仔细打量双林,见他虽是年纪略大些儿,眉目之间饶有媚态,更兼身段轻盈,走起路来直欲随风飞去,心中倒有些替他可惜起来。暗想:“这样一个人材,可惜从了流氓,做这紥火囤的勾当。”

自从这一天起,秋谷至陈文仙院中去了一趟,在栈内住了一夜,却并未到双林那边去。隔了一天,秋谷故意晚间回来,约摸不到十点钟的光景才到房中,娘姨已来相请。秋谷悄对娘姨说道:“此刻还有茶房在外,不便过来,停回等人静了,我来就是。”娘姨答应去了。那班茶房见秋谷与他们鬼鬼祟祟的,不免疑心,早已料着了七八分光景。只是上海地方视为常事,没有什么希奇,那有人来管你们的闲事?

只说秋谷心中想道:今夜他叫人来请,大约事情的发作就在今天。若要谨慎些儿,从此同他一刀两断,凭他们再有通天本事,也是无可如何。只是我正要看那王云生怎样开场,那里肯就此不去?只要我自家小心防备,料想也不怕他,我倒偏要冒险一遭,看他们究竟如何做作。

想定主意,又坐了一会,已敲过十二点钟,秋谷单穿一身纺绸衫裤,悄悄带上了自己房门,走将过去。见双林坐在灯下默默无言,见秋谷走进,立起身来,含笑拉他坐下。秋谷觉得双林今夜的神情甚是巴结,比平时大不相同,暗暗的说声“不好”,虽然胆大,倒底也不免拖着惊慌,只毫不放在面上。略坐一刻,双林先自睡了,秋谷也勉强登床,提心吊胆的听着外边。那时已有两点多钟,却没有一毫响动,略觉放心,或者今夜不来的了。那知心一放下,便觉得睡意朦胧。

正在将睡未睡之际,忽听得房门上“嘡”的一声,把个章秋谷登时惊醒,在床上直跳出来,知道一定是事情发作,连忙下得床来穿好鞋子。原来秋谷本来有心防备,所以不脱衣裳。秋谷下床之后,把两边衣袖往上梢了一梢,侧耳再听时,只听得房门上连连敲了几下,外边高声叫道:“快些起来开门,你们都睡死了么?敲了半天的门,没有人来答应!”秋谷听得十分清楚,正是王云生的口音。双林本来没有睡着,假作惊醒的样儿,听了外边云生敲门的声音,只装着吓得浑身乱抖,在床上起来,拉住秋谷的衣裳不肯放手,身上只穿着一件汗衫,一条洋布睡裤,口中只低说“如何是好”,满眼中流下泪来。秋谷见双林紧紧的拉住了自己衣服,明晓得是要借着惊吓的样子,拉住了他,好叫他脱身不得的意思。外边王云生见叫门不开,便把那房门一连踢几脚。你想那客栈房子那得坚牢?不多两脚,已被他踢得门摇轴动,“吱吱”的响起来。秋谷见风势已急,便想走到门前,预备好脱身出去,怎奈双林抵死的两手吊住,那里肯放?秋谷大怒,不由分说,把右手轻轻一洒,把个双林早洒得头晕眼花,立脚不住,一交筋斗直跌到墙脚边去。

头时迟,那时快,那两扇松木板门早被王云生用力一脚,“轰”的一声倒了一扇。秋谷在灯光之下,见王云生抢进来,门外还有三四个人,都是当差的打扮。王云生走进房内见了秋谷,假做吃惊道:“你为什么在这边房内,怪道我叫门不应,原来你们这班奸夫淫妇,干得好大的乾坤,真是混帐!”一面说着,抢步上前要扭秋谷,回头又叫门外的人道:“你们快些进来,与我把奸夫淫妇一齐捆了起来,明日送官究治。”门外一声答应,都拥进来。

王云生揎拳掳袖的正要动手,不料被秋谷把他拦腰一掌,王云生不及提防,一声“阿呀”,早已滚在一旁。秋谷不待他们动手,两手略略向人丛里一拉,拉得众人让开一线。秋谷一个蹲身,噗的早穿出房门去了。王云生急急的从地下扒起,带着众人追出来,见秋谷立在自己房门首。

此时茶房已经被王云生踢门惊醒,隔壁房门也还有未睡的客人,听见外边大闹起来,大家出来看视。只见王云生装做气得气喘呼呼的样子,指着秋谷骂道:“天下竟有这样的事情,我倒把你当作好人,托你招呼家口,你竟敢丧了良心,奸骗起人家的内眷来,难道世上没有王法的么?”正是:

锦瑟华年之恨,绮阁春深;含沙射影之场,书生胆大。

要知秋谷怎样脱身,但看下回分解。





第二十八回 吹大话满口牛屄 露真情一箱石块





且说章秋谷见云生追赶出来,不慌不忙,指着他微微冷笑道:“你这紥火囤的大胆奴才,你哄骗别人也还罢了,竟敢班门弄斧,在我面前做起这个勾当来!你未曾起意,也该打听姓名,我章秋谷可是这样人物,受你哄骗的么?我劝你快些息了念头,不要多开臭口,免得张扬,还是你的造化。你若再要扬威耀武,在这里混摆你的官腔,那时送到当官,追究羽党,莫怪我反面无情!”

王云生正在乱嚷乱跳之际,忽然听见秋谷这番说话,正如当心一拳打个正着,劈头浇了一桶冷水下来,免不得心中大大的吃了一惊。回过头一想,就算章秋谷看破机关,终久拿不住他的凭据,况且今夜的奸情,又是当场捉破,有双林的活口为凭,不怕他有本事跳上天去。便做出那铁铮铮的面色,暴跳如雷,口中叫道:“真是反了,你奸了我的内眷,还要说我是个紥火困的流氓,这里也和你分辩不清,我也没有工夫同你费气,我只问那贱妇便了。”便一片声叫捆那贱人出来。两旁家人听了,故意都不动手。王云生自家抢进房内,一把头发把双林拖了出来。双林哭哭啼啼,装得真是十分相像。王云生把他拖至门外,问着他道:“你这不要脸的烂污货,我不在此间,你干得好事!你们两人是从几时起手?从实说来!”双林呜呜咽咽的泪流满面,一句话也说不出来。王云生连喝:“快说!”双林看着秋谷的面孔,半晌方说出一句话道:“我当初原是不肯的,被他勾引了多时,一时没了主意,只求老爷耽待这一次,留了我的脸面罢。”王云生不待说完,火星直冒,只听“噗”

的一声,双林粉面上,早着了云生一掌,一面指着秋谷道:“你干了这样事情,倒像没事人儿一般模样,难道你假作痴呆,我就罢了不成?”又向旁边的人说道:“你们众位请看,可有这个道理么?”

秋谷见王云生这般做作,觉得甚是可笑,却故意拿他开心道:“我便算骗了你的家眷,是我一时之错,却已经追悔不来。现在据你的意思,要怎么样呢?或者要我出几个钱,遮遮你们的脸面,也要好好的商量,那有一味恃强的道理?”这几句话,直把个王云生气得拍着胸膊,大骂道:“你们听听,他自己干了犯法的事,反要寻我开心,我也不怕你飞上天去,明日同你到上海县讲便了。”

众人在旁听了,多替秋谷捏着一把冷汗,怪他既是干错了事,不应该一味蛮凶,暗暗的多在那里说他不知风色。秋谷却对着云生正色说道:“你还是当真到上海县去,还是说着大话吓人?若当真要到上海县去,认真究问起来,我倒没有什么虚心,只怕坏了你的钱树还在其次,并且出了名声,从此在上海地方做不得生意,岂不是我绝了你们的衣食么?我劝你不是趁收篷,彼此讲和的好。”

众人听了秋谷这番说话,不觉大家都笑起来。笑他说的话儿好似孩子一般,到了这个时候,还这样定心,随口说这般希松的说话,那里晓得他们两下的机关?只有王云生听这几句话儿,入耳钻心,由不得心上扑扑的跳个不住。但是明知没有被他拿住什么破绽,料想也不怕他,只得扳着面皮,喝道:“我还有这样工夫和你蛮闹,你倚着自己有些拳棒,一味恃强,还要说出这般撒赖的话来,真是岂有此理!

我只叫你好好的等着便了。“

秋谷哈哈的笑道:“我倒留你些儿体面,不肯翻出你的证据,你到这样的猖獗起来,我也晓得你们这班光棍,不叫你们见些手段,你也不肯死心。”说着四边一望,见栈内的帐房先生,身上披着一件短褂也走了进来,便招呼他道:“他们这些光棍想要紥我的火囤,我去取出他们的凭据来,烦你作个证见,不要被他们跑了。”

那帐房先生是个老于上海的人,见王云生半夜回来,并不是轮船到埠的时候,心上已是了然,但是章秋谷被他当场捉破,凭你再有通天的手段,一时也施展不来。这帐房先生向来同秋谷甚是要好,见秋谷这般说话,便走进一步,拉着秋谷,附耳说道:“你若拿不住真凭实据,万万不可出场,还是私下讲和的好。”秋谷也低声答道:“少停我自有证据给你们大家看视,你且不用心慌。”王云生听得分明,心上着急,想不出个落场的法儿,却还没有猜着秋谷已经开过了他的皮箱,急得只把眼看着双林,要想他出来硬证。

恰好秋谷一回身,如飞的抢进云生房中,要想去开他的箱子,双林立在门外,一把拉住了他的手臂,道:“你把我害到这般地步,还说我们紥你的火囤,你的良心何在?”秋谷大怒,觉得火上加油,兜面呸了双林一口,道:“我看你年纪轻轻的人,又生了这般的容貌,那样事儿不好去做,却姘着这班光棍,干这忘廉丧耻的勾当。你自己想想,可有什么出头?我倒替你十分可惜,你还要硬作证见,说出这样的话来,岂不真是可羞可恨!”说得个双林满面羞惭,满心懊悔。暗想:“果然为什么错了主意,要干这样无耻的事情。”登时耳热面红,放了秋谷的手,随他进去。

只见秋谷走进房中,两手提了两只箱子出来。众人不解其故。王云生一见,急得面色如灰,连忙指挥众人要夺秋谷手中的箱子,口中叫道:“你们众位请看,他破了奸情,还要硬抢我们的箱子,请你们众位发个公论何如?”秋谷见众人七手八脚想要夺还箱子,忙把手中箱子摔在地下,两手拦住众人,大声说道:“谁要你的箱子?我只把你箱子内装的东西给他们大家看看。”王云生听了虽然着急,口中却说不出来,只得嚷道:“你要开我的箱子,我也没有犯法的东西。但是我箱子里头都是要紧的物件,若走失了我一件,我们赔偿得起么?”一句话早恼了客栈的帐房先生,上前说道:“王先生,什么说话!大家多在此间看得明明白白,难道开了箱子就有人偷了你的物件么”况且你们两人现在各执一词,你便叫他奸骗,他却叫你们是紥他的火囤,大家都是一面之词,叫我们旁人何从捉摸?不如任他开了衣箱大家看看,他若拿不出你紥火囤的凭据,料想他也抵赖不来,那时任你将他官了私休,我们旁人自然也有个公论。“帐房先生这几句话儿方才出口,大家齐和一声,说这样办法方是平允。王云生到了此际,明知变化不来,急得他顿口无言,面青唇白,口里还想要硬挺几句,怎奈他受了惊吓的人,那一个舌头竟不肯由他做主,结结截截的说了半日,始终挣不出一句话来。

大家看他急得这样情形,早已心中明白,只不好多开口儿。双林早已躲进屋中去了。秋谷便问王云生要那箱子上的钥匙,王云生那里答应得出?秋谷见他不肯,便对着众人说道:“他既拿不出钥匙,只好把他的锁扭开,请你们大家看看,不要回来又说遗失了什么紧要东西。”那旁边看热闹的客人以及栈内的茶房,初时虽然并不开言,却大家暗怪着章秋谷恃蛮无理;现在见王云生神色仓皇,已经露了马脚,又见章秋谷语言清朗,神采飞扬,不觉暗暗的心中称羡,便大家附和起来,七张八嘴的道:“你只顾把锁扭开,里头有什么东西,我们自然都是见证。”秋谷听了甚是欢喜,便把那两只衣箱的锁轻轻一扭,把锁硬扭成两断,打开箱盖。大家近前看时,只见箱面上都是些半新不旧的男女衣服,并没有贵重之物。翻到一半,早把那包好的砖石翻了出来,每箱约有十余包的光景。众人把那纸包放在手里顿了一顿,觉得沉甸甸的,大家倒吃一惊,面面相觑,做声不得。秋谷笑道:“你们不要心慌,且把这包儿打开看看,可是什么东西?”众人便大家去拆那纸包。

王云生见了,真是急得汗流体战,魄荡魂摇,明知是难逃公道,看看手下的同党,早已乘空逃去了两人。还有这两个是向来扮作他的家人,脱身不得。正在着急之时,忽见众人一齐拥到前边去看他的箱子,他便想乘空脱逃,向那两个家人使了一个眼色,轻轻的绕到天井中间,一溜烟正待逃走。众人并不提防,秋谷却时刻留心防他弄鬼,忽地一回头不见了王云生,慌忙向外看时,见王云生的背影一闪,已到腰门。秋谷大怒,疾忙跳到窗外,就如燕子穿帘一般,只一步已扑到王云生背后,连肩夹背一把拖来,依旧把王云生扭了回去。大笑道:“你原来也只这点本领,一般害怕起来,刚才你的威风那里去了?”羞得王云生把头拜倒,不敢作声。

说话之间,众人已将纸包拆开几个,仔细看时,那里有什么宝玉明珠?尽是那砖头石块。一齐大噪道:“怪不得他形迹可疑,原来果然是个骗子!”秋谷对着众人说道:“我的说话何如?若没有拿住他的实据真赃,也不敢说这般满话。如今既是破露出来,想他在上海地方,不知害了多少青年子弟。既然撞在我的手内,我却就要替那以前受害的报仇。明日我托人写信到新衙门去,把他们一同解案,重重的办他,也好警戒他的下次。但是要屈你们做一个公正的证人方好。”章秋谷的意思,原不过呼吓他们,并不要一定送官究治,因为自己同双林既有交情,免不得先落一层不是,也占不着什么便宜,就是赢了官司,于自己又无益处,倒同这班小人结了个不解之仇。

只说众人听得秋谷要把他们送到当官,并且要旁人见证,不约而同一齐劝解。

双林躲在房中,听见秋谷要将他们一起送官,更吓得涕泪俱下,只得老着面皮走出房去,望着秋谷扑地跪将下去,也不开口,只把抽于这着脸儿,泪流不止,几乎哭出声来。王云生正在为难,见双林出来跪下,便由不得也赶过来一同屈膝。正是:斋

盲风怪雨浑闲事,舞袖弓腰妒莫愁。

要知秋谷如何发放他们,下回分解。





第二十九回 写伏辩光棍无颜 听良言名花有主





且说王云生哀求秋谷道:“我们虽然丧了良心,章老爷却并没有落了我们的圈套,只求章老爷看破些儿,高抬贵手,免了送官究治,我们就感激万分了。不瞒章老爷说,我们凑了许多本钱,原想做着这注生意,现在弄得人财两空,还丢了这般脸面,我们当光棍的人落到这个下场,总算可怜的了,只求章老爷开个思典罢。”

说着就叩了几个响头。双林更是羞容可掬,掩面欷歔. 秋谷见了,心早软了一半,又听着云生的话虽然可笑,却也是句句真情,便一手先把双林扶起,又叫王云生起来。双林低头立在一旁,深锁蛾眉,半含珠泪,秋谷更觉得心中不忍起来,便向众人说道:“我本待把他送到当官,但既是你们众位同声相劝,我也不好意思扫了众位的面光。现在他们既然自家认错,我看着大家分上放过了他,免了他一场出丑。

但还有一件,今夜的事情是你们当场共见,不要我转背之后,他倒同我说话起来,那时事过无凭,我也。奈何他不得。这须要叫他写张伏辩方好。“众人听了,都赞秋谷的见识不差。

原来王云生虽做流氓,却上海不曾犯案,所以极怕见官。当下听见要叫他写张伏辩,虽是心中不愿,料想推托不来,只说:“这张伏辩,不知章老爷要叫我怎生写法。”秋谷道:“这个容易,我起个稿子,你誊就是了。”随叫家人取出笔墨,秋谷随意起了一个稿子,递给众人看了,便叫王云生用端楷誊好。王云生勉强写好了一张,秋谷取过,同众人看时,只见那伏辩上写道:

立伏辩王云生,今因冒充官长,图诈未成,求免送官究治。此后如再有讹诈等情,听从惩治,立此伏辩是实。

后面写着年月,并王云生亲笔的几个字儿。秋谷看罢,见他写得不差,又叫他在名字底下画了一个花押,收在身边。却向众人举手,谢道:“今天多有费神,改日再谢。”众人多称“好说”,见事已停当,渐渐的散去。

一番扰攘,不觉天已大明,秋谷正要进房略睡片刻,见栈内帐房走了进来,手中拿了一篇单帐交给云生道:“你闹了这样事情,我们这里是不能再住的了,你快把栈帐算清,立刻就搬出去。并不是我们赶你动身,你可知这里是租界地方,捕房的规矩十分严紧,设或被包探查了出来,这容留匪类的名儿,我们却担当不起。”

可怜王云生好容易花了无数本钱,结交了章秋谷,想要在他身上捞回一大注钱,不料章秋谷看破机关,弄得个人财两失。此时手中正是空空洞洞的时候,那里拿得出钱来,看一看那张单子,倒开着六十余元,心上万分着急,只得老着脸皮央求帐房道:“我此时手中实在无钱,请你们暂时宕欠,待我出去设法归还,两三内日决不误事便了。”那帐房见他没有钱,就变了面孔道:“这个不能!你说得倒狠是容易,我刚刚同你说过,你今天还想住在我们栈内么?我实对你说罢,我们的房饭帐是不能少的,你休想短了一毫。你若真没有钱,我只把你们的行李衣箱一齐留下,算个押头,你去取了钱来赎回行李,就是这两句说话,没有别的商量。并且结好了帐,还要快些请你出门,免得叫我们受累。”云生听了,无可如何,只得走进房去与双林说知。

原来王云生的衣箱虽是空的,却还有几件单夹罗纱的时新衣服,连着双林的衣饰,并那床上的熟罗帐子以及烟盘烟枪各物,也还值得一二百块钱。云生和双林商量,要暂时当了他的首饰去付栈内的房饭钱。双林自从秋谷拉他起来,晓得秋谷还有些可怜他的意思,只懊悔自家打不定主意,上了他们的当,被他们包了出来,做着这无耻的勾当。眼看着章秋谷这样一个风流人物,反要去哄骗他,现在弄得破了机关当场出丑,从此回到苏州去有什么面目见人?愈悔愈惭,愈惭愈恨,不觉咬牙切齿的恨起王云生来。正在那里暗泣,忽见王云生进来,要将他的首饰去抵当栈帐,心中忍不住怒恨交并,便恨恨的道:“我上了你这般大当,弄到出丑当场,这还是我自家不好,不该听信你的言语跟你出来。亏你还说得出这般说话,问我要起首饰来。我的首饰是我自己带来,又不是你出钱置备,怎么要拿我的东西去抵你的栈帐!”

说着,越想越是愧悔,止不住两行珠泪直流下来,那说话的声音早已岔了。

秋谷在外,听得甚是明白,心中不忍,便把双林叫了出来,问道:“你还是打算跟他回到苏州,还是怎样?”双林拭泪应道:“我一时听了他们的哄骗跟了出来,现在弄得这般结局,叫我回去怎样的见人?”不觉呜咽起来。秋谷慨然道:“你既是不肯同他回去,不妨你在此间耽搁数天,等他们先自回去。至于你们的栈帐既然拿不出来,我同你总算认得一场,这几个钱我来出了就是。”双林听了,感激秋谷,真是重生父母一般。王云生也十分欢喜,谢了秋谷,自去收拾行李,立刻搬出栈去。

这里秋谷向帐房说明,把他们所欠的房饭钱,一并归在秋谷帐上。双林归并了自己的物件,仍旧住在原旧房内。

秋谷打发了他们,觉得畅满非常,便歪上床去,一觉直睡到日中时候方才起来。

对面双林听得秋谷起身的声息,连忙走了过来,含羞带愧,双泪盈盈,对着秋谷又要行下礼去。秋谷看他态度惺讼,神情寂寞,低眉承睫,煞是可怜,老大的心中怜惜,急把他一把拉住道:“你好多礼呀,这件事情都是他们不好,与你有什么相干?

你不过受他们的指使罢了。我方才放松他些,一半为的是你,只要从今改过,就是好人,倒不必把这件事儿放在心上。“双林听了,又谢了秋谷,含情凝照的说道:”我懊悔自家没有主意,冒冒失失的跟了这班光棍出来,非但受这一场羞辱,并且被他们拖累了名声,将来不知怎样的收场,真算得十分命苦的了。“说着,眼圈儿早又红了,不觉哽咽起来。

秋谷见双林的情景实是真心懊悔,并不是那随口之言,便趁势劝他道:“你虽然从前错了念头,犹幸你现在回头甚早。只要你真心傀悔,自然不至于流落终身。

但我替你想来,你有了这样的姿容,何苦要做着这般生意,何不留心物色,拣一个合意的客人嫁了他去。就是年纪比你略略大些,或者家中并不十分富足,只要大家中意,不妨成就姻缘。切不可倚着自家的容貌不肯嫁人,一年一年的耽搁下去,白白的辜负了自己的春青,岂不可惜!从来树高千丈,叶落归根。凭你有薛涛、苏小的清才,樊素、小蛮的丰调,若要仅着在枇杷花下做这卖笑的生涯,只怕不到几年,终久免不了车马稀疏,门前冷落。趁着自己妙龄之际不肯从良,到了那年华老大之时方才回过念头,急急的想要嫁人,那时更有谁来要你?再说起你们这般勾当,更不如堂子里做生意的倌人。赔了自家的身体冒险担惊,就使敲到了别人的竹杠,却是花了无数本钱,装出许多圈套,传扬开去,还不免坏了名头,在我替你想来已经不值。再要遇着那一班精明的人物,看破阴谋,将你们一起送官究治,那时问起供来,免不得受些刑罚。我看你这样的娇柔身体,那里受得起堂上的官刑?比如昨日的事情,若是换了别人,恐怕不见得把你轻轻放过,到了那懊悔嫌迟的时候,他们一班光棍可替得你么?“

好个章秋谷,果然舌吐莲花,词霏金玉,随处苦心劝说,指点迷途。双林先前尚呆呆的听着,听到一半,已经止不住泪滚珍珠。及至秋谷说到后来,竟是不顾别人,滚在秋谷怀中低声掩泣,虽然不敢出声,却已涕泪汛澜,罗衣尽湿,连章秋谷也不知不觉的替他凄惋起来,倒着实温慰了他几句。当夜秋谷又细细的劝他一番,更把现在那一班嫁人复出的倌人,出来之后倚着有些金珠积蓄,贴戏子,姘马夫,闹得一塌糊涂,拖了许多亏空不算外,还带了一身的毒疮这些情事,和他详细演说,要想把他提醒痴迷。又道:“还有一个最近的倌人,因为不肯从良,弄得穷饿而死。

二十年前的朱桂宝,大名鼎鼎,是个上海花榜的状元。当初时候真是缠头千万,车马如云,大家争着要娶他回去,他却恋着堂子里的风光,不肯答应。不多几年,年纪大了,渐渐的无人过问起来,穷到无可如何,只得在四马路巷堂一弄,捻着一只竹篮卖些瓜子花生度日,岂不可怜!“把个李双林说得毛骨悚然,通身是汗,感激秋谷的心念直透心脾。

秋谷把他留了几天,给他一百块钱,叫家人送他回苏州去。双林千恩万谢,临走的时候依依不舍,望着秋谷,只顾把罗巾拭泪,点点滴滴的把一件纱衫上湿了好些,一步九回头的走了出来。秋谷也只得硬着心肠,任他去了。后来双林回到苏州,果然听了秋谷的话,留心择配,嫁了一个阊门内开绸缎庄的老板,居然生了一个儿子,齐眉到老。此是后话不提。

只说秋谷在栈中方要出去,忽见茶房传了一张请客票进来,却是辛修甫请在西安坊龙蟾珠家,上写着“竹酒两叙,务请早光”的字样。秋谷看了,叫茶房回他就来。

秋谷随穿好了衣服,先到林黛玉处。黛玉要留他晚饭,秋谷不肯,说在西安坊有应酬,黛玉便不好留他。秋谷略坐一刻,直到西安坊来。进了房间,只见主人之外,王小屏、葛怀民已经在座,还有一个四十上下的客人,并不认识。见章秋谷进来,便起身一揖道:“章秋翁,久仰久仰。”秋谷连忙还礼。问起姓名时,方知这人姓陈号海秋,是个广东南海县的拔贡,现在都中当一个七品小京官,是辛修甫的好友。新在京城出来,听得辛修甫极赞秋谷是个当今名士,肝胆照人,所以甚是仰慕。当下两人周旋了一会,陈海秋看着章秋谷,绮年玉貌,大雅不群;章秋谷看着陈海秋,气宇深沉,老成持重,彼此甚相爱敬。坐谈未久,已见娘姨进来排开桌子。

派好筹码,议定章秋谷、陈海秋、王小屏与主人辛修甫四人一局,五十块底二四。

秋谷道:“我们彼此朋友,不见得想要赢钱。五十块底二四不太大么?”修甫道:“我原没有什么一定,今天是陈海翁的意思,要略略碰得大些。”秋谷听是陈海秋要碰大些,就不开口。扳了位,轮该秋谷起庄,碰了两圈,台上甚是平稳,没有大牌。

秋谷正在起牌之际,蓦地抬起头来往对面一看,只见辛修甫背后坐着蟾珠,正在那里同一个二十岁上下的女子咬着耳朵说话。秋谷留心看去,见这个人的神气打扮不像娘姨,不像大姐,随身衣服懒散梳妆,却生得体态娇娆,风姿艳丽,一眼瞅着秋谷,正与蟾珠说话。秋谷见了他的面貌吃了一惊,寻思他这付神气好似二年前在天津东阎乐的陆畹香,越看越像,不觉看得出了神去,把手内的牌乱发起来。恰好秋谷自己的庄,修甫坐在对面,已经碰出三张西风,手中做的是万子一色,三张二万,三张白板,一对中风,一对九万,已经等张。秋谷自己手中本有一对中风,一张白板,恰好碰了三张一索,打算要发去白板便好等张,说也可笑,秋谷往对面看得认真,正在心中摹拟那陆畹香的丰度,不觉忘其所以,有些模模糊糊起来,本来要抽出白板,一个不留心误抽了一张中风出去,辛修甫“扑”的把牌摊了出来。

秋谷见他和了这样一副大牌,又有三张中风,诧异起来,连忙把自己的牌摊出一看,见白板依然不动,中风却少了一张,方才晓得误发了一张中风,致被辛修甫和了一副倒勒,忍不住哈哈大笑道:“我真是有些昏了,你们来看,喏,一对中风竟会打了一张出去,被他和了这样一副大牌,你说可笑不可笑!”正是:

旧日之桃花无恙,小杜魂销;重来之人面依然,徐娘未老。

欲知后事,请看下回。





第三十回 章秋谷乱叉麻雀 陆畹香暗印灵犀





且说章秋谷发错了一张中风,哈哈大笑。对面那人先前见秋谷看得诧异,已觉得有些好笑,及至见他翻出牌来,自家本有一对中风,不知怎的会误打了一张出去,忍不住“噗嗤”一声笑得扭过脸去,弯着腰,扶了修甫的椅背立不起来。秋谷见如此情形,更加狂笑。好容易大家收住笑声,方才算帐,秋谷自己的庄,要输一底多些码子,秋谷照数付讫。

修甫方问他道:“你倒底为着何事这样的失神落智,连碰和都会错误起来?”

秋谷指着对面道:“我看见了他甚是面熟,好像我从前在天津做过的陆畹香。”龙蟾珠不等秋谷说完,急叉口道:“俚耐就是陆畹香呀,到仔上海勿多两日勒。”那陆畹香连忙走过来,仔细把秋谷认了一认,方才认得,忙笑着道:“阿呀!真格是二少,倪隔仔两年,实头勿认得哉。”

原来这陆畹香前两年在上海生意不好,所以到天津去看看情形。谁知刚到天津,便是哄然一声,名声大震,各处的堂子老板,大家拿着重金去罗致他。陆畹香就搭了东阎乐的班子,年纪又轻,品貌又好,更兼唱的梆子、京腔、昆曲、小调,无一不好,又弹得一手的好琵琶,应酬更不必说。天天的冠盖如云,甚是热闹,比在上海的光景大不相同。陆畹香高兴非常。

那时,正值章秋谷进京路过,天津的同乡便同他去打茶围。秋谷一见畹香,甚是赏识,畹香也见章秋谷相貌堂堂,倾心结纳,正彼此有些意思。秋谷因家中有事打电报来催他回去,匆匆归棹,不免怅然。

后来,拳匪闹事,联军破了天津,陆畹香逃到德州住了两月,因德州做不出生意,便折回天津,由天津进京,想要做些生意。那知兵乱之后景象萧条,那里支持得住?那时李文忠公已经同外国讲和,把天津地方退还中国,那侯家后的窑子,依旧的笙歌彻夜,灯火连云。这陆畹香只得重到天津,搭在宝华班内。那知他花运已退,生意大不如前,竟一节不如一节起来。没奈何离了天津,回到上海,要想做个住家,摆只碰和台子。他与龙蟾珠是旧时姊妹,所以到了上海,住在蟾珠院中,暂时帮他应酬照应。不想无意之中遇着了章秋谷,两年不见的旧交,重新相遇,自然欢喜,连忙极力的应酬。

秋谷一面碰和,一面絮絮的问他别后的光景,畹香一一的告诉他,二人就谈个不住。那知秋谷一面同畹香说话,分了神思,早不觉又打错了几张牌。畹香在旁看得明白,恐怕他要输钱,叫秋谷不要和他说话,一心一意的碰和。秋谷那里肯听?

还是口中杂七杂八的寻着说话问他,一个不留心,发了一张东风出去,又被下家王小屏和了一副一百二十和的筒子一色。恰恰的小屏又是庄家,秋谷差不多又要输他半底码子,急得陆畹香和他嚷道:“叫耐勿要说话,耐偏生勿旨,瞎碰一出,输得一塌糊涂,倪来替耐碰仔两副罢。”修甫也说秋谷心神乱了,不妨等畹香替你代碰两圈。秋谷不肯,笑道:“你们就把我看得这般无用,输了两副就要请起替身来?

通共碰了不到四圈,就见得出什么输赢么?“大家听了,不好再说,于是重复掳牌。

秋谷果然不替畹香说话,用心用意的碰起来。畹香坐在秋谷背后静静的看他。

这一副却是秋谷和了一副,止有三十二和。接着陈海秋的庄,秋谷又和了一副五十六和的万子浑一色。

轮到秋谷做庄,起出牌来。畹香看秋谷的牌时,只见一对东风,一对西风,一张南风,一张北风,还有三张万子,三张索子,两张筒子。秋谷把头摇了一摇,皱着眉头略略想了一想,不打南风,反打了一张索子出去。畹香见了,连忙把秋谷一拉道:“耐打错仔一只牌哉。”秋谷不语,只叫他不要多言。接着王小屏打了一张东风,秋谷连忙一碰,便又发了一张筒子,下家不要。辛修甫便发了一张南风,接着王小屏又摸出一张北风,随手打出。秋谷见南风北风已经见过,打算他打北风,便先打了北风出去,再去摸牌。不料刚刚凑巧,摸起的牌恰恰是张北风,秋谷连忙把前发的北风缩了进来,打去一张筒子。辛修甫发出一张西风,秋谷又是一碰,再发一张索子。陈海秋见了,忙招呼小屏同修甫道:“庄家东风西风一齐碰出,刚才又缩进一张北风,一定是手中做着四喜,我们须要小心。”秋谷微笑不语。

过了一转,秋谷又摸起一只南风,发出了一只索子,已经等张,南北风对碰和倒。恰好王小屏摸起一张南风,放在手中,正要发时,被陈海秋拦住道:“南北风万发不得,庄家一定是等这两张。”小屏听了,只得扣住南风,拆了一张搭索子。

轮到陈海秋摸牌时,刚正摸着一张北风,放在手中,向王小屏一扬道:“我又摸得一只北风,大约庄家的牌被我们扣住的了。”秋谷看台上时,南北风已经有了两张,自家现有两对,他们两人每人扣了一张,死也不肯发出,这牌断断和不出来。看那牌时,已差不多将要到底,止有二十余张,秋谷猛然想出一个主意,要出奇制胜的冒险一回,正摸了一张九索,这九索是台上极熟的牌张。秋谷故意把九索翻了转来,明叫众人看见,却拆了北风对子,打出一张北风。畹香见了,急得连声咳嗽,拉着秋谷的衣裳,想叫他缩回重打。秋谷只作不知,凭你怎样,他只如无其事的样儿。

气得个陆畹香走了开去,对龙蟾珠道:“我看二少今朝格碰和,实头有点昏哉,从来韵看见歇格号打法。”

秋谷听见陆畹香的话只微微而笑。王小屏见秋谷打了一张北风,料想不是四喜,又明明看见他上了一张九索,便放心大胆的不怕他,把先前扣住的一张南风发了出去。秋谷急忙一碰,却故意装作懊悔道:“早晓得还有南风出来,刚刚不该把北风发掉。”王小屏道:“你通是说的痴话,你不把北风发掉,我肯放南风给你么?”

秋谷又故作踌躇了一会,方才发了一张九索。

大家那里留心?只有陆畹香听秋谷碰了南风,发去九索,方觉恍然大悟,他用的是那欲擒故纵的法儿,暗暗甚是佩服秋谷的心机圆活。陈海秋坐在秋谷的上家,见秋谷才打北风,料他不要,便也打了一张北风,道:“你刚刚不要北风,我且顶你一只北风何如?”扑的把牌打出。秋谷大笑一声,将牌摊出道:“你现顶北风,我就现领你的盛情。”三家见秋这副牌和得诧异,一个个目定口呆,只把一个陆畹香喜得心花怒开,满心奇痒,张开了一张樱桃小口,笑得“吱吱格格”的再合不拢来。大家看了秋谷的牌,方才明白他拆掉北风对子,是要骗出王小屏的南风,却又明知陈海秋手中还扣着一张北风,所以翻转身来,重吊北风和倒。算一算,四喜要加三倍,不消说已经倒勒。秋谷这一副牌,就赢了三底半筹码,除了前输一底半之外,恰好还赢着两底。大家便重新洗起牌来。

正碰之际,忽见贡春树同着吕仰正一前一后,匆匆的走进来。大家招呼过了,修甫问他为什么到此刻才来。春树道:“我在路上遇见仰正,同去打了两处茶围,所以迟了。”秋谷便告诉他刚才和了一副四喜的缘故,春树也说秋谷这副牌和得十分巧妙,便也坐下看牌。

直到八圈碰完,已有十点钟的光景,各人都觉得腹中有些雷响起来,修甫便一叠声叫:“快摆台面。”娘姨们早摆上四碟点心。秋谷等随意点饥,相将坐下,算起和帐来,秋谷恰恰的赢了一百五十块钱,海秋、小屏各输一半,修甫没有输赢。

当下王小屏同陈海秋取出一叠钞票,点了数目,双手交与秋谷。秋谷不肯就接道:“这几个钱儿什么要紧,难道还一定要现钱交易么?”仍旧要送还他们,叫他们不妨以后碰和再算。二人那里肯依,道:“我们玩耍原为大家消遣,并不是一定要斗输赢,况且通共这点儿洋钱,你若一定不收,倒不是豪士的举动了。”秋谷只得收下。

这一席酒,辛修甫做了主人,殷勤相劝,无不尽欢。龙蟾珠的应酬本来不错,又添了一个陆畹香帮着招呼,客人更是高兴。陆畹香应酬了一会台面,便来坐在秋谷背后,咬着耳朵,遮着面庞,密密切切的不知说些什么,直至陈文仙出局到来,方走了开去,又朝着秋谷横波微笑道:“耐绰仔倪格烂污,是倪勿成功格嗫。”秋谷只点点头,并不开口。贡春树见了,一把搀着畹香的手,要问他什么事情,却被陆畹香把手洒脱,跑了开去。春树一个没趣,面上竟红起来,却被秋谷看见,狂笑道:“你今天剪边,明天剪边,今夜遇着了他,可碰在顶子上了。”众人听了,不觉都笑起来。春树发急道:“你见我剪过谁的边?这般胡说,定要罚你一杯。”就取过一只大杯,斟了满满的一杯送到秋谷面前。秋谷也不推辞,却自家不饮,回过头来见陆畹香远远的立着,正在着衣镜内端详自己的形容,又侧过头去整理鬓发,便向他招招手儿,叫他走来。陆畹香见秋谷向他招于,微微含笑,却扭过身去,像个不肯来的样儿。秋谷见他不动,又连连招手。畹香方才忍着笑,趑趑趄趄、欲前不前的走了两步,又回身坐在榻上。背着脸笑个不住。秋谷见他娇痴可掬,又连叫了两声,畹香才立起来,慢慢的轻移莲步,慢款纤腰,袅袅婷婷,一步一步的走到秋谷身畔,好似蜻蜓点水,荷叶随风,轻回掌上之身,低蹴鞋尖之凤,更不数汉家飞燕,洛浦凌波,把合席的人都看得呆了,不由齐声喝起彩来。陆畹香听得众人喝彩,略略有些羞愧的意思,两颊微醉,秋波凝睇,一手弄着衣角,一手摸着云鬟,倚在秋谷椅背之上,问道:“哈格事体叫倪?”秋谷一手携着他一纤腕,一手端着那杯罚酒,道:“这一杯酒是你的作成,你代了我罢!”说着,把酒杯直送到他口边,陆畹香待要吃时,见众人的眼光多注在他一人身上,看得畹香面上越红起来,桃腮薄晕,杏脸含瞋,似怒非怒的瞅了秋谷一眼,道:“勿要实梗嗫,等倪自家慢慢里吃末哉。”秋谷见他被众人看得急了,恐怕他当真起来,便放了他的手。畹香接过酒杯一饮而尽,洋洋的走到那边去了。

秋谷自同主人说话,又和众人搳了一通关,秋谷输了十余杯,陈文仙代了三杯,跟局娘姨代了三杯,秋谷自家连吃了七八杯,觉得头上蒸蒸汗出。陈文仙取出丝巾,替他拭汗。

秋谷有了些酒意,兴会勃然,自家提起精神,笑语劝酬。风生四座。陆畹香在傍偷看见章秋谷丰姿灌灌,骨格珊珊,目比春星,神同秋水;李泌九仙之骨,何郎十日之香;坐在席上,就如玉山在座,清朗照人。再看别人时,虽然也都气度翩翩,却那里比得章秋谷?只有贡春树丰仪出众,同秋谷比起来似乎在伯仲之间。但是贡春树神情妩媚,就像个大家闺秀一般,靦靦觍觍的全没有一点昂藏体态。两下比较起来,毕竟还是章秋谷棱棱风骨,英气逼人。陆畹香暗暗称羡,觑首秋谷不觉看得出神。

秋谷一面虽在那里敷衍着修甫等一班主客,却只是望着陆畹香,把眼光不住的飘来。可煞作怪,章秋谷的眼光飘到畹香头上,畹香便不知不觉的连忙去对着穿衣镜整理云鬓;章秋谷的眼光飘到畹香脚下,畹香便不因不由的连忙把三寸春纤搁在膝上,重加约束;徘徊弄影,跌宕生姿。那陆畹香的一笑一颦,竞和那章秋谷的一顾一盼互相关合,差不多就和无线电机一般,不期而然的两边相应。这一种灵犀暗逗的深情,就是吴道子的画工也万万描摹不出,叫作书的在下那里演说得来?列公中有温柔乡里的惯家,脂粉场中的老手,一定也晓得这种情形,须不是在下欺人之论。

闲话休提,只说章秋谷与陆畹香眉来眼去,正在得意。众人都没有留意,只有贡春树最是留心,看得甚是亲切,看了一会,猛然对众人笑道:“我一向不知,秋谷吊膀子的本事,竟是绝顶工夫。你们来看他们吊膀的样儿,真是再要好看没有。”

众人听了一齐好笑,陆畹香被春树说得不好意思,面上一红道:“啥格叫吊膀子,倪是勿懂格。唔笃末总是实梗瞎三话四,说出闲话来阿有啥格淘成?”正是:

西川公子,犹开东阁之樽;北地胭脂,重入南朝之选。

直教:

鞋凤暗钩南浦月,指尖亲掠楚山云。

要知后事如何,且待下回分解。





第三十一回 西安坊名士讲嫖经 高升栈优伶夸大口





且说贡春树说得陆畹香面上一红,一扭身跑进后房去了。春树又道:“秋谷吊膀子的手段真个不差,就是他在堂子里头做的倌人,也是做一个要好一个,不晓得他到底是什么本事?看他也不过是随随便便的样儿,却把那些倌人一个个哄得死心塌地。我们同着他到堂子里来玩耍,真是吃亏。”秋谷笑道:“你只顾寻我的开心,你不自己去照照镜子,那付顾影自怜的态度,真个是我见犹怜,好像个有名的花旦,全没有一毫男子的神情。怪不得张书玉为了你,要同金小宝吃起醋来。”春树被他说着毛病,早不觉脸上生红,有些惭愧,却又回答不出,只瞟了秋谷一眼,并不开言。

修甫便问秋谷究操何术,那些有名的红倌人个个倾心,人人要好。秋谷道:“天下的事情总不外‘晴理’二字,我在堂子里头玩耍,也不过是依着情理而行,并不是有什么秘密的口诀。你们总说堂子里头的妓女待人没有真心,这一句话固然不错。然而仔细想来,倌人们做着这门生意,万不能纯用真心,不得不用些假情假意。譬如你做了一个倌人,面子上十分要好,但是堂子卫头人来客往,并不是单单做你一人,或者他昨夜留了别的客人住夜,今天却又留你住在院中,他可肯对你讲着真话,说他昨天接客的么?假使他果然纯用真心,竟对你说了真话,你可肯坦坦平平、不着一毫醋意么?总而言之,倌人见了客人,总有几分顾忌,到了那转弯不来的地处,左右为难,只好说些假话瞒过客人。原为恐怕客人动气,所以要两下遮瞒,卫顾客人的面子,这是他们倌人体贴客人的好心,凡事之中留着客人的地步。

无奈那些瘟生、曲辫子的客人,不懂情形,不知规矩,动不动要发标吃醋,闹得一塌糊涂,岂不埋没了倌人的一片苦心、一腔好意?倌人遇着了这样不知甘苦的客人,那里还肯真心相待?自然就要坏着良心敲起他的竹杠来。你们试想,他们做了倌人,挂着牌子,无论什么家人皂隶都可以走进院中,不能把他们赶了出去。在倌人也是无可如何,怪他不得,何苦要争风吃醋,弄得那倌人进退两难,又有什么趣味?假使那倌人见客人这样歪缠,他也用些蛮派出来,不顾客人的面子,无论什么话儿竟是直言拜上,毫不遮瞒,那时你又将他怎样?难道他挂了牌子,你好不许他接第二个客人么?“

修甫等秋谷说完,击节叹赏道:“你的说话,真是花柳场中千古不磨之论,比到那场面上的劝人说话更觉深进一层。但是你说了半天,还没有提到正文,究竟你用的是什么法儿呢?”秋谷道:“要他们真心要好,却也不难,大约不外三层做法:第一不发标,第二不吃醋,第三不认真。久而久之,那些倌人就自然而然的同你要好起来。再用些体贴的工夫、温存的伎俩,神而明之,存乎其人,不怕他不一个个死心塌地。你想这班倌人,平日之间只有巴结别人,何曾受过别人的熨贴;忽然的客人倒反迁就起来,那有不喜欢的道理?所以我在堂子里头并不认真,把倌人当作孩子一般随口哄骗,把他们哄得喜欢,图个一时的快乐,再不去吃醋发标,自寻懊恼。这便是我章秋谷一生得力的地方。不知你们诸位的意思如何?”

秋谷说到此处,王小屏猛然笑道:“如此说来,你不是同那一班马夫、戏子一样的主意么?”秋谷也笑道:“在外边看去,原也和他们差得不多;其实内里的情形,却是迥然各别。他们那一班马夫、戏子和倌人轧了姘头,非但不肯花钱,并且还专要倌人倒贴,自然就只好颠倒过来,倒反去奉承妓女了。我在堂子里头虽然不闹什么脾气,却也是一样花钱的客人,不过到了他们为难的时候体贴他些便了。到了后来,你越见体贴他的艰难,他越是感激你的情意,所以我做的倌人,起初的时候,两边要好原是假的;及至做到一年半载,渐渐的倒真心要好起来。可不是乐得这样的么?”三席话,说得席上的主客个个点头。

??席散之后,秋谷将要告辞,陆畹香从后房走出,和秋谷两人靠在烟榻之上,一面烧烟,不知悄悄的又说了些什么。秋谷临走,在怀中取了一卷钞票交与畹香。畹香笑迷迷的接了过来,秋谷就去了。

你道陆畹香和章秋谷说了半晌,是什么事情?原来陆畹香到了上海,想暂时不做生意,先摆起一只碰和台子来,但是两手空空,就是碰和台子用不着什么垫场,却也不是空手做得的事。现在畹香遇见了章秋谷,是二年前在天津要好的客人,便悄悄的告诉他一番苦境,并要问秋谷借二百块钱,说得情词恳切。章秋谷本来是个慷慨丈夫,昂藏男子,况且前在天津又甚是同他要好,那有不肯的道理?便慨然应允。畹香大喜,又向他说:“只要一有了钱,诸事好办。明天我去看看房子,大约三五天内可以舒齐,那时搬进新居,再来请你过去。”秋谷就把刚才碰和赢的钞票,自家又添了五十元,一并交给畹香。

果然隔了一天,畹香出去看了几处房子,看中了聚宝坊的一家房子,两楼两底,房租甚是便宜,便又置备了些木器,用了一个娘姨、两个大姐。不到一礼拜工夫,畹香已经搬了进去。章秋谷十分高兴,约了一班朋友替他碰了两场和。畹香因感激章秋谷备了二百块钱,当晚就留他住下。这一夜誓海盟山,两情缱绻。

到了明日,秋谷去后,畹香直至午后起来,想到自己的身世飘零之恨,不觉呆了一回。又想章秋谷为人慷慨,性格温存,我见了无数客人,竟没有这般人物,心上盘算了一会,竟一心一意的想要嫁起章秋谷来,但一时不使出口,想随后再看秋谷的情形。

到了晚间,章秋谷因听人传说张园的烟火甚好,便坐了马车到聚宝坊,要约畹香同去。畹香欣然,换了衣服一同登车。马车在泥城桥一带行来,晚风拂面,露气当空,甚是凉爽。到了张园,便同着畹香在草地上徘徊一刻,回身拣了一张桌子,离着那烟火架子远些,免得火星飞落。

坐得不多一刻,烟火将要开场,秋谷忽见一个滑头滑脑的人,穿着一身极华丽的衣服,带一副金丝眼镜,头上边的刘海发竟有二寸多长,口中衔着一支吕宋烟,襟上插一个茉莉花球,香风触鼻,摇摇摆摆的晃了过来。走到桌子面前把秋谷仔细看了一会,忽然回头除下眼镜,叫了一声“章老爷”。秋谷听了大诧起来,立起身将那人认了一会,方才隐隐约约的想起来道:“你可是苏州丹桂戏园的赛飞珠么?”

原来果然是他。这赛飞珠是苏州丹桂的著名旦脚,秋谷极是赏识他,曾在上海替他登报揄扬。后来秋谷到苏,赛飞珠亲到秋谷寓所称谢,所以彼此认得。

当下赛飞珠答道:“果然章老爷的眼力不差。”秋谷便问他来此何干,赛飞珠道:“丹桂园主因生意清淡,恐怕开不下去,托我来到上海替他请人,住在高升栈内,隔几天就要回去。”说话之间,赛飞珠就飞了陆畹香一眼。畹香微笑,也还飞一个眼风。秋谷何等留心,早已看见,只作不知。赛飞珠和秋谷立谈一会便走了开去,秋谷任其自便,不去留他。恰好烟火已经放起,流星满地,月炮横飞,火树银花,五光十色,做得甚是巧妙,大家喝彩如雷。一连放了八套方才放完,游客纷纷各散,秋谷也同畹香回去。

又过两日,畹香对着秋谷渐渐的要露出嫁他的意思来。在畹香,料着秋谷以为不至推辞,那知秋谷听了,冷冷的并不接口,却对他笑道:“不瞒你说,我自从十七岁出来玩耍,花丛柳阵整整混了五年。这五年之中,同我要好的倌人一时也数他不尽。那初落交情的时候,一个个都是盟山誓海,一定要跟我终身,那甜蜜蜜的话儿说得一连串的,好似漳州的百子炮一般,我也记不得许多。我当时狠是痴心,把他们说的都当作真话,认真的要娶起他来。那晓得那班倌人听得你真要娶他,便指西话东的和你白赖,不是说老鸨不从,就是说父母不肯,再不就说自己的亏空太多。

闹了多时,许多要好的倌人终久没有娶成一个,反冤枉花了无数瘟钱,方晓得倌人们说要嫁人,是一句随口应酬的说话,并没有一点真心,客人们若要当起真来,就免不得要落他的圈套了。你自然不是那样的人,我也没有什么不信。但是我们要好在心,也不必一定要讲到嫁娶,万一你嫁我之后,将来有些不像意思地方,那就不妥当了。我看还是慢慢的再谈罢!“

这几句话,秋谷也未免说得过分了些,把个陆畹香直气得呆了,花容失色,面罩浓霜,心头一股酸气透到顶门之上,一直酸到鼻尖上来,再也耐忍不住,两行珠泪直滚下来。也不言语,径自走到床边,面向里床睡下,暗暗流泪。秋谷见了,方觉得自己的话说得太急了些,懊悔不该这般老辣,便也走到床边来。叫了几声不应,坐在床沿上又温存劝解了一番,仍不见畹香开口。秋谷便一把挽着他的纤手,勉强扶起他来。宝髻横斜,花钿不整,容光渗淡,珠泪阑干,真似那雨打梨花,风吹菡萏。秋谷见他甚觉可怜,便自家认错道:“我说的并不是你,休得这样多心。如今也不必说了,总是我的说话太过了些,惹得你这般生气,只好你原谅些儿的了。”

畹香听了,只是一言不发,听凭章秋谷怎样温存,如何劝解,只当没有听见一般,把秋谷的手推开,别转头去。把章秋谷磨得急了,欲待不去理他,觉得心上过意不去,只得说道:“我这样的认错,你还是不发一言,究竟你要怎样方好呢?”畹香方才说道:“耐勿答应末也只要回报一声,倪勿见得好自家挜上仔门格。倪又勿是林黛玉、陆兰芬,好借仔嫁自家淴浴。耐拿倪说得实梗坏法,叫倪阿要动气?”秋谷又劝了一回,畹香只是紧锁双眉,全无喜色。

秋谷没法,想道:“看他这种样儿,或者竟是真心也未可定。我不妨姑且答应了他,博得个大家欢喜,随后再想法儿回他便了。”便道:“你这个样儿真是叫人难过。只要你欢欢喜喜的不要动气,凡事总好商量。我方才的说话,是怕你将来有些过不惯的地方,并不是我不肯。只要你自家情愿,我岂有颠倒不肯的道理?”畹香两手齐摇道:“阿唷!倪呒拨格号福气,勿要折煞仔人,耐就是实梗仔罢,倪格闲话才是假格。耐豪燥当心点,勿要上仔倪格当。”秋谷倒笑起来,又着实安慰了一番,畹香方才有点笑容,道:“倪好好里勒浪天津,拨格断命格外国人打仔进来,吓末拨俚吓煞快,逃来逃去,吃仔几几化化格苦头,总算逃仔一条性命。故歇倪想起来,勿到天津去末,也吃勿着格个大吓头,阿是总是吃仔格碗堂子饭格勿好。倪想来想去,直头无啥趣势。譬如倪勒浪天津格辰光,拨外国人杀脱仔,故歇是随便啥格事体,倪才看穿哉。只想拣着一个客人,嫁拨仔俚完结,勿壳张倪刚刚说仔一句,就吃着耐格个钝杠,耐想耐格人阿要刁枭?”

秋谷听他这几句话,像似真的一般,虽然含糊答应了他,不免也在心中思索,懊悔自家不该粘花惹草,到处留情,牵惹出这些枝节。虽然娶个侧室也不算什么希奇,无奈堂子出身的人,总是一般脾气:在堂子里的时候,终日应酬客人忙忙碌碌,不知不觉的把日子混了过去;一到嫁人之后,无事可做,英雄无用武之地,就不免有些懊闷起来。况且他们生长在堂子里头,耳濡目染的都是些无耻的行为,司空见惯,不以为奇,竟不知世界之间尚有廉耻。就使他们的嫁人果是真心,没有什么歹意,但是他们看惯了这些勾当,不晓得妇人名节是最重的事情,那里好做得良家妇女?万一他将来见了个风流子弟,保不住他不起邪心。做过妓女的人,看得这偶然轧个姘头更是希松的事,好似他平常出去坐回马车,吃顿大菜,借此消遣性情的一般,非但算不得背主通情,并且也不是昧良失节。你想那倌人可是娶得的么?方才看那陆畹香的情形,或者竟是真心也未可定,然而与其将来懊恼,不如眼下推开。

但已经答应了他,说得结结实实的,怎样好无故反悔呢?章秋谷的心上左轮右转,一时就如辘轳一般转移不定。忽然想起一个人来,想道:何不如此这般试他一试,他若全然不动,便是个娶得的人,不妨竟把他娶回家去,料也不至龃龉;若是他中了机关,我就当他的面一口叫穿,只不要同他翻面,此后照旧往来,料他不好意思再提嫁我的一层说话,只要彼此暗中明白就是了。

主意已定,过了一夜,明天一早起来,一直赶到赛飞珠的寓处高升栈内,寻着了赛飞珠。那赛飞珠正在和人说话,忽见章秋谷走了进来,出其不意,连忙迎出房中,笑道:“章老爷,什么风儿把你吹到此地?”秋谷笑道:“我因有一件事情同你商议,所以一早到来,你务必要帮我一个忙儿。”赛飞珠听了,诧异道:“章老爷有什么事情要托起我来,可是要定什么堂戏么?若是我办得到的,一定效劳。”

秋谷微笑,叫赛飞珠走到面前,附着他的耳朵说了半晌。只见赛飞珠连连含笑摇头道:“这件事我却答应不来,请章老爷照顾别人罢。现在章老爷虽是这般说法,不过是一时高兴,说着玩罢了。设或将来懊悔,吃起醋来,我却担当不起。”正是:

推出窗前之月,分付梅花;移来别岫之云,温存桃叶。

不知秋谷怎生说法,请听下回。





第三十二回 吊膀子小丑帮忙 掉枪花秋娘中计





却说章秋谷见赛飞珠不肯答应,又附耳说了一回,又道:“这是我央你的事情,你若肯帮我的忙,我只有感激你的,那有反来怪你之理?你若果然办得成这件事儿,我一定重重的谢你。赛飞珠方才点头答应。又向秋谷道:”这件事情,不是我在章老爷面前夸句口儿:手到擒来,十分容易。但是办成了也没有什么凭据,他又万不肯说出口来,难道我好去和他当面质对么?“秋谷一想,果然不错,踌躇了一会,便向赛飞珠道:”这个不难,我教你给一个法子。“又低低的说了几句道:”你只消如此这般。到手之后便送到我栈内来,我自然从丰酬谢。但是你在外边千万谨言,切不可向人提起,万一被他得了风声,就莫想他肯来上钩了。“赛飞珠听了心领神会,连连点头。秋谷便回栈去了。

一连过了几天,秋谷也常到陆畹香家走走,并不提起那天早起的事情,这一天下午,正在栈内会着客人,忽见茶房领着一个娘姨进来。秋谷认得是林黛玉的娘姨,便问他来此何事。那娘姨向秋谷道:“大小姐叫倪来请二少过去,有格苏州来格先生勒浪倪搭,说俚一径认得二少格,要请二少过去说两声闲话。”秋谷听了,摸不着头路,便问那娘姨道:“我在苏州虽然认得几个倌人,然而同你们大小姐都不认得,况且无缘无故也不见得到上海来寻我,你可晓得他的名字么?”娘姨道:“倪勿晓得俚叫啥格名字,像煞是姓金格。”秋谷想了一会,依然记不起来,便道:“你先回去,说我少停一刻就来。”娘姨答应而去。

秋谷等得客人去了,急于要到惠福里去看看那来的究竟是个什么人儿,便忙忙的走出吉升栈,上了包车,飞一般的到惠福里来。不多几步,已到门前。秋谷下车进弄,直走进去,三脚两步的走上扶梯。进房一看,只见一个丽人正坐在窗前,和林黛玉低声说话。香肩琐琐,艳影亭亭。秋谷定睛看时,早吃了一惊,原来不是别人,正是那大金月兰。当下连忙问道:“你说到上海来的,为什么直到如今才到?

在苏州有什么事情?“月兰见了秋谷不免有些惭愧,答应不出来,转是林黛玉替他把来去的情事一一说明,又道:”俚耐现在人末到仔上海,事体弄得尴尬哉,俚耐心浪原要想跟耐转去,耐看那哼?“

原来这金月兰自从在常熟和秋谷分手之后到了苏州,他却不到上海,仍在佛照楼住了两天。他自家打算上海去,又没有什么熟人,又不敢再做生意,只得且住苏州,耽搁几时再作道理。住了不多几日,早又姘了一个姓潘的,叫潘吉卿,住在闾门城内,却是个有名的败落乡绅。这潘吉卿平日之间专用那吊膀子的工夫,衣服一天要换三回,辫子一天要打两次,那引见皂、口香糖、嫩面粉、花露水,更是随身法宝,时刻不离。到了堂子里头不肯花一个大钱,专想倌人倒贴,真是一个花丛蟊贼,体面流氓。他在佛照楼客栈遇见了金月兰,便留心去吊他的膀子。那相貌的好歹,这潘吉卿倒出不论:无论再是半老秋娘,暮年名妓,鸠盘一般的面貌,夜叉一样的形容,只要肯倒贴银钱,他也肯欣然笑纳。只因打听得金月兰是在黄相国府中逃走出来,料想他手中必定有些积蓄,所以竭力的笼络他。不上两天,居然被他上手。住了两夜,竟明目张胆的把金月兰同转家中。

这潘吉卿的正室久已病亡,家中止有几个家人、仆妇,那敢管他?潘吉卿的本意,原想要大大的骗月兰一注银钱,等到银钱骗到手中,再慢慢的想个法儿把他打发出去。这个主意,比那倌人淴浴、光棍折梢还要恶毒了几倍。不料那金月兰在天津遇了兵乱,单单逃得一个空身,就连那箱子里头的二百块钱,还是章秋谷送他的。

潘吉卿高高兴兴的把他骗到家中,想不到扑了一个空,大失所望,方晓得金月兰两手空空,一无所有,把他留在家中,反要赔贴饭食。潘吉卿气得发昏,便渐渐的寻着事端,与金月兰吵闹非止一次。

月兰已经看破了潘吉卿的行为,心中也十分怨恨,便也要想一个绝户计儿,拿出那以前在黄府内的手段来,把他一捞一个罄净。便故意把自家的几件衣饰并秋谷送他的二百块钱,一齐交在潘吉卿手内,凡遇潘吉卿与他吵闹,月兰并不争执,一味的认错低头。

潘吉卿并不防备他有什么歹意。不料金月兰有心算计着他,和带来的娘姨合成一路,趁着潘吉卿出去,把房间内的细软金珠,还有些古董字画,打了两个大包。

乘着天色将晚,那娘姨挟着两个包,一溜烟走出后门,叫了一号小船,放在船上,把船一直放出城去,停在那丝厂码头,悄悄的等候月兰。这里月兰不慌不忙的叫家人去叫一乘轿子,说是要出城去看戏。那些家人见月兰平日常常出去看戏,不以为奇;又见他是个空身,那轿夫又是向来相熟的靠班,更加大意,梦里也想不到月兰逃走起来。那知月兰上了轿子,一直抬出盘门,到了戏园,便在包厢坐下,吩咐轿夫散戏场的时候再来相接。轿夫并不疑心,乐得自去。月兰略坐一会,看轿夫时,并不见他们的影子,心中大喜,霍地起身望外便走。戏园内人多于蚁,那有人来查问?他出了园门,雇了一部马车直到丝厂码头,寻着了小船,便叫那船家开到洋关左近的地方停了一夜。等到明天,三公司的小火轮验过了关开过来,半路叫住轮船,登时带缆拖在后边,径往上海而去。

到了码头,月兰就寓在后马路晋升栈内。虽然走了出来,心上总有些儿鹘突,恐怕被那潘吉卿赶到上海寻访出来,那时两案齐发,不是玩的。虽然杭州的事情已经结案,却担不起再加一个卷逃的罪名。想来想去,无计可施,打听得林黛玉现在上海,更一直寻到黛玉院中,要同他商议一个安身的法儿。黛玉也是束手无策,便想到把秋谷请来,或者想得出什么主意,也未可知。

月兰听得秋谷也在此间,惊喜交集。便向黛玉把他在苏州和秋谷相处的情形细说一遍,但是走的时候曾经说过即日回来,现在又闹了这样的事儿,未免有些惭愧。

黛玉道:“格是说勿得格哉。耐既然居格辰光说过歇要嫁俚末,故歇正好跟仔俚耐转去避避风头啘。”月兰一想,真是顾不得许多,便点头称是。

及至秋谷来了,听得金月兰又在苏州潘家逃了出来,暗想道:“这真是江山好改,本性难移。幸而我当初乖觉些儿,不然,几乎上了他圈套!”因鄙薄月兰的为人,不免微含怒意。又听黛玉说月兰想要同他回去,连忙摇手,微微的冷笑道:“这件事儿免劳照顾了罢!他刚刚在潘家走了出来,我却连忙把他同回家去,将来被人晓得风声,这不明明是我叫他逃走的么?况且他这样的性情,我也不敢领教,劝你少管些儿闲事罢!”

月兰见秋谷回得斩钉截铁,好似钢刀削了他的面皮一般,红云满面,眦泪溶溶,满心的委屈。正待开口,忽见秋谷的家人闯了进来,道:“栈里有客人立等老爷说话,说有要紧的话儿。”秋谷趁此立起来,向黛玉、月兰说道:“我有事要回去,你们还有什么说话,明天再说罢!”说罢就走了出去。黛玉拉他不住,只得由他。

秋谷疑疑惑惑的,不知那客人到底是谁,问那家人时,家人说向来不认得他,好像个外路的口音。秋谷听了心中一动,想外路口音的人,不要是赛飞珠来了?回得栈中看时,果然是赛飞珠坐在那里。秋谷大喜,问:“那事儿怎么样了?”赛飞珠微笑,走上一步,怀内取出一个黄澄澄的戒指来,递与秋谷。秋谷急看时,只见这戒指雕镂工细,花样时新,中间嵌着一粒小小的钻石,果然是自己在银楼定制、前几天被陆畹香要去的那只戒指,不觉呆了一呆。停了一刻,方向赛飞珠笑道:“果然你的本领不差,费心得狠,等我把这件事儿交涉清楚再行酬谢。”赛飞珠道:“章老爷笑话了!我是因为章老爷再三重托,碍着面情,不好意思不答应,难道我是贪这一点儿谢仪么?”秋谷见他说得认真,倒不便一定怎样,只得笑道:“既然如此,我们随后再说就是了。”赛飞珠方才欢喜,辞别去了。

秋谷便把戒指藏在身边,匆匆的到聚宝坊去见了畹香。畹香满面堆下笑来,请坐下,说了几句闲话。秋谷忽问畹香道:“我前日给你的那个戒指,可在这里么?”

畹香突然被秋谷这一问,不觉陡吃一惊,面上早红起来,顿了一顿,方说道:“耐问俚做啥?自然勒倪搭畹。耐阿是舍勿得哉?倪勿成功格!”秋谷笑道:“那只戒指虽有一粒金刚钻在上面,也不值什么钱,不过花样打得好些罢了。前天有个朋友看见这个戒指,要照样去定一只,所以问我要个样儿。他只要拿去看一看,立刻还来,并不是我舍不得给你。你不要这般小气,快些去寻出来。”畹香被秋谷逼住,腾挪不得,迟迟疑疑的不肯去寻。秋谷催了他几次,又逼他道:“你不肯寻,难道我要骗你一只戒指么?”畹香见秋谷将要动气,无可如何,只得走进后房,一步挪不了三寸,慢慢的进去,假装着寻了一回,故作惊诧之声道:“阿唷!格只戒指勿知拨倪弄到仔陆里去哉!”又叫娘姨来寻,那里有什么戒指的影儿?秋谷听他们装神做鬼,暗中甚是好笑。

畹香乱了一会,又在后房和娘姨密密切切的讲了一会,不知说的什么。恰才走出来,面有愧色,吞吞吐吐的向秋谷说道:“格只戒指实头诧异!倪昨日仔还带格,今朝勿知放仔陆搭去哉!”秋谷尚未开口,旁边的娘姨接口道:“大小姐耐记记看,像煞昨日仔大阿姐来借仔两只戒指,勿知阿就是二少格一只?”畹香拍子道:“划一,大阿姐昨日仔拿仔两只戒指去,倪格记性实头坏得呒拨仔淘成哉!”又向秋谷道:“耐要做样子末,只好明朝到大阿姐搭去拿格哉。”秋谷微微笑道:“只怕这只戒指不是大阿姐借去,是高升栈的四阿哥来借去的罢!”畹香一听,就如当头一个霹雳一般,慌忙说道:“啥格四阿哥,倪是勿晓得格。耐说说末咦要瞎三话四哉。”

秋谷微笑,也不回言,向衣袋取出那只戒指来,向陆畹香面前一掷,道:“你看,这不是四阿哥借去的戒指被我要回来的么?”

章秋谷这一来,真是出于意外,满房人众齐吃一惊,面面相觑,不敢开口,只把一个陆畹香羞得满面飞红,急得浑身香汗,一句话也回不出来,恨不得有个地洞钻了下去。正是:

暗赠搔头之玉,绮梦缠绵;强追约指之银,萧郎薄幸。

欲知章秋谷和赛飞珠商量的究竟是甚事情,陆畹香为什么见了一个戒指便要这般惭愧,编书的在下写到此间,笔秃不花,灯昏无焰。权且学些近日时下说书的习气,到了紧要之处把笔墨收束起来,直至三集书中再行分解。还有许多嫖界、官场的现状,卑鄙龃龊的情形;倒脱靴再行骗局,康中丞帷薄不修等诸般事实,请看三四续集,便知分晓。





第三十三回 姘戏子苦劝陆畹香 扳差头驳倒花筱舫





前回书中做到陆畹香见了戒指,满面羞惭,无言可答,恨不得当时有个地洞钻了下去。

潇湘花侍做到此间,暂停笔墨,作个《九尾龟》二集的收场,正要续成三集,就有一位花丛的大涉猎家来批驳在下道:“你初、二集书中,记那四大金刚和大金月兰、陆畹香的事迹,虽然大半都是实情,但是他们出现的时代和那来去的行踪,却不免有些舛错。为什么呢?你说金月兰在杭州黄中堂府内逃走出来,一直径到天津去搭了东天保的班子。后来拳匪闹事,联军破了天津,金月兰同着林黛玉等一班名妓狼狈逃归,一无所有。这金月兰几年内的历史是不错的了,但是林黛玉嫁了邱八之后,重又闹了出来,上海议论纷纷,存身不住,方才无可如何的北上津沽,打算要作个孤注一掷。及至遇了拳匪之乱,一直由天津逃到山东,在山东再折回上海,这便是林黛玉在津沪来去的行踪。你却说他在邱八家中出来之后就在上海做了住家,并不提起天津一节,这不是老大的一个岔子么?况且那年庚子之乱,上海的倌人大家逃避,是在六七月内的事情,你的书中好像是二三月的样子,你何不将前二集书中这几段的舛误之处重新改正,把这一部书成了全璧呢?”

潇湘花侍哑然一笑,回答他道:“在下做这部书,一半原是寓言醒世,所以上半部形容嫖界,下半部叫醒官场,处处都隐寓着劝惩的意思,好叫列位看官看看在下的这部小说,或者有回头警醒的人,这也总算是在下编书的一片苦心,一腔热血;并不是闲着笔墨,旷着功夫,去做那嫖界的指南、花丛的历史。若要把在下这部小说当作历代兴亡的史鉴、泰西各国的蓝皮书,那就差之毫厘,谬以千里了。”知

闲话休提,书归正传。只说前回的章秋谷和那赛飞珠鬼鬼祟祟,到底商量什么事情?章秋谷送给陆畹香的戒指,怎么又会到了赛飞珠的手中?真是一本算不清的糊涂帐目,在下不说明白,料想看官们有细心推究的,也有些想得出当日的情形。

原来章秋谷因陆畹香定要嫁他,推辞不脱,堂子里头的规矩,若是那客人要娶倌人,倌人不肯;倌人要嫁客人,客人不要:这两件事真是那天字第一号的坍台,竟有不共戴天的光景。章秋谷被陆畹香缠住了不得开交,又不肯当面回绝叫畹香的面子下不来,左思右想甚是为难。忽被他想着了一个刁钻主意:他以前在苏州,晓得赛飞珠吊膀子的工夫甚好,便到高升栈去寻着了他说明原委,要他去吊陆畹香的膀子。料想堂子里的倌人,那里有什么定力?况且赛飞珠的身段甚好,相貌也在中上之间,就口馒头落得慨然领受。赛飞珠初时不肯应承,秋谷许了他的谢仪,方才答应。又怕没有凭据,秋谷便叫他上手之后问陆畹香要个戒指作为表记,又向他说了畹香手上戒指的样式,叫他诸事小心在意,切不可露了口风。赛飞珠欣然答应,便借着去探望秋谷,到聚宝坊来见了陆畹香。

戏子们吊膀子的工夫果然利害,别有心传,不多几天,三言二语的,那陆畹香那里晓得是章秋谷叫来做弄他的,容容易易竟是被他吊上。过了两夜,便问畹香要个戒指。畹香正是同他打得火热的时候,自然情情愿愿的给他。赛飞珠却嫌着这一个戒指的样式不好,那个戒指的宝石不精。畹香拿了几个出来,换来换去都不中意,就赌气不要了。畹香急了,就拿章秋谷给他的那一个戒指拿出来,替他带在手上,方才欢喜。那知他刚得转身,就飞一般跑到吉升栈来找秋谷,把戒指给与秋谷,又将前后的情节述了一番。秋谷便把戒指带在身上,径到聚宝坊来,问畹香要取那一个戒指。畹香吃了一惊,暗想:“天下真有这般巧事,怎么一边刚才带去,一边就忽然的要起来?”只得假做寻了一回,支吾半晌,暗地和娘姨说明,说是被大阿姐借去。秋谷当时说道:“只怕是高升栈的四阿哥来借去的罢。”

原来那赛飞珠排行第四,人人都赶着他叫“滑头阿四”,所以秋谷说这个影射的话儿,要叫他自家明白。陆畹香听了,当顶门就是一针,勉强装作不知,强颜为笑,还想要用言遮饰。不料章秋谷当时取出戒指,送到畹香面前。这一来,把个陆畹香逼得目定口呆,好似那深山樵子忽闻虎豹之声,弥月婴儿乍被雷霆之震。只见他低下头去,一言不发,那面上一阵阵的泛出红来。看他那惭愧的神情,真是万分难过。在章秋谷的意思,原不要和他翻面绝交,只因畹香定要嫁他,腾挪不得,所以想出这一个偷天换日的奇谋,拿住了他姘戏子的真赃实犯,那嫁的一层说话自然说不出来。却想不到自己这个主意虽然不错,却忒嫌刻毒了些儿。你想那陆畹香一副嫩郁郁吹弹得破的脸皮,那里禁得起这般砢碜?秋谷见了,觉得也有些懊悔起来,倒向畹香笑道:“我不过和你说了一句笑话,你何必这样的认真,我又不来怪你,只要你自家明白就是了。难道我们认得了这几年,你还没有晓得我的脾气,这些小事一定要和你过不去么?”

陆畹见香秋谷非但并不翻面,倒如无其事的去安慰着他,心上狠是感激秋谷遇事含容,不肯出他的丑,又羞又喜,一个头低了下去,那头上好像有一座泰山压住的一般,羞怯怯的只是抬不起来。秋谷见了,点头暗赞畹香天良未泯,还有些羞耻之心,想来还可劝化得转,不免再费一番唇舌把他提醒一场,也算不枉了两年相识。

便携着畹香纤手,把他拉到烟榻旁边,两下对面躺下。秋谷看着畹香面上还是两颊绯红,羞态可掬,正是:

红上胭脂之颊,两涴桃花;春横却月之眉,羞颦杨柳。

秋谷觉得有些怜惜起来,便低低的向他说道:“这件事儿,你也不过是一时之错。我虽然晓得,决不向人传说,坏你的名头,你只顾放心,不必放在心上。况且现在上海滩上,有些名气的倌人,那一个不要姘几个戏子?算不得什么希奇。”畹香听秋谷说到此处,越发羞得背过脸去,把一方白绸小手巾掩住两眼,几乎要哭出来。

秋谷见了甚觉可怜,携着他的手温存一会,方又说道:“姘几个戏子虽然算不得希奇,但是你们堂子里的倌人犯了这个毛病,被外头传说出来,非但生意上头大有妨碍,而且从此露了名头,真是一件有损无益的事。为什么你这样一个聪明绝顶的人,这件事儿恰看他不透?你想,那戏子同倌人轧了姘头,不肯花钱,专要想倌人的倒贴。倌人们辛辛苦苦在客人身上敲了竹杠出来,去供那戏子的挥霍,好像不是戏子姘着倌人,倒是倌人嫖着戏子一般。到了倌人的银钱用尽、供应不来的时候,他就立时立刻翻转面孔,和你断了交情。轧姘头轧到这个样儿,可还有什么趣味?

从来妓女无情,优伶无义,你们做倌人的在客人身上虽然没有良心,独到和戏子轧了姘头,却是真心相待,偏偏遇着那班戏子,平时看待别人也还不到得这般刻毒,一到姘着了一个倌人,就出奇的天良尽丧起来。我也不懂这个里头到底是怎么的一个讲究。再说起那班爱姘戏子的倌人来,以前的周双林,现在的花玉笙,那一个不是姘了戏子弄得声名狼藉,车马稀疏,到后头拆姘头的时候,还免不了一场吵闹。

从没有姘戏子的有个好好的收场。你如今趁着外边没有风声,快快的回头改过,不要到了将来,和周双林、花玉笙一样起来,那时就懊悔嫌迟了。我劝你的一番说话,却是句句良言,你不要认错了我的意思,当作故意来坍你的台,那就埋没了我的一片真心了。“

陆畹香听了章秋谷这一番提醒的良言,觉得无一句不体贴,无一字不婉转,不由得那感激秋谷的心念,就感激到二十四分。暗想:“如今世上那里还有这样好人,晓得我姘了赛飞珠,他不吃醋也罢了,还肯这样苦口劝人,说得这般真切;并且留着我的面子不肯高声,恐怕被娘姨们听见不好意思,真是个天字号的好客人!”这样一想,便慢慢的回过脸来,握着秋谷的手,含情带愧,相视无言。忽又自家懊悔不该姘了戏子,做出这样事儿,料想要嫁他的一层说话,是不消提起的了。眼看着章秋谷这样的一个风流名士,倜傥才人,自家做错了事情,消受不起,不觉由感生惭,由惭生悔,懊悔到极处,竟忍不住两行珠泪直滚下来。秋谷明晓得他的意思,安慰一番也就罢了。

秋谷略坐一会,正欲起身,忽见辛修甫同陈海秋走了上来。大家相见过了,秋谷道:“我道客人是谁,原来是你们二位,想来有什么事情么?”修甫笑道:“也没有什么别事,今天是陈海翁专诚请你在东合兴花筱舫家吃酒,恐怕你有了应酬不到,所以我们特地自己过来相请,可好就此同行?”秋谷笑道:“既然陈海翁赏光请我,岂有不到之理?但是时候尚早,何必这样要紧,尽可在此宽坐一回再去,十分早去了,也没有什么道理。”修甫道:“在我多坐一回也不要紧,但陈海翁是个性急的人,我们还是就去的好,省得他发躁起来。”秋谷一笑,便也起身。

三人一路同到东合兴来,秋谷走进弄堂,就看见第三家门左高高的挂着一块花筱舫的金字招牌。陈海秋当先走进,秋谷等跟着上了扶梯。进得房来,娘姨招呼坐定,却不见倌人出来。秋谷便问那娘姨道:“你家先生可是堂差出去了么?”娘姨陪笑道:“倪先生勒浪后房就出来哉。”秋谷听了,暗想:倌人既然没有出去,为什么不来应酬?心上就有些不然起来。

坐不到一盏茶时,方见一个十八九岁的倌人从床后走将出来,五短身材,面貌也还秀丽,小花宝髻,石竹罗衣,虽无倾国之姿,大有回风之态。只是一张瘦骨脸儿,觉得露筋显骨的没有那妩媚的神情。走到面前,大落落的,慢慢的叫了一声“陈老!”也不招呼客人,便一屁股坐在凳上。忽回头见了章秋谷仙骨珊珊,五山朗朗,似有一道光华射将过去,吃了一惊,连忙又立起来走到秋谷身旁,问他尊姓。

秋谷此时见花筱舫一面孔的时髦倌人,架子甚大,心上十分有气,不去理他。见他来请问姓名,勉强回称姓章。花筱舫倒着实应酬了他几句。修甫便向筱舫笑道:“怎么你不应酬我,单应酬他,可是见他面孔生得标致么?”筱舫被修甫说破心事,面上不免一红道:“格位章大少是今朝第一转来,耐是同仔陈老日日来格,倪自然要先应酬仔生客,再挨着耐格熟客,慢慢里来,耐勿要性急嗫。”说着,便走了开去。

陈海秋便问筱舫道:“请客的可曾回来?我们先摆起台面来罢!”花筱舫冷冷的答道:“耐请格客人倒有一半勿来,才勒浪搭耐客气,耐阿要再去请仔两个罢。”

秋谷听了冷笑一声,向修甫道:“陈海翁请的客人有一半不到,是替他客气也还罢了,怎么他们这里的花头,今天也只有陈海翁一个,难道这样的红倌人,那班吃酒的客人也同他客气不成?”修甫听了一笑。

筱舫听章秋谷的说话来得锋铓,知道一定是个花丛老手,只把他说得连耳根满面通红,瞅了秋谷一眼,又不好发作,只得笑道:“倪是勿会应酬格,闲话说得勿好。章大少看陈老面浪包涵倪点,勿要扳倪格差头。”秋谷听了正要回答,听得楼下高叫“客人上来”,秋谷同陈海秋起身看时,却是贡春树来了,便打断了话头。

略谈几句,先摆起台面来。随后客人陆陆续续的到了几个,原来王小屏等一班旧识。

入席之后,陈海秋鼓起酒兴,叫相帮去大菜馆内拿了几瓶会司克来,开了瓶,斟在玻璃缸内,要合席和他照杯,众人只得勉强相陪。干了一杯,陈海秋还不肯歇,又自己干了两杯,不觉就有了七八分醉意。正是:

银屏锦帐,缠绵杜牧之情;冶叶狂花,辜负韦郎之意。

欲知陈海秋醉后如何,下回分解。





第三十四回 杀风景莽客醉飞觞 意缠绵良宵花解语





且说陈海秋多喝了几杯酒,醉眼朦胧,有些糊糊涂涂的,斟了一满杯酒,要和章秋谷对饮。秋谷不胜酒力,连忙摇手推辞。陈海秋见秋谷不饮,回过头来,见筱舫坐在身后,便把满满的一杯会司克递在花筱肪手中,要他代饮。筱舫接了酒,仍旧放在席间,冷笑道:“章大少勿肯吃酒末,阿关得倪啥事,那哼叫倪来代章大少格酒介?”陈海秋见他不饮,酒醉的人最易提动肝气,已有了几分怒意,也不开口,仍向席间取了酒杯,直送到花筱舫口边,一定要他和秋谷代饮。花筱舫袖着两手,不住的摇头,那里肯接?陈海秋一手拿着酒杯,伸了出去,竟缩不回来,就乘醉大声道:“你当真不喝,我灌也要灌你一杯。”便踉踉跄跄的直立起来。花筱舫恐怕他真要硬灌起来,只得一手接过酒杯,一手推他坐下,道:“勿要来动手动脚,像啥格样式?等倪自家来吃末哉。”陈海秋见他肯吃,方才住手,却不肯坐下,要候花筱舫吃于了这一杯。花筱舫只得皱着眉头勉强吃了一口。那知这会司克的酒性燥烈非常,花筱舫又是向来不能吃酒的人,一口酒刚到喉咙,没有下咽,就觉得一股辣气直透入脑门里来,不由得连忙回过头去,把一口酒吐将出来,又急急的取过茶碗喝了两口茶,方才罢了。

不料陈海秋见花筱舫刚刚接过酒杯吃得一口,仍旧一齐回了出来,认作他有心不吃,心上登时大怒。乘着酒兴,一手抢过那一杯满满的酒来,连酒连杯子望地下一摔,只听豁啷一声,杯子打得粉碎,把秋谷等大家都吃了一惊,齐声相劝。花筱舫却扳着面孔,冷笑道:“倪从来勿会吃酒,大家才晓得格,就是客人笃代酒末,也有娘姨勒浪啘。故歇格客人才有点阴阳怪气,倪勿做生意末,把势饭也吃仔两年哉,勿壳张今朝耐吃醉仔格酒,来瞎起倪格花头,阿要诧异!”

秋谷听花筱肪的说话,夹七夹八的不知说的什么,便也动起气来,正色向筱舫道:“你刚才一番说话,还是有心说着我们这起客人,还是说的陈老?你若要说着我们,我们却并没有叫你吃酒;若是说的陈老,客人们要倌人吃酒,也是常有的事,算不得什么希奇。况且陈老今天已经醉了,你们既是多时相好,却不该说出这样的话儿,索性连我们客人也骂在一起。我倒要请教请教,你们堂子里头,可有这个规矩么?”

花筱舫被秋谷扳住错处,开口不得,心上虽然暗恨,却不得不敷衍他们,勉强忍住了气,向秋谷道:“倪是一句无心闲话,章大少勿要动气,索性费耐章大少格心,劝劝陈老。倪也是一时之错,勿要作倪格过意。”秋谷听得花筱舫自家认错,方不开口。

陈海秋掼碎了一只酒杯,不觉酒涌上来,口中却还在那里乱嚷道:“我不过叫他吃一杯酒,他一定不肯,有心坍我的台,难道我就罢了不成?”说着便立起来又去斟酒,一定要叫花筱舫和他照杯。王小屏在旁劝道:“他既然不能吃酒,你何必定要强他,不如让他喝了一杯绍酒,过过你的场罢。”陈海秋还不肯答应,当不得众人大家称是,又劝他:花柳场中本是寻欢取乐的地方,何必要斗这般闲气?陈海秋无奈,只得点头,自己取过酒壶,斟了满满的一杯绍酒递与筱舫,立逼着要他一气饮干。

花筱舫见方才一番说话犯了众怒,明知不得开交,只得接过酒杯,在口边试了一试,蹙起双眉,把那一大杯酒慢慢的一口一口,刚刚吃得一半──原来不能吃酒的人,那里喝得下这许多酒──不觉喉中一呛,那酒直冲起来,把那刚才咽下的酒往外直冲,口内冲不迭,连鼻孔内也冲出酒来。花筱舫身上穿一件湖色闪光外国纱衫,吐得浑身湿透,就是陈海秋身上也沾着些儿。把个花筱舫直吐得粉黛霪霪,胭脂狼藉,更兼头痛眼花,说不出来的一种难过,不由心中大怒,把心肠横了一横,顾不得客人挑眼,把手内的酒杯竟是也往地下一掼,一言不发,立起身来,跑进后房去了。

陈海秋这一气非同小可,连忙跳起来,要走进后房去追赶筱舫。秋谷等大家见此情形,十分诧怪;又见陈海秋要赶进后房,一把将他拉住道:“你不要这般性急,筱舫虽然可恶,你就是打掉他的房间,也没有什么道理。将来传说出去,终是一件杀风景的事情,反说是我们酒醉滋事。你且不要动气,且去叫他出来,看他有何理说。”

陈海秋见秋谷说得不差,捺住了一股恶气,便和娘姨说道:“你们去叫了先生出来,他方才好好的,又没有人得罪了他,为什么无缘无故的跑了进去?娘姨听了,便向后房去叫筱舫。叫了半天,非但筱舫不来,连那娘姨也躲在后房不见出来。陈海秋等了一回,甚是焦躁,又直着喉咙叫了筱舫两声,竟不见后房答应。海秋冷笑道:”我倒从没有遇着上海滩上的倌人有这样大的牌子!既是这样,你又何必要出来做什么生意呢?“听得筱舫在里房高声说道:”倪人末吃仔格碗把势饭,倒也勿在乎此格。唔笃高兴末,赏赏倪格光,倪也无啥希奇;勿高兴末,随便唔笃未哉。

“

此时章秋谷见花筱舫这般说法,有心得罪客人,暗想:“这样的倌人无从与他讲理,只好想一个计较,也用野蛮手段去对待他。”眉头一皱,早已想了一个法儿。

只见陈海秋气得喘吁吁的一句话也说不出来,秋谷一面劝他,一面附着陈海秋的耳朵说了几句,海秋大喜,连连点头。

秋谷明知后房没有客人,只有花筱舫和娘姨两人在内,竟自走了进去。见花筱舫满面怒容,把一件纱衫卸去,单穿一件粉红汗衫,正在那里对着镜子,重施脂粉,再点铅黄。娘姨立在身后也不言语,见秋谷走进,并不招呼,口中说道:“阿呀!

间搭是龌龊煞格,章大少请外势坐罢。“秋谷走近一步,含笑说道:”我特来请你出去,为什么要这般动气?就是陈老叫你吃杯酒儿,也不算得罪着你;况陈老已经醉了,你也须原谅他些,无论你再有天大的委屈,有我在这里一面招陪,快些出去应酬,不要冷了台面。“

花筱舫见章秋谷满面春风进来相劝,把方才的气恼早已丢过一边,只不好意思当时出去,把秋谷瞟了一眼,微笑一笑。秋谷见他已经心肯,便趁势上前携着花筱舫的手,低低笑道:“就算陈老得罪了你,却与我们客人无涉。难道我自己进来请你,你还不肯赏光么?你若再要这般生气,不肯出去,就是有心坍我的台了。”说着不由分说,携着筱舫往外便走。说也奇怪,花筱舫的一个身体,不由软洋洋的跟着他立了起来,却瞋了秋谷一眼,道:“慢慢的叫看嗫,让倪着好仔衣裳看。”秋谷听了,暂时放手。娘姨另取了一件纱衫和他披上,钮好扣子,方才同着秋谷移步出来。

只见陈海秋颓然座上,酒意醺人。花筱舫虽然走了出来,不免还有几分怒气,在海秋背后一坐,默默无言。秋谷向花筱舫使一个眼色,筱舫只得立起身来,在席上斟了一巡酒,算是自家赔个不是的样儿,向修甫等说道:“倪刚刚进去换件衣裳,各位包涵点,勿要动气。”大家见花筱舫这个样儿,颇觉气愤,却又不好发作,只好勉强点头。无心吃酒,大家草草终席,一齐立起身来。陈海秋醉到十分,立脚不住,向秋谷道:“你们要紧回去,我却今天醉得挣紥不来,只好在这里借个干铺的了。”花筱舫听了,冷笑一声。秋谷见不是头路,便向海秋道:“我看你今天还是回去的好,借干铺是不便当的。”章秋谷一句话还未说完,陈海秋酒在肚里,事在心头。他本是个广东人,初入花丛,那里晓得堂子中的规矩?就大声说道:“我在他们这里摆了好几台酒,难道今天借个干铺都不行么?”花筱舫只是在旁冷笑。秋谷听陈海秋说的都是些曲辫子的话儿,不再去和他多讲,一手拉住他的衣袖往外就走。陈海秋那里拗得过他,被秋谷拉得七跌八铳的,跟着下楼。修甫等见了,甚是好笑。

大家一哄而下,走到门前。秋谷道:“还是我寓内近些,我们且到吉升栈去坐一会儿再说。”大家称是。出了东合兴,便直到吉升栈来开了房门,大家坐下。

陈海秋坐了一会,神气顿清。秋谷向海秋道:“你这个人真真的没有志气,闹到这般地位,还要在他们那里借起干铺来。要晓得我叫你不要发标,是卫顾你暂时的面子,是个好好的落场。你若要和他闹些脾气,他肯来认错张罗还好;万一他横了心肠,听凭你们怎样,他只是一个不见不闻,不来敷衍,那时你又怎的一个落场?

我们都是面子上人,又怎的坍台得起?所以我把你暂时劝住,遮过了当时的场面,然后慢慢的再想收拾他的法儿,你道可好?“

陈海秋听了章秋谷的说话,一想果然不错,便道:“你的说话虽是不错,但想个什么计较去收拾他呢?”秋谷道:“我早已打定了一个主意。明天我邀你在陈文仙处碰和,却把花筱舫叫来代碰,那时我们如此这般,管教要把他气一个发昏。你们众位看来,我想的这个法儿怎样?”众人一齐称是。陈海秋道:“万一他不来呢?”

秋谷道:“上海地方,熟客叫局那有不来之理?况且今天散的时候原是欢欢喜喜的,不露一毫马脚,他那里就看想得到有这一着棋子出来?这个你倒不必多虑。”陈海秋听了点头。坐了一会,大家告辞散了。秋谷却到陈文仙院中住了一夜。

文仙因秋谷多日不来,颇形怨望,并且文仙发痧方好,脸上瘦了些儿,从前是荷粉露垂,杏花烟润,如今却是腰低弱柳,眉销湘烟,低回西子之颦,天袅落花之舞,大有六铢衣薄、翠袖惊风的意态。秋谷便默然相对,细细的领略色香。文仙和他说话,竟不答应,只点头微笑。文仙道:“耐今朝啥格路道,跑得来口也勿开,阿是倪得罪仔耐哉,耐看见仔倪讨气?”秋谷依然不答,只是上上下下的看他,把个陈文仙呕得急了,走过来揪着秋谷的耳朵,道:“啥格倪搭耐讲章,耐一声勿响,耳朵到仔洛里去哉?”秋谷见文仙发起极来,方才立起来,哈哈一笑,便把陆畹香一节情事细细的告诉他。

文仙听完,把秋谷打了一下,又把嘴一披道:“耐格心思倒直头刻毒笃啘,就是陆畹香要嫁拨耐末,也是俚格要好。耐心浪勿高兴末,啥勿爽爽快快回头仔俚,要俚去上格种恶当。俚耐上仔耐格当,耐也无啥好处啘。倒看耐勿出,做起事体来实梗格刁枭法子,真真少有出见格。难下转倪也要当心点哉!”秋谷哈哈的笑道:“他是爱姘戏子,所以上了我的牢笼。你是向来不姘戏子的人,为什么要你当心,可是近来也有些……”秋谷说到此处口中顿了一顿,似笑不笑的看着文仙。文仙急了,板着面孔接下去问道:“有点啥末事介,说下去嗫。”秋谷道:“我不说了,若要直说出来,你岂不要生气?”文仙蛾眉半蹙,杏眼含瞋的,正色向秋谷说道:“二少,倪讲闲话是讲闲话,搂白相是搂白相,耐倒勿要勒浪随仔只嘴瞎说一泡,耐末是说格笑话,拨别人家当起真来,说仔出去,看耐那哼对倪得起!”

秋谷见文仙将要动气,便过来携住他的纤腕,道:“我是一句无心笑话,你何必要这样认真?”文仙道:“耐末说说笑话呒啥希奇,阿晓得倪吃勿消?”秋谷打着苏白笑道:“倪也朆说啥格呀,先生勿要动气嗫。”说着,就向文仙打了一拱。

文仙也忍不住笑道:“厚皮得来,才做得出格。”说罢,回过手去把秋谷膀子上拧了一把,道:“耐下转阿要瞎三话四哉?”秋谷被他拧得叫了一声“阿呀”,道:“你这个人岂有此理!大家说说玩话,怎么用劲拧起来?”文仙道:“啥人叫耐瞎说一泡格介,耐阿是嫌比勿痛,等倪再来补两把阿好?”秋谷连忙跑开,彼此一笑。

秋谷又向他说:“花筱舫有心得罪客人,十分可恶,明天要在你这里请客碰和,去叫花筱舫来代碰,好如此这般的翻他的本儿,当着众人的面,给他一个大大的下不来,也叫他以后自家晓得些儿难处。”正是:

熨贴檀郎之意,玉软香温;安排花信之风,嗔莺叱燕。

不知以后如何,请看下回交代。





第三十五回 暗提调碰和叫局 现开销当面坍台





且说陈文仙听了章秋谷的说话,瞋了他一眼,道:“别人家格事体,阿关得耐啥事,要耐去瞎起劲?就是花筱舫得罪仔客人末,耐也勿犯着来做格个冤家啘。”

秋谷听了,微笑不言。一夜无话,不提。

到了明日上灯时候,果然陈海秋拉着修甫同来。不多时,贡春树也来了。当下碰和脚色已齐,文仙亲手配了筹码,大家入座扳庄。秋谷道:“你们不要心慌,先发了局票再说。”修甫道:“果然,待我写起来就是了。”秋谷道:“今天碰和只有四人,我自己也叫一个,趁趁你们大家热闹。”文仙瞅了秋谷一眼,却不作声。

秋谷便叫了陆兰芬,修甫叫的龙蟾珠,贡春树不消说自然是金小宝了。修甫提笔在手,一一写好。秋谷拿过来点一点不错,就把花筱舫的一张局票抽出来搁在旁边,还有那三张局票一并交在娘姨手中,叫他传下楼去。陈海秋见了,诧异道:“一样的四张局票,自然一起去发,为什么要留下一张,难道还恐怕他来得太早了么?”

秋谷道:“不是这个讲究,少停你自然明白。”陈海秋不便开言,心上十分的疑惑。

修甫同春树也有些不懂起来,同声问道:“到底你是个什么意思?不妨此刻说明。”

秋谷笑道:“这是我的军机密事,岂能和你说明?你们不要开口,在旁看着就是了。”

说罢不由分说,自家坐下,便去扳庄。

陈海秋等见章秋谷不肯说出,也不晓得他葫芦里头卖的是什么药,又不好苦苦的追问,便只得归座扳庄。扳好了庄,转过坐位,碰不到两副,陆兰芬已经到了。

湘帘启处,莲步移时,香风已到。眉画初三之月,绿锁横波;鬓挑巫峡之云,花欹宝髻。戴一头翡翠押发,穿一身浅色衣裳,轻启朱唇,低开檀口,笑盈盈的叫了一声“二少”。秋谷还不曾答应,这一声不打紧,早把个贡春树叫得直跳起来,逼紧喉咙打着苏白道:“阿呀!先生格喉咙脆得来格,一声‘二少’,叫得倪骨头才酥脱格哉!”兰芬听了,免不得粲然一笑,别过头去就坐在秋谷身旁。修甫等大家哄堂大笑起来,秋谷也忍不住笑了,却向贡春树道:“你的一身功架固然不错,但是见了一个倌人就要吊膀子,我看你也有些应酬不来。就如张书玉一般,到得大家吃醋闹出事来,你却又把一个头直缩到腔子里去,倒要卸到我旁人身上,替你们调停这一件醋海的官司。像你这样的人,真是那天字第一号的滑头码子。”说得陆兰芬好笑起来,抿着嘴笑个不住。春树无言可答,只得笑道:“你这般发急,敢是怕我割了你的靴腰么?我虽然是个滑头,朋友面上也未免有些不好意思,你只顾放心就是了。”

秋谷狂笑道:“我向来不怕剪边,你只要看中了兰芬,尽管自家去做,我若有了一毫醋意,就罚我做一个万世的乌龟,与现在的康抚台一样。你道如何?”这一句话来得突兀,把辛修甫等三人又招得大笑不止。好一会,方才渐渐的止住笑声。

修甫笑道:“现在有多少道台知府,翰林举人,拼着性命奴颜婢膝的在那里巴结着康抚台,惟恐不当其意。你却把他比作乌龟,还借着他来赌神发咒,若被那班大人先生们听见,直要把你当作个一生的切骨之仇。从来惟口兴戒,以后还是收敛些儿为是。”秋谷听到此处,不觉肃然拱手,对修甫道:“多谢良言,有逾金石。我章秋谷一生的吃亏之处,就是处处以狂态逼人,以致场屋文章不中主司的绳尺,清流议论每来朋辈之讥评,想起来真是有损无益。如今定当随处留心,学为谦退,庶几不负你劝我的一片热心。”说罢,大家嗟叹不已。

陆兰芬见秋谷有些抑郁的神情,便提起了精神殷殷勤勤的和他说笑。秋谷一面应酬,一面碰和,把那一腔的豪情胜概登时又提了起来。刚才是拔剑斫地,搔首问天,大有四海无家,前路苍茫之恨;如今却又是俯观山海,高见风云,又有那斗酒十千,红绡买醉的神态。

正碰着和,陆兰芬忽地问着秋谷道:“唔笃常州有一个姓方格客人,说俚是安徽格候补知府,耐阿认得俚格?”秋谷听了,初时想不起来,细细想了一会,方才想出是他。原来章秋谷原籍本是常州,后来因住在南京多年,所以入了金陵籍贯,直至秋谷丁了外艰之后,方才移到琴川。常州有几处祖坟,每年春、秋二季,秋谷必到常州祭扫一趟。前书中贡春树初到上海之时,也曾表过,按过不提。

只说章秋谷猛然记起这个姓方的客人,同秋谷向来认识,家中也有二三十万家财。自家本是个目不识丁的人,你就是叫他写封平常通候的书信,他也写不出来。

恰又有一样脾气,最怕人家说他不通,最喜要结交一班名士。从前章秋谷回来扫墓,住在贡春树家,不知怎样的被他打听着了,晓得章秋谷是个风流才子,当代名家,连忙自己先来拜会,又请秋谷吃过几次酒,算是和他接风。秋谷见他这样的屈意殷勤,情不可却,只是看着他的言谈卑鄙,举止仓皇,自头上看到脚边没有一根雅骨,真是个俗不可耐的人,无可奈何,只得勉强和他来往。现在听了陆兰芬问他的话,想起他来,便笑道:“不错,我认得这个人,可是一个瘦骨脸儿,长挑身材,名叫方子衡的么?你要问他作甚?”兰芬道:“照耐说起来一点勿错,一定就是格挡码子。倪前日仔有格姓方格客人,来叫倪格局,到金谷春去,勿然是倪本来勿去格,为仔有倪一格姓王格熟客替俚代叫,勿好意思坍俚格台。就是格日仔夜里向,格个方家里跟到倪搭,摆好一格双台,接下去碰仔两场和,直到仔两三点钟,天亮辰光走格。昨日仔又是双酒双和,今朝故歇辰光还朆来。倪看格客人瘟得利害,诧异起来哉,所以问问耐阿认得格个人,到底是那哼一个路道?”秋谷笑向兰芬道:“恭喜恭喜,又做着了一个绝好的户头客人。这个方子衡不比那个方幼惲,虽然也有些啬刻的性情,但他专要爱装场面。你若把他挤在面子上,叫他转不过脸来,就是一万八千也肯忍着心痛挥霍,可不是一个绝好的客人么?”陆兰芬听了,甚是欢喜。

忽见金小宝和龙蟾珠两人一先一后走了进来,招呼了几句话儿,各自坐下。

秋谷见他们局已到齐,止有花筱舫未曾去叫,便连忙把局条发将下去,却对兰芬、小宝说道:“今天我们这一席却不是专为碰和,其中另有一番缘故。”遂把昨夜在东合兴花筱舫家吃酒的情形说了一遍,“所以今天我想了一个主意:在此碰和,叫筱舫来代碰,要把他羞辱一场,出出胸中的闷气。特地把你们三个叫来,和花筱舫合成一局,恰好四人,候他动手之后,方才慢慢的问他为什么要得罪客人!看筱舫如何回答,然后将他的局帐当面开销,大大的给他一个没趣。但是还有一层说话,要先和你们说明,等回儿筱舫到了,你们大家不要睬他,若有人和他说了一句话儿,便是瞧我们众人不起。你们大家记着,千万不可理他。”

陆兰芬和花筱舫向来相识,颇是要好,听得章秋谷这番说话,暗暗心惊,便想要劝他几句,叫他不要顶真,少停等筱舫到来,赔个不是也就过去了。正要开口,见小宝把舌头一吐道:“耐格主意倒直头来得刁枭,区得倪无啥差头拨耐扳着,要是一格勿当心得罪仔耐,是耐也要想仔法子来翻倪格本哉嗫。”秋谷一笑,又道:“此刻花筱舫将近就来,你们快些坐下,不要耽误了工夫。”于是陆兰芬代了章秋谷,金小宝和龙蟾珠代了修甫、春树,合着陈海秋四人,慢慢的碰起来。

陆兰芬还想着要解劝秋谷,便叫着秋谷道:“二少,耐过来嗫,倪有两句闲话要搭耐讲笃。”秋谷便走了过来,还未立定,已见花筱舫进来,淡淡的向陈海秋叫了一声“陈老”。陈海秋只当秋风过耳,没有听见的一般,一声不应。花筱舫见陈海秋竟不答应,已经气上心来,腮边现两朵红云,眉际起几分怒色。秋谷见了,恐筱舫不肯坐下碰和,连忙过来含笑招呼道:“今天我们碰和,陈老特叫你来代碰,快些下去替他代碰两副,好和他转转色头。陈老的一底筹码输得差不多了。”一面说着,陈海秋已经立起身来。秋谷捺着筱舫坐下,筱舫见秋谷等三人都是叫局代碰,推辞不得,只得就碰起来。又招呼了陆兰芬一声,觉得陆兰芬冷冷的神气,似理非理的应了一声,花筱舫心中不觉有些疑惑,偷眼再看秋谷等时,神情之内,都觉有些奇异,陈海秋更是双眉微竖,勃勃的现出怒气来。

正在心中摹拟之际,只听得陈海秋对着陆兰芬等一班叫来的倌人高声说道:“你们大众都是上海滩上有名的红倌人,请你们替我评评道理。我昨日在花筱舫院中请客,闹了一肚子的闷气出来,你们堂子里头可有这样的规矩么?”便又把昨日要他吃酒的情节重说一遍。又道:“堂子里头的筋络,我虽然是个外行,但是比他再红的倌人,我曾见过无数,从没有见过这种样儿!难道他既然吃了这碗堂子里头的饭,还混摆他的什么架子不成?”花筱舫听了,方才心中明白,假说叫局,骗他来羞辱一场,明知他不能不去,想不到陈海秋有这样的挖掐心肠,只气得泪滚珍珠,花容失色,几乎要哭出来,这里陆兰芬便立起来,咬着秋谷的耳朵,说了两句不知什么话儿,秋谷点头不语。

又听陈海秋盛气向花筱舫说道:“你这样的红倌人,我姓陈的也高攀不起。我们花了银子,原是到你们堂子里来寻个开心,想不到你们吃把势饭的,居然竟敢这样的放肆起来!不要说是你这样半红半黑的倌人,就是比你红了十倍的人,也不能这个样子。你也把我当作曲辫子的客人看待么?”此时陈文仙房内鸦雀无声,大家悄没声儿的寂然静听。花筱舫早气得呆在椅上,就如木偶一般,那眼内的泪珠只是滚个不住。

陈海秋又冷笑道:“你的局帐料想不肯抄来,我自家倒还记得明白,共是二十三个局钱,三台菜钱,一共四十七块。”说到此处,向身边摸出一把洋钱,数了一数,望着花筱舫身边一掼,“豁啷啷”一声滚得满房都是,声音清脆,入耳异常。

海秋又大声道:“我也没有这样的工夫和你生气,你拾了洋钱与我快些出去。你是个上海第一的红倌人,不要坐在此间沾了我一身霉气!”

花筱舫听了,真是冤愤填胸,无门可告,要想发作,又怕陈海秋动起蛮来,吃了现亏。气到极处,索性把眼泪揩乾,霍地立起身,待要走出门去,早被陈海秋抢上一步,挡住房门,喝道:“你不把局钱带去,还要我叫人送到你的门上么?”直把个花筱舫急得坐又不是,立又不是,哭又不是,笑又不是,那一刻工夫的神景,一枝笔那里形容得出来!

秋谷见花筱舫十分惭怒,暗想:“就是这样,总算翻了本儿,若再过分羞辱他,非但恐怕一时间逼出事来,心上也觉得有些不忍。”便向陆兰芬使个眼色。兰芬会意,走到筱舫身旁,软软的携住筱舫的手,道:“耐也勿要生气,倪同耐到后房去坐歇罢。”又回头向陈海秋道:“陈老勿要动气,等歇倪再叫俚出来,销陈老格气性。”说着,便同了花筱舫一径往后房便走。花筱舫正在又急又气之际,巴不得躲过他们,连忙同着陆兰芬进去。陈海秋还要开口时,秋谷急急止住。修甫朝着秋谷把大拇指伸了一伸,低低说道:“主意果然甚好,只是陈海翁说话过分了些。”秋谷也觉略略带些懊悔的意思,想等花筱舫定一定神,去安慰他几句。

等了一会,只见陆兰芬移步出来,望着秋谷招手,叫他进去。秋谷便走进后房,见花筱舫满面泪痕,靠在一张榻上啼妆惨淡,鬓影蓬松,别有一副可怜的神态。兰芬见章秋谷进来,便低声向他说道:“倪刚刚问明白哉。耐也勿要怪俚一干仔,陈老自家格勿好。”秋谷诧问:“为什么倒是陈海秋不好?”兰芬对他告诉出来。正是:

春掩胭脂之泪,绿怨红愁;风欺薄命之花,飘茵堕溷。

欲知后事如何,请看下回分解。





第三十六回 说大话满口吹牛 摆双台安心落局





且说陆兰芬向着章秋谷细细的讲说,陈海秋初做花筱舫情形:陈海秋生长广东,平日最是性急,兼之初到上海,不懂堂子里的规矩,自从辛修甫将筱舫荐与海秋之后,刚叫了三四个局,就想住夜起来。筱舫的娘姨向他说道:“倪长三堂子里向格先生,比不得么二搭仔野鸡,总要碰几场和,吃几台酒,到仔是实梗模样格辰光,再好讲到住夜浪去。耐实梗性急,是勿成功格。”陈海秋听了娘姨的话,当夜就摆了一台花酒,连着碰了一场和,接连又吃了一台酒。陈海秋的心上,以为吃了两台花酒,筱舫一定留他。谁知花筱舫身价自高,非但没有留他,并且应酬之间也是随随便便的样儿,并不十分巴结。陈海秋见筱舫并没有留他住夜,心上就着实的不快活起来,说那娘姨有意哄他摆酒,又装着身分不肯留客。“难道你们做了这个生意,还要装什么千金小姐的身分么?”花筱舫听了又气又笑,晓得他是个外行,着实抢白了他儿句。陈海秋虽然听见,不甚懂得他们的口音,也就罢了。昨夜陈海秋又到筱舫院中请客,筱舫一肚子的不高兴,那有好气待他?又值海秋醉后一定要强他吃酒,所以闹出这一件花城香国的风波,也不能全怪倌人的不是。

章秋谷听了方才明白,不住的点头,果然这件事儿做得过分了些。又见花筱舫泪涴罗衣,眉颦翠黛,倒可怜筱舫起来,又劝他道:“这件事儿陈老虽然性急,你也冒失了些。但陈老是个外路客人,不懂堂子里头的规矩,你何不将这些情节向我们朋友说明,等我们再去劝他,便没有今天这一场糟蹋了。如今事情已过,不必再谈,你看着我的面情,不消生气,我去向陈老说明,叫他进来陪你一个不是,只当没有这件事儿可好?”

花筱舫明晓得今天这场冤屈是章秋谷暗中提调众人,却又无可如何,坐起来用手巾拭了泪痕,道:“谢谢耐,对勿住,总是倪自家勿好,得罪仔客人。难下转请耐二少照应点倪,陈老搭说句好话。”秋谷听了,暗道:“这两句双关话儿,倒也来得利害,竟像晓得是我的主意一般。”心中想着,口内胡乱答应一声便走了出去,附耳和陈海秋说了几句。海秋初时不肯,禁不得被秋谷一把衣袖拉住了,直到后房。

花筱舫正和陆兰芬并肩坐着,不知口中低声悄语在那里说的什么。见章秋谷同了陈海秋进来,筱舫登时扳起面孔,别转头去,低头向壁不发一言。秋谷向陈海秋努一努嘴,海秋会意,抢到筱舫面前,搀着他的手,道:“刚刚二少已经和我说明,这件事情恰是大家不好。我虽然性急了些,你也不消动气。看着二少的面情,不要放在心上。”筱舫并不开口,夺过手来赌气避了开去。海秋只得又走过来向他央告道:“我方才也是一时性急,现在有章二少爷从中劝解,是再好没有的了,你何必定要这样认真?”筱舫听了就如没有听见的一般,低着头看自己手中的帕子。秋谷见了,晓得自家在此不便,碍了他们的眼睛,向陆兰芬把手招招,两人一齐退出房外,只有陈海秋同花筱舫两人在内。修甫等见秋谷出来,争问怎样,秋谷不语,只指着后房把手摇了二摇。

好一会,方见陈海秋走了出来。秋谷便仍旧同着兰芬进去,把筱舫拉了出来。

花筱舫见了众人,不免面上红了一红,有些惭愧。兰芬见他不好意思,便把他拉到靠壁二张椅上坐下,二人哝哝唧唧的谈心。陈海秋取过一碗茶来,喝了半碗,把余下的半碗递在筱舫手中。筱舫正在说话,不及提防,只认是娘姨给他倒茶,顺手接了过去。及至回过头来一看,方知就是陈海秋,又见众人的目光一并注在他一人身上,不禁羞得他满面通红,把海秋啐了一口,自己也撑不住笑了。又道:“刚刚搭倪反末也是耐,故歇末也是耐,耐格人……”说到此处,顿了一顿道:“赛过是戏台浪格三花面,一时一样面孔,才做得出格。啥人来看耐呀!”说着又低头而笑。

陈海秋见他笑了两声,心中方才快活,秋谷也是欣然。

忽听得贡春树向秋谷笑道:“你自己常对人说,堂子里头玩耍万万不可认真,你为什么今天又认起真来?”秋谷笑道:“你这个人说出来的话儿真是不通情理!

我说不要认真,是遇事将就,不必挑他们的眼儿。若是倌人把我们当作瘟生,任情得罪,自然也要认真起来,难道真是和那一班马夫、戏子一般,专想他们倌人的倒贴么?“一句话,早又把个花筱舫说得面红起来。秋谷觉得,连忙用别话混了开去。

筱舫略坐一会,起身去了。陆兰芬等也陆续要走,秋谷叫住兰芬又说几句话,问到那方子衡身上来。兰芬道:“俚耐日日八九点钟辰光到倪搭来请客,一连请仔两日哉,今朝勿得知阿要来?”略谈几句,也就走了。

陆兰芬回到院中,果然那方子衡已在房中高坐等了多时,见兰芬回来,大喜道:“今天什么人叫你的局,去了半天。我等了有一点多钟,为什么到此刻才来?”

兰芬微笑道:“倪从前格熟客叫倪去替碰和,坐勒浪厌烦煞。刚刚今朝呒拨转局,只好替俚一直格碰下去。倪人末勒浪替俚笃碰和,心浪末勒浪牵记仔耐,晓得耐故歇辰光一定要来快哉。方大人,对勿住耐,等仔倪多化辰光。”说着横波展笑,眉黛生春,笑迷迷的朝方子衡瞟了一眼。这一个眼风,几乎把方子衡的三魂七魄都钩了出来。爱到极处,迷着两只眼睛看定了陆兰芬嘻嘻的傻笑。

兰芬见了心中暗暗好笑,故意走到方子衡身边立定,把一只纤手搭着方子衡的肩膀,低低问道:“耐今朝阿要请客嗄?”方子衡正在色授魂飞之际,见兰芬走至身旁,更加欢喜,张开两手想要趁势把陆兰芬搂入怀中。早被兰芬觉着,连忙把他的两手挡开,低声笑道:“勿要嗫!拨俚笃看见仔,算啥格样式介?”方子衡听了,只得暂时住手,虽然已是动情,却晓得陆兰芬是个金刚队里的出色人员,平日之间,将就些儿的客人绝不肯假借一些词色。

方子衡不敢冒昧,恐怕兰芬要发那红倌人的标劲出来,只好规规矩矩的和他说话。又问他方才叫局究竟是什么客人,陆兰芬依实回答,又道:“姓章格客人说搭耐向来认得,耐倒底阿认得俚介?”方子衡听了,想起章秋谷来,跳起来道:“果然不错,我认得这个客人!原来他也在这里,巧极了。”便一叠连声,叫快拿笔砚来写请客票头,一面又叫先摆台面。方子衡早把请客票头写好,就到兆贵里陈文仙家去请秋谷,又请几个别处的客人。不一会,客人陆续到了。

章秋谷在陈文仙院中尚未回栈,众人已经散去,接到了方子衡的票头,本想不去,回过念头一想,未免有些不好意思,便也随后到来。到得兰芬院内,方子衡直接到楼梯边来,呵呵大笑道:“章秋翁,幸会幸会。怎么你既到上海,竟不给我一个信儿?今天幸而兰芬向我说起,方晓得你在此间,为什么不肯通知朋友?停回却要罚你一杯。”秋谷无暇回答,只是含笑招呼。跨进房中,和那一班先到的客人彼此通了名姓,也有认得的,也有不认得的,恰好那金汉良也在座中,秋谷略道几句寒暄。

方子衡最是性急,连声叫快起手巾,自家提起笔来替众人写好局票,交代娘姨,彼此相将入席。金汉良叫的金小宝却第一个先来,见秋谷也在席中,似有诧怪之状,叫了一声,方走至金汉良背,竞不招呼,只把头略略朝金汉良点了一点,便自坐下。

金汉良见他叫的局第一个先来,他本来是个瘟生,只乐得他摆尾摇头,身子坐在椅上不住的摇晃,闭着眼睛口内咕噜咕噜的不知说的什么。猛然睁开眼睛,向席上众人说道:“这堂子里头的玩耍,虽然不算什么正经事情,然而也着实的有些讲究。不是我兄弟说句夸口的话儿,无论再是有些名气的倌人,但凡兄弟做的地方,比起别人来总要多占一分面子。你们众位请看,小宝这样的红倌人,兄弟去叫起局来,总是第一个先到。若不是他把我兄弟当做恩客,那里肯巴结到这个样儿?不瞒你众位老哥说,兄弟在此间堂子里头颇有些名气。”

金汉良正要再说下去,金小宝坐在后面冷笑一声,止住汉良的话头道:“金大少,耐倒慢慢叫,闲话说清爽仔。倪啥辰光做耐格恩客,耐倒搭倪说说看?就是叫个把局,倪有转局末来得晏点,呒拨转局末来得早点,阿是倪来得早仔点,就算做仔耐格恩客哉?倪倒从来勿晓得做啥格恩客,那哼末叫恩客,那哼末叫勿恩,耐倒讲拨倪听听看。倪堂子里向格客人多多花花,象耐金大少一样格客人也多煞来浪,倪要碰碰就做恩客,是也好格哉。耐格只嘴说起闲话来,真真呒拨仔格淘成,阿要瞎三话四!”

金汉良正在高兴,被金小宝兜头拦住,说出一番冰冷的话来,把个金汉良说得又羞又气,顿口无言。章秋谷见他那一副可笑的神情,早想起前日在四马路中见他坐在小宝轿内的那种怪相,忍不住别转了头不住的暗笑。其时陈文仙出局已来,坐在秋谷背后,见秋谷这般好笑,悄问为甚,秋谷附耳和他说那金汉良的可笑情形,陈文仙也格格的笑个不住,又恐怕金汉良见了疑心,将一方手巾掩在嘴上,极力忍住。

方子衡搳了两个通关,见客人的局已经到齐,便一个个细细的浑身打量。只见这一个是惊鸿顾影,那一个是飞燕惊风;这个是艳影凌波,那个是纤腰抱月。正是:

绛辱珠袖,花飞一面之春;雾縠冰绡,红涴桃花之影。

方子衡看看这个,看看那个,又回头看看兰芬,觉得他的姿态清丽绝人,脂粉不施,衣裳雅淡,丰神整洁,眉目清扬,那顾盼之间别有一种动人之态。方子衡看了一回,忽地向兰芬问道:“你为什么都是穿的素色衣裳,浑身上下没有一些红色,同他们那一班时髦倌人的装束大不相同,可是你平日间不爱浓妆,所以这般装束么?”

兰芬听说,不觉长叹一声道:“倪格闲话说起来,三日两夜也说俚勿尽。”说着,早眼圈儿红了,桃腮挹露,眉黛含颦,似有许多幽怨说不出来。

方子衡不知什么缘故,连连问他,兰芬方才叹口气道:“倪故歇吃格碗堂子饭真叫无法,说起来也是坍台。”就把他当初嫁了个姓张的客人,因他正妻妒忌,别租了一所小公馆和他同住。两下如何要好,怎样恩情。不料不到一年,姓张的生起病来,医治无灵,竟自死了,那时无可奈何。兰芬说到此间,那声音早呜咽起来,用手帕去揩那眼梢,好像要流下泪来的光景。停了一会,又说死了不多几日,正室天天吵闹,不容他住在家中,寻事生非,闹得翻天覆地,存身不住,只得出来重落风尘,再做这行生意。这也叫红颜薄命,无可如何。一面说,一面蹙额低头,盈盈欲涕,装得十分相像。又道:“倪故歇想起来,总是倪自家格命苦,张格勿死末,倪也勿会出来,所以倪格衣裳才是素格,头浪也勿紥红头绳,赛过搭俚穿孝,总算是倪心浪勿忘记俚格意思。”

方子衡听了兰芬一番说话,暗想:“堂子里头竟有这样的多情妓女!若把他娶回家去,倒是一个好人,料想不至于闹什么笑话。”方子衡心上打了这个主意,便看着兰芬,竟越看越好起来。陆兰芬的面貌本自不差,方子衡看了他,竟是个吴王苑里的西施,汉帝宫中的合德,差不多把今来古往见于传载的那些倾城倾国的佳人合将拢来,也比不上陆兰芬的丰格。这真是情人眼里出西施了。

且说章秋谷听了陆兰芬的说话,暗暗的赞他迷人的手段不差,看来这方子衡又免不得要入他的圈套,我们做朋友的人该应要把他提醒,免得他堕落迷途,方是道理。但是这方子衡一钱如命,也不是什么好人。平日间有些不得意的亲友要向他借贷些须,就如割了他身上的肉一般。凡是向他借贷过一次的人,从此他见了你的影儿望风远避,比那穷人见了债主还要惧怕几分。果然是“富人怕借,穷人怕债”,说得不差,章秋谷想到此间,那里还肯去管他的闲事?只预备着看他们的笑话罢了。

正是:

三千选佛,输他荀令之香;十斛明珠,难买罗敷之嫁。

欲知后事如何,下回分解。





第三十七回 真急色春宵圆好梦 假堂差黑夜渡陈仓





且说章秋谷走后,众客人陆续告辞。依着方子衡意思,今夜就想要住在兰芬院中,怎奈陆兰芬身价甚高,等闲不敢开口,又不好意思露出那性急的样儿。俄延半晌,已有三点多钟,兰芬催他走了。自此之后,方子衡天天在兰芬院中吃酒碰和,竭力报效,有时也遇秋谷在座,却只是冷眼看他。

光阴迅速,不觉一连已有十余天。方子衡见兰芬虽是待他甚好,却是落落大方,全没有一些儿女温柔的情态。方子衡忍耐不住,微微的露些仰慕的意思出来。兰芬听了只是微笑,并不回言。方子衡急了,捉个空儿私下向着兰芬再三央告。兰芬着实沉吟了一会,方向方子衡附耳说了几句。方子衡不懂,连忙问他说的什么。兰芬又向他说了一遍,方子衡虽已听得,但不晓得兰芬是个什么意思,仍是漠然。兰芬十分好笑,把方子衡推了一把,道:“耐格人啥实概介?”又拉着方子衡去坐在榻床上,两人对面躺下,兰芬方才低声说道:“耐心浪格事体,倪蛮明白来浪。就不过有一件,倪为仔格件事体,心浪向也转仔几化念头哉。”方子衡连忙追问他究竟为着何事,兰芬方才叹口气,道:“故歇倪格身体赛过是个讨人,说拨别人家听仔阿肯相信?倪来浪张家里出来格辰光,一榻刮仔带仔一个衣包,耐想呒拨洋钱,陆里好做啥生意?衣裳头面,搭仔房间里家生,样式才要拿仔洋钱去办,格末间架头哉啘。区得有两个娘姨相帮,搭倪掮仔三千洋钱带挡,难末总算将就过去。陆里晓得格两个娘姨掮仔带挡,格末叫讨气,拆仔利钱勿算,另外还要搭倪讲啥个拆头。

做起客人来,倪自家一点点作勿来主。些少客人面浪推扳仔点末,俚笃就要咕噜哉,说倪做生意勿肯巴结。倪末一径是老老实实格人,勿会勒客人身浪敲俚格竹杠,俚笃又要说倪夹忙头里向做起恩客来哉。真真叫哑子吃黄连──有苦无处说。倪总想生意好点,多点洋钱下来,拿俚笃格带挡还脱仔末好哉。刚刚格两节格生意勿好,差勿多单做一个开消,格末也叫无说法。方大人耐想想看,叫倪陆里好做啥客人呀!“

方子衡听了陆兰芬的一派花言巧语,竟自信了。暗想:“他自己不能作主,不过客人多费些银钱,也没有什么做不到的事。”便又欠起身来,偎着兰芬的粉面,问他道:“既然你这般说法,我便去把娘姨叫了进来,当面商议可好么?”兰芬不语,只点点头。方子衡又道:“虽然如此,但也要你自家斟酌一番,可有什么勉强之处?”兰芬听了,瞅了方子衡一眼,把一个指头指一指方子衡,又指一指自己的心口,然后斜溜秋波,嫣然微笑。方子衡见了大喜,连忙叫了娘姨进来。

娘姨阿金走进房中,兰芬急朝他使个眼色。娘姨会意,不等方子衡开口,就拉着他坐到床上,咬着耳朵讲了一回。方子衡好像有些不肯的一般,微微的把头摇了一摇。阿金出声笑道:“阿唷!方大人耐勿晓得,倪先生来浪上海滩浪总算有点名气,客人笃转起念头来,用脱仔三千二千直头无啥希奇,换仔推扳点格客人,俚就洋钱再用得多点,倪倒也勿放来心浪。勿瞒耐方大人说,用仔洋钱近勿到身体格客人,多煞来浪。倪刚刚说格闲话,不过绷绷倪自家格场面,勿是敲耐啥竹杠,耐方大人也蛮明白来浪。”几句话,已把方子衡说得暗暗点头。阿金又道:“耐方大人是有名格阔客,比勿得啥别人,倘忙就是实梗随随便便攀仔相好,勿要说倪先生坍勿落格个台,拨俚笃说起来,就是耐方大人面浪也无啥趣势啘。”方子衡听了点头称是。当夜无话,不提。

只说陆兰芬自和方子衡有了相好,竟教他把行李搬到自己院中。兰芬的房间本来甚多,腾出一间房间叫他住下。方子衡被兰芬哄得终日昏昏沉沉的,也不去理会别的事情。兰芬要他代买了一付珍珠头面,又是一付金钏臂,差不多也化了二千开外。兰芬趁着没有客人的时候,便来陪着方子衡殷勤说笑;也有时客人连连络络的不断,直到天明之后方始回房,陪着方子衡睡觉。

事有凑巧,忽一天来了两个住夜客人。一个叫陆小廷,是银行董事;一个叫余芹甫,是个当铺东家。同兰芬多是几年相好,性情极是豪奢,银钱更加挥霍,不约而同的先后都到兰芬院中。兰芬知道今夜推辞不得,权且把他们二人安顿在两处房中,一面应酬,一面要想打个两全其美的主意。想了一会,蓦然计上心来,走到亭子间,叫了娘姨阿金,附耳与他说了一回。阿金点头领会,兰芬走了出来。

其时已有十二点钟,兰芬便走到方子衡和余芹甫二人房内,略略周旋了一会,却向余、方二人说道:“今朝来仔一个过路客人,格末叫来得讨气,一定要勒倪搭借一夜干铺,倪又勿好叫俚勿借,耐来浪房间里向坐歇,勿要走。倪去仔转来有闲话搭耐说。”二人听了,自然如奉着纶音恩旨一般,那敢违拗?果然静悄悄的坐在房中。兰芬安顿了他们二人,款步出房去了。

约等有一点钟光景,忽然楼下相帮高声叫起出局来。楼上问什么地方,相帮说是后马路王家厍,楼上默然不应。余芹甫只道陆兰芬真要出局,甚是心焦。不料不多一会,兰芬走了进来,含笑说道:“格个断命客人来浪要困快哉,倪勿去管俚,阿要倪也困罢?”余芹甫道:“你不是要去出局么?”兰芬带笑低声道:“后马路倪勿去哉,脱仔局也无啥希奇,勿要倪去仔,耐一干仔勒浪等人心焦。”余芹甫听了,自然感激非常,相将就寝。那知睡不多时,楼下相帮又高喊起来道:“徐大人叫到老旗昌去。”兰芬故作嗔道:“深更半夜,来叫啥个断命堂差!惹厌得来。”

余芹甫慌问他老旗昌叫局可去?兰芬道:“姓徐个是倪搭老客人。俚耐叫格局,倒勿好意思勿去。”余芹甫默然;又问他几时回来,兰芬道:“说勿定,耐勿去末,倪定规早点转来。”芹甫听了又欢喜起来,点头应允。

兰芬略照一照镜子,急急的到方子衡房内来,故意对着方子衡抱怨道:“格碗断命饭,倪直头吃得来勿要吃格哉。倪刚刚堂差转来,老旗昌又来叫局,阿要讨气?”

在方子衡房内约有一点余钟,也不知他做些什么,临走却叮嘱方子衡道:“倪出局去转来,长恐要天亮哉嗫,耐定心点困歇。”子衡答应,兰芬瞥然去了。

到得将近天明,兰芬却仍到余芹甫房内。芹甫正在朦胧之际,被他惊醒,问道:“你可是刚刚回来?”兰芬点头,便又上床睡下。睡了一会,见芹甫已经睡熟,悄悄的踅下床来,不知何处去了。

芹甫这一觉,直到十点余钟方醒,睁眼看时,不见兰芬在床上,房内静悄悄的,便叫了兰芬几声,不见答应。只见阿金急急的走进来,问芹甫道:“余老爷要啥?”

余芹甫问他:“先生那里去了?”阿金道:“倪先生刚刚起来,勒浪梳头,阿要去喊俚来?”芹甫点头不语。阿金去了多时,方见兰芬云髻半偏,秋波饧涩。一面打着呵欠,慢慢的走进来。芹甫道:“时候尚早,你为什么要紧起来?”兰芬含笑道:“倪困勿着哉呀,难末起来去梳个头,听见耐来浪喊倪,倪头也朆梳,要紧奔得来看耐,啥勿困歇起来介?”芹甫道:“我店中有事,十二点钟一定要自家到店,现在已将近十一点钟,也差勿多了。”兰芬见他要走,知道他向来如此,并不相留,但道:“格末耐吃仔点心去,勿要饿仔肚皮,叫俚笃去叫仔一碗鸡丝面来阿好?”

芹甫点头。不多时叫来,娘姨送上,芹甫吃了匆匆而去。那边房内的陆小廷,七点钟已经回去。

兰芬一时打发了两人,原到方子衡房内,殷殷勤勤的陪着他。方子衡那里晓得兰芬一夜之内接了两个客人,依旧欢天喜地的照常相待。陆兰芬见他瘟得利害,便把自己的全身伎俩施展出来,把个方子衡骗得伏伏贴贴的,竟把他当作世界之内有一无二的好人,渐渐露出要娶他回去的意思。

兰芬听了,正中下怀,却故意不肯答应,向方子衡说道:“倪从前嫁仔格人,看看像煞蛮好,陆里想得到故歇再要出来做生意。倪吃格嫁人格苦,吃得足里足格哉,故歇倪想起来,再要去嫁人倒有点放心勿落。耐方大人肯讨倪转去,再要好也无拨。不过倪格两年生意勿好,亏空加二来得大哉,倪想再做两节下去,倘忙生意好点,还脱仔格亏空,格末再说到嫁人,阿是就容易哉。”

方子衡听得陆兰芬的口风推托,心上有些不快活起来,便道:“如此说来,你是不肯嫁我的了?”兰芬听了,慌忙问道:“啥人说勿肯嗄?耐格人末,一句闲话缠夹仔大腿浪去。倪要嫁人,像耐方大人一样格人勿嫁末,再要去嫁啥人?不过倪心里来里想,倪格亏空,故歇好像拖得重点,再做仔两节下去,阿好拨轻点亏空就好哉。故歇倪总算是自家身体,只要无拨仔亏空,倪拍拍身体跟仔耐方大人就走,阿有啥人来要倪格身价洋钱?耐方大人故歇就要讨倪转去,刚刚正是尴尬格辰光,多花几千洋钱,耐方大人自然是呒啥希奇,不过倪自家像煞有点意勿过。”

方子衡听了,沉吟一会,又问陆兰芬道:“你究竟有多少亏空,可有一万么?”

兰芬道:“一万末勿到,也差勿多笃哩。”方子衡道:“既是不到一万洋钱,料想我还开销得起,我来和你还清债务何如?”兰芬道:“耐方大人肯来搭倪开销,倪阿有啥勿要格道理?不过倪搭耐想起来,耐也勿犯着实梗破费啘。”方子衡听了不觉愕然,呆了一会,方问兰芬:“为什么犯不着这般破费?你这个话儿来得诧异,倒把我说得糊涂起来。”

兰芬忍住了笑,走过来,袅袅婷婷的坐在方子衡身上。方子衡看兰芬时,见他双鬟滴翠,高髻盘云。梨涡颊上之痕,低偎檀口;杨柳怀中之玉,醉倚纤腰。真个是花月为神,琼瑶作骨,把个方子衡看得骨软筋酥,刚才和他说的什么话儿,早一齐忘在九霄云外去了。兰芬低声说道:“勿是呀,耐就是一定要讨倪转去,倪有一个阿哥来里,大家也要商量商量,故歇热煞格天气,也做勿出啥格事体,索性让倪做仔一节,下节脱仔牌子收场,倪外势格局帐,也好去收收,多少收点转来,贴补贴补。故歇倪搭仔耐赛过自家人哉,耐少用一个铜钱,倪心浪好像快活点。晓得耐有铜钱人勿在乎此,省仔洋钱下来搭倪多创点物事末哉,瞎用脱俚做啥?方大人阿对?”方子衡听了,心上十分欢喜。

看官,方子衡虽然是个富家,但如今世上的情只有嫌少,那有嫌多的道理?况且他认定了陆兰芬是个有情的女子,兰芬的一番说话,又句句打到他心坎中间,那得不入他的罗网?有分教:

吹箫引凤,凄凉秦女之台;金屋银屏,辜负高唐之梦。

不知陆兰芬究竟肯嫁方子衡与否?请听下回分解。





第三十八回 还带挡做成圈套 订白头再捉瘟生





且说方子衡听了陆兰芬一番说话,非但不要他的身价,而且还替他打算省钱,心里喜欢得毛骨悚然,十分畅快。便问兰芬可要先付些洋钱,慢慢的还清债项。兰芬连连摇手道:“格末谢谢耐,勿要实概性急,就是娘姨笃面浪,耐也勿要说起,赛过无拨格件事体。倘忙一格勿当心,拨俚笃说仔出去,大家晓得仔,格是勿要说啥生意哉,连搭仔局帐一钱才收勿着,去便宜俚笃格排客人,也勿犯着啘。”

方子衡听了,觉得甚是有理,心中自是喜欢,但不免还有些儿不满之处,便向兰芬道:“你既是一心嫁我,何必定要多做一节生意?就有些局帐收不下来,我也不是这般啬刻的人,那有不肯代还的道理?况且你的身子已经嫁我,这些局帐自然要我包场,你又何必一定要替我节省呢?”陆兰芬听了,把眉尖一皱,颦蹙道:“耐格人啥总归实概性急得来,格个嫁人格事体,勿是一句两句闲话说得清爽格。

倪末也总算商量商量,耐末也自家想想,勿要就是实概妈妈虎虎,故歇倪格身体总归要嫁拨耐格哉,阿好再去接啥格客人?就是生意做到下节,不过场面浪实概说法,赛过嫁拨仔耐一样啘。“方子衡听了,方才放心。

兰芬见方子衡已经受了牢笼,这件事儿便有了二十四分拿手,正要乘着这个机会,狠狠的砍他一下斧头,还要叫他情情愿愿的报效出来,一毫不觉得陆兰芬是个敲竹杠的都头,砍斧头的名手。正是:

准备金笼关彩凤,安排香饵钓神鳌。

闲话休提,书归正传。忽一日陆兰芬院中来了一个客人,是阿金同来的熟客,兰芬却讪讪的不甚应酬,过去略坐了一回便走了出来,把那客人丢在房中,佯佯不睬。那客人坐了半天仍不见兰芬出来,心中未免也有些生气,起身要走,却被阿金拉住不放,急急的过来和兰芬说了,要他出去应酬。兰芬坐着不动,那里睬他?阿金见了这个样儿,不知何故,呆呆的立在旁边,见兰芬只当没有听见一般,忍不住又催一遍。兰芬冷笑一声,也不言语。阿金见连催了两三遍,兰芬只是不理,发起火来,也冷笑道:“做生意勿做生意,生来勿关倪娘姨啥事,倪阿好来管耐?不过耐挂仔牌子,客人来仔勿应酬末,做啥格生意介?”兰芬听了不觉面上一红,道:“个把客人,倪勿做末勿做哉啘,要耐生瞎巴结俚格啥?倪做仔生意,倒挨着耐格娘姨来管起倪来哉,阿要笑话!”阿金听了更加火冒,按捺不住,大声说道:“倪娘姨末娘姨,倒也三千洋钱笃哩,耐末是先生,倪末是娘姨,客人做勿做生来勿关倪事,只要耐拿格三千洋钱带挡还拨仔倪,格末随便那哼随耐格便,勿然末倪也有两句闲话勒浪说说。”陆兰芬听得阿金竟是顶撞起来,那说话的神情十分可恶,只气得蛾眉倒竖,粉面生红,把一双小脚在地下一跺道:“耐一塌刮仔三千洋钱带挡,啥格希奇勿煞,还仔耐格洋钱末,才完结哉啘,阿挨得着耐来瞎噪,嘤嘤喤喤,啥格样式!直头无拨仔淘成哉。”阿金冷冷的把手一摊道:“还仔倪格洋钱末顶好哉啘,倪有仔三千洋钱,阿怕无拨仔生意?勿要耐故歇末说得蛮好,停歇歇要起洋钱来原是无拨,格是定规勿成功格嗫。”

兰芬怒极,转向方子衡说道:“耐听听俚格闲话,阿要气煞仔人,二三千洋钱才拿勿出仔末,直头拨耐钝光格哉。”阿金呵呵冷笑道:“耐实概格红倌人,阿怕拿勿出仔洋钱,就不过还有倪经手格店帐好像勿少,耐倒记记明白,一淘交代仔倪,等倪去还拨仔俚笃完结,明朝等耐舒齐好仔倪来拿。”说罢,竟自走了出去,头也不回,自去回覆那客人去了。只把个陆兰芬气得呆了多时,一言不发。

方子衡婉婉转转的劝了兰芬一回,兰芬长叹说:“总归倪要仔俚笃格带挡勿好,耐看俚格样式,标得来,阿像啥格娘姨,赛过比仔本家再要利害,故歇倪也说得勿哉,想点法子还仔俚格洋钱,看俚阿再有啥格说话?”说到此处,便登时愁锁双眉,着实的踌躇起来。方子衡问他为什么这般着急?兰芬道:“阿金格带挡洋钱,倪答应末答应仔俚哉,故歇想起来,一时三刻,陆里拿得出几化洋钱?格件事体倒直头尴尬哉嗫。”方子衡笑道:“这些小事极是容易,何必要这般的着起急来,明天我就去打张票子来替你还了他的带挡可好?”兰芬摇头道:“耐勿要实概性急,等倪到别处借借看,倘忙无借处,再搭耐说。”方子衡诧异道:“前日我早已和你说明,替你代还债项,为什么忽然的不要起来?”兰芬道:“勿是呀,耐勿要缠错哩,耐搭倪还债末倪阿有啥勿要?耐搭格洋钱放来浪,总归一样格呀,等倪下节勿做好生意,再拨倪好哉。”方子衡听他说得有理,点头称是。

隔了一天,兰芬说是出去借钱,去了半晌,方才愁眉不展的回来。方子衡急问他可曾借到?兰芬拍手道:“无借处嗫,啥人肯借拨倪呀!倪问格客人要借五千洋钱,俚勿借倒也罢哉,陆里晓得俚说出来格闲话,格末来得讨气,俚倒说耐借得忒多哉啘,一借就是五千,叫倪陆里来得及”勿比三百五百洋钱,倪还好应酬应酬。

倪拨俚气婚哉,对俚说倪穷末穷,几百洋钱倒也勿在乎此,倪要老仔格面皮,问客人笃来借格三百五百洋钱,格是好煞格哉,难末倪一径跑仔转来,耐说阿要勿色头?

“方子衡道:”既然如此,我一准去划了票子来可好?“兰芬道:”难是生来只好问耐方大人借哉,不过耐方大人末,看仔几千洋钱无啥希奇,倪自家心浪意勿过煞来里。“

方子衡果然去后马路汇划庄上,划了一张五千洋钱的汇票来,交与兰芬。兰芬接在手中,低声笑道:“谢谢耐,倪今朝拿仔耐格洋钱,赛过就是收仔耐格定洋,故歇耐搭倪两家头……”兰芬说了半句,觉得似乎有些不好意思,两颊微红,回头匿笑。方子衡看了这种含羞佯笑的情形,浅逗轻挑的言语,只把他喜得眉飞色舞,乐不可支。

陆兰芬接了银票,便立刻唤了阿金上来,又从妆台抽屉内取出一叠发票,一一的算清。合起来连那三千带挡洋钱统通在内,竟有五千多些。兰芬又开了拜匣,取出几张钞票,一齐交与阿金,当面言明,从此两无交涉。又把阿金数说了一番,说他不该这样的全无义气,无缘无的和他吵闹起来。阿金银钱到手,并不计较,只冷笑两声,接过票子,收拾衣装,扬长去了。

这里兰芬便问方子衡道:“倪收末收仔耐五千洋钱,阿要写张借票拨耐?”一句话,把个方子衡说得哈哈的笑起来道:“岂有此理!难道我不相信你么?”说得兰芬也一笑道:“勿是呀,常恐耐勿相信,说倪骗仔耐格洋钱。”

自此以后,兰芬便和方子衡商量,要办红裙披风、珠花首饰,一切嫁人应用之物,估计起来也有三千开外。方子衡那里晓得兰芬不是真心,一味的拿出钱来任凭布置。兰芬因天气甚热,借着歇夏的名头不出堂差,夜间的和酒也就少了些儿。

方子衡忽然想起要坐马车,便向兰芬说知,要他同去。兰芬道:“一淘去也无啥,就不过倪去末总要带个娘姨,一部车子坐勿落啘。”方子衡道:“一部坐不下就叫两部,什么大不了的事情?”兰芬方才欢喜,叫相帮去雇两部橡皮马车。相帮去不多时,马车已是来了。方子衡便催着兰芬,叫他快换衣裳。兰芬将就洗一把面,略施脂粉,重整云鬟,换了一套衣服,越显得娇如解语,弱不胜衣,扶在娘姨肩上向方子衡笑道:“价末倪去哩。”方子衡只是讪笑,要让兰芬先行,兰芬不肯,道:“倪勿要呀,耐豪燥点走嗫。”方子衡一面笑,一面同着兰芬出门,上了马车。

马夫加上一鞭,跑开四蹄,径往大马路泥城桥一带跑来。

此时正是六月初天气,新月在天,明河倒影,碧天如水,萧然无云,已觉得心旷神怡,烦恼尽去。再过了跑马厅一带,无数的重阴密树,接干交柯,树阴之内漏出一角月光,那树枝的影儿不住的往来弄影,风飘翠袖,露湿罗衣,好像到了清凉世界一般。到了张园,方子衡和陆兰芬下了马车,就在草地上拣一张桌子泡茶坐下。

不多一刻,那班有些名气的倌人陆续到来,也有泡茶的,也有并不泡茶到各处去闲走的,内中有认得兰芬的倌人走过来招呼两句,兰芬含笑应酬。忽见随后又是一班少年客人蜂拥而来,在一班倌人的桌子面前走来走去,穿个不了,口内评头品足的恣意说笑。那班倌人也有背过脸儿不去理会的,也有打情骂俏兜揽生意的,更有和客人动手动脚扭作一团的。兰芬看不入眼,扭转身子向方子衡说道:“故歇格倌人真真笑话,耐看俚笃,当仔几几化化人做出实梗样式,阿要面孔?连搭仔倪格台才拨俚坍完格哉。”方子衡点头称是。

兰芬正在说话,忽然背后伸过一双手来,两手交叉,把兰芬的眼睛紧紧掩住。

兰芬不晓得什么人和他玩笑,待要发作,又恐是个熟人不好意思,发极喊道:“啥人介,勿要实梗噪嗫!”就这一声喊里,背后的人方才放手,哈哈的笑起来,兰芬急回头看时,原来不是别人,就是那章秋谷。兰芬见了,故意沉下脸来埋怨秋谷道:“耐末总是实梗无淘成,倪拨耐吓煞快,认仔是个流氓要拆倪格梢哉。”说着不禁也笑了,又反手摸摸头发,用豆蔻盒的镜子照了一照。秋谷随便坐下,招呼了方子衡。陈文仙随在秋谷身后,便也坐在一旁。

秋谷向子衡道:“多时没有见你出来,怎么今天居然有空儿坐起马车来了。你们贵相知竟许你出来么?”方子衡一笑,尚未回言,陆兰芬面上早不知不觉的红起来,睄了秋谷一眼,道:“耐末总无拨好闲话说,狗嘴里阿会生得出象牙?方大人出去勿出去,阿关得倪啥事?随便啥格闲话,到仔耐格嘴里向末就无拨仔淘成哉。”

秋谷正待再说,方子衡拦住道:“你们不要大家斗口,还是我们来谈谈罢。”就把椅子往前挪了一挪,低声诉说:要把兰芬娶回家去,可好托他做个现成媒人?秋谷听到此间,便把兰芬着实钉了一眼,兰芬低着头装着不见,自在那里和陈文仙交头接耳的密密谈心。秋谷等方子衡说完,方才笑道:“原来你就要纳宠,所以这样喜欢,我竟没有晓得风声,不曾和你道喜。但是你要我做个现成媒人,虽然极是容易的事情,这个媒人我却做不来的。”正是:

画中爱宠,难销金谷之春;天上兰香,一现昙花之影。

欲知后事,请看下回。





第三十九回 陆兰芬雨后试新妆 方子衡花前申旧约





且说章秋谷向方子衡道:“你要我做个媒人,我却不能答应。为什么呢?一则我向来没有经手过这些事情;二则在堂子里头讨个把倌人回去,老实说也用不着什么媒人,你们自家早已两下言明,这个媒人岂不是个多余的饭桶。”说得方子衡同兰芬都笑起来。

秋谷又道:“此时我不做媒人可担不着将来的干系,不要你们回来有了什么说话,又来寻起我来。”方子衡听得秋谷口风诧异,连忙问他将来好好的有什么说话?

秋谷微笑,正要回答,那边兰芬咳嗽一声,向秋谷递个眼色,似乎教他不要多说。

陈文仙坐在背后,更把秋谷的衣裳乱扯。秋谷不觉笑了一笑,转口说道:“不是别的,你们既然请了我做媒人,将来免不了有什么开销赏项,以及脱牌子的喜封等,狠是一件累赘的事情,你想我弄得来这个么?”几句话就把方才的情形遮掩过了,兰芬方觉放心。方子衡本来没有留心,那里估量得到他们的话中有话?便把这一层说话丢过一边。

方子衡问秋谷道:“明天你可有应酬?若是没有什么应酬么,明天我就在兰芬那里摆个双台,请你们多吃杯喜酒。”秋谷攒眉道:“多谢盛情,我却未必能到。

这样的热天,吃酒有什么味儿?我向来六月天气不去应酬,你还是另请了别人罢。“

方子衡听了直跳起来,嚷道:“岂有此理!我专诚请你,你竟不肯赏我的光,可是瞧我不起么?”秋谷尚在迟疑,经不得方子衡一定不肯,兰芬也在旁边说着,方才勉强点头。

秋谷略坐了一会,不耐久坐,霍地立起身来向方子衡道:“亏你们都有这样的耐心,呆呆的坐在此间有什么趣味,我天天到此一趟,总不过打一个圈子,若不是遇见熟人,一刻也不能久坐。”兰芬道:“难倪也要去快哉。”秋谷便用手搭着凉篷,四围一望,见自己的马夫正在前面,连忙招手叫他。那马夫跑来问道:“阿是去哉?”秋谷更不言语,只点一点头。马夫去不多时,便拉了一部橡皮两轮快车过来,停在草地旁边。秋谷指挥陈文仙,叫他先上车去,然后向方子衡拱手告辞,撩衣摸裳,耸身一跃,早坐在马车上面,回头向着兰芬微微一笑,飞个眼风,一手顺过丝缰,一手拔出鞭子,把鞭梢扬了一扬,马背上加上一鞭,那马跑开四蹄,电卷风驰,径往园外而去。顷刻之间早已烟尘滚滚,不见影儿,只听得远远的马蹄声响。

正是:

草软沙平,十里春风之路;香车宝马,一鞭陌上之尘。

陆兰芬看得出神,不由得口中喝一声彩,方子衡绝不理会,随后也叫娘姨去寻着了马车,一同回去。

次日,直睡到午后方才起身。梳洗已毕,差不多有两点余钟。其时正是万里无云,一轮赤日热得十分利害,流金烁石,鸦雀无声。兰芬房间内一齐都装着风扇终日扇风,那里解得这天中的烦热!不但方子衡热得走头无路,连陆兰芬也热得微微娇喘,汗透罗衣。正在无可奈何之际,忽见西北角上推起一片黑云,方子衡道:“好了好了,天上堆起云来,像是要下雨的光景。”就拉了兰芬同他坐到窗前去看。

果然那一堆云起,渐渐的移过来,移到天中,不知不觉的已把日光遮没。不多一会,就遮得满天都是乌沉沉的,就如晚间的天色一般,辨不出东西南北。兰芬看得有些害怕起来,拉着方子衡的手,道:“倪进去罢,怕煞个,看俚啥介。”

两人手挽着手正要进去,大风起于西北,汹汹涌涌直卷过来,就像那钱塘江上的潮水一般,有千军万马、金戈铁马之声自远而近,把楼上的几扇玻璃窗吹得互相撞击,砰訇有声。只听“豁啷”一声,早打碎了两块玻璃,吓得兰芬拉着方子衡,三脚两步的跑了进去。再看那天上时,风声怒吼,云气迷漫,愈觉暗得异样,差不多像大米的泼墨山水,满纸淋漓,天低如盖,那云昏雾暗之中隐隐约约的现出万道金蛇,周回乱掣。兰芬慌忙叫娘姨们去关上纱窗,话犹未了,又是一阵凉风吹进,吹得人毛骨悚然,然后电光一闪,霹雳一声,大雨倾而降。一班娘姨七手八脚的关上窗棂。霎时间狂风骤雨,把房屋震得岌岌动摇。兰芬素来胆小,最怕雷声,吓得伏在方子衡怀内,自己用两手紧紧掩住耳孔,又叫方子衡用衣袖遮护着他的头面,一动也不敢动。方子衡甚是好笑,只得两手揽住兰芬的粉颈,紧紧的抱着他。那窗外的雨一阵大似一阵,好似那匡庐瀑布,大海飞湍,白茫茫的一片,平空直泻下来。

夹着那闪闪烁烁的电光四周飞舞,直射入屋子中间,照得人毫发肌肤纤毫毕见。雷声又隆隆而起,轰轰隐隐不绝于耳,震得大家心骇耳聋。兰芬靠紧了方子衡,浑身乱战。好一会,雷声渐止,檐溜仍淙淙不绝。停了一会,渐渐的也小了。兰芬方才放大了胆,放开子衡立起身来。已经揉擦得脂粉模糊,云鬟散乱,连身上的纱衫裤子,也皱得不像样儿。兰芬走到着衣镜内端详了一回,自己也不由好笑,忙忙的换了衣裳,重新梳洗。

方子衡自己走到窗前,推开窗子向外看时,残雨未消,晚烛初散,尚兀自有些跳珠激浪的余势。再向天上看时,断虹明灭,霞彩满天,那天上的颜色就如用水洗过的一般,苍翠欲滴。约莫正是七点多钟时候,那林梢屋角之间,尚隐隐的有些薄雾,暝色四围,苍然欲合,早露出一钩新月,斜挂天中。这一阵急雨,把方才的暑气不知赶到何处去了。晚风吹袂,凉气袭人,当户披襟,开轩送爽,竟是深秋天气,那里像什么三伏炎天?方子衡心中大乐,便连声叫取笔砚过来,写了几张弯弯曲曲的请客票头。

正要叫人去发,恰好陆兰芬晚妆初罢,缓步走来。换了一身白罗衫裤,拖着一双湖色拖鞋,淡扫蛾眉,不施朱粉,只淡淡的点了一点唇上的胭脂,秋波送媚,巧笑多姿,娇如解语之花,皎若中秋之月。眉如远黛,八字斜描;腰似垂杨,三眠初起。加以云鬟耀眼,凤翼低垂,梳得竟没有一根乱发,夺目争光,只带着一支全绿翡翠押发,鬓边髻上簪着一排茉莉珠兰,妖艳动人,香风扑鼻,又夹着一种花露水的香气,十分甜静。灯影迷离之下,竟是花香人气一例模糊,好像兰芬身上有一道光华射到面前,把方子衡的眼光罩住,越看越不得分明起来。

看官听者,这样的一身妖艳,满面风流,就是那目中有妓、心中无妓的有名道学先生,到了此时也万万把持不住。何况这方子衡不过是一个公子哥儿,没有什么阅历,又是个头等瘟生,著名冤桶,那里逃得过这陆兰芬捉怪降妖的绳索、勾魂摄魄的兵符?

当下方子衡见了陆兰芬这一身打扮,不由的三魂七魄一齐飞出顶门,不知去向,一口气放了出去,几乎收不转来。正在那飘飘荡荡的时候,忽然觉得有一个人把他的肩膀乱推,方才把他推醒。回转头来,见陆兰芬立在身后,一只手扶在自家肩上用力乱摇,却笑得面红耳赤,腰都立不起来,趁势伏在方子衡背上,笑作一团。方子衡不知何故,冒冒失失的问了一声,兰芬更加好笑,笑了半天,方说道:“耐心浪想着仔啥格老相好哉?倪问仔耐几声,一响勿响,阿是朆听见?”方子衡听见,不觉自家也笑起来。兰芬又问子衡道:“吃酒末,晏歇正好来啘,啥格要紧得来,阿嫌忒煞格早仔点。”方子衡道:“趁着这一场雨后暑气全消,正好趁此摆起台面,略早些却也不妨。”兰芬听了,便叫相帮一面去发请客票头,一面摆好台面。

请的客人却是章秋谷第一个先到,刚刚走进房门,便笑道:“好大的一天风雨,一会儿就凉快了许多,真是一雨成秋,绝不是六月间的天气了。”方子衡点头道是:“我见今日比昨天更热,还怕你不肯赏光,不料天公凑越,下了这一场大雨,好像代我邀客一般。”

说话之间,兰芬也来应酬两句,不觉又谈起兰芬身上的事来。方子衡问秋谷道:“你看兰芬的为人何如?”秋谷听了,看着兰芬微微而笑,不发一言。兰芬正和秋谷并坐,连忙用金莲踹一踢秋谷的脚。秋谷忍着笑,答道:“兰芬的为人还有什么不好,待你也煞是多情,但是依我看来,吃惯了这碗堂子饭儿,恐怕做不来良家妇女,你道如何?”方子衡正在一团高兴的时候,巴不得要旁人帮衬,不料被章秋谷兜头浇了一桶冷水,心中大不为然,默然不答。陆兰芬却急了,叉口说道:“人家人末也是人,倪堂子里向末也是人,阿是吃仔堂子饭就勿好做人家人格哉?倪归格辰光,一班姊妹嫁人格多煞,故歇才是蛮好来浪,也朆出歇啥格花头啘!独剩仔倪一干仔运气勿好,嫁仔人再出来做格个断命生意,一径也朆碰着歇对劲格客人,故歇难得格方大人搭倪要好,说好仔要讨倪转去。耐二少是方大人格朋友,该应要照应倪点,方大人心浪有啥勿舒齐末搭倪说两声好话,勿壳张耐格二少爷好话勿说,倒说起倪格邱话来,耐阿对倪得起,也无拨该号道理啘,方大人阿对?”方子衡听了只是点头。陆兰芬说完了这一番说话,又暗暗的拉了秋谷一把,斜溜了他一眼送个眼风。秋谷料想方子衡已经堕落在情海中间,那里翻腾得起?此刻徒劳口舌,劝他也是枉然,便趁着兰芬拉他的机会,立起身来哈哈笑道:“算了算了。我通共讲了一句无心说话,把被你叽哩咕噜说了一大篇,难道我有心破败你们的好事么?”

兰芬也笑道:“耐自家勿好啘,啥人叫耐瞎三话四介。”说着又使一个眼色,把秋谷调至外房,悄悄埋怨他道:“耐格人末,直头少有出见格。别人末只有帮帮倪格腔,耐倒来弄倪格嘴舌,阿要讨气!故歇倪搭耐说明白仔,勿要去多说多话,阿晓得?”秋谷也笑道:“姓方的是我的朋友,我不提醒他一句,好像不好意思。”

兰芬嗔道:“耐再要说,姓方格又勿是耐同得来格客人,随便俚去那哼,勿关耐事,要耐去瞎说格多花啥?”秋谷听了也觉不差,只得点头答应,又笑道:“你要我不开口却也不难,我坐在这里,你朝我磕了一个响头,我便不露你的马脚。不然就要对你不起。”恨得个陆兰芬又气又笑,咬紧了牙齿,把他搡了一搡。秋谷趁势走进房去,回头望着兰芬咳嗽一声,急得兰芬远远的向他摇手,又合掌当胸朝他拜了几拜,似乎央告他的意思,章秋谷方才微微的点了一点头。兰芬放下了心,跟进房来。

方子衡问道:“你们同到外房说些什么?”兰芬一笑不答。秋谷道:“你们贵相知将我调到外房,不过要打听打听你的家世,并没有什么别的事情。”正说着,只见金汉良也高高兴兴的走进房来。随后客人先后都到,写了局票,起过手巾,方子衡邀客入席,陆兰芬亲身斟酒,甚是殷勤。

不多一会,相帮叫局回来,把金小宝的局票带回,放在台上,说:“金大少叫金小宝勿来,说谢谢哉。”众人相顾错愕,都看着金汉良的面色,看他说出什么来。

正是:知

落花有意,犹开半面之妆;流水无情,不逐胡麻之饭。

要知金小宝为甚不来,下回分解。





第四十回 蓝桥咫尺旧雨不来 芳草天涯王孙归去





且说金汉良叫了金小宝的局,小宝回说不来,方子衡也觉得十分诧异,多看着金汉良的面色,想着他下不来台,定要发作一场,重写局票去叫。不料金汉良不慌不忙,面上也没有一些愧色,竟是若无其事的一般,慢慢的说道:“我昨天在小宝院中,小宝这两日受了暑气,我就料他今日未必出来,果然今夜不能出局。这原是我自家不好,不应就去叫他。”众人不料金汉良说出这一番遮掩的话来,一个个十分好笑,却又不好说明,只含着笑看他的神色。

金汉良见无人应接,自觉脸上也有些发起热来,只得又向方子衡说道:“小宝的为人却甚是和平,没有一些时下倌人的习气。兄弟深晓得他的性情,他却也不把兄弟一定当做客人看待,差不多就像自家人的一般。所以他偶然有些差错之处,兄弟也并不怪他。今天他一定是撑不起来,才回了兄弟的条子。若换了别的时候,只要他勉强得来,兄弟去叫他的局,万没有不来的道理。”

方子衡虽然是个外行,然而毕竟是个世家子弟,终不像金汉良的草包,听了他这一派怯排场的说话也觉好笑。章秋谷更觉得胸胃中作恶起来,皱着眉头瞪了金汉良一个大大的白眼。暗想:这样的东西,怎么也到应酬场中现眼,亏他这般老脸,叫局不到,还说出这般混摆架子的话来!待要骂他几句,却想起来与自家无涉,不必去做这冤家,便忍住了,只在鼻子眼里笑了一声。

那金汉良不知好歹,索性把喉咙提高了一调,高谈阔论起来道:“不瞒你们众位说,金小宝在上海滩上是一个有名气的倌人,排在四大金刚之内。你们请想,要不是他色艺兼全,那里数得着他呢?兄弟此番到了上海地方,也不过要闹些名气,所以就做了小宝,没有再去做过别人。小宝的看承兄弟,也是竭力张罗,十分巴结。

论起小宝的为人来,虽然没有什么脾气,却总有些红倌人的性情,往往一个不高兴,免不得就要得罪客人。独有我做兄弟的到了小宝院中,无论如何烦恼,总是笑面相迎,从没有得罪过一句。“说到此处,又笑嘻嘻的低声说道:”就是攀相好的时候,也没有花费什么银钱,那许多要好的情形真是一言难尽。想众位在这件事儿之内都是些过来人,也用不着兄弟细说的了。“这一席话尚未说完,台面上的一众客人早已笑声盈耳。金汉良全然不觉,还在那里手舞足蹈的数说金小宝如何要好,那样多情。

章秋谷实在忍不住了,把桌子猛然一拍,哈哈大笑道:“金汉兄,你还认着金小宝和你真心要好,敢是在那里做梦么?你上了他一趟轿子,他就敲你四十块钱的竹杠,还说了你无数刁尖刻薄的话儿。这也还罢了,今天你好好的叫他的局,竟自谢了不来,上海地方可有这般规矩?你是小宝的恩客,尚且这般相待;那不是恩客的人,又当怎样?岂不更要受他的糟蹋么?他吃了堂子饭,要是这样的得罪客人,也不必什么生意了。金汉良兄,我倒有一言相劝,你既然不懂,不必满口胡吹,还是少说些儿为妙。这是我的金玉良言,你却不须动气。”

这几句话儿,把一个惯吹牛屄的金汉良说得顿口无言,羞得面红耳赤,那头上的汗就如荷叶上的露水一般往下乱滴。众人见了金汉良这般局促的情形,又听了章秋谷这样发松的说话,一齐哈哈大笑起来。笑得金汉良愈加着急,拿出手巾来揩了头上的汗珠,又不住的用扇子乱扇,看他那个样儿,好生难过,脸上一阵红、一阵白,忽然又逼得面皮紫胀,口内发起喘来,一刻之间,就露出许多怪象,最苦的是白白的被章秋谷这般打趣,不敢认真。众人笑了一回,毕竟方子衡是个主人,见金汉良急到这般模样,有些过意不去,朝着众人连连摇手,止住笑声。

金汉良过了老大一回,方才渐渐的回过两色,暗暗的切齿痛恨秋谷,却又无可如何,只得搭讪着向方子衡笑道:“既然小宝不来,我却没有别人可叫,台面上未免寂寞了些,只好借重方子翁和我代叫一个的了。”方子衡道:“也不必另外再叫别人,你看台面上的局已到齐,你自己拣个中意的倌人,转一个局过去不好么?”

金汉良听了,便四围看了一遍,见倌人、大姐、娘姨等挤得密密层层,却仔细看来,没有什么好的。只有章秋谷背后坐着一个倌人,约有十八九岁光景,柳眉贴翠,檀口含朱,妙丽无双,容华绝代,正在那里遮着扇子和秋谷密谈。金汉良暗想:这一定就是什么陈文仙了。却为方才被秋谷无故骂了一顿,不好意思转他的局。对面方子衡看了,已知其意,便唤秋谷道:“章秋翁,有人要转一个文仙的局,不知可肯割爱么?”秋谷失笑道:“奇了!倌人挂着牌子,无论何人都好叫他的局,怎么问起我来?难道我有什么不肯么?”回头对陈文仙道:“你只管坐过去就是了。”方子衡和金汉良大喜。不料陈文仙听秋谷叫他转局,登时沉下脸来,把身子一扭道:“倪一帮里向客人勿做两个格,耐末无啥稀奇,倪倒呒拨格号规矩。”秋谷一笑,金汉良又碰了一个钉子,连方子衡都不好意思起来。金汉良气得呆呆的,半晌不言。

还是方子衡怕他下不来台,叫兰芬去转个本堂局,坐在金汉良肩下。兰芬勉勉强强的去坐了一坐,仍旧回来。

方子衡见台面甚是冷落,便鼓起兴来,要摆三十杯的庄。陆兰芬不许,瞅了方子衡一眼道:“勿要实梗嗫,晏歇吃醉仔,倪搭是无拨啥人来浪替耐吃酒。”方子衡道:“我就一人独吃,不用你们替代何如?”兰芬也笑道:“倪勿要嗄。”就把方子衡手内的酒壶夺去。方子衡再三央告,陆兰芬只是不许。合席的人都笑起来。

章秋谷笑道:“我来同方大人讲个情儿,许他摆了十杯拳庄罢。”兰芬还不肯应,秋谷打着苏白笑道:“耐也就是实梗仔罢,勿要来浪做啥格生意经哉。”大家哄然又笑。兰芬听了,急把酒壶放下,瞪着眼睛,一手指着秋谷道:“耐格号人末,实头……”兰芬说到此处,自觉有些碍口,顿住不说。秋谷也忍笑无言。方子衡却不甚明白,只把酒壶取过来,先斟了五杯,便要和章秋谷搳拳。方子衡却却的连输五拳。兰芬咕噜道:“难生来等耐自家去吃,吃醉仔勿关倪事。”方子衡果然直着喉咙灌了五杯,便又去寻别人对搳。一时叫来出局的倌人,会搳拳的一齐出手。霎时间红飞翠舞,玉动珠摇,那手上带的金玉腕钏,互相摩击,铿锵作声。方子衡看了大乐,秋谷也微微而笑。丝哀竹急,履错钗横,红粉两行,金钗十二。方子衡左顾右盼,骇瞩流光。

正在乐不可支之际,忽见留在栈内的一个家人满头大汗闯进房中,后面跟一个信差模样的人,手中拿的像是一封电报。方子衡不觉呆了一呆。果然那家人走近面前垂手回道:“家内来了一封电报,不晓得是什么事情,请老爷过目。”就向那信差手中接过电报,递在方子衡手中,两人便退了出去。方子衡拆开电封看时,那知都是洋码,并未翻出,涂鸦书蚓的就如天书一般,一个字也认不得。便又叫了家人进来,要叫他带到局里去翻。章秋谷向他摇手,问陆兰芬道:“你们可有官商便览的历本么?”兰芬应声道:“有。”即叫娘姨取来,送在秋谷手内。秋谷向方子衡要过电报,一字一字的翻了出来。不多时早已翻好,取笔写出。秋谷略略一看,皱皱眉头并不言语,即便交与方子衡。子衡接过看时,只见那一张报纸上写着道:

上海名利栈方子衡,父病重,速回常,万勿迟误。铨。

方子衡看了登时变色,半晌说不出话来。众人看他神色惨淡,知道家中有了变故,一齐拥上前来看了电报,一个个闭口无言,默然相对。还是章秋谷道:“既是你令尊病重,你自然该应连夜赶回,这里如有什么不了的事情,我尽可代你料理,你也不必心慌。”方子衡听了,方才立起来道:“这个自然,好在我在此间没有什么大事,可以立刻动身。但是今天苏州的轮船已经开了,我想只好到轮船局去和他商议,单雇一只小火轮,一直拖带回去,你道好么?”秋谷连声道是。

陆兰芬听得方子衡的父亲病重,立时就要赶回,也吃了一惊,却一刻之间也想不出什么主意,只紧紧的拉了方子衡的手,看着他的面孔像要说话,却说不出什么来。章秋谷见他如此,料想他们一定还有什么体己的话儿要说,况且方子衡此时心思已乱,大家不好久坐,章秋谷第一个立起告辞,又淡淡的慰劝了几句,便先走了。

秋谷走后,大家也一哄而散,单剩了方子衡和陆兰芬二人。陆兰芬拉着方子衡同向榻床躺下,悄悄问道:“阿是唔笃老太爷来浪生病,叫耐转去?”方子衡点一点头。兰芬又道:“价末耐明朝阿走介?”方子衡道:“我想明朝一早就走。”兰芬着急道:“耐阿好耽搁一日。”方子衡摇头。兰芬便欠身凑到方子衡一边枕上,推开烟盘,脸贴脸的问道:“耐就要转去末,倪先起头说个闲话,耐阿是勿记得哉。”

方子衡又摇摇头。兰芬把一点朱唇凑着方子衡的耳朵,道:“耐倒底阿记得,说嗫?”

方子衡停了半晌,方才开口道:“我此时心上实在不得主意。你想家内来了电报,叫我立时回去,我此刻的身体还在上海,不能飞到常州,家内的情形现在也不知道怎样,叫我的心上怎生好过,那里还想得出什么主意来?你的事情,只好我下次再来的了。”兰芬听了,假作发极道:“耐实梗说起来,是耐来浪想搳脱仔倪,再讨别人哉啘。倪一句闲话说出仔口,总归是耐格人,好好坏坏搭耐来浪一淘,故歇倪生意末也勿做哉,大家才晓得耐要讨倪转去,耐倒想要搪脱仔倪,要倪下节再做格断命生意。耐想想看,倪再有啥面孔来浪上海滩浪见人?耐要倪随便那哼,倪总无啥勿肯。耐要搳脱仔倪,叫倪再做生意末,倪就是死仔,倪格魂灵也要寻着耐格!”

一句话尚未说完,已止不住泪流满面,宛转娇啼,春深眉黛之愁,红掩灵芸之泪,回眸掩面,悲不自胜,把个方子衡的心上搅得就如乱丝一般,又有些怜惜起来。究竟那老父的死生抵不得美人的情重,不知不觉的早把他父亲病重丢在一边,打叠起许多的软语深情,陪着笑面着实劝慰。兰芬一面把方子衡两手推开,一面还呜呜咽咽的掩面而哭,又道:“耐再要来骗倪,耐格闲话啥人来听耐嗄。”说罢又哭。

方子衡被他哭得柔肠百结,凭你如何解劝,只当作没有听见的一般。方子衡急了,勾着兰芬的肩项轻轻问道:“依你要怎么样呢?只要你说出口来,我总依你就是了。”兰芬听了,方才趁势慢慢的收住了哭声,却还口中咕噜道:“耐搳脱仔倪,倪是不过死仔末哉,也无啥希奇,只要耐自家摸摸良心,阿对倪得起?”方子衡只是讪讪的笑了两声,又问他究竟打的什么主意。兰芬不答。经不得方子衡千求万告的,勉强把他拉了起来,又用手巾替他拭干眼泪,兰芬方才,隆慢的说道:“依仔倪格心浪末,故歇就跟耐转去,不过倪搭再有几化债户勿曾开销,耐明朝就要转去,总归勿成功,叫倪陆里来得及?耐去仔又勿见得就来。倪过仔该节,下节定归勿做生意格哉。勿做生意末,住来里上海做啥?生来只好跟耐转去哉啘。倪想起来,勿如耐先转去仔,留一个当差格住来里倪搭,等倪舒齐好仔,同俚一淘到常州来,耐说阿对?”方子衡听了,觉得果然不差,心上十分欢喜,把那家内的事情一时间就撇在九霄云外,竟自携着兰芬一同归寝。

看官请想,方子衡起初接了家中电报,想要连夜赶回,总算他天良未泯;后来被陆兰芬两行珠泪、一片虚情,哄得他把一个病重的父亲也置之不顾,反和着陆兰芬两人同到温柔乡里,携云握雨起来。正是:

多情神女,飘烟抱月之腰;无赖襄王,暮雨朝云之梦。

欲知方子衡究竟何时回去,且听下回。





第四十一回 骂瘟生西楼惊好梦 唱骊歌南浦黯销魂





且说方子衡本来急欲回家,被兰芬灌了一阵迷汤,竟把一个病危的老父丢在家中,全没有一毫着急的念头,也不想赶回家去。他二人倒趁着雨后新凉,珍簟初铺,碧天如水,竟是鸳鸯并宿,翡翠双栖,春深玳瑁之床,香暖合欢之枕。陆兰芬更拿出全身手段,枕边软语,被底风情,说不尽的山盟海誓,倒风颠鸾,把一个方子衡哄得如入黄河之阵,如穿九曲之珠,千变万化,不可端倪,一个身子觉得飘飘荡荡的,说不出那心中的快乐来。

良宵易度,一刻千金,早又是红日满窗,晓风入户,窗外有许多鸟雀在那里钩辀格磔的群噪弄晴。方子衡和陆兰芬香梦初回,模糊未醒。方子衡睡在枕上,见陆兰芬睡意惺忪,春情满面,酥胸半露,星眼微开,那一种娇憨的态度煞是可怜。方子衡待要起来,却又踌躇不忍,把枕头挪了一挪,重复并头睡下。陆兰芬正要收服方子衡的心,见他如此,正中下怀,自然的软语喁喁,殷勤相对。他二人一个是秋娘未老,一个是季子多金,果然似漆投胶,如鱼得水,不觉重又霍然睡去。

看官试想,上海堂子里倌人,那一等勾魂摄魄的功夫可利害不利害?凭你有些主意的人,不落他的圈套便罢,若要落了他的圈套,就免不得被他们哄得个神志昏迷,梦魂颠倒,甚至败名失操,荡产倾家。古今来多少英雄才子,到了这一个色字关头,往往打他不破,英雄肝胆变做儿女心肠,辜负了万斛清才,耽误了一生事业,你道可怕不可怕?

闲话休提,只说章秋谷昨夜辞别了方子衡,仍到陈文仙家住了一夜。午刻起身,梳洗已毕,想到方子衡昨日接了电报,今天不知曾否动身,有些放心不下,要到陆兰芬处去看看他。文仙叫他吃了饭去,秋谷不肯,文仙再三挽留,秋谷只得坐下。

文仙知他爱吃雅叙园的京菜,便暗暗叫娘姨下去,令相帮去叫了几样菜、一壶酒来。

不多时已是来了,娘姨便一样一样的搬了上来。秋谷看时,见是一盆生拌腰片,一盆糟鸭,一碗虾子扁尖,一大碗生川火腿汤。秋谷皱皱眉头道:“为什么要去叫这许多?”文仙忙笑道:“阿唷!二少勿要客气,倪搭就是请耐勿到,格两样菜勿中吃格。”秋谷也不禁笑了。文仙自己过来斟酒,就坐在下首相陪。秋谷要文仙同吃,文仙因章秋谷是个极熟的客人,并不推托,却因天热不敢吃酒。恐怕呛坏了喉咙,只陪着秋谷吃了半碗饭。秋谷因急于要到兰芬院内去探望方子衡,随便吃了几杯酒就不吃了。吃了饭,洗一把面,穿上长衫急急到兰芬家来。

那知进了大门,一直走上扶梯,楼上相帮喊了一声,只有一个粗做娘姨走到楼梯边来招呼秋谷。秋谷一脚跨进穿堂,见两个大姐都靠在榻上打盹,静悄悄的不见一人。秋谷心中疑惑起来,想是方子衡已经走了。正要问时,两个大姐听得脚步声音走进客堂,晓得有客人来了,连忙揉一揉眼睛,一骨碌扒起身来,见是章秋谷,笑嘻嘻的低声说道:“二少!阿是看方大人格?方大人搭仔倪先生两家头才朆起来。

二少房里去坐嗫。“秋谷听了,更加诧异,随口问道:”方大人昨日没有走么?你们可晓得他几时动身回去?“一个大姐叫做巧宝的,抢先笑道:”方大人昨日来浪说今朝要动身转去,难末拨倪先生说仔一泡,方大人倒好格,听仔倪先生闲话,今朝勿转去哉。“

章秋谷听了,真是没头没脑,摸不着究竟是怎么一回事儿,暗想:“定是兰芬放出功夫,把方子衡迷住,要叫他慢些回去,好趁着这个机会大大的敲他一下斧头。

但是方子衡昨天说得明明白白的,要去单雇轮船连夜赶回家去,怎么忽然变起卦来?

难道为了一个陆兰芬,就连他自己的生身老父病在垂危也置之不顾?这岂不竟是禽兽的行为么?天下竟有这般奇事!可谓天下之大,无奇不有的了。“又自己心中转一个念头道:”方子衡虽不是什么好人,何至于丧心病狂到这步田地?大约是大姐听错了说话,以讹传讹也未可知。“一面心中盘算,一面走进房去坐下,又以心问心的想道:”此刻也用不着胡思乱想,少停等方子衡起来之后问他一个明白。如方才大姐所说的话果是真情,我不免要把他正言戒责一番,叫他及早回头,免得众人唾骂。如若执迷不悟,须要把他痛骂一场,从此与他绝交也不为过。“

正在心中思想,见一个大姐走进房来,巧宝随后踵至,揭开大床帐子低声叫唤。

方子衡毕竟心中有事,叫了一声便已惊醒,张开两眼便问什么事情。巧宝道:“方大人,朋友来哉,阿要起来罢,一点钟刚刚敲过哉。”方子衡听说朋友来看,已经一点多钟,自家还在高卧,不免吃了一惊;又有些不好意思,连忙坐起穿好衣服,跨下床来,把陆兰芬也惊醒了,朦胧问道:“啥要紧起来介?”方子衡还未回言,巧宝接口道:“辰光勿早哉,方大人有朋友来里。”兰芬听说,便也坐起身来打了几个呵欠。

这里方子衡跨到床下,见是章秋谷端端正正的坐在窗前,那面上的气色似乎有些不善,早又吃了一惊。原来方子衡许多朋友之中最是敬畏章秋谷,每每的方子衡有些错处,秋谷就要正言厉色教训起来,以此方子衡见了秋谷虽然十分爱重,却是如对师保一般。当下见了秋谷,自觉有些虚心,脸上讪讪的红了起来。彼此招呼过了,秋谷便问方子衡道:“你昨夜亲口向我说过,要连夜赶回,为什么直到今日还不动身,更兼睡到此时未起?你接了一封电报,倒也亏你放得下心。”说着就冷笑了一声。方子衡听了十分惭愧,口内支支吾吾的说道:“本要今日动身回去,但我身体之中着实有些不快,恐怕不得动身,大约要到明朝的了。”

秋谷听了,方才大姐的一番说话竟是真的,不觉大怒起来。秋谷本来性急,一时怒发,激得他满面通红,怒气横飞,双眉倒竖,高声说道:“你家内令尊病重,发了电报来叫你立刻回去,你却恋着一个倌人,连自己的生身父母都不放在心上。

你倒自家想想,天下可有这样的道理么?我与你虽然朋友,却不愿意认得你这样无父无君的人!我们从此讲明,彼此绝交,大家不认。我将来到了常州之后,还要把你们亲友请到当场,把你的荒唐地方和他们讲个明白,也好泄泄我一肚子的不平。“

说着怒气冲冲的立起身来要走。

方子衡虽然受了陆兰芬的骗局,毕竟天良难昧,自己心中也觉不安,如今被章秋谷突然骂了一场,却平空的把他提醒,羞惭满面,无地可容。又见秋谷立起身来往外就走,竟要与他绝交,连忙赶上前来,一把拉住衣袖道:“你的说话句句是金石之言!我如今自己深知愧悔,今天一定动身,只求你不要说绝交的话。”一头说着,想起他父亲病重,天良发现,止不住流下泪来。

秋谷方才的一番言语原是一时的愤激之谈,现在看见方子衡赶来拉住,又见他流下泪来,知道他真心愧悔,心中也是欢喜,便立住了脚道:“你既知改悔,今日就可动身。遥想你们令尊既在病中,不知怎样的望你回去,你还忍心在此稽迟?万一你迟到一天,竟抱了终天之恨,你抚心自问,可不成了个名教中的罪人么?”方子衡听了,更加毛骨悚然,浑身汗下,也没有什么别的说话,只是诺诺连声。

此时陆兰芬已在床上起身,不及与秋谷相见,掩至大床背后小遗。章秋谷责备子衡的话,也被他依稀听见,只是不甚清楚,大约是催他回去的意思。好在昨天晚上已经两面说明,方子衡答应留下五千洋钱和他还债,并留一个家人名叫刘贵的,住在兰芬院中。一过秋节,候陆兰芬把上海的事情料理清楚,便同着刘贵一起同到常州,为的是留下一个家人,一半好监押着他,叫他不能翻悔的意思。所以兰芬听得秋谷要催逼方子衡回去,并不十分着急。

当下兰芬在床后走了出来,云鬟散乱,玉体慵抬。秋谷见兰芬出来,瞅了他一眼。兰芬便低下头去,叫了秋谷一声,问道:“二少,阿是催方大人转去?”秋谷点一点头,随口说道:“你可肯放他回去么?”兰芬面上一红道:“笑话哉,方大人屋里有仔病人,生来该应早点转去,阿有啥问起倪来哉?倪阿好叫俚勿要转去?”

便把方子衡的衣袖一拉道:“耐自家说哩,阿是倪来浪叫耐勿要转去?”方子衡默然不言。秋谷一笑,便打断他的话头道:“现在长话短说,你既然今天要走,料想趁搭轮船是来不及的了。我却有个认得的人在船局内,我和你写张条子知会一声,叫他代备一号小火轮一直开到常州,立刻生起火来,上灯时候就可登舟。我同他向来认得,价钱里头料想不至吃亏,你道好么?”方子衡此刻被章秋谷数言提醒,想着他父亲的病不知怎么样了,心上边焦躁异常,归心如箭,听了秋谷的话,拱手致谢。

秋谷果然立刻写了一张条子,叫了方子衡的家人上来,令他送去。兰芬却向方子衡说道:“章二少搭耐说格闲话句句才是好格,耐听仔俚格闲话早点转去。倪是早晏点总归是耐格人,勿要牵记仔倪,误仔耐格事体。倪事体舒齐好仔,马上就到常州,耐放心转去末哉。”方子衡听了也不言语,秋谷却甚是诧怪,正要问时,方子衡拉了秋谷过来,请他坐在炕上,把兰芬昨夜的言语告诉一番,又说现在留下一个家人同他回去,但终怕倚靠不住,要请秋谷代他料理一切,过节之后,把陆兰芬一直送到常州。秋谷连连摇手道:“这样事情,我向来不能料理,就是我自家的事也还要转托别人,那里办得来这样的肐瘩帐?你们既已两下言明,又有一个家人在此,料不至于有什么意外的事情,你难道信不过兰芬的话么?”方子衡听秋谷不肯担认,也只得罢了。转过身去,和陆兰芬轻轻悄悄的说了许多密语,又开了箱子取出一只洋漆嵌螺甸的拜匣,在拜匣内不知拿了些什么交与兰芬,兰芬欢天喜地的接了过去。章秋谷在榻上横着,远远看他,虽没有看见是什么东西,心中早已十猜八九。

恰好刚刚到船局去的那个家人走了进来,呈上一封回信。秋谷拆开看时,大略说轮船已经代备,刻下正在生火,就泊在本局码头。价目一层,彼此至交,不能多要,照着自己的本钱核算,并不多赚一文,共合八十块洋钱,连轮船酒钱统通在内。

后面又说令亲如有急事,八点钟即可开行的话。秋谷看了,把信递与子衡,叫于衡也看一遍,道:“八十块钱虽然并不吃亏,却也不见十分便宜。”方子衡看了拱手称谢,便叫家人先去收拾了行李衣箱发下船去。兰芬因方子衡尚未吃饭,便去叫了几样菜来。方子衡邀秋谷一同吃饭,秋谷因先已吃过,推辞不用。方子衡却草草的吃了些儿,只觉得心中好像有千头万绪,一时说不出口来,不知道腹中是饥是饱,将就吃了半碗饭,也辨不出什么味儿,只紧握着陆兰芬的手,你看着我,我看着你,说不尽的那一种缠绵宛转的神情。兰芬更是两只眼睛水汪汪的含着两眶眼泪,不则一声。秋谷看了暗中好笑,想他们堂子里头的妓女惯会做出一番的假意虚情,但是到那要紧时候居然迸得出一付急泪,也算亏他。便催促他道:“现在已经不早,你还是早些上船的为是。”方子衡听了,只得硬着心肠要走。兰芬把脚儿在地下一跺道:“慢慢交哩,倪还有闲话来里。”方子衡又立住了,眼睁睁的看他,兰芬低声叮嘱了几句,方子衡连声答应,兰芬方放了手。方子衡硬着头皮走了两步,又回过头来看看兰芬。兰芬直送下扶梯,秋谷也同到门口。方子衡一步一步的挨出大门,兰芬立在客堂门口,还说道:“倪格闲话耐勿要忘记脱仔嗫。”方子衡回头答应。

秋谷也说了几句套活道:“论理我要送到船上,我们还可谈谈,但是你此番回去是急如风火的事情,就是到了船上也不得畅谈,还是出来再见罢。”方子衡也谢了一声,彼此一拱而别。

秋谷立在门前,看他坐上马车电卷风飞的去了。秋谷便回上楼来,想要和兰芬说话。走到房内,见兰芬刚刚坐下,见了秋谷进来,不觉向他一笑,展齿嫣然。正是:

惆怅银屏之梦,青鸟难通;荒唐云雨之踪,玉人何处。

欲知兰芬如何说法,但听下回。





第四十二回 吃大菜粲花生妙谑 错房间无意遇名姝





且说章秋谷见陆兰芬向他一笑,便也笑道:“你骗客人的功夫果然不错!偏偏两个姓方的都被你骗得死心塌地,吃了你的空心汤团。怪不得你说常州来的客人都是一班土地码子,这班人却也实在瘟得利害,竟是一些不懂的东西。若要换了我做你的客人,就要对你不起。”兰芬听了,嗤的笑了一声,把秋谷背上打了一下道:“难阿好谢谢耐,勿要去多说多话。倪一径待耐勿曾错歇,就算仔耐是老白相,也勿犯着替倪做个格冤家啘。倪做仔生意,生来才靠两个客人,像俚笃格档码子,敲仔俚格竹杠,俚笃也勿晓得啘。”秋谷倒被他说得无言可答,略坐了一会便回栈去了。兰芬这边按下不提。

只说章秋谷走出陆兰芬家,觉得无事可做,信步掠去,意思要到新马路辛公馆去看看修甫,先到西安坊龙蟾珠家,去问辛老爷可在院中。刚刚凑巧,辛修甫竟在里面,却是方才走到,坐未多时。秋谷大喜,款步登楼,与修甫相见坐下。龙蟾珠也走过来应酬两句,穿着一身湖色洋纱衫裤,内衬妃色紧身,梳一个懒妆髻,发光可鉴,兰气袭人,簪着几朵珠兰,不施脂粉,不衫不履的样儿,打扮得甚是雅素。

秋谷见了,喝一声:“好!直头出色。”龙蟾珠微笑说道:“倪是勿好格,就不过为仔天热,衣裳着得清爽点,有啥格好嗄?”

秋谷却不理会他说的什么,转向辛修甫说话,又把昨天方子衡接着电报的一段故事,以及他自己今天责备的话儿,一一的向修甫说个明白。修甫又笑又叹道:“这方子衡被你骂了一场,居然还晓得自家惭愧,究竟还算是个好人。陆兰芬这番举动,大约又要借他淴一个浴。但是我真不懂,如今世上那里来这许多痴子,情情愿愿的供给他们,难道这班人都是没有心肺的么?”大家笑了一会,秋谷道:“这些花柳场中逢场作戏的地方,自然免不得花费。但是另有一层道理,也不必一味奢华,凡是面子上的银钱,这是自家的场面,不妨多出些儿;若是塞狗洞的地方,你就是花了一万八千,好像丢在水里一般,响声也没有一点,这样的银钱却万不可出,非但闹不出名气,而且还被他们当作瘟生。总而言之,场面上的银钱不能不出,塞狗洞的花费尽可无须。这却要做客人的自家斟酌,只要看准了嫖界的方针,便不至误落倌人的圈套。若要一毛不拔,和他们斤斤的计较锱铢,那就还是不嫖的为是,免得闹出笑话来。”修甫听了,点头叹服。龙蟾珠也在旁边听着,默然不语,若有所思。忽然目不转睛的注视秋谷,两边颊上渐渐红晕起来。秋谷一眼瞧见,微笑一笑,倒反背过脸去。

修甫便问秋谷:“今晚没有应酬,我们到一品香去可好么?”秋谷点头道好。

便邀蟾珠同去,蟾珠也答应了。秋谷道:“我们两人先去,你随后坐了轿子就来。”

蟾珠点头。章秋谷便和辛修甫出门先走。出了西安坊口,路上的马车、东洋车连络不断,那车声就如雷响一般,隆隆不绝。二人慢慢的沿着马路走到一品香,上了扶梯。因龙蟾珠尚未到来,恐怕他找寻不着,便就在扶梯旁边第五号房内坐下。侍者送上茶来,问可要请客。秋谷想本来人数太少,便取客票,写到迎春坊金小宝家去请贡春树,连小宝也请在里头,又写了龙蟾珠、陈文仙的两张客票,便叫细崽去发。

那侍者刚刚出去,已另有一个人引着龙蟾珠进来,便叫回先前的细崽,把西安坊的一张抽去,一面便先点起菜来。秋谷点的是鲍鱼汤、铁牌鸡、炸虾球、牛奶冻四样,又点了一客樱桃梨。修甫也和秋谷一般,只换了一个鸡绒汤,添了一样咸牛舌。秋谷又叫蟾珠点菜,蟾珠只要了鲍鱼汤和樱桃梨两样,都是吃不饱的东西。秋谷不由分说,替他添了一样禾花雀,又叫侍者先开两瓶冰冻荷兰水上来,并拿了两瓶皮酒和两杯克力沙,一齐放在桌上。

秋谷先举起一杯荷兰水来一口气吃个干净,觉得一股冷气直透心脾,其凉震齿。

龙蟾珠在旁调笑他道:“二少,耐当心点格好,晏歇点吃勿消格嗫。”秋谷一笑,又取过一杯来向龙蟾珠说道:“你不要寻我的开心,且先顾着你自家再说。若是你昨夜没有这般如此,你就做个好汉,把这一杯冰水吃下腹中,不要推三阻四,我便佩服你是个好的。”蟾珠红着脸道:“啥格实梗?实梗倪是勿晓得格,耐倒泌拨倪听听看。”秋谷大笑道:“你一定要我演说出来,我却没有这般福气。”用手把辛修甫一指道:“只好你们两个试法试法,看是如何?”说得蟾珠脸上更加红了,啐了秋谷一口,别转了头,忍不住笑道:“二少爷,倪一径搭耐规规矩矩,今朝啥高兴得来,单单来浪寻倪格开心,阿作兴实梗格。”秋谷笑道:“你昨天晚上若是干干净净的,我说我的话儿不干你事,为什么要你这般着急?一定你有了虚心的毛病,我的说话刚刚枭着了你的痛疮,所以着急得这个样子。”一句话把龙蟾珠说得当真发起急来,把面孔胀得通红,十分腼腆,口中咕噜道:“好好里一句闲话,拨耐说得来加二无拨仔淘成哉。真真歪嘴吹喇叭──股邪气。耐说格闲话倪一塌刮仔勿懂,随便耐去说啥末哉。”

秋谷见他急得面红头赤,更加狂笑起来。忽见贡春树携了金小宝同走进来,春树开口笑道:“你们为的什么事情这般好笑,可好分些给我笑笑么?”修甫也笑着把方才章秋谷和蟾珠斗口的话说了一遍,春树、小宝齐笑起来。正在笑得热闹,陈文仙也走了进来,笑道:“俉笃啥格事体来浪好笑,倒闹忙笃啘。”秋谷便叫他们坐下。贡春树也点了五样菜,又和小宝、文仙点了几样,都是大同小异的,差不多。

把菜单交与侍者,一面先喝起酒来。

这三人都是年少风流、倜傥自喜的人物,芝兰结契,金石同心,高见古人,俯视流辈,自然谈得十分契合,水乳交融。更兼各人带了相好坐在一起,一个个明眸皓齿,粉颈纤腰,媚态旁生,妍容侧聚,更是心上快然,毫无拘束。

正在豪饮雄谈之际,忽听见一个绝清脆的喉音,呖呖莺声在门外问道:“啥人叫格嗄?阿是该搭介?”秋谷等方在诧异,已见一个倌人扶着一个大姐,约有十七八岁光景,轻移莲步走进门来。秋谷举目看时,只见他腰肢纤小,态度安详,面如春晓之花,眉画初三之月,明眸善睐,一顾倾城,暖玉凌波,双弯贴地,云光外露,秀气内含,浑身上下竟有一道宝光射将过来,不由得心迷目眩。那倌人走进来见一个也不认得,知道认错了房间,回头一笑便欲退出。秋谷见陈文仙朝他点了点头,想是向来认得。又听见那倌人问道:“该搭阿是六号嗄?”文仙道:“该搭是五号,六号来浪隔壁。”那倌人便回转身来,又向着众人一笑,方才走了出去。

秋谷看他走出房门,连背影都不看见了,方回过头来说道:“不意风尘中竟有这般人物,我们为什么竟没有看见过他?”便问陈文仙道:“他和你说话,想是你认得他么?”文仙掩着嘴格格的笑道:“阿是耐看中仔俚哉,等倪来替耐做媒人阿好?勿要连耐格眼睛带仔隔壁房间里去。”说得大家都笑起来。秋谷问叫什么名字?

文仙道:“俚叫王佩兰,就勒浪兆贵里,本底仔倪也勿认得俚,有转把台面浪碰着仔,难末认得起格,头俚搭倪讲讲说说,倒蛮要好。俚自家说一径来浪苏州仓桥滨做生意,为仔苏州生意勿好,难末到上海来,故歇到仔勿多两节,还是该节调到仔倪兆贵里来。耐看看俚阿中意嗄?”秋谷听了笑而不答,便取过客票写了一张请吃大菜的票头,叫侍者送到隔壁房间请王佩兰。

不多时,王佩兰竟是姗姗其来,笑道:“洛里一位大少姓章?”秋谷尚未回答,文仙朝着王佩兰将秋谷指了一指,又将秋谷身旁一把椅子拖开,王佩兰会意,便走向秋谷身旁坐下,含笑不言。秋谷却打着苏州白,向着王佩兰笑道:“阿唷!先生时髦得来,跑进来赛过一只电气灯。”王佩兰也笑道:“阿唷,章大少客气得势!

倪是勿好格呀,陆里说得着时髦倌人?章大少来浪寻倪格开心哉!“秋谷连说勿要客气,口中在那里随口应酬,眼内却仔仔细细的把他自头至足看个尽情,果然是比玉生香,如花有韵,丰姿婀娜,骨格轻盈,心上十分欢喜。回头再看陈文仙时,珠光照彩,艳影惊鸿,太真出浴之妆,西子捧心之态,和王佩兰比较起来,却也不相上下。但细细评论两人的丰格,又觉得各不相同:陈文仙是一身的爱好天然,清华都丽;王佩兰是一派的妖娆荡逸,意气飞扬。看起来还是陈文仙较胜一筹,绝不是王佩兰那一种专取轻佻的模样。章秋谷在这边细看佩兰,王佩兰也在那边细看秋谷,见他丰神跌宕,气宇端凝,眉目之间别有一种英爽之气,回眸顾盼,丰彩动人,潘安仁逸世之姿,卫叔宝羊车之度,就是旁座的两个客人也觉得气概非常,仪容出众。

王佩兰看了多时,满心欢喜。秋谷叫他点菜,佩兰推道:“倪刚刚吃过夜饭,吃勿落来里,章大少请慢慢交用末哉。”秋谷见他不吃,也不相强,只寻些话说来引动他,又问他儿时到的上海,生意可好,王佩兰见秋谷问得殷勤,也不觉亲热起来,一一回答,也回秋谷几句,竟密密的谈起来。陈文仙见了免不得有些醋意,但是不好意思放在面上,只神色之间默然不悦。

秋谷和王佩兰谈得正是投机,那里去理会到陈文仙身上?倒是辛修甫寻些话与文仙兜搭兜搭,文仙也只得含笑应酬。贡春树忽问秋谷道:“我有一个手卷要你做一篇序文,随便什么体格,四六骈体不拘,就是散体也好,你可有工夫么?”秋谷皱眉头道:“我于文字一道,荒疏已久,你偏要和我歪缠,放着辛修翁这样有名的一个古文大家不去请教,可不是有心要我献丑么?”春树道:“就是辛修翁我也放他不过,明日我把手卷取来你看,笔意狠是工致,就请你们二位赐题。”辛修甫谦让了几句。秋谷问春树是什么手卷?春树道:“就是苏州那一个的小照,我新近托人钩了下来,另外补些花木,我自己的小照也一同画在上边。”

秋谷听了,方才想起春树初到上海时托他的一番说话,便道:“你一定要我和你做篇序文,也未始不可。但我平日的性情,向来不肯题诗跋画,学那班斗方名士的行为,或者我替你做一篇四六,仿着《玉台新咏》的体裁,直叙你们的事迹不好么?”春树道:“你肯做篇四六,是再好没有的了。我多时没有请教你的骈文,觉得数日不见珠玉,顿令胸中鄙念复生。别人的四六骈文未尝不清华绮丽,但是看起来好像总没有你的来得熨贴,虽然外面看去平淡无奇,却是格律谨严,一字不能移动,也不知是个什么缘故。或者我的见解与近时的名士不同,所以看了他们的文字,终觉得格格不入。何以我看了古人的文字,那见解又和别人差不多呢?这我就想不明白了。”说得章秋谷狂笑起来道:“这是他们的文情古奥,你看了,一时间解说不来,你要将来中了进士,点了词林,就懂得他们的文字了。”修甫和春树都不觉好笑。

金小宝等一班倌人在旁听着,一些不懂,见他们大家好笑,认是说笑他们,小宝把一张樱桃小口撅得高高的,口中说道:“唔笃来浪说啥?阿是笑倪?倪勿来格?”

说得三人重新又笑起来。这一笑不知不觉的菜已陆续完了,侍者呈上一篇帐来,夹着一张鉴字纸。秋谷看帐时,只得五元几角,甚是便宜,当下照着数目签好了字,大家起身。

秋谷又向王佩兰说了几句套话,佩兰乘机要约秋谷去院中小坐,秋谷应允,说少刻就来,佩兰便先走了。这里辛修甫同着贡春树先下楼来,见门前有一堆人在那里嚷闹。听不出是什么事情。两人连忙走到门口看时,见门外停着一部极精致三湾头的包车,漆得十分光亮,点着一对药水车灯,闪闪烁烁的耀得人眼都睁不开来,车上外国纱绣花围垫一色簇新,那轴上车沿包的都是银錾起花的什件。正是:

忽遇玉台之选,名士倾心;惊逢狐兔之成,小人得志。

欲知后事如何,但听下回分解。





第四十三回 章秋谷痛骂无耻奴 王佩兰暗吃山西醋





且说贡春树同辛修甫走到一品香门口,见停着一辆包车,却不晓得是何人吵闹,便急急的走出门外看时,只见一个年少车夫,十分精壮,头上戴着一顶极细的外国窄边草帽,身上穿一件玄色拷绸号衣,四围用湖色金阊纱滚着灵芝如意,品蓝生丝裤子,玄色夹纱快靴,靴上也用绿皮镶成如意头的样子,那样儿甚是时髦。春树暗想:不知是那里的车夫,打扮得这般邪气。又见那车夫揎拳掳背的,揪着一个衣裳破碎的老头儿,白须白发,已有七十多岁光景。只听得那车夫口中骂道:“我把你这个瞎眼的乌龟!好好的自家走路,怎么撞到别人身上?几乎把我撞了一交,还把我的衣裳扯破。你好好的赔了我的衣裳便罢,若说一声不肯,我就请出我们的老爷来,一张名片,把你送到巡捕房锁押起来,看你走路还撞人不撞?”

那老头儿听了这一派利害的话儿,早把他吓得浑身乱抖,面容失色,没口子的求告那个车夫道:“我一时自不留心,把你撞了一撞,可怜我是个穷人,那里赔得起你的衣服?只求你行个方便,放我去罢。”那车夫那里肯听,圆睁两眼,大声说道:“你这个老死囚,谁叫你走路这般乱撞,你赔不起难道就算了么?”那老头儿听了更加着急,再三哀告,车夫只是不依,拉住不放,却看着他自己身上穿的一身衣服,扬扬得意的样儿,摇头晃脑的向旁边看的人说道:“我这一身号衣穿了还不多几次,偏偏今天遇着这老乌龟,走路就如逃命一般,没命的撞过来,把我簇新的衣裳拉了一道口子,你想可恼不可恼?”说着,便提起那拉破的地方给众人瞧看。

春树看时,原来是那衣裳叉口里头,少微脱了些儿线缝,并不是要紧地方,明是这车夫倚着主人的势焰,狐假虎威,在那里欺压良善。春树见车夫满面得意的样儿,挺胸凸肚指手划脚的揪着那老头儿的衣领,定要赔了衣裳才罢,气势汹汹,像要打他的样子。这老头儿本来是个老实乡愚,又不会说话,被那车夫讹住,急得他无可如何,看他那个样子,像要哭出来的光景,不住口的认错,说:“我是个苦人,那里赔偿得起,只算放了一个生罢。”旁观的人听了,都甚可怜那老头儿,争着上前劝解。那车夫那里肯听,不觉心中焦躁起来,顺手把那老头儿着力一拖,听得“哈”的一声,早把那老头子领口撕破,直豁到背脊上来。老头子没有防备,站立不稳,扑地跌了一交,扒起来不敢开口,还在那里央求。

春树见此光景,心中十分忿恨,打算要替那老头儿抱个不平,便抢步上前,分开众人,向那车夫说道:“你的衣服虽然破了些儿,却是脱了线缝,算不得什么损伤!你一定要他赔你的衣服,你看这老头儿的样儿可是赔得起衣服的人么?况且他不过撞你一下,你就要他赔还衣服,你把他的衣裳撕破,难道是不要赔的么?据我看来,还是两边扯直,放他去罢,你就是和他闹到明天,他也赔不出你的衣服,何必要这般的倚势横行?”

贡春树说这一番话儿,自以为是极和平的了,那车夫料无不听之理。不料那车夫听了把脸一沉,睁着一双贼眼冷笑一声道:“先生,你走你的路儿,不要来多管我们的闲事!你不晓得我家老爷的利害,一身新做的号衣给我穿了出来,如今破了一块,给他看见他肯答应么?这个老乌龟如若定不肯赔,管教他到巡捕房里坐上几天,吃些眼前的苦楚,他才晓得利害呢!”几句话,把一个贡春树气得发昏。

辛修甫在后边听得也是心中不忍,走上来向车夫说道:“这老头儿虽然穷苦,却总是我们四万万国民内的同胞,你不能照应他些,已经不能尽同类的义务了,为什么倒反施着野蛮的手段,用压力去禁制他,你难道没有一些儿国民思想的么?”

那车夫听了,那里懂得他讲的是什么东西,满口叽哩咕噜的说不清楚,只认辛修甫说的是外国话,倒也不敢得罪他,只向修甫摇了摇头,似乎是不懂得他话说的意思。

修甫自家也觉好笑,便向他讲了一句平话道:“你放那老头儿去罢,他穷到这个样儿,你难道没有一些恻隐之心么?”那车夫听得明白,方知他刚才的说并不是外国话儿,又翻起那一张势利面孔恶狠狠的瞪了修甫一眼,竟不理会于他,却只顾朝着老头儿暴跳如雷的道:“怎么样,你延捱一会子就不要你赔不成?我没有多大的工夫在这里等你,我可要喊巡捕去了。”气得个辛修甫走了开去,不忍看他,向着贡春树叹口气道:“你看他穿着一身奴隶的衣服,不晓得一些惭愧,反觉得一面孔的得意非常,靠着他主人的势力,糟蹋自己的同胞。就和现在的一班朝廷大老一般,见了外国人侧目而视,侧耳而听,你就叫他出妻献子,他还觉得荣幸非常,仗着外国人的势头,拼命的欺凌同种,你道可气不可气?怪不得外国人把我们中国的人种比作南非洲的黑人,这真是天地生成的奴隶性质,无可挽回。你想我们中国,上自中堂督抚,下至皂隶车夫,都是这般性质,那里还讲得到什么变法自强?只好同三两岁的孩子一般,说几句梦话罢了。”

春树道:“这个车夫实在的可恶,怪不得激出你平日的牢骚。但不知这个时候秋谷恰恰走到那里去了,若得他来解劝,这车夫若是不知风色,不免就要吃亏。偏偏我们两人都是个弄笔书生,没有一些气力,到了这些地方,可见平日懂些拳棒也有用处。刚才只要我有些气力,我便不管他什么捕房的规矩、租界的章程,且先将这车夫痛打一顿,出出这一口不平的恶气,只当做陈琳的一篇草檄,祢衡的三挝渔阳。”

贡春树正还要说将下去,不料章秋谷早已随后下来,见门口有人吵闹,不知何事,便也挤出来。看时,见贡春树正在和那车夫说话,秋谷暗笑春树这样斯斯文文的话儿,这班山精野兽一般的人,那里肯听他的说话?果然那车夫非但不听,反把贡春树抢白了两句。又见辛修甫抢上前去,和车夫背了一大套的新名词,秋谷更加好笑,跟在二人的后面,听他们再说什么。那车夫闹事,他们两人劝解的情形,一一被他看得明白,听得分明,此刻再忍不住,在他们二人背后直跳出来,大笑道:“你用这些说话去劝这种绝无意识的畜生,真真是对牛弹琴,枉费了多少功夫,他却一毫不懂。你想一个拉包车的蠢物,他有这样高的人格么?”修甫听了,也不觉自家好笑起来。秋谷又道:“要打发他们这些禽兽一点不难,自然另有一番说法,不信你看我来。”

说时迟,那时快,只见那车夫扭着老头儿的衣服,高声叫起巡捕来。那老头儿急得战抖抖的涕泪俱下。幸而叫了一声,巡捕尚未听见,秋谷急忙走上前去,两手一拦,说一声:“且慢!”就这一拦里,早把那车夫的手松开,两人一齐倒退了几步。车夫见秋谷的手势来的利害,不觉吃了一惊,又见秋谷人才轩爽,衣服鲜华,凤眼含瞋,双眉微竖,带着一团怒气,未曾开口,先觉得有些怕他。秋谷拦开了他们两个,向那车夫喝道:“你的主人是何等样人?现做什么生意?与我叫他出来!

你不过是他的一个车夫,连个奴才也不如的脚色,居然就敢在马路之上这样的欺人。

你可知租界的章程,相打相骂都是犯规。你在马路上边和他揪扭,你自己先犯了捕房的规矩,还要呼吓别人,满口混说。我劝你赶紧放他去了,还是你的便宜,否则我叫巡捕到来,把你们两人一同送到捕房,有话明天再说。只怕问明白了,你还要赔他的衣服呢!你当巡捕房内的捕头,就是你主人做的么?好个不要脸的奴才,还不与我快滚!“那车夫听章秋谷的话头利害,想一想果是不差,摸不着秋谷是何等人物,想着要叫他的主人出来说话,一定是个大大的来头,那敢得罪?被秋谷骂得诺诺连声,低头倒退。那老头儿正是着急,无意之中倒遇着了章秋谷这个救星,千恩万谢的走了。

秋谷回过头来,向着修甫和春树二人笑道:“何如?”修甫道:“这却实在亏你,装得真像。”春树忽诧问道:“小宝他们那里去了?”秋谷道:“还等得你来查问,你们劝架的时候,他们早已回去的了。我们也快些走罢!”说着,便邀二人同到王佩兰家去打个茶围。二人应允,便从四马路穿过石路,径进兆贵里来。春树问他陈文仙处可去,秋谷摇头。

三人联步行来,寻着了王佩兰的牌子,走进客堂,问王佩兰房间。相帮说在楼上。秋谷当先走上楼去,早有王佩兰的大姐走出来招呼进去。佩兰刚刚出局回来,含笑叫了一声:“章大少!”秋谷笑道:“我排行第二,堂子里头都赶着我叫老二,你以后也不必叫什么大少爷、二少爷,竟直直捷捷的叫我一声老二就完了。”佩兰把眼一瞟,笑道:“阿唷!格末倪叫差哉,二少勿要动气。”秋谷拍手道:“刚刚一句说话,叫你不要叫我什么大少爷、二少爷,你又叫我二少。”佩兰带笑说道:“别人家勿叫二少爷,叫耐老二,格是有道理格啘,像倪该搭二少难得赏赏倪格光,生来总要客气点,倪阿好去跟仔别人叫耐啥格老二?倪也无拨格号交情啘。”说罢,又向秋谷飞了一眼,道:“二少爷阿对?”修甫、春树见了,不约而同齐齐的叫一声“好”。秋谷笑道:“我同别人家有什么交情?你倒要说说我听。”佩兰又笑道:“阿唷!格是倪勿晓得格啘。耐二少爷搭俚笃格交情,倪陆里会晓得?不过倪想起来,拿仔客人格排行当仔称呼,实梗格窝心,还说无拨交情,说拨随便啥人听听看,阿肯相信?”秋谷走上一步,低声说道:“如今说来,定要有了交情,方好把排行当作称呼的了。”佩兰道:“格是自然嗫,无拨交情也办勿到啘。”秋谷道:“自此以后,你就叫我老二何如?”王佩兰把嘴一披,道:“倪阿有格好福气?拨陈文仙晓得仔,是反得来好白相煞哉。”秋谷道:“陈文仙倒向来不是这样的人,你不要混冤枉他。”王佩兰道:“阿唷!倒会帮笃啘,阿是说仔耐格相好,耐来浪帮俚哉。”说得大家笑了。

秋谷暗想:王佩兰面貌虽然不错,说起话来着实有点醋意,只怕性情不好,比不上陈文仙的阔大和平。这种人做了他,恐怕没有什么趣味,便觉得心上冷了好些。

又转一个念头想道:虽然如此,但是做个把倌人,不过是逢场作戏的勾当,合着脾气的多走两次,性情不好的少去两趟,又不是要娶他回去,何必拣得这样顶真?这般一想,便决计想要做他,要想把陈文仙和王佩兰做个一箭双雕,方才满意。

闲话休提。只说秋谷等三人随意坐下,见房间甚是宽阔,陈设极精,房内一个娘姨、一个大姐也甚是伶俐,应酬得颇为周到。秋谷坐了一会,因修甫有事要走,便也走了。自此秋谷在王佩兰院中连吃了几台酒,接连碰了两场和,倒着实的报效了几天。秋谷和佩兰两人,差不多都有些意思。

有一天,秋谷独自一人到佩兰家来打茶围,佩兰恰好在家,亲手替他脱了长衫挂在衣架上,请他坐下。自己坐在旁边,用一把雕翎扇轻轻的与他扇风,笑道:“今朝一干仔来,清清爽爽倒无啥。”又低声说道:“耐要来末一干仔来好哉,啥事体同仔几花朋友闹得一塌糊涂,倪要说两声闲话才无拨空,格末叫讨气。”秋谷听了甚喜,问他有什么说话?佩兰笑道:“倪想仔闲话,要问耐末耐倒勿来;故歇耐来仔,倪格闲话倒又忘记脱格哉。”秋谷一笑,明知他是一句随口应酬的话,也不追问。佩兰忽问秋谷道:“格两日耐陈文仙搭阿去?”秋谷道:“不去。”佩兰把指头在秋谷额上推了一下,道:“耐末再要瞒倪,唔笃老相好阿有勿去格道理?

耐格鬼话也说得勿像啘。“秋谷也笑了,两人谈了一回,无意之中谈到如今堂子里的倌人,做起客人来也有许多难处。王佩兰道:”故歇格客人划一来得讨气,做起倌人来,东边做这一个,西边再做一个,呒拨一定格地方,做到仔后来,做来做去,总归呒拨要好格倌人。耐想客人脾气勿好,东做做,西做做,倌人阿会搭俚要好?“

正是:

消受莺花之妒,梅子含酸;欲争邢尹之妍,蛾眉暗画。

欲知后来何事,请看下回,便知分解。





第四十四回 有情人都成新眷属 懊恼记重仿玉台文





且说章秋谷听了王佩兰的说话,不觉对他笑道:“你的说话虽是不差,也看倌人的脾气。碰着个会吃醋的倌人,就要把客人吃住,不放他到别处去再做别人;也有性气好些的,做了客人,却也并不是这个样儿。就如陈文仙,我做他将及两年,虽不见得十分要好,却是大家客客气气的,从没有看见他和人吃醋。不像你这般脾气,就和山西老表一般,一身儿都是酸气。”王佩兰听了,不好意思起来,洋洋的走了开去,道:“耐格两声闲话倒诧异笃啘。倪啥辰光搭陈文仙吃醋?耐倒说拨倪听听看。耐欢喜陈文仙末,只顾到俚搭去末哉,倪阿好叫耐勿去?为啥要牵牵连连,拿倪一淘说?倪末搭俚吃啥格醋?耐自家想想看,勿要缠错仔人。”秋谷晓得堂子里倌人最犯忌的是说他吃醋,况秋谷和王佩兰没有落过相好,自然更加避讳的了,因此笑了一笑,便也不提。

两人谈了一会,秋谷叫娘姨取过长衫要着,王佩兰一把拦住道:“耐着仔长衫,要紧到啥场化去?”秋谷佯笑道:“我不到别处去,要回栈去睡了。”王佩兰鼻子里哼了一声,似笑非笑的道:“耐末要紧到陈文仙搭去,阿怕倪勿晓得,今朝倪定规勿许耐去,看你有啥格法子?”秋谷却故意笑道:“你不许我去,把我留在此间做甚?”佩兰面上一红,假作没有听见,口中说道:“勿然是倪也勿来叫耐勿去,故歇耐再要瞒倪末,倪定规勿成功。”说着,半真半假的趁势往秋谷身上一坐,撒娇道:“倪勿来,耐下转阿要实梗?”秋谷也随随便便的和佩兰鬼混一回。看看钟上已经两点多钟,秋谷故意立起身来像个要走的样子,佩兰嗔道:“耐阿是咦要去哉?”秋谷低声笑着学他的话道:“勿去末无啥事体啘,倪两家头来碰对对和阿好?”佩兰呸的啐了秋谷一口,羞得别转头去,面上发起烧来。秋谷兀自假意要起,佩兰一手拉着秋谷的衣袖,道:“勿要来浪假痴假呆哉,搭我去坐来浪。”秋谷问他可有什么话说?佩兰说不出来,只把秋谷瞪了半日,不声不响。娘姨在旁说道:“二少爷勿要去哉,倪先生从来朆自家留过歇客人,挨着耐格二少爷还是头一转来啘。”秋谷方才一笑无言。

娘姨开上稀饭来吃了,伏侍佩兰卸过头面,掩上房门,大家退出。这里章秋谷和王佩兰,一个是敷粉欺朱,平叔莲花之面。一个是飘烟抱雨,小蛮杨柳之腰。自然是人面田田,脂香满满,不消说是一双两好的了。

只说秋谷一连在王佩兰家住了几天,陈文仙院中竟绝迹不去。王佩兰又说陈文仙的品行如何不好,娘姨门的应酬更不讲究,叫秋谷不要再去做他。秋谷口中含糊答应,心上虽然不信,却就此陈文仙家的踪迹疏了好些。

忽一日,王佩兰竟敲起章秋谷的竹杠来,要他打一支十五两重的金水烟袋。秋谷大为诧异。欲待不答应他,恐怕当面受他的奚落;若要当真去和他打造,不但对不住陈文仙,连自己也对不住。回想自家在花城香界之内整整混了五年,也颇颇的有些名气,就是一等再时髦的倌人从没有这样的大敲竹杠,所以挥霍的都是面子上的银钱,自家其实所费不多。旁人看了他的豪华气概,差不多就像个有名的阔客一般。每每见那一班曲辫子的客人和倌人去买这样办那样,鞠躬尽瘁的一种光景,笑他是个大大的瘟生。不料如今轮到自家身上,也被王佩兰当作瘟生看待,敲起大注的竹杠来。懊悔当初不该钻头觅缝的去做他,如今却弄得这般结局,觉得王佩兰这个人势利异常,全没有一些情义。便又想着陈文仙,做了多时,从没有敲过他的竹杠,可见如今世上都是王佩兰一路的人;要如陈文仙这个样儿,已经难得的了。当下笼笼统统的答应了他一声。王佩兰便正色道:“耐答应仔是要去拿得来格捏,勿要故歇末答应,歇仔两日绰倪格烂污,是倪勿来格嗫。”秋谷见王佩兰惟利是图,含着一腔怒意,面上却不露出来,故意笑道:“我既然答应了,停两日自然拿来,难道我是哄你的么?”王佩兰听了,见秋谷说得斩钉截铁,料想不是假的,方才满心欢喜,喜孜孜的放出满面春风。又问他几时打好。秋谷道:“这却我也不知,要去问那银楼里头方得明白。大约一礼拜,只怕也差不多了。”佩兰屈着指头算道:“今朝是礼拜一,耐礼拜日仔拿得来阿好?”秋谷勉强点一点头。坐了一会,觉得没有什么意思,起身要走。佩兰送到楼门,又千叮万嘱的叫秋谷不要忘了。

秋谷出了王佩兰家,心想王佩兰这般可恶,想要把他处置一番,一时又想不出什么主意,只好到了礼拜日慢慢的耽搁他,叫他自家晓得,不来开这口儿,也就罢了。一面想着,脚下随便乱走,低着头只往前撞,不知不觉早出兆贵里的弄堂。只听得迎面有人叫了他一声,秋谷抬起头来一看,却是贡春树,手中拿着一卷不知是什么东西,正要举步进弄,恰见秋谷低头急走出来,故而叫了一声。秋谷立住了脚,含笑问道:“你到兆贵里,可是去寻我的么?”春树笑着点头。秋谷又问他手内是什么东西?春树道:“就是要给你看的那个手卷。我一连几天不得工夫看你,今天特地带着手卷前来看你一趟,一来要请教你的珠玉,二来请你看看这个手卷的笔意画得如何?”秋谷道:“我刚在王佩兰家出来,要想回去,此间立谈不便,还是回栈去坐一回儿罢。”春树应允,两人同到吉升栈来。

到了栈内,走进房坐下,秋谷就把贡春树手内的手卷取了过来打开细看。只见那一幅纸儿约有二尺余长,绫锦装潢,十分华丽。上面画着一座工细楼台,纱窗半掩,青琐横斜,高高的吊起一挂湘帘,栏杆屈曲,映衬着楼外边几树垂杨,随风飘拂。重杨之下便是一湾流水,停泊着几只画船。那楼窗内倚着一个美人,露着半身,凭栏凝睇,春山敛恨,秋水含颦,微微的带着病容,丰神酸楚,那一双眼光紧紧的注在楼下一只船上。船头上也立着一个少年,玉立亭亭,丰仪整洁,和春树甚是相像,呆呆的仰望高楼,四目相视,神气之间画得甚是活泼,发纹衣褐,工细异常,大有赵子昂的笔意。

秋谷看了一回,赞道:“这一个手卷居然画得不差,却像个近时名家的手笔,可是吴友如画的么?”春树道:“不是,吴友如听说已经死了几年,这个手卷是我们常州一个画家名叫黄松寿画的。”秋谷不语,只点点头。春树便接过手卷,把后面放开,见后面空着丈余长的素纸,摊在台上,道:“就请你的大笔一挥何如?”秋谷摇头道:“这些事儿我素来没有弄过,我还是和你做一篇四六序文,这题的一层,你赶紧去请教别人,我却不能破例。”春树见他不肯,也只得罢了。把手卷收起,向秋谷笑道:“你既然一定不肯,我也不能勉强,只把那一篇序文快快做来,好待我开开眼界。”秋谷笑道:“你还是这般性急,待我慢慢的想起来,你却不要在旁打岔。”说着,便立起来在房内走了几步,不到一刻钟,腹稿已经打好,却笑向春树道:“我想做一篇短短的四六,题目就叫《懊恼记》;你那一个手卷,索性也叫他做《懊恼图》,何如?”春树拍手叫好。

当下秋谷取了一张冷金笺铺在案上,提起笔来飕飕的便写。一笔赵松雪的行草就如兔起鹘落的一般,写得满纸上龙蛇飞舞。春树见他写得神速,差不多就是个再生的曹子建,转世的温八叉,暗暗的心中佩服。不一会,秋谷已是写完,把笔一掷,立起身笑道:“虽然潦草文成,幸而还没有什么不通之处,你来看看,如有不妥的地方,我们大家酌改。”春树笑道:“你又来说违心之论了。老实说,我们做出来的文字,无论再是不通,总还比近来名士文章高了几倍。况且你的四六也极好的了,我们一班同辈之中,那里赶你得上?”秋谷一笑无言。

春树便走近案前看时,只见写着道:

琵琶沦落,商妇工愁,小玉多情,十郎薄幸。所以情天不老,韩寿圆割臂之盟;密约难忘,徐令合惊闺之镜。彩鸾已嫁,嗟绿叶之成阴;飞燕重来,笑花枝之独照。未还珠于合浦,先种玉于蓝田。扬州杜牧之狂,太白西川之痛。桃花易老,银汉难通,此《懊恼记》之所由作也。则有门承通德,家庆弹冠。刘晏七龄,能为正字;邺侯四岁,解赋方圆。少登北海之堂,长有羊车之誉。而且何郎怀袖,春留十日之香;李泌丰神,夜抱九仙之骨。长卿善病,叔宝多愁。未逢绿绮于临邛,先得倾城于吴会。罗敷相见,遗玉佩以归来;卓氏私奔,脱貂裘而换酒。天上双星之会,碧落团圆;人间倩女之魂,红绡惆怅。盖飘萧华发,依然卫玠之姿;落拓江湖,未改潘安之度。三生慧业,一见倾心。蚌已含珠,人难化鹤。海天蜃气,辨幻影于楼台;情海生波,更惊心于风雨。匆匆归去,歌残白练之裙;好好题诗,剔破桃花之纸。花开造次,心未死而先灰;莺苦丁宁,泪将流而未敢。公河莫渡,指白水以为盟;比翼相期,愿青天之作证。从此相思刻骨,远梦惊心。丁香之眉结难开,莲子之心期终苦。押衙已死,叱拨何来;碧血招魂,黄衫安在?使君打鸭,可怜花底之鸳鸯;公子思乡,谁解笼中之鹦鹉?愁如春水,不解西流;泪似大江,还期东去。嗟乎!冯京宅里,何来金带之招?温峤堂前,未有玉台之聘。当年相遇,愿为连理之枝;他日重逢,长作相思之树。

春树看了又看,爱不释手,朗吟了几遍方才放下,向秋谷道:“这一篇四六做得香云缭绕,花雨缤纷,词意缠绵,文情宛转,真个是鹿锦风绫之艳,珊瑚玉树之珍。我们实在望尘不及,甘拜下风。但是一样,把我却抬举的过分了些。虽然一字之褒,荣于华衮。我自家心上却总觉有些过意不去,当不起这样的揄扬。”秋谷大笑道:“文字中的褒贬,扬之可使上天,抑之可使入地,有什么一定的讲究?你果然自家过意不去,只把我这一篇文字当作是说的别人,何必要这般呆实?”说得春树也笑了。春树又道:“我把你这一篇草稿带去给修甫他们大家看看,明天在密采里请你们吃顿大菜,你可有工夫到么?秋谷道:”你请我吃大菜,那怕再没有功夫也要到的。“春树大喜,丁宁而别。

到了明天晚上,春树果然亲到栈中,邀着秋谷到密采里。坐了不多一会,修甫等大家都已到来,又有几个常州乡亲,秋谷素不认识,一一的招呼过了。末后又走进一个人来,一进房间就向主人作了一个大揖,众人觉得甚是好笑。原来不是别人,就是那有名饭桶,第一瘟生的金汉良。秋谷不觉格声一笑。金汉良抬头一看,见是章秋谷,心上就吃了一惊,暗想今天真是倒运,恰恰又遇着了这个冤家。勉强大家入座。这一席是章秋谷倡议不要叫局,为的是大家好细细的谈心,若一叫了局来,众人个心,便一齐移到倌人身上,没有说话的功夫。

当下坐定之后,贡春树便取出秋谷做的那一篇《懊恼记》来,给修甫、小屏等大家传看。修甫等看了一遍,一个个极口称扬,秋谷不免谦让几句。春树又把那一个手卷交与修甫,要请他们大家题些什么。修甫、小屏齐声说道:“我们构思颇差,那里赶得上你们的这般神速,万不能即席挥毫。你一定要我们当场献丑,只好把这个手卷我们带了回去,慢慢的构思起来可好?”春树拱手应允。

这一席因没有叫局,大家谈得十分热闹。只有金汉良一人坐在席上,没有人去理他,呆呆的听着众人讲话,却又不懂他们说的是什么东西,自家觉得没趣起来。四边一看,见章秋谷的那一张草稿,众人看过之后没有收起,还在那桌子中间。金汉良伸手取了过来,约略看了一遍,也有懂的,也有不懂的,因要卖弄他自家的才情,假充通品,便闭着眼睛,摇头拍手的做出许多丑态,竟高声朗诵起来,不知不觉的念出多少骑马句子,还有无数的白字。这一来,早把众人的话头打断,都看着金汉良暗暗的好笑。金汉良还是一毫不觉。正是:

浣花笺纸,凄凉金缕之歌;杨柳楼台,懊恼王钩之梦。

欲知后事,请听下回。





第四十五回 说官话小子无知 困春悉萧娘多病





且说章秋谷等听得金汉良念出许多白字,甚是好笑。章秋谷便埋怨贡春树道:“今天我们一班朋友都是性命之交,正好趁此良宵快谈风月,为什么偏要带着这一个蠢货,被他搅得满坐不欢?难道这样的一身俗骨的畜生,你还要和他来往么?”

春树听了,也觉有些懊悔,忽又笑道:“他这样混混沌沌的人物,正好给你做一味下酒的佳肴,比到用《汉书》下酒,还胜强百倍呢!”秋谷听了,忍不住狂笑起来。

修甫等在旁听得分明,一个个放声大笑。

金汉良正在那里念得出神,那里去管他们是笑的什么?也万想不到笑的就是自家,还在那里提起了毛竹一般的喉咙,念得十分得意。众人虽然惹厌,也只得由他。

好容易一会儿的工夫才算念毕,方才咳嗽一声,吐了一口浓浓的涎沫,抬起眼睛打量众人时,见秋谷等还是笑容满面,心中暗想:幸而我今天显了一显才情,他们就登时瞧得起我起来。又见章秋谷今天没有开口取笑着他,心上更是欢喜。不料这一阵欢喜,顿时忘了平时的顾忌,不觉露了他的本来面目出来,便张牙舞爪的立起来,打着那不三不四的官话,对着众人说道:“像这样的文章,兄弟小时也曾读过。记得还是十九岁的时候,先生叫兄弟念了一部古文。后来又出了几个什么论题,要兄弟做什么策论,兄弟却也狠费了些工夫。可惜现在荒了多年,只怕做出来没有这般的顺口了。”

众人听他打着一口京腔,南腔北调的十分可笑。章秋谷忍不住问金汉良道:“金汉兄是什么贵班?想就要到省的了。果然你们官场中人毕竟有些儿气派,不要说是别,就是你这一口京腔,也说得十分圆熟,比那戏子唱的京调,倌人说的苏白,觉得还要好听些。”

金汉良听章秋谷问到他的功名,这是他生平第一件快心得意的事情,正要逢人卖弄,只把他得意的身子摇子两摇,好像一个身体都没有放处的一般。只见他满面精神的说道:“兄弟是个尽先候选的知县,现在已经指了直隶的省分。不瞒你老哥说,兄弟报捐这个知县,倒也狠费了一笔大钱,如今打算就要到省去,领了制台的咨文,再进京去引见,早些到省,或者当个什么差使,也好捞转两个本钱。到底这做官的赚起钱来,比到那做生意容易多子。”说罢,哈哈大笑。

章秋谷听到此际,实在忍不住,便驳他道:“你既然是个候选班,该应归部铨选,怎么又平空的指起省来?况且向来的章程,大凡各省报捐的候补人员,都要先行引见,领了部里的文凭方能到省。你金汉兄才说要先去领了制台的咨文再去引见,请问这制台的咨文可是给皇上的么?”金汉良听了,知道自家说错了,面上红了一阵,老着面皮说道:“这是他们引见过的人员出来说的。他们是过来的人,说的话儿料想不错,只怕还是你章秋翁记错了罢。”秋谷忍住了笑。又道:“想必是你金汉兄做了吏部,和他们改了章程。我本来没有捐过什么功名,那里晓得这里头的规矩?”说得金汉良面上一红一白好不难过,还亏得他的脸皮甚厚,挨了一回也就罢了,便不和秋谷说话,又同贡春树谈心起来。

秋谷见他不知羞耻,真是天下无难事,只怕老画皮,竟奈何他不得。想了一会,便又向众人笑道:“我有一个笑话,讲给你们大家听听何如?”众人估料一定又是骂着金汉良的笑话,都要听他又编出什么故事来,大众齐声说好。秋谷含笑说道:“那公冶长不是会听鸟语的么?你们却不晓得公冶长还有一个兄弟,叫作公冶短。”

春树等听了公冶短的名字,已忍不住先笑起来。秋谷又道:“那公冶长能解禽言,不料这公冶短也有一般绝技,能通兽语。公冶短的住房间壁,是个磨豆腐的磨房,养着一个驴子,每天四更起来,把这驴子上了笼头叫他磨麦。不想有一天,这驴子忽然带着笼头乱进乱跳,高声大叫起来,叫得驴主人恼了,把鞭子狠狠的打他。谁知打者自打,叫者自叫,凭你怎样的乱抽,他还是叫个不住。这驴主人诧异得了不得,连忙过隔壁去请了公冶短来,和他说了,要他听听这驴子说的是什么话儿。公冶短走到驴子身边仔细听了一会,驴子还在那里昂头掉尾的嘶鸣,似有得意之状。

公冶短听了,把头摇了一摇,侧耳再听一回,依然不懂。公冶短焦躁起来,抢过一根鞭子。“秋谷说到这里,走过来把手在金汉良肩上一拍,道:”把那驴子狠狠抽了一鞭,口中骂道:“你这个不要脸的畜生,放着好好的话儿不说,偏要学起蓝青官话来。你这样的畜生,人格还没有完全,配说什么官话,难道你也想学着他们一班捐官的人,报捐了什么州县,去到省候补么?‘”众人听了,这一阵笑声就如那春雷震耳,一个个笑得话都说不出来。贡春树笑到极处,一个不留神,竟连人连椅望后一仰,滚在地下,还在那里大笑。众人正在笑得有趣,猛然听见“扑通”一声,急急的看时,见贡春树跌在地下,一张椅子也倒在一旁。众人更加好笑,秋谷连忙过去把春树拉了起来。

金汉良被章秋谷的一场笑话说得他满面通红,又被众人这一阵笑声笑得浑身汗出。待他认真发作起来,料想他们口众人多,那里说他得过?只得勉强忍住了,觉得自家面上一阵阵的热气直升上来,直把他气得坐立不安,好生难过,坐在席上如坐针毡一般。巴得他们吃完了,立起身来,金汉良急急的穿好长衫,就如那笼中鸟雀,网内鱼虾,连忙别了主人飞一般的逃了出去。这里众人说说笑笑,一路回去,又去打了几个茶围,方才分手。

到了礼拜的那一天,王佩兰因秋谷几天不去,晓得事情有些不妙,起了一个绝早,梳好了头,竟到吉升栈内来看秋谷。其时约有十点多钟光景,秋谷尚未起来。

当差的进来叫醒秋谷,睁眼一看,见王佩兰扶着一个小大姐,婷婷袅袅的进来,就坐在秋谷床上,向秋谷嫣然一笑,说道:“耐到好格,几日天勿到倪搭去,倪牵记得来!”秋谷也作苏白答道:“好哉好哉,勿要来浪生意经哉。”佩兰“嗤”的一笑,把秋谷拧了一把。秋谷披衣坐起,问他为什么来得这般早法,佩兰道:“为仔耐几日勿去,常恐耐有啥格勿舒齐,所以倪来看看耐呀!”秋谷含笑道:“多谢多谢,看是不敢当的。你有什么事情,只顾请说。”佩兰道:“倪也无啥别样事体,就是格支烟筒,耐今朝好去拿得来哉啘?”秋谷假作失惊道:“该死该死,我竟忘了,没有到银楼去定,只好等回儿再去的了。”王佩兰见说,不依道:“耐前日仔搭倪说得明明白白,今朝啥格假痴假呆,说忘记脱哉。耐吃饭困觉阿会忘记?倪勿要,耐豪燥点去搭倪拿得来!”秋谷只是笑,也不说拿,也不说不拿。王佩兰见秋谷不肯,焦躁起来,拉着秋谷的手着紧问道:“耐到底阿去搭倪拿介?”连问几声,秋谷并不开口。王佩兰更加着急,把秋谷乱推,道:“耐说哩,啥一声勿响哉呀?”

秋谷方开口笑道:“你也不要去拿什么烟筒了,倒是我去拿一把斧头来送你用用罢。”

王佩兰听了,跳起来嚷道:“唔笃听听看,说出来格闲话,阿要气煞仔人!耐自家绰仔倪格烂污,倒说倪敲耐格竹杠。耐格人阿有良心?”秋谷笑道:“有了良心,还肯敲客人的竹杠么?”

王佩兰听秋谷的话一句紧似一句,更觉生气,冷笑一声,一言不发。秋谷也不理会,跨下床来洗脸漱口。诸事完毕,回身仍旧坐在床沿,向佩兰笑道:“为什么半天并不开口,可是没有和你去拿烟袋,所以生了气么?”佩兰冷冷的答道:“倪末陆里敢生气?只要耐二少爷勿生仔气末是哉。”停了一停,又道:“倪要耐拿一只烟筒,也勿算敲耐格竹杠啘。耐勿情愿末,好好里说末哉,倪也无啥希奇。勿壳张耐当时末来浪答应,骗得倪欢喜煞,到仔故歇原是放仔倪个生,还要说倪敲耐格竹杠,耐倒直头好意思格。”说着就低下头去,眼波溶溶,好像要流下泪来的样子。

又道:“故歇倪房间里格排娘姨,才晓得耐来浪搭倪打金烟筒,连搭仔楼下底格本家才晓得哉,停歇歇俚笃问起倪来,耐是生来无啥要紧,倪阿好意思说得出?”

秋谷听他说到此间,不觉已是几分怒意,又听他说道:“耐故歇歇就是拿拨仔倪,一塌刮了几百洋钱格事体,耐二少爷实梗格场面,也勿在乎此啘。老实说,推板点格客人,送仔倪两付金钏臂,倪理也勿去理俚,勿要说落啥格相好哉,耐末…

…“说到此,口中顿了一顿道:”再要说倪敲竹杠?“秋谷不觉笑道:”如此说来,反是我得了便宜了。“王佩兰面上也红了一红,星眼流波,蛾眉半锁,瞅了秋谷一眼,又道:”耐是有名气格客人啘,故歇为仔一只烟筒放倪格生,倪是就不过坍仔点台末哉。耐为仔格点点小事体,倒卖脱仔自家格牌子。倪搭耐想起来啥犯着嗄?“

秋谷听王佩兰说得十分尖刻,不觉勃然大怒,面上已经红了,勉强捺住了怒气,冷笑道:“我不过和你说句玩话罢了,难道真要绰你的烂污么?此刻我就同你一同到银楼去何如?”佩兰听了方才大喜,顿时眼笑眉开的道:“倪也晓得耐勿是格排滑头码子,推扳点客人,倪也勿肯做俚啘。”秋谷不待说完,截住了道:“不用说了,我叫人去雇部马车,我们一同就去。”

恰好那一天,阴阴沉沉的没有日光,甚是凉爽。佩兰此时心满意足,再不多言。

一会儿马车放在门前,佩兰叫跟来的大姐先自回去,同着秋谷坐上马车。马夫问明去向,加上一鞭,直向杨庆和门前停下。秋谷因和那杨庆和的老班杨宝宝素来相识,向有往来,便同着佩兰下车进内,和那柜内管帐的先生说明,要打一只金水烟筒,大约十四五两的光景,明天就要来拿。管帐的听说明天就要,踌躇道:“明天恐怕打造不来,可好略停两日?”秋谷和那管帐的再三商量,央他连夜赶做。管帐的却情不过,只得点头。秋谷略坐一会,拱手辞别。王佩兰不肯放他回栈,便直到兆贵里来。王佩兰欢天喜地的同着秋谷进去,那一种要好巴结的情形竟比往常时加了几倍,难以尽述。

留秋谷吃过了饭,王佩兰要坐马车到张园去,秋谷也同王佩兰坐在一马车上。

到张园泡了一碗茶,坐得不多一刻,只见一个倌人从上首转了过来,态度温存,风姿淡雅,走到秋谷面前朝他点一点头,停住脚步微微含笑,似欲有言。秋谷看时,见是陈文仙同院住的倌人金湘娥,也朝他笑了一笑。湘娥悄问秋谷道:“耐阿晓得文仙来浪生病呀?”秋谷吃了一惊道:“我几天不去,不晓得院内的事情,他为什么又生起病来?”湘娥道:“为仔耐几日勿去,认仔耐动气勿来哉,难末心浪一径勿舒齐。格两日局才勿出,才是倪搭俚代格。耐今朝阿去看看俚呀?”秋谷点了一点头道:“我停回晚间就去,托你回去和他先说一声。”湘娥应允,也不坐下,姗姗的去了。

王佩兰虽坐在秋谷对面,却并未留神,不去理会,只认做金湘娥也是秋谷做的相好。候他去了,方向秋谷笑道:“耐格相好倒多笃啘?”秋谷笑而不辨,心上却狠记忆着陈文仙,要想张园出来就去看他,王佩兰死命的拉住,那肯放松?撒娇撒痴的定要秋谷送他回来。秋谷摆脱不来,只得把佩兰送到院中,一同进去。佩兰提起了全副的精神应酬秋谷,无如秋谷心上想着陈文仙,总有些无精打采的样子。佩兰也猜不着他有什么心事,只是伴住了不肯放他。

到得差不多十二点钟,秋谷立起身来,一定要走。佩兰拦阻不住,发起急来,喊道:“唔笃豪燥点来嗫,二少爷要去哉!”就这一声喊里,后房房外跑进四五个大姐娘姨,一齐拥上,竟是打了一个拷拷圈儿,把一个章秋谷团团围住,好像那杨国忠的肉屏风,石季伦的锦步障,一些儿水泄不通,七张八嘴的挽留,七手八脚的乱扯。秋谷见此光景暗中好笑,料想走不脱身,只好安心住下。

这一夜,王佩兰尽力应酬,倾心巴结;双钩抱月,半面偎云;花飞锦帐之春,水满蓝桥之路。若换了差不多些的客人,早已被他迷得丧心失志,当不得章秋谷歌场酒阵阅历多年,那一样事儿没有见过?近数年来,更是结束铅华,屏除丝竹,差不多就有些杜司勋梦觉扬州、王摩诘西风禅榻的光景,不过是借着这载酒看花,消遣那牢骚郁勃,所以凭着那王佩兰如何做作,只是淡淡的勉强应酬。看看佩兰的一片虚情假意,反觉得有些惹厌起来,越发把一个陈文仙深深的印入脑筋,竟有些儿丢撇不下。正是:

疑云怨雨,缠绵宋玉之情;金枕银环,辜负丁娘之索。

不知后事如何,且听下回分解。





第四十六回 争闲气怒掷缠头 恶跳槽气伤名妓





却说章秋谷在王佩兰院中住了一夜,明天不到九点钟时候,秋谷已自起来,佩兰也便惊醒,见秋谷起身,连忙也揉一揉眼睛,跨下床来,不肯再睡。秋谷暗暗的好笑,便披上长衣匆匆要走。王佩兰一手拉住,道:“故歇辰光,耐要紧到啥场化去”就是要去看唔笃格相好,晏歇点也正好勒啘。耐看耐格辫子,啥格毛得来实梗样式,阿要倪来搭耐打条辫子,吃仔点心,慢慢交去末哉。“

秋谷本要径到陈文仙院内去看他的病,看看钟上还不到十点钟,也觉得似乎太早,料想他们还没有起来,便点头应允,就在窗口藤椅上坐下。王佩兰取了牙梳发篦过来。立在秋谷身后,替他慢慢的拆开,先梳通了头发,又用发篦编了一会,然后编起辫子来。编好之后,又用刨花水刷了又刷,直把秋谷的一条辫发刷得没有一根乱丝,黑漆漆的宝光如镜,方才完事。又问秋谷要吃什么点心。秋谷道:“还是去叫碗面来的好。”佩兰晓得他平日爱吃九华楼鸡丝面,便叫相帮到九华楼去,叫了一碗钱六分的生川鸡丝面来。秋谷吃了,王佩兰便坐在秋谷旁边,对镜梳洗,却把一个身子斜倚在秋谷身上,低声笑道:“倪搭耐打格辫子阿好?勿是倪来里说,别人阿肯实梗呀?”

秋谷见王佩兰睡态未消,余香犹腻,娇波流慧,顾盼生妍,不由的心中一动,暗想:“王佩兰这般姿态,也算蛾眉队里一个出色的人材,可惜他看待客人没有一些儿良心,只晓得一味的混敲竹杠,将来一定没有好好的收成。”想了一会,方才立起身来。王佩兰挽留不住,又咬着耳朵叮嘱了一番,叫他晚间务必要把金水烟筒带来。秋谷微笑答应,出了王佩兰家门口,径到陈文仙家来。

走上扶梯,相帮高叫一声,只见陈文仙的娘姨宝珠姐蓬着头走了出来,正和秋谷打个照面,登时满面上堆下笑来,道:“咦,二少爷多日勿来哉啘,倪先生牵记得耐来勒浪生病,房里向去坐嗫。”推着秋谷的背,进房坐下。

陈文仙本来尚未起身,被宝珠姐在外间说话惊醒,听得秋谷到来,心中大喜,便坐起身来。秋谷见文仙已经坐起,一直到床沿坐下,握着文仙的手正要问时,只听得文仙先说道:“二少爷,耐一径勿来,倒好意思格?”说到此际便顿住了,不说下去。秋谷看他云鬓忪惺,不施脂粉,果然消瘦了好些,心上好生怜惜;要想几句安慰他的说话,却急切里一时想不出来,只紧紧握住他的手,彼此默然。文仙又道:“倪是一径朆待差歇耐,耐别地方去做仔相好,倪搭勿来末,只要凭耐格良心末哉。倪做客人总不过实梗样式,呒拨啥格别样花头,勿像别人有多花迷人格功架。”

说着又低下头去,玉容寂寞,眉黛含颦,大有凄凉之态。秋谷觉得甚是过意不去,只得着意温存了一会,文仙方才有点笑容。

秋谷问他可有什么不快,文仙道:“倪人是倒也无啥,就是心浪向勿舒齐,勿晓得啥格道理。”一面说着,便走下床来。秋谷直候他梳洗完了,方把王佩兰敲竹杠的一层情事,细细的告诉了陈文仙。文仙听了,心上自是畅快,面上却冷冷的道:“晤笃两家头实梗格要好,耐去搭俚打一支金水烟筒也无啥要紧啘。”章秋谷知他醋意未消,便抱着文仙坐在膝上,密密的说了一回。文仙面有喜色,故意说道:“格是耐自家情愿格,勿半得倪啥事,勿要隔仔两日,再要说倪敲耐格竹杠。”秋谷连连摇手道:“你只管放心,我难道肯说这样的话么?”文仙方才不说。

秋谷到得天晚,便到杨庆和银楼去了一趟,把那昨天定打的金水烟筒取了回来,共是十四两金子,连工钱在内,合要七百三十块钱。秋谷带了金水烟筒,却不到兆贵里去,一直到吉升栈来,把烟筒交代当差的,又教了几句说话,方到兆贵里来。

王佩兰见秋谷进来,仍是一双空手,不觉登时变了面色,连忙问道:“金水烟筒啥勿搭倪拿得来?”秋谷道:“我刚刚去了一趟,要停一会儿方有,我叫当差的在那里坐等,一直拿到你这里来。今天决不绰你的烂污,你放心就是了。”佩兰听了,方才转过面皮,笑逐颜开,春风满面。这一刻时候,王佩兰恨不得要把章秋谷心坎温存,眼皮供养,要哄他这一支金水烟筒。

秋谷坐了一会,向佩兰道:“我今天本想要请几个客人,就此刻吃了一台罢。”

佩兰更是欢喜,连忙关照下去。秋谷一面写票请客,一面叫摆起台面来。不多时,请客已经来了,写好局票交与相帮,大家入席。秋谷却添叫了一个陈文仙。王佩兰看见,连忙伸手过去,把那一张局票抢了过来,撕得粉碎,口中咕噜道:“耐说陈文仙搭勿去哉,故歇为啥要去叫俚格局?”秋谷笑道:“你不用这般着急,我为今天客人太少,叫的局又不多,所以多叫一个,台面上热闹些儿,并不是要再去做他。”

王佩兰嗔道:“倪勿要呀,耐末总是实梗。”秋谷暗暗好笑,便把王佩兰拉了过来,低低的说了几句,佩兰方才依了。秋谷又重写一张局票交代下去。不多时,陈文仙已经来了,走进房内叫了一声,便默然坐下,一言不发。秋谷只顾应酬客人,并不理会。王佩兰见此光景,心中暗喜,倒与陈文仙问答几句。秋谷摆了二十杯庄,要人代酒,方回头过去,将两杯酒递与陈文仙。文仙一气饮干,王佩兰也代了几杯。

这一席酒,不觉已吃到十点多钟,将近散席。王佩兰等来等去,候了多时,不见当差的到来,便伏在秋谷肩上,悄悄的问他:“为什么金水烟筒还不送来?”秋谷故意诧异道:“这奴才真是没要紧,为什么还不赶紧送来?此刻已经十点多钟,大约也差不多了。”说着,早搬上干稀饭来,大家随意吃了些儿,起身散座。其时叫来的局已经散尽,惟有陈文仙催了几趟转局,兀自坐着不走。王佩兰看看陈文仙的面孔,着实诧异,连那班客人也奇怪起来。

王佩兰正和秋谷在那里附耳密谈,陈文仙立起身来要走,秋谷一把拦住道:“慢些儿,我还有话说。”文仙佯嗔道:“台面也散哉,独剩仔倪一干仔,坐来浪算啥嗄?”秋谷道:“你为什么这般性急,难道说一句话的功夫都没有么?”文仙方立住了脚,问道:“有啥格闲话,豪燥点说嗫。”秋谷尚未开口,只见门帘一起,当差的高福走了进来,手中拿着一支金水烟筒,黄澄澄的辉煌夺目。王佩兰一见,喜得娇含杏靥,笑晕梨涡,那搓酥捏粉的脸上,喜孜孜现出两朵红云,粉融融添了一团春色。轻移莲步,走近前来正要伸手去接,高福把身子往后一退,载过身来交在秋谷手中。王佩兰觉得有些没趣,见秋谷把金水烟筒接在手中,王佩兰的一双俊眼,就跟着秋谷的金水烟筒周围乱转,心上早突突的跳起来,眼花撩乱的看不清楚。

定了一定心神,方才看见秋谷手内的那一支金水烟筒,打造得十分工细,雕镂精巧,光彩照人。修甫等也走近前来一同观看,都说果然打得不差,大家心上都觉得章秋谷此举有些瘟气。只有贡春树心中暗想:“秋谷平日时常说别人是个瘟生,如今轮到自家身上,也做起瘟生来了。可见得‘色’之一字最易迷人,章秋谷这样的花丛老手,都受了他的圈套,其余的人可想而知,更不必说的了。

正在彼此疑惑之际,只见秋谷笑问王佩兰道:“你看这一支烟筒何如?”王佩兰此际得意已极,并不言语,只笑着点点头。秋谷又回过脸来问陈文仙,陈文仙道:“打工倒无啥,倪看也无啥希奇。”秋谷一笑。王佩兰却瞅了陈文仙一眼,微微冷笑,大有看不起他的样子。不提防秋谷把那一支水烟筒,竟自递在文仙手内,向他说道:“我自从做你,将及两年,从来没有敲过我的竹杠。我如今送你一支金水烟筒,好等那一班专爱银钱、死敲竹杠的倌人看个样儿,我姓章的并不是不肯出钱的客人。”文仙把金水烟筒接在手中,笑迷迷的道:“谢谢耐,晏歇请过来。”说罢也不作别,往外便走,三脚两步的去了。

王佩兰万料不到章秋谷使出这一着棋子来,见了这般光景,这一来,就是那石破天惊,云垂海立,也没有这样的惊奇。这一气非同小可,真似那冷水淋头,闷雷击顶一般,直把一个王佩兰气得来脸泛秋霜,眼流珠泪,面青唇白,半晌不言。到了这个时候,方才懊悔自家差了主意,不该一味的混敲竹杠,做出那一付神情,恰恰的钉头碰着铁头,遇着了个花柳惯家、温柔名手的章秋谷。竹杠没有敲成还在其次,偏偏的章秋谷把陈文仙叫了过来,千不给,万不给,单单的给了陈文仙,还带着把王佩兰骂了几句,燥燥他的脾胃,叫他在房看着,心上已自难过,当着这大庭广众之中,彼此相形之下,你叫那王佩兰的面上怎生的下得来?

辛修甫等大家看了章秋谷这样的作为,一个个方才心服,未免众人的视线一齐逼到王佩兰身上,看得佩兰愈加惭愧,满面飞红。待要和秋谷不依,却又不好怎样。

那一时的神景实在好看。秋谷本意原要待陈文仙走后,对着众人尽情把他数落一番,好叫他自家懊悔;现在见王佩兰这般模样,面红头胀,珠泪双垂,又觉得有些不忍起来。想着那定情之顷,山盟海誓,何等缠绵,毕竟有些怜惜,便也不去合他多话,把手招招众人,起身便走。又似笑非笑的向王佩兰道:“但愿你以后多做几个阔客,不要像我一般。我留心看你就是了。”佩兰正在气得发昏,听了也没有什么说话。

秋谷便同着一班朋友走了出来,一直就走到陈文仙院内。文仙接进房中,自是欢喜。

贡春树说:“秋谷这件事情未免太过些儿。王佩兰虽是不该混敲竹杠,你也不应这样的反面无情,究竟你和他总算有过交情。凡事须要将就些儿,为什么这般刻薄?”秋谷听了也有些自悔孟浪,便道:“我生平作事,无论什么事情,专要取那一时的快意,过后也觉得过分了些。”众人谈了一会各自散去,按下这边。

且说方子衡回去之后,留下家人刘贵住在陆兰芬院中,痴心妄想陆兰芬过了中秋,还清债项,便好和刘贵同到常州,一心一意的嫁他了。那晓得上海的红倌人,不是轻易招惹得的,何况是金刚队里坐第一把交椅的陆兰芬。枇杷花下,车马如云,三千选佛之场,十万缠头之锦,那一班坠鞭公子、走马王孙,落了他的圈套,要娶他回去的人,也不知多多少少,那里把一个方子衡这样的曲辫子客人放在心上?大凡上海倌人的外交政策,差不多都是一般,无论见了什么客人,只要一有交情,就满口的山盟海誓,定要嫁他。及至客人被他灌了迷汤,入了他的圈套,他却只要银钱到手,就登时翻转面皮,把那以前的被底风情、枕边盟誓一笔勾销,好似素不相识的一般,也不管客人的死活。其实倌人见了客人,起初也不是有意奉承,后来也不是负心背约,总而言之,都是堂子里头照例的事儿,算不得什么丧心负义。你想他做了妓女,吃的本来就是这碗饭儿,不骗客人的钱,却骗那个的钱,难道要他自己赔钱不成?所以堂子里的倌人做了客人,那倌人的说话行为千篇一律,就如一个模子里头印出来的一般,跳不出这个圈子。

依着在下的意见想来,倌人们哄骗客人,却也怪他不得。为什么呢?他们既做了这行生意,自然就要指着生意开销,若要对着客人说起真话来,那里还有什么生意?这哄骗客人,岂不是他们应尽的义务么?最可恶的是那一种嫁人之后,复又出来重做生意的人。你想既已嫁人,便是良家妇女,如何又要下堂求去,重新做起生意来?这便是他生成贱骨,爱落风尘,拔超不出的了。在下这一番议论,原是凭着自家的意见,一时拟议之谈,未知看官们以为然否?

闲话休提,书归正传。只说方子衡把刘贵留在上海,住在兰芬院中,一天到晚没有一些事情,正是两餐老米饭,一枕黑甜乡。不觉过了几天,那刘贵实在无聊到极处,便和那些相帮随口闲谈,说到他主人方子衡要娶陆兰芬,两下已经说定,所以主人把他留在此间,好同兰芬回去的一层说话。那班相帮听了,你看着我我看着你,大家冷笑一声不来理会。刘贵看见这般光景,免不得疑惑起来,便向那班相帮迫问。相帮等那肯说明,只是不住的冷笑。刘贵打听不出,晓得事有蹊跷,暗想方子衡临走的时候曾经分付过他,要他一过中秋便把陆兰芬同回家去。现在这个样儿看着有些不像,心中着实慌忙。正是:

惆怅温郎之镜,天上人间;重寻渔父之津,落花流水。

未知陆兰芬后来究竟肯嫁方子衡与否,试听下回。





第四十七回 负心郎黄衫求作合 薄命女紫玉竟成姻





却说刘贵见兰芬的样儿不像,未见得肯嫁人,心上不免着急起来,只得候陆兰芬起来之后,正在对镜梳头,一步步的踅上楼梯,走到房内,立在一旁。正要开口,兰芬早已看见,故作不知,问他道:“耐是啥场化来格?倒倪搭阿有啥格事体?耐有啥闲话,到帐房里去说嗫,啥格一直跑到仔房间里向来?”刘贵听了兰芬的话,不觉呆了一呆,心上明知不好,只得说道:“我就是方大人留在这里的家人,怎么又不认得起来?”兰芬听了,方才笑道:“噢,原来耐就是方大人搭格管家,倪倒像煞勿认得哉。”娘姨在旁边插口道:“俚耐住来浪倪搭呀,住仔好几日哉。”兰芬听了点一点头。又向如贵道:“唔笃大人阿要几时出来,倪倒牵记煞来里?”刘贵听兰芬的话不是头路,更加慌了,便道:“我们大人临走的时候把我留在此间,叫我过了中秋就要把先生送回家去,难道他没有说明么?”兰芬故意摇头道:“倪陆里有功夫到常州去?俚耐走格辰光,也朆搭倪说过歇啘,就是实梗妈妈虎虎要叫耐同倪转去?”说得这一句,就鼻子里哼了一声,回头向背后梳头的娘姨道:“阿要一厢情愿?”刘贵听陆兰芬说出来的话愈加不对起来,把一个刘贵说得急了,便直说出来道:“我们大人没有动身的时候,你自家亲口答应定要嫁他,还要我们大人替你还债,所以才把我留在上海,要把你同转常州。说得明明白白的,怎么现在又忽然变卦起来?”兰芬听了,“嗤”的笑了一声道:“阿是我陆兰芬嫁拨唔笃大人实梗容易?老实对耐说仔罢,倪堂子里向见仔客人,生来才是实梗样式,无啥稀奇。倪吃仔格碗把势饭,碰碰就要嫁起人来,也呒拨几化客人来浪嫁啘。唔笃格大人阿,勿是倪勒浪说俚,直头是格伉大,一句闲话就要当倪格真。耐想倪堂子里说出来格应酬闲话,阿好作准?倪就是要嫁人,也呒拨实梗容易啘!”兰芬说毕,不觉又好笑起来。

刘贵听了这一番言语,好似顶门上浇了一桶雪水下来,方知果然是自己主人入了陆兰芬的圈套,无可奈何,又勉强争道:“你既然不肯,为什么要满口应承,有心哄骗?何不早些回复了他?”兰芬又冷笑道:“倪做仔生意,生来要应酬客人。

俚一团高兴,要付倪转去,倪阿好勿答应,坍俚格台?老实说,倪嫁起人来,像唔笃大人格号客人,勿见得靠得住。耐去想哩,唔笃大人一塌刮仔几十万银子格家当,也勿算啥格大家私。再说起功名来,一个候补知府,加二挨俚勿着。倪搭格客人,比仔唔笃大人再要阔点,想讨倪转去格多煞来浪,倪眼睛角落里向稍也朆稍着,勿要说啥唔笃格大人哉。“

这几句,把刘贵说得哑口无言,又急又气,只得说道:“我原是奉上差遣,没有我的事情。但是你既已当面应承,现在又是这般变卦,叫我们当家人的回去怎样的销差?你也要替我想想才是。”兰芬道:“格号事体啥格销差勿销差?希奇勿煞!

耐转去搭俚说,有啥闲话末,叫俚自家来搭倪说好哉,勿关得耐啥事,倪总勿见得怕仔俚勒逃走,耐只顾放心转去末哉。“说着,又叫娘姨去衣橱内搬出一只小拜匣来。兰芬开了盒盖,检出六张十元的银行钱票,递与刘贵道:”格点小意思,请耐吃顿点心,耐转去就拿倪格闲话搭唔笃大人说末哉。“刘贵待要不接,明知无奈他何,只得伸手过来接了钞票,快怏的走下楼去,心中暗想:住在此间无益,只可赶早动身回去,禀了主人再作道理。又想:方子衡平日最敬重的是章秋谷,姑且去和他商议,或者有什么法儿也未可定。主意已定,便急急的走到吉升栈来寻章秋谷。

不料秋谷已经两夜不回,寻了几处地方,直到陈文仙院中方才寻着,见秋谷在房中正与陈文仙说笑。刘贵走进房去,请了一个安,垂手侍立。秋谷见刘贵进来,似乎有些认识,却模模糊糊的记不清,问道:“你可是在方大人那里当差的么?”

刘贵走上一步,答应了一声:“是。”秋谷问他可有什么事情,刘贵就把方子衡留他在此,并陆兰芬忽然变卦的缘由诉说了一遍。又道:“主人把家人留在此间,原叫家人要同着陆兰芬回去。现在他忽然变了口风,家人回去怎好销差?可好请章老爷想个法儿,家人实在不得主意。”说着又请了一个安。

秋谷听了,大笑道:“我早就料到这件事儿定有一番口舌。你们贵上那时正在迷惑之际,劝他一定不依,反要失了我们的和气。依我看去,陆兰芬忽然改悔起来,还是你们贵上的运气。他们堂子出身的人,那里受得人家的规矩?与其将来闹出什么笑话,坏了你们贵上名声,不如现在听他反悔的为是。你回去同你贵上请安,就说我劝他不必放在心上,痴心妄想的还想娶他。上海的倌人,不是轻轻易易的就可以娶回家去,万一将来闹出事来,那时就懊悔嫌迟了。你住在上海,也没有什么事情,莫若早些回去,免得你贵上等得心焦。”刘贵听了不敢多言,只好连声答应。

辞了秋谷,出得门来,想着章秋谷的话儿实在不错,只得到陆兰芬家取了铺盖,急急的回常州告诉方子衡去了。

闲话休提。只说章秋谷见刘贵去了,向陈文仙笑道:“天下竟有这般痴子,上了陆兰芬的恶当,花掉了银钱不算,还要把自己一个家人留在上海,想要把陆兰芬同到常州。在上海滩上要讨一个堂子里的倌人,那有这般容易?真是个世界之上有一无二的瘟生!”陈文仙也笑了一会。

不觉又过了几日,其时已是七月中旬,桂魄初生,金风未动,已经凉快了好些。

秋谷因离家已久,家中又连次信来催他回去。穷年索寞,旅舍萧条,虽然酒阵歌场,尽有温柔之梦,却是十年一觉,偏多落魄之悲,前路苍茫,华年似水,不免便有些张季鹰秋风莼菜之思。想要暂时回去一趟,随后再来上海,却又有些迟迟疑疑的自家作不定主意。

这一天正在栈内检点朋友往来的信札,已经聚了一大堆来信,多没有写回书,便拣那要紧的先写了几封。正要叫人去送,忽见贡春树闯了进来,形景仓惶,面有忧色,走进来一屁股坐下,也不言语,皱着双眉,好像有什么心事一般。秋谷觉得有些诧异,便追问贡春树到底为着什么事情这般着急,春树叹了一口气,走至秋谷身旁,附耳朵说了几句。秋谷笑道:“这也不算什么大事,我早已知道了,何用急得这个样儿?”春树顿足道:“在你看来,原没有什么希奇,只在旁边说两句现成话儿,可有什么用外?你不晓得这件事儿的关系,万一闹了出来,我怎的对人得起?

你以前答应我的话儿到底怎样,可有什么法子么?“秋谷冷笑道:”你既晓得对人不起,为什么一到上海,就拚命的乱吊膀子,混轧姘头?难道你这般胡闹,就对得起人么?“春树听了哑口无言,想秋谷的话果然不错,一时脸上红红的竟说不出话来。呆了半晌,见秋谷装着冷面不去理他,只得立起来走到秋谷面前,深深的打了一拱,道:”你向来是个极有血性的人,这件事儿总得替我想个法子,除了你,别人也没有这样的担当。“

秋谷起初推托不肯,当不起贡春树再三再四苦苦的求告,推辞不得,只得应了。

便道:“这件事儿我虽然应了下来,却又鲁莽不得,须得我自己赶到苏州方有把握。

但是你自己闹了乱子,却无缘无故的要我来替你张罗。你的朋友甚多,为什么单要寻我,不去照顾别人,这是什么讲究?“春树怕他又要改口,再三央告,急得几乎要流下泪来。秋谷方笑道:”论起理来,我们读书子弟不应去做这样事情。但是据你说来,若不趁早想个法儿,一定要闹乱子,这也只好急则治标,从权些儿的了。“

春树听了大喜,举手称谢。秋谷又道:“我既然应了,也不必耽误日期,明天就好动身同你一同前去。但想个什么主意,也要预先商量方好。”

正在打算,见茶房又传了一封信进来。秋谷看封面时,见是方子衡在常州寄来的。拆开封袋看,倒是方子衡的亲笔,写得歪歪斜斜的,白字连篇,那文理又似通非通的十分费解。秋谷甚是好笑,仔细摹拟了一回,方才略略懂得他的大意。

原来方子衡赶到家中,他父亲的病居然好了些儿。这方子衡虽然是勉强在家,却一心一意的记念着陆兰芬,一刻也放他不下,觉得那陆兰芬声容笑貌没有一天不在他心目之中,差不多竟是害了单思病,恨不能一刻儿飞到上海来,好和那意中人会面。无奈他父亲有病,不得脱身,只把个方子衡恨得咬着牙齿,咒骂不已。正在那梦魂颠倒、胡思乱想的时候,不料那刘贵赶了回来,一五一十的把陆兰芬的说话直言拜上,不曾掉了一些,说到高兴的地方还要添些枝叶。这一下不打紧,把方子衡气了一个发昏,想来想去没有什么法儿,叹了几口冷气,只得罢了。却又痴心不断,自己写一封信给章秋谷,要请他去问那陆兰芬为什么无故变卦。

章秋谷看了他的来信,微笑一笑,把信递与贡春树道:“你看竟有这样到死不悟的瘟生,我那有功夫去碰陆兰芬的钉子?”春树把信接在手中,还没有到眼,听见秋谷说到陆兰芬三字,不及看信,连忙向秋谷道:“说到陆兰芬,你可晓得陆兰芬已经死了么?”秋谷吃了一惊,急问道:“那有这般奇事,可是真的么?不要是外头的瞎话,为什么我这里没有风声?”春树道:“确而又确,还是昨天半夜的事情,我今天早上听小宝家的相帮传说,方才知道,断断不是传来的谣言。并且我还听见相帮们自家议论,说兰芬身上的亏空倒有两万多些,听得兰芬死了,一齐赶到,有的还去投报捕房,现在不知怎么样了。”秋谷听了,料想是真,因子日间兰芬和他虽然没有交情,却是相待甚好,现在听他死了,不觉有些心中酸酸的不忍起来,便又问春树道:“你可晓得他是什么毛病,就死得这般快当?”春树道:“我也弄不清楚。好像听他们说发痧刚好,夜间留了一个客人,登时反复,霍乱吐泻的发作起来,不到一天功夫便断了气,却不晓得究竟如何?”

秋谷听了,便拉贡春树作伴,要同到兰芬院内去看看那班债主怎样的开销。春树应允,立刻同出栈门,到兰芬住的大洋房来。走到门口,只见有一个印度巡捕立在门内,那出进的人纷纷不绝。秋谷便同着春树纵步登楼。往日间走上楼梯,便有娘姨应客,雏婢呼茶,青琐回灯,湘帘卷月,真个是桃花门巷,杨柳楼台。如今章秋谷走上楼来,那些旧日的娘姨大姐一个不见,鼻观之中,只闻得一股纸钱灰气直逼进来,那里还有什么花香人气?正是:

风月依然,倾城何处?惆怅昙花之影,燕子楼空;凄凉倩女之魂,华清梦醒。

秋谷忍不住一阵心酸,勉强忍住了,走到房内,见大床上的帐子已经卸去,直挺挺的躺着陆兰芬,那生前如花如玉的丰神,宜喜宜嗔的态度,不知往那里去了,只觉得口开目闭,形状怕人,身上直穿着一身半旧的竹布衫裤。秋谷别转头去不忍再看。房内的衣橱、箱子一齐贴着封皮,客堂内有一簇人在那里纷纷议论。有一个人把一本账簿摊在桌上,在那里不知写的什么,想就是兰芬生前的债户了。

秋谷正在徘徊感慨之际,忽见人丛内挤出一个人来,把秋谷一把拉住,大哭道:“二少,耐看看难末叫我那哼?”秋谷吃了一惊,急看时,原来就是陆兰芬的亲生娘,泪流满面,头发蓬松。秋谷见了也不禁恻然,只好将就安慰他几句。兰芬的娘哭道:“俚耐刚刚死得勿多辰光,就有几化格债户同仔巡捕房里向格人赶得来,一塌刮仔格物事,才上仔封皮,动也勿许倪动,说是要拍卖仔洋钱替俚还债。故歇洋钱末呒拨,借也无借处,叫我那哼弄法?”说罢又大哭起来。秋谷心上十分酸楚,只得对他说道:“兰芬生前虽有许多亏空,要拿他的衣裳首饰拍卖抵偿,却照例要另外留出一分作为治丧的费用。事已如此,你也不必这样伤心,我们一班和兰芬素来要好的人,只要可以帮忙的地方,没有不尽力的。”说着便向身边取出一卷钞票,点了一点;又问贡春树身边可有钞票,春树连声说:“有。”便也取出一卷来递与秋谷。秋谷接过来看一看,检了几张,和自己的合成一百块钱,把余多的仍旧还了春树。正是:

红颜薄命,伤心天宝之歌;黄土埋香,肠断真娘之墓。

要知后事如何,但听下回分解。





第四十八回 章秋谷惊散野鸳鸯 霍春荣排演花蝴蝶





却说章秋谷闻得兰芬病死,甚是凄然,拉着贡春树同去看他。遇见了陆兰芬的亲生娘,拉住秋谷放声大哭,秋谷十分不忍,给了他一百块钞票,叫他凑着使用。

兰芬的娘千恩万谢的接了,又道:“倪囡仵活浪格辰光,客人笃来来去去,格末叫忙;故歇俚死仔是,格排勿要面孔格客人,勿要说啥帮倪格忙,连搭仔欠来浪格局账,一塌刮仔漂脱。像耐二少实梗好人,故歇陆里再有呀!”秋谷听了,转觉心酸,痛紫玉之成烟,感华年之似水,彩云易散,情海难填。再想起自家的际遇来,身世飘零,江湖落拓,阮步兵驱车痛哭,李谪仙酒肆逃名,登广武而欷歔,望中原而叹息,易求骏足,难遇孙阳,把自己的一腔抑塞一齐都提上心来,再也存身不住,急急的同着春树下楼。

兰芬的娘还想挽留,秋谷那里肯住,一路出了大门,对着春树叹口气道:“这就是他们名妓的下场,真是不堪回首。想那陆兰芬在生时节何等锋铓,差不多些的客人倒反要仰承他的颜色。他的枇杷门巷差不多竟像个督抚衙门,车马如云,往来不绝。只为他吃惯了堂子饭儿,做不来良家妇女,倚仗自家的色技不肯嫁人。这般的一个有名妓女,今日之下却弄得这等的收场,真是可怕!”说着不觉得言下怆然,春树也叹息不已。

一面走着,顺路到迎春坊金小宝家,和他说明要去苏州的缘故,一礼拜就可回来。金小宝初时不肯,后来经秋谷帮着解说,方勉强应了,但向春树道:“耐去仔要豪燥点来格嗫。倪也无啥闲话,来勿来听耐自家格良心。”春树连连答应。

秋谷又讲到兰芬死后的情形。金小宝兔死狐悲,物伤其类,免不得挥下几点泪来。秋谷又道:“他若趁着方子衡没有回去的时候,安安稳稳的嫁了他,不要一味地乱敲竹杠,如今死了倒也博得些儿死后的风光,不至于弄到这般地位。可见你们吃堂子饭的人总以嫁人为是,只看陆兰芬这样的收场,也该觉悟回头,惊心动魄。

你想做男人的沉迷不醒,尚且每每弄得荡产倾家、身名扫地,何况你们是个倌人?“

金小宝不等说完,便截住道:“耐格闲话自然勿错,不过倪想起来,各人有各人格打算,倒勿在乎嫁人勿嫁人,只要自家有点主意好哉。倪格排人要嫁起人来,格末叫讨气。唔笃去想哩,好好交格人家,啥人肯讨格倌人转去做大老母?推扳点格人家,倪又勿肯嫁俚。就算嫁仔一格好好里格人家,也不过一个小老母,总归有多化勿称心格地方,阿是也呒啥趣势?”

秋谷听了,觉得他的道理倒也不差,便问他道:“依着你的意见,不嫁人便怎么样呢?”小宝道:“倪从小头里吃仔格碗堂子饭,身体散淡惯哉,再要去做格人家人,像煞受勿来俚笃格规矩。只要自家有点主意,生意浪多点洋钱下来,勿要去贴啥格马夫、戏子,像俚笃实梗欠得一塌糊涂,自家阿有啥格好处?现在格世界,只要有仔铜钱,样式才办得到。倪有仔钱铜,就是勿做生意也无啥希奇啘。再要做起客人来,老老实实点,勿要去拨俚笃吃啥格空心汤团,到仔着末完结,总归原要出来,拨别人叫声老枪,也无啥好听啘。二少耐说阿对?”章秋谷听了不住的点头,道:“你这个主意倒也不差,真是有些阅历,并不是同他们一样一味的哄骗客人。

想不到你竟有这般见识,也算是庸中佼佼的了。“

秋谷说罢又向春树道:“你既要同去,赶紧去雇一只中号快船,好叫轮船拖带;到了苏州便好住在船上,省得住在岸上,露了风声不是玩的。”春树诺诺连声。

秋谷便到兆贵里去坐了一回。陈文仙出局未回,觉得无趣。起身出院,想到新马路辛修甫公馆内去看他,并和他说一声要暂去苏州耽搁。因修甫这几日有些小恙,知他在家养病,并不出门,便坐上包车径到新马路昌寿里来。

修甫在家正是独坐无聊,见秋谷来了心中甚喜,留他吃了晚饭,又谈了一回。

秋谷把要去苏州的话向他说了,修甫问几时回来,秋谷道:“说不定,或者一礼拜内就可回头。”说着,听见自鸣钟当当的已敲了十二下,便辞了修甫坐车回去。

那车夫因时已不早,拉着车子飞一般的向前直走。刚到新马路转弯之处,秋谷坐在车上,见有两三个人在跑马厅迎面走来。一个个不着长衫,都是官纱短衫裤,也有生丝裤衫,一齐散着裤腿,走起路来摇摇摆摆,凸肚挺胸。秋谷看得明白,晓得定是一班流氓,不去理会。那车夫拉着包车,腾云驾雾的一般跑过头去。秋谷忽听得背后那班流氓,口中高高的打了一个哨子,又把掌心击了一下。秋谷分明听见,疑惑起来,低低的叫车夫停下车子,从黑影里绕过头去看时,只见那几个流氓正立在转弯角上,对着一座洋楼。那洋楼本是个姓王的铁路委员租的公馆,沿着马路,两间楼面,湘帘不卷,隐隐的露出灯光。秋谷看了,恍然大悟,晓得那班流氓方才的哨子是个吊膀子的暗号。秋谷平日本来爱管闲事,索性立住了看他究竟如何。又见那班流氓等了一会不见动静,悄悄的说道:“咦,倒诧异笃啘。”便又打了一声哨子,比先前高了好些。秋谷一声不响,隐在黑影里偷看他们。这班流氓那里晓得有人窥探,只眼睁睁的看着楼上,目不转睛。

不多一会,果然那湘帘里面影影绰绰的映着灯光,露出一个人影,揭起帘缝,倚着栏杆,往下张看。秋谷在暗处看得真切,是个二十余岁的妇人,那身材态度,觉得甚是苗条,面目虽不甚清楚,却也红腻桃腮,绿堆云鬓。约略看去,不是什么粗蠢的人材。秋谷正在细看,又听得呀的一声,那两扇大门轻轻的开了一扇,走出一个小大姐来,看来只有十三四岁的样子,低低的说道:“里向去哩。”那流氓之内便有一个正要举步进门。秋谷看了多时,早已心中忿忿,暗想这班流氓引诱良家妇女,真是死有余辜。便忍不住咳嗽一声,在黑影里急抢出来,喝一声:“且慢!”

那班流氓出其不意,大吃一惊。那个开门的小大姐更是吃吓,急急的把大门关上,也顾不得那班流氓,七跌八铳的逃了进去,连那楼上的妇人,也吓得回身进去,连忙把两盏点着自来火的灯一齐集灭。一霎时玉钩全下,帘影沉沉。秋谷看了十分畅快。

那班流氓见破了他的道儿,心中大怒,一齐回过身来要与秋谷寻事;及见秋谷身上衣裳穿得甚是齐整,不觉呆了一呆。一个流氓便开口喝道:“你是什么人,为什么鬼头鬼脑的掩在黑影里头?”秋谷未及答应,又一个流氓插口道:“看他这个样儿,深更半夜不声不响的掩在这里,一定不是个好人。”说着七手八脚的齐赶上来。看着秋谷的一身衣服华丽非常,又有金边眼镜,钻石戒指,着实值几个钱,众流氓看得垂涎起来,倚着新马路地方冷静,大有攫取的意思。还未动手,早听得章秋谷哈哈冷笑道:“你们这班不知进退的流氓,我还没有盘问你们的来历,你倒反来问我起来。我正要问你,你们这班不三不四的人,半夜三更在人家公馆门前探头探脑,口内还打暗号,做的什么事儿?你们可懂得租界的章程么?况且我走我的路,与你们什么相干,要你们来多嘴?你们趁早的与我走开便罢,不然,把你们送到捕房,问你们一个引诱妇女的罪名,看你们可吃得起吃不起?”

众流氓不听犹可,一听章秋谷这番说话,一个个顿时大怒起来,嚷道:“你倒说得这般容易,要把我们送到捕房,真是你自己不知进退。你既说这般大话,我们且叫你吃些眼下的现亏,先打你一顿再说。”说着不由分说,两三个人一齐拥上。

一个身材高大的流氓抢上前来,先把秋谷劈胸一拳,秋谷不慌不忙,霍地闪过,扑的一个箭步早已跳在旁边。那流氓那里肯舍,当先赶上,照着秋谷的脑袋又是一拳下来。秋谷把左手轻轻一格,觉得也似乎有些力量,便顺着他的来势,右手劈胸一拳。这一下来得势猛,那流氓站脚不住,踉踉跄跄的直跌出去。又有一个流氓上来,想要扭住秋谷的胸前衣服。秋谷也不躲闪,趁势把他胁下一掌,也便滚在一旁。这一来,把后面第三个流氓吓得不敢动手,眼睁睁的看着他。秋谷甚是好笑,正拔步要走时,不防那抢先动手的流氓却也懂得些儿拳棒,见秋谷手势伶俐,知是惯家,便在地下一溜烟爬起身来,趁着秋谷走过身旁不及提防之际,把身子一伏,俯身下去,就想要挤秋谷的肾囊。果然秋谷轻看他们,毫不防备,见他来挤肾囊,吃了一惊,招架不及,把左足腾开一步,就地飞起右腿,正踢在那流氓肩窠之上。用得力猛,把那流氓踢得直掼开去有四五步远近,觉得好似踢折了肩骨一般,这一痛直钻入心窝里去,那里挣紥得住?由不得高声喊叫起来。

秋谷见他喊叫,倒吃了一惊,恐怕巡捕到来。马路上的规矩,同人相打,两造都要同入捕房,岂不失了体面?急急的四边一看,幸而还好,正是十二下钟巡捕换班的时候,落班的已经去了,接班的尚未到来。暗暗的叫了一声“惭愧”,急忙三脚两步跳上车去。那班流氓已经被他打怕了的,谁敢上前拦阻?眼睁睁的看着秋谷车子飞也似的跑了,转眼之间不见踪影,也是这些流氓的一个小小报应,只好自认晦气,被他白打了一场罢了。

且说章秋谷坐在车上,沿路喝叫车夫快走,一直到陈文仙家,心上甚是高兴。

陈文仙见他这般快活,问他为什么事情。秋谷把方才的事告诉了他一遍,倒把个陈文仙吓得粉面通红,埋怨他道:“耐末总是实梗,呒拨仔格清头。俚笃来浪吊膀子,关耐啥事?要耐去管俚笃格闲帐。结仔冤家还勿算数。倘忙真格拨巡捕拉仔巡捕房里去,阿要坍台?”咕咕噜噜的埋怨一个不住。秋谷始而大笑,笑他的胆子忒小;后来仔细一想,他的说话倒也不差,倘然真被巡捕拉到捕房,等到问明白了,连忙释放出来,已是失了自家的体统,何苦把自家的名气去拚那班不要脸的流氓?如此一想,便觉有些后悔起来。又兼陈文仙坐在秋谷身上,挽着他的手,不住的揉搓,口内埋怨道:“倪勿来格,难下转勿要实梗,闯仔穷祸,呒啥人来替耐,阿晓得?”

秋谷见陈文仙一片天真,深情缱绻,转着实安慰了他一番,又答应他此后不去闹事,文仙方才罢了。一夜无话。

明日秋谷起来,要回栈去检点行李。文仙叮嘱他早去早回,秋谷答应。刚刚起身要走,文仙叫住道:“慢点去看嗫,倪有闲话说呀。”秋谷又回来坐下,问他有什么说话,文仙看着秋谷的面孔,看了半晌却说不出什么话来。彼此相对了一刻,文仙道:“倪像煞有几几化化格闲话来浪心浪,要搭耐说,不过好像心浪横七竖八格勿好过,勿知说仔陆里一句格好,故歇直头一句也说不出,耐总归豪燥点转来就是哉。”秋谷听了,似觉得也有些儿惆怅,又吩咐了文仙几句,方才走了。

秋谷回到栈内收拾带去的行李,因为天热,只带一个皮包,装着几件替换的衣服,一条番席,一个气枕,都塞在皮包里头;又带一只考篮,放些笔墨书本。又恐人多口杂,把两个当差的高福、顾升都留在栈中,叫他们小心照应。刚刚收拾停妥,贡春树早已到来,把物件发下船去。二人随后登舟,径往苏州去了。

看官且慢,贡春树要求秋谷和他设法同到苏州,到底是什么事情?自《九尾龟》初集之内,就是一个闷葫芦,直到如今尚未打破,这是什么体格呢?看官们且休性急,只把那《九尾龟》第三集第三卷内的一篇《懊恼记》细细的追寻,便有了七八分影子。且待在下做到四集,把这件事情的下落演说出来,好叫看官们心中明白,如今且说些时下编书的俗套,待看官们自家慢慢的揣摹。

闲话休提。且说章秋谷和贡春树二人到了苏州,把船便开到南壕,紧靠着一家水阁下边停泊。秋谷进城去访方小松。小松见了,故友相逢,十分欢喜,便一起同出阊门,到船上去见了春树。小松和春树都是一般的裙屐少年,见面自然投合。小松便邀秋谷、春树一同上岸,到新开的一家堂子高桂宝家小坐。

原来章秋谷自在苏州回去,不到半年,阊门开了马路,渐渐的热闹起来,那盘门青阳地的生意就登时冷落,所有的戏园堂子一齐搬到阊门外来。那先前的丹桂戏园因为折了本钱关了,现在新开了一家丽华。那盘门外的马路依然是景象荒凉,人烟冷落,只有上海轮船到了埠头,还有些儿市面,真个是盛衰一瞬,沧海桑田。秋谷打听分明,心上不由的顿生感慨;又问花云香、许宝琴的踪迹,方知许宝琴早已嫁人,花云香也回无锡,更觉怅然。

小松见他不乐,便请他就在桂宝家吃酒,好让他提些兴会出来。秋谷看高桂宝时,姿容娇小,态度玲珑,颇觉得动人怜爱,便欣然应允。小松又道:“你既到苏州,可晓得丽华园内新到了一个武小生霍春荣么?”秋谷喜道:“原来霍春荣到了苏州。此人我前在上海看见过他的戏,相貌既好,武功更是讲究,恰算是武行内一个出色的人材,但不知他今天唱什么戏。现在天已不早,我们先去看戏,再来吃酒何如?”小松道:“先去看戏也好,我们略坐一回便去。”桂宝听了,也要同去看戏。小松应了,叫他快些打扮。等得桂宝换了衣裳,重施脂粉,便一同坐了马车,同到戏园门口。下车进去,检一张正桌坐下,案目送上戏单。秋谷看时,恰好是霍春荣的《花蝴蝶》。小松也看了戏单,向秋谷道:“你可晓得这霍春荣的来历么?

他还是中堂的门婿呢!“有分教:

多情蝴蝶,春留枕上之香;懊恼鸳鸯,惊起花间之梦。

还有下文贝小姐包厢、霍春荣被捉、章秋谷夜盗红绡、王云生再拖骗局等许多节目,都在四集书中,请看续回,便知分解。





第四十九回 方小松演说风流案 贝夫人看戏丽华园





且说前回书中章秋谷同着贡春树、方小松,并带了高桂宝,同到丽华戏馆,要看霍春荣的戏。章秋谷坐定之后,检看戏单,见今天霍春荣排的恰好是《花蝴蝶》。

方小松向章秋谷说道:“你可晓得霍春荣的历史么?他还是中堂的门婿呢!”章秋谷和贡春树听了不觉大为诧异,章秋谷便问小松道:“怎么说霍春荣是中堂的门婿?

这句话儿我却有些不信,那里有这样的事儿?他既是中堂的门婿,为什么不去做官?

只要拿了他丈人的一封八行,那一省不好去当差署缺,还肯在苏州唱戏,做这种卑贱的勾当么?“方小松听了哈哈的笑道:”你这个人怎么这般老实,难道真个中堂的门婿肯来唱戏么?“秋谷也笑道:”既然如此,为什么你又要这样说呢?“

刘、松道:“这件事儿,说也话长,真是江苏省内唯一无二的新闻。待我慢慢儿的和你细说。”一面说着,就回过眼光两旁一看,把手指着一间包厢内道:“你看这里头坐的却是的的真真中堂的小姐、翰苑的夫人,这个新闻就出在他们府上,你在上海难道没有一点风声?”秋谷听了,不及回答小松,连忙转过眼光,跟着方小松手指的包厢里面仔细看去,只见包厢内坐着一位服御辉煌的中年妇人,旁边还坐着一个少妇。那中年妇人约莫有四十余岁,面上却还不甚看得出来,看着只像个三十多岁的样子。徐娘年纪,未褪娇红;中妇风情,犹传眉妩。那两只秋波水汪汪的十分活泼,就像那秋月无尘,春星照彩,明显着一付娇娆的态度出来。这样的妇人,若在少年时可想而知一定是个尤物。再看那旁坐的少妇时,更是冰雪为肌,琼瑶作骨,芙蓉如面,杨柳为腰。太真红玉之香,洛浦凌波之影,低鬟顾影,媚态横生。真是宝月祥云,明珠仙露,把个章秋谷竟看得呆了多时。又见他珠翠满头,纱罗被体,那头上的簪饰映着保险灯的光彩,珠光宝气,晔晔照人,背后更有许多俊俏青衣成群围列。那包厢之外,立着几个家人垂手侍立,肃然无声。

章秋谷看罢:方才向方小松道:“看他们这个样儿,一定是个贵家内眷。不过那神情意态,觉得甚是飞扬,眉目之间隐隐有些荡意。你怎么说他们府内出的什么新闻,快些把这件新闻的原委细细讲来,好待我们静听。”春树也异口同声的叫小松快讲。方小松微笑一笑,方才附耳低声,把这件故事细细的讲说出来。

看官,在下做到此间,只好把章秋谷一边按下,且把这件新闻一一的演说出来,好叫看官们不至茫无头绪。

闲话休提,书归正传。你道那厢房内的妇女究竟是何等人家的内眷?说将起来,来历却也不小。原来这中年妇人的母家姓余,他父亲名叫余颂南,翰苑出身,历任京秩,后来熬炼得资格深了,辈数老了,就荐升了刑部尚书,并在军机处赞画枢务,居然就是一位中堂。这余中堂生平只有一个女儿,十分溺爱。嫁与苏州贝太史为室,丰姿虽是娇娆,情性却甚为悍戾。偏偏这位贝太史又是个惧内庸夫,到了外边天不怕地不怕的人儿,一到进了自己的房门,看见了床头的这尊菩萨,便由不得神魂飞越,毛骨悚然。久而久之,这位贝太史便不知不觉的做了重生的陈季堂,再世的裴御史。贝太史自从点了庶常,也放了一任主考,不知怎的,外间物议沸腾,声名甚是狼籍,都说他出卖举人。至于这件事儿的有无,在下做书的当时并不在场,隔着一个省分,究竟是怎么一回事情,在下没有亲知灼见,却也不敢一定下什么断语。

只说贝太史的口碑传入都中,就被一个御史参了一本。那班京城里头的都老爷照例是这个样儿。若遇着那势焰薰天、威权炙手的人,凭着他怎样的卖官纳贿、枉法徇私,这班都老爷在一旁看着听着,都是袖手旁观,罚咒也不敢去动他一动。若有一个御史参动了头,还要窥测天颜的喜怒,要是皇上看了御史的参本果然震怒起来,免不得要传旨查办,这班都老爷得着了这个消息,一个个都发起狠来,你参一本,我参一本,大家都去射那死老虎。称想这个人既经参奏,已属是个待罪人员,何苦趁别人的热闹再去参他?这位贝太史就吃了这个苦头,给这班都老爷横参一本,竖参一本。那本上说的话儿,什么“似此败坏科场,贿通关节,若不从严查办,何以正士气而肃官方”。皇上看了这许多参本,从来说众口成城,自然也要震怒起来,便将原折发交浙江巡抚认真查办。

幸亏这位余中堂晓得这件事儿,心上虽然恨着女婿不该做出这样事儿,削他的颜面,却又看着女儿面上,不得不替他嘱托弥缝。这科场贿通关节的事儿,闹了出来不是顽的,就是从轻办理,也要问一个边远充军。余中堂无可奈何,只得替他上上嘱托,安顿了那几个原参的御史,又自己亲笔切切实实的写了一封信,托那浙江抚台替他辩护,方才把这一桩天字第一号的风波平了下来。浙江巡抚果然上了一个折子,替贝太史竭力辩护,无非是查无实据、合无仰恳天恩、免其议处的这些话头。

这个折子到了军机,又有余中堂在里头照应,方得从轻发落,把贝太史议了一个回籍闲住的处分。

贝太史回得苏州,刚刚进门,就被这位夫人指着脸儿痛骂了一顿,说:“你这样不要脸的东西,怎么竟敢这般大胆,连举人也卖起来?若不亏我父亲在京城里头同你竭力想法,这个时候只怕你这个狗头早已滚下来了。像你这样不争气的人儿受了王法,让我做了寡妇,到也干净些儿,省得你活在世上现眼!”把这位贝太史骂得满面羞惭,满心惶恐,低着头屏息而立,连哼都不敢哼一声。贝夫人骂了多时,见他不敢开口,也就消了几分怒气,到了晚间,贝太史少不得也要奴颜婢膝,陪着无数小心,方才哄得夫人欢喜。

自此之后,贝太史时常想起丈人的救命之恩,见了夫人越发怕得神出鬼入。更兼贝太史本来是个寒士出身,他封翁虽曾做过几年道台,家中却没有什么积蓄。你想一个当穷翰林的人,那里挣得起家产?刚刚巴得放了一任试差,又被那班不近人情的御史参了回来,依旧是两袖清风、一肩行李,渐渐的就有些支持不住起来。幸亏这位余中堂的小姐嫁过门来奁资丰富,足足的二三十万;他又善于居积,数年之内又赚了无数的利钱出来。他见贝太史手中竭蹶,金尽囊空,不免又要将他谩骂一场;骂过之后,索性不要他管了,自己拿出钱来供给贝太史的用度。贝太史乐得坐享其成,随意挥霍。但是贝太史现在的身家性命都是从老婆身上得来,家庭之内不得不曲意承颜,格外又加了二十四分恭顺。贝夫人的性气一天狠是一天,贝太史的惧内却一日甚于一日??怕老婆怕到极处。这贝夫人自然就趾高气扬、飞扬跋扈起来。

贝夫人将近中年,止生了一个女儿,却生得似玉如花,千娇百媚。贝夫人溺爱这个女儿,一言难尽,总而言之,也和余中堂的溺爱贝夫人差不多。

贝小姐到十九岁上,就嫁了一个常熟人姓彭的,也是一位太史公,家道十分寒素,相貌又甚不扬,更兼生性不羁,疏狂放荡,骄态逼人。贝夫人听了贝太史的话儿,又被媒人撺掇,便把一个心爱的女儿轻轻易易的许了这位彭太史,说定招赘进门,择了吉期,就把彭太史赘了进来。

贝夫人只道彭太史少年翰苑,定是个风流佳婿,蕴藉才郎。不料新郎官进得门来,贝夫人见他面目不扬,身材短小。说也奇怪,贝小姐倒还没有什么,把一个做丈母的贝夫人气得个发昏,默默无言。当夜就使出他那一种野蛮手段,硬硬的把贝小姐叫了进来,和自己同床睡觉,不许他出去和彭太史成婚。一连三天都是如此,把彭太史气得目瞪口呆。待要和他讲个明白,却又是已觉得有些碍口,说不出来,只得放在心中隐忍不发。那贝小姐年幼娇痴,毕竟和彭太史有些夫妻的情愫,也只好偷寒送暖,暗地关情。见贝夫人这样作为,不晓得他究竟是怎么一个意见,又不好意思去问他。久而久之,这贝小姐受了专制的压力,不知不觉把从前心上的夫妇爱情都消入东洋大海去了。

看官且住,从来男大当婚,女大当嫁。做父母的见那女儿出阁,自然要指望他“琴瑟和鸣,夫妻好合”才是道理,怎么这位贝夫人用着野蛮手段禁制了自己的女儿,不许他夫妇合婚成礼,天地之内那有这样诧怪的事情?若果然竟有这样人儿,那也可算得宇宙之大,无所不有的了。你们试想,贝夫人究竟是怎样一个心思?原来他仗着自己是中堂之女、翰苑之妻,更兼门第清华,家财百万,女儿的面貌又生得珠圆玉润,柳媚花娇,算计自家这样的女儿,那般的声势,一定要配一个风流熨贴的如意郎君,方不辜负他女儿的才貌。见了彭太史这般模样,气到极处,便想出一个极糊涂的主见来,忘了那“嫁鸡随鸡,嫁狗随狗”的两句俗语,倚着那一往无前的气势,竟想替贝小姐于正门之外另辟一个便门,好任他拣选入才,评量面目,差不多有那山阴公主面首三十人的样子。你想这贝夫人的意见,糊涂到怎么一个田地!而且贝夫人虽然将近中年,却是意气飞扬,神情荡越,绝不像贵家命妇的规模。

贝太史虽然晓得,心中也有些不以为然,却那里敢来问他一问?随着这贝夫人带领了小姐各处烧香随喜,看戏游园,渐渐的风声不雅起来。贝太史也只好眼开眼闭,装作痴聋。贝小姐更是个少年女子,有什么定见?近朱者赤,近墨者黑,跟着贝夫人这样的一个尤物,今天看戏,明日烧香,到处卖弄风骚,招蜂引蝶。贝小姐看了这种样子,慢慢也便乐此不疲。那苏州城内,贝家太太的名声,却是通国皆知的了。

有一天,贝夫人带了贝小姐到城外丽华戏馆包了一个包厢,一同看戏。恰恰的霍春荣新自上海到苏,演得不多几日。那一天霍春荣排的戏正是《白水滩》。霍春荣的面貌本来不错,加以浑身结束伶俏非常,衣服鲜明,声情激越。那几步抬步的身段,更觉得气概高华,丰仪出众。刚刚出得场门,只听得一片喝彩之声轰然震耳。

到得打翻青面虎的一场,霍春荣本来武功纯熟,一路棍法,使得旋转如风,虽然傀儡登场,却也有些惊心动目。贝夫人仔细看那霍春荣时,只见他蜂腰猿臂,英武过人,而眼媚横波,眉含黛色,眉目之间却又有些媚态。贝夫人看得出神,贝小姐也眼波澄澄,只注在霍春荣一人身上。那霍春荣是个著名吊膀子的都头,一见了标致些儿的女人,便要百计千方钻头觅缝的谋他到手,何况今夜是送上门的买卖?又见贝夫人等衣装炫耀,仆从如云,料想是个大家内眷,吊上了他们的膀子一定有些好处,不比寻常,便也越发的在台上卖弄精神,把眼光注定在贝夫人包厢之内,一连飞了他们几个眼风,把贝夫人母女二人看得心旌摇摇,六神无主。

贝夫人忽然想出一个主意,叫了包厢的案目上来,指名要点霍春荣的戏,点了一出《义旗令》。霍春荣见他们点戏,晓得已经入彀,甚是欢喜,便进去换了衣服,重扮了黄天霸出来。这一出戏唱得更是认真。贝夫人叫家人放了一封赏洋,只听得“锵啷啷”一声,那雪白的洋钱就如雨点一般在台上四周乱滚。霍春荣见了十分得意,做到吃紧之际,贝夫人放出那绝娇必脆的喉咙高叫一声:“好呀!”这一声喝彩,惊动了合园看戏的人,一个个回头张望。有分教:

狼腰猿臂,惊回蝴蝶之魂;燕颔虎头,飞入鸳鸯之队。

欲知后事如何,请听下回分解。





第五十回 巧姻缘良夜渡银河 杀风景三更飞黑索





且说贝夫人看到得意之时,不觉一声喝彩,早惊动了合园看戏的人。大家回过头看时,早看见贝夫人母女二人坐在包厢看戏,看得眉飞色舞,壹志凝神,如承丈人之蜩,如射大夫之雉。看的人也有认得的,也有不认得的,见了这个样儿,免不得一个个暗中好笑,却也不去管他。

这贝夫人坐在包厢,只顾和台上的霍春荣眉来眼去,及至《义旗令》做完之后,霍春荣换了一身簇新的纱罗衣服,故意走到包厢,向着贝夫人请安谢赏。贝夫人眉花眼笑,慌忙叫他不要多礼,便搭讪着和霍春荣问答起来,那一对眼光就如电光石火一般,忽来忽往,飘疾如风。贝小姐坐在贝夫人背后,羞怯怯的低下头去,再也抬不起来,红晕腮痕,绿凝眉妩,却时时在暗中飞过眼风,偷看霍春荣的面貌,一汪秋水,漠漠含情。一班仆婢侍立在旁,虽然也都看见,只是素来畏惧这位夫人,连贝太史尚且怕他,不敢去管他的帐,何况这班小人?可想而知是怕他的了。当下贝夫人和霍春荣缠绵情话,直到散了戏场,方才回去。

自这一天之后,贝夫人每夜带着小姐出来看戏,又时常把霍春荣叫到公馆中去。

每每晚上十二点钟进去,直至明天午后方才出来,也不晓得他们在内干的什么事儿,这却在下没有看见,不敢乱说。但是霍春荣有时拿出绝精致的扇袋荷包给旁人观看,说是贝夫人母女亲手制造送给他的。这样去看起来,只怕霍春荣在贝府中一箭双雕,恩情美满,也未可知。只苦了两位太史公,担了惧内的名头,还要受这般的糟蹋,在下虽然是个旁人,却也免不得有些气愤。

这一件事儿,苏州省内把他当作新闻,茶坊酒肆,三三两两,谈的都是贝府的新闻。方小松久在苏州,那有不知之理?恰值章秋谷同贡春树到了苏州,要到丽华去看霍春荣的戏,方小松同着秋谷、春树走进戏园,一眼先看见了贝夫人母女二人早已端端正正的坐在包厢里面,不觉暗中好笑,方向秋谷说出一句顽话儿来,说:“你不要轻看了霍春荣,他还是中堂的门婿呢!”章秋谷听了十分疑怪,似信不信的追问他,究竟这里头怎么一回事儿,方小松方才把贝夫人和霍春荣的故事一一的演说出来。

秋谷听了甚是气愤,道:“不信天下竟有这般奇事,这贝太史难道是没有血气的么?怎么任着老婆这样的出来胡闹!”小松大笑道:“岂敢。他果然有了血气,也不至于怕老婆怕到这种样儿。我们多是旁人,何必去管他们的闲事?落得看看他们的情形。”正在说话,台上早换了筱荣祥的《文昭关》上来。这筱荣祥台容甚好,嗓音也还不差,唱过了《文昭关》,就是霍春荣的《花蝴蝶》了。

霍春荣出得台来,秋谷定睛凝视,只见他穿一件织金云缎玄色夹衣,内衬绣花短袄,绣花叉裤,浑身钉着水钻,行动时光华照目,映着那台上保险灯的影儿,分外精莹。品貌果然甚好,丰姿不减当年,更兼口白清亮,身段圆融,煞是可爱。只见包厢内的贝夫人母女,两双眼睛钉在霍春荣身上,目不转睛只顾呆呆的观看。到了交手的一场,霍春荣的一把单刀旋转如飞,满身围绕,但觉得刀光闪烁,灯影迷离,浑身上下,但见一线寒光,丝毫不漏。连秋谷在台下看着,也不觉高声喝起彩来。再做到《水战鸳鸯桥》的一场,霍春荣扑那两交斤斗,更是十分快捷。台下看戏诸人,叫好之声哄然不绝。

秋谷暗想:霍春荣的面貌着实不差,又有这一身本领,也算得梨园角色之内一个出色的人材,怪不得这班妇女见了他就要把持不定。正在心中转念,霍春荣早已走进戏房,换了衣服走下台来,竟到贝夫人坐的包厢里面,坐在贝夫人背后,贝夫人和他说说笑笑,甚是亲热。章秋谷看了,气愤非常,向方小松道:“怎么如今世上竟有这样无耻的妇人!”小松笑道:“你真是少见多怪,可晓得如今风气不比从前,还有什么讲究么?”秋谷听了不觉一声太息,默默无言。又坐了一会,因看不惯贝夫人和霍春荣那种肉麻样子,便拉了方小松和贡春树先自走了出来,高桂宝也同出戏园,方小松同着秋谷、春树仍到桂宝院中。

方小松摆酒接风,荐了两个倌人给秋谷、春树二人,一个叫金媛媛,一个叫朱素卿。秋谷便叫了金媛媛,春树便叫了朱素卿。不多时,两人一齐到了。秋谷看金媛媛时,身材袅娜,骨格轻盈,虽然赶不上陈文仙,也还罢了。再看朱素卿,面貌也和金媛媛仿佛,都是中上之材。秋谷虽叫了金媛媛的局,却并不在意,倒是金媛媛和朱素卿见他们举止豪华,风仪秀美,格外的巴结起来,秋谷也只得略略应酬。

这一席直到了三点多钟方才散席。秋谷同春树一起回到船上歇息,方小松不必说起,自然就是住在高桂宝家的了。

按下秋谷一边,只说丽华戏园。那一天章秋谷等走后,闹了一场风波,你道是什么事情?原来贝夫人在丽华看戏,恰好包厢对面另有一个看戏客人,这人姓郭,是个广东的候补道,苏州人氏,和贝太史狠有交情,为人任侠,喜抱不平,气概高华,性情慷爽。只是有一桩坏处,性如烈火,急躁非常,向和贝太史诗酒往来,互相爱敬。这贝太史原是一个诗酒名家,风流才子,若单看他的表面,那里晓得他是个惧内的都头、怕老婆的领袖!这位郭观察虽是和他要好,却一向不晓得他的家事,只道贝夫人是个名门闺秀,自然是贝太史的内助,三从俱备、四德兼全的了。

有一天,郭观察在亲戚家中听见了贝夫人这些笑话,郭观察那里肯信!反说那亲戚不该污蔑闺门。那亲戚向他力辨道:“这件事儿并不是我一人知道,苏州城内到处皆知,你只顾去细加察访就是了。我和贝府上又没有什么仇恨,为什么要捏造这些说话呢!”郭道台听了,觉得他亲戚的话甚是有理,然而终是半疑半信的,不肯当真。隔了几天,郭道台自家出去细细的打听了一回,果然众口相同,大家都把贝夫人姘戏子的事儿当作新闻传说。

郭道台打听得实,直气得他气涌心头,双眉倒竖,一时忍耐不住,一口气直走到贝太史家来,要见了贝太史和他当面说明,叫他以后当心防范。那知事有凑巧,贝太史刚刚不知为了什么事情,两天之前往上海去了。郭道台见不着贝太史,恨得他擦掌摩拳,气无可出。暗想:“贝太史这样一个人,也算有些名气,怎么娶着这般妇女?怎不叫人和他代抱不平?”气了一会,忽又转一个念头,想道:“天下的事情,眼见是实,耳闻是虚。虽然众口一辞,我却究竟没有看见,难保不是他人捏造的话儿。我何不到丽华去看几天戏,一则解了自家的疑惑,二则看看他们情形,岂不是好?”主意已定,便到丽华戏馆一连看了几天,把贝夫人和霍春荣的情事一齐看在心上,十分愤恨,无计可施。

这郭道台和江苏臬台朱竹君交情极好,并且是结拜弟兄。这一天见了朱臬台,偶然提起这件事情,还气得咬牙切齿的,问朱臬台可有什么法儿?朱臬台也诧异道:“天下竟有这般恶棍,难道贝太史竟是丝毫不觉,也不约束约束的么?”郭道台又把贝太史家事,怎样的惧内,如何的情形,把近来听见的话儿和盘托出。朱臬台想了一回道:“这件事儿,要办他也甚容易,只要办他个外来流棍,把贝府的这些事情隐过不提,料想贝夫人也没有什么法子庇护着他,你道这个办法如何?”郭道台听了大喜,道:“这样办法果然甚好。像这样的淫棍,把他留在苏州,真是害人不浅的东西,办掉了他,也是你的一件德政。”说着,立起来打了一躬,朱臬台笑道:“究竟你和他有什么冤仇,要你这般着急?”当下又谈了一回,定了主意,郭道台就走了。

朱臬台次日上院,把这件事细细的禀了抚台,抚台勃然大怒,便叫他下去立刻饬县提人,从严究办。朱臬台答应下来,恐怕饬县提人漏了信息,被他逃走;或者霍春荣得了这个消息,竟去躲在贝府里头,又不好去派人搜捉,岂不便宜了这个棍徒?当下不露风声,密密的下了一个密札给那马路工程局的委员李兰生,札内还附了一个访牌,话头说得十分利害,叫他立刻会同捕房连夜拿人。

原来苏州马路止有一个捕房,没有会审公廨。凡有马路讼案以及华洋交涉这些事情,都是工程局委员兼管,所以工程局在马路极是有权。李兰生接到这角公文,不敢怠慢,连忙叫上四个能干差役吩咐一番,又去知照捕房,派了两个巡捕协同拿捉。这班差捕到得戏园,霍春荣正在台上唱戏,不便去拿;及至唱完了戏下台,又在贝夫人包厢里面谈谈说说,甚是开心。此时丽华园主已经知道,再三央恳廨差巡捕不要在园内拿人,待他出了戏园再行拿捉。差人等初时不肯,又送了他们一笔差钱,方才答应守在戏园门口,等他出去顺手擒拿,不怕他飞上天去。

那贝夫人等到戏场将散,便上轿进城,霍春荣慢吞吞跟在轿子后头,想要跟进城内。不提防刚刚一脚跨出园门,早有一个差人走上前来,就是劈胸一把。霍春荣梦里也不晓得朱臬台叫人捉他,只认做或者是他的仇家,要想同他拚命;那时止不住心头火发,用了一个解手法儿,左手把廨差的手托开,霍地将身子闪过,右手向廨差的额下随手一叉。这个廨差不曾防备他要动手,招架不及,早被他叉得仰面一交,直跌得有四五步远近。两旁的人一齐吃惊。还有三个差人、两个巡捕见了这般光景,一个个心中大怒,便一拥上前,高声喊道:“我们是臬台朱大人派来拿你。

你这个东西,好生大胆,竟敢动手殴差!你还不好好的跟了我们前去,直要自讨苦吃么?“霍春荣听得臬台拿他,这一惊却也非同小可,那里还敢动手?又见巡捕把号叫放在手中,预备着要吹的样子,越发不敢怎样。凭着他们四五人把他横拖倒曳,扭辫子的扭辫子,揪胸脯的揪胸脯。差人又在身边取出铁链来,哗啷一声将他锁上。

正拖着要走,前面贝夫人坐在轿中听得后边喧嚷,不晓得什么事情,叫一个家人回来打听。那家人见霍春荣被他们一班差人、巡捕锁了起来,连忙走到贝夫人轿前说知备细。贝夫人大惊失色,急急的又叫两个家人回去问那差人:霍春荣犯的是什么案情;可好暂时交保,到了过堂的时候不妨竟到贝府提人。又大大的许他们重酬差费。在贝夫人的意思,想着如今世上只重银钱,凭你再是天大的官司,只要用银钱承抵,料想没有办不到的事情,万想不到霍春荣的案情就是为他自己。那些差人听得贝府许他银子,心上虽然欢喜──从来公门中人,见了银钱就似苍蝇见血一般,那肯轻轻的放过?无奈霍春荣的案情甚重,怎敢受他们的贿赂?正是:

三更怪雨,摧残并蒂之花;一夜罡风,惊散同心之鸟。

欲知后事,请看下回。





第五十一回 美优伶驳翻堂上官 懦太史不问河东吼





却说廨差和巡捕在戏园门口锁了霍春荣,正要走时,见贝府的家人急急的赶来询问,并重重的许了谢仪。若是换了别人,只要案情犯得轻些,这班差人便好得钱买放,怎奈这霍春荣是臬台的公事提人,更兼犯的案情甚重。若要买放了他,就是工程局委员也耽不起这个处分,何况这班差人,那敢怠慢?一个差人便冷笑一声道:“我们是奉上差遣,概不由己。这霍春荣是臬台朱大人立等提案的人,我们耽不起这个干系。你想,朱大人的性情何等利害!我们若把他放走,我们自己还要性命么?倒是请你们太太回去,叫贝大人写封信到朱大人那里和他说个情儿,料想朱大人没有不答应的。此刻向我们话说,却是没用。”一面说着,一面把霍春荣前推后拥径自去了。

贝夫人在轿子里头看得分明,听得真切,见霍春荣铁索钉铛的被一班差人拉着,脚不点地的走了过去。贝夫人看了这般光景,止不住一阵心酸,早流下泪来。想来霍春荣的案情犯得重了,所以臬台立刻提人。自家想来想,想不出一个搭救的法儿,只得要依着差人的说话,叫贝太史写信去保他出来。偏偏的贝太史又到上海去了,不在苏州,一时不得回来。只得自行回去,在轿中跺脚恨道:“平日间用他不着的时候,他偏要挨在家中,这个当儿要用着他起来,却又走到上海去了。”

贝夫人回到家中,母女二人十分懊恼。贝小姐红着眼圈,含了一汪珠泪,默然不语。贝夫人也背过脸儿暗中流泪,口内却还在那里安慰着贝小姐道:“你不要心慌,待我慢慢儿的想法。好在你父亲也就要回来。等他回来之后,叫他写信,或者亲去见那朱臬台。难道咱们这等一分人家,要保一个人都保不下来么?”贝小姐听了,略略心上安了些儿,却终是满心不快,便也睡了。

一夜之中,一个半老徐娘,一个卢家少妇,不知流掉了许多眼泪。锦帏虚掩,宝枕横陈;蜡泪未消,春痕犹腻。红愁绿怨,凄凉斗帐之春;冰簟银床,辜负华清之梦。好容易盼到次日,贝夫人一早起来,便叫一个家人到电报局去,打个急电到上海去,要叫贝太史立刻回来;又叫两个家人去到臬台衙门打听霍春荣的消息。那知这件事儿异常机密,再也打听不出来。

这一天工夫,贝夫人好像热锅上的蚂蚁一般,茶饭无心,坐立不定。又过一天,贝太史在上海接着了家中一个急电,叫他立时回去,不晓得家中出了什么事情,倒大大的吃了一惊,果然立刻趁了轮船回到苏州。贝夫人见丈夫回来了,略觉放心。

这个时候,正是用得着他的时候,免不得也要放些笑面出来,便叫他写信给朱臬台,保那霍春荣出来。贝太史听了,呆了一呆,不敢开口。

原来贝夫人和霍春荣的事实,贝太史也有点风声,虽然心中愤恨,却也无可如何,又不敢把霍春荣怎样。现在听得朱臬台访拿他,正在心中快活,不提防他这位夫人竟堂堂皇皇的叫他写信,要把霍春荣取保出来,不觉呆了半晌,一句话也说不出来。

贝夫人见他并不开口,已经有些怒意,便问道:“怎样样,为什么一句口都不开?难道我烦你这点事儿,你都不答应么?”贝太史见他夫人发怒,粉面生红,蛾眉微竖,又吓得手足慌忙,满心里想要教训他几句,无奈见了他的影子,听了他的声音,更觉得筋酥骨软。此刻见夫人发起火来,那里还敢驳回,挣了半晌方才挣出一句话来道:“我不晓得他犯的是什么案情,怎么就好写信?况且朱竹君也不是遇事生风的人,这件事儿一定内中有个道理。若是冒冒失失的写封信去就要保人,他答应了还好;若不答应,可不是落了一个下风?你也要替我想想才是。”贝夫人怒道:“我不管他犯的是什么案情,横竖是冤枉的就是了。你不肯写信,难道就罢了不成?”说着把一对秋波狠狠的瞪着贝太史,差不多又要发作。若是贝太史是个有些性气的人,把正言责备夫人几句,就是贝夫人再要凶悍些儿,也不能把贝太史当真怎样。无奈贝太史向来惧内,真是闻风胆落,望影惊心。现在见他夫人倒竖双眉,又将发作,就吓得诺诺连声的道:“我也没有说一定不肯写信,不过问问他的案情,好像被他们看了,说你连他犯的什么案情也没有弄得清楚,还要来保什么人?所以我和你商量一回儿,并不是不肯听你的话,你休要这般动气。”

贝夫人听了贝太史一番说话,方才收了怒容,却又冷笑一声道:“他犯的什么案情,我知道么?你一个做男子的,这点事儿打听不出,反来问起我来,可不是个笑话?”

贝太史又碰这个钉子,也只好低头忍受,便向贝夫人道:“你既然一定要去保他,我就去写信就是了。”贝夫人听他肯写,立刻换了一面的笑容,向贝太史笑道:“我不过叫你写一封信儿,你就装腔做势的不肯答应,一定要呕上我的气来才肯去写,我真不懂你是个什么性情。”说着,又笑了。又问道:“你清早进城,可曾吃过点心?”贝太史道:“我接了你的电报,不知家中有什么事情,急得我一夜没有合眼。轮船一到码头。我就忙着上岸赶紧回来,这早晚何曾吃过什么点心?”贝夫人听了,慌忙替他张罗点心。一会儿来了,贝夫人即向他笑道:“你今天没有吃过点心,想是有些饿了,快些吃罢!吃饱了好去写信。”贝太史这一刻儿的快活,真是他有生以来从没有受过他夫人这般优待,只把他乐得抓耳挠腮,不知怎样才好,把方才那一肚皮的怨气早消化到九霄云外去了。吃了点心,急匆匆的往外便走。贝夫人叫住他问道:“可是去写信么?”贝太史连连答应,果然走到书房内,顺着他夫人的意思,实实结结的写了一封信,拿进来给贝夫人看了。贝夫人甚是欢喜,叫他快些送去。贝太史又在信中加盖了一方名字图章,叫了一个能干家人,当面吩咐了几句说话,叫他把这一封信送到臬台衙门,面见朱臬台,要讨一个回信。家人答应去了。

不料家人去了一会,空手回来,也没有回信。贝太史甚是诧异,急问:“怎么没有回信,可是没有见着朱大人吗?”家人道:“见是见着的。朱大人正在签押房着公事,家人把老爷的的信呈上,并说要求大人赏封回信,好待家人回去销差。不想朱大人拆开了信看了一回,冷笑一声,问道:”这霍春荣这案情,难道你家大人竟不晓得么?“家人回道:”小的主人初从上海回来,实在不知备细,总求大人开恩准他取保,小的主人就感激不尽了。‘朱大人听了不但不肯答应,反又冷笑两声,对家人说:“你回去上复你们贵上,这霍春荣是抚台的访牌,不干我事,况且犯的案情十分暖昧,你们贵上就不管这件事情也罢。’家人无法,只得回来,听老爷的示下。”

贝太史听了,尚在沉吟,贝夫人早急得手足如冰,花容失色,急向贝太史道:“他既是这般说法,你最好径去拜会他一趟,打听打听究竟是怎样一个道理,或者再写封信给那抚台,料想讨了情儿也还使得。不然像咱们这样人家,一个戏子都保不下来,以后还要想办得了事么?”贝太史听了贝夫人一派一厢情愿的话头,虽是心中狠不愿意,又不敢推辞,只得说道:“这个抚台我和他没有来往,写信去也是枉然,还是朱臬台和我的交情还好,或者到他那里问了个明白,和他商议一个法儿。

只是朱臬台答应了,叫他取保,料想抚台也没有什么不肯。你道何如?“贝夫人听见丈夫肯去,又欢喜起来,立刻替他取出衣冠,亲手和他穿带。这又是向来没有的事情,破题儿第一次。贝太史受了这般恩宠,不觉的有些感激涕零起来,自然尽心竭力的和他办事。

不料轿子到了臬台衙门,投进贴子,隔了半天也不叫请。贝太史呆呆的坐在轿内,等得好不心焦;又等了好一会,方见一个家人拿着名贴慢吞吞的走了出来,走到轿子面前说声“挡驾”,请一个安。贝太史十分疑惑,连忙把来的家人叫住,细细问他为什么今天不见。那家人把眼看着贝太史的面上,嘻的笑了一声,方才答道:“大人有公事,不能见客。”说了这一句,竟自走了进去。

贝太史看了这般光景,只得回来向贝夫人说了。贝夫人也无计可施,只同着小姐无情无绪的暗中流泪。贝太史看在眼中也不敢问。

贝夫人想了一天,忽然想了一个主意出来,心中大喜。你道他想的是什么主意?

他忽然想起父亲现在军机声名赫奕,只要打个电报给他父亲,请他父亲在京里一个电报打给江苏巡抚,和霍春荣说个情儿。料想外省督抚一个个都要巴结军机处的人员;就是一个军机章京,他也不肯得罪,何况他父亲做了相国十年,那有办不到的事情?想定了主意,便逼着贝太史和他拟了一个极长的电稿,约有二百多字,说了无数的谎话,也不晓得怎样措辞,做书的人当初没有看见他的底稿,也只好付之阙如的了。

当下拟好了电报,叫家人到电报局内打了一个三等商电,这一个电报却就花了一百四五十块钱,立时立刻发了出去。

贝夫人自从发了这个电报,指望余中堂听了他的说话,打个电报给苏州抚台,眼见得霍春荣不日便可放出狴犴,重圆绮梦,眼睁睁的只望霍春荣出来。那知过了两天,余中堂外来了一回电,电报局翻好号码送了过来。贝夫人见了余中堂回电,心中大喜,只道霍春荣的事情有些指望,谁知拆开来一看,那电码端端正正的不多几个字儿,除了住处、姓名之外,只有八个大字,是“事涉优伶,毋庸过问”。贝夫人看了,气得他把一张电报撕得粉碎,掼在地上,又把他父亲咒骂了一场。自此之后,贝夫人无可奈何,只得死心塌地的,暗暗的叫人去看了霍春荣几次,花了好些使费,因此霍春荣虽然拘禁县监,倒也并不吃苦。

贝夫人一边的事按下不提,只说霍春荣被差人拿去,在巡捕房关了一夜,工程局委员问了一堂,霍春荣自己也昏天黑地的说不出为了什么事情。工程局委员道:“你的事情本来是上宪提人,我也不来问你,只把你解到臬台那里,看你的远气罢了。”说着就叫廨差押下去,备了文书,将他申解到臬台衙门。臬台朱竹君看了文书,也不提讯,把霍春荣发到元和县来,叫他问供。

元和县大老爷接到了臬台的公事不敢怠慢,立刻升堂,把霍春荣带上堂来。那霍春荣到了县堂,跪在地下,不等县大老爷开口,先是高声问道:“小的究竟犯了什么罪犯,要朱大人这样的费心搜捉?”县大老爷见他这般强项,不由也动起火来,把惊堂一拍道:“你这个该死的棍徒,你引诱贝大人的妻女,夜入人家,还说没有罪么?本县看你还是好好的招成,免受刑罚。”霍春荣见县大老爷这般问法,胆又放大了几分,定一定神,又高声答道:“戏子唱戏为生,向来安分,不敢做这样的事情,求大老爷明鉴。”县大老爷又拍着惊堂道:“现在有真赃实据,你还要希图抵赖么?”霍春荣心中暗想:“事到如今,左右难逃公道,落得索性把他挺撞一番。”

便又高声道:“大老爷既说现有真赃实据,请问大老爷是个什么赃据呢?”县大老爷又喝道:“你时常自己拿着什么扇袋、荷包给人观看,说是贝夫人母女亲手制造送给你的,难道还不算真赃实据不成?”

霍春荣听到此处,竟哈哈大笑起来,笑得满堂差役相顾失色。县大老爷又羞又怒,高声喝道:“你笑的什么!”难道本县问错了么?“霍春荣笑了一会方才回道:”就是这荷包、扇袋,就算做引诱的凭据么?不瞒你大老爷说,戏子在京城里头唱戏,那些王爷、中堂的太太、小姐们说戏子唱得好戏,时常叫到府中说说闲话,不算什么希奇。再说起荷包、扇袋来,戏子在京城里,常有太太们赏些活计,更算不了什么事情。大老爷说戏子引诱贝大人家的妻女,戏子唱戏为生,那有这般大胆?

不过贝大人的太太常到戏园看戏,贝大人又是个头等乡绅,点了戏子的戏,戏子不能不唱。贝太太放了赏钱,戏子不能不上去谢赏。谢赏的时候,贝太太叫住戏子,问几句话儿,戏子不敢不应。贝太太一团好意,和戏子说句话儿,难道戏子就好跑掉了么?至于大老爷说戏子夜入人家,戏子一个唱戏的人那敢向人家混走?都是贝太太几次叫人来叫戏子进城,戏子方敢进去。况且贝大人家是何等的规矩,那样的门墙,就凭着戏子这样一个人儿,里边没有招呼,就走进得去么?这样的事情,大老爷要说是戏子的罪名,戏子就死也不服。大老爷若是不信,只顾叫人到贝府上去打听,若有一定虚言,听凭大老爷怎生惩罚。“正是:

一夕公庭之供,口利如风;三千堂上之刑,鞭飞碧血。

欲知后事,且看下回。





第五十二回 霍春荣利口受官刑 宋子英丧心施骗局





且说霍春荣在元和县堂上侃侃凿凿的说出一番口供,把所有的事情都推在贝夫人身上,自家却卸得干干净净的,好像与他无涉一般。这位元和县大老爷听了他一番口供,竟被他顶得目瞪口呆,那里敢再问下去?怕他再要说出别家闺阃的事来,得罪了苏州城内的乡绅不是顽的。当下坐在公堂上面,一句话都问不出来,停了一回方才说道:“你方才说的话儿都是胡闹,难道贝大人的太太和你有什么交涉不成?”

霍春荣听了又冷笑道:“大老爷不是方才问着戏子,说是戏子引诱了贝大人的妻女,戏子才敢从实供招;此刻怎么又说这般说话,可是大老爷忘记了么?”这几句话,说得两旁差役都好笑起来,虽然不敢喧嚷,却已一个个掩口葫芦。县大老爷听了大怒道:“你这大胆的棍徒,这般可恶!连本县都顶撞起来。”吩咐左右掌嘴。差人答应一声,喊了一声堂威,正要上前,霍春荣两手一拦道:“且慢,戏子若是说错了什么活儿,或是真犯了什么罪名,才好领大老爷的刑法,戏子到底在大老爷案下犯的何等事情?还请大老爷明鉴。”

县大老爷被霍春荣这一顶,竟是无言可答。呆了一刻,方才咬牙大怒道:“你仗着这般利口顶撞本县,本县今天偏要打你一遭。”说着,又喝差役快些动手。差役见本官发怒,不敢怠慢,不由分说,上来了几个差人,把霍春荣按住,一五一十的打了四十,打完了放他起来。县大老爷又道:“你既然不肯供招,本县一天到晚的公事甚多,那有工夫问你?浑深你是臬宪解来的人,且待本县去禀复了朱大人再来问你。”说完这几句话,便喝叫差人带他下去。

霍春荣被差人带了下来,仰着脸儿冷笑道:“我自己的罪名通没有晓得,倒打了四十个嘴巴,岂不可笑!”一面说着,一面挺着胸脯,大踏步走了下去。

这里县大老爷完了堂事,一径便到臬台衙门禀见。朱臬台慢慢的踱了出来,说了几句闲话,便问:“霍春荣的案子问得怎么样了?”元和县便从袖中取出一纸供单,鞠躬献上。朱臬台看了一遍,就冷笑一声,问那元和县道:“我不懂你的问案为什么这样的糊涂?你想这个事情关涉人家内眷,怎么好和他当面说明?惹得他牵牵连连的说了这么一大篇儿,还是听了他的好呢,还是不听他的好呢?将来传扬出来,得罪了绅士还在其次,何苦去坏人家闺阃的名声?”说得元和县面红耳赤,跼蹐不安,连忙立起身来请了一个安,道:“大人明鉴,这都是卑职糊涂,没有想到这层道理。卑职下去再问就是了。”朱臬台又冷笑道:“不敢劳动,还时老兄下去,仍旧将霍春荣申解上来,我自己来问罢。”元和县听了,满面羞惭,只得诺诺连声的退了下去,果然仍把霍春荣解了上来。

朱臬台听得霍春荣解到,便传呼伺候,立刻升堂。臬台升坐大堂,不比州县,那两旁伺候的吏书兵役黑压压的站了一堂,甚是威武。朱臬台踱出大堂,端然正坐。

两旁吏役齐齐的喊了一声。霍春荣提到堂上,却也有些心惊,偷眼看那朱臬台时,只觉得满面霜威,棱棱可畏。他还当是昨日在元和县堂上一般,朱臬台还没有开口问他,霍春荣倒反跪上了一步,高声问道:“蒙大人赏提,戏子不知犯了什么案情,要求大人的明示。”臬台听了微微的冷笑道:“你这个利口刁徒,到了本司这里还敢巧言狡展,本司只问:你既是唱戏为生,平日就该安分,为什么拆梢打架,遇事生风,学那流氓的行径?本司久已访闻,你是一个不安本分的棍徒,你还不晓得自己的罪名么”你可知本司这个地方,比不得元和县堂上,不准你开口多言!“说着把惊堂一拍,喝一声打:”打!“

霍春荣正要分辩,无奈臬台衙门的差人十分凶狠,况是朱臬台预先分付下的,一声喝打,立时就拥了七八个人上来,凭着霍春荣高声叫冤,众人只是不理。鹰拿燕雀的一般,把霍春荣揪翻在地,剥去背上的衣服,露出脊梁,两个行刑的皂隶手中拿着一对藤鞭,一起一落的向着霍春荣背上便打。霍春荣大叫道:“话还没有说得明白,怎么就这般混打起来?”朱臬台只当作不听见的,只是敲着旗鼓,喝叫重重的打这狗头。原来刑杖之中惟有藤鞭最是利害,京津一带惩治青皮都用这个藤鞭,仿佛就和站笼一般。

当下打了二百多鞭,霍春荣的背上已是条条见血,打到五百更是血肉模糊。好个霍春荣,咬定了牙齿一声不哼;痛到极处,反高声大叫道:“我到底犯了何等重罪,要受这样的刑罚?不说一个明白,就把我打死也是枉然!”朱臬台冷笑道:“你要问你的罪名,本司就是办你外来的流棍……”霍春荣不等朱臬台说完,又喊道:“就是外来的棍徒,也没有这般的打法。”朱臬台向着旁边站的书吏说道:“你们看他这个样儿,真是目无官长,他在本司这里尚且这样的咆哮公堂,平日之间可想而知,一定不是个安守本分的了。”说着又喝叫结实再打。打到后来,一鞭下来,那背上的血四围乱溅,打得浑身上下真是一个血人,差不多气咽声嘶,只有一丝游气,朱臬台方才喝住。那时霍春荣已打得和死人一般,热血攻心,眼睛倒插,四个差人把他扛下堂去。

朱臬台见霍春荣打得这个样儿,心上十分畅快,当下叠成文卷,定罪申详,把霍春荣当作个著名流棍,定了五年的监禁罪名。从此霍春荣收在县监,鞭痕利害,沉重非常,这也是他到处贪欢的风流业报。幸亏贝夫人暗暗的叫人进监看视,花了许多使费,又按月接济他的用度,所以霍春荣虽在监中,倒也并不吃苦。只苦的是贝夫人母女二人,哑吃黄连,无从诉说。最恨的萧郎咫尺,门外天涯;对月伤心,背灯弹泪。这相思病儿,也不知害到何时方能了结。真是心期凄惋,宝髻慵梳,睡思惺忪,熏笼愁椅。春蚕半死,犹留未尽之丝;蜡炬成灰,尚有将燃之泪。贝夫人更是恹恹牵牵的大病了一场,医了多时方才全愈,这也不去管他。如今且把霍春荣和贝太史的新闻一齐按下,再说起章秋谷、贡春树的正文来。

且说章秋谷和贡春树在船上住了一夜。次日,小松出城看望,说起霍春荣被臬台拿去的事情,秋谷拊掌称快。小松道:“虽然如此,但是苏州戏馆却少一个人材。”

三人谈了一会,秋谷便同着小松进城,看了几家亲友。有一位陆侍郎的公子叫做陆仲文,请秋谷游了一天虎丘,坐的是小陈家双开门的船,酒菜甚是洁清。陆公子带的一个局,叫做王小宝,面貌也在中上之间,应酬却甚是周到。秋谷看他云鬟腻绿,杏靥浮红,香辅微开,星眸低缬,和陆公子不住的咬着耳朵,凭肩私语。秋谷看了,想起花云香和许宝琴二人,不觉微微叹息,停杯不饮。幸亏金媛媛十分要好,见秋谷有些不乐的样儿,想些说话和他解闷。接着主人陆仲文摆起拳庄来,要找秋谷掊拳,方把秋谷的心事混了过去。

过了几天,陆仲文又请章秋谷、贡春树二人在王小宝家吃酒,却只有章秋谷一人到来。陆仲文诧问:“春树那里去了,为什么不来?”秋谷微笑道:“春树么,他有一件切己的事情,今天料理去了。”仲文又问:“春树有什么切己的事情?”

秋谷笑而不答。

这一席酒,却是秋谷叫的金媛媛第一个先来,到了台面上,先用一对秋波四围飞了一转,然后对着秋谷低鬟一笑,方才坐了下来。坐定之后,张开了折扇遮着面孔,和秋谷密密切切的谈得甚是投机。却被陆仲文一眼看见,先自笑着嚷道:“唔笃两家头啥要好得来,到仔台面浪还是格付架形。就是有啥闲话末,晏歇点到仔被头里向也好说格啘。”说得秋谷一笑,回转头来。金媛媛涨得粉面通红的道:“陆大少末,总是实梗瞎三话四。倪搭章二少客客气气,无啥交关,耐勿要来浪说得像煞有介事。”陆仲文拍手笑道:“章二少故歇末客客气气,停歇歇到仔床浪就勿客气哉,阿怕倪勿晓得?”金媛媛无言可笑,只得也笑了。一座客人都笑起来。

忽见娘姨传过一张请客票头来,递在陆仲文手内,陆仲文接过看时,众人也都要看,只见一张票头写着:

飞请

陆仲文少老爷,至如意里王黛玉房酒叙,千万勿却。座客无多,乞代邀数位。

至要。此请

冶安 英订

陆仲文看了道:“原来是他请客。”便叫娘姨关照下去“少刻就来”,便向秋谷、小松道:“这人姓宋,号子英,却是个狠爱朋友的人,和我的交情狠好,你们可肯一同前去,赏赏他的光么?”章秋谷和方小松的意思,原是不肯同往。禁不得陆仲文再三苦邀,只得允了同去。

散席之后,陆仲文便拉了二人,径到如意里来。好得是王小宝家离如意里只有一箭多路,不多几步已经到了。陆仲文是认得的,便当先走进踏上扶梯,刚刚走得一半,早有一个三十多岁的人走至楼口相迎,王黛玉也跟在后面。秋谷、小松素不相识,免不得大家一揖,通过名姓,方知就是宋子英。子英问了秋谷、小松的名姓,满面堆下笑来道:“今天兄弟托了陆仲翁的福,居然二位都肯赏光,真是幸会!”

又竭力把秋谷恭维了一番。秋谷听他的谈吐也还不俗,抬起眼来看时,见他面貌也还清秀,身上的衣服甚是时新,觉得这个人也还不甚可厌,便也应酬了他几句。

当下等了一会,又来了两个客个,秋谷并不认得,却都是陆仲文的旧交。宋子英见客人已经到齐,便叫快摆台面。陆仲文道:“一席酒,宾主止有六人,可不觉得寂寞么?”宋子英道:“客人虽然少些,我们多叫几个局来,叫他们凑个热闹也好。”仲文听了,点头称是。宋子英便取过局票来,央陆仲文和他代写。仲文叫的是王小宝、王二宝、沈芸仙;小松叫的是高桂宝、洪彩珍;秋谷没有别人可叫,就叫了金媛媛和朱素卿。那两个客人每人也叫两个来,宋子英自己也叫了一个吴小卿。

陆仲文一一写好,点了一点共是十二张局票,交与娘姨去发。房间里人早绞上手巾,起过手巾大家入坐。宋子英便请秋谷首坐。秋谷不肯,要让别人时,宋子英抵死不肯,只得坐了。小松坐了第二,其余以次坐定。不多一会,叫的局陆续到来,一时柳舞花飞,钗摇钏动。这一席直吃到十二点钟方才散席。

秋谷起身别过主人,径回船内,只见贡春树先已回来,坐在床上尚未睡下,呆着脸儿好像有万分心事一般。秋谷见春树这个样儿,知道不妙,急问事情怎样。春树叹一口气道:“不必说他,这事情真个有些不妙。”便附着秋谷耳朵说了一回,秋谷呆了一会。

看官且住,这贡春树的事情在《九尾龟》初集中间已经提起,不过没有说破,有心叫看官猜个闷葫芦,到底是件什么事儿。章秋谷此次到苏何事,究竟没有说明,这个闷葫芦一直闷到如今,看官们始终没有明白。列位休得心慌,待在下慢慢的表白出来。正是:

桃花人面,空怀合浦之珠;杨柳春风,先种蓝田之玉。

欲知后事如何,请看下回分解。





第五十三回 弱书生几成薄幸郎 老学究怒责亲生女





且说前回书内;章秋谷和贡春树同到苏州,究竟所为何事,且听在下说来。

原来贡春树住在常州,本来寄籍苏州城内,狠有些儿房产,还有几处住房。春树每年必到苏州两次,为的是收取房租。另有一所极大的住房,坐落在观前宫巷,却赁与春树自家的亲戚潘玉峰居住。每到苏州收取房租,春树就住在潘玉峰家内。

今年正月春树到了苏州,在潘家住了一月有余,正想要动身回去,不期事有凑巧,无意之中撞着了一个风流孽障,欢喜冤家。潘玉峰有一个干亲家,姓吴,叫做幼勋,教读为生,南濠人氏,只有一个女儿,从幼时就与潘玉峰的内眷往来。潘玉峰就把程幼勋的女儿认为继女。这程小姐长到十六岁上,生得妩媚出,丰姿绝世,齐齐整整,袅袅婷婷。汉宫飞燕之腰,洛浦惊鸿之影,真是个十全十美、倾城倾国的佳人。

潘玉峰的太太以及上下人等,没有一个不欢喜他。

这一天也是合当有事,程小姐要到潘玉峰家看看干娘,刚刚走进中门,恰恰的贡春树在里边走出,和程小姐擦肩走过,彼此定睛一看,大家吃了一惊。春树只觉得程小姐蛾眉淡扫,星眼流波,肩若削成,腰如束素。内家装束,穿一套缟素衣裳;时样梳妆,挽一个轻盈鬟髻。见了春树,不觉面上一红,低下头去,那一付娇羞的态度画也画不出来,走的那几步儿更是杨柳随风,春云出岫,一步步的移将过来。

贡春树自有生以来从没有看见过这样的女儿,不觉得神魂飞越,心花怒开。最可恨的是一边进去,一边出来,那一个花娇柳媚的影儿只在眼前一闪,已经走进中门,只得立定了回过头来看他的背影。不想春树回头之际,那女子恰恰也回过头来,一对水汪汪的俊眼正和贡春树的眼光射个正着。只见他红晕梨涡,春融杏靥,低头一笑,就扶着随来的侍婢急急的走了进去。春树被他回头一看,只看得骨节皆酥,暗想不知是何等人家的女子,竟是一个十分出色的人材。且不要管他是谁,回过身来,闯进房去,好再看他一个仔细。原来苏州规矩,内眷见客甚是大方,并不做那小家的样子,乱逃乱躬的神情。

当下贡春树重又闯进房内,见刚才这个女儿正和潘太太坐在一起,拉着手儿有说有笑的甚是亲热。见了春树进去,假意立起身来含羞欲避,却被潘太太一把拉住道:“这是我娘家的侄儿,为人甚是诚实,不必避他。”又向贡春树道:“这是我的干女儿,你来见个礼儿,日常也好见面。”贡春树听了大喜,便向程小姐深深打了一拱。程小姐红着脸儿回个万福。潘太太拉他坐下道:“我这个侄儿就如儿子一般,你不必同他客气。”春树也在一旁坐下,搭讪着寻些闲话和他扳谈。程小姐十句之中,也回他四五句。

看官,你想程小姐年当及笄,情窦已开,又是个千伶百俐的性情,不免就有些秋恨春愁的心事。看着贡春树这样的一个翩翩公子,浊世才郎,更兼举止温存,仪容俊爽,那有不动心的道理?向来这位程小姐到潘玉峰家来探望干娘,必要留他住在家中,隔了几天或是半月方肯放他回去。自此程小姐住在潘家,天天与春树见面,偏偏贡春树的卧房就在潘太太对面,不多几日,贡春树放出偷香的手段,不知怎的竟和程小姐暗中成了这件事儿。

眷属疑仙,姻缘美满,贡春树的得意自不必说。潘太太慢慢的也有些晓得风声,背地里着实埋怨了贡春树几次,说他怎样做出这种事情。“你是已经娶亲的人,又不能娶他回去,将来你却怎样对得住他?”贡春树见事已败露,对着潘太太赌神设誓的,说将来必要想个法儿娶他回去。潘太太见他们木已成舟,也没有什么话说。

程家因此回住得久了,屡次叫了人来要接程小姐回去,都是贡春树怂恿潘太太出头留住不放。潘太太心上虽然不愿,为的是娘家只有一个侄儿,平日甚是疼他,拦阻不住,也只得随他胡闹。

不觉一连就是两月有余,不想程小姐和春树暗度春风,腹内已经留了一个种子。

蓝田玉茁,合浦珠芽,渐渐的程小姐怀酸呕食,竟是病妊起来。春树急了,要求潘太太到程府和他做媒。潘太太那里肯去说?“你是已经娶过的人,我怎好到那边去说?将来闹了什么事儿,我耽不住这个干系。”

贡春树见潘太太不肯去说,更加着急,再??求告。求得个潘太太推辞不得,只得坐了轿子去到程家,要和他女儿说亲。不料程幼勋这个老头儿自从小中了书毒的人,情性十分古拙,一口回绝。只是只有一个女儿,要把他许在苏州本城,舍不得嫁到别处。潘太太碰了一个顶子,没有什么话说,只得回来。

贡春树无计可施,程小姐更加急得要死,晓得他父亲的性情不好,若回到家中,知道了这桩丑事,就是性命交关。更兼程小姐的肚子一天大似一天起来,那里遮掩得住?急得只要寻死。

贡春树忽然想起章秋谷现在上海,便想前去寻他,和秋谷商议一个计较。平日间贡春树最是佩服秋谷精明练达,应变多才,更兼为人任侠,喜抱不平。倘能寻着了他,或者有个主意也未可知。想来想去,只有这一个计较,更想不出别的法儿。

到了这山穷水尽的地方,也只得姑且试他一度。打定主意,硬着头皮和程小姐说了,一直径到上海访寻秋谷。一见面的时候,就把这件事儿恳他。

秋谷虽然答应了他,却打算直到上海的正事完毕之后,顺路回到苏州,再行替他设法。不料章秋谷在上海耽搁住了,不能动身,贡春树也有些迷恋烟花,乐而忘返。直到七月里头,贡春树接了潘玉峰的一封来信,说程小姐回去之后,肚皮渐渐大了,隐藏不住,被程老头儿看了出来,气得个发昏半死,便盘问女儿究竟与谁人苟合,做出这样不要脸的事情。程小姐那里肯说,只推是停经鼓胀,并没有什么私情。程老头儿虽然不信,却也有些疑心,便把他女儿关在后面一间楼上,要等他当真分娩,然后问他。信上边并且责备了春树几句,说他到了上海,既然朋友已经寻着,为什么不赶紧回来?若再不回来想个法儿,大家计较,直到他月足临盆,可不枉害了程小姐的一条性命?

春树接到了这封急信方才当真发起极来,千求万告的央着秋谷同到苏州。秋谷虽是当时答应,但仔细想来,这件事儿没有一些门路,怎好下得手来?一到苏州,便叫春树先到潘家打听消息,依着春树的意思,还想要叫潘太太到程小姐家去看看他到底怎生光景。那晓得程老头儿道是潘家引诱了他的女儿干了这般丑事,又不能当面和他理论,却恨得咬牙切齿的,差不多彼此成了不共戴天之仇,如何还肯与潘家来往?春树听了焦急非常,想要寻一个同程家素来认识的人,进去和程小姐通个线索。好容易寻了几日,才寻着一个程家数年前用过的一个粗做娘姨,许了他的谢仪,又教于他许多说话,指望叫他进去见着了程小姐,做一个传消递息的红娘。

那知娘姨去了半晌,垂头丧气的回来道:“这件事儿是办不到的,我也不想赚你们的谢仪。”说着转身就走。春树连忙把他叫回,要问他一个底细。娘姨叹口气道:“我到了他家,见过奶奶,坐了一回,问起小姐为何不见。我刚刚问得一句,还没有说出什么别的话儿,就被那老头子突出了眼睛,挠起了胡须,叱喝了两声,说:”这个贱人,我家已当他是死过的了,你还来提他做甚?‘那个样儿好像人都吃得的,把我倒吓了一跳。后来我打听他们用的小大姐,方晓得小姐被他们关在后楼,不许他下楼一步,连楼门都锁了起来。您想别人还见得着他么?“春树听了十分叹息,只得给了那娘姨几块洋钱,让他去了。这些事儿,都是三五天之内的事情。

春树等那粗做娘姨去了,奔出阊门,径到船上,要和章秋谷商议。岂知到得船上,秋谷尚未回来,春树十分焦躁,却又无处去寻,直等到一点多钟,秋谷方才回来。见春树神色仓皇,晓得事情尴尬,急急的问他事情怎么样,春树便把方才粗做娘姨的话照样说了一回。秋谷听了,皱着眉头想了一会,想着这件事儿十分棘手,便说:“此刻我也打不出什么主意,最好明天你把昨日的粗做娘姨叫来,待我细细的问他,或者想得出什么法子,也未可知。”春树听了,虽然少觉放心,终觉得满心忐忑,睡在床上翻来覆去的再也不得合眼。

勉强过了一夜,约莫不到六点钟时候,春树已经起身,秋谷却还在沉睡。春树胡乱洗了个面,把秋谷叫醒了,嘱付他:“在船老等,切不可到别处耽迟,我去了立刻就来。”说着,便急急的上岸去了。秋谷等春树走了,便也起来洗面,并吃些点心,等到十点钟左右,果然春树回来,背后跟着一个四十岁上下的娘姨,跑得满头是汗,同上船来。

秋谷盘问了那娘姨一会,也想不出什么计较来,便又问那娘姨道:“你既然在他家做过娘姨,他家共有几间房子,你自然是晓得的了,可晓得他家小姐究竟关锁在什么地方?”那娘姨指手画脚的说道:“程家的房屋就在前面桥边,离此没有多远。他家共有两厅正屋;后面还有两间水阁,却是临着河滩。他家小姐就锁在后面的两间楼上。你想外边有人进去,怎的见得到他?”秋谷听了,猛然双眉一皱,计上心来,暗想必须如此这般,方能成事。若这件事儿办他不到,我章秋谷还算什么当今侠客,说什么当世奇才?当下打定主意,不觉面有喜色,急问娘姨道:“那两间水阁既是沿河,立在船头上可看得见么?”娘姨用手望东边一指道:“那不是程家的房子么?”秋谷连忙跨出船头,把那娘姨也叫了出来,顺着他手指的地方向东看去,果然见酱园隔壁有两间水阁,门窗紧闭,人影全无,估量着也不甚高大。秋谷疑惑这两间水阁不像有人住在里边的样子,又细细的问了娘姨一回,问得确确实实的一毫不错,便在身上取出一张十元钞票赏与那粗做踉姨,对他说:“现在没有什么事儿,你且先行回去,将来有用你的地方再来叫你。”那娘姨接了钞票,欢天喜地,千恩万谢的去了。

秋谷回身走进中舱,贡春树慌问:“怎么?”秋谷笑而不答。春树见秋谷这般模样,知道他一定是想着了什么法儿,再三追问。秋谷笑道:“法子是想了一个,至于办得成办不成,却要听你自家运气。我总尽心竭力的为你代谋。倘若真做不成,那就不干我事了。”春树急问他:“是甚法儿?”秋谷含着笑,附耳和他说了一遍。

春树喜得满心奇痒,满面笑容,连说:“这个招儿甚是稳妥,一定是手到功成。”

秋谷道:“要说我这个主意是一个稳妥的法儿,却也未必,不过事到如今,不得不这般做法,叫做尽我们的人事罢了。”春树点头称是。秋谷忽又跌足道:“这件家伙我都掉在常熟,现在一时却无从置备,这便如何是好?”正是:

窥帘贾午,春留韩寿之香;曲院红绡,夜试昆仑主持。

欲知章秋谷究竟如何设法,请看下回





第五十四回 拍马屁流氓讨好 抱春愁侠客传书





且说章秋谷盘问了粗做娘姨一会,忽然心中得了一个主意,想起从前大金月兰嫁与黄大军机的长孙公子,后来逃走出来,是预先设法买通了船户,在水阁上边用腰带吊着身子吊下来的。现在听那娘姨数说,程小姐关锁在水阁后头,不觉登时得计。又细细的想了一会:这件事儿却又与大金月兰不同。一边是金月兰有心逃走,一边程小姐却无意私奔。最好是要和程小姐彼此说通,方能下手。无奈程小姐关锁楼中,无从见面,这个消息怎的传递得通?想了一会,无计可施。偶然想起自己幼年间投师习武的时候,学过一种袖箭,是用右手中指抻发出去,二三十步之内可以暗地伤人。不过是如今时局迁移,英雄无用武之地,只好把他当做顽耍的事儿一般。

但是秋谷寻常习练的几枝毛竹箭儿,一齐掉在家中,不觉跌足自悔。

春树慌问:“究竟是什么东西掉在常熟,说得这般郑重?”秋谷和他说了。春树呆了一会,道:“这个时候,你还想着这不要紧的东西有甚用处?”秋谷又附了他的耳朵说了几句,春树方才恍然大悟,眉开眼笑的道:“几枝毛竹箭儿值得什么,我们难道不好重做几枝么?”秋谷道:“你是个外行,晓得什么?袖箭的做法不是单用毛竹,并且不是一天工夫做得成的。先要认准了粗细长短,用细竹削做竹签,却还要配着分两,熔些铅锡或是铜铁灌在竹节里头,须要分两配得停匀,发出去方才有力。若单是一支竹签,那里有这般力量?你难道这点关节都不懂的么?”春树道:“我又没有学过这个东西,那里晓得这里头还有这许多讲究?如今只好立刻赶造。你先画个图样出来。”

秋谷听了摇一摇头,一言不发;想了一回,方才立起身来开了船上台子的抽屉,取出一枝带着铜笔套的水笔,放在手中试了一试轻重。又把这枝水笔放平在右手掌中,用大指、无名指捺住了中间的笔管,中指抻着笔头做了一个手势,便觉面有喜色。向春树笑道:“这枝水笔大是可用,就不必去重新赶造了。”春树听了也甚是欢喜。

秋谷便叫船户进来,叫把船移到酱园码头停泊。船户道:“那边的码头甚是拥挤,况且上岸起来没有此间便当,我看还是就在此间的好。”秋谷道:“你不要多管闲事,叫你开船只顾开就是了,为什么要这样的噜苏?”船户听了不好再说,答应一声,便把船移到那边停下,打好了桩,系上缆绳,搭好跳板。秋谷因见时候尚早,在船上不免等得心焦,便吩咐春树在船坐守,并叫他留心看那上面楼窗到底开与不开。秋谷便上岸去了,想想没有什么正事,便到高桂宝家去看方小松。

不料小松不在桂宝院中,秋谷却扑了一个空,便又走到王小宝院中,打算要问陆仲文。恰好陆仲文昨夜因闹得晚丁,没有进城,就住在小宝那边,这时候刚刚起来梳洗。见秋谷来了,大喜,便拉他坐下,谈了一回。仲文留他就在小宝院中吃饭,秋谷答应。因秋谷爱吃京菜,仲文叫相帮到德花楼去叫了几样菜来,两人小酌。饭毕,仲文觉得枯坐无聊,要拉秋谷出去兜个圈子,秋谷道:“兜个圈子也没有什么味儿,还是我们再去请两个客人,今天在这里碰一场和可好?”陆仲文尚未答应,其时王小宝新妆已竟,走进来坐在旁边,听得秋谷说要碰和,慌忙接口道:“章二少有心照应倪点蛮好,阿要就去请起客来?”仲文沉吟道:“请什么人的好呢?若要到城里头去请客碰和,实在相离太远,马路左近又没有什么熟人。”

正在踌躇,忽听得楼下相帮叫了一声:“客人上来!”楼梯上脚步响处,早走进一个客人,不是别人,原来就是方小松。他出城之后,先到桂宝院中,晓得秋谷已经去过,又想他没有别处地方,一定是到王小宝家去寻陆仲文去了,所以急急的赶来。陆仲文见了方小松,大喜,便道:“我们正要请客碰和,你来得正好,只要再请一个客人便可入局。”仲文说罢想了一想,便取过一张请客票来,到石路长安栈去请宋子英。

相帮去了不多一会,果然宋子英来了,彼此寒温了几句,便大家入位扳庄。子英便问仲文多少底码。仲文道:“我们相好弟兄,难道谁想赢钱不成?不过是寻个消遣罢了。但是底码打得太小了也没有什么味儿,我看打二十块底二四,说大不大,说小却也不小,你们众位的心上如何?”众人听了点头道好。扳好了庄,定了座位,便碰起和来。碰了几付,章秋谷的牌风甚好,连和了几付大牌。及至碰完结账,方小松没有进出,陆仲文输了二十元,宋子英大输,输了四十余元,多是章秋谷一人赢的,给了八块和钱,其余的一齐收下。

原来苏州堂子与上海规矩不同。上海地方是吃酒碰和都是十二块钱,并且客人吃酒,房间里人没有什么好处,不过是绷个外场。若遇客人碰和,房间里人方有些些好处。这是花柳场中人人都晓得的。苏州堂子却又不然。本来只有吃酒,没有碰和,偶而遇着客人高兴,约些朋友碰一场和,那和钱随便开销,也有四块,也有六块,没有一定。到得后来,有一班爱算小钱的人,只去碰和不去吃酒,虽然没有和钱,倒是烟茶酒饭闹得一塌糊涂。本家同倌人吃亏不起,方才也学着上海堂子一般,行出碰和的名目,却每场和只要八块洋钱。至于客人吃酒,更比上海的情形大是不同,每一台酒虽然也只十二块钱,却另有许多名目。吃酒的无论主客,却要出什么台面洋钱,每人两元,却要现开销的。叫来的局又要出什么坐场洋钱,每人一元,也要当场开发。若是台面上八个客人,每人叫一个局,就要开销十六块台面洋钱,八块坐场洋钱,多在正价十二块钱之外。这便是倌人的好处。所以上海的堂子只愿碰和,不愿吃酒;苏州的堂子却是只巴吃酒,不愿碰和。这也是上海、苏州彼此不同的风气。再如苏州地方,在堂子里头摆酒请客,那请的客人必定是和主人家向来要好方肯到来。因为开销台面,要自家拿出现钱,不比上海地方没有这些名目,就是客人叫局,也要和倌人素来相识方肯应酬,为的是客人局账,倒要逢节开销;倌人出局的坐场洋钱,先要自家垫出。这些情形,在下初集书中已经说过,不过没有说得这般详细。看官们有欢场阅历的人,料也晓得这些规矩的,并不是在下的信口胡言。

如今闲话休提,书归正传。章秋谷和陆仲文等在王小宝家碰了一场和。碰完之后,差不多已有七点多钟,娘姨们捧上碰和饭菜,摆好杯箸,王小宝过来斟了一巡酒,陪着坐在旁边,四人谈谈说说,甚是投机。那宋子英的应酬甚好,谈笑生风,把章秋谷、方小松二人恭惟得十分欢喜。你想如今世上,那有不爱巴结的人?凭你章秋谷这样的高明,免不得着了道儿,险些上了第二次倒脱靴的恶当。

当下宋子英和三人谈了一回,忽地回过头来问陆仲文道:“前天我托你的事情怎么样了?昨日接到一封来信,我们舍亲已经进京引见,只要完结了正事,立时径到苏州,先派了他家里头一个账房来此和他办事,只怕差不多将要到了。你那边的事可有什么眉目么?”陆仲文皱皱眉头和宋子英说道:“我已经替你问过几家,多是不甚凑巧,我那里有功夫和你们办这样的事情,或者我替你再去托托别人倒还可以。”说着便回过头来向秋谷、小松二人说道:“这位宋子翁的亲戚邹介卿,他是安徽有名的富户,现在捐了个候补道,已经分发江苏,引见之后就要出来到省,要在城内买一所大些的住房,屋价不拘多少。宋子翁几次托我,要我和他寻找,你想我那里有这样的工夫?你可晓得那里有出卖的住房么?”

秋谷听了,不觉接口道:“若说住房,春树就有好几所房子,也有大的,也有小的,只不知他可肯出卖,这却要与他商量。”宋子英听了大喜,连忙立起来朝着秋谷深深一揖道:“贡春翁当真有几所房子,那是再好没有的了,只是还要费秋翁的心,前去同他商议。”秋谷连称不敢,道:“这点事儿值得什么,也要这般多礼,我回去问他就是。”宋子英又谆谆嘱咐了一番。

秋谷因记念着春树的事情,不知在船上怎生光景,便别了三人先自走了。到得船上,见春树伏在船上假寐,秋谷唤了他一声,春树失忙张致的跳起身来,两边张望,见是秋谷回来,方才坐下。秋谷问春树可曾看见那两间水阁开过楼窗,春树摇头叹道:“我在船头上等了半天,望得眼睛都有些酸溜溜的,那里见他开甚楼窗?

并且连人声都一毫没有,不要是上了那妇人的当罢。“秋谷道:”宁可信其有,不可信其无。不管他是假是真,姑且试他一试。“一面说着,一面掏出表来一看,已是十点十五分,秋谷便取一张东洋纸信笺铺在桌上,提起笔来不知要写什么。忽然一想道:”坏了,坏了。“急问春树:”程小姐可能识字?“春树道:”眼前的几个字儿尚还认得,就是粗浅些的小说或是信札,也都懂得意思。“秋谷喜道:”这便还好。若是一个不识字的,便又要另想法儿。“说罢,取过笔来向笺纸上一挥而就,写了几个字儿。春树倚在案头,看他写的是”贡春树到明日早十点钟“。就是这十个字儿。春树迟疑道:”何不写得明白些儿,却要这般含混?“秋谷把春树呸了一口道:”你这个人真是糊涂!这不过预先问个信儿,我自己也保不定十分把稳。

若依着你的意思,写些私情话儿,万一射到楼中被第二个人拾去,还了得么?所以我只写这几个字儿,就使被旁人拾去,也想不出这里头再有什么机关,你还嫌我写得少么?“几句话说得贡春树又羞又喜,暗想章秋谷这人真是精细,我此番央他同来,也不枉了我一番跋涉。如今世上那里还有这样的好人,为着朋友的事情肯这样尽心竭力?心上这般着想,却感激到万分。

只见秋谷把方才写好一张信笺,折了一个方胜,取一条麻线,结结实实的紥在笔梗中间,把手招招春树,走出舱去。春树也随后出来,到船头上立定。

正是那七月中旬的时候,玉宇无尘,银河倒影;纤云四卷,清风吹空。一轮明月高高的挂在天中,照得水面上十分澄澈,万籁无声,那景物甚是凄楚。

秋谷走出船舱,举头仰望,见那上面的楼窗依然紧闭,月光照着,好像里面隐隐有灯火一般。秋谷把那一支袖箭放在手中,又仔细打量了一会,见那楼窗的样子都用竹纸糊在外边。秋谷翻身走到船边,离开数步,放出眼力觑得较亲,用尽平生之力发了一箭。只听得“呼”的一声,那支袖箭竟穿入楼窗里面去了。秋谷大喜,春树倒吃了一惊,低低的赞了几声“好箭”。秋谷见那支袖箭一直穿入楼窗,便同春树两人在船上坐了一会。冷露无声,西风拂面,虽是新秋天气,却也有些凉回枕簟,露冷罗衣的光景,便拉着春树进去睡了。

春树睡在床上,千思万想的,这一夜又不知转了多少念头。好容易巴到天明,叫醒了秋谷一同起来,吃过点心,说些闲话。差不多十点钟,秋谷又取一张东洋信笺写了一回,却不许春树近前来看,只叫他到船头上去等候。一面仍旧折成方胜,又寻了一枝笔,照依昨日一般,如法炮制的制备定当,藏在袖中,走出船头立定,目不转眼的看着那上面的楼窗。不多一会,果然只听得“呀”的一声,楼窗开了一扇。秋谷眼力最尖,早看见一个丽人,腰肢袅娜,骨格轻盈;眼含秋水之波,眉锁春山之恨;云鬟半卸,脂粉不施,娇怯怯的倚在楼窗向着下边张望。面上好像带着几分病态,越显得弱不胜衣;更兼泪眼惺忪,愁容寂寞,那一付带病含愁的丰格煞是动人,仿佛是一树带雨梨花,娇柔欲坠。秋谷见了暗暗喝彩,想怪不得春树这般着急,果然面貌不差。那丽人开了楼窗,探出半身往下看时,恰恰的和春树打了一个照面,一时又惊又喜,心上边也不知是什么味儿,好像有多少的酸甜苦辣,一霎时并在一堆。一个楼上,一个船头,彼此你看着我,我看着你,看了半晌。春树只觉得一阵心酸,忍不住泪珠欲滴。程小姐更是蹙着双眉,含情欲泣。男女两人虽然对面,却不能说一句话儿。

正在彼此相看之际,秋谷猛然把春树推开数步。春树刚刚回过头来,只见他翻身舒臂,轻轻的把右手一扬,听得“呼”的一声,秋谷手内的一枝袖箭早飞入楼上窗中,在程小姐耳边擦过。程小姐大吃一惊,一连倒退几步,几乎跌倒。秋谷早拉着贡春树走进舱中去了。程小姐定一定神,方才看那飞进来的是什么东西。只见原是一支水笔,套着一个白铜笔管,有一个红纸方胜系在中间,和方才拾着的差不多的样子。程小姐连忙拾起,拆开看时,见上面写得明明白白的,叫他怎样脱身,如何走法,自有人在下边接应,叫他不用心慌,就是这几句说话。程小姐看了虽然欢喜,却终久是个年轻女子,不免有些胆战心惊,只得大着胆子,硬了头皮,悄悄的收拾了一回。喜得是程小姐被他们锁在后楼,就是送饭与他,也在壁间开个一尺见方的小门,叫人传递。这两间屋内,竟是个人迹不到的地方,所以凭你如何做作,也没有看见的人。

直等到晚上十一点钟,月明如水。照进纱窗。程小姐把楼窗开了两扇月光之下,已看见春树立在船头,秋谷立在春树身后。船头上叠了一张茶几,茶几上边又叠了两张椅子,就和楼窗的高低差得不多,只低了四五尺光景。程小姐见他们已经预备,满心欢喜,放大了胆,把两条绉纱腰带接做一条,一头系在自己腰间,一头系在楼窗柱上,系得十分结实。章秋谷在船头上已经看见,两下打了一个照会,便叫春树立上椅子去接他一接。那知春树向来胆小,刚刚上得茶几,两只脚早索索的抖个不住,急得章秋谷悄悄的顿足,埋怨他道:“现在这一刻儿的时候正是要紧,怎么你这般胆小,不被你误了大事么?”春树连连摇手,一句话也说不出来。正是:

黄衫挟弹,暗传青鸟之书;红粉衔恩,合受花枝之拜。

欲知后事如何,但听下回分解。





第五十五回 一封书琴心通绿绮 百尺楼黑夜盗红绡





且说章秋谷立在船头,见程小姐将腰带拴好两边,正要跨出窗棂,急叫贡春树上去接他一接。那晓得贡春树上了茶几,两足发起抖来,再也跨不上去,急得秋谷连连顿足,埋怨他为甚这般无用。春树正在心慌之际,回过头来要与秋谷说话,不提防脚下软了一软,一个鹞子翻身,早扑通的跌了一交。幸而秋谷立在旁边,眼明手快,一把将他扶住,好的是船头阔大,没有跌在河中。

说时迟,那时快,秋谷眼见楼上的程小姐全身探出,坐在窗棂上边,两手紧紧的拉着腰带,却是战战兢兢的看着下边不敢放手。你想一个未出闺门的少女,那里有这般大胆?看了一会,终久不敢下来,要想船上有人上前去接。秋谷见了这般光景,着急非常;回头看春树时,跌了一交,还在那里叫痛;远远的又听见摇橹之声,想是有船来了,秋谷更加着急。这个时候,顾不得什么嫌疑,把春树推过一边,飞身而上,立在椅子上面,恰恰的够近楼窗,不由分说,竟把程小姐抱在怀中,轻轻的下了椅子,一跃而下。急忙将程小姐放在船头,招手叫春树过来,替他解下了腰间的绉纱腰带,叫春树赶紧将他扶进船舱。早听得后面欸乃之声渐来渐近,秋谷急了,手忙脚乱的把两张椅子一齐掇了下来,又把程小姐吊下来的腰带打个结儿,用力往一丢,恰好仍旧的丢进楼窗去了。

秋谷见事情已经停当,回围一看,除了上面的两扇楼窗之外,没有什么形迹可寻。后边早来了一只小船,船梢上有两人摇橹,正在秋谷大船旁边掠过。那小船上的人,见大船上这个时候还有人在船头张望,又有茶几、椅子排在船头,不免有些诧异。但是他们摇船度日的人,那有工夫来管你这般闲事?擦肩的摇过去了,把个章秋谷吓了一身冷汗出来。暗想今天真是十分侥幸,后先之际,只争一刻儿的工夫,几乎被那小船撞破。弄出事来,被程老头儿告到当官,说是奸拐了他的女儿,还当了得!一面心中盘算,便也移步进舱。只见贡春树和程小姐两人手对手儿坐在旁边榻上。程小姐云鬓不整,玉体横斜,珠泪半含,蛾眉深锁。春树也眼圈儿红红的,眼中含着泪痕,正在那里嘁嘁喳喳的不知讲些什么。见了秋谷进来,男女二人一齐立起;程小姐免不得有些惭愧的样儿,眉黛低颦,红潮上颊,若前若却,脉脉含情。

春树不待秋谷开口,指着秋谷向程小姐说道:“这便是章家伯伯,你我的事情不污他出力帮扶,那有今日这般团聚?真是我们的一个大大的恩人,你快些过去行个礼儿,谢谢他一片热肠,一腔热血。”程小姐听了春树这般说话,那当时感激心绪也不晓得从何说起,感激到极处便又流下泪来,不等春树说完,早花飞柳舞的一般朝着秋谷行下礼去。春树立在一旁,想着这样良朋如今难得,若不是他这般出力,这件事儿怎得收场?白白的送了程小姐的性命。想到此处,不因不由的也推金山倒玉柱的跪在一旁。男女二人一齐拜倒在地,忙得个章秋谷还礼不迭。急忙把春树一把拉住,又把程小姐扶了起来,不觉哈哈大笑。章秋谷这一会儿的得意,差不多就是洞房花烛见了个绝代佳人,金榜题名却又是传胪第一,任是什么事儿,也赶不上他那一番得意。

当下秋谷笑向春树道:“这点事儿算得什么,也要行起礼来?我虽然费了一片心机,却成就了你们的两桩好事,总算不枉我姓章的和你们出力一场。但是还有一句话儿,你却也要自家裁度:你是娶过正室的人,将来把这位小姐同到家中,能否相安无事?再者,你过了三年五载,保不定要秋风团扇,弃旧怜新,那时岂不是依旧误了他的终身,却叫他如何结局?这些事情,虽是不干我事,却不是不替他虚到这层;况且今天这样一来,将来这位小姐自然是无家可归的了,你又不得不格外体贴他些。你道我这层说话何如?”程小姐在旁听了秋谷的说话,觉得句句入情入理,没有一个字儿不是打入心脾,并且还替他虑日后的仳离,将来的结局,如今世上那有这般精细的好人?又听他说到自己日后无家可归的一层说话,不觉牵动伤心,忍不住泪流满面,呜咽起来。又听春树向他说道:“你的说话虽已虚得不差,但我却断断不是这般人物,你只顾放心就是了。若万一将来有甚差池,凭你怎生理论,你可信得过么?”秋谷听了方才微笑点头。程小姐此时感激秋谷直到二十四分,因又走近前来,向秋谷行了一个全礼。秋谷不及提防,搀扶不迭,忙叫春树扶他起来。

程小姐起来,低低的叫了一声“伯伯”。秋谷请他坐在旁边榻上,自和春树也坐下来,商议以后怎生安置。

程小姐此刻方才抬起头来,偷转秋波,暗回粉头,细细的偷看秋谷。见秋谷坐在灯下,面如冠玉,奕奕有光;目若朗星,英英露爽;长身玉立,猿臂蜂腰;气概昂藏,丰神俊美。真个是素腰压沈,粉面欺何;春留荀令之香,夜抱邺侯之骨。和贡春树坐在一处,觉得章秋谷光芒外露,华彩照人,两人比并,还是章秋谷较胜些儿。程小姐不觉吃了一惊,暗想春间初见春树的时候,觉得他丰调过人;现在见了秋谷这般仪表,和春树两边比较,春树不免逊了一筹,不信世界中间竟有这般人物!

“程小姐看了一会,不觉粉面微红。这边章秋谷坐在一旁,也在那里仔仔细细的评量姿态,只见他叙亸香肩,半欹云髻;长眉掩鬓,笑靥承颧;春融却月之姿,红上春风之面,真是宜嗔宜喜,如玉如花。

秋谷也看得呆了一会,方才开口向春树道:“现在事情已经办妥,此刻却就要和你商量善后的事宜。这个地方也不是久居之地,我想你只好把他送回家内,然后再到苏州。我在客栈里头暂住几天,等你回来,一同再到上海。你想我这个主意如何?”春树听了,便问程小姐打算怎样。程小姐低低答道:“我是个没有主意的人,况且既已……”程小姐说到此际,面上不由的起了一阵红云,顿了一顿,接下去说道:“自然和你一同回去,依着章家伯伯的说话罢了。”贡春树问明了程小姐的口风,便道:“你的主意甚好,一准明天动身回去便了。”

秋谷道:“但是还有一件事情,我们大家计较。程小姐虽然走了出来,那程老头儿失了女儿,怎肯轻轻罢手?自然要报官追捕,招贴寻人。我们这个船家又不是我们一党,他明天起来见忽然多了一个女人,定要心中疑忌;那时不得不把真话和他说明,一时露了风声,知道他心迹是好是坏?万一他说出口来,被人晓得,我们那里耽得起个拐逃的罪名?据我想来,我们明人不做暗事,索性等到明天亲自到他家内,见了第头儿和他一一说明。到了这个时候,一则如今木已成舟,二则恐怕风声传播,免不得忍气吞声,卫顾自家,你道何如?”春树听了,连忙摇手道:“这个不好,那里有拐了他家的人口私逃还自己上门承认的道理?倘被他翻转面来,吃在你的身上,要交还他的女儿,或者竟和你打起官司来,如何了得?”秋谷笑道:“你终是见理不明,所以这样胆小,我却料定这件事儿起不出什么风波。你只顾放心,不要替人着急。若我没有这样口才,那里敢去自家承认?难道我是不怕王法的么?”春树听了,不好拦阴,心上终是觉得不甚妥当,但也只好由他。

秋谷见时候不早,便立起身来道:“今天我到外舱安息,让你们说说话儿,天明了再打主意。”春树一把拉住道:“怎么还要这般客气,避的是什么嫌疑,难道我们还有这些过节不成?”秋谷一定不肯,道:“大凡男女嫌疑,到了无可如何之际,自然也只好从权。现在还不是从权的时候。”说着,回身向着外舱便走。春树苦苦的拉住,程小姐也说道:“伯伯是我们的救命恩人,何必要避什么嫌疑?这个样儿叫我们心上如何过意得去?”秋谷还不肯依。后来春树急了,赌神发咒起来,秋谷方才依了,暂时和春树同在一床睡下。春树的床便让与程小姐睡了。三人辛苦了一夜,和衣略睡,一入睡乡。

直睡到明天十一点钟,还是秋谷先醒,还有些睡眼模糊,见窗缝内日光射入,知道迟了,连忙唤了春树几声。程小姐先自惊醒,急急的坐了起来。春树也自醒了,一同起来。外面船家听得秋谷起身,舀了两盆脸水走进舱来。见多了一个少年女子,不觉呆了一呆,却又不敢多问,只是站在一旁,做嘴做脸的做出许多怪相。秋谷却正颜厉色的把船家唤近前来,约略把这件事情和他说了几句,又向箱子内取出一封洋钱,约有二十余块,一齐赏了船家,叫他不许外边漏泄。船家得了这注意外横财,不胜之喜,连连的答应几声,接了洋钱又谢了几句退了出去。

秋谷也起身上岸,又叫贡春树也上岸去置办些妇女服用的东西,自己却径向程家去了。春树拦他不住,眼睁睁的看他敲门进去,心上鹘鹘突突的怀着一肚子鬼胎,只得上去买了些镜子梳具、胭脂洋粉等零件送上船来,看着程小姐对镜梳头,等候章秋谷的信息不提。

再说章秋谷上得岸来,走到酱园隔壁,认准了门户,轻轻的把门敲了两下。早听得呀的一声,两扇门开了一扇,门内有人道:“是什么人敲门?”秋谷不及答应,一脚跨进门来,刚刚和门内的人打个照面。秋谷停住脚步,举目看时,只见开门的是一个五十多岁的老头子,拱肩缩颈,曲背弯腰,面皮起了皱纹,须发已经花白,那形状甚是可笑,却满面带着怒容,还有些气喘吁吁的样子。秋谷看了心中暗想:这个老头儿神色这般呆滞,一定就是程小姐的父亲,便开口问道:“这位老先生就是程幼翁么?”

原来程幼勋今天早起不见了女儿,气得他暴跳如雷,大骂不止。待要报官追捉,又怕坏了自家一世的名声。嚷闹了一回,没有法想。此刻正在家中纳闷,忽听见外面敲门,叫了几声小大姐,没人答应,赌气立起身来自家出去把门开了。见章秋谷撞将进来,开口第一句就问他的名字,又见他衣裳楚楚,相貌堂堂,却也不敢怠慢,忍着怒气,请秋谷进堂坐下,方才说道:“这位老兄尊姓,有何贵干,打听小弟的贱名?”秋谷听了,立起来把手一拱道:“原来就是程老先生,兄弟不知,多多得罪。”说着随又通了自己的名姓,大家坐下。

程幼勋便问秋谷:“有甚事情降临寒舍?”秋谷微笑答道:“府上可有走失的内眷么?”这一句话把个程幼勋说得好像当心打了一拳,面上的神色登时一红一白的不定起来,硬着头皮回道:“你这话儿来得奇怪,我们这里好好的世代清门,那里有什么人走失,你这个人可是有些痰气的么?”口内这般说着,心中却暗想:“这个人来得蹊跷,我家中出了这件事儿,并没别人晓得,怎么他突然开口就问这样的话儿?”又听得秋谷笑道:“我是好意前来报信,怎么你竟出口伤人?既是没有这件事儿也就是了。依我看来,劝你不必这般遮掩,和我说了真话,或者有些消息也未可知。”正是:

瘦损香桃之骨,小玉多情;荒唐割臂之盟,十郎薄幸。

欲知后事如何,但听下回分解。





第五十六回 真大胆登门报信 假小心曲意邀欢





且说章秋谷见了程幼勋,劈头就问他可有家人走失。程幼勋虽然觉得秋谷说话希奇,却还口中胡赖,不肯承认。后见秋谷说出这一番说话,方才着实的有些怪异,又把秋谷打量了一回,料道他不是个来历不明的骗子,便倒反问着秋谷道:“就算我家中有人走失,却是外边没有风声。你一个素不相识的人,怎么倒说得这般清楚,究竟可有什么消息呢?”

秋谷微微笑道:“我不说一个明白,料你那里得知?但是和你讲明,你却不可动气。”程幼勋听了这样话风,更加疑惑,急急的逼着秋谷要他说明。秋谷便把自己坐的椅子挪前一步,附着耳朵,把当初贡春树和程小姐怎样私通,如何怀孕,贡春树如何着急,赶到上海要求他想个法儿,自己念着朋友之情,如何答应,如何同到苏州,怎样叫人打听,又如何自己暗中通信,把程小姐救出牢宠,现在程小姐还在自家船上,一五一十的好像背书一般,滔滔滚滚说了一遍。又说:“这件事儿,多要怪你自家不好。从来男大当婚,女大当嫁。你误了他的摽梅之候,怪不得要闹出事来。我虽然是个旁人,却不忍见死不救,眼睁睁看着你女儿一条性命生生丢在水中,所以我想个权宜之计,将他救了出来。如今事已如此,本来也不消和你说明,但是我明人不作暗事,特来和你讲个明白,好叫你自家心上分明。”

秋谷一面说,一面看那里老头儿的面色。只见他初起时低头不语,听到一半,早气得他满面通红,满头流汗,那颈顶上的青筋都一根根的爆将起来,就有些忍耐不住的光景。再听得后来许多说话,直把他气得七窍生烟,双睛出火,浑身乱抖,一口冷气塞住了咽喉,几乎透不转来。不等秋谷说完,再也按捺不住跳起身来,把秋谷胸前衣服一把扭住,大骂道:“你这个人好生大胆,你拐了我的女儿,还敢前来送信!你好好的把我女儿送出,万事全休;如若不然,我把你扭到当官,这拐逃的罪名看你可吃得起吃不起?”

秋谷见了这个样儿,甚是好笑,只是哈哈冷笑道:“你不用这样野蛮,有话只管请说。你家女儿好好的现在船上,又没有逃出苏州。我好意前来送个信儿,要和你商量个善后事宜,免得坏了两家的名气,你倒这样的横跳一丈、竖跳八尺起来,也不想个情理。你想天下那有这样大胆的棍徒,拐了你的女儿还敢自己上门送信,好等你送到当官,自寻烦恼,可有这样的痴子么?我劝你暂时放手,我倒有句话儿和你商量。我若怕你送官,也不自己跑到你家来了,难道我既然来了,又肯跑掉了么?”

程幼勋虽然愤恨,却听着秋谷的一番说话实是不差,又怕这个事儿闹了出来,自家平日极是个言规行矩的人,生了这样的女儿不能管束,还有什么脸面见人?不如听着他的话儿,还好暂时遮掩。想到此间,那一扭着秋谷胸膛的手,早不知不觉的缩了进来,长叹一声,重新坐下,却还是上气不接下气,张开大口,气喘吁吁,对秋谷道:“你…你有…有…有什么说…说话,和我…我…我商量,快些说来,说说你…你…你把我…我女儿,拐到那…那…那里去了。”

秋谷见那老头儿气急败坏的样子,忍不住要笑出来,勉强忍住了,正色和他说道:“你是个读书明理的人,怎么全不懂事情的轻重”你家女儿既已失足在前,你不叫他嫁姓贡的,却叫他去嫁那个?难道还好再嫁别人么?至于我,本来是个旁人,与我丝毫无涉,原犯不着来管你们的闲事,但我替你仔细想来,这件事儿已经如此,不如将错就错,彼此认了亲家,凭着姓贡的把你女儿带回家内,只当没有这件事儿。

到了明年二三月内,暗暗的把你女儿送回。那时叫姓贡的堂堂皇皇的托人说亲,圆成好事,一则掩了旁观的耳目,二则全了自己名声。若是你一定不肯通融,定要送官究办,我是旁人,自然只好由你。姓贡的和你女儿都安安顿顿的现在船中,凭你去将他怎样。但想姓贡的既然送到当官,你令爱也不免当场出丑,就是你老先生自己也免不得匐伏公堂。姓贡的犯的罪名不过是一个和奸,又不是什么谋反叛逆,将来这件事儿张扬开去,你却怎的见人?况且就是把姓贡的办了一个罪名,于你有何益处?你家令爱又不能重嫁别人,就算是堂上官员秉公判断,也是只有断合,没有断离,那有叫你家令爱重去嫁人之理?照这样的想起来,你那方才的盛气自然而然的一齐消化。还是听了我旁人的解劝,做个半截汉子,落一个好好的收场。请你自家斟酌一番,到底如何办法,官私两样,凭你怎样便了。“

程幼勋起先听了章秋谷解劝的话儿,还是咆哮不服,不料听到后来,越听越是有理,更兼章秋谷的粲花妙舌,说得来八面玲珑,没有一句话儿不是入情入理。真是那黄河九曲,层出不穷;三峡春泉,倒倾瀑布。就是再顽钝些的顽石,听了这般说法也要点头,何况程幼勋虽然闭塞不通,毕竟还是个人类,这些利害岂有不知?

听了这番说话,好似暗室逢灯,旱苗得雨,一霎时心地光明,觉得章秋谷的说话当真不错,渐渐的面上的气色也回了过来,沉吟了一回,叹口气道:“只是便宜了姓贡的这个畜生,实在有些不服。他引诱了我的女儿不算,还想要把他拐着同逃,难道就是这么让他过去不成?”秋谷笑道:“你不要这样糊涂。你令爱既然嫁了姓贡的,姓贡的就是你的东床。你若要把他送到当官照例惩办,非但伤了你家令爱的心,就是你老先生的面子上边又有什么好看?况且这件事儿原是万不得已,方才不顾危险,做这样干犯名教的事情。这正是姓贡的一片血诚,不肯负心的好处。若是换了将就些儿的人物,早把这件事儿撇在一边,那里还管别人的死活,却叫你家令爱将来怎的收场?如此看来,姓贡的也算不得什么坏人,不过是犯了些儿风流罪过,没有什么天大的事情。俗语说得好:”毛厕越掬越臭。‘我看还是将就些儿,凭他去了的好。“

程幼勋听了,想想实在不差,虽然有些强词夺理的地方,却是想不出一句驳他的说话。左思右想了一会,实实的无计可施,只得长叹一声道:“罢了,罢了!我就听了你的说话,便宜了这个畜生。我也只当没有这个女儿,也不用遮人耳目。那以后的话儿再也不消提起,这样掩耳盗铃的事情尽可不必。”秋谷道:“这却你又错了。我今天的来意,原是卫顾你们的府上的名声,你怎的倒是这般说话?”说着,又附了程幼勋的耳朵说了几句不知什么话儿,随后又道:“到了这个时候,仍旧把你们令爱暗暗的送到苏州,那时一样的央媒说合,一般的迎娶过门,那些不知细底的人那里看得出什么破绽?岂不把先前的这件事儿一齐都盖过了么?”

秋谷说毕,程幼勋正在沉吟,秋谷突然见屏门背后走出一个半老的妇人,约有五十多岁,走出屏门便向秋谷深深万福。秋谷连忙回礼。这妇人一屁股回身坐下,便对程幼勋道:“适才这位先生的话,我在后面已听得明明白白,真是再好没有的了。难得这位先生这样费心,顾全我们的面子,你还不快些答应,难道还想什么念头么?”程幼勋忽然被他的老婆走出来夹七夹八的说了一阵,想想除了这般办法,实在也没有别的法儿,只得勉强应允。

秋谷见他已经答应,立起身来便想要走,却被这妇人拦住道:“这位先生不要性急,且请坐下,我还有话说呢。”秋谷只得重又回身坐下,问他有什么话儿,叫他快说。妇人便唠唠叨叨的盘问起贡春树的家世来,秋谷一一的回答。妇人又问可曾娶过正妻,秋谷一想,这倒不好瞒他,便答道:“这个不好隐瞒,实在已经娶过的了。”妇人听了呆了半晌:眼中便流下泪来。秋谷明晓得他的意思,便接着说道:“他虽然室有正妻,府上的小姐过去,一定是姊妹称呼,决不亏待,这倒我可以和他做个保人。”那妇人又道:“现在事已这般,也说不得的了。只是他将来要是亏待了我的女儿,我却要和他们说话的。”秋谷道:“这个自然,但请放心就是。”

秋谷因费了半天口舌,说得他舌敝唇焦,巴不得要立时回去。只听那妇人道:“你们的船停在那里,我还要到你们船上看看女儿,还有他的衣箱、镜箱随身动用的东西,让他带去。”一句话还未说完,程幼勋睁起眼珠,向那妇人说道:“这样不要脸的东西,你还去看他做甚?难道台还给他坍得不够么?”他老婆听了正要和他争论,章秋谷因急于要走,便打断他的话头道:“程老先生的话儿却是不错,此刻正要遮人耳目,还是不要去的为是。就是衣服、镜箱也都不必拿去,免得露了风声。这些物件自有姓贡的和他置备,不消费心。”说着立起来把手一拱,急急的走出门去。任那妇人在后边呼唤,秋谷只作不闻,飞也似的回到自家船上。见春树已经回来,置买了多少服用之物,正和程小姐在那里挑看衣服。

秋谷看程小姐已经梳洗,梳了一个懒妆髻,薄施脂粉,又换了一件衣服,出落得别样风流,千般袅娜。昨天晚上还是粗服乱头,花枝寂寞,如今却已是明妆丽服,环佩凌波,小蛮杨柳之腰,樊素樱桃之口,双涡晕酒,二笑倾城,比起昨夜好像换了一个人的一般。见了秋谷回来,一齐立起。春树连忙问道:“到底怎么样,没有碰到钉子么?我倒狠狠的替你耽心,幸而还没有怎样。你想那有拐了人家内眷,还自己送上门去告诉他?虽然没有闹什么乱子,这个胆量也就佩服你了。”秋谷笑道:“你只是一味的胆小,晓得什么!我是看准了这件事儿准定闹不出什么乱子,所以才这般胆大。你想我章秋谷要是没有这般胆量,那里担当得起这样的事情?”说着,便把刚才的说话一一说了一遍,又笑道:“这一本戏文,生、旦、净、丑都是我一人独唱,作成你做一个现成快婿、自在东床,你还不要好好的谢谢媒人么?”

春树听了,也无别话可说,不住的点头痛赞,佩服秋谷的辩才智慧直到二十四分,感激秋谷的侠骨热肠更是五体投地。连程小姐在旁听着,也是感激万分,那心上的感情深深的印入脑筋,竟是个留了终身纪念。这也不去说他。

只说秋谷和春树商量,叫他坐着原船和程小姐一同回去,秋谷便在苏州城外暂落客栈,等贡春树到了苏州,一同再到上海。计议已定,秋谷忽又想起一件事来,便问春树苏州的几所住房那一处最大些,可肯出卖。春树道:“我的房子只有宫巷的一所住屋最是大些,只要有人肯出价钱,那有不肯出卖之理?”秋谷便把宋子英和亲戚代寻房屋的事同他说了,并道:“你既然肯卖,不妨找了子英,同他去看,好在你今天不能动身,我们就同去一趟可好?”

春树答应了,一同上岸,先到王小宝院中寻着了陆仲文,再托陆仲文写张条子,当场把宋子英约来,和他说了。子英大喜,便要立刻去看。当时由院内相帮雇到三乘轿子,章秋谷和春树同宋子英三人同坐,一直到宫巷潘玉峰家。春树请秋谷、子英暂在大厅少坐,自己进去了一会方才出来。有分教:

画舫笙歌之夜,檀板金尊;呼卢喝雉之场,崖勒马。

要知后事,且看下回。





第五十七回 贡春树一棹载名花 章秋谷良宵圆好梦





再说贡春树同宋子英、章秋谷到潘玉峰家,暂请他二人在客厅坐下,自己进去了。一会出来,便请宋子英和章秋谷二人同到里边,春树陪着在前领路。宋子英前前后后各处看了一回。那一所房屋一共有五开间五进,头门进去,便是五间大厅,第三进是三间花厅,两旁另有两间书室,花厅背后有一座月亮门,一个大大的院落,有几处鱼池山石,松阴藤架,花木萧疏,布置得十分幽雅,再往后边两进便是上房。

宋子英看了一遍甚是合式,口中不住的赞好,重新回到大厅坐下。那大厅的前进便是头门,大门却开在偏左一边,进了大门向右转湾,却还有三间轿厅,头门左首便是门房,宋子英也去看了一遍,便向春树请问价钱。春树道:“我们既是要好弟兄,我也不说虚价,老实和??翁说,你们令亲果然要买,叫他出一万银子。这还是你老哥来说,又有章秋翁一力作成,要是换了别人,他就是多出些儿,我也未必肯卖。”

宋子英听了,道:“一万银子并不算贵。既承你春翁答应肯卖,我便竟是斗胆代我们舍亲定了下来。但是还有一件事儿要和你春翁商酌。如今的规矩,置备什么产业都要先付定洋,这所房子既然兄弟答应下来,理应先付些儿定洋才是,无奈兄弟到此已经日久,旅费有限,一时凑不出大注银钱。好在前日接着安徽来信,说舍亲已经进京,先派一个姓箫的账房到此替他料理事情,大约总在这几天可到。等他到了之后再付定银,不知你春翁可能相信得过?”贡春树连忙一口答应道:“定银不定银尽管随便,你我既然相识,何必要这样拘泥?况且有章秋翁在里头经手,难道我还有什么不放心么?”宋子英道:“虽然如此,也要预先说明,既承你春翁看得起我,那是再好没有的了。”说着便仍旧同着秋谷,春树坐轿出城,宋子英便拉着秋谷二人到王黛玉家小坐。

王黛玉要叫宋子英吃酒,宋子英起头不甚愿意,没有爽爽快快的答应。王黛玉见他不肯,便走过来和他不依,坐在宋子英身上,一手勾着他的颈项,一手揪着他的耳朵,两人滚作一团。王黛玉更伸出一只玉笋一般的纤手,在宋子英两边脸上,“劈劈拍拍”的不住乱打,打的那声音好像知县堂上打着犯人的一般。章秋谷和贡春树坐在一旁,看见这般怪相,忍不住哈哈大笑起来。王黛玉只当没有听见,更加力的去拧宋子英的大腿,拧得个宋子英抱着头苦苦的告饶。王黛玉只是不理,直到了宋子英答应了他吃一台酒,方才放他起来,却还口中咕噜道:“耐阿敢勿答应呀,勿答应末,晏歇点办耐格生活。”宋子英刚刚坐起身来,听见了,把舌头一伸,打着苏白“嗤”的笑道:“耐格生活,倪昨日仔夜里向已经吃着格哉;今朝再要办倪格生活,是倪吃勿消格哩。”一句话说得秋谷等又笑起来。王黛玉急了,又要走过来拧他的嘴。宋子英连忙告饶,方才罢了。

王黛玉用一个手指头,用着气力在宋子英额上点了一点,道:“耐格人末勿知啥格骨头,敬酒勿吃要吃罚酒,倪恨得来!”宋子英正要回答,秋谷剪住他的话道:“算了罢,不用大家斗口,还是早些摆起台面来,我们吃了还要早些回去,今天晚上还有些料理的事情。”宋子英依言,便写了几张请客票头,叫相帮快些去请。

除了陆仲文、方小松之外,还有两个客人,一个姓顾,一个姓李,也都是城内有名的绅富。

相帮去了一会,方小松同陆仲文同来;又等了一回,顾、李两人也就到了。宋子英见客已到齐,发过局票,请客人席。那姓顾的名叫顾云卿,叫一个小清倌人,叫花二宝。姓李的名叫李子刚,叫的倌人叫金惠卿。当下坐了不多一会,又是金缓缓第一个先来。方小松见了先喝一声彩,众人也随声附和了几句,随后各人的局也都来了。宋子英酒量颇好,便抢着先要摆庄,众人因他是个主人,让他先摆。宋子英就独摆了五十杯,先和李子刚出手,五魁对手的乱喊起来。

秋谷本来是个爱静的人,不去理会他们,只回过头来和金媛媛密密的谈心。金媛媛道:“耐来仔好几日哉,阿要到倪搭去吃一台酒,请请客人?”秋谷一笑,尚未开口,金媛媛接着说道:“勿然是倪也勿是一定要耐吃酒,像煞俚笃说起来,总说倪搭仔耐两家头,做末做得蛮要好,为啥酒也勿吃一台?轧实倪做仔客人,搭客人要好起来,倒勿在乎吃酒勿吃酒。不过?俚笃格排人,总是实梗说法。耐阿好去吃仔一台,绷绷倪场面?”秋谷听了,不觉暗暗赞叹,便点头答应道:“你既然这般说法,我自然要绷绷你的场面,等回儿这边散席之后,翻台过去便了。”金媛媛听了大喜,加倍奉承。秋谷口内这般说着,心上却想着“金媛媛的应酬实在不差,不意苏州地方也有这般名妓”,便不觉也和金媛媛亲热起来。

这边席上,宋子英摆了五十杯庄,众人轮流交手,互有输赢。方小松等一个个一齐轮过,只有秋谷只顾和金媛媛说话,也不去管搳拳的输赢,直至宋子英要找他交手,方才打断了话头,两个便交起拳来。不料章秋谷意不在此,随便应酬,竟连输了十几拳,喝了十余杯急酒,不觉就有些头晕眼花。金媛媛看了,便把台面上的两盆水果──一盆荸荠一盆甘蔗拿了过来,叫秋谷吃些过酒。又亲手取两个荸荠放在秋谷口中,秋谷吃了几个,方才觉得头目清凉。因为连输了十余拳,不肯伏输,攘肩而起,又和宋子英搳了十拳。这回秋谷不敢怠慢,用着十分的小心去对付他。

果然宋子英被他捉住,也输了八九拳,方才把宋子英拳庄打掉。方小松连着又摆了三十杯,秋谷打了十拳,输了四杯。秋谷将四大杯拳酒折在一个玻璃缺内,正要叫金媛媛代吃,方小松嚷道:“不准代酒,代的要罚十大杯。”秋谷听了,只得仍把玻璃缸放在自己面前,却被金媛媛从肩上身伸过手来抢了过去,一口气咕嘟嘟的饮干,放下杯子,面上早添了一层红晕。方小松见金媛违例代酒,也不言语,自家取过酒壶,又叫娘姨取了三只大玻璃杯过来,放在桌上,斟了满满的三大杯酒,向金媛媛笑道:“你有心违令,定要罚你三杯。”秋谷和金媛媛讨情道:“他见我刚才多吃了几杯,有些醉意,怕我喝醉了,才和我代的,并不是有心违令,你不要这样顶真。”方小松那里肯听,一定要罚他三杯。金媛媛瞅了方小松一眼道:“方大少倪搭耐讲讲格个道理。看耐搭二少是要好朋友,不比啥格别人。二少吃醉仔酒末,只有耐方大少劝劝二少,叫俚少吃两杯,勿要吃坏仔自家格身体,格末像格要好朋友啘。阿有啥朋友吃醉仔酒,再要灌俚两杯,倪搭俚代仔,翻转来倒要罚倪格酒,唔笃想想看,阿有格道理?”金媛媛这几句话,把个方小松倒说得哑口无言,只得笑道:“晓得你们两个是恩相好,所以要在我们面上摆个样儿。”秋谷见方小松这般说法,知道他理屈词穷,乘势再和媛媛讨情,方小松也便依了。秋谷又约众人翻台到媛媛家去,众人一齐应允。散席之后,同到金媛媛家,一个个逸兴横飞,豪情遄发,直吃到晚上十点多钟,方才大家散了。春树自回船上,秋谷便住在媛媛院中。

到了次日,因贡春树要送程小姐回去,午刻便要开船,秋谷便到船上,把自己的几件行李发上岸来,就在宋子英住的长安栈内暂住。叮嘱了春树一番说话,叫他快去快来。又问:“他房子的事情怎样,可要等你回来?”春树道:“你在这边也是一样,诸事听你如何调度。尚若那边付了定洋过来,你不妨和我代收。我们这样的交情,难道还分什么彼此么?”当下贡春树又交代了宋子英一番,叫他房子的事情只要去请问秋谷,定洋也交在秋谷手中,“凡是他答应的什么事儿,我决不参差反悔。”说着,又和秋谷说了几句,匆匆的下船走了。

再说章秋谷住在苏州,专等贡春树到来同走,却没有什么事情,只天天和陆仲文、方小松在堂子里头打混。等了几天,贡春树还不见来,秋谷甚是焦躁。

那一天秋谷住在栈中,直睡到午后方才起身,略略吃些点膳,觉得甚是无聊,便走到宋子英房内,打算要和他谈谈。刚刚走进房门,只见子英房内挤了一房的人,坐得满满的,七张八嘴的不知在那里谈论些什么。秋谷觉得不便,缩住了脚,正待退出,早被宋子英看见,连忙立起身来招呼进内。秋谷见他房内人多,不愿意进去,对着子英摇摇头道:“你只顾招呼朋友,不必同我客气。我们停会在王黛玉那里见罢。”宋子英见他不愿进房,只得罢了,却再三嘱付:“少停一定要到王黛玉家,我在那边等你。”秋谷答应了,便信步走出栈门,想到王小宝家去,问一声陆仲文可在那里。

走得不多几步,劈面来了两担行李,十分沉重,看那挑夫样儿挑得甚是吃力,头上的汗就如珠子一般。行李后面跟着一个人,低头急走,身上衣服虽然华丽,却宽袍大袖的不合时样,看他那样子就是一个寿头。那人跟着两担行李,急急的转了一个弯。不防章秋谷正在那转角上走来,正和他撞了一个对面,那人低着头儿,那里看见?竟是一直的向章秋谷怀里撞来。两边避让不及,躲闪不开,眼看着就要撞在一起,幸亏章秋谷眼明手快,伶俐非常,见对面有人直撞过来,急把身子略略一偏,趁着势儿就让了开去。对面的人来得势猛,那里收得住步儿,又被章秋谷把身子往左一偏,上面撞了一空,脚下绊了一绊,立脚不住,一个狗吃屎直扑下去,跌得他脊背朝天,胸膛着地。两旁走路的人看了这般光景,一齐大笑起来。秋谷也甚是好笑,反立定了脚看他。只见他跌在地下,扒了半天还扒不起。秋谷倒有些过意不去起来,走过去,轻轻一把就把他拉了起来。看他的面貌时,獐头鼠目,缩嘴短腮,不像是本城人氏,果然听他开出口来,是安徽一带的声气。当下那人跌了一交,跌得他浑身生痛,正在扒不起来的时候,忽然秋谷过来把他扶起,不免倒谢了几句,便各自分头走了。

秋谷回头看时,见他跟着挑夫径到长安栈里去了。秋谷暗想:原来也是住栈的人,却也不去管他。一直就走到王小宝家,一问陆仲文不在那里,并连王小宝也不在家,和仲文一同去坐马车去了。娘姨要请秋谷进房略坐,秋谷不肯。走出王小宝的大门,见有几部马车停在道左,正在那里兜揽客人。还有几匹川马,一般的歇在路旁,锦辔雕鞍,昂头掉尾,形状甚是神骏。秋谷暗想:怎么马路上边也有这般好马“正要近前打量,不防马车上有两个马夫认得秋谷,晓得就是上半年余香阁点书、甘棠桥跑马的章老爷,便围将拢来,你言我语的兜搭,要想做秋谷的生意。秋谷正在纳闷,便拣了一部绣花靠枕、闪光纱车垫的马车。那两个马夫都穿着一身外国纱的号衣,精光射目。正是:

珠帘十里,谁家白面之郎;玉漏三更,何处行云之路。

欲知后事如何,请看下回分解。





第五十八回 驰宝马争看绿衣郎 博枭庐埋冤曲辫子





且说章秋谷拣了一部最精致的马车,叫马夫放到石路口金媛媛家门口等候,自己却不坐马车,又拣了一匹小川马,把右手在马鞍略略的一搭,飞身而上,马夫递过丝鞭,秋谷加上一鞭,追上前面的马车。到了金媛媛门口,跳下马来急急的进去。

不一刻,同了金媛媛出来,叫他坐上马车,自家依旧骑马相随。到了马路中间,秋谷骑在马上放出手段,带紧丝缰,马后股连加几鞭,那马放开四蹄,就如腾云驾雾一般往前跑去。秋谷扬鞭揽辔,意态自豪,一霎时早追过了几十辆马车,耳边只听得呼呼风响。那些马路两帝的住家倌人,到了三四点钟差不多夕照衔山的时候,一个个坐在洋台凭栏眺望,见秋谷骑在马上灵便非常,更兼衣服鲜华,形貌秀丽,那马飞一般的在马路上往来驰骤,风吹衣袂飘飘欲仙。那些倌人见了,不约而同齐声喝彩。秋谷在马上听见甚是得意,跑了几个圈子方才勒转马头,追上金嫒媛的马车,慢慢的走。又跑了几趟,已经将近上灯,秋谷也觉兴尽,同着金媛媛回来,开发了马夫,把金媛媛送到楼上。想着宋子英约他在王黛玉家,恐他久等,便走到黛玉院中。一问宋子英已经来了一趟,有什么朋友约他出去说话,临走的时候,招呼房间里娘姨,请秋谷进房坐等。秋谷也无可不可的进房坐下,王黛玉陪着。

闲谈了一回,宋子英还不见来,秋谷觉得无味。正待立起身来要走,忽见门帘一起,走进一个人来。秋谷以为定是宋子英来了,岂知定睛一看,竟不是宋子英,就是方才在长安栈门口跌了一交的那个寿头码子,又换了一身簇新的衣服,后面还跟着一个人,匆匆的举步进房,正和章秋谷撞个正着。王黛玉见了两人,也不认得,还只认是和秋谷相识的熟人。秋谷当时摸不着头脑,见他们无缘无故的闯进房间,不觉怒从心起,竖起双眉,刚才开口骂了一句:“你这两个糊涂虫,怎么人也不认识,乱闯别人的房间?”正还要骂下去,猛见门外又走进一个人来,哈哈大笑道:“不要骂了,都是自己一家人。”秋谷听丁,方才住口不骂,举眼看时,原来第三个进来的人便是宋子英。秋谷晓得自家性急了些,却又不肯认错,只得向宋子英笑道:“我一时失口,得罪了你的贵友,莫怪莫怪。但是还有一层道理,不能怪我出口伤人。为什么呢?这里王黛玉院内是你宋子翁做的地方,这两位既是初到此间,你却不该让他先走,自家倒反缩在后面。我看见了他们两位,只认是闯房间的客人,所以开口骂了几句。你想这件事儿可不是你的错处么?”宋子英不等说完,哈哈笑道:“算了算了,就算是我的错处何如?你不晓得我们这位同乡,没有到过苏州、上海,老实说是个曲辫子儿,不懂堂子里头的规矩。他们刚刚走上楼梯,便三脚两步的走进房门,我那里追赶他们得上!恰恰的来迟一步,你已经在房里骂起来。你想想,叫我那里有这么的长脚?”秋谷听了不觉好笑起来,不再去和他说话。回过头来,便问那两人的姓名,彼此寒暄了一回。

原来那先走的叫萧静园,便是宋子英说的邹观察派来办事的账房;后随的叫汪慕苏,也是宋子英的亲戚,到苏州来顽的。当下一一通名已毕,章秋谷留心打量二人究竟是个怎样的人物。看了半晌,觉得这两人的形景甚是好笑:身上的衣服虽然华丽,却真有些像曲辫子的样儿,坐在那里动也不敢动,头也不敢抬,低着头目不邪视,好像高僧入定一般。萧静园更是好笑,他听见宋子英说他们是曲辫子,他虽然不懂,却牢牢的记在心中,私自拉着宋子英问道:“你刚才说的‘曲辫子’是个什么东西?我的辫子,是刚在栈房里头叫剃头的打得好好儿的,怎么一回儿就得弯呢?”宋子英不听此言犹可,听了他这般说法,忍不住笑得前仰后合,拍手弯腰,眼泪都笑出来了。章秋谷更笑得蹲在地上立都立不直,气都透不过来。王黛玉也笑得“格格支支”的,把一方小手巾掩紧了口,兀自笑得伏在桌上,几乎要滚入宋子英怀中。房间里娘姨大姐等人,一个个都笑不可仰。好一会,才大家止住笑声。萧静园还不懂笑的是他,鼓着腮帮子,一付正经面孔,问道:“你们为什么这般好笑,说了些什么东西,怎么我一句也听不出来呢?”宋子英听了又笑起来,拍着萧静园的肩膀道:“老弟,你算了罢,不用怄人了,这里头的筋络,你那里一回儿就弄得清楚?下回我劝你少说些儿,省得给别人笑话。”萧静园听了,方知他们笑的是他,只把他羞得满面通红,一言不发;连汪慕苏听了,脸上也红起来。秋谷见了恐怕他们老羞成怒,大家不好看相,便用别的话儿忿了开去。

当夜宋子英和萧、汪二人接风,就在王黛玉家吃了一台。席间说起房子的事情,宋子英便向萧静园道:“前天我看了一所房子,甚是合式,但是还没有付得定洋,不知你带了多少钱来?”萧静园道:“我虽然带了些银子出来,要付定银只怕不够。”

宋子英道:“定银不拘多少,就少些也不妨,明天我同你先去看一趟房子,再付定银可好?”萧静园点头应充。宋子英又和秋谷说明,要请他同进城去,秋谷也答应了。当下席终之后各自散去。

到了明天,果然宋子英同着萧静园来约秋谷一同进城。萧静园看了房子也说甚好,便问秋谷要付多少定银。秋谷道:“这个不拘多少,听凭尊便就是了。”宋子英一口答应,先付一千银子定洋,约定日期照付,暂交秋谷代收,萧静园也就应了。

三人仍旧一同出城,萧静园因要到钱庄去照验汇票,就在半路分头自去,秋谷只同了宋子英一起出城。

隔了两天,约付定银的日期到了,只见宋子英走来说道:“这两天那位萧公同着汪慕苏甚是奇怪,看他心神不定,好像一刻都坐不住的一般。昨天晚上没有回来,临走的时候我还问他,应付的定洋明天怎样,他说已经预备,只要去划好了票子送来。今天到这个时候还不回来,我倒狠替他们耽着心事,不要他们两个土地码子到各处混跑,闹了什么乱子出来,这可不是顽的。”秋谷道:“他们虽然初到苏州,料还闹不出什么乱子,你只顾放心。”

正说着,已见萧静园走了进来,子英埋冤他道:“你怎么这样的忙法?昨天没有回来,今天直到这个时候方才回栈。不知你在那里耽搁了一夜工夫?如今也不必说了,前天说的定洋怎样,票子可曾带来?”萧静园听了,低着个头一句话也说不出来。宋子英连问了几遍,不见萧静园答应,十分诧异起来。秋谷也觉得不解。宋子英立起身来,逼近萧静园的身旁再三追问,方见他无精打彩,丧气垂头,一付不高兴的样子。宋子英看了,明知事有蹊跷,越发逼住了问他。萧静园起初还不肯说,后来被宋子英追得急了,方才叹一口气道:“不必说了,总是我自家不好。忽然一时高兴,和他们赌起钱来。一夜工夫,输了一千九百多两银子,把带来的两张汇票一齐输掉,定钱是付不成的了,只好随后再想法儿。”宋子英还没有听他说完,直跳起来道:“怎么说,你一夜工夫输了一千九百两银子,你在此间没有认得的朋友,怎就有人合你赌钱,又怎的会输这许多?你且说说我听。”此时秋谷在傍听了,也不觉惊心,便侧着耳朵听他说些什么。萧静园料想隐瞒不过,只得实说道:“我原不认得这一班人,多是汪慕苏的朋友,还有几个是钱庄上人。昨日他们雇了一号灯船,请汪慕苏去游虎丘,连我请在里头。他们一班人闹到晚上,高兴起来,便约我们二人同赌,我同汪慕苏不合一时答应了他,胡乱入局。起先原是想赢的,不料入局之后,有输无赢,输到后来,大家发起火来,便一百两、二百两的重打,不到半夜,把两张汇票一齐输得精光。你想这件事儿如何是好?若是我自家的钱,输掉了也还罢了,偏偏都是东家的银子,叫我带到苏州和他办事,如今输得两手空空,叫我怎生设法?”萧静园一头诉说,急得满头是汗,那面上的形景做得甚是为难。宋子英听了,连连顿足道:“你怎么做出这样的事来?如今银子已经输得精光,还有什么法儿可想!你自家想想,可怎的对得起人?”萧静园听了,那里答应得出来,默默无言,逼得面红颈赤。

宋子英又想了一回,问萧静园道:“你们还是赌的牌九,还是赌的摇摊?怎会输这许多,不要你寿头寿脑的去上了别人的当罢?”萧静园道:“赌的不是牌九,也不是摇摊,他们说起来叫做什么‘抓摊’,是用一把棋子盖在茶碗里头,叫人打的。”宋子英道:“做庄的人可是随意抓一把棋子,把茶碗合在上边,那茶碗上横搁一只筷子,等你们大家打定,再把茶碗移开,用筷子拨着棋子的多少,可是这样的赌注么?”萧静园道:“一些不错,正是这个样儿。”宋子英把桌子一拍道:“如此说来,你果真上了别人的当,冤冤枉枉的去送掉这许多的钱,真是糊涂到极处的了。”萧静园听了,有些疑疑惑惑的,不肯相信道:“据我看来,这个抓摊里头,不见得做得出什么手脚。况且这一班人都是汪慕苏的朋友,料想不至于做弄着他,若说是汪慕苏串同了别人前来哄我,我看慕苏虽不是一定什么正人君子,但他是个有钱的人,决不肯做这样的事情。更兼他昨天晚上比我输得更多,那里做得出什么花样?我劝你不必疑心,不过我的运气不好,所以输这许多罢了。”宋子英冷笑道:“你这个人真是二十四分的糊涂,自己输了银钱还说没有上当,天下那有这般痴子!你还当汪慕苏的一班朋友都是好人么?他们遇着了你们这一对寿头码子,不弄你们的钱,却弄那个的钱?难道他们做了这行翻戏的生意,喝西北风不成?”

萧静园听了似乎觉得有理,便有些半疑半信起来。还未开口,宋子英又道:“说起那汪慕苏来,自然不是有心做你,但他的为人比你更加无用,自己已经输得一塌糊涂,还能来照顾你么?你说抓摊里头做不出什么手脚,待我细细的说与你听。

他不是做庄的时候,要拿一只筷子搁在茶碗的底面么?这就是他们的暗号:用一个指头拈那一根筷子,便是做的幺门;两个指头,便是二门;三门,四门都是一样。

他们一班同伙的人在旁看了,自然领会得来。这里头的弊病真是说他不尽,怎的你还这样的糊涂?“萧静园听他说得抓摊的毛病,方才恍然大悟,自家懊悔万分。宋子英又道:”如今事已过去,追悔他也是枉然,倒是你自己的事情要紧,输了二千两银子,一时那里弥补得来?最好今天你先想个法儿,把房子上的定银付了,其余的或者我再替你慢慢的弥缝,若叫我们舍亲晓得,你这碗账房的饭那里还吃得成?“

萧静园道:“我正要请你和我想个法儿,你在此间认得的人多了,或者有些法想,也未可知。”宋子英皱着眉头道:“我虽然有些认得的朋友,却没有通融钱债的交情,你何不到汪慕苏那边暂借一二千银子,救了如今的燃眉之急,随后便好慢慢商量。”正是:

欲擒故纵,淮阴背水之兵;一掷千金,刘毅呼卢之技。

欲知以后如何,且听下回分解。





第五十九回 萧静园输钱重约赌 王云生设计报前仇





且说萧静园听了宋子英的话,皱着眉头连连摇手道:“你还要提起汪慕苏,还当他是什么慷慨人物么?我不然也不至于到此刻回来,就是在汪慕苏那里坐了半天和他商量,要向他暂借一千银子凑着付今天的定钱,慢慢的再设法还他。谁知他非但分文不借,反把我数说了一场,说我不应这样的荒唐,刚刚到得苏州,便把带来的银子一齐输掉。又说他现在虽有几千银子,因为昨天输多了,要做翻本的本钱,那有多余的钱出借。唠唠叨叨的说了一大篇儿,我被他气得昏了,一句也没有回答他,只得跑回栈房向你设法。你还没有晓得汪慕苏的脾气,输起来一千八百,三千五千,不以为奇。越是输得利害,越是赌得利害。若是有个朋友要问他设法借钱,他就立刻翻转面皮,回答得斩钉截铁,真猜不出他是个什么性情。”

宋子英听了,沉吟中语,停了一回方又问道:“昨天晚上慕苏输了多少,可曾拿来现钱来么?”苏静园道:“慕苏输得比我更多,输了三千一百多两银子。见他拿了一张三千两的庄票出来,其余的多是现洋。”宋子英诧异道:“你们总算是书房赌,怎么会输这许多?”萧静园道:“我是输到后来发了火性,打得大了,所以输了这些。慕苏自己虽然打得不大,却专爱移吃别人的注目,把别人压的不论多少,通通吃到自己一门,开出来偏偏又被庄家吃了,慕苏却要照数赔人,所以上家虽然赢钱,下风却个个不输,单单的输了我们两个。你想这不是性气么?”

宋子英扑嗤的冷笑了一声道:“明明是你们两个寿头去上了他们的圈套,却还在这里糊涂。如今钱已输掉,追也追不转来,你做了这一笔亏空,总要想个法儿才好,难道凭他这样么?”萧静园听了,呆了一回方开口道:“你想我初到此间,有什么法儿好想?不比你在此地长来长往,无论如何总有几个熟人,这件事情总要仰仗你的大力替我想个法儿,料理开了我自然日后也有补报得着你的地方,千万不要推诿。”说着,就立起身来朝宋子英作了一个揖。宋子英摇头道:“我如今是个客边,和你一样,怎么一刻儿工夫就借得出这许多银子?就是借起钱来,只好二三百银子,多至四五百银子,还好和你转转手儿,那里凑得出一千银子?”萧静园听了宋子英真无法想,不觉双眉紧锁,满面愁容。又附着宋子英的耳朵说了半晌,仿佛都是央恳他的话儿,看那萧静园的神气,十分着急,脸上边显出为难的样子来。

秋谷在旁听了半天,觉得自己叉不进话去,便立起身来要想出去,却被宋子英拦住道:“章秋翁且请坐下,兄弟还有事情要和秋翁计议。”秋谷听了只好坐下。

只见宋子英听了萧静园的说话,一会儿点,一回儿摇头,不知他心上想些什么。直到萧静园把话说完,宋子英也不言语,默然半晌,好像心上在那里打算什么事情,约有两刻钟的工夫。秋谷看着心焦,又不好走了出去。又等了一回,宋子英方向萧静园道:“法子是替你想了一个在此,只是我不犯着为你的事,做出这样事情,如今也说不得了,要救你的一时之急,只好这般办法,拿他来顶个缸儿的了。”这几句话儿,不但萧静园听了摸不着头脑,连章秋谷也不懂起来,急急的要听他说下去。

萧静园更是眼睁睁的看着宋子英的脸上发怔。宋子英看了笑道:“我不说个明白,你们自然不懂,在我的主意,要把你昨天晚上输掉的钱一齐在汪慕苏身上拿他回来。

好在你输的钱,都是汪慕苏的朋友赢了进去,你本来不认得这一班人,算起来总算是他连累你的;况且他眼见你输了二千银子,方才问他开口借钱,他竟是一毛不拔,还要把你数说一番。像他这样的啬刻,也不是什么有肝胆的好人,我们就是算计了他,也算不得伤天害理。“

宋子英这一番说话说得没头没脑的,萧静园更不知他说的什么。章秋谷素来是一个性急的人,这一下子的闷葫芦可把他呕得急了,立起来向宋子英道:“你说了半天的话,牵枝带叶的一大套儿,我听了半天听不出你是什么意思,不知你说的到底是那一路的话儿?真是京戏里头《翠屏山》潘老丈说的:”你不说我还有点明白,给你这么一说,我可更糊涂了。“你方才的一篇说话可真把我搅糊涂了。”宋子英听了,自己也觉好笑道:“这是我自家不好,没有说得明白,难怪你们不懂。待我慢慢的说出缘故来,你们就晓得了。”说罢,便问萧静园道:“你不是说那汪慕苏的赌品十分利害么?”萧静园道:“怎么不是!”这个赌法我从来没有见过,可真是少少儿的。并且他还有一种脾气,不懂他是个什么性情,你们压着幺门,他偏要吃到三门上去;你们压在四上,他偏要吃到二门上来;你们越是压得多,他越是吃得高兴,凭你压得再大些儿,他也总是要吃。若是他本来压的进门,只要见别人跟了他一记进门,他就赌气把自己的注目连别人的注目,移的移,吃的吃,一齐放到出门去了。一刻儿的时候,输了一千二千,他却毫不放在心上,你想这般赌法,不是有意和银钱作对么?“宋子英听了大喜道:”既是如此,这是再好没有的了。我想他既然爱赌,只要有人约他赌钱,他一定没有不到的。我们何不约几个人,凑些本钱,去把他约到此间和他赌上一赌,彼此打个暗号,齐心捉弄着他,怕他不输掉三千二千银子么?那时把你输掉的钱在他身上翻了回来,可不是个稳稳当当的主意?

虽然论起理来,这样的事情不是我们做的,但是你输了这一笔钱,事体十分尴尬,也叫作出于无奈,不得不这样的腾挪。况且他是个有钱的人,也不在乎这几千银子,与其叫他去输给别人,落得补补你的亏空,你想我的主意可还不差?“

萧静园不等宋子英说完,连连的点头道好道:“你这个主意想得真是聪明。一时除了这个从权的法儿也想不出什么道路,顾不得他平日的交情,只得是要这般一做的了。”宋了英道:“还讲什么朋友的交情!他若还念着平日交情,见你这样为难,就该和你想个法子才是,难道他是拿不出银子的人么?”萧静园听了连声道是。

宋子英又向章秋谷道:“刚才兄弟的话儿,秋翁想已听得明白,不知可好屈尊些儿,到那约赌的一天请秋翁等一同到场。人多了,觉得好看些儿,总请秋翁枉驾帮帮静园的忙。”章秋谷起初听得宋子英忽然想出这个主意,要翻汪慕苏的钱,心上就有些觉得不以为然,却为的与自家无涉,不好去劝阻他们,后来又听得宋子英要约他同去,便想一口推辞。不料一刻之间又转了一个念头,想道:“这件事儿,不晓得他们究竟怎生做法?我却从来没有看过。到了那一天去看看热闹也是好的。”

想罢,便高高兴兴的答应了一声。宋子英不胜之喜,拱手相谢,连那萧静园也说了无数的好看话儿。宋子英又细细的和秋谷说明关节:“只要看做庄的人拿筷子的时候是几个指头,倘或是一个指头,便是进门,赶紧先把自家的注目放到进门上去。

汪慕苏既是这般公子哥儿的脾气,一定要把我们的注目吃到别门上去,好显他的威风。你们只要压得大些,怕不赢他三千五千银子?只消把静园输的捞了转来,也就罢了,我们也不是做这样事情的人。“秋谷听了,只得也随口答应。

萧静园道:“话虽如此,却打算在什么地方呢?”宋子英道:“这个地方,却要想得稳当些儿,客栈里是不便的,堂子里更加耳目众多,给他们传说出来不是顽的。”想了一想道:“有了有了,你前日输钱,是他们请你坐灯船逛虎丘。如今七月天气,正是游虎丘的时候,我们不如也雇一号灯船,专请汪慕苏去游虎丘,索性连陆仲文、方小松都请里头,多几个人,也好壮壮我们的威势,你道这般可好?”

萧静园听了连连点头,又恭维了宋子英几句,便也散了。

章秋谷回到自家房内,却不免心上有些疑惑起来,想着他们好好的忽然要赌起钱来,虽然他是想骗姓汪的银钱,原与别人无涉,但是同在一起的人,免不得总要小心防备,不要他们内中有甚圈套,上了他的钓钩,那时就懊悔嫌迟了。想了一会,觉得他们似乎有些形迹可疑的地方。忽又回头一想,断没有这个理儿,他们骗姓汪的,又不要我旁人拿出钱来,何必这样的瞎费心思,多疑多虑。况且姓汪的也是他们一帮,就是他们赢了他一千二千银子,又不是外帮的人,与我什么相干?再看萧静园的样子,一付土头土脑的神情,不像会什么假话,就算他竟是假的,我也要看看他们到底怎样的骗人,如何的下手,也算是我到苏州来阅历一番。不要说是他们这几个人儿,就是夏间在上海的时候,王云生做那仙人跳的勾当,被他拿着了真凭实据,尚且凭着我的嬉笑怒骂,竟是无可如何。这样冒险的事情我都不怕,难道到了今日之下,倒怕了他们这几个人么?想到此间,便不知不觉的放宽心事,看着宋子英、萧静园这般人物好像小孩子的一般。

看官,你道宋子英和萧、汪二人究竟是何样的人物?原来果然是一班倒脱靴的赌棍、翻天印的流氓,就是王云生的一班党羽。章秋谷梦里也想不到,他们和王云生都是一起的棍徒。王云生自从在上海吉升栈内被章秋谷说破机关,栈内存身不住,只得当时回转苏州。可怜花了多少本钱,费了许多心血,指望好把章秋谷当场讹住,诈一注大大的银钱,想不到章秋谷机警过人,精明出众,非但弄他不倒,反被他当场叫破,吃了一场天字第一号的大亏,从此上海地方不能再到。王云生回到苏州,把个章秋谷恨得咬牙切齿的,恨不得当时把他捉住通上几刀,方出这一口恶气。气到极处,只得会齐了一班流氓戏子、光棍马夫计议这件事情,要报这个仇恨。无奈章秋谷现在不在苏州,也不着他的什么花样,想来想去想不出什么法儿,只得大家叹一口气,认个晦气也就罢了。

近来王云生因合着一班流氓在租界上拆梢,被巡捕扭到捕房关了一夜,解到工程局来。工程局委员问了一堂,把他枷在青莲阁门口示众。伽到一月期满,责释出来。租界上犯了这件案情,出头不得,只得又去给了宋子英等一班赌棍,做那翻天印、倒脱靴的勾当。城里狠有几个初出茅庐的乡绅子弟吃了他们的亏。近来宋子英又看上了陆仲文,想着他滥赌狂嫖,一定有些油水,便要想个计较去交结他。有一天,陆仲文正在蔚南村大餐馆内请客,却只有主客二人。宋子英串同了细崽,叫他进去和陆仲文商量,说是客人拥挤,没有房间,有一个单身客人要和他拼个座儿。

陆仲文是个公子出身,那肯答应,不想话犹未了,宋子英早已走了进来,对着陆仲文就是深深一揖道:“实在对不起尊驾,暂时拼个座儿。”陆仲文见他人品不俗,衣服风华,又是这样的谦恭客气,一时倒翻不转面来,只得说道:“一样多是客人,拼个座儿何妨,这间客座又不是我包下来的,何必这般客气?”宋子英见他答应,心中大喜,趁势坐了下来。有分教:

看破樗蒲之战,五木无灵;怒挥子路之拳,流氓丧胆。

欲知陆仲文怎样上他们的圈套,请看下,回便知分晓。





第六十回 吃大菜贵绅中计 游虎丘画舫嬉春





且说宋子英见陆仲文答应和他拼座,欢喜非常,搭讪着就和陆仲文坐在一起,彼此问过了姓名。陆仲文心上虽然不甚舒服,却又没本事叫他出去,只得略略应酬。谁知不去理他还好,这一理他可就惹出事情来了。宋子英放出和身本事,十分巴结,满口恭维,把一个公子脾气的陆仲文应酬得甚是欢喜,渐渐的和宋子英知己起来。及至一顿番菜吃完,宋子英进门的时候预先把钱放在柜上,抢着和陆仲文一齐付了。陆仲文那里肯叫他破钞,自己拿出钱来交给侍者。无奈这个细崽早已受了宋子英的贿赂,死也不肯接他的钱。陆仲文无可奈何,只得罢了,脸上倒有些讪讪的样儿,向宋子英道:“怎么今天竟扰了你的,可不是笑话么?”宋子英连忙说道:“陆仲翁说那里的话,你们二位是请也请不到的,难得今天赏我的脸,作个小东,只要你仲翁不嫌简慢,我就承了你的情了。”说着哈哈的笑起来。陆仲文听他这般说法,倒不好再说什么,只得谢了一声,一同出去。宋子英又再三拉着他们二人,到王黛玉家去打茶围,陆仲文本是个无可不可的人,就答应了。只有陆仲文请的那个客人再三不肯同去,就先告辞进城去了。这里宋子英见他走了,乐得少一个人,免得他在旁碍眼,便同了陆仲文到王黛玉家来。又竭力的恭维了陆仲文一顿,那胁肩谄笑的样儿,一时那里形容得出。

自此一连几天,宋子英都和陆仲文顽在一起,又请陆仲文吃了几台花酒,陆仲文少不得也要回请他。那消半个月的工夫,早把陆仲文骗得死心塌地,意服心输,觉得世界之内,朋友之中,只有一个宋子英是大大的好人,是知己的朋友,除了宋子英一个,再没有什么别人赶得上他们两个的交情。宋子英看着陆仲文的这般坚信,差不多已经水到渠成,若要动起手来,是拿得住千稳万当的了。

正要下手这个当儿,奇巧不巧,恰恰章秋谷同着贡春树也到苏州。陆仲文应酬秋谷,不免也耽误了两天工夫,却被王云生的党羽打听着了,便邀了宋子英一同商议,要想报上海的冤仇。大家斟酌了一回,斟酌不出个道理。他们晓得章秋谷世代簪缨,出身贵介,苏州地面自然总有相识的亲朋,要和他打起官司来,是万万打他不过的。这个念头也不用去转他。只有聚起一班光棍邀他个狭路相逢,或是把他羞辱一场,打他一顿,也算报了这个冤仇。等到他明天送官究治,一则并无证据,二则不识姓名,料想他一定无从查访。但是又有一件难处:章秋谷自幼投师习武,技勇过人,等闲十个八个人儿近身不得。何况苏州这班流氓都是风吹得倒的烟鬼,那里禁得起秋谷的尊拳,谁敢轻身尝试?所以王云生和宋子英想了几天,终是奈何他不得。后来还是宋子英出了一个主意,说:“陆仲文既是与他认得,我们何不想个法儿,把他们两个打在一起,狠狠的翻他一场,只叫姓章的大大的输掉一注银钱,我们也算报了仇了。”众人听了,都说宋子英的主意不差。

当下宋子英和一班同党的人细细商议了一番,把诸事安排停妥,却故意写条子去请陆仲文吃酒,叫仲文代请几位客人。果然章秋谷被陆仲文拉着同来。他又拿出好巴结陆仲文的工夫来巴结秋谷,果然章秋谷着了他的圈套,把他当作好人。又假说个姓邹的亲戚要买房子,托仲文、秋谷二人代他留心寻觅。章秋谷并不疑心,和贡春树说了,同进城去看过房屋,就问价银。宋子英却故意一口允许,又说只要等姓萧的帐房一到就好先付定钱。这个道儿,凭你是个神仙化身的人也是参他不透,免不得要着了道儿,何况是一个目空一世的章秋谷,一个纨袴出身的陆仲文。为什么呢?你想大凡世上的骗局,总是骗着别人拿出钱来,那有做骗子的人倒反拿出钱来,买所住房之理?况且房屋这件东西是生根的产业,和那金珠宝贝不同,不是可以骗了人家的房子就好逃走的,有这几层道理,所以就是章秋谷那般利害,这样机灵,一时也被他们糊涂住了,想不出他们的鬼计来。

如今闲话休提,书归正传。且说宋子英见章秋谷已经上当,把他当作个老实商人,却绝口不提起“赌钱”两字。到了付定钱的时候,故意的把萧静园一挤,不知不觉的把萧静园赌输的一桩公案挤了出来,却慢慢的从萧静园设法借钱,再落到汪慕苏身上,好叫章秋谷在旁看着绝不疑心。这样的调度安排,真算得是韩信奇兵,陈平妙计,果然一毫马脚也没有露出来。不料,章秋谷是个绝顶聪明的人物,虽然一时瞒过了他,那里防备得许多破绽?听他们说到“赌钱”两字,不觉起了一番疑心。又为他们要翻姓汪的钱,与自己并无干涉,又不要自己出钱,倚仗着自家胆大才高,不把这些人放在心上。要看看他们如何的举动,怎样行为,也好自己长些见识,便只当没有这件事的一般。

过了一夜,果然宋子英雇了小陈家的灯船,把章秋谷、陆仲文一同请到,只有方小松有事不来。宋子英隔夜已经和陆仲文说得明明白白,要他帮帮萧静园的忙,赢了汪慕苏的钱,三七开拆。陆仲文本来是个爱赌的人,又听得许他进款,自然乐得答应。

秋谷到得船上时,陆仲文已经来了,只有汪慕苏还没有来。宋子英又问秋谷可曾备些资本,“等少停入局之时,大家动手一齐重打,只要看着我的指头暗号,自然不差。汪慕苏既有这脾气,一定要把你们打的吃到别门,输出他的火来,定要记记重打。静园前天输掉的二千银子,不怕不在他身上回来,但总要你们二位帮他的忙才好。”陆仲文听了自然是一口答应。章秋谷却微微的笑道:“我虽然带了些本钱,却是旅资不够,所以带得少些。但是我兄弟向来没有做过这样的事情,为的是萧静翁输得多了,又是你宋子翁的意思,不得不勉强应酬,凑你们大家的兴,只是资本不多,恐怕赔不上你们的豪兴。”宋子英听了,就觉呆了一呆。陆仲文接着道:“你这个人真是多虑,本钱不够怕什么,放着我们这几个朋友,难道不和你想法不成?”章秋谷尚未开口,宋子英又道:“陆仲翁说的话儿一些不错,我们本来单是算计那汪慕苏,要想赢他的钱补静园的亏空。至于我们这几个人,暗中都是一起,大家可以通融,章秋翁不消多虑。况且我们这个法儿,原不用什么本钱,赢了下来,大家都有些儿好处,我晓得你们二位是不在乎此的,只算得个彩头罢了。”陆仲文听了,连连称是。

章秋谷此时已经起了疑心,差不多心上已有三分明白,面上却假作不知,依旧微微冷笑道:“宋子翁的说话自然不差,但我兄弟从来不要这样的钱。这三七对分的话再也休提。我不过看着你们二位的面情,今天和你装些幌子。若一定提起分拆的一层说话来,我却立刻就要告辞,不敢领教了。”宋子英和萧静园听得章秋谷的说话来得锋芒,晓得事体有些不妙,那面上顿时就变了颜色,发起楞来。章秋谷冷眼看他们的神气,心中已猜着了五分,却又恐怕被他们看出,倒回过脸去,故意寻些闲话和陆仲文随口攀谈。

宋子英停了一刻方才回过面色来,立起来便向秋谷打了一躬,道:“既是如此,我也不敢勉强,但是承秋翁这般关切,义气过人,我和静园只好放在心上,随后补报的了。”萧静园在旁听着,也跟着宋子英打了一拱。章秋谷连忙还礼,不免又谦让了几句。陆仲文见了却大不为然,口中咕噜者道:“你这个人的脾气实在希奇,放着教你赢钱,你却自家不要,天下那有这般痴子!要晓得如今世上,凭这良心天理是万万行不去的。只好把你这个良心暂时收拾起来,或者将来还有得法的日子。”秋谷听了只是微笑,也不回言。

陆仲文正在说着,汪慕苏已经来了,坐了一乘簇新的蓝呢中轿,跟了两个年轻的俊俏跟班。轿子停在岸边,汪慕苏走出轿来,这里的船家早已搭好扶手,扶着汪慕苏慢慢的走上船头。宋子英和萧静园一齐迎到头舱,汪慕苏只朝着他们弯了一弯腰,就大摇大摆的走进中舱,那架子狠有些儿可厌。宋子英和萧静园跟在他的后边,进得中舱,秋谷和仲文,免不得立起招呼;汪慕苏却非常客气,他们本来认得,不免又要寒暄一番。宋子英便问汪慕苏船上可要带局,汪慕苏道:“大远的路去游虎丘,不带个把倌人,有何趣味?”萧静园听了,便问船家要了笔砚,写起局票来。先写了汪慕苏的如意堂陆韵仙,又写了自己的翠凤堂金宝珠,宋子英仍叫王黛玉,陆仲文和章秋谷不用说自然是王小宝和金媛媛了。

秋谷趁他们正写局票,便把陆仲文拉了一把,立起来望船头上走了出去。陆仲文会意,随后也跟出来,问他有什么话说。秋谷道:“今天看他们的样儿不对,恐怕事有蹊跷。你不要去上了他们的圈套,只要跟着我的眼风行事,包你不差。停回儿上起场来,你看我打得多,你也打得多些,我打得少,你也不要重打,总看着我就是了。”陆仲文听了那里肯信,况且他心上只把一个宋子英认作心腹之交,章秋谷那里说他得动。当下把眉头连皱几皱道:“你也太小心了,为什么要这样多疑?依我看来,宋子英的为人甚好,一定不肯做这样的事情。你不要这般疑惑,我和他出个保单何如?”章秋谷还待和他细说,禁不得宋子英叫萧静园到船头上来请秋谷内舱去坐,便把话头打断。秋谷和仲文一同进去。

坐了一回,各人的局陆续到了。宋子英便叫水手开船,水手们答应一声,抽起跳板把船拦开,点了一篙,那船便顺流而下。起先没有开船的时候,坐在舱中甚是燥热,开船之后,顿觉得清风徐起,水波不兴。秋谷等坐在舱内谈谈说说,甚觉开怀。不多时,那船已开到山塘左近,波平如镜,碧天无云,看着两边岸上的景致,不知不觉的立时间心地清凉。只见这一边画阁凌云,那一处垂杨拂面;这面是栏杆映水,那边是红袖凭栏,说不尽的许多景物。秋谷暗想,他们这一班俗不可耐的人,只晓得赌钱、吃酒,料想他们不懂这些,落得待我赏鉴赏鉴。正在倚着船窗留连凭眺,觉得背后一阵香风,一个人将秋谷肩背上拍了一下。秋谷急回头看时,原来就是金媛嫒立在自家背后,清胪照彩,巧笑流波,含笑向他说道:“耐一干仔来浪看啥?让倪也来看看虐!”秋谷便携着金媛媛的纤腕,一同倚在船窗向外观看。

恰好船已到了山塘,就在吉公祠前几株垂杨下边停泊,众人约了秋谷,并带了一班倌人,一齐步上岸来。鬓影撩人,和香扑面。到吉公祠内吃了一碗茶,徘徊一会,方才仍旧上船。

船家已在中舱摆起台面,果盘、小吃排得整整齐齐。宋子英便请众人入席。那些倌人都坐在客人身后,履舄交错,钗弁纵横。那小陈家的船菜是苏州有名气的,比起上海堂子里头的菜来真是高了几倍。有分教:

破机关于顷刻,杯酒戈矛;惊豪士之风神,黄衫挟弹。

欲知后事如何,且听下回分解。





第六十一回 倒脱靴两番骗局 破机关一怒挥拳





且说小陈家的船菜,是通省最精致的烹庖。端上菜来,十分精洁可口,众人极口称赞,秋谷倒饱餐了一顿。众人因饭后就要赌钱,都不吃酒,只略略的吃几杯酒,应个景儿,便请主人赐饭。一时间饭毕,船户递上手巾,收过台面,又泡上茶来,出??自去。

这里众人喝了几口茶,便要商量上局。先是汪慕苏头一个答应,嚷着还叫快些。

宋子英便把预备的一把围棋子、一只铜盘拿了出来,放在台上;又取了一只茶杯,再问船家要了一只象牙筷了。宋子英便让汪慕苏做庄。汪慕苏道:“我向来不做上家,你不必和我客气。”宋子英听了,又让秋谷、仲文二人上去做庄,两人一齐不肯。宋子英笑道:“既然你们大家不肯出手,只好待我自做庄家便了。”说着,便坦然高坐,把棋子抓在手中,看他在袖内做了一回,就把棋子放在盘中,用茶碗向上头一盖。仲文却呆了一呆道:“这个顽意,不要亮宝的么?”宋子英道:“亮宝是骰子摇摊,要看他的宝路,才要先亮三摊。这个抓摊却没有什么宝路,凭着庄家的高兴随便做去,一些没有毛病,所以不用亮摊。”陆仲文听了方才明白,当下大家动手。秋谷又附着耳朵悄悄嘱付仲文,叫他不要重打。这个时候,见宋子英两个指头拈了筷子放在碗底上面,秋谷就取出一张十元钞票打在二门上。陆仲文因是第一摊,也只打了十元。萧静园只打五块钱的一张钞票,只有汪慕苏打了五十块钱青龙,又把萧静园打的也吃到青龙上去。

看官且住,章秋谷既然心上有些疑惑,为什么还肯跟着他们一起赌钱,岂不是在下做书的人自相矛盾么?看官要晓得,章秋谷的心中虽有几分疑惑,却究竟揣摸不定他们的情形,也不过是个悬想之词罢了。况且他自恃才高胆大,一定不至吃亏,所以把自己的疑惑放在心中,面子上和他们混在一堆,究竟要看看他们怎样。这是章秋谷一生好奇冒险的性情,如今不在话下。

宋子英开出宝来一数,齐齐整整的十个棋子,恰恰是个白虎,应配秋谷和仲文的六十元,吃了青龙上慕苏的五十五元,宋子英照数配出。汪慕苏除了自己输的五十元之外,还要赔还萧静园的注目,连本二十元,输得汪慕苏有些发火起来。宋子英又做了一宝,那拿筷子的时候是用一个指头,这回汪慕苏压得大了,身边取出一张一千两的银票来,再扑一记青龙,就在银票上打了三百,又把章秋谷、陆仲文打在进门上的每人五十元一齐吃到青龙上去。开出来准是个进门,气得他目瞪口呆,只得向秋谷、仲文道:“我今天带的都是一千两的票子,我共该赔还你二位四百块钱,可好少停一刻再算?”秋谷听了并不开口,陆仲文却十分信他,连说:“不妨不妨,这几百块的事情,难道我们不相信你么?”汪慕苏道:“虽然如此,也要你们相信才好。”

说着,宋子英又做了一摊,汪慕苏仍旧扑了一记青龙,原在银票上打了四百,向秋谷说道:“你们两位为什么不多打些儿,就是赢了也好算些。”秋谷因接连赢了两摊,胆就放大了几分,因看宋子英做的暗号仍旧是个进门,便在进门上打了二百。陆仲文跟上去也打四百,萧静园也打了五十块钱,汪慕苏看他们已经摆好,伸过手来,把他们摆的注目一注一注的都吃到青龙上去。秋谷暗暗心中好笑,想:“这个人真真是个赌痴。”及至开出宝来,宋子英把一只筷子分开数目,那知竟是二十粒棋子,端端正正的是个青龙。宋子英假作大惊失色,面上现出一付懊恼的神情来。陆仲文见了也觉有些诧异,章秋谷看了这般光景,陡的把一桩事儿提上心来,暗想方才好好的赢了两摊,怎么又忽然变局?顿时把那先前的几分疑虑直变到二十四分,不觉豁然大悟,果然是他们弄的玄虚,做那倒脱靴的勾当。正在心中委决不下,却见宋子英皱着眉头,也取出一张票子赔了汪慕苏,回头向秋谷和仲文使了一个眼色,假作解手,走出舱去。秋谷只当作没有看见一般,坐着兀然不动,只有陆仲文跟了出来。

到得船头,宋子英不等陆仲文开口,先自家说道:“我真是糊糊涂涂的鬼摸了头,不知怎么少数了一个棋子,把好好的进门变作青龙,连我自己也有些不信。如今也不必说了,总是我自家不好,带累你们赔钱,只好我用心些儿再做几摊,你们重重的加倍打上几记,让他吃了过去,加倍输钱。好在他是个有钱的人,输掉几千银子也不要紧,你想是么?”陆仲文听了深以为然,正待开口,却听得汪慕苏在里头嚷起来,叫着子英道:“怎么你解个手儿要这许多时候,可是你才输了一摊,就把你的胆子吓破了么?”宋子英听了,慌忙进去。陆仲文也随后进来。宋子英向汪慕苏道:“你说的什么话儿,可是瞧我不起么?老实说输这几个钱还不放在心上。

你通共才赢了一摊,就要这般性急,不要停回输得多了,朝我讨起饶来。“

两人一面斗口,宋子英又做了一摊,却伸了三个指头。陆仲文趁着宋子英和汪慕苏说话,附着秋谷的耳朵,将宋子英的话向秋谷说了一遍,又叫他这一下务必重打些儿,秋谷微笑不答。这一回汪慕苏打得更大,除了把自己的银票收回之外,就在宋子英的银票上打了六百。再扑一记青龙,又把一张赢的五百块一张的银票还了秋谷和陆仲文二人。秋谷到了这个时候已是十分明白,待要发作出来,又想且慢,我就依着他的说话再打一记出门,看那汪慕苏怎样。想着就把方才还来的银票一齐放在出门上边。陆仲文更在出门上打了一千,秋谷眼睁睁的看着汪慕苏,只见他果然又把出门上的注目,一齐吃了过来,放在自家一起。宋子英见已经打定,满心欢喜,心上想着,凭你姓章的这般利害,不由的也着了我的道儿,等到你心上边明白过来,已经输了千把银子,总算我和王云生报了上海的冤仇,一面想着,正要伸手揭去茶杯。就这个闪电穿针的时候,猛然章秋谷立起身来,长眉倒竖,凤目圆睁。

何郎粉面,现出两朵红云;沉令丰姿,变作一团杀气。从宋子英肩上伸过一只手来,把桌上的茶杯按住,喝一声:“且慢!”这一声不打紧,在别人听见原也不算什么,无奈宋子英等三个都是贼人胆虚,听他一声呼喝,看他满面怒容,就好像青天起个霹雳一般,彼此相看,一个个大惊失色。宋子英只得勉强问道:“章秋翁这是为何?”

陆仲文也觉不解,向秋谷道:“为什么这个样儿,可不是疯了么?”章秋谷冷笑一声,且不说破,只对着他们高声说道:“我晓得这摊棋子一定是个青龙,待我揭了茶杯大家观看,若是我说得错了,你们台上的注目,我情愿一概通赔。”宋子英听了,知道章秋谷已经识破机关,真是疾雷不及掩耳,只急得目定口呆,汗流体战。

待要和他硬挺几句,又晓得章秋谷武艺精通,不是好惹的人物,况且王云生吃过他的亏苦,被他轻轻的随手一掌,就跌了一个鹞子翻身。俗语说的:“光棍不吃眼前亏。”若要和他硬挺,挺发他的火性,动起手来,那一个是他的对手?可不是白白的吃了他一顿拳头,却上那里去喊冤枉?所以宋子英和萧静园面面相觑,不敢开口,只勉强挣出几句道:“章秋翁为甚这般生气?我们彼此客客气气的从不敢得罪秋翁,有什么开罪的地方,还请秋翁明讲。”说着又央告陆仲文,叫他劝解。陆仲文糊里糊涂的摸不着头脑,果然上去劝他道:“我们都是要好弟兄,何必这般动火?他们又没有得罪着你,为什么要做这种样儿,快些放了手,有话好说。”陆仲文的话还未说完,早被章秋谷迎面狠狠的呸了一口,大声说道:“你这个糊涂虫,自家上了别人的当,一些儿不懂,还来替他们劝和!我也没有多大的工夫和你细说,只把这一摊亮给你们看看到底可是青龙,就晓得我的说话不差了。”说罢,正要翻转茶杯叫他细看,宋子英等此刻真是万分着急,无计可施。汪慕苏只得硬挺几句道:“我们几个人在一起顽耍,本来只算是个书房局,算不得什么赌钱,就是有些输

赢也是常事。章秋翁也犯不着做出这个样儿。”秋谷听了更加大怒,厉声喝道:“好个无耻的棍徒,还敢多嘴!今天不打你,你也不认得我姓章的是何等样人!”就着就把左手向他胁下一叉,早把个汪慕苏叉得踉踉跄跄直跌出去。幸亏有船窗挡着,不然,几乎跌入河中。章秋谷把汪慕苏叉了一交,不由分说,就把茶杯一翻了转来,也用一根筷子,细细的拨着,叫陆仲文在旁细看,数来数去,只有十六个棋子,不是青龙是个什么?陆仲文直到此际方才明白过来。章秋谷早把注目收回,哈哈大笑道:“你可明白了么?”陆仲文连连点头。当下宋子英见事情败露,急得满面通红,心头乱跳,口中却还在那里支支吾吾的不知说些什么,秋谷也不去理他。

汪慕苏吃了一交筋,自家扒了起来,口内却还不服道:“反了反了,到底为了什么事情这样的穷凶极恶,难道如今世上没有王法的么?”秋谷冷笑一声,正要回答,忽回头见金媛媛立在自家身边,吓得花容惨淡,泪眼惺忪,那几个叫来的局都摸不着头脑,一个个急得愁蛾双锁,珠泪欲流。汪慕苏叫的陆韵仙,见汪慕苏跌了一交,恐怕连累到自家身上,更吓得面无人色,几乎要哭出来。秋谷见了这般光景,忍不住有些可怜他们的意思,便向金媛媛说道:“这事与你们无干,不必这般害怕,你同着他们到房舱去坐一回儿,免得在此碍手碍脚。”金媛媛巴不的这一声,连忙同着王小宝等一齐躲入后舱。这里秋谷向汪慕苏道:“你们这一班赌棍,平时做着那翻天印、倒脱靴的勾当,也不知被你们害了多少好人。今天在我面前还要装着糊涂,自家掩饰。你们未曾举意,也该打听打听我章秋谷可是受骗的人!上海的那一班赌棍何等的神通,尚且不敢在我跟前弄什么手脚,不要说你们这起无用的东西!”

这几句话儿,把他们骂得十分惭愧,只有汪慕苏勉强回道:“就算我们是个赌棍,可有什么凭据被你拿住?这样无凭无据的事情,都好随口乱说的么?”秋谷又冷笑道:“你说你的赌棍没有凭据么?哼哼,我若要认真追究起来,只怕你们翻戏的罪名还在其次,那私刻钱庄图记、私造庄票的罪名,你们那里担承得起?我劝你不如听了我的说话当场认错,赔个礼儿,好在我们没有输钱,那有功夫来同你们作对,岂不还是你的便宜?若要一口咬定,不肯服输,那就莫怪了。”说着,手中拿出一张银票,朝他们扬了一扬道:“真赃现在,你们还能抵赖得过么?”原来方才秋谷收回注目之时,一并把汪慕苏打的一张银票取在手中,明晓得他们的银票都是假的,只有汪慕苏刚刚赔还秋谷、仲文的一张五百块钱的银票却是真的,不过把来摆个样儿。正是:

人情变幻,蜃楼海市之奇;世界沧桑,石火电光之影。

欲知后事,请看下回。





第六十二回 讨局帐当场出丑 托微波名士多情





且说章秋谷拿着一张银票向他们扬了一扬,宋子英看了更加着急,又听得秋谷朗然说道:“论起理来,你们做了圈套到处害人,本该把你们送官究治;但是你们都是穷苦出身,总算出于无奈,我也不来和你们做这个冤家。不过我替你们想起来,你们年纪正轻,人品也还漂亮,不是那巴结不出的人;那一样事儿不好去做,却要做这样倒脱靴、翻天印的事情?将来总有一天被人看破,送到当官,从此犯了案情,没有出头之日,何苦把父母的遗体这般糟蹋?难道你们除了这行生意,就没有别的事情好做么?”秋谷说到此际,声音倒反和平了些。宋子英等听了秋谷这几句心平气和的说话,不由得也有些良心发现起来,又听得章秋谷好好的向他们说道:“现在我也不来难为你们,只要你们把自己的来踪去迹,以及受了何人的指使,一一说得分明,从此洗心革面,大家痛改前非,切不可再做这行生意,我便把你们当场

释放,免了你们这天字第一号的官司。若再是这般不肯认差,那时莫怪我送官究治。到了公堂之上,凭你人心似铁,当不起官法如炉。到了那个时候,依然还要供招,可有什么趣味?“

宋子英听了,晓得秋谷的话虽然霸道些儿,却是实在不错,待要直说出来,又实在觉得面上惭愧,回过头来看萧、汪二人时,也是面上一红一白,那个样子甚是为难。宋子英明知今天这个局面是抵赖不来的了,左思右想,跑是跑不了,打又打他不过,只得要从实供招,红着脸支吾半晌,说出一句话来道:“这件事儿,与我们这在座的三人全然无涉。”说到这里,又半吞半吐的不肯直说出来。偏偏的这个当儿,宋子英的舌头也不听他的呼唤起来,期期艾艾的说了一句,倒缩住了半句。

章秋谷不懂得他的说话,焦躁起来,便向陆仲文说道:“他们既是不肯说明,我们也说不得了。我在这边船上守着他们,你赶紧上岸,到阊门去拜总巡,叫他派几个人来,把他们带去看押,再移县问他们的案情。好在这个事情是一面的官司,就是无人送办,也是他们巡察的责成。一定没有不准的。”陆仲文起初不知底细,真把宋子英当作好人,此刻被秋谷当场说破机关,他心上方才明白,由不得就恨起这班人来。听了秋谷的说话,答应一声,当真便要上去。

宋子英急了,心想也顾不得许多,只好直说出来,作个脱身之计罢了,便一一的向着秋谷、仲文从头细说:如何想了主意,本来只想去哄骗仲文;如何章秋谷到了苏州,被王云生的手下看见,他为了上海的事情结下了仇恨,要想个法子报仇;如何自己串同了萧、汪二人,要想把秋谷和仲文一齐打下水去,从头至尾一字不遗,细细的说了一遍。章秋谷恍然声道:“原来又是王云生这个奴才。”陆仲文不晓得这件事情,急问:“王云生是谁,和你有何嫌隙?”章秋谷约略把夏间的事情说了几句。陆仲文方才明白,却咬牙恨道:“原来他们是来算计我的,我还把他们当着好人。不亏你提醒了我,几乎上了他们的大当。”秋谷道:“如今也不必说了,他们既然认罪服输,我们又没有输什么钱,让他们走了罢。”宋子英等三人听了,好像逢了郊天大赦一般,免不得谢了秋谷一声,穿好了衣裳就要上岸。秋谷又叫住他们道:“你把方才赢我们的钞票,仍旧彼此掉换,我也把银票还你。”说着,便把一真一假两张银票取了出来给还了他们,仍把自己钞票收回。

正在掉换,忽见房舱内走出三个倌人。原来就是宋子英等叫来的局王黛玉、陆韵仙和金宝珠。他们一班倌人坐在后舱,把前舱的说话听得明明白白,晓得宋子英等三人是个倒脱靴的赌棍,王黛玉等就吃了一惊,想着自家的局帐恐怕有些不妥,又听得秋谷要释放他们上岸,更加着急,一齐拥了出来,每人拉住一个不放。只听得王黛玉先开口道:“宋大少,倪一径当耐是个好客人,局帐洋钱拨耐欠仔几化,故歇勿壳张耐是实梗样式。唔笃赌铜钱勿赌铜钱,生来勿关倪事,倪也勿好来管唔笃格事体;不过倪搭格局帐洋钱,阿好请耐开销脱仔,省得倪叫人到栈房里来哉。”

陆韵仙和金宝珠也是一般说法。宋子英满面通红,勉强说道:“现在又不是年,又不是节,收什么局帐!况且我又不是不来,停回到你院中再说。”王黛玉冷笑道:“宋大少,勿是倪来里说望门闲话。倪堂子里向名气要紧,耐宋大少阿好去照应仔别人罢,倪格局帐洋钱末,请耐开销脱仔,勿要晏歇点弄得大家难为情。”宋子英被他逼住了,开不出口来,待要发作,又怕章秋谷要帮着他们,待要赌气照数给他,又舍不得这许多的钱。正在迟疑不决,果然秋谷开口问王黛玉道:“他一共欠你多少局帐?”王黛玉急应道:“说起来是也无啥希奇,一塌刮仔勿到一百洋钱格事体。”

秋谷听了道:“这也不多。”又问金宝珠和陆韵仙时,每处不到五十块钱,合来也只有二百块钱上下,秋谷便向宋子英道:“一共二百块不到,你们料想也还拿得出来。他们堂子里头吃亏不起,你拿一百六十块钱出来,待我和你们分派。”宋子英听了虽然心痛,却是不敢不依。只得凑了一卷钞票出来,交与秋谷。秋谷接过,点了一点,分作三注,向王黛玉道:“你的局帐拿了八十块钱,他们两个合分八十,所差已是不多,也不必计较了。”王黛玉接了钞票,甚是感激,一同谢了又谢,方才放了宋子英等三人,回身坐下。宋子英满面羞惭,满心懊恨,同着萧静园、汪慕苏两个抱头鼠窜的上岸去了。这里船上的章秋谷同陆仲文叫船户把船回到阊门,分头登岸。

章秋谷倒贴了一天的船钱,又在苏州等了两天,贡春树已经来了。秋谷因他来得迟了,不免埋怨他一番,立刻收拾行李,发上上海轮船。章秋谷又到金媛媛处把局帐开销清楚,辞别了陆仲文和方小松。金媛媛却一直送到船上,嘱付了无数的话儿,无非是要叫他就来的意思,直至将要开船,小火轮的气筒鸣鸣的连放了几遍,方才上去。正是:

未免有情,芳草天涯之路;谁能遣此,销魂南浦之歌。

只说章秋谷和贡春树上了轮船,在船上没有什么消遣,把宋子英这件倒脱靴的公案细细的讲给贡春树听。春树抚掌称快,又道:“我正在疑惑,怎么不见宋子英,因为你匆匆促促的上船,没有工夫问你。原来我走了不多几日,出了这样的一件事情。但是那王云生吃了你两场亏苦,冤家结得更深了一层,以后倒要防备他些才是。”

秋谷道:“这样酒囊饭桶的奴才,难道我章秋谷怕了他么?”春树道:“不是这般说法,蜂虿有毒,那里防得尽许多?总是小心的为妙。”秋谷方点头称是。

过了一夜,不到七点钟,轮船已到码头。秋谷起身上岸,便拉了贡春树同住吉升栈,春树自然应允。秋谷到得栈房,当差的接着,开了房间。秋谷进房坐下,恰好对面有个客人走了,空了一间禄字官房,秋谷便叫茶房把春树的行李搬到对房安放。坐不多一会,当差的送上一搭名片来,还有几封别处寄来的信,秋谷一一看过。

当差又送上几张倌人名片,秋谷看时,见也有陈文仙的,也有张书玉、陆畹香的,惟有林黛玉的名片最多,竟有七八张光景。秋谷诧异起来,问当差的:“怎么林黛玉的名片有这许多?”当差的回道:“这林黛玉自己来过两次,又天天叫人到栈内来打听少爷几时回来,说有要紧的事情要和少爷商议。再三再四的分付家人,少爷一到上海,立刻要请少爷过去。也不晓得有什么事情。”秋谷听了甚是疑惑。暗想:“黛玉有什么要紧的事情要和我商议?少停且去看他一趟就晓得了。”

章秋谷离了上海已有十几天,少不得要出去拜拜客人,会会朋友,料理些未了的事情。又到辛修甫、王小屏等各处去了一转,倒整整的忙了一天。辛修甫见秋谷回来,心中大喜,急急的问他办的事情怎样?秋谷也不隐瞒,细细的向修甫说了一遍,修甫不胜叹服。当夜修甫请他在一品香晚膳,又请了小屏、春树作陪,宾主只有四人,小屏问修甫可要叫局,修甫笑道:“今天他们两位初到上海,自然要把他们的相好叫来,一则好让我们看看他们的情景,二则他们一日三秋,也好叙叙这十来天的阔别。”这几句,说得三人都笑了。当下修甫写了局票,叫侍者发了出去,不消说各人是叫的老相好了。发了局票,各人又点了一张菜单,交与侍者,随意闲谈。

秋谷正把苏州的事在那里告诉王小屏,不想第一道菜还没有上来,叫的局陈文仙已经来了,扶在娘姨的肩上款步进来。先向修甫等打了一个招呼,慢慢的回身坐下,方才似嗔似喜、含怨含颦的叫了一声“二少”,随接下去说道:“耐倒好格,阿记得动身格辰光搭倪说一礼拜就转来,故歇耐算算看去仔几日,只怕三格礼拜要来快哉,倪末倒牵记煞耐。”秋谷听了,且不回答,抬起头来细细的打量他。见他穿一身白纱衫裤,头上只带着一排茉莉花条,趁着那杨柳纤腰,梨花白面,趁显得柔情似水,媚态如春。那头上的花香夹着些脂香粉气,一阵阵的透人鼻观中间,秋谷看得十分畅满。

看了一回,方向陈文仙道:“我到苏州去原为一件要紧事情,前几天事情没有办好,所以不得回来,并不是有心耽搁。”陈文仙不肯相信,把嘴一披道:“倪勿相信,耐有啥格要紧事体,倒说拨倪听听看。”秋谷因刚才和小屏说话,还未讲完,被陈文仙进来打断,王小屏又急于要听,秋谷便从头至尾把搭救程小姐的事情,看破宋子英的骗局,又一一说了一遍。王小屏也甚是佩服,不免称颂了一番。陈文仙却听得呆呆的,想了一会,好似想什么心事一般,回身把秋谷一推道:“耐格人末……”说了这一句,顿然闭了口说不下去,面上早红起来。秋谷听了他说了半句便不说了,摸不着他是说什么话儿,连忙问道:“我便怎么样,为什么不说下去?”

陈文仙飞了秋谷一眼,默然主语,那两边颊上红得就如雨后桃花,娇妍可爱。秋谷见了愈加疑惑,再三追问,文仙只是说不出来。修甫等看着陈文仙的神情,不觉哈哈大笑。陈文仙被他们笑得愈加不好意思,只得又向章秋谷说道:“耐替别人家赶事体,倒起劲煞。”说了又顿住不言,索性低下头去。红上春风之面,笑晕梨涡;羞融却月之眉,春添媚妩。秋谷到此,方觉心中明白,就是辛修甫等也猜着了几分。

正待要大家追问,只见金小宝笑盈盈的走了进来,先叫了秋谷一声,不等坐下,就向贡春树笑道:“阿唷!我道仔耐勿来格哉,今朝啥格好风吹仔耐转来,耐倒直头有良心格。”春树笑道:“我本来早想回来,无奈有些事体不得脱身。”金小宝不待说完,便问:“耐勒浪苏州有啥格事体?”春树笑而不答,小宝再三追问,王小屏听得不耐烦,正待说时,贡春树急使个眼风,王小屏便顿口不说。金小宝咕噜道:“倪勿来,耐阿搭倪说?”春树笑道,附在小宝肩上悄悄的说了几句,金小宝方才罢了。章秋谷也和陈文仙咬着耳朵讲了半天,不知说些什么。辛修甫在旁看着,只是微笑,向王小屏道:“你看他们的形状要好非常。我们虽在花丛阅历多年,那里赶得上他的资格?”正是:

前度刘郎,重访天台之路;巫山神女,空为朝暮之云。

不知后事如何,且听下回分解。





第六十三回 会审官左袒黑心妇 金月兰不认薄情郎





且说章秋谷在一品香出来之后,少不得到陈文仙院内住了一宵,直睡到次日午间方才起身回栈。当差的上来回道:“昨天少爷出去之后,林黛玉那边又有娘姨过来打听,晓得少爷回来,说一定要请少爷过去。”秋谷听了并不言语,只点一点头,当差的便退了下去。秋谷略坐一回便到惠秀里来,刚刚走进弄堂,早见一个娘姨从弄内劈面走来。见了秋谷连忙一把拉住,叫一声:“二少,为啥昨日勿来?倪大小姐牵记得来。”秋谷看时,原来就是林黛玉用的娘姨,便跟着他举步进门,匆匆的走上楼去。那娘姨先就嚷道:“大小姐,二少来哉!”秋谷刚刚走上楼梯,早见林黛玉一身素服,满面春情,袅袅婷婷的从房内掀着门帘走了出来,一把搀着章秋谷的手,同进房中坐下。

黛玉就坐在秋谷身旁,笑盈盈的说道:“长远勿见哉,身体阿好?倪一径来浪牵记耐呀。”秋谷也含笑应酬了几句,黛玉又笑道:“耐是昨日仔转来格,转来仔为啥勿来?阿是先要去看看唔笃老相好,倪搭是想勿着来格哉?”说着,秋波斜睨,启齿嫣然。秋谷见了黛玉这般态度,如此风情,任是阅历再深些儿的人,也由不得心飞神荡。暗想林黛玉的一身功架着实不差,陈文仙的面貌虽然比他好些,那外面的应酬那里赶他得上?便不由也携着黛玉的手,低声笑道:“你难道不是我的老相好么?我除了你,还有什么相好?”黛玉回眸一笑,答道:“阿唷,二少爷勿要客气,倪陆里有格号福气,只要耐二少长到倪搭坐坐,赏赏倪格光好哉,只怕倪搭小地方请耐格二少爷勿着啘。”

秋谷倚在榻上温存调笑了一回,方问林黛玉:“有什么事情要和我商议?”黛玉道:“耐阿晓得大金月兰吃仔官司,拨包打听捉得去哉。新衙门问仔一堂,故歇移到县里,耐啥还朆晓得介?”秋谷听了失惊道:“我昨日刚在苏州回来,那里就会晓得?月兰的性情本来不好,几次三番在人家逃走出来,我早料到他一定要吃了官司才罢;现在果不其然,闹了乱子出来,我也没有工夫管他这般闲事,你也不必管他。”黛玉听了,把秋谷手臂上拧了一把道:“耐格人生得实梗狠心,倒直头看耐勿出。月兰格脾气勿好,待耐总算勿差,千日格坏处末,也有一日格好处,耐总算看倪面浪,搭俚想想法子,也是唔笃两家头相好仔一场。”

秋谷起先还不肯答应,禁不得林黛玉向来和金月兰甚是要好,再三央告秋谷和他想个法儿,又道:“月兰带信出来,叫倪托耐阿好搭俚想想法子。俚说上海地方无拨啥格熟客,只有章二少是格好人,总要托俚说句好话。谢谢耐格,耐总算看俚苦恼,去保仔俚出来罢。”秋谷听到这几句说话,不觉恻然。想起苏州初次相逢,自成心许,何等缠绵,春融蝴蝶之魂,帐暖鸳鸯之梦。不想到了上海,不满半年,金月兰又闹了这个乱子。想着黛玉的话倒也不差,心上便有几分活动;又被林黛玉撒娇撒痴的一定要他答应,秋谷乐得顺水推船的做一个现成人情,便答应了黛玉。

却又说道:“我虽然答应了你,却还没有晓得月兰犯了什么案情,怎么好替他说话?”

黛玉就把金月兰的事情向秋谷说了一番,秋谷方才晓得,便去寻着了辛修甫,托他出一封信到县里去和金月兰说情。如今且把章秋谷一边暂时按下,先把金月兰的情节细说一番。

只说大金月兰自从在潘吉卿家卷了许多金珠首饰逃走出来之后,到了上海,本来要到旧时姊妹的院中暂时借住。无奈他的那一班姊妹都晓得他本是黄中堂家的逃妾,现在又是从苏州逃走回来;你想这些堂子里的倌人个个怕事,那里担得起这般风火?所以一个个都支吾推托,不肯留他。月兰无奈,只得在四马路上一个栈房内暂时住下。不想潘吉卿因金月兰卷了金珠逃走,直把他气得一个发昏。潘吉卿向来吊膀子的工夫甚好,所以有些女人都肯倒贴银钱。潘吉卿历年积聚下来的私财,多是一班妇女倒贴他的,这一下子被金月兰卷了一个精光,丝毫不剩。潘吉卿一生精力仅仅乎博得这点金珠,如今被他卷得精光,潘吉卿如何不气?气到极处,发起恨来,连夜到轮船局单雇了一只小火轮,立时生火开船赶上前去,罚咒一定要寻到月兰和他算帐。那知小火轮虽然单放,却开船的时候已是十二点钟,依旧赶他不上。

潘吉卿到了上海,落了一家后马路的栈房,便托了许多朋友四处打听金月兰的消息。又叫了包探来,交给他一张月兰的照片并一张失物的清单,叫他用心杏访,寻到了从重酬谢。那包探接了照片和失单,自然明查暗访,格外当心。不到一礼拜,居然被他访缉出来,晓得金月兰住在鼎升栈内,立刻照会了潘吉卿,禀了捕头,带了几个探捕,径到四马路鼎升栈搜捉。

金月兰刚刚起来梳洗,正在簪花顾影,对镜梳头的时候,那里想得到有人捉他?

几个包探巡捕一拥而入,自然是瓮中捉鳖,手到擒来。连金月兰的行李衣箱,一齐都带到捕房里面。金月兰免不得在巡捕房内关了一夜,明天九点钟解到公堂,会审官问了几句,便传了原告上来,当堂对质。金月兰听得潘吉卿告他卷物私逃,并说他是自家的逃妾。金月兰虽然有些胆寒,到此也顾不得了,只得按定心神,细细的想了一会,方才供道:“俚耐格闲话,才是瞎说,大老爷勿要听俚。倪一径来浪天津做格生意,今年二月里向刚刚转到苏州,拨俚耐碰着仔一转,倒说看中仔倪哉,要包倪一节生意,叫倪同俚转去。勿壳张倪到仔俚耐屋里向住仔一节,洋钱末无拨,倒说勿肯放倪出来。倪也叫无说法,只好等俚出门格辰光,自家走仔出来,故歇俚耐顶倒说倪是俚格小老姆,还说倪拐仔俚格物事逃走。大老爷问俚,阿有逃走格凭据?阿有讨倪格婚书?格号冤枉,要求大老爷搭倪伸冤!”会审官听了金月兰的口供,觉得甚是有理,便又问了潘吉卿一回,果然没有婚书,也没有卷逃的凭据。会审官便有不直潘吉卿之意,无奈潘吉卿一口咬定金月兰卷逃是实,会审官道:“你既然没有婚书,这金月兰便算不得你的妻妾,怎么好说他卷物私逃?”

原来这位会审大老爷也是个风流人物,他见金月兰语言伶俐,丰态温存,不由的就存了一个开脱他的意思,所以语言之内有些偏护着他。潘吉卿见会审官不肯认真追究,便着了急,又上去回道:“不瞒老公祖说,他临走的时候委实卷了几千银子的东西,如今只求老公祖把他的赃物追了出来,别的事儿也就不必提起了。”当下会审官听了,只得正颜厉色的把金月兰叫到公案旁边认真追问。怎奈金月兰死也不肯认帐,只说:“实在没有卷他的什么东西。”会审官问了一回,又把金月兰的衣箱行李吊上公堂,一件一件的打开,当堂查看,却是些半旧不新的衣服,还有些香水、手巾、肥皂等妇女应用的东西,并没有潘吉卿失单上的物件。原来金月兰到得上海,把苏州卷出来的金珠,通通寄放在别人家内,预备潘吉卿万一访着了风声,只要没有真赃,便好和他白赖,你想金月兰的心思可利害不利害?

只说当时会审官见并无赃证,便冷笑了两声,直把一个潘吉卿急得满心火发,七孔烟生。但是自家身在公堂,又不敢当真怎样,只得忍住了怒气,再三求那会审官要他追究赃物。会审官听得不耐烦道:“这样没头没脏的事情,又没有证人,叫本府怎生追究?况且会审公堂的案件,一天最少也有十余宗,都像你这样歪缠,本府那有工夫和你管这般闲事?”说着不由分说,叫了廨差过来,分付把金月兰取保释放。潘吉卿听了更加着急,此时顾不得利害,高声嚷道:“老公祖,不要这样糊涂,这金月兰放是万万放不得的。他就是黄大军机府中的逃妾,苏杭上海都存过案的,老公祖难道忘了么?”会审官听说金月兰就是黄中堂府中的逃妾,倒不觉吃了一惊;又听潘吉卿说他糊涂,登时大怒,把公案一拍道:“你既然晓得他是黄相国府中的逃妾,为什么要把他留在家中,难道你是不知法律的么?”那潘吉卿方才原是气愤头上,一个不防备脱口说了出来,被会审官一句话儿提醒,心中懊悔起来。

暗想:我怎的这样糊涂,一时说了出来,我自己收留人家的逃妾,先有一个处分,这不是自寻苦吃么?又听得会审官向金月兰说道:“你既是黄府中的逃妾,我也不来问你,只把你们移到上海县去,听他发落就是了。”便叫廨差把金月兰押下堂去,又叫潘吉卿回寓候传,这且不表。

只说金月兰到了上海县中,暂时押在官媒家里,甚是心集,只得带信出来,叫林黛玉转请章秋谷替他设法。不料章秋谷又到苏州去了,好容易等得秋谷回来,被林黛玉缠绕不过,只得去和辛修甫商量,托他发信到县和金月兰说情。辛修甫本来和这位县大老爷的交情甚好,果然写了信去,不到十天,金月兰已经放了出来。你道金月兰的这一场风波为何消化得这般容易?原来金月兰在杭州逃走出来,这位黄大军机的长孙公子想着月兰虽然可恶,却又碍了自家相府的名声,不便追拿到案,只在上海、苏州两处存了一个县案,不准他到处为娼,原没有办他的意思。上海县接了新衙门的公事,只把他暂时收押,也没有问过一堂。凑巧辛修甫写信到县和他说情,上海县也乐得做个现成人情,立刻叫他取保。

金月兰出来之后,免不得到林黛玉家来见章秋谷。秋谷见他脂粉不施,花容瘦损,觉得他也甚可怜。金月兰见了秋谷,却是十分惭愧,满面通红,几乎要滚出珠泪来,勉强忍住了,默默无言。秋谷明知他的意思,只好反安慰他几句。从前的话一字不提,又恳恳切切的劝了他一潘。金月兰也甚是感激,在黛玉处住了两天,想上海立脚不住,又不愿再入风尘,只得摒挡行李仍到天津去了。到得天津,搭了宝华班的班子,生意甚好,居然车马盈门。这是后话,不必提起。

只先苦了黄伯润,后苦了一个潘吉卿。黄伯润花了八千银子把他娶到家中,真个是心坎温存,眼皮供养。徐娘姽婳,正当碧玉之年;夫婿温柔,况有潘郎之貌。

这也算得是一双两好,地久天长的了。谁知这金月兰得福嫌轻,自寻烦恼,不晓得他为了什么事儿,偏要想着方法一溜烟逃出来。可怜这位黄公子的爱情,那里一时就割舍得下?气得一个半死,醋得一个发昏,人财两空也还罢了,还落了一腔闷气发泄不来。遇着了那月夕花朝,免不了就要长吁短叹。这还不必说他,最苦是潘吉卿,他平日间千刁万恶,无所不为,专靠着倌人倒贴的银钱,供给他日用起居的挥霍。他晓得金月兰是在中堂府内逃走出来,一定有些积蓄,便把生平吊膀子的手段施展出来,要想金月兰的倒贴。不想他运气不好,失了眼睛,非但倒贴不曾想着,反把自己的十余年积蓄贴在里头,被他卷得精光,还不说一个“谢”字。正是:

当年渔父,难寻洞口之春;旧日萧郎,肠断天涯之路。

欲知后事如何,请听下回分解。





第六十四回 章秋谷有心试名妓 玉太史临老入花丛





且说潘吉卿妄想痴心,要想月兰倒贴,不料非但不能如愿,反被金月兰卷了几千银子的金珠首饰逃走出来。潘吉卿历年以来的积蓄都被他一卷而空,自家想想,花了无数的精神,拚着自家的性命,去巴结那班妇女,方才得来的这点东西,一齐卷得干干净净,尺寸不留。看官,你道这潘吉卿如何不急?看着金月兰这般样子,你想这班倌人何等狠心,那般辣手,那里还有什么天良。所以堂子里的倌人万万娶他不得,这些说话都是在下阅历有得之言,并不是信口开河,有心捏造。

闲话休提,书归正传。只说章秋谷自从到了上海,便有辛修甫等一班好友,一个个轮流接风。不知不觉的过了几天,看看节近中秋,金风送爽。秋谷这一节的局帐,止有王佩兰和陈文仙两处多些,其余都不过七八个局,或是一两台酒,为数不多。王佩兰家自从为了金水烟筒跳槽之后,一直没有去过,算来也是有限。只有陈文仙院中有二十几台酒钱,八九十个局钱。秋谷约略算了一算,自家在常熟出来,带了几千银子,没有用掉多少,算起来尽够开销。局帐倒不过四百块钱的光景,倒是杨庆和银楼帐目有七百多些,就是拿了一支金水烟筒,也没有什么别的。秋谷一注一注的算了一回,大约连戏园、大菜馆、马车行这几处的零碎帐目并在一起,差不多也要一千五百块钱。便开了皮包,取出一张一千二百两银子的汇票,到后马路钱庄上去兑了许多钞票回来,先到杨庆和去把帐算清,便回到兆贵里来。

这几天将近中秋,大家收帐,堂子里头的生意狠是清淡。陈文仙恰好在家。秋谷进去坐了一回,忽然心上转了一个念头,暗想:上海的倌人只爱银钱,只要有了银钱,没有办不到的事体。就是倌人见了客人,装出多少假情假义,十分要好的样儿,这也是银钱买出来的,并不是倌人当真爱着这个客人。如今文仙的待我总算不差,但是他究竟心上如何,我却无从晓得,何不趁着开销局帐的时候,想个法儿试他一试?只说我盘缠用尽,家里的钱还没有寄出来,你们这些局帐只好一齐等到节后开销,现在却无从想法。看他听了我的说话,神色如何,那待我的心是真是假就看出来了。

想定主意,就向陈文仙招手,叫他过来,自家脸上故意装出一付气闷的样子。

陈文仙见秋谷招手叫他,慢慢的走过来,坐在秋谷肩下,问道:“啥格事体,说哩。”

秋谷假作皱着双眉,摇头不语,文仙连问了几声,见秋谷依然不答,发起急来道:“耐今朝啥格道理,跑得来阴阳怪气,一付勿高兴格面孔;问耐闲话末,一声勿响,阿是倪得罪仔耐哉?”秋谷听了,方才抬起头来,把文仙的纤手握在手中,叹一口气道:“你也没有什么得罪我的地方,这会儿我有我自家的心事。”文仙听了章秋谷的说话,抬起秋波,向他注视,果然见秋谷双眉深锁,一付不高兴的神情。陈文仙不知为了何事,反着实吃了一惊,连忙问他为甚事情,秋谷却默然不语,呕得陈文仙急了,赌气立起身来。秋谷方又叹口气道:“我的事情和你说也没用。”便又顿住了不说下去,急得陈文仙媚眼微睃,金莲双顿的埋怨他道:“有啥格事体,豪燥点说哩,耐看格付架形,阿要讨气。”

秋谷见陈文仙当真急了,暗暗好笑,方才附着他的耳朵悄悄的告诉他道:“我到了上海已经一节多些,家里带出来的钱差不多将要完了。前天我寄了一封家信回去带钱;还没有接到回信,恐怕节前是来不及的了。不瞒你说,我节边的开销帐目共要一千多些,勉强借贷了些,却还只有一半,还有堂子里头的局帐,也要差不多五百块钱,实在想不出个法儿,这个秋节如何过得下去?你想,现在已经逼近中秋,正是大家收帐的时候,那里一时想得出什么法儿?况且堂子里头的局帐,节边都要开销,更是有关场面,叫我心上怎生不急?”陈文仙听了方才明白,倒觉放下了心,“嗤”的一笑道:“倪当仔耐啥格事体实梗格发极,一塌刮子少仔几百洋钱,也用勿着实梗样式啘。倪搭格局帐是倒无啥希奇,耐有末开销仔点,无拨末也勿要紧。

秋谷听了心中暗暗欢喜,索性逼紧一步道:“你还没有晓得我的意思,你这里的局帐虽不要紧,但是这一班房间的娘姨、大姐,都是天字第一号的势利心肠,我若是到了节边局帐都付不出来,以后还有什么脸儿再到你院中行走?”说着,脸上做出那懊恼万分的样子,又向文仙道:“我今天来了一趟,明天还要出去寻人设法。

若是这几天之内借到了钱,还了你们局帐,我自然在你院中照旧往来;若是借不到钱,那就要直等家里的钱寄了出来,方能再到你院中走动。所以我今天特地到你这里来暗中和你说明,节前若是不来,你不必叫娘姨寻我。“说罢,又做出一付无奈的神情,对着陈文仙大有黯然销魂之意。陈文仙被秋谷这一番做作,不觉也把他的心事提了上来,蛾眉蹙黛,凤目含波,看着秋谷的样儿,也觉有些凄楚;便一把搀着秋谷的手,梨涡低熨,檀口斜偎,似笑不笑的对他说道:”耐慢慢交看嗫,勿要实梗性急,就是局帐勿开销末,也无啥希奇啘。“秋谷又附耳向他说道:”不是这般说法,这班带房间的相帮,掮带挡的娘姨、大姐,都不是什么好人,将来他们传说出去,还要说你做了恩客,所以连局帐都没有开销。你们做倌人的名气要紧,那里禁得起这样的一个名声,你想我这句话可是不是?“陈文仙听了,觉得章秋谷的说话果然不差,便也觉得为难起来。

想了一会,忽然想着了一个主意,便咬着秋谷的耳朵说了一回。秋谷连连摇头道:“这个办法不妥,况且我也不是这样的人。”文仙听了皱着双眉,又向秋谷耳边说了一会,秋谷还不肯答应。文仙不由分说,支开了房里的娘姨,取出首饰匣来,捡了两件不知什么东西,忙忙的仍把首饰匣儿收好,跑过来就塞在章秋谷衣袖管内。

秋谷看时,见是一只金刚钻戒指,一付蒜苗梗式的金镯头。暗想:陈文仙的为人果然不错,真是上海的平康队里数一数二的好人。

此时章秋谷的心上十分畅满,一时间心花大放,色舞眉飞,忍不住哈哈大笑起来。这一笑,笑得个陈文仙摸头不着,疑诧异常。外房间的娘姨人等听得秋谷放声大笑,不晓得他为的什么,一齐赶了进来,见文仙呆呆的立在秋谷旁边,也不开口,宝珠姐便问秋谷道:“二少为啥实梗高兴,阿好说拨倪大家听听。”秋谷听了,把一只戒指,一付金镯在袖中掏了出来,放在桌上,陈文仙看了着急起来,连连的咳嗽几声,似乎叫他不要说出来的意思。秋谷虽然听见,那里管他,对着宝珠姐等把方才的说话讲了一遍,只把骂他的话掩过不提。又说自己要试试文仙的心到底是真是假,所以掉了一个枪花,撒了一番大谎,“幸而你们先生果然是个好人,居然没有上当。要是换了一个势利些的倌人,说话中间得罪了我一句两句,哈哈,我姓章的今天就要对你们不起了。”宝珠姐等听了,倒大家呆了一回,宝珠姐方才开口笑道:“阿唷,看耐二少爷勿来,倒直头来得利害跺,区得倪先生待耐二少是轧轧实实格一片真心,勿然是今朝推扳一点露仔马脚出来哉。”

陈文仙到此方才恍然大悟。暗想:这个人的心思实在很刻毒,今天险些儿被他试了出来。不觉的桃花敛恨,柳叶含颦,佯嗔薄怒的对秋谷瞪了个白眼道:“耐倒好格,倪啥格地方得罪仔耐,洛里一样事体倪待耐勿是真心,耐倒说拨倪听听看!”

耐自从到仔倪搭直到故歇,说勿长久末也五六节哉,阿有啥两三年格老客人,勿晓得倪格脾气,想出格号挖掐心思来拨当倪上,阿要讨气?区得倪勿是格号坏人只认得铜钱勿认得人格脾气,勿然是拨耐说起来也好哉啘。耐自家想想看,两三年工夫倪阿曾待错歇耐,勿要说是故歇,总算有点……“文仙说到此间,说了半句就顿住了口,似乎有些说不出来,两颊微红,横波斜溜,向着秋谷掩口而笑,又在秋谷肩上打了一下道:”耐格人,就叫无拨良心。“说着又向宝珠姐等说道:”倪故歇想起来,上海滩浪格客人直头无拨良心!倪刚刚要是推扳仔俚一点,是只怕俚又要跳槽,跳到王佩兰搭去哉。“说得宝珠姐等大家笑了。

那秋谷此时满心欢喜,倒也说不出什么话来,只是细细的看着文仙微微含笑。

此时八月初旬,天气尚热,文仙穿着一身本色金阊纱衫裤,光艳照人。宝髻盘云,蛾眉掠月,真个是雪肤花貌,素口蛮腰。秋谷本来和文仙甚是要好,现在却凭空的添了几分爱情。文仙为了方才的事情,说是瞧不起他,不免还要咕噜几句。秋谷只得温存安慰了一番,文仙方才罢了。秋谷看着文仙十分清丽,十分快意,就十分的密爱幽欢。这一夜,秋谷自然不回栈房,就在文仙院中住下。正是:

鹊桥乍渡,蟾月刚圆;宝帐低垂,炉烟不动。春掩铜屏之影,鞋凤双翘;暗传膏沐之香,云鬟半卸。口脂微度,香融雀舌之酥;宝靥低偎,斜背春灯之影。嫣薰兰被,私语轻轻;冰簟银床,清宵细细。

真个是:

但为蝴蝶甘同梦,愿作鸳鸯不羡仙。

如今且把章秋谷一边暂时按下。提起一位前辈的太史公来。这位太史公姓王,号叫伯深,却是常熟人氏,同章秋谷总算是个同乡,还是他的父执。这位王太史本来是寒士出身,家中一无所有,直考到五十多岁才点了翰林。留馆之后,他想着在京城里头当这个穷翰林,也没有什么趣味,况且当翰林的就同那外省的候补人员一样,是要倒赔浇裹的。京城里米珠薪桂,他那里当得起这个翰林?想来想去,想着了一条道路,托了一个同乡的京官,把他荐到上海道幕中,差不多就算是这上海道的顾问官一样。那时维新的风气未开,八股还没有废掉,这位观察公也是个守旧家,同王太史谈论起来倒也意见相合,水乳交融,宾主之间甚是相得。那江海关道是关道中著名的好缺,所以王太史的束修每年竞有二千余金。玉太史喜出望出,索性把家眷搬到上海,住在一起。手内有了束修银子,登时就花天酒地阔绰起来。

原来这班专读死书、专做八股的书呆子,往往少年时节不敢荒唐,一到中年以后,中了进士,点了翰林,自以为是功成名就的了,免不得就要嫖赌起来。却是不嫖则已,一经涉足花丛,定是那天字号的曲辫子;不赌则已,一经走到赌场,便是那专输钱的冤大头。这位王太史少年寒素,没有中举人的时候,抱着一部直省闱墨,拚命揣摩;买了一部策府统宗,尽心摹仿。一天到晚只想着怎么好中进士,如何能点翰林,把那心地中间本来所有的一点平旦之气,早已磨灭得干干净净,那里还有工夫来想这样的事情!现在点了翰林,处着这般优馆,又住在上海这花营柳阵的地方,自然也要不安本分起来,天天在四马路堂子里头碰和吃酒,闹得一塌糊涂。却又实在是个外行,弄出许多笑话,他自己还扬扬得意的不以为奇。正是:

放着个玉堂学士,须发飘萧;辜负他金雀丫鬟,风情旖旎。

还有王太史许多笑话,九尾龟出现新闻,都在下回分解。





第六十五回 老风流艳福难销 美少年名花独占





且说前回书中说起王太史的现状,免不得先把王太史的许多笑话一一的演说出来。只说王太史有一天,在人家席间看见了一个公阳里的姑苏金寓,不知怎的就十分倾倒起来。这金寓原是上海滩上数一数二的人物,年纪已有二十五六的光景,虽然半老徐娘,却是尚饶丰致,更兼丰容盛翦,皓齿明眸,应酬甚是圆融,谈吐也还不俗。只是有一件,脾气不好,爱做恩客,爱姘戏子、马夫,正经花钱的客人,反把他高高的搁一在旁,正眼儿也不去看他一看,往往惹得客人发起火来,从此绝迹。他却毫不放在心上,依然还是我行我素,不改丝毫。一连嫁过几回,都是贵家公子,殷实商人。嫁过去到一年,便拚命的百般吵闹,寻死觅活,闹得个不可开交。人家被他闹不过了,赌气放他出来,凭他怎样,他却安安稳稳的重落风尘,琵琶再抱。你想这样的人,那里有什么良心?王太史却偏偏的看中了他。当夜席上转了一个局,翻台过来,就吃了一台酒,又连着碰了一场和,从此就天天在金寓院中走动,尽心竭力的报效起来。

王太史每年的修金虽然也有二千多银子,那里禁得起他这样的狂嫖,免不得要向别人借贷。只要金寓说出来的说话,他无有不依:金寓说一声要上天,他就立刻去搬梯子;金寓说一声要入地,他就立刻去挖深坑。总而言之,王太史待承金寓的这一番“鞠躬尽瘁,死而后已”的光景,若用在父母家庭之内,便是那孝感动天的孝子;用在君臣纲纪之间,便是那精忠贯日的忠臣。

王太史这样的待承金寓,只指望有些情愫到他,谁知金寓的心思却大是不然。看着王太史腰驼背曲,又留了一嘴的胡须,这样的神情还要去勉强学那风流解数,嘻皮笑脸动手动脚的,心中甚是厌烦。凭着王太史万种温存,十分巴结,从没有一些儿笑面待他,只是冷冰冰的面孔,待理不理的样子。王太史那里晓得,还是在他身上拚命的花钱。依着金寓的意思,那里肯叫他沾着自家的身体!却被房间里娘姨苦苦相劝,为的生意起见,没奈何勉勉强强的留了王太史两夜。

王太史受了金寓的特别利益,自以为荣幸非常。看见了不论何人,见一个告诉一个,只说是自己这般年纪,居然也有和他相好的倌人,这真是难得的奇缘,一生的知己。差不多就是西子太真之遇,瑶台月府之逢。别人听了他这般说法,都在背地里笑他,只不好当面说破罢了。王太史那里晓得,只是昏昏沉沉的,一天到晚除了办几件公事、拜几处客人之外,其余的工夫都是销磨在金寓那边。自从三月间做起,直到四月尽边,差不多也花了一千多两银子。在王太史的心上想来,可算得是和盘托出,竭尽绵薄的了。王太史的出身虽然本是宦家子弟,却是家世清贫,看得这一千多两银子的人情,真是天大地大,无大不大,出了一身臭汗,忍着满心难过,方才高高的捧了出来。这要是除了金寓,换了第二个人,未必想得动他这一千银子。怎奈在金寓眼中看了却是平淡无奇,看得他屁也不值一个。

有天晚上,这位王太史在金寓院中张筵请客,到了许多客人,金寓却只是冷冷的样子,酒也不斟,曲也不唱,只懒洋洋的在王太史背后坐了一回。恰好别处有人叫局,相帮传了局票进来,金寓趁此霍地立起身来,换了衣服,也不招呼台面,竟是一言不发的翩然而去。台面上客人看了金寓这般慢客,一个个都有些代抱不平。王太史却是一毫不觉,只当没有这件事儿,依然高兴非常,春风满面。众客人看见主人这般的大度优容,倒不好意思开口,只得罢了。及至金寓出局回来,仍是默然坐下,没有一句话儿。房间里的娘姨替客人装水烟,一个个一齐装到,独独的空过了王太史一个主人。

众人看见这样情形,十分不忍。就有一位姓陆的客人,叫做陆云峰,少年口快,慷爽非常。他见了王太史瘟得利害,再也忍耐不住,向着他冷笑一声道:“王伯翁真好耐性,真是十年养气,方有这样的忍耐工夫。要是换了我们这班少年,早已对他不起的了。”王太史听了,还不甚明白他的意思,连忙问他说的是那一路的话儿。陆云峰又冷笑道:“我们做客人的花钱吃酒,又不漂他的账,又不借他的光,为什么要受他们的这般闷气?”

王太史还未开口,金寓早瞅了陆云峰一眼,微微笑道:“陆大少勿要动气,倪是生来老老实实格脾气,无拨啥格多化瞎巴结,瞎应酬。刚刚碰着格王大人,搭倪一样格脾气,所以王大人到倪搭来,倪赛过当俚自家人,随随便便,总归是实梗样式,王大人也从来朆扳歇倪格差头格。像俚笃格排时髦倌人,嘴里向末说得蛮好,轧实肚皮里向一肚皮才是格枪花,格号样式倪是生来勿会格,只好请唔笃各位大少包涵点倪格哉。”陆云峰听了金寓的一番花言巧语,一时竟说不出什么来,只好冷笑一声,也就罢了。只有王太史听了金寓的话,说是把他当作自家人一般看待,这一喜,喜得非同小可,霎时间手舞足蹈,眼笑眉开,好似那甘露人心,醍糊灌顶,甜迷迷的,不知不觉直望耳朵里钻了进来。便对着众人说道:“你们大家都说我是个瘟生,那里晓得我们的要好!你想,倌人做了客人,把客人当作自己家里的人一般看待,要好到这个分儿,还要打算叫他怎样呢?”大家听了虽觉好笑,却又不好当面驳他,只好放在肚里,勉强附和几声。

金寓坐在王太史身后,听见他这般说法,忍不住把嘴披了一披,背过脸去扑嗤一笑。偏偏的又被陆云峰看见,实在气他不过,对着王太史笑道:“王伯翁的话儿果然不错,金寓和你十分要好,竟把你当作自己家人,这才算得你王大人的颜色。要是换了别人,那里有这般资格?恭喜恭喜,指日你一定要发大财。”这几句话说得刻毒,把一个王太史说得满面通红,又不好当真发作,只得说道:“我们一向客客气气的,这是什么顽笑,真是岂有此理!”众人听了陆云峰的说话已是要笑,再看了王太史面红耳赤的那种样儿愈加好笑,由不得一齐放声大笑起来。笑得个王太史认真又不是,不认真又不是,面上越发红了,坐在席上如坐针毡,好生难过。亏得有两个懂些事务的客人恐怕王太史恼羞变怒,连忙对着众人摇手,使一个眼风,止住了众人的笑声,又寻些别的话儿和王太史问答,方才把这一件事儿叉了开去。当下敷衍了一回,众人见时候已经不早,便请主人赐饭。王太史叫娘姨搬上干稀饭来,大家吃了些儿,谢过了扰,一哄而散。正是:

酒阑人散,灯回宝帐之春;漏尽更残,烟袅金炉之篆。

王太史坐着不走,心上自然要想金寓留他。金寓却总是一付无精打彩的样子,好像心上有什么心事一般。王太史猜不出他的意思,只是陪着笑脸挨近金寓身旁,贼忒嘻嘻的问长问短,金寓总不理他。

原来金寓新做了一个姓陈的客人,是现任通州知州的儿子,却是绮年玉貌,倜偿不群,顾影临风,风流自赏。这样的人物到了嫖界,不用说,自然到处沾光。从来鸨儿爱钞,姐儿爱俏。金寓自从做了这姓陈的客人,不多几天已经有了相好,要好异常。王太史那里赶他得上?况且王太史这般年貌,满面的晦气,一嘴有髭须,和姓陈的两边比较起来,一边就是那控鹤监的傅粉郎君,一边便是那终南山的虬髯进士;又好比那戏上的岑彭马武一般,相形之下,实在是玉石攸分,妍媸愈判。你想那金寓这样一个如花似玉的名妹,眼中那里看他得上?又有个姓陈的和他比较,自然越发的厌恶起王太史来。

刚刚这一天晚上和姓陈的客人预先约定,叫他十二点钟之后一定要来。偏偏又撞着了王太史这个瘟生,也想在金寓院中住夜,只把个金寓恨得金莲暗顿,媚眼横睃,暗暗的心中咒骂。却又不好赶他出去,只得由坐着,不去理他。反自走到窗前坐下,取出一付牙牌,点了一支洋烛,倒定定心心的打起五关来。王太史觉得有些没趣,讪讪的走到烟榻上边和衣睡倒。金寓见王太史竟不肯走,也无可如何,只得由他睡着。直迸到两点多钟,金寓心上甚是着急,恐怕姓陈的客人来了,看见有客在房,和他吃醋。想了一会忽然计上心来,暗想只好这般如此,方才遣得动他。想定主意,便走到榻旁坐下。

此时王太史已经有些迷迷糊糊的睡着,金寓把烟盘推过一边,软绵绵的和王太史并头睡下,脸贴脸的揉了一揉。王太史登时惊醒,金寓笑迷迷的低声问道:“辰光勿早哉,耐阿要转去罢,唔笃屋里向格太太,要骂倪格哩。”王太史起先见金寓睡在身旁,杨柳垂条,花枝低亚,又是香肩并倚,玉体轻偎,悄语低言的和他说话。这位王太史公自从和金寓落了相好,花了无数银钱,受了许多冷淡,那里受过金寓的这般恩宠。现在见了他这样的一番做作,早把个王太史弄得好像雪狮子向火,融化了半边,不知不觉的心荡神摇,六神无主。及至听得金寓叫他回去,却不由的呆了一呆,便也低声说道:“这个时候你还要叫我回去么?我们又不是没有落过相好,就在你院中住了一夜也不算什么希奇。”金寓听了,皱着眉头勉强答道:“勿是呀,耐格个人啥格实梗多心!倪倒是一片好心,为仔耐身体勿好,恐怕淘坏仔耐格身体呀。耐自家想哩,连浪几日吃酒碰和,总要到啥格辰光,一格人洛里有实梗几化精神?耐自家末无拨清头,倪倒有点替耐放心勿落。今朝勿要住来浪倪搭哉,豪燥点转去,养仔日把再出来罢。倪倒勿比格排倌人,单晓得自家寻开心。”金寓说到这一句,似乎有些不好意思,红晕两腮,娇羞满面,就顿住了口不说,用小手巾掩着樱唇,微微含笑。

王太史听了他这般说法,真是灵犀一点,沁人心脾。那一时心上的快活,在下做书的也说不出来。但是见了金寓这样的风神旖旎,情话温存,却又舍不得自家回去,便含笑向金寓道:“你的说话实是不错,我就听了你的话儿,在你这里借一夜干铺,明天回去如何?”金寓见王太史老着面皮只不肯走,登时就烦恼起来,面上却不露声色的仍旧向他笑道:“耐格人啥总是实梗介,倪搭耐讲格闲话,总归一句才勿肯听格,倪未来浪替耐发极,耐末倒杀死格糟蹋自家身体,阿要讨气,说起来像煞还是倪勿肯。”说着又微微的含笑。

王太史本来已是心神撩乱,也听不清他说的什么,只见他星眼微朦,云鬟半卸,口脂低度,兰气暗吹。忍不住心神荡漾,便把两手合抱拢来,把金寓搂入怀里肉麻了一会。金寓正色催他道:“格歇辰光耐好去哉呀,勿想等歇点路浪转去受仔风寒,出起毛病来,倪倒耽当勿起。”王太史听得金寓连连的催他回去,虽然不敢不走,却免不得有些疑心,还是蝎蝎螫螫的不肯就走。惹得个金寓火冒起来,春山半蹙,秋水含嗔,似怒非怒的瞅着王太史道:“耐格种人直头少有出见格。倪是顾惜耐自家格身体呀,耐定规勿肯转去,像煞倪总归有点勿放心。耐勿肯听倪格闲话末,随便耐去那哼,本来勿关得倪啥事,倪阿好来管耐?”说着坐起身来,推开了王太史的双手,掠掠头发就要跑开。王太史见金寓翻起面来,便吓得他不敢开口,只得也洋洋的立了起来。正是:

何郎年少,独看上苑之花;潘岳中年,仅有闲情之赋。

不知后事如何,且听下回分解。





第六十六回 苦温柔太史多情 空缋绻秋娘薄幸





且说王太史听得金寓催他回去,虽是心中不愿,却又不敢不依。原来王太史少年时节功名心切,拚命的萤窗雪案,苦志用功,那里晓得什么迷香洞里的风情,温柔乡中的况味?所以现在见了金寓这般模样,由不得骨软筋酥,那敢违背?只得满口答应。立起身来又叮嘱了金寓几句话儿,金寓只把头略点一点,秋气说道:“晓得哉,勿要多烦哉,豪燥点去罢!”王太史方才没情没趣的走了。

又过了不多几日,早已是蒲艾盈门,榴花照眼,薰风送暖,午节将临。王太史免不得要在金寓那里报效几个双台。除了照例开销之外,金寓还格外向他借了几百块钱。上海滩上的红倌人是端阳节后照例要歇夏的,金寓也把公阳里的房子回了,在观盛里赁了两楼两底的房子,暂且收场,就搬到观盛里去。

金寓忽然转了一念头,要叫王太史替他开销用度。明晓得王太史是个头等瘟生,乐得骗他一骗,便立刻去请了王太史,和他说明了要在观盛里暂停一节;更兼自己做生意做得怕了,最好拣一个合意的客人嫁了他,从此跳出风尘,脱离苦海,只是一时没有娶他的客人。一面这般说着,却把一只媚眼注在王太史身上,目不转睛只顾呆看。那一种娇羞的态度,一付憨媚的神情,王太史不看犹可,一见他这般模样,早已神迷目眩,魄魂魂销,不觉就要毛遂自荐起来。当下一口应允,替他开销门口,又露出些要娶他回去的话风。金寓不答应,也不回绝,只说:“格是倪一生一世格事体,勿是瞎来来格,慢慢里倪再商量。故歇除脱仔耐,倪总无拨啥第二格人,赛过就是耐格人啘。”王太史听了这几句说话,总算已经得了他的允许,就如捧了纶音凤旨一般,自然是感激涕零的了。

自此以后,金寓的开销日用,都是王太史一力承当。金寓还要拚命的敲他竹杠,今天要做衣裳,明天要打首饰,又要天天出去坐马车,吃吃大菜。看看一个多月,王太史已是所费不资。这金寓虽是出来歇夏,那观盛晨的房租日用却都是王太史出钱供给,差不多就是王太史包他一节一般。论理不该再和别的客人来往。金寓却只等王太史前脚走了,后脚便叫娘姨去寻了那姓陈的客人来,暗中双宿双飞,早已订了婚姻之约,只瞒着王太史一人。娘姨等虽然心上不以为然,却为的金寓本是自家身体,又不欠什么带挡,只好由他。可怜王太史那里晓得,还是妄想痴心打算要娶他回去,托了许多朋友去和金寓做媒。金寓不得不暂时答应,只说要王太史先付一千银子算做定钱,等到过了中秋再行择吉,讲定了身价四千,一切费用统通在内。

那做媒的朋友听了,估量着不甚妥当,只得和王太史一一说明。那知王太史听了并不疑心,把他的说话当作千真万真,心上十分欢喜,果然先付了金寓一千银子。金寓收了他的定钱也不写张收票,落得安安稳稳的用着王太史不心痛的银钱。从此以后,这金寓就要算是王太史的人了。

王太史因要谢谢媒人,有天晚上约了几个客人,就在金寓那边吃酒。金寓心上老大的不愿意,却没有法子回他。王太史向来本与辛修甫相识,这一席酒也把辛修甫请在里头。辛修甫虽也有些风闻,却还不晓得他们的情节,接了王太史的请客条子立刻就来。走进房间,恰恰与金寓打了一个照面,修甫把金寓打量了一回,暗想:“果然就是那公阳里的姑苏金寓。这是上海平康队里有名的辣手倌人,王太史那里是他的对手?”心上这般打算,不好竟说出来。王太史见辛修甫来了,连忙立起相迎。修甫进房,招呼了一会,见请客已经到齐,有几个不认得的,免不得彼此请教姓名,敷衍一回。王太史请客入坐,众人一齐坐下,齐修甫一面应酬众人,一面留心看那金寓的举动。觉得他落落寞寞的,面上明露着一付不高兴的神情,好像在那里想什么心事。王太史搭讪着和他说话,他也是待理不理的样儿。修甫看了甚是疑惑,却又不好问他。停了一会,那金寓忽然立起身来走到王太史身旁,附着耳朵说了几句,王太史连连点头。原来金寓对王太吏说的话儿是心上烦热,要出去坐一回夜马车,王太史那敢拗他,就点头答应。

偏偏的事不凑巧,陆云峰的坐位紧靠着王太史身旁,这几句话儿恰恰的被他听得明明白白。陆云峰的酒量本不甚高,今天多吃了几杯,已经有了七八分醉意,听得金寓要出去坐马车,明摆着是惹厌他们,要躲避出去的意思。不由得那腹中的酒直涌上来,按不住怒气,只听得“当”的一声,陆云峰把手内的酒杯向台上一放,冷笑道:“我们这样的吃酒,有什么趣味,吃出一肚子的气来。你要出去坐马车,那一天不好去坐,偏要拣着今天。我们在你院中吃酒,你就要去坐起马车来,这不是明明的惹厌我们这班人物,故意要躲了出去么?你要晓得这里的房子是王大人租的,我们是王大人请来的客人,与你什么相干,难道我们吵闹了你么?”金寓本来一肚子的没好气,正要发作,巴得有人引动他,听了陆云峰的说话,霎时间面泛浓霜,双眉倒竖,还没有开口,早听得王太史向陆云峰说道:“陆云翁不可这般动气,你不晓得内中的细情。他近来的身体着实有些不好,一天到晚只是恹恹牵牵的没有舒服的时候,好似有些暑病一般。我恐怕他郁出病来,所以叫他出去闲散闲散,坐坐马车,并不是他自己的意思,你不要错怪了他。况且他现在是歇夏期内,又不做什么生意。他已经答应节后一定嫁我,总算已经是我的人,比不得先前挂着牌子,不能得罪客人,你们总要原谅他些才好。”陆云峰听了倒说不出什么来,只在鼻子眼里哼了一声道:“如此说来,倒是我的不是了。”金寓还要开口,却被王太史拉了一把,金寓也乐得收逢,就立起身来开了衣厨,换了一身衣服,扶着一个小大姐,竟是头也不回姗姗的去了。

辛修甫一旁看着,也有些不忿起来,便向王太史道:“王伯翁,我说句不怕你见怪的说话,依我看来,这位贵相好却不是什么一定的好人,你这样的待他,他却这般的待你,那心地也就可想而知的了。”辛修甫的意思,原想要说出一番利害,把王太史劝醒转来,免得受了金寓的骗局,原是一片热心。岂知王太史听了,心上竟大大的不以为然,登时就露出不悦之色,冷冷的答道:“你们劝我的话儿虽然也是好话,但是我已经五十多岁的人,那里就会上了别人的当?况且我再三再四的和你们说了几回,他是个有病的人,总要体贴他些。他现在又不做生意,你们怪他的无非是说他目中无人,不肯应酬,殊不知他的不肯随便应酬,正是他的好处。你们众位见不到此,总是说他的坏话,又说他不是好人,真是‘以貌取人,失之子羽’了。”辛修甫听了王太史这一番糊里糊涂的说话,又好气又好笑,待要再和他争论几句,却想着别人的事与自家什么相干,劝他不听也就算了,何必这般起劲,做这个空头的冤家,想到此间,便佯笑了几声,不去和他分辩,大家闷闷的又饮了几杯。

忽然听得隔壁亭子间内有男女嬉笑之声,又像有人在那里密密切切的说话,座中惟有辛修甫最是留心,就侧耳而听。听了一会,仿佛好像就是金寓的声音,心上已是明白,正要开口问时,恰恰的陆云峰也听见了。陆云峰本来已经大醉,听见了这般声音,霍的立起身来,脚步歪斜,踉踉跄跄的走出房去,众人也没有理会他。

那晓得陆云峰走了出去,一直踅到亭子间门口,巴着门帘,在缝内留心张看,只见一个少年男子朝外坐着,生得长眉俊目,白面朱唇。金寓却坐在那少年男子的身上,两人搂作一团,脸偎脸的不知在那里说些什么。陆云峰见了气上心来,忍不住在房外大声说道:“哈哈,你坐马车坐到亭子间来了。”说了这一句,便仍旧回身进去。

这一声不打紧,把亭子间里的男女二人齐齐的大吃一惊。那少年男子连忙把金寓推开,立起来高声问道:“什么人在这里窥探?”

这个时候陆云峰已经走进内房,没有听见,却酒气冲冲的把方才看见的情形对着大众诉说。王太史还不甚相信,道:“只怕你看错了罢,我看金寓总不是这样的人。”陆云峰听了气得目瞪口呆,一句话也说不出,只一把拉了王太史的衣服,叫他同去看来。两人一同举步,刚刚走出房门,劈面撞着金寓进来,把王太史一把拦住道:“出去做啥,勿要瞎闯瞎闯啘,搭倪到里向去,好好里坐来浪。”说着就仍把王太史拉了进来,捺他向交椅上坐下。

好笑这位王太史虽然不信陆云峰的说话,却未免起了些疑心,原想要到亭子间去看看,究竟那少年男子是个什么样的客人。不料被金寓拉了一把就不知不觉的慢吞吞跟了进来,身不由己的软洋洋坐了下去。陆云峰看了这个样子,真是气破胸脯,却又无法可想,只得眼睁睁的看着他。

王太史坐了一会,免不得把那疑心又提了上来,吞吞吐吐的向着金寓问道:“刚才亭子间内的客人是谁?”金寓听了,由不得面上一红,心头乱跳,定了一定神方才说道:“亭子间里格客人?”金寓说了这一句,又顿了一顿道:“耐也勿必去问俚。耐王大人是蛮明白格人,一径体贴倪格,阿有啥勿晓得倪格难处。倪做仔倌人,吃仔格碗断命饭,总归有几化说勿出来格事体,像倪故歇实梗样式,阿好说是人家人,说出去别人阿肯相信倪?故歇想起来,顶好耐马上搭倪还清仔债,拿倪讨仔转去,依仔倪心浪越快越好,巴勿得明朝就跟耐转去,省得别人总归讲倪格丘话,说倪无拨真心。”说着双眉锁恨,杏靥凝愁,做出那一付幽怨不胜的样子。一双俊眼,水汪汪的剪水横波,好像是泪珠欲落。王太史听了这样的甜言蜜语,见了这般的弱态娇姿,禁不住魂魄齐飞,心神大乱,早把方才的一点疑心撇在不知何处去了。倒反着实的安慰了他一番,又回身对着众人说道:“何如?我早晓得他决不是这样的人,一定还有隐情在内,你们那里晓得这里头的细情!”众人虽然替他气愤,却是劝他不转,晓得无可如何,只得彼此默然不语,草草终席,也就散了。

又隔了一月有余,王太史正在家内和人代写寿屏,忽见陆云峰闯了进来,王太史因陆云峰几次要和金寓作对,心上有些怪他,又因陆云峰和他本有世谊,不能因此绝交,见他走进客堂,不免起身相见,谈谈的招呼几句。陆云峰不等让坐,劈头就问王太史道:“这两天你在金寓那里,可打听着什么新闻么?”王太史见他开口又提金寓,心上更加不乐,冷冷的说道:“金寓那里出了什么新闻,为什么要来问我?”陆云峰笑道:“这样说来,料想你还没有晓得,我倒和你打听着一件新闻,特地到你这边说个明白。你可晓得金寓和一个姓陈的恩客订了婚姻,今天就要动身回去么?”王太史听了那里肯信,只向陆云峰道:“你这个风声是那里去打听来的?

真是虚无缥缈的事情。我昨天晚上还在金寓那边,他正在那里发着肝气,睡在床上坐也坐不起来,那里今天就会跟着姓陈的动身回去?你这个慌话也说得太不像了。“

陆云峰顿足道:“到了这步田地,你还是这样痴情,怪不得要上别人的当。如今也不必说别的话儿,竟算我是说的谎话,我和你到观盛里去看看他究竟如何。”这一来有分教:

隔断蓝桥之路,拥雨停云;重寻白板之门,桃花人面。

不知王太史肯同陆云峰一同去否,且看下回分解。





第六十七回 桃花人面惆怅刘郎 细雨斜风重寻关盼





且说王太史正在家中写字,恰恰的陆云峰走了进来,把金寓要嫁陈姓客人,当夜就要动身回去的话和他说了。王太史那里肯信,只向陆云峰冷笑道:“你说是一厢情愿的话儿,不管事情的真假,你想金寓那边我天天过去,要真有这样的事情,我那有不晓得的道理。他嫁人不嫁人我不知道,难道你倒比我明白些么?”陆云峰听王太史只是一派的糊涂话,更加有气道:“现在不用说什么别的,我只要同你到观盛里去看他一趟,要是没有这件事情,凭你怎生罚我。我是一片好意,特地赶来给你一个信儿,你们的事与我有什么相干,难道我打了你们的破败,就有什么好处不成?”王太史听了只是不信,道:“无论你怎生说法,总而言之,他的病还没有全好,怎么就好嫁人,可不是笑话么?”这几句话把个陆云峰气得昏了,一把拉着王太史的衣裳,定要和他同去看个明白。王太史没奈何,只得勉勉强强的换了衣裳,一同出门。

那时已有掌灯的时候,陆云峰本来坐着包车来的,王太史就坐了自己的包车,一先一后,如飞的直往观盛里来。到了弄堂门口停下包车,王太史和陆云峰一同进弄,走到门口推门进去,王太史头一个进门,看那光景就觉有些不像。客堂里只点一盏壁灯,保险灯也不见了,楼上更是黑洞洞的没有灯光,更没有一些声息。王太史见此光景,晓得事情不妙,口中只叫得一声“阿呀”,急急的奔上楼去。陆云峰跟在后边,一同走进房内,只见房内的木器家生都是横七竖八的堆得满地。窗前梳妆台上只点着一盏半明不灭的长颈灯台,结了一个大大的灯花,光焰摇摇,闪烁不定。大床上的被褥帐子已经不见,连金寓的四只衣箱也不知那里去了。衣厨的门开得壁直,内中也是空空的没有什么东西。王太史见了这般形景,只气得目瞪口呆,默然无语。陆云峰立在后面,冷笑一声道:“何如?”

王太史此时心上千回万转,也不知是苦是甜,是酸是辣,辨不出自家心上是个什么味儿。呆了一回,想不出什么主意,还是陆云峰提醒他道:“金寓虽然逃走,一定还有未曾带去的人,或是粗做娘姨,或是厨子之类,方才我们进门的时候,看那样儿不像一人没有的。你姑且叫他们一声,把他们叫了上来问问他们,究竟是如何逃走,或者还可追得转来。”王太史听了方才醒悟,便高声在楼上叫了两声,听得楼下厨房内隐隐的有人答应,却是厨子的声音。果然不多一会,就听得“登登登”

的脚步声音走上扶梯,直到房内,见了王太史,不觉呆了一呆。王太史见了厨子上来,连忙问道:“他们那一班人那里去了,怎么一个人都不见呢?”厨子听了大为惊异,便从实说道:“我只晓得他们搬到归仁里去,不晓得什么别的事情。”王太史不等说完,急急的又问道:“他们既然搬去,为什么又单把你留在这边呢?”厨子道:“他们先把随身的箱子搬去,留下这些粗重的家具,叫我在这里看家,明天再来搬取,不晓得他们是到那里去的。”

王太史听了半晌并不开口,陆云峰却代他气愤道:“他们既是走了,料想一定是往通州一路去的,此刻轮船还没有开,我们赶到船上追问,一定寻得着他。”王太史一口气梗在胸口透不出来,挣了半天方才抽出一口冷气,问陆云峰道:“你怎么晓得他们是往通州一路,难道他们和你说过的么?”云峰道:“我还没有和你细说,那姓陈的客人是通州知州的儿子,年纪甚轻,品貌也好,所以金寓一心一意的定要嫁他。你虽然是个太史公,却已是五旬开外的人了,那里赶得上他们这一班少年浪子,专在倌人面上用那修饰的工夫,你想我们那里做得出这般模样?你当初不肯信我的话儿,如今懊悔已是嫌迟的了。”

王太史听了也不回答,呆呆的想了一回。陆云峰催他道:“怎么样?要去追问却要快些,何必在此间耽搁?”王太史朝他摇摇手道:“我想这件事儿还是我自己认些晦气,不要提起的为妙。你想金寓虽然答应嫁我,不过是一句话儿,又没有什么凭据;付那一千银子定银的时候,是我自己亲手交给他的,又没有一人见证,没有一个收条。就是赶到船上寻着了他,他若是老羞变怒,和我硬挺起来,也不能当真将他怎样,那时岂不是更觉坍台?所以我的意思,也不必再去追他,只算我瞎了双眼把他当作好人,上了他这样的一盘恶当。从今以后只当没有这件事儿,绝口不要提起,还要托你在朋友面前替我遮瞒一二,切不可逢人便说,弄得我没脸见人。”

陆云峰起初原是一团怒气,恨不得把王太史的事当作自己的事情,寻见了金寓,不知要打算将他怎样。及至听了王太史的一番说话,回心一想觉得实是不差。金寓虽然口说嫁他,却自己又不是媒人,又不是见证,没有什么一定的凭据,那里说得过他?万一金寓翻转脸皮抢白几句,说他们霸阻从良,那时放手又不是,不放手又不是,难道真好不叫他嫁人不成?如此一想,便把那一腔烈火一时间瓦解冰销,叹一口气道:“罢了罢,虽然不是我的事情,却实在替你气愤不过。你的说话也是不差,只是不去追他,就这般把他放走,却是便宜了这个良心丧尽的东西。”说着,又觉又埋怨王太史道:“都是你当初不肯听人说话,现在却弄到这个样儿。”王太史道:“事到如今,不必提起,我也追悔不来的了。”说罢,回头见厨子还自痴痴呆呆的立在一旁听他们说话,王太史当时就分付了那厨子几句话儿。那些木器家伙本来都是租的,只消叫人搬去就是,又叫那厨子暂看一夜,明天叫人来搬,又把那房子退了,厨子也辞了自去。这且按下不提。

只说王太史自金寓逃走之后,心上虽然懊恼,那花柳场中的兴趣却是一毫不减,只想要再看一个比金寓好些的人。果然不到几天,又被他看中了一个东尚仁的花彩云,也是一个著名的老蟹。王太史却又偏偏的拣中了他,做了不多两日,吃过两三台酒,碰过三四场和。花彩云见王太史呆头踱脑的不甚内行,明放着是一个土地码子,便想放出辣手弄他一注银钱,轻轻易易就和王太史做了相好。住过一夜,就撒娇撒痴的要嫁他。王太史见花彩云待他十分要好,不比金寓总是冷冷的样儿,心上就甚是欢喜,认定了花彩云是个好人,便请辛修甫和他做媒。辛修甫明晓得花彩云也不是个肯嫁王太史的人,又是一番骗局,恳恳切切的劝了他几次。怎奈王太史执意不从,口中只说:“花彩云决计不是金寓一般的人,你们不要胡说。你若是不肯和我做媒,我也不好勉强,只好待我去另请别人便了。”辛修甫听了无可奈何,只得和他去说,讲定五千身价,先付二千。这回的王太史却比前一回老到了许多,付定洋的时候叫辛修甫从中经手,还要花彩云写了一张收条,画了花押,又叫吴鉴光看了一个合婚的吉期。王太史自以为是千妥万当的了,不想到了吉期的前一天,又闹出花样来。

看官,你道是什么花样?原来花彩云接了王太史的定洋,打算要想个法儿逃,不料事机不密,不知怎样的走了风声,被辛修甫打听了出来,不觉勃然大怒,好在付过定洋,立有婚书,不比金寓的逃走一毫把握没有,所以不能追他;这花彩云既然出过婚书,又有收银的字据,和他出起场来,不怕他飞上天去。当下辛修甫得了这个信息,便立刻报了捕房,先派了一个警察来守住了花彩云的门口,随后修甫自己赶到彩云院中,当面问他不应这般混帐。谁知花彩云不慌不忙,一口咬定并没有这样的事情,竟是赖得干干净净。修甫听了,也指不出个什么逃走的凭据来。花彩云倒逼住了辛修甫,问他这句话儿是那个同他说的,一定要修甫指出这个人来,倒说得修甫无言可答。花彩云又道:“倪堂子里向嫁人勿嫁人,总归全靠一个名气,格号说话倪陆里担当得起?之修甫想了一回,方开口冷笑道:”据我看来,你的说话还是老实些儿,不要大宽大转的远兜圈子。你既是不愿从良,这也不能勉强,这是一生一世的事情,勉强不来的。与其嫁了过去将来闹什么笑话出来,不如现在一刀两断,讲个明白的好。我看你把他付过的定洋还了出来,我到他那边说法,从此两不相关,免得你心中不愿,否则你今天这件事儿闹了出来,既有婚书,又有现成的收据,恐怕你到了公堂难逃公道,不知你心上如何?“

花彩云听了不觉红泛桃花,低头无语,半晌方说道:“倪堂子里向格嫁人,勿是好弄白相格。故歇倪嫁王大人,外势大家才晓得格哉,一排老客人听见说倪要嫁人,来也勿来,生意才无拨格哉。辛老拜托耐,去搭王大人说声,倪是打打算算嫁拨俚格,故歇俚翻过来说勿要末,只要俚摸摸良心,自家说仔一声末哉。”修甫道:“你不要认错了我的意思,这件事情王大人还没有晓得,这是我的主意,还要去同他商量起来。”花彩云道:“故歇也无啥话说,倪虽然做仔倌人,也勿见得自家挨上仔别人家格大门,老实说,要讨倪格客人也勿止王大人一干仔。俚耐勿要,倪也无啥希奇,只要俚自家想想,说末说仔一泡,弄到仔故歇,原是一场无结果,阿对倪得起?辛老倪格生意瞒耐勿过,耐看倪故歇阿有啥格生意,还要叫倪还俚二千洋钱,叫倪陆俚搭去变格二千洋钱出来?”辛修甫听了,也晓得花彩云的意思,无非想赖掉这一笔定洋不拿出来。当下说来说去说了半天,又呼吓了花彩云几句道:“你若一定不肯,巡捕现在门口,我便叫他进来,先把你解到捕房再说别的。”花彩云吃了这一惊,怕吃巡捕官司,方才勉强答应。

辛修甫便出了东尚仁,直到酱园弄王太史家,把花彩云要暗中逃走,自己叫了警察看住了他的大门;又把花彩云的说话、自己的主意,详详细细说了一遍。在辛修甫的意思,以为花彩云既要逃走,就是勉强把他娶回家去,也要闹出笑话来,只有这样的一个法儿,叫他还出定洋,从此一刀两断,庶几不至吃亏。辛修甫在王太史的身上,也可算得是尽心竭力。那晓得王太史这两天正在高高兴兴的准备着要做那芙蓉帐里的新郎、玉镜台旁的花侍。正是:

准备画眉之笔,京兆风流;安排荀令之香,萧郎旖旎。

那心上的欢喜是不问可知的了。那里晓得辛修甫忽地跑来报了这个信息,好似青天霹雳,平地波涛,这一气直气得面青唇白,半晌无言。辛修甫又劝他道:“那花彩云本来是上海有名的辣手倌人,你就是把他娶到家中,也是养他不起,不如还是听了我的说话,仍旧把定钱收了回来,还是你的运气。”王太史寻思了一会,却又舍他不得起来,似信不信的道:“既是这般说法,我们两人同到彩云院中,看他怎生打算,我们再打主意便了。”辛修甫晓得王太史有些呆气,不肯舍他,却也无可如何,只好同着他径到东尚仁去。

花彩云见了王太史,登时做作起来,把眼睛挤得红红的,倒在王太史怀中。王太史见花彩云这般做作,娭光眇视,薄怒佯嗔,宝靥偎云,纤腰昵抱,又闻得一阵脂粉香水的味儿,早把个王太史弄得肢体皆酥,神魂欲化,头脑之内不由得有些浑淘淘的样儿,一点主意也没有了。再经花彩云把方才对着辛修甫的说话又对王太史说了一遍,更兼一手揪着他的胡须,一手扭牢他的耳朵,口中几哩咕罗的,倒把王太史抱怨了一个不了。正是:

雕笼押羽,池边共命之禽;宝槛移花,墙外春风之恨。

欲知后事如何,且听下回分解。





第六十八回 花彩云有意骗痴郎 王太史两番逃爱宠





且说花彩云和王太史两人扭作一团,揉成一块。王太史年纪高大,那里禁得起他这等的揉搓,早已气喘吁吁,上气不接下气的说道:“你有话只顾好说,为什么要这般动手动脚?”我们读书人那有这般的气力。“花彩云见了也觉好笑,方才放松了他,口中咕噜道:”别人家才来浪说倪逃走,倪好好里格人,为仔啥格事体要逃走?格号闲话勿知啥人格杀千刀,瞎三话四说出来格,连搭仔倪自家也勿懂。

“辛修甫此时正坐在旁边,眼睁睁的只好由他去骂。花彩云又接着说道:”倪格嫁人是自家情愿格,也无拨啥人来吃牢仔倪嫁人,勿壳张里笃格挡码子,才来浪说倪格丘话,故歇索性说倪要逃走哉。耐去想嗫,倪真格要逃走末,老早走脱格哉,陆里等到故歇?格号闲话,说得阿要勿色头?再有耐格饭桶,加二来得讨气,听仔别人家一句闲话,鸡毛当仔令箭,当仔真哉!说得明明白白格事体,耐故歇翻过来勿要。耐阿晓得别样事体末好搂白相,格个嫁人格事体勿是好弄白相格。一歇说要,一歇说勿要,才是耐一干仔格花头,也无拨实梗容易啘。虽然倪做仔倌人,名气倒要紧格;耐勿要末,勿见得倪就勿嫁仔人,不过耐自家想想,格个辰光耐搭倪那哼说法,故歇为仔一句无拨对证格闲话,弄得实梗样式,倪也勿来说耐,耐问问自家格良心好哉。“

花彩云这一席话说得有开有合,面面皆圆。王太史听了,自己回心一想,果然觉得对不起他,暗想这都是辛修甫无缘无故的造言生事,几乎离间了我的一场美满姻缘。心上这般想,面上却又不好怪他,只得对着花彩云极力辩白,说这件事儿并不是他自己的意思,是别人告诉他的,又极意的温存慰劝了一回,花彩云方才罢了。

把一个辛修甫气得满面通红,发作又发作不出,提醒又提醒不来,也只好怪着自家多事,按下不提。

只说王太史回去,过了几日果然清音彩轿,灯担堂名的把花彩云娶了进来,王太史的得意,自不必说。花采云自从嫁了过来之后,真个是随心贴意,百顺千依,把王太史哄得个死心塌地。这个时候,就是叫他把自己的性命交给花彩云,大约他也没有什么不肯。

隔了半个月,花彩云忽向王太史道:“故歇倪嫁拨仔耐,总算是格人家人。倪嫁仔过来,承耐格情,待倪总算好格。倪屋里向有格妩姆来浪,倪想转去看看倪妩姆,叫里快活快活。说起来,总算是倪嫁耐一场,让倪转去绷绷场面,勿得如耐阿肯勿肯?”王太史此时已经被花彩云迷得神志昏迷,梦魂颠倒,把个花彩云恨不得一天到晚含在口中,擎在掌上,看得他就是神圣父母一般,那里敢违背他的说话?

就连连的点头,一口应允。花彩云大喜。隔了一天,果然收拾了一个衣包,坐了马车,临走的时候还向王太史横波一笑,又分付他道:“倪今朝夜里向就转来格,耐勿要出去。”王太史诺诺连声的一直送出大门,看他上车自去。

原来花彩云未走之前,已向王太史说明,他的娘家住在新北门内,马车坐到城门口,再换了轿子进城。王太史还不放心,叫一个当差的跟去伺候。岂知去不多时,当差的一个人先自回来。王太史见了,急问他为什么不跟着奶奶进城,当差的回道:“奶奶分付,恐怕家中有事,叫家人先自回来,到晚上十点钟再放马车去接。”

王太史听了并不疑心,一直到了晚间,才慢慢吞吞的叫当差的配了马车到城门口去接那位新姨太太,王太史自坐在家中老等。那知左等也不来,右等也不来,直等到两点多钟,连当差的也不回来了。王太史到了这个时候,方才觉得有些不妙,却还想不到花彩云竟是一去不来。看看将近天明,王太史十分着急,连忙自己坐着包车,也到新北门外探望花彩云的信息。到了城外河边,停下车子四处一寻,只见自己当差的正在那里和马夫吵闹。马夫嚷着不肯再等,说:“你们说的十二点钟卸载,现在将近三点钟了,等不着他的人,不回去可做什么?”王太史听了晓得不妥当,急得心头火发,毛发烟生,看着这花彩云竟是做了断线的风筝,出笼的黄鹄,那里还有一个影儿?王太史等到天明,没法儿只得打发马车回去。打开花彩云的箱笼看时,一只只都是空的,不多几件旧衣服,不值什么钱。

原来花彩云有心逃走,趁着王太史有时出去,暗暗的把衣裳首饰搬运一空。王太史那里想得他这般一着,花了五千银子不算,还惹了一肚子的腌躜。起初的时候,要是听了辛修甫的说话,也还不至吃亏。偏偏的王太史执迷不悟,拚命的和银钱做对,一定要多送几千银子入了他的圈套才罢。你想,王太史虽然是个翰林,一时要借这三五千银子也不是容易的事情,到后来只落得泡影无常,电光一瞬,落花有意,流水无情。从此王太史为了金寓、花彩云两个倌人负了一身亏累,惹了无数牢骚,你想可有什么趣味?

看官且住,在下做书的做到此间,却有一言奉劝,一班花柳场中的坠鞭公子,走马王孙,且灰问柳之心,请听粲花之舌。大凡一班嫖界中人,必定要有嫖界的资格方才不至吃亏。什么叫做资格呢?第一要身段风流,第二要少年都丽,第三要郭家的金穴,第四是要嫪毒的大阴。这四件事儿样样完全,桩桩不缺,方算得花柳从中的飞将,温柔队里的班头。在下说到此处,就是人来问着在下道:“从来说鸨儿爱钞,姐儿爱俏,你怎么把身段放第一,面貌倒放在第二呢?”在下就回答他道:这个话儿却不是这般说法,你且安心静听,待在下一一的道来。

大抵堂子里的客人,只要有些阅历,自然随处占些便宜,那初出茅庐、一毫阅历没有的客人,自然到处要吃些亏苦。就是一味的少年美貌,也要有这一身功架帮衬着他,方才做得堂子里头上客。若是单靠着自家面貌,一些儿没有阅历,样样都是外行,那歌场酒阵的规模丝毫不懂,竟是个寿头码子、土地老儿,尽着在堂子里头呆头踱脑的乱闯,枉可的生了一付面貌,那里占得着什么便宜?就如倌人的资格一般,相貌好了,还要看他的应酬;应酬好了,还要拣他的功架。若单是面貌好些,身段应酬一些没有,像了那虎丘山上的泥娃子,楚王宫里的息夫人,不言不语的默然相对,可有什么味儿?照这样的看起来,不得不把客人的功架推为第一,那面貌只好靠后些儿,算作第二的了。至於嫖客的银钱自然也是一件逢时利器,但尽有那些曲辫子的客人看中了一个倌人,转着他的念头,往往花了一千八百、三千二千,倌人的身体也没有碰着一碰。可见虽然钱可通神,也有办不到的事体,所以这银钱一道只好排在第三。再讲起那武则天的淫经,张昌宗的秘记,这却要先有了上面的这三桩资格,方才做得到这个分儿,不是和那倌人一见儿面就可以如此如此得的,那就不得不把这件事儿排到第四去了。这是讲那做客人的资格。

如今再提起倌人的现状来,倌人们的看待客人,本来都是虚情假意,这却不好怪他。为什么呢?他做的就是这个迎新送旧的生涯,暮李朝张的本分,若要做了客人,一个个都把真心相待,不敲他的竹杠,不要他的银钱,倌人的首饰衣服,动用开支,却叫他出在那里?难道要叫他倒贴了银钱,把自家的身体供给客人的顽笑么?

从来说青楼妓女只爱银钱,没有情意,这句话却是大谬不然。他做着这行生意,不要银钱,可要什么?就是客人上了他们的当,也是客人们自家情愿,伏伏贴贴的把大把的金银双手奉送,不放一个屁儿。他们做倌人的难道好做了强盗,硬抢客人的钱么?这样的平心和气细细想起来,倌人们没有良心,实在怪他不得。只要做客人自家随处留心,不要上他们的圈套,到了那个时候,栽了筋斗,埋怨地皮,可是懊悔不来的了。

最可怜的是一班大人先生,自家的年纪差不多将近中年,堂子的情形却又是一毫不懂,偏偏的要学那丝竹东山的谢太傅、戎装骏马的陈季常,一天到晚,尽着在堂子里头选舞征歌,追欢寻梦。提着那一身的精神气血,捏着那几根的八字胡须,在倌人面前扮出了许多丑脸,做尽了无数戎腔。在上司面前做不出的奴颜婢膝,只要一见了倌人,他就自然而然、不知不觉的没有一样不做出来。在他自己想来:“我这样的降心迁就,屈意温柔,倌人面上可以告得无罪的了。”岂知倌人们见了那班大老,面上虽然应酬着他,心上却在那里十分好笑。赵是大人们卖弄风流,越是倌人们满心厌恶。见了他们那般动手动脚、嬉皮笑脸的丑态,不由得满身毛孔都皱了起来,成了一身的鸡皮疙瘩。这几句说话,实是在上海一个有名的倌人口内演说出来,并不是在下自家杜撰。列位试想,这老人花丛可有什么趣味?

如今闲话休提,书归正传。只说王太史不见了姨太太,无可如何,只得把一肚皮的气一齐发作在家人身上,把当差??大骂一场,说他为什么这般贪懒,先自回来,不跟着他们一起进城,以致闹出这般笑话。当差的一肚子的委屈,不敢回嘴,只好诺诺连声,连说:“家人该死。”王太史骂了他一顿还不出气,立刻把他撵了出去,方才完事。

王太史自从经了这两番笑柄,谁知他并不灰心,又在人家席上看中了陈文仙,一连叫了十几个局,吃了两三台酒。陈文仙虽然不比金寓和花彩云这一班辣手倌人,却总有些红倌人的习气,见了王太史这般年纪,须发皆苍,那里有什么真心相待?

只是面子上淡淡的应酬他。王太史却看见陈文仙相貌甚好,身段玲珑,真是润脸呈花,圆姿替月;赵后回风之态,梁家七宝之妆。从前的花彩云和金寓两人的丰格,都觉得赶不上他。这位王太史就癞蛤蟆想吃起天鹅肉来,每每的在陈文仙院中一直坐到夜深还不肯走,微微的露出些仰慕的口风,要想陈文仙留他住夜。陈文仙那里睬他,只装着糊涂,不懂他的意思,就是这般一天一天的挨了过去。

王太史初做陈文仙的时候,章秋谷正在苏州,所以秋谷并不曾晓得。到得章秋谷回来之后,因为借着中秋的局帐,试出陈文仙的真心,未免又加了几分情爱,每天晚上竟不回去,十天之内,倒有六七天住在陈文仙的院中。这一天正逢礼拜,秋谷晓得堂子里头礼拜的生意总比别天好些,恐怕去得早了,有些碰和吃酒的客人还没有散局,一则陈文仙分不开身,二则呆呆的坐在那里也觉得没有什么趣味,有心去得迟些,直到十二点钟之后方到兆贵里。在章秋谷的心上,以为这个时候一定没有什么客人的了。岂知到了那里,房间挤得满满的,一些没有空儿,大房间内有一个客人正在摆着双台,另外还有两三场和碰得甚是热闹。秋谷去了,没有房间,只得在大房间背后一间小小的后房内权且坐下。

秋谷见了这般光景转身要走,陈文仙赶了进来,一把拉住死也不放。秋谷只得坐下,和陈文仙讲不多几句说话,忽又听得楼下相帮高叫客人上来。陈文仙立起身来往外便走,迎出房门。秋谷坐在房内,只听得陈文仙对那来的客人说道:“王大人,对勿住,今朝房间勿空,阿好等一歇?”又叫宝珠姐姐道:“耐到楼底下花丽卿搭去看看,阿有空房间?”宝珠姐答应自去,又听得那客人说道:“既是房间不空,也不必去另借房间,我去一回儿再来也好。”那说话的声音是常熟口音,并且觉得十分相熟。正是:

谢太傅中年丝竹,别有深情;潘黄门两鬓霜华,犹多绮思。

不知来的客人究竟是谁,静听下回分解。





第六十九回 兆贵里翰林出丑 春申浦名士吟秋





且说章秋谷坐在房内听那房外的客人声音,送入耳中十分相熟,但是一时之内急切辨不出他是谁,便走到后房门口,巴着门帘向外张望。仔仔细细的打量那来的客人时,原来不是别人,就是那著名蜡烛、第一瘟生的王太史。论起世谊来,王太史还是章秋谷的父辈。平日之间,章秋谷见了王太史的面儿总是循规蹈矩,恭恭敬敬的按着后辈的礼数。这位王太史却是倚老卖老的,每逢见面的时候总要说两句凿四方眼的话儿,一个不高兴,还要教训几句。章秋谷虽然年少才高,天资疏放,目空一世,睥睨不群,不把王太史放在眼内,却因为他是个多年的父执,不好去得罪他,碰了他几次钉子,心上也觉得有些不快。

刚刚的事有凑巧,今天和王太史混在一堆。章秋谷见了王太史,暗想:“这个老头儿平日间满口道学,好像一个正派人儿,今天难得和他遇见,不如把他让进房来,大家坐在一起,塞了他的口儿,省得他一见了面就要罗罗苏苏的,说那些道学的扳谈。”想罢正要走出来招呼,忽见王太史转身要走,章秋谷连忙一手把门帘掀起,笑容满面的向王太史道:“原来果然是老世伯,久违了,怪道说话的声音十分相熟,一时几乎想不起来。今天他们这里的房间不空,老世伯何不就在这里坐一会儿?”

王太史无意之间突然遇着了章秋谷,觉得有些不好意思,又不能一定要走,只好讪讪的进来坐下,满身的不得劲儿,和章秋谷讲了几句应酬话儿,脸上还有些红红的,好容易停了一回方得自在。抬起头来再看陈文仙时,只见文仙和秋谷并着香肩坐在一张榻上,纤腰斜亸,素手同携,和秋谷咬着耳朵不知说些什么。说了一回,又看着王太史回头匿笑,仿佛是在那里笑他,那一种要好的样儿,一时也说他不尽。

更兼榻床对面恰恰的摆着一面小小的墙镜,正照着陈文仙和章秋谷两个的影儿,真个是一对璧人,两株玉树。一个是飘烟抱雨,丽华杨柳之腰;一个是敷粉涂朱,平叔莲花之面。琼枝照夜,宝靥回春;赵家掌上之身,汉殿春风之影。王太史不看犹可,一见章秋谷和陈文仙这般亲热,一股酸气直从脚底下冒了起来,涌到心头,按捺不住,不由得冷笑一声,对着秋谷说道:“老侄,我有一句话儿劝你,你可不要见怪。你们年纪轻轻的人,比不得我们年纪大了,自然只好借着到堂子里头走走,寻寻开心。老实说,我虽然老朽无能,却也挣了一名进士,点了一个翰林,读书一层总算交代过了。你现在年纪方交二十,又没有成就功名,这个当儿正是在窗下用功的时候,将来或者博得一个科名,不枉了你是个世代书香、宦家子弟,何苦尽着在堂子里头寻花问柳,弃掷了这些有用的光阴,我倒有些替你可惜。并不是我自己倚着多年的父辈,说这些倚老卖老的话儿,你可知去日苦多,书囊无底?我看你还是敛迹些儿的好。”

章秋谷本来不佩服王太史的学问,说他除了做八股策论、写白所摺试策之外,一样也不懂什么。现在听他居然教训起来,不觉满心发火,顾不得他是什么父执的了,当时便推开了陈文仙立起身来,鼻子管里笑了一声,向着王太史说道:“世伯的话果然不错,小侄今天多多承教了,只是还有一句话儿不得明白,要求世伯指教。”

王太史听了,一时也不得明白,便问秋谷道:“你有什么不懂的话儿要我指教?”

章秋谷冷笑道:“据世伯这样说来,像我们这般年轻的人,是不该在堂子里头顽耍的了。请问世伯,我们这样的年纪不该顽耍,难道直要到年纪大了,腰驼背曲、鹤发鸡皮的时候才好在堂子里头顽要么?如今的这班大人先生,年轻时候读了几句死书,一概的世故人情全然不懂,那里还有工夫来考察这嫖界中间的学问?到得上了年纪,自以为是功成名遂的了,免不得倒过头去重新顽耍起来,却不想自家事事外行,那里有嫖界的资格?闹出许多笑话,惹了无数牢骚,把自家辛辛苦苦的银钱,大把儿撩在水中,讨不出倌人一个‘好’字。更兼潘鬓将斑,何郎已老,勉勉强强的涎着脸儿去讨倌人的欢喜,费了自家的精力,博得那无谓的风情,应了那‘一树梨花压海棠’的一句说话。如此的看来,到了这般年纪,何苦的还要自家卖弄风流,到头来落得一场没趣?不如还是趁着少年时节及时行乐,春花秋月尽是可怜,檀板金尊居然无赖,也未尝不是一个消遣的法儿。要晓得来日无多,春华易晚,若是到了你老世伯这般年纪方才要及时行乐起来,可是来不及了。”章秋谷还未说完,陈文仙听他说得好笑,忍不住“扑嗤”的笑了一声。

王太史听得章秋谷的话风,句句是说着自己,气得他双眉倒竖,两眼圆睁睁,嘴上的几根稀稀郎郎的胡子一根根都直立起来。又听章秋谷郎然说道:“至于学问一层,小侄虽然年幼,自问还不弱于人,不过时运不济,没有取得科名罢了。一个人的文章经济,都是在少年时节得来,若到了二十以外还要用什么功,读什么书,这个人也就是一钱不值的了。”

王太史自出娘胎,从没有受过别人这般教训,只见他的脸上一会儿红,一会儿白,一会儿青黄不定,好似开了一个颜色铺子一般;直把他骂得气塞胸膛,火星直冒,眼睁睁的看着章秋谷。看了半晌方才说出一句话来道:“好,好,我是好心劝你,你倒教训起我来!我活了五十多岁年纪,没有受过这般糟蹋。你这个人真真的不知好列!你想你在外面荒唐,与我什么相干?我不过念着你们尊大人和我的交情,所以这样的苦心相劝,想要保全你的名誉,不想你倒这样的把我顶撞,眼眶内看不起人。就算你是怎样的高才,我总算是你的父执,可该把我这样糟蹋的么?”说着气喘呼呼的,把一把象牙油纸扇儿不住的乱扇,头上的汗珠竟有黄豆一般大小,口内连说“岂有此理”。

章秋谷见了甚是好笑,又见他气得这般模样,好像心上也觉得有些过意不去起来,便含笑说道:“老世伯言重了,小侄怎敢这般大胆,糟蹋起你老世伯来?但是小侄性情伉直,心上留不住一句话儿,所谓‘骨鲠在喉,吐之为快’,还求老世伯的大量海涵,不要和小侄一般见识才是。你老世伯是十年读书,十年养气,比不得我们这一班少年性急的人。”说着,便立起身来打了一躬。

王太史听了章秋谷的说话,虽然恨他切骨,却是无可奈何,只得顿住了口,默然不语。陈文仙此时走到前房,应酬客人去了。王太史坐了一刻,觉得心中余怒未平,坐在此间无谓,便起身要走。秋谷也不相留,任他先走。陈文仙赶到后房相送,王太史临走的时候,似笑不笑的向着陈文仙道:“恭喜你,有这样的漂亮客人在你院中来往,怪不得你要做他的恩客,果然生得不差。像我们这样的老头儿,你面子上虽然一样应酬,那心上究竟是勉强的。”陈文仙听了,变了面皮,正要回答,不料王太史晓得自己说他不过,三步并做一步,急急的走下楼梯,头也不回,竟自去了。陈文仙又气又笑,回转后房对着秋谷笑道:“耐听听看,格号闲话阿要气数?”

章秋谷也不觉笑了。按下不提。

只说章秋谷在上海过了中秋,应办的事情差不多将次完结,秋谷打算过了重阳,束装回去。恰恰的金小宝过了秋节不做生意,另外租了几间房子和贡春树住在一起,只留下章秋谷一个人住在吉升栈中。花朝月夜,甚是无聊,除了和几个知己些的朋友谈谈,便往陈文仙院中走走,每每整天整夜的不到栈房。

这一天,秋谷正在栈内检点往来的信札,忽然见王小屏走了进来,秋谷大喜,让他坐下。谈了一回,王小屏随意把案上的书本翻看,只见一本《玉溪诗集》,内夹着两张写过的冷金笺,写的一笔赵松雪行楷,甚是秀挺。第一张上面写首“秋谷八章”的题目,下边写着“憔翠青衫客旅稿”。原来这憔翠青衫客,便是章秋谷的别名。王小屏看了,晓得是章秋谷的近作,便朗吟起来道:

十二阑干映画塘,水心亭子好招凉;

夜深独立无人问,一点流萤过曲廊。

画船载酒听湖歌,十里湖光压芰荷;

行到六桥烟外路,碧湖深处晚凉多。

珠帘不卷夜星低,独倚银屏望翠微;

坐久不知风露冷,满身香影湿罗衣。

一夜新凉透碧棂,谁家玉笛暗中听;

当时七夕真虚度,惆怅牵牛织女星。

三更凉露湿秋千,云母屏风隔半偏;

冰簟银床眠不得,碧天如水夜如烟。

锦帏半掩睡惺忪,昨夜轻寒力更慵;

八尺龙须人未起,月明庭院冷梧桐。

两岸溪光拥板桥,岸花开处泊兰桡;

可怜扶荔宫中柳,瘦尽当年一捻腰。

大堤残柳乱栖鸦,灯火帘栊月又斜;

一夜西风秋不管,隔滩闲煞白苹花。

王小屏念完,不觉击节叫好。秋谷道:“你不要谬选,还有几首《秋闱怨集唐》,好像集得好些,你一总看了再说。”王小屏听了,便又取过第二张来,高吟道:

倦倚东床白玉床,为谁销瘦减容光;

今宵始觉房栊冷,卧后清宵细细长。

露床风簟半欹斜,深掩妆窗卧碧纱;

二十五弦弹夜月,不知秋思在谁家?

象齿薰炉未觉秋,天河迢递笑牵牛;

相思一夜知多少,春入眉心两点愁。

深院沉沉独闭门,为君惆怅又黄昏;

一钩冷雾悬朱箔,金屋无人见泪痕。

月过花西尚未眠,月光如水水如天;

晚来怅望君知否,织女佳期又隔年。

已凉天气未寒时,桂魄初生秋露微;

直道相思了无益,残宵犹得梦依稀。

王小屏看完了,真是佩服得五体投地,拍案称赏,又把那两张诗翻来覆去的,看了几遍道:“你这《秋词八首》直是逼真的王渔洋,渔洋七绝全取丰神,不食人间烟火,真个是锦心绣口,我们那里做得出来?”秋谷笑道:“你这个人,无论什么事情总有一番谦逊,其实我们这样的交情,何必定要拘着这些俗套。你的著作我是拜读过的,真如大海长江,波澜万里,若令当世竖儒见了,一定要挢舌不下者三日。像我这样风云月露的才子,那里赶得上你的大才。”王小屏不等秋谷说完,哈哈大笑道:“算了算了,你说我无论什么事情总有一番谦逊,你为什么也要这般的谦逊起来?”正是:

折倒迂儒之论,名士高谈;狂吟子夜之歌,王郎绝唱。

不知王小屏还有什么说话,请看下回便知分解。





第七十回 好良宵诗征出阁词 留学生弹打章秋谷





且说王小屏向章秋谷说道:“你说我过于俗套,为什么你自己也要谦虚?我们大家只好算个扯直罢了。”秋谷不觉也笑起来。王小屏坐了一回便向秋谷道:“你可晓得辛修甫的令妹就要出阁么?”秋谷惊道:“我这几天没有见着修甫,不晓得这件事情,即是他家有喜事,我们还该备个公分才是。”王小屏道:“我正为要约公分,特为来和你商量,你看还是怎么的一个约法?”秋谷道:“据我看来,还是等他回门的那一天,送一班髦儿戏,大家热闹不好么?”王小屏即说道:“我也是这般的想,既是你也是这个主意,好极了!我们就立刻写好贴子,我们两人为头,去约那一班朋友,可好么?”秋谷点头道:“好。”当下就取过一付全帖,写好知单,交与王小屏带去代发。那单上的人差不多也有二三十位,一时不去提他。

只说不多几天,辛府吉期已到,秋谷等一班朋友一齐穿着衣冠,前去道喜。真个是车马盈门,十分热闹。隔了一天,新郎、新妇归宁,辛府中更加热闹。章秋谷和王小屏两人到得最早,不多一会客人陆陆续续的到来。琼筵坐花,羽觞醉月,哀丝豪竹,添酒回灯。春开孔雀之屏,褥隐芙蓉之绣。整整的闹了一夜,直到四更将尽,方才宾主尽欢而散。章秋谷即席挥毫,赋了八首《出阁词》。下笔如风,文不加点,一时传诵沪滨,脍炙人口。那诗是八首五律,做得深情如水,宛转关生,旖旎风光,一时无两。在下倒还有些记得呢,免不得背诵出来给列位看官听听:

绮阁辞亲日,爬瓤问字年。

含情依阿母,掩泪整花钿。

临镜还惆怅,妆成亦自怜。

不知为底事,眉黛蹙湘烟。

自画檀蛾浅,梳妆拟大家。

风前停玉佩,天上驻云车。

宛转回鸾袖,逡巡换绣鞋。

娇羞扶不起,妒煞海棠花。

箫管送星蛾,天孙意若何。

轻风吹鹊驾,微雨渡银河。

红泪阑干湿,矜持宝靥酡。

欹斜偎画烛,未敢展双蛾。

灯火拥楼台,端详宝扇开。

双痕留晕脸,羞态压蛾眉。

嫁得乘龙婿,应怜倚凤才。

蓬山应早到,玉漏漫相催。

微觉口脂香,春风夜正长。

寻声轻唤婢,背影暗窥郎。

侧坐犹低首,迟徊末卸妆。

却嫌红烛下,夫婿太轻狂。

背人无语处,睡意已惺忪。

玉箸啼痕浅,鲛绡腻粉红。

牢钩金屈戊,稳放玉玲珑。

春梦迷何处,蓬山十二重。

妆台携手立,私语嘱殷勤。

未必檀郎信,还防侍婢嗔。

低鬟时敛笑,凝睇更含颦。

珍重罗帏里,还疑梦里人。

此夜最魂销,银屏倚素腰。

钗光和影颤,春色泥人娇。

惆怅温家镜,徘徊弄玉箫。

怜他孤馆客,坐听雨潇潇。

再说辛修甫自从办了这桩喜事,倒整整的忙了半月有余,好容易才得料理停当,仍旧和章秋谷、王小屏等一班朋友天天来往。这一天,到了午后三四点钟,大家到陈文仙院中去寻章秋谷。寻到了秋谷,彼此谈了一回,秋谷就同着辛修甫、王小屏二人到一品香去吃大菜。陈文仙听了也要同去,秋谷答应,叫他随后就来。

三人一同到了一品香,占了一间房间,恰好开出去就是洋台,甚是轩爽。秋谷和修甫随便坐下,谈了一回,听得隔壁房内的客人,高谈阔论的十分热闹,还夹着些馆人的燕语莺声。章秋谷留心听去,只听一个人的声音说道:“你们都说日本妇女的面貌甚好,然而我却不爱他。你想他们身上穿着一身和尚一般的衣服,脚下又踏着一双高低不平的木屐,走起路来踢踢跶跶的像个什么”所以我在东京的时候,我始终没有陪着你们到堂子里头去过一趟,就是这个缘故。“又有一个人接着说道:”我们中国妇女的打扮实在娇淫得狠,不要说是别的,你只看他们缠那一只小脚,走起路来,好似那出水荷花,随风杨柳,不由得令人魂魄俱销。中国的人,都是把些有用的精神消磨在一班妇人身上,那里还做得出什么事业?你看他们这样的小脚,缠起来不知吃了许多痛苦,费了如许工夫,却只供得一班嫖客的玩具。“说着,忽听见倌人的声音嚷道:”勿要嗫,啥实概介?“

章秋谷听了他们起先的一番说话,晓得定是一班出过洋的留学生,听到此处忍耐不住,便立起身来走到洋台上面,隔着玻璃窗看去。只见三个穿西人服式的少年,一式的都戴着金丝边眼镜,三个留学生倒叫了六个倌人。更有一个留学生把一个倌人抱着坐在身上,一手在他胸前乱摸,丑态百出。那倌人挣又挣不脱,跑又跑不开,只把他急得满面通红,口中“阿唷阿唷”的喊个不住。又有一个把个倌人的粉面双手捧住了,不住的在他脸上乱闻乱嗅,那倌人躲闪不过,急得几乎要哭将出来。其余的倌人见了,恐怕连累到自家身上,有的背过脸去暗笑,有的立起身来走开。秋谷见了他们这个样儿大不入眼,冷笑一声走了开去。辛修甫也在后面看见,跟了过来,一同倚在栏干上低头俯眺。辛修甫叹息道:“留学生是最高的人格,怎的现出这样的怪像来?这一班人真是那留学生中的败类。”

秋谷此时心上十分作恶,听了辛修甫的说话,由不得惹起他的议论来,大声说道:“你还没有晓得,我们中国的人,只有留学生的人格最高,亦惟有留学生的品途最杂;不论什么娼优皂隶,只要剪了头发,穿了一身洋装,就可以充得留学生的样子。你道这班留学生将来有什么用处么?他开口革命流血,闭口独立自由,平日之间专会吹牛皮说大话,不论你是个什么人儿,也不是他们的对手。好像为了同胞的国民,真肯把自家的身命当作牺牲,去供那野蛮政府的刀锯鼎镬;其实到了那要紧的时候,不要说是叫他流血,就是在公堂之上轻轻的打他几下手心,他也要吓得屁滚尿流,汗流浃背。”

章秋谷说到此处,听得隔壁的门窗一响,那三个留学生一齐走了出来,走得皮靴声响咯支咯支的,也到洋台上来。却是一个个怒容满面,似乎已经听见了章秋谷的说话一般。辛修甫回头一看,晓得他们已经听见,那班留学生的性情,无论什么事情别人做不出来的,他都做得出来,便把章秋谷的衣服拉了一把,叫他不要再说的意思。那知章秋谷本来脸向那边,没有理会,况且他向来胆大,那里顾得这些,接下去大声说道:“虽然他们里面也有一两个好人,看得清时势阽危,担得住支那全局,却是这样的人一千个里头恐怕还拣不出一个,倒有九百九十九个是这般的斯文败类,凉血畜生。”章秋谷正在说得高兴,还要说下去的时候,忽然那边的留学生内走过一个身材高大的人来,立在章秋谷面前。秋谷眼光一闪,早看见就是隔壁房间里的学生。只见他眼露凶光,眉横杀气,怒容满面的对着章秋谷道:“你也是国民中的一分子,为什么要这样的毁骂同胞?难道我们一班留学生都是像你口中说的这般败类么?”说着把手在衣袋里头一摸,竟摸出一管小小的手枪来,抢上一步对着章秋谷开机便打。

说时迟,那时快。章秋谷初时看见他这般样子,怒气冲冲的,早料定他不怀好意,急忙把子腾开一步,却也还想不到他竟要拚起命来。当下见他在衣袋里头摸出手枪,擎在手中正要开放。这一下子,可把那旁边的辛修甫,里面的王小屏,吓得一身冷汗,手脚慌忙,不约而同的齐叫一声:“阿唷!”就这一声里,这个时候,章秋谷正是“会得不忙,忙家不会”,不等他手枪放出,早已把头一低,扑地一个箭步,穿到他的身旁,一手警住他的手腔,趁势飞起一脚,不竖不斜,正踢在那人的臂弯上面。不由得骨节酸麻,手内一松,那弹子还没有放出来,早被章秋谷轻轻的一把将手枪夺去,顺手把他的颔下一叉,那人立脚不定,连退了几步,仰面朝天扑地一交。辛修甫和王小屏看了方才放下心来,暗暗的叫了一声“侥幸”。再看章秋谷时,虽然似乎也有些惊慌的样子,却是面上不红,口中不喘,好像没有这件事儿,手中拿着一管手枪,微微含笑。那跌了一交的人也自家扒起,立在一旁呆呆的不发一言,却也并没有惊惧的意思。

章秋谷并不动气,走过去笑咪咪的向他说道:“方才我的说话虽是过于激烈了些,但不过是这么一句话儿,算不得什么睚眦之怨,何至于要弄到这般的白刃相加,和我拚起命来呢?况且我说的是那一班无耻的学生,并不是指名说你,你只要不是这样的人也就是了,为什么要勉强把这些留学生的罪过,都揽在自己一人身上,又是个什么意思呢?”几句话把那个人说得哑口无言,十分惭愧。秋谷又道:“今天这件事,幸而遇见了我,没有受伤,若是换了别人,一时间定要闹出一场人命。你说我是国民的一分子,不应该毁骂同胞,难道你放枪打我,残害同胞又是应该的么?

你可知租界上边,那里容得你这般胡闹?本该把你扭到捕房,解堂问罪,但是我也不是这样多事的人,只要自家没有受伤也就算了,免得你们又要说我借着警署的势力欺压同胞。不过你虽然和我为难,我倒还有一句良言相劝,下次须要自己小心,切不可这般冒失,若是落在别人的手内,恐怕你没有这样便宜。“说着,便哈哈冷笑,羞得那人面涨通红,低着头一句话也说不出。秋谷又把方才抢下的手枪替他放在衣袋之内,说声”少陪了“,便举步进房,不去管他。

辛修甫和王小屏接着秋谷道:“今天真是你的运气,没有受伤。”秋谷笑道:“我倒没有什么,恐怕你们的心上倒受了一个大大的惊吓。”正在说着,别处房间里的客人听得有这般奇事,一齐拥了出来,都要看看这姓章的是何等人物。顿时洋台上拥了无数的人,连着一班侍者也挤在里边,七张八嘴的纷纷议论。再看那动手的学生时,早已不知去向,悄悄的溜回自己房中。

原来那两个同来的人,见同党无故行凶失利,也是出其不意,着实吃了一惊。

拉既拉不住,走又走不开,都吓得回到房内,探头探脑的往外边张看消息。后来见章秋谷随随便便的还了他的手枪,并不鸣捕,方觉放心。恰恰的动手的学生溜了进来,连忙算了菜帐,打发了来的倌人,悄悄的鸦雀无声,抱头鼠窜而去。这且不表。

再说章秋谷坐在榻上,见拥了一大班人立在门口,咕咕哝哝的不知大家在那里说些什么。章秋谷正觉得有些厌烦,忽然门外走进一个人来,身体魁梧,丰仪高爽,一把拉了秋谷的手,哈哈大笑道:“我听见他们说什么姓章的客人,就有些疑心到你。果然一点不差。”秋谷举眼看时,原来是他的同窗好友,是个常熟城内有名的富翁,差不多也有二三百万光景,年纪止有二十多岁,已捐了个浙江候补道,姓李,单名一个煜字,表字子宵。这李子霄虽是个富家子弟出身,却是精明得狠,差不多些的事情都瞒不过他,在上海开着几家钱庄,几处当铺,生平敬重的朋友止有秋谷一人。这一回到上海来盘查帐目,就住在后马路自己的钱庄里头。今天同着一个朋友姓沈的,也在一品香吃大菜,听得隔壁人声嘈杂,便叫了侍者进来,问他为什么这般吵闹。侍者把留学生放枪打人,反被一个姓章的客人夺了手枪的事情,一一的朝他说了。李子霄听了,也要去看看这姓章的是什么一个样儿。所以也到门口窥探,不想一眼早看见了章秋谷,心中大喜,走进来招呼。秋谷见是李子霄,也觉欢喜,便邀他一同坐下谈谈。李子霄不肯道:“我那边还有客人,还是你倒我那边去坐一回儿的好。”说着不由分说,拉着便走。又让辛修甫、王小屏两个先走。秋谷见李子霄甚是爽直,只得依着他一同过去。正是:

偶失睚眦之意,白刃自如;重逢车笠之交,故人无恙。

欲知后事如何,且看下回分解。





第七十一回 李子霄他乡逢旧友 辛修甫谈笑讽良朋





且说李子霄不由分说,拉了三人就走,章秋谷因李子霄为人性直,便并不推辞,向着修甫、小屏招招手儿,一同跟了过去。李子霄先请辛修甫和王小屏二人坐下,他们素不相识,免不得彼此客套一番。章秋谷到了子宵那边,见还有一个客人,年约三旬,身材中等,倒也和霭近人,春风满面。秋谷便朝他拱一拱手,请教他的姓名,方知也是常熟富户,叫做沈仲思,因为他排行第六,大家都叫他沈六。秋谷应酬了他几句,正要坐下,忽见李子霄和沈仲思都是坐在两旁,主位上空着没有人坐,觉得有些诧异。正要问时,只听得莺声呖呖,从洋台上转进一个倌人:宝髻盘云,珠光照采;衣裳艳丽,态度妖娆;眉横远岫之烟,眼媚湘江之水。一步步的走到面前:好似那华月初升,春云乍展;仿佛惊鸿之影,依稀照月之妆。莲步移来,香风到处,倒把章秋谷的眼光提了一提。仔细看那倌人时,原来不是别人,就是自家的相好,四大金刚里头的张书玉。暗想:这可糟了,我合他们闹到一起来了。

张书玉见了秋谷,也不觉呆了一呆,停了一刻方开口道:“倪当仔是啥人,想勿到就是耐。”说着向秋谷微微一笑,点了点头,便向主位上坐了下去。秋谷见了觉得诧异,忙问:“为什么这般坐法,今天请客,可是你的主人么?”张书玉横波一盼,启齿嫣然,还未开口,李子霄见张书玉和秋谷这般熟落,好似素来相识的一般,不觉疑惑起来,插口问书玉道:“你和这位章大少可是一向认得的么?”书玉听了李子霄这样口风,晓得他有了醋意,便连忙转口掩饰道:“格位章二少爷,来浪上海滩浪真真是多年格老牌子哉,稍微有点名气格倌人,陆里一个勿认得俚?勿要说是倪,就是金刚里向格林黛玉搭仔金小宝,也才认得俚格呀。”一面说着,暗中伸一只小脚,把章秋谷钩了一下,又微微的递了一个眼风,似乎叫他不要说穿的意思。秋谷会意,乐得假作不知,轻轻的几句话儿就被他遮过去。

李子霄听了,心上不觉释然。张书玉方回头过来向秋谷道:“今朝是倪专诚请格位李大人搭仔沈大人,到该搭来吃大菜,难得碰着耐格二少,也肯赏倪格光,总算倪靠仔李大人格福气,今朝借花献佛,绷绷倪格场面。”秋谷听他说得文绉绉的十分客气,觉得好笑,便也调侃他道:“阿唷,今朝书玉先生请客,是百年难遇格事体,倪阿好勿领耐格情,只怕倪无拨格号福气,吃仔耐格大菜,转去生起病来末尴尬哉。”这几句话说得好笑,修甫等一齐大笑起来。张书玉也忍不住抿着嘴儿好笑,笑了一回,书玉方才向秋谷说道:“刚刚倪听见俚笃说,有两个外国人吃醉仔酒,拿仔洋枪打人,倪倒拨俚吓仔一跳,只怕外国人勿讲理性,瞎打一泡,打起倪来末,那哼弄法!勿壳张就是耐,耐啥格道理搭仔外国人两家头吵起来,阿好讲拨倪听听看?”秋谷听书玉说得夹七夹八的甚是可笑,不免约略和他说了一番。

正在还没有说完的时候,只见门帘起处,又走进一个倌人来。秋谷只道是陈文仙来了,正要叫他,却一眼看去似乎要比陈文仙长些,缩住了口没有叫出来,再聚起眼光仔细看他时:秋水丰神,远山眉黛;西子凌波之步,夜来红玉之香。好像有些认得,却又叫不出他的名字来。那倌人走到席间,先叫了沈仲思一声,又招呼了李子霄,然后回过头来,向章秋谷等微微一笑,就在沈仲思身旁坐下。秋谷见了,晓得就是沈仲思做的倌人,见他年纪也有二十四五岁的样儿,风头却还甚好,两只眼睛水汪汪的,射来射去甚归妖媚。秋谷暗暗的问张书玉,方晓得那倌人是兆富里的洪月娥。

当下书玉便请各人点菜,秋谷和修甫等随意点了几样。秋谷向修甫道:“文仙为什么这个时候还不见来?”修甫道:“或者有什么客人,耽搁住了也未可知。”

说着又等一会,陈文仙方走了进来。张书玉因是主人,立起来招呼了几句。陈文仙就坐在秋谷左边,张书玉先开口向陈文仙道:“刚刚耐阿晓得险格虐!”陈文仙并不晓得这件事儿,没头没脑的被张书玉这般一说,不觉呆了一呆,微笑答道:“啥格事体,倪勿晓得啘。”张书玉便把方才的事和他说了一遍,倒把个陈文仙吓得来香汗淋漓,花容失色,半晌方透过一口气来。章秋谷见陈文仙这般关切,不觉触起心事来,低头默默,如有所思。陈文仙定一定神,急忙回头过来问秋谷可曾被他打着,秋谷不觉哈哈笑道:“若是被他打着了,我还能好好的坐在这里么?你怎么说出痴话来了。”修甫等听了都觉好笑。陈文仙自己觉得岔了话头,面上一红,趁势拉着秋谷的手和他不依道:“耐格种人直头少有出见格,倪搭耐说格闲话,总归一句也勿肯听。别人家勿好阿关得耐啥事?要耐去嘤嘤喤喤瞎说一泡,几乎弄出性命交关格事体。区得耐运气还好,朆拨俚笃打着,倘忙一格勿当心,拨俚笃打仔一枪,耐阿犯着豁脱仔自家格性命,去拼格排杀千刀格强盗坯。”文仙说着又道:“格个辰光,耐来浪新马路打啥格流氓,阿记得倪劝仔耐几几化化格闲话,勿壳张耐一句也勿听,总归原是格付脾气,格末也叫真真无说法。”文仙说罢不觉烦恼起来,背过脸去佯佯不睬,秋谷和他说话,只是不理。秋谷没奈何,咬着陈文仙的耳朵说了几句,文仙故意嗔道:“晓得格哉,啥烦得来!”秋谷一笑,回过头来搭讪着和李子霄谈了一回,当下照例点菜叫局,自不必说。

吃到十点多钟方才散席,各人自到相好那边小坐,只有辛修甫不到西安坊,同着章秋谷到兆贵里去。到了院中,文仙先已回来,招呼坐下。文仙免不得又把章秋谷埋怨一回,秋谷只好笑而不辩。辛修甫向秋谷道:“今天这件事情,倒把我吓了一大跳,幸而文仙没有看见,不受虚惊。你没有见那当时的样儿,真正人也吓得坏的。”修甫说首,又向秋谷道:“我原晓得他们那班留学生,随便什么奇奇怪怪的事情没有一样做不出的,所以我暗中把你的衣裳拉了几回。你正是说得高兴,没有觉着,果然被他们听见,要和你拼起命来,你虽然没有被他打着,却也受了一个虚惊。究竟这样的人,正该把他送到捕房,问他一个凶器伤人的罪名,也好警戒警戒他的下次,怎么轻轻易易的竟是把他放走,可不便宜了他!”秋谷道:“你不晓得这当中的道理,我说出一个缘故来你就明白了。他们开枪打我,自然情理难容。我们就把他送到当官,也不算什么罗织。但是他们和我没有什么冤家,不过听我骂他们的说话骂得刻毒了些,一时气极了,不顾利害做出这样的事情。究竟我和他们不是什么不共戴天的仇恨,我既然没有受伤,放了他就是了,何必定要惊天动地的闹到当官,结这个无谓的冤家作甚?万一为了这事弄假成真,他们这一班留学生当真的结了团体和我做起对来,从来暗箭难防,明枪易躲,我虽然不怕他们,却也防备他不尽,不如还是放他去了的好。我想他人非草木,此后也不至于再来和我为难,你想我这话可是不是?”修甫听了恍然,不住的点头道是。

秋谷便对修甫说起打算就要回去的话,修甫也劝他不必久在上海,还是回去的好。文仙听了,急问秋谷道:“阿是耐说要转去?”秋谷点头,文仙又道:“格末倪搭耐讲格闲话,到底那哼!”秋谷微笑,朝他摇一摇头,文仙发急道:“耐格人啥格总是实梗。归格辰光,倪搭耐说格闲话,耐阿记得?故歇又是实梗搭倪格浆,倪定规勿成功。”说着,便柳眉颦蹙,杏眼含珠,着实的横了秋谷一个白眼。修甫在旁看了这个样儿,已经猜着了八九分的光景,只听得秋谷向陈文仙笑道:“你不晓得我的家事也有多少为难。第一,太夫人性情严厉;第二,我家计不过中资。如今若是趁了一时高兴,做了这件事情,将来万一有什么说话出来,我怎的对你得起?

到了那个时候,不是要好,反是害了你的终身,你也要自家想想。“章秋谷这几句说话原是真心,不料陈文仙听了眼圈儿一红,反止不住掉下泪来。停了一回方说道:”故歇倪也无啥说头,耐到陆里倪跟到陆里,随便耐叫倪那哼,倪总无啥勿肯。“

秋谷又笑道:“话虽如此,但是我晓得自家福薄,消受不起你这样的人,所以不敢答应。”文仙听了他这样话风,生起气来道:“照样耐实梗说法,是拿倪当仔坏人,恐怕将来要出啥格毛病,耐倒自家想想看,倪阿曾有啥格地方待错仔耐,无拨真心拨耐看仔出来,耐倒说拨倪听听看。”秋谷笑道:“实不相瞒,我自从十七岁上出来,纵情花柳,歌场酒阵,整整的阅历了五年,做了无数的倌人,攀了许多的相好,没一个不是密意缠绵,深情宛转,赌神罚咒的定要从良,到得后来,一个也没有成功。所以你虽然一片真心,我却不敢相信。”

陈文仙听了气得粉面通红,蛾眉斜竖,逼着问道:“耐既然实梗格念头,为啥倪问耐格辰光一口答应,阿是拿倪来浪弄白相,寻倪格开心?嘴里向说出来格闲话赛过放屁,耐自家想想阿对得起人?故歇倪只有一句闲话,耐答应末也是实梗,耐勿答应末也是实梗。阿有啥闲话说得明明白白,到仔故歇倒装起妈虎来哉,倪末白白里快活仔一泡,耐自家心浪阿有点意勿过?”秋谷听了自己回心一想,果然有些对不起他,但是要答应他却又有好些的为难之处,没奈何,只得附耳和陈文仙细细的说了一番,指望他回心转意。不料陈文仙听了,愈加动气起来道:“倪晓得自家格命苦,所以落到堂子里向做仔倌人,勿想嫁啥格大人老爷,过啥格好日脚,勿壳张碰着格客人,又是实梗样式。”说到此处便咽住了,说不出来,眼中珠泪一行行向下直挂。秋谷见了心上觉得可怜,想要劝慰他几句,不想陈文仙倒动了真气,娇喘微微,泪流满面。

秋谷正在无可如何之际,辛修甫坐在旁边呆呆的听着他们讲话,因为插不下口去,不便开言,见陈文仙气到这般模样,忍不住向秋谷道:“这件事儿却是你的不该,为什么既然答应了他,如今又要变卦?其实你们成就了这样好事,总算是一段美满姻缘,为何你一定不肯答应?”秋谷道:“不瞒你说,并不是我不肯答应,实在有为难的事情,不好向你们细说的。况且他们堂子里头的人,总是吃惯用惯,我不过一个中人之产,那里供给得来?你想他们做着倌人的时候,把多少客人的家财精力,通通用在一人身上,尚且横不愿意,竖不称心,讨不着他们的欢喜,不要说一个人的财力,那里填得满无底的深坑?你想这件事儿,我那敢冒冒失失的就答应他?”修甫道:“你的话虽然不错,我看陈文仙还不是这样的人,将来决不至于闹什么笑话,你只顾放心就是了。”秋谷听了正在踌躇,修甫忽然笑道:“我有一句话儿你可不要见怪,你这个人,在朋友面上极有义气,极有交情,若要讲到倌人面上的交情,却实在有些说不过去,委实的没有良心。”秋谷听了诧异起来,忙问:“你这话儿怎生说法?”陈文仙正在气得昏头搭脑的时候,忽听得修甫这样说法,也觉诧异,倒住了哭,呆呆的听他怎生说法。

只听得修甫笑道:“大凡一个客人做着一个倌人,虽然不要处处认真,上了倌人的圈套,却也不好过于诈伪,学那王莽的谦恭。从来男女居室,人之大欲存焉,天下的事情,惟有这样地方最是看得出一生的品行。若是一个人到了这等地方还是满口胡言,满身诈伪,没有一点真心,这个人的居心就不可问了。你想花丛柳阵的地方,粉黛笙歌的境界,最容易激发真心,你虽然是个个中老手,却不能太上忘情,不过阅历既深,有些强制的工夫罢了。却不晓得资格渐深,天良渐泯,做了一个倌人,无论那倌人和他怎生要好,总是随随便便的没有真心。我说句不怕你生气的话儿,像你这样的一个风流人物,又天天混在那脂粉丛中,绮罗队里,居然毫不动心,没有一丝儿迷惑。不是那元奸巨恶,和曹孟德一样的行为;就是个木偶刍灵,和晋惠帝一般的人物。我劝你还要诚实些儿,宁可做一个明知故犯的瘟生,不要学那些奸巧刁钻的行径,你的意思以为何如?”这一席话,竟把一个能言善辩的章秋谷骂得顿口无言,眼睁睁的看着修甫。看了半晌,忽然哈哈大笑道:“骂得好,骂得好!

我自从出世以来,没有个人把我骂得这般结实,你今天的几句说话却正搔着我的痒处,说到我心眼上来,真是佩服得狠。“修甫听了也笑起来道:”我不是有心骂你,不过是议论现在的嫖客罢了,你可不要多心。“秋谷笑道:”我也不是个怕骂的人,只要你骂得有理,就多骂几句何妨。“说着两人又笑了一会,陈文仙又向修甫诉说道:”辛大少,耐想想看,格号事体俚阿对倪得起?“修甫听了,又委曲劝解了陈文仙一番,却向秋谷说道:”我看文仙狠可娶得,你不妨答应了他,不要学那李益一般,做那负心男子。“正是:

水殿春风之影,镜里情郎;摩登软幛之图,中爱宠。

欲知后事如何,请听下回分解。





第七十二回 章秋谷名花成眷属 张书玉陌上遇萧郎





且说陈文仙对着辛修甫说道:“俚耐说倪勿是真心,倪格心只有自家晓得,勿好挖仔出来拨俚看看。故歇倪只有两句说话,无啥别样花头:第一勿要俚格洋钱,第二随便俚那哼分付。闲话说到仔实梗样式,俚耐还要说倪勿是真心末,听凭俚自家格良心好哉。辛大少,倪格事体瞒勿过耐。要讨倪转去格客人勿止一格,倪要无拨真心待俚末,老早嫁仔人哉,陆里等得到故歇!”修甫听了点头叹息,便又开导了秋谷一会。

秋谷此时见陈文仙果是真心,心上已有八九分懊悔,不该这样的回他,现在又被辛修甫劝了几句,自然顺水推船,一口应允。文仙见秋谷已经答应,方才眉锁重开,梨涡浅晕,收拾了一天烦恼,打叠起无限娇矛,喜孜孜的提起精神,应酬他们两个。秋谷便向修甫道:“这件事情我虽是已经应允,却还要回去一趟,和家内说明了委曲的情形,方能成就,现在却不能就这般草草的娶他。”文仙瞅了秋谷一眼道:“耐格闲话,有点妈妈虎虎,勿好算数。倪倒勿相信耐格枪花。”秋谷道:“这一回不比前番,有修甫在中间介绍,不是我们两个的事情。我若再要反悔,非且对不住你,并且对不住朋友了。”当下彼此商量一会,说明秋谷过了月半回去一趟,至多耽搁一月,再回上海来办陈文仙的事情,三面讲得明白。文仙恐怕章秋谷还要反悔,又问得着着实实的,估量着没有什么变动。好个陈文仙,当时叫了娘姨进来,和他说明嫁人,叫相帮去把牌子除下。娘姨呆了一回,虽不愿意,但陈文仙不欠他们的带挡,不好拦他,只得骨都着嘴,自去分付。

辛修甫见文仙做事这般剪绝,暗暗称赞。秋谷见他如此,自是欢喜。文仙又当场叫了本家上来,叫他把帐算清,房钱认他一节,因是节后不多几日,不过四十几台菜钱,算起来倒还不甚吃重,又叫秋谷和他去看房子,预备搬场,回报了一个娘姨,一个大姐,宝珠姐仍旧暂时服侍。文仙还有一个小大姐,也叫他一同过去。秋谷替他算了一算本家的帐,约着不到一千块钱,便打了一张一千块洋钱的票子,交与文仙,叫他开销一切,又另外赏了房间里一百块钱。文仙起初还不肯要,秋谷道:“你虽然不要我的身价,难道好倒反要你贴钱?况且我也不是这样的人,你不必这般客气。”文仙方才收了,章秋谷一连看了几天房子,在新马路租了一所两楼两底的洋房,把陈文仙搬了过去,自己也把吉升栈内的行李搬到新马路来,和陈文仙住在一起。正是:

花枝并蒂,春融秦女之箫;蛱蝶同心,月满温家之境。双星无恙,碧落团圆;三千天女之场,一枕风流之梦。脂香满满,未销宝鼎之烟;人面田田,占尽柔乡之福。

章秋谷这边的事按下不提。如今且把李子霄、沈仲思的来历补叙一番。

看官且住,在下这部小说,原名叫做《九尾龟》,又叫作《四大金刚外传》,如今做到五集,差不多就要结束全书,不得不把他们的事实再细细的补叙一回。那四大金刚里头,陆兰芬已经死了,金小宝暂时收场,不做生意,却和贡春树住在一处。林黛玉住在惠秀里内,算个住家,有向来相熟的客人,也可过去坐坐,他自己却竟是销声匿影的不大出来。只有张书玉仍旧住在新清和坊,艳帜高张,香名愈噪,真是枇杷花下,车马如云。每天牵算起来,总有五六场和,十余台酒,那生意比先前好了几倍。书玉得意扬扬,十分高兴。

有一天,书玉坐着轿子在一品香出局回来,轿子走到大新街口,忽然迎面撞过一个客人,正在四马路走过,轿子走得甚快,那客人也低着个头直撞过来,恰恰的撞了一个照面,轿夫避让不及,彼此一碰,把那客人仰面朝天的跌了一交。那客人在地下扒了起来,心中大怒,一把扭住了轿夫的衣服,喝道:“你走路不带眼睛的么?乱撞你娘的什么?”轿夫见那客人衣服都丽,气概出众,却也不敢得罪他,况且委实把他撞了一交,只得陪着笑面,说声:“对不住,实在没有看见。”那客人那里肯放,要叫巡捕到来,把轿夫带到捕房里去。张书玉坐在轿中,一眼看见那客人的手上带着三个金刚钻戒指,晶宝夺目,光彩照人,身上穿着一身外国缎子的衣服,颜色配搭得甚是匀称,更兼仪表轩昂,身材俊伟,生得倒还不俗。看了他这般气派,晓得定是个有钱的阔客,便有心要笼络着他,对他嫣然一笑道:“大少对勿住,总是轿夫勿好,碰仔耐一交筋头,勿得知身浪向阿曾碰痛?”说罢星眸低漾,杏脸微红,含羞带笑的瞧了那客人一眼。这一个眼风,就把那客人的身体酥了半边。

动弹不得,本来是一腔怒气不肯干休,被张书玉这样一来,不知不觉的把心上的焦躁,一霎时销化个干干净净,连忙放了轿夫,笑嘻嘻的答道:“不妨不妨,没有什么要紧。”那眼睛却紧紧的钉着张书玉看个不住。张书玉见了,晓得他已经入彀,又微微一笑道:“晏歇点阿到倪搭去坐歇?倪来浪新清和第三家。”那客人听了大喜道:“狠好狠好,停回儿我一定过去。”书玉笑道:“晏歇点要来格哩!”那客人连连答应,轿夫放开脚步径自前行。临走的时候,书玉还欠起身来回头一笑,略略的朝他点点头儿,一直回新清和去了。

那客人见张书玉径自去了,只觉得晃晃荡荡的好像神魂还没有归窍一般,虽然想起没有问他的名字,到清和坊那里去寻,便急急的那边一看,见张书玉的轿子,影影绰绰的还在前边,连忙三脚两步赶上前去,把轿后的龟奴一把扯住。轿夫倒吃了一惊,问他为什么这般样子。那客人便问他倌人的姓名,轿夫见他气喘吁吁的甚觉好笑??便替他说了。书玉坐在轿中听见,把跟局的娘姨金珠叫了过来道:“倪先坐仔轿子转去,耐同仔格位大少慢慢交来。”金珠答应一声,那客人更是欢喜,同着金珠在马路上慢慢的走着,一头夹七夹八的扳谈。

大新街口到新清和坊本来不多几步路儿,不一刻已经到了。金珠在前引路,那客人跟在后边,上了扶梯,已见张书玉换了一身衣服,笑迷迷的立在楼门口道:“倪晓得耐就要过来,倪等仔耐一歇哉。”那客人到了此时,神魂飘荡,觉得身体虚飘飘的,好似在云雾中的一般。张书玉拉着他进了大房间,亲手替他宽了马褂,推他坐下,方才问他的姓名。你道这客人是谁?原来就是那李子霄。当下敬过瓜子,书玉着实的敷衍了他一番,当夜就摆了一个双台,闹到三更多天方才散席。

自此一连几天,李子霄夜夜碰和,朝朝摆酒,闹得烟雾尘天。在李子霄的意思,原想要转张书玉的念头,无奈张书玉虽是待他要好,晚间却总不留他,李子霄也不好意思开口。论起这李子霄的为人来,却也甚是精明,随便什么世故人情一概瞒他不过,就是在嫖界里头也着实的有些资格,不比那一班土头土脑的瘟生。但是有一桩毛病不好,见了倌人,一个个都是好的,并且一见了面,就想要转他的念头。虽然狠肯花几个钱,却自家打家主意,不肯落他们的圈套,所以有些倌人都要嫁他,他却咬定了牙齿不肯答应。不料一见了张书玉的面,就由不得神魂颠倒起来。那四大金刚的手段名不虚传,他不想你的念头则已,想了你的念头,却总要比他人来得辣些。这几天,张书玉放出全身本事,把一个李子霄哄得一心一意都在张书玉的身上。张书玉却又拿定主意,不肯叫他轻易近身,故意打情骂俏的做出那一种亲热的样子,弄得李子霄这又不好,那又不好,好似热锅上的蚂蚁一般,团团乱转。

有一天,李子霄在票号里头刚刚起身,还未梳洗。张书玉要笼络李子霄的心,起了一个大早,打扮得花枝招展,丰态娇娆,带了一个娘姨,坐了轿子竟到李子霄票号里来。其时刚敲十二点钟,由李子霄的家人引进房内,笑盈盈的叫了一声。李子霄见了喜出望外,连忙叫他坐下,只听得张书玉道:“李大人,耐啥格刚刚起来,阿是昨日仔辛苦哉?”李子霄听了一呆道:“我除了碰和吃酒,没有别的事情,我有什么辛苦?”张书玉掩口笑道:“勿是呀,作兴耐昨日仔到仔相好搭去住夜,辛苦仔点,所以今朝起来得晏哉,耐自家照镜子看哩!”说着又低声问道:“李大人阿对?”李子霄听了笑道:“你这说话甚是奇怪,我昨日若真个住在相好院中,现在这个时候怎么就得回来?况且我在上海除了你,那里还有什么相好?你倒说说我听。”书玉面上一红道:“倪末陆俚有格号福气?”说着就溜了李子霄一眼,李子霄见了满心欢喜,一面洗脸,一面和张书玉天南地北的扳谈。书玉又见李子霄的头发蓬了,便问他要出梳具来,要自己和他梳头。李子霄打着苏白答道:“阿唷,书玉先生实梗格红倌人搭倪来打辫子,格是勿敢当格啘。”书玉听了,对着那个姨娘道:“耐听听看,说得阿要好听。”又向李子霄道:“李大人耐勿要实梗客气,故歇倪搭耐打条辫子,耐就要搭倪客气,晏歇点……”张书玉说到此间,粉颊低垂,含羞微笑的说不下去。李子霄逼着问道:“你怎么说话只说半句?说下去。”张书玉又嫣然一笑,接下去道:“也客气勿尽啘。”李子霄听了这两句话儿,真是乐不可支,满心奇痒。当下张书玉和李子霄打了一条辫子,李子霄又留他在票号里头吃饭,书玉一口应允,并不推辞。

李子霄也是个老于此道的人,晓得倌人有时看望客人,不肯在客人那边吃饭,一定要客人在那倌人面上有了非常资格,方才做得到这般田地。张书玉看待李子霄虽然要好,却还只是那表面上的交情,并没有什么密切的关系,今天居然破格赏光,肯在李子霄那里吃起饭来,也算得是李子霄特别的场面了。当时李子霄叫当差的去关照一声厨房,说有客人吃饭,叫他们另添几样菜来。当差的去不多时,已经开进饭来。本来是六碗饭菜,如今有了客人,添了四个热炒,四只荤盆,另外又是一壶绍酒。李子霄便让张书玉坐下,竟是两人对酌起来,那菜虽是不多几样,却做得甚是精致。张书玉竟不客气,吃了几杯酒,又吃了一碗饭。因李子霄酒量颇好,书玉亲自与他斟酒,直至完了一壶方才吃饭。当差的舀上一盆水来,娘姨拿出带来的镜匣放在桌上,书玉对着镜子略略的添些脂粉,又揩了一把面;回头过来,见李子霄恰好吃完了饭正要洗面,书玉便亲手绞了一把手巾,走过去和李子霄并肩一坐,一手搭着他的肩头,一手拿着手巾和他揩了一把。李子霄只闻得一阵剩粉残脂的香气在那手巾上直透出来。正是:

碧城十二,相思六曲之屏;金粉三千,云雨前身之梦。

欲知后事如何,且听下回分解。





第七十三回 李子霄销魂春照夜 沈剥皮拼命死贪财





且说李子霄闻得一阵香气直钻入鼻孔里来,觉得今天张书玉陪他吃一顿饭竟是破格的事情,心上十分高兴。张书玉又向他笑道:“倪生意末做仔好几年,从来朆到客人搭吃歇过饭。今朝耐李大人说仔,倪勿好勿答应,晏歇点说起来,总说是倪坍仔耐李大人格台,换仔别人留倪吃饭,倪阿肯答应?”李子霄听了更是欢喜。张书玉和他说说笑笑,甚是投机。直到傍晚时分,张书玉竟是坐着不走。李子霄暗觉诧异,问他可有什么话说。书玉佯嗔道:“阿是无拨事体,倪勿好来格。”正在还要说下去的时候,早见书玉的相帮走了进来,手中拿着一搭局票递与娘姨,又说了一遍,无非是姓张的叫到聚丰园,姓李的叫到金谷春,要叫书玉早些回去。书玉故意皱着眉头道:“啥要紧呀,耐转去说。”转过来又回头向李子霄道:“格排客人末叫讨气,叫啥格断命堂差!”倪难得今朝一日天,搭耐讲讲闲话,心浪倒蛮快活,刚刚俚笃又来叫啥格堂差,勿得知啥格道理,看见仔俚笃格付架形,就觉着心浪勿舒齐。说来说去,倪格碗堂子饭直头勿要吃哉,赛过勿是自家格身体,真真作孽。

“

李子霄倒解劝了书玉一番。停了一回,书玉并不想走,院中接连来了两个相帮,说叫局的催过了两回,又有两起客人坐在房内等他回去。书玉听了把头一别道:“哈格希奇勿煞,要唔笃实梗发极,一转两转吵勿清爽,阿怕倪勿晓得。”相帮听了不敢开口,倒是李子霄看了不过意,便对书玉道:“你院中既有客人,又要出局,我看你还是回去应酬客人,不必在此间耽搁,不要回来脱了局,得罪了客人,要是闹些闲话出来,叫我心上怎么过意得去?”书玉听李子霄叫他回去,斜了他一个白眼,嗔道:“耐倒好格!阿是来浪讨厌倪,赶倪转去?倪好心来看看耐,耐倒是实梗样式,耐格人阿有良心?老实说,格号客人,倪本来勿高兴做,脱仔局也无啥希奇。比方耐李大人叫倪格局,倪阿好勿来?像俚笃格排客人,倪生来勿去应酬,高兴末多来来,勿高兴少来来,倪也勿见得靠仔格挡码子绷啥格场面,李大人,耐说阿是?”李子霄见张书玉这般要好,不好再说什么,口内虽是这般说法,叫他不要得罪客人,心上却自是欢喜。

张书玉直坐到上灯以后,约有九点多钟,院中的相帮一连来了几趟叫他回去。

书玉装出无奈的样子,又向李子霄叮嘱了无数的话,叫他今晚一定要来,李子霄自然答应。张书玉方才一步一回头的坐了轿子走了。子霄又到别处去了一转回来,便直到书玉院中,当夜又摆了一个双台,请的客人,便是那沈仲思首座。

原来这沈仲思本来是杭州人氏,寄籍虞山,他父亲名叫沈近园,足足的二三百万产业,不要说是别的,就是常熟城内的田,竟被姓沈的占去十分之二,你想可利害不利害?这沈近园生了七个儿子,那五个都是少年夭折,只存了沈仲思兄弟二人。

沈仲思还有一个兄弟,排行最小,名叫沈幼吾,因他排在第七人,都管着他叫沈老七。但是沈近园虽是个头等富家,生性却十分吝啬,真是一毛不拔,算尽锱铢。你要和他商议别件事儿,他总没有什么不肯,若要和他商议到银钱上去,这却杀了他的头他也不肯拿出一个钱来。他又有一件毛病,不肯把银子放到庄上去生利钱,只说:“这些钱庄都靠不住,他要是把我的银子拐在家里,自己却一溜烟跑了,我可到什么地方去找他去呢?”所以情愿把银子放在家里,再也不拿出来。在家里另外起造了一间房子,四边都是铁打的窗棂,只有一扇小门出入,这间房子专为存放银钱,除了他自己一个人,余外的任是什么人儿也不放进这间密室。他放钱的法儿却又与众不同,也不是用保险钱箱,也不是用太平银柜,你道他怎生的放法?说也奇怪,他把那历积蓄的洋钱一封一封的排在地下,又怕没有数目,自己年纪大了记不上来,他又想了一个法儿,把一万块钱堆作一排,整整的堆了数十余排,他却对人说道:“我若不是这般排法,万一有贼进来,偷了三百五百,一千八百,我那里查考得出?像这样的一万洋钱一排,那做贼的任是再有通天本事,也拿不动这一万洋钱。”人家听了都笑他是个痴子,他也不以为意。

沈近园虽然吝啬,家中倒有好几房的小老婆,头上插的,手上带的,都是金器,身上穿的,却又都是布草衣裙。有些好事的人问他道:“你家里那几个如夫人,为什么插带的都是金器,穿的却又都是布衣?你既是舍不得钱给他们穿着,怎么又肯花钱打造首饰呢?”他却回答得好,说:“你们晓得什么?我的算盘真是精益求精,你们那里想得这步田地?你想金银首饰带在他们头上身上,就是隔了十年二十年,也还是这般轻重,没有什么吃亏。那绸缎衣服花了许多的钱做来着在身上,着了一年半载,最多的也不过三年五年,着得稀碎破旧的,一个大钱也不值,岂不是白白的赔钱?”那问的人听他这般说法,不觉哈哈大笑,佩服他的算计真是精明,出来对别人说了。从此就送了他一个外号叫做“沈剥皮”。

这沈剥皮虽然啬刻,他的那两个儿子却是著名的洋盘,在外边结识了一班篾片,一天到夜的各处乱闯乱跑,大把的银子捧出来,就像水一般的往外直淌。但是沈剥皮的家教极严,等闲不许他儿子走出大门一步。这两个宝贝只是背着沈剥皮,在外面打架闹事,无所不为,沈剥皮犹如醉在梦里一般,那里查察得着。但有一样,沈剥皮的银钱都是自家经手,这两个儿子摸不着他一个大钱。他们又想出一个主意,兄弟两个大伙儿商量,偷偷的叫了铜匠配了银房的钥匙,候着晚间,沈剥皮睡了,开了房门进去,偷了一个饱。又为偷得少了,恐怕被沈剥皮查了出来,索性一偷就是一排。偷了一万块钱出来,兄弟二人大家分用。这沈剥皮虽然算计精明,却只晓得要钱,别的事情都有些糊里糊涂的。他以为把历年积蓄的银钱放在这间密室里头,四边又是铁打的窗棂,就着生了翅膀,扁着身子,也不用打算进去,心上道是千妥万当的了,就是进去安放洋钱的时候,也不去查点数目,就是这样糊糊涂涂的过去。

这兄弟二人偷了一万洋钱出来,用完了便再进去偷,一连偷了好几回,见沈剥皮并不查点,越发放大了胆,索性多偷几排,挥霍一个畅快。

又偷了几次,沈剥皮渐渐的有些疑心起来,对他两个儿子说道:“怎么我的洋钱,只有一排一排的堆上去,不见他一排一排的长出来,老是这个样儿,可是个什么缘故呢?”他儿子听了吃了一惊,连忙遮掩道:“你老人家不要多疑多虑,那里有这样的事情,难道我们这样的高房大屋还有什么贼人进来么?”沈剥皮听了,想想儿子的说话不错,也就罢了。

沈幼吾又嫌家里的住房不好,在自己对门买了一块大大的地基,造起一座洋房,又怕被沈剥皮晓得了是不得了的,便叫一个手下的篾片捏一个假名,径到沈剥皮家中拜会。见了沈剥皮,只说是苏州人氏,为的常熟地方甚好,所以买块地基起些房屋,算他是别业一般,现在工程将要落成,特来拜拜邻舍。沈剥皮听了甚是相信,反恭恭敬敬的送了他出去。隔了几天,沈剥皮穿得衣冠齐楚的过来回拜,恰恰的沈幼吾坐在中堂,高谈阔论的和那一班清客讲话。抬起头来,看见沈剥皮穿靴戴帽的走进中堂,只把他吓得屁滚尿流,一溜烟从后门逃了出去,却叫一个家人出来挡驾。

沈剥皮还心中有气,说他瞧不起人。

沈剥皮一天到晚只是呆呆的坐在家中,除了吃饭睡觉之外,便是盘算银钱,别的事情一件也不在他心上。早晨不到天亮就要起来,晚间刚刚天黑就叫关了大门大家睡觉。临睡的时候,还要自己到各处门口细细的查看一回,又亲手把一重重的门通通锁得结实,方才放心。到了晚上不许家人们点灯睡觉,他明说是小心火烛,其实却是节省灯油。大约沈剥皮的家里,从正月初一到十二月三十,也用不了一斤灯油。沈剥皮这样的小心防范,算得是顶真的了。谁知他那两位贤郎候他睡了,拿出身边预备的钥匙把一重重门上的锁一齐开了出去,直到三更四更方才回来,悄悄的仍旧把门锁好,一些也看不出来,沈剥皮那里晓得?

有一回,沈剥皮打发儿子沈仲思到上海的一爿什么当店里头盘查帐目,顺便查查别处的什么钱庄、绸缎店的出入。沈剥皮以为他生出来的儿子一定也和他自己一般,所以竟是放心大胆的叫他前去。不想这沈仲思在常熟的时候虽是荒唐,不免总有些儿忌惮,恐怕沈剥皮晓得风声不是顽的;现在到了上海,真是海阔从鱼跃,天空任鸟飞,那里还有什么顾忌?更兼上海这个地方是花天酒地的擅场,纸醉金迷的世界。沈仲思到了上海,便是拼命的狂嫖,不管三七二十一,嫖得昏天黑地,一塌糊涂,竟把好好的两处钱庄,一处绸缎号,一处洋货号,轻轻易易的盘给别人,顿时手头有了四五十万银子,越发的不想回去,只在上海地方昏昏沉沉的度日。沈剥皮连连的写信到来催他回去,他也置之不理。

不知怎的这件事情漏了风声,竟被沈剥皮晓得,只气得怒发冲冠,浑身乱抖,气到极处圆睁两眼,一句话都说不出来,一口气接不上,竟是一个鹞子翻身,跌在地下晕了过去。家人们慌了,连忙去寻了沈幼吾回来,请了两三个医生开方施救,直到半夜方才渐渐的醒转,吐出一口浊痰,慢慢的说出话来。还是气得咬牙切齿的,想要亲自赶到上海去和他儿子拼命。无奈刚刚晕了过去,人的元气未复,手脚瘫软,一动也动不来,无可奈何,只得罢了。却因儿子不肖,败了他的家财,恨入骨髓,预备了一条极粗的麻绳,要等沈仲思回来,用绳把他勒死,只恨的自己一时不能全愈,活动不来,发狠说:“养好了病,定要亲到上海找他,这样的儿子还不如死了的干净。”

照这样的说起来,沈仲思的一条性命,竟有些岌岌可危。幸而沈仲思的妻子在家,听了沈剥皮的说话,到底事不关心,关心则乱,不由的心惊胆战起来,急急的写了一封信,寄到上海和沈仲思说知缘故,叫他千万不可回来。沈仲思得了这个信息,大吃一惊,晓得沈剥皮的脾气,别样事儿还好将就得过,惟有用了他的银钱,却是不共戴天的仇恨。他说得出来,却就做得出来,这件事儿竟没有个挽回的方法,想来想去想不出一个计较来,只急得咳声叹气,抓耳搔腮。就有一个篾片教他主意,叫他发信回家,装得自家病重,要叫家里一个人来。到得家人来了,竟用一口空棺装些砖头石块充作死人,停在公所,让那家里的来人把棺材搬回家去。自己却有了银钱在手,没有什么做不得的事情,尽顾租了房子,长长久久的住在上海,一则免了家中拘束,二则躲了这场是非,岂不是绝妙的一个主意?沈仲思听了这个主意,心中大喜,连赞:“好个妙计,他们那里想得出来?”当下果然就如法炮制的打了一个电报回去,假说自家病重,要叫他夫人赶紧前来,一面安排了一口空棺停在会倌里头,什么灵牌孝幔,一齐预备停当。这叫做“装龙像龙,装虎像虎”,免得别人看见样儿不像,要起疑心。

那边沈剥皮接着了病重的电报,非但并不吃惊,反说:“这样的不肖子孙留他何用,让他死了也罢。”沈仲思的夫人听了,倒大大的吃了一惊,连忙收拾收拾,要到上海去看仲思的病。正是:

瞒天造谎,犹留鸿爪之前;同室操戈,岂有天伦之义。

欲知后事,请听下回分解。





第七十四回 假病危瞒天造谎 打官司教士分家





且说沈仲思假装病重,打了一封电报回去,他夫人那里晓得这个信是假的,认真的着急起来,收拾些随身衣服,便要到上海去。本来要想邀沈幼吾一同前去,路上好有些招呼,谁知沈剥皮深恨仲思,不许幼吾同去,只得罢了。当下沈仲思的夫人雇了一只快船,一路凄凄惶惶的赶到上海,偏偏又遇着了顶头逆风,足足的走了三天方才到了。

船刚到岸,沈仲思夫人心急如箭,连忙打发了一个家人上去问信,自己随后上岸,也不坐轿子,只坐了一部东洋车赶上岸来。不料那家人赶到沈仲思的寓处一问,他们是预先计划好的,一见有人来问仲思的信,仲思便自己躲了起来,叫人回复道:“沈某人已经死了两天,灵柩都停到公所去了,你还来问的什么信儿?”原来沈仲思恐怕他兄弟同来,被他撞见,所以分付手下的人这般说法,想不到他兄弟不来,来的倒是他夫人一个。当下那来的家人听了不觉大惊,连忙拔起脚来,飞一般奔回原路。恰恰的在半路上遇见了少夫人的车子,只见他满头大汗,气喘喘吁吁的极声喊道:“少奶奶,不好了,少爷已经故世了两天,连棺材都停在浙江会馆去了。”

仲思的夫人听了,好似那高楼失足,大海沉舟,一霎时万箭穿心,却一句话都说不出,只觉得哄的一声,三魂七魄一齐飞出顶门,飘飘荡荡的不知散归何处,几乎跌下车来。幸而跟来的一个娘婧有些见识,便向家人说道:“既然事已如此,也不必再到寓所去了,还是一直径到浙江会馆停灵的地方去了再说。”家人听了点头称是,便叫车夫掉过车头,回去浙江会馆。此时沈仲思的夫人坐在车上就似木雕泥塑一般,那眼中的珠泪一片汪洋往下乱滚。在马路上又不好放声大哭,恨不一步就跨到浙江会馆来。

不一刻,到了门前停下,沈仲思的夫人三脚两步走了进去,问明了停灵柩的地方,扶着妨姨的肩头,一路哭着直抢进去。只见一间灵室,高高的挂着孝幔,供着灵牌,两枝白蜡辉煌,一段香烟缭绕。沈仲思的夫人见了这般光景,止不住一阵心酸,号淘大哭,直抢进灵帏里面,抱着灵柩哭得死去活来,泪干声尽。这里沈仲思的夫人正在呼天抢地,痛不欲生的时候,忽地灵的帏一起,走进一个人来。旁边的娘姨反起头来一看,这一惊非同小可,直吓得魂飞天外,魄散九霄,一交跌在地上,色色的抖个不住,那喉咙口好像塞了一个棉团,要叫喊也叫喊不出。这来的人竟走到他夫人身畔,拍着他的肩头道:“不要哭了,这棺材是个假的,我好好的现在这里,一些也没有什么。你且住了哭,定一定神再和你说。”沈仲思的夫人正哭得发昏,忽听得有人和他说话,好像自己丈夫的声音,急忙勉强忍住了哭,抬头一看不觉也吃了一惊。

你道来人是谁?原来就是沈仲思。他本来派了两个手下的人在停灵地方照看香烛,又晓得家内有人到来,恐怕露了破绽,连忙叫一个人到浙江会馆去打听消息。

到得那里,听见他夫人在那里号啕痛哭,甚是伤心,晓得叉了话头,却又不好上前去劝,只得急急的回去报知。沈仲思听了连连顿足道:“坏了,坏了,都是我自己粗心,这里那里说起?”连忙的跳上马车赶到会馆,早听见他夫人在里面哭得伤心,打动了沈仲思的心肠,就也落了几点眼泪,大踏步走进孝幔,也不及说什么别的,只好先劝住了他的哭再作计较。

他夫人抬头见了不免也是一惊,忽然一个念头赶上来,把沈仲思拦腰抱住,哭道:“我和你十余年的夫妇,你就是死了我也不怕。我活在世上也没有什么味儿,你快些同了我去。”一面说,一面哭,倒把个沈仲思牵动情肠,十分感激,由不得也吊下泪来,连忙安慰他道:“你不要这样的伤心,我实在并没有死。”就把自己有心装死,躲过这场是非的话和他夫人说了一遍。他夫人还不肯相信,沈仲思又重新把前事说了一番。他夫人又呆呆的想,想了多时,见沈仲思说话有声,行步有影,方才相信他真没有死。定了一定神,向沈仲思道:“我这身体觉得虚飘飘的一些也没有着落,到底今天的事情是真是假,不要是我在这里做梦么?”沈仲思笑道:“青天白日,好好的人,那里做什么梦?你放定了心,不要疑疑惑惑的。”他夫人听得这般说法,方得明白,却痛定思痛,喜极生悲,又觉又哭起来,沈仲思连忙劝住了,他夫人免不得要把沈仲思埋怨一番。沈仲思低头谢过,一同走出孝堂。娘姨在地上听了,方才扒起身来,跟着二人一同出去。见了沈仲思,还是做眉做眼的有些害怕。

那知走到中间,刚刚常熟来的两个家人也撞了进来,正和沈仲思撞了一个劈面。

两个家人一见沈仲思在内走出,只认白日显魂,吓得个冷汗浑身,毫毛直竖。一个胆小的家人大叫一声,跌倒在地。一个胆大些的回过头去,撒腿便跑。沈仲思甚是好笑,正要叫他,恰好跟着沈仲思来的家人也走进来,拦住了他说明原委,方把他同了回来。又把地下的那一个也扶起来和他说了。那两个家人立在一旁,兀是有些心惊胆战。沈仲思便同了他的夫人回到寓处,住了一夜。大家商议停妥,沈仲思叫他的夫人假装穿孝,扶了灵柩回去,好瞒住那沈剥皮。他夫人起初不肯,沈仲思再三央恳,只得勉勉强强的应允了。沈仲思又和他夫人说明,回去之后再想法子接他出来。他夫人当真搬了一具空柩,回到常熟。沈剥皮那里晓得,并不伤心,只说:“这样没出息的东西,死了还是家门之幸。”沈幼吾本来和沈仲思兄弟不合,也不把这件事儿放在心上。沈仲思的夫人又分付了带去的家人仆妇不许乱说,果然一些破绽也看不出来。

谁知隔了多时,终久事机不密,被沈幼吾看了些儿毛病出来,便暗暗的盘问家人,被他问得个明明白白,便写一封信去给沈仲思,说他不应诈死骗人,干得好事。

又吓唬他哥哥道:“这件事儿虽是父亲没有晓得,究竟不该瞒他,回来万一晓得了风声,连我也担当不起,若要我替你遮瞒这事,每年须要津贴一万洋钱,总算你自己买条活命。”这封信到了上海,沈仲思见于又惊又气。想了一会,竟没有什么法儿,只得忍气吞声,依了他兄弟的话,每年孝敬他一万洋钱,差不多就像纳贡一般,不敢推扳一点。

直至后来沈剥皮死了,沈仲思方敢回来,要和他兄弟分家,不想沈幼吾又起了个独吞家产的念头,竟是咬定牙齿一些不认,说:“我哥哥已经死了几年,如今葬都葬了,这是大家晓得的,那里又跑出一个哥哥来,要分什么家产,这不是有心图赖么?”沈仲思听了他兄弟这般说法,心中大怒,便请了许多的公亲族长,来商议这件分家的事儿。有几个无耻的亲簇,受了沈幼吾的贿赂,便帮着沈幼吾说话;有几个公正些的,只好两边劝解,无奈沈幼吾咬定牙齿坚不承认,只说他当初怎样的荒唐,沈剥皮要用绳子把他勒死,他着了急,方才想出这一个装死的法子来,如今却又要承受遗产,那里有这样的事儿?又向沈仲思道:“你开口闭口总说一样的儿子,为什么承受不得遗产。你可晓得父亲存日,早巳不把你当作儿子,你如何还要想来顶受家财?比如人家的儿子已经贴了革条,革出祠堂,难道也好承受产为么?”

议论了一天,也议不出个道理。沈仲思气极,便往常熟县告了一状。那知批出来仍是亲族理处。兄弟两个一连争闹了几天,究竟田房产来都在沈幼吾的手中,沈仲思思竟闹他不过,没奈何回到上海和人计较。

又有一个人和他出主意,叫他拜在一个天主教士的名下,要请他出来帮忙,说明分家之后,把所有的家财产业,提出二成捐入教会。那教土听了大喜,果然同了沈仲思径到常熟,先到县里拜了县官,和他说了,要他秉公审断。那知县大老爷见是外国人的事情,那敢违拗,诺诺连声的答应,立时立刻的出了一张传票,传沈仲思兄弟二人到案,沈幼吾,听得有外国人帮着他哥哥出头打官司,登时就吓矬了一尺,要请几个亲族出来做个见证。那些亲族听见说有外国人在内,谁敢多事?一个个缩着头颈死也不肯去。沈幼吾没奈何,只得硬着胆子自己到案。县大老爷着实训斥了他几句,叫他听断具结,把父遗财产兄弟均分。沈幼吾不敢不听,只得当堂具下结来,兄弟二人一齐退出。此时沈仲思得意扬扬,沈幼吾垂头丧气,到了家中,邀齐亲族,把所有的现钱产业分作两分,兄弟二人各得一分。

沈仲思得了这些财产,便在上海买了一处房子,把家眷接在一起,竟不想回到常熟去了。果然把那财产提出二成来,也有十多万银子,送与教土,一齐捐入教堂。

算起来他们兄弟分家,只便宜了一个教士,轻轻易易的几句话儿,就卖了十数万银子,这叫做“鹜蚌相争,渔翁得利”。

看官试想,天下只有儿子死了,旁人瞒着他的父母不叫晓得。那有儿子现在好端端的活着,却瞒着父母说是死了的道理?这可是一件绝妙的新闻,更可笑的是沈仲思怕他兄弟在父亲面前漏了风声,每年孝敬他兄弟一万洋钱,买他个不开口。从古以来,只有将钱买命,那有花了银钱自家装死的道理?这样的笑话不要说是自家眼见,就是听也不曾听过,可算得少见多怪。无偶独有的了。

闲话休提,书归正传。只说沈仲思叫了个兆富里的洪月娥,一到台上便咬着沈仲思的耳朵,唧唧哝哝的讲个不住。李子霄晓得洪月娥和沈仲思是有交情的,看见他们台面上这般要好,不觉心上有些热刺刺的起来。张书玉坐在背后把李子霄的衣服一扯,李子霄回头过来,书玉低声笑道:“耐看俚笃两家头恩得来!”李子霄微笑不语,一会儿看看洪月娥,一会儿又看看张书玉,书玉低问:“看啥?”李子霄不答,只是呆呆的看。书玉伸手过来拧了他一把,背过脸去,却慢慢的回转秋波,偷看李子霄的脸面。不防李子霄也在那里看他,恰恰的四目偷窥,两心相印,书玉不觉低鬟一笑,脉脉含情,李子霄趁此也咬着书玉的耳朵说了无数的话。书玉只是含笑摇头,李子霄怃然若失,又见洪月娥和沈仲思恩爱缠绵,一直坐着不走,等到将要散席,逼着他一同回去。沈仲思还有些迟迟疑疑的,月娥一定不肯,把自己的轿子让与沈仲思坐了,自己坐了东洋车回去。

李子霄见了甚是艳羡,忽然的眉头一皱,计上心来,一连喝了几大杯酒,装作大醉的样儿,伏在桌上,连客人要走,他也装作不知,只是沉沉的打睡。只听得张书玉走近身畔叫了几声,李子霄不应,书玉低低的向姨娘们说道:“李大人吃醉哉,搀俚到大床浪去靠歇罢。”就有一个娘姨帮着书玉,把李子霄搀到床上,轻轻的放他睡下,又叫娘姨们小心伺候,自己到别处房间应酬客人去了。李子霄在大床上假装睡着,等得好不心烦,直等到十二点钟,书玉方才进来。一进房门,便问:“李大人阿曾困醒?”娘姨答道:“一径朆醒歇。”书玉轻轻的移步床上来,把手摸一摸李子霄的额角,又附耳叫了他两声,李子霄只是不应。书玉坐在床沿,低声向娘姨说道:“格个李大人勿知那起风来,阿要喊应仔俚,问声俚看?”说着,便软绵绵也睡到床上来,又叫了他几声,李子霄听得张书玉对着娘姨这般说法,心上甚是感激着他。张书玉叫了两声,便装作刚刚睡醒的样子,开眼问道:“有什么时候?”

书玉道:“一点钟也敲过哉。啥格耐一困就困到仔故歇,阿是有啥勿舒齐?”一面说着,一面把一双儿罗绵的纤手在他背上轻轻的挺了几下,又对他说出一番话来。

正是:

玉软香温之夜,此福难销;金迷纸醉之天,深情如许。

欲知后事如何,且听下回分解。





第七十五回 撩云拨雨夜渡银河 辣手狠心朝施毒计





且说张书玉对李子霄说道:“耐刚刚啥格吃仔两杯酒,就吃醉哉。倪摸摸耐头浪,像煞有点发热,难下转勿要去瞎吃瞎吃,倘忙吃出仔点毛病,总是耐自家格身体吃亏。耐故歇一干仔来浪上海,夷无拨啥自家格亲人,有仔毛病,阿有啥人好来替耐,倪是白白里替耐发极,也无拨啥格用场。耐下转阿好当心点,勿要拿仔自家格身体弄白相,耐想倪格闲活阿对?”李子霄听了满心快活,一时说不出来,暗想:“我做的倌人也不知多少了,恰都是虚情假意的一些儿没有真心,我却也从来没有上过他们的当。如今看这张书玉的样儿,实是和我真心要好,不是那虚情假意的人,但是我几次转他的念头,他终是糊里糊涂的含糊答应,不肯爽爽快快的应承,不晓得他是什么意见。今天且待我再结结实实的问他,看他怎生回来。若是他再有什么推三阻四,我也不必再在这里花这冤枉的银钱,决计撇下了他再寻别个。”想罢,便低声向张书玉道:“你的说话自然不差,但不瞒你说,我多吃几杯酒儿倒还没有什么,实是吃了你的空心汤团,所以心上觉得有些不快。”书玉听了“嗤”的一笑,道:“耐格人啥实梗呀,闲话勿说勿明,倪搭耐说明白仔,耐就晓得哉。倪人末做仔倌人,本底子也是好人家格囡仵。倪娘拿倪卖出来,吃仔格碗堂子饭,也叫无说法;再加仔倪格抚蓄娘格末叫利害,勿知吃尽仔几化苦头。”书玉说到此间,顿时眼圈儿一红,声音就低了好些,一对秋波含着一眶眼泪。

李子霄见他说得好好的,忽然好像要哭出来,心上十分痛惜,连忙用手帕和他拭去泪痕,又款款轻轻的安慰一番。张书玉方才接下去说道:“故歇总算赎仔身体出来,自家做生意。耐想倪好好里格人家人,吃到仔格碗断命饭,阿要作孽?再有格排一厢情愿格客人,总说倪摆啥格架子,勿肯巴结客人,俚笃说起来,倒说倪既然挂仔招牌,做格行生意,勿管俚是啥人,只要有仔铜钱,大家才好进来,摆啥格时髦倌人格架子。轧实勿瞒耐说,倪十六岁出来做生意,故歇念三岁,做仔七八年格生意,有过歇相好格客人直头勿多几格。一节里向,一塌刮仔留仔两三格客人。

老实说,格排客人才勿勒倪心浪。客人见仔千千万万,总规无拨对劲格人。故歇碰着仔耐,勿知啥格道理,心心念念,放耐勿落,耐一日天勿来,像煞倪心浪掉脱仔啥格物事,横来竖去总归一格勿舒齐,倪格辰光见仔别格客人,一向朆有歇实梗样式,格当中啥格道理,连倪自家也解说勿出。想起来,要末是倪两家头前世有点缘分。“说着,就看着李子霄低头微笑,那眉稍眼角露出两朵红云,升起十分春色,星眸曼视,粉颈低垂,说不尽那许多的情态。

张书玉做作了一会,又道:“故歇耐翻转来倒说倪拨空心汤团耐吃。倪怕耐淘坏仔自家身体,所以勿肯……”张书玉说了半句,那半句却咽住了,没有说出来。

李子霄故意问道:“不肯什么?为什么说了半句就不说了。”书玉掩口一笑,把李子霄打了一下,却口中低低的咕哝道:“耐自家一声勿响,倒说吃仔倪格空心汤团,叫倪那哼好……”书玉说了这半句,又不说了。李子霄明晓得张书玉的意思已经许了他的特别利权,心中大喜,便乘势两手合抱拢来,把书玉搂入怀内。书玉半推半就的听他轻薄了一回,推开李子霄的手,坐起来向他说道:“耐刚刚吃醉仔酒,空心饿肚,身体陆里吃得消?倪搭耐炖好莲心桂圆来浪,阿要吃仔点勒困。”李子霄此时,正觉得肚皮有些咕噜噜的作响,正用得着,便点一点头。

书玉自己跨下床来,取了一只白磁盖碗,亲手把莲子壶里炖好的莲心盛了一碗,又取一个银匙,送到李子霄口边。不用李子霄坳手,一匙一匙的和他送进口中,李子霄觉得这一碗莲子的滋味十分甜美,好像生平从没有吃过这样好吃的东西。李子霄直把这碗莲子吃完,果然觉得精神抖擞。张书玉问他可还要吃些,李子霄摇一摇头说:“不要了。”又劝书玉自己吃些。书玉也吃了几匙,娘姨送上手巾,李子霄抹了一把,原来是预先留着的开水。当下张书玉含羞带笑的,向李子霄说道:“难末勿要紧哉。”当夜张书玉就留李子霄住下。真是:

天上人间,花香月满。洞口桃花之浪,潭水清深;高唐神女之云,鸳攀梦稳。

李子霄住了一夜,自然是恩情美满,云雨迷离,给了四十块钱住夜下脚,这且不在话下。

且说张书玉自与李子霄落了相好,心上想要借着他淴一个浴,便向李子霄说生意做得怕了,想要嫁人,隐隐约约的露出些要嫁他的意思。怎奈李子霄虽然见色心迷,却毕竟是个花丛老手,有些见识,晓得上海的时髦倌人,不是可以娶得回去的人。拿定了主意,凭着张书玉怎生说法,他总不去兜揽。书玉竟弄得无可如何,暗想我这个金钟罩,随便什么一等利害的客人也跳不出我的圈套,怎么这姓李的竟是这般棘手?想了一会,被他想了一个极恶毒的主意出来,你道是什么恶毒主意?下文再表,如今且不必说明。

只说李子霄在张书玉院中一连住了几夜,忽然有一天早上起身,吃了一碗莲子,觉得腹内大大的不受用,翻肠搅肚的响了一阵,竟是狠狠的泄泻起来。一刻儿的工夫就泻了有五六次,泻得他头晕眼花,四肢无力,倒在床上只是喘气。把一个张书玉吓得手脚慌忙,泪流满面,蛾眉锁恨,杏脸凝愁的向李子霄说道:“倪实梗格劝耐,叫耐保重点自家格身体,耐总归勿肯,故歇真格有仔毛病哉,叫倪那哼弄法!

耐到底身浪向啥格勿舒齐,阿要请格郎中先生来看看?“李子霄见书玉两眼红红的含着一包珠泪,心上暗暗的赞他,却有气无力的说道:”今天早上起来好好的,不知什么原故,一时间腹内绞痛起来,一连泻了几次,觉得头痛耳鸣,心头霍乱,睡了一会略略的好些,你且不要心慌,或者将息一天两天好了也未可知,医生且慢些去请。上海地方也找不出什么名医,药不对症,反要被他吃坏。“书玉听了点头称是,却只是愁眉不展,坐在床沿目不转睛的看着李子霄,好像真是十分关切,就是自家夫妇也没有他这样的留心。停了一会,李子霄腹中又痛起来,皱着眉头连叫几声”阿呀“,想要挣起身来到床后去用便桶,不想泻了几次,浑身无力,再也挣不起身。书玉连忙用手相扶,一步一步的挨到床后,又是连泻了四五遍。李子霄有些来不得了,只见他冷汗直流,面皮雪白,两脚虚飘飘的好似在云雾中一般,勉强扶着书玉的肩头蹭到床上一头睡下。

李子霄睡了一回,定一定神,睁开双眼,自觉着这个病儿有些不妥,便对张书玉道:“我这个病来势不轻,恐怕一时不得就好,你还是叫相帮们打乘轿子把我送回寓处,好待我安心调理几天。”书玉听了那里肯放,道:“耐故歇病得实梗样式,阿好坐啥轿子,就是转去仔,耐又无拨家眷来浪上海,一塌刮仔几个当差,啥人肯搭耐当心,好好里服侍耐?倪想起来,还是勿要转去格好,来浪倪搭住仔两日,养好好病再说。老实说,倪搭人手多点,包耐服侍得勿差。”李子霄听了,又想了一会道:“虽然如此,但是你们堂子里头比不得别处,我怎好占住了你们的房间?况且我是个病人,你又有别处的应酬,我住在这里,那里有回去的便当,再要累你这般服侍,我自家心上也觉得不安。”书玉见李子霄这样说法,便紧紧搀着他的手道:“耐故歇有毛病,勿要再去转啥格念头,倪搭仔耐两家头勿比别人,再有啥格客气?就是占仔倪一间房间,也勿算啥格事体。勿瞒耐说,倪看见仔耐生病,心浪几乎急杀快,再有啥格心想做格个断命生意!耐豪燥点自家养病,随便啥格事体勿要放勒心浪。耐想倪一生一世,总算做着仔耐实梗一个中意客人,正来浪要好头浪,夹忙头里耐咦生起病来哉,叫倪阿要发极勿要发极?”说着就背过脸去,用手帕拭那泪痕,又道:“再加仔耐勿肯住来倪搭,定规要想转去,叫倪陆里放心得落?”

说罢又流下泪来。

李子霄见他这般关切,倒是十分感激着他。说话之间,李子霄又起来泻了两次,竟是支持不住起来,合着眼睛喘个不住。慢慢的定了一回,方才睁开两眼。见张书玉半扒半坐的坐在床上,正呆呆的看着他,眼中的珠泪一滴一滴的下来,直淌到李子霄面上。李子霄见了心中欢喜,就觉得精神好些,上气不接下气的和书玉说道:“你不必这般着急,我这会儿觉得略略的好些。”书玉和他脸对脸的含泪说道:“倪明朝吃仔净素,替耐到大马路虹庙里向去烧烧香,求服仙方转来,等耐吃仔试试看,保佑耐毛病好仔,倪再去替耐还愿。”又叫娘姨下去招呼相帮们一声,道:“耐下去关照俚笃一声,有局票来叫局末,说倪到仔苏州去哉,勿管啥格客人,勿要让俚进来,等李大人毛病好仔再说。”娘姨答应自去。李子霄在床上听见,觉得有些过意不去,便道:“你何苦为我一个客人,得罪他们这一班熟客?我看你还是出去应酬,留个娘姨在这里招呼招呼就是了。”书玉皱眉道:“耐勿晓得倪格心浪赛过来浪打结,看仔耐生病,替耐勿落,咦无拨啥格法子好想,格个心浪,格末叫难过,陆里再有实梗高兴去做啥格生意。”李子霄听了,格外的意服心输,死心塌地。

张书玉果然这一天不做生意,把浑身的手段都使出来,用在李子霄一人身上。

一天到晚竟是坐在房中,动也不动,连饭也不肯吃,只随意吃了些儿稀饭,只是愁眉泪眼的坐在床上看着李子霄。到了晚间,更是衣不解带的殷勤服侍。李子霄着实的过意不去,叫他略睡片时,他那里肯睡?

一直坐到天明,便叫醒了娘姨,要早些梳洗到虹庙去烧香,当下梳好云鬟,只带一支押发,别的插带一些没有,穿了一身素服,觉得有些缟袂临风飘飘欲仙的丰态。此时李子霄睡了一夜,已经觉得好些,只腹中似乎还有些儿隐隐的绞痛,却比昨天好得多了。张书玉打扮好了,回身走到床前,携着李子霄的手道:“倪搭耐去烧香,求仔仙方来,保佑耐吃仔就仔,耐定心仔靠一歇,倪去仔就来。”说罢便飘然去了。

李子霄躺在床上,等人心焦,足足的等了两点多钟的时候,书玉方才回来,手内拿着一纸仙方给李子霄看。李子霄看了这个仙方,见是三钱薏米、三钱冰糖,开水煎送,明知是吃不好、吃不坏的药品,见张书玉郑重其事的设着香案,恭恭敬敬的煎起来,又指着自己的裙裤给李子霄看道:“耐看倪格衣裳浪,跪仔两格影子,倪刚刚来浪庙里向,足足里替耐跪仔一点钟辰光。”李子霄听了,留心看他的膝盖,只见两边中衣上,果然沾了两个碗口大的灰尘影。此时的李子霄心上,已经二十四分服贴,没有一些疑心,看着张书玉把药煎好倒在碗内,凉了一凉,又亲自试了一试,方才送到李子霄口边。李子霄闻得一阵糯米香,觉得甚是开胃,便一口气吃了下去,甜津津的也没有什么别的味儿。说也奇怪,这一碗仙方吃下肚去十分受用,登时的头目清凉,连声音都响亮了许多,竟慢慢的走下床来,勉强扶着书玉走了几步,仍复回身坐下,书玉又炖好了燕窝粥给李子霄吃了一碗,精神更觉好些。这一天,到晚竟没有泻。张书玉欢喜非常,合着两手向空拜了几拜,道:“阿弥陀佛,谢天谢地,总算好哉,几乎拿倪急杀快。”

又一连过了几天,李子霄已经好了,张书玉又到虹庙去了一趟,算是和他还愿。

正是:

相如善病,惊回倩女之魂;小玉多情,疗得檀奴之渴。

欲知后事如何,且听下回分解。





第七十六回 假温柔瘟生中计 真淴浴名妓私奔





且说李子霄在张书玉院中一连病了几天,张书玉服侍得十分周到,真是要长便长,要短便短,千依百顺的奉承得李子霄好不欢喜。李子霄本来原没有什么毛病,不知怎样突出其来的泄泻起来,接连泻了十几遍,就也着实的有些支撑不住,却又不知怎的,吃了张书玉在虹庙求来的一服仙方,就是这样容容易易的好了。来也来得神速,去也去得稀奇,连李子霄自己也不晓得是什么道理,只当是偶然受了风寒,腹中作怪。见张书玉这样的殷勤服侍,着急非常,好像恨不得自己替他的样儿,更兼趁着夜深人静没有人在面前的时候,把李子霄灌了无数迷汤,说了许多好话。真是:

宛转枕屏之上,海誓山盟;缠绵五夜之情,怜声倚影。

直把个李子霄骗得心花怒开,看着书玉就是天下第一个好人,再没有第二个人赶得上他的了。心上这般一想,便觉得李子霄般般多好,色色俱佳;乱头粗服随处增妍,浅笔轻颦无时不媚。再加张书玉到了晚间总是目不交睫,打起精神,彻夜伺候,凭着李子霄怎样的叫他安息,他只是不肯,反向李子霄说道:“耐格病故歇总算好点,真真还是倪格运气,倪故歇来浪服侍耐,心浪倒蛮快活,辛苦点无啥希奇。

耐一定要叫倪去困,丢仔耐一干仔来浪,倪倒有点勿放心。耐故歇自家格身体还朆复元,勿要来管倪格事体,养好仔身体再说。“这几句说话,就是那春蚕自缚的情丝,大海钓鳌的香饵,把李子霄的心钩得牢牢结实,那里还撒手得开,果然心中快活,病也好得快些。

李子霄病好之后,心中暗想张书玉待我这般要好,服侍得这般殷勤,自己家中正少这样一个贴身伏伺的人,决计打算要娶他回去。料想他这般相爱,将来不至于闹什么笑话出来。想定了主意,便和书玉说知,问他可肯嫁人,要多少身价,可有什么债项。张书玉见李子霄果然中了计策,甚是欢喜。暗想这个主意使得真是不差,凭你李子霄这般的主意坚牢,也跳不出我的圈子,还要乖乖的自己送上门来。

看官,你道张书玉使的什么计策,就把李子霄骗到这般?原来张书玉在上海滩上专爱姘那一班不要脸的马夫、戏子,情愿倒贴银钱,只要马夫、戏子姘上了他,向他开口,他就大把的洋钱钞票拿出来供给他们的挥霍,左右用的是那些曲辫子客人不心痛的银钱,那里放在心上?就是刚刚遇着他没有钱的时候,也要千方百计的敲了客人们的竹杠,拿来送给他们。近来张书玉姘了两个戏子,拿着整千整百的洋钱倒贴,贴到后来为数大了,客人们也渐渐的晓得风声,一个个绝脚不去。书玉的用度又大,收敛不来,一节下来竟欠了五千多些的债,张书玉不免也有些着急起来。

不期事有凑巧,刚刚做着了李子霄,晓得他是个虞山富户,在倌人身上花费一万八千、三千五千银子不算什么,便有心要大大的敲他一下竹杠。

倌人们要敲客人的大注竹杠,除了说要嫁他,更无别法。那知李子霄虽然是个富翁,在堂子里头也着实的有些阅历,任凭张书玉怎生打动,他却只是一口咬定,不放一点儿口风,张书玉急了,便想了一个极恶毒的主意出来。你想李子霄好好的可有什么毛病?他却忍心害理的买了些巴豆夹和在莲子里头,一同煎好,大着胆子给李子霄吃了。果然就一霎时大泻起来,书玉趁着李子霄生病,做出那一心关切、着急万分的样子。到得隔了一天,书玉到虹庙去烧香,求了仙方回来。他那里真去求什么仙方,只在虹庙里头问香火要了一张吃不坏的仙方回来,装了恭恭敬敬的样儿把仙方煎好,却暗暗把糯米饮搀在里头,这糯米饮是专解巴豆毒的,所以李子霄吃了,居??一天好似一天。他又不惜工本,殷殷勤勤的服侍了他几天,把李子霄骗得伏伏贴贴,那里想得到他做出这般恶毒的事情?看官,你想倌人们的心思可刻毒不刻毒?

当下张书玉听得李子霄问他,心中暗喜,却又故意沉吟了一回,方才说道:“耐李大人格闲话,倪阿好勿答应?不过倪有一句闲话,故歇搭耐说明白仔,勿要等两日大家心浪勿高兴。”李子霄听了倒觉一呆,急问他有什么话说,书玉却正正经经的说道:“耐要讨倪转去,格是倪想也想勿到格事体,陆里再有啥格勿肯?不过唔笃格排男人才是无拨良心格多,倪人末做仔倌人,倒是老老实实格脾气,比勿得格排时髦倌人,今朝接仔姓张,明朝再接姓李,无啥希奇。再说起唔笃客人来,加二讨气,一个勿高兴,扳仔倪点差头,就要跳槽,说起来总是倪做倌人格勿好。

耐勿要故歇一时辰光高兴头上说得蛮好,拿倪讨好转去,歇格一年两年勿高兴哉,丢脱仔倪再要去讨别人,格是倪勿成功格虐,耐去想虐,唔笃男人讨仔一格再讨一格无啥要紧,像倪嫁仔人阿好再要出来?“

李子霄听了,越发觉得张书玉身分比别人不同,更是一心一意的要娶他回去。

便托了一个朋友出来做媒,一切讲得明明白白。身价共是八千,先付一半,张书玉欢天喜地的一口允许。李子霄便在大马路赁了一处公馆,三楼三底的洋房,甚是齐整,拣了一个吉日,清音彩轿的把张书玉娶进门来。李子霄的一班朋友,也有送髦儿戏的,也有送酒席的,说不尽的筵开玳瑁,镜掩芙蓉,炉焚百和之香,春照双星之影。整整的闹了三天,方才安静。

张书玉自从嫁了过来,一心一意的装出人家人的样儿,没有一些不高兴的神气,在李子霄面上更是事事尽心,般般周到,李子霄冷眼看他,心中甚喜。有时倒是李子霄恐怕书玉在家气闷,要同他出去看看戏,或是坐坐马车,书玉反不肯天天出去,只对着李子霄道:“故歇倪嫁拨仔耐,总算是人家人,比勿得做倌人格辰光,总归还是少出去格好,”李子霄听了更是放心,但终久怕他不惯,勉强拉他出去散心。

书玉嫁了李子霄半月有余,一共只出去了两次。

这一天李子霄没有应酬,便坐在家中和书玉说说笑笑,甚是开心,觉得另有一种趣味。李子霄和张书玉商量道:“我到了下月想要回去一趟,不知你可肯跟我回去?你若是心中不愿,就住在上海也好,我在常熟、上海两边走走却也无妨。”书玉含笑答道:“倪靠仔耐格福气,嫁拨仔耐,总算无啥,故歇耐要转去末,倪自然跟耐转去,倪既然嫁耐,就算是耐格人,嫁鸡跟鸡,嫁狗跟狗,阿有啥耐转去仔。

倪一干仔住来浪上海,也无拨格号道理啘。“李子霄听了心中暗喜,又道:”不是这般说法,你若是跟我回去,我家内却现有正妻,况且我家老太太的规矩甚严,恐怕你回去了过不来这般拘束的日子,所以我要和你商量一声。“书玉笑道:”耐格闲话倒直头来得稀奇,勿知说到仔啥格地方去哉,倪既然嫁拨仔耐,早晏点总要转去,阿有啥一直勿转去格道理?就是唔笃老太太凶点,倪只要规规矩矩,无拨啥格坏处,勿见得老太太有心来寻倪格事,倘忙有点啥格闲话出来,倪总归打定主意,骂仔勿开口,打仔勿动手,也才完结哉啘。“李子霄大喜道:”原来你竟有这般见识,真算是观德无双,但是要你回去,这般的陪着小心,我终久有些过意不去。“

书玉笑着,把头一扭道:“倪搭耐两家头再有啥格客气?只要耐将来勿要有仔别人,忘记脱仔倪好哉。

自此李子霄和张书玉又加了几分爱情,心上十分相信书玉是天下有一无二的好人,把自己的要紧物件、钞票、银洋、帐簿、珠宝,都交与张书玉收管。书玉起先还假意推辞不肯,李子霄再三的叫代收管,方才一一的收了下来,细细的查点了一番。李子霄因在客边,没有什么重大的物件,却还差不多有两万多的光景。张书玉心中暗喜。李子霄住在上海,打算一月满月,便同着书玉一同回去,不想平空的闹了一桩笑话来。

这一天晚上李子霄出去应酬,回来得迟了些儿,约有十二点钟的光景。走到房内,见书玉不在房中,并连书玉贴身伏侍、在堂子里带过来的两个娘姨大姐也都一个不见。李子霄见了,这一惊非同小可,晓得事情不妙,中了张书玉的苦肉计儿。

一时又惊又气,大声叫喊当差的上来,问他姨太太那里去了。当差的回说:“老爷刚刚出去不多时,姨太太说心中气闷,要到丹桂去看戏,套了马车,带了两个娘姨一同前去,叫家人等散戏场的时候套车去接。现在李升已经去了,家人因家内人少,所以没有同去,此刻差不多戏场已散,想来也好回来了。”李子霄听了,明知不妥,只得自宽自解,想书玉或者是真去看戏也未可知。又问家人:“为什么姨太太要一人出去,你们不来报我一声?”当差的回道:“平日间老爷尚且信他,家人们怎敢拦阻?”李子霄听了顿口无言。

等了一会,竟是石沉大海,那有什么人影儿回来?李子霄暴跳如雷,急叫当差的再到戏园去看,自己一面开了铁箱查点物件。巧巧的不见了张书玉的一张婚书,三千多洋钱的钞票,还有些翡翠玉器珠子也不见了,约摸着也值六七千银子,连自己帽子上的一个玻璃绿翎管也带了去。再开书玉的衣橱箱子看时,只有一只首饰匣被他带去,其余的衣服,整整齐齐一件不少。只把一个李子霄气得就如死人一般,坐在床上,两眼睁睁的看着保险灯一言不发。暗想:“我自从二十多岁在花柳场中混了十年,从没有上过倌人的这般恶当,不想如今上了张书玉的一场大当,把我好像三岁的孩子一般,由着他性儿撮弄。这本来是我自家不好,他们做倌人的那有什么良心,我却着了他的道儿,把他娶了回来。如今只叫作人财两空,还落了一肚皮的腌臜闷气。想起这堂子里头顽耍,真真的没有什么味儿!”

想了一会,忽然又想起当初的一场病来得甚是蹊跷,不要是张书玉在饮食里头和了什么泻药,所以一时间拼命的大泻起来。他却假做出那一付关切的样儿,好叫我看了他这般要好,感激他的深情。那时吃了他的迷汤,真把一个张书玉轻怜痛惜,百种温存,感激他尚且来不及,那里想得到这步田地。想来想去,越想越是不差。

又想:“那张书玉竟是下得这般辣手,幸而我自家本元还好,不至于弄出性命之忧。

倘换了一个身体虚弱的人,那里禁得起他一服泻药?就是这般容容易易的一条性命送在他的手中,却向何处去伸冤理枉?“越想越恨,恨得他咬牙切齿,恨不得立刻把张书玉拿来打死。

正在无可奈何,只听得楼梯上一阵脚步声音,当差的已经回来,和那先去的李升一同走了进来,神色张皇,满头流汗,失张落智的回道:“家人们到丹桂门口候了多时,又到厢楼各处去寻了一遍,不见姨太太的影儿,现在戏场已经散了多时,家人们只得回来,请老爷的示下。”李子霄呆呆的半晌,长叹一声,又听家人还叫他是姨太太,不觉怒气直冲,一声喝住道:“还叫什么姨太太,都是你们这班混帐东西不肯留心,闹出这样的事,你们还有脸来见我么?”两个当差的不敢开口,诺诺连声的垂着手侍立一旁。李子霄又想了一会,方向当差的道:“我开一张失单出来,你们立刻去报捕房,叫他派个包打听来,明天我再去拜上海县,存一个案,再想追缉的法儿。”当差的答应了一声,李子霄就在保险灯下草草的开了一张失单出来,约莫已有一万开外,正要交给家人拿去,忽又转念想道:“这样的事情,就是报了捕房查缉出来,也没有什么好看;若是查缉不着,岂不是白白的坏了名声?”

这样的一想,便有些踌躇不决起来,向当差的道:“今天已有两点多钟,捕房里头就明天去报也好。你们明天早上赶紧到沈仲思沈大人那里,说我有事和他商议,请他立刻过来。沈大人在上海住了多年,料想一定有个主意。”当差的又连连的应了几声是,见李子霄没有什么话说,便退了下去。李子霄见时候不早,只得走到大床上,和衣略睡片时。正是:

一夜高唐之梦,神女成虹;十年杜牧之狂,青楼薄幸。

不知后事如何,且听下回分解。





第七十七回 楼空燕子神女成虹 帘卷西风檀郎懊恼





且说李子霄因张书玉忽然不见,懊恼万分,要等明天请了沈仲思来,和他商议一个办法。看看表上已是指到三点钟,只得就在床上和衣少睡。那知睡到床上,翻来复去,眼睁睁的再也睡不着。往日间是好梦易醒,春宵苦短。金钗暗堕,香融被底之春;玉体横陈,软试怀中之玉。如今张书玉走了,只剩了李子霄一个人住在楼上,冷冷清清的,鸳鸯瓦冷,翡悴衾寒;宝鸭不温,银釭无焰。辜负高唐之梦,商妇归来;凄凉锦瑟之歌,玉人何处?这一种的孤凄情况,李子霄那里销受得来?心上边万转千回的,就如蜘蛛结网,膏火自煎,不知怎样的才好。张着两眼,看着那一盏孤灯摇摇不定,更觉得窗外远远的一阵一阵的风声,夹着些秋虫的声响,玻璃窗上好像有隐隐的一股凉气,直透到床上来。李子霄暗觉诧异道:“往日间书玉没有逃走的时候,只觉得睡到床上,一会儿天就明了,从来没有这样的孤凄,真是那俗语说的‘欢娱夜短,寂寞更长’了。”一直躺在床上,直到四点多钟还没有睡着。

渐渐的窗上透进微微的亮光来,好容易盼到天色大明,李子霄方有些朦朦胧胧有睡着。正在神魂颠倒的时候,猛然又听得晓鸟“呀”的一声,便霍然惊醒,开眼一看,窗上已经有了日光,便也懒懒的起来洗面。当差的上来伺候,李子霄问:“沈大人可曾去请?”当差的回:“已经去了。”李子霄便眼巴巴的等着沈仲思来,好告诉他这件事儿。

那知李子霄这边张书玉夜间逃走,出了这件事情。沈仲思也在洪月娥那边受了他的骗局。这两个人,一个是李子霄的欢喜冤家,一个是沈仲思的风流孽障。你道沈仲思怎样受了洪月娥的骗局?在下做书的一枝笔儿提不得两家的事,只好撇了李子霄这边的事,先把沈仲思的事一一的演说出来。闲话休提,书归正传。

只说沈仲思做了洪月娥,彼此十分要好。洪月娥因为沈仲思是个狠肯花钱的人,面子上不能不巴结,其实还是把他当作瘟生,沈仲思那里晓得。恰恰的到了礼拜那一天,沈仲思要同洪月娥去坐马车,洪月娥虽然口中答应,却不肯和沈仲思坐在一车,便向沈仲思掉了一个枪花道:“倪今朝有点头里痛,坐仔皮篷马车只怕勿局,耐另外叫一部轿车阿好?”沈仲思听了,心上自然有些不快,便赌气说道:“你不去也没有什么,我就一个人去也好。”洪月娥见沈仲思动了气,便把口风翻了过来,连忙分辩道:“啥人说勿去呀?耐格闲话,倪阿曾勿听过歇?今朝耐勿要倪去,倪倒定规要跟牢仔耐一淘去,省得耐来浪瞎三话四,说倪勿肯。”沈仲思听了,回嗔作喜的道:“你不过怕和我坐在一车,有人说你做了我的恩客,这怕什么,你就是做了恩客,只要那客人不要你们倒贴,这也算不得什么。老实说,你若把我当作客人,我们便坐在一处同去;若要把我当作瘟生,你也不必客气,竟是我自己一个人去。”洪月娥听了着急起来,赶过去拉了他的手道:“耐格闲话倒来得调皮笃啘!

倪几时当耐瘟生,耐倒说拨倪听听看。“沈仲思笑道:”你既然没有把我当作瘟生,为什么不肯和我坐在一起子“洪月娥被他驳住了,没有话说,眉头一皱计上心来,向沈仲思道:”勿瞒耐说,倪勿肯同耐坐勒一淘,也有一格讲究,格辰光一排做倪格客人,才要倪同仔俚笃一淘坐马车,倪心浪勿高兴,回报仔俚两转,说倪从来朆搭仔客人一淘坐歇马车,格挡码子勿肯相信,搭倪反仔一泡,实梗格故歇有啥格客人叫倪坐马车,倪总归回报俚笃勿去。今朝耐沈大人搭倪说仔,倪勿好勿答应,不过倪想起来,勿要拨俚笃看见仔瞎三话四放倪格谣言,倪堂子里向名气要紧,耐沈大人是蛮明白格人,阿有啥勿晓得倪格苦。“说着就蹙着双眉,做出那一付幽怨可怜的样子。

沈仲思听了,想一想倒也不差,忽又问道:“你既然有这一层缘故,为什么不早些和我说明?却定要藏头露尾的说什么头痛,可见你们的说话,真真的有些儿不老成。”洪月娥听了,一时回答不出来,顿了一顿方转口说道:“勿是呀,倪说仔真话,怕耐沈大人要生气,耐高高兴兴要倪一淘去坐马车,倘忙为仔倪勿去洛,光火起来,阿是无啥趣势,叫倪心浪也过意勿落啘。”好个洪月娥,一时间就捏出这许多说话,把沈仲思先前的盛气不知说到那里去了。当下沈仲思听他说得婉转可听,又十分的情义动人,反连连的点头称是。洪月娥见沈仲思已经被他说动,反撒娇撒痴的和他不依道:“倪倒是一片格真心,耐再要说倪无拨真闲话,耐自家去想想看,耐来浪倪搭做仔一节光景,阿曾有啥洛里一句闲话勿替耐说,耐末再要当倪坏人,说起来真真讨气。”说着便滚在沈仲思怀中,口内咕噜道:“倪勿来格,耐下转阿要实梗?”沈仲思被他一阵胡闹,心上也有些浑淘淘起来,觉得自家好像真有些对他不起,倒安慰了洪月娥一番,月娥方才顺水推船的罢了。沈仲思听了洪月娥的说话,果然多雇了一部马车,沈仲思自己独坐一车,洪月娥带着一个大姐同坐一车。

到了张园下车,进去泡了一碗茶,也有些认得沈仲思的朋友,彼此招呼。坐了一会,又到四马路去兜了一转,便也回来了。这一夜沈仲思自然住在洪月娥院内,不消说起。

看官须要晓得这边的沈仲思,这几天夜拥名花,销尽温柔之福。那边的李子霄,便也是这几天春融金屋,新成鹣鲽之盟。沈仲思见了李子霄的请酒帖子,方才晓得这件事儿,又是羡慕,又是眼热,便鼓起兴来约了许多朋友,大家出个公分,足足的在李子霄的新公馆里头热闹了三天。沈仲思天天被他们灌得大醉,过了一天还觉得头目之中森然作晕。却为见了李子霄把张书玉娶到家中,玉暖香温,一双两好,更兼那一天晚上的情景,真是艳锦裁云,新绫织凤,画屏无睡,银烛摇红,把个沈仲思在旁看了,由不得自家心上也跃跃欲试起来。暗想他娶得张书玉,难道我就娶不得洪月娥?便把这个意思和洪月娥商量。

须知洪月娥的巴结沈仲思,全是巴结他的钱,并不是看中他的人品,那些面子上的应酬本来原是假的,在洪月娥心上原不把沈仲思放在眼中。无奈月娥虽是自家身体,房间里娘姨的带挡洋钱却欠到三千开外。娘姨有了带挡,自然倌人面上也作得来几分主意。从前沈仲思初做月娥的时候,月娥不肯留他,房间里娘姨为着生意起见,勉强着月娥把他留下。月娥又说不出一定要做恩客的这一句话,被娘姨们逼住了,只得委委屈屈的留下了他,娘姨们见沈仲思狠肯花钱,大家都二十四分的巴结。洪月娥面子上也只好敷衍着他,不敢得罪。其实月娥心上没有一点真心。现在见沈仲思自家开口说要娶他,月娥心上自然不愿,却心中暗想道:他既自这般说法,我不妨应许了他,叫他和我将这些娘姨的带挡一概还清,省得他们有了些儿带挡,便要碍手碍脚的混出主意。只要把带挡还了,以后的事便好想个法儿,再作脱身之计,料想姓沈的决计防不到这一层。想定主意,便一口应允,并向沈仲思道:“倪吃格碗把势饭,也叫无说法,只要耐肯讨倪转去,是再好无拨格事体啘,阿有啥倪倒勿肯格道理?轧实搭耐说仔,倪刚刚做耐格辰光,就转格条念头,只怕倪末一心一意看中仔耐,耐倒看倪勿中,翻转面孔来说声勿要,倪阿有啥格趣势?唔笃做客人格要讨倌人,倌人勿肯倒无啥希奇;倪做仔倌人挨拨客人,客人勿要,耐想倪阿坍得落格个台?”沈仲思听了更是欢喜,便叫了房间里人上来,细细的和他们说了。

一班娘姨听得洪月娥竟肯嫁他,觉得诧异,都有些支支吾吾的不肯答应,一个个都看着月娥,听他怎样。月娥暗暗的和他们递了一个眼风,方才一口应许,并不作难。

沈仲思大喜,也不用别人打话,竟是和着洪月娥等三面言明。月娥口口声声不要沈仲思的身价,只要替他还清了债务,就好跟他回去。沈仲思问他一共有多少债?

月娥说:“有六千洋钱。”其实月娥身上只有三千多债,衣裳首饰差不多也值四五千,沈仲思那里晓得?当下讲得明明白白,还债六千,开销二千,说明叫沈仲思先付六千,还有二千等轿子到门再付,沈仲思一一答应。洪月娥欢欢喜喜的叮嘱沈仲思道:“故歇倪两家头格事体总算说定格哉。依仔倪心浪,巴勿得明朝就跟耐转去,省得倪勿做仔生意住来浪该搭地方,拨别人家说起来好像无啥好听。耐豪燥点去看好仔房子,等倪早点过去,也算完结仔一桩事体。”沈仲思本来性急,又被洪月娥这般一说,便急如星火的先托人去看好了房子,瞒着家里的人悄悄的在外边布置。

不几天,已经布置得十分妥贴,又看了一个吉期,便先打了一张六千洋钱的即期庄票,亲手交与洪月娥。还算沈仲思有些见识,付了定洋,要问洪月娥取张婚据,洪月娥故作猛然省悟道:“勿错勿错,格样物事倒是要紧格。”说着,又想一想道:“故歇倪搭无拨人来里,只好明朝写好仔再交拨耐,勿然末,就是耐搭倪写仔一张也无啥。”沈仲思笑道:“别的东西我都可以代写,独有这个婚书,却一定要你们这边的人写的,我怎的好代你写起婚书来?”洪月娥笑道:“倪是才勿懂格,洛里晓得格当中有实梗格几花讲究。要末耐只好明朝来拿仔罢,勿得知耐阿放心勿放心末?刚刚格张票子耐原带仔转去。”沈仲思道:“你真是说笑话了,我自从做你以来,直到如今却差不多也有两个月的光景,何曾有什么不信你的地方?不要说这一张票子。”洪月娥听了,也便收了。沈仲思梦里也想不到洪月娥要骗他的六千银子,心上还在那里打算,到了那一天怎样的风光,如何的热闹。正是:准备着银屏金屋,销受他楚雨巫云;星娥七宝之妆,神女洛川之佩。这沈仲思的高兴,是不言可知的了。

那知隔了一天,沈仲思又到洪月娥院中,要问他取那一张婚据。走到洪月娥房内,见情形不好,先就吃了一惊。只见房内坐着一个少年男子,月娥的本家坐在旁边,正在那里不知说些什么,却不见洪月娥的影儿。房间里也撇得乱七八糟的不像了样儿,连台上摆的自鸣钟和台花都不见了。沈仲思看了这般模样,心上晓得不好,只得怀着鬼胎,举步进房。本家见沈仲思进来,立起身叫了一声:“沈大人来得刚好,格件事体勿关得倪啥事。倪开仔堂子,洛得担得起格号风火?”沈仲思听了本家的说话,真是夹七夹八的一句也不懂。便先问一声:“月娥到那里去了,为什么不见出来?”本家未及答应,早见那少年男子立起身来,睁开两只龟眼,一脸的怒气,迎着沈仲思说道:“你可就是姓沈的么?来得正好,我正要问你要人。”沈仲思抬头一看,并不认得他是谁,听他这般说法,不觉怒气直冲,高声答道:“我和你并不认得,你是个什么东西,却来问我要人,真是诧异!”那少年男子听了,冷笑一声,说出一番话来。正是:

万金买笑,空余宝枕之香;七夕苍茫,望断银河之影。

欲知后事如何,请听下回分解。





第七十八回 洪月娥有心讹曲辫 沈仲思同病劝瘟生





且说沈仲思在洪月娥家见一个少年男子向他发话。沈仲思出身豪富,从来只有别人巴结着他,那里受过别人的糟蹋?这一气非同小可,便也回骂了几声。不料那少年男子反是冷笑一声,对他说道:“你自己干得好事,还要推说不知。老实对你说了罢,只我就是洪月娥的本夫。你鬼鬼祟祟的把月娥藏到那里去了?我好好的一个人,如今被你弄得踪迹全无,我不问你要人,却叫我再去问那一个?”沈仲思听了摸头不着,好似当头打了一个闷雷;又听那少年男子的说话,没一句不是诬赖着他,口口声声的叫他把人交出,万事全休,不然便要把他扭到捕房,告他拐骗。沈仲思此时真是一盆烈火直透顶门,须发皆张,双眉倒竖,大叫道:“反了,反了!

你们做的好事,骗了我整整的六千洋钱,如今把他藏了起来,反来问我要人。难道我六千块钱就是这般轻轻易易被他骗去,世上没有王法的么?“此刻沈仲思方才心中明白,澈底澄清,晓得是洪月娥有心哄骗着他,骗得银钱到手,自家却躲在一边,串通了娘姨本家和他白赖,只恨得咬牙顿足,恨不得一时跳破了天。

只见那少年男子听了沈仲思的说话,不慌不忙,微微冷笑道:“据你口中的话,月娥骗了你六千洋钱,但是你和月娥也不过寻常的相好,并没有什么格外的交情,为什么无缘无故的给他六千洋钱?这句话儿凭你说到随便什么地方,我也不来信你。

你不说自家拐了他的身体,还要随口讹人,你未曾开口,也该打听打听我是个什么样人,可是讹得动的么?“沈仲思听了,更加大怒道:”你是什么东西!不过是个乌龟罢了,也要来吓什么人?若要问我为什么无缘无故的给他六千洋钱,你那里晓得我们的道理。前两天月娥说明嫁我,讲定一共八千身价洋钱,六千还债,二千开销,要我先付六千给他还债,所以我昨天付了他一张庄票。当时原要叫他写个婚书,他却托故推辞,说什么无人会写,骗我今日来拿。我倒把月娥当作好人,并不疑心。

谁知他骗了我的银子,自家背地私逃,还串出你们这一班人来通同图赖,难道他躲过了就好白白的胡赖不成?“那少年男子听他这般说法,那里肯听,只冷洋洋的问道:”你倒说得好一篇道理,吹得好一口牛厌,我且问你,你付了六千洋钱可有什么凭据,或当时有什么在旁看见的证人?“

沈仲思听了倒呆了一呆,这件事儿明明是自己过于托大了些,所以坏事。这六千洋钱的票子委实是自己亲手交在洪月娥手中,如今洪月娥躲了起来,给你个无人对证,既没有托人经手,又没有取得收条,这样的事就是到官司也是无凭无据的事情。明晓得有些尴尬,口中却不肯服输,高声嚷道:“这件事情,房间里的娘姨和本家们大家晓得。我当着月娥的面和他们三面言明,你只要问他们就是了。”那人听了回过头来正要问时,本家立在旁边听得明白,连忙抢先说道:“沈大人,勿是倪勿肯帮耐,就是月娥先生要嫁耐沈大人,倪本底子也勿晓得。原是耐沈大人叫倪上来讲啥格身价,难末倪刚刚晓得。勿瞒耐沈大人说,俚耐是自家身体,亦勿是倪个讨人。俚耐说要嫁人,倪也勿好说闲话。故歇耐沈大人说付过歇六千洋钱,倪轧实朆看见;再加仔故歇月娥格人勿知到仔陆里去哉,赛过死人无对证格事体,倪也朆看见啥格六千洋钱,连搭仔月娥到仔啥场化去倪也勿晓得。倪开仔堂子,陆里耽得起实梗格风火?真真前世倒仔霉,碰着格号事体。”沈仲思见那本家的口风,明明的袒护着那少年男子和自己为难,心上虽然愤恨,却又驳不倒他,只得说道:“照你这般说法,倒是我没有付钱,有心图赖你们的了?”本家急忙分剖道:“勿是呀,沈大人付俚洋钱格辰光,倪轧实朆看见,阿好瞎说瞎说。”

那本家正要说下去,被那少年男子一声喝住道:“不要多讲,且待我来问他。”

便从从容容向沈仲思说道:“你说你付过六千洋钱,又拿不出付钱的凭据。你想,六千洋钱的事情虽然说大不大,说小却也不小,比不得六百六十不算什么希奇,那有付了六千块钱没有一个凭据的道理?况且本家们既然晓得这件事情,你付钱的时候,为什么不把他们听上楼来当面交代一个明白,却要鬼鬼祟祟的私相授受?你们大家听听,天下可爱这样的痴子么?老实和你说,月娥这里,这一个月里头除了你天天来往之外,没有什么别处的客人,现在无缘无故的月娥不知走到那里去了,不是你干的事儿还有那一个?若说这件事儿不干你事,为什么他别的时候不走,偏偏拣了这个时候逃走呢?你好好的还我人来便罢,如若不然,哼哼,恐怕你难逃公道!”

沈仲思听了这般无赖的说话,只气得面泛沈霜,满身乱抖,明知自己失于检点,被洪月娥骗去了六千洋钱,却怕的沈仲思不肯干休,又想了这个极毒的主意出来,反客为主的一口咬定问他要人。看那少年男子的样儿,挺胸凸肚,怒气冲冲,只想寻事,晓得没有理讲。那班本家娘姨们又都是帮他说话,最苦的自己手中没有证据,说不出来,只得想暂时避过锋头,再想翻本的计较,便忍着一口气立起身来道:“你们这一班人真真的没有理讲,明明是你们通同一路,把月娥藏了起来,却还要这般说法。我今天也没有工夫和你讲理,明天再和你们说话就是了。”说着,匆匆的起身就走。不料那少年男子听他这样说法,又见他立起来便走,不觉勃然大怒,跳起身来两手一横,把沈仲思去路拦住道:“你倒说得这般轻可,容容易易的就想要走么?今天你不好好的交出人来,我便和你同到巡捕房去,凭你是什么大人老爷我也不怕。从来王子犯法,庶民同罪。做官的人杀了人,就好不要抵命么?”一面说着,揎拳掳袖的竟想要扭沈仲思的衣裳。

沈仲思见他真要动手,不觉慌了,大叫:“岂……岂……岂有此理,这……这是那……那……那里说……说起,方……方才你……你说我……我……我付过六千块钱,没有什么凭据,难……难道我……我……我指使月娥逃走,又……又……又有什么凭据么?”那人冷笑道:“我那管你有凭据没有凭据,只要问他们本家就是了。你天天贼形怪状的不知来说些什么,偏偏的这几天里头就会不见了人,你还想要赖到那里去了?今天我们的官司是打定了。老实说,我是个无名小卒,就是官司输了,也算不得什么,你却是个场面上人,看你怎样的坍台得起?”一面说,一面眼睁睁的就有个动手的意思。沈仲思见了势头不妙,要走又不能,不走又不好,竟十分的着急起来。暗想:“他们的说话虽然可恨,情理却是不差,丢掉了六千块钱还在其次,倘然真个的拉拉扯扯动起手来,被他们扭到捕房,虽然真者自真,假者自假,自然有一个水落石出的收场,但受了这般的糟蹋,以后还有什么脸面再在上海见人?”这样一想,便心中只想脱身。无奈那个乌龟怒目横眉的迎头拦住,心上正在忐忑,幸而那班本家和娘姨做好做歹的上来解劝说:“沈大人不是这样的人物,你不要瞎疑心,月娥虽然不见,我们慢慢的在外边访问,料想寻得出来。”那乌龟还是装腔做势的不肯。本家和哄着,又劝了一回,又把他拥出房去,那本家回头过来,向沈仲思飞了一个眼风,似乎叫他走的意思。沈仲思正在着急,巴不得立时就走,便三脚两步的走了出来,一路垂头丧气的回去。想了一夜,气得发昏。隔了一天,沈仲思还是昏昏闷闷的无精打采,只恨着洪月娥没有良心。

这一天沈仲思睡在床上还没有起来,忽地传进一个李子霄的名片,说要请他立刻过去。沈仲思不晓得是什么事情,想着这几天受了一肚子的闷气,正想要到子霄那里和他谈谈,便在床上起身,梳洗过了,吃过点心,直到李子霄新赁的公馆里来。

李子霄听得沈仲思来了,叫请楼上去坐,沈仲思就觉得有些诧异,暗想:楼上是他和张书玉两个的卧房,怎么叫我楼上去坐?心上这般想着,就跟着家人走到楼上,径进卧房。

沈仲思留心一看,见对面一间房门关着,这边房内却不见张书玉的影儿,连娘姨、大姐都一个不见。沈仲思见了十分诧怪。刚刚走进房门,李子霄起身迎着,彼此招呼了一声,沈仲思见他面上一付无精打采的样儿,正要动问,李子霄早一团盛气的,迎着仲思,把张书玉忽然逃走并拐去许多的东西的事说了一遍,并问沈仲思可有什么法儿,还是径去投报捕房,还是另想别法。沈仲思听了,方晓得书玉不见的缘故,原来也是落了他的圈套,和自家正是同病相怜,不觉哑然笑道:“原来你也上了书玉的当,怪不得要这个样儿。但是你还没有晓得我的事儿,我被洪月娥骗去了六千洋钱,如今躲得人影也不见一个,反串同了一个什么流氓,说得洪月娥的本夫,翻过来吃住了我,要我还他的人,我竟一时被他们逼住了,无言可答,幸得本家娘姨等大家相劝,才得脱身出来,你想想可是笑话不是?我为了这件事整整的气了一天,正要赶到你这里来和你商议,不想事有凑巧,你这里也闹了这么一个乱儿。”

李子霄听了大怒,不待沈仲思说完,抢着说道:“你为什么这般无用,竟被他们吃住了,一句口也不开?洪月娥既然逃走,就该问兆富里的本家要人,你花了六千块钱,难道就这样的罢了么?你既是这般胆小,待我来和你出头,若不把这件事儿追一个澈底澄清,我这‘李’字也不姓了。”说着立起身来要邀着沈仲思同走,沈仲思连忙止住他道:“你不要这般性急,我的话还没有说完,且等我说完了,我们再商议。”李子霄听了,方重新气愤愤的坐下,倒把自己张书玉逃走的事情放在一边,且听沈仲思的说话。只见沈仲思向他说道:“这件事儿实是我自家不好,过于大意了些,虽然付了他六千洋钱,却是我亲手交与月娥,此处交没一人知道。现在月娥的人不知那里去了,不见我的面儿,却串出本家娘姨等一班人来,咬定口风和我白赖,倒反问我要人。你想就是要打官司,也要有付钱的凭据,或者有什么证人,只是空口说白话,没有着实的收据婚书,这样的官司,凭你什么利害的人也想不出个万全之策。何必为了一个倌人,惊天动地的坏了自家的名气?好在我们也不是吃亏不起的人,虽然花了几千洋钱,却也总算长了一番见识,自家认个晦气,叹口气儿,譬如自家病了一场也就罢了。”

李于霄听了细细想了一回,觉得沈仲思的话儿句句有理,便道:“你的事情总算就是这般罢了,我的事情你可有什么法儿替我想想?”沈仲思沉吟了一刻道:“这件事儿据我想来,倒也狠有些棘手。你的婚书已经被他带走,当初又没有什么保人,就算报了捕房,把他退了回来,堂上的官员也不见得肯将他十分严办。但是在你这边想来,你娶了他不到半月,便被他卷物私逃,别人晓得了,显见得你是个瘟生,上了他的圈套。况且他已经逃走,就算追寻得着,也是没有真心,那时还是听凭他发堂择配,还是你自家仍旧收回?依我看来,这件事儿闹将起来,非但你没有什么好处,反是闹得通国皆知,还落了一个瘟生的名气,这又何苦呢?”李子霄听了,呆呆的瘫在椅子,长叹一声道:“罢了罢了,听你这般说法,果然闹将起来没有什么味儿,只得也和你一般认个晦气便了。”说着,还长吁短叹的十分不乐。

沈仲思也想着了自家的心事,彼此默然,停一回方说道:“青楼妓女,本来十个倒有十一个没有良心。我们经过了这样的一番阅历,以后须要看破些儿,只好逢场作戏,随便应酬,断不可再上他们的当,那就明知故犯,一误再误了。”李子霄听了不住的点头称是,两人又彼此互相劝慰了一番。

从此李子霄、沈仲思两人看破了倌人的伎俩,把那寻花问柳的念头淡了许多,就是做个把倌人,也不过叫几个局。吃几台酒,应酬朋友,从不去转他的念头,倒成了败子回头,悬崖勒马。正是:

结束铅华之梦,禅榻西风;屏除丝竹之情,电光石火。

欲知后事如何,且听下回分解。





第七十九回 论嫖界新小说收场 结全书九尾龟出现





旦说沈仲思和李子霄自从受了张书玉、洪月娥的骗局,居然勘破痴情,忏除绮业,这也总算难得的了。看官听者,从来泡影无常,昙花一瞬,兰因絮果,一切茫茫。金樽檀板,销磨儿女之情;秋月春花,短尽英雄之气。或有五陵豪客,裘马轻狂;湖海词人,风情旖旎;貂裘夜走,株叶朝迎。十年歌舞之场,一万缠头之锦,送客留髡之夜,誓海盟山;酒阑香烬之宵,飘烟抱雨。这样的风流艳福,自然是见者侧目,闻者倾心。但是上诲滩上的倌人,覆雨翻云,朝张暮李,心术既坏,伎俩更多。将就些儿的人入了她的迷魂阵,哪里跳得出来?没有一个不是荡产倾家,身败名裂。在下做这部书的本旨,原是要唤醒诸公同登觉岸,并不是闲着工夫,形客嫖界。所以在下这部书中,把一班有名的倌人,一个个形容尽致,怎样的把客人当作瘟生,如何的敲客人的竹杠,各人有各人的面目,各人有各人的口风。总而言之,都是哄骗了嫖客的银钱,来供给自家的挥霍。那些千奇百怪的情形,一时也说他不尽,看准了那客人的脾气,便专用那一种的手段去笼络他,定耍把这个客人迷得他意乱神昏,敲得他倾囊倒箧,方才罢手。在下这部小说,把他们那牛鬼蛇神的形状,一样一样的曲笔描摹,要叫看官们看了在下的书,一个个回头猛省,打破情关,也算是在下著书劝世的一番好意。在下书中的这些说活,虽不免有些过分的地方,却这些事迹,一大半都是真情,并不是在下自家杜撰。做书的做到此处,便算是一部四大金刚的外传收场;如今且把这些闲话一齐收起,就是那章秋谷,也暂且不提;先要提起那“九尾龟”的正文来。兔不得要把他的出身来历,一一的铺叙一番,好作个全书的结束。

且说无锡城内有一家暴发的乡绅,姓康,官名汝楫,表字己生,由附生出身,捐了一个候补道,署了两任事,又放过一任关道,慢慢的升到江西抚台。他老太爷倒是个进土出身,做过一任知府,在知府任上,不知怎的就弄了十几万银子回来。这位康太守,有了钱就不做官了,一直回到无锡,就着这几个钱,收些利息,也还用下了,倒也无拘无束的,十分自在。康太守中年无子,直到五十岁上方才生了这康己生。因为他是己年生的,所以就叫他己生。康太守得了这个儿子,欢天喜地,把这康己生好象明珠异宝一般的擎在手中.一口大气也不敢呵他。康己生长到五六岁上,便请了一个有名的孝廉公来做先生。无如这位康公子的心性若明若昧的,不甚明白,又来肯用心读书。先生见他不肯用功,晓得这个学生是东家溺爱的,便也不十分去做那空头冤家。首尾教了十二年,把这个康己生也教了个半瓶醋的学问。己生自六岁上学读书,到了十八岁上,那先生辞了馆地。这位康太守也糊里糊涂的,不去考查儿子的学间。己生见康太守这般,乐得说些大话,满口胡吹,自以为自家的学问数一放二的了,看得那些举人进士,就如在手心里一般。

过一年,适遇督学按临,己生也要打算去考。这督学公是十科前辈,现任刑部左侍郎,姓王,号兰佩,名体仁。性情甚是古怪。每到考的那一天,他却一天到晚顶冠束带的坐在大堂暖阁里头,把这些童生拘管得十分苦楚,背地里无不咒骂这位宗师。且说康己生要去应考,府、县两试,倒也不前不后的,取在二圈里头。府、县考过了,便去钻头觅缝的,打听了一位王大宗师的同年陆太史,放过一任福建学台,现在恰好丁忧在籍。平日间与王侍郎相与得十分稠密。原来王侍郎和陆太史都是现在余大军机的得意门生,所以他们两人的交情,格外比众不同。不知怎佯被康己生打听着了,花了五百两银子,托人去求了陆太史一封书信。到了江阴,谁知去得迟了两天,童生正场已经考过,后来的人一概不准补考。康己生急得没法,在寓中咒天骂地的,把带来的一班家人厨子,一个个骂得垂头丧气,胆战心惊。有一个得用家人叫做石升,素来十分伶俐,最得这位少爷的喜欢。见己生甚是焦急,便悄悄的对己生道:“据家人看来,少爷且把陆大人的信送进去,试他一试,看这位大人如何打发。虽说不准补考,从来打官话的都是这般说法,哪里就一定不准补考了吗?就是学台当真不肯通融.我们这里有的是银子.再花上些银子,什么大不了的事情!”己生听了,心上大喜,高兴得直跳起来,笑道;“我原说我带来的几个家人,就是你一个人靠得住,还能干些事儿。只是为什么不早替我说,害得我直急了关天。我们此刻马上放送信进去,看他如何说法?”就叫石升带了红缨大帽,穿了马褂,登上快靴,飞也似的赶到学院衙门投信。到了学院,直进号房,把陆太史的信交在号房手内,请他送进,自己便坐在号房候信。

且说号房投进书去,这位王侍郎拆书看了,心中很有些儿委决不下,暗想道:“这陆太史也很糊涂,我向来规矩极严,从不受一些请托。况且正场已过,这康汝楫有意迟到几天,落得回复他去。”忽又转念想道;“若是叫他回去,却又碍着同年的脸面,不好看相。就是余老师分上,也有些不好意思。”想来想去,想了多时,究竟那皇上家的关防抵不过同年的情分。正在踌躇未决之际,恰恰的事有凑巧,门上传了几个禀帖进来。原来是十几个外县童生,也为到迟了两天,不能补考。这班童生慌了,联名具禀,要求王侍郎补考大收,禀帖上说得十分恳切。王侍郎看了,暗想:“既然如此,我也乐得听了陆太史的来信,做个顺水推船的人情。”

想定主意,便吩咐出去,叫康汝楫在外候着。号房传出话去,那石升得了这一声,便飞跑出来。一路跑着,一路又打算主意,要想撒一个谎,骗他主人的钱。一口气跑至寓所,走进大门,看见这位少爷正在房内踱来踱去,低着头不住的搓手,约摸着是心中在那里打算念头。猛一抬头,见石升气急败坏的跑进来,急问事情怎样?石升方才在路上的时候,已经打算得停停当当,此刻不慌不忙,对着康己生指手划脚的说道;“家人到了学院衙门,送进信去。王大人把家人叫进去,当面问了一会,便道:我这里的规矩,向来不准补考,你回去对你主人说,叫他下次来罢。那时家人也不敢多说,只得退了出来。”石升还没有说完,康己生早急得瞪着眼睛,连说:“这怎么好?这怎么好?”石升又接下去说道:“家人退了下来,后来一想,要真是这样,不是少爷白白的来了一趟了么?家人便去寻着了文巡捕吴大老爷,再三的求他想法。这位巡捕老爷答应是答应了,只是有一句话,家人不敢说,要求少爷宽恕了家人,家人才敢说呢。”说罢,把两手逼在背后,又请了一个安,直挺挺的站在一旁,一声不响。康己生以前听得学院不准补考,已急得满头流汗,遍体生津,好容易听见巡捕肯替他想法,甚是欢喜,正在扯长了耳朵,听他说下去,见他说了一半,就不说了,心上十分焦躁,连连的跺脚道:“糊涂东西.你不看见我在这里着急么?怎么说了一半,就不说了?”石升见他急得头红面胀,心中暗暗好笑,便凑上一步,又说道:“那吴大老爷开口,定要五百银子,一些也不肯短少。家人好容易从一百两银子说起,一直添到三百银子,是再少不来的了。家人大胆竟应允了他,现在他还在巡捕厅等家人的回信,不晓得少爷心上如何?”己生听了,“呸”的啐了石升一口,又骂道:“这点小事,你去办了不结了么?三百银子,什么大事,还要在我这里蝎蝎螫螫的,滚你妈的蛋罢!”骂得石升又羞又喜,口内连声应是。又立了一回,见己生不开口了,便侧着身子,退出来。便走到同他主人来往的钱庄上,取了三百银子的洋钱,到街上各处去空走了一趟,便跑了回来。又把方才的银子藏得严严密密,方向己生说道:“银子三百两,家人已经当面交与吴大老爷了。吴大老爷答应明后两天便有信息。”己生听了,欢喜非常,便磨拳擦掌的在寓中等侯。

到了明日绝早,果然学院衙门高高的挂出一扇牌来,一共补出十七个童生,康汝楫自然也在其内。到了补考的这一天,己生收拾考具,坐了轿子,几个家人前呼后拥的到学院衙门等候。不多一刻,里头升炮开门,王侍郎升坐大堂,点名给卷。康己生领了卷子,归号作文。原来这一回补考,一共只有十七个人,王侍郎叫承差在大堂旁边安设桌椅,叫他们就坐在两旁。封门之后,承差掮出一扇高脚牌来,上写首题目,首题是:“先生以利说秦楚之王”;二题是:“其至尔力也”。这原是王侍郎调侃康己生的意思,头题是明知那陆太史的一封书信,是花了重价得来;二题是说这来到江阴,是你的力量,下文明明的含着其中非尔力也的两句意思。虽然如此,这康己生原是个富贵公子出身,哪里晓得题目的命意。但是腹内空空的,要做这二文一诗,甚是吃力。倒也亏他,居然勉强做得出来,这正是破题儿第一次,当下勉强交卷。隔了几天,贴出酌覆的案来,康汝楫居然补在里头。康己生随众进覆,依然草草的敷衍完场。出场之后,随着出案,把康汝楫高高的取了第五名。己生喜欢得拍手打脚的,笑个不了,好像痴子一般。拜了教谕,见过宗师,便收拾行李,回到常州。得了一个秀才,便如天塌下来的一场宫贵。那些亲戚朋友为他有钱,便一个个都去奉承他,秦承得这位新秀才十分欢喜,浑身骨节都觉得痒飕飕的,连自己也有些不信起来,竟是自己的文章换来的一般,把自己的本来面目通通忘了。见了别人,把一个脸儿仰得高高的,一副得意的样儿。这可合着了一句骂人的俗语,则做“龙门未折三秋桂,狗脸先飞六月霜”了。

闲话休提。只说康已生兴兴头头的专等明年乡试,预备着乡会连科。却自从得了这领青衿,便把文章书籍一概丢失,不是寻花问柳,便是引类呼朋,却像这进土举人,毕竟会自己飞到家里来的一般。康太守以前虽把这儿子看如珍宝,有时还拘束拘束他,现在看见他儿子得了功名,虽然不过是个小小的秀才,常言“秀才是宰相的根苗”,便也自譬自解的不去管他,竟等封翁做了。春来秋往,早已过了一年。到了秋间,又早是乡试的时侯。康己生带了许多仆役,雇了一只大船,门枪旗灯,十分煊赫,就像是什么现任官员赴任的样儿。到了南京,寻了一所精致河房,他一人住下。那录科领卷的这些照例事儿,总不必去提他。只说录科已过,康己生专等入闱,却心上忐忑,恐伯万一不中举人,如何是好。就打发家人四出寻访门路,自己却只在钓鱼巷堂子里头住宿,整天整夜的也不回寓。就这般糊糊涂涂的过了两天。己生正住在钓鱼巷还未起来,石升同了一个长随打扮的人来找他。等了一会,已经午后,方见己生睡得铺眉蒙眼的,披着衣裳,趿着鞋子,口中不住的打着呵欠,走了出来,问道:“有什么事,这时候就来寻我?石升抢步上前,附耳说道:“家人寻着了一家门路,是最稳当不过的,请少爷回寓去说罢。”己生一听大喜,便连忙走进去,穿好衣服,又走出来。那轿子是石升带来接的,便坐轿回寓。还未坐定,石升上来说道:“这同小的来的,是桃源县郑大老爷的签押房家人,名叫陈贵。郑大老爷是翰林散馆出来,就放了甘泉县,现在又调到桃源县来。”正是:生公说法,欲点顽石之头;阿堵无灵,销尽豪华之气。欲知后事如何,请看下回交代。





第八十回 通关节花钱遭巨骗 捐道员拜客出风头





且说石升低低的向康己生说道:“这郑大老爷今年点了第一房房官,又和副主考汪大人是同年,方才这个家人对小的说道,只要有银子,拿得定就是一个举人,并且还可以同着去见郑大老爷当面交代。家人想这条道路倒还稳当,所以同他来见少爷的。”己生听了,便说叫他进来。

当下石升便去同了那陈贵进来。向着己生也把腿略弯一弯,算是请安,便站在旁首。己生看陈贵时,面目清秀,举动伶便,却像一个现任州县的亲随,当时问道:“你同我家人说的那件事儿,要多少银子?倒底稳当不稳当呢?”陈贵走上一步,轻轻的说道:“这银子原不是家人要的,就是讲定了数目,交银子的时候也得你少爷自己交给敝上,省得要经别人的手儿,只是这数目敝上说一定要三千银子,如或短少是不必去说的。”己生道:“三千银子,我不好去捐个知县,不比买这个举人好的多么?”陈贵道:“这是你少爷自己的名气,中了举人,体面却好得多了,即如少爷今年中了举人,明年还要中进士,点翰林,将来一样也好放得学台主考,这是不能说的,你少年自己打主意就是了,我们当家人的还能勉强着办么?”己生听陈贵这一番说话讲得十分中听,便道:“只要一定靠得住,我就出三千银子也不算什么,但只能先付一半,放榜之后,再行找足如何?”陈贵道:“这一半的说话,家人却不敢答应,请你少爷到我们公馆里头去当面说就是了。”己生道:“也可以,我立刻和你同去。”便换了衣冠,坐着轿子,因为恐怕招摇耳目,只带了石升一人,陈贵也跟在轿后。

轿子走到武定轿左首,说是到了,只见陈贵抢先一步赶进大门。石升便拿着治晚生的名帖,跟着陈贵走了进去,那轿子就在大门外暂时站住。己生在轿中看时,见这门楼高大,彩画辉煌,大门上贴着一张朱笺,上写着“特授淮安府桃源县正堂郑公馆”几个大字,又有两张朱笺贴在两旁,写着“回避”,那字写得铁画银钩,十分的端丽,却像个玉堂中人写的。正在观看,忽听得远远的喊了一声“请”,便有十来个人接接连连的喊出来,早听“吱”的一声,两扇中门分开左右,陈贵立在门内,手中举着名帖高声道“请”。己生的轿子便由中门进去,到了大门下轿,陈贵在前侧身引道,到了花厅便又退出去了。己生坐在花厅等了好一刻,才见陈贵又来把帘子高高打起。那位郑大老爷顶冠束带的走了进来,背后跟着四五个当差的,己生连忙恭恭敬敬的行下礼去,郑公却止还半礼,起来让坐,早送上茶来,彼此又打一恭,方才坐下。

郑公先开口道:“尊帖本不敢当,只因小价来说,吾兄有事来此商量,将来不免有个师生之谊,兄弟却有僭了些。”说轻呵呵的笑了。己生又着实谦逊了一番,方才抬头看时,只见郑公花白胡须,方面丰卧,眉目清朗,举止凝重,言语安详,称得起个官场的品格,便又把要买关节的意思说了一番。说到先付一半的话,郑公便截住道:“这件事儿,原是大家取信,不必勉强。况且兄弟的意思不过想要多收几个门生,并不是于中取利。既是我兄尊意不甚相信,竟请吾兄别寻道路,兄弟倒并不介怀的。”己生碰了这个钉子,便慌了道:“既是公祖这般说法,治晚何敢有违?立刻就将该项当面交割,省得另日叫人送来。不知公祖的心上怎样?”郑公听了道:“这个也悉凭尊便,兄弟不便撺掇的。”

当下己生主意已定,使叫石升进来,叫他到钱庄去开银票,石升飞一般的去了。

不多时已经回来,把一张银票双手递上,己生看了不错,立起身来,双手又送与郑公。那郑公却不自己用手去接,只向着背后的家人把嘴努了一努,就有一个俊俏跟班上来接去。己生见话已说妥,便起身告辞。走出花厅,又说了两句叮嘱的话,大约是怕他落空的意思。不料这位郑大老爷却拂然不悦,冷笑一声道:“老兄看得人太不值钱了,难道我这桃源县知县,止值这三千银子么?”己生吃了一惊,连声“不敢”,打拱告辞。他送到滴水檐前,就不往外送,遂把身子躬了一躬,大摇大摆的踱进去了。己生上轿回寓,虽然花了三千银子,心上却说不出的得意。

在寓中休息了几天,早已场期到了,石升便料理考篮、风炉、书本、茶食、油布、号帘,一一停当,初八日五更就叫了己生起来,五六个家人前后簇拥的出门而去。

到了贡院,领了卷子,石升是来过几次的,便当先引路,掮着书箱,依着卷面上刻的字号寻着了号子,替他解了考篮,钉好号帘,铺好号板,又把风炉拿出来烧了炭,炖好茶水,方才一齐出去。己生到了号内,只见通共只有一张方桌的地方,吃,喝、睡觉都在里头。己生是在家受用惯了的人,何曾受过这般苦楚?觉得坐立不是起来,焦躁了一回,也是没法,只得捺住了心,勉强睡下,却倒睡着了。直睡到午后方醒,已经听得明远楼上的号筒不住的呜呜价吹,吹手不住的吹打,远远的又听得炮声,想是已经封门了。腹内却觉得有些饿起来,便叫号军取开水来,将带的风米泡了两碗,又取出路菜火腿、薰鱼等胡乱吃了一顿,便又呆呆的坐在号中。

听得外面的一班考生呼朋唤友高谈阔论的十分热闹,己生也不去管他,到晚间又随便吃了些茶食,便自睡了。

约莫四更时分,己生正在睡熟,忽觉有人在他身上连连的推了几下。己生糊里糊涂的还认是在自己寓中,不知何人把他推醒,心中大怒,坐起身来方欲骂时,头上“鼙冬”的一声,早把自己的头撞了一下。这一下,直撞出一个疙瘩来,方才记得是在场内,自己不觉好笑。连忙看时,却是号军送了题纸来了,便手接题纸,点起火来看时,只见头题是“大哉圣人之道”,二题是“此之谓大丈夫”,三题是“西子蒙不洁,则人皆掩鼻而过之”;诗题是“诸君何以答升平,得平字五言八韵”。

己生看了,却呆了一会,觉得这几个题目不知从何处做起,只得铺下草稿,定心做去。

早过了一天,已是初十日午后了,己生刚做了头次二题,第三题尚未做完,早见邻号的人纷纷交卷,外面已放二牌。己生惟恐来不及抄写,便急忙忙的把一文一诗凑完,连忙取出卷子誊真。好容易誊到第二篇,正在闷着头写,忽见几个人掀起号帘来,抬头一看,见这一班人都戴着红缨大帽,又有一个拿着一个大号筒照着他的面孔,呜呜的吹。己不知何故,倒着实的吃了一惊,急问时,方知是净场催缴卷的,心中越急,越写不上来,勉强潦潦草草的乱了一阵,抄完了去交卷时,场中早已静悄悄的不多几个人了。连忙收拾了考具,叫号军掮着到龙门口,自有人接出大门。大门之外,石升带着众人等得不耐烦,见主人出来,急抢上来接过考具。坐上轿子,回寓便睡了。

有话即长,无话即短,二三场一样的进场,草草完事。十六日出场,己生累得狠了,足足睡了两日,方才起来。又过了四五天,便收拾行李回到常州。到家之后,把那似通非通的文摘,抄了几篇送给亲友观看,自以为花了三千银子,这个举人是稳稳的飞不到别处去了。那各亲友中也有有些见识的,见己生的文稿都暗暗的摇头,却当面不肯说出,只是一味的奉承。

说时迟,那时快,早已过了九月十五,差不多要放榜了。到了放榜的前一天,算计五更可以得信,康己生便约了各家亲友,治了酒肴,大家欢呼畅饮的在那里等榜。已生做了主人,高谈阔论的只在那里背他的场作,又摇头摆尾的道:“若说这样的文章试官不中,今年常州府内就没有可中的人了。”各亲友听了免不得附和一番。大家饮酒至三更光景,又叫了几个土娼来陪酒,弹起琵琶唱了几支京调小曲,说说笑笑的不知不觉已有五更。只见石升飞跑进来道:“外面报房已经开报,我们还没有报来,只怕少爷中在五名之内呢。”说犹未了,早听得远远的锣声自北而南,镗镗的敲过来,己生不觉直立起身,竟向门外迎去,各亲友也随后跟来,到了大门之内,眼睁睁的看着那一班敲锣的报子走了过去,竟是头也不回。己生便觉得心上有些把不稳起来,却还倚着走过门路,不至落空,或者竟中在前面也未可定,便又大胆起来,重新进去,再邀亲友们饮酒。

众人见报人不来,心上都道是没分的了,面上还不肯露出来,依旧在那里敷衍着他,乐得开怀畅饮。只有己生等了一会还没有信息,身子虽坐在席中,那心上就如十五个吊桶打水一般──七上八落的,面色青黄不定,看他那个样儿,煞是难过。

延挨了一会,早已天色大明,东方日出,众亲友见此光景,料难再留,各自起身告别。免不得说几句套话,安慰己生道:“功名迟早有定,下科一定高魁,那时再叨喜酒。”己生没精打采的送出大门,彼此一拱而别。己生回到书房,心上越想越气,便把石升叫来大骂了一顿,吓得石升诺诺连声,跪在地下自家认罪。原来这件事儿,却是南京的一班骗子做的圈套,石升并不得知。康己生又是个寻常纨绔,那里看得出什么人情世故,所以刚刚的着了道儿。当下己生把石升骂了一顿,也无可奈何,只得罢了,闷闷的坐在家里。

坐了几天,就有一班朋友劝他不必应试,越着现在捐例大开的时候,不如竟去捐一个官,你又不是捐不起的人,就是捐个道台也不是什么难事。己生听了如梦方醒,恍然大悟,便和他父亲康太守说了,想要捐个道台。这位康太守素来溺爱己生,那有不听?果然拿出钱来交给己生,托人上兑。己生要图体面,索性加了一个二品顶戴,差不多也花到一万三四千银子的样儿。从附生上一直报捐道员,却是从来没有的,也算得一件奇闻。更兼康己生自从捐官之后,自己想想不过花了一万多银子,居然就是惶惶的一个大员,十分得意,整天的带着珊瑚顶,拖着孔雀翎,大摇大摆的坐着轿子,在街上拜客。却想着自己现在是个道台,照例要坐绿呢轿子,方合大员的体制。无奈这绿呢轿子无锡城内竟是借不出来。己生的性儿又是今天等不到明天的,十分性急,只得到丧衣店里头,赁了一乘绿呢四轿,坐着拜客,别人看见他这般怪相,没有一个不是掩口葫芦。康己生那里晓得?还是扬扬自得,荣幸非常,一连拜了几天客,便要打算进京,去办引见到省的事情。

那时已经有了轮船,甚是快当,不多几日已到北京,暂住在一个同乡家内。这同乡也是一个京官,叫马申甫,少年点了探花,不多两年就用了军机章京,推升了达拉密,那一班军机处的王爷、中堂们多器重他。康己生住在他家,晓得他是中堂们的红人儿,竭力拉拢,又把自己的女儿许给他的亲侄儿,后来又不知怎的,康己生居然走着了章凤藻章中堂的门路,送了一分厚礼,把章中堂拜作老师。章中堂倒甚是器重这个门生,给他一个明保,康己生就顿时的显赫起来。不多几时,放了一任天津道,章中堂又在里面照应着他,便又调了江苏上海道。十多年的光景,康己生熬炼资格,论俸推升,竞直做到江西巡抚,这真是“孤始愿不及此,今及此,岂非天乎”了。康己生在天津道任上的时候,还有许多帷薄不修的丑事,传播官场,没有一个不晓得这位康观察的笑话。料想列位看官也有些晓得,用不着做书的在下替他一一宣扬,这一回书却就是《九尾龟》的全书结局,诸公若一定还要打听这位中丞的历史,或者待在下费些笔墨,再续他一部出来,现在却是限于篇幅,只得就着这些事迹,作个《九尾龟》五集的收场。

本来在下这部小说虽然名叫《九尾龟》,不过是借着他作个楔子,究竟并不是嫖界醒世小说的正文。看官们不要认错了在下作书的宗旨,正是:

一把辛酸之泪,回首销魂;十年风月之场,现身说法。





第八十一回 演前文重见九尾龟 醒迷途续成新小说





上回第五集书中,正说着那位康己生康观察乡试不中,便捐了个河南候补道到省候补,后来居然暑了一任开归陈许道,又调补丁直隶天津道,不到一年的工夫,升授了河南按察使,得了直隶总督陆制军一个密保,便升补了江西布政司。到任不及两个月,刚刚的江西抚台德中丞调了热河都统,这位康方伯便升授了江西巡抚。

这也算得是一帆风顺,宦运亨通了。如今在下且把康中丞的一面按下不题,再把章秋谷的事实演说一番,诸公静听,待在下慢慢的说来。

只说章秋谷自从娶了陈文仙之后,两个人自然是似漆投胶,如鱼得水,频伽共命,鹣鲽同心。凌华十五之年,初逢韩寿;碧玉小家之女,来嫁王昌。地久天长,一双两好。秋谷也怕文仙散淡惯了,坐在家里头要气闷,便也时常同他出去跑跑马车,看看夜戏。在上海约有住了三个月,忽然接了家里头太夫人的一封来信,叫秋谷快些回去。依着秋谷的意思,要想把陈文仙留在上海,自己回去省亲,倒是文仙不肯道:“我既然嫁了你,嫁鸡随鸡,嫁狗随狗。你如今回去,我自然应该跟你回去,那有我一个人住在上海的道理?”秋谷忽地哈哈的笑道:“好呀,你索性把我比起畜生来了。”文仙听了一面笑着瞪了秋谷一个白眼道:“你这个人实在的难说话,一句无心的话儿,你又要挑起眼来,难道我和你两个人还要这些过节儿不成?”

秋谷笑道:“我们两个人自然用不着讲什么过节,我也不过是说说罢了。但是你既然要跟我回去,我现有老母在堂,家中又有正室,虽然没有什么别的,那礼数关节是不能错的。你是向来散淡惯了的人,那里受得起这般拘束?到了那个时候,万一有什么委屈你的地方,叫我心上怎样的对你得起?”文仙听了把头一别道:“怎么你这样的明白人,也会说出这样的糊涂话来?你家里有老太太,有正室少奶奶,我是向来知道的。如今既然嫁了你,不跟你回去和老太太、少奶奶住在一起,难道倒要另外一个人住在上海,叫你身心两地不成?再说起什么老太太、少奶奶面上的礼数关节来,那更是我分内的事情,算不得什么,你只顾放心同我回去,不要这般七上八下的拿不定主意。”

章秋谷听了陈文仙的一番说话,低着头沉吟了一回,方才说道:“你的说话自然不错,但是我心上好像总觉得有些不妥当,万一到了那个时候你受了什么委曲,或是闹了什么口舌,心上抱怨起来,那就懊悔嫌迟了。”文仙道:“这是我自己愿意跟你回去的,那有懊悔的道理?况且我们两个人住在上海,你的家眷又不在这里,不尴不尬的,究竟不是个长久的法儿,如今跟你回去是再好没有的了。”秋谷听了心中暗暗的欢喜,故意再逼他一逼道:“你果然情愿跟我回去么?不要是一时高兴头上讲的顽话罢。”文仙正色道:“顽是顽,笑是笑,这样的事儿那里好和你顽笑?”

秋谷听了笑道:“既然如此,是再好没有的了。”

当下便和陈文仙商议了一回,把那些家具动用的东西,本来有一半是租的,便都退还了店家,自己的家具拣好的带了回去,粗笨些的便都丢掉了不要。商议定了,文仙倒忙忙碌碌的收拾了两天。到了动身的隔晚,文仙把自己的东西和秋谷的行李都收拾得妥妥贴贴。陈文仙本来身体娇弱,又是一双凌波三寸的金莲,忙了一回,只把她累得娇喘微微,浑身香汗。章秋谷在旁边看着只是微微的笑,也不开口,也不动手。文仙喘息了一回便对着章秋谷道:“你不来帮助我也还罢了,只顾看着我笑些什么?”秋谷一面嘻嘻哈哈的笑着,一面问道:“你这两天忙些什么,无缘无故的为什么要忙到这般模样?”文仙听了诧异道:“原是你自己和我讲的,收拾了东西好同你回去,怎么你倒反来问起我来?难道你贵人忘事,已经忘了不成?”秋谷又笑道:“看你这个样儿,真要收拾了东西同我一起回去么?”文仙听了摸不着一些头脑,只得说道:“不是真的倒是假的不成?你怎么平空又说出这样的话来?”

秋谷听了抢步过去,走到文仙面前深深的打了一拱道:“多谢多谢。”陈文仙见了章秋谷这般张智,更觉摸头不着,只得说道:“你这个人不要是发了痴罢,怎么无缘无故又打恭作揖起来?”秋谷慨然说道:“我章秋谷半生落拓,百事殢邅,天壤茫茫,竟没有遇着一个知己。不料如今居然娶着了你这样的一个人,既不贪我的钱,又不图我的势,却这样的和我一心一意,没有些儿势利的心肠,你叫我怎样的不感激,怎样的不欢喜?”说着不觉言下黯然,大有独立苍茫,四海无家之恨。

陈文仙本来是个情种,听了章秋谷这一番说话,不觉打动了他的情肠,流出两行珠泪,紧紧握了章秋谷的手,四目相视,脉脉含情,觉得心上千头万绪的不知有多少话儿要说,却一句也说不出。停了一回,陈文仙方才笑道:“我既然已经嫁你,我这个人就是你的,自然该应跟你回去,自己人还用得着这般么客气么?”秋谷在袖子里头取出一方丝巾来,和文仙拭干了面上的眼泪,口中说道:“你还没有看见上海地方,多少有钱有势的客人,娶了个倌人不肯回去,住在上海的多得狠在那里,那里能一个个都像你这般贤德。”文仙道:“说起‘贤德’两个字来,我也不敢当。

不过自己还保得定不至于闹什么笑话罢了。老实和你讲罢,那些嫁了人不肯回去、一定要住在上海的倌人,都是有心淴浴,不是真要嫁人。若果然真要嫁这个人,自然要和他想个安稳法儿,那有不肯住在一起的道理?“秋谷听了微微一笑,便搀着陈文仙在榻上并肩坐下,恳恳切切的对他说道:”既然如此,我却有几句推心置腹的话儿和你讲个明白,你却不要生气。“

看官,你道章秋谷是当真要同着陈文仙一同回去么?原来秋谷的太夫人陈氏性情严厉,不许秋谷在外边娶妾,在下做书的在初集书中已经提过。如今秋谷在上海娶了陈文仙,原是瞒着他那位太夫人的,那里敢就是这般的同他回去?只因陈文仙自从嫁了章秋谷以来,虽然是倚影怜声,双心一袜;鸳鸯比翼,蛱蝶同心,但秋谷心上毕竟还有些儿疑惑。想着文仙虽是一心嫁我,没有什么别样的心肠,但是如今是把他放在上海,吃的、穿的、用的虽然不见得怎样的奢华豪侈,却也般般不缺,样样现成,既没有一些儿愁烦,又没有一些儿拘束,过着这样的日子,那里现得出什么真心?不如我假意和他说明,要把他留在上海,看他怎样的一个说法。章秋谷想定了主意,便常常的对着陈文仙说,家里头的太夫人家教怎样的方严,规矩又怎样的利害。陈文仙听了,只微微笑着并不开口,秋谷一时也看不出他心上的意思来。

刚刚这个时候,太夫人写信叫他回去,秋谷便趁着这个当儿,假意去和陈文仙商量,要把他留在上海。那知陈文仙自家不肯,一定要跟着章秋谷一同回去,秋谷听了心上自然欢喜,便细细的把自己家里头的事情和陈文仙说了一遍,又说明不能同他回去的缘故,叫文仙仍旧住在上海等他。

陈文仙听了不觉俊眼横睃,蛾眉微蹙,哨了秋谷一眼道:“你这个人的心不知是怎么生的?凭着别人向你呕出了心肝,你依旧是指东画西的不肯说一句真话。幸而我的嫁你还是真心,你试不出什么马脚,万一我心上存了一丝一毫的假意,被你试了出来,那还了得么?我平日待你究竟怎么样,可得罪过你没有,你自己去想想!

如今无缘无故的又要这般鬼鬼祟祟起来,你怎样的对人得起?“说着便别转头去,洒脱了秋谷的手,一言不发,不觉有些烦恼起来。眉锁湘烟,眸回秋水,那一付含怨含颦的丰态,直似那雨中菡萏,霜里幽兰。章秋谷少不得深深的抚慰一番,又对着文仙说道:”不是我这样的一番做作,也显不出你的一片真心,你又何必这样的动气呢?“文仙听了方才破涕为笑,当下走到窗下一张梳妆桌上,对着镜子重掠乌云。秋谷便站在陈文仙背后,细细的打量那镜子里头的陈文仙,只见他宝靥偎霞,蛾眉却月,西子捧心之态,太真倾国之姿。觉得真个是国色天香,一时无两,把一个章秋谷看得呆了。陈文仙在镜子里头,看着秋谷这般呆看,便在镜子里头对他笑道:”你看些什么,难道到了如今,你还没有看够么?“说着那两边颊上,不觉升起两朵红云,越显得十分媚妩。这一晚桥填乌鹊,水溢银河;雨殢阳台,云迷巫峡。

檀奴归去,匆匆唱南浦之歌;凤女相思,缓缓结芳兰之佩。

过了一天,章秋谷安顿了陈文仙,把自己在上海经手首尾的事情料理了一番,又到辛修甫、王小屏、陈海秋等几个要好朋友那里去辞了一回行。大家都不知道他要回去,如今听得秋谷说立刻就要动身,辛修甫怪他为什么早些不讲。秋谷道:“我此番回去省亲,不多时就要出来的,你们不必挂念。”依着陈海秋,还要和他饯行,王小屏拦住道:“你不听见他说立刻就要动身么?那里还来得及饯什么行。”

秋谷也向陈海秋拱一拱手道:“我们知己弟兄,相交在心,本来不必拘什么形迹,我心领盛情就是了。”说着,便匆匆要走。辛修甫等都要到船上送他,秋谷拦阻不住,只得自己先回去,嘱付了陈文仙几句话儿。陈文仙也要送到船上,秋谷便同陈文仙同坐一辆马车,星飞电转的赶到常熟轮船码头上。秋谷是自己雇的一号快船,兼雇轮船拖带。当下秋谷同陈文仙上船坐下,刚刚讲得几句话儿,早见岸上远远的两辆马车,风一般的赶到秋谷船边焦下。正是:

将离赠别,佳人南国之思;寸草春晖,游子天涯之感。

不知章秋谷此去何日再来,请看下回便知分晓。





第八十二回 送萧郎南浦赠将离 返故乡天涯留别恨





且说章秋谷刚刚同着陈文仙上得船去,早见岸上两辆马车飞也似的赶来,秋谷知道是辛修甫等赶来送行,便自己跨出船头拱手相迎。辛修甫和陈海秋、王小屏上得船来,秋谷便让他们进舱坐下。陈文仙见了,想要回避进去,秋御叫道:“我们都是知己朋友,你过来见见不妨。”陈文仙听了,便回过身来,慢款湘裙,轻移莲步,低着头向辛修甫等三人一连道了三个万福,辛修甫也作揖相还。陈文仙道过万福,便低头立在一旁。辛修甫等偷眼看时,只见他体态依然,丰姿如昔,只身上穿着一身玄色衣服,曳着一条玄色长裙,淡扫蛾眉,薄施脂粉,铅华不御,芳泽无加;头上只带着一支珍珠押发,一个珠骑心簪,千干净净的没有一些儿珠翠,低眉敛袖的立在那里,不笑不言,竟没有一些儿荡逸轻扬,全是一派的大家丰范。辛修甫见了,暗暗地十分赞叹。陈文仙略略的站了一回,便也转身进去。王小屏料想章秋谷和陈文仙一定还要说几句体己的话儿,我们不要在这里讨他的厌,便和辛修甫、陈海秋使一个眼色,大家立起身来告辞,彼此打了一拱,辛修甫等三个人便自上岸去了。

这里章秋谷和陈文仙两个人你看着我,我看着你,一言不发。陈文仙只觉得各种酸甜苦辣的滋味一古脑儿都并到心上来。正在这个时候,猛然听得船上“呜呜”

的两声汽笛,秋谷便道:“轮船将要开行,你上岸回去罢。”陈文仙听了勉强点一点头。章秋谷便扶着陈文仙上了码头,说一声:“你自家保重。”踊身一跃,早已跳上船头。船家把缆绳带在拖船的后面,“呜”的一声,轮船已经开动。章秋谷立在船头上,眼睁睁的看着陈文仙;陈文仙坐在马车里头,也眼睁睁的看着章秋谷,直看到烟波浩渺,人影模糊,陈文仙方才懒懒的回去。这且按下不题。

只说章秋谷立在船头上,直至望不见文仙的影儿,方才叹了一口气进舱坐下。

真个是风情遐思,凄凉南浦之歌;别恨离愁,辜负高唐之梦。那上海到常熟本来水路不多,不到五更已经到了。

章秋谷离家已久,也觉得要紧回去看看家里头的情形,便把船上的行李都交给那两个家人,自己便跳上岸去,赶到家中,见了太夫人,又见了他夫人张氏。秋谷见太夫人身体十分康健,心中自然欢喜。太夫人见秋谷回来,心中也十分欢喜,问问这样,问问那样,又把自己家里头几个月里头的事情,夹七夹八的告诉了秋谷一遍。秋谷在家里头休息了两天,不免出去到各亲友那里去应酬一番,一班亲友也有上门来探望的,也有备酒和他接风的,倒把个章秋谷忙了好几天。秋谷自回之后,也没有什么事情,只陪着太夫人讲讲闲话,叙叙家常。他夫人张氏,秋谷本来原是因他才貌平常,所以和他不合。幸而他这位夫人性情极是平和,脾气也还柔顺,倒深得太夫人的欢心。章秋谷听了太夫人的解劝,便也渐渐的两下和睦起来,所以秋谷在家,倒也狠不寂寞。

一连过了十余日,太夫人对秋谷讲起佃户的抗租不完来,秋谷道:“这班种田的人,虽然种了几亩田,却往往穷得衣不遮身,食不充腹,想起来也狠可怜。若是欠得不多,不如听他去罢。”太夫人道:“若是穷佃户欠租不完,自然不必去问他追讨。这个欠户,听说狠有钱的,靠着他儿子的丈人是县里头的差役,作威作福的狠不安分。种了我们五十几亩田,三年的工夫一个大钱都不肯完,你想世上那有这般道理?要是一班佃户,大家都学着他的样儿不肯完租起来,叫田主人怎么样呢?”

秋谷听了勃然大怒道:“原来就是黄阿润这个混帐东西,去年他没有还租,我就要把他送县押追,一向只道他是个贫户,那晓得他竟敢倚着一个差役的靠山,抗不完租,这还了得!明天待我自己去拜常熟县刘大令,托他立刻提了黄阿润,押追欠租就是了。”太夫人道:“只要他好好的把租还了出来,或者先还一半,也就罢了,不必一定要把他送官押追,他们乡里人究竟吃不起惊吓。”秋谷听了答应一声,便把收租的帐目查了一查,见欠租不完的,十个里头差不多倒有四五个,不觉怒道:“这都是大家看了黄阿润的样儿不肯完租,要不好好的办他一下子,明年的租就不用收了。”想着,便把几个欠户的名儿都开了下来。

到了明天,章秋谷换了衣冠,坐着轿子去拜那位常熟县刘大老爷。投进帖子等不多时,只听得“吱喽喽”的一声中门大开,一个执帖家人手中举着帖子,说一声“请”。秋谷的轿子便直进二堂歇下。执帖家人斜着身子,把帖子举得高高的在前引道,把秋谷让到花厅坐下。等不多时,这位刘大老爷便在里面走了出来,秋谷和他行过了礼,叙了几句寒温,便提起佃户欠租的事来,要请他出票提人。刘大老爷听了,一口应允,并不作难。秋谷不免和他说了几句客气的话儿,便端起茶碗来喝了一口,起身告辞。刘大老爷送到轿旁,打过一拱,便走了进去。

章秋谷的轿子便一直抬出大堂来。刚刚抬出暖阁,早看见对面飞也似的来了一乘青布小轿,一直抬到大堂上,便停下来。轿子里头走出一个少妇,不先不后,刚刚和章秋谷打了一个照面。章秋谷早吃了一惊,只见这个少妇风目凝波,蛾眉锁翠,衣裳缟素,举止端详,狠像个大家命妇的风范,却是眼中含着一泡珠泪,面上又显着一派怒容,低着了头直走出来。章秋谷看了心上不由的疑惑起来。暗想这样的一个人,狠像一个贵家命妇,怎么会无缘无故的跑到这个地方来,难道和人家打什么官司不成?看他脸上的那付形容,明摆着一腔冤愤,也不知他究竟是什么事情,不如在这里略等一回,看看他的情形,若是可以相助的地方,我也不妨帮他一下子。

想着,便叫轿夫略停一停。秋谷坐在轿内也不出来,只仔仔细细看那少妇的举动。

只见那少妇后面还跟着两个差役,慢慢的走过来。那少妇回过头来问那两个差役道:“县大老爷在那里,快些儿请他出来。”那两个差役听了微微冷笑道:“你说得好容易的话儿,县大老爷是一方之主,也是轻易见得的么?你既然来了,且到官媒那里等候一回再说。”那少妇听了着急道:“既然县大老爷没有坐堂,为什么你们又把我撮弄到这个地方来呢?”一个差役又冷笑道:“大老爷既然提你,自然有坐堂的日子,你只好好的等着就是了。”那少妇听了更力着急道:“依着你们这般说法,要等到什么时候呢?”一个差役又道:“那我们也不知道,大老爷高兴几时坐堂理事,就是几时坐堂理事,我们当差役的那一个敢去催他?你只到官媒那里去好好候着,自然有你一个快活。”那少妇听了差役的口风不对,不觉心中大怒,只见他抬起头来厉声说道:“你们两个嘴里头放的都是什么屁儿,我一个寡妇,你们无缘无故的平空把我叫到这个地方,如今县大老爷又不肯坐堂,倒反要把我押起官媒来。那官媒家里是好好的人可以住的么?你们瞎了眼睛,难道把我也当作那班没骨气的人不成?”一面说着,虽然声色俱厉,却止不住两行珠泪直挂下来。连忙别转头去,自己拭干了眼泪,蛾眉倒竖,凤目圆睁,又高声对着那两个差役道:“到底怎么样,你们只请县大老爷出来就是了,若要把我押到官媒那里去,你们不要想昏了头,我是死也不去的。”两个差役听了,你看着我,我看着你,彼此做了一个眼色,一个差役便呵呵的笑道:“伙计,你听听,好大的口气。老实对你说了罢,大老爷的吩咐,去不去由不得你。你愿意去也是要去,你不愿意去也是要去。

我劝你还是好好的走罢。“

章秋谷看了这样的一种情形,又听了那般的一番言语,虽然还没有知道是怎么一回事情,心上早瞧料了五六分,不由得怒从心起,便自己走出轿来,一直走到那少妇身旁站定,睁开两眼看着那两个差役。那两个差役抬起头来,见平空来了这样的一个人,心上虽然有些诧怪,却也还不在心上,只恶狠狠的对着少妇说道:“怎么样,大老爷的话儿难道你竟敢不听么?怪不得祁乡绅对着大老爷说你是个泼妇呢。”

那少妇听了不慌不忙,冷笑一声道:“原来就是祁八这个畜生干出来的事情。好,好!”那两个差役道:“好也罢,歹也罢,只请你快快的走罢,在这里挨一会儿也当不了事,”那少妇听了忽然把眉头一皱,大声说道:“你们真要把我押到官媒那里去么?”那两个差役冷冷的说道:“岂敢,难道是和你取笑的不成?”那少妇忽地咬一咬牙齿,顿一顿金莲,“飕”的一声从衣袖里头掣出一把明晃晃的小刀,望着自己喉咙便刺。两个差役见了,只吓得灵魂出窍,毛骨皆酥,口中一个字儿都喊不出来,两个人四只脚儿就如钉在地下生了根的一般,一步也走不上去。大堂上一班家人、差役见了这般形景,一个个也都大吃一惊,连忙七手八脚的赶过来想要去夺,那里来得及。说时迟,那时快,章秋谷这个时候已经立在那少妇身旁,见他一转眼的工夫掣出刀来望着自己颈中便刺。饶你章秋谷这般胆大,由不得也吓出一身冷汗来。到了这个间不容发的当儿,那里还顾得什么男女的嫌疑,疾忙抢进一步,轻舒猿臂,只一把把那小刀夺了过来,凭我章秋谷这样的眼明手快,那刀锋已经刺入喉咙约有一寸多深,血花飞溅,一个身体软瘫下来,坐在地上动弹不得。幸而还是章秋谷抢得快了些儿,那刀锋虽然刺进喉咙,没有割破食气两管,不至于有伤性命,却一时间怒气攻心,刀疮迸裂,鲜血直喷出来,晕了过去。正是:

邹衍下狱。天飞六月之霜;齐妇含冤,泪迸三年之血。

欲知这位少妇究竟是什么样人,性命如何,且看下回便知分解。





第八十三回 风凄繐帐泣凤悲麟 月冷空房鸾孤鹄寡





上回书中正说着那位少妇在大堂上晕了过去,但是这位少妇究竟是个何等样人,为着什么事儿,要弄到一时短见,慷慨轻生?在下做书的都没有讲得明白,就是这样糊里糊涂,没头没脑的一来,看官们一时间那里弄得清楚,如今列位看官且休性急,待在下做书的一一说来。

只说那个时候,常熟县有一位致仕的乡绅,姓钱,叫做钱韬叔,是一个榜举人的大挑知县,做过几任州县,倒也狠有政声。无奈读书人出来做官,总带着那一点儿先天的书毒,一心想做好官,不肯巴结上司,上司因此和他不对,借着公事上的一些不合,便把他撤任察看,把这位钱大老爷只气得一个发昏章第十一,索性告了个假不做官了。回到常熟地方,自己修一个小小的花园,种竹养鱼,栽花莳药,一天到晚的只在自己的花园里头吟风啸月,饮酒赋诗。虽然地方不大,却也房廊曲折,花木萧疏,榆柳两行,梨桃百树,布置得狠有些儿丘壑。

钱大老爷夫人黄氏早年就死了,钱大老爷伉俪情深,不肯续娶。黄夫人生了一子一女,儿子名叫康寿,女儿名叫纫秋,都生得目秀眉清,唇红齿白,真是两株玉树,一对璧人。这钱纫秋长到十七岁上,更长得如花如玉,倾国倾城,冰雪为肌,琼瑶作骨;更兼性情和顺,资质聪明,对着钱大老爷真是千依百顺的,从不肯叫钱大老爷生气。钱大老爷钟爱的这个女儿,真个也像是掌珠拱璧一般,自己教他读书识字,又请了一个绣娘教他女工刺绣。这位儿小姐一学就会,一会就精,不上五六年的工夫,钱小姐早已女工针刺无一般不会,诗词歌赋无一样不精。到了十七岁上,钱大老爷便和他对了一头亲事,是个本城贡生的儿子,名叫王芝宇,家况甚是贫寒。

这王芝宇却生得白面长身,一表非俗,更兼天资卓越,学问渊深。钱小姐嫁了过去,自然意合情投,一双两好,闺房之乐,甚于画眉。这也不必去提他。那知钱小姐嫁了王芝宇不及一年,钱大老爷忽然生起病来,医治不好,呜呼哀哉死了。钱小姐姊弟两个的哀痛迫切,也不必去说他。

又过了几年,常熟县城内忽然倒了一家有名的钱庄,钱大老爷本来是个清官,一生所积的宦囊,一古脑儿都存放在这爿钱庄里头,如今被他倒得干干净净,那钱庄上的经理也逃得无影无踪,一个大钱也要不回来。钱康寿和钱小姐也无可如何,只好由他。从此之后,钱康寿便有些度日艰难起来,勉强敷衍了几年,越发支不住,只得把自己住的房屋和花园典给本城的祁彦文祁侍郎家,典了几千银子,钱康寿便捐了一个功名,到湖北去候补。王芝宇本来是个寒士,家无担石之储,囊无一钱之蓄的,以前钱家有钱的时候,还可以常常的通融借贷;如今钱家穷了,王芝宇不免也更加拮据起来。若单是穷苦些儿也还罢了,谁知道祸不单行,福无双至,大凡天心最妒忌的是男子一个“才”字,女子一个“色”字。所以古今来往往才士坎坷,红颜薄命。如花美眷,消不得似水流年;绮思风情,辜负了良辰美景。十个里头倒有九个都是这个样儿。这还不必去说他,更有一件最犯忌的事情,便是那倾国名妹,嫁着了个风流才子;江南名土,娶着了个燕赵佳人。像这样的一班人物,上天却断不肯轻轻易易的放过了他,一定要千万百计的想着法儿把他磨折得九死一生,方才肯罢。

看官,你想王芝宇和钱小姐这样一对才貌相当的夫妇,一个具着这样的清才,一个生着那般的丰貌,那里能够就是这样安安稳稳的过去?平空的王芝宇又害起病来,急得钱小姐烧香拜佛,问卜求医,没有一件法儿没有想过,那里有什么用处?

不上半个月,把一个王芝宇又送到阎王家去了。钱小姐呼天抢地,泣血捶心,几次三番的哭晕了去。家里头的人见了慌作一团,连忙七手八脚的把他救醒。

看官,可知道这一边王芝宇地下修文之日,正是那一边钱康寿玉楼赴召之时。

原来钱康寿到了湖北候补了几年,没有得着一个差使,心中十分懊闷,得了病又没有好好的医生调治,不上几时,也跟着王芝宇一起儿往阎王家去了。钱小姐得了这个信息,更加痛不欲生,屡次的想要自尽,都被一班人看守得牢牢的,展不得手脚,也是无可如何。刚刚事有凑巧,正在这个当儿,又接得钱康寿夫人一封来信,说钱康寿的棺木现在还停在湖北省城一个古庙里头:要想扶柩回来,却一个大钱也没有。

钱小姐看了这封来信,心上更加悲痛,不免又赶到王芝宇灵前去痛哭了一场。哭过之后,钱小姐定一定神,心中暗想:“兄弟的棺木现在停在湖北,路远迢迢的又没有盘费,一时那里搬得回来?虽然有几家族中叔伯可以托他们料理,但是如今世上的人都是势利不过的,听得钱康寿死在湖北,身后萧条,一个个早巳躲得远远的,恐怕过了穷气,那里还肯来帮你们的忙?想想姓钱的一家,如今只剩了自己一个,自己不去料理他的灵柩回来,还有那一个肯来多管这般的闲事?”想着便把殉节的念头撇过一边。盘算了一回,想着钱康寿没有儿子,少不得要把族中的子侄承嗣,这是第一件大事,更兼搬取灵柩办理丧葬,免不得大大的要一笔经费,这一笔钱,一时又从那里去打算呢?呆呆的想了一回,忽然想起自己家里头的房子现在典给祁彦文住着,这祁彦文祁侍郎向来为人狠好,不如我自己亲去见他一趟,问他借几百银子,一起并在典价上算,料想他没有什么不肯的。况且靠屋借钱,向来就有这个规矩,不是我一个人闹出来的新样儿。想着,定了主意,便换了一身素服,雇一乘轿子,竟到祁侍郎大门上来。这个时候,王芝宇已经死了三个多月,一切丧葬的事情已经办妥,所以钱小姐一心一意要办兄弟的事儿。

轿子到了门外,门上人问明来意,便放他进去,见了祁侍郎的夫人,含着眼泪把钱康寿死在湖北、棺木不得回来的情形细细的说了一遍,要问祁侍郎借五百银子。

祁夫人见他神色凄凉,言词宛转,心上也不觉侧然,便请了祁侍郎进来见了钱小姐,和他说了。那知这位祁郎本来是个财迷,一个大钱在他手里头拿出来也要惦个分两,如今听得钱小姐一开口就是五百两银子,倒把他吓了一跳,口中支支吾吾的不肯答应。钱小姐便对他说道:“这所宅子连着后面的花园,当初有人估价原是值一万银子,如今府上典价止有六千银子,再加上五百银子,也不过六千五百银子,有房屋在这里作低,料想没有什么不妥当,请只顾放心就是了。”祁侍郎听了沉吟一回道:“五百银子的事情似乎数目大了些儿,一时也不能决定,请隔几天再来问信罢。”

钱小姐听了便起身告辞,先自回去。

祁侍郎见他走了,一个人坐在书房里头以心问口,以口问心的足足踌躇了大半天的工夫,方才打定了主意道:“他虽然向我借钱,这所房子却不止这个价钱,我只管借给他就是了。”想着便走出来,叫帐房先生先去打五百银子的银票。那位帐房先生答应一声,正要走出去,忽听得外面有人说道:“要五百银子做什么?”祁侍郎抬头一看,只见一个獐头鼠目的人在外面大摇大摆的走了进来。不是别人,原来就是祁侍郎的族侄,叫做祁祖元。做过一任福建道台,到任的时候,正碰着要和外国人划定地界,办起事来左右为难。要是帮着外国人和百姓为难罢,百姓大家不服,万一个聚众闹事,闹了个什么乱子出来,不是顽的;要是帮着百姓和外国人过不去罢,如今的世界都是外国人的势力圈,不但外国人不答应,做官的人担当不起,就是上司也要不答应的。祁观察到任之后,看了这样的一个情形,好像个猴儿抓着了一把屎的一般,那里摆布得来?更兼外国人天天的朝着他絮聒,只说着他不肯出力,纵容百姓们和他为难,意思里头十分嗔怪着他,只把个祁观察急得手足无措,想不出一个两全其美的法儿。就有一个他自己幕府里头的人和他出了一个主意道:“这件事情,横来竖去总是不讨好的。要帮了他们外国人办事,不但坏了功名,而且还要受那万人的唾骂,不如索性转过头来,一味的帮着百姓和外国人硬挺,外国人一定不肯答应的。上司见外国人和我们不对,自然要想个法子把我们调到别处去,那时既躲过了这一场棘手的事情,又可保全了自己的声誉。人家说起来,只说是为着硬帮百姓和外国人不合,方才调到别处去的,这样一来岂不是名利双收么?”

祁观察听了,觉得他这一番话儿倒也狠是不错,仔细想了一想,连连的自己点头。暗想这件事儿果然是办不好的,与其帮着外国人。弄到后来仍旧是一个丢官,不如还是咬着牙齿帮着百姓和外国人为难,丢掉了这个功名,也觉得荣耀些儿。想罢,心上究竟还有些舍不得这个功名,又问着那个幕府道:“我们这样的办法,可以保得不至于丢官么?”那幕府大声说道:“你要我保着你一定不丢官,那是我保不来的。不过依着我的意见想起来,做上司的碰着了这样的事情,要顾全外国人的面子,无非是一个调省察看,至多也不过是一个撤任罢了。只要等这件事情冷了些儿,那时仍旧可以出来的,虽然暂时蹉跌了一下子,却得了个天字第一号上好的名声,你道我这个主意可好不好?”祁观察听了心上十分欢喜,便依着他的主意,处处帮着百姓和外国人为难。果然外国人心中不对,一个电报打到福建省城去给闽浙总督周制军,要请周制军参他的官。周制军便上了一个摺子,把祁观察参了个实降两级,不准抵销,立时挂出牌来,把祁观察先行撤任,派员接印,赶算交代,倒忙碌了一番。这一来,只把这位祁观察气得个脑胀头昏,要死不活拍着桌子,把周制军大骂了一顿,又要找那位幕府和他拚命。正是:

孤鸾寡鹄,结幻梦于三生;玉碎珠沉,子浮生于一瞬。

不知后事如何,且看下回分解。





第八十四回 办交涉庸奴降秩 谄大官观察欺贫





且说祁观察得了周制军参他降级的信息,只把他气得一个发昏。在祁观察本来的意思,原是听着那位幕府的话儿,有心取巧,明晓得个这当儿事情十分难办,所以故意充个好汉,帮着百姓和外国人为难,外国人不答应起来,预备着上头把他调任别处,或者把他调省察看;就是再顶真些,也不过一个撤任罢了,只要等这件事儿的风头过了,上头一定要大大的把他调剂一番。那知人有千算,天有一算,偏偏碰着了这位周制军也不把他调任别处,也不把他调省察看,单单的把他降了两级,好好的一个道台,降了一个通判,你叫他如何的不气?

闲话休提,只说祁观察自从降官之后,便和那位幕府吵闹,说他出错了主意,那位幕府朝着他呵呵冷笑道:“你不要这般模样,幸而我教了你这样的一个主意,方才落得这样的一个收场。若凭着你的主意拼命的巴结外国人,做他的奴才,只怕百姓们大家不服,鼓噪起来闹了个大大的乱子,那时你又怎么样呢?如今你虽然降了官,却得了个绝好的声名,将来总可以找个出路,你不感激我教你的主意也还罢了,还要平空的和我吵闹起来,这不是笑话么?”祁观察听了这一番说话,哑口无言,一句话也说不出来,只得收拾收拾回到常熟,做起绅士来。

这常熟县分本来是个小地方,没有什么大绅士,祁彦文虽然是个侍郎,却向来不肯干预公事的。这位祁观察回到常熟,便干预起地方上的公事来。不但民间词讼争论的事情他要插进去帮个忙儿,就是地方上的公款,常平仓里头的积谷,他也要千方百计的想着法儿出来混闹。地方上有了这般一个无耻的绅士,就有许多卑鄙龌龊的刁生劣监,挺身出来做他的走狗,在外面招揽词讼,把持衙门,无事生风,招摇撞骗,把常熟一县的人弄得一个个叫苦连天,恨入骨髓。刚刚这个当儿,两江总督刘制军和两广总督寿制军连衔保奏祁祖云老成练达,才识兼优,便开复了原职。

祁观察到了这个时候,当了几年绅士得着了滋味,觉得当这个绅土,比出去做官的进款还要多些,便立定主意不出去做官,也不进京引见,只拼命的在本地想着法儿搜括银钱。这个时候,正碰着各省举行新政,房屋田地都要加捐,祁观察借着这个名色,假公济私,行出许多新法,把这班百姓捐了又捐。捐出钱来,开办地方上的新政,又都是祁观察一个人经手,凭着他怎样中饱私囊,敛钱肥己,那一个敢道一个字儿?

这位常熟县刘大老爷又是一位不理民事的糊涂虫,他衙门里头有个通房的丫环,年纪止得十八岁,却生得山眉水眼,皓腕纤腰,刘大老爷收他做了通房,便想把他升做姨太太。不想刘大老爷在家乡带来一个侄儿,到了任上就叫他管理帐房。这位侄少爷年纪止有二十三四岁,翩翩年少,顾影自怜,不知怎样的一刮两刮,和这个丫环竟刮上了。偏偏的事情不巧,那一天两个人正掩在书房里面轻轻悄悄的??话,不料刘大老爷正在外面走过,听得书房里面有男女嬉笑的声音,便掩着身子从门缝里张了一张,不觉心中大怒,那一把无明业火从脚心底下焰腾腾的直冲到顶门上来,按捺不住,当时就要发作。忽然转一个念头,想道家丑不可外扬,这件事情要是闹了出来,别人只说我没有家教,所以闹出这样的事来,我的面上怎么下得去?想到这里便勉强忍住了。悄悄的走了进去,一个人坐在签押房里,想那处置的法儿。想着:“这个贱人我何等的抬举他!想是他嫌我年纪大了,不愿意跟我,所以做出这样的事来。这个小畜生尤甚可恶,他明晓得这个人是我收过房的,竟近起禁脔来。”

心上这般想着,越想越气,立刻把那位侄少爷叫了进来。反转脸皮,叫他收拾行李立时回去。这位侄少爷见了这般声势,明知道是那件事儿发作,不敢多讲,只说帐房里头还有许多经手的事情,恐怕一时不能就走,要等料理明白了方才好交代别人。

刘大老爷大声说道:“不用你这般小心,帐房里头不是你一个人,你只顾回去就是了,给我立刻动身,不许耽搁。”这位侄少爷听了无可如何,只得拜别了刘大老爷,垂头丧气的自家回去。

刘大老爷撵走了侄儿,把这个丫环叫到面前痛打了一顿,叫了一个家人、一个仆妇进来,叫他们带着这个丫环,到上海去卖给堂子里头。大家听了面面相觑,不晓得这位老爷是什么意思,这个仆妇便上前说道:“禀老爷的话,仆妇的儿子高福已经三十岁了,还没有成过家,可否求老爷的恩典,抬一抬手,不要卖他到堂子里去,赏给仆妇做了儿媳妇罢,老爷要卖多少钱,仆妇情愿照数缴上来。”刘大老爷听了,心中大怒,拍着桌子大声说道:“你晓得什么,我正为这个贱人没有良心,所以要把他卖到堂子里头去,有意叫他受些磨折,吃些苦头,你们不准多话!”这个丫环听得要卖他到堂子里去,只吓得芳魂飞散,珠泪纵横,跪在地上哭哭啼啼的苦苦哀求。刘大老爷铁青了脸,一言不发。这一闹,闹得里头那位夫人也走了出来,也劝着刘大老爷道:“你心上不喜欢这个人,好好打发他嫁人就是了,何必一定要把他卖到堂子里头去呢?这样的事情不是我们做官人家做的,譬如做个好事,把他放了出去罢。”刘大老爷冷笑道:“你不要来多管闲事,这件事情我主意已经打定,凭你什么人来说也是不中用的。”这位刘夫人本来性情懦弱,衙门里头的事情做不得主,听了刘大老爷说得这样的斩钉截铁,便也不敢多口,凭着他去胡闹。当下刘大老爷立刻打发这一班男女动身出门,临出门的时候,还再三再四的吩咐他们一定要卖到野鸡堂子里去,卖了二百五十块钱,刘大老爷方才出了这一口恶气。

看官,你想这个卖良为娼、买良为娼,是照例禁止的,做地方官的人碰着了这般的案子,一定要把犯罪的人重重的惩办他一下,以儆后来。如今这位刘大老爷非但不能禁止,倒反自己把好好的良家女子卖到堂子里去为娼,你想如今做官的人还有什么交代?

闲话休提,只说刘大老爷到了常熟县任上,不到一年就闹了一起诬良为盗的案子。本地的绅士大家联名出了公呈,到江苏巡抚丞中丞那里去告他。朱中丞想要把他撤任,刘大老爷听得这个消息十分着急,便求了祁观察和他设法。刚刚祁侍郎的朱中丞是同年,祁观察便不顾死活的求了祁侍郎的一封信给朱中丞,着实和刘大老爷讲了几句好话,朱中丞接了祁侍郎的信,便把这件事情搁了下来,只当没有这件事儿,刘大老爷方才放下心来。白此以后,感激这位祁观察就如亲生父母一般,差不多常熟一县的公事,都要听着这位祁观察的指挥。以前祁观察在地方上把持公事,刘大老爷心上还有些不以为然,自从经过了这一番,祁观察做起事来越发顺手,没有一些儿阻碍的地方。祁侍郎见他在地方上作威作福,也着实劝过他几番,见他不听,也只得罢了。

这一天也是合当有事,祁侍郎正要叫帐房先生反打银票,恰恰碰着了祁观察进来,问起为什么要打银票,祁侍郎把钱小姐的事情和他说了。只见他把眉头一皱道:“天下的事情那有这般容易!他家里头死了人,与我们什么相干?要是典房子的人大家都要找起价来,那里找得尽许多?”祁侍郎听了这一番说话,心上又舍不得那五百银子起来,便道:“你的话儿虽也不差,但是我叫他隔几天来听信的,倘若他一定要在房屋上靠借五百银子,却叫我怎样的回覆他呢?”祁观察听了拍着胸脯道:“不要紧,这件事情交给我就是了。就是靠屋借钱,也要两下情愿,难道好硬借不成?”祁侍郎听了,虽然觉得不甚妥当,但究竟心上蝎蝎螫螫的想要省这五百银子,便依着祁观察的话儿。

等钱小姐来了,祁侍郎也不请他进去,自己也不出来,只请他在厅上坐着,叫人请了祁观察来,见了钱小姐,一口回绝。钱小姐见了祁观察那般神气,大模大样的目中无人,心上早有了三分不快,便问他道:“靠屋借钱是我们这里的常事,府上又不是拿不出钱人家,为什么不肯通融一二?”祁观察道:“靠屋借钱也要两家愿意,我们不愿意借,便怎么样呢?”钱小姐听了,怫然不悦道:“既然府上不愿意,这个房屋却是姓钱的产业,如今我要请府上外加五百银子的典价,那也算不得什么。”祁观察冷笑道:“当初典屋的时候说明六千银子典价,原是两下愿意的,如今为什么平空的又要加起价来?”钱小姐道:“我也不是无故加价,这里头也有一个不得已的苦衷……”说到这里,正还要说下去,不料祁观察早立起身来,脸上现出一付不耐烦的神色,口中说道:“不必多讲,我也没有工夫和你讲话。我只晓得出了钱典你们的房子,并没有一些儿亏负你们的地方,至于什么借钱不借钱,加价不加价,我一概不管。你还是快些回去罢,年纪轻轻的妇人,来去出头露面,也狠不便的。”钱小姐听得祁观察这一番说话一味的不讲道理,只气得面罩浓霜,花容失色,不由得冷笑一声道:“怎么平空的说出这般不讲理的话来,可不是奇事么?”

祁观察听了也怒道:“我好好的和你讲话,是赏你的脸,你倒连我都冲撞起来,你说我不讲理,我就不讲理,看你可有什么法儿?劝你趁此早些回去,还好保全面上的光辉,如若不然,那就莫怪得罪了。”钱小姐听了,这一气非同小可,一时怒发起来顾不得什么,立起身来大声说道:“天下的人讲天下的理,难道你们做官的人就好不讲理的么?枉了你们还算都是世家子弟,原来一个个都是些不成材料的草包!”

祁观察听他骂得尖利,也不由得心中大怒,高声喝道:“你是什么东西,还不给我快些滚出去,这个地方是容你放肆的么?”钱小姐听了,转觉得有些好笑。便又冷笑一声道:“这个地方是我们姓钱的产业,你既然住了我们的房子,我和你便是宾东,难道你这个地方是皇上的紫禁城,我们到不得的么?”正是:盲风怪雨,摧残上苑之春;叱燕嗔莺,惆怅金铃之使。

不知后事如何,且听下回分解。





第八十五回 负奇冤烈女骂奸雄 溅热血公堂飞白刃





且说钱小姐在祁侍郎家厅上,把祁观察着实抢白了一番,祁观察只气得白瞪着两只眼睛,一句话都讲不出来,只一叠连声的叫道:“来,来,来,来,来!”就这几声里头,早有五六个家人在外面走进来,垂着手站在一旁。祁观察把手指着钱小姐道:“快给我把这个泼妇撵出去!”众家人听了,面面相看,不敢动手。钱小姐听了直气得浑身发抖,心肺俱伤,对着那一班家人道:“你们既然住了我的房子,我就是你们的房东,你们那一个敢动手?”说着顺手取过一个茶碗来,咬牙切齿的,对着祁观察劈头就掼过去。祁观察不及防备,吃了一惊,疾忙把头一侧,只听得“飕”的一声,一个茶碗从耳朵旁边飞了过去;又是“豁啷啷”的一声,茶碗落在地上打得粉碎。祁观察头上身上,却淋淋漓漓的泼得一身的茶。钱小姐不等他开口,赶过去把天然几上的一个大磁瓶用力一推,推在地下,也打得粉碎。祁观察急得双脚乱跳,对着那班家人大骂道:“你们这班没用的奴才,叫你们撵一个人都撵不掉,倒反容他这般放肆起来,你们到底当的什么差使?”说着,便自己抢步上去,揎拳掳袖的想要动手。

那位祁侍郎本来是躲在里面听他们讲话的,如今见闹得不成体统,连连顿足道:“糟了,糟了!”急急的走出来对着祁观察把手乱摇道:“不要动手,有话好好的讲。”这个时候,钱小姐气到无可如何,已经把天然几上的东西,一古脑儿推在地下。见了祁侍郎出来和他讲话,便道:“前天我们当面讲得好好的,为什么今天又要变起卦来?”祁侍郎道:“如今事已如此,不必说他。你只顾先请回去,我自然有个安顿的法儿。”钱小姐听了,头也不回一直走了。这里祁观察见他走了,也气得目瞪口呆,拍着胸脯道:“好一个利害的泼妇,我有生以来没有见过这样的人。”

祁侍郎见一个花瓶和两个帽架都跌在地下跌得粉碎,觉得十分心痛,口中却说不出来。大家呆了一回,方才商议这件事儿,依着祁侍郎的意思,就依着他借给五百银子,并在房价上头核算。祁观察那里肯依,道:“我们平空的被他这般糟塌,把厅上陈设的东西都打一个稀烂,难道罢了不成?若不好好的给他一个利害,我这个‘祁’字也不姓了。”祁侍郎起先还劝他不要多事,祁观察不肯,只说:“闹了什么事情出来,有我一个人承当,决不牵到二叔身上。二叔只顾放心就是了。”祁侍郎听了也只得由他,暗想:“凭着他去胡闹,我乐得省下五百银子。”想着便不去管他。

祁观察立刻坐了轿子去拜常熟县刘大老爷,只说这个王钱氏是个女光棍,要想平空讹诈银钱,要他出签提人,提到了也不要坐堂审问,只把他押在官媒那里吓唬他一下子,叫他以后不敢再来讹诈。这位刘大老爷听了祁观察的话儿,糊里糊涂的不问情由,便派了两名差役去立提王钱氏到案审问。那两名差役便跑到钱小姐那里去,大呼小叫的逼着钱小姐要走。钱小姐不慌不忙,问他们究竟为的什么事情。两个差役又不肯和他说,只逼着钱小姐立刻就走。钱小姐虽然心上不怕什么,却明晓得祁观察不是个好惹的人,如今自己得罪了他,恐怕他串通了常熟县,有心和自己为难。便暗暗的取了一把小刀放在袖子里头,预备见了刘大老爷,把自己的苦衷对他哭诉一番。那里晓得到了常熟县堂上,既见不着刘大老爷,又要把他押到官媒那里去,一时急气攻心,便拔出小刀想要寻个自尽。幸而遇着了章秋谷,把他手中的刀夺了下来。

当下章秋谷见钱小姐晕了过去,连忙指挥众人取过一方白布,先紥了他颈上的刀伤,又叫取过热水来灌了一回,渐渐的醒转来。一面又叫自己的家人赶回去取了刀伤药来,替他敷治;一面指着那两个差役冷笑道:“你们这两个奴才,几乎闹出人命交关的事来,好得狠,这才算会当差使呢!”那两个差役本来已经吓得昏了,如今被章秋谷骂了几句,看看章秋谷这般气派,料想是有些来历的,不敢开口。章秋谷回过头来,问着那大堂上的一班人役道:“这个人姓什么,为的什么事情,怎么无缘无故的要寻起自尽来?”那些人役还没有开口,早有秋谷自己的轿夫抢步过来,指手画脚的说道:“这件事儿,我都知道得明明白白,待我细细的讲给老爷听就是了。”说着,便把这件事儿的始末根由,一一的说了一遍。秋谷听了不觉大怒,一言不发,回过身来叫过号房,叫快去请刘大老爷出来,我有话说。号房答应一声,转身进去。不多时便走出来,把秋谷请到花厅。

只见那位刘大老爷慌慌张张的问道:“那王钱氏的刀伤怎么样,可要紧不要紧?”

秋谷微笑道:“方才要不是治弟手快,赶紧把刀夺了过来,等到这个时候,再有一百个也死了。”刘大老爷连连向着秋谷打拱道:“有劳得狠,有劳得狠。”秋谷又微笑一笑,连忙回礼道:“不敢当,不敢当。但是这件事儿,老公祖打算怎样的一个办法呢?”刘大老爷呆了一呆道:“只有且先把他送回家去,随后再讲别的事儿。”

秋谷冷笑道:“这件事儿,本来是祁绅不是,倚着乡绅的势力,在外面鱼肉乡里,欺负平民。老公祖不该听了祁绅的一面之词,冒冒失失的出差提人,几乎闹了个大大的乱子,老公祖以后还要小心些儿才是。”几句话把一个常熟县刘大老爷说得面红过耳,只得答应一声:“老哥的高论不差。”秋谷又说:“那两个差役,作威作福的十分可恶,方才这件事情,就是他们两个威逼出来的,要请刘大老爷惩办他们一下子,也好叫后来的人不敢效尤。”刘大老爷听了一口应允,立刻坐出堂去,传了那两名差役上来,不问情由,每人打了一千板子。秋谷眼见这两个差役打得皮开肉绽,鲜血迸流,心上十分痛快,便也辞了出来。

这个时候钱小姐虽然已经醒转,还有些昏昏沉沉的讲不出话来,刘大老爷已经叫人把他送回家去。章秋谷一路回来,心上甚是不忿,想着要和钱小姐帮个忙儿。

过了几天,秋谷派了一个家人出去打听钱小姐的事情,这个家人出去打听了一回,走回来便一一告诉了秋谷。

原来祁侍郎听得这个消息心上也慌了,便托了人出来和解。钱小姐的刀伤本来不重,这几天的工夫已经平复了五六分,便对着祁侍郎的来人说道:“我知道这件事儿不是他的主意,统通是祁八这个畜生一个人闹出来的事情,将来冤各有头,债各有主,叫他只管放心。但是我的事情,是我的事情;我兄弟的事情,是我兄弟的事情。如今他死在他乡,没有搬柩回籍的盘费,我不给他料理,还有那一个给他料理呢?我以前开口就说要借五百银子,如今仍旧还止要五百银子,把我兄弟的丧葬事情办妥,就算我身上的责成完了,别的事情一概不必说起。”那来人听了,便如一如二的把这一番说话告诉了祁侍郎。祁侍郎倒有心要照数给他,无奈那祁观察手下的一班走狗要讨祁侍郎的好,便七张八嘴的纷纷议论。这个说五百银子是白花掉的,那个又说这房子是钱家的产业,钱小姐虽然是钱家的女儿,却算不得钱家的人,不能听他的说话。祁侍郎本来有些色抖抖的心痛这五百银子,听了众人的说话心上也作不定主意起来。一连议了几天,还没有议决。

章秋谷听了这个信息,心中大怒,便亲自赶到祁侍郎那里打听情形。祁侍郎本来原是认得章秋谷的,如今见了章秋谷的面,觉得有些不好意思起来,口中支支吾吾的说不出一句话儿。秋谷开口便问钱小姐的事情,祁侍郎还没有开口,早有个走狗叫做康长龄的抢着说道:“据晚生看来,这一笔钱老先生可以不必拿出来。就是老先生格外体恤他们,给些丧葬银两,也用不着许多,至多给他一百两银子罢了。”

说着,又有一个走狗叫做经伯成的也插口说道:“清平世界,朗朗乾坤,要都像他们家里死了人就来讹诈起来,那还了得!”一句话还没有说完,早见章秋谷霍地立起身来,剔起双眉,睁开两眼,大声说道:“你们两位这般巴结,替祁府上省了银子,这银子可送给你们两位么?人家家里死了人,没奈何,靠屋借些银子,原是不得已的事情,又不是无故借贷。你们不知道帮衬些儿也还罢了,为什么还要这个一句,那个一句,打他们的破坏?你们的良心何在,天理何在?”几句话说得这两个走狗满面通红,一言不发。

章秋谷又接着说道:“你们可知道祁府上多出几百银子不算什么事情,姓钱的得了这几百银子,却可以大大的办些事情,你们何苦一定要这般的无端拦阻,这是个什么缘故?”说罢,越发把这两个走狗说得无缝可钻,十分难过。祁侍郎见了他们两个这般模样,便插口说道:“他们两位也不过这般讲讲罢了,兄弟今天已经打好了五百银子的银票,正要叫人送过去,老哥请不必生气。”章秋谷道:“并不是晚生善于生气,这件事儿地方上的公论狠有些儿不平,想来老先生也该知道。”说着便起身告辞。祁侍郎送出大门,便拱一拱手,自家进去。

章秋谷走出大门,正要举步,忽见祁观察远远的大踏步从对面走来,章秋谷一见了祁观察的面,就觉得怒从心起,恶向胆生,心上想要过去骂他几句。忽然转了一个念头,暗想不如如此如此,叫他小小的吃些苦头。想着便低着个头,一直走将过去,看看至近,故意把身子一横,一个箭步抢过去,正和祁观察扑个满怀。章秋谷用力一撞,祁观察不曾防备,那里当得住?只听得祁观察口中“阿哟”一声,一个身体就如个皮球一般,直跌出去有七八步远近,仰面一交躺在地下,只把个祁观察跌得浑身酸痛,骨节酥麻,口中哼哼的哼一个不住。章秋谷见了,心上暗暗的好笑,急忙抢步过去,把祁观察在地上扶将起来,口中连连的说道:“得罪,得罪!

对不起得狠。“祁观察被这一跌,只跌得头晕眼花,也看不清楚扶他的是什么人,直至定一定神,回过一口气来,睁开两眼,把章秋谷看了一看。正是:

瑶琴照夜,何来变徵之声;剑气凌云,谁是黄衫之客?

不知后事如何,且听下回分解。





第八十六回 归故里堂上奉慈亲 泛轻舟姑苏逢旧友





却说祁观察被章秋谷撞了一交,撞得昏头搭脑的,一时那里扒得起来?直至章秋谷抢步过去把他扶起,祁观察定了一定神,方才抬起头来看时,认得是章秋谷,知道自己方才跌这一交,是章秋谷把他撞倒的,不觉心中大怒。待要发作几句,却又觉得脊梁上的几根骨头一根根都像跌折了的一般,痛不可当,痛得他弯着个腰,嘴里头哼哼的哼成一片。更兼章秋谷赔着笑脸再三认错,只说:“方才实在没有留心,把尊驾撞了一交,不知跌痛了那里没有?”说着,又连连的自己说道:“实在荒唐得狠,实在荒唐得狠。”祁观察见了章秋谷这样的赔着小心,一时发作不出,更兼背上实在痛得利害,说不出什么话来,只恶狠狠的瞪了秋谷一个白眼。秋谷假意在祁观察背上抚摩几下,口中说道:“可是跌痛了背上么?这都是晚生不好,老先生千万不要生气。”祁观察被他灌了一大饱的米汤,有话也说不出来,只得熬着痛勉强说道:“多承老哥费心,幸而还没有跌伤致命的所在,大约还不要紧。”章秋谷听了,几乎要笑出来,连忙别转了脸,对着祁观察拱一拱手道:“得罪,得罪!

晚生先走一步。“说着,便头也不回的一直走了回去。

祁观察吃了这个苦头,明知道章秋谷是有心撞倒他的,面子上却讲不出来。见章秋谷走得远了,方才一步一步挨了进来,气喘吁吁的一屁股就在椅子上坐下,张开了口说不出话来。祁侍郎和经伯成、康长龄见了祁观察这般模样,大家都吃一惊,问他为什么事儿。祁观察坐着喘了一回,方才把章秋谷把他撞了一交的事情和祁侍郎等说了。又道:“这个小畜生十分可恶,无缘无故的平空把我撞这一交,究竟我和他虽然认得,向来又没有什么冤仇,也不知他为了什么事情。”两个走狗正在恨着章秋谷无故把他们骂了一顿,想要翻他的本,出口气儿,便也把方才的事情和祁观察说了一遍,道:“照这样的看起来,他竟是为着王钱氏的事儿出来打抱不平的。

所以今天跑到这里来先把晚生们骂了一顿,又有意撞了八大人一交。像这样的混帐东西,不给他一个下马威,他也不知道八大人的利害!“祁观察听了连连点头。

自此以后,祁观察和经、康两个人把个章秋谷恨得咬牙切齿,好似那不共戴天的仇恨一般,便千方百计的想要借着别的事儿陷害章秋谷。无奈章秋谷素来安分,又是个有名的旧家,那里想得出陷害他的法儿?依着祁观察的意思,还要叫祁侍郎不要出这五百银子,幸而祁侍郎究竟做人明白,不肯听他的话儿。这是后话,按下不提。

只说章秋谷在家里头住了几时,又有些静极思动起来。刚刚这个时候,贡春树在苏州写了一封信来,要请他到苏州去,说有房屋的事情和他商议。章秋谷见了这封来信,心中大喜,便拿着这封信给太夫人看了一看,说有个朋友请他到苏州去。

太夫人看了觉得心上有些不愿意,便问着秋谷道:“如今已经差不多将要过年,大大小小的人家总有些儿事情要料理料理,难道你要在外面过年不成?”秋谷见太夫人的意思有些不以为然,便慌忙解说道:“就是到苏州去上一趟,也不过几天的工夫,自然要回家过年的。”太夫人听了也不说什么。秋谷又说贡春树和自己的交情怎样怎样的要好,贡春树的看待自己,又怎样怎样的真诚;如今他特地写信相招,一定有什么正事,常熟到苏州又止八九十里路程,若一定不去,恐怕他心上见怪。

几句话把太夫人心上说得活动起来,便点头应允,只叮嘱他早些回来。秋谷大喜,走到自己房中,便叫他夫人张氏和他收拾行李。他夫人听得秋谷又要出门,心上未免有些不高兴,却又不好怎样的拦他,只得把秋谷的衣服行李一古脑儿收拾得停停当当。秋谷叫家人押着行李先上轮船,自己高高兴兴的别了太夫人,坐着轿子出城上船。

常熟到苏州的轮船本来止消半日,差不多一点钟的时候已经到了阊门。秋谷见轮船已到码头,便自己先跳上岸去,寻着了贡春树,旧友相逢,大家自然都十分欢喜。秋谷和春树讲了一回闲话,便问他什么房屋的事情,贡春树和他说了。原来贡春树在苏州有几处房屋,都是租给别人的,有一所护龙街的房子租给一个候补人员做公馆,那知这位候补老爷穷得要死,住了三年工夫,只付了一个月租钱。贡春树知道了这件事情,便自己上门去讨,讨得这位候补老爷急了,便假意对贡春树道:“你不要着急,今天和你算结就是了,你带了房租摺子来没有?”贡春树道:“房租摺子自然带来的。”说着,掏出摺子来,交给这位候补老爷拿了进去。贡春树在外面等不多时,只见这位候补老爷怒气冲冲的走了出来,大声说道:“我的房租都是按月给付的,不欠你们一个钱,怎么你无缘无故的来讨起三年的房租来,这不是个笑话么?”贡春树听了摸不着一些头脑,也大声说道:“怎么,怎么?我这所房屋自从租给你们府上以来,除了收过一个月房租之外,一个大钱也没有见你付过,怎么这会儿说出这样的话来?你不信,只看这房租摺就是了。”那位候补老爷听了,一声冷笑,把一个摺子一直送到贡春树的面前道:“你看,你看!摺子上写得明明白白的,怎么你还是这般说法?”贡春树听了心上十分诧异,便接过摺子来看时,不觉吃了一惊,只见这个摺子果然写得明明白白的,某时付房租若干,某时付房租若干,一个摺子上写得满满的,刚刚付到本年本月为止。照着这个摺子上看起来,果然一个大钱也不欠。贡春树见了,心上恍然大悟,知道自己大意,上了他的当了,却也料不到世界之上竟会有这样奇奇怪怪的事情。要想和他们争论几句,料想无凭无据的事也争不出什么来,倒不觉哈哈的笑道:“算了,算了!我一时冒失,上了你们的当,如今也不必去说他,但是你们府上既然困乏到这般田地,只该和我好好的商量,我也不见得不肯,为什么要做出这般的张智来。”说着也不再去和他们争论,一路哈哈的笑着出来。回到自己寓所,要想一个对付他的法儿,一时竟想不出来。忽然想着章秋谷现在常熟,何不写封信去请他到来,一则借此和他畅叙一番,二则也好叫他出个主意。

当下贡春树把这件事儿和章秋谷说了,要他想个法儿,秋谷呸了他一口道:“这样的小事情,也要来劳动起我来!”正说着,忽然春树的家人走进来回道:“护龙街的韩老爷现在已经委了浏河厘局总办,不日就要到差了。”秋谷听了,便对春树道:“恰好他委了厘差,你的房租可以去向他索取的了。”春树拍着手道:“你不要说得这般容易,收房租是要凭着房租摺子的,如今我的房租摺子被他这样的一来,那里还好去向他要钱?”秋谷道:“你这个人怎么笨到这般田地,难道除了死法,没有活法的么?”春树笑道:“你不要张口就骂我,且请问你这个活法是怎么一个法儿?”秋谷道:“像这样的人也不是有心要赖你的钱,无非到那实在没有法儿的时候,只得老着脸皮和你混赖,究竟并不是他的本心,如今他既然得了差使,料想不至于要赖这一笔钱。但是以前既然有了这样的一层情节,你若要彰明较著的问他追讨房租,恐怕他老羞成怒,脸上不好意思,你只要核计一下,三年的房租统计若干,写封信去问他借一笔钱,不必提起以前的事情,叫他心上自家明白,又彼此不伤和气,你道我这个主意何如?”贡春树想了一想,点头微笑道:“主意呢,果然不错,只是我为什么无缘无故的要落一个问他借钱的名气呢?”秋谷也笑道:“这件事儿只怪你自家不好,一时上了人家的圈套,到了如今还有什么法儿!

你可晓得如今世界上的事情只要有钱,怕什么名气不名气?人家千方百计的想着法儿要借钱,借不到的还多得狠呢!“贡春树听了点头称是,便当时提起笔来写了一张条子,加上一个封套,叫自己的家人送去。隔了一天,果然这位韩老爷叫个家人送了一封回信来,里头装着一张四百块钱的瑞昌庄票,并把贡春树的原信附回。

贡春树核计起来,每月十块钱的房租,三年的房租合起来三百几十块钱,他却送了四百块钱过来,算起来还多几十块钱,春树便和秋谷商量,买了些官礼送他,又送了他一桌官席。这且不必提他。

只说章秋谷在苏州住了一天,便想到上海去看陈文仙,春树苦苦的留他再住一天,秋谷起先不肯,还是春树和他说道:“这里庙堂巷有一个私货,生得曼丽非常,名字叫做阿娟,年纪止得十九岁。那一双眼睛更生得十分秀媚,真个是回眸一笑,百媚横生,直是那勾魂摄魄的兵符,拨雨撩云的照会。你既然来此,不可不去赏鉴一番。”秋谷听了贡春树说得这般好法,心中未免有些不信,便一口答应下来,要看看这个阿娟究竟是怎样的一个人物。

当下章秋谷在贡春树那里吃过了午膳,猛然想起那位东方小松来,便一个人走到小松家里头去,指望要和小松两个人畅叙一番。那知半个月之前,两广总督李制军把他聘请去了,秋谷不觉惘然,只得回过身来,到抚台衙门里头去看那几位亲戚。

原来秋谷有两位亲戚都是太史公,一个姓曾,叫做曾祖述;一个姓邓,叫做邓振邦,现在都在江苏巡抚幕府里头管理摺奏事件。两个人见章秋谷来了,大家谈了一回,就把秋谷留住在衙门里头吃了一顿夜膳。这一来有分教:斋

韦郎未老,香留白袷之衣;倩女多情,春满流苏之帐。

不知后事如何,且听下文分解。





第八十七回 卖风情陌路遇萧郎 感华年高楼圆好梦





只说章秋谷被那两位太史公留着吃了晚饭,忽然想起贡春树约他在阿娟那边吃酒,便苦苦的辞了出来,两位太史公留他不住,只得由他。章秋谷大踏步走将出来,出了抚署头门,恨不得一步就跨到贡春树寓所。一路慌慌张张的走过来,到了道前街,想着抄小路走近些,便回过头来抄入南面一条巷内。

这个时候已经八点多钟,路上十分黑暗,章秋谷心中性急,便不顾好走不好走,低着头,放开脚步飞一般的向前直冲。猛然听得对面马蹄声响,耳边有个人吆喝一声,章秋谷抬起头来,不觉大吃一惊。只见对面一个人骑着一匹快马,也是飞一般的直冲过来,那马把头一昂,早已碰着章秋谷的肩项。说时迟,那时快,章秋谷躲避不及,退让不来,这个骑马的人一时又收勒不住,这匹马正在放开四蹄,腾云驾雾的一般向前跑去,那里收得住。眼看着十分危险,两下都急出一身冷汗来。好个章秋谷,真是“忙者不会,会者不忙”,把身体往后一仰,伸出右手来霍地把马口内的嚼环揪住,轻轻的一个转身,早已转到马头的左道,把手内的嚼环用力一凝,那马便停住四蹄,屹然不动。秋谷睁开双眼看那马上的人时,只见也是一个二十多岁的少年,穿着一身时式的衣服,面上显着一付油滑样儿。秋谷伸过左手,抓住他的衣服往下一拉,这个少年身体一歪,坐不住鞍鞯,扑的跌下马来。秋谷正要骂他几句,忽听得对面一家人家的门内,发出一种轻清婉妙的声音,低低的一声“好”。

章秋谷听了这一声脆生娇生滴滴的声音,好似那乳燕呢喃,春莺宛转,不由得心中一动。闪过眼光往对面仔仔细细的看时,恰好这个地方有一个路灯闪闪烁烁的照着,只见门内立着一个二十余岁的女子,朱唇半启,香辅微开,眼波莹莹的正对着秋谷细看,却生得不长不短的身材,不瘦不肥的态度,云鬟宝髻,皓腕纤腰,润脸呈花,圆姿替月。比赵家之飞燕,宜喜宜嗔;方洛浦之灵妃,倾城倾国。掩着半个脸儿,立在门内,后面还立着一个十三四岁的丫鬟。

章秋谷不看犹可,一看了这个女子的样儿,觉得眼光一闪,好似一道电光射将过来,闪闪烁烁的耀得眼光都有些模糊起来。一时间章秋谷的眼睛里头,好像有十百千万个美人的影儿,前后左右的耀着他的视线,登时一个心上七上八下的在腔子里头乱跳,觉得有一种说不出的情感从心窝里头发越出来,把那方才的一团烈火都不知化到那里去了。只得勉强定一定神,对着那骑马的人正色说道:“你跑马有跑马的地方,怎么跑到这个地方来?马项上又不带响铃,就是这样的横冲直撞,你难道不懂规矩的么?今天幸而遇见了我,没有受伤;要是撞着了别人,那里有这般急智?闹了人命出来,你又怎么样呢?”秋谷口中虽然在那里和人说话,一双眼睛却不住的望着门内溜过来。那女子见了,知道秋谷已经有意,对着章秋谷低鬟一笑,飞了秋谷一个眼风,却故意别转头去,举起一双纤手把头上这云鬟慢慢的整理一番。

这个时候,章秋谷心中的快活,在下做书的也形容他不出来,只觉得心花怒开,心窝奇痒,浑身的四肢百体无一处不畅快,四万八千毛孔无一孔不熨贴。比那寒士登科,穷人暴富,觉得还要快活些儿,那里还顾得和那骑马的人说话。

那骑马的人在旁边看了这个情形,也觉得十分好笑,便对着章秋谷说道:“我的马上虽然没有响铃,你的走路却也太慌迫了些,我们两下都有不是,也不必再去提他。”依着这个骑马的人的意思,无故被章秋谷在马上揪了一交,心上好生不伏,还要想和他理论几句,但看着章秋谷这样的身材灵便,手脚玲珑,晓得他一定是个精通拳棒的惯家,便也不敢去惹他。说了这几句话儿,便不问情由,腾身上马,把缰绳一拎,这马放开四蹄向前便走,口中高声说道:“得罪,得罪!我要先走了。

今天这一撞倒便宜了你,你一个人在这里吊膀子罢!“

章秋谷心上糊里糊涂的也不知这个骑马的人和他说的是些什么话儿,只呆呆的看着那门内这女子,两下眉来眼去,卖弄风情。直至那骑马的人上马走了,说了几句取笑他的话儿,方才抬起头来看时,听得马蹄得得,只看见一个影儿早已走了一大段路。秋谷自己也觉得心中好笑,只见那门内的女子也用手帕掩着樱唇,对着他嫣然巧笑。章秋谷到了这个时候,知道大事将成,心上要想一个和他说话的法儿,却一时想不出来。只见那女子对着秋谷瓠犀微露,媚眼横斜,举起手来做了一个手势。章秋谷猛然心生一计,竟大踏步进门来,对着那女子笑道:“对不起,请问一声,刚才我不见了一点儿小东西,给那马平空的一冲,不知落到那里去了,可好容我在这里找一找么?”说着便抢步过去,深深的一拱到地。那女子也不回礼,只微微一笑背转脸去,红上桃腮,春生宝靥,口中说道:“这个不妨,只顾请便就是了。”

那丫环在背后插口说道:“倒客气得狠。”那女子举起手来,轻轻的打了丫环一下道:“不话多说。”章秋谷见了这般模样,便故意蹲下身去,两手在地上乱摸,渐渐的一步一步直接过来,一直摸到那女子的脚下。章秋谷趁势撩起他的裙来,把一只左手在他脚上碰了一下,那女子格的一笑,口中说道:“在这个地方规矩些儿,不要这般啰唣。”秋谷也笑道:“在这个地方要规矩些儿,在什么地方就可以不规矩呢?”那女子听了一言不发,瞪了秋谷一个白眼,回转身来往里便走。章秋谷到了这个时候色胆如天,竟是不分好歹,跟在女子的后面闯将进去。那女子虽然觉得章秋谷跟在他的后面,却头也不回,带着丫环一直的走进去。章秋谷跟进门内,仔细看时,原来不是大门,好像人家的后门的样儿。那女子放轻了脚步走过一层院子,转一个弯便是一个扶梯。那女子走上扶梯,秋谷大着胆子也跟上去。

到了楼上,章秋谷举目看时,见是一并三间的屋子,上首一间垂着门帘。那女子掀开门帘走了进去,章秋谷也走进来,又是深深一躬。那女子到了这个时候,倒觉得有些不好意思起来,低着头还了个万福,背着保险灯远远的坐下。秋谷到了房内,先看屋内的样儿。只见一张楠木大床朝外摆着,不多的几张桌椅,疏疏落落的排着。梳妆台上却排着几部小书、笔砚瓶花,位置得十分济楚。上首一带略略的有几件箱笼陈设。当门排着一张小小的条桌,上面还摆两盆梅花,疏影横斜,暗香骀荡。衬着这个灯下的美人,名花倾国,相映生辉。

章秋谷到了这个时候,觉得一个身体有些虚飘飘的样儿,如入天台,如登仙府,便不分好歹,走过去拉了他的纤手,拉他立起身来,向灯下并肩立着。再从头至脚的看时,只见他头上低低的挽着一个髻儿,插着不多的几件簪饰,穿一件蜜色皮袄,衬一条玄缎长裙,一双尖尖瘦瘦的金莲,一捻凌波纤不容握,穿着一双宝蓝绣花的弓鞋,都丽非常,丰神绝世。真个是说不尽的千般婀娜,写不出这万种风流。章秋谷见他羞怯怯的低着头不肯开口,便先问他的名姓,方才知道他姓楚,小字叫做芳兰。秋谷自己也通了名姓,嘲他又打一拱道:“我章秋谷的一双眼睛阅人多矣,从没有遇见你这样的一个人,真是天上神仙,人间珠玉。”芳兰听得秋谷这样的赞他,便回眸一笑,对着秋谷低低的说道:“你不要只管打拱作揖的做这许多怪相,人家要说你是痴子的。”秋谷紧紧的一把搀住了他的手,觉得兜罗一握,入手如绵,口中还对他说道:“别人叫我痴子,我一定的不答应,惟有你就是叫我痴子,我也狠高兴的,还恐怕我没有这般的福分呢!”

正说着,忽听得下面人声喧嚷起来,好像有三五人的脚步声音望着楼下直走进来。章秋谷吃这一惊非同小可,只认着又是什么仙人跳,有意诱他进去的,便推开了芳兰的手,揎拳掳袖的,要想打下楼来。芳兰一把把他拉住道:“不要紧,你不用着急,这是我父亲在外面回来,他们都不到这间房里的,你只坐在这里,不要声张就是了。”秋谷听了他的话儿,便悄没声儿的坐在那里,不敢开口,心上却还狠有些儿疑惑,侧着耳朵往下面听时,果然听得下面的人喧嚷了一回,却没有一个人上来。只听得有人说道:“老爷回来了,给老爷预备点心。”听得有个人答应一声,又听得有个人叫“来,来”的声音,又有几个人答应“嗻嗻”的声音。闹了一回,渐渐的没有声息,章秋谷方才放下了心,暗想这个样儿,一定是个本省的候补官,所以有这般势派,但是他女儿为什么又是这样呢?想着便问着芳兰道:“方才回来的可是令尊么?”芳兰点一点头,秋谷道:“你们令尊是什么班次?想来是这里苏州的候补人员了。”不料章秋谷一句话儿刚刚出口,芳兰早急地变了脸儿烦恼起来,一霎时粉面生红,蛾眉紧蹙,对着秋谷把手摇了两摇,默然不语,眼波溶溶的好像要流下泪来。秋谷见了他这般模样,便也不好再去问他,两个人默然相对。

秋谷又放出眼力,细细的注视他的面庞,只见他虽然皓齿明眸,雪肤花貌,却眉目之间明显着有一段牢骚,十分幽怨,好像有什么不得已的苦衷。暗想方才问他父亲是什么功名,便顿时心上这般不高兴起来,一定有什么说不出的心事,等会儿待我来把他好好的盘问一番,看他究竟是怎么的一回事情。想到这个地方,觉得芳兰这般模样狠有些儿可怜,更兼见他含情带恨,脉脉无言,眼眶中擎着两行珠泪,好似那风吹杨柳,雨打芙蓉,便深深款款的安慰了一番。正是:

三生慧业,一见倾心;刘郎之丰度依然,凤女之深情如许。琛钗暗堕,春融翡翠之衾;宝髻宵慵,香暖珊瑚之枕。

有分教:

巫云楚雨,十年小杜之狂;玉软香温,一枕高唐之梦。

要知后如何,请听下回,便知分解。





第八十八回 章秋谷意外得奇逢 贡春树开筵宴良友





且说章秋谷和楚芳兰邂逅相逢,良缘偶会,这一夜的恩情美满,鱼水和谐,海誓山盟,缠绵缱绻,也就可想而知的了。这些故事在下做书的也不必去提他,只讲章秋谷过了一宵,早是红日东升,鸡人报晓。秋谷恐怕迟了不能出去。便急急的起身,芳兰也不留他,只握着秋谷的手说一声:“后会有期,前途保重。”说到这里,那声音早有些哽咽起来,咽住了喉咙,说不下去。秋谷见了,不觉牵动了自家的万斛愁肠,半生心事,也陪着他凄楚起来,便安慰他道:“我们以后还可以想着法儿大家见面,你不必这个样儿。”芳兰也不开口,只对着秋谷摇一摇头。秋谷摸不着头脑,便在身上取出一张仁和的庄票,刚刚五十块钱,放在芳兰手内,口中说道:“这一点儿算不得什么,你留着赏人用罢。”芳兰一见了秋谷手内的一张银票,不知又是怎样的平空凄楚起来,眼圈儿红了一红,止不住泪珠滚滚,就如断线珍珠一般往下乱滴,呜呜咽咽的对着秋谷说道:“你若把我还当个人,请你不要把我这般糟蹋,我这般命苦,难道你还忍心糟蹋我么?”说着,几乎要放声哭将出来,这一下把个章秋谷也说得十分难过起来,想着:天壤茫茫,置身无地;青衫落拓,红粉飘零。扬意不逢,知音难得;才名画饼,忧患如山。就也不知不觉的洒了几点英雄眼泪。

停了一回,芳兰含着一汪珠泪,把那一张庄票仍旧塞在章秋谷衣袋里头,对着秋谷说道:“你还是走罢,在这里挨一会儿也不中用,倒弄得两下心中难过。”秋谷听了,觉得还有些依依不舍的样儿,又恳恳切切的对着芳兰说道:“你究竟是怎么的一回事情?为什么我昨夜这般问你,你咬定牙齿不肯露出一个字儿的风来?我们两个人一番相遇,也算是个意外的姻缘,你有什么心事什么话儿,何不讲出来?

我们两个人商量商量,或者我有什么可以出力的地方也未可知。不是我自家夸口,我章秋谷一身侠骨,万斛奇才,无论你身上再有什么天大的事儿,也要和你想一个万全的方法。“芳兰听了叹一口气道:”多谢你的盛情肯和我这般的出力,但是我的事儿是苦在我自己的心里,叫做哑子吃黄连,说不出的苦,就是和你说了,也没有什么法儿。我只恨着我自家命苦,两年以前没有遇着你这样的一个人,到了如今也是无从说起的了。“说着只见他把牙关一咬,毅然决然的对着秋谷说道:”你去罢,我不留你了。“

秋谷听了芳兰的一番说话,觉得一字一句里头都含着无限的辛酸,迸着许多的血泪,心上已经明白了几分,知道他自家心上,一定有天大的冤苦说不出来。想着他不肯要钱,又不便勉强他,要想送他一个戒指罢,秋谷向来又是不带戒指的。想了一想,便从身旁取出一个金表来,表上还带着一条黄澄澄的金链,递在芳兰手内道:“你好好的收着,算个我们的纪念罢!”芳兰看也不看,口中说道:“你方才送我五十块钱我尚且不收,如今又送起金表来,你把我当作什么样人?难道我也是那班堂子里头的倌人,只晓得问你要钱不成?”这几句话倒把章秋谷说得闭口无言,只得转口说道:“这是我错了,这是我错了,如今依你的意思便怎么样呢?”芳兰听了,便把秋谷手内用的一方全白丝巾拿了过来,放在自家袖里,把自己日常用的一方湖色丝巾换给了秋谷,又在自己手上脱下一个赤金嵌宝的戒指,也替秋谷带在左手小指上,叮嘱他不要送给别人。秋谷见了,只得在表链上解下一个全绿的翡翠猴儿来,放在芳兰手内,芳兰方才收了。又催着秋谷快走,秋谷看看表上已经将近八点钟,没奈何只得一步懒一步的要走。

芳兰握着秋谷的手送到扶梯旁道:“但愿上天保佑我们两个人后来再得相逢。

我们两个人要好一场,我只有一句话儿相劝,你是个读书人,家里头只要有一碗粥吃,千万不要出去做官;就是连粥都没得吃,情愿在家里头饿死,也千万不要出去做官。你若是记得我这个人,务必把我这几句话儿到处劝人,叫人知道这个官是万万做不得的,我也不白白的糟蹋了……“说到这个地方,说了半句,再也说不下去,眼中又流下泪来。秋谷看了实在代他凄惨,却又找不出什么话儿劝他,见那丫鬟立在旁边,眼圈儿也揉得红红的,秋谷便从袋内取出十块钱给他,口中说道:”昨天辛苦了你,你拿去买些花戴罢。“一面说着,一面把手内的丝巾和芳兰揩拭泪痕,芳兰珠泪纵横,玉容惨淡,停了一回方才长叹一声道:”咳,苦呀!“这一声好似那巫峡哀猿,衡阳孤雁。章秋谷听得十分酸鼻,不由的又落下泪来。芳兰一言不发,放开了秋谷的手,把手指着扶梯叫他下去。章秋谷一步一回头的走下楼来,那丫鬟在前引路,喜得静悄悄的没有一个人影儿,章秋谷悄悄的走出后门,那丫鬟便自掩门进去。

章秋谷走到街上,回过头来看时,楼阁依然,玉容深掩,倾城何处,梦境都非。

心上觉得十分怅惘,低着头往前走去,走了几步,又抬起头来看时,只见六扇黑漆大门和那昨夜的后门也隔得不多几步,大门上贴着一张朱笺,写着“楚公馆”的几个字儿,上面还写着许多官衔,秋谷见了把舌头一吐,心上已经明白了五六分,便急急的走回贡春树寓所来。

贡春树刚刚起来,洗脸漱口,见了章秋谷便嚷道:“你昨天晚上往那里去了一夜?害得我在阿娟那里直等了大半夜,一班客人都空着肚子等你一个人,究竟是个什么缘故?”秋谷听了微微的一笑道:“这件事儿说起来话长得狠,你且不要乱嚷,等一回儿和你慢慢的说。”贡春树听了,走近身来把秋谷脸上仔仔细细的打量一回,笑道:“我看你的样儿,一脸的不尴尬,既带着一团高兴,却又有些牢骚郁勃的样儿,一定是昨天晚上到仓桥滨去寻你的老相好,吃了败仗回来了。”秋谷笑道:“你不要这般混说,难道我也像你常常的打汇票不成?”春树听了呆了一呆,不晓得秋谷讲的是那一路的话儿,一时顿住了口说不出什么来,只眼睁睁的看着章秋谷的脸儿。章秋谷见了他这个样儿,只笑得拍手打脚的,口中说道:“何如今天你也居然有给我难倒的时候?”贡春树实在不懂什么叫做“打汇票”,只得也笑道:“今天算我输了,你且把这‘打汇票’的三个字儿细细的给我讲一讲。”秋谷哈哈的笑道:“你也总算是个上海的老白相,怎么‘打汇票’的这句话儿都不懂是什么讲究?真是个不中用的饭桶,怪不得金小宝要说你中看不中吃呢。”春树听了不觉脸上红了一红,道:“这句话儿是从那里来的,难道是金小宝自己告诉你的不成?”

秋谷把一个右手的大拇指在自己鼻子上一指,口中说道:“岂敢,不是小宝自己朝我说的,难道是我说谎的不成?”贡春树不听这句话儿犹可,一听了这句话,脸上越发的??起来,把秋谷呸了一口道:“你这个人真不是个好人。”秋谷见春树有些当真着急起来,不由哈哈大笑道:“算了算了,你不用这等疑心,虽然小宝对我实在有这句话儿,却是我和他两下取笑的时候说出来的话儿。我和你这样的交情,决不剪你的边。方才你自己说我到老相好那里去打了败仗,如今我不过回敬了几句,你就要急得这个样儿,难道只许你和我取笑,不许我和你取笑的不成?”贡春树听了,自己也不觉好笑起来,对着章秋谷说道:“算了,不用说了,就算我说你不过,怕了你如何?”秋谷道:“你本来说我不过,为什么要就算?”春树道:“你这个人,怎么一个字眼儿上都不肯将就些儿,吃一点亏,一定要自己占了便宜才罢?”

秋谷道:“这是如今世界上优胜劣败的公理,没有,什么说的。”春树道:“优胜也罢,劣败也罢,你且把昨天晚上的事情细细的和我说来。”

秋谷方才把昨天遇着芳兰的情节,一字不遗的和贡春树说了一遍。贡春树方才明白道:“原来你果然有了这般奇遇,今天罚你在阿娟那里吃一台酒,罚你的无故爽约,你道如何?”秋谷道:“罚我吃台把酒,算不得什么事情,但是我心上有一件不明白的事情。”说着,便把自己如何的心中疑惑,如何的盘问他,芳兰如何的死不肯说,临走的时候芳兰又是怎样的依依不舍,好像以后不能见面的一般,一一的和贡春树说了。又道:“看他的丰姿体态,绝没有一丝一毫的轻贱样儿,不知他心上究竟有什么不得已的苦衷,没奈何方才把他逼到这般田地。但是既然如此,要和我再图相会,也是狠容易的事情,为什么听他的口气又像有什么阻碍的一般。难道第一次没有阻碍,第二次倒有了阻碍么?你和我想一想,这里头究竟是个什么缘故?”贡春树听了,心上也觉得狠是诧异,大家猜想了一回,终久猜想不出来,便也只得由他。章秋谷的心上究竟还觉得有些依恋,一时撒不下来,好像心上忘了一件最紧要的大事一般,狠有些儿闷闷不乐,连午饭都不高兴吃,只略略的吃了些儿便放下了。

到了晚上,依着秋谷的意思,还要到芳兰那里去候他,希冀他和昨日一样的出来。贡春树因为已经约好了一班朋友,再三的不肯,硬拉着他望庙堂巷阿娟家里来。

秋谷一面走着一面口中说道:“这个地方都是些候补人员的公馆,怎么会住着这样的人家?”春树道:“苏州地方,那些半开门的私窝子门口贴着个公馆条子,假充候补官公馆的多得狠呢。”说着,已经走到一家门首,看看大门上倒也贴着一张公馆条子,上面写着“余杭马公馆”的几个字儿。贡春树便当先走进,秋谷和着春树的几个朋友也跟在后面走进来,走过了小小的三间花厅,便是三间上房。春树不分好歹,领着众人直闯进,只见房间里头空空的不见一个人。春树高声叫道:“客人来了,怎么人都没有,快给我滚一个人出来。”一句话还没有说得完,早听得房后莺声呖呖转出一个丽人。正是:

春风无恙,可怜白贮之歌;旧雨不来,辜负蓝桥之约。

不知出来的是什么人,下文交代。





第八十九回 闯房间流氓横索诈 惩无理名士怒挥拳



且说贡春树正在嚷着,叫滚出一个人来,早听得房后莺声呖呖,转出一个二十岁上下的丽人。未曾走到,早有一股香水的味儿,和着那一阵的脂香粉气芬芳馥郁的直透入众人鼻观中来。秋谷见了,知道一定就是阿娟。只见他轻移莲步,慢拢湘裙,直走到贡春树的面前,故意嗔道:“阿贡,耐勿要勒浪瞎三话四,啥格滚出来勿滚出来,倪倒一径< 曾忽> 滚过歇,勿晓得那哼格滚法,请耐滚拨倪看看。”说罢回过头来,把一双俊眼前后左右的四围的飞了一转,方才把手掠一掠头上的鬓发,对着众人笑道:“各位老爷请坐捏。”

章秋谷听了,便先叫一声“好”,阿娟又飞了章秋谷一眼道:“倪是勿好格,请唔笃各位包涵点。”章秋谷也打着苏州白讲道:“阿呀,耐勿要客气嗫,依仔倪看起来,是样样好式式好,格末叫好得来呒淘成。”阿娟听了把头一扭道:“好哉好哉,勿要勒浪钝哉,耐格位老爷啥实梗格介。”秋谷也不再说,只是上上下下的细细的看他,只见他穿一件铁青色摹本锻的皮袄,下面衬着一条品蓝绉纱的裤子,湖色缎子的弓鞋不盈四寸。蛾眉掠月,宝髻堆云;丰彩惊鸿,佩环回雪、那一双俊眼就如一泓秋水的一般,秋剪双瞳,横波欲活,一颦一笑,顾盼生姿,虽然算不得什么倾城倾国的丰神,却也态度鲜妍,容光飞舞。暗想贡春树的眼力总算不差,但是和昨日的芳兰比较起来,却还觉得差了些儿,赶他不上。正想着,忽听得阿娟开口对他说道:“格位老爷尊姓,阿就是章老爷?”秋谷还没有开口,春树在旁边接下去说道:“不错,这位就是章老爷。”阿娟道:“阿呀,章老爷昨日仔啥勒勿来介,阿贡搭仔几化客人等得来大家格肚皮才要饿杀快,阿是倪间搭小场化,章老爷勿肯过来?”秋谷不等他说完,便指着他的脸道:“你这个人,真有些岂有此理!”

阿娟听了不懂道:“啥格岂有此理,岂有此外介,倪才勿懂啘!”秋谷道:“姓贡的和我们一样的客人,为什么你见了他口口声声的叫他阿贡,难道我们就不是客人么?”阿娟听了,对着章秋谷把嘴一披道:“阿呀,章老爷,勿要扳倪格差头嗫。”

刚刚说到这里,忽听得外面许多脚步的声音直拥进来,不由分说竟一直闯进房内。贡春树和那几个人早吃了一惊,章秋谷不慌不忙,举目看时,只见七八个短衣窄袖的流氓从外面乱闯进来,都是身上单穿着一件皮马褂,敞着了怀,把一条腰带系在外面,一个个揎拳掳袖、怒目横眉,好像要寻人打架的样儿。为首的一个约有三十来岁,身材高大,面目凶横,对着章秋谷一班人点点头道:“对不起,你们已经来了多时,如今请你们到外面去坐一坐,这个地方让我们来开个心儿。你们要是懂事的,快快的给我出去,万事全休;如若不然,哼哼,那时却莫怪我们粗鲁。”

有几个客人听了那班流氓的口风,明晓得他是有心寻事,一个个心上都怕起来,立起来站在地下,你看我,我看你的要想逃走出去。

早听得秋谷大声说道:“你们不要害怕,有我在这里一面承当。”说着,便向众人喝道:“你们这班奴才,平空闯进来做什么?还不给我快些滚出去!你们睁开狗眼认认我是个什么样人,不要想昏了你们的头。”章秋谷这几句话儿方才出口,那个为首的流氓早勃然大怒,高声叫道:“你们看看他倒骂起人来,不给他一个利害他也不知道。”说着便俯身下去,在快靴统里头掣出一把冷森森明晃晃的尖刀拿在手内,大踏步走近章秋谷的身旁,睁开两眼,厉声说道:“老实和你说,这个地方是大家可以来的,你们通共出了一块钱来打个茶围,你们打算要坐到什么时候?

老子们有的是钱,今天也要来打个茶围,你们好好的出去,是你们的便宜。“说罢把手内的小刀用力往桌子上一插,只听得”铮“的一声,那把小刀插进桌子约有一寸深。这一来,把那几个贡春树的朋友吓得魂飞魄散,骨软筋麻,觉得一股冷气从脚底下直透起来,一个个只吓得浑身乱抖,一句话也不敢说。更兼那一班流氓耀武扬威七张八嘴的嚷道:”老大,还有这样的工夫去和他讲话?只拖了他们出去就是了。“

章秋谷坐在那里动也不动,只自己一个人在那里微微的冷笑。那为首的流氓见了秋谷这般模样,心上焦躁起来,便刘着秋谷喝道:“出去不出去?”秋谷微微的冷笑道:“我正要看看你们这班奴才有什么本领。我不出去,看你们这些奴才可有什么法儿。”那为首的流氓听了章秋谷这般说法,由不得心头火发,鼻孔烟生,抢过来一把抓住了秋谷胸前的衣服,想要撵他出去。早被章秋谷伸出右手,接住了他的手臂轻轻的一拧,这班流氓本来都是些鸦片烟鬼,大风都吹得倒的,那里当得起秋谷的神力?被他轻轻的把手臂一拧,拧得他“阿呀”一声,身不由己的跪在地上。

秋谷顺手一送,早把他跌了一个狗吃屎,倒在地下扒不起来。那同来的一班流氓见了,一齐怒道:“什么东西竟也这般可恶!我们大家上去打他一顿。”说着便七手八脚的拥上来。章秋谷见了,觉得实在好笑,慢慢的立起身来,把两手一拦,当头的两个流氓立脚不住,跌倒在地,后面的人看了,就立住了不敢上来。秋谷哈哈的笑道:“像你们这般没用的东西也敢出来讹诈?你们胆大的只顾上来。”一班流氓听亏,面面相看不敢动手。那起先跌倒的三个也都扒起身来,呆呆的站在一旁,秋谷对他们说道:“你们怎么样?怎么七八个人,一个都不敢上来?你们这班没用的奴才,不要在这里现世,快些给我滚你妈的蛋罢!”那一班流氓听了,不敢开口,只得垂头丧气的出来,连那方才插在桌子上一把小刀都不敢拿,一哄的都走了。

秋谷见他们走了,回过头来看那几位贡春树的朋友时,一个个都吓得屁滚尿流,唇青面白。贡春树站在秋谷背后,虽然也有心上惊慌,却向来知道章秋谷的本事,料想这几个人不是章秋谷的对手,所以也还不至于十分胆怯。只有阿娟一个人见那班流氓拥进门来,早吓得他香汗淋漓,花容失色,不顾三七二十一、四七二十八,连忙趁着大家扰乱的当儿,躲进床背后小房里去,和两个小大姐紧紧的把房门关上,不敢出来。直至章秋谷打退了一班流氓,他在里面听得明白,心中大喜,便开了门出来,对着众人说道:“格排杀千万格强盗坯,也勿知啥格路道,倪拨俚吓得来人野吓杀快。”又对着秋谷说道:“谢谢耐帮仔倪格忙。今朝区得耐勒浪倪搭,赛过救仔倪格性命。”秋谷笑道:“不要说是这两个人,就是来得再多些儿,也不放在我的心上。”贡春树是见惯的,不以为奇,只有那几个人在旁边看着秋谷的样儿气宇安闲,丰神潇洒,好像个手无缚鸡之力的一般;如今见他三拳两脚的打退了一班流氓,觉得心上十分诧异,大家都对着秋谷说道:“今天幸而秋翁先生和我们同在一起,没有吃他们的亏。如若不然,今天就不可问了。”秋谷也随意谦逊了几句,趁便走过去拉着阿娟的手道:“你以后不要叫我章老爷,只要叫我一声二少就是了,不信你问阿贡,我在上海,那些堂子里头的人都是叫我二少的。”阿娟听了,斜着眼把秋谷一看,只见他朱唇粉面,猿臂蜂腰,举止安详,丰神俊雅,眉宇之间觉得另有一种英气,奕奕照人。不觉面上一红,别转头去,口中说道:“晓得哉,格末就是二少。”秋谷又低声和他讲道:“我今天和你打退了这班流氓,你该应怎样的谢我?”阿娟听了也不开口,只慢转秋波,暗中飞了秋谷一眼,横波一转,脉脉含情。秋谷见了,倒把头低了一低,放开了阿娟的手。

回转身来刚刚同贡春树打了一个照面,春树对着他微微一笑道:“你这个人真有些岂有此理,剪别人的边也还罢了,怎么剪起我的边来?”秋谷听了也笑道:“我和你两个人认得了多年,你几时见我剪过朋友的边?难道我章秋谷也和你姓贡的一般,不顾朋友的交情一味的混闹不成?”贡春树还没有开口,早被阿娟走过来拉着他的手不依道:“啥格剪边勿剪边,耐勿要勒浪瞎三话四,倪勿来格。”说着,便坐在春树身上,扯着他一个耳朵,口中咕噜道:“倪勿要,耐下转阿要实梗?”

春树被他扯住了一个耳朵,扯得他口中叫道:“你有话好好的说,不要这般动手动脚。”秋谷对着阿娟笑道:“你好好的扯他一下,问他以后还瞎说不瞎说?”阿娟果然听了章秋谷的话儿,用着气力把他的耳朵扯了一下,扯得个贡春树抱着头直跳起来,口中乱叫道:“耳朵耳朵,扯掉了耳朵是没有价钱的。”阿娟一面格格的笑着,一面口中说道:“啥人叫耐实梗呀,耐下转阿要实梗瞎三话四哉?”春树脱了阿娟的手,连忙走过一边道:“你吊膀子只管吊膀子,我又不来管你的闲帐,你何必就要这般着急。”阿娟听了不由的着起急来,红着脸赶过去要和春树不依。春树见了连忙抱着头逃过这一面来,对着阿娟把双手乱摇道:“算了算了,总算我的不是,讲错了一句话儿,我还要留着耳朵摆个样儿呢。”一句话把大家都说得笑起来,阿娟也笑道:“耐自家勿好啘,耐下转阿敢哉?”春树朝着阿娟恭恭敬敬的打了一拱道:“千不是,万不是,总是小生不是。”说得秋谷哈哈大笑,对着阿娟道:“他既然这样的自家认错,你就饶了他罢。”阿娟听了方才一笑走开。春树见了又拍着手道:“到底章二少说的话儿比我灵应得多。”阿娟瞅了春树一眼道:“狗嘴里勿会出象牙,啥人来理耐呀。”春树正要开口,秋谷扯住他道:“时候已经不早,叫他们摆起台面来罢。”春树听了,便和阿娟说了几句,两个大姐走过来调开桌椅,摆上菜来。原来苏州的规矩,私窠子是没有什么摆酒不摆酒的,有时候客人要摆酒请客,便自己去叫菜。今天这一席菜是贡春树在三雅园叫来的,肴馔十分精致。正是:

桃花春水,谁家有蛱蝶之图?珠箔银屏,何处是天台之路?

要知后事,请听下回分解。





第九十回 银汉仙槎刘郎惆怅 秋风莼菜张翰归来





上回书中说着章秋谷和贡春树在阿娟那边晚膳,一时间觥筹交错,履舄纵横。

那几个客人也每人叫了一个和阿娟一样的开门的私娼,只有秋谷不认得这些人,无从叫起。贡春树要和他代叫一个,秋谷执意不要,也就罢了。当下开筵坐花,飞觞醉月,直闹到三更左右方才散席。大家都辞了主人先走,只有秋谷和春树两个人已经微微的有些醉意,还坐在那里。只见阿娟走过来和春树咬了一回耳朵,春树沉吟一回道:“一时找不出地方,搬到那里去呢?”秋谷听了,不晓得他们说的什么,便问着春树道:“什么搬不搬的,你们那一个要想搬家?”春树听了,便把这里头的情形和秋谷讲了一遍。

原来苏州地方的规矩,一班堂子里头的倌人开着一个门面,每每有许多地方的流氓跑到堂子里头去想他们的好处。一班倌人见了这一班流氓,一定要送他几块钱,还要对着他们说上许多好话,方才肯好好的出去。如若不然,这班流氓就要糟蹋他们的房间,得罪他们的客人。这班客人都是一班有身家的,见了这班流氓如何不怕?

自然大家都吓得不敢再来。这些流氓一味的拼命混闹,闹得一个天翻地覆,一定要拿着了钱才罢。除了租界上的堂子,这班流氓吃巡捕官司不敢去闹,其余城里城外的那些开堂子的人家都是他们的衣食饭碗。这些倌人见了那班流氓,没有一个不是怕得心惊胆战,非但一毫不敢得罪,而且还要好好的敷衍他们。若是那一班半开门的私娼,本来没有多少客人走动,这班流氓要是不知道也就没有法儿,万一个给他们打听了出来,一定也要带着几个人进来啰唣,想要讹诈客人们的钱。阿娟住在这个地方还不到一年,所以起先他们还不知道阿娟是个私娼,如今被他们晓得了风声,便大家闯进来想些好处。不料刚刚碰着了章秋谷,非但想不着好处,倒反吃了一个大亏;但是一时间虽然有个章秋谷挺身出来把他们打退,慢慢的他们一定要来翻本。

到了那个时候,章秋谷不能常常的在这里保护他们,只剩了阿娟一个人,那里受得他们的糟蹋,所以阿娟和春树计议要想搬到阊门马路上去,做个么二住家。春树想着,一时找不出这样的一处房子,有些踌躇起来。

当下春树和秋谷说了这个缘故,秋谷想了一想道:“也不必搬到城外去,你不是有几间房子在胥门里头么?现在正还空着没有人住,你何不就借给他住了,将来有起事来,你是个房主人,也可以出来讲话的。”春树想了一回,点一点头道:“这个主意倒也不差,只得暂时搬去,避过他们的耳目也就是了。但是这班流氓地痞是到处有的,万一搬了过去又有人去吵闹起来,这便怎么样呢?”秋谷道:“那倒不要紧,只要客人们出进的时候自己小心些儿就是了,那里顾得许多?就使再有流氓闹事,你如今是他的房东,也可以出来说几句话的。”春树听了。觉得秋谷的话不差,便打定了主意,又和阿娟说了些体己的话儿。秋谷要催着他回去,春树道:“时候已经不早,我们大家在这里借个干铺罢。”秋谷听了,拿出表来看时,果然已经三下多钟,便依着春树在阿娟那边借了一夜干铺。

到了明天,贡春树要请章秋谷到仓桥浜高桂林家吃酒,曾太史和邓太史两个人又写了一封信出来,约秋谷进城吃饭,秋谷一概都回了不去,写了一封回信给那两位太史公,只说已经动身回去。秋谷自己一个人却走到道前街巷内楚公馆的大门外面,来来往往的走了数十余次,要想候着芳兰出来见他一面,再续前缘。那里知道呆呆的等了多时,只看见有几个家人出入,连芳兰的影儿也看不见,一直等到二更以后方才无精打彩的回来。

到了第二天又去那里候他,刚刚走到楚公馆的门口,心上吃了一惊,只见大门上挂着红绸,中间的屏门开着,大厅上点着灯烛辉煌的,却静悄悄的不见什么人。

秋谷在门外踱了一回,想不出什么缘故,见门口没有人,便轻轻的蹑步走进,早听得有几个人的声音在门房里头谈论得十分热闹。秋谷侧耳听时,只听得一个人的声音说道:“我们老爷做起事来总有些鬼头鬼脑的,不知道是个什么缘故。你们想,今天小姐出嫁总算一件喜事,为什么要这般藏头露尾的不叫同寅们知道,好像把小姐送给人做姨太太的一般,你想可奇怪不奇怪?”秋谷听了这几句说话,心上好似触着了电气的一般。再仔细的听下去时,又听一个人说道:“我们老爷真是瞎了眼睛,把一个如花似玉的小姐去配给这样一个姑爷,又黑又丑,还是一脸的大麻子,走起路来一只脚又有些拐的,老爷不知怎样的平空拣中了他,不知小姐看了这样的姑爷,心上怎样的烦恼呢。”说着,又听得一个人接下去大声说道:“你们讲的都是些隔壁帐的话儿,我们老爷拣中这个姑爷,难道是拣他的才貌么?我们老爷的性情本来是势利不过的,见了他有财有势,所以连忙把女儿嫁他。将来一则好问他借几个钱,二则还好靠着他的势力自己弄个差使。只可惜我们小姐这样的才貌,却嫁着了这样的人,真是好块肥羊肉掉到狗口里去了。”众人听了,哈哈的笑起来。

章秋谷听到这里,心上早明白了八信分,觉得一股酸气从丹田底下直冲到鼻子里来,一个心乱七八糟的也不知是什么味儿,也不再听下去,大踏步走了回来。真个是:

银汉仙槎,桃花人面;牵牛西北,孔雀东南。凄凉巫峡之云,懊恼高唐之梦。

红楼隔雨,魂销婪尾之春;珠箔飘灯,肠断相思之字。

章秋谷当下一个人垂头丧气的回来。春树见了问他为什么这般模样,秋谷懒懒的也不开口,便上床睡了。这一夜的千般别恨,万种离愁,螺肠九回,珠丝百结,思佳人而不见,望秋水兮潆洄,这些情思也不必去提他。

只说章秋谷在家里头动身的时候,预先和太夫人说明,说到苏州去一两天就回来的,如今在苏州一连耽搁了五天,还要想到上海陈文仙那边去打个转身,算起日子来已经十分急促,便别了贡春树立刻上了轮船往上海去。轮船走了一夜,天还没有亮就到了上海。秋谷在大餐间里头直睡到八点钟方才起来,一直赶到文仙那里。

文仙迎着笑道:“我只道你今年不来的了,你倒居然没有失信,你们府上太夫人和少奶奶怎么倒都肯放你出来?”秋谷把别后的事情,约略告诉了陈文仙一遍,只瞒了苏州的事情没有提起。

秋谷坐了一回,便问起陈文仙年底有多少开销,陈文仙屈着指头算了一算道:“这里倒没有什么开销,就是年底下要用几个钱也看得见的,倒是那些店家的店帐,以及你堂子里头的酒帐局帐,只怕通算起来,倒也不少呢。”秋谷故意假作吃惊的样儿,口中说道:“我这一次来一个大钱都没有带,这便怎么样呢?”陈文仙瞪了秋谷一眼道:“你看你看,又来了,这样的假话只好对着人说上一次两次,人家或者还有些相信你的话儿。到了后来听得惯了,耳朵里头的老茧都听了出来,那里还有人相信?我劝你不要这样的装腔作势罢。”秋谷听了,自己也好笑起来,便在衣袋里头取出一张一千块钱的银票,交给陈文仙道:“我今天就要动身回去,一班朋友那里我也不去惊动他们,还有那些店帐和堂子里头的帐,我核算起来大约差不多也要七八百块钱,你等他们来了发票,便一一的和他们算清了,省得拖欠他们的钱。

堂子里头这一节本来不多,只有公阳里的陆丽娟、西鼎丰的梁绿珠,有几台酒几个局,你叫车夫去抄了局帐来,就叫车夫送去。我今年自从娶你进门以后,堂子里头没有去住过夜,所以没有欠什么钱。“陈文仙看着秋谷微微一笑道:”只怕不见得这样的克己罢。“秋谷也笑道:”看你这个样儿,难道要我跪下来赌一个咒不成?“

陈文仙道:“你们男人差不多大家都是这个样儿:见了家里头妻妾的面,一味的甜蜜语,拼命哄骗;等到转过身来,便把方才的话儿都忘到九霄云外去了。”秋谷道:“我向来不会骗人的,你看我平日之间可曾骗过什么人?况且你又不是一味吃醋不通道理的女人,我又何必骗你呢?”陈文仙听了笑了一笑,也不开口。

秋谷又问他新年里头要钱用不要钱用,陈文仙道:“我一个人住在上海,要用什么钱?自从你回去以后,我通共止出了一回门,是出去买洋货的,连马车都没有坐过一趟,你不信,只看那马车行的帐单就是了。”秋谷听了心上十分欢喜,却故意说道:“新年上没有什么事儿,虽然我不在上海,你一个人也好出去坐坐马车,吃吃大菜,或者戏园子里头去听听戏,借此消遣消遣开个心儿,不要呆呆的坐在家里,闷出病来倒不是顽的。”陈文仙道:“我本来不喜欢这些顽耍的事情,况且你不在这里,我一个人出去有什么兴趣。”

秋谷听了陈文仙这般说法,自然甚是放心,匆匆忙忙的叮嘱了陈文仙几句,便上了原来的轮船,赶回苏州。又趁上常熟的轮船,回到家里已经是十二月二十五了。

太夫人见秋谷回来,不免还要埋怨他几句,问他为什么到这个时候才来,秋谷随口掩饰了几句,便过去了。秋谷到了家里,少不得要料理些年事,到了新年上又要出去拜年,还有许多亲戚请秋谷去吃年酒,一连应酬了半个月,方才清静些儿。

光阴如驶,日月如飞。早又过了一个二月,这位章秋谷在家里住得腻烦起来,勉强过了一个三月,实在忍不住,便又告禀了太夫人要到上海去散散心,顺便算些帐目。太夫人心上本来不愿意章秋谷出去,但是眼见他恹恹悒悒的过了一春,提不起一些高兴,恐怕他闷出病来,便轻轻易易的一口应允。秋谷大喜,便急急的赶到上海来。正是:

桓司马重来灞水,风景依然,习凿齿再到襄阳,山河无恙。

不知章秋谷到了上海,又有什么事情,下文交代。





第九十一回 开花榜名妓占鳌头 掷金钱瘟生游北里





且说章秋谷得了太夫人的允许,再到申江。崔护重来,觉得殊有些人面桃花之感。章秋谷这边的事,权且按过一边。在下做书的再提起一个人来,把他的事情讲给看官们听听。

只说东方小松自从到了广东之后,两广总督李制军狠是器重他,请他办理摺奏。

刚刚李制军衙门里头有一位总文案,却是个广东候补道,姓陶,单名一个继字,表字伯瑰,本来是浙江山阴人,和方小松是亲戚,这一回李制军下了一个札子,委他到上海去采办军装。这位陶观察也久慕上海是个有一无二的繁华世界,满心想要去见识见识,但是陶观察这个人也是个没有阅历的土老儿,上海地方从来没有到过;知道方小松是久住上海的人,便托他介绍几个本地的朋友。方小松便写了两封信给他:一封是给章秋谷的,一封是给辛修甫的。信里头的话儿,无非是说陶观察现在到上海采办军装,托他们两个推情照拂。陶观察收好了信,便禀辞了李制军,上了轮船。不一日,早到了上海,在三洋泾桥泰安栈占了一间官房,便带了小松的信来找辛修甫和章秋谷,刚刚章秋谷已经回去,只有辛修甫还在上海。

在下做书的做到这里,便忽然又有一位爱说话的朋友来扳驳在下道:“你前集书中的东方小松,明明是复姓东方,为什么你的书中,有时候叫他东方小松,有时候叫他方小松,难道一个人有两个姓不成?”在下哑然笑道:“你这位老先生光景没有吃过花酒到过堂子罢?”那位宝贝听了不懂道:“我和你讲的方小松,怎么牵到吃花酒上去了?堂子里头的花酒我虽然没有吃过,我还记得几年之前有人同着我去打过一个茶围的。”在下听了止不住哈哈的笑道:“原来如此,那就怪不得了,你不晓得上海堂子里头的规矩,譬如这一个客人姓方,那班倌人自然是叫他方老爷,或者叫方大少;若是这个客人的姓有两个字儿,那班倌人嫌着两个字儿的姓叫得不顺口,便和他截掉一个字儿。比如这个客人双姓东方,倌人们有些事儿就叫他方大少;或者这个客人双姓欧阳,倌人便叫他阳大少。这位东方小松在堂子里头的时候,一班倌人大家都叫他方大少,所以在下做书的也就省一个字,把他写作方小松。古今来中国、外国都有省文的一条规例,并不是在下做书的自相矛盾、前后不同,算不得什么漏缝,你老先生不必费心。”那位朋友听了,方才闭口无言的去了。

如今闲话休提。只说这位陶观察到了上海,虽然没有什么熟人,却是大家都知道这位陶观察大人是从广东来采办军装的,就有一班洋行里头的滑头买办想要招揽生意,便大家都去拜他。又大家请他吃花酒,吃大菜,看戏游园,开口大人、闭口大人的拼命恭维,百般巴结。把这位陶大人巴结得十分欢喜,一个身体虚飘飘的好似在云雾里头一般。这班人又荐了两个倌人给他,一个叫做姚红玉,住在东荟芳;一个叫做薛金莲,住在福致里。姚红玉听了别人的说话,说这位陶大人是广东来办军装的,只要巴结上了他,一定有些好处,姚红玉便尽心竭力的巴结这位陶大人,不上几天就落了相好。只有薛金莲虽然做着陶观察的生意,却只是冷冷淡淡的样儿,并不十分巴结。偏偏这位陶观察又有些厌故喜新的脾气,虽然和姚红玉有了相好,却嫌他过于迁就了些,不上一个月,早已有些厌了,一心一意的要转薛金莲的念头。

说起这个薛金莲的出身来,本来是个大兴里的野鸡妓女出身,模样儿既不见得十分俊俏,身段儿也不见得怎样轻盈;既不会应客飞觞,又不会调丝度曲;却不知怎样的交了花运,做了几年野鸡妓女,却生意十分兴旺,慢慢的倒也积了些钱。这薛金莲既有了钱,便居移声,养移体,无缘无故的平空想升起长三来。好在薛金莲有的是钱,便在福致里租了一处三楼三底的房子,铺起房间,拣了一个日子烧路头进场,邀了那一班做野鸡时候的老客人来吃了几台酒,倒也十分热闹。无奈那一班老客人都是些上不得台盘的,也有机器厂里头的机匠,也有马车行的马夫,那里有什么钱常常的吃花酒?一时又找不着什么别的客人。只有一个恩客,是广东香山人,姓郑,叫做郑小麻子,薛金莲和这个郑小麻子虽然十分要好,无奈郑小麻子也是个穷光蛋,拿不出一个钱的。薛金莲见生意清淡,面子上实在过不去,便异想天开的想出一个主意来。

这个时候,正有一家小报馆里头要出花榜,薛金莲便去请了那一家报馆里头的主笔来,和他密密切切的商议了一回。那主笔点头应允,临走的时候,薛金莲又在首饰匣里头拣了几张钞票出来,往那主笔袖子里头一塞。那主笔接了,一张一张的看了一回,笑嘻嘻的对着薛金莲道:“请高升些,请高升些。”薛金莲听了,便又拣出几张来给了他。那主笔接了过来,满心欢喜,把那几张钞票翻来覆去的数了一遍,这才郑重其事的放在衣袋里头。立起身来辞了薛金莲往外便走,口中说道:“你只顾放心,这件事儿交给我,我给你格外说得好看些儿就是了。”薛金莲听了点一点头,连送也不送,由他自己去了。

隔了不多几天,果然这一家报馆里头出了一张花榜,把这个薛金莲高高的取了个一甲第一名状元,那几句评语里头说得十分热闹,什么说“藐姑仙子,无比清扬;越国西施,逊其都丽”。上海的一班人看见了这张报纸。觉得狠有些儿诧异。上海的事情,就是取一个花榜状元,也是论些资格的。如今这张报上平空把薛金莲取做状元,大家都不晓得这个人,便哄然一声,你也去叫,他也去叫。也是薛金莲的花运当阳,财星发达。这一班叫他的客人,大家都十分赏识他,不说他不会应酬,却说他狠有些儿大家丰范;不说他不能唱曲,只赞他还带着些闺阁娇羞。这样的一来,就一传十,十传百的把一个薛金莲高高的抬到天上去了,连薛金莲自己的心上也有些不相信起来。

说也奇怪,讲起这薛金莲和郑小麻子两个人的历史来,真真不知道是怎么一个缘故。看着薛金莲这样的一个人才,上海滩上不要说是长三书寓,就是野鸡幺二,面貌比他好的也不知多少,却不知怎样的,一班客人都把他当作天仙化人一般。只要和他有过相好的,一个个都是魄荡魂迷,心输意伏,也不知究竟是怎样的一回事情。再说起这个郑小麻子来更加奇怪,大凡上海滩上的倌人,只要是风头十足有些积蓄的人,那一个不要做个把恩客,自己寻寻开心,但是倌人们不做思客便罢,要是做起恩客来,自然总要拣个把少年貌美的客人,方才合着他们的意思。这个郑小麻子生得一个五短身材,两个眼睛抠了进去,一个鼻子高了起来,一脸漆黑的麻子。

这样的一付尊容,却又不知怎样的偏偏对了薛金莲的胃口,把他当做天字第一号的恩客,并且还讲明以后嫁他。这个郑小麻子非但一个大钱没有,而且还要常管着薛金莲,不准他接客。偏偏的薛金莲看看这个不对,看看那个不对,单单的看中了这样的一个郑小麻子,无论什么事情,都肯听他的话儿。这个里头,也不晓得究竟是怎么的一回事情。依着在下做书的摹拟起来,这两个人虽然外才不足,或者内才有余;一个就是那鸡皮三少的夏姬,一个就是那大阴专车的嫪毒,也未可知。

闲话休提,只说薛金莲的应酬功夫虽然不见得怎样的周到,却当了几年的野鸡妓女,阅历的客人多了。一见了陶观察的面,便料定了陶观察的性情:你越是待他冷淡,他越是转你的念头。更兼薛金莲这个时候已经狠有几个钱,虽然知道陶观察有钱,也不去想他什么念头。偏偏这个当儿,郑小麻子要想娶他回去,拼命的和一班客人吃醋,不许他留一个客人。所以陶观察死命的要想和薛金莲攀相好,薛金莲只是含含糊糊的,也不答应,也不回绝。弄得个陶观察好似鼻子上敷了糖的一般,枉是着急非常,不得到口。若是换了别个人呢,也就丢开了他,再去想别个的念头了。偏偏这位陶观察又是十分拙性,只说薛金莲的骨气不差,一定要想弄他到手,一连吃了十几台花酒,碰了七八场和,又送了他一个金刚钻戒指。薛金莲虽然受了他的戒指,谢也不谢一声,还只是这般冷冷的样儿。

陶观察没有法儿,只得来托辛修甫,请他在薛金莲那边做个说客。辛修甫那里肯答应?只对他说道:“我看你的相待薛金莲,也算得尽心竭力的了,怎么薛金莲的待你总是这样受理不理的样儿?看起来,一定是他心上不愿意和你要好。你有了钱,那里不好做个相好,何必一定要做他呢?”陶观察听了,呆了一回方才说道:“据我看来,他的待我也不见得怎样的冷淡,或者你的心上不欢喜这个人,所以觉得他有些不合,也未可知。”辛修甫听了暗暗的好笑,却当着面又不好十分的驳他,只得含含糊糊的说道:“照你这样说来,或者是我一时看错也是讲不定的。”陶观察又道:“今天我想在薛金莲那里吃个双台,再约几个人碰两场和,和他绷绷场面,但是我在这里不认得什么人,要请你和我转请几个客人。”修甫听了道:“转请几个客人是狠容易的事情,但是你要我去牵马拉皮条,那是我一生一世没有学过这个行业,这个生意还是请你照应了别人罢。”陶观察听了也觉得好笑,只得对他说道:“你不肯便罢,我也不敢勉强,但是等会儿晚上的局,你自己一定要到的,还有王小屏和陈海秋请他们一起过来。”修甫听了点头答应,陶观察便先去了。原来小屏、海秋都是辛修甫介绍和陶观察相见的,相见之后大家倒十分投合,所以陶观察在薛金莲那里吃酒,也把他们请在一起。正是:

桃花轻薄,才开半面之妆;柳絮颠狂,又作漫天之舞。

要知后事如何,但听下回交代。





第九十二回 红倌人安心施巧计 曲辫子拼命害相思





且说陶观察在薛金莲那边请客,辛修甫和他代请了几个客人,一同走到福致里薛金莲家,只见陶观察和着几个人正在那里碰和;见了辛修甫进去,连忙立起身来招呼,有几个和陶观察不认得的人,都是辛修甫邀来的,彼此通过了名姓,便又凑了一场和,两边打得十分热闹。

辛修甫素来不爱碰和,便随便坐下,留心四面一看,只见房间里头只有几个娘姨大姐在那里应酬,却不见薛金莲的影儿。修甫暗想:“这个时候还早得狠,难道已经出了堂唱么?”心上想着,口中也不去问他。坐了一回,一个人觉得无聊得狠,便对陶观察说道:“你们在这里碰和想来还有一会儿,我到西安坊去去就来。”陶观察听了他要走,连忙立起身来一把拉住了,再三留他坐下。辛修甫走不脱身,只得转过身来看着他们碰和。看了一回,八圈渐渐的碰完。辛修甫留心看那薛金莲时,依旧不见出来应酬,心上就觉得狠有些诧异。暗想:那有客人来碰了八圈麻雀,倌人还不出来应酬的道理?忍不住便悄悄的问陶观察道:“怎么我们来了多时,八圈牌都完了,倌人还不出来应酬,是个什么缘故?”陶观察听了呆了一呆,方才说道:“或者是他出局去了也未可知。”辛修甫笑道:“堂子里头那有这般规矩?就是出去应局,也要到客人面前招呼一下,打个转身,那有一声儿不响就去出局的理!”

陶观察想了一想道:“或者他没有出去,在里面有什么事情罢。”辛修甫道:“那越发岂有此理!倌人们挂着牌子做生意,应酬客人就是天字第一号的要紧事情,要是客人来了不肯应酬,又做什么生意呢?”陶观察一时听了说不出什么,只搭讪着叫娘姨们摆起台面来,一面请辛修甫和众人写好局票,发了出去,一面起过手巾,请那一班客人入席。

那一班客人,连着陶观察自己算上去,通共十一个人,今天的酒本来是个双台,十一个人坐着还是十分宽绰。辛修甫见大家已经定坐,薛金莲依然不见出来,那班娘姨连一句客气话儿也不说,径自上来斟酒。陶观察倒还没有什么,辛修甫和陈海秋等一班客人见了他们这般怠慢,一个个心上不快活起来。辛修甫实在熬不住了,便冷笑一声,对着那一班娘姨道:“你们先生究竟到那里去了?我们来了半天,没有见着你们先生的面,只怕上海地方的堂子,没有这个规矩罢?”那班娘姨听了辛修甫发起话来,大家都呆呆的你看着我,我看着你,说不出一句话来。停了一回,一个娘姨方才开口说道:“对勿住,陶大人搭仔各位大人,倪先生来浪吃饭。”修甫听了又冷笑一声道:“我们来了这大半天的时候,难道你们先生一径在那里吃饭的么?一顿饭要吃到这个时候,你们先生真真是个饭桶了。”几句话把大家听得都笑起来。一个大姐听着辛修甫的口风来得利害,便回转身来,一直跑下楼去。

直等到客人叫的倌人一个个都到齐了,还是不见薛金莲的影儿。一班客人个个都觉得有些气忿,有几个不好意思发作出来,只有陈海秋十分性急,便嚷着说道:“客人们差不多都要散了,怎么倌人还不见出来,这是什么缘故?”陈海秋叫的东尚仁范彩霞坐在陈海秋后面,把陈海秋拉了一把道:“勿要嗫,别人家格事体,阿关得耐啥事,嘤嘤喤喤,吵勿清爽,用勿着耐实梗格起劲啘!”陈海秋道:“你不晓得,我们已经来了半天,连倌人的影儿都没有见着,要不和他顶真一下,他还把我们这班客人都当作一些儿不懂的曲辫子呢。”范彩霞听了,把嘴一披道:“好哉好哉,勿要勒浪像煞有介事哉。”

正说着,薛金莲从外面走了进来,见了陶观察和辛修甫等一班客人也不开口,扬着个脸儿待理不理的,把个嘴唇皮略略的动了一动,也算打过了招呼。辛修甫见了薛金莲出来,以为他一定要说几句“对不起”的客气话儿,或者在众人面前斟一巡酒,胡弄局儿的过了场面,也就过去了。那知他坐在陶观察背后,还没有坐到五分钟的工夫,霍的立起身来;对着陶观察只说得一声:“倪出堂唱去。”竟自头也不回,转身便走。满台的客人,见了薛金莲对着陶观察这般模样,不知道到底是怎么的一回事儿,一个个都眼睁睁的看着陶观察,又不好意思问他。陶观察见薛金莲走了,倒一些儿没有怪他的意思,好像没有这件事的一般。

辛修甫本来在那里和龙蟾珠讲话,见了薛金莲这样情形,实在气他不过,冷笑道:“好大架子的倌人,我倒从来没有见过,等会儿等他来了,我倒要来问他一下,吃把势饭的人懂规矩不懂规矩?”陶观察起先听了陈海秋的一番话儿,心上已经有些不狠高兴,又被辛修甫这样的一说,心上更不舒服,只得对辛修甫道:“我们当客人的人,逢场作戏,原是出来寻开心的,倌人们应酬不到,做客人的只要原谅些儿也就是了,何必这样的顶真呢?况且我们赏识的是他的人,不是赏识他的应酬,就是应酬差些却也不妨,我劝你将就些儿,不要挑他的眼罢。一,对着他说道:”

我原是和你代抱不平,和你争这一口气儿,你既然自家愿意这般,那也不必说起。

本来人家捉你的瘟生,与我什么相干?“陶观察听了,自己也觉得有些不好意思,只支支吾吾的说道:”你们不要只顾一味的说他不好,其实他也有他的好处。据我看来,他那一派的形容举止,狠有些儿良家女子的样儿……“辛修甫不等他说完,早哈哈大笑道:”罢了罢了,我也不来管你们的闲帐,你也不必这样的掩耳盗铃。

“

正还要说下去,忽然一阵香风,早见姚红玉急急忙忙的走进来。宝髻垂云,蛾眉掠月,不施脂粉,只淡淡的在嘴唇上点一点胭脂,走进来就坐在陶观察背后,玉容寂寞,半晌无言。陶观察正在一肚子的不快活,见他来了,就盛气对他说道:“客都散了一半,你还来做什么?”姚红玉抬起头来,把两个批头在陶观察头上一推,咬着牙齿说道:“耐格个人……”说了一句,就咽住不说,眼中早掉下泪来。

停了一停,方才说道:“耐自家想想,良心到仔陆里去哉?”陶观察听了他这般说法,究竟抚心自问有些对他不起的地方,便也淡淡的安慰了他几句,姚红玉便起身去了。辛修甫见时候不早,便同着他的相好龙蟾珠一同到西安坊去,大家一哄而散。

陈海秋新做了个范彩霞,也在那里想转范彩霞的念头。这个范彩霞更比不得薛金莲,是个大名鼎鼎的倌人,和那四大金刚的名气差不多,那里看得上陈海秋这样的一个人!但是范彩霞平日之间最爱的是姘马夫、姘戏子,在客人那里千方百计弄来的昧心钱,依旧给那一班马夫、戏子骗得干干净净;更兼他向来服御奢华,用钱挥霍,一连的进款那里够他的用度?拖了一身的亏空再也弥补不来。这个陈海秋是范彩霞那里用钱最多的客人,所以范彩霞当着他的面儿,却也不肯得罪他,只不叫他近着自家的身体。凭着陈海秋怎样的用钱,总不肯露出一个字儿留他住夜。陈海秋想来想去想了无数的法儿,报效了许多的和酒,只指望范彩霞留他住夜,那里知道闹了几个月,依然还是一个不成功。

陈海秋焦躁起来,便也去寻着了辛修甫和他商议。辛修甫也想不出什么法儿,想了一回方才对陈海秋说道:“只有这一个法儿,却不知用起来中用不中用。这个范彩霞是著名倒贴的宝贝,现在差不多将近过年,这个宝贝一定是过不去的,你趁着这个当儿,除了还帐之外,格外借给他几百块钱,这件事儿一定到得手来。你说我这个主意可好不好?”陈海秋听了大喜,便拍着手道:“你的主意果然不差,我就照你这个法儿做去,一定没有不成的。”修甫道:“虽然如此,但是我保是保不来的,只好碰你自家的运气罢了。”陈海秋听了辛修甫的话儿,高高兴兴的竟到东尚仁范彩霞家来。

走进房间,见范彩霞一个人无精打彩的坐在那里,房间里头连娘姨大姐也不见一个。范彩霞见陈海秋走了进来,勉强陪着笑脸,立起身来,自家动手和陈海秋宽了马褂,拉着他坐下。陈海秋刚要开口,早见娘姨阿金、大姐阿玉两个人勾肩搭背,一路嘻嘻哈哈的笑进来。见了陈海秋,阿金便道:“咦,陈老几时来格?”陈海秋道:“我刚刚来的。你们什么事儿这般高兴?”阿玉听了,又掩着口“吱吱格格”

的笑起来。范彩霞皱着眉头道:“勿得知啥格事体,实梗格高兴。”说着便拿过一支金水烟筒,袅袅婷婷的走过去,和陈海秋并肩坐下,亲自和他装了几筒烟。陈海秋见范彩霞忽然这般的要好起来,心上十分欢喜,觉得浑身的骨头都有些痒飕飕的,便顺手把范彩霞抱了过来,坐在自己的身上。范彩霞趁势把纤腰一扭,一个身体便倒在陈海秋的怀中。陈海秋鼻中闻着范彩霞头上的一股头油香气,不觉色心大动,低下头来,脸贴脸的揉了一揉。范彩霞故意嗔道:“勿要实梗哩。”海秋也不理他,只仔仔细细的眯着一双眼睛,看着范彩霞的脸儿,目不转睛的只顾呆看。范彩霞被他看得别过头去,口中说道:“啥格好看呀,阿是勿认得倪?”说着便又格格的笑。

阿金在旁边说道:“勿要实梗高兴哉。今年年底下格开销,耐阿曾自家转转念头,勿要到仔格格辰光弄勿落。”范彩霞听了叹了一口气道:“横竖总归弄勿落,叫倪也呒说法。”阿金道:“陈老搭耐一径要好煞格,耐还是搭陈老商量商量罢。”范彩霞听了也不言语,只把一个脂香粉腻的脸儿紧紧的贴在陈海秋肩上,瞟了阿金一眼道:“耐倒说得实梗容易,只怕陈老勿见得相信倪呀。”说着横波斜溜,宝靥生春,向着陈海秋嫣然一笑。陈海秋被他一阵揉搓,心上早糊里糊涂的没有了主意;又被他这般一逼,更加心荡神迷,捉摸不定,不因不由的说出几句话来。正是:

风情霞思,莺花南国之诗;纸醉金迷,云雨巫山之梦。

不知后事如何,且听下回分解。





第九十三回 花低月亚虚度春宵 凤去台空可怜良夜



且说陈海秋被范彩霞一阵巴结,巴结得十分欢喜,便不因不由的问道:“你今年的生意怎么样?核算起来够开销不够开销?”范彩霞听了便长叹一声道:“勿要说起,房饭钿搭仔菜钿,才欠得一塌糊涂,外势格帐收煞收勿下,格两日倪也呒啥念头转,只好弄到陆里是陆里格哉。”海秋道:“你的客人有钱的也多得狠,为什么不去问他借几百块钱来开销一下,也就过去了,难道他们还会不答应么?”彩霞听了,把头一扭道:“阿呀,耐倒说得实梗容易,耐阿晓得故歇格班客人,用起铜钿来才要称称斤两,格末叫来得精明。俚只要勿漂仔倪格帐,已经算俚好格哉。耐还要去问俚借啥格铜钿,格末勿客气,两个去换俚一个,陆俚有啥个个客人才像耐陈老实便介。”陈海秋听了这几句话儿满心欢喜,口中却对他说道:“我也不见得一定就怎样的大量,你不要在这里灌我的米汤。”范彩霞道:“倪是勿会灌啥格米汤格,要末耐……”说到这个地方,把脸一红,飞了陈海秋一眼,低着头微微一笑。

这一笑,就把陈海秋的一个身体酥了半边,动弹不得。

又听得范彩霞郎然说道:“格号闲话,啥人高兴去搭俚笃说呀,洋钿拿勿着,白白俚坍仔自家格台,想起来啥犯着呀,勿比耐洛勒浪倪搭,赛过搭倪自家人一样,搭耐讲仔也无啥希奇”说着更把一个身体紧紧的往陈海秋怀中贴了一贴,附着陈海秋的耳朵低低说道:“耐一径啥洛勿来介,倪有几几化化心浪向格闲话要搭耐说。”

这一来,更把陈海秋弄得遍身瘫软,好似雪狮子向火──融化了半边,张开了一张大口再也合不拢来。停了一回,方才向范彩霞道:“你有什么话儿,何不趁着这个当儿和我讲个明白。”

彩霞听了暗暗的好笑,本来是随口讲的一句话儿,那里真有什么说话,只得瞪了陈海秋一个白眼道:“耐格人啥实梗性急呀,晏歇点慢慢里搭耐说。”陈海秋听了,这一刻儿的心上高兴,在下做书的也形容他不来,只对着范彩霞呵呵的痴笑,笑了一回,方才问着范彩霞道:“你过年要借多少钱,只顾问我拿就是了。”范彩霞听了,便道:“勿瞒耐说,倪间搭过年格开销,一塌刮仔总要五百洋钿。”陈海秋不等他说完,便接下去说道:“五百块钱,什么大不了的事,也值得急到这个样儿。”说着,便叫阿金去抄局帐。

阿金走了出去不多时,拿着一篇局帐走进来,陈海秋接过去看一看,只见通共三十几台酒,一百几十个局,差不多也有四百块钱的光景。陈海秋看了,便从身边取出一张一千块钱的汇票来,交在范彩霞手内,口中说道:“这一千块钱除了你借的五百块钱,还有四百块钱局帐,这余下来的一百块钱,就算了手巾送礼的开销罢。”

范彩霞见了,登时满面天花的伸手过去,把票子接了过来,口中却还说道:“谢谢耐,借仔几化洋钿拨倪,总算耐搭倪帮仔一个忙,勿然是今年底下倪直头一塌糊涂哉。”陈海秋听了,便低低的问着范彩霞道:“今天晚上你打算怎么样呢?”范彩霞听了,不由得春压肩梢,红生宝靥,一言不发的只看着海秋笑。陈海秋又问了一声,范彩霞嗔道:“晓得哉,耐格人啥是实梗格介,晏歇点……”说了这三个字儿,便顿住了口不说下去。

正在这个时候,忽然下面相帮高声叫道:“姓王格叫到一品香,姓陆格叫到金谷春,姓洪格叫到谦吉里。”范彩霞听了,故意眉头一皱,立起身来口中咕哝道:“格排断命客人,格末叫来得讨厌,倪格碗把势饭也吃得恨尽恨绝格哉。”说着,便又去陈海秋耳边说了几句,不知说的是些什么,见陈海秋连连点头。范彩霞换好了衣服,对着陈海秋道:“陈老,对勿住,倪出堂差去,耐勿许去格哩。”陈海秋道:“你出去应局,料想不是一刻儿的工夫,我去一去再来罢。”范彩霞听了不依道:“倪勿要,耐搭我好好里坐来浪。”说着回过头叫阿玉道:“耐搭倪看好仔俚,勿要放俚出去。”陈海秋哈哈的笑道:“好得狠,索性把我当起犯人来了。”范彩霞又分付了阿玉几句,自己同着阿金走了。

陈海秋坐着等了多时,范彩霞还没有回来,这个时候已经将近年底,堂子里头没有什么客人。陈海秋一个人坐在那里,呆呆的等了又等,等得陈海秋焦躁起来,跳起身来要走,又被阿玉死命拦住,不放他走。正在扭结固结,忽见阿金气喘吁吁的走了进来,陈海秋以为范彩霞回来了,登时又坐了下来。只见阿金走过来对着他说道:“先生勒浪谦吉里洪公馆里向代碰和,格格客人格末叫气数,碰仔八圈倒说再碰八圈,定规要倪先生搭俚代碰,倪先生恐怕陈老勒浪等仔心焦,叫倪赶转来搭陈老说一声,先生说请陈老勿要性急,俚就要转来快哉。”说着又叫了阿玉,两个人到后房去嘁嘁喳喳的讲了几句不知什么话儿,阿金便要要紧紧的走了。陈海秋本来等得十分焦躁,一定跳起来要走,却听了阿金的几句话儿,不知不觉又软绵绵的坐了下来。又等了一点多钟,看看身边的表时,差不多将近两点钟了,直把一个陈海秋等得意懒心灰,神疲气索,要想不管三七二十一毅然决然的走了罢,眼看着这样的到口馒头觉得有些舍不得,想了一回,心上转一个念头道:“他既然特地叫阿金回来把我留在这里,自然就要回来的,如若不然,他又何必这样的骗我呢?”正想着,阿玉端了一个茶碗进来递给海秋道:“陈老吃一点点杏仁露。”陈海秋正在口渴,接过来一口气就喝了一个干净,歪在炕上觉得有些睡意,朦胧的上眼皮找不起下眼皮来,便不觉和懵腾睡去。

这一觉不知怎样的直睡到红日三竿,方才觉得有个人在他身上乱推乱搡的。搡了一回,海秋猛然惊醒,睁眼看时,只见范彩霞和娘姨阿金、大姐阿玉一班人都立在面前。范彩霞一面推着,一面叫道:“陈老,辰光勿早哉,啥洛实梗好困介?”

陈海秋擦了一擦眼睛坐起身来,心上还有些模模糊糊的,把昨天晚上的事情早忘记了一半。看着范彩霞,呆了一回方才仿佛有些记起昨天的事来,却不知道范彩霞什么时候回来,自己又怎样的会一觉直睡到这般时候,想来想去不得明白,只得问着范彩霞道:“你是什么时候回来的?我昨天晚上直等你到四更时候,你还没有回来,不知怎样的我自己也困倦起来,直睡到这个时候。”范彩霞听了几乎要笑出来,恐怕被陈海秋觉着,连忙别转头去忍住了笑,打了两个呵欠,方才开口说道:“倪拨格个断命客人一径要拉牢仔搭俚碰和,煞死格勿肯放,倪心浪向牵记仔耐,几乎急杀快,一直搭俚碰到仔天亮,刚刚完结,倪转来仔也呒拨几化辰光。”

陈海秋见了范彩霞这样的一个人,婷婷袅袅的立在面前,两鬓惺忪,春情满面,那两边颊上隐隐的透出两朵桃花,越显得皓齿明眸,丰神绝世。想起昨日的事情来,自己觉得十分懊恼。暗想好容易得着了这个机会,看着一个大肥的河鸭盖在锅子里头的,梦想不到会出了这样的岔儿,到了这个时候已经是红日满窗,料想是不能的了,便似笑非笑的对彩霞说道:“我昨天晚上了你的当了。”彩霞听了不觉面上一红,春色横眉,娇羞上脸,走过来附着海秋的耳朵道:“耐勿要噪嗫,教倪阿要难为情!”陈海秋听了便不开口,立起身来胡乱洗了一把脸,便走了回去。

到得晚上,陈海秋一个人又跑到范彩霞那边来,一团高兴的要想在他那里请客。

那知到得范彩霞大房间里头,范彩霞的影也不见,只有大姐阿玉一个人坐在那里。

问他先生那里去了,阿玉把嘴往后面一努道:“倪先生来浪生病,耐进去看看俚虐。”陈海秋听了十分怪诧,刚才自己走的时候,明明的范彩霞还是有说有笑的并没有什么毛病,怎么一会儿的工夫就会生起病来。想着,便自己走到后面房间里去看他。只见范彩霞拥着一条湖色绉纱的绵被,和衣睡在铁床上;娘姨阿金正坐在床沿,和他密密切切的讲话。见阿玉同了陈海秋来进来,便道:“阿呀,格搭地方龌龊煞格,耐还是外势去坐歇罢。”陈海秋道:“不要紧,我听见你无缘无故的生病,所以来看你一下,你们何必同我这般客气。”彩霞听了,瞅了阿玉一眼道:“倪呒啥病呀,耐末总要实梗瞎说瞎说。”阿玉道:“耐自家昨日仔夜里向出堂差受仔风寒,一径勒浪吵肚里痛,倒要叫倪瞎说,倪倒一点点才< 曾忽> 瞎说虐。”陈海秋听了,便问:“为什么肚子痛,大约是昨天晚上受了寒罢?”范彩霞摇摇头道:“倪格肚里痛是老毛病呀,日常格辰光一径要发格,到仔一个月……”范彩霞说到这里,看着海秋一笑,顿住了口不说下去。海秋看了,不懂他是什么意思,正待问时,阿金从旁边接口说道:“陈老耐勿晓得,倪先生一径有个痛经格毛病,一个月里向到仔归格辰光,就要发一转肚里痛格毛病,郎中先生勿知请仔几几化化,总归医勿好。”阿金说到这里,范彩霞伸过手来他打一下道:“耐格号人,总归欢喜实梗瞎三话四。”阿金道:“陈老亦勿是啥别人,搭俚讲讲也无啥希奇啘。”陈海搂听了,心上大大的不高兴,明晓得自己的事儿又是不成功的了,却又不便说出什么来,只得嘿嘿无言,闷在心里。范彩霞见了陈海秋一言不发,知道他心中不乐,便把纤手对他招招,叫他走过来坐在自己身旁,和他低低的讲了几句,又对着他笑道:“对勿住,只好屁股里吃人参──后补格哉。”陈海秋听了只得点一点头。又坐了一回,范彩霞催他回去了。

在陈海秋的意思,还只当着范彩霞不是有心骗他,不过自家的运气不通,所以刚刚碰得这般凑巧,指望以后还要和范彩霞怎样的蛱蝶双飞,鸳鸯颠倒。那里晓得上海堂子里头倌人的伎俩,真真的好似那九尾神狐,通天魑魅。那些哄骗客人的方法千变万化,层出不穷。这些做嫖客的人,又一大半都是些曲辫子、土老儿,那里是他们的对手?正是:

碧城十二,苍茫情海之波;弱水三千,缥渺蓬莱之路。

不知后事如何,且看下回交代。





第九十四回 陈海秋痛恨范彩霞 章秋谷重游安垲第



且说陈海秋要想转那范彩霞的念头,白白的借了五百块钱给他,只指望要得些好处,那里知道受了范彩霞的圈套,花了无数的钱,毛也没有捞着一根。起先还只说彩霞当病好之后自然相就,还天天跑到东尚仁去看他,范彩霞的病本来原是假的,一天一天只得含含糊糊的搪塞他。

一直到了二十八的这一天,陈海秋也有些觉着范彩霞对着他不是真心,心上十分气愤,一口气跑到辛修甫那边,气呼呼的把范彩骗他的事情告诉了修甫一遍,又埋怨修甫道:“总是你和我出的主意,如今弄得羊肉吃不着,惹得一身骚,倒上了他的恶当。”辛修甫笑道:“主意虽然是我出的,我本来和你说明保是保不定的,况且这件事儿是你自家不好,所以上了他的当,与我什么相干?”陈海秋听了跳起来嚷道:“原是你出的主意,怎么倒说是我自家不好呢?”辛修甫道:“如今世界上的事情,第一贵重的就是金钱,只要有了钱,无论什么事情都是做得到的。他们一班堂子里头的倌人喜欢的是钱,他银钱没有到手的时候要想骗你的钱,不得不好好的巴结你,等到银钱已经到了他的手中,就是你吃了他的空心汤团,你也没有什么法儿。那个叫你急急的先把五百块钱送到他的手中,他不哄骗你难道哄骗我不成?

他们吃把势饭的好容易遇见了你这样的瘟生,不好好的敲你一下竹杠,专靠着几个叫局吃酒的钱来开销这个门户,那他们就都要喝西北风过日子了。“陈海秋听了,觉得辛修甫的话实在不差,便道:”这件事情果然是我一时大意,上了他的当,但是我平空吃了这个亏,难道罢了不成?毕竟要想个法子,把他弄得伏伏贴贴的自己降心相就,方才出得我这一口气儿。你可有什么主意没有?“

修甫想了一回想不出来,便道:“我在这个里头究竟还是个外行,可惜章秋谷不在这里,他便想得出那些千奇百怪的法儿。他常常对人说道:”天下的事情,除了穷苦的人没有钱用,害病的人医治不好,这两件事实在没有法儿,别的事情,凭你再是天大地大,无大不大的事,也有法儿好想的。兵来将挡,水来土掩。一个人只要有了思想,那有做不来的事情?‘若是他在上海的时候,只要和他商议一下,一定想得出一个主意。如今他既然不在这里,我又实在想不出什么法儿,只好随后再说的了。“陈海秋听了道:”你的说话不差,从前我做花筱舫的时候,花筱舫有心和我过不去,就是章秋谷在里头提兵调将,出了个主意在陈文仙那里碰和,把花筱舫叫了来大大的责备了他一顿。这件事儿是我平生第一件畅快的事情。如今可惜章秋谷已经回去,若是他在这里,一定要和我出个主意的。“

辛修甫忽然失声笑道:“天下的事情真是无独有偶的,你们两个人真算得一对大大的瘟生。”陈海秋听了觉得好笑道:“好好的和你说话,你又要取笑起来,像我这样的客人那里算得什么瘟生?那位陶观察才是个有一无二的瘟生呢!”辛修甫听了更哈哈的大笑道:“岂敢岂敢,我说的本来就是那位陶观察的事儿。你们两个人,一个要转范彩霞的念头,一个就要想充薛金莲的恩客;一个受了薛金莲的怠慢,一个就入了范彩霞的牢笼。有你们这样的一对客人,便有他们那般的两个妓女,你们两个人岂不是同病相怜,无独有偶么?”陈海秋听了,实在自己解说不来,只得笑道:“好了好了,不用说下去了,就算我们两个都是大大的瘟生,你只把陶观察的事儿讲给我听罢。”辛修甫听了,便把陶观察那一天同着他一同到福致里去送帐的情形,和陈海秋一一的说了。

看官,你道陶观察什么事儿?原来陶观察也和陈海秋一般,要想和薛金莲攀相好,薛金莲那里肯依。陶观察想去想不到手,便也想着趁着这个年底的当儿,送一笔钱给他,或者薛金莲感激涕零,竟肯以身图报也未可知。陶观察定了主意,便邀了辛修甫同去开销局帐,辛修甫听说“薛金莲”的三个字儿,心上便有些不大高兴,却又不好意思不去,只得同着陶观察往福致里来。

到了那里,陶观察和辛修甫两个人坐在房间里头,足足的坐了两个时辰,把个辛修甫等得火星直冒,薛金莲方才走了出来。陶观察便从衣袋里头取出两卷钞票,先拣了一卷,递给薛金莲道:“我的局帐菜帐大约不过三百几十块钱,这里头六百块钱的钞票,你且收了。”薛金莲谢也不谢一声,大模大样的接了过去。把那一卷钞票看了一看,又瞅了陶观察一眼,便把那一卷钞票一张一张的抖了开来,在那里一五一十的点。陶观察见了倒不觉得怎样,辛修甫心上不由的动气起来,冷笑一声道:“你当心点儿仔细看一看,陶大人的钞票都是假的,你不要上了陶大人的当。”

薛金莲听了辛修甫的这几句话儿,也有些觉得辛修甫是有心骂他的,便抬起头来看了辛修甫一眼,把钞票放了下来。陶观察又把另外的一卷钞票递过去道:“这是四百块钱,给你留着新年上用罢。”薛金莲见了,也不伸手来接,只把嘴望着烟盘里头一努道:“耐放勒浪末哉。”陶观察见他不肯来接,只得依着他的话放在烟盘里头。

薛金莲停了一回方才冷洋洋的道:“格个钞票拿得来做啥,阿是算送拨倪格?”

辛修甫听了,不待陶观察开口,早接过去说道:“这个自然。不是送给你的,难道是送给我的不成?”薛金莲微微一笑,口中说道:“格末陶大人请耐勿要实梗费心,留仔自家用用罢。倪穷末穷,过年格开销还开销得转勒里,用勿着耐陶大人实梗要好。”陶观察听了心上愕然,不懂薛金莲是什么意思,便道:“你为什么不肯受,可是嫌少么?”薛金莲道:“勿瞒耐说,倪是用俚勿着,唔笃格贵相好姚红玉极煞勒浪,耐还是拿得去送拨仔俚罢。”陶观察听了,只道他还在那里和姚红玉吃醋,便笑着说道:“你的器量怎么这般的狭窄,你自己想想,我待姚红玉是什么样儿,待你是什么样儿,你何必还要同他吃醋?”薛金莲听了把嘴披了一披,鼻子眼里“哼”的笑了一声,立起身来,他右手的一个中指一直送到陶观察嘴边,大声说道:“阿是倪要搭姚红玉吃醋,阿唷阿唷,耐勿要勒浪瞎三话四哉,倪搭姚红玉末吃啥格醋?啥格叫吃醋,耐倒搭倪讲讲看,倪搭耐亦< 曾忽> 攀啥格相好,为仔啥格事体要搭姚红玉吃醋介?格号闲话讲出来,赛过放屁,唔笃听听看,阿要像煞有介事。”陶观察平空被薛金莲教训了一顿,并不生气,还是笑嘻嘻的对薛金莲说道:“不要生气,不要生气,总算是我错了何如?”薛金莲听了,又瞪了他一眼道:“生来是耐错啘,啥格吃醋勿吃醋,瞎说瞎话。”陶观察听了,又把那烟盘里头的一卷钞票取过来,塞在薛金莲手内,口中说道:“吃醋不吃醋,不必再去提他,但是这个钱是我送给你的,你为什么一定不收,可是瞧我不起么?”

看官听着,世上的人,只有嫌着钱少的心肠,那有倒反嫌着钱多的道理。何况薛金莲是个堂子里头的人,见了白花花的四百块钱,又是自己送上门来的,那肯不受?不过平日之间摸着了陶观察的脾气,是个吃硬不吃软的人,明晓得这四百块钱是飞也飞不到那里去的,落得摆些身分不要他的,也好装装自己的腔。如今见陶观察这般说法,便趁势说道:“格末谢谢耐,送仔倪实梗几几化化格洋钿,不过倪有句闲话要搭耐讲明白仔,格个洋钿是耐自家情愿送拨倪格,倪是从来< 曾忽> 问耐借过歇啥洋钿,耐歇歇点跑出去搭别人家讲起来,只说薛金莲过年过勿落,要问耐借洋钿,格是倪定规勿成功格嗫。”陶观察道:“这个自然,我又不是个小孩子,那里会这样糊涂。”辛修甫在旁边听着薛金莲这些说话,已经气满胸膛,更兼看着陶观察这般模样,越发气得不可开口。暗想:“天下怎么竟会有这种人,糊涂到这般田地。”待要发作几句,忽然心上转一个念头道:他自己情愿受他们的怠慢,与我什么相干?更兼这位陶观察是个糊涂蛋,对着他说了薛金莲待他不好,他非但不知感激,而且倒反还要生起气来,我何必白寻烦恼,去管他们的闲事呢。想到这里,气早消了一半,立起身来对着薛金莲冷笑道:“陶大人有了钱,怕没有地方去用,特地恭恭敬敬的送到你这里来,你何必和他客气,不是落得受用的么?”说着,又向陶观察道:“你请一个人在这里坐一回儿,我有些小事要先走一步,不得奉陪。”

说罢往外就走。陶观察还想留他,辛修甫回过头来道:“我要再在这里坐一回儿,胀破了肚子叫那一个和我抵命呢?”说着急急的走了出去。这且按下不提。

只说章秋谷到了上海,便先去看陈文仙,两个人别后重逢,自然是欢畅非常,互相慰问,春云乍展,玉镜刚圆;宝扣亲除,银钩暗荡。证相思于此夜,人面依然;问洞口之桃花,渔郎无恙。秋谷在家过了一夜,直到明日十点多钟方才起来。

这个时候正是四月初的天气,春归南浦,绿满林皋。大家都换了单罗夹纱的衣服,秋谷便对陈文仙说道:“今天礼拜六,我也懒得出去,我们雇一辆马车到张园里看看如何?”文仙听了便也点一点头。吃过了饭,秋谷早叫人到善钟马房去雇了一辆橡皮亨斯美快车来,放在门首。秋谷换了衣服,看着陈文仙装饰好了,穿一件白罗夹袄,戴一头翡翠簪环,淡淡蛾眉,弯如新月;盈盈媚眼,静若澄波。慢慢的移步出来,同着秋谷一同坐上车去。秋谷拔出鞭子,理顺丝缰,只把右手的鞭子一扬,左手的丝缰一抖,那马早放开四蹄,泼喇喇的向前跑去。新马路到张园,本来没有多少路,风和日丽,草软沙平,秋谷的马车一路如飞跑去。

到了张园,秋谷循着曩例,把马加上一鞭,抢到安垲第门前停住。秋谷在车上轻轻的一跃而下,陈文仙也跟着下来。秋谷站在安垲第门首,抬起头来四面一望,只见绿阴遍地,碧草如茵,一阵白兰花的香气,夹着晚风直送到鼻管里来。

不知后事如何,且听下回分解。





第九十五回 当冤桶观察开心 吊膀子张园受辱





只说章秋谷同着陈文仙到了张园,两个人一同走进安垲第去,四周看了一看,见那些男男女女来吃茶的人倒也狠多,男的一个个都是画扇轻衫,女的一个个都是纤腰皓腕,来来往往的十分热闹。秋谷同着陈文现拣一张桌子坐下,泡了一碗茶坐了一回,觉得没有趣味,便招呼堂倌把茶留下。那几个堂倌本来都认得秋谷的,便诺诺连声的答应,秋谷便同着陈文仙走出来四面闲逛。

到了外面觉得空气清新,神气为之一爽。秋谷因为自己半年不到这个地方,便抬起头来细细的四面观看,只见还是那几处的亭台楼阁花木池塘,并没有添出什么来。秋谷同着陈文仙一面讲话,一面慢慢的向前走去,只见板桥几曲,流水一弯,树底残红,春魂狼藉,枝头新绿,生意扶疏,已经换了一派初夏的景候。各处走了一回,陈文仙只累得香汗淋漓,微微娇喘,秋谷见陈文仙有些走不动,便搀着他的手一路走回来。已经日色西沉,归鸦噪晚,安垲第门外却马龙车水的拥挤非常,都是那些堂子里头的倌人,一个个敷粉涂脂,争娇斗艳。那天上斜阳的光线一丝一缕的直射过来,飐着这些倌人头上的珠翠,便觉得光华飞舞,耀得人眼睛都有些花花绿绿的看不清楚。

秋谷同着文仙正走到安垲第门外将要进去的时候,只见滔滔滚滚的一连来了两辆马车。前一辆车内坐着一个四十多岁、方颐大耳、乌须白面的人,看他脸上的气派好像是个当道贵官的样儿。只见这个人跳下车来,立在门首且不进来,等着后面一辆马车过来。马车里头走出一个满头珠翠的倌人,这个人连忙要上前去搀他,那倌人把眉头一皱,嗔道:“勿要嗫,算啥介,耐搭倪先跑进去。”这个人听了,恭恭敬敬的答应一声,便依着那倌人的话儿先走进去。这个倌人在外面略略的站了一站,等着那前面的人已经走了几步,方才慢慢的走进来。秋谷见了,对着陈文仙道:“这个倌人分明就是那濂溪坊的薛金莲,怎么对着客人这般模样?”说着,便同着陈文仙跟在那薛金莲的后面也走进去。见他走进安垲第四面走了一转,那男子也跟在他的后头,薛金莲在前走着头也不回,径自拣了一张桌子坐下,刚刚紧靠着章秋谷、陈文仙的那张桌儿,正在章秋谷的对面。那个男子见薛金莲坐了下来,便也想在薛金莲旁边坐下。薛金莲登时变转脸来,把手在桌上一拍道:“耐勿要坐勒倪搭,坐勒格面去末哉啘。”那男子听了也不动气,连忙就走到旁边一张桌子上坐下。

堂倌泡上茶来,那个男子又跑到薛金莲面前,问他要吃什么点心不要。薛金莲皱着眉头道:“耐格人总归实梗鸭矢臭,一日到夜吵勿清爽,吵得倪头脑子也涨杀快。”

那男子听了,便又跑到那边坐了,还只顾目不转睛,看着薛金莲的脸儿。

这些情形都被秋谷、陈文仙看在眼里,文仙悄悄的对秋谷说道:“耐看格个曲辫子曲得来。”秋谷看了薛金莲看待客人这般模样,心上狠觉得有些愤愤不平,便对陈文仙说道:“天下真有这般的奇事,做嫖客出了钱到堂子里头去顽,原是要寻开心的,都照着这个宝贝的样儿,那就是自寻苦恼了。最可怪的,倌人们吃这碗饭原不过是为两个钱,怎么薛金莲的看待客人竟是这般模样,岂不是笑话么?”陈文仙道:“他怠慢他自己的客人,与我们什么相干,何必去管他的闲事。”秋谷道:“那个去管他们的闲事,不过我在旁边看着,心上气愤不过,这般讲讲罢了。”

正说着,忽见一个倌人从秋谷后面转将过来,丰态清扬,妆梳雅淡,山眉水眼,雾鬓风鬟,一步一步的慢慢向前走去,忽然回过头来把秋谷看了一眼,不觉失声叫道:“阿呀,二少啘,啥辰光来格呀。”秋谷听了连忙仔细看时,认得他不是别人,就是辛修甫的相好西安坊龙蟾珠,连忙微微含笑的立起身来,招呼他坐下。龙蟾珠又回过头来和陈文仙打了一个招呼,方才就向上首一张椅子上坐下。龙蟾珠向来因为章秋谷是辛修甫最知己的朋友,每逢秋谷同着辛修甫到他院中的时候,龙蟾珠应酬秋谷格外尽心。秋谷在朋友的一班相好中间,最赏识的就是龙蟾珠。说他沉静过人,丰姿出众。如今龙蟾珠殷殷勤勤的和他讲话,便也随意应酬了几句,又问他这几天可见辛修甫?龙蟾珠道:“辛老有一礼拜勿到倪搭来哉,耐看见仔俚,请俚到倪搭来。”秋谷随口答应了一声。龙蟾珠又道:“二少,耐格贵相知,今年才调仔头哉,一个来浪久安里,一个勒浪迎春坊,看见仔倪一径勒浪问耐呀。”秋谷笑道:“我如今还有什么相好,你说的是什么人?”龙蟾珠笑道:“陆丽娟搭仔梁绿珠,勿是耐格相好,是啥人格相好呀?”秋谷道:“那算不得相好,不过应酬朋友,随便叫几个局罢了。”

正在这个时候,忽听得陈文仙在对面咳嗽一声,秋谷不知道什么事情,连忙抬起头来看时,只见陈文仙把嘴往那边一指,秋谷顺着他指的一方面看过去。只见那边台上的薛金莲对着自己目不转睛的只顾呆看,两只眼睛水汪汪的,腮边颊上早现出两朵红云。秋谷见了,知道他在那里吊自己的膀子,但方才见他待那同来的客人那般怠慢,觉得自己也是个嫖客,兔死狐悲,物伤其类,心上很有些恨他,凭着他在那里弄眉挤眼的卖弄风情,只当作没有看见的一般,由着他一个人去做作。陈文仙和龙蟾珠看了,把手巾掩着嘴,格格的只是要笑。龙蟾珠忽地对着秋谷悄悄的把手往对面一指,低声问道:“耐阿认得俚呀?”秋谷也低低的说道:“濂溪坊的薛金莲。”龙蟾珠摇一摇头道:“倪勿是说俚呀,格个坐勒薛金莲左首格客人,耐阿认得俚,搭辛老一淘格朋友呀。”秋谷摇头道:“修甫的朋友我一个个都认得的,却从没有见过他,或者是修甫近来结交的朋友也未可知。”看官,你道这个寿头码子的客人,究竟是个什么人?在下做书的就是不说出来,看官们也一定知道,自然就是那位广东来的陶观察了。

当下龙蟾珠又把薛金莲和陶观察两下事情细细的和章秋谷低说一遍,章秋谷听了越发心上有些不伏,看着对面的薛金莲眉花眼笑,把眼风只顾望着秋谷溜来。秋谷只是洋洋不睬,不去理他,却故意对着陈文仙、龙蟾珠两个人大声说道:“我最恨的是那班野鸡妓女出身的倌人,凭你怎样的花运亨通,香名鼎盛,那一言一笑,一举一动,总还都带着野鸡妓女的下贱样儿。他自己虽然不觉,旁人的眼睛却看得狠清楚。”几句话把陈文仙和龙蟾珠说得都格格的笑起来,明晓得是有心骂薛金莲的。薛金莲正在那里吊膀子吊得出神,忽然听了章秋谷这般的一番说话,一句一字好像是有心骂他,枭他的痛疮。这一气非同小可,只气得他目瞪口呆,心窝冰冷,一天的高兴都不知那里去了。赌气立起身来往外便走,陶观察见了,连忙也跟了出来,两个人上了马车,一直回到濂溪坊去。薛金莲碰了章秋谷一个大大的钉子,无可奈何,便把陶观察来出他的气。只说出去坐了一趟马车有些头痛,埋怨着陶观察道:“倪原说格两日子探仔牌子勿出去哉,耐定规拖牢仔倪一淘出去,故歇害得别人家头脑子里向痛煞快。”陶观察见他生气,那敢多嘴,只低声下声的安慰他。

看官,你道薛金莲为什么平空除了牌子不做生意?原来薛金莲和那郑小麻子两个人搅在一起,搅得火一般的热,盟山誓海的说要嫁他,好在金莲的娘是亲生娘,薛金莲总算是自家身体,做了五六年野鸡,升了书寓;又做了两年,倒也和他挣了不少的钱。金莲一年以前早已和他的娘说明,将来嫁人不要他的身价。如今见金莲要嫁人,也不好一定怎样的阻格他,心上却嫌着郑小麻子是个穷光蛋,便和薛金莲说明了不要身价,只要郑小麻子自己拿出一千银子来,做院中的下脚开销,犒赏经费。薛金莲听了,明晓得郑小麻子是一个大钱没有的宝贝,平日的零用都是自己给他的,那里拿得出这一千银子?自己虽然有钱,究竟一千银子数目大子,白花花的拿出来,也觉得有些心痛。想了一回,便想出一个主意来,立刻叫娘姨金珠到泰安栈去请陶观察,请他即刻就来。陶观察听得薛金莲忽然来请他过去,好似奉着九重纶綍的一般,连忙飞一般的赶过去。正是:

落花有意,空留金谷之春;流水无情,不逐胡麻之饭。

欲知后事如何,下文再表。





第九十六回 借洋钱硬捉瘟生 呼将伯欣逢故友





且说陶观察听得薛金莲叫人请他,心中大喜,便立时立刻的赶到福致里来。薛金莲见了,殷殷勤勤的亲手和陶观察脱了马褂,推他在炕上坐下。这是陶观察自做薛金莲以来从来没有的事情,这一阵巴结,只把一个陶观察巴结得坐立不安起来。

薛金莲等他坐了一回,方才对他说道:“陶大人,倪今朝请耐来,有句闲话搭耐说,勿得知耐阿肯答应勿肯答应?”陶观察听了,连忙说道:“有什么事情,怕我不答应,你只顾讲就是了。”薛金莲便走过来,和陶观察并肩坐下,一只手搭在陶观察的肩上,口中说道:“倪格两日要嫁人哉,耐阿晓得?”陶观察吃了一惊,连忙问嫁什么人?薛金莲道:“就是归格姓郑格广东人,来浪倪搭,也是五六年格老客人,故歇夹忙头里倒说要讨起倪来哉。依仔倪自家心浪,老实说有点勿高兴,吃着倪格呒姆已经去答应仔俚哉,故歇也叫呒说法。不过倪来浪间搭欠仔几几化化的格债,故歇一塌刮仔才要还。倪呒拨洋钿来浪,格陆请仔耐陶大人来搭耐商量商量,勿得知耐阿肯搭倪想想法子?”

看官且住,陶观察虽然糊涂,究竟也是个人,天下那有倌人将要嫁人的时候,还要和客人借钱的道理?况且这位陶观察在二月里头的时候,想要把薛金莲娶回家去,无奈这件事儿觉得自己不好出口,特地到浙江原藉去接了太太出来,在大餐馆里头叫了薛金莲的局,这位太太就当面和他说明,说陶观察要娶他做姨太太,身价银子不论多少。那知薛金莲一口回绝,咬得斩钉截铁,不肯答应,陶观察也无可如何。以前既然有过这样的一重过节,如今薛金莲要嫁别人,怎么竟会和陶观察当面锣对面鼓的这般明讲,可不是在下做书的有心掉谎么?但是这件事儿实在是真真实实,的的确确,在下做书的不敢掉一个字儿的谎,这叫做理所必无,情所或有。看官们,莫提闲话,且听正文。

只说陶观察听了薛金莲的话儿,凭你再是怎么天字第一号的瘟生,心上究竟有些儿不快,低着头只在那里沉吟。薛金莲见了,知道陶观察心中不悦,便拉着他的手低低说道:“耐阿是听见仔倪要嫁人来浪动气?二月里向格事体勿关倪事,是倪格呒姆勿肯答应呀!陶大人,耐勿要来浪瞎转啥念头。老实搭耐说,倌人嫁人陆里肯告诉别人?倪为仔耐陶大人比勿得别人,一径待倪要好煞,赛过是倪自家人,告诉仔耐也呒啥要紧,陶大人阿对?”陶观察被他一阵米汤灌得满心欢喜,觉得自从薛金莲院中走动以来,薛金莲总是板起了一付吃生葱的脸儿,耳朵里头从没有听过这样的一番委婉温柔的好话,不知不觉的脱口答应出来道:“你要多少银子,只顾向我拿就是了。”当下陶观察立刻拿出一千五百块钱的银票给了薛金莲,又和他亲热了一回,方才被薛金莲催了回去。

隔了一天,薛金莲已经除了牌子,陶观察又跑了来,要和他同坐马车到张园去。

起先薛金莲不肯,只说除了牌子不便出去。倒是他的娘在旁边看不过,催着薛金莲同他出去。薛金莲没奈何,只得同着陶观察坐了两乘马车,到张园去坐了一坐。恰恰的章秋谷同着陈文仙也到张园,他们两个人的情形被章秋谷看得明明白白。薛金莲无意之间抬起头来,猛然看见了章秋谷,觉得这个少年意气非常,风华出众,长眉挹秀,凤目含威,好像眼睛里头从来没有见过这般人物,由不得心中一动,想要吊起章秋谷的膀子来。那里知道章秋谷心上正在恨他,那里还肯和他要好。薛金莲落了一场没趣,口里头又说不出来。这且按下不题。

再说起张园里头的章秋谷来,见薛金莲老羞变怒的起身去了,不觉回过头来对着陈文仙和龙蟾珠哈哈一笑,陈文仙和龙蟾珠也笑了一回。章秋谷坐了一会觉得没有什么趣味,见大家都纷纷的上车去了,便也别了龙蟾珠,同着陈文仙上车回来,到了新马路公馆里头,早已是夜气沉山,灯光照夜。坐了不多一刻,忽然听得外面雷鼓也似的敲门,当差的走出去把门开了,早听得陈海秋的声音一路大叫进来道:“秋谷兄,怎么你到了上海不来看看我们朋友,却先去逛起园来,这是什么道理?”

秋谷听了是陈海秋,便在楼上急急的走下来,彼此相见,打了一拱,知己重逢,故人相见,自然心上都十分欢喜。

陈海秋还没有坐下,便道:“你怎么回去了这许多时候,把我们这一班旧时的朋友都撇到脑后去了,是不是?”秋谷道:“那里有这个话,我回去之后家里有些家事,外面又有些应酬,忙得不得分身,并不是忘了旧时朋友。你有什么话坐下来讲,怎么尽着站在这里。”陈海秋听了方才一屁股坐下道:“成天的望你不来,连我的眼睛都几乎望穿,肚子都几乎气破。”秋谷听了诧异道:“怎么,你望我不来,眼睛几乎望穿也还罢了,怎么好好的会几乎气破肚子,这是个什么缘故?”陈海秋摩着肚子,口中说道:“不要说起,说起来就气死人。我吃了人家许多的亏苦,一向闷在心里说不出来,专等着你来了,好和我想个法儿。”秋谷听了,心上已经猜着了几分,知道他一定是在堂子里头吃了亏了,便问道:“究竟什么事儿,你且先和我讲个明白。我章秋谷虽然赶不上那黄衫客、古押衙,却也自负满腹经纶,一身侠骨,只要可以和你出力的地方,凭你什么天大的事儿也不在我的心上。”陈海秋听了,便把自己和范彩霞的事情和他讲了一遍:怎样的想他的念头,怎样的想不到手,怎样的辛修甫和他出主意,怎样的被他借去了五百块钱,到得后来终久还是不成功,详详细细,本本原原说得十分明白。

秋谷听了,低着头沉吟一回道:“这件事儿来得十分奇怪,缶么早不出局,迟不出局,偏偏到他留你住夜那一天,就有人要他代碰起和来,这还说是他们做成的圈套,不必说了。但是你平日之间并不一定怎样的贪睡,怎么刚刚的那天晚上你就会糊里糊涂的睡了一夜,直到明天早上才醒呢?况且你那个时候一个心正在那里七上八落的,预备着怎样的偎红倚翠,又是如何如何的惜玉怜香,那里就会睡得着?

不要是他们叫你睡的罢!“陈海秋听了,一时听不出秋谷话中的意思,便道:”你这个话儿错了,我不是个孩子,那里能由着他们指拨。“秋谷道:”不是这般说法,只问你未睡之前,吃过他们的什么东西没有?“陈秋海猛然醒悟,拍着手说道:”是了是了,我未上之前,吃了他们一碗杏仁露。我正心上诧怪,怎么无缘无故凭空这般的死睡起来。这样看起来,是他们有心在杏仁露里头放了什么东西,把我吃得这般沉睡,方才圆得过他们的谎来,你说他们可是这个主意不是?“秋谷道:”这个自然,何消说得?但是他们这个主意也只好暂时骗你一下,长久下去是不行的,难道你就不会另外想一个法子,上他的手不成?“陈海秋道:”不瞒你说,法子也不知想尽了多少,到得归根完结还是一个不成功。“章秋谷道:”你这个人真真是个大大的饭桶。你在范彩霞那里的资格也算得狠老的了,就是想他的念头也是分内的事情。你只要装着吃醉了酒的样儿,睡在那里不走,或者趁着狂风大雨的晚上,赶到他那里去借个干铺,难道他好把你推了出来么?“陈海秋道:”岂敢,这些事儿我都一一做过的了,我跑去借干铺,他叫我睡在大床里面,叫个大姐睡在中间,他自己和衣睡在床外,要想动他一动都不能的。我有一天又装着吃得烂醉,睡在那里不肯回去,他却叫个大姐把我扶到大床上去睡了,他自己却坐着不睡,拿出一付牙牌来过五关。娘姨劝他上床来睡,他也不肯,一直等到五更鸡唱,红日东升。

我没奈何只得起来,问他为什么不睡,他只说为着我吃醉了睡在床上,恐怕上床来睡惊醒了我。我听了也无可如何,又扳不着他什么错处,一时发作不来,你想叫我有什么法儿呢?“

秋谷听了低着头沉吟一会,便道:“法子是有一个在这里,但这个时候也不必和你说明,等我会见了修甫他们一班人,再说给你听不迟。但是我昨天到此,并没有出去拜客,你怎么会知道我已经来了,并且还知道我昨天到张园去的呢?你今天可看见修甫没有?”海秋听了便道:“我正忘了,修甫在龙蟾珠家请你吃酒,我正为着这件事儿要和你商量,等会儿在稠人广众的地方讲起来,我面上未免有些不好意思,所以在修甫那里讨了这个差,特地自己赶来请你。现在客人已经齐了,你就赶快同着我一起去罢。”秋谷听了便走上楼去,换了衣服。陈海秋本来是坐了马车来的,秋谷便坐了他的马车同到西安坊来。

原来这一天正是礼拜,修甫在龙蟾珠家摆酒请客,王小屏、刘仰正、陈海秋、陶观察等一班人统通都请在里头。龙蟾珠见修甫来了,便告诉他在张园遇见秋谷的事情。修甫听说秋谷来了,不觉大喜,便要写请客票叫相帮到新马路来请。陈海秋听得章秋谷已经到了,格外起劲,便对修甫说了自己赶到新马路来请章秋谷。当下陈海秋同着秋谷到了龙蟾珠院中,走进房间,见了辛修甫等,大家执手欢然,十分喜慰。秋谷略略招呼了一回,一眼见了陶观察也在这里,想起昨天张园里头的事情,不觉几乎要笑起来,连忙别过头去忍住了笑,和他打了一拱。辛修甫上前介绍说:“这位就是陶伯瑰陶观察,去年在广东来,有东方小松的信给我们两上人介绍,刚刚那个时候你已经回去多时,不在这里。”辛修甫说着,陶观察便在身边取出东方小松的信来递给秋谷,秋谷接过来看了一遍,大家都说了几句客气话儿,方才一同坐下。正是:

瘟生无用,浪挥曲院之金;名士多情,又入笙歌之队。

以后还有许多事实,章秋谷初到天津,范采霞降心相就,味莼园名妓争风等,都在下集书中再行交代,如今却要暂时搁笔了。





第九十七回 莺飞草长望断萧郎 添酒回灯重开夜宴





上回书中正说到章秋谷在西安坊龙蟾珠家与陶伯瑰陶观察相见,陶观察取出东方小松的信来,递在章秋谷手内,章秋谷顺手拆开看了一遍,大家又客气了一回。

辛修甫见客人已经到齐,便和众人代写局票,一个一个的写过来,到了陶观察面前,辛修甫问道:“你是不是还叫薛金莲?”陶观察听了叹一口气道:“薛金莲已经嫁了人,我就叫三马路的胡玉兰罢。”章秋谷听了跳起来问道:“怎么,薛金莲已经嫁了人么?”陶观察听了只点一点头,并不开口,章秋谷诧异道:“我昨天下午还看见你同着他在张园安垲第吃茶的,怎么会嫁起人来,不要你上了人家的当罢?”陶观察听了又叹一口气道:“我亲眼见他嫁人的,怎么会上人家的当!”秋谷听了十分诧异,不懂这个里头究竟是怎样的一件事情,便细细问了陶观察一遍。

陶观察也把薛金莲如何问他借钱、如何前天已除了牌子、如何今天嫁人的事情,一一的都告诉了章秋谷。秋谷听了哈哈的笑道:“如此说来,总算便宜了他。”陶观察听了,不懂秋谷的话是什么意思,只眼睁睁的看着秋谷的脸儿。秋谷正要开口,忽地里陈海秋接过来说道:“算了算了,你要想替人出气,也要看着各人的自家情愿。万一这个人不愿意要你和他出力,你又怎么样呢?”说着不由也哈哈的笑起来。

秋谷听了也笑道:“你又不是人家肚子里头的蛔虫,怎么知道人家不愿意呢?”

正说着,辛修甫走过来对着秋谷说道:“你还是那去年的两个旧相好的么?”

秋谷道:“我到了上海统共只有一天,那里又有什么新相好。”辛修甫点一点头,又问陈海秋道:“你呢,叫什么人?”陈海秋道:“叫西鼎丰林媛媛……”一句话还没有说完,章秋谷早拦住他道:“好好的范彩霞不叫,叫什么林媛媛。”说着又对辛修甫道:“你不要管他,只顾写范彩霞就是了。”陈海秋连忙说道:“你这个人岂有此理,我刚才和你说的话儿,你难道没有听见么?”章秋谷微微的笑了一笑道:“你不要多问,只依着我的话儿去做就是了,到了那个时候,我自然有个法儿。”

陈海秋听了,便逼着章秋谷要问他是个什么法儿。章秋谷一言不发,只看着陈海秋微微冷笑,陈海秋一连问了几声,章秋谷只是不答。陈海秋急了,走过来把秋谷推了一把道:“怎么样,你今天变了哑子么?怎么这般问你,总是一个不开口。”秋谷听了方才对他笑道:“你要我帮你的忙,就是这个样儿,须要听着我的指挥命令,并且不准你无故多言。如若不然,就烦你另请高明,我也没有工夫来管你的这些闲事。”陈海秋听了,没奈何只得谷都着一张嘴走了过去,口中咕噜道:“好好的讲明白了不好,一定要把这样的闷葫芦给人打,不知道是个什么意思。”秋谷见陈海秋一个人在那里自言自语,觉得狠有些儿好笑,便也立起身来,走过去附着陈海秋的耳朵低低的说了几句。陈海秋听了心中大喜,回过身来深深的向秋谷打了一拱,口中说道:“多谢费神。”但是陈海来还没有说出来,秋谷朝着他把手摇了一摇,叫他不要说穿,陈海秋点头会意。正在这个时候,辛修甫来请他们入席,打断了他们的话头,大家依次坐下。龙蟾珠过来斟了一巡酒,唱了一段《文昭关》,便立起身来对着大家说声:“对勿住,请宽用点,倪出堂差去。”便扶着大姐阿小妹的肩头姗姗而去。

这里龙蟾珠刚刚出去,那边范彩霞恰恰进来,莲步未移,香风已到。章秋谷的坐位刚刚对着房门,恰好和范彩霞打了一个照面。只见他穿一件闪光纱湖色夹袄,下面衬一条淡蜜色春纱裤子;髻云高拥,鬟凤低垂,檀口含朱,蛾眉挹翠,身材夭娜,骨格轻盈。走进门来,先抬起那一对秋波四周围飞了一转。刚刚转到章秋谷面前,忽然呆了一呆,不觉出声叫道:“阿唷,二少啘,几时来格呀?”秋谷也笑着朝他点一点头道:“我们一年不见,你竟居然记得我这么的一个人。”范彩霞听了不觉面上一红,别过头去见了陈海秋,待理不理的叫了一声“陈老”,一屁股就坐在陈海秋背后,回转头来再看章秋谷时,只见章秋谷的一双眼睛正在上上下下的打量着他。范彩霞见了“嗤”的一笑,不因不由的飞了章秋谷一个眼风,两个人便密密切切的谈起来。

正谈得高兴,早听得门外弓鞋细碎的声音,门帘一起,走进两个丽人,手搀着手的并肩进来。秋谷连忙举目看时,原来就是自己叫的两个倌人,一个久安里的陆丽娟,一个迎春坊的梁绿珠,婷婷袅袅的走到面前。只见陆丽娟身上着一件玄色外国纱夹袄,衬一条淡淡的浅蓝闪光纱裤;蛾眉欲蹙,皓齿微呈;丰彩惊鸿,佩环回雪。再看那梁绿珠时,只见他着一件本色春纱夹袄,衬着一条湖色裤子;秋水横波,春山敛黛;风鬟雾鬓,皓腕纤腰。两个人手搀手儿立在一处,恰好两个人的长短都差不多。当下梁绿珠和陆丽娟两个人走进门来,一眼早看见了章秋谷,两个人齐叫一声“二少”,便轻移莲步的走过来,坐在章秋谷身后。梁绿珠先开口道:“二少,耐倒好格,啥勒倪搭一径勿来介?”秋谷笑道:“我刚刚昨天到的上海,忙了一天,那里有工夫到你们那里去!”梁绿珠听了把嘴一披道:“耐呒拨工夫到倪搭去,吃花酒倒有工夫格?”秋谷道:“这是应酬朋友,算不得吃花酒。”梁绿珠听了,飞了秋谷一个白眼道:“应酬朋友未有工夫格,到倪搭去末呒拨工夫,阿对?”秋谷听了,一时竟回答不出什么来,只得哈哈一笑道:“算了算了,不用挑眼了,就算是我的不是何如?”陆丽娟听了,对着秋谷微微一笑,梁绿珠还在那里自己低低的说道:“生来是耐勿是啘。”陆丽娟趁着这个当儿,握着秋谷的手低低的问道:“耐阿是昨日来格,倪搭仔耐长远勿看见哉,耐身体浪向阿仔?啥勒一径勿到上海来价,倪末一径心浪向牵记煞。”秋谷听了,对着陆丽娟笑道:“多谢多谢,承情得狠。”说着,把手紧紧的握住了陆丽娟的纤手,四目相视,脉脉含情。秋谷正在出神,恰被梁绿珠扭过身来,附在秋谷耳朵上悄悄的说道:“恩得来,阿要窝心。”

秋谷出其不意,倒被他吓了一跳,便也回过头来,一把握着梁绿珠的手,左顾右盼,心花大开。

正在这个时候,忽然觉得肩头上有人一拍,抬起头来看时,只见范彩霞睁着一对水汪汪的眼睛,对着自己的脸儿似笑非笑的说道:“二少,倪去哉,晏点有功夫末,请到倪搭去坐歇,不过倪搭小地方,怠慢煞格,勿得知耐二少阿肯赏光勿肯赏光?”说着,又对着秋谷飞了一个眼风。秋谷听了,便也打着苏白回答他道:“阿唷,先生勿要客气,啥人勿晓得范彩霞先生是上海滩浪天字第一号格红倌人。”范彩霞不等他说完,把眼一瞟道:“好哉好哉,勿要钝哉!”一面说着,一面往外便走,走到房门回过头来,对着章秋谷嫣然一笑,急急的走了出去。章秋谷见了不由得叫一声“好”。梁绿珠接着说道:“勿要瞎拍马屁哉,阿是刚刚格马屁还朆拍足?”

秋谷听了也觉得好笑,正要开口,恰恰的陶观察要和他搳拳,便把这句话儿岔了过去。秋谷和陶观察搳了五拳,秋谷输了三拳,秋谷自己吃了两杯,梁绿珠吃了一杯。

陶观察打了一转通关,便立起身来对辛修甫说,别处还有应酬,匆匆的要走。辛修甫见他要去别处应酬,不便留他,由着他一个人去了。

秋谷等梁绿珠和陆丽娟走了之后,便也起身要走。辛修甫道:“你今天还有什么事情没有?”秋谷道:“事情是没有什么,但是等会儿要去看两个人。”辛修甫笑道:“你无非要到陆丽娟和梁绿珠处打两个茶围,等一回散席之后,我们一同去就是了,这个时候你就是去也是碰不着的。”秋谷听了觉得不差,便也依着辛修甫的话儿坐了一回。

大家散席之后,同着辛修甫、陈海秋、王小屏等一班人到陆丽娟院中坐了一回。

丽娟有心要拉拢章秋谷,竭力应酬,尽心巴结,奉承得章秋谷十分欢喜,在他那里坐了一点多钟的工夫,又同着众人到范彩霞那里去坐了一回。

范彩霞对着陈海秋还是那般冷冷落落的样儿,却打起精神来应酬秋谷。秋谷被他殷勤不过,也只得略略的领略些儿。陈海秋在旁边看了十分难过,口中又说不出什么来,只得催着秋谷叫他快走。秋谷也无可不可的出了院门,便别了众人自家回去。

到了明天,秋谷还没有起来,陈海秋已经来了,坐在楼下书房里头等了一回,章秋谷方才下来。陈海秋一见了章秋谷的面,便嚷道:“你这个人真真的岂有此理!

我托你的事儿你不肯和我想个法儿也还罢了,你自己倒和他吊起膀子来,天下那有这般道理?“秋谷听了笑道:”你不要这般性急。我既然答应了和你设法,自然总有一个好好的安排。至于吊膀子的一层,并不是我去吊他,却是他来吊我的,这样的就口馒头,我也落得寻寻他的开心,难道我当真要去吊他的膀子么?你若怕我剪了你的边,在旁边吃起醋来,这件事情就办不来的了。“陈海秋听了也笑道:”我也不过是这样说说罢了,我和他又没有什么交情,那里会吃什么醋?不过你既然答应了同我设法,何不把这个法儿和我讲个明白,也好等我心上喜欢一下,何必一定要叫我打这样的闷葫芦呢?“秋谷听了低着头想了一想,方才对陈海秋说道:”这件事情有个绝好的法儿在这里,管教大大的糟蹋他一下,出出你的闷气,但不知道你自己的意思怎么样?“当下章秋谷对着陈海秋说出一番话来。有分教:

望断蓝桥之路,无那萧郎;强寻巫峡之云,难为神女。

不知后事如何,且听下回分解。





第九十八回 范彩霞安心慢客 东尚仁叫局碰和





且说章秋谷对着陈海秋说道:“这件事儿,虽然我和你做个军师,究竟要你自家定个目的,你的意思到底怎么样呢?”陈海秋道:“我也没有什么一定的目的,只要你和我出了这口闷气也就是了。”章秋谷道:“就是你要翻他的本,出口气儿,也有几等几样的法儿,你老实说,你究竟心上怎么样?”陈海秋道:“我一时也想不出什么主意,你的意思又怎么样呢?”秋谷道:“依着我的心上想起来,你不过因为范彩霞看你不起,有心骗了你的钱,又不肯留你住夜,只要好好的想个主意,把他大大的糟蹋一下,出出你的气儿,你说可好不好?”陈海秋听了沉吟一回,把头摇了一摇道:“这个主意虽然不错,未免便宜了他,据我的意思想起来,他既然不肯留我住夜,我如今偏要……”陈海秋说到这里,觉得有些说不下去,便顿了一顿,说不出来。

章秋谷听了心上早已明白,故意问道:“偏要什么?说下去。”陈海秋面上一红,觉得有些不好意思,道:“算了罢,你不用假装糊涂了。”秋谷听了哈哈笑道:“照你这样说起来,无非还是想要他留你住夜。上海的倌人也狠多,就是面貌比他好的也还不至于找不出来,何必一定要看中这个范彩霞呢?”陈海秋听了面上红了一红,一时间回答不出来。停了一停方才慢慢的答道:“我也并不是一定要和他怎样,不过我在他面上花了无数的钱,他竟把我当作个天字第一号的瘟生,好像是理应孝敬他的一般,你想可恨不可恨呢?如今我的意思,要你和我想个法儿,叫他自家俯就。一则出了我的一腔恶气,二则也好借此坍坍他的台,只不知可做得到做不到?”秋谷听了道:“有什么做不到?你只要依着我的话儿行事,我叫你怎么样你便怎么样,到了那个水到渠成的时候,自然有一个叫他不得不如此的法儿,你只好好的等着就是了。”

陈海秋听了心上甚是喜欢,却故意做着不相信的样儿道:“你不要这样的拿得千稳万稳的。范彩霞这个混帐东西比不得别人,我不信你就有这般手段。”秋谷听了冷笑道:“你不信就罢,请你自家去另想法儿,与我不相干。”陈海秋一听秋谷推托,心上又着急起来,再三的央求秋谷和他想法。秋谷到了这个时候,方才把自己的主意细细的和他说了一遍,喜得个陈海秋直跳起来道:“这个主意,拿得定他一定上钩的么?”秋谷道:“这个自然。若是换了别人,我不敢说他一定怎样;至于范彩霞这个东西,我久已知道他的历史,还你百发百中,手到拿来。”陈海秋听了十分欢喜,又坐了一回,说了些天南地北的闲话,方才告辞去了。

章秋谷从这一天起,接连拜了几天客,应酬了几天。这一天下午,刚刚在金谷春大菜馆里头走出来,劈面又撞着了陈海秋,便拉着秋谷一同到东尚仁去。秋谷一路走着,同陈海秋讲道:“你拉我到东尚仁去,你不怕我要剪你的边和范彩霞吊膀子么?”陈海秋也笑道:“凭你去怎样吊法,我总不吃你们的醋就是了。”两个人说说笑笑,一路到东尚仁来。到了范彩霞院中,两人走进房内,范彩霞刚刚起来,正在那里梳洗,见了陈海秋进去,只微微的朝他点一点头,忽然抬起头来见了章秋谷在陈海秋的后面,登时满面添花,立起身来口中说道:“阿唷,二少,今朝陆里一阵好风,吹仔耐来哉,几日天勿见哉,唔笃格位姨太太阿好?”章秋谷含笑点头道:“多谢多谢,托福托福。”一面说着,一面走到范彩霞后面,把一只手轻轻的在他肩上搭道:“请坐请坐,你只管办你的公事,不要客气。”范彩霞回头一笑,两颊生红,对着秋谷笑道:“倪无啥事体呀,耐二少是难得请过来格客人,今朝赏倪格光,到倪间搭小地方来坐歇,总要客气客气格啘,二少爷阿对?”范彩霞一面说着,一面自己坐了下来,指着靠窗的一张椅子对章秋谷道:“二少坐嗫。”

章秋谷听了,也随随便便的坐下,却细细的抬起眼睛来打量范彩霞时,只见他身上穿着一件半旧的熟罗短袄,春生宝靥,红上眉梢,一缕漆黑的头发,一个娘姨替他解开了直拖下来,差不多直垂到地,透出一股冰桂兰麝的味儿。胸前两颗钮扣儿没有扣好,微微的露出里面杨妃色的抹胸,扣着一条黄澄澄的金练,衬着那纤腰婀娜,云鬓惺忪,觉得无限娇娆,十分妖艳。章秋谷看了这般的一付样儿,也不知不觉的心上怦怦欲动。范彩霞一面梳头,一面偷眼见了章秋谷这般模样,越发的眉梢眼角卖弄精神。秋谷到了这个时候,免不得也要略略应酬,只把一个陈海秋丢在那里,既没有人和他讲话,也没有人去理他,好似老僧入定一般,坐在那里无声无息。

章秋谷始终意不在此,便立起身来对陈海秋道:“我们没有什么事情,还是约几个人来碰和罢。”陈海秋听了道:“也好,我们就去约了陶伯瑰和辛修甫来碰一场和,但不知他们来不来?”范彩霞听了接口道:“耐写仔请客票,叫相帮去请请看末哉,今朝辰光勿晏,陶大人搭仔辛老勿见得出去格。”说着又飞了秋谷一眼,好像打个照会一般。陈海秋写了两张请客条子,叫相帮去请辛修甫和陶伯瑰。相帮去不多时,早听得楼下相帮高叫客人上来,陈海秋和章秋谷方才立起身来,辛修甫已经匆匆走进。秋谷笑道:“请客的还没有回来,客人倒已经来了。”辛修甫见了陈海秋和章秋谷,也略略的讲了几句套话。

这个时候,范彩霞的头已经梳好,便立起身来应酬了修甫几句。等了一回,陶观察也来了。范彩霞便叫娘姨大姐调开桌椅,取出一付乌木牌并一付筹码来,问陈海秋筹码怎生配法?陈海秋还没有开口,陶观察抢着说道:“自然打现的,那个来打什么筹码。”秋谷微笑不言,范彩霞听了,便把筹码拿了过去,把那一付牌倒在桌子上,拣出东南西北四张放在中间。秋谷顺手拿过一张牌来看时,原来是象牙牌面,雕得甚是精致,不觉顺口赞道:“好讲究的牌,果然这个地方和别处不同。”

范彩霞听了,只道有意赞他,便抬起头来又对着秋谷一笑。秋谷却没有留心,见范彩霞对他一笑,心上方才明白,心上倒觉得有些儿不得劲儿,便搭讪着问辛修甫叫局不叫。辛修甫道:“我们四个人碰和,我看不必叫局罢。”秋谷道:“叫几个人来,觉得热闹些。”辛修甫听了便也答应。秋谷便代他们写起局票来,辛修甫叫龙蟾珠,陶伯瑰叫胡玉兰,陈海秋也叫了一个西鼎丰的林媛媛。章秋谷不消说,自然就是梁绿珠和陆丽娟了。

当下大家讲明打五十块钱一底的二四,大家扳了坐位便碰起来。碰了几副,叫的局已经来了,梁绿珠和陆丽娟坐在秋谷身后,默默的看他发牌,起先的几付牌,平平的都没有什么输赢。陈海秋碰了两圈,便叫林媛媛和他代碰,刚刚遇着他的庄,一起手便是中风开了个暗杠。陶观察又打了一张东风,林媛媛又碰了出来,转了几转,秋谷见林媛媛的牌只打了一张万子,便和陶观察同修甫道:“庄家是万子一色,你们留神一点。”一句还没有说完,陶观察忽然打了一张发风出来,林嫒媛见了把牌摊出,计算起来四百和牌,给他和了一个倒勒。辛修甫等大家算清了帐,便问陶观察为什么无缘无故的打出一张发风,陶观察道:“我自己要和,怎么不要打这张发风呢?”秋谷听了,心上觉得狠有些好笑,狠想问他,你自己想和,如今可想到了没有?却又为着和他认识没有许多时候,恐怕他动气,便也微微一笑,并不言语。

那知自此以后,林媛媛的牌风大旺起来,一连庄上和了几付,接着辛修甫和了一付两翻的索子一色,不到四圈牌,章秋谷已经输了一百四五十块钱。陆丽娟见了,便要和秋谷代碰,秋谷便立起身来让他去碰。陆丽娟碰了两圈,输得比秋谷更多,秋谷诧异道:“我平日碰和,从来没有输得这般利害,今天什么缘故,忽然这个样儿?”便叫陆丽娟立起来,原是自己坐下去碰。范彩霞见秋谷一霎时的功夫,已经差不多输了三百块钱,便走过来站在秋谷身后,指手画脚的指点他。只见秋谷起出牌来,都是七不搭八的,没有一张好牌。范彩霞见了皱着眉头,把头连连的摇了几摇,忽然上家陶观察发出一张二索不。范彩霞说一声“吃”,秋谷只当没有听见的一般,范彩霞不懂秋谷的意思。转了一转,陶观察又打出一张九万,范彩霞道:“碰。”秋谷还是只当没有听见,径去摸牌。范彩霞在旁边看了,忍不住问道:“二少耐碰错哉,碰和勿是实梗碰法,蛮好格九万,啥格道理勿碰呀。刚刚只要听仔倪格闲话,吃仔二索,碰仔九万,故歇和也和脱格哉。”秋谷道:“我有我的道理在里头,用不着你和我着急。”范彩霞听了,那里肯信,口中只在那里咕哝道:“阿有啥碰和勿碰九万格道理,唔笃大家听听看。”秋谷听了道:“等一回儿碰完了,再和你细细的讲这个里头的道理,这个时候没有工夫。”说着,又历乱掳牌,范彩霞仍旧立在秋谷后面看他。对面辛修甫打了一张七万,秋谷说一声“碰”。便打出一张八万。范彩霞见了,嚷道:“格只七万随便那哼,呒拨碰格道理,豪燥点勿要碰。”秋谷微笑道:“这个道理你不懂的,不要来和我混闹。”范彩霞听了愈加不服,把身躯一扭,走到烟榻上一屁股坐下,对着梁绿珠、陆丽娟两个说道:“倪看今朝格二少有点输昏仔头哉。”正是:

樗蒱陆博,偏多制胜之方;蹴鞠弹棋,亦有神明之化。

不知后事如何,且看下文便知分晓。





第九十九回 叉麻雀名士讲牌经 卖风情倌人吊膀子





且说范彩霞见章秋谷碰和这般碰法,心上大大的不以为然,口中咕噜着说道:“倪从来朆看见碰和实梗样式。”秋谷听得范彩霞这样的替他着急,心上也觉得有些好笑,便对他说道:“我的碰和和别人不同,另外有我的法儿,你不信你只走过来好好的看一下子,就知道里头的道理了。”范彩霞听了便又走过来,站在秋谷后面细细的看着。

这番秋谷的庄,恰和了一付,又接着连了一付七十二和的筒子一色。接着,辛修甫和了一付,轮着林媛媛的庄。范彩霞在秋谷背后看着他起出牌来,也是平平常常的,不见得怎样好法。碰了两转,上家陶观察发出一张五索,秋谷不吃,顺手去摸一张东风来,打出一张四索。范彩霞看了也不开口,只把秋谷的衣服一拉,秋谷微笑摇头,一转过来,秋谷去起出一张三万,成了三四五万的一搭,便又打出一张六索。辛修甫见了诧异道:“你与其拆掉四索六索,为什么不吃他的五索呢?”秋谷笑道:“照这样的一付牌,就是和了也不过一个平和,有什么希罕。”等了一回,辛修甫发出一张南风,秋谷碰了出来,发出一张九索。这个时候,林媛媛早已碰了三张白板放在桌上,一转过来轮到陶观察发牌,陶观察却顺手发出一张东风来。林媛媛见了大喜,扑的把牌摊出,口中说道:“难末咦敲着仔唔笃一记哉。”大家举目看时,原来是东风和一索对碰和出,是一付索子一色,里头还有三张八索,三张七索,又是个对对和。林媛媛屈指一算道:“对对和要外加一翻,刚刚咦是一付倒勒。”林媛媛正在高兴,不提防章秋谷伸过手去,把那一张东风抢了过来。林媛媛嚷道:“作啥呀,拿倪一张东风抢得去。”

秋谷不慌不忙,把自己的牌摊在桌子上,口中说道:“请你们看看,我的牌怎么样?”辛修甫和陶观察大举眼看时,只见齐齐正正的三张八筒,三张一万,三张三四五万,一张东风,还有三张南风已经碰在桌上。修甫见了,诧异道:“你是独等东风么?”秋谷不答,只点一点头,把陶观察方才打的那张东风和自己的东风放在一起,只把一个背后的范彩霞喜欢得笑得“吱吱格格”的,一张樱桃小口再也合不拢来。辛修甫和陶观察见章秋谷拦了林媛媛的和,心上自然高兴。只有林媛媛谷都着一张嘴,十分扫兴,瞪了秋谷一眼道:“倪勿来,勿作兴实梗格。耐要拦倪格和,为啥勿早点说呀。”秋谷笑道:“你的手脚十分神速,对面的一张东风,刚刚打出,你已经飞一般的抢了过去,叫我那里来得及?”林媛媛听了也觉好笑,便把自己的牌一推,历历碌碌的掳起牌来。

秋谷方才对着范彩霞讲道:“何如?这一下子你有些明白了么?你刚才看着我不吃二索不碰九万以为错了,你不知碰和这样东西,虽然是一件游戏的事情,里头也有些儿反败为胜的道理。大约上家的牌风狠旺,便不当吃的吃他一下,把上家的牌落到自己手里头来,或者下家的牌风狠旺,便当吃的不吃,把下家的牌提到自己手里头来。我刚才看见下家的牌风好得狠,所以故意不碰不吃,有心揽他一下,果然给我一下子揽过来。你想方才要是吃了上家的一张五索,自己三六万等张,这一张东风岂不是给下家拿了去么?下家要是拿着了东风,早已和出来的了,那里还等得到这个时候。”章秋谷一面说着,林媛媛和辛修甫、陶观察都停了手呆呆的听。

范彩霞听了秋谷的一番说话,不觉连连点头,想了一想便又问道:“既然耐说勿碰勿吃,为啥好好里有仔八万九万,要碰对家格七万呀?”秋谷道:“今天的牌只有他们两家的好些,所以对面打了一张七万,我拆掉了自家的八万九万,去碰他那张七万,本来是不应该碰的,如今我碰他一下,或者可以把对面的好牌碰过这一面来,这也是一个反败为胜的法子。”

辛修甫和陶观察听了秋谷这一番说话,觉得甚是津津有味。辛修甫便问秋谷道:“据你说来,碰和里头也有这许多奥妙,但是除了这几个法儿,还有什么别的方法没有?”秋谷道:“碰和的方法,第一不要让下家多要自己的牌,看着给他吃一下子没有什么要紧,就是和了出来,无非是十和二十和的牌,也算不得什么。人家往往在这个里头不狠留心,随随便便的混打,却不知道虽然人家和一付小小的牌不算什么,你要是一连给他和了几付,牌风一顺,他的牌就忽然间大好起来,真是拉朽摧枯,势如破竹,到了那个时候,你就是再要扣他的牌,凭你怎样也扣不住的了。

那班碰和的饭桶,自己输了钱还要抱怨自己的牌风不好,那里想得到别人的牌风为什么这般好法,就是自己不肯留神闹出来的。大凡碰和的人,虽然要顾自己手里头的牌,却也要顾着台面上的牌风怎样,到了那差不多大家等张的时候,只要留神看着台面上的牌,已经打出去的是几张什么,合着自己手里的牌算计起来,别人等张是等的什么牌大约总有几分拿手。总之,不论自己的牌风好与不好,只要少发生张,不开大炮,一定不至于出什么乱子的。至于讲起自己的发牌来,那是碰和里头最要紧的一件事情,在自己牌风不好的时候,自然不好混打;就使自己的牌风狠好,也要自己留神些儿,不好乱发。一个不小心给人家和了去,凭你自己的牌再大些儿,也不值一个大钱,倒反把牌风弄得大坏起来。如今那些碰和的人都是这个样儿,倚仗着自己的牌风狠好,便不管三七二一随手乱打,打到后来总是输得他一个要死,这几句话儿虽然没有许多窍妙,碰和里头的方法也就差不多了。“

辛修甫、陈海秋和范彩霞等听了,都是心领神会,只有陶观察听了有些不以为然,便道:“据我看起来,碰和一道原不过是我们借他消遣的事儿,何必要这样的在里头讲究?况且我们一班人大家聚在一起顽顽,输赢都不算什么,用不着这样认真,你们看我的话可是不是?”秋谷接着说道:“这个话儿自然不差,但是这个‘赌’字的字义,本来就是彼此争胜的意思。无论什么人,你不沾到这个‘赌’字便罢,要是沾到了这个‘赌’字,凭你亲戚、朋友、父子、兄弟都没有一些儿退让的心肠,一定要自己胜了,人家输了,心上方才快活。至于我们的打牌本来算不得赌钱,不过是个消遣的法儿罢了。但是虽然消遣,大家心上未免总有些争胜的意思,断没有一个人上了赌场,只想输不想赢的道理。不过我们的赌钱与别人不同,没有那些死想赢钱的期望,赢了固然狠好,就是输了也没有什么希奇。至于说起我们大家赌起钱来,一定的希望着自己输钱,那也不过是这么一句话儿讲讲罢了。”陶观察听了,和辛牙甫都点头称是。

陈海秋一个人在炕上躺了一回,觉得有些困倦,便立起身来叫林媛媛让他坐下,几个人又碰起来。等到完了八圈,差不多时候已经六七点钟,叫来的倌人一个个都走了。大家算起帐来,陶观察一个人大输,输了一百三十多块钱。辛修甫也输五十块钱,陈海秋只赢了二十块钱,章秋谷非但把方才输的都捞了回来,还透赢了一百六十几块钱。秋谷对着范彩霞道:“何如?你说我打错了牌,如今你相信不相信?”

范彩霞听了嫣然一笑,也不开口,只对着秋谷微微的朱唇一动。

秋谷一笑,别过头去对陈海秋说道:“这个时候,差不多就要上灯,我看你就在这里吃一台酒罢。”陈海秋听了点头答应,便和范彩霞说了,叫他预备一台菜。

范彩霞听了自然欢喜,连忙叫娘姨下去招呼。不多时,早已摆得齐齐整整,陈海秋又请了两个招商局里头的朋友,大家闹了一回,这一台酒差不多直吃到十点钟的光景,方才大家回去。范彩霞趁着陈海秋送客的时候,一把拉住了秋谷的手,低低的问道:“耐明朝几点钟来?倪有两句闲话要搭耐说。”秋谷微微笑着,答应他道:“明天我一定同了陈老爷过来就是了。”范彩霞听了把头一扭,把一个指头轻轻的在秋谷头上点了一点道:“耐格人啥实梗介……”正还要说下去,刚刚陈海秋送了客进来,酒气冲冲的口中说道:“彩霞到那里去了,为什么不来送送客人?”范彩霞把双眉一皱,连忙扭过身来答道:“倪勒浪啘,刚刚章二少搭倪说两声闲话,夹忙头里向客人去哉。”秋谷趁着这个时候对陈海秋说道:“我们回去罢,明天就是我们原班四个人,在这里再碰一场和可好不好?”陶观察和辛修甫自然答应,秋谷便别了众人,自己回新马路去了。

自从这一天章秋谷在范彩霞那里碰过了一场和之后,陈海秋天天约着他们三个在范彩霞院中碰和,又天天请客,在范彩霞院中吃酒。秋谷也有时约着他们几个到梁绿珠、陆丽娟家去碰和吃酒。陆丽娟自从认得了这位章秋谷以来,觉得章秋谷华彩非常,丰仪出众,好像自己相与的客人里头没有一个赶得上章秋谷的,便十分巴结起来。章秋谷也爱着陆丽娟的性格温柔,风情旖旎,几天工夫便有了相好。一个是江南名士,倜傥非常;一个是越国佳人,深情如许。自然的十分恩爱,格外缠绵。

在下做书的也不必去提他。

不多时,早到了五月初三,转瞬之间已经是端午佳节,榴花照眼,暑气迎人。

那班堂子里头的娘姨、大姐,一个个都在四马路上穿梭一般的来往不绝;更有那起抬轿子的乌龟,挑着送礼的东西,满街上乱走。有些漂帐的客人,到了这个时候都躲得个无影无踪,累得那班娘姨、大姐寻得一个发昏。章秋谷恰早早的把那些堂子里头的酒局帐和那些店帐,都开销得清清楚楚。到了初三那一天,为着陆丽娟叫他去吃司菜,便约了辛修甫和陈海秋两个人同去。到了陆丽娟那里,只见陆丽娟梳好了头,正和个大姐在那里说笑,见秋谷进来,便笑微微的叫一声“二少”。正是:

倾城名士,重翻子夜之歌;暮雨朝云,又入高唐之梦。

欲知此后如何,且看下回交代。





第一百回 打茶围乌龟送礼 出奇谋嫖客施威





且说陆丽娟见章秋谷同了辛修甫、陈海秋三个人一起走进来,便立起身来含笑招呼。秋谷同着辛修甫、陈海秋进房坐下,房间里头的人见章秋谷狠肯花钱,便十分巴结。一个娘姨叫做金宝的,便叫相帮拿进四样节礼,放在章秋谷面前,笑道:“送到二少公馆里向去,长恐唔笃姨太太心浪勿舒齐;就来浪间搭送仔罢。二少勿要客气,一塌刮仔受仔末哉。”秋谷看那四色礼时,见无非是些火腿、粽子、鲜藕、枇杷之类,便也对着金宝笑道:“别人的姨太太要吃醋,我的姨太太是从不吃醋的,你只顾放心送去就是了。”

秋谷的话还没有说完,早见陆丽娟瞅了秋谷一眼道:“唔笃勿要听俚格瞎三话四,俚笃姨太太凶得野笃。”秋谷听了诧异道:“我章秋谷自从有生以来,从来没有怕过妻妾,你这句话儿是那里来的?倒要讲给我听听。”丽娟“嗤”的一笑道:“勿工勒浪海外哉,故歇末说得像煞有介事,晏歇点距起踏板来吃勿消格,阿晓得?”

秋谷听了,实在不懂丽娟是什么意思,只呆呆的看着他。丽娟看着秋谷的脸,忍不住又笑道:“昨日仔阿记得,极得来呒淘成?”秋谷听了这两句话儿,心上方才恍然大悟,哈哈一笑道:“原来你为昨天晚上的那件事儿,所以好端端的平空说出这许多的怪话来,你却不知道昨天所以一定要回去的缘故,是我在家里头出去的时候和他们讲明白了一定回去的,恐怕他们在那里呆等,所以定要回家,并不是不肯陪你。”陆丽娟听到这里,不由得面上一红,啐了秋谷一口道:“啥人要耐陪呀,说说末就是实梗瞎三话四。耐怕勿怕距踏板勿距踏板,才勿关倪啥事!”说到这里,秋谷大笑道:“我倒从没有跪过什么踏板,或者看你面上给你跪一下子,也不可知。”

陆丽娟道:“倪是呒拨格号福气,唔笃听听看,说得阿要诧异!”说着,也忍不住笑起来。

秋谷一面笑着,一面在衣袋里头取出一卷钞票来,随手拣了三张十元的,放在烟盘里头道:“送礼手巾和司菜的钱都在里头。”金宝接了过去,谢了一声,又向秋谷道:“格末格个节盘,阿要送到二少公馆里去呀。”秋谷连连摇头道:“算了算了,我不过这样的说,那个要你们送去。”说着,相帮送上手巾,口中说了一声“多谢二少”。秋谷只略略点头。一会儿金宝走了出去,陆丽娟走到秋谷身旁悄悄的说道:“刚刚耐啥事体要拨俚笃几化洋钿呀?”秋谷道:“连司菜的钱在内一共三十块钱,也不算什么。”陆丽娟嗔道:“一塌刮仔出仔廿块洋钿好哉,耐就是多拨点俚笃,俚笃也勿见得见耐格情,推扳点再要说耐曾生,格号铜钱冤冤枉枉出俚做啥?老实搭耐说,该应用格辰光自然搳脱两钿,无啥要紧,勿该应用格辰光,耐也勿必摆啥格架子,难下转勿要实梗,阿晓得?”秋谷听了陆丽娟的这一番说话来得十分诚切,知道他倒是一片真心,心上狠觉得有些感动,便也悄悄的附着他的耳朵道:“你的话自然不错,但是我在你身上不要说是这几个钱,就是再多些儿我心上也高兴的。”陆丽娟听了心上自然十分欢喜,却故意说道:“倪勿要,耐下转阿要实梗勒。”

秋谷还没有开口,早听得陈海秋嚷道:“你们这两个人真真岂有此理。到了这个地方,便两个人密密切切的讲话,把我们两个客人干搁起来,理也没有人理;就是有什么说不尽的话儿,等会儿到了床上,凭着你们去怎样讲法就是了。为什么偏偏要在这个时候,当作我们的面做出这种样儿,难道故意做给我们看的么?”陆丽娟听了陈海秋取笑他的话儿,不觉涨得满面通红。秋谷回过头来对陈海秋道:“海秋不要胡说,人家在这里好好的讲话,你又要取笑起来。”说着,见陆丽娟低着个头口中咕噜道:“随便唔笃去说啥末哉。”秋谷便握着他的手道:“我们老夫老妻那里还怕人取笑,凭他去讲些什么,我们不要管他就是了。”陆丽娟听了更觉不好意思,挣脱了手,把秋谷背上打了一下道:“耐格个人,实头呒拨仔淘成哉,说出格号闲来,阿要气数!”说着自己也不由得“嗤”的一声笑起来。秋谷正要和他讲话,只见大姐阿金妹走进房来,向陆丽娟使个眼色,丽娟见了,就走过去低低的分付了他几句,阿金妹走了出去。

一会儿相帮早端上菜来,本来堂子里头的司菜,照例是一碗鱼翅,一碗整鸭,一碗鸡,一碗蹄子。秋谷一眼看去,见那四样例菜之外。又另外加了一大盆鲥鱼,一贫白汁排翅,一碗清燉火腿,一碗鲍鱼汤。还有四个碟子:一样凉拌腰片,一样凉拌鸡丝,一样凉拌猪肝,一样虾米煮黄瓜。这几样菜都是章秋谷平日最爱吃的。

另外两把酒壶,装着满满的两壶花雕,还有一瓶薄荷酒,一齐都放在桌子上。秋谷见了,把头一皱道:“今天你怎么忽然和我客气起来,平空的添这许多的菜做什么。”

陆丽娟笑道:“倪为仔格两样菜无啥吃头,所以另外点仔几样,总算是倪一点点意思,耐勿要客气哩。”说着,便取过一个玻璃小酒杯,倒了一杯薄荷酒放在秋谷面前,又问辛修甫、陈海秋道:“辛老、陈老,唔笃两位吃啥格酒?”陈海秋本来酒量狠大,要了薄荷酒,辛修甫不会吃酒,便要了花雕。陆丽娟斟了辛修甫、陈海秋两个人的酒,口中说道:“怠慢唔笃,请宽用一杯。”章秋谷便叫他过来陪着同吃,陆丽娟便也坐在秋谷下首,自己斟了一杯酒,四个人浅斟低酌起来。这一席虽然没有什么别的客人,却大家都十分高兴,说说笑笑,不觉已是下午三点多钟。秋谷便对着陈海秋说道:“我们回去罢,那个家伙只怕差不多要去的时候了。”陈海秋听了会意,便同着章秋谷、辛修甫出了陆丽娟的院中,一路回去。

这个时候,陈海秋正在后马路一家谦泰土栈里头,这个土栈,就是陈海秋一个人开的。当下陈海秋邀了辛修甫、章秋谷一同到得谦泰土栈,坐在陈海秋的卧室里头,陈海秋叫家人泡上茶来。坐不多时,果然见范彩霞那里的大姐阿小妹同着两个相帮,拖拖带带的送进四样节礼来。见了陈海秋,春风满面的叫了一声“陈老”,陈海秋只点一点头,阿小妹道:“陈老,今朝啥勿到倪搭去呀,倪先生勒浪牵记耐呀。”陈海秋听了冷笑一声,道:“用不着这般的客气,只要我到你们先生那里去的时候不要做出那付阴阳怪气的样儿,已经是好的了,什么牵记不牵记,像我这样的惹厌客人,那里配你们先生牵记。”阿小妹听了呆了一呆,笑道:“陈老末咦要实梗瞎三话四哉,倪先生搭耐蛮要好,啥辰光搭耐阴阳怪气呀!像陈老格号好客人,再要说惹厌,是真真天理良心呒拨仔淘成格哉。”说着回过头来对着秋谷和修甫道:“二少搭仔辛老想想看,倪格两声闲话阿对?”辛修甫和章秋谷听了,只好点一点头。海秋又道:“算了算了,不用多讲了。你今天无非是送礼和讨帐的两件事情。”

说着,便开了保险箱,取出一大卷钞票放在桌子上,随手取出两张十块钱的钞票,交在阿小妹手里头,口中说道:“这几件礼物,我也用他不着,就烦你们和我带了回去。这二十块钱,连节盘和手巾的钱都在里头,今天交给你,省得我又要叫人送来。”阿小妹接了钞票口中说道:“陈老啥实梗客气,一样物事才勿受呀。”陈海秋对着他连连的摇头,只说:“你不要和我客气,我这里委实用他不着。”阿小妹道:“格末谢谢耐。”相帮也跟着谢了一声。

陈海秋又问阿小妹道:“我的酒局帐抄好没有?”阿小妹听了,便从身旁衣袋里头取出一篇开现成的酒局帐来,还有一张范彩霞的大字名片,一齐交给陈海秋,口中还在那里说道:“陈老慢慢交末哉呀,啥洛实梗要紧介。”陈海秋接过来一看,见通共二十六台菜钱,十九场和钱,一百二十多个局钱,还有那一天陈海秋在他们那里碰和,没有带钱,就同范彩霞借了一百块钱做本钱,后来没有还他,一古脑儿合算起来,差不多要六百多块钱。陈海秋看了一看,把那一篇帐单放在桌子上,正色对阿小妹道:“你今是想来要钱的是不是?”阿小妹道:“陈老末总归实梗瞎疑心,洋钿勒浪陈老格搭,阿怕会少……”阿小妹正还待说下去,陈海秋接着说道:“如今不必空费这些口舌,总之一句话儿,我今天欠你们先生的局帐,一个大钱也不能给他。”阿小妹听了呆了一呆,还只认是陈海秋和他取笑,却见陈海秋正颜厉色的对他讲道:“我姓陈的并不是没有钱,钱狠多在这里,但是凭着你们先生这样的一个人,要想用我姓陈的钱,只怕还嫌太早些儿。”说着便把桌子上的那一大卷钞票,一张一张的摊了开来,给阿小妹看,一古脑儿统统是五十块的,只有几张十块的在里头,合计起来,这一大卷钞票至少也有二三千块钱在里头。把一个阿小妹只看得目定口呆,眼花撩乱,觉得自己的一双眼睛花碌碌的,只顾随着桌子上的一卷钞票前后左右四周乱转,直等得陈海秋把那些钞票仍旧放在保险箱里头去,方才把心定了一定,一时说不出一句话来。只听得陈海秋又对他讲道:“你回去只把我这几句话儿,讲给你们先生听就是了。”阿小妹呆了一回,心上不知道陈海秋究竟为着什么,转了一回念头,只得开口说道:“阿呀,陈老为仔啥格事体实梗动气呀,阿是倪先生得罪仔耐哉,阿好讲拨倪听听看,到底那哼格一件事体?”陈海秋听了,便睁着眼睛对阿小妹说出几句话来,正是:

落花堕劫,魂销南浦之歌;飞絮沾泥,肠断西楼之梦。

要知后事如何,且看下回分解。





第一百零一回 扣局帐陈海秋发标 留夜厢范彩霞中计





却说阿小妹听了陈海秋这一番说话,那里摸得着一些头脑?只眼睁睁的看着陈海秋,满心疑惑。只听得陈海秋朗朗的对着自己说道:“这件事儿与你不相干,我也并不怪你;都是你们先生一个人的不好。但是今天你既来收帐,不得不和你讲个明白。我只问你,你们先生既然挂着牌子在上海滩上做生意,吃了这碗把势饭,可懂得把势上的规矩不懂?”

阿小妹听得陈海秋的话风利害,心上也有几分明白,却也不便和范彩霞分辨什么,只得陪着笑脸道:“倪先生有啥勿好格地方末,请耐陈老包涵点……”陈海秋不等他说下去,接着说道:“包涵不包涵的话儿如今不必提他,只讲现在的话。讲起你们先生来,在上海滩上做生意,拼着自家的身体给客人糟蹋,为的是些什么?

无非为一个‘钱’字罢了!自从我和你们先生认得以来差不多将近一年光景,酒也不知吃了多少次,和也不知碰了多少场,一古脑儿合算起来,差不多也花了二三千块钱。像我这样的客人,老实说,上海地方虽然不少,却也不多!为什么你们先生见了我的面总是那一付爱理不理的样儿,连好好的一句应酬话儿都没有讲过?不要说什么住夜不住夜了!像我这样的一个人,又在他身上花了这许多的钱,难道和他攀个相好都够他不上么?老实和你讲,既然吃到了这碗把势饭,就有把势上的规矩。

你们先生在我面上这般模样,简直是不讲情理,硬欺我是个瘟生!他既然把我当作瘟生,不讲情理,我倒今天也要回敬他一下。你们先生要想向我要钱,钱有在这里,六百多块钱的帐,一个大钱也不少他的。不要说是六百,就是六千也现成在这里。

但是要想拿我姓陈的钱,也要有些本领!看他有什么本领来拿我的钱!“

阿小妹听了这一大篇说话,心上不由得吃了一惊。要是别个人的帐,几十块钱的事情,或者一百八十块钱,也还不算什么。偏偏陈海秋这一节的帐,比别节格外多些。明知道范彩霞平日十分挥霍,到了节边狠有些接济不上来,专望着陈海秋这一笔钱来抵挡节底下的开销,那里经得起这样一来!呆了一回,只得立起身来走近陈海秋身边,拍着他的肩膀笑道:“陈老勿要动气,倪先生一径搭倪说,客人里向只有陈老末是个好人。耐勿要缠错,倪先生搭耐一径蛮要好,不过面孔浪像煞有点难为情,说勿出留耐住夜格句闲话。陈老耐也总算是倪搭格老客人哉,勿要实梗瞎想心思哩。倪先生吃仔格碗把势饭,要真真实梗样式,洛里好做啥生意呀?”

陈海秋听了阿小妹的一番说话,要是换了别的时候,早已被他说得心动的了。

这个时候却心上拿定了主意,不肯听他的话儿,只对着阿小妹冷笑道:“不是这般说法。我以前的时候已经和他说过几次,要在他那里住夜,他只是装聋做哑的不肯答应。我又不是白住不出钱的,为什么要受这般的怠慢呢?你回去和他讲,叫他只顾放心,六百块钱暂时放在这个地方,到了那个时候自然给他;这会儿叫他不用心焦,就心焦也不中用!”阿小妹听了,一时也讲不出什么来,只得说道:“依仔陈老格心浪末,要倪先生那哼呢?”陈海秋道:“依着我的心上么,也不是什么难事。

我从前再三的迁就他,他却装腔做势的把我这般冷落。如今只要和他转一个身,叫他收了那以前的架子,到我这里来自家俯就,也就罢了。你快些回去,把我这番说话和你们先生讲个明白,叫他自家斟酌。“阿小妹听了陈海秋这般说法,知道无可再说,只得怏怏走了回去。

去了不多一回,阿小妹忽然又赶到谦泰土栈里头来,见了陈海秋便道:“倪先生请耐到倪搭去,有闲话搭耐说。”陈海秋道:“这会儿我有公事,没有工夫。你们先生请我去,料想也没有什么要紧话说;如若真有什么要紧的话儿要和我讲,请你们先生自己到我这里来就是了。”阿小妹见陈海秋一定不肯去,便匆匆忙忙的往外便走。

陈海秋见阿小妹走了,对着章秋谷伸出一个大指,口中说道:“你的主意果然不差,这样的一逼,等会儿一定自己要来的了。但是他来了,我又怎么样的对待他呢?”章秋谷听了,又细细的教了他许多的法儿,陈海秋大喜,磨拳擦掌的专等着范彩霞来。等了一回,早听得辛修甫口中说道:“来了,来了。”陈海秋立起身来举目看时,只见范彩霞扶着阿小妹的肩膀,从对面屏门外面冉冉的转将过来,那几步路儿就如风吹杨柳一般,走得十分圆稳。陈海秋见了,故意别转了头,装作没有看见。当下范彩霞走进房来,先招呼了辛修甫和章秋谷,又半嗔半喜的瞅了秋谷一眼,方才走近陈海秋身旁,低低的叫了一声:“陈老。”陈海秋回过头来,把范彩霞打量一番:只见他穿着一身玄色外国纱衫裤,下面衬着一双品蓝缎子挑绣的弓鞋,头上只挽着一个懒妆髻,春山淡淡,秋水盈盈,脂粉慵施,铅华不御,低着一双俊眼,好像有些不快的样儿,娇怯怯的站在一旁,把手扶着陈海秋的椅背,口中说道:“耐啥事体实梗动气?就是倪有啥勿好末,耐好好里搭倪讲末哉。倪是无啥要紧,耐气坏仔身体啥犯着呀!”陈海秋听了这几句软软款款的话儿不觉心中一动,连忙忍住了,淡淡的答道:“你不要和我客气,像我这样惹厌的客人,你那里看在眼里!”

范彩霞听了,把一双纤手握着陈海秋的手,说道:“耐勿要实梗嗫,冤枉仔倪,作业格嗫。倪一径搭耐蛮要好,耐勿要听仔别人格闲话,扳倪格差头。耐自家赛过像格哑子,一声勿响,倒说倪……”范彩霞说到这个地方,不觉面上一红,低眸一笑。

又说道:“故歇勿要说哉,一塌刮子才是倪勿好;今朝请耐到倪搭吃酒,总算倪得罪仔耐,赔耐格礼。故歇就请过去末哉。”

陈海秋被范彩霞自己赶过来轻轻的三言两语,已经心上岌岌欲动;现在听得范彩霞邀他过去,便抬起头来看秋谷的眼色。只见秋谷微微的把头一点,陈海秋便也答应。范彩霞本来是马车来的,便拉着陈海秋同车回去。秋谷也有马车,同着辛修甫同坐一车。一路风驰电掣的到东尚仁来。一刻儿的工夫,早到东尚仁门口。大家下车进去。这番不比别的时候,范彩霞竭力巴结,拼命张罗,就是房间里头的人也换了一付样儿。秋谷见了由不得心中暗笑。当下范彩霞和陈海秋并肩坐在炕上,咬着耳朵说了一回。早见一班娘姨、大姐七手八脚的调开桌椅摆上菜来。原来今天这一席酒,是范彩霞和陈海秋赔礼,专请陈海秋的。范彩霞见碟子排了上来,便问海秋还有什么朋友。陈海秋还没有开口,秋谷在旁说道:“我看今天这一席不便请什么外人,只请了王小屏和陶伯瑰两人,何如?”陈海秋听了点头称是,当下写了请客票叫相帮送去。请客的去不多时,客人来了,大家入席畅饮。这一席酒,因是范彩霞专请陈海秋和他赔礼的;肴馔十分精致。范彩霞殷勤相劝,满场飞舞,八面张罗,打起了全副的精神,竭力应酬。陈海秋高兴非常,大家也都开怀痛饮。

到得酒阑人散的时候,已经差不多有十一点钟。辛修甫和章秋谷略坐一回,便都立起来要走。陈海秋也跟着往外就跑,却被范彩霞赶上来一把拉住道:“勿许走,倪还有几几化化闲话要搭耐说。”陈海秋故意笑道:“你留我在这里做什么事儿?

我们先讲明白了再说别的话儿。要我再像前一次的一般吃你的空心汤团,那是再不上当的了!“说着便又要走。急得范彩霞一手拉住陈海秋的衣服不肯放手,面上却一阵阵的红起来。陈海秋故意逼着问他道:”留我在这里,究竟怎么样?我上了一次恶当,再不上第二次的了。“范彩霞听了,口中实在说不出来,顿了一顿方才说道:”耐格个人,啥格实梗假痴假呆介。“说着,阿小妹也赶过来帮着挽留。陈海秋道:”你讲的话不中用,我信不过你的话儿,一定要叫你们先生自己和我讲个明白。“

范彩霞到了这个时候,明晓得陈海秋有意作难,无奈生刺刺的讲不出口来。又见章秋谷和辛修甫两个人都望着他嘻嘻的笑,越发不好意思。没奈何只得把金莲一顿,对着章秋谷道:“二少帮仔倪留留陈老嗫!”秋谷笑道:“我和你把陈老留在这里是狠容易的事情,但是你留住了他在这里干什么呢?”范彩霞听了又羞又怒,又不敢发作,只瞪了秋谷一个白眼道:“耐也装起妈虎来哉!故歇倪想起来,总归是吃仔把势饭格勿好,真真叫呒说法。”说着别过头去,眼圈儿一红。

章秋谷见了这般模样,知道作弄得他够了,便对陈海秋道:“他既然这般留你,你就今天住在这里也没有什么。”陈海秋道:“你不要弄错了夹壁帐。他那里是当真留我,不过当着你们的面儿,讲句好看话儿罢了。”这一句话说得范彩霞发起急来,对着陈海秋道:“天理良心!耐再要讲出实格梗话闲来,只好随耐去说啥格哉!

倪闲话说到实梗样式,耐勿听末,倪也呒啥法子想!只要耐自家想想,阿对倪得起?“

说着扭过头去,不觉流下泪来。章秋谷见了,不由得哈哈的笑道:“算算,算了。”

一面对着陈海秋道:“我们先走一步,明天来看你罢。但是你要小心些儿,不要打了败仗,给他赶到地板上去睡,是与别人不相干的。”陈海秋听了忍不住也笑道:“不要混说。看你这个样儿,光景是长给人赶到地板上去睡的。”范彩霞听了也笑起来,拭了眼泪道:“说说末,就要瞎说一泡,真真歪嘴吹喇叭──一团邪气。”

正是:

酒柬灯炧,缠绵午夜之情;送客留髡,宛转中宵之语。

不知以后如何,请看下回便知分晓。





第一百零二回 酒阑人散软语缠绵 送客留髡深情缱绻





却说陈海秋见章秋谷同着辛修甫要走,想着这样的一来,居然坍了范彩霞的台,出了自己的多时闷气,大功告成,心上十分得意;更兼范彩霞紧紧的拉着他两只手不肯放松,把一个身体差不多全个儿都扑在陈海秋身上,一个脸儿就紧紧的贴着他的肩膀,面粉口脂,暗香发越。陈海秋鼻子中间,觉得有一阵阵的香气直透进来,更觉踌躇满志,却做意再说一句道:“你虽然殷勤留我,但是这件事情是要各人自己愿意的。你要是不愿意,勉勉强强的敷衍一下,我也没有什么味儿。你心上究竟怎样?倒是讲明白了的好。”范彩霞听了,不由得皓齿微呈,蛾眉欲蹙,含怨含颦的说道:“谢谢耐,阿好推扳点,就是实梗仔罢。”说着眼圈儿又是一红,眼眶里头水洋洋的含着一汪珠泪,好似那梨花带雨,芍药当风。陈海秋见了范彩霞这般模样,觉得自己心里头也有些七上八下的不得劲儿。那以前的旧恨,早不知丢到那里去了。看看范彩霞这样的赔着小心,觉得他又是可怜,又是可爱,不由的微微含笑,看着范彩霞??脸儿。这个时候,陈海秋心上的那一番得意,在下做书的一时也形容不出来。

只说章秋谷看了他们两个人的一番情景,知道这个时候的陈海秋,已经入了范彩霞的温柔圈套,便趁势对陈海秋道:“我们两个人走了。你们两口儿好好的装枪备马,预备登场。我们要少陪了。”说得大家都笑起来。连范彩霞也忍不住笑,只用衣袖掩着嘴,格格的要笑出来。秋谷也不等陈海秋再说什么,便拉着辛修甫一同走了。

这边范彩霞好容易把陈海秋留了下来,自然也拿出浑身本事来笼络他。只见锦帏半掩,罗帐四垂;街鼓沉沉,清宵细细。杨柳怀中之玉,软语温存;梨花颊上之痕,风情熨贴。这一夜陈海秋的满心得意,范彩霞的格外牢笼,说不尽的万种绸缪,千般旖旎。一直睡到明天十二点钟,两个人还是春梦迷离,睡得十分甜蜜。

陈海秋正睡得恍恍惚惚的,好像耳朵里头有个人在那里叫他。睁开两眼看时,原来就是章秋谷,满面春风的站在床侧,一手撩起帐子,哈哈的笑道:“怎么睡到这个时候还不起来?想是昨天晚上辛苦了,所以这般困倦。”陈海秋见了章秋谷的面,打了一个呵欠,自己也觉得有些不好意思。看那范彩霞时,枕着自己一只手臂,还微微的睡着,星眸双合,香梦沉酣。陈海秋见了觉得十分可爱,顾不得章秋谷在旁看着,不由得把自己的脸去贴着范彩霞的脸儿,紧紧的揉了一揉。秋谷看着,不觉叫一声“好”!这一下子,早把个范彩霞惊醒。睁开俊眼,早见了章秋谷笑迷迷的站在那里。羞得个范彩霞脸涨通红,无地可避,连忙没头没脑的把头缩进夹纱被窝里面去。听得章秋谷笑道:“你不要不好意思。上海地方的倌人,那一个不是这个样儿?为什么见了我就急到这般模样?”范彩霞听了也不开口,只把被窝兜着自己的头,好像没有听见的一般。

陈海秋坐起身来穿好衣服,跨下床去,往床后转了一转,便向章秋谷说道:“你怎么今天这个时候就来了?”秋谷笑道:“这个时候还早么!差不多已经将近十二点,你们两个人还在这里睡觉,未免太舒服了!”陈海秋听了一笑,也不言语。

接着范彩霞遮遮掩掩的从床上溜下来。秋谷走过去,拉着他的手道:“恭喜,恭喜!”

范彩霞红着个脸,头也不抬,洒脱了手,一溜烟逃到床后去了。停了好一回,才慢慢的走出来。见了章秋谷觉得有些羞怯怯的,再也不抬起头来。挨了一会儿,范彩霞方才问章秋谷道:“耐阿曾吃点心?阿要叫俚笃去叫得来,搭陈老一淘吃?”秋谷笑道:“我是吃过的了。多谢盛情,不必这般客气。你还是料理你们的陈老爷罢!”

范彩霞听了,把眼一瞟道:“耐格个人,总归呒拨好闲话说出来格。陈老末陈老哉啘,啥格是倪格介。”秋谷哈哈一笑道:“你们昨天晚上恩到这般地步,今天早上睡到这个时候还不起来,恨不得两个人挤作一团,并作一块,还说不是你的?难道不是你的,倒是我的不成?”说得陈海秋好笑起来。

范彩霞委实不好意思,只得说道:“随便耐去说啥末哉!”说着,便低低的问海秋要吃什么点心。陈海秋道:“叫他们去叫一碗一钱六分的生炒鸡丝面罢。”不一会,相帮端上面来。陈海秋吃了,便同着章秋谷起身想走。范彩霞那里肯放,道:“耐格辫子毛哉,搭耐打好仔辫子去。”说罢,取过梳篦,自己和陈海秋拆开辫发,慢慢的梳。秋谷在旁看着。只见范彩霞把陈海秋的几根头发梳得通了,用刨花水刷了又刷,刷得没有一根松的,方才顺着头发,一路一路的编起来。一面编着,又用刨花水刷那松出来的头发。一根辫子,直打了半点钟的工夫,果然亮油油的十分好看。秋谷在旁看着,不觉说一声:“打辫子的本事!果然不差!”范彩霞回过头来,把手在自己头上打个手势,微微的对着秋谷一笑。秋谷见了,连忙把头摇了一摇。陈海秋打完了辫子,要和秋谷同走。范彩霞一把拉住问道:“晏歇点阿来?”

陈海秋道:“自然来的。”范彩霞道:“晏歇点要来格啘,绰仔倪格烂污是,倪勿来。”陈海秋道:“等会儿晚半天一定来就是了。”范彩霞听了,方才放手。

陈海秋刚才举步,忽然想起一件事来,停止脚步笑道:“几乎忘了一件最要紧的事情。”说着,便从衣袋里头取出几张庄票,对范彩霞说道:“我的酒局帐,合算起来,通共六百几十块钱,如今统通给你。”说着顿了一顿,又道:“节底下你的开销怎么样?”范彩霞沉吟一会,方才说道:“倪间搭节底下也呒拨几化开销,有限煞的。收下来格局帐,拿得来开销开销,刚刚正好。”陈第秋听了,便拣出一张一千块钱的一张即期庄票,放在范彩霞手中道:“你和我给他们四十块钱下脚,多下来的,送你买几件衣服罢。”范彩霞欢欢喜喜的接了过来,口中说道:“陈老再要实梗客气,放来浪陈老搭末一样格啘。”陈海秋摇摇手道:“节底下比不得平时,大家都要开销的,你也不用和我客气。”范彩霞听了方才接了过来,谢了一声。

陈海秋便同着章秋谷走了出去,两个人一前一后的到了马路上。章秋谷对着陈海秋笑道:“好贵的打辫!打一条辫子足足的一千块钱!”陈海秋听了也笑个不住。

当下章秋谷同陈海秋两个人坐上马车,一路讲着闲话,一同到辛修甫公馆里头坐了一回,辛修甫他们两个吃饭。吃过了饭又谈一会,秋谷取出表来看时,见刚刚正指三点,想着昨天约着陆丽娟坐马车到张园去的,便辞了辛修甫,说要和陆丽娟去坐马车。辛修甫道:“我也要到西安坊去,我们一同出去罢。”章秋谷道:“既然如此,我们何不大家到张园顽顽?”辛修甫道:“也好,我们大家到张园会罢。”

说罢便换了衣服,就趁了章秋谷、陈海秋的马车先到了西安坊,辛修甫便下车进去。

秋谷候马车到了久安里门口。因陈海秋要到东尚仁,秋谷便跳下马车,自家进

去。

到了陆丽娟院中,只见陆丽娟早已梳好了头,换了衣服在那里等候。见了秋谷进来,便笑吟吟的迎上前来,搀着秋谷的手笑道:“耐倒好格,昨日仔讲明白仔三点钟同倪去坐马车,故歇三点钟敲过哉!”秋谷微微笑着坐下来,叫相帮到善钟马房去叫一辆自拉缰的亨斯美来;一面和陆丽娟道:“你还是一个人坐,还是和我一起坐?”陆丽娟道:“生来一淘坐哉啘!”秋谷道:“和我坐在一起虽然没有什么希奇,但是万一个给人看见了,说你做我的恩客,便怎么样呢?”陆丽娟听了把秋谷一推道:“随俚笃去说末哉!倪是勿怕格。就算倪做仔耐格恩客末,也勿关俚笃啥事啘!”秋谷笑道:“你当真不怕人家说我是你的恩客么?”陆丽娟嗔道:“耐格人啥烦得来,阿是勒浪讨厌倪?勿要倪搭耐一淘坐?”

秋谷听了正还要和他取笑,只见马夫阿荣跟着一个相帮走上楼来,对着秋谷说道:“二少爷,马车来哉。”秋谷听了便立起身来,同着陆丽娟一同下去。走到久安里门口,只见一匹小小的川马浑身漆黑,神骏非常,驾着一辆双轮马车停在弄口。

秋谷先叫丽娟坐上车去,自己也跳上车来。阿荣递过丝缰,秋谷顺手接过,轻轻的一提,那马已跑开四蹄,向前便走。秋谷见四马路一带人来人往的十分热闹,便带住丝缰,慢慢的走;到了大马路一带,地方宽阔,秋谷把缰绳紧了一紧,拔出鞭子来只轻轻的在马背上一掠。那马见了鞭子的影儿,便电掣风驰,飞一般的向前直驶。

一会儿早已过了泥城桥,直到张园门首。秋谷的马车一直放到安垲第门前停住。

秋谷和陆丽娟下得车来,走进安垲第,四面兜了一转,却不见一个熟人。正要回身出来到老洋房去,早见迎面走进两个人来。一个男的,穿着一件湖色单纱长衫,玄色外国纱马褂,带着一顶极细的草帽,眉清目秀,齿白唇红,却有些滑头滑脑的样儿;一个女的,倌人打扮,一身银灰色闪光纱衣服,长挑身材,鹅蛋脸儿,皓齿明眸,丰容盛翦。两个人一前一后的走进来。秋谷猛然见了这个倌人,觉得他十分面熟,好像在那里见过的一般,一时却想不起来。这个倌人和秋谷擦肩过去,眼波澄澄的,正和秋谷的眼光碰个正着,登时也呆了一呆。秋谷这个时候,身不由己的跟着这个倌人缩进安垲第来。陆丽娟不知为的什么事儿,只得也跟着进来。正是:

飘零红粉,偏多迟暮之悲;落拓青衫,谁有穷途之泪?主

要知后事,且听下回分解。





第一百零三回 味莼园遇旧感前游 金小宝寻春逢浪子





且说章秋谷看着那个倌人的模样,觉得面熟得狠,却想不出他叫什么名字来。

见那倌人同着那个男子走进安垲第,四面看了一看,便拣一张桌子坐下。秋谷便也拣了对面的一张桌子坐了下来,目不转睛的看着那个倌人。那个倌人也秋波澄澄的看着章秋谷。两下正看之间,忽见辛修甫同着龙蟾珠款款行来。龙蟾珠一直走到面前,含笑招呼道:“二少,耐阿是来仔一歇哉?”秋谷也含笑让坐。那知龙蟾珠这一声“二少”,猛然把那对面的倌人提醒,不觉失声道:“阿唷!勿壳张是二少!

多时勿见哉啘。刚刚倪碰着仔耐,像煞有点面熟蓦生,肚皮里向想来想去,总归想勿出是陆里搭看见歇格。故歇想仔出来哉,实头是二少。“秋谷听得那倌人和他讲话,说话的声音十分熟溜,不觉也恍然想起道:”原来是你!差不多一年勿见,几乎大家都认不出来。“

看官,你道这个倌人是谁,原来叫做祝小春,也是上海滩上一位大名鼎鼎的人物。以前秋谷做陈文仙的时候,祝小春和陈文仙狠是要好,两下常常来往;和章秋谷言来语去的,狠有些儿意思。陈文仙见了,虽然不怪秋谷,但未免总有些儿吃醋的意思。对着祝小春总是淡淡的,不狠应酬他。后来祝小春做着了一户好客人,包了他一节,又在苏州做了差不多半年。如今回到上海来再筑香巢,芳名大震。就在清和一包了楼上三间房间。章秋谷和他一年不见,两下见面都模模糊糊的想不出来。

当下章秋谷见了祝小春,便也和他讲些闲话,又说说陈文仙的话儿。小春道:“文仙阿姊跟着仔耐,总算是俚格福气。故歇辰光,倌人要嫁格好好里客人,倒勿容易哩!”秋谷听了正要回答,忽然一眼看去,见那个和小春同来的男子满面怒容,眼睁睁的看着自己。秋谷见了,知道是和他吃醋,便微微一笑,对着祝小春道:“我们改天再谈罢!”小春听了还没有开口,早见那个男子恨恨的催着小春道:“这里没有什么味儿,我们还是到弹子房去罢!”祝小春还不知道什么意思,随口答道:“刚刚来得勿多一歇,等倪坐歇再去末哉!”那个男子听了那里肯依,只在那里死命的催促。祝小春还在那里延延挨挨的不肯走,忽然看见章秋谷对着他微微含笑,把嘴往那边一努,祝小春方才回过头来看了一看,只见那个男子已经气得满面通红,恶狠狠的催着他要走。祝小春心上方才明白,冷笑一声,只得跟着他一同出去。章秋谷这边的事,权且按下不提。

只说那四大金刚里头的金小宝,自从贡春树回去之后,心上觉得好生眷恋,便天天坐着马车到张园去兜个圈子,借此消遣。这一天金小宝正坐着马车从四马路兜转泥城桥,望着张园、静安寺一路跑去。将近张园门口,忽然见一个西洋装束的少年,年纪不过二十多岁,穿着一身极细的黑呢衣服,身材伶俐,举止轻扬,坐着一辆自行车,好似星飞电转的一般,从背后直赶过来,抢出金小宝马车的上首。见了小宝,飞了一个眼风,微微一笑,把身体往前一伏,两脚用力向前一送,只见那一辆脚踏车,就如箭一般的直赶过去。金小宝看了,不知怎样的觉得心上微微一动。

一转眼的工夫,马车早到了张园门口。小宝一眼看去,早又看见那方才的少年男子站在道旁,把那一辆脚踏车倚在一棵树上。见了小宝的马车过去,对着小宝微微的又笑一笑,接着跳上脚踏车,飞也似的又赶过金小宝前面,直到安垲第门口方才一跃而下。等金小宝的马车停住,下了马车,轻移莲步往内便走,这个少年男子便也在后跟来。

金小宝见了,明知道是有心吊他的膀子,便偷着回过头来细细的打量这少年男子。只觉得他细腰窄背,骨格风华,面貌倒也不俗。小宝看了,便也对着他嫣然一笑。这一笑不打紧,只把这个少年男子喜欢得眉飞色舞,手舞足蹈,越发的紧紧跟着一步不离。见小宝拣一张桌子坐下泡茶,他也在隔壁桌子上坐下泡茶。四目相对,你来我往就好像空中的流电一般,渐渐的两下都有些意思了。等了一回,只见那少年男子叫过堂倌来,说了几句不知什么,堂倌走过来对小宝说道:“这里的花钱有了。”小宝回头一笑,尚未开口,早见那少年男子抢步过来,对着小宝点一点头道:“小宝先生,今天怎么有空到这里来?”金小宝听了,觉得好像有些不好意思,却又没本事不答应他,只得把那一点朱唇略略的动了一动,就算答应过了。那少年男子又对着金小宝道:“我姓牛,堂子里头的人大家都叫我小牛。”小宝听他说到这里,禁不住“嗤”的一笑。那少年也不理会,接着说道:“我们老太爷放过美国的参赞大臣,如今已经故了。我久仰小宝先生的大名,本来想要去看你,如今刚刚我们两个人在这里遇见了,也是三生有幸!”金小宝听得他说出来的话儿十分巴结,心上早有几分欢喜,横波一转,笑口微开,便对着那小牛说道:“牛大少,请间搭坐歇。”小牛巴不得小宝有这一句话儿,诺诺连声的坐了下来。金小宝和他谈了一会,觉得这个人狠是知趣,便存了个和他款洽的念头。

看官,你道这个人究竟是谁?原来果然是出使美国大臣牛康伯的儿子,叫做牛幼康。牛康伯放了一任美国钦差就死了,止有牛幼康一个儿子。差不多也有二三十万银子的家产。牛幼康从牛康伯死后,隔了几年,渐渐长成,却生得十分清秀,读书也甚是聪明。只有一件毛病不好,见了一个女人,就如苍蝇见了血的一般。瞒着家里头的人,在外头死命的嫖。偏偏的牛康伯那位夫人治家整肃,严厉非常。牛幼康除了问他母亲要几个钱零用之外,捞不着一个大钱。没有法子,便只好靠着自己的年轻貌美做个幌子,到处去哄骗那些倌人,只说他还没有娶过正室,要娶他去做正室夫人。从来鸨儿爱钞,姐儿爱俏。这班倌人见了这样的一个标致少年,那有不爱的道理!更兼倌人们最不愿意的,是嫁给人家做姨太太;最喜欢的,是有人娶他去做正妻。牛幼康对着这班倌人,便把这些说话来哄骗他们,骗得那些倌人一个个都随手而转,大家都要想做牛幼康的结发夫人,把个牛幼康就当作天字第一号的恩客,非但不要他用钱,而且还肯倒贴他两个。无奈上海的倌人十个里头倒有九个是穷的。牛幼康虽然不要化什么钱,却也弄不着什么大好处。也是金小宝合当晦气,偏偏撞见了这个宝贝!

闲话休提。只说金小宝和牛幼康谈了一回,金小宝掏出一个打簧金表来看时,已经五点一刻,便立起身来要走。对牛幼康说道:“倪先去哉,牛大少晏歇点请到倪搭来。”牛幼康恭恭敬敬的答应一声道:“我就立刻过去和你请安。”金小宝笑道:“阿唷!请安是勿敢当格。牛大少啥实梗客气呀!”牛幼康道:“小宝先生那里比不得别处,只要肯赏我的脸,容我到那边去坐一回儿,就是我的福气了!”小宝听得牛幼康这般说法,自然高兴。从来世上的事情,千穿万穿,马屁不穿。何况牛幼康又是个堂堂一表的青年,自然的更加有效。金小宝便对牛幼康说道:“牛大少勿要客气,搭倪一同转去阿好?”牛幼康听了大喜,便同着金小宝一起出来。金小宝坐上马车,牛幼康坐着脚踏车跟在后面。一路上牛幼康卖弄精神,故意把脚踏车放得慢慢的,和马车同走。一霎时早已到了惠秀里门口,金小宝同着牛幼康进去。

牛幼康到了金小宝房间里头,便四面看了一看,口中啧喷叹羡道:“好精致的房间!不是小宝先生,也配不上这样的房间!”金小宝笑道:“倪间搭是勿好格,小地方龌龊煞,请牛大少包涵点。”牛幼康看了一回,向小宝说道:“这样精致的房间,我想要借你这里请几个朋友,不知你答应不答应?”小宝道:“牛大少要请客末蛮好,只怕耐牛大少勿来照应,阿有啥倪倒勿肯格道理?”牛幼康听了十分欢喜,走到小宝面前深深的打了一拱道:“多谢小宝先生赏我的脸。”正是:

高唐云雨,谁偷韩椽之香;醋海风波,妒煞宓妃之枕。

不知后事如何,且听下回分解。





第一百零四回 跳空槽滑头得志 翻醋罐名妓争风



却说牛幼康走到金小宝面前深深的打上一拱,金小宝见了,连忙把身体扭了过去,格格的笑道:“牛大少,勿要嗫。拨别人看见仔,阿要难为情!”牛幼康笑着说道:“老实说,若是换了别人,不要说叫我给他打拱,就是翻过来他给我打拱,我还有些不高兴呢!如今在小宝先生这里,不要说打个把拱,就是叫我天天给你叩一个头,我也没有什么不情愿!”小宝掩着口笑道:“倪陆里有格号福气呀!”牛幼康道:“我没有这般福气是真的,怎么你倒说起这样的笑话!”金小宝对着一班娘姨大姐笑道:“唔笃大家听听看,说得阿要好听!”小宝口中虽是这般说法,心上却着实高兴,便也应酬了牛幼康一番。牛幼康更加得意,两个人谈了一回,牛幼康写起请客票来,叫相帮送去。不多一刻的工夫,请的客人陆续到来。这一席酒,直闹到二更天气,一班客人方才散去。

自此以后一连几天,牛幼康在金小宝院中请客,拼命的奉承金小宝,把个金小宝奉承的心上迷迷糊糊起来,不多两天的工夫,竟落了牛幼康的圈套,留他住夜。

牛幼康便又把那一套骗人的话儿说给金小宝听,只说自己尚未娶妻,要把金小宝娶为正室。金小宝听了他的一番谎话,心上虽然欢喜,却又有几分疑惑的意思,不敢相信。暗想牛幼康这般家世,家里头又有太夫人在堂,那里肯娶个倌人回去做媳妇?

金小宝心上有了这个意思,对着牛幼康却不便说出来。无奈这牛幼康哄骗倌人的本领实在不差,慢慢的骗来骗去,竟把金小宝骗得个死心塌地,一心一意的想嫁起牛幼康来。

看官,你想四大金刚里的金小宝是何等的人物!本来打定主意不想嫁人的,就是贡春树和他这样的深情缱绻,恩爱缠绵,也没有要嫁他的意思。这样一个阅历深沉的人,却给牛幼康一阵鬼混,鬼混得活动起来,这牛幼康骗人的本领,可想而知的了。闲话休提。只说金小宝自从和牛幼康落过相好以后,便不肯要牛幼康花一个钱,就是牛幼康自己身上的开支,都是小宝和他代付。一班娘姨、大姐见了牛幼康这般模样,没有一些儿好处到他们身上,一个个心上都觉得十分不快,渐渐的都放到脸上来,见了牛幼康的面,大家都不狠理他。小宝的生意本来是狠好的,小宝为着一心一意想嫁牛幼康,见了别的客人都冷冷的不狠应酬。客人里头也有知道这件事情的,讲出去给人听了,登时一传十、十传百,大家都知道了这件事儿。小宝身上的一班熟客,慢慢的都裹足不前起来。依着小宝的意思,叫牛幼康立刻娶他回去,无奈牛幼康讲的本是一片谎话,那里有个影儿?便一天一天的支吾过去。

这一天牛幼康正和小宝坐着讲话,忽见小宝的梳头娘姨、绰号叫做强盗阿金的,满面怒容走进房来,对着牛幼康瞪了一眼,便一屁股坐下。小宝觉得诧异,还没有开口,早听得阿金大声讲到:“倪间搭故歇里鬼也呒拨一个来格哉!格扇招牌挂俚做啥?探探脱末拉倒哉啘!”小宝听了心上早已有些明白,便皱着眉头道:“呒拨客人来勿关耐事,用勿着耐来嘤嘤喤喤,算啥格样式介,规矩也呒拨格哉!”阿金冷笑道:“耐有客人呒客人,生来勿关倪事。不过倪刚刚来格辰光,讲明白生意浪有拆头格。故歇勿要说到拆头,连拆脚才勿着杠。屋里向几几化化人,才靠仔倪一干仔吃饭,一塌刮子拿仔三块洋钿一月,陆里开销得转?倪要去哉!梳头娘姨末,耐自家另外去寻,勿关倪事!”

金小宝猛然听了这一番没情没理的话儿,只气得气满胸膛,花容失色,一时倒也说不出什么话来。停了一回方才咬着牙齿,把金莲一顿道:“耐要去末,去末哉啘!阿有啥人来留耐呀?说出格号放屁格闲话来,阿要气熬仔人!”阿金立起身来淡淡的说道:“倪是娘姨,生来勿好管耐格事体,只要耐勿要上别人家格当好哉!”

小宝越发生气道:“就算倪上仔别人家格当末,也勿关得耐啥事。耐搭倪滚出去!

勿要勒浪吵勿清爽。“阿金道:”去末去末哉,呒啥希奇;耐勿要反嗫。倪倒要张开仔眼睛,看看耐格位牛府浪格少太太笃!“小宝听了气得浑身乱抖,拍着桌子口中乱骂。阿金口中也有些不干不净的话骂出来。小宝气到极处,叫进相帮来,立时立刻的把他撵了出去。又把他的东西铺盖一古脑儿都丢出门外,方才气平了些。想着这场口舌,是为着要嫁牛幼康起的,便叫相帮立刻把牌子除了下来。相帮心中虽然不愿意,却又不敢不听,只得除下牌子,送进房间。

金小宝见牌子已经除了,便催着牛幼康央媒择日,讲明不要他一个钱身价。牛幼康还想支吾,金小宝那里肯听?牛幼康只得暂时答应,心上却在那里打算脱身的主意。过了一天,问小宝借了两付金镯子,只说有人要照样打造,要借去看个样儿。

金小宝绝不疑心,慨然交付。那知这一下子就如断线风筝,出笼黄鹄,一连去了几天,连个影儿也不见来。

小宝自从和牛幼康认得以来,两个人没有一天不见面的,如今忽然几天不来,小宝还十分记挂,只道病了,狠觉得不放心。叫个人到牛幼康家里头去问信,又问不出来。细细的在外面打听了几天,方才知道牛幼康有一天同着朋友在戏园里头看戏,遇见了祝小春也在包厢听戏,两个人眉来眼去的吊膀子,竟吊上了。牛幼康当时跟着祝小春回去,只摆了一台的酒,轻轻易易的就有了相好。从来男子的性情,都是得新忘故的。牛幼康看了祝小春的一颦一笑、一举一动,觉得都比金小宝高些,便把以前哄骗金小宝的那一番手段,都移到祝小春身上来,一连在祝小春院中住了几天,金小宝那边竟是绝迹不去。

这个信息传到金小宝耳中,金小宝不听犹可,听了这句话儿,这一气非同小可,觉得眼迸金花,耳鸣钟鼓,登时地转天旋的,心上就有些恍恍荡荡起来。想着他骗了两付金镯子去还没有什么希奇的,最可恨的,拿了自己的金镯子倒反去送给祝小春,真是有生以来从没有上过这般恶当!呆呆的气了一回,要想就是这样的割断了罢,毕竟心中有些割舍不得。便叫手下的娘姨大姐,到祝小春那边去请。无奈到了那里,小春院中的人总回说不在这里,一连去了七八次都是这般。小宝气得无可如何,只得忍着,再叫人细细探听。想着牛幼康躲在祝小春院中不便去找,只好趁着他们两个人一同出门的时候再去找到了他,和他理论。小宝为了这件事儿,心上二十四分的抑郁,也不梳头,也不出门,恹恹闷闷的过了几天。

这一天下午,小宝吃过了饭,一个人坐在那里,捧着一支金水烟袋,呆呆的只顾出神。只见一个大姐阿囡匆匆的走进来,对着小宝说道:“今朝牛家里搭仔祝小春两家头一淘坐仔马车到张园去哉。”小宝听了跳起来问道:“阿是真格呀?”阿囡道:“自然是真格啘。刚刚一大里向马夫阿龙来搭倪说格,阿有啥假格呀!”小宝道:“耐豪燥点,叫阿龙拉一部马车来,倪两家头一淘去。”阿囡答应着去了。

不一刻,马车早已放到门口,小宝把头略略的拢了一拢,薄施脂粉,换了一件衣服,立刻同着阿囡坐上马车,赶到张园。先到安垲第内略略的看了一看,便到老洋房照相馆去兜了一趟,不见牛幼康的影儿。小宝见找不到,心上甚是懊恼,只得又到弹子房来。刚刚走进门口,就看见牛幼康正高高兴兴的同着几个人在那里打弹子。祝小春立在牛幼康一起,两个人指指点点的不知道在那里说些什么。

金小宝走进门来,一眼看见了牛幼康,不觉怒从心起,蛾眉紧皱,粉面通红,抢步上去冷冷的对着牛幼康说道:“耐倒好格,几日天勿到倪搭去,倒一干仔勒浪舒齐!”牛幼康猛然见了金小宝进来,由不得心上大吃一惊,带耳根连脖子都胀得通红。听了金小宝的几句话儿,一个字儿都回不出,就如一个木偶一般呆呆的站在那里。金小宝又冷笑道:“请耐同仔倪一淘转去,倪有两声闲话要问问耐。”牛幼康听了好像没有听得的一般,站在那里动也不动一动。金小宝道:“去嗫,阿曾听见呀!”牛幼康听了,还是一个不动。气得个金小宝赶过去,伸出右手,拉住牛幼康一只耳朵,拉着便走。拉得牛幼康抱着头,叫声“阿唷坏”。

金小宝正拉着牛幼康的耳朵要走,忽然祝小春抢上前来,一手拉住牛幼康,一手拦住金小宝,高声说道:“耐是啥人介,拉仔牛大少到啥地方去?有啥闲话,好讲出来拨大家听格啘。拉拉扯扯,算啥样式介!”金小宝正一肚子的没好气,也大声说道:“倪末就是金小宝。牛大少是倪搭格客人,倪同俚转去,有闲话搭俚说,勿关耐事。用勿着耐来多管!”祝小春冷笑道:“啥人说勿关倪事介。牛大少末是倪格客人,耐要搭俚说闲话末,到俚府上去请末哉!故歇勒浪归搭末,就叫勿成功!”

说着,又对着旁边的众人说道:“唔笃大家看看,也呒拨实梗样式格啘!勒浪归搭地方,几几化化格人,动手动脚,真真面孔才勿要格哉!”金小宝听了气得大骂道:“耐格< 毛乍> 千人格烂污婊子!倪搭格客人做得好好里来浪,拨耐个烂污婊子拉仔过去,再有面孔搭倪瞎吵!”祝小春听也了大怒道:“倪是烂污婊子,耐阿是人家人呀?大家才是一样路,呒啥海外!耐说倪拉仔耐格客人,阿是倪到耐屋里去拉客呀?上海滩浪客人末,也勿是做一个倌人;倌人末,也勿是做一个客人!挂仔牌子末,只要是客人末,大家好做格。耐格客人末那哼呢?阿是耐格客人,就勿许倪做格?老实说,勿要说倪朆拉耐啥格客人,就算倪拉仔耐格客人末,耐也只好两只眼睛望望倪!耐有本事末拉牢仔客人,勿要放俚出来。故歇自家做勿牢客人,客人跳仔槽,再要说出实梗格闲话来,阿要鸭屎臭!”

金小宝听了祝小春这番说话,一时竟想不出什么话来回他,只得也骂道:“耐自家勿要面孔!拉牢仔客人勿放,再要说别人鸭屎臭!”祝小春微微冷笑道:“唔笃大家听听看,到底是啥人勿要面孔?耐是勿挂牌子格住家呀,倒有面孔到归搭来拉客人格,就是四马路浪格野鸡末,也勿糙至于实梗样式啘!”

这几句骂得十分刻毒,金小宝怒气冲天,放了牛幼康,伸出手来把祝小春劈面一掌。祝小春不提防他要动手,出其不意“拍”的一声,左边脸上着了一下,只打得祝小春粉面生烟,星眸出火,大声骂道:“勿要面孔格烂污婊子!再有面孔打人!”

说着便也伸出手来,一把扭住金小宝胸前衣服,还他一掌。小宝急忙一闪,立脚不定,身体向前一晃,扑倒在地下。祝小春扭住了金小宝衣服不肯放手,一同跌下地去。两个人就在地下滚作一团。阿囡立在旁边,见小宝倒在地下,想要抢过去帮时,早被祝小春跟来的一个娘姨拦住。就这个一转眼的时候,人丛里早转出一个人来。

正是:

嗔莺叱燕,何来娘子之军;绿舞红飞,不数鸳鸯之队。

不知这个出来的究竟是什么人?下回交代。





第一百零五回 祝小春得意占情郎 章秋谷正言讥浪子





却说金小宝和祝小春两个人正滚在地下,人丛里早挤出一个人来。这个人究竟是谁,料想列位看官也不用在下做书的饶舌,一定知道是章秋谷了。

只说章秋谷走上前来,轻轻的把金小宝同祝小春两个人在地下扶了起来,一手拉着一个,口中说道:“你们有话好说,何必动手动脚,失了体统!”祝小春还没有开口,金小宝早听得章秋谷的声音,心上就吃一惊。抬起头来看时,果然就是章秋谷,只羞得个金小宝满面通红,心头乱跳,几乎要急出泪来,恨不得有个地洞让他钻了进去。低着个头,再也不敢抬起来。只听得章秋谷朗然说道:“你们为什么这般争闹?把这件事儿讲出来给我听听,或者可以和你们说句话儿。”祝小春听了,便抢着把自己和金小宝的事情对秋谷讲了一遍。秋谷点一点头。又问小宝道:“你这般生气,究竟什么原故?”小宝没奈何,只得也把这件事儿略略述了一遍。秋谷听了,便正色向牛幼康道:“尊姓是牛,想来是牛钦使的少君了?还没有请教台甫,是那两个字儿?”

牛幼康见章秋谷两只手两边挽着金小宝和祝小春,心上狠不愿意,却又说不出来;如今见秋谷问他的号,没本事不答应,只得顺口答道:“不敢,贱字幼康。”

章秋谷正颜厉色的对他说道:“牛幼翁,不是兄弟大胆,说句放肆的话儿,这件事儿,他们两个人都没有错处,都是你老兄一个人不好。你既然借了小宝的两付镯子,不该应一连几天不去,怪不得小宝动了疑心,出来找你。小春见自己的客人平空被别人拉了去,不晓得这里头还有这样的一回事情,出来讲话,却也不能怪他的不是。

如今事情既然已经闹到这个地步,你老兄打算怎么样呢?“说着,便回向祝小春、金小宝两个人说道:”据我看来,你们两个人平日之间又没有什么仇恨,何必为着这点儿小事大家吵闹!况且说起来,无非为着客人身上的事情,传说出去也没有什么好听。不如你们两下都看在我的面上,讲了和罢。“祝小春听了抢着说道:”倪好好里搭俚讲闲话,俚倒勿问三七廿一,四七廿八,拔出手来就打。格是啥格道理?

倪倒要问问俚笃!“秋谷笑道:”不必说了,你们相骂无好言,相打无好手。他虽然平空打你一下,你也把他拉了一交,大家只算得一个扯直。依着我的话儿,大家只当没有这件事儿也就算了。“

这个时候的金小宝,心上觉得好生惶愧。偏偏这样的事儿又给章秋谷来撞见了,又羞又悔,一句话也讲不出来,恨不得立刻跑了开去。无奈一只手被章秋谷紧紧拉住,无可如何。听了章秋谷的一番说话,巴不得两下讲和,便抬起头来含羞说道:“二少格闲话蛮准,大家只当呒拨格件事体末,拉倒哉啘。”祝小春起先的意思还有些装腔作势的不肯答应,如今见金小宝先答应了,觉得自己占了上风,便也高高兴兴的点头应允。

秋谷见两下都答应了,心中自是欢喜。回过头来对着牛幼康说道:“老兄还借了小宝的两付镯子没有还他是不是?”牛幼康蓦然之间听了这一句话儿,不觉面上一红道:“那是有的。兄弟连日有事,没有工夫,所以直到如今还没有带去给他。”

秋谷微笑道:“小宝那里,你老兄的去与不去,我们旁人不能一定要你怎样;至于这个镯子的事情,似乎该应赶紧还他方才是个道理。如若不然,给别人传说起来,不说你老兄一时匆促没有工夫;只说你老兄这般家世,还要吞没倌人的东西,未免有些不好听。”牛幼康听了心上十分不快,待要发作几句,又发作不出来,只得红着脸说道:“这是那里说起。我兄弟也何至于做这样没出息的事儿!如今明天就叫人送去就是了。”秋谷听了,知道他心上不快,便又对他说道:“论起理来,这件事儿与旁边人不相干。不过照理上看起来,该应是这般办法就是了。”说着便放了祝小春,携着金小宝的手说道:“我们还到那边安垲第去坐一回儿。”金小宝答应一声,轻移莲步跟在秋谷后面。陆丽娟和辛修甫、龙蟾珠等也一起跟来。

秋谷临出弹子房门口的时候,回过头来和祝小春打了一个照会,笑微微的说道:“我们隔天再见。”祝小春见章秋谷携着金小宝的手和他同走,那样儿甚是亲热,不觉心上也有些酸溜溜的起来,对着秋谷把嘴披了一披,也不言语。秋谷会意,只是微微的笑,也不去理会牛幼康,同着金小宝一干人竟转到大洋房来,重新拣了一张桌子,五个人团团坐下。

金小宝虽然坐在桌子上,只是面红耳热的不好意思。秋谷见了,便对小宝说道:“坐在这里也没有什么道理,我们出去走走好不好?”金小宝听了,巴不得这样,便立起身来和辛修甫、陆丽娟等打了一个招呼,同着秋谷一直的走到草地上去。秋谷恐怕小宝走不上来,便慢慢的走。走了一段,小宝已经觉得有些娇喘微微。秋谷搀着他的手,在树阴里头歇了一回。小宝忽然抬起头来,朱唇微动,好像要和秋谷说话的样儿,却又脸上一红,低下头去。秋谷见了,已经猜料了七八分,问他有什么话说。小宝延挨了一回,方才吞吞吐吐的说道:“谢谢耐。今朝格件事体,阿好……”金小宝说到这两个字儿,顿了一顿说不下去。秋谷接着说道:“你只顾放心,贡春树那边,我决不提起就是了。其实这件事儿,也没有什么希奇,吃了把势饭,没有法儿,就是春树知道了,也不能怪你。”小宝听了,抬起头来望了秋谷一望,樱唇红绽,笑口微开,低低的对秋谷说道:“格末谢谢耐。倪吃仔格碗把势饭,也叫呒说法。”秋谷和他取笑道:“我记得那一回,你和张书玉两个人吃醋,也在这个地方。一班马夫七手八脚的把你团团围住,还是我挺身出来和你们两个人讲和,方才了事。”说到这里,金小宝脸又一红,顺手把秋谷拉了一把道:“耐闲话讲明白仔哩,格是张书玉来搭倪吃醋呀!倪几时搭俚吃过啥格醋介?”秋谷笑道:“就算我说错了,是张书玉和你吃醋。如今又在这里和你同祝小春讲和,一连和你当了两次苦差,你该应怎样的谢谢我呢?”

金小宝听了,不觉低头一笑,也不开口,把手去掠着头上的云鬟。秋谷再问一遍,小宝方才格格的笑道:“耐搭贡大少是好朋友呀!”秋谷笑道:“我和春树虽然是要好朋友,但是春树是我荐给你的。两下比较起来,我的资格又要比春树老些。”

小宝沉吟了一回方才说道:“只怕呒拨实梗格规矩嗫。”秋谷道:“堂子里头什么规矩不规矩。真讲规矩的人,不到堂子里头去顽了。”小宝没有话说,只看着秋谷微笑。秋谷见小宝薄施脂粉,丰韵天然,不觉心上狠有些眷恋的意思。忽然转过念头来想道:小宝是春树的相好,我和春树的交情比不得别人,到底有些不便。正想着,忽听得小宝讲道:“倪转去罢,辰光勿早哉。”秋谷听了,抬起头来看时,果然霞彩满天,斜阳欲没,四围螟色,一片苍烟。便也同着金小宝转进安垲第来。

只见范彩霞同着陈海秋也来了,坐在辛修甫一班人一起。秋谷见了范彩霞,朝他点一点头,便问陈海秋道:“你们为什么到这个时候才来?”陈海秋道:“我正要来的时候,刚刚有个朋友找到东尚仁去和我讲话,直到这个时候方得脱身。”说罢,陆丽娟已经立起身来,对着秋谷说道:“倪去罢。”这个时候,金小宝悄悄的拉一拉秋谷的衣服,附耳说道:“耐一淘到倪搭去。”秋谷便对陆丽娟说了,叫他自己坐车回去。陆丽娟听了,未免有些不愿意,勉强答应。秋谷便同着金小宝要走。

辛修甫叫住他道:“等回儿请你在西安坊吃酒。你有别处应酬没有?”陈海秋也要请秋谷和修甫在范彩霞院中吃酒。秋谷想了一想道:“今天虽然有两个人约我吃酒,但这两个人也不是什么知己朋友,就不去也不要紧。或者我跑到那里,略略的坐一回儿,就到你们那边也好。”辛修甫、陈海秋听了,都点头答应。

秋谷便同着金小宝走出大洋房门口,叫马夫把马车放过来。秋谷因为自己坐的是亨斯美两轮车,便叫金小宝把马车换给陆丽娟坐。金小宝的大姐阿囡,便和陆丽娟一车。秋谷自己拉缰,和小宝同坐。陆丽娟满心委屈,却又不便说什么,只着着实实的钉了秋谷一眼。秋谷见了,觉得今天的事情有些对他不起,想着也顾不得许多,只得由他。正是:

双星无那,银河七夕之槎;一笑相逢,洛浦飞仙之影。

不知后事如何,且看下回交代。





第一百零六回 危崖勒马虚度清宵 宝镜孤鸾枉辜良夜





且说章秋谷同着金小宝上了马车,秋谷把丝缰一带,从老洋房弹子房那一面大宽转兜过来,马车路过老洋房门外,只见老洋房门口站着一个淡妆素服的丽人,头上打着一条油松大辫,发光可鉴,膏沐照人。身上一身本色单罗衫裤,胸前簪着一朵红花;下面的裤管高高吊起,露出一双尖尖瘦瘦的金莲,穿着一双大红缎绣花弓鞋,真个是一搦凌波不盈三寸。那一身打扮好像是个髦儿戏班里头的人。见了章秋谷自己拉缰过去,便嫣然微笑,送了一个眼风。秋谷的马车飞一般的过去,只觉得两下的眼睛一错,眼睛里头霍的电光一闪,秋谷的马车早已过去了三五丈远的地方。

依着秋谷的心上,要想把马车再兜转老洋房门口,细细的认他一认,怎奈那匹马四蹄飞动,就像星飞电卷的一般,一时勒他不住。更兼那边的地方不大,马车一时间转不过身来。又有一个金小宝同在车上,似乎觉得不好意思,只得由着那匹马的性儿望前跑去。心上却十分惆怅,不由得问着金小宝道:“方才老洋房门口站着的一个女子,好像也是个倌人,你认得这个人不认得?”金小宝听了微微含笑,对着秋谷摇一摇头。秋谷不知不觉的说出一句道:“可惜。”小宝含笑问道:“耐可惜啥物事呀?”秋谷道:“方才那个女子,模样儿长得狠不错。可惜你又不认得他。”

金小宝斜着一双俊眼笑问道:“耐格人阿,真真是苏州人打话,叫声化子吃死蟹──只只好。”秋谷听了不觉也好笑起来。

两个人一路说着话儿,不知不觉的马车已经到了惠秀里门口,秋谷扶着金小宝跨下车来。小宝要留秋谷进去坐一回儿,秋谷也无可不可的,跟着小宝进房,坐下谈了一回。秋谷要走,小宝不肯放他道:“倪两家头难得碰头,刚刚坐得一歇,啥咦要去哉呀?”

秋谷本来心上是狠爱小宝的,但是秋谷的性情,喜欢这个倌人,却不是一定要和他落水,不过大家有些意思罢了。如今见了金小宝这样苦留。便道:“既然如此,我这会儿还要到别处去应酬一下,回来我到西安坊和东尚仁的时候,我们同去何如?”

金小宝道:“俚笃咦朆请倪,同仔耐去,算啥样式呀!”秋谷道:“那怕什么。你和辛修甫、陈海秋认得也不是一天了,就算个闯席的客人何妨。”金小宝想了一想,方才应允。又叮嘱秋谷道:“耐去仔要就来格嗫。”秋谷道:“这个自然。”说着便立起身来,走出门外,跳上马车,赶到东荟芳黄菊英家,是一个什么吴淞钓捐局委员姓郑的朋友请他的。秋谷只略略的坐了一回,又到别处去应酬了一转,惦记着小宝等他,便辞了主人,径到小宝院中来。

只见小宝换了一身男妆衣服,穿着一件湖色单罗长衫,单纱一字襟半臂,胸前一个花球香风扑鼻,面上的脂粉一齐洗掉,梳了一条大辫,脚下也换了一双夹纱衬金的小靴,越显得水眼山眉,雪肤花貌。见了秋谷便笑道:“耐看倪改仔男妆阿好?”

秋谷自头至脚细细打量了一番,口中赞道:“真个是巫山神女、姑射仙人,可惜我没有这般福分。”小宝听了,把秋谷打了一下道:“勿要瞎三话四哉,倪去罢。”

说着便移步下楼,同着秋谷坐上马车,只转一个弯,便到了西安坊门口。秋谷同着小宝一同进去。

辛修甫一眼见秋谷同了一个男子进来,没有看得清楚,只道是秋谷同来的朋友。

立起身来一看,方才知道就是金小宝改的男妆。金小宝见了修甫,却恭恭敬敬的打了一个拱。修甫大笑起来,口中说道:“今天小宝先生居然肯赏我的光,实在意想不到!”秋谷坐下来,便问局票写了没有。修甫道:“都写好了,只等你一个人。”

秋谷拿过来看了一看,见自家名下,仍旧是写的陆丽娟和梁绿珠,便点一点头,交给娘姨发出去。修甫见客已齐了,便叫起手巾,大家入席。依着辛修甫,要请金小宝坐首席,小宝不肯,和章秋谷并肩坐了。不一会,叫的局一个个陆续到来,别人都没有什么,只有陆丽娟见金小宝和秋谷并肩执手,密密切切的讲话,心上有些醋意,低头不语。梁绿珠和秋谷没有落过相好,心上倒没有什么。这一席大家因为还要翻台到东尚仁去,便略略吃些,都不尽量。上过了头四道,大家一哄的都到东尚仁范彩霞院中来,又闹了一回,已经十二点钟了。陆丽娟走的时候,悄悄的问秋谷道:“耐晏歇点阿来?”秋谷沉吟道:“来的。”陆丽娟道:“格末倪来浪等耐,勿要绰烂污嗫!”秋谷点一点头。

等着席散之后,秋谷同金小宝依旧双双回去。到了小宝院中,小宝见秋谷有些醉意,便自己开了一瓶荷兰水给秋谷吃了,方才两个人促膝深谈。小宝便把自己本来不愿嫁人的意思和这一番上了牛幼康圈套的原因,细细的和秋谷讲了一遍,叹一口气道:“上海格客人总归靠勿住。就像贡大少末,故歇看看好像呒啥,慢慢里也勿知到底那哼。”说着不觉有些凄楚起来,眼角里头盈盈的含着一汪珠泪。秋谷深深款款的安慰一番,看着小宝的样儿似离似合,眉目含情,便握着小宝的手道:“我们两个人……”说到这里停了一停,又叹一口气道:“只好做个朋友罢!”小宝听了,眼波溶溶的看着秋谷,看了一回不觉也长叹一声,低下头去。秋谷见了这般模样,觉得一个心七上八下的不妥当,好像要直跳出腔子外来。暗想:若是小宝一定不肯放我走时,我也只得应酬一遭的了。小宝挨了一会,抬起头来对着秋谷说道:“二少格闲话勿错,倪也勿好……”说到这里,那下半句竟说不出来。秋谷咬一咬牙齿,硬着心肠道:“时候不早,我要回去了。”小宝也不开口,只点一点头。

秋谷正要走时,小宝又道:“耐慢慢交走。”秋谷便立定了,等他说出什么来。小宝停了一停道:“耐身上阿冷?”秋谷摇一摇头,就走出房门。小宝也送出来。秋谷对他摆手,叫他进去。小宝不语,一直送下扶梯,走到门口,看着秋谷上了马车,方才进去。

秋谷回到新马路公馆里头,差不多已经天亮。陈文仙还一个人坐着等他,见秋谷回来,便立起来打了一个呵欠,笑着说道:“我晓得你今天晚上一定回家,所以没有睡觉。”秋谷见桌子上排着一本牙牌神数,又有一付牙牌放在桌上,便道:“你在这里起牙牌数么?”文仙笑道:“等了你半天,你不回来,一时气闷,借着这个消遣,也不知灵与不灵。”秋谷道:“这些事情本来是骗骗小孩子的,那里会灵?”文仙道:“你不要不信。世上鬼神的事情都是有的。”秋谷听了,知道文仙妇女性质,迷信甚深,一时劝化不过来,便也只得由他。只问一句道:“我不在家,你冷静不冷静?”文仙笑道:“你回来就不冷静了。”秋谷道:“却是对你不起。

我在外面这样的打茶围、吃花酒,却要累着你深更半夜守在这里。其实我们如今是自己人,可以不必这个样儿。“文仙道:”你既然知道我们是自己人,你又何必和我这般的客气呢?“秋谷听了,没有话说,便也微微一笑,相携就寝。一夜无话,不提。

过了一晚,章秋谷到九点钟方才起来,便有许多朋友都来贺节。秋谷倒应酬了一回,免不得也坐着马车到各处去走了一转。猛然想起昨天答应陆丽娟到他院中去的,便吩咐马夫一直放到久安里门口。秋谷下了车,径到陆丽娟院中来。

陆丽娟见了秋谷,似笑非笑的说道:“阿呀,章二少贵人勿踏贱地,那哼跑到??倪搭小地方来哉?勿要踏错仔门堂子哩!”说着便别过头去。秋谷见丽娟脂粉不施,玉容寂寞,知道他为着昨天金小宝的事情不快,便抢步上去,拉着陆丽娟的手道:“昨天晚上对不起,累你空等一回。不知怎样的,糊里糊涂就忘了这件事儿。”

丽娟冷笑道:“本来倪自家勿好,倪搭实梗格小地方,陆里请得动耐格位二少!”

秋谷道:“你不要生气,我和你陪个礼儿好不好?”说着就对着陆丽娟打了一拱。

陆丽娟别转了头,只当没有看见的一般,口中说道:“勿敢当。倪也朆生着格付骨头。”说罢,停了一停又道:“倪看耐昨日仔直头有点浑淘淘哉!拨别人家迷昏仔,陆里还记得到倪搭来!”秋谷道:“你不要疑心小宝和我有什么相好。我和他两个人都是干干净净的。那里有什么别的事情!况且小宝的相好客人姓贡的,是我最要好的朋友。我也不肯做这样的事儿。”陆丽娟听了那里肯信,冷笑道:“耐格号闲话只好去骗骗三岁小干仵。耐去搭金小宝那哼那哼,勿关得倪啥事;倪也勿好来管仔耐,叫耐勿要做哩!不过,耐就搭倪讲明白仔,也呒啥希奇。啥事体定规要瞒牢仔倪,勿搭倪说?耐倒搭倪讲讲格个道理看。”秋谷看了陆丽娟娇嗔满面,情不自禁,便婉婉曲曲的对他说道:“老实说,我就是和金小宝落了相好,我也不必瞒你。

但是的没有这件事儿。你只想我和你认得了差不多也有一年,那一件事儿是瞒过你的?你不信,只顾去问辛修甫、陈海秋他们一班人,究竟可有这件事情?“陆丽娟听了,还有些似信不信的。秋谷又去安慰了他一番。

坐了一回,忽然又想起昨日在张园老洋房门口遇见的那个人来,想要想个法子去找他。盘算了一回,想着那一身打扮,一定是个髦儿戏班里头的人。只要今天再到张园去一趟,到楼上去看髦儿戏,一定找得着的。想罢,便对陆丽娟道:“我还有些事情,去一去就来。”陆丽娟道:“格末耐格双台几点钟来吃呀?”秋谷想了一想道:“今天端午,朋友们的台面很多,就晚上十点钟罢。”陆丽娟听了点一点头。秋谷便回到自家公馆里头,和陈文仙说了,要同他到张园去,文仙欣然答应。

略略的梳掠一回,换了衣服,同着秋谷直到张园来。正是:

看花载酒,十年杜牧之狂;对影闻声,一枕西楼之梦。

未知章秋谷到了张园,如何去找寻那个女子,且看下回交代。





第一百零七回 游张园初看髦儿戏 访萧郎又遇意中人





只说章秋谷同着陈文仙到了张园,只到安垲第去转了一转,便要到海天胜处去看髦儿戏。陈文仙道:“这个地方的髦儿戏没有什么看头的,我们何必去看他?”

秋谷也不瞒他,竟是直言拜上的,把昨天的事儿和陈文仙说了一遍。文仙听了只是微笑,也不言语。两个人同到海天胜处,走进戏场,拣了一张桌子,并肩坐下。

秋谷刚刚坐定,便抬起头往那戏场上看时,只见场上正在那里做《探亲相骂》的一出,那扮城里亲家的花旦,叫做玉兰花,却也生得眉目玲珑,身材娇小,狠有几分可爱,却不是昨天见过的那一个。秋谷留心看了多时,总不见他的影儿。秋谷心上有些疑惑道:昨天看他的打扮,明明是髦儿戏班里头的人,怎么今天竟没有这个人的影儿?正想着,忽然觉得陈文仙把自己衣服轻轻一扯。秋谷回过头来问时,文仙对着秋谷把嘴往东边一努,悄悄的说道:“你看那边一个,是不是你昨天遇见的?”秋谷顺着文仙指的一方面看将过去,只见离自己的坐位不远,坐着一个丽人,明眸皓齿,宝靥云鬟;小蛮杨柳之腰,攀素樱桃之口。正在那里和同坐的一个少妇交头接耳的,不知说些什么。虽然不是昨日的那一个人,却也彼此相衡,不相上下。

秋谷见了一回,把一双眼睛不住的周围上下仔仔细细的打量他。正看得高兴,忽然那女子回过头来,和章秋谷正打了一个照面。见了秋谷这般模样,不觉有些不好意思起来,红上眉梢,春融眼角,低下头去微微一笑。章秋谷见了,虽然明知道这一笑不见得就是有什么吊膀子的意思,却由不得心上的一缕情丝便有些摇曳起来。

这个时候,刚刚一个人在外面大踏步走进来,见了章秋谷呆呆的坐在那里,便抢步上去,伸出一只手来在秋谷肩头上一拍。秋谷正在那里出神,被他这一拍,猛然吃了一惊。直立起来看时,原来就是张园的总经理人,姓李号伯惠。秋谷同他向来认得,却没有什么大交情,便随意和他谈了几句。李伯惠就在秋谷后面一张凳子上坐了下来。秋谷问他髦儿戏班里头的花旦是那一个?李伯惠道:“就是方才做《探亲相骂》的玉兰花。还有一个叫做月月仙,却面貌生得狠平常,只好算个配角罢了。”章秋谷听了,便把昨日在老洋房门口遇见那个女子的事情一一和李伯惠说了。

又把他的面貌打扮,细细的和李伯惠讲过一遍,问李伯惠可认得这样的一个人?李伯惠听了想了一想,也说不认得。秋谷听了,心上十分惆怅起来,觉得咫尺山河,玉人何处。正低着个头,细细的心上在那里摹拟那个女子的体态,忽地听得那坐在左首的女子对着同坐的少妇口中说道:“我们回去罢!这个戏没有什么看头。”这两句话儿莺声呖呖,直送到章秋谷耳朵里头来。章秋谷听了不觉心中一动,早见这个女子款款的立起身来,同着那个少妇香飘拂的一步一步走过来,恰恰在章秋谷面前经过。起先隔着一张桌子,秋谷看得还未十分清楚,又不好意思走过去打量他,如今见他从自己身边走过,自然要细细的领略他的丰神。只见他俊眼流波,长眉却月;春云作态,秋水为神。那一种清华秀曼的丰姿,隐隐的都在眉目中间现出。更兼秾纤合度,修短得中,步步金莲,亭亭倩影,慢慢的走过来。走到章秋谷面前,不由得偷转秋波,把章秋谷看了一眼。那里知道章秋谷正在那里目不转睛的看他,两下的眼光刚刚的碰一个着。那女子见了章秋谷也在看他,连忙别过头去,装作没有理会的样儿,急急走了过去。

章秋谷到了这个时候,好像被那女子眼睛里头的电气吸了过去的一般,不管三七二十一,立起身来把陈文仙拉了一拉,立时立刻的跟在那女子后面往外便走。那女子一面在前走着,却也频频回过头来看看后面。一直走到安垲第门外,那女子便立定了脚步,觉得已经有些娇喘微微的样儿,把手掠着头上的鬓发,略略的立了一回,便叫了一声:“我们的马车在那里?”叫着,早见一个马夫跑过去说了几句话儿,便飞一般的向前跑去。不多时早拉过一辆皮篷车来,那个女子和着那个少妇两个人手挽手儿的一同上去。

这个当儿,章秋谷站在一旁,早已将自己的马车叫了过来,同着陈文仙坐上马车,把丝缰一抖,紧紧的跟着前面的皮篷马车跑出张园外。只见前面那辆马车走不多时,忽地带转马头,把丝缰略略一偏,竟望刺斜里爱文义路一带直跑过去。秋谷也拉马车紧紧相随。前后两辆马车,八个马蹄,好似追风逐电一般。

秋谷见这一条路上地人甚少,便使一个手段,把手内的丝缰的往前提了一提,拔出鞭子来,在马背上微微的一掠,那马放开四蹄,好似那羽箭离弦,弹丸脱手,一霎时早赶过皮篷的马车的前面。跑不上二三十丈地方,又把马头带转来,在皮篷马车的右边直擦过去。只见那女子坐在马车里面,对着秋谷微微展笑,后启嫣然。

两下的马车霍的电光一闪,早已两边错过。章秋谷等他的马车已经过去,依旧勒转马车,缓缓的跟在后面,一直钉到新马路人寿里门外,前面的马车方才停住。章秋谷也把马车停在一旁,吩咐陈文仙在车上暂坐一回,自己跳下车来,看那女子同着那少妇一同下了马车,走进弄内第三家,门口贴着个“平江伍公馆”的几个字儿。

那女子走到大门里面方才回过头来,看着章秋谷还一个人跟在后面,不觉“嗤”的一笑。听得“呀”的一声,两扇大门已经关上,把一个章秋谷关在门外。真个是阳台春杳,巫峡云封;苍茫银汉之波,惆怅蓝桥之路。一个人立在大门外面,细细的认了一认,便回转身来,同着陈文仙一同回去。

陈文仙见了秋谷这般模样,心上未免有些醋意,却不便说出来。秋谷只在自己公馆里头坐了一坐,想着今天端午,不但有许多朋友请他吃酒,就是自己也有两处台面,恐怕迟了来不及,忙忙的又跑了出来,各处应酬了一回,方才到陆丽娟院中吃了一个双台,直闹到两点多种方才散席。

陆丽娟要留秋谷住在院中,秋谷执意不肯。陆丽娟见留不住,心上就不愿意起来,把秋谷打了一下道:“耐要去末,去末哉!呒啥人来浪拉牢仔耐。倪格搭小地方,陆里放得落耐格位大人!”秋谷听了,还没有说出什么来,阿金妹早接过来说道:“今朝节浪,唔笃两家头自然要双双对对、团团圆圆末好畹。”秋谷笑道:“不瞒你们说,今天端午,我们姨太太一定在公馆里头等我回去,所以我不肯住在这里。”一句话还没有说完,陆丽娟抢步过来,推着秋谷的背道:“耐豪燥点搭倪请出去,好去陪唔笃格姨太太!晏歇点姨太太动起气来,勿要害耐吃生活!”

章秋谷见陆丽娟粉面生红,蛾眉微竖,认真动起气来,只得回转身来,拉着陆丽娟的手并肩坐下,对他笑道:“你不要生气。我讲一个道理给你听,你就明白了。

我章秋谷顶天立地,自然不是个怕姨太太的人。但是既然把他娶到家中,自然要处处和他同心合意方才是个道理。我今天出门的时候已经和他说过,今天一定回来。

如今不回去,自然没有什么要紧,但何苦哄他一个人在家里冷冷清清的坐等一夜呢!

我今天不肯冷落了姨太太,住在你的院中;到了别的时候就也不肯冷落了你,住在别人院内。如今我不肯辜负姨太太,别的时候就不肯辜负你!要是今天我听了你的话,住在这里,丢掉了姨太太;难保到了那个时候,也听了别人的话儿住在别处,丢掉了你。你只要细细想一想我的话儿,自然气就平了。“这一席话,说得陆丽娟一场烈火不知化到那里去了,低着头一言不发。秋谷见了,便又和他并倚香肩,低偎檀口的问道:”我的话儿可是不是?“陆丽娟听了一时转不过口来,只冷冷的回答道:”算耐会说。一只嘴翻来覆去,总归耐一干仔格闲话。“说着不觉横波一笑,立起身来把秋谷推开,口中说道:”耐转去罢,明朝要来格虐!“秋谷见了,知道他已经心平气和的了,便也趁势说了几句闲话,搭讪着走了。

回到公馆,见陈文仙一个人在灯下支颐独坐,好像心上在那里想什么事儿。秋谷笑着问他想些什么。文仙道:“我在这里想今天张园里头的情景。”秋谷听了,心上已经有几分明白他的意思,便拥着陈文仙在大床沿上坐下,默然相对;文仙也不开口。停了一回,秋谷忽然问道:“我遇着的妇女,也不知多多少少,没有一个不爱吃醋的人。怎么你在我身上,竟没有一些儿吃醋的意思,这是什么缘故?”文仙听了微微笑道:“老实和你说,天下但凡是个女子,没有个不吃醋的人。就是我自从嫁你之后,见你还是那般沾花惹草的性情,我心上也不免有些不快。但是我和你相处几年,狠知道你的性情;虽然外面这般模样,心上却还有些把握;不是那般不分好歹、不知黑白的人。只要你有了别人,不要得新忘故也就是了。”说着不觉微微的叹一口气。秋谷听着陈文仙这几句话儿说得楚楚可怜,觉得心上好生抱歉,跳起身来对着陈文仙打了一拱道:“总算我一生幸福,娶着了你这样的一个人!”

正是:

夜阑灯炮,罗帏之私语轻轻;倚影怜声,卧后之清宵细细。

不知后事如何,且看下回交代。





第一百零八回 情切切密意慰檀郎 意绵绵深情回倩女





却说章秋谷对陈文仙打了一拱,陈文仙连忙立起身来,背过脸去,口中说道:“为什么平空的又要打起拱来?”秋谷笑道:“我自从把你娶到家中之后,还是这样的沾花惹草,到处留情,你却从没有和我闹过一回,争过一句。仔细想起来,觉得狠有些对你不起。所以今天朝你打一个拱,总算和你陪个不是。”文仙听了也笑道:“自己人,何必还要这般客气?打拱作揖的,不要折了我的福分。”章秋谷道:“若要论起理来,你的嫁我,既没有要我的钱,又不是贪我的势。我娶着了你这样的一个人,总算心满意足,没有什么不合,不应该再在外面这般胡闹。但是我天生成是这般的性情,实在无可如何,你也只好将就一点的了。”陈文仙道:“我也知道你性情如此,和你争论也是不中用的,倒反大家存了意见。只要你把我这个人长长的放在心上,不要到了那个时候忽然反面无情起来,也就是了。”秋谷道:“这个你只顾放心。我也不是这样负心薄幸的人物。难道我们认得了这几年,你还不知道我的为人不成?”文仙听了,斜着一双俊眼微微笑道:“我也知道不会这般薄幸,所以凭你在外面这样混闹,没有什么不放心。如若不然,老实说我也不至于这般冒失!……”文仙说到这里顿了一顿,秋谷接着说道:“可是不嫁我么?”文仙含笑点一点头。

秋谷又道:“我家里虽然现有正室,我待他却很平常,没有和你这般熨贴。但是我在你面上,虽然别的没有什么,却免不得东去吊个膀子,西去做个倌人,自己想起来狠觉得有些过意不去。”陈文仙“嗤”的一笑道:“算了罢,不用灌米汤了。”

秋谷正色道:“我向来不说假话的。况且在你面前说假话做什么?不过我想起来,你当初嫁我,我没有出一个大钱的身价,一古脑儿只和你付了几百块钱的帐,又委屈你做我的姨太太……”秋谷正还要说下去,文仙秋波澄澄的看着秋谷说道:“你当真的过意不去么?”秋谷道:“自然当真过意不去。”文仙道:“你既然心上过意不去,天长地久,以后的日子多得狠。只要你放在心里头,慢慢的来就是了。”

秋谷听了,拉着他的手笑道:“不用慢慢的来,今天就要给你赔礼。”文仙面上不觉红了一红道:“赔礼是不敢当的,你去和陆丽娟赔礼罢。”秋谷哈哈的笑道:“你好没良心!刚才在陆丽娟那里,费了无数的唇舌,方才肯放我回来。你还要说这样的话儿!”文仙听了,不懂秋谷说的什么,连忙问时,秋谷便把方才陆丽娟留他在院中住夜的事情,同着自己开导的话儿,细细的告诉了陈文仙。文仙听了,虽然不说什么,心上却十分感激。

正在这个时候,章秋谷忽然觉得窗外一阵凉风直逼进来,打了一个寒噤。抬起头来看时,只见那几扇玻璃窗上已经隐隐的透出晓光来。秋谷道:“我们只顾讲话,连天明都不知道。”文仙到了这个时候,身上也觉得有些翠袖生凉,罗衣风冷,便也同着秋谷上床就寝。这两个人一个是离支侧挺,栽成婪尾之春;一个是桃李无言,嫁得金龟之婿。镜盟衫誓,玉软香温;帏中之小玉频呼,枕上之深钗欲堕。十分欢乐,十分熨贴,就十分的恩爱缠绵。这些琐事,在下做书的也不必去讲他。

只说章秋谷自从在张园见过那个女子之后,心上觉得十分的放他不下,自己亲自到人寿里去打听了好几回,方才知道那天看见的就是平江伍公馆里头的小姐,那同他坐在一起的少妇便是这位小姐的舅母。这位小姐的父亲叫做伍圭甫,本来是苏州人,在上海南市开了一家糖栈。娶妻周氏,生了一男一女,得病死了。伍圭甫有一个内弟,死的时候年纪很轻,遗下一个寡妻,无儿无女,便住在伍圭甫家里,靠着这位姑奶奶度日。自从周氏死了之后,伍圭甫不知怎样的勾勾搭搭,竟和这位舅太太勾搭上了,隔了一两年,伍圭甫又在堂子里头娶了一个倌人做姨太太。娶到家头没有一个月,就和这位舅太太吃起醋来,两下闹了个天翻地覆。伍圭甫恐怕传出去风声不雅,便把姨太太搬到南市去住。把自己的一个女儿、一个儿子,托给舅太太照应,另外在人寿里租了几幢房子,用了一个厨子,一个梳头娘姨,还有小姐的妈妈也跟着住在一处。伍圭甫一个月里头也回来住十多天,把这位舅太太竟作了他的外室。

这位小姐长到十七岁上便出落得态度清华,丰神婀娜皎若中秋之月,娇如解语之花。一班少年子弟见了伍小姐这般丰貌,一个个好像失了魂魄的一般,免不得一个个都要和他挤眉弄眼,卖些弄吊膀子的手段。无奈这位伍小姐虽然破瓜年纪,情窦已开,却向来不大出门的,那里知道什么吊膀子不吊膀子。更兼看着这一班油头滑脑的少年,眼睛里头也看他们不上。

这位舅太太虽然已经年过三旬,却还狠喜欢抹粉涂脂,画眉掠鬓;衣妆时世,体格风流,看上去也不过像个二十三四的样儿。时常也同着这位伍小姐出去坐坐马车,游游张园。也有时到戏馆里看看夜戏。这位舅太太十分高兴,伍小姐却是随随便便的。

这一日也是天缘凑巧,刚刚在张园遇着了章秋谷。伍小姐见了秋谷长身玉立,白面丰颐,顾盼非常,风华出众。觉得平日之间眼中从没有见过这般人物,不觉肚子里头暗暗的喝彩。又见秋谷同着陈文仙两个人在一起,好似那珊瑚连理,玉树交枝;一个丰彩照人,一个容光飞舞,合起来恰是一对儿,不相上下。伍小姐心上暗想道:这一对少年男女,也不知是那里来的?心上就也略略的动了一动,不免偷转秋波,着实的多看几眼。及至秋谷自己拉着马车,在他马车的前后左右兜了一个圈子,又连连的朝着伍小姐飞几个眼风,伍小姐是个绝世聪明的人,那有不领会的道理?不由得对着秋谷一笑。直到马车已经到了人寿里门口,伍小姐同舅太太差不多将要走进大门,回过头来,还看见章秋谷远远的跟在后面。伍小姐心上虽然明白,只说这个人有些痴气,却没有什么什么歪念。倒是这位舅太太见了章秋谷这样的一个人物,未免动了个怜才爱貌的心肠,心上觉得好生眷恋,对着伍小姐又说不出来。

这边的话权且按过一边。

只说章秋谷自从知道了这些消息,便一心一意要想做个跳粉墙的张君瑞,把一个好好的伍小姐就当做西厢待月的崔莺莺。无奈这里头没有个传书递柬的红娘,这件事儿那里弄得成功?一连在伍小姐家门外徘徊了几天,不要说没有见着伍小姐的面,就是伍小姐的声气也没有一些儿听见,找不出一个空儿。想要发一个狠丢掉了他,只当没有看见这一个人,无奈千思万想的,心上总放不下来。觉得自己的前后左右都有无数伍小姐的影儿团团围住,那里撇得开!自己心上诧异道:天下竟有这般奇事!我章秋谷平生看见的妇人女子也不知多少,就是和他一个样儿的也狠多,怎么我在别人面上从没有这样的痴心眷恋,独独的遇着了他就是这般模样,这是个什么道理?想了一回,也想不出个缘故来。又是这样的去守了几日,依然找不到一些门路,没奈何只得放过一边,无精打彩的在公馆里头过了几天,也不出去。

向来章秋谷到了夏间,差不多天天要坐马车到花园里头去顽的,如今心上有了这件事儿,只成日的坐在公馆里头,连大门都不出。陈海秋同陶观察等一班人也时时来邀他一同出去,秋谷心上不耐烦,只推有病不能出门。恹恹闷闷的过了几天,当真发寒发热的生起病来。陈文仙着了忙,又不便怎样苦苦的劝他,只得尽心服侍。

过了两三天,秋谷觉得好些,早上起来吃过一碗荷叶粥,和陈文仙讲些闲话。文仙趁势劝他道:“你一个男子汉,何苦为着这样没要紧的事情自己生起病来?你想老太太通共止生你一个儿子,要是知道你在这里生病,不知要怎样的着急呢!”秋谷听了悚然道:“你的说话委实不差,我也知道我这个单相思害得无谓,却不知怎样的心上总是放他不下,连我自己都不明白。”

正说着,只见一个十七八岁的女子走上楼来,穿着一身淡湖色洋纱衫裤,上身却衬着一件杨妃色汗衫。梳着一条乌光漆黑的油松大辫,一双天然脚穿着一双皮鞋,好像个女学生的打扮。倒生得眉清目秀,齿白唇红,一张圆圆的脸儿,不施脂粉,素净非常。手里头拿着一个筠篮,篮里头装着无数的鲜花,香风扑鼻。原来是卖花的苏州阿七。阿七走进房来,见了章秋谷,笑微微的叫了一声:“二少爷。”对着文仙道:“奶奶,今天要买些花不要?”文仙素性最爱花的,便拣了一个茉莉花球和一条茉莉花条,又拣了几剪珠兰,几剪白兰花。阿七便坐下来七搭八搭的和文仙扳谈,文仙却不甚理他。忽然蛾眉一皱,颊上的两个酒涡微微一动,便走近秋谷身旁附耳说了几句。秋谷登时喜上眉梢,连连点首。

文仙便走过去坐在一张美人榻上,招手叫阿七过来,问他道:“你在这里卖花,新马路一带公馆里头的花,一古脑儿都是你的是不是?”阿七道:“不错。这里新马路左近几个有名的大公馆,什么姨太太、少奶奶、小姐头上戴的花,都是我一个人送去的。有时自己园里出的花还不够分派。”这一来有分教:

蜂媒蝶使,偷来御苑之春;倚玉偎香,销尽温柔之福。

不知陈文仙和阿七说些什么,且看下回便知分晓。





第一百零九回 梦巫山良宵圆好事 忆倾城名士苦相思





却说陈文仙听了那卖花阿七的话儿心中大喜,便又问道:“人寿里有一家伍公馆,你可知道么?”阿七笑道:“他家大小姐,是我买花的长主客,天天带的花都是我送去的。”文仙听了再要问时,章秋谷坐在床上连忙和他递个眼风,陈文仙便不开口,故意做着无心的样儿,和阿七说了一阵闲话,方才付了一花钱,打发他去了。

文仙见阿七已经走了,便向秋谷笑道:“如今走内线的人倒弄了一个在这里了。

但不知这条内线怎样的一个走法?“秋谷听了默然不语。文仙忽然笑道:”那个时候,你常常自家夸口说:兵来将挡,水来土掩,天下没有做不到的事情。只有人生了笪病要死,和穷人没有钱用,这两件事儿想不出法儿。除此之外,凭你再有天大的事情,你也有对付的法儿。怎么遇着了这样的一件小事,就把你难到这般田地,甚至生起病来?今天这个主意,还是我和你想出来的,这是个什么道理呢?“秋谷听了也笑道:”这个里头另外有个道理,并不是我想不出法儿。我自从那一天和他相见以后,想来想去只有买通婢仆的一个法儿。无奈我又有一般脾气,那一班低三下四的人我又不肯陪着小心和他讲话;心上总想凭着我的一身本事、全套工夫,或者不用别人帮衬,竟夫成就也未可知。那里知道提心吊胆的候了好几天,钻不着一些门路。如今说不得,只好请个帮手帮帮忙的了。“说罢自己心上算计了一回,又和文仙商量一会,定了主意。

等了一天,等得阿七来了,秋谷便和他夹七夹八的讲些闲话,问他家里头还有什么人。阿七叹一口气道:“我家里还有一个父亲,一个哥哥。母亲是早已死掉了。

父亲同哥哥两个人都是坐在家里不会挣钱的,一天倒要吃半块钱鸦片烟,只靠着我一个人卖花度日。“秋谷又问他卖花的钱可够用不够用,阿七道:”平常的时候也还勉强敷衍得过。若是天气不好,没有什么人要买花,就要过不去了。“秋谷笑道:”卖花的利息是狠好的,你何不租些空地,开一个大大的花局子呢?“阿七也笑道:”二少爷说得这般容易。我们做这个卖花的生意,连自己的用度还有时候顾不来,那里有这许多钱来开什么花局子!“秋谷道:”这个不妨。我有一件事情要托你和我帮一个忙,只要你肯答应,我借一百块钱给你做本钱好不好?“阿七只认得是秋谷有心和他取笑,面上红道:”二少爷不要取笑,我们这样的人,那里会和二少爷帮忙?“秋谷趁势抢步过去,握住他的手道:”我不是和你取笑,实在有件事儿要和你商量。“

阿七见秋谷握他的手,发了急道:“二少爷不要这样,给奶奶看了,什么意思!”

秋谷笑道:“奶奶早已走到楼下去了,你不用这般胆小。”要七听了,抬起头来看时,果然陈文仙不知走到那里去了,房里头只剩下他和秋谷两个人。阿七不觉满面通红,心中乱跳,想要洒脱了手跑下楼去。怎奈章秋谷天生神力,紧紧的握住了他的手怎肯放松?阿七挣了一回,不得脱身,只得红着脸央告秋谷道:“二少爷,多谢你放我下去罢。等一回有人走进来,看了这般模样,叫我这个脸放到那里去?”

秋谷道:“你只顾放心,包管没有一个人进来。”阿七和秋谷扭了一回,心旌大动,面上一阵一阵的红云直升起来。秋谷是个花丛老手,这些门径那有不知道的理?霎时间并蒂花开,鸳鸯梦稳。云痴雨殢,未妨瑶岛之春;李代桃僵,且疗相如之渴。

过了一回,秋谷正和阿七款款深深的讲话,忽见门帘一启,陈文仙笑盈盈的移步进来,对着秋谷和阿七笑道:“恭喜,恭喜!”这一下只把个阿七羞得红云满面,坐立不安;背过脸去,恨不得地上生个大洞好让他钻了下去。文仙款步过去,挽了他的手,拉他一同坐下,笑道:“你不要这般怕羞。上海滩上这样的事情狠多,不是你一个,算不得什么希奇。”秋谷也道:“我们这位奶奶不比别人,不要说是醋,连酱油都不吃的。”文仙瞅了秋谷一眼,又宛宛转转的把阿七安慰一番。阿七只是低着个头,再也抬不起来。

原来这个阿七本来是个有名的私货,借着卖花做个名目,在几家公馆里头直出直人,带着勾搭些少年子弟,做那不要本钱的生意。这一班少年见阿七生得体态轻盈,性情流动,便起了他一个绰号,叫做“桃花阿七”。秋谷素来知道他的名气,狠有些想拉拢他。如今借着这件伍小姐的事儿,一举两得,把这个卖花女子当作个窃玉偷香的青鸟、传消递息的红娘。阿七虽然入了秋谷的网罗,却那里知道秋谷的这一般意思?

闲话休提,只说章秋谷和陈文仙两个人,你吹我唱的把阿七哄了一番,好似骗小孩子的一般,渐渐的把个阿七哄得抬起头来,却依然还是满面含羞,一言不发。

停了一回,方才羞羞涩涩的对着文仙讲道:“奶奶刚才到那里去的?我上了二少爷的当了!”一句话刚刚说出,面上又红起来。陈文仙又百般的寻着话儿去应酬他。

阿七到了这个时候,也只得老着面皮,讪讪的和文仙坐在一起。坐了一回,阿七起身要走。秋谷拿出一张五十块钱的钞票来给他,阿七假意不受。文仙勉强和他放在衣袋里头。看着他下楼去了,回过身来蛾眉半蹙,星眼横斜,似笑非笑的看着章秋谷说道:“别的且不必说他,这床枕席便怎么样?”秋谷笑道:“我知道你这般性格,没有糟蹋你的大床。”文仙摇着头道:“我不信,那有这般干净。”秋谷道:“你为我这般迁就,我心上已经二十四分的感激,那里还忍心哄你!”说着便对文仙做了一个手势,文仙方才信了。

自此以后,阿七一连两天不来。急得个章秋谷叫了自己的包车夫去寻他。去了多时,方才寻着了,一同回来。阿七走上楼来,觉得有些不好意思,趑趑趄趄的走进房门,见了章秋谷和陈文仙两个都笑哈哈的看着他,登时脸上又红起来。秋谷叫他坐下,和他讲些闲话,趁势问问伍小姐家里头的事情。阿七道:“伍小姐家我有二十多天不去了。听说他家老爷病重得狠,伍小姐和舅太太都到南市去看他,就住在那边公馆里。这个时候还不知他们老爷的病怎么样呢。”秋谷听了,心上恍然,方才明白那几天影响不见的缘故。便对阿七说道:“据你说来,伍小姐和你狠熟落的。我要托你想个法儿到伍小姐那里通个信息,不知你办得到办不到。如若事情成就,一定重重的谢你。”阿七听了连忙摇头道:“这个办不到的。这位伍小姐向来安分,从没有和人勾勾搭搭的事情。这个生意,免劳照顾了罢!”秋谷道:“你不要这样有心推托,我自然有个绝好的法儿在这里。”说着,便如此这般细细的教导了阿七一遍。阿七沉吟了一回方才说道:“我只好和你去误打误撞的。撞了一回,闹出事来却不与我相干的!”秋谷道:“这个自然。”便又取出几张钞票来交给他。

阿七接了钞票,欢欢喜喜的去了。

去了好半天,笑嘻嘻的回来对秋谷说道:“真正是你的运气!伍小姐刚刚由南市回来不多几天。我已经暗暗的和他奶娘王姆姆通了线索。你交给我的钞票,我止给他一张十块钱的,他已经千恩万谢的甚是喜欢。说他一个人不敢答应,要和舅太太商量,叫你好好的配一分礼去送给舅太太,只要他收了你的礼,这件事儿就有七八分指望。你今天赶紧去配好了礼物,交给我明天送去。”秋谷听了心中大喜,跳起身来,朝着阿七就是深深一揖。慌得阿七连忙躲开,却把一个纤指在自己脸上一连划了几划,做个羞他的样儿。

秋谷微微一笑,也不理会,只向陈文仙说道:“我想这一付入门的礼物,太重了恐怕事情办不到白花了钱,太轻了又不好看。我想去剪一件外国纱衣料,再搭一个嵌宝的戒指,且送去试他一试,看他怎样的一个说法。”文仙道:“衣料、戒指,我这里都有现成的,你拿去就是了,不必再去花什么钱。”秋谷摇一摇头道:“别的事儿拿你的东西还不必讲他,今天为着这件事情要拿你的东西,那有这般道理!

我自己心上也觉得过不去。还是花几个钱,到外边去买的为是。“陈文仙说道:”你说的通是痴话。我和你是什么人?你和我又是什么人?我的东西就是你的。你说出这样的话来,可不是笑话么!“说着不由分说,拿了一个首饰匣子出来,叫秋谷自家去拣。

阿七在旁边看了匣子里头的首饰,金珠照耀,翠玉玲珑,一样一样的光华四射,烨烨照人,不觉口中啧啧叹赏,心上却十分羡慕。只说:“奶奶真是福气!有了这许多首饰,就带一世也带不尽!”文仙听得他这般说法,便随手取了一个三钱重的金戒指替他带在手上。阿七还假意不肯受,谦逊了几句,也便谢了一声收了。

秋谷见文仙决意这样,也就拣了一个嵌红宝石的,约摸着也值四五十块钱。文仙还要叫秋谷拣一个嵌钻石的拿去,秋谷不肯道:“就是这一个已经够了,那里用得着这般贵华。”文仙方才把首饰匣子收了起来。秋谷又拣了一件玄色外国铁线纱的衣料,用红纸包得端端正正的,连着戒指匣子交给阿七。

阿七拿在手内,竟往伍公馆来。找着了王姆姆,暗暗的把这两件或西交给他。

王姆姆走到小姐房里,见这位小姐正横在一张大理榻上睡着,舅太太正在窗前闲坐。

王姆姆走近舅太太身旁低低的说道:“请舅太太到外面去说句话儿。”正是:

灵犀一点,未通鸩鸟之媒;彩凤双飞,讵有鸳鸯之翼?

未知后事如何,且听下回分解。





第一百一十回 传眉语喜遇秋娘 托微波暗通青鸟





只说舅太太听了王姆姆的话儿,不知什么事情,便跟着王姆姆走出房来,低低的问他什么事情。原来,这位舅太太少年守寡,独宿空房,每当那花朝月夕的良辰,不免总有些倒凤颠鸾的情思。更兼性情活泼,态度风流,到了那消遣不来的时候,也就不因不由的做些尴尬事情出来。这个奶妈,从小的时候便是舅太太娘家的丫环,后来荐到伍家做了奶妈,和舅太太十分合式。这些风流孽障的事情,也都是他一个人和舅太太传递消息。伍公馆里头,上上下下大大小小的人,没有一个人知道这些事情。如今阿七恰恰的找着了他,要他去走舅太太的门路,真正是合着了油瓶盖,刚刚正好。

闲话不提,只说王姆姆见了舅太太,把一件衣料和一个戒指都拿出来给舅太太看了一看,悄悄的说道:“这两件东西,有个姓章的送给舅太太的。”舅太太听了错会了意,只道是人家看上了他自己,要和他攀个相好。先把两件东西看了一看,觉得十分可爱,便道:“我和他向不相识,他为什么平空的送这两件东西?”王姆姆道:“自然他有事情要求你和他设法。你不要管他三七二十一,收了下来再说。”

舅太太故意说道:“他要求我有什么事情?要是办不到,怎么好混收人家的礼呢?”

王姆姆道:“自然是办得到的事情,你只顾收就是了。”

舅太太听了,低着头想了一回,便点一点头。又问王姆娟道:“这个人是何等样人,有多少年纪,你认得认不得?”王姆姆道:“我不认得这个人。只听说今年二十二岁,是个乡宦人家的少爷。据他自己说,端午那一天,在张园老洋房里头见过舅太太和大小姐的。”舅太太听了,知道就是那一天跟在后面的人,登时两颊生红,芳心暗动,对着王姆姆道:“不错,见是在张园见过一次的;但是他为什么无缘无故的、平空又会想到我的身上呢?”。

王姆姆听了这两句话儿,知道舅太太缠到隔壁去了,连忙说道:“他的意思想着我们这里大小姐,要请舅太太和他想个主意。”舅太太到了这个时候,方才知道他不是想的自己,一场欢喜扑了个空,不觉一团醋意直上心头,啐了王姆姆一口道:“你的讲话总是这样模模糊糊的,不分个皂白出来,叫人那里听得清楚!”王姆姆听了心中暗笑,也不去和他分辨,只问他一句道:“这件事儿舅太太看怎么样?”

舅太太皱着眉头道:“大小姐的性情你是向来知道的,那里肯做这样的事情!况且他父亲把他重托我们照应,我们怎么好把这些事儿来引诱他?情理上也讲不过去。

快些把这两件东西去送还了他,叫他不要胡闹。“

王姆姆听了默然不语。停了一回方才说道:“据我看来,上海这样事情也多得狠。舅太太有什么主意,和他想个法儿也好,乐得收他两件东西,连我也好得些好处。”舅太太方才的这番做作,原是和伍小姐吃寡醋吃出来的,其实自家心上也狠想见见这个人。如今听得王姆姆这般说法,正中下怀,便道:“收了他的东西,就要和他设法;得人钱财,与人销灾。但是想不出一个好好的法儿,便怎么样呢?”

王姆姆道:“这倒不要紧,他说只要舅太太同着大小姐再到张园去顽上一趟,他见了舅太太,大家慢慢的再想法子。”舅太太听了大喜,便问:“这个带信的是什么人?”王姆姆道:“是卖花的阿七。”舅太太道:“你叫他回去和那姓章的讲,明天在张园相会就是了。”王姆姆听了,便出去和阿七说了。阿七十分高兴,连忙回去报信,不必提他。

这里舅太太走进房来,见伍小姐横在榻上已经微微睡去。把一弯玉臂当作枕头,星眼眬眬咙,云鬟不整,额上略略的有些香汗,好似那梨花挹露,杨柳涵烟。那一种娇柔婀娜的丰姿,真个是倾国倾城,无双绝世。舅太太看了,未免有些自惭形秽起来。暗想这般风态,我见犹怜,怪不得姓章的要这般钻头觅缝的转他的念头。便叫了一声道:“起来罢!这个地方有风,睡不得的。”伍小姐被舅太太唤醒,便坐起身来道:“这几天十分困倦,心上总觉得有些不畅快,也不知道是什么缘故。”

舅太太道:“那几天你父亲病重的时候,你连日连夜的伏侍,辛苦了些,所以这几天这般困倦。”说着,伍小姐便叹了一口气。

原来伍小姐到了这般年纪,情窦已开,自从那一天见过章秋谷以后,虽然没有什么邪念,却总觉得心上有些不快。横也不好,竖也不好,也不知心里头想些什么,连伍小姐自己都讲不出来。如今听得舅太太提起父亲病重的事情,觉得自己一个身体没有一些着落,虽说倚着父亲做个靠山,但是一个人是说不定的;万一个父亲死了,叫自己去倚靠着那一个?想到这里,便不知不觉的长叹一声。舅太太趁势说道:“这两天,我看你总是这般闷闷的,好像有了什么心事的一般。明天我们还是到外面去散散心罢。尽着这般恹恹闷闷的,不要弄出病来,不是顽的。”伍小姐听了也无可不可的,点头应允。

隔了一天,果然舅太太哄着伍小姐梳洗停当,叫了一辆马车在门口等着。依着伍小姐的意思,要同着兄弟同去顽顽,舅太太道:“他好好的在书房读书,何必又去叫他出来?小孩子分了读书的心,将来要不肯用心读书的。”伍小姐听了觉得不错,便也不说什么,同着舅太太坐上马车,径往张园来。

到了大洋房,舅太太一眼早看见了章秋谷端端正正坐在进门左首的一张桌子上,眼睁睁的向外看着。舅太太见了,笑吟吟的送了一个眼风。章秋谷到了这个时候,方才觉得这位舅太太也在那里转他的念头,不觉心中暗笑。没奈何,只得也还他一个眼风,却细细的打量伍小姐今天的妆饰。只见他跟在舅太太后面,低着头款款行来,脸上觉得瘦了些儿。略施粉黛,淡淡的点一点胭脂,越觉得光彩照人,丰神绝世。秋谷见了伍小姐的面,不因不由的心上觉得发出一种说不出的感情来。

伍小姐刚刚走进,抬起头来已经看见了章秋谷,也不觉秋波一转,两颊微红。

暗想今天怎么这般凑巧,刚刚遇着了他。正想着,只见舅太太已经拣了秋谷身旁的一张桌子轻轻坐下,伍小姐便也一同坐了下来。凭着章秋谷目不转睛的呆看,伍小姐只是有意无意的,不狠兜搭。章秋谷无可如何,只得和舅太太眉来眼去了一回。

舅太太却十分高兴,卖弄精神,忽地立起身来,对着秋谷把嘴微微一动,又向伍小姐道:“你在这里坐一坐,我去一会儿就来。”说着往外便走。秋谷会意,也慢慢的跟出来。

舅太太走到门口,秋谷疾行几步,和舅太太擦肩过去。秋谷口内只低低的说“一品香”三个字儿。舅太太微微的把头一点。秋谷一直走出安垲第,假意四面望了一回,回身走进又坐了一刻。只见伍小姐无精打彩的立起身来,对着舅太太道:“我们到别处去顽顽罢。尽着坐在这里,气闷得狠。”舅太太听了点头称是,两个人一同走出安垲第,到老洋房弹子房去打了一个转身,又在照相馆拍了一个小影。

章秋谷在后面紧紧的跟着。伍小姐一面走着,也不免回过头来偷窥秋谷。伍小姐心上只觉得这个人跟前跟后,狠觉得有些痴气。

秋谷直等得伍小姐和舅太太两个人坐上马车,自己方才跳上车去,加上一鞭,在后面紧紧的跟着。到了分路的地方,秋谷把丝缰一带,霍地调过马头,回头过来,又和舅太太打个照会,便先到一品香去了。舅太太见了这般光景,连忙把伍小姐送回公馆,打发马车回去。一面重匀粉黛,再画蛾眉,对着伍小姐只说去看个亲戚。

伍小姐因他向来是这般惯的,也不疑惑,只说一句:“舅母既然还要出去,为什么打发马车回去?”舅太太支吾了伍小姐几句,一溜烟竟到一品香来。刚刚走上扶梯,便看见第六号门口牌上写着一个大大的章字。舅太太走到门口,探进半个身体望时,恰恰和章秋谷打个照面。秋谷见了,连忙立起来笑道:“恭候多时,请里面坐罢。”

舅太太觉得有些不好意思,红着脸一笑。秋谷又让一遍,舅太太方才轻移莲步,走进房来。

还没有坐下,秋谷迎着舅太太兜头就是一拱。舅太太也手捧胸膛还了一福。秋谷请他坐下,先开口道:“对不起,劳动得狠,今天总算赏光。”舅太太也道:“昨日多谢章少爷送的戒指、衣料,平空的怎么这般客气?”秋谷道:“那一点儿东西算不得什么。如今正有一件事情要仰仗大力,不知周奶奶肯答应不肯答应?”

舅太太故意问:“什么事情?”秋谷趁势走近舅太太身旁,把一张椅子移了一移,竟挨着舅太太并肩坐下。舅太太只把身体略略的侧了一侧,口中也不作声。秋谷低低说道:“这件事儿若换了别一个人,我也不便和他直说;如今,既然承周奶奶赏我的光,将来总是一条路上的人。”秋谷说到这里看着舅太太一笑,舅太太不觉把头一低。秋谷便伸手过去,挽着他的纤手。舅太太只不开口。秋谷附着舅太太的耳朵,把自己的意思细细讲了一遍。舅太太起初只是摇头不肯答应。秋谷又把这件事儿该应如何布置,怎样调度,说得井井有条。舅太太听了只得点一点头,口中说道:“我且去探一探他的口风,再想法儿。”说着只见细崽进来,请舅太太点菜。舅太太随意点了几样,细崽便走了出去。正是:

思想永夜,文君绿绮之琴;刻意伤春,杜牧青楼之恨!

不知后事如何,请看下回交代。





第一百一十一回 赋高唐东墙窥宋玉 隔巫峰云雨恼襄王





却说章秋谷心上暗想:“要想转这位伍小姐的念头,一定要把这位舅太太巴结好了,方才好借着他做个昆仑奴。”更兼看着这位舅太太虽然已经年过三旬,却也生得身段玲珑,丰神俊俏;心上虽然有些勉强,面上却做出十分欢喜的样儿,只说舅太太面貌怎样的纤秾,肌肤又是怎样的娇嫩,看上去还只像个二十多岁的人一般。

看官听者,大凡天下的妇女最喜欢别人恭维他的美貌。那一班妙龄已过的半老徐娘,又最喜欢别人说他年少。就是他不共戴天的杀父仇人,只要讲了这般一句话儿,泼天的仇恨也要消去一半!如今这位舅太太看着章秋谷这样一个唇红面白的美少年,讲的话儿又刚刚搔着他的痒处,自然十分喜欢,百倍缠绵。两个人谈谈说说,甚是投机。一直吃到差不多九点钟,方才吃毕。

舅太太立起身来要走,秋谷一把拉住道:“今天周奶奶既然来了,说不得只好委屈些儿,我们到虹口礼查去罢,他那里衾枕都有现成的。”舅太太面上一红,打了秋谷一下。秋谷笑道:“这一下打得十分爽快,等会儿请你多打几下何如?”说得舅太太嫣然一笑,瞟了秋谷一眼道:“我向来不住客栈的,况且我今天还有些事情,要回小房子去。”秋谷喜道:“原来你有小房子,在那里?何不早些和我讲个明白?”舅太太道:“我有小房子也不与你相干,为什么要和你说?”秋谷呵呵笑道:“就算我讲错了,何如?”舅太太似笑非笑的瞅了一眼,也不言语,往外便走。

秋谷急忙忙拿过帐单来签了个字,同着舅太太一同走了。

他们两个人,一个是半老徐娘,一个是江南名士。鸳鸯颠倒,春风半面之妆;云雨荒唐,锦帐三生之梦。掩灯遮雾,对影闻声;轻躯昵抱之时,玉体横陈之夜。

这一番情事,好像天外飞来的一般,章秋谷做梦也不曾想到!

一宵已过,舅太太回到伍公馆去,要想寻闲话打动伍小姐的春心,便对着伍小姐提起章秋谷来。只说:“这个少年好像疯子一般,只要一见了你的面,就跟前跟后的,不肯放松一步,不知他转的什么念头。”舅太太半真半假的说着,只指望要打动伍小姐,那知伍小姐听了这些说话只当没有听见的一般。舅太太说了几次说不上,只得暗中回覆章秋谷,叫他另想法儿。章秋谷听了,心上十分烦闷。暗想这样一个人天香国色的佳人,那有不知情爱的道理?大约一向在家里头,从来没有经过这样的事情,所以还有些糊里糊涂的不明白。想来想去又想了一个主意出来,自己口中自言自语的说道:“事情已经到了这般田地,就不得大胆子试他一试的了。”

章秋谷这边的事,按过一边。

只说伍小姐坐在家里过了几天。刚刚这几天的天气十分酷热,一轮烈日,万里无云,只把个伍小姐热得娇喘微微,浑身香汗,心上觉得烦躁。到了晚上还是这般酷热,院子里头没有一些儿风。舅太太便道:“今天热到这般田地,我们还是到张园去坐一会儿看看焰火罢。”伍小姐听了便也答应。舅太太登时妆束,立刻叫到一辆马车,两人坐了径到张园。在草地上拣了一张桌子坐定,就觉得微风吹袂,凉气入怀,一天暑气不知销到那里去了。

舅太太和伍小姐坐得不多一刻,忽然天上起了几阵大风,西北角上一阵阵的乌云直推上来。伍小姐见了有些害怕起来。催着舅太太回去。舅太太心中暗喜,坐着马车一同回来。马车走了一回忽然停住不走,说车轮坏了。两个马夫跳下来修了一回,还没有修好。舅太太忽然皱着眉头,双手捧了肚子,叫声“阿呀”。伍小姐忙问为什么,舅太太道:“一时腹痛起来,要找个地方解手。”伍小姐道:“这个地方,到那里去解手?舅母只好忍一回儿,回去再说。”舅太太道:“刚刚凑巧,有一个亲戚在这里,我去一去就来。”说着便跨下车来,又道:“你一个人坐在马车里头不便,不如你也同我一起进去坐一回儿,等他们修好了马车再走。”伍小姐听了,心上有些不愿意;还没有开口,早被舅太太不由分说,扶下车来。

伍小姐抬起头来,只见天上电光乱闪,四面的乌云都拢在一起,黑漆漆的好不怕人!伍小姐最怕雷响的,恐怕一个人坐在车上打起雷来无从躲闪,只得跟着舅太太走进弄内,又走进一家人家。只见一个十八九岁的少年女子笑吟吟的迎下楼来,便让伍小姐和舅太太楼上去坐。伍小姐见了这个女子,倒生得十分秀丽。当下舅太太同伍小姐跟着这个女子上楼坐下,刚刚走进房间,舅太太一个转身,走到大床后面去了。这个少年女子也对着伍小姐笑道:“请在这里坐一坐,我去去就来。”说着飘然去了。

伍小姐刚才进来的时候,也没有留心楼下房屋是个什么样儿,如今到了楼上,仔细看时,只见一并两间楼屋,一间便是客堂,左首一间卧室,却铺设得十分精致。

点着保险纱罩灯,一张红木大床,挂着湖色秋罗帐子。壁上也挂着许多字画。伍小姐正看间,忽然耳朵里头听得房门一响,连忙回头看时,见房门已经闭了,又听得门外落锁的声音。伍小姐摸不着头脑,心上十分诧异,暗想这个地方不像个好好的人家,为什么平空把我锁在这里?想着,不由得着急起来,连忙叫道:“舅母快来!”

那里晓得一句话儿方才出口,早听得床后脚步的声音,一个少年男子三脚两步的抢出来,对着伍小姐深深一揖。

伍小姐这一惊非同小可,连忙问道:“你是什么人?快些放我出去!”章秋谷不慌不忙,慢慢的说道:“小姐不必惊慌。我也断不敢在小姐面前放肆。自从那一天在张园见过小姐之后,已经眠思梦想的想了多时,也不知费了多少心血,方才把小姐请到这个地方。小姐请坐,有话慢慢的讲。”章秋谷虽然这般说着,伍小姐那里肯听?只急得香汗直流,芳心乱跳,口中只叫:“舅母那里去了?”几乎要哭出声来。秋谷见伍小姐急得这般模样,心上老大的不忍,只得又道:“小姐不要这般胆小。我说过不敢放肆,小姐只顾放心。只有几句话儿,和小姐说明白了,自然好好的送你回去。”

伍小姐方才见章秋谷突然在床后走了出来,急得眼花撩乱,那里还敢抬起头来看他!如今听得章秋谷言语温柔,没有一毫强暴的模样,方才略略放下了一二分心。

暗暗的偷看时,原来是两次在张园相遇的人,不觉心中又是一惊,只得腼腼腆腆的说道:“我和你向不相识,你把我关在这里做什么事情?我是好好的人家人,你不要弄错了。快把我舅母请出来,和我一同回去。”秋谷道:“小姐请先坐下,令舅母一会儿就来。”伍小姐那里肯坐!禁不得秋谷再三央告,只得勉勉强强的坐下道:“你有什么话说,快讲了放我出去。”章秋谷也坐下来,慢慢的把自己的思慕的情怀、相思的苦况,自头至尾说了个一字不遗;又道:“不瞒你说,我眼睛里头的女子也不知见过多少,从没有见过像你这般一个人。今天特地把你骗到这个地方,和你见一见面,就是立时死了,也不枉我章秋谷为人一世!”说着便立起身来,一步步走近伍小姐身畔。

伍小姐起先听了章秋谷的一番话儿说得十分诚恳,心上倒也有些感动,如今见了章秋谷走近身来,不知他要做什么,吓得连忙立起来,口中叫道:“舅母在那里?

快来同我回去!“秋谷摇手道:”小姐不必乱叫,叫也没有人来的。况且我已经讲明不敢得罪小姐,只求小姐赏一个光,和我讲一句话儿,我也不敢再想什么别的念头。“说罢,便伸手想要去握伍小姐的纤手;伍小姐吓得金莲倒退,脚步踉跄,一句话都说不出来。

秋谷见了伍小姐这般模样,不敢去勉强他,只得退后一步道:“小姐心上不愿意,我也不敢一定怎样。但是我为了你也不知费了多少的机谋,呕了许多的心血,已经成了个痰中带血的症候。小姐一定不肯,我又有什么法儿!”说着咳嗽一声,吐出一口痰来,秋谷把手巾接着,直送到伍小姐面前。伍小姐偷眼看时,果然那一口痰沫里头丝丝缕缕带着许多鲜血,不由得心中大动。登时两颊生红,低头不语。

秋谷见了伍小姐这般模样,知道事情有些指望,索性立着不动。伍小姐低了一回头,又抬起头来看了秋谷一眼。只见他丰度端凝,仪容俊爽,玉山朗朗,琪树亭亭,不由的叹了一口气。

秋谷趁着这个机会,抢步过来,一把携着伍小姐的手。伍小姐又叹一口气道:“我和你又没有什么冤仇,你何苦这般害我!”秋谷朗然说道:“这个怪不得我,是你自家不好。”伍小姐勃然变色道:“怎样是我自家不好!难道我叫你这般的么?”

秋谷道:“不是这般说法。谁叫你面貌生得这般都丽,方才惹出这般的事情来。若是生得将就些儿,就没有这些波折了。”伍小姐听了也不觉回头一笑,脉脉含情。

秋谷趁着这个当儿,便要放肆起来。伍小姐那里肯依?凭着章秋谷千方百计的哄他,伍小姐只是不肯,口中只说:“你要和我想想,教我将来怎么样呢!听你方才的口气,已经娶过正室的了,那里好这般一厢情愿的混闹!你们做男子的都是这般性格,把我们女子不知当作什么东西!难道只有你们是人,我们就不是人么?”

说罢不由的流下泪来。正是:

金堂夜永,三年心字之香;宝幄春温,一枕西楼之梦。

欲知后事如何,且听下回分解。





第一百一十二回 度良宵名花开并蒂 歌白纻病渴过三秋





且说章秋谷见伍小姐流下泪来,心上好生怜惜,替他拭了眼泪,口中说道:“我章秋谷平生忠厚待人,断不是负心的人物。你想,你和我素不相识,你又不知道我家里头的情形,我就是花言巧语的哄你一场,你也没有地方去问,何必一定和你讲真话呢!但是我想起来,我想着法儿,用了诡计,把你骗到这个地方,心上已经觉得二十四分的对你不起;若再要有心哄你,我自己心上实在过不去。你只要听这两句话儿,就晓得我不是有心哄你的了。只恨我自己没有这般福气,销受不起你这样的一个人。若是五年之前遇见了你,那就不是这般说法了!”伍小姐听了章秋谷这番说话,不知不觉的心软起来,沉吟了一回,只是摇头不语。

秋谷见伍小姐始终还是不肯,心中着急,暗想:“天下竟有这样的铁心石肠女子,凭着我的这般才貌,全付的工夫,竟是打他不动!这便想个什么法儿呢?”想了一想,只得又对伍小姐道:“小姐但请放心,这件事儿,将来没有什么乱子出来便罢,万一个闹了什么乱子出来,我章秋谷情愿与你同死同生,决不辜负你的一番情谊!”章秋谷说到这个地方,由不得心上一酸,便也长叹一声,眦泪欲滴。伍小姐听了,还在那里沉吟不决。

正在这个时候,猛然间一阵大风从窗外透进来,吹得人透体生凉,毫毛欲竖。

接着一个雪亮的闪电,白光一闪,直照得满室光明。这一下子,只吓得伍小姐四体皆酥,芳心乱跳,“阿呀”一声,连忙夺了手,把自己的耳朵紧紧掩住,口中乱叫:“舅母快来!”一霎时的工夫,那天上的雷声早排山震地的响起来,砰硼訇訇,震得人两耳欲聋。秋谷看伍小姐时,只见他吓得缩在椅子上,缩作一团,两手掩着耳朵,还在那里浑身乱战。秋谷见了心上十分怜惜,趁空儿走过去,轻舒两臂把伍小姐搂入怀中,口中说道:“小姐不要害怕,有我在这里,不妨事的。”说着不觉的斜倚香肩,低偎檀口,轻轻款款的安慰一番。伍小姐到了这个时候,心上害怕还来不及,那里还有什么工夫去和他拒却?自己的两只手儿又紧紧的按着耳朵,不敢放松,只得把头低垂,纤腰紧贴,伏伏贴贴的一动也不敢动。章秋谷心满意足,公然把伍小姐拥在怀中。这一阵的疾雷闪电,到像和章秋谷做了个媒人的一般,你道凑巧不凑巧?

当下过了一回,雷电已过,雨也渐渐的止了。伍小姐方才放了两手。抬起头来,见自己的一个身体竟坐在秋谷身上,不觉面上通红,挣着要走下地来。章秋谷那里肯放,不免渐渐的就要得步进步起来。私语温存,香肌熨贴;春情欲荡,欢意初融。

伍小姐到了这个时候,无可如何,只得半推半就的,默然不语。但见玉钩乍放,罗帐四垂;宝扣初松,明珰代解。汗湿梨花之颊,风回杨柳之腰。娇喘微微,清宵细细。半含雀舌,春融檀口之枚;低照云鬟,暗度麝兰之气。臂上之蛇医早褪,心中之凤卜初谐。宝髻惺忪,蛾眉罗转。东风羯鼓,催开上苑之花;瓜字年华,落尽深红之色。

过了一回,章秋谷走到门口,递了一个暗号,早听得门外“呀”的一声,房门开了,舅太太轻轻走进。伍小姐正对着镜子整整云鬟,见了舅太太进来,又羞又气,满心委屈,口中又说不出来,只说了一声:“舅母到那里去的?”一面说着,不知不觉的两行珠泪直挂下来。章秋谷深深的劝慰一番,舅太太也解说了几句。伍小姐心上总觉得有些忽忽不乐,临走的时候,委委屈屈的看了秋谷一眼道:“我上了你的当!”便咽住了说不下去。秋谷见了,没奈何只得自家引咎,说了许多好话。伍小姐方才同着舅太太一同回去。章秋谷也同着陈文仙回到新马路公馆来。

看官,你道伍小姐的马车为什么坏得这般凑巧?原来是章秋谷做成的圈套,和舅太太商量定了,故意叫舅太太这船做作。这个地方,就是舅太太的小房子。又怕伍小姐心上疑心,故意叫陈文仙充个屋主人。等得伍小姐登楼坐定,舅太太在后面偷偷的转了出去,陈文仙在外面锁了房门。章秋谷预先躯在床后,到了这个时候方才直跳出来。章秋谷因为傻小姐的事情费尽了心机,总是不能到手,便千方百计的想出这个法儿,果然伍小姐落了他的圈套。其实这件事情还是伍圭甫自家不好,平空的和这位舅太太勾搭上了,把女儿托他照应,又糊里糊涂的娶了个倌人做姨太太,两下泼起醋缸来,把好好的一家人家分作两起,糟蹋了一个冰清玉洁的女儿。这也总算伍圭甫的晦气了。

在下做书的写到这里,忽然有一位前辈先生来和在下说道:“你这部小说,名目叫做‘醒世小说’,自然是唤醒迷途,惊回春梦的意思。那些嫖界里头妓女骗人的事情,只说是唤醒那班嫖客,不要安心沉溺,拼命挥金,说说也还罢了;至于这位伍小姐和章秋谷的这件事儿,不过是一件伤风败俗的事情,窃玉偷香的公案,何必也要编在这部小说里头?还讲的这般详细,难道演说这些轧姘头、吊膀子的事情,也算改良风俗的么?你倒把这个道理讲给我听听。”在下做书的听了那位老先生这般说法,不慌不忙的对他说道:“老先生不消疑惑,请听在下一言。你老先生责备在下的一番说话虽然说得不差,但可惜没有把这件事儿的始末根由细细的推求一下。

你只想一想,这件事儿的原因是从那里来的?只要伍圭甫有些主意,不去和那位舅太太兜兜搭搭,也不至于把一家人口平空的分作两家;伍圭甫和舅太太没有牵连,章秋谷又那里走得这条门路?这叫做水腐而后蠛蠓生,酒酸而后醢鸡集。在下做书的所以把这件事情细细的演说出来,也好叫这班住在上海的大人先生看个样儿。从来欲齐其家,先修其身,先要整束了自家的品行,方才可以保得家里头没有暖昧的事情。这正是在下做书的劝人为善的意思。怎么你老先生倒反是这般说法?只怕错会了在下的意思罢!“那位前辈先生听了在下做书的一番说话,方才顿口无言,颠头播脑的走了开去。

闲话不提。只说章秋谷自从和伍小姐有了相好,章秋谷自己心上自然十分欢喜。

又为这件事儿,陈文仙非但没有吃醋的心肠,并且也帮着章秋谷在里头出力,章秋谷甚是感激。自此以后,在陈文仙面上不免也加了几分恩爱。依着章秋谷的意思,要想和舅太太讲明白了,买通了伍小姐家里的佣人,到伍公馆里头去和伍小姐重寻旧好。伍小姐恐怕泄漏出来不是顽的,再三的不肯;章秋谷只得约着伍小姐仍旧到舅太太的小房子里头相会。伍小姐一个月里头只肯出来一两次,只说时常出来恐怕给人晓得。好在伍小姐家里用的一个梳头娘姨晚上回家去的,这件事情影也不知。

王姆姆受了秋谷的贿赂,同他们打成一路,还有一个粗做娘姨和一个灶下的厨子,都是牛一般的蠢货,那里会知道这些事情?所以,这件事儿外面竟没有一毫消息。

只有章秋谷一个人,近几时来艳情深溺,香梦沉酣,好像个穿花蛱蝶一般,应酬了这一边,还要应酬那一面,不知不觉的,相如病渴,沈令衣宽,面上的丰彩竟销减了好些。陈文仙十分着急,婉婉转转的劝了几回。秋谷自己也觉得有些精神不济,向文仙要镜子来一看,不觉吃了一惊,暗想:“我章秋谷上有老亲,下有少妇,一个身体关系非轻,以后须要自家留意些儿。”想着,便依着文仙的话儿,在家里安息了几天。不觉金风乍起,玉露初零;凉生枕簟之秋,露冷屏风之影;早又是初秋时节。暑气潜消。正是:

西风昨夜,凄凉团扇之歌;秋雨茂陵,惆怅相如之病。

以后的书中情节,章秋谷初到珠江,安垲第大开胜会,康中丞挂冠归隐,赵娘姨看戏轧姘头。许多笑话,无数新闻,都在第八集书中出现。在下做书的到了这个时候却要歇息一回,和你们列位看官暂时告别了。





第一百一十三回 久安里旧雨续新欢 春申浦高朋宴良夜





上回书中说起章秋谷在家养病,养了十多天,觉得精神好些。坐在公馆里头,又觉得气闷起来。想着陆丽娟那里差不多有两个月不去了,便出了门,径到久安里陆丽娟家来。

陆丽娟本来和秋谷狠要好的,见秋谷多时不去,叫娘姨到秋谷公馆里头请了几次,秋谷只说有病不能出门。如今见秋谷来了,十分欢喜。一个大姐正在客堂里和相帮说话,见秋谷走进门来,连忙迎上来挽着秋谷的手道:“二少多时勿来哉。今朝啥格好风,吹到仔倪搭来介?”一面说着,拉着秋谷走上扶梯,口中叫道:“先生,二少来哉。”陆丽娟听了连忙走出来,接着秋谷笑道:“恭喜恭喜,耐格毛病好哉!倪一径来浪牵记煞。”说罢,同着秋谷进房坐下。陆丽娟见秋谷的面貌比以前消瘦了好些,便道:“耐面浪瘦仔几几化化哉,啥自家勿保重点呀!”秋谷笑道:“这个生病是没有躲闪的事情,叫我何从保重起呢?”陆丽娟瞅了秋谷一眼道:“只要少赶点正经好哉!”秋谷听了一笑,也不开口。

陆丽娟见秋谷坐在炕上,自己便也挨着秋谷身旁坐下道:“耐勒浪生病格辰光,倪心浪一径勒浪搭耐发极,叫金宝搭仔阿金妹去看看耐末,总归说得勿清勿爽。倪想自家到耐公馆里向来末,怕唔笃姨太太心浪勿舒齐。真真牵挂得来!难故歇阿好哉介?”秋谷听了微微一笑道:“算了罢,不用灌米汤了。你们当倌人的,做的客人也多得狠;要是客人病了,你就要急到这个样儿,你一个人那里来得及?”陆丽娟听了嗔道:“唔笃大家听听看,说出格号闲话来,阿要讨气!倪好好里搭耐讲闲话,耐咦是格付架形,真真良心才呒拨格!耐勿要勒浪勿相信,倪拨点末事耐看看。”

说着便走过去,在妆台抽屉里头拿出几张纸来,放在章秋谷手中。

章秋谷不知道是什么东西,接过来看时,只见几张纸上都批得花花绿绿的,原来是问病的课单。什么三马路吴鉴光、城隍庙知机子,批的病情都是十分危险,说了许多罗罗唆唆的话儿:什么冲犯家宅六神,故而致病;头昏心痛,寝食不安;又是什么幸有青龙星化解,转危为安,一派都是这些梦话。秋谷看了十分好笑,心上却也有些感动。又被陆丽娟撅了嘴咕噜了一阵,只得安慰了他一番方才罢了。

当下秋谷便在陆丽娟家摆了一台酒,请的客人无非是辛修甫、陶伯槐、王小屏、陈海秋等五六个人。大家因为秋谷多时不见,这一席酒吃得十分欢畅。陈海秋叫的范彩霞,到了席上见秋谷虽然瘦了好些,却还是那般的神彩飞扬,丰姿秀发,一顾一盼,卓荦不群,更觉得格外倾心,十分属意。也不知递了许多暗号,飞了多少眼风。秋谷却不过情面,只好将就应酬。陈海秋坐在席上,连影儿都不知道。范彩霞直到走的时候,还和秋谷咬了一个耳朵,方才走了。

这一夜章秋谷自然住在陆丽娟院中,不回去了。碧天如水,珍簟新铺。沉沉锦帐之云,闪闪银珰之焰。檀奴久别,夭娇非常;凤女多情,轻盈如许。这些闲事,都不必去管他。

只说章秋谷住在陆丽娟院中,一宵已过,起来的时候已经十一点钟。秋谷正在梳洗,忽听得楼下相帮高叫:“客人上来。”秋谷听了,心中甚是诧异,暗想怎么这个时候就有打茶围的客人?正想着,只听得那来的客人已经一步一步的走上楼来,口中问道:“有个姓章的章二少爷在这里不在?”秋谷在里面听得清楚,知道是贡春树的声音,不觉心中大喜,连忙抢步出房,叫道:“春树从那里来?我在这里!”

贡春树见了秋谷,也连忙走过来执手问讯。两个人知己相逢,心上自然高兴。

秋谷同着春树进房坐下。陆丽娟刚刚起来,见了贡春树丰神濯濯,仪表亭亭,不由心上吃了一惊。秋谷对他说道:“这位便是我平日和你讲过的贡春树贡大少。”

陆丽娟听了,知道是秋谷的要好朋友,便也殷殷勤勤的应酬一番,却偷转眼睛,细细的把章秋谷和贡春树打量一回,觉得两个人立在一起,还是章秋谷的气概胜些。

这里秋谷和春树谈了一回,便问贡春树到上海来有什么事情。春树道:“一则和你多时不见,特地来看你一趟;二则顺道看看小宝。却没有什么别的事情。”秋谷又问春树,怎么会找到这个地方来。春树道:“我先到你公馆里头,你们姨太太叫个大姐下来和我说的。”春树一面说着,一面细细的打量陆丽娟,看了一会,向着秋谷笑道:“你这位贵相好着实不差。你几时认得起的,我怎么不知道这么一个人?”秋谷道:“还是去年娶了文仙之后做起的。你看长得怎么样?”春树道:“真是天仙化人。也不知你几生修到的福分!”陆丽娟听得春树赞他,心上自然欢喜,微微一笑,也不开口。春树又道:“我看起来,和文仙也差不多。”秋谷道:“这两个人里头却有一个分别:一边是一味的丰彩清华,一边是一派的风神流丽。

两下比较起来,似乎还是文仙胜些。“春树听了,点一点头。

陆丽娟在旁听得秋谷这般说法,心上有些不快活,便道:“倪是勿好格,陆里比得上唔笃姨太太!”秋谷听了,一时说不出什么来,只得笑道:“你不用听错我的话儿,我说的是你们两个人各有各的好处。你话都没有听得明白,就要泼起醋瓶来。你这个吃醋似乎觉得过分些。”说得春树一笑。陆丽娟不好意思,便赶过来和秋谷不依,口中说道:“耐格人直头呒拨仔淘成哉!啥格吃醋勿吃醋,瞎说一泡!

只要唔笃姨太太勿吃别人格醋好哉!“秋谷笑道:”你只要心上不吃醋,我讲我的话儿,用不着你这般着急。一定这句话儿说着你的心病,所以要急到这般模样。“

陆丽娟听了,趁势往秋谷怀中一坐,想要伸手去拧他的腿;见贡春树对着他微微的笑,丽娟面上一红,连忙缩住了手;把秋谷打了两下,便立起来自去梳洗。

贡春树坐了一刻,忽然对秋谷道:“我正有一句话儿要问你一个明白。”秋谷便问:“什么话?”春树道:“去年你在苏州的时候,和我说什么打汇票不打汇票,我不懂是什么意思;正要问你时,被你一阵议论打断了话头,你也始终没确讲出来。

究竟是什么一句的话呢?“陆丽娟听了,在那里掩着嘴”格格“的笑。秋谷也笑道:”你这样一个聪明的人又是个老上海,怎么竟不懂这句话儿?这原是苏州人的一句俗语,男女对垒交锋,男人打了败仗,就叫做打汇票。你久在苏州,难道这句话儿都没有听人说过么?“春树听了心上方才明白,不觉也笑起来。笑了一回又问道:”我究竟不懂这句话是什么的一个意思,打败仗就直捷痛快的说打败仗就是了,为什么要叫做打汇票,这又是个什么道理呢?“秋谷道:”那些钱庄里头,每逢要用钱的时候,一时没有现银,便打一张汇票出去,叫他明天来拿。好像男女交锋,男人打了败仗,说句好听话儿,说明天再来,就是这个意思。“春树想了一想道:”这句话儿也没有什么意思。“秋谷道:”本来不过是句俗语,又不是什么通人大儒的格言,何必去考究他的意思呢!“

春树听了忽然想起一件事情来道:“你住在常熟,可知道钱纫秋的事情么?”

秋谷道:“这件事儿,差不多通省都传遍了,那一个不晓得?我去年不是和你讲过的么?”春树道:“他近来在南京自尽,你可知道不知道?”秋谷惊道:“有这样的事情么?不要你听了谣言罢!”春树道:“那里是谣言?我还带着金星精给你的信在这里。”说着,便在衣袋里头取出来递给秋谷。秋谷连忙接过来拆开封皮,看了一遍,叹一口气道:“这也总算个奇女子,可惜我们须眉男子都不能和他出来打个不平!讲起来也实在有些惭愧。”

看官,你道这位钱小姐,如何的会在南京自尽?这个写信给秋谷的金星精,又是一个什么人?原来钱小姐自从办过他哥哥的丧事以后,心中只恨着祁祖云祁观察一个人,平空叫阳湖县县尊出差提他到案,在大堂上出头露面,羞愤非常,心上早存了一个必死的念头,一心一意的想要报仇。知道本地的那些亲友都怕祁家势焰熏天,不敢惹他,便自己带了一个钱家的老家人,到湖北去寻族弟钱子瑶。见了面哭诉一番,要叫钱子瑶和他告状。钱子瑶本来是个胆小怕事的人,如今听得平空的要叫他去和别人作对,心上已经害怕;更兼祁祖云是个观察公,又把祁侍郎牵在里面,吓得把颈项一缩,舌头一伸,那里敢答应?钱小姐没奈何,只得自己做了一张冤单,要想到南京总督衙门去告。钱子瑶再三央求他,叫他不要惹事;又派了两个老妈,不由分说竟把钱小姐送到长江船上,要他回去。钱小姐心上本来想要到南京去告状,便上了船,直到南京,在城里一家客栈里头住下。正要自己坐着轿子到制台衙门去击鼓,忽然回心一想,如今的打官司有句俗话,叫做“八字公门荡荡开,有理无钱莫进来”。在地方官衙门里头尚且如此,何况制台衙门!自己身边又没有钱,这个官司那里打得赢?更兼世上的人情自然是官官相护,那一个来肯帮着我一个民妇和我出力?与其抛头露面、忍气吞声的受了许多委屈,依然还是扳他不倒,又何必多此一举呢!这一来有分教:

花残月缺,三年嫠妇之哀;烈魄贞魂,一夜西风之恨!

不知以后如何,请待后文交代。





第一百一十四回 弃尘寰烈妇捐躯 征挽联豪绅仗义





且说钱小姐想了一回,想不出个报仇的法儿。想着难道白白的受他一场羞辱不成?越想越气,越气越恨,不由的叹一口气。又心中自己打量道:“我本来是拼着一条性命和他打官司的,如今事势如此,没有法儿,不如趁个空儿决意自尽,或者我死之后,有那些热肠侠骨的人出来和我报仇也不可知。”想定主意,便预先偷着空儿,细细的写了一篇遗嘱,和那一张冤单一并放在一处。觑了一个便,竟自关起房门,悬梁自缢。真是:彩云易散,皓月难圆。三尺青绫,泪洒杜鹃之血;一场春梦,灰飞蛱蝶之魂。

那同去的老家人和那钱子瑶派来的两个仆妇,到得明天十二点钟的时候,见钱小姐的房门还是紧紧的关着,叫着也不答应,知道事情不好,打开了房门进去看时,已经高高的挂在梁上。老家人和仆妇猛然看见,吓得魂魄齐飞,六神无主。三个人六条腿好像钉住了的一般,连喊叫都喊叫不出。

这件事儿,霎时间已经传得大家知道,都赶来探听什么事情。依着店主人的意思,要去报官相验。幸而有几个明白事理的客人,把那老家人叫出来,问明了前后情节,知道是个烈妇,十分叹息;连忙拦住了店主,叫他不要报官;只叫老家人出名进个呈子,把这里头的情节略说几句,只说气忿自尽,恳求免验。那班做地方官的天天伺候上司还来不及,那里有工夫来管这些闲事!看了这个呈子,自然照例批准,不必提他。

只说老家人递呈回来,就在店里头草草的买棺装殓,扶着灵柩回来。常熟地方的一班绅士,除掉了祁观察手下的那几个走狗以外,都一个个嗟讶不已。也有几个热血的人,想要出头设法和钱小姐报仇。无奈钱小姐是自家自尽的,没有凌辱威逼的实迹;这位祁观察又是个有名的绅士,势焰熏天,炙手可热的,大家都不敢去惹他,只得叹恨一回,也就罢了。

只有一位绅士叫做金星精的,听了这件事儿心上十分痛恨。想出一个法子来,自己恳恳切切的做了一篇《钱烈妇行述》,刻了几千本各处分送;又发了许多传单,请了本地的绅士大家商议,要和钱烈妇设祭开丧。那些绅土里头,有几个狠有热血的人,自然大家赞成;有几个唯唯否否没有宗旨的人,一则却不过金星精的情面,二则心上也有些感动,便也都点头答应。只有那几个向来做祁观察走狗的人,一个都不来。金星精也不管他,便叫众人具了一个公呈,自己到常熟县知县莫大令那里和他说知,请他到开丧那一天去拈香致祭。原来金星精是个二甲进士出身,由刑部主事推升了刑部郎中,向来声名狠好,又是个江苏有名的才子。莫大令不好不答应,只得依他。金星精又各处去和他征祭文、征挽联,拣了一个日子和他开丧。

章秋谷在家的时候,和金星精时常往来,也是最要好的朋友。金星精此番做了这件事儿,心上十分得意,便写了一封信给章秋谷,细讲一番。正还没有寄,刚刚这个时候贡春树到常熟来游虞山,就住在金星精家里。住了几天,贡春树要到上海去看秋谷,金星精便把这封信交给春树,托他转致秋谷。

秋谷看了这封来信,又看了那本行述,心上也甚是不平。春树便对他说道:“金星精要叫你做挽联,你做不做?”秋谷道:“挽联自然做的。但是这个挽联,虽然没有什么难做的地方,却措词狠不容易得体。”春树道:“我看见兵部主事姚小知的一副对子,倒讲得狠痛快。”秋谷问是什么联语。春树念道:

凭天道断不令凶人漏网,

愧吾辈未能为匹妇复仇。

春树念毕又道:“你看这付挽联怎样?”秋谷道:“痛快是痛快得狠,但是这‘匹妇’两个字儿用得欠斟酌些。这样一个贞烈的人,不该应竞称他‘匹妇’。你细细想一想我的话可是不是?”春树听了连连点头道:“不错,不错。毕竟你的心比我来得细些。”

秋谷细细沉吟了一回,取过一枝秃笔,随手拉过一张局票来,兔起鹘落的写出一付挽联道:

一死等鸿毛百万同胞齐俯首,

双星圆碧落两行清泪奠贞魂。

写着便说道:“这里没有纸笔,只好明天写好再寄去的了。”说罢,递在春树手中。春树看了一看,点头叫好道:“你的笔墨,实在超脱非凡!”秋谷笑道:“又来了,又来了。我们知己朋友,怎么总是这般谬赞。”春树道:“并不是什么谬赞,好的自然是说好,坏的自然说不好。难道知己朋友就该应作违心之论的么?”

秋谷听了一笑。

春树问道:“听说新到一班马戏,你去看过没有?”秋谷道:“我差不多病了一个月,如今方才全愈,没有去看过。”春树问秋谷什么病。秋谷一时讲不出来,顿了一顿。春树笑道:“不是害的相思病罢?”秋谷也笑道:“什么相思病不相思病,不过是受了暑气,又带着感受新凉,所以有些不爽快罢了。”春树道:“今天晚上我们就去看好不好?”秋谷还没有答应,陆丽娟抢着说道:“蛮好,倪几家头一淘去。”秋谷也便答应。

陆丽娟问秋谷吃什么点心,秋谷摇一摇头。早有娘姨金宝端上一碗煨好的莲子来。秋谷也不推让,随意吃了些,便对春树道:“你还没有午膳,我们同到一品香去好不好?”春树道:“雅叙园的菜就狠好,我们何不往雅叙园去。”秋谷道:“雅叙园的菜虽然不差,却没有大菜馆的精洁。”正说着,陆丽娟接口道:“贡大少勿嫌怠慢末,就勒浪倪搭用仔便饭罢。故歇格大菜也呒啥吃头。”秋谷听了便问春树道:“你的意思怎么样?”春树道:“我无可无不可的,就在这里吃也好;但是叨扰了丽娟先生,心上未免有些抱歉。”丽娟‘忙笑道:“贡大少勿要客气,倪搭不过怠慢点,呒啥好莱。”贡春树道:“丽娟先生也不必这般客气,我就老老实实的叨扰了。”丽娟听了,便和娘姨金宝附耳说了几句,金宝便走出去;停了一回,早摆上四个碟子来。丽娟亲手去取一瓶巴德温来,取过两个玻璃小酒杯,斟了两杯酒,请秋谷和春树坐下吃酒。

秋谷看那四个碟子时,见是一样凉拌虾仁、一样粉皮鸡丝、一样醉虾、一样糟鸭,收拾得十分精致。春树见了道:

“多谢盛情。你们何必这般费事?”丽娟笑道:“呒啥物事,请唔笃随便用点。”

秋谷便邀着春树随意坐下,两人对酌。秋谷教丽娟同吃,丽娟便也坐在旁边陪着他们。等会儿相帮又端上菜来,虽然不多几样,却甚是精洁。秋谷因自己咯血还没有全好,便不敢多吃酒,只吃了两杯就不吃了。贡春树酒量甚好,一连干了几杯方才吃饭。饭后春树略坐一回,要同着秋谷去看金小宝,秋谷自然答应,两个同到惠秀里来。

金小宝见了贡春树的面,自然十分欢喜,携着春树的手道:“耐倒好格,一径搭倪说去仔就来、去仔就来。倪末一径勒浪牵记耐。”贡春树见了金小宝丰姿不减,华彩依然,也觉得眉飞色舞。两个人四目相视,倒说不出什么话来。金小宝见了章秋谷,想起那一天张园的事情,觉得狠有些儿惭愧,面上红红的也不开口。秋谷会意,便立起身来说:“我还有事情,等回儿我来同你们到马戏场去。”贡春树和金小宝两个人小玉重逢,韦郎久别,自然说不尽的相思况味,讲不完的别后情怀。见秋谷起身要走,也不相留。

秋谷从惠秀里出来,到自己公馆里头打了一个转身,又到久安里陆丽娟院中坐了一回。有两个朋友写条子请秋谷吃酒。秋谷因日间吃了两杯酒,觉得有些头晕,便辞了不去;叫陆丽娟熬了些荷叶粥略略吃了些,便同陆丽娟两个人坐着马车先到惠秀里去看贡春树和金小宝。

金小宝再三再四的邀着秋谷和丽娟进去坐一回儿。秋谷推却不得,只得略坐一会,催着贡春树和金小宝上了马车,直到跑马厅对面马戏场门口。秋谷先跳下车来,买了四张头等票,同着众人进去,拣了四个座位,大家坐下。那马戏场里头的一班看客,见了他们四个人进来,两个男子都是丰彩清华,衫裳倜傥,好似那琼树当风;两个女子,又都是容光焕发,态度娉婷,好似那花枝照夜。大家的眼光都不知不觉的注在他们身上,把他们细细的打量一番。秋谷和春树都没有留心,不去理会。

这个时候,刚刚一个十四五岁的少女,骑着一辆自行车登场献艺。空中绷着一条绳索,这个女子坐着白行车竟在绳上飞一般来来往往的行走,那一辆自行车好像贴在身上一般。满场的人都大家拍手。那女子献了一回技艺便进去了。里面又走出一个涂着花脸的洋人来,一面拍着手歌唱,一面哈哈大笑,口中叽哩咕噜的讲了一回。秋谷虽然略略懂些英文,却苦不甚精,听不出他说的什么,大约都是自己讥诮自己,引人发笑的话儿。停了一停,里面又走出一个洋人来,和这个涂脸的洋人互相问答了一会,这后来的洋人,就去伏在地上。正是:

春风良夜,勿惊虎豹之威;灯火深宵,曼衍鱼龙之戏。

不知后事如何,且听下回分解。





第一百一十五回 看马戏忽逢荡妇 闻狮吼惊散鸳鸯





且说那个后来的洋人和那涂脸的洋人讲了几句话,就去扒在地上,扒得伏伏贴贴的,四平八稳好像个乌龟一般。那个涂着花脸的洋人便抢步过来,一个斤斗在他背上打了过去,接着又是一个斤斗打过来,跳来跳去的跳得十分高兴。忽然地上的洋人跳起身来,照着翻斤斗的脸上就是一掌;只听得“拍”的一声,翻斤斗的“扑”

的跌倒,睡在地上不肯起来。秋谷看了十分好笑,一班看客也都拍手。

等了好一回,那涂脸的洋人方才在地上扒起来,不知从那里取出一枝点着的纸烟,放在口中慢慢的吃。里面又赶出一个洋人来,对他连连摇手,叫他不要吃烟,不由分说把他手中的纸烟抢了过去,往地下一掼。那涂脸的洋人候他走了,又取出一支出来放在口中;又赶出一个洋人来夺了过去。一连夺掉了七八支,也不知他在那里拿出来的。到得后来,四五个洋人都走出来,把他身上藏的纸烟一古脑儿都搜了出来,长长短短的,也有一二十支。那里知道这几个洋人刚刚转身,这个涂脸的洋人不知怎样的又取了一支出来,一面吸着,摇头晃脑的甚是得意。那几个洋人正要抢时,不料他在腰间取出一根三节棍,随手乱打。大家被他打得急了,跑进去拿了许多军器出来,什么腰刀、铁叉等类,混打一场,把他赶了进去。

随后又有一个少女骑着一匹黄马出来,身上止穿一层绝薄的紧身衣裤,都连在一起,远远望去,好像不穿衣服的一般;马背上也没有鞍辔,四围绕着戏场乱跑。

那女子在马上或坐或立,或睡或跳,颠颠倒倒的做出许多身段。只听得四围一片拍手的声音。

一套做完,只见推出一个虎笼来,就在场上用铁栏四围护住,两个洋人开了笼门,把个老虎放出笼来。两个洋人便百般的和他顽耍,一会儿把头放在他的唇边,一会儿又把手伸进他的口内,看的人都替他捏一把汗。

这个时候,章秋谷觉得这个顽意儿没有什么趣味,便抬起头来细细的打量那些座中的妇女。打量了一回,见虽然有几个面貌还好,却都不过平平常常的,算不得什么倾国倾城。看到西北角上的一面,忽然见两个俊俏大姐拥着一个少妇,头上戴着满头珠翠,只觉得珠光夺目,宝气照人,虽然相貌平常,却生得体格风骚,神情流荡,眉梢眼角大有风情。秋谷见了,未免回过头来多看几眼。那少妇见秋谷看他,便也卖弄精神,把一对水汪汪的秋波只顾望秋谷这边溜来。

秋谷正呆呆的看,忽然被陆丽娟用力在身上拧了一把。秋谷被他拧了一下,猛吃一惊。回过头来还没有开口,陆丽娟早附着秋谷的耳朵低低的说道:“耐格个人实头少有出见格,搭别人吊吊膀子还勿要去管俚,啥格戏子格姘头,耐也吊起膀子来哉!”秋谷听了,只说是陆丽娟有心吃醋,方才说出这样话儿,便也悄悄的回答他道:“你又不认得他是什么人,怎么知道他是戏子的姘头?”陆丽娟又低说道:“耐格眼睛到仔陆俚去哉?耐自家看哩!”

秋谷听了,连忙再往对面细看,果然见斜刺里头还坐着一个少年男子,也在那里和那少妇眉来眼去。那少妇一面对着章秋谷笑盈盈的飞个眼风,一面又喜孜孜的和这个少年男子打个照会,竟有些左顾右盼、应接不暇的样儿。那少年男子坐在那边,见了章秋谷这般模样,心上十分不快活,睁起眼睛望着秋谷。秋谷仔细看那少年男子的样儿,分明是桂仙戏园的武小生柳飞云。见他朝自己怒目而视,心上自然明白,不觉甚是好笑,却又自己心上暗想:“世上竟有这样风流放诞的妇人,双管齐下的吊膀子,未免有些过分了!”想着,便别转头去不去理他。在身边拿出表来看了一看,对陆丽娟道:“差不多已有十一下钟,我们大家回去罢。”

陆丽娟还没有答应,忽听得对面有个女人的声气叫声“阿呀”!接着有几个人都乱嚷起来,又夹着大家哈哈大笑的声音。章秋谷不知道什么事情,连忙举目看时,原来那个铁栏里头的老虎忽然要撒起溺来。那马戏的戏场,原是在中间划出一个大大的圆圈来,就算是个戏场。圆圈外面四周,都是排的一层一层的椅子,最近椅位就算头等,略远些的便算二等、三等。那坐在头等的,和那戏场的圆圈不过相离四五尺地方。偏偏的这个老虎走到圈边,撅起一条虎尾撒起溺来,好似那一道飞泉从空直泻,直射出去七八尺远。刚刚的把那位少妇和坐在两旁的两个大姐,还有坐在一起的几个女子,都溅得一头一脸,脂粉淋漓,衣裳湿透,连口内也溅了好些。这班人都是爱洁净的,怎禁得住这样一来?大家都叫声“阿呀”,又羞又恨,恨不得要哭出来。一时却又无可如何,只好把手巾去头面上乱揩乱抹,那里抹得干净了一班看戏的人见了这般光景,忍不住大家都哈哈大笑,只把这几个女子笑得无可如何,哭笑不得。出来的时候,原想倚着面貌出去出个风头,如今倒反出了这般的大丑!

没奈何,只得掩着脸儿急急的往外就走。武小生柳飞云也紧紧的跟着出来。

章秋谷看了,也不觉十分好笑,便也同着丽娟和春树、小宝四个人一起跟在他们后面出来。只见两个大姐扶着那少妇站在门口,见了小宝连忙别转了头。小宝也只作不曾看见,却低低的向秋谷说道:“耐阿认得俚?就是康家里格姨太太;勒浪外势轧姘头,轧得一塌糊涂。底子也是倌人出身,叫王素秋。格辰光为仔搭倪抢客人,吵仔一泡,一径到仔故歇,有辰光碰着仔倪,还是格付架形,耐想阿要好笑!”

秋谷听了点一点头,心中想道:“原来这个宝贝就是康己生的姨太太。康已生在江西巡抚任上,也不知弄了多少造孽钱,自然该有这般的报应。”说着,早见两个穿着号衣的马夫赶过一辆绝精致的橡皮轿车来,那位康姨太太还回过头来对着柳飞云看了一看,使个眼色,方才上了马车,一路回到虹口康公馆来。

康姨太太下了马车,急急的回到卧室。那些丫鬟、仆妇见了他们三个人都是这般模样,身上衣服一齐湿透,面上的脂粉更是斑斓狼籍的,一块红一块白,好像个妖怪一般,大家吃了一惊,不敢动问。康姨太太一肚子的没好气,发泄不来,一面忙忙的换了衣服,打水洗脸,一面打鸡骂狗的闹了一回,众人都不敢开口。康姨太太洗了一次,还恐怕洗不干净;又换过一盆水来,把上好的香肥皂在脸上细细的擦,擦了又洗,洗了又擦,一连换过了三四盆水方才罢了。正还要叫娘姨打开头发也洗一下,忽然一个念头,便问众人:“老爷到那里去了?”众人都说在内书房。康姨太太听了,便不管头发不头发,霍的立起身来,吩咐众人不许声张,自己一步一步的悄悄走到内书房门口。先侧着耳朵一听,果然听得里面有人在那里低低的讲话。

康姨太太听了心头火起,不由分说,竟自直闯进去。

这位康大人,平日原狠怕这位姨太太的,今天知道他去看马戏,要到十二点钟回来。这个时候只有十一点三刻,算定不得回来,正搂着个年纪狠轻的苏州娘姨在那里密密切切的说话。不料一时间这位姨太太走了进来,两下都大吃一惊。这个娘姨见了姨太太进来,吓得魂不附体,连忙飞一般的在后面逃了出去。康大人目定口呆,坐在椅上一句话都说不出来。康姨太太眼睁睁的看着康大人,看了一回方才把两个指头狠命的往康大人额上戳了一下,咬着牙齿道:“总是这样偷偷摸摸的性情,死也不肯改的!这样的一把年纪,还有什么脸见人?”康大人听了只得陪着笑脸道:“你不要这般多心,我和他又没有什么别的事儿。方才不过和他讲几句话,你又何必这般动气?”康姨太太冷笑一声道:“亏你讲得出这样的话来!一个做主人的,为什么要和娘姨干这些鬼鬼祟祟的把戏?也有这样不要脸的人来勾搭主人。有你这样的主人,自然就有那般的贱货!”说得康大人闭口无言,只是老着脸呵呵的笑。

康姨太太数说了一回,便要连夜的把那娘姨赶出公馆。康大人觉得心中不忍,只得再三替他央告,涎着脸缠了一回,只说:“这会儿为着这件事情赶他出去,人人有脸,树树有皮,万一他脸上下不来,逼出些意外的事来,我们虽然不怕,却也何苦呢!不如只当没有这件事儿,过几天借一件别的事情叫他出去,岂不干净?”

康姨太太先还不肯,当不得康大人苦苦的拦着,只得罢了。

列位看官,你道这位康大人是个什么人物?原来就是在下做书的在第五集里头讲的那位康己生康观察。这位康观察自从捐官以后不多几年,他那位老太爷就得病死了。康观察丁了三年的艰,在家里头没有什么消遣,又不好明公正气的嫖赌,只得悄悄的叫媒婆和他做媒,娶了两个姨太太。又把自己家里的一个丫鬟名叫彩云的,收在房里也算做小老婆。这三年丁忧期内,只成日成夜的和这几位姨太太滚在一起。

好容易盼得三年服满,便赶进京去,要想走了门路,去选个好好的缺。正是:

膏粱子弟,不知稼穑之艰;纨绔郎君,忽起簪缨之想。

不知后事如何,且听下回分解。





第一百一十六回 谋补缺观察入都 受苞苴奸奴作弊





却说康观察自从他老太爷死后,在家里头守制三年。这几年里头也不知闹了多少希奇古怪的顽意儿,早把他老太爷的一份家资去了十分之五。康观察想着,尽着这般的坐吃山空,也不是个长局;算来算去,只有还是去做官。自己本来捐了个候选道在身上,不如趁着自己年富力强的时候到官场里头去混一下,或者混得出什么好处。打定了主意,便带了几万银子的汇票赶进京来,拣了杨梅竹斜街的一家高升店住下。先拜了几天同乡,要想找个门路,却一时找不出来。康观察十分焦急,便有几个同乡京官和他说道:“你要找门路,不用到别处去混找,只要去找吏部的书办;找到了和他商议,没有不妥当的。”

看官听着,原来这个各部的书办,京城里头人都叫他作部办,最会营私舞弊,纳贿招权,差不多比那各部尚书的权柄还要大些。你道这个是什么缘故呢?一个小小的书办倒反比尚书的权柄大些,这句话儿讲出去给人听了,那一个肯相信,岂不是在下做书的有心说谎么!原来这个里头另有一个自然的道理在内,并不是在下做书的平空掉谎。看官们请休性急,待在下做书的一一道来。

那各部尚书虽然权重,却都是由别处调来的,三年也是一任,五年也是一任。

部里头情形不熟,办起公事来就也只好将就些儿。这班部办却是世世代代世袭下来的,从小的时候就把本部的历年档案,记得烂熟在肚子里头。那些部里头的司官,那里有他这般本事!我们中国的向例,办起公事来都要照着例案办的;没有例案可援的,便要请旨办理。每每的堂官接了一件公事,便交给那班司官,叫他援例办理。

司官那里记得部里这些档案,就只好来请教这班部办了。这班部办趁着这个当儿,便上下齐手的作起弊来。譬如这件事情部办已经得了贿赂,明明可以驳斥的,他一定要想着法儿引出一个例案来叫你核准;要是这件事情部办没有得到钱,明明可以批准的,他也一定要找出一个例案来叫你驳斥。你想,一个部里头历年案卷堆积如山,也不知有多少,除了这些部办,别人那里记得尽许多!那怕你一样的两件公事,同是一天的日期,同是一般的情节,他得了这一边的钱,就拉出某人某人的旧案来照例核准;那一边没有走他的门路,他就有本事又去拉出某人某人的旧例来平空驳斥。那班司官只图省事,那里还去管他们的得贿不得贿,作弊不作弊!那班堂官又都是尸居余气的,过得一天,便是两个半日。就是明知道他们在外面作弊,无奈本部办公都仗着这班部办,一天也离不了他们,也就只好眼开眼闭的装着糊涂,不去多管。看官,你道这些部办可利害不利害!

在下做书的做到此间,便又有一位友人不相信在下的说话,对着在下说道:“你这个话儿我就有些信不过。那部办不过是部里的一名书吏,那里就会这么利害起来。就算那些尚书、侍郎不知本部的情形,不熟本部的例案,那班司官也有二三十年还在一个部里头当差的,难道就没有一个熟悉例案的么?”

在下做书的听了笑道:“你的话只知其一,不知其二。你也细细的想一想这个里头的情形,再说别的。你想他们那班部办,从小儿不做别的事情,只捧着这些例案,当他四书五经一般死命的揣摩简练,还有父兄在那里细细的教他,自然的熟能生巧,好像是他们的看家本事一般。至于那班司官,从小儿先要揣摩八股,又要学些词章,还有什么策论表判的,已经闹得他一塌糊涂的了。再到后来中了个进士,分了个部曹,他心上又在那里算计如何如何的钻谋外放,如何如何的打点升官。成日成夜的把那一团卑鄙势利的思想横放在肚子里头,连那以前没有做官之前藏在肚子里头的一点良心,都汩没得干干净净的了,那里还有工夫来留心这些事情!况且他们那些司官们在部里头当差,那一个不想放个外官?那一个不想高升上去?不是打算一生一世在部里头混的。比不得那些部办,靠山吃山,靠水吃水;吃的也是本部,穿的也是本部,用的也是本部。有百年的部办,没有百年的堂官和司员。你只要细想一想,就明白这个道理了。”

那位友人听了在下这一番说话,低着头想了一想,便道:“照这样的说起来,一个部里头只要用个部办就够了,又何必要什么尚书、侍郎呢!”在下做书的听了,叹一口气道:“我们中国的事情向来如此。你认着那些尚书、侍郎大人先生有什么出类拔萃的大本领么?只要有部办的学问,已经是好的了,那班不如部办的还多得狠呢!就是如今的那班地方州县,难道一个个都是熟悉民情、谙练吏治的么?官场衮衮,宦海茫茫,我们又何从说起呢!”在下的那位友人也就长叹一声,默然不答。

如今闲话休提。只说康观察听了同乡的话儿,便同了一个同乡的内阁中书叫做张伯华的,同着他去找到了一个有名的部办,姓刘,号叫吉甫,住在绳匠胡同里头绝精致的一所宅子。康观察到了门前停了车,心中想道:“这所宅子倒像个什么一二品大员的住宅,若不讲明了是个部里的书办,外面那里看得出来?”想着,等了一回才请了他们进去;在一个客厅上又等了好一回,方才见这个刘吉甫匆匆的走了出来。见了张伯华笑道:“咱们多日不见了,一向可好?”张伯华连忙立起来。康观察也跟着和他客气了一阵。

刘吉甫略谈了几句,便问康观察道:“咱们一向少亲近得狠,今天同伯华兄光降,不知有什么见教的事情没有?”张伯华便道:“这位康己翁有件事儿,要奉求你老哥和他想个法儿。老哥如不嫌亵渎,请屈驾到饭庄子上坐一回儿,我们好慢慢的商议。”刘吉甫笑道:“不瞒你老哥说,兄弟今天还有些穷忙,不能出去。那饭庄子上的饭也没有什么吃的。我说句放肆的话,今天你们两位既然赏我兄弟的光,竟请不必客气,就在这里吃个便饭。不过没有菜,简慢些儿。”康观察还没有开口,张伯华知道刘吉甫的性情向来爽快,便也点头答应。

刘吉甫说了几句话儿,就说一声“失陪”,竟自走了出去。出去了好一回方才进来。张伯华便把康观察的来意和他说了一遍,又说:“这件事情总要请你老哥推我的情,帮个忙儿。至于谢仪一节,只要请你老哥吩咐一声,自然如数送过来。”

说着,早已摆上饭来,四盆四碗,还有一壶酒,虽然样数不多,却十分精致可口。

刘吉甫让他们坐下,一面吃着,一面细细的盘问康观察的捐官是在那一案的,什么年分,交了多少银子?康观察一一说了。

不一时吃完了饭,大家洗漱已毕,只见刘吉甫侧着个头,口中不知念些什么,又轮着指头算了一会,忽然笑道:“果然早得狠呢。”便对康观察笑道:“依着你老兄的这个班子,若要照例轮选起来,只怕还要好几年呢!如今在你前面还有四个压班的。要等这四个都选了出去,方才轮你得着。这还是没有岔子的说话。要是半路上跑出一个压班的来,那就还是一个不中用。如今外省道员出缺的又狠少,就是出了缺,又都是一次部选、一次外简的,像你这个班次,只怕三年五载候不着也不算什么。”

康观察听了心上着急起来,便和张伯华附耳说了一回,叫张伯华托他设法。张伯华正要开口,只听得刘吉甫慢慢的说道:“这个道缺,比不得什么州县;事情大了,上头的一班堂官们在这个里头也狠留心。今天要是换了别人来和我讲这个话儿,我兄弟也不是轻易答应的;无奈我和伯华兄相识多年,难道说这点儿情面都没有?

在你们两位老兄分上,做兄弟的自然要和你们两位设法效劳。依我看起来,只要把你老哥的名次和那几个压班的倒个过儿,回来外省出了道缺,就挨着你老哥轮选,这是妥当不过的事情。大约迟则三月、早则月余,你老哥就好到任。至于谢仪的一层,不瞒你们两位说,我兄弟平日之间也专爱的结交朋友,不是那只认得钱不认得人的人物。这件事儿,一则多蒙康己翁见爱,不去找别人,却来找我;二则我和伯华兄知己朋友,情面难却,并不是想什么钱。但是这件事儿不是我一个人的首尾,不得不点缀他们一下。至于我兄弟自己身上的什么谢仪不谢仪,咱们自家兄弟竟请不必客气就是了!“

张伯华知道刘吉甫的脾气,便道:“多谢老哥费神,但是究竟怎样的一个数目,还要请老哥核算一下。”刘吉甫听了,便取过一面算盘来滴沥搭拉的算了一阵,便对张伯华笑道:“里里外外的使费,一古脑儿要三万五千银子,这还是看你老哥分上,别人拿了五万银子,我还不见得答应他呢!”康观察听了刘吉甫的话儿,心上吃了一惊,暗想:“自己通共带了三万银子,家里头的钱所存不多,如今他一开口就要三万五千银子!”心上有些踌躇不决起来,一时答应不出,只看着张伯华的脸,和他使个眼色。正是:

衣冠扫地,侍中之貂尾何多;犬马登堂,灶下之羊头如许。

不知后事如何,且听下回交代。





第一百一十七回 严选政部办吃虚惊 出奇兵名优施巧计





且说康观察听了刘吉甫要三万五千银子,就不觉心上一惊,便立起身来走过去,悄悄的和张伯华说了一回。张伯华便陪着笑脸,对刘吉甫道:“兄弟还有一句不知进退的话儿,要和你老哥商量。”刘吉甫听了,心上也有些明白,便道:“你有什么话儿,只顾讲就是了。难道咱们这样的交情,还有什么通融不来的事情不成?”

张伯华听了,便拉着刘吉甫,两个人在一起坐下,婉婉转转的讲道:“方才你老哥讲的数儿,康己翁知道狠便宜,他心上也十分感激,那里还有不愿意的道理!无奈他也有一个苦情,要请你格外原谅他些。他现在只有三万银子,还有五千一时凑不出来。又知道你老哥办的清公事,不是和市上买东西一般,可以争多论少得的。他的意思,想先付三万银子,还有五千银子请你和他暂时垫付一下,随后再缓缓的归还。但是这件事儿,已经承了你的盛情和他谋干,没有什么好处到你身上也还罢了,倒反要你和他垫起钱来,他自己讲不出,托我和你说一声儿,不知你心上怎么样?”

刘吉甫听了,心上也知道康观察的意思想要少出五千银子,却又不好意思一定怎样的和他争论,索性说得好听些儿。想要不答应,无奈张伯华的这番说话实在说得情理兼到,推却不来,只得微微一笑道:“伯华兄,咱们大家都是明白人,打开桶子讲亮话,还是这么样罢:如若康己翁得了个好缺,这五千银子是不能少的,总算给他们吃个喜酒;或者缺分平常,不见得怎么好,这一笔钱也就不必拿出来了,总算我姓刘的结个朋友。何如?”张伯华听了,自然不好再说什么,同着康观察说了几句客气话儿,两个人一同回去。

康观察就把那一张三万两银子的汇票拿了出来,交给张伯华,托他明天送去。

张伯华起先不肯一个人送去,定要同康观察一同送去。康观察道:“你这个人何必这般拘执,难道我还信你不过么?”张伯华听了方才接了过来。想着几万银子的事情不是顽的,便不等明天,立刻又坐了车赶到绳匠胡同来。见了刘吉甫,把银票交代清楚,便要告辞。刘吉甫苦苦的留住,对他笑道:“这一笔钱咱们在里头经手的人,照例有个九扣的,一共三千银子,咱们两个人两下平分。方才你们两个人同在一起,所以我也没有提起。”张伯华听了喜出望外,自然乐得收领的了。

两个人谈了一回,张伯华问起康观察的这件事情怎样的一个调法?刘吉甫便也细细的把调换的法儿和他讲了一遍。张伯华低头想了一想道:“这样办法,我看不见得怎样妥当罢。万一个上头查了出来又怎么样呢?”刘吉甫笑道:“这个法儿在当时是一万年也查不出来的。除非后来查拣别件公事案卷,一个不防备查了出来,也或者有的。但是到了那个时候,他心上知道自家错了,断不肯认真追究的。要是认真追究起来,我们虽然要担不是,他自己先有了个失察错误的处分。所以那班堂官就是明知道我们作弊,也无非打个哈哈就过去了,历来都是这个样儿。”张伯华听了也微微一笑道:“照你这些说话看起来,难道那班堂官就没有一个弊绝风清的么?”刘吉甫道:“也有时遇着了个难说话的堂官,不许我们作弊。我们又有一个挟制他们的法儿,会齐了合部的大小书办一同告退。他们那班堂官,离了我们是一件公事也办不来的。这样的一来,他没有法儿,也就只好听凭我们去怎样怎样的了。

老实对你讲罢,我们本部里头的公事,要准起来,件件都是准的,要驳起来,件件都是驳的。“张伯华听了不懂,连忙问什么道理。刘吉甫道:”一样的两件公事,今天准了你的,明天驳了他的;也有今天驳了你的,明天却准了他的。所以我们在部里头当差的人没有作不来的弊,没有准不来的事情,也没有驳不来的案件。只怕撞着了个不顾前后不受情面的堂官,一味的和你混闹起来,那就糟了。“张伯华听了口中不说什么,只心中暗想:怪不得这班部办这般利害,也有这些道理在里头。

想着便起身告辞,又到康观察寓中坐了一坐,便也自己回去。

康观察自从出了这三万银子以后,天天坐在寓里头等候消息。隔了一个多月,刘吉甫来给他报信说:“如今浙江杭嘉湖道缺出,恰恰是应归部选。你的事情我已经和你打点得好好的,你只要预备谢恩就是了。”康观察听了心中大喜,呆呆的等了两天,连店门都不出。

这一天康观察刚刚起身洗脸,忽见刘吉甫大踏步走进来,脸上的神色十分不快,见了康观察,只说一句:“你的事情坏了。”康观察听了心中大惊,连忙问什么事儿。刘吉甫拍着手道:“你的事情我已经和你安排得停停当当的了。那里知道,昨天晚上忽然被堂官查了出来。如今正在那里查核例案。这件事情闹了出来,虽然没有什么大事,不过认个无心错误便过去了。但是你白白丢掉三万银子,叫我怎么对你得起呢!”

康观察听了,一时只急得目定口呆,做声不得,连忙问道:“好好的,怎么又会被他们查了出来?”刘吉甫道:“也是合该有事。我们本部的一个同事,和堂官的侄少爷有些亲戚,前天喝醉了酒,无心露了口风,今天就闹出这个乱子来。”康观察听了,心上二十四分的着急,便问:“可有什么解释的法儿没有?”刘吉甫道:“法子是有,只不知道你肯不肯。”康观察道:“我自己身上的事情,那里倒有什么不肯!只不知竟是个什么法儿,可妥当不妥当?”刘吉甫道:“这会儿且慢些提起,去请了张伯翁来,我们大家商议一下再说。”

康观察听了,也不好一定再问,只得叫人立刻去请了张伯华来。刘吉甫和他交头接耳的商量了一会,定了主意方才和康观察说,只要如此如此。康观察听了呆了一回,道:“别的不必说他,倒是这几千银子一时那里去找呢!”刘吉甫一面笑道,一面从靴统里头取出一个小小的靴页,拣出两张银票递给康观察道:“承你老哥瞧得起我,咱们总算是个知己朋友,要是这点事儿都不预先和你打算一下,那还算什么朋友!”康观察接过银票来看时,只见齐齐整整的三千一张,二千一张,心上方才放心。便也随口谢了刘吉甫几句。刘吉甫哈哈笑道:“算了,不用客气了。咱们如今就去讨个信儿罢。”说着便催着康观察套起车来。

三个人一同到了一处地方,大家下车进去,里面早迎出一个十六七岁的美少年来,生得粉面朱唇,细腰窄背。这个时候,正是十一月天气。这少年穿着一件淡密色缎子猞猁皮袍,上面衬一件枣红色缎四围镶滚的草上霜一字襟坎肩;头上戴着瓜皮小帽,迎面钉着一颗珍珠,光辉夺目;脚上薄底缎靴。一见了他们三个,便满面添花的说道:“三位老爷请里面坐。”把他们邀进一间绝精致的书房坐下。先问了康观察的名姓,便对着康观察略略的把腰弯了一弯,好像要请安的样儿。刘吉甫连忙一把扯住道:“康大人是自己人,不必客气。”那少年听了,回起身来也略略的朝着他们两个点一点头,笑迷迷的口中说道:“你们两位是常来的,我就大胆放肆了。”刘吉甫连忙笑道:“老佩,今天你和我这个样儿可是该的么!你把我们当起客人来了,快快的请坐了,好讲话。”那少年听了微微一笑,便轻轻的把身躯一扭,一个转身便坐在张伯华下首,那转过身来的时候,两面的衣裳角儿都是纹风不动。

真个的一身身段,圆转非常。

那少年坐了下来,先应酬了康观察几句,刘吉甫便抢着说道:“老佩,你不用尽着应酬。咱们今天的到你这里,有一件正经事儿要和你商量。”说着便把自己的椅子往那边挪了一挪,紧靠着那少年身旁坐下,低低的说了一回。又招手儿叫张伯华过去,三个人又说了一回。只听得那少年笑道:“这件事儿交给我就是了!”刘吉甫听了大喜,便走过去向康观察要出那一张二千银子的银票,塞在那少年手中。

那少年又笑道:“咱们还讲这个么!”刘吉甫道:“这一点儿算什么。只要你肯和我们帮个忙儿,就承情得狠了。”那少年道:“既然如此,我也不好不收,只好暂时收了再说。你们也不必回去吃饭,省得来来往往的费事,就在我这里吃顿便饭等一回儿,好不好?”刘吉甫听了连忙答应,又跑过去和康观察附着耳朵讲了几句。

康观察自然欢喜。

看官,你道这个美少年又是个何等样人?就是在下做书的不讲,列位看官料想心上也有几分明白。原来这个少年是京城里头数一数二的红相公。什么叫做红相公呢?就是那戏班子里头唱戏的戏子。这少年便是四喜班里头唱花旦的佩芳。京城里头的风气,一班王公大人专逛相公,不逛妓女。这些相公也和上海的倌人一样,可以写条子叫他的局,可以在他堂子里头摆酒。无论再是什么王侯大老,别人轻易见都见他不着的,只要见了这些相公,就说也有、笑也有,好像自己的同胞兄弟一般,成日成夜的都在相公堂子混搅。那窑子里头简直没有一个人去的,就是难得有一两个爱逛窑子的人,大家都说他下流脾气,不是个上等人干的事情。正是:

清歌妙舞,伶工傀儡之场;豪竹哀丝,太傅东山之宴。

不知后事如何,且听下回分解。





第一百一十八回 闹相公尚书中计 告病假巡抚归田





且说刘吉甫同着张伯华和康观察在佩芳那里吃了一顿便饭,佩芳嘱付了康观察许多说话,又教导了他许多礼节。这位康观察虽然外面的仪表长得不错,心上却狠有些糊涂,只听着刘吉甫和佩芳两个人的话儿连连点头。坐了好一回,只见一个小孩子飞一般走进来,向着佩芳做个手势道:“来了,来了。”佩芳霍的立起身来,叮嘱刘吉甫同着康观察:“宽坐一回,等会儿再来叫你。”说着便匆匆的去了。

康观察同着刘吉甫、张伯华闷坐在书房里头,连一声都不敢响。只听得里面嘻笑说话的声音,足足的等了半天。只见一个十一二岁的小孩子走了进来,口中说道:“请康大人快些进去。”刘吉甫听了,连忙推着康观察立起身来,叫他进去。康观察是已经习过仪注的,心中虽然有些七上八下的不得劲儿,却自己拿定了心,放大了胆子,一步一步的走过了一层院子。

院子里面,另外还有三间精室。听得上首一间屋内有个老头儿的声气,在那里和佩芳讲话。佩芳一面笑,一面讲道:“你管了这个吏部,不论京外各官,都要在你手里选出来的是不是?”佩芳说罢,只听得那一个老头儿也笑着说道:“这个自然。”佩芳道:“可惜我只会唱戏,不会做官;如今我有个亲戚,是个进京候选的道员,要想拜在你的门下,托你照应他些。”说到这里,便咳嗽一声。

康观察听了,连忙抢进房门;刘吉甫也跟着进去。举眼看时,只见一个六十多岁的老头儿,穿着一身半新不旧的衣服,方面大耳,一部花白胡须,正搂着佩芳坐在身上说笑。忽然抬起头来,见他们两个人平空的走到面前,心上十分诧异。正要开口问时,康观察早疾趋而进,双膝跪下,叩首有声。刘吉甫也跟着一同跪下。都在靴统里头取出手本来,恭恭敬敬的递上去。

那老头儿见了他们两个这般模样,摸不着头脑,连忙推开佩芳想要立起身来。

不料佩芳紧紧的一把拉了他的胡子,对他说道:“你不要慌,这就是我的亲戚。他要拜你做个老师,你就收了他罢。”那老头儿听了,睁着眼睛一时讲不出话来。佩芳早伸出手去,接了康观察和刘吉甫的手本;又把康观察手内的一个红封套接了过来,抽出三千两银子的一张银票,不由分说竟替那老头儿揣在怀中,口中笑道:“这是人家孝敬你的贽敬。”这一阵播弄,竟把那老头儿播弄得目瞪口呆,开口不得;定了一回神,方才说道:“这个使不得!”刚刚说了这一句,佩芳接上去说道:“有什么使不得?你不用累赘,只收了就是了;我在外面已经和他们讲明白了,你不答应,就是剥我的脸皮!”

原来这个老头儿,就是现任吏部堂官白礼仁白大人。这位白尚书别的都没有什么,只有个爱顽相公的毛病儿。见了相公们就如性命一般,一天不和相公在一起也是过不去的。这个佩芳更是向日最得意的人,天天完结了公事,一定要到佩芳寓里来顽的。如今见佩芳家里平空的走出这两个人来,明知道这两个人一定是买通了佩芳要来走他的门路,心上想要翻转脸来,喝令他们出去,一则佩芳撒娇撒痴的死缠着他,定叫他答应,不好意思一定怎样;二则自己也是个一位大员,本来不应常在外面这般混闹,万一个闹了出来,自己身上也有好些不便之处;更兼白尚书分明认得刘吉甫是本部的书办,自己是个堂官,如今在这个地方给他撞见了,脸上好像有些过不去的样儿。一时间心上七横八竖的不得主意起来,只得对着佩芳说道:“你这个孩子,不问什么事情,专要这般的多管闲事。”佩芳道:“他们两个都是我的亲戚,怎么又是我多管闲事呢?”白尚书听了也说不出什么来,只得说道:“你也不管是什么东西,受得受不得,就这样的混出主意!”佩芳道:“这是他拜师的贽敬,有什么受不得!你们做官的人,拜老师送贽敬是通行的,又不是你一个人,算不得什么大事。”白尚书听了,料想今天不答应是不行的,又见康观察和刘吉甫两个人还直挺挺的跪着不敢起来,便道:“你们且先起来,有话好说。”二人听了方才立起身来,垂着手站在一旁。白尚书只随随便便的问了几句话儿,佩芳便对着他们使个眼色,两个人都会意,便请一个安退了出去。

隔不多时果然一道谕旨出来,浙江杭嘉湖道就放了康观察。康观察自然欢喜,忙忙的预备谢恩,预备召见,忙了差不多有一个月,便到浙江去到任。事有凑巧,刚刚到那位浙江巡抚常恒常中丞,虽然是个旗人,却和康观察家有些世谊。康观察又放出浑身本事来巴结这位常中丞。常中丞十分欢喜,格外照应。到任不多几时,刚刚藩台调了江西,常中丞又和这位臬台不合,就委康观察署理藩司。康观察忙忙的到任接印,心上十分得意。不想过了两年,常中丞死了。康观察就调了直隶天津河间道。做了两年,康观察不知怎么的又走了一个军机大臣的门路,给了他一个密保,就升了云南按察使。康观察嫌着云南路远,就又钻营了门路,调署江西布政司。

也是康观察的官运亨通,不到一年就升补了湖南布政司。接着江西巡抚出缺,里头一班军机大臣知道康方伯江西的情形狠熟,就传旨出去把康方伯升授江西巡抚。

康中丞在江西足足做了五年,忽然有个御史参奏康中丞帷薄不修,官箴有玷;并且说他在天津道任上的时候,怎样怎样的放纵家属,怎样怎样的败坏伦常,要请皇上认真查办。这个消息传到康中丞的耳朵里头,不觉又羞又恨。就有人劝他趁着这个当儿告个病假,奏请开缺,随后慢慢的再想法儿。康中丞听了,心上还有些不决。刚刚那位军机大臣又打个电报给他,说近来参你的人狠多,不晓得究竟是什么缘故。事关暖昧,又不便一定怎么的和你深辩。不如暂时告病,以后再想法儿。康中丞得了这个电报,没奈何,只得立刻电奏请假。不一日,京里头回电来了,准他开缺。康中丞只得怏怏的带着家眷回到江苏,也不回无锡去,住在上海虹口,买了一所高大精致的洋房,自家住着。

看官,你道这个岔儿究竟是怎样的一回事情?原来康中丞在天津道任上的时候,有两位堂房姊妹住在衙门里头。这两位小姐的性情却生得十分古怪,一天到晚只知道同人顽笑。不管男的女的、老的少的,就是康中丞手下的亲兵和抬轿的轿夫,碰着这两位小姐心上高兴,也要和他们顽笑一回。康中丞虽有几个妻妾,那几个姨太太只晓得争风吃醋,大家闹得个一塌糊涂。这位太太又性情懦弱,弹压不住,凭着这两位小姐这般放纵,也不去管他们的闲事。这两位小姐见没有人说他,索性两个人都改了男装,出去混闹,也不知他们做的什么事情。天津一府的人,没有一个不知道这两位小姐的大名。这几个连衔参奏康中丞的御史公,原是个翰林出身,都是淮安府人,总算是康中丞的大同乡。康中丞在天津道任上的时候,这几位太史公一同进京,路过天津,想要向康中丞借些旅费;康中丞一毛不拔,不肯应酬。如今这几个宝贝都考取了御史,想起不肯借钱的仇恨,便大家联名参他一下。如今暂且按下。

再说起这位康中丞来,自从告病开缺以后,原想略略的等过一年半载,再想法子去走京城里头的门路。不想事机不顺,那位军机大臣忽然得了一个急病,呜呼死了。接着康中丞的后任春华中丞,为着库款的事情参了康中丞一下,说他办事颟顸,虚糜公款。幸而没有什么实迹,康中丞又已经离任,这件事情便也成了烂案。康中丞经过了这样的一来,一时找不出起用的门路,只得缓了下来。

这个时候,那两位小姐虽然已经出嫁,无奈天生成的薄命,嫁过去不到两年,男人都一病死了。这两位姑太太不肯住在家里,都搬回娘家来住,比以前闹得更加利害。康中丞也不去管他。从来说近朱者赤,近墨者黑。这两位姑太太闹到后来,连那几位康中丞的姨太太也学起他们的样儿来,成天的涂脂抹粉,扮得妖妖娆娆的,出去坐马车、看夜戏、吃大菜、游花园,闹得外面的名气沸沸扬扬,十分难听。康中丞虽然有些知道,却也无可如何,只得缩着个头,凭着他们去怎生闹法。

上海的地方原是天地间的一个极乐世界。康中丞虽然年过五旬,看着这些粉白黛绿的妖姬,过着那般酒地花天的日月,自然的未免有情,谁能遣此?便自己也在嫖赌场中混闹起来。看中了个倌人叫做王素秋的,花了七千块钱的身价,把他娶了回去。这个王素秋也是个数一数二的个中老手,那里肯嫁康中丞这样一个拱肩缩背的老头儿?本来原想借着他淴个浴的。不想到康中丞家内,康中丞宠爱非常,竟把他当个正室夫人一般,把家里头上上下下的事情一古脑儿交给他一个人管理。真个是一呼百诺,要一奉十,不敢有一些儿违拗他的地方。正是:

荀香何粉,三千选佛之场;锦帐银床,十二金钗之队。

不知后事如何,且听下回交代。





第一百一十九回 思淴浴名妓嫁衰翁 约空房家妈私爱妾





且说王素秋自从嫁到康中丞家,原想趁一个空卷了金银珠宝逃走出去、别抱琵琶的,想不到既嫁之后,康中丞待他甚好,又狠有些怕他,更兼看着那几个姨太太的样儿,成日的描眉画眼,卖弄风骚,绝不像个好好的人家人,康中丞只当没有这件事儿,说也不说一句。王素秋见了这般模样,心中暗想:“既然他不管闲事,乐得安安顿顿的不用私逃,省得逃了出来耽惊受怕。况且这样舒服的日子,就是逃走出去也未必过得着。”想定了主意,便索性拿出浑身手段来牢宠这位康中丞,只把个康中丞骗得骨软筋酥,心输意服,渐渐的由爱生畏起来。一天一天的下去,一个成了篾,一个成了铁。康中丞只要见了这位姨太太的面,就觉得有些毛骨悚然。

王素秋又使出本事来笼络那几位姨太太,大家面子上都十分要好;更兼他现在当家,那些姨太太都要到他一个人手里来讨生活,自然免不得大家都迁就他些。王素秋又拿着康中丞不心痛的钱在众人面上挥霍,不到一年,早已把康中丞公馆里头上上下下、大大小小、男男女女的一班人,都收得伏伏贴贴。大家不怕康中丞,只怕这位姨太太。要是得罪了康中丞,只要是姨太太欢喜的,康中丞也不能一定把他怎样;要是得罪了姨太太,一定立时立刻的发作出来,康中丞那里敢回护!

康中丞的正室夫人穆氏,本来和康中丞性情不合,自从娶了这个王素秋以后,老夫妇更加不睦,也着实吵闹了几场。穆夫人赌气不管事儿,自己回到母家去,和康中丞音信不通,好像毫不相干的一般。康中丞也自由他,不去理会。王素秋见了,心上自然更觉得意,渐渐的自己也做些暖昧事情出来,只瞒着康中丞一个。每每碰着心上不高兴的时候,每便坐着马车出去看戏,有时对着康中丞只说到亲戚家去,差不多要到晚上一两点钟的时候方才回来。康中丞还只说他是个正经人,那里知道这些秘密。

康中丞未娶王素秋之先,本来已经娶过四个姨太太,都是依着次序排下去的称呼。第一个娶的叫大姨太太,第二个娶的就叫二姨太太,娶到王素秋已经是第五个了,本来合家的人都叫他五姨太太的。偏偏的他又倚着康中丞十分宠爱,言听计从,硬要跨过这几个姨太太的前面去,逼着家里头的人要叫他大姨太太,其余的几个都排在他的肩下。众人听了自然不敢违拗,只得听从。

这位大姨太太平日之间本来最爱看桂仙戏园的戏,一连去看了几次,就看上了武小生柳飞云。两下眉来眼去的,狠有几分意思。倒是柳飞云知道他是康中丞的家眷,不敢造次,恐怕弄出事来。刚刚的事有凑巧,康姨太太在马戏场中看戏,又遇见了柳飞云也在那里。康姨太太心中大喜,便对着他搔头作态,龋齿弄姿,做出十二分丑态。正在得意,不想那不知趣的老虎偏偏又要撒起溺来,撒得他一脸一身,心上又羞又恨,那里坐得住,只好急急的赶回来。恰恰的又遇着了康中丞做些鬼戏,不由得把方才一肚子的闷气都发泄到康中丞身上来。闹了一回,康中丞再三自家认错,便也只好罢了。心上却只想着那柳飞云怎样怎样的身段玲珑,又怎样怎样的台容俊俏,一夜之间,颠颠倒倒做了许多好梦。到了明天,便觉得一个身体软哈哈的抬不起来。

康中丞不知道他害的是相思病儿,只道他当真有病,心上便着了慌,要叫人去请医生调治。倒是这位姨太太不肯,只说没有什么病,康中丞只得由他。还有那几位姨太太和那两位姑太太,听得大姨太太有病,便大家都来看他;康姨太太也免不得应酬一番。众人在康姨太太房里头坐了一回,见康姨太太只是有些懒懒的样儿,怕他心上厌烦,便都起身走了。

康姨太太看着他们走出回廊,只有二姨太太一个人走得慢些,落在后面,刚刚走到屏门左近;只见一个少年家人叫做陆升的,从外面走进来,见了二姨太太便使一个眼色。二姨太太微微一笑,把嘴向左首一努,匆匆的往外便走。那个少年家人抢前一步,也随后跟来。他们两个人只顾调情,忘其所以,那里想到大姨太太在后面帘子里头看得十分真切。

这个王素秋本来原是倌人出身,何等的精灵古怪,那一件事儿瞒得过他!看了他们两个人这样情形,不觉心中暗暗好笑。自己心上算计了一回,暗想这件事儿若是换了别人,也还罢了。这个二姨太太向来是和我面和心不和的,有时还要把我取笑几句,只说我是堂子里头出身,他是个好人家的女儿。今天既然落在我的眼中,说不得要给他一个利害。想到这里,猛然得了一个主意,连忙悄悄的叫进七八个娘姨大姐来,只说要到外面东厢房里头去拿东西。众人听了,你看着我,我看着你,彼此都狠诧异。有一个伶俐些的大姐便开口说道:“那东厢房里面收的都是些用不着的旧货,人都不进去的,不知道大姨太太要去拿什么东西?”康姨太太听了嗔道:“不用你多管,你们只跟着我悄悄的去,不许声张,大家都轻轻的走。”众人听了,大家都心中疑惑,却又想不出究竟是什么事儿,只得依着他的说话,大家都跟着他轻轻的走出去。

康姨太太带着众人,一步一步的径向方才二姨太太努嘴的地方走去。这个地方本来是堆放什物的,一家大小的人没有事情都走不到这里,静悄悄的一个人也没有。

康姨太太一直走到东厢房的窗外,站定脚步侧耳听时,果然听得有男女两个人的声音在那里低低说话。康姨太太听了心中大喜,便回过头来对着众人高声说道:“你们都走进去!”说着便自己第一个轻移莲步走进门来。

这一下子,把这里面的男女两个人吓得魂不附体,浑身乱颤。想要逃走时,那里逃走得掉!康姨太太早已走了进来。这两个人没奈何,只得双双跪下,口中只说:“我们该死!”那一班娘姨、大姐出其不意的见了这般的一出把戏,大家也都目瞪口呆。康姨太太却故意做出那一种十分惊骇的样儿,口中说道:“怎么?怎么?

你们两个人这般大胆!干起这个把戏来!你们难道王法都没有的么?“二姨太太跪在地下,羞得两颊通红,眼含珠泪,一句话也说不出。只有陆升连磕响头道:”大姨太太的明见,家人方才不过和二姨太太说了几句话儿,不敢放肆,大姨太太是看见的。只求大姨太太开恩。“康姨太太故意怒道:”你这大胆的奴才!二姨太太是狠规矩的,都是给你这个奴才引诱坏的!“说着停了一停髓:”如今叫我怎么样呢?

你们还是出去请了老爷进来罢!“

二姨太太听了,心上二十四分的着急,暗想单是老爷知道了,倒还没有什么;好在没有拿到什么凭证。但是这样一来,这件事儿就瞒不住的了。要是合宅的人都知道了,以后还有什么脸见人?只得老着脸皮苦苦的求道:“我也是一时该死,上了人家的当,只求你高抬贵手,瞒过了老爷,不要叫别人知道;我以后情愿和你当个丫头,伺候你一生一世。”说罢早不知不觉的挂下泪来。陆升见了这般光景,也连连的在地下磕着响头求饶。那班娘姨、大姐都是和陆升要好的,见了他们两个人形景可怜,便不约而同的大家替他告饶。只说:“大姨太太抬一抬手,饶了他们。

如若以后再敢这般,再请老爷定夺,也是大姨太太的一件阴骘。“

康姨太太本来知道康中丞的脾气,不过为着二姨太太和自己有些龃龉,如今借着这件事儿把他当场拿住,一则自己做个好人,二则从此以后就好借此挟制,叫他不敢和我作对。便趁势对二姨太太说道:“快些立起来,有话好好的讲。我们都是自己姊妹,何必要做出这个样儿来。只要你们以后诸事小心就是了。”说着便拉了二姨太太起来,对着陆升说道:“还不起来给我滚出去!今天真是你的造化!”正是:

西厢待月,未妨卓氏之琴;巫峡行云,惊破襄王之梦。

未知后事如何,且听下回分解。





第一百二十回 王素秋看戏轧姘头 柳飞云当场施绝技





且说康姨太太拉了二姨太太起来,又嘱付自己同去的那几个娘姨、大姐道:“这件事儿,你们看在二姨太太分上,不准声张,如若外面有人知道了风声,我只和你们几个人说话!”众人听了只得齐声答应。二姨太太羞得低着个头抬不起来,听得康姨太太这般分付,只道他是好意,不因不由的心上十分感激,对着康姨太太扑的又跪下地去。康姨太太连忙一把拉住,搀了起来,口中说道:“你再要这般模样,就不成个自家姊妹了。”二姨太太面红过耳,低低的说了一声“多谢”。又向那些娘姨、大姐说道:“对不起你们众位,只好慢慢谢你们的了。”

看官,你道这位二姨太太既然要做这样的事情,为什么不秘密些儿,却这样的粗心草率?康姨太太既是有心去捉他的破绽,又为什么不肯声张?难道还顾着康中丞的面子,不肯闹出来么?原来康中丞虽然做过封疆大员,家里头的家法却是一些也没有的。这位二姨太太,这样的事情也不止做了一次了,看得轧个把姘头、吊个把膀子没有什么希奇。就是这些娘姨、大姐也都看得惯了,并没有一些儿诧异的意思,好像是分内的常事一般。至于这位大姨太太的不肯声张,却另外有个道理在内。

既不是卫顾康中丞的面子,也不是周全二姨太太的脸儿,却为着这个陆升生得俊俏非常,语言伶俐。康姨太太初嫁康中丞的时候,就狠喜欢这个陆升,久已存着个要勾搭他的意思;倒是这个陆升有些蝎蝎螫螫的,不敢放肆。康姨太太见了这般光景,觉得自己毕竟还要留些身分,不好意思一定怎样去俯就他。好在康姨太太的事情狠多,只转了几个念头也就罢了。如今无意之间忽然见了陆升和二姨太太这般如此,不觉心上有些酸溜溜的吃起醋来,故意带几个人去真赃现获的捉住了他们两个,却又胡弄着不肯声张。一则好在陆升面上见一个情;二则收伏了二姨太太,做个自家的心腹。这也总算是天从人愿,一举两得了。

闲话休提,只说康姨太太自从在马戏场回来之后,心上只想着柳飞云的模样,觉得他一言一笑、一举一动,无一不好。便故意到桂仙戏园去包了一个厢。对着康中丞只说要请客,早早的吃过了晚饭,重施脂粉,再画蛾眉;头上挽着一个懒妆髻,疏疏的几件钗环;身上换了一身素罗衣裤,衬一条玄色纱裙;足下又换了一双簇新的挑绣弓鞋;淡妆素服,妖艳动人。打扮好了,又自己在镜子里头照了一会,坐上马车一直到桂仙戏馆来。

到了戏馆,走上厢楼,案目呈上戏单来。康姨太太接过来留心看时,只见排的柳飞云的《战宛城》。康姨太太便分付案目,叫挂出牌去,要点柳飞云的《白水滩》。案目答应一声,便走了出去。这个时候已经做到第三出了,正是小喜凤的《游龙戏凤》。小喜凤本来是上海数一数二的有名花旦,扮了酒店里头的李凤姐,和那老先生做的正德皇帝,两个人眉来眼去,卖弄风骚,看的人一个个齐声喝彩。康姨太太是醉翁之意不在酒,便只当没有看见的一般,只低着头一言不发的在那里想他自己的心事。

等了一回,柳飞云的《白水滩》上场。康姨太太眼睁睁的看着台上,目不转睛,要看柳飞云的身段。一回儿手锣一响,绣帘开处,柳飞云迅步登场。只见他戴一顶攒花箬笠,着一件织金玄缎夹衣,里面衬着一身品蓝衣裤,胸前绕着白绒绳蝴蝶扣儿;面上搽着血点一般的胭脂,画着长长的两道眉毛。俊眼流波,双眉入鬓,身材夭矫,台步从容。面貌本来生得十分俊俏,再衬着这样的一身结束,越显得蜂腰猿臂,鹤势螂形。这柳飞云听得康公馆的姨太太点他的戏,心上早知道了八九分,连忙结束登场;先抬起头来一看,就对着康姨太太飞了一个眼风。康姨太太也笑吟吟的和柳飞云使个眼色,两个人四只眼睛你来我往,一去一还,闪闪烁烁的好似电光一般满场飞舞。台下那一班看戏的人,也有几个老上海,看出他们两个吊膀子的情形,却都是事不干己,那个去管他们的闲事。

这个柳飞云见康姨太太有意吊他的膀子,越发放出他全副的精神来;那打倒青面虎的一场,一条棍棒耍得就如风车儿的一般;上三下四,左五右六,使得个风雨不透!临了儿更格外添出许多解数,翻出许多斤斗,只听得台下一片喝彩的声音。

把一个康姨太太看得眼花撩乱,张开了一张樱桃小口,一时间再也合不拢来。到了那吃紧的时候,康姨太太连忙在身上掏出一大卷钞票,也有五块一张的,也有十块一张的,举起手来,竟是往台上一撩。刚刚这个时候,柳飞云收了棍法,回转头来对着康姨太太微微一笑,便大踏步走进后台去了。

不多时,《白水滩》已经完了,柳飞云换了便衣上来谢赏。见了康姨太太,深深的请一个安,垂着手规规矩矩的站在旁边。康姨太太想要和他说几句话儿,觉得心上好像有许多话儿,一时却想不出来,不由得俊眼斜眸,红云上颊。停了好一回方才说出一句话来道:“你到上海有几年了?”柳飞云又请一个安道:“小的到上海两年了。”趁着请安下去的时候,柳飞云的右手早在康姨太太的一双脚尖儿上碰了一下。康姨太太回头一笑,脉脉含情。

两个大姐本来是和姨太太一路的,见了他们两个人这般形景,便对康姨太太说道:“我们回小房子去罢。”这个姐听了,便道:“我还有事,等一回儿就来,你跟着大姨太太先去。”康姨太太听了一言不发,只点一点头,对着柳飞云把眼一瞟,立起身来就是。那一个大姐见康姨太太走了,便同着柳飞云不知到什么地方去鬼鬼祟祟的打了一个转身,便把他一直领到新马路口的一处地方,悄悄的在后门进去。

柳飞云虽然色胆如天,到了这个时候也由不得心上有些心惊胆战起来。到了门内,转过前堂,走上扶梯,直到一间房内,却静悄悄的不见一个人。柳飞云举眼看时,只见是一所两楼两底的洋房,起造得十分小巧精致。房间里头都是些外国器具,一色雪白,耀得人夺目生辉。正中间摆着一张铁床,也挂着雪白的冰绡帐子,点着两盏纱罩自来火灯,照耀得满房内灯光闪烁。

柳飞云正在打量,早听得帷后弓鞋细碎的声音。康姨太太扶着一个大姐的肩膀慢慢的走出来,已经换了一身家常衣服。春山挹翠,秋水横波;神彩飞扬,丰姿婀娜。柳飞云早已看得呆了。康姨太太走出来,对着柳飞云微微的一笑。柳飞云抢步过去,直到康姨太太身旁,又请了一个安道:“姨太太的恩典。”康姨太太一把拉了柳飞云起来,笑道:“你请安叩头的混闹些什么!我难道是叫你来请安的么?”

说着,便又回过头去一笑。柳飞云到了这个时候,就口馒头,岂有不吃的道理?自然也要放肆起来。他们两个人,一个是男儿身手,解数非常;一个是中妇妖娆,风情如许。自然的巫峡云痴,银河水满;颠倒鸳鸯之字,迷离蛱蝶之魂。与别人的寻常欢会,大不相同。

到了明天,柳飞云恐怕有人知道,一早起来悄悄的溜了回去。康姨太太慢慢的起来梳洗,梳好了头,便同着两个大姐坐着东洋车,到他一个结拜姊妹的公馆里头,大家说了一回闲话,方才坐着自己的马车回去。

看官,你道康姨太太在外面住夜,康中丞为什么竟不疑心?原来康姨太太自小儿堕落平康,原没有什么父母姊妹,只不过有几个结拜姊妹,都是把势里头的倌人。

也有已经嫁人的,也有还做生意的。自从嫁了康中丞以后,便说和这几个人本来都是亲戚,硬要和他们来往。康中丞也不敢拦他。只要有了什么意中人,要在外面住夜,对着康中丞就说是到亲戚家去,要住过一夜方才回来。自己坐着马车,先到个已经嫁人的结拜姊妹家里,便打发马车回去。明天十二点钟,再叫马车来接。那班堂子里头出身的人,那里有什么好货!虽然嫁了人,大家原都是打成一路的。康姨太太这般做作,不过是瞒瞒外人的耳目罢了。至于那个马夫,本来原在四马路马车行里头的,一向做堂子里头的生意,和康姨太太也有些不明不白。见了康姨太太这样藏头露尾的行径,心上虽然明白,那里肯直说出来?乐得借着这个由头,向康姨太太借几个钱敲些竹杠。两个大姐都是在堂子里头带来的,自然是一路上的人。只瞒着康中丞一个。那几个姨太太里头,也有两个是堂子里头的倌人,看着康姨太太的行为,心上虽然有些疑惑,却想着大家井水不犯河水,又拿不着他的凭据,那个肯来做这样的空头冤家?更兼自己身上也都有些不明不白的勾当,做贼心虚,也不来管别人的闲事。只有那位二姨太太口直心快,对着康姨太太说话的时候,未免有些含着皮、包着骨头的话,所以康姨太太使出手段来,先收伏了这位二姨太太,叫他以后非但不敢再说什么,并且不得不和他一路。这也是康姨太太的一片深心、一番辣手了。按下不提。

只说康中丞的那两位堂房妹子,自从守寡之后,越发的风流放诞起来,天天的跑马车,游张园,只要遇见了个清俊些的少年子弟,就使出那勾魂摄魄的手段来勾引他。更兼这两个宝贝衣装华丽,态度风流,那一种娇娆荡佚的样儿,真个比堂子里头的倌人还要胜个几分。就是他不吊别人的膀子,别人还要来寻着他们;何况又是这样的两个头等名角!只引得那些滑头子弟,如蚁附膻,如蝇逐臭,大家都想他们的念头。正是:

文姬新寡,群登子反之床;卓氏私奔,谁有相如之渴?

要知后事如何,但看下文交代。





第一百二十一回 联美眷荡子迷香 破温柔滑头泼醋





且说贡春树自从到了上海之后,和金小宝久别重逢,自然的枕上风情,衾边缱绻,比往常加了几分。金小宝这个时候,本来除了牌子不做生意,便劝他索性搬到惠秀里来住。贡春树见金小宝虽然还有几个熟客在他那里来往,小宝却不大应酬;更兼小宝的房间不止一个,便也乐得应承,夜夜偎香,朝朝倚玉,两个人十分相爱,百倍缠绵。章秋谷也常常的到金小宝那边,和他们两个人讲讲那些花丛里头的典故,堂子里头的事情,却也并不寂寞。

转瞬中秋已过,又到重阳。露冷罗衣,风吹冰簟。章秋谷又回到常熟家里头去了一趟。只住了半个月,便又托着事故重到上海来。

贡春树在金小宝那里住了差不多两个月,狠有些“此间乐,不思蜀”的意思。

这一天贡春树饭后无事,便信步踱到马路上来??转过大新街,想要到久安里陆丽娟那里去看章秋谷。正走过大观楼门外,忽听得楼上有许多人的声气在那里纷纷扰扰的乱闹乱嚷,又夹着有人哈哈大笑的声音,也不知在那里闹些什么。

贡春树本来也是个少年好事的人,听了楼上这般热闹,不知不觉的就想上楼看看是件什么事情,便走上楼去。举眼看时,只见有十余个油头滑脑的少年,都坐在靠着楼梯的几张桌上,口中都在那里夹七夹八的乱嚷;另有一个少年,低着头坐在那里不敢出声。有一个滑头滑脑的少年,头上刷着一转一寸多高的刘海发,身上穿着一件湖色绉纱夹衫、玄色实地纱马褂,指着那少年的脸大声说道:“你可知道图奸寡妇是个什么罪名?你好好的写下一张伏辨来,我们便将就些儿放你回去。如若不然,我们就要对你不起,把你送官究治了!”那坐着的少年听了,只是一言不发,连头都不敢抬起来。众人见他并不开口,便大家乱嚷道:“你不用在这里装聋做哑的,就是装聋做哑也不中用!”又一个人大声道:“你们不用和他讲理,先把他送到捕房里去押起来再说!”

那少年听了他们说得这般利害,只得抬起头来,正要和他们说话,却一眼早看见了贡春树,不觉喜出望外,连忙叫道:“春树兄,你来得正好,请来和我评评这个理儿。”贡春树听了口音甚熟,就吃了一惊。连忙看时,原来果然就是他的两姨表弟杨慕陶。贡春树见了,便走过来,问他为了什么事儿这般模样。

杨慕陶正要开口,早见坐在他上首一个少年立起身来,睁开两眼对着贡春树喝道:“你是什么人?敢来管我们的闲事!快给我闭了嘴儿,不用多事!”贡春树见他这样的蛮横无理,心上不由得就生起气来,冷笑道:“我和他是亲戚,问一声也不要紧,何必做出这个样儿来。”那个少年听了,不觉心中大怒,抢近身来把贡春树劈胸一推。贡春树不曾防备,被他推了一个躘踵,几乎跌倒,心中十分愤怒,只得说道:“好好的讲话,怎么平空就动手动脚起来,难道没有王法的么!”那少年听了又喝道:“我就是没有王法!你又怎么样呢?你再在这里蝎蝎螫螫的,今天就打了你这个饭桶也没有什么希奇!”

贡春树听了不觉鼻端出火,心上生烟,正要发作,忽然转一个念头道:不好,不好,他们这般流氓都是些无法无天的宝货,更兼他们人多,我只得一个人,吃了他们的眼前亏,却到那里去翻他们的本?只好暂时忍住了,去把章秋谷找到这个地方来,给他们一个利害,也叫他们晓得我不是个好欺的人。想着便忍气吞声,也不开口,回过身来往下便走。只听得那一班流氓大家拍手笑道:“像他这样的一个饭桶,也要想来管我们的闲事!”贡春树虽然听得,却也无可如何,只得装着没有听得的一般,往前急走,径到久安里来寻章秋谷。这且按下不提。

看官,你道这个杨慕陶是何等样人?这件事情究竟是怎么的一个缘故?待在下做书的慢慢的演说出来。原来这个杨慕陶本来是上海本城人氏,和贡春树是姨表兄弟,却生得目秀眉清,唇红齿白,和贡春树的面貌狠有些儿相象。上海地方本来是个繁华世界,极乐洞天,杨慕陶幼年丧父,没有人管束他,成天成夜的只在嫖赌场中混搅。搅得久了,学着那一班滑头少年的习气,一天到晚只晓得到处看看女人,吊吊膀子,没有一些儿正经事情。偏偏的这个杨慕陶又是个色中饿鬼,只要看见了个面貌好些的妇女,一定要千方百计、钻头觅缝的去转他的念头。以前章秋谷和贡春树初到上海的时候,杨慕陶也同在一起吃过几台花酒。后来秋谷见他滑头滑脑的,满嘴大话,一身油气,觉得有些可厌,便不狠和他来往。杨慕陶见了秋谷却倒十分敬重,加倍恭维。秋谷有些不好意思,便也只好淡淡的应酬应酬他。贡春树听了秋谷的话儿,便也和他不甚亲热。好在杨慕陶的朋友狠多,也不把这件事儿放在心上,只顾忙忙碌碌的赶他自己的正经。到了夏天,也一般的同着一班朋友,天天坐着马车到张园去乘凉,借着这个乘凉的名儿,施展他那吊膀子的手段。

这一天恰逢七月七夕,又正是礼拜,张园的园主人定做了几套双星渡河的焰火在园里头施放。这一天晚上的人果然来得十分拥挤,杨慕陶也同着几个朋友同到张园。杨慕陶的意思,原不是专为要看焰火来的,便先往草地上四周围转了一回,仔仔细细的打量那班来的女客。觉得虽然一个个粉艳脂香,描眉画鬓,却都是些平常材料,没有什么出色的在里头。正要回身坐下,忽然鼻孔中间闻着一股素馨花露的香味,顺着风直飏过来;接着两个淡妆少妇手挽手儿的走过来,恰恰在杨慕陶身旁擦过。杨慕陶急忙仔细看时,只见这两个人体格苗条,腰肢袅娜;一身香艳,满面春情。虽然灯光闪烁,又在树阴底下,看得不狠明白,却觉得面粉口朱,芳芬竟艳。

两个人一面走着,一面低低的讲话,也不知讲些什么。

杨慕陶见了这样的两个尤物,不觉筋酥骨醉,意乱神迷,不由的口中“吱”的一声打了一个哨子。那两个少妇本来低着头走过去,没有留神杨慕陶这个人,如今听了这一声哨子,自然不期而合的一齐都回过头来。两对秋波注在杨慕陶身上细细的一看,不觉也都呆了一呆,对着杨慕陶嫣然展笑。杨慕陶是个风月丛中的老手,见了他们这般模样,便斜着眼睛瞟了他们一眼,把手中的一方白丝巾朝着他们轻轻的飏了几飏. 那两个少妇见了,又是微微一笑。转过身来走到草地,拣了一个僻静些儿的地方,两个人双双坐下。杨慕陶不分好歹,跟在他们身后,也紧紧的靠着他们两个的身旁拣张椅子坐下。

那张园到了夏间放焰火的时候,便把桌子、椅子,都搬在安垲第外草地上,预备来的客人好坐着看放焰火。那草地上没有灯火,都是黑沉沉的。虽然有一两盏电灯,却也照得隐隐约约的不狠清楚。杨慕陶趁着这个当儿,涎着脸儿便和这两个少妇说话。一面说着,却觉得心上突突的跳,自己也不知道说些什么。那两个少妇起先只微微的笑,不去理他,后来也渐渐的回答他一两句。

一回儿放起焰火来,那明火的光芒,照耀得满园里就如白昼一般。杨慕陶趁着这个光线,又细细的打量这两个少妇,越显得山眉水眼,粉颈香肩,腰细惊风,鬟低敛雾。两个少妇见杨慕陶细细的看他,便也抬起两双俊眼,也细细的看杨慕陶。

男看女如出水芙蓉,女看男如临风玉树。三心相印,六日偷窥,三个人你看着我,我看着你,竟看得呆了。连那放的焰火是怎么的一个样儿也没有看见。

那两个少妇坐了一回,和杨慕陶低低的说了几句话儿,立起身来先走。杨慕陶慢慢的跟在后边。这也总算是杨慕陶的修来夏福,左拥右抱,一箭双雕。双开姊妹之花,并织鸳鸯之锦。这些蝶亵的事情,在下做书的也不来细细的说他。

只说杨慕陶自从那一天以后,知道这两位宝贝就是那位江西巡抚康己生康中丞的堂房妹子,上海滩上有名的康姑太太,心上十分得意,差不多天天都在张园里头和这两位康姑太太相会。到得后来,索性明目张胆的三个人同坐一车招摇过市,连人也不避了。

就是这样的过了几时,这一天,杨慕陶同着这两位康姑太太在小房子里头过了一夜。直到明天十二点钟,三个人方才起身梳洗。猛然听得外面人声嘈杂,有几个人在外面叩门,叩得那门上的声音就如擂鼓一般。康姑太太叫过大姐下去问时,外面只说我们有要紧事儿来请杨少爷的。那大姐听了,便把门开了让他们进来。那知刚刚把门开得一扇,门外早拥进七八个少年男子来,身上都是长袍短褂的穿得十分齐整。拥进大门,不由分说一个个就往楼梯上跑。那开门的大姐见了他们这样,知道事情不好;连忙要想拦住他们,那里拦得住!只急得那大姐口中乱叫。

说时迟,那时快,这一班少年男子早走上楼梯,闯进房间。杨慕陶出其不意,那里躲闪得及!康大姑太太和康二姑太太两个人正在对镜梳头,一眼就看见了这几个人走进房来,心上吃了一惊,不由的目瞪口呆,做声不得。这一班少年男子里头,有两个为首的对着康姑太太冷笑一声。正是:

名花并蒂,猖狂昨夜之风;翡翠双栖,惆怅三珠之树。

要知后事如何,下文交代。





第一百二十二回 闹茶楼扬慕陶受窘 抱不平章秋谷解围





且说康姑太太见了那几个少年男子不由分说一直闯进房来,心上十分着急,口中说不出话来。有两个为首的男子看着康姑太太冷笑一声道:“你们三个倒得意得狠!”康姑太太听了,只低着个头不敢开口。杨慕陶见了他们声势汹汹的,更觉摸头不着。只见众人向着他高声喝道:“你是何等样人?竟敢擅入人家,图奸寡妇!

今天被我们真赃现获的捉住了,看你可还有什么法儿?“杨慕陶听了一时不敢开口,又不知道这一班宝贝究竟是康姑太太的什么人,只眼睁睁的看着姑太太的脸,要看他说出什么来。那里知道这两位康姑太太都红着个脸,一句话都说不出来。

杨慕陶见了这般模样,心上也不由得有些着急起来。又见众人都抢步过来,对他喝道:“你干了这样的事情,究竟打算怎么的一个主意?难道想就是这样的过去不成!”杨慕陶听了,没奈何只得说道:“我又不认得众位是这里的什么人,叫我打算什么主意呢?况且这里又不是我的地方,不过我和他们是亲戚,有时常常来往就是了。你们众位方才说的什么图奸孀妇,擅入人家,那里有这样的事情!你们众位不信,只顾问这里的主人就是了。”那一班人不等杨慕陶说完,大家都哈哈的笑道:“你这样掩耳盗铃的说话,想瞒那一个!你说只顾问这里的主人,如今两个主人都在这里,你自己去问一问,究竟你和他们是什么亲戚,等他们自己讲就是了。”

杨慕陶听了心中大喜,便走过来对着康姑太太说道:“这些人我一个都不认识。

我也不知道今天是怎么的一回事情。只请你们当着他们的面讲个明白,省得他们这般啰唣,传说出去在你们面上也不好听。“在杨慕陶的心上以为康姑太太一定是帮着自己一边说话的。那里知道康姑大太太和康二姑太太听了杨慕陶的这番说话,两个人都把头一低,红潮晕颊,默默无言。杨慕陶见了,心上十分着急,便又逼着问道:”怎么你们两位都不敢开口,这是个什么缘故呢?“康大姑太太和康二姑太太听了只当没有听见的一般,只低着个头,还是给他一个不开口。

杨慕陶到了这个时候,看了这样情形,不由的又急又气,一时倒也说不出什么来,只得回转身来想要走下楼去。那里走得脱?早被众人拉住,口中喝道:“到了这个时候,你还要想逃走,不要想昏了你的头!老实和你讲罢,你今天做出这样的事情来,你可知道是什么罪名?我们如今好好的和你讲话,还是留你的脸儿!如若不然,我们竟把你捆绑起来,送官追究治,你又有什么法儿呢?如今我们倒留了你的脸儿,你倒这样装腔做势的不肯自家服罪,还要满嘴混说什么亲戚不亲戚,你难道到了公堂上也敢这样的胡说不成?”

杨慕陶听了,心上觉得七横八竖的,狠有些儿胆小起来。呆了一回,只得说道:“你们要我自家认错,我就自己认个不是就是了。”众人听了,又大家冷笑几声道:“你说得好容易的话儿!难道这样一件事情,就是这般轻轻易易的认个不是就过去了不成?”杨慕陶着急道:“刚才你们众位自己说的,要我自家服罪。如今我认了不是,又说没有这般容易。依着你们众位的意思,要叫我怎么样呢?”众人道:“也不要你怎么样,只要你自己亲笔写个伏辨,只说不合图奸寡妇,擅入人家,今已自知悔过。以后如敢再犯,甘愿治罪。”

正说到这里,众人里头又有一个人高声说道:“慢些,慢些。这件事情这般办法还不见得妥当。这个伏辨也不过是个名色罢了。以后他就是再犯,我们这班人又从那里去查考他?不如罚他一千银子,叫他在伏辨上声明情愿罚充公款,也好借此儆戒儆戒他的下次。你们大家看怎样?”众人听了,自然大家都点头道好。便立逼着杨慕陶要他写个这伏辨。

杨慕陶这个时候虽然被他们搅得心上七颠八倒的,却究竟还有些儿主意,暗想:“这一千银子倒还不必说他,我也不穷在这千把银子。这个伏辨是万万写不得的。

万一个他们拿着了这张伏辨,常常的来和我歪缠起来,却叫我怎么样呢?“想着,便连连摇头道:”别样事情还好商量,这个伏辨是写不来的。我又没有犯什么法,为什么要我写伏辨呢?“众人见他不肯,一个个都横眉怒目的道:”你做出这样的事情来,还说没有犯法!如今我们也不来和你多讲,且到茶会上去评个理儿,再说别的!“说着大家不由分说,七手八脚的把杨慕陶推推拥拥的拉着就走。

杨慕陶还想康姑太太和他出头讲话,不料这两位康姑太太平日之间说起话来好像那会叫的画眉一般,凭你什么人也说他不过;不知怎么的到了这个时候闭口无言,一个字儿都说不出,凭着这班宝贝在那里夹七夹八的混闹,只是不敢开口。杨慕陶见了觉得心上十分纳罕,却又不知道究竟是怎样的一件事情,只得由着众人把他半推半搡的拥下楼去。又见有两个二十多岁的少年男子走进康姑太太身旁,不知说了几句什么话儿。杨慕陶见了心上甚是诧异,又不好去问他,只得同着他们一同坐上东洋车到大观园来。大家纷纷扰扰的闹了一回。

杨慕陶一个人那里说他们得过?正在着急,忽然见贡春树立在一旁,不由心中大喜。连忙叫住他,要想把这件事儿告诉了他,请他帮一个忙。不料那班人不讲道理,连贡春树也碰了一个钉子。杨慕陶见了着急非常,心上七上八落的想不出个主意。

看官,你道这班宝贝到底是康姑太太的什么人?为什么无缘无故的平空和杨慕陶为难?康姑太太见了他们这几个人何以竟不敢开口,凭着他们去这般混闹?这是个什么缘故呢?原来这两个为首的少年,一个姓李叫李洛卿,一个姓林叫林柱甫。

平日间和这两位康姑太太也有些不尬不尴的首尾。自从康姑太太姊妹两个认得杨慕陶以后,山盟海誓,对影闻声,未免和李洛卿、林柱甫生疏起来。李洛卿和林柱甫起先还不晓得是什么缘故呢,便细细在外面探听,方才知道这件事情。两个人由妒生醋,由醋生恨,便大家商量着要和杨慕陶为难。李洛卿、林柱甫这两个人,本来是个破落户的绅衿子弟,平日交接的朋友不是流氓,就是滑头,那里有什么好好的人物!听了李洛卿和林柱甫的话儿,便如此如此的商议出一个法儿来。候着杨慕陶和康姑太太在小房子里头相会的时候,叫门进去,一直闯进房门。

康姑太太虽然口角伶俐,蓦然之间见了这两个人的脸儿,一时满面通红,腾挪不得。看着这时候雨横风狂的暴客,便是那时间香温玉软的萧郎,旧雨归来,新人惆怅,凭着康姑太太的脸皮再老些儿,也忍不住十分惭愧。一个是今日的画眉夫婿,两个是当时的傅粉郎君,真个是左右为难,一身无主。你叫这两位康姑太太究竟帮了那一个的好呢?况且看着这李洛卿和林柱甫的模样,声势汹汹,明晓得是他们和杨慕陶吃醋吃出来的事情,自己若再要帮着杨慕陶的一边讲话,今天这件事情一定要闹出笑话来。虽然不怕什么,究竟于声名上有些妨碍,只好一言不发,凭着他们去糊里糊涂的混闹。

在李洛卿、林柱甫两个人的心上,却也并不是一定要来捉什么奸;不过和杨慕陶吃醋,想要出出气儿,大大的吓他一下,借此敲他一下竹杠,叫他知道了利害,以后不敢再来。好在杨慕陶虽然是个老上海,却究竟还有些纨挎子弟的习气,不懂外面的事情,被他们一吓就吓倒了。当下李洛卿和林柱甫两个人见杨慕陶入了他们的圈套,心中大喜,便越发扬威耀武的要写伏辨、要逼罚款。

杨慕陶被他们逼得无可如何,正在心上二十四分的惶急,忽听得楼梯声响,贡春树同着章秋谷两个人一前一后匆匆的跑上楼来。杨慕陶见了章秋谷,不由得心中大喜,连忙高叫道:“秋谷先生,请这边坐!”原来杨慕陶知道章秋谷生平好事,最喜欢和人排难解纷,见贡春树同了秋谷上来,早已料定是贡春树特地去请来的了,登时心上就放了几分。只见章秋谷大踏步直走过来,对着杨慕陶只把头略略的点了一点,也不坐下,便大声问道:“你们在这里闹些什么?为着什么事儿?快些和我讲个明白!”众人见了章秋谷仪容俊伟,举止轩昂,凤目含威,长眉隐秀,料想这个人有些来历,比不得别人,便也不敢得罪他。只大家眼睁睁的都看着章秋谷一个,看他说出些什么来。

杨慕陶听得秋谷问他,便细细地把这件事情的始末和秋谷说了一遍,却瞒过了和康姑太太相好的一段事儿。只说本来和康家有些亲戚,今天偶然去看看他们,就闹出这样的事来。秋谷听了心上早已明白,只微微的冷笑,口中说道:“你的事情也不用瞒我,这个时候也没有工夫和你多讲,等回儿再和你说就是了。”说罢,便回过身来对着众人说道:“马路有马路的规矩,你们众位在这里闹些什么?”

众人听了章秋谷的话风利害,大家都呆了一呆,李洛卿便勉强说道:“我们有我们的事情,不与你相干。请你不用多管闲事。”秋谷冷笑一声道:“天下人管天下的事,什么多管不多管!况且千差万差,旁人不差。你们不分好歹,连旁人都得罪起来,这是什么原故?”正是:

韦郎无恙,春风之眉黛新描;旧雨重来,昨夜之星辰如故。

要知后事如何,下回分解。





第一百二十三回 大观园流氓争口舌 乐仁里名士见秋娘





且说章秋谷对李洛卿和林柱甫两个人说道:“天下的人管天下的事情,为什么不好管你们的闲事?况且你们既然叫人不要管你们的闲事,你们又为什么管他们的闲事呢?”李洛卿和林柱甫听了,呆了一回方才说道:“我们和康家是亲戚,不得不和他帮个忙儿。”秋谷冷笑道:“康家的事情,自有姓康的人出来说话,与你们什么相干?”李洛卿听了,一时回答不出来,停了一停道:“这件事情本来原与我们无涉,是姓康的托我们出来说话的。”秋谷又冷笑道:“别样事情,托个旁人出来料理也还罢了,这样的事情怎么也托起旁人来?那有这般道理!如今这些话儿也不必说他,只问你们诸位,把杨慕陶兄挤在这个地方,是什么意思呢?”李洛卿和林柱甫听了,便抢着说道:“我们的意思也不是一定要他怎样,只叫他写一个悔过的伏辨也就算了。”秋谷不慌不忙的说道:“为什么要叫他写悔过的伏辨呢?”林柱甫不等李洛卿开口,连忙说道:“他图奸寡妇,擅入人家。”秋谷不等林柱甫说完,接下去问道:“他图奸孀妇,擅入人家,可有什么凭据?”众人齐声答道:“我们都亲眼看见的。我们这几个人都是凭据。”秋谷道:“捉贼捉赃,捉奸捉双。

你们究竟有什么实在的证据没有?你们众人嘴里头的话儿是不能算凭据的。“

众人见章秋谷驳得认真,大家都发怒起来。有一个十八九岁的少年跳起身来,一直抢到章秋谷面前,指手画脚的说道:“那里跑出这样的一个人来,也来多管闲事!我劝你还是省事些儿的好!如若不然,我们大家就要对你不起了!”秋谷看了他们这一班饭桶,明晓得都是些没用的东西,那里把他们放在心上,站在那里屹然不动。一面大声说道:“你们对我不起便怎么样呢?像你们这样的一班饭桶,我要怕了你们,连上海滩上也不用住了!”

众人听了章秋谷这样的藐视他们,由不得一个个心中大怒。李洛卿倚着自己有些蛮力,便抢上一步把秋谷劈胸一搡,口中说道:“给我走你的清秋路罢!”好个章秋谷,忙者不会,会者不忙,略略的把身体一偏,右手接住了李洛卿的手腕轻轻的一拧,拧得李洛卿“阿呀”一声;接着又把他轻轻一推,李洛卿立脚不住,连连的往后倒退,踉踉跄跄的一直退到他自己坐的一张椅子上方才坐下。秋谷冷笑道:“这样不中用,也来和我动手动脚。我好好的和你们讲理,你们偏要和我动粗。你们有胆子的只顾上来。不要说你们这七八个人,就是再多些儿,我也不把你们放在心上!”

众人见李洛卿吃了个败仗,又听秋谷这般说法,虽然一个个心中不服,却都不敢动手。章秋谷等了一回,不见他们开口,便又微微冷笑道:“原来你们的本事也不过这般,刚才又何必这样的装腔做势呢!”众人听了都面上通红,说不出一句话来。林柱甫只得勉强说道:“你老兄不必动气,我们有话好好的讲就是了。刚才原是他们一时性急,请你老兄原谅些儿。”秋谷道:“你们既要和我讲理,我就和你们讲理;你们有什么话,只顾大家商议就是了。”

林柱甫到了这个时候,知道章秋谷不是好惹的人物,便恭恭敬敬的请他坐下吃茶,又请问秋谷的姓名。秋谷不耐烦和他多讲,便道:“如今闲话少说,据你们众位的意思,究竟要杨慕陶兄怎样,方才肯了结这件事情呢?”林柱甫道:“他做了这样的事情,若就是这样的放他过去,天下也没有这样便宜的事情。就是看在你老兄的分上,不要他写伏辨,也要罚他拿出一笔钱来算作罚款,方可了结这件事儿。”

秋谷听了,不觉哈哈的笑道:“算了罢,不用说了。这个事情办不到的。据你们说起来,不过说姓杨的图奸寡妇,擅入人家。你可知道,律例上头载得明明白白的,叫做‘指奸勿论’。就使到了公堂上,也要本人到案,指证明白,方才可以照例治罪。那里有这样糊里糊涂,只凭着你们一面的话,就好定案的道理?况且你们既不是在奸所捉获的,又没有什么蝶狎嬉笑的情形,你们又何以知道他是图奸寡妇,就一口咬定了他呢?”章秋谷说到这里,林柱甫连忙说道:“你这几句话儿错了。

他图奸未成,当场捉获,这是有凭有据,众目共见的。康家的人和他并没有什么首尾,你不要认错了人。“秋谷道:”依着你们的话儿,竟算他是图奸未成,当场捉获。该应姓康的有人出来把他送官究治,和你们什么相干?难道这样的事情,也好请旁人出来替代的么?“

林柱甫和众人听了这一番说话,一个个面面相看,一言不发。秋谷又道:“老实和你们讲罢,就算姓杨的和康家的人有什么暖昧不明的形迹,你们也不是可以出来讲话的人!这样的事情,除了本夫之外,只有父母家长方才可以出来说话,就是兄弟至戚也不能多讲一句话儿。你们一非本夫,二非家长,怎样好出来管人家这样的事情?安知你不是有什么意外的仇恨,挟嫌诬蔑,借此报仇呢?我说句不怕你们见怪的话儿,像这样的事情,到了公堂上只怕没有断定别人的罪名,先把你们几个问个挟嫌生事、聚众拆梢呢!你们可知道马路章程?在茶坊酒肆聚众滋闹,是外国人最恨的。只怕到了那个时候,你们想要就是这般太太平平的过去也是不能的了。

依我的言,相劝你们还是省些烦碎,把这件事儿就是这样的一笔抹倒,一概不提,省得将来闹出什么乱子来大家面上都不好看。“

众人听了章秋谷这番说话,不觉大家目瞪口呆。眼看着一块好好的肥羊肉已经到口,平空走出这么的一个章秋谷来,把他们的肥羊肉从口中抢了出去,一个个心上恨得要死。无奈听着这番说话又是实在不差,本来这样的事儿原只好骗骗杨慕陶,却那里骗得过章秋谷!大家都眼睁睁的看着秋谷的脸儿,要看他究竟怎样。

只见章秋谷霍的立起身来,对着众人说道:“今天总算我姓章的出来排解一场,这里的茶钱,一古脑儿都归我给就是了!”说着,从身上掏出一张五块钱的钞票放在桌上,左手挽着贡春树,右手拉着杨慕陶,口中只说一声:“你们众位不要见怪,我们失陪了。”一面说着,大踏步往楼下便走。众人见了,拦又不是,不拦又不是。

别人也还罢了,只有李洛卿和林柱甫更加着急。两个人不分好歹,抢在章秋谷面前拦住去路。林柱甫陪着笑,口中说道:“请略停一步,我们还有话讲。”秋谷微笑道:“我的话已经讲完,再讲也不过是这几句话儿。你们不用拦阻,就拦阻也不中用。”李洛卿、林柱甫那里肯放!秋谷又笑道:“你们不要这样拉拉扯扯的,马路上斗殴,是犯规矩的。等回儿闹得巡捕来了,我是有名片的,只怕你们就要吃亏了。”

说着放了贡春树和杨慕陶,两手轻轻一分,在章秋谷不过是用了一二分气力,李洛卿和林柱甫已经东倒西歪,立脚不住,没奈何,只得让在一旁。章秋谷回过头来对着贡春树和杨慕陶道:“你们都跟我来。”三个人大摇大摆的走下楼来,竟没有人敢再来拦阻。

秋谷刚刚走到门口,早听得楼上在那里乱嚷乱骂,嚷成一片。章秋谷眉头一皱,便问贡春树道:“今天这件事儿,平空的被他们骂上几句,是你作成我的好生意!”

春树还没有开口,杨慕陶忙连连拱手,深致不安道:“总是为着兄弟的事情,实在不安得狠。要是今天没有秋谷先生来和兄弟解这个围,那就了不得了!”秋谷也谦让了几句。春树插口说道:“他们的骂人,就和那驴鸣狗吠一般,那有这样的工夫去听他。”秋谷听了一笑,便同着他们两个,同到久安里陆丽娟那里坐了一回。

杨慕陶千恩万谢的说了许多感恩图报的话儿。秋谷道:“朋友的事情本来理应相助,算不得什么。倒是你怎么平空的会去吊上了这两个宝货的膀子?”杨慕陶听了不觉面上一红,口中还想支吾。秋谷笑道:“你不用瞒我,你只和我从实细讲就是了。”杨慕陶听了,知道瞒他不过,便从头至尾细细说了一遍。又道:“这一班流氓也不知是他们两个的什人,他们见了那两个为首的人,好像狠有些怕他们的样儿”

秋谷听了早已心中明白,只微微一笑,也不开口。却对着杨慕陶说道:“今天我的意思,要和你同去见见你们那两位贵相好,不知你答应不答应?他们既然和你要好,看着他们一班流氓把你拥了出去,一定心上狠不放心的;你也应该去给他们一个信儿,省得他们心上记念。”杨慕陶听了满口答应,便同着章秋谷和贡春树一同到后马路乐仁里二弄一个门口。杨慕陶叫秋谷和春树略等一等,自己敲门进去。

秋谷同春树站在门外等了一回,方才见杨慕陶走出门来,请他们两个人进去。

上了扶梯,走进房间,早见两个淡妆少妇袅袅婷婷的立在门内,见秋谷和春树两个人同走进来,便朝着他们一笑,说了一声:“请坐。”秋谷是本来认得这两位宝货的,现在不免又细细的把他们打量一回,见他们虽然差不多都有三十余岁,却还是细腰长腕,皓齿明眸,看上去也不过二十几岁的样儿。便把方才在大观园的情形略略的和他们说了一遍。又说:“据我看来,既然闹了这个乱子,这个地方是住不得的了,还是换个地方秘密些儿的好。万一他们有心寻事,三更半夜的打了进来,虽然不怕他,却究竟面上下不去。”正是:

徐娘半老,犹为堕马之妆;孙寿中年,尚作回风之舞。

要知后事如何,请看下文便知分解。





第一百二十四回 王素秋家庭翻醋瓮 康已生中冓咏新台





且说康大姑太太和康二姑太太听了章秋谷的话儿,免不得也谢他几句;一面偷转秋波,细细的打量他们两个。看着这样的两个少年男子,一个是玉山朗朗,华彩非常;一个是琪树亭亭,丰姿照夜。杨慕陶生得虽然俊俏,和他们两个人立在一起,就觉得差了好些。康姑太太看了又看,不觉心上狠有些儿羡慕的意思,便把两对秋波只顾望着秋谷、春树这边溜来。秋谷虽然看见,却故意别过头去和春树说话。

只听得杨慕陶问着康姑太太道:“方才那一班流氓,究竟是你们的什么人?你们为什么都这样怕他?”康姑太太还没有开口,章秋谷早接着讲道:“你这个人真是有些糊涂。这班宝货那里有什么好人,无非总是大家赌气赌出来的事情,你又何必去问他!”康姑太太听了这几句话儿不觉面上一红,低下头去。杨慕陶听了也不觉恍然大悟,心中彻底皆明。暗想我这个人怎么这样糊里糊涂的,一时竟想不出来。

章秋谷说了几句闲话,便立起身来对着康姑太太讲道:“他们那班人都不是什么好货,今天吃了下风,一定要想着法儿来报复你们的。不如今天就把这几间房子还了房东,随后慢慢的再找地方,觉得妥当些儿。你们的意思看怎么样?”杨慕陶听了连连答应。康姑太太见秋谷同着春树立起身来要走,心上未免有情,明知道留不住的,只得起身相送。横波一瞥,脉脉含情,看着贡春树、章秋谷两个人出门走了,方才回身进来。

果然听着章秋谷的话儿,立刻把房子还了房东。有些动用器具没有安放的地方,便和房主人说明了,暂时寄放。好在房租已经付到月底,这些器具暂时存放一下也不要紧。料理了一回,又和杨慕陶说了几句话儿,叫他在外面另寻房子。杨慕陶答应了,便起身先走。

康大姑太太和康二姑太太便也慢慢的回到虹口康公馆来。刚刚走到花厅,就听得里面有许多人的声音在那里吵闹,又夹着些女子的哭声。康姑太太听了,心上甚是疑惑,不知道闹的什么事儿,便连忙赶过去看。急急忙的走过一重院子,那吵嚷的声音直钻进耳朵里来,听得十分真切。只听得大姨太太的声气在那里哭着乱嚷道:“你这样一把年纪,还是这样的不要脸,成天的和那些娘姨、大姐拉拉扯扯的混闹。这还不必讲他。如今索性连自己的媳妇也要拉拉扯扯起来,那里还像个人家!

我虽然是堂子里头出身,眼睛里头却从来没有看见你们这样的一家人家,不论上上下下、大大小小的,都是嘻嘻哈哈的没有一些儿规矩。“一面说着,又有许多丫鬟娘姨的声音,七张八嘴的劝道:”大姨太太,不要气坏了自己的身体,有话好好的讲就是了。“

康姑太太听了,见闹得这般利害,连忙走进去看时,只见那位大姨太太紧紧的一把揪住了康中丞的胸前衣服,把头往康中丞身上乱撞;一把眼泪一把鼻涕的,口口声声的只说:“你把绳子来勒死了我,省得在你面前讨你的厌!”康中丞被这位大姨太太一阵的乱揪乱扭,弄得没了主意,只说:“你放了手,有话好好的讲。如今做出这个样儿来,给人家看了算什么样儿!”大姨太太那里肯放,只滚得髻鬟散乱,粉黛模糊,那流下来的涕泪,连康中丞的花白胡须上也沾了好些,身上的衣服更湿了一大爿。七八个丫头、娘姨在旁边拉着,也拉不开来。康中丞虽然着急,却又无可如何。

康姑太太见了这般模样,心上狠有些怪着大姨太太不该应闹到这步田地。便抢步上去,一边一个拉开了大姨太太,捺他坐下,口中说道:“什么事儿,闹得这样天翻地覆的?且把这件事儿讲给我们听听。”大姨太太听了,便又在椅子上立起身来,含着一泡眼泪告诉康姑太太道:“他这样的一把年纪,也是五十几岁、将及六十岁的人了,还是这样的没正经;在别人身上也还罢了,自己的媳妇也和他眉来眼去的,做出那种贼形怪状来。我看在眼睛里头已经不是一天了,劝了他几次,他只当没有听见。今天索性两个人在内书房里头动手动脚起来。我走进去说了几句,他不但不听,倒反和我横跳一丈、竖跳八尺的闹起来。你们想想,可有这般道理?”

康姑太太听了正在沉吟,康中丞觉得脸上过不去,便连忙说道:“没有这件事情。我不过和二少奶奶说了几句话儿,他一时看错,就和我闹起来。”大姨太太听了,又抢过来拉着康中丞的衣袖说道:“你没有这件事情,是我冤枉你的?我和你当天发一个誓好不好!”康姑太太见了,连忙分开了大姨太太的手,劝他道:“你不必这般生气,凡事只好忍耐些儿。就算果然真有这件事情,你也不便这般吵闹,传出去给人知道,我们这样人家将来还有什么脸见人?”大姨太太听了,一时说不出什么话来,只得说道:“我的意思,原想不要闹出来的,无奈我只说了一句,他倒瞪着眼睛、提起喉咙和我寻事,把我的气提了上来,方才和他翻脸的。你们想想,究竟是我不是还是他的不是?”康姑太太道:“自然是他的不是,那里有派你不是的道理?但是这样的事情传了出去,也没有什么好听,还是好好的劝他为是。”

大姨太太听了,觉得这几句话儿说得不差。况且平日之间,大姨太太不怕别人,见了这两位姑太太心酸口辣,说又说得出,做又做得出,心上狠有些馁他。更兼这件事情,仔细想起来实在是自己性急了些,不该闹得合府皆知的,便也只得点头说好。康姑太太又安慰了他一回,又劝说了康中丞几句。康中丞也没有话说。

康姑太太正要回到自己房间里去,忽然想起一件事情来,便问:“二少奶奶到那里去了?”康中丞道:“他只说我们有意和他过不去,当时就坐着马车走回娘家去了。”康姑太太想了一想道:“这件事情不妥当。无论这个事儿的有没有,始终没有什么凭据,回来他叫了娘家的人出来和我们讲起理来,只说我们污蔑他的名节,那时又该怎样呢?”康中丞听了也把手一拍道:“这个话儿不错。该应怎么的一个说法呢?只好请你们两位和我想个法儿的了!”康大姑太太听了,低着头沉吟一会道:“据我看来,不如立刻派个人去和他讲明白了,说刚才大姨太太的话儿不是说他,他不要认错了。一则过过他的面子,二则总算和他赔个礼儿。只要他面上过得去,自然也就罢了。”康中丞道:“这个主意虽然不错,却派那一个去说呢?要是派个不会说话的人去,万一个说僵了,更不好。”说着,想了一想,便对康姑太太说道:“这个媒人,本来是你们二位做的,只好请你们两位去走一趟的了。”康姑太太听了,义不容辞,只得点头应允。

康中丞道:“要去这个时候就去。要是迟到明天,他们那里有人先来说话,我们这边的话儿就难讲了。”康姑太太听了,便走回自己房间去打扮了一回,两个人坐着马车,去了多时方才回来。

康中丞见他们来了,分外关心,连忙问他们怎么样。康姑太太笑道:“费了我们两个人许多唇舌,他们方才没有话说。只说留他在家里头住上几日,再打发他回来。”康中丞听了,便立起身来,朝着他们两个深深打了一拱,口中说道:“一切费心得狠。”康大姑太太和康二姑太太见了康中丞这般形景,忍不住“格格”的笑,还了一个万福道:“我们自己人,还说这个么!”说着坐了一回,便都走了出去。

康中丞见房里头静悄悄的一个人也没有,少不得要在大姨太太面前做个矮人,陪个不是。大姨太太起先背过脸去,不肯理他。康中丞左打一拱,右打一拱的,口中说了许多软话,方才把大姨太太的气骗了下来,“嗤”的笑了一声道:“你不用在我面前做这般的腔调。我不是喜欢这个样儿的!”康中丞见他笑了,心上方才高兴,便想出许多说话来骗他。

大姨太太见他这样的陪小心,便故意问他道:“你不要对着我花言巧语,你只和我实说,你和他究竟上过手没有?”康中丞也故意装糊涂道:“你问的那一个?

什么上手不上手?“大姨太太冷笑一声,又咬着牙齿把一个指头用力在康中丞头上点了一点道:”你还要和我装糊涂!难道今天我看得这样的明明白白,你还要假装干净么?“康中丞也笑道:”你要说出究竟是那一个来,也好叫我自己心中明白。

你如今只是含着皮、包着骨头的不肯说出来,叫我那里想得到呢?“大姨太太听了,气得把颈项一扭,别转头去,口中说道:”你不肯和我讲,你就赌个气儿,从此以后不要和我讲一句话!那一个再要和我讲话的,便是个没志气的畜生!“

康中丞见他又生了气,便连忙说道:“你这个人,怎么这般的会生气。和你说一句顽话,你就当起真来。老实和你讲罢,我和他虽然彼此有些意思,只不过大家讲几句笑话罢了,实在没有别的事情。你不相信,咒都可以赌得的。”大姨太太听了,知道不是假话,便道:“还说是世代乡绅的千金小姐,做出这样的事情来,以后看他把脸放在什么地方去!我们堂子出身的人,只要嫁了人,倒是规规矩矩的,也没有他这般轻贱。”康中丞连忙朝他摇手道:“和你说了,你又这般混闹。请你少说几句,留我点儿面子罢!”大姨太太听了,停了一回道:“原来你也知道要面子的么?如今第二个新媳妇差不多又要进门了,你再去扒灰去罢!”急得康中丞摆手顿足的道:“叫你少说两句,你越发说出好听的来了!”正是:

河水新台之咏,老子风流;墙茨中荐之羞,佳人难得。

不知后事如何,且看下文分解。





第一百二十五回 闹花厅白昼敦伦 闯深闺黄昏惊梦





且说大姨太太自从和康中丞闹了一回之后,康中丞陪了无数的小心,认了许多的不是,方才夫妇如初。康中丞也忙忙碌碌的打点要和第三个儿子娶媳妇。

原来康中丞只生一个女儿、两个儿子。女儿到八九岁上就一病死了,如今只存两个儿子。第二个儿子娶了媳妇,已经过了几年,现在第三个儿子也长成了,便和他择日迎亲。里头的事情,都是大姨太太一个人料理。但是大姨太太本来是个堂子出身的倌人,嫁娶的规矩那里懂得,只得请了两位姑太太来帮忙。外头的事情,自有那一班走狗和他料理。

闹了几天,到了吉期。康公馆里头摆设得绿舞红飞,花团锦簇,真个是笙歌匝地,灯火连云,堂开玳瑁之筵,褥隐芙蓉之绣。那些官场商界的客人,纷纷扰扰的往来不绝。一回吉期已到,一乘花轿,几对仪仗,把新媳妇娶了过来。一切坐床撒帐、交拜庙见的这些礼节,料想看官们也都懂得,用不着在下做书的来铺排。

只说康中丞见了这位新娶的媳妇,丰神活泼,体态娇娆,比那位二少奶奶还要胜过几分;更兼性情宽厚,待人和气。真个是俊眼乍回,春云偷展,朱唇未启,巧笑先闻。康中丞看了,便也十分得意。康中丞这位公郎,娶着了这般一个尤物,自然的夫妻恩爱,鱼水缠绵,恨不得把两个身体捏作一团,并成一块。

康公馆的房子本来狠宽,有三间小小的花厅,四周都种着些梧桐竹子,窗明几净,花木参差,是康中丞向来会客的地方。花厅后面隔着小小三间翻轩,这个地方康中丞就叫他内签押房。本来这个签押房的名目,是签押公事的地方,不是现任官员、就是现有差使的人,方才用得着。如今康中丞既不做官,又不当差,简直的叫他内书房就是了,为什么还要叫做什么签押房?原来这个康中丞生有官癖,此番自己奏请开缺,原是不得已的举动,心上总存着个希冀起用的意思,所以把内书房叫作内签押房。平日之间除了见客和休的时候,看书写字都在这个内签押房里头。自从娶了这位三少奶奶回来之后,康中丞一向忙忙碌碌的,有好几天不到内签押房去。

这个当儿忽然接到了京城里头吕大军机的一封来信,康中丞拆开看了一看,连忙到内签押房去写回信。为着这封信上的话儿是要和他代谋起复,恐怕家人们闯进来看见了,传出去不便,便把内签押房的门关得紧紧的,吩咐一班家人许进来。自己一个人坐在内签押房里头,悄没声儿的在那里想着怎样的写回信。

想了一回,只听得外面“吉吉各各”的弓鞋细碎的声直走到花厅上来。康中丞不知道是什么人,便由他在外面,自己却一言不发。等了一回,又听得轻轻的一声咳嗽。康中丞听得真切,知道不是别人,正是那位新娶来的三少奶奶,不觉心中一动,便躲在里面一言不发。只听得那位三少奶奶口中自言自语的说道:“还是这几间房子造得比别处好些。”康中丞正在那里侧着耳朵听他说话,忽听得外面又来了一个男子的声音,朗然说道:“今天怎么你跑到这里来了!”康中丞听着这个说话的声气就是他那位令郎,心上便一个没趣,只得索性不响。听得三少奶奶笑道:“今天你出去了,我一个人觉得有些烦闷,闲着没有事情,所以出来各处走走。”

那位三少爷也笑道:“这里是老头子会客的地方,今天老头子出去了,所以这样静悄悄的。”

康中丞在里面听着,心上暗想道:“他见我关着门,只说我出门去了,我倒要躲在这里,看看他们两个人做些什么。”想着便轻轻的蹑着脚步走到门口,在门缝里头看时,只见他那位令郎和那位三少奶奶本来两个人并肩坐在一处的,忽然间三少爷附着三少奶奶的耳朵不知说了一句什么,三少奶奶“格支”一笑,举起手来打了三少爷一下。三少爷道:“这里又没有人,怕什么?这个地方只要老头子出去了,是没有一个人来的。”三少奶奶道:“我不要,你便怎么样呢?”三少爷笑道:“你不要也由不得你!”说着便走过去把门帘放下,关上了门,走过来不由分说,轻轻的一把把三少奶奶抱了起来。两个人霎时间并蒂花开,鸳鸯梦稳;尤云碲雨,倒凤颠鸾。只把一个里面的康中丞气得软作一团。看着这两个宝贝这样的风流放诞,青天白日的竟在花厅上串起戏来,你叫他怎的不气?

当下康中丞赌气掩过一边,不去看他,只听得两个人“支支格格”的笑作一团。

停了一回,康中丞忍不住又去看他。只见三少爷又把三少奶奶抱起来,坐在肩上,就和那堂子里头的相帮掮着倌人的一般,掮着满厅乱走。康中丞在里面看着,又好气又好笑。不料那位三少爷走了一回,走得高兴起来,竟自走到内签押房门口,“呀”的推开了门,就要进来。这一下子,把康中丞大大的吃了一惊,一时无可如何。人急智生,便想出一个法子来,只当他是家人送茶进来的样儿,口中喝道:“我不要吃茶,端进来做什么,给我端出去!”那位三少爷不听这几句话儿便罢,听了这几句话儿,这一惊倒也非同小可!不管三七二十一,回转身来没命的往外乱跑。三少奶奶也吃了一惊,又羞又怕,由不得身体一歪,在三少爷肩上直跌下来,跌得他“阿呀”一声,遍身酸痛,连弓鞋都跌掉了一只。三少爷见了,心上更加着急,也顾不得他跌痛了那里,连忙一把拉了起来,两个人飞一般的拉开了门,往着上房逃去。

这个时候,刚刚大姨太太打发两个大姐出来寻康中丞,不知有什么话说,奇巧不巧的,和三少爷、三少奶奶碰了一个正着。只见这位三少奶奶衣裳不整,鬓发蓬松,同着三少爷拼命的往里面跑。这两个大姐见了,心上十分诧异;走到花厅上,又见地上落下一只弓鞋,知道是三少奶奶的,顿时大家传说起来,一个公馆里头的人没有一个不知道这件事情。

康中丞躲在里面,眼睁睁的看着这两位宝贝走了出去,方才叹一口气,走了出来。劈面又撞着了这两个大姐,知道他们已经看见,又没有本事按住他们的嘴,叫他们不要声张,只得装痴做聋的,凭着他们去大家传说。自己对着大姨太太,也免不得把这件事儿和他细细的说上一番。大姨太太倒笑了一会,又埋怨他不应该惊动他们。你只悄悄的躲在里面不要作声,等他们走了再出来,就闹不出这般笑话来了。

康中丞顿着脚道:“你倒说得好风凉的话儿!我起先原是躲在里面不敢作声的,到了后来,这两个宝贝不分好歹,竟要闯到里面来,我若再不开口,他们就要走进来了,你想可有什么法子呢!”大姨太太听了也没有话说。

那三少爷和三少奶奶两个宝贝,自从闹了这个笑话以后,觉得没脸见人,两个人只得装着生病,连房门都不出,一直躲在房间里头。躲了一个多月,方才老着脸皮出来见人。三少奶奶见了康中丞??还是满面通红的,连头都抬不起来。这件事儿传说开去,上海地方的人就把他当作笑话一般,茶坊酒肆讲的都是康中丞家的事情。

康中丞虽然知道,却又无可如何,只得借着事儿把他那位令郎骂了几场,打了一顿,方才罢了。

康中丞自从娶了位大姨太太之后,大姨太太拿出堂子里头骗人的本领来,把康中丞骗得伏伏贴贴,又爱又怕,一个月里头差不多有二十天住在大姨太太房里,那四个姨太太不过是挂个名儿罢了。倒是这位大姨太太有些不过意,劝着康中丞也到别个姨太太房里去应酬应酬。康中丞越发相信大姨太太是个天字第一号的正经人。

有一天大姨太太坐了马车出去,不知买什么东西。康中丞便踱到三姨太太房里头去,讲了一回闲话。大姨太太回来了,康中丞便坐在大姨太太房间里头,两个人说说笑笑的,康中丞十分高兴。正要收拾安睡,忽然想起日间有件马褂脱在三姨太太房里头,马褂袋里头有一封紧要电报,一时忘了收拾,便和大姨太太说了,要自己去拿。大姨太太道:“一件马褂,只要叫个人去拿来就是了,何必早要自己去拿?”

康中丞道:“我刚刚想起,今天还要到内签押房去写几封信,你只顾先睡就是了。”

说着,便立起身来往外便走。

一路走到三姨太太房门外面,静悄悄的不见一个人,康中丞口中说道:“怎么这些人都到那里去了,这里一个人都不见?”一面说着,一面跨进门来。只见这位三姨太太,两颊飞红,衣裳不整,一个人坐在房里的一张榻上;还有一个平日跑上房的家人胡德,慌慌张张的立在旁边。

康中丞见了不觉大诧道:“你们在这里做些什么?怎么房里头一个人也不见?”

又对胡德厉声说道:“你这个时候,一个人跑到这个地方来做什么?”吓得胡德诺诺连声,不敢开口。三姨太太慢慢的说道:“你不要骂他,是我叫他进来的。”康中丞听了,瞪了三姨太太一眼道:“你叫他进来做什么?虽然他是派值上房的,这个时候叫他进来,房里头又只有你一个人在这里,算什么样儿!”三姨太太不慌不忙的说道:“我今天发了肝气,痛得无可如何,三更半夜的,又不便惊天动地的乱闹。偏偏我平日吃的十香丸又没有了,没奈何只好叫他连夜去买,又怕他们说不明白,所以叫他进来,我自己吩咐他。你当是什么事儿,又是这样的动起气来!”说着,便把一双纤手捧紧了胸膛,口中哼个不住。正是:

惊破高唐之梦,好事多磨;吹残巫峡之云,襄王何处?

不知后事如何,且听下文交代。





第一百二十六回 感风寒中丞卧病 乱人伦令子宣劳





且说康中丞听了三姨太太的一番说话,心中半信半疑,心中暗想:又没有拿到什么证据,闹是料想闹不出的。又回过头来看着三姨太太那般模样,双蛾欲蹙,皓齿微呈,太真病肺之妍,西子捧心之态,不觉把一个心早软了一半。看着那胡德还站在那里一动也不敢动,便对他喝道:“你还不赶快去买丸药,站在那里做什么?”

胡德得不的这句话儿,好似得了赦书一般,连忙答应一声往外便走。

康中丞又问着三姨太太道:“你既然发了肝气,他们那些人都到什么地方去了?”

三姨太太一面哼着,一面抬起头来说道:“绿云、祥云两个,是我叫他们去拿开水的。还有几个,我就不知道他们到那里去了。”康中丞听了,低着头想了一想,便对三姨太太说道:“你以后须要留心些儿,不要这般大意。像今天这样事情,房间里头一个人也没有,就是你和胡德两个人。要是换了个疑心重些的人,已经不知闹到怎样的一步田地了。”三姨太太听了,娇怯怯的说道:“我发了肝气,痛得十分利害,那里还顾得房间里有人没有人。这都是他们贪懒,看见我病了,就一个个不知躲到那里去了,你还要向我说这样的话儿!难道你拿到了什么凭据么?”说着,皱着眉头把身体扭了几扭,连叫几声“阿呀”,一谷碌就倒在榻上。

康中丞见了这般做作,早把方才的一片疑心不知跑到那里去了,心上倒发起急来,连忙问道:“你到底什么地方痛,可要叫个人来和你捶一下子?”三姨太太听了也不开口,只把手对着自己的胸膛指了几指。康中丞看了,便自己走过来,就在榻旁坐下,把两只手替换着在三姨太太胸间轻轻摩抚。又把几个娘姨大姐都叫进房来,康中丞骂了他们几句道:“怎么三姨太太在这里生病,你们这班人一个都不来伺候!躲到什么地方去了?那里有这般规矩!”众人听了都呆了一呆,彼此做个眼色,便不开口。康中丞这一夜就住在三姨太太这边,倒伏侍了三姨太太一夜,这且不提。

只说康中丞的那位二令郎,今年已经二十九岁,官名一个杞字,号就叫少己。

从小的时候康中丞也延师教他读书,无奈康少己的质地鲁钝非常,竟比康中丞自己还加了一倍。读了整整的十五年书,连《十三经》都没有读完,写个寻常通候的条子也写不出来。康中丞气得要死,他自己却毫不放在心上,倒对着人说:“如今的做官只要有钱。我们老头子也是捐班出身,也做过一任江西巡抚。难道捐班出身的就是不是人么?”这句话儿传到康中丞耳朵里,康中丞听了心上虽然气忿,转过念头来一想,觉得也无可如何,只有这个法儿。便只得拿出钱来,和他捐了一个主事,到部里头去候补了几年,赔掉了无数的银钱,还闹了许多笑话。康中丞赌气把他叫了回来。

这位康少己到了上海,便花天酒地、朝歌夜弦的乱闹起来。偏偏的康少己肚子里头虽然没有一些儿墨水,外面的丰貌却生得漂亮非常,面子上的应酬又来得十分活泼。一班堂子里头的倌人,见了这位康二少爷,没有一个不喜欢的。康少己又专爱在女人面上用些工夫,献些殷勤。就是康中丞的那几位姨太太,见了康少己也都是十分亲热,格外殷勤,大家都有些跃跃欲试的意思。这位康少己本来也不是什么正经人物,看了几位姨太太这般模样,便也存了个代父从军的念头;却是回过念头来一想,始终觉得有些碍手碍脚的,不甚妥当。

自从那一回大姨太太为着二少奶奶的事情和康中丞闹了一回之后,虽然康中丞吩咐一班娘姨、大姐不许传说出去,都是同在一家的人,那里瞒得过?这个信息早传到康少己耳朵里头,不觉心中大怒。想道:这个老头子这样的不知廉耻!自己有了五个花枝一般的姨太太,还要调戏起自己的媳妇来!我倒留你的脸皮,不肯不分皂白的混搅,你倒这样的不顾人伦,那就怪不得我了!想着,又私地里把自己的老婆盘问一番。

这位二少奶奶本来是个外交名手,自然另外想出一番话来和他敷衍,把自己的不是一古脑儿都推在康中丞身上。只说康中丞时常要调戏他,想转他的念头。康少己听了老婆这样的一番话,自然气得双睛出火,七孔生烟,暴跳如雷的道:“这个老东西真个这般无耻!说不得我也顾不得许多,只好做到那里算到那里的了!他们五六十岁的老头儿尚且要这般混搅,我们年纪轻轻的人,更是分内的事情了!”自此以后,一直无话。

光阴迅速,早又是秋去冬来,朔风乍紧,霜气中人。康中丞偶然受了寒气,觉得头痛鼻塞,身体有些不快。康少己听得康中丞病了,虽然不把这件事儿放在心上,却这一点儿面子上的规矩不能不要,便也同着众人照例进去问安,淡淡的问了几句。

康中丞见了儿子来问他的病,不觉心上欢喜,就叫他坐在床沿上,和他讲些闲话。

这个时候,正有一个大姐煎好了一碗药递将上来。大姨太太便接在手中,二姨太太走过去,把康中丞扶了起来坐在床上,大姨太太把一碗药放在康中丞口边,康中丞自己一口一口的喝。康少己在旁见了,不知怎么忽然天良发动起来,连忙抢过去,在银吊子里头斟了半碗冰糖燕窝汤,自己拿着立在一旁,要等康中丞吃过了药给他过口。

不一时,康中丞一碗药已经吃毕,康少己端上茶来。康中丞吃了两口,忽然一眼看见康少己左手指头上光华闪烁,带着一个钻石戒指。那钻石差不多比那最大的黄豆还要大些。康中丞见了,心上早吃了一惊。记得这个戒指,是去年自己买给五姨太太的。买的时候着实地看过一番,又是时常见五姨太太戴在手上的,心上十分诧异,不由的开口问道:“你这个戒指是几时买的?脱下来给我看看。”

康少己出其不意,心中大吃一惊。不知不觉的全身一震,右手一松,拿不住茶碗,“豁啷啷”的一声跌在地下,连康中丞身上也泼了许多燕窝汤。康中丞看了这般模样,心中已经猜料了几分,便冷笑道:“什么事情这样慌慌张张的,把茶碗都跌下来?叫你把戒指脱下给我看一看,为什么急得这个样儿?”

康少己听了满面通红,口中支支吾吾的说不出话来,那心上好像有十五个吊桶在那里打水的一般,七上八下跳个不住。没奈何硬着头皮,在手上除下来递在康中丞手内。

康中丞接过来仔细看了一看,越看越像,不由得怒气填胸,胡须倒竖,勉强忍住了不发出来。只问着康少己道:“你这个在什么地方买的?花了多少钱?其实这些东西,都是女人的装饰品,我们堂堂男子何必要带这样东西呢?”康少己一时说不出话来,嗫嚅了一会方才说道:“这个东西是一个出洋的朋友送的。据他自己讲,在美国纽约买来的,花了二百五十元美金,合起我们中国的钱来,差不多也有五百块钱。”康中丞听了那里肯信,冷笑一声道:“你的那个朋友同你的交情倒狠好,居然送你这样贵重的东西!”康少己红着个脸答应不出。

康中丞正要骂他几句,忽然心上一想,虽然如此,究竟不知这件事情的真假何如。万一个没有这件事儿,不过偶然相像,惊天动地的吵闹起来什么意思?就使这件事儿竟是真的,家丑不可外扬,我自己先是这样彰明较著的闹起来,给人家传了出去,我的脸上有何光彩!想到这里,只得把心上的怒气捺了一捺,叹一口气,瞪了康少己一个白眼,仍旧把戒指交还了他。康少己怀着鬼胎,不敢开口,接过戒指来也不敢再带,勉强站在那里敷衍了一回,便回转身来一溜烟跑了出去。

康中丞本来没有什么大病,不过着了些儿风寒,觉得心上有些饱闷。富贵人家的习气,只要稍稍的觉得有些不快,就要延医服药的闹得一塌糊涂。每每有本来不妨的小病,吃了几贴药吃出病来的。康中丞的生病便也是犯着这个毛病。

当下康中丞见康少己走了出去,自己盘算了一回,正要去叫了五姨太太来和他说话,恰恰的门帘启处,那位五姨太太已经轻移莲步走了进来,宝靥微红,秋波不定,好似受了什么惊吓的一般,走进来就坐在康中丞床上,和康中丞说了几句闲话。

康中丞留心看他的手上,只见那个钻石戒指高高的戴在手上。康中丞看了,心上顿时一块石头落地。暗想果然是我疑心错了,他的戒指明明的在他手上,怎么会到别人手里头去呢?幸而没有吵闹出来,总算我自己有些耐性。想着,心上正是欢喜。忽然心上又想道:天下的事情都是无从逆料的,或者他方才见我要他的戒指来看,心上已经明白,连忙把这个戒指去送还了他,也未可知。一会儿心上又想五姨太太的为人,平日之间狠是稳重,料想不至这般轻贱。一刻儿的工夫,康中丞的一个心,就如井上的辘轳一般,转了无数的念头。

五姨太太在房间里头坐了一回,忽把双眉一皱,对着康中丞说有些肚子痛。康中丞叫他回房歙息。五姨太太便慢慢的走了出去。

停了一回,康少己又走进来,问长问短的十分亲切。康中丞口中不语,却偷眼看他手上,见方才的戒指依旧带在手上,纹风不动。康中丞到了这个时候,方才把满心疑惑都化得干干净净。又仔仔细细的把康少己手上的戒指看了一回,觉得和五姨太太手上的那个直是一个样儿,没有一丝一毫的分别,就是有心制造的,也制造不出来。正是:

珠帘金屋,魂迷韩掾之香;锦帐银床,春满宓妃之枕。

不知后事如何,且听下文交代。





第一百二十七回  锡佳名注释九尾鱼 写牢骚演说烟花史





且说康中丞看了康少己手上的戒指,竟和五姨太太手上的一个样儿,好像是天生一对的样儿,不由的看了又看,心中暗想:“天下竟有这样相像的东西!若不是方才有些涵养,当时没有闹出来;冒冒失失的混闹了一下子,那就懊悔不及了。”

自此以后,康少己见康中丞这般糊里糊涂的,免不得更加大胆起来,渐渐的丑声外播,大家都知道这位康中丞家有些帷薄不修。甚至上海有一班滑头子弟,编出三十首《竹枝词》来,专讲康中丞家里的那些故事。康中丞公馆里那些大大小小的人,也没一个不知道的,只瞒着康中丞一个。甚而至于康中丞的亲戚里头有一班轻薄少年,故意抄着那几十首《竹枝词》给康中丞看。康中丞看了,有些懂得的,有些全然不懂,却糊里糊涂的,不晓得他说的是那一家的事情。还带了回来给家里头的人看,只说这个诗上说的不知是什么人家,怎么好好的人家会弄到这般模样?始终没有知道这三十首《竹枝词》就是说他自己家里头的事情,你道可笑不可笑?

看官且住,在下做书的做到这个地方,又出了一个岔子,用不着列位看官指摘,在下做书的先自己举发出来。

从来天下的人,胳膊折了往袖子里藏,无论什么事儿总要帮着自己亲戚的;就使亲戚家中闹了什么笑话,出了什么乱子,对着外人尚且要千方百计的替他遮盖,怎么康中丞的这些亲戚,不替他遮盖一下也还罢了,倒反有意把康中丞当个顽意儿一般的捉弄起来,好像狠有些幸灾乐窝的意思,这是个什么缘故?难道康中丞的那些亲戚,都是些红毛国里头的野人不成?

原来这个里头却也有个道理。自从康中丞的那位正室夫人回籍以后,康中丞把一切家里头的事情,一切亲戚朋友的应酬,都是交给大姨太太一个人管理。这位大姨太太虽然能干,究竟是个倌人,那里懂得这些事情?那些亲戚家里该应送礼的也不去送,该应遣人问候的也不叫人去。再碰着那些婚丧凶吉的事情该应要内眷出去应酬的,这位大姨太太更加出不得场,缩着个头死也不肯出去。

那班亲戚心上本来已经有些不快活,更兼见康中丞这般糊涂,把好好一个正室夫人搁在家里,连娶媳妇这般喜事都不去接他出来,只凭着那几个姨太太在里头混搅,大家多狠有些不以为然。再是康中丞恃着自家有钱有势,未免有些富贵娇人的样儿,所以那些亲戚一个个都和康中丞不合,竟没有一个肯帮他的人。听见有人在那里骂他,这些亲戚非但不肯和他辩护,碰着一个高兴的时候,还要连自己也凑下去点缀两句。这个里头有这样的几层缘故,所以那些亲戚一个个都不肯帮他。并且有些秘密的话儿,外人不知道的,也是那些亲戚背地里传出来的。你想康中丞家这样的深闺内院,青琐高楼,这些闺房狎昵的事情,外人那里打听得出来?

更兼上海滩上的人都是那些不顾廉耻的滑头少年,听了康中丞家有这样的几个尤物,便大家前前后后的想要转他们的念头;不但是癞蛤蟆想吃天鹅肉,并且还心上存着个人财两得的念头,想着要骗他们的钱。就是这样的一传十、十传百,沸沸扬扬的。就是实在没有这件事情,这班滑头少年也要造些话出来说,竟把康中丞家里的那些宝货,当作个历史里头大有关系的人物一般,今天说的也是这几个人,明天说的也这几个人。说来说去,里头就有轻薄少年把康中丞起了一个绰号,就叫作“九尾龟”。

有人问他这个“九尾龟”是什么意思?他说也没有什么深微奥妙的意思在里头,不过为着这位康中丞家里头有五个姨太太,有两个姑太太,有两个少奶奶,恰恰是九个人,又恰恰的九个人都是这样风流放诞的宝贝,我所以把这位中丞公起个徽号叫做“九尾龟”。你们闭着眼睛想一想,这个情形可像不像?问的人听了他这一番说话,觉得虽然没有什么道理,这个情形恰委实有些相像,便也一笑走开。

从此外面那些和康中丞不对的人,只要提起康中丞来,大家都不说他的名姓,只叫他是“九尾龟”。在下做书的便借着这个“九尾龟”的名目,编成这一部醒世新书。虽然康中丞这个人并不是书中的正脚色,但是在下的这部小说既然名目就叫作“九尾龟”,在下做书的,自然也不得不把这位元绪先生姑且当作全书中间的主人翁,好好的演说一番,总算交代过了书中的一个节目。

看官们若毕竟要问着在下做书的,这部小说里头那一个是书中的主人翁?这却连在下做书的自己也不曾晓得。看官们意中把那位当作主人,在下做书的就把那位算作主人。就是把在下做书的局外人,扭进局内去做一个全书的主人翁,也未尝不可。究竟三千大千世界,谁主谁宾?恒河沙数众生,无人无我。在下做书的随口说出,信手拈来,本来没有存着那个是主、那个是宾的念头。列位看书的酒罢茶余,消遣世虑,也不必存什么那个是主、那个是宾的意见。无非姑妄言之,姑妄听之罢了!

咳!如今世上的事情,为着办事的人胸中存了个宾主的念头,因此坏事的也不知多少!何况在下这样一部汗牛充栋的小说,洒一腔之涕泪,谁是知音?掬满腹之酸辛,畴能遣此!寓言醒世,俳语成文;东方滑稽之谈,南国烟花之史。知我者怜其沦落,或者方诸阮籍之穷途;罪我者咋其疏狂,方且指为灌夫之骂座!文章憎命,时运不济,时逢白眼之人,尽有揶揄之鬼!寄闲情于风月,惆怅扬州;感逝水之华年,凄凉锦瑟。借着那青楼中冶叶狂花的姿态,做一部世界上劝人讽世的清谈。把那些上海滩上以前的四大金刚,以后的十二花神,都一古脑儿收聚拢来,做了这一部小说中间的资料。这也总算是现身说法,皆大欢喜了!

如今闲话休提,把这位康中丞撇到一边去,再提起那位章秋谷来。

只说辛修甫这个时候在后马路开了一家极大的书局,就请章秋谷做个总经理,兼任编辑事务,每一个月送他二百两银子。章秋谷本来原不愿意就的,自己想了一想,一则太夫人还在常熟,陈文仙又在上海,好好的一个人家分作两起,终久不是长局。况且自己又要回去侍奉太夫人,不能长在上海,把陈文仙一个青年少妇丢在外面,未免身心两地,不甚放心。如今就了这个馆地,便可把太夫人接到上海来住,免得两边来来往往的,十分不便。更兼这个书局又是辛修甫一个人独股开的,秋谷也想要和他整顿一番,自己也好借着这件事儿多看些书,长些学问,便慨然应了。

辛修甫十分欢喜。

秋谷到书局里去料理了几天,先把事情理出一个眉目来,聘请了几个编辑新书和小说的人。又请了几个翻译,译那些东西书籍。把书局里头几个朋友的执事,都分派得清清楚楚:管批发的管批发,管机器的管机器,管出入的管出入。秋谷倒忙了好几天,便和修甫说了,要回常熟去接家眷出来。修甫自然赞成。

章秋谷回到常熟和太夫人说了,太夫人听了自然十分欢喜。依着太夫人的意思,要过了年再搬。禁不得秋谷在旁撺掇,只说书局事多,不能回家过年,一个人在上海又不放心。太夫人听了这几句话说得不差,便也依他。忙忙碌碌的差不多料理了半个月,方才到了上海。在新马路眉寿里看了一处三楼三底的洋房,甚是宽敞,大家欢欢喜喜的过了几时。

秋谷心上想着一个陈文仙住在外面,好像个外室一般,终久不妥当,只得和几个亲戚密密的商议了好几天,定了主意,趁着太夫人喜欢的时候,几个亲戚婉婉转转的把这件事儿和太夫人讲了一遍。太夫人听了,果然心中大怒,便叫人到书局里去立刻把秋谷叫了回来,当着亲戚的面前,便叫秋谷跪下。几个亲戚连忙相劝。

劝了一回,太夫人怒气稍稍平些,叫秋谷立起来,对着秋谷说道:“你是我的儿子,你的事情为什么要瞒着我,不叫我知道?你难道是当我已经死了的么?若是到了那个时候,我真个闭上眼睛,自然不来管你的事!如今我还有一口气在,你就瞒着我在外面这般混闹,你究竟是个什么意思?”秋谷听了,低着头不敢开口。太夫人又道:“就是一件极平常的小事,也该应和我讲一声儿,何况这样的事情。天下那有纳妾好瞒着父母的道理?你就是做了皇帝,家庭里头也要由我做主!难道你比皇帝还大些不成?”

秋谷听了委实无言可答,只得跪下又叩了一个头,起来站在一旁,口中说道:“这件事情,都是儿子的不是。儿子情愿领母亲的责罚。”几个亲戚见了,又着实在旁相劝。

太夫人心上虽然不快,看着秋谷叩头认罪,又满口自家认错,心上早已有些回转;又被几个亲戚你言我语的劝了一番,便对着秋谷道:“如今看众位亲戚面上,况且生米已成熟饭,只好由你去闹到那里算到那里的了。但是好好的一家人家,断没有妻妾分居的道理,只好把你那位姨奶奶接到这里来一同居住。只不知道堂子出身的人,安本分不安本分?”秋谷道:“这个母亲只顾放心。这个人的性情十分温厚。就是住在一起的儿,他也和儿子说过几次,情愿守着规矩住在一起。母亲不信,只等他来了再看就是了。”太夫人听了,不觉开颜一笑道:“人还没有来,你就这样拚命的帮他。将来你那位老婆,不知你还要把他怎么样呢!”秋谷见太夫人笑了,也陪着笑道:“这也不至于的。”正是:

小星三五,银河昨夜之波;孔雀东南,中妇前宵之泪。

不知以后如何,且看下回交代。





第一百二十八回 换桃符阳春回大地 喧爆竹风雪度残年





却说章秋谷想着陈文仙住在外面终不是个久计,便请了几个亲戚宛宛转转的和太夫人讲了一番;又大家都劝了太夫人一阵。太夫人起先虽然有些动气,后来见秋谷自己口口声声的认罪,又被几个亲戚劝了一番,便也回嗔作喜,叫秋谷拣个日子,把陈文仙搬了进来一同居住。

到了那一天,陈文仙明妆靓服的过来,恭恭敬敬的先拜见了太夫人。太夫人把他搀了起来,仔仔细细的从头到脚看了一遍。只见他蛾眉挹翠,檀口含朱,眼媚春波,腰欺弱柳。更兼丰容婀娜,态度端庄,既没有一些儿风流放诞的样儿,又没有一些儿儇薄轻佻的气派,那里像什么堂子里头出身的倌人,看上去竟是一个大家闺秀。太夫人看了十分欢喜,心上暗想:“这个人倒不像是个倌人出身,将来一定不至于闹什么笑话的。”便也和颜悦色的抚慰了文仙几句。文仙拜见了秋谷的那位正室夫人,也规规矩矩的,甚是小心。

秋谷的那位夫人起先听了这个消息,心上自然十分不快。只说这个陈文仙既然是个妓女,不知怎样飞扬跋扈的一个人。如今见了陈文仙这样的循规蹈矩,没有一些儿撒娇恃宠的样儿,倒觉得出于意外,便也欢欢喜喜,好好的相待。陈文仙究竟是个倌人出身,骗人的工夫狠好,用出浑身手段来巴结太夫人和少夫人,不上半个月,就把这两位骗得二十四分的欢喜。秋谷见了,自然也十分快活。

不知不觉的早到了十二月二十八的那一天,腊鼓迎年,屠苏献岁,万家爆竹,大地回春。秋谷在家里头没有什么事,便和太夫人讲些外面的事情,说些街巷的笑话。有时候带着一妻一妾,同着太夫人抢状元筹、掷升官图;掷得不耐烦,便四个人打一局麻雀,和哄得太夫人甚是高兴。

过了两天,早又是除夕了。秋谷想着梁绿珠同陆丽娟那里有些帐没有开发,这两天和哄着太夫人顽,连大门都没有出,把这件事情竟不知忘到那里去了,直到这个时候才忽然想起来,便和太夫人说了一声,要出去还些帐目。太夫人道:“你无非是要出去还嫖帐就是了。把有限的几个钱这般用法,将来用完了,我看你怎么样!”

秋谷听了呆了一呆,答应不出,恐怕太夫人生气,站在那里不敢就走。偷眼去看看太夫人脸上的神色,却还是一脸的笑容,‘心上方才放心。便慢慢的退了出来,赶到楼下自己书房里头,开了铁箱,带了一卷钞票,一溜烟直到久安里来。

看官,你道太夫人既然知道他是出去还嫖帐,怎么并不生气,许他出去?原来太夫人自从到了上海以后,也微微也有些知道秋谷在嫖场里面狠有些儿声名;又向来知道秋谷的脾气风流自喜,倜傥非常,更兼住在上海滩上,这样花天酒地的地方,自然的就有选舞征歌的兴会。从来说知子莫若母,明知道就是管也管他不住的。平日之间常常听得秋谷讲的那些堂子里头的情形,那些倌人骗人的圈套,讲得个穷形尽相,色舞眉飞,知道他是嫖界里头的惯家,不至于再会上什么倌人的圈套,便也随随便便的,不十分去拘管他。只对他说:“你们在面子上的人,逢场作戏自然是免不来的。但是你究竟年纪还轻,恐怕一个不留神,上了倌人们的当,到了那个时候,就想懊悔都来不及了。我虽然不来管你,你也要诸事留神些儿。”又叫秋谷把陆丽娟和梁绿珠两个人叫到大菜馆来,太夫人细细的打量了他们一番,又和他们问答了一阵,便对秋谷道:“这两个人里头,还是陆丽娟天真烂漫,我看起来比梁绿珠好些。梁绿珠脸上虽然没有什么,我看他心计深得狠,说的话儿亦狠有斟酌,你以后不要做他,就做陆丽娟一个人罢。”

秋谷听了,口中自然答应,心中却有些不相信的意思。自己心中暗想:“凭你梁绿珠再要狡猾些儿,凭着我章秋谷的一身本领,料想也还对得过他。”想着太夫人的话儿也不过是揣度之词罢了。

如今闲话休提。只说章秋谷径到久安里陆丽娟院中,大踏步走进房间,见丽娟一个人坐在房里,静悄悄的不见别人。丽娟把一只纤手托着香腮,坐在那里好像想什么心事的样儿。见了秋谷进去,立起身来懒洋洋的打了一个呵欠,笑道:“耐好几日勿来哉啘,勒浪屋里向陪仔姨太太,两家头窝心得来,连大门才勿想出格哉!

今朝倒那哼想着仔到倪间搭小地方来走走?“秋谷听了笑道:”你这个人,真是浸在醋缸里过日子的,一开口就有些酸溜溜的味儿。“陆丽娟不等秋谷说完,把身子一扭道:”耐格闲话倒来得诧异笃啘!啥人勒浪搭耐吃醋呀?“说着不觉蛾眉微竖,俊眼流波,狠狠的瞪了秋谷一个白眼。秋谷便笑道:”你不要发急,我不过和你说句笑话,你就急到这般田地。“

说着便走过去搀着陆丽娟的手问道:“怎么这里只剩了你一个人,他们都到那里去了?”丽娟道:“俚笃才勒浪外势收帐,一塌刮仔才出去哉。”秋谷道:“你今年的帐怎么样,收得齐收不齐?”丽娟蹙着眉头道:“有几户老客人,才到仔别场化去哉。倪间搭几格户头,才是看得见格。有格排滑头客人,倪也勿去做俚!故歇倪帐浪一塌刮仔算起来,差勿多二千多点。除脱仔两格勿勒浪上海格客人,倒去脱仔四百多。再有一千六百洋钿,收着仔一格八折帐就算好哉!”秋谷听了,便又问道:“你今年年底的开销怎么样?”丽娟道:“倪搭格开销,是耐晓得格,一节不过一千洋钿。帐浪收落来,刚刚正好。”

秋谷听了,故意和他说道:“我要和你商议一件事情,不知你答应不答应?”

丽娟听了倒呆了一呆,看着秋谷的脸道:“啥格事体,耐要搭倪商量?”秋谷低低的和他说道:“我今年的酒局帐,差不多也有三百块钱,虽然数目不多,我今年亏空做得大了,一时周转不来。我想和你商量,把你这里的钱暂时耽搁一下,等明年正月里头再想法子给你。只要过了一个年,就不怕没有法儿,不知你心上怎么样?”

丽娟听了,似信不信的看着秋谷道:“阿是真格呀?耐格闲话一径来浪瞎三话四,有点靠勿住。”秋谷正色道:“别的事儿说说笑话罢了,这个事情是于我面子上大有关系的,我怎么肯说谎骗你?难道我无缘无故的平空倒掉自己的牌子么?”

陆丽娟听了,心上已经有几分相信的意思,却究竟还有些儿疑惑,停了一回,方才说道:“倪间搭格二三百洋钿倒呒啥希奇,耐也勿要放勒心浪。倒是梁绿珠格搭格帐,耐去还拨仔俚,勿要搭俚杂格乱拌。阿晓得?”秋谷道:“这个自然。就是你这里,也为我们两个人平日之间总算是彼此要好的,我才来和你商量。要是换了第二个人,我无论怎样也要想个法子还他,免得折了自家的志气,去和他商议。”

说着,又对丽娟道:“但是你这里也要开销别人的,平空的少掉了一笔钱,你又怎么样呢?”丽娟道:“倪搭倒呒啥要紧。倒是耐自家格开销那哼?”秋谷道:“那些戏园、菜馆、马车行、绸缎店的帐,一古脑儿也不过三百块钱,这一点儿不算什么。”丽娟道:“倪一径搭耐说,勿要实梗瞎用。故歇格世界,铜钿银子顶要紧。

耐总归勿肯听倪格闲话。到仔故歇辰光,耐阿是也来勿及哉!下转勿要实梗,阿晓得?“

秋谷听了,点一点头,却故意对他笑道:“像我这样的蹩脚客人,还要在你院中走动,给他们一班娘姨、大姐看了,也觉得不好看。”陆丽娟瞟了秋谷一眼道:“啥格蹩脚勿蹩脚,只要倪搭耐两家头──”丽娟说到这里地方觉得接不下去,便顿了一顿,看着章秋谷一笑。章秋谷也看着陆丽娟一笑。丽娟把头一低。秋谷又道:“万一有人说你做我的恩客,你又怎么样呢?”丽娟笑道:“随俚笃去说末哉。

说算倪做仔耐格恩客末,也勿关别人啥事。“秋谷听了,不觉哈哈一笑。丽娟倒呆了一呆道:”耐笑啥呀?“正是:

残年风雪,万家爆竹之声;萧鼓春城,大地河山之影。

第九集书中,还有张园赛会,江北水灾,章秋谷初到天津,方小松重来上海,这些说话都在下集书中。在下做书的做到这个地方,却要暂时搁笔,休息几天的了。





第一百二十九回 假漂帐嫖客行权 真索债倌人受骗





上集书中,正说到章秋谷把家眷接到上海,就在上海过年。到了除夕的那一天下午,章秋谷忽然想起有几处局帐还没有开发,便先到陆丽娟院中,故意要试试陆丽娟和自己的交情究竟怎么样。假意只说今年的局帐来不及,要等到明年再付,要看陆丽娟听着这个话儿怎生回答。不想丽娟听了没有一些儿勉强,竟自一口答应。

秋谷心上自是十分欢喜,当下对着陆丽娟哈哈一笑。丽娟摸不着头脑,不懂他是什么意思,呆呆的对着秋谷道:“啥格事体,耐实梗好笑呀?”

秋谷也不开口,在衣袋里头取出一卷钞票放在桌上,对着丽娟笑道:“今天还好,居然竟没有坍台,总算我们两个人的交情不错。”陆丽娟听了,起先还不知是什么缘故。想了一想方才恍然大悟,口中说道:“怪勿得倪原说耐格位二少爷,勿糙至于实梗样式啘?倪晓得耐格闲话靠勿住,故歇到底那哼?”秋谷一面笑着,一面在那一卷钞票里头拣出六张五十块一张的递在陆丽娟手内道:“手巾和送礼的钱前几天已经开销的了。我的酒帐,局帐,通共二百七十几块钱,多的二十几块钱,就给了你房间里头的人罢。”陆丽娟把钞票接在手内,看也不看便放在桌子上,口中说道:“耐格帐一塌刮仔二百七十几块洋钿,付仔二百八十洋钿好哉。房间里人末,有下脚拆格啘,拨俚笃做啥?耐就是拨仔俚笃,俚笃也勿见得见耐格情啘!”

秋谷道:“这班人都是小人,格外赏他们几个钱,也好叫他高兴一点。”陆丽娟不肯道:“耐末总是实梗。格号铜钿出俚做啥?真正到仔要用格辰光,阿怕倪勿晓得?

故歇耐总归是实梗马马虎虎。俚笃拿仔耐格洋钿,再要当耐瘟生,啥犯着呀!“

秋谷听了,觉得这几句说话委实不差,便对丽娟道:“你的说话自然不差。但是我在你面上用几个钱,就是多花了些,我也没有什么不愿。你怕他们拿了我的钱还要当我瘟生。不是我在你面前说句大话,我章秋谷在嫖界里头阅历了五六年,别的不敢说,只这‘瘟生’的两个字儿大约自问还可以免得。料想你们堂子里头的人也没有人把我当作瘟生的。在我的意思想起来,我们两个人总算是狠要好的,房间里头的人也没有一个不知道。如今我多出几个钱,总算是给他们的赏钱,在你面上也觉得好看些儿。况且我虽然不是个有钱的人,这几个钱也还不算什么,又何必一定要省这几十块钱呢!”陆丽娟听了,想了一想方才点一点头。又问着秋谷道:“耐今朝到倪搭来吃年夜饭,阿好?”秋谷随口答应。

坐了一回,正起身要走,陆丽娟忽然说道:“耐格个人倒来得挖掐笃啘!”秋谷笑道:“怎么你想了半天,没头没脑的说出这样的一句话来。”陆丽娟听了自己也觉得好笑起来,一面笑着一面又道:“倪故歇想起来,耐来浪对仔倪瞎说一泡,啥格呒拨洋钿,咦是啥格今年来勿及。区得倪勿是格号只认得铜钿,勿认得人格人,答应仔耐呒啥闲话说,勿然是,耐故歇搭倪跳得来好白相煞哉!倪倒今朝问问耐:倪勒浪耐面浪,阿曾有啥推扳?耐要搭倪实梗样式?耐倒自家想想看,阿有格号道理?”秋谷见丽娟星眸敛恨,宝靥微红,觉得另有一种丰韵,便连忙笑道:“你不要生气,你要晓得不是我这样一来,那里试得出你的心迹?你不谢我,也还罢了,倒反要怪我起来。”丽娟“嗤”的一笑道:“索性越说越好听哉!啥人来听耐呀。”

口中虽然这般说法,心上却甚是喜欢,拉着秋谷在炕床上并肩坐下,又密密切切的讲了一回,叮嘱他晚上早来。

秋谷便出了久安里,从大新街直穿过迎春坊,来到了梁绿珠院中。走上楼去,梁绿珠正和一个小大姐拿着一付骨牌在那里打天九顽,见了章秋谷,满脸上堆下笑来,喜孜孜的叫了一声“二少”。连忙和秋谷宽了马褂,推着秋谷坐下,那相待的样儿甚是亲热。秋谷趁势说道:“像我这样的漂帐客人,你何必这般客气?”梁绿珠听了,不懂秋谷的意思,便道:“勿要来浪瞎三话四,啥人是漂帐客人呀?漂啥人格帐呀?”秋谷不慌不忙,把一个大拇指在自己鼻子上一指道:“漂帐客人就是我。漂的就是你这里的帐。”绿珠听了,越发不知道说的是那一路的话儿,只呆呆的看着秋谷的脸。秋谷笑道:“你不要在这里装糊涂,我要漂你的局帐,你答应不答应?”梁绿珠那里肯信,口中说道:“阿是耐要漂倪格帐,说得阿要像点。像耐二少爷实梗格客人要漂倪格帐末,上海滩浪一塌刮仔才变仔漂帐客人哉!”

秋谷听了梁绿珠的口气又是一种,和陆丽娟不同,便也不去和他多话,只微微一笑,立起身来做个要走的样子。梁绿珠连忙拉住问道:“啥实梗要紧去介,晏歇点阿来?”秋谷故意摇一摇头道:“今天除夕,我家里头还有事情,等会儿未见得有工夫再来。我们明年再见罢。”说着往外要走。梁绿珠连忙紧紧的拉住了秋谷的衣服,不肯放手,口中只说:“耐慢慢交去,倪有闲话搭耐说。”秋谷听了,便回身坐下,对着梁绿珠道:“你有什么话,只顾讲就是了。”梁绿珠支支吾吾的,又一时说不出来,只说道:“耐啥格事体实梗要紧?倪搭呒拨啥格老虎勒浪,勿见得吃脱仔耐格,耐放心末哉。”秋谷笑道:“我要走,你又不叫我走,说有话说;如今我问你什么话儿,你又不说。这是个什么缘故呢?”梁绿珠没有话说,只得把金莲在地下一顿道:“倪勿要!耐搭倪坐来浪!”

秋谷忽然大笑道:“我明白了,我明白了。”梁绿珠也笑道:“啥格明白不明白,啥人搭耐唱‘三娘教子’呀。耐明白啥物事?倒说拨倪听听看。”秋谷笑着说道:“实不相瞒,今天我原是出来还帐的,不料到了你这里坐了一回,把还帐的这件事儿忘了。怪不得我要走,你不叫我走,说有什么话和我说,一定就是这件事情了。你何不早些和我讲个明白,却这样吞吞吐吐的不说出来,难道还怕不好意思不成?”说着便取出三张五十块钱的钞票,递给梁绿珠。

梁绿珠被章秋谷一席话儿说中了他的心病,未免有些不好意思,一时间颊泛桃红,脸生春色。见秋谷手内拿着几张钞票要递给他,便缩着手不肯接,口中说道:“慢慢交,耐放勒浪仔看。啥格倪要搭耐说句闲话,耐倒说,倪问耐讨帐,勿肯放耐,格两声闲话,倒要搭耐弄弄明白笃!”秋谷含笑道:“你先收了钱,再说话也还不迟。”梁绿珠填道:“倪勿要。”秋谷道:“依着你的意思,要怎么样呢?”

梁绿珠道:“倪也呒啥别样,只要叫声耐,倪好好里叫耐坐歇再去,耐倒说要问耐讨帐,耐勒浪倪搭做仔一年多点哉,几时间耐讨过歇啥格帐?耐倒搭倪说说看!”

秋谷道:“既然如此,你有什么话要和我说,又为什么支支吾吾的讲不出来呢?”

梁绿珠被秋谷逼住了,一时造不出什么话,只得随口说道:“倪要问问耐,格两日阿是一径勒浪陆丽娟搭,啥洛倪搭一径勿来?啥格讨帐勿讨帐介!”

秋谷听了,知道他有心掩饰。待要再驳他几句,却看着他的样儿已经面红头胀的,狠有些儿发急;恐怕他理屈词穷,老羞成怒,那时倒觉得没有味儿,便也微微一笑,不去驳他,只对他说道:“既是你这般说法,就算我讲错了何如?但是这个局帐是我本来要付的,不过我一时忘了,所以迟了几天,同这件事情毫不相干的,为什么你又不肯收呢?”说着便又把方才的三张钞票递过去,放在绿珠手内。绿珠口中还说:“放勒浪末哉,用勿着实梗要紧啘!”口内这般说着,却不知不觉的已经伸手过去接了过来。秋谷笑道:“今天已经十二月三十,你还说用不着这般要紧,那就真要漂帐过年的了。”梁绿珠也不觉一笑。秋谷又略略坐了一回。临走的时候,梁绿珠要留他吃年夜饭,秋谷摇摇头道:“年夜饭是没有工夫来吃的了,明年来吃开台酒罢。”说着,便走下楼梯。

刚刚走出大门,忽然一个人劈面走来,一把拉着秋谷道:“我找了你半天,居然给我找着了!”秋谷抬头看时,原来是自己的一个远房表叔,姓马,号山甫,家里头狠有几个钱,捐了一个户部郎中。如今丁了外艰,便在上海合了几个人,在新闸地方开了机器公司。这个马山甫还有一位老太太,也是住在常熟的。平常的时候,都是在上海、常熟两处来来往往,差不一年里头也有半年住在上海。这个时候,刚刚马山甫的老太太打发马山甫到上海来结算公司里头的帐目。

马山甫来的时候,原打算赶回去过年的。不料到了上海,做了一个倌人,叫做陆韵仙,住在清和坊一弄。这位马山甫本来是个嫖客里头的瘟生,陆韵仙又是个烟花队中的老将,两个人自从有了相好之后,如鱼得水,如漆投胶,一刻也离不开来。

马山甫虽然家里头狠有几个钱,却生得性情啬刻,那怕用一个大钱,也要心里掂一掂轻重方才肯拿出去。陆韵仙放了他几回差,马山甫都含含糊糊的不肯答应。陆韵仙只认他还没有死心塌地,所以不肯花钱,要想个笼络他的法儿,便索性劝马山甫把行李搬到他院中去住。

马山甫也不想一想该应怎么的一个价值,还只说陆韵仙和自己要好,方才要他搬去,心上二十四分的欢喜,冒冒失失的带着一个家人竟搬到清和坊来。陆韵仙的房间本来狠多,便腾出一间房间来给他住了,应酬得十分周到,供给又甚是丰盈。

连马山甫的零用,都是陆韵仙代出,不要他花一个着钱,预备着到了年底的时候好大大的敲他一下竹杠,料想他一定不好意思推却。马山甫那里知道。正是:

银环金枕,丁娘十索之歌;雨散云飞,宋玉三年之恨。

不知后事如何,且听下回分解。





第一百三十回 享温柔误人销金窟 敲竹杠偏遇守财奴





且说陆韵仙把马山甫留到自己院中来住,韵仙自己提着全付精神的来应酬他,连他的零用都和他代付,不叫他出一个钱,照应得十分周到。原想等到年终,要问马山甫借几百块钱,敲他一下竹杠,料想马山甫一定不好意思不答应的。这个过年的盘缠,就要想出在马山甫身上。

可怜马山甫那里知道,好像在那里做梦的一般。心上还只在那里算计:住了陆韵仙的房子,又享受了他的供给,这里头倒好着实省几个钱。又怕陆韵仙要和他纠缠,便不等年底,预先早早的叫陆韵仙抄出帐来,和他算得清清楚楚。自己想着,这件事情做得十分干净。

不料陆韵仙到了十二月二十七那一天的晚上,一个人悄悄对他说,要问他借五百块洋钱。马山甫听了,吃了一惊,一时间回答不出,只得含含糊糊的答应一声道:“几百块钱的事情,也是小事。你不要性急,明天再说就是了。”

陆韵仙的心上,以为自己特地空了一个房间给他住着,别的客人都不放进来,更兼供给他主仆两个人的伙食,马山甫又是个公子哥儿的脾气,在他一个人身上琐琐屑屑的今天要这样,明天要那样,不肯将就些儿,这半个月之内,用在马山甫身上的钱,已经差不多有一百块钱。再加上过年的费用,新年的开销,合算起来,也要二百块钱的光景。如今问他借五百块钱,拿定他一口应承,断没有不答应的道理。

那里晓得马山甫听了他的说话,脸上就是一呆。回答他的话儿,又觉得狠有些儿勉强,支支吾吾的露出些不愿意的样儿。

堂子里头的人,何等狡猾。陆韵仙看了马山甫的样儿,心上已经有些明白;看着他那种半吞半吐的神情,却又不明白马山甫的意思。只认着马山甫见他一开口就要借五百块钱,嫌他狮子大开口,要得太多,所以这样的踌躇不决。或者想要打他一个折扣,多则四百块钱,少则三百块钱,这件事情也就过去了。陆韵仙一面想着,一面在肚子里头暗笑。

只见马山甫沉吟了一回,开口问道:“你要借五百块钱有什么用处?难道像你这般生意,年底的开销还不够么?”陆韵仙听了满肚子的不愿意,却又不好发作出来,只得冷冷的答道:“故歇堂子里向格生意,格末叫难做。看看面子浪生意蛮好,像煞呒啥;到仔节浪向搭仔年底下,划算起来总归是格勿灵。耐放心末哉,倪总勿见得来敲耐啥格竹杠,耐勿要勒浪发极。轧实搭耐说仔罢,今年倪搭开销,刚刚再少一千洋钿。耐搭借仔五百,再有五百勿着杠,倪也只好到仔归格辰光再讲格哉!”

马山甫听了,心上有些半信半疑的。停了一回方才说道:“你要借钱,你又不早和我说,前几天我把这里的几千银子一古脑儿都汇到常熟去了。留在我自己身边的,不过几百块钱,还要预备过年的零用。如今你要问我借钱,只好等我明天出去到朋友那里去,托他们和我转借就是了。”陆韵仙听了心上自然狠有些不像意,微微的笑了一笑,口中说道:“实梗说起来,倒费仔耐格心,谢谢耐,对勿住。”马山甫也不知道这几句话儿是陆韵仙有意反激他的,一些儿也不觉得,欢欢喜喜的过了一夜。

到了二十八早上,马山甫故意出去打了一个转身。回到陆韵仙院中,假意蹙着眉头,对陆韵仙道:“事情不成功,这便怎么样呢?”陆韵仙听他竟自爽爽快快的回报出来,觉得甚是诧异,便说道:“阿唷!耐勿要来骗倪!像耐实梗一个蛮阔蛮大格马大少,要借五百洋钿才呒借处,耐勒浪骗啥人介?”马山甫连忙说道:“并不是我骗你,实在这个时候已经年底,大家都不肯通融。我虽然有几处来往的钱庄,到了这个时候,他们只有归帐,那里还肯放出?若在平日之间,不要说五百,就是五千,我姓马的也还拖欠得动。如今刚刚碰着年底,实在想不出什么法儿。这件事情却要怪你自家不好,为什么一向不肯开口,直到这个时候方才讲出来,这是个什么缘故?”

??韵仙听了一言不发。停了一回,方才冷笑道:“耐勿要勒浪搭倪瞎三话四。

耐肯借末借仔,勿肯借末也呒啥希奇,老老实实搭倪说末哉;啥格实梗阴阳怪气,假痴假呆,阿要气数!“马山甫到了这个时候,还没有听出陆韵仙的意思来,连忙分辩道:”你不要动气,我实在是没有法儿。若是有了法儿,不肯借给你,凭你怎么样罚我就是了。“陆韵仙道:”倪是呒啥那哼,只要耐自家心浪去想想好哉!“

马山甫听了,糊糊涂涂的想不出什么来,只说道:“我想不出什么,你叫我想什么呢?”

陆韵仙见马山甫糊涂到这般田地,又不好明说出来,心上又好气,又好笑。只得走过去坐在马山甫身旁,伸出纤手来,紧紧的拉住了马山甫的手,大声说道:“倪勿要!耐勿要勒浪假痴假呆!搭倪去借得来!别人家倌人才有仔相好,送格一千搭仔八百洋钿拨倌人过年,也勿算啥希奇。只有耐格个人末,真真苏州人攀谈,拔出仔──”

陆韵仙说到这里说不下去,面上一红,不觉看着马山甫一笑。停了一停,陆韵仙又道:“别人家倌人敲客人竹杠格,蛮多来浪。耐倒自家想想,天理良心,倪阿曾敲过歇耐啥竹杠?听见耐到仔上海,常恐耐住来浪公司里向勿舒齐,赶紧叫耐到自家屋里向来住。一塌刮仔,才是倪一干仔搭耐开销,勿要耐出一个铜钿。耐想想别人家格倌人,阿有实梗样式?故歇倪一塌刮仔不过问耐借得五百洋钿,耐就是实梗格瞎二话四,假痴假呆。耐去问问看,勿要说上海滩浪,世界路浪阿有格号道理!”

马山甫听了,虽然觉得陆韵仙的意思狠有些儿不高兴,但是这一点儿后天长出的情苗,那里抵得过先天带来的贪念?想了一想便立起身来,朝着陆韵仙深深的打一个拱,口中说道:“承你的情,留我住在这里,一切都费你的心,我心上感激得狠!”

看官且住!这个打拱作揖,虽然是个男子在女人面上陪小心、拉交情的一件利器,但是只可以用在大家口角争论的时候,借着他作个和事老人;或者用在彼此有些情愫的当儿,借着他作个天然媒妁。若要把他当实实在在的一件东西,和那世界上天字第一号宝贵的金钱比较起来,不要说是打拱作揖,就是跪在地下磕破了头皮,也是不中用的。你们诸位看官要是不相信在下的说话,只消请你们诸位大家回去,把自己的夫人试验一下子,问他还是愿意天天给他几个钱,还是愿意天天向他打几个拱、叩几个头,就晓得在下做书的一番说话不是无稽之谈了。

闲话休提。只说陆韵仙见马山甫虽然对他打拱作揖的十分客气,却依然不提借钱的事情,不由得心上更加不快。若在乎日之间,陆韵仙见客人对着他这样小心,这般恭顺,自然心上喜欢。恰恰的这个时候是为着银钱上的事儿,非同小可。看了马山甫朝他打拱,非但没有一些儿喜欢的意思,心上倒反觉得厌恶非常,连眼睛都望着别处,不去看他。冷冷淡淡的说道:“勿要实梗嗫,拨俚笃进来看见仔,算啥格样式呀!故歇用勿着啥个打拱作揖,只要耐爽爽快快搭倪说一声到底那哼。有未有,呒拨末也呒啥希奇。”马山甫朗然说道:“我已经和你说过的了,如今年底的时候实在没有法儿。难道我们两个人这样交情,这点儿事情我都不肯出力不成?我看还是这样罢,你不论什么地方去通融几百块钱,只要过了年底,就有法想。明年正月里头我来还他就是了。”陆韵仙冷笑道:“谢谢耐说得实梗好听!倪要紧要借洋钿,一塌刮仔才是年底格开销,洛里等得到开年?等到仔开年是,倪也勿要借啥格洋钿哉!像耐实梗格大少爷,要借几百洋钿才呒借处,叫倪再到洛里搭去借?加二勿灵哉啘!”

马山甫听了陆韵仙这番说话,知道陆韵仙心上着实不快,假意说道:“虽然如此,但是你年底的开销又怎么样呢?我们两个人这样的交情──”马山甫刚刚说到这里,早被陆韵仙接过去说道:“好哉,好哉,勿要说哉!耐勿要提起倪两家头格交情倒也罢哉,说起交情勿交情格句闲话,真正叫枉空嗫!倪实梗格人末,阿好搭耐格马大少爷攀啥格交情?本底子也勿配啘!”

马山甫被陆韵仙说了这番话儿,心上也觉得有些鹘突起来,暗想:“韵仙的待我总算不差,如今年底的时候要问我借几百块钱,也不好算什么敲竹杠。”想到这里,心上便有了几分活动,想给他三百块钱。忽然心上又转一个念头道:“三百块钱的事情不是顽的,只要我把脸皮老一老,挨他几句说话便过去了,虽然受些冷淡,却究竟省了几百块钱。”想着,便坐在那里,也不开口。

陆韵仙见这样的激他,他还是一个老不开口,只得又道:“耐勿要当仔倪问耐借仔洋钿,呒拨还耐。耐借仔五百洋钿拨倪,来浪倪开年格帐浪扣末哉。”马山甫听了,心中暗隐:“这句话儿不过是随口骗骗人罢了,那有堂子里头的倌人问客人借了钱,肯在帐上扣算的道理?”想着便老着脸道:“你不要见怪,我并不是不肯和你出力,实在是力不从心。我向来不说谎话的,这件事儿委实的办不到。”陆韵仙听了,娇嗔满面的说道:“阿是真格呒借处?”马山甫道:“自然是真的,我为什么要骗你,难道在我脸上有什么光彩么?”陆韵仙听了把身躯一扭,霍的立起身来。正是:

春风榆荚,还飞买笑之钱;十斛珍珠,不作缠头之锦。

未知陆韵仙说些什么,且待下回分解。





第一百三十一回 聚家庭天伦全乐事  度残年骨肉庆团圆





却说陆韵仙听了马山甫回得这样斩钢截铁,料想是不肯借的了,一时间由不得心中大怒,蛾眉倒竖,俊眼横睃,把身躯一扭,忽然立起身来,一言不发往外便走。

马山甫见了陆韵仙这般模样,知道他心上在那里生气,自己心中暗想:“亏得我做事老到,老一老脸皮,省掉了三百块钱。像这样的钱,就是双手捧着送给他,他也不见得见我的情。只怕拿了我的钱还要说我是个瘟生,也是保不定的。”

正想着,只见门帘一起,陆韵仙慢慢的走了进来,手中拿着一篇红纸帐单,递在马山甫手内,口中说道:“马大少,请耐看看,勿得知俚笃阿曾开错?”马山甫见了,心上甚是疑惑,只说:“我的局帐已经算清的了,这又是什么东西?”说着接了过来,举目看时,只见那篇帐单上,第一行就开的马大少房租洋八十元。马山甫见了吃了一惊,连忙问道:“什么房租,难道我住在你们这里──”马山甫说到这里地方,觉得这句话儿有些碍口,便不由顿了一顿。陆韵仙早含笑道:“倪格房间四十洋钱一月,耐住来浪倪搭,住到开年过仔正月半动身,刚刚两个月租钿。”

马山甫听了,说不出什么别的话儿,只口中咕哝一句道:“怎么这里的房租贵到这般田地?”陆韵仙笑道:“马大少,耐放心末哉。耐真格勿放心末,只顾到经租帐房里向去问声看,倪阿曾赚耐格铜钿。”马山甫听了,没奈何只得再看下去,只见开得乱七八糟的,又是什么伙食,又是什么零用赏钱,一篇帐上合起来,差不多要三百块钱。

马山甫看了目瞪口呆,半晌说不出话来。陆韵仙笑迷迷的,对着马山甫道:“马大少,耐勿要动气,倪老实搭耐说仔罢。上海滩浪格事体,洛里一样勿是铜钿?

耐带仔个二爷,两家头住仔一间房间,耐自家算算,房钱、伙食、零用,一塌刮仔算起来,要几化开销?叫倪洛里调头得转?依仔倪格心浪,问耐借仔五百洋钿开销脱仔,到仔开年再说。格篇细帐放来浪倪搭,勿拨耐看,省得耐看仔心浪勿舒齐。

勿壳张耐格位大少爷洋钱末勿借,一根毛才勿肯拔,难末倪僵哉啘!再加仔格个断命本家,总说耐一干仔占仔一间房间,别格客人勿好进来,心浪一径来浪勿舒齐,加二逼得起劲点。马大少,耐想想看,叫倪阿有啥法子?“说罢故意叹了一口气,别转头去口中自言自语的说道:”格几个铜钿,豪燥点拨仔俚笃,省得俚笃一径来浪板面孔。“

马山甫听了陆韵仙的这番说话,觉是甚是有理;要找句话儿去驳他,一时那里找得出来。自己心中暗想:“这件事情,毕竟是我自家不好,住在这里,要想占他们的便宜。要想他们的钱是从那里来的?只有算进没有算出,那里占得着他们的便宜!如今便宜没有占着,倒反吃了一个大亏,平空的要拿出二百几十块钱去。”心上自然十分舍不得,却又没有法儿。想来想去,料想这一笔钱是一定要给他的了。

正要开口,忽然心上又转一个念头道:他这个帐上算我两个月的房租,我乐得住到明年再说。想着,便赌气在身上掏出几张钞票,凑满三百块钱,递在陆韵仙手内。

陆韵仙竟不客气,老老实实的接了过来,随手交给娘姨阿五,叫他送到楼下帐房里去。却对着马山甫说道:“刚刚今年生意勿好,掐掐做格开销,勿然是就算仔倪格也呒啥希奇。晏歇点拨别人家说起来,再要说倪敲仔耐格竹杠。”马山甫听了陆韵仙这两句话儿,那里知道陆韵仙是有心轻薄他。只说陆韵仙待他究竟不差,总算有些良心。虽然花掉了三百块钱心上有些心痛,究竟马山甫家里有钱,几百块钱的事情不算什么。便依然还是高高兴兴的,不把这件事儿放在心上。

陆韵仙自从砍了这下斧头之后,摸着了马山甫的脾气,平常时候是不肯拿出钱来的,一定要硬逼着他方才肯拿出钱来;便换了一付样儿看待他,绝不像那以前旖旎温和模样。马山甫一些儿也不知道,还在那里打算:到了明年,要想娶他回去。

过了一天,已是除夕,马山甫忽然要请起客来,高高兴兴的和陆韵仙说了,叫他预备一个双台。那知请客条子发了出去,请的客人倒有大半不来。相帮跑了半天,只请到了三位客人,其余的影都不见。马山甫见连着自己只有四个人,四个人吃一个双台,面子上下不过去。只得自己跑出来,要想去请几个同乡,恰恰遇见了章秋谷。马山甫见了大喜,一把拉住了那里肯放。章秋谷被他拉着打一个转儿,又请了三个客人,马山甫大喜道:“好了,好了。今天这个双台吃得成了。”说着不由分说,把他们拉到清和坊陆韵仙院中。大家坐下,立刻摆起台面来。

秋谷的意思,本来狠不愿意来吃酒,只说一个人有一个人的事情,怎么到了除夕还在堂子里头吃酒?又不算年夜饭,又不算辞年酒,这算个什么路道?无奈马山甫死拖活拉的不肯放手,只得勉勉强强跟了来。又见陆韵仙对着马山甫不瞅不睬的,满面露着不愿意的样儿,不由得心上添了几分不快。章秋谷看了多时,便对着陆韵仙微微冷笑道:“今天我们这几个人里头,那一个得罪了你,请你讲给我听听。我看你今天满身满脸都是一付不高兴的样儿,这是什么道理?”

陆韵仙听得秋谷挑他的眼,便吃了一惊,抬起头来看了秋谷一眼,觉得这个人丰仪照眼,华彩凌云,嫖客里头难得遇着这般人物。不由得把头一低,大宽转的飞了一个眼色,一面微微的笑道:“章大少,阿好请耐勿要扳倪格差头。倪有啥怠慢格场化,请耐包涵点。”说着便立起身来,自己去斟了一碗茶,走过来递给秋谷;嘴唇一动,眼睛一瞟,低低的笑道:“章大少,请用茶。”秋谷见了,自然心中会意,便也对着他把头略略的摇了一摇,口中打着苏白说道:“先生勿要客气,谢谢耐,对勿住。”陆韵仙见了也不开口,只把嘴披了一披,扭过身躯回身就走。

陆丽娟坐在秋谷背后看得明白,忍不住“格”的一笑。这一笑不打紧,只把一个陆韵仙笑得连耳根带脖子都扯得通红,瞅了陆丽娟一个白眼,赌气仍旧跑到马山甫背后坐下。马山甫眼睁睁看着他们,摸不着一些头脑。

这一席酒,虽然马山甫做了主人,殷殷相劝,却是已经到了这般时候,一班客人大家都未免有些琐琐屑屑的事情,便不等终席,一个个告辞要走。马山甫也不好强留,一时间几个客人都走了。只有章秋谷一个人还坐在那里,见大家都走了,便也立起身来道谢告辞,却悄悄的和马山甫说道:“我看这个陆韵仙的样儿,和老表叔不见得怎样的要好。老表叔如若有什么事儿,只顾和我讲个明白,或者我可以和老表叔帮个忙儿也未可知。”马山甫这个时候还是糊里糊涂的,只认着陆韵仙待他不差,这一笔钱是本家敲他的竹杠,和陆韵仙不相干。便随口谢了秋谷几句,只说没有什么事情。

秋谷心中暗笑,不便再说,便辞了马山甫,一径回到新马路公馆里来。见了太夫人,也没有什么话说,只说了几句闲话便退出来。只见他那位夫人同着陈文仙两个人正在那里指挥着铺设炕围椅垫,秋谷也略略的料理一回。

江南的风俗,到了除夕晚上一定要接什么财神,又是供什么佛。秋谷虽然不信这些事情,却是老母在堂,不便违拗,自然也要依样葫芦的忙碌一番。一会儿摆上家宴来,太夫人坐在中间,秋谷坐在上首,他夫人和陈文仙便一顺坐在下面,大家说说笑笑的十分高兴。差不多吃到十二点钟光景,方才撤席。

这个时候,大家都在那里迎接灶神,只听得一片的爆竹声喧,“劈劈拍拍”的络绎不绝。秋谷也胡乱跑到厨房里面去磕了几个头,便走出来和老太太说道:“要到朋友人家去辞年,恐怕有几个知己些的人要留着吃年夜饭,一时不得回来。”太夫人不晓得上海的风俗,只说上海地方的人家都是这个样子,便点一点头。

秋谷回到自己房里头去换了一身衣服,正要走时,恰恰陈文仙走进房来,对着秋谷低鬟一笑道:“耐到底要到啥场化去吃年夜饭?搭倪讲明白仔洛去。”秋谷还没有开口,他夫人接着说道:“那里是到什么朋友那里去辞年,只怕你这个朋友是住在堂子里头的!”秋谷听了,对着他们两个人一笑,又朝着他夫人摇一摇手道:“你不要这般不高兴,等回儿我回来,好好的和你辞一个年,总算我陪个不是何如?”

他夫人听了不由得两颊生红,别转头去啐了一口道:“不要这般混说,快些去和你的相好辞年罢!我是用不着的!”文仙在旁边听了,也不觉回头一笑,对着秋谷把眼睛瞟了一瞟。秋谷哈哈的笑着,一路走下楼去,坐上包车,风驰电掣的到久安里来。

到了陆丽娟院中,只见辛修甫和王小屏两个人已经坐在那里,秋谷见了大喜。

不一回陈海秋也走了进来。原来秋谷日间在久安里的时候,已经写了条子叫相帮送去,约他们十二点钟在久安里吃年夜饭。这几个人见是秋谷请的,知道不能不到,只得大家拨冗到来。陆丽娟问着秋谷道:“阿再要去请啥客人?”正是:

残年风雪,谁开东阁之樽?良夜迢遥,应有高唐之梦。

不知后事如何,且听下回分解。





第一百三十二回 设华筵良朋守岁 兜喜神名妓迎春





且说章秋谷听了陆丽娟的话,便对他摆一摆手道:“没有别的客人,你就叫他们摆罢。”丽娟听了,便指挥着一班娘姨、大姐七手八脚的排起来。秋谷便邀客入座,辛修甫等便也随意坐下。

秋谷看那桌子上的菜时,见齐齐整整的排着十六个碟子,弄得十分精致。堂子里头年夜饭的菜,本来原只得十二个碟子,四大四小,一个暖锅。如今陆丽娟格外要好,在例菜之外又另外添了几样。一会儿,相帮带着红缨帽送上鱼翅。秋谷见了,不觉把双眉一皱。陆丽娟知道秋谷的性情,见他眉头一皱,便对他笑道:“耐勿要实梗嗫,生来规矩是实梗样式呀。”秋谷一笑,也不开口。等了一回又送上一个暖锅,开了盖看时,却是一锅绝清的鸡汤,没有一些儿渣滓。接着又送上几个盘子,盘子里头都装着生片的山鸡片、腰片、鸡片、肉片。原来陆丽娟知道秋谷喜欢吃这个东西,特地为他预备的。秋谷见了心中大喜。无奈虽然爱吃,方才已经在家里头吃了一顿来的,肚子里装不下许多。只吃了几口汤,烫了几片山鸡片吃了,就放下牙箸不吃。陆丽娟还在那里尽着让他,秋谷摇一摇头道:“方才吃饱了来的,不能多吃。难道在你这里我还和你客气么?”丽娟听着方才罢了。

一回儿大家散席,立起身来。秋谷意思想要回去,丽娟拦住道:“故歇辰光差勿多天亮快哉,耐搭仔俚笃三位来浪倪搭坐歇,大家讲章讲章。晏歇点等天亮仔,大家一同出去兜喜神方阿好?”秋谷还没有答应,辛修甫先拍手道好,陈海秋和王小屏听了也都十分高兴,秋谷便听了陆丽娟的话儿,回身坐下。

丽娟叫娘姨泡上茶来,秋谷端起来茶碗来随便喝了一口,觉得这个茶和方才的茶不同,满口清醇,风生两腋,连忙拿起来仔细看时,只见细叶浮香,螺芽荡影,竟是色、香、味三者兼备的好茶。秋谷便问陆丽娟道:“你们这里那里来这样的好茶?”丽娟道:“格个茶叶是江西客人送拨倪格。倪也勿晓得俚好勿好。倪搭多煞来浪,耐要末拿两瓶去阿好?”秋谷听了大喜,连忙道谢。丽娟斜了秋谷一眼笑道:“耐搭倪客气起来哉,阿是?”秋谷听了微微一笑,也不再说。

不多一会,已经听得远远的鸡声唱晓,玻璃窗上微微的透进曙光。陆丽娟忙忙碌碌的对着镜子洗了个脸,重匀粉面,再画蛾眉,换了一身衣服,朝着秋谷笑道:“难倪去罢。”秋谷听了,便同着辛修甫等立起身来,同着陆丽娟走下楼梯。还有几个陆丽娟的同院姊妹也同着走在一起,大大小小男男女女约有十几个人。

秋谷一面走着,一面细看他们的打扮:只见他们一个个都是戴着满头珠翠,身上也有草上霜皮袄,也有狐皮袄。下面都是大红绉纱百褶宫裙,飘飘的垂着许多裙带。陆丽娟还穿着一双红缎弓鞋。一个个都打扮得裙袄鲜明,花枝招展。一群人走出大门,陆丽娟立定了脚道:“今年喜神方是东南方,倪穿过同庆里去阿好?”大家都依着他的话儿,一直走进同庆里去。在四马路兜了一个转身,在路上遇见无数的倌人,都是出来兜喜神方的。一个个都是打扮得满面春情,一身香艳。也有几个倌人认得秋谷的,都朝着他点头微笑。秋谷也略略招呼,只觉得眼睛里头印着无数的美人影子,差不多就有些像那河阳满县之花,金谷回风之队。秋谷一边走着,一边细看,心上十分高兴。

兜了一回,大家都回到久安里来。秋谷和修甫等方才坐下,只见陆丽娟笑迷迷的,走过来对着众人说道:“难末倪要拜年哉。”修甫等连忙拦住。大姐阿金妹在旁笑道:“二少,今朝开仔果盘去罢。”秋谷不答,只点一点头。阿金妹便招呼出去。一会儿果盘上来,又有许多娘姨、大姐的小孩子,七长八短的和秋谷等拜年。

秋谷便拿出几张钞票递给陆丽娟,叫丽娟替他开发。丽娟接过来,点了点头道:“用勿着实梗几化啘。”秋谷摆手道:“你去开发就是了,不要管他多少。”修甫和小屏等也都拿出一张钞票来给那几个小孩子做压岁钱。秋谷略坐一回,便立起身来同着众人走了。

一个新年里头,秋谷虽然没有什么事情,但人来人往的,许多朋友都来拜年,秋谷也免不得一家一家的挨门回礼,倒着实忙了几天。直忙到过了正月初五,方才略略空闲些儿。

到了初六那一天,秋谷早上起来,刚刚吃过点心,忽然家人传进一张名片来,说有人拜会。秋谷接过名片看时,只见名片上端端正正的写着“王定”两个大字。

原来这个人叫做王安阁,也是秋谷的同乡。秋谷平日之间虽然和他相识,却彼此不甚往来。当下秋谷看了这个名片,心上狠觉得有些诧异,暗想他无缘无故的来找我做什么?便叫家人请在书房里坐,自己穿上马褂,随后走进书房。

王安阁一见了秋谷的面,便慌慌张张的说道:“你们令表叔病重得狠,现在住在我们轮船公司里头,请你去探望一下。万一出了什么乱子,你们是亲戚,大家也好有个商量。”秋谷听了摸头不着道:“你说的究竟是那一个?我们亲戚在上海的多得狠,表叔也不止一个,你这样没头没脑的,我知道是说的那一个呢?”王安阁听了方才说道:“就是那位马山甫先生,你难道不知道他的事情么?”秋谷愕然道:“我那里知道他什么事情!只去年除夕的那一天,他还在陆韵仙那里请我吃酒,我看他精神狠好,那里会病得这般快当!”

王安阁听了叹一口气道:“他这个病,就是为着陆韵仙身上气出来的,你还提什么陆韵仙不陆韵仙。”秋谷听了吃了一惊,连忙问道:“到底怎么的一件事情?

你且讲给我听听。“王安阁道:”这件事儿,说起来话长得狠,一时也说不清楚。

我今天是特地来请你过去,大家好商量个主意。马车现在门外,请你就去一趟,我们在马车里头慢慢的讲何如?“秋谷听了,自然答应。便立时立刻的同着王安阁走出大门,坐上马车。在马车里头,王安阁方才把马山甫和陆韵仙的交涉一一的和章秋谷说了一遍。

看官,你道马山甫究竟为着什么事情要气到这般田地?原来马山甫住在陆韵仙院中过了除夕,又到新春。正月初一那一天,陆韵仙自然好好的和哄着他,哄得马山甫十分欢喜。马山甫既然住在那里,自然免不得要开个果盘,又有许多相帮、娘姨都进来和他拜年。马山甫不知道开销的规矩,只说去年平空花了三百块钱,今年的一切开销都要省俭些儿,要想在陆韵仙身上省出这三百块钱来,便一古脑儿只拿了十块钱出来。陆韵仙大为诧异。无奈是正月初一,新年的第一天,不好向他争论。

过了一天,陆韵仙方才对着马山甫说道:“耐昨日仔格十块洋钱,到底还是付格果盘洋钿呢,还是拨俚笃格压岁洋钿?”马山甫听了道:“什么压岁不压岁钱,我是一古脑儿开销在里头的。”陆韵仙听了冷笑一声,也不言语。马山甫糊里糊涂的,那里看得出来。

到了晚间,陆韵仙又来和马山甫说道:“有件事体要来搭耐商量,勿知耐阿答应勿答应?”马山甫问:“什么事情?”陆韵仙道:“今朝倪房间里向有几个吃酒格客人,房间摆勿落哉。阿好委屈点耐,请耐到后房去坐歇,横竖耐是倪搭格老客人哉,总呒啥勿好商量格。”马山甫听得要他让出房间来给别的客人吃酒,心上自然不愿意;无奈听了陆韵仙的两句话儿,说他是老客人,心上又高兴起来,不因不由的点头答应。陆韵仙便同着他到后房坐下,又说了几句对勿住,便匆匆的走了出去。

马山甫一个人冷冷清清的在后房坐了半天,听着那前房的客人猜拳吃酒,又夹着倌人唱曲的声音,闹作一团。马山甫心上不由得有些发起酸来,便一个人踱出后房,到外面去打了一个转身,看得清清楚楚。原来陆韵仙本来有四个房间,马山甫占了一间,还有三间。马山甫起先只认着他几个房间里头都有客人吃酒,不料自己出去看了一看,只见那几个房间都静悄悄的,人影儿也不见一个,刚刚只有自己住的一间房间有个客人在那里摆酒。这原是陆韵仙有心怠慢马山甫,取瑟而歌的意思。

到了这个时候,马山甫就是个石头做成的人,也不由得大怒起来。想要立刻叫了陆韵仙出来问他,却又没有个人去叫他。好容易等了一回,方才见一个小大姐在房里头跑了出来,马山甫连忙叫住他,叫他去叫陆韵仙出来。那个小大姐听了也不答应,也不回言,只抬起头来看着马山甫“嘻”的一笑,便跑了开去。马山甫气得发昏。又停了一会,见陆韵仙的跟局大姐出来,马山甫气冲冲的和他说了。那大姐冷冷的答应一声,回身走进房去。不多时又走了出来,只对着马山甫说道:“先生呒拨工夫。”刚刚说了这一句,便把身体一扭,回身便走。

马山甫这一气非同小可,想要闯进房去发作一场,转念一想:“上海地方比不得别处,堂子里头是不能混闯房间的。万一个别的客人不答应起来,那时自己的气出不成,倒反受别人的一场羞辱。”想来想去想不出一个法儿,只得忍着一肚子的气,跑到轮船公司来找王安阁。正是:

眼前恩爱,都成一霎之花;心上温存,剩有双栖之影。

欲知后事如何,请看下回分解。





第一百三十三回 让房间安心慢客 受讥评当面坍台





且说马山甫忍着满肚子的气恼,跑到轮船公司来找王安阁。原来这个轮船公司开设在老闸桥左首,专走苏、杭、常、镇一带的内河小轮。马山甫也是个有股份的东家。王安阁就是轮船公司的经理,也是马山甫荐进去的。马山甫平日之间和王安阁狠是要好,两个人无话不谈。这一番马山甫受了陆韵仙的一场怠慢,心上气忿不过,没奈何想要来和王安阁商量。

当下见了王安阁的面,马山甫便把这件事情自头至尾和王安阁说了一遍,要请王安阁和他想一个报复的法儿。王安阁想了一想,一时间也想不出什么主意来,便道:“今天时候已经不早,你也不必再去住在他那里,就在这里住了一夜罢。明天我们两个同到他那里去问他,看他怎样的说法。”马山甫听了,只得点头答应,就在公司里头将将就就的住了一夜。

马山甫住在陆韵仙院中是热闹惯的,这一夜鸳鸯瓦冷,翡翠衾寒;凄凉云雨之台,辜负高唐之梦。翻来覆去的睡在床上,对着一盏孤灯想起千般心事,再也睡不着。一直醒到五更鸡唱,方才略略的睡着了一回。等到醒来,已经十点多钟。王安阁陪着他吃了点心。依着马山甫的意思,这个时候就要同着王安阁到陆韵仙那里去问他。倒是王安阁拦住他道:“你也是个老白相了,难道还不知道堂子里头的情形?

这个时候,那些倌人正在那里做他的好梦,那里就会起来?不如等回儿,在这里吃过了饭去罢,何必这般性急。“马山甫听了觉得不差,只得依着他的话儿等会再去。

王安阁见马山甫没精打采的,神气十分索漠,便劝了他一番。马山甫虽然口里头胡乱在那里答应着他,却又是咳声叹气的,没有一些笑容。开上饭来,马山甫也只吃了几口便不吃了。忙忙的洗过了脸,便催着王安阁一同到清和坊来。

到了陆韵仙院中,走到楼上还是静悄悄的,陆韵仙还没有起来。马山甫不管三七二十一,同着王安阁就要闯进房去。早有一个娘姨抢步过来拦住马山甫,低低的笑道:“马大少,对勿住,格面房间里坐罢。”马山甫听了,知道那间房里有了客人,心上更加不快。只得回过身来,在对面一间房间坐下。那娘姨也连忙跟了过来。

马山甫对着他冷笑道:“你们这里的空房间也多得狠,为什么你们先生定要把住夜客人留在我住的这一个房间里头,这是个什么意思?还是有意要和我过不去呢,还是怎么样?”那娘姨听了呆了一呆,便笑道:“马大少,勿要动气。倪先生一径搭耐蛮要好,洛里会有心搭耐过勿去?昨日仔格个客人,吃醉仔酒,坐勒浪格间房间里,一动才勿肯动,倪也只好让俚去歇。”

马山甫听了,想了一想又道:“既然如此,你们先生为什么不叫他到别个房间里去吃酒,一定要占我的房间呢?”那娘姨又分辩道:“勿瞒耐马大少说,格几间房间才是几个客人老早就定好来浪格。”马山甫不等说毕,又道:“就算竟是如此,也要等客人来了再说让的话儿,为什么又要预先叫我让呢?况且到了后来,我要叫你们先生出来问他一句话儿,你们先生又为什么不肯出来呢?”那娘姨一时支吾不过来,只得吞吞吐吐的道:“格号事体,倪也勿晓得倪先生心浪到底那哼格道理。

晏歇点等先生自家来搭耐说末哉!“马山甫听了便不开口。

王安阁插口问道:“你们先生起来没有?”那娘姨道:“起来格哉,勒浪有点事体。对勿住,马大少,请坐歇。”王安阁又道:“看这个光景,是昨天晚上有了住夜客人,所以到了这个时候还陪着客人没有起来?”那娘姨听了笑了一笑,也不说什么。

两个人等了一回,听得对面房间里头有男子咳嗽的声音。接着又听得陆韵仙的笑声“支支格格”的,也不知他和那男子说些什么,却只不见他走过来。只气得个马山甫心头出火,鼻孔生烟,恨不得跳过去一把把陆韵仙抓了过来。又等了好一回,方才见陆韵仙慢慢的走过来,鬟髻惺忪,衣裳不整,红添颊上,春透眉梢。见了马山甫,淡淡的叫了一声,又向王安阁把朱唇微微的动了一动,便一屁股回身坐下。

马山甫一股盛气的问道:“你昨天吃酒客人倒多得狠,统通都来了没有?”陆韵仙不慌不忙的答道:“自然来格啘,阿有啥勿来格道理。勿来末,也勿要搭耐商量房间哉啘。”

马山甫起先的意思,原只要陆韵仙自家认个不是,一天的云雾就也都消散了。

如今听了陆韵仙的口气说得甚是轻松,好像没有这件事情的一般,不由得心上又添上了几分烦恼,便冷笑道:“昨天我走的时候,明明看见几个房间里头都是空的,这是个什么缘故?”陆韵仙慢慢的说道:“才是客人先付仔洋钿定好来浪格。倪堂子里向规矩,客人吃酒付仔现洋钿末,赛过就是定房间,随便啥人总归要让还俚格。”

马山甫道:“这也罢了,为什么吃酒的客人还没有来,就先要占我的房间,难道别个房间不好吃酒的么?”陆韵仙听了顿了一顿,说不出来。

马山甫又道:“这些事情也还罢了,总都不必去管他。但是昨天晚上我要请你出来,和你讲句说话,我竟不肯赏我的光。这个道理,今天倒要请你讲给我听听。”

陆韵仙听了眉头一皱,口中说道:“喔唷,耐格闲话倒来得希奇笃啘!阿是耐今朝有心要来扳倪格差头?昨日仔耐叫倪格辰光,倪刚刚来浪应酬客人,呒拨工夫呀!

勿是实梗末,阿有啥勿来格!“

王安阁在旁听了半日,一言不发,听到这个地方实在忍不住,插进去说道:“你这个话儿倒也不错。吃了堂子饭,姓张的跑进来也是客人,姓李的跑进来也是客人,大家都是一样的客人。应酬了这一个,也要应酬那一个。最不好的是应酬一个,得罪一个。做了个倌人,连个客人都不会骗,这样的人,也就是个饭桶了!”

陆韵仙听得这几句话儿有些棱角,知道是有心骂他,便回过头来打量了王安阁一眼,对他笑道:“格位大少尊姓?‘王安阁道:”我姓王,去年不是马大少常常在你这里请我吃酒的,怎么你又不认得我起来?“陆韵仙笑道:”对勿住,王大少,勿要动气。倪有啥闲话勿到家格场化,请耐王大少爷包荒点。勿瞒耐王大少说,倪格碗把势饭格末叫难吃!王大少,耐想嗫,客人笃跑到倪堂子里向来,大家才是一门心思。看见倪搭再有第二个客人,心浪总归勿舒徐格。倪应酬格面格客人,归面格客人咦来浪勿高兴;应酬仔归面格客人,格面格客人咦来浪说闲话,叫倪应酬啥人格好呢?王大少,耐想想看:耐做仔倪,那哼一格弄法?王大少,耐勿是把势出身,洛里晓得倪堂子里向格苦!“这几句话儿,把王安阁顶得闭口无言,心中暗想:这个东西真是混帐,平空的取笑起我来!却又不好和他认真,只得冷笑了一声,一言不发。

马山甫见陆韵仙说得十分干净,竟丝毫不肯认错,只得气愤愤的说道:“不用说了,说来说去总是你的理长。总而言之,别人在你这里走动,你就当他是个客人。

我姓马的在你这里走动,你就当我不是个客人!我姓马的是不出钱的,白叨你们的光!“马山甫说到这里,正还要说下去,陆韵仙怫然变色,立起身来对着马山甫摇一摇手道:”马大少,耐格号闲话才勿要来搭倪说,客人笃到倪堂子里向来白相末,生来要出铜钿格。耐看见啥人勿出铜钿格呀?寻仔开心,再要勿出铜钿,上海滩浪也呒拨格号规矩啘!倪吃仔格碗把势饭,跑进来格才是客人,倪阿好赶俚出去?耐马大少肯照应倪,倪野是实梗样式;勿肯照应倪,倪野是实梗样式。独有耐末,总归是实梗枝枝节节,阿要鸭屎臭。“

马山甫平空被他抢白了一场,由不得心中大怒,双眉倒竖,面泛浓霜,一时间却又说不出什么来,只得大声说道:“好得狠,好得狠!你说出这样的话来,就算是你的应酬客人。其实你不愿意,只顾爽爽快快的讲就是了,何必做出这个样儿!

去年十二月里头和我讲的话儿,也不知是那个混帐东西的口中讲出来的。我只算自己糊涂,上了你的当就是了!“

陆韵仙的意思,本来原是有心激怒马山甫,好叫他从此不来。如今见了马山甫这般生气,正中下怀,不慌不忙的在那里看着他冷冷的笑。听了马山甫说出这几句说话来,刚刚枭了他的痛疮,不由得面上一红,两朵嗔霞从腮颊边直泛过来。略略的顿了一顿,也大声说道:“倪吃仔格碗把势饭,来格才是客人,呒拨啥格愿意勿愿意。倪也蛮明白来浪,耐来浪倪搭做做勿高兴哉;勿知看中仔格啥人,要想跳槽过去,实梗洛碰碰扳倪格差头。格末老实搭耐说仔,上海滩浪像耐实梗格客人,蛮多来浪,呒啥希奇。耐高兴多照应照应,勿高兴少照应照应,倪也勿见得来拉牢仔耐!客人有仔铜钿,勿怕做勿着倌人;倌人挂仔牌子,勿怕做勿着客人。耐心浪勿高兴末,随便耐去耶哼末哉!耐说上仔倪格当,倪倒问声耐:耐到底上仔倪啥格当哉?阿是倪骗仔耐格铜钿呢,还是骗仔耐格人?就算耐真格上仔倪格当末,也是耐自家情愿上当格,勿关得别人啥事。”正是:

妙粲莲花之舌,气煞瘟生;横遭白眼之讥,伤心冤桶。

不知马山甫说些什么,请看下文便知分晓。





第一百三十四回 忍恶气冤桶无颜 遭白眼瘟生致病





且说马山甫一腔盛怒的同着王安阁跑到陆韵仙那里去,只指望大大的数说他一场,出出这一肚子的闷气。不料陆韵仙有意要和他过不去,非但不肯自家认错,而且还连嘲带笑的顶撞了他一番,只把一个马山甫气得无可如何,眼瞪瞪的看着陆韵仙的脸,一个字都说不出来,只得说道:“总算我瞎了眼睛,一时晦气,平空的要住在你这里。如今也不必说了。”一面说着,一面喝叫家人收拾行李,立刻搬到轮船公司去。

陆韵仙听了也不留他,只淡淡的说道:“倪搭小地方,耐马大少勿中意,勿肯赏光,倪也勿好留耐。倪搭有啥怠慢格地方末,请耐马大少包涵点,勿要动气。”

马山甫这个时候已经气到极处,浑身乱颤,面白唇青,只连连的在那里催着家人快些收拾,陆韵仙说的话儿一句也没有听见。坐在那里等了一回,等得那家人收拾停当,便同着王安阁立起身来,对着陆韵仙要想说些什么,却又说不出来,只勉强冷笑道:“今天大年初三,我也不说什么。但愿你以后做的客人大家都好好的有始有终,不要像我这个样儿。”陆韵仙听了马山甫这句的话儿,不觉良心发现,面上一红,别转头去。

马山甫赌气同着王安阁走出陆韵仙大门,回到轮船公司来。马山甫埋怨王安阁:“为什么不帮着我骂他几句?”王安阁摇一摇头道:“我刚刚开口说了几句,他就夹七夹八的把我取笑了一场。他们吃把势饭的,那一张嘴练得就像个纯钢锥子一般,翻来覆去的凭着他怎么说法。你我们那里说得过他?”马山甫听了,想了一回道:“照你这样的说起来,白白的受他一场糟蹋,难道就是这样的罢了不成?”王安阁道:“你想有什么法儿?就是依着你的话儿,他也没有什么大不是,不过是有心怠慢客人,情形可恶。倌人们怠慢客人,也是上海滩上常有的事情,算不得什么希奇。就是他明明白白的自家承认有心怠慢你,你又把他怎么样?还是和他到茶会上去讲理呢,还是为了这般小事,和他到新衙门去打官司呢?”

马山甫听了想了一想,觉得王安阁的话也狠不错,实在没有什么法儿,便气忿忿的说道:“我不管三七二十一,约几个朋友去打掉他的房间,你看好不好?”王安阁连忙摇手道:“上海地方比不得内地,万一个他们去报了捕房,你又怎么样呢?”

马山甫道:“就是他报了捕房,我们也不见得吃亏。”王安阁道:“虽然如此,难道我们还为了这件事情和他打官司么?况且到了那个时候,你说他有心怠慢,是没有凭据的事情。我们打毁他的房间,却是件犯法的举动。万一个外国人说我们违背了他的马路章程,一定要公事公办起来,罚几个钱还在其次,我们的面子又放在那里去呢?你只要前前后后的想上一想,就知道这件事情不是可以动得蛮的。”

马山甫听了一言不发,只低着个头,坐在那里,王安阁和他说话他也不答应。

到了晚上,连晚饭也不肯吃。王安阁劝了他一回,马山甫只是给他一个不开口,王安阁也只得由他。一会儿大家睡觉,马山甫衣服也不脱,只和衣躺在床上。王安阁劝他宽了衣服再睡,他也不肯,王安阁只得自去安歇。

到了明天早上,王安阁绝早起来,走到马山甫房里来看他。只听得马山甫睡在床上,口中不住的在那里哼哼唧唧的哼。王安阁连忙揭开帐子看时,只见马山甫一个脸儿都烧得通红,合着两眼睡在那里。王安阁见了这般形状,心上便吃了一惊。

叫了两声,马山甫也不答应,只是昏昏的睡着。

原来这位马山甫出身富贵,平日之间父母溺爱,奴婢承迎,一呼百诺,要一奉十,从来没有受过这般的闷气。如今平空碰了这样一个钉子,自然的怒填肺腑,气塞胸膛。更兼以前和陆韵仙彼此要好的时候深情宛转,恩爱缠绵,海誓山盟,千金一刻。春宵苦短,双飞蛱蝶之图;宝帐四垂,同命鸳鸯之影。未免的朝朝交颈,夜夜成双,欢乐得过度了些,自然就把身体淘碌得虚弱起来。又受了陆韵仙这般怠慢,把天大的气恼都郁在心里,发作不出,登时就生起病来,满身发热,神识不清,来势十分沉重。王安阁见他病到这般模样,便不由的慌了手脚,连忙请了医生来和他诊脉。这个医生姓庄,外号叫做庄一帖;因为他两耳重听,大家又叫他庄聋聱。

当下庄聋聱看了马山甫两手的脉,又看了舌苔,细细的问了病原,只是摇头,口中说道:“这个病势来得不轻,你们须要小心些儿。”说着便提起笔来,忙忙的开了一张方子,递给王安阁道:“吃了这帖药再看情形罢。”一面说着,一面立起身来。

王安阁听着他这般口气,心上甚是担惊,便道:“请先生细看一看,他这个病究竟能好不能好?”庄聋聱见他啰苏,心中便有几分不快,冷笑道:“我们做医生的,只会给人治病,要保着别人不死,那是办不到的事情。就是我们自己,将来也要死的,难道做了医生就会有什么不死的秘诀不成?”

王安阁平空受了他一场抢白,不觉心中不快起来,暗想:怎么这个医生这般无礼?待要和他争论几句,却转念头想道:今天是请他来看病的,何必和他斗口?想到这里,便忍住了不开口。等得庄聋聱走了,连忙叫人去赎了药回来,自己看看煎好了,给马山甫吃了下去,却也没有什么动静。

不料过了一天,到了夜半的时候,马山甫忽然沉重起来,口中谵语,身上烧得就如炭火一般,头上却没有一些汗气,昏昏沉沉的连人都不认识。时时刻刻的在床上坐起身来,掀开盖的棉被,要走下床去。口中只嚷着要到陆韵仙那里去,问他为什么这样的没有良心。慌得王安阁连忙把他按住了,仍旧捺他睡下,闹了一夜。

王安阁十分着急,恐怕马山甫有些好歹,他一个人担不起这般郑重,便想起章秋谷来。马山甫常常的和他讲起,章秋谷的为人怎样的缓急可恃,怎样的仗义多才。

王安阁本来原和秋谷相识,听了马山甫这般说法,觉得心上十分佩服这个人。如今忽然想起他来,便立时立刻的赶到章秋谷公馆里头去,把章秋谷拉了出来。在马车里头,方才把这件事情的始末根由,一一的和章秋谷说了。

秋谷不觉大怒道:“天下那有这样的事情!一个吃把势饭的倌人竟敢这般放肆,真是没有王法的了!或者这个里头另外还有什么缘故,也未可知。”王安阁道:“这里头有别的缘故没有,我也弄不清楚。据山甫自己口中讲起来,却没有什么别情在内。”正说着,马车已经到了公司门外,停住车轮。

秋谷和安阁都跳下马车,走进去见了马山甫。只见他脸上通红,浑身发热,连嘴唇都是紫黑的。见了章秋谷也不认识,只是忽笑忽哭的,口中混说。秋谷见了这般病势,不由得也是吃惊。便走上去,把手向马山甫头额上边一摸,只觉得炙手可热,烧得甚是利害。秋谷取过几张药方来看了一看,只见方子上开的药味,都是些荆芥、防风、陈皮、甘草,一派稀松的药。秋谷看了道:“这些药都是不中用的。

病势重到这般田地,怎么还吃这些平平常常的药?“说着,便低着头想了一想。

王安阁在旁看了,也不知他想的什么,只对着秋谷说道:“这件事情真是累赘,偏偏的病在这个地方!万一个有些好歹,这个干系放在那一个身上呢?”说着心上二十四分的着急,咳声叹气,顿足捶胸,只急得在屋子里头走来走去的,四面乱转。

秋谷见了便和他说道:“你也不必这般着急,天有不测风云,人有意外祸福,那里预先料得定?又不是你害他生病的,这也是没有法子的事情。倒是他们老太太那边,该应打个电报去通知一下,这才是个道理。”正是:

三更怪雨,凄凉病榻之禅;一夜西风,憔悴无家之客。

欲知后事如何,且看下文分解





第一百三十五回 发电信开函惊老母 抱不平疗病出奇方





且说章秋谷见马山甫病势这般沉重,心上也觉得有些不妥当,便和王安阁商量,先打了一个电报到常熟去给马山甫的老太太。只说马山甫病危,要请他老太太赶紧到上海来,和他设法疗治。一面又和王安阁说道:“据我看起来,我们这位老表叔的病,分明是被陆韵仙气出来的,吃这些草根树皮那里中用?不如还是去把陆韵仙设法叫来,叫陆韵仙在他面前自家认错,好好的安慰他一番。解铃还仗系铃人,或者竟有效验,也未可知。”王安阁听了道:“你的话虽然有理,无奈陆韵仙这个烂污货十分可恶,他不肯自家认错,我们有什么法儿呢!”秋谷笑道:“这个不难,待我去和他讲就是了。老实说,也不怕他不肯。”

王安阁口中虽然在那里答应,心上却狠有些不相信的意思,面子上却不好说出来。章秋谷见了王安阁这般模样,心上早已明白,便对王安阁说道:“这个时候,已经差不多十二点钟,我就到清和坊去,把陆韵仙立刻叫来。”说着便匆匆的跳上马车,一口气赶到陆韵仙院中。

陆韵仙刚才起来,正在那里梳洗,见章秋谷走了进来,心上虽然有些诧异,却只说他是来找马山甫的,笑迷迷的起身让坐,口中说道:“章大少,阿是来寻马大少格?马大少勿知为仔啥格事体,前日仔搭倪反仔一泡,搬仔物事去,倒说就此勿来哉呀──”

秋谷不等他说下去,便截住他的话头道:“如今闲话少说,你们那位马大少为了你的事情在那里生病,病得九死一生。你们总算是老相好,难道不去看看他么?”

陆韵仙听了呆了一呆道:“耐格闲话说得勿明勿白,啥格马大少为仔倪格事体勒浪生病,阿是真格呀?”秋谷微微一笑道:“我们客客气气的,难道我在你面上会讲假话不成?”陆韵仙听了,心上觉得甚是诧异,口中说道:“马大少生病末,勿关得倪啥事啘。为仔倪啥格事体呀?”秋谷道:“据他自己讲,是给你气出来的。我也不知道你们两个人究竟是怎么的一件事情。”

陆韵仙听了顿了一顿,还没有开口,那站在他身后和他梳头的娘姨便插口说道:“格末真正阿弥陀佛,天理良心!马大少来浪倪搭,倪先生一径搭俚蛮要好。啥格俚自家生病,倒说是拨倪先生气出来格呀!”秋谷道:“如今也没有工夫来讲这些闲话,只要请你梳好了头,立刻到轮船公司去看他一趟,好好的安慰他一番,或者他这个病竟会好起来,也是论不定的。”

陆韵仙听了,正在那里沉吟不决,那娘姨又连忙说道:“马大少生病末,豪燥请郎中先生看嗫!倪先生咦勿是郎中先生,去做啥格事体呀?”秋谷听了,正色向陆韵仙说道:“据我看来,今天是一定要请你去一趟的。马车现在门外,你梳洗好了,我们一同去罢。”陆韵仙低头不语。那娘姨又向陆韵仙使一个眼色道:“昨日仔王大人说,要搭耐坐马车呀。到仔马大少格搭转来再坐马车,阿来得及呀?”

秋谷听了那娘姨的话儿,心上觉得狠有些儿不高兴。又见陆韵仙低着个头,在那里踌躇不决,暗想我好意留还他们的面子,好好的和他讲,他们倒这样的不识好歹起来。既然如此,我也乐得教训他们一顿,借此好燥燥自己的脾。想罢,便忽然变转脸皮,对着陆韵仙冷笑道:“你不用在那里踌躇不决。老实和你说,吃了把势饭的人,身体就不是自己的。今天你愿意去,也要你去上一趟;你就是不愿意去,也要你委屈一下,去上一趟。我劝你还是爽爽快快,同着我快些去罢。”

陆韵仙听了章秋谷的话儿,说马山甫的病势十分沉重,心上本来有些害怕。如今又听得秋谷这般说法,未免心上也就有些不快活起来,便也冷冷的笑道:“依仔耐章大少实梗说起来,是倪一定要去格哉?不过倪今朝轧实有点事体,呒拨工夫,阿好明朝去仔罢。”秋谷慢慢的道:“不管你有工夫没工夫,一定要请你今天去一趟。”

陆韵仙听了心上更加不快,便似怒非怒的瞅了秋谷一眼道:“既然章大少实梗说法,倪倒说句笑话,比方倪定规勿去末,耐章大少那哼弄法?”娘姨听着章秋谷的话儿说得这般强硬,心上狠不愿意,也在旁边笑道:“真格比方倪先生勿肯去末,耐章大少阿有啥格法子?”

秋谷听了,不慌不忙的道:“天下的事情,总无非是讲个情理。况且你们把势里头的人,虽然是末等的生涯,却是头等的规矩。好好的客人,既没有欠你们的钱,又没有嫖你们的帐,平空的把他这般怠慢,这里那里来的规矩?你们倒讲给我听听,也好叫我见识见识。”陆韵仙和那娘姨起先听了章秋谷的话儿,还只道他是随口讲的顽话。如今见秋谷正颜厉色讲出这几句话来,字字当行,言言有理,方才吃了一惊,知道章秋谷不是个好缠的人物。

陆韵仙想了一想,方才开口说道:“章大少,耐勿要去相信马大少格闲话,俚耐一塌刮子才是瞎说。倪搭待俚一径才是客客气气,啥格怠慢勿怠慢呀。”秋谷听了哈哈的笑道:“明人面前不讲暗话。我章秋谷既不是那种没用的瘟生,又不是那般颟顸的饭桶。你们在我面前,也不必讲这样敷衍的话儿,只老老实实的,给我讲了真话就是了。”陆韵仙听了口中还想支吾。秋谷接着说道:“如若你们一定不肯讲出来,我也不能勉强。只怕你们今天在我面前敷衍得过去,回来到了茶会上的时候就敷衍不过去了。”陆韵仙听得秋谷话风利害,便又吃一惊,连忙转口笑道:“倪也不过说说罢哉。耐章大少面浪,阿有啥勿去格道理?”秋谷微微一笑,也不开口,看着陆韵仙梳好了头,立起身来换了一件衣服。

秋谷又对他说道:“你和马大少大家好好的,怎么会平空闹出这样的岔子来?

这里头究竟是个什么道理?其实去年我在这里吃酒的那一天,看着你那般模样,就知道有些不妥。马大少糊里糊涂的看不出来。究竟你们为了什么原由,要和他这样的过不去呢?“

陆韵仙听了,便袅袅婷婷的走过来,拉着秋谷的手,到榻上并肩坐下,细细的把马山甫如何不肯借钱,本家和房间里娘姨如何的背地里埋怨他,前前后后的许多情节一一和秋谷说了。秋谷方才明白,笑道:“我本来原在这里诧异,你们两个人以前既是这般要好,为什么忽然这般的大决裂起来?但是这件事情,马大少虽然自家不好,你们却也过分了些。吃了堂子饭,就有堂子里头的规矩,怎么把房间里头的客人赶了出来,让别人在房间里摆酒,这又是那里来的规矩?”

那娘姨听了还想遮盖,便又插口道:“勿瞒章大少说,格日仔倪间搭格房间轧实勿空,才是客人笃定好来浪格。”秋谷听了,瞪了那娘姨一个白眼道:“你这样的话儿,只好对着姓马的讲,怎么对着我也说出这样的话来!就算依着你的话儿,那一天的房间都是客人预定,马大少是住在你们这里过年的长客人,难道不是预定的么?难道别人可以定你们的房间,姓马的就定不得的么?老实和你们讲罢,你不用在我面前讲这般大话,就是林黛玉、金小宝这样的红倌人,在正月十五以前,也没有多少吃酒的客人。不要说你们先生算不得什么有名的红倌人,那里会有这般生意。你难道把我也当作马大少么?”

一席话,说得那娘姨闭口无言。陆韵仙脸上却添了一层红晕,瞟了那娘姨一眼道:“耐阿好少说两声,唤唤喤喤,勿知算啥格样式。”说得那娘姨撅着个嘴跑了开去。陆韵仙方才拉着章秋谷笑道:“一塌刮仔才是倪格勿好,耐章大少勿要动气。

故歇随便耐要那哼,倪总呒啥勿肯。“说着不觉脸上又是一红。秋谷不觉一笑道:”这件事情本来不干我事,我不过出来抱个不平罢了。我也没有什么生气,我也不要什么。我就要什么,也没有这般福分。“

陆韵仙见秋谷的话儿说得针锋相对,瞅了秋谷一眼,低下头去。秋谷道:“你们那位马大少,病重得狠,如今事不宜迟,我们赶紧同去看他一看。”陆韵仙听了,便懒懒的立起身来,也不带娘姨,同着秋谷上了马车。

秋谷在马车里头又教了他几句说话,说着又对他笑道:“你只要把初次哄骗马大少的那些勾心摄魄的话儿,翻过来和他再讲一遍,管保他的病就会立时立刻的好起来。”陆韵仙听了,红着脸,把秋谷打了一下道:“倪骗马大少啥格闲话介,阿是耐听见格?”秋谷笑道:“你也不必瞒我。倌人们和客人相好,总有几句山盟海誓的话儿,方才拉得住客人们的心。这是你们做生意不得不如此,有什么不好意思?”

陆韵仙被秋谷顶住了,没有话说,只得笑道:“听耐实梗说起来,比仔倪做倌人格再要熟点,像煞耐倒是格倌人出身。”秋谷听了,也笑道:“我好意教你,你倒反把我取笑起来。如今世上的人,真是没有良心!”

秋谷和陆韵仙一面说着话儿,那马车走得飞快,不一刻,早已到了轮船公司门外。秋谷同着陆韵仙急急的走到里面。马山甫一个人睡在那里,口中还在那里喃喃的说着谵语道:“你们同我到清和坊,我要问问他,为什么这样的和我过不去?”

秋谷听了也觉心酸,便指挥陆韵仙,叫他走上前去。陆韵仙见马山甫病到这般模样,心上也觉得有些害怕起来。正是:

爱河滚滚,难浮灵府之槎;情海茫茫,不见回头之岸。

不知马山甫性命如何,且听下回分解。





第一百三十六回 抱沉疴三宵占勿药 起乡心千里整归装





却说章秋谷同着陆韵仙来看马山甫的病,陆韵仙走上一步,看着马山甫病到那般模样,昏沉不醒,遍体发烧,心上不觉有些害怕,趑趄着脚儿不敢走近身去。章秋谷见了,便和他说道:“你不用害怕,且走过去叫他一声,看他知道不知道。”

陆韵仙听了,没奈何只得走近床前,低低的叫了一声:“马大少。”马山甫仍是不应,只合着眼睛呼呼的喘气。陆韵仙又叫一声,马山甫又不答应。陆韵仙到了这个时候,由不得天良发现;想着那往日的缠绵,看着他这般的委顿,心上一酸,两行珠泪直挂下来,不由得轻移莲步,走到马山甫的身旁;就在床沿上坐了下来,一手拉着马山甫的手,低下头去,在马山甫耳边叫了一声。

说也奇怪,马山甫病了几天,热得昏昏沉沉的,连人都不认得。吃下药去也如石投水,不见一些儿效验。如今听了陆韵仙叫他一声,好似触着了电气一般,登时浑身一震,睁开双眼,把陆韵仙看了一看,忽然说出话来道:“我病了几天,你也不来看我一看。”陆韵仙见马山甫忽然和他说起话来,竟是清清楚楚的,不像个病重的样儿,心上也不由得暗暗称奇。王安阁站在门外,看了也觉得甚是诧异。章秋谷更是眉飞色舞的,看着王安阁道:“何如?”王安阁只点一点头,微微含笑。

陆韵仙又对马山甫低低说道:“马大少,耐啥洛好好里生起病来哉呀,耐自家保重点嗫。”原来马山甫病了几天,心上糊里糊涂的,把陆韵仙和他过不去的事情,都忘得干干净净。如今听得陆韵仙问他为什么生病,猛然把这件事情记了起来,呆呆的看着陆韵仙。看了一回却说不出什么来,只对着陆韵仙长叹一声,流下两点眼泪。

陆韵仙见了,心上狠觉得有些过意不去,便连忙取出一方丝巾和他拭泪,在他耳旁轻轻的说道:“耐勿要实梗动气,一塌刮仔格事体,才是倪勿好。耐自家身体要紧,豪燥点好好里养病,勿要去心浪瞎转啥格念头。阿晓得?耐来浪倪搭,也总算老客人哉,倪有啥得罪耐格场化末,耐包荒点,勿要捉倪格过意。耐有啥闲话,只管搭倪说末哉。就是耐心浪向勿舒齐,骂倪一场,打倪一顿,倪倒也呒啥希奇。

像实梗气坏仔耐自家格身体,啥犯着呀!“马山甫听了陆韵仙这几句话儿,一霎时好像那甘露沁心,醍醐灌顶,登时精神就爽快了许多。觉得这几句温柔宛转的话儿,甜迷迷的钻进耳朵,软融融的直走心脾,五脏六腑没有一处不走到,浑身骨节没有一根不松爽,直比那华佗、扁鹊的神方,起死回生的灵药,还要效验些儿。

停了一停,马山甫心上还有些糊里糊涂的不得明白,便问着陆韵仙道:“你怎么跑到这里来,那一个叫你来的?”陆韵仙听了,回过头来看了秋谷一眼。秋谷远远的对他做一个手势,陆韵仙会意,便道:“倪听见耐来浪生病,心浪搭耐发极,实梗洛跑得来看看耐格呀!呒拨啥人叫倪来啘。”马山甫听了心上更是欢喜,便大声说道:“你这话儿是真的么?”陆韵仙道:“自然真格啘!阿有啥假格呀!”

马山甫听了更喜,便拉着陆韵仙的手,想要坐起身来。不想病了几天,饮食不进,那里坐得起!只觉得眼迸金花,耳鸣石磬,早挣出一头冷汗来,马山甫不由得“阿呀”一声道:“怎么我病了几天,就会病到这般田地!”陆韵仙连忙说道:“耐自家勿晓得,耐生仔病,别人家替耐急煞快,豪燥点勿要实梗。”说着不觉面上一红,回转头来瞟了秋谷一眼。秋谷知道他有些话儿不好在众人面前讲出来,便拉着王安阁走到外面,凭着陆韵仙和马山甫两个人在房内。

陆韵仙趁着这个当儿,着实的安慰了马山甫一番。至于他那安慰的话儿究竟是如何说法,在下做书的当时没有听见,不便捏造一番说话出来,只好请诸位看官自家去揣摩想象的了。

如今闲话休提。只说章秋谷和王安阁在外面坐了一回,听见马山甫嚷着要吃粥,秋谷大喜,便叫王安阁赶紧送进去。马山甫吃了一碗,又微微的出了一身汗,秋谷方才走进房去和他相见,却绝不提起去叫陆韵仙的事情。马山甫见了秋谷,也略略的应酬几句。秋谷也随便讲了几句套话,便走了出来。

陆韵仙也走到外面。秋谷见了陆韵仙,便对他笑道:“何如?我的主意怎么样?”

陆韵仙笑道:“格末真真诧异,倪自家也勿懂啥格道理。”说着,便又向秋谷说道:“故歇马大少好仔点哉,倪转去仔,明朝再来,阿好?”秋谷听了,摇一摇头道:“这个不能,你看他现在虽在好些,却是靠不住的。只好委屈你在这里住上几天,等马大少病好了回去。”陆韵仙听了呆了半晌,方才说道:“格是勿局格嗫。”秋谷道:“有什么不行?马大少的病是为你身上起的,论起理来你也该应在这里陪他几天。”陆韵仙道:“来浪间搭住几天,倒呒啥希奇,不过倪搭有几几化化事体──”

陆韵仙说到这里,秋谷截住他的话道:“我知道你的事情,无非是要应酬客人,不能分身。只要和本家说明,有什么客人来,只说你有事情到苏州去了,四五天就回来的。客人叫局,也好托别的倌人代应,有什么大不了的事情?”陆韵仙听了推托不得,呆了一回只得又道:“倪是倒呒啥,就怕倪搭格断命本家勿肯。”秋谷哈哈笑道:“这个事情,交给我就是了。本家不肯放你住在这里,无非怕少了生意,我立刻同着你回到清和坊去,当面和他讲,每天包你二十个局就是了。你们挂着牌子做生意,也无非为的是钱。难道有了钱,还办不到么?”

陆韵仙见秋谷许他二十个局一天,心上虽然还有些不满意,口中却说不出来。

更兼方才已经领过这位章秋谷先生大教,知道是个平康巷里的惯家,烟花队中的侠客,想着就是不答应,也不中用,只得点一点头道:“只要本家呒啥闲话说,倪总归肯格。”说着又把秋谷的衣服拉了一下,洋洋的笑道:“耐章大少面浪嗫,换仔别人来是,倪就老实勿客气哉!”秋谷笑道:“承情得狠,承情得狠!如今闲话少说,我们就同去罢。”陆韵仙听了点头微笑,便同着秋谷坐上马车,跑回清和坊一弄。

秋谷到得院中,立刻把女本家叫了上来,和他说了情由,问他心上怎么样,还是肯与不肯?那女本家见了章秋谷丰裁凛凛,相貌堂堂,言语惊人,目光如电,先就觉得有几分怕他。又听得讲着马山甫的事情,口口声声的只说你们吃把势饭的人不该这般模样,把那女本家说得哑口无言。起先听得章秋谷的话儿要把陆韵仙留在那里伺应病人,心上大大的不愿意。直至秋谷说得每天包他二十个局,有一天算一天,方才心中欢喜,满口应承。却又对着秋谷说道:“倪有一句闲话要搭章大少商量:故歇刚刚开果盘格辰光,请章大少照应点倪。”秋谷笑道:“既然如此,就每天包你三十个局,何如?天下的事情只怕你不要钱,没有法子。只要你肯要钱,事情就容易办了。”说着,便叫陆韵仙收拾些随身衣服和梳洗的器具,带一个娘姨回去,也好遇事招呼。陆韵仙到了这个时候,知道不能不去,只得草草的收拾起一个衣包,同着秋谷一同前去。

果然马山甫自此以后,耳朵里头听着陆韵仙的娇音嘹呖,眼睛里头看着陆韵仙的倩影娉婷,一时展动便来纤手扶持,说句话儿又是芳心熨贴,药炉茗碗搀和着粉气脂香,春恨秋悲都化着欢苗爱叶。这几天之内,马山甫倒着实享些艳福,那病便一天一天的好起来。不到一礼拜,马山甫已经全愈。

马已甫的那位老太太和他夫人接了电报,吓得魂不附体,连忙星夜赶来。章秋谷见了马老太太,便把马山甫起病的情由和自己的打算细细的说了一遍。马老太太千恩万谢,感激非常。陆韵仙见马老太太同着少太太一同来了,自己心上不安,便告辞要走。秋谷也不拦他,叫王安阁给他二百块钱,另外付二十块钱给那娘姨,陆韵仙便同着娘姨告辞走了。马老太太和马少夫人见了陆韵仙妖妖娆娆的样儿,又知道马山甫的病是给他气出来的,不觉心上十分恨他;马少太太更是眼中出火,恨不得揪他过来打他一顿。幸而秋谷预先和马老太太说过不要难为他,不好将他怎样,只直着眼睛一直瞪着陆韵仙出去。

章秋谷倒为着这件事情忙了好几天。光阴迅速,不觉又是一月有余。这一天秋谷在书局里头完了公事,没有什么事情,便同着辛修甫走到龙蟾珠院中去打茶围。

坐了一回,龙蟾珠要留他们吃饭。辛修甫忽然想起,对秋谷道:“葛怀民昨日在湖北回来,你可知道么?”秋谷摇一摇头道:“不知道,他没有到我那边去。”修甫道:“我也是小屏和我说的,不如今天和他接个风,就在这里吃一台酒何如?”秋谷听了点头道好。辛修甫写了几张请客票,叫相帮分头送去。

一会儿,葛怀民第一个先到,三人相见叙了些多时阔别的友情,又谈了些湖北地方的风景。早见王小屏、刘仰正、陈海秋等都陆续到来。辛修甫叫摆台面,大家入席,一面吃酒,一面高谈阔论起来。秋谷和他们议论了半天,不知不觉的,又讲起嫖经来。秋谷对他们说道:“‘嫖’的一个字儿,全在要讲资格,就同如今官场里头,吏部截取资俸挨次轮选,外官记算劳绩委署差缺的一般。有了资格的,到处不至吃亏。没有资格的,就是有了钱也不中用。”正是:

星桥横过,苍茫银汉之波;鹊架飞回,惆怅黄姑之恨。

不知后事如何,且看下回分解。





第一百三十七回 讲嫖经名士高谈 打茶围瘟生吃醋





且说章秋谷正讲得高兴,刘仰正便对他说道:“你这个话儿我不敢附和。据你讲,做嫖客全要资格,就是有钱也不中用。难道有了资格的嫖客,就可以白嫖,不用出钱的么?”

秋谷笑道:“你这个话儿又不是这般说法。你只要听我细细的讲,你就明白了。

如今那些堂子里头的倌人,一个个都是精灵古怪的,那里还比得从前?差不多些的客人跑到堂子里头去,要是个漂亮些儿的还好,只要略略的有些土气,或有些不合款式的地方,那般倌人看了心上就不高兴起来,不但是暗中奚落,甚而至于还要当面欺凌。更兼如今的堂子里头另有一般习气,以前的倌人挂着牌子做生意,只要是个肯花钱的客人就是了,那里去管他什么瘟生不瘟生,曲辫子不曲辫子?就是做着了天字第一号的曲辫子客人,也没有什么人去笑他。现在的那班倌人,只要做着了一个土头土脑的客人,大家便要指指点点的笑他,只说他做着了土地码子。就是有钱的人,也不过背地里灌几句米汤,骗他几个钱,面子上那里肯好好的待他!至于那班有资格的嫖客,比起那些曲辫子的客人来,却是大不相同。本来是嫖界的惯家,花丛的老手,堂子里头的那些规矩件件皆知,倌人们的喜怒性情般般都晓,既没有一句惹厌的话儿,又没有一些瘟生的举动。倌人们见了这样客人,非但不敢得罪,而且还要好好的巴结着他。所以如今的嫖客,有了钱又有资格的自然是个天字第一号的客人。就是有资格没有钱的客人,堂子里头也不敢怎生的怠慢。独独的碰着了那班只有银钱、没有资格的客人,骗了他无数的银钱,还不说一句好话。这些情形,是我近年以来在堂子里头细细的考察出来的。你若不信,只要你自家慢慢的细心查察,就知道我的说话不是欺人之谈了。“

秋谷说罢,席上的人大家都点头道是。只有刘仰正听着还觉得有些不信,又对着秋谷道:“你虽然说得甚是有理,我的心上却始终觉得有些疑惑。那班堂子里头的倌人专要喜欢那有资格的嫖客,有什么好处呢?嫖客的有资格没有资格,是惯家不是惯家,又与倌人什么相干?照这样说起来,那班倌人挂了牌子做生意,不是做的钱,难道是做的资格不成?”秋谷笑道:“这个话儿你又说得太过了些。我方才说的没有钱,不是说有了资格的客人就可以一毛不拔,不过用起钱来,有些斟酌,不是那般一曲千金、一笑万金的用法。难道他们做倌人的不要客人的钱,拿着钱出来倒贴不成?”

刘仰正听了,方才点头一笑道:“这还罢了。方才你的话儿说得含含糊糊的,不狠明白,所以我就不懂你的意思了。但是这个里头的事情,我究竟还有些索解不得:那些有了钱没有资格的嫖客,为什么倒要吃亏呢?”秋谷道:“那些嫖客虽然有几个钱,堂子里头的规矩却一毫不懂。该应用钱的地方,他不肯用;不该用钱的时候,他又偏要乱用。用了无数的钱,倌人身上却没有一些儿好处。比不得那些嫖场的老手,用的钱一个一个都是用在面子上的,既闹了自己的声名,倌人又受了他的实惠,明明的只用了一千块钱,给别人看了却好像用了三千、五千的一般。要是你做了倌人,碰着了这样的两个嫖客,两下比较起来,究竟你还是巴结那一个的呢?”

刘仰正听了这一大篇议论,方才顿口无言,心上十分佩服。暗想:秋谷这个人真是精明,会把堂子里头的情形看得这般透澈。想着口中说道:“我们好好的讲话,你无缘无故的又要和我取笑,该应罚你一杯。”便取过酒壶来,斟了满满的一大杯递给秋谷。

秋谷也不推辞,哈哈一笑接过来,一饮而尽。又道:“虽然如此,究竟这个‘嫖’字实在不是什么好事情。即如我们同乡有个姓马的,叫做马山甫,好好的到上海来结算帐目,忽然高兴起来,做了个清和坊一弄的陆韵仙,两个人恩爱非常,恨不得化做一团,合成一块。不知怎样的,平空为了几百块钱的事情,两个人争论起来,闹了一回,气得生了一场大病,病得个九死一生。若不是我章秋谷出来和他帮个忙儿,只怕一条性命就保不住了。为了一个倌人,几乎白白的送掉了自家的性命!

你想这个‘嫖’有什么好处?“

原来马山甫的事情只有辛修甫一个人知道,别人都不晓得这件事儿,如今听了他这般说法,便大家七张八嘴的问他。秋谷到了这个时候,方才把马山甫和陆韵仙的事情细细的和众人说了一遍。大家听了都嗟叹不已,只有王小屏一个人,坐在那里低着个头,默然不语,好像有什么心事的一般。秋谷留心看着觉得诧异,便问道:“小屏兄,你为什么这般模样?你心上有什么委决不下的事情,何妨讲出来给大家听听,或者我章秋谷有可以和你出力的地方,做个现在的古押衙,再世的黄衫客,也未可知。”

王小屏听了,抬起头来看了秋谷一眼,叹了一口气,口中说道:“我没有什么事情。”章秋谷看着他那般模样,双眉紧皱,神彩黯然,知道他一定有什么不得已的事情,便又道:“我们在座的这几个人,都是金石同心、芝兰结契的朋友,朋友身上的事情,就是自己身上的事情。你有什么为难的事,为什么不肯讲出来给我们大家听听?难道我们这班人,够不上你的交情,算不得你的朋友么?”

几句话儿把王小屏说得发起急来,只得说道:“你既是这般说法,我不得不和你们讲个明白。但是这件事情,是无从措手的,我就和你们讲了,你们也不能帮我的忙。”秋谷道:“不用管他能帮忙不能帮忙,你先把这个事情讲给我们听听。”

王小屏方才说道:“我以前做的倌人,是公阳里郑菊香,你们都知道的。今年我又做了个东荟芳的洪素卿,方才叫来的就是他。”说到这里,陈海秋大笑道:“我知道你事情,一定是害了单思病!这样的事儿也值得要放在心上!只要我秋谷兄出个主意就是了,包管一霎时握雨携云,颠鸾倒凤。”

王小屏皱皱眉头,连连的摇手道:“你不要混说,我那里害什么单思病?你们只慢慢的听就是了。我自从做了这个洪素卿以后,不上一礼拜就落了相好。”陈海秋又插嘴道:“如此说来,一定是你要娶他回去,请我们和你做个媒人,可是不是?”

王小屏摇一摇头道:“不是,不是。”秋谷对陈海秋道:“你不要和他打岔,我们听他讲下去。”说罢,大家便不开口。

王小屏又道:“这个洪素卿待我甚是殷勤,应酬也十分圆到。不想一礼拜之前,素卿那里来了个姓焦的客人,听说是什么洋行里头的小老板,我也不知道他究竟是个什么东西。我虽然和他并不相识,他却专门的和我作对。每逢我到素卿那里去的时候,他一定占住了房间,死不肯让,素卿也无可如何。一连这样的两天,我被他呕得气不过,就和素卿说了,叫他叫个双台,立刻就摆。我的意思,原想要赶掉这个混帐东西。不想他听得我叫双台,他就叫个双双台。大家屏来屏去,我吃了一个四双双台,他也吃了一个四双双台,赖在那间房内死也不肯出来。就是这样的一连闹了两天,花了三百多块钱,始终还是屏他不过。方才素卿悄悄的问我,为什么伏伏贴贴的情愿让他?你们和我想想,叫我怎样的回答他呢?”

秋谷听了,哑然笑道:“你这个人也太认真了,这样的事情有什么希奇!要是将就些儿,不用顶真,就让了他也算不得什么大事。何必把这般小事放在心上?”

王小屏道:“你的话儿虽然不错,但是你没有身亲其境,自然是冰凉雪冷,平淡稀松,说起来不值一笑。要是你做了我,设身处地的自己当着这般的境界,就知道我的话儿是不错的了。”

秋谷听了,想了一想,这句话儿却也不差,便道:“据我看来,你们两下争论的都是些无谓的闲气,何必这样顶真?要是倌人和你是要好的,也还罢了。万一个倌人对着你是一团假意,向着别人倒是一片真心,你还要不顾死活的去吃这般冷醋,那就可以不必了。”秋谷说着,辛修甫和刘仰正、葛怀民等也大家道是,都劝着王小屏不要再去发痴。

王小屏那里肯听,只对着他们说道:“方才我已经和你们说过,洪素卿的待我,委实是十分要好。不过这件事情,素卿也叫作无可如何。挂了牌子做生意,走进来的都要应酬,不能赶他出去。我想来想去,实在想不出个驱逐他的法儿。请你们几位和我想想,有什么法儿没有?”辛修甫听了道:“这个法儿倒狠不容易,你想大家都是一样的客人,更兼他有的是钱。堂子里头只要有钱就可以进去,有什么法儿禁止他?”

秋谷低着头想了一想,忽然心中触动了一个念头,便向王小屏说道:“有是有一个法儿在这里,这个时候却不便说出来。我们同到洪素卿那边去,待我细细的下一番研究的工夫,或者竟有个禁止他的法儿,也是论不定的。”王小屏听了,问他是什么法儿,秋谷那里肯说,只说:“这个时候不能和你说;和你说了,你要泄漏出来的。”

王小屏听了,只得由他。辛修甫和陈海秋等一班人,也问他究竟想的是什么法儿,秋谷只微微的笑,一言不发。王小屏便道:“既然你一定不肯说明,我们也勉强不来,如今我就请你们翻台过去,到洪素卿那边去吃酒何如?”大家点了一点头,胡乱叫了干稀饭来,随便吃了些,主客一齐起身,径到东荟芳来。

到了洪素卿院中,果然那姓焦的早在那里占住了素卿的正房,王小屏只得同着众人到对面房间坐下。正是:

青楼薄幸,荒唐得宝之歌;云雨迢遥,懊恼迷香之洞。

不知以后如何,且听下文交代。





第一百三十八回 洪素卿昧良施巧计 章秋谷谈笑破奸谋





且说王小屏同着众人在对面房间坐下,洪素卿满面添花的走出来,叫了一声“王大少”,又一一问了众人的名姓,应酬得甚是周到。应酬了一回,便拉着王小屏的手到榻上坐下,把眉头一皱,低低的向王小屏说道:“耐啥洛勿早点来呀!刚刚格个断命客人跑得来勿多歇,赶咦赶俚勿脱,真正拿俚无那哼,格末叫讨气得来!”

王小屏听了,心上自是不快,便道:“我今天一定要在你正房间里头请客,你去和他讲一声儿,他要是个知事的,赶紧给我滚出去!”洪素卿听了,点头答应。

秋谷便问道:“这个姓焦的究竟是做什么事情的,他和你讲过没有?”洪素卿道:“俚自家说起来是海外得来,啥格荣德洋行、协顺祥银号、宝昌钱庄,才是俚笃一干仔开格。”秋谷听了,微微一笑,也不开口。只见洪素卿立起身来,对着王小屏说道:“倪过去搭俚说一声。”说着便慢慢的走过去。

秋谷见素卿过去,便留神细听,要听那姓焦的怎样的一个说法。只听得素卿走过去,竟朗朗的高声说道:“焦大少,对勿住,格间房间有客人来请客,谢谢耐,阿好请耐到亭子间里去坐歇?”秋谷听了,心上猛然一动,连忙提着耳朵再听下去,早听得那姓焦的大声说道:“你倒说得好轻松的话儿!别人要请客,难道我不要请客的么?老实说,这个房间,姓焦的占定了!别人在你这里吃酒,那怕他吃一百台、五十台,我姓焦的一定奉陪。只要他占得住这个房间,就算他是好的。”

王小屏在对面房间里头,听了心上十分生气,却又发作不出来,只对着秋谷说道:“你们听听,可有什么法儿?”辛修甫和陈海秋等听了那姓焦的说话这般放肆,大家也觉得有些愤愤不平。只有章秋谷只对着他们摇手,叫他们大家不要开口。看一看房间里头,只有一个大姐坐在那里。秋谷“霍”的立起身来,向着床后便走。

大家看了,只说他要小便,到床后去找便壶,便也不去管他。

那里知道,秋谷从房后的小门里面一溜烟溜出来,转到前面,一直走到正房门外,放轻了脚步,悄悄的在门帘缝里偷窥。只见一个油头滑脑的少年正把洪素卿拥在身上,两个人密密切切的在那里贴着耳朵讲话,咕咕唧唧的一个字都听不出来。

只见洪素卿点一点头,满面笑容的对着对面房间,把手做一个手势,那少年也点一点头,洪素卿立起身来。秋谷连忙轻轻的蹑步回去,故意到大床后面去转了一转,方才走出来。

辛修甫问道:“怎么你一个小便去了这许多时候?”秋谷不语,只对他摇头。

辛修甫不知道什么意思,正要问时,早听见弓鞋声响,洪素卿缓步进来,对着王小屏摇一摇头道:“格个断命客人,格末叫讨气,叫倪那哼弄法?”王小屏断了,怒气填胸,一时却又说不出什么来。

正在这个当儿,忽然章秋谷立起身来对王小屏道:“你们请在这里略坐一回,我有些小事去去就来。”说着,便急急的走了。辛修甫看了这般光景,料想今天房间是占不成的了,便向众人使一个眼色,大家立起身来。辛修甫对王小屏说道:“堂子里头本来是逢场作戏的地方。今天没有房间,还有明天,明天没有房间,还有后天,何必这样认真,平空的和人斗气?据我看起来,不如暂时去了,明天再来何如?”辛修甫的话还没有说完,王小屏跳起身来拦住众人,口中说道:“房间不房间不要管他,难道别人可以在这个地方请客,我就不好在这个地方请客的么?你们诸位又没有什么紧要的事情,既然来了,何必这般匆促。”众人听了,大家都只得重复坐下。王小屏一面叫洪素卿招呼摆台面,一面和众人代写局票。辛修甫道:“秋谷还没有来,你们可要等他一等?”

正说着,只听得对面房间里头的客人高声大叫起来,拍着桌子道:“你们的人都到那里去了?怎么我一个人坐了半天,连人影儿也不见一个!”洪素卿听了皱着眉头,连忙移步进去,对他嗔道:“啥格嘤喤吵得来,拨别人家听见仔阿要好听?”

那姓焦的大声说道:“我叫你过来,没有别的事情,赶快和我照式照样的叫一个双台下去,立时立刻给我摆上来!”

王小屏和辛修甫等听了,大家都是面面相看,想不出一个主意。停了一会,猛然听得楼下相帮一声高叫:“客人上来!”就这一声里,早听得脚步声响,章秋谷满面笑容飞奔上来。辛修甫问道:“你一个人跑到什么地方去的?”秋谷只是笑,也不开口,走进房来就对着众人摇手,叫不要混闹。众人不知道什么道理,便大家都不开口,眼睁睁的十余只眼睛,都看着章秋谷,要看他做些什么。

只见他不慌不忙,慢慢的走到王小屏身旁,低低的问道:“你身上带钱没有?”

王小屏听了甚是诧异,便对他说道:“今天我身上有些钞票,却也不多,止有一百多块钱。你平空问他做什么?”秋谷低低的说道:“你不要多讲,你们大家不要开口,只听着我的调度。我要怎么样,你们就依着我怎么样,等会儿包你有个法儿把那个混帐东西赶他出去。”王小屏听了半疑半信的,心上狠有些儿疑惑。秋谷又走过去,问着辛修甫和陈海秋、刘仰正、葛怀民等,问他们有钱没有。也有带着钱的,也有不带的,几个人合起来,也有二百多块钱。秋谷又叫他们:“把带的钱一古脑儿都拿出来,等回儿再还你们。”

众人听了,心上大家都诧异起来,辛修甫先问道:“究竟你为着什么事情,何妨说给我们听听。”章秋谷道:“你们不要慌,等一回儿自然明白。”辛修甫道:“怎么这样糊里糊涂的。”秋谷不等他说下去,连忙摇手道:“你们不要开口,我得了一个极好的主意,要替小屏出出气儿,你们等会儿看就是了。大家不用开口,看我一个人发挥。如今你们把钱赶紧拿出来交给我,赶着这个当儿,不要给素卿瞧见。”众人听了不懂他是什么意思,只得大家把带的几张钞票都拿出来交给秋谷。

秋谷接在手内,又在自己身上掏出几张钞票,并在一起,一起交给王小屏,口中说道:“你好好的收起来,等回儿自有用处。”王小屏摸头不着,连忙问道:“我又没有问你借钱,交给我做什么?”秋谷皱着眉头道:“你不用多说,只依着我的调度。少停一刻,包管和你把那姓焦的驱逐出去,叫你大大的出口气儿。”王小屏听了疑疑惑惑的,也不知章秋谷是什么意思。辛修甫道:“秋谷的为人样样都好,就是有一件性情不好,专喜欢叫人打他的闷葫芦。”

一言未毕,只见洪素卿姗姗而来,走进房门,对着王小屏把金莲在地上一顿,咬着牙齿,把手指着对面低低的骂道:“格个杀千刀末,直头是格强盗坯!定规呒拨好死法格!”王小屏听了洪素卿骂那姓焦的,心上自然高兴。却为着屏房间屏不过他,究竟心中不乐。还没有开口,早听得秋谷大声说道:“难道我们吃酒就在这个地方么?”洪素卿听了,连忙抢过步来,对着秋谷道:“章大少,勿要动气,格个断命客人,煞死格坐来浪仔勿肯走!王大少吃双台,俚也要吃双台,真正叫拿俚呒那哼!──”

秋谷不等说完,朗然说道:“我今天倒要学着他们那班曲辫子,发一个痴,一定要赏鉴赏鉴你的卧室,今天就吃个双双台!”洪素卿还没有答应出来,早听得对面房间里头那个姓焦的,也在那里高声说道:“我也吃个双双台!”秋谷听了,微微一笑道:“狠好,他要和我斗气,那是他的造化来了。既然如此,我就吃个四双双台!”那姓焦的也是大声应道:“什么造化不造化,堂子里头吃酒,只要有钱的,那一个不是大爷?我也吃四双双台!”秋谷哈哈大笑道:“好得狠,好得狠。我再加一倍,三十二台!”那姓焦的也应道:“我也三十二台!老实和你说罢,不要说三十二台,就是三百二十台,我姓焦的也要陪你一下!”秋谷又哈哈的笑道:“三十二台酒,差不多要四百多块钱,不是顽的,可是真的么?”那姓焦的高声答道:“不是真的,倒是假的不成?几百块钱的事情,算什么大事!”

这个时候,刘仰正和葛怀民等忽然见秋谷这般举动,十分诧异。就是王小屏自己心上,也觉得有些不以为然。想着花几个钱争得回面子,也还不要说他。花了无数的钱争不到一丝一毫的面子,觉得大可不必。刘仰正便走过去拉了秋谷一把道:“你平日之间讲起那班吃醋屏房间的客人,笑他们是个痴子,怎么你今天自己也做起痴子来?况且这个地方又不是你的相好,你也不便这个样儿。”秋谷听了,回头对他笑道:“我自有我的布置,这会儿不用你们多管。”

辛修甫在旁看了秋谷这般举动,心上已经有了几分明白,便走过来拉着刘仰正道:“他有他的道理,我们不必管他,只看他怎样的一个布置就是了。”刘仰正听了便不开口,大家静悄悄的站在那里。

只听得秋谷口中说道:“我吃三十二台,你也吃三十二台么?不要等回儿反悔起来。”那姓焦的冷笑一声道:“那一个反悔的是个畜生!”秋谷大笑道:“好好,反悔的是个畜生!”一面笑着,一面大踏步走出房门,三脚两步的竟向着对面直闯进去。

王小屏和辛修甫等见秋谷闯进对面房间去,不知道他葫芦里头究竟是卖的什么药,不由得大家面面相看,做声不得。素卿出其不意,吃了一惊。连忙赶出房来,要想拉他回来,口中叫道:“章大少,勿要进去嗫!倪堂子里向呒拨实梗规矩格呀!”

说时迟,那时快,秋谷早已闯了进去,那里叫得回来?这一来,有分教:

识破黔驴之技,名妓惊心;幸逃子路之拳,滑头丧胆。

不知以后如何,且看下文分解。





第一百三十九回 闯房间痛骂滑头 驱恩客难为名妓





且说章秋谷大踏步跨进对面房间,那姓焦的正在那里摇头摆脑的自鸣得意。猛然见章秋谷闯了进来,也不觉吃了一惊,心上狠觉得有些忐忑,连忙立起身来,口中说道:“你平空闯我的房间,是何道理?难道堂子里头没有规矩的么?”

正说着,洪素卿已经赶了过来。那姓焦的见了洪素卿,便大声说道:“你们堂子里头究竟有规矩没有规矩?怎么好好的平空有人闯起房间来!”洪素卿不及回答,连忙走过去拉着秋谷的衣服陪笑道:“章大少,谢谢耐,请到格面去坐。堂子里向格规矩,章大少阿有啥勿晓得?”

秋谷笑道:“你不要这般害怕,我只要和你们这位焦大少说一句话儿,有什么事情都是我一个人承当,与你不相干,你只顾放心。”说着,便对着那姓焦的把手一拱,含笑道:“我闯了你老哥的房间,是我一时卤莽,你老哥不要见怪。如今有一句话儿要请问你老哥。”

那姓焦的见秋谷无故闯他的房间,心上自是十分不快。但是从来有一句俗话,叫作“楚霸王的尊拳,不敌张子房的笑脸”。那姓焦的心上虽然焦躁,看着秋谷笑容满面的好好和他讲话,便也发作不出来,只得答道:“你要问什么话儿,只顾说。”

秋谷又笑道:“论起理来,这件事情与我毫无干涉,我也不必来管这般闲帐。但是今天既然同着我们敝友跑到这个地方,你们两位又彼此斗起气来,我们做旁人的免不得也要出来说句话儿。请问老哥,今天当真的要和敝友斗气,吃三十二台酒么?”

那姓焦的大笑道:“这个话儿方才已经讲过的了,那一个反悔的便是个不要脸的畜生!如今何必又来提起?”秋谷又道:“既然如此,你老哥吃酒的钱怎么样?还是现付的呢,还是赊帐的呢?”

那姓焦的听了,面上不由就呆了一呆,停了一停忽然哈哈的笑道:“你又不是堂子里头的管帐先生,用不着你来多管。”秋谷道:“不是这般说法。你们两位既然彼此斗气,大家争的就是这一点儿面子。要是一时间混闹一阵,闹得一塌糊涂,到了后来拿不出现钱来,这个面子争他做什么呢?方才听你老哥的口气说起来,不要说是三十二台,就是三百二十台你老哥也要陪我们一下。既然为着大家斗气,你老哥总带着现钱来的。我说句放肆的话儿,请你老哥把身上带的钱拿出来给大家看一下,一则显了你老哥自己的声名,二则也好叫我们敝友心服。我们敝友今天跑到这个地方来,只带了四百多块钱,合计起来差不多刚刚三十二台酒钱。如今我也叫他把带的钱拿出来给你老哥看一下子。”说着,便回过头来叫着王小屏道:“你把身上带的钞票拿出来,给大家看一下子。”王小屏听了,果然在身上掏出一卷钞票来,走过去递给秋谷。秋谷点了一点,把钞票放在桌子上,对着那姓焦的说道:“请你老哥看看,一古脑儿四百五十五块钱。你老哥身上的钱在那里?也请拿出来,我们大家瞻仰瞻仰。”

那姓焦的到了这个时候,脸上的神色未免就有些不对起来,只得勉强支吾道:“我带钱不带钱,与你什么相干?我就是有钱,你也没有一定叫我拿出来的权利。

我不给你看,你又有什么法儿?“秋谷冷笑道:”我们自然没有一定叫你拿出钱来的权利,但是今天的事情不比别的,原是你们两家赌气,大家闹个阔大爷的牌子,那有不带现钱的道理?不是我在这里讲一句不中听的话儿,今天拿不出钱来的,就是那天字第一号的滑头码子!你老哥可不要见怪。“

那姓焦的听得章秋谷的口气越逼越紧了,一时间腾挪不得,脸上竟红起来。停了一停,只得又道:“上海滩上的客人,要是在堂子里头吃酒都要付起现钱来,那就连路都不用走了。况且我在他们这里欠帐,自然和他们有欠帐的交情,只要他们自己放心就是了,要你来着急做什么?”秋谷大声道:“欠帐不欠帐,交情不交情,我都不管。总而言之,今天这件事情,有钱的就是上风,没有钱的就是饭桶!你当了个嫖客,连这几个钱都拿不出来,还混闹你的什么架子!我劝你还是早早的跑到别处去罢,省得当场出丑,面上无光!”

那姓焦的听了秋谷这番说话,面子上一时过不去,大怒道:“你究竟是个什么人?我认都不认得你,你平空闯了我的房间,还要在这里满嘴里混说!我那有这样的闲工夫来和你斗口,快快的给我请出去!”秋谷淡淡的笑道:“我闯了你的房间是我的不是,等会儿自然向你服礼。如今只要请你把身边的钱取出来给我们大家看看,一则坍了我们的台,二则装了你自家的幌子。到了那个时候,我们情愿自认下风,尘土不沾,拍腿就走。难道这样光天化日的世界,你拿出钱来,我们会抢了你的不成?”

那姓焦的听了一言不发,只把一双眼睛不住的望着洪素卿。洪素卿把那一点朱唇略略的动了一动,一双俊眼微微的斜了一斜。那姓焦的得了这个暗号,立时立刻的胆大起来,对着章秋谷冷笑道:“你认着我姓焦的真个的拿不出钱么?老实和你们讲,这个时候身上却没有带来。既然你们一定要看,我就立刻回去取来给你们大家看看!”说着立起身来往外就走。

章秋谷抢上一步,把两手一拦,口中喝一声道:“且慢!”那姓焦的见了这般光景,只得立定了脚道:“你这个人不要是犯了疯病罢?我要回去拿钱,为什么你来拦阻?平空的和我这样歪缠,我今天也不知是那里的晦气!”秋谷正颜厉色的说道:“你们既是大家赌气,那有身上一个钱都没有的道理?分明是你们大家通同作弊,有心硬捉姓王的,把他当个瘟生!这样鬼鬼祟祟的事情,我章秋谷眼睛里头也不知见了多少,你哄骗别人也还罢了,竟想平空的骗起我来!你们未曾举意,也该应打听一下我章秋谷是个何等样人,那里会上你们的圈套?就算据你自家口中的说话立时立刻的回去拿钱,安知不是你们彼此商量妥贴,暂时拿出钱来糊里糊涂的搪塞一下?就算搪塞过去,又叫我们到什么地方去查呢?”

章秋谷说到这里,那姓焦的不觉形容大变,一时说不出什么来。洪素卿见了这般模样,心上十分着急,只得赶着说道:“焦大少来浪倪搭,倒一径规规矩矩格──”秋谷不容他说下去,就截住他的话道:“算了,算了,不用说了。劝你少讲几句罢。我是留着你的面子,不肯和你为难。你们的事情那一件是瞒得过我的?到了这个时候,你就是再要帮他说话,也是不中用的了。”洪素卿听了,满面上涨得通红,低下头去,不敢开口。

那位焦大少爷见了,心上也不觉拍拍的跳。但是事情到了这个地步,不得不大着胆子硬挺一下。便按定心神大声说道:“别样话儿你混说也还罢了,怎么这般说话你也好混说起来!你说我和他们通同作弊,有心捉他的瘟生,可有什么凭据没有?

我倒要请教请教。“秋谷笑道:”这些事情,凭据不凭据我都不管。如今世界上的事情只要有钱,不论什么事都办得到。有钱的便是嫖客,没有钱的就算滑头。你如今既然拿不出钱,就是个滑头码子。这个地方,不是你可以挨在这里的,请你快些出去罢。“

那姓焦的听了,知道秋谷已经窥破他的底蕴,索性把心横了一横,口中嚷道:“如今上海地方连王法都没有的了!我有钱没有钱与你什么相干?你又不是开堂子的老板,为什么要你这般着急?”秋谷冷笑道:“堂子里的老板也罢,倌人也罢,总而言之,长话短说,今天你拿不出钱来,就请你快些出去!”

那姓焦的索性立起身来,把桌子一拍道:“你闯我的房间,我还没有赶你,你倒要赶起我来,真是笑话!”秋谷道:“你满口牛皮,虽然说得十分相像,无奈你那几处的钱庄、银号,都没有和他们打个照会,他们都不肯承认你这位东家。如今好好的请你出去,老实说??还是你的便宜!如若不然,你借着钱庄、银号的声名在外面招摇撞骗,哼哼,只怕到了那个时候吃不了兜着走呢!”

那姓焦的听了不觉得毛骨悚然,回答不出。正还想支吾几句,秋谷早抢步过来,一手拉着他的衣服道:“我也不来难为你,劝你好好的走罢。”说着,轻轻的拉着他就走下楼去。那姓焦的本来是个一两几钱的老枪,又是酒色淘虚了的躯壳,那里禁得起秋谷的神力,口内连连的嚷道:“不要拉,不要拉。”一个身体却不因不由的跟着秋谷往外飞跑。

秋谷一直把他拉到大门外面方才放手。只把他拉得喘作一团,上气不接下气的说道:“上海地方是有巡捕的,你怎么这样的动蛮?”秋谷笑道:“我又没有和你动手打架,不过好好的请你出去,什么动蛮不动蛮!你要和我打巡捕官司,我在这里恭候。”那姓焦的又喘嘘嘘的说道:“你是个好的,不要逃走!”秋谷明知道这几句话儿,不过是个自己落场的法儿,笑着应道:“我在这里候你十年。”说着,又对他把手一拱道:“今天多多冒犯,对不起你老哥,我们等回再见。”说罢,便笑吟吟的走了进去。

陈海秋见了秋谷走进来,立起身来把手在秋谷肩上一拍,把一个大指一伸道:“今天这件事情,是你一个人的功劳,我要记你大功一次!”秋谷一笑,回转身来对着洪素卿道:“这个姓焦的是个上海滩上的大滑头,你们不该听他的话儿。得罪几个客人还不必说他,这样的声名传出去,给人家知道了,以后怎样的做生意呢?”

正是:

刘郎前度,桃花人面之思;杜牧扬州,芳草天涯之梦。

要知后事,请听下回。





第一百四十回 感良朋深交铭肺腑 论时艰极目痛山河





且说洪素卿见那位焦大少爷平空竟被章秋谷撵了出去,心上十分不乐,却口中说不出来。没奈何换出一脸的笑容,忍着满心的烦恼,委委曲曲的应酬他们。如今又听得秋谷这几句话儿,明知道这些把戏已经给他看破,只得勉强陪着笑道:“章大少格闲话勿错,格个断命客人,倪上俚格当倒上得勿大勿小。嘴里向枪花掉得蛮好,倪陆里晓得俚是滑头呀!章大少,倪也是一时之错,故歇阿好请耐章大少帮帮倪格忙?”秋谷听了微微一笑,点一点头道:“如今事情已经过去,也不必再去提他,我们吃我们的酒就是了。”

洪素卿听了,眼睛一动,含笑道:“格末谢谢耐。”秋谷回过头来对王小屏道:“今天这个饭桶已经给我赶了出去,什么双双台,四双双台,是用不着的了,还是吃个双台罢。”王小屏听了点头称是。秋谷又对洪素卿说道:“今天他们两家赌气,你一笔狠好的生意生生的给我平空打破,又把你的客人赶了出去。你虽然面子上说不出来,心上不知怎么的在那里恨我呢!”洪素卿陪笑说道:“章大少末总是实梗,倪是做生意,叫呒说法呀。倪堂子里向格苦,耐章大少阿有啥勿晓得格!”

众人听着洪素卿的话儿说得七不搭八的,大家都不懂他是什么意思。只有辛修甫心中会意,在那里暗暗点头。

一会儿摆好台面,大家入席。王小屏便向秋谷殷勤道谢,又问他怎么知道那姓焦的是个滑头。秋谷道:“这个时候不便和你说,你一定要问什么原因,明天细细的说给你听,何如?”陈海秋便道:“明天我作个东道,十二点钟在一品香请你们吃饭,就便听听这件奇事,你们大家有工夫没有?”众人听了都说明天上午没有事情,一定大家奉扰。当下散了席各自回去。

到了明天,果然陈海秋自己坐着马车到书局里头来,邀了秋谷和修甫一同前去。

到了一品香,拣个房间坐下。陈海秋便写了几张请客票,叫侍者分头请客。陈海秋本来性急,便不等客人到齐,先要秋谷把识破他们机关的始末根由说给他听。秋谷道:“提起这件事情的始末根由,话长得狠,一时也说不尽。请你略等一回,等他们大家都到了,再细细的说给你们听,省得我再费一番口舌。”陈海秋听了,只得依着他的话儿耐心等着。

不多一刻,王小屏同着葛怀民、刘仰正都陆续到齐。王小屏开口便问秋谷:“昨天的事情,怎么会无缘无故的就知道他是个空心滑头?”秋谷听了慢慢的笑道:“天下的事情总不出一个理字。只要处处关心,时时留意,没有考察不来的事儿。

你们诸位都是不肯遇事留心,所以就未免见理不明,料事不透。即如昨天的那件事情,我只要把这里头的始末原因一一的细说出来,原不过极平常的节目,你们大家都知道的。并不是什么神出鬼入的算计,通天彻地的机关,你们听我讲就明白了。

我昨天晚上听了小屏的一番说话,说那姓焦的天天同他屏房间,我心上就觉得有些疑惑:就是他们两下吃醋,也不过偶然之间彼此相逢,都倚着一团盛气,不肯让出房间来,不过一次两次的事情。只要占着了个上风也就是了,那有天天如此的道理!

这不是有心和银钱作对么?我心上横着这个疑团,决计要来细细的看他一下究竟是个何等样人?及至到了那里,看着那洪素卿的情形,对着我们是这样的和颜悦色,下气低声。对着那姓焦的讲起话来,却又是那样的高声大气,说话里头,更明明的含着不高兴的意思,全不把他当个客人。你想那姓焦的要果然是个肯花钱的客人,少年清秀,气概豪华,既不是那籧篨戚施的丑鬼,又不是个一钱不舍的财奴,这样的客人和你比较起来,大家都是差不多的资格。更兼你连日和他斗气,都被他占了上风。堂子里头的做生意,本来只认得钱,做着了这样的客人,那有得罪他的道理?

又那有待你这样温存,待他那般冷淡的道理?这个姓焦的又不是个痴子,难道看不出来,听不出来的么?就说姓焦的真个看不出来,听不出来,难道洪素卿又是个痴子么?况且你和他彼此都是一样的客人,就使他们要巴结生意,不肯得罪客人,也该好好的两面应酬,怎么好把一样的客人,一个那样恭维,一个这般得罪?这不是明明的有心偏重,故意叫我们知道的么?他既然有心偏重,自然有个偏重的道理在里头。究竟是个什么缘故呢?这不是明明的姓焦的和洪素卿两个人通同作弊想骗你的钱么?要是换了个寻常些儿的人,他也未必用出这般恶计。偏偏的看着你的样儿也是个多年的老上海,不是容易上钩的。他们两个人想来想去就想出这个法子来:请将不如激将,故意叫那姓焦的和你两下斗气。素卿在你面前又死命的巴结你,巴结得你心上十分畅快;便死命的痛骂那姓焦的客人,骂得你心中甚是燥脾。把你扛在面子上去,叫你落不下台,不得不自家告个奋勇,和他硬挺一场。那姓焦的口中虽然说得十分热闹,背地里却一个大钱都不用拿出来。只苦了你这位王大少爷,铁铮铮的一个一个都要挖出钱来。想不到你这样的一个老上海,竟会上这样的一盘恶当!送了无数银钱,还惹了许多烦恼,也总算是出于意外的了!“

小屏和众人听了,方才一个个恍然大悟。想了一回,觉得那前前后后的情形真是一丝不错。辛修甫便道:“照这样的说起来,你平空的出去一趟,又是到什么地方去的呢?”秋谷道:“那个时候,我虽然看着他的形状已经猜着了八九分,却究竟还有些儿拿把不定,万一个冒冒失失的闹了开去,落不得场,这便怎么样呢?恰恰我听着素卿口中的话儿,什么荣德洋行、协顺祥银号,又是什么宝昌钱庄,刚刚的冤家撞着了对头,我有个朋友是宝昌钱庄的经理。我自己想起来,不如赶到他那里去问他一下,究竟他们股东里头有个姓焦的没有。一口气跑到那里,找着了那个朋友问起他来,非但没有个姓焦的东家,连伙计里头也没有姓焦的。依着我的意思,想要同着那个朋友到素卿那里去,见了姓焦的当面证他一下,无奈他正有要事,不得分身。况且这句话儿又是素卿口中说出来的,算不得什么凭据。这般一想,我便立时立刻的赶了回来。这个混帐东西,也总算是他的流年不利,撞在我章秋谷的手内,平空的出了这样一个大丑,也就是他意想不到的了!”

王小屏听了,立起身来朝着秋谷深深的打上一拱,口中说道:“这件事儿实在仰仗清神,总算和我出了一场闷气。我今天再请一个双台,算个谢仪何如?”秋谷立起来还了一拱,笑道:“我们这几个人都是肝胆相交的朋友,这般小事和朋友帮个忙儿,那算什么!你还和我闹这个么?但是我还有一句话儿要和你说,你那个洪素卿,我看你以后也可以不做了罢。虽然这样的事情算不得什么,这个人的心地也就可想而知的了。就是再做下去,也没有什么味儿。你说我这个话儿可是不是?”

王小屏听了,自然点头称是。

辛修甫想了一回,便又问秋谷道:“据你说起来,洪素卿不该待小屏这样温存,待那姓焦的这般冷淡。你就在这个里头,看出他们的破绽来。但是我仔仔细细的想起来,你究竟不是什么仙人,看不出他们肚子里头的心事,你又安知不是洪素卿把小屏当做恩客,方才做出这般样儿的呢?”

秋谷笑道:“你虽然在上海多年,堂子里头的阅历,究竟不深。你想要是洪素卿果然把小屏当做恩客,又那里肯叫他和别人赌意气,冤冤枉枉的平空花这许多的钱?明明是他们两个人通同作弊,彼此讲明白了,故把小屏抬得高高的,叫他跳不下来,自然不因不由的就要入他的陷阱。这是个一定的道理,那里什么恩客不恩客。”

辛修甫听了,想了一想果然不错,便也微微一笑,不说什么。

陈海秋本来是个最性急的人,嚷着说道:“事情已经过去,你们还在这里议论军机大事一般的议论些什么!不如还是叫几个局来消遣消遣罢。”秋谷道:“你这个人真是个外行。这个时候,那些倌人正在那里陪着客人睡觉,何必一定去惊动他们。况且就是把他们叫了起来,他们还要慢慢的梳头洗脸,抹粉涂脂,不知等到什么时候才来,我们那里等得及?不如劝你免了罢。”陈海秋听了觉得有理,就也依允。

一会儿,侍者端上菜来,秋谷本来酒量不差,便叫开了一瓶克里沙来,和陈海秋两人对酌。辛修甫同着王小屏等都不能吃酒,只略略的吃了些。六个人一面吃酒,一面谈论,讲一回国家的现势,说一回衰弱的原因,论一回列强环伺的艰难,谈一回内政外交的失策。刘越石闻鸡起舞,祖士雅击楫中流。大陆苍茫,风云惨淡。伤心时事,聊为梁父之吟;蒿目河山,尽有唐衢之恸!大家讲了一会,不由得相对凄惶起来。秋谷更觉得别有伤心,无从索解。大家你看着我,我看着你,彼此黯然。

秋谷勉强笑道:“好好的讲话,为什么大家忽然烦恼起来?给别人看见了我们这个样儿,岂不是无病而呻么?”辛修甫也道:“这个缘故,连我自己也讲不出来,平空的忽然觉得心中不乐,不知是个什么道理?”秋谷道:“我们还是喝酒罢!说着,倒了一杯克里沙,一饮而尽。陈海秋也干了一杯。秋谷高吟道:

丈夫及时贵行乐,歌舞任侠人称豪。举杯一歌行路难,酒阑钟歇风萧萧。

吟罢,又一连干了几杯,不觉有了几分醉意。正是:

后庭玉树,犹为亡国之歌;天地蒿莱,独洒狂生之涕!

不知后事如何,且听下回分解。





第一百四十一回 恨天涯深闺挥别泪 折将离南浦送檀郎





且说章秋谷同着辛修甫等在一品香,大家谈论到那时事艰难之处,不觉触起了大家的心事,不由得相对凄然。秋谷更觉得满腹酸辛,无人可诉,一腔抑塞,无泪可挥。吃了几杯闷酒,不由得就有了几分酒意,便辞了大家先走,回到公馆里头。

陈文仙见他闷闷的十分不乐,少不得深深款款的安慰一番。

从来有事即长,无事即短。光阴迅速,不觉又是春末夏初,婪尾花残,酴醿香谢。几声鶗鹓,催残金谷之春;一夜东风,落尽夭桃之色。章秋谷同着辛修甫等一班朋友,花朝月夜,选舞征歌,南陌看花,东门载酒,倒也并不寂寞。就是这样一天一天的过去,不知不觉的一春已过。

转瞬间到了四月,差不多将近端阳,秋谷忽然接着了一封天津来的电报,是一个直隶候补道金云伯金观察打给他的,要请他立刻动身到天津去。秋谷接了这个电报,倒觉得有些踌躇起来。

看官,你道这个金云伯金观察是个何等样人?平空的又为什么打个电报给章秋谷?又为了什么事情要请秋谷到天津去?

原来这位金云伯金观察的父亲,和秋谷的祖老太爷是个联衿兄弟。金观察在十六七岁的时候家计甚是艰难,同着兄弟金霞仲两个人都在章府读书。金观察到了十九岁上,同着兄弟金霞仲捐了个北籍监生,去应顺天乡试。就在这一年上,兄弟同科都中了举人。金观察和兄弟会试了几场不中,便两个人都就了大挑。金观察得了一个知县,金霞仲得了一个教官。

金观察掣出签来,掣了个山东的省分。到省不上两年,就补了沂水县。金观察做了两年沂水县,和山东巡抚张中丞甚是合式。上游器重,僚辈揄扬,几年之间就升了济南府知府。不想这个当儿张中丞一病死了,后任巡抚夏中丞却和金观察不甚合式,借了个盗案,就把金观察参了一下。部议下来,降了一个同知。这一来,只把个金观察气了一个发昏,便赌气不肯做官,告假回去。

刚刚那个时候,直隶津海关道陈宣甫陈观察,和金观察有些世谊,便把金观察请到天津去,在道署里头当个总文案。这位金观察本来丰采过人,衫裳倜傥,办起笔墨上的公事来又是个惯家,那一枝笔来得十分熟溜。陈观察倒也十分敬重。在陈观察那里当了几年总文案,金观察又托陈观察把他荐到直隶总督章中堂幕府里头,也是当个文案。章中堂见了金观察丰神凝重,气慨安详,知道这个人将来必成大器,便也十分器重起来。

金观察趁着这个时候,就在同知上加捐了个候补道,指分直隶,在章中堂手内狠当过几次要差。后来拳匪扰乱,联军进京,章中堂在两广总督任上派了议和全权大臣,便调了金观察一同进京,叫他当个随员。不料事机不巧,恰恰的章中堂一病身亡,金观察止得了一个军机处记名的保举,仍回本省候补。幸而新任直隶总督方安阁方制军和金观察本来是旧友,到任不到三个月,就把金观察委了个洋务局总办,又兼了个营务处。顿时一个金观察就声名大振起来。

金观察自从到了洋务局以后,觉得办起交涉来十分棘手。更兼这个当儿已是那班外国人刚刚交还天津的时候,不得不略略迁就他些。金观察虽然是个通才,也不免有些发付不下。洋务局里头虽然有几个会办、提调,却都是些酒囊饭袋,只晓得吃饭拿钱,那里会办什么交涉。偶然有件事情要和他们商量起来,便大家都是你看着我,我看着你,一句话都回答不出:竟没有一个可以商议的。

这位金观察和章秋谷虽然形迹稀疏,却素来知道章秋谷的为人满腹经纶,一腔热血。有时金观察和章秋谷偶尔相逢,大家议论起来,金观察二十四分的佩服,总说秋谷是个奇才。如今忽然之间心上想起这个人来,把手一拍道:“远在天边,近在眼前!何不打个电报去把他立刻请来,将来有了什么紧要的事情,大家也好有个商议。”想着,便立刻发了一个电报,要请秋谷立刻束装。

秋谷接了这个电报,心上委决不下。待要依着他的话儿立刻就去罢,上海书局里头的事情又没有个可以替代的人。待要回绝了不去罢,觉得自己和金观察是三代至亲,金观察和自己又是十分要好,若竟是毅然决然的不去,未免觉得有些不好意思,却不过他的情面。

当下秋谷拿着电报以心问心的沉吟了一回。刚刚辛修甫走来,便把这个电报递给辛修甫道:“你看看这个电报。这样的事情叫我怎么样?”辛修甫接过来看了一看,便问道:“你心上打算去不去?”秋谷皱眉道:“我一时自己也没有主意,不知你的意思怎么样?”修甫道:“你的去不去,我虽然不能和你做主。但是这个书局里头的事情忙碌得狠,你一时走了,叫那一个人和你代庖?”

秋谷听了低头一想,觉得果然不错,自己的事情别人是代劳不来的,便道:“如此说来,只好不去的了。明天打个电报去回他就是了。”辛修甫听了大喜,连忙点头道是,立逼着秋谷起了个电报的稿子,只说自己上海有事,不得分身。

谁知这个电报发去之后,一连又接了金观察的两封电报,再三劝驾,一定要请他去,那电报里头说得十分恳切。秋谷连接两封电报,觉得实在却不过面情,只得把这件事情告知了太夫人,请太夫人的主意。太夫人便道:“我们和金观察是三代的至亲,如今既是他一定要来请你,你也不得不去上一趟。这里书局的事情,只要请个人和你代理就是了。”

秋谷听了太夫人的话儿,心上便定了主意,和辛修甫说明白了,请王小屏暂时代理书局里头的事情。虽然勉强些儿,却也还可以将就得过。修甫心上虽然狠不愿意,却也知道秋谷的苦衷。这趟行役,秋谷原是不愿意的,只为着迫于情面,无可如何,便也不说什么。秋谷当下便请了王小屏来,和他说了,要请他暂时代理。王小屏也无可无不可的,点头应允。秋谷把书局里头的事情当着王小屏交代一回,交代得清清楚楚。那辛修甫和王小屏等一班朋友,大家都要设席饯行,一连吃了几天花酒。

恰恰到了四月二十六的那一天,招商局的安平船轮开往天津。秋谷便定了安平船上的一间官舱,未免也要回去把行李收拾收拾。他那位夫人和陈文仙,见秋谷平空的要出起门来,少年夫妇恩爱非常,心上自然狠有些儿不乐,却又不便阻挡他叫他不去,未免有许多牵衣执手的离悰,珍重丁宁的别绪。秋谷平日的胸襟虽是十分阔大,到了这个挥泪临歧的时候,不因不由的也觉得神采黯然,一言不发。没奈何走上楼去,告辞了太夫人。太夫人分付了一番说话,无非是叫他沿途保重的意思。

秋谷也嘱咐了他夫人和陈文仙几句话儿,叫他们小心门户,善事高堂。说罢,头也不回的一直走出门去。

他夫人和文仙两个人,手搀手儿的跟在秋谷的后面,一直送到门首。文仙只得说一声:“你在路上没有人照应,须要自家保重些儿!”文仙口中说着,不觉一股酸气一直透到鼻尖,那说话的声音已经岔了,几乎流下泪来。秋谷听了,回转身来看着他们两个人的脸,要想说几句安慰他们的话儿,觉得心上千头万绪的,不知从那一句说起。定了一定神,方才说道:“你们不必挂念,我此去多则半年,少则三个月,一定要回来的。”文仙听了,忍着泪点一点头。他夫人也对他说了几句一路保重的话儿。秋谷便挥手叫他们进去。他夫人和文仙不肯,立在门外,一直眼睁睁的看着秋谷上了马车,风驰云卷的去了,方才同着进去。

当下章秋谷坐着马车一直到久安里陆丽娟院中。走进房间,辛修甫和王小屏两个人已经坐在那里。原来秋谷为着大家和他饯行,今天也在陆丽娟院中吃个双台,算个留别的意思。陆丽娟听得章秋谷要到天津去,心上自然不愿意,未免也有些长亭惜别,南浦牵衣的情态。秋谷也密密切切的安慰了他一番。陆丽娟总觉得有些恹恹闷闷的,在席上勉强应酬,提不起兴趣来。直至到了秋谷临行的时候,陆丽娟同着辛修甫等一班朋友都送到船上来。辛修甫等略略的坐了一回,便起身走了。只有陆丽娟坐着不走,咕咕哝哝的嘱付了许多话儿,软语缠绵,深情宛转;惆怅檀奴之别,凄凉婪尾之歌。

两个人谈了一回,不知不觉的已是五更鸡唱。秋谷带去的那个家人叫做刘升的,走进来回道:“这个时候,差不多将要开船,送行的人请上岸去罢。”丽娟听了立起身来要走。秋谷同着他一直走上船面甲板,两个人倚着栏杆又说了几句话儿。丽娟走了两步,又回转过身来对秋谷说道:“倪搭耐讲格闲话,耐记好仔,勿要忘记脱。出门格辰光,勿比勒浪屋里向,一塌刮仔格事体,耐自家当心点,勿要实梗马马虎虎,阿晓得?出门人除脱仔自家当心,再有啥人来照应耐呀?耐就是带仔当差格去末,俚也勿肯搭耐当心啘!糟蹋仔自家格身体,啥犯着呀!”秋谷听了丽娟这一番说话,不觉暗暗点头。正是:

一声珍重,魂销南浦之歌;十里长亭,肠断京华之路。

不知后事如何,请看下文交代。





第一百四十二回 出吴淞离怀随逝水 走津沽壮志破长风





且说章秋谷听了陆丽娟那一嘱咐丁宁的说话,觉得深深款款,无限柔情,未免心上也有些儿感动,不由的暗暗点头。陆丽娟一面说着,眉头一皱,那一双俊眼水汪汪的含着一泡珠泪,看着秋谷的脸儿,一步一回头的,依依不舍。秋谷也看着丽娟,两个人脉脉含情。

停了一回,秋谷忽然笑道:“你这个样儿,倒也装得十分相像,果然名下无虚。”

陆丽娟忽然听得秋谷说出这两句话来,真是出于意外。一时间倒呆了一呆,方才皱着眉头道:“阿是倪格闲话才是假格?耐格人阿有良心?说笑话末,也勿是实梗说法格啘!”秋谷笑道:“你就是假的,我心上也狠喜欢,你又何必一定要这般辩白?”

陆丽娟听了,恨得把金莲一顿道:“耐格良心到仔陆里去哉!说出格号闲话来,阿要作孽!”

秋谷听了,一面笑着,一面走过来握着丽娟的手道:“就你算是真的,我的不是,如何?”说着又附着陆丽娟的耳朵,说了几句不知什么的话儿。丽娟不觉微微一笑,故意嗔道:“耐格人末,直头少有出见格。”秋谷笑道:“时候还早得狠,回去再坐一回也不要紧。难道怕我真个把你带上天津去么?”丽娟瞅了秋谷一眼道:“倪勿要,倪要去哉。”秋谷点一点头道:“送君千里,终须一别。况且你一夜没有睡觉,还是早些回去休息休息罢。”丽娟听了眼圈儿一红,低低的说了一声“一路顺风”,便别转头去也不再说,急急的上了跳板。走到岸上,回过头来对着章秋谷打一个手势。秋谷倚着栏杆,也向他挥一挥手。陆丽娟一步懒一步的坐上马车,一径回到久安里去。秋谷直望着陆丽娟的马车去得远了,方才懒懒的回到官舱,没精打采的睡了。

这一睡,直睡到差不多十二点钟方才睡醒。轮船早已开行。秋谷起来洗了个脸,饭也不吃,便一个人走上甲板来。浪静风平,海天如镜;波涛无际,极目苍茫。只有许多海燕跟在轮船后面,前后左右的四围飞舞。远远的望见几点黑影,隐隐的露出帆樯,原来都是那浮海的沙船,在那浪花里面一上一下、一高一低的乱滚。真个是神山一发,白浪千寻,潮来则天地皆青,风起而鲛人欲泣。

秋谷立在船面上举头四望,心旷神怡;更兼一阵阵的海风劈面吹来,拂袖动裾,更觉头目豁然,形神俱适。看了一回,便回到官舱坐了。闷闷的没有事情,便在网篮里面拿出几本小说来,歪在榻上看了一回,不觉又鹓眬睡去。直到刘升来请吃晚饭,方才起来,走到外面广厅,杂着众人坐下。

原来轮船上的规则,官舱客人吃起饭来,是大家聚在一起吃的,肴馔十分精致。

秋谷随便吃些,又走出官舱,到甲板上来闲眺。只见有两个二十上下的少年,都是天津口音,两个人站在一起谈得甚是热闹。秋谷见了,便慢慢的走近他身畔侧耳细听,要听他们在那里谈些什么。

只听得那少年长叹一声道:“我们中国人的事情,都是自己弄坏的。即如招商局初开的时候,搭客的价目原分主、仆两等,当差的只收半价。那知到了后来,就有那班打小算盘的人出来有心弄巧。明明两个人都是一样的搭客,他却贪图便宜,算做一主一仆。甚至同伴四五个人,他却算做一主三仆,或者一主四仆。后来给招商局里头的人知道了,索性删除了这条规例,搭客不论主、仆,一律收取全价。他们那班人到了这个时候,大家都你看着我,我看着你,无可如何。你想我们中国的人,都是这般卑鄙龌龊的性格,那里还有什么顾全公益的胸襟、组织团体的观念?

这样的小事尚且如此,大事可知。我们中国前途的希望,也就可想而知的了!“

那一个少年听了也叹一口气道:“以前李鸿章到美国去的时候,住在一家客店里头。那客店的头等客房一天要一百五十元美金,合起墨西哥银币来,差不多要三百几十块钱。李鸿章嫌他价钱太贵,就住了二等房间,参随人等都是住的三等,一班美国人都讥笑他的慳吝。我们中国头等的人物,倒去住他们美国的二等房间。你想像李鸿章这样的富豪,那般的声望,尚且要这般的贪小利、打算盘,不顾国家的体统,别人更不必说了,你又何必还去责备他们呢!”

秋谷听了他们两个的一番说话,觉得这样的一番议论,不是寻常的人讲得出来的。更兼看着那两个少年的样儿,也都是目秀眉清,气度不俗,便想和他们做个萍水相逢的朋友。不由的对着那两个少年把手一拱道:“方才听着你们两位的高论,果然抱负非常。请教你们两位的贵姓大名,不知你们两位肯赐教不肯赐教?”

那两个少年蓦然见秋谷走近身来和他们讲话,出其不意,不觉倒吃了一惊。及至抬起头来看时,只见站在面前的也是一个二十上下的少年,却生得粉面朱唇,蜂腰猿臂,长眉入鬓,凤目含威,亭亭天表之姿,濯濯灵和之柳。从来名士相怜,倾城互惜。那两个少年见了秋谷这般仪表,不觉都有些自惭形秽起来。那一个年纪大些的少年,连忙拱手含笑,通了姓名。

原来两个都是天津县人,住在天津城内。一个年纪大些的姓姚,叫姚小峰;一个年纪小些的姓傅,叫傅仲骏。是天津县里头两家著名的绅士。却又都是少年好学,声望不凡;腹有经纶,胸多块磊。在天津地方狠有些儿名望。当下傅仲骏和姚小峰也问了章秋谷的姓名,略略的谈了几句,大家都觉得十分合式。秋谷便把他们邀进官舱坐下,彼此高谈阔论起来。从此之后,章秋谷和姚、傅两个成了朋友,芝兰结契,金石论交,一路上谈谈说说,倒也并不寂寞。

不一日轮船早到天津。原来轮船到了大沽口,还要曲曲折折的弯进七十二沽,方才到得紫竹林租界。春夏两季,大沽口内水深,轮船可以直抵紫竹林租界。到了秋冬两季,口内水浅,轮船不能进去,就只好停在大沽口外面。一班搭客都另趁小火轮登岸,狠有些儿不便。刚刚这个时候夏令水深,轮船可以进去。在大沽口外停泊了一夜,到了明天,慢慢的鼓轮进去。走了半日,方才到了码头。

早有金观察接了秋谷的电报,知道他坐的“安平”,便派了一乘四人大轿,四名差弁,两个家人,到码头上来迎接。章秋谷便把刘升留在船上,叫他押着行李慢慢的来。秋谷坐上轿子,一直到东门内卢家胡同金观察公馆里头。

秋谷刚刚出轿,早见金观察呵呵大笑的直走出来,一把拉住了秋谷道:“我算计你该应到了。”秋谷也笑吟吟的抢步上前,执手招呼。两个人手挽手儿的走到厅上。秋谷为着金观察是长亲,对着他不得不行个全礼,便对着金观察屈一屈膝,早被金观察一把拉了起来,大笑道:“我们至亲,还闹这些过节儿么!”秋谷又请了金观察的夫人出来拜见过了。金观察便把秋谷邀到内书房内坐下,谈了一回,早不觉红日沉西,暮烟四合。金观察对着秋谷笑道:“你今天初到,我要和你接风。久仰你是个粉阵花围的老手,今天就请你到一个地方去见识见识,何如?虽然你是在上海顽惯的人,也要叫你看看这里的风景。”秋谷听了自然答应。一会儿,金观察备了两乘轿子,同着秋谷到侯家后宝华班来。

原来天津地方的侯家后,就像上海的四马路一般,无数的窑子,都聚在侯家后一处地方。更兼天津地方的嫖场规则和上海大不相同。上海地方把妓女叫作倌人,天津却把妓女叫作姑娘。上海的妓院叫做堂子,天津却把妓院叫作窑子。窑子里头又分出许多名目,都叫作什么班、什么班,就如那优人唱戏的班子一般。班子里头的姑娘,都是北边人的,就叫作北班。班子里头都是南边人的,就叫作南班。南班和北班比较起来又是大同小异:到北班里头打个茶围,要两块钱;到南班去打茶围,却只消一块钱。那怕你一天去上十趟,打上十个茶围,就要十次茶围的钱,一个都不能短少。南班里头吃酒碰和,都是十六块钱,住夜是六块钱。北班里头的碰和也是十六块钱,吃酒却要二十二块钱,住夜是五两银子。叫局不论南班、北班,都是五块钱。请倌人出局,只要三块钱。若是没有去过的生客,走进窑子里头去,合班的姑娘都要出来见客,凭着客人自己拣择。拣中了那个姑娘,就到他房间里头去打个茶围。万一那个客人眼界甚高,一个都拣不中,尘土不沾,立起身来便走,也不要他花一个大钱。住夜的客人不必定要碰和吃酒,碰和吃酒的客人也不必定要住夜。

住一夜是一夜的钱,住十夜是十夜的钱,狠有些像那上海么二堂子里头的规矩。这些事情,在下做书的既然做到这里,不得不把天津妓院里头的规矩,细细的演说一番,好叫看官们看了在下的这部小说,心上有个头绪,不至于看到紧要的地方茫然不解,漠然不知,就知道在下的这番演说不是赘瘤之谈了。闲话休提。

只说章秋谷同着金观察到了侯家后宝华班内,金观察领着章秋谷走到一个房间里头坐下。秋谷举目看时,见房间里头的陈设也和上海差不多,墙壁上挂着许多的单条字画。正中向外,放着一架红木床,挂着熟罗帐子。两旁也摆着两口红木衣橱。

秋谷看了一回,早见门帘一起,一个十七八岁的淡妆女子走了进来。正是:斋

南都石黛,偏开上苑之花;北地胭脂,重入唐宫之选。

不知以后如何,请听下回分解。





第一百四十三回 金观察夜走宝华班 章秋谷重到侯家后





却说金观察同着章秋谷到侯家后宝华班,走进一间房内坐下。不多一刻,早见一个十七八岁的淡妆女子款款走了进来,轻启朱唇,对着金观察,叫了一声“金大人”。回转头来,向着秋谷一笑,口中问道:“格位老爷贵姓?”金观察便对他说道:“这位老爷姓章,今天从上海到的。”又指着那女子的脸,对秋谷道:“这个就是我招呼的,名叫金兰,你看怎么样?”原来北边班子里头的规例,客人做了姑娘,就说某老爷招呼某姑娘,大家都是这般说法,没有什么做与不做的,和上海的名目不同。

只说章秋谷听了金观察的话,便抬起头来细细的把金兰打量一番:只见他身上穿着一身白罗衣裤,下面衬着一双湖色挑绣弓鞋。头上挽着一个时新宝髻,刷着一圈二寸多长的刘海发,带一支翡翠押发。那一身妆饰,和上海的样儿也差不多。再往脸上看时,只见他脂粉不施,铅华不御,两道淡淡的蛾眉,一双盈盈的杏眼,虽然没有十分姿态,却也生得轻盈柔媚,尽足动人。说起话来一口的上海白,不像苏州人的口音。

秋谷看了点一点头,对金观察道:“老表伯的眼力着实利害,这个贵相知生得果然不错。”金观察听了,心上甚是得意,拈着几根胡子哈哈的笑道:“你不要作违心之论,有意面谀。你们在上海玩惯的人,那里看得上这般人物?”秋谷也笑道:“那倒不是这般讲法。上海的倌人也不见得个个都是好的,天津的倌人也不见得个个都是坏的。小侄记得几年之前到过天津一次,见过几个倌人,色艺都狠不错,可惜如今都不知那里去了。就是上海那几个有名的红倌人,林黛玉、张书玉、顾兰荪等,也都到天津做过生意。”

正说着,只见金兰一个转身,手内托着两个瓜子碟子,一碟西瓜子,一碟北瓜子,走近身旁来敬秋谷。秋谷随意拈些,金兰便把两个碟子放在桌上。金观察笑道:“你这个东西,怎么只敬章老爷,不来敬我?难道我不是客人么!”金兰听了也笑道:“金大人末总是实梗,咦要来瞎扳差头哉!”金观察听了一笑,也不言语。

停了一停,忽听得房门外一阵脚步的声音一步步走进房来。秋谷举目看时,只见一顺的早进来三个女子,一色的都穿着竹布衫裤。说话的声气,好像是镇江、扬州一带的口音。眉目口鼻都生得不大平正,脸上却搽着许多脂粉。走进房来各叫了一声“金大人”,便都一屁股坐下。秋谷看了一眼,便别过头去不去看他。

金观察忽然向秋谷道:“我倒忘了一件事儿,你初到这里,没有相好,就在这里的倌人里面拣选一个,何如?”秋谷听了,点头应允。金观察便对金兰道:“快叫他们出来见客。”金兰答应一声,走出房去。

只听得房外高叫一声:“见客!”金兰便翻身走了进来。一霎时笑语喧哗,花枝招展,七长八短的,走进十数个女子来。也有大的,也有小的,也有妍的,也有媸的,拥拥挤挤的都挤在一间房内。有的打情骂俏,有的弄眼丢眉,有的“咭咭咯咯”的笑作一团,有的动手动脚的顽做一块:一个个徘徊顾影,卖弄风情。

秋谷细细的一个一个看过来,觉得不是有些俗眼俗眉,便是有些土头土脑,没有什么出类拔萃的在里头。只有一个最后进来的倌人,年纪约有十八九岁,身穿着一件玄色铁线纱夹袄,湖色春纱裤子,一双四寸金莲,着一双宝蓝平金弓鞋,头上止挽一个懒妆髻,没有一些首饰,越衬得明眸皓齿,玉面朱唇,月挂双眉,霞蒸两靥。虽然比不上陈文仙的那般清丽,陆丽娟的那样风华,却也姿态娇娆,丰神姽婳.秋谷看了他一眼,便指着他问金观察道:“这叫什么名字?”金观察拍手笑道:“果然你的眼力不差!他叫云兰,也是从上海新到的,是这个宝华班里头的翘楚,如今却被你选中了。”

秋谷听了便走过去,一把握着云兰的纤手,细细的看了一回。云兰被秋谷看得不好意思起来,瞟了秋谷一眼道:“做啥呀,慢慢里看末哉呀。”秋谷微微一笑,把手一松,云兰对着秋谷飞个眼色,回过身来低低的叫一声“上碟子”。早听得外面答应一声,递进两个瓜子碟子来。云兰接在手内,先敬观察,后敬秋谷,却对着秋谷低鬟一笑。秋谷便拉着他叫他坐下,一长一短的和他讲话。那一班落第的倌人,起先进来的时候看着秋谷这样翩翩年少,跌宕多姿,大家都觉得有些心动,眉迎目送,脉脉含情。如今见他选中了云兰,大家都知道自家没分,又羞又妒,一哄的都走出来。

金观察见他们走了,心中大喜,和金兰坐在一处,密密切切的讲话。讲了一回,金观察便叫金兰预备摆酒,取过请客的纸片,写了几张客票。忽然抬起头来,见秋谷和云兰并肩执手的坐在那里,低低的不知在那里讲些什么,讲得正是热闹。金观察不觉大笑道:“怪道别人都说你喜欢在女人身上用功。今天你们两个人第一次相见,就有这许多说话,果然名不虑传!”云兰听了脸上一红,立起身来道:“耐勿要来浪搭倪瞎三话四,倪规规矩矩讲两声闲话,也无啥希奇啘。”金观察哈哈笑道:“本来没有什么希奇,我不过这样的说一声罢了,你又何必这样的做贼心虚!”

云兰被金观察说了这几句取笑的话儿,面上越发红起来,讪讪的走了开去,口中咕噜道:“随便唔笃去说啥末哉。”

秋谷一笑,立起身来,走近金观察身畔,问他请的是那几个客人。金观察道:“都是几个同乡,并没有什么外客。”说着,早见几个男班子进来摆设桌面。原来北边的男班子,就是南边的相帮。当下金观察便把客票交给他们,叫立刻就去催请客人。

不一会,早见一个三十余岁的男子从外面大踏步走进来。秋谷连忙看时,认得是金观察的亲戚余太守,便立起身来,彼此招呼坐下。金观察道:“今天你居然来得狠早,接到我催请的客票没有?”余太守笑道:“我方才接到你的来信,说请我吃花酒,当陪客。我一听得有人请我吃花酒,我心上高兴极了,连忙办结了今天的公事,急急的就赶过来,那里还等得及你来催请!”说得金观察和章秋谷都笑起来。

停了一会,又到了两个客人。秋谷却不认得,彼此请问名姓,方才知道一位是营务处发审委员、直隶候补同知杨玉甫,一位是制台衙门里头的幕府、兵部主事言立身,都是秋谷的同乡。秋谷也不免应酬了一阵。

这个时候,只见金兰和云兰两个人一前一后姗姗而来。云兰趁着他们大家在那里说话,拉着秋谷的手悄悄的讲道:“耐到倪房间里向去坐歇,倪要搭耐说闲话。”

秋谷跟着他走出房去,穿过一个院落,方才是云兰的房间。云兰把秋谷拉进房间坐下,两个人谈了一回,早有金观察叫人相请。秋谷同着云兰一同走过去,只见又来了三个客人,桌面已经摆好,大家在那里高谈阔论的讲话。

秋谷走进房去,对着那三个新来的客人拱一拱手,问过姓名。金观察便向秋谷道:“你的本堂局票,已经和你发了出去,只怕一个人不够,我再荐一个人给你,好不好?”云兰跟在秋谷后面,连忙悄悄的把秋谷衣服一拉。秋谷会意,便向金观察道:“小侄也不过逢场作戏,叫了一个本堂也就算了。”金观察道:“既如此,客人已经到齐,就请诸位入座。”今天这一台酒,原是金观察专请秋谷的,要请秋谷首座。秋谷再三谦让,大家都不肯就坐,秋谷方才坐了。

金兰斟过了酒,便有几个乌师在门外拉起胡琴,打起锣鼓。金兰慢慢的立起身来走到帘底,把脸向着门外,唱了一段《取成都》。回过身来就坐在金观察后面,把一柄白纸折扇递在金观察手内。金观察便把这柄纸扇递给秋谷,口中说道:“你爱听什么,随意点就是了。”秋谷接过来打开看时,只见上面写着许多戏目,也有二簧,也有西皮,也有梆子。秋谷心上暗想道:古时清歌妙舞,歌舞原是连的,所以教坊中人有舞衫歌扇的名目。如今这个舞学久已失传,这柄纸扇大约就是古时的歌扇了。正是:

樊素樱桃之口,逸响停云;小蛮杨柳之腰,流光回雪。

不知后事如何,应听下文交代。





第一百四十四回 舞衫歌扇清夜无愁 大道青楼良宵载酒





且说章秋谷接过扇子来看了一看,便递给那位言立身言主政让他来点。言主政也不肯点,大家推让了一回,公点了一出《朱砂痣》。金兰唱毕,接着云兰也唱了一出《黄金台》。叫的局已经来了几个。金兰又斟了一巡酒,便向金观察告一个假,走了出去。

看官,你道什么叫做告假?在下做书的在上海烟花队里整整的混了十年,从来没有见过倌人要向客人告假的。原来这个告假,也是北边窑子里头的规矩。客人们叫倌人的局,那倌人直要等到大家散席,方才可以告退。若是遇着有别人叫局,或者有人在他院中吃酒碰和,便在那叫局的客人面前告一个假,到别处去打个转身再来应酬。甚至叫一个局,有连告好几次假的。金观察虽然没有叫局,却照例吃酒的时候有个台面局的,所以金兰照着叫局的规条,向金观察告假。

在下做书的写到此间,就有个老于上海的朋友驳斥在下的说话道:“你这句话儿错了。要是照着你的说话,倌人出来应局,直要等到大家散席方可脱身,遇着有别人叫局,又要向客人告假。万一个天津的倌人也和上海的倌人一般,一天里头出上二三十个局,甚至四五十个局的都有,要是一个一个都要向客人告起假来,那里告得尽许多?那些倌人又怎样的分身得开?难道真个像《西游记》上孙猴子一般,当真有什么分身法不成?”

在下听了笑道:“你的说话虽然有理,却还没有知道这里头的实在情形。天津地方的带局比不得上海,止要一块钱,可以一转眼的工夫立起身来就走。在天津叫一个局,足足的要五块钱,又大半都是现钱,没有什么赊帐的。若要叫一个局,不给现钱,一定要是向来要好的熟客方才办得到。这个里头也有一个道理:倌人应局的规例,不论什么地方,除了叫到戏馆和叫到自家公馆之外,一概都要出一块钱的坐场钱,和苏州的叫局规则一般。不过苏州规矩,只有在堂子里头叫局方才要出坐场的钱,酒馆、大菜馆都没有的。天津的大菜馆和酒馆也是这般。那班倌人出来应一个局,若是客人赊帐,就要自己贴掉一块钱。所以天津倌人每逢有素不相识的人叫他的局,多半是推托不去。就算是勉强去了,也一定要当面向他讨钱。那里像上海的这般模样,出一个局一古脑儿只有一块钱,还要大家赊帐。若是一两个局,就是嫖了也不能算嫖帐。彼此的情形不同。如此自然天津倌人的局少,上海倌人的局多了。上海的红倌人,一夜工夫竟有出五六十个局的。天津的倌人,就是天字第一号头等名角,一夜工夫至多也不过出上六七八个局。你没有到过天津,不懂那边窑子的情形,只拿着上海堂子里头的情形来两边印证,自然觉得大大的不合了。”那位老上海听了在下这一番滔滔滚滚的说话,方才俯首无言,走过一边去了。

闲话休提。只说云兰见金兰告假走了,也向秋谷告一个假走了出去,便有几个本班的倌人走进房来应酬台面。应酬了一回,这几个走了出去,又换了几个进来。

原来天津那些班子里头的姑娘好像上海么二堂子的倌人一般,不是捆帐伙计,就是分帐伙计,再不然就是老鸨的讨人,从没有一个人是自己身体的。那班子里头也没有什么包房间做伙计的名目,合班的倌人不论红的黑的、大的小的,都要听老鸨的节制号令。就是那个时候的林黛玉、张书玉到天津做生意,也是包帐伙计,算不得自己身体。那第一天进门的时候,一般的也要向着老鸨叩头。所以天津窑子的倌人,大家都是混在一起的,你的客人,我也可以应酬;我的客人,你也可以陪待,分不出什么界限。

当下章秋谷看着那班倌人你来我去,你出我入的,好似穿花蛱蝶一般,倒也甚是热闹。秋谷看了一回,忽然又见几个倌人嘻嘻哈哈、拉拉扯扯的,口中说着满口的扬州白直闯进来,三个人坐在一起,夹七夹八的和客人说笑。

秋谷见就是方才进来那三位宝货,便连忙把头别过去,不去看他,心上觉得十分惹厌。更兼听他们你言我语的,打着满口的江北乡谈,却口口声声的讲我们苏州怎么样、我们苏州那么样。秋谷听得清楚,心上又好气又好笑,忍不住问道:“你们几个人都是苏州人么?”那三位宝贝听了,大家觉得甚是得意,齐齐的答应一声。

秋谷笑道:“我看起来,你们这几个苏州人着实有些西贝。”那三个人听了,不懂秋谷的话是什么意思,便道:“什么叫做西贝?我们不懂。”秋谷道:“你们既是苏州人,怎么连这句话儿的意思都不懂?你们姑且讲几句苏州话来给我们大家听听,看你们究竟是苏州人不是?”

原来我们中国全国,苏、杭两处是个繁华富丽的地方。苏、杭两处的女子,就也是个姽婳娇娆的尤物。这几个宝贝平日之间总假充是苏州人。好在那些客人,本来辨不出他们的口音什么叫做扬州话,什么叫做苏州话,当真都把他们几个认做真的苏州人。这三位宝贝假冒苏州人冒得久了,忘其所以,自己也有些不信自己起来,好像自己真是苏州人的一般。不料今日之下忽然冤家遇了对头,平空的跑出一个章秋谷,要考起他们的苏州话来。这几个宝贝那里说得出什么苏州话?被章秋谷逼住了,无可如何,只得胡乱说了几句扬州不像扬州、镇江不像镇江的话,就算是苏州话,只指望章秋谷也不懂苏州话,糊里糊涂的搪塞过去也就算了。

那里知道章秋谷听了他们的这几句话儿,不觉哈哈大笑道:“这个就算你们的苏州话么?好得狠,好得狠。这才是有一无二的苏州白呢!我听着你们三个的口音,明明是个扬州人,为什么一定要假充苏州人?难道假充了苏州人有什么??处吗?”

这几句话儿,把那三位宝贝说得做声不得,脸上都涨得通红,只得勉强说道:“扬州人也是个人,苏州人也是个人,难道苏州人还比扬州人多个眼睛、鼻子么?”秋谷微笑道:“你们既然知道扬州人也是人,苏州人也是人,为什么自己又要假充苏州人?这是个什么道理?”那三个宝贝被秋谷顶住了,腾挪不得,一句话都说不出,赌气大家立起身来往外便走,口内咕咕哝哝的不知说些什么。秋谷也不去理他。金观察见了,便对着秋谷笑道:“他们好好的坐在这里,被你几句话儿把他们逼得跑了出去,他们心上不知要怎样的恨你呢!”秋谷笑道:“这样的牛鬼蛇神,但愿他心中怀恨,绝迹不来,倒干净了许多。”

正说着,云兰已经走了进来。秋谷对着云兰皱一皱眉头,又把手打个手势,似乎把方才的事情告诉他。云兰会意,微微的一笑,也皱着眉头低低的说道:“耐勿要实梗嗫。大家才是姊妹淘里向,讲起来阿要难为情?”秋谷也不开口,只伸过手去紧紧的握住了云兰的纤腕叫他坐下,两个人四目相对,彼此默然。

正在这个时候,客人叫的局陆续陆续的到齐,大家拉开嗓子唱起来。秋谷候他们唱过之后,一个个从头至脚打量一番。只见也有北班里头的,也有南班里头的。

北边人和南边人的装束,也没有什么大分别。北边人多半是紥着裤腿,那眉梢眼角都是吊得高高的,全没有一些儿温柔枭娜的丰神。秋谷看着心中想道:“究竟这班人生长北方,总觉得有些儿体态刚强、丰姿生硬,那里比得上我们江苏人的样儿!

究竟北地胭脂,不及南朝金粉,这是一定的道理。“正想着,恰恰的言主政要打通关,先和金观察五魁对手的乱叫起来,方才打断了章秋谷的思想。

大家闹了一回,一班客人都散席告辞。金观察掏出表来看了一看,对秋谷道:“今天时候还狠早,我们出去打几个茶围再回去,可好不好?”秋谷听了自然高兴,便点头答应,立起身来想走。云兰一把拉住,口中低低的问道:“倪刚刚搭耐说格闲话,阿是忘记脱哉?”秋谷摇一摇头道:“今天不便,改一天再讲罢。”

云兰听了默然不语。秋谷附着云兰的耳朵说了几句,不知说的什么。云兰回眸一笑,启齿嫣然,一面说道:“间搭勿比上海,耐勿吃酒也呒啥希奇。”秋谷道:“虽然没有什么,我总觉得有些不安,同你绷个场面,就同绷我的场面一般。”云兰听了,把嘴披了一披,也不开口。秋谷便同着金观察起身就走。金兰和云兰送出房门,云兰又叮嘱一句道:“勿要忘记脱仔哩。”秋谷笑道:“不劳分付,我的心上更要比你性急些儿。”云兰脸上忽然一红,把头一扭道:“好哉,好哉。阿好请耐格两声勿要响。”

金观察听了他们两个人的话,心上早已明白,也对云兰笑道:“你们两个人不用猜哑谜,有什么话儿何必瞒我!等我来和你们做个媒人,可好不好?总算你的眼力不差,看中了这位章老爷。你也不必遮遮掩掩的,只管说明白了就是了。”几句话把个云兰说得不好意思起来,不由得红上眉梢,春生颊际,对着金观察道:“耐说仔几几化化格闲话,倪一塌刮仔才勿懂。耐勿要来浪搭倪瞎三话四!”说着,便拉着金兰一同进去。

金观察同着章秋谷走出宝华班大门,走不多几步,便是一个北班,叫做东天保的,本来是个著名的班子,房屋十分宽大。秋谷和金观察走了进去,在一间客座里头坐下,便有许多的本地倌人挨挨挤挤的走出来。秋谷约略看了一看,却没有一个好的在里头。正是:

春风二月,忽逢解语之花;大道青楼,又绾同心之结。

以下的许多情节:安垲第大开赛珍会,章秋谷再到沪江,试真情红倌人中计,都在第十集里头出现。列位看官不须性急,听我慢慢的道来。





第一百四十五回 走章台良宵开夜宴 入花丛蓦地遇无盐





上集书中说到章秋谷到了天津,金观察同他到侯家后去,在宝华班金兰那里和他摆酒接风。席散之后,金观察又同着秋谷到东天保去打茶围。刚刚坐下,早见七长八短的拥出十余个倌人来。秋谷约略看了一回,只见不论妍媸、大小,都紥着一双裤腿,缠着一双金莲。那一双金莲虽然一个个都缠得不盈四寸,却都是趾圆背厚,臃肿非常,那里像什么两瓣香莲,那里像什么一钩新月!比起那驿路旁边的马足、磨坊里面的驴蹄来,倒觉得有些相像。

看官请想,好好一对增娇助媚的三寸金莲,像了那最龌龊、最不雅观的驴蹄、马足,可想而知,还有什么好看!更兼北边女人的习惯,走起路来都挺着胸脯仰着个脸,雄赳赳、气昂昂的,全没有一些儿袅娜温柔,只觉得满面上都带着一团怒气。

秋谷见了皱着眉头,向金观察打着乡谈道:“这太难了,拣不出一个好的,便怎么样呢?”金观察看了一看,也把双眉一皱道:“没奈何,将就些儿选一个就是了。”秋谷道:“就是矮子里头选将军,也选不出来,这有什么法儿?”金观察听了,摇头不答。

正在这个时候,门外又走进一个倌人来,黑面长身,腰圆背厚,浓眉大眼,阔口方腮,挺着个肚子摇摇摆摆的走进来。章秋谷见了,不觉吃了一惊,向金观察道:“这样的奇形怪状,吓也被他吓死了!就是上海的花烟间娼妓,也要比他好些。”

章秋谷的意思,只道天津人不懂苏州话,所以这几句话儿也是打着苏白讲的。那里知道这个最后进来的丑鬼,听了秋谷这两句说话,不觉脸上变色,一张漆黑的脸泛出一阵红云,大声说道:“你们两位老爷,怎么跑上门来骂人?什么叫作不如上海的花烟间?”秋谷出其不意,忽然听得这位宝贝说起话来声音洪亮,就如破锣败鼓一般,倒被他吓了一跳,一时间倒回答不出来,只得勉强和他支吾道:“你不要听错了我们的话儿,听到隔壁去了。我们讲的是上海的事情,说上海花烟间娼妓,一样也有好的,并不是说你们,那里有上门骂人的道理?”那倌人见秋谷向他分剖,明晓得是秋谷说谎,不便再说,只把秋谷瞪了一眼。

秋谷不觉毛骨悚然,有些坐不住,便向金观察道:“我们究竟怎么样?”金观察无可如何,只得随意指着自己身旁一个倌人,问他叫什么名字。那倌人便答应道:“我叫福喜,你们两位老爷到我房间里头去坐罢。”秋谷听了连忙立起身来,同着金观察跟着他就走,直走到福喜房内坐下。登时觉得如释重负,心上松爽非常。

金观察见了,忍不住对着秋谷微微一笑。秋谷自家也觉得好笑起来,一面笑着,一面抬起头来看时,只见房间里头倒收拾得十分干净,湘帘棐几,锦帐银钩,花气融融,芸香拂拂。秋谷看了不觉暗暗称奇,暗想不料北边的窑子里面,竟有这样的地方!可惜这班人物,一个个都是奇形怪状、牛鬼蛇神,未免辜负了这般精室。心上想着,再看那福喜时,只见他黑漆漆的一头头发,水汪汪的一对眼睛,虽然姿貌平平,却还没有什么怪相。

当下金观察同着章秋谷坐了一回,又听福喜唱了一个天津小调。秋谷催着金观察要走,金观察也就立起身来,在身上掏出两块钱放在烟盘里面,便同着秋谷出了大门。

金观察便和他取笑道:“你向来自负是个嫖界中的高手,怎么今天也这样的耳红面赤,话都说不出来?”秋谷自己也笑道:“小侄只说他是不懂苏州话的,无意中说了这几句,那知他竟认真起来。一时间不好回答,只好扯一个谎的了。小侄在上海地方,歌场酒阵整整的混了六年,从来没有吃过一些儿亏,今天恰恰的遇着了这个妖魔,却是第一次碰这样的大钉子!”金观察听了不觉大笑起来。

两个人一面笑着,早又走进一家南班子的寓所,叫做五凤班。这个班子一古脑儿只有五个倌人,那四个都是扬州人。只有一个叫月芳的是苏州人,倒也生得骨格娉婷,腰肢婀娜。只是年纪大了些儿,看上去已经有三十内外的模样。梨涡熨贴,未褪娇红;眉黛温存,犹余浅绿。虽是秋娘半老,却还狠有些徘徊顾影的丰神。

月芳见了秋谷,不觉心中一动。又听得金观察说,秋谷是从上海来的,更觉得十分巴结,百倍殷勤,对着秋谷飞个眼风道:“章老爷来浪上海白相惯仔,天津地方格两个倌人,章老爷陆里看得上?只好将就点哝哝格哉。”秋谷微笑道:“你们这里只几个人,老实说我都看不中,刚刚的只看中了你一个。你的房间在那里?我们过去坐一会儿。”月芳听了道:“阿是真格呀?”秋谷道:“自然是真的。”月芳一笑道:“倪搭别人家做媒人,倒做到仔自家身浪来哉!”说着便握着秋谷的手,走到自家房里。金观察也同着过来。月芳敬过瓜子,提起全付的精神应酬一番。

原来月芳在上海做生意的时候,叫做陆月卿,十年之前狠有些儿名气,枇杷花下,车马常盈。过了几年,不知怎么的忽然门前冷落起来。上海立不住,就到天津来做。在天津做了几年生意,也不见得怎样热闹。月芳回忆当日的繁华,想着如今的落寞,对着那花朝月夕,未免有许多的旧恨新愁。如今见了章秋谷,虽然是初次见面,却把秋谷当作个旧时恩客一般,把自己的遭逢身世约约略略的和秋谷说了一番。金观察和章秋谷听了,都叹息不已。

秋谷见月芳虽然将近中年,芳时已过,却是语言伶俐,丰格清华,心上便觉得有些属意。略略的坐了一坐,便向金观察道:“时候已经不早,差不多将近五更,我们还是回去罢。”金观察点一点头,便同着坐轿回去。

秋谷因晚间困倦,又路上辛苦,直睡到十点钟方才起身。金观察已经上了衙门回来,和秋谷商议,要请他当洋务局的总文案。秋谷想了一想,也便答应。秋谷本来有个候选同知的功名,就是安中堂办顺直捐的时候,秋谷太夫人听得人说,这一次开捐以后就要永远停捐,那顺直捐的折扣又实在来得便宜,就出了七百多两银子,和秋谷捐了个候选同知。秋谷心上不愿用捐班出身,这个头衔从来没有用过。如今金观察要请秋谷当洋务局总文案,官场里头的规矩,没有功名的人是不能当差的,这个洋务局总文案又是个紧要的差使,不能不搬出这个功名来装一装场面。

金观察因秋谷素日性情高傲,一定不肯受他的委札,便把委札改了个照会,用上关防,自己亲手送交秋谷。秋谷接过来看时,见不是札子,方才道谢一声,收了下来。又向金观察说道:“小侄蒙老表伯的垂爱,本应立刻到差。但是千里长途,未免有些劳顿,要在老表伯这里告假三天,小侄也好借此休息。”金观察听了自然一口答应。

到了晚间,金观察又在双福班请秋谷吃了一台酒。秋谷又看中了一个十三岁的清倌人,名叫月香,邀同众人到月香房间里头去打了一个茶围。

一连闹了几天,秋谷假期已满,金观察同着秋谷到洋务局去到差视事。又引着他见了会办宋观察、帮办徐观察、提调召太守。秋谷见了宋观察、徐观察、召太守等,并不请安,也不行礼,只打了一个拱。那知这位宋观察和徐观察,是最有官场习气,最爱闹牌子的,见了秋谷这样的礼数疏狂,语言直率,心上大大的不以为然;只碍着金观察的面子,不好说出什么来。只有提调召太守,是个举人出身,少年时也是个有名的狂士,见了章秋谷这样的丰裁俊爽,举止从容,知道不是寻常人物,便有心要结识这个人。两个人常常聚在一起谈天说地,我佩服你的意气,你羡慕我的才华,倒成了披肝沥胆的朋友。

秋谷自到洋务局以后,金观察每逢有了疑难的交涉,便和秋谷商量。秋谷感激金观察推诚相待,也是推心置腹的和他尽心策画,竭力扶持,宾主之间十分相得。

有时遇着事情棘手的地方,秋谷又援照各国的条约,和外国人反复辩论,外国人也无可如何。

这一天,秋谷正在洋务局里头和召太守讲论那中外约章的失败。讲论了一回,又提起近来交涉的困难来,秋谷便向召太守道:“我们中国到了如今的这般时候,再要和洋人办交涉,自然是困难非常。但是这个原因,不在于如今那班办交涉的人员,却在于当初那些定条约的饭桶。为什么呢?这个条约原是国际里头一件最紧要、最重大的东西。另外有这样的一家学问,深文钩义,和别的文法大不相同,不是局外的人可以弄得来的。所以他们泰西各国订定条约,另有条约专家,一字一句细细的斟酌,就是一个半个字儿也不是轻易用的。那里像我们中国一般,把这样紧要的事情一古脑儿都交给那一班不谙交涉、不懂条约的大员,自然闹出许多笑话、种种失败来了。更兼这个商订条约的这一种学问,里头的道理甚是精微,你就是放着几个博古通今的名士,熔经铸史的大儒,在这里要是叫他和外国人订起条约来,也未见得一定就会妥当。总之,这个学问别是一种工夫,另有一家门路。就和我们中国的公文案牍一般,尽有那一班下笔千言的才子,你叫他办个照例的公牍,他倒提不起笔来。那些州县衙门里头的书吏,平时写个条子都写不上来的,办起公事来倒办得清清楚楚,没有一些儿不通的地方。商订条约,办理交涉,也就是这个样儿,一丝一毫都错不得的。比如你当个办交涉的人员,和洋人订一个条约,那条约里头的话儿看上去都是平平常常,并没有什么紧要的地方;那里知道,到了日后洋人忽然来和你交涉起来,认定了条约里头的一句说话,当作个和你交涉的凭据,只说约章里面早已订明,叫你无从回驳。其实你当初和他立约,条约里面虽然有这样的一句话儿,却不是这般解决的。禁不起洋人忽然翻过脸皮,把好好的一句说话颠倒了一个过儿,硬要这般解决起来。到了那个时候,你反悔又反悔不来,磋磨又磋磨不下,方才知道这个条约不是靠着政府里头的一二大员冒冒失失、糊胡涂涂就可以乱定得的。你想,我们中国那几个最初订定条约的人,那一个是明白外交的?那一个是熟谙条约的?那些损失国权、关系体统之处说也说不尽许多!虽然是那班人不中用的饭桶办理不善,却也不能全怪他们,政府里头的人也有些儿不是。他们那些人自少至老只晓得吃饭拿钱,请安叩首,何曾知道这‘条约’两个字儿是个什么东西?平空的叫他们去和外国人订起什么条约来,好象抓着了个北郭的农夫定要叫他持筹握算,捉住了个南山的石匠定要叫他镂玉雕金。闹到后来,终久还是个一物不成、一事不就!究竟还是农夫、石匠的不是呢,还是指使的人不是呢?”正是:

大好河山,寂寞新亭之涕;可怜明月,凄凉庾亮之楼。

要知后事如何,请看下回交代。





第一百四十六回 论交涉清言讥俗吏 纵微辞谈笑说官场





只说召太守听了章秋谷的话儿,连连的点头称是道:“你的话儿实在讲得透澈。

如今的那班办交涉的宝贝,一个个都是坐了这个毛病。当初订定条约的时候,糊里胡涂就是这样的一来,那里懂得什么条约的学问?比不得他们外国派出来商订条约的人,一定是长于外交、熟谙例约,办起交涉来自然不至茫无把握。我们中国这班人那里是他的对手!据我想起来,这些商订约章、办理交涉的事情,另有一种专门的学问,不是那些门外汉可以率尔操刀、鲁莽从事得的。更兼商订条约,关系非常,一个不小心就要损失许多的权利。就是一个无关轻重的字儿,一句绝无系属的说话,也一定要再三审慎,没有一些儿疏忽的地方,方才保得将来不另生枝节。你若是一时忽略,不去细细的推敲,只说这句话儿、这个字儿是不关紧要的,随随便便的就答应了;那里知道,将来就在这个不关紧要的地方平空生出许多枝节,闹出绝大的交涉来!这样的事情,我在这里见了也不止一次。我以前也曾上过一个条陈,请在总理衙门里头设一个外交馆,专门培植那些办理交涉的人才。无奈人微言轻,大家非但不以为然,倒反一个个都说我无故多事。这些话儿,我以前也和金观察说过,金观察倒深以为然。无奈金观察也没有什么大权力,在上的人置之不理,说来也是枉然。方才你说的一席话儿,真是一句一字都打到我心坎里去,没有一句不是我心上要说的话儿,真是英雄所见略同,不是那班庸庸碌碌的人可以妄参末议的。“

章秋谷听了笑道:“极承推许,惭愧非常。但是我的心上还有一个意见:如今那班办交涉的人──”

秋谷正说到这里,只见金观察在外面走了进来,章秋谷和召太守连忙立起。金观察忙道:“请坐,请坐。我们都是自己人,何必要讲这些过节。”说着金观察自己便也坐了下来,章秋谷和召太守也就一同坐下。金观察道:“你们谈论得正在十分热闹,被我进来打断了你们的话儿。如今你们只顾谈你们的,待我来做个旁听的人何如?”秋谷笑道:“小侄和召太尊方才讲的,就是我们中国交涉失败的原因。”

说着,便把方才一番议论约略述了一遍。金观察也不住的点头称是。

秋谷又道:“据小侄的意见看起来,如今我们中国的交涉失败还有一种原因:第一种原因是条约失败,方才已经讲过,不必再去提他。第二种原因,却都是给那班办理交涉的官员闹坏的。他们那班饭桶,好容易花了无数的银钱,走了许多的门路,方才谋得一个功名,钻得一个差使,兢兢业业的捧着脑袋过日子,一个树叶子下来也怕压破了头。平时见了上司,一味的只晓得掇臀放屁,捧卵呵脬,这样的人要叫他去办交涉,你想可中用不中用?只要一见了外国人的影儿,不等他开口说话,早已吓得魂飞魄散,骨软筋融,一味的唯唯诺诺,凭他要怎么样就怎么样,那里敢驳他一个字的回!在他自己心上想起来,得罪了上司还好请个旁人解释解释,或者行些贿赂也就罢了;要是得罪了外国人,就是上司和他十分合式,也是偏袒不来的。

所以办起交涉来,凭着那外国人怎样的要求、那般的强硬,也不敢说半个不字、放一个屁儿。他那里知道,外国人的办交涉也是专用诡谲手段的。他自己明晓得这件事情不合条约,有妨公法,未见得办得到,他却故意装个胡涂,姑且向我们中国要求一下。若是我们中国的外交官据着条约公法和他抗辩,他也就不来提起,只当没有这件事儿一般。在他原没有一些儿损失,不过费他一个照会就是了。万一个那班办理交涉的人不明条约、不谙公法,竟是轻轻易易的答应了下来,他就得步进步,要求无已;并且从此以后还要把这件事儿当作旧例,节节挟制,事事诛求。他们那班饭桶只说外国人的事情不是顽的,遇着有什么交涉的事件免不得将就些儿,敷衍一下,叫他心上喜欢,以后或者可以省些困难。那里知道,如今这般的竞争世界,只有进步,没有退步的。就是一件至微极细的事情也一定要和他据理力争,退让不得。若是遇事退让,处处将就,今天退让来,明天将就去,一天一天的让来让去,我们中国缩退一步,他们外国人便占进一步,得寸进寸,得尺进尺,到了后来一定要弄得无可退让,无从将就。那其间退让不得,将就不来,势必至于彼此决裂,酿成重要的交涉。与其遇事将顺,到后来依然还是收拾不来,不如在交涉之初,就正正堂堂的和他磋磨辩驳,据约争持,到后来还不至于这样的溃败决裂,不可挽回。

在他们外国人的一方面看起来,却也怪不得他们痛恨,以前的种种要求,没有一件不肯,没有一事不允,到了如今忽然两下龃龉起来,自然是恨入骨髓的了。就是如今各省的民变、闹教的案件,那一件不是地方官激出来的?要是那些地方官能够放大了胆,逢着民教交哄的事情,一秉至公的按律办理,不要袒护教土,凌虐百姓,也何至于闹出这样的事情来!总而言之,做官的人要是存了个患得患失的心,就断断不能办事。小侄狂瞽之论,老表伯以为何如?“

金观察拍手道:“你的话儿一些不错,正和我的意见相同。如今那班办交涉的人要是个个都能依着你的话办事,我们中国的利权何至这般丧失!我们中国的百姓何至这样受欺!”说着三个人不免嗟叹一番。金观察道:“如今官场中人的卑鄙龌龊,比那前十年的情形更是不同,就是说也说不尽许多。别的都还不必说他,最可笑的就是我们这班候补道,你只看全国行省里头那些最重要的差使,什么银元局、铜元局、铁路、矿务、军政、警军,那一处的总办、会办不是候补道当的?好象世上的人只要是个候补道,就无所不通,无所不晓,不论什么事情都是内家,不管什么要差都是熟手。好象不是候补道就不胜其任的一般。你想,那些候补道里头大半都是些有钱的纨袴子弟,仗着家里头的有几个钱,捐个功名出来顽顽,那里会办什么事情?虽然候补道里头也未尝没有几个精明强干、有才有识的人,却是十个里头找不出这样的一个。把国家的大事,一古脑儿的都交给这一起酒囊饭袋的庸才,我们中国的前途那里还有什么希望!”说着不觉长叹一声。

秋谷道:“老表伯这番说话委实不差。如今那班候补道里头,像老表伯一般的人不要说十个里头找不出一个,就是全国的候补道一古脑儿合拢起来,只怕也拣不出几个!”金观察笑道:“这句话儿你是违心之论了。像我这般的人,在候补道里头虽不是什么酒囊饭袋,却也算不得什么奇材异能。不过抚心自问,还不是那班尸位素餐的人物罢了。你的说话未免称誉得过当些儿。”

召太守接着说道:“秋谷兄的话儿却也不是过赞,委实如今直隶通省里头和大人一般热心办事、才识兼优的,却是寥寥无几。”金观察哈哈的笑道:“今天什么道理,你们两个人忽然这样的谬赞起来。”章秋谷道:“小侄的为人,老表伯是向来知道的,从不肯胁肩谄笑,当面阿谀。就是召太尊,也不是这般卑鄙的人物。”

章秋谷正说到这里,忽然外面有人来拜会金观察。当差的传了进来,金观察连忙起身出去。临走的时候对着秋谷道:“今天余太守请你在上林春晚饭,你去不去?”

秋谷道:“如若老表伯去,小侄一定奉陪。”金观察点一点头,匆匆的走了出去。

当下章秋谷又和召太守谈了一回,又办了些日行的公事,看看日色西斜,便回到卢家胡同金观察的公馆里头来。只见余太守已经来了,在金观察书房里头谈天,见了秋谷连忙拱手道:“我只怕秋谷先生不肯赏光,所以特地自己过来奉请。”秋谷道:“岂敢岂敢!多承赐饭,深扰郇厨,那有不到的道理!”余太守道:“好说,好说。秋谷先生为什么要这般客套?”金观察便取笑他们道:“我看你们两个不是在这里讲什么话,大约是你们两个结了新亲,今天在我这里会亲,所以一个这般客气,一个又是那样谦恭,不然为什么要这般拘束呢?”说得秋谷和余太守两个都笑起来。

余太守坐了一会,便向秋谷道:“如今差不多有六下钟,我们就去好不好?”

金观察便对秋谷道:“今天我听说天仙戏馆里头,来了个上海新到的女伶冯月娥,花旦戏串得甚好,我们何妨早些吃了晚饭赏鉴他一下子?”余太守听了先自高兴,口中说道:“狠好,狠好。我们吃过了立刻就去。想不到我今天这个东道主人做得竟不折本!”

金观察和章秋谷听了都微微一笑。章秋谷不说什么,金观察却对着余太守道:“你的算计既然这样精工,何不索性连今天的一顿晚饭都不要请,岂不更占便宜?”

余太守听了,跳起来对着金观察打了一拱道:“既然如此,今天对不起,一客不烦二主,爽性我奉托了你老哥和我代作了今天的主人,何如?”金观察大笑道:“好得狠,好得狠。你既然舍不得花钱,我今天非但不要你出一个大钱,爽性再送五块钱给你用用好不好?”

章秋谷听到这里,忍不住“格”的一笑。余太守也笑道:“不好,不好。给你占了便宜去了。”金观察道:“你自己情情愿愿、伏伏贴贴的叫我来占你的便宜,我不好意思推却,自然只好领你的情的了。”余太守笑着,“呸”了一口道:“小孩子没有规矩,满嘴里乱讲的是些什么话儿!”金观察拈着自己的胡须,对着秋谷道:“你听听他,倒叫我是小孩子!你想可笑不可笑?”

三个人一面说笑,大家都坐上轿子到日本租界的上林春番菜馆来,拣了楼上的一间房间坐了。余太守便写了几张催请客人的条子交给细崽,叫他立刻送去。请的客人就是言主政和杨司马两个,宾主只有五个人。正是:

胭脂照夜,楼台歌管之春;粉墨登场,傀儡衣冠之恨。

不知后如何,且待下回交代。





第一百四十七回 演活剧刻意绘春情 儆淫风当场飞黑索





且说余太守在上林春请客,金观察和章秋谷是和余太守一同去的,还有言主政和杨司马两个人一会儿也都来了。金观察便和众人写起叫局的条子来。原来京津一带,不说叫局,只说是叫条子。当下金观察叫了宝华班的金兰,余太守叫五凤班的桂红,杨司马叫东天保的贵喜,言主政叫富贵班的银珠,章秋谷自然是叫宝华班的云兰不用说了。

条子发了出去,余太守便请众人点菜,写好菜单交给细崽拿了出去。不多一刻,细崽端上汤来,叫的姑娘也都来了,一个个坐在客人后面。金兰和桂红,秋谷本来认得;贵喜和银珠,秋谷虽然也在金观察席间见过一次,却看得不甚清楚,又仔仔细细的打量一番:虽然比不上金兰和云兰两个,却也还五官端正,身段玲珑,并不十分惹厌。

那桂红见了秋谷,忽然想起招呼月芳的客人,连忙问道:“章老爷,你不是招呼月芳的么?为什么不去叫他?”秋谷微笑,摇一摇头。云兰却瞪了桂红一眼。金观察便道:“月芳和你狠要好的,你就多叫一个也没有什么。”秋谷道:“我们今天要去听戏,一会儿就要走的,改天再叫罢。”金观察听了,也就不说什么。

云兰却拉着秋谷的手,附着耳朵悄悄的说道:“耐勿要去做啥格石灰布袋,阿晓得?今朝看过仔戏,阿到倪搭去呀?”秋谷略一沉吟道:“等一会再说,不来也说不定。”云兰又低声说道:“倪勿要。晏歇点定规要耐去格!”秋谷听了,便也附着云兰的耳朵说了几句,云兰面上一红道:“倪是勿晓得格。

金观察见他们两个附耳说话,便喝一声采道:“你们两个人不用这般鬼鬼祟祟的样儿,今天我来和你们做个媒人何如?”章秋谷微微一笑,也不言语。云兰接口说道:“格末蛮好,就请耐金大人搭倪做个媒人,勿得知倪阿有格号福气?”说着自觉有些不好意思,红着脸回头一笑,恰恰和章秋谷打了一个照面。秋谷便握着他的纤手,定睛细看时,只见他宝靥微红,梨涡欲笑;柳挹双眉之翠,花飞一面之春;头上带着两条茉莉花条,一阵茉莉花香直送到章秋谷鼻孔中来。

秋谷到了这个时候,不由得心中一动,两只眼睛一瞬不转只是静静的看。云兰被他看得有些不好意思,不觉“嗤”的笑道:“耐格人啥实梗呀!”秋谷微微一笑,一言不发,只细细的领略那静中香色、个里温柔。云兰见他看得诧异,不由得脸上竟红起来,推开了秋谷的手,口中低低说道:“耐勿要实梗哩,拨别人家看仔,阿要难为情!”说着便立起身来走到那边,对着壁上的着衣镜理了一理鬓发,又取出一个小小的牙梳来把前刘海梳了一梳。回过头来对着章秋谷嫣然展笑。秋谷也对着他微微的飞个眼风。

余太守见了便嚷道:“你们两个人有什么话儿只顾当着我们讲就是了,何必要挤眉弄眼的做出这个样儿来!”秋谷听了还没有开口,言主政便也笑道:“秋谷兄既然这样的赏识云兰,明天何不就在他那里吃一台酒,也好等我们做个现成媒人。”

正说着,忽然听得笛声嘹亮,金兰低低的唱起昆曲来,大家要听曲子,便打断了话头。秋谷原是个惯家,听他唱的是《八阳》,便按着节拍一句一句的听下去,觉得一字一转,音节缠绵,便不由得喝一声采。接着云兰唱了一段《二进宫》,却也唱得平平稳稳的,没有什么舛误,大家也不免得赞了一声。桂红是不会唱的。贵喜、银珠都唱了一支天津小调。

五道菜已经陆续上完,桂红和贵喜先自去了。金兰尚有别处转局,便也匆匆走了。只有云兰和银珠要同着众人一起去听戏,秋谷和言主政自然答应。一会儿细崽送上帐来,余太守签过了字,大家谢过主人,出了上林春,竟到东门外天仙戏园来。

这个时候已经差不多有八点多钟。金观察是预定的包厢,大家一哄上楼,各自坐下。举目看时,已经挤得个人山人海,连包厢都挤得满满的了。原来天津、京城的戏园规则和上海不同,上海是不论包厢正桌,一样都是上等人的座位,只有同着女客的方才去坐那包厢。平常的人大半都坐正桌,看得清楚些儿,听也听得明白些儿。京城和天津的戏园,上等人出来听戏大家都坐包厢。那池子里头的正桌,都是些下流社会的人物,上等人一个都没有的,表过不提。

只说金观察邀着大家坐下,先拿过戏目来看时,只见戏目上排着男伶高福安的《金钱豹》、青菊花的《珍珠衫》、小陈长庚的《奇冤报》,又是女伶尹鸿兰的《空城计》、小菊英的《烧骨记》、冯月娥的《卖胭脂》。原来天津戏馆都是男女合演的,所以生意十分发达,地方官也不去禁他。

这个时候,台上正在那里演《金钱豹》。这个高福安本来也是个著名的武生,台容既好,武工也狠不差。这出《金钱豹》更是他的拿手好戏。到那飞叉的一场,高福安卖弄精神,拿着一把明晃晃的真叉飞得穿梭一般的,没有一些儿渗漏。那个做配角接叉的开口跳刘燕云,也接得十分神捷,伶俐非常。大家都称赏不已。

《金钱豹》演毕,就是青菊花《珍珠衫》上场。那青菊花穿著一身艳服,婷婷袅袅的走到当场,恰生得骨肉停匀,丰神妍丽。比临风之玉树,粉面凝脂;同出水之芙渠,纤腰约素。好似那一朵彩云,慢慢的飞到台前的一般。那态度神情,也不像什么男扮女妆,竟是逼真的一个大家闺秀!出得场来,流波四盼,狠有些娇羞腼腆的神情。

秋谷见了,先叫一声“好”,对着金观察等道:“这个青菊花狠不错。据我看起来,比那上海的什么高彩云、周凤林还要胜些。”一面说,一面看,看着那青菊花的做工也觉得甚是到家。直到小陈长庚唱完了《奇冤报》,方才是女伶出场,尹鸿兰起着孔明出来。秋谷仔细看时,见他短短的一个身材,台容也不见得十分出色,唱工倒还没有什么,就是喉音低些。秋谷便有些不高兴看,回过头来低低的和云兰握手谈心,也不去看那戏台上做些什么。

一会儿的工夫,小菊英《烧骨记》唱过,就是冯月娥的《卖胭脂》。刚刚出得戏房,就听得楼上楼下的人齐齐的喝一声采,轰然震耳,倒把个章秋谷吓了一惊。

章秋谷在上海的时候也看过冯月娥的戏,觉得平平常常的,也没有什么出类拔萃的地方。如今见了冯月娥,又细细的打量了一番,觉得还是和从前差不多。面貌本出平常,唱工又不见得大好。只有那一对秋波生得水汪汪的,横波一顾,剪水双清,着实有些勾魂摄魄的魔力。章秋谷看了暗想:“虽然一双眼睛生得好些,却究竟不是全材,唱工、做工也都狠是平常,为什么天津地方的人要这般的赏识他?”想着,又留意看他的做工,觉得似乎比以前做得老到些儿。那里知道这个冯月娥做到“买脂调戏”的一场,竟当真和那小生捻手捻脚,两个人滚作一团,更兼眉目之间隐隐的做出许多荡态,只听得楼上楼下一片声喝起采来。

秋谷本来最不喜欢看的就是这些淫戏,如今见冯月娥做出这般模样,不觉浑身的鸡皮疙瘩都直竖起来,别过了头不去看他,口中只说:“该死!该死!怎么竟做出这个样儿来,真是一些儿廉耻都不顾的了!”金观察等看了也说形容得太过了些,未免败坏风俗。只把一个云兰看得满面通红,低着个头,抬都抬不起来,拉着章秋谷的手,口中说道:“格号浪形,勿知区俚那哼做得出格!看仔阿要勿色头。”章秋谷附耳和他说道:“你不要说他浪形,等回儿我们两个人也去串一下子给众人看看,何如?”云兰打了秋谷一下道:“倪是勿懂格,请耐一干仔去串罢。”说着忍不住一笑,面上更红起来。

秋谷正和云兰说笑,忽然又听得那些座客齐齐的喝起采来。秋谷连忙看时,只见冯月娥索性把上身的一件纱衫卸了下来,胸前只紥着一个粉霞色西纱抹胸,衬着高高的两个鸡头,嫩嫩的一双玉臂。口中咬着一方手帕,歪着个头,斜着个身体,软软的和身倚在那小生的肩上,好似没有一丝气力的一般。鬓发惺忪,髻鬟斜亸,两只星眼半开半合的,那一种的淫情荡态,就是画都画不出来。

这个时候,不要说引得那班听戏的人人人心动,个个神摇,就是章秋谷这样的一个曾经沧海的人,也不因不由的心上有些跳动起来。云兰坐在秋谷背后,也有些杏眼微饧,香津频咽。耳中只听得一片喝采的声音,好似那八面春雷,三千画角,直震得人头昏脑痛,两耳欲聋。

正在闹得沸反盈天之际,猛然见外面走进几个人,分开众人,一直挤到台前。

头上都戴着缨帽,脚下都穿著黑布快靴,好象衙门里头的差役一般。众人见了,大家摸不着头路,不知道是来做什么的,大家都眼睁睁的看着。

不想这几个人到了台前,抬起头来向台上看了一看,竟大家登着台前的桌子跳上台来。台上的人见了十分诧异,正要开口问时,说时迟,那时快,有一个为首的人抢上一步,抢到冯月娥身旁,“豁啷”的一声,袖管里头掏出一根铁练,呼的就向冯月娥头上套去。冯月娥正在卖弄精神的时候,不提防竟有这样的事情,一时间大惊失色。想要开口问时,张口结舌的一时那里问得出来。

台下那班听戏的人见了这个样儿,大家都七张八嘴的嚷个不住。早见那几个人取出一张访牌,向着台下众人扬了一扬,大声说道:“我们是天津县沈大老爷手下的衙役。沈大老爷奉了天津府林大人的访牌,要立拘这个冯月娥到府听讯。我们是奉上差遣,概不由己,列位不要见怪。”说着便牵着冯月娥向戏房里走了进去。正是:

桃花轻薄,荒凉洞口之春;柳絮颠狂,辜负东风之意。

不知后事如何,请待下文交代。





第一百四十八回 印深情软语留春 谐好事平康选梦





只说金观察和章秋谷等见冯月娥被天津县差役拿去,虽然吃了一惊,大家心上却甚是畅快。秋谷只说:“拿得好,拿得好!若是凭着他一味的这般混闹,不去问他,将来各处戏馆都大家效尤起来,地方上的人心风俗还可问么!”金观察等听了,大家都点头称是。只有一个云兰倒大大的吃了一吓,吓得个目瞪口呆,紧紧的拉着章秋谷的衣服几乎要哭出来。秋谷见他这般胆小,觉得甚是好笑,连忙安慰他道:“你不用害怕。他们拿的是冯月娥,与你什么相干?”云兰道:“倪只怕俚也要来捉起倪来末,那哼弄法呢?”秋谷笑道:“你好好的没有犯法,断没有什么人来捉你的;你只顾放心就是了。”云兰听了方才觉得放心,却还拉着秋谷不放。

这一出戏本来是排在结末的,如今这样的一来,一霎时止鼓停锣,收场罢演。

那一班听戏的人也大家扫兴而归,就如潮水一般的拥出门外。金观察见挤得利害,便招呼众人索性停一回儿,等人少些再慢慢的走,大家依言坐下。云兰趁势低低的和秋谷说,要秋谷送他回去。秋谷沉吟道:“今天时候不早,差不多已经十二点钟。

我明天还有要办的公事,一准明天晚上来罢。“云兰拿着秋谷的手放在自己胸间道:”耐摸摸看,倪格心跳得来掏掏,吓得倪来要死。耐末再要实梗勿肯送倪转去。“

秋谷听了,果然把手去摸他胸膛时,真个一个心拔拔的跳个不住。

这个时候,正是五月底的天气,倌人们着的都是绝薄的纱衣。秋谷轻轻一摸,早觉得双峰腻玉,触手如酥,由不得心旌摇荡。更兼云兰对着他俊眼微饧,眉尖斜蹙,看着他的脸,要说什么却又说不出什么来,好似央告他的一般,便也只好点头答应。却又故意问他道:“你叫我送你回去做什么事情?”云兰把眼一瞟,佯嗔道:“勿要瞎三话四哉,烦得来!”秋谷道:“你既然这般说法,我也不必送你回去,省得你心上厌烦。我请个代庖的人送你回去,何如?”云兰低低笑道:“阿育,阿是算扳倪格差头呀!”

金观察坐在那里,看着他们两个人的样儿,觉得目送眉迎,若离若合,别有一种缠绵款曲的神情,暗想:他们两个人认得没有多少时候,怎么就要好到这个样儿?

真是奇怪。正在呆呆的看,被余太守肩上拍了一拍道:“他们两个人头里是有些浑的了,难道你的头里也浑了么?人都差不多散尽了,你们不走,等在这里做什么?”

金观察和章秋谷连忙看时,只那些人果然都已经散得干干净净,便连忙都立起身来。

余太守看着云兰笑道:“你们有什么秘密的话儿,等一会儿到床上去说不好?

何必要这般性急,在戏馆里头做出这个样儿来?“云兰听了,红着脸口中咕噜道:”狗嘴里阿会生得出象牙!耐格只嘴,总归呒拨啥好闲话说格!“余太守虽然是江苏人,却从小儿生长在天津地方,不大懂得苏州话,听了云兰在那里咕噜,虽然听不明白,却知道一定是骂他的,对着云兰把头颈缩了一缩道:”你不要发急,我从此再不开口,何如?“云兰听了一笑,也不理会。

依着章秋谷的意思,要请金观察、余太守等一同到宝华班去,余太守等都说夜深不便,各自别去。言主政也和银珠一同回去。只有金观察一个人,同着秋谷到了侯家后宝华班。

金观察便拉着秋谷先到金兰房间里头去稍坐,秋谷依言,一同走进金兰房内。

金兰立在门口,含笑相迎,亲自和金观察卸下长衫,云兰也照样把秋谷身上着的那件淡湖色金阊纱长衫卸了下来。

坐了一回,云兰要请秋谷到自己房间去坐,秋谷故意道:“等一回儿我就要回去,就在这里坐一下罢。”云兰斜着眼睛瞪了秋谷一眼,似笑非笑的道:“耐今朝阿敢转去!”秋谷笑道:“有什么不敢回去,你又不是我的太太,我为什么要怕你?”

云兰不等说毕,举起扇子把秋谷头上“拍”的打了一下道:“耐勿要来浪搭倪调皮!”

秋谷道:“我规规矩矩的并不调皮,所以要今天回去。若是当真的和你调皮,今天那里还要回去?”云兰坐在秋谷膝上撒娇道:“倪勿来格,耐自家心浪阿意得过?”

说着,直把一个脸儿紧紧的偎着秋谷的脸,附耳低声道:“耐勿作兴实梗样式格。

今朝勿要去哉呀!“

秋谷见他说得这般委婉可怜,早已心中默许,却故意沉吟一会,口中一言不发。

云兰见他始终还是一个不开口,便挽着他的手道:“耐啥格一声勿响介,阿是变仔哑子哉?”说着又回过头来对金观察道:“金大人,耐说搭倪做媒人格呀,帮仔倪留留二少哩!”金观察笑道:“他是有心在你面前装腔做势,你不要去信他。包在我的身上,今天还你一个章二少。如若走了,我赔也赔你一个。”云兰听了,不觉低鬟一笑,立起身来道:“倪是不过实梗哉,耐阿好推扳点。”秋谷听了,不由得也笑起来,拉着云兰对金观察道:“老表伯的严命,,小侄不敢不遵。明天再请老表伯吃酒。”又对云兰道:“我们两个不要在这里惹厌。我们走了,好等金大人放马登场;我们也去办我们的公事罢!”说罢拉着云兰往外就走。云兰面上一红,软软的跟着章秋谷走了过来。

到了那边房内相将坐下,一个娘姨端上茶来。秋谷抬头看时,只见这个娘姨穿著一身玄色铁线纱衫,玄色铁线纱裤,里面衬着一身粉霞色洋纱衣裤。脚下一双玄缎弓鞋,只有三寸多些。玉笋凌波,金莲贴地,比云兰的觉得还要小了好些。头上挽着个懒妆髻,插着两朵白兰花。丰态轻盈,腰肢婀娜。虽然差不多年过三旬,却还狠有些动人的姿态:盈盈凤目,淡淡蛾眉。腮凝新荔,未褪娇红;颊晕梨涡,犹余妩媚。看着秋谷,只是微微的笑。

秋谷见了倒不觉吃了一惊,立起身来,拉着他的手道:“你叫什么名字?怎么我前两天没有看见你这样的一个人?想不到天津地方的娘姨,也有你这般的漂亮人物!”那娘姨见秋谷恭维他的漂亮,心上甚是得意,对着秋谷一笑道:“倪是勿好格,耐勿要来浪瞎三话四。”秋谷道:“像你这样的人再要说不好,世界上的人也没有好的了。”那娘姨把秋谷推了一推道:“耐就是实梗仔罢,阿好请耐少说两声!”

秋谷一笑道:“你到底叫什么名字?为什么前两天没有见你?”那娘姨道:“倪叫老二,刚刚来浪上海来,今朝七点钟到格搭格。”秋谷听了道:“怪不得,我说这里天津地方那里有你这样电气灯一般的人!原来果然是上海来的。”说着不由分说,猛然把他搂在膝上,脸贴脸的偎了一偎。

云兰见了,瞪了秋谷一眼,别转头去,口中说道:“耐勿要实梗哩!格个是倪格娘呀!”那老二也微微笑道:“耐勿要来浪实梗瞎俏。俚是倪格囡仵,耐就是倪格女婿;阿有啥女婿搭丈母吊起膀子来格?晏歇点倪囡仵小姐吃起醋来,耐吃勿消格嘘!”云兰听了,把身躯一扭道:“呒姆末总归实梗,啥格吃醋勿吃醋介!”说着不因不由的两边颊上泛起两朵红云。

秋谷听了他们的说话,起先还不相信,只说是讲的笑话,连忙问道:“难道你当真是他的亲生娘不成?”老二笑道:“勿是真格,倒是假格?的的刮刮,俚是倪亲生囡仵。耐勿相信,自家问俚末哉!”秋谷听了便放了老二,立起身来,对着他深深的打一个拱道:“我实在不知道你就是我的丈母太太,多多得罪。如今只好在丈母太太面前陪个礼儿,休怪方才放肆。”说着又打一拱。老二扭转脸去,只是“格格”的笑。云兰道:“唔笃看看俚阿要厚皮,一塌刮仔才做得出格。”秋谷回过身来,对着云兰,也打一拱道:“我已经在这里打拱服礼,你还吃这般的冷醋做什么?”云兰啐了秋谷一口道:“耐说说末就是歪嘴吹喇叭,难勿搭耐说啥哉。”

秋谷听了,也不去理会他说的什么,只招手把老二叫了过来,问他以前在上海做过生意没有。老二回说:“十年前在上海的时候,叫姑苏林寓。”秋谷虽然以前在上海没有见过他,却知道有个姑苏林寓,善唱青衫,也是个鼎鼎有名的人物。便和他讲些花丛兴废的原因,并上海近来生意的难做。老二拍手道:“二少格闲话蛮准,故歇上海格生意格末叫难做。倪吃仔格碗把势饭,真正叫呒说法。”两个人长篇大套的谈论了一回,讲的都是堂子里头的事实,讲的人手指口划,讲得个娓娓忘疲,听的人也心领神会,听得个津津有味。直讲到差不多两点多钟。

云兰坐在一旁呆呆的听,没有一些儿倦意。还是秋谷觉得时候不早,掏出表来用手轻轻一按,只听得铮铮的打了两下,又打一下,秋谷道:“我们只顾在这里讲话,不知不觉的已经两点一刻了。”老二也立起身来,懒洋洋的打了一个呵欠,笑道:“倪要困觉去哉。唔笃两家头也早点困罢。”说着便叫房间里的人端上稀米饭。

秋谷随意吃些,云兰也吃了半碗,相携就寝。金堂夜永,宝幄香温,绣枕暗推,流苏悄颤;檀口之脂香微度,酥胸之春意初融;艳语轻轻,重帏悄悄,钗堕绿云之髻,汗凝红玉之肤;水泛横塘,云飞巫峡;冰蕈银床之夜,花香月满之宵。一夜无话。

到了明朝,章秋谷直睡到十点钟还没有起来,好梦初回,双晴乍启,只见云兰枕着自己的手臂,还在那里蒙眬酣睡。额上微微的沁出几点汗珠,剩粉末消,残脂犹腻,一缕漆黑的头发拖在枕边。秋谷看着这个样儿,觉得一个心在腔子里头不由的怦怦自动,想要再睡一回,却又睡不着,一个手臂却被云兰枕得有些麻木起来。

见他睡得正浓,却又不忍唤醒他。

正在这个当儿,忽见老二蓬着个头,悄悄的在外面走进来,蹑着脚步走到床前,轻轻的把帐子揭开,探头一望,见秋谷已经睡醒,便低低笑道:“辰光早来浪,困歇起来末哉。”正是:

徐娘半老,犹多姽婳之姿;杜牧重来,尽有烟花之恨。

不知以后如何,请看下回便知分解。





第一百四十九回 遇秋娘一箭贯双雕 卖丰姿春风描倩影





且说章秋谷听了老二叫他再睡一回,便也低低答道:“我睡醒多时,就要起来了。”这两句话儿虽然低低的说,却已经把云兰惊醒,蒙蒙眬眬的睁开眼来看时,只见他母亲正一手拉着帐子,在那里和章秋谷说话。这个时候云兰身上只穿著一身汗衫睡裤,一个头又枕在秋谷臂上,觉得有些不好意思,便一谷碌坐起身来,挽了一挽头发,便跨下床去。秋谷也便起身盥洗。

吃过点心正待要走,老二见秋谷的辫子有些蓬蓬松松的,便拉住他道:“耐来浪倪搭坐歇,倪搭耐打条辫子阿好?”秋谷正觉得头上的发辫有些累赘,便也点一点头,只说:“你是丈母太太,怎么要你打起辫子来,这是不敢当的。”老二笑道:“勿要客气哩。打条辫子末也用勿着实梗客气嘛?”说着便取了一个牙梳、一个竹篦,对秋谷笑道:“倪到对过亭子间里向去风凉点。”秋谷不懂他什么意思,自然应允。老二拉着秋谷的手往外就走。云兰见了,轻轻的咳嗽一声。秋谷听了也不介意,同着老二径到对面房间来。

老二一面和秋谷梳发,一面夹七夹八的和秋谷讲话。秋谷的头发本来不多,一霎时已经打就。秋谷握着他的手,随口谢了一声。不想这个老二,趁着秋谷和他握手,把身体轻轻的一侧,直侧人秋谷怀中,看着秋谷微微的笑道:“昨日夜里向阿曾辛苦?”秋谷见老二忽然做出这般模样来,心上十分明白,只得也向他笑道:“我是没有什么辛苦,倒是你昨天晚上,恐怕不见得睡得着罢?”老二道:“倪困勿着末,总是耐勿好嘛!”

秋谷见他话风逼得甚紧,只得用别话岔开去道:“你和云兰两个人,说是母女,我看起来总有些儿不像,差不多倒有些像姊妹的样儿。你的面上还是十分娇嫩,掐得出水来的一般,那里像什么三十多岁的人?”说着想要立起身来,却被老二把一个身体紧紧的贴着他,一时立不起来。只听得老二低低的说道:“倪是老太婆哉,就是心浪想要巴结耐二少末,也巴结勿上格哉。二少陆里要倪格号人嗄,二少阿对?”

说着竟是纤腰紧贴,雀舌全舒,和秋谷亲热起来,春上眉梢,波横眼角,隐隐的露出几分荡意。

这一番情事好象天外飞来的一般,竟把个章秋谷弄得个解脱不开,推辞不得,没奈何,只得略略应酬。晓日当窗,熏风拂面,鸳鸯选梦,蛱蝶栖云。香销汉殿之屏,春人秋娘之梦。一会儿,秋谷笑道:“今天这件事儿,真是出于意外的。”老二道:“堂子里向,有啥格交代。老实说,吃仔格碗把势饭,陆里讲究得尽实梗几几花花。”说着两个人依旧手搀手的走过来。

云兰见秋谷和他母亲走了过去,一些声息都听不见,早已心中明白了,心上也未免有些发起酸来。见了秋谷走进来,一言不发,只对着他把嘴披了一披。秋谷倒不由的面上红了一红,有些不好意思。倒是老二坐在那里,好象没有这件事儿的一般。秋谷搭讪着走近云兰身旁,轻轻的和他讲了几句不知什么。云兰“格”的一笑,把头摇了一摇;又趁着老二回过头去的时候,把一个指头对着秋谷,在自己脸上划了几划,做个羞他的样儿。

秋谷也不好再说什么,只得胡卢一笑,便问金观察起来没有。老二道:“金大人七点钟就起来,老早转去格哉。”秋谷听了,便连忙立起身来,穿了衣服,在衣袋里头拣出两张十块钱的钞票,交给云兰。云兰看了一看道:“勿要实梗几化嘛。”

秋谷挥手道:“多的就算了下脚。”老二接着道:“间搭天津呒拨下脚格呀。”秋谷道:“这几个钱,何必还去计较他。”云兰把两张钞票里头检了一张,仍旧塞在章秋谷衣袋里头,口中说道:“晓得耐勿在乎格几块洋钿,不过倪间搭呒拨实梗格规矩末,去多拨俚笃做啥?多拨仔也是白白里格嗄,啥犯着呀。耐倒是今朝到倪搭来吃一台酒,搭倪绷绷场面罢。”秋谷见云兰这般说法,只得依他,把钞票收了起来道:“今天的酒是横竖一定要来吃的,你们何必要替我省这几个钱。”云兰笑道:“耐格铜钿忒嫌俚多,送点拨倪用用末哉,去送拨俚笃格号人做啥?”秋谷听了微微一笑,便也坐着轿子回去。到了晚间,秋谷在云兰那里吃了一台酒,又碰了一场和,便一连在云兰那里住了三天。

这几天的工夫,秋谷觉得酒食征逐,有些厌烦起来,便打着主意要静静的休息几天。那知刚刚吃过晚饭坐在房内,余太守忽然跑了进来,谈了一回,金观察也来了,讲些闲话,不觉又讲到嫖经上去,讲论起天津地方的那些倌人来,毕竟比不上上海的那班人物。金观察偶然讲起五凤班的月芳,说:“虽然年纪大些,倒还着实有些风韵。”余太守听了,便要大家同着去五凤班打个茶围,要认认月芳究竟是怎么的一个样儿。秋谷心上不愿意出去,只说这几天身体有些疲乏,想要好好的休息几天。无奈余太守不由分说,一定拉着要去,秋谷被他拉得不好意思,只得勉强应允,和金观察一同出门,一路望五风班来。

到了五风班,月芳见了十分欢喜,一把拉着秋谷的手道:“二少,耐啥洛一径勿来介?倪牵记得来。说二少格两日到仔洛里去哉,长恐耐相好做得多仔,倪搭勿想着格哉!阿对?”说着满面春风的回过身来,先问了余太守的姓,又应酬了金观察和余太守一番。

余太守见他见了秋谷十分巴结,只说是和秋谷有交情的,便对金观察道:“怎么他来得不多两天,已经有了两处相好?你看这个样儿真是十分、二十分的要好,怪不得上海的那班人,一个个都叫他是嫖学大家,果然名不虚传。”金观察听了还没有开口,月芳早对他笑道:“余大人耐弄错哉。倪搭二少客客气气,呒拨啥格相好格。像倪实梗格人末,阿有实梗福气?二少洛里会看中倪介!就是要巴结末,也巴结勿上嘛!”说着,又对着章秋谷笑道:“倪格日仔一看见耐,就晓得耐是老牌子,标致搭仔年轻格相好,勿知几化来浪,洛里会挨得着倪呀!”说罢,把那一双俊眼微微的飞了一个眼风,檀口微开,樱唇略动,对着秋谷把头侧了一侧,嫣然一笑。在秋谷面前打了一个转身,轻轻坐下,翘起金莲搁在自家膝上,细细的结束了一回,札缚得瘦若纤锥,峭如菱角。一面在那里结束,一面时时的斜转秋波,留心看着章秋谷的举动。

章秋谷本来原是狠赏识他的,如今又见他这般的卖弄风情,徘徊顾影。那方才的一个转身,几步路儿,转得甚是娉婷,走得十分圆转,好似那夭桃荡影,杨柳当风;更兼眼波澄澄,只向着秋谷身上转个不住。虽然年纪大些,比不上云兰的那般娇娜;那一种婉转随人的情态,倒觉得比云兰还要胜些。章秋谷到了这个时候,不知不觉的脱口叫一声:“好!”

月芳斜了秋谷一眼道:“啥格好呀?天津人格功架,才是另有一工格。所以洛格排天津人看仔倪,像煞总归勿对,倪来浪间搭生意也清煞。区得今朝碰着仔耐二少,只好请耐二少包涵点倪格哉。”秋谷听了微微的笑道:“我倒并不是在这里拍你的马屁,委实你的一身功架实在不差。不要说天津地方像你这样身段的狠少,就是上海地方,像你这般身段的一古脑儿也不多几个。”

月芳听得秋谷赞他,心上自是欢喜。趁着这个当儿,袅袅婷婷的立起身来,走到秋谷身旁,一手扶着秋谷的肩头,一手整理自己的鬓发。秋谷便把自己坐的椅子让出半张来,挽着他并肩坐下。月芳便道:“勿瞒耐二少说,倪格功架自然勿见得那哼大好。不过比起格排天津人来,老实说,随便那哼总要比俚好点。再讲起格排本地客人来,格末叫来得讨气!勿说俚自家曲辫子,倒说倪苏州人身架勿局。只有耐二少末,真真老牌子哉!晓得格里向格道理,别人洛里明白呀!”秋谷听了,也便点头称是。

余太守不懂这个“功架”是什么东西,便拉着秋谷要问。秋谷道:“这个‘功架’的两个字儿,也没有什么一珲的道理在里头。据我心上想起来,这个功就是功夫的功,这个架就是架子的架。那像那骑马的人和拉弓的人,一定要摆着个四平八稳的架子,方才是个惯家。但是这个架子,也不是个个人都可学得来的,一定要好好的用些功夫上去,方才摆得出这个架子来,这就是‘功架’两个字的命意了。”

正是:

云英有意,春融玉杵之霜;公子多情,月照西楼之梦。

不知以后如何,请看下文交代。





第一百五十回 矢从良缠绵倾肺腑 悲身世老大感年华





且说余太守不懂什么叫做“功架”,秋谷便和他讲道:“这个‘功架’就是北边人的身段。上海地方最讲究的就是这个‘功架’。当倌人的只要功架是好的,就是面貌生得将就些儿,还不要紧;若是没有功架,那就老老实实没有一个人来请教的了。”余太守听了,方才明白。

坐了一回,大家起身要走,月芳早已把秋谷的那件金阊纱长衫捉个空儿不知放在什么地方去了。秋谷虽然看见,却有意装个胡涂,不去理会。到了这个时候,金观察和余太守穿上长衫要走,见秋谷坐在那里不动。金观察一眼看去,不见了章秋谷的长衫,心上自然明白,便对章秋谷笑道:“你在这里坐一会儿,我们还要到别处去走走,明天再来和你道贺罢。”说着回身要走。

秋谷一把拉住道:“这个时候还早,我们何不就在这里碰一场和?老表伯的贵相知,只顾把他叫到这里来就是了。”金观察道:“我们只有三个人,还缺一个,再去请那一个呢?”秋谷道:“何用再去请人?我一个人坐了两分,叫月芳代碰就是了。”金观察便问余太守道:“你有什么事情没有?”余太守本来是最爱碰和的,连忙应道:“我没有事情,我们碰起来就是了。就是有什么紧要的事情,只要有人和我打牌,我也是一定来的。”

月芳听得秋谷叫替他碰和,心中大喜,连忙叫了男班子进来,搭开桌子,配好筹码,大家扳庄坐下。月芳却对着秋谷笑道:“谢谢耐,总算耐二少照应倪格。”

秋谷点一点头,也不言语,大家掳起牌来。

秋谷的麻雀经本来是绝精的,月芳也是个惯家。金观察还不过略略差些,和他们两个人也差得不多。只有这个余太守,和他们差了八九个底子,如何是他们的对手?八圈碰完,余太守输了七十多块,五十块钱一底,差不多输了底半。金观察只输了七八块钱,不算什么。章秋谷也不过赢了二十几块钱。月芳一个人大赢,赢了六十多块钱。

一会儿的工夫收过牌筹,开上稀饭。金观察和余太守略略吃些,辞了先去。章秋谷明知今天是一定走不掉的了,只得随随便便的住下。银釭背影,璧月流光,一晌缠绵,三生缱绻。和那老二的事情一般,都是章秋谷做梦也想不到的。

月芳在枕上对着秋谷叙述自家的遭遇,如何的父母双亡,如何的叔父把他卖人烟花;如何的做了几年,自己竭力赎身,却欠了一身的债;如何的在上海生意不好,没奈何只得到天津地方来。哝哝唧唧的直讲到半夜。讲到那堕溷飘茵之恨,不由得酸酸的流下泪来。秋谷不免款款的安慰一番。月芳说如今年纪大了,只求有个人和他还清债项,把他拔出火坑。秋谷问他身上有多少债,月芳说数目有限,差不多只要一千块钱。月芳见秋谷问他债项多少,只道秋谷有意要娶他,便盟山誓海的十分熨贴,百倍缠绵,定要秋谷娶他回去。

秋谷听他的话儿说得甚是诚切,知道他不是谎话,便也把自己的家事和他说了一遍。只说如今已经有了一个姨太太,太夫人家教方严,断不许再娶第二个的。

“只恨我没有艳福,消受不起你这样的一个人。只好答应了你,和你留心找一个好好的客人,娶你回去。辜负了你的一番好意,也是无可如何。”月芳听了,呆了半晌道:“勿是耐呒拨福气,总归是倪自家格命苦,呒啥说头,一径碰勿着对景格客人。刚刚碰着仔耐二少,倪末倒快活煞,洛里晓得原是一个勿成功!耐阿好照应点倪,搭倪想想法子呀?”说着,由不得两行珠泪直挂下来。

章秋谷见他这般模样,也觉得有些替他心酸,只得好好的劝他道:“你们吃把势饭的,只有赶快拣个合意的客人嫁了他去,方才可以图一个好好的收成。那班不肯嫁人的倌人,年轻的时候客人情愿娶他,他自己倒反不愿。到得后来有了几岁年纪,就是急急的赶着要嫁人,都已经迟了,还有那一个肯来要他?像你这样的人,如今自然不要紧。若再是过了几年,颜色衰零,年华老大,那就真个的要门前冷落,车马稀疏,要想做一个商妇都不可得了。所以我劝你趁着这个时候,放出眼力好好的拣选一个靠得住的客人,嫁了他去,图一个下半世的收场。你想我这几句话儿可是不是?”

月芳听了章秋谷劝他的这一番说话,心上感激非常。感激到极处,又不由得鼻涕、眼泪都滚出来,把一个头紧紧的钻在秋谷怀中,玉体轻偎,云环低熨。那流的眼泪,把秋谷身上的一件汗衫都湿了好些。

秋谷见他听了自己的说话狠有感动的意思,便索性再激他一激道:“据你说起来,做了几年生意不但没有剩钱,而且还做下许多亏空。你想,一个人拼着父母生下来的身体这般糟蹋,无非是为的一个‘钱’字。如今你做了这些亏空,一个大钱不得到手,又何苦要吃这碗把势饭呢?咳!可怜,可怜!你也是个好人家的儿女,一般的也爱体面,一般的也有廉耻。丢掉了体面和廉耻,来吃这碗把势饭,索性多几个钱也还罢了,如今还拖下许多债项,究竟你贪图的是些什么?难道你就不是个人,不是父母生出来的么?”秋谷说到这个地方,不因不由的自己也觉得酸鼻起来,说话的声音已经岔了,眼中也流出两点泪来。

月芳听了秋谷劝他的话儿说得这般沉痛,更觉得一阵心酸,从肚子底下一直透到心窝里来,看着这烟花的苦趣,想着那身世的飘零,止不住泪滚珍珠,鲛绡尽湿,呜呜咽咽的几乎要哭出来。秋谷见了,暗赞他天良未昧,廉耻犹存,将来有人把他拔出风尘,一定不像那林黛玉、张书玉的样儿嫁人复出,重落平康,倒可以保得不出什么乱子。章秋谷这般想着,心上便存了一个要把他拔出火炕的念头。无奈自己已经有了陈文仙,太夫人断断不肯让他再娶第二个。更兼月芳的年纪倒反比自己大着七八岁,也觉得有些不合。只得拿定主意不答应他,只应允替他留意,寻个好好靠得住的客人。月芳见他回得这般决绝,明知道就再说也是枉然,委委屈屈的泪流不止。秋谷免不得温存婉款的慰劝一番。

自此以后,秋谷也常常的在月芳那里走动,月芳便和他说下个月要调头到宝华班去。秋谷诧异道:“这个时候,既不是年,又不是节,你掉的是什么头?”月芳道:“间搭天津地方勿比上海,堂子里向格帐才是一个月一算格,实梗洛调头也是一个月一调。”

秋谷听了暗想:“宝华班里头,自己有个相好在那里,不要等会儿他们两个人大家吃起醋来。”想着,便对月芳道:“宝华班里头,我有一个相熟的在那里,叫做云兰,想来你总认识的。”月芳道:“实梗说起来,定规是耐格恩相好哉嘛。倪搭俚一径来浪台面浪碰头格,有啥勿认得?”秋谷笑道:“我的恩相好,只有一个五风班的月芳,和我是狠要好的。那里还有第二个恩相好?”月芳把眼睛瞟了一瞟道:“像耐实梗格二少,倪洛里巴结得上,搭耐要好?耐要好格人勿知几化来浪,挨着倪不过是应酬应酬罢哉。二少,倪格闲话阿对?”说着不觉低头微叹。秋谷听了,觉得自己的待他,真个有些对他不起的地方,不免心上有些惭愧,连忙把别的话儿岔了开去。依着月芳的意思,调头的那一天要秋谷去吃一台酒,碰一场和。秋谷想了一想,也便点头应允。

那知到了月芳调头的那几天,秋谷忽然发起痧来。叫了一个剃头的人来,在身上打了几针;又请医生服了几帖药。虽然没有什么大病,却差不多一礼拜不能出门。

直到一礼拜之后,方才同着金观察等到宝华班去看月芳。

月芳见秋谷面上瘦了些儿,便问道:“耐一径勿来,面孔浪像煞瘦仔点哉,身体浪阿好呀?”秋谷道:“这几天忽然平空的发起痧来,一连七八天,大门都没有出。”月芳道:“倪晓得耐格日仔勿到倪搭来,定规有个道理来浪里向。格两日阿好点呀?”说着便走过来,把秋谷的头上按了一按,对着他说道:“出门人样式样要当心点格哩,生仔病有啥人来搭耐当心呀?”秋谷听了不觉心中一动,只点一点头,也不开口。略略的坐了一坐,秋谷要到云兰那边去坐。刚刚老二拿着茶碗走了过来,月芳也和他敷衍两句。看着老二对着秋谷那般亲热,心上也有七八分明白,不觉对着秋谷鼻子里轻轻的哼了一声。秋谷只作不知,别过头去。

一会儿,老二拉了秋谷的手,同到那边房内。云兰接着,淡淡的笑了一笑道:“倪搭小地方,今朝勿晓得洛里格一阵好风拿耐格位章二少吹仔过来?耐到搭倪讲讲看,前格两日来浪五凤班里向那哼格窝心,今朝咦那哼肯放耐过来?倪看耐格两日面孔浪瘦仔几几化化,拍马屁末也勿是实梗拍法格嘛!拿仔自家格身体去拍别人格马屁,耐格人阿有啥淘成!”秋谷笑道:“真是冤枉,我在金大人公馆里病了几天,那里有这些事情?你不信,只问金大人就是了。”

云兰听了,起先还不相信,抬起头来把秋谷细细的打量一下,见果然有些病容,方才信了。停了一回,又对着秋谷冷冷的说道:“二少,耐格恩相好时髦得来,间搭宝华班里才是别脚倌人,洛里比俚得上?”秋谷不觉一笑道:“你不用这般酸溜溜的样儿,劝你将就些罢。我的做他,也不过应酬应酬罢了,那里有什么恩相好不恩相好?你只要自己心上想一下子,我的待他怎么样,待你怎么样,就知道我的话儿不是假的了。”云兰听了,想了一想果然觉得不差,便也不说什么,只问秋谷前几天生的是什么病。秋谷和他说了,云兰道:“耐既然勿舒齐,为仔啥事体再要跑出来?阿是出来看看格位新相好?几日天勿碰头,牵记得势,阿好?”秋谷听了,立起身来朝着云兰打了一拱。正是:

春风好去,吹残扬柳之枝;红泪阑干,落尽桃花之色。

不知后来怎样,请看下文,便知分晓。





第一百五十一回 两调头翡翠共移巢 三鼎足鸳鸯齐比翼



且说章秋谷立起身来对云兰打了一拱道:“我有了你这样的相好,不来看你,还要去看什么人?你口口声声的只说他是我的恩相好,你的醋劲也未免来得过度些儿。如今就算我的不是,向你陪个礼儿,以后不要提起这件事儿,如何?”云兰听了把头一扭道:“啥格吃醋勿吃醋呀,倪是勿懂格。耐到说拨倪听听看!”秋谷笑道:“你这个样儿,不是吃醋,难道是吃酱油不成?”云兰走过来,把秋谷背上打了一下,道:“倪是勿会吃啥酱油格,倒是当心别人家来浪吃醋!耐豪燥点去罢,晏歇点吃起生活来是勿关倪事格嘘。”说着,便推着秋谷的背,想要推他出去。秋谷趁势拉着云兰到榻床上去坐下,不免陪个小心,抚慰一番,云兰方才欢喜。

停了一回,云兰忽然正容说道:“二少,倪听见别人家说,耐要开海货行,到底阿有介事?”秋谷诧异道:“你听见那一个讲的?没有这件事儿。”云兰道:“常恐是真格嘘。”秋谷道:“我自己的事情自己不知道,难道你倒比我知道不成?”

云兰忍着笑道:“既然耐勿开海货行末,为啥老蟹腌蟹,一塌刮仔才要收格介?”

秋谷起先没有留心,只道他说的真话,如今听了他这两句话儿,不觉哈哈的笑起来,一面说道:“今天我上了你的当了。我说平空的那里有这件事情。”云兰也把手巾掩着嘴,“格格”的笑个不住。老二听了,心上大大的不舒服,着实瞪了云兰一眼,把身躯一扭,立起来往外便走。秋谷看得十分清楚,却只作没有理会的一般。

老二刚刚出去,早见两三个十二三岁的清倌人,手挽手儿的走进来。见了秋谷,有一个清倌人叫道:“咦,章二少嘛!”秋谷听得有人叫他,连忙举目看时,只见一个穿著男装的清倌人,眉目清澄,肌肤白腻,长条身材,瓜子脸儿,别有一种旖旎动人的姿态。原来不是别人,就是那双福班的月香,便对他笑道:“你是几时调过来的?我竟一些儿都不知道。”月香道:“倪是初一调过来格呀,耐啥洛一径勿见介。”

秋谷嘴里在那里和他讲话,心上在那里暗想:天下竟有这样奇巧的事情!刚刚我在天津地方做了三个倌人,刚刚的这三个人都调在一个班子里头来。好在月香是个清倌人,没有什么要紧。只要云兰和月芳这两个人面前想个调停的法儿就是了。

想着,和云兰混了一回,又到月香那里去坐了一坐。云兰又在秋谷耳边咕咕哝哝的埋怨他,只说他是石灰布袋、垃圾马车。秋谷道:“我在天津地方一古脑儿只做了你们这三个人。不料事有凑巧,偏偏的把你们三个拢到一处来。真是奇事!”云兰那里肯信,只说:“耐格号闲话只好去骗骗三岁小干仵。耐一塌刮仔做仔倪三家头,刚刚三家头才来浪一个班子里向,也呒拨实梗凑巧嘛!”秋谷听他说得有理,料想辩白不来,只说道:“你不信,慢慢的看就是了,这个时候我也不来和你分辨。”

云兰听了,方才不说什么。秋谷坐了一回,便同着金观察一同回去。

一连隔了几天。这一天,秋谷自己在上林春番菜馆请客。请的客人是刑部郎中陈小梅,恰是秋谷的父执,就请金观察和言主政两个人作陪。这位陈部郎恰是个端方古执、拘谨非常的人,所以秋谷不去再请别人,恐怕和他说不到一处。

到了上林春,不多一刻,陈部郎已经来了,金观察便问陈部郎叫那一个的条子。

陈部郎正色道:“我是向来不破这个例的,你们诸位只顾叫就是了。”秋谷道:“今天我们大家谈谈,就不叫也好。”陈部郎道:“你不要为着我一个人,败了你们大家的清兴。逢场作戏,这又何妨?”金观察听了,便写了一个条子去叫金兰,言主政仍叫银珠。秋谷心上暗想:“若是叫了云兰和月芳来,有过相好的,神气之间未免总有些看得出来,不如叫了月香罢。月香是个清倌人,给这个老头儿看了觉得干净些儿。”主意已定,便和金观察说了,写了月香。

一会儿银珠先到,金兰和月香还没有到。等不多时,忽听得门外“咭咭咯咯”

一阵的弓鞋声响,一时间走进三个人来。章秋谷见了这三个人,不由得目瞪口呆,做声不得。看官,你道这三个人究竟是什么人?为什么章秋谷见了他们要诧异到这般田地?原来这三个人不是别人,就是章秋谷的两个相好,一个云兰,一个月芳,还有一个就是方才去叫的月香。三个人齐齐的走进门来:云兰满面凄凉,一言不发;月芳也低眸俯首,神彩黯然;只有月香喜孜孜的叫了一声“二少”。三个人齐齐的在秋谷背后坐下。那位陈部郎见秋谷一叫就来了三个人,心上大不耐烦,微微冷笑。金观察和言主政见了他们三个人一起同来,心上也觉得十分诧怪。章秋谷更是呆呆的看着他们,摸不着头路,不知道究竟是怎么的一回事情。看看这个,看看那个,一时竟说不出什么话来。

云兰见他只是呆呆的看,冷洋洋的说道:“看啥呀,阿是勿认得倪呀?”秋谷听了,方才开口问道:“怎么你们三个人一起来了?只怕你们弄错了罢。”云兰冷笑道:“就是弄错仔末,倪来也来格哉!耐阿有本事,赶仔倪出去。格个末就叫‘人有千算,天有一算’。耐想勿叫倪两家头,倪自然有包打听来浪外势。耐勿叫末,倪两家头自家跑得来,看耐阿有啥法子!”秋谷听了这两句话儿,心上更加不明白起来,又不好问他:你们两个人来做什么?更兼本来原为着陈部郎性情古板,所以有意叫个清倌人的。如今他们两个人不由分说大家都跑了进来,糊里胡涂的不知道葫芦里头卖的是什么药。霎时间,把一个足智多谋的章秋谷弄得左右为难起来。还是月香含笑和他说道:“天津地方格规矩,一径是实梗格呀。一个客人来浪一个班子里向做仔两个倌人,叫起条子来就要一叫两个,吃起酒来就是一吃两台。耐就是条子浪只写一干子格名字,来起来总归是两家头一淘来。间搭地方大家才是实梗样式,耐想阿要诧异。”秋谷听了,方才恍然大悟,如梦方醒,连金观察在天津候补多年,也不知道有这样的一个规矩。

当下章秋谷看着云兰和月芳两个,都是懒懒的没精打彩,好象有什么心事一般,暗想今天的这件事情,在面子上看起来果然有些说不过去。正要和他们说明原委,忽然回过头来把陈部郎看了一看。只见他只顾举着一杯薄荷酒在那里细细的自斟自酌,正眼儿也不看他们一看,知道今天的事情弄巧成拙。若是早知道天津地方有这般的规矩,也就想个法儿,到别处去叫一个了。如今他们三个人既然来了,便也只得由他。等这位陈部郎走了,再去安慰他们也还不迟。

章秋谷心上这般想着,面上却不得不殷殷勤勤的应酬这位陈部郎。一会儿番菜吃完,算过了帐,叫来的条子也都走了,陈部郎急急告辞。章秋谷免不得邀着金观察和言主政到宝华班去,不知费尽了许多口舌,陪尽了无数小心,方才骗得云兰和月芳心中欢喜。又当真和云兰、月芳、月香三个人,一个人吃了一台酒。

流光如驶,不觉又过了几天。章秋谷虽然也常到宝华班去走走,却比以前不便了好些。打个茶围,一打就是三处,叫个条子,一叫就是三个,觉得十分累赘。想要到别处去另做一个,却一时没有个看得上眼的人。

这一天秋谷在云兰房里坐阗,讲起这件事情的不便来,云兰道:“耐自家勿好嘛。啥人叫耐去做石灰布袋,东揩一的的、西揩一的的格呀?倪搭耐讲格闲话,耐总归一句才勿肯听,格末叫讨气。”秋谷听了,一时倒也回答不出什么话来,只说道:“你们这个规矩究竟不好,难道有个客人在你们这里住夜,也是三个一连牵滚作一堆不成。”云兰道:“格是规矩犯就嘛,三家头一连牵滚来浪一堆是勿见得格,不过应酬仔格面,再应酬归面末哉。”秋谷听了,不觉笑道:“既然如此,我今天倒要住在这里,看看你们到底怎样的一个布置。”云兰也笑道:“倪倒从来< 曾勿>碰着歇今朝实梗事体,倪也勿晓得该应那哼。要末叫仔本家进来问问俚,看俚那哼说法。”说着,果然出去叫了宝华班的女本家来。

秋谷便问他道:“你们这里的云兰和月芳,都是和我有交情的。如今我今天想要在这里落厢,究竟是怎样的一个规矩?”本家回道:“那是听老爷吩咐的。老爷说怎么样就是怎么样?”秋谷忍着笑道:“譬如我要叫他们两个人并作一个房间,可办得到办不到?”那本家想了一想道:“要是老爷喜欢这么样,也没有什么办不到。只要请老爷分付一声,叫那位姑娘并过那位姑娘的房间就是了。”

秋谷听了,止不住大笑起来。云兰连忙把秋谷拉了一把道:“耐格人真真呒拨仔淘成哉!客人淘里末并并房间罢哉,阿有啥格件事体也好并啥格房间格?倪是勿来格。请耐去照应仔别人罢。”秋谷道:“你不要发急,我不过说说罢了,那里并什么房间?我自然有我的道理。”便对着那本家说道:“我住在这里,包你两个厢的钱就是了,别的你不用管。月香是清倌人,不在里头的。”那女本家答应一声,退了出去。

云兰撅着个嘴,狠不高兴。秋谷少不得又要好好的温存一会,又在衣袋里头取出一张五十块钱的钞票来,放在云兰手内道:“我本来想和你买些衣服,但是我不知道你爱穿的是些什么颜色,什么样儿。如今这几个钱,给你自己去买两件衣服罢。”

云兰瞅了秋谷一眼,把钞票仍旧放在秋谷手中,口中说道:“耐今朝啥格闸生里想着仔拨起洋钿倪来哉呀?倪也勿要买啥衣服,勿要用啥洋钿。放来浪耐搭仔再说,等倪要用格辰光,再问耐拿末哉。”正是:

春风良夜,双姝开并蒂之花;拥月猥云,鼎足入巫山之梦。

不知章秋谷说些什么,请看下文,使知分晓。





第一百五十二回 循旧例双美拥檀郎 闹相公新知结幽愫





且说章秋谷见云兰不肯拿他的钱,只说他还是有意吃醋,便道:“这一笔钱,我本来早就想要给你的,前几天不知怎样的,心上竟忘了这件事儿,所以直到今天方才给你。你不要,难道是嫌少么?做客人的和倌人有了相好,这一下子竹杠迟早都是逃不掉的,你又何必和我客气?”云兰听了,皱着眉头,把金莲一顿道:“耐格闲话,倒诧异勒海嘛,阿是倪来敲仔耐格竹杠哉!耐自家想想,倪阿曾敲过歇耐一块两块洋钿格竹杠?老实说,故歇倪呒啥用场,耐拨倪自然勿要。等到倪真正要用起来,倪自家会得问耐讨格。耐拿倪当仔别人一样,只认得铜钿勿认得人,格末耐看豁仔边哉。”秋谷听了,看着他的样儿约莫着也不是什么假话,不好再去勉强他,便只得收了回来。这一夜秋谷睡在云兰房内,没有回去。

到了一点多钟的时候,秋谷走到月芳房间里头,只见月芳穿著一件湖色汗衫,卸了头上的钗环,乌云乱挽,坐在灯下,一手托着香腮,一手拿着一个茉莉花球在那里翻来覆去的。看见了秋谷进来,动也不动,只说:“辛辛苦苦,再要跑到倪搭来作啥呀?”秋谷走到月芳面前,低低的笑道:“对不起,累你等了半夜,如今特来陪你。”月芳道:“耐自家身体要紧,轧实勿要过来哉呀。倪是勿搭别人家吃醋格,横竖耐二少自家心浪蛮明白来浪。只要耐照应照应倪好哉,吃仔格碗把势饭真正作孽,再去搭别人家吃啥格醋呀!”

秋谷听了月芳的一番说话,觉得他另有一种口角,说得悱恻可怜,不由得心上也有些替他凄惋,便也拿出一张五十块钱的钞票来送给他。月芳那里肯受,推了半天,月芳始终不肯。秋谷恳恳切切的和他说道:“在你心上的意思,不肯无缘无故要我花钱,我心上也狠明白。但是你欠了一身亏空,可想而知不是有钱的人,手里头也狠是拮据。我和你就是没有相好,平常的时候总算甚是投机,就送你几十块钱帮帮你的忙,也不算什么事情。何况我虽然不是什么巨富,这几个钱也还不在心上。

你若一定要和我客气,那就是瞧我不起了。“

月芳听了,推却不得,只得谢了一声收了下来。却又平空的一阵心酸,泪珠欲落,连忙别转了头,把手巾拭了泪痕,转身对章秋谷道:“倪十四岁落仔堂子,做生意做到仔故歇,客人碰着仔几几化化,勿是靠勿住格滑头末,就是踱头踱脑格曲辫子。直到仔故歇,刚刚碰着仔格耐末,样式才对景。说格闲话,赛过就是倪肚皮里向挖出来格。碰着仔耐实梗格人,倪咦呒拨格号福气。”说到这里,眼圈儿不觉又红起来。秋谷又婉婉转转的安慰了他一番。

自此以后,章秋谷到宝华班去住夜,总是这般的左右逢源,前偎后拥。至于这个里头怎样的一箭双雕,怎样的双管齐下,那却要去问云兰和月芳的房间里人方才知道。在下做书的没有和他们打过梆子,却就不得而知的了。

闲话休提。只说这位金观察,本来原是个举人出身,笔底下狠来得,而且洞明时务,博览群书。这个时候正是皇上家开经济特科的时候。吏部尚书王凤山王冢宰,素来极佩服金观察的学问,就专折奏保了金观察的经济特科。到了六月中旬,已经到了考试的时候。金观察便禀请方制军,派了津海关道李伯溪李观察,来兼理洋务局总办的事情。金观察匆匆忙忙的交卸了一切公事,便带了两个差官,两个家人,克日进京。

临走的时候,和章秋谷商量,想要请秋谷同他进京,两个人住在一起,觉得大家谈谈有兴些儿。恰恰章秋谷也为着金观察进京去了,那几位会办大人和帮办大人大家都和他不合。本来章秋谷的当这个洋务局总文案原是看着金观察的交情,自己原不是一定怎样。如今见金观察去了,那里肯留在洋务局里头当这个没意思的差使。

早就和金观察说过要想辞差,所以金观察趁着这个当儿请他一同进京,章秋谷一口答应。金观察大喜,却不肯叫他辞差,只叫他告了一个月的假。找了一个人和他代理文案上的事情,便同着秋谷上了火车。

天津到京只有二百多里路,不到半天已经到了。金观察本来是常州府阳湖县人,就同着秋谷住在青厂武阳会馆里头。金观察自去料理应考的事情。秋谷没有什么事儿,便出去拜了几天客。就有那班同乡的亲戚朋友,大家都来拜望,也有请他吃饭的,也有请他听戏的,秋谷倒忙了好几天。

这一天秋谷在会馆里头刚刚起来,见当差的传进一个名片来,说姚大人来拜望。

秋谷接过名片来看时,只见名片上写着“姚潇”的两个大字,秋谷便叫快请进来。

原来这个姓姚的名潇,号子湘,也是个直隶候补道,现当京津铁路的督办,和秋谷既是同乡,又是亲戚,向来狠要好的。性情豪宕,学问精纯。以前在常熟的时候,和秋谷也是朝夕过从,契合非常。如今听得秋谷来了,昨日又去拜了他一趟,这位姚观察便连忙起个大早,到武阳会馆来看秋谷。秋谷见了名片,连忙叫请。

当差的出去不多时,早见一个三十多岁年纪的人,大踏步在外面走进来,气概轩昂,英姿飒爽,目光如电,华彩凌云。见了秋谷还在那里洗脸,便笑道:“我只怕来得迟了,你要出去,怎么这个时候你才在这里洗脸?”秋谷道:“这个时候不过八点多钟。若是我们在上海的时节,这个时候正在大槐国里看招亲呢。”姚观察坐下来谈了一回,便对秋谷道:“我们几年不见,今天要好好的和你畅叙一天。这个时候,你就同着我一同回到公馆去,就在我公馆里头吃过了饭,请你到中和园去听小叫天的戏。听过了戏,就请你到升平班小兰那里去吃饭,我们畅畅快快的叙一天,你看怎么样?”

秋谷听了大喜,连忙对着姚观察打了一拱道:“你请我别处吃饭,我不谢你。

你请我吃相公饭,我却感激得狠。我自从那一年出京之后,想着相公饭的滋味,别处地方,凭你怎么样总吃不到这样的好东西,正在这里求之不得。你忽然要请我吃起相公饭来,真叫作天从人愿了。“姚观察见秋谷向他打拱,便哈哈的笑道:”你这一个拱,好象是下了定钱的一般,我就要倒扳桨也不中用了。“

正在说笑,金观察也从自己房内走了进来,姚观察便请他同去。金观察想着这几天刚刚没有什么事情。便也答应。姚观察便立起来对着秋谷同金观察道:“你们既然没有什么事情,坐在这里也没有什么趣味,还是早些到我那里去谈谈罢。”金观察道:“你们两位请先去,我还要去拜一个客,一会儿就到你府上来。”

姚观察听了点一点头,便同着章秋谷一同坐了骡车,直到绳匠胡同姚观察公馆里来。进了大门,姚观察让着秋谷到一间小小的书室里头坐下。秋谷举目看时,只见这间书室收拾得十分精致:一帘花影,四壁图书。案头摆着的,都是些夏鼎商彝,斑烂绝俗。架上放着的,都是些金签玉管,名贵非常。两面都挂着斑竹帘儿,不透一些日色。地上也铺着织花地席。帘外更摆着几盆珠兰茉莉,微凤一动,便有一阵阵的花香从帘隙中间直透出来。

秋谷到了这个地方,一霎时觉得头目爽然,尘襟尽扫,好似服了一服清凉散的一般,便对姚观察道:“到了你这个地方,直可扑去俗尘三斗。不意京城里头这样人海烦嚣之地,居然也有这等地方!”坐了一回,金观察也来了,走进书房四面看了一看,啧喷叹赏道:“好地方,好地方!看了这样的书室,就可见主人胸襟之雅。”

姚观察听了,不免也随口谦让几句,不多一时,又来几个客人:一个就是刑部郎中金星精,是金观察的族侄,本来和秋谷极知己的;一个是浙江道御史郑兰任;一个是军机章京翰林院编修陆云峰。

大家塞暄了一回,姚观察便拱请众人入席。郑侍御便要姚观察去叫小兰,姚观察便问众人怎样,陆太史也点头说好。只有章秋谷没有相识的人,姚观察便荐了一个小兰的师弟小菊给他。一会儿,小兰同着小菊一起到来。秋谷举目看时,只见他们两个人一色的都穿著蝉翼纱衫,手中拿着雕翎扇,脚下踏着薄底靴。小兰是长长的一个鹅蛋脸儿,长眉俊目,白面朱唇,狠有些顾影翩翩的姿态。小菊却是一个圆圆的脸儿,骨格娇柔,风情流动,狠有些天然憨媚的样儿。

小菊一走进来,便问姚观察那一位是章老爷,姚观察和他说了。小菊看了秋谷一眼,走过来就对秋谷请了一个安。秋谷一把拉住,细细的看了一看。小菊笑了一笑,回过身来招呼了席上众人,方才坐下。接着,众人叫的也都来了。秋谷一个一个的打量一番,觉和虽然也有好的在里头,却都不及小兰的身段玲珑,丰神婀娜。

就是小菊,也比小兰差些。秋谷看着,都放在心上,也不言语。大家吃了几杯酒,家人们送上菜来,是姚观察自己公馆里头的厨子做的,做得甚是精美。

席间大家谈起北京人的闹相公来,秋谷便问姚观察道:“我听人说,以前的时候那班京城里头的大老,每逢宴会一定要叫几个相公陪酒,方才高兴。那班窑子里头的妓女却从没有人去叫他陪酒的。偶而有个人叫了妓女陪酒,大家就都要笑他是个下流社会里头的人。自从庚子那一年联军进京以后,京城里头却改了一个样儿,叫相公的狠少,叫妓女的却渐渐的多起来。究竟是怎么一个道理?我记得前几年在京城里头的时候,闹相公的人还狠多,为什么如今丢掉了旱路,忽然又去走起水路来呢?”姚观察听了,叠着指头说出一席话来,正是:

繁华如昨,春城罗绮之天;风月无边,冠盖京华之路。

不知姚观察说的什么,请待下回分解。





第一百五十三回 中和园书生听戏 升平班观察开筵





且说姚观察听了秋谷的话,便对他说道:“你的话儿却是不错。京城里头自从庚子以后,果然变了一个风气。但是这个里头也有一个道理,你听我慢慢的和你讲究就是了。你可知道以前的时候,他们那班大老大家都叫相公,不叫妓女,是个什么道理?”

秋谷道:“大约是为着那班相公究竟是个男人,应酬狠是圆融,谈吐又狠漂亮,而且猜拳行令,样样事情都来得。既没有一些儿扭捏的神情,又没有一些儿蝶狎的姿态,大大方方的陪着吃几杯酒,说说话儿,偎肩携手,促膝联襟,觉得别有一种飞燕依人的情味。不比那些窑子里头的妓女,一味的老着脸皮,丑态百出,大庭广众之地,他也不顾一些儿廉耻。别人讲不出来的话儿,他会讲得出来;别人做不出来的样儿,他会做得出来。若是面貌生得好些,或者身段谈吐漂亮些儿也还罢了,偏偏的一个个都是生得个牛头马面,蠢笨非常,竟没有一个好的,那班大老那里看得中意?妓女既然是这个样儿,自然是万万叫不得的了。那班大老却又觉得不叫一个陪酒的人席上又十分寂寞,提不起兴趣来,所以每逢宴会一定要叫个相公陪酒。

这就是大家都叫相公不叫妓女的原因了。“

姚观察听了道:“你的话儿虽然不错,却还有一层道理在里头。京城里头的妓女自然断断叫不得。就是和上海的倌人一般,百倍娇柔,十分漂亮,这个里头也到底有些窒碍。为什么呢?做妓女的究竟是个女子,比不得当相公的是个男人,凭你叫到席上的时候,怎样的矜持,那般的留意,免不得总有些儿淫情冶态在无心中流露出来。这班当大老的人一个个都是国家的柱石,朝廷的大臣,万一个叫了个妓女陪酒,在席上露了些马脚出来,体统攸关,不是顽的,倒不如叫个相公,大大方方的,没有什么奇形怪状的丑态发现出来。你想我的这一席话可是不是?”秋谷拍手道:“是极,是极!你的一番说话正和我心上的意见相同,不过我放在心上,没有讲出来就是了。”

姚观察又道:“庚子以前,京城里头的妓女都是些本地方人,梳着个干嘉以前的头,穿著一件宋元以后的衣服,紥着个裤腿,挺着个胸脯。我们南边人见了他这个样儿,那一个敢去亲近他?那一个见了不要退避三舍?如今的妓女,却比那庚子以前大大的不同了。那些下等的妓女依旧是本地人,不必去说他。那班上等的妓女却大半都是南边人了。虽然扬州、镇江的人多,苏州、上海的人少,却究竟比本地人高了好些。所以以前不叫妓女的,如今也渐渐叫起妓女来。但是那班大人先生宴会的时候,叫了个妓女在席上拉拉扯扯的,毕竟有些不雅。所以到了如今,叫妓女的人固然狠多,叫相公的人却也不少。但是像以前那般的实事求是,要想中阿行雨,陆地操舟的,却是绝无仅有的了。”

秋谷听了,低头想了一想道:“据这样的看起来,大约妓女里头是优长的占了胜点,劣陋的居于败点;相公里头却是上流的天演竞存,下流的就渐渐人于天然淘汰之列了。”姚观察听了笑道:“不错,不错。妓女里头虽然给外路人占了胜点,那班本地人究竟还不至于到天然淘汰的地位。那班相公里头的下流,如今却当真没有一个人去请教的了。虽然是社会上凤俗的迁移,却究竟逐膻的人多,附臭的人少,这也不是人力可以挽回的。”秋谷道:“既然如此,以前那些专做这个生意,开拓后庭,肉身布施的人,如今又怎么样呢?”

秋谷说到这里,只见那几个相公的脸上都不觉红了一红。小菊却拉了秋谷一把道:“章老爷,这些事情还去提他做什么,我们来猜拳罢。”说着把眼睛微微的向秋谷斜了一斜,伸出一个粉团一般的拳头来,和秋谷猜了五拳,秋谷倒输了三拳。

小菊直打了一个通关,也吃了七八杯酒,吃得个两颊生红,星眸斜睇,觉得越添了几分风韵。秋谷趁着他们大家猜拳的时候,细细的打量这几个叫来的相公,觉得他们的一举一动,一言一语,都狠有些娟媚动人之处。暗想:怪不得他们那班人一个个都只叫相公,不叫妓女,原来相公也有相公的好处在里头。想着,便不由得回过头来看看小菊一眼。小菊见秋谷看他,便寻些说话出来和秋谷讲论。两个人谈人了港,竟是密密切切的长谈起来。直至姚观察要打通关,方才打断了他们两个人的话。

姚观察见他们两个人谈得津津有味,便哈哈的笑道:“你们两个人讲的什么话儿,讲到这般密切。”小菊道:“我们讲的都是些京城里头的事情,不是什么体己话。”姚观察大笑道:“我不过问了一句,并没有疑心你们讲的是体己话,你何必这样的心虚?”小菊听了一笑,也不言语。秋谷也只是微微的笑,不说什么。姚观察对着众人说道:“以前我同着秋谷住在上海的时候,不知怎样的,他做的倌人十个里头倒有九个和他要好的。你们只看今天小菊到来,和他并不相识,就是这般的谈谈说说,熟落非常,好象他身上含着电气的一般,有天然的吸引力,可以吸得动人。这个里头不知是怎么的一个道理?”众人听了,大家都笑起来,都争着要问秋谷究竟有什么秘诀。

秋谷道:“讲起这个里头的关节来,一时就讲也讲不尽许多,只好约略讲个大概就是了。”说着,便把那些对付倌人的法儿,略略的说了几句:如何如何的逢场作戏,认不得真,一认了真必定是自家吃苦;如何如何的随机应变,不可拘泥,看着倌人用出那一等的手段来,便是那一等的对付。众人听了一个个都点头称是。小菊暗暗的把秋谷拉了一把。秋谷回过头来,小菊笑容满面的把一个大指对秋谷伸了一伸。秋谷倒觉得有些儿不得劲儿起来,也对着小菊摇一摇头。不提防被对座的金星精金部郎看见,对着姚观察笑道:“他们两个人果然有些意思,你的话儿委实不错。”大家听了哄然一笑,大家都目不转睛的望着秋谷和小菊两个人。看得小菊脸上竟红起来,立起身来走到帘下去看花,只作不曾理会。

大家又说笑了一回,吃过了饭,一班相公都要回去唱戏,便急急的告辞回去。

婚观察同着章秋谷等略停一停,便大家同到中和戏园来,拣了一间厢楼,大家坐下。

看那戏目时,只见排着水仙花的《翠屏山》,金秀山、朱素云的《飞虎山》,龚处的《目莲救母》,王俊卿的《三岔口》,谭鑫培的《文昭关》。只有这几个人都是狠负时望的,那以前的几出配戏,都是些无名小卒,不必说他。

一连唱过了三出配戏,方才是金秀山、朱素云的《飞虎山》上场。金秀山起李克用,朱素云起李成孝,两个人唱得工力悉敌。那朱素云的喉音高亮非常,声声合拍。不比上海的那班唱小生的唱起《飞虎山》来,不是喉音太高,和老生不相上下,便是腔调太低,像了文小生和花旦。秋谷听了不觉击节道:“这才算得是武小生的正宗,果然名不虚传。”龚处的《目莲救母》也唱得淋漓顿挫,沉郁得神。水仙花的《翠屏山》虽然唱工做工都还不差,无奈年纪大了些儿,台容未免差些。王俊卿的《三岔口》也做得翻腾跌扑,色色到家。

临了儿,方才是小叫天的《文昭关》。出场的时候,大家先轰雷一般的喝了一声彩。这个小叫天,是中国伶人里头天字第一号的人物,自然的台步气概比别人来得不同。等得唱到“一轮明月”一段的时候,除了场上胡琴鼓板的声音,那楼上楼下挤得水泄不通的看客,大家都敛息宁神,侧耳细听,偌大的一个戏场竟没有一些儿声息,就是丢掉一根绣花针的声音也听得出来。秋谷也跟着众人侧着耳朵,一字一句的细细听去。只觉得叫天儿的喉音高低上下,圆转如意,他自己要怎么样便是怎么样,声韵圆活,音节沉雄,一字数顿,一顿数转,却又并不依着一定的节拍。

有的地方本来没有摇板的,他随意添上几板;有的地方本来是有摇板的,他却蓦然截住,凭着自己的意思翻来倒去。凭你唱到那极生极涩的地方,他却随随便便的一转便转了过来,不费一些儿气力,真个是清庙明堂之乐、黄钟大吕之音。又好象天马行空,飞行绝迹,凡间的羁勒,那里收得住他?秋谷听了,由不得也跟着众人喝起采来。姚观察等也大家啧啧称赏,多说叫天儿是曲中神品,别人唱起来那里有他这样雍容大雅、裂石穿云?

等到叫天儿的《文昭关》唱完,已经差不多有六点多种。姚观察便邀众人一直到小兰那里去。到了那里,小兰同着小菊都接出来,小兰便请众人到他房里坐下。

众人进去看时,只见是一间大大的屋子,隔作一横两竖的三间。靠东首的一间是小兰的卧房,外面两间做了客座。壁上挂着许多条对,都是些大人先生的亲笔。屋中陈列着许多古玩,湘帘宰地,冰簟当凤,花气融融,篆香袅袅,别有一种潇洒的样儿。房屋中间放着个大大的玻璃冰桶,冰桶里头浸着许多莲子和菱藕。章秋谷同着姚观察等刚刚从戏园里头出来,虽然北边天气,六月里头不见得十分炎热,那稠人广众的地方未免总有些汗香人气,大家心上都觉得有些烦躁。一到了这个地方,恍如到了清凉世界的一般。更兼小兰和小菊,亲自把冰桶里头剥现成的莲子取了许多出来,放在白磁盘子里头,请众人大家随意吃些,真个是凉溅齿牙,芳回肺腑。秋谷笑道:“怪不得如今那些大人先生,成天的爱在相公堂子里头混闹。这般的地方委实是天上琼楼,人间瑶岛。”正是:

珠喉玉貌,云郎之风格何如?雪藕调冰,公子之豪情未已。

欲知后事如何,请听下回分解。





第一百五十四回 吃大菜安心寻绮梦 走歧途着意访名姝





且说姚观察在小兰那里请客,相公堂子里头的菜本来是京城有名的,那些时鲜莱蔬,都是别处没有的。什么春不老炒冬笋,豌豆苗炒虾仁,都是在新鲜的时候藏在地窖里头的,到了这个时候还像鲜的一般,大家吃了都极口赞叹。这一席酒,差不多直吃到十二点钟方才散席。

到了明天,秋谷要去窑子里头逛逛,便拉着姚观察一同去走了几家班子。虽然也有两家南班,却都是些扬州人,满口的扬州白,一个苏州人都没有,北班更不必说他。秋谷同着姚观察走了半天,没有一个合意的,觉得十分败兴。秋谷便问姚观察道:“我听见人说,京城里头的大餐馆有几家简直是男女的台基,并且有外路人去的。他还可以和你拉皮条,甚而至于富贵人家的内眷都会被他们引诱出来。这句话儿不知究竟怎么样?如若当真有这样的事情,我却狠想要来试他一下。这个顽意儿却不知道你顽过没有?”

姚观察笑道:“我也听见人说过这条事情,我自己却没有顽过,不知这里头是怎样的一回事情。大约没有去过的人,须要一个熟人介绍,方才可以成事。如若不然,他摸不着我们是个何等样人,恐怕万一闹出乱子来。所以没有去过的人,没有熟人同去是办不到的。若是你一定要去,我却不能奉陪。我们一班相识的人里头,只有郑兰生在这里头最熟,你就和他同去何如?”

秋谷听了大喜,立逼着姚观察一同到郑侍御公馆里头去拜他。郑侍御恰好在家,出来相见,姚观察便把秋谷的来意说了一遍,郑侍御笑着一口应允。章秋谷见郑侍御一口答应,一刻也等不及的就要逼着郑侍御立刻同去。郑侍御也无可无不可的,套起车来,同着章秋谷一同前去。姚观察要去见识见识,便也同着郑侍御等坐车同去。

到了东交民巷左首的一家番菜馆门首,骡车停了下来,三个人下车走进。看那门外的商标时,只见写着大大的“凤苑春”三个黑字。极大的一座三层高楼,甚是宽敞。三个人直到第三层楼上,拣了一个大房间坐下。那侍者是认得郑侍御的,笑嘻嘻的送上茶来,口中说道:“郑都老爷,今天是不是照顾小店的生意?”郑侍御点一点头,对着他把三个指头伸了一伸。侍者便答应了一声“是”,回过身来就跑了出去。

秋谷问郑侍御:“这是什么暗号?”郑侍御道:“也算不得什么暗号,他来问我们是不是照顾他的生意,就是问我们要叫人不要叫人。若是要叫人的,只要向他点一点头,要叫几个,就伸几个指头。他见了心上自然明白。”秋谷道:“譬如我们一个人叫两个,可行不行?”郑侍御道:“一个人叫两个可不行。一个人只能叫一个,并且是无从挑选的,只好看各人的运气。叫来的人也有好的,也有不好的。

若是你的运气好些,或者叫得着一个好的也不可知。“秋谷道:”譬如叫来的人我们看不中意,便怎么样呢?“

郑侍御摇手道:“你不要看得这般容易。你要知道,这班出来的宝贝,大半都是达官贵人的姬妾出来找些野食吃的,并不是做生意的妓女。见了男子,先要他自己看中了这个男子,方才肯和他款洽;若是他看不中意,略坐一坐起身便走,休想留得住他。所以这个看得中看不中的问题,男人是没有主权的。你看中了他,他看不中你,依旧还是枉然。你还当作和上海的妓女一个样儿么?”秋谷呆了一呆道:“照如此的说起来,我们这个钱花他做什么,那有出了银钱在外面寻开心的大爷们,倒反要受他们鉴赏的道理?”郑侍御道:“那十两银子是给番菜馆里头的,你当是给那女人的么?这班宝贝也是和我们一般的出来寻个开心,非但一个大钱不要,并且还要格外拿出钱来赏给这些菜馆的人。甚而至于有男子和他合式的,只要老着脸皮卑躬屈节的拍他的马屁,一般也肯整千整万的银子拿出来倒贴男人,也不算什么事情。甚而至于靠着这条门路升官发财的,也不知多少。若是老老实实的说穿了,这个顽意儿就叫做女人倒嫖男子。不过好好的人,虽然做这个顽意儿的狠多,却不肯拿他们的钱,比那做妓女的究竟有些分别就是了。”

秋谷听了想了一回,忽然说道:“不好,不好!万一个运气不好,撞着了个奇形怪状、丑到极处的人,我们看不中他,他倒看中了我们,强要和我们如此如彼起来,这便怎么样呢?”郑侍御狂笑道:“这是我也保不定的。若果然有这样的事情,逃又逃不脱,推又推不掉。最怕的你不肯应酬他,他却老羞成怒,翻起脸来,只说你调戏他,那可不是顽的。也只得咬着牙齿应酬他一次的了。”

姚观察听了他们两人的话,不由的也笑起来,一面对章秋谷道:“据我看来,大凡这班宝贝,都是些放诞风流的人物,一定都有几分姿色,不过有个高下之分罢了。若果然是丑到极处的人,他自己也一定知道知难而退,那里再出来做这样的事情!”章秋谷笑道:“你的话虽然不错,却也有那些不顾廉耻的男子情愿交结个嫫母、无盐,只要想那女人的财物。如今世上这般的人也狠多。”说着,侍者已经送上来。大家听着,一面谈心,直吃到第四样菜,还没有什么人来,秋谷十分焦躁。

正在这个时候,忽然间门帘一起,走进一个少年女子来。走进门内便立定了脚,抬起秋波四围飞了一转,眼波莹莹飞到秋谷身上,不觉钉了秋谷一眼。回转身来,一言不发,走到壁间着衣镜面前照了一照。接着门外弓鞋琐碎的声音,又走进两个少年女子。三个人一色的都穿著闪光纱衫、蝉翼纱裙,脚下都穿著夹纱衬金纸的平底弓鞋,头上都挽着时新苏州式的玲珑云髻。一般的都是长条身材,削肩细腰,华彩飞扬,丰神流丽。看着这三个女子的模样,好似嫡亲姊妹的一般,螓首蛾眉,横波巧笑。只有那先进来的身材略略长些,月挂双眉,霞蒸两靥,觉得比后来的两个还要胜些。那两个女子走进门来,也和那先进来的一般,四围一看,也是一言不发。

这个时候,姚观察等三个人都立起身来,章秋谷便走到那先进来女子的身后,口中只说一声“请坐”,那女子听了,漠然不答,却在镜中微微一笑。秋谷也在镜中和他飞了一个眼风。那女子不由得回过头来看了秋谷一眼。秋谷趁势伸过手去,握着他的纤手,口中说道:“请那边坐罢。”那女子听了也不开口,却软软的被秋谷拉着走了过来,竟和秋谷并肩坐下。姚观察和郑侍御一个人搀了一个,相将坐下。

秋谷亲自取过酒瓶,斟了一杯薄荷酒双手送过去。那女子伸出手来,把一杯酒接了过去慢慢的吃了半杯,却仍把这个酒杯放在秋谷面前,也不开口。秋谷会意,举起酒杯来一饮而尽,把杯子对面照了一照。那女子似笑非笑的瞅着章秋谷,略略把樱唇动了一动。秋谷眉飞目舞,得意非常,握着那女子的手低低说道:“今天我姓章的不料竟有这般的奇福,遇着这样的佳人,也不知是那一世里修得来的。”那女子听了章秋谷这样的恭维他,免不得开颜一笑,脉脉含情,却依旧还是一个不开口。

姚观察和郑侍御也千方百计的想着法儿要想那两个女子开口说话。无奈这两个宝贝也是和那先来的一般,只是低头敛手的坐着,默默无言。

秋谷见他们三个凭你怎样的引逗,总是一个无声无臭,好象是个哑子的一般,便对着他们三个人说道:“今天你们三位为什么总不肯开口讲话?难道是我们得罪了你们三位么?”那三个人听了,只当没有听见的一般。秋谷又道:“你们三位这样的天仙化人,我们三个自然配不上和你们讲话。但是你们三位既然赏光下降,没奈何也只好委屈些儿的了。”那两个女子听了,只抬起头来看了秋谷一眼。那先来的女子轻轻的推了秋谷一把,低声说道:“有话等一回儿再说,这个时候性急什么?”

秋谷得了这几句话儿,心中大喜,一连答应了几声“是是是是”。一面说着,两个人的眼睛就如流星闪电的一般,大宽转的飞来飞去,那眼角眉梢之上,大家都含着无限的深情,一时间说不出来。正是:

为有前宵之梦,明月怀中;未妨昨夜之风,珍珠掌上。

不知后事如何,请待下回分解。





第一百五十五回 访天台三士入桃源 定花榜群芳登上第





且说章秋谷同着姚观察、郑侍御,想要挑逗那三个女子讲话。那知他们三个人都是缄口无言,止有那先进来的女子,开口说了一句话儿。却大家都是眉来眼去的,你看着我,我看着你,几乎大家都看出了神,三对男女,都是默默的一言不发。只见这一个流光眇视,那一个笑靥回春;这一个忽然慢展双眉,那一个又是暗抬俊眼。

一个个都是心期缱绻,眉语惺忪。一室之中静悄悄的,没有一些儿声息。那一种幽欢密爱的情形煞是好看。就是赵子昂、仇十洲著名善画的人,也描摹不出这般缠绵熨贴的情形,况在下做书的一支拙笔,那里描写得尽?

闲话休提。只说章秋谷和那先进来的女子彼此对看了一回,那女子忽然立起身来,看了秋谷一眼,眼光中间打了一个照会,回身便走。章秋谷紧紧的跟在后边。

那女子走到那壁间嵌着一面大着衣镜的地方,蓦地里把纤手在镜旁一按,不知怎样的“呀”的一声,那一面镜子忽然不见,露出一个小小的门来。那女子推门进去,章秋谷也跟着进去。好一会的工夫,方才双双的携手出来。只见姚观察和郑侍御依然坐在那里,那两个女子正在对着壁间的着衣镜顾影徘徊,眉敛湘烟,脸生春色,衣裳不整,云鬓惺忪。见了那女子出来,由不得大家相视一笑。

三个人略略的整了一整衣裳,掠了一掠鬓发,大家都立起身来有个要走的意思。

秋谷连忙走上去,附着他的耳朵说了几句话。只见那女子蛾眉一皱,神色黯然,看着秋谷好象有什么话要说的一般,却又默然不语。停了一停,方才低低的和秋谷说了一句:“改天再见。”说着,在手上脱下一个宝石指环来,套在秋谷手上。秋谷也连忙把表链上挂的一个外国金相合解了下来,递在他的手内。那两个女子见了,也照样脱下一个戒指,放在姚观察和郑侍御手中。姚观察解下一个翡翠扇坠,郑侍御随身没有什么东西,只得在衣袋内取出一个金表来。大家交换,总算是个表纪的意思。大家对面相视,都有些依依惜别的情肠。三个女子延挨了一回,只听得门上轻轻弹指的声响,三个人只得回身便走。那先进来的一个走到门口,又回头过来把手举了一举,又把头摇了一摇。秋谷心上明白他的意思,连忙打个手势,点一点头。

眼睁睁的看着他们三个走了出去,好似做了一场好梦的一般。姚观察忽然笑道:“这三个人倒狠有些意思。”郑侍御道:“这三个人真是嫡亲姊妹,可惜不知道他们的姓名。”姚观察道:“这些人为什么不能问他的姓氏?究竟是个什么道理?”

郑侍御道:“他们这班人最忌的是有人问他的姓名,好象有心要查考他的根脚一般。

也无非讳莫如深,怕人张扬出去的意思。“只有章秋谷只是微微的笑,不说什么。

郑侍御道:“今天这个媒人是我和你做的,你该应怎样的谢我?”秋谷笑道:“我和你当个侦探,就算是大媒的谢仪,可好不好?”郑侍御诧异道:“什么侦探?

难道你竟知道了他们的姓名么?“秋谷笑着走过去,附着郑侍御的耳朵说了几句,又转身和姚观察附耳说了几句,两个人都不觉把舌头伸了一伸。姚观察想了一想道:”既然是这个话儿,三个人都好好的嫁有夫家,为什么要出来这般胡闹?“秋谷笑道:”大约是当差不力的缘故。“姚观察和郑侍御都笑起来。

大家坐了一刻,吃过咖啡,那先前的侍者拿着一纸帐单从外面走了进来,把帐单放在桌上,满面添花的躬身侍立。秋谷和姚观察都取出十两银子的银票来,交给侍者。侍者接过来,谢了一声。郑侍御也付了一张银票。大家出了凤苑春,各自回去。章秋谷回到武阳会馆。

过了几天,金观察殿试已毕,取了个二等第二。陛见谢恩下来,却没有什么好处,只在候补道上加了个军机处存记。一班应试的人都大失所望,金观察倒随随便便的,没有什么。拜过了几个阅卷老师,便收拾行李,同着章秋谷一同出京,回到天津来。

恰恰的金星精金部郎要到天津避暑,便也告了个病假,同着秋谷和金观察一起出京,也住在金观察公馆里头。秋谷同着他出去顽了几天,金部郎看中了一个宝华班里头上海新到的小洪宝宝,又看中了一个富贵班的桂珠。那小洪宝宝生得清丽非常,丰神绝俗,有王夫人林下之凤。那桂珠生得丰肩腻体,素口蛮腰,有袁宝儿娇憨之态。金部郎做了这两个人,一连吃了好几台酒,不知不觉的一连就是几天。

这个时候,方制军把金观察委了个北洋大学堂总办,那洋务局总办的事情,依然还是金观察兼理。依着金观察的意思,要请章秋谷当北洋大学堂的总教习,兼办洋务局文案的事儿。章秋谷再三推却,要想告辞回去。金观察那里肯放,再四挽留。

章秋谷只说要回去省亲,又要回南乡试。金观察听他说到省亲、乡试的两层题目,知道挽留不来,心上却十分惆怅,只得再三约他闱后再来,切勿失信。秋谷只得答应,定了七月初十搭招商局安平轮船回去。

算起来,到初十还有四五天,金观察便和金部郎商议,要趁着七月初七这一天牛女渡河的良夜,在宝华班替秋谷饯行。商议定了,金观察和金部郎便走到秋谷房间里头和他说知。两个人刚刚跨进房门,只见秋谷正坐在那里,低着头振笔疾书,不知写些什么,连他们两个人走进都不知道。金观察便笑道:“你在这里写些什么,写得这样认真?”秋谷听了,连忙搁了笔,立起身来含笑相迎。金观察走近看时,只见案上铺着一张大大的柳絮笺,写着一纸的草书,写得兔起鹘落,满纸淋漓。金观察和金部郎走过来定睛看时,只见第一行写着“津门南榜”四个大字,下面又注着“扬人不录”的四个字儿。

金观察道:“这是你定的花榜么?你倒居然还有这样的心情,来弄这些笔墨。

想来是专取那些南班里头的人,所以叫做南榜。但是天津地方本地人也尽有几个好的,不可一概抹煞。就是那些南班子里头的人,扬州人也有几个狠好的在里头,苏州、上海人却不多几个。你既然取名南榜,怎么又不取扬州人呢?况且南班子里头的人,扬州人差不多十居七八,苏州、上海人却不过十之二三。你要专取苏州、上海人,那里找得出许多?“秋谷道:”那班北班子里的人,虽然也有面目清秀些的,却眉目之间总带着一股犷气。南班子里的扬州人,虽然狠有几个面貌不差的,却神色之间总带着一股贱气。那里比得上苏州、上海人,一举一动别有一种温柔软媚的神情。所以小侄拣选花榜人才,非但北人不录,连扬州人也是一个不取。“秋谷说到这里,金部郎拍手道:”你的话儿一些不错,平日间我的意思也是这般。古来那班诗人名士,一个个都夸说扬州佳丽,真是徒有虚名,毫无实际,那里当得起‘佳丽’的两个字儿!“

金观察听了不由得点了一点头,就在秋谷坐的那张椅子上坐了下去。仔仔细细的看那花榜时,只见上面写着:

第一甲第一名小洪宝宝。

评曰:

花输旎旖,雪逊温柔,姽婳无双,丰神第一。西子捧心之态,秋敛青蛾;太真红玉之肤,香融宝靥。

诗曰:

小立风前斗晚妆,松松云髻薄罗裳。梅花清瘦桃花俗,合让姚黄压众芳。

第一甲第二名云兰。

评曰:

神彩惊鸿,佩环回雪,金莲贴地,玉笋凌波。皎如琼树之流光,灼若芙蕖之照夜。

诗曰:

心上烧香掌上怜,丽娟肤发丽华年。

倾城一笑真无赛,疑是瑶台月下仙。

第一甲第三名金兰。

评曰:

镂玉为肌,团琼作骨,山眉水眼,皓齿明眸。正当二九之年,恰称芳菲之选。

诗曰:

为有春情透脸霞,东风无力舞腰斜。

夜深独背银釭坐,自弄钗头茉莉花。

第二甲第一名桂珠。

评曰:

素面纤腰,丰容盛剪,秋月乍满,奇花初胎。歌喉遏巫峡之云,皓腕比蓝田之玉。

诗曰:

碧玉丰神绛雪肤,凤情天付有谁如?

歌喉宛转谁堪拟?百八牟尼一串珠。

第二甲第二名月香。

金观察看到这个地方,见底下没有了,便又翻过来看了一遍,道:“你的笔墨实在松秀得狠。若要叫我如今再弄这些笔墨,是再也弄不来的了。”金部郎倚在案头,金观察看的时候,也早已看得明白,便对秋谷道:“你自己的相好怎么不取作第一,倒把别人的相好取作状元,这是个什么道理?”秋谷道:“品评花榜,是不能心上有一毫私见的,要大家看了,一个个都点头心服,方才算得平允,不是可以把一个人的爱憎作众人的爱憎的。”正是:

秾桃艳李,春风联玉笋之班;大道青楼,旭日照金泥之榜。

不知来怎样,请看下文,便知分晓。





第一百五十六回 饯长亭良朋悲远别 脱火坑名士作冰人





且说章秋谷把小洪宝宝定作花榜的状元,金部郎心上自是欢喜,却故意对章秋谷说道:“据我看起来,云兰和小洪宝宝也不相上下。云兰的姿貌也狠不差,为什么你一定要把小洪宝宝拔居云兰之上?”秋谷笑道:“老实和你说罢,如今的人凭你怎样,心上便总有一个私心,那里能当真大公无我的没有一些儿私曲的地方?云兰是我的相好,那里有不回护他些的道理?无奈这个云兰和小洪宝宝两个人比较起来,一边是虽多婀娜之姿,略欠清扬之态;一边是既具纤秾之致,兼饶林下之风。

这般的赏鉴,却不是粗心人可以领略得来的。因此没奈何,把小洪宝宝取了第一。

若是在你未来之前,这个小洪宝宝就预先到了天津,我也早已收罗在我的门下,那里还轮得到你?“金部郎听了,便也笑了一笑,不说什么。

金观察便问秋谷道:“你既然不取北方人和扬州人,苏州、上海人那里有这许多?”秋谷道:“取在榜上的,原不过二十个人的模样。宁缺毋滥,只好凭他少几个人的了。”金观察和金部郎又把那几首诗读了一遍,金观察道:“你的笔墨果然绮丽非常,做这样的香奁艳体,刚刚合你的笔路。”秋谷谦逊道:“这些笔墨已经抛弃多时。三日不弹,手生荆棘,如今再要提起笔来就觉得十分生涩。这里头未免有不妥当的地方,还要请老表伯指正才是,怎么老表伯先自这般的谬赞起来?”金观察呵呵的笑道:“我们自己人,你还和我客气么?”秋谷也不觉微微一笑。

金观察和金部郎坐了一刻,把明天饯行的事和秋谷说了。秋谷随口谢了一声道:“明天老表伯和星精兄赐饮,断断不敢不到。”金观察道:“你还是这般客气,索性具个手本上来禀谢何如?”说笑了一回,金观察和金部郎走了。

章秋谷又坐下来,把那张没有写完的花榜一挥而就。一共只取了二甲十名,三甲五名,连着三鼎甲,只得十八个人。把月芳取了个二甲第四。二甲里头,取了林湘君、林妃君、卓文君、李香玉等。又把桂红、小芬等几个人勉强取了个三甲。立刻找了一纸冷金笺,半真半草的誊写出来,预备明天带到宝华班去。又把草稿送到津沽风月报馆里头,请他登报。

到了明天,已经是七月初七,天上佳期,人间良夜,银河无浪,乌鹊不惊,盈盈一水之波,脉脉双星之恨。金观察料理了日间应行的公事,急急的回到公馆里头来,邀了金部郎和章秋谷同到宝华班去。又到别处去请七八个客人,主客一共十一个人,在金兰房间里头摆了一个双台,算是金观察和金部郎两个的主人。一台是金兰的,一台却算是小洪宝宝的。依着小洪宝宝的意思,原想要叫金部郎不要和金观察混在一处,这一台酒就在他自己房间里头吃的。无奈今天的酒是金观察和金部郎两个人合在一起和秋谷饯行的,章秋谷一个人不能分作两个,金部郎便和小洪宝宝商议叫他将就些儿,这一台酒就摆在金兰房间里头,也是一样的。小洪宝宝便也答应。金部郎又把章秋谷把他取做状元的事情和小洪宝宝说了,小洪宝宝只说是金部郎有意哄他,不肯相信。金部郎道:“你不信,我把花榜给你看。”说着便回过头来,要问章秋谷要那一张花榜。

不想章秋谷不在房中,到月芳那里去了。金部郎便走到月芳房间里去,向他要时,只见云兰、月香两个人都在月芳房内,大家正在看那花榜。秋谷站在那里,指指点点的在那里解说给他们听。金部郎等他们看过之后,便拿着那张花榜走到小洪宝宝那边来。章秋谷同着云兰、月香、月芳也跟着过来。小洪宝宝本来认得几个字的,看了那张花榜上的字儿,一甲一名,果然是他自己的名字。金部郎又把那几句评语和一首七绝的意思,细细的和他讲解一遍。小洪宝宝不觉心中大喜,杏靥春回,樱唇红绽,对着章秋谷笑道:“谢谢耐,像煞说得忒嫌好仔点哉。”秋谷也笑道:“我是向来不会拍马屁的,好的就说好,不好的就说不好,你又何必和我客气?”

章秋谷说到这里,云兰和月芳两个都瞟了秋谷一眼。秋谷见了,心上自是明白,却只当没有看见的一般。不多一刻,金观察叫金兰过来,请秋谷入席。秋谷便同着金部郎一同过去,小洪宝宝和云兰等也随后跟来。

那些班子里头的倌人听说章秋谷定了个花榜,只说自己一定在花榜里头,大家争着拥到金兰房里头来看。连着那个女本家也走进房来,见了众人一一的招呼过来。

金观察便对他笑道:“恭喜!恭喜!这位章老爷定的花榜,状元、榜眼、探花,都出在你们一个班子里头。这个风声传扬开去,你们这个班子一定要发大财。”那女本家听得三鼎甲都是他家班子里头的人,心上自然欢喜,随口谢了秋谷,便回身退出。还有几个班子里头的苏州倌人,大家拉着金观察,要金观察把花榜上的名字,一个一个的都念出来给他们听。金观察只得依着他们念了一遍。有几个榜上有名的自然高兴,有几个落第的就不免要暗中把章秋谷咒骂几句。更有那班扬州人,听说凡是扬州帮的倌人一概没有名字,更是恨得咬牙切齿,气愤非常,背地里也不知把个章秋谷骂了多少。

只说章秋谷坐在席上,看着云兰的神色倒还没有什么,只有月芳坐在那里闷闷的一言不发。秋谷知道他的意思,咬着耳朵敷衍了他几句,只说本来要把他取作第三名探花的,不知怎么样,一时错误,竟取了个二甲第四。月芳听了,只微微的笑道:“像倪实梗格别脚倌人,陆里挨得着啥格探花!倒是归格辰光,倪搭耐说格闲话,耐阿记得?”秋谷听了,猛然提起一件心事来,暗想以前曾经亲口许他,一定要想个法儿把他提出火坑的,如今自己的归期在即,一时那里想得出什么法儿?低着个头想了一回,由不得为难起来。

正在这个时候,忽然觉得有人在后面拉他一把。秋谷回过头去看时,只见云兰坐在后面,附着他耳朵低低问道:“阿是耐真格要转去?慢慢交末哉呀?啥格实梗要紧?”秋谷对他说道:“我有正经事情,不能不回去。初十一准要走的。”云兰听了,登时蹙着双蛾,黯然不乐,低下头拉着秋谷的手揉搓一会,默默无言。停了好一回,方才抬起头来说道:“格末耐去仔,阿要几时来呀?”秋谷道:“自然就要来的。金大人再三再四的一定要我来。金大人的面上,不来觉得不好意思。”云兰道:“格末几时来呀?阿是真格呀?”秋谷道:“自然是真的。回去不过一个多月的勾留,大约八月底九月初就可以到这里的了。”云兰听了,把一个粉面偎在秋谷肩上,道:“格是倪到仔九月里向,等耐格嘘。”说了这一句顿了一顿,眼圈儿已经红了。

秋谷见了这般模样,倒不觉心上有些跳动起来。名士多情,佳人难得,杨柳长亭之路,将离南浦之思,两个人四目相视,狠觉得有些依依不舍的心情。云兰见秋谷脸上呆呆的,露出十分惆怅的样儿,更觉得别绪满怀,泪珠欲滴。月芳也附着秋谷耳朵低声说道:“耐阿好勿要去哉!耐去仔,叫倪那哼呀?谢谢耐,搭倪想想法子。”

秋谷听了,便伸出手来,左手挽住了月芳,右手拉住了云兰,这边看看,那边看看。看了一回,忽然别转头去叹一口气,把双手一齐放下,立起身来拉着金观察到榻上坐下,和他商量月芳的事情。把月芳如何的情愿从良,自己又如何的情愿帮他的忙,一一说了一遍,要把这件事情转托金观察。

金观察听了,矍然道:“你不说我几乎忘了,恰好有一个凑巧的机会在此。孙英玉去年断了弦,不愿意再娶正室,想要娶一个姨太太操持家政,就是堂子里头出身的人也不妨,只要一心一意肯嫁他,他也没有什么不愿意。和我说了几遍,要托我替他做个媒人。如今既然月芳情愿从良,我看月芳这个人狠有些厌倦凤尘的意思,倒也不是个娶不得的人。孙英玉娶了他回去,一定可以彼此相安,不至于闹什么笑话。好在英玉今天也在这里,待我去把他叫过来问他一下,看他愿意不愿意。”

说着,便走过去把那位孙英玉叫了过来,把这件事儿和他说了一遍。孙英玉十分欢喜,一口应承。秋谷见孙英玉已经答应,便又回转身来和月芳咬了几句耳朵。

月芳呆了一呆,还没有开口,秋谷又低声对他说道:“这个人是狠靠得住的,虽然功名小些,是个直隶候补县丞,却上司都狠剪他得起。年纪也只得四十一岁,不算狠大,面貌也平平正正的,不是什么麻胡黑丑的尊容。你自己看就是了。”说着,便把孙英玉指了一指。月芳便回过头来,把孙英玉着着实实的看了两眼,便对着秋谷一笑,不说什么。

秋谷知道他心上已经许可,便一手拉着月芳,直拉到孙英玉面前,把月芳的手一直送到孙英玉的手内,口中说道:“你们两个人都是自家情愿的了,有什么话,你们两个人自己讲罢。”月芳红着个脸,半推半就的竟在孙英玉身旁坐了下来。

孙英玉看着月芳,虽然年纪大些,却还着实有些丰采,喜得笑嘻嘻的,看着月芳一时倒说不出什么话来。停了好一会,方才开口问问月芳的出身家世,月芳一一的回答,也问了孙英玉几句。两个人登时低声促膝的谈心起来。章秋谷和金观察见了他们两个人这般情景,便故意回到席上去应酬一会,好让他们两个人细细的谈心。

正是:

风尘沦落,谁怜多病之徐娘;湖海飘零,讵有黄衫之侠客?

未知以后如何,且待下文分解。





第一百五十七回 解腰缠豪情成义举 翻醋翁冷语试深心





且说席上的那班客人见章秋谷和金观察低声谈了一回,又把个孙英玉拉了过去,不知道讲些什么。言主政便问道:“你们这几个人,鬼头鬼脑的究竟说些什么?”

秋谷听了,便对着大众,把月芳想要从良的事儿,约略说了一遍。大家听了,都说月芳的主意不差。

秋谷虽坐在席上,却时时刻刻的留意剪着孙英玉和月芳两个人的情形。只见他们两个人谈了一回,孙英玉忽然皱着眉头沉吟起来。秋谷见了,连忙拉着金观察出席问他。孙英玉对着他们说道:“方才据月芳说起来,身上现有一千多块钱的亏空,还有些零碎帐目,差不多要一千四百块钱,合起来要一千银子方才可以还清各债。

不瞒金大人和秋谷先生说,我的家计原不见得十分宽裕,竭力拚凑起来,五六百银子是拚凑得出的,还有四百银子却叫我到那里去设法呢?看起来,这件事儿只好暂时从缓的了。“

秋谷听了还没有开口,月芳听了心上却甚是着急,两只眼睛水汪汪的只看着秋谷,却说不出什么话来。秋谷慨然对金观察道:“据小侄看起来,这件事情总算是成人之美,何不大家帮他个忙,也是一件狠好的事情。”金观察听了欣然说道:“你的话狠不错,我就帮他五十两银子,其余或者和他同乡里头告一个帮,料想大家也都是乐于成全的。”秋谷道:“既然如此,我也帮他五十两银子。有了这两笔一百两银子,还差三百两,只好请老表伯和他告一个帮的了。凭着老表伯的面子,这几个钱料想不难。”金观察听了,点一点头。

席间的几个客人,除了孙英玉之外,还有七个人,只有一个是山东人,其余的六个都是江苏的同乡。观察把告帮的意思和他们说了,大家一口许诺,也有三十两的,也有二十两十两的,登时凑了一百四十两银子。金部郎也出了三十两。那位山东人候补知府戚太守,却是个山东有名的富室,见大家解囊倾助,便也欣然帮了五十两,一共有了三百二十两。尚少八十两银子,凑不满四百两的数儿。章秋谷慨然道:“这件事儿是我发起的,如今功亏一篑,我自然该应竭力成全,所少的八十两银子,算我一个人的就是了。”金观察道:“这件事情是我们两人发起的,怎么好叫你一个人出?我们两个人一个人出一半就是了。”众人听了,大家都说章秋谷和金观察两个人轻财仗义,慷慨非常。金观察和章秋谷不免也谦逊几句。

孙英玉看了众人这样的成全,心上万分感激。便抢步过来,对着众人一个人打一个拱,口中说道:“我孙英玉蒙诸位这般的格外周全,感铭肺腑,却叫我将来怎样的报答得来?古人说的,‘大恩不谢’,我也只好把这件事儿长长的放在心上了。”

众人都说:“这般小事,何足挂齿?”章秋谷却含笑对他说道:“你老哥不必打拱作揖的和我们客气,只要你们两个人将来地久天长,一双两好,就不枉我们几个人的这番举动了。”大家听了,一个个都点头称是。孙英玉听了,更诺诺连声的答应不迭。月芳在旁听着,见章秋谷这样的和他尽力,心上真是感激到二十四分;感激到极处,却又不由得落下泪来。只见他慢慢的立起身来走到席前,立定了脚,口中朗朗的说道:“今朝格事体,区得唔笃几位大人老爷,大家才肯搭倪帮忙。倪也呒啥别样,只好多磕两个头,谢谢唔笃几位大人老爷格哉。”大家听得他要叩头,连忙向他摇手,叫他不要多礼。月芳那里肯听,不由分说,插烛也似的跪下地去。众人回礼不及,只得大家立起身来,背过脸去。

月芳拜了四拜,方才起来。一眼看见章秋谷站在那里呆呆的望着他,不知不觉的想起那以前的情款,不由的心上有些凄恋起来。想着今天这件事儿,多亏他一个人竭力周全,方能成事。如今世上居然也还有这样的人。若是换了第二个人,听得自己的相好倌人想要嫁人,不吃醋已经够了,那里还肯这般出力?可惜事机不凑,不能嫁他。若是嫁着了这样一个人,好算得心满意足的了。如今嫁了这个姓孙的,虽然一个愿娶,一个愿嫁,没有什么不合意的地方,但是摆着秋谷这样的一个风流年少,自己却没有福气嫁他,心上未免总觉得有些不足。想到这里,便也对着章秋谷呆呆的看,星眸斜睇,波光四流。

章秋谷眼快,早已看得甚是清楚。想着那往时的恩爱缠绵,看着这现在的神光离合,只觉得一个心七上八下的十分眷恋,无限凄怆。明知道这个时候已经算是孙英玉的人,不好再是怎样的和他亲热,恐怕孙英玉脸上下不来。便在身上掏出一张六十两银子的银票,递在月芳手内,口中说道:“我们两个人相识一场,大家总算狠要好的。你的事情,我也总算和你竭力周全,没有辱命。你的景况我是狠知道的,这几个钱,你拿去办些应用的东西,总算是我一点儿意思。从此以后,但愿你们两个人夫妇齐眉,白头偕老,我就没有什么记挂了。”月芳听了,起先还不肯接。秋谷低低的道:“我们两个人相识一场,这几个钱算得什么,你又何必和我客气?况且自此以后,你是孙府上的姨太太了,我又要回到上海去,知道我们两个人见面在什么时候?”

章秋谷说到这个地方,便顿住了口不说下去。月芳却再也忍不住,把头一低,那眼中的泪就如断线的珍珠一般乱滴下来,一面呜咽着一面说道:“耐实梗样式,叫倪心浪洛里意得过!”秋谷听了也觉得有些酸鼻,几乎也要滴下泪来。却恐怕别人见了要笑他,勉强忍住了,对月芳说道:“你们两个人天缘凑合,是一桩大大的喜事,怎么倒这样的伤心起来?”说罢又低低说道:“只要你嫁过去夫妻和睦,我也就放下了一条心。如今你这个样儿,我看了心上倒觉得十分难过。这也是注定的我们没有缘分,说他也是枉然。”月芳听了方才抬起头来拭了眼泪,握着秋谷的手道:“像煞倪有几几化化格闲话要搭耐说,故歇勿晓得那哼,一句才说勿出,耐自家保重点。”秋谷听了回答不出什么,只把头点了一点。硬着头皮回转身来,走到席上坐下。

那几个宝华班里的人──云兰、金兰和小洪宝宝,坐在席上都看得呆了。云兰停了一回,方才把秋谷拉了一把道:“耐格个大媒人,倒做得呒啥,总算月芳阿姊格运气。”说着,便向月芳道:“月芳阿姊,恭喜耐。实梗格喜事,要请倪吃喜酒格嘘!”小洪宝宝同着金兰等,也向月芳贺喜。月芳两颊微红,不免也要谦让几句。

小洪宝宝却向章秋谷道:“章二少真正是个好人,肯实梗格帮月芳阿姊格忙。客人里向像耐二少实梗格人,实头少格嘘!”秋谷为着做了这个媒人,把月芳提出火坑,心上却甚是得意,便多吃了几杯酒,脸上红红的有些酒意上来。金观察见席上众人的酒也吃得差不多了,便和众人打了一个通关,又敬了章秋谷几杯酒,大家都覆杯告止。

秋谷略略的吃些稀饭,便也立起身来。依着云兰,要秋谷今天住在院中。秋谷因多了几杯酒,觉得有些胸中作恶,便没有答应,只说回去还有些事情。云兰瞪了秋谷一眼道:“耐格人末,就叫讨气!”秋谷笑道:“并不是讨气,委实的还有事情。”云兰谷都着嘴,口中咕噜道:“啥格事体呀!耐格事体倪阿有啥勿晓得,豪燥点跑到相好格搭去,晏仔点是要吃生活格。”说着,便推着秋谷的背道:“豪燥点去嘘!格两日天就要动身哉,自然要到恩相好搭去辞辞行格嘛,阿对?”章秋谷听了笑道:“真正极天冤枉,我除了你们这里,那里别处还有什么相好?”云兰道:“啥人晓得耐呀!耐有相好呒拨相好末,也勿关得倪啥事嘛”说着,不觉双眉紧皱,俊眼微睁,狠狠的钉了秋谷一眼。秋谷见他娇嗔满面,情不自禁只得过去,携着他的手道:“你不要生气,你就是我的恩相好,那里再有别人。我就今天不走,在这里和你辞行何如?”云兰别转头去,口中说道:“啥人要耐辞行呀!耐豪燥点请出去,像倪实梗格别脚倌人,洛里好比别人?再要说起啥格恩相好勿恩相好,是真正枉空嘛!耐实梗一个章二少,倪阿配搭耐做啥格恩相好,也亵渎仔耐章二少格身分哉嘘!”

秋谷听了云兰的这几句话儿,觉得他话中有眼,明明是指着月芳说的。回心一想,把月芳和云兰两个比较起来,却委实的有些轩轾。在月芳身上的事情,便肯这样的和他出力。在云兰身上,他要挽留自己住在院中都不肯答应他。若要拿他们两个人的交情说起来,还是和云兰要好些儿,却也怪不得他要说这般的话儿。想到这里,便回头向月芳看时,只见月芳低着头,假做没有听见一般的,脸上却有些红红的不好意思。秋谷咳嗽一声,打个暗号。月芳回过头来,秋谷对着他使个眼色,月芳会意,便走了出去。

云兰见了,便也立起身来,冷笑一声道:“耐有啥闲话末说末哉。倪跑出去,让唔笃随便那哼说法。”说着向外便走。秋谷连忙一把拉住,在他耳边说道:“你不要这般生气,给人看了,还只说你是吃醋。你只要自己想一想,你的年纪还没有满二十岁,生意又是狠好的,比不得月芳已经三十多岁的人,又欠了一身的债,那里还做得起什么生意?如今和他成就了这段因缘,想起来你们同院姊妹该应可怜他些,替他喜欢才是,怎么你倒和他吃起醋来?”正是:

落花堕劫,飘零金谷之春;飞絮沾泥,惆怅灵和之柳。

不知云兰听了秋谷的话说些什么,且待下文交代。





第一百五十八回 逢醉鬼狭路动干戈 数前尘花丛谈掌故





且说云兰本来是一肚子的不高兴,如今听了章秋谷这样一番有情有理的话儿,倒觉得无言可答,心上的怒气倒平下了许多,对着秋谷横波一笑,口中说道:“耐个人末勿晓得缠到仔洛里去哉!月芳阿姊一径搭倪蛮要好格,啥人去搭俚吃醋呀!”

秋谷听了,知道这几句话儿无非是有心掩饰,好自己做一个落场,便也对他一笑。

又去咬着耳朵温存了好一回,云兰方才欢喜。这一夜,章秋谷自然不回去的了。连着金观察和金部郎两个,都给小洪宝宝同金兰挽留不放,住在院中。珍簟新铺,秋宵苦短,三对儿鸾交凤友,一时间雨殢云封,温存掌上之躯,宛转怀中之月。这些说话不关紧要,也不必去说他。

只说章秋谷从宝华班回来便收拾了一天行李,又出去辞了一天行。那招商局的安平轮船十一早上就要开的,秋谷一到初十,就把行李都发上船去。又有两三个同乡,在凤苑春和燕宾楼和他饯行。秋谷情不可却,每处都去坐了一坐,便连忙赶到宝华班来。原来金观察为着轮船一早开行,搭客至迟到晚上两三点钟一定要上船的,早早的跑上船去坐着,却又没有意思,便约着金部郎、孙英玉,连着秋谷四个人,在宝华班碰一场和,碰完了和上船去刚刚正好。秋谷赶到宝华班,金观察已经先在,谈了一回,便大家碰起和来。

云兰为着秋谷今天要走,未免有些依依惜别的心情,坐在那里呆呆的不甚开口。

月芳嫁人的事情,秋谷已经当面和本家说过,帐目都付清了,月芳便不肯再见客人。

但是章秋谷到来的时候,月芳却还依旧出来,敛袖低眉,淡妆素服,竟是个人家人的样儿。秋谷看着这般模样,觉得玉人依旧,咫尺天涯,狠觉有些惆怅。再三叫他不要出来,月芳那里肯听。只两下谈心的时候,大家都是面上淡淡的,不能够握手牵衣,偎肩接膝,像以前的那种样儿。今天月芳听得秋谷一定要走,自然心上也狠是酸辛,也是坐在秋谷背后,一言不发,只静静的看着他们碰和。等得八圈庄碰过,已经十二点钟,秋谷便也不免对着月芳、云兰说些告别的话儿。又拉着云兰坐在床上,咕咕唧唧的不知说了些什么。月香也走过来,对着秋谷说些套话。

不多一刻,已经听见自鸣钟“铮铮”的响了两声。秋谷立起身来要走,云兰和月芳再送到船上,秋谷再三阻拦,他们那里肯听,秋谷也只得由他。金观察和金部郎也一定要送秋谷到船上去,秋谷推却不得,只好听凭他们怎样。金观察和秋谷等本来都是轿子来的,秋谷忽然想起有一个清芬班里头的玉凤,曾经叫过他两个局,没有付钱,便叫轿夫把轿子搭在弄口去等,又叫云兰等略候一回。秋谷同着金观察等急急的到清芬堂去付过了钱,连忙出来再到宝华班去,会齐了云兰和月芳,叫他们坐轿在前先走。秋谷同金观察等慢慢的一步一步走出侯家后来。

那侯家后的地方,原是一条极窄的小弄,弄外便是新造的马路。秋谷等刚刚走出弄口,劈面撞见了一个同乡兵部主事严克任严主政。大家止步招呼,不想斜刺里有两个洋兵吃得烂??,七跌八撞的直撞过来;不左不右,不前不后,刚刚撞在那位严主政的身上。严主政还没有开口,不料那洋兵撞了严主政一下,顿时发起酒风来,一手扭住丁严主政的衣服,口中“钩辀格磔”的不知骂些什么;一手在腰间拔出小刀来,望着严主政肩窝便刺。严主政措手不及,大吃一惊,连忙把身体一侧,那把小刀正刺在严主政的嘴唇上面,直刺得唇开肉破,鲜血直涌出来,刀尖撞着门牙,连牙齿都撞缺了一个。严主政“阿呀”一声,要想回身走时,怎奈衣服被他拉住,脱不得身。

正在十分危急,早恼了那位章秋谷,一个箭步直抢过来,起左手臂开了他拉着衣服的手,右手轻轻一转,早把小刀抢在手中,左手顺势一送,那洋兵本来已经醉到十二分的了,那里经得起章秋谷的神力,早已踉跄直倒过去,扑的仰面一交。说时迟那时快,章秋谷正要看严主政的伤痕时,只觉得脑后一阵风直扑过来,也不回头去看,把身体“霍”的一扭,右脚往后一登,只听得“扑”的一声,那一个洋兵也是仰面一交。这个时候恰恰的没有巡警在那里,凭着他们去闹,没有人去问他。

金观察等却多替章秋谷捏一把汗,恐怕万一个闹出大交涉来不是顽的。章秋谷却并不放在心上,立在那里不动,只看着那两个洋兵。只说他一定还要起来混打,那里知道这两个洋兵醉到极处,心上那里还有什么知觉,一个人吃了章秋谷一交筋斗,睡在地上也不扒起身来,倒反口中“呜呜”的唱起歌来。

这个时候正是微雨初过,地下还有些泥泞,这两个洋兵满地乱滚,滚得浑身上下好象个泥母猪的一般。秋谷看了又气又笑,料想这两个醉猫是扒不起来的了,便回过头来看严主政的伤处。只见严主政把衣袖掩着嘴唇,那流出来的血连衣袖都湿透了。大家问他怎么样,严主政说:“还没有大伤,回到寓所去找些伤药敷一敷就不妨事的了。”说着,又向秋谷谢道:“今天幸而遇见了你们几位,和我解了这个围。如若不然,那就不堪设想了。”秋谷谦逊几句,只说这般小事,理应相助的。

一面说着,严主政已经叫了一辆人力车,叫到江苏会馆。秋谷等还要送他回去,严主政再三不要,谢了众人,上车自去。

秋谷又对金观察道:“这两个醉鬼躺在地上,虽然与我们不相干,但是这个地方又不见有巡警在那里,万一闹了个什么乱子出来,酿成交涉,老表伯当着洋务局的总办,这个责任是跑不掉的。不如叫几个巡警把他们送到领事衙门去,觉得妥当些儿。”金观察点头道:“你的话儿不差,闹出交涉来还是洋务局的干系。”说着左右一望,见就近竟没有一个巡警的影儿。便叫轿夫去叫了一名巡警来,对他说了这个缘故。那巡警垂着手,诺诺连声的答应。金观察吩咐过了,便同着大家坐上轿子,到紫竹林招商码头安平轮船上来。

到了船上,云兰和月芳已经坐在官舱里头等了好一回,问他们来迟的缘故,秋谷把路上遇着的这件事儿和他们说了一遍。云兰和月芳吐舌道:“阿要怕人势势,区得倪韵碰着俚,要叫倪碰着仔格号酒鬼格外国人,是魂也吓脱格哉!”秋谷同着众人,想着中国的这般衰弱,以致受侮外人,不由大家嗟叹一番。金观察见开船在即,究竟和秋谷相处了好几个月,平日之间又是狠合式的,心上自然怅惘非常,不免有几句分袂丁宁的话。云兰和月芳更是脉脉相看,凄然欲泣。秋谷到了这个时候,也觉得一腔别绪,满腹离愁。和金观察说几句,和云兰、月芳又说几句,只觉得心上许多衷曲,一时那里说得出来。无奈坐不多时,早已是曙色在天,残星无影,差不多已经有三点多钟。船上的那些水手大家喧嚷起来,急忙忙的起锚解缆,预备开船。云兰和月芳只得立起身来,对着秋谷说了句“一路平安”,懒懒的走上岸去。

金观察也对着秋谷说道:“但愿你秋凤第一,直上青云,我们良晤有期。前途珍重!”

说罢,便也同着众人一同登岸回去。

这一边章秋谷的事情且自按下不题。如今且再说起上海的事情来。只说上海地方,虽然是个中外通商的总码头,那些市面上的生意却一半都靠着堂子里头的倌人。

那班路过上海的人,不论是什么一钱如命、半文不舍的宝贝,到了上海他也要好好的顽耍一下,用几个钱,见识见识这个上海的繁华世界。凭你在别处地方啬刻得一个大钱都不肯用,到了堂子里头就忽然舍得挥霍起来,吃起花酒来一台不休,两台不歇,好象和银钱有什么冤家的一般。所以上海市面的总机关,差不多大半都在堂子里头倌人的身上。堂子里头的生意狠好,花钱的客人狠多,市面上的资本家也狠多。若是堂子里头的生意不好,花钱的客人也不狠多,那市面上的经济就有些不妙了。这是个什么缘故呢?堂子里头是嫖客最肯花钱的地方,要是堂子里头的生意都不济起来,那市面上的恐慌自然是可想而知的了。但是如今上海地方的堂子,比起十年以前的光景来却是大大的不同。客人的经济,一天窘似一天。堂子里头的规则,却一天坏似一天。以前那班堂子里头的倌人,一个个都还有些自爱的思想,见了客人也都大大方方、规规矩矩的;既没有那般飞扬荡佚的神情,又没有那种鄙薄客人的思想。若是有一个倌人姘了戏子,或者姘了马夫,就当作个惟一无二的耻辱,不但做客人的剪他不起,就是同辈姊妹里头,也都把这个人当作下流,传为笑柄。所以那个时候,倌人们姘戏子的狠少,就是或者有几个,也都是讳莫如深,不肯自家承认。如今的倌人却不是这个样儿,一个个庞然自大,见了客人,面子上虽然不说什么,心上却狠有些轻鄙客人的思想。那生意不好的倌人,也还不必说他。最可恨的是那些生意狠好的红倌人,一味的只晓得姘戏子、轧马夫,闹得个一塌糊涂,不成话说。非但没有一些儿惭愧的意思,而且还得意扬扬的十分高兴,那脸皮上面好象包了一层铁皮的一般。以前堂子里头倌人的品行,比如今那些倌人的品行高了好些,却对着客人不摆一些儿架子。如今的倌人品行坏到极处,那一付无大不大的架子,却比以前的倌人大了好些。就是那些旧时花丛里头的先正典型、老成规则,也都差不多删除净尽,颓落无存。正是:

回黄转绿,春残苏小之楼;月谢花蔫,肠断琵琶之梦

未知以后如何,请看下文交代。





第一百五十九回 范彩霞歇夏观盛里 陆丽娟独游味莼园





且说上海那些堂子里头的习气一天一天的愈染愈深,那班倌人们的人品便也愈趋愈下。面貌好些的倌人不是一味的飞扬跋扈,廉耻全无,就是拼命的作态妆妖,矜持太过。那些面貌不好的却又一个个都是怪丑无比,粗犷非常。要想找一个性情和软、举止大方的,一时间那里找得出这样的一个人?那班客人们到堂子里头去顽的,若不是在嫖界里着实的有些资格,免不得言语之间就要受他们的怠慢,神色之际更要受他们的欺凌。但是如今的那些嫖客,那一个是有十二分嫖场资格的?大半都是些土头土脑的曲辫子。这样的人到了堂子里头这样的地方,那就真是求荣反辱、自寻苦吃了。就是那些资格狠老、事情内教的客人,若是逢场作戏、随随便便的只当是个消遣的顽意儿,那还没有什么;若是当真的狂嫖起来,却也没有什么趣味。

花了无数的银钱,耗了许多的时刻,还要拼着自己的精神,来应酬这些倌人,更要费了自己的思想,来对付他们。花了钱到堂子里头去顽,原是要图个自在、寻个开心的,若像如今到堂子里头的这般时势,做客人的也要步步留心起来,还寻个什么开心、图个什么自在?这可不是花了银钱自家买罪受么?看官们看着如今堂子里头的这样情形,听着在下做书的这番说话,再仔仔细细自己想起来,这个“嫖”字可还有什么味儿!

如今闲话休题,只说辛修甫自从章秋谷到了天津去以后,狠觉得有些寂寞,虽然刘仰正、王小屏等都在上海,却都不如章秋谷的交情格外来得密切些。所以一个五月里头,辛修甫坐在家里头不狠出来,就是花酒也比往时吃得少些。只天天到自己书局里头走上一趟,料理些印刷的事情。

这一天,辛修甫正在书房里头和王小屏闲谈,忽然见陈海秋从外面闯了进来,见了辛修甫便道:“你这几天躲在家里有什么事情?连龙蟾珠那里都不去,这是什么缘故?”修甫道:“也没有什么缘故,不过我为着这几天天气热得狠,懒怠出门。

前几天听刘仰正说你到苏州去了,是几时回来的?“陈海秋道:”我到苏州去了足足的十天,昨日一早才到上海的。今天你们想来没有什么应酬,我请你们到西鼎丰林嫒媛那里去吃酒。“辛修甫皱一皱眉头道:”这样的炎天盛暑,到堂子里头去吃花酒,实在没有什么味儿。你若是还有别人可请,我就心领了罢。“陈海秋道:”这个使不得。今天我是吃的双台,因为天热,人多了十分拥挤,只请了九个客人,连我自己只有十个人。你若是不去,小屏一定也是不去的了。八个人吃个双台,似乎面子上不甚好看,只得委屈你一次,和我绷个场面的了。

修甫听得陈海秋说在林嫒嫒那里吃双台,便觉得有些诧异,道:“林媛媛那里你又没有交情,平空去报效他做什么?”陈海秋笑道:“你不要管我有交情没有交情,只要屈你的驾去上一趟就是了。”王小屏插口说道:“这样说起来,林媛媛那里你又下了水了,怪不得范彩霞要说你是垃圾马车。好好的做了范彩霞,为什么又要跳起槽来?”陈海秋道:“我也并不是跳槽。彩霞这一节在观盛里歇夏,我一个月贴他二百块钱,不做生意。所以我自端午节之后,在林媛媛那里走得勤些。”辛修甫听了陈海秋话,微微一笑也不开口。王小屏便问道:“彩霞在观盛里歇夏,你当夏一个月给他二百块钱么?”陈海秋道:“自然是真的,难道哄你不成?”王小屏笑道:“难道他在观盛里只有你一个人去,别的客人都不去的不成?”陈海秋摇头道:“那是他和我讲明的,歇夏的时候开销不够,要我一个月帮他二百块钱。那班旧日的客人,除我之外只有一两个熟客偶然去走走,别人是一概都走不进去的。”

王小屏听了,不由得鼻子眼里“哼”了一声道:“照你这样的讲起来,你一个月给他二百块钱,简直是你和他开销的了。论起理来,就不该应再走别的客人,为什么他那里的客人又不止你一个呢?”陈海秋道:“你到说到这般容易。二百块钱一个月那里够他挥霍?他自己亲口和我说过,一个月房租多少、伙食多少、坐夜马车的钱多少、吃大菜看戏的钱多少,还有相帮、娘姨的工钱,一切大小的零用,他口中算起来差不多一个月要七八百块钱,那里二百块钱就包得住他的用度?”

王小屏听了笑了一笑,还想要开口和他说时,被陈海秋拦住道:“闲话少说,今天是礼拜六,张园里头十分热闹,我们坐在这里也没有什么意思,还是到张园去坐一回儿何如?”辛修甫点一点头道:“我们同到张园去也好,只要到一大去叫他放一辆马车来就是了。”陈海秋道:“你们不用另叫马车,我这辆马车是借章季居章京卿的,是船式的双马车,十分宽敞,不要说坐三个人,就坐四个人也坐得下。”

辛修甫听了,也便点头应允。大家一同走出弄口,坐上马车,果然三个人坐在里头甚是宽绰。那马夫把丝缰一带,加上一鞭,便滔滔滚滚的一路往味莼园来。

到了安垲第,辛修甫同着王小屏、陈海秋下车进去,就在台阶上拣张桌子坐下。

这个时候,正是六点多钟的时候,夕阳西下,晚风徐来。那一班来乘凉的人倒着实不少,一个个都在辛修甫等面前过去。倌人里头也有几个认得的人,见了辛修甫等大家点一点头。

辛修甫等正在游目骋怀之际,忽见一个丽人缓缓的从后面转过来,腰细惊凤,鬟低敛雾,宜主娇娆之态,凌华婀娜之姿,扶着一个十六七岁的小大姐,走到辛修甫面前,凝眸一视,便停步含笑道:“辛老长远勿见哉嘛。”辛修甫连忙抬头看时,原来不是别人,就是那章秋谷的相好陆丽娟,便也向他含笑点头,招他坐下。丽娟又招呼了王小屏和陈海秋两个,便也慢慢的坐下来,开口便问道:“辛老,章二少到天津去仔阿有信来?阿晓得俚几时转来呀?”修甫道:“信是常常有的,信上说七月里头一定要回来乡试。你和他是狠要好的,难道他去了,信都没有给你一封不成?”丽娟面上一红道:“倪搭一塌刮仔接着仔俚一封信。”

陆丽娟刚说到这里,忽然王小屏拉了辛修甫一把道:“你看,你看!”辛修甫连忙回过头去看时,只见一男一女从斜刺里慢慢的走过来。那女子的模样只好二十来岁的样儿,穿著一件白官纱衫,玄色外国纱裙,里面衬着淡妃色金阊纱裤,面上不施粉黛,止淡淡的点着一点儿胭脂,顾盼飞扬,丰神流动。一面走着,一面时时的溜转眼光,照顾那同来的男子,笑吟吟的露出一团媚妩,软怯怯的妆成满面风情。

那男子随在女子背后,年纪约有三十多岁,穿著一件白香云纱长衫,手中拿着一把雕翎扇,那头上的前刘海差不多有一二寸长,刷得一截齐的,发光可鉴。眉清目秀,齿白唇红,却是一张瘦骨脸儿,两边的颧骨生得高高的,满脸上堆着一团滑气。手上却带着一个全绿玻璃翠班指、两个金刚钻戒指,灿灿烁烁的,光彩照人。紧紧的跟在那女子的后面,两只眼睛骨碌碌的四围飞射。

辛修甫看了一眼,猛然想起这个男子的样儿,分明就是天仙戏园里头的武小生廉小福。那个女子虽然狠有些面熟,却一时想不起是什么人。看着他们男女两个的那种样儿,狠觉得有些看不上眼。陆丽娟也看见了,连忙别过头去不去看他,口中低低的说道:“格号人,晤笃去看俚做啥!”辛修甫便也低低的问王小屏道:“这一个男的是廉小福,那一个女的又是什么人?你认得不认得?”王小屏附耳说道:“女的就是前节在东尚仁的姚月仙,新嫁了电报局总办宣柳生的,你难道不认得么?”

辛修甫听了恍然大悟,原来这个姚月仙,刘仰正也做过的,辛修甫同着王小屏等在席上和他相遇过几次。辛修甫见了他觉得好生面熟,却一时间想不起来,如今听了王小屏的说话,心上方才明白。暗想上海的这班红倌人,真是十分可恨,好好的嫁了人,却又偏要出来这般混闹。

正想着,只见廉小福和姚月仙在草地上兜了一个圈子:回身走上台阶,就在对面的一张桌子上双双坐下。那一种眉来眼去的神情,眼波四飞,双眉欲动,委实的十分好看。陆丽娟看不上眼,便立起身来,辞了辛修甫等,往老洋房那一边便走。

那一班男男女女的游客,见了廉小福和姚月仙两个人,觉得他们那般情景,知道一定不是什么好好的来历。更兼廉小福也是一个有名的武小生,天天登台演剧,认得他的人狠多,便不免大家都在背地里窃窃议论起来。廉小福、姚月仙见了,知道议论的一定是他们两个,也觉得有些坐不住,只好付过了茶钱,立起身来便走。

辛修甫见他们走了,方才对王小屏和陈海秋说道:“如今上海的风气一天坏似一天,像这样的事情还不足为奇。更有好好的大家内眷,也似这般的一味在外边胡闹,廉耻的两个字儿竟是没有的了。以后的人心俗,不知要坏到怎样的一步田地呢!”

说着,不觉大家嗟叹一番。正是:

桑间濮上,采兰赠芍之风;北阁西厢,待月期星之约。

未知后事如何,且待下文交代。





第一百六十回 吊膀子淫令得意 闹包厢戏馆争风





且说辛修甫和陈海秋等在味莼园回来,便一直到西鼎丰林媛媛院中。陈海秋忙忙的写起请客票来。一会儿客人来了,陈海秋分付摆起台面来。一班客人为着天气十分炎热,略略的坐了一回,便大家谢了主人,散席回去。

辛修甫想着回去也没有什么事情,便约着王小屏和陈海秋等到天仙去看戏。王小屏摇头道:“这般天气到戏馆里头去听戏,可不是自己找苦吃么?”修甫道:“包厢里看戏的人少些,又有风扇,我们只要去包他一间厢就是了。看戏虽然苦热,回到家里去也是一般。还是找些消遣的事情,觉得比坐着些好。”陈海秋道:“今天礼拜六,这个时候已经差不多九点多钟,只怕包厢早已挤满的了。”王小屏忽然笑道:“我们方才看见的廉小福和姚月仙,廉小福恰恰是天仙里头的武生,姚月仙自从和廉小福有些首尾,想来一定是天天要到天仙去看戏的,我们今天去看看他们两个人的把戏也好。”陈海秋听了甚是高兴,催着辛修甫快去,迟了恐怕没有坐位。

辛修甫便同着他们几个走出西鼎丰弄口,一路往天仙戏园来。

进了戏馆,自有认得的案目赶忙招呼。辛修甫便问:“还有全间的包厢没有?”

那案目弯背躲身、满面添花的道:“别人来是腾不来的了,如今辛老爷要,让也要让出一间来。”说着,便引着众人走上楼去,果然让了一间包厢出来,请辛修甫等进去坐下。

辛修甫举目看时,只见楼下正桌上的客人虽然不见得十分拥挤,却也坐得满满的没有什么空位,只有楼上的人略略少些。随手拿过一张戏单来看时,只见排的廉小福的《长阪坡》、谢月亭的《四郎探母》、小连生的《四进士》。台上已经做到一阵风的《泗州城》,《泗州城》完了,就是小连生的《四进士》,做得甚是精神。

《四进士》做完,便是谢月亭的《四郎探母》。手锣一响,谢月亭缓步出来。辛修甫等素来闻得谢月亭的声誉,知道是个新出来的著名老生,不免大家都细细的看他。

只见他面如满月,肤若凝脂,骨格玲珑,身材稳称。更兼喉音高亮,清脆非常,唱到那几句摇板,直唱得十分沉郁,无限凄凉,好象一声声、一句句都唱出眼泪来。

辛修甫听了十分叹赏道:“真个名不虚传,不愧是个后起之秀。”

一面听着,一面留神往厢楼上两旁一看,只见两边楼上有好几个不尬不尴的少年女子,都目不转睛的看着那台上的谢月亭。这一个眼波斜溜,那一个檀口微开;这一边方才巧笑承迎,那一边又是娇声引逗。那一种妖娆冶荡的样儿,一时间那里摹绘得出。更兼那几个女子的样儿十分诧异,说他是人家人罢,又实在不像是人家人。说他是堂子里头的倌人罢,又不像是个吃把势饭的样儿。辛修甫看了诧怪非常,口中叹一口气道:“怎么上海地方的风气如今竟坏到这般田地?我记得前几年的时候还不是这个样儿,怎么隔不多时竟会现出这般怪状?”王小屏道:“前几年已经都是这般的了,不论什么人家人和堂子里头的人,吊起膀子来都是在戏馆里头,把戏馆当做他们的台基一般。你向来不狠听戏,所以没有留心罢了。”

辛修甫听了,便也不说什么,只细细的看那台上的谢月亭,看他怎样的对付那班女子。只见那班女子,虽然一个个眉花眼笑,卖弄精神,把一双眼睛钉定在谢月亭身上,目不转睛的看,那谢月亭却只顾做他的戏,不甚理会。虽然也有时回他们几个眼风,却终是随随便便的,不大经意。

辛修甫看了,不懂这个里头是什么道理,心上疑惑:或者是那班女子面貌丑陋,看不上眼,所以不去理会也未可知。便又对着那班女子看了一看。只见那几个女子,也有面貌生得平平常常不狠出色的,也有生得十分出色、艳丽非常的,却没有一个丑陋的在里头。辛修甫想来想去,始终想不出这里头究竟是怎么的一回事情,便和王小屏、陈海秋两个人说了。王小屏和陈海秋也留心看了一回,果然觉得那几个女子虽是十分挑逗,谢月亭却有意无意的不甚兜揽。王小屏和陈海秋也想不出这个道理来。

这个时候,台上的谢月亭已经做到“别妻被擒”的一场,那一个抢背筋斗也跌得十分圆稳。陈海秋喝一声采道:“这个小孩子委实可爱,怪不得这班没廉耻的妇女要一心一意吊他的膀子!”王小屏听了,便取笑他道:“这样说起来,你若是做了女子,也一定要和他吊膀子的了。”陈海来也笑道:“我不过是这般说说罢了,你又没下巴起来。”

正说着,忽然陈海秋回过头来,一眼看见隔壁二包里头空空洞洞的,一个人也没有,却铺着台布,装着碟子,还有两个花插,里头插得满满的都是鲜花,摆设得狠是精致。陈海秋便道:“怎么二包里头的客人,到这个时候还没有来?”辛修甫微微笑道:“我是进来的时候早已看见的了。这个包厢,一定是那位电报局总办宣观察的姨太太长包在这里的了。”陈海秋不信,道:“今天是礼拜六,他为什么到这个时候还没有来,只怕不是他包的罢。”辛修甫笑道:“你不要性急,等会儿廉小福的戏出场,他自然会来的。”

说犹未了,早听得一阵脚声,一个案目当头领着一班大大小小的妇女,一窝蜂都走进二包里来。陈海秋连忙回头看时,只见一个少年女子领着两个娘姨、两个大姐,嘻嘻哈哈的做一堆儿坐下。果然不是别人,就是在张园里头看见的那个姚月仙。

这个时候的妆束和方才大不相同,打扮得粉腻脂浓,珠围翠绕,穿著一身外国纱衫裤,越显得花嫣柳媚,玉润珠圆。那姚月仙坐了下来,也不看台上的戏,只和那两个大姐咬着耳朵,咕咕唧唧的说了一会,也不知他说些什么。

一会儿谢月亭的戏已经演毕,便是廉小福的《长阪坡》登场。廉小福穿著一身簇新的白缎绣甲,捻着一根短短的白蜡杆枪,气昂昂、雄赳赳的走上场来,台容甚是整齐,台步也十分稳称。这个时候,不但是姚月仙的一双眼睛目不转睛的注定在廉小福身上,就是那一班楼上楼下的看客,也大家的眼光都拢在廉小福一个人身上。

廉小福抬起头来,往两边包厢里头把眼睛飞了一转,见了姚月仙喜孜孜的在包厢里头看着他微微展笑,便不由得心花大放,越趁精神。那混战的一场,一路枪花使得水屑不漏。“投井”的一场,更添出几个大翻身,旋转如飞,身段活泼,演得甚是认真。只把个姚月仙在包厢里面喜得满心奇痒,张开了一张樱桃小口再也合不拢来。

辛修甫等一面看着戏台上面廉小福的戏,一面又要看着包厢里头姚月仙的戏,倒觉得有些应接不暇起来。正看到好处,忽然听得“豁啷啷”一声响亮,一个茶碗从头包里面直飞到二包里来,刚刚的不歪不斜,正飞在姚月仙的头上,直把个姚月仙吓了大大的一惊,头上淋淋漓漓的淋了许多的水,一枝翡翠押发折作两截,珠花也掉了一支。接着,听得头包里头有一个女子的声音,娇滴滴的骂道:“格只烂污货末,直头少有出见格,嫁仔人再要出来吊膀子,面孔才勿要格哉!”这一下子,登时二包里头闹哄哄的大乱起来。

姚月仙吃了这一个惊吓,更听得隔壁有人骂他,明晓得这个隔壁的人一定也是廉小福的相好,顿时又恨又妒,心头那一股酸气直升到脑门里头来,再也按捺不住;不顾好歹,也跳起身来厉声骂道:“耐是啥人介?倪认也勿认得耐,吃醋末也勿是实梗吃法格嘛。耐倒有面孔骂倪,说倪勿要面孔,耐阿是要面孔格呀?要仔面孔末,也勿操至于到戏馆里向来吃醋哉嘛!倪吊膀子末,勿关耐格事体,挨勿着耐来瞎三话四。耐有本事末,跑出来等倪认认耐格大好老嘘。拿仔茶碗躲来浪隔壁打人,连搭仔王法才呒拨格哉!耐打断仔倪一根押发,搭倪好好里赔得来,少仔一个铜钱末,耐试试看!”一面说着,喝叫手下的那几个娘姨、大姐:“唔笃大家才跟仔倪,到隔壁去问问格只烂污货看!”说罢,便立起身来往外就走。

那头包里头的那个宝贝,听得姚月仙把他这般痛骂,更气得一佛出世,二佛生天,把两只小脚在地下乱顿道:“倪吃醋末,自然有吃醋格道理,你倒再有面孔说得出格号闲话?老实对耐说,廉小福搭倪末四五年格老相好哉。倪挂仔牌子规规矩矩做生意,搭戏子轧姘头,呒啥希奇。耐是嫁仔人格人家人,宣家里格姨太太呀,再有面孔出来轧姘头?”一面说着,一面也挺身而出,直迎上来,刚刚和姚月仙打了一个照面。

姚月仙好好的坐在那里,被他泼了一头的水,又打断了一支押发,直恨得咬牙切齿,恨不得把他一把扭过来打个半死,方才爽快。见他直迎上来,不免抬起头来看他一眼。只见这个女子约莫也不过二十多岁的样儿。头上梳着一条油晃晃的朴辫,没有一些插戴。身上也穿著一身外国纱衫裤,不穿裙子。身量苗条,丰神妖丽,蛾眉直竖,粉面通红,恶狠狠的直扑过来。正是:

月照明河之梦,神女生涯;风吹妒海之波,摩登业界。

在下做书的做到此处,却要暂歇一回。以后的许多事迹,都要在十一、十二两集里头出现的了。





第一百六十一回 泼醋当场争口舌 单相思狭路劫伶人



上回书中说到辛修甫同着陈海秋等在天仙看戏,忽然头包里头一个少年女子和那二包里头的姚月仙大闹起来。姚月仙那里肯让,便也挺身而出,要到隔壁去打他。

那女子也怒气吽吽畔的直扑过来,两下相隔止有二三尺路。两下正要动手,幸而有几个案目,听得楼上大闹,连忙飞一般的赶上楼来,急急的两边拦住,横身劝解。

这个时候,辛修甫见他们大闹起来,便也立起身来张望。只见那姚月仙被案目横身插劝,不得近前,更觉得满心火发,便指着那个女子对着众人道:“唔笃大家听听看,世界路浪阿有实梗少有出见格事体。别人家吊膀子末,吊来浪肚皮里向,吃醋末也吃来浪肚皮里向,阿有啥像俚实梗,吃醋吊膀子才放来浪面孔浪向,倒说廉家里搭俚四五年格老相好哉。四五年格老相好末那哼呀?区俚说得出实梗格闲话!”

俚自家末挂仔牌子做生意,倒要管牢仔相好,勿许俚去吊膀子,世界路浪也呒拨格号道理嘛!“

那女子听了姚月仙这番说话,更气得金莲乱顿,烈火横飞,也指着姚月仙骂道:“倪吃仔把势饭,吊膀子当官格,呒啥希奇。耐格勿要面孔格< 毛乍> 千人,再有面孔出来吊膀子!阿是耐姨太太做做,做得勿高兴哉,再要出来做倌人?别人搭俚吊膀子,倒还勿要去说俚,独独挨着耐要搭俚吊膀子末,倪定规勿许,看耐阿有啥法子!”

姚月仙把舌头一伸,头颈一缩道:“阿唷阿唷!格是倪吓得来魂灵才吓脱格哉!

耐勿许倪吊俚格膀子末,阿是耐格家主公呀?耐有本事末,管牢仔俚,勿要放俚出来吊膀子。耐说勿许倪吊末,老实勿客气,倪定规吊定格哉,耐有啥格法子末来末哉,倪等好来浪!耐说倪< 毛乍> 千人,倪倒勿曾挂啥< 毛乍> 千人格牌子哩!“

一席话,把那女子说得又气又恨,只指着他的脸大声说道:“耐再有面孔来浪嘤嘤喤喤,倪立时立刻去叫仔宣家里格老乌居来,看耐再敢勿走!”姚月仙听了这句话,倒不觉吃了一惊,一时说不出话来。

这个时候,楼上楼下的那些看客,听得楼上闹得这样的天翻地覆,不由得大家都立起身来回头探望,却又不知究是怎么的一件事情。一霎时人语喧哗,万头攒动。

那门口的红头印捕,也靴声橐橐的走上楼来。姚月仙见势头不好,又被那几个案目苦苦的解劝,又怕那个女子说得出来做得出来,万一竟去叫了宣观察来,这倒不是顽的,只得自己做个落场道:“今朝便宜仔耐格烂污货,明朝再搭耐说闲话!”说着,便头也不回的转身便走。那个:女子见了红头印捕走上楼来,心上也觉得有些害怕;更兼见姚月仙已经走了,总算自己占了上风,便也不敢再说什么,也带着两个大姐回身便走,一面口中咕咕哝哝的讲道:“格只老乌居,讨仔实梗格一个姨太太转去,真正叫作业!”

辛修甫等看着他们做出那般的形状,又听着他们说出那样无耻的话儿,一个个心上都觉得十分好笑。如今见他们两个人都已经走了,台上的戏已经做到《长阪坡》后段的汉津口,辛修甫等见时候不早,便都无心看戏,大家一同下楼回去。刚才慢慢的走下扶梯,戏台上戏已经演毕,登时,那些看戏的人就和潮水一般的直拥出来。辛修甫便拉了陈海秋一把道:“我们不用去和他们挤在一起,等一会再走就是了。”王小屏道:“我们走侧门出去也是一样的。”辛修甫道:“侧门的路狠难走,而且也狠拥挤,不如还是等一回儿罢。”王小屏听了便点头应允,等着那班人略略的散了一散,方才一同走出门外。

到了门外,辛修甫一眼看见一个面貌狠好的倌人,一个人站在门外,好象等什么人的一般。辛修甫仔细一看,便认得是公阳里的沉二宝。只见他秋波侧盼,两颊微红,目不转睛的看着那些门内去来的人。辛修甫便叫了一声二宝道:“你在这里等什么人?”沉二宝抬起头来看了一看,见是修甫,脸上不觉呆了一呆,随口说道:“倪等格个断命格阿招,勿晓得那哼再勿出来。”支吾了两句,辛修甫也不去理会他的话儿是真的假的,对着他一笑,点一点头,便同着陈海秋等走了过去。

沈二宝见辛修甫等走了,依旧还是目不转睛的望着门内出来的人。等了一回,只见门内走出一个十六七岁的少年男子,面如满月,肤若朝霞,猿臂蜂腰,肩平身削,匆匆的在门内走出来。刚刚一脚跨出大门,沉二宝见了大喜,登时间笑容满面,心花怒开,不顾好歹走上‘步,一把便拉住了那少年男子的手,口中说道:“耐啥格到故歇出来介?倪等仔耐半日哉!”那少年男子出其不意,被他平空的这样一来,倒不觉吃了一惊,连忙回过头来楞着眼珠说道:“你是个什么人,平空拉我做什么?”

沉二宝到了这个时候也顾不得廉耻,笑吟吟的对他低声说道:“勿要实梗嘘,到倪搭去坐歇末哉!”那少年男子听了他这两句话儿,由不得心中一动。更兼沉二宝这样满面添花和他讲话,口中一阵阵的香气直送过来,娇喉巧啭,脂香暗吹,不知不觉的抬起眼睛来把沉二宝细细的一看。只见这个沉二宝红腻桃腮,波凝杏眼,容光飞舞,体态风骚,觉得眼睛里头好象电气灯的一般霍的一闪。这个少年男子看了这样的一个丽人站在眼前,又是自己凑去上和他勾搭,心上那有不动的道理?便也不因不由的对着沉二宝微微一笑。沈二宝见了那少年男子居然向他一笑,只喜得眉飞色舞,毛骨悚然,那一种说不出来的快活直从心窝里头直发出来,几乎连自己的生年月日都一概忘记得干干净净。

正在这般时候,猛然又从门内走出一个五十多岁的人来,一眼见了沉二宝拉着那少年男子的手,由不得心头火发,鼻孔烟生,抢上一步劈手把沉二宝的手尽力一拆,拆了开来,睁着两个眼睛对沉二宝骂道:“你是个女子,怎么一些儿廉耻都不顾,千人百众的所在,做出这个样儿来?他一个小孩子懂得什么,你这样的凭空引诱他?天下那有像你这般的人,还不给我走开去!”

这没头没脑的一席话儿,沉二宝虽然脸皮狠老,也被他骂得脸上一阵一阵潮热起来。要想就此撒手罢,看着这样的个风流俊俏的人儿,心上那里舍得下。要想和他扭结固结的软缠一下罢,看着这个人气势汹汹的,两只眼睛直勾勾的瞅着他,好象要一口把他吞下肚去的样儿,又觉得有些怕他。暗想这个混帐东西不知是他的什么人。我常常听得人说,他的父亲谢云奎拘管儿子得十分利害,不许他在外面混闹,不要就是他罢。想着,便叹了一口冷气,想要回转身去。忽然心上又转一个念头,觉得好容易今天候着了他,究竟有些放他不下,便老着脸儿,硬着头皮走上一步,对着那个人说道:“耐勿要来浪嘤嘤喤喤,倪格事体勿关得耐啥事!倪吊膀子末,也挨不着耐来管!”

那个人听了沉二宝这几句说话,倒反呵呵的冷笑道:“你吊膀子不用我管,说得好轻松的话儿!你吊别人的膀子,自然和我不相干,不来管你的闲事。如今你要和我的儿子吊起膀子来,难道也说不与我相干,不要我管不成?”沉二宝听了,方才知道他真是谢月亭的父亲谢云奎。一时间闭口无言,十分惭愧,只得低着头连连往后倒退。

谢云奎回过头来,一眼看见他那位公郎呆呆的站在一旁,还在那里不住的偷眼注视方才的那个女子。谢云奎看了心上甚是生气,望着他喝了一声道:“你还不快快的回去,站在这里看什么!”谢月亭被他父亲一喝,也吓了一跳,连忙往外便走。

谢云奎紧紧的跟在后面,一同回去。

沉二宝眼睁睁的看着谢月亭走了,好似不见了一颗夜光珠的一般,心上十分不乐。却又不敢去拉他,只得自己慢慢的一步一步捱到马路边上。那包车夫阿二、阿福两个,已经把一对药水车灯点了起来,照耀得精光四射,已经在那里等了好一会。

沉二宝却好象没有看见一般,还在那里东张西望的寻他的包车。直至阿二叫了他一声:“二小姐看什么?车子在这里。”沉二宝正在心猿意马的拴缚不定,神飞意荡的收束不牢,突然听得车夫叫了一声,方才猛然醒悟,讪讪的坐上车去。

到了公阳里,跑上楼去连衣服也不换,跑到榻床上去一头睡倒,咳声叹气的心上狠不自在。一班娘姨大姐明知道他的心事,只好大家静悄悄的不说什么。偏偏的这个时候又来了一起打茶围的客人,沈二宝那里肯出去应酬?只叫娘姨们出去和客人说:“先生有病睡在床上,不能起来。”一班房间里人听了沉二宝这样的怠慢客人,大家心上都有些不以为然。却又为着沉二宝是自己身体,又不欠什么债,不好说他什么,只得由他。幸而这几个客人都是狠本分的人,听见二宝有病,便不肯多坐,略略的坐了一回,便大家起身散去。

这一起客人刚刚跑了出去,接着又听得楼下相帮高叫:“大人上来!”楼梯上靴声橐橐的又走了一个客人上来。几个娘姨、大姐见了,大家都眉花眼笑的迎上前来。正是:

月暗蓝桥之路,好事多磨;波横银汉之桥,仙槎不渡。

要知后事如何,下回交代。





第一百六十二回 杜春心严亲怜少子 困债台名妓叹穷途





且说沉二宝房间里头的那班娘姨大姐听得相帮叫了一声“大人上来”,便一个个都迎出房来。一个大姐阿招,便去叫沉二宝道:“先生豪燥点起来,潘大人来哉!”

沉二宝正在满肚子的不高兴说不出来的时候,只当没有听见的一般,动也不动一动。

阿招叫了两声,见沉二宝不理他,便发起急来,走上去把沉二宝推了一把道:“先生起来嘘,晏歇点潘大人要发脾气格嘘!”

看官,你道这个里头究竟是怎么的一回事情?这位潘大人又是个什么人?为什么相帮不叫潘大人,竟是这样的叫他大人?

原来这个沉二宝,本来是也个狠有名气的红倌人,做客人的工夫也狠不错,但是有一件堂子里头最犯忌的毛病,一味的爱姘戏子。只要见了一个有些名气的戏子,不论他的面貌如何,一定要千方百计的吊他的膀子。差不多上海的几个有名戏角,都和沉二宝有些牵牵缠缠的不清楚。

那一天沉二宝到天仙戏园去看戏,恰恰谢月亭第一天上台,年纪又轻,品貌又好,衣服又甚是鲜明,唱工又十分出色。沉二宝的眼睛里头,从来没有见过这样的一个玉雪可念的人物,便一心一意的想要吊谢月亭的膀子。也不知想尽了许多方法,用尽了无限心机,无奈这个谢月亭一则年纪狠小,有些孩子,不狠去理会他;二则他父亲谢云奎约束得十分严紧,每天都是和谢月亭同进同出,寸步不离,生恐怕有那班无耻的倌人要转他的念头,吊他的膀子,非但淘碌坏了身体不是顽的,并且恐怕倒了嗓音不能唱戏。他们吃唱戏饭的人,全靠着嗓子卖钱,倒了嗓子唱不出来,还有那个园子里头肯来请教他?所以谢月亭在戏台上做戏的时候,凭着沉二宝怎样的卖弄凤骚,有心挑逗,谢月亭却始终正眼儿也没有去看他一眼。沉二宝一连看了一礼拜的戏,竟想不出一个钩他上手的法儿。

其实,谢月亭这个小孩子虽然可爱,却也不是什么上天下地有一无二的美男子。

无奈情人眼里出西施,在沉二宝眼睛里头看起这个谢月亭来,真是个子都再世,叔宝重生,越看越好,越好越爱。这个爱情,直从心眼里头发出来的。偏偏的这个谢月亭只是凭他做作,不去理他。沉二宝看着谢月亭在台上唱戏的时候,恨不得一把将他拉了过来、立时两个人捏作一团,合成一块,方才爽快。只是这样的到眼不到手,直把个沉二宝熬得清水直流,满心奇痒,差不多害了单思病的一般。

前两年的时候,沉二宝住在南平安,生意十分发达。后来不知怎样的,一班客人大家都知道他爱姘戏子,一个个都绝脚不去。沉二宝又是手里用惯大钱的,虽然见生意不好,他却一些儿都不放在心上,依旧还是吃大菜,看夜戏,坐马车,吊他的膀子,寻他的开心。不到一年的工夫,身上欠了三千多的债,本家的房饭钱、菜钱、外面的店帐,到了年底下催逼起来,只把一个沉二宝逼得个上天无路人地无门,没有一些主意。想来想去,想不出个解结法儿。看看的差不多到了二十一二的那几天,沉二宝一古脑儿把帐上算了一算,一切的饭钱和菜钱,还有带挡的利钱和那些店家的帐,差不多要一千七八百块钱,方才可以敷衍得过去。看看堂簿上的局帐和酒帐,止有一千不到。就是那班客人一个钱都不少全数收了回来,也还差着一千上下。年近岁逼,那里去弄这一千块钱?

这一天已经到了十二月二十五日的晚上,沉二宝一个人坐在房间里头,局也没有人来叫,看着别人的房间里头虽然生意比平常的时候清些,却一样也还有人来碰和吃酒。只有自己的房间里头冰清水冷的,不但没有人来碰和吃酒,连打茶围的客人都没有一个跑进来。连着那房间里头的娘姨大姐也都一个个无精打彩的冷面相向,只是咕咕哝哝的埋怨沉二宝不肯好好的做生意,一味的在外面和那班戏子混搅,如今弄得这般模样,连累了房间里头的人一个大钱都摸不着。

沉二宝受了他们的埋怨,委实无言可答,只得忍气吞声的听着。思前想后,心上也觉得有些懊悔起来,懊悔以前生意好的时候,不该应这般胡闹。如今到了这般时候,跳又跳不出去,弥补又弥补不来。想着若是实在打算不出什么法儿来,只好咬定牙齿,暂落帐房,找一个有钱的人来,把自己捆出去。但是捆了出去之后,这个身体就不是自己的身体了。自己又是散淡惯的,那里过得惯这般日子?想到这里,恨不得有个地洞好等自己钻了下去,免得这般出丑。不由得两泪交流,一个人呜呜咽咽的哭起来。

哭了一回,见娘姨小妹娘走进房来,沉二宝叫他倒盆水来洗脸。那知小妹娘只当没有听见一般,也不开口,把个脸儿板得铁生生的冷笑一声,竟自走到妆台前,开了妆台抽屉不知拿了一件什么东西,回过身来往外便走。沈二宝见了小妹娘这般模样,只气得呆呆的瞧着他,一时倒也说不出什么来。要想骂他几句罢,这个小妹娘不比别人,是有五百块钱带挡的,万一个和他闹翻了,他立时立刻要起钱来,一时那里有钱来还他?只好勉强忍住,不说什么,长长的叹了一口气。

忽然门帘一动,又走进一个人来。沉二宝只道就是小妹娘重又进来,把头别转不去看他。却听那进来的人口中说道:“先生长远勿见哉嘛。”沉二宝听得不是小妹娘的声音,却是自己旧日一个大姐叫做阿玉的声音。沈二宝平日狠喜欢这个阿玉的,便抬起头来看时,见果然不是别人,果然就是旧时的跟局大姐阿玉,便对他勉强笑道:“耐倒还想着倪实梗格倒霉人,到间搭来走走。”阿玉听得沉二宝这般说法,心上已经有些明白。又仔仔细细的向沉二宝脸上一看,便失惊道:“先生哈格事体实梗样式,阿好说拨倪听听呀?”说着,便挨着二宝身旁坐下。

二宝到了这个山尽水穷的时候,见了阿玉就好象见了个前世亲人一般,便拉着阿玉的手,把为难的情形一一和他说了一遍,说罢又不觉流下泪来。阿玉听沉二宝说得这般可怜,心上也狠有些不忍,只得竭力劝慰一番。沈二宝见阿玉身上穿的、头上戴的,都甚是齐整,便又对他叹一口气道:“耐来浪妹子搭倒蛮好,耐妹子生意阿好呀?”阿玉道:“倪妹子生意格年把总算呒啥,格一节做着仔个姓潘格客人,搭倪妹子蛮要好。一节勿曾到,洋钿用仔四五千。像实梗格客人,故歇总算要让还俚天字一号格哉!”沉二宝听了心中忽然一动,便故意问道:“格个潘家里啥地方人呀?”阿玉道:“就是湖南格潘大人呀。耐啥忘记脱哉呀?格辰光也做耐格呀。”

二宝想了一想,方才知道就是那潘中堂的嫡孙,世袭侯爵的潘广平潘侯爷。

讲起这位潘侯爷来,本来性格风流,贪花好色,差不多一天到晚都是在堂子里头过日子的,更兼家赀巨万,年少封侯,又是个堂子里头的惯家、花柳丛中的老手,有财有势。那些倌人那一个不巴结他?但是这潘侯爷却是出身富贵,养尊处忧,一呼百诺惯的,把性情惯得十分矜贵。到了堂子里头,只要一句话儿不合,便立时立刻的翻转脸皮,把那倌人痛骂一场,就此绝迹不去。若是有了个和他合得上来的倌人,用起钱来,一千八百,三千五千,甚而至于竟是一万八千都不算什么事情。那阿玉的妹子叫做花婷婷,本来是个杭州人家的姨太太,后来不安于室,逃了出来,在上海做生意。把他的娘也在苏州乡间接了出来,又把阿玉叫了回去,就算了房间里头的做手娘姨。这个阿玉以前在沉二宝那里的时候,两个人甚是要好。沉二宝和戏子吊膀子,一半都是阿玉的牵头。所以虽然到了花婷婷那里,心上还是十分想念。

这一天,阿玉跟着花婷婷在一家春番菜馆里出局,这位潘侯爷也在席上。见了花婷婷一身袅娜,满面风情,便看中了他。当时就转了一个局,接着就跟到花婷婷院中去吃了一个双台。花婷婷知道潘侯爷是个天字第一号的好客人,便提起全付精神来殷殷勤勤的应酬一番。潘侯爷见他宛转依人,圆融出众,大大方方的,却没有一些儿装娇作态的样儿,刚刚合上了潘侯爷的意思,当夜就有了相好。那花婷婷自然拿出那勾魂摄魄的手段来,千般昵就,万种缠绵,把个潘侯爷奉承得十分欢喜。

一连几个月,着实花了几个钱在花婷婷身上。不但婷婷狠有些儿储积,就是阿玉当个房间里头的大姐,一节的工夫也多了几百块钱。到了十二月二十日的那一天,潘侯爷早已除局帐之外,另外给了花婷婷一千块钱给他开销各帐,又给了二百块钱给房间里头的人。

花婷婷自从做了潘侯爷之后,只有几户老客人来往,新客一概不做。堂簿上的酒局帐,除了潘侯爷之外,不过七八百块钱。到了二十三,已经把酒局帐收齐。八百块钱只打了一个九折,已经算是极好的了。花婷婷收齐了帐,便也把所欠的一切帐目都早早付清。

到了二十五那一天,阿玉坐在院中没有事情,忽然想起沉二宝来,差不多有一年多些不见了,不知现在的生意怎么样?以前想去看他,都为生意上事情狠忙,不得分身。如今趁着年底没有事情,何不到公阳里去看他一看?这一来有分教:

暮雨襄王之梦,家令重来;春风淫女之禅,摩登无恙。

未知以后如何,且听下回分解。





第一百六十三回 逢旧待深宵谈秘戏 索新逋软语媚干娘





且说阿玉换了衣服,到公阳里来看沉二宝。花婷婷是住在西荟芳的,从后面穿出西荟芳弄堂,不多几步就是公阳里。当下阿玉见了沉二宝,沉二宝把自己的情形告诉了他一遍,便托他不论什么地方,和他借几百块钱,就利钱重些也不要紧。阿玉沉吟一会,便答应了三百块钱,却要四分起息。沈二宝自然答应,觉得略略放心。

阿玉坐了一回,便??辞去。沉二宝一把拉住那里肯放,只说多时不见,要和他谈谈,留他吃过了晚饭去。阿玉也便答应。大家手拉手的坐在一起,讲得十分亲密。

阿玉又说起潘侯爷要叫花婷婷学坐自行车,花婷婷学了一天,跌了一交,就此不敢再学。沉二宝听了,猛然又触动了心上的一件事情,记得潘侯爷初做自己的时候,曾经说过最爱的是能坐自行车的女人。女人坐了自行车,自有一种说不出来的天然丰韵,可惜堂子里头没有这样的一个人。那个时候听了这番说话,一则为着自己不会坐什么自行车,二则正和那一班戏子搅得天昏地黑的时候,不把这件事儿放在心上。潘侯爷做了自己不到一个礼拜,看着自己的样儿并不十分巴结,心上生了气,便从此绝脚不来。如今听了阿玉的话儿,刚刚自己在八九月中姘了一个戏子叫做十二红,这个十二红也是最爱坐自行车的,成天的教着自己坐自行车,倒练得十分精熟。不如趁着这个当儿再去用些手段,把潘侯爷引了回来,说不定可以借着他淴一个浴也未可知。

想到这里,便不由得心上欢喜起来,也不瞒着阿玉,就把这个念头对阿玉说了。

并道:“勿然是耐妹子格客人,倪勿好去拉俚。不过潘家里来浪上海滩浪,堂子里向做格相好也都得势,勿是耐妹子一干仔。就是到倪搭来仔,耐妹子格搭也都一样格。耐想倪格闲话阿对?”阿玉听了,想了一想果然不错,便也点头称是。沉二宝又细细的打听潘侯爷的性情嗜好,阿玉也细细的和他说了一遍。两个人又谈起以前吊膀子的旧话来。吃了晚饭,一直谈到十二点钟。沈二宝便留阿玉住了一夜再去,阿玉也便依允。沉二宝就在自己大床上留他住了一宵,两个人唧唧哝哝的直讲了一夜的话,直到天明方才睡去。到了明天十二点钟,沉二宝同着阿玉起来梳洗,又留阿玉吃了饭,阿玉方才别去。

阿玉走得不多时,早见女本家金姐走进房来,对着沈二宝冷冷的说道:“二小姐,耐也转转念头哉哩。倪格房饭钱搭仔菜帐,本底子不要紧,不过今年格事体,勿比旧年搭仔前年,倪自家开销才开销勿转。尴尬头来里,实梗洛今年格房饭钿菜帐才要付清。耐是格外勿比别人,再有四百块洋钿借头,耐今朝阿好先付几百洋钿,等倪去开销开销,再有格到仔年底再算阿好?”沉二宝听了大惊,好似兜头泼了一瓢冷水的一般,只得对着金姐说道:“妩姆勿瞒耐说,倪帐浪一塌刮仔收着仔一百几十洋钿,零零碎碎老早用完结格哉。格件事体末那哼弄法,总要请耐妩姆帮帮倪忙格哉。”原来这个沉二宝是金姐的干女儿,所以沈二宝也叫他妩姆。当下金姐听了沈二宝的话,板着个脸儿冷笑一声道:“世界路浪格事体,铜钿银子真公事,叫倪那哼帮耐格忙?倪搭耐是一径蛮要好,大家格心思也蛮对劲,不过今年格事体直头尴尬,耐想倪自家开销勿够,洛里再好帮耐格忙?耐总要豪燥点想法子末好,勿要到仔格个辰光,大家难为情。”

沈二宝听得金姐的口气甚紧,心上更觉着急,暗想如今世上的人,真真是世态炎凉,不堪回首。前两年自己生意狠好的时候,就是一个大钱也不给他都不要紧。

就是这个金姐,平日之间也不知受了自己的许多礼物,占了自己的无数便宜,如今却这样的反面无情,逼迫得这般利害。想着不觉叹一口气,便又对着金姐恳恳切切的说道:“妩姆格待倪一径勿错,倪只要有法子想,洛里肯实梗样式?故歇实在一个铜钿才呒拨来里,只好请妩姆停脱格一两天,等倪到外势去想法子──”金姐不等他说完,顿足说道:“耐末说得蛮舒徐,呒啥要紧,耐阿晓得今朝是啥格日脚哉?

今朝已经廿六,再要停脱格一两日,已经小年夜哉!谢谢耐,耐总算照应倪格。拿笔房饭帐菜钿算清爽仔,耐真正弄勿落末,倪大家慢慢里再想法子。耐总算看倪面浪,拨倪一个面子。要是耐一干仔勿拿出来,大家也才看仔耐格样子,才勿拿出来,格是倪僵哉嘛!照式实梗样式,上海滩浪格本家洛里还有人做?卖脱仔自家格身体来赔,也勿够嘛!“

沈二宝见金姐这样顶真,没奈何,只得含着一胞眼泪,拉着金姐的手,宛宛转转的央告道:“妩梅请坐仔,倪有两声闲话要搭妩媳商量。”金姐铁铮铮的洒脱了手道:“格是呒啥商量格!耐呒拨洋钱,搭倪商量;倪呒拨洋钱,去搭啥人商量呀?

今朝搭耐说明白仔,耐豪爽点自家去转点念头,勿要到仔归格辰光,大家面子浪过勿去,倒说倪坍仔耐格台!“说着便回身要走。沉二宝忍气吞声的一把拉住了道:”妩媳勿是呀,倪有生意浪格闲话搭妩姆商量呀。“

金姐听了,方才回转身来,催他有话快说。沈二宝便把潘侯爷的性情专爱能坐自行车的女人,和自己昨日心中的意思要想在潘侯爷身上弄他一大笔钱,宛宛转转的和金姐说了一遍。又蹙着眉头道:“倪格人妩梅也晓得格,只要潘家里跑进仔倪格门,老老实实勿怕俚跑到啥地方去。不过格件事体总归是开年格事体哉,今年年里向,洛里有洋钿开销?妩姆就是拿倪随便那哼,也逼勿出一个铜钿。衣裳首饰,好格老早当脱格哉。故歇格点衣裳首饰,一塌刮仔几百洋钿格事体,再要去当脱仔,新年里向那哼出去做生意?倪想起来,只得求求妩姆,赛过做好事,搭倪随便洛里去借几百洋钿,拿格房饭帐菜钿付清仔,就是五分八扣也说勿得格哉。倘忙到仔开年,靠仔妩梅格福气,生意浪多点洋钿,总归搭妩姆二八分帐末哉。倪待妩姆一径勿曾错歇,赛过自家格亲生娘,妩媳待倪也赛过自家格亲生囡仵。妩姆总算照应自家格囡仵,倪受仔妩梅格好处,心浪也明白来浪。”说到这里不觉眼圈儿一红,心上觉得十分委屈。又道:“倘忙妩姆定规勿肯答应,倪也勿怪妩梅,总归才是倪自家勿好。到仔故歇,懊悔也懊悔勿来格哉。妩梅再勿肯照应倪点,是今生今世总归呒拨出头日脚格哉!”说着不由得两泪交流,几乎要哭出来。金姐听了这番说话,却着实的沉吟了一回,登时面上露出笑容来。

看官,你道金姐是听了沉二宝的一番话儿说得十分恳切,方才打动了他的心么?

那里知道世界上当老鸨的人,都是那狼心狗脸、鼠肚鸡肠,只认得钱不认得人的宝货。就是他亲生父母欠了他的钱,也是一文不饶,两文不让的,何况沉二宝不过是他的干女儿,那里肯放他过去?这个金姐在上海当了二十年的老鸨,手里头着实有几个钱。方才问沉二宝着紧的讨钱,并不是自己过不去,为着这两年沉二宝的生意不好,又知道他拖了几千块钱的债,恐怕他得空同着戏子逃走,给你一个远走高飞,不是顽的。早已暗暗分付沉二宝的娘姨、大姐一步步的紧紧跟随。如今又有心逼他归帐,预备他还不出来,就把他所有东西统通扣住,给他一个先下手为强。外面的店帐,凭着沉二宝自己去设法支吾,他只要自己的钱到了手中,那里还管别人的死活。如今平空听了沉二宝的这一席话,又许他二八分帐,不免就有些贪得起来。更兼知道潘侯爷是上海地方数一数二的阔客,沉二宝又是个堂子里头香名鼎盛的倌人,以前生意不好,是他自己爱姘戏子闹坏的事情,以致客人裹足。如今既肯回心转意,改悔前非,好好的做生意,原是一定做得出来的。不如趁此做个人情,不去追他的房饭帐和菜帐,面子上只说和他在外面转借了钱来开销这一笔帐。既然赚他一笔大大的利息和扣头,还白白的得他一个二八提来,料想将来这个潘侯爷一定逃不出沉二宝的圈套。那时沉二宝有了钱,一个大钱都不会少的。想到这里,便不因不由的脸上露出笑容来。

沉二宝看了,知道他心上已经答应,自己心上的一块石头方才落地。金姐看了沈二宝一眼,故意叹一口气道:“二小姐,你是年纪轻,勿晓得上海滩浪格把势饭勿容易吃嘘。耐放仔好好里格客人勿做,去搭仔格排唱戏格戏子吊膀仔。耐看仔格排戏子巴结得耐蛮舒徐,蛮高兴,只当俚笃是好人,洛里晓得格排滑头码子,才来浪想耐格洋钿,洛里有啥格真心待耐?等到耐洋钿呒拨哉,俚笃也勿来哉。倪格辰光一径搭耐说,格排戏子靠勿住,耐勿肯听倪格闲话,故歇弄得实梗。早点听倪两声闲话,洛里会到实梗样式?二小姐啊,吃格碗把势饭,苦煞格嘘!拿仔自家身体去换别人家铜钿,洛里会几化称心?耐末贪图仔戏子称耐格心样式样,才依仔耐,耐要俚笃那哼,俚笃就听耐那哼。阿晓得自家身体称仔心,铜钿勿称心哉呀!”金姐说到这里,还待要再说下去,只把一个沉二宝说得满心惭愧,满面羞惶,凭着沉二宝的脸皮再厚些儿,也不由带耳根连脖子都涨得通红。金姐便顿住了口不说下去。

正是:

金空岁暮,何来避债之台;逝水华年,讵有翾风之宠?

不知金姐还说些什么,请看下回便知分晓。





第一百六十四回 逼残年倌人借债 丧良心小子探囊



且说金姐见沈二宝羞得面红过耳,二十四分的不好意思,便剪住了话头不说下去。停了一停方说道:“二小姐,耐勿要见气哩,倪是不过望耐生意好点,大家有点好处,实格洛劝劝耐。等耐心浪明白点,倘忙耐要见气起来,格是倪下转连搭仔口才勿敢开格哉。二小姐,耐想倪格闲话阿对?”沉二宝红着脸道:“妩姆格闲话,说到仔洛里搭去哉!妩姆搭倪讲格,才是好闲话。倪归格辰光,煞死勿肯听妩姆格闲话,故歇弄得实梗样式。早点听仔妩姆格闲话,也勿操至于实梗格样式。故歇倪也呒说法格哉,只好拿前头格事体一塌刮仔才丢脱,赛过呒拨实梗格事体。到仔开年,规规矩矩,一心一意做生意。倘忙生意好点,也是妩姆照应仔倪一场,总算韵落空。妩姆刚刚搭倪讲格闲话,倪一句一句才记来里心浪向。故歇除脱仔妩媳,再有啥人肯搭倪说格号闲话呀。”金姐听了拍手道:“难末二小姐耐明白哉!倪说耐实梗一个明白人,洛里会实梗胡涂?耐真正肯拿从前格事体丢脱仔,一心一意做生意,格是定规做得好格,几千洋钿格债啥格希奇!”

说到这里,便又故意作难道:“故歇别样事体才缎去管俚,倒是耐要借洋钿,真生活。”说着又屈着指头算了一算道:“房饭账搭仔菜钿,算俚七百,再有四百洋钿借头,故歇过年格辰光,洛里去借啥洋钿?要借洋钿,要末到中尚仁萧三大搭去借,不过利钿重得野笃。”沉二宝到了这个时候,那里还管什么利钱重不重,就是要他对本对利,他也没有什么不答应。便再三重托了金姐,托他去做保代借,明知道金姐自己有钱,萧三大的话儿不过是做个推托罢了。

当下,金姐还故意作难了一回,沉二宝又再三再四的央告,金姐方才答应。故意到外面去走了一个转身,便回来和沉二宝说:“萧三大虽然肯借,却要四分起息,先付三个月利钱,又要打个八扣。合算起来,要借一千六百块钱,方才敷衍得过去。

一千六百块钱打个八扣,先扣去了三百二十块钱,再付三个月利钱,一百九十二块钱,还有什么代写借据和中保人画押的钱,帐房先生的回用,整整的又是八十块钱。

合起来只得一千多块钱到手,还要贴出一百块钱,方能把房饭钱菜帐付清。还有那些煤炉上和厨房里头的零碎开销不在其内。“

沉二宝听了,心上算了一遍,竟要生生的吃亏六百块钱,虽然心上有些舍不得,但是到了这个时候,明知道金姐是捉着自己做的,不怕自己不答应,脸上又不敢露出那一种不愿意的神色来,只得勉强装着笑容连声称谢,一一依从。金姐拿出一张写好的借据来,叫沉二宝在上面画了一个十字,便收好借据。去了一回,果然带了一千块钱的一张庄票和八块现洋回来,除了付给金姐一千块钱,沉二宝自己止落下八块钱,还欠了金姐一百块钱的找头没有给他,言明停两天再付。沉二宝自己心上盘算了一回,觉得开销差不多够了,客人的局帐收了五百几十块钱,阿玉答应借的三百块钱恰恰的也送了来,就是差些,也所少有限。沉二宝心上方才宽了一宽。

到了二十八的那一天,沉二宝正拿着几篇店铺的发票,请帐房先生进来和他代算。算了一回结出一个总数,一古脑儿要七百多块钱,马车行、戏馆和大菜馆最多。

沉二宝通盘一算,还差一百多块钱,便请了金姐进来和他说明,那结欠的一百块钱请他暂缓一下,明年再付。金姐虽然不甚愿意,却又不得不答应。

金姐前脚走了出去,接着外面相帮便一声高叫,早有一个客人大踏步走了进来。

沉二宝正开了橱门,要把那八百几十块钱都搬出来,开发那些店铺。本来和他们说明,叫他们二十八下午来的,这个时候已经两点多种,料想差不多都要来了,便把那几封洋钱一封一封的都搬出来。刚刚搬了两封,听得客人走进来,便连忙把洋钱依然收在橱内,随手掩上橱门。回过身来看进来的客人时,不觉大大吃了一惊。原来这个进来的也不是什么客人,竟是桂仙戏园里头的小丑小飞珠,和沉二宝也是有些交涉的。这个小飞珠本来是个最下流的戏子,就是他同班的伶人大家也都瞧不起他,不知沉二宝怎样的看上了他,两下就轻轻易易的成了好事。到了后来,沉二宝有了别人,便不大理他。这个小飞珠见沉二宝不理他,便也赌一个气,裹足不前,从此和沉二宝绝了来往。到了今年,小飞珠在外亏空闹得大了,不得过年,忽然想起沉二宝是个有名的红倌人,一定手里有钱,不如跑到他那里去,问他借几百块钱,如若他回绝不借,便一口把这件事情叫穿出来,料他也不敢不借。好在这个小飞珠本来是个卑鄙不堪、龌龊非常的人,那里知道什么羞耻,便一个人高高兴兴的跑到公阳里来。

沉二宝猛然见了小飞珠,不觉吃了大大的一惊,又不能叫他走出去,只得低低的问道:“耐到倪搭来做啥?间搭堂子里向勿便当格呀?”小飞珠听了也不多说,只把自己的意思对沉二宝说了一遍,要向他借五百块钱。沉二宝听了又气又笑,对他说道:“倪故歇自家弄勿落来里,再有啥洋钿来借拨耐?请耐去搭别人借仔罢。”

小飞珠听他不答应,便睁起两个眼睛,口中说道:“你橱里头现放着许多洋钱是做什么的?怎么我问你借,你就推托起来?”沉二宝见了小飞珠这样其势汹汹的样子,好象是理应要借给他的一般,心上自然十分生气,却又怕他把以前的事情当着众人直说出来,不敢一定对他怎样,只说道:“耐洛里晓得,倪橱里向一塌刮仔七八百洋钿,自家付帐才勿够来里。倪有洋钿格辰光,是耐来借就借点拨耐末哉。故歇刚刚过年格裆口,叫倪啥地方去调洋钿借拨耐呀?”

小飞珠听得沉二宝一口回绝,定不肯借,不由得气忿忿的拍着胸脯,口中乱嚷道:“你这个时候姘了别的人,把我丢到脑后,你想就是这样的算了么?”沉二宝听了,急得连忙赶过来拉着小飞珠的手,低低说道:“耐阿好少说两句,倪也一径<曾勿>待错歇耐。有啥闲话,慢慢里商量末哉。”说着连忙回头看时,恰好一个娘姨小妹娘回去看他女儿去了,一个大姐阿金和也不在房间里头,不知到外面去做什么。沈二宝见房里没有第三个人,便索性把小飞珠拉到榻上,并肩坐下,附着耳朵说丁几句不知什么话儿。想着今天他既然要想借钱,料想贼无空过,只好认个悔气,送他一百块钱,且把他敷衍走了再说。

正想着,忽然肚子里头绞肠刮肚的一般大痛起来。沉二宝皱紧了眉头,连叫“阿呀”,急急的跑到床后去。这个时候,肚子痛都来不及,那里顾得别样事情?

就在这一会儿的工夫,忽听得小飞珠在前面说了一声:“我还有事情到别处去,等一回儿再来。”沉二宝听了答应一声,暗想他没有拿到钱,怎么居然肯走,想来一会儿就要来的。想着,便听着小飞珠脚声橐橐的走出房去。

停了一回,听得大姐阿金和的声音,同着一个楼上李小兰房间里头的大姐一路说笑进来。刚刚走进房门,忽然失惊倒怪的叫道:“先生哩,到仔洛里搭去哉呀?

橱门为啥开直来里,啥人开格呀?“沉二宝听了这两句话儿,心上吃了一惊,便在床后应声道:”倪为仔肚子里痛,来里解手呀,橱门倪< 曾勿> 开嘛。耐豪燥点看看橱里向格物事嘘!“阿金和听了,连忙走进一步,看了一看,不觉大惊道:”先生,耐洋钿阿曾动呀?“沉二宝听了这句话,知道事情不妙,那心头的小鹿儿上上下下的撞个不住,连忙嚷道:”洋钿倪< 曾勿> 动呀!“一面说着,一面也顾不得肚子痛,七跌八撞的从净桶上立起来,连手都顾不得洗,急急的赶出来,直急得两手如冰,满身香汗。早听得阿金和嚷道:”洋钱剩仔四百块哉,啥人来得拿去格呀?“

沉二宝更急得芳心乱跳,两泪交流,连忙自己赶过去查点起来。恰恰的止剩了四百七十多块钱,那四百块钱却是不翼而飞,不胫而走了。正是:

青楼胠箧,惊残名妓之魂;白日探囊,恨煞无良之盗。

不知以后如何,且待下文交代。





第一百六十五回 逐香尘游春驰绮陌 骋飞车奋勇捉瘟生



且说沉二宝见橱门大开,橱里头的洋钱只剩了四百多块,还有那四百块钱,不知到那里去了。明知道这一转眼的工夫没有别人,一定是小飞珠趁着自己一个疏忽,悄悄的开了橱门,顺手牵羊的偷到手中,却故意说一声“有事到别处去,等一会儿就来”,急急的跑了出去,安安稳稳的受享那四百块钱去了。只把一个沉二宝急得口呆目定,话都说不出来。想着这个小飞珠这样的没有良心,趁着这般时候还来偷了几百块钱去,不由得两行眼泪扑簌簌的直挂下来。

这个时候,女本家金姐也知道了,连忙赶过来看了一看,便问沉二宝究竟是怎么的一回事情。沉二宝定了一定,方才含着眼泪把刚才的事情告诉了金姐一遍,只把小飞珠是个戏子的话儿瞒了起来,只说是一个姓马的客人。好在沉二宝和小飞珠已经断了多时,所以阿金和同着那几个客堂里的相帮都不认得他是个戏子。

当下金姐听了沈二宝的说话,便道:“听耐实梗说起来,格个洋钿定规是格个杀千刀偷得去哉。俚耐住来浪啥地方,倪大家赶到俚屋里向去。”沈二宝听得金姐:喧问客人的住处,只得又说几句谎话道:“格个杀千刀,还是两年前头倪来浪美仁里格辰光,来浪倪搭吃歇过一台酒。本底子倪勿认得俚,也是客人同来格朋友,吃仔一台酒,一径< 曾勿> 来歇。倪刚刚来浪开仔橱门拿洋钿,吃着格个杀千刀冒冒失失跑得来。刚刚说呒拨三句闲话,夹忙头里倪肚里痛起来哉,痛得来呒淘成。

勿壳张格个杀千刀趁倪来浪解手格辰光,倒说偷仔洋钿就跑,叫倪洛里想得着?“

说着,不由得眼泪又直流下来,对着金姐说道:“难末叫倪那哼!”

金姐想了一想:便道:“勿是倪来浪说,格件事体是耐自家勿好,忒嫌大意仔点哉。耐就是肚里痛,要去解手末,为啥勿叫个人进来嗄?陌陌生生格客人,咦勿是啥一径来格熟客,洛里好实梗勿当心?”沉二宝道:“格个辰光肚皮里向痛煞快,洛里晓得格个杀千刀来偷格洋钿?”金姐冷冷的道:“难看耐那哼弄法,格个客人咦勿晓得俚住来浪啥格地方,要追也呒追处嘛。自家勿小心,只好自家认仔悔气格哉。”说着,恐怕沉二宝又要和他缠扰,急急的走了出去。沉二宝见了长叹一声,默然无语。大家略略的安慰几句,也跟着一哄散去。

不多一刻,那班收帐的店家陆续到来。沉二宝拣那必不可少的几家店铺都付清了,有几家不甚要紧的,再三和他们商量先略略付些,其余的等过年再付。那知这班店铺里头的人也和金姐一般,都是十分势利。若是这个欠帐的是个有钱的人,你就一个钱都不给他,他也没有什么不放心。惟有遇着个那些没有钱的债户,好象是他不共戴天的杀父仇人一般,那里肯放松一点。沉二宝的那些店帐本来端午、中秋两节都没有付清,那些店铺里头的人心上已经在那里十分懊悔,如今到了年底,如何还肯通融?不但不肯缓到明年,连一刻儿都不肯等候。大家坐在沉二宝房间里头,七张八嘴的催逼,只把个沉二宝逼得束手无策,哭笑皆难。到了晚上,大家勉勉强强的散去。明天一早,已经都绝早的赶来,坐在沈二宝房里坐索,渐渐的吵闹起来。

沉二宝没奈何,只得又叫小妹娘去请本家金姐。金姐知道一定又要借钱,起先还不肯来。沉二宝一连叫人请了他三次,方勉勉强强的走进房来,口中说道:“二小姐叫倪做啥?倪事体忙煞来浪,耐总要自家打打主意末好呀,寻着倪有啥格用场?”

沈二宝见了金姐的面,便一把拉到后面小房间里头,滴泪苦求道:“今朝格件事体,总要请妩姆救救倪急格哉!”

金姐听了,便正色数说道:“二小姐,耐勿要看得铜钿实梗容易嘘!耐阿晓得,倪为仔耐身浪格事体,搭耐借仔几化洋钿,一千六百块洋钿笃呀,勿是啥格少嘘!

耐故歇自家勿小心,失脱仔洋钿,咦要问倪借,倪咦勿开啥钱庄银行,洋钿洛里会来得实梗容易?老实搭耐说仔罢,格两日拨耐格两个要帐格断命人,吵得头脑子才空格哉。楼浪向李小兰搭仔筱花丽卿,一径来浪叽哩咕噜,啥格钝仔俚笃格色头哉,咦是坍仔俚笃格台哉。格号闲话,倪轧实听勿惯。勿是倪今朝来里说耐,耐也忒嫌勿当心仔点哉,好好里洋钿放来浪橱里向,那哼就会拨俚偷得去?倪想起来,也呒拨实梗容易嘛。格个里向勿得知到底啥格讲究?“说着,便瞟了沉二宝一眼。

沉二宝被他这几句话儿正说着了他心上的心病,不由得心上突突的跳起来,知道金姐老奸巨滑,那里瞒得过他?万分无奈,只得用出看家本事来,立起身来拉住了金姐两手,竟是双膝跪下,把一个脸儿伏在金姐膝上,口中说道:“今朝格件事体,只得请妩姆再搭倪借四百洋钿格哉!妩姆真正勿肯帮倪格忙,倪也照呒说法,照式实梗样式,横竖生意也做勿成功,只好随俚笃去那哼格哉,格辰光一塌刮仔格事体,故歇也勿必说起俚,总归妩姆救仔倪格急,倪心浪也有数目来浪。”

金姐起先听了沉二宝的话儿,倒吃了一惊,暗想:万一他当真横着心肠,听凭他们怎样,索性不做生意,绰一个大大的烂污,往公堂上一跑,只说他自愿从良,那就把自己的一千几百块钱都送到水里去了,这倒不是顽的。后来又转念一想道:一个当倌人的转到这样念头,一定是山穷水尽,无可如何,方才肯走到这一条路上去,但凡有一丝一毫的法儿可想,也一定不肯这般的。像二宝这样的人,岂是肯走这条路的?想着,便故意一面拉着沉二宝的手,去拉他起来,只说:“二小姐豪燥点起来,折脱仔倪格福气格呀!”一面却又装腔作势的说,没有地方去借钱。沉二宝跪在地下那里肯起来,只说:“妩姆末赛过倪亲生娘嘛,本底子该应受倪格礼格呀。今朝妩姆勿答应,是倪一径距来里勿起来格哉!”说着不觉一阵心酸,眼泪直滚出来。

可怜这个沉二宝也是个数一数二的红倌人,平日之间最是心高气傲的,就是把金姐认做干娘,也是金姐看着他生意实在不差,想要沾些小利,哄骗得沉二宝心上甚是高兴,方才认着了这个干女儿。这个干娘是倒过来奉承干女儿的,沈二宝那里把他当真当作干娘。如今不过少了几个钱,金姐就顿时变转脸来。沉二宝受了他的数说,连屁都不敢放一个。思前想后,想着那往日的锋芒,看着这今时的景象,你叫他怎样的不要委屈?怎样的不要感伤?

闲话按下,只说金姐见沈二宝跪在那里不肯起来,心上十分得意,却又假意做出个无可如何、情面难却的样儿,勉勉强强的点头答应,替他再借四百块钱,拉了沉二宝起来。沉二宝谢了又谢,说了无数的感激的话儿。金姐果然又去拿了四百块钱来交给沉二宝,把店帐开销清楚,沉二宝草草的过了一个年。

过了元旦,沉二宝便又向金姐借了一百块钱,用八十五块钱去电飞脚踏车行里头买了一辆飞轮女车。到了初五的下午,沉二宝到了十二下钟就起来梳洗妆饰,加意打扮了一回,直到三点多钟方才修饰完备。自己用两面镜子照了一回,又走到着衣镜前左右端详了好一会,又叫金姐和小妹娘等进来细看一回。金姐见沈二宝加意梳掠出来,果然比别人不同,身段风流,衣裳熨贴,就是那几步路儿也是上海滩上数一数二的俏步,不是那班饭桶倌人可以学得到的,不由得连连道好。小妹娘等大家看着,自然也都说好。沉二宝见大家都啧啧称羡,便叫一个相帮和他推了自行车,喜孜孜的对着众人点了一点头,口中说一句:“倪晏歇就转来。”金姐也对他说一声:“恭喜发财!”沉二宝便出了公阳里,跨上自行车,由二马路转出大新街,望大马路泥城桥一带驶来。

上海地方坐自行车的人虽然狠多,却都是些男人,除了泰西妇女也一般乘坐自行车之外,那些中国的妇女从没有坐着自行车在马路上跑的。如今蓦然见了沉二宝居然坐起自行车来,大家心上都觉得甚是诧异,不由得大家的视线就聚拢在沉二宝一个人身上。更兼沉二宝貌美年轻,骨格娉婷,衣装艳丽。而且这个沉二宝坐自行车的本领狠是不差,踏得又稳又快,一个身体坐在自行车上动也不动;那些人的眼光都跟着沉二宝的自行车,往东便东,往西便西,还有几个人拍手喝采的。沈二宝也不去理会他们,一直过了泥城桥跑马厅。只见马路上的马车,一线齐的滔滔滚滚,络绎不绝。马车里头坐的,大半都是些堂子里头的倌人和那些滑头年少的游客,却也狠有几个大家内眷、绣阁名姝在里头。上海的风俗,都把正月初五当作财神日。

那班倌人,大家都浓妆艳抹的出来迎接财神,所以马路上的马车比别的时候格外来得多些。沉二宝一心一意的只想要去找那一位潘侯爷,好放出手段来笼络他,头也不回,一直往斜桥一带地方跑去。

那潘侯爷的公馆,就在斜桥总会隔壁,和张园离不多路。沈二宝走过潘公馆门外,便把那自行车缓缓的踏,慢慢的走过去。走不到两三丈路,便停了自行车,跨下车来把自行车倚在树旁。略略休息了一会,便又在潘公馆门外打个转身。一连这样的三次,不见潘侯爷出来。看看天将旁晚,斜日西沈,沉二宝没奈何,只得自己坐着自行车先到味莼园去。到了安垲第又等了好一回,依然不见潘侯爷的影儿。正是:

春云冉冉,未销倩女之魂;秋水迢迢,不见伊人之影。

不知后事如何,请看下文交代。





第一百六十六回 巧机关深谋排陷阱 奇遇合豪客入牢笼





且说沉二宝想要在张园里头等候那位潘侯爷,要在潘侯爷面前卖弄他坐自行车的本领,摩拳擦掌的一连等了两天,连潘侯爷的影都不见。沉二宝十分没趣,回到院中和金姐说了,叫小妹娘到潘公馆左右去打听,方才知道这位潘侯爷感冒凤寒,这天不能出门。沉二宝只得捺定了心呆呆的等候,一连等了四天,已是正月初十。

沉二宝又坐着自行车往潘公馆左右候了一回,又到张园去泡了一碗茶,依然不见这位潘侯爷出来。沉二宝等得心上甚是烦躁,看看时候不早,那些游客一个个都纷纷散去,沉二宝也懒懒的跨上自行车,慢慢的回来。

刚刚走过泥城桥,忽见一辆小小的亨斯美两轮马车从迎面飞也似的直跑过来。

沉二宝把自行车略略的向左一偏,那马车已经在沉二宝右边擦过。马车里头的人和沉二宝两下眼光一错,只听得那马车里头的人高叫一声:“好呀!”沉二宝听了这一声喝采,不觉心中一动。暗想:方才坐在这个马车里头的人虽然擦肩过去,看不清楚,却一眼看过去彷佛有些像那潘侯爷的样儿,不要当面错过了。想着,便“霍”

地把自行车拨转,回过身来。不想后面也正有一个坐着自行车跟在沉二宝背后,紧走紧赶,慢走慢赶。沉二宝回转身来,不偏不歪,刚刚和背后的人打个照面。沉二宝举眼看时,原来不是别人,却是金姐的兄弟叫做阿德的,就是院子里头的帐房先生。

当下这位烧汤大叔阿德劈面撞见了沉二宝,觉得不好意思,只得叫了一声“先生”。沉二宝见了十分诧异,待要问时,两下的自行车已经过去。沉二宝想了一想,心上忽然大悟,想一定是金姐叫他暗暗跟随,怕我欠了许多亏空,要乘空逃去的缘故。想着便回过头去远远一看,果然见阿德也拨过车来,隐隐的跟在后面。沉二宝觉得心中好笑,不去理他。

沉二宝心上在那里转着念头,那脚底下就未免慢了好些。那前面的马车却跑得十分神速,一转眼的工夫已经过去了七八丈远近。那马车里头的人,还在那里不住的回头张望。沉二宝便把脚底下紧了一紧,飞一般的直追过去,一霎时早已追过了头。仔细看那马车里头的人时,却不是什么潘侯爷,约莫也有四十来岁年纪,却穿著一身极鲜明的衣服。见沉二宝赶了过来,又目不转睛的向他细看,只说是和他吊膀子,心中大喜,便也眉花眼笑的对着沉二宝嘻嘻的笑。沈二宝见不是潘侯爷,那模样也没有相像潘侯爷的地方。沉二宝见了心上暗暗诧异,暗想这个人并不像潘侯爷,怎么平空的会看错了。一面想着,那自行车去得飞快,不知不觉的又到了潘公馆门首。沈二宝不去理会那马车里头的人,只把自行车轻轻拨转,望着原路回去。

今天又扑了一个空,心上十分懊恼。去踏了半天自行车,觉得有些腰酸力软,便把腰伸了一伸,缓一口气,沿着那马路左首只顾慢慢的走。

忽然,后面又有一辆自行车如飞似箭的赶过去,从沉二宝右首直穿过去,那自行车上的人却目不转睛的看着沉二宝。沉二宝抬头一看,不觉心中大喜,好似天上掉了个斗大的夜明珠下来的一般。原来这个自行车上的人不是别人,正是沉二宝一连候了好几天候他不到的潘侯爷。这个时候沉二宝一见了潘侯爷,不由得精神陡长,连忙用尽平生之力,把脚下一紧,星飞电闪的一般赶上前去。潘侯爷坐着自行车赶过了沉二宝的头,却还不住的回过头来往后张望。见沉二宝也催动自行车直赶上来,暗想:不料上海地方也有会坐自行车的女子。方才走过去的时候,却没有十分看得清楚,不知他面貌如何?想着便故意把自行车略略放得缓些,凭着沉二宝赶过头来。

恰恰的两车相并,中间只隔着三四尺路,两下都看得十分清切。

潘侯爷细细的打量沉二宝时,只见他穿著一件玄色泰西缎狐皮紧身短袄,下面衬着一条淡湖色泰西缎裤子。脚下踏着一双小小的尖头缎靴,尖尖瘦瘦的,差不多只有四寸。头上打着一条油松朴辫。再往面上看时,只见他腻粉搓酥,秧脂滴露。

长眉人鬓,青含远岫之云;俊眼流光,碧漾明湖之水。轻同飞燕,婉若游龙。更兼身量苗条,丰神流动,坐在自行车上,香凤飘拂,华彩飞扬,好似一朵彩云从平地上涌出来的一般。回波顾影,浅笑迎人,别有一种媚妩玲珑的态度。这样的一个美人,坐在自行车上自然比别人格外要好看些儿。更兼这位潘侯爷又有一个癖性,一生一世最喜欢的就是会坐自行车的女人。无奈上海地方的那班倌人,一百个里头倒有五十对是不会坐自行车的。如今偶然见了一个能坐自行车的女人,又具着这般的姿态,虽然不是什么倾国倾城的颜色,却也狠有些宜嗔宜喜的丰神。更兼这个沉二宝出奇制胜的地方还不在面貌上,全仗着一对秋波,一付身段,做个勾魂摄魄的招牌。横波一盼,能回铁石之肠;纤步轻移,不数昭阳之态。只把一个潘侯爷看得眼前撩乱,心上回旋,觉得自己所见的那些倌人都赶不上他这般丰态。更兼沈二宝是有心挑逗,自然的丝丝入扣,一拍就合。故意的对着潘侯爷嫣然展笑,以目送情,更把潘侯爷引得意马心猿,拴束不定,一时间六神无主起来,也对着沉二宝微微含笑,好象要和他说话的一般。

沉二宝见了这般模样,知道潘侯爷已经人彀,心中暗喜。却又故意卖弄精神,把身体向前一伏,把头一低,脚下用一用力,只见沉二宝的那一辆自行车好似天边飞鸟一般,一直线向前跑去。潘侯爷见了那肯放松,连忙催着自行车也赶上来。两辆自行车在马路上互相追逐,直像那弹丸脱手,羽箭离弦。路上的人见了,不由得一个个都拍手叫好。一霎时,沉二宝和潘侯爷早由大马路一直穿出黄浦滩,直到了三马路口。沉二宝方才慢慢的转进三马路,潘侯爷的自行车也紧紧的跟着转弯。

沉二宝虽然坐自行车的本领不差,却毕竟是柔弱女子,和潘侯爷追逐了一回,早已有些娇喘吁吁,额角上沁出几点香汗。潘侯爷看得清楚,趁势和他说道:“对不起,辛苦,辛苦!”沉二宝回头一笑道:“啥格对勿起呀,倪勿懂耐格闲话。”

潘侯爷笑道:“你在大马路上走得好好的,都是我平空的要和你比赛,冤冤枉枉的害你费了许多气力,岂不是我对你不起么?”沉二宝听了也不说什么,竟瞟了潘侯爷一眼,把嘴唇动了一动。潘侯爷见了,十分高兴,便又趁势问他住在什么地方。

沉二宝听了,忽然假作失惊道:“阿唷,耐是潘大人嘛,啥勿认得倪哉呀!”潘侯爷听了,又把沉二宝仔细认了一认,觉得虽然有些面熟,却一时想不起这个人来,便对沉二宝笑道:“我和你虽然狠觉得面熟,却想不出是在什么地方见过的。”沉二宝掩着嘴,“格”的一笑道:“倪故歇来浪公阳里,耐阿到倪搭去坐歇,马路浪向勿好讲闲话格。”说着,便放开自行车在前缓缓的走,潘侯爷紧紧的跟在后面。

到了公阳里,沉二宝下车进弄,走到自家门口,把手招着潘侯爷道:“请里向来坐,倪搭小地方,不过怠慢点。”潘侯爷连说:“不用客气。”一脚跨进房来,对着沉二宝又细细的看了一看。沈二宝对着潘侯爷把头侧了一侧,眼波斜溜,樱口微开。潘侯爷看了沉二宝这般模样,觉得一个心吸吸的动个不住,连要问沉二宝的名字都忘记掉了。停了一回,忽然想起道:“你可就是沉二宝么?怪不得我看着你面熟得狠。”沉二宝听了微微一笑,也不开口,只对着潘侯爷点一点头。潘侯爷方才明白果然是沉二宝,便问他这两年生意怎么样。沉二宝不肯和他说真话,只说:“生意也呒啥好,哝哝罢哉。”说着,又向潘侯爷一笑道:“耐啥洛吃仔一台酒,一径勿来呀?阿是倪怠慢仔耐动气哉?今朝勿是倪马路浪碰着仔耐,耐洛里会到倪搭来?贵人勿踏贱地,倪搭实梗格小地方,就等到仔开年,耐也勿见得肯来嘛,耐是要到花婷婷搭去格,倪洛里请耐得到?”

潘侯爷听了诧异道:“我做花婷婷还是上节做起的,你怎么就会知道?”沉二宝把眼一瞟,笑道:“倪自然有呒线德律凤格嘛,耐格事体洛里瞒倪得过?”说着,便趁势走过去,坐在潘侯爷左首,紧紧的靠着潘侯爷的肩傍道:“倪腰里向痛得来,勿得知啥格讲究?”潘侯爷趁着沉二宝说腰痛,轻轻的伸出双臂,把他拥人怀中。

沉二宝也不推却,只把身体扭了两扭,把纤腰紧紧贴在潘侯爷身上。潘侯爷见了沉二宝这样的俯就,心上自然欢喜,把一只右手捏着拳头,轻轻的在沉二宝背上捶了几下道:“你腰痛,我和你捶捶好不好?”沉二宝把一只纤手拉着潘候爷的手道:“谢谢耐,勿敢当,要折仔倪格福气格呀。”潘侯爷听了便低下头去,附着沉二宝的耳朵悄悄说了几句。沈二宝有意无意的略略点头,低眸不语,那眉间眼角却渐渐的红晕起来。这一夜,潘侯爷自然是住在沉二宝院中不回去了。娇郎抱日,倩女停云,海燕双栖,文鸳比翼。一个是江南名妓,一个是三楚通侯,你爱我的丰姿,我慕你的富贵,自然比别人的情景不同。正是:

金堂夜永,香销宝鸭之烟;锦幄春温,灯颤流苏之影。

要知后事如何,且待下文交代。





第一百六十七回 蓄深心连环施妙策 狙缠头反扑出奇丈



只说潘侯爷虽然和沉二宝有了相好,却平日之间听得别人说过沉二宝爱姘戏子,未免有些疑惑的意思。沉二宝心中明白,索性把以前自己爱姘戏子的事情,一一和潘侯爷说明,又装点出许多的话儿,只说那班唱戏的人怎样怎样的反面无情,怎样怎样的卑鄙无耻,自己看破了这般宝贝没有一个好人,心上二十四分的懊悔,以前不该这样的胡涂。如今既然遇着了你这样的一个人,自然死心塌地的守着你一个人的了。我自从吃了把势饭,眼中的客人也不知见了多少,却从没有遇着个像你这样温柔爽快的人,所以把这般的心腹的话儿一古脑儿都告诉了你,你却切不可再去告诉别人。沉二宝说到这里,不觉面上一红,羞怯怯的把个脸儿伏在潘侯爷怀里再也不肯抬起来。

潘侯爷虽然是个惯家,到了这个时候,听了沉二宝这样的一番说话,也不由得心上有些着迷起来,便拉着他的手,叫他抬起头来。沈二宝越发把个头紧紧的钻在潘侯爷胸前,一动也不动,口中却喃喃呐呐的说道:“倪搭耐讲仔,耐勿要动气嘘。

耐要动气,是倪勿来格。“潘侯爷笑道:”这些事情都是以前的把戏,与我什么相干?只要你以后知道改悔就是了,我为什么要动气?“说着,便把两手捧着沉二宝的脸,自己低下头去轻轻的偎了一偎。只见沉二宝的两边颊上红得十分鲜艳,好象那带露玫瑰,酣妍欲滴。见了潘侯爷兀自把两手掩着眼睛,似笑非笑的别转头去。

潘侯爷看了心满意足,酣畅非常。

自此以后,潘侯爷便和沉二宝约法三章,要他遵守:第一,不到戏园看戏;第二,不留客人住夜;第三,但是潘侯爷来了,不论什么客人在房间里头,都要让他。

沉二宝如何不肯?千依万顺,满口应承。潘侯爷又和沉二宝讲明,每月贴他四百块钱,吃酒叫局外算。只把个沉二宝喜得一个无可不可,心花大开。

潘侯爷从那一天住在沉二宝院中,到了明天起来,原想给他一千块钱的。忽然转念一想,故意一个大钱都不给,要看沉二宝怎么样。那里知道这个沉二宝是何等的手段,早已和金姐商量得停停当当的了。刚刚下床梳洗,便在拜匣里头拿出一百块钱的钞票来,交给小妹娘道:“格个是潘大人赏给唔笃格下脚,唔笃拿得去。”

小妹娘接了,谢了潘侯爷一声,便走了出去。

潘侯爷见了心上自是高兴,便对沉二宝道:“这下脚的钱怎么要你拿出来,我还给你就是了。”说着,便取出一个皮页子要拣钞票。沉二宝连忙拦住,笑道:“耐拿洋钿做啥,阿是还倪呀?还倪末谢谢耐。就要还倪末,也慢慢交末哉,用勿着实梗性急嘛。”潘侯爷先还不肯,只说下脚的钱断没有要叫你出的道理。沈二宝斜了潘侯爷一眼道:“阿唷,耐倒分得明白笃嘛!倪两家头比勿得别人,承耐格情看倪得起,倪也一径当耐自家人格,格两个铜钿啥格希奇?耐拨俚笃也好,倪拨俚笃也好,耐故歇实梗还拨倪,倒勿像──”

沉二宝说到这里,顿住了口不说下去,望着潘侯爷一笑。潘侯爷听了这些说话,觉得甜蜜蜜的,一字一句都钻进心坎里头去,心上甚是高兴,倒不好意思一定还他,只得罢了。过了一天,潘侯爷便另外送他一千块钱。沉二宝再三不受,口口声声只说的潘侯爷剪他不起。潘侯爷无奈,只得罢了,心上却甚是过意不去。

过了几天,潘侯爷在公馆里头吃过了饭,便到沉二宝那里来。沉二宝刚刚起来,正在那里梳头,见了潘侯爷,立起身来叫了一声,潘侯爷便坐在沉二宝旁边,看着他涂脂傅粉,掠月挑云,看得甚是得意。正在这个当儿,忽见女本家金姐走进房来,叫了一声“潘大人”,便去附着沉二宝的耳朵唧唧的讲了一回。沉二宝顿时皱着眉头,十分不乐,偷转秋波看了潘侯爷一眼,好象怕他听见的一般。潘侯爷看了他们这般鬼鬼祟祟的做作,不知道他们葫芦里头卖的是什么药儿。

正要开口问时,早见沉二宝对着金姐使个眼色道:“妩姆末总是实梗,早勿说,晏勿说,恰恰来浪格个辰光缠勿清爽。有啥事体,晏歇点再说末哉!”金姐听了,便回过头来看了潘侯爷一眼,方才说道:“格末昨日仔一篇帐拿得来,等倪交拨来帐房先生,叫俚搭耐算算。”沉二宝听了,便在贴身的衣袋里头取出一篇帐来交给金姐,却又回头看着潘侯爷,又好象怕他看见的模样。潘侯爷见了这般模样,那里忍得住?便问沉二宝道:“你们鬼鬼祟祟的说些什么?这一篇帐又是什么东西?快拿来给我看!”沉二宝听了,面上一呆道:“勿关耐事,耐缎要去问俚。”说着,又催着金姐道:“耐豪燥点去罢,勿要来浪多说哉!”

潘侯爷听了更加疑惑,叫住了金姐,不放他走,对沉二宝道:“你们究竟闹的什么鬼戏?快和我说个明白!”沉二宝道:“搭耐说勿关耐事,耐要问俚做啥?”

潘侯爷听了沉二宝这样的言词闪烁,金姐又那般的形迹可疑,心上不觉有些不快起来,冷笑道:“就是不干我事,也要和我讲个明白。”沉二宝把眉头一皱道:“耐格人啥实梗格呀,倪勿搭耐说,自然有勿搭耐说格道理来浪里向,耐慢慢交看末就晓得哉。”

潘侯爷见沉二宝始终含含糊糊的不肯和他讲实话,不由得心上生气起来,睁着眼睛看定沉二宝道:“我不管什么道理不道理,今天一定要问个明白!你们做的事情不用在我面前闹鬼。我不在你们这里走动,你们的事与我不相干;如今我既然在你院中走动,你又要去寻别人的开心,还要把我当作小孩子一般随口哄骗,那是办不到的!”沉二宝听了,不慌不忙对着金姐说道:“晤笃听听看,阿要气数。”金姐也笑道:“二小姐,耐末也有点勿着勿落。潘大人要看末,拨俚看看末哉嘛,为啥要瞒仔潘大人呀?”说着便走过一步,把手中的一篇帐目交在潘侯爷手中道:“潘大人勿要动气,格个是二小姐格帐呀,耐请看末哉。”

潘侯爷接过来看时,见果然是一篇帐目,什么房饭帐多少,家生店多少,绸缎店多少,洋货店银楼多少,零零碎碎的一篇帐目,差不多也有三千多块钱的样儿。

潘侯爷看了不懂,便问沉二宝道:“这是你的帐么?前天不是你和我讲过不欠别人的债么?”沉二宝听了,呆着个脸低头不语。金姐接口说道:“二小姐格两年生意勿局,一径亏空下来格呀,不过二小姐勿肯搭耐说罢哉。”

潘侯爷听了,想了一想,还没有开口,金姐又道:“说起二小姐格事体来,再要讨气也呒拨。格两年格生意,说末说勿好,到底还哝得过去,勿会去欠啥格债,吃着俚屋里向一个娘、两个阿哥、一个兄弟,四家头四支老枪,单是鸦片烟要三两开外哚。一榻刮仔才靠仔二小姐一干仔,一年里向格开销,少说点也要一千几百洋钿。旧年加二勿对哉,啥格阿哥讨家小,兄弟做生意,七七八八,去脱仔三千外势。

耐想二小姐前两年生意好点还勿要紧,刚刚旧年仔格生意只好做一个开销,洛里来实梗几化洋钿?实梗洛二小姐身上背仔三千多块洋钿格债,轧实说起来,俚自家一个铜钿才< 曾勿> 用着,阿要作孽!“金姐说到这样,沉二宝抬起头来对他说道:”耐少说两句哉呀!“一面说着,两只眼睛里头水汪汪的,含着一泡珠泪。

潘侯爷听了沉吟了一会,便又问金姐道:“二宝既然有这许多亏空,为什么瞒着我,不和我说?像这样的事情,也算不得什么大事,又为什么不早些和我商量?

多了我拿不出来,三千、五千的事情,也还算不了什么,为什么有心要不叫我知道呢?“金姐道:”倪一径搭二小姐说,叫俚搭耐潘大人商量,潘大人勿在乎此格,二小姐勿肯呀。“潘侯爷笑道:”这是个什么缘故呢?“说着,便回顾二宝。二宝斜倚在榻床上,把一只纤手托着香腮,低鬟敛袖的,只当不听见的一般。潘侯爷又问一声,二宝只不开口。金便含笑道:”倪搭耐潘大人说仔罢,二小姐是勿肯说格哉。二小姐格心浪,总道仔俚搭耐潘大人轧实是真心要好,勿是啥格假情假义,实格洛俚身浪欠仔债,瞒仔耐勿肯响起。晓得耐听见仔格件事体,定规要拨俚洋钿,教俚去还债格。俚要受仔耐格洋钿呢,好象是搭耐勿是啥真心要好,不过是有心想耐两个铜钱罢哉。要定规勿受呢,咦怕耐潘大人心浪要动气。潘大人耐想俚有仔实梗一个念头来里心浪向,自然勿肯搭耐说哉呀。“

这一席话,说得来圆转非常,有情有理,直把个潘侯爷听得好象醍醐灌顶,醇酒醉心,那心上的快活,一时间都说不出来,只微微含笑,把眼睛去看着沉二宝。

沉二宝也把眼光注在潘侯爷身上,好象有无限的深情流露出来。金姐又接着说道:“故歇上海滩浪格倌人,大家才是只认得铜钿勿认得人,对仔客人洛里有啥真心。

倪二小姐倒轧实勿是格号人嘛。耐潘大人< 曾勿> 来格辰光,二小姐一径搭倪说起,说上海格客人才靠勿住,只有耐潘大人末,气魄咦大,脾气咦好,上海滩浪实头难得碰着格。实梗洛格日子,二小姐肯留耐呀,勿然是洛里有实梗容易?格辰光,李宝珍李家里放仔三千洋钿──“金姐说到这里,沉二宝忽然”霍“的立起身来,红着脸说道:”耐末说说就要瞎三话四,越说越好听哉!豪燥点去罢,勿要勿着勿落格瞎说!“正是:

春满迷香之洞,宋玉魂销;花飞扶荔之宫,襄王梦断。

未知以后如何,且看下回分解。





第一百六十八回 假缠绵爱语稳痴人 真懊恼芳心乖宿愿





只说沉二宝推着金姐的背叫他出去,金姐知道这个时候大功已成,便呵呵的笑着走了出去。潘侯爷见他走了,自然要和沉二宝亲热一番,软语温存,柔情婉转,那相爱的情愫自然是十分熨贴,百倍缠绵,也不必去说他的了。

到了明天,潘侯爷拿着一张四千块钱的庄票,要给沉二宝还债,却婉婉的对他说道:“你不肯拿我的钱,自然是和我真心要好。但是这个里头也有一个分别,若是你不欠什么债务,有心敲我的竹杠问我要钱,自然对我不起。如今你委实欠了一身的债,我又不是没有钱的人,我们两个人这样的交情,理应和你代还债项,算不得是敲我的竹杠。况且是我自家愿意给你,又不是你问我索取的,你受了怕什么?”

沉二宝听了,正颜厉色的说道:“潘大人,阿有处请耐照应点倪,勿要实梗。倪欠别人家格铜钿末,等倪自家去想法子。耐要搭倪还债末,慢慢叫末哉,故歇用勿着。”

潘侯爷见他说得这样侃侃凿凿的定不肯受,心上更加欣服,暗想:如今上海堂子里头居然也有这样的人。便也正色问道:“你一定不肯受我的钱,究竟是个什么道理?你倒要讲给我听听!难道你剪我不起,所以不要受我的钱么?”沈二宝把金莲一顿道:“耐格人真正缠煞哉!倪要看耐勿起末,也勿要搳脱仔几几化化客人,独做耐一千仔哉嘛!”潘侯爷道:“既然不是嘛我不起,为什么不肯受我的钱?”

沉二宝呆着个脸不肯说。潘侯爷再三追问,方才叹一口气道:“老实搭耐说仔罢,倪格做耐潘大人,勿是为啥铜钿,也勿是为啥势利。格辰光倪搭耐刚刚碰头,心浪向就有仔耐实梗一个人,一径丢耐勿脱。耐吃仔一台酒,一径勿来,倪心浪末牵记煞,面孔浪末说勿出。倪碰着格客人几几化化,一塌刮仔才勿来浪倪心浪。独独看见仔耐,像煞心浪有一种说勿出格念头,总归耐说一句闲话,跑一步路,你看仔总归呒啥勿对劲。格个里向,连搭仔倪自家也说勿出是啥格讲究。直到仔今年马路浪碰着仔耐,承耐格情看倪得起,搭倪也蛮要好,别人家看仔倪两家头总说呒啥希奇,洛里晓得倪心浪格事体。老实说,耐要倪那哼,只要耐说一声,倪总呒啥勿肯。故歇耐晓得倪欠仔亏空,搭倪还债,拨别人家看起来,好象倪搭耐要好才是假格,为仔自家欠仔别人家格债,呒说法洛,有心骗耐搭倪要好,叫耐搭倪还债。耐想拨俚笃一说,倪阿要难为情。就是耐自家心浪想起来,也要勿相信格呀!总当仔倪搭金姐两家头串通仔调耐格枪花,倪就生仔一百张嘴,也搭耐讲勿明白嘛。实梗洛倪情愿自家去想法子,勿要搭倪还啥格债,等别人家看看倪到底阿是格号只认得铜钿勿认得人格人。”

这几句话儿,真个说得来恩上加恩,爱中添爱。潘侯爷听了,不由得满面添花的道:“你的话虽然不错,但是你现在欠着别人的债项,这是讲不来的。我不知道也还罢了,我既然知道了,这件事情那有不和你还的道理?若是你一定不肯受,那就倒反不是真心和我要好,好象是假意撇清的了。”沉二宝听了,低着头沉吟一会,叹一口气道:“说起来,倪做仔生意,客人拨倪洋钿,阿有啥勿要格道理?不过今朝拿仔耐格洋钿,拨别人家说起来,总归说倪有心做仔圈套,敲耐格竹杠。轧实倪搭耐两家头要好,是样式样对劲仔格要好,勿是为啥洋钿勿洋钿。故歇实梗一来,像煞仔倪想耐格洋钿洛,格外巴结。轧实倪也勿是格号勿要面孔格人,耐也勿是格号碰碰上当的曲辫子,俚笃洛里晓得?”

潘侯爷听沉二宝说他不是轻易会上当的曲辫子,心上更觉合拍,便又对他说道:“你的话儿都是多虑,别人说你不是真心和我要好,只顾凭他们去说就是了。只要我自己心上明白,别人的讲论何必再去管他?如今你的真心我也知道的了,若要叫我看着你欠了一身的债,不来和你想个法儿,非但我心上过不去,你叫我的面子上也怎么的下得去?你们当倌人的人若真个一个钱不要,又何必要做什么生意?”

沉二宝正色道:“潘大人,耐倒勿要实梗说。倪吃仔格碗把势饭,做客人也有几等几样做法格呀!老实搭耐说,格个客人要是搭倪勿对劲格,等俚去多用脱两个铜钿,心浪像煞开心点。碰着仔搭倪对劲格客人,像煞俚多用仔一个铜钿,倪心浪总归有点勿舒齐。勿是啥吃仔把势饭,就拿铜钿买得动格。买倪格身体倒呒啥希奇,要买倪格心倒勿容易嚏!耐总当仔倪做倌人格末,总归只认得铜钿,勿认得交情,格末耐真正看错仔人哉!”

潘侯爷听了,连忙走过来对着沉二宝打了一拱道:“我的不是,说错了一句话儿,不要生气。”沉二宝忍着笑别转头去,道:“勿要嘘,算啥格样式呀!”潘侯爷又道:“你一定不肯受我的钱,我也没有别的法儿。我如今只有两条道路,凭你自家去拣。你若是不愿意我在你院中走动,你就不要受我的钱,我从今日起再也不来的了;你若是愿意我来走走的,你就老老实实的受了,不必和我客气。”沉二宝听了,呆了一回,方才说道:“格末真正也叫呒说法,耐说到仔实梗闲话,叫倪那哼再好勿受?”说着,便把那一张四千块钱的汇票接了过来,对着潘侯爷笑道:“谢谢耐!”潘侯爷也笑道:“今天这一张汇票,我不知费了许多的气力,说了无数的话,你方才肯赏我的光收了下来,我还要谢谢你呢!”沉二宝也微微一笑。

看官,你道沉二宝的这一篇反扑文章,可做得利害不利害?凭你潘侯爷这样的精明漂亮,也不因不由的一头钻进了他的圈套,一时间那里看得出来?自此以后,不到三个月的工夫,沉二宝的亏空都已经还得清清楚楚,头上手上的首饰金珠翡翠办得件件俱全,身上的衣服更不必说。论起理来,这个沉二宝以前上了姘戏子的这般恶当,几乎落在帐房里头,跌到么二上去。幸亏想着了个潘侯爷,居然被他钩上了手,做了他一个大大的救星,一节不到,差不多用了八九千块钱在他身上。在下做书的和他想起来,该应改悔前非,死心塌地的守着潘侯爷才是。那里知道他饱暖思淫,清闲不惯,以前为着姘戏子碰了这样的一个大钉子,他却一些儿警忌的心都没有。到了如今,亏空刚刚还掉,手里头才多了几个钱,不由得又想起那旧日的营生来,偷偷的瞒着潘侯爷,自己一个人到戏园里头去看戏,刚刚又是孽缘天凑,碰着了这个谢月亭。

沉二宝自从一见谢月亭之后,便眠思梦想的,害了个闻声对影的单相思。茶里也是谢月亭,饭里也是谢月亭,一天到晚只把个谢月亭的形容放在心上,车轮一般的旋转。就是见了潘侯爷,也有些失神落智的样儿。潘侯爷虽然有些觉得,只说他或者身体有什么不爽快,方才是这个样儿,便问他为什么这般模样,身体觉得怎么样。沈二宝随口支吾了几句,一心一意只想着个谢月亭一个人。想来想去,想不出个引他的法子,便硬着头皮,在戏园门口候着谢月亭出来,一把拉住了他,试他一试。虽然知道谢月亭的父亲管束得十分严紧,却只说不见得一天到晚看守住了这个儿子,不分好歹,且去碰个机会再说,或者竟会成就了好事也未可知。那里知道偏偏运气不好,遇见了谢云奎,受了他一场抢白。

回到公阳里院中,长吁短叹的好似失了心的一般。听得大姐阿招叫他起来,他赌气不答应。阿招一连叫了几声,发起急来,潘侯爷早已走上扶梯。沉二宝起先在公阳里的时候,本来是楼下房间,如今做了潘侯爷以后,便搬到楼上去,三间楼面都是沉二宝一个人的。当下阿招见沉二宝睡着不肯起来,心上十分着急,只得高声说道:“潘大人要动气格呀!”这个时候潘侯爷已经走进房来,见了沉二宝睡在那里竟不起身,心上也觉得有些不快,便对阿招说道:“他起来不起来,凭他的便,你去叫做什么!”

沈二宝听得潘侯爷发话,心上有些忐忑,便趁着阿招推他,一骨碌坐起身来,故意嗔道:“耐嘤嘤喤喤吵啥物事?潘大人来末,让俚来末哉嘛,俚咦勿是啥今朝头一转来格生客,要耐来浪发啥格极呀!”说着,便回过头来,对着潘侯爷说道:“耐听听看,俚笃赛过来浪当耐生客,阿要讨气!”潘侯爷见沉二宝睡着不理他,只说他有心怠慢,正要发作,听了沉二宝这几句话儿,不知怎样的方才心上的气不知走到那里去了,顿时盛气齐平,一言不发,欢欢喜喜和沉二宝谈了一回,方才就寝。

这里潘侯爷和沉二宝的事情姑且按过,再讲起那位从天津回来乡试的章秋谷来。

章秋谷自从在天津回来,回到新马路自己家中,见了太夫人和夫人并陈文仙等,自然大家甚是欢喜。这个时候已在七月十五之后,秋谷知道,要回到常熟本籍起了录遗文书,再到南京去录遗,是来不及的了。便去商约大臣陈荫孙陈宫保那里,求他起一套送考的咨文。这位陈宫保本来和章秋谷是同乡,又彼此都有了世谊,自然一口应允。隔了一天,果然就差一个差官送了一件咨文过来。秋谷接了这口咨文,免不得又自己去陈宫保那里道谢。陈宫保倒着实和秋谷谈了一回,见秋谷口如悬河的滔滔不绝,不由得心中暗暗称奇。秋谷谢过了陈宫保,正打算动身赴试,不想平空的有个岔子出来。正是:

相如善病,茂陵秋雨之宵;樊素多情,绮阁春风之夜。

不知后事如何,且待下文交待。





第一百六十九回 阻观光无端婴小极 喜同心着意护檀郎





且说章秋谷在家里头住了几天,正要动身到南京去,不想平空的忽然害起病来。

原来章秋谷素来怯热,到了夏间最爱吃那大莱馆里头的冰忌濂。只说这样东西十分爽口,到了嘴里头真个是凉沁心脾,寒凝齿颊,比那冰水浸的瓜果更觉得爽口些儿。

在上海的时候差不多天天要吃的,吃得多了,未免寒气凝积在脏腑里面发泄不出来。

到了秋天一定要啾啾唧唧的害些小病,秋谷也不去管他。此番由天津回来,在船上的时候天气正是十分炎歊,秋谷晚间睡觉,把那官舱里头的玻璃开得直直的,着实受了些海面上的风寒。到了上海,多吃了些冰忌濂。他夫人和陈文仙那里,檀郎久别,凤女多情,想来未免要接一接风。

偏偏这一个立秋很早,到了七月二十的那一天,便下了一场大雨,金风萧瑟,枕簟生凉,把一天暑气都赶得干干净净。章秋谷却就在这几天之内生起病来。二十二的那天晚上,章秋谷把书籍行李都收拾得停停当当,预备着明晚下船。那里知道到了二十三早上,章秋谷刚刚起身,便觉得身上有些不自在,眼花头晕,立脚不定。

章秋谷本来自己也懂些医道,他太夫人的医理也狠有些门路的,当下太夫人诊了秋谷的脉,知道是发寒热,便叫他在榻床上睡下,取了一条夹纱被,和他盖在身上。

直到夜间两点多钟,头上的热方才退清楚了,微微的出了一身汗。章秋谷自觉身躯疲乏,吃了一碗稀饭,便也上床睡了。

到了明天,章秋谷的寒热又来了,比上一回却觉得重了些儿。他太夫人等他退热之后便和他商议,叫他南京不必去罢,就错过于一场乡试,下科再去就是了,也算不得什么事情。依着章秋谷的性情,看着这个举人进士的功名本来原是可有可无的,所以在天津几千里路的赶回家来,一定要去乡试,原为着这位太夫人期望甚深,不容不去。如今听了太夫人这样的和他商议,自己也觉得有些支持不住,便对着太夫人道:“虽然错过一科没有什么,但是可以支持得来,还是去的为是。明天只要这个劳什子的寒热不来,立刻赶上船去,还赶得上学台的录遗,再迟就赶不上了。”

太夫人笑道:“你就是明天好了,我也不放心叫你一个人上路。你不要把我也当作那班势利龌龊的人,把功名富贵看得十分郑重。在我心上看起来,看着这个举人进士倒也是狠平淡的。不过你们章氏世代簪缨,门承通德,不得不在这里头图个出身就是了。”秋谷听了也笑道:“既然母亲决意如此,儿子一定不去就是了。”

太夫人又笑道:“若是我一定要逼着你扶病出门,不要说别的,只你这两位夫人只怕心上就要不快活了。”陈文仙在旁听了,微微含笑,也不作声。秋谷也笑道:“这个他们怎敢?”说着,太夫人见秋谷有些疲乏的样儿,便吩咐了陈文仙几句话,叫他好好招呼,自己便回房去了。

那里知道章秋谷的这个寒热发得甚是蹊跷,吃了几服药,非但不见一些儿功效,倒反的一天重似一天起来了。上一次的余热未清,接着第二次的重寒又至,到了后来竟是热得发狂谵语起来。只把一个章秋谷的夫人和陈文仙吓得个魄散魂飞,六神无主,只说这样的病热是有些尴尬的了。两个人衣不解带的昼夜伏伺,却一天到晚的愁眉泪眼,着急非常。还是章秋谷的那位太夫人,见了章秋谷这般病势,虽然心上有些焦躁,却究竟在脉理上有些把握,知道这个病是没有性命之忧的。见了他们两个人急到这般模样,免不得安慰一番,叫他们不要着急。这两个人听了略略放心。

章秋谷整整的病了两个礼拜,方才寒热来得轻些。他夫人和陈文仙两个却整整的伏伺了两个礼拜,这两个礼拜里头茶饭无心,梦魂不定,真累得这两个花容憔悴,神彩疏慵。

这一天章秋谷睡醒热退,睁开眼睛在床上四围一看,只见他夫人坐在床沿上,拉着他的手紧紧的贴身坐着。陈文仙却坐在里床,捏着一只粉团一般的拳头轻轻的和他捶腿。见秋谷睁开两眼,他夫人便连忙把手到他额上去试了一试,觉得余热已退,便问道:“你这个时候心上觉得怎么样?”秋谷道:“这个时候倒觉得狠爽快。”

他夫人便去倒了一杯温凉可口的洋参茶来。秋谷觉得寒热已经退了,便一谷碌在床上坐起。他夫人连忙要来扶他,秋谷摇头不要,接过洋参茶一饮而尽。陈文仙对着秋谷笑道:“你寒热才退,再睡一回儿养养精神也好。”秋谷道:“这个时候我觉得精神狠好,头目清凉,坐一回儿不妨。”

说着便抬起头来看了他们一会,觉得他夫人和陈文仙两个人的脸上比以前瘦了好些,狠有些楚楚可怜的丰致。从前是红衬湘桃,花呈妙靥,如今却是六铢衣薄,掌上身轻了。秋谷知道自己寒热来得利害的时候,他们两个人都是通宵彻旦的伏伺,心上十分感激,却对他夫人和陈文仙笑道:“我害了十几天的病,把你们两个人都累坏了。多谢,多谢!我心上感激得狠!”他夫人听了,握着他的手道:“阿弥陀佛,真正谢天谢地!如今巴得你渐渐好起来,我们已经心满意足的了。你生了病,我们伏伺你,这是我们做妇女的分内事情,那里当得你这般客气?难道我们还用得着客气么?”说着不觉一笑。

陈文仙也道:“如今你的病渐渐见轻,真是大家的运气。那几天寒热来得最重要的时候,昏迷不醒,连人都认不得,真是人都吓得死的!我生长二十岁,还是第一次受着这般的惊吓。如今我们虽然一般在这里伏伺你,心上却是十分宽畅。比不得那几天,真是急得上天无路,人地无门,替又替你不来。吃了药下去,仍没有一些儿效验。你想那个时候,怎样的叫人不要着急?如今幸而天地保佑,祖宗灵感,你的寒热也渐渐的退了,病也渐渐的轻了,我们心上一块石头也落下地了。至于你为着我们在你病中伏侍了你,你平空的忽然的和我们客气起来,那是再也不敢当的。

只要你以后处处自家保重身体,不要叫老太太和我们耽惊着急,我们就是不论怎么样,心上也是高兴的。辛苦些儿算得什么。“说着,也是横波一笑,目光澄澄的看着秋谷,好象要说什么话儿,却又没有说出来。秋谷听了陈文仙的这一席话儿,自然点头道是。他夫人听了,也不由得连连点头道:”二妹的话儿一些儿都不错,你以后自家要保重些儿才是。“

原来秋谷的这位夫人自从陈文仙进门之后,见他和婉非常,温柔有礼,两下谈论起来竟是二十四分的要好。陈文仙虽然不敢越分,这位秋谷夫人却早已和他姐妹称呼的了。当下章秋谷听了他夫人的话,也不开口,只把头略略的点了一点,却把左手挽了他夫人的手,右手握着陈文仙的手,三个人六只眼睛,就如闪电流光的一般,你看着我,我看着你,深深凝睇,脉脉含情,大家都觉得有无限的深思厚爱,在眼光中间流露出来。三个人无言相视了一回,秋谷觉得坐在床上不耐烦,便跨下床来走了几步。陈文仙恐怕他病后力弱,连忙拉着他的右手,紧紧的贴身扶着他。

章秋谷也觉得头目森然,脚下无力,便随意躺在榻床上,和他们两个人讲些闲话。

一会儿,太夫人走过来看他,见他精神甚好,也自欢喜。

自此以后,章秋谷又在家里头一连养了半个月的病,方才精神复旧,二竖潜逃。

这半个月里头在家里没有事情,一天到晚除了陪侍太夫人讲些闲话之外,成天的只和一妻一妾相对,喁喁对语,款款相偎,纤手扶搔,芳心熨贴。茗碗药炉之畔,搀和着许多的粉晕脂痕;添香伴影之宵,平添出无限的幽欢密爱。章秋谷虽然在家养病。却倒享受了许多的艳福。从此以后,章秋谷和妻妾的恩爱平空的又添了几分。

到了中秋节后,章秋谷已经照常出门。辛修甫和王小屏两个听了秋谷病愈,便两个人同着来看他。秋谷和他们谈了一回,辛修甫和王小屏为着他错过了乡试,甚是替他可惜。修甫道:“如今乡试改了策论,你是向来留心古学的,一定可以有些把握,可惜你又偏偏生起病来!”王小屏也道:“你这一场病生得真是凑巧,早不生病,迟不生病,偏偏的正在那几天录遗的时候生起病来,眼看着一个举人生生的送掉了,岂不可惜!”

秋谷笑道:“承你们两位这般关切,足见盛情。但是据我想起来,现在的这般时局,国势阽危,前途黑暗,这个举人就使中了,也没有什么道理。我的性情你们是知道的,本来不把功名不功名的事情放在心上,就是错过了也算不得什么。”辛修甫道:“虽然如此,但是如今这般势利卑鄙的时代,中个举人却要占无数的便宜,你也不要把举人看得这样的一个大钱不值。”秋谷笑道:“你们两位都是举人出身,我也不是一定把举人、进士看得一文不值。但是一个人的声价,是从学问经济上来的。一个人只要有了真学问真经济,就不中举人、进土,他的声价也不见得就会低些。那一班没有学问的饭桶,就是中了举人、进士,依然还是一个庸庸碌碌的饭桶。

照这样看起来,这个举人又何必一定要中他呢?“正是:

高谈惊座,春生舌本之莲;往事如烟,肠断秋娘之泪。

不知以后如何,且待下文交代。





第一百七十回 发清言高论寄牢骚 访桃源良朋联伴侣





却说辛修甫和王小屏听了章秋谷的话儿,辛修甫便又向他说道:“你的话虽然不错,无奈我们既然生在这般卑鄙龌龊的时代,大家都把这个举人、进士当作一件最宝贵的东西,这个举人、进士便也自然而然的做了读书人必不可少的对象。即如你具着这般雕龙绣虎的才华,又怀着这般治世长民的经济,功名的两个字儿自然不放在你心上的了。但是你平日之间常常的对我们说,大丈夫不能独当一面,建节拥旄,便当为节度参军、平章幕府,庶几虽然不握大权,还好借着这个机会做些事业。

照你这般说起来,如今只要有个督抚大员来请你当个幕府,你是一定愿意的了。但是如今的那班督抚,也都是些以耳为目、不分黑白的人。若是放着个一窍不通的太史公或者进士公在那里,再放着个才学兼优的你在这里,两下比较起来,你看他还是愿意聘请个有功名的太史公、进士公,还是愿意聘请个没功名的你?你只要这般一想,就知道这个举人、进士也不是当真没用的废物了。“

章秋谷听了,笑着说道:“承你这般谬赞,把我说得这般的才学兼优,只怕你未免有些违心之论罢。”辛修甫道:“我倒不是违心之论,只怕你倒有些拂意之谈。

如今闲话休提,你只说我的话儿究竟可是不是?“秋谷想子一想道:”就大势看起来,自然是你的话儿不错。如今的那些督抚部院的大员,都是庸庸碌碌的多。矫矫铮铮的少。但是十步之内必有芳草,十室之邑必有忠信。现在的大员里头也未始没有爱才如命,求贤若渴,和毕秋帆、林则徐、尹继善一般的人,不过我们没有遇着就是了。大抵这样的人自然的腹有经纶,胸藏韬略,秉天独厚,得气之清,和那班酒囊饭袋的督抚不同。所以他看起人来也能独具只眼,拔英雄于未遇之时,识豪杰于穷途之会,卑躬屈己,任贤使能,自然的就能功盖国家,泽及百姓。这样的人,我们当他的幕府,借着他的力量,自然好做些事业出来。若是那种瞎了眼睛,全无经济的督抚,我们就使在他的幕府里头,他也未见得肯听我们的话儿,我们也未见得做出什么事业。像这样的人,本来只认得翰林、进士,那里晓得什么叫做学问,什么叫做经济?这样的去取,那里有什么声华价值?我们躲着他还恐怕来不及,那里还肯去当他的幕府?“

王小屏和辛修甫听了章秋谷的这番议论,心上十分叹服。辛修甫便点一点头道:“你这番议论真个痛快非常。但是你把那班酒囊饭袋骂得未免过分了些。万一给人听见,传到这一班宝贝的耳朵里头去,一定要把你当做个不共戴天的仇人一般,你也何苦去做这样冤家呢?以后我劝你还是收敛些儿,不要这般的冲墙倒壁,无故骂人,这才是个明哲保身的道理。”秋谷听了修甫这几句劝他的话儿,觉得心上悚然一动,对着修甫拱一拱手道:“你劝我的说话真是金玉之言,我以后自当谨慎。

但是我方才的话儿原是平空发论的,并不是有心骂人,况且我也不是把他们那班做大员的人一笔抹倒,把他们看得没有一个好人,也不过随口说说罢了。多谢良言,永当铭佩。“王小屏听了接口笑道:”你向来是个狠豪爽的人,怎么如今似变了一个人的一般,文绉绉的这般客气,这是什么道理?“

章秋谷听了,不觉有些好笑起来。正要开口,王小屏又对他说道:“闲话少说,你可知道我们今天到你这里来,是为着什么事情?”秋谷道:“你们两位大概是听说我近来在家养病,所以跑到这里来看我一下,想要和我谈谈,可是不是?”辛修甫道:“我们今天的跑到你这里来,虽然也可以算得是为着问病来的,却究竟不是我们心上的事情。你在上海多年,你可知道有个卧云阁在什么地方?”秋谷听了,不知道他们心上是一件怎么的事儿,更兼满肚子里想不出这个卧云阁是个什么店号,沉吟了一会道:“这个卧云阁,我实在肚子里头想不起来,你要问这个卧云阁做什么?”王小屏笑道:“你这个人岂有此理!怎么记忆力竟是这般不济?去年十二月里头的事情,难道就当真忘了不成?”秋谷听了,兜的把这件事儿提上心来,方才恍然大悟。

看官,你道究竟是怎样的一回事情?原来章秋谷去年十二月在一品香遇着一个少妇,看他的年纪却差不多已经有二十八九岁的样儿,却生得身段玲珑,丰姿活泼。

那一双俊眼闪闪烁烁的,波光飞舞,流动非常,好似那两丸水银、一汪秋水,觉得别有一种飞扬流丽的丰神。秋谷看了他一眼,不觉心中一动,暗想这个人虽然年纪大些,身段却着实不差。想着便不由得回过头来去再看一眼。那少妇正从扶梯上缓缓的走上楼来,忽见第八号门内立着一个二十上下的美少年,细腰窄背,白面朱唇,气概轩昂,仪容俊伟,端端正正的和他打了一个照面。那少妇见了心上也不觉跳了一跳,把头一低,走了过去。心上暗想:这是个什么人?觉得眼睛里头从没有见过这般人物。心上这般想着,便也不因不由的回过头来,刚刚的又和秋谷打了一个照面。两下的眼光一对,那少妇不觉面上一红,急急的别转头去。走到第十一号房间门口,又回头瞟了一个眼风,便轻移莲步,走了进去。

章秋谷看丁,心上狠有些儿摇动,便也跟着他走到第十一号房间门外,有意无意的立定了脚,往里一张。只见那少妇同着一个滑头滑脑的少年男子并肩促膝的坐在一处,正在那里交头接耳的不知说些什么。秋谷见了,心上暗暗的好笑,知道他们两个人也不是什么好勾当,便趁着他们两个人都没有看见,连忙缩了回去。回到房内,正见侍者拿着一瓶克里沙进来,秋谷便问他:“十一号里头的那个少妇,你认得不认得?”侍者笑道:“这个人就是大马路聚贤坊卧云阁的女东家,上海租界上狠有名的一个私货。怎么章老爷倒不认得?”秋谷听了,方才知道就是卧云阁烟灯的女东家,以前也听见别人说过有这样的一个人。暗想这个人倒狠不差,看着他这样的身段圆融,秋波宛转,他一定是风情旖旎,格调温存。几时倒要去赏识赏识他,看究竟是怎样的一个风味。

隔了一天,章秋谷便想要到卧云阁去请教请教这位女东家,便邀着辛修甫、王小屏、刘仰正,四个人一起同去。到了卧云阁门口,只见是个两楼两底的住房格式,下面两间横七竖八的铺着几张烟榻,许多短衣窄袖的人横在榻上吸烟,吸得烟雾腾腾的。章秋谷和辛修甫等看了这般模样,如何坐得下去?正想回身走出,只见屏门背后走出一个少妇,把他们几个人看了一眼,就满面堆下笑来,口中打着一口绝圆的苏州白道:“唔笃几位阿是来吃烟?问搭地方龌龊煞格,阿要到楼浪去罢?”间秋谷一眼看去,果然就是昨日在一品香相遇的人。听得请他们到楼上去,便对着众人把手招招,跟着那少妇一同走上楼去。那少妇高高兴兴的在前引导。

走到楼上,也是一并两间。那少妇同着秋谷竞走到自己卧房里去。秋谷等举眼看时,见一房间都是红木器具,铺设得狠是整齐。靠窗一张红木烟榻,明晃晃的点着一盏烟灯。那少妇请他们坐下,叫一个小大姐倒上四杯茶来,自己又拿出一付烟具来摆在大床上,点好了灯,对着秋谷笑道:“请靠歇吃筒烟哩。”秋谷摇手道:“我们都不吃烟的,你不用让我们,你自己请罢。”那少妇对着秋谷把嘴唇动了一动道:“倪也勿吃格呀。”说着,便问四个人尊姓。秋谷一一和他说了,不免也问问他的来历,那少妇也一一和他们说了一遍。原来这个少妇本来是常熟人,娘家姓尹,是个江苏候补道的姨太太。后来男人死了,大太太分了几千银子给他,把他打发出来。如今没奈何,只得在这里开个烟灯,暂图糊口。正是:

多情杨柳,谁怜昔日之腰?薄命桃花,莫问东流之水。

不知后事如何,请待下回交待。





第一百七十一回 证心期三生传慧业 听眉语一晌醉风情





且说那位卧云阁的女东家,把自己的出身来历约略和章秋谷等讲了一遍。说到那身世飘零之处,不由得有些凄楚起来,低着头叹一口气。章秋谷便走过去,握着他的手,上上下下的打量一番,喝一声采道:“好得狠,真是个绝代佳人,将来不知道那一个人有福消受你这样的一个人呢!”那女东家听了脸上一红道:“倪是老太婆哉,啥格好呀!”说着,却把章秋谷的手紧紧的握了一握,笑盈盈的飞了一个眼风。秋谷也还飞了他一眼。正在有些意越神飞之际,忽然听得楼下人声鼎沸起来,许多人的声气闹成一片。

章秋谷和辛修甫等都吃一惊,大家立起身来,问楼下什么事情。那女东家按住了章秋谷道:“俚笃格排流氓坯,一径是实梗格。呒啥事体,唔笃坐末哉。”秋谷听了把眉头皱了一皱,正要开口,忽然又听得楼下的那几个人大嚷大笑在那里讲话,讲的话儿一句句的听得十分清楚。只听得一个人笑着说道:“今天老二找着了主顾,这个老枪的身段却着实的不差,今天晚上广东货吃了。”说罢,大家都拍手打脚的哈哈大笑,闹得个鸦飞雀乱,烟起尘喧。这个女东家听了这几句话儿,不由得脸上一阵阵的红起来,含羞带笑的对着章秋谷说道:“耐听听看,格排杀千刀阿要面孔,随便啥格闲话总归说得出格。”

章秋谷的性情本来最恨的喧嚣烦嚷,最喜的沉静清闲。方才进门的时候,看着那些吃烟的人都是些不三不四的流氓,连一个规规矩矩的人都没有在里头,就有不愿意进去的意思,却被这位女东家自己走出来,把他们邀上楼去。章秋谷虽然跟着他一同上去,心上却暗暗想道:这个地方,那班来的人未免太庞杂了些,不是我们可以常常来的。如今听得楼下喧扰到这步田地,那里还坐得住,便急急的立起身来要走。那女东家一把拉住了秋谷的衣服,再也不放,只问他为什么要去。章秋谷对着他把头摇了一摇,也不说别的,只说我们有要紧事情去了,改日再来。那女东家听了,明知道是为着方才楼下喧闹的缘故,所以急着要去,心上十分不舍,便低低的对秋谷道:“耐阿是嫌比倪搭地方龌龊,坐才勿肯坐歇?倪要搬场哉呀,搬仔场蛮清爽,呒拨啥别人来,耐要来格嘘!勿然末倪一淘吃大菜去阿好?”秋谷听了,知道他有心俯就,便去他耳边低低的说了几句。那女东家呆了一呆道:“格末耐几时有工夫呀?”秋谷道:“明后天有空就来。”那女东家又拉着秋谷道:“耐勿要骗倪呀!耐骗仔倪,是倪勿来格。”秋谷道:“这个自然,那有哄你的道理?”

辛修甫见了微笑不语。王小屏见了便哈哈的笑起来,对着章秋谷扮个鬼脸道:“你吊膀子的本领着实不差,我们和你在一起吊膀子,总吊你不过,这是个什么缘故?”那女东家听了把头一扭道:“啥格吊膀子勿吊膀子,倪才勿懂格。”王小屏笑道:“你懂也罢,不懂也罢,停几天你们两个人做成了交易,看你再说不懂!”

那女东家听了着实的有些不好意思,要说什么却又没有什么说的,只得别转头去,洋洋的笑道:“倪一塌刮仔才勿晓得,耐去瞎三话四,勿关倪事。‘’王小屏正还要和他取笑,章秋谷连忙对他摇一摇头道:”算了,算了,我劝你少说几句罢。

“王小屏笑道:”阿唷!你们大家看看,刚刚吊膀子吊得有些意思,就这般舍命相帮。我也劝你将就些儿罢。“说得大家都哈哈一笑。

章秋谷道:“你要和他闹俏皮,讲笑话,听你一个人坐在这里,慢慢的闹你的就是了。我们却没有工夫奉陪,要先走一步了。”王小屏把舌头一伸道:“那还了得!这个人已经是你的禁脔,我就有天大的胆量,也不敢挨他一下。万一个你和我吃起醋来,你的气力又大,拳棒又精,我区区鸡肋,那里当得起你的尊拳?给你一拳打死了,叫我到那里去叫冤?”这几句话儿,说得连女东家也笑起来。章秋谷笑道:“这个时候,我也没有工夫和你斗口。”说着便走过去,一把拉着王小屏的手往下便走,好似提着个小鸡一般。王小屏连连叫道:“我走,我走,你不要动手!”

秋谷听了,方才放手。大家走下扶梯,那女东家竟送下楼来,直送到屏门外面方才回去。到了明天,章秋谷把这件事儿不知道忘到什么地方去了,竟从此没有去过,也从此没有见过这个人。

如今听得王小屏提起去年旧事,心上方才想起这个人来,便也笑道:“怎么我如今的记忆力竟弱到这般田地,竟把这件事儿遗忘得干于净净?不是你们提起,我那里还想得出来。但是这个人,我自去年直到如今一径没有见过他的面,可不知道这个时候还在大马路不在大马路?”王小屏道:“老实对你讲了罢,我和修甫昨日两点钟到南诚信去找个朋友,恰恰的就遇见了他。我和修甫和他只见过一面,模模糊糊的一时记不起来,他却不知怎样的,一见了我们两个就认得我们是和你一起的人。我们倒和他谈了半天,他说如今搬到法马路去了,再三再四的和我们说,要请你去一趟。今天下午四点钟,他在南诚信老等,等候我们去了,大家一同到他那里去。在我们面前说了许多好话,一定要我们和你同去,说是有什么紧要的话儿他要和你说。我和修甫倒一口答应了他,讲明今天和你一同到南诚信去,所以我们两个人特地前来奉邀同去。这个时候已经差不多有三点多钟,我们就此起马何如?”

秋谷忽然笑道:“我倒忘了,还没有和你们贺喜。”辛修甫和王小屏都愕然不解道:“我们有什么喜事,要你贺喜?”秋谷笑道:“你们两个新做了卧云阁女东家那里的相帮,头衔新晋,封号荣加,堂堂的二品封典,松翎绿顶,荣耀非常,怎么不要和你们贺喜呢?”这几句话,把辛修甫和王小屏说得都狂笑起来。王小屏笑着说道:“你这个人委实的可恶,我们辛辛苦苦的和你带了一个信,不指望你的酬谢罢了,倒反要取笑我们!把我们当做烧汤乌龟,天下那有这般情理?”章秋谷笑道:“你们既没有当他的相帮,为什么要拼命的和他拉客人?这叫做箭在弦上,不得不发!”

修甫微微一笑,对着秋谷道:“我们已经来了多时,骂也给你骂了,取笑也给你取笑了,我们就算是个相帮,来请你这个客人的,就请你和我们一同去罢。”秋谷慢慢的笑道:“这几句话儿不过大家打个哈哈罢了,也不是安心要骂你们。”王小屏连忙拦住他道:“走罢,走罢,不用讲闲话了!”秋谷故意问道:“走到什么地方去?”王小屏听了嚷道:“你不用装胡涂,装胡涂也不中用!”秋谷笑道:“我不是装胡涂,委实这几天还不能出门,只好改天再奉陪你们的了。”王小屏道:“你要说谎也不是这般说法的。你说这几天不能出门,昨天晚上在陆丽娟那里吃晚饭的是那一个?”秋谷笑道:“昨天觉得精神好些,所以到丽娟那里去坐一回儿。

今天忽然又觉得精神不济起来,所以不能出门。这个算不得说谎。“

王小屏听了,一时说不出什么来,只得说道:“我们昨天已经一口应许了他,一定和你同去。今天无论如何也要委屈你些同去一趟的了。”秋谷听了便立起身来,对着王小屏打了一拱道:“对不起,我今天当真不能出去,先给你陪个礼儿好不好?”

王小屏听了,不由得心上有些着急起来,道:“你的去不去不干我事,但是我昨天在他面前拍着胸脯一力担承的,今天你不肯去,好象面上有些不好看。更兼他和我当面说明,只要把你同到南诚信去,便重重的送我一分酬仪。如今你不去,连我的酬仪都不得到手了,这便怎么样呢?”秋谷听了一笑,也不开口。

辛修甫对着王小屏笑道:“怎么你这样的一个人也忽然胡涂起来?这样就口馒头的事情,他那里肯不去,不过口中说说罢了。”王小屏听了恍然大悟,也笑道:“我只为急于要得他的谢仪,就连这件事情的利轻利重都忘了。这件事情在他身上是大有便宜的,我不过想得些表面上的利益就是了。只想着自己身上的便宜,却忘了别人身上的利益。这样一件小小的事情尚且如此,怪不得如今的那班饭桶办起公事来,只知道一味的拼命要钱,却不顾以后的许多祸患。‘利令智昏’,古人的说话果然不错。”秋谷笑道:“讲讲闲话,忽然发出这样的大议论来,足见你是个古文家,讲的话儿都是胎息《史》《汉》的。”王小屏不觉笑道:“算了罢,不用俏皮了。你要是去的,我们就一同去;你若是不去,我们就对不起,要少陪了。”

秋谷不语,却把桌子上的电铃一按,“噶啷啷”的响了一阵。门帘起处,便走进一个家人来,秋谷叫他去取件夹纱马褂出来。辛修甫便向王小屏道:“何如?我就知道他不肯不去的。”秋谷微笑不语。一会儿马褂取了出来,三个人一同出门,各人坐上包车,不到一刻,早已到了法大马路南诚信门外。

原来这个南诚信是个绝大的广膏烟灯,却是个住家野鸡的总会。上海的那班野鸡妓女,只有那些住家野鸡里头着实有几个出色的,大马路长裕里头的已经差了好些,那些在四马路拉客人的野鸡妓女都是些下等的蹩脚货。所以上海那班爱打野鸡的人,略略上等些的,都是到南诚信去细细的物色那班住家野鸡。每天下午四点钟的时候,那些野鸡妓女便接踵而来,老的少的,妍的媸的,似海滩上晒蚌蛤的一般,挤得个层层叠叠。章秋谷等来的时候,正是那班野鸡妓女上市。章秋谷刚刚走到第二层楼上,早见迎面走过一个三十多岁的丽人来。正是:

绛唇珠袖,十年烟月之狂;泥玉焚兰,一觉风尘之梦。

不知以后如何,且待下文分解。





第一百七十二回 赋皇华小星随使节 开绮席大尉遇佳人





且说章秋谷同着辛修甫等走到南诚信第二层楼上,蓦然见一个三十多岁的丽人从斜刺里慢慢的走过来。秋谷远远的看着,只说就是那位卧云阁的东家,紧着抢过几步,想要和他说话。那里知道走到面前,两下的眼风刚刚碰了一个针锋相对。那丽人见了秋谷,秋波一定,好象要和他说话的一般。秋谷见了不觉呆了一呆,原来不是那位卧云阁的东家,别是一个袅袅婷婷的少妇。只见他身上穿著一件湖色熟罗夹袄,下着玄色绉纱夹裤,内家结束,雅淡梳妆。盈盈宝靥,经酣春晓之花;浅浅蛾眉,黛画初三之月。纤腰约素,莲步凌波,大大方方的走过来;没有一些儿小家子的气派,觉得另有一种雍容华贵的丰神,竟像个大家眷属一般。却是皱着个眉头,垂着个眼睛,无精打彩的好象有心事的样儿。秋谷和他擦肩走过,细细的打量一回,心中暗想这个人怎么这般面熟,看他这个样儿,一定心上有什么不得已的苦衷。红颜薄命,从古以来都是如此。

正在这个时候,早见那丽人忽然回转身来,抢行几步,把章秋谷等几个人着着实实的看了几眼,忽然对着辛修甫说道:“阿呀,辛老爷嘛!多时勿见,实头勿认得哉!”辛修甫也猛然想起道:“你是北京的赛金花!听说你吃了官事,回到苏州去了,怎么会到这个地方来?”赛金花听了,叹一口气道:“倪格事体,一时说勿尽几化,故歇就来浪格搭小房间里向坐歇,等倪慢慢里搭耐说。”辛修甫听了点一点头,便同着赛金花走到左首一间房内,大家坐下。章秋谷到了这个时候,方才也想起这个北京城中香名鼎鼎的赛金花来,便笑着对他说道:“你认得我不认得?”

赛金花看了秋谷一眼道:“面熟是面熟煞,想倒想勿出嘛。”秋谷笑道:“四年之前,你在天津东天保的时候,我在你那里碰过一场和。今年六月里头,你还没有闹那银翠的事儿以前,我同着一个姓姚的到你那里去过一次。只怕你贵人多忘事,记不得我这样一个人的了。”赛金花听了,又抬起眼睛来看了秋谷一眼,忽然面上一红道:“划一耐是章二少嘛!六月里向耐来仔一埭,一径勿来,倪末倒一径心浪牵记煞。”章秋谷笑道:“多谢,多谢!不敢当。”

王小屏在旁看了,“格”的一笑。赛金花乖觉。连忙说道:“耐也是一径照应倪格老客人,生来该应牵记格嘛,啥格客气得来。”说到这里,便又回过头来向辛修甫道:“说起倪格事件来,格末真正叫作孽。”赛金花说到这里,章秋谷叉口说道:“我自从七月出京以后,在天津听得你遇了官事,后来又听得说你回苏州去了,这个里头究竟怎样的一回事情?你何不讲给我们大家听听。”赛金花听了,便把自己的事情略说了一遍。

看官,你道这个赛金花究竟是什么人?原来这个赛金花,就是那以前的状元夫人傅钰莲、中间的江南名妓曹梦兰、后来的议和大臣赛二爷。在我们中国的历史里头,狠有些儿系属的。那傅钰莲在历史,有一部《孽海花》的小说里头,已敷叙得明明白白,把那位状元公改了个名字叫金雯青,把傅钰莲改了个名字叫傅彩云。后来这位状元公死了,这傅钰莲正是水葱儿的一般,水也掐得出的人,那里守得住?

那位状元公的太太也知道他万不是个守节的人,便给了他几千银子,好好的打发他出去。傅钰莲自从出来之后,便改了个名字叫曹梦兰,到上海去重做生意。枇杷花下,倒也车马如云,并不寂寞。这个傅钰莲本来是个色艺双绝的名妓,做起生意来自然十分顺手。一班客人知道他是那位殿撰公的姨太太,大家都还赶着他叫状元夫人,这状元夫人曹梦兰的声名便大燥起来。过了几年,曹梦兰的年纪渐渐的大起来,生意却渐渐的退起来。曹梦兰心中着急,听得人说天津地方的生意狠是好做,便又改了个名字叫赛金花,到天津去做了几年。果然香名大噪,着实多了几个钱。便买了几个讨人,到京城里头开了一家堂子,赛金花便做起本家来。

那一年联军进京,德国的华德生是个联军总统,赛金花听了这个华德生的名字,猛然想起以前的事情来。原来傅钰莲跟着那位殿撰公出使德国的时候,华德生还是个陆军大尉,在跳舞会里头见了傅钰莲,觉得眼睛里头从没有见过这样的丽人,心上十分羡慕。傅钰莲看着华德生也觉得有些心动。你爱我的英姿飒爽,我爱你的倩影娉婷,四目偷窥,两心互印,早已种下了一个相思种子在两个人的心里头。华德生看了一回,想要和钰莲讲话,无奈欧洲各国的礼法,男子见了女子,若没有相识的人介绍是不能冒昧自荐的。华德生徘徊了一会,恰恰遇着一个外务部的朋友和傅钰莲素来相识,华德生大喜,便托他做了介绍,和傅钰莲执手相见。傅钰莲的德语本来是狠好的,两下殷殷勤勤的谈了一回,脉脉深情,盈盈遥愫,眼波互证,心事交期。两个人虽然不说什么,心上恰都存着一个偷香窃玉的心期,送雨推云的襟绪。

从此以后,华德生便常常的和傅钰莲来往,傅钰莲也往华德生寓里头去了好几次。

至于他们两个人究竟有无暖昧的事情,在下做书的却没有调查确实,又没有自家眼见,不敢一定说是怎么样,只好付之缺如,作个疑案的了。

只说傅钰莲自从回了中国之后,和华德生两个人一个在亚洲之东,一个在亚洲之北,波涛万里,萧艾三秋,床空翡翠之衾,枕冷鸳鸯之梦,绣帏锁夜,宝鸭无温,未免觉得十分惆帐。起先的时候,两下还常有书信往来,直到那位殿撰公天上修文,傅钰莲风尘再堕,两止下方才绝了音信。如今听得联军的总统是华德生,不觉得旧梦重温,余情复续。却还怕这个华德生不是自己意中人,便写了一封德文信去给这位联军总统,问他是不是一千八百九十二年,在德国京城曾任陆军大尉的华德生,下面注了个傅钰莲的德文名字,想个法儿叫人送去。

这一封信去不多时,早见四个德国马兵牵着一匹空马,拿着一封华德生的回信来,给赛金花看了。那信上无非历叙如何如何的想念,怎样怎样的相思,如今得了他的消息,又怎样怎样的喜慰,请他立刻就到行营相见。赛金花看了来信,知道这个联军总统果然就是自己的意中人华德生,心上自然欢喜更喜他事融多年,地位又彼此大相悬绝。从前在德国相见的时候,一个是堂堂的公使夫人,一个是小小的陆军武弁,两下比较起来,还觉得傅钰莲的地位胜些。如今隔了多年,华德生已经升了陆军大将,此番奉命专征,又是各国公举的联军总统,威权赫奕,势位非常。更兼掌着全军的生杀大权,一个北京城都在他掌握之内,就是我们中国的大皇帝,到了这个兵败势危的时候也要让他三分。这个赛金花却是丽质埋尘,红颜薄命。飘茵堕溷,转徒流离,凄凉金谷之花,寂莫章台之柳,年华老大,憔悴堪怜。和华德生比较起来,一个当年的公使夫人,如今却做了风尘娼女;一个是当日的陆军大尉,如今却升了阃外元戎:真个是贵贱悬殊,云泥分隔。赛金花虽然写了这一封信,心上却也虑着他未见得还记得我这样的一个人。那里知道华德生回了一封信来,信里头说了许多情话,说得个缠绵宛转,眷念非常。并且还派了四名马兵牵着一匹空马,要请赛金花立刻就去。

赛金花自然喜出望外,便连忙重施脂粉,再挽云髻,换了一身衣服,打扮得娇娇滴滴的,千般旖旎,万种风流,虽然年纪大些,却着实还看得去。赛金花本来原会骑马,便上了马按辔徐行,一直进了内城。从午门进去,只见龙楼如故,凤阁依然,日射昭阳,花飞御苑,依旧还是旧日的规模,只不见一个内官宫女,眼睛里头看见的,都是些异言异服的洋兵。赛金花看了,不觉也动了些爱国的热心,心上十分感慨。

一面看着,不觉已经到了正大光明殿侧首的南书房。华德生满面笑容的抢步相迎,两个人紧紧的拉着手握了一握,相携坐下。赛金花看那华康生时,只见比以前雄壮了好些,气概堂堂,威风凛凛,深目隆准,火色鸢肩,胸前佩带着许多的宝星,闪闪烁烁的光华飞舞,耀得人眼睛都睁不开来。赛金花便对着他嫣然笑道:“恭喜你立功万里,总统诸军。地球上的人,那一个不知道你是个绝世的英雄,过人的豪杰!我们自从那一次在德国公园别后,只道今生今世再见不着你的了。不想天缘凑合,居然彼此相逢,真是再也想不到的。”说着,不觉眼圈儿一红,低下头去。华德生见赛金花和自己隔绝多年,依然的华彩照人,丰姿替月,眉弯浅黛,颊晕深红,觉得他走到面前,好似一盏绝大的电灯一般耀得眼光霍霍的,一时捉摸不定。正是:

萧郎久别,莺花南国之思;倩女离魂,烟雨西方之梦。

不知华德生说些什么,请看下回去便知分晓。





第一百七十三回 慰离悰倾心结幽愫 上手本屈膝拜红裙





且说华德生见了赛金花,心上十分高兴,紧紧的握着赛金花手,对他说道:“我们一别十数年,不意又在此间相遇。且喜你丰姿不改,颜色依然。我们两个人的这番相见,虽然不是天缘凑合,却也全亏了你们中国的那班团匪闹出事来,我们两个人方才得有这般欢聚。论起来,还是这班团匪的功劳。”说着,不觉拈着胡子哈哈大笑。赛金花听了也笑起来。两个人诉了一回别后的相思,说了一番多年的离绪。华德生便把自己的事情,怎样的和内阁大臣的女儿结婚,怎样的推升陆军大将,怎样的奉诏东征,约略说了一遍。赛金花也把自己夫死复出,重落风尘的事情,一字不瞒,告诉了华德生一遍,叹了一口气道:“我们十余年不见,你却十分得意,官居大将,名动全球。我就弄得这般模样,萍飘蓬转,重入火坑,将来还不知作何归结。想起那以前的事情来,真个是追想当年,不堪回首!”说到这里,不觉天良激发,打动了他的心事,一阵心酸,扑簌簌的流下泪来。华德生见赛金花忽然下泪,连忙携着他的手,切切的安慰他道:“你不必这般伤感,我们故人相见,正该大家欢喜才是,怎么倒伤心起来?你心上有什么不遂意的事情,只顾和我讲就是了。只要我办得到的,无不和你尽力。”说着,便取出素巾,和他拭泪。

不想这个时候,赛金花当真的把自家的心事提了起来。想着自家年纪已经将近中年,婪尾花残,茶蘼香老,春光零落,前路苍茫,终究不是个了局。将来自己的这个身体都不知怎样的一个归结。想着那以前的锦绣繁华,看着这现在的风尘沦落,心上已经酸酸的要流下泪来。更兼想着以前那位殿撰公没有死的时候,待自己也着实不差,偏偏的要这般拼命的混闹,想起来委实有些对他不起。想到这里,不由得天良萌现,竟呜呜咽咽的哭起来。华德生见赛金花竟哭起来,心上十分难过,连忙拉着他的手,低低的劝慰一番。赛金花触动了真伤心,一时那里劝得住。华德生虽然是个一刀一枪的马上英雄,到了这个时候也被他哭得儿女情长、英雄气短起来。

呆呆的看了一回,看着他无可劝解,只得附着赛金花的耳朵,说了无数的柔情软意话儿,央恳他不要再哭。

赛金花见他这样婉婉转转的殷勤相助,觉得自己吃了半世的把势饭,相识的客人也不知多少,从没有遇着这样一个温柔熨贴的人。就是那位状元公,看待自己虽然狠好,也没有这样真心体贴的。心上觉得感激非常,便拉着华德生的手,委委曲曲的泪流不止。华德生看了,知道他拉着自己的手向他流泪,是感激他的意思,不知怎样的,也有些酸鼻起来。深深款款的慰藉了一番,赛金花方才拭泪回欢,敛悲作喜。这一夜,赛金花自然是不回去的了。十年契阔,一晌温柔。一个是南国佳人,风情无限;一个是欧洲名将,华彩非常。玉漏宵沉,凤城夜永,枕上之云鬟斜堕,暗中之芳泽微闻,春融红玉之酥,露渍胭脂之汁。罗帷私语,声声之小凤频呼;玉体横陈,惜惜之檀郎欲醉。这一夜的情形,自然和别人的情景不同。

到了明天,华德生和赛金花说,中国派了议和大臣洪理章前来议和,刚刚营里头没有精通中国文字的翻译,要请赛金花当个翻译的文案。赛金花觉得有趣,便一口应允。从此以后,华德生和赛金花十分相得,一切事情都和赛金花商量。赛金花心中暗想:我虽然是个妓女,却究竟是个中国人,遇着可以帮助中国的地方,自然要出力相助。便趁势劝华德生不要虐待中国人,又劝他把以前监禁的中国官员,只要不是团匪的头目,都释放出来,叫他们照常办事,华德生一一答应。这个消息传了出去,大家哄然一声,都知道赛金花是华德生的腻友,赛金花说的话儿,华德生没有不听的。便有许多无耻的中国官员,钻头觅缝的来寻赛金花的门路。赛金花觉得甚是好笑,一概不去理会他们。遇着那不关紧要的事情,也对华德生说一下子,却是不说便罢,有说必应。

赛金花在华德生那里一连住了几天,想着自己家里的事情,这几天自己没有回去,狠有些不放心,便和华德生说了要回去料理一下,耽搁一两天再来。华德生自然答应。赛金花便辞了华德生,回到自己院中料理了一回院里头的事情。那几个讨人便对赛金花说:“这几天里头,来问信的人一起一起的不知多少,都问说几时回来。”赛金花正待根问,忽见一个从上海带来的娘姨叫做银姐的,笑嘻嘻的手里拿着一个手本走了进来,口中说道:“倪倒一径勿曾听见过歇,到堂子里向来要用啥手本格,阿要诧异仔点。”赛金花听了,心中明白,知道又是要走他们路的人。

原来赛金花自从遇见了华德生以后,那班中国的无耻官员,凡是拿着手本来见华德生的,一定另有一个手本,和赛金花请安。赛金花见得多了,司空见惯,不以为奇,顺手接过手本来一看,只见上面的几个字儿却写得比众不同,端端楷楷的写着“沐恩工部郎中卜蔼廉”的九个字儿。赛金花看了倒不觉呆了一呆,暗想他是个工部官员,我又不是他的堂官,他又不受我的统属,怎么平空的写起“沐恩”的两个字儿来?吃把势饭的人,虽然也有人来上手本称沐恩,真是个有一无二的奇事。

正在沉吟,只听得银姐说道:“格个就是旧年仔一径来浪倪搭吃酒格、大人呀,啥格拿仔格手本,叫倪拿进来拨耐看。倪叫俚自家进来,俚倒说定规勿肯呀。倒搭倪说呒拨实梗规矩格,要耐叫俚进来末,俚好进来,耐勿叫俚进来,俚勿好进来格。

带仔格红樱帽子,拖仔格花翎,海外得来,勿得知啥格事体,倒说搭耐换仔格名字,叫耐啥格宗脱牵太太。倪说大小姐勿姓宗嘛,耐阿是弄错哉。俚倒说耐勿晓得格,请仔宗脱牵太太出来,有要紧闲话要当面讲。耐想阿是少有出见格事体?“

赛金花听了,想起去年的那位卜部郎来,着实在京城里头闹了几个月,和自己有过相好的。想着他用那“沐恩”的两个字儿,大约就是指着和自己有过相好的缘故,倒不觉面上微微的红了一红,对着银姐啐了一口道:“俚是倪搭格熟客呀,耐叫俚进来末哉。啥格实梗神妖鬼怪,几几化化格七搭八搭介,真真气数得来!”银姐一面走出去,口中咕噜道:“倪本底仔叫俚自家进来,俚定规勿肯呀。”走到外面,只见那位卜部郎还直挺挺的站在那里,垂着两手,低着个头,静静的等候传见。

见银姐走过去,推了他一把道:“倪大小姐请耐进去,勿要来浪假痴假呆哉!”卜部郎得了这个分付,连忙恭恭敬敬的答应了一声“嗻”,跟在银姐后面,循规蹈矩的一步一步的走进去。

到了赛金花卧房里面,赛金花立起身来,含笑相迎。见他果然穿得衣冠济楚,翎顶辉煌,更兼袖手低头,鹅行鸽,好象参见上官的一般。便向他笑道:“耐啥格事体着好大衣裳,跑到倪搭来呀?阿有啥到堂子里来白相,着仔大衣裳来格?耐格人阿要伉。”赛金花一面说着,便伸手去拉他,想要叫他脱了衣服,再说别的话儿。

那里知道,这位卜大人见了赛金花伸手要拉他,吓得连连倒退,口中说道:“沐恩今天特地专诚来和总统宪太太贺喜的。”说着不由分说,早已双膝跪下地去,恭恭敬敬的叩了四个头。赛金花见他平空叩起头来,出其不意,着实吃了一晾,连忙笑道:“卜大人,耐算啥呀,拨别人看仔,难为情格呀!”说着急急的伸手去拉他,却那里拉他得起?赛金花见拉他不起,没奈何,只得自己也跪下去还礼。那位卜大人还连连的说道:“总统宪太太,怎么这般客气?”赛金花起先见他无故的跪下叩头,已经觉得十分好笑,却还勉强忍住了不笑出来。到了这个时候,再也忍不住的了,不由的“格格”的笑出声来。那几个讨人和娘姨大姐,看了这般怪相,也不约而同都嘻嘻哈哈的看着卜大人笑。

这位卜大人却心平气和的,没有一些儿惭愧的模样,从从容容的叩过了四个头,扒起身来又深深的请了一个安,站在一旁垂手侍立,连坐也不肯坐。赛金花再三让他坐下,他死也不肯,只说总统宪太太在上,那有沐恩的坐处。赛金花道:“耐定规勿坐,是只得倪也陪仔耐勿坐格哉。”卜大人听了,方才斜着身体坐下。

赛金花对他笑道:“卜大人,倪搭耐一年勿见,耐啥格变得实梗样式哉呀?文绉绉格,客气得来。”卜大人听了,连忙立起身来答道:“沐恩自从受了总统宪太太的格外栽培,心上的感激一时也说不尽。如今在总统宪太太面前,那里敢放肆?”

赛金花听得卜大人叫他做总统宪太太,暗想怪不得方才银姐听错了,认作什么宗脱牵太太,想着,不因不由的又笑起来道:“耐格称呼勿对嘛,啥格总统宪太太,杂格乱拌,倪懂才勿懂。耐一径叫倪老大,故歇也叫倪老大末哉。啥格总统洛粥桶,拨别人家听见仔阿要好听呀。”正是:

庸奴无耻,樊英拜侍女之床;中妇多情,都尉屈黄金之膝。

不知后事如何,且待下文交代。





第一百七十四回 暮夜金奸奴行重贿 美人计相国赠明珠





且说赛金花听得那位卜大人竟叫他做总统宪太太,叫他不要这样的称呼。卜大人那里肯改口,只说这是理应这样称呼的。赛金花又对他笑道:“卜大人,耐是倪搭格熟客呀,为啥要实梗客气呀?”卜大人听了,连忙又立起来请了一个安道:“总统宪太太,这样的称呼不敢当,只叫沐恩的名字就是了。”这一阵的巴结,倒把个赛金花巴结得局蹐起来。

卜大人恭维了一回,便道有几样东西要请总统宪太太赏收。说着,亲自走去拿进一个红绫锦匣,里面放着四样首饰:一对珠花,一对金镯子,一只金钢钻戒指,一付翡翠押发。双手捧着,交在赛金花的手中,口中说道:“这一点儿东西,不过聊表沐恩孝敬的意思,算不得什么。”赛金花接过来看时,只见珠子、翡翠和金钢钻都是上等的货物,那付镯子也打造得十分精巧,精光外溢,宝气内含。约莫看上去,这几件东西少说些也要值一二千银子。从来天下的女子生性最贪,又最爱的金珠首饰。赛金花见了这几件东西,由不得喜得眉花眼笑,拿在手中看了又看,爱不忍释,便对着卜大从笑道:“谢谢耐,送倪实梗几几化化物事,常恐要几千洋钿笃嘘!倪想起来,倪呒拨一点点好处来浪耐卜大人面浪,受仔耐实梗几化物事,心浪洛里意得过?耐有啥事体要倪搭耐帮忙,耐只顾搭倪说末哉,勿然倪也勿好意思受耐格物事。”卜大人听了,正中下怀,便走近一步,附着赛金花的耳朵,悄悄的说了几句。

原来这位卜大人也是附和端王的人,也曾当过团匪头目。如今联军进京查办罪人,要把他提去治罪。幸亏这位卜大人的手臂极长,耳目极灵,早已得了信息,连忙拿着许多的造孽钱各处运动。便有人和他说:“你运动别人不中用,除非去运动华德生方才有用。”这几天之内,这位卜大人十分着急,东奔西走的找寻门路,被他打听出赛金花的这条门路来。卜大人想着这个赛金花是和自己有过交情的,觉得更加放心。却又恐怕带着一双空手去要赛金花和他说情,赛金花未必就肯答应,便配了这几样首饰,卑词厚币的跑到赛金花那里,要托他在华德生面前说些好话。

赛金花听了他的一番说话,想了一想,觉得这件事情也没有什么关系。更兼这位卜大人究竟和自己有些瓜葛,虽然不是什么一定怎样的恩客,却到底芳心辗转,未免有情。又恐平空的受了他这许多的礼物消缴不来,自然一口答应竭力和他关说。

卜大人见赛金花容容易易竟答应了他,心中大喜,立起来对着赛金花一连请了几个安,只说:“多谢总统宪太太格外施恩,沐恩感激不尽。”接着又说了许多感激涕零、受恩图报的话儿,把一个赛金花也说得有些肉麻起来。

卜大人方才走了,接着外面传进无数的手本来,都是要见赛金花的。赛金花见了,委实觉得有些好笑,只得把他们一个一个的都请进来。赛金花慢慢的出来相见,也有向来认得的,也有不认得的,无非都是要走赛金花门路的人。那个时候,洪中堂虽然已经到了北京开议和约,那京城地面的政权,却差不多还在华德生手内。那些九卿六部的官员,也没有一个不要承问他的颜色。只要是华德生保举的人,立刻就在平地飞升,非常的快速。华德生索办的人,不是拿问,便是革职,甚而至于把个脑袋都请了下来。所以这班忘廉丧耻、贪利蔑义的人,一个个都想走华德生的门路,希冀升官发财。无奈这个华德生却不是那般贪受贿赂、上下其手的人。不得已而思其次,便大家都想到赛金花身上,想借着用个间接的法儿,料想他说的话儿,十句里头华德生便有九句听的。一个赛金花的门外,顿时的冠盖如云,车马杂沓起来。两三天的工夫,赛金花收受的那些礼物几乎挤满了屋子,比那外省的督抚到任还要热闹些儿。赛金花只拣那没有什么大关系的事情答应下来,那真有关系的,便把他的礼物退还不收。回来住了两天,倒觉得十分忙碌,直到晚上十点多钟的时候,方才没有人来缠扰。

赛金花正想安睡,忽然外面又传进一张名刺来,名刺上写着“杨言”的两个字儿,说有要事商量。赛金花便把他请进卧房看时,却和他素不相识。那姓杨的见了赛金花,便疾趋而进,低低说道:“我是洪中堂的手下的随员,洪中堂特地派我前来,有国家大事和你商量。”赛金化听了不觉呆了一呆道:“洪中堂有啥格事体搭倪商量呀?”那姓杨的说道:“你这里人多口杂,恐怕万一个传了出去泄漏风声,却大大的不便,须要找个清静些儿的地方才好讲话。”赛金花听了心上疑疑惑惑的,不知道洪中堂要和他商议什么事情,便引着那姓杨的到后面一间小小的斗室里头坐下,预先分付了那班娘姨、大姐,叫他们不准窃听。

赛金花先让姓杨的坐下,又把双扉掩上,方才回身问道:“洪中堂搭倪一径勿认得格嘛,故歇有啥格闲话搭倪说呀?”那姓杨的把坐下的交椅往前移了一移,紧靠着赛金花坐下,悄悄的说道:“中堂听你和联军总统华德生甚是要好,你的话儿他没有不听的。”赛金花不觉面上一红,有些不好意思起来,红着脸说道:“倪搭俚也勿是一定啥格要好,不过归格辰光来浪德国,一径搭俚认得格,故歇多年勿见哉,碰仔头像煞要好点。”那姓杨的又低声说道:“你不要这般客气,难得华德生竟肯和你要好,是再好没有的了。如今的华德生脾气大得狠,就是洪中堂和他说话,也常常碰他的钉子。中堂听说你和他狠要好,并且狠听你的话儿,心上十分欢喜,所以特地遣我到来,要请你在里头帮个忙儿。中堂知道你是个狠有才识胆略的人,只要拿出本领来,好好的哄着华德生,料想他逃不出你的手掌。况且你又是个中国人,一定帮着中国的。”赛金花听了,想了一想,心上已有几分明白,不由得脸上又红起来,低低的对着那姓杨的说道:“到底啥格事体,洪中堂要倪帮忙呀?只要倪办得到格事体,倪阿有啥勿肯。”

那姓杨的先立起身来,开了门往外面看了一看。见门外一个人没有,便又随手把门掩上,翻身进来,方才向赛金花说道:“实不相瞒,洪中堂此番奉命议和,别国的钦差都还没有什么,只有华德生,为着他们本国的公使克林德被团匪无端杀害,忿恨万分。那议和的条款和赔偿兵费,别国都肯通融办理,惟有华德生一力坚持,不肯丝毫退让。洪中堂再四和他商议,请他看着国家的交谊,退让些儿,他却对着洪中堂说道:”只要你肯还我一个活活的克公使,万事都好商量。如若不然,只得休怪了。‘洪中堂屡次受他的抢白,无可如何。若是和议不成,可怜我们中国的大局就不堪设想了。如今洪中堂听得华德生狠肯听你的话儿,说总算我们中国的洪福,特地叫我前来奉托,要请你在华德生那里设法劝他一下,叫他退让些儿。你若果然办成了这件事情,一则不杠你有了这般才识,二则调和了中外的邦交,三则扶助国家的气运。料想你这样的一个奇女子,一定有过人的胆量、出众的机权。这个事儿竟要靠托在你一个人身上的了。洪中堂说,只要你肯答应,将来事成之后凭你要什么,只要是洪中堂办得到的事儿,都没有不答应的。你总要看看洪中堂的情面,也看在国家分上,耽任了这件大事罢!“

赛金花听了,低着头沉吟了一会,慨然说道:“既然洪中堂要倪搭俚帮忙,倪自然呒啥勿肯。不过格个讲和格事体,倪有点弄勿明白,阿好请耐搭倪讲明白仔,难末倪再慢慢里搭俚说,像煞稳当点,耐说倪格闲话阿对?”那姓杨的听了,便粗枝大叶的把议和的条款约略说了一遍,华德生如何的要内地各处通商,厘金关税都归他们监理,如何的定要赔款七百兆,洪中堂如何的想把赔款减少,如何的想要竭力磋磨,都和赛金花说了。又道:“洪中堂分付过的,你若是肯在里面相助,却只好随机应变的想着法子劝他,万不可说出真情,说是洪中堂的意思。他们欧洲各国的人都是狠爱名誉的,你若是和他说了真话,他只说为着儿女的私情贻误国家的公事,非但不肯答应,并且还恐怕要生出别的枝节来。你只要有意无意的只当作和他谈论一般,婉婉转转的劝他几句,叫他勿为已甚,他一定没有不听的。”

赛金花听了点头道:“格是倪晓得格,倪自然有法子教俚听倪格闲话。耐转去格洪中堂说,请俚放心末哉。谢勿谢倪倒勿来浪心浪,只要唔笃大家看仔,晓得倪吃把势饭格人也勿是一点点用场才呒拨格饭桶。故歇别人家说起倪堂子里向倌人,总说才勿是好人,阿是也勿见得。”那姓杨的笑道:“像你这样的人,如今的那班堂子里头的倌人那里还有?”赛金花微微一笑,也不言语。杨观察临走的时候,又在袖中取出一个小小的玻璃锦匣,里面装着四个绝大的珍珠,光华夺目,送给赛金花道:“这是洪中堂送给你的,将来事成了,再大大的酬谢。”正是:

鲸鲵跋浪,踏翻西海之涛;烽火连云,拥出大官之骑。

不知赛金花肯受与否,且待下回便知分晓。





第一百七十五回 联中外名妓说英雄 闹平康宵有张虐焰





且说赛金花见了那四颗绝大的珍珠,心上十分欢喜,略略的推让几句便也收了下来。从此以后,赛金花果然在华德生面前,一早一晚随时劝解。华德生起先还不肯听,经不起赛金花的一张嘴儿好象娇鸟调音、雏莺弄舌的一般,说得有情有理,不由得华德生不听;更兼洪中堂再四磋磨,请他酌减赔款,一切通融办理,华德生便将机就计答应了。登时就把中外和议的草约议成签字,各国的钦差也都答应,没有异言。想不到这样一件天大的事情,却是一个弱女子在里头宛转相助,成就了这件绝大的功劳。

论起来这位议和大臣洪中堂,既然用了这个美人计,便应该大大的酬谢赛金花一下才是。偏偏的洪中堂年纪高大,吃不起辛苦,看着中国这般的时势,荆榛遍地,豺虎当涂,蒿目山河,惊心烽火。看着自己的年纪已经将近八十岁的人,那里还能和国家出什么力,心上未免总有些郁郁不舒。更兼跋涉风尘,驰驱舆马,进京的时候本来已经有病,无奈这个时候国事紧急,不得偷安,没奈何只得力疾从事。开议和约的时候,未免又要受些委屈,忍些烦恼,心上一忧一急,那病便一天一天的重起来。究竟上了年纪的人,那里禁得起?不等到和约签字,便呜呼哀哉死了。

洪中堂既死之后,偏偏的那位姓杨的随员也丁了外艰,奔丧回去。这两个人死的死了,走的走了,别人那里知道这件事儿的内容?就是有几个知道的人,那里还来多管这般闲事,想着要酬谢赛金花的这件事儿?老老实实的把赛金花的这番劳绩挂在瓢底里头去的了。好在赛金花本来不想什么酬谢,便也不把这件事儿放在心上。

到了后来不知怎样的,京城里头的那班人大家都把赛金花的这件事儿传说出来。又见他常常穿著男子衣冠,同着华德生并马出游,大家都不叫他赛金花,都叫他作赛二爷。又为着他帮着洪中堂议成和约,大家便又叫他作议和大臣。这个议和大臣赛二爷的芳名,竟是京城里头没有一个不知道的。

后来华德生撤兵回国,赛金花想要同着他到德国去。华德生为着奉命出师还没有回国复命,不能带个女子回去,赛金花便也只得罢了。华德生临走的时候,两个人依依不舍。长亭惜别,南浦伤神。蘼芜远道之思,杨柳征人之恨。柔肠百结,春销凤女之魂;别泪三声,目断西溟之水。赛金花直送华德生到天津,上了兵轮,方才洒泪别去。自此以后,赛二爷的生意,比以前更是日盛一日。过了几时,赛金花想着恋恋风尘究竟不是长策,趁着如今手里头着实有了几个钱,想要拣个好好的客人嫁了他,作个叶落归根的算计。

刚刚这个时候,那位卜蔼卜部郎借着赛金花的扶持,走到了华德生的门路,非但没有追究他附从拳匪的事情,而且华德生还在中国议和大臣面前,和他讲几句好话。这个时候华德生的话儿,就好象皇上、皇太后的谕旨一般,那一个敢不听他的说话!连忙把这位卜部郎一保两保,平地飞升,不到半年,已经升署了刑部右侍郎。

这位卜侍郎的运动手段又十分利害,皇上、皇太后回銮之后,那一班跟着到西安去的大臣,一个个不是军机大臣,便是尚书、部院,却不知怎样的一个个都受了卜侍郎的运动,都说他是个狠有才干的人。这位卜侍郎本来是贪花好色的都头、醇酒妇人的首领,如今仕途得意,越发成日的花天酒地,选舞征歌,邀结公卿,交通权贵,赛金花院中也常常的去摆酒请客。但是当着那华德生没有回国的时候,卜侍郎虽然也常到赛金花院中去,却口口声声的总统宪太太长、总统宪太太短,不是送衣服,就是送首饰,规规矩矩的连笑话也不敢说一句,那里敢在赛金花院中摆酒?如今华德生走了,卜侍郎却登时变了样儿,见了赛金花的面,也不称他总统宪太太,自己也不称沐恩,依旧嬉皮笑脸的动手动脚起来。

赛金花见他忽然变了样儿,不像那以前的恭敬,虽然不甚放在心上,却也觉得有些好笑。卜侍郎在赛金花那里混了几时,知道赛金花狠有几个钱,就是华德生在京城里头的时候,那些别人送他的金珠首饰,也值好几万银子,便存着个人财两得的念头,想要娶他回去。无奈赛金花想起他以前要走华德生门路的时候,对着自己一味的叩头请安,不顾廉耻,后来华德生走了,又趾高气扬的翻转脸来,和以前好象两个人的一般,心上是有些瞧他不起,不肯嫁他。卜侍郎和他说了几次,赛金花都一口回绝。卜侍郎一连碰了几次钉子,心上便大大的不快起来,对着赛金花常常的藉端发作,一会儿说他怠慢了客人,一会儿又说他回绝了生意。赛金花虽然是个妓女,却倒是个狠爽直的人,见他这样的有心挑剔,只说他是闹着顽的,也不放在心上。

这一天正逢礼拜,赛金花那里来了无数的客人,把六七个房间都挤得满满的,摆酒的摆酒,碰和的碰和,甚是热闹。只把一个赛金花忙得个八面张罗,满场飞舞,凭你赛金花的这般老手,也有些手忙脚乱的应酬不过来。在忙得个手口不闲之际,刚刚的卜侍郎又同着几个朋友吃得醉醺醺的,闯了进来,要在赛金花院中碰和。赛金花见了卜侍郎,只说自己以前帮过他的忙,救过他的患难,更兼华德生没有回国的时候,这位卜侍郎见了赛金花的面好象小鬼见了阎王、老鼠见了猫一般,连屁都不敢放一个。如今虽然华德生遄回德国,卜侍郎已据要津,在赛金花眼中看起卜侍郎来,却还是以前的卜侍郎一般,并没有什么分别,那里把他放在心上。当下便对卜侍郎笑道:“卜大人耐来得勿巧,几间房间才勿空来浪,只好请唔笃几位晏歇再来格哉。”

卜侍郎听得房间勿空,赛金花叫他等一回儿再来,心中甚是不快,乘着醉意,睁开了一双鼠目,便想发作。却被一个同来的人说道:“我还要宝香堂去,这里的房间不空狠好,我们去一会儿再来。”说着,拉了卜侍郎便走。卜侍郎只得同着他去到宝香堂坐了一回。转过身来,方才又到赛金花院中,那几间大房间依旧还没有空,只有一间极小的斗室,里头没有人,卜侍郎只得勉勉强强的坐在这个小房间里面。赛金花正在那里和客人代碰和,听得来的客人就是卜侍郎,赛金花便只顾碰他的和,没有过去应酬。

卜侍郎等了好一回,要等赛金花出来,那知等来等去,赛金花的影也不见。卜侍郎不由得心火发起来,喝令娘姨:“去把你们大小姐叫过来,我有话和他讲!”

偏偏的这几个娘姨大姐,也为着以前的卜大人对着赛金花这样的奴颜婢膝,如今的卜大人对着赛金花却又这样的装腔作势起来,一个个的心上也都在那里剪他不起。

看了他这样其势汹汹的样儿,心上越发的不愿意,冷冷的连应都不应。卜侍郎见了他们这般待理不理的神情,更觉得火上浇油,薪边措炭,心上一盆烈火直透青云,再也忍耐不住,跳起身来把桌子上的茶碗抢在手中,用力往地下一摔,摔得粉碎,口内大声喝道:“怎么我来照顾你们的生意,你们都是这般不瞅不睬的样儿?难道我姓卜的不是出钱的么?”

那班娘姨大姐见了卜侍郎忽然的这般发作起来,倒也都吃了一惊。一个大姐便飞一般的去和赛金花说,娘姨银姐便上前按住了卜侍郎陪笑劝解。卜侍郎那里肯听银姐的话,只是气忿忿的乱嚷。一刻儿的工夫,赛金花急急的赶了过来,见了卜侍郎便微微一笑道:“倪当仔啥人来浪发脾气,勿壳张是卜大人!卜大人,耐是勿比别人,倪搭格老客人哉嘛!俚笃有啥勿到家格场化得罪仔耐卜大人,阿好看倪面浪勿要动气。”卜侍郎见了赛金花说得这样轻描淡写的,知道有心藐视,更觉生气,把桌子一拍道:“别人得罪了我,叫我看在你的脸上不要生气;如今就是你自己得罪了我,却叫我看在那一个人的脸上呢?”

赛金花见卜侍郎忽然这样的平空变起脸来,心上廿四分的诧异,却还只道他吃醉了酒,不是有心来寻事的,便笑着说道:“倪是呒啥得罪耐卜大人格地方嘛,耐今朝啥实梗动气呀?阿是好吃仔酒哉?”卜侍郎铁铮铮的说道:“我吃了酒也用不着你来多管。客人来了差不多一点多钟的时候,你影儿都没有看见,这样的还说是没有得罪,你要怎么样方才算得罪呢?”

赛金花听了卜侍郎这样的口风,分明是有心扳他的错处,心上方才也有些生气起来,暗想天下竟有这样忘恩负义的人,便也正色说道:“卜大人,耐闲话说错哉!

耐卜大人来浪倪搭,老实说,比勿得别人,倪就是得罪仔耐,耐也勿好意思扳倪格差头哩,卜大人阿对?“卜侍郎听了顿了一顿,硬着头皮又道:”这是什么话儿,得罪了我,我也不好意思挑你的眼儿!我到要问问你,为什么我不好意思挑你们的眼?难道我姓卜的就不是客人么?“赛金花冷笑一声道:”卜大人耐自家心浪也蛮明白来浪,定规要倪说出来,是呒啥趣势!“说着又叹一口气道:”故歇世界浪事体,格末叫稀奇。倪倒勿壳张耐卜大人会有实梗格一来,阿要诧异!“正是:

辜负红梨之梦,雨怨云愁;猖狂遥夜之风,花啼柳泣。

不知后事如何,且听下回分解。





第一百七十六回 杀风景恶客试尊拳 弃尘寰佳人悲薄命





只说卜侍郎听了赛金花的说话,越发暴跳如雷的道:“你这个东西近来着实的放肆!你在别人面上放肆也还罢了,如今竟在我面前都敢这般放肆起来,这还了得!

最可笑的,无影无踪的平空讲出这般混话,倒说我自己心上明白,我今天定要请教请教你,究竟是什么话儿?“赛金花听了卜侍郎一番说话,把以前的事情竟是一笔抹煞,只气得目定口呆,一时连话都说不出来。停了一停,方才冷笑道:”倪来浪别人面浪,倒才是客客气气格。独有来浪耐浪末,就是推扳点也呒啥希奇。耐阿记得,跪来浪地浪叫总统宪太太格辰光,倪对仔耐是那哼样式,阿是忘记脱哉?“

卜侍郎听了虽然面上红了一红,却假作不懂他说话的意思,别过脸来对着那几个朋友说道:“你们听听他讲的,都是些什么乱七八糟的,我简直不懂他讲的是些什么话儿!”赛金花鼻子里哼了一声道:“唔笃做官格,大家才靠天老爷来浪照应。

倪吃把势饭格,也靠仔天老爷来浪照应。一个人有仔良心,总归有好日子格。做仔格人呒拨仔良心,是勿局格嘘!耐说出实梗格闲话来,耐良心到仔陆里去哉?倪倒要洗清仔眼睛,看看耐格位卜大人那哼格升官发财!倪是呒啥希奇,总归靠仔天老爷过日子。耐卜大人要扳倪格差头,随便耐去那哼末哉!“卜侍郎听了赛金花的说话,一句紧似一句,来得甚是锋芒,知道说他不过。想要打掉他的房间,又怕被人知道了风声不雅,要想找句话儿出来扳驳他,却又一时找不出来。

刚刚这几个朋友里头也有知道卜侍郎这件事情的人,明知道说来说去一定说不出什么好话,便拉着卜侍郎说道:“你们两个人,今天大家都在气头上的时候,从来相打没有好手,相骂没有好口。你们两个好几年的老相好,那里真有什么一定过不去的事情,有什么话明天再讲就是了。”赛金花瞪了卜侍郎一眼,对着众人说道:“勿说起老相好格句闲话,倒还勿要去说俚。说起仔老相好格句闲话来,格末真正叫枉空!”卜侍郎被那几个朋友拉着往外便走,也就将机就计,回过头来对着赛金花说道:“你自己小心在意,不要撞在我的手里头就是了!”赛金花气到极处,那里还管他什么侍郎不侍郎,高声答道:“倪等好来里,耐有啥本事末,来末哉!”

卜侍郎还要说话,却被那几个朋友不由分说,推推拥拥的拉着他一哄出去。赛金花连送也没有送,卜侍郎真恨得咬牙切齿的,发誓要想个法儿收拾他。偏偏事有凑巧,也是赛金花运遇邅迍,出了一件不大不小的事情。赛金花院中本来有两个讨人,一个叫金红,一个叫银翠。这个金红,恰生得十分狡猾,一味的巴结赛金花,巴结得赛金花十分欢喜,把他就当作自己的亲生女儿一般,一切贵重的东西都交给金红一个人掌管。这个银翠,却刚刚和金红生得反了一个意见,不但不肯奉承,而且性情生硬,就是见了客人也每每要排墙倒壁的任意冲撞,赛金花心上本来狠不愿意他。

就是这个银翠,见赛金花把个金红这般的抬举,把自己却这样的冷淡,两下比较,未免有些相形见绌的地方。

这一天,有个在银号里头管帐的山西客人,到赛金花院中来摆酒请客。刚刚赛金花和金红都出条子去了,没有回来,只有银翠在家,身上有些寒热,睡在床上没有出来应酬。那客人不知道他生病,要去拉他起来,银翠不肯。那客人本来也是个蛮牛一般的人物,那里有什么怜香惜玉的心肠,见银翠不肯起来,只说他有心慢客,心上生气,一定要叫他起来。自己跑过去,不分好歹生生的把银翠拉了起来。银翠心中大怒,着实把他冲撞了一顿。那客人受了这番没趣,不觉得老羞成怒起来,跳起身来,伸出巨灵一般的手掌对着银翠的左边颊上“呼”的就是一掌。银翠不及提防,只听得“拍”的一声,粉嫩的脸上早现出五个指印,红了半边。说时迟,那时快,银翠还没有回身,右边脸上早又是“呼”的一掌飞来。银翠一连受了两掌,又羞又痛,又气又怒,不觉掩面大哭起来。一面哭着,一面骂着,只说:“你要打,索性打死了我,不敢打的就是个畜生!”那客人那里忍得住,再要奔上去打时,却被一班娘姨、大姐大家拦住,七张八嘴的解劝,大家闹作一团。

正在闹得沸反盈天之际,幸而赛金花出局回来,连忙上前把那客人劝住。那客人还气得乱嚷乱跳,只说银翠得罪了他,定要赛金花打他一顿,方才肯罢。赛金花听了,知道这件事情银翠没有什么大不是,又知道他身上有病,不肯打他。禁不得这位西老儿一味的和赛金花混闹,死也不肯干休,逼得赛金花没奈何,只得把银翠叫了来,当着那客人的面,轻轻的打了几下,又淡淡的骂了几句,那客人方才罢了。

那里知道,这个银翠平空被那客人打了两下,正在有冤没处伸的时候,不想赛金花又当着那客人的面,把他打了几下,一腔冤忿,无可发泄。想着流落风尘,将来终究没有好好的结局,平日之间既不得赛金花的欢心,今天又受了这样的一番奇冤极枉,越想越气,就萌了个短见的心肠,悄悄的取了一合生鸦片烟吞了下去。一霎时芳魂渺渺,艳魄悠悠;阆苑雪消,高堂云散。灯昏柝死,香销离恨之天;月黑风凄,春冷芙蓉之府。等到赛金花院中的人知道银翠吞了生烟,大家手忙脚乱的想要施救时,早已脉息停断,直僵僵的挺在床上,呜呼哀哉了。

赛金花慌了手脚,想要私自殓埋,不想左右邻居的那些班子里人,都与赛金花家不合,嫌他夺了生意。如今听得他出了人命,不由分说,竟去坊官那里报案。坊官听得赛金花家出了命案,心中大喜,知道生意来了,便差了几个差役,跑到赛金花那里去和他打话,要想大大的敲他一下竹杠。赛金花起先已经答应了一千块钱。

在坊官的意思,拿了他一千块钱,也就罢了。倒是有几个老年的差役,见赛金花答应得这般容易,大家要想他的好处起来,撺掇着坊官一定要他一万块钱。赛金花那里肯出?坊官想要吓他一吓,便径去报了刑部,刑部照例差官相验。在坊官的心上,原说就是报了刑部,也没有什么大事,只要哄他多出几个钱,原可以撕掳得开的。

不想刑部里头刚刚正有一个赛金花的冤家卜侍郎,虎视眈眈的在那里候着,正想要寻赛金花的事情。如今听得他院中自尽了一个妓女,喜得直跳起来,哈哈大笑。

连忙和刑部尚书寿少山寿尚书、卢英之卢尚书说了,只说赛金花逼良为娼,凌虐至死,要重重的办他。卢尚书和寿尚书听了他的话儿,自然授意司官叫他从严办理。

一霎时风行雷厉的认真起来,把银翠面上的伤痕,只说是赛金花打的,顿时把赛金花提到刑部监禁起来。这个时候的赛金花,直吓得胆裂魂飞,手足无措。没奈何,只得叫金红到几相相识的京官那里去,求他们想个开脱的法儿。又备着许多的银钱礼物,去走刑部堂官的门路。那一班刑部司员,知道赛金花是块绝大的肥肉,大家都掂着脚儿,仰着头儿,希冀发归自己审问,好大大的发一笔财。

隔了一天,里头传出消息来,说寿尚书要把赛金花一案发交云南司承审。大家听了,知道这个云南司主事白熙泉白主政,是寿尚书的门生,心上又羡又妒,便大家约齐了,到白主政那里去贺喜。白主政也得了消息,心中大喜,便邀了那班同寅,到四喜新班花旦喜凤寓里去吃饭,猜拳行令,直闹了一个通夜方才回来。

不知怎样的,这件事儿传到寿尚书和卢尚书的耳朵里头,寿尚书大怒道:“我并没有把这赛金花一案发交云南司的意思,这个消息是那一个传出去的?”当下查问了一回,也查问不出什么来。卢尚书和寿尚书便传齐了全部司员,大加申饬,只说你们当了刑部司官,责任狠重,该应怎样的矢廉矢慎,方才是个道理。怎么你们听得赛金花一案发交云南司承审,你们都到云南司去和他贺喜?这承审案件是何等的事情,难道你们都把审案当作利薮的么?若真是这个样儿,那还成个什么体统?

一班司员受了堂官的申饬,一个个都诺诺连声,不敢开口。依着卢尚书的意思,定要奏参几个以儆效尤。还是左右两堂出来和他们缓颊,卢尚书方才罢了。却为着有了这样的一来,不便把赛金花的一案随意发交司员审问。一班司员大家都把这个赛金花当作个头等的美差,究竟发给那一个的好呢?卢尚书和寿尚书等商量了一回,学着吏部掣签选官的法儿,把一班司员大家都聚在刑部堂上,叫他们掣签为定。掣出签来,却是浙江司掣着了,便把赛金花发交浙江司承审。卜侍郎又授意浙江司主事叫他重办。亏得这位浙江司主事洪小连洪主政狠有些风骨,不是那一味巴结上官的人,暗想卜侍郎一个堂堂的刑部堂官,要重办一个妓女,有何难处?却要暗中授意于我,做个间接的交涉,这是个什么道理?不要他别有什么隐情罢?正是:

鲛宫蜃气,楼台之变幻无穷;覆雨翻云,世态之炎凉何极!

《九尾龟》第十一集已经告成,还有许多事实以及全书的结束都在第十二集中出现,看官们休嫌濡滞。这样的五月炎天,让在下做书的调冰雪藕、沈李浮瓜的歇息一回,再来演说给诸公听何如?





第一百七十七回 罡风无赖折柳摧花 眉语彷徨双心一抹





上回书中正说到洪主政受了卜侍郎的属托,心上甚是疑惑。把赛金花提出来问了一堂,又把赛金花院中的几个娘姨、大姐,都传到堂上对了一遍口供。大家都说赛金花并没有逼良为娼、凌虐至死的事情。大家的口供,都和赛金花自己的口供一般。洪主政便存了个开脱赛金花的心。依着洪主政的意思,要把那山西客人提来质对。那山西客人得了这个消息,心中大惊,究竟是人命重情,不是顽的,便找了个积年的刑部书办和他商量,只说现在有病,不能到堂。一面求了几个素日往来的京官,托他们写信到洪主政那里去,恳求免其提讯。恰恰的赛金花的门路也走到了,卢尚书和寿尚书两个都分付洪主政,把赛金花一案早日讯结,无用株连,明明就是叫他从宽办理的意思。卜侍郎心上虽然不快,但两个堂官做主,怎敢不依?凭着洪主政把赛金花议了一个流娼滋事的罪名,把他发到该管地方官那里去,由地方官派差递解回籍。

这一场官司虽然没有什么大碍,却花了无数的银钱,在刑部监里头,又受了许多的狼藉。赛金花明知自己这件事情一定是卜侍郎有心和他做对,心上十分恨忿,懊悔当初不该在华德生面前和他缓颊。越想越气,越气越恨,却又把他无可如何。

只得和宛平县派来的差役打通了关节,暂时留住几天,料理京城里头那些未了的事情。讲明了在京城里头多住了一天,给解差二百两银子,有一天算一天。赛金花心上虽然烦恼,却还仗着手里头着实还有几个钱,还有一个讨人金红,到了上海去一般的也好做生意。就是从此不做生意,有这几个钱一生一世也吃着不尽。

那里知道,福无双至,祸不单行,这个金红竟席卷了赛金花的所有银钱、首饰,跟了个赛金花的车夫不知逃到那里去了。连几件值几个钱的衣服,也都卷得一个干干净净,一件不留。赛金花急得气塞咽喉,几乎晕倒。呆了一会,由不得号淘大哭起来。到了这个时候,方才懊悔以前嫁了洪殿撰,偏偏要重落青楼。到了第二次风尘再堕,又不肯早些嫁人,如今只落到这般田地。哭了一回,娘姨银姐走过来再三相劝,方才勉强把他劝住哭声。赛金花呆呆的想了一回,最苦的自已是递解回去的人,不能出面,只好眼睁睁的让他逃走,不敢报官,真是说不出的苦恼。赛金花住了哭,把对象点了一回,银钱、衣饰都是一卷精光,只剩得几箱旧式的衣服和些陈设器具,多算些也不过值上一二千银了。那班娘姨大姐见了这般光景,一个个都去自寻门路,走得一个也不见。幸而这个银姐是赛金花的旧人,倒狠有些良心,依依不舍,情愿同着赛金花一同到苏州去,赛金花十分感激。

在京城里头住了五天,那解差便来催着要钱。赛金花只得悄悄的亲自到几个旧时相好的客人那里,把金红逃走的事情哭诉一遍,要向他们借些盘费,借了一千几百两银子。又把所有的衣服、器具一齐卖掉了,一古脑儿不到三千银子,却被那几个解差,足足的讹了一千六百两银子去。

到了苏州,住了一个多月,想着坐吃山空不是久计,只得同着银姐到上海来,在法界连福里租了两幢房屋,摆开碰和台子。又好象是个半开门的私窝子一般,常常同着银姐两个人到南诚信去坐一回儿,借此兜兜生意。不想今天无意之中却遇着了辛修甫和章秋谷两个。

章秋谷虽然也算是做过他的客人,却一古脑儿只吃了一台酒,不算什么。不过秋谷以前在天津的时候,知道这个赛金花就是状元夫人曹梦兰,是个著名的人物,不免要去赏鉴他一下,并没有什么别的意思。这个辛修甫恰是在上年人京会试的时候和赛金花有过交情的,两下甚是要好。所以赛金花见了辛修甫心上十分欢喜,好似他乡遇故的一般,不免把自己的这番蹉跌对着辛修甫等一一的讲说出来。说到银翠的吞烟、金红的卷逃和自己的监禁刑部,不觉?圈儿就红了,说话的声音,也有些颤抖抖得岔起来,好似那微风振箫,幽凄欲咽,山阳闻笛,喑呜可怜。辛修甫和章秋谷也不觉心上凄然,着着实实的安慰了他一会。

赛金花又说起卜侍郎的一番把戏,引得章秋谷等都哈哈大笑起来,都说:“天下那有这般的奇人奇事?你也未免形容得太刻薄了些。”赛金花正色说道:“格个卜家里格事体末,真正天理良心,倪勿曾瞎说俚一句。唔笃勿相信末,倪罚个咒拨唔俚听听:倪造仔俚半句闲话末,要烂脱舌头根格。倪搭俚咦呒啥冤家,为啥要造俚格闲话呀?格个辰光,唔笃才勿曾看见京城里向格排勿要面孔格京官,一径拿仔手本,到倪门浪来挂号请安格,耐说阿要奇希!”

章秋谷听了赛金花这番说话,知道不是假的,便也对他笑道:“如今那班堂子里头的倌人,都比不上你的这般资格:六年的状元太太,三年的公使夫人,更兼又是联军总统的腻友。许多堂堂中国的官员,一个个都向你上手本、称晚生,这也真算得荣誉达于极点的了。但是到了如今的时候,抚今追昔,回想当年,廿年风月之场,一霎昙花之梦,想起那以前的事情来,心上不知怎么样的感慨呢!”章秋谷这几句话儿,原是有心讥刺他的,不想却触起了赛金花的一腔心事,无限凄惶,迸出两滴眼泪,几乎要哭出来。章秋谷见了,自己也懊悔未免说得太激切了些,平空引动了赛金花的伤感。连忙过去拉着他的手劝慰道:“总是我不好,几句话儿引动了你的心事。但是如今的这般时代,人生几何,去日苦多,你也何必这般认真?”赛金花拭了眼泪,瞟了秋谷一眼,慢慢的说道:“繁华一瞬,富贵沧桑,倪自家懊悔来浪盛年格辰光,勿晓得早点自家做格终身之计;到仔现在格辰光,好梦难常,华年易逝,再要懊悔也来勿及格哉!”

章秋谷听得赛金花忽然的满口调起文来,这几句话儿却说得十分蕴藉,竟像个名士的吐属一般,不觉喝声采道:“你的谈吐真是十分出色。想见当日妙年的时候,倾城颜色,绝代风华,洪殿撰也不知前世怎样修来的艳福,方才娶着你这样的一个人。可惜我章秋谷迟了数年,就没有这般福分。”赛金花听了不觉回眸一笑,颊上生红,看着章秋谷笑道:“倪故歇是老太婆哉,洛俚再有啥格讲究?”秋谷道:“徐娘虽老,丰韵犹饶,着实的不差!”赛金花听了,又是微微的一笑,别过头去不说什么。辛修甫乖觉,在旁“格”的一笑,笑得赛金花和章秋谷都有些不好意思起来。赛金花别转头去,章秋谷便也回过头来和王小屏说话。

修甫在烟榻上坐起身来,对着秋谷招了一招手。秋谷见了,便走过来问什么事情。辛修甫拉着秋谷,就在榻旁坐下,附耳说了几句。秋谷一面听着,一面抬起头来打量了赛金花一眼,摇一摇头道:“我和你是要好朋友,恐怕没有这个道理罢?”

修甫笑道:“你和我也是一样的客人,怕什么?”秋谷道:“虽然如此,究竟有些不便。”修甫道:“这是我自己愿意的,又不是你的意思,有什么不便?”赛金花坐在那里,见辛修甫、章秋谷附耳说话,章秋谷又抬起头来看他,心上早有几分明白,脸上便红起来,低下头去。却又溜转秋波,暗暗的偷看他们两个人的举动。只见章秋谷对着辛修甫还是不住的摇头,修甫切切实实的对他说道:“这个事儿是用不着客气的,你又何必这般的推托?况且这个里头别有一个缘故,我细细的和你说就明白了。”说着,便又附着章秋谷的耳朵说了几句。秋谷又看了赛金花一眼,眼珠一动,微微的笑了一笑。辛修甫附耳又说几句,章秋谷方才点一点头道:“虽然如此,但是你也要问他一下,不知他自己心上怎么样?万一个你答应了,他不答应,可怎么样呢?”辛修甫把赛金花看了一看,呵呵的笑道:“你不必这般过虑。你们方才已经私自先行交易的了,那有到了这时候,倒反不答应的理?包你一说一依,十说十依就是了。”

章秋谷听了一笑,不说什么,只回过眼光和赛金花打了一个照会。赛金花咳嗽一声,也瞟了秋谷一眼。辛修甫便向赛金花笑道:“你走过来,我有句话儿要和你说。”赛金花明知道他说的一定就是方才和秋谷说的话儿,心上早已十二分明白,越发的不好意思起来,坐在那里动也不动,只把手中的一方绸巾细细的看。修甫见他不肯过来,便自己走过去,和他唧唧咕咕的说了半天。赛金花一言不发,只是不住的摇头。辛修甫忽然笑嘻嘻的悄说几句,赛金花不觉一笑,面上隐隐的透出红来,把辛修甫用力推开道:“勿要瞎三话四哉!”辛修甫听了,知他心上已经情愿,便向章秋谷做个手势。

章秋谷正要开口,只见王小屏在外面同着一个中年丽人款步进来,对着秋谷似笑不笑的叫了一声“二少”。秋谷连忙看时,原来就是那位卧云阁的女东家老二。

秋谷连忙答应一声,对着他点一点头。老二星眼微横,蛾眉半蹙,瞅了秋谷一眼道:“二少,耐倒有良心格!”正是:

徐娘半老,难为堕马之妆;商妇多情,谁有青衫之泪?

不知后事如何,且待下文分解。





第一百七十八回 渡银河秋娘联旧好 谐凤侣名士结新欢





且说章秋谷见了那位老二,倒不觉呆了一呆。看着老二的那般模样,狠觉得有些不尴不尬的样儿。回过头来再看赛金花时,只见赛金花侧着脸坐在那里,看着秋谷的脸微微展笑。见章秋谷蓦地里忽然回过头来看他一眼,赛金花会意,连忙别转头去,忍不住“格”的笑出声来。章秋谷到了这个时候,凭你是个花粉丛中的老手、绮罗队里的惯家,也不由得有些左右为难起来。只得对着老二道:“我们到那一边去,拣了房间坐一回儿何如?”

老二把嘴一披,只不开口。王小屏哈哈的笑道:“今天你也太觉得快活了些,如今该应要碰个钉子。你还没有知道,老二已经来了好一会儿,就在隔壁房间里头看了多时,我们都没有知道。”秋谷听了,知道老二翻了醋罐,没奈何,只得要向他陪个小心的了。便又回过头来,先向赛金花飞了一个眼风。赛金花是何等伶俐的人,心上早已明白,微微一笑,立起身来,对着辛修甫说道:“倪去哉,晏歇点一淘请过来。倪来浪连福里第九号,勿要忘记脱仔。”说着,又飞了章秋谷一眼,竟自姗姗的去了。

这里老二见赛金花走了,便对着章秋谷冷笑一声道:“二少,耐倒好格,倪末一径来浪等耐,耐倒来里寻开心!”秋谷笑道:“真正冤枉,我何曾在这里寻什么开心?不过这个人是辛老爷的旧相好,多年不见,如今在这里遇见了,大家免不得讲几句话儿,与我什么相干?”老二又冷笑道:“既然是辛老爷格相好,勿关耐事,耐为啥要搭俚吊膀子?朋友面浪,耐去剪俚格边,阿要难为情呀!”秋谷道:“你们听听,这又是信口栽埋人的话,我何曾和他吊什么膀子?”老二瞟了秋谷一眼道:“像耐实格规矩人,洛里肯搭别人吊膀子?刚刚来浪吊膀子格,是只众生!”秋谷叫一声“阿呀”道:“你这个人怎么开口就讲骂人?”老二呸了一口道:“耐说勿曾吊膀子呀,倪骂格排吊膀子格杀千刀,勿是骂耐嘛!”秋谷不觉笑道:“算了,算了!不用再骂了,就算是我错了如何?”

老二停了一停,又对着章秋谷冷笑道:“二少,耐阿是做仔邵万生格东家哉?”

秋谷听了,已经明白他的意思,待要开口时,老二早接着说道:“耐勿开南货店末,要几几化化老蟹做啥?”这一句话儿,说得大家都好笑起来。秋谷却对着老二做个手势,又往自己鼻子上指了一指。老二猛然省悟,不觉得面上红了一红,伸过手来把秋谷打了一下道:“只有耐末总归比别人家刁枭点。”秋谷一笑,也不开口,大家也没有留心。

略略的坐了一回,秋谷便同着辛修甫等几个人,到老二那里去吃了一顿便饭,秋谷又邀着辛修甫打了八圈牌,给了二十块钱的头。老二谢了一声,收了进去。这一夜,章秋谷自然是不得回来的了。刘郎再到,天台之旧路依然;神女多情,巫峡之行云无恙。惊鸾顾影,飞燕回风。宝钮郎当,罗衣熨贴。就日偎云之梦,飘烟抱月之腰。这些情节,也不必去提他。

只说章秋谷在老二那里住了一夜,便回到新马路公馆里头来。见了太夫人,太夫人对他说道:“你昨天晚上住在什么地方去的?为什么不预先招呼一下?害得他们两个人昨天晚上直等了一夜。”秋谷只微微的笑,不说什么。太夫人略略训戒了几句,便也罢了。

秋谷回到他夫人房内,见他夫人睡在床上,微微的有些睡着。秋谷也不去叫他,又走到陈文仙房内看时,只见陈文仙独自一个人靠窗坐着,一手托着香腮,好象想什么心事。见了秋谷,便慢慢的立起身来,微微笑道:“你昨天到那里去的?”秋谷走过来,握着陈文仙的纤手道:“对不起,你昨天等了一夜。”文仙笑道:“自己人何必这般客气?我只问你昨天到底在什么地方?”秋谷便把老二和赛金花的事情和他说了一遍,陈文仙听了,脸上不由得呆了一呆,一言不发。秋谷见了,心上觉得有些过意不去,不免温温存存的安慰一番。

到了晚间,辛修甫同着王小屏、陈海秋三个人,都到章秋谷公馆里来,邀着他一同出去。秋谷换了衣服,又到太夫人那里禀知。太夫人问道:“今天回来不回来?”

秋谷觉得有些答应不出,只看着太夫人嘻嘻的笑。太夫人道:“看你这个样儿,今天又是不回来的了。就是在外面应酬,也要自己有些分寸,不要落了他们的圈套才是。”秋谷听了,只得撒一个谎道:“这两天的应酬是必不得已的。杭州到了一个朋友,不得不应酬他一下。只要过了两三天,敷衍得他走了,就没有事情了。”太夫人听了点一点头。陈文仙站在太夫人后面,对着章秋谷嫣然一笑,把两个指头在自己脸上划了几划,做个羞他的样儿。秋谷看了忍不住也是一笑,急急的走了出去,同着辛修甫等,大家一阵风都到连福里来。

进了门,只见赛金花笑吟吟的迎上来,穿著一件玄色绉纱夹袄、玄色绉纱裤子、玄色缎子弓鞋,一身黑色,越显得山眉水眼,云鬓花颜。虽然年纪略略觉得大些,却还是体态娇娆,丰姿清丽。见了辛修甫和章秋谷等,便对着章秋谷等笑道:“二少,今朝那哼有工夫到倪搭来,昨日仔阿曾吃生活?倪牵记得来!”秋谷听了,面上也不觉红了一红道:“昨天打碎了醋缸,今天又泼翻了醋瓶,怪不得熏得我心上都有些酸溜溜的。”赛金花也不由的脸上一红,道:“二少,耐勿要缠夹嘘!啥格醋缸醋瓶,才勿关倪啥事嘛。”秋谷听了也不开口,只对着赛金花把嘴唇动了一动,眼睛撇了一撇。赛金花见了把身体一扭,一言不发,低下头去。辛修甫在旁边看得十分明白,心上暗暗好笑,便邀着大家进房坐下。赛金花亲自送上茶来,秋谷连忙立起身来接了茶,口中连说:“不敢当,不敢当。”辛修甫笑道:“你们两个人何必这般客气?难道等会儿到了那个时候,也是这般的客气么?”一句话把赛金花说得连脖子带耳根都涨得通红,讪讪的走了出去。

王小屏对辛修顿足道:“他们两个人方才有些意思,给你这样的一来,把那一个说得跑了。”章秋谷听了不觉也微微一笑,回头和辛修甫说道:“这件事儿,我觉得始终有些不妥当。”辛修甫笑道:“你这个人怎么这样的矫情?我昨天已经和你讲得明明白白的了,怎么今天又说出这样的话来?”参欧谷道:“我和你是要好朋友,怎么好意思剪你的边呢?”辛修甫皱着眉头道:“这里头另有一个道里,你难道昨天还没有听清楚么?”秋谷想了一想道:“既然如此,只好且去试他一下。

料想凭着我这样的一个人,也还不至于退避三舍。“

王小屏和陈海秋两个人在旁听了,全然不懂,不知道辛修甫和章秋谷说的是那一路的话儿。陈海秋本来是个性急的人,那里忍得住,大声嚷道:“你们说的都是些什么话儿?我们一句都不懂。”辛修甫笑道:“你不要性急,慢慢的和你讲就是了。”陈海秋再三根问,辛修甫只是微微的笑,一言不发。陈海秋没奈何,只得由他。

等了一回,赛金花娉娉婷婷的从外面进来,看了秋谷一眼,便去坐在修甫身旁,密密切切的讲了一回。辛修甫又在赛金花耳边说了几句。赛金花把头一低,星眸斜漾,宝靥生红,偷偷的瞟了秋谷一眼,口中却不说什么。辛修甫一面笑着,一面又附耳和赛金花说了几句。赛金花忍不住“扑嗤嗤”的笑将出来,把一个指头对着辛修甫头上用力推了一推,口中说道:“耐格个人真正气数得来!随便啥格闲话总归说得出格,啥人有工夫来听耐呀!”说着立起身来,走过章秋谷身旁,趁着大家没有留神,暗暗的把章秋谷的衣服拉了一把。章秋谷被他拉了一拉,不由得心上有些摇动起来,也对着赛金花回头一笑,还他一个眼风。辛修甫看了,只作没有看见的一般,只催着那班娘姨、大姐搭开桌椅,大家碰起和来。

碰了四圈,赛金花指挥那班娘姨、大姐摆出齐齐整整的一桌菜来,这是辛修甫预先招呼的。当下修甫便邀着大家人座,大家免不得叫局吃酒的闹了一回。到得后四圈麻雀碰毕,已经差不多十一点钟。辛修甫同着王小屏等别了章秋谷要走。章秋谷究竟觉得有些不好意思,便也立起身来要和修甫同走。修甫呵呵大笑道:“你不用和我打哈哈儿,你只老老实实的在这里伺候这位状元夫人的为是。须要小心谨慎,好好的出力当差。万一个当差不力,给人赶到地板上来睡觉,却与我不相干的。”

陈海秋到了这个时候,心上方才明白,便对着赛金花嘻嘻的笑道:“你今天遇着了这样的一个有名人物,你要自己留意些儿。”赛金花红着个脸,口中说道:“唔笃总归是实梗瞎三话四,真正歪嘴吹喇叭邪气。”辛修甫笑着,大踏步走了出去。正是:

花低月亚,香融玉杵之云;李代桃僵,春暖金茎之露。

不知后事如何,且待下文交代。





第一百七十九回 真阅历发明攻战术 正比例研究床第谈





且说章秋谷住在赛金花那里,这一夜的情景果然比别人不同,真个是:春魂照夜,玉艳临波;一桨穿红,双桡剪绿。熨贴云鬟之影,惺忪暗麝之香。徐娘之丰调依然,名士之风怀未减。香肩倚月,飞来帐底之云;檀口偎云,捧出怀中之月。娇喉乍颤,雀舌初舒。汗融合德之肤,春满华池之液。金釭闪闪,玉漏丁丁,好梦未醒,罗帏不动。这些秾情艳语,在下做书的也不便细细的形容,只好将就着说个约略罢了。

到了明天,章秋谷和赛金花刚刚起来,辛修甫已经来了,走进房来。赛金花见了辛修甫,不由得满脸通红,立起身来,一溜烟走到后房去了。辛修甫细细的把章秋谷脸上看了一看,摇一摇头道:“看你这个样儿,色势不好,不要是打了汇票罢?”

章秋谷微微一笑,也对着辛修甫摇一摇头,口中低低的说道:“等回儿和你细细的讲。”辛修甫随便坐下,和秋谷谈了一回。赛金花也从后房走了出来,对着辛修甫总觉得有些腼腆。辛修甫笑道:“这是三面言明的事情,你何必还要这般模样?”

赛金花听了,越觉得不好意思起来,斜溜了辛修甫一眼,别转头去。辛修甫和章秋谷坐了一回,两个人都起身要走。赛金花留他们吃了饭去,秋谷不肯道:“我还有公事要去料理一下,等回儿再来罢。”赛金花立起身来送了几步,对着秋谷把头略略的侧了一侧,眼珠微微的动了一动。这一对水汪汪的秋波里面,好象有万千情愫传送出来的一般。秋谷见了一笑,把头点了一点,便一直同着辛修甫向书局里头去了。

到了晚间,便是辛修甫在龙蟾珠那里请客,请的客人无非原是章秋谷等一班人。

入座之后,辛修甫便问章秋谷道:“你们昨天究竟怎么样?”秋谷微微笑道:“你的话儿果然不错。虽然比不得什么鸡皮三少的夏姬,却也差不多像个内视丰盈的赵飞燕,果然是个劲敌。如今上海滩上的那班人物,除了胡宝玉之外,只怕第二个就要轮着他了。”

王小屏等起先听了辛修甫的说话还不甚懂,如今听了章秋谷的这一番说话便心上都有七八分明白。刘仰正第一个开口问道:“秋谷,你平日之间常常的对着我们说些大话,说什么有彭祖御女之玉,如今我倒要请教请教,要你把这个御女之术讲给我们大家听听。”这句话儿方才出口,陈海秋先拍手道:“仰正的话儿一些不错,我正在这里有疑惑,看看那班倌人,和他没有交情的便罢,只要和他有了交情,十个里头倒有九个是和他要好的。这个里头一定有个道理,今天定要你讲给我们听听。”

秋谷笑道:“你们要我讲不难。但是这件事儿是极秽极亵的勾当,却教我一时怎样的讲得出口来?万一将来有个什么人,把我们这些人的事迹编成一部小说发行起来,岂不是污了看官们的眼睛么?”

辛修甫道:“你这个话儿虽然不错,却是只知其一,未知其二。将来万一个有人把我们的事实编成小说,这样洋洋洒洒一部绝大的嫖界小说,那些嫖客的胡涂、倌人的伎俩、魑魅魍魉的现状、神奸巨蠹的面目,一桩桩一件件的,都载得明明白白,独独这件最紧要的真实工夫,却没有提起一个字儿,未免是个缺点。你又何妨把这个里头的精微奥妙之处说给我们大家听听,公诸同好呢?”秋谷听了,想了一想方才笑道:“既然你们大家都要请我演说,我也无从推托的了。但是把这样龌龊的事情形诸齿颊,实在觉得有些不雅。如今我把别的事情和这件事情作一个正式比例,免得旁人听了不好意思,你们以为何如?”辛修甫笑道:“你果然能够把别的事情做个比例,自然更好。你只顾发议肆论,我们大家都在这里洗耳恭听就是了。”

秋谷听了故意咳嗽一声,口中说道:“你们大家静听,我要升座说法了。”大家听了都不觉一笑,果然一个个都正襟危坐,静静的听着。

秋谷把眼光四面飞了一个转,看了他们这般模样,不觉大笑起来。大家见了,都不知他笑的是什么事情,问他为什么平空这般好笑。秋谷笑道:“你们这班人听了这般秽蝶的话儿,便大家都伏伏贴贴,鸦雀无声的静听。要是今天有个人在这里讲起什么正心诚意的工夫、葆德崇性的学问来,只怕你们众人不用等他开口,早把他轰驴马的一般轰出去了。照这样的看起来,如今世上那班人的人格,真是一天不如一天、一个低似一个了。你想我们这班人尚且如此,那些不学无术的小人更是可想而知的了。”辛修甫不觉笑道:“你这几句话儿骂得结实,如今也没有工夫和你斗口,请你快些的开篇罢。”陈海秋也道:“我们骂也给你骂了,你若不好好的讲些玄精微理出来给我们听,我们大家就要鼓噪了。”

秋谷方才慢慢的说道:“如今我把两个开战的国度作个正式比例:男子的对于女子,好象是个悬师千里、深入敌境的国度一般;女子的对于男子,好象是个坚守险阻、声色不动的国度一般。那悬师千里、深入敌境的人,费了无数精神气力,始终还是不知道路,不谙虚实,事倍功半,未免总觉得要吃亏些儿。那坚守险阻、声色不动的人,却是安安逸逸、随随便便的,不发一矢,不出一兵,凭着那敌人在那里胡闹,只作没有知道的一般,事半功倍,自然的总要得些便宜。一边是以劳待逸,一边是以逸待劳,这个里头已经差了一个底子。所以明明的两个强国,工力都是悉敌的,却有了这个缘故在里头,攻守异势,劳逸殊形,就自然而然的有些支吾不过起来。那以逸待劳的人,却是到了粮尽兵疲、十分支吾不来的时候,究竟还好勉勉强强的敷衍一下。那以劳待逸的人,却是不打败仗便罢,若是打了一个败仗,那就一败涂地,全军覆没,再也收拾不来的了。总而言之,那以劳待逸的人对于那以逸待劳的人,一定要估料着此国的攻战力比彼国的攻战力胜过一倍,方才可以刚刚得个平手。若是彼此的攻战力大家相等,断没有不打败仗的,你们把这个情形细细的去想一想,就知道我的话儿是阅历有得之谈了。”众人听了,大家垂着头想了一想,不由得都点一点头。

王小屏又问道:“你这些话儿,不过是皮毛上的议论,我还有一句话要问你:照你这样的说起来,男子的对于女子,是以劳待逸;女子的对于男子,是以逸待劳。

一定要此国的攻战力胜过彼国一倍,方才得个平手;就是彼此工力相当,也一定要打败仗,是不是呢?“章秋谷道:”这个自然。“王小屏道:”万一个遇着了个攻战力远胜于我们的,这便该应怎么样?还是抱头鼠窜、临阵脱逃呢?还是硬着头皮,勉强迎敌呢?“秋谷笑道:”若果然遇到了这样的人,这却没奈何,要用奇兵取胜的了。“王小屏道:”怎么叫作奇兵?这个奇兵又是怎样的一个用法呢?“

秋谷道:“若是遇着了这样的人,躲又躲不掉,逃又逃不脱,只好到了临阵交绥的时候,故意慢慢的虚与周旋,千方百计的挑逗他,直挑逗得对阵的敌人战心勃发,急于求斗,这一边却养精蓄锐的按兵不动。一边是火杂杂的怒如虓虎,一边静悄悄的屹若长城。直等得敌人求战不得,十分性急,这一边却才慢慢的布阵出队,慢慢的和他合战。那敌人的性情,不是刚刚合阵就会战酣兴发的。那起先合阵的时候,也不过是些虚空的架势。这一边却只是随随便便的应酬他,敌来我去,敌去我还,不用战斗的全力。直要到得对阵的敌人战酣兴发,二十四分的性急起来,那中军的马队拼命的向前近凑,两边的枝队拼命的四面包抄,那远远的游击队也四面紧紧的合将拢来。到了这个时候,这一边方才用出十二分的全力来,奋勇当先,狂冲乱突,不按着什么阵式步法,只一味的和他垓心肉薄,短兵相接。这个时候,那一边的精神差不多已经发越得干干净净,成了个强弩之末的势儿。这一边却是保守着全身精力,没有一丝一毫的亏损。一个是一鼓作声,一个是三鼓气衰,凭你两下的战斗力不能相等,这样的一来,自然的五雀六燕,轻重适当,刚刚得一个对手。这是我从这个里头细细的再三考察,考察出来的不二秘方。你们想想我这个话儿可是不是?”众人听了,一个个就如维摩听讲,顽石点头,不因不由的大家都微微的笑。

辛修甫道:“今天这番议论,倒也真个是闻所未闻。倒难为你居然考察得十分切实,比起如今那班出洋考察的大人先生来,考查详细得多了。”大家听了都不觉笑起来。章秋谷笑了一回,又对着众人说道:“大概如今世上的人,那班以逸待劳的人,大半都是战斗力十分强盛的;那班以劳待逸的人,却又大半都是失了战斗力,不能临阵的,所以如今的人,只有男子躲避内差,从没有女子躲避外差的。就是或者有个把女子躲避外差的,也不过千万分中的一二罢了。”众人听了,又都哈哈的笑起来。

章秋谷正和辛修甫等说得十分高兴,忽然从秋谷背后伸出一只纤纤玉手来,把章秋谷拉丁一把道:“唔笃杂格乱拌到底来浪讲啥物事?为啥倪来浪听仔半日,一句才勿懂呀?”秋谷回头看时,只见一个修眉俊眼的丽人,笑吟吟的站在他身后。

那一种清华的姿态,好似那春云乍吐,华月初升。原来不是别人,就是自己的相好陆丽娟。便对着他一笑道:“我们讲的是我们的话儿,就和你们讲了,你们也是不懂的。”陆丽娟听了也不再问,只附着秋谷的耳朵道:“耐生病刚刚好得勿多两日呀,自家总要保重点,勿要来浪外势瞎俏,阿晓得?”秋谷听了点一点头。陆丽娟又道:“就是花酒也少吃两台格好,搳脱两个铜钿呒啥希奇,自家格精神要紧,二少阿是子”秋谷听了陆丽娟几句这软绵绵的话儿,心上竟着实的动起来。伸过一只右手,把陆丽娟的手紧紧握着,四目相对,呆呆的看了一回,盈盈不语,脉脉含情。

这个时候,辛修甫等也都在那里应酬自己的相好,没有人来留意他们的举动。两个人互视了一回,又密密的谈起心来。正是:

徐娘身世,飘零薄命之花;飞燕光阴,惆怅慢天之絮。

不知以后如何,请待下文分解。





第一百八十回 忆前尘同游钓鱼巷 怀旧事重访莫愁湖





且说章秋谷趁着大家都在那里和倌人讲话,两个人便细细的谈起心来。在陆丽娟的意思,狠想章秋谷和他还了债项,娶他回去。章秋谷明知道这件事情,太夫人那里一定办不到的,况且自己已经娶了一个陈文仙,当初娶的时候陈文仙又没有要他的身价。如今若要再娶一个倌人回去,不用说太夫人面上不答应,就是陈文仙面上也未免有些对他不起。便恳恳切切的把自己为难的情形和陆丽娟讲了一遍,道:“像你这样的人,肯一心一意的嫁我,我岂有倒反不愿意的道理?但是我家里头已经有了一妻一妾,如今再把你娶了回去,我自己心上想想,在你分上也觉得有些交待不过。你们当倌人的嫁个人,也是一生一世的大事。不要到了那个时候万一个有些不合起来,那时进退不得,岂不误了你的终身?我们如今看起来是狠要好的,将来娶了回去,一妻两妾,未免总有口舌相争的地方。到了那个时候,弄得个有始无终,你叫我又怎的对你得起?况且我们老太太的家法又是十分利害,你嫁了过去,那里拘束得来?与其到了后来为好成歹,大家都不好看,不如还是这个时候硬着心肠,不要冒冒失失、懊悔嫌迟的好。”陆丽娟听了,知道章秋谷说的是真话,拉着秋谷的手一言不发。呆了一回,不知不觉的眼波溶溶,眉峰紧紧,几乎要掉下泪来,口中说得一句道:“阿是真格呀?”秋谷低低的说道:“我们这样的交情,那有哄你的道理?总是我章秋谷没有福气,消受不起你这样的一个人。”

正说到这里,忽然半空中飞下一件东西来,把章秋谷和陆丽娟一齐裹住。两个人不由大大的吃了一惊。陆丽娟吓得高声叫道:“啥人呀,勿要实梗哩!”章秋谷虽然叫了一惊,却明知道一定是别人和他取笑,连忙伸出手来,把头上裹的那件东西撕掳开了。举眼看时,原来是陈海秋的马褂。看着他们两个人讲得这般热闹,悄悄的把一件衣服往他们两个人头上一蒙。大家见了,都拍着手笑作一团。章秋谷也不觉跟着众人笑了一阵。随手把那件马褂“扑”的往窗外一丢。陈海秋连忙来夺时,那里来得及?大家见了,不免又笑一阵。陆丽娟还口中咕噜道:“陈老末总是实梗,倪吓得来!”说着,早有相帮把陈海秋的马褂送上楼来。陈海来看了一看,见还是干干净净的,没有什么污泥在上面,便也不说什么。一会儿大家散席,章秋谷别了主人先走。

光阴迅速,不知不觉的又过了一年。到了秋间,恰恰的又是恩科乡试。章秋谷的性情,本来原不把富贵功名放在心上。想要不去时,当不得他夫人和陈文仙再三相劝。太夫人又和他说道:“我们姓章的上代祖父,多半是科第出身。我虽然未见得一定逼着你去干功名,但是你若果然能中了一个举人,你的读书排场也就算交代过了。况且他们两个人心上总想你中个举人,心中二十四分的期望,你就去走上一趟也好。”

章秋谷听了太夫人的这番说话,只好连声答应。收拾了行李,匆匆的上了轮船竟往南京来。到了南京,免不得合了几个同伴租了一处文德桥下的河房,三间两进,甚是宽敞。录遗过了,时候还早得狠,便有几个朋友来拉着秋谷去逛钓鱼巷。那钓鱼巷里头挨门沿户的都是些娼寮。秋谷同着那几个朋友拣了一家有名的薛家,进去坐了一回,见房间倒收拾得十分齐整。无奈那些倌人,大半都是些扬州人,走起路来,一撅一撅的甚是难看。秋谷见了不住的摇头,连连的催着那几个朋友快走。

那几个朋友没奈何,只得走了出来,在路上和他分辩道:“这个地方是南京最有名的妓院,你难道一个都看不中么?只怕你的眼睛也未免太高了些。”秋谷笑道:“我生平最不赏识的就是扬州人,如今见了许多扬州的螃蟹,满口‘辣块辣块’的,倒还不必去管他。更兼浑身上下都是直撅撅的,没有一些儿柔媚的样儿,我眼睛里头那里看得上这样的人?”那几个朋友道:“照你这样的说起来,上海的那班倌人你也是看不上的了?”秋谷道:“上海的倌人那里像这班宝贝的模样?一个个都是语言柔软,态度温存。就是面貌差些,也觉得楚楚堪怜,婷婷可爱。凭着这班宝贝的样儿,叫他去和上海的倌人拾鞋皮,还未见是得要他呢!”那几个朋友道:“你这几句话儿,未免有些一偏之论。照着这般的说起来,是上海的倌人个个都是好的,别处的倌人个个都是不好的了。况且你这般偏见,只取身段,不取面貌,难道叫个无盐、嫫母来学些娉娉袅袅的丰姿,你也当他是好的么?难道身段不好的人,就是真个的天生丽质,你也不赏识的么?”秋谷道:“这个话儿却不是这般说法。

你们要知道,如若真个奇丑非常的无盐、嫫母,断断学不出娉娉袅袅的丰姿。就是勉强学些,也和那东施效颦一般,不见其美,只见其丑。那身段玲珑、语言伶俐的女子,就是面貌差些,一定都是中人之质,不是那缺唇龋齿、挛腰偻背的宝贝。至于天生丽质,我何尝不赏识?无奈如今的时候,要我找个平头整脸不甚丑怪的人,尚且难得的狠,那里还寻得着什么天生丽质?若是果然见了这样的一个人,我也自然有目共赏的。“

那几个朋友听了秋谷的这番说话,一个个都闭口无言。有一个人还在那里咕噜道:“这些地方原不过是逢场作戏,何必这样的顶真?”秋谷笑道:“我看你的样儿,狠有些失魂落魄的,十分迷恋。你还没有知道那班妓院里头的倌人,都把我们这班乡试的人唤作考呆子,专骗我们考呆子的钱。面子上虽然勉强应酬,实在心上狠有些不愿意。你只看方才那个什么巧云,口中一面和你说话,两只眼睛却骨碌碌的看着别处,正眼儿也没有剪你一剪,就可想而知他们是勉强敷衍的了。”

那几个朋友听了秋谷的话,细细的想了一想,觉得果然不错,便大家都向秋谷说道:“你说的话狠不差。他们既然不愿意我们光降,我们有的是钱,难道还怕没有使用的地方么,何必再去送给他们用?”秋谷拍手道:“这几句话儿才说得十分明白。我们花了银钱,原是要想寻开心的。不要寻开心没有寻到,倒遇着了几个妖魔鬼怪一般的人物,回来吓死了,那个给我们抵命?”这几句话把大家说得哈哈大笑起来。章秋谷同着几个朋友一面走着,一面说着,一直走到章秋谷寓中。大家坐了一回,秋谷留他们吃了晚饭,方才走了。

到了明天,秋谷一个人雇了一只游艇,在秦淮河里荡了一回。荡到钓鱼巷那边一带,只见杨柳垂波,珠帘拂槛,那些娼寮里头的人,都一个个浓妆艳抹的坐在帘内,把珠帘高高的挂起,一阵阵的香气扑过来。秋谷约略看了一看,虽然看得不狠清楚,却倒觉得狠有些迷离掩映的丰神,比那当面平视倒反觉得好些。荡了一回,又从东往西荡过来。那些沿着秦淮河的河房,都深深的垂着湘帘,里面隐隐的露出许多鬓影钗光,遮遮掩掩的偷看那往来的游客。秋谷见了,不觉得心窝里面倒有些痒痒的起来。游了一天,倒觉得十分畅快。又顽了一天玄武湖,顽了一天莫愁湖,觉得那玄武湖绿滟波光,云横山色,遥峰挹翠,远树含烟,倒狠有些远水近山的景致。惟有那莫愁湖却没有什么景物,只供着个中山王和莫愁的小像。正是:

英雄老去,湖山一代之愁;金粉消亡,家国千年之恨。

不知以后如何,且看下回,便知分晓。





第一百八十一回 吃花酒騃儒得意 入乡闱词客观光





且说章秋谷在莫愁湖亭上徘徊了一回,看着那几朵开残的莲花,赏玩一会。又看着中山王和莫愁的小像,细细的端详一回。只见一个是白面长须,英姿照日;一个是风鬟雾鬓,倩影惊鸿。秋谷见了,不免也有些心中感慨起来。在湖亭上泡了一碗茶,坐了一回,直到红日西斜,晚风吹袂,方才慢慢的回来。又在寓里头过了几天,已经到了八月初旬的时候。秋谷到了这个时候,便也未免要抱抱佛脚起来,把那些带去的书籍翻出来,略略的看了一遍。

这一天正在寓里头静静的坐着,忽然又来了一个同乡朋友叫作黄少农的,要拉他去钓鱼巷吃酒。秋谷心上狠有些不愿意去,只推说身体有些不快,不能出门。黄少农不由分说,拉着就走。拉到钓鱼巷一个韩家老班里头,便有一个倌人出来应酬,秋谷抬头看时,只见这个倌人生得圆圆的一个脸儿,觉得团头团脸的,也晶评不出什么好歹。黄少农却得意洋洋的指着那倌人对秋谷说道:“这是南京有名的韩家小翠子,你看他生得怎么样?”秋谷又细细的打量了小翠子一眼,觉得虽然没有什么奇形怪状的丑相,却也没有什么娇娆袅娜的姿容,不过勉勉强强的看得过去罢了。

看了一看,没本事说他不好,只得勉勉强强的说一声“好得狠”。黄少农听得秋谷赞他的相好,心上二十四分的高兴。小翠子也扭扭捏捏的扭捏出许多的身段来。秋谷看了,只是暗暗的好笑。

黄少农略坐一坐,便取过笔砚来,写了几张请客票,叫了男班子的掌班进来,身边摸出一块钱来,连着请客票一古脑儿都交给他,口中说道:“这一块钱是给你的车钱,快些去给我请客。”那男班子答应一声,接了过去。章秋谷看着,已经觉得二十四分的诧异。正要开口,忽然又见小翠子抢步过来,斜着眼睛把那男班子手里头的请客票看了一眼,半笑不笑的对着黄少农道:“你请的客人狠多,给他一块车钱只怕不够罢?”黄少农听了点点头,连忙又拿出一块钱来交在那男班子的手内。

只把一个章秋谷看得心上更加诧异,真个是见所未见,闻所未闻。

一会儿客人到了,排上席来。黄少农见秋谷没有相好,想要荐个相好给他,秋谷再三再四的推辞。黄少农那里肯听,不由分说,硬硬的荐了一个什么薛亚仙给他。

章秋谷举目看时,只见这个薛亚仙生得矮矮的一个身材,匾匾的一个脸儿,眉眼不甚周详,鼻梁有些四塌,也是个中等以下的人材。秋谷见了,把眉头皱了一皱,也不言语。黄少农却指着薛亚仙向秋谷道:“你不要轻看了他,这也是南京地方大名鼎鼎的人物。”秋谷听了,不觉鼻子孔里“哼”了一声。黄少农又对着薛亚仙道:“这位章老爷在上海的时候,嫖界里头狠有声名的,你须要好好的应酬,将来我还要吃你的喜酒呢。”

薛亚仙听了,把手帕子掩着嘴笑了一声,回过头来,上上下下的把章秋谷不住的打量。章秋谷被他看得不耐烦起来,别转头去。原来薛亚仙见了章秋谷这样的少年英俊,气宇非常,心上倒着实有些垂涎,便存着个屈身俯就的意思。见章秋谷只是淡淡的不理他,便故意找些话儿说出来和章秋谷讲,章秋谷也只得随随便便的应酬几句。一会儿,竟撒娇撒痴的拉拉扯扯起来,对着章秋谷不住的扭头掉颈,卖弄风骚,做出无数的丑态来。章秋谷看了他这般做作,不由得心中暗暗好笑,觉得甚是肉麻,周身的鸡皮疙疸都森森的直立起来,心上二十四分不愿意,只得假托腹痛;出了席去躺在榻上。无奈这位薛亚仙紧紧的跟着,问东问西,十分的献勤讨好,直把一个章秋谷拘束得如受桎梏,如坐针毡,又好笑,又好气,却又说不出来。好容易巴得薛亚仙走了,方才如释重负,畅快非常。黄少农糊里胡涂的,还对着章秋谷把大指一竖道:“何如?我荐给你的人不错么?你们两个人初次相逢,就是这般的要好,论理该应谢谢媒人才是。”

章秋谷正含着一块烧鸭在嘴里还没有咽下去,听了黄少农这番说话,再也忍不住,“扑嗤”的一声一口气冲上喉咙,要笑出来。口中的这块烧鸭就留不住了,“扑”的从口中直飞出来,刺斜里飞过去,直飞到一个十四五岁的雏妓面上。说也凑巧,刚刚不偏不倚的直中在他鼻梁上面。大家都哄然大笑起来,秋谷自己也觉得十分好笑。连忙看那雏妓时,原来是一个姓杨的客人叫的,却不知道他叫什么名字。

正默默的坐在那里,不提防一块烧鸭劈面飞来,刚刚飞在鼻梁上面,躲闪不及,只得把头一偏,那块烧鸭就落在地下。那雏妓出其不意,倒吃了一惊,连忙用手巾往脸上按了一按,身边取出镜子腮了一照。见面上油了一块,连忙讨盆脸水抹了一把,口里头喃喃呐呐的说了几句,也不知说些什么。

章秋谷觉得有些过意不去,等他抹过了脸,便走过来对着他就是深深的一拱到地。那雏妓倒吃了一惊,口中说道:“这是怎么!这是怎么!”章秋谷立起身来,口中说道:“方才一个不小心,把一块烧鸭直飞在你的脸上,特地来和你陪个礼儿。”

那雏妓微微一笑道:“这也算不得什么大事,何必这般客气?”章秋谷听了那雏妓说话的声音十分圆转清脆,不由的抬起头来把他打量一下。只见他高高的挽着一个云髻,淡淡的画着两道蛾眉,檀口含朱,横波挹翠,身材纤小,骨格停匀,虽然不是什么倾城倾国的佳人,却狠有些宜喜宜嗔的丰态。比起那小翠子和薛亚仙来,直是天壤云泥,相差甚远。秋谷看了,不由的心中动了一动,暗想:这个地方一般也有这样的人材,可见天地生才,原是不拘资格的。想着,便故意上上下下的把那雏妓细细的看,看得他脸上红起来,啐了一口道:“你上上下下的看些什么?难道要和我画个小照,回去供在家堂里面么?”秋谷笑道:“你不要见怪,像你这样的标致人儿,就是多看一会,也是前生修来的福分。”

看官听着,原来天下的女子,只要听得别人赞他貌美,心上总是高兴不过的,何况是个堂子里头的人物?听了章秋谷这几句话儿,不知不觉的酣迷迷、软洋洋,钻进心坎里去,登时春风满面的对着秋谷道:“你不用这般混说,像我这样的一个人,那里合得上你们的眼睛?”章秋谷笑道:“阿唷,你不用这般客气!若再要这般的谦让起来,把这里的房子牵得坍掉了,却不与我相干。”那雏妓斜着眼睛瞅了他一眼道:“算是你一个人会讲话,快些去坐了罢。”说着,便轻移莲步,慢慢的走过去,刚刚和章秋谷擦肩挨过。章秋谷趁着众人不见,暗暗把他手拉了一把。那雏妓秋波澄澄的也不言语,只把嘴对着那姓杨的客人努了一努,又摇了一摇头。

秋谷会意,便也慢慢的归座,悄悄的问黄少农:“这个雏妓叫什么名字?”少农大笑道:“你敢是看上他么?他叫银喜,就是这里韩家本班的。我来和你们做个介绍人,转一个局就是了。”秋谷听了,便回过头来看了那姓杨的一眼。只见那姓杨的满面怒容,正襟危坐,只当没有听见的一般。秋谷知道那姓杨的醋劲发作了,连忙朝着黄少农连连摇手。黄少农看了姓杨的这般模样,料想这个媒人不是轻易做得成的,便也笑了一笑不说什么。只凭着这个章秋谷和银喜两个人在席上眉黛传情,秋波送睇,案底之莲钩暗蹴,尊前之宝靥轻回。大家都在搳拳吃酒的十分热闹,却没有看见他们两个人的这番情景。只怕自此以后,竟是这般的暗渡蓝桥,私谐鸳侣,也未可知。

这且不必去说他,只说章秋谷在寓里头休息了几天,准备着秋风一战。到了初八日进场的那一天,秋谷进了号舍。那跟进去的家人把号帘挂了起来,钉好了号围,又把食篮收拾好了,笔砚纸墨都取了出来,方才出去。秋谷在号里头没有什么事情,便立在号门口闲看。看了一回,忽然见隔壁号里钻出一个人来,赤着膊,盘着辫子,一张漆黑的脸儿,两个绝高肩膀,粗眉糙目,一部大大的连鬓胡须,走出号舍,刚刚和秋谷打个照面。秋谷鼻子中间,就觉得有一阵汗臭和着那一股狐腋的臊气直冲进来,秋谷连忙别转头去掩面不迭。

只见这个人走出号舍东西张望了一回,忽然又走进号去,捉出一个绝大的鸭子来,左手拿着一把明晃晃的牛耳尖刀,右手把那鸭子紧紧的捺在地下,那鸭子还叫个不住。章秋谷看了觉得十分诧异,不由得走近一步细细的看他。只见这位宝贝左手拿着刀,调转右手,照着那鸭子的项下就是一刀,鲜血直冒出来。那班同号的朋友见忽然有人在这里杀起鸭子来,也觉得甚是诧异,大家都赶过来看他。只见他揎拳掠袖的,向号军要了一瓤热水,把鸭子的毛持得干干净净。又拿出一个瓦罐,生起一炉火,把那鸭子慢慢的煮起来。正是:

出门一笑,秋风吹桂子之香;下笔千言,璧月吐奇葩之彩。

未知以后如何,请待下回再行交代。





第一百八十二回 闹新闻撞墙翻瓦罐 洒霜毫论史出奇文





且说我们中国乡试的号舍,原是最逼狭的地方。那间号舍的地位,前后左右方圆不到三尺,刚刚只容得一个人的坐处,连晚上睡觉的地方都没有。要睡起来,只好和狗一般的,就在那问号舍里头圈着,那里还有什么地方安放对象?那班乡试的人都把一个铁叉插在号舍对面的墙缝里头,铁叉上有个圈儿,把个小小的炉灶就放在圈儿里面,烧菜煮饭都在这付炉灶上头。如今这个宝贝也把这个炉子如法泡制的放在墙上,慢慢的把那只鸭子煮起来。无奈他这付炉灶也不知从那里定制来的,果然的硕大无朋。那号舍里头的过弄只有一尺多宽,给他这样的一来,差不多就占了一半地位,来往的人已经都要侧着身子过去。更兼炉灶上面加上一个绝大的瓦罐,煮得热气腾腾的。那班来往的人到了这个地方,没奈何只得低着头,斜着身体过去。

章秋谷看了这般情景,觉得心上也狠有些嫌他,暗想天下怎么竟有这般奇事。

正想着,只见一个同号的朋友叫作石仲瑛的,走了过来。见秋谷站在号舍外面,便立定了脚,随意和他闲谈。忽然间回过头来,刚刚那瓦罐里头的热气丝丝缕缕的直腾上来,直扑到石仲瑛脸上。那鸭子本来没有洗得干净,那热气里头却夹着一股臊气,直冲人石仲瑛鼻子里头。石仲瑛掩鼻不迭,觉得一个恶心,嘴里头吐出一口清水来。秋谷见了,不觉有些好笑起来,便把方才的事情,打着乡谈和他讲了一遍。

石仲瑛回过头来看了一看,口中说道:“天下那有这样的人?竟带着活鸡活鸭进场烧煮,想来是个厨夫的儿子。我们何不想着法儿,跑过去撞他一撞,把他的宝货撞掉了,叫他不得到口,岂不爽快?”石仲瑛说到这里,只见那考生回过头来,恶狠狠的瞅了他们两个人一眼。秋谷见了,便悄悄的把石仲瑛拉了一把,低低说道:“你不要随口混说,他懂得我们的话儿。”石仲瑛笑道:“他就是懂得我们的话儿,我们也不怕他。”

正说着,只见远远的一个长大身材的人大摇大摆的走来。秋谷眼快,早已看见是东方小松的族弟东方柏生。便道:“柏生来了。”那东方柏生远远的一直跑来,直走到秋谷面前,方才看见了秋谷和仲瑛,口中叫道:“秋谷兄,仲瑛兄,原来你们都在这里。”一面说着,眼睛望着他们两个人直撞过来。秋谷看势头不好,东方柏生的身体,离那煮鸡子的瓦罐中间,相隔不过只有四五寸的地位,连忙说道:“小心些,留心别人的东西!”一句话还没有说得完,早见东方柏生一个转身,那一只右手轻轻的在那瓦罐上带了一带,只听得“阿呀”的一声,那个瓦罐早翻了一个身,从炉座上直跌下去。“格啷啷”一声,把个瓦罐跌得一个四分五裂,连那煮的鸭子也丢在地下。

东方柏生呆了一呆,正要开口,早见那考生拧拳掳袖的直抢上来,劈胸一把拉住了东方柏生的衣服,口中嚷道:“你走路不带眼睛么,乱撞你娘的什么?快快的赔我鸭子和瓦罐来!”石仲瑛见了,连忙走上一步,劝道:“朋友,我们有话好好的讲,何必动粗?快放了手,有话总好讲的,况且他是一时无心之失,不是有心和你作对的。”那考生把石仲瑛看了一看,睁起了一双眼睛,“呸”了一口道:“你还说他是无心。你们两个方才已经在那里商议了好一会,要想法子撞翻我的鸭子,叫我不得到口。分明是你们三个人有心串合,故意前来寻我的开心。还亏你有脸来和他讲情,我不和你讲话已经是好的了。”石仲瑛平空的碰了他一个大大的钉子,一时倒也回答不出什么来。那考生紧紧的拉住了东方柏生的胸前衣服不肯放松,一面还口中嚷道:“你们几个人想要来寻我的开心,你们也没有打听打听我是个什么人!”

章秋谷听了一回,看着那考生十分放肆,口中牵枝扯叶的只顾乱嚷,不觉怒从心起,抢步上前,把那考生的手腕轻轻的一把握住,往下一顿,那考生不由的“阿呀”了一声,不知不觉的就放了手。秋谷正色对他说道:“我们都是读书人,有理讲理,为什么要这样动手动脚的,那里还像个斯文人儿?”那考生被秋谷顿了一顿,知道这个人气力不小,不是好惹的,只得勉强说道:“你们几个人有心撞翻了我的鸭子,你如今又无故干预我的事情,难道你是不讲理的么?”秋谷大笑道:“你倒说我不讲理,你恃蛮拉住了别人的衣服不肯放手,讲理的人是应该这样的么?我不过是个旁人,好意解劝你们一下,怎么倒是我不讲理?”

那考生道:“他撞掉了我的东西,难道我不要拉了他,叫他赔偿的么?”秋谷道:“他打碎了你的东西,你只顾好好的叫他赔偿就是了,为什么要这般粗卤莽撞,动手动脚?他撞了你的东西,你要叫他赔偿你的东西,你扯了他的衣服,却叫那个赔偿他的礼面呢?”那考生听了,顿口无言了一会,方才气忿忿的说道:“你们大家串同一气,有心毁坏我的东西,和无心毁坏的不同。”秋谷大笑道:“天下的事情只要杀人偿命,欠债还钱,管什么有心无心。有心也是这个样儿,无心也是这个样儿,只要偿还了你的东西,就是有心便怎么样呢?

那考生听了,口中支支格格的不知想说什么,却一时说不出来,停了一停方才说道:“你们须要赔还我的原物。”秋谷大笑道:“你要赔还原物,非但没有这个例,而且也没有这个理。亏你读书明理的人,怎么讲出这样的无意识的说话来?”

那考生听了满面羞惭,无言可答。秋谷便取出两块钱来,递在那考生手内道:“这两块钱赔你的鸭子和瓦罐,好不好?”那考生见了白晃晃的两块钱,顿时改了满面的笑容道:“论理不该和你老人家较量,只是两块钱委实少些,请高升些儿。”秋谷见了微微冷笑,又取出一块钱来给他道:“你只要肯要钱,事情就好办。”那考生把三块钱揣在腰内,口中还谢了秋谷一声。

东方柏生便也向秋谷谢了一声道:“今天幸而你在这里,和我解了一个围。”

石仲瑛笑道:“方才那般其势汹汹的样儿,一见了钱就软绵绵的变了一个样儿,可见如今世上银钱的力量大得狠。”秋谷道:“就是如今的那班王爷、中堂,平时见了人那脸上好象刮得下霜的一般,只要一见了白晃晃的银子,就是见了他的父母妻子也没有这般的亲热,顿时春风满面,和气迎人。那班大人先生尚且如此,何况这样一个不成气候的饭桶?”石仲瑛听了狂笑道:“好好的说话,你的牢骚话儿又来了。”秋谷听了微笑不言。大家谈了一回,也就散了。

一会儿,听得三声大炮,明远楼上鼓角齐鸣,知道已经封了门。一会儿又封了号门,不许大家来往。到了晚间,秋谷觉得肚子里头有些饿了,便取出炒米,胡乱泡了一泡,就带着的火腿、熏鱼吃了两碗。又吃了一杯茶,便半半睡的合目安息。

起先睡的时候觉得浑身都不畅快,再也睡不着,翻来覆去的。直到二更将尽,却倒睡着了。睡到四更将尽,主考发下题纸,号军按着号数一号一号的送进来。秋谷蒙蒙眬眬的接了题纸,看也不看,随手放下,仍复睡去。直睡到晨鸡报晓,玉漏无声,方才睡醒。坐起身来,叫号军取些热水,洗一个脸,又胡乱吃了些干果糖点,方才展开题纸。看时,只见一张大大的题纸上刻着五道论题:第一题是“汉武帝时,征吏民有明当世之务、习先圣之术者,县次续食,令与计偕论”;第二题是“识时务者在乎俊杰论”;第三题是“谢安登冶城,悠然遐想,有高世之志论”;第四题是“张九龄上千秋金鉴录论”;第五题是“明太祖诏商税毋定额论”。秋谷看了这几个题目,觉得都狠容易,况且又都是素来知道的,连查也不用去查,略略的想了一想,便都有了主意。铺下草稿纸,提起笔来,振笔直书。这章秋谷本来是个有名的江南名士,真个是文不加点,倚马万言,平翻北海之潮,倒卷黄河之水。还不到十一点钟的时候,五艺早已脱稿。略略的休息一回,吃了饭,便誊真起来。一口气写到下午五点多钟,已经誊毕,又自己细细的看了一回。

正看着,只见石仲瑛从外面探进头来,看了一看,失惊道:“你都完了罢,好快手,好快手!我刚刚做了首次两篇,第三篇还只做了一半。”说着,便伸手过来,取了章秋谷手中的卷子,略略的看了几行,就啧啧叹赏道:“笔仗好得狠!逼真是胎息《史》《汉》的文法。”秋谷笑道:“我不要这般谬赞,你只看下去就是了。”

石仲瑛听了,便果然一行一行的看下去。看到第三篇上,看得得意极了,竟高声朗诵起来。只听得石仲瑛提着那正宫调的嗓子,一腔三板的读道:

入广武门而闻阮籍之唏嘘,登平乘楼而听桓温之太息,俯视天下,感慨系之。

盖尝读史,至谢安之为人,而叹其度之不可及也。古之君子,尚黄老之学,崇淡泊之治。内无所惧,外无所营。虽有帝王之尊、卿相之贵,雷霆震惊于前,虎豹奔走于后,而此心漠焉冥焉,终不为动。此平日学问有以养之,非镇物矫情之所能也。

晋之士习崇尚虚无,卿相以清淡为事,儒林以论答为能。安性好声律,期功之惨,不废丝竹,士大夫效之,遂以成俗。又尝与王羲之同登冶城,悠然遐想,有高世之志,当世非之。然其为政也,尽忠王室,竭忠辅卫。斯时也,内有权臣,外有强敌。

晋以偏隅之地、积弱之势,北面而争天下。胜败之机,间不容发;天下大势,岌岌可危。而安以谈笑应之,处之晏如,无所畏葸。卒能折桓温于内,败苻坚于外。悬一发于千钧,奠国家于盘石。其晋室之所以不风亡者,徒以有安在也。夫清净之学,沉思若愚,拊几若得;高见风云,俯视山水;啸傲天下,凌铄古今;以卿相富贵为敝屣,与天地精神相往来。安之为人,有类于此。观其与王坦之同迎桓温,坦之流汗沾衣,倒持手版;安从容就席,神色自若,亦可以见其度矣。或谓其闻谢玄之胜,至于折屐,矫情镇物,非大臣所宜。然三代以上,惟恐好名;三代以下,惟恐不好名。东晋之政,棼于乱丝,而安以淡泊治之,无内外相乘之乱。盖其经济足以应之,非特以黄老相尚而已也。其与羲之同登冶城,登高遐想,慨然有世外之志,而不以富贵功名为念,此其胸次为何如?而后人乃以小节议之,谓其矫镇,抑亦苛矣!

石仲瑛读了一遍,觉得爱不忍释。又反反复复的重看一遍,不觉击节叹赏道:“这几篇文字,雄浑高古,音节非常。而且顿挫宛转,丰神独绝,真个不愧是个古文的作家!”秋谷笑道:“你看看也还罢了,何必要说这许多应酬的套话?”石仲瑛道:“那一个说应酬套话的就是个乌龟。”秋谷大笑道:“骂得好,骂得好,算你会说何如?”石仲瑛回心一想,不觉也笑起来,口中说道:“你不要见怪,我是一句无心的话儿,不是有心骂你。”

章秋谷笑了一笑,便也向石仲瑛要做好的草稿来看。石仲瑛便在胸前一个卷袋里头取出草稿来,递给秋谷,笑着说道:“我没有你这般洋洋洒洒的笔仗。你看了有什么不妥之处,请你改削改削,不要客气。”秋谷笑道:“太谦了,太谦了,这‘改削’的两个字儿断不敢当。”一面把他的草稿看了遍,觉得见识也还开通,议论也不通达,只是笔力来得软些,气魄来得小些,未免有些小家气。便也随口赞了几句,又和他斟酌了几处不妥当的地方,石仲瑛方才走了。

又见隔号的那个考生走了过来,满头大汗的对着秋谷拱手道:“老先生这个时候五艺都一齐完了,佩服得狠!只是小弟有一件事儿要来求教。”正是:

鹿锦凤绫之艳,彩笔生花;珊瑚玉树之珍,文章有价。

不知那考生问的什么话儿,且待下文交代。





第一百八十三回 传急电游子还乡 开花榜庸奴得贿





且说章秋谷忽然见那隔壁的考生急得满头大汗,来和他兜搭说话,又说要请教他什么事情,心上早已明白,只说:“你有什么话,只顾请讲。”那考生陪笑道:“请问老先生,这二题的出处在什么地方?小弟查了整整的半天都没有查到。这样空空洞洞的一句话儿,教人从何查起?”秋谷听了,忍不住笑道:“你连这句话儿的出处都不知道么?这个容易得狠,待我查给你看就是了。”说着,便取出一本《御批通鉴》来,把那司马徽的一段话儿查给他看了一遍。只把这个宝贝喜得个手舞足蹈,抓耳挠腮,也不知怎样才好,再三谢了章秋谷,一步一摆的去了。秋谷暗想:这样的人也要充什么读书人,这样的眼前典故都不知道,直是个目不识丁的草包。

当下秋谷把自己的卷子细看了一遍,见没有什么舛误的地方,便也把他放人卷袋。又到石仲瑛那里去走了一趟。回到自己号里,也觉得有些两臂酸麻起来,便下了号帘,静悄悄的睡了一夜。到了明天,绝早的交卷出来。

到了二场,秋谷照旧进去。原来这个时候的考试,已经改了新章,不用什么经文八股,第一场是五篇史论,第二场是五道时务策,第三场是三篇“四书”义。秋谷看了第二场的五道策题,也都是狠容易的空策,用不着什么考据。只有第五题,是问“俄取高加索,并别设禁令以制山民”的事情,略略的要加些考证。

章秋谷进第一场的时候,笑话已经听了无数,什么把谢安当作谢灵运,又把张九龄当作明朝的宰相,这些笑话不一而足,秋谷都听在肚子里头。第二场的笑话更加多了,秋谷连听都听不尽许多。只听得对面号里有三四个人讲话的声音,一个人高声说道:“他问高中索是朔方何部,这个朔方就是北方,大约就是我们中国的北京了。只不知道这个高加索是那一府属的地方?”又一个说道:“他问的什么禁令,一定就是我们的大清律例。我们只要抄上几条律例,把卷子上挤得满满的,把那班房官吓上一吓也好。”

秋谷听了这些说话,几乎要放声大笑起来,暗想:这班宝贝真是饭桶中间的饭桶,也要来出这个丑做什么?笑了一会,也不去管他们怎样,只把自己的文字斟酌了一回,连忙誊真,又是第一个交卷出去。到了三场放牌,格外放得早些,十四夜间四更出了题目,十五一早就收卷放牌。原来南京本地的人赴试的都有这个规矩,一个个都要赶十五晚出场,好回去人月同圆的意思。章秋谷本来文思敏捷,这几篇“四书”又那里在他心上,提起笔来,一挥而就。到了十五一早,就去交了卷子,慢慢的出场,到寓里头睡了一天。

到了十六那一天,秋谷刚刚起来,忽然家人周升手里头拿着一封电报走了进来,把电报交给秋谷说:“这是上海来的电报。”秋谷听了,心上就觉得一惊。接过电报来看时,见封面上果然写着自上海发的,暗想这一定是家里头的电报,不知道有什么紧要事情,难道是太夫人有什么病痛不成?想到这里,不由得满心乱跳,连那只拿着电报的手都颤动起来,呆呆的看着那封电报,竟不敢去拆封。定了一定神,只得硬着头皮拆开那封电报来。看时,只见写得明明白白的几个字儿道:“其盛倒,母病,速回。”秋谷见了这“母病”的两个字儿,好象兜头浇了一瓢冷水一般,心上“扑扑”的跳个不住。连忙叫家人收拾了李立刻搬出城去,上了轮船,回到上海。

这边章秋谷的事儿且自按过一边,只说上海地方,一年一年的时势变迁,人事代谢,市面一天衰败似一天,堂子里的生意也一天寥落似一天。就是那班堂子里头的有名人物,到了这个时候,老的老了,嫁的嫁了,死的死了,繁华一瞬,歌舞无常,飘零金谷之花,摇落章台之柳。那班曲院中的老辈人物,除了胡宝玉之外,还有什么前四金刚、中四金刚、后四金刚的名目。前四金刚是陆兰芬、金小宝等四个,中四金刚是左翠玉、秦薇云等四个,后四金刚是张扬、王宝宝等四个,都是那一班小报馆里头的主笔提倡出来的。又有什么蕊珠仙榜、十二花神等种种色色的许多名目,在下做书的一时也实在写他不尽。但是以前那班报馆的开花榜,虽然未免有些阿私所好的弊病,却究竟还有几分公道。即如南亭亭长选拔花榜状元,有了色艺,还要考证他的资格;有了资格,还要察看他的品行;直要色艺、资格、品行件件当行,桩桩出色,方可以把他置诸榜首,独冠群芳。所以那个时候的花榜状元,倒着着实实的有些声价。

到了后来,就渐渐的闹得大不是起来。那一班没有廉耻的小报主笔,本来是穷得淌屎,囊无一钱的。当了个小报主笔,薪水不过一二十块钱,至多的也不过三十块钱,那里够他们的挥霍?到了那穷到无可如何之际,便异想天开的开起花榜来,拣那有了几个钱的倌人,叫个旁人去和他打话,情愿把他拔作状元,只要他三百块钱或者二百块钱。那状元以下的探花、榜眼、传胪等,名次来得低些,价目也来得贱些。渐渐的递减下去,甚而至于十块五块钱的贿赂都收下来,胡乱给他取个二甲的进士,或者三甲的进士。看官请想,我们中国的科举毒是人人最深的,古今来多少的英雄豪杰都跳不出这个圈子去,情愿拼着毕生的心血,去博这个无谓的科名。

何况这班倌人,都是些不读书、不明理的女子,那里打得破这个关头,翻得过这个筋斗?听得只要花几个钱,就可以把他取作状元、榜眼,况且又都知道自己的名字登了花榜,名气自然要来得响些,生意自然也来得大些,这花掉的几个钱不算什么,将来可以收得回来的。只要这般一想,自然大家都情情愿愿、伏伏贴贴的拿出钱来。

到了发榜以后,那些报馆里头的人又格外想出个生财的法子。略略的花几个本钱,去漆匠铺子里头做了几块状元、榜眼、探花、传胪的匾额,上面插了金花,雇几个人抬了匾额,带着红缨大帽,雇了一班吹手,携带着许多鞭炮,一窝蜂的都赶到那新贵人院中去报喜讨赏,多的一百块钱、五十块钱的都有,至少的也要二三十块钱。就是那班三甲里头的进士公,也要叫一个人带着那一张花榜沿门分送,放上一串短短的鞭炮,讨起赏来也要一两块钱,也有三块五块的。又有什么赏元贺魁的筵宴,那前十名的新贵人,每家都要整治一桌盛席,延请这位主笔先生、花榜总裁赴宴,好象那京城里头的黄榜团拜、白榜团拜一般。这位主笔先生免不得也要呼朋引类的大嚼一番,吃完了抹抹嘴就走,连下脚的都是倌人自己出的。这种种无耻的举动,在下做书的一时间也说不尽他许多。

看官,请想这个评选花榜的事情,闹到这样不可收拾的一个田地,那花榜上的人还有什么声价?非但不论品行,不拘资格,连色艺都是随随便便的了。头面还没有长得平正,便说他是有一无二的国色天香;曲子还没有唱得周全,便说他是当世无双的仙音法曲。只要有钱的倌人,便无盐、嫫母也是佳人;那些没有钱的倌人,便西子、南威也是丑鬼。那班极小极穷的报馆,每每穷到山穷水尽支持不来的时候,便开起花榜来,借此做个救急疗贫的妙策。开一次花榜,就是最少也有几百块钱。

到得后来,竟有一家报馆半年之内连开四五次花榜的,开了色榜又开艺榜,开了艺榜又开叶榜,闹得个一塌糊涂。就是那些堂子里头,如今的风气也一天坏似一天,比起那十年以前的光景来真有天渊之隔。这些说话,且把他暂时按过一边,慢慢的再和列位看官细说。如今在下做书的,且讲一件嫖界中间的故事出来给列位看官们大家听听。正是:

宛转三生之誓,名士倾心;缠绵一晌之情,佳人难得。

不知以后如何,下回交代。





第一百八十四回 挥别泪红杏嫁东风 讶奇遇仙云吐华月





且说辛修甫自从做了龙蟾珠以后,前后整整的五年,虽然也做几个别的倌人,却都是没有交情的。惟有龙蟾珠和辛修甫性情相合,嗜好相投,做了五年彼此没有口角过一句。龙蟾珠狠想叫辛修甫娶他回去,辛修甫也狠想娶他。无奈辛修甫的那位太太,虽然有些才貌,却抵死的吃醋,不许辛修甫娶妾。辛修甫恪遵阃令,不敢擅违。龙蟾珠也知道辛修甫有些惧内,只好把这件事儿阁起不提。

这一天辛修甫在西安坊龙蟾珠那边请客,龙蟾珠淡淡的不甚应酬,比平日的样儿大不相同。辛修甫觉得十分诧异,暗想蟾珠向来不是这个样儿,一定有什么道理在里头。到得客人散了再问他不迟。一会儿酒阑人散,漏永宵深,龙蟾珠一把拉着辛修甫的手道:“耐今朝呒拨啥事体末,勿要去哉,倪有闲话要搭耐说。”辛修甫正要问他今天为什么这般模样,便也点头答应,坐着不走。

龙蟾珠拉着辛修甫坐在炕上,自己紧紧的挨着他身旁坐下,搀着他的手悄悄的问道:“辛老,耐一径搭倪说唔笃太太凶煞,勿许耐讨小老姆,到底阿有介事?”

辛修甫听了叹一口气道:“自然是真的,你看我几时向你说过假话的?”龙蟾珠听了也叹一口气道:“格末倪两家头格事体到底那哼?倪一径做仔耐五年下来,勿曾说过歇一句。弄到仔故歇,再弄也弄勿过哉。实梗洛倪要问问耐,耐格心浪到底是那哼格意思?”辛修甫听了,皱着眉头道:“你的意思我自然知道的,我心上也狠愿意娶你回去。无奈这件事儿委实的办不到,你叫我怎样呢?如若不然,我早已把你娶回家去的了,那里还要等到今日?”龙蟾珠嘿然了一会,看着辛修甫一言不发,含着一泡水汪汪的眼泪,秋波溶溶的几乎要流出来。辛修甫看了心上早已有了几分明白,便也对着龙蟾珠细细的看。

龙蟾珠和辛修甫对看了一回,慨然说道:“倪有一句闲话要搭耐说,耐听仔勿要动气,倪也叫呒说法。”辛修甫道:“你只顾说就是了,岂有动气的道理?”龙蟾珠又长叹道:“做个仔人,总规随便啥事体一塌刮仔勿称心,格末叫苦恼!”辛修甫接口说道:“这世上的烦恼,是不论什么人都不能免的,何况是我们两个人?

你有什么事情,只顾和我说就是了。想起来大约还是我没有福气,消受不起你这样的一个人。“

龙蟾珠听了,呆了一回方才说道:“格件事体,说起来倪也真真叫呒说法。”

说着,便把有个姓葛的客人也是个江苏候补道,要出三千银子娶他回去的事情,和辛修甫说了一遍。又道:“倪吃仔格碗把势饭,总规呒拨结果格。趁仔勒浪年纪轻格辰光,拣格好好里靠得住格客人,嫁拨停俚,总算完结仔一生一世格事体。倪搭耐两家头一径倒蛮要好,刚刚唔笃太太来得笋,看上去总规是格勿成功。就是实梗弄来弄去,弄到仔故歇已经五年哉。再要弄下去,年纪大仔,再有啥人来要倪?实梗洛倪今朝要搭耐商量。耐格心浪到底是那哼格意思,倪横竖总归听耐格闲话。耐说那哼,倪依仔耐那哼。耐就是格个辰光办勿到,耐只要说定仔一句闲话,倪慢慢里等来浪,也呒啥要紧。耐只要说一句好哉。”

辛修甫听了,沉吟一会,也紧紧的拉着龙蟾珠的手,对他说道:“依我的心上看起来,你既然有人要娶你回去,这个机会狠好,你只顾答应他就是了。如今上海地方,靠得住的客人狠少。这个姓葛的客人想来是狠靠得住的,错过了这样的客人,一时那里再去找第二个?至于我们两个人的交情,自然原是狠好的。但是我们家里那一位实在来得累赘,不是个好惹的人。我若要不由分说的把你娶了回去,将来一定要闹得一个天翻地覆,海沸江号。到了那个时候,你怎样的闹得过他?我又怎样对得住你的?所以我想起来,如今既是有人娶你,自然赶快答应他的为是。在我们两个人这一面看起来,自然有些割舍不得。但这是你一生一世的大事,我自己既然不能娶你回去,怎么好把我自己心上的私见耽误你一世的事情?你说我这个话儿可是不是?”

龙蟾珠听了把头点了一点,不由得心上一阵心酸,望着辛修甫扑簌簌的流下两行珠泪。辛修甫到了这个时候也有些熬忍不住,几乎要流下泪来,只得携着手,殷殷勤勤的劝慰一番。这一夜,辛修甫自然是不回去的了。笑啼并作,悲喜交并。结万斛之愁肠,春心宛转;倒一腔之别绪,玉箸纵横。烛影摇红,钗光照夜。匆匆别去,羌有恨以无言;缓缓归来,欲双栖而未得。

过了两天,辛修甫知道龙蟾珠的嫁期已在十日之内,连牌子都除了下来。辛修甫觉得以后不便再去,便在自己手上脱下一只金刚钻戒指来,套在龙蟾珠手上,口中说道:“我们两个人,从此以后是不能再叙的了。但愿你嫁了过去,白头偕老,琴瑟和谐。”说到这里,喉咙竟咽住了,说不出来。龙蟾珠泪流满面,哭得两个眼睛都肿了起来,拉着辛修甫的衣服,好似生离死别的一般不肯放手。要说什么又说不出来,呜呜咽咽的把一个小小的绢包递给辛修甫道:“格点物事耐带得去,总算是倪格记念。”说了这几句,不由得眼中珠泪好似雨点一般的落下来。辛修甫这一回儿那心上的难过竟是从来没有经过的,再也忍不住,眼中也流下泪来。接了龙蟾珠手内的绢包,那眼泪竟斑斑点点的把绢包湿了好几处。几个娘姨、大姐在旁看了他们这样的依依不舍,也觉得大家有些心酸起来。龙蟾珠哽咽了一回,方才竭力挣出几句话来道:“耐去罢,自家保重点身体,勿要妈妈虎虎,阿晓得?倪是真正叫呒说法。”龙蟾珠说到这里,就咽住了说不下去,掩着脸把手向辛修甫摇了几摇,便去倒在一张美人榻上吞声暗哭。辛修甫也知道久留无益,只得也硬着心肠,走了出去。

一直回到自己公馆里头,瞒着他那位夫人,把龙蟾珠给他的绢包拆开看时,只见一支漆黑的头发;一个绉纱兜肚;一双玄色缎绣白花平底弓鞋,尖尖瘦瘦的,只好四寸光景,鞋底上面只有微微的一些儿泥污,还有七八分新。辛修甫见了,明知道龙蟾珠的心事,给他这几件东西,是好象天天和他并头贴体的意思。看了这几件东西,更觉得魂销心动起来。过了好几天,心上还觉闷闷不乐。一个人独坐嗟呀,书空咄咄,心中目中都是惦记着一个龙蟾珠,觉得龙蟾珠的声音笑貌,一天到晚只在辛修甫心中间,上下左右的周旋来往,一时那里抛撇得下!直过了一月有余,方才把这个龙蟾珠的事情放了下来。

辛修甫的性情本来最爱听戏,每到心上不高兴的时候,便去听戏消遣。如今这个时候,一个最要好的倌人龙蟾珠是嫁了人了,还有那几个知己些的朋友,如章秋谷、王小屏等那班人,守制的守制,出山的出山,止有一个陈海秋还在上海。辛修甫觉得心上有些懊恼,便去寻着陈海秋,同到戏馆去听戏。

这一天,辛修甫正同着陈海秋到丹桂去听戏。这个时候,正是夏月润等弟兄几个初到丹桂的时候,生意十分热闹,上下都挤得满满的。辛修甫见楼下正桌的人太多,便同着陈海秋到包厢里面拣了两个座位坐下。看了一回夏月润的《花蝴蝶》,登场一出后台,大家便齐齐的喝一声采。辛修甫举目看时,只见那夏月润立在当台,打扮得衣服甚是鲜明,结束得身材十分伶俐,雄赳赳、气昂昂的,倒也狠有些儿英雄气概。一会儿上起杠来,手脚甚是活溜,把两只手臂牢牢的圈住了台上的铁杆,一个身体好似风车儿的一般,在杠子上旋转起来。大家看了,又不觉齐齐喝采。

辛修甫是坐在头包里面的,刚刚抬起头来,往对面包厢里头一看,只见一个少年丽人,生得容华艳冶,态度娇娆,黛色浮香,珠光聚彩。这个时候,正是十月天气,这个丽人穿著一件铁青色珠皮袄,下面穿的什么裙裤,却隔着栏槛看不出来。

头上带着许多珠翠,把那一对秋波刺斜里向着对面溜来,恰恰和辛修甫打了一个照面。辛修甫见了不觉呆了一呆,暗想这个人真来得有些诧怪,怎么平空的和我吊起膀子来?一面想着,便也对着那丽人飞了一眼,微微的把头动了一动。只见那丽人着实的把自己钉了一眼,便低下头去,略略的呆了一会;顿时抬起头来,眉欢眼笑,卖弄风情,一连对着辛修甫使了几个眼色,又远远的对辛修甫把头点了一点;回过头来,对一个大姐附耳说了几句。正是:

肠断京华之路,崔护重来;魂销春水之波,桃花无恙。

要知后事如何,且待下文交代。





第一百八十五回 辛修甫良宵逢旧识 汤娟娘薄命堕风尘





且说辛修甫眼睁睁的看着那对面三包里面的丽人,心上狠觉得有些诧异,暗想:“我这个人是向来不用膀子工夫的,怎么他竟会看中了我?”心上想着,只见那丽人叫过一个十八九岁大姐来,附耳说了几句不知是什么话儿,又指指点点的对着辛修甫指了一会。一会儿的工夫,早见那大姐从人丛里面挤上楼来,带着银水烟筒直走到辛修甫背后,笑迷迷的对着修甫说道:“格位阿是辛老?倪先生说,请耐到倪搭去。”辛修甫倒吃了一惊,道:“你怎么认得我姓辛?你们先生是什么人?”

那大姐笑道:“倪先生叫苏青青,来浪三马路美仁里,说搭辛老一径认得格。等歇点定规要请过去格嘘!”辛修甫听了,想了一回,始终想不出这个苏青青是什么人,把眼光拢了一拢,再往对面看时,觉得这个人虽然有些面熟,一时间那里想得出来?

问那大姐时,那大姐也说不明白,只说是新来的,弄不清楚。辛修甫也只得点一点头,预备着等会儿到美仁里再去细细的问他。那大姐装了几筒水烟,便也去了。

辛修甫看着那大姐挨挨挤挤的回到对楼,和苏青青说了几句,苏青青抬起头来,远远的对辛修甫一笑。辛修甫见了,便也对着他微微一笑。这个时候,场上正在做着七盏灯的《烈女传》。这七盏灯本来是个有名角色,唱做俱佳,声容并茂。台下的许多看客,都目不转睛的看着台上的七盏灯。只有苏青青的两只眼睛,只顾目不转睛的看着辛修甫,一顾一盼,便有许多送意推情的诚愫流露出来。辛修甫虽然是个老于上海的人,不是什么色中饿鬼;但是世上的男子,断没有见了个少年美貌的女子在那里和自己吊膀子,倒反要拒绝不纳的道理。况且辛修甫自从龙蟾珠嫁人以后,怀着一腔的情愫,含着满腹的牢骚,一时又找不出个替代龙蟾珠的人。如今见了苏青青,一见倾心,三生慧果。目成眉语,托诚愫于微波;拨云撩雨,隔星娥于银浦。芳悰叩叩,密意沉沉,未谐风卜之欢,先有鸳盟之订。这一段情事,却是辛修甫意想不到的,自然觉得心上十分高兴。两下里遥遥的对着,眉来眼去,卖弄风情,连台上做的什么戏也都糊里胡涂的不知道,只觉得你的心上只有一个我,我的心上只有一个你。一片的爱河浩瀚,无边的情海汪洋,都在这两个人的眼中滚来滚去,把个身体都深深的埋在里面,再也跳不出来。

辛修甫只顾呆呆的望着苏青青呆看,陈海秋和他说话都不听见。陈海秋见辛修甫这般模样,便把他拉了一把道:“你吊膀子只管吊膀子,为什么要吊得失魂落魄的这般模样?”辛修甫被他拉了一把,猛然吃了一惊。回过头来,慌慌张张的问道:“什么,什么?你有什么话说?”陈海秋不觉狂笑道:“你这个人向来常常的说见色不迷的,怎么今天见了一个苏青青,就把你迷到这般模样?”辛修甫听了,不由的也有些不好意思起来。待要分辩,却又分辩不出什么,只得也对着陈海秋哈哈一笑。

陈海秋还待再说,辛修甫忽见对面苏青青立起身来,对着他做了一个手势,回过身来便走了。辛修甫见了,知道他要走,便也拉着陈海秋一同走下来,直到戏园门口。等了一刻的工夫,方才见苏青青扶着方才的那个大姐姗姗的走来。见了辛修甫,便自樱唇半启,笑靥微开,喜孜孜的叫了一声:“辛老。”辛修甫正待问时,苏青青对他摇手道:“故歇慢慢交,有啥闲话,到倪搭去慢慢里说末哉。”辛修甫点一点头。

早见两个车夫拉过一辆簇新的橡皮包车来,车前点有两盏药水灯,精光夺目。

苏青青便对着辛修甫嫣然笑道:“辛老,耐坐仔倪格车子先去阿好?”辛修甫摇头道:“我们都有包车,你只顾先走,我们慢慢的来就是了。”说着,辛修甫和陈海秋两个人的车夫,也拉着车子过来。苏青青不肯先去,定要看着辛修甫和陈海秋坐上了车子,自己方才也坐着包车跟在后面。三辆包车飞一般的竟奔美仁里来。

那消一刻工夫,早已到美仁里,弯进弄去,到苏青青门口停下。苏青青同着辛修甫和陈海秋两个人走上楼来,到房间里头坐下。苏青青先问了陈海秋的姓名,方才笑容可掬的对着辛修甫说道:“辛老,耐想想看,到底阿认得倪?”辛修甫想了好一回,还是一个想不出,只得对着苏青青摇头道:“看着你的样儿觉得好生面熟,一时委实想不出来。”苏青青微微的笑道:“辛老,耐阿记得,格辰光有个阿娟,住来浪唔笃隔壁?阿是贵人多忘事,忘记脱格哉?”辛修甫听了,不由得心中一动道:“原来你就是阿娟!怎么忽然会落到堂子里头来?你的父亲和哥哥到那里去了?”

苏青青叹一口气道:“说起倪格闲话来,格末真正叫坍台。”说着,便把他父亲病故,他哥哥嫖赌吃烟,不务正业,把他卖人烟花的事情,细细和辛修甫说了一遍,不觉溶溶欲涕。辛修甫听了,更觉得替他难过,嗟叹不已。眼看着这样的一个旧家的女儿陷入烟花圈套,心中老大的不忍,便存着个要把他提出火坑的念头。

看官,你道这个阿娟是个什么人?他的父亲、哥哥又是个什么人?原来辛修甫年幼的时候,是住在上海城内新北门里面的。那个时候,辛修甫的隔壁住下一家人家,姓汤,官名一个澄字,却是个江苏候补巡检,分道差遣的当了几年的保甲局委员。不知怎样的被他上下其手,倒狠弄了几个钱。这位汤巡检的太太已经死了多年,止有一子一女。女儿的小名就叫阿娟。到了十多岁,却出落得一表人才,十分俊俏,汤巡检甚是溺爱这个女儿。止有这个儿子,天生的性格甚是惫赖,打街骂巷,无事生风,没有一个人不恨他。这个阿娟,却和他哥哥大不相同,天生的口角灵变,最会哄人,就是那左邻右舍的人家,见了阿娟,也没有一个不欢喜的。辛修甫那个时候,正是十八九岁,阿娟却止有十岁,时常到辛修甫家里去顽耐。辛修甫见了这样粉装玉琢的女孩子,虽然和他没有什么情愫,心上却狠赏识他。后来不多两年,辛修甫家搬到城外来住,便从此和他家音信不通。

不想这位汤巡检患病不起,一命呜呼。他那位儿子,平日之间见他父亲捏住了家财,一个大钱都不肯给他,早已恨得咬牙切齿。如今见父亲死了,心上十分欢喜,眼泪都没有一滴,只一天到晚到赌场里头去赌钱。这个“赌”字没有底止的,一晚的工夫输了几千几万都不算什么。汤巡检虽有几个刮地皮钱,究竟是个小官,就有钱也有限得狠。不消两个月,早把这一分薄薄家私,输得一个干干净净,寸草不留。

渐渐的典卖衣饰、典卖器具起来。衣饰和器具都典卖尽了,便想到妹子身上来,把他典了六百块钱,典在堂子里头。可怜这个阿娟还止得十四岁,晓得什么事情?听得哥哥把他典在堂子里头,一时也无可如何,只得依着老鸨,做起生意来。生生的把一个宦家小姐,落在把势里头去了。

苏青青做了几年生意,倒是枇杷花下,车马如云,生意十分热闹。过了几年,便自己赎了身出来,一连做了三年生意,虽然不差,无奈苏青青的用度开销二十四分的浩大,狠有些儿支持不来。勉勉强强的移东补西,过了两节,实实的过不去。

今天刚刚在丹桂看戏,遇着了辛修甫,觉得十分面熟。低着头想了一想,记得好象辛家大少爷的模样。他素来知道辛修甫家狠有几个钱,虽然算不得上海地方的首富,却也是个数一数二的有名富家。不管是他不是他,姑且叫大姐过去撞个木钟再说。

那知这个木钟用不着两撞三撞,只消一撞便撞着了。当下辛修甫听了苏青青的一番说话,心上倒着实的有些替他伤感。看着那苏青青宛转娇啼,水绡泪渍,更觉得楚楚可怜,免不得要温温存存的安慰一番。

陈海秋坐在旁边,呆呆的听了一回,觉得时候不早,便取出表来一看,已经四点多些,便立起来对着辛修甫笑道:“你们慢慢的谈心罢,我却不能奉陪,要先回去了。”辛修甫听了,还没有开口,早见苏青青抢步过来,在辛修甫耳边说了几句。

辛修甫点了一点头,苏青青便走过来,对着陈海秋说道:“陈老慢慢交,坐歇末哉,勿要去,辛老要请耐吃酒呀!耐吃过仔酒,就来浪倪搭借仔格干铺末哉。”陈海秋听了,先向辛修甫看了一看,又向苏青青看了一看,便笑嘻嘻的对着苏青青打个手势道:“恭喜,恭喜!”只把个苏青青羞得别转头去,一言不发。

一会儿,娘姨们调开桌椅,排上一个双台。陈海秋道:“一个双台,只有我们两个人吃,未免太寂寞了些。”辛修甫道:“这个时候到那里去请客?只好把青青这里的娘姨、大姐,一古脑儿都叫来坐在席上,胡乱吃上一顿也就算了。”陈海秋忽然长叹一声道:“如今我们这班朋友,也都一个个风流云散,只有我们两个人还在上海。”修甫听了,也不由得叹了一口气。正是:

后夜之相思何处,月殿云廊;当年之丰度依然,飘烟抱雨。

要知究竟如何,且待下回再行交代。





第一百八十六回 证前因深情结遥誓 出奇计险语试倾城





且说辛修甫在苏青青院中吃了一个双台,自然苏青青不肯放他回去的了。镜盟衫誓,倚影偎声,春浮银汉之槎,水泛桃源之洞;子夫散发,合德横陈,红添两颊之云,绿展双眉之黛。

辛修甫直到明天十一点钟方才起身,见房门虚掩,静悄悄的没有一个人,便走到对面房间里头,去看那借干铺的陈海秋时,见陈海秋一个人睡在床上,还在那里呼呼的打鼾,没有睡醒。辛修甫把他推了一推,陈海秋方才坐起身来,把眼睛揉了一揉,见了辛修甫,口中还含含糊糊的道:“时候还早得狠,你怎么倒先起来?”

辛修甫笑道:“这个时候已经差不多十二点钟,还不起来做什么?想睡在这里过一世么?”陈海秋听了一谷碌跳下床来,定醒了一回,方才同着辛修甫走到对面屋内。

苏青青早已起来,坐在窗前梳洗。陈海秋走到苏青青面前,仔仔细细的把他看了一会。又回转身来,把辛修甫仔仔细细的看了一会。辛修甫笑道:“你这样看法,为的什么事儿?”陈海秋笑道:“我昨天晚上,和你们两个人预算决算了一回,大约无论怎么样,总要睡到下午一两点钟起来。如今你十一点钟就居然起来,不是临阵脱逃,就是事前躲避,我所以要看看你们两个人的脸儿。”辛修甫笑道:“想来是你临阵脱逃惯的,所以要这般平空的替人着急。”陈海秋又向苏青青笑道:“你和我直说,他究竟临阵脱逃没有?”苏青青红着脸道:“耐格闲话,倪一塌刮仔才勿懂。”陈海秋哈哈笑道:“你真个不懂,待我来教你何如?”苏青青听了把头一扭,不去理他。

辛修甫走过来,一把把陈海秋拉了过去,口中说道:“你这个人,成天的专讲和人取笑,取笑得的也要取笑,取笑不得的也要取笑,这像个什么样儿?”陈海秋把手抹着自己的脸羞他道:“阿呀!显见得你们两个人是恩相好,所以要这般回护。”

辛修甫笑道:“算了罢,不用说了。”苏青青听了,也侧过头来,把陈海秋看了一眼,便向辛修甫说道:“辰光勿早哉,唔笃两家头吃仔饭去阿好?”修甫听了便也点头应允,坐了下来。苏青青梳好了头,陪着他们两个人吃了午饭,辛修甫方才同着陈海秋去了。

自此以后,辛修甫和苏青青两个人的交情打得火一般热,真是个鹣盟蝶誓,密爱幽欢。苏青青拿出乎生的手段来,窝着辛修甫,竟不接别的客人。辛修甫也想着法儿,试过了苏青青几次,却试不出什么破绽来,辛修甫心上自然欢喜。

恰恰的事有凑巧,辛修甫的那位夫人,本来原是个专会泼醋的人物,不知怎样的得了一个吐血的症候,延医服药,一些儿效验都没有,不上两个月的工夫便呜呼哀哉死了。只把个辛修甫闪得个风折鸳分,形单影只。沧海巫山之恨,无地招魂;金钗沽酒之诗,心伤旧配。免不得要着实的伤感一番。过了几时,渐渐的把伤感的意思丢掉了些,却又兜的把龙蟾珠的事情提了起来,暗想:“若是这件事儿出得早了些儿,龙蟾珠也不至于给别人娶去。如今是事已成事,木已成舟,无可奈何的了。”

真个是佳人已属沙咤利,义士今无古押衙,未免的心上狠有些儿惆怅。想了一回,忽然转一个念头道:“如今幸而还有个苏青青在这里,虽然我和他相知不久,却是看他的样儿和我二十四分的要好,不如竟把他娶了回去,料想还不至于怎样的不妥当。况且我以前曾经历试过他几次,试不出什么破绽,一定靠得住的。”

想到这里,忽然又是心中一动道:“上海倌人岂是可以娶回家去的?我平日之间看着苏青青的样儿,虽然一心待我要好,没有什么三心二意,但是冷眼看他的起居服用,却又奢侈放荡,不像是个肯做人家人的。俗语说得好:画虎画皮难画骨,知人知面不知心。知道他将来嫁人之后,究竟怎么的一个样儿?不如还是趁着这个时候,再把他试上一试,试出他的真心来再说别的。”

辛修甫定了主意,便和陈海秋等一班朋友,大家商议了一回,商议出一个法儿来,要想趁着个空儿试验他的真假。恰恰的这个时候,苏青青知道辛修甫夫人死了,便越发的使出浑身手段,全付工夫,把个修甫哄得脑筋里面有些迷迷糊糊的起来,撒娇撒痴的只要辛修甫娶他回去。辛修甫虽然被苏青青哄得十分高兴,却毕竟心上有些把握,见了苏青青这般模样,便想着要把这个试验的法儿施展出来。

这一天,走到苏青青房间里头坐下,不住的咳嗽叹气,神色张皇,坐在那里一言不发。苏青青见了心上诧异,便问着辛修甫道:“辛老,耐今朝啥格事体实梗样式,阿是身体浪有点勿舒齐?”辛修甫听了只是摇头,一句话儿都说不出。苏青青一连问了两三遍,辛修甫只是不开口。苏青青问得着起急来,走过来把辛修甫的耳朵一把拉住,口中说道:“耐格人啥实梗呀?好好里问耐闲话,啥格一声勿响,阿是变仔哑子哉?”辛修甫皱着眉头对苏青青道:“我的事情弄糟了,你知道不知道?”

苏青青吃了一惊道:“耐啥格事体弄坏哉呀?阿好搭倪讲讲呀?”辛修甫道:“说起这件事情,真叫作一言难尽。就是和你说了,也没有什么用处,还是不和你讲的好。”苏青青听了更加着急道:“耐格人总规是实梗阴阳怪气,豪燥点搭倪说嘘!”

辛修甫听了便故意装着一派愁容,瞎七瞎八装装点点的和苏青青说了一遍,只说:“自己前两年有一封信写给朋友,这封信上的话儿是得罪皇太后的。如今不知怎样的,这封信给一个仇人拿了去,在京城里头告发起来。幸而有个要好的朋友暗地里通了一个信给我,叫我快走,不消几日,京城里头就有电报出来,着落地方官要拿我。我若是不走,万一个给地方拿住了送进京去,就是熬得一条性命出来,最轻也要问一个烟瘴充军的罪名。如今我也没有别的法儿,只得把家产托人照管,自己逃到日本暂时躲避。所以没奈,只得来和你说一声儿,我们两个人以前的话儿,我如今自己的生死还不可知,怎好平空的把你拖下水去?以前的那些嫁娶的话儿,如今一古脑儿都一笔抹过,只当没有这句话的一般。我就在这几天之内,就要动身到东洋去,你的事情委实不能兼顾的了。但愿你摞梅迨吉,燕尔新欢,好好的拣一个人,不要和我一般的有始无终,辜负了你的一番好意。”说着,把眼睛挤了一挤,挤得眼皮儿红影影的,好象要哭出来。

苏青青听了辛修甫的说话,起先倒也呆了一呆,顿时的花容失色。直听得辛修甫这一番说话说完了,不由得低下头去,沉吟一会。忽然抬起头来,对着辛修甫把头摇了一摇道:“耐格闲话定规是假格,倪实头勿相信。为啥别人家呒拨格号事体,独独到仔耐身浪,就有几化希奇古怪格事体出来?耐阿是来浪骗小干仵?”辛修甫听了,故意顿足道:“这是千真万确的事情,我怎么肯来骗你?别的事情或者和你取笑,哄哄你也还罢了。那有这样的风火事儿都和你取笑的道理?难道我们两个人这样的要好,你还信不过我的说话么?”苏青青见辛修甫说得这样活龙活现,不由的也有几分相信起来。

踌躇了一会,慢慢的走过来,扶着辛修甫的肩膀,低下头去和辛修甫脸贴脸的偎了一偎,口中说道:“辛老,听耐实梗说起来,到底阿是真格呀?”辛修甫连连顿足道:“我心上这般着急,你还在这里慢条斯理的这般模样。你想我为什么要哄你?就是哄信了你,在我身上有什么好处?”苏青青听到这里,心上有些鹘鹘突突的起来,便对辛修甫说道:“辛老,格末阿要紧格呀?”辛修甫把舌头一吐道:“你说的真是风凉话儿,还问要紧不要紧。若是当真的给他们拿进京去,非但人亡家破,连这脑袋保得住保不住都是不可知的事情。若果然到了那个时候,你也不必感伤纪念,只要你心上记着我这样一个人就是了。”

辛修甫一面说着,不觉流下泪来。苏青青也泪珠莹莹的握着辛修甫的手道:“辛老,格末那哼介?”辛修甫皱着眉头道:“如今只要早些逃走,料想也闹不出什么别的事情。但是从此以后,我姓辛的在中国地界之内就算个犯法的罪人,若不遇赦典,是一生一世不得回来的了。我心上原觉得狠有些割舍不得你,却又无可如何。想来你也知道我的苦衷,这是出于意外的事情,没奈何只得要劳燕分飞的了。”

苏青青听了这番说话,不觉双蛾敛恨,宝靥含嗔,似嗔似喜的瞅了辛修甫一眼道:“耐倒说得实梗容易,倪勿成功格。格个嫁人格事体,勿是好搂白相格。阿有啥一塌刮仔说得明明白白,故歇倒说勿成功?拨别人家晓得仔,阿要难为情?倪故歇只有一句闲话搭耐说,随便耐那哼,倪总归是耐格人,今生今世,除脱仔耐姓辛格,要倪去再嫁第二个人客人,格末老老实实办勿到。故歇耐末拍拍身体东洋去哉,留仔倪一干仔来浪上海,耐打算那哼?”

辛修甫听了,想了一回道:“这个时候,那里想得出什么安置你的法儿?要便立刻收了牌子,同着我一同到日本去。但是我细想起来,你们当倌人的好容易嫁一个人,不指望他什么好处也就是了;如今嫁了人,倒反把你们带下水去,我辛修甫天良不昧,怎样的心上过得去?”苏青青听了,接着说道:“倪搭耐自家人,格号客气闲话,故歇用勿着。总归倪既然嫁仔耐,就是耐格人。耐到洛里,倪跟到洛里,呒拨啥第二句闲话。

修甫听了,走过来对苏青青打了一拱道:“我倒想不到你有这般的志气,可敬得狠!既然你自家愿意跟着我走,我也自然不能拦你。但是还有一句话儿要预先和你说明,如今你跟着我,我还是有钱的时候,你还没有什么;万一个到了将来,我的家产保守不住,到了没有钱的时候,你那里过得惯这样的苦日子?”苏青青把头一扭道:“耐格闲话笑话哉!倪既然跟耐,总归要苦末大家一淘苦,要甜末大家一淘甜,呒啥过得惯过勿惯。”正是:

回黄转绿,人生之祸福无常;地老天荒,金石之深盟未改。

要知后事如何,且待下文交代。





第一百八十七回 甘同梦永夜听鸡声 困洪波长堤成漏泽





且说辛修甫对着苏青青造了一番谎话,只说他一定要自家反悔,不肯嫁他。那里知道这个苏青青竟是斩钉截铁的一口咬定,情愿跟着他到日本去。这一喜非同小可,暗想这个苏青青居然能够始终不变,立志不渝,在上海倌人里面总算是难得的了,便想要把这件事儿的来历和他说明。忽然又想道我何不再着着实实的逼他一下,也好试试他的真心究竟怎样。

想着,便又道:“既然你肯同甘共苦,这是我求之不得的。但是事不宜迟,我明天便要动身,万一个被他们拿住了,走不脱身,倒不是顽的。你既要跟着我一同走,这个时候就要和本家娘姨等讲个明白,把牌子除了下来,还清了他们的帐目,好预备一同上路。只不知道你来得及来不及?”苏青青听了,略略的顿了一顿,便慨然说道:“倪是说走就走,有啥格来勿及呀?只要叫仔本家进来,搭俚说声探脱仔牌子好哉。”说着,便叫了大姐阿金进来,叫他去叫本家老鸨。原来辛修甫的这番说话,本来是咬着耳朵说的,那班房间里人,见了他们这般模样,便故意都躲出去,好凭着他们两个人密密切切的谈心,所以这件事儿说了半天,那班娘姨、大姐还大家都不知道。如今听得苏青青叫阿金去叫本家老鸨,阿金答应一声,便当真要走下楼去。辛修甫连忙把阿金叫回来,口中说道:“你慢慢的去叫他,不要性急。”

苏青青司道:“早点去叫仔俚上来,搭俚算清仔帐末拉倒哉呀,为啥耐咦要叫俚慢慢交?”辛修甫对着苏青青哈哈一笑,又对着他打一个拱道:“我如今和你实说,你不要见怪。”苏青青是何等伶俐的人儿,更兼以前被辛修甫试过几次,如今见了辛修甫朝着他哈哈一笑,心上早已明白,便把辛修甫推了一推道:“耐格人末,实头少有出见格,总归瞎三话四,呒拨一句真闲话。耐自家想想看,阿该应勿该应?

前两转格事体,还说是搭倪讲笑话,呒啥要紧。今朝是加二勿对哉,啥格皇太后也来哉,犯人也来哉,倒骗得倪蛮相信,阿要气数!“辛修甫听了又打一个拱道:”我原是有心试验你一下的,看你口中说得这般铁铮铮的,到底是真心不是真心。

若不是我这样的一来,也显不出你的真心实意。千万不要生气,我原是和你要好的意思。“

苏青青听了,瞪了辛修甫一眼道:“耐格个人真正就叫讨气!耐试仔一转勿算数,再要试第二第三转。区得倪格嫁人勿是假格,呒拨啥枝枝节节格事体。勿然是拨耐试仔出来,也好哉!”说罢,咬着牙齿用一个指头在辛修甫头额上用力点了一点,口中又说道:“唔笃格排男人,总归是翻转仔面孔就勿认得人。刚刚倪要拨耐试仔出来,故歇勿知要办倪那哼格罪名哉!”辛修甫笑着,拉着苏青青的手道:“这件事儿,总是我的不是,你千万不要生气。”苏青青故意把手一摔,洒脱了辛修甫的手,别过头去假装不去理他。辛修甫到了这个时候,这心上的高兴就中醍醐灌顶、醇醪醉心的一般,直觉得骨节奇痒,心花怒开。一时间在下做书的也形容不出他的喜欢来。见苏青青扭转身体不来理他,免不得要软软的央告安慰一番。

自此以后,辛修甫和苏青青平空的又添了几分恩爱,竟有些迷惑起来。一天到晚都在美仁里鬼混,连书局里头的事情都不去理会,只和苏青青商议着那临时嫁娶的典礼。依着苏青青,要辛修甫从此不娶正室,又要什么风冠霞帔、清音彩轿,要和娶正室的礼节一般。辛修甫虽然十分溺爱这个苏青青,不忍拂他的意思,却为着这几件事儿关系来得大了,不能轻轻易易的一口应允。自己心上忖度了一番,只许了他五年之内生了儿子,便不娶正室;如若五年不生儿子,别的再说。又许了他用清音彩轿和披风红裙。苏青青还故意作难,一定要用风冠霞帔。辛修甫一口咬定了不肯答应,只推说这是我们的家法,我就是答应了,也还有别人不答应,我一个人也做不来主。苏青青听了,知道再说也不中用,也只得罢了。两个人说得停停当当的,只等着万国救荒赛珍会举行过了,便要花开并蒂,月照三星;春融翡翠之巢,水荡鸳鸯之影。辛修甫到了这个时候,也没有工夫再管别的事情,只一味的屈着指头轮算那未来的日子,静静的等候佳期。幸而辛修甫也是个花丛老手,还不至于十分颠倒,和那淫魔色鬼一般。

看官且住,这个万国赛珍救灾会到底是怎么的一回事情?辛修甫要娶苏青青,和这个赛珍会是不相干的,为什么要等万国赛珍会举行之后方才迎娶?看官们不嫌烦碎,在下做书的少不得要一一的铺叙一番。

原来我们中国的江浙漕米,本来是由运河运到北通州交纳的。京城里头的食米,全是靠的南粮。所以那个时候政府特设漕运总督一缺,专管这漕运的事情。这个运河却是我们中国人工所成的第一大川,自浙江杭州府起,直贯穿江苏、山东两省,直至直隶通州为止,有二千五百多里长。自隋炀帝时兴工开挖,唐宋而后,直到元明,本朝也不知费了无数的金钱,用了许多的人力,方才成了这个运河。这运河的水势自浙江至江苏淮安、扬州一带,河运都十分利便。到了淮安清江浦以北,那河水便渐渐的干涸起来,一路都筑了许多水闸,随时开闭,节制运河的水量。遇着那水浅年分,粮船不能行走,便把第一闸的水放到第二闸来,等粮船差不多要走到第二闸的时候,却又把第二闸的水放到第三闸来。这样一闸一闸的过去,直要等得粮船过了水浅的地方,方才把末一闸的水又逐段的倒放过来。那运河水势最大的地方,就是淮安以南、扬州以北的一段。运河之西有高邮湖、邵伯湖、白马湖、宝应湖,运河之东有吴公湖、大枞湖、获金湖、广洋湖,水势都甚汪洋汹涌,也都有一百多里长、六七十里宽,都流人运河,和运河竟是通连的一般。更有安徽、江苏交界的洪泽湖,也是流人运河的。

看官,请想这般的许多湖泊都是流人运河的,把运河当作漩窝之地,众水所归,小小的运河能有多大的气魄?遇着天干水浅的时候,还不要紧。遇着个雨水过多、河水泛滥的时候,那里容纳得下?所以那个时候,漕运总督在运河东西两岸,筑起两条极高极坚的堤岸,在堤岸中间开一个节制水量的水门。每逢水浅的年分,便把水门开了,放进东、西两湖的水来。逢着水满的时候,便又开了水门,把运河的水放进东、西两湖去。借着这两条堤岸,做个运河的紧要机关。年年修造,岁岁兴工,也不知花费了许多帑项。淮、扬一带地方,也借着这个运河的力量,水旱不荒,年年的收成十足。

到了后来,河运改了海运,又省力又神速,并且还节省许多经费,政府便把漕运的事情永远改了海运,把漕运总督一缺也裁掉了。自此以后,这条运河便永远没有人来挑浚,这条堤岸便永远没有人来修整,由着他年深月久的淤塞坍塌,没有一个人来理会,直把这淮、扬一带的东西两岸渐渐坍塌得一个干干净净。那东、西湖的水,便一古脑儿都流人运河里面来,却没有了开闭机关,只有来路,没有去路。

一条运河里面安放不下,便都顺着下流一带灌注进来。那班淮、扬的百姓正眼巴巴的望得田禾成熟,大家高兴。那里知道被水一冲,都冲得个一物不留,一茎不剩。

今年如此,明年也是如此。一班百姓,还大家只说天公降饥荒,没有一个知道是运河年久失修,以致湖水顺势灌人的缘故。那淮、扬一带的居民,都是穷苦的多,富饶的少,那里禁得起这样的年年饥馑、岁岁凶荒?自然便都是流离转徙、奔走道路起来。一个个都是扶老携幼的望着镇江府、常州府、长江下流一带的地方来逃荒就食。常、镇两府的地方官,见这班饥民越来越多,到得后来连那淮、扬、徐、海三府一州的饥民,大家都逃避过来。地方官一时没有安置他们的地方,只得把地方上所有的寺庙都借给那班饥民居住。再到了后来,连寺庙也挤不下了,只得在城外拣一方大大的空地,胡乱搭些草棚,安顿那些饥民。那一种辗转沟壑的情形,琐尾流离的惨状,在下做书的一时间也描写不出来。那些地方官和那班本地的绅商,虽然也都募捐经费,设了几个粥厂,按日施粥,但是不能持久的。

这个时候,便有几个上海的巨绅大商出来发起劝捐了半个多月,虽然有些捐款下来,也是杯水车薪,无济于事。便又有一位慈善家想出一个救急的法儿来。这位慈善家姓孙,官名一个厚字,号伯义,是个山西候补道。从小的时候便在德国留学,却也算得如今世上一个熟谙洋务的人才。见了这班淮、海一带的饥民,一个个都在那里嗷嗷待哺,也未免有些蒿目伤心,暗想:“欧洲各国每每的举行什么慈善会,不论什么命妇贵女,都在会中执业,借着妇女的魔力,去吃收那社会的银钱,一古脑儿都供这个慈善会的用度。我们中国却没有举行过,何不趁着这个时候借着味莼园的地方也开一个慈善会,普请那些绅商人家的内眷都在会中执役。预先印了入场券各处分销,每张卖一块钱,大约这一笔人场券的钱倒也不少。”想着便又转个念头道:“这件事儿,最好请陈宫保做个发起人,好在他也是江苏人,向来在慈善事业上很肯花钱的,料想他病怀桑梓,一定不推却的。”

想着,便立刻坐了马车,到斜桥陈宫保的行辕里头来,禀见这位商约大臣陈寅孙陈宫保。手本投了进去,候了一回,陈宫保慢慢的出来。孙观察便把自己的意思和陈宫保说了一遍,要请陈宫保做个发起人。陈宫保听了,喜道:“我正在这里踌躇这淮、海饥民的事情,如今你出这个主意好得狠。我是个江苏人,这担任发起的事儿自然是无从推诿的。便是我想起来,就是发卖入场券,也卖不出什么钱,不如合着上海全埠的绅商内眷,大家都在张园里头设肆售物,把卖下来的钱都充作捐款,你说这个主意怎么样?”正是:

牺牲名誉,救亿万之同胞;递泪江皋,听中宵之鸿雁。

不知这个慈善会怎样的一个开法,且待下回交代。





第一百八十八回 悯哀鸿仁人兴义举 泛明湖好景入诗囊





且说孙观察听了陈宫保的话儿,便大喜道:“究竟是陈宫保想得周到,职道却一时想不出来。这样的一来,一定可以多得几万块钱。多得一块钱,就可以多救一条性命,这都是宫保的功德。”陈宫保也谦逊几句道:“这是我们分内的义务,算得什么功德?”说着便又和孙观察商议了一回,把会里头一切章程都议得停停当当。

陈宫保又道:“专靠我们中国人,究竟没有几个肯出大钱的。最好要想个法儿,把那些寓沪的西人也拉进会里头去,方才热闹。”孙观察想了一想道:“待职道先到工部局拜会局董,看他的意思怎样。大约据职道看起来,那些欧美各国的人,在慈善事业上大家都肯出力帮忙的,料想没有不答应的道理。”说罢,便辞了陈宫保,先到虹口地方来,看他一个英国女友叫做哈罗利夫人的,和他商议一番。

这哈罗利夫人向来和孙观察狠要好的,却又和工部局总董叫做喀伦达立夫的两下狠说得来。当下听了孙观察的说话,便拍手赞成道:“我们虽然是大英国的人,却居留在你们贵国,又和你们贵国有邻国的谊分,这件事情也是我们分内应尽的义务。料想我们英国人都有仗义好善的性格,断没有不答应的。如今我先到喀伦君那里去问他一下,看他的意思如何。”孙观察听了连忙殷勤致谢,说了许多感激的话儿。

当下,哈罗利夫人果然立刻到工部局去见了那位喀伦达立夫,把孙观察的说话和自己的意思都说了一遍。那位工部局总董事喀伦达立夫君也十分赞成这桩义举,又和各国领事商量了一回,大家都是十分高兴,拍手赞成。并且那十三国领事都情愿叫自己的夫人也在张园里头设肆售物,把卖出来的钱都交在中国慈善会里头去,拨作徐、海、淮、扬的赈款,尽个邻国的义务。哈罗利夫人听了大喜,连忙和孙观察说了,孙观察自然甚是欢喜。

当下陈宫保、孙观察议定了会中的一切布置、一切章程,便推举了二十名干事员,分头办事。恰恰的这位辛修甫也被他们推举在里头,做了个干事长。那些指定的方向、铺设的会场、预备的商店,都要辛修甫一个人往来奔走,流汗相属,忙得一个发昏章第十一,那里还有工夫来娶什么小老婆?

这些闲话我且按下不题。只说贡春树和刘仰正两人,都在浙江杭州地方。贡春树是捐了个知县,分发浙江;刘仰正应了杭州将军的聘请,和他管理折奏。两个人虽然时常相叙,却每每的当着那茶余酒后的时光,遇着那月夕花晨的佳日,大家都不免常常的要想起章秋谷、辛修甫这一班要好朋友来。这一天,刘仰正雇了一只湖船,邀着贡春树一同去游西湖。船上的人解了缆,一路轻轻的荡过来。这个西湖,本来是中国第一著名的胜地。这个时候又正是四月初旬的时候,沿着湖堤一带还有些开不尽的桃花,三三五五的临风招展,夹着那些绿沉沉的扬柳,衬着那波光一碧,微微的有些摇动,好似那轻罗薄觳一般。那四围的山色也是午岭浮青,遥峰界碧,直是天地生成的妙景,连画图上都画不出来。那西湖的水本来是十分澄澈的,看着那水底的行藻纵横,看得甚是清楚;船上的人影倒入水中,须眉毕见,好象是一面大镜子的一般。贡春树和刘仰正坐在船中凭栏玩赏,只见楼台隐约,烟水迷离,嶂影涵青,波光漾碧,只觉得神怡心畅,头目爽然。

贡春树和刘仰正谈了一回,刘仰正道:“这般景物,可惜秋谷、修甫等都不在这里!”春树道:“秋谷自从太夫人逝世,回到常熟去闭门守制,连至好的朋友都不狠通信。两年之间,我一连发了五六封信去,只接了他一封回信,不知道是什么缘故。计算起来,他的服已经满了,为什么躲在家里还不出来?”仰正叹一口气道:“秋谷近两年来运气也狠不好,自从其盛倒闭,被徐齐甫吞没巨款之后,家产便去了大半。去南京乡试,虽然三场满荐,又被主考落掉了。又为着教演拳棒的事情,大家竟都谣传他是个会匪的头目。你想可笑不可笑?直是曲高和寡,少所见而多所怪了。”春树道:“前天我接了修甫的一封信,说就在这个四月里头要娶姨太太,只等过了万国赛珍会便要举行大礼。我们何不到上海去走上一趟?这个赛珍会是难逢难遇的,我们去看了赛珍会,再去扰修甫的喜酒,不知你的意思怎么样?”仰正拍手道:“我正有这个意思,不想你和我竟有同志。我们明天就去,何如?”春树道:“我们就明天去也好。”两个人定了主意,便一同上了轮船,直到上海来。

到了上海,两个人都住在辛修甫的公馆里头。过了一天,恰恰的张园赛珍会已经开场。贡春树和刘仰正两个,少不得也要买两张入场券进园游览。贡春树刚刚走进园门,早见一个十六七岁的丽人浓妆艳抹的,手中提着满满的一篮花,袅袅婷婷的迎将上来,对着贡春树嫣然一笑,在篮里头取出一朵花来,对贡春树道:“请买一朵花,尽个同胞的义务。”贡春树被他说了这一句,倒觉得有些不好意思起来,连忙把那丽人手中的花接了过去,扣在钮扣上。那丽人微微一笑,又照样的取了一朵来,递给刘仰正。刘仰正也接了。贡春树便取出两块钱来交在那丽人手内。那丽人接了,笑盈盈的对着他们两个点一点头,走到那边去了。两个人慢慢的走到安垲第来,只见那安垲第中间陈列着许多东西,都是些泰西士女在那里四处兜揽生意的。

那安垲第的两旁隔作十几处,好象是十数间厢房的一般,却是十三国领事的夫人分厘列货的在那里掌柜。安垲第的前后,又有许多欧美各国的女士,也有设着博彩摊的,也有卖点心食物的,大半都是些少年貌美的人,一个个都打扮得金钻照眼,锦绣流光。两人一路走来,东看看,西看看,真有些应接不暇的光景。出了安垲第,又到老洋房去看了一回,都是陈设的珠玉绣货、古玩字画,陈设得五光十色,光怪陆离。再转过河边,便是一方草地,围了一个艺场,有几个中国人在那里舞枪弄棒。

两个人各处走了一遍,觉得有些腿酸起来,正要寻个歇息的地方。忽见一群的滑头滑脑的少年,大家都勾肩搭背,一窝蜂直拥过来。听得人丛里头有一个少年哈哈的笑着说道:“我们出了一块钱,倒像打了一个中西合璧的大茶围一般。”这一句话方才出口,猛听得对面有个人大声喝道:“这算什么话儿,真是混帐!”那说话的少年听了有人骂他,也就回骂道:“你是个什么人?敢于这般放肆!我说我的话,与你什么相干,要你来起什么劲儿?”对面那个人听了,更加大怒道:“今天是他们那班中西女士不惜牺牲名誉,来拯救我们中国的灾民。你也是中国人,该应感激才是,怎么的放出这样屁来!”那少年听了也大怒道:“你的说话便是放屁!

像你这样的道学话儿,上海地方用不着,劝你还是少讲几句罢!“那对面的人听了,怒不可遏,忍不住大踏步的抢过来,一把抓住了那少年的衣服,好象拎着个小鸡的一般,口中说道:”我把你这个全无心肝的东西,你自己错了,还敢这般倔强!如今我也不来和你说什么,只和你当着大众评个理儿,这样的说话,你究竟应该出口不应该出口?“这个时候,已经有许多的人听得有人吵闹,大家都围将拢来。

贡春树和刘仰正两个起先听得对面那个人的声音甚熟,明明是章秋谷的声音,两个人不由的满心大喜。大家都抢上一步,举目看时,果然不是别人,就是那位文妙天下、厥性好骂的章秋谷。两人一眼见了秋谷,正待要叫时,只见章秋谷一手扯着那方才说话的少年,对着大众朗然说道:“你们诸位听着,今天的赛珍会,是中西女士为着那班淮、海的饥民嗷嗷待哺,所以大家都牺牲名誉,开这个赈荒赛珍会,用意十分可感。我们做男子的人不能够帮着他们尽些义务也还罢了,怎么方才这个东西竟会说出那样轻薄的话来?说什么出了一块钱,倒打了一个中西合璧的大茶围。

你们众位请想,这样的话儿可该说不该说?可荒不荒唐?“

众人听了,有几个胆小怕事的便走了开去,有几个有些义气的,大家也都数说方才说话的那个少年,说他不应说出这般轻薄刁钻的说话。那少年起先被章秋谷一把拉住了衣服,觉得这个人气力不小,英毅非常,心上已经有了几分馁意。却又受了那几个同伴的激发道:“你口中说话是你的自由权,怎么他平空的干涉起来?这还了得!”那少年受了众人这般一激,便也想要装些虎势出来。无奈看着这章秋谷两只眼睛光芒闪闪的,只是凛凛的对他看着。更兼被章秋谷一把抓住了挣紥不来,动弹不得,不由得心中有些害怕,口中却支支吾吾的说不出什么来。如今又听了众人的话儿,许多的人异口同声的都怪他不该如此,早已吓出一身冷汗来,连忙向着章秋谷道:“你且先请放手,有话再说。我方才的说话,实在是一句信口的话儿,并不是有心轻薄。你们众位不消生气,我自己认一个错就是了。”章秋谷听了那少年自家认错,方才放了手道:“既是你自己认错,我也不来和你计较。”那少年见秋谷放手,好容易得脱了身,一言不发,三脚两步的望着弹子房那边走去。

章秋谷方才回转身来,早听得有人叫道:“秋谷兄,我们多时不见了,渴想得狠!”秋谷听了连忙回头看时,见果然是贡春树和刘仰正两个,不觉心中大喜。连忙走过来大家相见,拉着手寒温了几句。刘仰正道:“这里不便讲话,那边有一个东洋茶棚,我们去坐一会儿也好。”章秋谷听了点点头儿,便同着他们两个走进茶棚去,拣个座儿大家坐下。刘仰正便问问秋谷这两年来在家里头的情形,秋谷长叹一声道:“说起我的事情来,真是一言难尽。”

看官,你道这位章秋谷这两年之间为什么不到上海来,却这样销声匿迹的躲在家里,这是个什么道理?原来章秋谷自从那一年在南京得了上海家里头的电报,连忙赶回上海,急急的赶到新马路公馆里头,看太夫人时,原来太夫人是个秋痁,虽然来势利害,却也没有什么大碍,只为着有一家合本的典铺叫做其盛的,被管事人徐齐甫亏空了本钱,故意放火,把一个黄铺烧得个干干净净,一物不遗,还欠了外面的许多帐目。正是:

垄断尽东西不利,市会之良;火攻出决死之军,奸奴大胆。

不知后事如何,请待下回分解。





第一百八十九回 吞存款市侩昧良 萎慈萱北堂弃养





且说章秋谷自从老太爷故后,虽然有些宦囊,却也不多。历年以来,章秋谷在外面挥金结客,慷慨非常,已经花费了许多。更兼这几年之内,轻裘肥马,访柳评花,名妓倾心,良朋聚首,阅历了无数的歌场酒阵,经过了许多的荡叶狂花,真个是鹿锦缠头,貂裘换酒,买笑则珍珠一斛,留欢则黄金百斤。虽然章秋谷是个惯家,不至于受了倌人的迷惑,但这个嫖的一个字儿,凭你怎么精明剥削的人,也是有出无人、有绌无盈的。秋谷在上海堂子里头混了几年,却也着实花掉了几个钱,不知不觉的把这些有限的银钱,渐渐的用得干涸起来。

幸而章秋谷的那位太夫人性情豁达,不是那爱钱如命的人物,见家里头的钱给章秋谷用掉了一大半,心上也不狠着急,只说:“凭着自己这样的一个儿子,将来一定不是池中之物,这几个钱不过是身外的东西,何足挂齿?”章秋谷听了太夫人这番说话,越发的把银钱看得真个就如傥来的对象一般,随意挥霍。到了这个时候,刚刚只剩得其盛典铺一万五千银子的股本、汇丰银行的一万三千银子存款,统统合起来,不到三万银子。

这个其盛典铺的管理人叫做徐齐甫,本来是个当铺里头的小伙计,却是章秋谷的那位老太爷一手提拔出来的,先合了几个股东,开设这个其盛典铺,叫他在里头管帐。又在外面和他各处的揄扬,一时间传说开去,就在别个典铺的东家来请他去当经理。不上几年,竟大大的得意起来。章秋谷的那位老太爷故后,他便不知怎样移花接木的先吞没了一笔存项。那个时候,章秋谷正在哀痛忙乱的时候,况且年纪还狠轻,一时间那里查察得出?只说这个徐齐甫古板诚实,是个靠得住的好人。那里知道他外假忠诚,内怀鬼蜮,故意的放火把典铺烧了,把别人家典的东西,拣贵重些的金玉珠宝,一古脑儿都暗暗的搬回家去。等到火息之后,查起帐目来,典铺里头的六万银子,本钱一卷而光不算外,外面还欠一万几千银子的亏空,这是要几家股东拿出来的。那其余的三家股东,都还当着徐齐甫是个好人。只有章秋谷心上早已明白,但是查不出他的什么凭据,一时也无可如何,只暗暗的把自己疑惑的意思和那三个股东说了一遍。那三个股东听了,大家甚是相信,便和秋谷商议,要禀了上海县把他看押,追他的钱。秋谷道:“禀官提押的事情,虽然可以做得,但要想他把我们的钱拿出来,是没有这件事情的了。只要这样的一来,我们不至于再拿出钱去,也就罢了。”

章秋谷为着这件事情,倒一连闹了半个月,方才弄得清楚。虽然没有倒转拿出钱来,这一万五千银子却是丢到水里头去了,连响声也没有听得一点。章秋谷回到自己家里头,却不敢和太夫人说,只把几句假话搪塞过去。只说已经收了一万银子回来,还有五千银子立了一张期票,明年归还。太夫人听了,起先还不相信。章秋谷恐怕太夫人病中发急,只得假造了一张汇票和一张期票,给太夫人看了一看,方才放下心来,那病体就轻了好些。章秋谷的那位夫人却悄悄的埋怨他道:“你这个人怎么这般的好说话!白白的一万五千银子送了别人,这是什么缘故?你常说天下的事情,不论什么人、什么事,总有法儿好想,只有穷人没有银钱和病人沉重要死的这两件事情,却是没有法儿。如今这样一个小小的徐齐甫,怎么平空被他吞没了一万五千银子,想不出一个处治他的法儿?难道就是这样的罢了么?”

秋谷道:“你们没有见过这个人,那里知道他的可恶?他凭你怎样的和他生气,要打他要告他,他只是和你软缠,笑嘻嘻的满口自认不是,抱怨自己不小心。你若是打他一顿,他只是一个不开口、不动手。你若是把他送到当官,他拼着看押起来,暂时不要出去。你若是要他赔钱,他又满口说是应该赔的,可惜拿不出钱来。你想这样的一个人,有什么法儿处治他?最可恨的是那三个股东,都情愿自认晦气,这笔钱是不要的了,难道我一个人去追他的钱么?况且就追也追不到的,又访查不出他放火吞财的证据,还是落得装个大方的好。”他夫人听了章秋谷这番说话,嘿然半晌道:“如此说来,这一万多银子竟是白送给他的了?”秋谷道:“他虽然这样瞒心昧己的弄了几个钱,但是他那个后娶的老婆成天的在那里和人吊膀子,拚命的倒贴;更兼他那几个公郎,虽然一个个都目不识丁,却倒是吃、着、嫖、赌件件俱全的。他这几个钱,悖人的一定悖出,那里会保守得住?真叫做人有千算,天有一算,我们何必再去和他计较?”他夫人听了,也就不说什么。

过了几天,章秋谷见太夫人的病一天好似一天,心上好生欢喜。不想事机不巧,晦运忽临。这一天,太夫人正坐在房中看了一回小说,觉得有些闷倦,便慢慢的起来试走。章秋谷和陈文仙一边一个扶着。走得不多几步,突然见个小丫鬟名叫采菱的,手中拿着一封电报走进房来交给章秋谷。秋谷一眼看去,见封面上写的“常熟电报”,心上先是一惊,遮掩不及。太夫人也早已看见,便吃惊道:“常熟电报是什么事儿,快拆开来我看!”秋谷虽然心中着急,却又没奈何,只得把电报拆了开来,把一张电码递在太夫人手内。暗想:“只要是没有翻好的,我便好在里头做个手脚了。”一面想着,侧着头去看时,却偏偏又是翻好的。说时迟那时快,正在这般时候,早听得太夫人叫了一声“阿呀”道:“不好了,我的小萱死了!”说着,便把手中的电报掼在地下,放声大哭。

看官,你道这个小萱是什么人?原来章秋谷在常熟城内本来还有一处住宅,如今太夫人为着秋谷在上海就馆,心上十分惦记,所以带着他夫人一同到上海来住。

章秋谷的那位太夫人一生就生了二男三女。长男就是秋谷的胞兄,也是文行俱优的人物,到了二十一岁上,便得病死了。寡嫂史氏,是过门守节抱着木主成亲的。第二个就是秋谷。第三个女儿就是秋谷的胞妹,乳名叫做小萱,已经出阁,嫁给无锡文氏。第四第五个女儿名叫小芙、小蕙,都已经字人,尚未出阁。太夫人自到上海之后,便把这位文姑奶奶接回家中,同着那位大少奶奶和四、五两位小姐,一同看守住宅。起先,原说在上海住上半年三个月也就要回来的,谁知一住就住了差不多两年光景。

这位文姑奶奶为着那位文姑爷出门去了,便安安心心的长住在娘家。也曾到上海来过两次,住了一两个月便又回去了。如今却不知怎样的,一时感冒,染了喉症,请错了医生,把极重的喉痧当作伤寒,只一贴药便闭了喉管,焦热上冲,不到两天把好好的一个人送到阎王家去了。那位大少奶奶,起先只说不要紧的,知道太夫人在上海生病还没有全愈,只恐惊了太夫人,不肯发信。到得病势沉重起来,方才慌了手脚,要打电报去叫章秋谷时,那里来得及!一霎时的工夫,病人已经气绝。没奈何,只得打个电报通知秋谷,刚刚被太夫人亲手接着。章秋谷纵有通天手段,一时也施展不出来。

只说当下太夫人接了这个电报,偏偏这位文姑奶奶在三个女儿之中又是最钟爱的一个,忍不住放声大哭起来。秋谷站在旁边,早已看见了那封电报上的字儿。章秋谷平日之间,本来最是笃于手足,一班女兄弟们和秋谷也都甚是相爱。看了这封电报,不由得心肠搅痛起来,一霎时泪如泉涌。却又看着太夫人这般悲痛,自己不敢放声大哭,只得勉勉强强的忍住了,倒反来劝慰太夫人,只说母亲病后须要自己保养些儿。太夫人那里肯听,直哭得泪干气尽,力竭声嘶,方才住了哭。倒在床上,却顿时旧病又发起来,那来势比前更重,抖得浑身的骨节都“格格支支”一片声的怪响。秋谷慌了手脚,连忙去请了医生来,吃了一贴药竟不退热,索性的发狂谵语起来。秋谷衣不解带的伏侍,一连这样的五天,头上的焦热依然不退。一班医生都说不出这是个什么病儿,只葫芦提定个脉案,开个药方,那里中用?只把一个章秋谷急得好象个掏了头的苍蝇一般,没奔一头处。

又过了几天,太夫人的焦热虽然退了,却微微的有些气喘上来。太夫人自知不起,便叫了儿女、媳妇都到床前。原来这个时候,那位大少奶奶和四、五两位小姐已经从常熟赶到上海侍疾,所以一家的人一古脑儿都在这里。太夫人一个个看了一遍,叹了一口气,先向章秋谷道:“你的为人狠有些儿气骨,我也没有什么不放心。

这家里的几个钱,是我死之后料想保不住的了。凭着你这个人,也不愁挣不出这几个钱来,我也没有什么放不下。我所不放心的,是你平日之间一味的恃才傲物,在外面结了无数的冤家,将来一定要受他们的陷害。你自今以后须要处处留心,不要这样的眼高于顶,终久没有什么好处的。你们等我死后,一切发送都从省俭。服满之后,快些给两个妹子完了姻事,这是最要紧的事情。至于你平日间专爱到堂子里头去混闹,别人都说你不该这样,只有我一个人知道你的意思,无非是为着心上不得意,便故意到堂子里头去这般混闹,借此发泄你的牢骚,所以我也从没有说你一句。只要你把这个恃才傲物的性格改掉了,我就死了也瞑目的了。“

章秋谷听了太夫人这番说话,那心胞里面好似万刃攒刺、万箭激射的一般,那眼中的泪便像那峰顶飞泉、檐头急溜,滔滔滚滚的直冲下来。却又不好放声哭出来,恐怕太夫人听了心上更加难过,只得竭力忍住了连声答应。太夫人把几个媳妇和女儿都叫过来,都嘱付了一番。又把陈文仙叫到床前,对他说道:“别人家娶倌人的,每每到后来总弄得一个有始无终,惹人笑话。你却不比别的倌人,一定没有这些举动。但愿你和少奶奶妻妾和谐,早些生个儿子,也不枉你嫁人一场。”陈文仙泪流满面的答应了。

一会儿灵风习习,瓶内的两枝桂花发出一阵一阵的香来。太夫人觉得有些喘呃起来,便慨然说道:“一个人那一个能不死?不过迟早些罢了。你们也不必悲伤,我也没有什么挂碍。这个时候,一个心觉得空空洞洞的,只你们一班儿女,觉得还有些爱情牵惹,割舍不得。”说到这里,不由得落下两点泪来,微微的叹一口气,蓦然的合上双眼,一言不发。秋谷等连忙叫时,已是喉间气绝,脸上却还带着笑容。

正是:

蓼莪抱憾,心伤陟屺之诗;风木终天,血染思亲之泪。

不知以后如何,下文交代。





第一百九十回 章秋谷闭门守制 祁祖云挟忿兴谣





且说章秋谷见太夫人已经气绝,不觉得心肝俱裂,肺腑皆摧,抢上一步,抱住了太夫人嚎啕大哭,一连哭晕了数次,直哭得石人下泪,铁汉伤心。那位大少奶奶见秋谷哭到这般模样,着急起来,倒反自己先住了哭,又劝止了大家的哭,几个人走过来苦苦的劝止秋谷。只说办事要紧,如若你哭坏了,有什么人来和你经理殡葬的事情?秋谷哭到这个时候,只哭得四肢皆颤,口中呕出大口血来,还在那里拼命的号哭。大家见不是头,不由分说,把秋谷生生的拥了开去。在太夫人床前地下铺了一床芦席,把秋谷捺着睡下。秋谷要想挣紥起来,却觉得浑身上下没有一些力量,不由得又痛哭起来。那位少奶奶见了秋谷这样一丝两气的样儿,当真的着起急来,便同着那两位小姐一齐跪在秋谷面前,苦口劝解。只说你是如今最要紧的人儿,万一个有了什么差池,叫我们大家怎么样呢?秋谷见了嫂嫂和两个妹子都跪下相劝,自己又立不起来,只得连忙叫了他夫人和陈文仙过来,把那位大少奶奶和两位小姐都扯了起来,自己也只得勉勉强强的忍住了哭,一面连忙请了几个亲戚朋友来帮办丧事。

这几天之内,秋谷的悲恸痛切自不必说。到了大殓过了,章秋谷悲痛过度,卧床不起,直病了二十多天方才挣得起来。章秋谷为着太夫人在生的时候最信的是佛教,便到常州天宁寺里头去打了一场七天七夜的水陆,差不多也花了一千块钱。又连忙看了安葬的日子,家奠领帖、出殡举襄,都办得停停妥妥的。以前第一集书中已经表过章秋谷的祖父坟墓都在常州,所以在常熟地方受吊一次,举襄一次,到了常州地方又要受吊一次,举襄一次,比起别人来更加糜费。好容易风风光光的把太夫人殡葬事情都办妥了,免不得痛定思痛,又把心上的悲恸提了起来,便静静的坐在家内,闭门守制。

谁知福无双至,祸不单行,又闹出一件意外的岔儿来。原来这个时候,正值江苏各地袅匪横行,地方官畏葸怕事,不敢过问。甚而至于大帮枭匪把地方官的稿案、家人都掳了去,要他出钱来赎。地方官只好眼睁睁的看看他,无可如何。地方官见了枭匪,尚且要怕到这般田地,别人更不必说了。渐渐的纵容得这般袅匪愈加放肆起来,强买强卖,遇事生风,闹得一班地方上的百姓,一个个都畏之如虎,不得安居。

常熟这个地方和福山相近,也算是个沿江近海的地方。那班贩卖私盐的枭匪,每每的到常熟地方来骚扰,大家都束手无计,没奈他何。就有几家绅士家的子弟来和章秋谷商议,说枭匪这样横行,官兵不敢过问,这便怎么样呢?秋谷慨然说道:“如今的世界,比不得以前的太平时代,要想倚仗着法律保护身家是靠不住的了。

只有一家家的人一个个都熟习武技,人自为战,那时不但可以抵挡这些枭匪,就是再利害些儿的也不怕他。“这班人听了章秋谷的说话,大家都说不错,便真个的想要人自为战起来。聚拢了一二十个人,都是些绅衿人家的子弟,大家都缠着章秋谷要他教习拳棒。章秋谷起先不肯,后来被他们大家再三央告,便也点头应允。天天到了下午三点钟的时候,便都往秋谷家里头来。秋谷耐着心一一教授。

一连教了几个月,那班徒弟一个个都学会了几套拳法、几件兵器。那班人原都是些少年好事的人物,如今学了拳棒,更加的胆大起来,未免要在外面任意闯事。

秋谷一连告戒了几次,他们大家那里肯听!有一天不知怎样的,见了祁祖云祁侍郎的家人在门外强买对象,众人不服起来,一拥而上齐声喝阻。那家人是平日放肆惯了,看得这班人那里在他心上?三言两语争闹起来。众人心中大怒,先把那家人打了一顿,又堵着祁侍郎的门口骂了一场。祁侍郎见人多了,不敢出头,凭着众人骂了一场去了。祁侍郎心中怀恨,便叫个门下的走狗叫做康长垣的出去打听了一回,方才知道这几个人都是章秋谷的徒弟。

祁侍郎听得提到章秋谷的名字,便觉得怒从心起,恶向胆生,口中说道:“这个小畜生前一次把我撞了一交,我还没有去寻着他,他倒指使了这班混帐东西来上门骂人。我若不给他一个手段叫他知道我的利害,我这个‘祁’字也不用姓了!”

说着,便会齐了那些走狗,密密的商议。一个走狗便走上前来,附耳说道:“他聚众教拳,本来有干例禁的。我们如今只说他是会匪的头目,聚了许多党羽教演拳棒。

只要这个风声一传出去,只怕他吃不了要兜着走呢!“祁侍郎听了十分欢喜,连连的点头道好。又鬼鬼祟祟的商量了一回,方才大家散了。

果然不多几日,常熟城内传出几句谣言来,只说章秋谷是会匪的头目。更有几个无耻的劣绅,大家都附和起来。章秋谷的一班亲戚、朋友听了这些说话,大家都十分不忿,一个个都对着章秋谷说,叫他设法分辩。章秋谷却付之一笑,不去理会,只说:“一个人的毁誉是说不定的。他们这般传说,只顾凭他们去传说就是了。我只要问心无愧,何必要去分辩?况且这般龌龊小人,即如华廷栋和祁伯田等这班宝贝,素来被我讥诮奚落惯的,恨我好似切骨仇人的一般。就是向他分辩,他还只道我自己心虚,所以这般着急。还是凭他去怎样兴谣造诼,将来自有明白的日子。”

他夫人和陈文仙听了,也只得由他。

自此之后,章秋谷索性闭门守制,不与外事,连几个知己些的朋友都不相往来。

渐渐的,这个信息一传十、十传百的大家都传说起来。再加上华廷栋和祁伯田这几个宝贝竭力的吹风纵火,说得活龙活现的十分相像。除了几个章秋谷的亲戚朋友不肯相信,其余的人大家都不由不信起来。慢慢的这个信息竟传到商约大臣陈寅孙陈宫保耳朵里头,心上大为诧异,便写了一封信给章秋谷,叫他到上海去。

章秋谷也不知什么事情,只得立刻坐了小火轮径到上海来,见了这位陈宫保。

陈宫保第一句就问起这件事情来,只说:“我听得人说你人了会党,究竟有这样的事情么?”秋谷微微一笑道:“宫保的明见,看晚生可像个会党么?这些谣传的话儿也有一个缘起,却是晚生自己不好。晚生平日之间少年盛气,未免有许多得罪人的地方。那几个捏造谣言的人,都是和晚生向有仇恨的。这样的谣传非但无从辩起,并且也不屑去和他分辩。宫保请想:晚生纵然胡涂,却也幼读诗书,长知道义,怎么会平空人起会党来?况且人了会党,于晚生又有什么好处?这样有损无益的事情,那一个肯干?只求宫保细细的想一想,就明白这些说话一定是谣传了。”

陈宫保听了;想了一想,觉得秋谷的话不错,便也点一点头,嘿然不语。停了一会方才开口说道:“据你这般说起来,这件事儿原是你自己招出来的,和别人不相干。自今以后,你那瞧不起人的性格,还该收敛些儿。古来的圣人处事,也都是谦和为贵,何况我们这般人,究竟不是圣人呢。一定要嬉笑怒骂的,到处锋芒太露,傲态向人,在世路上结了无数的冤家,究竟在自己身上没有一些儿的好处,这又何苦?”秋谷听了陈宫保劝他的一番说话甚是关切,心上狠觉得有些感动,便也说道:“晚生自恨从小儿多读了几卷书,以致到了这个时候眼高不低,肠直不曲,委实和那班龌龊无耻的小人拉拢不来,只得凭着他们去怎样的了。”陈宫保听了,也不免嗟叹了一番,又着实的劝了几句。章秋谷暗想此公虽然有些富贵习气,却倒具着这样的热心。心上想着,口中少不得连声答应,退了出来。原来这位商约大臣陈宫保和章秋谷的老太爷是总角之交,陈宫保的夫人又是章秋谷的亲戚,所以和章秋谷倒狠关切。

只说章秋谷回到常熟,依旧闭门不出。辛修甫因为书局里头没有办事的人,屡次写信请他到上海来,秋谷只写了一封回信给他,叫他另请别人,自己仍旧不肯出来。直到得守满了两年二十七个月孝服,秋谷守着太夫人的遗训,急急的和两个妹子料理出阁的事情,倒也整整的忙了几个月。等得那两位姑奶奶一齐出阁之后,章秋谷把家里头的计算一番,刚刚只剩了七千五百银子,合起来差不多也有一万块钱。

秋谷便和他夫人商议,要索性把住宅典给别人,搬到上海去住。

陈文仙插口说道:“住到上海地方去,开销大得狠,不如还住在这里,现现成成的房屋,每月可以着实省几文钱。”秋谷想了一想道:“我如今把这一笔汇丰存款,一古脑儿都提了出来,放在当铺里头,可以每月多些利息,一个月也有七八十块钱。你们家里头的开支,有了这几个钱也勉强够了,只是我的用度却没有在里头。”

陈文仙道:“你要用钱,我还有一千多块钱,原是你经手给我存放的,你只顾用就是了。再有什么不够,我还有些首饰,也还可以算得几个钱,一时间料想也还不至缺乏。”秋谷笑道:“你只顾放心,我如今虽然不比从前,却也还不至于要用你的钱。倒只怕你在上海的时候舒泰惯了,如今过不惯这般日月,那就要另想法儿了。”

文仙正色道:“这个不用你费心,我若过不惯这般日月,我又何必要嫁什么人?”

秋谷笑道:“虽然如此,只是你嫁我一场,没有得到什么好处,却倒反要你熬清受淡的过起这样的清苦日子来,我心上委实觉得过意不去。”文仙微笑道:“一个人住了现成的房屋,吃了现成的茶饭,还有什么不惯?老实和你说了罢,我们当倌人的嫁人,只要果然嫁着了好好的客人,自己心上没有什么不愿意,那些身外的事情都是可以随便得来的。那班不愿意嫁人的倌人,方才横又不是、竖又不是的有心挑眼,好借此闹着出去。若是当真愿意嫁人的人,将来总是自己一家人,有什么过不去的地方?”

秋谷听到这里,一面微微的笑,一面上上下下的打量了陈文仙一眼道:“果然只要心上没有什么不愿意,别的事情便都是可以将就的么?”文仙听了忽然面上一红,瞅了秋谷一眼,回转身来往外便走。秋谷看了又是微微一笑,不说什么。他夫人风了,不懂是什么意思。正要开口问时,章秋谷对着他夫人做了手势,他夫人方才明白,也是面上一红,啐了一口。正是:

十年落拓,司勋之绮恨偏多;风里风尘,狂白之黄金欲尽。

要知后事如何,下文交代。





第一百九十一回 救灾黎大开赛珍会 放焰火普照不夜城





且说章秋谷把家计安排了一会,便商订行期,自己一个人到上海来提取汇丰银行的存款,兼带着看看万国赛珍会的情形。此时常熟到上海已有小轮船,只消一夜的工夫,往来狠是便捷。这一天,章秋谷到了上海,在吉升栈占了一间官房住下,也不出去探问朋友,便叫当差去叫了一部亨斯美双轮马车,提鞭按按,径往张园。

从石路转出大马路,风驰电卷的一直线望西而行,蹄声得得,转眼已到。下车进门,但见旗帜飞扬,满园内花团锦簇的,热闹非常。秋谷至各处游览了一周,忽然听得那个少年说就出这样的话来,不觉得心中火起,抢出来抱个不平,却刚刚的遇着了刘仰正和贡春树两个朋友。

当下,贡春树和刘仰正两个听了章秋谷的一番说话,不觉心中气忿起来,把那祁伯田、华廷栋着实的骂了一顿。秋谷倒笑道:“你们何必去骂他?像他们这样的人都是禽兽一般的畜类,我们不犯着去骂他。譬如一个人给疯狗咬上一口,难道也去和他讲理不成?”正说着,只见一个侍者送上三盘点心来。秋谷看时,见是每盘一块奶饼、一方蛋糕、两方糖饼。三个人也随意吃了些。

秋谷又抬起头来,四下里看了一看,只见四下里有许多日本少年女子,都打扮得脂香粉艳、锦衣绣裳的,在那里穿梭一般的应酬游客,却是别有一般诧异。这班日本女子见了个西洋人走进来,便争先恐后的巧笑承迎;见了个中国人走进来,便眉斜眼瞪的洋洋不睬,只叫那中国侍者过来伺候。秋谷看在肚里,暗暗的心中好笑,便对着贡春树和刘仰正道:“这班日本女子是势利不过的,我手上向来不带戒指,你们两个何不走过去,把手上的钻石戒指在他们面上晃上两晃,看他们怎么样?”

贡春树和刘仰正听了,果??故意大摇大摆的走过去,把手上的戒指故意露出来,在他们面前打了两个转身,依然慢慢的归座坐下。只见那班日本女子一个个俊眼斜睃,秋波微动,一窝蜂的都拥到这边桌上来,七手八脚的添茶伺水,应酬不迭。秋谷见了不觉哈哈大笑,对着他们两个人道:“何如?”他们两个人看着秋谷也只是笑。

三个人一面笑着,一面立起身来付过了钱,走出门去。走了一回,忽然又见两三个中年妇女,托着一个盘,盘里头放着几匣纸烟,几方手巾,硬硬的拦住章秋谷等不肯放走,把一匣纸烟塞在章秋谷手内,强要他买。秋谷把他们看了一看道:“这个会场里面,凡是兜卖对象的女士,都有天足会的徽章,你们几位的徽章在什么地方?那边纠察员来了。”这几句话儿,把那几个人说得满面生红,回身便走。

章秋谷见了哈哈一笑。

一会儿又走到安垲第面前,只见安垲第的右手一带,一连接着十几间铺面,陈列着无数的东西。原来是商约大臣陈寅孙陈宫保的夫人带着一班少年妇女在那里兜卖对象。章秋谷恰恰的走过去,被那位陈夫人一眼瞧见,招手叫他过来,要他买些东西。秋谷便随意买了一柄扇子,走了开去。又去找着了辛修甫,闲话一番。

到了晚间,那些会里的人役,把些椅子、茶几都搬到外面草地上来,好预备演放焰火。章秋谷也同着刘仰正等拣几张椅子坐下。不多一会,早已男男女女的接踵联袂,相率偕来,把那些椅位都坐得满满的,水泄不漏。章秋谷留心举目往四下里细细的看时,只见那班少年男女一个个都在黑地里遮遮掩掩、鬼鬼祟祟的,不知道做些什么事情。这一边携手殷勤,那一边凭肩款曲;这一处纤腰倚玉,那一厢玉笋钩云,真个是一双双的同命鸳鸯,一对对的双飞蝴蝶,连焰火也顾不得看,一味的在那里安心熨贴,着意厮缠。

秋谷看得不耐烦起来,看着那几套焰火也没有什么好看,便同着刘仰正等立起身来,顺着池边一带慢慢的走去。走到一带树林左畔,秋谷的耳朵最尖,早听得有男女两个人的声音低低的在那里说话。一个女子声音说道:“你要我叫你什么?你行三,我就叫你三哥哥何如?”又一个男子的声音说道:“你叫我三哥哥,我就叫你四妹妹。”章秋谷听了,连忙轻轻的赶上一步,举眼看时,只见一株大松树的后面隐着一男一女两个人,男的学生打扮,女的也像是个女学生的样儿,两个人紧紧的搂作一团。秋谷故意高高的咳嗽一声,把那男女两个人吓了大大的一跳,连忙放了手,回身就走。

大家笑了一番,又往前走了几步。贡春树忽然扯了秋谷一把道:“你看,你看!”

秋谷回过头来,果然见丛林里面隐隐的男女两人并肩站着。只见那男子附着女子的耳朵不知说了些什么,那女子回过头来,把一个指头向着那男子一伸,大声说着英国话道:“辟因斯!”秋谷虽然不懂西文,那浅近些的话儿也还懂得,听了不觉眉头一皱,抢过一步,刚刚和那女子打个照面。只见这个女子穿著一身男装衫服,却也生得眉目清秀,体态风流。一眼看见了章秋谷,嘻笑自若,没有一些惭愧的样儿,目光炯炯的把章秋谷钉了两眼,倒反握着那男子的手,迎面直走过来,和章秋谷等一干人擦肩过去。章秋谷倒噤住了口,一时说不出什么来。

看他走得远了,秋谷方才说道:“世界之上竟有这般无耻的女子,真个是无奇不有的了。”贡春树问道:“方才那女子说的一句是什么话儿?”秋谷笑道:“这个‘辟因斯’便是男子的生殖器。”大家听了都笑起来。刘仰正笑道:“你平日之间最会骂人,今天为什么不骂他几句,却像了个寒蝉噤口一般,这是什么道理?”

秋谷笑道:“骂他几句是容易。你想,这样的人岂是肯受人辱骂的?一定要惊天动地的弄得大闹起来。常言‘男女不相争’。他吊他的膀子,与我们不相干,何必去管他的闲事?况且,这样的人是不论什么话儿都说得出来的,万一个被他破口骂上几句,或者把我们牵扯几句,我们就不值得了。”春树笑道:“如此说来,你也是欺善怕恶的人。”

正说到这里,只听得后面有人叫道:“前面走的可是秋谷么?”秋谷听了,连忙回身看时,只见后面两个人急急的走上前来。两个一般的都有五十多岁年纪,鸳肩鹤背,白面乌须。秋谷仔细看时,认得不是别人,是王子渊、王子深弟兄两个,一般都是同榜的太史公。这位王子渊王太史,却是个海内的书家,真、草、隶、篆无一不会,无一不精。南北十余省,没有一个人不知道这位王太史的书法。和秋谷的老太爷是拜兄弟,为人却十分诚实,古道非常。当下秋谷见了王太史弟兄两个,忽然想起王子深王太史的事情,数年之前,曾在陈文仙院中和他相遇,两下着实顶撞过一回的。如今见了面,不觉有些不好意思起来,要想躲避,却又躲避不及,只得走过来见了他们弟兄两个。

王子渊王太史便开口说道:“我们久不通信,心上十分惦念。去年忽然听了无数的谣言,也不知是那里来的,我们两个人甚是和你气忿。到底是怎么的一回事情?

你说给我们听听。“秋谷微笑,把这件事儿的原委略略说了一遍。他们两个听了,都摩拳擦掌,十分愤激。王子深王太史便又问问秋谷近来在家里头的情形,绝不提起以前的那番话儿,意思里头甚是关切。倒是章秋谷自己觉得过意不去起来,暗想:这位王太史毕竟是个不念旧恶的好人,究竟老辈行为来得十分厚道。懊悔以前在陈文仙院中好好的不该得罪他。只得自己先开口说道:”以前小侄无知,冒犯老伯。

如今老伯虽然不念旧恶,小侄自己想起来却觉得十分颜赧。“王子深王太史听了哈哈大笑,一手拉着秋谷道:”这些小事我久已忘记的了,你又何必再去提他?“秋谷打了一拱道:”足见老伯的雅量。“王子渊王太史又道:”这里说话不便,明天我想请你去舍间吃顿便饭,不知你赏光不赏光?“秋谷忙道:”两位老伯赏饭,怎敢不到?“王子深王太史道:”你何必这般客气?明天上午,我们在舍间恭候就是了。最好请早些来,我们可以谈谈。“说着,便同着王子渊王太史别了秋谷,一同走了。

秋谷回过头来看刘仰正和贡春树时,早已不知到那里去了。叫了几声,方才听得远远的答应。秋谷连忙走过去看时,只见他们两个人立在桥上,低着头在那里看玩水中倒影的焰火。见了秋谷,便道:“你们那里来的这许多说话?直说了这半天。”

秋谷把方才的事情一一向他们说了,又把自己和王太史顶撞的事情也向他们说了一番。贡春树笑道:“这两个人,我们平日还说他是书迂;如今看起来,却是个不可多得的好人。

看了一回,秋谷觉得没有什么趣味,便要回去。刘仰正等也觉已经兴尽,便去寻着了马夫,叫他配起马车来。这个当儿,三个人偶然又走到安垲第那边去打了一个转身。只见安垲第门内走出一个中年妇人来,虽然年纪已有四十多岁,却生得蛾眉螓首,玉面朱唇,别有一种婀娜动人的姿态。见了章秋谷,含笑和他点一点头,章秋谷也向他鞠躲。正在这般时候,刺斜里又走过一个学生装束的少年男子来,和那妇人做了一个鬼脸,那妇人顿时眉花眼笑的也还他一个眼风。只说章秋谷没有看见,谁知偷转眼来一看,章秋谷的这双眼睛竟是全付精神的注在他们两个人身上。

那妇人不觉脸上红起来,一个转身,便走进安垲第去。

秋谷叹一口气道:“这个就是孙伯义孙观察的如夫人。本来是个半开门的私娼出身,手里头着实有几个钱,并且也通些文墨。自从嫁了这位孙观察之后,宠爱非常,把家事都给他掌管,那位正室夫人倒反成了赘瘤。如今附着孙观察的声誉,居然当了什么女学堂的监督。你看他到了这般的年纪,还是这般的回眸顾影,卖弄风情,那里还像个人家人的样儿!”。一面说着,马车已经来了,章秋谷等便各自登车回去。

到了明天,秋谷一早起来,坐了马车去拜了几个客。差不多九点多钟的时候,便到归仁里王公馆里。见了王太史弟兄两个,相让坐下,谈了一回,秋谷见他们十分关切,便把自己的家计也和他们弟兄说了。王子渊王太史便竭力劝他到上海来就馆,对他说道:“像你这般的才干,就个每月一二百金的馆地手到擒来,有什么难处?那时就是同了宝眷住在上海,这几个钱也就差不多了。”王子深王太史接着说道:“你若一时没有机会,总在愚弟兄两个人的身上和你推荐就是了。”秋谷听了他们这番说话,虽然不想他荐什么馆地,心上却狠有些儿感激,不免谢了几句。

文说到这一次赛珍会的事情来,王子渊王太史气忿忿的道:“好好的一个慈善会,如今弄成了一个大台基,还不如不开这个会,还觉得干净些儿。”秋谷听了道:“老伯这个意见却错了。这个赛珍会虽然被他们弄成了个大台基,却究竟那班饥民还得些实惠。”王子深王太史听了,摇一摇头道:“照你这样的说起来,这些败坏风化的举动都是应该的了?据我看来,赈济饥民的事小,不过患在一时;败坏风化的事大,却是患在久远。两下里比较起来,究竟有些轻重的分别。”秋谷道:“老伯的话自然不错,却又是只知其一,不知其二。上海这个地方本来是风俗狠坏的,就是没有这个赛珍会,依然也是这个样儿,并不是开了这个赛珍会方才败坏风化的。不开这个会,风俗未见得就会变好;开了这个会,却实实的在灾民身上有些益处。这样的比较起来,还是赈济饥民的事情来得重些。两位老伯以为何如?”王子渊和王子深两个人听了,低着头想了一想,觉得当真不错,也便点头称是。一会儿端上菜来,清清疏疏的几样,却甚是精致。座中就是主客三个,不请别人。秋谷吃了几杯酒,有了几分酒意,不觉提起满腹的牢骚来。放下酒杯,叹一口气,霍地立起身来,口中高吟道:“姮娥老大无归处,独倚银轮哭桂花!”吟罢,就不觉凄然欲涕。王子渊王太史听了,对着他兄弟叹道:“古之伤心人!”说着,又把这两句诗在口中翻来覆去的念了两遍,击节叹赏道:“好诗,好诗!”说着,又问秋谷道:“是近作么?好象这两句诗在古人诗集上没有见过。”秋谷笑道:“这两句是钱虞山的《秋兴》诗,是本朝干嘉年间禁晶,坊间没有刻本的。”王太史听了点一点头道:“他的诗你还记得不记得?可好请抄写几首出来,也好叫我们见识见识?”

秋谷听了,便向王太史索了纸笔,提起笔来,风雨一般的就写了二十余首。放下笔来道:“还有一半没有写出来,却记忆不全了。”王太史接过来,高声朗诵了一遍。

又递接他兄弟看了一遍,两个人都啧啧叹赏。秋谷道:“他这个诗都是慷慨激烈之音,觉得比平常的诗要容易见长些。”王太史兄弟都点头称是。

秋谷又吃了几杯酒,王太史见秋谷酒量不差,叫换过大杯来,又灌了秋谷几杯。

秋谷不觉有了七八分酒意。一眼看见壁上挂了一口古剑,便走过去取在手中,拔出鞘来看了一看,却是一口双剑,赞道:“这把剑虽然算不得宝剑,却也狠有些儿身分。”一面回过头来对着王太史兄弟两个说道:“小侄酒酣耳热,要大胆在两位老伯面前放肆一下,舞一回剑,和两位老伯佐饮何如?”王太史兄弟两个齐声说道:“狠好,狠好!我们正要请教。”说着,便大家立起身来。秋谷早把身上衣服略略的结束了一下,仗着双剑走到院中,慢慢的舞动起来。起初的时候,只见那剑光一闪一闪的耀得人眼光不定,还看得见人影儿。舞到后来,只见万道寒光高低驰骤,一团白气上下纵横,好似那大雪漫天,梨花乱落,看不见一些儿人影,锋芒四射,咄咄逼人。王太史看了,倒觉得有些胆寒起来。一会儿剑光一闪,用了一个金鸡独立的架势,收住剑法,露出一个人来,提着双剑走进屋中,把剑插入鞘内,面上微微的有些红影,向着王太史弟兄两个拱一拱手道:“放肆,放肆。”

王太史携了秋谷的手,仔细端详了一回,口中说道:“不想你竟有这般绝技!

不枉了我那位老友一生忠厚,如今却留下你这般一个材兼文武的佳儿。“说到这里,不由得神色凄然。秋谷听得王太史提起他老太爷来,更觉衔哀欲涕。王太史见了,恐怕提起了秋谷心中伤感,便也把几句别的话儿岔了开去。秋谷心中暗想:如今的这般世界,这样的笃于友谊的人,也总算是难得的了。这般想着,便越发的心中感动起来,不免要把他们两个恭维几句,他们也不免要谦逊一番。吃过了饭,又谈了一回,方才别去。

过了两天,张园的赛珍会已经完了,辛修甫一定要邀着秋谷到他公馆里头去住,秋谷也便答应,辛修甫便把要娶苏青青的一层情节和他说了。秋谷在常熟的时候,已经接了辛修甫的信告诉他这件事情。又听了贡春树和刘仰正与他细说,早已知道这件事儿的根由始末。如今听了辛修甫的话,故意沉吟一会道:“你当真要娶苏青青么?”辛修甫道:“自然是当真的,难道我和你说谎不成?”秋谷摇一摇头道:“你常说,将来娶妾,断不要堂子里头的倌人,怎么如今又要起倌人来?上海的倌人岂是可以娶得的么?”辛修甫道:“你常常说,真有良心的倌人,是可遇而不可求的。如今我恰恰的遇着了这个苏青青,就和你的遇着陈文仙一般。”

秋谷不等他说下去,哼了一声道:“只怕没有这般凑巧罢!”修甫道:“这个人我狠信得过他,委实的真心向我,没有什么三心两意,我可以和他出得保结的。”

秋谷哈哈的笑道:“万一个竟是假的便怎么样呢?”修甫道:“这个人我不但试过他一次,已经一连试过他三四次的了。”说着,便把几次试验的情形和秋谷说了一遍。秋谷听了,低着个头着实沉吟了一会,又细细的把那试验的情节一字不漏的问了一遍,又想了一回,方才对辛修甫笑道:“据我看起来,还是个假的。”辛修甫跳起来:“这件事情你却未免过虑了些。我这样破釜沉舟的试验他都试验不出来。

他竟肯除了牌子同着我一起往日本去,那里还有什么虚假?你们要把这个虚假的道理说给我听听。“

秋谷笑道:“你不用这般乱跳,待我慢慢的和你讲就知道了。那班堂子里头的倌人,要是给客人一试就试出原形来的,本来是个不中用的饭桶。若是有些阅历的老辈,你那里试他得出?凭你去试他的人,口中说得怎样的危险、那般的紧急,他却不问你是真是假,先把你几句迷汤灌住了,再说别的。为什么呢?你的说话就使果然是真的,这个时候也还不知道究竟怎样。果然到了那个时间,见了实在的情形,当真的要他怎样怎样起来,他再借个缘由,翻转脸来,和你做一个决绝,也还不迟。

这个时候和你说几句好话,灌几句迷汤,却是他的本等家园货,又不要花钱置买。

就是白丢掉了,也没有什么稀罕。若是你的话儿果然是假的,他就更加的‘得其所哉,得其所哉’了。你想他们那班倌人,要是听了你们这班客人的话儿,一时间就冒冒失失的翻转脸皮吵闹一阵,要万一个是假的,不但客人脸上过不去,将来这个没良心的声名传说出来,他那里还好做什么生意?你想我的话儿可是不是?“

修甫听了,想了一想道:“你的话儿却狠不错,我也狠佩服你的见识。但是这些说话,你也不过是揣度之词,没有什么实在的凭据,你又究竟怎样知道他是假的呢?”秋谷笑道:“这个狠容易明白的。你想,他既是和你恩深义重,发誓不嫁别人,听了这样至危极险、性命交关的话儿,该应二十四分的着急才是,那里还有工夫来指驳你的说话?如今,你只看他知道了这个信息,全没有一些儿张皇迫切的神情,痴一味软款缠绵的把你哄住,说了许多深恩厚爱的话儿。照这般的样儿,不是假的难道倒是真的?”

修甫听了,侧着头踌躇了好一刻,方才说道:“据你这般说来,要怎么一个样儿才是真的呢?”秋谷道:“这也不难。只要他果然除了牌子跟你到日本去,到了日本的船上,那就是真的了。”修甫叹一口气道:“这是我自家性急了些,没有隐藏到底。如今何不我们同去,请你细细的评理他一下,看看他究竟是真是假。”正是:

十年载酒,魂迷照玉之屏;一枕惊秋,梦断鲛红之被。

不知后事如何,且看下回分解。





第一百九十二回 阻星期曲房惊好梦 行酒令东阁宴嘉宾





且说辛修甫要章秋谷同到苏青青那里去,看看他的真假何如。章秋谷连忙摇手道:“如今的时候,就是我亲去试验他,也试验不出来的了。你若就是这样不问真假,糊里胡涂的把他娶了回去,便也不必去说他。若真个的要试验他的真心,我却有一个主意在这里。这个时候却不能和他见面,只要你肯割爱就是了。”辛修甫听了,不懂他是什么意思,眼睁睁的看着他。

秋谷见他不懂,便又和他说道:“你们这位贵相好,如今既然除了牌子想要嫁你,自然是不接别人的了。”修甫听了,点一点头。秋谷道:“如今的时候,要试倌人的真假,只有一个法儿。两个要好的朋友大家预先约齐了,去同做一个倌人,却只作大家不认得的一般。又故意的大家赌气吃醋,你骂我,我骂你的,听那倌人的口气怎么样。虽然堂子里头的规矩,对着姓张的照例要骂姓李的,对着姓李的又照例要骂姓张的,却是那里头的轻重情形总有些看得出来的。到了那个时候,两个人约齐了,大家当着那倌人的面前说出真情来,把那些背后的话儿,都一古脑儿讲得个明明白白。虽然计策来得毒些,却除了这个法儿,再没有第二个法儿了。”

辛修甫听了,拍手称是道:“这个主意果然来得十分挖掐。”说到这里,忽然顿了一顿道:“但是他如今是不接客人、不做生意的了,却怎样的再去试他?”秋谷微笑道:“只要你不要掀翻醋罐,我自然有个法儿去算计他。”辛修甫想了一想,奋然说道:“罢了,被你这般的一说,把我说得果然疑惑起来,只得要凭你去把他怎样的了。”秋谷道:“既然如此,明天你就和他坐马车到张园去。到了张园,你只推说有紧要的事情先要回去,那时你便坐了马车先走,只说等一会儿再打发马车来接他。到了这个时候,你就交代给我,不用管,我自然有我的法儿。”修甫叹了一口气道:“也只得如此的了。”

到了明日,果然辛修甫如法泡制的同着苏青青到张园去。进了安垲第,就在进去的地方拣张桌子,泡一碗茶。刚刚坐下,早见那位章秋谷换了一身衣服,刺斜里劈面走过来。那时四月中旬天气,章秋谷穿著一件白纺绸长衫,衬着一件玄色外国纱马褂,丰裁朗朗,仪表亭亭,翩翩潘玉之姿,濯濯王恭之度,眉稍敛意,眼角含情,面白颐丰,神清气爽。辛修甫见了,觉得眼光一动,便故意别转头去,只作没有看见。章秋谷走近身来,恰恰的和苏青青打个照面。苏青青忽然抬起头来,见了章秋谷,不由得呆了一呆。那一对秋波,就不知不觉的射到章秋谷身上来。章秋谷见了,知道有些意思,便软软的飞了一个眼风,苏青青回头一笑。秋谷又把手中的一方丝巾对着苏青青扬了一扬,苏青青把头一低。章秋谷便急急的走了过去,偷眼看辛修甫时,只见他呆着个脸儿,正把眼睛注在那边桌子上一班倌人的身上。秋谷暗想:装得狠是相像。便故意去各处兜了一趟。

慢慢的走回来,果然辛修甫已经走了,苏青青一个人坐在那里,手托香腮,呆呆的在那里出神。见了章秋谷走过来,便有意无意的瞟他一眼。章秋谷微微的笑着,索性立到苏青青对面去,上上下下的仔细打量。看得个苏青青不好意思起来,不觉“嗤”的一笑,对着秋谷把头略略的摇了一摇。秋谷索性走近一步,对着苏青青笑道:“我们两个人面熟得狠,好象是认得的。请问可是前年在西鼎丰的苏青青么?”

苏青青听了,粲然一笑道:“倪正是苏青青,格位大少贵姓?”秋谷道:“原来果然是青青先生,我的眼力果然不错。你可还记得那个时候在你房里头借干铺的章二少么?”原来章秋谷以前本来没有做过苏青青,明欺他们做倌人的张三李四,身上的客人多得狠,那里记得出来?当下苏青青听了,想了一回,想不起来,只得笑道:“二少,对勿住,隔仔几年,倪直头忘记脱格哉。”秋谷一面和他说话,一面故意把眼光只顾向他身上溜来。苏青青见了,心上甚是高兴,便指着旁边一张椅子道:“二少,耐请坐哩。”秋谷便也软绵绵的坐了下来。两个人谈了一回,谈得十分密切。秋谷一面和他讲话,那桌子底下的脚未免要不规矩起来。苏青青只是微微的笑,不说什么。

秋谷正和苏青青讲话,忽然叫了一声“呵呀”道:“我听人说,你就要恭喜嫁人,可是真的么?”苏青青斜了他一眼,并不开口。秋谷叹一口气道:“那个娶你回去的客人,也不知是那一世里修来的福气。”苏青青故意嗔道:“耐勿要来浪瞎三话四哉。”说着,把秋谷背上打了一下。秋谷趁势低低的附耳说道:“等回儿请你到一品香去,不知你肯赏光不肯赏光?”苏青青不答,只略略的点一点头。秋谷便又向苏青青耳旁说了几句,苏青青不觉脸上一红,呸了秋谷一口道:“勿要来浪像煞有介事!”一会儿,苏青青的马车来了。苏青青便立起身来,把秋谷瞟了一眼,往外便走。秋谷会意,连忙随后走出安垲第,坐上自己的马车,紧紧的跟着苏青青的马车。一路上追风逐电的跑到一品香门口停下,两个人一同下车进去。

自这一天起,章秋谷放出全付的工夫笼络那苏青青。当日晚上,就和苏青青有了交情。辛修甫得了这个信息,虽然心上有些酸气,却也无可如何,只得依着秋谷的分付。到了明天一早,便赶到永吉里来。进了永吉里的弄口转一个弯,只见一家门首写着“姑苏归公馆”的五个字儿,暗想这里是了。便一一依着秋谷的话儿,推门进去。见秋谷的车夫站在门内,见了辛修甫,把手招招,又往屏门背后一指。修甫会意,轻轻的转进屏门,走上楼去。见上首的一间房门,果然房门虚掩,便站在门外,轻轻的咳嗽一声。只听得房内也是轻轻的一声咳嗽。修甫得了秋谷的暗号,方才放大了胆一脚跨进房去。只见银钩不动,锦帐低垂,宝鸭沉沉,房栊寂寂。修甫抢进两步,揭开帐子。章秋谷已经坐起身来,见了修甫倒觉得有些不好意思起来,只把一只手指着里床。修甫举眼往床里看时,果然见一个少年女子,侧着身体向外睡着,星眸不起,宝靥微红,剩粉末销,残指犹腻,两只玉臂双双的抛在床外,一头黑发软软的堆在枕边。原来不是别人,果然就是他那位现在情人、将来爱宠的苏青青。

辛修甫见了又好笑,又好气,不由分说赶过去扯着苏青青的一只手,把他拉了起来,口中大声喝道:“你这不要脸的东西,干得好事!”苏青青正在香梦迷离、春情撩乱的时候,忽然被修甫扯了起来,又是这样的大声一喝,早把个苏青青在睡中惊醒,大吃一惊,直吓出一身香汗。连忙开眼看时,一眼光见了辛修甫对着他怒气冲冲的,口中不知在那里说些什么。又见章秋谷也在那里嘻嘻的看着他笑。这一来,只把个苏青青搅得心上胡涂起来,好象是做梦的一般。看看这个,看看那个,一句话也说不出。

修甫又向他喝道:“你已经收了我的定钱,除了牌子,怎么如今又和别人吊起膀子来?”苏青青听了还是摸不着头脑。看着章秋谷立在床前,好似没事人儿的一般。苏青青心上越发的不得明白起来,呆呆的坐在床上,一言不发。

章秋谷见了,便走过来对着苏青青打了一拱,口中说道:“一切事情都是我的不是,你不要生气。”苏青青听了这几句话儿,又见章秋谷得意扬扬的对着辛修甫只是笑,想了一想,心上方才恍然大悟,彻底澄清,知道是他们两个人串合了做弄他的。到了这个时候,凭你苏青青的脸皮再厚些儿,也由不得满面上涨得通红,低下头去。辛修甫又大声问道:“你以前和我讲的话儿是怎么讲的,如今又怎么平空的变起卦来,这是个什么道理?”苏青青听了,顿了一顿,一时回答不出,只好低着个头,嘿然不语。辛修甫冷笑道:“你装聋做哑的,难道罢了不成?”

苏青青到了这个时候,明知道事情已经决裂,心上便定了主意,挽一挽头发,跨下床来对着辛修甫道:“辛老,耐末也勿要动气,听倪好好里搭耐说。格件事体是倪自家勿好,对耐勿起。故歇事体已经弄到仔实梗格样式,也勿必再去说俚。格辰光倪搭耐两家头格闲话,赛过勿曾说,黑板浪写白字,揩脱。下转耐肯照应倪格,请到倪小地方去坐坐,请请客,碰碰和,绷绷倪场面,格是再好勿有。耐真正勿肯照应倪格,倪也叫呒说法。不过格个辰光,端午节要到快哉,倪末探脱仔牌子预备嫁人,勿做生意,故歇再要挂仔牌子做起生意来,格末真正尴尬头。”

说到这里,章秋谷不觉喝一声采道:“好得狠!这几句话儿,真是说得道地──”一句话还没有说完,早被苏青青一把拉住了道:“耐到好格,倪搭耐咦呒拨啥冤家,啥事体耐要搭倪实梗混俏?倪末总算上仔耐格当哉,耐倒底打算那哼?”说着,又走到辛修甫身畔,握着他的手,亲亲切切的说道:“辛老,倪末总算上仔别人家格当,对耐勿起。耐也勿作兴格嘘!耐自家想想看,阿有点心浪意勿过?上海滩浪好好里格人家人,上别人当格多熬来浪,勿要说啥堂子里向格倌人哉。倪老实搭耐说仔,故歇辰光倪就懊悔勿转格哉。不过嫁人是嫁人,要好是要好,嫁人格事体勿成功,倪两家头要好是呒啥勿成功嘛。”

辛修甫起先只说苏青青一定要扭结固结的和他不肯开交,预备着许多决绝的话儿,要燥燥他的脾。不想苏青青不等他开口,先自大大方方的讲出这样一番说话来,心上也暗暗的赞他,倒不好再说什么。如今又听了这几句话儿,只觉得心上非但并不恨他,倒像觉得自己真个有些不是的一般。推开了苏青青的手,微微笑道:“算了罢,不用再提了。我们从此不提今天的事情。”苏青青回过身来,指着秋谷,把金莲在地下一顿道:“才是耐勿好!”秋谷不去理会他的话儿,却对着他把一个大指一伸道:“真正利害,不愧是个头等名角!”

苏青青想了一想,倒笑起来,口中说道:“唔笃格两个人,直头是少有出见格,阿有啥两家头串通仔合着一只靴子。”苏青青说到这里,面上也红了一红,顿住了口不说下去。章秋谷和辛修甫听了,都笑起来。秋谷笑着走过去,拍一拍苏青青的肩头道:“这样说起来,你这个靴子定是内城定造的上等京靴了。”苏青青听了,忍不住“扑嗤”一笑。自此以后,苏青青要嫁辛修甫的这件事儿,虽然被这位章秋谷平空打散,辛修甫同着章秋谷两个却依然在他院中走动。

一言表过不提。只说章秋谷在上海住了几天,把汇丰银行里头的存款,果然一古脑儿提了出来,回到常熟去,存在一个大昌当铺里头。把家事布置了一番,便又到上海来。原来辛修甫见章秋谷到了上海,便再三再四的邀他仍到书局里头去,章秋谷便也答应。此番再到上海,却和以前在上海的时候大不相同,陆丽娟和梁绿珠都不知到那里去了,习凿齿再到襄阳,桓司马重来灞水,摇落江潭之柳,凄凉湘水之波,狠有些儿沧海桑田的感慨。更兼看着自己这般境遇,桩萱凋谢,朋旧销沉,十年湖海之游,一霎邯郸之梦,司勋落魄,阮籍猖狂,感身世之无聊,抚头颅之如许,便不知不觉的郁郁不乐,黯然神伤。

就是这样的过了几个月。忽然东方小松从广东解饷回来,一到上海,便先去看章秋谷。章秋谷见了方小松,不觉心中大喜。良朋久别,知己重逢,自然有一番款曲。两个人畅叙了十多天。方小松见秋谷郁郁不快,怀着一肚子的牢骚,便劝他同到广东去顽一趟。秋谷也为着广东地方是个最先通商的口岸,又是南洋群岛的门户,本来心上狠想去游历一趟。听了方小松邀他同去,心上十分高兴,便一口答应。又和辛修甫说了要告几个月假到广东去。辛修甫挽留不住,只得由他自去。章秋谷又荐了贡春树暂时代理书局里头的事情,自己便同着方小松到广东来。

到了广东地方,休息了几天,方小松备酒和他接风。席间的陪客除了几个同乡候补官之外,有一个实缺潮州府知府程梅谷程太守,现充法政学堂监督,是个进士出身,和方小松是极要好的朋友。久已听得方小松说起这位章秋谷先生的大名,和秋谷谈得十分合式。秋谷看了这位程太守生得丰裁出众,气概非常,两只眼睛炯炯的光芒直射,知道不是个寻常人物,便也肃然起敬。

到了明天,程太守便托了方小松致意,要请章秋谷当个总教习。章秋谷起先不肯,只说我是到这里来游历一下的,至多不过几个月的勾留,何必多此一举。当不起程太守再三再四的敦请,方小松又劝他道:“你就借着这个机会到学界里头去阅历一下也好。到了要回去的时候,你只顾辞了馆地回去,他也决不能勉强留你。”

秋谷听了,一想不错,便也点头答应。自此以后。秋谷便把行李搬到法政学堂去,每天三四点钟的课程倒也不觉得辛苦。

这一天,秋谷方才完了课程,正要想到方小松那里去,忽然家人传进一个帖子,说水师提督黎绳甫黎军门来拜。秋谷听了,心上觉得诧异。接过帖子来看了一看,心上想道:“这位黎军门听说在广东声名狠好,虽然和我同乡,曾有一面之识,却向来没有什么来往,怎么忽然纡尊降贵的拜起我来?这是什么原故?”想着,便叫那家人出去请黎军门在花厅上坐,自己换了衣服,立刻出来见了那位黎军门,不免大家要说几句套话。

原来这位黎军门知道章秋谷是个江南名士,所以先来拜会。章秋谷一面和黎军门说话,一面细细的打量这位黎军门时,只见这位黎军门生得虎头燕额,猿背狼腰,声若洪钟,目如闪电,真是个桓桓名将,矫矫虎臣。那谈吐举止,更是高华名贵,俊雅无俦。秋谷看了,心上暗暗的赞叹。更兼这位黎军门没有一些儿官场里头的习气,也不摆什么架子,和秋谷谈了一回,觉得甚是契合。直谈了一点多钟,方才走了。隔了一天,秋谷少不得要去回拜。黎军门接着,又谈了好一回,便约秋谷明天在他衙门里头吃饭,秋谷应了别去。

到了明天,差不多十点钟还没有到,黎军门便来催请。秋谷到了那里看时,见方小松也在坐中,其余的客也都是些素来相识的同乡。一个姓杨的杨安之,也是个江南名士,书画俱精,却是黎军门那里的文案。有两个姓江的,却是同胞兄弟,一个叫江伯临,一个叫江仲吉,都是广东候补知府,也都少年英俊,倜傥不群。还有一个姓陆的陆善卿,也是江苏人。只有一个姓戚的戚珍三,却是个四川人。当下大众寒暄了一阵,相让坐下。黎军门讲起他自己平生的战绩来,如何如何的冲锋打仗,如何如何的运筹克敌。讲到紧要的时候,讲得意气飞扬,须眉欲动。大家都不觉叹羡一回,黎军门也谦逊几句。

一会儿酒菜排齐,大家入席。黎军门的厨夫是广东全省第一个烹调名手,烹调出来的肴馔十分精致。大家吃着,一个个都赞赏不置。

一会儿酒过三巡,食供五套,江仲吉便道:“闷酒无味,我们何不行过酒令消遣呢?”秋谷道:“我的性情素来不爱行什么酒令。你想好好的吃酒,何必要来呕什么心血,绞什么脑汁?还是拇战觉得爽快些儿。”说着,黎军门点头称是。大家拇战了一回。江仲吉定要行令,便行了一回席上生风的射覆,大家吃了几杯酒。

黎军门道:“我们如今把射覆的字儿分作上下两截,须要依着上下的次序,不准颠倒,还觉得耐些寻味。”大家听了,都点头称是。方小松便说一个“布”字、一个“沙”字。杨安之想了一回,一眼看见江伯临面前有一盘彩蛋,心上便明白了,便射了一个底下的“达”字。方小松点一点头,大家一笑。戚珍三和陆善卿听了,不懂他们说些什么,便问道:“你们覆的覆,射的射,可好讲给我们听听么?”方小松道:“我是把一个‘蛋’字分作两截,一个‘疋’字,一个‘虫’字,上面的‘布’字是布疋,下面的‘虫’字是虫沙,他射的下面一个‘达’字,是虫达,汉高祖功臣中之一。”说到这里,江仲吉便道:“我给一个你射,看你射得着射不着我的上下两个字儿,就是那京戏《翠屏山》里头‘杀山’两个字儿。”方小松听了想了一回,却想不出。江仲吉道:“你吃一杯酒,我和你说了罢。”方小松果然干了一杯。江仲吉把手指着案上一盘芥酱道:“上面是霜华杀草的‘杀草’两个字,下面是‘介山’两个字,是个‘芥’字。”方小松听了,便忙忙斟了两杯酒,放在江仲吉面前道:“你先吃了我一杯酒,再罚了一杯酒,我再和你讲话。”江仲吉那里肯吃,嚷道:“难道我这个覆得错了么?你先讲出我的错处来,我再吃酒不迟。”

方小松道:“你这个‘杀草’的两个字虽然的可以用得,但是这个‘芥’字拆了开来,上面的草头不是成字的。我早已想到这个‘芥’字,为着不妥当,所以没有说出来。快快的把这两杯酒给我吃下去!”江仲吉起先还不肯吃,只说:“这个草字头是‘草’字的古体。”小松道:“我们是在这里射覆,不是在这里考据古学。你抬出古体字来也不中用。”江仲吉说他不过,只得一口气把两杯酒灌了下去。第三个就轮着章秋谷。秋谷却低着头,好似想什么心思一般。直至小松叫他,方才抬起头来,随口说了一下,却被黎军门射着。接着,大家都轮了一次。

杨安之道:“这个令也没趣得狠。”秋谷道:“你们要行有趣的酒令,我倒带着一付酒筹在这里。本来是一个朋友托我作的,后来这个人到关东去了。这付酒筹刚刚带在这里,行起来却狠有些味儿。”众人听了,便问是什么酒筹。秋谷道:“这付筹上都刻着《石头记》的人名,下面刻着四六评话,应贺应罚,也都注在上面。”众人听了都大喜道:“你快去取来,我们行个新酒令也好。”秋谷听了,便叫家人回去,把箱子里头的一付竹筹立刻取来。

家人去不多时,果然取来送上。大家争着看时,只见一个大大的竹筒,装着满满的一筒竹筹,虽然是竹的,却雕得十分工致。众人要去拔出筹来看时,秋谷拦住道:“预先看过了没有什么趣味,我们慢慢的抽就是了。只是你们既要行这个令,却要推我做个令官,大家都听我的号令行事。”众人道:“这个自然。”秋谷便把这个竹筒放在中间,口中便道:“我是令官,该应自令官左首的人行起。”

方小松正坐在秋谷左首,便揎拳掳袖的掣了一枝出来,口中说道:“要掣一个好的,不要受罚才好!”大家争着看只见筹上刻着几行字道:

史湘云豪情弱质,侠骨柔肠,楚山缥缈之云,湘水潆洄之恨。玉山颓倒,香留芍药之茵;宝月温存,春入衡芜之梦。得史湘云者,合席皆贺两杯,自饮两杯。量洪者与湘云对饮一杯。如座有宝玉,宝玉应为湘云斟酒;除贺酒外,再与湘云对饮一杯。遇宝钗、黛玉,与湘云对饮一杯。

秋谷看了笑道:“你抽着了史湘云,却没有什么累赘,不过吃几杯酒就是了。”

方小松道:“这个时候横竖没有宝玉在这里,我吃过了三杯令就是了。”秋谷连忙道:“这个不能,要等大家抽齐了才算的。如若不然,那先抽的人岂不是占了便宜,迟抽的人岂不是吃了亏么?”大家听了,都点一点头。

第二个便是杨安之,也抽出一枝筹来。众人大家看时,只见刻着道:

薛蝌 千里京华,三年荆棘。花空散雨,絮不沾泥。裙布钗荆,宜室宜家之梦;吹箫引凤,式金式玉之音。 得薛蝌者,合席皆贺一杯,自饮一杯。遇薛蟠,亦与薛蝌对饮一杯。如座中有夏金桂,作怒容,不饮。

第三个便是戚珍三,恰恰掣着了薛蟠,上面刻着道:

霸王雅号,壮士雄风。河东之狮吼无常,郭外之南风不竞。貂裘走马,章台杨柳之云;鸳锦缠头,绮阁湘桃之月。得薛蟠者,合席不贺,自饮一杯。惧内者与薛蟠对饮一杯。遇宝钗、宝玉,对饮一杯。遇夏金桂,当低眉承睫,亲敬三杯,薛蟠自陪一杯。如遇柳湘莲,应饮酱油一杯,并受打三拳。

戚珍三道:“这个虽然累赘,只要座中没有柳湘莲、夏金桂就是了。但是这个吃的一杯酱油,是个什么道理?”秋谷笑道:“这个酱油,是那苇根下泥水的替代品,你难道不知道么?”众人都哄然笑起来,都说这个替代品想得狠好。

第四个就是主人黎军门,伸手掣了一枝筹出来。戚珍三一眼看见,便嚷道:“完了!完了!”众人大家连忙看时,原来奇巧不奇巧的,黎军门刚刚掣着了柳湘莲,众人都不觉哈哈大笑。只见上面刻着道:

酒人唐突,怒挥子路之拳;凤女离魂,愁洒荀郎之泪。高情照日,侠气凌云。

万金宝刃,纵横秋水之光;满马春愁,撩乱绣鞍之影。 得柳湘莲者,合席皆贺两杯,自饮一杯。习武者与湘莲对饮一杯。遇宝玉、秦钟,对饮一杯。遇尤三姐,受罚一杯。主

黎军门看了笑道:“这倒很爽快。”

第五个便是陆善卿,刚刚掣了一枝出来,自己一看,便“呸”了一口,要仍旧放进筒去。早被黎军门一把抢了过来,大家看了一看,不觉又笑起来。原来这个陆善卿刚刚掣着了个夏金桂,上面刻的按语道:

香囊叩叩,未销真个之魂;鸳梦沉沉,推出窗前之月。芳心无主,春色难销。

熏衣理鬓,长窥宋玉之墙;撩雨拨云,愿作陈平之嫂。 得夏金桂者,合席不饮,夏金桂受罚一杯。有外遇者,与金桂对饮一杯。遇薛蟠者,作怒容,嘿饮三杯。遇宝玉,作媚态,对饮一杯。遇薛蝌,作媚态,牵衣握手,亲敬三杯,薛蝌不饮,金桂作眉语自饮。

大家看了,都笑道:“这个令儿狠有趣味,今天我们倒要看看陆善翁的媚态如何?”陆善卿和戚三珍都发急道:“怎么今天这个令儿专专的和我们两个人作对?

这是个什么道理?“大家听了,又笑个不住。

第六、第七就是江伯临、江仲吉兄弟两个。江伯临掣着了李绮,是大家公贺一杯,自饮一杯。遇李纨、李纹、邢岫烟、薛宝琴,各对饮一杯。江仲吉掣着了柳五儿,是大家公贺一杯,自饮一杯。遇宝玉、芳官,对饮一杯。遇林之孝家的,当受罚一杯,俯首低眉,安坐不动。江仲吉看了笑道:“只要巴着章秋谷不是林之孝家的,我就不怕了。”

临了儿,秋谷吃了一杯令酒,伸手掣了一枝出来。大家看时,只见刻着道:

探春 轻盈二八,正当瓜字之年;霹雳一声,飞出巨灵之掌。明明如月,婉婉当春。东风红杏,移来上苑之花;凤阁鸾台,嫁得金龟之婿。得探春者,公贺两杯,自饮一杯。有功名者,与探春对饮一杯。官至一二品者,与探春对饮合卺双杯。遇宝玉、宝钗、黛玉,对饮一杯。

秋谷看了笑道:“这真真是作法自毙了。”座中的几个客人,刚刚的都是广东候补官,黎军门又恰恰是水师提督,秩居一品。秋谷只得和众人对饮一杯,又和黎军门对饮两杯,笑道:“这个令官吃亏得狠。”

秋谷过了令,便是方小松的史湘云,座中止有章秋谷和黎军门两个酒量大些,便三个人大家照了一杯。又轮着杨安之的薛蝌,大家公推黎军门和方小松两个是有贤内助的,两个人便吃了一杯。第四个戚珍三的薛蟠,大家说杨安之和江伯临有些惧内,要他们两个人吃酒。他们不肯吃,便也只得罢了。秋谷便拿起席上的酱油碟子来,倒了满满的一酒杯要戚珍三吃。大家都望着他笑。戚珍三皱着眉头勉强吃一口,几乎要吐出来,便道:“我情愿多罚几杯酒罢,这酱油委实的难吃。”大家听了,又都笑起来。秋谷那里肯依,道:“酒令严如军令。你一个人不遵令,别人就都要不服令官的号令了。”戚珍三没奈何,只得咽着气,把一杯酱油吃了下去,众人看着笑个不住。第五个黎军门的柳湘莲,习武的人止有秋谷一个,便吃了一杯。

黎军门又走过去,把戚珍三背上轻轻的打了三下。第六个轮着陆善卿的夏金桂,大家都知道章秋谷和杨安之、方小松三个都是有外遇的,派着他们都吃了一杯。戚珍三便走过来,恭恭敬敬的敬了陆善卿三杯。陆善卿笑了一笑,被章秋谷罚了一杯,说要作怒容,不准嘻笑。戚珍三的酒敬过了,便该陆善卿去敬杨安之。陆善卿作难了一回,知道强不过去,只得斟了三杯酒,笑盈盈的走到杨安之身旁,拉着他的手,把酒杯放在杨安之唇边。杨安之果然作出怒容,推开不饮。陆善卿又把第二杯酒送过来,斜着眼睛钉了他一眼。杨安之只不开口,坐着不动。陆善卿便取过酒杯,刚要吃时,秋谷在旁说道:“你这个眉语要好好的做,做得不好是要罚的。”陆善卿把双眉一动,望着杨安之把眼睛飞了一转。秋谷看了,不觉喝一声彩,大家也都叫起好来。

这一席酒只吃到日色平西,这个酒令直行了四五转,行出许多笑话来。大觉都十分高兴,尽欢而散。章秋谷同着方小松一同回去,方小松便问他道:“你既然不爱酒令,为什么今天这般高兴起来。”秋谷笑道:“这里却有一个道理,万一个将来有人把我们的事情编成小说,这个酒令的一门却是少不得的。我不过和那做书的人预备一个地位罢了。”

隔了几天,又有几个同乡公请章秋谷在紫洞艇上和他接风。这个紫洞艇差不多就是西湖的游船一般,里面却是一色紫榆嵌螺甸的桌椅,锦帏绣幔布置得簇簇生新。

又叫了许多广东本地倌人和几个外省马班子里头的姑娘前来陪酒。秋谷看那些广东倌人时,只见一个个都是宽衣博袖,大脚花鞋,面上搽得雪白的一脸铅粉,连嘴唇都搽得白了,却没有一些儿胭脂,好象《三上吊》里头的缢鬼一般;更兼体态生硬,身段倔强,见了人理都不理。秋谷见了,把舌头伸了一伸。又看那班马班子的姑娘时,见虽然有一两个略略生得好些,却没有一些儿身段架子,比起上海的倌人,大不相同。正是:

烟波万重,苍茫海上之槎;风月清宵,惆怅江南之客。

自此以后,章秋谷便暂住在广东。还有些广东的官场笑柄、嫖界奇闻,在下做书的也来不及一一登载,这部《九尾龟》小说,却就在这里算个总结的了。

【完】








\backmatter

\end{document}