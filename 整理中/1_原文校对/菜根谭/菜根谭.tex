% 菜根谭
% 菜根谭.tex

\documentclass[12pt,UTF8]{ctexbook}

% 设置纸张信息。
\usepackage[a4paper,twoside]{geometry}
\geometry{
	left=25mm,
	right=25mm,
	bottom=25.4mm,
	bindingoffset=10mm
}

% 设置字体,并解决显示难检字问题。
\xeCJKsetup{AutoFallBack=true}
\setCJKmainfont{SimSun}[BoldFont=SimHei, ItalicFont=KaiTi, FallBack=SimSun-ExtB]

% 目录 chapter 级别加点(.)。
\usepackage{titletoc}
\titlecontents{chapter}[0pt]{\vspace{3mm}\bf\addvspace{2pt}\filright}{\contentspush{\thecontentslabel\hspace{0.8em}}}{}{\titlerule*[8pt]{.}\contentspage}

% 设置 part 和 chapter 标题格式。
\ctexset{
	chapter/name={},
	chapter/number={}
}

% 图片相关设置。
\usepackage{graphicx}
\graphicspath{{Images/}}

% 设置古文原文格式。
\newenvironment{yuanwen}{\bfseries\zihao{4}}

% 设置署名格式。
\newenvironment{shuming}{\hfill\bfseries\zihao{4}}

% 注脚每页重新编号,避免编号过大。
\usepackage[perpage]{footmisc}

\title{\heiti\zihao{0} 菜根谭}
\author{洪应明}
\date{}

\begin{document}

\maketitle
\tableofcontents

\frontmatter
\chapter{前言}



\mainmatter

\chapter{修身}

\begin{yuanwen}

\end{yuanwen}


欲做精金美玉的人品,定从烈火中煅来;思立掀天揭地的事功,须向薄冰上履过。
一念错,便觉百行皆非,防之当如渡海浮囊,勿容一针之罅漏;万善全,始得一生无愧。修之当如凌云宝树,须假众木以撑持。
忙处事为,常向闲中先检点,过举自稀。动时念想,预从静里密操持,非心自息。
为善而欲自高胜人,施恩而欲要名结好,修业而欲惊世骇俗,植节而欲标异见奇,此皆是善念中戈矛,理路上荆棘,最易夹带,最难拔除者也。须是涤尽渣滓,斩绝萌芽,才见本来真体。
能轻富贵,不能轻一轻富贵之心;能重名义,又复重一重名义之念。是事境之尘氛未扫,而心境之芥蒂未忘。此处拔除不净,恐石去而草复生矣。
纷扰固溺志之场,而枯寂亦槁心之地。故学者当栖心元默,以宁吾真体。亦当适志恬愉,以养吾圆机。
昨日之非不可留,留之则根烬复萌,而尘情终累乎理趣;今日之是不可执,执之则渣滓未化,而理趣反转为欲根。
无事便思有闲杂念想否。有事便思有粗浮意气否。得意便思有骄矜辞色否。失意便思有怨望情怀否。时时检点,到得从多入少、从有入无处,才是学问的真消息。
士人有百折不回之真心,才有万变不穷之妙用。立业建功,事事要从实地着脚,若少慕声闻,便成伪果;讲道修德,念念要从虚处立基,若稍计功效,便落尘情。
身不宜忙,而忙于闲暇之时,亦可儆惕惰气;心不可放,而放于收摄之后,亦可鼓畅天机。
钟鼓体虚,为声闻而招击撞;麋鹿性逸,因豢养而受羁縻。可见名为招祸之本,欲乃散志之媒。学者不可不力为扫除也。
一念常惺,才避去神弓鬼矢;纤尘不染,方解开地网天罗。
一点不忍的念头,是生民生物之根芽;一段不为的气节,是撑天撑地之柱石。故君子于一虫一蚁不忍伤残,一缕一丝勿容贪冒,便可为万物立命、天地立心矣。
拨开世上尘氛,胸中自无火焰冰竞;消却心中鄙吝,眼前时有月到风来。
学者动静殊操、喧寂异趣,还是锻炼未熟,心神混淆故耳。须是操存涵养,定云止水中,有鸢飞鱼跃的景象;风狂雨骤处,有波恬浪静的风光,才见处一化齐之妙。
心是一颗明珠。以物欲障蔽之,犹明珠而混以泥沙,其洗涤犹易;以情识衬贴之,犹明珠而饰以银黄,其洗涤最难。故学者不患垢病,而患洁病之难治;不畏事障,而畏理障之难除。
躯壳的我要看得破,则万有皆空而其心常虚,虚则义理来居;性命的我要认得真,则万理皆备而其心常实,实则物欲不入。
面上扫开十层甲,眉目才无可憎;胸中涤去数斗尘,语言方觉有味。
完得心上之本来,方可言了心;尽得世间之常道,才堪论出世。
我果为洪炉大冶,何患顽金钝铁之不可陶熔。我果为巨海长江,何患横流污渎之不能容纳。
白日欺人,难逃清夜之鬼报;红颜失志,空贻皓首之悲伤。
以积货财之心积学问,以求功名之念求道德,以爱妻子之心爱父母,以保爵位之策保国家,出此入彼,念虑只差毫末,而超凡入圣,人品且判星渊矣。人胡不猛然转念哉!
立百福之基,只在一念慈祥;开万善之门,无如寸心挹损。
塞得物欲之路,才堪辟道义之门;弛得尘俗之肩,方可挑圣贤之担。
容得性情上偏私,便是一大学问;消得家庭内嫌雪,才为火内栽莲。

事理因人言而悟者,有悟还有迷,总不如自悟之了了;意兴从外境而得者,有得还有失,总不如自得之休休。
情之同处即为性,舍情则性不可见,欲之公处即为理,舍欲则理不可明。故君子不能灭情,惟事平情而已;不能绝欲,惟期寡欲而已。
欲遇变而无仓忙,须向常时念念守得定;欲临死而无贪恋,须向生时事事看得轻。
一念过差,足丧生平之善;终身检饬,难盖一事之愆。
从五更枕席上参勘心体,气未动,情未萌,才见本来面目;向三时饮食中谙练世味,浓不欣,淡不厌,方为切实工夫。

\begin{yuanwen}

\end{yuanwen}\begin{yuanwen}

\end{yuanwen}\begin{yuanwen}

\end{yuanwen}\begin{yuanwen}

\end{yuanwen}\begin{yuanwen}

\end{yuanwen}\begin{yuanwen}

\end{yuanwen}\begin{yuanwen}

\end{yuanwen}\begin{yuanwen}

\end{yuanwen}\begin{yuanwen}

\end{yuanwen}\begin{yuanwen}

\end{yuanwen}\begin{yuanwen}

\end{yuanwen}\begin{yuanwen}

\end{yuanwen}\begin{yuanwen}

\end{yuanwen}\begin{yuanwen}

\end{yuanwen}\begin{yuanwen}

\end{yuanwen}\begin{yuanwen}

\end{yuanwen}\begin{yuanwen}

\end{yuanwen}\begin{yuanwen}

\end{yuanwen}\begin{yuanwen}

\end{yuanwen}\begin{yuanwen}

\end{yuanwen}\begin{yuanwen}

\end{yuanwen}\begin{yuanwen}

\end{yuanwen}\begin{yuanwen}

\end{yuanwen}\begin{yuanwen}

\end{yuanwen}\begin{yuanwen}

\end{yuanwen}\begin{yuanwen}

\end{yuanwen}\begin{yuanwen}

\end{yuanwen}\begin{yuanwen}

\end{yuanwen}\begin{yuanwen}

\end{yuanwen}\begin{yuanwen}

\end{yuanwen}\begin{yuanwen}

\end{yuanwen}\begin{yuanwen}

\end{yuanwen}\begin{yuanwen}

\end{yuanwen}\begin{yuanwen}

\end{yuanwen}\begin{yuanwen}

\end{yuanwen}\begin{yuanwen}

\end{yuanwen}\begin{yuanwen}

\end{yuanwen}\begin{yuanwen}

\end{yuanwen}\begin{yuanwen}

\end{yuanwen}\begin{yuanwen}

\end{yuanwen}\begin{yuanwen}

\end{yuanwen}\begin{yuanwen}

\end{yuanwen}\begin{yuanwen}

\end{yuanwen}

\chapter{应酬}

操存要有真宰,无真宰则遇事便倒,何以植顶天立地之砥柱!应用要有圆机,无圆机则触物有碍,何以成旋乾转坤之经纶!
士君子之涉世,于人不可轻为喜怒,喜怒轻,则心腹肝胆皆为人所窥;于物不可重为爱憎,爱憎重,则意气精神悉为物所制。
倚高才而玩世,背后须防射影之虫;饰厚貌以欺人,面前恐有照胆之镜。
心体澄彻,常在明镜止水之中,则天下自无可厌之事;意气和平,常在丽日光风之内,则天下自无可恶之人。
当是非邪正之交,不可少迁就,少迁就则失从违之正;值利害得失之会,不可太分明,太分明则起趋避之私。
苍蝇附骥,捷则捷矣,难辞处后之羞;萝茑依松,高则高矣,未免仰攀之耻。所以君子宁以风霜自挟,毋为鱼鸟亲人。
好丑心太明,则物不契;贤愚心太明,则人不亲。士君子须是内精明而外浑厚,使好丑两得其平,贤愚共受其益,才是生成的德量。
伺察以为明者,常因明而生暗,故君子以恬养智;奋迅以为速者,多因速度而致迟,故君子以重持轻。士君子济人利物,宜居其实,不宜居其名,居其名则德损;士大夫忧国为民,当有其心,不当有其语,有其语则毁来。
遇大事矜持者,小事必纵弛;处明庭检饰者,暗室必放逸。君子只是一个念头持到底,自然临小事如临大敌,坐密室若坐通衢。
使人有面前之誉,不若使其无背后之毁;使人有乍交之欢,不若使其无久处之厌。
善启迪人心者,当因其所明而渐通之,毋强开其所闭;善移风化者,当因其所易而渐及之,毋轻矫其所难。
彩笔描空,笔不落色,而空亦不受染;利刀割水,刀不损锷,而水亦不留痕。得此意以持身涉世,感与应俱适,心与境两忘矣。
己之情欲不可纵,当用逆之之法以制之,其道只在一忍字;人之情欲不可拂,当用顺之之法以调之,其道只在一恕字。今人皆恕以适己而忍以制人,毋乃不可乎!
好察非明,能察能不察之谓明;必胜非勇,能胜能不胜之谓勇。
随时之内善救时,若和风之消酷暑;混俗之中能脱俗,似淡月之映轻云。
思入世而有为者,须先领得世外风光,否则无以脱垢浊之尘缘;思出世而无染者,须先谙尽世中滋味。否则无以持空寂之后苦趣。
与人者,与其易疏于终,不若难亲于始;御事者,与其巧持于后,不若拙守于前。
酷烈之祸,多起于玩忽之人;盛满之功,常败于细微之事。故语云﹕"人人道好,须防一人着脑;事事有功,须防一事不终。"
功名富贵,直从灭处观究竟,则贪恋自轻;横逆困穷,直从起处究由来,则怨尤自息。
宇宙内事要力担当,又要善摆脱。不担当,则无经世之事业;不摆脱,则无出世之襟期。
待人而留有余,不尽之恩礼,则可以维系无厌之人心;御事而留有余,不尽之才智,则可以提防不测之事变。
了心自了事,犹根拔而草不生;逃世不逃名,似膻存而蚋仍集。
仇边之弩易避,而恩里之戈难防;苦时之坎易逃,而乐处之阱难脱。

膻秽则蝇蚋丛嘬,芳馨则蜂蝶交侵。故君子不作垢业,亦不立芳名。只是元气浑然,圭角不露,便是持身涉世一安乐窝也。
从静中观物动,向闲处看人忙,才得超尘脱俗的趣味;遇忙处会偷闲,处闹中能取静,便是安身立命的工夫。
邀千百人之欢,不如释一人之怨;希千百事之荣,不如免一事之丑。
落落者,难合亦难分;欣欣者,易亲亦易散。是以君子宁以刚方见惮,毋以媚悦取容。
意气与天下相期,如春风之鼓畅庶类,不宜存半点隔阂之形;肝胆与天下相照,似秋月之洞彻群品,不可作一毫暧昧之状。
仕途虽赫奕,常思林下的风味,则权且之念自轻;世途虽纷华,常思泉下的光景,则利欲之心自淡。
鸿未至先援弓,兔已亡再呼矢,总非当机作用;风息时休起浪,岸到处便离船,才是了手工夫。
从热闹场中出几句清冷言语,便扫除无限杀机;向寒微路上用一点赤热心肠,自培植许多生意。
随缘便是遣缘,似舞蝶与飞花共适;顺事自然无事,若满月偕盂水同圆。
淡泊之守,须从浓艳场中试来;镇定之操,还向纷纭境上勘过。不然操持未定,应用未圆,恐一临机登坛,而上品禅师又成一下品俗士矣。
廉所以戒贪。我果不贪,又何必标一廉名,以来贪夫之侧目。让所以戒争。我果不争,又何必立一让的,以致暴客之弯弓。
无事常如有事时,提防才可以弥意外之变;有事常如无事时,镇定方可以消局中之危。
处世而欲人感恩,便为敛怨之道;遇事而为人除害,即是导利之机。
持身如泰山九鼎凝然不动,则愆尤自少;应事若流水落花悠然而逝,则趣味常多。
君子严如介石而畏其难亲,鲜不以明珠为怪物而起按剑之心;小人滑如脂膏而喜其易合,鲜不以毒螫为甘饴而纵染指之欲。
遇事只一味镇定从容,纵纷若乱丝,终当就绪;待人无半毫矫伪欺隐,虽狡如山鬼,亦自献诚。
肝肠煦若春风,虽囊乏一文,还怜茕独;气骨清如秋水,纵家徒四壁,终傲王公。
讨了人事的便宜,必受天道的亏;贪了世味的滋益,必招性分的损。涉世者宜蕃择之,慎毋贪黄雀而坠深井,舍隋珠而弹飞禽也。费千金而结纳贤豪,孰若倾半瓢之粟,以济饥饿之人;构千楹而招来宾客,孰若葺数椽之茅,以庇孤寒之士。
解斗者助之以威,则怒气自平;惩贪者济之以欲,则利心反淡。所谓因其势而利导之,亦救时应变一权宜法也。
市恩不如报德之为厚。雪忿不若忍耻之为高。要誉不如逃名之为适。矫情不若直节之为真。
救既败之事者,如驭临崖之马,休轻策一鞭;图垂成之功者,如挽上滩之舟,莫少停一棹。
先达笑弹冠,休向侯门轻曳裾;相知犹按剑,莫从世路暗投珠。
杨修之躯见杀于曹操,以露己之长也;韦诞之墓见伐于钟繇,以秘己之美也。故哲士多匿采以韬光,至人常逊美而公善。
少年的人,不患其不奋迅,常患奋迅而成卤莽,故当抑其躁心;老成的人,不患其不持重,常患以持重而成退缩,故当振其惰气。
望重缙绅,怎似寒微之颂德。朋来海宇,何如骨肉之孚心。
舌存常见齿亡,刚强终不胜柔弱;户朽未闻枢蠹,偏执岂能及圆融。

\chapter{评议}


物莫大于天地日月,而子美云﹕“日月笼中鸟,乾坤水上萍。”事莫大于揖逊征诛,而康节云﹕“唐虞揖逊三杯酒,汤武征诛一局棋。”人能以此胸襟眼界吞吐六合,上下千古,事来如沤生大海,事去如影灭长空,自经纶万变而不动一尘矣。
君子好名,便起欺人之念;小人好名,犹怀畏人之心。故人而皆好名,则开诈善之门。使人而不好名,则绝为善之路。此讥好名者,当严责君子,不当过求于小人也。
大恶多从柔处伏,哲士须防绵里之针;深仇常自爱中来,达人宜远刀头之蜜。
持身涉世,不可随境而迁。须是大火流金而清风穆然,严霜杀物而和气蔼然,阴霾翳空而慧日朗然,洪涛倒海而坻柱屹然,方是宇宙内的真人品。爱是万缘之根,当知割舍。识是众欲之本,要力扫除。
作人要脱俗,不可存一矫俗之心;应世要随时,不可起一趋时之念。
宁有求全之毁,不可有过情之誉;宁有无妄之灾,不可有非分之福。
毁人者不美,而受人毁者遭一番讪谤便加一番修省,可释回而增美;欺人者非福,而受人欺者遇一番横逆便长一番器宇,可以转祸而为福。
梦里悬金佩玉,事事逼真,睡去虽真觉后假;闲中演偈谈元,言言酷似,说来虽是用时非。
天欲祸人,必先以微福骄之,所以福来不必喜,要看他会受;天欲福人,必先以微祸儆之,所以祸来不必忧,要看他会救。
荣与辱共蒂,厌辱何须求荣;生与死同根,贪生不必畏死。
作人只是一味率真,踪迹虽隐还显;存心若有半毫未净,事为虽公亦私。
鹩占一枝,反笑鹏心奢侈;兔营三窟,转嗤鹤垒高危。智小者不可以谋大,趣卑者不可与谈高。信然矣!
贫贱骄人,虽涉虚骄,还有几分侠气;英雄欺世,纵似挥霍,全没半点真心。
糟糠不为彘肥,何事偏贪钩下饵;锦绮岂因牺贵,谁人能解笼中囵囮。
琴书诗画,达士以之养性灵,而庸夫徒赏其迹象;山川云物,高人以之助学识,而俗子徒玩其光华。可见事物无定品,随人识见以为高下。故读书穷理,要以识趣为先。
姜女不尚铅华,似疏梅之映淡月;禅师不落空寂,若碧沼之吐青莲。
廉官多无后,以其太清也;痴人每多福,以其近厚也。故君子虽重廉介,不可无含垢纳污之雅量。虽戒痴顽,亦不必有察渊洗垢之精明。
密则神气拘逼,疏则天真烂漫,此岂独诗文之工拙从此分哉!吾见周密之人纯用机巧,疏狂之士独任性真,人心之生死亦于此判也。
翠筱傲严霜,节纵孤高,无伤冲雅;红蕖媚秋水,色虽艳丽,何损清修。
贫贱所难,不难在砥节,而难在用情;富贵所难,不难在推恩,而难在好礼。
簪缨之士,常不及孤寒之子可以抗节致忠;庙堂之士,常不及山野之夫可以料事烛理。何也?彼以浓艳损志,此以淡泊全真也。
荣宠旁边辱等待,不必扬扬;困穷背后福跟随,何须戚戚。
古人闲适处,今人却忙过了一生;古人实受处,今人又虚度了一世。总是耽空逐妄,看个色身不破,认个法身不真耳。
芝草无根醴无源,志士当勇奋翼;彩云易散琉璃脆,达人当早回头。
少壮者,事事当用意而意反轻,徒汛汛作水中凫而已,何以振云霄之翮?衰老者,事事宜忘情而情反重,徒碌碌为辕下驹而已,何以脱缰锁之身?

帆只扬五分,船便安。水只注五分,器便稳。如韩信以勇备震主被擒,陆机以才名冠世见杀,霍光败于权势逼君,石崇死于财赋敌国,皆以十分取败者也。康节云﹕"饮酒莫教成酩酊,看花慎勿至离披。"旨哉言乎!
附势者如寄生依木,木伐而寄生亦枯;窃利者如蝇虰盗人,人死而蝇虰亦灭。始以势利害人,终以势利自毙。势利之为害也,如是夫!
失血于杯中,堪笑猩猩之嗜酒;为巢于幕上,可怜燕燕之偷安。
鹤立鸡群,可谓超然无侣矣。然进而观于大海之鹏,则眇然自小。又进而求之九霄之凤,则巍乎莫及。所以至人常若无若虚,而盛德多不矜不伐也。贪心胜者,逐兽而不见泰山在前,弹雀而不知深井在后;疑心胜者,见弓影而惊杯中之蛇,听人言而信市上之虎。人心一偏,遂视有为无,造无作有。如此,心可妄动乎哉!
蛾扑火,火焦蛾,莫谓祸生无本;果种花,花结果,须知福至有因。
车争险道,马骋先鞭,到败处未免噬脐;粟喜堆山,金夸过斗,临行时还是空手。
花逞春光,一番雨、一番风,催归尘土;竹坚雅操,几朝霜、几朝雪,傲就琅玕。
富贵是无情之物,看得他重,他害你越大;贫贱是耐久之交,处得他好,他益你深。故贪商旅而恋金谷者,竟被一时之显戮;乐箪瓢而甘敝缊者,终享千载之令名。
鸽恶铃而高飞,不知敛翼而铃自息;人恶影而疾走,不知处阴而影自灭。故愚夫徒疾走高飞,而平地反为苦海;达士知处阴敛翼,而巉岩亦是坦途。秋虫春鸟共畅天机,何必浪生悲喜;老树新花同含生意,胡为妄别媸妍。
多栽桃李少栽荆,便是开条福路;不积诗书偏积玉,还如筑个祸基。
万境一辙原无地,着个穷通;万物一体原无处,分个彼我。世人迷真逐妄,乃向坦途上自设一坷坎,从空洞中自筑一藩蓠。良足慨哉!
大聪明的人,小事必朦胧;大懵懂的人,小事必伺察。盖伺察乃懵懂之根,而朦胧正聪明之窟也。
大烈鸿猷,常出悠闲镇定之士,不必忙忙;休征景福,多集宽洪长厚之家,何须琐琐。
贫士肯济人,才是性天中惠泽;闹场能学道,方为心地上工夫。
人生只为欲字所累,便如马如牛,听人羁络;为鹰为犬,任物鞭笞。若果一念清明,淡然无欲,天地也不能转动我,鬼神也不能役使我,况一切区区事物乎!
贪得者身富而心贫,知足者身贫而心富;居高者形逸而神劳,处下者形劳而神逸。孰得孰失,孰幻孰真,达人当自辨之。
众人以顺境为乐,而君子乐自逆境中来;众人以拂意为忧,而君子忧从快意处起。盖众人忧乐以情,而君子忧乐以理也。
谢豹覆面,犹知自愧;唐鼠易肠,犹知自悔。盖愧悔二字,乃吾人去恶迁善之门,起死回生之路也。人生若无此念头,便是既死之寒灰,已枯之槁木矣。何处讨些生理?
异宝奇琛,俱民必争之器;瑰节奇行,多冒不祥之名。总不若寻常历履易简行藏,可以完天地浑噩之真,享民物和平之福。
福善不在杳冥,即在食息起居处牖其衷;祸淫不在幽渺,即在动静语默间夺其魄。可见人之精爽常通于天,天之威命寓于人,天人岂相远哉!

\chapter{闲适}


昼闲人寂,听数声鸟语悠扬,不觉耳根尽彻;夜静天高,看一片云光舒卷,顿令眼界俱空。
世事如棋局,不着得才是高手;人生似瓦盆,打破了方见真空。
龙可豢非真龙,虎可搏非真虎,故爵禄可饵荣进之辈,必不可笼淡然无欲之人;鼎镬可及宠利之流,必不可加飘然远引之士。
一场闲富贵,狠狠争来,虽得还是失;百岁好光阴,忙忙过了,纵寿亦为夭。
高车嫌地僻,不如鱼鸟解亲人。驷马喜门高,怎似莺花能避俗。
红烛烧残,万念自然厌冷;黄梁梦破,一身亦似云浮。
千载奇逢,无如好书良友;一生清福,只在碗茗炉烟。
蓬茅下诵诗读书,日日与圣贤晤语,谁云贫是病?樽垒边幕天席地,时时共造化氤氲,孰谓非禅?兴来醉倒落花前,天地即为衾枕。机息坐忘盘石上,古今尽属蜉蝣。
昂藏老鹤虽饥,饮啄犹闲,肯同鸡鹜之营营而竞食?偃蹇寒松纵老,丰标自在,岂似桃李之灼灼而争妍!
吾人适志于花柳烂漫之时,得趣于笙歌腾沸之处,乃是造化之幻境,人心之荡念也。须从木落草枯之后,向声希味淡之中,觅得一些消息,才是乾坤的橐龠,人物的根宗。
静处观人事,即伊吕之勋庸、夷齐之节义,无非大海浮沤;闲中玩物情,虽木石之偏枯、鹿豕之顽蠢,总是吾性真如。
花开花谢春不管,拂意事休对人言;水暖水寒鱼自知,会心处还期独赏。
闲观扑纸蝇,笑痴人自生障碍;静觇竞巢鹊,叹杰士空逞英雄。
看破有尽身躯,万境之尘缘自息;悟入无坏境界,一轮之心月独明。
木床石枕冷家风,拥衾时魂梦亦爽;麦饭豆羹淡滋味,放箸处齿颊犹香。
谈纷华而厌者,或见纷华而喜;语淡泊而欣者,或处淡泊而厌。须扫除浓淡之见,灭却欣厌之情,才可以忘纷华而甘淡泊也。
"鸟惊心""花溅泪",怀此热肝肠,如何领取得冷风月;"山写照""水传神",识吾真面目,方可摆脱得幻乾坤。富贵得一世宠荣,到死时反增了一个恋字,如负重担;贫贱得一世清苦,到死时反脱了一个厌字,如释重枷。人诚想念到此,当急回贪恋之首而猛舒愁苦之眉矣。
人之有生也,如太仓之粒米,如灼目之电光,如悬崖之朽木,如逝海之一波。知此者如何不悲?如何不乐?如何看他不破而怀贪生之虑?如何看他不重而贻虚生之羞?
鹬蚌相持,兔犬共毙,冷觑来令人猛气全消;鸥凫共浴,鹿豕同眠,闲观去使我机心顿息。
迷则乐境成苦海,如水凝为冰;悟则苦海为乐境,犹冰涣作水。可见苦乐无二境,迷悟非两心,只在一转念间耳。
遍阅人情,始识疏狂之足贵;备尝世味,方知淡泊之为真。
地宽天高,尚觉鹏程之窄小;云深松老,方知鹤梦之悠闲。
两个空拳握古今,握住了还当放手;一条竹杖挑风月,挑到时也要息肩。
阶下几点飞翠落红,收拾来无非诗料;窗前一片浮青映白,悟入处尽是禅机。
忽睹天际彩云,常疑好事皆虚事;再观山中闲木,方信闲人是福人。
东海水曾闻无定波,世事何须扼腕?北邙山未省留闲地,人生且自舒眉。
天地尚无停息,日月且有盈亏,况区区人世能事事圆满而时时暇逸乎?只是向忙里偷闲,遇缺处知足,则操纵在我,作息自如,即造物不得与之论劳逸较亏盈矣!

"霜天闻鹤唳,雪夜听鸡鸣,"得乾坤清纯之气。"晴空看鸟飞,活水观鱼戏,"识宇宙活泼之机。
闲烹山茗听瓶声,炉内识阴阳之理;漫履楸枰观局戏,手中悟生杀之机。
芳菲园林看蜂忙,觑破几般尘情世态;寂寞衡茅观燕寝,引起一种冷趣幽思。
会心不在远,得趣不在多。盆池拳石间,便居然有万里山川之势,片言只语内,便宛然见万古圣贤之心,才是高士的眼界,达人的胸襟。
心与竹俱空,问是非何处安脚?貌偕松共瘦,知忧喜无由上眉。
趋炎虽暖,暖后更觉寒威;食蔗能甘,甘余便生苦趣。何似养志于清修而炎凉不涉,栖心于淡泊而甘苦俱忘,其自得为更多也。
席拥飞花落絮,坐林中锦绣团裀;炉烹白雪清冰,熬天上玲珑液髓。
逸态闲情,惟期自尚,何事处修边幅;清标傲骨,不愿人怜,无劳多买胭脂。
天地景物,如山间之空翠,水上之涟漪,潭中之云影,草际之烟光,月下之花容,风中之柳态。若有若无,半真半幻,最足以悦人心目而豁人性灵。真天地间一妙境也。
"乐意相关禽对语,生香不断树交花",此是无彼无此得真机。"野色更无山隔断,天光常与水相连",此是彻上彻下得真意。吾人时时以此景象注之心目,何患心思不活泼,气象不宽平!
鹤唳、雪月、霜天、想见屈大夫醒时之激烈;鸥眠、春风、暖日,会知陶处士醉里之风流。
黄鸟情多,常向梦中呼醉客;白云意懒,偏来僻处媚幽人。
栖迟蓬户,耳目虽拘而神情自旷;结纳山翁,仪文虽略而意念常真。
满室清风满几月,坐中物物见天心;一溪流水一山云,行处时时观妙道。
炮凤烹龙,放箸时与虀盐无异;悬金佩玉,成灰处共瓦砾何殊。
"扫地白云来",才着工夫便起障。“凿池明月入”,能空境界自生明。
造花唤作小儿,切莫受渠戏弄;天地丸为大块,须要任我炉锤。
想到白骨黄泉,壮士之肝肠自冷;坐老清溪碧嶂,俗流之胸次亦闲。
夜眠八尺,日啖二升,何须百般计较;书读五车,才分八斗,未闻一日清闲。

\chapter{概论}


君子之心事,天青日白,不可使人不知;君子之才华,玉韫珠藏,不可使人易知。
耳中常闻逆耳之言,心中常有拂心之事,才是进德修行的砥石。若言言悦耳,事事快心,便把此生埋在鸩毒中矣。
疾风怒雨,禽鸟戚戚;霁月光风,草木欣欣,可见天地不可一日无和气,人心不可一日无喜神。
醲肥辛甘非真味,真味只是淡;神奇卓异非至人,至人只是常。
夜深人静独坐观心;始知妄穷而真独露,每于此中得大机趣;既觉真现而妄难逃,又于此中得大惭忸。
恩里由来生害,故快意时须早回头;败后或反成功,故拂心处切莫放手。
藜口苋肠者,多冰清玉洁;衮衣玉食者,甘婢膝奴颜。盖志以淡泊明,而节从肥甘丧矣。
面前的田地要放得宽,使人无不平之叹;身后的惠泽要流得长,使人有不匮之思。
路径窄处留一步,与人行;滋味浓的减三分,让人嗜。此是涉世一极乐法。
作人无甚高远的事业,摆脱得俗情便入名流;为学无甚增益的工夫,减除得物累便臻圣境。
宠利毋居人前,德业毋落人后,受享毋逾分外,修持毋减分中。
处世让一步为高,退步即进步的张本;待人宽一分是福,利人实利己的根基。
盖世的功劳,当不得一个矜字;弥天的罪过,当不得一个悔字。
完名美节,不宜独任,分些与人,可以远害全身;辱行污名,不宜全推,引些归己,可以韬光养德。
事事要留个有余不尽的意思,便造物不能忌我,鬼神不能损我。若业必求满,功必求盈者,不生内变,必招外忧。
家庭有个真佛,日用有种真道,人能诚心和气、愉色婉言,使父母兄弟间形骸两释,意气交流,胜于调息观心万倍矣。
攻人之恶毋太严,要思其堪受;教人以善毋过高,当使其可从。
粪虫至秽变为蝉,而饮露于秋风;腐草无光化为荧,而耀采于夏月。故知洁常自污出,明每从暗生也。
矜高倨傲,无非客气;降伏得客气下,而后正气伸;情欲意识,尽属妄心消杀得,妄心尽而后真心现。
饱后思味,则浓淡之境都消;色后思淫,则男女之见尽绝。故人当以事后之悔,悟破临事之痴迷,则性定而动无不正。
居轩冕之中,不可无山林的气味;处林泉之下,须要怀廊庙的经纶。处世不必邀功,无过便是功;与人不求感德,无怨便是德。
忧勤是美德,太苦则无以适性怡情;淡泊是高风,太枯则无以济人利物。
事穷势蹙之人,当原其初心;功成行满之士,要观其末路。
富贵家宜宽厚而反忌克,是富贵而贫贱,其行如何能享?聪明人宜敛藏而反炫耀,是聪明而愚懵,其病如何不败!
人情反复,世路崎岖。行不去,须知退一步之法;行得去,务加让三分之功。
待小人不难于严,而难于不恶;待君子不难于恭,而难于有礼。
宁守浑噩而黜聪明,留些正气还天地;宁谢纷华而甘淡泊,遗个清名在乾坤。
降魔者先降其心,心伏则群魔退听;驭横者先驭其气,气平则外横不侵。
养弟子如养闺女,最要严出入,谨交游。若一接近匪人,是清净田中下一不净的种子,便终身难植嘉苗矣。

欲路上事,毋乐其便而姑为染指,一染指便深入万仞;理路上事,毋惮其难而稍为退步,一退步便远隔千山。
念头浓者自待厚,待人亦厚,处处皆厚;念头淡者自待薄,待人亦薄,事事皆薄。故君子居常嗜好,不可太浓艳,亦不宜太枯寂。
彼富我仁,彼爵我义,君子故不为君相所牢笼;人定胜天,志壹动气,君子亦不受造化之陶铸。
立身不高一步立,如尘里振衣、泥中濯足,如何超达?处世不退一步处,如飞蛾投烛、羝羊触藩,如何安乐?
学者要收拾精神并归一处。如修德而留意于事功名誉,必无实诣;读书而寄兴于吟咏风雅,定不深心。
人人有个大慈悲,维摩屠刽无二心也;处处有种真趣味,金屋茅檐非两地也。只是欲闭情封,当面错过,便咫尺千里矣。
进德修行,要个木石的念头,若一有欣羡便趋欲境;济世经邦,要段云水的趣味,若一有贪着便堕危机。
肝受病则目不能视,肾受病则耳不能听。病受于人所不见,必发于人所共见。故君子欲无得罪于昭昭,先无得罪于冥冥。
福莫福于少事,祸莫祸于多心。惟苦事者方知少事之为福;惟平心者始知多心之为祸。
处治世宜方,处乱世当圆,处叔季之世当方圆并用。待善人宜宽,待恶人当严,待庸众之人宜宽严互存。
我有功于人不可念,而过则不可不念;人有恩于我不可忘,而怨则不可不忘。
心地干净,方可读书学古。不然,见一善行,窃以济私;闻一善言,假以覆短。是又藉寇兵而济盗粮矣。
奢者富而不足,何如俭者贫而有余。能者劳而俯怨,何如拙者逸而全真。
读书不见圣贤,如铅椠佣。居官不爱子民,如衣冠盗。讲学不尚躬行,如口头禅。立业不思种德。如眼前花。
人心有部真文章,都被残编断简封固了;有部真鼓吹,都被妖歌艳舞湮没了。学者须扫除外物直觅本来,才有个真受用。苦心中常得悦心之趣;得意时便生失意之悲。
富贵名誉自道德来者,如山林中花,自是舒徐繁衍。自功业来者,如盆槛中花,便有迁徙废兴。若以权力得者,其根不植,其萎可立而待矣。
栖守道德者,寂寞一时;依阿权势者,凄凉万古。达人观物外之物,思身后之身,宁受一时之寂寞,毋取万古之凄凉。
春至时和,花尚铺一段好色,鸟且啭几句好音。士君子幸列头角,复遇温饱,不思立好言、行好事,虽是在世百年,恰似未生一日。
学者有段兢业的心思,又要有段潇洒的趣味。若一味敛束清苦,是有秋杀无春生,何以发育万物?
真廉无廉名,立名者正所以为贪;大巧无巧术,用术者乃所以为拙。
心体光明,暗室中有青天;念头暗昧,白日下有厉鬼。
人知名位为乐,不知无名无位之乐为最真;人知饥寒为忧,不知不饥不寒之忧为更甚。
为恶而畏人知,恶中犹有善路;为善而急人知,善处即是恶根。
天之机缄不测,抑而伸、伸而抑,皆是播弄英雄、颠倒豪杰处。君子只是逆来顺受、居安思危,天亦无所用其伎俩矣。
福不可邀,养喜神以为招福之本;祸不可避,去杀机以为远祸之方。
十语九中未必称奇,一语不中,则愆尤骈集;十谋九成未必归功,一谋不成则訾议丛兴。君子所以宁默毋躁、宁拙毋巧。

天地之气,暖则生,寒则杀。故性气清冷者,受享亦凉薄。惟气和暖心之人,其福亦厚,其泽亦长。
天理路上甚宽,稍游心胸中,使觉广大宏朗;人欲路上甚窄,才寄迹眼前,俱是荆棘泥涂。
一苦一乐相磨练,练极而成福者,其福始久﹕一疑一信相参勘,勘极而成知者,其知始真。
地之秽者多生物,水之清者常无鱼,故君子当存含垢纳污之量,不可持好洁独行之操。
泛驾之马可就驰驱,跃冶之金终归型范。只一优游不振,便终身无个进步。白沙云﹕"为人多病未足羞,一生无病是吾忧。"真确实之论也。
人只一念贪私,便销刚为柔,塞智为昏,变恩为惨,染洁为污,坏了一生人品。故古人以不贪为宝,所以度越一世。
耳目见闻为外贼,情欲意识为内贼,只是主人公惺惺不昧,独坐中堂,贼便化为家人矣。
图未就之功,不如保已成之业;悔既往之失,亦要防将来之非。
气象要高旷,而不可疏狂。心思要缜缄,而不可琐屑。趣味要冲淡,而不可偏枯。操守要严明,而不可激烈。
风来疏竹,风过而竹不留声;雁度寒潭,雁去而潭不留影。故君子事来而心始现,事去而心随空。
清能有容,仁能善断,明不伤察,直不过矫,是谓蜜饯不甜、海味不咸,才是懿德。
贫家净扫地,贫女净梳头。景色虽不艳丽,气度自是风雅。士君子当穷愁寥落,奈何辄自废弛哉!
闲中不放过,忙中有受用。静中不落空,动中有受用。暗中不欺隐,明中有受用。
念头起处,才觉向欲路上去,便挽从理路上来。一起便觉,一觉便转,此是转祸为福、起死回生的关头,切莫当面错过。
天薄我以福,吾厚吾德以迓之;天劳我以形,吾逸吾心以补之;天扼我以遇,吾亨吾道以通之。天且奈我何哉!
真士无心邀福,天即就无心处牖其衷;险人着意避祸,天即就着意中夺其魂。可见天之机权最神,人之智巧何益!
声妓晚景从良,一世之烟花无碍;贞妇白头失守,半生之清苦俱非。语云﹕"看人只看后半截",真名言也。
平民肯种德施惠,便是无位的卿相;仕夫徒贪权市宠,竟成有爵的乞人。
问祖宗之德泽,吾身所享者,是当念其积累之难;问子孙之福祉,吾身所贻者,是要思其倾覆之易。
君子而诈善,无异小人之肆恶;君子而改节,不若小人之自新。
家人有过不宜暴扬,不宜轻弃。此事难言,借他事而隐讽之。今日不悟,俟来日正警之。如春风之解冻、和气之消冰,才是家庭的型范。
此心常看得圆满,天下自无缺陷之世界;此心常放得宽平,天下自无险侧之人情。
淡薄之士,必为浓艳者所疑;检饬之人,多为放肆者所忌。君子处此固不可少变其操履,亦不可太露其锋芒。
居逆境中,周身皆针砭药石,砥节砺行而不觉;处顺境内,眼前尽兵刃戈矛,销膏靡骨而不知。
生长富贵丛中的,嗜欲如猛火、权势似烈焰。若不带些清冷气味,其火焰不至焚人,必将自焚。
人心一真,便霜可飞、城可陨、金石可贯。若伪妄之人,形骸徒具,真宰已亡。对人则面目可憎,独居则形影自愧。
文章做到极处,无有他奇,只是恰好;人品做到极处,无有他异,只是本然。
以幻迹言,无论功名富贵,即肢体亦属委形;以真境言,无论父母兄弟,即万物皆吾一体。人能看得破,

认得真,才可以任天下之负担,亦可脱世间之缰锁。
爽口之味,皆烂肠腐骨之药,五分便无殃;快心之事,悉败身散德之媒,五分便无悔。
不责人小过,不发人阴私,不念人旧恶,三者可以养德,亦可以远害。
天地有万古,此身不再得;人生只百年,此日最易过。幸生其间者,不可不知有生之乐,亦不可不怀虚生之忧。
老来疾病都是壮时招得;衰时罪孽都是盛时作得。故持盈履满,君子尤兢兢焉。
市私恩不如扶公议,结新知不如敦旧好,立荣名不如种阴得,尚奇节不如谨庸行。
公平正论不可犯手,一犯手则遗羞万世;权门私窦不可着脚,一着脚则玷污终身。
曲意而使人喜,不若直节而使人忌;无善而致人誉,不如无恶而致人毁。
处父兄骨肉之变,宜从容不宜激烈;遇朋友交游之失,宜剀切不宜优游。
小处不渗漏,暗处不欺隐,末路不怠荒,才是真正英雄。
惊奇喜异者,终无远大之识;苦节独行者,要有恒久之操。
当怒火欲水正腾沸时,明明知得,又明明犯着。知得是谁,犯着又是谁。此处能猛然转念,邪魔便为知真君子矣。
毋偏信而为奸所欺,毋自任而为气所使,毋以己之长而形人之短,毋因己之拙而忌人之能。
人之短处,要曲为弥缝,如暴而扬之,是以短攻短;人有顽的,要善为化诲,如忿而嫉之,是以顽济顽。
遇沉沉不语之士,且莫输心;见悻悻自好之人,应须防口。
念头昏散处,要知提醒;念头吃紧时,要知放下。不然恐去昏昏之病,又来憧憧之扰矣。
霁日青天,倏变为迅雷震电;疾风怒雨,倏转为朗月晴空。气机何尝一毫凝滞,太虚何尝一毫障蔽,人之心体亦当如是。
胜私制欲之功,有曰识不早、力不易者,有曰识得破、忍不过者。盖识是一颗照魔的明珠,力是一把斩魔的慧剑,两不可少也。
横逆困穷,是煅炼豪杰的一副炉锤。能受其煅炼者,则身心交益;不受其煅炼者,则身心交损。
害人之心不可有,防人之心不可无,此戒疏于虑者。宁受人之欺,毋逆人之诈,此警伤于察者。二语并存,精明浑厚矣。
毋因群疑而阻独见,毋任己意而废人言,毋私不惠而伤大体,毋借公论以快私情。
善人未能急亲,不宜预扬,恐来谗谮之奸;恶人未能轻去,不宜先发,恐招媒孽之祸。
青天白日的节义,自暗室屋漏中培来;旋乾转坤的经纶,从临深履薄中操出。
父慈子孝、兄友弟恭,纵做到极处,俱是合当如是,着不得一毫感激的念头。如施者任德,受者怀恩,便是路人,便成市道矣。
炎凉之态,富贵更甚于贫贱;妒忌之心,骨肉尤狠于外人。此处若不当以冷肠,御以平气,鲜不日坐烦恼障中矣。
功过不宜少混,混则人怀惰隳之心;恩仇不可太明,明则人起携贰之志。
恶忌阴,善忌阳,故恶之显者祸浅,而隐者祸深。善之显者功小,而隐者功大。
德者才之主,才者德之奴。有才无德,如家无主而奴用事矣,几何不魍魉猖狂。
锄奸杜幸,要放他一条去路。若使之一无所容,便如塞鼠穴者,一切去路都塞尽,则一切好物都咬破矣。
士君子不能济物者,遇人痴迷处,出一言提醒之,遇人急难处,出一言解救之,亦是无量功德矣。

反己者触事皆成药石,尤人者动念即是戈矛,一以辟众善之路,一以浚诸恶之源,相去霄壤矣。
事业文章随身销毁,而精神万古如新;功名富贵逐世转移,而气节千载一时。君子信不当以彼易此也。
鱼网之设,鸿则罹其中;螳螂之贪,雀又乘其后。机里藏机变外生变,智巧何足恃哉。
作人无一点真恳的念头,便成个花子,事事皆虚;涉世无一段圆活的机趣,便是个木人,处处有碍。
事有急之不白者,宽之或自明,毋躁急以速其忿;人有切之不从者,纵之或自化,毋操切以益其顽。
节义傲青云,文章高白雪,若不以德性陶镕之,终为血气之私、技能之末。
谢事当谢于正盛之时,居身宜居于独后之地,谨德须谨于至微之事,施恩务施于不报之人。
德者事业之基,未有基不固而栋宇坚久者;心者修裔之根,未有根不植而枝叶荣茂者。

\backmatter

\end{document}