% 素书
% 素书.tex

\documentclass[12pt,UTF8]{ctexbook}

% 设置纸张信息。
\input{../../../common/page}

% 设置字体,并解决显示难检字问题。
\xeCJKsetup{AutoFallBack=true}
\setCJKmainfont{SimSun}[BoldFont=SimHei, ItalicFont=KaiTi, FallBack=SimSun-ExtB]

% 目录 chapter 级别加点(.)。
\usepackage{titletoc}
\titlecontents{chapter}[0pt]{\vspace{3mm}\bf\addvspace{2pt}\filright}{\contentspush{\thecontentslabel\hspace{0.8em}}}{}{\titlerule*[8pt]{.}\contentspage}

% 设置 part 和 chapter 标题格式。
\ctexset{
	part/name= {第,卷},
	part/number={\chinese{part}},
	chapter/name={第,篇},
	chapter/number={\arabic{chapter}}
}

% 图片相关设置。
\usepackage{graphicx}
\graphicspath{{Images/}}

% 设置署名格式。
\newenvironment{shuming}{\hfill\zihao{4}}

% 注脚每页重新编号,避免编号过大。
\usepackage[perpage]{footmisc}

% 设置古文原文格式。
\newenvironment{yuanwen}{\bfseries\zihao{4}}

% 列表项向右偏移。
\usepackage{enumitem}

\title{\heiti\zihao{0} 素书}
\author{黄石公}
\date{汉}

\begin{document}

\maketitle
\tableofcontents

\frontmatter
\chapter{前言、序言}

\mainmatter

% 增加空行
~\\

% 增加字间间隔,适用于三字经、诗文等。
 \qquad  

\chapter{原始}

夫道、德、仁、义、礼五者,一体也。

道者,人之所蹈,使万物不知其所由。

德者,人之所得,使万物各(得/德)其所欲。

仁者,人之所亲,有慈慧恻隐之心,以遂其生存。(生存 一作:生成)

义者,人之所宜,赏善罚恶,以立功立事。

礼者,人之所履,夙兴夜寐,以成人伦之序。

夫欲为人之本,不可无一焉。

贤人君子,明于盛衰之道,通乎成败之数,审乎治乱之势,达乎去就之理。故潜居抱道,以待其时。

若时至而行,则能极人臣之位;
得机而动,则能成绝代之功。如其不遇,没身而已。
是以其道足高,而名重于后代。

\chapter{正道}

德足以怀远,信足以一异,义足以得众,才足以鉴古,明足以照下,此人之俊也。

行足以为仪表,智足以决嫌疑,信可以使守约,廉可以使分财,此人之豪也。

守职而不废,处义而不回,见嫌而不苟免,见利而不苟得,此人之杰也。

\chapter{求人之志}

绝嗜禁欲,所以除累。抑非损恶,所以(禳/让)过。贬酒阙色,所以无污。

避嫌远疑,所以不误。博学切问,所以广知。高行微言,所以修身。

恭俭谦约,所以自守。深计远虑,所以不(穷/彰)。亲仁友直,所以扶颠。

近恕笃行,所以接人。任材使能,所以济物。(殚/瘅)恶斥谗,所以止乱。(殚恶 一作:瘅恶)

推古验今,所以不惑。先揆后度,所以应卒。设变致权,所以解结。

括囊顺会,所以无咎。橛橛梗梗,所以立功。孜孜淑淑,所以保终。

\chapter{本德宗道}

夫志,心笃行之术。

长莫长于博谋,安莫安于忍辱,先莫先于修德,乐莫乐于好善,

神莫神于至诚,明莫明于体物,吉莫吉于知足,苦莫苦于多愿,

悲莫悲于精散,病莫病于无常,短莫短于苟得,幽莫幽于贪鄙,

孤莫孤于自恃,危莫危于任疑,败莫败于多私。

\chapter{遵义}

以明示下者(暗/闇),有过不知者蔽,迷而不返者惑,

以言取怨者祸,令与心乖者废,后令缪前者毁,

怒而无威者犯,好众辱人者殃,戮辱所任者危,

慢其所敬者凶,貌合心离者孤,亲谗远忠者亡,

近色远贤者(昏/惛),女谒公行者乱,私人以官者浮,

凌下取胜者侵,名不胜实者耗。

略己而责人者不治,自厚而薄人者弃废。

以过弃功者损,群下外异者沦,既用不任者疏,

行赏吝色者沮,多许少与者怨,既迎而拒者乖。

薄施厚望者不报,贵而忘贱者不久。

念旧(恶)而弃新功者凶 ,用人不得正者殆,强用人者不畜,

为人择官者乱,失其所强者弱,决策于不仁者险,

阴计外泄者败,厚敛薄施者凋。

战士贫,游士富者衰;货赂公行者昧;

闻善忽略,记过不忘者暴;所任不可信,所信不可任者浊。

牧人以德者集,绳人以刑者散。

小功不赏,则大功不立;小怨不赦,则大怨必生。

赏不服人,罚不甘心者叛。赏及无功,罚及无罪者酷。

听谗而美,闻谏而仇者亡。能有其有者安,贪人之有者残。

\chapter{安礼}

怨在不舍小过,患在不预定谋。福在积善,祸在积恶。

饥在贱农,寒在堕织。安在得人,危在失(士/事)。

富在迎来,贫在弃时。上无常操,下多疑心。

轻上生罪,侮下无亲。近臣不重,远臣轻之。

自疑不信人,自信不疑人。枉士无正友,曲上无直下。

危国无贤人,乱政无善人。爱人深者求贤急,乐得贤者养人厚。

国将霸者士皆归,邦将亡者贤先避。

地薄者大物不产,水浅者大鱼不游,树秃者大禽不栖,林疏者大兽不居。

山峭者崩,泽满者溢。

弃玉取石者盲,羊质虎皮者柔。

衣不举领者倒,走不视地者颠。柱弱者屋坏,辅弱者国倾。

足寒伤心,人怨伤国。山将崩者下先隳,国将衰者人先弊。

根枯枝朽,人困国残。与覆车同轨者倾,与亡国同事者灭。

见已生者慎将生,恶其迹者须避之。畏危者安,畏亡者存。

夫人之所行,有道则吉,无道则凶。吉者,百福所归;凶者,百祸所攻。

非其神圣,自然所钟。务善(策)者无恶事,无远虑者有近忧。

同志相得,同仁相忧,同恶相党,同爱相求,同美相妒,同智相谋,

同贵相害,同利相忌,同声相应,同气相感,同类相依,同义相亲,

同难相济,同道相成,同艺相规,同巧相胜:此乃数之所得,不可与理违。

释己而教人者逆,正己而化人者顺。

逆者难从,顺者易行,难从则乱,易行则理。

如此理身、理家、理国,可也!

\backmatter

\end{document}