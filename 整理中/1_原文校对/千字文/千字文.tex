% 千字文
% 千字文.tex

\documentclass[a4paper,12pt,UTF8,twoside]{ctexbook}

% 设置纸张信息。
\RequirePackage[a4paper]{geometry}
\geometry{
	%textwidth=138mm,
	%textheight=215mm,
	%left=27mm,
	%right=27mm,
	%top=25.4mm, 
	%bottom=25.4mm,
	%headheight=2.17cm,
	%headsep=4mm,
	%footskip=12mm,
	%heightrounded,
	inner=1in,
	outer=1.25in
}

% 设置字体,并解决显示难检字问题。
\xeCJKsetup{AutoFallBack=true}
\setCJKmainfont{SimSun}[BoldFont=SimHei, ItalicFont=KaiTi, FallBack=SimSun-ExtB]

% 目录 chapter 级别加点(.)。
\usepackage{titletoc}
\titlecontents{chapter}[0pt]{\vspace{3mm}\bf\addvspace{2pt}\filright}{\contentspush{\thecontentslabel\hspace{0.8em}}}{}{\titlerule*[8pt]{.}\contentspage}

% 设置 part 和 chapter 标题格式。
\ctexset{
	chapter/name={},
	chapter/number={}
}

% 设置古文原文格式。
\newenvironment{yuanwen}{\bfseries\zihao{4}}

% 设置署名格式。
\newenvironment{shuming}{\hfill\bfseries\zihao{4}}

% 注脚每页重新编号,避免编号过大。
\usepackage[perpage]{footmisc}

\title{\heiti\zihao{0} 千字文}
\author{周兴嗣}
\date{南朝梁 · 公元 502--549 年}

\begin{document}

\maketitle
\tableofcontents

\frontmatter
\chapter{前言}

南朝梁武帝时期(公元502--549年),员外散骑侍郎周兴嗣奉皇命从王羲之书法中选取1000个字,编纂成文,是为《千字文》。文中1000字本来不得有所重复,但周兴嗣在编纂文章时,却重复了一个“洁”字(洁、絜为同义异体字)。因此,《千字文》实际只运用了999字。

除周兴嗣版《千字文》之外,另有《续千字文》(侍其玮著)、《叙古千字文》(胡寅著)、《新千字文》(高占祥、赵缺著)等不同版本的《千字文》。其中,“高、赵版《新千字文》”被认为是超越“周版《千字文》”的真正经典之作。

《千字文》在中国古代的童蒙读物中,是一篇承上启下的作品。它那优美的文笔,华丽的辞藻,是其他任何一部童蒙读物都无法望其项背的。

《千字文》以儒学理论为纲、穿插诸多常识,用四字韵语写出,很适于儿童诵读,后来就成了中国古代教育史上最早、最成功的启蒙教材。宋明以后直至清末,《千字文》与《三字经》、《百家姓》一起,构成了我国人民最基础的“三、百、千”启蒙读物。旧有打油诗云:“学童三五并排坐,天地玄黄喊一年”,此之谓也!不仅汉民族用作儿童启蒙教材,一些兄弟民族也使用,甚至传到了日本。

同时,《千字文》在中国文化史上也有独特地位,是历代各流派书法家进行书法创作的重要载体。隋唐以后,凡著名书法家均有不同书体的《千字文》作品传世。 

公元六世纪初,南朝梁武帝时期在建业(今南京)刻印问世的《千字文》被公认为世界使用时间最长、影响最大的儿童启蒙识字课本,比唐代出现的《百家姓》和宋代编写的《三字经》还早。《千字文》可以说是千余年来最畅销、读者最广泛的读物之一。

《千字文》乃四言长诗,首尾连贯,音韵谐美。以“天地玄黄,宇宙洪荒”开头,“谓悟助者,焉哉手也”结尾。全文共250句,每四字一句,字不重复,句句押韵,前后贯通,内容有条不紊的介绍了天文、自然、修身养性、人伦道德、地理、历史、农耕、祭祀、园艺、饮食起居等各个方面。

\mainmatter

\begin{yuanwen}
天地玄黄\footnote{玄青,深黑色。古人认为天为玄青色,地为黄色。},宇宙洪荒\footnote{混沌蒙昧的状态,此处指宽阔辽远。}。日月盈\footnote{充满。}昃\footnote{z\`e,太阳偏西。},辰\footnote{日、月、星的统称。}宿\footnote{xi\`u,中国古代将天上某些星的集合体称为宿,共二十八宿,分别为东方七宿,即角、亢、氐、房、心、尾、箕;南方七宿,即井、鬼、柳、星、张、翼、轸;西方七宿,即奎、娄、胃、昴、毕、觜、参;北方七宿,即斗、牛、女、虚、危、室、壁。}列张。
\end{yuanwen}

玄,天也;黄,地之色也;洪,大也;荒,远也;宇宙广大无边。

天空青,大地黄,宇宙茫茫,无边无际。日出日落,月圆月缺,星辰布满天空。

\begin{yuanwen}
寒来暑往,秋收冬藏。闰\footnote{一回归年的时间为365天5时48分46秒。阳历把一年定为365天,所余的时间约每四年积累成一天,加在二月里;农历把一年定为354天或355天,所余的时间约每三年积累成一个月,加在一年里。这样的方法,在历法上叫作闰。}余成岁,律吕\footnote{古代用竹管制成的校正乐律的器具,以管的长短来确定音的不同高度。从低音管算起,成奇数的六个管叫作“律”,又叫阳律;成偶数的六个管叫作“吕”,又叫阴律。后来用“律吕”作为音律的统称。}调阳。
\end{yuanwen}

寒暑循环变换,来了又去,去了又来;秋天收获,冬季储藏。历法以闰余积成闰月,放在闰年里;音乐用律吕调节阴阳。

\begin{yuanwen}
云腾致雨,露结为霜。金生丽水\footnote{即金沙江,相传江中盛产黄金。},玉出昆冈\footnote{即昆仑山,相传山里盛产美玉。}。
\end{yuanwen}

云气上升遇冷就形成了雨,露水遇冷凝为霜。金子出产于金沙江底,玉石出自昆仑山岗。

\begin{yuanwen}
剑号巨阙\footnote{qu\`e,传说春秋时期越王勾践有一口锋利的宝剑,名巨阙。},珠称夜光\footnote{即夜明珠。传说南海鲸鱼的眼珠,在黑夜中能发出光亮。}。果珍李柰\footnote{n\`ai,花红果。},菜重芥\footnote{一年或二年生草本植物。种子黄色,有辣味。磨成粉末,叫芥末,用作调味品。芥菜品种很多,形态各异。}姜。
\end{yuanwen}

最锋利的宝剑叫“巨阙”,最贵重的明珠叫“夜光”。
水果中珍品是李和柰,蔬菜中最重要的是芥和姜。

\begin{yuanwen}
海咸河淡,鳞\footnote{鱼鳞,此处泛指鱼类。}潜羽\footnote{羽毛,此处泛指禽类。}翔。龙师\footnote{伏羲氏以龙命官,称为“龙师”。}火帝\footnote{即钻木取火的燧人氏。},鸟官\footnote{少昊氏以鸟命官,称为“鸟官”。}人皇\footnote{古代传说中的帝王,通常称伏羲、燧人、神农为三皇,或者称天皇、地皇、人皇。}。
\end{yuanwen}

海水咸,河水淡,鱼儿在水中潜游,鸟儿在空中翱翔。龙师、火帝与鸟官,天皇、地皇和人皇。

\begin{yuanwen}
始\footnote{初,开始。}制\footnote{制:造。}文字,乃服衣裳。推位\footnote{禅让之意。}让国,有虞\footnote{上古传说中的舜帝。}陶唐\footnote{上古传说中的尧帝。}。
\end{yuanwen}

苍颉造文字,嫘祖制服装。禅让王位的贤人,要数有虞和陶唐。

\begin{yuanwen}
吊民伐罪\footnote{慰问受苦的民众,讨伐有罪的统治者。},周发\footnote{即周武王姬发。}殷汤\footnote{商朝的建立者。}。坐朝问道,垂拱\footnote{垂衣拱手。古时多指统治者以无所作为、顺其自然的方式统治天下。}平章\footnote{商量处理国事。}。
\end{yuanwen}

抚慰百姓讨伐罪孽,典范便是姬发和殷汤。
安抚百姓,讨伐暴君,有周武王姬发和商君成汤。
安抚百姓,讨伐暴君,是周武王姬发和商王成汤。

\begin{yuanwen}
爱育黎首\footnote{百姓。},臣伏戎羌\footnote{泛指古代少数民族。}。遐\footnote{xi\'a,远。}迩\footnote{\v{e}r,近。}一体,率宾归王\footnote{指土地和人民都统一于某个君王。《诗经》:“普天之下,莫非王土;率土之滨,莫非王臣。”}。
\end{yuanwen}

\begin{yuanwen}
鸣凤在竹,白驹食场\footnote{ch\'ang}。化被草木,赖及\footnote{传导;波及。}万方。
\end{yuanwen}

他们端坐于朝垂衣拱手,与臣子把国家大事商量。他们爱护养育百姓,四方各族归附向往。远近地域都实现了统一,万民相率服从于君王。

贤君身坐朝廷,探讨治国之道,垂衣拱手,和大臣共商国事。

他们爱抚、体恤老百姓,四方各族人都归附向往。

远远近近都统一在一起,全都心甘情屈服贤君。

凤凰在竹林中欢鸣,白马在草场上觅食,国泰民安,处处吉祥。
凤凰和鸣于竹林,白驹觅食在草场。君王的道德教化覆盖草木,君王的恩泽德行遍及四方。

贤君的教化覆盖大自然的一草一木,恩泽遍及天下百姓。

贤明的君主坐在朝廷上向大臣们询问治国之道,垂衣拱手,毫不费力就能使天下太平,功绩彰著。

他们爱抚、体恤老百姓,使四方各族人俯首称臣。普天之下都统一成了一个整体,所有的老百姓都服服贴贴地归顺于他的统治。

\begin{yuanwen}
盖\footnote{发语词,无义。}此身发,四大\footnote{古人认为人的身体由地、水、火、风“四大”组成,其中,骨、肉、毛发属地;血液、眼泪属水;体温属火;体内循环属风。}五常\footnote{即仁、义、礼、智、信。}。恭\footnote{恭敬。}惟鞠\footnote{j\=u,抚养;养育。}养,岂敢毁伤。
\end{yuanwen}

\begin{yuanwen}
女慕\footnote{向往。}贞洁,男效\footnote{效仿。}才良。知过必改,得\footnote{获得;获取。}能\footnote{能力;知识。}莫忘。
\end{yuanwen}

女子要仰慕贞洁纯净的情操,男人要效法德才兼备的贤良。知道过错必须改正,学习知识不应轻忘。

人的身体发肤分属于“四大”,一言一动都要符合“五常”。
我们的身体发肤分属“四大”,我们的行为准则必须符合“五常”。父母给予的身体应该爱护,岂能有一丝一毫的毁伤。

恭蒙父母亲生养爱护,不可有一丝一毫的毁坏损伤。

女子要思慕那些为人称道的贞妇洁女,男子要效法有德有才的贤人。

知道自己有过错,一定要改正;适合自己干的事,不要放弃。

人的身体发肤分属于“四大”,一言一动都要符合“五常”。诚敬的想着父母养育之恩,哪里还敢毁坏损伤它。

女子要仰慕那些持身严谨的贞妇洁女,男子要仿效那些有才能有道德的人。知道自己有过错,一定要改正;适合自己干的事,不要放弃。

\begin{yuanwen}
罔\footnote{不要。}谈彼短,靡\footnote{无;没有。}恃\footnote{sh\`i}己长。信\footnote{诚实,守信。}使可覆\footnote{审查,考察。},器\footnote{器量。}欲难量。
\end{yuanwen}

\begin{yuanwen}
墨\footnote{墨子。}悲丝染,诗赞羔羊。景行\footnote{景仰。}维贤,克念\footnote{克制私欲。}作圣。
\end{yuanwen}

不要谈论别人的短处,也不要依仗自己有长处就不思进取。诚实的话要经得起考验,器度要大,让人难以估量。
切莫背后议论别人的短缺,不要仗恃自己的优长。诚实守信方能经受时间考验,器量大度别人就难以度量。

墨子悲叹白丝被染上了杂色,《诗经》赞颂羔羊能始终保持洁白如一。 要仰慕圣贤的德行,要克制私欲,努力仿效圣人。
墨子叹息易染的蚕丝,《诗经》赞美洁白的羔羊。效仿景仰贤者的言行,克制私欲使自己成为圣人。

不要去谈论别人的短处,也不要依仗自己有长处就不思进取。

诚实的话要能经受时间的考验;器度要大,让人难以估量。

墨子为白丝染色不褪而悲泣,「诗经」中因此有「羔羊」篇传扬。

高尚的德行只能在贤人那里看到;要克制私欲,努力仿效圣人。

\begin{yuanwen}
德建\footnote{养成。}名立\footnote{树立。},形\footnote{形状;形体。此处指动作体态。}端表\footnote{外面;外表。此处指仪表风度。}正。空\footnote{空旷。}谷传声,虚\footnote{宽敞。}堂习听。
\end{yuanwen}

\begin{yuanwen}
祸因恶积,福缘善庆。尺璧非宝,寸阴是竞\footnote{争。}。
\end{yuanwen}

养成了好的道德,就会有好的名声;就像形体端庄,仪表也随之肃穆一样。
养成好的道德,树立好的名声,体态端庄,仪表庄重。空旷的山谷声音可以传得很远,宽敞的厅堂说话也能荡起回声。

空旷的山谷中呼喊声传得很远,宽敞的厅堂里说话声非常清晰。

祸害是因为多次作恶积累而成,幸福是由于常年行善得到的奖赏。

养成了好的道德,就会有好的名声;就如同形体端庄了,仪表就正直了一样。空旷的山谷中呼喊声传得很远,宽敞的厅堂里说话声非常清晰。

\begin{yuanwen}
资\footnote{资助;帮助。}父事君,曰严\footnote{敬畏。}与敬。孝当竭力,忠则尽命。
\end{yuanwen}

祸害是因为多次作恶积累而成,福分是由于常年行善的结果。一尺之璧玉不算宝贵,一寸之光阴值得珍惜。

一尺长的璧玉算不上宝贵,一寸短的光阴却值得去争取。

供养父亲,待奉国君,要做到认真、谨慎、恭敬。
供养父母侍奉君主,必须认真恭敬。尽孝当竭尽全力,尽忠应不惜生命。

对父母孝,要尽心竭力;对国君忠,要不惜献出生命。

要“如临深渊,如履薄冰”那样小心谨慎;要早起晚睡,让父母冬暖夏凉。

灾祸是作恶多端的结果,福禄是乐善好施的回报。一尺长的美玉不能算是真正的宝贝,而即使是片刻时光也值得珍惜。

奉养父亲,侍奉君主,要严肃而恭敬。孝顺父母应当竭尽全力,忠于君主要不惜献出生命。

\begin{yuanwen}
临深履薄\footnote{临深渊,履薄冰。},夙\footnote{早。}兴\footnote{起。}温\footnote{温暖。}凊\footnote{清凉。}。
似兰斯\footnote{这,此。}馨\footnote{散布得很远的香气。},如松之盛。
\end{yuanwen}

\begin{yuanwen}
川流不息,渊澄\footnote{清澈。}取映\footnote{照。}。容\footnote{容貌。}止\footnote{举止。}若思,言辞安定。
\end{yuanwen}

侍奉君主要像临深渊履薄冰时那样小心谨慎,奉养父母则应注意冬暖夏凉,嘘寒问暖,使之感受亲情。这样的品德才像兰花一样馨香,像松柏一般茂盛,像江河水川流不息地流淌,似碧潭水清亮澄澈地照映。

能这样去做,德行就同兰花一样馨香,同青松一样茂盛。
还能延及子孙,像大河川流不息;影响世人,像碧潭清澄照人。
仪态举止要庄重,看上去若有所思;言语措辞要稳重,显得从容沉静。
要“如临深渊,如履薄冰”那样小心谨慎;要早起晚睡,侍候父母让他们感到冬暖夏凉。让自己的德行像兰草那样的清香,像松柏那样的茂盛。
还能延及子孙,像大河川流不息;影响世人,像碧潭清澄照人。仪容举止要沉静安详,言语措辞要稳重,显得从容沉静。
无论修身、求学、重视开头固然不错,认真去做,有好的结果更为重要。这是一生荣誉的事业的基础,有此根基,发展就没有止境。
无论修身、求学、重视开头固然不错,认真去做,有好的结果更为重要。

\begin{yuanwen}
笃\footnote{厚;忠实。}初诚美,慎终宜令\footnote{美好。}。荣业\footnote{光耀事业。}所基,籍\footnote{名声,声誉。}甚\footnote{很,极。}无竟\footnote{停止。}。
\end{yuanwen}

行为要端正安详,言辞要从容沉静。专注创造良好的开始诚然必要,谨慎追求完美的结果尤应注重。孝成而德备是终身事业的基础,声誉日隆绵延无尽。

有德能孝是事业显耀的基础,这样的人声誉盛大,传扬不已。

学习出色并有余力,就可出仕做官,担任一定的职务,参与国家的政事。

书读好了就能做官,可以行使职权参加国政。周人怀念召伯的德政,召公活着时曾在甘棠树下理政,他过世后老百姓对他更加怀念歌咏。

\begin{yuanwen}
学优登仕,摄\footnote{执掌;治理。引申为做官。}职从政。存以甘棠\footnote{传说西周周公召南巡途中曾在甘棠树下理政,其品行风度让当地百姓深受感染。在周公召离去后,当地人民一直保留着这棵甘棠树,并创作《甘棠》诗吟咏他,以纪念其人其德。},去而益\footnote{增,更加。}咏。
\end{yuanwen}

\begin{yuanwen}
乐殊\footnote{不同,差异。}贵贱,礼别\footnote{区分,区别。}尊卑。上和下睦,夫唱妇随。
\end{yuanwen}

\begin{yuanwen}
外受傅\footnote{负责教导或传授技艺的人,此处指老师。}训,入奉母仪\footnote{礼节;礼仪;规范。}。诸姑伯叔,犹子比儿。
\end{yuanwen}

音乐划分贵贱,礼仪区别尊卑。在上者和在下者要和睦相处,丈夫和妻子之间要和谐相随。

学问做好了就可走上仕途,谋取职位参与政事为国效命。要学周公召甘棠树下理政,他离开后人们还作歌长久吟咏。

召公活着时曾在甘棠树下理政,他过世后老百姓对他更加怀念歌咏。

选择乐曲要根据人的身份贵贱有所不同;采用礼节要按照人的地位高低有所区别。

长辈和小辈要和睦相处,夫妇要一唱一随,协调和谐。

音乐要根据人们身分的贵贱而有所不同,礼节要根据人们地位的高低而有所区别。上下要和睦相处,夫妇要一唱一随,协调和谐。

在外接受师傅的训诲,在家遵从父母的教导。对待姑姑、伯伯、叔叔等长辈,要像是他们的亲生子女一样。
在外虚心接受老师教诲,在家恭谨遵从母亲家训。姑母伯父叔父都是长辈,要尊敬爱戴如同他们的亲生儿女。


在外面要听从师长的教诲,在家里要遵守母亲的规范。


\begin{yuanwen}
孔\footnote{大,很。}怀\footnote{爱戴,爱护。}兄弟,同气\footnote{气类相投。}连枝。交友投分\footnote{f\`en,情分相投。},切磨\footnote{切磋砥砺之意。}箴规\footnote{zh\=en,规劝。}。
\end{yuanwen}

对待姑姑、伯伯、叔叔等长辈,要像是他们的亲生子女一样。
兄弟之间要非常相爱,因为同受父母血气,犹如树枝相连。

\begin{yuanwen}
仁慈隐\footnote{怜悯。}恻\footnote{悲伤。},造次\footnote{匆忙,仓促。}弗离。节义廉退\footnote{谦让之意。},颠沛\footnote{穷困,受挫折。}匪亏\footnote{欠缺;短少。此处指丧失。}。
\end{yuanwen}

\begin{yuanwen}
性静情逸,心动神疲。守真\footnote{人之纯洁性情。}志满,逐物意移。
\end{yuanwen}

兄弟姐妹同出父母一脉,要相亲相爱如同相连的树枝。结交朋友要注重情投意合,学习上相互切磋,品行上相互勉励。

常怀仁慈恻隐之心,即便遇到危难都不放弃;常守节义廉退之德,即便颠沛流离也不丧失。


结交朋友要意相投,学习上切磋琢磨,品行上互相告勉。
仁义、慈爱,对人的恻隐之心,在最仓促、危急的情况下也不能抛离。
气节、正义、廉洁、谦让的美德,在最穷困潦倒的时候也不可亏缺。
品性沉静淡泊,情绪就安逸自在;内心浮躁好动,精神就疲惫困倦。
保持纯洁的天性,就会感到满足;追求物欲享受,天性就会转移改变。
兄弟之间要相互关心,因为同受父母血气,如同树枝相连。结交朋友要意气相投,要能学习上切磋琢磨,品行上互相告勉。

仁义、慈爱,对人的恻隐之心,在任何时候,任何地方都不能抛离。气节、正义、廉洁、谦让这些品德,在最穷困潦倒的时候也不可亏缺。

保持内心清静平定,情绪就会安逸舒适,心为外物所动,精神就会疲惫困倦。保持自己天生的善性,愿望就可以得到满足,追求物欲享受,善性就会转移改变。

\begin{yuanwen}
坚持雅\footnote{高尚,完美。}操\footnote{操守。},好爵\footnote{爵位,此处指职位。}自縻\footnote{m\'i,系住。}。都邑\footnote{都城。}华夏,东西二京\footnote{汉高祖刘邦建都于长安,史称西京;汉光武帝刘秀建都洛阳,史称东京。}。
\end{yuanwen}

\begin{yuanwen}
背邙\footnote{m\'ang,北邙山,位于河南洛阳北。}面\footnote{面临,面向。}洛\footnote{洛河,发源于陕西,流入河南。},浮\footnote{跨。}渭\footnote{渭河,发源于甘肃,经陕西流入黄河。}据\footnote{依傍。}泾\footnote{泾河,发源于宁夏,流入陕西。}。宫殿盘\footnote{弯曲。}郁\footnote{茂盛。},楼观\footnote{楼台之类。}飞惊。
\end{yuanwen}

坚持高尚铁情操,好的职位自然会为你所有。
心性平静,情绪便安逸自在;心性躁动,精神就紧张疲惫。保持纯洁性情内心就充实满足,追逐物欲享受意志便改变转移。坚持高尚情操,好的职位就会属于自己。
华夏有京城,东京和西京。洛阳背靠北邙,前临洛河,长安左跨渭河,右傍泾水。
宫殿重重叠叠,曲折盘旋,楼阁高耸欲飞,让人心惊。上面绘着飞禽走兽,彩画着仙人神灵。

古代的都城华美壮观,有东京洛阳和西京长安。

东京洛阳背靠北邙山,南临洛水;西京长安左跨渭河,右依泾水。

宫殿盘旋曲折,重重迭迭;楼阁高耸如飞,触目惊心。

坚定地保持着高雅情操,好的职位自然就会属于你。中国古代的都城华美壮观,有东京洛阳和西京长安。

洛阳北靠邙山,面临洛水;长安北横渭水,远据泾河。宫殿回环曲折,楼台宫阙凌空欲飞,使人心惊。

宫殿上绘着各种飞禽走兽,描画出五彩的天仙神灵。


\begin{yuanwen}
图写禽兽,画彩仙灵。丙舍\footnote{宫廷中的配殿。}旁启,甲帐\footnote{最好的帐幕。}对楹\footnote{堂屋前部的柱子。}。
\end{yuanwen}

\begin{yuanwen}
肆\footnote{铺陈之意。}筵\footnote{y\'an}设席,鼓瑟吹笙。升阶纳陛\footnote{殿前台阶。},弁\footnote{bi\`an,古代男子戴的帽子,此处指官帽。}转疑(移)星。
\end{yuanwen}

\begin{yuanwen}
右通广内\footnote{汉宫殿名,主要用于藏书,后泛指帝王的图书库。},左达承明\footnote{汉宫殿名,为朝臣休息的地方。}。既集坟典\footnote{即《三坟》、《五典》,传说中的古书名,其中,《三坟》记三皇之事,《五典》记五帝之事。此处泛指各种书籍。},亦聚群英。
\end{yuanwen}

\begin{yuanwen}
杜\footnote{指后汉书法家杜度,善草书。}稿钟\footnote{三国时魏书法家钟繇,善隶书。}隶,漆书\footnote{以漆书字于竹简之上。}壁经\footnote{传说秦始皇焚书时,孔子八世孙孔腾将《书经》藏于壁中。到了汉代,鲁共王在孔子旧宅中寻得,这就是《古文尚书》。漆书、壁经在这里泛指书籍之多。}。府罗\footnote{排列;罗列。}将相,路侠\footnote{通“夹”字。}槐卿\footnote{周朝时,在宫廷外面种植三槐九棘,公卿大夫分立其下,正面三槐为三公,两侧九棘为卿大夫。槐卿在此处泛指大臣。}。
\end{yuanwen}

正殿两边的配殿从侧面开启,豪华的账幕对着高高的楹柱。
宫殿中大摆宴席,乐人吹笙鼓瑟,一片歌舞升平的景象。
登上台阶进入殿堂的文武百官,帽子团团转,像满天的星星。
右面通向用以藏书的广内殿,左面到达朝臣休息的承明殿。
这里收藏了很多的典籍名著,也集着成群的文武英才。
书殿中有杜度的草书、钟繇的隶书,还有漆写的古籍和孔壁中的经典。
宫廷内将想依次排成两列,宫廷外大夫公卿夹道站立。
正殿两侧大门开启,珍珠宝玉装饰柱楹。美味佳肴布满筵席,鼓瑟吹笙庆祝升平。文武百官沿着台阶步入殿堂,官帽晃动像漫天的繁星。

宫殿里画着飞禽走兽,还有彩绘的天仙神灵。正殿两边的配殿从侧面开启,豪华的帐幕对着高高的楹柱。

宫殿里摆着酒席,弹琴吹笙一片欢腾。官员们上下台阶互相祝酒,珠帽转动,像满天的星斗。

右面通向用以藏书的广内殿,左面到达朝臣休息的承明殿。这里收藏了很多的典籍名著,也集着成群的文武英才。
宫殿两侧可通两座大殿,右边名广内,左边为承明。既收藏着很多典籍名著,也汇聚了众多雄才精英。

里边有杜度草书的手稿和钟繇隶书的真迹,有从汲(jí)县魏安厘王冢(zhōng)中发现掘出来的漆写古书,以及汉代鲁恭王在曲阜(fù)孔庙墙壁内发现的古文经书。宫延内将想依次排成两列,宫廷外大夫公卿夹道站立。
这里珍藏着杜度的草书、钟繇的隶书,还有漆书的竹简和孔子旧宅中的经文。宫廷里将相们排列成行,出行时大臣们夹道送行。

他们每家都有八县以上的封地,还有上千名的侍卫武装。


他们每户有八县之广的封地,配备成千以上的士兵。


\begin{yuanwen}
户封八县,家给千兵。高冠陪辇\footnote{皇帝乘坐的车子,此处代指皇帝。},驱毂\footnote{g\v{u},车轮,此处代指车辆。}振\footnote{动。}缨\footnote{系帽的带子。}。
\end{yuanwen}

\begin{yuanwen}
世\footnote{世袭。}禄侈\footnote{ch\v{i}}富,车驾\footnote{此处代指马。}肥轻。策\footnote{谋划。}功茂\footnote{茂盛。}实,勒\footnote{刻。}碑刻铭\footnote{刻在器物上歌功颂德的文字。}。
\end{yuanwen}

\begin{yuanwen}
磻溪\footnote{p\'an,故址在今陕西省宝鸡县东南,传说中姜太公吕尚钓鱼处。周文王在磻溪与姜太公相遇,聘其为军师。后来姜太公帮助周武王灭掉商朝。磻溪在这里代指姜太公。}伊尹\footnote{商汤王宰相。},佐时阿衡\footnote{商朝官名。阿意为倚,衡意为平。}。奄\footnote{y\v{a}n,取得。}宅\footnote{居住。}曲阜,微\footnote{没有。}旦\footnote{周公旦。}孰营。
\end{yuanwen}

他们戴着高高的官帽,陪着皇帝出游,驾着车马,帽带飘舞着,好不威风。
他们的子孙世代领受俸禄,奢侈豪富,出门时轻车肥马,春风得意。
朝廷还详尽确实地记载他们的功德,刻在碑石上流传后世。
周武王磻溪遇吕尚,尊他为“太公望”;伊尹辅佐时政,商汤王封他为“阿衡”。
周成王占领了古奄国曲阜一带地面,要不是周公旦辅政哪里能成?
戴着高大帽子的官员们陪着皇帝出游,驾着车马,帽带飘舞着,好不威风。

他们的子孙世代领受俸禄,奢侈豪富,出门时轻车肥马,春风得意。 

朝廷还详尽确实地记载他们的功德,刻在碑石上流传后世。

周武王磻溪遇吕尚,尊他为“太公望”;伊尹辅佐时政,商汤王封他为“阿衡”。周成王占领了古奄国曲阜一带地面,要不是周公旦辅政哪里能成?
皇上封赏每人八县的土地,并且配给成千的士兵。官员们头戴高高的官帽陪着皇帝出巡,驱动车马飞奔,帽带随风飘动。
他们的子孙世袭俸禄,生活奢侈,出门时轻车肥马。他们为国家出谋划策,功勋卓著,石碑和器物上的铭刻,称诵着他们的功绩。

姜子牙与伊尹,都是辅佐君主匡时济世的一代名臣。封地曲阜,如果没有周公旦有谁能把鲁国来经营?


\begin{yuanwen}
桓公\footnote{春秋时期齐国君主,春秋五霸之一。}匡\footnote{匡正,帮助。}合\footnote{指汇合诸侯。},济\footnote{接济,救援。}若\footnote{同“弱”。}扶\footnote{扶持,扶助。}倾。绮\footnote{q\v{i},即汉初高士绮里季。秦末天下大乱,绮里季与东园公、夏黄公和甪里先生等三位老人避乱于商山,称“商山四皓”。汉高祖曾想废掉太子,张良请来“商山四皓”游说,使汉高祖最终打消了废太子的念头。}回汉惠\footnote{指汉惠帝,即前文所述汉高祖曾经想废的太子,他于汉高祖死后即位。},说\footnote{yu\`e,商朝武丁时期贤相傅说。传说他隐居在傅岩,武丁梦上帝赐予贤相,醒来后将梦中所见绘成图画,最后找到在山中采石的傅说,遂拜为相。在傅说的辅佐下,商朝得以中兴。}感武丁\footnote{商朝君主。}。
\end{yuanwen}

\begin{yuanwen}
俊乂\footnote{泛指德才兼备之人。古人以在千人之上的英才为俊,百人之上的英才为乂。y\`i}密勿\footnote{勤勉之意。},多士寔\footnote{sh\'i,通“是”。}宁\footnote{安定;安宁。}。晋楚更\footnote{更替;交替。}霸,赵魏困横\footnote{连横。战国时期,苏秦游说六国(赵、魏、韩、齐、楚、燕)合纵以抗秦,后来张仪则游说六国以事秦。}。
\end{yuanwen}

\begin{yuanwen}
假途灭虢\footnote{gu\'o,春秋时期,晋献公欲伐虢国,途中需经过虞国,晋献公用谋臣荀息之计,以宝玉、宝马买通了虞国国君,晋军得以通过虞国。灭掉虢国后,晋军在回师途中又灭掉虞国。},践土会盟\footnote{践土,地名,在今河南省开封市附近。春秋时期,晋文公于城濮之战击败楚军后,在践土会合各国诸侯,相约效命于周王室。}。何\footnote{汉高祖时期丞相萧何。}遵约法\footnote{秦末,刘邦在攻入咸阳后,与父老约法三章,“杀人者死,伤人及盗抵罪。”},韩\footnote{战国时期思想家韩非。}弊烦刑。
\end{yuanwen}

\begin{yuanwen}
起\footnote{战国时期秦国名将白起。}翦\footnote{ji\v{a}n,战国时期秦国名将王翦。}颇\footnote{战国时期赵国名将廉颇。}牧\footnote{战国时期赵国名将李牧。},用军最精。宣威沙漠,驰\footnote{传播。}誉丹青\footnote{红色或青色的颜色,借指绘画;史册,史籍。}。
\end{yuanwen}

\begin{yuanwen}
九州\footnote{传说中的我国上古行政区划,即冀州、豫州、雍州、扬州、兖州、徐州、梁州、青州、荆州,后用作中国的代称。}禹迹,百郡\footnote{秦始皇统一中国后,将当时的疆土分为三十六郡,汉朝则进一步分为一百零三郡,此处代指天下。}秦并。岳\footnote{五岳,即东岳泰山,西岳华山,南岳衡山,北岳恒山,中岳嵩山。}宗\footnote{尊。}泰岱\footnote{即泰山。},禅\footnote{封禅。}主\footnote{主持。}云亭\footnote{云山和亭山,都是泰山之下的小山。}。
\end{yuanwen}

\begin{yuanwen}
雁门\footnote{雁门关,在今山西代县。}紫塞\footnote{长城的别称。秦始皇筑长城,土色为紫色,故称紫塞。},鸡田\footnote{古代驿站名,在今宁夏灵武县。}赤诚\footnote{传说为蚩尤居住之处,在今河北境内。}。昆池\footnote{即昆明滇池。}碣石\footnote{山名,在今河北昌黎县境内。},钜野\footnote{j\`u,湖名,在今山东巨野县,已干涸。}洞庭\footnote{湖名,在今湖南省境内。}。
\end{yuanwen}

\begin{yuanwen}
旷\footnote{空而宽阔。}远绵邈\footnote{遥远苍茫。},岩岫\footnote{xi\`u,山洞;山。}杳\footnote{y\v{a}o,深。}冥\footnote{昏暗。}。治本于农,务兹(资)稼穑\footnote{s\`e,种植和收割,泛指农业劳动。}。
\end{yuanwen}

齐桓公九次会合诸侯,帮助弱小诸侯国使其免遭灭顶。绮里季挽回了汉惠帝的王位,傅说以治国才能感动武丁。
英才勤勉工作是国家的福气,正是他们使国家获得安宁。晋楚两国相继称霸,赵魏两国先后受困于连横。

齐桓公九次会合诸侯,出兵援助势单力薄和面临危亡的诸侯小国。
汉惠帝做太子时靠绮里季才幸免废黜,商君武丁感梦而得贤相传说。
能人治政勤勉努力,全靠许多这样的贤士,国家才富强安宁。
晋、楚两国在齐之后称霸,赵、魏两国因连横而受困于秦。
晋献公向虞国借路去消灭虢国;晋文公在践土与诸侯会盟,推为盟主。
萧何遵循简约刑法的精神制订九律,韩非却受困于自己所主张的严酷刑法。
晋献公向虞国借路讨伐虢国,晋文公在践土地方会盟诸侯,为周王室效命。萧何遵循高祖的约法三章,成功治理,韩非主张苛酷刑法,自己却在严刑中丧命。
白起、王翦、廉颇和李牧,带兵有方,用兵如神。他们的声威远播沙漠,他们的英名史册永存。

秦将白起、王翦,赵将廉颇、李牧,带兵打仗最为高明。
他们的声威远传到沙漠边地,美誉和画像一起流芳后代。
九州处处有留有大禹治水的足迹,全国各郡在秦并六国后归于统一。
五岳中人们最尊崇东岳泰山,历代帝王都在云山和亭山主持禅礼。
名关有北疆雁门,要塞有万里长城,驿站有边地鸡田,奇山有天台赤城。
赏池赴昆明滇池,观海临河北碣石,看泽去山东钜野,望湖上湖南洞庭。

齐桓公匡正天下诸侯,都打着“帮助弱小”、“拯救危亡”的旗号。汉惠帝做太子时靠绮里季才幸免废黜,商君武丁感梦而得贤相传说。

贤才的勤奋谨慎,换来了百官的各安其位。晋文公、楚庄王先后称霸,赵国、魏国受困于连横。

晋国向虞国借路去消灭虢国,晋文公在践土召集诸侯歃血会盟。萧何遵奉汉高祖简约的法律,韩非惨死在他自己所主张的苛刑之下。

秦将白起、王翦,赵将廉颇、李牧,用兵作战最为精通。他们的声威远扬到北方的沙漠,美名和肖像永远流传在千古史册之中。

九州之内都留下了大禹治水的足迹,全国各郡在秦并六国后归于统一。五岳以泰山为尊,历代帝王都在云山和亭山主持禅礼。

名关有北疆雁门,要塞有万里长城,驿站有边地鸡田,奇山有天台赤城。赏池赴昆明滇池,观海临河北碣石,看泽去山东巨野,望湖上湖南洞庭。
九州遍布大禹的足迹,秦始皇统一天下百郡。五岳之尊为泰山,天子封禅在云亭。

江河源远流长,湖海宽广无边。名山奇谷幽深秀丽,气象万千。
名关有雁门,要塞有长城,驿站有鸡田,奇山有赤城。赏池去滇池,眺海临碣石,观泽奔巨野,看湖下洞庭。苍茫大地辽阔无涯,山峦起伏壮丽幽深。


\begin{yuanwen}
俶\footnote{ch\`u,开始;整理。}载\footnote{从事。}南亩\footnote{出自《诗经·大田》:“以我覃耜,俶载南亩。”此处代指田地。},我艺\footnote{种植。}黍稷\footnote{sh\v{u} j\`i,黄米和小米,此处代指百谷。}。税熟贡新,劝\footnote{勉励。}赏黜陟\footnote{zh\`i,官员的降职为黜,升职为陟,此处代指处罚或奖励。}。
\end{yuanwen}

\begin{yuanwen}
孟轲\footnote{孟子。}敦\footnote{崇尚,纯厚。}素\footnote{纯洁;精纯。},史鱼\footnote{史鰌,字子鱼,春秋末年卫国史官。}秉\footnote{秉持。}直。庶几\footnote{差不多。}中庸\footnote{不偏不倚。},劳\footnote{勤。}谦\footnote{谦虚;谦恭。}谨敕\footnote{ch\`i,戒除。}。
\end{yuanwen}

\begin{yuanwen}
聆\footnote{l\'ing}音察理,鉴貌辨色。贻\footnote{遗留;赠送。}厥\footnote{ju\'e,其;他的。}嘉\footnote{善;美好。}猷\footnote{y\'ou,计划;谋划。},勉其祗\footnote{zh\=i,恭敬。}植\footnote{树立。}。
\end{yuanwen}

治国的根本在发展农业,要努力做好播种收获这些农活。
一年的农活该开始干起来了,我种上小米,又种上高粱。
收获季节,用刚熟的新谷交纳税粮,官府应按农户的贡献大小给予奖励或处罚。
孟轲夫子崇尚纯洁,史官子鱼秉性刚直。
做人要尽可能合乎中庸的标准,勤奋、谦逊、谨慎,懂得规劝告诫自己。
中国的土地辽阔遥远,没有穷极,名山奇谷幽深秀丽,气象万千。把农业作为治国的根本,一定要做好播种与收获。

一年的农活该开始干起来了,种植着小米和黄米。收获季节,用刚熟的新谷交纳税粮,庄稼种得好的受到表彰和赏赐,种得不好的就要受到处罚。
治国之本在于农,稼穑之事须上心。辛勤劳作在农田,黄米小米忙播种。粮食丰收缴新税,官府赏罚最分明。

孟子崇尚朴素,史官子鱼秉性刚直。 做人要尽可能合乎中庸的标准,必须勤劳谦逊,谨慎检点,懂得规劝告诫自己。
孟子崇尚纯洁,史鱼追求刚直,可算近乎中庸的道理,再加上勤奋谦虚,谨慎自励。

听人说话要审察其中的道理,看人容貌要看出他的心情。

听别人说话,要仔细审察是否合理;看别人面孔,要小心辨析他的脸色。


\begin{yuanwen}
省\footnote{x\v{i}ng}躬讥诫\footnote{儆戒。},宠\footnote{尊荣;尊贵。}增\footnote{益;更加。}抗\footnote{益;更加。}极\footnote{尽,最终。}。殆\footnote{d\`ai}辱近耻,林皋\footnote{g\=ao,水边的高地。}幸即\footnote{就。}。
\end{yuanwen}

\begin{yuanwen}
两疏\footnote{指西汉宣帝时太子太傅疏广,太子少傅疏受,两人均以年老辞位而归。}见机\footnote{征兆;预兆。},解\footnote{解脱。}组\footnote{绶;印绂。}谁逼?索\footnote{萧索;孤独。}居闲\footnote{悠闲。}处,沉默寂寥\footnote{寂静;空旷;清静无为。}。
\end{yuanwen}

\begin{yuanwen}
求\footnote{寻觅。}古寻论,散\footnote{排解;消散。}虑\footnote{思虑;思念。}逍遥。欣\footnote{欢喜。}奏\footnote{凑;聚集。}累\footnote{操劳;费力。}遣\footnote{打发;消除。},戚\footnote{忧郁;忧愁。}谢\footnote{辞去;拒绝。}欢招。
\end{yuanwen}

\begin{yuanwen}
渠\footnote{人工开凿的水道,此处指池水。}荷的历\footnote{光亮灿烂的样子。},园莽\footnote{密生的草。}抽条。枇杷晚翠,梧桐蚤\footnote{通“早”。}凋。
\end{yuanwen}

听人说话明辨是非,看人面孔辨析心理。虚心学习他人美好品德,勉励他人谨慎立身处世。

要给人家留下正确高明的忠告或建议,勉励别人谨慎小心地处世立身。
听到别人的讥讽告诫,要反省自身;备受恩宠不要得意忘形,对抗权尊。
知道有危险耻辱的事快要发生,还不如归隐山林为好。
疏广疏受预见到危患的苗头才告老还乡,哪里有谁逼他们除下官印?
离群独居,悠闲度日,整天不用多费唇舌,清静无为岂不是好事。
想想古人的话,翻翻古人的书,消往日的忧虑,乐得逍遥舒服。
轻松的事凑到一起,费力的事丢在一边,消除不尽的烦恼,得来无限的快乐。
池里的荷花开得光润鲜艳,园中的草木抽出条条嫩枝。

要给人家留下正确高明的忠告或建议,勉励别人谨慎小心地处世立身。

听到别人的讥讽告诫,要反省自身;备受恩宠不要得意忘形,对抗权尊。如果知道有危险耻辱的事快要发生就退隐山林,还可以幸免于祸。
听到别人的指责讽刺,要反省自己。受到恩宠过多,被抬举重用过甚,可能会有危险耻辱之事来临,避居山林即可平安度日。疏广疏受预见危险便急流勇退,辞官归隐有谁逼迫?
甘于离群独居,悠闲度日,心境沉默寂静,探求古书中的哲理,寻觅其精辟高论,忧愁消散,乐得自在逍遥。

汉代疏广、疏受叔侄。

探求古人古事,读点至理名言,就可以排除杂念,自在逍遥。轻松的事凑到一起,费力的事丢在一边,消除不尽的烦恼,得来无限的快乐。
高兴的事接踵而来,劳神的事抛到脑后,远离了忧郁,欢欣就在身边围绕。池塘荷花,开得艳丽灿烂;园中草木,竞显鲜嫩枝条。枇杷到了岁末,仍然青翠欲滴;梧桐一逢秋日,便尽失妖娆。

池塘中的荷花开得多么鲜艳,园林内的青草抽出嫩芽。到了冬天枇杷叶子还是绿的,梧桐一到秋天叶子就凋了。

枇杷到了岁晚还是苍翠欲滴,梧桐刚刚交秋就早早地凋谢了。


\begin{yuanwen}
陈根委\footnote{曲折;抛弃。}翳\footnote{y\`i,遮蔽;树木枯死。},落叶飘摇。游鹍\footnote{k\=un,同“鲲”,鸟名,鲲鹏,传说中的大鸟。}独运\footnote{转动。},淩(凌)\footnote{升高,在空中。}摩\footnote{迫近。}绛\footnote{深红色。}霄\footnote{云;天空。}。
\end{yuanwen}

\begin{yuanwen}
耽读玩市\footnote{东汉哲学家王充因家贫买不起书,经常在书铺里站着读书,以至于废寝忘食。耽:沉溺;入迷。},寓目\footnote{过目。}囊箱。易輶\footnote{y\'ou,轻。}攸\footnote{所。}畏,属\footnote{进。}耳垣\footnote{yu\'an,墙。}墙。
\end{yuanwen}

\begin{yuanwen}
具\footnote{备;办。}膳餐饭,适口充肠。饱饫\footnote{y\`u,饱食;饱足。}烹宰,饥厌\footnote{满足。}糟糠\footnote{酒糟、米糠等粗劣食物。}。
\end{yuanwen}

\begin{yuanwen}
亲戚故旧,老少异粮。妾御\footnote{管理;掌管。}绩纺\footnote{绩麻纺纱。},侍巾帷房\footnote{指夫妻居住的内室。}。
\end{yuanwen}

陈根老树枯倒伏,落叶在秋风里四处飘荡。
寒秋之中,鲲鹏独自高飞,直冲布满彩霞的云霄。
树根陈腐,老树倒地,风吹落叶,空中飘摇。鲲鹏在寒风中独往独来,展翅凌空,冲向彩霞密布的九霄。

汉代王充在街市上沉迷留恋于读书,眼睛注视的都是书袋和书箱。
说话最怕旁若无人,毫无禁忌;要留心隔着墙壁有人在贴耳偷听。
安排一日三餐的膳食,要适合各位的口味,能让大家吃饱。
的时候自然满足于大鱼大肉,饿的时候应当满足于粗菜淡饭。
亲属、朋友会面要盛情款待,老人、小孩的食物应和自己不同。

老树根蜿蜒曲折,落叶在秋风里四处飘荡。只有远游的鲲鹏独立翱翔,直冲布满彩霞的云霄。

汉代王充在街市上沉迷留恋于读书,眼睛注视的全是书袋和书籍。换了轻便的车子要注意危险,说话要防止隔墙有耳。
读书该像王充那样沉迷,满目都是书袋和书箱。说话谨慎加小心,隔墙有耳应提防。

平时的饭菜,要适合口味,让人吃得饱。饱的时候自然满足于大鱼大肉,饿的时候应当满足于粗菜淡饭。
安排膳食不要十分讲究,适合口味能充饥肠就行。饱时不满足于鱼肉,饥时不嫌弃糟糠。


亲属、朋友会面要盛情款待,老人、小孩的食物应和自己不同。小妾婢女要管理好家务,尽心恭敬地服待好主人。
亲戚故旧盛情来款待,老人小孩食物要不同。侍妾勤勉纺绩,侍候主人。


小妾婢女要管理好家务,尽心恭敬地服待好主人。


\begin{yuanwen}
纨扇\footnote{w\'an,很细的丝织品;细绢。纨扇,用细绢制成的团扇。}圆洁,银烛炜煌\footnote{w\v{e}i hu\'ang,火光炫耀的样子。}。昼眠夕寐,蓝笋\footnote{青色。蓝笋,用青篾编成的竹席。}象床\footnote{用象牙装饰的床。}。
\end{yuanwen}

\begin{yuanwen}
弦歌酒宴,接杯举觞\footnote{sh\=ang,一种酒器。}。矫\footnote{高举。}手顿足,悦豫\footnote{快乐。}且康\footnote{安乐。}。
\end{yuanwen}

\begin{yuanwen}
嫡\footnote{宗法制度下指家庭的正支(跟“庶”相对)。}后嗣\footnote{接续;继承。}续,祭祀\footnote{旧俗备供品向神佛或祖先行礼,表示崇敬并求保佑。}烝尝\footnote{祭祀之名。秋天祭祀为尝,冬天祭祀为烝。}。稽颡\footnote{q\v{i} s\v{a}ng,以额至地。颡:额,脑门子。}再拜,悚\footnote{恐惧。}惧恐惶。
\end{yuanwen}

细绢制成的团扇浑圆洁白,银色的烛台灯火辉煌。白日休憩,夜晚安歇,有青篾编成的竹席和象牙镶嵌的寝床。

绢制的团扇像满月一样又白又圆,银色的烛台上烛火辉煌。
白日小憩,晚上就寝,有青篾编成的竹席和象牙雕屏的床榻

酒宴上弦歌不绝于耳,美酒一觞接一觞。客人兴高采烈手舞足蹈,大家纵情欢乐享受美好时光。

奏着乐,唱着歌,摆酒开宴;接过酒杯,开怀畅饮。
情不自禁地手舞足蹈,真是又快乐又安康。
子孙继承了祖先的基业,一年四季的祭祀大礼不能疏忘。
嫡子继承先祖事业,祭祀大事不能遗忘。以额触地一拜再拜,表达敬意诚惶诚恐。

圆圆的绢扇洁白素雅,白白的蜡烛明亮辉煌。白日小憩,晚上就寝,有青篾编成的竹席和象牙雕屏的床榻。

奏着乐,唱着歌,摆酒开宴;接过酒杯,开怀畅饮。情不自禁地手舞足蹈,真是又快乐又安康。

子孙一代一代传续,四时祭祀不能懈怠。跪着磕头,拜了又拜;礼仪要周全恭敬,心情要悲痛虔诚。

跪着磕头,拜了又拜;礼仪要周全恭敬,心情要悲痛虔诚。


\begin{yuanwen}
笺\footnote{ji\=an,注解;写信或题词用的纸。}牒\footnote{di\'e,文书或证件;簿册。}简要,顾答(达)\footnote{回答。}审\footnote{详细;审查;知道。}详。骸\footnote{h\'ai,借指身体。}垢\footnote{污秽;肮脏。}想浴,执\footnote{持;拿着。}热愿凉。
\end{yuanwen}

\begin{yuanwen}
驴骡犊\footnote{d\'u,小牛。}特\footnote{公牛。},骇\footnote{h\`ai,惊吓;震惊。}跃超骧\footnote{xi\=ang,马奔跑。}。诛\footnote{杀。}斩贼盗,捕获叛亡\footnote{逃跑。}。
\end{yuanwen}

\begin{yuanwen}
布\footnote{东汉末年名将吕布,擅长射箭。}射僚\footnote{春秋时期楚国人熊宜僚,擅长玩弹丸。}丸,嵇\footnote{j\=i,三国时期魏国文学家嵇康,擅长弹琴。}琴阮\footnote{三国时期魏国文学家阮籍,专擅长啸。}箫(啸)。恬\footnote{ti\'an,战国时期秦国名将蒙恬,传说他发明了毛笔。}笔伦\footnote{东汉发明家蔡伦,他发明了造纸术。}纸,钧\footnote{三国时期魏国人马钧,他改进了织绫机,制造了指南车、水车,发明了一种攻城用的转轮式发石机。}巧任\footnote{任公子,传说中的人物,擅长钓大鱼。事见《庄子》。}钓。
\end{yuanwen}

\begin{yuanwen}
释\footnote{消除。}纷利俗\footnote{大众的。},并(竝)\footnote{b\`ing}皆佳妙。毛\footnote{战国时期美女毛嫱。}施\footnote{春秋时期越国美女西施。}淑\footnote{温和善良;美好。}姿,工\footnote{长于;善于。}颦\footnote{p\'in,皱眉。}妍\footnote{y\'an,美丽。}笑。
\end{yuanwen}

给别人写信要简明扼要,回答别人问题要详细周全。
身上有了污垢,就想洗澡,好比手上拿着烫的东西就希望有风把它吹凉。
家里有了灾祸,连牲畜都会受惊,狂蹦乱跳,东奔西跑。
对抢劫、偷窃、反叛、逃亡的人要严厉惩罚,该抓的抓,该杀的杀。
吕布擅长射箭,宜僚有弄丸的绝活,嵇康善于弹琴,阮籍能撮口长啸。
蒙恬造出毛笔,蔡伦发明造纸,马钧巧制水车,任公子垂钓大鱼。
他们的技艺有的解人纠纷,有的方便群众,都高明巧妙,为人称道。


给人的书信要简明扼要,回答别人的问题时要审慎周详。身上脏了就想洗个澡,捧着热东西就希望它有风把它吹凉。
撰写书信要简明扼要,回答问题要准确周详。身体有了污垢自然想要沐浴,接触发热物体便愿尽快清凉。

家里有了灾祸,连驴子、骡子,大小牲口都会受惊,狂蹦乱跳,东奔西跑。官府诛杀盗贼,捕获叛乱分子和亡命之徒。
驴子骡子牛犊虽是牲畜,灾祸降临也会奔跑发狂。要使社会安定,家庭安宁,必须诛杀盗贼,捕获叛亡。

吕布善于射箭,宜僚善玩弹丸,嵇康善于弹琴,阮籍善于撮口长啸。蒙恬制造了毛笔,蔡伦发明了造纸,马钧发明了水车,任公子善于钓鱼。
吕布擅长射箭,宜僚长于弹丸,嵇康精于弹琴,阮籍专擅长啸。蒙恬创造毛笔,蔡伦发明纸张,马钧制指南车,任公子钓术高。

他们或者善于为人解决纠纷,或者善于发明创造有利于社会,这些都非常巧妙。毛嫱、西施,姿容姣美,哪怕皱着眉头,也像美美的笑。

毛嫱、西施年轻美貌,哪怕皱着眉头,也像美美的笑。
消除纠纷方便大众,受人拥戴为人称道。毛嫱西施姿容美好,即使皱眉亦如娇笑。

\begin{yuanwen}
年矢\footnote{箭。}每\footnote{经常。}催,曦\footnote{x\=i,指清晨的阳光。}晖\footnote{hu\=i,阳光。}朗曜\footnote{y\`ao,日光;照耀。}。璇玑\footnote{xu\'an j\=i,古代称北斗星的第一颗星至第四颗星。}悬斡\footnote{w\`o,旋转。},晦\footnote{hu\`i,昏暗;不明显;夜晚。}魄\footnote{晦魄:此处指月亮。}环照。
\end{yuanwen}

\begin{yuanwen}
指\footnote{通“旨”,示。}薪\footnote{柴火。}修祜\footnote{h\`u,福。},永绥\footnote{su\'i,安好。}吉劭\footnote{sh\`ao,劝勉。}。矩步\footnote{画直角或正方形、矩形用的曲尺。矩步:方步。}引领,俯仰\footnote{低头和抬头,此处引申为周旋、应付。}廊庙\footnote{此处代指朝廷。}。
\end{yuanwen}

可惜青春易逝,岁月匆匆催人渐老,只有太阳的光辉永远朗照。
高悬的北斗随着四季变换转动,明晦的月光洒遍人间每个角落。
行善积德才能像薪尽火传那样精神长存,子孙安康全靠你留下吉祥的忠告。
如此心地坦然,方可以昂头迈步,应付朝廷委以的重任。

青春易逝,岁月匆匆催人渐老,只有太阳的光辉永远朗照。 高悬的北斗随着四季变换转动,明晦的月光洒遍人间每个角落。
时光如箭催人渐老,只有阳光永远辉耀。北斗高悬旋转不息,月光明暗环球普照。

顺应自然,修德积福,永远平安,多么美好。如此心地坦然,方可以昂头迈步,一举一动都象在神圣的庙宇中一样仪表庄重。
积德行善薪火相传自然修来福分,子孙赖以永远安康吉祥美好。行走从容得体,方可俯仰自如,穿戴端正庄重,才能高瞻远眺。

如此无愧人生,尽可以整束衣冠,庄重从容地高瞻远望。
孤陋寡闻一如愚昧无知,两者同样受到讥讽嘲笑。这些道理孤陋寡闻就不会明白,只能和愚味无知的人一样空活一世,让人耻笑。

\begin{yuanwen}
束带矜\footnote{j\=in,慎重;拘谨。}庄\footnote{庄重。},徘徊瞻\footnote{zh\=an}眺\footnote{ti\`ao}。孤陋寡闻,愚蒙等\footnote{同样。}诮\footnote{qi\`ao,责备;讥讽。}。
\end{yuanwen}

衣带穿着整齐端庄,举止从容,高瞻远瞩。这些道理孤陋寡闻就不会明白,只能和愚味无知的人一样空活一世,让人耻笑。

\begin{yuanwen}
谓语助者,焉哉乎也。
\end{yuanwen}

语气助词,“焉”、“哉”、“乎”、“也”。

\end{document}