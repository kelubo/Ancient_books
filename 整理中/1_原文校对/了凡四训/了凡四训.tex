% 了凡四训
% 了凡四训.tex

\documentclass[12pt,UTF8]{ctexbook}

% 设置纸张信息。
\usepackage[a4paper,twoside]{geometry}
\geometry{
	left=25mm,
	right=25mm,
	bottom=25.4mm,
	bindingoffset=10mm
}

% 设置字体,并解决显示难检字问题。
\xeCJKsetup{AutoFallBack=true}
\setCJKmainfont{SimSun}[BoldFont=SimHei, ItalicFont=KaiTi, FallBack=SimSun-ExtB]

% 目录 chapter 级别加点(.)。
\usepackage{titletoc}
\titlecontents{chapter}[0pt]{\vspace{3mm}\bf\addvspace{2pt}\filright}{\contentspush{\thecontentslabel\hspace{0.8em}}}{}{\titlerule*[8pt]{.}\contentspage}

% 设置 part 和 chapter 标题格式。
\ctexset{
	part/name= {第,卷},
	part/number={\chinese{part}},
	chapter/name={第,篇},
	chapter/number={\chinese{chapter}}
}

% 图片相关设置。
\usepackage{graphicx}
\graphicspath{{Images/}}

% 设置古文原文格式。
\newenvironment{yuanwen}{\bfseries\zihao{4}}

% 设置署名格式。
\newenvironment{shuming}{\hfill\bfseries\zihao{4}}

% 注脚每页重新编号,避免编号过大。
\usepackage[perpage]{footmisc}

\title{\heiti\zihao{0} 了凡四训}
\author{袁了凡}
\date{明}

\begin{document}

\maketitle
\tableofcontents

\frontmatter

\chapter{前言}

《了凡四训》为明代袁了凡,于六十九岁时所作,以此来教戒他的儿子袁天启,认识命运的真相,明辨善恶的标准,改过迁善。

全文分“立命之学”、“改过之法”、“积善之方”、“谦德之效”四个部分。文章篇幅虽然短小,但是寓理内涵深刻,兼融儒释道三家思想,尽现真善美中华文化,论证“种瓜得瓜”、“善有善报”、“积极进取”、“有愿皆成”的道理。平实而无虚华,深奥而不迷信。所以数百年来历久不衰,时至今日。仍然被人们广为传颂,脍炙人口。

\mainmatter

\chapter{立命之学}

“立命之学”是袁了凡“四训”中的第一篇,是他晚年总结人生经验,训诫儿子的《立命篇》。在这一篇中,了凡先生把他自己改造命运的经过,同他所看到的一些改造命运的人的各种效验,来论述“命由我作,福自己求”的思想。

了凡先生在“立命之学”中,意欲让自己的儿子明白命运是可以改变的,要自己把握住自己的命运,并要建立改造命运的信心。告诉他不要被“命”束缚住,要竭力去做各种善事,不可以做坏事,从而去获得一个快乐美满的人生。

\begin{yuanwen}
余童年丧父,老母命弃举业\footnote{为应科举考试而准备的学业,旨在求取功名。}学医,谓可以养生\footnote{养活自己和家庭。},可以济人\footnote{以医术救济别人。},且习一艺以成名,尔父夙心\footnote{平素的心愿。}也。
\end{yuanwen}

我在童年的时候就失去了父亲,母亲让我弃文从医,她说学医可以养活家庭,也可以救济别人,而且精通一技之长能够以此成名,这也是父亲生前的夙愿。

了凡先生自称在其童年之时父亲就不幸去世了,只得与母亲相依为命,父亲的过早离世使家境陷入困顿。为了维持生计,母亲要求他放弃读书考取功名的举业,改学医术,这样既可以养家糊口,又可以悬壶济世、治病救人。

古时读书人始终是以步入仕途、兼济天下为人生最高旨趣的。不过科举进仕并非易事,其路途可谓漫长而艰辛。从生存的角度考虑,学医不失为一个切实可行的办法。并且习得一技之长,技艺精湛而成为一代名医,这也是他父亲生前的夙愿。其实,治病救人与读书救国在古往今来的知识分子的心目中往往是有相通之处的。古有“不为良相,便为良医”之说,二者最关乎民生痛痒。宋朝名相范仲淹的志向就是做宰相和医生,言唯有此二者能救人,后来果然位列朝班,却能居庙堂之高,处江湖之远,“先天下之忧而忧,后天下之乐而乐”。近代知识分子、新文化运动旗手鲁迅早年也曾远渡东瀛、立志习医以救国民,后又弃医从文,投身到唤醒国人精神与灵魂的战斗中去。从治疗人的躯体和生命到心怀家国天下,这是中国历来知识分子们内在所具有的精神品格。

童年时期的了凡先生,就这样听从母亲的劝告,放弃做官的念头而改为学医。

\begin{yuanwen}
后余在慈云寺\footnote{原为中国古代官署名。后佛教用以称僧众供佛和聚居修行的处所,在我国主要指佛寺。},遇一老者,修髯\footnote{rán,两侧面颊腮部的胡子,也泛指胡子。古人有长髯、美髯、白发苍髯等说法,古人认为髯的多少及色泽好坏与血气盛衰有关。}伟貌,飘飘若仙,余敬礼之。
\end{yuanwen}

后来我在慈云寺,遇到一个老人,长须飘飘、相貌堂堂,飘飘然宛若神仙一样,我对他非常尊敬并以礼相待。

了凡先生为什么会到寺院来,表面上看来似乎是偶然和巧合,其实历来古代的知识分子们大都喜欢流连于寺院。清幽的古刹往往是居住、读书的绝佳之境。宋代大文豪苏轼就曾有《宿蟠桃寺》诗云:“板阁独眠惊旅枕,木鱼晓动随僧粥。”古代文人和僧人常有交往,诗歌唱和,书画过从。当然,也有僧人嫌弃落魄文人寄居寺庙白吃白住的,在历史上也是有此趣闻典故的。

在人生行进的道路上,经常会遇到一些令人生发生转折和改变的人。了凡先生在慈云寺便碰见了这样一位老人。老人长得相貌魁伟,仙风道骨,更有一捧长长的须髯。了凡先生见到此老者一派飘飘欲仙的模样,不敢怠慢,连忙行礼以示恭敬。

\begin{yuanwen}
语余曰:“子仕路\footnote{指做官的途径。}中人也,明年即进学\footnote{科举时,童生参加岁试,被录取入府县学肄业,称为进学。进学的童生被称为秀才。},何不读书?”

余告以故,并叩老者姓氏里居。

曰:“吾姓孔,云南人也。得邵子\footnote{即邵雍,字尧夫,谥康节。邵雍本是中国思想史上一位著名的易学家,他以《易传》为基础,以象数为中心,以易图为张本,创立先天易学,并在此基础上,用元、会、运、世等概念来推算天地的演化和历史的循环。邵雍在世时便以“遇事能前知”而名声在外,民间流传的“二雀争梅”与“邻人借斧”的故事,就是他即兴占卜而应验的佳话。其代表作有《皇极经世书》(《观物内外篇》和《渔樵问对》)、《伊川击壤集》等。}皇极数正传,数该传汝。”

余引之归,告母。母曰:“善待之。”

试其数,纤悉\footnote{细微详尽。}皆验。余遂启读书之念,谋之表兄沈称,言:“郁海谷先生,在沈友夫家开馆\footnote{开设学馆教授生徒。},我送汝寄学甚便。”

余遂礼郁为师。
\end{yuanwen}

老人对我说:“你注定是仕途上的人啊,明年就可以参加岁试进入学校当秀才了,不知你现在为什么不读书呢?”

我就将其中的缘由据实告诉了他,并且询问老人的姓名和籍贯。

他说:“我姓孔,是云南人。得到了邵雍先生《皇极经世书》的正统传授,命中注定应该再传授给你。”

我把他请回家,并禀告了老母亲。母亲说:“你要好好跟他学习。”

我们多次试验他的占卜之术,事无大小都能应验。我这才开始有了读书的念头,与表兄沈称商量,表兄说:“郁海谷先生正在沈友夫家中开馆授徒,我送你到那里跟他们一起读书也很方便。”

于是我便拜郁海谷先生为师了。

老人告诉了凡先生,其仕途比较发达,命里官运亨通,并且明年就能考取秀才,因此对了凡先生拥有做官的命却不读书求取功名感到很奇怪。了凡先生如实相告转述了母亲的意愿,同时他恭敬地向老人询问尊姓大名以及来自何处。老人告诉了凡先生,自己姓孔,云南人,已经得了邵雍皇极数术的真传,并且运数上正应该传授给了凡先生。

了凡先生此时年仅十五岁,却有此奇遇,并非完全是机缘巧合,他能对陌生的老者礼敬有加,是很重要的因由,说明他谦逊知礼、诚心待人,具有很好的根器和气禀。了凡先生听了孔姓老人的一番言语后,将他请至家中,向母亲做了禀报,母亲着他好好善待老人。其后,了凡先生试探了老人的术数,尽皆应验,分毫不差。于是便有了读书的念头。家中只有寡母,只好找到表兄沈称与之商量。表兄思考之后对他说:知道有位名叫郁海谷的私塾先生正在沈友夫家授课教学,可以送了凡先生去那里跟随寄读,也十分便利。于是,了凡先生便拜郁海谷先生为师开始读书。

\begin{yuanwen}
孔为余起数\footnote{占卜用语,通过象,即各种现实中已存在的事物的表征,根据既定的规则换算为数,搭配成卦,进而通过分析卦变的各种可能性,来推断出事物发展的未来走向。}:县考童生\footnote{由知县主持的考试,试期多在二月。要参加科举考试取得功名的士子,先向本县礼房报名应试,须填写姓名、籍贯、年岁、三代履历,并取得本县廪生保结。考五场,各场分别试八股文、试帖诗、经论、律赋等。一般来说,第一场录取后即有资格参加上一级的府试。县考,即县试。童生,明清时期,府、州、县学的应考者称为童生,或称儒童、文童。名称中虽则有“童”字,但是童生的年龄是大小不一的。只要未取得府、州、县学的生员资格,均称童生。},当十四名;府考\footnote{即府试。科举制度中由府一级进行的考试称府试。经县试录取的童生得以参加管辖该县的府(或直隶州、厅)的考试。试期多安排在四月,报名手续与县试略同。府试录取以后,即取得参加院试的资格。}七十一名,提学\footnote{学官名。宋崇宁二年(1103)在各路设提举学事司,管理所属州、县学校和教育行政,简称“提学”。每年视察各地学校,查考师生勤惰优劣。历代沿制。明正统元年(1436)改为提调学校司,设两京提学御史,由御史充任。各省设提督学道以按察使、副使或佥事充任,称提学道。从此提学成为专管地方教育文化的最高行政长官。}考第九名。

明年赴考,三处名数皆合。

复为卜终身休咎,言:某年考\footnote{又称“岁考”。明代提学官和清代学政,每年对所属府、州、县生员、廪生举行的考试。分别优劣,酌定赏罚。}第几名,某年当补廪\footnote{明清科举制度,生员经岁、科两试成绩优秀者,增生可依次升生,称为“补廪”。廪,lǐn 。},某年当贡\footnote{科举制度从府、州、县生员(秀才)中选拔入京师国子监读书的学子。生员(秀才)一般是隶属于本府、州、县学,除应乡试中举人为“正途”之外,其余未中式者而考选升入京师国子监读书以谋求出身的,称为贡生,意思是以人才贡献给朝廷以备选用。},贡后某年,当选四川一大尹,在任三年半,即宜告归。五十三岁八月十四日丑时\footnote{夜里一点至三点。},当终于正寝,惜无子。余备录而谨记之。
\end{yuanwen}

孔先生为我占了一卦,结果是:县考童生时,应当考中第十四名;府考时为第七十一名,提学主持的考试中为第九名。

第二年我去参加考试,三处考试的名次都完全相符。

孔先生又为我占卜一生的吉凶祸福,他说:某某年会考中第几名,某某年应当升为廪生,某某年可以选拔进京师国子监读书成为贡生,入贡后某年,应当被选为四川某方面的大官,在职三年半后,便应该告老辞官还乡。五十三岁那年的八月十四日的丑时,会寿终正寝,可惜最后没有儿子。我把这些都完整地记录下来并牢牢记住。

文中孔术士为了凡先生所做的预测,一般来说,应是运用出生时间的天干地支也就是俗称的生辰八字作为起数的依据。孔先生通过对了凡先生生辰八字所成卦象的分析,将其一生的命运,主要是仕途发展的前景揭示出来,栩栩如生、历历在目。相比邵雍祖师的神占来说,孔先生之能只算是雕虫小技,邵雍仅靠邻人五声叩门就推算出他是来借斧头的;但在了凡先生看来,孔术士的术数是很了不起的,故而他要“备录而谨记之”。对预测之术的虔信与他后来改变自己命运的决心形成了鲜明对比。

孔先生替了凡先生起数所推算的命运是:在第二年的县考中是第十四名,在其后的府考中是第七十一名,提学考则是第九名。了凡先生十六岁这年,果然考试考取了并且名次与孔先生推算的完全一致。在这一年内的三次考试中,名次和结果也都一如孔先生所说,可谓毫厘不差。这令了凡先生内心完全折服,于是请孔先生为他算定终生命运的吉凶祸福。孔先生告诉他,哪一年考试会考第几名,哪一年会补而成为廪生,作为秀才的一个级别,康生就可以领取国家发给的米粮了。哪一年可以当贡生,达到秀才的最高级别,获取入太学即国家大学读书的机会和资格。孔先生甚至告诉他,在他出贡后的某一年还会当上四川一个县的知县。任职三年半后就应告老还乡、辞官退隐。在五十三岁那年的八月十四日丑时离开人世,寿终正寝。还有一点就是命中无子。了凡先生将孔先生为他所推算的一生之流年休咎均备录在案给自己做一个参考。

\begin{yuanwen}
自此以后,凡遇考校,其名数先后,皆不出孔公所悬定\footnote{预定、算定的意思。}者。独算余食廪米\footnote{指官府按月发给在学生员的粮食。}九十一石五斗当出贡\footnote{科举考试中屡试不第的贡生,可按资历依次到京,由吏部选任杂职小官。某年轮着,称为“出贡”。};及食米七十(余/一)石,屠宗师即批准补贡,余窃疑之。
\end{yuanwen}

从此以后,凡是遇到考试,每次考试所得名次,都与孔先生所算的一样。只有一件不同:孔先生算我为廪生时领取官府九十一石五斗廪米时就可以按资历到京出贡选职了。但当我领取七十一石的时候,屠宗师便批准我可以补上贡生了,我暗地里也很怀疑。

儒家思想中有“知命”的思想,《论语》中孔子曾经受到当时一些隐者的讥讽,认为他“知其不可而为之”。这里,儒家思想是指每个人都有他应该做的,每个人所能够做的,就是一心一意尽力去做我们知道应该做的事,而不要计较成败,这就是儒家“知命”的思想。

知命就是承认世界本来客观存在的必然性,不为外在成败而萦怀牵累,做到这一点,也就能“永不言败”,也就是儒家所谓的君子。

了凡先生这里则是流于宿命的思想,因为从此以后,凡是遇到考试,他的结果名次都如孔先生所事先算定的一样。命运流转似乎已经毫无悬念。唯独在推论了凡先生做廪生领取到国家九十一石五斗粮食就能出贡一点上,第一次出现了差池。因为当了凡先生领到米粮七十一石时,就被当时掌管一省教育的姓屠的提学批准补贡,这一点似乎算得不太准确。

\begin{yuanwen}
后果为署印\footnote{text}杨公所驳,直至丁卯年\footnote{text},殷秋溟(míng)\footnote{text}宗师见余场中备卷,叹曰:“五策\footnote{text},即五篇奏议也,岂可使博洽淹贯之儒\footnote{text},老于窗下乎!”

遂依县申文准贡,连前食米计之,实九十一石五斗也。余因此益信进退有命,迟速有时,澹(dàn)然\footnote{text}无求矣。
\end{yuanwen}

①署印:代理官职。这里指代理提学之职的杨姓官员。
②丁卯年:即1567年。
③殷秋溟:指殷迈(1512-1581),字时训,号秋溟,直隶南京人。嘉靖二十年(1541)进士,授户部主事,历江西参政、南京太常寺卿。
④策:古代考试的一种文体。
⑤博洽淹贯:这里形容知识广博、深通广晓。治,指对理论了解得透彻。淹,指文义透彻。贯,指功夫一以贯之。
⑥澹然:清心寡欲的样子。
译文
后来这件事果然被代理提学的杨先生驳回,直到丁卯年,殷迈宗师看到我正式考试所作文字的备卷,赞叹道:“这五篇策论,就是给朝廷的五篇奏议啊,怎么能让渊博而明理的儒者老死于窗下呢!”于是便依从屠宗师的申文而批准出贡,连同前边领取的米一起计算,恰好九十一石五斗。我因此更加相信一进一退皆有天命,发迹的快慢也都各有因缘,所以也就无所欲求了
浅 释
正当了凡先生疑虑猜度之时,其结果是屠宗师批准补贡的文件被杨姓代理提学给驳回了,一直到丁卯年,了凡先生三十三岁上,此时主持教育的殷秋溟在闲暇时不经意地翻阅到了凡先生的旧卷文,不由得感慨系之:这五篇策论文章,写得如此之好简直就是五篇朝堂上的奏议!怎么能让如此见闻广博、学识丰富而又有德行的人老于窗下,当一辈子穷秀才呢?古代读书人最害怕的就是一辈子终老于窗下,唯希望“十年窗下无人问,一举成名天下知”,从此尽享人间富贵。读书的旨趣发生了质变,包括封建统治阶层也正是利用这一点来利诱古代读书人的,而这种以华堂美色、车马奴仆为诱饵,使得读书变得越发功利了,反成为读书人的思想禁锢,失去了其原初的意义。
殷秋溟再次为了凡先生申请补贡,获得批准,又一次应验了孔先生的预言,确实

\begin{yuanwen}

\end{yuanwen}\begin{yuanwen}

\end{yuanwen}\begin{yuanwen}

\end{yuanwen}\begin{yuanwen}

\end{yuanwen}\begin{yuanwen}

\end{yuanwen}\begin{yuanwen}

\end{yuanwen}\begin{yuanwen}

\end{yuanwen}\begin{yuanwen}

\end{yuanwen}\begin{yuanwen}

\end{yuanwen}\begin{yuanwen}

\end{yuanwen}\begin{yuanwen}

\end{yuanwen}\begin{yuanwen}

\end{yuanwen}\begin{yuanwen}

\end{yuanwen}\begin{yuanwen}

\end{yuanwen}\begin{yuanwen}

\end{yuanwen}\begin{yuanwen}

\end{yuanwen}\begin{yuanwen}

\end{yuanwen}\begin{yuanwen}

\end{yuanwen}\begin{yuanwen}

\end{yuanwen}\begin{yuanwen}

\end{yuanwen}\begin{yuanwen}

\end{yuanwen}\begin{yuanwen}

\end{yuanwen}\begin{yuanwen}

\end{yuanwen}\begin{yuanwen}

\end{yuanwen}\begin{yuanwen}

\end{yuanwen}\begin{yuanwen}

\end{yuanwen}\begin{yuanwen}

\end{yuanwen}\begin{yuanwen}

\end{yuanwen}\begin{yuanwen}

\end{yuanwen}\begin{yuanwen}

\end{yuanwen}\begin{yuanwen}

\end{yuanwen}\begin{yuanwen}

\end{yuanwen}\begin{yuanwen}

\end{yuanwen}\begin{yuanwen}

\end{yuanwen}\begin{yuanwen}

\end{yuanwen}\begin{yuanwen}

\end{yuanwen}\begin{yuanwen}

\end{yuanwen}\begin{yuanwen}

\end{yuanwen}\begin{yuanwen}

\end{yuanwen}\begin{yuanwen}

\end{yuanwen}\begin{yuanwen}

\end{yuanwen}\begin{yuanwen}

\end{yuanwen}\begin{yuanwen}

\end{yuanwen}\begin{yuanwen}

\end{yuanwen}\begin{yuanwen}

\end{yuanwen}\begin{yuanwen}

\end{yuanwen}\begin{yuanwen}

\end{yuanwen}\begin{yuanwen}

\end{yuanwen}\begin{yuanwen}

\end{yuanwen}\begin{yuanwen}

\end{yuanwen}\begin{yuanwen}

\end{yuanwen}\begin{yuanwen}

\end{yuanwen}\begin{yuanwen}

\end{yuanwen}\begin{yuanwen}

\end{yuanwen}\begin{yuanwen}

\end{yuanwen}\begin{yuanwen}

\end{yuanwen}\begin{yuanwen}

\end{yuanwen}\begin{yuanwen}

\end{yuanwen}\begin{yuanwen}

\end{yuanwen}\begin{yuanwen}

\end{yuanwen}\begin{yuanwen}

\end{yuanwen}\begin{yuanwen}

\end{yuanwen}\begin{yuanwen}

\end{yuanwen}\begin{yuanwen}

\end{yuanwen}\begin{yuanwen}

\end{yuanwen}\begin{yuanwen}

\end{yuanwen}\begin{yuanwen}

\end{yuanwen}\begin{yuanwen}

\end{yuanwen}\begin{yuanwen}

\end{yuanwen}\begin{yuanwen}

\end{yuanwen}\begin{yuanwen}

\end{yuanwen}\begin{yuanwen}

\end{yuanwen}\begin{yuanwen}

\end{yuanwen}\begin{yuanwen}

\end{yuanwen}\begin{yuanwen}

\end{yuanwen}\begin{yuanwen}

\end{yuanwen}\begin{yuanwen}

\end{yuanwen}\begin{yuanwen}

\end{yuanwen}\begin{yuanwen}

\end{yuanwen}





贡入燕都,留京一年,终日静坐,不阅文字。己巳归,游南雍,未入监,先访云谷会禅师于栖霞山中,对坐一室,凡三昼夜不瞑(míng)目。

云谷问曰:凡人所以不得作圣者,只为妄念相缠耳。汝坐三日,不见起一妄念,何也?

余曰:吾为孔先生算定,荣辱生死,皆有定数,即要妄想,亦无可妄想。

云谷笑曰:我待汝是豪杰,原来只是凡夫。

问其故?曰:人未能无心,终为阴阳所缚,安得无数?但惟凡人有数;极善之人,数固拘他不定;极恶之人,数亦拘他不定。汝二十年来,被他算定,不曾转动一毫,岂非是凡夫?

余问曰:然则数可逃乎?曰:命由我作,福自己求。诗书所称,的为明训。我教典中说:求富贵得富贵,求男女得男女,求长寿得长寿。夫妄语乃释迦大戒,诸佛菩萨,岂诳(kuáng)语欺人?

余进曰:孟子言:求则得之,是求在我者也。道德仁义,可以力求;功名富贵,如何求得?

云谷曰:孟子之言不错,汝自错解耳。汝不见六祖说:一切福田,不离方寸;从心而觅,感无不通。求在我,不独得道德仁义,亦得功名富贵;内外双得,是求有益于得也。

若不反躬内省,而徒向外驰求,则求之有道,而得之有命矣,内外双失,故无益。

因问:孔公算汝终身若何?余以实告。云谷曰:汝自揣(chuǎi)应得科第否?应生子否?

余追省良久,曰:不应也。科第中人,类有福相,余福薄,又不能积功累行,以基厚福;兼不耐烦剧,不能容人;时或以才智盖人,直心直行,轻言妄谈。凡此皆薄福之相也,岂宜科第哉。

地之秽者多生物,水之清者常无鱼;余好洁,宜无子者一;和气能育万物,余善怒,宜无子者二;爱为生生之本,忍为不育之根;余矜惜名节,常不能舍己救人,宜无子者三;多言耗气,宜无子者四;喜饮铄(shuò)精,宜无子者五;好彻夜长坐,而不知葆(bǎo)元毓( yù )神,宜无子者六。其余过恶尚多,不能悉数。

云谷曰:岂惟科第哉。世间享千金之产者,定是千金人物;享百金之产者,定是百金人物;应饿死者,定是饿死人物;天不过因材而笃(dǔ),几曾加纤毫意思。

即如生子,有百世之德者,定有百世子孙保之;有十世之德者,定有十世子孙保之;有三世二世之德者,定有三世二世子孙保之;其斩焉无后者,德至薄也。

汝今既知非。将向来不发科第,及不生子相,尽情改刷;务要积德,务要包荒,务要和爱,务要惜精神。从前种种,譬如昨日死;从后种种,譬如今日生;此义理再生之身也。

夫血(xuè)肉之身,尚然有数;义理之身,岂不能格天。太甲曰:天作孽,犹可违;自作孽,不可活。诗云:永言配命,自求多福。孔先生算汝不登科第,不生子者,此天作之孽,犹可得而违;汝今扩充德性,力行善事,多积阴德,此自己所作之福也,安得而不受享乎?

易为君子谋,趋吉避凶;若言天命有常,吉何可趋,凶何可避?开章第一义,便说:积善之家,必有余庆。汝信得及否?

余信其言,拜而受教。因将往日之罪,佛前尽情发露,为疏一通,先求登科;誓行善事三千条,以报天地祖宗之德。

云谷出功过格示余,令所行之事,逐日登记;善则记数,恶则退除,且教持准提咒,以期必验。

语余曰:符箓(lù)家有云:不会书符,被鬼神笑;此有秘传,只是不动念也。执笔书符,先把万缘放下,一尘不起。从此念头不动处,下一点,谓之混沌开基。由此而一笔挥成,更无思虑,此符便灵。凡祈天立命,都要从无思无虑处感格。

孟子论立命之学,而曰:夭寿不贰。夫夭与寿,至贰者也。当其不动念时,孰为夭,孰为寿?细分之,丰歉不贰,然后可立贫富之命;穷通不贰,然后可立贵贱之命;夭寿不贰,然后可立生死之命。人生世间,惟死生为重,曰夭寿,则一切顺逆皆该之矣。

至修身以俟(sì)之,乃积德祈天之事。曰修,则身有过恶,皆当治而去之;曰俟,则一毫觊觎(jì yú),一毫将迎,皆当斩绝之矣。到此地位,直造先天之境,即此便是实学。

汝未能无心,但能持准提咒,无记无数,不令间断,持得纯熟,于持中不持,于不持中持。到得念头不动,则灵验矣。

余初号学海,是日改号了凡;盖悟立命之说,而不欲落凡夫窠臼(kē jiù)也。从此而后,终日兢(jīng)兢,便觉与前不同。前日只是悠悠放任,到此自有战兢惕(tì)厉景象,在暗室屋漏中,常恐得罪天地鬼神;遇人憎我毁我,自能恬然容受。

到明年礼部考科举,孔先生算该第三,忽考第一;其言不验,而秋闱(wéi)中式矣。

然行义未纯,检身多误;或见善而行之不勇,或救人而心常自疑;或身勉为善,而口有过言;或醒时操持,而醉后放逸;以过折功,日常虚度。自己巳岁发愿,直至己卯岁,历十余年,而三千善行始完。

时方从李渐庵入关,未及回向。庚辰南还。始请性空、慧空诸上人,就东塔禅堂回向。遂起求子愿,亦许行三千善事。辛巳、生男天启。

余行一事,随以笔记;汝母不能书,每行一事,辄(zhé)用鹅毛管,印一朱圈于历日之上。或施食贫人,或买放生命,一日有多至十余圈者。至癸未八月,三千之数已满。复请性空辈,就家庭回向。九月十三日,复起求中进士愿,许行善事一万条,丙戌登第,授宝坻知县。

余置空格一册,名曰治心编。晨起坐堂,家人携付门役,置案上,所行善恶,纤悉必记。夜则设桌于庭,效赵阅道焚香告帝。

汝母见所行不多,辄颦(pín)蹙(cù)曰:我前在家,相助为善,故三千之数得完;今许一万,衙中无事可行,何时得圆满乎?

夜间偶梦见一神人,余言善事难完之故。神曰:只减粮一节,万行俱完矣。盖宝坻之田,每亩二分三厘七毫。余为区处,减至一分四厘六毫,委有此事,心颇惊疑。适幻余禅师自五台来,余以梦告之,且问此事宜信否?

师曰:善心真切,即一行可当万善,况合县减粮、万民受福乎?吾即捐俸银,请其就五台山斋僧一万而回向之。

孔公算予五十三岁有厄,余未尝祈寿,是岁竟无恙,今六十九矣。书曰:天难谌(chén),命靡(mí)常。又云:惟命不于常,皆非诳语。吾于是而知,凡称祸福自己求之者,乃圣贤之言。若谓祸福惟天所命,则世俗之论矣。

汝之命,未知若何?即命当荣显,常作落寞想;即时当顺利,当作拂逆想;即眼前足食,常作贫窭(jù)想;即人相爱敬,常作恐惧想;即家世望重,常作卑下想;即学问颇优,常作浅陋想。

远思扬祖宗之德,近思盖父母之愆(qiān);上思报国之恩,下思造家之福;外思济人之急,内思闲己之邪。

务要日日知非,日日改过;一日不知非,即一日安于自是;一日无过可改,即一日无步可进;天下聪明俊秀不少,所以德不加修、业不加广者,只为因循二字,耽阁一生。

云谷禅师所授立命之说,乃至精至邃(suì)、至真至正之理,其熟玩而勉行之,毋自旷也。


\chapter{改过之法}

春秋诸大夫,见人言动,亿而谈其祸福,靡(mí)不验者,左国诸记可观也。

大都吉凶之兆,萌乎心而动乎四体,其过于厚者常获福,过于薄者常近祸;俗眼多翳(yì),谓有未定而不可测者。

至诚合天,福之将至,观其善而必先知之矣。祸之将至,观其不善而必先知之矣。今欲获福而远祸,未论行善,先须改过。

但改过者,第一、要发耻心。思古之圣贤,与我同为丈夫,彼何以百世可师?我何以一身瓦裂?耽(dān)染尘情,私行不义,谓人不知,傲然无愧,将日沦于禽兽而不自知矣;世之可羞可耻者,莫大乎此。孟子曰:耻之于人大矣。以其得之则圣贤,失之则禽兽耳。此改过之要机也。

第二、要发畏心。天地在上,鬼神难欺,吾虽过在隐微,而天地鬼神,实鉴临之。重则降之百殃,轻则损其现福;吾何可以不惧?

不惟此也。闲居之地,指视昭然;吾虽掩之甚密,文之甚巧,而肺肝早露,终难自欺;被人觑(qù)破,不值一文矣,乌得不懔(lǐn)懔?

不惟是也。一息尚存,弥天之恶,犹可悔改;古人有一生作恶,临死悔悟,发一善念,遂得善终者。谓一念猛厉,足以涤百年之恶也。譬如千年幽谷,一灯才照,则千年之暗俱除;故过不论久近,惟以改为贵。

但尘世无常,肉身易殒(yǔn),一息不属,欲改无由矣。明则千百年担负恶名,虽孝子慈孙,不能洗涤;幽则千百劫沉沦狱报,虽圣贤佛菩萨,不能援引。乌得不畏?

第三、须发勇心,人不改过,多是因循退缩;吾须奋然振作,不用迟疑,不烦等待。小者如芒刺在肉,速与抉剔(tī);大者如毒蛇啮(niè)指,速与斩除,无丝毫凝滞,此风雷之所以为益也。

具是三心,则有过斯改,如春冰遇日,何患不消乎?然人之过,有从事上改者,有从理上改者,有从心上改者;工夫不同,效验亦异。

如前日杀生,今戒不杀;前日怒詈(lì),今戒不怒;此就其事而改之者也。强制于外,其难百倍,且病根终在,东灭西生,非究竟廓然之道也。

善改过者,未禁其事,先明其理;如过在杀生,即思曰:上帝好生,物皆恋命,杀彼养己,岂能自安?且彼之杀也,既受屠割,复入鼎镬( huò),种种痛苦,彻入骨髓;己之养也,珍膏罗列,食过即空,疏食菜羹(gēng),尽可充腹,何必戕(qiāng)彼之生,损己之福哉?

又思血气之属,皆含灵知,既有灵知,皆我一体;纵不能躬(gōng)修至德,使之尊我亲我,岂可日戕物命,使之仇我憾我于无穷也?一思及此,将有对食伤心,不能下咽者矣。

如前日好怒,必思曰:人有不及,情所宜矜(jīn);悖(bèi)理相干,于我何与?本无可怒者。

又思天下无自是之豪杰,亦无尤人之学问,行有不得,皆己之德未修,感未至也。吾悉以自反,则谤毁之来,皆磨炼玉成之地;我将欢然受赐,何怒之有?

又闻谤而不怒,虽谗(chán)焰薰天,如举火焚空,终将自息;闻谤而怒,虽巧心力辩,如春蚕作茧,自取缠绵;怒不惟无益,且有害也。其余种种过恶,皆当据理思之。此理既明,过将自止。

何谓从心而改?过有千端,惟心所造;吾心不动,过安从生?学者于好色、好名、好货、好怒、种种诸过,不必逐类寻求;但当一心为善,正念现前,邪念自然污染不上。如太阳当空,魍魉(wǎng  liǎng)潜消,此精一之真传也。过由心造,亦由心改,如斩毒树,直断其根,奚必枝枝而伐,叶叶而摘哉?

大抵最上者治心,当下清净;才动即觉,觉之即无。苟未能然,须明理以遣之;又未能然,须随事以禁之;以上事而兼行下功,未为失策。执下而昧上,则拙矣。

顾发愿改过,明须良朋提醒,幽须鬼神证明;一心忏悔,昼夜不懈,经一七、二七,以至一月、二月、三月,必有效验。

或觉心神恬(tián)旷;或觉智慧顿开;或处冗沓(rǒng  tà)而触念皆通;或遇怨仇而回嗔(chēn)作喜;或梦吐黑物;或梦往圣先贤,提携接引;或梦飞步太虚;或梦幢幡(chuáng  fān)宝盖,种种胜事,皆过消罪灭之象也。然不得执此自高,画而不进。

昔蘧(qú)伯玉当二十岁时,己觉前日之非而尽改之矣。至二十一岁,乃知前之所改,未尽也;及二十二岁,回视二十一岁,犹在梦中,岁复一岁,递递改之,行年五十,而犹知四十九年之非,古人改过之学如此。

吾辈身为凡流,过恶猬集;而回思往事,常若不见其有过者,心粗而眼翳(yì)也。

然人之过恶深重者,亦有效验:或心神昏塞,转头即忘;或无事而常烦恼;或见君子而赧(nǎn)然消沮;或闻正论而不乐;或施惠而人反怨;或夜梦颠倒,甚则妄言失志;皆作孽之相也,苟一类此,即须奋发,舍旧图新,幸勿自误。

图片

\chapter{积善之方}

易曰:积善之家,必有余庆。昔颜氏将以女妻叔梁纥(hé),而历叙其祖宗积德之长,逆知其子孙必有兴者。孔子称舜之大孝,曰:宗庙飨(xiǎng)之,子孙保之,皆至论也。试以往事征之。

杨少师荣、建宁人。世以济渡为生,久雨溪涨,横流冲毁民居,溺死者顺流而下,他舟皆捞取货物,独少师曾祖及祖,惟救人,而货物一无所取,乡人嗤(chī)其愚。逮(dài)少师父生,家渐裕,有神人化为道者,语之曰:汝祖父有阴功,子孙当贵显,宜葬某地。遂依其所指而窆( biǎn)之,即今白兔坟也。后生少师,弱冠登第,位至三公,加曾祖、祖、父,如其官。子孙贵盛,至今尚多贤者。

鄞(yín)人杨自惩,初为县吏,存心仁厚,守法公平。时县宰严肃,偶挞(tà)一囚,血流满前,而怒犹未息,杨跪而宽解之。宰曰:怎奈此人越法悖理,不由人不怒。自惩叩首曰:上失其道,民散久矣,如得其情,哀矜(jīn)勿喜;喜且不可,而况怒乎?宰为之霁(jì)颜。

家甚贫,馈遗(kuì  wèi)一无所取,遇囚人乏粮,常多方以济之。一日,有新囚数人待哺,家又缺米;给囚则家人无食;自顾则囚人堪悯(mǐn);与其妇商之。妇曰:囚从何来?曰:自杭而来。沿路忍饥,菜色可掬(jū)。因撤己之米,煮粥以食囚。后生二子,长曰守陈,次曰守址,为南北吏部侍郎;长孙为刑部侍郎;次孙为四川廉宪,又俱为名臣;今楚亭、德政,亦其裔也。

昔正统间,邓茂七倡乱于福建,士民从贼者甚众;朝廷起鄞县张都宪楷南征,以计擒贼,后委布政司谢都事,搜杀东路贼党;谢求贼中党附册籍,凡不附贼者,密授以白布小旗,约兵至日,插旗门首,戒军兵无妄杀,全活万人;后谢之子迁,中状元,为宰辅;孙丕(pī),复中探花。

莆田林氏,先世有老母好善,常作粉团施人,求取即与之,无倦色;一仙化为道人,每旦索食六七团。母日日与之,终三年如一日,乃知其诚也。因谓之曰:吾食汝三年粉团,何以报汝?府后有一地,葬之,子孙官爵,有一升麻子之数。其子依所点葬之,初世即有九人登第,累代簪缨(zān  yīng)甚盛,福建有无林不开榜之谣。

冯琢(zhuó)庵太史之父,为邑庠(yì  xiáng)生。隆冬早起赴学,路遇一人,倒卧雪中,扪(mén)之,半僵矣。遂解己绵裘衣(yì)之,且扶归救苏。梦神告之曰:汝救人一命,出至诚心,吾遣韩琦为汝子。及生琢庵。遂名琦。

台州应尚书,壮年习业于山中。夜鬼啸集,往往惊人,公不惧也;一夕闻鬼云:某妇以夫久客不归,翁姑逼其嫁人。明夜当缢(yì)死于此,吾得代矣。公潜卖田,得银四两。即伪作其夫之书,寄银还家;其父母见书,以手迹不类,疑之。既而曰:书可假,银不可假;想儿无恙(yàng)。妇遂不嫁。其子后归,夫妇相保如初。

公又闻鬼语曰:我当得代,奈此秀才坏吾事。旁一鬼曰:尔何不祸之?曰:上帝以此人心好,命作阴德尚书矣,吾何得而祸之?应公因此益自努励,善日加修,德日加厚;遇岁饥,辄(zhé)捐谷以赈之;遇亲戚有急,辄委曲维持;遇有横逆,辄反躬自责,怡然顺受;子孙登科第者,今累累也。

常熟徐凤竹栻(shì),其父素富,偶遇年荒,先捐租以为同邑之倡,又分谷以赈贫乏,夜闻鬼唱于门曰:千不诓(kuāng),万不诓;徐家秀才,做到了举人郎。相续而呼,连夜不断。是岁,凤竹果举于乡,其父因而益积德,孳(zī)孳不怠,修桥修路,斋僧接众,凡有利益,无不尽心。后又闻鬼唱于门曰:千不诓,万不诓;徐家举人,直做到都堂。凤竹官终两浙巡抚。

嘉兴屠康僖(xī)公,初为刑部主事,宿狱中,细询诸囚情状,得无辜者若干人,公不自以为功,密疏其事,以白堂官。后朝审,堂官摘其语,以讯诸囚,无不服者,释冤抑十余人。一时辇(niǎn)下咸颂尚书之明。公复禀曰:辇毂(gǔ)之下,尚多冤民,四海之广,兆民之众,岂无枉者?宜五年差一减刑官,核实而平反之。尚书为奏,允其议。时公亦差减刑之列,梦一神告之曰:汝命无子,今减刑之议,深合天心,上帝赐汝三子,皆衣紫腰金。是夕夫人有娠(shēn),后生应埙(xūn)、应坤、应堎,皆显官。

嘉兴包凭,字信之,其父为池阳太守,生七子,凭最少,赘(zhuì)平湖袁氏,与吾父往来甚厚,博学高才,累举不第,留心二氏之学。一日东游泖(mǎo)湖,偶至一村寺中,见观音像,淋漓露立,即解橐(tuó)中得十金,授主僧,令修屋宇,僧告以功大银少,不能竣(jùn)事;复取松布四疋(pǐ),检箧(qiè)中衣七件与之,内纻褶(zhù  zhě),系新置,其仆请已之。凭曰:但得圣像无恙,吾虽裸裎何伤?僧垂泪曰:舍银及衣布,犹非难事。只此一点心,如何易得。后功完,拉老父同游,宿寺中。公梦伽蓝来谢曰:汝子当享世禄矣。后子汴,孙柽(chēng)芳,皆登第,作显官。

嘉善支立之父,为刑房吏,有囚无辜陷重辟,意哀之,欲求其生。囚语其妻曰:支公嘉意,愧无以报,明日延之下乡,汝以身事之,彼或肯用意,则我可生也。其妻泣而听命。及至,妻自出劝酒,具告以夫意。支不听,卒为尽力平反之。囚出狱,夫妻登门叩谢曰:公如此厚德,晚世所稀,今无子,吾有弱女,送为箕(jī)帚妾,此则礼之可通者。支为备礼而纳之,生立,弱冠中魁,官至翰林孔目,立生高,高生禄,皆贡为学博。禄生大纶(lún),登第。

凡此十条,所行不同,同归于善而已。若复精而言之,则善有真、有假;有端、有曲;有阴、有阳;有是、有非;有偏、有正;有半、有满;有大、有小;有难、有易;皆当深辨。为善而不穷理,则自谓行持,岂知造孽,枉费苦心,无益也。

何谓真假?昔有儒生数辈,谒(yè)中峰和尚,问曰:佛氏论善恶报应,如影随形。今某人善,而子孙不兴;某人恶,而家门隆盛;佛说无稽(jī)矣。中峰云:凡情未涤,正眼未开,认善为恶,指恶为善,往往有之。不憾己之是非颠倒,而反怨天之报应有差乎?众曰:善恶何致相反?中峰令试言其状。一人谓詈(lì)人殴(ōu)人是恶;敬人礼人是善。中峰云:未必然也。一人谓贪财妄取是恶,廉洁有守是善。中峰云:未必然也。众人历言其状,中峰皆谓不然。

因请问。中峰告之曰:有益于人,是善;有益于己,是恶。有益于人,则殴人,詈人皆善也;有益于己,则敬人、礼人皆恶也。是故人之行善,利人者公,公则为真;利己者私,私则为假。又根心者真,袭迹者假;又无为而为者真,有为而为者假;皆当自考。

何谓端曲?今人见谨愿之士,类称为善而取之;圣人则宁取狂狷(juàn)。至于谨愿之士,虽一乡皆好,而必以为德之贼;是世人之善恶,分明与圣人相反。推此一端,种种取舍,无有不谬;天地鬼神之福善祸淫,皆与圣人同是非,而不与世俗同取舍。凡欲积善,决不可徇(xùn)耳目,惟从心源隐微处,默默洗涤,纯是济世之心,则为端;苟有一毫媚世之心,即为曲;纯是爱人之心,则为端;有一毫愤世之心,即为曲;纯是敬人之心,则为端;有一毫玩世之心,即为曲;皆当细辨。

何谓阴阳?凡为善而人知之,则为阳善;为善而人不知,则为阴德。阴德,天报之;阳善,享世名。名,亦福也。名者,造物所忌;世之享盛名而实不副者,多有奇祸;人之无过咎而横被恶名者,子孙往往骤发,阴阳之际微矣哉。

何谓是非?鲁国之法,鲁人有赎人臣妾于诸侯,皆受金于府,子贡赎人而不受金。孔子闻而恶(wù)之,曰:赐失之矣。夫圣人举事,可以移风易俗,而教道可施于百姓,非独适己之行也。今鲁国富者寡而贫者众,受金则为不廉,何以相赎乎?自今以后,不复赎人于诸侯矣。

子路拯(zhěng)人于溺,其人谢之以牛,子路受之。孔子喜曰:自今鲁国多拯人于溺矣。自俗眼观之,子贡不受金为优,子路之受牛为劣;孔子则取由而黜(chù)赐焉。乃知人之为善,不论现行而论流弊;不论一时而论久远;不论一身而论天下。现行虽善,而其流足以害人;则似善而实非也;现行虽不善,而其流足以济人,则非善而实是也;然此就一节论之耳。他如非义之义,非礼之礼,非信之信,非慈之慈,皆当抉择。

何谓偏正?昔吕文懿(yì)公,初辞相位,归故里,海内仰之,如泰山北斗。有一乡人,醉而詈(lì)之,吕公不动,谓其仆曰:醉者勿与较也。闭门谢之。逾年,其人犯死刑入狱。吕公始悔之曰:使当时稍与计较,送公家责治,可以小惩而大戒;吾当时只欲存心于厚,不谓养成其恶,以至于此。此以善心而行恶事者也。

又有以恶心而行善事者。如某家大富,值岁荒,穷民白昼抢粟于市;告之县,县不理,穷民愈肆,遂私执而困辱之,众始定;不然,几乱矣。故善者为正,恶者为偏,人皆知之;其以善心而行恶事者,正中偏也;以恶心而行善事者,偏中正也;不可不知也。

何谓半满?易曰:善不积,不足以成名,恶不积,不足以灭身。书曰:商罪贯盈,如贮(zhù)物于器。勤而积之,则满;懈(xiè)而不积,则不满。此一说也。

昔有某氏女入寺,欲施而无财,止有钱二文,捐而与之,主席者亲为忏悔;及后入宫富贵,携数千金入寺舍之,主僧惟令其徒回向而己。因问曰:吾前施钱二文,师亲为忏悔,今施数千金,而师不回向,何也?曰:前者物虽薄,而施心甚真,非老僧亲忏,不足报德;今物虽厚,而施心不若前日之切,令人代忏足矣。此千金为半,而二文为满也。钟离授丹于吕祖,点铁为金,可以济世。吕问曰:终变否?曰:五百年后,当复本质。吕曰:如此则害五百年后人矣,吾不愿为也。曰:修仙要积三千功行,汝此一言,三千功行已满矣。此又一说也。

又为善而心不著善,则随所成就,皆得圆满。心著于善,虽终身勤励,止于半善而已。譬如以财济人,内不见己,外不见人,中不见所施之物,是谓三轮体空,是谓一心清净,则斗粟可以种无涯之福,一文可以消千劫之罪,倘此心未忘,虽黄金万镒(yì),福不满也。此又一说也。

何谓大小?昔卫仲达为馆职,被摄至冥司,主者命吏呈善恶二录,比至,则恶录盈庭,其善录一轴,仅如箸(zhù)而已。索秤称之,则盈庭者反轻,而如箸者反重。仲达曰:某年未四十,安得过恶如是多乎?曰:一念不正即是,不待犯也。因问轴中所书何事?曰:朝廷尝兴大工,修三山石桥,君上疏谏(jiàn)之,此疏稿也。仲达曰:某虽言,朝廷不从,于事无补,而能有如是之力。曰:朝廷虽不从,君之一念,已在万民;向使听从,善力更大矣。故志在天下国家,则善虽少而大;苟在一身,虽多亦小。

何谓难易?先儒谓克己须从难克处克将去。夫子论为仁,亦曰先难。必如江西舒翁,舍二年仅得之束修,代偿官银,而全人夫妇;与邯郸张翁,舍十年所积之钱,代完赎银,而活人妻子,皆所谓难舍处能舍也。如镇江靳(jìn)翁,虽年老无子,不忍以幼女为妾,而还之邻,此难忍处能忍也;故天降之福亦厚。凡有财有势者,其立德皆易,易而不为,是为自暴。贫贱作福皆难,难而能为,斯可贵耳。

随缘济众,其类至繁,约言其纲,大约有十:第一、与人为善;第二、爱敬存心;第三、成人之美;第四、劝人为善;第五、救人危急;第六、兴建大利;第七、舍财作福;第八、护持正法;第九、敬重尊长;第十、爱惜物命。

何谓与人为善?昔舜在雷泽,见渔者皆取深潭厚泽,而老弱则渔于急流浅滩之中,恻(cè)然哀之,往而渔焉;见争者皆匿其过而不谈,见有让者,则揄(yú)扬而取法之。期年,皆以深潭厚泽相让矣。夫以舜之明哲,岂不能出一言教众人哉?乃不以言教而以身转之,此良工苦心也。

吾辈处末世,勿以己之长而盖人;勿以己之善而形人;勿以己之多能而困人。收敛才智,若无若虚;见人过失,且涵容而掩覆之。一则令其可改,一则令其有所顾忌而不敢纵,见人有微长可取,小善可录,翻然舍己而从之;且为艳称而广述之。凡日用间,发一言,行一事,全不为自己起念,全是为物立则;此大人天下为公之度也。

何谓爱敬存心?君子与小人,就形迹观,常易相混,惟一点存心处,则善恶悬绝,判然如黑白之相反。故曰:君子所以异于人者,以其存心也。君子所存之心,只是爱人敬人之心。盖人有亲疏贵贱,有智愚贤不肖;万品不齐,皆吾同胞,皆吾一体,孰非当敬爱者?爱敬众人,即是爱敬圣贤;能通众人之志,即是通圣贤之志。何者?圣贤之志,本欲斯世斯人,各得其所。吾合爱合敬,而安一世之人,即是为圣贤而安之也。

何谓成人之美?玉之在石,抵掷(dǐ zhì)则瓦砾(lì),追琢(zhuó)则圭璋(guī zhāng);故凡见人行一善事,或其人志可取而资可进,皆须诱掖(yè)而成就之。或为之奖借,或为之维持;或为白其诬而分其谤;务使之成立而后已。

大抵人各恶(wù)其非类,乡人之善者少,不善者多。善人在俗,亦难自立。且豪杰铮(zhēng)铮,不甚修形迹,多易指摘;故善事常易败,而善人常得谤;惟仁人长者,匡直而辅翼之,其功德最宏。

何谓劝人为善?生为人类,孰无良心?世路役役,最易没溺。凡与人相处,当方便提撕,开其迷惑。譬犹长夜大梦,而令之一觉;譬犹久陷烦恼,而拔之清凉,为惠最溥(pǔ)。韩愈云:一时劝人以口,百世劝人以书。较之与人为善,虽有形迹,然对证发药,时有奇效,不可废也;失言失人,当反吾智。

何谓救人危急?患难颠沛,人所时有。偶一遇之,当如痌瘝(tōng  guān)之在身,速为解救。或以一言伸其屈抑;或以多方济其颠连。崔子曰:惠不在大,赴人之急可也。盖仁人之言哉。

何谓兴建大利?小而一乡之内,大而一邑(yì)之中,凡有利益,最宜兴建;或开渠导水,或筑堤防患;或修桥梁,以便行旅;或施茶饭,以济饥渴;随缘劝导,协力兴修,勿避嫌疑,勿辞劳怨。

何谓舍财作福?释门万行,以布施为先。所谓布施者,只是舍之一字耳。达者内舍六根,外舍六尘,一切所有,无不舍者。苟非能然,先从财上布施。世人以衣食为命,故财为最重。吾从而舍之,内以破吾之悭(qiān),外以济人之急;始而勉强,终则泰然,最可以荡涤(dí)私情,祛(qū)除执吝(lìn)。

何谓护持正法?法者、万世生灵之眼目也。不有正法,何以参赞天地?何以裁成万物?何以脱尘离缚?何以经世出世?故凡见圣贤庙貌,经书典籍,皆当敬重而修饬(chì)之。至于举扬正法,上报佛恩,尤当勉励。

何谓敬重尊长?家之父兄,国之君长,与凡年高、德高、位高、识高者,皆当加意奉事。在家而奉侍父母,使深爱婉容,柔声下气,习以成性,便是和气格天之本。出而事君,行一事,毋谓君不知而自恣(zì)也。刑一人,毋谓君不知而作威也。事君如天,古人格论,此等处最关阴德。试看忠孝之家,子孙未有不绵远而昌盛者,切须慎之。

何谓爱惜物命?凡人之所以为人者,惟此恻隐之心而已;求仁者求此,积德者积此。周礼、孟春之月,牺牲毋用牝(pìn)。孟子谓君子远庖(páo)厨,所以全吾恻隐之心也。故前辈有四不食之戒,谓闻杀不食、见杀不食、自养者不食、专为我杀者不食。学者未能断肉,且当从此戒之。

渐渐增进,慈心愈长。不特杀生当戒,蠢动含灵,皆为物命。求丝煮茧,锄地杀虫,念衣食之由来,皆杀彼以自活。故暴殄(tiǎn)之孽(niè),当于杀生等。至于手所误伤,足所误践者,不知其几,皆当委曲防之。古诗云:爱鼠常留饭,怜蛾不点灯。何其仁也?

善行无穷,不能殚述;由此十事而推广之,则万德可备矣。

\chapter{谦德之效}

易曰:天道亏盈而益谦;地道变盈而流谦;鬼神害盈而福谦;人道恶盈而好谦。是故谦之一卦,六爻皆吉。书曰:满招损,谦受益。予屡同诸公应试,每见寒士将达,必有一段谦光可掬。

辛未计偕(xié),我嘉善同袍凡十人,惟丁敬宇宾,年最少,极其谦虚。予告费锦坡曰:此兄今年必第。费曰:何以见之?予曰:惟谦受福。兄看十人中,有恂(xún)恂款款,不敢先人,如敬宇者乎?有恭敬顺承,小心谦畏,如敬宇者乎?有受侮不答,闻谤不辩,如敬宇者乎?人能如此,即天地鬼神,犹将佑之,岂有不发者?及开榜,丁果中式。

丁丑在京,与冯开之同处,见其虚己敛容,大变其幼年之习。李霁岩直谅益友,时面攻其非,但见其平怀顺受,未尝有一言相报。予告之曰:福有福始,祸有祸先,此心果谦,天必相之,兄今年决第矣。已而果然。

赵裕峰、光远,山东冠县人,童年举于乡,久不第。其父为嘉善三尹,随之任。慕钱明吾,而执文见之,明吾,悉抹其文,赵不惟不怒,且心服而速改焉。明年,遂登第。

壬辰岁,予入觐(jìn),晤(wù)夏建所,见其人气虚意下,谦光逼人,归而告友人曰:凡天将发斯人也,未发其福,先发其慧;此慧一发,则浮者自实,肆者自敛;建所温良若此,天启之矣。及开榜,果中式。

江阴张畏岩,积学工文,有声艺林。甲午,南京乡试,寓一寺中,揭晓无名,大骂试官,以为眯目。时有一道者,在傍微笑,张遽(jù)移怒道者。道者曰:相公文必不佳。张益怒曰:汝不见我文,乌知不佳?道者曰:闻作文,贵心气和平,今听公骂詈,不平甚矣,文安得工?张不觉屈服,因就而请教焉。

道者曰:中全要命;命不该中,文虽工,无益也。须自己做个转变。张曰:既是命,如何转变。道者曰:造命者天,立命者我;力行善事,广积阴德,何福不可求哉?张曰:我贫士,何能为?道者曰:善事阴功,皆由心造,常存此心,功德无量。且如谦虚一节,并不费钱,你如何不自反而骂试官乎?

张由此折节自持,善日加修,德日加厚。丁酉,梦至一高房,得试录一册,中多缺行。问旁人,曰:此今科试录。问:何多缺名?曰:科第阴间三年一考较,须积德无咎者,方有名。如前所缺,皆系旧该中式,因新有薄行而去之者也。后指一行云:汝三年来,持身颇慎,或当补此,幸自爱。是科果中一百五名。

由此观之,举头三尺,决有神明;趋吉避凶,断然由我。须使我存心制行,毫不得罪于天地鬼神,而虚心屈己,使天地鬼神,时时怜我,方有受福之基。彼气盈者,必非远器,纵发亦无受用。稍有识见之士,必不忍自狭其量,而自拒其福也。况谦则受教有地,而取善无穷,尤修业者所必不可少者也。

古语云:有志于功名者,必得功名;有志于富贵者,必得富贵。人之有志,如树之有根,立定此志,须念念谦虚,尘尘方便,自然感动天地,而造福由我。今之求登科第者,初未尝有真志,不过一时意兴耳;兴到则求,兴阑则止。孟子曰:王之好乐甚,齐其庶几乎?予于科名亦然。



第一篇 立命之学

我童年时父亲就去世了,母亲要我放弃学业,改行学医术,她说行医可以谋生,也可以治病救人。并且精通一门技艺而后扬名天下,正是你父亲的愿望。

后来我在慈云寺碰到一位修髯伟貌、飘飘若仙的老人家,就恭敬地向他行礼。老人家对我说:“你是官场中人,明年就可考中秀才,你为什么不读书呢?”我就把母亲叫我放弃读书而去学习医术的缘故告诉他,又请问老人家的姓名籍贯等事。老人家说:“我姓孔,是云南人,学得宋朝邵康节先生《皇极经世》的真传。因为缘分,我应该把《皇极经世》传给你。”我就带老人家到家里。母亲要我好好侍侯他。

孔先生推算的事情很详尽,连细节问题也应验无误。因此我相信孔先生的推算,动了读书的念头,就和表哥沈称商量。表哥说:“我的好朋友郁海谷先生在沈友夫家里开馆授徒,送你过去寄读很方便。”于是我便拜郁海谷先生为师。

孔先生推算我县考时考第十四名,府考时考第七十一名,提学考时考第九名。第二年,我参加考试,三次考试所得名次和孔先生推算的完全相符。

孔先生又替我推算终生的命数:某年考取第几名,某年补廪生,某年做贡生,又在贡后某年,被选为四川某县县令,做县令三年半后,最好辞官回乡,五十三岁那年八月十四日丑时,就寿终正寝,可惜命中没有儿子。这些话我都作了记录,并牢记在心。

从此以后,凡是参加考试,所考名次先后,都不出孔先生所算。唯独算我做廪生领到九十一石五斗米时才能出贡这件事起了波折,当我领到七十一石米时,学台屠宗师就批准我补贡,我当时就怀疑孔先生的推算不灵了。

谁知后来学台代理杨宗师驳回屠宗师的批准,不准我补贡生。直到丁卯年,殷秋溟宗师查看考场的“备选试卷”时,看到我的试卷,动情地说:“这份卷子所做的五篇策论,竟有奏议的水准。怎么可以让这么有学问的读书人埋没到老呢?”于是他就吩咐县官写申请公文,准我补了贡生。我这段时间领的廪米,加上以前那部分,刚好是九十一石五斗。

经过这番波折,我更加相信命数的存在,即使屠宗师提前让我补贡,因为不符合命数,必然要被杨公驳回,当符合命数要求时,自然有殷宗师帮我补贡。所以,我把一切事情都看淡了,变得无所追求。

补贡以后,我到北京国子监读书一年。这一年里,我整天都在静坐,也不看书写字。到了己巳年,回南京国子监,还没报到时,我先到栖霞山拜见云谷禅师。我同禅师在禅房里对坐,三天三夜没有闭过眼睛。

云谷禅师问我:“普通人不能成为伟人,只是因为妄念一个接着一个,束缚了自己的行动。而你静坐三天,我没见过你起一个妄念,这是什么缘故呢?”

我说:“我的命运早被孔先生算定了,一生中生死荣辱都有定数,没有办法改变。想也是白想,真要想也不知该想些什么。”

云谷禅师笑着说:“我还以为你是个豪杰,谁知也不过是个凡夫俗子。”

我问他为什么这么说。云谷禅师说:“普通人未能进入‘无心’境界,一定要被命数控制,怎能说没有命数呢?但是只有普通人,才会被命数所束缚,大善人或者大恶人,命数是控制不住他的。二十年来你被孔先生算死,不能改变一分一毫。你不是凡夫,是什么?”

我问云谷禅师:“人真能摆脱命运的束缚吗?”

禅师说:“命由我造,福自己求。诗书中说的,的确是至理名言。佛家经典也说,诚心追求富贵可得到富贵,诚心追求儿女可得到儿女,诚心追求长寿可得到长寿。说谎是佛家大戒,佛菩萨可能说假话吗?”

我还是不明白,又问:“孟子曾说,‘求则得之,是求在我者也’,我内心的道德仁义可以通过自己的努力修来,功名富贵等身外之物,怎么求得到呢?”

云谷禅师说:“孟子的话没错,是你理解错了。你看六祖慧能大师也说,‘一切福田,不离方寸,从心而觅,感无不通’,在自己的心田里求,既可求得内心的道德仁义,又能求得身外的功名富贵,就是‘内外双得’,就说明求有助于得,求则得之。反之,若不立足自心福田,经常检讨反省自己,而是盲目向外攀缘,追求名利福寿,那么只能听天由命了,就是‘内外双失(失就是失控,自己作不了主的意思)’,正说明求无益于得,自己根本无法摆脱命运的束缚。”

云谷禅师又问:“孔先生算你终身的命运如何?”我将孔先生的推算和盘托出。云谷禅师说:“你自己仔细想想,你应该考得功名吗?应该有儿子吗?”

我想了很久才说:“我不应该考得功名,也不应该有儿子。因为有功名的人都有福相,而我相薄福浅。第一,没有行善积德,以增长福报;第二,性格急躁,怕麻烦,气量小,不能包容别人;第三,恃才傲物,锋芒毕露,经常用自己的才华去贬低别人,说话直来直去,经常信口开河,滔滔不绝。这些都是福薄的表现,怎么能考得功名呢?

“我也不应该有儿子,第一,地之秽者多生物,水之清者常无鱼,而我有洁癖的毛病;第二,和气能育万物,我却脾气暴躁,经常生气发怒;第三,爱为生生之本,忍(残忍)为不育之根,而我却执著自己的名节,不能舍己为人,助人为乐;第四,我说话总是口若悬河滔滔不绝,容易耗费元气;第五,我嗜酒,经常喝得酩酊大醉,容易消散精神;第六,我喜欢整夜长坐,不睡觉,不晓得保护元气养育精神。其它还有许多过失,说也说不完。”

云谷禅师说:“岂只是功名,你得不到的事情还多着呢!世上拥有千金家产的人,一定有享用千金的福报,拥有百金家产的人,一定有享用百金的福报;应该饿死的,一定是要受饿死报应的人。上天不过根据他本来的材质进行雕琢加工,什么时候强加过丝毫意思?就像生儿子,譬如一个人积了一百代的功德,就一定有一百代的子孙来保住他的福;积了十代的功德,就一定有十代的子孙来保住他的福;积了三代或者两代的功德,就一定有三代或者两代的子孙来保住他的福;至于绝后,那是因为他的功德极薄。

“你既然知道自己的问题,就应该把你得不到功名,以及不生儿子的种种不良行为习惯,彻底改过来。一定要行善积德,一定要大度包容,一定要和气待人,一定要爱惜精神。从前有害身体健康,折损福报的行为习惯都要彻底改掉,从今天开始要脱胎换骨,做一个明白事理,懂得用正确方式追求幸福的人。我们这个血肉之躯,尚且还有命运定数,通达命运的人,怎么不能感动上天,改变自己的命运呢?

“《书经·太甲篇》上说,‘天作孽,犹可违,自作孽,不可活’(意思是:天降给你的灾祸,或者可以避开;而自己造的孽,一定要自己受报应),《诗经》也说,‘永言配命,自求多福’(意思是:人要随时检讨自己的行为是否符合天道的要求,幸福的生活要自己去创造)。孔先生算你不得功名,命中无子,这就是上天降给你的灾祸,还可以改变!你只要研究宇宙人生的真相,提到自己的道德水平,多做善事多积阴德,加大自己的福报,哪有享受不到自己福报的道理?

“《易经》帮助正人君子趋吉避凶,如果命运真的不能改变,那么吉祥怎么追求,凶险如何避免?《易经》开头就将核心思想揭示出来,‘积善之家,必有余庆’(意思是:经常行善的家庭,不仅自家有福报,还能福荫子孙),你相信这个道理吗?”

我相信云谷禅师的话,拜谢他的教诲,同时把以前所做的错事所造的罪业,到佛前检讨忏悔,还写了一篇疏文,祈求能考得功名,并发誓要做三千件善事,来报答天地祖先的恩德。

云谷禅师听了我的誓言,就拿出功过格给我,要我将所做的善事恶事,逐日逐条记录在功过格上,做了善事可加分,做了恶事要扣分。云谷禅师还教我念准提咒,确保我能达成所愿。

云谷禅师又说:“孟子讲立命的道理时说,‘夭寿不贰’(意思是短命和长寿没有分别)。短命和长寿怎能说没有分别呢?要晓得婴儿还在娘胎里时,谁晓得这孩子是短命还是长寿?孩子出生时,立即带上前生前世的业因果报,就有了短命和长寿的分别。要改变自己的命运,就要利用因果报应的规律,通过破迷断恶、行善积德来积累福报,将短命变成长寿。这个道理推而广之,就是丰收跟歉收、穷困跟通达、短命跟长寿本来相同,只因业缘不同而有差别,通过断恶修善就可改变或贫穷或富裕、或尊贵或卑贱,或生存或死亡的命运。人生在世,最重要的事情就是生死问题,明白了如何面对生死,也就能从容应对人生中的一切顺境逆境。

“孟子说,‘修身以俟之’(意思是:自己只管修心养性、行善积德,机缘成熟,上天一定会安排福报的降临),说的是行善积德,向天祈祷的事。说到修字,就是自己身上所有的过失罪恶,都要彻底改掉。讲到俟(等待)字,就是连一丝一毫的非份之想、攀缘念头(如走后门走捷径等),都要彻底根除。能够这样,就可算是走上通往‘无心’,即将成圣成贤的康庄大道了。这才是人世间最实在最受用的学问。

“你虽然还达不到‘无心’的水平,但若能念准提咒,一遍一遍念下去,不要记数,不要间断,念到极熟的时候,自然就会口里在念,心里不觉得在念,在不念的时候,心里不觉仍在念,到了念头不起时,就可暂时得到‘无心’的体验,这时你就会有许多意想不到的收获。”

我原来的名号叫做“学海”,今天改为“了凡”。因为我超越了世俗的观念,明白了立命的道理,不愿再做普通人了。从此以后,我就整天小心谨慎,感觉和以前大不相同。以前整天糊里糊涂,无所事事。现在变得谨小慎微,战战兢兢,如履薄冰,如临深渊。在暗处无人监管时,我也常恐怕得罪天地鬼神;受到别人嗔恨诽谤,我也能安然接受,而不斤斤计较,争论是非曲直。

第二年,我参加礼部科考,孔先生算定我应该考第三名,结果却是第一名,孔先生的话开始不灵了。秋天我又参加乡试,竟然中了举人,这是我命里本来没有的。我改造命运的计划初见成效,就更加相信云谷禅师的话,深信命运是可以改造的!

我虽然改了不少毛病,却很不彻底,还有很多不足之处。例如做好事的行动不够坚决,帮助别人时自己还会犹豫,做了好事喜欢四处张扬,清醒时还能严格要求自己,酒醉后又放肆胡来。虽然做了善事,又在不断犯错,将功抵过,就这样虚度光阴,从己巳年发愿,直到己卯年,十年时间才把三千件善事做完。

当时,我正和李渐庵先生,从关外回来,没来得及把三千件善事回向。到了庚辰年,我从北京回到南方,才请性空、慧空两位大和尚,借东塔禅堂完成回向的心愿。这时,我又起了求生儿子的心愿,就许下再做三千件善事的大愿。到了辛巳年,生了你,取名“天启”。

我每做一件善事,都用笔记下来。你母亲不识字,她每做一件善事,都用鹅毛管,印一个红圈在日历上,或是送食物给穷人,或是买活物放生,都要印圈。有时她一天可印十几个红圈呢!到癸未年八月,三千件善事才做完。我又请性空和尚等,在家里做回向。到那年九月十三,我又起求中进士的愿,并许下做一万件善事的大愿。到了丙戌年,居然中了进士,吏部就让我补了宝坻县县令的缺。

我做县令时,准备了一本有空格的小册子,我称之为《治心篇》,就是要治理内心的邪思歪念的意思。每天早晨坐堂审案时,我叫家里人拿《治心篇》给差役,将我每天所做的善事恶事,不管大小都详细记录在《治心篇》上。到晚上,我就在家里庭院中摆下桌子,效仿宋朝的铁面御史赵阅道,焚香祷告天帝。

你母亲见我所做的善事不多,常常皱著眉头说:“我从前在家帮你做善事,所以三千件善事能够做完。现在你许了做一万件善事的心愿,在衙门里又没什么善事可做,要等到什么时候才能做完呢?”

我也犯了愁,有次晚上睡觉我做梦,梦中遇到一位天神,我就将一万件善事不易做完的事告诉天神,天神说:“只算你当县令减轻田赋这件事,就可抵一万件善事。”宝坻县的田赋,本来每亩要收银两分三厘七毫,我觉得百姓负担太重,就把全县的田地重新清理测算,并将每亩应缴的田赋减少到一分四厘六毫。虽然确有其事,但我心里还是有疑惑。

恰好幻余禅师从五台山来到宝坻,我就把梦境告诉了禅师,并问禅师这件事的可信度。幻余禅师说:“做善事若能存心真切,不图回报,那么一件善事跟一万件善事没有差别。况且你减轻全县的田赋,全县得到恩惠的何止万人?我听了禅师的话,就捐出我的俸银薪水,请禅师在五台山替我斋僧一万人,并将功德回向。

孔先生算我只能活到五十三岁,我没有向天祈寿,五十三岁那年竟然平安无事,现在已经六十九岁了。《书经》上说:“天难谌,命靡常”(意思是天道难测,命运无常)。又说:“惟命不于常”(意思是人的命运并非一成不变)。这些话一点都不假,因此我相信,命运由自己决定是圣人的观点,命运由天注定是世俗人的观点。

你的命运不知究竟怎样。就算命中应该兴旺发达,也要以不得意相看待;就算行运顺风顺水,也要以不称心相看待;就算眼前衣食无忧,也要以贫困相看待;就算旁人敬重你,也要诚惶诚恐、小心谨慎;就算你家世显赫,也要以出身低微相看待;就算你学识渊博,也要以肤浅相看待。

长远看,要努力发扬祖先的德气;近处看,要尽力弥补父母的过失;向上,要热爱祖国,时刻准备为国家作贡献;对下,要尊老爱幼,为全家人造福;对外,要乐于助人;对内,要检讨反省自己。

你每天都要发现自己的缺点并立即改正,若有一天没有发现缺点,或者没有改正错误,那这一天算是白过了。天下许多有才气的人,到最后学问却不见得有多高深,事业也没多大成就,就是因为他们因循守旧,得过且过,以至虚度年华,耽搁一生。

云谷禅师所说的话,包含着精妙深邃,至真至正的道理,你要深入研究,努力践行,千万不要浪费光阴!


第二篇 改过之法

春秋时期,各诸侯国的高级官吏,能从一个人的言谈举止、气质神态,判断出他的命运祸福,而且判断没有不灵验的。《左传》和《国语》等史书有许多这方面的记载。通常情况下吉祥和凶险的预兆,都从心里发出而体现在人的言谈举止、气质神态上。譬如某人厚道稳重,就说明他的福报已近,某人刻薄*,就说明他的灾祸不远。普通人不会看气色面相,就说祸福无常,命运无法预测。人心至诚无妄,就能与天道互相感应,这样的人就能通过他人的言谈举止来预测他的祸福情况。如一个人福报即将来临时,他的行为多半是善的,灾祸就要降临时,他的行为多半是不善的。所以,人要想生活幸福,远离灾难,就要多做善事。但是在做善事之前,一定要先要把自己的过失改掉。

改过的方法,第一要发“羞耻心”。想想古时圣贤,和我一样,都是男子汉大丈夫,为什么他们可以流芳百世、为人师表,而我却碌碌无为、贱如破瓦?就是因为我贪图享乐,偷偷做了种种不应该做的事,还以为旁人不知道,毫无惭愧之心,就这样渐渐沉沦下去,变成禽兽自己还不知道。世界上没有比这件事更令人羞耻的了。孟子说:“耻之于人大矣”(意思是羞耻心对人最重要)。人懂得羞耻,就可以成为圣贤;若不晓得羞耻,就跟禽兽无异。培养羞耻心,就是改过的关键。

改过的第二个方法,是要发“敬畏心”。举头三尺有神明。在大家看不到的地方干了坏事,天地鬼神看得清清楚楚。过失重的,种种灾祸立即降临;过失轻的,也要折损现在的福报,这怎么能不怕呢?

不仅如此,就是在自己家里也离不开神明的监察,我虽然把过失遮盖得十分严密,掩饰得十分巧妙,但实际上,我的言谈举止、气质神态早已表现出来,出卖了我。若是被人看破,人格更是变得一文不值,这又怎能不畏惧呢?

再说,人还活着时,就算犯下滔天罪过,还是可以忏悔改过的。古时候就有人,作了一辈子坏事,临终前悔悟,发一个善念而得善终。这就是说,人转一个真切勇猛的善念,便可以把百年所积的罪恶洗净。譬如千年黑暗的山谷,只要一盏灯就可以把千年的黑暗除尽。所以过失不论长久新近,能改就行。但是,绝对不可以认为平时犯过无所谓,临终前改了就没事。因为世事无常,谁知自己的死期?那天一口气缓不过来,怎么去改过?明的报应,在阳间你要承担千百年的恶名,即使有孝子贤孙也不能替你洗清恶名;暗的报应,在阴间还要千百劫沉沦在地狱里受无量无边的苦难,即使碰到圣贤、佛菩萨也无从救拔接引。这又怎能不怕呢?

第三,要发“勇猛心”。一个人之所以有过不改,多是因为得过且过,不肯振作奋发而自甘堕落的缘故。改过一定要下定决心、振奋精神、勇往直前。小的过失,像尖刺戳在肉里一样要赶紧挑掉;大的过失,像手指头被见血封喉的毒蛇咬到一样要立即切掉,不能有丝毫犹疑,如迅雷不及掩耳要越快越好。

如果具备“羞耻心”、“敬畏心”、“勇猛心”这三种心,那么就能有过即改了,就像春天里的薄冰碰到太阳光一样,能不立即融化吗?改过有从事上着力,有从理上着力,也有从心上着力的,三种方法不同,所得到的效验也不一样。

第一,从事上着力。譬如前天杀生,今天起不再杀生了。前天发火骂了人,今天起不再发火骂人了。这些就是在事上着力来改过。这是从外部施加压力,难度比较大,而且病根没有根除,很容易旧病复发,不是彻底改过的方法。

第二,从理上着力。善于改过的人,未采取行动时,先搞清道理。譬如一个人犯了杀生的罪过,就要这样想:上天有好生之德,凡是动物都爱惜自己的生命。杀它的生命来养我的身体,自问心能安吗?而且动物在入口前还要受屠宰之痛,烧煮煎熬之苦,各种痛苦一直透进骨髓。实际上,人活着,各种美味佳肴吃了之后,便成渣滓,什么都没有了;而蔬菜素食素汤等,也吃得饱,为什么一定要伤害生命,造杀生的罪孽,减自己的福报呢?又想,凡是有血气的东西,都有灵性知觉,既然都有灵性知觉,那么和我本来一体,就算自己不能将道德修到极高的境界,使它们都来尊重我、亲近我,也不能天天伤害它们,使它们与我结仇,以致冤冤相报无尽期呀!能想到这些,就会吃不下桌上那些本来有血肉,有生命的菜肴。

又譬如喜欢发怒,应该想到:如果是对方不懂事,我应该哀怜他才对;如果是对方不讲道理,那错在他,与我有什么关系呢?本来就不值得发怒生气呀!又想到:天下没有自以为是的英雄豪杰,也没有专门用来怨天尤人的学问。因此一个人做事不称心,都是因为自己的道德没修好,功德没修满,无法感化别人,所以要时刻反省检讨自己。别人诽谤我,反而变成磨炼我、成就我。我应该愉快地接受别人对我的批评,哪里还有怨恨呢?如果被诽谤而不生气,那么不管诽谤有多厉害,烧上天了也不怕,最后一定要熄灭;如果因诽谤而生气,那么就是百般辩解也无济于事,就如作茧自缚,越解释越混乱,越描越黑。可见,生气不但无益,而且有害。

其它各种过错,也都要像上面讲的那样,先从道理上搞清楚,道理明白了,过错自然就改过来了。

第三,从心上着力。人各种各样的过失,都是由心造出来的,我的心要是不动,那么过失从何而生?凡是读书人,或是喜欢美色,或是喜欢名声,或是喜欢财物,或是喜欢发火,各种各样的过失,不必一一列出。只要心善,至诚无妄,那么各种邪念妄想,自然就污染不了了,譬如丽日当空,所有妖魔鬼怪都不敢出现,这就是最精妙绝伦的修心补过诀窍。过失由心而造,也要由心而改,正像斩除毒树一样,只要直接砍断树根就行了,又何必一枝一枝地裁剪,一叶一叶地摘掉呢?

改过最高明的方法是修心。如果心地清净,邪念起时自己已经觉察到,只要消掉邪念就不会犯错了;做不到心清净,就要在理上下功夫,搞清道理,也不会轻易犯错;理上功夫做不到,只好在事上加把劲,有错就改,又错又改,尽量弥补过失。这三种方法也可同时使用,犯了错就改,同时搞清道理,还要历事练心,提高修心的功夫。若是只知从事上改过,而不在理上问个所以然,更不修心;或者虽然明白了事理,却不修心,都不能从根本上改正过失,更不能通达生命,了脱生死,成就伟业。

发愿改过,最好能得到多方帮助。日常生活中,要有良师益友在身边时时提醒;心灵深处,要树立坚定的信仰,要对着自己的信仰虔诚忏悔,从早到晚,绝不放松,经过一个七天,两个七天,直到一个月,两个月,三个月……这样坚持下去,一定会有意想不到的效验!

改过忏悔后,会有各种体验。例如你会觉得精神舒畅,心旷神怡;或觉得突然智慧大开,灵感涌动;或是虽然处在复杂沉闷的环境里,心中静如止水,又无所不通;或碰到冤家仇人时,能将怨恨转为欢喜;或是梦见自己吐出黑物;或是梦见受到古时圣贤的提携牵引;或是梦见自己在太空中漫步;或是梦见各种彩旗华盖。出现诸如此类各种各样的好征兆,都是说明你已经消除掉好多罪过业障了。但是也不能因此而骄傲自满、固步自封,还要百尺竿头、更进一步。

春秋时代卫国的贤大夫蘧伯玉在二十岁时,已经能反醒过去的过失,作深刻检讨并彻底改过了。到了二十一岁,又觉得以前所改的过失并不彻底;到了二十二岁,再回忆二十一岁时,还像在梦中一般,就这样一年一年地过去,一年一年地改过;直到五十岁时,还觉得过去的四十九年,仍然有不足。古人改过就是这样孜孜不倦,永不停息。

我们都是平凡人,过失罪恶,就像刺猬身上的刺一样,满身都是。回想过去却经常看不到自己的过失,简直就是心粗眼患病。

如果人业障深重、罪孽缠身,也会有各种不好的体验。例如:整天精神恍惚,萎靡不振,魂不守舍;或者是健忘,遇事转头就忘;或者是莫名其妙自生烦恼;或者是见到品德高尚的人就觉得无地自容,自惭形秽,垂头丧气;或者是听到光明正大的道理,却不开心;或者是有恩惠给别人,对方却不领情反生怨恨;或者是夜里噩梦不断,甚至语无伦次,好像中邪一样。如此种种都是不正常的现象,都是干了坏事的表现。假使你有上面所说的某种情形,更应该即刻振作精神,奋发向上,努力改过,重新开始一条人生的康庄大道,希望你不要耽误自己!

第三篇 积善之方

《易经》上说:“积善之家,必有余庆”(意思是:积善的家庭,一定会有很多喜庆的事,还能福荫子孙)。从前姓颜的人家,要把女儿许配给孔子的父亲叔梁纥,就将孔家先祖所积之德一件一件罗列出来,由此推知,孔家子孙一定有大有作为的人,后来果然出了孔子。还有,孔子称赞舜的孝行时说:“他会进宗庙供子孙祭祀,且世代不断,子孙还能长久得到他的福荫。”这些都是确实无误的,我们还能在历史记录中找到很多相同的例子。

当朝少师(太子的老师)杨荣,是福建省建宁人,他家祖上世代以摆渡为生。有一次,连日下雨以至山洪暴发,洪水冲毁了上游不少民房,被淹的人顺流而下。别的船都去捞取水上漂来的财物,只有少师的曾祖父和祖父,专门去救灾民,而财物一件都不捞。乡人都笑他们是傻瓜。等到少师的父亲出生时,家里开始渐渐宽裕起来。有一位神仙化做道士模样,跟少师的父亲说,你的祖父和父亲,积了许多阴功,子孙应该发达做大官。你可以将你的先祖葬在某处。少师的父亲就把祖父和父亲移葬在道士所说的地方。这座坟就是现在大家都知道的白兔坟。后来少师出生了,二十岁就中进士,不断升迁,直到位列三公。皇帝还追封他的曾祖父、祖父、父亲。而且少师的子孙,都非常兴旺,直到现在还有许多贤能之士。

浙江宁波人杨自惩,起初在县衙做书办,他心地厚道,遵纪守法,做事公正。当时的县官,为人严厉方正,有一次发怒把一个囚犯打得血流满地,还不解恨。杨自惩就跪下替囚犯求情,请县官饶过囚犯。县官说:“这个囚犯不守法律,违背常理,实在是太可恶了,由不得人不生气啊!”杨自惩一边叩头一边说:“现在朝廷失却正理,政治黑暗*,人心散失已经很久了。审出案情时应该替他们伤心,可怜他们不明事理,误蹈法网,而不能雀跃欢喜。欢喜尚且不可,又怎么可以发火呢?”县官听了杨自惩的话,非常感动,态度和缓下来,气也消了!

杨自惩家其实很穷困,别人送他东西,他却一概不受。碰到囚犯缺粮时,他就想方设法弄些米来救济他们。有一次来了几个新囚犯,没有东西吃,非常饥饿。他自己家里刚好米也不多,若是拿来给囚犯吃,自家人就要挨饿;要是自己吃,囚犯又饿得很可怜。实在没办法,就跟妻子商量。妻子问他:“犯人从什么地方来的?”“从杭州来的,他们一路挨饿,现在饿得脸上一点血色都没有。”两夫妇就从自家锅里撤出一些米,用来煮稀饭给新来的囚犯吃。后来他们生了两个儿子,大的叫守陈,小的叫守址,分别当了南北吏部侍郎,大孙子当刑部侍郎,小孙子当四川按察使。两个儿子,两个孙子都是名臣,当朝两个名人楚亭和德政,也是杨自惩的后代。

明朝英宗正统年间,邓茂七在福建一带造反。福建有很多读书人和老百姓跟随他造反。皇帝派出曾经担任都御使的鄞县人张楷,去搜捕围剿造反者。张楷用计捉住了邓茂七,又委派福建布政司谢都事去屠杀造反余党。谢都事不肯乱杀人,就向各处收集造反者名册,凡不在名册上的人就暗中给一面小白旗,要他们在官兵搜查贼党时把小白旗插在家门口,并且禁止官兵乱杀好人。就这样保存了一万多人的性命。后来谢都事的儿子谢迁,就中了状元,官至宰相,孙子谢丕,也中了探花。

福建省浦田县的林家,上辈中有一位老太太喜欢做善事,经常用米粉做粉团给穷人吃。只要有人向她要,她就立刻给,而且毫不厌烦。有一位仙人,变作道士,每天早晨向她讨六七个粉团。老太太每次都给他,一连三年,天天如此。仙人晓得她作善事的诚心,就跟她说:“我吃了你三年的粉团,要怎样报答你呢?这样吧,你家后面有一块地,若是你死后葬在这块地上,将来子孙有官爵的,就会像一升麻子那样多。”后来老太太去世了,她的儿子依照仙人的指示,把老太太安葬下去。林家子孙第一代发科甲的,就有九人。后来世世代代,做大官的人非常多。因此福建省竟有“无林不开榜”(意思是林家参加考试的人很多,发榜时榜上也不会没有姓林的人)的民谣。

冯琢庵太史的父亲还在县学里做秀才时,一个寒冬早上他去县学,路上碰到一个人倒在雪地里,就快冻死了。冯老先生马上脱下皮袄替他穿上,并扶他到家里取暖施救。冯老先生救人后,有一晚上梦见一位天神告诉他:“你救人一命完全出自至诚之心,所以我要派韩琦投生到你家,做你的儿子。”后来老先生生了儿子琢庵,就起名为“冯琦”(韩琦是宋朝一位文武双全的贤能宰相,现在天神安排他来投胎转世,当冯老先生的儿子)。

浙江台州有一个应大猷尚书,壮年时在山中读书,晚上,鬼常聚在一起吼叫吓人,只有应公不怕。有一夜,应公听到一个鬼说:“有一个妇人,丈夫出远门好久没回来,公婆认定儿子已经死了,逼妇人改嫁,但是妇人要守节不肯改嫁,准备明晚在这里上吊。真开心,我终于可以找到替身了。”应公听到这些话,就暗中卖田,并以妇人丈夫的名义写信回家,随信附上卖田所得的四两银子。那人的父母看信以后,因为笔迹不像怀疑信是假的,后来看到银子就说:“信可以是假的,但是银子假不了!儿子一定还活着,平安无事。”所以他们就不再逼媳妇改嫁了。后来他们的儿子真的回来了,夫妇得以保全,像新婚一样甜甜蜜蜜过日子。

又有一次,应公听到那个鬼说:“我本来可以找到替身的,可惜这个秀才坏了我的好事。”

旁边一个鬼说:“你为什么不去害死他?”

那个鬼说:“天帝因为这个人心好有阴德,已经派他去做阴德尚书了,我怎么害得了他?”

应公听了两个鬼所说的话,就更加努力,善事一天一天去做,功德也一天一天在增加。碰到荒年时,就捐米谷救人;碰到亲戚有急难,也想尽办法助其渡过难关;碰到不如意的事,就反省检讨自己,心平气和地接受事实。因为应公这样为人处世,所以他的子孙得到功名官位的,到现在已经连成串了!

江苏省常熟县有一位徐凤竹先生,他的父亲很富有,也很有善心,碰到荒年,就先把自家应收的田租完全免去,做全县有田人的榜样,而后还捐出自家稻谷救济穷人。有一天夜里,他听到一群鬼在门口唱:“千不诓,万不诓,徐家秀才,做到了举人郎。”那些鬼连续不断呼叫,夜夜不停。这一年,徐凤竹去参加乡试,果然考中了举人。他的父亲因此更加努力不倦地做善事积功德。例如修桥铺路,施斋饭供养出家人。凡是对别人有好处的事情,无不尽心尽力去做。后来他又听到鬼在门口唱:“千不诓,万不诓,徐家举人,直做到都堂。”后来徐凤竹果然当了两浙巡抚。

浙江省嘉兴县有一位叫屠康僖的人,起初在刑部里做主事,夜里就住在监狱里,经常仔细盘问囚犯,结果发现被冤枉的有不少人。但是屠公并不觉得自己有功劳,只是秘密地把这件事上报刑部堂官。后来秋审提堂时,刑部堂官根据屠公所提供的材料,纠正了很多冤假错案,释放了十几个被冤枉的人,囚犯们没有不心服口服的。因此京城里的百姓都称赞刑部尚书明察秋毫。

后来屠公又上公文给堂官说:“在天子脚下,尚且有这么多被冤枉的人,全国这样大的地方,哪会没有被冤枉的人?所以应该定期派出减刑官,到各省去核实罪案,纠正冤假错案。”尚书代为上奏皇帝,皇帝批准了他的建议,派出减刑官到各省去查察,刚巧屠公也在委派之列。

有一天晚上屠公梦见天神告诉他说:“你命里本来没有儿子,但是因为你提出减刑的建议,正与天心相合,所以天帝赐给你三个儿子,将来都可以穿紫袍、束金带、做大官。”这天晚上,屠公的夫人就有了身孕,后来生下了应埙、应坤、应竣三个儿子,果然都作了高官。

有一位嘉兴人,叫包凭,字信之。他的父亲是安徽池州府太守,有七个儿子,包凭最小。包凭被平湖县姓袁的人家招赘为女婿,和我父亲常有来往,交情很深。他学识渊博,才华横溢,对佛道之学也很有研究,但是每次考试都考不中。

有一天,他向东去卯湖游玩,偶然到乡村一个破落的佛寺,看见观世音菩萨的圣像露天而立,被雨淋湿了。当时就掏出十两银子给寺里的住持和尚,叫他修理寺院房屋。和尚说:“修寺的工程大,银子太少,不够用,没法完工。”他又拿出四匹松江出产的布料,再从竹箱里捡出七件衣服给和尚。这七件衣服里,有一件是用麻料做的新夹衣,佣人要留下来,但是包凭说:“只要观世音菩萨的圣像能够安好,不被雨淋,我就是赤身露体又有什么关系呢?”和尚听后流着眼泪说:“施送银两和衣服布匹,还不算难事,只是这一片诚心,岂是谁都有的?”

后来房屋修好了,包凭就拉着父亲同游这座佛寺,并且住在寺中。那天晚上,包凭梦见寺里的护法神跟他说:“你做了这些功德,你的子孙可以世世代代享受官禄了。”后来他的儿子包汴,孙子包柽芳,都中了进士,做了高官。

浙江省嘉善县有一个叫做支立的人,他的父亲在县衙的刑房当书办。有一个囚犯被人陷害,判了死罪。支书办可怜他,有意帮他伸冤。那囚犯晓得支书办的好意之后,跟妻子说:“支公的好意,我们无法报答,明天请他到乡下,你就嫁给他,他或许会在念这份情上鼎力相助,那么我就有活命的机会了。”他的妻子哭着答应了。第二天,支书办到乡下,囚犯妻子出来劝支书办喝酒,并且把丈夫的意思告诉他。支书办不答应,但最后还是尽力相助,把案子*了。囚犯出狱后,夫妻到支书办家里叩头拜谢说:“您这样厚德的人实在少有。现在您没有儿子,我有一个女儿,愿意许给您做扫地的小妾。”这在情理上是说得通的,支书办就预备了礼物,把囚犯的女儿迎娶回家,后来生了儿子支立,才二十岁就中了举人,为官至翰林院的书记,后来支立的儿子支高,支高的儿子支禄,都被保荐做州县学里的教官,而支禄的儿子支大纶,也考中进士。

以上十个故事,每人所做的各不相同,但都是在行善积德,福荫子孙。若是进一步说明,那么做善事有真的假的,有直的曲的,有阴的阳的,有是的不是的,有偏的正的,有半满的圆满的,有大的小的,有难的易的。这些都要仔细辨别。若是做了善事,却没有深入考究,只知盲目苦干,要真的把善事办成恶事,那就是白费苦心,得不到一点益处啊!

第一,真假。元朝时有几个读书人,去拜见天目山高僧中峰和尚,问道:“佛家讲善恶因果报应,如影随形,为什么现在某人不停地在行善,他的子孙反而不兴旺,而某人作恶多端,家里反而发达得很呢?佛说的因果报应有凭据吗?”

中峰和尚回答说:“平常人被世俗的见解所蒙蔽,看不到事实真相,经常把真的善行反认为是恶的,把真的恶行反认为是善的。他们不反思自己颠倒是非,却去怀疑因果报应规律。”

大家又说:“善就是善,恶就是恶,善恶怎么会反过来呢?”

中峰和尚听了之后,便叫他们把自认为是善的、恶的事情各说几件出来。有一个人说:“骂人、打人是恶,恭敬人、礼貌待人是善。”

中峰和尚回答说:“你说的不一定对喔!”

另外一个人说:“贪财、乱要钱财是恶,不贪财、清清白白守正道是善。”

中峰和尚说:“你说的也不一定对喔!”

那些读书人,讲了好多善恶的行为,但是中峰和尚都说不一定。他们就问和尚其中的道理。中峰和尚说:“做有益于人的事情是善,只为自己就是恶。若是做的事情可以使别人受益,那怕是骂人打人也都是善;而有益于自己,那怕是恭敬人礼貌待人,也都是恶。所以一个人做善事,使旁人得到利益的就是公,公就是真;只考虑自己的利益,就是私,私就是假。还有,从心上发出来的善行是真,只不过做个样子的是假。还有,为善不求报答、不露痕迹的是真,为了某一目的才去做善事的是假。像这些都要仔细地考察。”

第二,端曲。现在人们称谨小慎微的人是善人,也很器重这类人。而古时圣贤却更欣赏积极向上、独立特行的人,他们评价谨小慎微的人,即使全乡人都喜欢他,也不过是伤害道德的贼。这样看来,世俗人的善恶观念,分明跟圣人相反。

依此类推,世俗人的种种善恶取舍,都跟圣人的观点相反。然而天地鬼神庇佑善人惩罚恶人,却和圣人的看法一致,而跟世俗人的观点相反。所以凡是要积功德,绝对不可以被世俗的观点所影响,一定要在起心动念的隐微之处,默默洗净自己的心,千万不可让污浊的环境污染了自己的真心。

纯是救济世人的心,是直;如果存有一丝讨好世俗的念头,就是曲。纯是爱人的心,是直;如果有一丝一毫对世人怨恨不平的念头,就是曲。纯是恭敬别人的心,就是直;如果有一丝玩弄世人的念头,就是曲。这些都应该仔细分辨清楚。

第三,阴阳。凡是一个人做善事被人知道,就叫阳善;做善事而别人不知道,就叫阴德。有阴德的人,由上天给他降福;有阳善的人,由世间给他美名。享受好名声,虽然也是福,但是名声这个东西,为天地所忌讳。如果一个人在世上享受极大的名声,实际上却名不副实,那么他很有可能要遭遇横祸;如果一个人并没有过失差错而被冤枉,或者无缘无故被人栽上恶名,那么他的子孙常常会忽然间发达起来。阴德和阳善的分别很微妙!

第四,是非。春秋时鲁国有一法律,凡出资赎回被别国抓去做奴隶的鲁国人的,可以到官府领取赏金。但是孔子的学生子贡,虽然也赎了人回来,却不肯接受赏金。孔子听到之后,很不高兴地说:“这件事子贡做得不对,凡是圣贤无论做什么事情,都要考虑能否移风易俗,以教化引导百姓做好人,而不能只考虑自己的名声。现在鲁国富人少,穷人多,若是受了赏金就算是贪财,那么谁还愿意去赎人?恐怕从此以后,鲁国再没有人向其他国家赎人了。”

有一次子路救了一个溺水的人,那个人就送一头牛来答谢子路,子路欣然接受。孔子知道了,很欣慰地说:“从今以后,鲁国就会有很多人,愿意到深水大河中救人了。”

用世俗的眼光看,子贡不接受赏金是好的,子路接受牛是贪婪。孔子反而称赞子路而责备子贡。由此看来,判断一件事的善恶,不能只看当前的行为,而要看对以后的影响;不能只论一时的是非,而是要看长远的影响;不能只论个人的得失,而是看它对天下大众的影响。

现在的行为,虽然是善的,却会危害后人,那就是表面是善而实际上不是善;现在的行为,虽然不是善,但是流传下去却能够利益后人,那就是表面不善而实际上是善!这不过是举一个例子而已,其它种种,如“非义之义,非礼之礼,非信之信,非慈之慈”都要仔细分辨清楚。

第五,偏正。明朝宰相吕文懿刚辞官回乡时,所有人都很敬佩他,唯独一个乡下人,喝醉酒后大骂吕公。吕公并没有生气,他对佣人说:“这个人喝醉酒了,不要和他计较。”吕公就关门不理睬他。过了一年,这个人犯了死罪入狱,吕公方才懊悔地讲:“若是当时稍微惩治一下,把他送到官府治罪,藉小惩罚而施大儆戒,他就不至于犯下死罪了。我当时只想心存厚道,那知道,反而养成他天不怕地不怕的恶性。”这就是存善心,反而做了恶事的例子。

也有存恶心反而做了善事的例子。像有一个大富人家,碰到荒年,穷人大白天在市场上抢米。这个大富人家便告到县官那里,县官偏偏不受理这个案子,穷人因此更加肆无忌惮。于是大富人家就把抢米的人捉起来关在一起,辱骂他们。那些抢米的人反而安定下来,否则就天下大乱了。这就是存恶心反而做了善事的例子。

大家都知道善是正,恶是偏。善心办恶事,叫正中的偏,恶心办善事,叫偏中的正,这道理大家不能不知。

第六,半满。《易经》说:“善不积,不足以成名,恶不积,不足以灭身。”(意思是一个人不积善不会成就好名声,不积恶则不会引来杀身之祸)《书经》上说:“商罪贯盈,如贮物于器。”(意思是商朝的罪孽,用绳索串起来也能穿满,用器皿装起来也能装满)如果你很勤奋地积福,那么日积月累,总有一天你会积满福来改变自己地命运;如果你很懒惰,三天打鱼两天晒网,那么你很难积够福。这是讲半善满善的第一种说法。

从前有一户人家的女子,到佛寺里去,准备捐点香油钱,可惜身上只有两文钱,就都捐了出来。虽然只是两文钱,寺里的首席和尚竟然亲自替她在佛前回向,求忏悔灭罪。后来这位女子进皇宫做了贵妃,富贵之后便带几千两银子来寺里布施,但是主僧只安排徒弟替她回向。那女子有些不解,就问主僧说:“我从前不过布施两文钱,师父就亲自替我忏悔。现在我布施了几千两银子,而师父不替我回向,不知是什么道理?”主僧说:“从前布施的银子虽然少,但是你布施的心,很真切虔诚,所以非我老和尚亲自替你忏悔,便不足以报答你布施的功德;现在布施的钱虽然多,但是你布施的心,不像从前那么真切,所以叫人代你忏悔,也就够了。”这就是为什么几千两银子的布施只能算是半善,而两文钱的布施却是满善的道理。这是讲半善满善的第二种说法。

汉朝人钟离送金丹给吕洞宾,让他用金丹点白铁,使白铁变成黄金,再用黄金救济世上穷人。吕洞宾问钟离说:“这些金子最后会不会变回铁呢?”钟离说:“五百年以后,仍旧要变回铁。”吕洞宾又说:“这样就会害了五百年以后的人,我不做这样的事。”钟离见吕洞宾心地善良,就对他说:“修仙要积满三千件功德,听你这句话,你的三千件功德已经做圆满了。”这是讲半善满善的第三种说法。

一个人做善事,而内心没有叨念,若无其事、了无牵挂,那么不管做什么善事,都算功德圆满;若是做善事,心就牢记着这件善事,虽然一生都很勤勉的做善事,也只不过是半善而已。譬如拿钱去救济人,要内不见布施的我,外不见受布施的人,中不见布施的钱,这才叫三轮体空,也叫一心清净。如果能够这样布施,那么纵使布施一斗米,也可以种下无边无际的福了;即使布施一文钱,也可以消除一千劫所造的罪业。如果这个心,不能够忘掉所做的善事,那么即使用二十万两黄金去救济别人,还是不能够得到圆满的福报。这是讲半善满善的第四种说法。

第七,大小。从前有一个人叫做卫仲达,在翰林院里做官,有一次他的魂魄被鬼卒引到阴间。阴间的主审判官,吩咐手下的书办,把他在阳间所做的善事、恶事两种册子送上来。等册子送到一看,他的恶事册子,竟然摊满了整个院子,而善事的册子,只不过像一支筷子那样小。主审官又吩咐拿秤来过秤,那摊满院子的恶册子反而比较轻,而像筷子那样小的善册子反而比较重。卫仲达就问说:“我年纪还不到四十岁,怎会犯下这么多的罪过?”主审官说:“只要一个念头不正,就是罪过,不必等到你去实施。”卫仲达又问善册子里记的是什么事。主审官说:“有一次皇帝要大兴土木,修三山地方的石桥。你上奏劝皇帝不要修,免得劳民伤财,这是你的奏章底稿。”卫仲达说:“我虽然讲过,但是皇帝不听,最后还是动工了。我的奏章并没有发生作用,怎么还有这么重呢?”主审官说:“皇帝虽然没有听从你的建议,但是你这个念头,目的是要使千万百姓免去劳役。倘使皇帝听从你的建议,那善的重量就更大了!”所以立志做善事是为了利益天下百姓,那么不管善事多么小,功德都很大。假使只是为了自己,那么不管善事有多少,功德都很小。

第八,难易。从前读书人都说,克制自己的私欲,要从最困难的地方开始。孔子也讲,仁道要在最难的地方下工夫。孔子所说的难,也就是除掉私心,并应该先从最难做,最难克除的地方做起。一定要像江西的舒老先生,用自己教书两年所得的薪水,帮助一户穷人还清欠官府的债务,使他们夫妇不被拆散。或者像河北邯郸县的张老先生,用自己十年的积蓄,替一位穷人赎回妻儿,使妻儿能生活下去。像舒老先生和张老先生,都是在最难以布施的地方布施。又像江苏镇江的靳老先生,年老没有儿子,因不忍心误了邻家少女的青春,而拒绝纳其为妾。这就是在最难忍的地方忍。所以上天赐给这几位老先生的福,也特别的丰厚。

凡是有财有势的人要立些功德,比平常人来得容易,但是容易做,却不肯做,那就是自暴自弃了;而没钱没势的穷人,要积些福,相对比较因难,难却能做到,这才真是可贵啊!

我们为人处事,应该顺着因缘去救济众人,救济众人的种类很多,可以简单地归纳为十类:

第一,与人为善。古时候,舜在雷泽湖边看渔夫捕鱼。看到那些年轻力壮的都占据水深流缓鱼多的地方,而那些年老体弱的只能在水浅流急鱼少的地方,舜心里就很不舒服。他想了一个方法,自己也去捕鱼,见到那些占据好地方的人,不吭声,见到那些让出好地方的年轻人,便到处称赞他,拿他作榜样。就这样过了一年,年轻力壮的渔夫都能主动让出鱼多的地方给老年渔夫了。像舜那么深明事理的人,为什么就不能说几句中肯的话来教化众人,而一定要亲自参与呢?舜不用言语来教化众人,而是率先垂范、以身作则,使人自觉渐愧而主动改正错误,这真是用心良苦啊!

我们生在末法时代、多元社会里,千万不要用自己的长处掩盖别人的短处;也不要用自己的善行彰显别人的恶习;也不要用自己的才华使别人难堪;要虚怀若谷、大智若愚,时刻收敛自己的聪明才智;看到别人有过失,要包容他,帮他掩盖,这样一方面给他改过自新的机会,另一方面促使他有所顾忌而不敢太过放肆;看到旁人的小长处小善行,都要放下自己的成见,虚心向他学习,并且称赞他,替他广为传扬。一个人若能在日常生活的一言一行,一举一动中,彻底去除自私自利的念头,只留下为社会大众谋福利、做贡献、树榜样的心,那么他就是胸怀天下的伟人。

第二,爱敬存心。君子与小人,简单从言谈举止上看很容易混淆,但只要分析他们的本心(好像我们讲的世界观价值观)善恶,就会发现他们的差别其实很大,简直就是黑白分明。所以孟子说:“君子所以异于人者,以其存心也。”(意思是说君子与常人不同的地方就在他们的本心)

君子的本心,是爱人敬人的心。人虽然有亲近、疏远、尊贵、低微、聪明、愚笨、高尚、下流等种种不同,但都是我们的同胞,本来跟我们一体,哪一个不该爱敬呢?爱敬众人,就是爱敬圣贤人。能够明白众人的意思,就是明白圣贤的意思。这是为什么呢?因为圣贤本来希望全世界的人都能安居乐业,生活幸福美满。我们若能处处爱人敬人,使我们周围的人,都能生活得幸福美满,也就是帮助圣贤达成所愿。

第三,成人之美。一块未经雕刻的玉石,如果随便丢弃,那么它将与瓦片碎石无异。如果加以雕刻琢磨,那么它将变成宝物圭璋。人要成才,也离不开正确的教育劝导,所以看到别人要做善事,或者别人追求上进,而且天资不错时,就要尽力引导帮助他,使他得成所愿。或者夸奖赞扬,或者激励扶持,如果他受了冤屈,就要替他辩解冤屈、澄清诽谤,一定要使他立身于社会,这才算是尽了我的心意。

常人都讨厌与他不同类型的人,我们世俗人的习性善少恶多,所以善人处在世俗里,常常被人孤立。况且豪杰多数刚正不屈,且不拘小节,很容易落人口实,受人攻击。所以世间善事常要失败,善人也常被诽谤。只有仁人长者,才能坚持正义,主持公道,纠正歪风邪气,帮助善人得以成就。成人之美的功德实在宏大!

第四,劝人为善。人生在世,哪一个没有良心?然而世道坎坷,社会上人人都在追求名利权势,因而障蔽真心迷失自性。一个人如果没有足够的定力,很容易受影响而走上歪路。所以,有缘跟迷失自性的人相处共事时,一定要随机提醒他,使他破除迷惑走上正道。就如当他做梦时叫醒他,使他清醒过来,觉悟真理,又如在他烦恼时,使他摆脱困境,重拾舒畅轻松心情。这些恩惠功德最为广博。

从前韩文公说:“一时劝人以口,百世劝人以书。”(意思是以口来劝人,只在一时,以书来劝人,可以流传百世)劝人为善跟前面讲的与人为善比较起来,虽然是有所为而为之,带着较强的目的性和功利色彩,然而却能对症下药,经常会有特殊的效果,这种方法也不可以放弃。如果劝人时话不投机,或者如对牛弹琴对方毫无反应,或者对方并不接受甚至反唇相讥,就要检讨自己是否操之过急。

第五,救人危急。陷入困境或者颠沛流离的生活,人一辈子中难免要遇上。偶然碰到身处困境的人,要当作发生在自己身上一样,赶快相助解救。或者用话语帮助他申辩冤屈,或者想方设法帮他渡过难关。明朝崔子曾经说:“惠不在大,赴人之急可也。”(意思是恩惠不要求大,只要能在别人危急时施以援手就行)这句话真正是仁者的话呀!

第六,兴建大利。小至一个乡,大到一个县,凡是有益公众的事,最适宜开展。或是开辟沟渠来引水灌溉农田;或是建筑堤岸来预防水灾;或是修筑桥梁以方便人们出行;或是施送茶饭以救济饥饿口渴的路人。只要有机会,就要劝导大家,同心协力,共商义举,不要因为避嫌、怕辛苦、担心受埋怨而退缩不作。

第七,舍财作福。佛门里有万种善行,以布施最为重要。讲到布施,实际上就一个“舍”字。真正明白“舍”的道理的人,什么都“舍”得掉。对内能舍掉眼睛、耳朵、鼻子、舌头、身体、念头等六种感官的功能,对外不受色、声、香、味、触、法六种感觉的影响,一切东西感觉念头没有舍不掉的。若能如此,那就身心清净,了无牵挂,潇洒自由,生死自在了。如果没有这种功夫,就从钱财布施着手吧!世间人都把穿衣吃饭,看得像生命一样重要,因此钱财上的布施也最为重要。如果我能够痛痛快快地施舍钱财,那么对内我们可以破除吝啬小气的毛病,对外我们可以救济别人的急难。布施钱财也不容易做到,刚开始时有些勉强,坚持下去也就自然了。这种方法最有利于消除贪念私欲,也可以除掉自己对钱财的执著与依赖。

第八,护持正法。法就是真理,就是千万年来有灵性有生命的众生的眼目。如果没有正法,天地如何造化万物?万物如何生化成长?人们如何摆脱束缚走向自由世界?又如何治理天下,脱离生死轮回的苦海?所以凡是看到圣贤的寺庙、雕像、经典、遗训,都要加以敬重,有破损不全的,要进行修补整理。至于宣传佛法,报答佛恩,就更应该全力以赴地进行了。

第九,敬重尊长。对家里的父亲、兄长,国家的君王、长官,以及年岁、道德、职位、见识高的人,都要虔诚敬重、细心侍奉。在家里侍奉父母,要有深爱父母的心,要委婉和顺、心平气和并习以为常,因为和气可以感动天心;在外服务人民,处理政事,不能以为别人不知道就肆意妄为;审理罪犯疑人,也不能以为别人不知道就耀武扬威;服务人民,要像面对上天一样恭敬。这是古人所订立的规矩,跟阴德的关系最大。不信你看,凡是忠孝人家的子孙,没有不长久发达、兴旺昌盛的,所以一定要小心谨慎地去做。

第十,爱惜物命。人之所以为人,就在于他有一片恻隐之心。求仁的,就是求这一片恻隐之心,积德的,也是积这一片恻隐之心。《周礼》说:“孟春之月,牺牲毋用牝。”(意思是每年正月时,祭品不能用雌性的)孟子说:“君子远庖厨。”(意思是君子不肯住在厨房附近)这都是要保全自己的恻隐之心。所以,前辈有四种肉不吃的禁忌:动物被杀时听到声音的不吃,被杀时看见的不吃;自己养大的不吃,因为我而杀的不吃。后辈的人一下子断不了荤食,也应该从不吃这四种肉开始减少吃荤食。渐渐地减少吃肉,慈悲心也在渐渐地增加。不但杀生应戒,就是昆虫等有生命的动物,都不应该伤害它们的性命。像用丝来做衣服,就要把蚕茧放在水里煮,锄地种田也会杀害地下生物的性命。想想我们穿的衣服、吃的饭都要由牺牲很多生命而换来。所以糟蹋粮食,浪费东西的罪孽,跟杀生的罪孽差不多。至于随手误伤的生命,脚下误踏而死的生命,又不晓得有多少,这都要设法防止。宋朝的苏东坡写的诗说:“爱鼠常留饭,怜蛾不点灯。”(意思是说担心老鼠会饿死,就为老鼠留些饭;哀怜飞蛾会扑到灯上烫死,所以不点灯)这话多么仁厚慈悲呀!

善事无穷无尽,哪能说得完?只要把上边说的十件事,推广发扬,那么无数的功德就能圆满了。


第四篇 谦德之效

《易经·谦卦》说:“天道亏盈而益谦,地道变盈而流谦,鬼神害盈而福谦,人道恶盈而好谦。”(意思是天道使盈满的受亏损,让谦虚的得益;地道使盈满的改变,使而谦虚的滋润;鬼神之道,使盈满的受害,使谦虚的得福;人之道,厌恶骄傲自满的人,喜欢谦虚谨慎的人)《易经》上六十四卦,每一卦爻中都有凶有吉,唯有这个谦卦,每一爻都是吉祥的。《书经》上也讲到:“满招损,谦受益。”(意思是:自满就会遭到损害,谦虚却能受益。)我多次参加考试,经常看到那些将要高中的贫寒读书人,脸上都洋溢着一片谦和安详的光彩。

辛未年(公元1571年),我到京城参加会试,嘉善同乡大约有十个人,丁敬宇最年轻也最谦虚,我告诉费锦坡说:“这位老兄,今年一定考中。”费锦坡问我怎么看出来的。我说:“只有谦虚的人才可以承受福报。你看我们十人当中,有像敬宇这样的吗?诚实厚道、不抢人先,恭恭敬敬、逆来顺受、小心谨慎,受人侮辱而不回应,被人诽谤而不争辩。一个人能够做到这样,就是天地鬼神也要保佑他,岂有不发达的道理?”等到放榜时,丁敬宇果然考中了。

丁丑年(公元1577年)我在京城里,和冯开之住在一起,看见他总是虚心自谦,面容和顺,大大改变了他小时候的习气。他有一位正直又诚实的朋友季霁岩,经常当面指责他的缺点,他都能平心静气地接受,从来不反驳一句话。我告诉他说:“一个人有福,一定有福的根苗;有祸,也一定有祸的预兆。只要能够谦虚,上天一定会帮助他的,你老兄今年必定能够登第!”后来冯开之果然考中了。

赵裕峰,名光远,山东省冠县人。不满二十岁就中了举人,后来参加会试却多次不中。他的父亲做嘉善县的主任秘书,裕峰随同父亲上任。裕峰非常倾慕嘉善县的名士钱明吾的学问,就拿自己的文章去见他。那晓得钱先生,竟然将他的文章都涂掉了。裕峰不但不发火,还心服口服地把自己文章的缺失改过来。如此虚心用功的年轻人实在是少有,第二年,裕峰就考中了。

壬辰年(公元1592年)我入京城觐见皇帝,见到一位叫夏建所的读书人,说法低声下气,脸上洋溢着谦虚的光彩。我回来告诉朋友说:“凡是上天要使让人发达,在没有发他的福时,一定先发他的智慧,智慧一发,浮滑的人也会变得诚实,放肆的人也会自动收敛。夏建所温和善良到这种地步,一定是发了智慧,上天就要发他的福了。”等到放榜时,夏建所果然考中了。

江阴有一位读书人,名叫张畏岩。他的学问很深,文章写得很好,在读书人当中颇有名声。甲午年(公元1594年)南京乡试,他借住在一处寺院里。放榜时榜上没有他的名字,他不服气,大骂考官瞎了眼,看不到他的文章好。那时候有一个道士在旁微笑,张畏岩马上把怒火发到道士身上。道士说:“你的文章一定不好。”张畏岩愤怒地说:“你没有看到我的文章,怎么知道我的文章不好?”道士说:“我听人说,写文章最要紧是要心平气和,现在听到你大骂考官,表示你的心非常不平,气这么冲,你的文章怎么会好呢?”张畏岩听了道士的话,不觉屈服了,就转过头来向道士请教。道士说:“考功名全靠命运,命里没有的,文章再好也没用,这时候想考中,就一定要改变自己的命运。”

张畏岩问道:“既然是命,怎样能改变呢?”道士说:“造命的虽然是天,但立命的却是你自己。只要你肯努力去做善事,多积阴德,什么福求不到呢?”

张畏岩说:“我是一个穷书生,能做什么善事呢?”

道士说:“行善事,积阴功,都是从你的心做出来的。只要常有做善事、积阴德的心,功德就无量无边了。就像谦虚这件事,又不要花钱,你为什么不自我反省,反而去骂考官不公平呢?”

张畏岩听了道士的话,就降低身份,克制自己,天天加功夫去修善积德。到了丁酉年(公元1597年),有一天,他梦中到了一处很高的房屋,看到一本名册,中间有许多的缺行。他看不懂,就问旁边的人。那个人说:“这是今年考试录取的名单。”张畏岩问:“为什么名册内有这么多缺行?”那个人回答说:“阴间每三年要考查一次那些可以高中的人,有积德没过失的,这册里才会有名字。名册前面的缺额,就是那些本该考中,但最近造了罪业而被除名的人。”那个人又指着一行说:“你三年来能严格要求自己,也许有机会补上这个缺,希望你珍重自爱!”果然张畏岩就在这次会考中,考中第一百零五名。

从上面所讲的看来,举头三尺高,一定有神明在监察每个人的行为。趋吉避凶,绝对是由自己决定的。自己只有存善心,约束一切不善的行为,丝毫不得罪天地鬼神,谦虚自处,总是委屈自己成就他人,使天地鬼神时时哀怜自己,这样才可以加固福报的根基。那些心高气傲的人,一定难成大器,就算能发达,也不会长久。稍有见识的人,一定不肯把自己的肚量弄得很狭窄,而拒绝自己本来可得的福报。况且谦虚的人,经常可以得到别人的教导,学习别人的长处。尤其是进德修业的人,更不能缺少谦虚的品格。

古话说:“有志于功名者,必得功名;有志于富贵者,必得富贵。”(意思是,有心要求功名的,一定可以得到功名;有心要求富贵的,一定可以得到富贵。)人要有远大的理想,就像树要有根一样。意志坚定,还要每一个念头、每一次行动都与人方便,这样就能感动天地,就能为自己造福从而改变自己的命运了。可惜现在那些求取功名的人,没有真正用心的,不过是一时的兴致罢了,心血来潮时就去行善积德,兴致退了就停止。
\backmatter

\end{document}