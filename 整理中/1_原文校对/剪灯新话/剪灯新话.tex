% 剪灯新话
% 剪灯新话.tex

\documentclass[a4paper,12pt,UTF8,twoside]{ctexbook}

% 设置纸张信息。
\RequirePackage[a4paper]{geometry}
\geometry{
	%textwidth=138mm,
	%textheight=215mm,
	%left=27mm,
	%right=27mm,
	%top=25.4mm, 
	%bottom=25.4mm,
	%headheight=2.17cm,
	%headsep=4mm,
	%footskip=12mm,
	%heightrounded,
	inner=1in,
	outer=1.25in
}

% 设置字体,并解决显示难检字问题。
\xeCJKsetup{AutoFallBack=true}
\setCJKmainfont{SimSun}[BoldFont=SimHei, ItalicFont=KaiTi, FallBack=SimSun-ExtB]

% 目录 chapter 级别加点(.)。
\usepackage{titletoc}
\titlecontents{chapter}[0pt]{\vspace{3mm}\bf\addvspace{2pt}\filright}{\contentspush{\thecontentslabel\hspace{0.8em}}}{}{\titlerule*[8pt]{.}\contentspage}

% 设置 part 和 chapter 标题格式。
\ctexset{
	part/name= {卷,},
	part/number={\chinese{part}},
	chapter/name={},
	chapter/number={}
}

% 设置古文原文格式。
\newenvironment{yuanwen}{\bfseries\zihao{4}}

% 设置署名格式。
\newenvironment{shuming}{\hfill\zihao{4}}

\title{\heiti\zihao{0} 剪灯新话}
\author{瞿佑}
\date{明洪武十一年}

\begin{document}

\maketitle
\tableofcontents

\frontmatter

\chapter{序一}

余既编辑古今怪奇之事。以为《剪灯录》,凡四十卷矣。好事者每以近事相闻,远不出百年,近止在数载,襞积于中,日新月盛,习气所溺,欲罢不能,乃援笔为文以纪之。其事皆可喜可悲,可惊可怪者。所惜笔路荒芜,词源浅狭,无嵬目鸿耳之论以发扬之耳。既成,又自以为涉于语怪,近于海淫,藏之书笥,不欲传出。客闻而求观者众,不能尽却之,则又自解曰:《诗》、《书》、《易》、《春秋》、皆圣笔之所述作,以为万世大经大法者也;然而《易》言龙战于野,《书》载雉雊于鼎,《国风》取淫奔之诗,《春秋》纪乱贼之事,是又不可执一论也。今余此编,虽于世教民彝,莫之或补,而劝善惩恶,哀穷悼屈,其亦庶乎言者无罪,闻者足以戒之一义云尔。客以余言有理,故书之卷首。

~\\

\begin{shuming}
洪武十一年岁次戊午六月朔日,山阳瞿佑书于吴山大隐堂
\end{shuming}

\chapter{序二}

昔陈鸿作《长恨传》并《东城老父传》,时人称其史才,咸推许之。及观牛憎孺之《幽怪录》,刘斧之《青琐集》,则又述奇纪异,其事之有无不必论,而其制作之体,则亦工矣。乡友瞿宗吉氏著《剪灯新话》,无乃类是乎?宗吉之志确而勤,故其学也博,具才充而敏,故其文也贍。是编虽稗官之流,而劝善惩恶,动存鉴戒,不可谓无补于世。矧夫造意之奇,措词之妙,粲然自成一家言,读之使人喜而手舞足蹈,悲而掩卷堕泪者,盖亦有之。自非好古博雅,工于文而审于事,曷能臻此哉!至于《秋香亭记》之作,则犹元稹之《莺莺传》也,余将质之宗吉,不知果然否?

~\\

\begin{shuming}
洪武三十年夏四月,钱塘凌云翰序
\end{shuming}

\chapter{序三}

余观宗吉先生《剪灯新活》,其词则传奇之流。其意则子氏之寓言也。宗吉家学渊源,博及群集,屡荐明经,母老不仕,得肆力于文学。余尝接其论议,观其著述,如开武库。如游宝坊,无非惊人之奇,希世之珍;是编特武库、室坊中之一耳。然则观是编者,于宗吉之学之博,尚有愆也。

~\\

\begin{shuming}
洪武十四年秋八月,吴植书于钱塘邑庠进德斋
\end{shuming}

\chapter{序四}

余观昌黎韩子作《毛颖传》,柳子厚读而奇之,谓若捕龙蛇,搏虎豹,急与之角,而力不敢暇;古之文人,其相推奖类若此。及子厚作《谪龙说》与《河间传》等,后之人亦未闻有以妄且淫病子厚者,岂前辈所见,有不逮今耶?亦忠厚之志焉耳矣。余友瞿宗吉之为《剪灯新话》,其所志怪,有过于马孺子所言,而淫则无若河间之甚者。而或者犹沾沾然置噱于其间,何俗之不古也如是!盖宗吉以褒善贬恶之学,训导之间,游其耳目于词翰之场,闻见既多,积累益富。恐其久而记忆之或忘也,故取其事之尤可以感发、可以惩创者,汇次成编,藏之箧笥,以自恰悦,此宗吉之志也。余下敏,则既不知其是,亦不知其非,不知何者为可取,何者为可讥。伏而观之,但见其有文、有诗、有歇、有词、有可喜、有可悲、有可骇、有可嗤。信宗吉于文学而又有余力于他著者也。宗吉索余题,故为赋古体一首以复之云。

山阳才人畴与侣?开口为今闔为古!

春以桃花染性情,秋将桂子薰言语。

感离抚遇心怦怦,道是无凭还有凭。

沉沉帐底昼吹笛。煦煦窗前宵剪灯。

倏而晴兮忽而雨,悲欲啼兮喜欲舞,

玉萧倚月吹凤凰,金栅和烟锁鹦鹉。

造化有迹尸者谁?一念才荫方寸移,

善善恶恶苟无失,怪怪奇奇将有之。

丈夫未达虎为狗,濯足沧浪泥数斗,

气寒骨耸铮有声,脱帻目光如电走。

道人青蛇天动摇,下斩寻常花月妖,

茫茫尘海沤万点,落落云松酒半瓢。

世间万事幻泡耳,往往有情能不死,

十二巫山谁道深,云母屏凤薄如纸。

莺莺宅前芳享述,燕燕楼中明月低,

从来松柏有孤操,不独鸳鸯能并栖。

久在钱塘江上住,厌见潮来又潮去,

燕子衔春几度回?断梦残魂落何处?

还君此编长啸歌,便欲酌以金叵罗,

醉来呼枕睡一觉,高车驷马游南柯。

~\\

\begin{shuming}
洪武己巳六月六日,睦人桂衡书于紫薇深处
\end{shuming}

\mainmatter

\part{}

\chapter{水宫庆会录}

至正甲申岁,潮州士人余善文于所居白昼闲坐,忽有力士二人,黄巾绣祆,自外而入,致敬于前曰:“广利王奉邀。”善文惊曰:“广利洋海之神,善文尘世之士,幽显路殊,安得相及?”二人曰:“君但请行,毋用辞阻。”遂与之偕出南门外,见大红船泊于江浒。登船,有两黄龙挟之而行,速如凤雨,瞬息已至。止于门下,二人入报。顷之,请入。广利降阶而接曰:“久仰声华,坐屈冠盖,幸勿见讶。”遂延之上阶,与之对坐。

善文局蹐退逊。广利曰:“君居阳界,寡人处水府,不相统摄,可毋辞也。”善文曰:“大王贵重,仆乃一介寒儒,敢当盛礼!”固辞。广利左右有二臣曰鼋参军、鳖主簿者,趋出奏曰:“客言是也,王可从其所请,不宜自损威德,有失观视。”广利乃居中而坐,别设一榻于右,命善文坐。乃言曰:“敝居僻陋,蛟鳄之与邻,鱼蟹之与居,无以昭示神威,阐扬帝命。今欲别构一殿,命名灵德,工匠已举,木石咸具,所乏者惟上梁文尔。侧闻君子负不世之才,蕴济时之略,故特奉邀至此,幸为寞人制之。”即命近侍取白玉之砚,捧文犀之管,井鲛绡丈许,置善文前。善文俯首听命,一挥而就,文不加点。其词曰:

伏以天壤之间,海为最大;人物之内,神为最灵。既属香火之依归,可乏庙堂之壮丽?是用重营宝殿,新揭华名;挂龙骨以为梁,灵光耀日;缉鱼鳞而作瓦,瑞气蟠空。列明珠白璧之帘栊,接青雀黄龙之舸舰。琐窗启而海色在户,绣闼开而云影临轩。雨顺风调,镇南溟八千余里;天高地厚,垂后世亿万斯年。通江汉之朝宗,受溪湖之献纳。天吴紫凤,纷纭而到;鬼国罗刹,次第而来。岿然着鲁灵光,美哉如汉景福。控蛮荆而引瓯越,永壮宏观;叫闾阖而呈琅玕,宜兴善颂。遂为短唱,助举修梁。

抛梁东,方丈蓬莱指顾中。笑看扶桑三百尺,金鸡啼罢日轮红。

抛粱西,弱水流沙路不迷。后衣瑶池王母降,一双青鸟向人啼。

抛梁南,巨浸漫漫万族涵。要识封疆宽几许?大鹏飞尽水如蓝。

抛梁北,众星绚烂环辰极。遥瞻何处是中原?一发青山浮翠色。

抛梁上,乘龙夜去陪天仗。袖中奏罢一封书,尽与苍生除祸瘴。

抛梁下,水族纷纶承德化。清晓频闻赞拜声,江坤河伯朝灵驾。

伏愿上粱之后,万族归仁,百灵仰德。珠宫贝阙,应无上之三光,衮衣绣裳,备人间之五福。

书罢,进呈。广利大喜。卜日落戍,发使诣东西北三海,请其王赴庆殿之会。翌日,三神皆至,从者千乘万骑,神鲛毒蜃,踊跃后先,长鲸大鲲,奔驰左右,鱼头鬼面之卒,执旌旄而操戈戟者,又不知其几多也。是日,广利顶通天之冠,御绎纱之袍,秉碧玉之圭,趋迎于门,其礼甚肃。三神亦各盛其冠冕,严其剑珮,威仪极俨恪,但所服之袍,各随其方而色不同焉。叙暄凉毕,揖让而坐。善文亦以白衣坐于殿角,方欲与三神叙礼,忽东海广渊王座后有一从臣,铁冠而长髭者,号赤餫公,跃出广利前而请曰:“今兹贵殿落成,特为三王而设斯会,虽江汉之长,川泽之君,咸不得预席,其礼可谓严矣。彼白衣而末坐者为何人斯?乃敢于此唐突也!”广利曰:“此乃潮阳秀士余君善文也,吾构灵德殿,请其作上梁文,故留之在此尔。”广渊遽言曰:“文士在座,汝乌得多言?姑退!”赤餫公乃赧然而下。已而酒进乐作,有美女二十人,摇明璫,曳轻裾,于筵前舞凌波之队,歌凌波之词曰:

若有人兮波之中,折杨柳兮采芙蓉。振瑶环兮琼珮,璆鏘呜兮玲瓏。衣翩翩兮若惊鸿,身矫矫兮如游龙。轻尘生兮罗袜,斜日照兮芳容。蹇独立兮西复东,羌可遏兮不可从。忽飘然而长往,御泠泠之轻凤。

舞竟,复有歌童四十辈,倚新妆,飘香袖,于庭下舞采莲之队,歌采莲之曲曰:

桂棹兮兰舟,泛波光兮远游。捐予玦兮别浦,解予珮兮芳洲。波摇摇兮舟不定,折荷花兮断荷柄。露何为兮沾裳?風何为兮吹鬓?棹歌起兮彩袖挥,翡翠散兮鸳鸯飞。张莲叶兮为盖,缉藕丝兮为衣。日欲落兮风更急,微烟生兮淡月出。早归来兮难久留,对芳华兮乐不可以终极。

二舞既毕,然后击灵鼍之鼓,吹玉龙之笛,众乐毕陈,觥筹交错。于是东西北三神,共捧一觥,致善文前曰:“吾等僻处遐陬,不闻典礼,今日之会,获睹盛仪,而又幸遇大君子在座,光采倍增,愿为一诗以记之,使流传于龙官水府,抑亦一胜事也。不知可乎?”善文不可辞,遂献水宫庆会诗二十韵:

帝德乾坤大,神功岭海安。

渊宫舟栋宇,水路息波澜。

列爵王侯贵,分符地界宽。

威灵闻赫羿,事业保全完。

南极常通奏,炎方永授官。

登堂朝玉帛,设宴会衣冠。

凤舞三檐盏,龙驮七宝鞍。

传书双鲤跃,扶辇六鳌蟠。

王母调金鼎,天妃捧玉盘。

杯凝红琥珀,袖拂碧琅玕。

座上湘灵舞,频将锦瑟弹。

曲终汉女至,忙把翠旗看。

瑞雾迷珠箔,祥烟绕画栏。

屏开云母莹,帘卷水晶寒。

共饮三危露,同餐九转丹。

良辰宜酩酊,乐事称盘桓。

异昧充喉舌,灵光照肺肝。

浑如到兜率,又似梦邯郸。

献酢陪高台,歌呼得尽欢。

题诗传胜事,春色满毫端。

诗进,座间大悦。已而,日落咸池,月生东谷,诸神大醉,倾扶而出,各归其国,车马駢阗之声,犹逾时不绝。明日,广利特没一宴,以谢善文。宴罢,以玻璃盘盛照夜之珠十,通天之犀二,为润笔之资,复命二使送之还郡。善文到家,携所得于波斯宝肆鬻焉,获财忆万计,遂为富族。后亦不以功名为意,弃家修道,遍游名山,不知所终。

\chapter{三山福地志}

元自实,山东人也。生而质钝,不通诗书。家颇丰殖,以田庄为业。同里有缪君者,除得闽中一官,缺少路费,于自实处假银二百两。自实以乡党相处之厚,不问其文券,如数贷之。至正末,山东大乱,自实为群盗听劫,家计一空。时陈有定据守福建,七闽颇安。自实乃挈奏子由海道趋福州,将访缪君而投托焉。至则缪君果在有定幕下,当道用事,威权隆重,门户赫弈。自实大喜,然而患难之余,跋涉道途,衣裳褴缕,客貌憔粹,未敢遽见也。乃于城中僦屋,安顿其妻孥,整饰其冠服,卜日而往。适值缪君之出,拜于马首。初似不相识,及叙乡井,通姓名,方始惊谢。即延之入室,待以宾主之礼。良久,啜茶而罢。明日,再往,酒果三杯而已,落落无顾念之意,亦不言银两之事。自实还家,旅寓荒凉,妻孥怨詈曰:“汝万里投人,听干何事?今为三杯薄酒所卖,即便不出一言,吾等何所望也!”自实不得已,又明日,再往访焉,则似已厌之矣。自实方欲启口,缪君遽曰:“向者承借路费,铭心不忘,但一宦萧条,俸入微薄,故人远至,岂敢辜恩,望以文券付还,则当如数陆续酬纳也。”自实悚然曰:“与君共同乡里,自少交契深密,承命周急,素无文券,今日何以出此言也?”缪君正色曰:“文券承有之,但恐兵火之后,君失之耳。然券之有无,某亦不较,惟望宽其程限,使得致力焉。”自实唯唯而出,怪其言辞矫妄,负德若此,羝羊触藩,进退维谷。半月之后,再登其门,惟以温言接之,终无一钱之惠。展转推托,遂及半年。市中有一小庵,自实往缪君之居,适当其中路,每于门下憩息。庵主轩辕翁者,有道之士也,见其往来颇久,与之叙话,因而情熟。时值季冬,已迫新岁,自实穷居无聊,诣缪君之居,拜且泣曰:“新正在迩,妻子饥寒,囊乏一钱,瓶无储粟。向者银两,今不敢求,但愿捐斗水而活涸辙之枯,下壶飨而救翳桑之饿,此则故人之赐也。伏望怜之悯之,哀之恤之!”遂匍匐于地。缪君扶之起,屈指计日之数,而告之曰:“更及一旬,当是除夕,君可于家专待,吾分禄米二右及银二锭,令人驰送于宅,以为过岁之资,幸勿以少为怪。”且又再三丁宁。毋用他出以候之。自实感谢而退。归以缪君之言慰其妻子。至日,举家悬望,自实端坐于床,令椎子于里门覘之。须臾,奔入曰:“有人负米至矣。”急出俟焉,则越其庐而不顾。自实犹谓来人不识其家,趋往问之,则曰:“张员外之馈馆宾者也。”默然而返。顷之,稚子又入告曰:“有人携钱来矣。”急出迓焉,则过其门而不入。再住扣之,则曰:“李县令之赆游客者也。”怃然而惭。如是者凡数度。

至晚,竟绝影响。明日,岁旦矣,反为所误,粒米束薪,俱不及办,妻子相向而哭。自实不胜其愤,阴砺白刃,坐以待旦。鸡鸣鼓绝,径投缪君之门,将俟其出而刺之。是时,晨方未启,道无行人,惟小庵中轩辕翁方明烛转经,当门而坐,见自实前行,有奇形异状之鬼数十辈从之,或握刀剑,或执椎凿,披头露体,势甚凶恶;一饭之顷,则自实复回,有金冠玉珮之士百余人随之,或击幢盖,或举旌幡,和容婉色,意甚安闲。轩辕翁叵测,谓其已死矣。诵经已罢,急往访之,则自实固无恙。坐定,轩辕翁问曰:“今日之晨,子将奚适?何其去之匆匆,而回之缓缓也?愿得一闻。”自实不敢隐,具言:“缪君之不义,令我狼狈!今早实砺霜刃于怀,将往杀之以快意,及至其门,忽自思曰:‘彼实得罪于吾,妻子何尤焉。且又有老母在堂,今若杀之,其家何所依?宁人负我,毋我负人也。’遂隐忍而归耳。”

轩辕翁闻之,稽首而贺曰:“吾子将有后禄,神明已知之矣。”自实问其故。翁曰:“子一念之恶,而凶鬼至;一念之善,而福神临。如影之随形,如声之应响,固知暗室之内,造次之间,不可荫心而为恶,不可造罪而损德也。”因具言其所见而慰抚之,且以钱米少许周其急。然而自实终郁郁不乐。至晚,自投于三神山下八角井中。其水忽然开辟,两岸皆石壁如削,中有狭径,仅通行履。自实扪壁而行,将数百步,壁尽路穷,出一弄口,则天地明朗,日月照临,俨然别一世界也。见大宫殿,金书其榜曰:“三山福地。”自实瞻仰而入,长廊昼静,古殿烟消,徘徊四顾,阒无人踪,惟闻钟磐之声,隐隐于云外。饥馁颇甚,行末能前,困卧石坛之侧。忽一道土,曳青霞之裾,振明月之珮,至前呼起之,笑而问曰:“翰林识旅游滋味乎?”自实拱而对曰:“旅游滋味,则尽足矣。翰林之称,一何误乎?”道士曰:“子不忆草西蕃诏于兴圣殿乎?”自实曰:“某山东鄙人,布衣贱士,生岁四十,目不知书,平生未尝游览京国,何有草诏之说乎?”道士曰:“子应为饥火所恼,不暇记前事耳。”乃于袖中出梨枣数枚令食之,曰:“此谓交梨火枣也。食之当知过去未来事。”

自实食讫,惺然明悟,因记为学士时,草西蕃沼于大都兴圣殿侧,如昨日焉。遂请于道士曰:“某前世造何罪而今受此报耶?”道士曰:“子亦无罪,但在职之时,以文学自高,不肯汲引后进,故今世令君愚懵而不识字;以爵位自尊,不肯接纳游士,故今世令君漂泊而无所依耳。”自实因指当世达官而问之曰:“某人为丞相。而贪饕不止,贿赂公行,异日当受何报?”道士曰:“彼乃无厌鬼王,地下有十炉以铸其横财,今亦福满矣,当受幽囚之祸。”又问曰:“某人为平章,而不戢军士,杀害良民,异日当受何报?”道士曰:“彼乃多杀鬼王,有阴兵三百,皆铜头铁额,辅之以助其虐,今亦命衰矣,当受割截之殃。”又问:“某人为监司,而刑罚不振;某人为郡守,而赋役不均;某人为宣慰,不闻所宣之何事;某人为经略,不闻所略之何方,然则当受何报也?”道士曰:“此等皆已杻械加其身,缧绁系其颈,腐肉秽骨,待戮余魂,何足算也!”自实因举缪君负债之事。道士曰:“彼乃王将军之库子,财物岂得妄动耶?”道士因言:“不出三年,世运变革,大祸将至,甚可畏也。汝宜择地而居,否则恐预池鱼之殃。”自实乞指避兵之地。道士曰:“福清可矣。”又曰:“不若福宁。”言讫,谓自实曰:“汝到此久,家人悬望,今可归矣。”自实告以无路,道士指一径令其去,遂再拜而别。行二里许,于山后得一穴出,到家,则已半月矣。急携妻子径往福宁村中,垦田治圃而居。挥钁之际,铮然作声,获瘫银四锭,家遂稍康。其后张氏夺印,达丞相被拘,大军临城,陈平章遭掳,其余官吏多不保其首领,而缪君为王将军者所杀,家资皆归之焉。以岁月记之,仅及三载,而道士之言悉验矣。

\chapter{华亭逢故人记}

松江士人有全、贾二子者,皆富有文学,豪放自得,嗜酒落魄,不拘小节,每以游侠自任。至正末,张氏居有浙西,松江为属郡。二子来往其间,大言雄辩,旁若无人。豪门叵族,望风承接,惟恐居后。全有诗曰:

华发冲冠感二毛,西风凉透鹣鹴袍。

仰天不敢长嘘气,化作虹霓万丈高。

贾亦有诗曰:

四海干戈未息肩,书生岂合老林泉!

袖中一把龙泉剑,撑拄东南半壁天。

其诗大率类是,人益信其自负。吴元年,国兵围姑苏,未拨。上洋人钱鹤皋起兵援张氏,二子自以严庄、尚让为比,杖策登门,参其谋议,遂陷嘉兴等郡。未几,师溃,皆赴水死。洪武四年。华亭士人石若虚,有故出近郊。素与二子友善,忽遇之于途,随行僮仆救人,气象宛如平昔。迎谓若虚曰:“石君无恙乎?”若虚忘其已死,与之揖让,班荆而坐子野,谈论逾时。全忽慨然长叹曰:“诸葛长民有言:‘贫贱长思富贵,富贵复履危机。’此语非确论。苟慕富贵,危机岂能避?世间宁有扬州鹤耶?丈夫不能流芳百世,亦当遗臭万年。刘黑闼既立为汉东王,临死乃云:‘我本在家锄莱,为高雅贤辈所误至此!’陋哉斯言,足以发千古一笑也!”贾曰:“黑闼何足道!如汉之田横,唐之李密,亦可谓铁中铮铮者也。横始与汉祖俱南面称孤,耻更称臣,逃居诲岛,可以死矣,乃眩于大王小侯之语,行至东都而死。密之起兵,唐祖以书贺之,推为盟主,及兵败入关,乃望以台司见处,其无知识如此!大丈夫死即死矣,何忍向人喉下取气耶?夫韩信建炎汉之业,卒受诛夷;刘文静启晋阳之祚,终加戮辱。彼之功臣尚尔,于他人何有哉!”全曰:“骆宾王佐李敬业起兵,檄武氏之恶,及兵败也,复能优游灵隐,咏桂子天香之句。黄巢扰乱唐室,罪下容诛,至于事败,乃削发被缁,逃遁踪迹,题诗云:‘铁衣著尽著僧衣。’若二人者,身为首恶,而终能脱祸,可谓智术之深矣。”贾笑曰:“审如此,吾辈当愧之矣!”全遽曰:“故人在坠,不必闲论他事,徒增伤感尔。”因解所御绿裘,令仆于近村质酒而饮。酒至,饮数巡,若虚请于二子曰:“二公平日篇什,播在人口,今日之会,可无佳制以记之乎?”于是筹思移时,全诗先成,即吟曰:

几年兵火接天涯,白骨丛中度岁华。

杜宇有冤能泣血,邓攸无子可传家。

当时自诧辽东豕,今日翻成井底蛙。

一片春光谁是主,野花开满蒺藜沙。

贾继诗曰:

漠漠荒郊鸟乱飞,人民城郭叹都非。

沙沉枯骨何须葬,血污游魂不得归。

麦饭无人作寒食,绨袍有泪哭斜晖。

生存零落皆如此,惟恨平生壮志违。

吟已,若虚骇曰:“二公平日吟咏极宕,今日之作,何其哀伤之过,与畴昔大不类耶?”二人相顾无语,但愀然长啸数声。须臾,酒罄,告别而去。行及十数步,阒无所见。若虚大惊,始悟其死久矣。但见林梢烟瞑,岭首日沉,乌啼鹊噪于丛薄之间而已。急投前村酒家,访其历以取质酒之裘视之,则触手纷纷而碎,若蝶翅之抟风焉。若虚借宿酒家,明早急回。其后再下敢经由是路矣。

\chapter{金凤钗记}

大德中,扬州富人吴防御居春风楼侧,与宦族崔君为邻,交契甚厚。崔有子曰兴哥,防御有女曰兴娘,俱在襁褓。崔君因求女为兴哥妇,防御许之,以金凤钗一只为约。 既而崔君游宦远方,凡一十五载,并无一字相闻。

女处闺闱,年十九矣。其母谓防御曰:“崔家郎君一去十五载,不通音耗,兴娘长成矣,不可执守前言,令其挫失时节也。”防御曰:“吾已许吾故人矣,况成约已定,吾岂食言者也。”女亦望生不至,因而感疾,沉绵枕席,半岁而终。父母哭之恸。 临敛,母持金凤钗抚尸而泣曰:“此汝夫家物也,今汝已矣,吾留此安用!”遂簪于其髻而殡焉。殡之两月,而崔生至。防御延接之,访问其故,则曰:“父为宣德府理官而卒,母亦先逝数年矣,今已服除,故不远千里而至此。”防御下泪曰:“兴娘薄命,为念君故,得疾,于两月前饮恨而终,今已殡之矣。”因引生入室,至其灵几前,焚楮钱以告之,举家号恸。 防御谓生曰:“郎君父母既殁,到途又远,今既来此,可便于吾家宿食。故人之子,即吾子也,勿以兴娘殁故,自同外人。”

即令搬挈行李,于门侧小斋安泊。将及半月。时值清明,防御以女殁之故,举家上冢。 兴娘有妹曰庆娘,年十七矣,是日亦同往。惟留生在家看守。

至暮而归,天已曛黑,生于门左迎接;有轿二乘,前轿已入,后轿至生前,似有物堕地,铿然作声,生俟其过,急往拾之,乃金凤钗一只也。欲纳还于内,则中门已阖,不可得而入矣。遂还小斋,明烛独坐。自念婚事不成,只身孤苦,寄迹人入门,亦非久计,长叹数声。方欲就枕,忽闻剥啄扣门声,问之不答,斯须复扣,如是者三度。乃启关视之,则一美姝立于门外,见户开,遽搴裙而入。生大惊。女低容敛气,向生细语曰:“郎不识妾耶?妾即兴娘之妹庆娘也。向者投钗轿下,郎拾得否?”即挽生就寝。生以其父待之厚,辞曰:“不敢。”拒之甚厉,至于再三。女忽赪尔怒曰:“吾父以子侄之礼待汝,置汝门下,汝乃于深夜诱我至此,将欲何为?我将诉之于父,讼汝于官,必不舍汝矣。”生惧,不得已而从焉。至晓,乃去。自是暮隐而入,朝隐而出,往来于门侧小斋,凡及一月有半。一夕,谓生曰:“妾处深闺,君居外馆,今日之事,幸而无人知觉。 诚恐好事多魔,佳期易阻,一旦声迹彰露,亲庭罪责,闭笼而锁鹦鹉,打鸭而惊鸳鸯,在妾固所甘心,于君诚恐累德。莫若先事而发,怀璧而逃,或晦迹深村,或藏踪异郡,庶得优游偕老,不致睽离也。”生颇然其计,曰:“卿言亦自有理,吾方思之。”因自念零丁孤苦,素质亲知,虽欲逃亡,竟将焉往?尝闻父言:有旧仆金荣者,信义人也,居镇江吕城,以耕种为业。 今往投之,庶不我拒。至明夜五鼓,与女轻装而出,买船过瓜州,奔丹阳,访于村氓,果有金荣者,家甚殷富,见为本村保正。生大喜,直造其门,至则初不相识也,生言其父姓名爵里及己乳名,方始记认,则设位而哭其主,捧生而拜于座,曰:“此吾家郎君也。”生具告以故,乃虚正堂而处之,事之如事旧主,衣食之需,供给甚至。生处荣家,将及一年。女告生曰:“始也惧父母之责,故与君为卓氏之逃,盖出于不获已也。今则旧谷既没,新谷既登,岁月如流,已及朞矣。且爱子之心,人皆有之,今而自归,喜于再见,必不我罪。况父母生我,恩莫大焉,岂有终绝之理?盍往见之乎?”生从其言,与之渡江入城。将及其家,谓生曰:“妾逃窜一年,今遽与君同往,或恐逢彼之怒,君宜先往觇之,妾舣舟于此以俟。”临行,复呼生回,以金凤钗授之,曰:“如或疑拒,当出此以示之,可也。”生至门,防御闻之,欣然出见,反致谢曰:“日昨顾待不周,致君不安其所,而有他适,老夫之罪也。幸勿见怪!”生拜伏在地,不敢仰视,但称“死罪”,口不绝声。防御曰:“有何罪过?遽出此言。愿赐开陈,释我疑虑。 ”生乃作而言曰:“曩者房帷事密,儿女情多,负不义之名,犯私通之律,不告而娶,窃负而逃,窜伏村墟,迁延岁月,音容久阻,书问莫传,情虽笃于夫妻,恩敢忘乎父母!今则谨携令爱,同此归宁,伏望察其深情,恕其重罪,使得终能偕老,永遂又于飞。 大人有溺爱之恩,小子有宜家之乐,是所望也,惟翼悯焉。”防御闻之,惊曰:“吾女卧病在床,今及一岁,饘粥不进,转侧需人,岂有是事耶?”

生谓其恐为门户之辱,故饰词以拒之,乃曰:“目今庆娘在于舟中,可令人舁取之来。”防御虽不信,然且令家僮驰往视之,至则无所见。方诘怒崔生,责其妖妄,生于袖中,出金凤钗以进。 防御见,始大惊曰:“此吾亡女兴娘殉葬之物也,胡为而至此哉?”疑惑之际,庆娘忽于床上欻然而起,直至堂前,拜其父曰:“兴娘不幸,早辞严侍,远弃荒郊,然与崔家郎君缘分未断,今之来此,意亦无他,特欲以爱妹庆娘,续其婚耳。如所请肯从,则病患当即痊除;不用妾言,命尽此矣。”举家惊骇,视其身则庆娘,而言词举止则兴娘也。父诘之曰:“汝既死矣,安得复于人世为此乱惑也?”对曰:“妾之死也,冥司以妾无罪,不复拘禁,得隶后士夫人帐下,掌传笺奏。妾以世缘未尽,故特给假一年,来与崔郎了此一段因缘尔。”

父闻其语切,乃许之,即敛容拜谢,又与崔生执手歔欷为别。 且曰:“父母许我矣!汝好作娇客,慎毋以新人而忘故人也。”言讫,恸哭而仆于地,视之,死矣。急以汤药灌之,移时乃苏,疾病已去,行动如常,问其前事,并不知之,殆如梦觉。 遂涓吉续崔生之婚。生感兴娘之情,以钗货于市,得钞二十锭,尽买香烛楮币,赉诣琼花观,命道士建醮三昼夜以报之。复见梦于生曰:“蒙君荐拔,尚有余情,虽隔幽明,实深感佩。小妹柔和,宜善视之。”生惊悼而觉。 从此遂绝。 呜呼异哉!

\chapter{联芳楼记}

吴郡富室有姓薛者,至正初,居于阊阖门外,以粜米为业。有二女,长曰兰英,次曰蕙英,皆聪明秀丽,能为诗赋。遂于宅后建一楼以处之,名曰兰蕙联芳之楼。适承天寺僧雪窗,善以水墨写兰蕙,乃以粉涂四壁,邀其绘画于上,登之者蔼然如入春风之室矣。二女日夕于其间吟咏不辍,有诗数百首,号《联芳集》,好事者往往传诵。时会稽杨铁崖制西湖《竹枝曲》,和者百余家,镂版书肆。二女见之,笑曰:“西湖有《竹枝曲》,东吴独无《竹枝曲》乎?”乃效其体,作苏台《竹枝曲》十章曰:

姑苏台上月团团,姑苏台下水潺潺。

月落西边有时出,水流东去几时还?

馆娃宫中麋鹿游,西施去泛五湖舟。

香魂玉骨归何处?不及真娘葬虎丘。

虎丘山上塔层层,夜静分明见佛灯。

约伴烧香寺中去,自将钗钏施山僧。

门泊东吴万里船,乌啼月落水如烟。

寒山寺里钟声早,渔火江枫恼客眠。

洞庭金柑三寸黄,笠泽银鱼一尺长。

东南佳味人知少,玉食无由进尚方。

荻芽抽笋楝花开,不见河豚石首来。

早起腥风满城市,郎从海口贩鲜回。

杨柳青青杨柳黄,青黄变色过年光。

妾似柳丝易憔悴,郎如柳絮太颠狂。

翡翠双飞不待呼,鸳鸯并宿几曾孤!

生憎宝带桥头水,半入吴江半太湖。

一纲凤髻绿于云,八字牙梳白似银。

斜倚朱门翘首立,往来多少断肠人。

百尺高楼倚碧天,阑干曲曲画屏连。

侬家自有苏台曲,不去西湖唱采莲。

他作亦皆称是,其才可知矣。铁崖见其稿,手写二诗于后曰:

锦江只说薛涛笺,吴郡今传兰蕙篇。

文采风流知有自,联珠合璧照华筵。

难弟难兄并有名,英英端不让琼琼。

好将笔底春风句,谱作瑶筝弦上声。

由是名播远迩,咸以为班姬、蔡女复出,易安、淑真而下不论也。其楼下瞰官河,舟楫皆经过焉。昆山有郑生者,亦甲族,其父与薛素厚,乃令生兴贩于郡。至则泊舟楼下,依薛为主。薛以其父之故,待以通家子弟,往来无间也。生以青年,气韵温和,性质俊雅。夏月于船首澡浴,二女于窗隙窥见之,以荔枝一双投下。生虽会其意,然仰视飞甍峻宇,缥缈于霄汉,自非身具羽翼,莫能至也。既而更深漏静,月堕河倾,万籁俱寂,企立船舷,如有所俟。忽闻楼窗哑然有声,顾盼之顷,则二女以秋千绒索,垂一竹兜,坠于其前,生乃乘之而上。既见,喜极不能言,相携入寝,尽缱绻之意焉。长女口占一诗赠生曰:

玉砌雕栏花两枝,相逢恰是未开时。

妖姿未惯风和雨,吩咐东君好护持。

次女亦吟曰:

宝篆烟消烛影低,枕屏摇动镇帏犀。

风流好似鱼游水,才过东来又向西。

至晓,复乘之而下,自是无夕而不会。二女吟咏颇多,不能尽记。生耻无以答,一夕,见案有剡溪玉叶笺,遂濡笔题一诗于上曰:

误入蓬山顶上来,芙蓉芍药两边开。

此身得似偷香蝶,游戏花丛日几回。

二女得诗,喜甚,藏之箧笥。已而就枕,生复索其吟咏。长女即唱曰:

连理枝头并蒂花,明珠无价玉无瑕。

次女续曰:

合欢幸得逢萧史,乘兴难同访戴家。

长女又续曰:

罗袜生尘魂荡漾,瑶钗坠枕鬓。

次女结之曰:

他时泄漏春消息,不悔今宵一念差。

遂足成律诗一篇。又一夕,中夜之后,生忽怅然曰:“我本羁旅,托迹门下;今日之事,尊人惘知。一旦事迹彰闻,恩情间阻,则乐昌之镜,或恐从此而遂分;延平之剑,不知何时而再合也。”因哽咽泣下。二女曰:“妾之鄙陋,自知甚明。久处闺闱,粗通经史,非不知钻穴之可丑,韫椟之可佳也。然而秋月春花,每伤虚度,云情水性,失于自持。曩者偷窥宋玉之墙,自献汴和之璧。感君不弃,特赐俯从,虽六礼之未行,谅一言之已定。方欲同欢衽席,永奉衣巾,奈何遽出此言,自生疑阻?郑君郑君,妾虽女子,计之审矣!他日机事彰闻,亲庭谴责,若从妾所请,则终奉箕帚于君家;如不遂所图,则求我于黄泉之下,必不再登他门也。一日,登楼,于箧中得生所为诗,大骇。然事已如此,无可奈何,顾生亦少年标致,门户亦正相敌,乃以书抵生之父,喻其意。生父如其所请。仍命媒氏通二姓之好,问名纳彩,赘以为婿。是时生年二十有二,长女年二十,幼女年十八矣。吴下人多知之,或传之为掌记云。

\part{}

\chapter{令狐生冥梦录}

令狐譔者,刚直之士也,生而不信神灵,傲诞自得。有言及鬼神变化幽冥果报之事,必大言折之。所居邻近,右乌老者,家赀巨宫,贪求不止,敢为不义,凶恶著闻。一夕,病卒;卒之三日而再苏。人问其故,则曰:“吾殁之后,家人广为佛事,多焚楮币,冥官喜之,因是得远。”譔闻之,尤其不忿,曰:“始吾谓世间贪官污吏受财曲法,富者纳贿而得圭,贫者无赀而抵罪,岂意冥府乃更甚焉!”因赋诗曰:

一陌金钱便返魂,公私随处可通门!

鬼神有德开生路,日月无光照覆盆。

贫者何缘蒙佛力?富家容易受天恩。

早知善恶都无报,多积黄金遗子孙!

诗成,朗吟数过。是夜,四烛独坐,忽有二鬼使,状貌狞恶,径至其前,曰:“地府奉追。”譔大惊,方欲辞避,一人执其衣,一人挽其带,驱迫出门,足不履地,须臾已至。见大官府若世间台、省之状。二使将譔入门,遥望殿上有王者被冕据案而坐。二使挟譔伏于阶下,上殿致命曰:“奉命追令狐譔已至。”即闻王者厉声曰:“既读儒书,不知自检,敢为狂辞,诬我官府!合付犁舌狱。”遂有鬼卒数人,牵捽令去。譔大惧,攀挽槛楣不得去,俄而槛折,乃大呼曰:“令狐譔人间儒士,无罪受刑,皇天有知,乞赐昭鉴!”见殿上有一绿袍秉笏者,号称明法,禀于王曰:“此人好訐,遽尔加罪,必不肯伏,不若令其供责所犯,明正其罪,当无词也。”王曰:“善!”乃有一吏,操纸笔置于譔前,逼其供状。譔固称无罪,不知所供。忽闻殿上曰:“汝言无罪,所谓‘一陌金钱便返魂,公私随处可通门’,谁所作也?”譔始大悟,即下笔大书以供曰:

伏以混沦二气,初分天地之形;高下三才,不列鬼神之数。降自中古,始肇多端。焚币帛以通神,诵经文以諂佛。于是名山大泽,咸有灵焉;古庙丛祠,亦多主者。盖以群生昏瞆,众类冥顽,或长恶以不悛,或行凶而自恣。

以强凌弱,恃富欺贫。上不孝于君亲,下不睦于宗党。贪财悖义,见利忘恩。天门高而九重莫知,地府深而十殿是列,立锉烧舂磨之状,具轮回报应之科,使为善者劝而益勤,为恶者惩而知戒,可谓法之至密,道之至公。然而威令所行,既前瞻而后仰;聪明所及,反小察而大遗。贫者入狱而受殃。宫者转经而兔罪。惟取伤弓之鸟,每漏吞舟之鱼。赏罚之条,不宜如是。至如譔者,三生贱士,一介穷儒。左枝右梧,未免儿啼女哭,东涂西抹,不救命蹇时乖。偶以不平而鸣,遽获多言之咎。悔噬脐而莫及,耻摇尾而乞怜。今蒙责其罪名,逼其状伏。批龙鳞,探尤颔,岂敢求生;料虎头,编虎须,固知受祸。言止此矣,伏乞鉴之!

王览毕,批曰:“令狐譔持论颇正,难以罪加,秉志不回,非可威屈。今观所陈,实为有理,可特放还,以彰遗直。”仍命复追乌老,置之于狱。复遣二使送譔还家。譔恳二使曰:“仆在人间,以儒为业,虽闻地狱之事,不以为然,今既到此,可一观否?”二使曰:“欲观亦不难,但禀知刑曹录事耳。”即引譔循西廊而行,别至一厅,文簿山积,录事中坐,二使以譔入白,录事以朱笔批一帖付之,其文若篆籒不可识。譔出府门,投北行里余,见铁城巍巍,黑雾涨天,守卫者甚众,皆牛头曳面,青体绀发,各执戈戟之属,或坐或立于门左右。二使以批帖示之,即放之入,见罪人无数,被剥皮刺血,剔心剜目。叫呼怨痛,宛转其间,楚毒之声动地。至一处,见铜柱二,缚男女二人于上,有夜叉以刃剖其胸,肠胃流出,以沸汤沃之,名为洗涤。譔问其故。曰:“此人在世为医。因疗此妇之夫,遂与妇通。已而其夫病卒,虽非二人杀之,原情定罪,与杀同也,故受此报。”又至一处,见僧尼裸体,诸鬼以牛马之皮覆之,皆成畜类。有趦趄未肯就者,即以铁鞭击之,流血狼藉。譔又问其故。曰:“此徒在世,不耕而食,不织而衣,而乃不守戒律,贪淫茹荤,故令化为异类,出力以报人耳。”最后至一处,榜曰:“误国之门。”见数十人坐铁床上,身具桎梏,以青石为枷压之。二使指一人示譔曰:“此即宋朝秦桧也。谋害忠良,迷误其主,故受重罪。其余亦皆历代误国之臣也。每一朝革命,即驱之出,令毒虺噬其肉,饥鹰啄其髓,骨肉糜烂至尽,复以神水洒之,业风吹之,仍复本形。此辈虽历亿万劫,不可出世矣。”譔观毕,求回。二使送之至家。譔顾谓曰:“劳君相送,无以为报。”二使笑曰:“报则不敢望,但请君勿更为诗以累我耳。”譔亦大笑。欠伸而觉,乃一梦也。及旦,叩乌老之家而问焉,则于是夜三更逝矣。

\chapter{天台访隐录}

台人徐逸,粗通书史,以端午日入无台山采药。同行数人,惮于涉险,中道而返。惟逸爱其山明水秀,树木阴翳,进不知止,且诵孙兴公之赋而赞其妙曰:“赤城震起而建标,瀑布泉流而界道。”诚非虚语也。”更前数里,则斜阳在岭,飞鸟投林,进无所抵,退不及还矣。踌躇之间,忽涧水中有巨瓢流出,喜曰:“此岂有居人乎?否则必琳宫梵宇也。”遂沿涧而行,不里余,至一弄口,以巨石为门,入数十步。则豁然宽敞,有居民四五十家,衣冠古朴,气质淳厚,石田茅屋,竹户荆扉,犬吠鸡鸣,桑麻掩映,俨然一村落也。见逸垒,惊问曰:“客何为者?焉得而涉吾境?”逸告以入山采药,失路至此,遂相顾不语,漠然无延接之意。惟一老人,衣冠若儒者,扶藜而前,自称太学陶上舍,揖逸而言曰:“山泽深险,豺狼之所嗥,魑魅之所游,日又晚矣,若固相拒,是见溺而不援也。”乃邀逸归其室。坐定,逸起问曰:“仆生于斯,长于斯,游于斯久矣,未闻有此村落也。敢问。”上舍颦蹙而答曰:“避世之士,逃难之人,若述往事,徒增伤感耳!”逸固请其故。始曰:“吾自宋朝已卜居于此矣。”逸大惊。上舍乃具述曰:“仆生于理宗嘉熙丁酉之岁,既长,寓名太学,居率履斋,以讲《周易》为众所推。度宗朝,两冠堂试,一登省荐,方欲立身扬名,以显于世,不幸度皇晏驾,太后临朗,北兵渡江,时事大变。嗣君改元德祐之岁,则挈家逃难于此。其余诸人,亦皆同时避难者也。年深岁久,因遂安焉。种田得粟,采山得薪,凿井而饮,架屋而息。寒往暑来,日居月诸,但见花开为春,叶脱为秋,不知今日是何朝代,是何甲子也。”逸曰:“今天子圣神文武,继元启运,混一华夏,国号大明,太岁在閼逢摄提格,改元洪武之七载也”上舍曰:“噫,吾止知有宋,不知有元,安知今日为大明之世也!愿客为我略陈三代兴亡之故,使得闻之。”逸乃曰:“宋德祐丙子岁,元兵入临安,三宫迁北。是岁,广王即位于海上,改元景炎。未几而崩,谥端宗。益王继立,为元兵所迫,赴水而死,宋祚遂亡,实元朝戊寅之岁也。元既并宋,奄有南北,逋至正丁未,历甲予一周有半而灭。今则大明肇统,洪武万年之七年也。盖自德祐丙子至今,上下已及百岁矣。”上舍闻之,不觉流涕。已而山空夜静,方籁寂然,逸宿于其室,土床石枕,亦甚整洁,但神清骨冷,不能成寐耳。明日,杀鸡为黍,以瓦盎盛松醪饮逸。上舍自制《金缕词》一阙,歌以侑觞曰:

梦觉黄粱熟。怪人间、曲吹别调,棋翻新局。一片残山并剩水,几度英雄争鹿!算到了谁荣谁辱?白发书生差耐久,向林间啸傲山闾宿。耕绿野,饭黄犊。市朝迁变成陵谷。问东凤、旧家燕子,飞归谁屋?前度刘郎今尚在,不带看花之福,但燕麦兔葵盈目。羊胛光阴容尾过。叹浮生待足何时足?樽有酒。且相属。

歌罢,复与逸话前宋旧事,叠叠不厌,乃言。“宝祐丙辰,亲策进士,文天样卷在四,而理皇易为举首。贾似道当国,造第于葛岭,当时有‘朝中无宰相,湖上有平章’。之句。一宗室任岭南县令,献孔雀二,置之圃中,见其驯扰可爱,即除其人为本郡守。襄阳之围,吕文焕募人以蜡书告急于朝,其人恳于似道曰:‘襄阳之围六年矣,易子而食,析骸而爨,亡在朝夕。而师相方且鋪张太平,迷惑主听,一旦虏马饮江,家国倾覆,师相亦安得久有此富贵耶?’遂扼吭而死。谢堂乃太后之侄,殷富无比,尝夜宴客,设水晶帘,烧沉香火,以径尺玛璃盘,盛大珠四颗,光照一室,不用灯烛;优人献诵乐语,有黄金七宝酒瓮,重十数斤,即于座上赐之不吝。谢后临朝,梦天倾东南。一人擎之,力若不胜,蹶而复起者三。己而一日坠地,傍有一人捧之而奔,觉而遍访于朝,得二人焉,厥状极肖,擎天者文天祥,捧日者陆秀夫也,遂不次用之。江万里去国,都民送之郭外者以千计,攀辕忍舍去,城门既阖,多宿于野。贾似道出督,御白银铠,真珠马鞍;千里马二,一驮督府之印,一载制书并随军赏格,以黄帕覆之,都民罢市而观。出师之盛,末之有也。”又论当时诸臣曰:“陈宜中谋而不断,家铉翁节而不通,张世杰勇而不果,李庭芝智而不达,其最优者,文天祥乎!”如是者凡数百言,皆历历可听。

是夕,逸又宿焉。明旦,告归,上舍复为古风一篇以饯行,曰:

建炎南渡多翻覆,泥马逃来御黄屋。

尽将旧物付他人。江南自作龟兹国。

可怜行酒两青衣,万恨千愁谁得知!

五国城中寒月照,黄龙塞上朔风吹。

东窗计就通和好,鄂王赐死蕲王老。

酒中不见刘四厢,湖上须寻宋五嫂。

累世内禅罢言兵,八十余年称太平。

度皇晏驾弓剑远,贾相出师茄鼓惊。

携家避世逃空谷,西望端门捧头哭。

毁车杀马断来踪,凿井耕田聊自足。

南邻北舍自成婚,遗风仿佛朱陈村。

不向城中供赋役,只从屋底长儿孙。

喜君涉险来相访,问旧频扶九节杖。

时移事变太匆忙,物是人非愈怊怅。

感君为我暂相留,野蔌山肴借献酬。

舍下鸡肥何用买,床头酒熟不须蒭。

君到人间烦致语,今遇升平乐安处。

相逢不用苦相疑,我辈非仙亦非鬼。

遂送逸出路口,挥袂而别。逸沿途每五十步插一竹枝以记之。到家数日,乃具酒醴,携肴馔,率家僮辈赍往访之,则重冈叠蟑,不复可寻,丰草乔林,绝无踪迹。往来于樵蹊牧径之间,但闻谷鸟悲鸣,岭猿哀啸而己,竟惆怅而归。逸念上舍自言生于嘉熙丁酉,至今则百有四十岁矣,而颜貌不衰,言动详雅,止若五六十者,岂有道之流欤?

\chapter{滕穆醉游聚景园记}

延祐初,永嘉滕生名穆,年二十六,美风调,善吟咏,为众所推許。素闻临安山水之胜,思一游焉。甲寅岁,科举之詔兴,遂以乡书赴荐。至则侨居涌金门外,无日不往于南北二山及湖上诸刹,灵隐、天竺、净慈、宝石之类,以至玉泉、虎跑、天龙、灵鹫。石屋之洞,冷泉之亭,幽涧深林,悬崖绝壁,足迹殆将遍焉。七月之望,于曲院赏莲,因而宿湖,泊雷峰塔下。

是夜,月色如昼,荷香满身,时闻大鱼跳掷于波间,宿鸟飞鸣于岸际。生已大醉,寝不能寐,披襟而起,绕堤观望。行至聚景园,信步而入。时宋亡已四十年,园中台馆,如会芳殿、清辉阁、翠光亭皆已颓毁。惟瑶津西轩岿然然独存。生至轩下,倚栏少憩。俄见一美人先行,一侍女随之,自外而入。风鬟雾鬓,绰约多姿,望之殊若神仙。生于轩下屏息以观其所为。美人言曰:“湖山如故,风景不殊,但时移世换,令人有《黍离》之悲尔!”行至园北太湖石畔,遂咏诗曰:

湖上园林好,重来忆旧游。

征歌调《玉树》,阅舞按《梁州》。

径狭花迎辇,池深柳拂舟。

昔人皆已殁,谁与话风流!

生放逸者,初见其貌,已不能定情。及闻此作,技痒不可复禁,即于轩下续吟曰:

湖上园亭好,相逢绝代人。

嫦娥辞月殿,织女下天津。

未会心中意,浑疑梦里身。

愿吹邹子律,幽谷发阳春。

吟已。趋出赴之。美人亦不惊讶,但徐言曰:“固知郎君在此,特来寻访耳。”生问其姓名,美人曰:“妾弃人间已久,欲自陈叙,诚恐惊动郎君。”生闻此言,审其为鬼,亦无所惧。固问之,乃曰:“芳华姓卫,故宋理宗朝宫人也。年二十三而殁,殡于此园之侧。今晚因往演福访贾贵妃,蒙延久坐,不觉归迟,致令郎君于此久待。”即命侍女曰:“翘翘,可于舍中取裀席酒果来,今夜月色如此,郎君又至,不可虚度,可便于此赏月也。”翘翘应命而去。须臾,携紫氍毹,设白玉碾花樽,碧琉璃盏,醪醴馨香,非世所有,与生笑谑笑咏,词旨清婉。复命翘翘歌以侑酒。翘翘请歌柳耆卿《望海潮》词,美人曰:“对新人不宜歌旧曲。”即于席上自制《木兰花慢》一阕,令翘翘歌之曰:

记前朝旧事,曾此地,会神仙。向月砌云阶,重携翠袖,来拾花钿。繁华总随流水,叹一场春梦杳难圆。废港芙渠滴露,断堤杨柳垂烟。两峰南北只依然,辇路草芊芊。恨别馆离宫,烟销凤盖,波浸龙船。平时银屏金屋,对漆灯无焰夜如年。落日牛羊垅上,西风燕雀林边。

歌竟,美人潜然垂泪,生以言尉解,仍微词挑之,以观其意。即起谢曰:“殂谢之人,久为尘土,若得奉侍巾栉,虽死不朽。且郎君适间诗句,固已许之矣。愿吹邹子之律,而一发幽谷之春也。”生曰:“向者之诗,率口而成,实本无意,岂料便为语谶。”良久,月隐西垣,河倾东岭,即命翘翘撤席。美人曰:“敝居僻陋,非郎君之所处,只此西轩可也。”遂携手而入,假寝轩下。交会之事,一如人间。将旦,挥涕而别。

至昼,往访于园侧,果有宋宫人卫芳华之墓。墓左一小丘,即翘翘瘗也。生感叹逾时。迨暮,又赴西轩,则美人已先至矣。迎谓生曰:“日间感君相访,然而妾止卜其夜,未卜其昼,故不敢奉见。数日之后,当得无间矣。”自是,无夕而不会。经旬之后,白昼亦见。生遂携归所寓安焉。已而生下第东归,美人愿随之去。生问:“翘翘何以不从?”曰:“妾既奉侍君子,旧宅无人,留其看守耳。”生与之同回乡里,见亲识,绐之曰:“娶于杭郡之良家。”众见其举止温柔,言词慧利,信且悦之。美人处生之室,奉长上以礼,待婢仆以恩,左右邻里,俱得其欢心。且又勤于治家,洁于守己,虽中门之外,未尝轻出。众咸贺生得内助。

荏苒三岁,当丁巳年之初秋,生又治装赴浙省乡试。行有日矣,美人请于生曰:“临安,妾乡也。从君至此,已阅三秋,今愿得偕行,以顾视翘翘。”生许诺,遂赁舟同载,直抵钱塘,僦屋居焉。至之明日,适值七月之望,美人谓生曰:“三年前曾于此夕与君相会,今适当其期,欲与君同赴聚景,再续旧游可乎?”生如其言,载酒而往。

至晚,月上东垣,莲开南浦,露柳烟篁,动摇堤岸,宛然若昔时之景。行至园前,则翘翘迎拜于路首曰:“娘子陪侍郎君,遨游城郭,首尾三年,已极人间之欢,独不记念旧居乎?”三人入园,同至西轩而坐。美人忽涕泪而告生曰:“感君不弃,侍奉房帷,未遂深欢,又当永别。”生曰:“何故?”对曰:“妾本幽阴之质,久戚阳明之世,甚非所宜。特以与君有夙世之缘,故冒犯条律以相从耳。今而缘尽,自当奉辞。”生惊问曰:“然则何时?”对曰:“止在今夕耳。”生凄惶不忍。美人曰:“妾非不欲终事君子,永奉欢娱。然而程命有限,不可违越。若更迟留,须当获戾。非止有损于妾,亦将不利于君。岂不见越娘之事乎?”生意稍悟,然亦悲伤感怆,彻晓不寐。及山寺钟鸣,水村鸡唱,急起与生为别,解所御玉指环系于生之衣带,曰:“异日见此,无忘旧情。”遂分袂而去,然犹频频而顾,良久始灭。生大恸而返。

翌日具肴醴,焚镪楮于墓下,作文以吊祭之曰:

惟灵生而淑美,出类超群。禀奇姿于仙圣,钟秀气于乾坤。粲然如花之丽,粹然如玉之温。达则天上之金屋,穷则路左之荒坟。托松楸而共处,对狐兔之群奔。落花流水,断雨残云,中原多事,故国无君。抚光阴之过隙,视日月之奔轮。然而精灵不泯,性识长存。不必仗少翁之奇术,自能返倩女之芳魂。玉匣骖鸾之扇,金泥簇蝶之裙。声泠泠兮环珮,香蔼蔼兮兰荪。方欲同欢而偕老,奈何既合而复分!步洛妃凌波之袜,赴王母瑶池之樽。即之而无所睹,扣之而不复闻。怅后会之莫续,伤前事之谁论。锁杨柳春风之院,闭梨花夜雨之门。恩情断兮天漠漠,哀怨结兮云昏昏。音容杳而靡接,心绪乱而纷纭。谨含哀而奉吊,庶有感于斯文。呜呼哀哉,尚飨!

从此遂绝矣。生独居旅邸,如丧配耦。试期既迫,亦无心入院,惆怅而归。亲党问其故,始具述之,众咸叹异。生后终身不娶,入雁荡山采药,遂不复还。

\chapter{牡丹灯记}

方氏之据浙东也,每岁元夕,于明州张灯五夜,倾城士女,皆得纵观。至正庚子之岁,有乔生者,居镇明岭下,初丧其耦,鳏居无聊,不复出游,但倚门伫立而已。十五夜,三更尽,游人渐稀,见一丫鬟,挑双头牡丹灯前导,一美人随后,约年十七八,红裙翠袖,婷婷嫋嫋,迤逦投西而去。生于月下视之,韶颜稚齿,真国色也。神魂飘荡,不能自抑,乃尾之而去,或先之,或后之。行数十步,女忽回顾而微哂曰:“初无桑中之期,乃有月下之遇,似非偶然也。”生即趋前揖之曰:“敝居咫尺,佳人可能回顾否?”女无难意,即呼丫鬟曰:“金蓬,可挑灯同往也。”于是金莲复回。生与女携手至家,极其欢昵,自以为巫山洛浦之遇,不是过也。生问其姓名居址,女曰:“姓符,丽卿其字,漱芳其名,故奉化州判女也。先人既殁,家事零替,既无弟兄,仍鲜族党,止妾一身,遂与金莲侨居湖西耳。”生留之宿,态度妖妍,词气婉媚,低帏昵枕,甚极欢爱。天明,辞别而去,暮则又至。如是者将半月,邻翁疑焉,穴壁窥之,则见一粉骷髏与生并坐于灯下,大骇。明旦,詰之,秘不肯言。邻翁曰:“嘻!子祸矣!人乃至盛之纯阳,鬼乃幽阴之邪秽。今子与幽阴之魅同处而不知,邪秽之物共宿而不悟,一旦真元耗尽,灾眚来临,惜乎以青春之年,而遂为黄壤之容也,可不悲夫!”生始惊惧,备述厥由。邻翁曰:“彼言侨居湖西,当往物色之,则可知矣。”生如其教,径投月湖之西,往来于长堤之上、高桥之下,访于居人,询于过客,并言无有。日将夕矣,乃入湖心寺少憩,行遍东廊,复转西廊,廊尽处得一暗室,则有旅榇,白纸题其上曰:“故奉化符州判女丽卿之柩。”柩前悬一双头牡丹灯,灯下立一明器婢子,背上有二字曰金莲。生见之,毛发尽竖,寒栗遍体,奔走出寺,不敢回顾。是夜借宿邻翁之家,忧怖之色可掬。邻翁曰:“玄妙观魏法师,故开府王真人弟子,符箓为当今第一,汝宜急往求焉。”明旦,生诣观内。法师望见其至,惊曰:“妖气甚浓,何为来此?”生拜于座下,具述其事。法师以朱符二道授之,令其—置于门,一置于榻,仍戒不得再往湖心寺。生受符而归,如法安顿,自此果不来矣。一月有余,往衮绣桥访友。留饮至醉,都忘法师之戒,径取湖心寺路以回。将及寺门,则见金莲迎拜于前曰:“娘子久待,何一向薄情如是!”遂与生俱入西廊,直抵室中。女宛然在坐,数之曰:“妾与君素非相识,偶于灯下一见,感君之意,遂以全体事君,暮往朝来,于君不薄。奈何信妖道士之言,遽生疑惑,便欲永绝?薄幸如是,妾恨君深矣!今幸得见,岂能相舍?”即握生手,至柩前,柩忽自开,拥之同入,随即闭矣,生遂死于柩中。邻翁怪其不归,远近寻问,及至寺中停柩之室,见生之衣裾微露于柩外,请于寺僧而发之,死已久矣,与女之尸俯仰卧于内,女貌如生焉。寺僧叹曰:“此奉化州判符君之女也,死时年十七,权厝于此,举家赴北,竟绝音耗,至今十二年矣。不意作怪如是!”遂以尸柩及生殡于西门之外。自后云阴之昼,月黑之宵,往往见生与女携手同行,一丫鬟挑双头壮丹灯前导,遇之者辄得重疾,寒热交作;荐以功德,祭以牢醴,庶获痊可,杏则不起矣。居人大惧,竞往玄妙观谒魏法师而诉焉。法师曰:“吾之符箓,止能治其未然,今祟成矣,非吾之所知也。闻有铁冠道人者,居四明山顶,考劾鬼神,法术灵验,汝辈宜往求之。”众遂至山,攀缘藤草,蓦越溪涧,直上绝顶,果有草庵一所,道人凭几而坐,方看童子调鹤。众罗拜庵下,告以来故。道人曰:“山林隐士,旦暮且死,乌有奇术!君辈过听矣。”拒之甚严。众曰:“某本不知,盖玄妙魏师所指教耳。”始释然曰:“老夫不下山已六十年,小子饶舌,烦吾一行。”即与童子下山,步履轻捷,径至西门外,结方丈之坛,踞席端坐,书符焚之。忽见符吏数辈,黄巾锦祆,金甲雕戈,皆长丈余,屹立坛下,鞠躬请命,貌甚虔肃。道人曰:“此间有邪祟为祸,惊扰生民,妆辈岂不知耶?宜疾驱之至。”受命而往,不移时,以枷锁押女与生并金莲俱到,鞭菙挥扑,流血淋漓。道人呵责良久,令其供状。将吏以纸笔授之,遂各供数百言。今录其略于此。

乔生供曰:

伏念某丧室鳏居,倚门独立,犯在色之戒,动多欲之求。不能效孙生见两头蛇而决断,乃致如郑子运九尾狐而爱怜。事既莫追,侮将奚及!

符女供曰:

伏念某青年弃世,白昼无邻,六魄虽离,一灵未混。灯前月下,逢五百年欢喜冤家;世上民间,作千万人风流话本。迷不知返,罪安可逃!

金莲供曰:

伏念某杀青为骨,染素成胎,坟垅埋藏,是谁作俑而用?面目机发,比人具体而微。既有名字之称,可乏精灵之异!因而得计,岂敢为妖!

供毕,将吏取呈。道人以巨笔判曰:

盖闻大禹铸鼎,而神奸鬼秘莫得逃其形;温峤燃犀,而水府龙宫俱得现其状。惟幽明之异趣,乃诡怪之多端。遇之者不利于人,遭之者有害于物。故大厉入门而晋景殁,妖豕啼野而齐裹殂。降祸为妖,兴灾作孽。是以九天设斩邪之使,十地列罚恶之司,使魑魅魍魉,无以容其奸,夜叉罗刹,不得肆其暴。矧此清平之世,坦荡之时,而乃变幻形躯,依附草木,天阴雨湿之夜,月落参横之晨,啸于梁而有声,窥其室而无睹,蝇营狗苟,牛狠狼贪,疾如飘风,烈若猛火。乔家子生犹不悟,死何恤焉。符氏女死尚贪淫,生可知矣!况金莲之怪诞,假明器而矫诬。惑世诬民,违条犯法。狐绥绥而有荡,鹑奔奔而无良。恶贯已盈,罪名不宥。陷人坑从今填满,迷魂阵自此打开。烧毁双明之灯,押赴九幽之狱。

判词已具,主者奉行急急如律令。即见三人悲啼踯躅,为将吏驱捽而去。道人拂袖入山。明日,众往谢之,不复可见,止有草庵存焉。急往玄妙观访魏法师而审之,则病瘖不能言矣。

\chapter{渭塘奇遇记}

至顺中,有王生者,本士族子,居于金陵。貌莹寒玉,神凝秋水,姿状甚美,众以奇俊王家郎称之。年二十,未娶。有田在松江,因往收秋租,回舟过渭塘,见一酒肆,青旗出于檐外;朱栏曲槛,缥缈如画;高柳古槐,黄叶交坠;芙蓉十数株,颜色或深或浅,红葩绿水,上下相映;白鹅一群,游泳其间。生泊舟岸侧,登肆沽酒而饮,斫巨螯之蟹,烩细鳞之鲈,果则绿橘黄橙,莲塘之藕,松坡之栗,以花磁盏酌真珠红酒而饮之。肆主亦富家,其女年十八,知音识字,态度不凡,见生在座,频于幕下窥之,或出半面,或露全体,去而复来,终莫能舍。生亦留神注意,彼此目成久之。已而酒尽出肆,怏怏登舟,如有所失。是夜遂梦至肆中,入门数重,直抵舍后,始至女室,乃一小轩也。轩之前有葡萄架,架下凿池,方圆盈丈,甃以文石,养金鲫其中;池左右植垂丝桧二株,绿荫婆娑,靠墙结一翠柏屏,屏下设石假山三峰,岌然竞秀;草则金钱绣墩之属,霜露不变色。窗间挂一雕花笼,笼内畜一绿鹦鹉,见人能言。轩下垂小木鹤二只,衔线香焚之。案上立一古铜瓶,插孔雀尾数茎,其傍设笔砚之类,皆极济楚。架上横一碧玉箫,女所吹也。壁下贴金花笺四幅,题诗于上,诗体则效东坡四时词,字画则师赵松雪,不知何人所作也。

第一幅云:

春风吹花落红雪,杨柳荫浓啼百舌。

东家蝴蝶西家飞,前岁樱桃今岁结。

秋千蹴罢鬓鬖髿,粉汗凝香沁绿纱。

侍女亦知心内事,银瓶汲水煮新茶。

第二幅云:

芭蕉叶展青鸾尾,萱草花含金凤嘴。

一双乳燕出雕梁,数点新荷浮绿水。

困人天气日长时,针线慵拈午漏迟。

起向石榴阴畔立,戏将梅子打莺儿。

第三幅云:

铁马声喧风力紧,云窗梦破鸳鸯冷。

玉炉烧麝有余香,罗扇扑萤无定影。

洞箫一曲是谁家?河汉西流月半斜。

要染纤纤红指甲,金盆夜捣凤仙花。

第四幅云:

山茶未开梅半吐,风动帘旌雪花舞。

金盘冒冷塑狻猊,绣幕围春护鹦鹉。

倩人呵笔画双眉,脂水凝寒上脸迟。

妆罢扶头重照镜,凤钗斜压瑞香枝。

女见生至,与之承迎,执手入室,极其欢谑,会宿于寝。鸡鸣始觉,乃困卧篷窗底耳。

自后归家,无夕而不梦焉。一夕,见架上玉箫,索女吹之。女为吹《落梅风》数阕,音调嘹亮,响彻云际。一夕,女于灯下绣红罗鞋,生剔灯花,误落于上,遂成油晕。一夕,女以紫金碧甸指环赠生,生解水晶双鱼扇坠酬之,既觉,则指环宛然在手,扇坠视之无有矣。生大为奇,遂效元稹体,赋会真诗三十韵以记其事。诗曰:

有美闺房秀,天人谪降来。风流元有种,慧黠更多才。

碾玉成仙骨,调脂作艳胎。腰肢风外柳,标格雪中梅。

合置千金屋,宜登七宝台。妖姿应自许,妙质孰能陪?

小小乘油壁,真真醉彩灰。轻尘生洛浦,远道接天台。

放燕帘高卷,迎人户半开。菖蒲难见面,豆蔻易含胎。

不待金屏射,何劳玉手栽。偷香浑似贾,待月又如崔。

筝许秦宫夺,琴从卓氏猜。箫声传缥缈,烛影照徘徊。

窗薄涵鱼魫,炉深喷麝煤。眉横青岫远,鬓軃绿云堆。

钗玉轻轻制,衫罗窄窄裁。文鸳游浩荡,瑞凤舞毰毶。

恨积鲛绡帕,欢传琥珀杯。孤眠怜月姊,多忌笑河魁。

化蝶能通梦,游蜂浪作媒。雕栏行共倚,绣褥坐相偎。

啖蔗逢佳境,留环得异财。绿荫莺并宿,紫气剑双埋。

良夜难虚度,芳心未肯摧。残妆犹在臂,别泪已凝腮。

漏滴何须促,钟声且莫催。峡中行雨过,陌上看花回。

才子能知尔,愚夫可语哉!鲰生曾种福,亲得到逢莱。

诗讫,好事者多传诵之。明岁,复往收租,再过其处,则肆翁甚喜,延之入内。生不解意,逡巡辞避。坐定,翁以诚告之曰:“老拙惟一女,未曾适人,去岁,君子所至,于此饮酒,偶有所睹,不能定情,因遂染疾,长眠独语,如醉如痴,饵药无效,昨夕忽语曰:‘明日郎君至矣,宜往侯之。’初以为妄,固未之信,今而君子果涉吾地,是天假其灵而赐之便也。”因问生婚娶未曾,又问其门阀氏族,甚喜。肆翁即握生手,入于内室,至女所居轩下,门窗户闼,则皆梦中所历也;草木台沼、器用什物,又皆梦中所识也。女闻生至,盛妆而出,衣服之丽,簪饵之华,又皆梦中所识也。女言:“去岁自君去后,思念切至,每夜梦中与君相会,不知何故。”生曰:“吾梦亦如之耳。”女历叙吹箫之曲,绣鞋之事,无不吻合者。又出水晶双鱼扇坠示生,生亦举紫金碧甸指环以问之。彼此大惊,以为神契。遂与生为夫妇,于飞而还,终以偕老,可谓奇遇矣!

\part{}

\chapter{富贵发迹司志}

至正丙戌,泰州士人何友仁,为贫寠所迫,不能聊生。因谒城隍祠,过东庑,见一案,榜曰:“富贵发迹司。”友仁祷于神像之前曰:“某生世四十有五,寒一裘,暑一葛,朝、晡粥饭一盂,初无过用妄为之事。然而遑遑汲汲,常有不足之忧,冬暖而愁寒,年丰而苦饥,出无知已之投,处无蓄积之守。妻孥贱弃,乡党绝交,困阨艰难,无所告诉。侧闻大神主富贵之案,掌发迹之权,叩之即有闻,求之无不获。是以不避呵责,冒渎威严,屏息庭前,鞠躬户下。伏望告以倘来之事,喻以未至之机,指示迷途,提携晦迹,俾枯鱼蒙斗水之活,困鸟托一枝之安,敢不拜赐,深仰于洪造!如或前事有定,后事无由,大数既已难移,薄命终于不遇,亦望明彰报应,使得预知。”祷毕,跧伏案幕之下。是夜,东西两廊,左右诸曹,皆灯烛荧煌,人物骈杂,惟友仁所祷之司,不见一人,亦无灯火。独处暗中,将及毕夜,忽闻呵殿之音,初远渐近,将及庙门,诸司判官,皆趋出迎之。及入,纱笼两行,仪卫甚严。府君朝服端简,登正殿而坐,判官辈参见既毕,皆回局治事。发迹司主者亦自殿上而来,盖适从府君朝天使回尔。坐定,有判官数人,皆幞头角带,服绯绿之衣,入户相见,各述所理之事。一人曰:“某县某户藏米二千石,近因旱蝗相继,米价倍增,邻境闭籴,野有饿莩,而乃开仓以赈之,但取原价,不求厚利,又为饘粥以济贫乏,蒙活者甚众。昨县神申上于本司,呈于府君,闻已奏知天庭,延寿三纪,赐禄万鍾矣。”一人曰:“某村某氏奉姑甚孝,其夫在外,而姑得重痼,医巫无效,乃斋沐焚香祝天,愿以身代,割股以进,固遂得愈。昨天符行下云:某氏孝通天地,诚格鬼神,令生贵子二人,皆食君禄,光显其门,终为命妇以报之。府君下于本司,今已著之福籍矣。”一人曰:“某姓某官,爵位已崇。俸禄亦厚,不思报国,惟务贪饕,受钞三百锭,枉法断公事,取银五百两,非理害良民。府君奏于天庭,即欲加其罪,缘本人颇有顽福,故稽延数年,使罹灭族之祸。今早奉命,记注恶簿,惟俟时至尔。”一人曰:“某乡某甲,有田数十顷,而贪纵无厌,务为兼并。邻田之接坟者,欺其势孤无援,贱价售之,又不还其值,令其含忿而死。冥府帖本司勾摄入狱,闻已儿身为牛,托生邻家,偿其所负矣。”诸人言叙既毕,发迹司判官忽扬眉盱目,咄嗟长叹而谓众宾日:“诸公各守其职,各治其事,褒善罚罪,可谓至矣。然而无地运行之数,生灵厄会之期,国统浙衰,大难将作,虽诸公之善理,其如之奈何!”众问曰:“何谓也?”对曰:“吾适从府君上朝帝阍,所闻众圣推论将来之事,数年之后,兵戎大起,巨河之南,长江之北,合屠戮人民三十余万,当是时也,自非积善累仁,忠孝纯至者,不克免焉。岂生灵寡福,当此涂炭乎?抑运数已定,莫之可逃乎?”众皆颦蹙相顾曰:“非所知也。”遂各散去。友仁始于案下匍匐而出,拜述厥由。判官熟视良久,命小吏取簿籍至,亲自检阅,谓友仁曰:“君后大有福禄,非久于贪困者,自兹以往,当日胜一日,脱晦向明矣。”友仁愿示真详,乃取朱笔,大书一十六字以授友仁曰:“遇日而康,遇月而发,遇云而衰,遇电而没。”友仁听讫,以所授置之于怀,因再拜辞出。行及庙门外,天色已曙。急探怀中,则无有矣。归而话于妻子以自慰。不数日,郡有大姓傅日英者,延之以训子弟,月奉束修五锭,家遂稍康。凡居其馆数岁。已而高邮张氏兵起,元朝命丞相脱脱统兵讨之,太师达理月沙颇知书好士,友仁献策于马首,称其意,荐于脱公,署随军参谋,车马仆从,一旦赫然。及脱公征还,友仁遂仕于朝,践历馆阁,翱翔省部,可渭贵矣。未几,授文林郎、内台御史,同列有云石不花者,与之不相能,构于大官,黜为雷州录事。友仁忆判官之言,日月云三字,皆已验矣,深自戒惧,不敢为非。到任二年,有事申总府,吏具牍以进,友仁自署其衔曰:雷州路录事何某。挥笔之际,风吹纸起,于雷字之下,曳出一尾,宛然成一电字,大恶之,亟命易去。是夜感疾,自知不起,处置家事,诀别妻子而终。因详判官所述众圣之语,将来之事。盖至正辛卯之后,张氏起兵淮东,国朝创业淮西,攻斗争夺,干戈相寻,沿淮诸郡,多被其祸,死于兵者何止三十万焉。是以知普夭之下,率土之滨,小而一身之荣悴通寒,大而一国之兴衰治乱,皆有定数,不可转移,而妄庸者乃欲辄施智术于其间,徒自取困尔。 

\chapter{永州野庙记}

永州之野,有神庙,背山临流,川泽深险,黄茅绿草,一望无际,大木参天而蔽日者,不知其数,风雨往住生其上,人皆畏而事之,过者必以牲牢献于殿下,始克前往,如或不然,则风雨暴至,云雾晦冥,咫尺不辩,人物行李,皆随失之。如是者有年矣。大德间,书生毕应详,有事适衡州,道由庙下,囊橐贫匮,不能设奠,但致敬而行。未及数里,大风振作,吹沙走石,玄云黑雾,自后隐至。回顾,见甲兵甚众,追者可千乘万骑,自分必死,平日能诵《玉枢经》,事势既危,且行且诵,不绝于口。须臾,则云收风止,天地开朗。所迫兵骑。不复有矣。仅而获全,得达衡州,过祝融峰,谒南岳祠,思忆前事,具状焚诉。是夜,梦駃卒来追,与之偕行,至大宫殿,侍卫罗列,曹局分市。駃卒引立大庭下,望殿上挂玉栅帘,帘内设黄罗帐,灯烛辉煌,光若白昼,严邃整肃,寂而不哗。应祥屏息俟命。俄一吏朱农角带,自内而出,传呼曰:“得旨问与何人有讼?”伏而对曰:“身为寒儒,性又愚拙。不知名利之可求,岂有田宅之足竞!布衣蔬食,守分而巳。且又未尝一入公门,无以仰答威问。”吏曰:“日间投状,理会何事?”应祥始悟,稽首而白曰:“实以贫故,出境投人,道由永州,过神祠下,行囊罄竭,不能以牲醴祭事,触神之怒,风雨暴起,兵甲追逐,狼狈颠踣,几为所及,惊怖急迫,无处申诉,以致唐突圣灵,诚非得已。”吏入,少顷复出,曰:“得旨追对。”即见吏士数人,腾空而去。俄顷,押一白须老人,乌巾道服,跪于阶下。吏宣旨诘之曰:“汝为一方神祗,众所敬奉,奈何辄以威祸恐人,求其祀飨,迫此儒士,几陷死地,贪婪苦虐,何所逃刑!”老人拜而对曰:“某实永州野庙之神也,然而庙为妖蟒所据,已有年矣,力不能制,旷职已久。向者驱驾风雨,邀求奠酹,皆此物所为。非某之过。”吏责之曰:“事既如此,何不早陈?”对曰:“此物在世已久,兴妖作孽,无与为比。社鬼祠灵,承其约束;神蛟毒虺,受其指挥。每欲奔诉,多方抵截,终莫能达。今者非神使来追,亦焉得到此!”即闻殿上宣旨,令士吏追勘。老人拜恳曰:“妖孽已成,辅之者众,吏士虽往,终恐无益,非自神兵剿捕,不可得也。”殿上如其言,命一神将领兵五千而往。久之,见数十鬼卒,以大木舁其首而至,乃一朱冠白蛇也。置于庭下,若五石缸焉。吏顾应祥令还,欠伸而觉,汗流浃背。事讫回途,再经其处,则殿宇偶像,荡然无遗。问于村甿,皆曰:“某夜三更后,雷霆风火大作,惟闻杀伐之声,惊咳叵测。旦往视之,则神庙已为煨烬,一巨白蛇长数十丈,死于林木之下,而丧其元。其余蚺虺螣蝮之属无数,腥秽之气,至今未息。”考其日,正感梦时也。应祥还家,白昼闲坐,忽见二鬼使至前曰:“地府屈君对事。”即挽其臂以往。及至,见王者坐大厅上,以铁笼罩一白衣绎帻丈夫,形状甚伟。自陈:“在世无罪,为书生毕应祥枉告于南岳,以致神兵降代,举族歼夷,巢穴倾荡,冤苦实甚。”应祥闻言,知为蛇妖挟仇捏诉,乃具陈其害人祸物、兴妖作怪之事,对辩于铁笼之下,往返甚苦,终不肯服。王者乃命吏牒南岳衡山府及帖永州城隍司征验其事。己而,衡山府及永州城隍司回文,与毕应祥所言实事相同,方始词塞。王者殿上大怒,叱之曰:“生既为妖,死犹妄诉,将白衣妖孽押赴酆都,永不出世!”即有鬼卒数人驱押之去,受其果报。王谓应祥曰:“劳君一行,无以相报”命吏取毕姓薄籍来,于应祥名下,批八字云:“除妖去害,延寿一纪。”应祥拜谢而返。及门而寤,乃曲肱几上尔。

\chapter{申阳洞记}

陇西李生,名德逢,年二十五,善骑射,驰骋弓马,以胆勇称,然而不事生产,为乡党贱弃。天历间,父友有任桂州监郡者,因往投焉。至则其人已殁,流落不能归。郡多名山,日以猎射为事,出没其间,末尝休息,自以为得所乐。有大姓钱翁者,以赀产雄于郡,止有一女,年及十七,甚所钟爱,未尝窥门,虽姻亲邻里,亦罕见之。一夕,风雨晦冥,失女所在,门窗户闼,扃鐍如故,莫知所从往。闻于官,祷于伸,访于四境,悄无踪迹。翁念女切至,设誓曰:“有能知女所在者,愿以家财—半给之,并以女事焉。”虽求寻之意甚切,而荏苒将及半载,竟绝音响。生一日挟鏃持弧出城,遇一麞,逐之不舍,遂越冈峦,深入涧谷,终莫能及。日巳曛黑,又迷来路,彷徨于垅坂之侧,莫知所适。已而烟昏云瞑,虎啸猿啼,远近黯然,若一更之后。遥望山顶,见一古庙,委身投之。至则尘埃堆积,墙壁倾颓,兽蹄鸟迹,交杂于中。生虽甚怖,然无可奈何,少憩庑下,将以待旦。未及瞑目,忽闻传导之声,自远而至。生念深山静夜,安得有此?疑其为鬼神,又恐为盗劫,乃攀缘槛楣,伏于梁间,以窥其所为。须臾,及门,有二红灯前导,为首者顶三山冠,绛帕首,披淡黄袍,束玉带,径据神案而坐。从者十余辈,各执器仗,罗列阶下,仪卫虽甚整肃,而状貌则皆豭貜之类也。生知为邪魅,取腰间箭,持满一发,正中坐者之臂,失声而走,群党一时溃散,莫知所之。久之,寂然,乃假寐待旦。则见神座边鲜血点点,从大门而出,沿路不绝,循山而南,将及五里,得—大穴,血踪由此而入。生往来穴口,顾盼之际,草根柔滑,不觉失足而坠。乃深坑万仞,仰不见天,自分必死。旁边微觉有路,寻路而行,转入幽遽,咫尺不辨。更前百步,豁然开朗,见一石室,榜曰:申阳之洞。守门者数人,装束如昨夕庙中所睹。见生,惊曰:“子为何人,而遽至此?”生磬折作礼而答曰:“下界凡氓,久居城府,以医为业。因乏药材,入山采拾,贪多务得,进不知止。不觉失足,误坠于斯。触冒尊灵,乞垂觅宥。”守门者闻言,似有喜色,问之曰:“汝既业医,能为人治疗乎?”生曰:“此分内事也。”守门者大喜,以手加额曰:“天也!”生请其故。曰:“吾君申阳侯,昨因出游,为流矢所中,卧病在床,而汝惠然来斯,是天以神医见貺也。”乃邀生坐于门下,踉跄趋入,以告于内。顷之,出而传其主之命曰:“仆不善摄生,自贻伊戚,祸及股肱,毒流骨髓,厄运莫逃,残生待尽。今而幸值神医,获赐良剂,是受病者有再生之乐,而治病者有全生之恩也。敢不忍死以待!”生遂摄衣而入,度重门,及曲房,帷幄衾褥,极其华丽。见一老猕猴,偃卧石榻之上,呻吟之声不绝。美人侍侧者三,皆绝色也。生诊其脉,抚其疮,诡曰:“无伤也,予有仙药,非徒治病,兼可度世,服之则能后天不老,而凋三光矣。今之相遇,盖亦有缘耳。”遂倾囊出药,令其服之。群妖闻度世之说,喜得长生,皆罗拜于前曰:“尊官信是神人,今幸相遇!吾君既获仙丹永命,吾等独不得沾刀圭之赐乎?”生遂罄其所赍,遍赐之,皆踊跃争夺,惟恐不预。其药盖毒之尤者,用以淬箭鏃而射鸷兽,无不应弦而倒。有顷,群妖一时仆地,昏眩无知矣。生顾宝剑悬于石壁,取而悉斩之,凡戳猴大小三十六头。疑三女为妖,欲并除之。皆泣而言曰:“妾等皆人,非魅也。不幸为妖猴所摄,沉陷坑阱,求死不得。今君能为妾除害,即妾再生之主也,敢不惟命是听!”问其姓名居址,其—即钱翁之女,其二亦皆近邑良家也。生虽能除去群妖,然无计以出。愤闷之际,忽有老父数人,不知自何来,皆身被褐裘,长须乌喙,推一白衣者居前,向生列拜曰:“吾等虚星之精,久有此土,近为妖猴所据,力弗能敌,屏避他方,俟其便而图之。不意君能为我扫除仇怨,荡涤凶邪,敢不致谢!”各于袖中出金珠之属,置于生前。生曰:“若等既具神通,何乃见欺于彼,自伏孱劣耶?”白衣者曰:“吾寿止五百岁,彼已八百岁,是以不敌。然吾等居此,与人无害也,功成行满,当得飞游诸天,出入自在耳。非若彼之贪淫肆暴,害人祸物。今其稔恶不已,举族夷灭,盖亦获咎于天,假手于君耳。不然,彼之凶邪,岂君所能制耶?”生曰:“洞名申阳,其义安在?”曰:“猴乃申属,故假之以美名,非吾土之旧号也。”生曰:“此地既为若等故居,予乃世人,误陷于此,但得指引归途,谢物不用也。”曰:“果如是,亦何难哉!但请闭目半晌,即得遂愿。”生如其言,耳畔惟闻疾风暴雨之声。声止,开目,见一大白鼠在前,群鼠如豕者数辈从之,旁穿一穴,达于路口。生挈挚三女以出,径叩钱翁之门而归焉。翁大惊喜,即纳为婿,其二女之家,亦愿从焉。生一娶三女,富贵赫然。复至其处,求访路口,则丰草乔林,远近如一,无复旧踪焉。

\chapter{爱卿传}

罗爱爱,嘉兴名娼也,色貌才艺,独步一时。而又性识通敏,工于侍词,以是人皆敬而慕之,称为爱卿。佳篇丽什,传播人口。风流之士,咸修饰以求狎,懵学之辈,自视缺然。郡中名士,尝以季夏望日,会于鸳湖凌虚阁避暑,玩月赋诗。爱卿先成四首,座间皆搁笔。诗曰:

画阁东头纳晚凉,红莲不似白莲香。

一轮明月天如水,何处吹萧引凤凰?

月出天边水在湖,微澜倒浸玉浮图。

搴帘欲共姮娥语,肯教霓裳一曲无?

手弄双头茉莉枝,曲终不觉鬓云欹。

珮环响处飞仙过,愿偕青鸾一只骑。

曲曲栏干正正屏,六铢衣薄懒来凭。

夜深风露凉如许,身在瑶台第一层。

同郡有赵氏子者,第六,亦簪缨族,父亡母存,家赀巨万,慕其才色,纳礼聘焉。爱卿入门,妇道甚修,家法甚饬,择言而发,非礼不行。赵子嬖而重之。未久,赵子有父党为吏部尚书,以书自大都召之,许授以江南一宫。赵子欲往,则恐贻母妻之忧,不往。则又失功名之会,踌躇未决。爱卿谓之曰:“妾闻男子生而桑弧蓬矢以射四方,丈夫壮而立身扬名以显父母,岂可以恩情之笃,而误功名之期乎?君母在堂,温凊之奉,甘旨之供,妾任其责有余矣。但年高多病,而君有万里之行,昔人所渭事主之日多,报亲之日少,君宜常以此为念。望太行之孤云,抚西山之颓日,不可不早归耳。”赵子遂卜日为京都之行,置酒酌别于中堂。酒三行,爱卿请赵子捧觞为太大人寿,自制《齐天乐》一阕,歌以侑之。其词曰:

恩情不把功名误,离筵又歌金缕。白发慈亲,红颜幼妇,君去有谁为主?流年几许?况闷闷愁愁,风风雨雨。凤折鸾分,未知何日更相聚!蒙君再三分付:向堂前侍奉,休辞辛苦。官诰蟠花,宫袍制锦,待要封妻拜母。君须听取:怕日薄西山,易生愁阻。早促归程,彩衣相对舞。

歌罢,坐中皆垂泪。赵子乘醉,解缆而行。至都,则尚书以病免,无所投托,迁延旅邸,久不能归。太夫人以忆子之故,感病沉重,伏枕在床。爱卿事之甚谨,汤药必亲尝,饘粥必亲煮。求神礼佛,以逭其灾;虚辞诡说,以宽其意。缠绵半载,因遂不起。临终,呼爱卿而告之曰:“吾子以功名之故,远赴皇都,遂绝昔耗。吾又下幸罹疾,新妇事我至矣!今而命殂,无以相报。但愿吾子早归,新妇异日有子有孙,皆如新妇之孝敬。苍天有知,必不相负!”言讫而殁。爱卿哀毁如礼,亲造棺椁。葬于白苎村。既葬,旦夕哭临灵几前,悲伤过度,为之瘦痟。

至正十六年,张士诚陷平江,十七年,达丞相檄苗军师杨完者为江浙参政,拒之于嘉兴。不戢军士。大掠居民。赵子之居,为刘万户者所据,见爱卿之姿色,欲逼纳之。爱卿以甘言给之,沐浴入閤,以罗巾自缢而死。万户奔救之,已无及矣。乃以绣褥裹尸,瘗于后圃银杏树下。未几,张氏通款,浙省杨参政为所害,麾下皆星散。赵子始间关海道,由太仓登岸,径回嘉兴,则城郭人民皆非旧矣。投其故宅,荒废无人居,但见鼠窜于梁,鸮鸣于树,苍苔碧草,掩映阶庭而已。求其母妻,不知去向,惟中堂岿然独存,乃洒扫而息焉。明日,行出东门外,至红桥倒,遇旧使老苍头于道,呼而问之,备述其详:则老母辞堂,生妻去世矣。遂引赵子至白苎村其母葬处,指松柏而告之曰:“此皆六娘子之所种植也。”指茔垅而告之曰:“此皆六娘子之所经理也。太夫人以郎君不归,感念成疾,娘子奉之至矣,下幸而死,卜葬于此。娘子身被衰麻,手扶棺榇,亲自负土,号哭墓下。葬之三月,而苗军入城,宅舍被占。有刘万户者,欲以非礼犯之,娘子不从,即遂缢死,就于后圃瘗之矣。”赵子大伤感,即至银杏村下发视之,颜貌如生,肌肤不改。赵子抚尸大恸,绝而复苏。乃沐以香汤,被以华服,买棺附葬于母坟之侧,哭之曰:“娘子平日聪明才慧,流辈不及。今虽死矣,岂可混同凡人,使绝音晌。九原有知,愿赐一见。虽显晦殊途,人皆忌惮,而恩情切至,实所不疑。”于是出则祷于墓下,归则哭于圃中。将及一旬,月晦之夕,赵子独坐中堂,寝不能寐,忽闻暗中哭声,初远渐近,觉其有异,即起祝之曰:“倘是六娘子之灵,何吝一见而叙旧也?”即闻言曰:“妾即罗氏也,感君想念,虽在幽冥,实所恻怆,是以今夕与君知闻耳。”言讫,如有人行,冉冉而至,五六步许,即可辨其状貌,果爱卿也。淡妆素服,一如其旧,惟以罗巾拥其项。见赵子,施礼毕,泣而歌《沁园春》一阕,其所自制也。词曰:

一别三年,一日空秋,君何不归?记尊嫜抱病,亲供药饵,高茔埋葬,亲曳麻衣。夜卜灯花,晨占鹊喜,雨打梨花昼掩扉。谁知道,把恩情永隔,书信全稀! 于戈满目交挥,奈命薄时乖履祸机。向销金帐里,猿惊鹤怨,香罗巾下,玉碎花飞。要学三贞,须拼一死,免被旁人话是非。君相念:算除非画里,重见崔徽!

每歌一句,则悲啼数声,凄惶怨咽,殆不成腔。赵子延之入室,谢其奉母之孝,莹墓之劳,杀身之节,感愧不已。乃收泪而自叙曰:“妾本倡流,素非良族。山鸡野骛,家莫能驯;路柳墙花。人皆可折。惟知倚门而献笑,岂解举案以齐眉。令色巧言,迎新送旧。东家食而酉家宿。久习遗凤;张郎妇而李郎妻,本无定性。幸蒙君子,求为室家,即便弃其旧染之污,革其前事之失。操持井臼,采掇蘋蘩。严祀祖之仪,笃奉姑之道。事以礼,葬以礼,无愧于心;歌于斯,哭于斯,未尝窥户。岂料昊天不吊,大患来临!毒手老拳,交争于四境;长枪大剑,耀武于三军。既据李崧之居,又夺韩翃之妇。良人万里,贱妾一身。岂不知偷生之可安,忍辱之耐久。而乃甘心玉略,决意珠沉。若飞蛾之扑灯,似赤子之入井,乃己之自取,非人之不容。盖所以愧夫为人妻妾而背主弃家,受人爵禄而忘君负国者也。”赵子抚慰良久,因问太夫人安在?曰:“尊姑在世无罪,闻已受生于人间矣。”赵子曰:“然则,君何以犹堕鬼趣?”对曰:“妾之死也,冥司以妾贞烈,即令往无锡宋家。托为男子。妾以与君情缘之重,必欲俟君一见,以叙怀抱,故迟之岁月耳。今既见君矣,明日即往降生也。君如不弃旧情,可往彼家见访,当以一笑为验。”遂与赵子入室欢会,款若平生。鸡鸣而起,下阶敛步,复回顾拭泪云:“赵郎珍重,从此永别矣!”因哽咽伫立。夭色渐明,欻然而逝,不复有睹。但空室俏然,寒灯半灭而己。赵子起而促装,径赴无锡,寻宋氏之居而叩焉,则果得一男子,怀妊二十月矣。然自降生之后,至今哭不辍声。赵子具述其事,愿请见之,果一笑而哭止,其家遂名之曰罗生。赵子求为亲属,自此往来馈遗,音问不绝云。

\chapter{翠翠传}

翠翠,姓刘氏,淮安民家女也。生而颖悟,能通诗书,父母不夺其志,就令入学。同学有金氏子者,名定,与之同岁,亦聪明俊雅。诸生戏之曰:“同岁者当为夫妇。”二人亦私以此自许。金生赠翠翠诗曰:

十二阑干七宝台,春风到处艳阳开。

东园桃树西园柳,何不移教一处栽?

翠翠和曰:

平生每恨祝英台,凄抱何为不肯开?

我愿东君勤用意,早移花树向阳栽。

已而翠翠年长,不复至学。年及十六,父母为其议亲,辄悲泣不食。以情问之,初不肯言,久乃曰:“必西家金定,妾已许之矣。若不相从,有死而已,誓不登他门也!”父母不得已,听焉。然而刘富而金贫,其子虽聪俊,门户甚不敌。及媒氏至其家,果以贫辞,惭愧不敢当。媒氏曰:“刘家小娘子,必欲得金生,父母亦许之矣。若以贫辞,是负其诚志,而失此一好姻缘也。今当语之曰:‘寒家有子,粗知诗礼,贵宅见求,敢不从命。但生自蓬筚,安于贫贱久矣,若责其聘问之仪,婚娶之礼,终恐无从而致。’彼以爱女之故,当不较也。”其家从之。媒氏复命,父母果曰:“婚姻论财,夷虏之道,吾知择婿而已,不计其他。但彼不足而我有余,我女到彼,必不能堪,莫若赘之入门可矣。”媒氏传命再往,其家幸甚。遂涓日结亲,凡币帛之类,羔雁之属,皆女家自备。过门交拜,二人相见,喜可知矣!是夕,翠翠于枕上作《临江仙》一阕赠生曰:

曾向书斋同笔砚,故人今作新人。洞房花烛十分春!汗沾蝴蝶粉,身惹麝香尘。 殢雨尤云浑未惯,枕边眉黛羞颦,轻怜痛惜莫嫌频。愿郎从此始,日近日相亲。

邀生继和。生遂次韵曰:

记得书斋同讲习,新人不是他人。扁舟来访武陵春:仙居邻紫府,人世隔红尘。誓海盟山心已许,几番浅笑轻颦,向人犹自语频频。意中无别意,来后有谁亲?

二人相得之乐,虽孔翠之在赤霄,鸳鸯之游绿水,未足喻也。

未及一载,张士诚兄弟起兵高邮,尽陷沿淮诸郡,女为其部将李将军者所掳。至正末,士诚辟土益广,跨江南北,奄有浙西,乃通款元朝,愿奉正朔,道途始通,行旅无阻。生于是辞别内、外父母,求访其妻,誓不见则不复还。行至平江,则闻李将军见为绍兴守御;及至绍兴,则又调屯兵安丰矣;复至安丰,则回湖州驻扎矣。

生来往江淮,备经险阻,星霜屡移,囊囊又竭,然此心终不少懈;草行露宿,丐乞于人,仅而得达湖州。则李将军方贵重用事,威焰赫奕。生伫立门墙,踌躇窥俟,将进而未能,欲言而不敢。阍者怪而问焉。生曰:“仆,淮安人也,丧乱以来,闻有一妹在于贵府,是以不远千里至此,欲求一见耳。”阍者曰:“然则汝何姓名?汝妹年貌若干?愿得详言,以审其实。”生曰:“仆姓刘,名金定,妹名翠翠,识字能文。当失去之时,年始十七,以岁月计之,今则二十有四矣。”阍者闻之,曰:“府中果有刘氏者,淮安人,其齿如汝所言,识字、善为诗,性又通慧,本使宠之专房。汝信不妄,吾将告于内,汝且止此以待。”遂奔趋入告。须臾,复出,领生入见。将军坐于厅上,生再拜而起,具述厥由。将军,武人也,信之不疑,即命内竖告于翠翠曰:“汝兄自乡中来此,当出见之。”翠翠承命而出,以兄妹之礼见于厅前,动问父母外,不能措一辞,但相对悲咽而已。将军曰:“汝既远来,道途跋涉,心力疲困,可且于吾门下休息,吾当徐为之所。”即出新衣一袭,令服之,并以帷帐衾席之属设于门西小斋,令生处焉。翌日,谓生曰:“汝妹能识字,汝亦通书否?”生曰:“仆在乡中,以儒为业,以书为本,凡经史子集,涉猎尽矣,盖素所习也,又何疑焉?”将军喜曰:“吾自少失学,乘乱崛起。方响用于时,趋从者众,宾客盈门,无人延款,书启堆案,无人裁答。汝便处吾门下,足充一记室矣。”

生,聪敏者也,性既温和,才又秀发,处于其门,益自检束,承上接下,咸得其欢,代书回简,曲尽其意。将军大以为得人,待之甚厚。然生本为求妻而来,自厅前一见之后,不可再得,闺阁深邃,内外隔绝,但欲一达其意,而终无便可乘。荏苒数月,时及授衣,西风夕起,白露为霜,独处空斋,终夜不寐,乃成一诗曰:

好花移入玉阑干,春色无缘得再看。

乐处岂知愁处苦,别时虽易见时难。

何年塞上重归马?此夜庭中独舞鸾。

雾阁云窗深几许?可怜辜负月团圆。

诗成,书于片纸,拆布裘之领而缝之,以百钱纳于小竖,而告曰:“天气已寒,吾衣甚薄,乞挤入付吾妹,令浣濯而缝纫之,将以御寒耳。”小竖如言持入。翠翠解其意,拆衣而诗见,大加伤感,吞声而泣,别为一诗,亦缝于内,以付生。诗曰:

一自乡关动战锋,旧愁新恨几重重!

肠虽已断情难断,生不相从死亦从。

长使德言藏破镜,终教子建赋游龙。

绿珠碧玉心中事,今日谁知也到侬!

生得诗,知其以死许之,无复致望,愈加抑郁,遂感沉痼。翠翠请于将军,始得一至床前问候,而生病已亟矣。翠翠以臂扶生而起,生引首侧视,凝泪满眶,长吁一声,奄然命尽。将军怜之,葬于道场山麓。翠翠送殡而归,是夜得疾,不复饮药,展转衾席,将及两月。一旦,告于将军曰:“妾弃家相从,已得八载。流离外境,举目无亲,止有一兄,今又死矣。妾病必不起,乞埋骨兄侧,黄泉之下,庶有依托,免于他乡作孤魂也。”言尽而卒。将军不违其志,竟附葬于生之坟左,宛然东西二丘焉。

洪武初,张氏既灭,翠翠家有一旧仆,以商贩为业,路经湖州,过道场山下,见朱门华屋,槐柳掩映,翠翠与金生方凭肩而立。遽呼之入,访问父母存殁,及乡井旧事。仆曰:“娘子与郎安得在此?”翠翠曰:“始因兵乱,我为李将军所掳,郎君远来寻访,将军不阻,以我归焉,因遂侨居于此耳。”仆曰:“予今还淮安,娘子可修一书以报父母也。”翠翠留之宿,饭吴兴之香糯,羹苕溪之鲜鲫,以乌程酒出饮之。明旦,遂修启以上父母曰:

伏以父生母育,难酬罔极之恩;夫唱妇随,夙著三从之义。在人伦而已定,何时事之多艰!;曩者汉日将颓,楚氛甚恶,倒持太阿之柄,擅弄潢池之兵。豕长蛇,互相吞并;雄蜂雌蝶,各自逃生。不能玉碎于乱离,乃至瓦全于仓卒。驱驰战马,随逐征鞍。望高天而八翼莫飞,思故国而三魂屡散。良辰易迈,伤青鸾之伴木鸡;怨偶为仇,惧乌鸦之打丹凤。虽应酬而为乐,终感激而生悲。夜月杜鹃之啼,春风蝴蝶之梦。时移事往,苦尽甘来。今则杨素览镜而归妻,王敦开閤而放妓,蓬岛践当时之约,潇湘有故人之逢。自怜赋命之屯,不恨寻春之晚。章台之柳,虽已折于他人;玄都之花,尚不改于前度。将谓瓶沉而簪折,岂期壁返而珠还。殆同玉萧女两世姻缘,难比红拂妓一时配合。天与其便,事非偶然。煎鸾胶而续断弦,重谐缱绻;托鱼腹而传尺素,谨致丁宁。未奉甘旨,先此申复。

父母得之,甚喜。其父即赁舟与仆自淮徂浙,径奔吴兴。至道场山下畴昔留宿之处,则荒烟野草,狐兔之迹交道,前所见屋宇,乃东西两坟耳。方疑访间,适有野僧扶锡而过,叩而问焉。则曰:“此故李将军所葬金生与翠娘之坟耳,岂有人居乎?”大惊。取其书而视之,则白纸一幅也。

时李将军为国朝所戮,无从诘问其详。父哭于坟下曰:“汝以书赚我,令我千里至此,本欲与我一见也。今我至此,而汝藏踪秘迹,匿影潜形。我与汝,生为父子,死何间焉?妆如有灵,毋齐一见,以释我疑虑也。”是夜,宿于坟。以三更后,翠翠与金生拜跪于前,悲号宛转。父泣而抚问之,乃具述其始末曰:“往者祸起萧墙,兵兴属郡。不能效窦氏女之烈,乃致为沙吒利之躯。忍耻偷生,离乡去国。恨以惠兰之弱质,配兹驵侩之下材。惟知夺石家买笑之姬,岂暇怜息国不言之妇。叫九阍而无路,度一日而三秋。良人不弃旧恩,特勤远访,托兄妹之名,而仅获一见,隔伉俪之情,而终遂不通。彼感疾而先殂,妾含冤而继殒。欲求袝葬,幸得同归。大略如斯,微言莫尽。”父曰:“我之来此,本欲取汝还家,以奉我耳。今汝已矣,将取汝骨迁于先垄,亦不虚行一遭也。”复泣而言曰:“妾生而不幸,不得视膳庭闱;殁且无缘,不得首丘茔垄。然而地道尚静,神理宜安,若更迁移,反成劳扰。况溪山秀丽,草木荣华,既已安焉,非所愿也。”因抱持其父而大哭。父遂惊觉,乃一梦也。明日,以牲酒奠于坟下,与仆返棹而归。

至今过者指为金、翠墓云。

\part{}

\chapter{龙堂灵会录}

吴江有龙王堂,堂,盖庙也,所以奉事香火,故谓之堂。或以为右崖陡出,若塘岸焉,故又谓之龙王塘。其地左吴淞而右太湖,风涛险恶,众水听汇,过者必致敬于庙庭而后行,夙著灵异,具载于范石湖所编《吴郡志》。元统间,闻生子述者,以歌诗鸣于吴下。因过其处,适值龙挂,乃白龙也,毊鬣下垂,如一玉住,鳞甲照耀,如明镜数百片;转侧于乌云之内,良久而没。子述自以为平生奇观,莫之能及。雨止,登庙,周览既毕,乃题古风一章于庑下曰:

龙王之堂龙作主,栋宇青红照江渚,岁时奉事孰敢违,求晴得晴雨得雨。平生好奇无与侔,访水寻山遍吴楚,扁舟一叶过垂虹,濯足沧浪浣尘土。神龙有心慰劳苦,变化凤云快观睹,毊尾蜿蜒玉柱垂,鳞甲光芒银镜舞。村中稽首朝翁姥,船上燃香拜商贾,共说神龙素有灵,降福除灾敢轻侮!我登龙堂共龙语,至诚感格龙应许。汲挽湖波作酒浆,采掇江花当肴脯。大字淋漓写庭户,过者惊疑居者怒。世间不识谪仙人,笑别神龙指归路。

题毕,回舟,卧于蓬下。忽有鱼头鬼身者,自庙而来,施礼于前曰:“龙王奉邀。”子述曰:“龙玉处于水府,贱子游于尘世,风马牛之不相及也。虽有严命,何以能至!”鱼头者曰:“君毋苦,但请瞑目,少顷即当至矣。”子述如言,但闻风水声,久之,惭止,开目,则见殿宇峥嵘,仪卫森列,寒光逼人,不可睇视,真所谓水晶宫也。王闻其至,冠眼剑珮而出,延之上阶,致谢曰:“日间蒙惠高作,伺旨既佳,笔势又妙,庙庭得此,光彩倍增。是以屈君至此,欲得奉酬。”坐未定,阍者传言客至,王遽出门迎接。见有三人同入,其一高冠巨履,威仪简重;其一乌帽青裘,风度潇洒;其—则葛巾野服而已。分次而坐。王谓子迷曰:“君不识三客乎?乃越范相国,晋张使君,唐陆处士耳,世所渭吴地三高是也。”王对三客言子述题诗之事,俱各传观,称赞不已。王曰:“诗人远临,贵客偕至,赏心乐事,不期而同。”即命左右设宴于中堂,凡铺陈之物,饮馔之味,皆非人世所有。酒至,方欲饮,阍者奔入曰:“吴大夫伍君在门。”王急起迎之。既入,范相国犹据首席,不能谦避。伍君勃然变色而谓王曰:“此地乃吴国之境,王乃吴地之神,吾乃吴国之忠臣,彼乃吴国之仇人也。吴俗无知,妄以三高为目,立亭馆以奉之。王又延之入室,置之上座,曩日吞吴之恨,宁忍忘之耶?”即数范相国:“汝有三大罪,而人罔知,故千载之下,得以欺世而盗名。吾今为汝一白之,使大奸无所容,大恶不得隐矣!”相国默然,请闻其说。乃曰:“昔勾践志于复仇,卧薪尝胆,十年生聚,十年教训。以此战伐,孰能御之?何至假负薪之女,为诲淫之事,出此鄙计,不以为惭。吴既已亡,又不能除去尤物,反与共载而去。昔太公蒙面以斩妲己,高颎违令而诛丽华,以此方之,孰得孰失?是谋国之不臧也。既已灭吴,以勾践为人,长颈鸟喙,可与共患难,不可与同逸乐,浮海而去,以书遗大夫种云:‘蜚鸟尽,良弓藏;狡兔死,走狗烹,子可以去矣。’夫自不能事君,又诱其臣与之偕去,令其主孤立于上,国空无人,于心安乎?昔鲍叔之荐管仲,萧何之追韩信,以此方之,轨是孰非?是事君之不忠也。既已去位,本求高蹈。何乃聚敛积实。耕于海滨,父子力作,以营千金,屡散而复积,此欲何为哉?昔鲁仲连辞金而不受,张于房辟谷而远引,以此方之,孰贤孰愚?是持身之不廉也。负此三大罪,安得居吾之上乎?”相国面色如土,不敢出声。久之,乃曰:“子之罪我则然矣!愿闻子之所事。”伍君曰:“吾以家族之不幸。遍游诸国,不避艰险,终能用吴以复父兄之仇,又能为夫盖复父之仇,则孝为有余矣。事吴至死不去,以毕志于其君,虽遭属镂之惨,终无怨词,则忠为有余矣。君不终用,至于临死,又能逆料沼吴之祸,而为身后之忧,则智为有余矣。使吾尚在,则会稽之栖,下可以复振;欈李之战,不可以诡胜;而越之君臣将不暇于朝食,又焉能得志于吾国乎?盖尝论之,吴之亡不在于西子之进,而在于吾之被谗,越之霸不在于种、蠡之用,而在于吾之受戮。吾若不死,则苎萝之妹,适足为后宫之娱;荣楯之华,适足为前殿之夸,姑苏之台,麋鹿岂可得游;至德之庙,禾黍岂至于遽生哉!惟自残其骨骾,自害其股肱,故仇人得以乘其机,敌国得以投其隙,盖有幸而然耳。岂子代国之功,谋国之策乎?”相国辞塞,乃虚位以让之。伍君遂据其上,相国居第二位,第三、第四位则张使君、陆处士,子述居第五,王坐于末席。已而酒行乐作。王请坐客各赋诗歌以为乐。伍君乃左抚剑,右击盆,朗朗而作歌曰:

驾艅艎之长舟兮,览吴会之故都。怅馆娃之无人兮,麋鹿游于姑苏。忆吴子之骤强兮,盖得人以为任。战柏举而入楚兮,盟黄池而服晋。何用贤之不终兮,乃自坏其长城。洎雨东而乞死兮,始踯躅而哀鸣。泛鸱夷于江中兮,驱白马于潮头。眄胥山之旧庙兮,挟天风而远游。龙宫鬱其嵯峨兮,水殿开而宴会。日既吉而辰良兮,接宾朋之冠珮。莫椒浆而酌桂醑兮,击金钟而戛鸣球。湘妃汉女出歌舞兮,瑞雾霭而祥烟浮。夜迢迢而未央兮,心摇摇而易醉。抚忙剑而作歌兮,聊以泄千古不平之气。

歌竟,范相国持杯而咏诗曰:

霸越平吴,扁舟五湖,昂昂之鹤,泛泛之鳬。

功成身退,辞荣避位,良弓既藏,黄金曷铸?

万岁千秋,魂魄来游,今夕何夕,于此淹留!

吹笙击鼓,罗列樽俎,妙女娇娃,载歌载舞,

有酒如河,有肉如坡,相对不乐,日月几何?

金樽翠爵,为君斟酌,后会未期,且此欢谑。

张使君亦倚席而吟侍曰:

驱车适故国,挂席来东吴。西风旦夕起,飞尘满皇都。

人生在世间,贵乎得所图。问渠华亭鹤,何似松江鲈?

岂意千年后,高名犹不孤。鬱鬱神灵府,济济英俊徒。

华筵列玳瑁,美酝倾醍醐。妙舞蹑珠履,狂吟扣金壶。

顾余复何人?亦得同歌呼。作诗记胜事,流传遍江湖。

陆处士遂离席而陈诗曰:

生计萧条具一船,笔床茶灶共周旋。

但笼甫里能言鸭,不钓襄江缩项鳊。

鼓瑟吹笙传盛事,倒冠落珮预华筵。

何须温峤燃犀照,已被旁人作话传。

子述乃制长短句一篇,献于座间曰:

江湖之渊,神物所居,

珠宫贝阙,与世不殊。

黄金作屋瓦,白玉为门枢,

屏开玳瑁甲,槛植珊瑚珠。

祥云瑞霭相扶舆,上通三光下八区,

自非冯夷与海若,孰得于此久踌躇!

高堂开宴罗宾主,礼数繁多冠冕聚,

忙呼玉女捧牙盘,催唤神娥调翠釜。

长鲸鸣,巨蛟舞,鳖吹笙,鼍击鼓。

骊颔之珠照樽俎,虾须之帘挂廊庑。

八音迭奏杂仙韶,宫商响切逼云霄,

湘妃姊妹抚瑶瑟,秦家公主来吹萧。

麻姑碎擘麒麟脯,洛妃斜拂凤凰翘,

天吴紫凤颠倒而奔走,金支翠旗缥缈而动摇。

胥山之神余所慕,曾谒神祠拜神墓。

相国不改古衣冠,使君犹存晋风度。

座中更有天随生,口食杞菊骨骼清,

平生梦想不可见,岂期一旦皆相迎。

主人灵圣尤难测,驱驾风云归顷刻,

周游八极隘四溟,固知不是池中物。

鲰生何幸得遭逢,坐令槁朽生华风!

待以天厨八珍之异馔,饮以仙府九酝之深鍾。

唾壶缺,麇柄折,醉眼生花双耳热。

不來洲畔采明珠,不去波间摸明月,

但将诗旬写鲛绡,留向龙宫记奇绝。

歌咏俱毕,觥筹交错。但闻水村喔喔晨鸡鸣,山寺隆隆晓钟击。伍君先别,三高继往。王以红珀盘捧照乘之珠,碧瑶箱盛开水之角,馈赠于子述,命使送还。抵舟,则东方洞然,水路明朗,乃于中流稽首庙堂而去。

\chapter{太虚司法传}

冯大异,名奇,吴、楚之狂士也。恃才傲物,不信鬼神,凡依草附木之妖,惊世而骇俗者,必攘臂当之,至则凌慢毁辱而后巳,或火其祠,或沉其像,勇往不顾,以是人亦以胆气许之。

至元丁丑,侨居上蔡之东门有故之近村,时兵燹之后,荡无人居,黄沙白骨,一望极目。未至而斜日西沉,愁云四起,既无旅店,何以安泊。道旁有一古柏林,即投身而入,倚树少憩。鸺鹬鸣其前,豺狐嗥其后。顷之,有群鸦接翅而下,或跂一足而啼,或鼓双翼而舞,叫噪怪恶,循环作阵。复有八九死尸,僵卧左右,阴凤飒飒,飞雨骤至,疾雷一声,群尸环起,见大异在树下,踊跃趋附。大异急攀缘上树以避之,群尸环绕其下,或啸或詈,或坐或立,相与大言曰:“今夜必取此人!不然,吾属将有咎”已而云收雨止,月光穿漏,见一夜叉自远而至,头有二角,举体青色,大呼阔步,径至林下,以手撮死尸,摘其头而食之,如啖瓜之状;食讫,饱卧,鼾睡之声动地。大异度不可久留,乘其熟寐,下树迸逸,行不百步,则夜叉已在后矣,舍命而奔,几为所及。遇一废寺,急入投之,东西廊皆倾倒,惟殴上有佛像一躯,其状甚伟。见佛背有一穴,大异计穷,窜身入穴,潜于腹中,自渭得所托,可无虞矣。忽闻佛像鼓腹而笑曰:“彼求之而不得,吾不求而自至,今夜好顿点心,不用食斋也!”即振迅而起,其行甚重,将十步许,为门限所碍。蹴然仆地,土木狼籍,胎骨糜碎矣。大异得出,犹太言曰:“胡鬼弄汝公,反自掇其祸矣!”即出寺而行。遥望野中,灯烛荧煌,诸人揖让而坐。喜甚,弛往赴之。及至,则皆无头者也,有头者则无一臂,或缺一足。大异不顾而走。诸鬼怒曰:“吾辈方此酣畅,此人大胆,敢来冲突!正当执之以为脯胾耳。”即踉蹡哮吼,或搏牛粪而掷,或攫人骨而投,无头者则提头以趁之。前阻一永,大异乱流而渡,诸鬼至永。则不敢越。蓦及半里,大异回顾,犹闻喧哗之声,靡靡不已。须臾,月堕,不辨蹊径,失足坠一坑中,其深无底,乃鬼谷也。寒沙眯目,阴气彻骨,群鬼萃焉。有赤发而双角者,绿毛而两翼者,鸟喙而獠牙者,牛头而兽面者,皆身如蓝靛,口吐火焰,见大异至,相贺曰:“仇人至矣!”即以铁纽系其颈,皮繂拴其腰,驱至鬼王之座下,告曰:“此即在世不信鬼神,凌辱吾徒之狂土也。”鬼玉怒责之曰:“汝具五体而有知识。岂不闻鬼神之德其盛矣乎?孔子圣人也,犹曰敬而远之。大《易》所谓载鬼一车,《小雅》听谓为鬼为蜮。他如《左传》所纪,晋景之梦,伯有之事,皆是物也。汝为河人,犹言其无?吾受汝侮久矣!今幸相遇,吾乌得而甘心焉。”即命众鬼卸其冠裳,加以棰楚,流血淋漓,求死不得。鬼王乃谓之曰:“汝欲调泥成酱乎?汝欲身长三丈乎?”大异念泥岂可为酱,因愿身长三丈。众鬼即捽之于石床之上,如搓粉之状,众手反复而按摩之,不觉渐长,已而扶起,果三丈矣,袅袅如竹竿焉。众笑辱之,呼为长竿怪。王又谓之曰:“汝欲煮右成汁乎?汝欲身矮一尺乎?”大异方苦其长,不能自立。即愿身矮一尺。众鬼又驱至石床上,如按面之状,极力一捺,骨节磔磔有声,乃拥支起,果一尺矣,团圞如巨蟹焉。众又笑辱之,呼为蟛蜞怪。大异蹒跚于地,不胜其苦。旁有一老鬼,抚掌大笑曰:“足下平日不信鬼怪,今日何故作此形骸?”乃请于众曰:“彼虽无礼,毖遭辱亦甚矣,可怜许,请宥之!”即以两手提挈大异而抖擞之,须曳复故。大异求还,诸鬼曰:“汝既到此,不可徒返,吾等各有一物相赠,所贵人间知有我辈耳。”老鬼曰:“然则,以何物赠之?”一鬼曰:“吾赠以拨云之角。”即以两角置于大异之额,岌然相向。一鬼曰:“吾赠以哨风之嘴。”即以一铁嘴加于其唇,尖锐如鸟喙焉。一鬼曰:“吾赠以朱华之发。”即以赤水染其发,皆鬅鬙而上指,其色如火。一鬼曰:“吾赠以碧光之睛。”即以二青珠嵌于其目,湛湛而碧色矣。老鬼遂送之出坑曰:“善自珍重,向者群小溷渎,幸勿记怀也。”大异虽得出,然而顶拨云之角,戴哨风之嘴,被朱华之发,含碧光之睛,俨然成一奇鬼。到家,妻孥不敢认;出市,众共聚观。以为怪物,小儿则惊啼而逃避。遂闭户不食,愤懑而死。临死,谓其家曰:“我为诸鬼所困,今其死矣!可多以纸笔置柩中,我将讼之于天。数日之内,蔡州有一奇事,是我得理之时也,可沥酒而贺我矣。”言讫而逝。过三日,白昼风雨大作,去雾四塞,雷霆霹雳,声振寰宇,屋瓦皆飞,大木尽拔,经宿始霁。则所堕之坑,陷为一巨泽,弥漫数里,其水皆赤。忽闻柩中作语曰:“讼已得理!诸鬼皆夷灭无遗!无府以吾正直,命为太虚殿司法,职任隆重,不复再来人世矣。”其家祭而葬之,肸蠁之间,如有灵焉。

\chapter{修文舍人传}

夏颜,字希贤,吴之震泽人也。博学多闻,性气英迈,幅巾布裘,游于东西两浙间。喜慷慨论事,叠叠不厌,人每倾下之。然而命分甚薄,日不暇给,尝喟然长叹曰:“夏颜,汝修身谨行,奈何不能润其家乎?”则又自解曰:“颜渊困于陋巷,岂道义之不足也?贾谊屈于长沙,岂文章之不贍也?校尉封拜而李广不侯,岂智勇之不逮也?侏儒饱死而方朔苦饥,岂才艺之不敏也?盖有命焉,不可幸而致。吾知顺受而已,岂敢非理妄求哉!”

至正初,客死润州,葬于北固山下。友人有与之契厚者,忽遇之于途。见颜驱高车,拥大盖,峨冠曳珮,如侯伯状,从者各执其物,呵殿而随护,风采扬扬,非复住日,投北而去。友人不敢呼之。

一日,早作,复遇之于里门,颜遽搴帷下车而施揖曰:“故人安否?”友人遂与叙旧,执手款语,不异平生。乃问之曰:“与君隔别未久,而能自致青云,立身要路。车马仆从,如此之盛;衣服冠带,如此之华,可谓大丈夫得志之秋矣!不胜健羡之至!”颜曰:“吾今隶职冥司,颇极清要。故人下问,何敢有隐,但途路之次,未暇备述,如不相弃,可于后夕会于甘露寺多景楼,庶得从容时顷,少叙间阔,不知可乎?望勿以幽冥为讶,而负此诚约也。”友人许之。告别而去。是夕,携洒而往,则颜已先在,见其至,喜甚,迎谓曰:“故人真信士,可谓死生之交矣!”乃言曰:“地下之乐,不减人间,吾今为修文舍人,颜渊、卜商旧职也。冥司用人,选擢甚精,必当其才,必称其职,然后官位可居,爵禄可致,非若人间可以贿赂而通,可以门第而进,可以外貌而滥充,可以虚名而攫取也。试与君论之:今夫人世之上,仕路之间,秉笔中书者,岂尽萧、曹、丙、魏之徒乎?提兵阃外者,岂尽韩、彭、卫、霍之流乎?馆阁擒文者,岂皆班、扬、董、马之辈乎?郡邑牧民者,岂皆龚、黄、召,杜之俦乎?骐骥服盐车而驽骀厌刍豆,凤凰栖枳棘而鸱鸮鸣户庭,贤者槁项黄馘而死于下,不贤者比肩接迹而显于世,故治日常少,乱日常多,正坐此也。冥司则不然、黜陟必明,赏罚必公,昔日负君之贼,败国之臣,受穹爵而享厚禄者,至此必妥其殃,昔日积善之家,修德之士,阨下位而困穷途者,至此必蒙其福。盖轮口之数,报应之条,至此而莫逃矣。”遂引满而饮,连举数觥,凭栏观眺,口占律诗二章,吟赠友人曰:

笑拍阑干扣玉壶,林鸦惊散渚禽呼。

一江流水三更月,两岸青山六代都。

富贵不来吾老矣,幽明无间子知乎?

旁人若问前程事,积善行仁是坦途。

满身风露夜茫茫,一片山光与水光。

铁瓮城边人玩月,鬼门关外客还乡。

功名不博诗千首,生死何殊梦一场!

赖有故人知此意,清谈终夕据藤床。

吟讫,搔首而言曰:“太上立德,其次立功。其次立言。仆在世之日,无德可称,无功可述,然而著成集录,不下数百卷。作为文章,将及千余篇,皆极深研几,尽意而为之者。奄忽以来,家事零替,内无应门之童,外绝知音之士,盗贼之所攘窃,虫鼠之所毁伤,十不存一,甚可惜也。伏望故人以怜才为念,恤交为心,捐季子之宝剑,付尧夫之麦舟,用财于当行,施德于不报,刻之桐梓,传于好事,庶几不与草木同腐此则故人之赐也。兴言及此,惭愧何胜!”友人许诺。颜大喜,捧觞拜献,以致丁宁之意。已而,东方渐曙,告别而去。友人吴中,访其家,除散亡零落外,犹得遗文数百篇,并所薯《汲古录》、《通玄志》等书,亟命工镂版,鬻之予肆,以广其传。颜复到门致渤。自此往来无间,其家吉凶祸福,皆前期报之。三年之启,友人感疾,颜来访问,因谓曰:“仆备员修文府,日月已满,当得举代。冥间最重此职,得之甚难。君若不欲,则不敢强;万一欲之,当与尽力。所以汲汲于此者,盖欲报君镂版之恩耳。人生会当有死,纵复强延数年,何可得居此地也?”友人欣然许之,遂处置家事,不复治疗,数日而终。

\chapter{鉴湖夜泛记}

处士成令言,不求闻达,素爱会稽山水。天历间,卜居鉴湖之滨,诵“千岩竞秀,万壑争流”之句,终日遨游不辍。常乘一叶小舟,不施篙橹,风帆浪揖,任其所之,或观鱼水涯,或盟鸥沙际,或蘋洲狎鷺,或柳岸闻鸳。沿湖三十里,飞者走者,浮者跃者,皆熟其状貌,与之相忘,自去自来,不复疑俱。而樵翁、耕叟、渔童、牧竖遇之,不问老幼,俱得其欢心焉。初秋之夕,泊舟千秋观下,金凤乍起,白露未零,星斗交辉,水天一色,时闻菱歌莲唱,应答于洲渚之间。令言卧舟中,仰视天汉,如白练万丈,横亘于南北,纤云扫迹,一尘不起。乃扣船舷,歌宋之问明河之篇,飘飘然有遗世独立,羽化登仙之意。舟忽自动,其行甚速,风水俱駃,一瞬千里,若有物引之者。令言莫测。须臾,至—处,寒气袭入,清光夺目,如玉田湛湛,琪花瑶草生其中,如银海洋洋,异兽神鱼泳其内。乌鸦群鸣,白榆乱植。令言度非人间,披衣而起,见珠宫岌然,宫阙高耸。有一仙娥,自内而出,被冰绡之衣,曳霜纨之帔,戴翠凤步摇之冠,蹑琼纹九章之履。侍女二人,一执金柄障扇,一捧玉环如意,星眸月貌,光彩照人。至岸侧,谓令言曰:“处士来何迟?”令言拱而对曰:“仆晦迹江湖,忘形鱼鸟,素乏诚约,又昧平生,何以有来迟之问?”仙娥笑曰:“卿安得而识我乎?所以奉邀至此者,盖以卿夙负高义,久存硕德,将有诚悃,籍卿传之于世耳。”乃请令言登岸,邀之入门,行数十步,见一大殿,榜曰:天章之殿。殿后有一高阁,题曰:灵光之阁。内设云母屏,铺玉华箪,四面皆水晶帘,以珊瑚钩挂之,通明如白昼。梁间悬香球二枚,兰麝之气,芬芳触鼻。请令言对席坐而语之曰:“卿识此地乎?即人世所谓天河,妾乃织女之神也。此去尘间,已八万余里矣。”令言离席而言曰:“下界愚民,甘与草木同腐。今夕何幸,身游天府,足践仙宫,获福无量,受恩过望。然未知尊神欲托以何事,授以何言?愿得详闻,以释尘虑。”仙娥乃低首敛躬,端肃而致词曰:“妾乃天帝之孙,灵星之女,夙禀贞性,离群索居。岂意下土无知,愚民好诞,妄传秋夕之期,指作牵牛之配,致令清洁之操,受此污辱之名。开其源者,齐谐多诈之书;鼓其波者,楚俗不经之语;傅会其说而倡之者,柳宗元乞巧之文,铺账其事而和之者。张文潜七夕之咏。强词雄辩,无以自明;鄙语邪言,何所不至!往往形诸简牍,播于篇章,有曰:‘北斗佳人双泪流,眼穿肠断为牵牛。’又曰:‘莫言天上稀相见,犹胜人间去不回!’有曰:‘未会牵牛意若何,须邀织玄弄金梭。’又曰:‘时人下用穿针待,没得心情送巧来。’似此者不一而足,亵侮神灵。罔知忌惮,是可忍也,孰不可忍!”令言对曰:“鹊桥之会,牛渚之游,今听神言,审其妄矣。然如嫦娥月殿之奔,神女高唐之会,后土灵佑之事,湘灵冥会之诗,果有之乎,抑未然乎?”仙娥怃然曰:“嫦娥者,月宫仙女;后土者,地祗贵神;大禹开峡之功。巫神实佐之;而湘灵者,尧女舜妃。是皆圣贤之裔,贞烈之伦,乌有如世俗所谓哉!非若上元之降封陟,云英之遇裴航,兰香之嫁张硕,彩鸾之配文箫,情欲易生,事迹难掩者也。世人咏月之诗曰:‘嫦娥应悔偷灵药,碧海青天夜夜心。’题峡之句曰:‘一自高唐赋成后,楚他天云雨尽堪疑。’夫日月两曜。混沦之际,开辟之初,既已具矣,岂有羿妻之说,窃药之事,而妄以孤眠霜宿侮之乎?云者,山川灵气,雨者,天地沛泽,奈何因宋玉之谬,辄指为房帷之乐,譬之衽席之欢?慢伸渎天,莫此为甚!湘君夫人,帝舜之配,陟方之日,盖已老矣。李群玉者,果何人欤?敢以淫邪之词,溷于黄陵之庙曰:‘不知精爽落何处,疑是行云秋色中。’自述奇遇。引归其身,诞妄矫诬,名检扫地!后土之传,唐人不敢明斥则天之恶,故假此以讽之耳。世俗不识,便谓诚然,至有‘韦郎年少眈闲事,案上休看《太白经》’之句。夫欲界诸天,皆有配耦,其无耦者,则无欲者也。士君子于名教中自有乐地,何至造述鄙猥,诬谤高明,既以欺其心,又以惑于世,而自处于有过之域哉!幸卿至世,悉为白之,毋令云霄之上,星汉之间,久受黄口之谗,青蝇之玷也。”令言又问曰:“世俗之多诳,仙真之被诬,今听神言,知其伪矣。然如张骞之乘槎,君平之辨石,将信然欤?抑妄谈欤?”仙娥曰:“此事则诚然矣!夫博望侯乃金门宜吏,严先生乃玉府仙曹,暂谪人间,灵性具在,故能周游八极,辨识异物。岂常人之可比乎?卿非三生有缘,今夕亦乌得至此!”遂出瑞锦二端以赠之,曰:“卿可归矣,所托之事,幸勿相忘。”令盲拜辞登舟,但觉风露高寒,涛澜汹涌,一饭之顷,却回旧所,则淡雾初生,大星渐落,鸡三鸣而更五点矣。取锦视之,与世间所织不甚相异,藏之篋笥,以待博物者辨之。后遇西域贾胡,试出示焉,抚玩移时,改容言曰:“此天上至宝,非人间物也。”令言问:“何以知之?”曰:“吾见其文顺而不乱,色纯而不杂。以日映之,瑞气葱葱而起,以尘掩之,自然飞扬而去。以为幄帐,蚊蚋不敢入,以为衣帔,雨雪不能濡。隆冬御之,不必挟纩而燠;盛夏张之,不必乘风而凉。其蚕盖扶桑之叶所饲,其丝则天河之水所濯,岂非织女机中之物乎?君何从得此?”令言秘之,不肯述其故。遂轻舟短棹,长游不返。后二十年,有遇之于玉笥峰者,颜貌红泽,双瞳湛然,黄冠布裘,不巾不带。揖而问之,则御凤而去,其疾如飞,追之不能及矣。

\chapter{绿衣人传}

天水赵源,早丧父母,未有妻室。延祐间,游学至于钱塘,侨居西湖葛岭之上,其侧即宋贾秋壑旧宅也。源独居无聊,尝日晚徙倚门外,见一女子,从东来,绿衣双鬟,年可十五六,虽不盛妆浓饰,而姿色过人,源注目久之。明日出门,又见,如此凡数度,日晚辄来。源戏问之曰:“家居何处,暮暮来此?”女笑而拜曰:“儿家与君为邻,君自不识耳。”源试挑之,女欣然而应,因遂留宿,甚相亲昵。明旦,辞去,夜则复来。如此凡月余,情爱甚至。源问其姓氏居址,女曰:“君但得美妇而已,何用强知。”问之不已,则曰:“儿常衣绿,但呼我为绿衣人可矣。”终不告以居址所在。源意其为巨室妾媵,夜出私奔,或恐事迹彰闻,故不肯言耳,信之不疑,宠念转密。

一夕,源被酒,戏指其衣曰:“此真可谓‘绿兮衣兮,绿衣黄裳者’也。”女有惭色,数夕不至。及再来,源叩之,乃曰:“本欲相与偕老,奈何以婢妾待之,令人忸怩而不安,故数日不敢侍君之侧。然君已知矣,今不复隐,请得备言之。儿与君,旧相识也,今非至情相感,莫能及此。”源问其故,女惨然曰:“得无相难乎?儿实非今世人,亦非有祸于君者,盖冥数当然,夙缘未尽耳。”源大惊曰:“愿闻其详。”女曰:“儿故宋秋壑平章之侍女也。本临安良家子,少善弈棋,年十五,以棋童入侍,每秋壑朝回,宴坐半闲堂,必召儿侍弈,备见宠爱。是时君为其家苍头,职主煎茶,每因供进茶瓯,得至后堂。君时年少,美姿容,儿见而慕之,尝以绣罗钱箧,乘暗投君。君亦以玳瑁脂盒为赠,彼此虽各有意,而内外严密,莫能得其便。后为同辈所觉,谗于秋壑,遂与君同赐死于西湖断桥之下。君今已再世为人,而儿犹在鬼箓,得非命欤?”言讫,呜咽泣下。源亦为之动容。久之,乃曰:“审若是,则吾与汝乃再世因缘也,当更加亲爱,以偿畴昔之愿。”自是遂留宿源舍,不复更去。源素不善奕,教之弈,尽传其妙,凡平日以棋称者,皆不能敌也。

每说秋壑旧事,其所目击者,历历甚详。尝言:秋壑一日倚楼闲望,诸姬皆侍,适二人乌巾素服,乘小舟由湖登岸。一姬曰:“美哉,二少年!”秋壑曰:“汝愿事之耶?当令纳聘。”姬笑而无言。逾时,令人捧一盒,呼诸姬至前曰:“适为某姬纳聘。”启视之,则姬之首也,诸姬皆战栗而退。又尝贩盐数百艘至都市货之。太学有诗曰:

昨夜江头涌碧波,满船都载相公鹾;

虽然要作调羹用,未必调羹用许多!

秋壑闻之,遂以士人付狱,论以诽谤罪。又尝于浙西行公田法,民受其苦,或题诗于路左云:

襄阳累岁困孤城,豢养湖山不出征。

不识咽喉形势地,公田枉自害苍生。

秋壑见之,捕得,遭远窜。又尝斋云水千人,其数已足,末有一道士,衣裾褴褛,至门求斋。主者以数足,不肯引入,道士坚求不去,不得已于门侧斋焉。斋罢,覆其钵于案而去,众悉力举之,不动。启于秋壑,自往举之,乃有诗二句云:“得好休时便好休,收花结子在漳州。”始知真仙降临而不识也。然终不喻“漳州”之意,嗟乎!孰知有漳州木绵庵之厄也!又尝有艄人泊舟苏堤,时方盛暑,卧于舟尾,终夜不寐,见三人长不盈尺,集于沙际,一曰:“张公至矣,如之奈何?”一曰:“贾平章非仁者,决不相恕!”一曰:“我则已矣,公等及见其败也!”相与哭入水中。次日,渔者张公获一鳖,径二尺余,纳之府第。不三年而祸作。盖物亦先知,数而不可逃也。源曰:“吾今日与汝相遇,抑岂非数乎?”女曰:“是诚不妄矣!”源曰:“汝之精气,能久存于世耶?”女曰:“数至则散矣。”源曰:“然则何时?”女曰:“三年耳。”源固未之信。

及期,卧病不起。源为之迎医,女不欲,曰:“曩固已与君言矣,因缘之契,夫妇之情,尽于此矣。”即以手握源臂,而与之诀曰:“儿以幽阴之质,得事君子,荷蒙不弃,周旋许时。往者一念之私,俱陷不测之祸,然而海枯石烂,此恨难消,地老天荒,此情不泯!今幸得续前生之好,践往世之盟,三载于兹,志愿已足,请从此辞,毋更以为念也!”言讫,面壁而卧,呼之不应矣。源大伤恸,为治棺榇而殓之。将葬,怪其柩甚轻,启而视之,惟衣衾钗珥在耳。乃虚葬于北山之麓。源感其情,不复再娶,投灵隐寺出家为僧,终其身云。

\backmatter

\part{附录}

\chapter{秋香亭记}

至正间,有商生者,随父宦游姑苏,侨居乌鹊桥,其邻则弘农杨氏第也。杨氏乃延祐大诗人浦城公之裔。浦城娶于商,其孙女名采采,与生中表兄妹也。浦城已殁,商氏尚存。生少年,气禀清淑,性质温粹,与采采俱在童卯。商氏,即生之祖姑也。每读书之暇,与采采共戏于庭,为商氏所钟爱,尝抚生指采采谓曰:“汝宜益加进修,吾孙女誓不适他族,当令事汝,以续二姓之親,永以为好也。”女父母乐闻此言,即欲归之,而生严亲以生年幼,恐其怠于学业,请俟他日。生、女因商氏之言,倍相怜爱。

数岁,遇中秋月夕,家人会饮沾醉,遂同游于生宅秋香亭上,有二桂树,垂荫婆娑,花方盛开,月色团圆,香气浓馥,生、女私于其下语心焉。是后,女年稍长,不复过宅,每岁节伏腊,仅以兄妹礼见于中堂而已。闺阁深邃,莫能致其情。后一岁,亭前桂花始开,女以折花为名,以碧瑶笺书绝句二首,令侍婢秀香持以授生,嘱生继和,诗曰:

秋香亭上桂花芳,几度风吹到绣房。

自恨人生不如树,朝朝肠断屋西墙!

秋香亭上桂花舒,用意殷勤种两株。

愿得他年如此树,锦裁步障护明珠。

生得之,惊喜,遂口占二首,书以奉答,付婢持去。诗曰:

深盟密约两情劳,犹有余香在旧袍。

记得去年携手处,秋香亭上月轮高。

高栽翠柳隔芳园,牢织金笼贮彩鸳。

忽有书来传好语,秋香亭上鹊声喧。

生始慕其色而已,不知其才之若是也,既见二诗,大喜欲狂。但翘首企足,以待结褵之期,不计其他也。女后以多情致疾,恐生不知其眷恋之情,乃以吴绫帕题绝句于上,令婢持以赠生。诗曰:

罗帕薰香病裹头,眼波娇溜满眶秋。

风流不与愁相约,才到风流便有愁。

生感叹再三,未及酬和。适高邮张氏兵起,三吴扰乱,生父挈家南归临安,展转会稽、四明以避乱;女家亦北徙金陵。音耗不通者十载。吴元年,国朝混一,道路始通。时生父已殁,独奉母居钱塘故址,遣旧使老苍头往金陵物色之,则女以甲辰年适太原王氏,有子矣。苍头回报,生虽怅然绝望,然终欲一致款曲于女,以导达其情,遂市剪彩花二盝,紫绵脂百饼,遣苍头赍往遗之。恨其负约,不复致书,但以苍头己意,托交亲之故,求一见以觇其情。王氏亦金陵巨室,开彩帛铺于市,适女垂帘独立,见苍头趦趄于门,遽呼之曰:“得非商兄家旧人耶?”即命之入,询问动静,颜色惨怛。苍头以二物进,女怪其无书,具述生意以告。女吁嗟抑塞,不能致辞,以酒馔待之。约其明日再来叙话。苍头如命而往。女剪乌丝襴,修简遗生曰:

伏承来使,具述前因。天不成全,事多间阻。盖自前朝失政,列郡受兵,大伤小亡,弱肉强食,荐遭祸乱,十载于此。偶获生存,一身非故,东西奔窜,左右逃逋;祖母辞堂,先君捐馆;避终风之狂暴,虑行露之沾濡。欲终守前盟,则鳞鸿永绝;欲径行小谅,则沟渎莫知。不幸委身从人,延命度日。顾伶俜之弱质,值屯蹇之衰年,往往对景关情,逢时起恨。虽应酬之际,勉为笑欢;而岑寂之中,不胜伤感。追思旧事,如在昨朝。华翰铭心,佳音属耳。半衾未暖,幽梦难通,一枕才欹,惊魂又散。视容光之减旧,知憔悴之因郎;怅后会之无由,叹今生之虚度!岂意高明不弃,抚念过深,加沛泽以滂施,回余光以返照,采葑菲之下体,记萝茑之微踪;复致耀首之华,膏唇之饰,衰容顿改,厚惠何施!虽荷恩私,愈增惭愧!而况迩来形销体削,食减心烦,知来日之无多,念此身之如寄。兄若见之,亦当贱恶而弃去,尚何矜恤之有焉!倘恩情未尽,当结伉俪于来生,续婚姻于后世耳!临楮呜咽,悲不能禁。复制五十六字,上渎清览,苟或察其辞而恕其意,使箧扇怀恩,绨袍恋德,则虽死之日,犹生之年也。诗云:

好姻缘是恶姻缘,只怨干戈不怨天。

两世玉箫犹再合,何时金镜得重圆?

彩鸾舞后肠空断,青雀飞来信不传。

安得神灵如倩女?芳魂容易到君边!

生得书,虽无复致望,犹和其韵以自遣云:

秋香亭上旧姻缘,长记中秋半夜天。

鸳枕沁红妆泪湿,凤衫凝碧睡花圆。

断弦无复鸾胶续,旧盒空劳蝶使传。

惟有当时端正月,清光能照两人边。

并其书藏巾笥中,每一览之,辄寝食俱废者累日,盖终不能忘情焉耳。生之友山阳瞿佑备知其详,既以理谕之,复制《满庭芳》一阕,以著其事。词曰:

月老难凭,星期易阻,御沟红叶堪烧。辛勤种玉,拟弄凤凰箫。可惜国香无主,零落尽露蕊烟条。寻春晚,绿阴青子,鶗鴂已无聊。蓝桥虽不远,世无磨勒,谁盗红绡?怅欢踪永隔,离恨难消!回首秋香亭上,双桂老,落叶飘颻。相思债,还他未了,肠断可怜宵!

仍记其始末,以附于古今传奇之后,使多情者览之,则章台柳折,佳人之恨无穷;仗义者闻之,则茅山药成,侠士之心有在。又安知其终如此而已也!

\chapter{寄梅记}

朱端朝,字廷之,宋南渡后,肄业上庠,与妓女马琼琼者善,久之,情爱稠密。端朝文华富赡,琼琼识其非白屋久居之人,遂倾心焉,凡百资用,皆悉力给之。屡以终身为托。端朝虽口从,而心不之许,盖以其妻性严,非薄幸也。值秋试,端朝获捷,琼琼喜而劳之。端朝乃益淬励,省业春闱,揭报果复中优等。及对策,失之太激,遂置下甲。初注授南昌尉。琼琼力致恳曰:“妾风尘卑贱,荷君不弃。今幸荣登仕版,行将云泥隔绝,无复奉承枕席。妾之一身,终沦溺矣!诚可怜悯!欲望君与谋脱籍,永执箕帚。虽君内政谨严,妾当委曲遵奉,无敢唐突。万一脱此业缘,受赐于君,实非浅浅。且妾之箱箧稍充,若与力图,去籍犹不甚难。”端朝曰:“去籍之谋固易,但恐不能使家人无妒。吾计之亦久矣。盛意既浓,沮之则近无情,从之则虞有辱,奈何!然既出汝心,当徐为调护,使其柔顺,庶得相安,否则计无所措也。”一夕,端朝因间,谓其妻曰:“我久居学舍,虽近得一官,家贫,急于干禄,岂得待数年之阙?且所得官,实出妓子马琼琼之赐。今彼欲倾箱箧,求托于我。彼亦小心,能迎合人意,诚能脱彼于风尘,亦仁人之恩也。”其妻曰:“君意既决,亦复何辞。”端朝喜谓琼琼曰:“初畏不从,吾试叩之,乃忻然相许。”端朝于是宛转求脱,而琼琼花籍亦得除去,遂运橐与端朝俱归。既至,妻妾怡然。端朝得琼琼之所携,家遂稍丰。因辟一区,为二阁,以东、西名,东阁以居其妻,令琼琼处于西阁。阙期既满,迓吏前至。端朝以路远俸薄,不欲携累,乃单骑赴任。将行,置酒相别,因瞩曰:“凡有家信,二阁合书一缄,吾覆亦如之。”

端朝既至南昌,半载方得家人消息,而止东阁一书。端朝亦不介意。既栽覆,西阁亦不及见,索之,颇遭忌嫉,乃密遣一仆,厚给裹足,授以书,嘱之曰:“勿令孺人知之。”书至,端朝发阅,无一宇。乃所画梅雪扇面而已。反复观玩,后写一《减字木兰花》词云:

雪梅妒色,雪把梅花相抑勒。梅性温柔,雪压梅花怎起头?芳心欲破,全仗东君来作主。传语东君,早与梅花作主人。

端朝自是坐卧不安,日夜思欲休官。盖以侥幸一官,皆琼琼之力,不忘本也。寻竟托疾弃归。既至家,妻妾出迎,怪其未及尽考,忽作归计,叩之不答。既而设酒,会二阁而言曰:”我羁縻千里,所望家人和顺,使我少安。昨见西阁所寄梅扇词,读之使人不遑寝食,吾安得不归哉!”东阁乃曰:“君今已仕,试与判此孰是。”端朝曰:“此非口舌可尽,可取纸笔书之。”遂作《浣溪沙》一阕云:

梅正开时雪正狂,两般幽韵孰优长?且宜持酒细端详。

梅比雪花输一白,雪如梅蕊少些香,无公非是不思量。

自后二阁欢会如初,而端朝亦不复仕矣。

\end{document}