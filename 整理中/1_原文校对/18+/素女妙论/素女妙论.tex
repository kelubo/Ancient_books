% 素女妙论
% 素女妙论.tex

\documentclass[12pt,UTF8]{ctexbook}

% 设置纸张信息。
\usepackage[a4paper,twoside]{geometry}
\geometry{
	left=25mm,
	right=25mm,
	bottom=25.4mm,
	bindingoffset=10mm
}

% 设置字体,并解决显示难检字问题。
\xeCJKsetup{AutoFallBack=true}
\setCJKmainfont{SimSun}[BoldFont=SimHei, ItalicFont=KaiTi, FallBack=SimSun-ExtB]

% 目录 chapter 级别加点(.)。
\usepackage{titletoc}
\titlecontents{chapter}[0pt]{\vspace{3mm}\bf\addvspace{2pt}\filright}{\contentspush{\thecontentslabel\hspace{0.8em}}}{}{\titlerule*[8pt]{.}\contentspage}

% 设置 part 和 chapter 标题格式。
\ctexset{
	part/name= {第,卷},
	part/number={\chinese{part}},
	chapter/name={第,篇},
	chapter/number={\chinese{chapter}}
}

% 设置古文原文格式。
\newenvironment{yuanwen}{\bfseries\zihao{4}}

% 设置署名格式。
\newenvironment{shuming}{\hfill\bfseries\zihao{4}}

% 注脚每页重新编号,避免编号过大。
\usepackage[perpage]{footmisc}

\title{\heiti\zihao{0} 素女妙论}
\author{}
\date{}

\begin{document}

\maketitle
\tableofcontents

\frontmatter
\chapter{前言、序言}

《素女妙论》,中医性学书籍,此书包括了一些摘取自《素女经》 、《洞玄子》的文字,然后重写及重新编排,组成连贯性的文字讨论,在其中空白不连贯的地方加上了编纂者的意见和看法。

中文名
素女妙论
类    型
中医性学书籍

包    括
《素女经》 、《洞玄子》
含    义
黄帝和素女的对话写成等

目录

1图书简介
2评价
3片段赏析

图书简介
播报
编辑
全书以黄帝和素女的对话写成,似乎是一本男子实用性指南。此书虽然还保留了过去一些性书中还精及用性作治疗的内容,但是却没有吸精及有关道教邪术的叙述。
评价
播报
编辑
高罗佩认为,此书是保留至今的明代性书中最真挚、最没有歪理的,它老老实实地研究了还精及性治疗,值得全面介绍。

\mainmatter

% 增加空行
~\\

% 增加字间间隔,适用于三字经、诗文等。
 \qquad  

\chapter{1}
\section{1}
\section{2}


《素女妙论》-原始篇
在昔,轩辕问素女曰:“朕闻上古圣人,寿有千岁,或八百岁,又有二百岁;中古圣人有百二十岁;今时人寿或三十或二十而亡,又五七岁而亡,又二三岁而亡,安逸者少,抱疾者多矣。其故何哉?幸谕开悟,愿勿吝其要。”
素女答曰:“凡人之生,感父精母血而受始,合地水火风而成形。盖寿夭之际,其因不一。有二三岁、五七岁至十二三岁而亡者,皆由父母受胎而无禁忌,故生子不寿。或二三十岁而亡者,其人四大本虚,初无坚固之质,不能学养生之术;年及少壮,血气方刚,而迷恋欲情,使精气耗散,疾病多生,而不识治疗之方,是乃自丧其本源,岂得望延年益寿乎!”
帝问曰:“太级剖判,阴阳肇分,轻清为天,混浊为地,乾道成男,坤道成女,惟人处乎其中,万物生焉,何者而无阴阳矣哉!天无阴阳则日月不明,地无阴阳则万物不生,人无阴阳则伦道绝矣。阴阳交感,不可一日而无焉也。卿之斯言,朕有未悟,男女交合之要、疾病治疗之方,幸望备道其详,以济人寿。”
素女答曰:“甚哉甚哉!凡男女交合,乃一阴一阳之道也。是以阴中有阳,阳中有阴,阴阳男女,天地之道也。然失其要,则疾病起矣。”又曰:“抱阴而负阳,阳极则阴生,阴极则阳萌矣。凡女子阴中自具阴阳,其间刚健柔顺,各有快美之趣。”
帝问曰:“夫妇交感之道,既已闻之,未达其旨。而夫妇交媾之时,亲之不得其美情,愿为是说。”
素女答曰:“凡男女交合之道及补精采气之法、按摩导引之义,返本还元,深根固蒂,得其长久之情。若非采阴之法,徒劳交合,不得其畅美,终为杏冥,而不能通美快之意,此因人不能慕其道也。若行此法,实为养生之秘要也。凡男女交合,其女人阴中自有美快之秘,而知其趣者少焉矣。故只多感其情,遂以致两情不乐,虚劳交合,而不美快。且夫女子精液未发,而阴中干,若男子勉强行之,玉茎钻制空亏,只劳神思而无适用也。或女子欲火已动,男子玉茎不刚坚,精津离形,而意未舒畅,女子心中不快不满,终生憎恶之心。”
《素女妙论》-九势篇
帝问曰:“男女者,人道大欲,而万物化生之源也。而今忽之者,未得其要之故也乎?”
素女答曰:“其言慎微,愚者以为亵,而非诲淫导欲之说,实乃养生之妙术。交媾之秘诀,其法有九,名具以提之:
一曰龙飞势;
二曰虎步势;
三曰猿搏势;
四曰蝉附势;
五曰龟腾势;
六曰凤翔势;
七曰兔吮势;
八曰鱼唼势;
九曰鹤交势。”
帝问曰:“九势已闻其目,而行之有法哉?”
素女答曰:“每一势有一法,只拟其物状而为势,故目云曰‘九势’。
“一、龙飞势。令女人仰卧其体,两足朝天。男子伏其上,据其股,含其舌。女人自举起牝户,而受玉茎刺入玄牝之门,抽出扣其户,举身动摇,行八深六浅之法,则阴中壮热,阳物刚强,男欢女悦,两情娱快,百疾消除。其法如蛟龙发蛰攀云之状。
“二、虎步势。令女人胡跪低头。男子踞其后,抱其腰而插入玉茎于牝门,行五浅六深之法,抽出百回,玉钳开张,精涎涌出,水火既济,尽丹鼎之妙。烦懑已除,血脉流通,补心益志。其法如虎豹出林啸风之状。
“三、猿捕势。令女人开起两股,坐在男子于两腿上,牝门开张滑滑。插玉槌,数扣yin户,次行九浅五深之法,女子哝哝不休,津液溢流,男子固济阳匮而不泄。百疴忽除,益气长生,不饥。其法如猿搏枝取果实之状,最以快捷为妙。
“四、蝉附势。令女人直舒左股而屈右股。男子踞其后,曳玉如意叩其赤珠,行七深八浅之法,红球大张,快活泼泼,极活动之妙。通利关脉,久久利人。其法如金蝉抱树吸露清吟之状,只含蓄不吐。
“五、龟腾势。令女人仰卧,澹然虚无,如忘其情。男子以两手指托两腿,抬起过乳房,伸出其头,忽入红门,深撞谷实。忽缩忽伸,如龟头伸缩。能除留热,遂五脏邪气。其法如玄龟游腾之状,坚甲自守,曳尾泥中,而全其真。
“六、凤翔势。令女人横身仰床上,手自举两股。男子以两手紧抱搂女腰,将金槌插玉门,左右奔突,至阴中壮热,女体软动,行九浅八深之法,则女悦微喘,滑液沸出。能补诸虚,填精髓,轻身,延年不老。其法如丹山瑞凤搏扶摇而翱翔寰中之状。
“七、兔吮势。先男子仰卧床上,直伸两股。令女人反骑跨男子股上,手握郎中探房门,直穿琴弦,觉玉条坚硬而后行浅深之法,则养血行气,除四股酸疼。其法如玉兔跳跃之快,忽蹲忽跳,出没不定,只不失其真,则能捉蟾魄于九霄。
“八、鱼唼势。令二女子一仰一俯,互搂抱以为交接之状。牝户相合,自摩擦则其鱼口自开,犹游鱼唼萍之形。男子箕坐其侧,俟红潮嘲发,先以手探两口相合处,将茎安其中间,上下乘便,插入两方交欢。大坚筋骨,倍气力,温中补五劳七伤。其法如游鱼戏藻之状,只以唼清吐浊为要。
“九、鹤交势。令女人搂男子之颈,以右足负床上。男以右手提女之左股而提肩上,两体紧巾,微抽玉茎,窥其菱齿,徐徐撞谷实,摇摆轻漫,行九浅一深之法,花心忽开,芳液浸润。保中守神,消食开胃,疗百病,长生不饥。其法如丹鹤回旋之状,张翎不收,自至妙境。
《素女妙论》-浅深篇
帝问曰:“火候浅深,炼丹之要旨也。然调停不得法,各有所捐哉?”
素女答曰:“浅而不足者戾意,深而大过者懊人。又有三十六种、七十二般之法,能合甜情、益快意,然其理深邃。”
帝问曰:“男女交媾之道,行浅深之法则多损伤,而补益者少焉矣。尝闻有采补秘奥以济人寿,愿示其详。”
素女答曰:“男子须察女人情态,亦要固守自身之宝物,勿令轻漏泄。先将两手掌摩热,坚把握玉茎;次用浅抽深入之法,耐久战,益美快。不可太急,不可太慢,又勿尽意深入,深则有所损焉。刺之琴弦,攻其菱齿,若至其美快之极,女子不觉噤齿,香汗喘吁,目合面热,芳蕊大开,滑液溢流,此快科学研究之极也。又女子阴中有八名,又名‘八谷’:
一曰琴弦,其深一寸;
二曰菱齿,其深二寸;
三曰妥豀,其深三寸;
四曰玄珠,其深四寸;
五曰谷实,其深五寸;
六曰愈阙,其深六寸;
七曰昆户,其深七寸;
八曰北极,其深八寸。”
帝问曰:“交合所伤,亦生何病?”
素女答曰:“交会之要,切忌太深,深则伤于五脏。若至谷实则伤肝,其病眼昏眵泪,四肢不遂;至愈阙则伤肺,其病恶心哕逆,痰喘昏晕;至昆户则伤脾,面黄腹胀,烦懑冷痢;至北极则伤肾,腰脚痿软,骨蒸潮热;忽浅忽深则伤心,其人面热虚嗽,梦魇遗精。所以交合不可太深。女子丹穴在脐下三寸,勿令伤之。又不可太速,不可太慢,太速则伤血,太慢则损气,并有损而无益焉矣。”
帝问曰:“火候调度浅深之要既审领之,损害之理亦不可忽之。尚有禁忌之功者,冀无吝之。”
素女答曰:“炼丹避忌,若误犯之,大者天地夺其寿算,鬼神殃其身,三彭窥其隙,抱疾罹厄;其生儿夭促,不慧不肖,或顽劣凶恶,遗害于父母。可不谨哉!夫天地晦冥震动之时,迅雷烈风暴雨之日,及晦朔弦望、大寒酷暑、日月薄蚀、神圣诞辰、庚申甲、自己本命之辰、三元八节、五月五日、月煞月被、披麻红杀,皆不可犯焉。又天地五岳川渎祠坛之近侧、神圣祠宇,及诸神鬼像前、井灶溜厕之傍,各有害,多令人夭亡或生怪形奇状之子也,交媾之际,亦有避忌。大饥勿犯,大饱勿犯,大醉勿犯,神劳力倦勿犯,忧愁悲恐勿犯,病新瘥勿犯,丧服勿犯,女子经中勿犯。”
《素女妙论》-五欲五伤篇
帝问曰:“少壮努力,则有衰败之患。欢乐之极,必多哀伤。至人节之以道,故曰御百女寿比天地。而今人不至半百,筋痿肉脱,火盛水枯,终为败物者,何乎?”
素女答曰:“凡人,有五欲、五伤、十动之候,若得其宜,则意满欲足;不得其宜,则各有所伤。
“五欲之候:
“第一,面上潮红者,其意有所思欲,先刺入玉茎,徐徐摇动,慢慢抽出,多在户外,探其情;
第二,鼻孔吐气者,欲火微动,先以玉茎穿阴户,刺其谷实,不可太深,宜俟火候之至;
第三,咽喉干嗄者,情动火炽。抽玉茎,俟其眼闭舌吐、喘气为声,出入任意,渐至佳境。
第四,红球浸润者,心火大盛,刺之则滑泽外溢,轻及菱齿,一左一右,一缓一急,随便如法;
第五,金莲擎抱者,火候既足焉,必以足缠腰上,两手搂肩背,舌吐不缩,宜刺愈阙,其时四肢通快。
“五伤之候:
“第一,阴户尚闭未开者,不可刺之,刺之伤肺,肺伤则痰喘声喝;
第二,情兴已至,金茎软痿,其兴过而后渐交者,伤心,心伤则经水不调;
第三,以少阴合老阳,欲火空燃,而不得所欲者,伤肝,肝伤则心眩目昏;
第四,欲足情满,阳兴未休者,伤肾,肾伤则带下崩漏;
第五,月厄未尽遇逼合者,伤脾,脾伤则,颜色痿黄。
“十动之候:
“一、玉手抱男背,下体自动,援吐舌相偎者,令男子动情,兴之候也;
二、芳体仰卧,直伸手足而不动,鼻中微发喘急者,欲刺抽之候也;
三、伸腕开掌,握睡汉之玉槌而动转者,垂涎之候也;
四、言语戏喋,眼来眉去,时发懊?之声者,春情极到之候也;
五、自以两手抱金莲,露张玄牝之门者,情熟意快之候也;
六、口含玉如意,如醉如睡,阴中隐痒者,俗浅深奔突之候也;
七、长伸金莲,勾挽玉槌,如进如退,低发呻吟者,阴潮涌来之候也;
八、忽得所欲,而微微转腰,香汗未彻,时带笑容者,恐阳气发泄兴情已尽之候也;
九、甜情已到,美快渐多,精液发泄,尚抱搂紧紧者,意未满之候也;
十、身热汗洽,足缓手慢者,情极愿足之候也。”
《素女妙论》-大伦篇
帝问曰:“人之大伦,有夫妇而后有子孙。妇德妇貌,不可不撰乎?”
素女答曰:“妇德,内美也;妇貌,外美也。先相其皮,而后相内。若妇人发焦黑,骨大肉粗,肥瘦失度,长短非常,年岁不合者,子孙不育。言语雄壮,兴动暴忽,阴内干涩,子宫不暖,及淋露赤白浊沥胡臭者,大损阳气。”
帝问曰:“损伤阳气之说已闻之。或以药饵为补导者,如何?”
素女答曰:“男女交合,非为淫乐也。今时之人,不晓修养,勉强临事,故多损精败气,疾病依生焉;或误饮食服饵而损性丧命,良可哀哉!”
帝问曰:“夫妇之道,为子孙之计,而今无子者,何乎?”
素女答曰:“三妇无子,三男无子。男子精冷滑者、多淫虚惫者、临敌畏缩者、无子也;妇人性淫见物动情者、子藏虚寒藏门不开者、夫妇不如妒忌火壮者,无子也。”
帝问曰:“若人无子,取之以何术乎?”
素女答曰:“求子之法,按阴合阳合之数,用黄纱黄绫黄绢之属,造衣被帐褥之类,以黄道吉日,取桃枝书年庚,放之卧内。又九月三日,取东引桃枝书姓名,插之床上,须察妇人月经已止过三四日,各沐浴炷香,祈天地鬼神,入帐中而为交合。其时子宫未合闭,故有子也。御法:进退如法,洗心涤虑,勿戏调戏弄,勿借春药,勿见春宫册。若犯之,损父母,不利生子。”
帝问曰:“阴阳之道,名之为交接者,何乎?”
素女答曰:“阴阳交合,男施女接,故名之为交接也。女人阴中自有明兆焉,无刺琴弦而及菱齿,美快之极,放露真宝,阴血包阳精则生男,阳精包阴血,则生女。谓之阴阳交接之道矣。”
帝问曰:“交接,人伦之原也。而有不相和悦者,何故也乎?”
素女答曰:“盖因女子不能察丈夫之意,男子亦不晓妇人之性,此不达人伦之道、生育继嗣之理也。各顽劣多淫,各怀不足,互填愤怨;或弃自己妻妄而通外妇,又欺丈夫而野合奸淫;又男子痿软不满欲情,或强阳悍无休息,后终生厌恶。”
帝问曰:“夫妻亲睦相敬爱者,人伦之常也。而敬爱之情,因何乎生焉?”
素女答曰:“既济者,顺也;未济者,逆也。八庚相合,少壮应时者,顺也;八字不协,老幼不遇者,逆也。才貌两全,意气相合者,顺也;蠢丑相背,狠戾反目者,逆也。但恩受契合则生敬恭,敬恭则富贵长命而子孙蕃育。”
《素女妙论》-大小长短篇
帝问曰:“男子宝物,有大小长短硬软之别者,何也?”
素女答曰:“赋形不同,各如人面。其大小长短硬软之别,共在禀赋,故人短而物雄,人壮而物短,瘦弱而肥硬,胖大而软缩。或有专车者,有抱负者,有肉怒筋胀者,而无害交会之要也。”
帝问曰:“郎中有大小长短硬软之不同,而取交接快美之道,亦不同乎?”
素女答曰:“赋形不同,大小长短异形者,外观也;取交接快美者,内情也。先以爱敬系之,以真情按之,何论大小长短哉!”
帝问曰:“硬软亦有别乎?”
素女答曰:“长大而萎软,不及短小而坚硬也;坚硬而粗暴,不如软弱而温藉也。能得中庸者,可谓尽美尽善焉矣。”
帝问曰:“方外之士,能用药物,短小者令其长大,软弱者令其坚硬,恐遗后患乎?将有补导之益乎?”
素女答曰:“两情相合,气运贯通,则短小者自长大,软弱者自坚硬也。有道之士能之,故御百女而不痿。得修养之术,则以阴助阳,呼吸吐纳,借水救火,固济真宝,终夜不泄,久久行之,则益寿除疾。若用五石壮阳之药,腽肭增火之剂,虚炎独烧,真阳涸渴,其害不少。”
帝问曰:“有修养之术者,亦不禁乎?”
素女答曰:“气运巡环,临事而合,应时而止,只量力而施,其余勉强迷惑,则修养之士亦至枯败焉。服药三朝,不如独宿一宵,前哲之诫也。”
《素女妙论》-养生篇
帝问曰:“养生之道,以何为本?”
素女答曰:“养生之道,以气为本。气能运血,血能化精,精能养神,神在则生,神散则死也。气者,神之本也。能炼气者,入火不焦,入水不溺。固守其精而不散,故终夜御女而不泄。若不能保守精神,而狂妄任意者,必失神丧气,名之为夺命之斧。”
帝问曰:“若人专守养生之道而不行夫妇房帏之礼,则人伦已绝,后继将断。”
素女答曰:“凡人年少之时,血气未充足,戒之在色,不可过欲暴泄。年已及壮,精气满溢,固精厌欲,则生奇病。故不可不泄,不可太过,亦不可不及。”帝问曰:“时泄而遣其兴,能其精而养神乎?”
素女答曰:“不然也。若常泄而偶不漏,反生疮痈;常秘而偶泄,则患暴虚。各害养生之道。”
帝问曰:“男子精血盈满,神气充足,何以知之乎?”
素女答曰:“男子二八天癸至,而血气不足,精神未定,故戒之也。年至二十,血气渐盛,而精聚肠胃,三十日而一泄焉。三十而血气壮盛,而精在两股,五日一泄焉。年四十,精聚要脊,七日一泄焉。五十而血气将衰,精聚背膂,半月一泄。年至六十四岁,天癸尽,卦数满,血气衰,精液竭矣。六十已上,能保全余气,兴壮者尚可泄。至七十,不可妄思欲动情。”
帝问曰:“朊知无赖之子,自赖强壮,一日三泄或五泄者,何乎?”
素女答曰:“暴泄者暴虚,后必痿辟。若泄而不休,自招天亡。”
帝问曰:“人阳气夜间勃然起立,腾然兴发者,何?”
素女答曰:“晨昼暮夜,此一日中之四时也。故阳气生子时,于卦为复;至丑时而二阳生下,于卦为临;寅时三阳已全,于卦为泰。若人半夜暴泄,则阳精枯损,年未五十,必发头晕腑痛目昏耳塞。又有五伤:
其一,男女交会,精泄而少者,为气伤;
其二,精出而者,为肉伤;
其三,泄而疼者,为筋伤;
其四,精出而涩者,为骨作;
其五,临门忽痿垂涎者,为血伤。各泄精过度、精液竭乏所致,何可不谨哉!”
《素女妙论》-四至九到篇
帝问曰:“男女好述,未发言语而知其情。机微慎密,以何术恻之趣之?”
素女答曰:“凡男子欲探女子私情,先以言语戏谐挑其意,以乎足扭捏趣其情。男子有四至,女子有九到,若四不至九不到而交俣者,必有后患。”
帝问曰:“男子四至者,若何?”
素女答曰:“玉茎不强者,阳气未至也;刚强而不动者,肌气未至也;振摇而不怒者,骨气未至也;怒张而不久者,肾气未至也。若一不至而犯之,必有损伤。”
帝问曰:“女人九到者,若何?”
素女答曰:“倦伸欠息,而睡觉朦胧,肺气未到也;门户不润,屈股不开,心气未到也;目不流视,举止不忻,脾气未到也;手扪玉茎而情意不悦,血气未到也;手软足缓,横卧不动,筋气未到也;抚弄两乳,意向无味,骨气未到也;瞬波微动,莺口不开,肝气未到也;举身向人,桃颊不红,肾气未到也;玉关仅润,口中不渴,液气未到也。九候已到而后行九浅一深之法,则阴阳调和,情思缠绵。能助阳气,补虚之劳损。
帝问曰:“何为九浅一深之法?”
素女答曰:“浅插九回,深刺一回,每一回以呼吸定息为度,谓之九浅一深之法也。自琴弦至玄珠为浅,自妥至谷实为深。凡太浅不美快,太深有所伤。”
帝问曰:“丹鼎调度,火候慎微,水火既济之妙既详闻之,尚有余蕴,愿尽其理,博施救世之仁,令万世无夭亡之患,亦无虚羸绝嗣之忧。”
素女答曰:“天地交泰,阴阳会施,先察其情兴,审辨其气候到不到,极抽出插入添炭之妙,固济自己阳匮,香吻相偎,吸阴精而补阳气,引鼻气以填脊髓,含津液以养丹田,令泥丸势气透切,贯通四支,溢益气血,驻颜不老。”
帝问曰:“采补修养,炼内丹第一妙义也。含灵之者,不可不达其理焉?”
素女答曰:“然矣,如帝命。此延龄益寿之妙要也。夫天不足西北,故男子阳气有余、阴血不足;地不满东南,故女子阴血充实、阳气不足。能达玄微者,以有余补不足,虽至期颐,不改其乐。快活娱乐,无穷极之时,长生久视,寿俦天地。宜录之金石,长传后蕊,则普济德泽,亦不少也矣。”
帝斋戒沐浴,以其法炼内丹八十一日,寿至一百二十岁。而丹药已成,铸鼎于湖边,神龙迎降,共素女白日升天。

\backmatter

\end{document}