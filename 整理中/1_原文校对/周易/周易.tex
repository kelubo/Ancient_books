% 周易
% 周易.tex

\documentclass[12pt,UTF8]{ctexbook}

% 设置纸张信息。
\usepackage[a4paper,twoside]{geometry}
\geometry{
	left=25mm,
	right=25mm,
	bottom=25.4mm,
	bindingoffset=10mm
}

% 目录 chapter 级别加点(.)。
\usepackage{titletoc}
\titlecontents{chapter}[0pt]{\vspace{3mm}\bf\addvspace{2pt}\filright}{\contentspush{\thecontentslabel\hspace{0.8em}}}{}{\titlerule*[8pt]{.}\contentspage}

% 设置 part 和 chapter 标题格式。
\ctexset{
	part/name={},
	part/number={},
	chapter/name={},
	chapter/number={\chinese{chapter}}
}

% 设置古文原文格式。
\newenvironment{yuanwen}{\bfseries\zihao{4}}

% 注脚每页重新编号,避免编号过大。
\usepackage[perpage]{footmisc}

\usepackage{tikz-bagua}

%\usepackage{tikz}

\title{\heiti\zihao{0} 周易}
\author{伏羲、姬昌、孔子}
\date{}

\begin{document}

\maketitle
\tableofcontents

\frontmatter
\chapter{前言}

《周易》即包括《易经》和《易传》,“三易”之一\footnote{另有观点认为易经即三易,而非周易。},是传统经典之一。约形成于西周初年,原本是筮\footnote{sh\`{i}}占的工具书。在我国被尊为“六经之首,大道之源”。

《汉书·艺文志》中记载《周易》的形成是“人更三圣,世历三古”。相传,上古伏羲,留天地之象;中古周文王,演《易经》之道;近古孔子及弟子后学,注《易经》而成《易传》。经传合一,终成《周易》。

《经》主要是六十四卦和三百八十四爻,卦和爻各有说明(卦辞、爻辞),作为占卜之用。《传》包含解释卦辞和爻辞的七种文辞共十篇,统称《十翼》,相传为孔子所撰。

春秋时期,官学开始逐渐演变为民间私学。易学前后相因,递变发展,百家之学兴,易学乃随之发生分化。自孔子赞易以后,《周易》被儒门奉为儒门圣典,六经之首。儒门之外,有两支易学与儒门易并列发展:一为旧势力仍存在的筮术易;另一为老子的道家易,易学开始分为三支。

《四库全书总目》将易学历史的源流变迁,分为“两派六宗”。两派,就是象数学派和义理学派。六宗,一为占卜宗,二为禨祥宗,三为造化宗,四为老庄宗,五为儒理宗,六为史事宗。

《周易》是中国传统思想文化中自然哲学与人文实践的理论根源,是古代汉民族思想、智慧的结晶,被誉为“大道之源”。内容极其丰富,对中国几千年来的政治、经济、文化等各个领域都产生了极其深刻的影响。
《易经》为群经之首,设教之书。

\section{成书背景}

中国早期社会由于生产力低下,科学不发达,先民们对于自然现象、社会现象,以及人自身的生理现象不能作出科学的解释,因而产生了对神的崇拜,认为在事物背后有一个至高无上的神的存在,支配着世间一切。当人们屡遭天灾人祸,就萌发出借助神意预知突如其来的横祸和自己的行为所带来的后果的欲望,以达到趋利避害的目的。在长期的实践中发明了种种沟通人神的预测方法,其中最能体现神意的《周易》就是在这种条件下产生的。

\section{成书时间}

成于西周

顾颉刚关于《周易》卦爻辞制作年代的考证结论,认为周初作。李学勤也认为顾颉刚此文“推定经文卦爻辞‘著作年代当在西周初叶’”,“为学者所遵信,可以说基本确定了《周易》卦爻辞年代的范围,是极有贡献的”。

成于战国

大多学者认为《易传》成书于战国。易传的成书问题,自欧阳修《易童子问》与苏轼之后,数百年来已经罕有人再信是孔子所做了。

钱穆、顾颉刚、冯友兰、郭沫若、李镜池等等海内外的学者名家均断定司马迁《史记》的说法不足信。其称易传为孔子所做,若非司马迁之误,就必是汉儒刘歆所伪窜。根据《易传》的内容来看,应是在孟子、荀子的性命天道之学出现以后的作品,有明显的黄老道家与阴阳家色彩。

成于西汉

顾颉刚根据箕子和康侯两条卦爻辞,指出《周易》卦爻辞为文王所作的传统说法不可信。顾颉刚继《周易卦爻辞中的故事》之后作《论易系辞传中观象制器的故事》,是对其所说《周易》卦爻辞所无的观象制器故事的专门考论,意在证明《系辞传》观象制器章讲到古史帝系人物的话是西汉后期人的说法。

关于《系辞传》与《世本》的关系,胡适指出,“《世本》所据传说,必有一部分是很古的,但《世本》是很晚的书,《系辞》不会在其后”,“《世本》不采《系辞》,也许是因为《系辞》所说制作器物太略了,不够过瘾。《系辞》那一章所说,只重在制器尚象,并不重在假造古帝王之名。若其时已有苍颉沮诵作书契之传说,又何必不引用而仅泛称‘后世圣人’呢?”

\section{周易作者}

《汉书·艺文志》在描述《周易》的成书过程时,称“人更三圣,世历三古”。三圣,即伏羲,文王和孔子。

否定孔子作《易经》的大学问家很多,如欧阳修、龚自珍。当代学者钱玄同精心考据后认为“孔丘无删或制作‘六经’之事。”鲁迅在《汉文学史纲要》中断言:“谁为作者殊难确指,归功一圣亦凭臆之说”。清儒家学者几乎一边倒认为“孔子以前不得有经”,康有为则认为:“凡‘六经’皆孔子所作”。

根据现代学界研究,一般认为,《周易》(64卦及卦爻辞)为周初周人所作,重卦出自文王之手,卦爻辞为周公所作。

\section{名称由来}

《周易》又称《易经》,分为经部和传部,经部之原名就为《周易》,是对四百五十卦易卦典型象义的揭示和相应吉凶的判断,而传部含《文言》、《彖传》上下、《象传》上下、《系辞传》上下、《说卦传》、《序卦传》、《杂卦传》,共七种十篇,称之为“十翼”,是孔门弟子对《周易》经文的注解和对筮占原理、功用等方面的论述。

“周易”一词的解释,众说纷纭,但归纳起来主要有二种:一种认为《周易》是周代占筮之书;一种认为《周易》是讲变化的书,英文中TheBookofChanges的译文就是取此义。

春秋时,就有《周易》提法,在《春秋左传》这部史书当中,多次提到“周易”,但从当时人们运用的《周易》看,只包括六十四卦的卦画(符号)卦爻辞。

战国时,以解释《周易》为宗旨的《易传》成书。《周易》和《易传》并称为《易》,如《庄子》所谓“易以道阴阳”、《荀子》所谓“善为易者不占”之“易”包含了《易传》。

西汉以降,汉武帝为了加强中央集权制,采纳了董仲舒“独尊儒术”的建议,把孔子儒家的著作称为“经”。《周易》和《易传》被称为《易经》,或直接称为《易》。自此以后,《周易》、《易经》、《易》混合使用,有称《周易》,有称《易经》,有称《易》,其实含义一致,均指六十四卦及《易传》,一直沿用至今,仍然没有严格区分。有的学者为了区分《周易》经传之不同,称六十四卦及卦爻辞为《周易古经》,称注释《周易古经》的十篇著作(易传)为《周易大传》。

\section{“周”之涵义}

具体地说,“周”字,有二义:
①:指周普、普遍,即易道广大,无所不包。东汉郑玄《易论》,认为“周”是“周普”的意思,即无所不备,周而复始。 [12]
②:指代号,即周朝,古代常称周朝的书为周书,如《周礼》、《周语》等。唐代孔颖达《周易正义》认为“周”是指岐阳地名,是周朝的代称。也有人认为《易经》流行于周朝故称《周易》,亦有人依据《史记》的记载“文王拘而演周易”,认同《易经》因周文王得名。 [12]

\section{“易”之涵义}

①易由蜥蜴而得名,为一象形字,此说出自许慎《说文解字》;而蜥蜴能够变色,俗称“变色龙”,所以“易”的变易义,为蜥蜴的引申义。 [13]
②理解西周之“易”,理当以西周礼乐制度的变革为条件。
③日月为易,象征阴阳。从文字学看,“易”字的构成是“日”、“月”。“易”上为“日”。
④日出为易。“干”的本义。
⑤易是占卜之名。
⑥变易、变化的意思,指天下万物是常变的,故此《周易》是教导人面对变易的书。
⑦交易,亦即阴消阳长、阳长阴消的相互变化。如一般太极图所示。有说“易”的甲骨文为象征将一器皿水(或酒)倒入另一器皿之中,以示变换、交易。
⑧易’即是“道”,恒常的真理,即使事物随着时空变幻,恒常的道不变。《系辞传》:“生生之谓易”。
⑨指古代卜筮之书的代名词。《周礼》“太卜”的记载中,有《连山》、《归藏》、《周易》三部筮书称为“三易”,故易是筮书专有名词。 [14]
东汉郑玄的著作《易论》认为“易一名而含三义:易简一也;变易二也;不易三也。”这句话总括了易的三种意思:“简易”、“变易”和“恒常不变”。即是说宇宙的事物存在状能的是:
1.顺乎自然的,表现出易和简两种性质;
2.时时在变易之中;
3.又保持一种恒常。
日月的运行表现出一种非人为的自然,这是简易;其位置、形状却又时时变化,这是变易;然而总是东方出、西方落这是“不易”。 [4]

\section{周易内容}

内容解说
《易经》由本文的“经”和解说的“传”构成。“经”由六十四个用象征符号(即卦画)的卦组成,每卦的内容包括卦画、卦名、卦辞、爻题、爻辞组成。
一:爻,卦画的基本单位为“爻”,爻分奇画与偶画,奇画由一条长的横线而成“—”,俗称“阳爻”;偶画是以两条断开的横线而成“--”,俗称“阴爻”。每一卦从最底层数起,总共有六爻,六爻以不同的奇画偶画配搭,形成八八六十四种不同的组合。按易天地人三才观,初、二爻代表地,奇画为刚偶画为柔,三、四爻代表人,奇画为义偶画为仁,五、上爻代表天,奇画为阳偶画为阴。所以用“阳爻”“阴爻”称谓奇画与偶画,是泛阴阳论的表现。 [4]
二:卦画(卦的符号),即由六条“—”“--”奇偶画爻组成。
三:卦名,顾名思义即前面卦画之名,如“坤”“复”“既济”等。
四:卦辞,在卦名后,对六爻的综合总结,如“元亨利贞”,“同人于野,亨。利涉大川,利君子贞”等。
五:爻题,即爻位名称,表示某一爻在六爻中的具体位置及奇偶画性质,六爻卦位自下而上数起,分别为初(即一)、二、三、四、五、上(即六)。“—”为九,“--”为六。如“初六”“九三”“六五”“上九”等。
六:爻辞,指单条爻的说明、描述文辞,一卦有六爻,故共有六条爻辞,如“九二,见龙在田,利见大人。”同一卦六条爻辞间相对独立、相对静止但又相互关联、相互作用,表示不同时间、阶段事物的发展状态,以构成完整的发展、发生过程即全卦六爻整体内容总结—卦辞。
七:上卦与下卦和内卦与外卦,因六十四卦最初由三爻八经卦重之演变而成:“八卦成列,像在其中矣;因而重之,爻在其中矣。”所以六爻卦亦可以分解为上半部分和下半部分,四、五、上为(上卦)或“外卦”,初、二、三为“下卦”或“内卦”。如“复”卦,上“地”下“震”,内“震”外“地”,“谦”卦为“坤”上“艮”下,内“艮”外“坤”。 [4]

周易古经
一:古经分篇
《周易》古经分为上下两篇,上篇三十卦,下篇三十四卦,共六十四卦,每一卦六爻,共三百八十四爻。
二:卦的构成
1.卦符构成
《周易》每一卦有六爻,即六个符号组成,六个符号由两部分组成,即上卦和下卦,上卦和下卦分别取八卦中的某一卦。何为八卦?八卦指乾,坤,震,巽,坎,离,艮,兑。 [4]
为了记住这八卦的符号,古人总结了顺口溜:
乾三连,坤六断。
震仰盂,艮覆碗。
离中虚,坎中满。
兑上缺,巽下断。
八卦符号两两相重,构成了《周易》六十四卦卦画,8×8=64。为了区分八卦之卦和六十四卦之卦,古人称八卦为“经卦”。称六十四卦为“别卦”。因此,任意两个经卦相重叠可以得一别卦。六十四卦是由八经卦相重而成。故从卦画看,一别卦由两经卦组成:居下部分称内卦(又称下体),另一部称外卦(又称上体)。
由于八卦相重成六十四卦,故往往用八卦卦象称呼六十四别卦。八卦最基本的象是八种自然物:
乾为天、坤为地、震为雷、巽为风、艮为山、兑为泽、坎为水、离为火。 [4]
为了记住六十四卦卦象,以两个经卦卦象称呼一别卦。如天地否即看成由天地组成的卦画称为否卦。天山遁可看成由天山组成的卦画可称为遁卦。 [15]
2.《周易》每一卦的组成
《周易》中每一卦除了卦画(符号)外,还有卦名、卦爻辞,按照先后次序,《周易》每一卦有四部分组成。
①卦画(卦的符号),即六条符号组成,如坤。 [4]
②卦名,所在卦画后面的叫卦名,如乾,乾就是卦名,卦名是对卦画最简要的说明,它是这个卦的主题。
③卦辞,在卦名后面有一段文字,这段文字叫卦辞,卦辞是对一卦六爻总的说明。
④爻辞,一卦共六爻,即由六个符号组成,每爻都有一个意思,表达这个意思的文辞叫做爻辞。一卦有六爻,故共有六条爻辞。在卦辞下,六条爻辞有“九”、“六”作为爻题,阳爻称九,阴爻称六。一卦六爻自下而上,若为阳爻依次为初九,九二,九三,九四,九五,上九;若为阴爻依次为初六,六二,六三,六四,六五,上六。 [16]
3.卦爻辞的结构
《周易》卦爻辞,一般分为两部分。一部分是取象,说明事理;另一部分是断语。
所谓取象,就是叙述一件事,或描述某一自然现象,以此说明一个道理。所谓断语,就是下结论,多用吉、凶、悔、吝等辞。《周易》卦爻辞之所以要由两部分组成,原因就是为了占问。在占问时,遇到某一卦或某一卦中的某一爻,先看卦爻辞取象部分,表示占问者处境,然后看判断结果。
卦爻辞两部分是一种因果关系,有其前因必有其后果,但是这种因果性不具有客观性、普遍性,也就是说从其前因中推不出其后果,而且有许多事是偶然发生的,不具有普遍性,而《周易》作者将这些不具有真实性,不带有普遍性的东西加以整理,作为《周易》的卦爻辞。以隐语形式普遍应用于占卜之中来预测未来。 [4]
《周易》卦爻辞并不是每一条都是有两部分组成,情况比较复杂,有时没有取象部分,直接下断语。如《恒》九二“悔亡”。或者没有断语,如《大畜》九二“舆说輹”(车子与车轴脱节,指车子坏了)。也有的断语很长,如《坤》“利牝马之贞”(此占适合乘雌马)。
《周易》卦爻辞断语常用辞及含义:
吉(善,福祥)
利(顺利,适合)
吝(很难)
厉(危险)
悔(悔恨,穷困)
咎(灾患)
凶(祸殃,大的灾难) [4]
周易卦序
一:卦序指六十四卦排列的顺序。 [17]
《周易》六十四卦的排列,有着内在的根据,按照古人说法,这种排列反映了世界产生、发展、变化的过程,以乾坤为首,象征着世界万物开始于天地阴阳,乾为阳,为天;坤为阴,为地。乾坤之后为屯、蒙,屯、蒙,象征着事物刚刚开始,处于蒙昧时期。……上经终于坎、离,坎为月,离为日,有光明之义,象征万物万事活生生地呈现出来。 [4]
下经以咸恒为始,象征天地生成万物之后,出现人、家庭、社会,咸为交感之义。指男女交感,进行婚配。恒,恒久,指夫妇白头到老。社会形成以后,充满矛盾,一直到最后为既济、未济。既济,指成功,完成。未济表示事物发展无穷无尽,没有终止。《周易》作者力图使《周易》六十四卦排列符合世界进化过程。
但是这种排列并不是唯一的。1973年在湖南长沙市东郊的马王堆汉墓中发现了写在帛上的《易经》叫帛书《易经》,帛书《易经》排列完全不同于今本《周易》,它是按照八卦相重的原则,把《周易》六十四卦分成八组,叫八宫,六十四卦分属于八宫。
二:六十四卦卦画排列的特点
唐人孔颖达曾用“二二相偶,非覆即变”来概括六十四卦卦画排列的特点。
所谓“二二相偶”,是指《周易》六十四卦两两为对,共三十二对,如乾坤为一对,屯蒙为一对,按顺序依次为对。所谓“非覆即变”,是指《周易》三十二对每一对的卦画不是颠倒,就是相反。覆,颠倒;变,相反,如(屯)倒置为(蒙),(需)倒置为(讼),这是覆。(乾)与(坤)相反,乾六爻全为阳爻,坤六爻全为阴爻,(颐)与(大过)相反,颐上下为阳爻中间四爻为阴爻,大过上下为阴爻,而中间四爻为阳爻,二者卦画完全相反,这就是变。
《周易》六十四卦三十二对,有二十八对为“覆”,有四对为“变”,即除了乾坤、颐大过、坎离、中孚小过变卦外,其它与对皆为覆卦。 [4]
爻及含义
一:爻含有三才之道。
八卦由三画组成,如乾坤三画象征着天地人,天地人即“三才”,其中下爻代表地,中爻代表人,上爻代表天。六十四卦由八卦相重而成,故六十四卦中也含有三才之道。一卦六爻,初二爻为地,三四爻为人,五上爻为天。
二:爻所处位置代表事物不同阶段
初爻:代表事物开始;二爻:代表事物崭露头角;三爻:代表事物大成;四爻:代表事物进入更高层次;五爻:代表事物成功;上爻:代表事物终极。乾卦比较典型。 [18]
三:爻所处位置代表人的身体不同的部分
初爻:代表脚趾(因脚趾在最下);二爻:代表小腿;三爻:代表腰(三爻居中,腰也居中)四爻:代表上身;五爻:代表脸;上爻:头。咸卦、艮卦比较典型。 [19]
四:爻所处的位置代表社会不同等级
按照汉人对《周易》的注释,情况如下:初爻在下,代表民;二爻居中,代表君子、卿大夫;三爻在二爻之上,代表诸侯;四爻近五,为近臣;五爻在上居中,为天子;上爻在最上,为宗庙(或太上皇)。
五:爻所处的位置代表不同性质事类
一般说来,二爻五爻居中,以示行中之道(即不偏不倚,不过无不及,古人称为大德),故多荣誉,多有功绩。也就是说,《周易》二五两爻辞多是吉利的。三爻居内卦之上,过中。故多凶险。四爻近五爻,五爻为天子,故近天子之人,多恐惧,即所谓伴君如伴虎。初爻代表事未成,上爻以示事已过。 [4]
六:爻位
爻所居的位置叫爻位。爻位有一定的规律:初为阳位,二为阴位,三为阳位,四为阴位,五为阳位,上为阴位,即奇为阳位,偶为阴位,初、三、五为阳位,二、四、上为阴位。在《周易》中,阴阳位与阴阳爻并非一一对应,即阴爻并非居阴位,阳爻亦并非居阳位。而多为阴阳杂居,如阳居阴位,阴居阳位,故《周易》中有当位、不当位(或得位、失位)问题。一般说来,阳居阳位,阴居阴位为当位。阳居阴位,阴居阳位为失位。在《周易》六十四卦中,全当位者为既济卦,全失位者为未济卦。
周易大传
《周易大传》,或称《易传》,因共十篇,又称“十翼”。传,有解说之义。在古代,凡解说、阐发经典著作意义的书和文字,皆可称为“传”。如《春秋左传》,是左丘明为《春秋》所作的注释。《诗经毛传》是毛亨为《诗经》所作的注释。翼,本指鸟虫之翅膀,此是指《周易》不可缺少的,与《周易》相辅相成的,注释解说《周易》的著作。
《易传》十篇指:《彖》上、下,《象》上、下,《文言》,《系辞》上、下,《说卦》,《序卦》,《杂卦》。 [4]

一:《易传》的贡献
《易传》是现存最早、最系统的注释《周易》的著作。它的成书对《周易》产生了很大的影响,是学习、研究《周易》的必读之书。
1.从抽象意义上对《周易》作了注释,即将《周易》六十四卦三百八十四爻上升到理论高度进行概括说明和解释。如《易传》从宇宙宏观角度探讨《周易》起源,它认为《周易》是古代圣人仰观俯察,对大自然进行模拟、效法的结果,因而《周易》中八卦及六十四卦体现了天地阴阳变化的规律。经过这样一解说,使《周易》理论变得博大精深。在注释《周易》卦爻辞时联系卦爻画注释,《周易》卦爻辞多是记录和叙述某一件事和某一现象。而《易传》把具体的卦爻辞上升到抽象的阴阳关系,从卦的含义及爻所处的位置进行解释,从理论水平看,远超《周易》古经。
2.《易传》从整体上对《周易》六十四卦加以排列和说解,揭示了卦与卦之间、卦象与卦辞之间、爻象与爻辞之间、卦与爻之间的内在联系,使《周易》六十四卦由原来散乱不堪,变成了一个有机的,具有一定逻辑性的相互联系的统一体。 [4]
3.《易传》对《易》的体例(如卦象、爻象、爻位等)作了详细说明,还保留了中国古代原始的古筮方法——大衍法,在《易传》产生之前的春秋时代,虽然用《周易》占问非常盛行,但对《周易》体例、筮法都没有说明,而《易传》在这一方面作了说明。对研究《周易》体例、筮法的起源有很大意义,若无《易传》,今人将不知古代《周易》体例和占筮的方法。 [4]

二:《易大传》的内容
1.《彖传》(或称《彖》)
彖,即材,通“裁”,有裁断之义。裁断一卦之义的文辞,叫彖辞。彖辞也叫卦辞。对彖辞(卦辞)的解释称为“彖传”。《易传》以“彖”作篇名,实指“彖传”。《彖传》共64条,按照六十四卦分为上下经的分法,《彖传》分为上、下,即《彖传》上,《彖传》下。《彖传》专释《周易》卦辞,其方法如下:
①以八卦之象释卦辞。八卦最基本的象是八种自然物:乾代表天,坤代表地,震代表雷,巽代表风,坎代表水,离代表火,艮代表山,兑代表泽。以八卦之象解释卦名卦辞。如《泰·彖》即《彖》释《泰》曰:
“天地交而万物通,上下交而其志同也”。
所谓天地交,是指泰上坤下乾,坤为地,乾为天,天本在上,而今在下,地本在下,而今在上,以示天阳之气下降,入地气之中,地阴之气上升而入天气之中,二气交感,故有“天地交”之义。由于阴阳二气交感,万物生生不已,故为通达,即是“万物通”。 [4]
所谓上下交,是指泰上坤下乾,乾为天阳本在上而今在下,坤为地阴本在下而今在上,故有上下交感之义,就社会而言,乾象征君,坤象征民众,在泰卦,君在下,众在上,以示君民上下交感,志向相通。
自然界的阴阳二气交感,万物通达生长;社会中君臣上下交感,志向一致,故天下泰平和谐,因而称此卦为《泰》卦。泰。就是泰平,通达。
②取义理释卦辞。八卦皆含有义理,如乾为刚,坤为柔,震为动,巽为入为风,坎为险,离为丽,艮为止,兑为说(悦)。《彖传》以八卦所含义理释卦辞,如《讼·彖》,即《彖》释《讼》曰:
“上刚下险,险而健,讼。”
所谓上刚下险,是指《讼》上乾下坎,乾为刚在上,坎为险在下,故有“上刚下险”。所谓险而健,是指《讼》内卦坎为险,外卦乾为刚,刚即健,故称“险而健”。因此卦乾阳刚健在上,坎水阴险在下,上下争讼,故为讼卦。
③取爻位说释卦辞。所谓爻位说,是指爻所处的位置。它主要有这样几方面内容:中位、乘、得位、失位、应位。《彖传》以爻位说释卦辞,如:A.以中位说释之:
《蒙·彖》:“‘初筮告’,以刚中也。”
所谓以刚中,是释“初筮告”,蒙卦有九二爻为阳爻,并居内卦中位,故称“刚中”。 [4]
B.以得位、应位释之:
《小畜·彖》:“柔得位而上下应之。”
小畜一阴五阳,六四爻以阴居阴位,故称“柔得位”,五阳爻分居六四上下,故称“上下应”。
C.以乘释之:
《彖传》所谓乘,是指阴爻居阳爻之上。乘,有乘凌之义。《彖传》以乘释卦辞,如《夬·彖》:
“扬于王庭,柔乘五刚也。”夬一阴五阳,阴为柔,阳为刚,此卦有一阴柔乘凌五刚之象,故柔乘五刚。
④以卦变释卦辞:
卦变,是指由于阴阳爻的变动,而使一卦变成另一卦,它反映了卦与卦之间存在着一种相互变化的关系。《彖传》运用这种卦与卦之间的关系注释卦辞。如《损·彖》曰:“损下益上,其道上行。”
所谓损下益上,是指《损》来自《泰》卦,即泰九三爻与上六爻交换位置而成,从上下卦看,是减损泰卦下体一阳爻而增益到其上体来。其道上行,是说减损下一阳而增加到上,阳通行在上。
而《益·彖》所说“损上益下”与《损·彖》相反,它是指《益》来自《否》,是减损《否》上一阳爻而增益到下,即《否》九四爻与初六爻互易而成。 [4]
2.象(又称《象传》)
卦象取法自然之象,自然之象是指自然界事物所呈现的容貌、形态,如日月星辰所呈现的象称为天象,山川草木所呈现的象叫地象。《周易》中的象是对自然界中的物象加以概括整理,并通过卦表现出来的。《易传》中作为篇名的“象”是指《象传》,从上下经来看,它分为《象》上、《象》下两篇,从释一卦来看,又可分为两部分:大象、小象。大象是释卦象,小象是释爻象。
大象释卦的方法,是先用八卦之象解释卦,然后比拟人事,说明根据此卦象当如何去行动。如《比·象》:“地上有水,比,先王以建万国,亲诸侯。”比上坎下坤,坎为水,坤为地,坎在坤上,故地上有水。水性下,水行地上,没有间隙,有亲比之义,故此卦为比。比:亲比。先王效法此卦象当建立许多诸侯国,并以仁爱的态度去对待诸侯。从卦画看,比卦九五居上中为王,上下全为阴爻,故称万国、诸侯。 [4]
《小象》释辞的方法:
①取爻位说
如释《屯》六二曰:“六二之难,乘刚也。”这是说,《屯》六二有“难”之文辞,是在于:《屯》六二之阴乘凌初九之阳刚。
又如释《否》九五爻辞:“‘大人’之吉,位正当也。”这就是说,《否》九五有“大人吉”之辞,在于此爻是以阳爻而居阳位,且居外卦之中位,即所谓“位正当”。
②《小象》有只从爻辞自身来解释爻辞,也就是说辞不明确,小象换一种说法表达这个意思。如《屯》六三曰:“即鹿无虞。”《象》释之曰:“‘即鹿无虞’,以从禽也。”这是说,在没有虞人作向导的前提下去追鹿,这只有被动地跟从禽兽。禽是释鹿。 [4]
③从义理方面释爻辞
如《恒》六五云:“贞妇人吉,夫子凶。”《小象》解释说:“妇人贞吉,从一而终也;夫子制义,从妇凶也。”这是说妇道人家守正则有吉。所谓守正就是妇人在婚姻方面从一夫而终,即一辈子只能嫁一夫,夫死不能再嫁,只有这样,才能有吉祥,而作为男子有绝对的权利只宜妇人从他,若他从妇人则有凶。《象》在这里很显然是以儒家的伦理道德来注释爻辞,这是《恒》辞中所没有的,《恒》六五爻辞是说,占问遇此爻,妇人则有吉,而男人则有凶。
3.《文言》
“文言”之辞,古者多解,有说以文饰乾坤的,有说依文言理,有说卦爻辞为文王所作,故曰文言,也有说乾坤德大,持以文饰而为文言。
本篇以“文言”为名,是指《文言传》。《文言》文字不多,专门释《乾》《坤》两卦卦爻辞。《文言》通过注释《乾坤》卦辞,阐发了天地阴阳变化之理,君臣上下、安邦治国、修心养性之道,它的注释,无论是从思想内容还是理论深度,远远超过了《乾》《坤》卦爻辞。 [4]
4.《系辞》
系,系属;辞,文辞。系辞,指系属在卦爻之下的文辞,即卦爻辞。《易传》以系辞为篇名,专指《系辞传》,其含义为系附在《周易》后面关于《周易》通论的文辞。
《系辞》分为上、下两篇,称《系辞》上,《系辞》下。《系辞》分章,古代不统一,一般说来,多采用两种分法。其一,分上篇为13章,分下篇为11章;其二,分上篇为12章,分下篇为12章。
《系辞》是对《周易》总的说明,内容博大精深,是今本《易传》7种中思想水平最高的作品,是学易必读之篇,阐述了乾坤在《周易》中的地位以及内在的根据,追述了《周易》起源、形成、作者、成书年代,揭示了《周易》的作用为认识事物规律、预知未来、道德修养、安邦治国、观卦象制作器具;解释了十六卦十八辞,以补充《彖》《象》之不足,说明了《周易》体例,包括卦位、作用、爻位、爻德等;保留了古代原始的占筮方法——大衍筮法,并对其客观根据作了说明。 [4]
5.《说卦》
又称《说卦传》,是系统地解说八卦的专著。《说卦》一般被分为十一章,主要说明八卦产生、性质、功用、方位,以及八卦所代表的卦象。其中八卦的性质、基本卦象是分析《周易》卦象,进行筮占的基础。
6.《序卦》
对六十四卦排列及排列的客观根据进行总的说明,以“有天地然后有万物”说明乾坤居《周易》之首,又以因果联系、物极必反、相生相成观点,解释卦与卦之间的关系。以物不可以终穷解释《周易》未济为最后一卦。《序卦》的解释多牵强附会。 [20]
7.《杂卦》
杂揉六十四卦,分六十四卦为三十二对,简要地说明卦名之义。之所以称“杂”,是因为它打乱了《序卦》六十四卦的排列,错综六十四卦而进行解说卦义。 [21]

周易演变
先天八卦
先天八卦,源自《说卦第二章》:“天地定位,山泽通气,雷风相薄,水火不相射,八卦相错。”
按乾坤、兑艮、离坎、震巽相错排列,所谓“乾坤纵而六子横”。
后天八卦
后天八卦,亦称文王八卦,其出现远比先天八卦早。后天八卦源自《说卦第五章》:“帝出乎震,齐乎巽,相见乎离。致役乎坤,说言乎兑,战乎乾,劳乎坎,成言乎艮。”所谓“震兑横六卦纵”。
歌诀
周易六十四卦的排列顺序有多种,如“今传本”卦序、京房易传的八宫卦序、邵氏易卦序、长沙马王堆三号墓出土的周易卦序等。 [22]
八卦歌诀:
乾三连,坤六断,震仰盂,艮覆碗,离中虚,坎中满,兑上缺,巽下断。
八卦代数:
乾一,兑二,离三,震四,巽五,坎六,艮七,坤八。
八卦五行:
乾、兑(金);震、巽(木);坤、艮(土);离(火);坎(水)。
八卦生克:
乾、兑(金)生坎(水),坎(水)生震、巽(木),震、巽(木)生离(火),离(火)生坤、艮(土),坤、艮(土)生乾、兑(金)。
乾、兑(金)克震、巽(木),震、巽(木)克坤、艮(土),坤、艮(土)克坎(水),坎(水)克离(火),离(火)克乾、兑(金)。
八卦旺衰:
乾、兑旺于秋,衰于冬;震、巽旺于春,衰于夏;
坤、艮旺于四季,衰于秋;离旺于夏,衰于四季;
坎旺于冬,衰于春。(四季是指每个季节的后一个月)

学术研究
性质
《周易》的性质学界长期以来存在分歧,原因在于漫长的历史发展中,《周易》随着政治变迁、理论需求以及自身地位变化,性质也有所不同。 [4]
一:《周易》产生及早期,是一部筮书,为人们提供行动的准则。
中国早期社会由于生产力低下,科学落后,先民对于自然现象、社会现象等不能作出科学的解释,当遭受意外的天灾人祸后,就萌发出借助于神意预知横祸或自己行为会带来何种后果的需求,以达到趋利避害。基于此,在长期的实践中发明了种种沟通人神的预测方法,其中最能体现神意的《周易》就是在这种条件下产生的。当时《周易》只是满足人们生产和生活预测的需要,这种一直持续到春秋战国时期。 [4]
二:经过演化,成为安邦治国、修身养性的哲学典籍。
汉代,《周易》的性质有所变化,在保留原有占筮性质的同时得到了充分发展。易学家们克服了大衍法的种种局限,创立了比较完备的新筮法。焦延寿作《易林》创立焦氏筮法,其弟子京房更胜一筹,对筮法进行了彻底变革,创立纳甲法,因而在汉代筮法趋向完备。
另一方面,《周易》也有了新的功能,因为《周易》中包含了深刻的人生哲理,尤其经过《易传》解释和发挥,其哲理化程度达到新的高度,《周易》遂成为一部博大精深的哲学典籍。也正是这个原因,《周易》得到了汉代统治者的青睐,由原来卜筮之书,而成为官方安邦治国、修身养性的哲学之书,被称为五经之首,大道之源。《周易》思想渗透到当时社会生活的各个领域,变成了统治者治国的理论根据。自此以后,《周易》包含了二重性,一方面在历代统治者加封之下,其理论指导作用日益显露和光大;另一方面,民间术士不断更新,完善筮法体系,至今《周易》二重性还是十分明显。 [4]
编次
《周易》编次在古代十分混乱,各种版本编次存在很大差别。通行的编次是阮元所刻《十三经注疏》中的《周易正义》编次,或后人篡改的朱熹《周易本义》编次,其编次特点经传混合,把《易传》解释《周易》卦爻辞的有关部分放到对应的地方。《易传》中凡属于总论,或者无法分割的部分,放在《周易》古经之后。 [23]
古代对《周易》编次的意见不尽一致,但总的看来可以分为两大派,一派是经传合,另一派是经传异。
战国时,《易传》十篇成书,《周易》经传分离,到汉代,《周易》与《易传》合称《易经》,但《周易》经传是分开的,一段把《易经》分为十二篇:《周易》古经上下两篇,《易传》十篇。
西汉末,费直首次以《易经》解经,经过郑玄、王弼传费氏易,将《易传》分割,附到《周易》古经相应的地方,并在书中称有“彖曰”、“象曰”,这就是如今通行的编次。 [4]
唐孔颖达采用了王弼本作《周易正义》,使王弼易本定为一尊,但也有人认为这个本编次不完善,还应该把其它传再分,如唐李鼎祚《周易集解》又在王弼本基础上把《序卦》分割,逐条放到每—卦经文之前。也有按照王弼本的乾卦的分法,将《易传》有关部分附到每一卦之后。
宋代有许多易学家不同意这种方法,恢复了汉初《周易》十一篇,让《周易》经传不相混杂,如吕祖谦、朱熹等人都是采用了经传分离的编次。朱熹的《周易本义》后经人篡改,次序被打乱,《周易本义》多是被篡改次序的本子。
将《周易》经传合,有利于对照传文解释《周易》古经,可以节省时间,这是有利的。但不利的一点是容易使经传不分,特别是打乱了各自的体系,不利于对二者的体系及各自的特点进行研究。而把《周易》经传分开,保持了《周易》经传原貌,克服了以上的缺点,但是,不能直接对应经传读《易》,二者各有千秋。 [4]
历代研究
从易学发展史上看,先秦易学发展到老孔时代,形成了道家易、儒家易及术家易三支。其後秦始皇焚书,易以卜筮之书独存,然易学至此盛极而衰。三支易学的这一总趋势,到先秦以下乃由隐而显:透过淮南子,道家易的趋势明朗化;透过董仲舒的春秋繁露,儒门易的趋势明朗化;透过占断灾异之学,术数易的趋势明朗化 [24]
。
《周易》的解释学传统至汉代而一大变。汉代对《易》的解释最重要的学派有三:一是以孟喜和京房为代表的象数之学,一是以费直为代表的义理学派,一是以严遵为首的黄老学派。
孟喜、京房之易以奇偶之数和八卦所象征的物象解释《周易》经传文,同时讲卦气说,并继承今文经学的传统,利用《周易》讲灾变。费直易学解经多取道德教训之意,用《彖》、《象》、《文言》中所讲的道理发挥《周易》经传文。严遵著《道德经指归》,以《周易》之义解释《老子》。这三家最重要的是孟、京一派的易学。此派易学最重要的是卦气说和纳甲说,将《周易》的卦与二十四节气及七十二物候相配,和干支、五行相配,将《易》坐实为一个定型的框架,这个框架可以装进不同的内容,框架的各个部分也可以由其规则推论而知,削弱了《周易》通过比喻、暗示、象征等进行范围广阔的意义诠释的有效性。汉易的象数传统对后世易学影响极大。 [25]
魏晋时代的王弼易学则转一方向,尽扫汉易象数学中滋蔓出来的各种学说,恢复义理学传统。他在解释《周易》经文中引入老庄哲学和东汉古文经学的传统,在解易体例上主取义说、一爻为主说、爻变说、适时说等。这在他介绍注易体例的《周易略例》中有详细说明。王弼非常重视《周易》的解释学传统,如他在《明象》中提出“得意忘象”说,主张通过卦象获取卦义,而获取卦义后就可忘掉卦象。这一说法的核心在通过解释学即象以见义,而一义能表现为不同的物象,故象不可拘泥执定。 [26]
程颐的《周易程氏传》是王弼《周易注》之后一部以义理方法解易的名著,在这部书中,解释学的方法得到了更为广阔的运用。程颐认为,《周易》是对宇宙万物的摹拟,但所要表达的,不是可用数量摹画的外在相状,而是一种道理。世界是一种道理和法则的宣示,《周易》也是一部道理和法则的宣示。六十四卦是这个总的道理在各卦所代表的特殊境遇中的体现,虽然代表六十四种境遇,但它经过解释,可以代表天下无尽的境遇。
在《周易程氏传》中,对于易学、理学的概念和范畴有很多解读,这些解读主要围绕如何重建儒家道德形上学的思想体系,从而服务于“内圣外王”的社会政治理想的实现。在《周易程氏传》中,从道德形上学的“理”,到践履完成它的工夫“诚”,都有一定的诠释,这不仅充分体现了程颐高度推崇、弘扬儒家人伦道德、王道政治理念,而且还为人们从本体到工夫,从明道到行道,提出了一条非常可行的实现“内在超越”之路径。 [27]
朱熹在评论程颐《易传》时说:“《易传》明白,无难看。但伊川以天下许多道理散入六十四卦中,若作《易》看,即无意味;唯将来作事看,即字字句句有用处。 [28] ”就是说,《周易程氏传》将《易》来做个载体讲他所见的道理,或者说是借《周易》卦爻发挥他自己的哲学思想。故无一句讲到卜筮,通篇皆在讲事理。朱熹还说:“伊川见得个大道理,却将经来合他这道理,不是解易。……他说求之六经而得,也是于濂溪处见得个大道理,占地位了。 [29]
”这也是说,程颐先熟读六经,尤其于周敦颐处特有颖悟,以此中义理为基础,然后借《周易》发挥所见。不是伊川解释《周易》,而是《周易》解释伊川。
程颐的这种易学观,对王夫之影响很大。王夫之以《周易》为道德训诫之书,就是在程颐这一基调之上继续延伸。朱熹不同意程颐以《周易》为言理之书,作《周易本义》,欲恢复《周易》本为卜筮之书,后来的易学家从中发挥出道理这一本来面目,强调《周易》的卜筮性质。王夫之吸取了朱熹的看法,不废卜筮而讲道德训诫,重点放在知得卜筮结果之后君子何以自省,何以接受道德教训而避凶趋吉。 [25]
现代中国学术界尤其是哲学史研究领域,《周易》一直被视为占筮书而难登大雅之堂。一方面承认其为传统社会官学典籍,另一方面又弃如蔽履。
20世纪60年代末,台湾学者劳思光在新编《中国哲学史》第一卷中,介绍了《易经》中的“宇宙秩序”观念。80年代,《周易》开始为大陆学术界所注意,从中国哲学史研究队伍中分离出专门从事《周易》研究的学术群体,以《周易》流传本及出土《周易》竹书、帛书文本为主,相继出版和发表大量高质量专著、论文;另一方面从逻辑学研究队伍中分离出来的专门从事中国逻辑史研究的学术群体,从逻辑学的角度审视和分析《周易》,相继出版和发表了一些专著和论文,并在21世纪初,将《周易》的逻辑思想作为中国逻辑史的起点,编写入国家级重点教材,由教育部研究生办公室推荐为高校研究生教学用书。 [30]

主要思想

哲学思想
古代中国学者的哲学思考,通过对易经的研究得到启发,哲学思辩能力也多数是在对易经的分析阐解和不同意见的争鸣中得到训练和提高。老子将易经的思想精华融入《道德经》中,创造了一个以辩证思维为核心的哲学体系。他在经卦阴阳相抱三爻成卦的组合方式的基础上,构造了一个“道生一,一生二、二生三、三生万物”的万物起源图式,揭示了事物内部所包含的种种势力的对立统一。“万物负阴而抱阳,冲气以为和”。阴阳相抱这一思想在易经还是一目了然的符号图,到了老子便有了种种具体的事物形象的分析。其间流传后世对中国哲学影响最大的命题莫过于“祸兮,福之所倚;福兮,祸之所伏”,矛盾对立的双方,必有一方为主,另一方为次。物极则反,对立面相互转化的思想,在易经是通过爻辞,对爻象在卦体中的不同位置使用吉凶等结语加以反映的。而在老子这里,已经到了社会、政治、伦理等一切方面,曲则全、枉则直、洼则盈、蔽则新、少则得、多则惑、认为委曲总是由保全转化。屈枉总是向伸直转化,卑下总是向充盈转化,蔽旧总是向新奇转化,这种辫证思维方式,是老子观察世界的方法。圣人抱一为天下式,他运用这条物极则反原理,对世间万物进行着辩证概括,兵强则灭,木强则折,坚强处下,柔弱处上,他又用这一条法则,提出了一系列处理问题的具体办法,老子这些从易经中得到启发而形成的辩证思想谱写了中国哲学史上颇有特色的一页华章。 [31]
孔子深得易经之道了,最显著者有二:一是关于正名这一政治主张,二是关于举一反三类推思想。在易经的推论规则中,有一条是关于阴阳爻与阴阳位是否一致的“当位律”。这一条思维规律要求在自然递进推演时,每一爻的阴、阳性质必须与所在位置的阴阳属性进行对照,一般而言,凡阳爻居阳位,或阴爻居阴位即“当位”表示此爻所象符合(顺)事物发展规律,倘若阳爻居阴位,或阴爻居阳位,则不当位,即此爻不符合(逆)事物发展规律,孔子把这一条推演规律扩大到了社会政治领域,提出了“正名”学说。在他看来社会政治领域中人与人之间的位置关系,也应当如同阳爻居阳位、阴爻居阴位那样当位才能使一个国家秩序井然局面稳定,否则名不正、言不顺,言不顺则事不成,事不成则礼乐不兴,礼乐不兴则刑罚不中,刑罚不中则民无所措手足,不仅不能越位,而且不在其位、不谋其政,不能产生不当位思想,孔子这一思想又被后人推广。 [31]
被考定成于战国中期的“周易大传”,是当时学者在总结前人认知易经的成果基础上,对易经卦爻符号体系及卦爻辞所作的一整套注释和阐解,其中既有老子、孔子这些哲学大家的思辩成果,也有更多古代优秀哲学家的思辩结晶,正是他们使易经这部古典在哲学方面产生了巨大影响,成为中国古代哲学的发源处。
先秦时期百家争鸣,所有学派或多或少都受到易经影响。之后的古代哲学发展每个重要时期,易经思想都充当着轴心角色。到了唐代,易经不仅受到学者重视,也得到了统治阶级如唐太宗的青睐,钦命孔子后裔孔颖达博采众长主编《周易正义》,推动了以易经研究为中心的哲学研究的开展,因此之后不少学者纷纷自注易经。
宋代哲学家如邵雍、周敦颐、张载、程颢、朱熹等人对易经都有很深的造诣,邵雍根据《易传》关于八卦形成的解译,构造了一个宇宙构造图式,创立了被称为“先天学”的理学象数学派;周敦颐根据“易传”和道家思想指出了一个简单而又系统的宇宙构成论《太极图说》;朱熹在《周易本义》中深入探讨,认为易经的核心是讲事物内部矛盾的对立统一,易只消阴阳二字括尽,对立统一是事物发展的普遍规律。明清之际的王夫之,在《周易外传》中将哲学研究和易经研究更进一步,提出“实道而器虚”的命题,明确指出“道者器之道,器者不可谓之道之器”。也就是说,一般原理存于具体事物中,而不是具体事物依存于一般原理,矛盾双方相反相成、相互转化,杂因纯起,即杂以成纯,变合常全,奉常以处变,则相反而固会其通杂和纯变和常相反相成对立统一。在推理时即要推出情之所必至,也要推出势之所必反,即要存其通,即掌握一般的相通之理,又要存其变即因时、因地等不同条件而灵活推论。只有把握了这种相通之理,才能行于此而不碍于彼这易经所特有的思维方法,用以指导行为,必无往而不胜。
易经之于中国传统哲学的深远影响,以及它在中国传统哲学中的地位、作用,是其他任何一部古典著作所不可及的,这可能就是其列为群经之首的根本原因。 [31]
法治思想
《周易》中有多处关于法律问题的论述,还有讼与噬嗑两卦专门讨论法律问题。 [32]
第一,君权神授思想
法是统治阶级意志的最集中体现,而这种体现,首先表现在立法权上。“君权神授”必然要求君主为立法的主体,人民没有立法权,也没有司法权,法律面前不平等,与之相应的是专制,“雷电噬嗑,先王以明罚敕法”正是这种意思的集中表达。《易传》的根本思想是专制,“法自君出”就成为题中应有之意,所以《易传》中反而没有对此加以论述。
第二,“刑罚清”与“刑罚中”
豫卦的下体是坤,上体是震;坤的性质为顺,震的性质为动。上下结合形成顺与动的特点,而顺是动的前提条件。自然界的日月星辰、昼夜交替、四时更替,都是按照一定的顺序先后出现的,这才保证了自然界的正常运转;人类社会是自然界的必然产物,法律的实施也要依此而行。这应该是自然法思想的最早表述,不过,近代自然法思想的出发点是强调一切权利来自于自然,《周易》强调“法权君出”,与人们所讲的自然法完全不同,而是从司法的角度,要求各级官员应仿效自然界的运行特点来执法。 [32]
第三,明罚敕法与明慎用刑
噬嗑“大象”:“先王以明罚敕法。”这是从立法的角度讲的,因为要实现“明罚清”与“明罚中”,前提是“明罚敕法”。也就是成文法或法律的公开化问题,把定罪与量刑以成文的形式固定下来,并公之于众,使天下民众清楚明白,知道其可为与不可为之事,尽可能不触犯法律;即使触犯法律,也因为有明文规定而定罪量刑适中,民心折服。以尽量避免执法者的主观随意性,显示法律的公正性。所以,“明罚敕法”的重点不在于罚,在敕不在法;在于育民教民,不在于制民刑民,与法家的思想根本不同。这是“为政以德”的延续,主张德教为先,先教后刑,德主刑辅。
第四,息讼思想。
《周易》中专门有一卦———“讼”讲诉讼问题的,但卦义却不鼓励人们争讼,更不教人们如何取得诉讼的胜利。作为一个忠实诚信的人,即使遇到不公正的待遇和委屈,也应保持内心平静,戒骄戒躁,能不诉讼就不诉讼;如果迫不得已非要诉诸法律时,也要保持冷静,不可采取过激行为,只有这样,才能最终获吉。“终凶”,卦中指上九,上九有终极其讼之象,也就是说,把官司彻底打到底的意思,这种行为无论胜诉还是败诉,皆凶;在《周易》看来,无讼为最理想境界,虽有争讼出现,但经过调解而平息争讼也不错,不听劝解把诉讼进行到底最不好。“利见大人”,需由有德有威望的大人物听讼———即由德才兼备的法官断案,才能息讼;“不利涉大川”,大川是大险大难,这是说,当某人陷入争讼的旋涡时,不可涉险其它危险之事,因为此时的人心浮气躁,运时不佳,涉险其它,很容易出其它问题;“君子以做事谋始”,与“终凶”对应,从另一个角度告戒人们,与其争讼不止,不如一开始就谨慎从事,理顺各种关系,从根本上杜绝诉讼。 [32]
经济思想
(一)两仪与经济
周易在经济上的重大启示,是阴阳两仪的动静观念。依据阴阳两仪的动静观念,人类经济活动总源头太极的第一个创化,是从消费者的主观价值中找出稳定合理的主观价值,其后分别有私有及公共消费财的后续创化。经济学认为,一切最终财货都直接或间接充当生活欲望的手段,凡能满足人类生活欲望的财货便是具有效用。个体的消费理论及资产选择理论,即建基于此。经由交换交易,再加上交易市场竞争性的提高,可衍生出合理稳定的客观价值之创化。客观价值不只可以适用于消费者与生产者等微观经济的测量分析,也能适用于一个国家宏观经济的测量分析,宏观经济的客观价值,通常只是微观经济的直接叠加。 [33-35]
由一个国家经济活动价值流汇集起来所得到的产业关联价量模型合理的两仪解析,还可印证周镰溪《太极图说》无极而太极,太极动而生阳”的“太极动而生阳”。此外,根据厂商的活动、经济环境对厂商无言的弱约定、及厂商对员工有言的强约定,经由就业与投资市场致中和之力的解析,也能指出中和是中华文化的分析观念,不只可以包容均衡,亦可以包容“为道也屡迁”之各种动因。 [36-37]
(二)三才与经济
《周易·系辞上》:“六交之动,三极(即天地人三才)之道也。”世界受天地人三才之主宰,因而三才是有意志、有意识、有感情的。要主宰就必须有精神作为,因此三才亦有为精神性万事之本的涵义。 [38]
经济学假设人类的欲望是无穷的,所以每一位消费者追求主观直接效用的极大,就是精神性的天;但受到消费支出预算的客观限制,是精神性的地。同理,社会集体消费追求社会福利的极大是精神性的天;但受到政府消费支出预算的客观限制是精神性的地。每一位生产者追求客观利润的极大是精神性的天,但受到生产技术限制是精神性的地。 [38]
若将消费活动的正而问题再视作天位阳交,政经领导下社会集体消费的正面问题视作人位阳交,生产活动的正面问题视作地位阳爻,然后联立起来,已十足展现出乾卦的卦象。若再将消费活动的对偶问题视作天位阴爻,政经领导下社会集体消费的对偶问题视作人位阴爻,生产活动的对偶问题视作地位阴乏,然后联立起来,则十足展现出坤卦的卦象。八卦以乾坤象征天地,而定上下之位,这就是易经说卦的天地定位。 [39]
(三)五行与经济
《春秋繁露·五行对》论述,五行相生循环是一种生、长、养、收、藏的生产性循环。而由藏再到生,其实就是反者道之动。所以生产活动不与消费、投资活动合理连结,无法形成一个完整的经济活动因果循环。 [40]
后世影响
编辑
播报
“易为群经之首、大道之源” [41]
,又为新道家列为“三玄”之冠 [42] ,集中体现了中华民族的思维模式、价值取向等哲学品格。 [43]汉武帝“独尊儒术”,《易经》被尊为六经之首。汉代被称为经学时代,“经学”高于一切学术。中国历代图书分类是:经、史、子、集四大类,经列于首。《周易》为“六经”之首,自然也就是群书之首,即中国的第一部典籍,影响极大 [42] 。历代学术思想发展之契机亦多建基于“易经”,两汉经学自不待言,魏晋“新道家”谈玄,亦将其列为“三玄”之一。若无《易经》之启发,“北宋五子”的学问几乎不能成立。 [43]《周易》不仅对中国哲学发展产生重大影响,而且对各个学科发展都发生了作用。《四库全书总目·经部易类小序》中说:“又《易》道广大,无所不包,旁及天文、地理、乐律、兵法、音员学、算术,以逮方外之炉火,皆可援《易》以为说,而好异者又援以入《易》,故《易》说愈繁。”正如国学大师、新儒家开山祖师熊十力所言“:中国一切学术思想,其根源都在《大易》,此是智慧的大宝藏。” [42]

\section{历史评价}

\subsection{总评}

《周易》是中国本源传统文化的精髓,是中华民族智慧与文化的结晶,被誉为群经之首,大道之源,是中华文明的源头活水,是中国古代杰出的哲学巨著,历经七千多年的历史至今经久不衰,奠定了中华文化的重要价值取向,开创了东方文化的特色,对中国的文化产生不可取代的重要价值和巨大影响。

《易经》的思想智慧已经渗透到中国人生活的方方面面,它的内容极其丰富,对中国几千年来的政治、经济、文化等各个领域都产生了极其深刻的影响。无论孔孟之道,老庄学说,还是《孙子兵法》,抑或是《黄帝内经》,《神龙易学》,无不和《易经》有着密切的联系。

《黄帝内经》是元素论五行文化和阴阳文化结合的典范,解决了大易“医病”的问题;思孟学派的《五行》是德性论五行文化与阴阳文化融汇的渊薮,解决了大易“医人”的问题。一言以蔽之:大易医国、医人、医病。

《周易》研究被称为“易学”,是一门高深的学问。“三易”在周朝不是随便可以见到的,孔子在得到周易之后爱不释手,《汉书·儒林传》记载:“孔子读易,纬编三绝,而为之传。”《易经》代代相传,释家林立,许多学者皓首穷经,考证训诂,留下了三千多部著作,蔚为大观。

《周易》历经数千年之沧桑,已成为汉族文化之根。易道讲究阴阳互应、刚柔相济,提倡自强不息、厚德载物。在五千年文明史上,汉民族之所以能够久历众劫而不覆,多逢畏难而不倾,独能遇衰而复振,不断地发展壮大,根源一脉传至今,与对易道精神的时代把握息息相关。

\subsection{中国评价}
孔子:加我数年,五十以学《易》,可以无大过矣。

孙思邈:不知易,不可以为医。

虞世南:不读易不可为将相。

苏东坡:抚视《易》《书》《论语》三书,即觉此生不虚过。

毛泽东:中国古人讲“一阴一阳之谓道”,不能只有阴没有阳,或者只有阳没有阴,这是古代的两点论,形而上学是一点论……

郭沫若:《易经》是一座神秘的殿堂。

冯友兰:《周易》不仅是中国的,也是东方的,更是世界的,不仅是古代的,也是现代的,更是未来的。

南怀瑾:我始终怀疑《易经》的文化是上一个冰河时期留下来的,不是这一个冰河时期的产物,因为它的科学、哲学的道理太高明了。

张协和:近代学者由于易理之启示获得诺贝尔奖金者已有四:德国汉森堡,其论文为《测不准原理》;丹麦之玻尔教授,其论文为《相生相克原理》,并在庆祝酒会上以太级八卦纪念章赠人;中国杨振宁、李政道,其论文为《不对等定律》,并自称得之易经之启示。今后由此书而得奖者,当犹有其人,愿周易学者多为现代科技服务。

\subsection{国际评价}
爱因斯坦:西方科学家做出的成绩,有不少被中国古代科学家早就做出来了。这是什么原因呢?原因之一是古代科学家自幼学习《周易》,掌握了一套古代西方科学家们不曾掌握的一把打开宇宙迷宫之门的金钥匙。

李约瑟认为易经的太极图显示了宇宙间力场的正极和负极的作用:“中国文明在科学技术史中曾起过从来没有被认识到的巨大作用。”

黑格尔:《易经》代表了中国人的智慧。就人类心灵所创造的图形和形象来找出人之所以为人的道理,这是一种崇高的业。

荣格:谈到世界人类惟一的智慧宝典,首推中国的《易经》,在科学方面,我们所得出的定律常常是短命的,或被后来的事实所推翻,惟独中国的《易经》亘古常新,相距六千年之久,依然具有价值,而与最新的原子物理学颇多相同的地方。

英国学者汤因比:中国诞生于2000年前的阴阳易学对人类文明的起源,起到了乐曲般的推动作用。

19世纪下半叶在明治维新时期,日本政府规定,不懂易经者,不得入阁。 

松下幸之助:中国古代的哲学,是天下之最。我公司职员必须顶礼膜拜,认真总结、背诵,灵活运用,公司才能兴旺发达。

美国哲学家卡普拉:可以把《易经》看成是中国思想和文化的核心。权威们认为它在中国二千多年来所享有的地位只有其它文化中的《吠陀》和《圣经》可以相比。它在二千多年中,保持了自己的生命力。

《周易》为儒家经典。六经之中,它因其体例和内容而被特别地列为占卜之书,散放着一股神秘气息。《周易》在人眼中之神秘,在于它从江湖术士滔滔不绝的颠倒阴阳中透露出宿命的玄机;《周易》在智者眼中之神秘,在于其深刻的思辩智慧和朴素的辩证观念。它肯定事物的运动变化永无穷尽,认为物极必反,否极泰来;大音稀声,大象无形。《周易》因其直击万事万物根本之理而为各科所援用,正所谓“易道广大,无所不包,旁及天文、地理、乐律、兵法、韵学、算术,以逮方外之炉火,皆可援易以为说”。孔子读《易》而“韦编三绝”;量子物理学家玻尔惊叹几千年前的太极图与他发明的并协(互补)原理颇为吻合,并以太极图为其族徽核心;哲学权威捷恩则说:“谈到世界人类唯一的智慧宝典,首推中国的《易经》。”《周易》之影响,由此可见一斑。

《周易》的智慧如清泉,绵远流长,汲之不尽。古诗云:“鸳鸯绣罢从君看,不把金针度与人”,本书却撩开《周易》神秘的面纱,把鸳鸯与金针和盘托出。“佛主拈花,伽叶微笑”,期望更多有心人心有灵犀,观此顿悟。


\mainmatter

\part{易经上}

\chapter{乾}

乾卦在自然中象征天,为圜;在人伦中象征父、君;在人体部位中象征首,就是大脑;在动物之中象征龙、马;另外还象征雾、金、冰,还有表示万物的颜色,是大赤,就是大红大紫的那种赤。

\begin{yuanwen}

\Bagua{111111}[3] \ (乾为天)乾上乾下。

乾\footnote{乾是本卦标题。卦义:天,健,刚性,阳性,矫健等。乾指北斗星,用来代表天。本卦的内容主要与天有关。},元\footnote{大。}亨\footnote{通。}利\footnote{顺利。}贞\footnote{正,固,定。}\footnote{元亨、利贞是两个表示吉祥的贞兆辞,表明是两个吉占。元亨的意思约等于大吉。利贞的意思是吉利的贞卜。元亨利贞代表了仁、义、智、礼四种美德。}。
\end{yuanwen}

乾卦,大,亨通,顺利,正定。

“元亨利贞”四个字系于卦之下,谓之卦辞。“元”的意思可以解释成:开始、创始、初始。“亨”,古代的“亨”字与“享”字是相通的。享是指祭祀,是指供奉天、供奉先祖时献上供品。“利”,即为收获,有利。贞就是占卜之用,古时的卜实际上是观察天象,观察时间,预测方位,当然也预测事情。贞的引申意是“正”字,正就是立,正立,成立,建立。乾卦卦辞的“元亨利贞”四字表面上没有取象,用的是比较抽象的概念。

\begin{yuanwen}
《彖\footnote{tu\`an}》曰:大哉乾元!万物资始,乃统天。云行雨施,品物\footnote{万物。}流形\footnote{传播。}。大明终始\footnote{若二与五爻变阴,成离卦,离为明,因而说“大明终始”。},六位\footnote{六十四卦,每卦都有六个爻位。最下位称“初”,最上位称“上”,余四位,分别称二、三、四、五。如果是阳的符号\liangyi{1},就称“九”,是阴的符号\liangyi{0},就称“六”。乾六位都是阳,所以都称“九”。}时成\footnote{乾卦的六个爻,自初往上,都分别代表一个时间单位,或是一月的五天,或是一旬的一日。},时乘六龙\footnote{震卦象征龙,阳爻为震主爻,乾卦象征天空。龙从初爻开始上行。初九当复卦的位置,九二当临卦的位置,九三当泰卦的位置,九四当大壮卦的位置,九五当夬卦的位置,上九当乾卦的位置,象征龙从一阳到六阳“与时俱进”。}以御天。乾道变化,各正性命。保合大\footnote{同“太”,极大。}和乃利贞,首出庶物,万国咸宁。
\end{yuanwen}

《彖传》说:大啊,乾元!万物由此开始大生并统属于天。云行雨落,万物普遍传播。太阳升起又落下,从初爻到上爻代表了时间的跨度。乾乘六阳时统御着天道。乾卦刚健变化的道理,时刻规范着万物的本性和命运。天道保持着伟大中和正义,是普利万物的首要因素,所有邦国都会因此得到稳定和安宁。

\begin{yuanwen}
《象》曰:天行健。君子以自强不息。
\end{yuanwen}

《象传》说:天道的运行刚健不息,君子观看这一卦象,要树立“自强不息”的志向。

\begin{yuanwen}
初九\footnote{初九是本卦第一爻名称,以 下“九二”、“九三”等也是。以“九”标示阳爻,以“六”标示阴爻。一个卦画由六爻组成,从下向上排列,依次用初、二、三、四、五、 上表示,如“六三”、“上六”、“九二”、“上九”等。它们都是表示爻的阴阳性和排列顺序的名称。}:潜\footnote{潜行。}龙\footnote{易经把每个人比作一条龙。},勿用\footnote{不要使用(出动)意思应是,隐藏实力,伺机出动。}。
\end{yuanwen}

为什么称初九呢?因为“初”为本卦中的开始一爻,也是最下一个爻位。虽然为阳爻,根据重卦规则,初、二爻为地道,三、四爻为人道,五、上爻为天道。“九”,是阳爻的名称,所以是“潜龙勿用,阳在下也”。但身居下位,也得遵守规则,暂时将阳气潜藏于下。爻辞指出,本爻的结果是“勿用”,深层意义上“勿用”不是不用,而是暂时不用;不是不行动,而是暂时不行动;不是不为,而是时机成熟了大有作为。

\begin{yuanwen}
《象》曰:“潜龙勿用”,阳在下\footnote{text}也。
\end{yuanwen}

\begin{yuanwen}
九二:见\footnote{xian,出现。}龙\footnote{龙星。}在田\footnote{天田。},利见大人\footnote{指王公贵族。}。
\end{yuanwen}

九二称“见龙在田”。“见龙在田”,龙已脱离潜伏状态,出现在地上,到了该发挥作用,有所施行的时候了。“见龙”是把握“现”的时机;“在田”是面对现实,面对大众,崭露头角,初显身手;是先以善行美德施行于众人,利益于社会,然后自然得利于众人。

\begin{yuanwen}
《象》曰:“见龙在田”,德施普\footnote{text}也。
\end{yuanwen}

\begin{yuanwen}
九三:君子\footnote{指有才德的贵族。}终日乾乾\footnote{勤勉努力。},夕\footnote{夜晚。}惕\footnote{tì,敬惧。}若厉\footnote{危险。},无咎\footnote{过失,灾难。}。
\end{yuanwen}

九三,阳爻居阳位,位置得当,但刚得过重。因为这里又是阳位,又是阳爻,刚中有刚,就是刚得过重。“终日乾乾”,终日戒慎恐惧,自强不息。“夕惕若”,即使到了晚上,还是心怀忧惕,不敢有一丝一毫的松懈。若,语助词。厉,危厉。九三在下卦之上,是多凶的危厉之地,本来要有过咎,由于它刚处得正,能够“终日乾乾,夕惕若”,有咎可以转变为无咎。个人发展到这个地步,社会地位也有了,但还没有达到事业的顶峰,还有许多值得忧虑的地方,因此必须终日勤勤恳恳,时时警惕自己。

\begin{yuanwen}
《象》曰:“终日乾乾”,反复道\footnote{text}也。
\end{yuanwen}

\begin{yuanwen}
九四:或\footnote{有人,指贵族。}跃在渊\footnote{跳进深潭。},无咎。
\end{yuanwen}

九四,或跃在渊。九四是阳爻居阴位,阳刚居于柔中,刚中有柔,柔中有刚。九四为上经卦的下位,又接近九五的尊位。尤须小心谨慎,不可轻举妄动,时可进则进,时不可进则退。进退依时而定,故有龙或跃起或在渊之象。跃是跳跃。有时身居高位,有时跃入低谷,并不是行为中有什么邪恶所至,而是可上可下,上下可居的一种修为的自在状态。

\begin{yuanwen}
《象》曰:“或跃在渊”,进无咎也。
\end{yuanwen}

\begin{yuanwen}
九五:飞龙\footnote{龙星。}在天,利见大人。
\end{yuanwen}

九五,这是阳爻居阳位,又居中位,是典型的中正之位,刚健得中,不偏不倚,纯粹精微,所以称为君位,“九五之尊”一词即来源于此。得此位者,能与见解相合者产生共鸣,与意趣相同者相互吸引。九五一爻对于个人而言,都是事业和社会地位的巅峰时期,因为有了以前各种各样的磨炼和休养,已经可以做到按照天的规律即自然的规律办事,可以做到抢在时间的前面而没有违背天的规则。

\begin{yuanwen}
《象》曰:“飞龙在天”,大人造也。
\end{yuanwen}

\begin{yuanwen}
上九:亢龙\footnote{龙升腾到极高处的龙星。},有悔\footnote{不吉利的占筮。}。
\end{yuanwen}

上九位已是本卦中最高位,也是最末位。本卦六个阳爻至此是阳气将消,阴气将长,说明“龙”飞得太高,成了“亢龙”。因为离开了“九五”的尊位,这里所以称“亢龙”,是指只知进,不知退;只知道生,却不明白死;只知道取得,却不知付出。亢是过的意思。九五飞龙在天,龙已经到了极高处,到了止进而退的时候,继续前进而不退,便要走向反面,故有悔。

\begin{yuanwen}
《象》曰:“亢龙有悔”,盈不可久也。
\end{yuanwen}

\begin{yuanwen}
用九\footnote{乾卦特有的爻名。《易经》的乾卦和坤卦都多一爻(坤卦为‘用六),专门表 示这两卦是全阳、全阴。“用九”表示乾卦的全阳爻将尽变为全阴爻。}:见群龙无首\footnote{等于说卷龙。龙卷曲起来就见不到头。},吉。
\end{yuanwen}

乾卦与坤卦都有“用九”与“用六”的断语,在其他六十二卦中是没有的。这里是“用九”,即用刚。为何用呢?“见群龙无首”,见,即表现,展现。就人事而言,宋代理学家程颐释“无首”为无自为首。意思是说资质刚健的人物不要自认为可以作为领导者,而需要让其他人拥戴自己为领导者。想要达到这样的效果,就要把握好一个度,这个度就在这个“用”字上,善于用刚则不会离开龙群,换句话说要保持刚健自强的状态,不脱离社会群体,不强自出头,始终让自身存在的社会群体表现为“群龙无首”的状态和心态。这样,对于自身而言,才能得到吉祥。

\begin{yuanwen}
《象》曰:“用九”,天德不可为首也。
\end{yuanwen}

乾卦:大吉大利,吉祥的占卜。

初九:龙星秋分时潜隐不见,不吉利。

九二:龙星出现在天田星旁,对王公贵族有利。

九三:有才德的君子整天勤勉努力,夜里也要提防危险,但最终不会有灾难。

九四:有些大人君子跳进深潭自杀,并不是他们本身的过失。

九五:龙星春分时出现在天上,对王公贵族有利。

上九:龙星上升到极高的地方,是不吉利的征兆。

用九:卷曲的龙见不到头,是吉利的兆头。

\chapter{坤卦}

坤卦的卦画是六个阴爻,是纯阴,从卦画上看它像一个已耕过的土地的面貌。依物理现象说,有和顺之象,有广远之象,有长养成物之象,有承天而广载之象,有与乾匹配之象。

坤是本卦标题。坤的卦象是六个阴交,用来表示大地以及阴柔的事物。 本卦的内容与人在地上的生活有关。

\begin{yuanwen}
\Bagua{000000}[3] \ (坤为地)坤上坤下

坤:元,亨。利牝\footnote{p\`in,母马。}马之贞。君子有攸\footnote{所。}往,先迷,后得主\footnote{主人,这里指接待旅客的房东。},利。西南得朋\footnote{朋贝,周代的货币。十枚贝壳串在起就是朋。},东北丧朋。安贞吉。

《彖》曰:至哉坤元!万物资生,乃顺承天。坤厚载物,德合无疆。含弘光大,品物咸亨。“牝马”地类,行地无疆。柔顺利贞。“君子”攸行,先迷失道,后顺得常。“西南得朋”,乃与类行;“东北丧朋”,乃终有庆。“安贞”之“吉”,应地无疆。

《象》曰:地势,坤,君子以厚德载物。

初六:履霜,坚冰至。

《象》曰:“履霜,坚冰”,阴始凝也。驯致其道,至坚冰也。

六二:直、方、大,不习,无不利。

《象》曰:六二之动,直以方也。“不习无不利”,地道光也。

六三:含章可贞,或从王事,无成有终。

《象》曰:“含章可贞”,以时发也。“或从王事”,知光大也。

六四:括囊,无咎无誉。

《象》曰:“囊括无咎”,慎不害也。

六五:黄裳元吉。

《象》曰:“黄裳。元吉”,文在中也。

上六:龙战于野,其血玄黄。

《象》曰:“龙战于野”,其道穷也。

用六:利永贞。

《象》曰:“用六”永贞,以大终也。
\end{yuanwen}

(7)安贞吉:占问定居而得到吉利的 预兆。

(8)直,方,大:指地貌平直、方正、辽阔。

(9)习:熟悉。

(10)含章:指周武王伐纣,战胜商纣王。

(1(4))可贞:称心如意的占卜。

(12)王事:大事,指战争。战争和祭把在古代都是最重要的事。

(13)括:收束,扎紧。囊:布口袋。

(14)黄裳:黄色的裙或裤。这是尊贵吉祥的标志。

(15)玄黄:血流的样子,是说血流得很多。

(16)用六:坤卦特有的交名。“用六”表示坤卦的全阴交将尽变为全阳交。
译文

坤卦:大吉大利。占问母马得到了吉利的征兆。君子贵族外出旅行经商,开始时迷了路,后来遇上招待客人的房东。往西南方向走有利,可以获得财物;往东北方向走会丧失财物。占问定居,得到吉利的预兆。 初六:脚下踩到了薄霜,结成坚实冰层的时令就快要到了。 六二:大地的形貌平直、方正、辽阔;虽然去到不熟悉的陌生地方,也不会有什么问题。 六三:周武王战胜殷商,是很好的占卜。有人参与战争,虽然没有战绩,但结局却很好。六四:把收成装进口袋捆好,收成不好不坏。 六五:黄色裙裤是大吉大利的象征。 上六:龙在旷野上争斗,血流遍地。 用六:这是永久吉利的最好征兆。


【注释】

①坤是本卦标题。坤的卦象是六个阴交,用来表示大地以及阴柔的事物。 本卦的内容与人在地上的生活有关。
②牝(pin)马:母马。
③攸(y0U):所。
(4)主:主人,这里指接待旅客的房东。
(5)朋:朋贝,周代的货币。十枚贝壳串在起就是朋。
(6)安贞吉:占问定居而得到吉利的 预兆。
(7)直,方,大:指地貌平直、方正、辽阔。
(8)习:熟悉。
(9)含章:指周武王伐纣,战胜商纣王。
(10)可贞:称心如意的占卜。
(11)王事:大事,指战争。战争和祭把在古代都是最重要的事。
(12)括:收束,扎紧。囊:布口袋。
(13)黄裳:黄色的裙或裤。这是尊贵吉祥的标志。
(14)玄黄:血流的样子,是说血流得很多。
(15)用六:坤卦特有的交名。“用六”表示坤卦的全阴交将尽变为全阳交。

【译文】

坤卦:大吉大利。占问母马得到了吉利的征兆。君子贵族外出旅行经商,开始时迷了路,后来遇上招待客人的房东。往西南方向走有利,可以获得财物;往东北方向走会丧失财物。占问定居,得到吉利的预兆。
初六:脚下踩到了薄霜,结成坚实冰层的时令就快要到了。
六二:大地的形貌平直、方正、辽阔;虽然去到不熟悉的陌生地方,也不会有什么问题。
六三:周武王战胜殷商,是很好的占卜。有人参与战争,虽然没有战绩,但结局却很好。
六四:把收成装进口袋捆好,收成不好不坏。
六五:黄色裙裤是大吉大利的象征。
上六:龙在旷野上争斗,血流遍地。
用六:这是永久吉利的最好征兆。

【赏析】

大地是人们赖以生存的根基。它虽然没有上天那么高高在上、 神圣而神秘,却让人感到实在、亲切。“坤”卦几乎涉及到了人们在大地上所从事的衣、食、住、行等全部重要活动,不由得让我们想到古人凭直感体验到的贴近大地胸膛的那种亲切而深情的眷念,因而从大地占得的征兆都是吉祥顺意的。这样一种认识和现念,简直可以说是一首大地母亲的颂歌。

人类由远古的采集、狩猎的生存方式,过渡到相对稳定和有保障的从事农牧商业的生存方式,是从漂泊、冒险、为生存而挣扎向安居乐业、休养生息的巨大飞跃。在这个飞跃过程中,必定会产生人类对大地无尽的亲情。西方传说中的巨人,只有紧贴大地才会获得无穷的力量。可见,对大地的亲情是一种具有普遍性的人类情感。

上有神圣幽远的苍天可以崇仰,下有广袤坚实的大地可以依靠,于是,人类的肉体和灵魂便有了寄居之所,寻到了永恒的园。世事的推移,人间的沧桑,在永恒的天与地之间,像一条动着的河流,昼夜不舍地向前奔腾。生命的律动,就在天、地、人的交融感应中显现出来。

\chapter{屯卦}

\Bagua{000000}[3] \ (水雷屯)坎上震下
\begin{yuanwen}


《屯》:元亨,利贞。勿用有攸往。利建侯。

《彖》曰:屯,刚柔始交而难生。动乎险中,大亨贞。雷雨之动满盈。天造草昧,宜建侯而不宁。

《象》曰:云雷,屯。君子以经纶。

初九,磐桓,利居贞。利建侯。

《象》曰:虽“磐桓\footnote{hu\'an}”,志行正也。以贵下贱,大得民也。

六二,屯如邅如,乘马班如。匪寇,婚媾。女子贞不字,十年乃字。

《象》曰:六二之难,乘刚也。“十年乃字”,反常也。

六三,即鹿无虞,惟入于林中,君子几不如舍,往吝。

《象》曰:“即鹿无虞”,以从禽也。君子舍之,往吝穷也。

六四,乘马班如,求婚媾。往吉,无不利。

《象》曰:“求”而往,明也。

九五,屯其膏,小,贞吉;大,贞凶。

《象》曰:“屯其膏”,施未光也。

上六,乘马班如,泣血涟如。

《象》曰:“泣血涟如”,何可长也?
\end{yuanwen}

注释

(1)屯(zhun)是本卦标题。屯的意思是困难,卦象是表示雨的“坎”和表示雷的“震”相叠加。本卦的内容是讲各种困难的事情。

(2)磐(pan) 桓(huan):徘徊难行。

(3)利居贞:占问安居,得到吉兆。

(4)屯如邅(zhan)如:想前进又不前进的样子。

(5)班如:回旋不前进的样子。

(6)匪寇:不是强盗。

(7)字:怀孕。

(8)即:接近。这里指追逐。鹿:麋 鹿。虞:掌管山林的官,这里指熟悉山林的人。

(9)惟:思考,想。

(10)几:当机智的“机”用。

(11)吝:很艰难。

(12)屯:当囤积的“囤”用; 膏:肥肉。

(13)涟如:水波荡漾的样子,这里形容血泪不断地流淌。
译文

屯卦:大吉大利,吉祥的占卜。出门不利。有利于建国封侯。 初九:徘徊难行。占问安居而得到吉利的征兆。有利于建国封侯。 六二:想前进又难于前进,乘着马车在原地回旋。这不是强盗前来抢劫,而是来求婚。占卜的结果是这个女子不能怀孕,十年之后才能生育。 六三;追捕康鹿时没有熟悉山林的人当向导,正在想进入密林中去。君子很机智,认为不如放弃追捕。进入密林很艰难。 六四:乘着马车在原地回旋,因为是去求婚。前进的结果吉利,没有什么不利。 九五:把肥肉囤积起来。占问小事吉利,占问大事凶险。 上六:乘着马车在原地回旋,悲痛得血泪流淌不断。



【注释】

①屯(zhun)是本卦标题。屯的意思是困难,卦象是表示雨的“坎”和表示雷的“震”相叠加。本卦的内容是讲各种困难的事情。
②磐(pan) 桓(huan):徘徊难行。
③利居贞:占问安居,得到吉兆。
④屯如邅(zhan)如:想前进又不前进的样子。
⑤班如:回旋不前进的样子。
(6)匪寇:不是强盗。
(7)字:怀孕。
(8)即:接近。这里指追逐。鹿:麋 鹿。虞:掌管山林的官,这里指熟悉山林的人。
(9)惟:思考,想。
(10)几:当机智的“机”用。
(11)吝:很艰难。
(12)屯:当囤积的“囤”用; 膏:肥肉。
(13)涟如:水波荡漾的样子,这里形容血泪不断地流淌。

【译文】

屯卦:大吉大利,吉祥的占卜。出门不利。有利于建国封侯。
初九:徘徊难行。占问安居而得到吉利的征兆。有利于建国封侯。
六二:想前进又难于前进,乘着马车在原地回旋。这不是强盗前来抢劫,而是来求婚。占卜的结果是这个女子不能怀孕,十年之后才能生育。
六三;追捕康鹿时没有熟悉山林的人当向导,正在想进入密林中去。君子很机智,认为不如放弃追捕。进入密林很艰难。
六四:乘着马车在原地回旋,因为是去求婚。前进的结果吉利,没有什么不利。
九五:把肥肉囤积起来。占问小事吉利,占问大事凶险。
上六:乘着马车在原地回旋,悲痛得血泪流淌不断。

【赏析】
屯卦用诗一般的语言为我们展示了人世间生存的艰难情景: 外出路难行,求婚受挫,追猎受阻,踌躇徘徊和悲痛欲绝的心境。 天地神灵固然可以成为精神上的支撑和鼓舞,而摆脱困境的难题, 却必须由人们凭借自己的努力来解答。“谋事在人,成事在天”这句古训,让我们在感到无可奈何的同时,也分明听见了人们不甘 于向困难和命运低头的心声。成功和胜利的机会,一半掌握在人的手中,另一半掌握在非人力所及的神灵手中。大胆前行,勇敢 追求,在不断耕耘的过程中去收获。人类就是这样在与天、地、人 不断地斗争中一步一步从远古走向今天的。那企图深入密林的追 鹿人,那再次前去求婚的有情人,实在体现了一种英雄气概—一 不屈不挠地追求。这种精神,也让我们联想到了西方神话传说中不断推巨石上山的西西弗斯。区别在于:一个是真实生活的情景, 一个是想象出来的神话。 

\chapter{蒙卦}
\Bagua{000000}[3] \ (坤为地)坤上坤下
\Bagua{111111}[3]
(山水蒙)艮上坎下
《蒙》:亨。匪我求童蒙,童蒙求我。初筮告,再三渎,渎则不告。利贞。
初六,发蒙,利用刑人,用说桎梏,以往吝。
九二,包蒙,吉。纳妇,吉。子克家。
六三,勿用取女,见金夫,不有躬。无攸利。
六四,困蒙,吝。
六五,童蒙,吉。
上九,击蒙,不利为寇,利御寇。

【注释】

(1)蒙是本卦标题。蒙的意思是高地上草木丛生。由于“蒙”字在本卦中多次出现,所以用它来作标题。全卦内容主要讲开荒垦植,也涉及到了家庭婚事等。
(2)我:占筮的人。童蒙:蒙昧愚蠢的人,指求筮的人。
(3) 渎:不恭敬,这里指亵读占筮。
(4)发蒙:垦荒时割草伐木。刑人:受过 刑的人,指奴隶。
(5)说:等于“脱”。
(6)以:等于“如”,如果。
(7)包蒙:捆扎割下的荒草。
(8)纳妇:迎娶妻子。
(9)克家:建立家庭。
(10)取女:抢夺女子成婚。
(11)金夫:武夫,拿着武器的男人。
(12)不有躬:丧失生命。
(13)困蒙:捆扎荒草。
(14)童蒙:童用作“撞击”的撞。童蒙的意思是砍伐树木。
(15)击蒙:砍伐树木。
(16)寇:强盗,侵略者。

【译文】

蒙卦:亨通。不是我请教蒙昧愚蠢的人,而是蒙昧愚蠢的人请教我。把第一次占筵的结果告诉了他,他却不恭敬地再三占筮; 对不恭敬的占筮,神灵不会告知。吉祥的占卜。
初六:最好利用有罪的奴隶去伐木开荒,因此解开他们身上的枷锁。如果外出,不吉利。
九二:捆扎割下的荒草,吉利。正式礼聘迎娶妻子,吉利。男女一起建立家庭。
六三:不要抢夺女子成婚,碰上拿着武器的人,会丧失性命。这样做没有什么好处。
六四:捆扎荒草。有危险。
六五:砍伐树木。吉利。
上九:割草伐木。充当强盗不利,抵御强盗有利。

【赏析】

向神灵请教,要诚心诚意;割草伐木开荒,要脚踏实地;诚心娶妻成家,要以礼相待。一个“诚”字,道出了为人处世、建功立业的秘诀。诚心真心可以感天动地惊鬼神,所以古人在说 “精诚所至,金石为开”这句话时,不仅有无数真实的体验作铺垫, 而且也包含着对人的能力的自信。 相待以诚,大概是人类从蒙昧走向文明之初最朴实的道德伦理准则。为了使人们确信这一准则的权威性,便构想出了神灵也 偏爱诚信的依据。有了这种说法,多少使人们在心理上有了戒惧, 行为有了规范。不过,越轨者在任何时代都有。远古没有严密、带强制性的法律,越轨者试图不劳而获,凭借暴力(肉体的和武器 的)手段强夺。这一来,就给本来就在为生存艰难奋斗的古人平添了防盗御寇这一重任,在与天斗的同时,还得与人斗,以至有 了战争。

历史的经验值得注意。“铭记上天神灵偏爱诚买守信的人们,肯定有助于我们在生命历程中的过渡。

\chapter{需卦}
\Bagua{000000}[3] \ (坤为地)坤上坤下
\Bagua{111111}[3]
(水天需)坎上乾下
《需》:有孚,光亨。贞吉,利涉大川。
初九,需于郊,利用恒,无咎。
九二,需于沙,小有言,终吉。
九三,需于泥,致寇至。
六四,需于血,出自穴。
九五,需于酒食,贞吉。
上六,入于穴,有不速之客三人来,敬之终吉。

【注释】

①需是本卦标题。需的本义是天上下雨,卦象是表示天的“乾”和表示 云的“坎”相叠加。本卦因“需”字多次出现,便用它作标题。全卦内容主要是出行和客居。
(2)孚:本义是俘虏,也指获利。
(3)光亨:意思是 大亨,元亨。
(4)需:爻辞中“需”的意思是等待,停留。
(5)用:以, 于。恒:常,长久。
(6)沙:沙地,难走的地。”
(7)言:当作想用,意 思是过错。
(8)泥:.泥泞的地方。
(9)血:血污的地方。
(10)穴;古 时的住所,依地势挖建而成,下半是在地下挖出的小土穴,上半是在地面搭 建的屋顶。
(11)速:请,招。不速:没有邀请。

【译文】

需卦:捉到俘虏。大吉大利,吉祥的占卜。有利于渡过大江大河。
初九:在郊野停留等待,这样长久下去是吉利的,没有危险。
九二:在沙地停留等待,出了一点小过错,最后结果是吉利的。
九三:在泥泞中停留等待,引来了强盗抢劫。
六四:陷入到血污之中,从地穴住处里逃脱出来。
九五:在酒席上留连等待,征兆吉利。
上六:进入地穴住处,来了三个不请自来的客人。主人殷勤地接待他们,结果吉利。

【赏析】
古人出行客居,自然与今人游山玩水、消闲遣闷、联络友情不同。他们没有那么多闲逸轻松的时光,日常时光和精力大多被生产、生活中谋生的活动占据了,出行客居总同某一具体的实用 目的有关,主要是经商贸易或征战、求婚等。因而,自然山川风光的绮丽,季候物象变幻与内在心境的共鸣,似乎被视而不见。在道路阻隔、交通工具简陋的情况下,首先让人关心’的是顺利与否, 出行前就必定要叩问神灵。出行中有泥泞坎坷风雨霜雪等天然险 阻,有强盗出没洗钱害命等人祸,当然也有路途坦荡、酒足饭饱睡香的愉悦畅快。透过这幅吉凶交织、苦乐掺杂的出行客居图,我们在驰骋的想象中完全可以领悟到:这是漫漫人生旅途的缩影。 

\chapter{讼卦}
\Bagua{000000}[3] \ (坤为地)坤上坤下
\Bagua{111111}[3]
(天水讼)乾上坎下
《讼》:有孚窒惕,中吉,终凶。利见大人。不利涉大川。
初六,不永所事,小有言,终吉。
九二,不克讼,归而逋。其邑人三百户,无眚。
六三,食旧德,贞厉,终吉。或从王事,无成。
九四,不克讼,复既命渝。安贞吉。
九五:讼,元吉。
上九:或锡之鞶带,终朝三褫之。

【注释】

①讼是本卦标题。讼的意思是争斗。本卦的内容主要讲人与人之间的纠 纷和斗争。
②窒;用作“侄”,意思是戒惧。窒惕:戒惧警惕。
③ 永:长久。不永所事:做事不能坚持长久。
④克:胜利,成功。
⑤ 通( bu):逃亡。邑人:采邑中的人,实际上是奴隶。
(6)眚( sheng):灾 祸,过错。
(7)旧德:从先人那里继承下来的遗产。
(8)厉:艰险。
(9)复:返回。即:服从。命渝:命谕,指判决。
(10)锡:赐。鞶(pan) 带:皮革做成的大腰带,供身居要职的贵族佩带,这里借指官位。
(11)终朝:一整天。褫(chi):剥夺。

【译文】

讼卦:抓获了俘虏,但要戒惧警惕。事情的过程吉利,结果凶险。对王公贵族有利,对涉水渡河不利。
初六:做事不能坚持长久,出了小过错,而结果吉利。
九二:争讼失败,回到采邑,。邑中奴隶逃跑了三百户。没有灾祸。
六三:靠从先人那里继承下来的遗产过活。占卜的征兆险恶,结果吉利。如果参与战争,不会获胜。
九四:争讼失败,返回服从判决。占问平安,得到吉兆。
九五:争讼。大吉大利。
上九:君王赏赐官职,但一天之内三次将赐予的官职剥夺。

【赏析】

凡是有人群的地方,总免不了有争斗,古今中外,概莫能外。 争斗的原因,林林总总,不一而足:或者为权力,或者为金钱,或者为名誉,或者为恋爱婚姻,或者为家务琐事……一言以蔽之,人们之间的争斗,总与利益有牵涉。小则动口舌、动手脚,大则动干戈、搞暴动。讼卦为我们展现的,便是几千年前古人争斗的真实图景。

争斗未必全是坏事,其中肯定有正义和非正义、进步与反动的原则区分。参与争斗也未必是好斗,《水许》中的英雄好汉是被逼造反,“官逼民反,民不得不反”。世事的险恶,多半映衬出人心的险恶。可以说,只要人类社会存在一天,总会伴随着争斗,在一定意义上,它也是推动社会前进的一种动力。

\chapter{师卦}
\Bagua{000000}[3] \ (坤为地)坤上坤下
(地水师)坤上坎下
《师》:贞丈人吉,无咎。
初六,师出以律,否臧凶。
九二,在师中吉,无咎,王三锡命。
六三,师或舆尸,凶。
六四,师左次,无咎。
六五,田有禽。利执言,无咎。长子帅师,弟子舆尸,贞凶。
上六,大君有命,开国承家,小人勿用。

【注释】

(1)师是本卦标题。师在这里的意思是指军队。本卦既因“师”字多次出 现,又因内容主要与军队出征作战有关,所以用师作标题。
(2)文人:这里的意思是军队的总指挥。
(3)律:军纪,纪律。
(4)否(pi)臧 (zang):不好。这里指不守军纪。
(5)王:君王。锡命:赐命,意思是下 令嘉奖。
(6)舆尸:用车运送尸体。
(7)左次:驻扎在左边。
(8)田: 田猎,打猎。禽:鸟兽。
(9)执言:意思是抓获俘虏。
(10)长子:指挥作战的长官。
(11)弟子:指挥运送尸体的副官。
(12)大君:国君。
(13)开国。分封诸侯。承家:分封大夫。

【译文】

师卦:占问总指挥的处境,吉利,没有危险。
初六:行军征战要守军纪,不守军纪,必打败仗。
九二:主帅身在军中,吉利,没有灾祸,君王三次下令嘉奖。
六三:军中有人用车运送尸体,战败。
六四:军队驻扎在左边,没有危险。
六五:打猎获取猎物,打仗抓获俘虏,没有灾祸。长官率领军队作战,副官指挥运送伤亡者,贞兆凶险。
上六:国君下令赏功,分封诸侯大夫。不能重用无才德的小人。

【赏析】

战争被古人看作最重要的事情之一。攻城掠地,发财致富,讨伐异己,争权夺利,都要诉诸武力。成者为王,败者为寇,似乎成了天经地义的真理。

战争的结果总有胜负,因此事前严肃认真地对待,请教神灵,祈求神灵保佑,寻找正当的理由,然后大张旗鼓地兴师讨伐。古人几乎把战争当作一种仪式来对待,当作一门艺术来研究,从神灵的意向、天时地利,一直研究到习武组织、制胜谋略、论功行赏等等规则。中国古代兵书的发达,世所罕见,在技术和艺术层 面上都堪称一流。

由此反思,古人称战争为“王者之事”,早已把它升华成了治国平天下的头等大事。相形之下,西方人后来称战争是政治的继续,就逊色多了。欲王天下者,精通战争艺术应是第一课。

\chapter{比卦}
\Bagua{000000}[3] \ (坤为地)坤上坤下
(水地比)坎上坤下
《比》:吉。原筮,元,永贞,无咎。不宁方来,后夫凶。
初六,有孚比之,无咎。有孚盈缶,终来有它,吉。
六二,比之自内,贞吉。
六三,比之匪人。
六四,外比之,贞吉。
九五,显比,王用三驱,失前禽,邑人不诫,吉。
上六,比之无首,凶。

【注释】

(1)比是本卦标题。比的本义是亲密,在本卦中为一词多义。由于“比”字 多次出现,本卦用它来作标题。全卦的内容主要讲交往和团结。
(2)原筮: 再筮,指三人同时再占问。
(3)不宁方:不安宁的邦国,不愿臣服的邦国。
(4)后夫:迟到的诸侯。
(5)比:亲近、安抚。
(6)缶(fou):瓦盆。 盈缶:用瓦盆装满酒饭。
(7)终来:即使。有它:有变故,有意外。
(8)比:团结一致。自内:自己内部。
(9)比:结党营私。匪人:不正派的 人。
(10)外:外部,外国。
(11)显:外,这里表示广泛。
(12)王用三 驱:君王打猎时让卫队从左右后三面把猎物驱赶到中间以便射猎。
(13)诫: 用作“骇”,惊吓。(14)比:互相倾轧。无首:没有头脑,指没有核心。

【译文】

比卦:吉利。三人同时再占问,占问长久吉凶,没有灾祸。不愿服从的邦国来了,迟迟不来的诸侯要受罚。
初六:抓到俘虏,安抚他们。没有灾祸。抓到俘虏,装满酒饭款待他们。即使有变故,结果吉利。
六二:自己内部团结一致,贞兆吉利
六三:与不正派的人结党营私。
六四:与外国结盟亲善,贞兆吉利。
九五:广泛亲善。君王打猎时三面包围,只留一面让猎物逃走。邑中百姓毫不惊骇,吉利。 上六:小人互相倾轧,不能团结一心,凶兆。

【赏析】

讲了战争,紧接着讲团结、外交。古人真是聪明无比,深谙“胡萝卜加大棒”的真谛。真正能王天下的人,必定能抓住时机,恰到好处地施展软硬两招,绝不会四面树敌,把自己逼到火山口上,也不会不以实力为后盾而盲目亲善妥协。

团结和外交也是一门大学问。上下左右,圈内圈外,国内国外,东西南北中,都要纳入视线之中。对弱者,为其撑腰打气。对叛逆,三面合围,给一条出路。对强者,谦恭又不失节。对君子, 彬彬有礼,动口不动手。对小人,威胁加利诱。手腕众多,可操作性很强,可借鉴的历史经验也不少,反正,要随机应变,灵活 机动,因时制宜,因地制宜,因人制宜,才能立于不败之地。

四海之内皆兄弟,普天之下皆王土。讲团结,搞外交,请不要忘了这个道义上的准则。

\chapter{小畜卦}
\Bagua{000000}[3] \ (坤为地)坤上坤下
(风天小畜)巽上乾下
《小畜》:亨。密云不雨。自我西郊。
初九,「复自道,何其咎?吉。
九二,牵复,吉。
九三,舆说辐。夫妻反目。
六四,有孚,血去,惕出无咎。
九五,有孚挛如,富以其邻。
上九,既雨既处,尚德载。妇贞厉。月几望,君子征凶。

【注释】

(1)小畜是本卦标题。畜的意思是田地里谷物滋生,草木茂盛。卦象是表 示天的“乾”和表示风的“粪”相叠加,卦辞、爻辞主要讲农业生活。本卦标题是根据内容加的。
(2)我:王公贵族的自称。
(3)复:返回。道:田间的 道路。
(4)牵复:拉回来。
(5)舆:车。说:用作“脱”。辐:车轮上的辐条,这里指车轮。
(6)血:用作“恤”,意思是担忧。惕:提防。
(7)挛如:捆绑得很紧的样子。
(8)富:用作“辐”。
(9)既:已经。处:停止。
(10)德:“得”的意思。载:用作“栽”。尚德载:还可以栽种作物。
(11)几:接近。望:农历每个月十五日月圆的时候,也叫做月望。

【译文】

小畜卦:吉利。在我西边郊野上空阴云密布,但雨却没有落下来。
初九:沿田问道路返回,没有什么灾祸。吉利。
九二:拉回来。吉利。
九三:车子坏了一个轮子,夫妻俩互相埋怨。
六四:抓到俘虏,免除了担忧,还是要注意提防,不会有灾祸。
九五:抓到俘虏后把他们紧紧捆住,与邻村邻族共同分享快乐。
上九:雨已降下,又已停止,还可以栽种作物。女子占问得到凶兆。月亮已是接近十五时的满月,君子离家出行,贞兆凶险。

【赏析】

从事农业劳动的生活平淡而琐碎,今天身处现代化大都市钢筋水泥丛林中的我们,难以想象其中苦、乐、喜、忧、烦、闷、愁、 淡等体验的具体滋味。生动切肤的感性体验,早已被抽象的文字 符号扼杀和深埋起来了。唯有想象力,才能透过冰冷僵死的文字符号,深入到真切具体的古人生存的事实中去,虽然这也是以我们今天的感性体验作为基础的。

没有现代化的农业机械和交通运输工具,没有电灯、电话、电视机和歌舞戏曲。日出而作,日落而息,冬去春来,年复一年艰辛体力劳动的印痕,渐渐在简陋土屋昏暗油灯的阴影中隐去。生存的现实是严峻的,活下去是人生的首要问题,除了自身能力之外,神灵似乎能带给人们精神上的慰藉。于是,春播秋收,天旱雨雪,虫灾鼠害,人祸徭役,都得叩问上苍的意向,都被严肃认真地对待。我们透过这些似乎神秘的占卜祷告,真切地听见了那来自远古的沉重的喘息和感叹,看见了祖先们布满厚茧的双手和满是皱纹的古铜色脸庞。

\chapter{履卦}
\Bagua{000000}[3] \ (坤为地)坤上坤下
(天泽履)乾上兑下
《履》:履虎尾,不咥人。亨。
初九,素履往,无咎。
九二,履道坦坦,幽人贞吉。
六三,眇能视,跛能履,履虎尾,咥人,凶。武人为于大君。
九四,履虎尾,愬愬,终吉。
九五,夬履,贞厉。
上九,视履考祥,其旋元吉。

【注释】

(1)本卦标题是履。原经文卦象后无“履”字。履的意思是踩踏,引伸为 行为和行为准则。由于“履”字在本卦中出现次数多,所以用它作为标题。全 卦内容主要讲人的行为修养。
(2)”咥(die):咬。
(3)素:洁白,引伸 为纯洁。素履;行为清正纯洁。
(4)履道:这里指人的行为修养。坦坦:宽 广坦荡。
(5)幽人:被监禁的人。
(6)眇(miao):一只眼睛小。
(7) 大君:国君。
(8)愬愬(su):恐惧的样子。
(9)夬(guai):“快”的本 字,意思是快速。夬履:意思是行为养撞急躁。
(10)视:察看,审视。视 履:意思是行为审慎。
(11)考祥:全面仔细地考虑。旋:反复。

【译文】

(履卦):踩到老虎尾巴,老虎不咬人。吉利亨通。
初九:行为清正纯洁,如此下去,没有灾祸。
九二:为人处世胸怀坦荡,即使无故蒙冤也会有吉祥的征兆。
六三:眼睛不好却能看,跛了脚却能走路。踩到老虎尾巴,老虎咬人,征兆凶险。军人掌握政权成为国君,也是凶兆。
九四:踩到老虎尾巴,让人害怕,但结果还是吉利。
九五:行为莽撞急躁,占问得到不利之兆。
上九:行为小心谨慎,反复仔细考虑,大吉大利。

【赏析】

这一卦以梦中所见踩到老虎尾巴的景象,来占问平时所作所为的吉凶兆头,探问神的意旨,这就是所谓“梦占”。古人迷信, 认为梦中所见所思,与日常的言谈举止有着某种必然的、神秘的内在联系,是神的意志的显现。因此,梦占便成了占筮的重要内容之一。

今天我们对梦的了解远比古人深入得多,虽然还没有达到了若指掌的地步,但已拨开了蒙在梦境之上的不少迷雾、不过,透过神秘之雾,我们发现古人关注的焦点集中在如何做人、如何使自己的行为合符仪轨之上。常言道,日有所思,夜有所梦。由梦联系到行为规范,可见古人对为人处世的重视。

我们不难发现,作者认为一个有教养的人应当行为清正纯洁, 胸怀坦荡,光明磊落,同时又沉着冷静,机敏细致,才可能有所作为。这个标准大概就是君子与小人、王者与野心家的分界线所在吧。由此可以想到,重视人伦道德纲常的儒家,何以要把《易》当作经典,也可以明白孔子所说的“君子坦荡荡”、“君子不 忧不惧”所包含的内容了。

\chapter{泰卦}
\Bagua{000000}[3] \ (坤为地)坤上坤下
(地天泰)坤上乾下
《泰》:小往大来,吉,亨。
初九,拔茅茹以其汇。征吉。
九二,包荒,用冯河,不遐遗。朋亡,得尚于中行。
九三,无平不陂,无往不复。艰贞无咎。勿恤其孚,于食有福。
六四,翩翩,不富以其邻,不戒以孚。
六五,帝乙归妹,以祉元吉。
上六,城复于隍,勿用师,自邑告命。贞吝。

【注释】

①泰是本卦标题。泰的意思是交通和畅,卦象为表示地的“坤”和表示 天的“乾”相叠加,以示阴阳交通和畅。全卦内容主要讲对立面的相互转化。
②小往大来:失去的小,得到的大。
③茅茹:一种可作红色染料的 草。
(4)汇:种类。
(5)包:用作“枹”,指枹瓜。荒:空。包荒:将枹 瓜挖空(用来绑在身上渡河)。
(6)冯(ping) 用作“淜”,徒步过河叫淜。
(7)不遐:不至于。遗:下落,下沉。
(8)得尚:得到帮助。中行: 中途,半路上。
(9)陂:斜坡。
(10)艰:通“旱”。艰贞:占问旱灾。
(11)孚:相信。其孚于食:相信粮食不成问题。
(12)翩翩:用作“谝谝”, 意思是巧言善辩,说大话。
(13)富:用作“福”。不富:遭殃。以:连累。
(14)戒:警惕。孚:俘虏。
(15)帝乙:殷代最后第二个帝王,纣王的父 亲。归:嫁。妹:少女。
(16)祉(ZhT):福。以祉:有福,得福; 隍:没有水的护城濠(有水的护城濠叫池)。

【译文】

泰卦:由小利转为大利,吉利亨通。
初九:拔掉茅茄草,按它的种类特征来分辨。前进,吉利。
九二:把匏瓜挖空,用它来渡河,不至于下沉。财物损失了, 半路上又得到别人帮助。
九三:平地总会变成起伏的斜坡,外出离开终归要返回。占问旱情,没有灾难。不用担心,相信会有粮食吃,会有福份。
六四:骗人说大话,使邻近的人一同遭殃,没有提防,还有人成了俘虏。
六五:殷王帝乙把女儿嫁给周文王,因此得福,大吉大利。
上六:城墙被攻破,倒塌在城濠中。从邑中传来命令,要停止进攻。占问得到不吉利的征兆。

【赏析】

中国传统思想注重对立面的相互转化,在《周易》中已初露端倪。以后的历代思想家不断谈到这方面的问题,将这方面的思想不断深化光大。老子就是一个突出代表。

对立面的相互转化,核心就是一个彼此沟通、转移的问题。天与地、自然与人类、国家与国家、一群人同另一群人、国君与臣民、丈夫与妻子,都存在相互联系和沟通的问题。通则畅,畅则和,和则万物兴旺繁盛。对立、对抗,只能导致敌意、矛盾冲突, 以至暴力战争。现代社会中的人们,已越来越认识到了相互沟通与和谐发展的重要性。

古人谈论对立面转化的立足点在一个“和”字,向他们更看重的是双方的转化:由生到死,由盛到衰,由好变坏,由大到小, 由福到祸。转化过程就是一个运动和变化的过程,这表明他们是用动态的观点来看待万事万物的存在。其中既有来源于真实生活的切身体悟(“包荒,用冯河,不遐遗。”),也有理性抽象的思辨 (“无平不陂,无往不复。”),应该说是相当深刻。我们现在更进一步认识到,对立转化需要一定的条件,比如由量变到质变 比如使用技术手段或政治、军事手段,而我们始终不应忘记的是古人早已阐明了的道理:万物顺遂和畅就是泰。

\chapter{否卦}
\Bagua{000000}[3] \ (坤为地)坤上坤下
(天地否)乾上坤下
《否》:否之匪人,不利君子贞,大往小来。
初六,拔茅茹以其汇。贞吉,亨。
六二,包承,小人吉,大人否。亨。
六三,包羞。
九四,有命,无咎,畴离祉。
九五,休否,大人吉。其亡其亡,系于苞桑。
上九,倾否,先否后喜。

【注释】

①本卦的标题是否(Pi)。原经文卦象后无“否”字。否的意思是闭塞, 不好,与“泰”的意思相反。卦象是表示天的“乾”和表示地的“坤”相叠 加,以示阴阳阻塞,不相通畅。全卦内容仍然是讲对立面相互转化的道理。
②否:不好,这里指做坏事。匪人:败类,小人。
③包:用作“庖”, 指庖厨。承:用作“普”,意思是肉。
④羞:即“饈”的本字,意思是美 味。
⑤有命:君王有赏赐的命令。
(6)畴:谁。离:受到,得到。祉:福。
(7)休否:不要干坏事。
(8)亡:危险,败亡。其亡其亡:危险到了极点。
(9)苞桑:苞草和桑树枝。
(10)倾:覆灭,倒下。倾否:干坏事要倒霉。

【译文】

(否卦):做坏事的是小人,这是对君子不利的征兆。由大利 转为小利。
初六:拔掉茅茹草,按它的种类特征来分辨。征兆吉利、亨通。
六二;庖厨中有肉,这对平民百姓是好事,对王公贵族算不上好事。
六三:庖厨中有美味。
九四:君王有赏赐的命令,没有灾祸,但不知得到赏赐的人是谁。
九五:不要干坏事,王公贵族知道就好。多么危险呵,国家命运就像系在苞草和桑枝上一样。
上九:干坏事要倒霉。先碰上恶运,最后还是可以交好运。

【赏析】

不好的症结在于阻隔,在于相互对立。明白好、坏相互转化的道理,意在使人增强忧患意识,处世做人要时时小心谨慎,瞻前而顾后,居安而思危,然后才能立于不败之地。有了阻隔,产生了对立,就要想办法疏通、消解,把不利变为有利,把坏事变成好事。

强烈的忧患意识,大概是中华民族积淀已久的心态结构。翻检一下我们的俗语、成语中累积起来的这方面的智慧(如“晴带雨伞,饱带饥粮”,如 未雨绸缪”等等), 就可以发现我们的祖先正是时刻怀揣着这种忧患意识从远古走来;我们民族强大的生命力,也正在于在不断变幻的宇宙天地人间中时刻保持清醒的头脑,以内在的智慧来对付各种不利的挑战。

\chapter{同人卦}
\Bagua{000000}[3] \ (坤为地)坤上坤下
(天火同人)乾上离下
《同人》:同人于野,亨。利涉大川。利君子贞。
初九,同人于门,无咎。
六二,同人于宗,吝。
九三,伏戎于莽,升其高陵,三岁不兴。
九四,乘其墉,弗克攻,吉主
九五,同人先号啕而后笑,大师克,相遇。
上九,同人于郊,无悔。

【注释】

①本卦的标题是同人。原经文卦象后没有“同人”二字。同的意思是聚 合,同人就是聚合众人。由于“同人”二字出现的次数多,本卦用来作标题。 全卦的内容专门讲作战打仗。
②野:郊外。古时称邑外为郊,郊外为野。
③门:这里指王门。古时遇到国家大事,常在王门前聚众训话、誓师、演 练等。
④宗:祭祖祖先的宗庙。
⑤戎:军队。伏戎:把军队隐蔽起 来。莽:茂密的树林草丛。
(6)高陵:高地。
(7)三岁:三年,这里指 多年。兴:举,起,这里指夺取。
(8)乘:登上。墉(yong):城墙。
(9)号咷(tao):嚎陶,大声哭喊。

【译文】

(同人卦):在郊外聚集众人,吉利。有利于渡过大江大河。对君子有利的占问。
初九:在王门前聚集众人,没有灾祸。
六二:在宗庙聚集众人,不吉利。
九三:把军队隐蔽在密林草丛中, 并占领了制高点,但却长时间不能取胜
九四:登上敌方的城墙,仍然没有把城攻下。吉利。
九五:会集起来的众人先大声哭喊,然后欢笑,因为大军及时赶到,转败为胜。
上九:在郊外聚集众人 没有悔咎。

【赏析】

重大的事情需要反复讲,从各个角度讲,讲深讲透,讲精彩。 战争就是这种事情之一。这一次不讲道理如何,不讲战争与其它诸事的关系如何,也不讲战争谋略,而是记述作战打仗的真实经过。从战前的仪式、誓师,到伏击战、攻坚战、突围战,以至最终获胜、祝捷,依次写来。简约的文字,给我们发挥想象力留下了广阔空间。

誓师时的群情激愤昂扬,祭祀时的庄严肃穆,出征时的整齐威武,伏击时的紧张刺激,攻坚时的艰难顽强,被围时的绝望挣扎,获胜时的欣喜若狂,祝捷时的皆大欢喜,加上铿锵作响的兵 器碰撞声,人喊马叫声,哭声骂声笑声,全都历历在目,声声在耳。

我们发现,古人已经把残酷的战争仪式化了,艺术化了,哲理化了。以神圣的名义制定出复杂的规则,投入巨大的智力、物力、 财力和人力,来演出人世间一幕幕惊天动地的场面,浓缩人生的意义。这的确是值得我们好好思索的一个问题。

\chapter{大有卦}
\Bagua{000000}[3] \ (坤为地)坤上坤下
(火天大有)离上乾下
《大有》:元亨。
初九,无交害匪咎。艰则无咎。
九二,大车以载,有攸往,无咎。
九三,公用亨于天子,小人弗克。
九四,匪其彭,无咎。
六五,厥孚交如威如,吉。
上九,自天佑之,吉,无不利。

【注释】

①大有是本卦标题。有的意思是丰收,大有就是大丰收。全卦的内容同 农业丰收有关。
②交害:互相侵害。
③艰:天旱,旱灾。
④公: 指众大臣。亨:同“享”,意思是宴会。
⑤匪:用作“昲”,意思是用太 阳晒。彭:用作“尪”(wang),意思是破足的男巫。
(6)厥:其。孚:俘 虏。交:同“绞”,这里指捆绑。交如:捆绑得很紧的样子。威如:气势汹汹的样子。

【译文】

大有卦:大亨大通。
初九:不要互相侵害,没有灾祸。即使天旱,也没有灾祸。
九二:大车大车的装载收成。有所往,没有灾祸
九三:天子设宴款待群臣。小人不能参与。
九四:用太阳晒男巫以求雨。没有灾祸。
六五:把抓到的俘虏紧紧捆住,但还是气势汹汹,不肯屈服。吉利。
上九: 上天保佑。吉利, 没有不吉利。

【赏析】

民以食为天。这一古老的思想早已深入到中华民族的骨髓之中,口福之乐是人生最大的快乐之一。丰收意味着一年的辛劳有了令人满意的结果,温饱有了着落,口福之乐可以得到满足,生命又能延续下去。人生之中恐怕再难以有如此意义重大的事情了。

丰收的喜悦之中的确包含着众多的心理体验:祈求苍天风调顺的期盼,担忧不劳而获的盗贼的强夺,对王公贵族铺张奢侈的不满,面朝黄土背朝天的辛劳,三四亩地两头牛老婆孩子热炕头的美梦,全都化作了对至神至圣的上天的虔诚。严峻的生活现实难以孕育出轻松浪漫的情怀。从远古面对上天的祈告声中,我们还能依稀听见几声无奈的喟叹。

\chapter{谦卦}
\Bagua{000000}[3] \ (坤为地)坤上坤下
【原文】
(地山谦)坤上艮下
《谦》:亨。君子有终。
初六,谦谦君子,用涉大川,吉。
六二,鸣谦,贞吉。
九三,劳谦君子,有终,吉。
六四,无不利,□谦。
六五,不富以其邻,利用侵伐,无不利。
上六,鸣谦,利用行师征邑国。

【注释】

①谦是本卦标题。谦的意思是谦虚、谦让。全卦内容主要讲道理上的谦 虚、谦让,并且“谦”字多次出现,所以用它来作标题。
②有终:拥有 好结果,有所成就。
③用:有利,利于。
④鸣:用作“明”,意思是 明智的。
⑤劳:勤劳,刻苦、
(6)伪(huT):用作“挥”,意思是奋勇向前。
(7)侵伐:这里的意思是讨伐敌人。
(8)行师;出兵作战。

【译文】

谦卦:亨通。君子谦让将会有好结果。
初六:谦虚再谦虚是君子应当具备的品德。有利于渡过大江大河,吉利。
六二:明智的谦让。吉祥的占卜。
九三:勤劳刻苦的谦让,君子会有好结果。吉利。
六四:没有什么不利,奋勇向前而又谦让。 由于不警惕使邻人一起遭殃,应当讨伐来犯之敌。没有什么不利.
上六:明智而谦让,有利于出兵讨伐邑国。

【赏析】

这一卦专门讨论谦虚这一道德品质。不难看出,讨论的前提是既定的:谦虚是一种美德,并且是有身份、有地位、有教养的君子必须具备的。从这个前提出发,再进一步从各个角度来深化 谦虚的内涵,把它与其它的品质联系起来。

传统上对道德伦理问题的关注似乎被当成了儒家哲学的专利,由此我们可以进一步明白《周易》成为儒家经典的内在原因。 说白了,谦虚作为一种极受推崇的美德,是为人处世的准则之一。它的基本要求就是才高而不自持,心高而不自傲,功高而不自居,名高而不自夸。这个准则一旦落实到行动上,应该说有相当的难度。它需要吃五谷杂粮的活人抵御各种欲望的诱惑和腐蚀,言谈举止小心谨慎,“战战兢兢,如履薄冰”,克己复礼。从另一个角度讲,正因为难以企及,才具有极大的吸引力,才使君子鹤立鸡群,卓越不凡,让人高山仰止,倾慕心仪。这大概是“内圣外王”的境界之一吧。

但愿我们都像真正的君子那样谦虚起来。


\chapter{豫卦}
\Bagua{000000}[3] \ (坤为地)坤上坤下
(雷地豫)震上坤下
《豫》:利建侯行师。
初六,鸣豫,凶。
六二,介于石,不终日,贞吉。
六三,盱豫,悔,迟有悔。
九四,由豫,大有得,勿疑。朋盍簪。
六五,贞疾,恒不死。
上六,冥豫,成有渝。无咎。

【注释】

①豫是本卦标题。豫的意思是犹豫、疑虑和预计、熟虑。全卦内容主要讲人的思想行为。豫既是多见词,又与内容有关,所以用它来作标题。
②鸣:用作“明”,意思是明亮,这里把白天、
③介:夹。
④盱:意思是缓慢。
(5)由豫:即犹豫。
(6)得:得到朋贝(货币)。
(7)盍(he):合。簪:古时盘头发的一种头饰。朋盍簪:用朋贝作成簪笄。
(8)冥:晚上。
(9)渝:变故。

【译文】

豫卦:有利于封侯建国,出兵作战。
初六:白天做事犹豫不决,凶险。
六二:夹在了石缝中不到一天被救出来。占得吉兆
六三:思想迟钝糊涂足以让人后悔;行动缓慢不定,更使人后悔莫及。
九四:经商先犹豫不决,反复考虑觉得会有大收获,便不再疑虑。后来把得到的朋贝制成头饰
六五:占问疾病,会痊愈并长久不死
上六:晚上反复考虑,事情是成功还是有变故。 结果没有变故。

【赏析】

作者显然是主张知行合一的,不赞成游移不定、没有主见、以至影响到行动和结果。前三爻讲到犹豫不决的坏处,后三爻说的是行动前要反复考虑,要求三思而后行。思想上明确之后,就要 坚决及时地付诸行动。

这是一种现实主义的态度,也是为人处世取得成功的一条原则。认识与实践之间有着内在的逻辑联系。只有脑子里认识清楚了,想明白了,行动才不会盲目;再好的想法和愿望,如果不踏踏实实地付诸实践,只能是空想。将知与行统一起来,才会有所作为,有所成就。

从根本上说,这种现实主义的态度是具有积极的人生意义的 它把人生看成是一个不断进取、奋斗的过程而不是消极地逃避,也不是一场充满幻想的白日梦。进取和奋斗就如同一场战斗:参与前反复思虑,想清楚后便投入。

\chapter{随卦}
\Bagua{000000}[3] \ (坤为地)坤上坤下
(泽雷随)兑上震下
《随》:元亨,利贞,无咎。
初九,官有渝,贞吉,出门交有功。
六二,系小子,失丈夫。
六三,系丈夫,失小子,随有求,得。利居贞。
九四,随有获,贞凶。有孚在道,以明,何咎?古
九五,孚于嘉,吉。
上六,拘系之,乃从维之,王用亨于西山。

【注释】

①随是本卦标题。随的意思是顺从,相随。全卦的内容讲商人出门做生意和贩卖奴隶的情况。
(2)官:古“馆”字,指馆舍,旅馆。
(3)交:交往,这里指互相帮助。
(4)小子:这里指年龄小的奴隶。
(5)丈夫:这里指成年的奴隶。
(6)求得:意思是希望获得利益。
(7)明:这里用作“盟”,意思是订立盟约。
(8)嘉:周代一个小国的名称,也称“有嘉”。
(9)维:捆绑。
(10)王:指周文王。西山:指歧山。

【译文】

随卦:大吉大利,吉利的占卜,没有灾祸。
初九:旅馆中发生了变故,但占得吉利。出门同行互相帮助有好处。
六二:抓住了年少的奴隶,成年的奴隶逃跑了。
六三:抓住了成年的奴隶,年少的奴隶逃跑了。结伴出门经商是为了获利。占问居住处得到吉兆。
九四:商人结伴出门是为了获利,占问得到凶兆。押送俘虏上路,互相订立了盟约,没有灾祸。
九五:俘虏了嘉国的人,吉利。
上六:把俘虏拘禁起来,紧紧捆住。周文王在歧山把他们作人牲祭视神灵。

【赏析】

这一卦所讲的出门经商所干的勾当是贩卖奴隶,其中透露出奴隶的两个来源:一个是商业买卖,一个是战争中的俘虏。同时, 我们也了解到了那时还用奴隶的生命来祭祀神灵。

这些情况至少让我们立刻联想到两点。首先是用现代人道主义的观点来看,贩卖和残杀奴隶是野蛮和骇人听闻的,在今天看 来是不可思议的。其次是西方近代史上殖民主义者打着人道主义 旗号贩卖奴隶、残害奴隶的血腥史实(法国作家梅里美的小说《塔曼戈》和美国作家斯陀夫人的小说《汤姆叔叔的小屋》均有真 切描述)。用道德化的观点来看历史,可以完全肯定地说,奴隶制 是人类历史上黑暗、野蛮、残酷、血腥、丑恶的一页,理所应当口诛笔伐。

不过,也可以用另一种观点来看待历史,那就是历史唯物主义的态度。从这种观点来看,奴隶制是人类社会历史发展过程中的必然。它为人类的物质和精神文明也作出过巨大的贡献。当然,这个推动力的主角,不是奴隶主,而是被他们侮辱和残害的奴隶。 这一点,作者无论如何是想不到的。

\chapter{蛊卦}
\Bagua{000000}[3] \ (坤为地)坤上坤下
(山风蛊)艮上巽下
《蛊》:元亨。利涉大川,先甲三日,后甲三日。
初六,干父之蛊,有子,考无咎。厉,终吉。
九二,干母之蛊,不可贞。
九三:干父小有晦,无大咎。
六四,裕父之蛊,往见吝。
六五,干父之蛊,用誉。
上九,不事王侯,高尚其事。

【注释】

①蛊(gu)是本卦标题。蛊的意思是“事”。全卦的内容主要讲儿子继承父业的事。由于蛊是全卦中的多见词,所以用它来作标题。
②先甲三日, 后甲三日:这是占问日期。古人记录时间的方法是;每年十二个月,每个月分三旬,每旬为十天,这十天依次用甲、乙、丙、丁、戊、己、庚、辛、壬、 癸十个字表示。按照这种方法,先甲三日就是辛日,后甲三日就是丁目。
③干:用作“贯”,意思是继承,这里指继承父业。
④考:用作“孝”, 子考就是儿子孝顺。
⑤裕:发扬光大。
(6)吝:艰难。
(7)用誉:得到赞誉。

【译文】

蛊卦:大吉大利。有利于渡过大江大河。在甲日前三天的辛日和甲日后三天的丁日出发。
初六:能继承父亲的事业,就是孝顺的儿子。没有灾难,虽有危险,结果还是吉利。
九二:继承母亲的事业,吉凶无法占问。
九三:继承父亲的事业,虽有小过错,但没有大灾祸。
六四:发扬光大父亲的事业,实行起来会有困难。
六五:继承父亲的事业,得到了赞誉。
上九:不为国君公侯服务,一心看重继承父业。

【赏析】

后现代的女权主义者看到“蛊卦”所讲的内容多半会嗤之以鼻,也可能会跳起来反驳。不过,这里所讲的是三千多年前的情况,作者如实表达了经过母权制时代进入到父权制时代后流行的家庭伦理观。按照这种观点,父亲是一家之长,是家庭中的权力核心;儿子继承父亲的业绩是天经地义的,不仅要继承,还要发 扬光大,这便是“孝”的具体表现。儿子不继承父业,即使能升 官发财,在道义上也要受到指责。

由父权制社会产生的男权主义思想,以及由此导致的对女性的歧视和社会压迫,是后来的事,不可与上述观点混为一谈。正如不能用道德化的历史观来看待真实的史实一样。实际上,随着父权制社会的逐渐解体,男权主义思想也失去了存在的依据,传统的男权观念也遇到了前所未有的挑战和被颠覆。只不过我们还 不能过分乐观,因为传统的力量毕竟太强大。

但有一点是肯定的,现代社会的儿子们的离经叛道已在伤透脑筋的父亲们面前掘出了一条深深的鸿沟,人们已越来越不相信“子承父业”的教条了。于是,一些父亲显出了无可奈何,一些父亲则显出了宽容和理解(不管是伪装的还是真诚的),也残存着一些冥顽不化的父亲。

\chapter{临卦}
\Bagua{000000}[3] \ (坤为地)坤上坤下
(地泽临)坤上兑下
《临》:元亨,利贞。至于八月有凶。
初九,咸临,贞吉。
九二,咸临,吉,无不利。
六三,甘临,无攸利;既忧之,无咎。
六四,至临,无咎。
六五,知临,大君之宜,吉。
上六,敦临,吉,无咎。

【注释】

①临是本卦标题。临的意思是从高处往下看和治理。全卦内容主要讲治民之术。临是卦中多见字,又与内容有关,所以用作标题。
②至于八月, 有凶:到了八月天旱,有凶兆。这里用天旱盼雨喻民盼治。
③咸:用作 “感”,意思是感化,这里指感化政策。
④咸:这里用作“诚”,意思是温和,指温和政策。
⑤甘:作用“钳”,意思是钳制,指钳制政策。
(6) 至临:亲自处理国事。
(7)知:智,明智。
(8)敦:敦厚诚实。

【译文】

临卦:大吉大利,占问得吉利。到了八月天旱,有凶兆。
初九:用感化改策治民,征兆吉利。
九二:用温和政策治民,吉利,没有什么不吉利。
六三:用钳制政策治民,没有什么好处。如果忧民之所忧,就没有灾祸。
六四:亲自处理国事,没有灾祸。
六五:用聪明睿智治民,是国君应该做到的。吉利。
上六:以敦厚诚实治民,吉利,没有灾祸。

【赏析】

统治者如何进行统治,如何使臣民归顺服从,历来是政治家们关注的焦点。中国古代这方面的著述可以说是汗牛充栋。临卦专门讨论统治术,算得上是一篇政治专论。前三爻讨论感化、温和与忧民政策,讲的是德治;后三爻讨论统治者躬亲、明智和敦厚的品行,说的是人治。

从统治者、统治术的角度看,作者的讨论应当说是较全面的,可以看作是对贤明的君主的最高要求。但不可忽略的是,这种讨论的前提始终被划定在君臣与民众这种天然形成的统治与被统治的关系之内,并且始终是从统治者的立场来看问题。这样的前提和出发点,正是专制制度产生的基础。它把民众当作是没有个人独立意志、价值和尊严而可以任意支配的对象,而不是在人人平等基础上的相互制约。因此,过高地评价传统的统治术,肯定是不妥的。

\chapter{观卦}
\Bagua{000000}[3] \ (坤为地)坤上坤下
(风地观)巽上坤下
《观》:盥而不荐。有孚顒若。
初六,童观,小人无咎,君子吝。
六二,窥观,利女贞。
六三,观我生,进退。
六四,观国之光,利用宾于王。
九五,观我生,君子无咎。
上九,观其生,君子无咎。

【注释】

①观是本卦标题。观的意思是观察、观看。全卦的内容与政治统治有关. 观在卦中多次出现,也与所讲内容有关,所以用它来作标题。
(2)盥(guan): 古代祭祖时用酒灌地迎神。荐:献,指祭祖时的献牲。
③颙(yong)若 :头大的样子,这里是指俘虏的头被打肿了。
(4)童:儿童,这里指幼稚无知。
(5)闚观:一孔之见。
(6)我生:我姓,指亲族。进退:行动,这里指政策措施。
(7)光:光耀,这里指政绩光耀。宾:作宾客,这里指朝觐。
(8)其生:其他姓氏,指别的部落氏族。

【译文】

观卦:祭祀时灌酒敬神,不献人牲,因为作祭牲的俘虏头青脸肿,不宜敬神。
初六:看问题幼稚无知,这对小人来说没有什么,但对君子就有害了。
六二:目光短浅,这是对女子有利的兆头。
六三:体察亲族的动向,由此决定政策措施。
六四:观察国家政绩大小,以选择可以朝觐的君王。
九五:体察亲族的意向,君子从政就不会有困难。
上九:体察其他部族的意向,君子从政就不会有困难。

【赏析】

这一卦又是从有利于统治者的角度来劝诫他们从政要善于体察各个方面的情况,从而维护自己的统治和既得利益,保证权力地位的牢固。作者的用心不可谓不良苦,算得上是忠君之士。

历来的忠君之士大抵都以匡扶社稷、劝诫国君为己任,而历来的史传都把这样的人奉为供人景仰的爱国者。其中是与非,自有人评说。在他们的心目中,万众百姓是供君子大人驱赶的羔羊, 民生疾苦和家国兴衰的价值只不过是用来烘托君子大人的神圣贤明。

但愿现代社会中人们的观念里再也不要有君子与小人、为官与为民一类高下尊卑的等级观念。君子大人同样是肉身凡胎吃五谷拉人屎,并不比百姓小人聪明多少高贵多少能干多少。

\chapter{噬嗑卦}
\Bagua{000000}[3] \ (坤为地)坤上坤下
(火雷噬嗑)离上震下
《噬嗑》:亨。利用狱。
初九,屦校灭趾,无咎。
六二,噬肤灭鼻,无咎。
六三,噬腊肉遇毒,小吝,无咎。
九四,「噬干胏,得金矢。利艰贞,吉。
六五,噬乾肉得黄金。贞厉,无咎。
上九,何校灭耳,凶。

【注释】

①噬嗑(shi he)是本卦的标题。噬嗑的意思就是吃喝,读音和意义与 “吃喝”一样。全卦内容是讲与饮食有关的事。噬嗑是卦中多见词,且与内容有关,所以用作标题。
②狱:刑罚。
③屦(ju):拖,拉。校:木制的刑具,这里指加在足上的桎。灭:遮盖。
④肤:肥美的肉,这里指鲜 鱼嫩肉。
⑤干胏(zi):带骨头的干肉。
(6)金矢:铜制箭头。
(7) 黄金:指铜箭头。
(8)何:用作“荷”,意思是负戴。校:这里指加在脖子上的刑具枷。

【译文】

噬嗑卦:亨通。有利于施用刑罚。
初九:足上戴着刑具,遮住了脚趾,没有灾祸。
六二:大吃鲜鱼嫩内,连鼻子也被遮住了,没有灾祸。
六三:吃干腊肉中了毒,出了小问题,但没有灾难。
九四:啃带骨头的干肉,发现肉中有铜箭头。占问旱灾,吉利。
六五:吃干肉,发现铜箭头。占得凶兆,但没有灾祸。
上九:脖上戴着刑具,遮住了耳朵,凶险。

【赏析】

虽然是讲吃吃喝喝,却又说到了刑罚。谁在大享口福之乐、大快朵颐,谁在受刑挨罚,是不言而喻的。把反差如此强烈的享乐和受刑放在一起,的确耐人寻味。或许,作者是想说明恩威并施, 赏罚结合,一张一弛是治人治国之道,要善于巧妙利用。

不过,我们从中能获得一些有关社会生活的信息。首先是吃肉。鲜鱼嫩肉是美味佳肴,即使在物质生活极大丰富的今天也没有大的改变;在没有鲜鱼嫩肉的季节吃干肉,显然是为了经常享受口福之乐,为了保证有足够的营养。由于古人缺乏必要的科考知识,腌制干肉的技术有时会出问题,以至有吃干肉中毒的事发生。可以想见,能够常年吃肉,在生活资料匮乏的时代,要有足够的财富作为后盾,一般人显然难以办到。

其次是刑罚。我们惊叹自己的祖先在这方面的聪明才智和创造发明:如此精细,如此种类繁多,如此专门化,如此严密。现在想来,如果把这等精力和智慧用在其它方面,结果将会怎样呢?

\chapter{贲卦}
\Bagua{000000}[3] \ (坤为地)坤上坤下
【原文】
(山火贲)艮上离下
《贲》:亨。小利有攸往。
初九,贲其趾,舍车而徒。
六二,贲其须。
九三,贲如,濡如,永贞吉。
六四,贲如皤如,白马翰如。匪寇,婚媾。
六五,贲于丘园,束帛戋戋,吝,终吉。
上九,白贲,无咎。

【注释】

①贲(b i)是本卦标题。责的意思是装饰,文饰。在本卦中,贲还借用为“奔”和“豮”。全卦内容主要讲婚嫁之事,作标题的“贲”字为卦中多见 词。
②贲:文饰。
③徒:徒步行走。
④贲:借用为“奔”。濡: 汗湿。
⑤贲:借用为“奔”。皤(po):用作“燔”,意思是焚烧。
③ 翰:马头高昂,这里指马飞驰的样子。
(7)丘园:指女家附近的地方。
(8)束:五匹帛为一束。戋戋:一大堆的样子。
(8)贲:借用为“豮”,意 思是大猪。

【译文】

贲卦:亨通。外出有小利。
初九:把脚上穿戴好,不坐车而徒步行走。
六二:把胡须修饰好。
九三:奔跑得满身大汗。占问长久吉凶得吉兆。
六四:一路奔跑,太阳晒得像火烧,白马昂头飞驰。不是来抢劫,而是来娶亲。
六五:跑到丘园,送上一束束布帛。遇到了困难,结果还是吉利。
上九:送上白色大肥猪,没有灾祸。

【赏析】

婚礼嫁娶是人生大事之一。远古时代民间婚俗的情景,今天已难想象得十分具体生动。“贲卦”中的描绘,恰好是一幅民间婚俗的风情画:娶亲的男方穿戴修饰整齐,有车不坐,一路奔跑到女方家,献上结婚的礼物。一桩美满姻缘就此完成了。

据说这是原始社会中期对偶婚的遗俗。结婚时,男方全氏族的成员要迁到靠近女方氏族居住的地方。“贲卦”所描绘的,正是这种情形。虽然只有动作、行为等简单的情节,但足以让我们在想象中去体察新郎内心的状态:兴奋激动中又包含着几分急切和忐忑不安,未来生儿育女的家庭生活和共同劳作的情景,一次又一次在脑海中闪现。其中肯定也有对肩负更大责任的意识,因为那时结婚绝不是简单的个人行为,而是同整个氏族的利益联系在一起的。


\chapter{剥卦}

\Bagua{000000}[3] \ (坤为地)坤上坤下
(山地剥)艮上坤下
《剥》:不利有攸往。
初六:剥床以足,蔑贞凶。
六二:剥床以辨,蔑贞凶。
六三:剥之,无咎。
六四:剥床以肤,凶。
六五:贯鱼以宫人宠,无不利。
上九:硕果不食,君子得舆,小人剥庐。

【注释】

①剥是本卦标题。剥的意思是击打、分离、掉落。全卦的内容同政治有关.“剥”是卦中多见词,所以用作标题。
②剥:脱落。
③蔑:无 不用。
④辨:用作“牑”,意思是床板。
⑤之:代词,指床。
(6) 肤:这里指床上的席子。
(7)贯鱼:射中了鱼。
(8)剥:离开。庐: 草 房子。

【译文】

剥卦:不利于外出。
初六:床足脱落了。不必占问,凶险。
六二:床权脱落了。不必占问,凶险。
六三:床离散了,没有灾祸。
六五:宫人射中了鱼,得到参加祭祀的荣宠。没有什么不利。
上九:劳动果实自己不能享受,君子却出门有车坐,百姓要离开自己的草屋。
六四:床上的席子没有了,凶险。

【赏析】

这一卦多是梦占,即根据梦中所见情景,来占问事情的吉凶。 前此的“履卦”中已出现过。梦见安身之处的床支离破碎,无法安身;身无居处,意味着生活的基本需要没有保障,自然是不好的兆头。梦境表明,做梦者心有忧戚。孔子说过,君子坦荡荡,小人常戚戚。“履卦”讲君子坦荡,现在“剥卦”又讲小人忧戚,一正一反,正合先哲之意。

其实,只要是人,都有忧戚,只不过忧戚的具体内容不同罢了。再进一步讲,只要是人,都要为衣食住行而忧戚。先圣孔夫子,若不是有人供奉、送束脩,恐怕也没有那么多豪言壮语吧。没有衣食住行的后顾之忧,而后大谈君子、小人之别,放言君子如何高贵,小人如何下贱,确实让人疑心之后感到荒谬虚伪。
\chapter{复卦}

\Bagua{000000}[3] \ (坤为地)坤上坤下
( 地雷复)坤上震下
《复》:亨。出入无疾。朋来无咎。反覆其道,七日来复,利有攸往。
初九,不远复,无祗悔,元吉。
六二,休复,吉。
六三,频复,厉,无咎。
六四,中行独复。
六五,敦复,无悔。
上六,迷复,凶,有灾眚。用行师,终有大败,以其国君凶,至于十年不克征。

【注释】

(1)复是本卦标题。复的意思是往返。全卦内容是讲行旅。“复”与内容有关,又是卦中多见词,所以用作标题。
(2)朋:朋贝,指货币,钱财。
(3)祗:大。
(4)休:美满。
(5)频:用作“颦”,意思是皱眉头。
(6)中行:中途,半路。
(7)敦:匆忙,急迫。
(8)眚(sheng):灾 祸,过错。

【译文】

复卦:亨通。外出回家不会生病。赚了钱而没有灾祸。路上往返很快,七天就可以了。有利于出门。
初九:没走多远就返回来了,没有大问题,大吉大利。
六二:完满而归,吉利。
六三:愁眉苦脸地回来,遇到了危险,却没有灾祸。
六四:独自一人半路返回。
六五:匆忙返回,没有大问题。
上六:迷路难返,凶险,有灾难。出兵作战,结果将会大败, 并连累到国君,凶险。十年都不能恢复作战能力。

【赏析】

《周易》一再讲到外出旅行,大概因为这事很重要(经商贸易和行军打仗都要外出),同时也有不少困难:天然的地理障碍,饥渴生病,盗贼打劫,同路人之间的不和,迷失道路和方向,两手 空空而归……总之,有太多意料之外的不利因素和危险,全然不像今天的飞机、火车、汽车、轮船那么方便、快捷、安全、舒适。

古人的行路难(李白曾慨叹蜀道难难于上青天)倒让人想起人生一世正如行路,也有太多意料之外的偶然因素和插曲,否则就不会有“谋事在人,成事在天”一说。有些事是自己可以控制 把握和努力做到的,有些事则超出了个人控制和努力之外;有些事是意料之中、必然会发生的,有些事则在意料之外、偶然出现; 有些事凭个人努力可以改变,而有些事则要改变人本身。

行路的苦乐忧喜唯有行路人自知,人生的苦乐忧喜也只有靠自己去体验。因而,行路和人生都一样,是一种实实在在、真真切切的体验。重要的是体验的过程,而不是结果。

\chapter{无妄卦}
\Bagua{000000}[3] \ (坤为地)坤上坤下
(天雷无妄)乾上震下
《无妄》:元亨,利贞。其匪正有眚,不利有攸往。
初九,无妄往,吉。
六二,不耕获,不菑畲,则利用攸往。
六三,无妄之灾,或系之牛,行人之得,邑人之灾。
九四,可贞。无咎。
九五,无妄之疾,勿药有喜。
上九,无妄行,有眚,无攸利。

【注释】

①无妄是本卦的标题。妄的意思是乱,不正。无妄就是不要有不合正轨行为。全卦的内容是讲行为修养要合于正轨,不能乱来。无妄是卦中多见词,又与内容有关,所以用作标题。
②菑(ZT):新开垦的荒地。
(3)畬(yu):耕种了三年的熟地。
④无妄:意料之外。

【译文】

无妄卦:大亨大通,吉利的占问。如果思想行为不正当,就 会有灾祸。不利于外出有所往。
初九:不要有不合正道的行为,吉利。
六二:不耕种就要收获,不开垦荒地就想耕种熟地。妄想者的行为难道有利吗?
六三:意料之外的灾祸。有人将牛拴住,过路的人顺手把牛牵走了,邑人丢牛得了意外之灾。
九四:利于占问,没有灾祸。
九五:得了病不胡思乱想,不吃药也会痊愈。
上九:不要妄行。妄行有灾,没有什么好处。

【赏析】

这一卦的主题是告诫人们不要有非分之想,不要胡作非为,思想和行为都要合于正道。换句话说,就是要想得正,行得端;反过来说,就是人正不怕影子歪。

这个立意不能说不好。讲究“思无邪”、名正言顺、光明正大的中国传统思想,总是用各种理论、例证、乃至说教来加强和提高人们的自觉性,通过个人人格的修养,来确立人们心中的道德 律令、行动准则。

但是,道德的作用毕竟是有限的。社会行为的规范还必须辅之以律法,月一定的强制措施来制约那些越轨者。况且,道德的说教对某些人(包括历代统治者)难以发挥作用,甚至还有口是 心非、阳奉阴违的人,以及敢于公开挑战道德准则的人。因此,善良的愿望总得配合着切实可行的措施制度,才会如愿以偿。

\chapter{大畜卦}
\Bagua{000000}[3] \ (坤为地)坤上坤下
(山天大畜)艮上乾下
《大畜》:利贞。不家食吉。利涉大川。
初九,有厉,利已。
九二,舆说輹。
九三,良马逐,利艰贞,曰闲舆卫,利有攸往。
六四,童牛之牿,元吉。
六五,□豕之牙,吉。
上九,何天之衢,亨。

【注释】

①大畜是本卦的作题。畜的意思是聚积,大畜就是积蓄很多。全卦内容与农业和畜牧业有关。
②不家食:不在家里吃饭。
③巳:用作 “祀”,祭祀。
④说:用作“脱”。輹:用作“辐”,这里指车轮。
⑤ 逐:交配。马在交配时要奔跑追逐。
(6)闲:用作“娴”,意思是熟练,熟 悉。舆卫:车战中的防卫。
(7)童牛:犝牛,即公牛。牯(gu):牛角上的 木架。
(8)豮(fen)豕:奔突的大猪。牙:用作“互”,意思是猪栏。
(9)何;用作“荷”,意思是承受。衢:福禄。

【译文】

大畜卦:吉利的占卜。不在家里吃饭,吉利。有利于渡过大江大河。
初九:有危险,有利于祭祀神鬼。
九二:车上的车轮脱掉了。
九三:良马交配繁殖。占问旱灾得到吉兆。每天练习车战防卫。有利于出门行旅。
六四:用木架架住公牛的角,大吉大利。
六五:用围栏圈住奔突的大猪,吉利。
上九:得到上天的福枯,大吉大利。

【赏析】

男耕女织,是古时自然经济条件下农民理想的生活方式。种田放牧自然是男人的亨,正如骑马打枪、当兵打仗是男人的事一样,其中甘苦,唯有男人最有体验。

这里我们看到了远古畜牧业的情景。它与今天工业社会使用人工激素饲料的工业化畜牧大异其趣。那是一个动物凶猛的时代, 性情粗暴的公牛不必说了,就连现在被看成最蠢笨、最无战斗力的猪,也凶猛异常,尚未驯服,其它动物(包括人在内)的凶猛更可想而知。

工业化社会的全面异化,必然使人对生命存在的原生态心驰神往,对动物和人的雌性化深恶痛绝。回归自然的口号虽然早在9世纪就已喊出,但我们却离自然越来越远,与自然越来越隔膜,越来越对立。回归自然已经成了一个遥不可及的奢望。还是古人有福气,呜呼!

\chapter{颐卦}
\Bagua{000000}[3] \ (坤为地)坤上坤下
(山雷颐)艮上震下
《颐》:贞吉。观颐,自求口实。
初九,舍尔灵龟,观我朵颐,凶。
六二,颠颐拂经于丘颐,征凶。
六三,拂颐,贞凶,十年勿用,无攸利。
六四,颠颐,吉。虎视眈眈,其欲逐逐,无咎。
六五,拂经,居贞吉,不可涉大川。
上九,由颐,厉,吉。利涉大川。

【注释】

①颐(yi)是本卦的标题。颐的意思是养育,同饮食营养有关。全卦内 容主要讲养生之道。“颐”是卦中多见词,又与内容有关,所以用它作标题。
②观:观察,研究。
③口实:口中的食物,口粮。
④舍:放置。 灵龟:用于占卜,所以十分贵重。这里代指财宝,财富。
⑤朵颐:朵,动的意思,颐动即为咀嚼之意,指饮食之事。
6颠:用作“填”,意思是塞。
(7)拂经:开垦荒地。
(8)颐征:为了生计而去抢劫粮食。
(9)拂:违 背。
(10)眈眈:盯得紧的样子。
(11)逐逐:动得快的样子。
(12)由:遵 循。

【译文】

颐卦:占得吉兆。研究养生之道,要靠自己解决粮食问题。
初九:你自己放着大量财物,还来窥伺我的衣食。凶险。
六二:要解决生计问题,就得在山坡上垦荒开田。为了生计而去抢劫粮食,凶险。
六三:违背养生之道,占得凶兆。十年都很倒霉,没有什么好处。
六四:解决生计问题靠自己,吉利。像老虎一样盯住别人的衣食,想一下子扑过去抢夺。没有灾祸。
六五:垦荒开田,有利于定居的占问。不能渡大江大河。
上九:遵循养生之道,先艰难后吉利。有利于渡过大江大河。

【赏析】

人间正道是自己动手,丰衣足食;不劳动者不得食,不劳而获是遭天谴的行为。我们的老祖先深明这个大义,在这一卦中反复申明这个道理。

农业社会的生存之道就是如此:奖勤罚懒。qiǎo取豪夺不仅不是生存的长久之计,而且有体于天理良心,应当口诛笔伐,必要时还得以暴力对付暴力。如今社会变了,前进了,但是道理却依然适用,只不过人们对此的理解变得更复杂和深刻了。保护知识产权,保护发明和使用的专利权,保护有形资产和无形资产,都是不劳动者不得食这一道理的延伸和深化。

这也是一种社会正义。它以自然公理为基础,以舆论作为捍卫它的主要手段。现代社会则以法律形式来保证社会正义的实施,任何敢于越轨的人都将受到惩罚。这肯定比古人前进了一大步。可以想见,无论社会怎么前进,这个道理绝不会变,正所谓天变道不变。这大概也算是人类社会永恒的真理之一吧。

\chapter{大过卦}
\Bagua{000000}[3] \ (坤为地)坤上坤下
(泽风大过)兑上巽下
《大过》:栋挠,利有攸往,亨。
初六,藉用白茅,无咎。
九二,枯杨生稊,老夫得其女妻,无不利。
九三,栋桡,凶。
九四,栋隆,吉。有它,吝。
九五,枯杨生华,老妇得其士夫,无咎无誉。
上六,过涉灭顶,凶。无咎。

【注释】

①大过是本卦的标题。大的意思是太,大过就是太过。全卦的内容是讲一些过了头的事,标题是按内容取的。
②橈(nao):弯曲。
③藉:席,用作铺垫。白茅:一种柔软洁白,较贵重的草。
④梯:用作“荑”,意思 是草木新生、发芽。
⑤隆:中间高起来。
③它:指意外的事故。
(7)灭顶:水淹过头顶。

【译文】

大过卦:屋梁被压弯了。有利于出门行旅,亨通。
初六:用白茅铺垫以示恭敬,没有灾祸。
九二:枯萎的杨树重新发芽,老头儿娶了年轻女子为妻。没有什么不吉利。
九三:屋梁压弯了,凶险。
九四:屋梁隆起不弯,吉利。但有意外事故,不妙。
九五:枯萎的杨树重新开花,老妇人嫁了一个年轻丈夫。没有灾祸也没有好处。
上六:渡河涉水,水淹过了头顶,凶险,但没有灾祸。

【赏析】

对人对事采取中间态度,似乎是一种最好的选择:过头或不及都失之偏颇。相比之下,不及比过头还要好些,所以才有“树大招风”、“高处不胜寒”这样的说法,以及枪打出头鸟这样的做法。中国人早已习惯了中庸的思维和行为方式,无论我们现在怎样指责,这个现买,这种内在心理结构,一时难以改变。

退后一步想,撇开中庸是好是坏的争执,折中的确是处理矛盾冲突的一种现实的手段。远的不说,就说近几年世界两大阵营冷战状态的结束,巴勒斯坦与以色列的和解,中国用“一国两制”的政策解决港、澳、台问题,都是突出的事例。其实,西方明智的人士远比一些国人更钟情于中庸思想,更懂得运用它的奥妙。这至少表明,走中间道路,取中间态度,是现实的,聪明的, 也是有生命力的。

\chapter{习坎卦}
\Bagua{000000}[3] \ (坤为地)坤上坤下
(坎为水)坎上坎下
《习坎》:有孚维心,亨。行有尚。
初六,习坎,入于坎,窞,凶。
九二,坎有险,求小得。
六三,来之坎,坎险且枕,入于坎,窞,勿用。
六四,樽酒簋贰用缶,纳约自牖,终无咎。
九五,坎不盈,祗既平,无咎。
上六,系用徽纆,窴于丛棘,三岁不得,凶。

【注释】

①本卦的标题是坎。习坎的意思是重坎,是说卦象为两个坎卦相叠加。标题省去习字是为了方便称呼。坎的意思是坑,陷阱。全卦内容主要讲从渔猎时代到农业时代的社会发展变化,用多见词“习坎”作标题。
②尚:帮 助。
③窞(dan):双重坎坑。
④之:至,到达。
⑤枕:用作 “沈”,意思是深。
(6)樽:装酒的器皿。簋(gui):装饭的器皿。簋贰:两碗饭。
(7)缶:陶制的器皿。约:取。牖(you):窗户。
(8)祗:应为 “坁’,意思是小山丘。
(9)系:捆绑。徽纆(mo):绳索。三股叫徽,两 股叫纆。
(10)丛棘:这里指监狱。古代监狱外面围上荆棘,以防犯人逃跑, 所以用“丛棘”代指监狱。

【译文】

习坎卦:抓到俘虏。用好话劝说他们,亨通。路途中遇到帮助。
初六:坎坑重坎坑,陷入重坑之中。凶险。
九二:坎坑有危险,为了小收获只得冒险。
六三:来到坎坑,坎坑又险又深。陷入重坑之中,非常不利。
六四:用陶樽陶簋装酒饭,关在坎窖中的俘虏从窗户拿进送出。结果没有危险。
九五:坎坑没有被填满,小山丘被挖平了。没有灾祸。
上六:用绳索把犯人捆住,关进四周有丛棘的监狱中,多年还不能使犯人屈服。凶险。

【赏析】

社会转型期的巨大动荡和变迁,必然会给个人的命运带来深刻影响。有人一夜之间暴发,由平民、奴隶变为拥有权势和财富的显贵;有人转瞬间由贵族沦为奴隶、阶下囚。江山财富的更迭 转移,个人命运的沉浮,不能不使人感到人生社会之路的艰险坎坷,命运的变幻无常和不可捉摸。即使是无处不在、无所不知的神灵,恐怕也难以解答这一人生的难题。

一个又一个的陷阱,一个又一个的坎坑,似乎是人生之旅的最好说明。做个日出而作日落而息的平民百姓,可能遇到的陷研坎坑会少一点;若想满足权欲物欲财欲而出人头地,难以逃脱陷阶圈套和风口浪尖的摔打。《红楼梦》中说,“因嫌纱帽小,致使 锁枷扛”,算是悟透了人生的这一真谛。

涉足官场、战场、商场、情场等等,人的自我由于众多掣肘的因素而变得身不由己,自我日渐消失,日渐物化,成为被算计、 陷害、剥夺。吞噬、压榨、谋杀的对象。能挺过来,就是好汉;挺不过来,就是牺牲品,就是鱼虾。这样说来,明知山有虎,还是不向虎山行为好。

\chapter{离卦}
\Bagua{000000}[3] \ (坤为地)坤上坤下
(离为火)离上离下
《离》:利贞。亨。畜牝牛吉。
初九,履错然,敬之无咎。
六二,黄离,元吉。
九三,日昃之离,不鼓缶而歌,则大耋之嗟,凶。
九四,突如,其来如,焚如,死如,弃如。
六五,出涕沱若,戚嗟若,吉。
上九,王用出征,有嘉折首,获匪其丑,无咎。

【注释】

①离是本卦的标题。离的意思是“罹”,即遭遇灾祸。全卦内容主要讲战祸,标题与内容有关。
②履:步履,这里指脚步声。错然:杂乱的样子。
③敬;用作“儆”,意思是警戒。
④离:这里用作“螭”,意思是龙, 指天上像龙形的云、虹,即霓。黄离就是黄霓。
⑤昃(ze):太阳偏西。
(6)缶:陶制的乐器。
(7)大耊(die):老头儿。七十岁叫耊。
(8) 弃:使……变成废墟。
(9)涕:眼泪。沦若:泪如雨下的样子。
(10)戚: 忧伤。嗟:叹息。
(11)有嘉:周代的小国嘉。折首;意思是斩首。
(12) 匪:用作“彼”。丑:众,这里指敌方。

【译文】

离卦:吉利的卜问,亨通。饲养母牛,吉利。
初九:听到错杂的脚步声,马上警惕戒备,没有灾祸。
六二:天空中出现黄霓,是大吉大利的征兆。
九三:黄昏时天空出现虹霓,人们齐声高叫,没有唱歌时的乐器伴奏,老人们悲哀叹息。这是凶兆。
九四:敌人突然袭击,见房就烧,见人就杀,使这里变成一片废墟。
六五:泪如雨下,忧伤叹息。吉利。
上九:在王的率领下反击敌人,将有嘉国君斩首,抓获了很多俘虏。没有灾祸。

【赏析】

这里描述的是一场自卫反击战,从保持警惕,敌人突然袭击, 到国王率众反击,大获全胜。天象显然是战争中的重要因素,吉、 凶征兆交替出现,似乎是天意的显现,结果也应验了预兆。其次是战争的残酷。发动突然袭击的敌人是强悍的,并且毫不留情地烧光、杀光、抢光;罹难的民众虽然难以抵挡强敌,却也演出了一出悲壮的场面,齐声高叫,泪如雨下,忧心叹息;再次是国王的英明勇敢,消灭了敌国,铲除了心腹之患。

一场残酷的战斗似平显得那么简单:没有挖空心思的谋略,没有复杂的战略战术,也没有相持不下的反复争夺。但是,那浓厚的血腥味却是透过了纸背久久不散,“三光”的情景如在目前。原始的战术凭借的是体力的强悍,而不是复杂的计谋和精良的武器, 还有古人笃信的上天的意向。

比较之下,现代战争无论在哪个方面都发生了翻天覆地的变化。没有变的是血腥和残酷,以及用它们来换取自己的利益。

\part{易经下}

\chapter{咸卦}
\Bagua{000000}[3] \ (坤为地)坤上坤下
(泽山咸)兑上艮下
《咸》:亨。利贞。取女吉。
初六,咸其拇。
六二,咸其腓,凶。居吉。
九三,咸其股,执其随,往吝。
九四,贞吉。悔亡。憧憧往来,朋从尔思。
九五,咸其脢,无悔。
上六,咸其辅颊舌。

【注释】

咸是本卦的标题。咸的意思是受伤。全卦的内容主要是就梦中所见的 生活琐事进行占问。标题的“咸”字是卦中多见字。
取:用作“娶”。
拇:大脚趾。
腓(fei):小腿肚子。
执:同“咸”,意思 也是受伤。随:用作“隋”,指股下隆起的肉。
憧憧:即童童,意思是 往来不绝的样子。
朋:朋贝,货币。
脢(mei):背上的肉。
辅:意思是牙床骨。

【译文】

咸卦:亨通,吉利的占问。娶女为妻。吉利。
初六:脚大拇趾受了伤。
六二:小腿肚子受了伤,凶险。定居下来,吉利。
九三:大腿和大腿下部的内受了伤。伤后出行,会遇困难。
九四:占问吉利,没有悔恨。人来人往,实现了赚钱的愿望。
九五:背上受了伤,没有悔恨。
上六:牙床骨、面颊和舌头都受了伤。

【赏析】

身体的某个部位受伤,今天在我们看来不足为奇,古人却相信网运气的吉凶有必然联系,尤其是在梦中出现,就更不是偶然的了,所以当然得向神灵占问一下。其中很难说有什么深奥的秘 密或微言大义。

日常生活中难免有磕磕碰碰不留神的时候。倘若事无巨细都得去深入挖掘某种深层原因,那日子恐怕就过得太累了,甚至到疑神疑鬼的地步,简直会寸步难行。当然,我们无意用我们今天的看法去指责苛求古人;他们自有他们的道理。或许可以说,梦兆这玩意儿,心诚笃信就灵,不信则不灵。

不过,心理分析学家们可能有充分的论据来反驳,证明梦兆的深层心理根源。但对普通百姓而言,不大容易懂得那些过分专门的理论,宁可相信自己的经验。经验显然更可靠一些。

\chapter{恒卦}
\Bagua{000000}[3] \ (坤为地)坤上坤下
(雷风恒)震上巽下
《恒》:亨。无咎。利贞。利有攸往。
初六,浚恒,贞凶,无攸利。
九二,悔亡。
九三,不恒其德,或承之羞,贞吝。
九四,田无禽。
六五,恒其德,贞,妇人吉,夫子凶。
上六,振恒,凶。

【注释】

①恒是本卦的标题。恒的意思是久常。全卦内容是日常生活和生产上的事。作标题的“恒”字是卦中多见词。
②浚:挖土。
③德:用作“得”,指收获。
④承:奉送。羞:即“馐”,意思是美味。
⑤振:振动,动荡。

【译文】

恒卦:亨通,没有灾祸,吉利的占问。有利于出行。
初六:挖土不止。占问凶兆,没有什么好处。
九二:没有什么可悔恨。
九三:不能经常有所获,有人送来美味的食物。占得艰难的征兆。
九四:田猎打不到禽兽。
六五:经常有所获。占问结果,女人吉利,男人凶险。
上六:动荡不止。凶险。

【赏析】

希望过上好日子,并且希望好日子长久保持下去,这是人们最普遍、最朴素的愿望,理所当然要占问神灵这一基本愿望能否实现。 ”

愿望是美好的。而现实却是严峻的,日子并不好过。有天灾,天旱水涝,火患虫害,雷电霜雪,对农业生产、居家度日、出门经商都会构成威胁。也有人祸,朝政动荡,兵匪战乱,盗贼抢劫,平民百姓终日提心吊胆,生活难以为继。所以,古人的生活固然有田园牧歌的一面,而更多的却是生产繁忙,生活艰辛。

其实,任何时代的生活都有各自的难题和烦恼,哪里有恒常不变的舒适美满日子。现代人虽然在物质方面远胜于古人,却面对着物对人的异化和吞噬,日子同样不好过。但是,疲惫的现代人已没有了向上苍祈祷的虔诚了。

\chapter{遯卦}
\Bagua{000000}[3] \ (坤为地)坤上坤下
(天山遯)乾上艮下
《遯》:亨。小利贞。
初六,遯尾,厉,勿用有攸往。
六二,执之用黄牛之革,莫之胜说。
九三,系遯,有疾厉,畜臣妾吉。
九四,好遯,君子吉,小人否。
九五,嘉遯,贞吉。
上九,肥遯,无不利。

【注释】

①遯(dun)是本卦的标题。遯是遁的异体字,意思是隐退。全卦的内容与政治斗争有关。避是卦中多见词,又与内容有关,所以用作标题。
② 尾:全部,尽。
③执:抓住捆绑。
④胜:可能。说:用作“脱”。
⑤系:拖累,拘系。
(6)畜:豢养。臣妾;家奴。
(7)好:喜好,喜欢。
(8)嘉:赞美。
(9)肥:用作“飞”。肥论的意思是远走高飞。

【译文】

遯卦:亨通。有小利的占问。
初六:君子全部隐退,危险。不利于出行。
六二:用黄牛皮绳把马绑住。它不可能逃脱。
九三:羁系住隐退者,他心里很痛苦,危险。豢养奴婢,吉利。
九四:喜欢隐遁,这对贵族君子是吉利的,对小人则不利。
九五:赞美隐遁,占得吉兆。
上九:远走高飞隐道起来,没有什么不利。
九五:赞美隐遁,占得吉兆。
上九:远走高飞隐道起来,没有什么不利。

【赏析】

中国历来的君子、士大夫的人生之途,总在步入官场和归隐山林之间像荡秋千一样地来回摆动;他们的人生选择,似乎就只有这两个“对立的极”。一切都是生而注定了的,别无选择:要么在官场如鱼得水、志得意满,要么在山林放浪形骸、怡清悦性。表面看来,这两极好像互不相融,实质上中心依然是官场庙堂。人生最大的前途和成就,是立功、立德、立言,个人存在的价值和 意义就在为这“三不朽”而奋斗。归隐有时是不得已而为之,有 时是一种暂时的策略和手腕,有时是作为失意之后的一种心理补偿与心理平衡。真正志在山林做一个今天时髦的“自由知识分 子”的人,实在太少,即使陶渊明在“采菊东篱下,悠然见南 山”的时刻,恐怕也还有一只眼在偷窥着庙堂呢。

不过,单就山林本身而言,它确实对有较高心性修养的君子有着相当的诱惑力,中国传统文化中也有一整套对此大加赞赏的理论。“遯卦”所言,也可看作是这方面的先声。

\chapter{大壮卦}
\Bagua{000000}[3] \ (坤为地)坤上坤下
(雷天大壮)震上乾下
《大壮》:利贞。
初九,壮于趾,征凶,有孚。
九二,贞吉。
九三,小人用壮,君子用罔,贞厉。羝羊触藩,羸其角。
九四,贞吉,悔亡。藩决不羸,壮于大舆之輹。
六五,丧羊于易,无悔。
上六,羝羊触藩,不能退,不能遂,无攸利,艰则吉。

【注释】

①大壮是本卦的标题。壮的意思是强健,伤。全卦的内容主要与畜牧有关。标题的“壮”字是卦中多见词。
②壮:用作“戕”,意思是伤。
③壮:强壮,力大。
④罔:古“网”字,用来捕兽的工具。
⑤羝 (dT)羊:公羊。藩;篱笆。
(6)羸(lei):用作“累”,意思是用绳子捆住。
(7)决:破。
(8)輹:用作“辐”,这里指车轮。
(9)易:用作“场”,指放牧的牧场。
(10)遂:进。

【译文】

大壮卦:吉利的占问。
初九:脚趾受了伤。出行,凶险。有所收获。
九二:占得吉兆。
九三:奴隶狩猎凭力大,贵族狩猎用猎网。占得险兆。公羊用头角撞篱笆,却被篱笆卡住了。
九四:占得吉兆,没有悔恨。公羊撞破篱笆,摆脱了羁绊,又撞在大车轮子上受了伤。
六五:羊在牧场上逃掉了。没有悔恨。
上六:公羊用头角撞篱笆,角被卡住,退不了,进不了。没有什么好处,占问旱情则得吉兆。

【赏析】

《周易》一再讲到狩猎驯养的情景,可见这在当时的社会经济 中仍占有相当重要的地位。倘若这是社会生活的真实反映的话,那就表明周代尚处在由狩猎社会向农业社会过渡的阶段。狩猎所获, 或用作食物,或用来驯养(吃不完时养起来供以后食用)。这两种情况,《周易》都说到了。另一方面,从所记农业生产的情形看,显然还比较原始,而且经常发生抢夺粮食的暴力事件。这说明农 业生产还未成为人们生活的主要来源,还得靠狩猎和畜牧作为重要的补充。

周人的发祥地在今天的陕西一带。今昔对比,沧海桑田的巨变给人的感慨不知是忧是喜。今日的陕西,恐怕找不到一个地方 可以狩猎,找不到一处有茂密的原始森林,找不到一处可以开垦的处女地……今日的陕西,给人最突出的印象是黄土地、浑浊的黄河、信天游、窑洞和古铜色的黄皮肤。俱往矣,昔日肥沃繁茂的草原,今朝已变为现代化的钢筋水泥丛林和网络。

\chapter{晋卦}
\Bagua{000000}[3] \ (坤为地)坤上坤下
(火地晋)离上坤下
《晋》:康侯用锡马蕃庶,昼日三接。
初六,晋如摧如,贞吉。罔孚,裕无咎。
六二,晋如,愁如,贞吉。受兹介福于,其王母。
六三,众允,悔亡。
九四,晋如鼫鼠,贞厉。
六五,悔亡,失得,勿恤。往吉,无不利。
上九,晋其角,维用伐邑,厉吉,无咎,贞吝。

【注释】

①晋是本卦的标题。晋的意思是前进,指作战中的进攻。全卦的内容主 要讲战争。“晋”字既与内容有关,又是卦中多见词,所以用作标题。
②康侯:指周武王的弟弟康叔封。锡:用作“赐”,意思是赐予。蕃庶:繁育,繁殖。
③昼日:终日,一整天。三接:指多次交配。
④摧:摧毁,打 垮。
⑤罔:无。孚:抓,抢夺。裕:这里指财物。
(6)愁:用作 “遒”,意思是迫使投降。
(7)兹:此。介:大。
(8)王母:祖母。
(9)允:用作“郓”,意思是进,这里指进攻。
(10)鼫(shi)鼠:这里用来 形容胆小如鼠。
(11)失得:失败,失利。恤:担忧,气馁。
(12)其:则。 角:较量。
(13)维:考虑。

【译文】

晋卦:康侯用周成王赐予他的良种马来繁殖马匹,一天配种多次。
初六:进攻打垮敌人、占得吉兆。没有抢夺财物,没有灾祸。
六二:进攻迫降敌人,占得吉兆。获得这样的福祐,是受了祖母的庇护。
六三:万众进攻,没有悔恨。
九四:进攻时胆小如鼠,占得凶兆。
六五:没有悔恨,即使战败也不气馁。前进,吉利。没有什么不利。
上九:进攻敌人必须较量力量,可以考虑攻打敌方城邑。凶险,吉利,没有灾祸,占得险兆。

【赏析】

一而再,再而三地写战争,除了证明这一“王者之事”的重要外,也说明远古战争的频繁,几乎就像家常便饭,只要,心血来潮,就可以大动干戈,不顾百姓奴隶的死活,不管对生产生活造成的劫难。

战争的动因和目的,在那时不外乎攻城掠地,抢劫财物,抓获俘虏作奴隶和献祭的牺牲品,或者是镇压统治集团内部的异己势力。这种带有原始暴力色彩的战争,很难说有什么正义和非正义之分,全是“肉食者谋之”的事情,给平民百姓带来的结果除了灾难之外,没有任何好处。

可是,后来的所谓“史家”总要为某某君王讨伐某人的战争找出种种赞美的理由,竭力夸大某些并非真实的因素。“成者为王, 败者为寇”,几乎成了中国传统史家的心理定势。这个逻辑的实质,便是对强权、暴力和专制的顶礼膜拜,为战争贩子带上胜利的花冠。

今天来思考战争,应当完完全全跳出这种巢臼,站在平民百姓的立场上,从社会经济稳定繁荣的全方位角度来提出问题,才符合人类社会发展的大趋势。

\chapter{明夷卦}
\Bagua{000000}[3] \ (坤为地)坤上坤下
(离下坤上)明夷,利艰贞。
初九:“明夷于飞,垂其翼。君子于行,三日不食。”有攸往, 主人有言。
六二:明夷,夷于左股,用拯马壮。吉。
九三:明夷于南狩,得其大首。不可疾贞。
六四:入于左腹,获得夷之心,于出门庭。
六五:箕子之明夷,利贞。
上六:不明,晦。初登于天,后入于地。

【注释】

(1)明夷是本卦的标题。明夷在卦中有三种意思:一指鸣,即叫着的鹈鹕;二指鸣响的弓;三指太阳落下。全卦内容讲出行狩猎和隐退守洁,用多 见词作标题。
(2)明夷于飞:这里引用一首民歌作占,叫做谣占,在这里用来说明行旅之难。明夷:用作‘鸣夷’,意思是叫着的鹈鹕。鹈鹕是一种水 鸟,俗称淘河。
(3)言:指责,责难。
(4)明:这里指太阳。夷:来。明 夷就是太阳下山。
(5)夷:用作“痍”,意思是受伤。
(6)用:因为。拯: 得救。用拯马壮:意思是说因为壮善跑而得救。
(7)明夷:这里指鸣弓, 意思是说拉弓发射。南狩:南方的猎区。
(8)大首:大头,指大头的猛兽。
(9)可:利。
(10)腹:,古代半地下式房屋的复室。左腹就是左室,这里指隐居之处。
(11)明夷:太阳隐去,这里的意思是说隐退。
(12)箕子:殷纣王的哥哥。明夷:这里指隐退。

【译文】

明夷卦:有利于占问艰难的事。
初九:”鹈鹕在飞行,垂敛着羽翼。君子在旅途,多日无食粮。”前去的地方,受到主人责难。
六二:太阳下山的时候,左腿受了伤,因马壮得救。吉利。
九三:在南边的猎区拉弓射箭,猎获了大猛兽。不利于占问疾病。
六四:进入隐居之处,产生了归隐的念头,一出门就想返回。
六五:殷纣王的哥哥箕子到东方邻国去避难,吉利的占问。
上六:太阳下山,天黑了。太阳初升是天明,后来下山是天黑。

【赏析】

借日出日落、天明天黑,来表达君子出行时的内心体验,尤其突出了出行途中的艰难境遇:饥肠辘辘,房东的刁难,身体的伤病,油然而生归隐之心。当然,也有顺利之时:狩猎时所获甚丰。但全卦的语调却在突出行旅的艰难和归隐之思。

在这种行路难的倾诉之中,我们可以清晰地分辨出一种疲惫感无奈感。人生路漫漫,日出复又入,何时有尽头,何处是归宿?这种感慨之中,显然包含有对人生意义的形而上的追求和悲观的色彩,包含有对个人存在价值的内存关注。

人生固然是一场战斗,为了功名利禄、家国妻儿;但得到之后又怎样,自己的位置在哪里,为什么总得去获取,为什么不停下来抚慰创伤、静心思虑?太阳再辉煌也有消失的时候,事业再 辉煌同样有难以为继的时候,人生再顺畅照样免不了灾祸。这一切真是剪不断理还乱的思绪。不可能不去想,却又永远不出答案。

是呵,不如归去,隐没到与世隔绝的世外桃源,不看不想无欲无求无牵无挂,岂不是大好大了的境界!古往今来,有几人真正识透了个中妙谛?

归隐,大概也该算是人生“永恒主题”之一吧。

\chapter{家人卦}
\Bagua{000000}[3] \ (坤为地)坤上坤下
(风火家人)巽上离下
《家人》:利女贞。
初九,闲有家,悔亡。
六二,无攸遂,在中馈,贞吉。
九三,家人嗃□,悔厉吉;妇子嘻嘻,终吝。
九四,富家,大吉。
九五,王假有家,勿恤,吉。
上九,有孚威如,终吉。

【注释】

①家人是本卦的标题。家人的意思就是家庭。全卦专门讲家庭中的事,标题与内容有关。
②闲:防范。有:于。
③遂:用作“坠”,意思是失 误。
(4)中馈:家庭中的饮食之事。
⑤嗃嗃(he):用作“嗷嗷”意思是 众口愁叹。
(6)嘻嘻:笑声。
(7)富:用作“福”,意思是幸福。
(8) 假:用作“格”,意思是到达。有:于。家:这里指祭把祖先的家庙。
(9) 罕:俘虏。威如:发怒的样子。

【译文】

家人卦:有利于妇女的占问。
初九:提防家里出事,没有悔恨。
六二:妇女在家中料理家务,没有失职。占得吉兆。
九三:贫困之家哀号愁叹,嗷嗷待哺,有悔有险,但终归吉利。富贵之家嘻笑作乐,骄奢淫逸,结果要倒霉。
六四:幸福的家庭大吉大利。
九五:君王的家庙中祭祝祖先,不必忧虑。吉利。
上九:抓到的俘虏不肯屈服,发怒反抗,结果还是吉利。

【赏析】

这一卦专讲家庭之事,看来作者并未忽略家庭这个“社会细胞”。事实上,家庭结构,血缘关系,正是构成中国传统宗法社会的根本所在,想必作者深知这一点,才辟出专卦来谈论。

引人注目的是说这是对妇女有利的,即把妇女的地位和作用定位在家庭之中。她们不是一家之长,仅仅是专门负责料理家务(大概也包括生儿育女吧),无缘参与社会事务。这就是中国传统对妇女角色的定位。由此形成的结果是:家庭之中没有妇女是不行的,家中的大小事情要由妇女来操持;但是妇女在家中是被领导者,无权作出决定。一个好女人的标准是服从文夫,孝敬公婆,养育子女,安分守己地做家务。

按这样的标准,“幸福的家庭都是相似的;不幸的家庭各有各的不幸。”托尔斯泰《安娜·卡列尼娜》开篇所讲的这句话,肯定适用于中国古代的家庭情况。在今天看来,那时的幸福家庭一定沉闷得令人窒息,压抑得令人难以容忍——至少对妇女们来说是如此,因为无论她们怎样聪明能干,都只是奴隶般的角色。

\chapter{睽卦}
\Bagua{000000}[3] \ (坤为地)坤上坤下
(火泽睽)离上兑下
《睽》:小事吉。
初九,悔亡。丧马勿逐自复。见恶人无咎。
九二,遇主于巷,无咎。
六三,见舆曳,其牛掣,其人天且劓,无初有终。
九四,睽孤遇元夫,交孚,厉,无咎。
六五,悔亡。厥宗噬肤,往何咎?斋
上九,睽孤见豕负涂,载鬼一车,先张之弧,后说之弧,匪寇,婚媾。往遇雨则吉。

【注释】

①睽(kui)是本卦的标题。暖的意思是相违,矛盾。全卦记述旅人出行途中所见所闻,像一篇旅行日记。作标题的“睽朕”字与内容有关。
②复: 返回。
③舆:大车。曳:拖拉。
④掣:意思是牛角一俯一仰,形容牛拉车很吃力的样子。
⑤天:用作“颠”,意思是额部,这里 专指一种在额上刺字的刑罚。劓(yi):割掉鼻子(一种刑罚)。
(6)睽:乖离,这里指外出的旅人。跃孤的意思是说旅人孤单行路。
(7)元夫:元用作“兀”,元夫就是跛子。
(8)交:一起,全部。
(9)厥:其,这里指代旅人。厥宗:跟他同宗族的人。噬:吃。肤:这里指肉。
(10)豕:猪。涂: 泥巴。负涂:背上有泥。
(11)鬼:这里指用图腾打扮的人。
(12)张:拉开。弧:弓。
(13)说:用作“脱”,这里的意思是放下。

【译文】

睽卦:小事吉利。
初九:没有悔恨。马跑掉了,不必去追,它自己会回来。途中遇到容貌丑陋的人,没有灾祸。
九二:刚进小巷就遇到主人接待,没有灾祸。
六三:看到一辆拉货的车,拉车的牛很吃力,一步一使劲,牛角一俯一仰的,赶车的人是个被烙了额、割掉鼻的奴隶。开始时拉不动,最后拉走了。
九四:旅人孤身赶路,遇到一个踱子,同他一起被抓住。危险,结果却没有灾难。
六五:没有悔恨。看见同宗族的人在吃肉。往前走去,哪有什么灾祸?
上九:旅人孤身赶路,看到一头猪满身是泥,一辆大车载满了图腾打扮的人。他们起初拿起弓箭要射,后来放下了。这些人不是来抢劫,而是去订婚。旅人继续前行,虽然遇到下雨,但平安吉利。

【赏析】

我们不知道这位旅行者姓甚名谁,更不知道他此行的目的和终点。他给我们的感觉是心情悠闲轻松,虽有心情紧张的时候,但大体上是无忧无虑的,自在的,并且一路顺利。因此,他才有闲 情逸致记下所见所闻。

他边走边看,像旁观者。他也像导游或电影导演,通过他的眼睛的选择,引领我们同他一起去观看旅途的景象:丢失马匹,容貌丑陋的人,投宿顺利,拉车的牛和赶车的奴隶,弧身赶路被俘, 同族人吃肉,满身是泥的猪,订婚的一群人,遇雨却平安。 这里没有文学夸张,更没有潜台词和微言大义;匆宁说,它简直像二次大战后电影中的新现实主义所用的长镜头,跟踪一桩生活中发生的事件,力图真实地记录下来。看不少作者的倾向性,没有任何评点、议论,朴实得如同生活本身。

好处和价值也正在这里。让我们自己去感受,用自己的生活体验去充实其中的细节,去揣度人物的心理,去体味古人的生存状况。因此,它比诗更有诗意,比散文更精炼。同时,我们不要忘了,悠闲的心境和细致的体察,是产生这篇日记的关键因素。

\chapter{蹇卦}
\Bagua{000000}[3] \ (坤为地)坤上坤下
(水山蹇)坎上艮下
《蹇》:利西南,不利东北。利见大人。贞吉。
初六,往蹇来誉。
六二,王臣蹇蹇,匪躬之故。
九三,往蹇来反。
六四,往蹇来连。
九五,大蹇朋来。
上六,往蹇来硕,吉,利见大人。

【注释】

①蹇(jian)是本卦的标题。蹇的意思是艰难。全卦的内容是通过商旅来说明由难变为不难的道理。作标题的“赛”字既与内容有关,又是卦中的多见词。
②誉:用作“趋”,意思是安全舒适地行路。
③蹇蹇:难上加难。
④躬:自己。
⑤反:意思是高兴快乐。
(6)连:用作“辇”,意思是车。
(7)朋:朋贝,货币。
(8)硕:用作“蹁”,意思是跳跃,这里用来说明高兴。

【译文】

蹇卦:往西南方走有利,往东北方走不利。有利于会见王公贵族。占得吉兆。
初六:出门时艰难,回来时安适。
六二:王臣的处境十分艰难,不是他自身的缘故。
九三:出门时艰难,回来时快乐高兴。六四:出门时艰难,回来时有车可坐。
九五:经历了许多艰难,最终赚钱获利。
上六:出门时艰难回来时欢喜跳跃。吉利。有利于见到王公贵族。

【赏析】

作者一再想告诉我们的是:难与不难是可以互相转化的;此时艰难,彼时却是兴高采烈、手舞足蹈;此时身处逆境,彼时却可能飞黄腾达。以商人经商营利为例,要想赚钱赢利,就得投入 资金和精力,历经辛劳,才会有所收获。

这道理再明白不过了。天上掉不下馅饼,不开荒耕种就没有粮食吃,舍不得孩子就打不到狼,不入虎穴焉得虎子,我们的老祖宗反复申说,我们子子孙孙真的铭记在心,并身体力行,用实践来证明这个真理。子子孙孙传下来,铸就了中国人特别能吃苦耐劳的民族性格。这一点早有定论,无可非议。

有意思的是,《周易》的作者似乎不像后来的君子大人们对经商赢利抱有成见,不说“无商不奸,无奸不商”一类糟踏商人的话,倒显得十分看重商人的活动、这种态度,恐怕比后来解说 《周易》的儒生们要豁达宽容得多,也许还能适合社会主义市场经济的情形?

\chapter{解卦}
\Bagua{000000}[3] \ (坤为地)坤上坤下
(雷水解)震上坎下
《解》:利西南。无所往,其来复吉。有攸往,夙吉。
初六,无咎。
九二,田获三狐,得黄矢,贞吉。
六三,负且乘,致寇至,贞吝。
九四,解而拇,朋至斯孚。
六五,君子维有解,吉,有孚于小人。
上六,公用射隼于高墉之上,获之,无不利。

【注释】

①解是本卦的标题。解的意思是分解,解除。全卦内容主要讲商旅、狩猎和俘虏。标题的“解”字为卦中多见词。
②夙:早。
③黄矢:铜 箭头。
④解:用作“懈”,意思是懈怠。拇:脚大拇趾,这里代指脚。解 而拇:意思是说不想走路。
⑤朋至:获得朋贝,赚了钱。斯:则。
(6)维:系,束缚。有:又。解:解开,松开。
(7)罕:惩罚。
(8)公:这里指贵族。隼(sun):鹰。墉:城墙。

【译文】

解卦:往西南方走有利。如果没有明确的目的地,不如返回来,吉利。如目的明确,早去吉利。
初六:没有灾祸。
九二:田猎获得三只狐狸,身上带着铜箭头。占得吉兆。
六三:带着许多货物,背负马拉,惹人汪目,结果强盗来了。 占得险兆。
九四:赚了钱而懈怠不想走,却被人抓去。
六五:君子彼捆起后又被解开,吉利。小人将受到惩罚。
上六:王公贵族在高高的城墙上射中一只鹰,并抓住了。这没有什么不吉利。

【赏析】

这里记录的又是一次商旅经历。这次商人遭遇歹徒和拘禁,最终逢凶化吉,满载而归。

商人的运气似乎特别好,总有神灵在庇拓他们,差不多像希腊神话传说中的商业和盈利幻中赫尔墨斯,虽然历经艰辛,却始终是福星高照,次次走运。 他们的日子显然比开荒种地的农民好过得多。他们可以外出旅行,边走边看,见多识广,旅游赚钱兼得。他们生活的内容没有农民那么枯燥,甚至在行旅途中还可以打猎--既是娱乐,又可以吃野味。

用现在时髦的话来说,他们活得很溃洒自在。他们是社会生活中必不可少的重要角色,经济机器中的润滑剂。

\chapter{损卦}
\Bagua{000000}[3] \ (坤为地)坤上坤下
(山泽损)艮上兑下
《损》:有孚,元吉,无咎。可贞,利有攸往。曷之用?二簋可用享。
初九,已事遄往,无咎。酌损之。
九二,利贞。征凶,弗损,益之。
六三,三人行则损一人,一人行则得其友。
六四,损其疾,使遄有喜,无咎。
六五,或益之十朋之龟,弗克违,元吉。
上九,弗损,益之,无咎,贞吉,利有攸往,得臣无家。

【注释】

①损是本卦的标题。损的意思是减损。全卦的内容是说明损与益两个对 立方的关系。标题的“损”字是卦中多见词。
②曷:用作“瞌”,意思 是送食物。簋(gui):装饭的器物。
③享:宴享,祭享。
④已:用 作“祝”,意思是祭耙。遄(chuan):快,速。
⑤损:减轻,消除。
③使:使人祭把。有喜:这里指病愈。
(6)益之:送给。朋:朋贝,货币, 十枚一串贝为朋。
(7)违:离去。
(8)弗损益之:意思是说不减少不增 加。
(9)臣:奴隶。家:家人。无家:没有家人,意思是说单身汉。

【译文】

损卦:获得俘虏,大吉大利,没有灾祸,如意的占问。有利于出行。有人送来两盆食物,可以用来宴享。
初九:祭祝是大事,要赶快去参加,才没有灾祸。但有时可酌情减损祭品。
九二:吉利的占问。出讨他国,凶险。有时不能减损,要增益。
六三:三人同行必有一人因看法不一而被孤立,一人独行遇人可以作伴。
六四:减轻疾病,要赶快祭神,才会病愈,没有灾祸。
六五:有人送给价值十朋的大龟,不能不要。大吉大利。
上九:不增不减,完全依旧。没有灾祸,占得吉兆。有利于出行,可以获得单身奴隶。

【赏析】

损和益,一减一增,被当作两个相互联系的方面,既是对立的,又可以相互转化;或减或增,或减中有增,增中有减,或不增不减。如何取舍,如何抉择,没有固定不变的模式,要依据具体情况灵活运用。

在对立双方中寻求一个恰到好处的度,是我们的祖先喜欢并擅长的思维方式和处世为人的态度。他们从不走极端,从不抓住一方面紧紧不放,也不习惯穷很究底或以毒攻毒,雪上加霜。这让人不由自主联想到走钢丝:主旨是保持平衡状态,稳住不致掉下来;向左边斜了就往右一点,向右边歪了就往左倾一些,不偏不倚正是所需的“度”。

国人传统的智慧、技巧、知识都用在了保持这个恰到好处的度上。人生的过程仿佛就是一个走钢丝的过程,虽然很累,但四平八稳,没有大起大落,大灾大难,于是就心安理得了。

平衡,在很大程度上是保守的。它排斥进取、冒险、冲刺、拼搏,固守自我封闭的心态,好静不好动,使人老态龙钟,生气全无。

\chapter{益卦}
\Bagua{000000}[3] \ (坤为地)坤上坤下
(风雷益)巽上震下
《益》:利有攸往。利涉大川。
初九,利用为大作,元吉,无咎。
六二,或益之十朋之龟,弗克违。永贞吉。王用享于帝,吉。
六三,益之用凶事,无咎。有孚。中行告公用圭。
六四,中行告公,从,利用为依迁国。
九五,有孚惠心,勿问,元吉。有孚,惠我德。
上九,莫益之,或击之,立心勿恒,凶。

【注释】

①益是本卦的标题。益的意思是增益。全卦内容是说明损益的道理,与“损卦”构成一个组卦。标题的“益”字是卦中多见词。
②用:于。大作:大兴土木,建筑。
③益之:这里的意思是祭把时增加人牲。用:因为。凶事:丧事,这里指周武王去世。
④中行:中途。用圭:指祭把,因为祭把时要执畦(即圭),所以用来代指。
⑤依:即殷,指殷代。
(6)惠心:安抚,好心。勿问:不必追问。
(7)德:用作‘得”,这里指所获财物。
(8)莫:没有人。益:帮助。
(9)击:攻击。
(10)恒:长久,坚持不变。

【译文】

益卦:有利于出行。有利于渡过大江大河。
初九:有利于大兴土木。大吉大利,没有灾祸。
六二:有人送给价值十朋的大龟,不能不要。占得长久吉兆。周武王克商,祭祝天帝,吉利。
六三:因武王去世,祭祝时增加人牲,没有灾祸。抓到了俘虏,中途报告周公举行祭把。
六四:东征胜利后,班师回来的路上报告周公成王有命,把殷商遗民处理好有利。
九五:抓到俘虏,好心待他们,不必追究。大吉大利。抓到俘虏,用财物优待使他们对我感激。
上九:没有人帮助,还有人来攻击。这时内心不坚定,必然凶险。

【赏析】

为了与“损卦”对应,“益卦”进一步用周王朝由盛到衰、行将危亡的历史事实来阐发益损相互转化的道理,意在告诫周朝统治者及时采取措施,防止周朝的衰亡。

作者的良苦用心确实可以理解,说得深刻在理。然而,我们经常见到的情况是:道理与事实的背离,人的意愿与历史发展的趋势相左,以及统治者专断粗暴或刚愎自用或利令智昏奢靡淫逸而断送家国。

在中国传统社会中,帝王专制制度决定了真理的力量十分有限,起决定作用的政治因素是统治者个人意志。不受任何约束的帝王权力可以无限膨胀,道理是否起作用,完全取决于统治者的个人好恶。正如我们看到的,无论《周易》的作者把损、益的道理讲得多么深刻,仍然没能挽救周王朝的衰亡。

这使我们想到:真理要真正发挥作用,必须同权力(包括各 种形式:法律的,政治的,舆论的,制度的等等)相结合,以强制的或公众的形式诉诸于个人,而不是相反。从这个意义上来理解“从来就没有什么救世主”,恐怕更为深刻。

\chapter{夬卦}
\Bagua{000000}[3] \ (坤为地)坤上坤下
(泽天夬)兑上乾下
《夬》:扬于王庭,孚号。有厉,告自邑。不利即戎,利有攸往。
初九,壮于前趾,往不胜,为咎。
九二,惕号,莫夜有戎,勿恤。
九三,壮于頄,有凶。君子夬夬独行,遇雨若濡,有愠无咎。
九四,臀无肤,其行次且。牵羊悔亡,闻言不信。
九五,苋陆夬夬中行,无咎。
上六,无号,终有凶。

【注释】

①夬(guai)是本卦的标题。夬是“快”的本字,有快乐和快速两种意思。全卦内容主要讲防范敌人和行旅。标题取“夬”的字义。
②扬:拿着兵器跳的武舞。
③孚号:呼号。有厉:有敌人来侵犯。
④即戎:马上进行防御。
⑤壮:受伤。
⑥惕号:惊恐呼号。
⑦莫:“暮”的本字,意思是太阳下山。
⑧馗(qiu):颧骨。
⑨夬夬:急匆匆的样子。
(10)若:而。儒:淋湿。
(11)愠:不高兴,不满。
(12)肤:肉。臀无肤:这里是说臀部受了伤。烟次且:用作“越趄”,意思是走路很困难的样子。
(14)闻:用作“问”。言:用作“愆”,意思是亏损。信:申辩,说明。
(15)苋:细角山羊。陆:意思是蹦跳。中行:路中间。
(16)无:应为“犬”字。

【译文】

央卦:王庭中正在跳舞取乐,有人呼叫“敌人来犯”。邑中传来命命:“不利出击,严密防范。”有利于出行。
初九:脚趾受了伤,再前往,脚力不胜将遭难。
九二:有人惊呼,夜晚敌人来犯,但不必担心。
九三:颧骨受了伤,凶险。君子独自匆匆赶路,遇到下雨淋湿了全身,很不高兴,但没有灾祸。
九四:臀部受了伤,走起路来十分困难。牵羊去做买卖,悔恨羊丢失了,问怎么丢的,却说不清楚。
九五:细角山羊在路中间欢快蹦跳,没有灾祸。
上六:狗叫,结果将凶险。

【赏析】

看来,古人过日子很难有安定团结的时候,随时都可能受到外敌入侵,随时都会有伤亡疾病的威胁,因而提心吊胆,小心翼翼,不敢有丝毫松懈。居家度日是如此,外出经商是如此,寻欢作乐也是如此。一句话,任何时候都要有备无患。

存在着不安定的因素,便会产生忧患意识;有了忧患意识,才会设法寻求各种防范措施。这一卦所讲,不是防天灾,而是防人祸;不是防自己人,而是防外族。这种忧患意识,在中国历史上从来没有中断过,并且早已深入到了整个民族的深层意识之中。

有了这种忧患意识,不断学会保护自己,才经得起种种磨难。 正像犹太民族一样,三千年流离失所,在胆战心惊的恐惧心理笼罩下度日,却没有一刻忘记过为重新建立家园而奋斗,最后终于如愿以偿。历史上的中华民族曾经强大过,又何以在近代落伍了? 这里面有太多的话可说,怎能是一个忧患意识所能解答。

尽管如此,忧患意识的确是焦点所在,其根源,可以追溯到我们远古的祖先们那里。

\chapter{姤卦}
\Bagua{000000}[3] \ (坤为地)坤上坤下
(天风姤)乾上巽下
《姤》:女壮,勿用取女。
初六,系于金柅,贞吉。有攸往,见凶,羸豕孚蹢躅。
九二,包有鱼,无咎,不利宾。
九三,臀无肤,其行次且,厉,无大咎。
九四,包无鱼,起凶。
九五,以杞包瓜,含章,有陨自天。
上九,女后)其角,吝,无咎。

【注释】

①姤(gou)是本卦的标题。姤用作“遘”,意思是遇合,也用作婚媾的 “媾”。全卦的内容与出行和婚姻有关,并且都是占问梦中景象,即梦占。标 题取“姤”的两种意义。
②壮:受伤。
③勿用:不利。取;用作 “娶”。
(4)金柅(ni):铜制的纺车转轮把手。
⑤赢豕:瘦猪。孚:用 作“桴”,意思是牵引。踯躅(Zhi zhu):徘徊不前的样子。
(6)包:用作 “疱”,意思是厨房。
(7)宾:宾客,这里指宴请宾客。
(8)起:动。
(9)以:倚靠,缠着。包瓜:匏瓜。
(10)含章:很有文彩。
(11)陨:掉 下,落下。
(12)姤:遭遇,这里指遇上野兽。其:而。角:搏斗。

【译文】

始卦:女子受伤,不利于娶女。
初六:衣服挂在纺车转轮的铜把手上了,占得吉兆。占问出行,则见凶象。拉着不肯前进的瘦猪。
九二:厨房里有鱼,没有灾祸。不利于宴请宾客。
九三:臀部受了伤,走起路来十分困难。危险,但没有大灾难。
九四:厨房没有鱼,一动就凶险。
九五:缠着把树往上长的轮从很好看,突然从很高的地方掉下一个瓜。
上九:碰上野兽,同它搏斗,危险,结果没有灾祸。

【赏析】

在周代,算得上顶顶重要的事的,只有战争和祭祀,因而受到极度重视。除此之外,其它一切事,饮食男女,婚丧嫁娶,种田经商,生老病死,冬去春来等等,都在小事之列。隆重的仪式, 繁琐的规程,艺术的装点,虔诚的态度,深遂的智慧,统统都奉献给了祭扫和战争,“小事”是不配享用的。

这一卦所记,正是梦中所见的生活琐事,并且确有梦的特点: 事与事之间没有逻辑联系,思路有很大的跳跃性,一会是婚姻,一 会是出行,一会是瘦猪,一会是厨房中的鱼,一会是臀部受伤,一会是地瓜掉下,一会是与野兽搏斗,显得稀奇古怪,纷纸杂陈。

这些梦象本身是偶然出现的,同储存于脑中的记忆表象有关。 古人不理解这一点,以为是神秘的征兆;而后世的说《易》者,却一本正经地要从中去发掘深意,未免荒唐可笑,正如缘木求鱼,完全找错了地方。

\chapter{萃卦}
\Bagua{000000}[3] \ (坤为地)坤上坤下
(泽地萃)兑上坤下
《萃》:亨,王假有庙。利见大人。亨,利贞,用大牲吉。利有攸往。
初六,有孚不终,乃乱乃萃,若号,一握为笑,勿恤,往无咎。
六二,引吉,无咎,孚乃利用禴。
六三,萃如嗟如,无攸利,往无咎,小吝。
九四,大吉无咎。
九五,萃有位,无咎。匪孚,元永贞,悔亡。
上六,□咨涕洟,无咎。

【注释】

①萃是本卦的标题。萃在卦中用作“悴”,“瘁”,意思是忧虑。全卦的内容主要讲祭祀和政治态度。标题的“萃”字是卦中多见词。
②假:到,至。
(3)大牲:牛,古代祭祀时以牛为大牲。
④不终:没有结果,这里指俘虏被抓后跑了。
③乱:纷乱。
(6)若:而。号:呼号。
(7)一握:即嗌喔,咿喔,表示笑声。
(8)引:永久,长期。
(9)禴(yUe):祭祀的名称,指春祭。
(10)嗟:感叹。
(11)萃:用作“瘁”。有,于。位:职位。
(12)匪罕:没有俘虏。
(13)赍咨(jizi):咨嗟,叹息。泆:流鼻涕。

【译文】

萃卦:亨通。君王到宗庙祭祝。有利于见到王公贵族,亨通,吉利的占问。祭祀用牛牲,吉利。有利于出行。
初六:抓到俘虏,后来又跑了,引起一阵纷乱和忧虑,大家呼喊着追捕。追回来后嘻哈大笑,不再担忧。前行,没有灾祸。
六二:长久吉利,没有灾祸。春祭最好用俘虏作人牲。
六三:长久叹息。没有什么好处。前行,没有灾祸,只有小危险。
九四:大吉大利,没有灾祸。
九五:尽瘁于职守,没有灾祸。没有俘虏,占问长久吉凶,没有悔恨。
上六:感叹流涕,为国忧心,没有灾祸。

【赏析】

前一卦讲了梦中的小事,这一卦接着就讲大事要事--祭祝。 祭祝对古人而言,恐怕难以再有比它重要的了。祭祝的对象一为祖宗,这与以血缘关系为纽带的宗法观念密切相关,也是社会组织结构形成的始基;一为神鬼,人间万物都有神鬼的法力在支配, 不得不以恭敬虔诚的的态度来对待;一为天地,人的生存条件取决于天与地,天地的运行变化,自然要对人的命运产生重要影响, 因此必须崇拜。

总而言之,祭礼以神圣隆重的仪式把古人的心灵导向自身以外的崇拜对象,唯独不崇尚自己,自己的心目中没有自己(天子例外,因为他就是天地神灵在人世间的唯一代表)。人是没有价值的,他的责任和义务就是服从,从家长到官员到大臣到皇上到祖先到天地到鬼种,人活着的意义就是为这些被当作神圣的对象服务,充当仆人和奴隶,充当牛马和犬羊!个人就更不用说了。个人是为他人、群体活着,是一架巨大机器上的一颗没有思想、没有情感、没有生命的螺丝钉。所以,在祭犯中,奴隶可以像牲口一样被杀了来做牺牲品,而人们却认为这是理所当然,必须如此。

看了这样的祭祝,我们的心里总该有所动吧!

\chapter{升卦}
\Bagua{000000}[3] \ (坤为地)坤上坤下
(地风升)坤上巽下
《升》:元亨。用见大人,勿恤。南征吉。
初六,允升,大吉。
九二,孚乃利用禴,无咎。
九三,升虚邑。
六四,王用亨于岐山,吉,无咎。
六五,贞吉,升阶。
上六,冥升,利于不息之贞。

【注释】

①升是本卦的标题。升的意思是上升,发展。全卦的内容大致是讲周朝不断上升、强盛的历史。标题的“升”字是卦中多见词。
②允:意思是前进。
③虚邑:建在大山丘上的城邑。
④王:周玉。亨: 即“享”,意思是祭祝。
⑤升:登上。阶:阶梯。
(6)冥:晚上。
(7)不息:不停。

【译文】

升卦:大亨大通,有利于见到王公贵族,不必担忧。向南出征吉利。
初六:前进而步步发展,大吉大利。
九二:春祭最好用俘虏作人牲,没有灾祸。
九三;向建在山丘上的城邑进军。
六四:周王在歧山举行祭祝。吉利,没有灾祸。
六五:占得吉兆,沿阶而逐步上升。
上六:昼夜不停地发展,有利于不停发展的占问。

【赏析】

我们不应忘记,《周易》的作者耿耿于怀的是挽救周王室的危亡,所以一有机会就要表达这一意图。这一卦的主题是发展进步,用在行将衰亡的周朝之上,正切中了问题的关键。

其实,岂止是周代才需要不断地发展壮大!无论哪个时代、哪个社会,没有发展和进步,都只有死路一条。“流水不腐,户枢不蠹。”这应是千古不易的真理。

发展和进步,不应当是单一直线式的,而应是开放式的,放射式的,金方位的,需要有博大的胸襟和高众瞩的见识。只要于我有利的,就广采博纳,而不要分什么中、外,更不应死守什么“中体西用”的死胡同。远的不说,近代中国就深受“中体西 用”之争的危害。现在的反思首先就当跳出中、西之分的窠臼,然后才会有深入实质的体察,然后才谈得上向前看,大步走。

\chapter{困卦}
\Bagua{000000}[3] \ (坤为地)坤上坤下
(泽水困)兑上坎下
《困》:亨。贞大人吉,无咎。有言不信。
初六,臀困于株木,入于幽谷,三岁不觌。
九二,困于酒食,朱绂方来。利用享祀。征凶,无咎。
六三,困于石,据于蒺藜,入于其宫,不见其妻,凶。
九四,来徐徐,困于金车,吝,有终。
九五,劓刖,困于赤绂,乃徐有说,利用祭祀。
上六,困于葛藟,于臲<臬兀>,曰动悔有悔,征吉。

【注释】

①困是本卦的标题。困的意思是困厄,倒霉和关押。全卦专讲刑狱。 “困”字与内容有关,又是卦中多见词,所以用作标题。
②言:用作 “愆”,意思是罪过。信:申辩,说清楚。
③困:这里的意思是挨打。株 木:指打人的刑杖。
④幽谷;这里指监狱。
⑤觐(di):看见。
(6)朱绂(fu):红色的服装,这里代指穿红色服装的民族。
(7)石:嘉石。 古代树立在朝门左边当众的地方,用于惩罚犯人。
(8):蒺藜:这里代指监 狱。
(9)徐徐;慢行的样子。
(10)金:禁。金车:关押犯人的囚车。
(11)劓(yi):割掉鼻子。刖(yue):砍掉脚。
(12)徐;逐渐。说:用作 “脱”。
(13)葛藟(lei):一种有刺的蔓生植物,种在监狱外,以防犯人逃跑。
(14)臲兀(nie wu):木桩,围在监狱外,防止犯人越狱。

【译文】

困卦:亨通。占问王公贵族得吉兆,没有灾祸。有罪的人无法申辩清楚。
初六:臀部挨了刑杖打,被关进牢房,三年不见外界天日。
九二:酒醉饭饱,穿红衣的敌人来犯,于是祭犯求神。占问出征,得凶兆。没有灾祸。
六三:被捆在嘉石上示众,又被关在四周有蒺藜的牢里,释放回到家里,妻子却不在了,凶险。
九四:犯人被关在囚车里,慢慢走来。这很不幸,但最后被释放了。
九五:被穿红衣的人抓去,割掉鼻子,砍断了脚,后来逐渐逃脱,赶快祭祝求神保枯。
上六:被关在四周有葛廷和木桩的监狱里,想动身越狱的话,就会悔上加悔。占问出征,得到吉兆。

【赏析】

中国古代历史上,从来没有过民主政治,民众从来就是被治理和奴役的对象,而治理和奴役民众的人从来就不受约束,可以为所欲为,无法无天,因而有“刑不上大夫”之说。

“困卦”中所讲的刑狱,便是专门用来治理平民百姓的。刑罚的方式和种类可谓丰富多彩,手段可谓无所不用其极,历史可谓悠久辉煌。泱泱大国,能列为国之骄做的品种,理当在“四大 明”之外加上严刑酷罚。

为了怕犯人逃走,想出了断脚、割鼻、刺面。 为了防止犯人传宗接代,想出了宫刑。为了防止报复,想出了诛灭九族。为了警告众人,想出了担枷示众,囚车游街。为了让犯人记取教训,想出了棒打屁股。还有臼刑、五马分尸、鞭尸等等。总而言之,在用暴力手段统治民众方面,中国古代的统治者具有世界级的水准。

据说卡夫卡当年在慕尼黑当众朗读小说《在流放地》(其中详细描述一架“杀人机器”的功用和特点)时,有人吓得逃走,有人当场晕倒。我们读《周易·困卦》,该有什么反应呢?

\chapter{井卦}
\Bagua{000000}[3] \ (坤为地)坤上坤下
(水风井)坎上巽下
《井》:改邑不改井,无丧无得。往来井井。汔至,亦未繘井,羸其瓶,凶。
初六,井泥不食。旧井无禽。
九二,井谷射鲋,瓮敝漏。
九三,井渫不食,为我心恻。可用汲,王明并受其福。
六四,井甃,无咎。
九五,井洌,寒泉食。
上六,井收勿幕,有孚元吉。

【注释】

①井是本卦的标题。井指井田,水井,陷饼。全卦内容是记述村邑中劳动和生活的情景。作标题的“井”字是卦中多见词。
②改邑:改换封邑。 井:井田,指划分齐整的田地。
③汔(qi):水干枯。至:用作“窒”,意思是淤塞。蹫(ju):这里的意思是挖。
④羸(lei):用作 “儡”这里指打破。
⑤泥:淤泥。井泥:井水浑浊含泥。
(6)井:陷 阱,用于捕兽。
(7)井谷:井底。鲋:小鱼。
(8)瓮:汲水的耳器。敝:破。
(9)渫(xie):污浊。
(10)心恻:沁测,意思是淘净,澄清。
(11) 井甃(zhou):用砖石垒筑井壁。
(12)冽:水清澈。
(13)收:缩小。井: 陷饼。幕:盖。

【译文】

并卦:改换了封邑却没改变井田数目,没有损失也没有多得, 人们照样在田间未来往往。水井已经干枯淤塞,却不去挖淘,还打破了汲水瓶,凶险。
初六:井水浑浊如泥无法饮用。陷阱塌坏不能关野兽。
九二:张弓射井底的小鱼。水瓮又破又漏。
九三:井水污浊不能饮用,给我淘净澄清,就可以汲饮。君王英明,使众人都得到他的福佑。
六四:用砖石垒砌井壁,没有灾祸。
九五:并水清澈,凉泉可口,可以饮用。
上六:缩小陷阱口,不加阱盖,结果捕获了野兽,大吉大利。

【赏析】

水被称为“生命之源”,已被世所公认。现代化的工业社会正面临着水资源匾乏的困扰,保护水资源成了全球性的话题。

这个困扰着今人的问题,同样也困扰过古人。原始社会的人类从狩猎游牧到定居下来从事农业生产,最重要的条件之一便是要有水源,否则难以生存下去。周人定居的陕西一带,古往今来都是缺水地区,要靠挖井取地下水来保证生产和生活,因此水便显得特别珍贵。“井卦”所记,正是这种情形的真实写照。

在那时人们的观念中,最重要的事情是祭祝和打仗,而对平民百姓来说,生活平安幸福最为重要,他们关注的是衣食住行,油盐柴米,锅碗瓢盆。我们不要自鸣清高地鄙视这些东西。世问没有不吃五谷杂粮的神仙,王公贵族们也是凡胎肉身,其它的都是身外之物。

正如没有水就没有生命一样,不食人间烟火就成了非人。

\chapter{革卦}
\Bagua{000000}[3] \ (坤为地)坤上坤下
(泽火革)兑上离下
《革》:已日乃孚。元亨。利贞,悔亡。
初九,巩用黄牛之革。
六二,巳日乃革之,征吉,无咎。
九三,征凶。贞厉。革言三就,有孚。
九四,悔亡。有孚改命,吉。
九五,大人虎变,未占有孚。
上六,君子豹变,小人革面,征凶,居贞吉。

【注释】

①革是本卦的标题。革的意思是改变。全卦的内容主要与战争有关,用战争来说明变的思想。标题的“革”字既与内容相关,又是卦中多见词。
②巳:用作“祀”,指祭祀。
③巩:加固,束紧。
④言:用作 “靳”,意思是马的胸带。三就:三重。
⑤改命:改变命令。
(6)革面: 变脸。

【译文】

革卦:祭祝那天用俘虏作人牲。大亨大通,吉利的占问。没有悔恨。
初九:用黄牛的皮革加固束紧。
六二:祭祝的日子要改变。出征,吉利。没有灾祸。
九三:出征,凶险。占得险兆。把马的胸带绑三匝,打了胜仗,抓到俘虏。
九四:没有悔恨。捉到俘虏,改变了命令。言利。
九五:指挥官勃然大怒,未必会取得胜利。
上六:君子勃然大怒,小人不满反抗。出征,凶险。占问居处得吉兆。

【赏析】

用战争来说明改变、变更的道理,确实说到了点子上。一方面,战争为国之大事,用来作例证具有说服力;另一方面,战争中充满各种变化多端的因素,没有灵活机敏的头脑,难以适应,固此本身就是对变化多端的最好说明。

变的道理可以推广到宇宙人间的万事万物。宇宙人间万事万物的存在即在不断运动和变化之中,绝对没有静止不变的东西。 因而,以不变应万变,在某种意义上便是消极闭塞的表现。天在变,道在变,人也应当变。最能变者,就最能生存。中国古代的智慧对此给予了充分的重视,所谓“因地制宜”,“因时制宜”, “因人制宜”等等,就是讲的以变对变,灵活机动,才能立于不败之地。

\chapter{鼎卦}
\Bagua{000000}[3] \ (坤为地)坤上坤下
(火风鼎)离上巽下
《鼎》:元吉,亨。
初六:鼎颠趾,利出否,得妾以其子,无咎。
九二:鼎有实,我仇有疾,不我能即,吉。
九三:鼎耳革,其行塞,雉膏不食,方雨亏悔,终吉。
九四:鼎折足,覆公餗,其形渥,凶。
六五:鼎黄耳金铉,利贞。
上九:鼎玉铉,大吉,无不利。

【注释】

(1)鼎是本卦的标题。鼎为饮食器具。全卦的内容同饮食以及与饮食有关的事相关。“鼎”为卦中多见词。
(2)颠趾:翻倒后足向上。
(3)出:除 去。否:这里指坏人。
(4)实:内容,这里指鼎中的装的食物。
(5)仇:妻子。
(6)即:两人对食,就餐。
(7)革:脱落。
(8)行:指外出打猎。赛:阻碍。
(9)雉膏:肥野鸡肉。
(10)粟:粥。
(11)形渥(wo):汤汁狼藉遍地。
(12)黄耳:铜耳。铉:关鼎盖的横杠。金铉:铜鼎盖横杠。
(13)玉铉:玉制的鼎盖横杠。

【译文】

鼎卦:大吉大利,亨通。
初六:鼎翻倒而足向上,有利于清除坏人。得到他人的妻子和儿子作家奴,没有灾祸。
九二:鼎中没有食物,我妻子有病,不能和我同吃。吉利。
九三:鼎耳脱落,外出打猎有无阻碍?家里野鸡肉留着不吃,天正下雨,倒霉。结果吉利。
九四:鼎足折断,倾倒了鼎中王公的粥,弄得遗地狼藉。凶险。
六五:鼎耳和鼎盖横杠都用铜制成。吉利的占问。
上九:鼎盖横杠用玉制威,大吉大利,没有什么不利。

【赏析】

鼎,不过一器具而已,腹大,三足,放着敦实稳当,或用作盛食物的器皿,或用作祭祀器物。如此器具,也值得占问求神?

其实,卦中所记,是梦中之象。敦实稳重的鼎翻倒、断足、脱 耳、洒得汤汁遍地,都与鼎的形象不相吻合,有点神秘,是不是? 总是什么的预兆吧?我们不信,但古人相信,所以要向神灵请示,弄个明白。

撇开梦的因素,单就现象而言,结实稳重的鼎出现异常现象, 可以具有某种象征意义。比如,硕大无朋的象翻倒在地四脚朝天,意味着什么?比如,稳定性最佳的三只粗足断掉一只,失去稳定性和重心而倾覆,叉意味着什么?全是不祥之兆。卦中所记诸象, 似乎证实了这一点。

这提醒我们,再坚实庞大的东西,也有颠覆瓦解的一天;再耀眼的太阳,也有陨落的时刻;再辉煌的业绩,也有穷途末路的时候。是呵,有什么是永恒可靠的呢:这样想,虽然有点令人丧气,但总比糊里糊涂好。

\chapter{震卦}
\Bagua{000000}[3] \ (坤为地)坤上坤下
(震为雷)震上震下
《震》:亨。震来虩虩,笑言哑哑,震惊百里,不丧匕鬯。
初九,震来虩虩,后笑言哑哑,吉。
六二,震来厉,亿丧贝,跻于九陵,勿逐,七日得。
六三,震苏苏,震行无眚。
九四,震遂泥。
六五,震往来,厉,意无丧,有事。
上六,震索索,视矍矍,征凶。震不于其躬,于其邻,无咎。婚媾有言。

【注释】

①震是本卦的标题。震代表雷电。全卦内容讲人对雷电的感受。‘震”是 卦中多见词,又与内容有关。
②隙隙(xi):,意思是恐惧的 样子。
③哑哑:笑声。
④匕(bi):勺子。鬯(chang):用黑黍和香 草酿成的酒。
⑤亿:猜测。贝:货币。
(6)跻:攀登。九陵:九重山, 指极远。
(7)苏苏:疑惧不安的样子。
(8)遂:用作“坠”。
(9)意: 用作“亿”猜测。
(10)索索:用作“缩缩”,意思是脚步很小。
(11)矍罢 矍矍(jue):意思是鹰隼看得远而准,比喻有眼光。
(12)婚媾:这里指亲戚。 言:罪过。

【译文】

震卦:亨通。雷声传来,有人吓得打哆咳,有人谈笑自如。雷声震惊百里,有人手拿酒勺镇定如常。
初九:雷声传来,有人先吓得打哆咳,后来便谈笑自如。吉利。
六二:雷电交加,非常危险,商人担心损失财物,翻山越岭赶往市场。有人告诉他别赶了,七八天就可弥补损失。
六三:雷电让人疑惧不安。在雷电中前行,没遇上灾祸。
九四:雷电从天上掉落到地面。 六五:雷电闪来闪去,十分危险,心里想着不要有损失和事 故。
卜六:雷电交加,有人行动小心谨慎,日光四顾。出行,凶险。雷电不会击到他身上,而击到邻人头上。没有灾祸。这大概是因为邻人做了错事吧。

【赏析】

在所有的自然现象中,恐怕少有像电闪雷鸣那样令人触目惊心的了:有声有色,撕天裂地,震撼人心。对古人而言,雷电也是最不可思议的:究竟是谁有如此大的魔力在操纵着它?答案被归结到在天上的神灵,被归结到“雷公”发怒,要惩罚人间的恶人坏事。于是,有了“报应”说。据说,做了坏事要遭电打五雷轰。据说,人受了不白之冤也要引起“雷公”震怒,可以发出晴天霹雳。

今天,我们早已懂得了雷电产生的原因,抹去了罩在它之上 的神秘色彩,并且有各种防范雷电伤害的手段措施。“雷公”被解了密。

然而,从心理学意义上说,当一切“密”都被解除之后,这个世界也许就变得索然无味了。倘若这世上没有了神秘莫测的黑夜,一切都暴露在光天化日之下,大概人们再也写不出忧郁美丽的诗歌了。在伦理学的意义上,人们相信了“雷公”的存在,在行为上多少会有所戒惧,不至于那么肆无忌惮。

所以,我们还是愿意看到“雷公”在天上存在,怒目俯看人间的一切邪恶,并发出怒吼。

\chapter{艮卦}
\Bagua{000000}[3] \ (坤为地)坤上坤下
(艮为山)艮上艮下
《艮》:艮其背,不获其身,行其庭,不见其人,无咎。
初六,艮其趾,无咎。利永贞。
六二,艮其腓,不拯其随,其心不快。
九三,艮其限,列其夤,厉,熏心。
六四,艮其身,无咎。
六五,艮其辅,言有序,悔亡。
上九,敦艮,吉。

【注释】

①艮(gen)是本卦的标题。原文卦象后无“艮”字,这是为避免与卦辞重复而省略。艮的意思是止息,歇息,引申为保护。全卦的内容是讲注意保 护身体。
②获:用作“护”。
③庭:园宅。
④趾:这里代指脚。
⑤胖:腿肚子。
(6)拯:保护,随:用作“隋”,意思是肉,这里指肌 肉。
(7)限:腰部。
(8)夤(yin):用作“夤”,意思是胁部肌肉。
(9)薰: 同“熏”。薰心:意思是心像被火烧一样痛苦。
(10)辅:指面 部。
(11)敦:意思是额部,这里代指头部。

【译文】

(艮卦):注意保护背部而不保护全身,就像一座大园宅没有人居住一样。没有灾祸。
初六:注意保护脚。没有灾祸。有利于长久吉利的占间。
六二:注意保护腿肚,却不保护腿部肌肉,心中不愉快。
九三;注意保护腰部,但胁间的肉已裂开了,危险,使人心焦。
六四:注意保护胸腹部。没有灾祸。
六五:注意保护面部,汪意说话有分寸。没有悔恨。
上九:注意保护头部。吉利。

【赏析】

这一卦专讲养身,讲到不要太劳累,注意保护身体各部位,便可以免除灾祸。这些看法至少说明,古人很珍惜自己的生命,并且已经具备了相当的医学知识,卦中虽然未述及具体的保养方法, 却有一个主导思想:以静养动。这也算是传统保养观的特色之一吧。

正如文武之道是相互补充的一样,动与静也是相互补充的。一张一弛,一紧一松,身心在其中得到调节息养,保持着活力。这种观或的灵感,大概得自于自然,得自于自然万物运动变化的节律。它一方面以生命体验为基础,一方面又深得内心感悟的启迪,从而得出保持生命本身的自然节律便是最好的保养的看法。

由此出发,过度的紧张或过度的歇息,都是对生命运动节律的破坏,都必须加以调节。肉体(命)本身的平衡,心理(性)本身的平衡,两者之间的平衡,便是调节所要达到的度。

\chapter{渐卦}
\Bagua{000000}[3] \ (坤为地)坤上坤下
(风山渐)巽上艮下
《渐》:女归吉,利贞。
初六,鸿渐于干。小子厉,有言,无咎。
六二,鸿渐于磐,饮食衎衎,吉。
九三,鸿渐于陆。夫征不复,妇孕不育,凶。利御寇。
六四,鸿渐于木,或得其桷,无咎。
九五,鸿渐于陵,妇三岁不孕,终莫之胜,吉。
上九,鸿渐于陆,其羽可用为仪,吉。

【注释】

①渐是本卦的标题。渐的意思是渐进,缓进。全卦内容以鸿雁起兴,作为主线,占问日常生活中的事情。标题的“渐”字是全卦中多见词。
②归:女子出嫁。
③鸿:水鸟。干:山涧。
④小子:小孩子。厉:危 险。
⑤言:意思是谴责,河责。
(6)磐:涯岸。
(7)瞰 (kan):自得的样子。
(8)陆:高而平的地。
(9)桷(jue):屋顶上承瓦 的方木条。
(10)胜:欺凌,欺侮。
(11)阿:大山。
(12)仪:用鸟羽编 织的文舞道具。

【译文】

渐卦:女子出嫁,是吉利的事。吉利的占问。
初六:鸿雁走进山涧,小孩也去很危险,应当河责制止。
六二:鸿雁走上涯岸,丰衣足食,自得其乐。吉利。
九三;鸿雁走上陆地,丈夫出征没回来,妻子怀孕而流产。凶险。有利于抵御敌寇。
六四:鸿雁飞上树木,贵族已准备好了盖房的桶条。没有灾祸。
九五:鸿雁飞上山头,妻子多年没有怀孕,却始终没有受到欺侮。吉利。
上九:鸿雁飞上大山,它的羽毛可以作文舞的道具。吉利。

【赏析】

这一卦采用民歌常用的起兴手法,记述家庭生活,既洋溢着一种幸福的情调,又颇富有诗意。

幸福的家庭确是相似的:丰衣足食,凡孙满堂,而这里的幸福之家却有点不一样,那就是妻子多年不孕不育。按封建时代传统的观念,不孝有三,无后为大。女子最重要的职责之一便是生儿育女。不能生儿育女的妻子,要被休弃,或者纳妾以替代。卦中写到的妻子是幸运的,没有被休弃,也没有受到歧视欺侮。

显然,这不能看成具有普遍性的现象,不幸的女子,不幸的家庭却很多很多。如今的女子已无法体验古时妇女被迫缠足、束胸、守节的苦痈,无法体验苦守闺中思春的惆怅落寞。俱往矣,时代不同了,妇女地位变化了。相妻教子,孝敬长辈,操持家务成了男人的职责,女人获得了空前的自由和解放。

这是今天的女人的福气。倘若古人有知,一定羡慕得要死。

\chapter{归妹卦}
\Bagua{000000}[3] \ (坤为地)坤上坤下
(雷泽归妹)震上兑下
《归妹》:征凶,无攸利。
初九,归妹以娣。跛能履,征吉。
九二,眇能视,利幽人之贞。
六三,归妹以须,反归以娣。
九四,归妹愆期,迟归有时。
六五,帝乙归妹,其君之袂不如其娣之袂良。月几望,吉。
上六,女承筐无实,士刲羊无血,无攸利。

【注释】

①归妹是本卦的标题,归妹的意思是少女出嫁。全卦的内容是记述姊妹共夫的婚俗。“归妹”是卦中多见同,也与内容有关。
②以:当。娣:女 弟,即妹妹。
③吵:目盲。
④幽人:囚徒,这里指家庭妇女。
⑤须:用作“娶”,意思也是女弟,即妹妹。
(6)反归:被休弃回娘家。
(7)愆期:过期。
(8)时:待。
(9)帝乙,殷纣上之父。
(10)其君:这 里指君夫人。袂(mei):衣袖,这里代指嫁妆。
(11)几:接近。望:农历 每月十五日。
(12)承:捧着。承筐:捧着装祭品的器具。
(13)刲(kui): 割,宰杀。

【译文】

归妹卦:出行,凶险。没有什么好处。
初九:姊妹一同出嫁。脚跛却能行走。出行,吉利。
九二:眼瞎了却能看见。有利于女子婚嫁的占问。
六三:姊妹一同出嫁,后来又一同被休弃返回娘家。
九四:出嫁时超过了婚龄,迟迟不嫁是有所期待。
六五:殷帝乙把女嫁给周文王,妹妹的嫁妆比姊姊的还要漂亮。婚期选在将近月中,吉利。
上六:新娘捧着祭品的筐,但筐里没有东西;新郎提刀杀羊, 但羊没有出血。没有什么吉利。

【赏析】

男大当婚,女大当嫁,这是天经地义的事情,正如饿了要吃饭,困了要睡觉一样,没有什么可讨论的。然而,怎么娶法,怎么嫁法,却是有讲究的。观念不同,赋予婚嫁的意义便有所不同, 采取的方式也就不同。

“归妹卦”让我们看到的是群婚制的遗俗:女子到了出嫁年纪; 便带着嫁妆,带着妹妹一同嫁到夫家去;被休弃时,也连同妹妹一起回家,并提到了殷纣王之父帝乙将女儿嫁给周文王的史实,说明这种姊妹共夫是有所据的。女子嫁到夫家三个月之后,要举行祭祀仪式,新娘献上黍稷等进行祭奠,新郎则宰羊献牲,甚是隆重。据考证,这种婚俗到了西周之后的春秋时代,仍很盛行,被叫做“媵”(ying)。

卦中的记述采取了现实和梦象交织的手法。初九交爻的“跛能履”,九二爻的“眇能视”以及上六爻都是梦占。这是否包含有这样的意思:出嫁对女子来说是个人生的转折点,在这喜忧交集的时刻,总是充满了各种想象和幻想?即使作者没有这意思,现实情况多半会如此。

\chapter{丰卦}
\Bagua{000000}[3] \ (坤为地)坤上坤下
(雷火丰)震上离下
《丰》:亨,王假之。勿忧,宜日中。
初九,遇其配主,虽旬无咎,往有尚。
六二,丰其蔀,日中见斗。往得疑疾,有孚发若,吉。
九三,丰其沛,日中见沫,折其右肱,无咎。
九四,丰其蔀,日中见斗,遇其夷主,吉。
六五,来章有庆誉,吉。
上六,丰其屋,蔀其家,窥其户,阒其无人,三岁不觌,凶。

【注释】

①丰是本卦的标题。丰的意思是大,盛大。全卦的内容是讲行旅、商旅。 标题的“丰”字是卦中多见词。
②亨:用作“享”,意思是祭把。
③ 假:用作“格”,意思是到达,至。之:指代祭耙的地方。
(4)日中:中午 时分。
⑤配主:女主人。
(6)旬:意思是男女姘居结合。
(7)尚:支持,帮助。
(8)丰:扩大,增加,森:用作“菩”,意思是用 草或小席拼接起来。
(9)斗:北斗星。
(10)疑疾:多疑的病,怪病。
(11)发:用作“废”,意思是天生残废。
(12)沛:用作“芾”,意思是用草 盖房顶。
(13)沫:用作“昧”,意思是暗淡无光,这里是说日蚀。
(14)肱 (gong):手臂。
(15)夷:常。夷主:常主。
(16)来:获得。章:用作 “璋”,意思是美玉。
(17)庆,庆贺,誉:称赞。
(18)屋:指整座房子。
(19)家:指屋内。
(20)窥:意思是探看。
(21)阒(qu):寂静。
(22)觌:看见。

【译文】

丰卦:君王到宗庙祭祝。不必担心,时间最好在中午。
初九:途中受到女主人招待,跟她同居,没有灾祸。行旅得到了内助。
六二:用草和草席铺盖房顶,中午见到北斗星。行旅中得了怪病;买到残废了的奴隶。吉利。
九三:用草盖屋顶,中午出现了日蚀。折断了右臂。没有灾祸。
九四:用草和草席铺盖房顶,中午见到北斗星。途中遇到了老房东。吉利。
六五:获得美玉,大家庆贺称赞。吉利。
上六:房子大而空,用草和草席盖房顶。从门缝往里看,寂静无人,看来多年无人居住。凶险。

【赏析】

这一卦写商人长期在外经商漂泊。从根本上说,谁都不愿漂泊,都愿意呆在家中过舒适日子,因为家毕竟是人天生所渴术的。但商人的目的是水钱,为此就得外出,买进卖出,四处游走,实际上是不由自主地选择了漂泊。

漂泊的滋味可以说酸甜苦辣样样都有,唯有亲身体验的感受才会真切。这里的记述算得上丰富细致,尤为引人注意的是,商人在途中与客栈的女主人同居,可见那时的性关系和婚姻关系是比较自由的。这同后来封建礼教束缚下的情形有天壤之别。女主人显然是个寡妇,否则不会那么顺利地与商人结合。那时的寡妇不必守节,能得到舆论的同情和支持。长期在外漂泊的商人需要得到女人的温情和帮助,本在情理之中。当然他也可能见异思迁, 为女主人的美貌或贤慧所动而贪恋新欢。在实质上,商人的这种冲动,也可以说是漂泊者寻找家园的冲动的折射。

漂泊中的苦肯定多于乐,忧愁多于舒适:日晒雨淋,风霜雷电,山高路远,水深流急,忍饥挨饿,遭劫失财,亏本受骗,生病受伤,等等,这一切都不难想象到。最大的快乐莫过于赚了大钱,满载而归,为此似乎可以忘却一切。总之,那时的商人大概还比较朴实,比较真诚,才会历尽艰辛不辞漂泊。这也与今天的商人大不一样。

\chapter{旅卦}
\Bagua{000000}[3] \ (坤为地)坤上坤下
(火山旅)离上艮下
《旅》:小亨。旅贞吉。
初六,旅琐琐,斯其所取灾。
六二,旅即次,怀其资,得童仆,贞。
九三,旅焚其次,丧其童仆,贞厉。
九四,旅于处,得其资斧,我心不快。
六五,射雉,一矢亡,终以誉命。
上九,鸟焚其巢,旅人先笑后号啕。丧牛于易,凶。

【注释】

①旅是本卦的标题。旅的意思是行旅,商旅,全卦的内容也与此有关。标题的“旅”字是卦中多见词。
②琐琐:意思是三心二意。
③斯:离开。所:指住所。
(4)取灾:得祸。
⑤即:走近,走到。 次:用作“肆”,意思是市场。
(6)焚其次:市场夫火。
(7)处:止,指住处。
(7)资斧:钱财。斧:斧形钱币。
(9)命:名声。
(10)易:用。

【译文】

旅卦:小事通。占问行旅得吉兆。
初六:旅途三心二意,离开住所,结果遭祸。
六二:行到市场,怀揣钱财,买来奴隶,占得吉兆。
九三:行到着火的市场,买来的奴隶借机逃走,占得险兆。
九四:行到住处,虽然赚了钱,心里却不安。
六五:途中射野鸡,一箭中的,结果得到善射的美名。
上六:鸟巢失了火,旅人先高兴欢笑,后来却呼号哭泣,狄人抢去了牛羊。凶险。

【赏析】

又是商旅。为什么一写再写?只能说明商人重要,商业活动频繁,商业在经济生活中的地位超过了农业。也许,那时的社会思潮是重商轻处与后来刚好相反。

不过,这次差不多是不好的兆头。途中疑神疑鬼,离家就遭祸,市场失火,买到的奴隶跑掉了,赚了钱心里并不高兴,最后连家园都丧失了。真是祸不单行啊。

看来,赚钱并不是好事。赚钱本来是为了日子过得更好,结果饱尝艰辛之后连家园都保不住,还有什么意思?钱财本是身外之物,生不带来,死不带走。舍本逐末,岂不是太糊涂吗!

这些隐隐约约的潜台词,仿佛在提醒我们:赚钱并非是件好事,经商不一定是一种好选择,不必人人都去东奔西跑。别忘了回家,别忘了重要的是.日子平和安康。

\chapter{巽卦}
\Bagua{000000}[3] \ (坤为地)坤上坤下
(巽为风)巽上巽下
《巽》:小亨。利有攸往。利见大人。
初六,进退,利武人之贞。
九二,巽在床下,用史巫纷若,吉,无咎。
九三,频巽,吝。
六四,悔亡,田获三品。
九五,贞吉,悔亡,无不利,无初有终。先庚三日,后庚三日,吉。
上九,巽在床下,丧其资斧,贞凶。

【注释】

①巽(xun)是本卦的标题。巽也写作“算”,意思是算卦。 全卦内容主要讲史巫算卦。标题的“巽”字与内容有关,又是卦中多见词。
②史巫:古代从事祭祖和算命的人。祝史负责祭祖,巫师负责降神,消除 灾祸。纷若:纷乱的样子。
③频:用作“颦”,意思是皱眉,表示愁眉苦脸。三品:三种,指三种野兽。
④先庚三日:庚日前的第三日,即了日。
⑤后庚三日:庚日后的第三日,即癸日。

【译文】

巽卦:小亨通。有利于出行。有利于见到王公贵族。
初六:前进后退听命于他人,有利于军人的占问。
九二:在床下算卦,祝史巫师禳灾驱鬼,乱纷纷一团。吉利,没有灾祸。
九三:愁眉苦脸地算卦,危险。
六四:没有悔恨。田猎获得三种猎物。
九五:占得吉兆,没有悔恨,没有什么不利。开头不好,结局不错。时间在丁日和癸日,吉利。
上九:在床下算卦,丧失了钱财。占得凶兆。

【赏析】

祝史巫师是一类很特殊的人:他们有文化,受过教育,用今天时髦的话说叫“文化人”;他们懂得神灵世界的事情,又熟知人世间的事,通过特殊的方式,把人间事告诉神灵,又向人间传达神灵的信息,可以叫做“神与人之间的使者”。

从古到今,在信奉神灵的社会中,祝史巫师享有很高的名望和社会地位,即使王公贵族、达官责人都敬畏他们、迷信他们。他们没有掌握政治、军事权力,却掌握着支配人们灵魂的权力,用自己的行为控制着人们的精神生活。

从上述方面来看祝史巫师的祭方已算卦,比较容易理解人们为什么那么看重他们,那么敬畏他们,也比较容易理解他们何以那么认真隆重地用各种仪式来进行人与神在心灵上的对话。今天我们觉得很荒唐,是迷信和愚昧的表现。可是古人却是严肃认真的, 虔诚的,不容置疑地确信。是的,他们有他们的理由。

\chapter{兑卦}
\Bagua{000000}[3] \ (坤为地)坤上坤下
(兑为泽)兑上兑下
《兑》:亨。利贞。
初九,和兑,吉。
九二,孚兑,吉,悔亡。
六三,来兑,凶。
九四,商兑未宁,介疾有喜。
九五,孚于剥,有厉。
上六,引兑。

【注释】

①兑(yue)是本卦的标题。兑的意思是悦,高兴,愉快。全卦的内容主要是讲国与国之间的邦交。标题的“兑”字是卦中多见词。
②率兑:以 抓到俘虏为快事。
③来:使人归顺。
④商:谈判。宁:定下来,得出结果。
⑤介:小。介疾:小毛病。有喜:有好结果。
(6)剥:国名。
7引:引导。

【译文】

兑卦:亨通。吉利的占问。
初九:和睦愉快,吉利。
九二;以捉到俘虏为快事,吉利,没有悔恨。
六三:以使人归顺为快事,凶险。
九四:谈判和睦相处的问题,尚未得出结果。小摩擦容易解决。
九五:被剥国所俘虏,危险。
上六:引导大家和睦相处。

【赏析】

这一卦专讲国与国之间的邦交,用今天的话说,就是外交关系。

国与国、邦与邦,实际上是各不相同的利益集团。利益焦点和关系不同,便会造成分歧、摩擦、冲突,乃至战争。因此,战争也是利益之争,是用武力来争夺利益。

但是,武力并非解决利益冲突的唯一手段。攻城掠地劫夺财物容易,征服人心同化异族却极其困难。现实迫使人们认识到这一点,只有承认他人的利益和存在,才能确保自己的利益和存在。 所以,和睦相处首先是以此为前提的,否则就会发生冲突。 和睦相处不是手段,不是权宜之计,而是目的,是对他人的承认和尊重。

以此来调节利益关系,便会顺理成章,皆大欢喜。

\chapter{涣卦}
\Bagua{000000}[3] \ (坤为地)坤上坤下
(风水涣)巽上坎下
《涣》:亨。王假有庙。利涉大川,利贞。
初六,用拯马壮,吉。
九二,涣奔其机,悔亡。
六三,涣其躬,无悔。
六四,涣其群,元吉。涣有丘,匪夷所思。
九五,涣汗其大号,涣王居,无咎。
上九,涣其血,去逖出,无咎。

【注释】

①涣是本卦的标题。涣的意思是洪水。全卦的内容是讲水灾水患。标题的“涣”字与内容有关,又是卦中多见词。
②亨:用作“享”,意思是祭 祀。
③用:因为。拯:用作“乘”。壮:用作“册’,意思是受伤。
④奔:用作“崩”,意思是冲毁。机:用作“兀”,意思是房屋的地基。
⑤群:众人。
(6)有:于。丘:山丘。
(7)匪:非。夷:平常。
(8) 涣汗:水流盛大的样子。其:而。号:呼叫。
(9)王居:王者的住处,王宫。
(10)血:用作“恤”,意思是忧患。去:消除。逖:用作“惕”,意思 是警惕。出:产生。

【译文】

涣卦:洪水到来,君王到宗庙祭祖祈祷。有利于渡过大江大河。吉利的占问。
初六:洪水到来,因骑马逃避摔伤。吉利。  
九二:汹猛的洪水冲毁了屋基,悔恨极了。
六三:洪水冲到身上,无灾无悔。
六四:洪水冲向人群,人群跑得快,大吉大利。洪水冲向山丘,那情景平时难以想象。
九五:洪水滔滔,人们奔走呼号。洪水涨到王宫,。结果没有灾难。
上九:洪水的忧患消除了,但要提防灾难重现,就不会有灾祸。

【赏析】

洪水属于天灾,严重威胁到人们的生存,并且难以抗拒,古往今来都是如此。大凡超过了人们控制能力的事物,对人来说都是可怕的,人们面对它们时,只有求助于超人的力量。

实际上,超人的力量是不存在的,至多只能给人精神上的安慰,而不能解决实际问题。要说古人完全相信神灵的力量,并不完全是如此,否则就不会有夏禹和李冰父子治水的事迹了。这些 故事的广为传诵,至少说明了人们把战胜水患的希望寄托在人力身上。

从本卦的记述来看,那时的人们还是用非常理性化的态度来对待洪水这种天灾的。大概是因为这类灾难见得多了,祈求神灵也不管用,还得自己想法来应付(哪怕是逃避),多次应付之后, 也就不觉得可怕了,所以最后说到了要吸取教训,提高警惕,加强防范。

看来,洪水并不可怕。可怕的是愚昧。

\chapter{节卦}
\Bagua{000000}[3] \ (坤为地)坤上坤下
(水泽节)坎上兑下
《节》:亨。苦节,不可贞。
初九,不出户庭,无咎。
九二,不出门庭,凶。
六三,不节若,则嗟若,无咎。
六四,安节。亨。
九五,甘节,吉,往有尚。
上六,苦节,贞凶,悔亡。

【注释】

①节是本卦的标题。节的意思是节制、节俭和礼节。全卦的内容讲礼节和节约。标题的“节”字与内容有关,又是卦中的多见词。
②苦节:意 思是以节制为苦事。
③若;句尾的助词,没有实际意义。
④嗟:感 慨,叹息。
⑤安节:意思是安于节俭的生活。
(6)甘节:意思是以节制为乐事。
(7)尚:资助,帮助。

【译文】

节卦:亨通。以节制守礼为苦事,吉凶不可占问。
初九:在家室内不出门,没有灾祸。
九二:在庭院内不出门,凶险。
六三:不知节俭守礼,就会后悔叹息。知道就没有灾祸。
六四:安于节制守礼的生活,亨通。
九五:以节制守礼为乐事,吉利。出行会得到帮助。
上六:以节制守礼为苦事,占得凶兆,悔恨不已。

【赏析】

节俭和遵守礼节是人们的行为准则。一个社会没有礼节,犹如球场上的比赛没有规则,将会乱套。据说,周公曾经制“礼”,就是为了使社会生活有所规范,使人们行为有度。又据说,春秋 时代的孔子对周礼十分向往,主张‘克己复礼”,表明他对以礼治 国的重视。

中国历来被称做“礼仪之邦”,大概便是指的始自周代的尊礼传统。后来孔子的学说占了上风,他所推行的仁、义、礼、知、信也就成了历代治国者奉行的教条,或叫做“礼教”。

遵守礼节没有什么不好,可以算作是有教养的文明表现。节制克俭也没有什么不好,它可以使人保持清醒的头脑,懂得应当珍惜什么。这些美德,应当成为社会全体成员信奉和遵守的准则,尤其是统治集团的成员,更应成为表率。这不仅是因为他们代表着国家的形象,而且也因为他们手中握有可以使人头脑膨胀的权力。

\chapter{中孚卦}
\Bagua{000000}[3] \ (坤为地)坤上坤下
(风泽中孚)巽上兑下
《中孚》:豚鱼,吉。利涉大川,利贞。
初九,虞吉,有它不燕。
九二,鸣鹤在阴,其子和之。我有好爵,吾与尔靡之。
六三,得敌,或鼓或罢,或泣或歌。
六四,月几望,马匹亡,无咎。
九五,有孚挛如,无咎。
上九,翰音登于天,贞凶。

【注释】

①中孚是本卦的标题。中争的意思是心中诚信。全卦的内容是讲礼仪。标 题与内容有关。
②豚(tun)鱼:小猪和鱼。这两样东西是献祭和行礼时 常用的物品。、
③虞:丧礼,葬礼。
④它:意外事故。燕;用作 “宴”,指宴饮之礼。
⑤阴;用作“荫”,意思是树上荫蔽的地方。
(6) 爵:古代酒器,即酒杯,这里代指酒。
(7)靡:共享,同享。
(8)得敌: 克敌,战胜敌人。
(9)鼓:击鼓追击敌人。罢;停战,收兵。
(10)挛如: 捆得紧紧的样子。
(11)翰音:鸡,这里指用鸡祭天。

【译文】

中孚卦:行礼时献上小猪和鱼,吉利。有利于渡过大江大河。吉利的占问。
初九:行丧礼,吉利。如有变故,就不行燕礼。
九二:鹤在树荫中鸣叫,幼鹤应声附和。我有美酒,与你同 享。
六三:战胜了敌人,有的乘胜追击,有的凯旋收兵,有的高 兴流泪,有的放声歌唱。
六四:月近十五的时候,马匹丢失了。结果没有灾祸。
九五:抓到俘虏,紧紧捆住。没有灾祸。
上九:用鸡祭祖上天。占问得凶兆。

【赏析】

这一卦专讲礼仪,以内心虔诚为中心,依次讲了丧礼、宴礼、军礼和祭礼。虽然这些还不是全部礼义,但可见周代礼仪繁多复杂之一斑。不妨说,古人的生活方式是普遍仪式化了的,各种礼 仪都为某一特殊目的而设,礼仪活动渗透到了日常生活的方方面面。

表面上看,礼仪奢华繁琐,但在更深层次上,它是二种人类寻求意义的活动。这道理很显豁,因为人不仅在物质的、肉体的层面上存在,同时也通过语言、仪式、艺术、宗教而在诗意的意 义层面上存在。现代人似乎免去了众多‘繁文褥节”,但现代人所 失去的比“繁文得节”要多得多。他离物质越近,寓意义就越远; 他离肉体越近,离想象和诗意也就越远。他已经赤裸得如同走肉行尸,并且被阉割了。

不难想象,通过神圣隆重庄严肃穆的礼仪,我们的心灵将被提高到怎样的高度,我们的精神将得到怎样的净化和升华。遗憾的是,这离我们已越来越远。

\chapter{小过卦}
\Bagua{000000}[3] \ (坤为地)坤上坤下
(雷山小过)震上艮下
《小过》:亨。利贞。可小事,不可大事。飞鸟遗之音,不宜上,宜下,大吉。
初六,飞鸟以凶。
六二,过其祖,遇其妣。不及其君,遇其臣。无咎。
九三,弗过防之,从或戕之,凶。
九四,无咎。弗过遇之,往厉必戒,勿用永贞。
六五,密云不雨,自我西郊。公弋取彼在穴。
上六,弗遇过之,飞鸟离之,凶,是谓灾眚。

【注释】

①小过是本卦的标题。“过”有经过和责备两个意思,全卦的内容主要是 讲对批评的看法。标题的“过”字是卦中的多见词。由于前面已有“大过 卦”,所以这一卦叫“小过”。
②可:有利于。小事:周代祭祀和战争是 大事,其它都是小事。
③遗:留下。音:鸟的叫声。
④以:带来。
⑤过:责备,批评。祖:祖父。
(6)遇:礼遇,“过”的反义词。批 (bI):祖母。
(7)不及:赶不上,有缺点。
(8)从:用作“纵”,意思是 放纵,听任。戕(qiang) :伤害。
(9)无咎:这里的意思是没有过错。
(10)勿用:不利于。
(11)弋(yi):射鸟。
(12)彼:这里指代野兽。
(13) 离:用作“罗”,意思是网,指捕鸟的网。

【译文】

小过卦:亨通,吉利的占问。对小事有利,对大事不利。飞鸟经过,叫声还留在耳际。对大人不利,对小人有利。大吉大利。
初六:飞鸟经过,带来凶兆。
六二:祖父可以批评,祖母可以称赞。君王也有缺点,臣子 也可以夸奖。没有灾祸。
九三:不要过分指责,但要防止错误发展。倘若放任不管,就是害他。凶险。
九四:没有错误,就不要指责,而要夸奖。日后有出错的危险,一定要防止。不利于占间长久的吉凶。
六五:在我西边郊野上空,阴云密布,雨却没有下来。王公射鸟,却在洞穴抓到野兽。
上六:对没有错的人不表扬,反而批评,像网罗网飞鸟,凶险,这就叫灾祸。

【赏析】

在讨论批评之前,作者就先确立了一个原则界限:祭祀和战争这样的大事不可随便批评,此外的一切“小事”都可议论,可批评,可表扬。这说明了祭犯和战争的神圣。 重要的是,在可以批评的范围年,我们看到作者持的是一种开明的立场,特意指出批评对地位尊贵者(王公贵族,家长官长)不利,对他们皆可批评。同时,批评要及时,恰到好处,注意褒贬的适当运用。

如果在后来的专制时代,这样的言论肯定会被看作是“犯上作乱”,不会见容于统治者,说不定作者还会陷入“文字狱”。 美国作家房龙曾写过一本书叫《宽容》,讲述西方历史上统治者对思想、言论自由的不宽容。其中一些情况,在古代中国是有过之而无不及,不允许思想自由,更不用说对君王将相王公贵族 的过失错误评头品足了。

只要是人,都有所长有所短,批评褒扬实在大正常了。可是,本来很简单明白的道理,由于把人划分出等级,梗越弄越复杂,甚至会招来杀身之祸。这是人类的悲剧,还是历史的悲剧?宽容说来容易,行来就不那么容易了。

\chapter{既济卦}
\Bagua{000000}[3] \ (坤为地)坤上坤下
(水火既济)坎上离下
《既济》:亨小,利贞。初吉终乱。
初九,曳其轮,濡其尾,无咎。
六二,「妇丧其茀,勿逐,七日得。
九三,高宗伐鬼方,三年克之,小人勿用。
六四,繻有衣袽,终日戒。
九五,东邻杀牛,不如西邻之禴祭,实受其福。
上六,濡其首,厉。

【注释】

①既济是本卦的标题。既的意思是已经,济的意思是渡水和成功、成就。 既济是说事已成功。全卦内容是讲事情成功的道理,与下一卦“未济”构成 组卦。标题取“济”的意义。
②乱:变故。
③茀(fu):用作“禴”, 意思是头巾。
(4)高宗:殷国君武丁,曾与周联手攻打北方强敌鬼方。鬼方:殷周时北方的国名,属于严允部落之一。
⑤繻(ru):意 思是御寒的衣服。袽(ru):用作“絮”,指破烂的冬衣。
(6)戒:用作 “骇”,意思是惊惧不安。
(7)东邻:指殷人。西邻:指周人。禴(yue)祭古代祭名,指春祭。

【译文】

既济卦,亨通,有小吉利的古问。开始吉利,结果会出现变故。
初九,拉车渡河,打湿了车尾。没有灾祸。
六二:妇人丢失了头巾,不用去找,七天内会失而复得。
九三:殷高宗武丁讨伐鬼方国,用了三年才取胜。对小人不利。
六四:冬天穿的寒衣破烂不堪,整天心里惊恐不安。
九五:殷人杀牛祭祝,不如周人春祭,周人确实得到了神的福佑。
上六:过河时打湿了头部,危险。

【赏析】

有句古话,叫做“塞翁失马,安知非福”,说的是丢失马虽然是个损失,但谁叉能说这不是更大的福气到来的征兆呢。

福 与非福,成功与失败,损失与收获,都没有绝对不可逾越的界限,完全可以相互转变。 这个思想,可以用来概括“既济卦”的主题。殷人强大的时 候,周人还较弱小,不足以与之抗衡。后来周由弱变强,最终消灭了殷,情况发生了逆转。

这类情形在现实中大多了,所以作者才说丢了头巾不必寻找,它自己会回来,以此说明不济之中又有济。再夸张一点说,就是三十年河东,三十年河西;山不转水转,天无绝人之路。 看透了这些道理,多少会给人心理上一些慰藉,也可能会使人心胸开阔起来,精神境界更上一层楼。正如毛泽东说的,“风物长宜放眼量”,“不管风吹浪打,胜似闲庭信步”。 患得患失,反而可能得到的少,失去的多;洒脱无忧,却有可能得到的多,失去的少。

现在流行歌唱“箭洒走一回”,实际上 又有凡人真能漾溃洒洒走一回呢?

\chapter{未济卦}
\Bagua{000000}[3] \ (坤为地)坤上坤下
(火水未济)离上坎下
《未济》:亨。小狐汔济,濡其尾,无攸利。
初六,濡其尾,吝。
九二,曳其轮,贞吉。
六三,未济,征凶。利涉大川。
九四,贞吉,悔亡,震用伐鬼方,三年,有赏于大国。
六五,贞吉,无悔。君子之光,有孚吉。
上九,有孚于饮酒,无咎。濡其首,有孚失是。

【注释】

①未济是本卦的标题。未济的意思与既济相反,全卦接着申说上一卦的道理,仍以“济”的意思作标题。
②讫(qi):用作“几”,意思是将要。济:渡水。
③震:动。
④大国:这里指殷。
⑤光:光荣,荣耀。
(6)是:用作“题”,意思是额部,这里代指头部。

【译文】

未济卦:亨通。小狐狸将要渡过河,打湿了尾巴。没有什么吉利。
初六:打湿了尾部,倒霉。
九二:拉车渡河,占得吉兆。
六三:渡不了河。出行,凶险。有利于渡过大江大河。
九四:占得吉兆,没有悔恨。周人动员出征,讨伐鬼方,三年取胜,得到大国段的赏赐。
六五:占得吉兆,没有悔恨。打胜仗,抓俘虏,是君子的荣耀。吉利。
上九:抓到俘虏,饮酒庆功。没有灾祸。打湿了头部,抓到俘虏,砍下他们的头。

【赏析】

上一卦讲济与不济的转化,似乎意犹未尽,于是这一卦接着申说。理还是那个理,事多半还是那些事,主题还是那个主题,角度还是那个角度。一正一反,既济中有未济,未济中又有既济,于是,功德圆满了。

从这当中,再一次反映了中国人善于从正反两方面、对立两方面去体悟宇宙人间万事万物运动变化的思维习惯。不济中有济,讨伐鬼方就是;济中有不济,殷被周灭就是。正如有生命就意味着有死亡,也只有不断的死亡,才会有不断的新生命诞生出来。于是,万物生生不息,新陈代谢,推陈出新。

既然是运动变化不止,就没有永恒静止不变的东西。既济不会永远是既济,不济不会永远是不济。君不见,“江山代有人才出,各领风骚数百年。"这已是过去了的皇历。如今,谁能领风骚几年、甚至几个月,就很不错了。运动变化的速度大大加快,快得令人眼花纷乱,目不暇接,再也没有谁敢于自夸是“不倒翁”,再也没有谁敢吹嘘代表了“永恒真理”,“永恒正义”。

如果要说有什么永恒的话,新陈代谢,代谢无疆,就是真正的永恒。

\part{文言}
\part{系辞上}
\part{系辞下}
\part{说卦}
\part{序卦}
\part{杂卦}

\backmatter
\chapter{附录一 寻象以观意--卦象统计表}
\chapter{附录二 四幅易图}
\chapter{后记}

\end{document}