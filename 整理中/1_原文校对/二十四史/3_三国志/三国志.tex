% 三国志
% 三国志.tex

\documentclass[12pt,UTF8]{ctexbook}

% 设置纸张信息。
\usepackage[a4paper,twoside]{geometry}
\geometry{
	left=25mm,
	right=25mm,
	bottom=25.4mm,
	bindingoffset=10mm
}

% 设置字体,并解决显示难检字问题。
\xeCJKsetup{AutoFallBack=true}
\setCJKmainfont{SimSun}[BoldFont=SimHei, ItalicFont=KaiTi, FallBack=SimSun-ExtB]

% 目录 chapter 级别加点(.)。
\usepackage{titletoc}
\titlecontents{chapter}[0pt]{\vspace{3mm}\bf\addvspace{2pt}\filright}{\contentspush{\thecontentslabel\hspace{0.8em}}}{}{\titlerule*[8pt]{.}\contentspage}

% 设置 part 和 chapter 标题格式。
\ctexset{
	part/name= {卷,},
	part/number={\chinese{part}},
	chapter/name={},
	chapter/number={}
}

% 设置古文原文格式。
\newenvironment{yuanwen}{\bfseries\zihao{4}}

% 设置署名格式。
\newenvironment{shuming}{\hfill\bfseries\zihao{4}}

% 注脚每页重新编号,避免编号过大。
\usepackage[perpage]{footmisc}

\title{\heiti\zihao{0} 三国志}
\author{}
\date{}

\begin{document}

\maketitle
\tableofcontents

\frontmatter
\chapter{前言、序言}

《三国志》是记述东汉未年到西晋统一间这段历史的一部史学名著,它记述的主要是魏、蜀、吴三国的历史,故称《三国志》。《三国志》历来备受推崇,它与《史记》、《汉书》、《后汉书》合称前四史,而前四史被公认为是二十四史中成就最高的四部史书。《后汉书》的作者范晔是南朝刘宋人,成书时间晚于《三国志》一百多年,《三国志》实际是二十四史中紧承《史记》、《汉书》的第三部史著。

作者陈寿,字承祚,蜀汉巴西郡安汉县(今四川南充)人,生于蜀汉建兴十一年(233),卒于西晋元康七年(297),年六十五岁。陈寿从小好学,“聪慧敏识,属文富艳”,曾从蜀汉的著名史学家谯周学习,研读《尚书》、《春秋》等经史著作,尤精于《史记》、《汉书》,这为他后来撰写《三国志》打下了很好的基础。蜀汉被灭后,陈寿以亡国之臣的身份入魏。他历仕蜀汉、曹魏、西晋三朝,曾任东观秘书郎、散骑黄门侍郎、著作郎、平阳侯相、治书侍御史等职,一生仕途坎坷,官位不显,所以《华阳国志·陈寿传》说:“位望不充其才,当时冤之。”

陈寿的著述很多,撰有《益部耆旧传》十篇、《古国志》五十篇,又编纂《诸葛亮集》、《魏名臣奏事》、《汉名臣奏事》等书。吴平之后,陈寿始“鸠合三国史,著魏、吴、蜀三书六十五卷,号《三国志》”。《三国志》写成后,曾在士大夫间流传,立刻就得到了人们的认可,“时人称其善叙事,有良史之才”。当时人夏侯湛正在修《魏书》,见到陈寿的《三国志》,便将自己的书毁掉,不再继续撰写。朝廷重臣张华非常欣赏陈寿,认为他的史才要超过《史记》的作者司马迁和《汉书》的作者班固,准备将修《晋书》的事情托付给他。陈寿去世后,尚书郎范觏等人上书说:“故治书侍御史陈寿作《三国志》,词多劝戒,明乎得失,有益风化。虽文艳不如相如,而质直过之,愿垂采纳。”于是晋惠帝下诏,令河南尹、洛阳令到陈寿家抄录《三国志》。《三国志》得到官方的认可,正式流传。

《三国志》之所以备受推崇,是因为它有许多突出的优点。

《三国志》六十五卷,其中《魏书》三十卷、《蜀书》十五卷、《吴书》二十卷,是一部纪传体史书。在陈寿之前,司马迁的《史记》贯穿古今,是第一部纪传体史书;班固的《汉书》,则是一部纪传体的断代史。为同时并立的三国修史,是陈寿面对的新问题,于是他另辟蹊径,分作《魏书》、《蜀书》、《吴书》为三国各自修史,然后合为一书,总其名为《三国志》,非常巧妙地解决了这个难题。这充分显示了陈寿的史才,同时也使《三国志》成为二十四史中别具一格的史书。

陈寿修《三国志》,眼光未局限于三国,而是上接汉末,为群雄董卓、袁绍、袁术、吕布等人立传。这是很有见地的做法,因为讲三国的历史离不开汉末的历史,离不开当时的这些风云人物,如果缺了汉末群雄在历史大舞台上的表演,轰轰烈烈的三国历史将会大为减色。陈寿虽分为魏、蜀、吴三国修史,但以《魏书》为主,即所谓以魏为正统。魏的君主依帝王例立本纪,蜀、吴的君主则降低一格,分别立传。另外,在谴词用字、人物的称谓上也体现出这种区别,如称魏君主为帝,蜀君主称先主、后主,吴则称吴主或径称其名等。以魏为正统,是陈寿颇受非议的地方,但这却是陈寿的无奈之举。陈寿修《三国志》是在吴平之后,大体上是当代人修当代史。西晋统治者司马氏是取代魏登上帝位的,只有承认魏的正统地位,才可以证明取代者司马氏的正统。而实际上从修史体例来说,他是将蜀、吴二国当做与魏并列的王朝来处理的,蜀、吴二国君主的传记,都是以本纪的规格来写,即以编年为序来记述传主的言行事迹,并以此为纲来记述一朝的政治、经济、文化等方方面面的重大事件。其实这正说明了陈寿的苦心,由此说来,人们对他的这种非议是不足一驳的。

在《三国志》的撰写上,陈寿取材谨严,剪裁得当,坚持以求实的态度修史。陈寿修《三国志》,可供他选用参考的资料,有魏王沈的《魏书》、鱼豢的《魏略》,吴韦昭的《吴书》等,还有他自己掌握的蜀国资料。他在史料的选用取舍上非常用心,皆再三审慎斟酌后才予采用。清代学者赵翼在批评陈寿的同时,也不得不承认他“剪裁斟酌处,亦自有下笔不苟者,参订他书而后知其矜慎也”,并列举大量例证说明他在资料使用上的剪裁得当。

《三国志》叙事生动简洁,语言洗练干净,评论中肯得当,毫无繁冗之词。也有学者认为《三国志》质朴有余,文采不足。当然,如果与《史记》那样的极品相比,《三国志》整体上的文采确是略逊一筹,然而在具体章节上,却不乏精彩的描写。比如赤壁之战,陈寿将相关史料分别放在《魏书·武帝纪》、《蜀书·先主传》、《蜀书·诸葛亮传》、《吴书·吴主传》、《吴书·周瑜传》、《吴书·鲁肃传》等几个纪传中,通过各有侧重的描写,把赤壁之战渲染得有声有色,尤其是其中吴蜀双方、各自的君臣之间的对话,非常传神。所以宋代大史学家司马光修《资治通鉴》时,对于这段历史,基本采用了《三国志》的记述,有些地方如《蜀书·诸葛亮传》所载诸葛亮与孙权的对话,几乎全文照录,这也从侧面证明了陈寿文字功夫的精到。又他在纪传后面的评论,不仅颇具文采,而且大都贴切公允,寓意深刻,堪称点睛之笔。比如看了陈寿对曹操精彩的评论,就会发现后人对曹操所谓的翻案,并没有太大的意义,陈寿对曹操早已作了非常恰当的评价。

《三国志》也有一些不足之处:

1.有纪传而无志。志是专门记载政治、经济、天文、地理、礼乐等典章制度的,它是对当时社会生活的全面反映,是后人了解认识这一时期历史所凭依的资料。在《三国志》前边的《史记》、《汉书》,都有这方面的内容。《三国志》无志,可能有社会动乱不已、资料不足等多种原因,但一部完整的纪传体史书无志,不能不说是一个大的缺憾。

2.叙述过于简略。《三国志》叙事简洁是它的优点,但对于史书来说,还是要给后人尽量多地留下有价值的资料。这一点陈寿做得有些不够,一些重要历史人物的传记资料很少,与传主的身份地位很不相称。如关羽、张飞、赵云,乃是蜀汉的开国元勋,而《蜀书·关羽传》仅一千二百余字,《蜀书·张飞传》仅八百余字,《蜀书·赵云传》仅四百余字。又如徐幹、陈琳、应场、阮瑀等建安时期的著名文人,皆未立专传,只附记在《魏书·王粲传》中,而且除记陈琳事有三四百言外,其余皆寥寥数语。这其中也有资料不足的原因,《蜀书·后主传》的评论说:“国不置史,注记无官,是以行事多遗,灾异靡书。”不设史官,当然记录下来的资料就不会太多。

3.对曹氏、司马氏等统治者隐恶溢美,曲笔回护。这是陈寿最受垢病的地方。中国的修史传统,讲究直书实录,不隐恶,不扬善,给后人留下信史。唐代史学家刘知已批评陈寿对司马氏篡权弑主事未置一词,不肯如实记录。清代学者赵翼《甘二史札记》有“《三国志》多回护”一条,专论曲笔回护事。考之史实,这确实不是虚词。但所谓曲笔回护,是专制时代无法完全避免的。陈寿作为晋人,让他直指甚至揭露当朝者的丑恶行径是不现实的,有时甚至还要为他们粉饰,对他来说,这也是无奈之举。而实际上陈寿在对统治者有所回护的同时,也对他们作了一些隐讳的批评,如在《魏书·文帝纪》的评论中,他就对魏文帝的心胸狭隘进行了批评。而且同样是曲笔,要看是有意还是无奈,还要看程度的多少。考察《三国志》,毕竟直笔实录的多,曲笔回护的少,整体上是实录。

4.此外,还有人批评陈寿借修史谋取私利和发泄私愤,古今学者对此作了驳正,都是不足凭信的虚言。

《三国志》一个与众不同的特点是它的注。陈寿去世一百多年后,随着有关三国的史料的逐渐出现,南朝宋文帝刘义隆认为《三国志》过于简略,命时任中书侍郎的裴松之为之作注。裴松之字世期,河东郡闻喜县(今山西闻喜)人,在他的祖父时,裴家迁居江南。他从小好学,“八岁学通《论语》、《毛诗》。博览坟籍,立身简素”,著述甚丰。他的儿子裴骃也是著名的史学家,曾撰后来被称为《史记》三家注的《史记集解》一书。裴松之领命后,“鸠集传记,增广异闻”,于刘宋元嘉六年(429)将《三国志注》完成。书成奏上,宋文帝非常满意,赞扬说:“此为不朽矣。”

裴松之的注不同于传统的注,重点不在于对语言、名物、制度的考证解释,而在于对史事的补缺、备异、惩妄、论辩。其中补缺、备异,是资料的补充,是对于所引用资料的归纳与整理。惩妄、论辩,则是对于所引用资料的考证与批评。这就是说,他不仅仅是简单的搜集罗列资料,而且要经过校勘考证,提出自己的观点。有学者考证,裴松之引用的书达二百一十种,其中大部分已经亡佚,而他所引史事,大多首尾俱全,未加删节,这就为后人留下了大量珍贵的资料。所以《四库全书总目》说裴注“转相引据者,反多于陈寿本书焉”。比如关羽、赵云二人传记的简略问题,经裴注得到了很好的解决。《蜀书·关羽传》补充约一千字,其中有关羽喜读《左传》、许田射猎等事。《蜀书·赵云传》补充一千四百余字,使人物形象更加丰满,其中有赵云截江夺阿斗等事。这些大量的极其有用的资料,为后人阅读理解这部史学名著提供了很大的帮助,同时也为后人创作《三国演义》提供了素材。

对于裴松之的注,人们也有非议之词,比较突出的是认为他引用资料过于芜杂烦琐。《四库全书总目》就说裴松之“往往嗜奇爱博,颇伤芜杂”,这种说法有一定的道理。至于有的学者批评他“注之所载,皆寿之弃余”,就不够客观了,他所引用的资料,大部分出于陈寿同时人或后人的著作,陈寿根本没有见到它们的可能。而有些可能被批评者引以为据的东西,裴松之的初衷并不全是为补充资料,一定程度上是为了表达自己的史学观点。比如著名的“空城计”的故事,它最早见于西晋郭冲的《诸葛亮五事》,根据史实,是毫无根据的妄说。所以陈寿未予采纳。裴松之虽然引用,但应该是出于惩妄的目的。所以他在引用的同时,对这种说法的荒谬以及停于史实,都作了批评,认为“此书举引皆虚”。而这些虚妄的材料本身,以后人的眼光来看也不是毫无用处,它被后人加工借用,成为《三国演义》中非常精彩的一个章节。
总之,裴松之的注价值极高,它已经与《三国志》成为一体,读《三国志》必须要读裴注。清代学者钱大昭甚至认为,裴松之依据他所掌握的新材料,完全可以自成一史,是因为他自己谦虚,才附于《三国志》下作为注而存在。

由于具有巨大的史学价值和文学价值,《三国志》不仅在史学史上占有崇高的地位,对后世的社会文化也产生了极大的影响。《三国志》很早就流传到海外,有各种文字的版本,海外有众多的学者在研究《三国志》。大约在元末明初,罗贯中依据《三国志》创作《三国演义》,把作为正史的《三国志》通俗化为小说,它最初的名字就叫作《三国志通俗演义》。书中大多数的人物及故事情节都可以在《三国志》及裴注中找到根据或线索,材料的主要来源就是《三国志》及裴注。由于《三国演义》的推波助澜,三国故事在中国可谓家喻户晓,而来源于三国故事的人物、典故、语言等等,更是早已深入到人们的生活当中。各种文艺形式的创作也大量的取材于三国故事,比如作为中国国粹的京剧,甚至专有成套且具相当规模的三国戏。三国故事在日本及东南亚等地广为流传,在西方也有很大的影响。有关三国故事的方方面面早已形成为丰富多彩的三国文化,而三国文化的普及反过来也促进了人们对于《三国志》的了解。就社会影响及普及性而言,《三国志》在中国的史书中是绝无仅有的。总体而言,绚丽多姿的三国文化滥觞于陈寿的《三国志》,而由裴松之的广采博注助扬其波,至《三国演义》的流传,则蔚为大观而成江河了。

《三国志》问世以来,除裴松之外,为它作注的代不乏人,尤其是清代的一些大学者,在这方面更是下了很大的功夫。近代以来,卢弼的《三国志集解》是一部集历代研究成果之大成的著作,是非常详尽的注释本。上世纪八十年代由著名学者缪钺先生主编、中华书局出版的《三国志选注》,是一部注释精审的选本。近年以来,有不少注译本问世,也都各具特色。

\mainmatter

\part{魏书一}

\chapter{武帝纪第一}

《武帝纪》是《三国志》的第一篇,记述的是魏武帝曹操的事迹。陈寿全面而详尽地记述了曹操不平凡的一生。曹操是非常有名的历史人物,他以他的雄才大略,叱咤风云数十年,统一了我国北方,同时也奠定了曹魏王朝的基业。曹操将汉献帝控制在手中后,令由己出,已经“三分天下有其二”,这也是他被后人称为奸雄的原因,但他自己并没有废掉汉帝自立。他的儿子曹不代汉后,追尊他为武皇帝。曹操是大乱世造就的大英雄,他能在群雄中脱颖而出,成就霸业,是因为他具备成就大事的条件。他目光远大,有治平天下的雄心壮志,有百折不挠的勇气和顽强的毅力,有过人的胆识和谋略,有清醒的头脑和恢弘的气度。他精通兵法,用兵如神,善于发现和使用人才。他壮心不已,一生几乎都在征战之中,直至在征途中去世。曹操又是个文采风流、多才多艺的人。他喜欢读书,在军中三十余年,手不释卷,“昼则讲武策,夜则思经传”。他“登高必赋,及造新诗,被之管弦,皆成乐章”,开创了慷慨悲凉的一代诗风,他的文章清峻整洁,他是建安文学的代表作家。他善书法,可与当时的书法名家张芝、张昶相媲美;善围棋,能和当时的高手山子道、王九真一较高低。陈寿说他是“非常之人,超世之杰”,确实不是过誉之词。

\begin{yuanwen}
太祖武皇帝\footnote{曹操的儿子曹丕称帝后,追尊曹操为武皇帝,定庙号为太祖。},沛国\footnote{王国名,今安徽濉溪西北。}谯\footnote{县名,今安徽亳县。}人也,姓曹,讳\footnote{古代对帝王及尊长不能直呼其名,故称讳,以示尊重。}操,字孟德,汉相国\footnote{丞相的尊称。}参\footnote{曹参,西汉大臣,曾任相国。}之后。\footnote{曹瞒传曰:太祖一名吉利,小字阿瞒。王沈魏书曰:其先出于黄帝。当高阳世,陆终之子曰安,是为曹姓。周武王克殷,存先世之后,封曹侠于邾。春秋之世,与于盟会,逮至战国,为楚所灭。子孙分流,或家于沛。汉高祖之起,曹参以功封平阳侯,世袭爵土,绝而复绍,至今适嗣国于容城。} 桓帝\footnote{汉恒帝刘志,公元146--147年在位。}世,曹腾为中常侍\footnote{宦官名,掌管传达诏令及宫中文书,权力很大。}大长秋\footnote{官名,东汉时多由宦官担任,为皇后近侍,掌管传达皇后旨意及宫中事务。},封费亭侯\footnote{封爵名。侯爵的一种。东汉封侯,根据功劳大小分为县、乡、亭侯,各享有数目不同的食邑。}。\footnote{司马彪续汉书曰:腾父节,字符伟,素以仁厚称。邻人有亡豕者,与节豕相类,诣门认之,节不与争;后所亡豕自还其家,豕主人大惭,送所认豕,并辞谢节,节笑而受之。由是乡党贵叹焉。长子伯兴,次子仲兴,次子叔兴。腾字季兴,少除黄门从官。永宁元年,邓太后诏黄门令选中黄门从官年少温谨者配皇太子书,腾应其选。太子特亲爱腾,饮食赏赐与众有异。顺帝即位,为小黄门,迁至中常侍大长秋。在省闼三十余年,历事四帝,未尝有过。好进达贤能,终无所毁伤。其所称荐,若陈留虞放、边韶、南阳延固、张温、弘农张奂、颍川堂溪典等,皆致位公卿,而不伐其善。蜀郡太守因计吏修敬于腾,益州刺史种暠于函谷关搜得其笺,上太守,并奏腾内臣外交,所不当为,请免官治罪。帝曰:“笺自外来,腾书不出,非其罪也。”乃寝暠奏。腾不以介意,常称叹暠,以为暠得事上之节。暠后为司徒,语人曰:“今日为公,乃曹常侍恩也。”腾之行事,皆此类也。桓帝即位,以腾先帝旧臣,忠孝彰着,封费亭侯,加位特进。太和三年,追尊腾曰高皇帝。}养子嵩嗣\footnote{继承。},官至太尉\footnote{官名。三公之一,掌管全国军事,为全国最高军事长官。},莫能审\footnote{了解,明白。}其生出本末。\footnote{续汉书曰:嵩字巨高。质性敦慎,所在忠孝。为司隶校尉,灵帝擢拜大司农、大鸿胪,代崔烈为太尉。黄初元年,追尊嵩曰太皇帝。吴人作曹瞒传及郭颁世语并云:嵩,夏侯氏之子,夏侯惇之叔父。太祖于惇为从父兄弟。}嵩生太祖。
\end{yuanwen}

太祖武皇帝,沛国谯县人,姓曹名操,字孟德,是汉相国曹参的后代。汉桓帝世,曹腾任中常侍大长秋,封费亭侯,曹腾的养子曹嵩继承了曹腾的爵位,官至太尉,没人能搞清楚他的来历。曹嵩生下太祖。

\begin{yuanwen}
太祖少机警,有权数,而任侠放荡,不治行业\footnote{品行学业。},故世人未之奇也;\footnote{曹瞒传云:太祖少好飞鹰走狗,游荡无度,其叔父数言之于嵩。太祖患之,后逢叔父于路,乃阳败面喎口;叔父怪而问其故,太祖曰:“卒中恶风。”叔父以告嵩。嵩惊愕,呼太祖,太祖口貌如故。嵩问曰:“叔父言汝中风,已差乎?”太祖曰:“初不中风,但失爱于叔父,故见罔耳。”嵩乃疑焉。自后叔父有所告,嵩终不复信,太祖于是益得肆意矣。}惟梁国\footnote{王国名,今河南商丘南。}桥玄\footnote{东汉大臣,官至太尉,以知人名世。}、南阳\footnote{郡名。今河南南阳。}何颙\footnote{东汉大臣,曾与王允等人谋诛董卓,后忧愤而死。}异焉。玄谓太祖曰:“天下将乱,非命世之才\footnote{安邦济世的人才。}不能济也,能安之者,其在君乎!”\footnote{魏书曰:太尉桥玄,世名知人,鷪太祖而异之,曰:“吾见天下名士多矣,未有若君者也!君善自持。吾老矣!愿以妻子为托。”由是声名益重。续汉书曰:玄字公祖,严明有才略,长于人物。张璠汉纪曰:玄历位中外,以刚断称,谦俭下士,不以王爵私亲。光和中为太尉,以久病策罢,拜太中大夫,卒,家贫乏产业,柩无所殡。当世以此称为名臣。世语曰:玄谓太祖曰:“君未有名,可交许子将。”太祖乃造子将,子将纳焉,由是知名。孙盛异同杂语云:太祖尝私入中常侍张让室,让觉之;乃舞手戟于庭,踰垣而出。才武绝人,莫之能害。博览髃书,特好兵法,抄集诸家兵法,名曰接要,又注孙武十三篇,皆传于世。尝问许子将:“我何如人?”子将不答。固问之,子将曰:“子治世之能臣,乱世之奸雄。”太祖大笑。}
\end{yuanwen}

太祖从小机警,富有权数计谋,行侠仗义,放荡不羁,不大注意培养自己的品行学业,所以当时人没很看重他;只有梁国人桥玄、南阳人何颙对他非常赏识。桥玄对他说:“天下将要大乱,非安邦定国之才不能挽救,能安定天下的,大概就是您了!”

\begin{yuanwen}
年二十,举孝廉\footnote{汉代察举科目之一,被举之人须孝顺父母,行为廉洁。察举为汉代选举制度的一种。}为郎\footnote{郎官,汉代中郎、侍郎、郎中等官的通称。掌管皇帝侍从宿卫,并可参议朝政。},除洛阳\footnote{东汉都城,在今河南洛阳东。}北部尉\footnote{官名。汉代县设县尉,为县令长之副手,掌管纠察盗贼,维护地方治安。县尉大县二人,小县一人。洛阳为东汉都城,设尉不止一人,故有北部尉。},迁\footnote{升职。}顿丘\footnote{县名。今河南清丰西南。}令\footnote{县令,官名。汉代县之主官,万户以上县称县令,万户以下县称县长。},\footnote{曹瞒传曰:太祖初入尉廨,缮治四门。造五色棒,县门左右各十余枚,有犯禁,不避豪强,皆棒杀之。后数月,灵帝爱幸小黄门蹇硕叔父夜行,即杀之。京师敛夡,莫敢犯者。近习宠臣咸疾之,然不能伤,于是共称荐之,故迁为顿丘令。}征\footnote{征聘。汉代选举制度之一,由朝廷直接征聘。}拜\footnote{任命。}议郎\footnote{官名。郎官的一种,属光禄勋,掌管参议朝政。}。\footnote{魏书曰:太祖从妹夫□强侯宋奇被诛,从坐免官。后以能明古学,复征拜议郎。先是大将军窦武、太傅陈蕃谋诛阉官,反为所害。太祖上书陈武等正直而见陷害,奸邪盈朝,善人壅塞,其言甚切;灵帝不能用。是后诏书敕三府:举奏州县政理无效,民为作谣言者免罢之。三公倾邪,皆希世见诏用,货赂并行,强者为怨,不见举奏,弱者守道,多被陷毁。太祖疾之。是岁以灾异博问得失,因此复上书切谏,说三公所举奏专回避贵戚之意。奏上,天子感悟,以示三府责让之,诸以谣言征者皆拜议郎。是后政教日乱,豪猾益炽,多所摧毁;太祖知不可匡正,遂不复献言。}
\end{yuanwen}

二十岁时,太祖被举荐为孝廉,入朝任郎官,被任命为洛阳北部尉,升任顿丘县令,朝廷又征召他入朝,任命他为议郎。

\begin{yuanwen}
光和\footnote{汉灵帝 178--184}末,黄巾\footnote{黄巾军,东汉末年农民起义军。}起。拜骑都尉\footnote{官名。掌管皇帝侍从宿卫,属光禄勋,位次低于将军。},讨颍川\footnote{郡名。今河南禹县。}贼\footnote{指黄巾军,是统治者对农民起义军的诬蔑称呼。}。迁为济南\footnote{王国名。今山东济南东。}相\footnote{官名。汉代王国由朝廷委派国相一人,掌管王国政务,地位同于郡太守。},国有十余县,长吏\footnote{指县令长。}多阿附贵戚,赃污狼藉,于是奏免其八;禁断淫祀\footnote{指未经官方准许的祭祀。},奸宄\footnote{违法乱纪之徒。}逃窜,郡界\footnote{指济南国境内。国与郡为同级行政单位,封王于其地则为国,未封则为郡,故以郡称国。}肃然。\footnote{魏书曰:长吏受取贪饕,依倚贵势,历前相不见举;闻太祖至,咸皆举免,小大震怖,奸宄遁逃,窜入他郡。政教大行,一郡清平。初,城阳景王刘章以有功于汉,故其国为立祠,青州诸郡转相仿效,济南尤盛,至六百余祠。贾人或假二千石舆服导从作倡乐,奢侈日甚,民坐贫穷,历世长吏无敢禁绝者。太祖到,皆毁坏祠屋,止绝官吏民不得祠祀。及至秉政,遂除奸邪鬼神之事,世之淫祀由此遂绝。}久之,征还为东郡\footnote{郡名。今河南濮阳西南。}太守\footnote{官名。汉代郡之主官,又称郡守。};不就,称疾归乡里。\footnote{魏书曰:于是权臣专朝,贵戚横恣。太祖不能违道取容。数数干忤,恐为家祸,遂乞留宿韂。拜议郎,常托疾病,辄告归乡里;筑室城外,春夏习读书传,秋冬弋猎,以自娱乐。}
\end{yuanwen}

光和末年,黄巾军起事。太祖被任命为骑都尉,率军攻打颍川黄巾军。升任济南相,济南国下辖十余县,各县的县令长大多巴结讨好权贵,贪赃枉法,声名狼藉,太祖于是上奏朝廷,罢免了其中八个县的县令长;又在境内禁绝淫祀,违法乱纪之徒纷纷逃窜,境内安定,秩序井然。过了很久,太祖被朝廷征还,任为东郡太守;他没有去赴任,称病返回家乡。

\begin{yuanwen}
顷之,冀州刺史王芬、南阳许攸、沛国周旌等连结豪杰,谋废灵帝,立合肥侯,以告太祖,太祖拒之。芬等遂败。\footnote{司马彪九州春秋曰:于是陈蕃子逸与术士平原襄楷会于芬坐,楷曰:“天文不利宦者,黄门、常侍*(贵)**[真]*族灭矣。”逸喜。芬曰:“若然者,芬愿驱除。”于是与攸等结谋。}

灵帝欲北巡河间旧宅,芬等谋因此作难,上书言黑山贼攻劫郡县,求得起兵。会北方有赤气,东西竟天,太史上言“当有阴谋,不宜北行”,帝乃止。敕芬罢兵,俄而征之。芬惧,自杀。

魏书载太祖拒芬辞曰:“夫废立之事,天下之至不祥也。古人有权成败、计轻重而行之者,伊尹、霍光是也。伊尹怀至忠之诚,据宰臣之势,处官司之上,故进退废置,计从事立。及至霍光受托国之任,藉宗臣之位,内因太后秉
政之重,外有髃卿同欲之势,昌邑即位日浅,未有贵宠,朝乏谠臣,议出密近,故计行如转圜,事成如摧朽。今诸君徒见曩者之易,未鷪当今之难。诸君自度,结众连党,何若七国?

合肥之贵,孰若吴、楚?而造作非常,欲望必克,不亦危乎!”
\end{yuanwen}

\begin{yuanwen}
金城\footnote{郡名。今甘肃永靖西北。}边章、韩遂杀刺史\footnote{官名。西汉分全国为十三部(州),设置刺史,以诏书六条察问郡县,本属监察官,东汉后权力加大,成为郡之上的一级行政长官。此处指凉州刺史耿鄙,金城为凉州辖郡,凉州治今甘肃张家川回族自治区。}郡守\footnote{指金城太守陈懿。}以叛,众十余万,天下骚动。征太祖为典军校尉\footnote{官名。东汉末,灵帝时为加强军事力量,设置西园八校尉,典军校尉即其中之一。}。会\footnote{恰巧,赶上。}灵帝\footnote{汉灵帝刘宏,公元168--189年在位。}崩\footnote{皇帝去世。},太子\footnote{汉少帝刘辩,公元189年在位,后被董卓废黜杀掉。}即位,太后\footnote{何太后,少帝的生母,后被董卓杀掉。}临朝。大将军\footnote{官名。东汉将军的最高称号,位在三公之上,掌管统兵征伐并执掌朝政,权力极大,不常置。}何进\footnote{何太后的异母兄长,官至大将军,灵帝死后专断朝政,与袁绍等人谋诛宦官,事泄被杀。}与袁绍\footnote{东汉末群雄之一。}谋诛宦官,太后不听。进乃召董卓\footnote{东汉末年祸国权臣。},欲以胁太后,\footnote{魏书曰:太祖闻而笑之曰:“阉竖之官,古今宜有,但世主不当假之权宠,使至于此。既治其罪,当诛元恶,一狱吏足矣,何必纷纷召外将乎?欲尽诛之,事必宣露,吾见其败也。”}卓未至而进见杀。卓到,废帝为弘农王而立献帝\footnote{东汉末代皇帝刘协,少帝刘辩的异母弟,公元189--220年在位,在位时东汉已名存实亡。},京都大乱。卓表\footnote{上表举荐。}太祖为骁骑校尉\footnote{官名。统领京师禁卫军,掌管京师宿卫。},欲与计事。太祖乃变易姓名,间行\footnote{取小路悄悄走。}东归。\footnote{魏曰:太祖以卓终必覆败,遂不就拜,逃归乡里。从数骑过故人成皋吕伯奢;伯奢不在,其子与宾客共劫太祖,取马及物,太祖手刃击杀数人。世语曰:太祖过伯奢。伯奢出行,五子皆在,备宾主礼。太祖自以背卓命,疑其图己,手剑夜杀八人而去。孙盛杂记曰:太祖闻其食器声,以为图己,遂夜杀之。既而凄怆曰:“宁我负人,毋人负我!”遂行。}
\end{yuanwen}

金城人边章、韩遂杀掉凉州刺史和金城太守叛乱,聚集部众十余万人,天下惊扰不安。朝廷征召太祖入朝出任典军校尉。正好赶上汉灵帝去世,太子刘辩即位,何太后临朝听政。大将军何进与袁绍谋划诛杀宦官,何太后不肯听从。何进便征召董卓率军进京,想以此胁迫何太后答应,董卓还没有到达,而何进已先被杀害。董卓到京城后,废黜少帝刘辩为弘农王,另立献帝刘协,京城大乱。董卓上表举荐太祖为骁骑校尉,想和他商议政事。太祖于是改换姓名,从小路向东悄悄返回家乡。

\begin{yuanwen}
出关\footnote{虎牢关,今河南荥阳汜水镇。},过中牟\footnote{县名。今河南中牟东。},为亭长\footnote{官名。汉代乡间十里为一亭,设亭长一人,负责当地治安等事务。}所疑,执\footnote{捉拿,拘捕。}诣\footnote{y\`i,前往;到某地去。}县,邑中或窃识之,为请得解。\footnote{世语曰:中牟疑是亡人,见拘于县。时掾亦已被卓书;唯功曹心知是太祖,以世方乱,不宜拘天下雄鉨,因白令释之。}卓遂杀太后及弘农王。太祖至陈留\footnote{县名。今河南开封东南。},散家财,合义兵,将以诛卓。冬十二月,始起兵于己吾\footnote{县名。今河南宁陵西南。},\footnote{世语曰:陈留孝廉韂兹以家财资太祖,使起兵,众有五千人。}是岁中平\footnote{汉灵帝年号(184--189)}六年也。
\end{yuanwen}

出虎牢关,途经中牟时,一个亭长对太祖产生怀疑,把他抓起来送到县城。县城里有人暗中认出了他,便为他向县里求情,结果太祖被释放。这时董卓在京城杀掉何太后和弘农王。太祖到达陈留,散出家中的财物,聚集义兵,准备凭借这支军队消灭董卓。冬十二月,太祖开始在已吾起兵举义,这一年是中平六年。

\begin{yuanwen}
初平\footnote{汉献帝年号(190--193)。}元年春正月,后将军\footnote{官名。与前、左、右将军为四将军,掌统兵征伐,地位很高。}袁术\footnote{东汉末群雄之一。}、冀州\footnote{州名。今河北临漳西南。}牧\footnote{州牧,官名。汉代各州或置州牧,或置刺史,二者职位相同而州牧稍重。}韩馥\footnote{f\`u}、\footnote{英雄记曰:馥字文节,颍川人。为御史中丞。董卓举为冀州牧。于时冀州民人殷盛,兵粮优足。袁绍之在勃海,馥恐其兴兵,遣数部从事守之,不得动摇。东郡太守桥瑁诈作京师三公移书与州郡,陈卓罪恶,云“见逼迫,无以自救,企望义兵,解国患难。”馥得移,请诸从事问曰:“今当助袁氏邪,助董卓邪?”治中从事刘子惠曰:“今兴兵为国,何谓袁、董!”馥自知言短而有惭色。子惠复言:“兵者凶事,不可为首;今宜往视他州,有发动者,然后和之。冀州于他州不为弱也,他人功未有在冀州之右者也。”馥然之。馥乃作书与绍,道卓之恶,听其举兵。}豫州\footnote{州名。今安徽亳县。}刺史孔伷\footnote{zh\`ou}、\footnote{英雄记曰:伷字公绪,陈留人。张璠汉纪载郑泰说卓云:“孔公绪能清谈高论,嘘枯吹生。”}兖州\footnote{州名。今山东金乡西北。}刺史刘岱、\footnote{岱,刘繇之兄,事见吴志。}河内\footnote{郡名。今河南武陟西南。}太守王匡\footnote{英雄记曰:匡字公节,泰山人。轻财好施,以任侠闻。辟大将军何进府进符使,匡于徐州发强弩五百西诣京师。会进败,匡还州里。起家,拜河内太守。谢承后汉书曰:匡少与蔡邕善。其年为卓军所败,走还泰山,收集劲勇得数千人,欲与张邈合。匡先杀执金吾胡母班。班亲属不胜愤怒,与太祖并势,共杀匡。}、勃海\footnote{郡名。今河北南皮东北。}太守袁绍、陈留太守张邈、东郡太守桥瑁\footnote{英雄记曰:瑁字符伟,玄族子。先为兖州刺史,甚有威惠。}、山阳\footnote{郡名。今山东金乡西北,为兖州治所。}太守袁遗\footnote{遗字伯业,绍从兄。为长安令。河间张超尝荐遗于太尉朱鉨,称遗“有冠世之懿,干时之量。其忠允亮直,固天所纵;若乃包罗载籍,管综百氏,登高能赋,鷪物知名,求之今日,邈焉靡俦。”事在超集。英雄记曰:绍后用遗为扬州刺史,为袁术所败。太祖称“长大而能勤学者,惟吾与袁伯业耳。”语在文帝典论。}、济北\footnote{王国名。今山东长清南。}相鲍信\footnote{信事见子勋传。}同时俱起兵,众各数万,推绍为盟主。太祖行\footnote{代理。}奋武将军\footnote{官名。汉杂号将军,掌统兵征伐。汉代杂号将军名目繁多,不常置,因事临时而设,战事结束则罢。}。
\end{yuanwen}

初平元年春正月,后将军袁术、冀州牧韩馥、豫州刺史孔伷、兖州刺史刘岱、河内太守王匡、勃海太守袁绍、陈留太守张邈、东郡太守桥瑁、山阳太守袁遗、济北相鲍信同时起兵讨伐董卓,各有部众数万,推举袁绍为盟主。太祖代理奋武将军。

\begin{yuanwen}
二月,卓闻兵起,乃徙天子都长安\footnote{西汉都城,在今陕西长安西北。}。卓留屯洛阳,遂焚宫室。

是时绍屯河内,邈、岱、瑁、遗屯酸枣\footnote{县名。今河南延津西南。},术屯南阳,伷屯颍川,馥在邺\footnote{yè,县名。今河北临漳西南。}。卓兵强,绍等莫敢先进。太祖曰:“举义兵以诛暴乱,大众已合,诸君何疑?向使\footnote{假使。}董卓闻山东\footnote{崤山以东地区。}兵起,倚王室之重,据二周\footnote{指战国时西周、东周两个小国。西周位于今河南洛阳西,东周位于今河南洛阳东。}之险,东向以临天下;虽以无道行之,犹足为患。今焚烧宫室,劫迁天子,海内震动,不知所归,此天亡之时也。一战而天下定矣,不可失也。”遂引兵西,将据成皋\footnote{gāo,县名。今河南荥阳西北。}。邈遣将卫兹分兵随太祖。

到荥阳\footnote{县名。今河南荥阳东北。}汴\footnote{biàn}水,遇卓将徐荣,与战不利,士卒死伤甚多。太祖为流矢所中,所乘马被创,从弟\footnote{堂弟。}洪以马与太祖,得夜遁去。荣见太祖所将兵少,力战尽日,谓酸枣未易攻也,亦引兵还。
\end{yuanwen}

这年二月,董卓闻知袁绍等人起兵,便将献帝迁徙到长安,以长安为都城。董卓自己率军留驻洛阳,于是纵火焚烧洛阳的皇官。

当时袁绍驻军河内,张邈、刘岱、桥瑁、袁遗等人驻军酸枣,袁术驻军南阳,孔伷驻军颖川,韩馥驻军邺县。董卓兵力强盛,袁绍等没人敢率先进军。太祖说:“我们起义兵来诛灭暴乱,大军已经聚合,诸位还迟疑什么?假使董卓闻知山东起兵,依仗皇家的权势威望,据守洛阳东西两边的险要之地,向东进军以控制天下,那即使他的行为不合道义,也还是足以成为祸患。现在他却焚烧皇官,劫持迁徙天子海内震动,人们不知道去归附谁,这是上天灭亡他的好时机啊。只须一战就可安定天下,这个好机会不能失掉。”便自己领兵西进,准备占据成皋。张邈分出一部分将士交与手下将领卫兹统领,派遣他跟随太祖进军。

军至荥阳汴水,遇到董卓的将领徐荣,太祖与之交战不利,土卒死伤甚多。太祖被流箭射中,所乘战马受伤,他的堂弟曹洪把自己的战马给他乘骑,他才得以乘夜逃走。徐荣见太祖所率领的军队数量很少,却能够力战一整天,认为酸枣不容易攻下,便也率军返回。

\begin{yuanwen}
太祖到酸枣,诸军兵十余万,日置酒高会,不图进取。太祖责让\footnote{责备。}之,因为谋曰:“诸君听吾计,使勃海\footnote{指袁绍。}引河内之众临孟津\footnote{黄河渡口。位于今河南孟县西南,设有关隘。},酸枣诸将守成皋,据敖仓\footnote{大粮仓名。位于今河南郑州西北。},塞轘\footnote{hu\'an}辕\footnote{关名。位于今河南偃师东南。}、太谷\footnote{关名。位于今河南洛阳东南。},全制其险;使袁将军率南阳之军军丹\footnote{丹水县,今河南淅川西。}、析\footnote{县名。今河南西峡。},入武关\footnote{关名。位于今陕西丹凤东南。},以震三辅\footnote{西汉武帝时,以京兆尹、右扶风、左冯翊分管京师长安附件地区,称三辅。}:皆高垒深壁,勿与战,益为疑兵,示天下形势,以顺诛逆,可立定也。今兵以义动,持疑而不进,失天下之望,窃\footnote{敬词,私下。}为诸君耻之!”邈等不能用。
\end{yuanwen}

太祖回到酸枣,那里的各路军队十余万人,天天置办酒宴聚会作乐,不谋划进攻董卓的大事。太祖责备他们,并趁便为他们谋划说:“诸位请听我的计策,让勃海太守领河内的军队迫近孟津,酸枣的诸位将军镇守成皋,占据敖仓,阻断轘辕关和太古关,完全控制险要之地;让袁术将军率领南阳的将士进军丹水、析县,进人武关,以威慑三辅地区:各路军队都高筑壁垒,深挖堑壕,不与敌人交战,多设疑兵,显示出有利于我军的天下大势,以正义讨伐叛逆,可以立刻平定。现在我们以正义起兵,却心存疑虑不敢前进,令天下人大失所望,我私下为诸位感到羞耻!”张邈等人没能采纳太祖的计划。

\begin{yuanwen}
太祖兵少,乃与夏侯惇\footnote{dūn}等诣扬州募兵,刺史陈温、丹杨太守周昕\footnote{xīn}与兵四千余人。还到龙亢,士卒多叛。\footnote{魏书曰:兵谋叛,夜烧太祖帐,太祖手剑杀数十人,余皆披靡,乃得出营;其不叛者五百余人。刘岱与桥瑁相恶,岱杀瑁,以王肱领东郡太守。}至铚\footnote{zhì}、建平,复收兵得千余人,进屯河内。
\end{yuanwen}

\begin{yuanwen}
袁绍与韩馥谋立幽州牧刘虞为帝,太祖拒之。\footnote{魏书载太祖答绍曰:“董卓之罪,暴于四海,吾等合大众、兴义兵而远近莫不响应,此以义动故也。今幼主微弱,制于奸臣,未有昌邑亡国之衅,而一旦改易,天下其孰安之?诸君北面,我自西向。”}绍又尝得一玉印,于太祖坐中举向其肘,太祖由是笑而恶焉。\footnote{魏书曰:太祖大笑曰:“吾不听汝也。”绍复使人说太祖曰:“今袁公势盛兵强,二子已长,天下髃英,孰踰于此?”太祖不应。由是益不直绍,图诛灭之。}

二年春,绍、馥遂立虞为帝,虞终不敢当。
夏四月,卓还长安。
秋七月,袁绍胁韩馥,取冀州。
黑山贼于毒、白绕、眭固等*眭,申随反。*十余万众略魏郡、东郡,王肱不能御,太祖引兵入东郡,击白绕于濮阳,破之。袁绍因表太祖为东郡太守,治东武阳。
\end{yuanwen}

\begin{yuanwen}
三年春,太祖军顿丘,毒等攻东武阳。太祖乃引兵西入山,攻毒等本屯。\footnote{魏书曰:诸将皆以为当还自救。太祖曰:“孙膑救赵而攻魏,耿弇欲走西安攻临菑。使贼闻我西而还,武阳自解也;不还,我能败其本屯,虏不能拔武阳必矣。”遂乃行。}毒闻之,弃武阳还。太祖要击眭\footnote{su\=i}固,又击匈奴於夫罗于内黄,皆大破之。\footnote{魏书曰:于夫罗者,南单于子也。中平中,发匈奴兵,于夫罗率以助汉。会本国反,杀南单于,于夫罗遂将其众留中国。因天下挠乱,与西河白波贼合,破太原、河内,抄略诸郡为寇。}

夏四月,司徒王允与吕布共杀卓。卓将李傕\footnote{ju\'e}、郭汜\footnote{s\`i}等杀允攻布,布败,东出武关。傕等擅朝政。
\end{yuanwen}

\begin{yuanwen}
青州\footnote{州名。今山东淄博东北。}黄巾众百万入兖州,杀任城\footnote{王国名。今山东济宁。}相郑遂,转入东平\footnote{王国名。今山东东平东。}。刘岱欲击之,鲍信谏曰:“今贼众百万,百姓皆震恐,士卒无斗志,不可敌也。观贼众群辈相随,军无辎重,唯以钞略为资,今不若畜士众之力,先为固守。彼欲战不得,攻又不能,其势必离散,后选精锐,据其要害,击之可破也。”岱不从,遂与战,果为所杀。\footnote{世语曰:岱既死,陈宫谓太祖曰:“州今无主,而王命断绝,宫请说州中,明府寻往牧之,资之以收天下,此霸王之业也。”宫说别驾、治中曰:“今天下分裂而州无主;曹东郡,命世之才也,若迎以牧州,必宁生民。”鲍信等亦谓之然。}信乃与州吏万潜等至东郡迎太祖领\footnote{兼任。}兖州牧。遂进兵击黄巾于寿张\footnote{县名。今山东东平西南。}东。信力战斗死,仅而破之。\footnote{魏书曰:太祖将步骑千余人,行视战地,卒抵贼营,战不利,死者数百人,引还。贼寻前进。黄巾为贼久,数乘胜,兵皆精悍。太祖旧兵少,新兵不习练,举军皆惧。太祖被甲婴冑,亲巡将士,明劝赏罚,众乃复奋,承闲讨击,贼稍折退。贼乃移书太祖曰:“昔在济南,毁坏神坛,其道乃与中黄太乙同,似若知道,今更迷惑。汉行已尽,黄家当立。天之大运,非君才力所能存也。”太祖见檄书,呵骂之,数开示降路;遂设奇伏,昼夜会战,战辄禽获,贼乃退走。}购求\footnote{悬赏寻求。}信丧\footnote{遗体。}不得,众乃刻木如信形状,祭而哭焉。追黄巾至济北。乞降。冬\footnote{初平三年(192)冬。},受降卒三十余万,男女百余万口,收其精锐者,号为青州兵。
\end{yuanwen}

青州黄巾军上百万人进入充州,杀掉任城相郑遂,转头进入东平。刘岱想进兵攻打他们,鲍信劝速说:“现在贼众有百万人,百姓全都震惊惶恐,士卒没有斗志,抵挡不住他们,我看贼众有老小家属跟随,军中没有辎重,只靠劫掠来供给军用,现在不如积蓄将士们的力量,先据城固守。敌人想要交战没有机会,攻城又攻不下来,势必会分崩离散,到那时我们挑选精锐,占据要害之地,再发动进攻,就可以将他们打败了。”刘岱不肯听从,与黄巾军交战,果然兵败被杀。鲍信便与州吏万潜等人到东郡迎接太祖,请他兼任兖州牧。太祖于是进军,在寿张东与黄巾军展开激战,鲍信临阵战死,太祖军竭尽全力才勉强把黄巾军打败。悬赏寻找鲍信的遗体,没能找到,于是大家用木头刻成鲍信的像,哭着祭奠他。太祖追击黄巾军到济北。黄巾军请求投降。这年冬天,收纳黄巾军降卒三十余万,男女人口一百余万,挑选出其中的精锐士卒组成军队,号称青州兵。

\begin{yuanwen}
袁术与绍有隙,术求援于公孙瓒\footnote{zàn},瓒使刘备屯高唐,单经屯平原,陶谦屯发干,以逼绍。太祖与绍会击,皆破之。

四年春,军鄄\footnote{juàn}城。荆州牧刘表断术粮道,术引军入陈留,屯封丘,黑山余贼及於夫罗等佐之。

术使将刘详屯匡亭。太祖击详,术救之,与战,大破之。术退保封丘,遂围之,未合,术走襄邑,追到太寿,决渠水灌城。走宁陵,又追之,走九江。夏,太祖还军定陶。

下邳\footnote{pī}阙\footnote{què}宣聚众数千人,自称天子;徐州牧陶谦与共举兵,取泰山华、费,略任城。秋,太祖征陶谦,下十余城,谦守城不敢出。

是岁,孙策受袁术使渡江,数年间遂有江东。
\end{yuanwen}

\begin{yuanwen}
兴平元年春,太祖自徐州还,初,太祖父嵩,去官后还谯\footnote{qiáo},董卓之乱,避难琅邪,为陶谦所害,故太祖志在复仇东伐。\footnote{世语曰:嵩在泰山华县。太祖令泰山太守应劭送家诣兖州,劭兵未至,陶谦密遣数千骑掩捕。嵩家以为劭迎,不设备。谦兵至,杀太祖弟德于门中。嵩惧,穿后垣,先出其妾,妾肥,不时得出;嵩逃于厕,与妾俱被害,阖门皆死。劭惧,弃官赴袁绍。后太祖定冀州,劭时已死。韦曜吴书曰:太祖迎嵩,辎重百余两。陶谦遣都尉张闿将骑二百韂送,闿于泰山华、费间杀嵩,取财物,因奔淮南。太祖归咎于陶谦,故伐之。}夏,使荀彧\footnote{xún yù}、程昱\footnote{yù}守鄄城,复征陶谦,拔五城,遂略地至东海。还过郯\footnote{tán},谦将曹豹与刘备屯郯东,要太祖。太祖击破之,遂攻拔襄贲\footnote{b\=en},所过多所残戮\footnote{lù}。\footnote{孙盛曰:夫伐罪吊民,古之令轨;罪谦之由,而残其属部,过矣。}
\end{yuanwen}

\begin{yuanwen}
会张邈与陈宫叛迎吕布,郡县皆应。荀彧、程昱保鄄城,范、东阿二县固守,太祖乃引军还。

布到,攻鄄城不能下,西屯濮\footnote{p\'u}阳。太祖曰:“布一旦得一州,不能据东平,断亢父、泰山之道,乘险要我,而乃屯濮阳,吾知其无能为也。”遂进军攻之。布出兵战,先以骑犯青州兵。
\end{yuanwen}

\begin{yuanwen}
青州兵奔,太祖陈\footnote{通“阵”,指军阵。}乱,驰突火出,坠马,烧左手掌。司马楼异扶太祖上马,遂引去。\footnote{袁暐献帝春秋曰:太祖围濮阳,濮阳大姓田氏为反闲,太祖得入城。烧其东门,示无反意。及战,军败。布骑得太祖而不知是,问曰:“曹操何在?”太祖曰:“乘黄马走者是也。”布骑乃释太祖而追黄马者。门火犹盛,太祖突火而出。}未至营止,诸将未与太祖相见,皆怖。太祖乃自力劳军,令军中促为攻具,进复攻之,与布相守百余日。蝗虫起,百姓大饿,布粮食亦尽,各引去。
\end{yuanwen}

\begin{yuanwen}
秋九月,太祖还鄄城。布到乘氏,为其县人李进所破,东屯山阳。于是绍使人说太祖,欲连和。太祖新失兖\footnote{yǎn}州,军食尽,将许之。程昱止太祖,太祖从之。冬十月,太祖至东阿。

是岁谷一斛\footnote{h\'u,一种计量单位,一斛本为十斗,后来改为五斗。}五十余万钱,人相食,乃罢吏兵新募者。陶谦死,刘备代之。
\end{yuanwen}

\begin{yuanwen}
二年春,袭定陶。济阴太守吴资保南城,未拔。会吕布至,又击破之。夏,布将薛兰、李封屯钜野,太祖攻之,布救兰,兰败,布走,遂斩兰等。布复从东缗\footnote{m\'in}与陈宫将万余人来战,时太祖兵少,设伏,纵奇兵击,大破之。\footnote{魏书曰:于是兵皆出取麦,在者不能千人,屯营不固。太祖乃令妇人守陴,悉兵拒之。屯西有大堤,其南树木幽深。布疑有伏,乃相谓曰:“曹操多谲,勿入伏中。”引军屯南十余里。明日复来,太祖隐兵堤里,出半兵堤外。布益进,乃令轻兵挑战,既合,伏兵乃悉乘堤,步骑并进,大破之,获其鼓车,追至其营而还。}布夜走,太祖复攻,拔定陶,分兵平诸县。布东奔刘备,张邈从布,使其弟超将家属保雍\footnote{yōng}丘。秋八月,围雍丘。冬十月,天子拜太祖兖州牧。

十二月,雍丘溃,超自杀。夷邈三族。邈诣袁术请救,为其众所杀,兖州平,遂东略陈地。

是岁,长安乱,天子东迁,败于曹阳,渡河幸安邑。
\end{yuanwen}

\begin{yuanwen}
建安\footnote{汉献帝年号(196--220)。}元年春正月,太祖军临武平\footnote{县名。今河南鹿邑。},袁术所置陈相\footnote{陈国国相。陈国治今河南淮阳。}袁嗣降。太祖将迎天子\footnote{当时天子汉献帝久经颠沛流离,暂居于安邑县(今山西夏县西北)。曹操准备将汉献帝接到自己身边,以取得政治上的优势。},诸将或疑,荀彧、程昱\footnote{都是曹操的重要谋士。}劝之,乃遣曹洪将兵西迎。卫将军\footnote{官名。统领京师禁卫军,掌管京师及宫廷宿卫,权位稍逊于三公。}董承与袁术将苌\footnote{ch\'ang}奴拒险,洪不得进。
\end{yuanwen}

建安元年春正月,太祖军至武平,袁术设置的陈国相袁嗣投降。太祖准备去迎接汉献帝,众将中有人怀疑这件事,荀彧、程昱二人则极力赞成,太祖于是派遣曹洪率军西进迎接献帝。卫将军董承与袁术将领苌奴凭险阻挡,曹洪无法前进。

\begin{yuanwen}
汝南、颍川黄巾何仪、刘辟、黄邵、何曼等,众各数万,初应袁术,又附孙坚。二月,太祖进军讨破之,斩辟、邵等,仪及其众皆降。
\end{yuanwen}

\begin{yuanwen}
天子拜太祖建德将军,夏六月,迁镇东将军,封费亭侯。秋七月,杨奉、韩暹\footnote{xi\=an}\footnote{都是东汉末军阀混战中的群雄。}以天子还洛阳,\footnote{献帝春秋曰:天子初至洛阳,幸城西故中常侍赵忠宅。使张杨缮治宫室,名殿曰扬安殿,八月,帝乃迁居。}奉别屯梁\footnote{县名。今河南临汝西。}。太祖遂至洛阳,卫京都,暹遁走\footnote{逃走。走,跑,也特指逃跑。}。天子假\footnote{借,此处为授予之意。}太祖节\footnote{符节,皇帝授予臣子行使权力的凭证。}钺\footnote{yu\`e,古代兵器,状如大斧,常作为仪仗表示王权,这时又称黄钺。大臣假黄钺者,表示代表皇帝征讨四方,有权指挥全国军队。},录尚书事\footnote{总领尚书台事务。东汉尚书台是实际的行政中枢,录尚书事者无所不总,实即独揽朝政,后录尚书事成固定官职。}。\footnote{献帝纪曰:又领司隶校尉。}洛阳残破,董昭等劝太祖都许\footnote{县名。今河南许昌东。}。九月,车驾出轘辕而东,以太祖为大将军,封武平侯\footnote{封于武平县的县侯。}。自天子西迁,朝廷日乱,至是宗庙\footnote{皇帝祭祀祖先的处所。}社\footnote{祭祀土神的祭坛。}稷\footnote{祭祀谷神的祭坛。}制度始立。\footnote{张璠汉纪曰:初,天子败于曹阳,欲浮河东下。侍中太史令王立曰:“自去春太白犯镇星于牛斗,过天津,荧惑又逆行守北河,不可犯也。”由是天子遂不北渡河,将自轵关东出。立又谓宗正刘艾曰:“前太白守天关,与荧惑会;金火交会,革命之象也。汉祚终矣,晋、魏必有兴者。”立后数言于帝曰:“天命有去就,五行不常盛,代火者土也,承汉者魏也,能安天下者,曹姓也,唯委任曹氏而已。”公闻之,使人语立曰:“知公忠于朝廷,然天道深远,幸勿多言。”}
\end{yuanwen}

汉献帝任命太祖为建德将军。夏六月,太祖升任镇东将军,封费亭侯。秋七月,杨奉、韩暹护送献帝返回洛阳,杨奉另率一军驻扎梁县。太祖于是进军洛阳,捍卫京师,韩暹逃走。献帝授予太祖符节黄钺,使太祖录尚书事。洛阳已经残破不堪,董昭等人便劝太祖将都城迁至许县。九月,献帝出辍辕关向东进发,任命太祖为大将军,封武平侯。自从献帝西迁长安以来,朝廷上下日益混乱,到这时,宗庙社稷等祭祀制度才开始建立起来,

\begin{yuanwen}
天子之东也,奉自梁欲要\footnote{y\=ao,半路拦截。}之,不及。冬十月,公征奉,奉南奔袁术,遂攻其梁屯,拔之。于是以袁绍为太尉,绍耻班\footnote{等级,位次。}在公下,不肯受。公乃固辞,以大将军让绍。天子拜公司空\footnote{官名。掌管全国的土木建筑及水利工程,为三公之一。},行车骑将军\footnote{官名。统领京师禁卫军,掌管京师及官延宿卫,地位很高,位次同于三公。}。是岁用枣祗\footnote{zhī}、韩浩\footnote{都是曹操手下的重要将领,二人建屯田之议。}等议,始兴屯田\footnote{指曹操建立的强制百姓屯田制度。屯田制度的成功推行,保证了曹操的军粮供应,为他统一北方莫定了物质基础。}。\footnote{魏书曰:自遭荒乱,率乏粮谷。诸军并起,无终岁之计,饥则寇略,饱则弃余,瓦解流离,无敌自破者不可胜数。袁绍之在河北,军人仰食桑椹。袁术在江、淮,取给蒲蠃。民人相食,州里萧条。公曰:“夫定国之术,在于强兵足食,秦人以急农兼天下,孝武以屯田定西域,此先代之良式也。”是岁乃募民屯田许下,得谷百万斛。于是州郡例置田官,所在积谷。征伐四方,无运粮之劳,遂兼灭髃贼,克平天下。}
\end{yuanwen}

在献帝东迁许县的时候,杨奉从梁县进军准备拦截,但能赶上。冬十月,曹公进军征讨杨奉,杨奉向南去投奔袁

\begin{yuanwen}
吕布袭刘备,取下邳。备来奔。程昱说公曰:“观刘备有雄才而甚得众心,终不为人下,不如早图之。”

公曰:“方今收英雄时也,杀一人而失天下之心,不可。”
\end{yuanwen}

\begin{yuanwen}
张济自关中走南阳。济死,从子绣领其众。二年春正月,公到宛。张绣降,既而悔之,复反。

公与战,军败,为流矢所中,长子昂、弟子安民遇害。\footnote{魏书曰:公所乘马名绝影,为流矢所中,伤颊及足,并中公右臂。世语曰:昂不能骑,进马于公,公故免,而昂遇害。}公乃引兵还舞阴,绣将骑来钞,公击破之。绣奔穰\footnote{ráng},与刘表合。公谓诸将曰:“吾降张绣等,失不便取其质,以至于此。吾知所以败。诸卿观之,自今已后不复败矣。”遂还许。\footnote{世语曰:旧制,三公领兵入见,皆交戟叉颈而前。初,公将讨张绣,入觐天子,时始复此制。公自此不复朝见。}
\end{yuanwen}

\begin{yuanwen}
袁术欲称帝于淮南,使人告吕布。布收其使,上其书。术怒,攻布,为布所破。秋九月,术侵陈,公东征之。术闻公自来,弃军走,留其将桥蕤\footnote{ru\'i}、李丰、梁纲、乐就;公到,击破蕤等,皆斩之。术走渡淮。公还许。
\end{yuanwen}

\begin{yuanwen}
公之自舞阴还也,南阳、章陵诸县复叛为绣,公遣曹洪击之,不利,还屯叶,数为绣、表所侵。冬十一月,公自南征,至宛。\footnote{魏书曰:临淯水,祠亡将士,歔欷流涕,众皆感恸。}表将邓济据湖阳。攻拔之,生擒济,湖阳降。攻舞阴,下之。
\end{yuanwen}

\begin{yuanwen}
三年春正月,公还许,初置军师祭酒。三月,公围张绣于穰。夏五月,刘表遣兵救绣,以绝军后。\footnote{献帝春秋曰:袁绍叛卒诣公云:“田丰使绍早袭许,若挟天子以令诸侯,四海可指麾而定。”公乃解绣围。}公将引还,绣兵来追,公军不得进,连营稍前。公与荀彧书曰:“贼来追吾,虽日行数里,吾策之,到安众,破绣必矣。”

到安众,绣与表兵合守险,公军前后受敌。公乃夜凿险为地道,悉过辎重,设奇兵。会明,贼谓公为遁也,悉军来追。乃纵奇兵步骑夹攻,大破之。

秋七月,公还许。荀彧问公:“前以策贼必破,何也?”

公曰:“虏遏\footnote{è}吾归师,而与吾死地战,吾是以知胜矣。”
\end{yuanwen}

\begin{yuanwen}
吕布复为袁术使高顺攻刘备,公遣夏侯惇\footnote{dūn}救之,不利。备为顺所败。

九月,公东征布。

冬十月,屠彭城,获其相侯谐。进至下邳,布自将骑逆击。大破之,获其骁将成廉。追至城下,布恐,欲降。陈宫等沮\footnote{jǔ,阻止。}其计,求救于术,劝布出战,战又败,乃还固守,攻之不下。时公连战,士卒罢\footnote{pí,古同“疲”,累,劳累;困乏。},欲还,用荀攸、郭嘉计,遂决泗、沂水以灌城。

月余,布将宋宪、魏续等执陈宫,举城降,生禽布、宫,皆杀之。太山臧\footnote{zāng}霸、孙观、吴敦、尹礼、昌豨\footnote{x\=i}各聚众。布之破刘备也,霸等悉从布。布败,获霸等,公厚纳待,遂割青、徐二州附于海以委焉,分琅邪、东海、北海为城阳、利城、昌虑郡。
\end{yuanwen}

\begin{yuanwen}
初,公为兖州,以东平毕谌为别驾。张邈之叛也,邈劫谌母弟妻子;公谢遣之,曰:“卿老母在彼,可去。”谌顿首无二心,公嘉之,为之流涕。既出,遂亡归。及布破,谌生得,众为谌惧,公曰:“夫人孝于其亲者,岂不亦忠于君乎!吾所求也。”以为鲁相。[一]
注[一]魏书曰:袁绍宿与故太尉杨彪、大长秋梁绍、少府孔融有隙,欲使公以他过诛之。公曰:“当今天下土崩瓦解,雄豪并起,辅相君长,人怀怏怏,各有自为之心,此上下相疑之秋也,虽以无嫌待之,犹惧未信;如有所除,则谁不自危?且夫起布衣,在尘垢之间,为庸人之所陵陷,可胜怨乎!高祖赦雍齿之雠而髃情以安,如何忘之?”绍以为公外托公义,内实离异,深怀怨望。臣松之以为杨彪亦曾为魏武所困,几至于死,孔融竟不免于诛灭,岂所谓先行其言而后从之哉!非知之难,其在行之,信矣。
四年春二月,公还至昌邑。张杨将杨丑杀杨,眭固又杀丑,以其众属袁绍,屯射犬。夏四月,进军临河,使史涣、曹仁渡河击之。固使杨故长史薛洪、河内太守缪尚留守,自将兵北迎绍求救,与涣、仁相遇犬城。交战,大破之,斩固。公遂济河,围射犬。洪、尚率众降,封为列侯,还军敖仓。以魏种为河内太守,属以河北事。
初,公举种孝廉。兖州叛,公曰:“唯魏种且不弃孤也。”及闻种走,公怒曰:“种不南走越、北走胡,不置汝也!”既下射犬,生禽种,公曰:“唯其才也!”释其缚而用之。
\end{yuanwen}

\begin{yuanwen}
是时袁绍既并公孙瓒,兼四州之地,众十余万,将进军攻许。诸将以为不可敌,公曰:“吾知绍之为人,志大而智小,色厉而胆薄,忌克而少威,兵多而分画不明,将骄而政令不一,土地虽广,粮食虽丰,适足以为吾奉也。”

秋八月,公进军黎阳,使臧霸等入青州破齐、北海、东安,留于禁屯河上。

九月,公还许,分兵守官渡。

冬十一月,张绣率众降,封列侯。

十二月,公军官渡。
\end{yuanwen}

\begin{yuanwen}
袁术自败于陈,稍困,袁谭自青州遣迎之。术欲从下邳北过,公遣刘备、朱灵要之。会术病死。程昱、郭嘉闻公遣备,言于公曰:“刘备不可纵。”公悔,追之不及。备之未东也,阴与董承等谋反,至下邳,遂杀徐州刺史车冑\footnote{zh\`ou},举兵屯沛。遣刘岱、王忠击之,不克。\footnote{献帝春秋曰:备谓岱等曰:“使汝百人来,其无如我何;曹公自来,未可知耳!”魏武故事曰:岱字公山,沛国人。以司空长史从征伐有功,封列侯。魏略曰:王忠,扶风人,少为亭长。三辅乱,忠饥乏噉人,随辈南向武关。值娄子伯为荆州遣迎北方客人;忠不欲去,因率等仵逆击之,夺其兵,聚众千余人以归公。拜忠中郎将,从征讨。五官将知忠尝噉人,因从驾出行,令俳取頉间髑髅系着忠马鞍,以为欢笑。庐江太守刘勋率众降,封为列侯。}
\end{yuanwen}

\begin{yuanwen}
五年春正月,董承等谋泄,皆伏诛。公将自东征备,诸将皆曰:“与公争天下者,袁绍也。今绍方来而弃之东,绍乘人后,若何?”

公曰:“夫刘备,人杰也,今不击,必为后患。\footnote{孙盛魏氏春秋云:答诸将曰:“刘备,人杰也,将生忧寡人。”	臣松之以为史之记言,既多润色,故前载所述有非实者矣,后之作者又生意改之,于失实也,不亦弥远乎!凡孙盛制书,多用左氏以易旧文,如此者非一。嗟乎,后之学者将何取信哉?且魏武方以天下励志,而用夫差分死之言,尤非其类。}袁绍虽有大志,而见事迟,必不动也。”

郭嘉亦劝公,遂东击备,破之,生禽\footnote{通“擒”,捉拿、拿获之意思。}其将夏侯博。

备走奔绍,获其妻子。备将关羽屯下邳,复进攻之,羽降。昌豨叛为备,又攻破之。公还官渡,绍卒不出。
\end{yuanwen}

\begin{yuanwen}
二月,绍遣郭图、淳于琼、颜良攻东郡太守刘延于白马,绍引兵至黎阳,将渡河。

夏四月,公北救延。荀攸说公曰:“今兵少不敌,分其势乃可。公到延津,若将渡兵向其后者,绍必西应之,然后轻兵袭白马,掩其不备,颜良可禽也。”公从之。绍闻兵渡,即分兵西应之。
\end{yuanwen}

\begin{yuanwen}
公乃引军兼行趣白马,未至十余里,良大惊,来逆战。使张辽、关羽前登,击破,斩良。遂解白马围,徙其民,循河而西。绍于是渡河追公军,至延津南。公勒兵驻营南阪下,使登垒望之,曰:“可五六百骑。”有顷,复白:“骑稍多,步兵不可胜数。”公曰:“勿复白。”
\end{yuanwen}

\begin{yuanwen}
乃令骑解鞍放马。是时,白马辎重就道。诸将以为敌骑多,不如还保营。荀攸曰:“此所以饵敌,如何去之!”绍骑将文丑与刘备将五六千骑前后至。诸将复白:“可上马。”公曰:“未也。”有顷,骑至稍多,或分趣辎重。公曰:“可矣。”乃皆上马。时骑不满六百,遂纵兵击,大破之,斩丑。良、丑皆绍名将也,再战,悉禽,绍军大震。公还军官渡。绍进保阳武。关羽亡归刘备。
\end{yuanwen}

\begin{yuanwen}
八月,绍连营稍前,依沙塠\footnote{duī,古同“堆”,堆积。}为屯,东西数十里。公亦分营与相当,合战不利。\footnote{羽凿齿汉晋春秋曰:许攸说绍曰:“公无与操相攻也。急分诸军持之,而径从他道迎天子,则事立济矣。”绍不从,曰:“吾要当先围取之。”攸怒。}时公兵不满万,伤者十二三。\footnote{臣松之以为魏武初起兵,已有众五千,自后百战百胜,败者十二三而已矣。但一破黄巾,受降卒三十余万,余所吞并,不可悉纪;虽征战损伤,未应如此之少也。夫结营相守,异于摧锋决战。本纪云:“绍众十余万,屯营东西数十里。”魏太祖虽机变无方,略不世出,安有以数千之兵,而得逾时相抗者哉?以理而言,窃谓不然。绍为屯数十里,公能分营与相当,此兵不得甚少,一也。绍若有十倍之众,理应当悉力围守,使出入断绝,而公使徐晃等击其运车,公又自出击淳于琼等,扬旌往还,曾无抵阂,明绍力不能制,是不得甚少,二也。诸书皆云公坑绍众八万,或云七万。夫八万人奔散,非八千人所能缚,而绍之大众皆拱手就戮,何缘力能制之?是不得甚少,三也。将记述者欲以少见奇,非其实录也。按钟繇传云:“公与绍相持,繇为司隶,送马二千余匹以给军。”本纪及世语并云公时有骑六百余匹,繇马为安在哉?}绍复进临官渡,起土山地道。公亦于内作之,以相应。绍射营中,矢如雨下,行者皆蒙楯,众大惧。时公粮少,与荀彧书,议欲还许。彧以为:“绍悉众聚官渡,欲与公决胜败。公以至弱当至强,若不能制,必为所乘,是天下之大机也。且绍,布衣之雄耳,能聚人而不能用。夫以公之神武明哲而辅以大顺,何向而不济!”公从之。
\end{yuanwen}

\begin{yuanwen}
孙策闻公与绍相持,乃谋袭许,未发,为刺客所杀。

汝南降贼刘辟等叛应绍,略许下。绍使刘备助辟,公使曹仁击破之。备走,遂破辟屯。

袁绍运谷车数千乘至,公用荀攸计,遣徐晃、史涣\footnote{huàn}邀击,大破之,尽烧其车。公与绍相拒连月,虽比战斩将,然众少粮尽,士卒疲乏。公谓运者曰:“却十五日为汝破绍,不复劳汝矣。”
\end{yuanwen}

\begin{yuanwen}
冬十月,绍遣车运谷,使淳于琼等五人将兵万余人送之,宿绍营北四十里。绍谋臣许攸贪财,绍不能足,来奔,因说公击琼等。左右疑之,荀攸、贾诩劝公。公乃留曹洪守,自将步骑五千人夜往,会明至。琼等望见公兵少,出陈门外。公急击之,琼退保营,遂攻之。绍遣骑救琼。左右或言“贼骑稍近,请分兵拒之”。公怒曰:“贼在背后,乃白!”士卒皆殊死战,大破琼等,皆斩之。\footnote{曹瞒传曰:公闻攸来,跣出迎之,抚掌笑曰:“*(子卿远)**[子远,卿]*来,吾事济矣!”既入坐,谓公曰:“袁氏军盛,何以待之?今有几粮乎?”公曰:“尚可支一岁。”攸曰:“无是,更言之!”又曰:“可支半岁。”攸曰:“足下不欲破袁氏邪,何言之不实也!”公曰:“向言戏之耳。其实可一月,为之柰何?”攸曰:“公孤军独守,外无救援而粮谷已尽,此危急之日也。今袁氏辎重有万余乘,在故市、乌巢,屯军无严备;今以轻兵袭之,不意而至,燔其积聚,不过三日,袁氏自败也。”公大喜,乃选精锐步骑,皆用袁军旗帜,衔枚缚马口,夜从间道出,人抱束薪,所历道有问者,语之曰:“袁公恐曹操钞略后军,遣兵以益备。”闻者信以为然,皆自若。既至,围屯,大放火,营中惊乱。大破之,尽燔其粮谷宝货,斩督将眭元进、骑督韩莒子、吕威璜、赵叡等首,割得将军淳于仲简鼻,未死,杀士卒千余人,皆取鼻,牛马割唇舌,以示绍军。将士皆怛惧。时有夜得仲简,将以诣麾下,公谓曰:“何为如是?”仲简曰:“胜负自天,何用为问乎!”公意欲不杀。许攸曰:“明旦鉴于镜,此益不忘人。”乃杀之。}绍初闻公之击琼,谓长子谭曰:“就彼攻琼等,吾攻拔其营,彼固无所归矣!”乃使张郃\footnote{hé}、高览攻曹洪。郃等闻琼破,遂来降。绍众大溃,绍及谭弃军走,渡河。
\end{yuanwen}

\begin{yuanwen}
追之不及,尽收其辎重图书珍宝,虏其众。\footnote{献帝起居注曰:公上言“大将军邺侯袁绍前与冀州牧韩馥立故大司马刘虞,刻作金玺,遣故任长毕瑜诣虞,为说命录之数。又绍与臣书云:‘可都鄄城,当有所立。’擅铸金银印,孝廉计吏,皆往诣绍。从弟济阴太守□与绍书云:‘今海内丧败,天意实在我家,神应有征,当在尊兄。南兄臣下欲使即位,南兄言,以年则北兄长,以位则北兄重。便欲送玺,会曹操断道。’绍宗族累世受国重恩,而凶逆无道,乃至于此。辄勒兵马,与战官渡,乘圣朝之威,得斩绍大将淳于琼等八人首,遂大破溃。绍与子谭轻身迸走。凡斩首七万余级,辎重财物巨亿。”}公收绍书中,得许下及军中人书,皆焚之。\footnote{魏氏春秋曰:公云:“当绍之强,孤犹不能自保,而况众人乎!”}冀州诸郡多举城邑降者。
\end{yuanwen}

\begin{yuanwen}
初,桓帝时有黄星见于楚、宋之分,辽东殷馗*馗,古逵字,见三苍。*善天文,言后五十岁当有真人起于梁、沛之间,其锋不可当。至是凡五十年,而公破绍,天下莫敌矣。
\end{yuanwen}

\begin{yuanwen}
六年夏四月,扬兵河上,击绍仓亭军,破之。绍归,复收散卒,攻定诸叛郡县。

九月,公还许。绍之未破也,使刘备略汝南,汝南贼共都等应之。遣蔡扬击都,不利,为都所破。公南征备。备闻公自行,走奔刘表,都等皆散。
\end{yuanwen}

\begin{yuanwen}
七年春正月,公军谯,令曰:“吾起义兵,为天下除暴乱。旧土人民,死丧略尽,国中终日行,不见所识,使吾凄怆伤怀。其举义兵已来,将士绝无后者,求其亲戚以后之,授土田,官给耕牛,置学师以教之。为存者立庙,使祀其先人,魂而有灵,吾百年之后何恨哉!”遂至浚仪,治睢阳渠,遣使以太牢祀桥玄。[一]进军官渡。
注[一]褒赏令载公祀文曰:“故太尉桥公,诞敷明德,泛爱博容。国念明训,士思令谟。灵幽体翳,邈哉晞矣!吾以幼年,逮升堂室,特以顽鄙之姿,为大君子所纳。增荣益观,皆由奖助,犹仲尼称不如颜渊,李生之厚叹贾复。士死知己,怀此无忘。又承从容约誓之言:‘殂逝之后,路有经由,不以斗酒只鸡过相沃酹,车过三步,腹痛勿怪!’虽临时戏笑之言,非至亲之笃好,胡肯为此辞乎?匪谓灵忿,能诒己疾,怀旧惟顾,念之凄怆。奉命东征,屯次乡里,北望贵土,乃心陵墓。裁致薄奠,公其尚飨!”
\end{yuanwen}

\begin{yuanwen}
绍自军破后,发病欧\footnote{通“呕”,吐的意思。}血,夏五月死。小子尚代,谭自号车骑将军,屯黎阳。

秋九月,公征之,连战。谭、尚数败退,固守。
\end{yuanwen}

\begin{yuanwen}
八年春三月,攻其郭,乃出战,击,大破之,谭、尚夜遁。

夏四月,进军邺。

五月还许,留贾信屯黎阳。
\end{yuanwen}

\begin{yuanwen}
己酉,令曰:“司马法‘将军死绥’,[一]故赵括之母,乞不坐括。是古之将者,军破于外,而家受罪于内也。自命将征行,但赏功而不罚罪,非国典也。其令诸将出征,败军者抵罪,失利者免官爵。”[二]
注[一]魏书曰:绥,却也。有前一尺,无却一寸。
注[二]魏书载庚申令曰:“议者或以军吏虽有功能,德行不足堪任郡国之选,所谓‘可与适道,未可与权’。管仲曰:‘使贤者食于能则上尊,□士食于功则卒轻于死,二者设于国则天下治。’未闻无能之人,不□之士,并受禄赏,而可以立功兴国者也。故明君不官无功之臣,不赏不战之士;治平尚德行,有事赏功能。论者之言,一似管窥虎欤!”
秋七月,令曰:“丧乱已来,十有五年,后生者不见仁义礼让之风,吾甚伤之。其令郡国各修文学,县满五百户置校官,选其乡之俊造而教学之,庶几先王之道不废,而有以益于天下。”
\end{yuanwen}

\begin{yuanwen}
八月,公征刘表,军西平。公之去邺而南也,谭、尚争冀州,谭为尚所败,走保平原。尚攻之急,谭遣辛毗\footnote{pí}乞降请救。诸将皆疑,荀攸劝公许之,\footnote{魏书曰:公云:“我攻吕布,表不为寇,官渡之役,不救袁绍,此自守之贼也,宜为后图。谭、尚狡猾,当乘其乱。纵谭挟诈,不终束手,使我破尚,偏收其地,利自多矣。”乃许之。}公乃引军还。
\end{yuanwen}

\begin{yuanwen}
冬十月,到黎阳,为子整与谭结婚。\footnote{臣松之案:绍死至此,过周五月耳。谭虽出后其伯,不为绍服三年,而于再儙之内以行吉礼,悖矣。魏武或以权宜与之约言;今云结婚,未必便以此年成礼。}尚闻公北,乃释平原还邺。东平吕旷、吕翔叛尚,屯阳平,率其众降,封为列侯。\footnote{魏书曰:谭之围解,阴以将军印绶假旷。旷受印送之,公曰:“我固知谭之有小计也。欲使我攻尚,得以其闲略民聚众,尚之破,可得自强以乘我弊也。尚破我盛,何弊之乘乎?”}
\end{yuanwen}

\begin{yuanwen}
九年春正月,济河,遏淇水入白沟以通粮道。

二月,尚复攻谭,留苏由、审配守邺。公进军到洹\footnote{huán}水,由降。既至,攻邺,为土山、地道。武安长尹楷屯毛城,通上党粮道。

夏四月,留曹洪攻邺,公自将击楷,破之而还。尚将沮鹄\footnote{沮音菹jū,河朔闲今犹有此姓。鹄,沮授子也。}守邯郸,又击拔之。易阳令韩范、涉长梁岐举县降,赐爵关内侯。

五月,毁土山、地道,作围堑\footnote{qiàn},决漳水灌城;城中饿死者过半。

秋七月,尚还救邺,诸将皆以为“此归师,人自为战,不如避之”。公曰:“尚从大道来,当避之;若循西山来者,此成禽耳。”尚果循西山来,临滏水为营。\footnote{曹瞒传曰:遣候者数部前后参之,皆曰“定从西道,已在邯郸”。公大喜,会诸将曰:“孤已得冀州,诸君知之乎?”皆曰:“不知。”公曰:“诸君方见不久也。”}夜遣兵犯围,公逆击破走之,遂围其营。未合,尚惧,遣故豫州刺史阴夔\footnote{ku\'i}及陈琳乞降,公不许,为围益急。尚夜遁,保祁山,追击之。其将马延、张顗\footnote{y\v{i}}等临陈降,众大溃,尚走中山。尽获其辎重,得尚印绶节钺,使尚降人示其家,城中崩沮。

八月,审配兄子荣夜开所守城东门内兵。配逆战,败,生禽配,斩之,邺定。公临祀绍墓,哭之流涕;慰劳绍妻,还其家人宝物,赐杂缯\footnote{zēng
,古代对丝织品的统称。}絮,廪\footnote{lǐn
,由官府供给的米粮。}食之。\footnote{孙盛云:昔者先王之为诛赏也,将以惩恶劝善,永彰鉴戒。绍因世艰危,遂怀逆谋,上议神器,下干国纪。荐社污宅,古之制也,而乃尽哀于逆臣之頉,加恩于饕餮之室,为政之道,于斯踬矣。夫匿怨友人,前哲所耻,税骖旧馆,义无虚涕,苟道乖好绝,何哭之有!昔汉高失之于项氏,魏武遵谬于此举,岂非百虑之一失也。}
\end{yuanwen}

\begin{yuanwen}
初,绍与公共起兵,绍问公曰:“若事不辑,则方面何所可据?”公曰:“足下意以为何如?”
绍曰:“吾南据河,北阻燕、代,兼戎狄之众,南向以争天下,庶可以济乎?”公曰:“吾任天下之智力,以道御之,无所不可。”[一]
注[一]傅子曰:太祖又云:“汤、武之王,岂同土哉?若以险固为资,则不能应机而变化也。”
九月,令曰:“河北罹袁氏之难,其令无出今年租赋!”重豪强兼并之法,百姓喜悦。[一]天子以公领冀州牧,公让还兖州。
注[一]魏书载公令曰:“有国有家者,不患寡而患不均,不患贫而患不安。袁氏之治也,使豪强擅恣,亲戚兼并;下民贫弱,代出租赋,衒鬻家财,不足应命;审配宗族,至乃藏匿罪人,为逋逃主。欲望百姓亲附,甲兵强盛,岂可得邪!其收田租亩四升,户出绢二匹、绵二斤而已,他不得擅兴发。郡国守相明检察之,无令强民有所隐藏,而弱民兼赋也。”
\end{yuanwen}

\begin{yuanwen}
公之围邺也,谭略取甘陵、安平、勃海、河间。尚败,还中山。谭攻之,尚奔故安,遂并其众。公遗谭书,责以负约,与之绝婚,女还,然后进军。谭惧,拔平原,走保南皮。

十二月,公入平原,略定诸县。
\end{yuanwen}

\begin{yuanwen}
十年春正月,攻谭,破之,斩谭,诛其妻子,冀州平。\footnote{魏书曰:公攻谭,旦及日中不决;公乃自执桴鼓,士卒咸奋,应时破陷。}下令曰:“其与袁氏同恶者,与之更始。”令民不得复私仇,禁厚葬,皆一之于法。是月,袁熙大将焦触、张南等叛攻熙、尚,熙、尚奔三郡乌丸。触等举其县降,封为列侯。初讨谭时,民亡椎冰,\footnote{臣松之以为讨谭时,川渠水冻,使民椎冰以通船,民惮役而亡。}令不得降。

顷之,亡民有诣门首者,公谓曰:“听汝则违令,杀汝则诛首,归深自藏,无为吏所获。”

民垂泣而去;后竟捕得。
\end{yuanwen}

\begin{yuanwen}
夏四月,黑山贼张燕率其众十余万降,封为列侯。故安赵犊、霍奴等杀幽州刺史、涿郡太守。
三郡乌丸攻鲜于辅于犷平。[一]秋八月,公征之,斩犊等,乃渡潞河救犷平,乌丸奔走出塞。
注[一]续汉书郡国志曰:犷平,县名,属渔阳郡。
九月,令曰:“阿党比周,先圣所疾也。闻冀州俗,父子异部,更相毁誉。昔直不疑无兄,世人谓之盗嫂;第五伯鱼三娶孤女,谓之挝妇翁;王凤擅权,谷永比之申伯,王商忠议,张匡谓之左道:此皆以白为黑,欺天罔君者也。吾欲整齐风俗,四者不除,吾以为羞。”冬十月,公还邺。
\end{yuanwen}

\begin{yuanwen}
初,袁绍以甥高幹领并州牧,公之拔邺,幹降,遂以为刺史。幹闻公讨乌丸,乃以州叛,执上党太守,举兵守壶关口。遣乐进、李典击之,幹还守壶关城。

十一年春正月,公征幹。幹闻之,乃留其别将守城,走入匈奴,求救于单于,单于不受。公围壶关三月,拔之。幹遂走荆州,上洛都尉王琰捕斩之。
\end{yuanwen}

\begin{yuanwen}
秋八月,公东征海贼管承,至淳于,遣乐进、李典击破之,承走入海岛。割东海之襄贲、郯、戚以益琅邪,省昌虑郡。[一]
注[一]魏书载十月乙亥令曰:“夫治世御众,建立辅弼,诫在面从,诗称‘听用我谋,庶无大悔’,斯实君臣恳恳之求也。吾充重任,每惧失中,频年已来,不闻嘉谋,岂吾开延不勤之咎邪?自今以后,诸掾属治中、别驾,常以月旦各言其失,吾将览焉。”
三郡乌丸承天下乱,破幽州,略有汉民合十余万户。袁绍皆立其酋豪为单于,以家人子为己女,妻焉。辽西单于蹋顿尤强,为绍所厚,故尚兄弟归之,数入塞为害。公将征之,凿渠,自呼扨入泒水,*泒音孤。*名平虏渠;又从泃河口*泃音句。*凿入潞河,名泉州渠,以通海。
十二月春二月,公自淳于还邺。丁酋,令曰:“吾起义兵诛暴乱,于今十九年,所征必克,岂吾功哉?乃贤士大夫之力也。天下虽未悉定,吾当要与贤士大夫共定之;而专飨其劳,吾何以安焉!其促定功行封。”于是大封功臣二十余人,皆为列侯,其余各以次受封,及复死事之孤,轻重各有差。[一]
注[一]魏书载公令曰:“昔赵奢、窦婴之为将也,受赐千金,一朝散之,故能济成大功,永世流声。吾读其文,未尝不慕其为人也。与诸将士大夫共从戎事,幸赖贤人不爱其谋,髃士不遗其力,是夷险平乱,而吾得窃大赏,户邑三万。追思窦婴散金之义,今分所受租与诸将掾属及故戍于陈、蔡者,庶以畴答众劳,不擅大惠也。宜差死事之孤,以租谷及之。若年殷用足,租奉毕入,将大与众人悉共飨之。”
\end{yuanwen}

\begin{yuanwen}
将北征三郡乌丸,诸将皆曰:“袁尚,亡虏耳,夷狄贪而无亲,岂能为尚用?今深入征之,刘备必说刘表以袭许。万一为变,事不可悔。”惟郭嘉策表必不能任备,劝公行。
\end{yuanwen}

\begin{yuanwen}
夏五用,至无终。秋七月,大水,傍海道不通,田畴请为乡导,公从之。引军出卢龙塞,塞外道绝不通,乃堑山堙谷五百余里,经白檀,历平冈,涉鲜卑庭,东指柳城。未至二百里,虏乃知之。
\end{yuanwen}

\begin{yuanwen}
尚、熙与蹋顿、辽西单于楼班、右北平单于能臣抵之等将数万骑逆军。

八月,登白狼山,卒与虏遇,众甚盛。公车重在后,被甲者少,左右皆惧。公登高,望虏陈不整,乃纵兵击之,使张辽为先锋,虏众大崩,斩蹋顿及名王已下,胡、汉降者二十余万口。辽东单于速仆丸及辽西、北平诸豪,弃其种人,与尚、熙奔辽东,众尚有数千骑。
\end{yuanwen}

\begin{yuanwen}
初,辽东太守公孙康恃\footnote{shì,倚赖。}远不服。及公破乌丸,或说公遂征之,尚兄弟可禽也。公曰:“吾方使康斩送尚、熙首,不烦兵矣。”

九月,公引兵自柳城还,\footnote{曹瞒传曰:时寒且旱,二百里无复水,军又乏食,杀马数千匹以为粮,凿地入三十余丈乃得水。既还,科问前谏者,众莫知其故,人人皆惧。公皆厚赏之,曰:“孤前行,乘危以徼幸,虽得之,天所佐也,故不可以为常。诸君之谏,万安之计,是以相赏,后勿难言之。”}康即斩尚、熙及速仆丸等,传其首。

诸将或问:“公还而康斩送尚、熙,何也?”

公曰:“彼素畏尚等,吾急之则并力,缓之则自相图,其势然也。”
\end{yuanwen}

\begin{yuanwen}
十一月至易水,代郡乌丸行单于普富卢、上郡乌丸行单于那楼将其名王来贺。
\end{yuanwen}

\begin{yuanwen}
十三年春正月,公还邺,作玄武池以肄\footnote{yì,肄,以四反。三苍曰:“肄,习也。”}舟师。汉罢三公官,置丞相、御史大夫。夏六月,以公为丞相。\footnote{献帝起居注曰:使太常徐璆即授印绶。御史大夫不领中丞,置长史一人。先贤行状曰:璆字*(孟平)**[孟玉]*,广陵人。少履清爽,立朝正色。历任城、汝南、东海三郡,所在化行。被征当还,为袁术所劫。术僭号,欲授以上公之位,璆终不为屈。术死后,璆得术玺,致之汉朝,拜韂尉太常;公为丞相,以位让璆焉。}
\end{yuanwen}

\begin{yuanwen}
秋七月,公南征刘表。

八月,表卒,其子琮代,屯襄阳,刘备屯樊。

九月,公到新野,琮遂降,备走夏口。公进军江陵,下令荆州吏民,与之更始。乃论荆州服从之功,侯者十五人,以刘表大将文聘为江夏太守,使统本兵,引用荆州名士韩嵩、邓义等。\footnote{韂恒四体书势序曰:上谷王次仲善隶书,始为楷法。至灵帝好书,世多能者。而师宜官为最,甚矜其能,每书,辄削焚其札。梁鹄乃益为版而饮之酒,候其醉而窃其札,鹄卒以攻书至选部尚书。于是公欲为洛阳令,鹄以为北部尉。鹄后依刘表。及荆州平,公募求鹄,鹄惧,自缚诣门,署军假司马,使在秘书,以*(勤)**[勒]*书自效。公尝悬着帐中,及以钉壁玩之,谓胜宜官。鹄字孟黄,安定人。魏宫殿题署,皆鹄书也。皇甫谧逸士传曰:汝南王鉨,字子文,少为范滂、许章所识,与南阳岑晊善。公之为布衣,特爱鉨;鉨亦称公有治世之具。及袁绍与弟术丧母,归葬汝南,鉨与公会之,会者三万人。公于外密语鉨曰:“天下将乱,为乱魁者必此二人也。欲济天下,为百姓请命,不先诛此二子,乱今作矣。”鉨曰:	“如卿之言,济天下者,舍卿复谁?”相对而笑。鉨为人外静而内明,不应州郡三府之命。公车征,不到,避地居武陵,归鉨者一百余家。帝之都许,复征为尚书,又不就。刘表见绍强,阴与绍通,鉨谓表曰:“曹公,天下之雄也,必能兴霸道,继桓、文之功者也。今乃释近而就远,如有一朝之急,遥望漠北之救,不亦难乎!”表不从。鉨年六十四,以寿终于武陵,公闻而哀伤。及平荆州,自临江迎丧,改葬于江陵,表为先贤也。}益州牧刘璋始受征役,遣兵给军。

十二月,孙权为备攻合肥。公自江陵征备,至巴丘,遣张憙救合肥。权闻憙至,乃走。公至赤壁,与备战,不利。于是大疫,吏士多死者,乃引军还。备遂有荆州、江南诸郡。\footnote{山阳公载记曰:公船舰为备所烧,引军从华容道步归,遇泥泞,道不通,天又大风,悉使羸兵负草填之,骑乃得过。羸兵为人马所蹈藉,陷泥中,死者甚众。军既得出,公大喜,诸将问之,公曰:“刘备,吾俦也。但得计少晚;向使早放火,吾徒无类矣。”备寻亦放火而无所及。孙盛异同评曰:按吴志,刘备先破公军,然后权攻合肥,而此记云权先攻合肥,后有赤壁之事。二者不同,吴志为是。}
\end{yuanwen}

\begin{yuanwen}
十四年春三月,军至谯,作轻舟,治水军。秋七月,自涡入淮,出肥水,军合肥。辛未,令曰:“自顷已来,军数征行,或遇疫气,吏士死亡不归,家室怨旷,百姓流离,而仁者岂乐之哉?不得已也。其令死者家无基业不能自存者,县官勿绝廪,长吏存恤抚循,以称吾意。”
置扬州郡县长吏,开芍陂屯田。十二月,军还谯。
十五年春,下令曰:“自古受命及中兴之君,曷尝不得贤人君子与之共治天下者乎!及其得贤也,曾不出闾巷,岂幸相遇哉?上之人不求之耳。今天下尚未定,此特求贤之急时也。‘孟公绰为赵、魏老则优,不可以为滕、薛大夫’。若必廉士而后可用,则齐桓其何以霸世!今天下得无有被褐怀玉而钓于渭滨者乎?又得无盗嫂受金而未遇无知者乎?二三子其佐我明扬仄陋,唯才是举,吾得而用之。”冬,作铜雀台。[一]
注[一]魏武故事载公十二月己亥令曰:“孤始举孝廉,年少,自以本非岩穴知名之士,恐为海内人之所见凡愚,欲为一郡守,好作政教,以建立名誉,使世士明知之;故在济南,始除残去秽,平心选举,违迕诸常侍。以为强豪所忿,恐致家祸,故以病还。去官之后,年纪尚少,顾视同岁中,年有五十,未名为老,内自图之,从此却去二十年,待天下清,乃与同岁中始举者等耳。故以四时归乡里,于谯东五十里筑精舍,欲秋夏读书,冬春射猎,求底下之地,欲以泥水自蔽,绝宾客往来之望,然不能得如意。后征为都尉,迁典军校尉,意遂更欲为国家讨贼立功,欲望封侯作征西将军,然后题墓道言‘汉故征西将军曹侯之墓’,此其志也。而遭值董卓之难,兴举义兵。是时合兵能多得耳,然常自损,不欲多之;所以然者,多兵意盛,与强敌争,倘更为祸始。故汴水之战数千,后还到扬州更募,亦复不过三千人,此其本志有限也。后领兖州,破降黄巾三十万众。又袁术僭号于九江,下皆称臣,名门曰建号门,衣被皆为天子之制,两妇预争为皇后。志计已定,人有劝术使遂即帝位,露布天下,答言‘曹公尚在,未可也’。后孤讨禽其四将,获其人众,遂使术穷亡解沮,发病而死。及至袁绍据河北,兵势强盛,孤自度势,实不敌之,但计投死为国,以义灭身,足垂于后。幸而破绍,枭其二子。又刘表自以为宗室,包藏奸心,乍前乍却,以观世事,据有当州,孤复定之,遂平天下。身为宰相,人臣之贵已极,意望已过矣。
今孤言此,若为自大,欲人言尽,故无讳耳。设使国家无有孤,不知当几人称帝,几人称王。
或者人见孤强盛,又性不信天命之事,恐私心相评,言有不逊之志,妄相忖度,每用耿耿。
齐桓、晋文所以垂称至今日者,以其兵势广大,犹能奉事周室也。论语云‘三分天下有其二,以服事殷,周之德可谓至德矣’,夫能以大事小也。昔乐毅走赵,赵王欲与之图燕,乐毅伏而垂泣,对曰:‘臣事昭王,犹事天王;臣若获戾,放在他国,没世然后已,不忍谋赵之徒隶,况燕后嗣乎!’胡亥之杀蒙恬也,恬曰:‘自吾先人及至子孙,积信于秦三世矣;今臣将兵三十余万,其势足以背叛,然自知必死而守义者,不敢辱先人之教以忘先王也。’孤每读此二人书,未尝不怆然流涕也。孤祖父以至孤身,皆当亲重之任,可谓见信者矣,以及*(子植)**[子桓]*兄弟,过于三世矣。孤非徒对诸君说此也,常以语妻妾,皆令深知此意。孤谓之言:‘顾我万年之后,汝曹皆当出嫁,欲令传道我心,使他人皆知之。’孤此言皆肝鬲之要也。所以勤勤恳恳□心腹者,见周公有金縢之书以自明,恐人不信之故。然欲孤便尔委捐所典兵众以还执事,归就武平侯国,实不可也。何者?诚恐己离兵为人所祸也。既为子孙计,又己败则国家倾危,是以不得慕虚名而处实祸,此所不得为也。前朝恩封三子为侯,固辞不受,今更欲受之,非欲复以为荣,欲以为外援,为万安计。孤闻介推之避晋封。申胥之逃楚赏,未尝不舍书而叹,有以自省也。奉国威灵,仗钺征伐,推弱以克强,处小而禽大,意之所图,动无违事,心之所虑,何向不济,遂荡平天下,不辱主命,可谓天助汉室,非人力也。然封兼四县,食户三万,何德堪之!江湖未静,不可让位;至于邑土,可得而辞。今上还阳夏、柘、苦三县户二万,但食武平万户,且以分损谤议,少减孤之责也。”
十六年春正月,[一]天子命公世子丕为五官中郎将,置官属,为丞相副。太原商曜等以大陵叛,遣夏侯渊、徐晃围破之。张鲁据汉中,三月,遣钟繇讨之。公使渊等出河东与繇会。
注[一]魏书曰:庚辰,天子报:减户五千,分所让三县万五千封三子,植为平原侯,据为范阳侯,豹为饶阳侯,食邑各五千户。
是时关中诸将疑繇欲自袭,马超遂与韩遂、杨秋、李堪、成宜等叛。遣曹仁讨之。超等屯潼关,公敕诸将:“关西兵精悍,坚壁勿与战。”秋七月,公西征,[一]与超等夹关而军。公急持之,而潜遣徐晃、朱灵等夜渡蒲阪津,据河西为营。公自潼关北渡,未济,超赴船急战。
校尉丁斐因放牛马以饵贼,贼乱取牛马,公乃得渡,[二]循河为甬道而南。贼退,拒渭口,公乃多设疑兵,潜以舟载兵入渭,为浮桥,夜,分兵结营于渭南。贼夜攻营,伏兵击破之。
超等屯渭南,遣信求割河以西请和,公不许。九月,进军渡渭。[三]超等数挑战,又不许;
固请割地,求送任子,公用贾诩计,伪许之。韩遂请与公相见,公与遂父同岁孝廉,又与遂同时侪辈,于是交马语移时,不及军事,但说京都旧故,拊手欢笑。既罢,超等问遂:“公何言?”
遂曰:“无所言也。”超等疑之。[四]他日,公又与遂书,多所点窜,如遂改定者;超等愈疑遂。公乃与克日会战,先以轻兵挑之,战良久,乃纵虎骑夹击,大破之,斩成宜、李堪等。
遂、超等走凉州,杨秋奔安定,关中平。诸将或问公曰:“初,贼守潼关,渭北道缺,不从河东击冯翊而反守潼关,引日而后北渡,何也?”公曰:“贼守潼关,若吾入河东,贼必引守诸津,则西河未可渡,吾故盛兵向潼关;贼悉众南守,西河之备虚,故二将得擅取西河;
然后引军北渡,贼不能与吾争西河者,以有二将之军也。连车树栅,为甬道而南,[五]既为不可胜,且以示弱。渡渭为坚垒,虏至不出,所以骄之也;故贼不为营垒而求割地。吾顺言许之,所以从其意,使自安而不为备,因畜士卒之力,一旦击之,所谓疾雷不及掩耳,兵之变化,固非一道也。”始,贼每一部到,公辄有喜色。贼破之后,诸将问其故。公答曰:“关中长远,若贼各依险阻,征之,不一二年不可定也。今皆来集,其众虽多,莫相归服,军无适主,一举可灭,为功差易,吾是以喜。”
注[一]魏书曰:议者多言“关西兵强,习长矛,非精选前锋,则不可以当也”。公谓诸将曰:
“战在我,非在贼也。贼虽习长矛,将使不得以刺,诸君但观之耳。”
注[二]曹瞒传曰:公将过河,前队适渡,超等奄至,公犹坐胡黙不起。张合等见事急,共引公入船。河水急,比渡,流四五里,超等骑追射之,矢下如雨。诸将见军败,不知公所在,皆惶惧,至见,乃悲喜,或流涕。公大笑曰:“今日几为小贼所困乎!”
注[三]曹瞒传曰:时公军每渡渭,辄为超骑所冲突,营不得立,地又多沙,不可筑垒。娄子伯说公曰:“今天寒,可起沙为城,以水灌之,可一夜而成。”公从之,乃多作缣囊以运水,夜渡兵作城,比明,城立,由是公军尽得渡渭。或疑于时九月,水未应冻。臣松之按魏书:
公军八月至潼关,闰月北渡河,则其年闰八月也,至此容可大寒邪!
注[四]魏书曰:公后日复与遂等会语,诸将曰:“公与虏交语,不宜轻脱,可为木行马以为防遏。”公然之。贼将见公,悉于马上拜,秦、胡观者,前后重沓,公笑谓贼曰:“汝欲观曹公邪?亦犹人也,非有四目两口,但多智耳!”胡前后大观。又列铁骑五千为十重陈,精光耀日,贼益震惧。
注[五]臣松之案:汉高祖二年,与楚战荥阳京、索之间,筑甬道属河以取敖仓粟。应劭曰:
“恐敌钞辎重,故筑垣墙如街巷也。”今魏武不筑垣墙,但连车树栅以扞两面。
冬十月,军自长安北征杨秋,围安定。秋降,复其爵位,使留抚其民人。[一]十二月,自安定还,留夏侯渊屯长安。
注[一]魏略曰:杨秋,黄初中迁讨寇将军,位特进,封临泾侯,以寿终。
十七年春正月,公还邺。天子命公赞拜不名,入朝不趋,剑履上殿,如萧何故事。马超余众梁兴等屯蓝田,使夏侯渊击平之。割河内之荡阴、朝歌、林虑,东郡之卫国、顿丘、东武阳、发干,钜鹿之廮陶、曲周、南和,广平之任城,赵之襄国、邯郸、易阳以益魏郡。
冬十月,公征孙权。
十八年春正月,进军濡须口,攻破权江西营,获权都督公孙阳,乃引军还。诏书并十四州,复为九州。夏四月,至邺。
五月丙申,天子使御史大夫郗虑持节策命公为魏公[一]曰: Us朕以不德,少遭愍凶,越在西土,迁于唐、韂。当此之时,若缀旒然,[二]宗庙乏祀,社稷无位;髃凶觊觎,分裂诸夏,率土之民,朕无获焉,即我高祖之命将坠于地。朕用夙兴假寐,震悼于厥心,曰“惟祖惟父,股肱先正,[三]其孰能恤朕躬”?乃诱天衷,诞育丞相,保乂我皇家,弘济于艰难,朕实赖之。今将授君典礼,其敬听朕命。 Us昔者董卓初兴国难,髃后释位以谋王室,[四]君则摄进,首启戎行,此君之忠于本朝也。后及黄巾反易天常,侵我三州,延及平民,君又翦之以宁东夏,此又君之功也。韩暹、杨奉专用威命,君则致讨,克黜其难,遂迁许都,造我京畿,设官兆祀,不失旧物,天地鬼神于是获乂,此又君之功也。袁术僭逆,肆于淮南,慑惮君灵,用丕显谋,蕲阳之役,桥蕤授首,棱威南迈,术以陨溃,此又君之功也。回戈东征,吕布就戮,乘辕将返,张杨殂毙,眭固伏罪,张绣稽服,此又君之功也。袁绍逆乱天常,谋危社稷,凭恃其众,称兵内侮,当此之时,王师寡弱,天下寒心,莫有固志,君执大节,精贯白日,奋其武怒,运其神策,致届官渡,大歼丑类,[五]俾我国家拯于危坠,此又君之功也。济师洪河,拓定四州,袁谭、高干,咸枭其首,海盗奔迸,黑山顺轨,此又君之功也。乌丸三种,崇乱二世,袁尚因之,逼据塞北,束马县车,一征而灭,此又君之功也。刘表背诞,不供贡职,王师首路,威风先逝,百城八郡,交臂屈膝,此又君之功也。马超、成宜,同恶相济,滨据河、潼,求逞所欲,殄之渭南,献馘万计,遂定边境,抚和戎狄,此又君之功也。鲜卑、丁零,重译而至,*(单于)**[箄于]*、白屋,请吏率职,此又君之功也。君有定天下之功,重之以明德,班□海内,宣美风俗,旁施勤教,恤慎刑狱,吏无苛政,民无怀慝;敦崇帝族,表继绝世,旧德前功,罔不咸秩;虽伊尹格于皇天,周公光于四海,方之蔑如也。
Us朕闻先王并建明德,胙之以土,分之以民,崇其宠章,备其礼物,所以藩韂王室,左右厥世也。
其在周成,管、蔡不静,惩难念功,乃使邵康公赐齐太公履,东至于海,西至于河,南至于穆陵,北至于无棣,五侯九伯,实得征之,世祚太师,以表东海;爰及襄王,亦有楚人不供王职,又命晋文登为侯伯,锡以二辂、虎贲、鈇钺、秬鬯、弓矢,大启南阳,世作盟主。故周室之不坏,繄二国是赖。今君称丕显德,明保朕躬,奉答天命,导扬弘烈,缓爰九域,莫不率俾,[六]功高于伊、周,而赏卑于齐、晋,朕甚恧焉。朕以眇眇之身,托于兆民之上,永思厥艰,若涉渊冰,非君攸济,朕无任焉。今以冀州之河东、河内、魏郡、赵国、中山、常山、钜鹿、安平、甘陵、平原凡十郡,封君为魏公。锡君玄土,苴以白茅;爰契尔龟,用建頉社。昔在周室,毕公、毛公入为卿佐,周、邵师保出为二伯,外内之任,君实宜之,其以丞相领冀州牧如故。又加君九锡,其敬听朕命。以君经纬礼律,为民轨仪,使安职业,无或迁志,是用锡君大辂、戎辂各一,玄牡二驷。君劝分务本,穑人昏作,[七]粟帛滞积,大业惟兴,是用锡君衮冕之服,赤舄副焉。君敦尚谦让,俾民兴行,少长有礼,上下咸和,是用锡君轩县之乐,六佾之舞。君翼宣风化,爰发四方,远人革面,华夏充实,是用锡君朱户以居。君研其明哲,思帝所难,官才任贤,髃善必举,是用锡君纳陛以登。君秉国之钧,正色处中,纤毫之恶,靡不抑退,是用锡君虎贲之士三百人。君纠虔天刑,章厥有罪,[八]犯关干纪,莫不诛殛,是用锡君鈇钺各一。君龙骧虎视,旁眺八维,掩讨逆节,折冲四海,是用锡君彤弓一,彤矢百,玈弓十,玈矢千。君以温恭为基,孝友为德,明允笃诚,感于朕思,是用锡君秬鬯一卣,珪瓒副焉。魏国置丞相已下髃卿百寮,皆如汉初诸侯王之制。往钦哉,敬服朕命!简恤尔众,时亮庶功,用终尔显德,对扬我高祖之休命![九]
注[一]续汉书曰:虑字鸿豫,山阳高平人。少受业于郑玄,建安初为侍中。虞溥江表传曰:
献帝尝特见虑及少府孔融,问融曰:“鸿豫何所优长?”融曰:“可与适道,未可与权。”虑举笏曰:
“融昔宰北海,政散民流,其权安在也!”遂与融互相长短,以至不睦。公以书和解之。虑从光禄勋迁为大夫。
注[二]公羊传曰:“君若赘旒然。”何休云:“赘犹缀也。旒,旗旒也。以旒譬者,言为下所执持东西也。”
注[三]文侯之命曰:“亦惟先正。”郑玄云:“先正,先臣。谓公卿大夫也。”
注[四]左氏传曰:“诸侯释位以闲王政。”服虔曰:“言诸侯释其私政而佐王室。”
注[五]诗曰:“致天之届,于牧之野。”郑玄云:“届,极也。”鸿范曰:“鲧则殛死。”
注[六]盘庚曰:“绥爰有众。”郑玄曰:“爰,于也,安隐于其众也。”君奭曰:“海隅出日,罔不率俾。”率,循也。俾,使也。四海之隅,日出所照,无不循度而可使也。
注[七]盘庚曰:“堕农自安,不昏作劳。”郑玄云:“昏,勉也。”
注[八]“纠虔天刑”语出国语,韦昭注曰:“纠,察也。虔,敬也。刑,法也。”
注[九]后汉尚书左丞潘勖之辞也。勖字符茂,陈留中牟人。魏书载公令曰:“夫受九锡,广开土宇,周公其人也。汉之异姓八王者,与高祖俱起布衣,□定王业,其功至大,吾何可比之?”前后三让。于是中军师*(王)*陆树亭侯荀攸、前军师东武亭侯钟繇、左军师凉茂、右军师毛玠、平虏将军华乡侯刘勋、建武将军清苑亭侯刘若、伏波将军高安侯夏侯惇、扬武将军都亭侯王忠、奋威将军乐乡侯刘展、建忠将军昌乡亭侯鲜于辅、奋武将军安国亭侯程昱、太中大夫都乡侯贾诩、军师祭酒千秋亭侯董昭、都亭侯薛洪、南乡亭侯董蒙、关内侯王粲、傅巽、祭酒王选、袁涣、王朗、张承、任藩、杜袭、中护军国明亭侯曹洪、中领军万岁亭侯韩浩、行骁骑将军安平亭侯曹仁、领护军将军王图、长史万潜、谢奂、袁霸等劝进曰:“自古三代,胙臣以土,受命中兴,封秩辅佐,皆所以褒功赏德,为国藩韂也。往者天下崩乱,髃凶豪起,颠越跋扈之险,不可忍言。明公奋身出命以徇其难,诛二袁篡盗之逆,灭黄巾贼乱之类,殄夷首逆,芟拨荒秽,沐浴霜露二十余年,书契已来,未有若此功者。昔周公承文、武之夡,受已成之业,高枕墨笔,拱揖髃后,商、奄之勤,不过二年,吕望因三分有二之形,据八百诸侯之势,暂把旄钺,一时指麾,然皆大启土宇,跨州兼国。周公八子,并为侯伯,白牡骍刚,郊祀天地,典策备物,拟则王室,荣章宠盛如此之弘也。逮至汉兴,佐命之臣,张耳、吴芮,其功至薄,亦连城开地,南面称孤。此皆明君达主行之于上,贤臣圣宰受之于下,三代令典,汉帝明制。今比劳则周、吕逸,计功则张、吴微,论制则齐、鲁重,言地则长沙多;然则魏国之封,九锡之荣,况于旧赏,犹怀玉而被褐也。且列侯诸将,幸攀龙骥,得窃微劳,佩紫怀黄,盖以百数,亦将因此传之万世,而明公独辞赏于上,将使其下怀不自安,上违圣朝欢心,下失冠带至望,忘辅弼之大业,信匹夫之细行,攸等所大惧也。”于是公敕外为章,但受魏郡。攸等复曰:“伏见魏国初封,圣朝发虑,稽谋髃寮,然后策命;而明公久违上指,不即大礼。今既虔奉诏命,副顺众望,又欲辞多当少,让九受一,是犹汉朝之赏不行,而攸等之请未许也。昔齐、鲁之封,奄有东海,疆域井赋,四百万家,基隆业广,易以立功,故能成翼戴之勋,立一匡之绩。今魏国虽有十郡之名,犹减于曲阜,计其户数,不能参半,以藩韂王室,立垣树屏,犹未足也。且圣上览亡秦无辅之祸,惩曩日震荡之艰,托建忠贤,废坠是为,愿明公恭承帝命,无或拒违。”公乃受命。魏略载公上书谢曰:“臣蒙先帝厚恩,致位郎署,受性疲怠,意望毕足,非敢希望高位,庶几显达。会董卓作乱,义当死难,故敢奋身出命,摧锋率众,遂值千载之运,奉役目下。当二袁炎沸侵侮之际,陛下与臣寒心同忧,顾瞻京师,进受猛敌,常恐君臣俱陷虎口,诚不自意能全首领。赖祖宗灵佑,丑类夷灭,得使微臣窃名其间。陛下加恩,授以上相,封爵宠禄,丰大弘厚,生平之愿,实不望也。口与心计,幸且待罪,保持列侯,遗付子孙,自托圣世,永无忧责。不意陛下乃发盛意,开国备锡,以贶愚臣,地比齐、鲁,礼同藩王,非臣无功所宜膺据。归情上闻,不蒙听许,严诏切至,诚使臣心俯仰逼迫。伏自惟省,列在大臣,命制王室,身非己有,岂敢自私,遂其愚意,亦将黜退,令就初服。今奉疆土,备数藩翰,非敢远期,虑有后世;至于父子相誓终身,灰躯尽命,报塞厚恩。天威在颜,悚惧受诏。”
秋七月,始建魏社稷宗庙。天子聘公三女为贵人,少者待年于国。[一]九月,作金虎台,凿渠引漳水入白沟以通河。冬十月,分魏郡为东西部,置都尉。十一月,初置尚书、侍中、六卿。[二]
注[一]献帝起居注曰:使使持节行太常大司农安阳亭侯王邑,赍璧、帛、玄纁、绢五万匹之邺纳聘,介者五人,皆以议郎行大夫事,副介一人。
注[二]魏氏春秋曰:以荀攸为尚书令,凉茂为仆射,毛玠、崔琰、常林、徐奕、何夔为尚书,王粲、杜袭、韂觊、和洽为侍中。
马超在汉阳,复因羌、胡为害,氐王千万叛应超,屯兴国。使夏侯渊讨之。

十九年春正月,始耕籍田。南安赵衢、汉阳尹奉等讨超,枭其妻子,超奔汉中。韩遂徙金城,入氐王千万部,率羌、胡万余骑与夏侯渊战,击,大破之,遂走西平。渊与诸将攻兴国,屠之。省安东、永阳郡。
安定太守□丘兴将之官,公戒之曰:“羌,胡欲与中国通,自当遣人来,慎勿遣人往。善人难得,必将教羌、胡妄有所请求,因欲以自利;不从便为失异俗意,从之则无益事。”兴至,遣校尉范陵至羌中,陵果教羌,使自请为属国都尉。公曰:“吾预知当尔,非圣也,但更事多耳。”[一]
注[一]献帝起居注曰:使行太常事大司农安阳亭侯王邑与宗正刘艾,皆持节,介者五人,赍束帛驷马,及给事黄门侍郎、掖庭丞、中常侍二人,迎二贵人于魏公国。二月癸亥,又于魏公宗庙授二贵人印绶。甲子,诣魏公宫延秋门,迎贵人升车。魏遣郎中令、少府、博士、御府乘黄厩令、丞相掾属侍送贵人。癸酉,二贵人至洧仓中,遣侍中丹将冗从虎贲前后骆驿往迎之。乙亥,二贵人入宫,御史大夫、中二千石将大夫、议郎会殿中,魏国二卿及侍中、中郎二人,与汉公卿并升殿宴。
三月,天子使魏公位在诸侯王上,改授金玺,赤绂、远游冠。[一]
注[一]献帝起居注曰:使左中郎将杨宣、亭侯裴茂持节、印授之。
\end{yuanwen}

\begin{yuanwen}
秋七月,公征孙权。
\end{yuanwen}

\begin{yuanwen}
[一]
注[一]九州春秋曰:参军傅干谏曰:“治天下之大具有二,文与武也;用武则先威,用文则先德,威德足以相济,而后王道备矣。往者天下大乱,上下失序,明公用武攘之,十平其九。
今未承王命者,吴与蜀也,吴有长江之险,蜀有崇山之阻,难以威服,易以德怀。愚以为可且按甲寝兵,息军养士,分土定封,论功行赏,若此则内外之心固,有功者劝,而天下知制矣。然后渐兴学校,以导其善性而长其义节。公神武震于四海,若修文以济之,则普天之下,无思不服矣。今举十万之众,顿之长江之滨,若贼负固深藏,则士马不能逞其能,奇变无所用其权,则大威有屈而敌心未能服矣。唯明公思虞舜舞干戚之义,全威养德,以道制胜。”公不从,军遂无功。
干字彦材,北地人,终于丞相仓曹属。有子曰玄。
初,陇西宋建自称河首平汉王,聚众枹罕,改元,置百官,三十余年。遣夏侯渊自兴国讨之。
冬十月,屠枹罕,斩建,凉州平。
公自合肥还。
十一月,汉皇后伏氏坐昔与父故屯骑校尉完书,云帝以董承被诛怨恨公,辞甚丑恶,发闻,后废黜死,兄弟皆伏法。[一]
注[一]曹瞒传曰:公遣华歆勒兵入宫收后,后闭户匿壁中。歆坏户发壁,牵后出。帝时与御史大夫郗虑坐,后被发徒跣过,执帝手曰:“不能复相活邪?”帝曰:“我亦不自知命在何时也。”帝谓虑曰:“郗公,天下宁有是邪!”遂将后杀之,完及宗族死者数百人。
十二月,公至孟津。天子命公置旄头,宫殿设钟虡。乙未,令曰:“夫有行之士未必能进取,进取之士未必能有行也。陈平岂笃行,苏秦岂守信邪?而陈平定汉业,苏秦济弱燕。由此言之,士有偏短,庸可废乎!有司明思此义,则士无遗滞,官无废业矣。”又曰:“夫刑,百姓之命也,而军中典狱者或非其人,而任以三军死生之事,吾甚惧之。其选明达法理者,使持典刑。”于是置理曹掾属。

二十年春正月,天子立公中女为皇后。省云中、定襄、五原、朔方郡,郡置一县领其民,合以为新兴郡。
\end{yuanwen}

\begin{yuanwen}
三月,公西征张鲁,至陈仓,将自武都入氐\footnote{dī,我国古代民族。殷、周至南北朝时分布在今陕西、甘肃、四川一带,东晋时曾建立前秦和后凉。};氐人塞道,先遣张郃\footnote{hé}、朱灵等攻破之。

夏四月,公自陈仓以出散关,至河池。氐王窦茂众万余人,恃险不服。

五月,公攻屠之。西平、金城诸将麴\footnote{qū}演、蒋石等共斩送韩遂首。\footnote{注[一]典略曰:遂字文约,始与同郡边章俱著名西州。章为督军从事。遂奉计诣京师,何进宿闻其名,特与相见,遂说进使诛诸阉人,进不从,乃求归。会凉州宋扬、北宫玉等反,举章、遂为主,章寻病卒,遂为扬等所劫,不得已,遂阻兵为乱,积三十二年,至是乃死,年七十余矣。刘艾灵帝纪曰:章,一名*(元)**[允]*。}

秋七月,公至阳平。张鲁使弟韂(卫)与将杨昂等据阳平关,横山筑城十余里,攻之不能拔,乃引军还。贼见大军退,其守备解散。公乃密遣解慓、高祚\footnote{zuò}等乘险夜袭,大破之,斩其将杨任,进攻韂,韂等夜遁,鲁溃奔巴中。公军入南郑,尽得鲁府库珍宝。\footnote{魏书曰:军自武都山行千里,升降险阻,军人劳苦;公于是大飨,莫不忘其劳。}巴、汉皆降。复汉宁郡为汉中;分汉中之安阳、西城为西城郡,置太守;分锡、上庸郡,置都尉。
\end{yuanwen}

\begin{yuanwen}
八月,孙权围合肥,张辽、李典击破之。
\end{yuanwen}

\begin{yuanwen}
九月,巴七姓夷王朴胡、賨邑侯杜濩举巴夷、賨民来附,[一]于是分巴郡,以胡为巴东太守,濩为巴西太守,皆封列侯。天子命公承制封拜诸侯守相。[二]
注[一]孙盛曰:朴音浮。濩音户。
注[二]孔衍汉魏春秋曰:天子以公典任于外,临事之赏,或宜速疾,乃命公得承制封拜诸侯守相,诏曰:“夫军之大事,在兹赏罚,劝善惩恶,宜不旋时,故司马法曰‘赏不逾日’者,欲民速鷪为善之利也。昔在中兴,邓禹入关,承制拜军祭酒李文为河东太守,来歙又承制拜高峻为通路将军,察其本传,皆非先请,明临事刻印也,斯则世祖神明,权达损益,盖所用速示威怀而着鸿勋也。其春秋之义,大夫出疆,有专命之事,苟所以利社稷安国家而已。况君秉任二伯,师尹九有,实征夷夏,军行藩甸之外,失得在于斯须之间,停赏俟诏以滞世务,固非朕之所图也。自今已后,临事所甄,当加宠号者,其便刻印章假授,咸使忠义得相銟励,勿有疑焉。”
冬十月,始置名号侯至五大夫,与旧列侯、关内侯凡六等,以赏军功。[一]
注[一]魏书曰:置名号侯爵十八级,关中侯爵十七级,皆金印紫绶;又置关内外侯十六级,铜印龟纽墨绶;五大夫十五级,铜印环纽,亦墨绶,皆不食租,与旧列侯关内侯凡六等。臣松之以为今之虚封盖自此始。
\end{yuanwen}

\begin{yuanwen}
十一月,鲁自巴中将其余众降。封鲁及五子皆为列侯。刘备袭刘璋,取益州,遂据巴中;遣张郃击之。
\end{yuanwen}

\begin{yuanwen}
十二月,公自南郑还,留夏侯渊屯汉中。\footnote{是行也,侍中王粲作五言诗以美其事曰:“从军有苦乐,但问所从谁。所从神且武,安得久劳师?相公征关右,赫怒振天威,一举灭獯虏,再举服羌夷,西收边地贼,忽若俯拾遗。陈赏越山岳,酒肉踰川坻,军中多饶饫,人马皆溢肥,徒行兼乘还,空出有余资。拓土三千里,往反速如飞,歌舞入邺城,所愿获无违。”}
\end{yuanwen}

\begin{yuanwen}
二十一年春二月,公还邺。[一]三月壬寅,公亲耕籍田。[二]夏五月,天子进公爵为魏王。
[三]代郡乌丸行单于普富卢与其侯王来朝。天子命王女为公主,食汤沐邑。秋七月,匈奴南单于呼厨泉将其名王来朝,待以客礼,遂留魏,使右贤王去卑监其国。八月,以大理钟繇为相国。[四]
注[一]魏书曰:辛未,有司以太牢告至,策勋于庙,甲午始春祠,令曰:“议者以为祠庙上殿当解履。吾受锡命,带剑不解履上殿。今有事于庙而解履,是尊先公而替王命,敬父祖而简君主,故吾不敢解履上殿也。又临祭就洗,以手拟水而不盥。夫盥以洁为敬,未闻拟*(向)**[而]*不盥之礼,且‘祭神如神在’,故吾亲受水而盥也。又降神礼讫,下阶就幕而立,须奏乐毕竟,似若不*(愆)**[衎]*烈祖,迟祭*(不)*速讫也,故吾坐俟乐阕送神乃起也。
受胙纳*(神)**[袖]*,以授侍中,此为敬恭不终实也,古者亲执祭事,故吾亲纳于*(神)**[袖]*,终抱而归也。仲尼曰‘虽违众,吾从下’,诚哉斯言也。”
注[二]魏书曰:有司奏:“四时讲武于农隙。汉承秦制,三时不讲,唯十月都试车马,幸长水南门,会五营士为八陈进退,名曰乘之。今金革未偃,士民素习,自今已后,可无四时讲武,但以立秋择吉日大朝车骑,号曰治兵,上合礼名,下承汉制。”奏可。
注[三]献帝传载诏曰:“自古帝王,虽号称相变,爵等不同,至乎褒崇元勋,建立功德,光启氏姓,延于子孙,庶姓之与亲,岂有殊焉。昔我圣祖受命,□业肇基,造我区夏,鉴古今之制,通爵等之差,尽封山川以立藩屏,使异姓亲戚,并列土地,据国而王,所以保乂天命,安固万嗣。历世承平,臣主无事。世祖中兴而时有难易,是以旷年数百,无异姓诸侯王之位。
朕以不德,继序弘业,遭率土分崩,髃凶纵毒,自西徂东,辛苦卑约。当此之际,唯恐溺入于难,以羞先帝之圣德。赖皇天之灵,俾君秉义奋身,震迅神武,捍朕于艰难,获保宗庙,华夏遗民,含气之伦,莫不蒙焉。君勤过稷、禹,忠侔伊、周,而掩之以谦让,守之以弥恭,是以往者初开魏国,锡君土宇,惧君之违命,虑君之固辞,故且怀志屈意,封君为上公,欲以钦顺高义,须俟勋绩。韩遂、宋建,南结巴、蜀,髃逆合从,图危社稷,君复命将,龙骧虎奋,枭其元首,屠其窟栖。暨至西征,阳平之役,亲擐甲冑,深入险阻,芟夷蝥贼,殄其凶丑,荡定西陲,悬旌万里,声教远振,宁我区夏。盖唐、虞之盛,三后树功,文、武之兴,旦、奭作辅,二祖成业,英豪佐命;夫以圣哲之君,事为己任,犹锡土班瑞以报功臣,岂有如朕寡德,仗君以济,而赏典不丰,将何以答神只慰万方哉?今进君爵为魏王,使使持节行御史大夫、宗正刘艾奉策玺玄土之社,苴以白茅,金虎符第一至第五,竹使符第一至十。君其正王位,以丞相领冀州牧如故。其上魏公玺绶符册。敬服朕命,简恤尔众,克绥庶绩,以扬我祖宗之休命。”魏王上书三辞,诏三报不许。又手诏曰:“大圣以功德为高美,以忠和为典训,故□业垂名,使百世可希,行道制义,使力行可效,是以勋烈无穷,休光茂着。稷、契载元首之聪明,周、邵因文、武之智用,虽经营庶官,仰叹俯思,其对岂有若君者哉?朕惟古人之功,美之如彼,思君忠勤之绩,茂之如此,是以每将镂符析瑞,陈礼命册,寤寐慨然,自忘守文之不德焉。今君重违朕命,固辞恳切,非所以称朕心而训后世也。其抑志撙节,勿复固辞。”四体书势序曰:梁鹄以公为北部尉。
曹瞒传曰:为尚书右丞司马建公所举。及公为王,召建公到邺,与欢饮,谓建公曰:“孤今日可复作尉否?”建公曰:“昔举大王时,适可作尉耳。”王大笑。建公名防,司马宣王之父。臣松之案司马彪序传,建公不为右丞,疑此不然,而王隐晋书云赵王篡位,欲尊祖为帝,博士马平议称京兆府君昔举魏武帝为北部尉,贼不犯界,如此则为有征。
注[四]魏书曰:始置奉常宗正官。
冬十月,治兵,[一]遂征孙权,十一月至谯。
注[一]魏书曰:王亲执金鼓以令进退。
二十二年春正月,王军居巢,二月,进军屯江西郝溪。权在濡须口筑城拒守,遂逼攻之,权退走。三月,王引军还,留夏侯惇、曹仁、张辽等屯居巢。
夏四月,天子命王设天子旌旗,出入称警跸。五月,作泮宫。六月,以军师华歆为御史大夫。
[一]冬十月,天子命王冕十有二旒,乘金根车,驾六马,设五时副车,以五官中郎将丕为魏太子。
注[一]魏书曰:初置韂尉官。秋八月,令曰:“昔伊挚、傅说出于贱人,管仲,桓公贼也,皆用之以兴。萧何、曹参,县吏也,韩、陈平负污辱之名,有见笑之耻,卒能成就王业,声着千载。吴起贪将,杀妻自信,散金求官,母死不归,然在魏,奏人不敢东向,在楚则三晋不敢南谋。今天下得无有至德之人放在民间,及果勇不顾,临敌力战;若文俗之吏,高才异质,或堪为将守;负污辱之名,见笑之行,或不仁不孝而有治国用兵之术:其各举所知,勿有所遗。”
刘备遣张飞、马超、吴兰等屯下辩;遣曹洪拒之。
二十三年春正月,汉太医令吉本与少府耿纪、司直韦晃等反,攻许,烧丞相长史王必营,[一]必与颍川典农中郎将严匡讨斩之。[二]
注[一]魏武故事载令曰:“领长史王必,是吾披荆棘时吏也。忠能勤事,心如铁石,国之良吏也。蹉跌久未辟之,舍骐骥而弗乘,焉遑遑而更求哉?故教辟之,已署所宜,便以领长史统事如故。”
注[二]三辅决录注曰:时有京兆金祎字德祎,自以世为汉臣,自日磾讨莽何罗,忠诚显著,名节累叶。鷪汉祚将移,谓可季兴,乃喟然发愤,遂与耿纪、韦晃、吉本、本子邈、邈弟穆等结谋。纪字季行,少有美名,为丞相掾,王甚敬异之,迁侍中,守少府。邈字文然,穆字思然,以祎慷慨有日磾之风,又与王必善,因以闲之,若杀必,欲挟天子以攻魏,南援刘备。
时关羽强盛,而王在邺,留必典兵督许中事。文然等率杂人及家僮千余人夜烧门攻必,祎遣人为内应,射必中肩。必不知攻者为谁,以素与祎善,走投祎,夜唤德祎,祎家不知是必,谓为文然等,错应曰:“王长史已死乎?卿曹事立矣!”必乃更他路奔。一曰:必欲投祎,其帐下督谓必曰:“今日事竟知谁门而投入乎?”扶必奔南城。会天明,必犹在,文然等众散,故败。后十余日,必竟以创死。献帝春秋曰:收纪、晃等,将斩之,纪呼魏王名曰:“恨吾不自生意,竟为髃儿所误耳!”晃顿首搏颊,以至于死。山阳公载记曰:王闻王必死,盛怒,召汉百官诣邺,令救火者左,不救火者右。众人以为救火者必无罪,皆附左;王以为“不救火者非助乱,救火乃实贼也”。皆杀之。
曹洪破吴兰,斩其将任夔等。三月,张飞、马超走汉中,阴平氐强端斩吴兰,传其首。
夏四月,代郡、上谷乌丸无臣氐等叛,遣鄢陵侯彰讨破之。[一]
注[一]魏书载王令曰:“去冬天降疫疠,民有凋伤,军兴于外,垦田损少,吾甚忧之。其令吏民男女:女年七十已上无夫子,若年十二已下无父母兄弟,及目无所见,手不能作,足不能行,而无妻子父兄产业者,廪食终身。幼者至十二止,贫穷不能自赡者,随口给贷。老耄须待养者,年九十已上,复不事,家一人。”
六月,令曰:“古之葬者,必居瘠薄之地。其规西门豹祠西原上为寿陵,因高为基,不封不树。周礼頉人掌公墓之地,凡诸侯居左右以前,卿大夫居后,汉制亦谓之陪陵。其公卿大臣列将有功者,宜陪寿陵,其广为兆域,使足兼容。”
秋七月,治兵,遂西征刘备,九月,至长安。
冬十月,宛守将侯音等反,执南阳太守,劫略吏民,保宛。初,曹仁讨关羽,屯樊城,是月使仁围宛。
二十四年春正月,仁屠宛,斩音。[一]
注[一]曹瞒传曰:是时南阳闲苦繇役,音于是执太守*(东里箧)**[东里衮]*,与吏民共反,与关羽连和。南阳功曹宗子卿往说音曰:“足下顺民心,举大事,远近莫不望风;然执郡将,逆而无益,何不遣之。吾与子共暞力,比曹公军来,关羽兵亦至矣。”音从之,即释遣太守。
子卿因夜踰城亡出,遂与太守收余民围音,会曹仁军至,共灭之。
夏侯渊与刘备战于阳平,为备所杀。三月,王自长安出斜谷,军遮要以临汉中,遂至阳平。
备因险拒守。[一]
注[一]九州春秋曰:时王欲还,出令曰“鸡肋”,官属不知所谓。主簿杨修便自严装,人惊问修:“何以知之?”修曰:“夫鸡肋,弃之如可惜,食之无所得,以比汉中,知王欲还也。”
夏五月,引军还长安。
秋七月,以夫人卞氏为王后。遣于禁助曹仁击关羽。八月,汉水溢,灌禁军,军没,羽获禁,遂围仁。使徐晃救之。
九月,相国钟繇坐西曹掾魏讽反免。[一]
注[一]世语曰:讽字子京,沛人,有惑众才,倾动邺都,钟繇由是辟焉。大军未反,讽潜结徒党,又与长乐韂尉陈祎谋袭邺。未及期,祎惧,告之太子,诛讽,坐死者数十人。王昶家诫曰“济阴魏讽”,而此云沛人,未详。
冬十月,军还洛阳。[一]孙权遣使上书,以讨关羽自效。王自洛阳南征羽,未至,晃攻羽,破之,羽走,仁围解。王军摩陂。[二]
注[一]曹瞒传曰:王更修治北部尉廨,令过于旧。
注[二]魏略曰:孙权上书称臣,称说天命。王以权书示外曰:“是儿欲踞吾着炉火上邪!”
侍中陈髃、尚书桓阶奏曰:“汉自安帝已来,政去公室,国统数绝,至于今者,唯有名号,尺土一民,皆非汉有,期运久已尽,历数久已终,非适今日也。是以桓、灵之间,诸明图纬者,皆言‘汉行气尽,黄家当兴’。
殿下应期,十分天下而有其九,以服事汉,髃生注望,,遐迩怨叹,是故孙权在远称臣,此天人之应,异气齐声。臣愚以为虞、夏不以谦辞,殷、周不吝诛放,畏天知命,无所与让也。”
魏氏春秋曰:夏侯惇谓王曰:“天下咸知汉祚已尽,异代方起。自古已来,能除民害为百姓所归者,即民主也。今殿下即戎三十余年,功德着于黎庶,为天下所依归,应天顺民,复何疑哉!”王曰:“‘施于有政,是亦为政’。若天命在吾,吾为周文王矣。”曹瞒传及世语并云桓阶劝王正位,夏侯惇以为宜先灭蜀,蜀亡则吴服,二方既定,然后遵舜、禹之轨,王从之。及至王薨,惇追恨前言,发病卒。孙盛评曰:夏侯惇耻为汉官,求受魏印,桓阶方惇,有义直之节;考其传记,世语为妄矣。
\end{yuanwen}

\begin{yuanwen}
二十五年春正月,至洛阳。权击斩羽,传其首。

庚子,王崩于洛阳,年六十六。\footnote{世语曰:太祖自汉中至洛阳,起建始殿,伐濯龙祠而树血出。曹瞒传曰:王使工苏越徙美梨,掘之,根伤尽出血。越白状,王躬自视而恶之,以为不祥,还遂寝疾。}遗令曰:“天下尚未安定,未得遵古也。葬毕,皆除服。其将兵屯戍者,皆不得离屯部。有司各率乃职。敛以时服,无藏金玉珍宝。”谥曰武王。

二月丁卯,葬高陵。\footnote{魏书曰:太祖自统御海内,芟夷髃丑,其行军用师,大较依孙、吴之法,而因事设奇,谲敌制胜,变化如神。自作兵书十万余言,诸将征伐,皆以新书从事。临事又手为节度,从令者克捷,违教者负败。与虏对陈,意思安闲,如不欲战,然及至决机乘胜,气势盈溢,故每战必克,军无幸胜。知人善察,难眩以伪,拔于禁、乐进于行陈之间,取张辽、徐晃于亡虏之内,皆佐命立功,列为名将;其余拔出细微,登为牧守者,不可胜数。是以□造大业,文武并施,御军三十余年,手不舍书,昼则讲武策,夜则思经传,登高必赋,及造新诗,被之管弦,皆成乐章。才力绝人,手射飞鸟,躬禽猛兽,尝于南皮一日射雉获六十三头。及造作宫室,缮治器械,无不为之法则,皆尽其意。雅性节俭,不好华丽,后宫衣不锦绣,侍御履不二采,帷帐屏风,坏则补纳,茵蓐取温,无有缘饰。攻城拔邑,得美丽之物,则悉以赐有功,勋劳宜赏,不吝千金,无功望施,分毫不与,四方献御,与髃下共之。常以送终之制,袭称之数,繁而无益,俗又过之,故预自制终亡衣服,四箧而已。傅子曰:太祖愍嫁取之奢僭,公女适人,皆以皁帐,从婢不过十人。张华博物志曰:汉世,安平崔瑗、瑗子寔、弘农张芝、芝弟昶并善草书,而太祖亚之。桓谭、蔡邕善音乐,冯翊山子道、王九真、郭凯等善围澙,太祖皆与埒能。又好养性法,亦解方药,招引方术之士,庐江左慈、谯郡华佗、甘陵甘始、阳城蜔俭无不毕至,又习啖野葛至一尺,亦得少多饮鸩酒。傅子曰:汉末王公,多委王服,以幅巾为雅,是以袁绍、*(崔豹)**[崔钧]*之徒,虽为将帅,皆着缣巾。魏太祖以天下凶荒,资财乏匮,拟古皮弁,裁缣帛以为帢,合于简易随时之义,以色别其贵贱,于今施行,可谓军容,非国容也。曹瞒传曰:太祖为人佻易无威重,好音乐,倡优在侧,常以日达夕。被服轻绡,身自佩小鞶囊,以盛手巾细物,时或冠帢帽以见宾客。每与人谈论,戏弄言诵,尽无所隐,及欢悦大笑,至以头没杯案中,肴膳皆沾污巾帻,其轻易如此。然持法峻刻,诸将有计画胜出己者,随以法诛之,及故人旧怨,亦皆无余。其所刑杀,辄对之垂涕嗟痛之,终无所活。初,袁忠为沛相,尝欲以法治太祖,沛国桓邵亦轻之,及在兖州,陈留边让言议颇侵太祖,太祖杀让,族其家,忠、邵俱避难交州,太祖遣使就太守士燮尽族之。桓邵得出首,拜谢于庭中,太祖谓曰:“跪可解死邪!”遂杀之。常出军,行经麦中,令“士卒无败麦,犯者死”。骑士皆下马,付麦以相持,于是太祖马腾入麦中,□主簿议罪;主簿对以春秋之义,罚不加于尊。太祖曰:“制法而自犯之,何以帅下?然孤为军帅,不可自杀,请自刑。”因援剑割发以置地。又有幸姬常从昼寝,枕之卧,告之曰:“须臾觉我。”姬见太祖卧安,未即寤,及自觉,棒杀之。常讨贼,廪谷不足,私谓主者曰:“如何?”主者曰:“可以小斛以足之。”太祖曰:“善。”后军中言太祖欺众,太祖谓主者曰:“特当借君死以厌众,不然事不解。”乃斩之,取首题徇曰:“行小斛,盗官谷,斩之军门。”其酷虐变诈,皆此类也。}
\end{yuanwen}

\begin{yuanwen}
评曰:汉末,天下大乱,雄豪并起,而袁绍虎摉四州,强盛莫敌。太祖运筹演谋,鞭挞宇内,閴申、商之法术,该韩、白之奇策,官方授材,各因其器,矫情任算,不念旧恶,终能总御皇机,克成洪业者,惟其明略最优也。抑可谓非常之人,超世之杰矣。 
\end{yuanwen}

\part{魏书二}

\chapter{文帝纪第二}
\begin{yuanwen}
文皇帝讳丕,字子桓,武帝太子也。中平四年冬,生于谯。[一]建安十六年,为五官中郎将、副丞相。二十二年,立为魏太子。[二]太祖崩,嗣位为丞相、魏王。[三]尊王后曰王太后。
改建安二十五年为延康元年。
注[一]魏书曰:帝生时,有云气青色而圜如车盖当其上,终日,望气者以为至贵之证,非人臣之气。年八岁,能属文。有逸才,遂博贯古今经传诸子百家之书。善骑射,好击剑。举茂才,不行。献帝起居注曰:建安十*(五)**[三]*年,为司徒赵温所辟。太祖表“温辟臣子弟,选举故不以实”。使侍中守光禄勋郗虑持节奉策免温官。
注[二]魏略曰:太祖不时立太子,太子自疑。是时有高元吕者,善相人,乃呼问之,对曰:
“其贵乃不可言。”问:“寿几何?”元吕曰:“其寿,至四十当有小苦,过是无忧也。”后无几而立为王太子,至年四十而薨。
注[三]袁宏汉纪载汉帝诏曰:“魏太子丕:昔皇天授乃显考以翼我皇家,遂攘除髃凶,拓定九州,弘功茂绩,光于宇宙,朕用垂拱负扆二十有余载。天不慭遗一老,永保余一人,早世潜神,哀悼伤切。丕奕世宣明,宜秉文武,绍熙前绪。
今使使持节御史大夫华歆奉策诏授丕丞相印绶、魏王玺绂,领冀州牧。方今外有遗虏,遐夷未宾,旗鼓犹在边境,干戈不得韬刃,斯乃播扬洪烈,立功垂名之秋也。岂得修谅闇之礼,究曾、闵之志哉?其敬服朕命,抑弭忧怀,旁祗厥绪,时亮庶功,以称朕意。于戏,可不勉与!”
元年二月[一]王戌,以大中大夫贾诩为太尉,御史大夫华歆为相国,大理王朗为御史大夫。
置散骑常侍、侍郎各四人,其宦人为官者不得过诸署令;为金策着令,藏之石室。
注[一]魏书载庚戌令曰:“关津所以通商旅,池苑所以御灾荒,设禁重税,非所以便民;其除池□之禁,轻关津之税,皆复什一。”辛亥,赐诸侯王将相已下将粟万斛,帛千匹,金银各有差等。遣使者循行郡国,有违理掊克暴虐者,举其罪。
初,汉熹平五年,黄龙见谯,光禄大夫桥玄问太史令单扬:“此何祥也?”扬曰:“其国后当有王者兴,不及五十年,亦当复见。天事恒象,此其应也。”内黄殷登默而记之。至四十五年,登尚在。三月,黄龙见谯,登闻之曰:“单扬之言,其验兹乎!”[一]
注[一]魏书曰:王召见登,谓之曰:“昔成风闻楚丘之繇而敬事季友,邓晨信少公之言而自纳光武。登以笃老,服膺占术,记识天道,岂有是乎!”赐登谷三百斛,遣归家。
已卯,以前将军夏侯惇为大将军。濊貊、扶余单于、焉耆、于阗王皆各遣使奉献。[一]
注[一]魏书曰:丙戌,令史官奏修重、黎、羲、和之职,钦若昊天,历象日月星辰以奉天时。
臣松之案:魏书有是言而不闻其职也。丁亥令曰:“故尚书仆射毛玠、奉常王修、凉茂、郎中令袁涣、少府谢奂、万潜、中尉徐奕、国渊等,皆忠直在朝,履蹈仁义,并早即世,而子孙陵迟,恻然愍之,其皆拜子男为郎中。”
夏四月丁巳,饶安县言白雉见。[一]庚午,大将军夏侯惇薨。[二]
注[一]魏书曰:赐饶安田租,勃海郡百户牛酒,大酺三日;太常以太牢祠宗庙。
注[二]魏书曰:王素服幸邺东城门发哀。孙盛曰:在礼,天子哭同姓于宗庙门之外。哭于城门,失其所也。
五月戊寅,天子命王追尊皇祖太尉曰太王,夫人丁氏曰太王后,封王子叡为武德侯。[一]是月,冯翊山贼郑甘、王照率众降,皆封列侯。[二]
注[一]魏略曰:以侍中郑称为武德侯傅,令曰:“龙渊、太阿出昆吾之金,和氏之璧由井里之田;砻之以砥砺,错之以他山,故能致连城之价,为命世之宝。学亦人之砥砺也。称笃学大儒,勉以经学辅侯,宜旦夕入侍,曜明其志。”
注[二]魏书曰:初,郑甘、王照及卢水胡率其属来降,王得降书以示朝曰:“前欲有令吾讨鲜卑者,吾不从而降;又有欲使吾及今秋讨卢水胡者,吾不听,今又降。昔魏武侯一谋而当,有自得之色,见讥李悝。吾今说此,非自是也,徒以为坐而降之,其功大于动兵革也。”
酒泉黄华、张掖张进等各执太守以叛。金城太守苏则讨进,斩之。华降。[一]
注[一]华后为兖州刺史,见王凌传。
六月辛亥,治兵于东郊,[一]庚午,遂南征。[二]
注[一]魏书曰:公卿相仪,王御华盖,视金鼓之节。
注[二]魏略曰:王将出征,度支中郎将新平霍性上疏谏曰:“臣闻文王与纣之事,是时天下括囊无咎,凡百君子,莫肯用讯。今大王体则乾坤,广开四聪,使贤愚各建所规。伏惟先王功无与比,而今能言之类,不称为德。故圣人曰‘得百姓之欢心’。兵书曰‘战,危事也’是以六国力战,强秦承弊,豳王不争,周道用兴。愚谓大王且当委重本朝而守其雌,抗威虎卧,功业可成。而今□基,便复起兵,兵者凶器,必有凶扰,扰则思乱,乱出不意。臣谓此危,危于累卵。昔夏启隐神三年,易有‘不远而复’,论有‘不惮改’。诚愿大王揆古察今,深谋远虑,与三事大夫算其长短。臣沐浴先王之遇,又初改政,复受重任,虽知言触龙鳞,阿谀近福,窃感所诵,危而不持。”奏通,帝怒,遣刺奸就考,竟杀之。既而悔之,追原不及。
秋七月庚辰,令曰:“轩辕有明台之议,放勋有衢室之问,皆所以广询于下也。[一]百官有司,其务以职尽规谏,将率陈军法,朝士明制度,牧守申政事,缙绅考六艺,吾将兼览焉。”
注[一]管子曰:黄帝立明台之议者,上观于兵也;尧有衢室之问者,下听于民也;舜有告善之旌,而主不蔽也;禹立建鼓于朝,而备诉讼也;汤有总街之廷,以观民非也;武王有灵台之囿,而贤者进也:此古圣帝明王所以有而勿失,得而勿忘也。
孙权遣使奉献。蜀将孟达率众降。武都氐王杨仆率种人内附,居汉阳郡。[一]
注[一]魏略载王自手笔令曰:“*(吾)**[日]*前遣使宣国威灵,而达即来。吾惟春秋褒仪父,即封拜达,使还领新城太守。近复有扶老携幼首向王化者。吾闻夙沙之民自缚其君以归神农,豳国之众襁负其子而入丰、镐,斯岂驱略迫胁之所致哉?乃风化动其情而仁义感其衷,欢心内发使之然也。以此而推,西南将万里无外,权、备将与谁守死乎?”
甲午,军次于谯,大飨六军及谯父老百姓于邑东。[一]八月,石邑县言凤皇集。
注[一]魏书曰:设伎乐百戏,令曰:“先王皆乐其所生,礼不忘其本。谯,霸王之邦,真人本出,其复谯租税二年。”三老吏民上寿,日夕而罢。丙申,亲祠谯陵。孙盛曰:昔者先王之以孝治天下也,内节天性,外施四海,存尽其敬,亡极其哀,思慕谅闇,寄政頉宰,故曰“三年之丧,自天子达于庶人”;夫然,故在三之义惇,臣子之恩笃,雍熙之化隆,经国之道固,圣人之所以通天地,厚人伦,显至教,敦风俗,斯万世不易之典,百王服膺之制也。
是故丧礼素冠,郐人着庶见之讥,宰予降儙,仲尼发不仁之叹,子颓忘戚,君子以为乐祸,鲁侯易服,春秋知其不终,岂不以坠至痛之诚心,丧哀乐之大节者哉?故虽三季之末,七雄之弊,犹未有废缞斩于旬朔之间,释麻杖于反哭之日者也。逮于汉文,变易古制,人道之纪,一旦而废,缞素夺于至尊,四海散其遏密,义感阙于髃后,大化坠于君亲;虽心存贬约,虑在经纶,至于树德垂声,崇化变俗,固以道薄于当年,风颓于百代矣。且武王载主而牧野不陈,晋襄墨缞而三帅为俘,应务济功,服其焉害?魏王既追汉制,替其大礼,处莫重之哀而设飨宴之乐,居贻厥之始而坠王化之基,及至受禅,显纳二女,忘其至恤以诬先圣之典,天心丧矣,将何以终!是以知王龄之不遐,卜世之期促也。
冬十*(一)*月癸卯,令曰:“诸将征伐,士卒死亡者或未收敛,吾甚哀之;其告郡国给槥椟殡敛,*槥音韂。*送致其家,官为设祭。”[一]丙午,行至曲蠡。
注[一]汉书高祖八月令曰:“士卒从军死,为槥。”应劭曰:“槥,小棺也,今谓之椟。”应璩百一诗曰:“槥车在道路,征夫不得休。”陆机大墓赋曰:“观细木而闷迟,鷪洪椟而念槥。”
汉帝以众望在魏,乃召髃公卿士,[一]告祠高庙。使兼御史大夫张音持节奉玺绶禅位,册曰:
“咨尔魏王:昔者帝尧禅位于虞舜,舜亦以命禹,天命不于常,惟归有德。汉道陵迟,世失其序,降及朕躬,大乱兹昏,髃凶肆逆,宇内颠覆。赖武王神武,拯兹难于四方,惟清区夏,以保绥我宗庙,岂予一人获乂,俾九服实受其赐。今王钦承前绪,光于乃德,恢文武之大业,昭尔考之弘烈。皇灵降瑞,人神告征,诞惟亮采,师锡朕命,佥曰尔度克协于虞舜,用率我唐典,敬逊尔位。于戏!天之历数在尔躬,允执其中,天禄永终;君其祗顺大礼,飨兹万国,以肃承天命。”[二]乃为坛于繁阳。庚午,王升坛即阼,百官陪位。事讫,降坛,视燎成礼而反。改延康为黄初,大赦。[三]
注[一]袁宏汉纪载汉帝诏曰:“朕在位三十有二载,遭天下荡覆,幸赖祖宗之灵,危而复存。
然仰瞻天文,俯察民心,炎精之数既终,行运在乎曹氏。是以前王既树神武之绩,今王又光曜明德以应其期,是历数昭明,信可知矣。夫大道之行,天下为公,选贤与能,故唐尧不私于厥子,而名播于无穷。朕羡而慕焉,今其追踵尧典,禅位于魏王。”
注[二]献帝传载禅代众事曰:左中郎将李伏表魏王曰:“昔先王初建魏国,在境外者闻之未审,皆以为拜王。武都李庶、姜合羁旅汉中,谓臣曰:‘必为魏公,未便王也。定天下者,魏公子桓,神之所命,当合符谶,以应天人之位。’臣以合辞语镇南将军张鲁,鲁亦问合知书所出?合曰:‘孔子玉版也。天子历数,虽百世可知。’是后月余,有亡人来,写得册文,卒如合辞。合长于内学,关右知名。鲁虽有怀国之心,沉溺异道变化,不果寤合之言。后密与臣议策质,国人不协,或欲西通,鲁即怒曰:‘宁为魏公奴,不为刘备上客也。’言发恻痛,诚有由然。合先迎王师,往岁病亡于邺。自臣在朝,每为所亲宣说此意,时未有宜,弗敢显言。殿下即位初年,祯祥众瑞,日月而至,有命自天,昭然着见。然圣德洞达,符表豫明,实乾坤挺庆,万国作孚。臣每庆贺,欲言合验;事君尽礼,人以为谄。况臣名行秽贱,入朝日浅,言为罪尤,自抑而已。今洪泽被四表,灵恩格天地,海内翕习,殊方归服,兆应并集,以扬休命,始终允臧。臣不胜喜舞,谨具表通。”王令曰:“以示外。薄德之人,何能致此,未敢当也;斯诚先王至德通于神明,固非人力也。” Ui魏王侍中刘廙、辛毗、刘晔、尚书令桓阶、尚书陈矫、陈髃、给事黄门侍郎王毖、董遇等言:“臣伏读左中郎将李伏上事,考图纬之言,以效神明之应,稽之古代,未有不然者也。故尧称历数在躬,璇玑以明天道;周武未战而赤乌衔书;汉祖未兆而神母告符;孝宣仄微,字成木叶;光武布衣,名已勒谶。是天之所命以着圣哲,非有言语之声,芬芳之臭,可得而知也,徒县象以示人,微物以效意耳。
自汉德之衰,渐染数世,桓、灵之末,皇极不建,暨于大乱,二十余年。天之不泯,诞生明圣,以济其难,是以符谶先着,以彰至德。殿下践阼未儙,而灵象变于上,髃瑞应于下,四方不羁之民,归心向义,唯惧在后,虽典籍所传,未若今之盛也。臣妾远近,莫不凫藻。”
王令曰:“儣牛之驳似虎,莠之幼似禾,事有似是而非者,今日是已。鷪斯言事,良重吾不德。”于是尚书仆射宣告官寮,咸使闻知。 Ui辛亥,太史丞许芝条魏代汉见谶纬于魏王曰:
“易传曰:‘圣人受命而王,黄龙以戊己日见。’七月四日戊寅,黄龙见,此帝王受命之符瑞最着明者也。又曰:‘初六,履霜,阴始凝也。’又有积虫大穴天子之宫,厥咎然,今蝗虫见,应之也。又曰:‘圣人以德亲比天下,仁恩洽普,厥应麒麟以戊己日至,厥应圣人受命。’又曰:‘圣人清净行中正,贤人福至民从命,厥应麒麟来。’春秋汉含孳曰:‘汉以魏,魏以征。’春秋玉版谶曰:‘代赤者魏公子。’春秋佐助期曰:‘汉以许昌失天下。’故白马令李云上事曰:‘许昌气见于当涂高,当涂高者当昌于许。’当涂高者,魏也;象魏者,两观阙是也;当道而高大者魏。魏当代汉。今魏基昌于许,汉征绝于许,乃今效见,如李云之言,许昌相应也。佐助期又曰:‘汉以蒙孙亡。’说者以蒙孙汉二十四帝,童蒙愚昏,以弱亡。或以杂文为蒙其孙当失天下,以为汉帝非正嗣,少时为董侯,名不正,蒙乱之荒惑,其子孙以弱亡。孝经中黄谶曰:‘日载东,绝火光。不横一,圣聪明。四百之外,易姓而王。天下归功,致太平,居八甲;共礼乐,正万民,嘉乐家和杂。’此魏王之姓讳,着见图谶。易运期谶曰:‘言居东,西有午,两日并光日居下。其为主,反为辅。五八四十,黄气受,真人出。’言午,许字。
两日,昌字。汉当以许亡,魏当以许昌。今际会之期在许,是其大效也。易运期又曰:‘鬼在山,禾女连,王天下。’臣闻帝王者,五行之精;易姓之符,代兴之会,以七百二十年为一轨。有德者过之,至于八百,无德者不及,至四百载。是以周家八百六十七年,夏家四百数十年,汉行夏正,迄今四百二十六岁。又高祖受命,数虽起乙未,然其兆征始于获麟。获麟以来七百余年,天之历数将以尽终。帝王之兴,不常一姓。太微中,黄帝坐常明,而赤帝坐常不见,以为黄家兴而赤家衰,凶亡之渐。自是以来四十余年,又荧惑失色不明十有余年。
建安十年,彗星先除紫微,二十三年,复扫太微。新天子气见东南以来,二十三年,白虹贯日,月蚀荧惑,比年己亥、壬子、丙午日蚀,皆水灭火之象也。殿下即位,初践阼,德配天地,行合神明,恩泽盈溢,广被四表,格于上下。是以黄龙数见,凤皇仍翔,麒麟皆臻,白虎效仁,前后献见于郊甸;甘露醴泉,奇兽神物,众瑞并出。斯皆帝王受命易姓之符也。昔黄帝受命,风后受河图;舜、禹有天下,凤皇翔,洛出书;汤之王,白鸟为符;文王为西伯,赤鸟衔丹书;武王伐殷,白鱼升舟;高祖始起,白蛇为征。巨迹瑞应,皆为圣人兴。观汉前后之大灾,今兹之符瑞,察图谶之期运,揆河洛之所甄,未若今大魏之最美也。夫得岁星者,道始兴。昔武王伐殷,岁在鹑火,有周之分野也。高祖入秦,五星聚东井,有汉之分野也。今兹岁星在大梁,有魏之分野也。而天之瑞应,并集来臻,四方归附,襁负而至,兆民欣戴,咸乐嘉庆。春秋大传曰:‘周公何以不之鲁?盖以为虽有继体守文之君,不害圣人受命而王。’周公反政,尸子以为孔子非之,以为周公不圣,不为兆民也。京房作易传曰:‘凡为王者,恶者去之,弱者夺之。易姓改代,天命应常,人谋鬼谋,百姓与能。’伏惟殿下体尧舜之盛明,膺七百之禅代,当汤武之期运,值天命之移受,河洛所表,图谶所载,昭然明白,天下学士所共见也。臣职在史官,考符察征,图谶效见,际会之期,谨以上闻。”王令曰:“昔周文三分天下有其二,以服事殷,仲尼叹其至德;公旦履天子之籍,听天下之断,终然复子明辟,书美其人。吾虽德不及二圣,敢忘高山景行之义哉?若夫唐尧、舜、禹之迹,皆以圣质茂德处之,故能上和灵只,下宁万姓,流称今日。今吾德至薄也,人至鄙也,遭遇际会,幸承先王余业,恩未被四海,泽未及天下,虽倾仓竭府以振魏国百姓,犹寒者未尽暖,饥者未尽饱。夙夜忧惧,弗敢遑宁,庶欲保全发齿,长守今日,以没于地,以全魏国,下见先王,以塞负荷之责。望狭志局,守此而已;虽屡蒙祥瑞,当之战惶,五色无主。若芝之言,岂所闻乎?心栗手悼,书不成字,辞不宣心。吾闲作诗曰:‘丧乱悠悠过纪,白骨纵横万里,哀哀下民靡恃,吾将佐时整理,复子明辟致仕。’庶欲守此辞以自终,卒不虚言也。宜宣示远近,使昭赤心。”于是侍中辛毗、刘晔、散骑常侍傅巽、韂臻、尚书令桓阶、尚书陈矫、陈髃、给事中博士骑都尉苏林、董巴等奏曰:“伏见太史丞许芝上魏国受命之符;令书恳切,允执谦让,虽舜、禹、汤、文,义无以过。然古先哲王所以受天命而不辞者,诚急遵皇天之意,副兆民之望,弗得已也。且易曰:‘观乎天文以察时变,观乎人文以化成天下。’又曰:
‘天垂象,见吉凶,圣人则之;河出图,洛出书,圣人效之。’以为天文因人而变,至于河洛之书,着于洪范,则殷、周效而用之矣。斯言,诚帝王之明符,天道之大要也。是以由德应录者代兴于前,失道数尽者迭废于后,传讥苌弘欲支天之所坏,而说蔡墨‘雷乘干’之说,明神器之存亡,非人力所能建也。今汉室衰替,帝纲堕坠,天子之诏,歇灭无闻,皇天将舍旧而命新,百姓既去汉而为魏,昭然着明,是可知也。先王拨乱平世,将建洪基;至于殿下,以至德当历数之运,即位以来,天应人事,粲然大备,神灵图籍,兼仍往古,休征嘉兆,跨越前代;是芝所取中黄、运期姓纬之谶,斯文乃着于前世,与汉并见。由是言之,天命久矣,非殿下所得而拒之也。神明之意,候望禋享,兆民颙颙,咸注嘉愿,惟殿下览图籍之明文,急天下之公义,辄宣令外内,布告州郡,使知符命着明,而殿下谦虚之意。”令曰:“下四方以明孤款心,是也。至于览余辞,岂余所谓哉?宁所堪哉?诸卿指论,未若孤自料之审也。夫虚谈谬称,鄙薄所弗当也。且闻比来东征,经郡县,历屯田,百姓面有饥色,衣或短褐不完,罪皆在孤;
是以上惭众瑞,下愧士民。由斯言之,德尚未堪偏王,何言帝者也!宜止息此议,无重吾不德,使逝之后,不愧后之君子。” Ui癸丑,宣告髃寮。督军御史中丞司马懿、侍御史郑浑、羊秘、鲍勋、武周等言:“令如左。伏读太史丞许芝上符命事,臣等闻有唐世衰,天命在虞,虞氏世衰,天命在夏;然则天地之灵,历数之运,去就之符,惟德所在。故孔子曰:‘凤鸟不至,河不出图,吾已矣夫!’今汉室衰,自安、和、冲、质以来,国统屡绝,桓、灵荒淫,禄去公室,此乃天命去就,非一朝一夕,其所由来久矣。殿下践阼,至德广被,格于上下,天人感应,符瑞并臻,考之旧史,未有若今日之盛。夫大人者,先天而天弗违,后天而奉天时,天时已至而犹谦让者,舜、禹所不为也,故生民蒙救济之惠,髃类受育长之施。今八方颙颙,大小注望,皇天乃眷,神人同谋,十分而九以委质,义过周文,所谓过恭也。臣妾上下,伏所不安。”令曰:“世之所不足者道义也,所有余者苟妄也;常人之性,贱所不足,贵所有余,故曰‘不患无位,患所以立’。孤虽寡德,庶自免于常人之贵。夫‘石可破而不可夺坚,丹可磨而不可夺赤’。丹石微物,尚保斯质,况吾托士人之末列,曾受教于君子哉?且于陵仲子以仁为富,柏成子高以义为贵,鲍焦感子贡之言,弃其蔬而槁死,薪者讥季札失辞,皆委重而弗视。吾独何人?昔周武,大圣也,使叔旦盟胶鬲于四内,使召公约微子于共头,故伯夷、叔齐相与笑之曰:‘昔神农氏之有天下,不以人之坏自成,不以人之卑自高。’以为周之伐殷以暴也。
吾德非周武而义惭夷、齐,庶欲远苟妄之失道,立丹石之不夺,迈于陵之所富,蹈柏成之所贵,执鲍焦之贞至,遵薪者之清节。故曰:‘三军可夺帅,匹夫不可夺志。’吾之斯志,岂可夺哉?” Ui乙卯,册诏魏王禅代天下曰:“惟延康元年十月乙卯,皇帝曰,咨尔魏王:
夫命运否泰,依德升降,三代卜年,着于春秋,是以天命不于常,帝王不一姓,由来尚矣。
汉道陵迟,为日已久,安、顺已降,世失其序,冲、质短祚,三世无嗣,皇纲肇亏,帝典颓沮。暨于朕躬,天降之灾,遭无妄厄运之会,值炎精幽昧之期。变兴辇毂,祸由阉宦。董卓乘衅,恶甚浇、豷,劫迁省御,*(太仆)**[火扑]*宫庙,遂使九州幅裂,强敌虎争,华夏鼎沸,蝮蛇塞路。当斯之时,尺土非复汉有,一夫岂复朕民?幸赖武王德膺符运,奋扬神武,芟夷凶暴,清定区夏,保乂皇家。今王缵承前绪,至德光昭,御衡不迷,布德优远,声教被四海,仁风扇鬼区,是以四方效珍,人神响应,天之历数实在尔躬。昔虞舜有大功二十,而放勋禅以天下;大禹有疏导之绩,而重华禅以帝位。汉承尧运,有传圣之义,加顺灵只,绍天明命,厘降二女,以嫔于魏。使使持节行御史大夫事太常音,奉皇帝玺绶,王其永君万国,敬御天威,允执其中,天禄永终,敬之哉?”于是尚书令桓阶等奏曰:“汉氏以天子位禅之陛下,陛下以圣明之德,历数之序,承汉之禅,允当天心。夫天命弗可得辞,兆民之望弗可得违,臣请会列侯诸将、髃臣陪隶,发玺书,顺天命,具礼仪列奏。”令曰:“当议孤终不当承之意而已。犹猎,还方有令。”
Ui尚书令等又奏曰:“昔尧、舜禅于文祖,至汉氏,以师征受命,畏天之威,不敢怠遑,便即位行在所之地。今当受禅代之命,宜会百寮髃司,六军之士,皆在行位,使咸鷪天命。
营中促狭,可于平敞之处设坛场,奉答休命。臣辄与侍中常侍会议礼仪,太史官择吉日讫,复奏。”令曰:“吾殊不敢当之,外亦何豫事也!” Ui侍中刘廙、常侍韂臻等奏议曰:“汉氏遵唐尧公天下之议,陛下以圣德膺历数之运,天人同欢,靡不得所,宜顺灵符,速践皇阼。
问太史丞许芝,今月十七日己未直成,可受禅命,辄治坛场之处,所当施行别奏。”令曰;
“属出见外,便设坛场,斯何谓乎?今当辞让不受诏也。但于帐前发玺书,威仪如常,且天寒,罢作坛士使归。”既发玺书,王令曰:“当奉还玺绶为让章。吾岂奉此诏承此贶邪?昔尧让天下于许由、子州支甫,舜亦让于善卷、石户之农、北人无择,或退而耕颍之阳,或辞以幽忧之疾,或远入山林,莫知其处,或携子入海,终身不反,或以为辱,自投深渊;且颜烛惧太朴之不完,守知足之明分,王子搜乐丹穴之潜处,被熏而不出,柳下惠不以三公之贵易其介,曾参不以晋、楚之富易其仁:斯九士者,咸高节而尚义,轻富而贱贵,故书名千载,于今称焉。求仁得仁,仁岂在远?孤独何为不如哉?义有蹈东海而逝,不奉汉朝之诏也。亟为上章还玺绶,宣之天下,使咸闻焉。”己未,宣告髃僚,下魏,又下天下。 Ui辅国将军清苑侯刘若等百二十人上书曰:“伏读令书,深执克让,圣意恳恻,至诚外昭,臣等有所不安。何者?石户、北人,匹夫狂狷,行不合义,事不经见者,是以史迁谓之不然,诚非圣明所当希慕。且有虞不逆放勋之禅,夏禹亦无辞位之语,故传曰:‘舜陟帝位,若固有之。’斯诚圣人知天命不可逆,历数弗可辞也。伏惟陛下应干符运,至德发闻,升昭于天,是三灵降瑞,人神以和,休征杂沓,万国响应,虽欲勿用,将焉避之?而固执谦虚,违天逆众,慕匹夫之微分,背上圣之所蹈,违经谶之明文,信百氏之穿凿,非所以奉答天命,光慰众望也。
臣等昧死以请,辄整顿坛场,至吉日受命,如前奏,分别写令宣下。”王令曰:“昔柏成子高辞夏禹而匿野,颜阖辞鲁币而远迹,夫以王者之重,诸侯之贵,而二子忽之,何则?其节高也。故烈士徇荣名,义夫高贞介,虽蔬食瓢饮,乐在其中。是以仲尼师王骀,而子产嘉申徒。今诸卿皆孤股肱腹心,足以明孤,而今咸若斯,则诸卿游于形骸之内,而孤求为形骸之外,其不相知,未足多怪。亟为上章还玺绶,勿复纷纷也。”
Ui辅国将军等一百二十人又奏曰:“臣闻符命不虚见,众心不可违,故孔子曰:‘周公其为不圣乎?以天下让。是天地日月轻去万物也。’是以舜向天下,不拜而受命。今火德气尽,炎上数终,帝迁明德,祚隆大魏。符瑞昭鴋,受命既固,光天之下,神人同应,虽有虞仪凤,成周跃鱼,方今之事,未足以喻。而陛下违天命以饰小行,逆人心以守私志,上忤皇穹眷命之旨,中忘圣人达节之数,下孤人臣翘首之望,非所以扬圣道之高衢,乘无穷之懿勋也。臣等闻事君有献可替否之道,奉上有逆鳞固争之义,臣等敢以死请。”令曰:“夫古圣王之治也,至德合乾坤,惠泽均造化,礼教优乎昆虫,仁恩洽乎草木,日月所照,戴天履地含气有生之类,靡不被服清风,沐浴玄德;
是以金革不起,苛慝不作,风雨应节,祯祥触类而见。今百姓寒者未暖,饥者未饱,□者未室,寡者未嫁;权、备尚存,未可舞以干戚,方将整以齐斧;戎役未息于外,士民未安于内,耳未闻康哉之歌,目未鷪击壤之戏,婴儿未可托于高巢,余粮未可以宿于田亩:人事未备,至于此也。夜未曜景星,治未通真人,河未出龙马,山未出象车,蓂荚未植阶庭,萐莆未生庖厨,王母未献白环,渠搜未见珍裘:灵瑞未效,又如彼也。昔东户季子、容成、大庭、轩辕、赫胥之君,咸得以此就功勒名。今诸卿独不可少假孤精心竭虑,以和天人,以格至理,使彼众事备,髃瑞效,然后安乃议此乎,何遽相愧相迫之如是也?速为让章,上还玺绶,无重吾不德也。” Ui侍中刘廙等奏曰:“伏惟陛下以大圣之纯懿,当天命之历数,观天象则符瑞着明,考图纬则文义焕炳,察人事则四海齐心,稽前代则异世同归;而固拒禅命,未践尊位,圣意恳恻,臣等敢不奉诏?辄具章遣使者。”奉令曰:“泰伯三以天下让,人无得而称焉,仲尼叹其至德,孤独何人?” Ui庚申,魏王上书曰:“皇帝陛下:奉被今月乙卯玺书,伏听册命,五内惊震,精爽散越,不知所处。臣前上还相位,退守藩国,圣恩听许。臣虽无古人量德度身自定之志,保己存性,实其私愿。不寤陛下猥损过谬之命,发不世之诏,以加无德之臣。且闻尧禅重华,举其克谐之德,舜授文命,采其齐圣之美,犹下咨四岳,上观璇玑。今臣德非虞、夏,行非二君,而承历数之谘,应选授之命,内自揆抚,无德以称。且许由匹夫,犹拒帝位,善卷布衣,而逆虞诏。臣虽鄙蔽,敢忘守节以当大命,不胜至愿。谨拜章陈情,使行相国永寿少府粪土臣毛宗奏,并上玺绶。” Ui辛酉,给事中博士苏林、董巴上表曰:“天有十二次以为分野,王公之国,各有所属,周在鹑火,魏在大梁。岁星行历十二次国,天子受命,诸侯以封。周文王始受命,岁在鹑火,至武王伐纣十三年,岁星复在鹑火,故春秋传曰:‘武王伐纣,岁在鹑火;岁之所在,即我有周之分野也。’昔光和七年,岁在大梁,武王始受命,*(为)**[于]*时将讨黄巾。是岁改年为中平元年。建安元年,岁复在大梁,始拜大将军。十三年复在大梁,始拜丞相。今二十五年,岁复在大梁,陛下受命。此魏得岁与周文王受命相应。今年青龙在庚子,诗推度灾曰:‘庚者更也,子者滋也,圣命天下治。’又曰:‘王者布德于子,治成于丑。’此言今年天更命圣人制治天下,布德于民也。魏以改制天下,与*(时)**[诗]*协矣。
颛顼受命,岁在豕韦,韂居其地,亦在豕韦,故春秋传曰:‘韂,颛顼之墟也。’今十月斗之建,则颛顼受命之分也,始魏以十月受禅,此同符始祖受命之验也。魏之氏族,出自颛顼,与舜同祖,见于春秋世家。舜以土德承尧之火,今魏亦以土德承汉之火,于行运,会于尧舜授受之次。臣闻天之去就,固有常分,圣人当之,昭然不疑,故尧捐骨肉而禅有虞,终无□色,舜发陇亩而君天下,若固有之,其相受授,闲不替漏;天下已传矣,所以急天命,天下不可一日无君也。今汉期运已终,妖异绝之已审,阶下受天之命,符瑞告征,丁宁详悉,反复备至,虽言语相喻,无以代此。今既发诏书,玺绶未御,固执谦让,上逆天命,下违民望。臣谨案古之典籍,参以图纬,魏之行运及天道所在,即尊之验,在于今年此月,昭晰分明。唯阶下迁思易虑,以时即位,显告天帝而告天下,然后改正朔,易服色,正大号,天下幸甚。”令曰:“凡斯皆宜圣德,故曰:‘苟非其人,道不虚行。’天瑞虽彰,须德而光;吾德薄之人,胡足以当之?今让,冀见听许,外内咸使闻知。” Ui壬戌,册诏曰:“皇帝问魏王言:遣宗奉庚申书到,所称引,闻之。朕惟汉家世踰二十,年过四百,运周数终,行祚已讫,天心已移,兆民望绝,天之所废,有自来矣。今大命有所厎止,神器当归圣德,违众不顺,逆天不祥。王其体有虞之盛德,应历数之嘉会,是以祯祥告符,图谶表录,神人同应,受命咸宜。朕畏上帝,致位于王;天不可违,众不可拂。且重华不逆尧命,大禹不辞舜位,若夫由、卷匹夫,不载圣籍,固非皇材帝器所当称慕。今使音奉皇帝玺绶,王其陟帝位,无逆朕命,以祗奉天心焉。” Ui于是尚书令桓阶等奉曰:“今汉使音奉玺书到,臣等以为天命不可稽,神器不可渎。周武中流有白鱼之应,不待师期而大号已建,舜受大麓,桑荫未移而已陟帝位,皆所以祗承天命,若此之速也。故无固让之义,不以守节为贵,必道信于神灵,符合于天地而已。易曰:‘其受命如响,无有远近幽深,遂知来物,非天下之至赜,其孰能与于此?’今陛下应期运之数,为皇天所子,而复稽滞于辞让,低回于大号,非所以则天地之道,副万国之望。臣等敢以死请,辄敕有司修治坛场,择吉日,受禅命,发玺绶。”令曰:“冀三让而不见听,何汲汲于斯乎?” Ui甲子,魏王上书曰:“奉今月壬戌玺书,重被圣命,伏听册告,肝胆战悸,不知所措。天下神器,禅代重事,故尧将禅舜,纳于大麓,舜之命禹,玄圭告功;烈风不迷,九州攸平,询事考言,然后乃命,而犹执谦让于德不嗣。况臣顽固,质非二圣,乃应天统,受终明诏;敢守微节,归志箕山,不胜大愿。谨拜表陈情,使并奉上玺绶。” Ui侍中刘廙等奏曰:“臣等闻圣帝不违时,明主不逆人,故易称通天下之志,断天下之疑。伏惟陛下体有虞之上圣,承土德之行运,当亢阳明夷之会,应汉氏祚终之数,合契皇极,同符两仪。是以圣瑞表征,天下同应,历运去就,深切着明;论之天命,无所与议,比之时宜,无所与争。故受命之期,时清日晏,曜灵施光,休气云蒸。是乃天道悦怿,民心欣戴,而仍见闭拒,于礼何居?且髃生不可一日无主,神器不可以斯须无统,故臣有违君以成业,下有矫上以立事,臣等敢不重以死请。”王令曰:“天下重器,王者正统,以圣德当之,犹有惧心,吾何人哉?且公卿未至乏主,斯岂小事,且宜以待固让之后,乃当更议其可耳。” Ui丁卯,册诏魏王曰:“天讫汉祚,辰象着明,朕祗天命,致位于王,仍陈历数于诏册,喻符运于翰墨;神器不可以辞拒,皇位不可以谦让,稽于天命,至于再三。
且四海不可以一日旷主,万机不可以斯须无统,故建大业者不拘小节,知天命者不系细物,是以舜受大业之命而无逊让之辞,圣人达节,不亦远乎!今使音奉皇帝玺绶,王其钦承,以答天下向应之望焉。” Ui相国华歆、太尉贾诩、御史大夫王朗及九卿上言曰:“臣等被召到,伏见太史丞许芝、左中郎将李伏所上图谶、符命,侍中刘廙等宣□众心,人灵同谋。又汉朝知陛下圣化通于神明,圣德参于虞、夏,因瑞应之备至,听历数之所在,遂献玺绶,固让尊号。能言之伦,莫不抃舞,河图、洛书,天命瑞应,人事协于天时,民言协于天□。而陛下性秉劳谦,体尚克让,明诏恳切,未肯听许,臣妾小人,莫不伊邑。臣等闻自古及今,有天下者不常在乎一姓;考以德势,则盛衰在乎强弱,论以终始,则废兴在乎期运。唐、虞历数,不在厥子而在舜、禹。舜、禹虽怀克让之意迫,髃后执玉帛而朝之,兆民怀欣戴而归之,率土扬歌谣而咏之,故其守节之拘,不可得而常处,达节之权,不可得而久避;是以或逊位而不□,或受禅而不辞,不□者未必厌皇宠,不辞者未必渴帝祚,各迫天命而不得以已。既禅之后,则唐氏之子为宾于有虞,虞氏之冑为客于夏代,然则禅代之义,非独受之者实应天福,授之者亦与有余庆焉。汉自章、和之后,世多变故,稍以陵迟,洎乎孝灵,不恒其心,虐贤害仁,聚敛无度,政在嬖竖,视民如绚,遂令上天震怒,百姓从风如归;当时则四海鼎沸,既没则祸发宫庭,宠势并竭,帝室遂卑,若在帝舜之末节,犹择圣代而授之,荆人抱玉璞,犹思良工而刊之,况汉国既往,莫之能匡,推器移君,委之圣哲,固其宜也。汉朝委质,既愿礼禅之速定也,天祚率土,必将有主;主率土者,非陛下其孰能任之?所谓论德无与为比,考功无推让矣。天命不可久稽,民望不不可久违,臣等慺慺,不胜大愿。伏请陛下割撝谦之志,修受禅之礼,副人神之意,慰外内之愿。”令曰:“以德则孤不足,以时则戎虏未灭。若以髃贤之灵,得保首领,终君魏国,于孤足矣。若孤者,胡足以辱四海?至乎天瑞人事,皆先王圣德遗庆,孤何有焉?是以未敢闻命。” Ui己巳,魏王上书曰:“臣闻舜有宾于四门之勋,乃受禅于陶唐,禹有存国七百之功,乃承禄于有虞。臣以蒙蔽,德非二圣,猥当天统,不敢闻命。敢屡抗疏,略陈私愿,庶章通紫庭,得全微节,情达宸极,永守本志。而音重复衔命,申制诏臣,臣实战惕,不发玺书,而音迫于严诏,不敢复命。愿陛下驰传骋驿,召音还台。不胜至诚,谨使宗奉书。” Ui相国歆、太尉诩、御史大夫朗及九卿奏曰:“臣等伏读诏书,于邑益甚。臣等闻易称圣人奉天时,论语云君子畏天命,天命有去就,然后帝者有禅代。是以唐之禅虞,命在尔躬,虞之顺唐,谓之受终;尧知天命去己,故不得不禅舜,舜知历数在躬,故不敢不受;不得不禅,奉天时也,不敢不受,畏天命也。汉朝虽承季末陵迟之余,犹务奉天命以则尧之道,是以愿禅帝位而归二女。而陛下正于大魏受命之初,抑虞、夏之达节,尚延陵之让退,而所枉者大,所直者小,所详者轻,所略者重,中人凡士犹为陛下陋之。没者有灵,则重华必忿愤于苍梧之神墓,大禹必郁悒于会稽之山阴,武王必不悦于*(商)**[高]*陵之玄宫矣。是以臣等敢以死请。且汉政在阉宦,禄去帝室七世矣,遂集矢石于其宫殿,而二京为之丘墟。当是之时,四海荡覆,天下分崩,武王亲衣甲而冠冑,沐雨而栉风,为民请命,则活万国,为世拨乱,则致升平,鸠民而立长,筑宫而置吏,元元无过,罔于前业,而始有造于华夏。陛下即位,光昭文德,以翊武功,勤恤民隐,视之如伤,惧者宁之,劳者息之,寒者以暖,饥者以充,远人以*(恩复)**[德服]*,寇敌以恩降,迈恩种德,光被四表;稽古笃睦,茂于放勋,网漏吞舟,弘乎周文。是以布政未儙,人神并和,皇天则降甘露而臻四灵,后土则挺芝草而吐醴泉,虎豹鹿兔,皆素其色,雉鸠燕雀,亦白其羽,连理之木,同心之瓜,五采之鱼,珍祥瑞物,杂嗠于其间者,无不毕备。古人有言:‘微禹,吾其鱼乎!’微大魏,则臣等之白骨交横于旷野矣。伏省髃臣外内前后章奏,所以陈□陛下之符命者,莫不条河洛之图书,据天地之瑞应,因汉朝之款诚,宣万方之景附,可谓信矣*(省)**[着]*矣;三王无以及,五帝无以加。民命之悬于魏*[邦,民心之系于魏]*政,三十有余年矣,此乃千世时至之会,万载一遇之秋;达节广度,宜昭于斯际,拘牵小节,不施于此时。久稽天命,罪在臣等。辄营坛场,具礼仪,择吉日,昭告昊天上帝,秩髃神之礼,须禋祭毕,会髃寮于朝堂,议年号、正朔、服色当施行,上。”复令曰:“昔者大舜饭糗茹草,将终身焉,斯则孤之前志也。及至承尧禅,被*(珍)**[袗]*裘,妻二女,若固有之,斯则顺天命也。髃公卿士诚以天命不可拒,民望不可违,孤亦曷以辞焉?” Ui庚午,册诏魏王曰:“昔尧以配天之德,秉六合之重,犹鷪历运之数,移于有虞,委让帝位,忽如遗迹。今天既讫我汉命,乃眷北顾,帝皇之业,实在大魏。朕守空名以窃古义,顾视前事,犹有惭德,而王逊让至于三四,朕用惧焉。夫不辞万乘之位者,知命达节之数也,虞、夏之君,处之不疑,故勋烈垂于万载,美名传于无穷。
今遣守尚书令侍中*(顗)**[觊]*喻,王其速陟帝位,以顺天人之心,副朕之大愿。”
Ui于是尚书令桓阶等奏曰:“今汉氏之命已四至,而陛下前后固辞,臣等伏以为上帝之临圣德,期运之隆大魏,斯岂数载?传称周之有天下,非甲子之朝,殷之去帝位,非牧野之日也,故诗序商汤,追本玄王之至,述姬周,上录后稷之生,是以受命既固,厥德不回。汉氏衰废,行次已绝,三辰垂其征,史官着其验,耆老记先古之占,百姓协歌谣之声。陛下应天受禅,当速即坛场,柴燎上帝,诚不宜久停神器,拒亿兆之愿。臣辄下太史令择元辰,今月二十九日,可登坛受命,请诏王公髃卿,具条礼仪别奏。”令曰:“可。”
注[三]献帝传曰:辛未,魏王登坛受禅,公卿、列侯、诸将、匈奴单于、四夷朝者数万人陪位,燎祭天地、五岳、四渎,曰:“皇帝臣丕敢用玄牡昭告于皇皇后帝:汉历世二十有四,践年四百二十有六,四海困穷,三纲不立,五纬错行,灵祥并见,推术数者,虑之古道,咸以为天之历数,运终兹世,凡诸嘉祥民神之意,比昭有汉数终之极,魏家受命之符。汉主以神器宜授于臣,宪章有虞,致位于丕。丕震畏天命,虽休勿休。髃公庶尹六事之人,外及将士,洎于蛮夷君长,佥曰:‘天命不可以辞拒,神器不可以久旷,髃臣不可以无主,万几不可以无统。’丕祗承皇象,敢不钦承。卜之守龟,兆有大横,筮之三易,兆有革兆,谨择元日,与髃寮登坛受帝玺绶,告类于尔大神;唯尔有神,尚飨永吉,兆民之望,祚于有魏世享。”
遂制诏三公:“上古之始有君也,必崇恩化以美风俗,然百姓顺教而刑辟厝焉。今朕承帝王之绪,其以延康元年为黄初元年,议改正朔,易服色,殊徽号,同律度量,承土行,大赦天下;自殊死以下,诸不当得赦,皆赦除之。” Ui魏氏春秋曰:帝升坛礼毕,顾谓髃臣曰:
“舜、禹之事,吾知之矣。” Ui干窦搜神记曰:宋大夫邢史子臣明于天道,周敬王之三十七年,景公问曰:“天道其何祥?”对曰:“后五*(十)*年五月丁亥,臣将死;死后五年五月丁卯,吴将亡;亡后五年,君将终;终后四百年,邾王天下。”
俄而皆如其言。所云邾王天下者,谓魏之兴也。邾,曹姓,魏亦曹姓,皆邾之后。其年数则错,未知邢史失其数邪,将年代久远,注记者传而有谬也?
黄初元年十一月癸酉,以河内之山阳邑万户奉汉帝为山阳公,行汉正朔,以天子之礼郊祭,上书不称臣,京都有事于太庙,致胙;封公之四子为列侯。追尊皇祖太王曰太皇帝,考武王曰武皇帝,尊王太后曰皇太后。赐男子爵人一级,为父后及孝悌力田人二级。以汉诸侯王为崇德侯,列侯为关中侯。以颍阴之繁阳亭为繁昌县。封爵增位各有差。改相国为司徒,御史大夫为司空,奉常为太常,郎中令为光禄勋,大理为廷尉,大农为大司农。郡国县邑,多所改易。更授匈奴南单于呼厨泉魏玺绶,赐青盖车、乘舆、宝剑、玉玦。十二月,初营洛阳宫,戊午幸洛阳。[一]
注[一]臣松之案:诸书记是时帝居北宫,以建始殿朝髃臣,门曰承明,陈思王植诗曰“谒帝承明庐”是也。至明帝时,始于汉南宫崇德殿处起太极、昭阳诸殿。魏书曰:以夏数为得天,故即用夏正,而服色尚黄。魏略曰:诏以汉火行也,火忌水,故“洛”去“水”而加“佳”。
魏于行次为土,土,水之牡也,水得土而乃流,土得水而柔,故除“佳”加“水”,变“雒”为“洛”。
是岁,长水校尉戴陵谏不宜数行弋猎,帝大怒;陵减死罪一等。
二年春正月,郊祀天地、明堂。甲戌,校猎至原陵,遣使者以太牢祠汉世祖。乙亥,朝日于东郊。[一]初令郡国口满十万者,岁察孝廉一人;其有秀异,无拘户口。辛巳,分三公户邑,封子弟各一人为列侯。壬午,复颍川郡一年田租。[二]改许县为许昌县。以魏郡东部为阳平郡,西部为广平郡。[三]
注[一]臣松之以为礼天子以春分朝日,秋分夕月;寻此年正月郊祀,有月无日,乙亥朝日,则有日无月,盖文之脱也。案明帝朝日夕月,皆如礼文,故知此纪为误者也。
注[二]魏书载诏曰:“颍川,先帝所由起兵征伐也。官渡之役,四方瓦解,远近顾望,而此郡守义,丁壮荷戈,老弱负粮。昔汉祖以秦中为国本,光武恃河内为王基,今朕复于此登坛受禅,天以此郡翼成大魏。”
注[三]魏略曰:改长安、谯、许昌、邺、洛阳为五都;立石表,西界宜阳,北循太行,东北界阳平,南循鲁阳,东界郯,为中都之地。令天下听内徙,复五年,后又增其复。
诏曰:“昔仲尼资大圣之才,怀帝王之器,当衰周之末,无受命之运,在鲁、韂之朝,教化乎洙、泗之上,凄凄焉,遑遑焉,欲屈己以存道,贬身以救世。于时王公终莫能用之,乃退考五代之礼,修素王之事,因鲁史而制春秋,就太师而正雅颂,俾千载之后,莫不宗其文以述作,仰其圣以成谋,咨!可谓命世之大圣,亿载之师表者也。遭天下大乱,百祀堕坏,旧居之庙,毁而不修,褒成之后,绝而莫继,阙里不闻讲颂之声,四时不鷪蒸尝之位,斯岂所谓崇礼报功,盛德百世必祀者哉!其以议郎孔羡为宗圣侯,邑百户,奉孔子祀。”令鲁郡修起旧庙,置百户吏卒以守韂之,又于其外广为室屋以居学者。
*(春)*三月,加辽东太守公孙恭为车骑将军。初复五铢钱。夏四月,以车骑将军曹仁为大将军。五月,郑甘复叛,遣曹仁讨斩之。六月庚子,初祀五岳四渎,咸秩髃祀。[一]丁卯,夫人甄氏卒。戊辰晦,日有食之,有司奏免太尉,诏曰:“灾异之作,以谴元首,而归过股肱,岂禹、汤罪己之义乎?其令百官各虔厥职,后有天地之眚,勿复劾三公。”
注[一]魏书:甲辰,以京师宗庙未成,帝亲祠武皇帝于建始殿,躬执馈奠,如家人之礼。
秋八月,孙权遣使奉章,并遣于禁等还。丁巳,使太常邢贞持节拜权为大将军,封吴王,加九锡。冬十月,授杨彪光禄大夫。[一]以谷贵,罢五铢钱。[二]己卯,以大将军曹仁为大司马。十二月,行东巡。是岁筑陵云台。
注[一]魏书曰:己亥,公卿朝朔旦,并引故汉太尉杨彪,待以客礼,诏曰:“夫先王制几杖之赐,所以宾礼黄耇褒崇元老也。昔孔光、卓茂皆以淑德高年,受兹嘉锡。公故汉宰臣,乃祖已来,世著名节,年过七十,行不踰矩,可谓老成人矣,所宜宠异以章旧德。其赐公延年杖及冯几;谒请之日,便使杖入,又可使着鹿皮冠。”彪辞让不听,竟着布单衣、皮弁以见。
续汉书曰:彪见汉祚将终,自以累世为三公,耻为魏臣,遂称足挛,不复行。积十余年,帝即王位,欲以为太尉,令近臣宣旨。彪辞曰:“尝以汉朝为三公,值世衰乱,不能立尺寸之益,若复为魏臣,于国之选,亦不为荣也。”帝不夺其意。黄初四年,诏拜光禄大夫,秩中二千石,朝见位次三公,如孔光故事。彪上章固让,帝不听,又为门施行马,致吏卒,以优崇之。年八十四,以六年薨。子修,事见陈思王传。
注[二]魏书曰:十一月辛未,镇西将军曹真命众将及州郡兵讨破叛胡治元多、卢水、封赏等,斩首五万余级,获生口十万,羊一百一十一万口,牛八万,河西遂平。帝初闻胡决水灌显美,谓左右诸将曰:“昔隗嚣灌略阳,而光武因其疲弊,进兵灭之。今胡决水灌显美,其事正相似,破胡事今至不久。”旬日,破胡告檄到,上大笑曰:“吾策之于帷幕之内,诸将奋击于万里之外,其相应若合符节。前后战克获虏,未有如此也。”
三年春正月丙寅朔,日有蚀之。庚午,行幸许昌宫。诏曰:“今之计、*(考)**[孝]*,古之贡士也;十室之邑,必有忠信,若限年然后取士,是吕尚、周晋不显于前世也。其令郡国所选,勿拘老幼;儒通经术,吏达文法,到皆试用。有司纠故不以实者。”[一]
注[一]魏书曰:癸亥,孙权上书,说:“刘备支党四万人,马二三千匹,出秭归,请往扫扑,以克捷为效。”帝报曰:“昔隗嚣之弊,祸发栒邑,子阳之禽,变起扞关,将军其亢厉威武,勉蹈奇功,以称吾意。”
二月,鄯善、龟兹、于阗王各遣使奉献,诏曰:“西戎即□,氐、羌来王,诗、书美之。顷者西域外夷并款塞内附,[一]其遣使者抚劳之。”是后西域遂通,置戊己校尉。
注[一]应劭汉书注曰:款,叩也;皆叩塞门来服从。
三月乙丑,立齐公叡为平原王,帝弟鄢陵公彰等十一人皆为王。初制封王之庶子为乡公,嗣王之庶子为亭侯,公之庶子为亭伯。甲戌,立皇子霖为河东王。甲午,行幸襄邑。夏四月戊申,立鄄城侯植为鄄城王。癸亥,行还许昌宫。五月,以荆、扬、江表八郡为荆州,孙权领牧故也;荆州江北诸郡为郢州。
闰月,孙权破刘备于夷陵。初,帝闻备兵东下,与权交战,树栅连营七百余里,谓髃臣曰:
“备不晓兵,岂有七百里营可以拒敌者乎!‘苞原隰险阻而为军者为敌所禽’,此兵忌也。
孙权上事今至矣。”后七日,破备书到。
秋七月,冀州大蝗,民饥,使尚书杜畿持节开仓廪以振之。八月,蜀大将黄权率众降。[一]
注[一]魏书曰:权及领南郡太守史合等三百一十八人,诣荆州刺史奉上所假印绶、棨戟、幢麾、牙门、鼓车。权等诣行在所,帝置酒设乐,引见于承光殿。权、合等人人前自陈,帝为论说军旅成败去就之分,诸将无不喜悦。赐权金帛、车马、衣裘、帷帐、妻妾,下及偏裨皆有差。拜权为侍中镇南将军,封列侯,即日召使骖乘;及封史合等四十二人皆为列侯,为将军郎将百余人。
九月甲午,诏曰:“夫妇人与政,乱之本也。自今以后,髃臣不得奏事太后,后族之家不得当辅政之任,又不得横受茅土之爵;以此诏传后世,若有背违,天下共诛之。”[一]庚子,立皇后郭氏。赐天下男子爵人二级;□寡笃癃及贫不能自存者赐谷。
注[一]孙盛曰:夫经国营治,必凭俊箉之辅,贤达令德,必居参乱之任,故虽周室之盛,有妇人与焉。然则坤道承天,南面罔二,三从之礼,谓之至顺,至于号令自天子出,奏事专行,非古义也。昔在申、吕,实匡有周。苟以天下为心,惟德是杖,则亲簄之授,至公一也,何至后族而必斥远之哉?二汉之季世,王道陵迟,故令外戚凭宠,职为乱阶。*(于)*此自时昏道丧,运祚将移,纵无王、吕之难,岂乏田、赵之祸乎?而后世观其若此,深怀酖毒之戒也。
至于魏文,遂发一概之诏,可谓有识之爽言,非帝者之宏议。
冬十月甲子,表首阳山东为寿陵,作终制曰:“礼,国君即位为椑,*椑音扶历反。*存不忘亡也。[一]昔尧葬谷林,通树之,禹葬会稽,农不易亩,[二]故葬于山林,则合乎山林。封树之制,非上古也,吾无取焉。寿陵因山为体,无为封树,无立寝殿,造园邑,通神道。夫葬也者,藏也,欲人之不得见也。骨无痛痒之知,頉非栖神之宅,礼不墓祭,欲存亡之不黩也,为棺椁足以朽骨,衣衾足以朽肉而已。故吾营此丘墟不食之地,欲使易代之后不知其处。
无施苇炭,无藏金银铜铁,一以瓦器,合古涂车、刍灵之义。棺但漆际会三过,饭含无以珠玉,无施珠襦玉匣,诸愚俗所为也。季孙以玙璠敛,孔子历级而救之,譬之暴骸中原。宋公厚葬,君子谓华元、乐莒不臣,以为弃君于恶。汉文帝之不发,霸陵无求也;光武之掘,原陵封树也。霸陵之完,功在释之;原陵之掘,罪在明帝。是释之忠以利君,明帝爱以害亲也。
忠臣孝子,宜思仲尼、丘明、释之之言,鉴华元、乐莒、明帝之戒,存于所以安君定亲,使魂灵万载无危,斯则贤圣之忠孝矣。自古及今,未有不亡之国,亦无不掘之墓也。丧乱以来,汉氏诸陵无不发掘,至乃烧取玉匣金缕,骸骨并尽,是焚如之刑,岂不重痛哉!祸由乎厚葬封树。‘桑、霍为我戒’,不亦明乎?其皇后及贵人以下,不随王之国者,有终没皆葬涧西,前又以表其处矣。盖舜葬苍梧,二妃不从,延陵葬子,远在嬴、博,魂而有灵,无不之也,一涧之闲,不足为远。若违今诏,妄有所变改造施,吾为戮尸地下,戮而重戮,死而重死。臣子为蔑死君父,不忠不孝,使死者有知,将不福汝。其以此诏藏之宗庙,副在尚书、秘书、三府。”
注[一]臣松之按:礼,天子诸侯之棺,各有重数;棺之亲身者曰椑。
注[二]吕氏春秋:尧葬于谷林,通树之;舜葬于纪,市廛不变其肆;禹葬会稽,不变人徒。
是月,孙权复叛。复郢州为荆州。帝自许昌南征,诸军兵并进,权临江拒守。十一月辛丑,行幸宛。庚申晦,日有食之。是岁,穿灵芝池。
四年春正月,诏曰:“丧乱以来,兵革未戢,天下之人,互相残杀。今海内初定,敢有私复雠者皆族之。”筑南巡台于宛。三月丙申,行自宛还洛阳宫。癸卯,月犯心中央大星。[一]丁未,大司马曹仁薨。是月大疫。
注[一]魏书载丙午诏曰:“孙权残害民物,朕以寇不可长,故分命猛将三道并征。今征东诸军与权党吕范等水战,则斩首四万,获船万艘。大司马据守濡须,其所禽获亦以万数。中军、征南,攻围江陵,左将军张合等舳舻直渡,击其南渚,贼赴水溺死者数千人,又为地道攻城,城中外雀鼠不得出入,此几上肉耳!而贼中疠气疾病,夹江涂地,恐相染污。昔周武伐殷,旋师孟津,汉祖征隗嚣,还军高平,皆知天时而度贼情也。且成汤解三面之网,天下归仁。今开江陵之围,以缓成死之禽。且休力役,罢省繇戍,畜养士民,咸使安息。”
夏五月,有鹈鹕鸟集灵芝池,诏曰:“此诗人所谓污泽也。曹诗‘刺恭公远君子而近小人’,今岂有贤智之士处于下位乎?否则斯鸟何为而至?其博举天下鉨德茂才、独行君子,以答曹人之刺。”[一]
注[一]魏书曰:辛酉,有司奏造二庙,立太皇帝庙,大长秋特进侯与高祖合祭,亲尽以次毁;
特立武皇帝庙,四时享祀,为魏太祖,万载不毁也。
六月甲戌,任城王彰薨于京都。甲申,太尉贾诩薨。太白昼见。是月大雨,伊、洛溢流,杀人民,坏庐宅。[一]秋八月丁卯,以廷尉钟繇为太尉。[二]辛未,校猎于荥阳,遂东巡。论征孙权功,诸将已下进爵增户各有差。九月甲辰,行幸许昌宫。[三]
注[一]魏书曰:七月乙未,大军当出,使太常以特牛一告祠于郊。臣松之按:魏郊祀奏中,尚书卢毓议祀厉殃事云:“具牺牲祭器,如前后师出告郊之礼。”如此,则魏氏出师,皆告郊也。
注[二]魏书曰:有司奏改汉氏宗庙安世乐曰正世乐,嘉至乐曰迎灵乐,武德乐曰武颂乐,昭容乐曰昭业乐,云*(翻)**[翘]*舞曰凤翔舞,育命舞曰灵应舞,武德舞曰武颂舞,文*(昭)**[始]*舞曰大*(昭)**[韶]*舞,五行舞曰大武舞。
注[三]魏书曰:十二月丙寅,赐山阳公夫人汤沐邑,公女曼为长乐郡公主,食邑各五百户。
是冬,甘露降芳林园。臣松之按:芳林园即今华林园,齐王芳即位,改为华林。
五年春正月,初令谋反大逆乃得相告,其余皆勿听治;敢妄相告,以其罪罪之。三月,行自许昌还洛阳宫。夏四月,立太学,制五经课试之法,置春秋谷梁博士。五月,有司以公卿朝朔望日,因奏疑事,听断大政,论辨得失。秋七月,行东巡,幸许昌宫。八月,为水军,亲御龙舟,循蔡、颍,浮淮,幸寿春。扬州界将吏士民,犯五岁刑已下,皆原除之。九月,遂至广陵,赦青、徐二州,改易诸将守。冬十月乙卯,太白昼见。行还许昌宫。[一]十一月庚寅,以冀州饥,遣使者开仓廪振之。戊申晦,日有食之。
注[一]魏书载癸酉诏曰:“近之不绥,何远之怀?今事多而民少,上下相弊以文法,百姓无所措其手足。昔太山之哭者,以为苛政甚于猛虎,吾备儒者之风,服圣人之遗教,岂可以目翫其辞,行违其诫者哉?广议轻刑,以惠百姓。”
十二月,诏曰:“先王制礼,所以昭孝事祖,大则郊社,其次宗庙,三辰五行,名山大川,非此族也,不在祀典。叔世衰乱,崇信巫史,至乃宫殿之内,户牖之闲,无不沃酹,甚矣其惑也。自今,其敢设非祀之祭,巫祝之言,皆以执左道论,着于令典。”是岁穿天渊池。
六年春二月,遣使者循行许昌以东尽沛郡,问民所疾苦,贫者振贷之。[一]三月,行幸召陵,通讨虏渠。乙巳,还许昌宫。并州刺史梁习讨鲜卑轲比能,大破之。辛未,帝为舟师东征。五月戊申,幸谯。壬戌,荧惑入太微。
注[一]魏略载诏曰:“昔轩辕建四面之号,周武称‘予有乱臣十人’,斯盖先圣所以体国君民,亮成天工,多贤为贵也。今内有公卿以镇京师,外设牧伯以监四方,至于元戎出征,则军中宜有柱石之贤帅,辎重所在,又宜有镇守之重臣,然后车驾可以周行天下,无内外之虑。吾今当征贼,欲守之积年。其以尚书令颍乡侯陈髃为镇军大将军,尚书仆射西乡侯司马懿为抚军大将军。若吾临江授诸将方略,则抚军当留许昌,督后诸军,录后台文书事;镇军随车驾,当董督众军,录行尚书事;皆假节鼓吹,给中军兵骑六百人。吾欲去江数里,筑宫室,往来其中,见贼可击之形,便出奇兵击之;若或未可,则当舒六军以游猎,飨赐军士。”
六月,利成郡兵蔡方等以郡反,杀太守徐质。遣屯骑校尉任福、步兵校尉段昭与青州刺史讨平之;其见胁略及亡命者,皆赦其罪。
秋七月,立皇子鉴为东武阳王。八月,帝遂以舟师自谯循涡入淮,从陆道幸徐。九月,筑东巡台。冬十月,行幸广陵故城,临江观兵,戎卒十余万,旌旗数百里。[一]是岁大寒,水道冰,舟不得入江,乃引还。十一月,东武阳王鉴薨。十二月,行自谯过梁,遣使以太牢祀故汉太尉桥玄。
注[一]魏书载帝于马上为诗曰:“观兵临江水,水流何汤汤!戈矛成山林,玄甲耀日光。猛将怀暴怒,胆气正从横。谁云江水广,一苇可以航,不战屈敌虏,戢兵称贤良。古公宅岐邑,实始翦殷商。孟献营虎牢,郑人惧稽颡。充国务耕植,先零自破亡。兴农淮、泗间,筑室都徐方。量宜运权略,六军咸悦康;岂如东山诗,悠悠多忧伤。”
七年春正月,将幸许昌,许昌城南门无故自崩,帝心恶之,遂不入。壬子,行还洛阳宫。三月,筑九华台。夏五月丙辰,帝疾笃,召中军大将军曹真、镇军大将军陈髃、征东大将军曹休、抚军大将军司马宣王,并受遗诏辅嗣主。遣后宫淑媛、昭仪已下归其家。丁巳,帝崩于嘉福殿,时年四十。[一]六月戊寅,葬首阳陵。自殡及葬,皆以终制从事。[二]
注[一]魏书曰:殡于崇华前殿。
注[二]魏氏春秋曰:明帝将送葬,曹真、陈髃、王朗等以暑热固谏,乃止。孙盛曰:夫窀穸之事,孝子之极痛也,人伦之道,于斯莫重。故天子七月而葬,同轨毕至。夫以义感之情,犹尽临隧之哀,况乎天性发中,敦礼者重之哉!魏氏之德,仍世不基矣。昔华元厚葬,君子以为弃君于恶,髃等之谏,弃孰甚焉!鄄城侯植为诔曰:“惟黄初七年五月七日,大行皇帝崩,呜呼哀哉!于时天震地骇,崩山陨霜,阳精薄景,五纬错行,百姓呼嗟,万国悲伤,若丧考妣,*(恩过慕)**[思慕过]*唐,擗踊郊野,仰想穹苍,佥曰何辜,早世殒丧,呜呼哀哉!
悲夫大行,忽焉光灭,永弃万国,云往雨绝。承问荒忽,惛懵哽咽,袖锋抽刃,叹自僵毙,追慕三良,甘心同穴。感惟南风,惟以郁滞,终于偕没,指景自誓。考诸先记,寻之哲言,生若浮寄,唯德可论,朝闻夕逝,孔志所存。皇虽一没,天禄永延,何以述德?表之素旃。
何以咏功?宣之管弦。乃作诔曰:皓皓太素,两仪始分,中和产物,肇有人伦,爰暨三皇,实秉道真,降逮五帝,继以懿纯,三代制作,踵武立勋。季嗣不维,网漏于秦,崩乐灭学,儒坑礼焚,二世而歼,汉氏乃因,弗求古训,嬴政是遵,王纲帝典,阒尔无闻。末光幽昧,道究运迁,乾坤回历,简圣授贤,乃眷大行,属以黎元。龙飞启祚,合契上玄,五行定纪,改号革年,明明赫赫,受命于天。
仁风偃物,德以礼宣;祥惟圣质,嶷在幼妍。庶几六典,学不过庭,潜心无罔,抗志青冥。
才秀藻朗,如玉之莹,听察无向,瞻鷪未形。其刚如金,其贞如琼,如冰之洁,如砥之平。
爵公无私,戮违无轻,心镜万机,揽照下情。思良股肱,嘉昔伊、吕,搜扬侧陋,举汤代禹;
拔才岩穴,取士蓬户,唯德是萦,弗拘祢祖。宅土之表,道义是图,弗营厥险,六合是虞。
齐契共遵,下以纯民,恢拓规矩,克绍前人。科条品制,曪贬以因。乘殷之辂,行夏之辰。
金根黄屋,翠葆龙鳞,绋冕崇丽,衡紞维新,尊肃礼容,瞩之若神。方牧妙举,钦于恤民,虎将荷节,镇彼四邻;朱旗所剿,九壤被震,畴克不若?孰敢不臣?县旌海表,万里无尘。
虏备凶彻,鸟殪江岷,权若涸鱼,干腊矫鳞,肃慎纳贡,越裳效珍,条支绝域,侍子内宾。
德侪先皇,功侔太古。上灵降瑞,黄初叔祜:河龙洛龟,凌波游下;平钧应绳,神鸾翔舞;
数荚阶除,系风扇暑;皓兽素禽,飞走郊野;神钟宝鼎,形自旧土;云英甘露,瀸涂被宇;
灵芝冒沼,朱华荫渚。回回凯风,祁祁甘雨,稼穑丰登,我稷我黍。家佩惠君,户蒙慈父。
图致太和,洽德全义。将登介山,先皇作俪。镌石纪勋,兼录众瑞,方隆封禅,归功天地,宾礼百灵,勋命视规,望祭四岳,燎封奉柴,肃于南郊,宗祀上帝。三牲既供,夏禘秋尝,元侯佐祭,献璧奉璋。鸾舆幽蔼,龙旗太常,爰迄太庙,钟鼓锽锽,颂德咏功,八佾锵锵。
皇祖既飨,烈考来享,神具醉止,降兹福祥。天地震荡,大行康之;三辰暗昧,大行光之;
皇纮绝维,大行纲之;神器莫统,大行当之;礼乐废弛,大行张之;仁义陆沉,大行扬之;
潜龙隐凤,大行翔之;疏狄遐康,大行匡之。在位七载,元功仍举,将永太和,绝迹三五,宜作物师,长为神主,寿终金石,等算东父,如何奄忽,摧身后土,俾我□□,靡瞻靡顾。
嗟嗟皇穹,胡宁忍务?呜呼哀哉!明监吉凶,体远存亡,深垂典制,申之嗣皇。圣上虔奉,是顺是将,乃□玄宇,基为首阳,拟夡谷林,追尧慕唐,合山同陵,不树不疆,涂车刍灵,珠玉靡藏。百神警侍,来宾幽堂,耕禽田兽,望魂之翔。于是,俟大隧之致功兮,练元辰之淑祯,潜华体于梓宫兮,冯正殿以居灵。顾望嗣之号咷兮,存临者之悲声,悼晏驾之既修兮,感容车之速征。浮飞魂于轻霄兮,就黄墟以灭形,背三光之昭晰兮,归玄宅之冥冥。嗟一往之不反兮,痛閟闼之长扃。咨远臣之眇眇兮,感凶讳以怛惊,心孤绝而靡告兮,纷流涕而交颈。思恩荣以横奔兮,阂阙塞之峣峥,顾衰绖以轻举兮,迫关防之我婴。欲高飞而遥憩兮,惮天网之远经,遥投骨于山足兮,报恩养于下庭。慨拊心而自悼兮,惧施重而命轻,嗟微驱之是效兮,甘九死而忘生,几司命之役籍兮,先黄发而陨零,天盖高而察卑兮,冀神明之我听。独郁伊而莫愬兮,追顾景而怜形,奏斯文以写思兮,结翰墨以敷诚。呜呼哀哉!”
初,帝好文学,以著述为务,自所勒成垂百篇。又使诸儒撰集经传,随类相从,凡千余篇,号曰皇览。[一]
注[一]魏书曰:帝初在东宫,疫疠大起,时人雕伤,帝深感叹,与素所敬者大理王朗书曰:
“生有七尺之形,死唯一棺之土,唯立德扬名,可以不朽,其次莫如着篇籍。疫疠数起,士人雕落,余独何人,能全其寿?”故论撰所着典论、诗赋,盖百余篇,集诸儒于肃城门内,讲论大义,侃侃无倦。常嘉汉文帝之为君,宽仁玄默,务欲以德化民,有贤圣之风。时文学诸儒,或以为孝文虽贤,其于聪明,通达国体,不如贾谊。帝由是着太宗论曰:“昔有苗不宾,重华舞以干戚,尉佗称帝,孝文抚以恩德,吴王不朝,锡之几杖以抚其意,而天下赖安;
乃弘三章之教,恺悌之化,欲使曩时累息之民,得阔步高谈,无危惧之心。若贾谊之才敏,筹画国政,特贤臣之器,管、晏之姿,岂若孝文大人之量哉?”三年之中,以孙权不服,复颁太宗论于天下,明示不愿征伐也。他日又从容言曰:“顾我亦有所不取于汉文帝者三:杀薄昭;幸邓通;慎夫人衣不曳地,集上书囊为帐帷。以为汉文俭而无法,舅后之家,但当养育以恩而不当假借以权,既触罪法,又不得不害矣。”其欲秉持中道,以为帝王仪表者如此。胡冲吴历曰:帝以素书所着典论及诗赋饷孙权,又以纸写一通与张昭。
评曰:文帝天资文藻,下笔成章,博闻强识,才蓺兼该;[一]若加之旷大之度,励以公平之诚,迈志存道,克广德心,则古之贤主,何远之有哉!
注[一]典论帝自□曰:初平之元,董卓杀主鸩后,荡覆王室。是时四海既困中平之政,兼恶卓之凶逆,家家思乱,人人自危。山东牧守,咸以春秋之义,“韂人讨州吁于濮”,言人人皆得讨贼。于是大兴义兵,名豪大侠,富室强族,飘扬云会,万里相赴;兖、豫之师战于荥阳,河内之甲军于孟津。卓遂迁大驾,西都长安。而山东大者连郡国,中者婴城邑,小者聚阡陌,以还相吞灭。会黄巾盛于海、岱,山寇暴于并、冀,乘胜转攻,席卷而南,乡邑望烟而奔,城郭鷪尘而溃,百姓死亡,暴骨如莽。余时年五岁,上以世方扰乱,教余学射,六岁而知射,又教余骑马,八岁而能骑射矣。以时之多故,每征,余常从。建安初,上南征荆州,至宛,张绣降。旬日而反,亡兄孝廉子修、从兄安民遇害。时余年十岁,乘马得脱。夫文武之道,各随时而用,生于中平之季,长于戎旅之间,是以少好弓马,于今不衰;逐禽辄十里,驰射常百步,日多体健,心每不厌。建安十年,始定冀州,濊、貊贡良弓,燕、代献名马。时岁之暮春,勾芒司节,和风扇物,弓燥手柔,草浅兽肥,与族兄子丹猎于邺西,终日手获□鹿九,雉兔三十。后军南征次曲蠡,尚书令荀彧奉使犒军,见余谈论之末,彧言:“闻君善左右射,此实难能。”余言:“执事未鷪夫项发口纵,俯马蹄而仰月支也。”彧喜笑曰:“乃尔!”
余曰:“埒有常径,的有常所,虽每发辄中,非至妙也。若驰平原,赴丰草,要狡兽,截轻禽,使弓不虚弯,所中必洞,斯则妙矣。”时军祭酒张京在坐,顾彧拊手曰“善”。余又学击剑,阅师多矣,四方之法各异,唯京师为善。桓、灵之间,有虎贲王越善斯术,称于京师。河南史阿言昔与越游,具得其法,余从阿学之精熟。尝与平虏将军刘勋、奋威将军邓展等共饮,宿闻展善有手臂,晓五兵,又称其能空手入白刃。余与论剑良久,谓言将军法非也,余顾尝好之,又得善术,因求与余对。时酒酣耳热,方食芊蔗,便以为杖,下殿数交,三中其臂,左右大笑。展意不平,求更为之。余言吾法急属,难相中面,故齐臂耳。展言愿复一交,余知其欲突以取交中也,因伪深进,展果寻前,余却脚鄛,正截其颡,坐中惊视。余还坐,笑曰:“昔阳庆使淳于意去其故方,更授以秘术,今余亦愿邓将军捐弃故伎,更受要道也。”一坐尽欢。夫事不可自谓己长,余少晓持复,自谓无对;俗名双戟为坐铁室,镶楯为蔽木户;后从陈国袁敏学,以单攻复,每为若神,对家不知所出,先日若逢敏于狭路,直决耳!余于他戏弄之事少所喜,唯弹澙略尽其巧,少为之赋。昔京师先工有马合乡侯、东方安世、张公子,常恨不得与彼数子者对。上雅好诗书文籍,虽在军旅,手不释卷,每每定省从容,常言人少好学则思专,长则善忘,长大而能勤学者,唯吾与袁伯业耳。
余是以少诵诗、论,及长而备历五经、四部,史、汉、诸子百家之言,靡不毕览。博物志曰:
帝善弹澙,能用手巾角。时有一书生,又能低头以所冠着葛巾角撇澙。 
\end{yuanwen}

\part{魏书三}

\chapter{明帝纪第三}

\begin{yuanwen}
明皇帝讳叡,字符仲,文帝太子也。生而太祖爱之,常令在左右。[一]年十五,封武德侯,黄初二年为齐公,三年为平原王。以其母诛,故未建为嗣。[二]七年夏五月,帝病笃,乃立为皇太子。丁巳,即皇帝位,大赦。尊皇太后曰太皇太后,皇后曰皇太后。诸臣封爵各有差。
[三]癸未,追谥母甄夫人曰文昭皇后。壬辰,立皇弟蕤为阳平王。
注[一]魏书曰:帝生数岁而有岐嶷之姿,武皇帝异之,曰:“我基于尔三世矣。”每朝宴会同,与侍中近臣并列帷幄。好学多识,特留意于法理。
注[二]魏略曰:文帝以郭后无子,诏使子养帝。帝以母不以道终,意甚不平。后不获已,乃敬事郭后,旦夕因长御问起居,郭后亦自以无子,遂加慈爱。文帝始以帝不悦,有意欲以他姬子京兆王为嗣,故久不拜太子。魏末传曰:帝常从文帝猎,见子母鹿。文帝射杀鹿母,使帝射鹿子,帝不从,曰:“陛下已杀其母,臣不忍复杀其子。”因涕泣。文帝即放弓箭,以此深奇之,而树立之意定。
注[三]世语曰:帝与朝士素不接,即位之后,髃下想闻风采。居数日,独见侍中刘晔,语尽日。众人侧听,晔既出,问“何如”?晔曰:“秦始皇、汉孝武之俦,才具微不及耳。”
八月,孙权攻江夏郡,太守文聘坚守。朝议欲发兵救之,帝曰:“权习水战,所以敢下船陆攻者,几掩不备也。今已与聘相持,夫攻守势倍,终不敢久也。”先时遣治书侍御史荀禹慰劳边方,禹到,于江夏发所经县兵及所从步骑千人乘山举火,权退走。
辛巳,立皇子冏为清河王。吴将诸葛瑾、张霸等寇襄阳,抚军大将军司马宣王讨破之,斩霸,征东大将军曹休又破其别将于寻阳。论功行赏各有差。冬十月,清河王冏薨。十二月,以太尉钟繇为太傅,征东大将军曹休为大司马,中军大将军曹真为大将军,司徒华歆为太尉,司空王朗为司徒,镇军大将军陈髃为司空,抚军大将军司马宣王为骠骑大将军。
太和元年春正月,郊祀武皇帝以配天,宗祀文皇帝于明堂以配上帝。分江夏南部,置江夏南部都尉。西平曲英反,杀临羌令、西都长,遣将军郝昭、鹿盘讨斩之。二月辛未,帝耕于籍田。辛巳,立文昭皇后寝庙于邺。丁亥,朝日于东郊。夏四月乙亥,行五铢钱。甲申,初营宗庙。秋八月,夕月于西郊。冬十月丙寅,治兵于东郊。焉耆王遣子入侍。十一月,立皇后毛氏。赐天下男子爵人二级,□寡孤独不能自存者赐谷。十二月,封后父毛嘉为列侯。新城太守孟达反,诏骠骑将军司马宣王讨之。[一]
注[一]三辅决录曰:伯郎,凉州人,名不令休。其注曰:伯郎姓孟,名他,扶风人。灵帝时。
中常侍张让专朝政,让监奴典护家事。他仕不遂,乃尽以家财赂监奴,与共结亲,积年家业为之破尽。众奴皆惭,问他所欲,他曰:“欲得卿曹拜耳。”奴被恩久,皆许诺。时宾客求见让者,门下车常数百乘,或累日不得通。他最后到,众奴伺其至,皆迎车而拜,径将他车独入。众人悉惊,谓他与让善,争以珍物遗他。他得之,尽以赂让,让大喜。他又以蒲桃酒一斛遗让,即拜凉州刺史。
他生达,少入蜀。其处蜀事夡在刘封传。魏略曰:达以延康元年率部曲四千余家归魏。文帝时初即王位,既宿知有达,闻其来,甚悦,令贵臣有识察者往观之,还曰“将帅之才也”,或曰“卿相之器也”,王益钦达。逆与达书曰:“近日有命,未足达旨,何者?昔伊挚背商而归周,百里去虞而入秦,乐毅感鸱夷以蝉蜕,王遵识逆顺以去就,皆审兴废之符效,知成败之必然,故丹青画其形容,良史载其功勋。闻卿姿度纯茂,器量优绝,当骋能明时,收名传记。今者翻然濯鳞清流,甚相嘉乐,虚心西望,依依若旧,下笔属辞,欢心从之。昔虞卿入赵,再见取相,陈平就汉,一觐参乘,孤今于卿,情过于往,故致所御马物以昭忠爱。”又曰:“今者海内清定,万里一统,三垂无边尘之警,中夏无狗吠之虞,以是弛罔阔禁,与世无疑,保官空虚,初无*(资)**[质]*任。卿来相就,当明孤意,慎勿令家人缤纷道路,以亲骇簄也。若卿欲来相见,且当先安部曲,有所保固,然后徐徐轻骑来东。”达既至谯,进见闲雅,才辩过人,众莫不属目。又王近出,乘小辇,执达手,抚其背戏之曰:“卿得无为刘备刺客邪?”遂与同载。又加拜散骑常侍,领新城太守,委以西南之任。时众臣或以为待之太猥,又不宜委以方任。王闻之曰:“吾保其无他,亦譬以蒿箭射蒿中耳。”达既为文帝所宠,又与桓阶、夏侯尚亲善,及文帝崩,时桓、尚皆卒,达自以羁旅久在疆埸,心不自安。
诸葛亮闻之,阴欲诱达,数书招之,达与相报答。魏兴太守申仪与达有隙,密表达与蜀潜通,帝未之信也。司马宣王遣参军梁几察之,又劝其入朝。达惊惧,遂反。
干宝晋纪曰:达初入新城,登白马塞,叹曰:“刘封、申耽,据金城千里而失之乎!”
二年春正月,宣王攻破新城,斩达,传其首。[一]分新城之上庸、武陵、巫县为上庸郡,锡县为锡郡。
注[一]魏略曰:宣王诱达将李辅及达甥邓贤,贤等开门纳军。达被围旬有六日而败,焚其首于洛阳四达之衢。
蜀大将诸葛亮寇边,天水、南安、安定三郡吏民叛应亮。[一]遣大将军曹真都督关右,并进兵。右将军张合击亮于街亭,大破之。亮败走,三郡平。丁未,行幸长安。[二]夏四月丁酉,还洛阳宫。[三]赦系囚非殊死以下。乙巳,论讨亮功,封爵增邑各有差。五月,大旱。六月,诏曰:“尊儒贵学,王教之本也。自顷儒官或非其人,将何以宣明圣道?其高选博士,才任侍中常侍者。申敕郡国,贡士以经学为先。”秋九月,曹休率诸军至皖,与吴将陆议战于石亭,败绩。乙酉,立皇子穆为繁阳王。庚子,大司马曹休薨。冬十月,诏公卿近臣举良将各一人。十一月,司徒王朗薨。十二月,诸葛亮围陈仓,曹真遣将军费曜等拒之。[四]辽东太守公孙恭兄子渊,劫夺恭位,遂以渊领辽东太守。
注[一]魏书曰:是时朝臣未知计所出,帝曰:“亮阻山为固,今者自来,既合兵书致人之术;
且亮贪三郡,知进而不知退,今因此时,破亮必也。”乃部勒兵马步骑五万拒亮。
注[二]魏略载帝露布天下并班告益州曰:“刘备背恩,自窜巴蜀。诸葛亮弃父母之国,阿残贼之党,神人被毒,恶积身灭。亮外慕立孤之名,而内贪专擅之实。刘升之兄弟守空城而己。亮又侮易益土,虐用其民,是以利狼、宕渠、高定、青羌莫不瓦解,为亮仇敌。而亮反裘负薪,里尽毛殚,刖趾适屦,刻肌伤骨,反更称说,自以为能。行兵于井底,游步于牛蹄。自朕即位,三边无事,犹哀怜天下数遭兵革,且欲养四海之耆老,长后生之孤幼,先移风于礼乐,次讲武于农隙,置亮画外,未以为虞。而亮怀李熊愚勇之*(智)*[志],不思荆邯度德之戒,驱略吏民,盗利祁山。
王师方振,胆破气夺,马谡、高祥,望旗奔败。虎臣逐北,蹈尸涉血,亮也小子,震惊朕师。
猛锐踊跃,咸思长驱。朕惟率土莫非王臣,师之所处,荆棘生焉,不欲使千室之邑忠信贞良,与夫淫昏之党,共受涂炭。故先开示,以昭国诚,勉思变化,无滞乱邦。巴蜀将吏士民诸为亮所劫迫,公卿已下皆听束手。”
注[三]魏略曰:是时斗言,云帝已崩,从驾髃臣迎立雍丘王植。京师自卞太后髃公尽惧。及帝还,皆私察颜色。卞太后悲喜,欲推始言者,帝曰:“天下皆言,将何所推?”
注[四]魏略曰:先是,使将军郝昭筑陈仓城;会亮至,围昭,不能拔。昭字伯道,太原人,为人雄壮,少入军为部曲督,数有战功,为杂号将军,遂镇守河西十余年,民夷畏服。亮围陈仓,使昭乡人靳详于城外遥说之,昭于楼上应详曰:“魏家科法,卿所练也;我之为人,卿所知也。我受国恩多而门户重,卿无可言者,但有必死耳。卿还谢诸葛,便可攻也。”详以昭语告亮,亮又使详重说昭,言人兵不敌,无为空自破灭。昭谓详曰:“前言已定矣。我识卿耳,箭不识也。”详乃去。亮自以有众数万,而昭兵纔千余人,又度东救未能便到,乃进兵攻昭,起云梯冲车以临城。昭于是以火箭逆射其云梯,梯然,梯上人皆烧死。昭又以绳连石磨压其冲车,冲车折。亮乃更为井阑百尺以射城中,以土丸填堑,欲直攀城,昭又于内筑重墙。亮又为地突,欲踊出于城里,昭又于城内穿地横截之。昼夜相攻拒二十余日,亮无计,救至,引退。诏嘉昭善守,赐爵列侯。及还,帝引见慰劳之,顾谓中书令孙资曰:“卿乡里乃有尔曹快人,为将灼如此,朕复何忧乎?”仍欲大用之。会病亡,遗令戒其子凯曰:“吾为将,知将不可为也。吾数发冢,取其木以为攻战具,又知厚葬无益于死者也。汝必敛以时服。且人生有处所耳,死复何在耶?今去本墓远,东西南北,在汝而已。”
三年夏四月,元城王礼薨。六月癸卯,繁阳王穆薨。戊申,追尊高祖大长秋曰高皇帝,夫人吴氏曰高皇后。
秋七月,诏曰:“礼,王后无嗣,择建支子以继大宗,则当纂正统而奉公义,何得复顾私亲哉!汉宣继昭帝后,加悼考以皇号;哀帝以外藩援立,而董宏等称引亡秦,惑误时朝,既尊恭皇,立庙京都,又宠藩妾,使比长信,□昭穆于前殿,并四位于东宫,僭差无度,人神弗佑,而非罪师丹忠正之谏,用致丁、傅焚如之祸。自是之后,相踵行之。昔鲁文逆祀,罪由夏父;宋国非度,讥在华元。其令公卿有司,深以前世行事为戒。后嗣万一有由诸侯入奉大统,则当明为人后之义;敢为佞邪导谀时君,妄建非正之号以干正统,谓考为皇,称妣为后,则股肱大臣,诛之无赦。其书之金策,藏之宗庙,着于令典。”
冬十月,改平望观曰听讼观。帝常言“狱者,天下之性命也”,每断大狱,常幸观临听之。
初,洛阳宗庙未成,神主在邺庙。十一月,庙始成,使太常韩暨持节迎高皇帝、太皇帝、武帝、文帝神主于邺,十二月己丑至,奉安神主于庙。[一]
注[一]臣松之按:黄初四年,有司奏立二庙,太皇帝大长秋与文帝之高祖共一庙,特立武帝庙,百世不毁。今此无高祖神主,盖以亲尽毁也。此则魏初唯立亲庙,祀四室而已。至景初元年,始定七庙之制。孙盛曰:事亡犹存,祭如神在,迎迁神主,正斯宜矣。
癸卯,大月氏王波调遣使奉献,以调为亲魏大月氏王。
四年春二月壬午,诏曰:“世之质文,随教而变。兵乱以来,经学废绝,后生进趣,不由典谟。岂训导未洽,将进用者不以德显乎?其郎吏学通一经,才任牧民,博士课试,擢其高第者,亟用;其浮华不务道本者,皆罢退之。”戊子,诏太傅三公:以文帝典论刻石,立于庙门之外。癸巳,以大将军曹真为大司马,骠骑将军司马宣王为大将军,辽东太守公孙渊为车骑将军。夏四月,太傅钟繇薨。六月戊子,太皇太后崩。丙申,省上庸郡。秋七月,武宣卞后祔葬于高陵。诏大司马曹真、大将军司马宣王伐蜀。八月辛巳,行东巡,遣使者以特牛祠中岳。[一]乙未,幸许昌宫。九月,大雨,伊、洛、河、汉水溢,诏真等班师。冬十月乙卯,行还洛阳宫。庚申,令:“罪非殊死听赎各有差。”十一月,太白犯岁星。十二月辛未,改葬文昭甄后于朝阳陵。丙寅,诏公卿举贤良。
注[一]魏书曰:行过繁昌,使执金吾臧霸行太尉事,以特牛祠受禅坛。
臣松之按:汉纪章帝元和三年,诏高邑县祠即位坛,五成陌,比腊祠门户。此虽前代已行故事,然为坛以祀天,而坛非神也,今无事于上帝,而致祀于虚坛,求之义典,未详所据。
五年春正月,帝耕于籍田。三月,大司马曹真薨。诸葛亮寇天水,诏大将军司马宣王拒之。
自去冬十月至此月不雨,辛巳,大雩。夏四月,鲜卑附义王轲比能率其种人及丁零大人儿禅诣幽州贡名马。复置护匈奴中郎将。秋七月丙子,以亮退走,封爵增位各有差。[一]乙酉,皇子殷生,大赦。
注[一]魏书曰:初,亮出,议者以为亮军无辎重,粮必不继,不击自破,无为劳兵;或欲自芟上邽左右生麦以夺贼食,帝皆不从。前后遣兵增宣王军,又敕使护麦。宣王与亮相持,赖得此麦以为军粮。
八月,诏曰:“古者诸侯朝聘,所以敦睦亲亲协和万国也。先帝着令,不欲使诸王在京都者,谓幼主在位,母后摄政,防微以渐,关诸盛衰也。朕惟不见诸王十有二载,悠悠之怀,能不兴思!其令诸王及宗室公侯各将适子一人朝。后有少主、母后在宫者,自如先帝令,申明着于令。”冬十一月乙酉,月犯轩辕大星。戊戌晦,日有蚀之。十二月甲辰,月犯镇星。戊午,太尉华歆薨。
六年春二月,诏曰:“古之帝王,封建诸侯,所以藩屏王室也。诗不云乎,‘怀德维宁,宗子维城’。秦、汉继周,或强或弱,俱失厥中。大魏创业,诸王开国,随时之宜,未有定制,非所以永为后法也。其改封诸侯王,皆以郡为国。”三月癸酉,行东巡,所过存问高年□寡孤独,赐谷帛。乙亥,月犯轩辕大星。夏四月壬寅,行幸许昌宫。甲子,初进新果于庙。五月,皇子殷薨,追封谥安平哀王。秋七月,以韂尉董昭为司徒。九月,行幸摩陂,治许昌宫,起景福、承光殿。冬十月,殄夷将军田豫帅众讨吴将周贺于成山,杀贺。十一月丙寅,太白昼见。有星孛于翼,近太微上将星。庚寅,陈思王植薨。十二月,行还许昌宫。
青龙元年春正月甲申,青龙见郏之摩陂井中。二月丁酉,幸摩陂观龙,于是改年;改摩陂为龙陂,赐男子爵人二级,□寡孤独无出今年租赋。三月甲子,诏公卿举贤良笃行之士各一人。
夏五月壬申,诏祀故大将军夏侯惇、大司马曹仁、车骑将军程昱于太祖庙庭。[一]戊寅,北海王蕤薨。闰月庚寅朔,日有蚀之。丁酉,改封宗室女非诸王女皆为邑主。诏诸郡国山川不在祠典者勿祠。六月,洛阳宫鞠室灾。
注[一]魏书载诏曰:“昔先王之礼,于功臣存则显其爵禄,没则祭于大蒸,故汉氏功臣,祀于庙庭。大魏元功之臣功勋优着,终始休明者,其皆依礼祀之。”于是以惇等配飨。
保塞鲜卑大人步度根与叛鲜卑大人轲比能私通,并州刺史毕轨表,辄出军以外威比能,内镇步度根。帝省表曰:“步度根以为比能所诱,有自疑心。今轨出军,适使二部惊合为一,何所威镇乎?”促敕轨,以出军者慎勿越塞过句注也。比诏书到,轨以进军屯阴馆,遣将军苏尚、董弼追鲜卑。比能遣子将千余骑迎步度根部落,与尚、弼相遇,战于楼烦,二将*[败]*没。步度根部落皆叛出塞,与比能合寇边。遣骁骑将军秦朗将中军讨之,虏乃走漠北。
秋九月,安定保塞匈奴大人胡薄居姿职等叛,司马宣王遣将军胡遵等追讨,破降之。
冬十月,步度根部落大人戴胡阿狼泥等诣并州降,朗引军还。[一]
注[一]魏氏春秋曰:朗字符明,新兴人。献帝传曰:朗父名宜禄,为吕布使诣袁术,术妻以汉宗室女。其前妻杜氏留下邳。布之被围,关羽屡请于太祖,求以杜氏为妻,太祖疑其有色,及城陷,太祖见之,乃自纳之。宜禄归降,以为铚长。及刘备走小沛,张飞随之,过谓宜禄曰:“人取汝妻,而为之长,乃蚩蚩若是邪!随我去乎?”宜禄从之数里,悔欲还,飞杀之。
朗随母氏畜于公宫,太祖甚爱之,每坐席,谓宾客曰:“世有人爱假子如孤者乎?”魏略曰:
朗游遨诸侯间,历武、文之世而无尤也。及明帝即位,授以内官,为骁骑将军、给事中,每车驾出入,朗常随从。时明帝喜发举,数有以轻微而致大辟者,朗终不能有所谏止,又未尝进一善人,帝亦以是亲爱;每顾问之,多呼其小字阿稣,数加赏赐,为起大第于京城中。四方虽知朗无能为益,犹以附近至尊,多赂遗之,富均公侯。世语曰:朗子秀,劲厉能直言,为晋武帝博士。魏略以朗与孔桂俱在佞幸篇。桂字叔林,天水人也。建安初,数为将军杨秋使诣太祖,太祖表拜骑都尉。桂性便辟,晓博弈、□鞠,故太祖爱之,每在左右,出入随从。
桂察太祖意,喜乐之时,因言次曲有所陈,事多见从,数得赏赐,人多馈遗,桂由此侯服玉食。太祖既爱桂,五官将及诸侯亦皆亲之。其后桂见太祖久不立太子,而有意于临菑侯,因更亲附临菑侯而简于五官将,将甚衔之。及太祖薨,文帝即王位,未及致其罪。黄初元年,随例转拜驸马都尉。而桂私受西域货赂,许为人事。事发,有诏收问,遂杀之。鱼豢曰:为上者不虚授,处下者不虚受,然后外无伐檀之叹,内无尸素之刺,雍熙之美着,太平之律显矣。而佞幸之徒,但姑息人主,至乃无德而荣,无功而禄,如是焉得不使中正日朘,倾邪滋多乎!以武皇帝之慎赏,明皇帝之持法,而犹有若此等人,而况下斯者乎?
十二月,公孙渊斩送孙权所遣使张弥、许晏首,以渊为大司马乐浪公。[一]
注[一]世语曰:并州刺史毕轨送汉故度辽将军范明友鲜卑奴,年三百五十岁,言语饮食如常人。奴云:“霍显,光后小妻。明友妻,光前妻女。”博物志曰:时京邑有一人,失其姓名,食啖兼十许人,遂肥不能动。其父曾作远方长吏,官徙送彼县,令故义传供食之;一二年中,一乡中辄为之俭。傅子曰:时太原发頉破棺,棺中有一生妇人,将出与语,生人也。送之京师,问其本事,不知也。视其頉上树木可三十岁,不知此妇人三十岁常生于地中邪?将一朝欻生,偶与发頉者会也?
二年春二月乙未,太白犯荧惑。癸酉,诏曰:“鞭作官刑,所以纠慢怠也,而顷多以无辜死。
其减鞭杖之制,着于令。”三月庚寅,山阳公薨,帝素服发哀,遣使持节典护丧事。己酉,大赦。夏四月,大疫。崇华殿灾。丙寅,诏有司以太牢告祠文帝庙。追谥山阳公为汉孝献皇帝,葬以汉礼。[一]
注[一]献帝传曰:帝变服,率髃臣哭之,使使持节行司徒太常和洽吊祭,又使持节行大司空大司农崔林监护丧事。诏曰:“盖五帝之事尚矣,仲尼盛称尧、舜巍巍荡荡之功者,以为禅代乃大圣之懿事也。山阳公深识天禄永终之运,禅位文皇帝以顺天命。先帝命公行汉正朔,郊天祀祖以天子之礼,言事不称臣,此舜事尧之义也。昔放勋殂落,四海如丧考妣,遏密八音,明丧葬之礼同于王者也。今有司奏丧礼比诸侯王,此岂古之遗制而先帝之至意哉?今谥公汉孝献皇帝。”使太尉具以一太牢告祠文帝庙,曰:“叡闻夫礼也者,反本修古,不忘厥初,是以先代之君,尊尊亲亲,咸有尚焉。今山阳公寝疾弃国,有司建言丧纪之礼视诸侯王。
叡惟山阳公昔知天命永终于己,深观历数允在圣躬,传祚禅位,尊我民主,斯乃陶唐懿德之事也。黄初受终,命公于国行汉正朔,郊天祀祖礼乐制度率乃汉旧,斯亦舜、禹明堂之义也。
上考遂初,皇极攸建,允熙克让,莫朗于兹。盖子以继志嗣训为孝,臣以配命钦述为忠,故诗称‘匪棘其犹,聿追来孝’,书曰‘前人受命,兹不忘大功’。叡敢不奉承徽典,以昭皇考之神灵。今追谥山阳公曰孝献皇帝,册赠玺绂。命司徒、司空持节吊祭护丧,光禄、大鸿胪为副,将作大匠、复土将军营成陵墓,及置百官髃吏,车旗服章丧葬礼仪,一如汉氏故事;
丧葬所供髃官之费,皆仰大司农。立其后嗣为山阳公,以通三统,永为魏宾。”于是赠册曰:
“呜呼,昔皇天降戾于汉,俾逆臣董卓,播厥凶虐,焚灭京都,劫迁大驾。于时六合云扰,奸雄熛起。帝自西京,徂唯求定,臻兹洛邑。畴咨圣贤,聿改乘辕,又迁许昌,武皇帝是依。
岁在玄枵,皇师肇征,迄于鹑尾,十有八载,髃寇歼殄,九域咸乂。惟帝念功,祚兹魏国,大启土宇。爰及文皇帝,齐圣广渊,仁声旁流,柔远能迩,殊俗向义,干精承祚,坤灵吐曜,稽极玉衡,允膺历数,度于轨仪,克厌帝心。乃仰钦七政,俯察五典,弗采四岳之谋,不俟师锡之举,幽赞神明,承天禅位。祚*(建)**[逮]*朕躬,统承洪业。盖闻昔帝尧,元恺既举,凶族未流,登舜百揆,然后百揆时序,内平外成,授位明堂,退终天禄,故能冠德百王,表功嵩岳。自往迄今,弥历七代,岁暨三千,而大运来复,庸命厎绩,纂我民主,作建皇极。念重光,绍咸池,继韶夏,超群后之遐踪,邈商、周之惭德,可谓高朗令终,昭明洪烈之懿盛者矣。非夫汉、魏与天地合德,与四时合信,动和民神,格于上下,其孰能至于此乎?朕惟孝献享年不永,钦若顾命,考之典谟,恭述皇考先灵遗意,阐崇弘谥,奉成圣美,以章希世同符之隆,以传亿载不朽之荣。魂而有灵,嘉兹弘休。呜呼哀哉!”八月壬申,葬于山阳国,陵曰禅陵,置园邑。葬之日,帝制锡衰弁绖,哭之恸。适孙桂氏乡侯康,嗣立为山阳公。
是月,诸葛亮出斜谷,屯渭南,司马宣王率诸军拒之。诏宣王:“但坚壁拒守以挫其锋,彼进不得志,退无与战,久停则粮尽,虏略无所获,则必走矣。走而追之,以逸待劳,全胜之道也。”[一]
注[一]魏氏春秋曰:亮既屡遣使交书,又致巾帼妇人之饰,以怒宣王。宣王将出战,辛毗杖节奉诏,勒宣王及军吏已下,乃止。宣王见亮使,唯问其寝食及其事之烦简,不问戎事。使对曰:“诸葛公夙兴夜寐,罚二十已上,皆亲览焉;所啖食不过数升。”宣王曰:“亮体毙矣,其能久乎?”
五月,太白昼见。孙权入居巢湖口,向合肥新城,又遣将陆议、孙韶各将万余人入淮、沔。
六月,征东将军满宠进军拒之。宠欲拔新城守,致贼寿春,帝不听,曰:“昔汉光武遣兵县据略阳,终以破隗嚣,先帝东置合肥,南守襄阳,西固祁山,贼来辄破于三城之下者,地有所必争也。纵权攻新城,必不能拔。敕诸将坚守,吾将自往征之,比至,恐权走也。”秋七月壬寅,帝亲御龙舟东征,权攻新城,将军张颖等拒守力战,帝军未至数百里,权遁走,议、韶等亦退。髃臣以为大将军方与诸葛亮相持未解,车驾可西幸长安。帝曰:“权走,亮胆破,大将军以制之,吾无忧矣。”遂进军幸寿春,录诸将功,封赏各有差。八月己未,大曜兵,飨六军,遣使者持节犒劳合肥、寿春诸军。辛巳,行还许昌宫。
司马宣王与亮相持,连围积日,亮数挑战,宣王坚垒不应。会亮卒,其军退还。
冬十月乙丑,月犯镇星及轩辕。戊寅,月犯太白。十一月,京都地震,从东南来,隐隐有声,摇动屋瓦。十二月,诏有司删定大辟,减死罪。
三年春正月戊子,以大将军司马宣王为太尉。己亥,复置朔方郡。京都大疫。丁巳,皇太后崩。乙亥,陨石于寿光县。三月庚寅,葬文德郭后,营陵于首阳陵涧西,如终制。[一]
注[一]顾恺之启蒙注曰:魏时人有开周王頉者,得殉葬女子,经数日而有气,数月而能语;
年可二十。送诣京师,郭太后爱养之。十余年,太后崩,哀思哭泣,一年余而死。
是时,大治洛阳宫,起昭阳、太极殿,筑总章观。百姓失农时,直臣杨阜、高堂隆等各数切谏,虽不能听,常优容之。[一]
注[一]魏略曰:是年起太极诸殿,筑总章观,高十余丈,建翔凤于其上;又于芳林园中起陂池,楫棹越歌;又于列殿之北,立八坊,诸才人以次序处其中,贵人夫人以上,转南附焉,其秩石拟百官之数。帝常游宴在内,乃选女子知书可付信者六人,以为女尚书,使典省外奏事,处当画可,自贵人以下至尚保,及给掖庭洒扫,习伎歌者,各有千数。通引谷水过九龙殿前,为玉井绮栏,蟾蜍含受,神龙吐出。使博士马均作司南车,水转百戏。岁首建巨兽,鱼龙曼延,弄马倒骑,备如汉西京之制,筑阊阖诸门阙外罘罳。太子舍人张茂以吴、蜀数动,诸将出征,而帝盛兴宫室,留意于玩饰,赐与无度,帑藏空竭;又录夺士女前已嫁为吏民妻者,还以配士,既听以生口自赎,又简选其有姿色者内之掖庭,乃上书谏曰:“臣伏见诏书,诸士女嫁非士者,一切录夺,以配战士,斯诚权时之宜,然非大化之善者也。臣请论之。陛下,天之子也,百姓吏民,亦陛下之子也。礼,赐君子小人不同日,所以殊贵贱也。吏属君子,士为小人,今夺彼以与此,亦无以异于夺兄之妻妻弟也,于父母之恩偏矣。又诏书听得以生口年纪、颜色与妻相当者自代,故富者则倾家尽产,贫者举假贷贳,贵买生口以赎其妻;县官以配士为名而实内之掖庭,其丑恶者乃出与士。得妇者未必有欢心,而失妻者必有忧色,或穷或愁,皆不得志。夫君有天下而不得万姓之欢心者,寭不危殆。且军师在外数千万人,一日之费非徒千金,举天下之赋以奉此役,犹将不给,况复有宫庭非员无录之女,椒房母后之家,赏赐横兴,内外交引,其费半军。昔汉武帝好神仙,信方士,掘地为海,封土为山,赖是时天下为一,莫敢与争者耳。自衰乱以来,四五十载,马不舍鞍,士不释甲,每一交战,血流丹野,创痍号痛之声,于今未已。犹强寇在疆,图危魏室。陛下不兢兢业业,念崇节约,思所以安天下者,而乃奢靡是务,中尚方纯作玩弄之物,炫耀后园,建承露之盘,斯诚快耳目之观,然亦足以骋寇绚之心矣。惜乎,舍尧舜之节俭,而为汉武之侈事,臣窃为陛下不取也。愿陛下沛然下诏,万几之事有无益而有损者悉除去之,以所除无益之费,厚赐将士父母妻子之饥寒者,问民所疾而除其所恶,实仓廪,缮甲兵,恪恭以临天下。如是,吴贼面缚,蜀虏舆榇,不待诛而自服,太平之路可计日而待也。陛下可无劳神思于海表,军师高枕,战士备员。今髃公皆结舌,而臣所以不敢不献瞽言者,臣昔上要言,散骑奏臣书,以听谏篇为善,诏曰:‘是也’,擢臣为太子舍人;
且臣作书讥为人臣不能谏诤,今有可谏之事而臣不谏,此为作书虚妄而不能言也。臣年五十,常恐至死无以报国,是以投躯没命,冒昧以闻,惟陛下裁察。”书通,上顾左右曰:“张茂恃乡里故也。”以事付散骑而已。茂字彦林,沛人。
秋七月,洛阳崇华殿灾,八月庚午,立皇子芳为齐王,询为秦王。丁巳,行还洛阳宫。命有司复崇华,改名九龙殿。冬十月己酉,中山王兖薨。壬申,太白昼见。十一月丁酉,行幸许昌宫。[一]
注[一]魏氏春秋曰:是岁张掖郡删丹县金山玄川溢涌,宝石负图,状象灵龟,广一丈六尺,长一丈七尺一寸,围五丈八寸,立于川西。有石马七,其一仙人骑之,其一羁绊,其五有形而不善成。有玉匣关盖于前,上有玉字,玉玦二,璜一。麒麟在东,凤鸟在南,白虎在西,牺牛在北,马自中布列四面,色皆苍白。其南有五字,曰“上上三天王”;又曰“述大金,大讨曹,金但取之,金立中,大金马一匹在中,大*(告)**[吉]*开寿,此马甲寅述水”。凡“中”字六,“金”字十;又有若八卦及列宿孛彗之象焉。世语曰:又有一鸡象。搜神记曰:
初,汉元、成之世,先识之士有言曰,魏年有和,当有开石于西三千余里,系五马,文曰“大讨曹”。及魏之初兴也,张掖之柳谷,有开石焉,始见于建安,形成于黄初,文备于太和,周围七寻,中高一仞,苍质素章,龙马、麟鹿、凤皇、仙人之象,粲然咸着,此一事者,魏、晋代兴之符也。至晋泰始三年,张掖太守焦胜上言,以留郡本国图校今石文,文字多少不同,谨具图上。按其文有五马象,其一有人平上帻,执戟而乘之,其一有若马形而不成,其字有“金”,有“中”,有“大司马”,有“王”,有“大吉”,有“正”,有“开寿”,其一成行,曰“金当取之”。汉晋春秋曰:氐池县大柳谷口夜激波涌溢,其声如雷,晓而有苍石立水中,长一丈六尺,高八尺,白石画之,为十三马,一牛,一鸟,八卦玉玦之象,皆隆起,其文曰“大讨曹,适水中,甲寅”。帝恶其“讨”也,使凿去为“计”,以苍石窒之,宿昔而白石满焉。至晋初,其文愈明,马象皆焕彻如玉焉。
四年春二月,太白复昼见,月犯太白,又犯轩辕一星,入太微而出。夏四月,置崇文观,征善属文者以充之。五月乙卯,司徒董昭薨。丁巳,肃慎氏献楛矢。
六月壬申,诏曰:“有虞氏画象而民弗犯,周人刑错而不用。朕从百王之末,追望上世之风,邈乎何相去之远?法令滋章,犯者弥多,刑罚愈众,而奸不可止。往者按大辟之条,多所蠲除,思济生民之命,此朕之至意也。而郡国毙狱,一岁之中尚过数百,岂朕训导不醇,俾民轻罪,将苛法犹存,为之陷藊乎?有司其议狱缓死,务从宽简,及乞恩者,或辞未出而狱以报断,非所以究理尽情也。其令廷尉及天下狱官,诸有死罪具狱以定,非谋反及手杀人,亟语其亲治,有乞恩者,使与奏当文书俱上,朕将思所以全之。其布告天下,使明朕意。”
秋七月,高句骊王宫斩送孙权使胡韂等首,诣幽州。甲寅,太白犯轩辕大星。冬十月己卯,行还洛阳宫。甲申,有星孛于大辰,乙酉,又孛于东方。十一月己亥,彗星见,犯宦者天纪星。十二月癸巳,司空陈髃薨。乙未,行幸许昌宫。
景初元年春正月壬辰,山茌县言黄龙见。*茌音仕狸反。*于是有司奏,以为魏得地统,宜以建丑之月为正。三月,定历改年为孟夏四月。[一]服色尚黄,牺牲用白,戎事乘黑首白马,建大赤之旗,朝会建大白之旗。[二]改太和历曰景初历。其春夏秋冬孟仲季月虽与正岁不同,至于郊祀、迎气、礿祠、蒸尝、巡狩、搜田、分至启闭、班宣时令、中气早晚、敬授民事,皆以正岁斗建为历数之序。
注[一]魏书曰:初,文皇帝即位,以受禅于汉,因循汉正朔弗改。帝在东宫着论,以为五帝三王虽同气共祖,礼不相袭,正朔自宜改变,以明受命之运。及即位,优游者久之,史官复着言宜改,乃诏三公、特进、九卿、中郎将、大夫、博士、议郎、千石、六百石博议,议者或不同。帝据古典,甲子诏曰:“夫太极运三辰五星于上,元气转三统五行于下,登降周旋,终则又始。故仲尼作春秋,于三微之月,每月称王,以明三正迭相为首。今推三统之次,魏得地统,当以建丑之月为正月。考之髃艺,厥义章矣。其改青龙五年三月为景初元年四月。”
注[二]臣松之按:魏为土行,故服色尚黄。行殷之时,以建丑为正,故牺牲旗旗一用殷礼。
礼记云:“夏后氏尚黑,故戎事乘骊,牲用玄;殷人尚白,戎事乘翰,牲用白;周人尚赤,戎事乘騵,牲用骍。”郑玄云:“夏后氏以建寅为正,物生色黑;殷以建丑为正,物牙色白;
周以建子为正,物萌色赤。翰,白色马也,易曰‘白马翰如’。”周礼巾车职“建大赤以朝”,大白以即戎,此则周以正色之旗以朝,先代之旗即戎。今魏用殷礼,变周之制,故建大白以朝,大赤即戎。
五月己巳,行还洛阳宫。己丑,大赦。六月戊申,京都地震。己亥,以尚书令陈矫为司徒,尚书*(左)**[右]*仆射韂臻为司空。丁未,分魏兴之魏阳、锡郡之安富、上庸为上庸郡。省锡郡,以锡县属魏兴郡。
有司奏:武皇帝拨乱反正,为魏太祖,乐用武始之舞。文皇帝应天受命,为魏高祖,乐用咸熙之舞。帝制作兴治,为魏烈祖,乐用章*(武)**[斌]*之舞。三祖之庙,万世不毁。其余四庙,亲尽迭毁,如周后稷、文、武庙祧之制。[一]
注[一]孙盛曰:夫谥以表行,庙以存容,皆于既没然后着焉,所以原始要终,以示百世也。
未有当年而逆制祖宗,未终而豫自尊显。昔华乐以厚敛致讥,周人以豫凶违礼,魏之髃司,于是乎失正。
秋七月丁卯,司徒陈矫薨。孙权遣将朱然等二万人围江夏郡,荆州刺史胡质等击之,然退走。
初,权遣使浮海与高句骊通,欲袭辽东。遣幽州刺史□丘俭率诸军及鲜卑、乌丸屯辽东南界,玺书征公孙渊。渊发兵反,俭进军讨之,会连雨十日,辽水大涨,诏俭引军还。右北平乌丸单于寇娄敦、辽西乌丸都督王护留等居辽东,率部众随俭内附。己卯,诏辽东将吏士民为渊所胁略不得降者,一切赦之。辛卯,太白昼见。渊自俭还,遂自立为燕王,置百官,称绍汉元年。
诏青、兖、幽、冀四州大作海船。九月,冀、兖、徐、豫四州民遇水,遣侍御史循行没溺死亡及失财产者,在所开仓振救之。庚辰,皇后毛氏卒。冬十月丁未,月犯荧惑。癸丑,葬悼毛后于愍陵。乙卯,营洛阳南委粟山为圜丘。[一]十二月壬子冬至,始祀。丁巳,分襄阳临沮、宜城、旍阳、邔*邔音其己反。*四县,置襄阳南部都尉。己未,有司奏文昭皇后立庙京都。分襄阳郡之鄀叶县属义阳郡。[二]
注[一]魏书载诏曰:“盖帝王受命,莫不恭承天地以章神明,尊祀世统以昭功德,故先代之典既着,则禘郊祖宗之制备也。昔汉氏之初,承秦灭学之后,采摭残缺,以备郊祀,自甘泉后土、雍宫五畤,神只兆位,多不见经,是以制度无常,一彼一此,四百余年,废无禘祀。
古代之所更立者,遂有阙焉。曹氏系世,出自有虞氏,今祀圜丘,以始祖帝舜配,号圜丘曰皇皇帝天;方丘所祭曰皇皇后地,以舜妃伊氏配;天郊所祭曰皇天之神,以太祖武皇帝配;
地郊所祭曰皇地之只,以武宣后配;宗祀皇考高祖文皇帝于明堂,以配上帝。”至晋泰始二年,并圜丘、方丘二至之祀于南北郊。
注[二]魏略曰:是岁,徙长安诸钟懬、骆驼、铜人、承露盘。盘折,铜人重不可致,留于霸城。大发铜铸作铜人二,号曰翁仲,列坐于司马门外。又铸黄龙、凤皇各一,龙高四丈,凤高三丈余,置内殿前。起土山于芳林园西北陬,使公卿髃僚皆负土成山,树松竹杂木善草于其上,捕山禽杂兽置其中。汉晋春秋曰:帝徙盘,盘折,声闻数十里,金狄或泣,因留霸城。
魏略载司徒军议掾河东董寻上书谏曰:“臣闻古之直士,尽言于国,不避死亡。故周昌比高祖于桀、纣,刘辅譬赵后于人婢。天生忠直,虽白刃沸汤,往而不顾者,诚为时主爱惜天下也。建安以来,野战死亡,或门殚户尽,虽有存者,遗孤老弱。若今宫室狭小,当广大之,犹宜随时,不妨农务,况乃作无益之物,黄龙、凤皇,九龙、承露盘,土山、渊池,此皆圣明之所不兴也,其功参倍于殿舍。三公九卿侍中尚书,天下至德,皆知非道而不敢言者,以陛下春秋方刚,心畏雷霆。今陛下既尊群臣,显以冠冕,被以文绣,载以华舆,所以异于小人;而使穿方举土,面目垢黑,沾体涂足,衣冠了鸟,毁国之光以崇无益,甚非谓也。孔子曰:‘君使臣以礼,臣事君以忠。’无忠无礼,国何以立!故有君不君,臣不臣,上下不通,心怀郁结,使阴阳不和,灾害屡降,凶恶之徒,因间而起,谁当为陛下尽言事者乎?又谁当干万乘以死为戏乎?臣知言出必死,而臣自比于牛之一毛,生既无益,死亦何损?秉笔流涕,心与世辞。臣有八子,臣死之后,累陛下矣!”
将奏,沐浴。既通,帝曰:“董寻不畏死邪!”主者奏收寻,有诏勿问。后为贝丘令,清省得民心。
二年春正月,诏太尉司马宣王帅众讨辽东。[一]
注[一]干窦晋纪曰:帝问宣王:“度公孙渊将何计以待君?”宣王对曰:“渊弃城预走,上计也;据辽水拒大军,其次也;坐守襄平,此为成禽耳。”帝曰:“然则三者何出?”对曰:“唯明智审量彼我,乃预有所割弃,此既非渊所及,又谓今往县远,不能持久,必先拒辽水,后守也。”帝曰:“住还几日?”对曰:“往百日,攻百日;还百日,以六十日为休息,如此,一年足矣。”魏名臣奏载散骑常侍何曾表曰:“臣闻先王制法,必于全慎,故建官授任,则置假辅,陈师命将,则立监贰,宣命遣使,则设介副,临敌交刃,则参御右,盖以尽谋思之功,防安危之变也。是以在险当难,则权足相济,陨缺不预,则才足相代,其为固防,至深至远。及至汉氏,亦循旧章。韩信伐赵,张耳为贰;马援讨越,刘隆副军。前世之夡,着在篇志。今懿奉辞诛罪,步骑数万,道路回阻,四千余里,虽假天威,有征无战,寇或潜遁,消散日月,命无常期。人非金石,远虑详备,诚宜有副。今北边诸将及懿所督,皆为僚属,名位不殊,素无定分,卒有变急,不相镇摄。存不忘亡,圣达所戒,宜选大臣名将威重宿著者,盛其礼秩,遣诣懿军,进同谋略,退为副佐。虽有万一不虞之灾,军主有储,则无患矣。”□丘俭志记云,时以俭为宣王副也。
二月癸卯,以大中大夫韩暨为司徒。癸丑,月犯心距星,又犯心中央大星。夏四月庚子,司徒韩暨薨。壬寅,分沛国萧、相、竹邑、符离、蕲、铚、龙亢、山桑、洨、虹*洨音胡交反。
虹音绛。*十县为汝阴郡。宋县、陈郡苦县皆属谯郡。以沛、杼秋、公丘、彭城丰国、广戚,并五县为沛王国。庚戌,大赦。五月乙亥,月犯心距星,又犯中央大星。[一]六月,省渔阳郡之狐奴县,复置安乐县。
注[一]魏书载戊子诏曰:“昔汉高祖创业,光武中兴,谋除残暴,功昭四海,而坟陵崩颓,童儿牧竖践蹈其上,非大魏尊崇所承代之意也。其表高祖、光武陵四面百步,不得使民耕牧樵采。”
秋八月,烧当羌王芒中、注诣等叛,凉州刺史率诸郡攻讨,斩注诣首。癸丑,有彗星见张宿。
[一]
注[一]汉晋春秋曰:史官言于帝曰:“此周之分野也,洛邑恶之。”于是大修禳祷之术以厌焉。魏书曰:九月,蜀阴平太守廖惇反,攻守善羌侯宕蕈营。雍州刺史郭淮遣广魏太守王赟、南安太守游奕将兵讨惇。淮上书:“赟、奕等分兵夹山东西,围落贼表,破在旦夕。”帝曰:
“兵势恶离。”促诏淮敕奕诸别营非要处者,还令据便地。诏敕未到,奕军为惇所破;赟为流矢所中死。
丙寅,司马宣王围公孙渊于襄平,大破之,传渊首于京都,海东诸郡平。冬十一月,录讨渊功,太尉宣王以下增邑封爵各有差。初,帝议遣宣王讨渊,发卒四万人。议臣皆以为四万兵多,役费难供。帝曰:“四千里征伐,虽云用奇,亦当任力,不当稍计役费。”遂以四万人行。及宣王至辽东,霖雨不得时攻,髃臣或以为渊未可卒破,宜诏宣王还。帝曰:“司马懿临危制变,擒渊可计日待也。”卒皆如所策。
壬午,以司空韂臻为司徒,司隶校尉崔林为司空。闰月,月犯心中央大星。十二月乙丑,帝寝疾不豫。辛巳,立皇后。赐天下男子爵人二级,□寡孤独谷。以燕王宇为大将军,甲申免,以武韂将军曹爽代之。[一]
注[一]汉晋春秋曰:帝以燕王宇为大将军,使与领军将军夏侯献、武韂将军曹爽、屯骑校尉曹肇、骁骑将军秦朗等对辅政。中书监刘放、令孙资久专权宠,为朗等素所不善,惧有后害,阴图间之,而宇常在帝侧,故未得有言。甲申,帝气微,宇下殿呼曹肇有所议,未还,而帝少闲,惟曹爽独在。放知之,呼资与谋。资曰:“不可动也。”放曰:“俱入鼎镬,何不可之有?”乃突前见帝,垂泣曰:“陛下气微,若有不讳,将以天下付谁?”帝曰:“卿不闻用燕王耶?”放曰:“陛下忘先帝诏敕,藩王不得辅政。且陛下方病,而曹肇、秦朗等便与才人侍疾者言戏。燕王拥兵南面,不听臣等入,此即竖刁、赵高也。今皇太子幼弱,未能统政,外有强暴之寇,内有劳怨之民,陛下不远虑存亡,而近系恩旧。委祖宗之业,付二三凡士,寝疾数日,外内壅隔,社稷危殆,而己不知,此臣等所以痛心也。”帝得放言,大怒曰:“谁可任者?”放、资乃举爽代宇,又白“宜诏司马宣王使相参”,帝从之。放、资出,曹肇入,泣涕固谏,帝使肇敕停。肇出户,放、资趋而往,复说止帝,帝又从其言。放曰:“宜为手诏。”
帝曰:“我困笃,不能。”放即上黙,执帝手强作之,遂赍出,大言曰:“有诏免燕王宇等官,不得停省中。”于是宇、肇、献、朗相与泣而归第。
初,青龙三年中,寿春农民妻自言为天神所下,命为登女,当营韂帝室,蠲邪纳福。饮人以水,及以洗疮,或多愈者。于是立馆后宫,下诏称扬,甚见优宠。及帝疾,饮水无验,于是杀焉。
三年春正月丁亥,太尉宣王还至河内,帝驿马召到,引入卧内,执其手谓曰:“吾疾甚,以后事属君,君其与爽辅少子。吾得见君,无所恨!”宣王顿首流涕。[一]即日,帝崩于嘉福殿,[二]时年三十六。[三]癸丑,葬高平陵。[四]
注[一]魏略曰:帝既从刘放计,召司马宣王,自力为诏,既封,顾呼宫中常所给使者曰:“辟邪来!汝持我此诏授太尉也。”辟邪驰去。先是,燕王为帝画计,以为关中事重,宜便道遣宣王从河内西还,事以施行。宣王得前诏,斯须复得后手笔,疑京师有变,乃驰到,入见帝。
劳问讫,乃召齐、秦二王以示宣王,别指齐王谓宣王曰:“此是也,君谛视之,勿误也!”
又教齐王令前抱宣王颈。魏氏春秋曰:时太子芳年八岁,秦王九岁,在于御侧。帝执宣王手,目太子曰:“死乃复可忍,朕忍死待君,君其与爽辅此。”宣王曰:“陛下不见先帝属臣以陛下乎?”
注[二]魏书曰:殡于九龙前殿。
注[三]臣松之按:魏武以建安九年八月定邺,文帝始纳甄后,明帝应以十年生,计至此年正月,整三十四年耳。时改正朔,以故年十二月为今年正月,可强名三十五年,不得三十六也。
注[四]魏书曰:帝容止可观,望之俨然。自在东宫,不交朝臣,不问政事,唯潜思书籍而已。

即位之后,褒礼大臣,料简功能,真伪不得相贸,务绝浮华谮毁之端,行师动众,论决大事,谋臣将相,咸服帝之大略。性特强识,虽左右小臣官簿性行,名迹所履,及其父兄子弟,一经耳目,终不遗忘。含垢藏疾,容受直言,听受吏民士庶上书,一月之中至数十百封,虽文辞鄙陋,犹览省究竟,意无厌倦。孙监曰:闻之长老,魏明帝天姿秀出,立发垂地,口吃少言,而沉毅好断。初,诸公受遗辅导,帝皆以方任处之,政自己出。而优礼大臣,开容善直,虽犯颜极谏,无所摧戮,其君人之量如此之伟也。然不思建德垂风,不固维城之基,至使大权偏据,社稷无韂,悲夫!
评曰:明帝沉毅断识,任心而行,盖有君人之至概焉。于时百姓雕弊,四海分崩,不先聿修显祖,阐拓洪基,而遽追秦皇、汉武,宫馆是营,格之远猷,其殆疾乎!
\end{yuanwen}

\part{魏书四}

\chapter{三少帝纪第四}

齐王讳芳,字兰卿。明帝无子,养王及秦王询;宫省事秘,莫有知其所由来者。青龙三年,立为齐王。景初三年正月丁亥朔,帝甚病,乃立为皇太子。是日,即皇帝位,大赦。尊皇后曰皇太后。大将军曹爽、太尉司马宣王辅政。诏曰:“朕以眇身,继承鸿业,茕茕在疚,靡所控告。大将军、太尉奉受末命,夹辅朕躬,司徒、司空、冢宰、元辅总率百寮,以宁社稷,其与群卿大夫勉勖乃心,称朕意焉。诸所兴作宫室之役,皆以遗诏罢之。官奴婢六十已上,免为良人。”二月,西域重译献火浣布,诏大将军、太尉临试以示百寮。

丁丑诏曰:“太尉体道正直,尽忠三世,南擒孟达,西破蜀虏,东灭公孙渊,功盖海内。昔周成建保傅之官,近汉显宗崇宠邓禹,所以优隆隽乂,必有尊也。其以太尉为太傅,持节统兵都督诸军事如故。”三月,以征东将军满宠为太尉。夏六月,以辽东东沓县吏民渡海居齐郡界,以故纵城为新沓县以居徙民。秋七月,上始亲临朝,听公卿奏事。八月,大赦。冬十月,以镇南将军黄权为车骑将军。

十二月,诏曰:“烈祖明皇帝以正月弃背天下,臣子永惟忌日之哀,其复用夏正;虽违先帝通三统之义,斯亦礼制所由变改也。又夏正於数为得天正,其以建寅之月为正始元年正月,以建丑月为后十二月。”

正始元年春二月乙丑,加侍中中书监刘放、侍中中书令孙资为左右光禄大夫。丙戌,以辽东汶、北丰县民流徙渡海,规齐郡之西安、临菑、昌国县界为新汶、南丰县,以居流民。

自去冬十二月至此月不雨。丙寅,诏令狱官亟平冤枉,理出轻微;群公卿士谠言嘉谋,各悉乃心。夏四月,车骑将军黄权薨。秋七月,诏曰:“易称损上益下,节以制度,不伤财,不害民。方今百姓不足而御府多作金银杂物,将奚以为?今出黄金银物百五十种,千八百馀斤,销冶以供军用”八月,车驾巡省洛阳界秋稼,赐高年力田各有差。

二年春二月,帝初通论语,使太常以太牢祭孔子於辟雍,以颜渊配。

夏五月,吴将朱然等围襄阳之樊城,太傅司马宣王率众拒之。六月辛丑,退。己卯,以征东将军王凌为车骑将军。冬十二月,南安郡地震。

三年春正月,东平王徽薨。三月,太尉满宠薨。秋七月甲申,南安郡地震。乙酉,以领军将军蒋济为太尉。冬十二月,魏郡地震。

四年春正月,帝加元服,赐群臣各有差。夏四月乙卯,立皇后甄氏,大赦。五月朔,日有食之,既。秋七月,诏祀故大司马曹真、曹休、征南大将军夏侯尚、太常桓阶、司空陈群、太傅锺繇、车骑将军张郃、左将军徐晃、前将军张辽、右将军乐进、太尉华歆、司徒王朗、骠骑将军曹洪、征西将军夏侯渊、后将军朱灵、文聘、执金吾臧霸、破虏将军李典、立义将军庞德、武猛校尉典韦於太祖庙庭。冬十二月,倭国女王俾弥呼遣使奉献。

五年春二月,诏大将军曹爽率众征蜀。夏四月朔,日有蚀之。五月癸巳,讲尚书经通,使太常以太牢祀孔子於辟雍,以颜渊配;赐太傅、大将军及侍讲者各有差。丙午,大将军曹爽引军还。秋八月,秦王询薨。九月,鲜卑内附,置辽东属国,立昌黎县以居之。冬十一月癸卯,诏祀故尚书令荀攸于太祖庙庭。己酉,复秦国为京兆郡。十二月,司空崔林薨。

六年春二月丁卯,南安郡地震。丙子,以骠骑将军赵俨为司空;夏六月,俨薨。八月丁卯,以太常高柔为司空。癸巳,以左光禄大夫刘放为骠骑将军,右光禄大夫孙资为卫将军。冬十一月,祫祭太祖庙,始祀前所论佐命臣二十一人。十二月辛亥,诏故司徒王朗所作易传,令学者得以课试。乙亥,诏曰:“明日大会群臣,其令太傅乘舆上殿。”

七年春二月,幽州刺史毌丘俭讨高句骊,夏五月,讨濊貊,皆破之。韩那奚等数十国各率种落降。秋八月戊申,诏曰:“属到巿观见所斥卖官奴婢,年皆七十,或癃疾残病,所谓天民之穷者也。且官以其力竭而复鬻之,进退无谓,其悉遣为良民。若有不能自存者,郡县振给之。”

己酉,诏曰:“吾乃当以十九日亲祠,而昨出已见治道,得雨当复更治,徒弃功夫。每念百姓力少役多,夙夜存心。道路但当期于通利,闻乃挝捶老小,务崇脩饰,疲困流离,以至哀叹,吾岂安乘此而行,致馨德于宗庙邪?自今已后,明申敕之。”冬十二月,讲礼记通,使太常以太牢祀孔子於辟雍,以颜渊配。

八年春二月朔,日有蚀之。夏五月,分河东之汾北十县为平阳郡。

秋七月,尚书何晏奏曰:“善为国者必先治其身,治其身者慎其所习。所习正则其身正,其身正则不令而行;所习不正则其身不正,其身不正则虽令不从。是故为人君者,所与游必择正人,所观览必察正象,放郑声而弗听,远佞人而弗近,然后邪心不生而正道可弘也。季末闇主,不知损益,斥远君子,引近小人,忠良疏远,便辟亵狎,乱生近昵,譬之社鼠;考其昏明,所积以然,故圣贤谆谆以为至虑。舜戒禹曰'邻哉邻哉',言慎所近也,周公戒成王曰‘其朋其朋’,言慎所与也。【诗】书云:‘一人有庆,兆民赖之。’可自今以后,御幸式乾殿及游豫后园,皆大臣侍从,因从容戏宴,兼省文书,询谋政事,讲论经义,为万世法。”冬十二月,散骑常侍谏议大夫孔乂奏曰:“礼,天子之宫,有斫砻之制,无朱丹之饰,宜循礼复古。今天下已平,君臣之分明,陛下但当不懈于位,平公正之心,审赏罚以使之。可绝后园习骑乘马,出必御辇乘车,天下之福,臣子之愿也。”晏、乂咸因阙以进规谏。

九年春二月,卫将军中书令孙资,癸巳,骠骑将军中书监刘放,三月甲午,司徒卫臻,各逊位,以侯就第,位特进。四月,以司空高柔为司徒;光禄大夫徐邈为司空,固辞不受。秋九月,以车骑将军王凌为司空。冬十月,大风发屋折树。

嘉平元年春正月甲午,车驾谒高平陵。太傅司马宣王奏免大将军曹爽、爽弟中领军羲、武卫将军训、散骑常侍彦官,以侯就第。戊戌,有司奏收黄门张当付廷尉,考实其辞,爽与谋不轨。又尚书丁谧、邓飏、何晏、司隶校尉毕轨、荆州刺史李胜、大司农桓范皆与爽通奸谋,夷三族。语在爽传。丙午,大赦。丁未,以太傅司马宣王为丞相,固让乃止。

夏四月乙丑,改年。丙子,太尉蒋济薨。冬十二月辛卯,以司空王凌为太尉。庚子,以司隶校尉孙礼为司空。

二年夏五月,以征西将军郭淮为车骑将军。冬十月,以特进孙资为骠骑将军。十一月,司空孙礼薨。十二月甲辰,东海王霖薨。乙未,征南将军王昶渡江,掩攻吴,破之。

三年春正月,荆州刺史王基、新城太守州泰攻吴,破之,降者数千口。二月,置南郡之夷陵县以居降附。三月,以尚书令司马孚为司空。四月甲申,以征南将军王昶为征南大将军。壬辰,大赦。丙午,闻太尉王凌谋废帝,立楚王彪,太傅司马宣王东征凌。五月甲寅,凌自杀。六月,彪赐死。秋七月壬戌,皇后甄氏崩。辛未,以司空司马孚为太尉。戊寅,太傅司马宣王薨,以卫将军司马景王为抚军大将军,录尚书事。乙未,葬怀甄后於太清陵。庚子,骠骑将军孙资薨。十一月,有司奏诸功臣应飨食於太祖庙者,更以官为次,太傅司马宣王功高爵尊,最在上。十二月,以光禄勋郑冲为司空。

四年春正月癸卯,以抚军大将军司马景王为大将军。二月,立皇后张氏,大赦。夏五月,鱼二,见於武库屋上。冬十一月,诏征南大将军王昶、征东将军胡遵、镇南将军毌丘俭等征吴。十二月,吴大将军诸葛恪拒战,大破众军于东关。不利而还。

五年夏四月,大赦。五月,吴太傅诸葛恪围合肥新城,诏太尉司马孚拒之。秋七月,恪退还。

八月,诏曰:“故中郎西平郭脩,砥节厉行,秉心不回。乃者蜀将姜维寇钞脩郡,为所执略。往岁伪大将军费祎驱率群众,阴图闚,道经汉寿,请会众宾,脩於广坐之中手刃击祎,勇过聂政,功逾介子,可谓杀身成仁,释生取义者矣。夫追加褒宠,所以表扬忠义;祚及后胤,所以奖劝将来。其追封脩为长乐乡侯,食邑千户,谥曰威侯;子袭爵,加拜奉车都尉;赐银千鉼,绢千匹,以光宠存亡,永垂来世焉。”

自帝即位至于是岁,郡国县道多所置省,俄或还复,不可胜纪。

六年春二月己丑,镇东将军毌丘俭上言:“昔诸葛恪围合肥新城,城中遣士刘整出围传消息,为贼所得,考问所传,语整曰:‘诸葛公欲活汝,汝可具服。’整骂曰:‘死狗,此何言也!我当必死为魏国鬼,不苟求活,逐汝去也。欲杀我者,便速杀之。’终无他辞。又遣士郑像出城传消息,或以语恪,恪遣马骑寻围迹索,得像还。四五人靮头面缚,将绕城表,敕语像,使大呼,言‘大军已还洛,不如早降。’像不从其言,更大呼城中曰:‘大军近在围外,壮士努力!’贼以刀筑其口,使不得言,像遂大呼,令城中闻知。整、像为兵,能守义执节,子弟宜有差异。”诏曰:“夫显爵所以褒元功,重赏所以宠烈士。整、像召募通使,越蹈重围,冒突白刃,轻身守信,不幸见获,抗节弥厉,扬六军之大势,安城守之惧心,临难不顾,毕志传命。昔解杨执楚,有陨无贰,齐路中大夫以死成命,方之整、像,所不能加。今追赐整、像爵关中侯,各除士名,使子袭爵,如部曲将死事科。”

庚戌,中书令李丰与皇后父光禄大夫张缉等谋废易大臣,以太常夏侯玄为大将军。事觉,诸所连及者皆伏诛。辛亥,大赦。三月,废皇后张氏。夏四月,立皇后王氏,大赦。五月,封后父奉车都尉王夔为广明乡侯、光禄大夫,位特进,妻田氏为宣阳乡君。秋九月,大将军司马景王将谋废帝,以闻皇太后。甲戌,太后令曰:“皇帝芳春秋已长,不亲万机,耽淫内宠,沈漫女德,日延倡优,纵其丑谑;迎六宫家人留止内房,毁人伦之叙,乱男女之节;恭孝日亏,悖慠滋甚,不可以承天绪,奉宗庙。使兼太尉高柔奉策,用一元大武告于宗庙,遣芳归藩于齐,以避皇位。”是日迁居别宫,年二十三。使者持节送卫,营齐王宫於河内之重门,制度皆如藩国之礼。

丁丑,令曰:“东海王霖,高祖文皇帝之子。霖之诸子,与国至亲,高贵乡公髦有大成之量,其以为明皇帝嗣。

高贵乡公讳髦,字彦士,文帝孙,东海定王霖子也。正始五年,封郯县高贵乡公。少好学,夙成。齐王废,公卿议迎立公。十月己丑,公至于玄武馆,群臣奏请舍前殿,公以先帝旧处,避止西厢;群臣又请以法驾迎,公不听。庚寅,公入于洛阳,群臣迎拜西掖门南,公下舆将答拜,傧者请曰:“仪不拜”公曰:“吾人臣也。”遂答拜。至止车门下舆。左右曰:“旧乘舆入。”公曰:“吾被皇太后徵,未知所为”遂步至太极东堂,见于太后。其日即皇帝位於太极前殿,百僚陪位者欣欣焉。诏曰:“昔三祖神武圣德,应天受祚。齐王嗣位,肆行非度,颠覆厥德。皇太后深惟社稷之重,延纳宰辅之谋,用替厥位,集大命于余一人。以眇眇之身,讬于王公之上,夙夜祗畏,惧不能嗣守祖宗之大训,恢中兴之弘业,战战兢兢,如临于谷。今群公卿士股肱之辅,四方征镇宣力之佐,皆积德累功,忠勤帝室;庶凭先祖先父有德之臣,左右小子,用保乂皇家,俾朕蒙闇,垂拱而治。盖闻人君之道,德厚侔天地,润泽施四海,先之以慈爱,示之以好恶,然后教化行於上,兆民听於下。朕虽不德,昧於大道,思与宇内共臻兹路。书不云乎:‘安民则惠,黎民怀之。’”大赦,改元。减乘舆服御,后宫用度,及罢尚方御府百工技巧靡丽无益之物。

正元元年冬十月壬辰,遣侍中持节分適四方,观风俗,劳士民,察冤枉失职者。癸巳,假大将军司马景王黄钺,入朝不趋,奏事不名,剑履上殿。戊戌,黄龙见于邺井中。甲辰,命有司论废立定策之功,封爵、增邑、进位、班赐各有差。

二年春正月乙丑,镇东将军毌丘俭、扬州刺史文钦反。戊寅,大将军司马景王征之。癸未,车骑将军郭淮薨。闰月己亥,破钦于乐嘉。钦遁走,遂奔吴。甲辰,安风津都尉斩俭,传首京都。壬子,复特赦淮南士民诸为俭、钦所诖误者。以镇南将军诸葛诞为镇东大将军。司马景王薨于许昌。二月丁巳,以卫将军司马文王为大将军,录尚书事。

甲子,吴大将孙峻等众号十万至寿春,诸葛诞拒击破之,斩吴左将军留赞,献捷于京都。三月,立皇后卞氏,大赦。夏四月甲寅,封后父卞隆为列侯。甲戌,以征南大将军王昶为骠骑将军。秋七月,以征东大将军胡遵为卫将军,镇东大将军诸葛诞为征东大将军。

八月辛亥,蜀大将军姜维寇狄道,雍州刺史王经与战洮西,经大败,还保狄道城。辛未,以长水校尉邓艾行安西将军,与征西将军陈泰并力拒维。戊辰,复遣太尉司马孚为后继。九月庚子,讲尚书业终,赐执经亲授者司空郑冲、侍中郑小同等各有差。甲辰,姜维退还。冬十月,诏曰:“朕以寡德,不能式遏寇虐,乃令蜀贼陆梁边陲。洮西之战,至取负败,将士死亡,计以千数,或没命战场,冤魂不反,或牵掣虏手,流离异域,吾深痛愍,为之悼心。其令所在郡典农及安抚夷二护军各部大吏慰恤其门户,无差赋役一年;其力战死事者,皆如旧科,勿有所漏。”

十一月甲午,以陇右四郡及金城,连年受敌,或亡叛投贼,其亲戚留在本土者不安,皆特赦之。癸丑,诏曰:“往者洮西之战,将吏士民或临陈战亡,或沈溺洮水,骸骨不收,弃於原野,吾常痛之。其告征西、安西将军,各令部人於战处及水次钩求尸丧,收敛藏埋,以慰存亡。”

甘露元年春正月辛丑,青龙见轵县井中。乙巳,沛王林薨。

夏四月庚戌,赐大将军司马文王衮冕之服,赤舄副焉。

丙辰,帝幸太学,问诸儒曰:“圣人幽赞神明,仰观俯察,始作八卦,后圣重之为六十四,立爻以极数,凡斯大义,罔有不备,而夏有连山,殷有归藏,周曰周易,易之书,其故何也?”易博士淳于俊对曰:“包羲因燧皇之图而制八卦,神农演之为六十四,黄帝、尧、舜通其变,三代随时,质文各繇其事。故易者,变易也,名曰连山,似山出内云气,连天地也;归藏者,万事莫不归藏于其中也。”帝又曰:“若使包羲因燧皇而作易,孔子何以不云燧人氏没包羲氏作乎?”俊不能答。帝又问曰:“孔子作彖、象,郑玄作注,虽圣贤不同,其所释经义一也。今彖、象不与经文相连,而注连之,何也?”俊对曰;“郑玄合彖、象于经者,欲使学者寻省易了也。”帝曰:“若郑玄合之,於学诚便,则孔子曷为不合以了学者乎?”俊对曰:“孔子恐其与文王相乱,是以不合,此圣人以不合为谦。”帝曰:“若圣人以不合为谦,则郑玄何独不谦邪?”俊对曰:“古义弘深,圣问奥远,非臣所能详尽。”帝又问曰:“系辞云‘黄帝、尧、舜垂衣裳而天下治’,此包羲、神农之世为无衣裳。但圣人化天下,何殊异尔邪?”俊对曰:“三皇之时,人寡而禽兽众,故取其羽皮而天下用足,及至黄帝,人众而禽兽寡,是以作为衣裳以济时变也。”帝又问:“乾为天,而复为金,为玉,为老马,与细物并邪?”俊对曰:“圣人取象,或远或近,近取诸物,远则天地。”

讲易毕,复命讲尚书。帝问曰:“郑玄曰‘稽古同天,言尧同於天也’。王肃云‘尧顺考古道而行之’。三义不同,何者为是?“博士庾峻对曰:“先儒所执,各有乖异,臣不足以定之。然洪范称‘三人占,从二人之言’。贾、马及肃皆以为'顺考古道’。以洪范言之,肃义为长。“帝曰:“仲尼言‘唯天为大,唯尧则之’。尧之大美,在乎则天,顺考古道,非其至也。今发篇开义以明圣德,而舍其大,更称其细,岂作者之意邪?“峻对曰:“臣奉遵师说,未喻大义,至于折中,裁之圣思。”次及四岳举鲧,帝又问曰:“夫大人者,与天地合其德,与日月合其明,思无不周,明无不照,今王肃云‘尧意不能明鲧,是以试用’。如此,圣人之明有所未尽邪?”峻对曰:“虽圣人之弘,犹有所未尽,故禹曰‘知人则哲,惟帝难之’,然卒能改授圣贤,缉熙庶绩,亦所以成圣也。”帝曰:“夫有始有卒,其唯圣人。若不能始,何以为圣?其言‘惟帝难之’,然卒能改授,盖谓知人,圣人所难,非不尽之言也。经云:‘知人则哲,能官人。’若尧疑鲧,试之九年,官人失叙,何得谓之圣哲?”峻对曰:“臣窃观经传,圣人行事不能无失,是以尧失之四凶,周公失之二叔,仲尼失之宰予。”帝曰:“尧之任鲧,九载无成,汨陈五行,民用昏垫。至於仲尼失之宰予,言行之间,轻重不同也。至于周公、管、蔡之事,亦尚书所载,皆博士所当通也。”峻对曰:“此皆先贤所疑,非臣寡见所能究论。“次及”有鳏在下曰“虞舜”,帝问曰:“当尧之时,洪水为害,四凶在朝,宜速登贤圣济斯民之时也。舜年在既立,圣德光明,而久不进用,何也?”峻对曰:“尧咨嗟求贤,欲逊己位,岳曰‘否德忝帝位’。尧复使岳扬举仄陋,然后荐舜。荐舜之本,实由於尧,此盖圣人欲尽众心也。”帝曰:“尧既闻舜而不登用,又时忠臣亦不进达,乃使狱扬仄陋而后荐举,非急於用圣恤民之谓也。“峻对曰:“非臣愚见所能逮及。”

於是复命讲礼记。帝问曰:“太上立德,其次务施报'。为治何由而教化各异;皆脩何政而能致于立德,施而不报乎?”博士马照对曰:“太上立德,谓三皇五帝之世以德化民,其次报施,谓三王之世以礼为治也。”帝曰:“二者致化薄厚不同,将主有优劣邪?时使之然乎?”照对曰:“诚由时有朴文,故化有薄厚也。”

五月,邺及上洛并言甘露降。夏六月丙午,改元为甘露。乙丑,青龙见元城县界井中。秋七月己卯,卫将军胡遵薨。

癸未,安西将军邓艾大破蜀大将姜维于上邽,诏曰:“兵未极武,丑虏摧破,斩首获生,动以万计,自顷战克,无如此者。今遣使者犒赐将士,大会临飨,饮宴终日,称朕意焉。”

八月庚午,命大将军司马文王加号大都督,奏事不名,假黄钺。癸酉,以太尉司马孚为太傅。九月,以司徒高柔为太尉。冬十月,以司空郑冲为司徒,尚书左仆射卢毓为司空。

二年春二月,青龙见温县井中。三月,司空卢毓薨。

夏四月癸卯,诏曰:“玄菟郡高显县吏民反叛,长郑熙为贼所杀。民王简负担熙丧,晨夜星行,远致本州,忠节可嘉。其特拜简为忠义都尉,以旌殊行。”

甲子,以征东大将军诸葛诞为司空。

五月辛未,帝幸辟雍,会命群臣赋诗。侍中和逌、尚书陈骞等作诗稽留,有司奏免官,诏曰:“吾以暗昧,爱好文雅,广延诗赋,以知得失,而乃尔纷纭,良用反仄。其原逌等。主者宜敕自今以后,群臣皆当玩习古义,脩明经典,称朕意焉。”

乙亥,诸葛诞不就徵,发兵反,杀扬州刺史乐綝。丙子,赦淮南将吏士民为诞所诖误者。丁丑,诏曰:“诸葛诞造为凶乱,荡覆扬州。昔黥布逆叛,汉祖亲戎,隗嚣违戾,光武西伐,及烈祖明皇帝躬征吴、蜀,皆所以奋扬赫斯,震耀威武也。今宜皇太后与朕暂共临戎,速定丑虏,时宁东夏。”己卯,诏曰:“诸葛诞造构逆乱,迫胁忠义,平寇将军临渭亭侯庞会、骑督偏将军路蕃,各将左右,斩门突出,忠壮勇烈,所宜嘉异。其进会爵乡侯,蕃封亭侯。”

六月乙巳,诏:“吴使持节都督夏口诸军事镇军将军沙羡侯孙壹,贼之枝属,位为上将,畏天知命,深鉴祸福,翻然举众,远归大国,虽微子去殷,乐毅遁燕,无以加之。其以壹为侍中车骑将军、假节、交州牧、吴侯,开府辟召仪同三司,依古侯伯八命之礼,衮冕赤舄,事从丰厚。”

甲子,诏曰:“今车驾驻项,大将军恭行天罚,前临淮浦。昔相国大司马征讨,皆与尚书俱行,今宜如旧。”乃令散骑常侍裴秀、给事黄门侍郎锺会咸与大将军俱行。秋八月,诏曰:“昔燕刺王谋反,韩谊等谏而死,汉朝显登其子。诸葛诞创造凶乱,主簿宣隆、部曲督秦絜秉节守义,临事固争,为诞所杀,所谓无比干之亲而受其戮者。其以隆、絜子为骑都尉,加以赠赐,光示远近,以殊忠义。”

九月,大赦。冬十二月,吴大将全端、全怿等率众降。

三年春二月,大将军司马文王陷寿春城,斩诸葛诞。三月,诏曰:“古者克敌,收其尸以为京观,所以惩昏逆而章武功也。汉孝武元鼎中,改桐乡为闻喜,新乡为获嘉,以著南越之亡。大将军亲总六戎,营据丘头,内夷群凶,外殄寇虏,功济兆民,声振四海。克敌之地,宜有令名,其改丘头为武丘,明以武平乱,后世不忘,亦京观二邑之义也。”

夏五月,命大将军司马文王为相国,封晋公,食邑八郡,加之九锡,文王前后九让乃止。

六月丙子,诏曰:“昔南阳郡山贼扰攘,欲劫质故太守东里衮,功曹应余独身捍衮,遂免於难。余颠沛殒毙,杀身济君。其下司徒,署余孙伦吏,使蒙伏节之报。”

辛卯,大论淮南之功,封爵行赏各有差。

秋八月甲戌,以骠骑将军王昶为司空。丙寅,诏曰:“夫养老兴教,三代所以树风化垂不朽也,必有三老、五更以崇至敬,乞言纳诲,著在惇史,然后六合承流,下观而化。宜妙简德行,以充其选。关内侯王祥,履仁秉义,雅志淳固。关内侯郑小同,温恭孝友,帅礼不忒。其以祥为三老,小同为五更。”车驾亲率群司,躬行古礼焉。

是岁,青龙、黄龙仍见顿丘、冠军、阳夏县界井中。

四年春正月,黄龙二,见宁陵县界井中。夏六月,司空王昶薨。秋七月,陈留王峻薨。冬十月丙寅,分新城郡,复置上庸郡。十一月癸卯,车骑将军孙壹为婢所杀。

五年春正月朔,日有蚀之。夏四月,诏有司率遵前命,复进大将军司马文王位为相国,封晋公,加九锡。

五月己丑,高贵乡公卒,年二十。皇太后令曰:“吾以不德,遭家不造,昔援立东海王子髦,以为明帝嗣,见其好书疏文章,冀可成济,而情性暴戾,日月滋甚。吾数呵责,遂更忿恚,造作丑逆不道之言以诬谤吾,遂隔绝两宫。其所言道,不可忍听,非天地所覆载。吾即密有令语大将军,不可以奉宗庙,恐颠覆社稷,死无面目以见先帝。大将军以其尚幼,谓当改心为善,殷勤执据。而此儿忿戾,所行益甚,举弩遥射吾宫,祝当令中吾项,箭亲堕吾前。吾语大将军,不可不废之,前后数十。此儿具闻,自知罪重,便图为弑逆,赂遗吾左右人,令因吾服药,密因鸩毒,重相设计。事已觉露,直欲因际会举兵入西宫杀吾,出取大将军,呼侍中王沈、散骑常侍王业、尚书王经,出怀中黄素诏示之,言今日便当施行。吾之危殆,过于累卵。吾老寡,岂复多惜馀命邪?但伤先帝遗意不遂,社稷颠覆为痛耳。赖宗庙之灵,沈、业即驰语大将军,得先严警,而此儿便将左右出云龙门,雷战鼓,躬自拔刃,与左右杂卫共入兵陈间,为前锋所害。此儿既行悖逆不道,而又自陷大祸,重令吾悼心不可言。昔汉昌邑王以罪废为庶人,此儿亦宜以民礼葬之,当令内外咸知此儿所行。又尚书王经,凶逆无状,其收经及家属皆诣廷尉。”

庚寅,太傅孚、大将军文王、太尉柔、司徒冲稽首言:“伏见中令,故高贵乡公悖逆不道,自陷大祸,依汉昌邑王罪废故事,以民礼葬。臣等备位,不能匡救祸乱,式遏奸逆,奉令震悚,肝心悼栗。春秋之义,王者无外,而书'襄王出居于郑',不能事母,故绝之于位也。今高贵乡公肆行不轨,几危社稷,自取倾覆,人神所绝,葬以民礼,诚当旧典。然臣等伏惟殿下仁慈过隆,虽存大义,犹垂哀矜,臣等之心实有不忍,以为可加恩以王礼葬之。”太后从之。

使使持节行中护军中垒将军司马炎北迎常道乡公璜嗣明帝后。辛卯,群公奏太后曰:“殿下圣德光隆,宁济六合,而犹称令,与藩国同。请自今殿下令书,皆称诏制,如先代故事。”

癸卯,大将车固让相国、晋公、九锡之宠。太后诏曰:“夫有功不隐,周易大义,成人之美,古贤所尚,今听所执,出表示外,以章公之谦光焉。”

戊申,大将军文王上言:“高贵乡公率将从驾人兵,拔刃鸣金鼓向臣所止;惧兵刃相接,即敕将士不得有所伤害,违令以军法从事。骑督成倅弟太子舍人济,横入兵陈伤公,遂至陨命;辄收济行军法。臣闻人臣之节,有死无二,事上之义,不敢逃难。前者变故卒至,祸同发机,诚欲委身守死,唯命所裁。然惟本谋乃欲上危皇太后,倾覆宗庙。臣忝当大任,义在安国,惧虽身死,罪责弥重。欲遵伊、周之权,以安社稷之难,即骆驿申敕,不得迫近辇舆,而济遽入陈间,以致大变。哀怛痛恨,五内摧裂,不知何地可以陨坠?科律大逆无道,父母妻子同产皆斩。济凶戾悖逆,干国乱纪,罪不容诛。辄敕侍御史收济家属,付廷尉,结正其罪。”太后诏曰:“夫五刑之罪,莫大於不孝。夫人有子不孝,尚告治之,此儿岂复成人主邪?吾妇人不达大义,以谓济不得便为大逆也。然大将军志意恳切,发言恻怆,故听如所奏。当班下远近,使知本末也。”

六月癸丑,诏曰:“古者人君之为名字,难犯而易讳。今常道乡公讳字甚难避,其朝臣博议改易,列奏。”

陈留王讳奂,字景明,武帝孙,燕王宇子也。甘露三年,封安次县常道乡公。高贵乡公卒,公卿议迎立公。六月甲寅,入于洛阳,见皇太后,是日即皇帝位于太极前殿,大赦,改年,赐民爵及谷帛各有差。

景元元年夏六月丙辰,进大将军司马文王位为相国,封晋公,增封二郡,并前满十,加九锡之礼,一如前诏;诸群从子弟,其未有侯者皆封亭侯,赐钱千万,帛万匹,文王固让乃止。己未,故汉献帝夫人节薨,帝临于华林园,使使持节追谥夫人为献穆皇后。及葬,车服制度皆如汉氏故事。癸亥,以尚书右仆射王观为司空,冬十月,观薨。

十一月,燕王上表贺冬至,称臣。诏曰:“古之王者,或有所不臣,王将宜依此义,表不称臣乎!又当为报。夫后大宗者,降其私亲,况所继者重邪!若便同之臣妾,亦情所未安。其皆依礼典处,当务尽其宜。”有司奏,以为“礼莫崇于尊祖,制莫大于正典。陛下稽德期运,抚临万国,绍大宗之重,隆三祖之基。伏惟燕王体尊戚属,正位藩服,躬秉虔肃,率蹈恭德以先万国;其于正典,阐济大顺,所不得制。圣朝诚宜崇以非常之制,奉以不臣之礼。臣等平议以为燕王章表,可听如旧式。中诏所施,或存好问,准之义类,则‘燕觌之敬’也可少顺圣敬,加崇仪称,示不敢斥,宜曰‘皇帝敬问大王侍御’。至于制书,国之正典,朝廷所以辨章公制,宣昭轨仪于天下者也,宜循法,故曰‘制诏燕王’。凡诏命、制书、奏事、上书诸称燕王者,可皆上平。其非宗庙助祭之事,皆不得称王名,奏事、上书、文书及吏民皆不得触王讳,以彰殊礼,加于群后。上遵王典尊祖之制,俯顺圣敬烝烝之心,二者不愆,礼实宜之,可普告施行。”

十二月甲申,黄龙见华阴县井中。甲午,以司隶校尉王祥为司空。

二年夏五月朔,日有食之。秋七月,乐浪外夷韩、濊貊各率其属来朝贡。八月戊寅,赵王幹薨。甲寅,复命大将军进爵晋公,加位相国,备礼崇锡,一如前诏;又固辞乃止。

三年春二月,青龙见于轵县井中。夏四月,辽东郡言肃慎国遣使重译入贡,献其国弓三十张,长三尺五寸,楛矢长一尺八寸,石弩三百枚,皮骨铁杂铠二十领,貂皮四百枚。冬十月,蜀大将姜维寇洮阳,镇西将军邓艾拒之,破维于侯和,维遁走。是岁,诏祀故军祭酒郭嘉於太祖庙庭。

四年春二月,复命大将军进位爵赐一如前诏,又固辞乃止。

夏五月,诏曰:“蜀,蕞尔小国,土狭民寡,而姜维虐用其众,曾无废志;往岁破败之后,犹复耕种沓中,刻剥众羌,劳役无已,民不堪命。夫兼弱攻昧,武之善经,致人而不致於人,兵家之上略。蜀所恃赖,唯维而已,因其远离巢窟,用力为易。今使征西将军邓艾督帅诸军,趣甘松、沓中以罗取维,雍州刺史诸葛绪督诸军趣武都、高楼,首尾蹴讨。若擒维,便当东西并进,扫灭巴蜀也。”又命镇西将军锺会由骆谷伐蜀。

秋九月,太尉高柔薨。冬十月甲寅,复命大将军进位爵赐一如前诏。癸卯,立皇后卞氏,十一月,大赦。

自邓艾、锺会率众伐蜀,所至辄克。是月,蜀主刘禅诣艾降,巴蜀皆平。十二月庚戌,以司徒郑冲为太保。壬子,分益州为梁州。癸丑,特赦益州士民,复除租赋之半五年。

乙卯,以征西将军邓艾为太尉,镇西将军锺会为司徒。皇太后崩。

咸熙元年春正月壬戌,槛车徵邓艾。甲子,行幸长安。壬申,使使者以璧币祀华山。是月,锺会反于蜀,为众所讨;邓艾亦见杀。二月辛卯,特赦诸在益土者。庚申,葬明元郭后。三月丁丑,以司空王祥为太尉,征北将军何曾为司徒,尚书左仆射荀顗为司空。己卯,进晋公爵为王,封十郡,并前二十。丁亥,封刘禅为安乐公。夏五月庚申,相国晋王奏复五等爵。甲戌,改年。癸未,追命舞阳宣文侯为晋宣王,舞阳忠武侯为晋景王。六月,镇西将军卫瓘上雍州兵于成都县获璧玉印各一,印文似“成信“字。依周成王归禾之义,宣示百官,藏于相国府。

初,自平蜀之后,吴寇屯逼永安,遣荆、豫诸军掎角赴救。七月,贼皆遁退。八月庚寅,命中抚军司马炎副贰相国事,以同鲁公拜后之义。

癸巳,诏曰:“前逆臣锺会构造反乱,聚集征行将士,劫以兵威,始吐奸谋,发言桀逆,逼胁众人,皆使下议,仓卒之际,莫不惊慑。相国左司马夏侯和、骑士曹属朱抚时使在成都,中领军司马贾辅、郎中羊琇各参会军事;和、琇、抚皆抗节不挠,拒会凶言,临危不顾,词指正烈。辅语散将王起,说‘会奸逆凶暴,欲尽杀将士',又云'相国已率三十万众西行讨会’,欲以称张形势,感激众心。起出,以辅言宣语诸军,遂使将士益怀奋励。宜加显宠,以彰忠义。其进和、辅爵为乡侯,琇、抚爵关内侯。起宣传辅言,告令将士,所宜赏异。其以起为部曲将。”

癸卯,以卫将军司马望为骠骑将军。九月戊午,以中抚军司马炎为抚军大将军。

辛未,诏曰:“吴贼政刑暴虐,赋敛无极。孙休遣使邓句,敕交阯太守锁送其民,发以为兵。吴将吕兴因民心愤怒,又承王师平定 巴蜀,即纠合豪杰,诛除句等,驱逐太守长吏,抚和吏民,以待国命。九真、日南郡闻兴去逆即顺,亦齐心响应,与兴协同。兴移书日南州郡,开示大计,兵临合浦,告以祸福。遣都尉唐谱等诣进乘县,因南中都督护军霍弋上表自陈。又交阯将吏各上表,言‘兴创造事业,大小承命。郡有山寇,入连诸郡,惧其计异,各有携贰。权时之宜,以兴为督交阯诸军事、上大将军、定安县侯,乞赐褒奖,以慰边荒’。乃心款诚,形于辞旨。昔仪父朝鲁,春秋所美;窦融归汉,待以殊礼。今国威远震,抚怀六合。方包举殊裔,混一四表。兴首向王化,举众稽服,万里驰义,请吏帅职,宜加宠遇,崇其爵位。既使兴等怀忠感悦,远人闻之,必皆竞劝。其以兴为使持节、都督交州诸军事、南中大将军,封定安县侯,得以便宜从事,先行后上。”策命未至,兴为下人所杀。

冬十月丁亥,诏曰:“昔圣帝明王,静乱济世,保大定功。文武殊涂,勋烈同归。是故或舞干戚以训不庭,或陈师旅以威暴慢。至于爱民全国,康惠庶类,必先脩文教,示之轨仪,不得已然后用兵,此盛德之所同也。往者季汉分崩,九土颠覆,刘备、孙权乘间作祸。三祖绥宁中夏,日不暇给,遂使遗寇僣逆历世。幸赖宗庙威灵,宰辅忠武,爰发四方,拓定庸、蜀,役不浃时,一征而克。自顷江表衰弊,政刑荒闇,巴、汉平定,孤危无援。交、荆、扬、越,靡然向风。今交阯伪将吕兴已帅三郡,万里归命;武陵邑侯相严等纠合五县,请为臣妾;豫章庐陵山民举众叛吴,以助北将军为号。又孙休病死,主帅改易,国内乖违,人各有心。伪将施绩,贼之名臣,怀疑自猜,深见忌恶。众叛亲离,莫有固志,自古及今,未有亡徵若此之甚。若六军震曜,南临江、汉,吴会之域必扶老携幼以迎王师,必然之理也。然兴动大众,犹有劳费,宜告喻威德,开示仁信,使知顺附和同之利。相国参军事徐绍、水曹掾孙彧,昔在寿春,并见虏获。绍本伪南陵督,才质开壮;彧,孙权支属,忠良见事。其遣绍南还,以彧为副,宣扬国命,告喻吴人,诸所示语,皆以事实,若其觉悟,不损征伐之计。盖庙胜长算,自古之道也。其以绍兼散骑常侍,加奉车都尉,封都亭侯;彧兼给事黄门侍郎,赐爵关内侯。绍等所赐妾及男女家人在此者,悉听自随,以明国恩。不必使还,以开广大信。”

丙午,命抚军大将军新昌乡侯炎为晋世子。是岁,罢屯田官以均政役,诸典农皆为太守,都尉皆为令长;劝募蜀人能内移者,给廪二年,复除二十岁。安弥、福禄县各言嘉禾生。

二年春二月甲辰,朐县获灵龟以献,归之于相国府。庚戌,以虎贲张脩昔於成都驰马至诸营言锺会反逆,以至没身,赐脩弟倚爵关内侯。夏四月,南深泽县言甘露降。吴遣使纪陟、弘璆请和。

五月,诏曰:“相国晋王诞敷神虑,光被四海;震耀武功,则威盖殊荒;流风迈化,则旁洽无外。愍恤江表,务存济育,戢武崇仁,示以威德。文告所加,承风向慕,遣使纳献,以明委顺。方宝纤珍,欢以效意。而王谦让之至,一皆簿送。非所以慰副初附,从其款愿也。孙皓诸所献致,其皆还送,归之于王,以协古义。”王固辞乃止。又命晋王冕十有二旒,建天子旌旗,出警入跸,乘金根车、六马,备五时副车,置旄头云罕,乐舞八佾,设钟虡宫县。进王妃为王后,世子为太子,王子、王女、王孙,爵命之号如旧仪。癸未,大赦。秋八月辛卯,相国晋王薨。壬辰,晋太子炎绍封袭位,总摄百揆,备物典册,一皆如前。是月,襄武县言有大人见,长三丈馀,迹长三尺二寸,白发,著黄单衣,黄巾,柱杖。呼民王始语云:“今当太平。”九月乙未,大赦。戊午,司徒何曾为晋丞相。癸亥,以骠骑将军司马望为司徒,征东大将军石苞为骠骑将军,征南大将军陈骞为车骑将军。乙亥,葬晋文王。闰月庚辰,康居、大宛献名马,归于相国府,以显怀万国致远之勋。

十二月壬戌,天禄永终,历数在晋。诏群公卿士具仪设坛于南郊,使使者奉皇帝玺绶册,禅位于晋嗣王,如汉魏故事。甲子,使使者奉策。遂改次于金墉城,而终馆于邺。时年二十。

评曰:古者以天下为公,唯贤是与。后代世位,立子以適;若適嗣不继,则宜取旁亲明德,若汉之文、宣者,斯不易之常准也。明帝既不能然,情系私爱,抚养婴孩,传以大器。讬付不专,必参枝族,终于曹爽诛夷,齐王替位。高贵公才慧夙成,好问尚辞,盖亦文帝之风流也;然轻躁忿肆,自蹈大祸。陈留王恭己南面,宰辅统政,仰遵前式,揖让而禅。遂飨封大国,作宾于晋,比之山阳,班宠有加焉。

\part{魏书五}

\chapter{后妃传第五}

易称“男正位乎外,女正位乎内;男女正,天地之大义也”。古先哲王,莫不明后妃之制,顺天地之德。故二妃嫔妫,虞道克隆;任、姒配姬,周室用熙。废兴存亡,恒此之由。春秋说云天子十二女,诸侯九女,考之情理,不易之典也。而末世奢纵,肆其侈欲,至使男女怨旷,感动和气;惟色是崇,不本淑懿,故风教陵迟而大纲毁泯,岂不惜哉!呜呼,有国有家者,其可以永鉴矣!

汉制,帝祖母曰太皇太后,帝母曰皇太后,帝妃曰皇后,其馀内官十有四等。魏因汉法,母后之号,皆如旧制,自夫人以下,世有增损。太祖建国,始命王后,其下五等:有夫人,有昭仪,有婕妤,有容华,有美人;文帝增贵嫔、淑媛、脩容、顺成、良人。明帝增淑妃、昭华、脩仪;除顺成官。太和中始复命夫人,登其位於淑妃之上。自夫人以下爵凡十二等:贵嫔、夫人,位次皇后,爵无所视;淑妃位视相国,爵比诸侯王;淑媛位视御史大夫,爵比县公;昭仪比县侯;昭华比乡侯;脩容比亭侯;脩仪比关内侯;婕妤视中二千石;容华视真二千石;美人视比二千石;良人视千石。

武宣卞皇后,琅邪开阳人,文帝母也。本倡家,年二十,太祖於谯纳后为妾。后随太祖至洛。及董卓为乱,太祖微服东出避难。袁术传太祖凶问,时太祖左右至洛者皆欲归,后止之曰:“曹君吉凶未可知,今日还家,明日若在,何面目复相见也?正使祸至,共死何苦!“遂从后言。太祖闻而善之。建安初,丁夫人废,遂以后为继室。诸子无母者,太祖皆令后养之。文帝为太子,左右长御贺后曰:“将军拜太子,天下莫不欢喜,后当倾府藏赏赐。”后曰:“王自以丕年大,故用为嗣,我但当以免无教导之过为幸耳,亦何为当重赐遗乎!”长御还,具以语太祖。太祖悦曰:“怒不变容,喜不失节,故是最为难。”

二十四年,拜为王后,策曰:“夫人卞氏,抚养诸子,有母仪之德。今进位王后,太子诸侯陪位,群卿上寿。减国内死罪一等。”二十五年,太祖崩,文帝即王位,尊后曰王太后。及践阼,尊后曰皇太后,称永寿宫。明帝即位,尊太后曰太皇太后。

黄初中,文帝欲追封太后父母,尚书陈群奏曰:“陛下以圣德应运受命,创业革制,当永为后式。案典籍之文,无妇人分土命爵之制。在礼典,妇因夫爵。秦违古法,汉氏因之,非先王之令典也。”帝曰:“此议是也,其勿施行。以作著诏下藏之台阁,永为后式。”至太和四年春,明帝乃追谥太后祖父广曰开阳恭侯,父远曰敬侯,祖母周封阳都君及敬侯夫人,皆赠印绶。其年五月,后崩。七月,合葬高陵。

初,太后弟秉,以功封都乡侯,黄初七年进封开阳侯,邑千二百户,为昭烈将军。秉薨,子兰嗣。少有才学,为奉车都尉、游击将军,加散骑常侍。兰薨,子晖嗣。又分秉爵,封兰弟琳为列侯,官至步兵校尉。兰子隆女为高贵乡公皇后,隆以后父为光禄大夫,位特进,封睢阳乡侯,妻王为显阳乡君。追封隆前妻刘为顺阳乡君,后亲母故也。琳女又为陈留王皇后,时琳已没,封琳妻刘为广阳乡君。

文昭甄皇后,中山无极人,明帝母,汉太保甄邯后也,世吏二千石。父逸,上蔡令。后三岁失父。后天下兵乱,加以饥馑,百姓皆卖金银珠玉宝物。时后家大有储谷,颇以买之。后年十馀岁,白母曰:“今世乱而多买宝物,匹夫无罪,怀璧为罪。又左右皆饥乏,不如以谷振给亲族邻里,广为恩惠也。”举家称善,即从后言。

建安中,袁绍为中子熙纳之。熙出为幽州,后留养姑。及冀州平,文帝纳后于邺,有宠,生明帝及东乡公主。延康元年正月,文帝即王位,六月,南征,后留邺。黄初元年十月,帝践阼。践阼之后,山阳公奉二女以嫔于魏,郭后、李、阴贵人并爱幸,后愈失意,有怨言。帝大怒,二年六月,遣使赐死,葬于邺。

明帝即位,有司奏请追谥,使司空王朗持节奉策以太牢告祠于陵,又别立寝庙。太和元年三月,以中山魏昌之安城乡户千,追封逸,谥曰敬侯;適孙像袭爵。四月,初营宗庙,掘地得玉玺,方一寸九分,其文曰“天子羡思慈亲”,明帝为之改容,以太牢告庙。又尝梦见后,於是差次舅氏亲疏高下,叙用各有差,赏赐累钜万;以像为虎贲中郎将。是月,后母薨,帝制緦服临丧,百僚陪位。四年十一月,以后旧陵庳下,使像兼太尉,持节诣邺,昭告后土,十二月,改葬朝阳陵。像还,迁散骑常侍。青龙二年春,追谥后兄俨曰安城乡穆侯。夏,吴贼寇扬州,以像为伏波将军,持节监诸将东征,还,复为射声校尉。三年薨,追赠卫将军,改封魏昌县,谥曰贞侯;子畅嗣。又封畅弟温、〈革韦〉、艳皆为列侯。四年,改逸、俨本封皆曰魏昌侯,谥因故。封俨世妇刘为东乡君,又追封逸世妇张为安喜君。

景初元年夏,有司议定七庙。冬,又奏曰:“盖帝王之兴,既有受命之君,又有圣妃协于神灵,然后克昌厥世,以成王业焉。昔高辛氏卜其四妃之子皆有天下,而帝挚、陶唐、商、周代兴。周人上推后稷,以配皇天,追述王初,本之姜嫄,特立宫庙,世世享尝,周礼所谓'奏夷则,歌中吕,舞大濩,以享先妣'者也。诗人颂之曰:‘厥初生民,时维姜嫄。’言王化之本,生民所由。又曰:‘閟宫有侐,实实枚枚,赫赫姜嫄,其德不回。’诗、礼所称姬宗之盛,其美如此。大魏期运,继于有虞,然崇弘帝道,三世弥隆,庙祧之数,实与周同。今武宣皇后、文德皇后各配无穷之祚,至於文昭皇后膺天灵符,诞育明圣,功济生民,德盈宇宙。开诸后嗣,乃道化之所兴也。寝庙特祀,亦姜嫄之閟宫也,而未著不毁之制,惧论功报德之义,万世或阙焉,非所以昭孝示后世也。文昭庙宜世世享祀奏乐,与祖庙同,永著不毁之典,以播圣善之风。“於是与七庙议并勒金策,藏之金匮。

帝思念舅氏不已。畅尚幼,景初末,以畅为射声校尉,加散骑常侍,又特为起大第,车驾亲自临之。又於其后园为像母起观庙,名其里曰渭阳里,以追思母氏也。嘉平三年正月,畅薨,追赠车骑将军,谥曰恭侯;子绍嗣。太和六年,明帝爱女淑薨,追封谥淑为平原懿公主,为之立庙。取后亡从孙黄与合葬,追封黄列侯,以夫人郭氏从弟德为之后,承甄氏姓,封德为平原侯,袭公主爵。青龙中,又封后从兄子毅及像弟三人,皆为列侯。毅数上疏陈时政,官至越骑校尉。嘉平中,复封畅子二人为列侯。后兄俨孙女为齐王皇后,后父已没,封后母为广乐乡君。

文德郭皇后,安平广宗人也。祖世长吏。后少而父永奇之曰:“此乃吾女中王也。”遂以女王为字。早失二亲,丧乱流离,没在铜鞮侯家。太祖为魏公时,得入东宫。后有智数,时时有所献纳。文帝定为嗣,后有谋焉。太子即王位,后为夫人,及践阼,为贵嫔。甄后之死,由后之宠也。黄初三年,将登后位,文帝欲立为后,中郎栈潜上疏曰:“在昔帝王之治天下,不惟外辅,亦有内助。治乱所由,盛衰从之。故西陵配黄,英娥降妫,并以贤明,流芳上世。桀奔南巢,祸阶末喜;纣以炮烙,怡悦妲己。是以圣哲慎立元妃,必取先代世族之家,择其令淑以统六宫,虔奉宗庙,阴教聿修。易曰:‘家道正而天下定。’由内及外,先王之令典也。春秋书宗人衅夏云,无以妾为夫人之礼。齐桓誓命于葵丘,亦曰'无以妾为妻'。今后宫嬖宠,常亚乘舆。若因爱登后,使贱人暴贵,臣恐后世下陵上替,开张非度,乱自上起也。“文帝不从,遂立为皇后。

后早丧兄弟,以从兄表继永后,拜奉车都尉。后外亲刘斐与他国为婚,后闻之,敕曰:“诸亲戚嫁娶,自当与乡里门户匹敌者,不得因势强与他方人婚也。”后姊子孟武还乡里,求小妻,后止之。遂敕诸家曰:“今世妇女少,当配将士,不得因缘取以为妾也。宜各自慎,无为罚首。”

五年,帝东征,后留许昌永始台。时霖雨百馀日,城楼多坏,有司奏请移止。后曰:“昔楚昭王出游,贞姜留渐台,江水至,使者迎而无符,不去,卒没。今帝在远,吾幸未有是患,而便移止,奈何?”群臣莫敢复言。六年,帝东征吴,至广陵,后留谯宫。时表留宿卫,欲遏水取鱼。后曰:“水当通运漕,又少材木,奴客不在目前,当复私取官竹木作梁遏。今奉车所不足者,岂鱼乎?”

明帝即位,尊后为皇太后,称永安宫。太和四年,诏封表安阳亭侯,又进爵乡侯,增邑并前五百户,迁中垒将军。以表子详为骑都尉。其年,帝追谥太后父永为安阳乡敬侯,母董为都乡君。迁表昭德将军,加金紫,位特进,表第二子训为骑都尉。及孟武母卒,欲厚葬,起祠堂,太后止之曰:“自丧乱以来,坟墓无不发掘,皆由厚葬也;首阳陵可以为法。”青龙三年春,后崩于许昌,以终制营陵。三月庚寅,葬首阳陵西。帝进表爵为观津侯,增邑五百,并前千户。迁详为驸马都尉。四年,追改封永为观津敬侯,世妇董为堂阳君。追封谥后兄浮为梁里亭戴侯,都为武城亭孝侯,成为新乐亭定侯,皆使使者奉策,祠以太牢。表薨,子详嗣,又分表爵封详弟述为列侯。详薨,子钊嗣。

明悼毛皇后,河内人也。黄初中,以选入东宫,明帝时为平原王,进御有宠,出入与同舆辇。及即帝立,以为贵嫔。太和元年,立为皇后。后父嘉,拜骑都尉,后弟曾,郎中。

初,明帝为王,始纳河内虞氏为妃,帝即位,虞氏不得立为后,太皇卞太后慰勉焉。虞氏曰:“曹氏自好立贱,未有能以义举者也。然后职内事,君听外政,其道相由而成,苟不能以善始,未有能令终者也。殆必由此亡国丧祀矣!”虞氏遂绌还邺宫。进嘉为奉车都尉,曾骑都尉,宠赐隆渥。顷之,封嘉博平乡侯,迁光禄大夫,曾驸马都尉。嘉本典虞车工,卒暴富贵,明帝令朝臣会其家饮宴,其容止举动甚蚩騃,语辄自谓“侯身”,时人以为笑。后又加嘉位特进,曾迁散骑侍郎。青龙三年,嘉薨,追赠光禄大夫,改封安国侯,增邑五百,并前千户,谥曰节侯。四年,追封后母夏为野王君。

帝之幸郭元后也,后爱宠日弛。景初元年,帝游后园,召才人以上曲宴极乐。元后曰“宜延皇后”,帝弗许。乃禁左右,使不得宣。后知之,明日,帝见后,后曰:“昨日游宴北园,乐乎?”帝以左右泄之,所杀十馀人。赐后死,然犹加谥,葬愍陵。迁曾散骑常侍,后徙为羽林虎贲中郎将、原武典农。

明元郭皇后,西平人也,世河右大族。黄初中,本郡反叛,遂没入宫。明帝即位,甚见爱幸,拜为夫人。叔父立为骑都尉,从父芝为虎贲中郎将。帝疾困,遂立为皇后。齐王即位,尊后为皇太后,称永宁宫,追封谥太后父满为西都定侯,以立子建绍其爵;封太后母杜为郃阳君。芝迁散骑常侍、长水校尉,立,宣德将军,皆封列侯。建兄德,出养甄氏。德及建俱为镇护将军,皆封列侯,并掌宿卫。值三主幼弱,宰辅统政,与夺大事,皆先咨启於太后而后施行。毌丘俭、锺会等作乱,咸假其命而以为辞焉。景元四年十二日崩,五年二月,葬高平陵西。

评曰:魏后妃之家,虽云富贵,未有若衰汉乘非其据,宰割朝政者也。鉴往易轨,於斯为美。追观陈群之议,栈潜之论,適足以为百王之规典,垂宪范乎后叶矣。

\part{魏书六}
\chapter{董二袁刘传第六}

董卓字仲颖,陇西临洮人也。少好侠,尝游羌中,尽与诸豪帅相结。后归耕於野,而豪帅有来从之者,卓与俱还,杀耕牛与相宴乐。诸豪帅感其意,归相敛,得杂畜千馀头以赠卓。汉桓帝末,以六郡良家子为羽林郎。卓有才武,旅力少比,双带两鞬,左右驰射,为军司马。从中郎将张奂征并州有功,拜郎中,赐缣九千匹,卓悉以分与吏士。迁广武令,蜀郡北部都尉,西域戊己校尉,免。徵拜并州刺史、河东太守,迁中郎将。讨黄巾,军败抵罪。韩遂等起凉州,复为中郎将,西拒遂。于望垣硖北,为羌、胡数万人所围,粮食乏绝。卓伪欲捕鱼,堰其还道当所渡水为池,使水渟满数十里,默从堰下过其军而决堰。比羌、胡闻知追逐,水已深,不得渡。时六军上陇西,五军败绩,卓独全众而还,屯住扶风。拜前将军,封斄乡侯,徵为并州牧。

灵帝崩,少帝即位。大将军何进与司隶校尉袁绍谋诛诸阉官,太后不从。进乃召卓使将兵诣京师,并密令上书曰:“中常侍张让等窃幸乘宠,浊乱海内。昔赵鞅兴晋阳之甲,以逐君侧之恶。臣辄鸣钟鼓如洛阳,即讨让等。”欲以胁迫太后。卓未至,进败。中常侍段珪等劫帝走小平津。卓遂将其众迎帝于北芒,还宫。时进弟车骑将军苗为进众所杀,进、苗部曲无所属,皆诣卓。卓又使吕布杀执金吾丁原,并其众,故京都兵权唯在卓。

先是,进遣骑都尉太山鲍信所在募兵,適至,信谓绍曰:“卓拥强兵,有异志,今不早图,将为所制;及其初至疲劳,袭之可禽也。”绍畏卓,不敢发,信遂还乡里。

於是以久不雨,策免司空刘弘而卓代之,俄迁太尉,假节钺虎贲。遂废帝为弘农王。寻又杀王及何太后。立灵帝少子陈留王,是为献帝。卓迁相国,封郿侯,赞拜不名,剑履上殿。又封卓母为池阳君,置家令、丞。卓既率精兵来,適值帝室大乱,得专废立,据有武库甲兵,国家珍宝,威震天下。卓性残忍不仁,遂以严刑胁众,睚眦之隙必报,人不自保。尝遣军到阳城,时適二月社,民各在其社下,悉就断其男子头,驾其车牛,载其妇女财物,以所断头系车辕轴,连轸而还洛,云攻贼大获,称万岁。入开阳城门,焚烧其头,以妇女与甲兵为婢妾。至于奸乱宫人公主。其凶逆如此。

初,卓信任尚书周毖,城门校尉伍琼等,用其所举韩馥、刘岱、孔伷、张咨、张邈等出宰州郡。而馥等至官,皆合兵将以讨卓。卓闻之,以为毖、琼等通情卖己,皆斩之。

河内太守王匡,遣泰山兵屯河阳津,将以图卓。卓遣疑兵若将於平阴渡者,潜遣锐众从小平北渡,绕击其后,大破之津北,死者略尽。卓以山东豪杰并起,恐惧不宁。初平元年二月,乃徙天子都长安。焚烧洛阳宫室,悉发掘陵墓,取宝物。卓至西京,为太师,号曰尚父。乘青盖金华车,爪画两轓,时人号曰竿摩车。卓弟旻为左将军,封鄠侯;兄子璜为侍中中军校尉典兵;宗族内外并列朝廷。公卿见卓,谒拜车下,卓不为礼。召呼三台尚书以下自诣卓府启事。筑郿坞,高与长安城埒,积谷为三十年储。云事成,雄据天下,不成,守此足以毕老。尝至郿坞,公卿已下祖道於横门外。卓豫施帐幔饮。诱降北地反者数百人,於坐中先断其舌,或斩手足,或凿眼,或镬煮之,未死,偃转杯案间,会者皆战栗亡失匕箸,而卓饮食自若。太史望气,言当有大臣戮死者。故太尉张温时为卫尉,素不善卓,卓心怨之,因天有变,欲以塞咎,使人言温与袁术交关,遂笞杀之。法令苛酷,爱憎淫刑,更相被诬,冤死者千数。百姓嗷嗷,道路以目。悉椎破铜人、钟虡,及坏五铢钱。更铸为小钱,大五分,无文章,肉好无轮郭,不磨鑢。于是货轻而物贵,谷一斛至数十万。自是后钱货不行。

三年四月,司徒王允、尚书仆射士孙瑞、卓将吕布共谋诛卓。是时,天子有疾新愈,大会未央殿。布使同郡骑都尉李肃等,将亲兵十馀人,伪著卫士服守掖门。布怀诏书。卓至,肃等格卓。卓惊呼布所在。布曰“有诏”,遂杀卓,夷三族。主簿田景前趋卓尸,布又杀之;凡所杀三人,馀莫敢动。长安士庶咸相庆贺,诸阿附卓者皆下狱死。

初,卓女婿中郎将牛辅典兵别屯陕,分遣校尉李傕、郭汜、张济略陈留、颍川诸县。卓死,吕布使李肃至陕,欲以诏命诛辅。辅等逆与肃战,肃败走弘农,布诛肃。其后辅营兵有夜叛出者,营中惊,辅以为皆叛,乃取金宝,独与素所厚攴胡赤儿等五六人相随。逾城北渡河,赤儿等利其金宝,斩首送长安。

比傕等还,辅已败,众无所依,欲各散归。既无赦书,而闻长安中欲尽诛凉州人,忧恐不知所为。用贾诩策,遂将其众而西,所在收兵,比至长安,众十馀万。与卓故部曲樊稠、李蒙、王方等合围长安城。十日城陷,与布战城中,布败走。傕等放兵略长安老少,杀之悉尽,死者狼籍。诛杀卓者,尸王允于市。葬卓于郿,大风暴雨震卓墓,水流入藏,漂其棺椁。傕为车骑将军、池阳侯,领司隶校尉、假节。汜为后将军、美阳侯。稠为右将军、万年侯。傕、汜、稠擅朝政。济为骠骑将军、平阳侯,屯弘农。

是岁,韩遂、马腾等降,率众诣长安。以遂为镇西将军,遣还凉州。腾征西将军,屯郿。侍中马宇与谏议大夫种邵、左中郎将刘范等谋,欲使腾袭长安,己为内应,以诛傕等。腾引兵至长平观,宇等谋泄,出奔槐里。稠击腾,腾败走,还凉州;又攻槐里,宇等皆死。时三辅民尚数十万户,傕等放兵劫略,攻剽城邑,人民饥困,二年间相啖食略尽。

诸将争权,遂杀稠,并其众。汜与傕转相疑,战斗长安中。傕质天子於营,烧宫殿城门,略官寺,尽收乘舆服御物置其家。傕使公卿诣汜请和,汜皆执之。相攻击连月,死者万数。

傕将杨奉与傕军吏宋果等谋杀傕,事泄,遂将兵叛傕。傕众叛,稍衰弱。张济自陕和解之,天子乃得出,至新丰、霸陵间。郭汜复欲胁天子还都郿。天子奔奉营,奉击汜破之。汜走南山,奉及将军董承以天子还洛阳。傕、汜悔遣天子,复相与和,追及天子於弘农之曹阳。奉急招河东故白波帅韩暹、胡才、李乐等合,与傕、汜大战。奉兵败,傕等纵兵杀公卿百官,略宫人入弘农。天子走陕,北渡河,失辎重,步行,唯皇后贵人从。至大阳,止人家屋中。奉、暹等遂以天子都安邑,御乘牛车。太尉杨彪、太仆韩融近臣从者十馀人。以暹为征东、才为征西、乐征北将军,并与奉、承持政。遣融至弘农,与傕、汜等连和,还所略宫人公卿百官,及乘舆车马数乘。是时蝗虫起,岁旱无谷,从官食枣菜。诸将不能相率,上下乱,粮食尽。奉、暹、承乃以天子还洛阳。出箕关,下轵道,张杨以食迎道路,拜大司马。语在杨传。天子入洛阳,宫室烧尽,街陌荒芜,百官披荆棘,依丘墙间。州郡各拥兵自卫,莫有至者。饥穷稍甚,尚书郎以下,自出樵采,或饥死墙壁间。

太祖乃迎天子都许。暹、奉不能奉王法,各出奔,寇徐、扬间,为刘备所杀。董承从太祖岁馀,诛。建安二年,遣谒者仆射裴茂率关西诸将诛傕,夷三族。汜为其将五习所袭,死于郿。济饥饿,至南阳寇略,为穰人所杀,从子绣摄其众。才、乐留河东,才为怨家所杀,乐病死。遂、腾自还凉州,更相寇,后腾入为卫尉,子超领其部曲。十六年,超与关中诸将及遂等反,太祖征破之。语在武纪。遂奔金城,为其将所杀。超据汉阳,腾坐夷三族。赵衢等举义兵讨超,超走汉中从张鲁,后奔刘备,死于蜀。

袁绍字本初,汝南汝阳人也。高祖父安,为汉司徒。自安以下四世居三公位,由是势倾天下。绍有姿貌威容,能折节下士,士多附之,太祖少与交焉。以大将军掾为侍御史,稍迁中军校尉,至司隶。

灵帝崩,太后兄大将军何进与绍谋诛诸阉官,太后不从。乃召董卓,欲以胁太后。常侍、黄门闻之,皆诣进谢,唯所错置。时绍劝进便可於此决之,至于再三,而进不许。令绍使洛阳方略武吏,检司诸宦者。又令绍弟虎贲中郎将术选温厚虎贲二百人,当入禁中,代持兵黄门陛守门户。中常侍段珪等矫太后命,召进入议,遂杀之,宫中乱。术将虎贲烧南宫嘉德殿青琐门,欲以迫出珪等。珪等不出,劫帝及帝弟陈留王走小平津。绍既斩宦者所署司隶校尉许相,遂勒兵捕诸阉人,无少长皆杀之。或有无须而误死者,至自发露形体而后得免。宦者或有行善自守而犹见及。其滥如此。死者二千馀人。急追珪等,珪等悉赴河死。帝得还宫。

董卓呼绍,议欲废帝,立陈留王。是时绍叔父隗为太傅,绍伪许之,曰:“此大事,出当与太傅议。”卓曰:“刘氏种不足复遗。”绍不应,横刀长揖而去。绍既出,遂亡奔冀州。侍中周毖、城门校尉伍琼、议郎何颙等,皆名士也,卓信之,而阴为绍,乃说卓曰:“夫废立大事,非常人所及。绍不达大体,恐惧故出奔,非有他志也。今购之急,势必为变。袁氏树恩四世,门世故吏遍於天下,若收豪杰以聚徒众,英雄因之而起,则山东非公之有也。不如赦之,拜一郡守,则绍喜于免罪,必无患矣。”卓以为然,乃拜绍勃海太守,封邟乡侯。

绍遂以勃海起兵,将以诛卓。语在武纪。绍自号车骑将军,主盟,与冀州牧韩馥立幽州牧刘虞为帝,遣使奉章诣虞,虞不敢受。后馥军安平,为公孙瓒所败。瓒遂引兵入冀州,以讨卓为名,内欲袭馥。馥怀不自安。会卓西入关,绍还军延津,因馥惶遽,使陈留高幹、颍川荀谌等说馥曰:“公孙瓒乘胜来向南,而诸郡应之,袁车骑引军东向,此其意不可知,窃为将军危之。”馥曰:“为之奈何?”谌曰:“公孙提燕、代之卒,其锋不可当。袁氏一时之杰,必不为将军下。夫冀州,天下之重资也,若两雄并力,兵交於城下,危亡可立而待也。夫袁氏,将军之旧,且同盟也,当今为将军计,莫若举冀州以让袁氏。袁氏得冀州,则瓒不能与之争,必厚德将军。冀州入於亲交,是将军有让贤之名,而身安於泰山也。愿将军勿疑!”馥素恇怯,因然其计。馥长史耿武、别驾闵纯、治中李历谏馥曰:“冀州虽鄙,带甲百万,谷支十年。袁绍孤客穷车,仰我鼻息,譬如婴儿在股掌之上,绝其哺乳,立可饿杀。奈何乃欲以州与之?”馥曰:“吾,袁氏故吏,且才不如本初,度德而让,古人所贵,诸君独何病焉!”从事赵浮、程奂请以兵拒之,馥又不听。乃让绍,绍遂领冀州牧。

从事沮授说绍曰:“将军弱冠登朝,则播名海内;值废立之际,则忠义奋发。单骑出奔,则董卓怀怖;济河而北,则勃海稽首。振一郡之卒,撮冀州之众,威震河朔,名重天下。虽黄巾猾乱,黑山跋扈,举军东向,则青州可定;还讨黑山,则张燕可灭;回众北首,则公孙必丧;震胁戎狄,则匈奴必从。横大河之北,合四州之地,收英雄之才,拥百万之众,迎大驾於西京,复宗庙於洛邑,号令天下,以讨未复,以此争锋,谁能敌之?比及数年,此功不难。”绍喜曰:“此吾心也。”即表授为监军、奋威将军。卓遣执金吾胡母班、将作大匠吴脩赍诏书喻绍,绍使河内太守王匡杀之。卓闻绍得关东,乃悉诛绍宗族太傅隗等。当是时,豪侠多附绍,皆思为之报。州郡蜂起,莫不假其名。馥怀惧,从绍索去,往依张邈。后绍遣使诣邈,有所计议,与邈耳语。馥在坐上,谓见图构,无何起至溷自杀。

初,天子之立非绍意,及在河东,绍遣颍川郭图使焉。图还说绍迎天子都邺,绍不从。会太祖迎天子都许,收河南地,关中皆附,绍悔。欲令太祖徙天子都鄄城以自密近,太祖拒之。天子以绍为太尉,转为大将军,封邺侯,绍让侯不受。顷之,击破瓒于易京,并其众。出长子谭为青州,沮授谏绍:“必为祸始。”绍不听,曰:“孤欲令诸儿各据一州也。”又以中子熙为幽州,甥高幹为并州。众数十万,以审配、逢纪统军事,田丰、荀谌、许攸为谋主,颜良、文丑为将率,简精卒十万,骑万匹,将攻许。

先是,太祖遣刘备诣徐州拒袁术。术死,备杀刺史车胄,引军屯沛。绍遣骑佐之。太祖遣刘岱、王忠击之,不克。建安五年,太祖自东征备。田丰说绍袭太祖后,绍辞以子疾,不许,丰举杖击地曰:“夫遭难遇之机,而以婴儿之病失其会,惜哉!”太祖至,击破备;备奔绍。

绍进军黎阳,遣颜良攻刘延于白马。沮授又谏绍:“良性促狭,虽骁勇不可独任。”绍不听。太祖救延,与良战,破斩良。绍渡河,壁延津南,使刘备、文丑挑战。太祖击破之,斩丑,再战,禽绍大将。绍军大震。太祖还官渡。沮授又曰:“北兵数众而果劲不及南,南谷虚少而货财不及北;南利在於急战,北利在於缓搏。宜徐持久,旷以日月。”绍不从。连营稍前,逼官渡,合战,太祖军不利,复壁。绍为高橹,起土山,射营中,营中皆蒙楯,众大惧。太祖乃为发石车,击绍楼,皆破,绍众号曰霹雳车。绍为地道,欲袭太祖营。太祖辄於内为长堑以拒之,又遣奇兵袭击绍运车,大破之,尽焚其谷。太祖与绍相持日久,百姓疲乏,多叛应绍,军食乏。会绍遣淳于琼等将兵万馀人北迎运车,沮授说绍:“可遣将蒋奇别为支军於表,以断曹公之钞。”绍复不从。琼宿乌巢,去绍军四十里。太祖乃留曹洪守,自将步骑五千候夜潜往攻琼。绍遣骑救之,败走。破琼等,悉斩之。太祖还,未至营,绍将高览、张郃等率其众降。绍众大溃,绍与谭单骑退渡河。馀众伪降,尽坑之。沮授不及绍渡,为人所执,诣太祖,太祖厚待之。后谋还袁氏,见杀。

初,绍之南也,田丰说绍曰:“曹公善用兵,变化无方,众虽少,未可轻也,不如以久持之。将军据山河之固,拥四州之众,外结英雄,内脩农战,然后简其精锐,分为奇兵,乘虚迭出,以扰河南,救右则击其左,救左则击其右,使敌疲於奔命,民不得安业;我未劳而彼已困,不及二年,可坐克也。今释庙胜之策,而决成败於一战,若不如志,悔无及也。”绍不从。丰恳谏,绍怒甚,以为沮众,械系之。绍军既败,或谓丰曰:“君必见重。”丰曰:“若军有利,吾必全,今军败,吾其死矣。”绍还,谓左右曰:“吾不用田丰言,果为所笑。”遂杀之。绍外宽雅,有局度,忧喜不形于色,而内多忌害,皆此类也。

冀州城邑多叛,绍复击定之。自军败后发病,七年,忧死。

绍爱少子尚,貌美,欲以为后而未显。审配、逢纪与辛评、郭图争权,配、纪与尚比,评、图与谭比。众以谭长,欲立之。配等恐谭立而评等为己害,缘绍素意,乃奉尚代绍位。谭至,不得立,自号车骑将军。由是谭、尚有隙。太祖北征谭、尚。谭军黎阳,尚少与谭兵,而使逢纪从谭。谭求益兵,配等议不与。谭怒,杀纪。太祖渡河攻谭,谭告急於尚。尚欲分兵益谭,恐谭遂夺其众,乃使审配守邺,尚自将兵助谭,与太祖相拒於黎阳。自九月至二月,大战城尚败退,入城守。太祖将围之,乃夜遁。追至邺,收其麦,拔阴安,引军还许。太祖南征荆州,军至西平,谭、尚遂举兵相攻,谭败奔平原。尚攻之急,谭遣辛毗诣太祖请救。太祖乃还救谭,十月至黎阳。尚闻太祖北,释平原还邺。其将吕旷、吕翔叛尚归太祖,谭复阴刻将军印假旷、翔。太祖知谭诈,与结婚以安之,乃引军还。尚使审配、苏由守邺,复攻谭平原。太祖进军将攻邺,到洹水,去邺五十里,由欲为内应,谋泄,与配战城中,败,出奔太祖。太祖遂进攻之,为地道,配亦於内作堑以当之。配将冯礼开突门,内太祖兵三百馀人,配觉之,从城上以大石击突中栅门,栅门闭,入者皆没。太祖遂围之,为堑,周四十里,初令浅,示若可越。配望而笑之,不出争利。太祖一夜掘之,广深二丈,决漳水以灌之。自五月至八月,城中饿死者过半。尚闻邺急,将兵万馀人还救之,依西山来,东至阳平亭,去邺十七里,临滏水。举火以示城中,城中亦举火相应。配出兵城北,欲与尚对决围。太祖逆击之,败还,尚亦破走,依曲漳为营,太祖遂围之。未合,尚惧,遣阴夔、陈琳乞降,不听。尚还走滥口,进复围之急,其将马延等临阵降,众大溃,尚奔中山。尽收其辎重,得尚印绶、节钺及衣物,以示其家,城中崩沮。配兄子荣守东门,夜开门内太祖兵,与配战城中,生禽配。配声气壮烈,终无挠辞,见者莫不叹息。遂斩之。高幹以并州降,复以幹为刺史。

太祖之围邺也,谭略取甘陵、安平、勃海、河间,攻尚於中山。尚走故安从熙,谭悉收其众。太祖将讨之,谭乃拔平原,并南皮,自屯龙凑。十二月,太祖军其门,谭不出,夜遁奔南皮,临清河而屯。十年正月,攻拔之,斩谭及图等。熙、尚为其将焦触、张南所攻,奔辽西乌丸。触自号幽州刺史,驱率诸郡太守令长,背袁向曹。陈兵数万,杀白马盟,令曰:“违命者斩!”众莫敢语,各以次歃。至别驾韩珩,曰:“吾受袁公父子厚恩,今其破亡,智不能救,勇不能死,於义阙矣;若乃北面於曹氏,所弗能为也。”一坐为珩失色。触曰:“夫兴大事,当立大义,事之济否,不待一人,可卒珩志,以励事君。”高幹叛,执上党太守,举兵守壶口关。遣乐进、李典击之,未拔。十一年,太祖征幹。幹乃留其将夏昭、邓升守城,自诣匈奴单于求救,不得,独与数骑亡,欲南奔荆州,上洛都尉捕斩之。十二年,太祖至辽西击乌丸。尚、熙与乌丸逆军战,败走奔辽东,公孙康诱斩之,送其首。太祖高韩珩节,屡辟不至,卒於家。

袁术字公路,司空逢子,绍之从弟也。以侠气闻。举孝廉,除郎中,历职内外,后为折冲校尉、虎贲中郎将。董卓之将废帝,以术为后将军;术亦畏卓之祸,出奔南阳。会长沙太守孙坚杀南阳太守张咨,术得据其郡。南阳户口数百万,而术奢淫肆欲,徵敛无度,百姓苦之。既与绍有隙,又与刘表不平而北连公孙瓒;绍与瓒不和而南连刘表。其兄弟携贰,舍近交远如此。引军入陈留,太祖与绍合击,大破术军。术以馀众奔九江,杀扬州刺史陈温,领其州。以张勋、桥蕤等为大将军。李傕入长安,欲结术为援,以术为左将军,封阳翟侯,假节,遣太傅马日磾因循行拜授。术夺日磾节,拘留不遣。

时沛相下邳陈珪,故太尉球弟子也。术与珪俱公族子孙,少共交游,书与珪曰:“昔秦失其政,天下群雄争而取之,兼智勇者卒受其归。今世事纷扰,复有瓦解之势矣,诚英乂有为之时也。与足下旧交,岂肯左右之乎?若集大事,子实为吾心膂。”珪中子应时在下邳,术并胁质应,图必致珪。珪答书曰:“昔秦末世,肆暴恣情,虐流天下,毒被生民,下不堪命,故遂土崩。今虽季世,未有亡秦苛暴之乱也。曹将军神武应期,兴复典刑,将拨平凶慝,清定海内,信有徵矣。以为足下当戮力同心,匡翼汉室,而阴谋不轨,以身试祸,岂不痛哉!若迷而知反,尚可以免。吾备旧知,故陈至情,虽逆于耳,骨肉之惠也。欲吾营私阿附,有犯死不能也。”

兴平二年冬,天子败於曹阳。术会群下谓曰:“今刘氏微弱,海内鼎沸。吾家四世公辅,百姓所归,欲应天顺民,於诸君意如何?”众莫敢对。主簿阎象进曰:“昔周自后稷至于文王,积德累功,三分天下有其二,犹服事殷。明公虽奕世克昌,未若有周之盛,汉室虽微,未若殷纣之暴也。”术嘿然不悦。用河内张炯之符命,遂僣号以九江太守为淮南尹。置公卿,祠南北郊。荒侈滋甚,后宫数百皆服绮縠,馀粱肉,而士卒冻馁。江淮间空尽,人民相食。术前为吕布所破,后为太祖所败,奔其部曲雷薄、陈兰于灊山,复为所拒,忧惧不知所出。将归帝号於绍,欲至青州从袁谭,发病道死。妻子依术故吏庐江太守刘勋,孙策破勋,复见收视。术女入孙权宫,子耀拜郎中,耀女又配於权子奋。

刘表字景升,山阳高平人也。少知名,号八俊。长八尺馀,姿貌甚伟。以大将军掾为北军中候。灵帝崩,代王叡为荆州刺史。是时山东兵起,表亦合兵军襄阳。袁术之在南阳也,与孙坚合从,欲袭夺表州,使坚攻表。坚为流矢所中死,军败,术遂不能胜表。李傕、郭汜入长安,欲连表为援,乃以表为镇南将军、荆州牧,封成武侯,假节。天子都许,表虽遣使贡献,然北与袁绍相结。治中邓羲谏表,表不听,羲辞疾而退,终表之世。张济引兵入荆州界,攻穰城,为流矢所中死。荆州官属皆贺,表曰:“济以穷来,主人无礼,至于交锋,此非牧意,牧受吊,不受贺也。”使人纳其众;众闻之喜,遂服从。长沙太守张羡叛表,表围之连年不下。羡病死,长沙复立其子怿,表遂攻并怿,南收零、桂,北据汉川,地方数千里,带甲十馀万。

太祖与袁绍方相持于官渡,绍遣人求助,表许之而不至,亦不佐太祖,欲保江汉间,观天下变。从事中郎韩嵩、别驾刘先说表曰:“豪杰并争,两雄相持,天下之重,在於将军。将军若欲有为,起乘其弊可也;若不然,固将择所从。将军拥十万之众,安坐而观望。夫见贤而不能助,请和而不得,此两怨必集於将军,将军不得中立矣。夫以曹公之明哲,天下贤俊皆归之,其势必举袁绍。然后称兵以向江汉,恐将军不能御也。故为将军计者,不若举州以附曹公,曹公必重德将军;长享福祚,垂之后嗣,此万全之策也。”表大将蒯越亦劝表,表狐疑,乃遣嵩诣太祖以观虚实。嵩还,深陈太祖威德,说表遣子入质。表疑嵩反为太祖说,大怒,欲杀嵩,考杀随嵩行者,知嵩无他意,乃止。表虽外貌儒雅,而心多疑忌,皆此类也。

刘备奔表,表厚待之,然不能用。建安十三年,太祖征表,未至,表病死。

初,表及妻爱少子琮,欲以为后,而蔡瑁、张允为之支党,乃出长子琦为江夏太守,众遂奉琮为嗣。琦与琮遂为雠隙。越、嵩及东曹掾傅巽等说琮归太祖,琮曰:“今与诸君据全楚之地,守先君之业,以观天下,何为不可乎?”巽对曰:“逆顺有大体,强弱有定势。以人臣而拒人主,逆也;以新造之楚而御国家,其势弗当也。以刘备而敌曹公,又弗当也。三者皆短,欲以抗王兵之锋,必亡之道也。将军自料何与刘备?”琮曰:“吾不若也。”巽曰:“诚以刘备不足御曹公乎,则虽保楚之地,不足以自存也;诚以刘备足御曹公乎,则备不为将军下也。愿将军勿疑。”太祖军到襄阳,琮举州降。备走奔夏口。

太祖以琮为青州刺史、封列侯。蒯越等侯者十五人。越为光禄勋;嵩,大鸿胪;羲,侍中;先,尚书令;其馀多至大官。

评曰:董卓狼戾贼忍,暴虐不仁,自书契已来,殆未之有也。袁术奢淫放肆,荣不终己,自取之也。袁绍、刘表,咸有威容、器观,知名当世。表跨蹈汉南,绍鹰扬河朔,然皆外宽内忌,好谋无决,有才而不能用,闻善而不能纳,废嫡立庶,舍礼崇爱,至于后嗣颠蹙,社稷倾覆,非不幸也。昔项羽背范增之谋,以丧其王业;绍之杀田丰,乃甚於羽远矣!

\part{魏书七}
\chapter{吕布*(张邈)*臧洪传第七}

吕布张邈臧洪传原文
原文 ⇛ 段译

吕布字奉先,五原郡九原人也。以骁武给并州。刺史丁原为骑都尉,屯河内,以布为主簿,大见亲待。灵帝崩,原将兵诣洛阳。与何进谋诛诸黄门,拜执金吾。进败,董卓入京都,将为乱,欲杀原,并其兵众。卓以布见信于原,诱布令杀原。布斩原首诣卓,卓以布为骑都尉,甚爱信之,誓为父子。

布便弓马,膂力过人,号为飞将。稍迁至中郎将,封都亭侯。卓自以遇人无礼,恐人谋己,行止常以布自卫。然卓性刚而褊,忿不思难,尝小失意,拔手戟掷布。布拳捷避之,为卓顾谢,卓意亦解。由是阴怨卓。卓常使布守中閤,布与卓侍婢私通,恐事发觉,心不自安。

先是,司徒王允以布州里壮健,厚接纳之。后布诣允,陈卓几见杀状。时允与仆射士孙瑞密谋诛卓,是以告布使为内应。布曰:“奈如父子何!”允曰:“君自姓吕,本非骨肉。今忧死不暇,何谓父子?”布遂许之,手刃刺卓。语在卓传。允以布为奋武将军,假节,仪比三司,进封温侯,共秉朝政。布自杀卓后,畏恶凉州人,凉州人皆怨。由是李傕等遂相结还攻长安城。布不能拒,傕等遂入长安。卓死后六旬,布亦败。将数百骑出武关,欲诣袁术。

布自以杀卓为术报雠,欲以德之。术恶其反覆,拒而不受。北诣袁绍,绍与布击张燕于常山。燕精兵万馀,骑数千。布有良马曰赤兔。常与其亲近成廉、魏越等陷锋突陈,遂破燕军。而求益兵众,将士钞掠,绍患忌之。布觉其意,从绍求去。绍恐还为己害,遣壮士夜掩杀布,不获。事露,布走河内,与张杨合。绍令众追之,皆畏布,莫敢逼近者。

张邈字孟卓,东平寿张人也。少以侠闻,振穷救急,倾家无爱,士多归之。太祖、袁绍皆与邈友。辟公府,以高第拜骑都尉,迁陈留太守。董卓之乱,太祖与邈首举义兵。汴水之战,邈遣卫兹将兵随太祖。袁绍既为盟主,有骄矜色,邈正议责绍。绍使太祖杀邈,太祖不听,责绍曰:“孟卓,亲友也,是非当容之。今天下未定,不宜自相危也。“邈知之,益德太祖。太祖之征陶谦,敕家曰;“我若不还,往依孟卓。”后还,见邈,垂泣相对。其亲如此。

吕布之舍袁绍从张杨也,过邈临别,把手共誓。绍闻之,大恨。邈畏太祖终为绍击己也,心不自安。兴平元年,太祖复征谦,邈弟超,与太祖将陈宫、从事中郎许汜、王楷共谋叛太祖。宫说邈曰:“今雄杰并起,天下分崩,君以千里之众,当四战之地,抚剑顾眄,亦足以为人豪,而反制于人,不以鄙乎!今州军东征,其处空虚,吕布壮士,善战无前,若权迎之,共牧兖州,观天下形势,俟时事之变通,此亦纵横之一时也。“邈从之。太祖初使宫将兵留屯东郡,遂以其众东迎布为兖州牧,据濮阳。郡县皆应,唯鄄城、东阿、范为太祖守。太祖引军还,与布战於濮阳,太祖军不利,相持百馀日。是时岁旱、虫蝗、少谷,百姓相食。布东屯山阳。二年间,太祖乃尽复收诸城,击破布于钜野。布东奔刘备。邈从布,留超将家属屯雍丘。太祖攻围数月,屠之,斩超及其家。邈诣袁术请救未至,自为其兵所杀。

备东击术,布袭取下邳,备还归布。布遣备屯小沛。布自称徐州刺史。术遣将纪灵等步骑三万攻备,备求救于布。布诸将谓布曰:“将军常欲杀备,今可假手於术。”布曰:“不然。术若破备,则北连太山诸将,吾为在术围中,不得不救也。”便严步兵千、骑二百,驰往赴备。灵等闻布至,皆敛兵不敢复攻。布於沛西南一里安屯,遣铃下请灵等,灵等亦请布共饮食。布谓灵等曰:“玄德,布弟也。弟为诸君所困,故来救之。布性不喜合斗,但喜解斗耳。”布令门候于营门中举一只戟,布言:“诸君观布射戟小支,一发中者诸君当解去,不中可留决斗。”布举弓射戟,正中小支。诸将皆惊,言“将军天威也”!明日复欢会,然后各罢。

术欲结布为援,乃为子索布女,布许之。术遣使韩胤以僣号议告布,并求迎妇。沛相陈珪恐术、布成婚,则徐、扬合从,将为国难,於是往说布曰;“曹公奉迎天子,辅赞国政,威灵命世,将征四海,将军宜与协同策谋,图太山之安。今与术结婚,受天下不义之名,必有累卵之危。”布亦怨术初不己受也,女已在涂,追还绝婚,械送韩胤,枭首许市。珪欲使子登诣太祖,布不肯遣。会使者至,拜布左将军。布大喜,即听登往,并令奉章谢恩。登见太祖,因陈布勇而无计,轻於去就,宜早图之。太祖曰:“布,狼子野心,诚难久养,非卿莫能究其情也。”即增珪秩中二千石,拜登广陵太守。临别,太祖执登手曰:“东方之事,便以相付。”令登阴合部众以为内应。

始,布因登求徐州牧,登还,布怒,拔戟斫几曰:“卿父劝吾协同曹公,绝婚公路;今吾所求无一获,而卿父子并显重,为卿所卖耳!卿为吾言,其说云何?”登不为动容,徐喻之曰;“登见曹公言:‘待将军譬如养虎,当饱其肉,不饱则将噬人。’公曰:‘不如卿言也。譬如养鹰,饥则为用,饱则扬去。’其言如此。”布意乃解。

术怒,与韩暹、杨奉等连势,遣大将张勋攻布。布谓珪曰:“今致术军,卿之由也,为之奈何?”珪曰:“暹、奉与术,卒合之军耳,策谋不素定,不能相维持,子登策之,比之连鸡,势不俱栖,可解离也。”布用珪策,遣人说暹、奉,使与己并力共击术军,军资所有,悉许暹、奉。於是暹、奉从之,勋大破败。

建安三年,布复叛为术,遣高顺攻刘备於沛,破之。太祖遣夏侯惇救备,为顺所败。太祖自征布,至其城下,遗布书,为陈祸福。布欲降,陈宫等自以负罪深,沮其计。布遣人求救于术,自将千馀骑出战,败走,还保城,不敢出。术亦不能救。布虽骁猛,然无谋而多猜忌,不能制御其党,但信诸将。诸将各异意自疑,故每战多败。太祖堑围之三月,上下离心,其将侯成、宋宪、魏续缚陈宫,将其众降。布与其麾下登白门楼,兵围急,乃下降。遂生缚布,布曰:“缚太急,小缓之。”太祖曰:“缚虎不得不急也。”布请曰:“明公所患不过於布,今已服矣,天下不足忧。明公将步,令布将骑,则天下不足定也。”太祖有疑色。刘备进曰:“明公不见布之事丁建阳及董太师乎!“太祖颔之。布因指备曰:“是儿最叵信者。”於是缢杀布。布与宫、顺等皆枭首送许,然后葬之。

太祖之禽宫也,问宫欲活老母及女不?宫对曰:“宫闻孝治天下者不绝人之亲,仁施四海者不乏人之祀,老母在公,不在宫也。”太祖召养其母终其身,嫁其女。

陈登者,字元龙,在广陵有威名。又掎角吕布有功,加伏波将军,年三十九卒。后许汜与刘备并在荆州牧刘表坐,表与备共论天下人,汜曰:“陈元龙湖海之士,豪气不除。”备谓表曰:“许君论是非?”表曰:“欲言非,此君为善士,不宜虚言;欲言是,元龙名重天下。”备问汜:“君言豪,宁有事邪?”汜曰:“昔遭乱过下邳,见元龙。元龙无客主之意,久不相与语,自上大床卧,使客卧下床。”备曰:“君有国士之名,今天下大乱,帝主失所,望君忧国忘家,有救世之意,而君求田问舍,言无可采,是元龙所讳也,何缘当与君语?如小人,欲卧百尺楼上,卧君於地,何但上下床之间邪?”表大笑。备因言曰:“若元龙文武胆志,当求之於古耳,造次难得比也。”

臧洪字子源,广陵射阳人也。父旻,历匈奴中郎将、中山、太原太守,所在有名。洪体貌魁梧,有异於人,举孝廉为郎。时选三署郎以补县长;琅邪赵昱为莒长,东莱刘繇下邑长,东海王朗菑丘长,洪即丘长。灵帝末,弃官还家,太守张超请洪为功曹。

董卓杀帝,图危社稷,洪说超曰:“明府历世受恩,兄弟并据大郡,今王室将危,贼臣未枭,此诚天下义烈报恩效命之秋也。今郡境尚全,吏民殷富,若动枹鼓,可得二万人,以此诛除国贼,为天下倡先,义之大者也。”超然其言,与洪西至陈留,见兄邈计事。邈亦素有心,会于酸枣,邈谓超曰:“闻弟为郡守,政教威恩,不由己出,动任臧洪,洪者何人?”超曰:“洪才略智数优超,超甚爱之,海内奇士也。”邈即引见洪,与语大异之。致之于刘兖州公山、孔豫州公绪,皆与洪亲善。乃设坛场,方共盟誓。诸州郡更相让,莫敢当,咸共推洪。洪乃升坛操槃歃血而盟曰:“汉室不幸,皇纲失统,贼臣董卓乘衅纵害,祸加至尊,虐流百姓,大惧沦丧社稷,翦覆四海。兖州刺史岱、豫州刺史伷、陈留太守邈、东郡太守瑁、广陵太守超等,纠合义兵,并赴国难。凡我同盟,齐心戮力,以致臣节,殒首丧元,必无二志。有渝此盟,俾坠其命,无克遗育。皇天后土,祖宗明灵,实皆鉴之!”洪辞气慷慨,涕泣横下,闻其言者,虽卒伍厮养,莫不激扬,人思致节。顷之,诸军莫適先进,而食尽众散。

超遣洪诣大司马刘虞谋,值公孙瓒之难,至河间,遇幽、冀二州交兵,使命不达。而袁绍见洪,又奇重之,与结分合好。会青州刺史焦和卒,绍使洪领青州以抚其众。洪在州二年,群盗奔走。绍叹其能,徙为东郡太守,治东武阳。

太祖围张超于雍丘,超言:“唯恃臧洪,当来救吾。”众人以为袁、曹方睦,而洪为绍所表用,必不败好招祸,远来赴此。超曰:“子源,天下义士,终不背本者,但恐见禁制,不相及逮耳。”洪闻之,果徒跣号泣,并勒所领兵,又从绍请兵马,求欲救超,而绍终不听许。超遂族灭。洪由是怨绍,绝不与通。绍兴兵围之,历年不下。绍令洪邑人陈琳书与洪,喻以祸福,责以恩义。洪答曰:

隔阔相思,发于寤寐。幸相去步武之间耳,而以趣舍异规,不得相见,其为怆悢,可为心哉!前日不遗,比辱雅贶,述叙祸福,公私切至。所以不即奉答者,既学薄才钝,不足塞诘;亦以吾子携负侧室,息肩主人,家在东州,仆为仇敌。以是事人,虽披中情,堕肝胆,犹身疏有罪,言甘见怪,方首尾不救,何能恤人?且以子之才,穷该典籍,岂将闇于大道,不达余趣哉!然犹复云云者,仆以是知足下之言,信不由衷,将以救祸也。必欲算计长短,辩谘是非,是非之论,言满天下,陈之更不明,不言无所损。又言伤告绝之义,非吾所忍行也,是以捐弃纸笔,一无所答。亦冀遥忖其心,知其计定,不复渝变也。重获来命,援引古今,纷纭六纸,虽欲不言,焉得已哉!

仆小人也,本因行役,寇窃大州,恩深分厚,宁乐今日自还接刃!每登城勒兵,望主人之旗鼓,感故友之周旋,抚弦搦矢,不觉流涕之覆面也。何者?自以辅佐主人,无以为悔。主人相接,过绝等伦。当受任之初,自谓究竟大事,共尊王室。岂悟天子不悦,本州见侵,郡将遘牖里之厄。陈留克创兵之谋,谋计栖迟,丧忠孝之名,杖策携背,亏交友之分。揆此二者,与其不得已,丧忠孝之名与亏交友之道,轻重殊涂,亲疏异画,故便收泪告绝。若使主人少垂故人,住者侧席,去者克己,不汲汲于离友,信刑戮以自辅,则仆抗季札之志,不为今日之战矣。何以效之?昔张景明亲登坛喢血,奉辞奔走,卒使韩牧让印,主人得地;然后但以拜章朝主,赐爵获传之故,旋时之间,不蒙观过之贷,而受夷灭之祸。吕奉先讨卓来奔,请兵不获,告去何罪?复见斫刺,滨于死亡。刘子琪奉使逾时,辞不获命,畏威怀亲,以诈求归,可谓有志忠孝,无损霸道者也;然辄僵毙麾下,不蒙亏除。仆虽不敏,又素不能原始见终,睹微知著,窃度主人之心,岂谓三子宜死,罚当刑中哉?实且欲一统山东,增兵讨雠,惧战士狐疑,无以沮劝,故抑废王命以崇承制,慕义者蒙荣,待放者被戮,此乃主人之利,非游士之原也。故仆鉴戒前人,困穷死战。仆虽下愚,亦尝闻君子之言矣,此实非吾心也,乃主人招焉。凡吾所以背弃国民,用命此城者,正以君子之违,不適敌国故也。是以获罪主人,见攻逾时,而足下更引此义以为吾规,无乃辞同趋异,非君子所为休戚者哉!

吾闻之也,义不背亲,忠不违君,故东宗本州以为亲援,中扶郡将以安社稷,一举二得以徼忠孝,何以为非?而足下欲吾轻本破家,均君主人。主人之於我也,年为吾兄,分为笃友,道乖告去,以安君亲,可谓顺矣。若子之言,则包胥宜致命於伍员,不当号哭於秦庭矣。苟区区於攘患,不知言乖乎道理矣。足下或者见城围不解,救兵未至,感婚姻之义,惟平生之好,以屈节而苟生,胜守义而倾覆也。昔晏婴不降志於白刃,南史不曲笔以求生,故身著图象,名垂后世,况仆据金城之固,驱士民之力,散三年之畜,以为一年之资,匡困补乏,以悦天下,何图筑室反耕哉!但惧秋风扬尘,伯珪马首南向,张杨、飞燕,膂力作难,北鄙将告倒县之急,股肱奏乞归之诚耳。主人当鉴我曹辈,反旌退师,治兵邺垣,何宜久辱盛怒,暴威於吾城下哉?足下讥吾恃黑山以为救,独不念黄巾之合从邪!加飞燕之属悉以受王命矣。昔高祖取彭越于钜野,光武创基兆于绿林,卒能龙飞中兴,以成帝业,苟可辅主兴化,夫何嫌哉!况仆亲奉玺书,与之从事。

行矣孔璋!足下徼利於境外,臧洪授命於君亲;吾子讬身於盟主,臧洪策名於长安。子谓余身死而名灭,仆亦笑子生死而无闻焉,悲哉!本同而末离,努力努力,夫复何言!

绍见洪书,知无降意,增兵急攻。城中粮谷以尽,外无强救,洪自度必不免,呼吏士谓曰:“袁氏无道,所图不轨,且不救洪郡将。洪於大义,不得不死,今诸君无事空与此祸!可先城未败,将妻子出。”将吏士民皆垂泣曰:“明府与袁氏本无怨隙,今为本朝郡将之故,自致残困,吏民何忍当舍明府去也!”初尚掘鼠煮筋角,后无可复食者。主簿启内厨米三斗,请中分稍以为糜粥,洪叹曰:“独食此何为!”使作薄粥,众分歠之,杀其爱妾以食将士。将士咸流涕,无能仰视者。男女七八千人相枕而死,莫有离叛。

城陷,绍生执洪。绍素亲洪,盛施帏幔,大会诸将见洪,谓曰:“臧洪,何相负若此!今日服未?”洪据地瞋目曰:“诸袁事汉,四世五公,可谓受恩。今王室衰弱,无扶翼之意,欲因际会,希冀非望,多杀忠良以立奸威。洪亲见呼张陈留为兄,则洪府君亦宜为弟,同共戮力,为国除害,何为拥众观人屠灭!惜洪力劣,不能推刃为天下报仇,何谓服乎!”绍本爱洪,意欲令屈服,原之;见洪辞切,知终不为己用,乃杀之。洪邑人陈容少为书生,亲慕洪,随洪为东郡丞;城未败,洪遣出。绍令在坐,见洪当死,起谓绍曰:“将军举大事,欲为天下除暴,而专先诛忠义,岂合天意!臧洪发举为郡将,奈何杀之!”绍惭,左右使人牵出,谓曰:“汝非臧洪俦,空复尔为!”容顾曰:“夫仁义岂有常,蹈之则君子,背之则小人。今日宁与臧洪同日而死,不与将军同日而生!”复见杀。在绍坐者无不叹息,窃相谓曰:“如何一日杀二烈士!”先是,洪遣司马二人出,求救于吕布;比还,城已陷,皆赴敌死。

评曰:吕布有虓虎之勇,而无英奇之略,轻狡反覆,唯利是视。自古及今,未有若此不夷灭也。昔汉光武谬於庞萌,近魏太祖亦蔽于张邈。知人则哲,唯帝难之,信矣!陈登、臧洪并有雄气壮节,登降年夙陨,功业未遂,洪以兵弱敌强,烈志不立,惜哉!


\part{魏书八}
\chapter{二公孙陶四张传第八}

公孙瓒字伯珪,辽西令支人也。为郡门下书佐。有姿仪,大音声,侯太守器之,以女妻焉,遣诣涿郡卢植读经。后复为郡吏。刘太守坐事徵诣廷尉,瓒为御车,身执徒养。及刘徙日南,瓒具米肉,於北芒上祭先人,举觞祝曰:“昔为人子,今为人臣,当诣日南。日南瘴气,或恐不还,与先人辞於此。”再拜慷慨而起,时见者莫不歔欷。刘道得赦还。瓒以孝廉为郎,除辽东属国长史。尝从数十骑出行塞,见鲜卑数百骑,瓒乃退入空亭中,约其从骑曰:“今不冲之,则死尽矣。”瓒乃自持矛,两头施刃,驰出刺胡,杀伤数十人,亦亡其从骑半,遂得免。鲜卑惩艾,后不敢复入塞。迁为涿令。光和中,凉州贼起,发幽州突骑三千人,假瓒都督行事传,使将之。军到蓟中,渔阳张纯诱辽西乌丸丘力居等叛,劫略蓟中,自号将军,略吏民攻右北平、辽西属国诸城,所至残破。瓒将所领,追讨纯等有功,迁骑都尉。属国乌丸贪至王率种人诣瓒降。迁中郎将,封都亭侯,进屯属国,与胡相攻击五六年。丘力居等钞略青、徐、幽、冀,四州被其害,瓒不能御。

朝议以宗正东海刘伯安既有德义,昔为幽州刺史,恩信流著,戎狄附之,若使镇抚,可不劳众而定,乃以刘虞为幽州牧。虞到,遣使至胡中,告以利害,责使送纯首。丘力居等闻虞至,喜,各遣译自归。瓒害虞有功,乃阴使人徼杀胡使。胡知其情,间行诣虞。虞上罢诸屯兵,但留瓒将步骑万人屯右北平。纯乃弃妻子,逃入鲜卑,为其客王政所杀,送首诣虞。封政为列侯。虞以功即拜太尉,封襄贲侯。会董卓至洛阳,迁虞大司马,瓒奋武将军,封蓟侯。

关东义兵起,卓遂劫帝西迁,徵虞为太傅,道路隔塞,信命不得至。袁绍、韩馥议,以为少帝制於奸臣,天下无所归心。虞,宗室知名,民之望也,遂推虞为帝。遣使诣虞,虞终不肯受。绍等复劝虞领尚书事,承制封拜,虞又不听,然犹与绍等连和。虞子和为侍中,在长安。天子思东归,使和伪逃卓,潜出武关诣虞,令将兵来迎。和道经袁术,为说天子意。术利虞为援,留和不遣,许兵至俱西,令和为书与虞。虞得和书,乃遣数千骑诣和。瓒知术有异志,不欲遣兵,止虞,虞不可。瓒惧术闻而怨之,亦遣其从弟越将千骑诣术以自结,而阴教术执和,夺其兵。由是虞、瓒益有隙。和逃术来北,复为绍所留。

是时,术遣孙坚屯阳城拒卓,绍使周昂夺其处。术遣越与坚攻昂,不胜,越为流矢所中死。瓒怒曰:“余弟死,祸起于绍。”遂出军屯磐河,将以报绍。绍惧,以所佩勃海太守印绶授瓒从弟范,遣之郡,欲以结援。范遂以勃海兵助瓒,破青、徐黄巾,兵益盛;进军界桥。以严纲为冀州,田楷为青州,单经为兖州,置诸郡县。绍军广川,令将麹义先登与瓒战,生禽纲。瓒军败走勃海,与范俱还蓟,於大城东南筑小城,与虞相近,稍相恨望。

虞惧瓒为变,遂举兵袭瓒。虞为瓒所败,出奔居庸。瓒攻拔居庸,生获虞,执虞还蓟。会卓死,天子遣使者段训增虞邑,督六州;瓒迁前将军,封易侯。瓒诬虞欲称尊号,胁训斩虞。瓒上训为幽州刺史。瓒遂骄矜,记过忘善,多所贼害。虞从事渔阳鲜于辅、齐周、骑都尉鲜于银等,率州兵欲报瓒,以燕国阎柔素有恩信,共推柔为乌丸司马。柔招诱乌丸、鲜卑,得胡、汉数万人,与瓒所置渔阳太守邹丹战于潞北,大破之,斩丹。袁绍又遣麹义及虞子和,将兵与辅合击瓒。瓒军数败,乃走还易京固守。为围堑十重,於堑里筑京,皆高五六丈,为楼其上;中堑为京,特高十丈,自居焉,积谷三百万斛。瓒曰:“昔谓天下事可指麾而定,今日视之,非我所决,不如休兵,力田畜谷。兵法,百楼不攻。今吾楼橹千重,食尽此谷,足知天下之事矣。”欲以此弊绍。绍遣将攻之,连年不能拔。建安四年,绍悉军围之。瓒遣子求救于黑山贼,复欲自将突骑直出,傍西南山,拥黑山之众,陆梁冀州,横断绍后。长史关靖说瓒曰:“今将军将士,皆已土崩瓦解,其所以能相守持者,顾恋其居处老小,以将军为主耳。将军坚守旷日,袁绍要当自退;自退之后,四方之众必复可合也。若将军今舍之而去,军无镇重,易京之危,可立待也。将军失本,孤在草野,何所成邪!”瓒遂止不出。救至,欲内外击绍。遣人与子书,刻期兵至,举火为应。绍侯者得其书,如期举火。瓒以为救兵至,遂出欲战。绍设伏击,大破之,复还守。绍为地道,突坏其楼,稍至中京。瓒自知必败,尽杀其妻子,乃自杀。

鲜于辅将其众奉王命。以辅为建忠将军,督幽州六郡。太祖与袁绍相拒於官渡,阎柔遣使诣太祖受事,迁护乌丸校尉。而辅身诣太祖,拜左度辽将军,封亭侯,遣还镇抚本州。太祖破南皮,柔将部曲及鲜卑献名马以奉军,从征三郡乌丸,以功封关内侯。辅亦率其众从。文帝践阼,拜辅虎牙将军,柔度辽将军,皆进封县侯,位特进。

陶谦字恭祖,丹杨人。少好学,为诸生,仕州郡,举茂才,除卢令,迁幽州刺史,徵拜议郎,参车骑将军张温军事,西讨韩遂。会徐州黄巾起,以谦为徐州刺史,击黄巾,破走之。董卓之乱,州郡起兵,天子都长安,四方断绝,谦遣使间行致贡献,迁安东将军、徐州牧,封溧阳侯。是时,徐州百姓殷盛,谷米封赡,流民多归之。而谦背道任情:广陵太守琅邪赵昱,徐方名士也,以忠直见疏;曹宏等,谗慝小人也,谦亲任之。刑政失和,良善多被其害,由是渐乱。下邳阙宣自称天子,谦初与合从寇钞,后遂杀宣,并其众。

初平四年,太祖征谦,攻拔十馀城,至彭城大战。谦兵败走,死者万数,泗水为之不流。谦退守郯。太祖以粮少引军还。兴平元年,复东征,略定琅邪、东海诸县。谦恐,欲走归丹杨。会张邈叛迎吕布,太祖还击布。是岁,谦病死。

张杨字稚叔,云中人也。以武勇给并州,为武猛从事。灵帝末,天下乱,帝以所宠小黄门蹇硕为西园上军校尉,军京都,欲以御四方,徵天下豪杰以为偏裨。太祖及袁绍等皆为校尉,属之。并州刺史丁原遣杨将兵诣硕,为假司马。灵帝崩,硕为何进所杀。杨复为进所遣,归本州募兵,得千馀人,因留上党,击山贼。进败,董卓作乱。杨遂以所将攻上党太守于壶关,不下,略诸县,众至数千人。山东兵起,欲诛卓。袁绍至河内,杨与绍合,复与匈奴单于於夫罗屯漳水。单于欲叛绍,杨不从。单于执杨与俱去,绍使将麹义追击於邺南,破之。单于执杨至黎阳,攻破度辽将军耿祉军,众复振。卓以杨为建义将军、河内太守。天子之在河东,杨将兵至安邑,拜安国将军,封晋阳侯。杨欲迎天子还洛,诸将不听;杨还野王。建安元年,杨奉、董承、韩暹挟天子还旧京,粮乏。杨以粮迎道路,遂至洛阳。谓诸将曰:“天子当与天下共之,幸有公卿大臣,杨当捍外难,何事京都?”遂还野王。即拜为大司马。杨素与吕布善。太祖之围布,杨欲救之,不能。乃出兵东市,遥为之势。其将杨丑,杀杨以应太祖。杨将眭固杀丑,将其众,欲北合袁绍。太祖遣史涣邀击,破之於犬城,斩固,尽收其众也。

公孙度字升济,本辽东襄平人也。度父延,避吏居玄菟,任度为郡吏。时玄菟太守公孙琙,子豹,年十八岁,早死。度少时名豹,又与琙子同年,琙见而亲爱之,遣就师学,为取妻。后举有道,除尚书郎,稍迁冀州刺史,以谣言免。同郡徐荣为董卓中郎将,荐度为辽东太守。度起玄菟小吏,为辽东郡所轻。先时,属国公孙昭守襄平令,召度子康为伍长。度到官,收昭,笞杀于襄平市。郡中名豪大姓田韶等宿遇无恩,皆以法诛,所夷灭百馀家,郡中震栗。东伐高句骊,西击乌丸,威行海外。初平元年,度知中国扰攘,语所亲吏柳毅、阳仪等曰:“汉祚将绝,当与诸卿图王耳。”时襄平延里社生大石,长丈馀,下有三小石为之足。或谓度曰:“此汉宣帝冠石之祥,而里名与先君同。社主土地,明当有土地,而三公为辅也。”度益喜。故河内太守李敏,郡中知名,恶度所为,恐为所害,乃将家属入于海。度大怒,掘其父冢,剖棺焚尸,诛其宗族。分辽东郡为辽西中辽郡,置太守。越海收东莱诸县,置营州刺史。自立为辽东侯、平州牧,追封父延为建义侯。立汉二祖庙,承制设坛墠於襄平城南,郊祀天地,藉田,治兵,乘鸾路,九旒,旄头羽骑。太祖表度为武威将军,封永宁乡侯,度曰:“我王辽东,何永宁也!”藏印绶武库。度死,子康嗣位,以永宁乡侯封弟恭。是岁建安九年也。

十二年,太祖征三郡乌丸,屠柳城。袁尚等奔辽东,康斩送尚首。语在武纪。封康襄平侯,拜左将军。康死,子晃、渊等皆小,众立恭为辽东太守。文帝践阼,遣使即拜恭为车骑将军、假节,封平郭侯;追赠康大司马。

初,恭病阴消为阉人,劣弱不能治国。太和二年,渊胁夺恭位。明帝即位拜渊扬烈将军、辽东太守。渊遣使南通孙权,往来赂遗。权遣使张弥、许晏等,赍金玉珍宝,立渊为燕王。渊亦恐权远不可恃,且贪货物,诱致其使,悉斩送弥、晏等首,明帝於是拜渊大司马,封乐浪公,持节、领郡如故。使者至,渊设甲兵为军陈,出见使者,又数对国中宾客出恶言。景初元年,乃遣幽州刺史毌丘俭等赍玺书徵渊。渊遂发兵,逆於辽隧,与俭等战。俭等不利而还。渊遂自立为燕王,置百官有司。遣使者持节,假鲜卑单于玺,封拜边民,诱呼鲜卑,侵扰北方。二年春,遣太尉司马宣王征渊。六月,军至辽东。渊遣将军卑衍、杨祚等步骑数万屯辽隧,围堑二十馀里。宣王军至,令衍逆战。宣王遣将军胡遵等击破之。宣王令军穿围,引兵东南向,而急东北,即趋襄平。衍等恐襄平无守,夜走。诸军进至首山,渊复遣衍等迎军殊死战。复击,大破之,遂进军造城下,为围堑。会霖雨三十馀日,辽水暴长,运船自辽口径至城下。雨霁,起土山、脩橹,为发石连弩射城中。渊窘急。粮尽,人相食,死者甚多。将军杨祚等降。八月丙寅夜,大流星长数十丈,从首山东北坠襄平城东南。壬午,渊众溃,与其子脩将数百骑突围东南走,大兵急击之,当流星所坠处,斩渊父子。城破,斩相国以下首级以千数,传渊首洛阳,辽东、带方、乐浪、玄菟悉平。

初,渊家数有怪,犬冠帻绛衣上屋,炊有小儿蒸死甑中。襄平北巿生肉,长围各数尺,有头目口喙,无手足而动摇。占曰:“有形不成,有体无声,其国灭亡。”始度以中平六年据辽东,至渊三世,凡五十年而灭。

张燕,常山真定人也,本姓褚。黄巾起,燕合聚少年为群盗,在山泽间转攻,还真定,众万馀人。博陵张牛角亦起众,自号将兵从事,与燕合。燕推牛角为帅,俱攻癭陶。牛角为飞矢所中。被创且死,令众奉燕,告曰:“必以燕为帅。”牛角死,众奉燕,故改姓张。燕剽捍捷速过人,故军中号曰飞燕。其后人众寝广,常山、赵郡、中山、上党、河内诸山谷皆相通,其小帅孙轻、王当等,各以部众从燕,众至百万,号曰黑山。灵帝不能征,河北诸郡被其害。燕遣人至京都乞降,拜燕平难中郎将。是后,董卓迁天子於长安,天下兵数起,燕遂以其众与豪杰相结。袁绍与公孙瓒争冀州,燕遣将杜长等助瓒,与绍战,为绍所败,人众稍散。太祖将定冀州,燕遣使求佐王师,拜平北将军;率众诣邺,封安国亭侯,邑五百户。燕薨,子方嗣。方薨,子融嗣。

张绣,武威祖厉人,骠骑将军济族子也。边章、韩遂为乱凉州,金城麹胜袭杀祖厉长刘隽。绣为县吏,间伺杀胜,郡内义之。遂招合少年,为邑中豪杰。董卓败,济与李傕等击吕布,为卓报仇。语在卓传。绣随济,以军功稍迁至建忠将军,封宣威侯。济屯弘农,士卒饥饿,南攻穰,为流矢所中死。绣领其众,屯宛,与刘表合。太祖南征,军淯水,绣等举众降。太祖纳济妻,绣恨之。太祖闻其不悦,密有杀绣之计。计漏,绣掩袭太祖。太祖军败,二子没。绣还保穰,太祖比年攻之,不克。太祖拒袁绍於官渡,绣从贾诩计,复以众降。语在诩传。绣至,太祖执其手,与欢宴,为子均取绣女,拜扬武将军。官渡之役,绣力战有功,迁破羌将军。从破袁谭於南皮,复增邑凡二千户。是时天下户口减耗,十裁一在,诸将封未有满千户者,而绣特多。从征乌丸于柳城,未至,薨,谥曰定侯。子泉嗣,坐与魏讽谋反诛,国除。

张鲁字公祺,沛国丰人也。祖父陵,客蜀,学道鹄鸣山中,造作道书以惑百姓,从受道者出五斗米,故世号米贼。陵死,子衡行其道。衡死,鲁复行之。益州牧刘焉以鲁为督义司马,与别部司马张脩将兵击汉中太守苏固,鲁遂袭脩杀之,夺其众。焉死,子璋代立,以鲁不顺,尽杀鲁母家室。鲁遂据汉中,以鬼道教民,自号“师君”。其来学道者,初皆名“鬼卒”。受本道已信,号“祭酒”。各领部众,多者为治头大祭酒。皆教以诚信不欺诈,有病自首其过,大都与黄巾相似。诸祭酒皆作义舍,如今之亭传。又置义米肉,县於义舍,行路者量腹取足;若过多,鬼道辄病之。犯法者,三原,然后乃行刑。不置长吏,皆以祭酒为治,民夷便乐之。雄据巴、汉垂三十年。汉末,力不能征,遂就宠鲁为镇民中郎将,领汉宁太守,通贡献而已。民有地中得玉印者,群下欲尊鲁为汉宁王。鲁功曹巴西阎圃谏鲁曰:“汉川之民,户出十万,财富土沃,四面险固;上匡天子,则为桓、文,次及窦融,不失富贵。今承制署置,势足斩断,不烦於王。愿且不称,勿为祸先”鲁从之。韩遂、马超之乱,关西民从子午谷奔之者数万家。

建安二十年,太祖乃自散关出武都征之,至阳平关。鲁欲举汉中降,其弟卫不肯,率众数万人拒关坚守。太祖攻破之,遂入蜀。鲁闻阳平已陷,将稽颡归降,圃又曰:“今以迫往,功必轻;不如依杜濩赴朴胡相拒,然后委质,功必多。”於是乃奔南山入巴中。左右欲悉烧宝货仓库,鲁曰:“本欲归命国家,而意未达。今之走,避锐锋,非有恶意。宝货仓库,国家之有。”遂封藏而去。太祖入南郑,甚嘉之。又以鲁本有善意,遣人慰喻。鲁尽将家出,太祖逆拜鲁镇南将军,待以客礼,封阆中侯,邑万户。封鲁五子及阎圃等皆为列侯。为子彭祖取鲁女。鲁薨,谥之曰原侯。子富嗣。

评曰:公孙瓒保京,坐待夷灭。度残暴而不节,渊仍业以载凶,秪足覆其族也。陶谦昏乱而忧死,张杨授首於臣下。皆拥据州郡,曾匹夫之不若,固无可论者也。燕、绣、鲁舍群盗,列功臣,去危亡,保宗祀,则於彼为愈焉。

\part{魏书九}
\chapter{诸夏侯曹传第九}

夏侯惇字元让,沛国谯人,夏侯婴之后也。年十四,就师学,人有辱其师者,惇杀之,由是以烈气闻。太祖初起,惇常为裨将,从征伐。太祖行奋武将军,以惇为司马,别屯白马,迁折冲校尉,领东郡太守。太祖征陶谦,留惇守濮阳。张邈叛迎吕布,太祖家在鄄城,惇轻军往赴,適与布会,交战。布退还,遂入濮阳,袭得惇军辎重。遣将伪降,共执持惇,责以宝货,惇军中震恐。惇将韩浩乃勒兵屯惇营门,召军吏诸将,皆案甲当部不得动,诸营乃定。遂诣惇所,叱持质者曰:“汝等凶逆,乃敢执劫大将军,复欲望生邪!且吾受命讨贼,宁能以一将军之故,而纵汝乎?”因涕泣谓惇曰:“当奈国法何!”促召兵击持质者。持质者惶遽叩头,言“我但欲乞资用去耳”!浩数责,皆斩之。惇既免,太祖闻之,谓浩曰:“卿此可为万世法。”乃著令,自今已后有持质者,皆当并击,勿顾质。由是劫质者遂绝。

太祖自徐州还,惇从征吕布,为流矢所中,伤左目。复领陈留、济阴太守,加建武将军,封高安乡侯。时大旱,蝗虫起,惇乃断太寿水作陂,身自负土,率将士劝种稻,民赖其利。转领河南尹。太祖平河北,为大将军后拒。邺破,迁伏波将军,领尹如故,使得以便宜从事,不拘科制。建安十二年,录惇前后功,增封邑千八百户,并前二千五百户。二十一年,从征孙权还,使惇都督二十六军,留居巢。赐伎乐名倡,令曰:“魏绛以和戎之功,犹受金石之乐,况将军乎!”二十四年,太祖军于摩陂,召惇常与同载,特见亲重,出入卧内,诸将莫得比也。拜前将军,督诸军还寿春,徙屯召陵。文帝即王位,拜惇大将军,数月薨。

惇虽在军旅,亲迎师受业。性清俭,有馀财辄以分施,不足资之於官,不治产业。谥曰忠侯。子充嗣。帝追思惇功,欲使子孙毕侯,分惇邑千户,赐惇七子二孙爵皆关内侯。惇弟廉及子楙素自封列侯。初,太祖以女妻楙,即清河公主也。楙历位侍中尚书、安西镇东将军,假节。充薨,子廙嗣。廙薨,子劭嗣。

韩浩者,河内人。沛国史涣与浩俱以忠勇显。浩至中护军,涣至中领军,皆掌禁兵,封列侯。

夏侯渊字妙才,惇族弟也。太祖居家,曾有县官事,渊代引重罪,太祖营救之,得免。太祖起兵,以别部司马、骑都尉从,迁陈留、颍川太守。及与袁绍战于官渡,行督军校尉。绍破,使督兖、豫、徐州军粮;时军食少,渊传馈相继,军以复振。昌豨反,遣于禁击之,未拔,复遣渊与禁并力,遂击豨,降其十馀屯,豨诣禁降。渊还,拜典军校尉。济南、乐安黄巾徐和、司马俱等攻城,杀长吏,渊将泰山、齐、平原郡兵击,大破之,斩和,平诸县,收其粮谷以给军士。十四年,以渊为行领军。太祖征孙权还,使渊督诸将击庐江叛者雷绪,绪破,又行征西护军,督徐晃击太原贼,攻下二十馀屯,斩贼帅商曜,屠其城。从征韩遂等,战於渭南。又督朱灵平隃糜、汧氐。与太祖会安定,降杨秋。

十七年,太祖乃还邺,以渊行护军将军,督朱灵、路招等屯长安,击破南山贼刘雄,降其众。围遂、超馀党梁兴於鄠,拔之,斩兴,封博昌亭侯。马超围凉州刺史韦康於冀,渊救康,未到,康败。去冀二百馀里,超来逆战,军不利。汧氐反,渊引军还。十九年,赵衢、尹奉等谋讨超,姜叙起兵卤城以应之。衢等谲说超,使出击叙,於后尽杀超妻子。超奔汉中,还围祁山。叙等急求救,诸将议者欲须太祖节度。渊曰:“公在邺,反覆四千里,比报,叙等必败,非攻急也。”遂行,使张郃督步骑五千在前,从陈仓狭道入,渊自督粮在后。郃至渭水上,超将氐羌数千逆郃。未战,超走,郃进军收超军器械。渊到,诸县皆已降。韩遂在显亲,渊欲袭取之,遂走。渊收遂军粮,追至略阳城,去遂二十馀里,诸将欲攻之,或言当攻兴国氐。渊以为遂兵精,兴国城固,攻不可卒拔,不如击长离诸羌。长离诸羌多在遂军,必归救其家。若舍羌独守则孤,救长离则官兵得与野战,可必虏也。渊乃留督将守辎重,轻兵步骑到长离,攻烧羌屯,斩获甚众。诸羌在遂军者,各还种落。遂果救长离,与渊军对陈。诸将见遂众,恶之,欲结营作堑乃与战。渊曰:“我转斗千里,今复作营堑,则士众罢弊,不可久。贼虽众,易与耳。”乃鼓之,大破遂军,得其旌麾,还略阳,进军围兴国。氐王千万逃奔马超,馀众降。转击高平屠各,皆散走,收其粮谷牛马。乃假渊节。

初,枹罕宋建因凉州乱,自号河首平汉王。太祖使渊帅诸将讨建。渊至,围枹罕,月馀拔之,斩建及所置丞相已下。渊别遣张郃等平河关,渡河入小湟中,河西诸羌尽降,陇右平。太祖下令曰:“宋建造为乱逆三十馀年,渊一举灭之,虎步关右,所向无前。仲尼有言:‘吾与尔不如也。’”二十一年,增封三百户,并前八百户。还击武都氐羌下辩,收氐谷十馀万斛。太祖西征张鲁,渊等将凉州诸将侯王已下,与太祖会休亭。太祖每引见羌、胡,以渊畏之。会鲁降,汉中平,以渊行都护将军,督张郃、徐晃等平巴郡。太祖还邺,留渊守汉中,即拜渊征西将军。二十三年,刘备军阳平关,渊率诸将拒之,相守连年。二十四年正月,备夜烧围鹿角。渊使张郃护东围,自将轻兵护南围。备挑郃战,郃军不利。渊分所将兵半助郃,为备所袭,渊遂战死。谥曰愍侯。

初,渊虽数战胜,太祖常戒曰:“为将当有怯弱时,不可但恃勇也。将当以勇为本,行之以智计;但知任勇,一匹夫敌耳。”

渊妻,太祖内妹。长子衡,尚太祖弟海阳哀侯女,恩宠特隆。衡袭爵,转封安宁亭侯。黄初中,赐中子霸,太和中,赐霸四弟,爵皆关内侯。霸,正始中为讨蜀护军右将军,进封博昌亭侯,素为曹爽所厚。闻爽诛,自疑,亡入蜀。以渊旧勋赦霸子,徙乐浪郡。霸弟威,官至兖州刺史。威弟惠,乐安太守。惠弟和,河南尹。衡薨,子绩嗣,为虎贲中郎将。绩薨,子褒嗣。

曹仁字子孝,太祖从弟也。少好弓马弋猎。后豪杰并起,仁亦阴结少年,得千馀人,周旋淮、泗之间,遂从太祖为别部司马,行厉锋校尉。太祖之破袁术,仁所斩获颇多。从征徐州,仁常督骑,为军前锋。别攻陶谦将吕由,破之,还与大军合彭城,大破谦军。后攻费、华、即墨、开阳,谦遣别将救诸县,仁以骑击破之。太祖征吕布,仁别攻句阳,拔之,生获布将刘何。太祖平黄巾,迎天子都许,仁数有功,拜广阳太守。太祖器其勇略,不使之郡,以议郎督骑。太祖征张绣,仁别徇旁县,虏其男女三千馀人。太祖军还,为绣所追,军不利,士卒丧气,仁率厉将士甚奋,太祖壮之,遂破绣。

太祖与袁绍久相持於官渡,绍遣刘备徇〈氵隱〉强诸县,多举众应之。自许以南,吏民不安,太祖以为忧。仁曰:“南方以大军方有目前急,其势不能相救,刘备以强兵临之,其背叛固宜也。备新将绍兵,未能得其用,击之可破也。”太祖善其言,遂使将骑击备,破走之,仁尽复收诸叛县而还。绍遣别将韩荀钞断西道,仁击荀於鸡洛山,大破之。由是绍不敢复分兵出。复与史涣等钞绍运军,烧其粮谷。

河北既定,从围壶关。太祖令曰:“城拔,皆坑之。”连月不下。仁言於太祖曰:“围城必示之活门,所以开其生路也。今公告之必死,将人自为守。且城固而粮多,攻之则士卒伤,守之则引日久;今顿兵坚城之下,以攻必死之虏,非良计也。”太祖从之,城降。於是录仁前后功,封都亭侯。

从平荆州,以仁行征南将军,留屯江陵,拒吴将周瑜。瑜将数万众来攻,前锋数千人始至,仁登城望之,乃募得三百人,遣部曲将牛金逆与挑战。贼多,金众少,遂为所围。长史陈矫俱在城上,望见金等垂没,左右皆失色。仁意气奋怒甚,谓左右取马来,矫等共援持之。谓仁曰:“贼众盛,不可当也。假使弃数百人何苦,而将军以身赴之!”仁不应,遂被甲上马,将其麾下壮士数十骑出城。去贼百馀步,迫沟,矫等以为仁当住沟上,为金形势也,仁径渡沟直前,冲入贼围,金等乃得解。馀众未尽出,仁复直还突之,拔出金兵,亡其数人,贼众乃退。矫等初见仁出,皆惧,及见仁还,乃叹曰:“将军真天人也!”三军服其勇。太祖益壮之,转封安平亭侯。

太祖讨马超,以仁行安西将军,督诸将拒潼关,破超渭南。苏伯、田银反,以仁行骁骑将军,都督七军讨银等,破之。复以仁行征南将军,假节,屯樊,镇荆州。侯音以宛叛,略傍县众数千人,仁率诸军攻破音,斩其首,还屯樊,即拜征南将军。关羽攻樊,时汉水暴溢,于禁等七军皆没,禁降羽。仁人马数千人守城,城不没者数板。羽乘船临城,围数重,外内断绝,粮食欲尽,救兵不至。仁激厉将士,示以必死,将士感之皆无二。徐晃救至,水亦稍减,晃从外击羽,仁得溃围出,羽退走。

仁少时不脩行检,及长为将,严整奉法令,常置科於左右,案以从事。鄢陵侯彰北征乌丸,文帝在东宫,为书戒彰曰:“为将奉法,不当如征南邪!”及即王位,拜仁车骑将军,都督荆、扬、益州诸军事,进封陈侯,增邑二千,并前三千五百户。追赐仁父炽谥曰陈穆侯,置守冢十家。后召还屯宛。孙权遣将陈邵据襄阳,诏仁讨之。仁与徐晃攻破邵,遂入襄阳,使将军高迁等徙汉南附化民於汉北,文帝遣使即拜仁大将军。又诏仁移屯临颍,迁大司马,复督诸军据乌江,还屯合肥。黄初四年薨,谥曰忠侯。子泰嗣,官至镇东将军,假节,转封甯陵侯。泰薨,子初嗣。又分封泰弟楷、范,皆为列侯,而牛金官至后将军。

仁弟纯,初以议郎参司空军事,督虎豹骑从围南皮。袁谭出战,士卒多死。太祖欲缓之,纯曰:“今千里蹈敌,进不能克,退必丧威;且县师深入,难以持久。彼胜而骄,我败而惧,以惧敌骄,必可克也。”太祖善其言,遂急攻之,谭败。纯麾下骑斩谭首。及北征三郡,纯部骑获单于蹹顿。以前后功封高陵亭侯,邑三百户。从征荆州,追刘备於长坂,获其二女辎重,收其散卒。进降江陵,从还谯。建安十五年薨。文帝即位,追谥曰威侯。子演嗣,官至领军将军,正元中进封平乐乡侯。演薨,子亮嗣。

曹洪字子廉,太祖从弟也。太祖起义兵讨董卓,至荥阳,为卓将徐荣所败。太祖失马,贼追甚急,洪下,以马授太祖,太祖辞让,洪曰:“天下可无洪,不可无君。”遂步从到汴水,水深不得渡,洪循水得船,与太祖俱济,还奔谯。扬州刺史陈温素与洪善,洪将家兵千馀人,就温募兵,得庐江上甲二千人,东到丹杨复得数千人,与太祖会龙亢。太祖征徐州,张邈举兖州叛迎吕布。时大饥荒,洪将兵在前,先据东平、范,聚粮谷以继军。太祖讨邈、布於濮阳,布破走,遂据东阿,转击济阴、山阳、中牟、阳武、京、密十馀县,皆拔之。以前后功拜鹰扬校尉,迁扬武中郎将。天子都许,拜洪谏议大夫。别征刘表,破表别将於舞阳、阴叶、堵阳、博望,有功,迁厉锋将军,封国明亭侯。累从征伐,拜都护将军。文帝即位,为卫将军,迁骠骑将军,进封野王侯,益邑千户,并前二千一百户,位特进;后徙封都阳侯。

始,洪家富而性吝啬,文帝少时假求不称,常恨之,遂以舍客犯法,下狱当死。群臣并救莫能得。卞太后谓郭后曰:“令曹洪今日死,吾明日敕帝废后矣。”於是泣涕屡请,乃得免官削爵土。洪先帝功臣,时人多为觖望。明帝即位,拜后将军,更封乐城侯,邑千户,位特进,复拜骠骑将军。太和六年薨,谥曰恭侯。子馥,嗣侯。初,太祖分洪户封子震列侯。洪族父瑜,脩慎笃敬,官至卫将军,封列侯。

曹休字文烈,太祖族子也。天下乱,宗族各散去乡里。休年十馀岁,丧父,独与一客担丧假葬,携将老母,渡江至吴。以太祖举义兵,易姓名转至荆州,间行北归,见太祖。太祖谓左右曰:“此吾家千里驹也。”使与文帝同止,见待如子。常从征伐,使领虎豹骑宿卫。刘备遣将吴兰屯下辩,太祖遣曹洪征之,以休为骑都尉,参洪军事。太祖谓休曰:“汝虽参军,其实帅也。”洪闻此令,亦委事於休。备遣张飞屯固山,欲断军后。众议狐疑,休曰:“贼实断道者,当伏兵潜行。今乃先张声势,此其不能也。宜及其未集,促击兰,兰破则飞自走矣。”洪从之,进兵击兰,大破之,飞果走。太祖拔汉中,诸军还长安,拜休中领军。文帝即王位,为领军将军,录前后功,封东阳亭侯。夏侯惇薨,以休为镇南将军,假节都督诸军事,车驾临送,上乃下舆执手而别。孙权遣将屯历阳,休到,击破之,又别遣兵渡江,烧贼芜湖营数千家。迁征东将军,领扬州刺史,进封安阳乡侯。帝征孙权,以休为征东大将军,假黄钺,督张辽等及诸州郡二十馀军,击权大将吕范等於洞浦,破之。拜扬州牧。明帝即位,进封长平侯。吴将审德屯皖,休击破之,斩德首,吴将韩综、翟丹等前后率众诣休降。增邑四百,并前二千五百户,迁大司马,都督扬州如故。太和二年,帝为二道征吴,遣司马宣王从汉水下,休督诸军向寻阳。贼将伪降,休深入,战不利,退还宿石亭。军夜惊,士卒乱,弃甲兵辎重甚多。休上书谢罪,帝遣屯骑校尉杨暨慰谕,礼赐益隆。休因此痈发背薨,谥曰壮侯。子肇嗣。

肇有当世才度,为散骑常侍、屯骑校尉。明帝寝疾,方与燕王宇等属以后事。帝意寻变,诏肇以侯归第。正始中薨。追赠卫将军。子兴嗣。初,文帝分休户三百封肇弟纂为列侯,后为殄吴将军,薨,追赠前将军。

曹真字子丹,太祖族子也。太祖起兵,真父邵募徒众,为州郡所杀。太祖哀真少孤,收养与诸子同,使与文帝共止。常猎,为虎所逐,顾射虎,应声而倒。太祖壮其鸷勇,使将虎豹骑。讨灵丘贼,拔之,封灵寿亭侯。以偏将军将兵击刘备别将於下辩,破之,拜中坚将军。从至长安,领中领军。是时,夏侯渊没於阳平,太祖忧之。以真为征蜀护军,督徐晃等破刘备别将高详於阳平。太祖自至汉中,拔出诸军,使真至武都迎曹洪等还屯陈仓。文帝即王位,以真为镇西将军,假节都督雍、凉州诸军事。录前后功,进封东乡侯。张进等反於酒泉,真遣费曜讨破之,斩进等。黄初三年还京都,以真为上军大将军,都督中外诸军事,假节钺。与夏侯尚等征孙权,击牛渚屯,破之。转拜中军大将军,加给事中。七年,文帝寝疾,真与陈群、司马宣王等受遗诏辅政。明帝即位,进封邵陵侯,迁大将军。

诸葛亮围祁山,南安、天水、安定三郡反应亮。帝遣真督诸军军郿,遣张郃击亮将马谡,大破之。安定民杨条等略吏民保月支城,真进军围之。条谓其众曰:“大将军自来,吾愿早降耳。”遂自缚出。三郡皆平。真以亮惩于祁山,后出必从陈仓,乃使将军郝昭、王生守陈仓,治其城。明年春,亮果围陈仓,已有备而不能克。增邑,并前二千九百户。四年,朝洛阳,迁大司马,赐剑履上殿,入朝不趋。真以“蜀连出侵边境,宜遂伐之。数道并入,可大克也”。帝从其计。真当发西讨,帝亲临送。真以八月发长安,从子午道南入。司马宣王溯汉水,当会南郑。诸军或从斜谷道,或从武威入。会大霖雨三十馀日,或栈道断绝,诏真还军。

真少与宗人曹遵、乡人朱赞并事太祖。遵、赞早亡,真愍之,乞分所食邑封遵、赞子。诏曰:“大司马有叔向抚孤之仁,笃晏平久要之分。君子成人之美,听分真邑赐遵、赞子爵关内侯,各百户。”真每征行,与将士同劳苦,军赏不足,辄以家财班赐,士卒皆愿为用。真病还洛阳,帝自幸其第省疾。真薨,谥曰元侯。子爽嗣。帝追思真功,诏曰:“大司马蹈履忠节,佐命二祖,内不恃亲戚之宠,外不骄白屋之士,可谓持盈守业,劳谦其德者也。其悉封真五子羲、训、则、彦、皑皆为列侯。”初,文帝分真邑二百户,封真弟彬为列侯。

爽字昭伯,少以宗室谨重,明帝在东宫,甚亲爱之。及即位,为散骑侍郎,累迁城门校尉,加散骑常侍,转武卫将军,宠待有殊。帝寝疾,乃引爽入卧内,拜大将军,假节钺,都督中外诸军事,录尚书事,与太尉司马宣王并受遗诏辅少主。明帝崩,齐王即位,加爽侍中,改封武安侯,邑万二千户,赐剑履上殿,入朝不趋,赞拜不名。丁谧画策,使爽白天子,发诏转宣王为太傅,外以名号尊之,内欲令尚书奏事,先来由己,得制其轻重也。爽弟羲为中领军,训武卫将军,彦散骑常侍侍讲,其馀诸弟,皆以列侯侍从,出入禁闼,贵宠莫盛焉。南阳何晏、邓飏、李胜、沛国丁谧、东平毕轨咸有声名,进趣於时,明帝以其浮华,皆抑黜之;及爽秉政,乃复进叙,任为腹心。飏等欲令爽立威名於天下,劝使伐蜀,爽从其言,宣王止之不能禁。正始五年,爽乃西至长安,大发卒六七万人,从骆谷入。是时,关中及氐、羌转输不能供,牛马骡驴多死,民夷号泣道路。入谷行数百里,贼因山为固,兵不得进。爽参军杨伟为爽陈形势,宜急还,不然将败。飏与伟争於爽前,伟曰:“飏、胜将败国家事,可斩也。”爽不悦,乃引军还。

初,爽以宣王年德并高,恒父事之,不敢专行。及晏等进用,咸共推戴,说爽以权重不宜委之於人。乃以晏、飏、谧为尚书,晏典选举,轨司隶校尉,胜河南尹,诸事希复由宣王。宣王遂称疾避爽。晏等专政,共分割洛阳、野王典农部桑田数百顷,及坏汤沐地以为产业,承势窃取官物,因缘求欲州郡。有司望风,莫敢忤旨。晏等与廷尉卢毓素有不平,因毓吏微过,深文致毓法,使主者先收毓印绶,然后奏闻。其作威如此。爽饮食车服,拟於乘舆;尚方珍玩,充牣其家;妻妾盈后庭,又私取先帝才人七八人,及将吏、师工、鼓吹、良家子女三十三人,皆以为伎乐。诈作诏书,发才人五十七人送邺台,使先帝婕妤教习为伎。擅取太乐乐器,武库禁兵。作窟室,绮疏四周,数与晏等会其中,饮酒作乐。羲深以为大忧,数谏止之。又著书三篇,陈骄淫盈溢之致祸败,辞旨甚切,不敢斥爽,讬戒诸弟以示爽。爽知其为己发也,甚不悦。羲或时以谏喻不纳,涕泣而起。宣王密为之备。九年冬,李胜出为荆州刺史,往诣宣王。宣王称疾困笃,示以羸形。胜不能觉,谓之信然。

十年正月,车驾朝高平陵,爽兄弟皆从。宣王部勒兵马,先据武库,遂出屯洛水浮桥。奏爽曰:“臣昔从辽东还,先帝诏陛下、秦王及臣升御床,把臣臂,深以后事为念。臣言'二祖亦属臣以后事,此自陛下所见,无所忧苦;万一有不如意,臣当以死奉明诏'。黄门令董箕等,才人侍疾者,皆所闻知。今大将军爽背弃顾命,败乱国典,内则僣拟,外专威权;破坏诸营,尽据禁兵,群官要职,皆置所亲;殿中宿卫,历世旧人皆复斥出,欲置新人以树私计;根据槃互,纵恣日甚。外既如此,又以黄门张当为都监,专共交关,看察至尊,侯伺神器。离间二宫,伤害骨肉。天下汹汹,人怀危惧,陛下但为寄坐,岂得久安!此非先帝诏陛下及臣升御床之本意也。臣虽朽迈,敢忘枉言?昔赵高极意,秦氏以灭;吕、霍早断,汉祚永世。此乃陛下之大鉴,臣受命之时也。太尉臣济、尚书令臣孚等,皆以爽为有无君之心,兄弟不宜典兵宿卫。奏永宁宫,皇太后令敕臣如奏施行。臣辄敕主者及黄门令罢爽、羲、训吏兵,以侯就第,不得逗留以稽车驾;敢有稽留,便以军法从事。臣辄力疾将兵屯洛水浮桥,伺察非常。”

爽得宣王奏事,不通,迫窘不知所为。大司农沛国桓范闻兵起,不应太后召,矫诏开平昌门,拔取剑戟,略将门候,南奔爽。宣王知,曰:“范画策,爽必不能用范计。”范说爽使车驾幸许昌,招外兵。爽兄弟犹豫未决,范重谓羲曰:“当今日,卿门户求贫贱复可得乎?且匹夫持质一人,尚欲望活,今卿与天子相随,令於天下,谁敢不应者?”羲犹不能纳。侍中许允、尚书陈泰说爽,使早自归罪。爽於是遣允、泰诣宣王,归罪请死,乃通宣王奏事。遂免爽兄弟,以侯还第。

初,张当私以所择才人张、何等与爽。疑其有奸,收当治罪。当陈爽与晏等阴谋反逆,并先习兵,须三月中欲发,於是收晏等下狱。会公卿朝臣廷议,以为“春秋之义,‘君亲无将,将而必诛’。爽以支属,世蒙殊宠,亲受先帝握手遗诏,讬以天下,而包藏祸心,蔑弃顾命,乃与晏、飏及当等谋图神器,范党同罪人,皆为大逆不道“。於是收爽、羲、训、晏、飏、谧、轨、胜、范、当等,皆伏诛,夷三族。嘉平中,绍功臣世,封真族孙熙为新昌亭侯,邑三百户,以奉真后。

晏,何进孙也。母尹氏,为太祖夫人。晏长于宫省,又尚公主,少以才秀知名,好老庄言,作道德论及诸文赋著述凡数十篇。

夏侯尚字伯仁,渊从子也。文帝与之亲友。太祖定冀州,尚为军司马,将骑从征伐,后为五官将文学。魏国初建,迁黄门侍郎。代郡胡叛,遣鄢陵侯彰征讨之,以尚参彰军事,定代地,还。太祖崩于洛阳,尚持节,奉梓宫还邺。并录前功,封平陵亭侯,拜散骑常侍,迁中领军。文帝践阼,更封平陵乡侯,迁征南将军,领荆州刺史,假节都督南方诸军事。尚奏:“刘备别军在上庸,山道险难,彼不我虞,若以奇兵潜行,出其不意,则独克之势也。”遂勒诸军击破上庸,平三郡九县,迁征南大将军。孙权虽称藩,尚益脩攻讨之备,权后果有贰心。黄初三年,车驾幸宛,使尚率诸军与曹真共围江陵。权将诸葛瑾与尚军对江,瑾渡入江中渚,而分水军于江中。尚夜多持油船,将步骑万馀人,於下流潜渡,攻瑾诸军,夹江烧其舟船,水陆并攻,破之。城未拔,会大疫,诏敕尚引诸军还。益封六百户,并前千九百户,假钺,进为牧。荆州残荒,外接蛮夷,而与吴阻汉水为境,旧民多居江南。尚自上庸通道,西行七百馀里,山民蛮夷多服从者,五六年间,降附数千家。五年,徙封昌陵乡侯。尚有爱妾嬖幸,宠夺適室;適室,曹氏女也,故文帝遣人绞杀之。尚悲感,发病恍惚,既葬埋妾,不胜思见,复出视之。文帝闻而恚之曰:“杜袭之轻薄尚,良有以也。”然以旧臣,恩宠不衰。六年,尚疾笃,还京都,帝数临幸,执手涕泣。尚薨,谥曰悼侯。子玄嗣。又分尚户三百,赐尚弟子奉爵关内侯。

玄字太初。少知名,弱冠为散骑黄门侍郎。尝进见,与皇后弟毛曾并坐,玄耻之,不悦形之於色。明帝恨之,左迁为羽林监。正始初,曹爽辅政。玄,爽之姑子也。累迁散骑常侍、中护军。

太傅司马宣王问以时事,玄议以为:“夫官才用人,国之柄也,故铨衡专於台阁,上之分也,孝行存乎闾巷,优劣任之乡人,下之叙也。夫欲清教审选,在明其分叙,不使相涉而已。何者?上过其分,则恐所由之不本,而干势驰骛之路开;下逾其叙,则恐天爵之外通,而机权之门多矣。夫天爵下通,是庶人议柄也;机权多门,是纷乱之原也。自州郡中正品度官才之来,有年载矣,缅缅纷纷,未闻整齐,岂非分叙参错,各失其要之所由哉!若令中正但考行伦辈,伦辈当行均,斯可官矣。何者?夫孝行著於家门,岂不忠恪於在官乎?仁恕称於九族,岂不达於为政乎?义断行於乡党,岂不堪於事任乎?三者之类,取於中正,虽不处其官名,斯任官可知矣。行有大小,比有高下,则所任之流,亦涣然明别矣。奚必使中正干铨衡之机於下,而执机柄者有所委仗於上,上下交侵,以生纷错哉?且台阁临下,考功校否,众职之属,各有官长,旦夕相考,莫究於此。闾阎之议,以意裁处,而使匠宰失位,众人驱骇,欲风俗清静,其可得乎?天台县远,众所绝意。所得至者,更在侧近,孰不脩饰以要所求?所求有路,则脩己家门者,已不如自达于乡党矣。自达乡党者,已不如自求之於州邦矣。苟开之有路,而患其饰真离本,虽复严责中正,督以刑罚,犹无益也。岂若使各帅其分,官长则各以其属能否献之台阁,台阁则据官长能否之第,参以乡闾德行之次,拟其伦比,勿使偏颇。中正则唯考其行迹,别其高下,审定辈类,勿使升降。台阁总之,如其所简,或有参错,则其责负自在有司。官长所第,中正辈拟,比随次率而用之,如其不称,责负在外。然则内外相参,得失有所,互相形检,孰能相饰?斯则人心定而事理得,庶可以静风俗而审官才矣。”又以为:“古之建官,所以济育群生,统理民物也,故为之君长以司牧之。司牧之主,欲一而专,一则官任定而上下安,专则职业脩而事不烦。夫事简业脩,上下相安而不治者,未之有也。先王建万国,虽其详未可得而究,然分疆画界,各守土境,则非重累羁绊之体也。下考殷、周五等之叙,徒有小大贵贱之差,亦无君官臣民而有二统互相牵制者也。夫官统不一,则职业不脩;职业不脩,则事何得而简?事之不简,则民何得而静?民之不静,则邪恶并兴,而奸伪滋长矣。先王达其如此,故专其职司而一其统业。始自秦世,不师圣道,私以御职,奸以待下;惧宰官之不脩,立监牧以董之,畏督监之容曲,设司察以纠之;宰牧相累,监察相司,人怀异心,上下殊务。汉承其绪,莫能匡改。魏室之隆,日不暇及,五等之典,虽难卒复,可粗立仪准以一治制。今之长吏,皆君吏民,横重以郡守,累以刺史。若郡所摄,唯在大较,则与州同,无为再重。宜省郡守,但任刺史;刺史职存则监察不废,郡吏万数,还亲农业,以省烦费,丰财殖谷,一也。大县之才,皆堪郡守,是非之讼,每生意异,顺从则安,直己则争。夫和羹之美,在於合异,上下之益,在能相济。顺从乃安,此琴瑟一声也,荡而除之,则官省事简,二也。又幹郡之吏,职监诸县,营护党亲,乡邑旧故,如有不副,而因公掣顿,民之困弊,咎生于此,若皆并合,则乱原自塞,三也。今承衰弊,民人彫落,贤才鲜少,任事者寡。郡县良吏,往往非一,郡受县成,其剧在下,而吏之上选,郡当先足。此为亲民之吏,专得底下。吏者民命,而常顽鄙,今如并之,吏多选清良者造职,大化宣流,民物获宁,四也。制使万户之县,名之郡守,五千以上,名之都尉,千户以下,令长如故。自长以上,考课迁用,转以能升,所牧亦增,此进才效功之叙也,若经制一定,则官才有次,治功齐明,五也。若省郡守,县皆径达,事不拥隔,官无留滞,三代之风,虽未可必,简一之化,庶几可致。便民省费,在於此矣。“又以为:“文质之更用,犹四时之迭兴也,王者体天理物,必因弊而济通之。时弥质则文之以礼,时泰侈则救之以质。今承百王之末,秦汉馀流,世俗弥文,宜大改之以易民望。今科制自公、列侯以下,位从大将军以上,皆得服绫锦、罗绮、纨素、金银餙镂之物,自是以下,杂彩之服,通于贱人。虽上下等级,各示有差,然朝臣之制,已得侔至尊矣,玄黄之采,已得通於下矣。欲使市不鬻华丽之色,商不通难得之货,工不作彫刻之物,不可得也。是故宜大理其本,准度古法,文质之宜,取其中则,以为礼度。车舆服章,皆从质朴,禁除末俗华丽之事,使幹朝之家,有位之室,不复有锦绮之饰,无兼采之服,纤巧之物,自上以下,至于朴素之差,示有等级而已,勿使过一二之觉。若夫功德之赐,上恩所特加,皆表之有司,然后服用之。夫上之化下,犹风之靡草。朴素之教兴於本朝,则弥侈之心自消於下矣。”

宣王报书曰:“审官择人,除重官,改服制,皆大善。礼乡闾本行,朝廷考事,大指如所示。而中间一相承习,卒不能改。秦时无刺史,但有郡守长吏。汉家虽有刺史,奉六条而已,故刺史称传车,其吏言从事,居无常治,吏不成臣,其后转更为官司耳。昔贾谊亦患服制,汉文虽身服弋绨,犹不能使上下如意。恐此三事,当待贤能然后了耳。”玄又书曰:“汉文虽身衣弋绨,而不革正法度,内外有僣拟之服,宠臣受无限之赐,由是观之,似指立在身之名,非笃齐治制之意也。今公侯命世作宰,追踪上古,将隆至治,抑末正本,若制定於上,则化行於众矣。夫当宜改之时,留殷勤之心,令发之日,下之应也犹响寻声耳,犹垂谦谦,曰'待贤能',此伊周不正殷姬之典也。窃未喻焉。”

顷之,为征西将军,假节都督雍、凉州诸军事。与曹爽共兴骆谷之役,时人讥之。爽诛,徵玄为大鸿胪,数年徙太常。玄以爽抑绌,内不得意。中书令李丰虽宿为大将军司马景王所亲待,然私心在玄,遂结皇后父光禄大夫张缉,谋欲以玄辅政。丰既内握权柄,子尚公主,又与缉俱冯翊人,故缉信之。丰阴令弟兖州刺史翼求入朝,欲使将兵入,并力起。会翼求朝,不听。嘉平六年二月,当拜贵人,丰等欲因御临轩,诸门有陛兵,诛大将军,以玄代之,以缉为骠骑将军。丰密语黄门监苏铄、永宁署令乐敦、冗从仆射刘贤等曰:“卿诸人居内,多有不法,大将军严毅,累以为言,张当可以为诫。”铄等皆许以从命。大将军微闻其谋,请丰相见,丰不知而往,即杀之。事下有司,收玄、缉、铄、敦、贤等送廷尉。廷尉锺毓奏:“丰等谋迫胁至尊,擅诛冢宰,大逆无道,请论如法。”於是会公卿朝臣廷尉议,咸以为“丰等各受殊宠,典综机密,缉承外戚椒房之尊,玄备世臣,并居列位,而包藏祸心,构图凶逆,交关阉竖,授以奸计,畏惮天威,不敢显谋,乃欲要君胁上,肆其诈虐,谋诛良辅,擅相建立,将以倾覆京室,颠危社稷。毓所正皆如科律,报毓施行”。诏书:“齐长公主,先帝遗爱,原其三子死命。”於是丰、玄、缉、敦、贤等皆夷三族,其馀亲属徙乐浪郡。玄格量弘济,临斩东巿,颜色不变,举动自若。时年四十六。正元中,绍功臣世,封尚从孙本为昌陵亭侯,邑三百户,以奉尚后。

初,中领军高阳许允与丰、玄亲善。先是有诈作尺一诏书,以玄为大将军,允为太尉,共录尚书事。有何人天未明乘马以诏版付允门吏,曰“有诏”,因便驰走。允即投书烧之,不以开呈司马景王。后丰等事觉,徙允为镇北将军,假节督河北诸军事。未发,以放散官物,收付廷尉,徙乐浪,道死。

清河王经亦与允俱称冀州名士。甘露中为尚书,坐高贵乡公事诛。始经为郡守,经母谓经曰:“汝田家子,今仕至二千石,物太过不祥,可以止矣”经不能从,历二州刺史,司隶校尉,终以致败。允友人同郡崔赞,亦尝以处世太盛戒允云。

评曰:夏侯、曹氏,世为婚姻,故惇、渊、仁、洪、休、尚、真等并以亲旧肺腑,贵重于时,左右勋业,咸有效劳。爽德薄位尊,沈溺盈溢,此固大易所著,道家所忌也。玄以规格局度,世称其名,然与曹爽中外缱绻;荣位如斯,曾未闻匡弼其非,援致良才。举兹以论,焉能免之乎!

\part{魏书十}

\chapter{荀彧荀攸贾诩传第十}

\begin{yuanwen}
荀彧\footnote{yù}字文若,颍川颍阴人也。祖父淑,字季和,朗陵令。当汉顺、桓之间,知名当世。有子八人,号曰“八龙”。彧父绲\footnote{gǔn},济南相。叔父爽,司空。\footnote{续汉书曰:淑有高才,王畅、李膺皆以为师,为朗陵侯相,号称神君。张璠汉纪曰:淑博学有高行,与李固、李膺同志友善,拔李昭于小吏,友黄叔度于幼童,以贤良方正征,对策讥切梁氏,出补朗陵侯相,卒官。八子:俭、绲、靖、焘、诜、爽、肃、旉。音敷。爽字慈明,幼好学,年十二,通春秋、论语,耽思经典,不应征命,积十数年。董卓秉政,复征爽,爽欲遁去,吏持之急。诏下郡,即拜平原相。行至苑陵,又追拜光禄勋。视事三日,策拜司空。爽起自布衣,九十五日而至三公。淑旧居西豪里,县令苑康曰昔高阳氏有才子八人,署其里为高阳里。靖字叔慈,亦有至德,名几亚爽,隐居终身。皇甫谧逸士传:或问许子将,靖与爽孰贤?子将曰:“二人皆玉也,慈明外朗,叔慈内润。”}
\end{yuanwen}

\begin{yuanwen}
彧年少时,南阳何颙异之,曰:“王佐才也。”\footnote{典略曰:中常侍唐衡欲以女妻汝南傅公明,公明不娶,转以与彧。父绲慕衡势,为彧娶之。彧为论者所讥。臣松之案:汉纪云唐衡以桓帝延熹七年死,计彧于时年始二岁,则彧婚之日,衡之没久矣。慕势之言为不然也。臣松之又以为绲八龙之一,必非苟得者也,将有逼而然,何云慕势哉?昔郑忽以违齐致讥,隽生以拒霍见美,致讥在于失援,见美嘉其虑远,并无交至之害,故得各全其志耳。至于阉竖用事,四海屏气;左悺、唐衡,杀生在口。故于时谚云“左回天,唐独坐”,言威权莫二也。顺之则六亲以安,忤违则大祸立至;斯诚以存易亡,蒙耻期全之日。昔蒋诩姻于王氏,无损清高之操,绲之此婚,庸何伤乎!}永汉元年,举孝廉,拜守宫令。董卓之乱,求出补吏。除亢父令,遂弃官归,谓父老曰:“颍川,四战之地也,天下有变,常为兵冲,宜亟去之,无久留。”乡人多怀土犹豫,会冀州牧同郡韩馥遣骑迎之,莫有随者,彧独将宗族至冀州。而袁绍已夺馥位,待彧以上宾之礼。彧弟谌及同郡辛评、郭图,皆为绍所任。彧度绍终不能成大事,时太祖为奋武将军,在东郡,初平二年,彧去绍从太祖。太祖大悦曰:“吾之子房也。”以为司马,时年二十九。
\end{yuanwen}

\begin{yuanwen}
是时,董卓威陵天下,太祖以问彧,彧曰:“卓暴虐已甚,必以乱终,无能为也。”卓遣李傕等出关东,所过虏略,至颍川、陈留而还。乡人留者多见杀略。

明年,太祖领兖州牧,后为镇东将军,彧常以司马从。

兴平元年,太祖征陶谦,任彧留事。会张邈、陈宫以兖州反,潜迎吕布。布既至,邈乃使刘翊告彧曰:“吕将军来助曹使君击陶谦,宜亟供其军食。”众疑惑。彧知邈为乱,即勒兵设备,驰召东郡太守夏侯惇,而兖州诸城皆应布矣。时太祖悉军攻谦,留守兵少,而督将大吏多与邈、宫通谋。惇至,其夜诛谋叛者数十人,众乃定。

豫州刺史郭贡帅众数万来至城下,或言与吕布同谋,众甚惧。贡求见彧,彧将往。惇等曰:“君,一州镇也,往必危,不可。”

彧曰:“贡与邈等,分非素结也,今来速,计必未定;及其未定说之,纵不为用,可使中立,若先疑之,彼将怒而成计。”

贡见彧无惧意,谓鄄城未易攻,遂引兵去。又与程昱计,使说范、东阿,卒全三城,以待太祖。太祖自徐州还击布濮阳,布东走。

二年夏,太祖军乘氏,大饥,人相食。
\end{yuanwen}

\begin{yuanwen}
陶谦死,太祖欲遂取徐州,还乃定布。

彧曰:“昔高祖保关中,光武据河内,皆深根固本以制天下,进足以胜敌,退足以坚守,故虽有困败而终济大业。将军本以兖州首事,平山东之难,百姓无不归心悦服。且河、济,天下之要地也,今虽残坏,犹易以自保,是亦将军之关中、河内也,不可以不先定。今以破李封、薛兰,若分兵东击陈宫,宫必不敢西顾,以其间勒兵收熟麦,约食畜谷,一举而布可破也。破布,然后南结扬州,共讨袁术,以临淮、泗。若舍布而东,多留兵则不足用,少留兵则民皆保城,不得樵采。布乘虚寇暴,民心益危,唯鄄城、范、卫可全,其馀非己之有,是无兖州也。若徐州不定,将军当安所归乎?且陶谦虽死,徐州未易亡也。彼惩往年之败,将惧而结亲,相为表里。今东方皆以收麦,必坚壁清野以待将军,将军攻之不拔,略之无获,不出十日,则十万之众未战而自困耳。\footnote{臣松之以为于时徐州未平,兖州又叛,而云十万之众,虽是抑抗之言,要非寡弱之称。益知官渡之役,不得云兵不满万也。}前讨徐州,威罚实行,\footnote{曹瞒传云:自京师遭董卓之乱,人民流移东出,多依彭城间。遇太祖至,坑杀男女数万口于泗水,水为不流。陶谦帅其众军武原,太祖不得进。引军从泗南攻取虑、睢陵、夏丘诸县,皆屠之;鸡犬亦尽,墟邑无复行人。}其子弟念父兄之耻,必人自为守,无降心,就能破之,尚不可有也。夫事固有弃此取彼者,以大易小可也,以安易危可也,权一时之势,不患本之不固可也。今三者莫利,愿将军熟虑之。”

太祖乃止。大收麦,复与布战,分兵平诸县。布败走,兖州遂平。
\end{yuanwen}

\begin{yuanwen}
建安元年,太祖击破黄巾。汉献帝自河东还洛阳。太祖议奉迎都许,或以山东未平,韩暹、杨奉新将天子到洛阳,北连张杨,未可卒制。

彧劝太祖曰:“昔晋文纳周襄王而诸侯景从,高祖东伐为义帝缟素而天下归心。自天子播越,将军首唱义兵,徒以山东扰乱,未能远赴关右,然犹分遣将帅,蒙险通使,虽御难于外,乃心无不在王室,是将军匡天下之素志也。今车驾旋轸,东京榛芜,义士有存本之思,百姓感旧而增哀。诚因此时,奉主上以从民望,大顺也;秉至公以服雄杰,大略也;扶弘义以致英俊,大德也。天下虽有逆节,必不能为累,明矣。韩暹、杨奉其敢为害!若不时定,四方生心,后虽虑之,无及。”

太祖遂至洛阳,奉迎天子都许。天子拜太祖大将军,进彧为汉侍中,守尚书令。常居中持重,\footnote{典略曰:彧折节下士,坐不累席。其在台阁,不以私欲挠意。彧有群从一人,才行实薄,或谓彧:“以君当事,不可不以某为议郎邪?”彧笑曰:“官者所以表才也,若如来言,众人其谓我何邪!”其持心平正皆类此。}太祖虽征伐在外,军国事皆与彧筹焉。\footnote{典略曰:彧为人伟美。又平原祢衡传曰:衡字正平,建安初,自荆州北游许都,恃才傲逸,臧否过差,见不如己者不与语,人皆以是憎之。唯少府孔融高贵其才,上书荐之曰:“淑质贞亮,英才卓荦。初涉艺文,升堂睹奥;目所一见,辄诵于口,耳所暂闻,不忘于心。性与道合,思若有神。弘羊心计,安世默识,以衡准之,诚不足怪。”衡时年二十四。是时许都虽新建,尚饶人士。衡尝书一刺怀之,字漫灭而无所适。或问之曰:“何不从陈长文、司马伯达乎?”衡曰:“卿欲使我从屠沽儿辈也!”又问曰:“当今许中,谁最可者?”衡曰:“大儿有孔文举,小儿有杨德祖。”又问:“曹公、荀令君、赵荡寇皆足盖世乎?”衡称曹公不甚多;又见荀有仪容,赵有腹尺,因答曰:“文若可借面吊丧,稚长可使监厨请客。”其意以为荀但有貌,赵健啖肉也。于是众人皆切齿。衡知众不悦,将南还荆州。装束临发,众人为祖道,先设供帐于城南,自共相诫曰:“衡数不逊,今因其后到,以不起报之。”及衡至,众人皆坐不起,衡乃号啕大哭。众人问其故,衡曰:“行尸柩之间,能不悲乎?”衡南见刘表,表甚礼之。将军黄祖屯夏口,祖子射与衡善,随到夏口。祖嘉其才,每在坐,席有异宾,介使与衡谈。后衡骄蹇,答祖言徘优饶言,祖以为骂己也,大怒,顾伍伯捉头出。左右遂扶以去,拉而杀之。臣松之以本传不称彧容貌,故载典略与衡传以见之。又潘勖为彧碑文,称彧“瑰姿奇表”。张衡文士传曰:孔融数荐衡于太祖,欲与相见,而衡疾恶之,意常愤懑。因狂疾不肯往,而数有言论。太祖闻其名,图欲辱之,乃录为鼓【吏】史。后至八月朝,大宴,宾客并会。时鼓【吏】史击鼓过,皆当脱其故服,易着新衣。次衡,衡击为渔阳参挝,容态不常,节殊妙。坐上宾客听之,莫不慷慨。过不易衣,吏呵之,衡乃当太祖前,以次脱衣,裸身而立,徐徐乃着裈帽毕,复击鼓参挝,而颜色不怍。太祖大笑,告四坐曰:“本欲辱衡,衡反辱孤。”至今有渔阳参挝,自衡造也。融深责数衡,并宣太祖意,欲令与太祖相见。衡许之,曰:“当为卿往。”至十月朝,融先见太祖,说“衡欲求见”。至日晏,衡着布单衣,【疏巾】綀布履,坐太祖营门外,以杖捶地,数骂太祖。太祖敕外厩急具精马三匹,并骑二人,谓融曰:“祢衡竖子,乃敢尔!孤杀之无异于雀鼠,顾此人素有虚名,远近所闻,今日杀之,人将谓孤不能容。今送与刘表,视卒当何如?”乃令骑以衡置马上,两骑扶送至南阳。傅子曰:衡辩于言而克于论,见荆州牧刘表日,所以自结于表者甚至,表悦之以为上宾。衡称表之美盈口,而论表左右不废绳墨。于是左右因形而谮之,曰:“衡称将军之仁,西伯不过也,唯以为不能断;终不济者,必由此也。”是言实指表智短,而非衡所言也。表不详察,遂疏衡而逐之。衡以交绝于刘表,智穷于黄祖,身死名灭,为天下笑者,谮之者有形也。}

太祖问彧:“谁能代卿为我谋者?”

彧言“荀攸、钟繇”。

先是,彧言策谋士,进戏志才。志才卒,又进郭嘉。太祖以彧为知人,诸所进达皆称职,唯严象为扬州,韦康为凉州,后败亡。\footnote{三辅决录注曰:象字文则,京兆人。少聪博,有胆智。以督军御史中丞诣扬州讨袁术,会术病卒,因以为扬州刺史。建安五年,为孙策庐江太守李术所杀,时年三十八。象同郡赵岐作三辅决录,恐时人不尽其意,故隐其书,唯以示象。康字元将,亦京兆人。孔融与康父端书曰:“前日元将来,渊才亮茂,雅度弘毅,伟世之器也。昨日仲将又来,懿性贞实,文敏笃诚,保家之主也。不意双珠,近出老蚌,甚珍贵之。”端从凉州牧征为太仆,康代为凉州刺史,时人荣之。后为马超所围,坚守历时,救军不至,遂为超所杀。仲将名诞,见刘邵传。}
\end{yuanwen}

\begin{yuanwen}
自太祖之迎天子也,袁绍内怀不服。绍既并河朔,天下畏其强。太祖方东忧吕布,南拒张绣,而绣败太祖军于宛。绍益骄,与太祖书,其辞悖慢。太祖大怒,出入动静变于常,众皆谓以失利于张绣故也。钟繇以问彧,彧曰:“公之聪明,必不追咎往事,殆有他虑。”

则见太祖问之,太祖乃以绍书示彧,曰:“今将讨不义,而力不敌,何如?”

彧曰:“古之成败者,诚有其才,虽弱必强,苟非其人,虽强易弱,刘、项之存亡,足以观矣。今与公争天下者,唯袁绍尔。绍貌外宽而内忌,任人而疑其心,公明达不拘,唯才所宜,此度胜也。绍迟重少决,失在后机,公能断大事,应变无方,此谋胜也。绍御军宽缓,法令不立,士卒虽众,其实难用,公法令既明,赏罚必行,士卒虽寡,皆争致死,此武胜也。绍凭世资,从容饰智,以收名誉,故士之寡能好问者多归之,公以至仁待人,推诚心不为虚美,行己谨俭,而与有功者无所吝惜,故天下忠正效实之士咸愿为用,此德胜也。夫以四胜辅天子,扶义征伐,谁敢不从?绍之强其何能为!”

太祖悦。彧曰:“不先取吕布,河北亦未易图也。”

太祖曰:“然。吾所惑者,又恐绍侵扰关中,乱羌、胡,南诱蜀汉,是我独以兖、豫抗天下六分之五也。为将奈何?”

彧曰:“关中将帅以十数,莫能相一,唯韩遂、马超最强。彼见山东方争,必各拥众自保。今若抚以恩德,遣使连和,相持虽不能久安,比公安定山东,足以不动。钟繇可属以西事。则公无忧矣。”
\end{yuanwen}

\begin{yuanwen}
三年,太祖既破张绣,东擒吕布,定徐州,遂与袁绍相拒。孔融谓彧曰:“绍地广兵强;田丰、许攸,智计之士也,为之谋;审配、逢纪,尽忠之臣也,任其事;颜良、文丑,勇冠三军,统其兵:殆难克乎!”

彧曰:“绍兵虽多而法不整。田丰刚而犯上,许攸贪而不治。审配专而无谋,逢纪果而自用,此二人留知后事,若攸家犯其法,必不能纵也,不纵,攸必为变。颜良、文丑,一夫之勇耳,可一战而禽也。”
\end{yuanwen}

\begin{yuanwen}
五年,与绍连战。太祖保官渡,绍围之。太祖军粮方尽,书与彧,议欲还许以引绍。彧曰:“今军食虽少,未若楚、汉在荥阳、成皋间也。是时刘、项莫肯先退,先退者势屈也。公以十分居一之众,画地而守之,扼其喉而不得进,已半年矣。情见势竭,必将有变,此用奇之时,不可失也。”

太祖乃住。遂以奇兵袭绍别屯,斩其将淳于琼等,绍退走。审配以许攸家不法,收其妻子,攸怒叛绍;颜良、文丑临阵授首;田丰以谏见诛:皆如彧所策。
\end{yuanwen}

\begin{yuanwen}
六年,太祖就谷东平之安民,粮少,不足与河北相支,欲因绍新破,以其间击讨刘表。彧曰:“今绍败,其众离心,宜乘其困,遂定之;而背兖、豫,远师江、汉,若绍收其馀烬,承虚以出人后,则公事去矣。”

太祖复次于河上。绍病死。太祖渡河,击绍子谭、尚,而高幹、郭援侵略河东,关右震动,钟繇帅马腾等击破之。语在《繇传》。

八年,太祖录彧前后功,表封彧为万岁亭侯。\footnote{彧别传载太祖表曰:“臣闻虑为功首,谋为赏本,野绩不越庙堂,战多不逾国勋。是故曲阜之锡,不后营丘,萧何之土,先于平阳。珍策重计,古今所尚。侍中守尚书令彧,积德累行,少长无悔,遭世纷扰,怀忠念治。臣自始举义兵,周游征伐,与彧戮力同心,左右王略,发言授策,无施不效。彧之功业,臣由以济,用披浮云,显光日月。陛下幸许,彧左右机近,忠恪祗顺,如履薄冰,研精极锐,以抚庶事。天下之定,彧之功也。宜享高爵,以彰元勋。”彧固辞无野战之劳,不通太祖表。太祖与彧书曰:“与君共事已来,立朝廷,君之相为匡弼,君之相为举人,君之相为建计,君之相为密谋,亦以多矣。夫功未必皆野战也,愿君勿让。”彧乃受。}

九年,太祖拔邺,领冀州牧。彧说太祖“宜复古置九州,则冀州所制者广大,天下服矣。”

太祖将从之,彧言曰:“若是,则冀州当得河东、冯翊、扶风、西河、幽、并之地,所夺者众。前日公破袁尚,禽审配,海内震骇,必人人自恐不得保其土地,守其兵众也;今使分属冀州,将皆动心。且人多说关右诸将以闭关之计;今闻此,以为必以次见夺。一旦生变,虽有守善者,转相胁为非,则袁尚得宽其死,而袁谭怀贰,刘表遂保江、汉之间,天下未易图也。愿公急引兵先定河北,然后修复旧京,南临荆州,责贡之不入,则天下咸知公意,人人自安。天下大定,乃议古制,此社稷长久之利也。”太祖遂寝九州议。
\end{yuanwen}

\begin{yuanwen}
是时荀攸常为谋主。彧兄衍以监军校尉守邺,都督河北事。太祖之征袁尚也,高幹密遣兵谋袭邺,衍逆觉,尽诛之,以功封列侯。\footnote{荀氏家传曰:衍字休若,彧第三兄。彧第四兄谌,字友若,事见袁绍传。陈群与孔融论汝、颍人物,群曰:“荀文若、公达、休若、友若、仲豫,当今并无对。”衍子绍,位至太仆。绍子融,字伯雅,与王弼、钟会俱知名,为洛阳令,参大将军军事,与弼、会论易、老义,传于世。谌子闳,字仲茂,为太子文学掾。时有甲乙疑论,闳与钟繇、王朗、袁涣议各不同。文帝与繇书曰“袁、王国士,更为唇齿,荀闳劲悍,往来锐师,真君侯之勍敌,左右之深忧也。”终黄门侍郎。闳从孙【恽】煇字景文,太子中庶子,亦知名。与贾充共定音律,又作易集解。仲豫名悦,郎陵长俭之少子,彧从父兄也。张璠汉纪称悦清虚沈静,善于著述。建安初为秘书监侍中,被诏删汉书作汉纪三十篇,因事以明臧否,致有典要;其书大行于世。}太祖以女妻彧长子恽,后称安阳公主。彧及攸并贵重,皆谦冲节俭,禄赐散之宗族知旧,家无馀财。十二年,复增彧邑千户,合二千户。\footnote{彧别传曰:太祖又表曰:“昔袁绍侵入郊甸,战于官渡。时兵少粮尽,图欲还许,书与彧议,彧不听臣。建宜住之便,恢进讨之规,更起臣心,易其愚虑,遂摧大逆,覆取其众。此彧睹胜败之机,略不世出也。及绍破败,臣粮亦尽,以为河北未易图也,欲南讨刘表。彧复止臣,陈其得失,臣用反旆,遂吞凶族,克平四州。向使臣退于官渡,绍必鼓行而前,有倾覆之形,无克捷之势。后若南征,委弃兖、豫,利既难要,将失本据。彧之二策,以亡为存,以祸致福,谋殊功异,臣所不及也。是以先帝贵指纵之功,薄搏获之赏;古人尚帷幄之规,下攻拔之捷。前所赏录,未副彧巍巍之勋,乞重平议,畴其户邑。”彧深辞让,太祖报之曰:“君之策谋,非但所表二事。前后谦冲,欲慕鲁连先生乎?此圣人达节者所不贵也。昔介子推有言‘窃人之财,犹谓之盗’。况君密谋安众,光显于孤者以百数乎!以二事相还而复辞之,何取谦亮之多邪!”太祖欲表彧为三公,彧使荀攸深让,至于十数,太祖乃止。}
\end{yuanwen}

\begin{yuanwen}
太祖将伐刘表,问彧策安出,彧曰:“今华夏已平,南土知困矣。可显出宛、叶而间行轻进,以掩其不意。”

太祖遂行。会表病死,太祖直趋宛、叶如彧计,表子琮以州逆降。
\end{yuanwen}

\begin{yuanwen}
十七年,董昭等谓太祖宜进爵国公,九锡备物,以彰殊勋,密以谘彧。彧以为太祖本兴义兵以匡朝宁国,秉忠贞之诚,守退让之实;君子爱人以德,不宜如此。太祖由是心不能平。会征孙权,表请彧劳军于谯,因辄留彧,以侍中光禄大夫持节,参丞相军事。太祖军至濡须,彧疾留寿春,以忧薨,时年五十。谥曰敬侯。

明年,太祖遂为魏公矣。\footnote{魏氏春秋曰:太祖馈彧食,发之乃空器也,于是饮药而卒。咸熙二年,赠彧太尉。彧别传曰:彧自为尚书令,常以书陈事,临薨,皆焚毁之,故奇策密谋不得尽闻也。是时征役草创,制度多所兴复,彧尝言于太祖曰:“昔舜分命禹、稷、契、皋陶以揆庶绩,教化征伐,并时而用。及高祖之初,金革方殷,犹举民能善教训者,叔孙通习礼仪于戎旅之间,世祖有投戈讲艺、息马论道之事,君子无终食之间违仁。今公外定武功,内兴文学,使干戈戢睦,大道流行,国难方弭,六礼俱治,此姬旦宰周之所以速平也。既立德立功,而又兼立言,诚仲尼述作之意;显制度于当时,扬名于后世,岂不盛哉!若须武事毕而后制作,以稽治化,于事未敏。宜集天下大才通儒,考论六经,刊定传记,存古今之学,除其烦重,以一圣真,并隆礼学,渐敦教化,则王道两济。”彧从容与太祖论治道,如此之类甚众,太祖常嘉纳之。彧德行周备,非正道不用心,名重天下,莫不以为仪表,海内英隽咸宗焉。司马宣王常称书传远事,吾自耳目所从闻见,逮百数十年间,贤才未有及荀令君者也。前后所举者,命世大才,邦邑则荀攸、钟繇、陈群,海内则司马宣王,及引致当世知名郗虑、华歆、王朗、荀悦、杜袭、辛毗、赵俨之俦,终为卿相,以十数人。取士不以一揆,戏志才、郭嘉等有负俗之讥,杜畿简傲少文,皆以智策举之,终各显名。荀攸后为魏尚书令,亦推贤进士。太祖曰:“二荀令之论人,久而益信,吾没世不忘。”钟繇以为颜子既没,能备九德,不贰其过,唯荀彧然。或问繇曰:“君雅重荀君,比之颜子,自以不及,可得闻乎?”曰:“夫明君师臣,其次友之。以太祖之聪明,每有大事,常先谘之荀君,是则古师友之义也。吾等受命而行,犹或不尽,相去顾不远邪!”献帝春秋曰:董承之诛,伏后与父完书,言司空杀董承,帝方为报怨。完得书以示彧,彧恶之,久隐而不言。完以示妻弟樊普,普封以呈太祖,太祖阴为之备。彧后恐事觉,欲自发之,因求使至邺,劝太祖以女配帝。太祖曰:“今朝廷有伏后,吾女何得以配上,吾以微功见录,位为宰相,岂复赖女宠乎!”彧曰:“伏后无子,性又凶邪,往常与父书,言辞丑恶,可因此废也。”太祖曰:“卿昔何不道之?”彧阳惊曰:“昔已尝为公言也。”太祖曰:“此岂小事而吾忘之!”彧又惊曰:“诚未语公邪!昔公在官渡与袁绍相持,恐增内顾之念,故不言尔。”太祖曰:“官渡事后何以不言?”彧无对,谢阙而已。太祖以此恨彧,而外含容之,故世莫得知。至董昭建立魏公之议,彧意不同,欲言之于太祖。及赍玺书犒军,饮飨礼毕,彧留请间。太祖知彧欲言封事,揖而遣之,彧遂不得言。彧卒于寿春,寿春亡者告孙权,言太祖使彧杀伏后,彧不从,故自杀。权以露布于蜀,刘备闻之,曰:“老贼不死,祸乱未已。”臣松之案献帝春秋云彧欲发伏后事而求使至邺,而方诬太祖云“昔已尝言”。言既无征,回讬以官渡之虞,俯仰之间,辞情顿屈,虽在庸人,犹不至此,何以玷累贤哲哉!凡诸云云,皆出自鄙俚,可谓以吾侪之言而厚诬君子者矣。袁暐虚罔之类,此最为甚也。}
\end{yuanwen}

\begin{yuanwen}
子恽,嗣侯,官至虎贲中郎将。初,文帝与平原侯植并有拟论,文帝曲礼事彧。及彧卒,恽又与植善,而与夏侯尚不穆,文帝深恨恽。恽早卒,子甝、霬【音翼。】以外甥故犹宠待。恽弟俣,御史中丞,俣弟诜,大将军从事中郎,皆知名,早卒。【荀氏家传曰:恽字长倩,俣字叔倩,诜字曼倩,俣子寓,字景伯。世语曰:寓少与裴楷、王戎、杜默俱有名京邑,仕晋,位至尚书,名见显著。子羽嗣,位至尚书。】诜弟顗,咸熙中为司空。【晋阳秋曰:顗字景倩,幼为姊夫陈群所异。博学洽闻,意思慎密。司马宣王见顗,奇之,曰:“荀令君之子也。近见袁偘,亦曜卿之子也。”擢拜散骑侍郎。顗佐命晋室,位至太尉,封临淮康公。尝难钟会“易无互体”,见称于世。顗弟粲,字奉倩。何劭为粲传曰:粲字奉倩,粲诸兄并以儒术论议,而粲独好言道,常以为子贡称夫子之言性与天道,不可得闻,然则六籍虽存,固圣人之糠秕。粲兄俣难曰:“易亦云圣人立象以尽意,系辞焉以尽言,则微言胡为不可得而闻见哉?”粲答曰:“盖理之微者,非物象之所举也。今称立象以尽意,此非通于意外者也。系辞焉以尽言,此非言乎系表者也;斯则象外之意,系表之言,固蕴而不出矣。”及当时能言者不能屈也。又论父彧不如从兄攸。彧立德高整,轨仪以训物,而攸不治外形,慎密自居而已。粲以此言善攸,诸兄怒而不能回也。太和初,到京邑与傅嘏谈。嘏善名理而粲尚玄远,宗致虽同,仓卒时或有格而不相得意。裴徽通彼我之怀,为二家骑驿,顷之,粲与嘏善。夏侯玄亦亲。常谓嘏、玄曰:“子等在世涂间,功名必胜我,但识劣我耳!”嘏难曰:“能盛功名者,识也。天下孰有本不足而末有馀者邪?”粲曰:“功名者,志局之所奖也。然则志局自一物耳,固非识之所独济也。我以能使子等为贵,然未必齐子等所为也。”粲常以妇人者,才智不足论,自宜以色为主。骠骑将军曹洪女有美色,粲于是娉焉,容服帷帐甚丽,专房欢宴。历年后,妇病亡,未殡,傅嘏往喭粲;粲不哭而神伤。嘏问曰:“妇人才色并茂为难。子之娶也,遗才而好色。此自易遇,今何哀之甚?”粲曰:“佳人难再得!顾逝者不能有倾国之色,然未可谓之易遇。”痛悼不能已,岁馀亦亡,时年二十九。粲简贵,不能与常人交接,所交皆一时俊杰。至葬夕,赴者裁十馀人,皆同时知名士也,哭之,感动路人。】恽子甝,嗣为散骑常侍,进爵广阳乡侯,年三十薨。子頵嗣。【荀氏家传曰:頵字温伯,为羽林右监,早卒。頵子崧,字景猷。晋阳秋称崧少有志操,雅好文学,孝义和爱,在朝恪勤,位至左右光禄大夫、开府仪同三司。崧子羡,字令则,清和有才。尚公主,少历显位,年二十八为北中郎将,徐、兖二州刺史,假节都督徐、兖、青三州诸军事。在任十年,遇疾解职,卒于家,追赠骠骑将军。羡孙伯子,今御史中丞也。】霬官至中领军,薨,谥曰贞侯,追赠骠骑将军。子恺嗣。霬妻,司马景王、文王之妹也,二王皆与亲善。咸熙中,开建五等,霬以着勋前朝,改封恺南顿子。【荀氏家传曰:恺,晋武帝时为侍中。干宝晋纪曰:武帝使侍中荀顗、和峤俱至东宫,观察太子。顗还称太子德识进茂,而峤云圣质如初。孙盛曰“遣荀勖”,其馀语则同。臣松之案和峤为侍中,荀顗亡没久矣。荀勖位亚台司,不与峤同班,无缘方称侍中。二书所云,皆为非也。考其时位,恺寔当之。恺位至征西大将军。恺兄憺,少府。弟悝,护军将军,追赠车骑大将军。】
\end{yuanwen}

\begin{yuanwen}
荀攸字公达,彧从子也。祖父昙,广陵太守。\footnote{荀氏家传曰:昙字元智。兄昱,字伯修。张璠汉纪称昱、昙并杰俊有殊才。昱与李膺、王畅、杜密等号为八俊,位至沛相。攸父彝,州从事。彝于彧为从祖兄弟。}攸少孤。及昙卒,故吏张权求守昙墓。攸年十三,疑之,谓叔父衢\footnote{q\'u}曰:“此吏有非常之色,殆将有奸!”衢寤,乃推问,果杀人亡命。由是异之。\footnote{魏书曰:攸年七八岁,衢曾醉,误伤攸耳;而攸出入游戏,常避护不欲令衢见。衢后闻之,乃惊其夙智如此。荀氏家传曰:衢子祈,字伯旗,与族父愔俱著名。祈与孔融论肉刑,愔与孔融论圣人优劣,并在融集。祈位至济阴太守;愔后征有道,至丞相祭酒。}何进秉政,征海内名士攸等二十馀人。攸到,拜黄门侍郎。董卓之乱,关东兵起,卓徙都长安。攸与议郎郑泰、何颙、侍中种辑、越骑校尉伍琼等谋曰:“董卓无道,甚于桀纣,天下皆怨之,虽资强兵,实一匹夫耳。今直刺杀之以谢百姓,然后据崤、函,辅王命,以号令天下,此桓文之举也。”事垂就而觉,收颙、攸系狱,颙忧惧自杀,\footnote{张璠汉纪曰:颙字伯求,少与郭泰、贾彪等游学洛阳,泰等与同风好。颙显名太学,于是中朝名臣太傅陈蕃、司隶李膺等皆深接之。及党事起,颙亦名在其中,乃变名姓亡匿汝南间,所至皆交结其豪桀。颙既奇太祖而知荀彧,袁绍慕之,与为奔走之友。是时天下士大夫多遇党难,颙常岁再三私入洛阳,从绍计议,为诸穷窘之士解释患祸。而袁术亦豪侠,与绍争名。颙未常造术,术深恨之。汉末名士录曰:术常于众坐数颙三罪,曰:“王德弥先觉隽老,名德高亮,而伯求疏之,是一罪也。许子远凶淫之人,性行不纯,而伯求亲之,是二罪也。郭、贾寒窭,无他资业,而伯求肥马轻裘,光耀道路,是三罪也。”陶丘洪曰:“王德弥大贤而短于济时,许子远虽不纯而赴难不惮濡足。伯求举善则以德弥为首,济难则以子远为宗。且伯求尝为虞伟高手刃复仇,义名奋发。其怨家积财巨万,文马百驷,而欲使伯求羸牛疲马,顿伏道路,此为披其胸而假仇敌之刃也。”术意犹不平。后与南阳宗承会于阙下,术发怒曰:“何伯求,凶德也,吾当杀之。”承曰:“何生英俊之士,足下善遇之,使延令名于天下。”术乃止。后党禁除解,辟司空府。每三府掾属会议,颙策谋有馀,议者皆自以为不及。迁北军中候,董卓以为长史。后荀彧为尚书令,遣人迎叔父司空爽丧,使并置颙尸,而葬之于爽冢傍。}攸言语饮食自若,会卓死得免。\footnote{魏书云攸使人说卓得免,与此不同。}弃官归,复辟公府,举高第,还任城相,不行。攸以蜀汉险固,人民殷盛,乃求为蜀郡太守,道绝不得至,驻荆州。
\end{yuanwen}

\begin{yuanwen}
太祖迎天子都许,遗攸书曰:“方今天下大乱,智士劳心之时也,而顾观变蜀汉,不已久乎!”于是征攸为汝南太守,入为尚书。

太祖素闻攸名,与语大悦,谓荀彧、钟繇曰:“公达,非常人也,吾得与之计事,天下当何忧哉!”以为军师。

建安三年,从征张绣。攸言于太祖曰:“绣与刘表相恃为强,然绣以游军仰食于表,表不能供也,势必离。不如缓军以待之,可诱而致也;若急之,其势必相救。”

太祖不从,遂进军之穰,与战。绣急,表果救之。军不利。

太祖谓攸曰:“不用君言至是。”乃设奇兵复战,大破之。
\end{yuanwen}

\begin{yuanwen}
是岁,太祖自宛征吕布,\footnote{魏书曰:议者云表、绣在后而还袭吕布,其危必也。攸以为表、绣新破,势不敢动。布骁猛,又恃袁术,若纵横淮、泗间,豪杰必应之。今乘其初叛,众心未一,往可破也。太祖曰:“善。”比行,布以败刘备,而臧霸等应之。}至下邳,布败退固守,攻之不拔,连战,士卒疲,太祖欲还。攸与郭嘉说曰:“吕布勇而无谋,今三战皆北,其锐气衰矣。三军以将为主,主衰则军无奋意。夫陈宫有智而迟,今及布气之未复,宫谋之未定,进急攻之,布可拔也。”乃引沂、泗灌城,城溃,生禽布。
\end{yuanwen}

\begin{yuanwen}
后从救刘延于白马,攸画策斩颜良。语在《武纪》。太祖拔白马还,遣辎重循河而西。袁绍渡河追,卒与太祖遇。诸将皆恐,说太祖还保营,攸曰:“此所以禽敌,奈何去之!”

太祖目攸而笑。遂以辎重饵贼,贼竞奔之,陈乱。乃纵步骑击,大破之,斩其骑将文丑,太祖遂与绍相拒于官渡。

军食方尽,攸言于太祖曰:“绍运车旦暮至,其将韩𦳣\footnote{xún}锐而轻敌,击可破也。”\footnote{臣松之案诸书,韩𦳣或作韩猛,或云韩若,未详孰是。}

太祖曰:“谁可使?”

攸曰:“徐晃可。”乃遣晃及史涣邀击破走之,烧其辎重。会许攸来降,言绍遣淳于琼等将万馀兵迎运粮,将骄卒惰,可要击也。众皆疑。唯攸与贾诩劝太祖。太祖乃留攸及曹洪守。太祖自将攻破之,尽斩琼等。绍将张郃、高览烧攻橹降,绍遂弃军走。郃之来,洪疑不敢受,攸谓洪曰:“郃计不用,怒而来,君何疑?”乃受之。
\end{yuanwen}

\begin{yuanwen}
七年,从讨袁谭、尚于黎阳。

明年,太祖方征刘表,谭、尚争冀州。谭遣辛毗乞降请救,太祖将许之,以问群下。群下多以为表强,宜先平之,谭、尚不足忧也。攸曰:“天下方有事,而刘表坐保江、汉之间,其无四方志可知矣。袁氏据四州之地,带甲十万,绍以宽厚得众,借使二子和睦以守其成业,则天下之难未息也。今兄弟遘\footnote{g\`ou}恶,此势不两全。若有所并则力专,力专则难图也。及其乱而取之,天下定矣,此时不可失也。”

太祖曰:“善。”乃许谭和亲,遂还击破尚。其后谭叛,从斩谭于南皮。

冀州平,太祖表封攸曰:“军师荀攸,自初佐臣,无征不从,前后克敌,皆攸之谋也。”于是封陵树亭侯。

十二年,下令大论功行封,太祖曰:“忠正密谋,抚宁内外,文若是也。公达其次也。”增邑四百,并前七百户,\footnote{魏书曰:太祖自柳城还,过攸舍,称述攸前后谋谟劳勋,曰:“今天下事略已定矣,孤愿与贤士大夫共飨其劳。昔高祖使张子房自择邑三万户,今孤亦欲君自择所封焉。”}转为中军师。魏国初建,为尚书令。
\end{yuanwen}

\begin{yuanwen}
攸深密有智防,自从太祖征伐,常谋谟帷幄,时人及子弟莫知其所言。\footnote{魏书曰:攸姑子辛韬曾问攸说太祖取冀州时事。攸曰:“佐治为袁谭乞降,王师自往平之,吾何知焉?”自是韬及内外莫敢复问军国事也。}太祖每称曰:“公达外愚内智,外怯内勇,外弱内强,不伐善,无施劳,智可及,愚不可及,虽颜子、甯武不能过也。”

文帝在东宫,太祖谓曰:“荀公达,人之师表也,汝当尽礼敬之。”

攸曾病,世子问病,独拜床下,其见尊异如此。攸与钟繇善,繇言:“我每有所行,反覆思惟,自谓无以易;以咨公达,辄复过人意。”

公达前后凡画奇策十二,唯繇知之。繇撰集未就,会薨,故世不得尽闻也。\footnote{臣松之案:攸亡后十六年,钟繇乃卒,撰攸奇策,亦有何难?而年造八十,犹云未就,遂使攸从征机策之谋不传于世,惜哉!}

攸从征孙权,道薨。太祖言则流涕。\footnote{魏书曰:时建安十九年,攸年五十八。计其年大彧六岁。魏书载太祖令曰:“孤与荀公达周游二十馀年,无毫毛可非者。”又曰:“荀公达真贤人也,所谓‘温良恭俭让以得之’。孔子称‘晏平仲善与人交,久而敬之’,公达即其人也。”傅子曰:或问近世大贤君子,答曰:“荀令君之仁,荀军师之智,斯可谓近世大贤君子矣。荀令君仁以立德,明以举贤,行无谄赎,谋能应机。孟轲称‘五百年而有王者兴,其间必有命世者’,其荀令君乎!太祖称‘荀令君之进善,不进不休,荀军师之去恶,不去不止’也。”}
\end{yuanwen}

\begin{yuanwen}
长子缉,有攸风,早没。次子适嗣,无子,绝。黄初中,绍封攸孙彪为陵树亭侯,邑三百户,后转封丘阳亭侯。正始中,追谥攸曰敬侯。
\end{yuanwen}

\begin{yuanwen}
贾诩字文和,武威姑臧人也。少时人莫知,唯汉阳阎忠异之,谓诩有良、平之奇。\footnote{九州春秋曰:中平元年,车骑将军皇甫嵩既破黄巾,威震天下。阎忠时罢信都令,说嵩曰:“夫难得而易失者时也,时至而不旋踵者机也,故圣人常顺时而动,智者必因机以发。今将军遭难得之运,蹈易解之机,而践运不抚,临机不发,将何以享大名乎?”嵩曰:“何谓也?”忠曰:“天道无亲,百姓与能,故有高人之功者,不受庸主之赏。今将军授钺于初春,收功于末冬,兵动若神,谋不再计,旬月之间,神兵电扫,攻坚易于折枯,摧敌甚于汤雪,七州席卷,屠三十六【万】方,夷黄巾之师,除邪害之患,或封户刻石,南向以报德,威震本朝,风驰海外。是以群雄回首,百姓企踵,虽汤武之举,未有高于将军者。身建高人之功,北面以事庸主,将何以图安?”嵩曰:“心不忘忠,何为不安?”忠曰:“不然。昔韩信不忍一餐之遇,而弃三分之利,拒蒯通之忠,忽鼎跱之势,利剑已揣其喉,乃叹息而悔,所以见烹于儿女也。今主势弱于刘、项,将军权重于淮阴,指麾可以振风云,叱咤足以兴雷电;赫然奋发,因危抵颓,崇恩以绥前附,振武以临后服;征冀方之士,动七州之众,羽檄先驰于前,大军震响于后,蹈迹漳河,饮马孟津,举天网以网罗京都,诛阉宦之罪,除群怨之积忿,解久危之倒悬。如此则攻守无坚城,不招必影从,虽儿童可使奋空拳以致力,女子可使其褰裳以用命,况厉智能之士,因迅风之势,则大功不足合,八方不足同也。功业已就,天下已顺,乃燎于上帝,告以天命,混齐六合,南面以制,移神器于己家,推亡汉以定祚,实神机之至决,风发之良时也。夫木朽不雕,世衰难佐,将军虽欲委忠难佐之朝,雕画朽败之木,犹逆坂而走丸,必不可也。方今权宦群居,同恶如市,主上不自由,诏命出左右。如有至聪不察,机事不先,必婴后悔,亦无及矣。”嵩不从,忠乃亡去。英雄记曰:凉州贼王国等起兵,共劫忠为主,统三十六部,号车骑将军。忠感慨发病而死。}察孝廉为郎,疾病去官,西还至汧\footnote{qi\=an},道遇叛氐,同行数十人皆为所执。诩曰:“我段公外孙也,汝别埋我,我家必厚赎之。”

时太尉段颎,昔久为边将,威震西土,故诩假以惧氐。氐果不敢害,与盟而送之,其馀悉死。诩实非段甥,权以济事,咸此类也。
\end{yuanwen}

\begin{yuanwen}
董卓之入洛阳,诩以太尉掾为平津都尉,迁讨虏校尉。卓婿中郎将牛辅屯陕,诩在辅军。卓败,辅又死,众恐惧,校尉李傕、郭汜、张济等欲解散,间行归乡里。

诩曰:“闻长安中议欲尽诛凉州人,而诸君弃众单行,即一亭长能束君矣。不如率众而西,所在收兵,以攻长安,为董公报仇,幸而事济,奉国家以征天下,若不济,走未后也。”

众以为然。傕乃西攻长安。语在《卓传》。\footnote{臣松之以为传称“仁人之言,其利溥哉”!然则不仁之言,理必反是。夫仁功难着,而乱源易成,是故有祸机一发而殃流百世者矣。当是时,元恶既枭,天地始开,致使厉阶重结,大梗殷流,邦国遘殄悴之哀,黎民婴周馀之酷,岂不由贾诩片言乎?诩之罪也,一何大哉!自古兆乱,未有如此之甚。}

后诩为左冯翊,傕等欲以功侯之,诩曰:“此救命之计,何功之有!”固辞不受。

又以为尚书仆射,诩曰:“尚书仆射,官之师长,天下所望,诩名不素重,非所以服人也。纵诩昧于荣利,奈国朝何!”

乃更拜诩尚书,典选举,多所匡济,傕等亲而惮之。\footnote{献帝纪曰:郭汜、樊稠与傕互相违戾,欲斗者数矣。诩辄以道理责之,颇受诩言。魏书曰:诩典选举,多选旧名以为令仆,论者以此多诩。}

会母丧去官,拜光禄大夫。傕、汜等斗长安中,\footnote{献帝纪曰:傕等与诩议,迎天子置其营中。诩曰:“不可。胁天子,非义也。”傕不听。张绣谓诩曰:“此中不可久处,君胡不去?”诩曰:“吾受国恩,义不可背。卿自行,我不能也。”}傕复请诩为宣义将军。\footnote{献帝纪曰:傕时召羌、胡数千人,先以御物缯采与之,又许以宫人妇女,欲令攻郭汜。羌、胡数来闚省门,曰:“天子在中邪!李将军许我宫人美女,今皆安在?”帝患之,使诩为之方计。诩乃密呼羌、胡大帅饮食之,许以封爵重宝,于是皆引去。傕由此衰弱。}

傕等和,出天子,祐护大臣,诩有力焉。\footnote{献帝纪曰:天子既东,而李傕来追,王师败绩。司徒赵温、太常王伟、卫尉周忠、司隶荣邵皆为傕所嫌,欲杀之。诩谓傕曰:“此皆天子大臣,卿奈何害之?”傕乃止。}天子既出,诩上还印绶。是时将军段煨屯华阴,\footnote{典略称煨在华阴时,修农事,不虏略。天子东还,煨迎道贡遗周急。献帝纪曰:后以煨为大鸿胪光禄大夫,建安十四年,以寿终。}与诩同郡,遂去傕托煨。诩素知名,为煨军所望。煨内恐其见夺,而外奉诩礼甚备,诩愈不自安。
\end{yuanwen}

\begin{yuanwen}
张绣在南阳,诩阴结绣,绣遣人迎诩。诩将行,或谓诩曰:“煨待君厚矣,君安去之?”

诩曰:“煨性多疑,有忌诩意,礼虽厚,不可恃,久将为所图。我去必喜,又望吾结大援于外,必厚吾妻子。绣无谋主,亦愿得诩,则家与身必俱全矣。”

诩遂往,绣执子孙礼,煨果善视其家。诩说绣与刘表连和。\footnote{傅子曰:诩南见刘表,表以客礼待之。诩曰:“表,平世三公才也;不见事变,多疑无决,无能为也。”}太祖比征之,一朝引军退,绣自追之。

诩谓绣曰:“不可追也,追必败。”

绣不从,进兵交战,大败而还。诩谓绣曰:“促更追之,更战必胜。”

绣谢曰:“不用公言,以至于此。今已败,奈何复追?”

诩曰:“兵势有变,亟往必利。”

绣信之,遂收散卒赴追,大战,果以胜还。问诩曰:“绣以精兵追退军,而公曰必败;退以败卒击胜兵,而公曰必克。悉如公言,何其反而皆验也?”

诩曰:“此易知耳。将军虽善用兵,非曹公敌也。军虽新退,曹公必自断后;追兵虽精,将既不敌,彼士亦锐,故知必败。曹公攻将军无失策,力未尽而退,必国内有故;已破将军,必轻军速进,纵留诸将断后,诸将虽勇,亦非将军敌,故虽用败兵而战必胜也。”

绣乃服。是后,太祖拒袁绍于官渡,绍遣人招绣,并与诩书结援。绣欲许之,诩显于绣坐上谓绍使曰:“归谢袁本初,兄弟不能相容,而能容天下国士乎?”

绣惊惧曰:“何至于此!”

窃谓诩曰:“若此,当何归?”

诩曰:“不如从曹公。”

绣曰:“袁强曹弱,又与曹为仇,从之如何?”

诩曰:“此乃所以宜从也。夫曹公奉天子以令天下,其宜从一也。绍强盛,我以少众从之,必不以我为重。曹公众弱,其得我必喜,其宜从二也。夫有霸王之志者,固将释私怨,以明德于四海,其宜从三也。愿将军无疑!”

绣从之,率众归太祖。太祖见之,喜,执诩手曰:“使我信重于天下者,子也。”

表诩为执金吾,封都亭侯,迁冀州牧。冀州未平,留参司空军事。
\end{yuanwen}

\begin{yuanwen}
袁绍围太祖于官渡,太祖粮方尽,问诩计焉出,诩曰:“公明胜绍,勇胜绍,用人胜绍,决机胜绍,有此四胜而半年不定者,但顾万全故也。必决其机,须臾可定也。”

太祖曰:“善。”乃并兵出,围击绍三十馀里营,破之。绍军大溃,河北平。太祖领冀州牧,徙诩为太中大夫。

建安十三年,太祖破荆州,欲顺江东下。诩谏曰:“明公昔破袁氏,今收汉南,威名远著,军势既大;若乘旧楚之饶,以飨吏士,抚安百姓,使安土乐业,则可不劳众而江东稽服矣。”

太祖不从,军遂无利。\footnote{臣松之以为诩之此谋,未合当时之宜。于时韩、马之徒尚狼顾关右,魏武不得安坐郢都以威怀吴会,亦已明矣。彼荆州者,孙、刘之所必争也。荆人服刘主之雄姿,惮孙权之武略,为日既久,诚非曹氏诸将所能抗御。故曹仁守江陵,败不旋踵,何抚安之得行,稽服之可期?将此既新平江、汉,威慑扬、越,资刘表水战之具,藉荆楚楫棹之手,实震荡之良会,廓定之大机。不乘此取吴,将安俟哉?至于赤壁之败,盖有运数。实由疾疫大兴,以损凌厉之锋,凯风自南,用成焚如之势。天实为之,岂人事哉?然则魏武之东下,非失算也。诩之此规,为无当矣。魏武后克平张鲁,蜀中一日数十惊,刘备虽斩之而不能止,由不用刘晔之计,以失席卷之会,斤石既差,悔无所及,即亦此事之类也。世咸谓刘计为是,即愈见贾言之非也。}太祖后与韩遂、马超战于渭南,超等索割地以和,并求任子。诩以为可伪许之。又问诩计策,诩曰:“离之而已。”

太祖曰:“解。”一承用诩谋。语在《武纪》。卒破遂、超,诩本谋也。
\end{yuanwen}

\begin{yuanwen}
是时,文帝为五官将,而临菑侯植才名方盛,各有党与,有夺宗之议。文帝使人问诩自固之术,诩曰:“愿将军恢崇德度,躬素士之业,朝夕孜孜,不违子道。如此而已。”

文帝从之,深自砥砺。太祖又尝屏除左右问诩,诩嘿然不对。太祖曰:“与卿言而不答,何也?”

诩曰:“属适有所思,故不即对耳。”

太祖曰:“何思?”

诩曰:“思袁本初、刘景升父子也。”

太祖大笑,于是太子遂定。诩自以非太祖旧臣,而策谋深长,惧见猜疑,阖门自守,退无私交,男女嫁娶,不结高门,天下之论智计者归之。
\end{yuanwen}

\begin{yuanwen}
文帝即位,以诩为太尉,\footnote{魏略曰:文帝得诩之对太祖,故即位首登上司。荀勖别传曰:晋司徒阙,武帝问其人于勖。答曰:“三公具瞻所归,不可用非其人。昔魏文帝用贾诩为三公,孙权笑之。”}进爵魏寿乡侯,增邑三百,并前八百户。又分邑二百,封小子访为列侯。以长子穆为驸马都尉。

帝问诩曰:“吾欲伐不从命以一天下,吴、蜀何先?”

对曰:“攻取者先兵权,建本者尚德化。陛下应期受禅,抚临率土,若绥之以文德而俟其变,则平之不难矣。吴、蜀虽蕞尔小国,依阻山水,刘备有雄才,诸葛亮善治国,孙权识虚实,陆议见兵势,据险守要,泛舟江湖,皆难卒谋也。用兵之道,先胜后战,量敌论将,故举无遗策。臣窃料群臣,无备、权对,虽以天威临之,未见万全之势也。昔舜舞干戚而有苗服,臣以为当今宜先文后武。”

文帝不纳。后兴江陵之役,士卒多死。诩年七十七,薨,谥曰肃侯。子穆嗣,历位郡守。穆薨,子模嗣。\footnote{世语曰:模,晋惠帝时为散骑常侍、护军将军,模子胤,胤弟龛,从弟疋,皆至大官,并显于晋也。}
\end{yuanwen}

\begin{yuanwen}
	
\end{yuanwen}

评曰:荀彧清秀通雅,有王佐之风,然机鉴先识,未能充其志也。【世之论者,多讥彧协规魏氏,以倾汉祚;君臣易位,实彧之由。虽晚节立异,无救运移;功既违义,识亦疚焉。陈氏此评,盖亦同乎世识。臣松之以为斯言之作,诚未得其远大者也。彧岂不知魏武之志气,非衰汉之贞臣哉?良以于时王道既微,横流已极,雄豪虎视,人怀异心,不有拨乱之资,仗顺之略,则汉室之亡忽诸,黔首之类殄矣。夫欲翼赞时英,一匡屯运,非斯人之与而谁与哉?是故经纶急病,若救身首,用能动于崄中,至于大亨,苍生蒙舟航之接,刘宗延二纪之祚,岂非荀生之本图,仁恕之远致乎?及至霸业既隆,翦汉迹着,然后亡身殉节,以申素情,全大正于当年,布诚心于百代,可谓任重道远,志行义立。谓之未充,其殆诬欤!】荀攸、贾诩,庶乎算无遗策,经达权变,其良、平之亚欤!【臣松之以为列传之体,以事类相从。张子房青云之士,诚非陈平之伦。然汉之谋臣,良、平而已。若不共列,则馀无所附,故前史合之,盖其宜也。魏氏如诩之俦,其比幸多,诩不编程、郭之篇,而与二荀并列;失其类矣。且攸、诩之为人,其犹夜光之与蒸烛乎!其照虽均,质则异焉。今荀、贾之评,共同一称,尤失区别之宜也。】 [1]


\part{魏书十一}
\chapter{袁张凉国田王邴管传第十一}

袁涣字曜卿,陈郡扶乐人也。父滂,为汉司徒。当时诸公子多越法度,而涣清静,举动必以礼。郡命为功曹,郡中奸吏皆自引去。后辟公府,举高第,迁侍御史。除谯令,不就。刘备之为豫州,举涣茂才。后避地江、淮间,为袁术所命。术每有所咨访,涣常正议,术不能抗,然敬之不敢不礼也。顷之,吕布击术於阜陵,涣往从之,遂复为布所拘留。布初与刘备和亲,后离隙。布欲使涣作书詈辱备,涣不可,再三强之,不许。布大怒,以兵胁涣曰:“为之则生,不为则死。”涣颜色不变,笑而应之曰:“涣闻唯德可以辱人,不闻以骂。使彼固君子邪,且不耻将军之言,彼诚小人邪,将复将军之意,则辱在此不在於彼。且涣他日之事刘将军,犹今日之事将军也,如一旦去此,复骂将军,可乎?”布惭而止。

布诛,涣得归太祖。涣言曰:“夫兵者,凶器也,不得已而用之。鼓之以道德,征之以仁义,兼抚其民而除其害。夫然,故可与之死而可与之生。自大乱以来十数年矣,民之欲安,甚於倒悬,然而暴乱未息者,何也?意者政失其道欤!涣闻明君善于救世,故世乱则齐之以义,时伪则镇之以朴;世异事变,治国不同,不可不察也。夫制度损益,此古今之不必同者也。若夫兼爱天下而反之於正,虽以武平乱而济之以德,诚百王不易之道也。公明哲超世,古之所以得其民者,公既勤之矣,今之所以失其民者,公既戒之矣,海内赖公,得免於危亡之祸,然而民未知义,其惟公所以训之,则天下幸甚!”太祖深纳焉。拜为沛南部都尉。

是时新募民开屯田,民不乐,多逃亡。涣白太祖曰:“夫民安土重迁,不可卒变,易以顺行,难以逆动,宜顺其意,乐之者乃取,不欲者勿强。”太祖从之,百姓大悦。迁为梁相。涣每敕诸县:“务存鳏寡高年,表异孝子贞妇。常谈曰'世治则礼详,世乱则礼简',全在斟酌之间耳。方今虽扰攘,难以礼化,然在吾所以为之。”为政崇教训,恕思而后行,外温柔而内能断。以病去官,百姓思之。后徵为 谏议大夫、丞相军祭酒。前后得赐甚多,皆散尽之,家无所储,终不问产业,乏则取之於人,不为皦察之行,然时人服其清。

魏国初建,为郎中令,行御史大夫事。涣言於太祖曰:“今天下大难已除,文武并用,长久之道也。以为可大收篇籍,明先圣之教,以易民视听,使海内斐然向风,则远人不服可以文德来之。”太祖善其言。时有传刘备死者,群臣皆贺;涣以尝为备举吏,独不贺。居官数年卒,太祖为之流涕,赐谷二千斛,一教“以太仓谷千斛赐郎中令之家”,一教“以垣下谷千斛与曜卿家”,外不解其意。教曰:“以太仓谷者,官法也;以垣下谷者,亲旧也。”又帝闻涣昔拒吕布之事,问涣从弟敏:“涣勇怯何如?”敏对曰:“涣貌似和柔,然其临大节,处危难,虽贲育不过也。“涣子侃,亦清粹间素,有父风,历位郡守尚书。

初,涣从弟霸,公恪有功幹,魏初为大司农,及同郡何夔并知名於时。而霸子亮,夔子曾,与侃复齐声友善。亮贞固有学行,疾何晏、邓飏等,著论以讥切之,位至河南尹、尚书。霸弟徽,以儒素称。遭天下乱,避难交州。司徒辟,不至。徽弟敏,有武艺而好水功,官至河堤谒者。

张范,字公仪,河内脩武人也。祖父歆,为汉司徒。父延,为太尉。太傅袁隗欲以女妻范,范辞不受。性恬静乐道,忽於荣利,徵命无所就。弟承,字公先,亦知名,以方正徵,拜议郎,迁伊阙都尉。董卓作乱,承欲合徒众与天下共诛卓。承弟昭时为议郎,適从长安来,谓承曰:“今欲诛卓,众寡不敌,且起一朝之谋,战阡陌之民,士不素抚,兵不练习,难以成功。卓阻兵而无义,固不能久;不若择所归附,待时而动,然后可以如志。“承然之,乃解印绶间行归家,与范避地扬州。袁术备礼招请,范称疾不往,术不强屈也。遣承与相见,术问曰:“昔周室陵迟,则有桓、文之霸;秦失其政,汉接而用之。今孤以土地之广,士民之众,欲徼福齐桓,拟迹高祖,何如?”承对曰:“在德不在强。夫能用德以同天下之欲,虽由匹夫之资,而兴霸王之功,不足为难。若苟僣拟,干时而动,众之所弃,谁能兴之?“术不悦。是时,太祖将征冀州,术复问曰:“今曹公欲以弊兵数千,敌十万之众,可谓不量力矣!子以为何如?”承乃曰:“汉德虽衰,天命未改,今曹公挟天子以令天下,虽敌百万之众可也。”术作色不怿,承去之。

太祖平冀州,遣使迎范。范以疾留彭城,遣承诣太祖,太祖表以为谏议大夫。范子陵及承子戩为山东贼所得,范直诣贼请二子,贼以陵还范。范谢曰:“诸君相还儿厚矣。夫人情虽爱其子,然吾怜戩之小,请以陵易之。”贼义其言,悉以还范。太祖自荆州还,范得见於陈,以为议郎,参丞相军事,甚见敬重。太祖征伐,常令范及邴原留,与世子居守。太祖谓文帝:“举动必谘此二人。”世子执子孙礼。救恤穷乏,家无所馀,中外孤寡皆归焉。赠遗无所逆,亦终不用,及去,皆以还之。建安十七年卒。魏国初建,承以丞相参军祭酒领赵郡太守,政化大行。太祖将西征,徵承参军事,至长安,病卒。

凉茂字伯方,山阳昌邑人也。少好学,论议常据经典,以处是非。太祖辟为司空掾,举高第,补侍御史。时泰山多盗贼,以茂为泰山太守,旬月之间,襁负而至者千馀家。转为乐浪太守。公孙度在辽东,擅留茂,不遣之官,然茂终不为屈。度谓茂及诸将曰:“闻曹公远征,邺无守备,今吾欲以步卒三万,骑万匹,直指邺,谁能御之?”诸将皆曰:“然。”又顾谓茂曰:“於君意何如?”茂答曰:“比者海内大乱,社稷将倾,将军拥十万之众,安坐而观成败,夫为人臣者,固若是邪!曹公忧国家之危败,愍百姓之苦毒,率义兵为天下诛残贼,功高而德广,可谓无二矣。以海内初定,民始安集,故未责将军之罪耳!而将军乃欲称兵西向,则存亡之效,不崇朝而决。将军其勉之!”诸将闻茂言,皆震动。良久,度曰:“凉君言是也。”后徵迁为魏郡太守、甘陵相,所在有绩。文帝为五官将,茂以选为长史,迁左军师。魏国初建,迁尚书仆射,后为中尉奉常。文帝在东宫,茂复为太子太傅,甚见敬礼。卒官。

国渊字子尼,乐安盖人也。师事郑玄。后与邴原、管宁等避乱辽东。既还旧土,太祖辟为司空掾属,每於公朝论议,常直言正色,退无私焉。太祖欲广置屯田,使渊典其事。渊屡陈损益,相土处民,计民置吏,明功课之法,五年中仓廪丰实,百姓竞劝乐业。太祖征关中,以渊为居府长史,统留事。田银、苏伯反河间,银等既破,后有馀党,皆应伏法。渊以为非首恶,请不行刑。太祖从之,赖渊得生者千馀人。破贼文书,旧以一为十,及渊上首级,如其实数。太祖问其故,渊曰:“夫征讨外寇,多其斩获之数者,欲以大武功,且示民听也。河间在封域之内,银等叛逆,虽克捷有功,渊窃耻之。”太祖大悦,迁魏郡太守。

时有投书诽谤者,太祖疾之,欲必知其主。渊请留其本书,而不宣露。其书多引二京赋,渊敕功曹曰:“此郡既大,今在都辇,而少学问者。其简开解年少,欲遣就师。”功曹差三人,临遣引见,训以“所学未及,二京赋,博物之书也,世人忽略,少有其师,可求能读者从受之。“又密喻旨。旬日得能读者,遂往受业。吏因请使作笺,比方其书,与投书人同手。收摄案问,具得情理。迁太仆。居列卿位,布衣蔬食,禄赐散之旧故宗族,以恭俭自守,卒官。

田畴字子泰,右北平无终人也。好读书,善击剑。初平元年,义兵起,董卓迁帝于长安。幽州牧刘虞叹曰:“贼臣作乱,朝廷播荡,四海俄然,莫有固志。身备宗室遗老,不得自同於众。今欲奉使展效臣节,安得不辱命之士乎?”众议咸曰:“田畴虽年少,多称其奇。“畴时年二十二矣。虞乃备礼请与相见,大悦之,遂署为从事,具其车骑。将行,畴曰:“今道路阻绝,寇虏纵横,称官奉使,为众所指名。愿以私行,期於得达而已。”虞从之。畴乃归,自选其家客与年少之勇壮慕从者二十骑俱往。虞自出祖而遣之。既取道,畴乃更上西关,出塞,傍北山,直趣朔方,循间径去,遂至长安致命。诏拜骑都尉。畴以为天子方蒙尘未安,不可以荷佩荣宠,固辞不受。朝廷高其义。三府并辟,皆不就。得报,驰还,未至,虞已为公孙瓒所害。畴至,谒祭虞墓,陈发章表,哭泣而去。瓒闻之大怒,购求获畴,谓曰:“汝何自哭刘虞墓,而不送章报於我也?“畴答曰:“汉室衰穨,人怀异心,唯刘公不失忠节。章报所言,於将军未美,恐非所乐闻,故不进也。且将军方举大事以求所欲,既灭无罪之君,又雠守义之臣,诚行此事,则燕、赵之士将皆蹈东海而死耳,岂忍有从将军者乎!“瓒壮其对,释不诛也。拘之军下,禁其故人莫得与通。或说瓒曰:“田畴义士,君弗能礼,而又囚之,恐失众心。”瓒乃纵遣畴。

畴得北归,率举宗族他附从数百人,扫地而盟曰:“君仇不报,吾不可以立於世!”遂入徐无山中,营深险平敞地而居,躬耕以养父母。百姓归之,数年间至五千馀家。畴谓其父老曰:“诸君不以畴不肖,远来相就。众成都邑,而莫相统一,恐非久安之道,愿推择其贤长者以为之主。“皆曰:“善。”同佥推畴。畴曰:“今来在此,非苟安而已,将图大事,复怨雪耻。窃恐未得其志,而轻薄之徒自相侵侮,偷快一时,无深计远虑。畴有愚计,愿与诸君共施之,可乎?”皆曰:“可。“畴乃为约束相杀伤、犯盗、诤讼之法,法重者至死,其次抵罪,二十馀条。又制为婚姻嫁娶之礼,兴举学校讲授之业,班行其众,众皆便之,至道不拾遗。北边翕然服其威信,乌丸、鲜卑并各遣译使致贡遗,畴悉抚纳,令不为寇。袁绍数遣使招命,又即授将军印,因安辑所统,畴皆拒不【当】受。绍死,其子尚又辟焉,畴终不行。

畴常忿乌丸昔多贼杀其郡冠盖,有欲讨之意而力未能。建安十二年,太祖北征乌丸,未至,先遣使辟畴,又命田豫喻指。畴戒其门下趣治严。门人谓曰:“昔袁公慕君,礼命五至,君义不屈;今曹公使一来而君若恐弗及者,何也?“畴笑而应之曰:“此非君所识也。“遂随使者到军,署司空户曹掾,引见谘议。明日出令曰:“田子泰非吾所宜吏者。“即举茂才,拜为蓚令,不之官,随军次无终。时方夏水雨,而滨海洿下,泞滞不通,虏亦遮守蹊要,军不得进。太祖患之,以问畴。畴曰:“此道,秋夏每常有水,浅不通车马,深不载舟船,为难久矣。旧北平郡治在平冈,道出卢龙,达于柳城;自建武以来,陷坏断绝,垂二百载,而尚有微径可从。今虏将以大军当由无终,不得进而退,懈弛无备。若嘿回军,从卢龙口越白檀之险,出空虚之地,路近而便,掩其不备,蹋顿之首可不战而禽也。“太祖曰:“善。“乃引军还,而署大木表于水侧路傍曰:“方今暑夏,道路不通,且俟秋冬,乃复进军。“虏候骑见之,诚以为大军去也。太祖令畴将其众为乡导,上徐无山,出卢龙,历平冈,登白狼堆,去柳城二百馀里,虏乃惊觉。单于身自临陈,太祖与交战,遂大斩获,追奔逐北,至柳城。军还入塞,论功行封,封畴亭侯,邑五百户。畴自以始为居难,率众遁逃,志义不立,反以为利,非本意也,固让。太祖知其至心,许而不夺。

辽东斩送袁尚首,令“三军敢有哭之者斩“。畴以尝为尚所辟,乃往吊祭。太祖亦不问。畴尽将其家属及宗人三百馀家居邺。太祖赐畴车马谷帛,皆散之宗族知旧。从征荆州还,太祖追念畴功殊美,恨前听畴之让,曰:“是成一人之志,而亏王法大制也。“於是乃复以前爵封畴。畴上疏陈诚,以死自誓。太祖不听,欲引拜之,至于数四,终不受。有司劾畴狷介违道,苟立小节,宜免官加刑。太祖重其事,依违者久之。乃下世子及大臣博议,世子以畴同於子文辞禄,申胥逃赏,宜勿夺以优其节。尚书令荀彧、司隶校尉锺繇亦以为可听。太祖犹欲侯之。畴素与夏侯惇善,太祖语惇曰:“且往以情喻之,自从君所言,无告吾意也。“惇就畴宿,如太祖所戒。畴揣知其指,不复发言。惇临去,乃拊畴背曰:“田君,主意殷勤,曾不能顾乎!“畴答曰:“是何言之过也!畴,负义逃窜之人耳,蒙恩全活,为幸多矣。岂可卖卢龙之塞,以易赏禄哉?纵国私畴,畴独不愧於心乎?将军雅知畴者,犹复如此,若必不得已,请愿效死刎首於前。“言未卒,涕泣横流。惇具答太祖。太祖喟然知不可屈,乃拜为议郎。年四十六卒。子又早死。文帝践阼,高畴德义,赐畴从孙续爵关内侯,以奉其嗣。

王脩字叔治,北海营陵人也。年七岁丧母。母以社日亡,来岁邻里社,脩感念母,哀甚。邻里闻之,为之罢社。年二十,游学南阳,止张奉舍。奉举家得疾病,无相视者,脩亲隐恤之,病愈乃去。初平中,北海孔融召以为主簿,守高密令。高密孙氏素豪侠,人客数犯法。民有相劫者,贼入孙氏,吏不能执。脩将吏民围之,孙氏拒守,吏民畏惮不敢近。脩令吏民:“敢有不攻者与同罪。“孙氏惧,乃出贼。由是豪强慑服。举孝廉,脩让邴原,融不听。时天下乱,遂不行。顷之,郡中有反者。脩闻融有难,夜往奔融。贼初发,融谓左右曰:“能冒难来,唯王脩耳!“言终而脩至。复署功曹。时胶东多贼寇,复令脩守胶东令。胶东人公沙卢宗强,自为营堑,不肯应发调。脩独将数骑径入其门,斩卢兄弟,公沙氏惊愕莫敢动。脩抚慰其馀,由是寇少止。融每有难,脩虽休归在家,无不至。融常赖脩以免。

袁谭在青州,辟脩为治中从事,别驾刘献数毁短脩。后献以事当死,脩理之,得免。时人益以此多焉。袁绍又辟脩除即墨令,后复为谭别驾。绍死,谭、尚有隙。尚攻谭,谭军败,脩率吏民往救谭。谭喜曰:“成吾军者,王别驾也。“谭之败,刘询起兵漯阴,诸城皆应。谭叹息曰:“今举州背叛,岂孤之不德邪!“脩曰:“东莱太守管统虽在海表,此人不反。必来。“后十馀日,统果弃其妻子来赴谭,妻子为贼所杀,谭更以统为乐安太守。谭复欲攻尚,脩谏曰:“兄弟还相攻击,是败亡之道也。“谭不悦,然知其志节。后又问脩:“计安出?“脩曰:“夫兄弟者,左右手也。譬人将斗而断其右手,而曰'我必胜',若是者可乎?夫弃兄弟而不亲,天下其谁亲之!属有谗人,固将交斗其间,以求一朝之利,愿明使君塞耳勿听也。若斩佞臣数人,复相亲睦,以御四方,可以横行天下。“谭不听,遂与尚相攻击,请救於太祖。太祖既破冀州,谭又叛。太祖遂引军攻谭于南皮。脩时运粮在乐安,闻谭急,将所领兵及诸从事数十人往赴谭。至高密,闻谭死,下马号哭曰:“无君焉归?“遂诣太祖,乞收葬谭尸。太祖欲观脩意,默然不应。脩复曰:“受袁氏厚恩,若得收敛谭尸,然后就戮,无所恨。“太祖嘉其义,听之。以脩为督军粮,还乐安。谭之破,诸城皆服,唯管统以乐安不从命。太祖命脩取统首,脩以统亡国之忠臣,因解其缚,使诣太祖。太祖悦而赦之。袁氏政宽,在职势者多畜聚。太祖破邺,籍没审配等家财物赀以万数。及破南皮,阅脩家,谷不满十斛,有书数百卷。太祖叹曰:“士不妄有名。“乃礼辟为司空掾,行司金中郎将,迁魏郡太守。为治,抑强扶弱,明赏罚,百姓称之。魏国既建,为大司农郎中令。太祖议行肉刑,脩以为时未可行,太祖采其议。徙为奉尚。其后严才反,与其徒属数十人攻掖门。脩闻变,召车马未至,便将官属步至宫门。太祖在铜爵台望见之,曰:“彼来者必王叔治也。“相国锺繇谓脩:“旧,京城有变,九卿各居其府。“脩曰:“食其禄,焉避其难?居府虽旧,非赴难之义。“顷之,病卒官。子忠,官至东莱太守、散骑常侍。初,脩识高柔于弱冠,异王基于幼童,终皆远至,世称其知人。

邴原字根矩,北海朱虚人也。少与管宁俱以操尚称,州府辟命皆不就。黄巾起,原将家属入海,住郁洲山中。时孔融为北海相,举原有道。原以黄巾方盛,遂至辽东,与同郡刘政俱有勇略雄气。辽东太守公孙度畏恶欲杀之,尽收捕其家,政得脱。度告诸县:“敢有藏政者与同罪。“政窘急,往投原,原匿之月馀,时东莱太史慈当归,原因以政付之。既而谓度曰:“将军前日欲杀刘政,以其为己害。今政已去,君之害岂不除哉!“度曰:“然。“原曰:“君之畏政者,以其有智也。今政已免,智将用矣,尚奚拘政之家?不若赦之,无重怨。“度乃出之。原又资送政家,皆得归故郡。原在辽东,一年中往归原居者数百家,游学之士,教授之声,不绝。

后得归,太祖辟为司空掾。原女早亡,时太祖爱子仓舒亦没,太祖欲求合葬,原辞曰:“合葬,非礼也。原之所以自容於明公,公之所以待原者,以能守训典而不易也。若听明公之命,则是凡庸也,明公焉以为哉?“太祖乃止,徙署丞相徵事。崔琰为东曹掾,记让曰:“徵事邴原、议郎张范,皆秉德纯懿,志行忠方,清静足以厉俗,贞固足以幹事,所谓龙翰凤翼,国之重宝。举而用之,不仁者远。“代凉茂为五官将长史,闭门自守,非公事不出。太祖征吴,原从行,卒。

是后大鸿胪钜鹿张泰、河南尹扶风庞迪以清贤称,永宁太仆东郡张阁以简质闻。

管宁字幼安,北海朱虚人也。年十六丧父,中表愍其孤贫,咸共赠赗,悉辞不受,称财以送终。长八尺,美须眉。与平原华歆、同县邴原相友,俱游学於异国,并敬善陈仲弓。天下大乱,闻公孙度令行於海外,遂与原及平原王烈等至于辽东。度虚馆以候之。既往见度,乃庐於山谷。时避难者多居郡南,而宁居北,示无迁志,后渐来从之。太祖为司空,辟宁,度子康绝命不宣。

王烈者,字彦方,於时名闻在原、宁之右。辞公孙度长史,商贾自秽。太祖命为丞相掾,徵事,未至,卒於海表。

中国少安,客人皆还,唯宁晏然若将终焉。黄初四年,诏公卿举独行君子,司徒华歆荐宁。文帝即位,徵宁,遂将家属浮海还郡,公孙恭送之南郊,加赠服物。自宁之东也,度、康、恭前后所资遗,皆受而藏诸。既已西渡,尽封还之。诏以宁为太中大夫,固辞不受。明帝即位,太尉华歆逊位让宁,遂下诏曰:“太中大夫管宁,耽怀道德,服膺六艺,清虚足以侔古,廉白可以当世。曩遭王道衰缺,浮海遁居,大魏受命,则襁负而至,斯盖应龙潜升之道,圣贤用舍之义。而黄初以来,徵命屡下,每辄辞疾,拒违不至。岂朝廷之政,与生殊趣,将安乐山林,往而不能反乎!夫以姬公之圣,而耇德不降,则鸣鸟弗闻。以秦穆之贤,犹思询乎黄发。况朕寡德,曷能不愿闻道于子大夫哉!今以宁为光禄勋。礼有大伦,君臣之道,不可废也。望必速至,称朕意焉。“又诏青州刺史曰:“宁抱道怀贞,潜翳海隅,比下徵书,违命不至,盘桓利居,高尚其事。虽有素履幽人之贞,而失考父兹恭之义,使朕虚心引领历年,其何谓邪?徒欲怀安,必肆其志,不惟古人亦有翻然改节以隆斯民乎!日逝月除,时方已过,澡身浴德,将以曷为?仲尼有言:'吾非斯人之徒与而谁与哉!'其命别驾从事郡丞掾,奉诏以礼发遣宁诣行在所,给安车、吏从、茵蓐、道上厨食,上道先奏。“宁称草莽臣上疏曰:“臣海滨孤微,罢农无伍,禄运幸厚。横蒙陛下纂承洪绪,德侔三皇,化溢有唐。久荷渥泽,积祀一纪,不能仰答陛下恩养之福。沈委笃痾,寝疾弥留,逋违臣隶颠倒之节,夙宵战怖,无地自厝。臣元年十一月被公车司马令所下州郡,八月甲申诏书徵臣,更赐安车、衣被、茵蓐,以礼发遣,光宠并臻,优命屡至,怔营竦息,悼心失图。思自陈闻,申展愚情,而明诏抑割,不令稍脩章表,是以郁滞,讫于今日。诚谓乾覆,恩有纪极,不意灵润,弥以隆赫。奉今年二月被州郡所下三年十二月辛酉诏书,重赐安车、衣服,别驾从事与郡功曹以礼发遣,又特被玺书,以臣为光禄勋,躬秉劳谦,引喻周、秦,损上益下。受诏之日,精魄飞散,靡所投死。臣重自省揆,德非园、绮而蒙安车之荣,功无窦融而蒙玺封之宠,楶棁驽下,荷栋梁之任,垂没之命,获九棘之位,惧有朱博鼓妖之眚。又年疾日侵,有加无损,不任扶舆进路以塞元责。望慕阊阖,徘徊阙庭,谨拜章陈情,乞蒙哀省,抑恩听放,无令骸骨填于衢路。“自黄初至于青龙,徵命相仍,常以八月赐牛酒。诏书问青州刺史程喜:“宁为守节高乎,审老疾尫顿邪?“喜上言:“宁有族人管贡为州吏,与宁邻比,臣常使经营消息。贡说:'宁常著皂帽、布襦袴、布裙,随时单复,出入闺庭,能自任杖,不须扶持。四时祠祭,辄自力强,改加衣服,著絮巾,故在辽东所有白布单衣,亲荐馔馈,跪拜成礼。宁少而丧母,不识形象,常特加觞,泫然流涕。又居宅离水七八十步,夏时诣水中澡洒手足,闚於园圃。'臣揆宁前后辞让之意,独自以生长潜逸,耆艾智衰,是以栖迟,每执谦退。此宁志行所欲必全,不为守高。“

正始二年,太仆陶丘一、永宁卫尉孟观、侍中孙邕、中书侍郎王基荐宁曰:

臣闻龙凤隐耀,应德而臻,明哲潜遁,俟时而动。是以鸾鷟鸣岐,周道隆兴,四皓为佐,汉帝用康。伏见太中大夫管宁,应二仪之中和,总九德之纯懿,含章素质,冰絜渊清,玄虚澹泊,与道逍遥;娱心黄老,游志六艺,升堂入室,究其阃奥,韬古今於胸怀,包道德之机要。中平之际,黄巾陆梁,华夏倾荡,王纲弛顿。遂避时难,乘桴越海,羁旅辽东三十馀年。在乾之姤,匿景藏光,嘉遁养浩,韬韫儒墨,潜化傍流,畅于殊俗。

黄初四年,高祖文皇帝畴谘群公,思求隽乂,故司徒华歆举宁应选,公车特徵,振翼遐裔,翻然来翔。行遇屯厄,遭罹疾病,即拜太中大夫。烈祖明皇帝嘉美其德,登为光禄勋。宁疾弥留,未能进道。今宁旧疾已瘳,行年八十,志无衰倦。环堵筚门,偃息穷巷,饭鬻糊口,并日而食,吟咏诗书,不改其乐。困而能通,遭难必济,经危蹈险,不易其节,金声玉色,久而弥彰。揆其终始,殆天所祚,当赞大魏,辅亮雍熙。兖职有阙,群下属望。昔高宗刻象,营求贤哲,周文启龟,以卜良佐。况宁前朝所表,名德已著,而久栖迟,未时引致,非所以奉遵明训,继成前志也。陛下践阼,纂承洪绪。圣敬日跻,超越周成。每发德音,动谘师傅。若继二祖招贤故典,宾礼俊迈,以广缉熙,济济之化,侔于前代。

宁清高恬泊,拟迹前轨,德行卓绝,海内无偶。历观前世玉帛所命,申公、枚乘、周党、樊英之俦,测其渊源,览其清浊,未有厉俗独行若宁者也。诚宜束帛加璧,备礼徵聘,仍授几杖,延登东序,敷陈坟素,坐而论道,上正璇玑,协和皇极,下阜群生,彝伦攸叙,必有可观,光益大化。若宁固执匪石,守志箕山,追迹洪崖,参踪巢、许。斯亦圣朝同符唐、虞,优贤扬历,垂声千载。虽出处殊涂,俯仰异体,至於兴治美俗,其揆一也。

於是特具安车蒲轮,束帛加璧聘焉。会宁卒,时年八十四。拜子邈郎中,后为博士。初,宁妻先卒,知故劝更娶,宁曰:“每省曾子、王骏之言,意常嘉之,岂自遭之而违本心哉?“

时钜鹿张臶,字子明,颍川胡昭,字孔明,亦养志不仕。臶少游太学,学兼内外,后归乡里。袁绍前后辟命,不应,移居上党。并州牧高幹表除乐平令,不就,徙循常山,门徒且数百人,迁居任县。太祖为丞相,辟,不诣。太和中,诏求隐学之士能消灾复异者,郡累上臶,发遣,老病不行。广平太守卢毓到官三日,纲纪白承前致版谒臶。毓教曰:“张先生所谓上不事天子,下不友诸侯者也。此岂版谒所可光饰哉!“但遣主簿奉书致羊酒之礼。青龙四年辛亥诏书:“张掖郡玄川溢涌,激波奋荡,宝石负图,状像灵龟,宅于川西,嶷然磐峙,仓质素章,麟凤龙马,焕炳成形,文字告命,粲然著明。太史令高堂隆上言:古皇圣帝所未尝蒙,实有魏之祯命,东序之世宝。“事颁天下。”任令于绰连赍以问臶,臶密谓绰曰:“夫神以知来,不追已往,祯祥先见而后废兴从之。汉已久亡,魏已得之,何所追兴徵祥乎!此石,当今之变异而将来之祯瑞也。“正始元年,戴鵀之鸟,巢臶门阴。臶告门人曰:“夫戴鵀阳鸟,而巢门阴,此凶祥也。“乃援琴歌咏,作诗二篇,旬日而卒,时年一百五岁。是岁,广平太守王肃至官,教下县曰:“前在京都,闻张子明,来至问之,会其已亡,致痛惜之。此君笃学隐居,不与时竞,以道乐身。昔绛县老人屈在泥涂,赵孟升之,诸侯用睦。愍其耄勤好道,而不蒙荣宠,书到,遣吏劳问其家,显题门户,务加殊异,以慰既往,以劝将来。“

胡昭始避地冀州,亦辞袁绍之命,遁还乡里。太祖为司空丞相,频加礼辟。昭往应命,既至,自陈一介野生,无军国之用,归诚求去。太祖曰:“人各有志,出处异趣,勉卒雅尚,义不相屈。“昭乃转居陆浑山中,躬耕乐道,以经籍自娱。闾里敬而爱之。建安二十三年,陆浑长张固被书调丁夫,当给汉中。百姓恶惮远役,并怀扰扰。民孙狼等因兴兵杀县主簿,作为叛乱,县邑残破。固率将十馀吏卒,依昭住止,招集遗民,安复社稷。狼等遂南附关羽。羽授印给兵,还为寇贼。到陆浑南长乐亭,自相约誓,言:“胡居士贤者也,一不得犯其部落。“一川赖昭,咸无怵惕。天下安辑,徙宅宜阳。正始中,骠骑将军赵俨、尚书黄休、郭彝、散骑常侍荀顗、锺毓、太仆庾嶷、弘农太守何桢等递荐昭曰:“天真高絜,老而弥笃。玄虚静素,有夷、皓之节。宜蒙徵命,以励风俗。“至嘉平二年,公车特徵,会卒,年八十九。拜子纂郎中。初,昭善史书,与锺繇、邯郸淳、卫觊、韦诞并有名,尺牍之迹,动见模楷焉。

评曰:袁涣、邴原、张范躬履清蹈,进退以道,盖是贡禹、两龚之匹。凉茂、国渊亦其次也。张承名行亚范,可谓能弟矣。田畴抗节,王脩忠贞,足以矫俗;管宁渊雅高尚,确然不拔;张臶、胡昭阖门守静,不营当世:故并录焉。

\part{魏书十二}
\chapter{崔毛徐何邢鲍司马传第十二}

崔琰字季珪,清河东武城人也。少朴讷,好击剑,尚武事。年二十三,乡移为正,始感激,读论语、韩诗。至年二十九,乃结公孙方等就郑玄受学。学未期,徐州黄巾贼攻破北海,玄与门人到不其山避难。时谷籴县乏,玄罢谢诸生。琰既受遣,而寇盗充斥,西道不通。于是周旋青、徐、兖、豫之郊,东下寿春,南望江、湖。自去家四年乃归,以琴书自娱。

大将军袁绍闻而辟之。时士卒横暴,掘发丘陇,琰谏曰:“昔孙卿有言:‘士不素教,甲兵不利,虽汤武不能以战胜。’今道路暴骨,民未见德,宜敕郡县掩骼埋胔,示憯怛之爱,追文王之仁。”绍以为骑都尉。后绍治兵黎阳,次于延津,琰复谏曰:“天子在许,民望助顺,不如守境述职,以宁区宇。”绍不听,遂败于官渡。及绍卒,二子交争,争欲得琰。琰称疾固辞,由是获罪,幽于囹圄,赖阴夔、陈琳营救得免。

太祖破袁氏,领冀州牧,辟琰为别驾从事,谓琰曰:“昨案户籍,可得三十万众,故为大州也。”琰对曰:“今天下分崩,九州幅裂,二袁兄弟亲寻干戈,冀方蒸庶暴骨原野。未闻王师仁声先路,存问风俗,救其涂炭,而校计甲兵,唯此为先,斯岂鄙州士女所望於明公哉!”太祖改容谢之。于时宾客皆伏失色。

太祖征并州,留琰傅文帝於邺。世子仍出田猎,变易服乘,志在驱逐。琰书谏曰:“盖闻盘于游田,书之所戒,鲁隐观鱼,春秋讥之。此周、孔之格言,二经之明义。殷鉴夏后,诗称不远;子卯不乐,礼以为忌,此又近者之得失,不可不深察也。袁族富强,公子宽放,盘游滋侈,义声不闻,哲人君子,俄有色斯之志,熊罴壮士,堕於吞噬之用。固所以拥徒百万,跨有河朔,无所容足也。今邦国殄瘁,惠康未洽,士女企踵,所思者德。况公亲御戎马,上下劳惨,世子宜遵大路,慎以行正,思经国之高略,内鉴近戒,外扬远节,深惟储副,以身为宝。而猥袭虞旅之贱服,忽驰骛而陵险,志雉兔之小娱,忘社稷之为重,斯诚有识所以恻心也。唯世子燔翳捐褶,以塞众望,不令老臣获罪於天。”世子报曰:“昨奉嘉命,惠示雅数,欲使燔翳捐褶,翳已坏矣,褶亦去焉。后有此比,蒙复诲诸。”

太祖为丞相,琰复为东西曹掾属徵事。初授东曹时,教曰:“君有伯夷之风,史鱼之直,贪夫慕名而清,壮士尚称而厉,斯可以率时者已。故授东曹,往践厥职。”魏国初建,拜尚书。时未立太子,临菑侯植有才而爱。太祖狐疑,以函令密访於外。唯琰露板答曰:“盖闻春秋之义,立子以长,加五官将仁孝聪明,宜承正统。琰以死守之。”植,琰之兄女婿也。太祖贵其公亮,喟然叹息,迁中尉。

琰声姿高畅,眉目疏朗,须长四尺,甚有威重,朝士瞻望,而太祖亦敬惮焉。琰尝荐钜鹿杨训,虽才好不足,而清贞守道,太祖即礼辟之。后太祖为魏王,训发表称赞功伐,襃述盛德。时人或笑训希世浮伪,谓琰为失所举。琰从训取表草视之,与训书曰:“省表,事佳耳!时乎时乎,会当有变时。”琰本意讥论者好谴呵而不寻情理也。有白琰此书傲世怨谤者,太祖怒曰:“谚言‘生女耳’,‘耳’非佳语。‘会当有变时’,意指不逊。”於是罚琰为徒隶,使人视之,辞色不挠。太祖令曰:“琰虽见刑,而通宾客,门若市人,对宾客虬须直视,若有所瞋。”遂赐琰死。

始琰与司马朗善,晋宣王方壮,琰谓朗曰:“子之弟,聪哲明允,刚断英跱,殆非子之所及也。”朗以为不然,而琰每秉此论。琰从弟林,少无名望,虽姻族犹多轻之,而琰常曰:“此所谓大器晚成者也,终必远至。”涿郡孙礼、卢毓始入军府,琰又名之曰:“孙疏亮亢烈,刚简能断,卢清警明理,百炼不消,皆公才也。”后林、礼、毓咸至鼎辅。及琰友人公孙方、宋阶早卒,琰抚其遗孤,恩若己子。其鉴识笃义,类皆如此。

初,太祖性忌,有所不堪者,鲁国孔融、南阳许攸、娄圭,皆以恃旧不虔见诛。而琰最为世所痛惜,至今冤之。

毛玠字孝先,陈留平丘人也。少为县吏,以清公称。将避乱荆州,未至,闻刘表政令不明,遂往鲁阳。太祖临兖州,辟为治中从事。玠语太祖曰:“今天下分崩,国主迁移,生民废业,饥馑流亡,公家无经岁之储,百姓无安固之志,难以持久。今袁绍、刘表,虽士民众强,皆无经远之虑,未有树基建本者也。夫兵义者胜,守位以财,宜奉天子以令不臣,脩耕植,畜军资,如此则霸王之业可成也。”太祖敬纳其言,转幕府功曹。

太祖为司空丞相,玠尝为东曹掾,与崔琰并典选举。其所举用,皆清正之士,虽於时有盛名而行不由本者,终莫得进。务以俭率人,由是天下之士莫不以廉节自励,虽贵宠之臣,舆服不敢过度。太祖叹曰:“用人如此,使天下人自治,吾复何为哉!”文帝为五官将,亲自诣玠,属所亲眷。玠答曰:“老臣以能守职,幸得免戾,今所说人非迁次,是以不敢奉命。”大军还邺,议所并省。玠请谒不行,时人惮之,咸欲省东曹。乃共白曰:“旧西曹为上,东曹为次,宜省东曹。”太祖知其情,令曰:“日出於东,月盛於东,凡人言方,亦复先东,何以省东曹?”遂省西曹。初,太祖平柳城,班所获器物,特以素屏风素冯几赐玠,曰:“君有古人之风,故赐君古人之服。”玠居显位,常布衣蔬食,抚育孤兄子甚笃,赏赐以振施贫族,家无所馀。迁右军师。魏国初建,为尚书仆射,复典选举。时太子未定,而临菑侯植有宠,玠密谏曰:“近者袁绍以嫡庶不分,覆宗灭国。废立大事,非所宜闻。”后群僚会,玠起更衣,太祖目指曰:“此古所谓国之司直,我之周昌也。”

崔琰既死,玠内不悦。后有白玠者:“出见黥面反者,其妻子没为官奴婢,玠言曰‘使天不雨者盖此也’。“太祖大怒,收玠付狱。大理锺繇诘玠曰:“自古圣帝明王,罪及妻子。书云:‘左不共左,右不共右,予则孥戮女。’司寇之职,男子入于罪隶,女子入于舂槁。汉律,罪人妻子没为奴婢,黥面。汉法所行黥墨之刑,存於古典。今真奴婢祖先有罪,虽历百世,犹有黥面供官,一以宽良民之命,二以宥并罪之辜。此何以负於神明之意,而当致旱?案典谋,急恒寒若,舒恒燠若,宽则亢阳,所以为旱。玠之吐言,以为宽邪,以为急也?急当阴霖,何以反旱?成汤圣世,野无生草,周宣令主,旱魃为虐。亢旱以来,积三十年,归咎黥面,为相值不?卫人伐邢,师兴而雨,罪恶无徵,何以应天?玠讥谤之言,流於下民,不悦之声,上闻圣听。玠之吐言,势不独语,时见黥面,凡为几人?黥面奴婢,所识知邪?何缘得见,对之叹言?时以语谁?见答云何?以何日月?於何处所?事已发露,不得隐欺,具以状对。”玠曰:“臣闻萧生缢死,困於石显;贾子放外,谗在绛、灌;白起赐剑於杜邮;晁错致诛於东市;伍员绝命於吴都。斯数子者,或妒其前,或害其后。臣垂龆执简,累勤取官,职在机近,人事所窜。属臣以私,无势不绝,语臣以冤,无细不理。人情淫利,为法所禁,法禁于利,势能害之。青蝇横生,为臣作谤,谤臣之人,势不在他。昔王叔、陈生争正王廷,宣子平理。命举其契,是非有宜,曲直有所,春秋嘉焉,是以书之。臣不言此,无有时、人。说臣此言,必有徵要。乞蒙宣子之辨,而求王叔之对。若臣以曲闻,即刑之日,方之安驷之赠;赐剑之来,比之重赏之惠。谨以状对。”时桓阶、和洽进言救玠。玠遂免黜,卒于家。太祖赐棺器钱帛,拜子机郎中。

徐奕字季才,东莞人也。避难江东,孙策礼命之。奕改姓名,微服还本郡。太祖为司空,辟为掾属,从西征马超。超破,军还。时关中新服,未甚安,留奕为丞相长史,镇抚西京,西京称其威信。转为雍州刺史,复还为东曹属。丁仪等见宠於时,并害之,而奕终不为动。出为魏郡太守。太祖征孙权,徙为留府长史,谓奕曰:“君之忠亮,古人不过也,然微太严。昔西门豹佩韦以自缓,夫能以柔弱制刚强者,望之於君也。今使君统留事,孤无复还顾之忧也。”魏国既建,为尚书,复典选举,迁尚书令。

太祖征汉中,魏讽等谋反,中尉杨俊左迁。太祖叹曰:“讽所以敢生乱心,以吾爪牙之臣无遏奸防谋者故也。安得如诸葛丰者,使代俊乎!”桓阶曰:“徐奕其人也。”太祖乃以奕为中尉,手令曰:“昔楚有子玉,文公为之侧席而坐;汲黯在朝,淮南为之折谋。诗称'邦之司直',君之谓与!”在职数月,疾笃乞退,拜谏议大夫,卒。

何夔字叔龙,陈郡阳夏人也。曾祖父熙,汉安帝时官至车骑将军。夔幼丧父,与母兄居,以孝友称。长八尺三寸,容貌矜严。避乱淮南。后袁术至寿春,辟之,夔不应,然遂为术所留。久之,术与桥蕤俱攻围蕲阳,蕲阳为太祖固守。术以夔彼郡人,欲胁令说蕲阳。夔谓术谋臣李业曰:“昔柳下惠闻伐国之谋而有忧色,曰‘吾闻伐国不问仁人,斯言何为至于我哉’!”遂遁匿灊山。术知夔终不为己用,乃止。术从兄山阳太守遗母,夔从姑也,是以虽恨夔而不加害。

建安二年,夔将还乡里,度术必急追,乃间行得免,明年到本郡。顷之,太祖辟为司空掾属。时有传袁术军乱者,太祖问夔曰;“君以为信不?”夔对曰:“天之所助者顺,人之所助者信。术无信顺之实,而望天人之助,此不可以得志於天下。夫失道之主,亲戚叛之,而况於左右乎!以夔观之,其乱必矣。”太祖曰;“为国失贤则亡。君不为术所用;乱,不亦宜乎!”太祖性严,掾属公事,往往加杖;夔常畜毒药,誓死无辱,是以终不见及。出为城父令。迁长广太守。郡滨山海,黄巾未平,豪杰多背叛,袁谭就加以官位。长广县人管承,徒众三千馀家,为寇害。议者欲举兵攻之。夔曰:“承等非生而乐乱也,习於乱,不能自还,未被德教,故不知反善。今兵迫之急,彼恐夷灭,必并力战。攻之既未易拔,虽胜,必伤吏民,不如徐喻以恩德,使容自悔,可不烦兵而定。”乃遣郡丞黄珍在,为陈成败,承等皆请服。夔遣吏成弘领校尉,长广县丞等郊迎奉牛酒,诣郡。矣平贼从钱,众亦数千,夔率郡兵与张辽共讨定之。东牟人王营,众三千馀家,胁昌阳县为乱。夔遣吏王钦等,授以计略,使离散之。旬月皆平定。

是时太祖始制新科下州郡,又收租税绵绢。夔以郡初立,近以师旅之后,不可卒绳以法,乃上言曰:“自丧乱已来,民人失所,今虽小安,然服教日浅。所下新科,皆以明罚敕法,齐一大化也。所领六县,疆域初定,加以饥馑,若一切齐以科禁,恐或有不从教者。有不从教者不得不诛,则非观民设教随时之意也。先王辨九服之赋以殊远近,制三典之刑以平治乱,愚以为此郡宜依远域新邦之典,其民间小事,使长吏临时随宜,上不背正法,下以顺百姓之心。比及三年,民安其业,然后齐之以法,则无所不至矣。”太祖从其言。徵还,参丞相军事。海贼郭祖寇暴乐安、济南界,州郡苦之。太祖以夔前在长广有威信,拜乐安太守。到官数月,诸城悉平。

入为丞相东曹掾。夔言於太祖曰:“自军兴以来,制度草创,用人未详其本,是以各引其类,时忘道德。夔闻以贤制爵,则民慎德;以庸制禄,则民兴功。以为自今所用,必先核之乡闾,使长幼顺叙,无相逾越。显忠直之赏,明公实之报,则贤不肖之分,居然别矣。又可脩保举故不以实之令,使有司别受其负。在朝之臣,时受教与曹并选者,各任其责。上以观朝臣之节,下以塞争竞之源,以督群下,以率万民,如是则天下幸甚。”太祖称善。魏国既建,拜尚书仆射。文帝为太子,以凉茂为太傅,夔为少傅;特命二傅与尚书东曹并选太子诸侯官属。茂卒,以夔代茂。每月朔,太傅入见太子,太子正法服而礼焉;他日无会仪。夔迁太仆,太子欲与辞,宿戒供,夔无往意;乃与书请之,夔以国有常制,遂不往。其履正如此。然於节俭之世,最为豪汰。文帝践阼,封成阳亭侯,邑三百户。疾病,屡乞逊位。诏报曰:“盖礼贤亲旧,帝王之常务也。以亲则君有辅弼之勋焉,以贤则君有醇固之茂焉。夫有阴德者必有阳报,今君疾虽未瘳,神明听之矣。君其即安,以顺朕意。”薨,谥曰靖侯。子曾嗣,咸熙中为司徒。

邢颙,字子昂,河间鄚人也。举孝廉,司徒辟,皆不就。易姓字,適右北平,从田畴游。积五年,而太祖定冀州。颙谓畴曰:“黄巾起来二十馀年,海内鼎沸,百姓流离。今闻曹公法令严。民厌乱矣,乱极则平。请以身先。”遂装还乡里。田畴曰:“邢颙,民之先觉也。”乃见太祖,求为乡导以克柳城。

太祖辟颙为冀州从事,时人称之曰:“德行堂堂邢子昂。”除广宗长,以故将丧弃官。有司举正,太祖曰:“颙笃於旧君,有一致之节。”勿问也。更辟司空掾,除行唐令,劝民农桑,风化大行。入为丞相门下督,迁左冯翊,病,去官。是时,太祖诸子高选官属,令曰:“侯家吏,宜得渊深法度如邢颙辈。”遂以为平原侯植家丞。颙防闲以礼,无所屈挠,由是不合。庶子刘桢书谏植曰:“家丞邢颙,北土之彦,少秉高节,玄静澹泊,言少理多,真雅士也。桢诚不足同贯斯人,并列左右。而桢礼遇殊特,颙反疏简,私惧观者将谓君侯习近不肖,礼贤不足,采庶子之春华,忘家丞之秋实。为上招谤,其罪不小,以此反侧。”后参丞相军事,转东曹掾。初,太子未定,而临菑侯植有宠,丁仪等并赞翼其美。太祖问颙,颙对曰:“以庶代宗,先世之戒也。愿殿下深重察之!”太祖识其意,后遂以为太子少傅,迁太傅。文帝践阼,为侍中尚书仆射,赐爵关内侯,出为司隶校尉,徙太常。黄初四年薨。子友嗣。

鲍勋字叔业,泰山平阳人也,汉司隶校尉鲍宣九世孙。宣后嗣有从上党徙泰山者,遂家焉。勋父信,灵帝时为骑都尉,大将军何进遣东募兵。后为济北相,协规太祖,身以遇害。语在董卓传、武帝纪。建安十七年,太祖追录信功,表封勋兄邵新都亭侯。辟勋丞相掾。

二十二年,立太子,以勋为中庶子。徙黄门侍郎,出为魏郡西部都尉。太子郭夫人弟为曲周县吏,断盗官布,法应弃市。太祖时在谯,太子留邺,数手书为之请罪。勋不敢擅纵,具列上。勋前在东宫,守正不挠,太子固不能悦,及重此事,恚望滋甚。会郡界休兵有失期者,密敕中尉奏免勋官。久之,拜侍御史。延康元年,太祖崩,太子即王位,勋以驸马都尉兼侍中。

文帝受禅,勋每陈“今之所急,唯在军农,宽惠百姓。台榭苑囿,宜以为后。”文帝将出游猎,勋停车上疏曰:“臣闻五帝三王,靡不明本立教,以孝治天下。陛下仁圣恻隐,有同古烈。臣冀当继踪前代,令万世可则也。如何在谅闇之中,修驰骋之事乎!臣冒死以闻,唯陛下察焉。”帝手毁其表而竞行猎,中道顿息,问侍臣曰:“猎之为乐,何如八音也?”侍中刘晔对曰:“猎胜於乐。”勋抗辞曰:“夫乐,上通神明,下和人理,隆治致化,万邦咸乂。移风易俗,莫善於乐。况猎,暴华盖於原野,伤生育之至理,栉风沐雨,不以时隙哉?昔鲁隐观渔於棠,春秋讥之。虽陛下以为务,愚臣所不愿也。”因奏:“刘晔佞谀不忠,阿顺陛下过戏之言。昔梁丘据取媚於遄台,晔之谓也。请有司议罪以清皇庙。”帝怒作色,罢还,即出勋为右中郎将。

黄初四年,尚书令陈群、仆射司马宣王并举勋为宫正,宫正即御史中丞也。帝不得已而用之,百寮严惮,罔不肃然。六年秋,帝欲征吴,群臣大议,勋面谏曰:“王师屡征而未有所克者,盖以吴、蜀唇齿相依,凭阻山水,有难拔之势故也。往年龙舟飘荡,隔在南岸,圣躬蹈危,臣下破胆。此时宗庙几至倾覆,为百世之戒。今又劳兵袭远,日费千金,中国虚耗,令黠虏玩威,臣窃以为不可。”帝益忿之,左迁勋为治书执法。

帝从寿春还,屯陈留郡界。太守孙邕见,出过勋。时营垒未成,但立标埒,邕邪行不从正道,军营令史刘曜欲推之,勋以堑垒未成,解止不举。大军还洛阳,曜有罪,勋奏绌遣,而曜密表勋私解邕事。诏曰:“勋指鹿作马,收付廷尉。”廷尉法议:“正刑五岁。”三官駮:“依律罚金二斤。”帝大怒曰:“勋无活分,而汝等敢纵之!收三官已下付刺奸,当令十鼠同穴。”太尉锺繇、司徒华歆、镇军大将军陈群、侍中辛毗、尚书卫臻、守廷尉高柔等并表”勋父信有功於太祖“,求请勋罪。帝不许,遂诛勋。勋内行既脩,廉而能施,死之日,家无馀财。后二旬,文帝亦崩,莫不为勋叹恨。

司马芝字子华,河内温人也。少为书生,避乱荆州,於鲁阳山遇贼,同行者皆弃老弱走,芝独坐守老母。贼至,以刃临芝,芝叩头曰:“母老,唯在诸君!”贼曰:“此孝子也,杀之不义。”遂得免害,以鹿车推载母。居南方十馀年,躬耕守节。

太祖平荆州,以芝为菅长。时天下草创,多不奉法。郡主簿刘节,旧族豪侠,宾客千馀家,出为盗贼,入乱吏治。顷之,芝差节客王同等为兵,掾史据白:“节家前后未尝给繇,若至时藏匿,必为留负。”芝不听,与节书曰:“君为大宗,加股肱郡,而宾客每不与役,既众庶怨望,或流声上闻。今调同等为兵,幸时发遣。”兵已集郡,而节藏同等,因令督邮以军兴诡责县,县掾史穷困,乞代同行。芝乃驰檄济南,具陈节罪。太守郝光素敬信芝,即以节代同行,青州号芝“以郡主簿为兵”。迁广平令。征虏将军刘勋,贵宠骄豪,又芝故郡将,宾客子弟在界数犯法。勋与芝书,不著姓名,而多所属讬,芝不报其书,一皆如法。后勋以不轨诛,交关者皆获罪,而芝以见称。

迁大理正。有盗官练置都厕上者,吏疑女工,收以付狱。芝曰:“夫刑罪之失,失在苛暴。今赃物先得而后讯其辞,若不胜掠,或至诬服。诬服之情,不可以折狱。且简而易从,大人之化也。不失有罪,庸世之治耳。今宥所疑,以隆易从之义,不亦可乎!”太祖从其议。历甘陵、沛、阳平太守,所在有绩。黄初中,入为河南尹,抑强扶弱,私请不行。会内官欲以事讬芝,不敢发言,因芝妻伯父董昭。昭犹惮芝,不为通。芝为教与群下曰:“盖君能设教,不能使吏必不犯也。吏能犯教,而不能使君必不闻也。夫设教而犯,君之劣也;犯教而闻,吏之祸也。君劣於上,吏祸於下,此政事所以不理也。可不各勉之哉!”於是下吏莫不自励。门下循行尝疑门幹盗簪,幹辞不符,曹执为狱。芝教曰:“凡物有相似而难分者,自非离娄,鲜能不惑。就其实然,循行何忍重惜一簪,轻伤同类乎!其寝勿问。”

明帝即位,赐爵关内侯。顷之,特进曹洪乳母当,与临汾公主侍者共事无涧神系狱。卞太后遣黄门诣府传令,芝不通,辄敕洛阳狱考竟,而上疏曰:“诸应死罪者,皆当先表须报。前制书禁绝淫祀以正风俗,今当等所犯妖刑,辞语始定,黄门吴达诣臣,传太皇太后令。臣不敢通,惧有救护,速闻圣听,若不得已,以垂宿留。由事不早竟,是臣之罪,是以冒犯常科,辄敕县考竟,擅行刑戮,伏须诛罚。”帝手报曰:“省表,明卿至心,欲奉诏书,以权行事,是也。此乃卿奉诏之意,何谢之有?后黄门复往,慎勿通也。”芝居官十一年,数议科条所不便者。其在公卿间,直道而行。会诸王来朝,与京都人交通,坐免。

后为大司农。先是诸典农各部吏民,末作治生,以要利入。芝奏曰:“王者之治,崇本抑末,务农重谷。王制:‘无三年之储,国非其国也。’管子区言以积谷为急。方今二虏未灭,师旅不息,国家之要,惟在谷帛。武皇帝特开屯田之官,专以农桑为业。建安中,天下仓廪充实,百姓殷足。自黄初以来,听诸典农治生,各为部下之计,诚非国家大体所宜也。夫王者以海内为家,故传曰:‘百姓不足,君谁与足!'富足之田,在於不失天时而尽地力。今商旅所求,虽有加倍之显利,然於一统之计,已有不赀之损,不如垦田益一亩之收也。夫农民之事田,自正月耕种,耘锄条桑,耕种麦,获刈筑场,十月乃毕。治廪系桥,运输租赋,除道理梁,墐涂室屋,以是终岁,无日不为农事也。今诸典农,各言'留者为行者宗田计,课其力,势不得不尔。不有所废,则当素有馀力。’臣愚以为不宜复以商事杂乱,专以农桑为务,於国计为便。“明帝从之。

每上官有所召问,常先见掾史,为断其意故,教其所以答塞之状,皆如所度。芝性亮直,不矜廉隅。与宾客谈论,有不可意,便面折其短,退无异言。卒於官,家无馀财,自魏迄今为河南尹者莫及芝。

芝亡,子岐嗣,从河南丞转廷尉正,迁陈留相。梁郡有系囚,多所连及,数岁不决。诏书徙狱于岐属县,县请豫治牢具。岐曰:“今囚有数十,既巧诈难符,且已倦楚毒,其情易见。岂当复久处囹圄邪!”及囚室,诘之,皆莫敢匿诈,一朝决竟,遂超为廷尉。是时大将军爽专权,尚书何晏、邓飏等为之辅翼。南阳圭泰尝以言迕指,考系廷尉。飏讯狱,将致泰重刑。岐数飏曰:“夫枢机大臣,王室之佐,既不能辅化成德,齐美古人,而乃肆其私忿,枉论无辜。使百姓危心,非此焉在?”飏於是惭怒而退。岐终恐久获罪,以疾去官。居家未期而卒,年三十五。子肇嗣。

评曰:徐奕、何夔、邢颙贵尚峻厉,为世名人。毛玠清公素履,司马芝忠亮不倾,庶乎不吐刚茹柔。崔琰高格最优,鲍勋秉正无亏,而皆不免其身,惜哉!大雅贵“既明且哲”,虞书尚“直而能温”,自非兼才,畴克备诸!

\part{魏书十三}
\chapter{锺繇华歆王朗传第十三}

锺繇字元常,颍川长社人也。尝与族父瑜俱至洛阳,道遇相者,曰:“此童有贵相,然当厄於水,努力慎之!”行未十里,度桥,马惊,堕水几死。瑜以相者言中,益贵繇,而供给资费,使得专学。举孝廉,除尚书郎、阳陵令,以疾去。辟三府,为廷尉正、黄门侍郎。是时,汉帝在西京,李傕、郭汜等乱长安中,与关东断绝。太祖领兖州牧,始遣使上书。傕、汜等以为“关东欲自立天子,今曹操虽有使命,非其至实”,议留太祖使,拒绝其意。繇说傕、汜等曰:“方今英雄并起,各矫命专制,唯曹兖州乃心王室,而逆其忠款,非所以副将来之望也。”傕、汜等用繇言,厚加答报,由是太祖使命遂得通。太祖既数听荀彧之称繇,又闻其说傕、汜,益虚心。后傕胁天子,繇与尚书郎韩斌同策谋。天子得出长安,繇有力焉。拜御史中丞,迁侍中尚书仆射,并录前功封东武亭侯。

时关中诸将马腾、韩遂等,各拥强兵相与争。太祖方有事山东,以关右为忧。乃表繇以侍中守司隶校尉,持节督关中诸军,委之以后事,特使不拘科制。繇至长安,移书腾、遂等,为陈祸福,腾、遂各遣子入侍。太祖在官渡,与袁绍相持,繇送马二千馀匹给军。太祖与繇书曰:“得所送马,甚应其急。关右平定,朝廷无西顾之忧,足下之勋也。昔萧何镇守关中,足食成军,亦適当尔。”其后匈奴单于作乱平阳,繇帅诸军围之,未拔;而袁尚所置河东太守郭援到河东,众甚盛。诸将议欲释之去,繇曰:“袁氏方强,援之来,关中阴与之通,所以未悉叛者,顾吾威名故耳。若弃而去,示之以弱,所在之民,谁非寇雠?纵吾欲归,其得至乎!此为未战先自败也。且援刚愎好胜,必易吾军,若渡汾为营,及其未济击之,可大克也。”张既说马腾会击援,腾遣子超将精兵逆之。援至,果轻渡汾,众止之,不从。济水未半,击,大破之,斩援,降单于。语在既传。其后河东卫固作乱,与张晟、张琰及高幹等并为寇,繇又率诸将讨破之。自天子西迁,洛阳人民单尽,繇徙关中民,又招纳亡叛以充之,数年间民户稍实。太祖征关中,得以为资,表繇为前军师。

魏国初建,为大理,迁相国。文帝在东宫,赐繇五熟釜,为之铭曰:“於赫有魏,作汉藩辅。厥相惟锺,实幹心膂。靖恭夙夜,匪遑安处。百寮师师,楷兹度矩。”数年,坐西曹掾魏讽谋反,策罢就第。文帝即王位,复为大理。及践阼,改为廷尉,进封崇高乡侯。迁太尉,转封平阳乡侯。时司徒华歆、司空王朗,并先世名臣。文帝罢朝,谓左右曰:“此三公者,乃一代之伟人也,后世殆难继矣!”明帝即位,进封定陵侯,增邑五百,并前千八百户,迁太傅。繇有膝疾,拜起不便。时华歆亦以高年疾病,朝见皆使载舆车,虎贲舁上殿就坐。是后三公有疾,遂以为故事。

初,太祖下令,使平议死刑可宫割者。繇以为“古之肉刑,更历圣人,宜复施行,以代死刑。”议者以为非悦民之道,遂寝。及文帝临飨群臣,诏谓“大理欲复肉刑,此诚圣王之法。公卿当善共议。”议未定,会有军事,复寝。太和中,繇上疏曰:“大魏受命,继踪虞、夏。孝文革法,不合古道。先帝圣德,固天所纵,坟典之业,一以贯之。是以继世,仍发明诏,思复古刑,为一代法。连有军事,遂未施行。陛下远追二祖遗意,惜斩趾可以禁恶,恨入死之无辜,使明习律令,与群臣共议。出本当右趾而入大辟者,复行此刑。书云:‘皇帝清问下民,鳏寡有辞于苗。'此言尧当除蚩尤、有苗之刑,先审问於下民之有辞者也。若今蔽狱之时,讯问三槐、九棘、群吏、万民,使如孝景之令,其当弃巿,欲斩右趾者许之。其黥、劓、左趾、宫刑者,自如孝文,易以髡、笞。能有奸者,率年二十至四五十,虽斩其足,犹任生育。今天下人少于孝文之世,下计所全,岁三千人。张苍除肉刑,所杀岁以万计。臣欲复肉刑,岁生三千人。子贡问能济民可谓仁乎?子曰:‘何事於仁,必也圣乎,尧、舜其犹病诸!’又曰:‘仁远乎哉?我欲仁,斯仁至矣。’若诚行之,斯民永济。”书奏,诏曰:“太傅学优才高,留心政事,又於刑理深远。此大事,公卿群僚善共平议。”司徒王朗议,以为“繇欲轻减大辟之条,以增益刖刑之数,此即起偃为竖,化尸为人矣。然臣之愚,犹有未合微异之意。夫五刑之属,著在科律,自有减死一等之法,不死即为减。施行已久,不待远假斧凿于彼肉刑,然后有罪次也。前世仁者,不忍肉刑之惨酷,是以废而不用。不用已来,历年数百。今复行之,恐所减之文未彰于万民之目,而肉刑之问已宣于寇雠之耳,非所以来远人也。今可按繇所欲轻之死罪,使减死之髡、刖。嫌其轻者,可倍其居作之岁数。内有以生易死不訾之恩,外无以刖易釱骇耳之声。”议者百馀人,与朗同者多。帝以吴、蜀未平,且寝。

太和四年,繇薨。帝素服临吊,谥曰成侯。子毓嗣。初,文帝分毓户邑,封繇弟演及子劭、孙豫列侯。

毓字稚叔。年十四为散骑侍郎,机捷谈笑,有父风。太和初,蜀相诸葛亮围祁山,明帝欲西征,毓上疏曰:“夫策贵庙胜,功尚帷幄,不下殿堂之上,而决胜千里之外。车驾宜镇守中土,以为四方威势之援。今大军西征,虽有百倍之威,於关中之费,所损非一。且盛暑行师,诗人所重,实非至尊动轫之时也。”迁黄门侍郎。时大兴洛阳宫室,车驾便幸许昌,天下当朝正许昌。许昌偪狭,於城南以毡为殿,备设鱼龙曼延,民罢劳役。毓谏,以为“水旱不时,帑藏空虚,凡此之类,可须丰年。“又上”宜复关内开荒地,使民肆力於农。”事遂施行。正始中,为散骑常侍。大将军曹爽盛夏兴军伐蜀,蜀拒守,军不得进。爽方欲增兵,毓与书曰:“窃以为庙胜之策,不临矢石;王者之兵,有征无战。诚以干戚可以服有苗,退舍足以纳原寇,不必纵吴汉于江关,骋韩信於井陉也。见可而进,知难而退,盖自古之政。惟公侯详之!”爽无功而还。后以失爽意,徙侍中,出为魏郡太守。爽既诛,入为御史中丞、侍中廷尉。听君父已没,臣子得为理谤,及士为侯,其妻不复配嫁,毓所创也。

正元中,毋丘俭、文钦反,毓持节至扬、豫州班行赦令,告谕士民,还为尚书。诸葛诞反,大将军司马文王议自诣寿春讨诞。会吴大将孙壹率众降,或以为“吴新有衅,必不能复出军。东兵已多,可须后问”。毓以为“夫论事料敌,当以己度人。今诞举淮南之地以与吴国,孙壹所率,口不至千,兵不过三百。吴之所失,盖为无几。若寿春之围未解,而吴国之内转安,未可必其不出也。”大将军曰:“善。”遂将毓行。淮南既平,为青州刺史,加后将军,迁都督徐州诸军事,假节,又转都督荆州。景元四年薨,追赠车骑将军,谥曰惠侯。子骏嗣。毓弟会,自有传。

华歆字子鱼,平原高唐人也。高唐为齐名都,衣冠无不游行市里。歆为吏,休沐出府,则归家阖门。议论持平,终不毁伤人。同郡陶丘洪亦知名,自以明见过歆。时王芬与豪杰谋废灵帝。语在武纪。芬阴呼歆、洪共定计,洪欲行,歆止之曰:“夫废立大事,伊、霍之所难。芬性疏而不武,此必无成,而祸将及族。子其无往!”洪从歆言而止。后芬果败,洪乃服。举孝廉,除郎中,病,去官。灵帝崩,何进辅政,徵河南郑泰、颍川荀攸及歆等。歆到,为尚书郎。董卓迁天子长安,歆求出为下邽令,病不行,遂从蓝田至南阳。时袁术在穰,留歆。歆说术使进军讨卓,术不能用。歆欲弃去,会天子使太傅马日磾安集关东,日磾辟歆为掾。东至徐州,诏即拜歆豫章太守,以为政清静不烦,吏民感而爱之。孙策略地江东,歆知策善用兵,乃幅巾奉迎。策以其长者,待以上宾之礼。后策死。太祖在官渡,表天子徵歆。孙权欲不遣,歆谓权曰:“将军奉王命,始交好曹公,分义未固,使仆得为将军效心,岂不有益乎?今空留仆,是为养无用之物,非将军之良计也。”权悦,乃遣歆。宾客旧人送之者千馀人,赠遗数百金。歆皆无所拒,密各题识,至临去,悉聚诸物,谓诸宾客曰:“本无拒诸君之心,而所受遂多。念单车远行,将以怀璧为罪,愿宾客为之计。”众乃各留所赠,而服其德。

歆至,拜议郎,参司空军事,入为尚书,转侍中,代荀彧为尚书令。太祖征孙权,表歆为军师。魏国既建,为御史大夫。文帝即王位,拜相国,封安乐乡侯。及践阼,改为司徒。歆素清贫,禄赐以振施亲戚故人,家无担石之储。公卿尝并赐没入生口,唯歆出而嫁之。帝叹息,下诏曰:“司徒,国之俊老,所与和阴阳理庶事也。今大官重膳,而司徒蔬食,甚无谓也。”特赐御衣,及为其妻子男女皆作衣服。三府议:“举孝廉,本以德行,不复限以试经。“歆以为”丧乱以来,六籍堕废,当务存立,以崇王道。夫制法者,所以经盛衰。今听孝廉不以经试,恐学业遂从此而废。若有秀异,可特徵用。患於无其人,何患不得哉?”帝从其言。

黄初中,诏公卿举独行君子,歆举管宁,帝以安车徵之。明帝即位,进封博平侯,增邑五百户,并前千三百户,转拜太尉。歆称病乞退,让位於宁。帝不许。临当大会,乃遣散骑常侍缪袭奉诏喻指曰:“朕新莅庶事,一日万几,惧听断之不明。赖有德之臣,左右朕躬,而君屡以疾辞位。夫量主择君,不居其朝,委荣弃禄,不究其位,古人固有之矣,顾以为周公、伊尹则不然。絜身徇节,常人为之,不望之於君。君其力疾就会,以惠予一人。将立席几筵,命百官总己,以须君到,朕然后御坐。”又诏袭:“须歆必起,乃还。”歆不得已,乃起。

太和中,遣曹真从子午道伐蜀,车驾东幸许昌。歆上疏曰:“兵乱以来,过逾二纪。大魏承天受命,陛下以圣德当成康之隆,宜弘一代之治,绍三王之迹。虽有二贼负险延命,苟圣化日跻,远人怀德,将襁负而至。夫兵不得已而用之,故戢而时动。臣诚愿陛下先留心於治道,以征伐为后事。且千里运粮,非用兵之利;越险深入,无独克之功。如闻今年徵役,颇失农桑之业。为国者以民为基,民以衣食为本。使中国无饥寒之患,百姓无离土之心,则天下幸甚,二贼之衅,可坐而待也。臣备位宰相,老病日笃,犬马之命将尽,恐不复奉望銮盖,不敢不竭臣子之怀,唯陛下裁察!”帝报曰:“君深虑国计,朕甚嘉之。贼凭恃山川,二祖劳於前世,犹不克平,朕岂敢自多,谓必灭之哉!诸将以为不一探取,无由自弊,是以观兵以闚其衅。若天时未至,周武还师,乃前事之鉴,朕敬不忘所戒。”时秋大雨,诏真引军还。太和五年,歆薨,谥曰敬侯。子表嗣。初,文帝分歆户邑,封歆弟缉列侯。表,咸熙中为尚书。

王朗字景兴,东海郯人也。以通经,拜郎中,除菑丘长。师太尉杨赐,赐薨,弃官行服。举孝廉,辟公府,不应。徐州刺史陶谦察朗茂才。时汉帝在长安,关东兵起,朗为谦治中,与别驾赵昱等说谦曰:“春秋之义,求诸侯莫如勤王。今天子越在西京,宜遣使奉承王命。“谦乃遣昱奉章至长安。天子嘉其意,拜谦安东将军。以昱为广陵太守,朗会稽太守。孙策渡江略地。朗功曹虞翻以为力不能拒,不如避之。朗自以身为汉吏,宜保城邑,遂举兵与策战,败绩,浮海至东冶。策又追击,大破之。朗乃诣策。策以朗儒雅,诘让而不害。虽流移穷困,朝不谋夕,而收恤亲旧,分多割少,行义甚著。

太祖表徵之,朗自曲阿展转江海,积年乃至。拜谏议大夫,参司空军事。魏国初建,以军祭酒领魏郡太守,迁少府、奉常、大理。务在宽恕,罪疑从轻。锺繇明察当法,俱以治狱见称。

文帝即王位,迁御史大夫,封安陵亭侯。上疏劝育民省刑曰:“兵起已来三十馀年,四海荡覆,万国殄瘁。赖先王芟除寇贼,扶育孤弱,遂令华夏复有纲纪。鸠集兆民,于兹魏土,使封鄙之内,鸡鸣狗吠,达於四境,蒸庶欣欣,喜遇升平。今远方之寇未宾,兵戎之役未息,诚令复除足以怀远人,良宰足以宣德泽,阡陌咸修,四民殷炽,必复过於曩时而富於平日矣。易称敕法,书著祥刑,一人有庆,兆民赖之,慎法狱之谓也。昔曹相国以狱市为寄,路温舒疾治狱之吏。夫治狱者得其情,则无冤死之囚;丁壮者得尽地力,则无饥馑之民;穷老者得仰食仓廪,则无喂饿之殍;嫁娶以时,则男女无怨旷之恨;胎养必全,则孕者无自伤之哀;新生必复,则孩者无不育之累;壮而后役,则幼者无离家之思;二毛不戎,则老者无顿伏之患。医药以疗其疾,宽繇以乐其业,威罚以抑其强,恩仁以济其弱,赈贷以赡其乏。十年之后,既笄者必盈巷。二十年之后,胜兵者必满野矣。”

及文帝践阼,改为司空,进封乐平乡侯。时帝颇出游猎,或昏夜还宫。朗上疏曰:“夫帝王之居,外则饰周卫,内则重禁门,将行则设兵而后出幄,称警而后践墀,张弧而后登舆,清道而后奉引,遮列而后转毂,静室而后息驾,皆所以显至尊,务戒慎,垂法教也。近日车驾出临捕虎,日昃而行,及昏而反,违警跸之常法,非万乘之至慎也。”帝报曰:“览表,虽魏绛称虞箴以讽晋悼,相如陈猛兽以戒汉武,未足以喻。方今二寇未殄,将帅远征,故时入原野以习戎备。至於夜还之戒,已诏有司施行。”

初,建安末,孙权始遣使称藩,而与刘备交兵。诏议“当兴师与吴并取蜀不”?朗议曰:“天子之军,重於华、岱,诚宜坐曜天威,不动若山。假使权亲与蜀贼相持,搏战旷日,智均力敌,兵不速决,当须军兴以成其势者,然后宜选持重之将,承寇贼之要,相时而后动,择地而后行,一举更无馀事。今权之师未动,则助吴之军无为先征。且雨水方盛,非行军动众之时。”帝纳其计。黄初中,鹈鹕集灵芝池,诏公卿举独行君子。朗荐光禄大夫杨彪,且称疾,让位於彪。帝乃为彪置吏卒,位次三公。诏曰:“朕求贤於君而未得,君乃翻然称疾,非徒不得贤,更开失贤之路,增玉铉之倾。无乃居其室出其言不善,见违於君子乎!君其勿有后辞。”朗乃起。

孙权欲遣子登入侍,不至。是时车驾徙许昌,大兴屯田,欲举军东征。朗上疏曰:“昔南越守善,婴齐入侍,遂为冢嗣,还君其国。康居骄黠,情不副辞,都护奏议以为宜遣侍子,以黜无礼。且吴濞之祸,萌於子入,隗嚣之叛,亦不顾子。往者闻权有遣子之言而未至,今六军戒严,臣恐舆人未畅圣旨,当谓国家愠於登之逋留,是以为之兴师。设师行而登乃至,则为所动者至大,所致者至细,犹未足以为庆。设其傲狠,殊无入志,惧彼舆论之未畅者,并怀伊邑。臣愚以为宜敕别征诸将,各明奉禁令,以慎守所部。外曜烈威,内广耕稼,使泊然若山,澹然若渊,势不可动,计不可测。”是时,帝以成军遂行,权子不至,车驾临江而还。

明帝即位,进封兰陵侯,增邑五百,并前千二百户。使至邺省文昭皇后陵,见百姓或有不足。是时方营修宫室,朗上疏曰:“陛下即位已来,恩诏屡布,百姓万民莫不欣欣。臣顷奉使北行,往反道路,闻众徭役,其可得蠲除省减者甚多。愿陛下重留日昃之听,以计制寇。昔大禹将欲拯天下之大患,故乃先卑其宫室,俭其衣食,用能尽有九州,弼成五服。勾践欲广其御儿之疆,馘夫差於姑苏,故亦约其身以及家,俭其家以施国,用能囊括五湖,席卷三江,取威中国,定霸华夏。汉之文、景亦欲恢弘祖业,增崇洪绪,故能割意於百金之台,昭俭於弋绨之服,内减太官而不受贡献,外省徭赋而务农桑,用能号称升平,几致刑错。孝武之所以能奋其军势,拓其外境,诚因祖考畜积素足,故能遂成大功。霍去病,中才之将,犹以匈奴未灭,不治第宅。明恤远者略近,事外者简内。自汉之初及其中兴,皆於金革略寝之后,然后凤阙猥闶,德阳并起。今当建始之前足用列朝会,崇华之后足用序内官,华林、天渊足用展游宴,若且先成阊阖之象魏,使足用列远人之朝贡者,脩城池,使足用绝逾越,成国险,其馀一切,且须丰年。一以勤耕农为务,习戎备为事,则国无怨旷,户口滋息,民充兵强,而寇戎不宾,缉熙不足,未之有也。”转为司徒。

时屡失皇子,而后宫就馆者少,朗上疏曰:“昔周文十五而有武王,遂享十子之祚,以广诸姬之胤。武王既老而生成王,成王是以鲜於兄弟。此二王者,各树圣德,无以相过,比其子孙之祚,则不相如。盖生育有早晚,所产有众寡也。陛下既德祚兼彼二圣,春秋高於姬文育武之时矣,而子发未举於椒兰之奥房,藩王未繁於掖庭之众室。以成王为喻,虽未为晚,取譬伯邑,则不为夙。周礼六宫内官百二十人,而诸经常说,咸以十二为限,至於秦汉之末,或以千百为数矣。然虽弥猥,而就时於吉馆者或甚鲜,明'百斯男'之本,诚在於一意,不但在於务广也。老臣慺慺,愿国家同祚於轩辕之五五,而未及周文之二五,用为伊邑。且少小常苦被褥泰温,泰温则不能便柔肤弱体,是以难可防护,而易用感慨。若常令少小之缊袍,不至於甚厚,则必咸保金石之性,而比寿於南山矣。”帝报曰:“夫忠至者辞笃,爱重者言深。君既劳思虑,又手笔将顺,三复德音,欣然无量。朕继嗣未立,以为君忧,钦纳至言,思闻良规。”朗著易、春秋、孝经、周官传,奏议论记,咸传於世。太和二年薨,谥曰成侯。子肃嗣。初,文帝分朗户邑,封一子列侯,朗乞封兄子详。

肃字子雍。年十八,从宋忠读太玄,而更为之解。黄初中,为散骑黄门侍郎。太和三年,拜散骑常侍。四年,大司马曹真征蜀,肃上疏曰:“前志有之,‘千里馈粮,士有饥色,樵苏后爨,师不宿饱’,此谓平涂之行军者也。又况於深入阻险,凿路而前,则其为劳必相百也。今又加之以霖雨,山坂峻滑,众逼而不展,粮县而难继,实行军者之大忌也。闻曹真发已逾月而行裁半谷,治道功夫,战士悉作。是贼偏得以逸而待劳,乃兵家之所惮也。言之前代,则武王伐纣,出关而复还;论之近事,则武、文征权,临江而不济。岂非所谓顺天知时,通於权变者哉!兆民知圣上以水雨艰剧之故,休而息之,后日有衅,乘而用之,则所谓悦以犯难,民忘其死者矣。”於是遂罢。又上疏:“宜遵旧礼,为大臣发哀,荐果宗庙。”事皆施行。又上疏陈政本曰:“除无事之位,损不急之禄,止浮食之费,并从容之官;使官必有职,职任其事,事必受禄,禄代其耕,乃往古之常式,当今之所宜也。官寡而禄厚,则公家之费鲜,进仕之志劝。各展才力,莫相倚仗。敷奏以言,明试以功,能之与否,简在帝心。是以唐、虞之设官分职,申命公卿,各以其事,然后惟龙为纳言,犹今尚书也,以出内帝命而已。夏、殷不可得而详。甘誓曰‘六事之人’,明六卿亦典事者也。周官则备矣,五日视朝,公卿大夫并进,而司士辨其位焉。其记曰:‘坐而论道,谓之王公;作而行之,谓之士大夫。’及汉之初,依拟前代,公卿皆亲以事升朝。故高祖躬追反走之周昌,武帝遥可奉奏之汲黯,宣帝使公卿五日一朝,成帝始置尚书五人。自是陵迟,朝礼遂阙。可复五日视朝之仪,使公卿尚书各以事进。废礼复兴,光宣圣绪,诚所谓名美而实厚者也。”

青龙中,山阳公薨,汉主也。肃上疏曰:“昔唐禅虞,虞禅夏,皆终三年之丧,然后践天子之尊。是以帝号无亏,君礼犹存。今山阳公承顺天命,允答民望,进禅大魏,退处宾位。公之奉魏,不敢不尽节。魏之待公,优崇而不臣。既至其薨,榇敛之制,舆徒之饰,皆同之於王者,是故远近归仁,以为盛美。且汉总帝皇之号,号曰皇帝。有别称帝,无别称皇,则皇是其差轻者也。故当高祖之时,土无二王,其父见在而使称皇,明非二王之嫌也。况今以赠终,可使称皇以配其谥。”明帝不从,使称皇,乃追谥曰汉孝献皇帝。

后肃以常侍领秘书监,兼崇文观祭酒。景初间,宫室盛兴,民失农业,期信不敦,刑杀仓卒。肃上疏曰:“大魏承百王之极,生民无几,干戈未戢,诚宜息民而惠之以安静遐迩之时也。夫务畜积而息疲民,在於省徭役而勤稼穑。今宫室未就,功业未讫,运漕调发,转相供奉。是以丁夫疲於力作,农者离其南亩,种谷者寡,食谷者众,旧谷既没,新谷莫继。斯则有国之大患,而非备豫之长策也。今见作者三四万人,九龙可以安圣体,其内足以列六宫,显阳之殿,又向将毕,惟泰极已前,功夫尚大,方向盛寒,疾疢或作。诚愿陛下发德音,下明诏,深愍役夫之疲劳,厚矜兆民之不赡,取常食廪之士,非急要者之用,选其丁壮,择留万人,使一期而更之,咸知息代有日,则莫不悦以即事,劳而不怨矣。计一岁有三百六十万夫,亦不为少。当一岁成者,听且三年。分遣其馀,使皆即农,无穷之计也。仓有溢粟,民有馀力:以此兴功,何功不立?以此行化,何化不成?夫信之於民,国家大宝也。仲尼曰:‘自古皆有死,民非信不立。’夫区区之晋国,微微之重耳,欲用其民,先示以信,是故原虽将降,顾信而归,用能一战而霸,于今见称。前车驾当幸洛阳,发民为营,有司命以营成而罢。既成,又利其功力,不以时遣。有司徒营其目前之利,不顾经国之体。臣愚以为自今以后,傥复使民,宜明其令,使必如期。若有事以次,宁复更发,无或失信。凡陛下临时之所行刑,皆有罪之吏,宜死之人也。然众庶不知,谓为仓卒。故愿陛下下之於吏而暴其罪,钧其死也,无使汙于宫掖而为远近所疑。且人命至重,难生易杀,气绝而不续者也,是以圣贤重之。孟轲称杀一无辜以取天下,仁者不为也。汉时有犯跸惊乘舆马者,廷尉张释之奏使罚金,文帝怪其轻,而释之曰:'方其时,上使诛之则已。今下廷尉。廷尉,天下之平也,一倾之,天下用法皆为轻重,民安所措其手足?'臣以为大失其义,非忠臣所宜陈也。廷尉者,天子之吏也,犹不可以失平,而天子之身,反可以惑谬乎?斯重於为己,而轻於为君,不忠之甚也。周公曰:‘天子无戏言;言则史书之,工诵之,士称之。’言犹不戏,而况行之乎?故释之之言不可不察,周公之戒不可不法也。”又陈“诸鸟兽无用之物,而有刍谷人徒之费,皆可蠲除。”

帝尝问曰:“汉桓帝时,白马令李云上书言:‘帝者,谛也。是帝欲不谛。’当何得不死?”肃对曰:“但为言失逆顺之节。原其本意,皆欲尽心,念存补国。且帝者之威,过於雷霆,杀一匹夫,无异蝼蚁。宽而宥之,可以示容受切言,广德宇於天下。故臣以为杀之未必为是也。”帝又问:“司马迁以受刑之故,内怀隐切,著史记非贬孝武,令人切齿。”对曰:“司马迁记事,不虚美,不隐恶。刘向、扬雄服其善叙事,有良史之才,谓之实录。汉武帝闻其述史记,取孝景及己本纪览之,於是大怒,削而投之。於今此两纪有录无书。后遭李陵事,遂下迁蚕室。此为隐切在孝武,而不在於史迁也。”

正始元年,出为广平太守。公事徵还,拜议郎。顷之,为侍中,迁太常。时大将军曹爽专权,任用何晏、邓飏等。肃与太尉蒋济、司农桓范论及时政,肃正色曰:“此辈即弘恭、石显之属,复称说邪!”爽闻之,戒何晏等曰:“当共慎之!公卿已比诸君前世恶人矣。”坐宗庙事免。后为光禄勋。时有二鱼长尺,集于武库之屋,有司以为吉祥。肃曰:“鱼生於渊而亢於屋,介鳞之物失其所也。边将其殆有弃甲之变乎?”其后果有东关之败。徙为河南尹。嘉平六年,持节兼太常,奉法驾,迎高贵乡公于元城。是岁,白气经天,大将军司马景王问肃其故,肃答曰:“此蚩尤之旗也,东南其有乱乎?君若脩己以安百姓,则天下乐安者归德,唱乱者先亡矣。”明年春,镇东将军毌丘俭、扬州刺史文钦反,景王谓肃曰:“霍光感夏侯胜之言,始重儒学之士,良有以也。安国宁主,其术焉在?”肃曰:“昔关羽率荆州之众,降于禁於汉滨,遂有北向争天下之志。后孙权袭取其将士家属,羽士众一旦瓦解。今淮南将士父母妻子皆在内州,但急往御卫,使不得前,必有关羽土崩之势矣。”景王从之,遂破俭、钦。后迁中领军,加散骑常侍,增邑三百,并前二千二百户。甘露元年薨,门生縗绖者以百数。追赠卫将军,谥曰景侯。子惲嗣。惲薨,无子,国绝。景元四年,封肃子恂为兰陵侯。咸熙中,开建五等,以肃著勋前朝,改封恂为氶子。

初,肃善贾、马之学,而不好郑氏,采会同异,为尚书、诗、论语、三礼、左氏解,及撰定父朗所作易传,皆列於学官。其所论駮朝廷典制、郊祀、宗庙、丧纪、轻重,凡百馀篇。时乐安孙叔然,受学郑玄之门,人称东州大儒。徵为秘书监,不就。肃集圣证论以讥短玄,叔然駮而释之,及作周易、春秋例,毛诗、礼记、春秋三传、国语、尔雅诸注,又注书十馀篇。自魏初徵士敦煌周生烈,明帝时大司农弘农董遇等,亦历注经传,颇传於世。

评曰:锺繇开达,华歆清纯德素,王朗文博富赡,诚皆一时之俊伟也。魏氏初祚,肇登三司,盛矣夫!王肃亮直多闻,能析薪哉!

\part{魏书十四}
\chapter{程郭董刘蒋刘传第十四}

程昱字仲德,东郡东阿人也。长八尺三寸,美须髯。黄巾起,县丞王度反应之,烧仓库。县令逾城走,吏民负老幼东奔渠丘山。昱使人侦视度,度等得空城不能守,出城西五六里止屯。昱谓县中大姓薛房等曰:“今度等得城郭不能居,其势可知。此不过欲虏掠财物,非有坚甲利兵攻守之志也。今何不相率还城而守之?且城高厚,多谷米,今若还求令,共坚守,度必不能久,攻可破也。”房等以为然。吏民不肯从,曰:“贼在西,但有东耳。”昱谓房等:“愚民不可计事。”乃密遣数骑举幡于东山上,令房等望见,大呼言“贼已至”,便下山趣城,吏民奔走随之,求得县令,遂共城守。度等来攻城,不能下,欲去。昱率吏民开城门急击之,度等破走。东阿由此得全。

初平中,兖州刺史刘岱辟昱,昱不应。是时岱与袁绍、公孙瓒和亲,绍令妻子居岱所,瓒亦遣从事范方将骑助岱。后绍与瓒有隙。瓒击破绍军,乃遣使语岱,令遣绍妻子,使与绍绝。别敕范方:“若岱不遣绍家,将骑还。吾定绍,将加兵于岱。”岱议连日不决,别驾王彧白岱:“程昱有谋,能断大事。”岱乃召见昱,问计,昱曰:“若弃绍近援而求瓒远助,此假人於越以救溺子之说也。夫公孙瓒,非袁绍之敌也。今虽坏绍军,然终为绍所禽。夫趣一朝之权而不虑远计,将军终败。”岱从之。范方将其骑归,未至,瓒大为绍所破。岱表昱为骑都尉,昱辞以疾。

刘岱为黄巾所杀。太祖临兖州,辟昱。昱将行,其乡人谓曰:“何前后之相背也!”昱笑而不应。太祖与语,说之,以昱守寿张令。太祖征徐州,使昱与荀彧留守鄄城。张邈等叛迎吕布,郡县响应,唯鄄城、范、东阿不动。布军降者,言陈宫欲自将兵取东阿,又使氾嶷取范,吏民皆恐。彧谓昱曰:“今兖州反,唯有此三城。宫等以重兵临之,非有以深结其心,三城必动。君,民之望也,归而说之,殆可!”昱乃归,过范,说其令靳允曰:“闻吕布执君母弟妻子,孝子诚不可为心!今天下大乱,英雄并起,必有命世,能息天下之乱者,此智者所详择也。得主者昌,失主者亡。陈宫叛迎吕布而百城皆应,似能有为,然以君观之,布何如人哉!夫布,粗中少亲,刚而无礼,匹夫之雄耳。宫等以势假合,不能相君也。兵虽众,终必无成。曹使君智略不世出,殆天所授!君必固范,我守东阿,则田单之功可立也。孰与违忠从恶而母子俱亡乎?唯君详虑之!“允流涕曰:“不敢有二心。”时氾嶷已在县,允乃见嶷,伏兵刺杀之,归勒兵守。昱又遣别骑绝仓亭津,陈宫至,不得渡。昱至东阿,东阿令枣祗已率厉吏民,拒城坚守。又兖州从事薛悌与昱协谋,卒完三城,以待太祖。太祖还,执昱手曰:“微子之力,吾无所归矣。”乃表昱为东平相,屯范。

太祖与吕布战于濮阳,数不利。蝗虫起,乃各引去。於是袁绍使人说太祖连和,欲使太祖迁家居邺。太祖新失兖州,军食尽,将许之。时昱使適还,引见,因言曰:“窃闻将军欲遣家,与袁绍连和,诚有之乎?”太祖曰:“然。”昱曰:“意者将军殆临事而惧,不然何虑之不深也!夫袁绍据燕、赵之地,有并天下之心,而智不能济也。将军自度能为之下乎?将军以龙虎之威,可为韩、彭之事邪?今兖州虽残,尚有三城。能战之士,不下万人。以将军之神武,与文若、昱等,收而用之,霸王之业可成也。愿将军更虑之!”太祖乃止。

天子都许,以昱为尚书。兖州尚未安集,复以昱为东中郎将,领济阴太守,都督兖州事。刘备失徐州,来归太祖。昱说太祖杀备,太祖不听。语在武纪。后又遣备至徐州要击袁术,昱与郭嘉说太祖曰:“公前日不图备,昱等诚不及也。今借之以兵,必有异心。”太祖悔,追之不及。会术病死,备至徐州,遂杀车胄,举兵背太祖。顷之,昱迁振威将军。袁绍在黎阳,将南渡。时昱有七百兵守鄄城,太祖闻之,使人告昱,欲益二千兵。昱不肯,曰:“袁绍拥十万众,自以所向无前。今见昱兵少,必轻易不来攻。若益昱兵,过则不可不攻,攻之必克,徒两损其势。愿公无疑!”太祖从之。绍闻昱兵少,果不往。太祖谓贾诩曰:“程昱之胆,过于贲、育。”昱收山泽亡命,得精兵数千人,乃引军与太祖会黎阳,讨袁谭、袁尚。谭、尚破走,拜昱奋武将军,封安国亭侯。太祖征荆州,刘备奔吴。论者以为孙权必杀备,昱料之曰:“孙权新在位,未为海内所惮。曹公无敌於天下,初举荆州,威震江表,权虽有谋,不能独当也。刘备有英名,关羽、张飞皆万人敌也,权必资之以御我。难解势分,备资以成,又不可得而杀也。”权果多与备兵,以御太祖。是后中夏渐平,太祖拊昱背曰:“兖州之败,不用君言,吾何以至此?”宗人奉牛酒大会,昱曰:“知足不辱,吾可以退矣。”乃自表归兵,阖门不出。

昱性刚戾,与人多迕。人有告昱谋反,太祖赐待益厚。魏国既建,为卫尉,与中尉邢贞争威仪,免。文帝践阼,复为卫尉,进封安乡侯,增邑三百户,并前八百户。分封少子延及孙晓列侯。方欲以为公,会薨,帝为流涕,追赠车骑将军,谥曰肃侯。子武嗣。武薨,子克嗣。克薨,子良嗣。

晓,嘉平中为黄门侍郎。时校事放横,晓上疏曰:“周礼云:‘设官分职,以为民极。’春秋传曰:‘天有十日,人有十等。’愚不得临贤,贱不得临贵。於是并建圣哲,树之风声。明试以功,九载考绩。各脩厥业,思不出位。故栾书欲拯晋侯,其子不听;死人横於街路,邴吉不问。上不责非职之功,下不务分外之赏,吏无兼统之势,民无二事之役,斯诚为国要道,治乱所由也。远览典志,近观秦汉,虽官名改易,职司不同,至于崇上抑下,显分明例,其致一也。初无校事之官干与庶政者也。昔武皇帝大业草创,众官未备,而军旅勤苦,民心不安,乃有小罪,不可不察,故置校事,取其一切耳,然检御有方,不至纵恣也。此霸世之权宜,非帝王之正典。其后渐蒙见任,复为疾病,转相因仍,莫正其本。遂令上察宫庙,下摄众司,官无局业,职无分限,随意任情,唯心所適。法造於笔端,不依科诏;狱成於门下,不顾覆讯。其选官属,以谨慎为粗疏,以謥詷为贤能。其治事,以刻暴为公严,以循理为怯弱。外则讬天威以为声势,内则聚群奸以为腹心。大臣耻与分势,含忍而不言,小人畏其锋芒,郁结而无告。至使尹模公于目下肆其奸慝;罪恶之著,行路皆知,纤恶之过,积年不闻。既非周礼设官之意,又非春秋十等之义也。今外有公卿将校总统诸署,内有侍中尚书综理万机,司隶校尉督察京辇,御史中丞董摄宫殿,皆高选贤才以充其职,申明科诏以督其违。若此诸贤犹不足任,校事小吏,益不可信。若此诸贤各思尽忠,校事区区,亦复无益。若更高选国士以为校事,则是中丞司隶重增一官耳。若如旧选,尹模之奸今复发矣。进退推算,无所用之。昔桑弘羊为汉求利,卜式以为独烹弘羊,天乃可雨。若使政治得失必感天地,臣恐水旱之灾,未必非校事之由也。曹恭公远君子,近小人,国风讬以为刺。卫献公舍大臣,与小臣谋,定姜谓之有罪。纵令校事有益於国,以礼义言之,尚伤大臣之心,况奸回暴露,而复不罢,是衮阙不补,迷而不返也。“於是遂罢校事官。晓迁汝南太守,年四十馀薨。

郭嘉字奉孝,颍川阳翟人也。初,北见袁绍,谓绍谋臣辛评、郭图曰:“夫智者审于量主,故百举百全而功名可立也。袁公徒欲效周公之下士,而未知用人之机。多端寡要,好谋无决,欲与共济天下大难,定霸王之业,难矣!”於是遂去之。先是时,颍川戏志才,筹画士也,太祖甚器之。早卒。太祖与荀彧书曰:“自志才亡后,莫可与计事者。汝、颍固多奇士,谁可以继之?”彧荐嘉。召见,论天下事。太祖曰:“使孤成大业者,必此人也。”嘉出,亦喜曰:“真吾主也。”表为司空军祭酒。

征吕布,三战破之,布退固守。时士卒疲倦,太祖欲引军还,嘉说太祖急攻之,遂禽布。语在荀攸传。

孙策转斗千里,尽有江东,闻太祖与袁绍相持於官渡,将渡江北袭许。众闻皆惧,嘉料之曰:“策新并江东,所诛皆英豪雄杰,能得人死力者也。然策轻而无备,虽有百万之众,无异於独行中原也。若刺客伏起,一人之敌耳。以吾观之,必死於匹夫之手。”策临江未济,果为许贡客所杀。

从破袁绍,绍死,又从讨谭、尚于黎阳,连战数克。诸将欲乘胜遂攻之,嘉曰:“袁绍爱此二子,莫適立也。有郭图、逢纪为之谋臣,必交斗其间,还相离也。急之则相持,缓之而后争心生。不如南向荆州若征刘表者,以待其变;变成而后击之,可一举定也。”太祖曰:“善。”乃南征。军至西平,谭、尚果争冀州。谭为尚军所败,走保平原,遣辛毗乞降。太祖还救之,遂从定邺。又从攻谭於南皮,冀州平。封嘉洧阳亭侯。

太祖将征袁尚及三郡乌丸,诸下多惧刘表使刘备袭许以讨太祖,嘉曰:“公虽威震天下,胡恃其远,必不设备。因其无备,卒然击之,可破灭也。且袁绍有恩于民夷,而尚兄弟生存。今四州之民,徒以威附,德施未加,舍而南征,尚因乌丸之资,招其死主之臣,胡人一动,民夷俱应,以生蹋顿之心,成觊觎之计,恐青、冀非己之有也。表,坐谈客耳,自知才不足以御备,重任之则恐不能制,轻任之则备不为用,虽虚国远征,公无忧矣。”太祖遂行。至易,嘉言曰:“兵贵神速。今千里袭人,辎重多,难以趣利,且彼闻之,必为备;不如留辎重,轻兵兼道以出,掩其不意。”太祖乃密出卢龙塞,直指单于庭。虏卒闻太祖至,惶怖合战。大破之,斩蹋顿及名王已下。尚及兄熙走辽东。

嘉深通有算略,达於事情。太祖曰:“唯奉孝为能知孤意。”年三十八,自柳城还,疾笃,太祖问疾者交错。及薨,临其丧,哀甚,谓荀攸等曰:“诸君年皆孤辈也,唯奉孝最少。天下事竟,欲以后事属之,而中年夭折,命也夫!”乃表曰:“军祭酒郭嘉,自从征伐,十有一年。每有大议,临敌制变。臣策未决,嘉辄成之。平定天下,谋功为高。不幸短命,事业未终。追思嘉勋,实不可忘。可增邑八百户,并前千户。”谥曰贞侯。子奕嗣。

后太祖征荆州还,於巴丘遇疾疫,烧船,叹曰:“郭奉孝在,不使孤至此。”初,陈群非嘉不治行检,数廷诉嘉,嘉意自若。太祖愈益重之,然以群能持正,亦悦焉。奕为太子文学,早薨。子深嗣。深薨,子猎嗣。

董昭字公仁,济阴定陶人也。举孝廉,除癭陶长、柏人令,袁绍以为参军事。绍逆公孙瓒于界桥,钜鹿太守李邵及郡冠盖,以瓒兵强,皆欲属瓒。绍闻之,使昭领钜鹿。问:“御以何术?”对曰:“一人之微,不能消众谋,欲诱致其心,唱与同议,及得其情,乃当权以制之耳。计在临时,未可得言。”时郡右姓孙伉等数十人专为谋主,惊动吏民。昭至郡,伪作绍檄告郡云:“得贼罗候安平张吉辞,当攻钜鹿,贼故孝廉孙伉等为应,檄到收行军法,恶止其身,妻子勿坐。”昭案檄告令,皆即斩之。一郡惶恐,乃以次安慰,遂皆平集。事讫白绍,绍称善。会魏郡太守栗攀为兵所害,绍以昭领魏郡太守。时郡界大乱,贼以万数,遣使往来,交易市买。昭厚待之,因用为间,乘虚掩讨,辄大克破。二日之中,羽檄三至。

昭弟访,在张邈军中。邈与绍有隙,绍受谗将致罪於昭。昭欲诣汉献帝,至河内,为张杨所留。因杨上还印绶,拜骑都尉。时太祖领兖州,遣使诣杨,欲令假涂西至长安,杨不听。昭说杨曰:“袁、曹虽为一家,势不久群。曹今虽弱,然实天下之英雄也,当故结之。况今有缘,宜通其上事,并表荐之;若事有成,永为深分。”杨於是通太祖上事,表荐太祖。昭为太祖作书与长安诸将李傕、郭汜等,各随轻重致殷勤。杨亦遣使诣太祖。太祖遗杨犬马金帛,遂与西方往来。天子在安邑,昭从河内往,诏拜议郎。

建安元年,太祖定黄巾于许,遣使诣河东。会天子还洛阳,韩暹、杨奉、董承及杨各违戾不和。昭以奉兵马最强而少党援,作太祖书与奉曰:“吾与将军闻名慕义,便推赤心。今将军拔万乘之艰难,反之旧都,翼佐之功,超世无畴,何其休哉!方今群凶猾夏,四海未宁,神器至重,事在维辅;必须众贤以清王轨,诚非一人所能独建。心腹四支,实相恃赖,一物不备,则有阙焉。将军当为内主,吾为外援。今吾有粮,将军有兵,有无相通,足以相济,死生契阔,相与共之。”奉得书喜悦,语诸将军曰:“兖州诸军近在许耳,有兵有粮,国家所当依仰也。”遂共表太祖为镇东将军,袭父爵费亭侯;昭迁符节令。

太祖朝天子於洛阳,引昭并坐,问曰:“今孤来此,当施何计?”昭曰:“将军兴义兵以诛暴乱,入朝天子,辅翼王室,此五伯之功也。此下诸将,人殊意异,未必服从,今留匡弼,事势不便,惟有移驾幸许耳。然朝廷播越,新还旧京,远近跂望,冀一朝获安。今复徙驾,不厌众心。夫行非常之事,乃有非常之功,愿将军算其多者。”太祖曰:“此孤本志也。杨奉近在梁耳,闻其兵精,得无为孤累乎?”昭曰:“奉少党援,将独委质。镇东、费亭之事,皆奉所定,又闻书命申束,足以见信。宜时遣使厚遗答谢,以安其意。说'京都无粮,欲车驾暂幸鲁阳,鲁阳近许,转运稍易,可无县乏之忧'。奉为人勇而寡虑,必不见疑,比使往来,足以定计。奉何能为累!”太祖曰:“善。”即遣使诣奉。徙大驾至许。奉由是失望,与韩暹等到定陵钞暴。太祖不应,密往攻其梁营,降诛即定。奉、暹失众,东降袁术。三年,昭迁河南尹。时张杨为其将杨丑所杀,杨长史薛洪、河内太守缪尚城守待绍救。太祖令昭单身入城,告喻洪、尚等,即日举众降。以昭为冀州牧。

太祖令刘备拒袁术,昭曰:“备勇而志大,关羽、张飞为之羽翼,恐备之心未可得论也!”太祖曰:“吾已许之矣。”备到下邳,杀徐州刺史车胄,反。太祖自征备,徙昭为徐州牧。袁绍遣将颜良攻东郡,又徙昭为魏郡太守,从讨良。良死后,进围邺城。袁绍同族春卿为魏郡太守,在城中,其父元长在扬州,太祖遣人迎之。昭书与春卿曰:“盖闻孝者不背亲以要利,仁者不忘君以徇私,志士不探乱以徼幸,智者不诡道以自危。足下大君,昔避内难,南游百越,非疏骨肉,乐彼吴会,智者深识,独或宜然。曹公愍其守志清恪,离群寡俦,故特遣使江东,或迎或送,今将至矣。就令足下处偏平之地,依德义之主,居有泰山之固,身为乔松之偶,以义言之,犹宜背彼向此,舍民趣父也。且邾仪父始与隐公盟,鲁人嘉之,而不书爵,然则王所未命,爵尊不成,春秋之义也。况足下今日之所讬者乃危乱之国,所受者乃矫诬之命乎?苟不逞之与群,而厥父之不恤,不可以言孝。忘祖宗所居之本朝,安非正之奸职,难可以言忠。忠孝并替,难以言智。又足下昔日为曹公所礼辟,夫戚族人而疏所生,内所寓而外王室,怀邪禄而叛知己,远福祚而近危亡,弃明义而收大耻,不亦可惜邪!若能翻然易节,奉帝养父,委身曹公,忠孝不坠,荣名彰矣。宜深留计,早决良图。“邺既定,以昭为谏议大夫。后袁尚依乌丸蹋顿,太祖将征之。患军粮难致,凿平虏、泉州二渠入海通运,昭所建也。太祖表封千秋亭侯,转拜司空军祭酒。

后昭建议:“宜脩古建封五等。”太祖曰:“建设五等者,圣人也,又非人臣所制,吾何以堪之?”昭曰:“自古以来,人臣匡世,未有今日之功。有今日之功,未有久处人臣之势者也。今明公耻有惭德而未尽善,乐保名节而无大责,德美过於伊、周,此至德之所极也。然太甲、成王未必可遭,今民难化,甚於殷、周,处大臣之势,使人以大事疑己,诚不可不重虑也。明公虽迈威德,明法术,而不定其基,为万世计,犹未至也。定基之本,在地与人,宜稍建立,以自藩卫。明公忠节颖露,天威在颜,耿弇床下之言,朱英无妄之论,不得过耳。昭受恩非凡,不敢不陈。”后太祖遂受魏公、魏王之号,皆昭所创。

及关羽围曹仁於樊,孙权遣使辞以“遣兵西上,欲掩取羽。江陵、公安累重,羽失二城,必自奔走,樊军之围,不救自解。乞密不漏,令羽有备。”太祖诘群臣,群臣咸言宜当密之。昭曰:“军事尚权,期於合宜。宜应权以密,而内露之。羽闻权上,若还自护,围则速解,便获其利。可使两贼相对衔持,坐待其弊。秘而不露,使权得志,非计之上。又,围中将吏不知有救,计粮怖惧,傥有他意,为难不小。露之为便。且羽为人强梁,自恃二城守固,必不速退。”太祖曰:“善。”即敕救将徐晃以权书射著围里及羽屯中,围里闻之,志气百倍。羽果犹豫。权军至,得其二城,羽乃破败。

文帝即王位,拜昭将作大匠。及践阼,迁大鸿胪,进封右乡侯。二年,分邑百户,赐昭弟访爵关内侯,徙昭为侍中。三年,征东大将军曹休临江在洞浦口,自表:“愿将锐卒虎步江南,因敌取资,事必克捷;若其无臣,不须为念。”帝恐休便渡江,驿马诏止。时昭侍侧,因曰:“窃见陛下有忧色,独以休济江故乎?今者渡江,人情所难,就休有此志,势不独行,当须诸将。臧霸等既富且贵,无复他望,但欲终其天年,保守禄祚而已,何肯乘危自投死地,以求徼幸?苟霸等不进,休意自沮。臣恐陛下虽有敕渡之诏,犹必沉吟,未便从命也。”是后无几,暴风吹贼船,悉诣休等营下,斩首获生,贼遂迸散。诏敕诸军促渡。军未时进,贼救船遂至。

大驾幸宛,征南大将军夏侯尚等攻江陵,未拔。时江水浅狭,尚欲乘船将步骑入渚中安屯,作浮桥,南北往来,议者多以为城必可拔。昭上疏曰:“武皇帝智勇过人,而用兵畏敌,不敢轻之若此也。夫兵好进恶退,常然之数。平地无险,犹尚艰难,就当深入,还道宜利,兵有进退,不可如意。今屯渚中,至深也;浮桥而济,至危也;一道而行,至狭也:三者兵家所忌,而今行之。贼频攻桥,误有漏失,渚中精锐,非魏之有,将转化为吴矣。臣私慼之,忘寝与食,而议者怡然不以为忧,岂不惑哉!加江水向长,一旦暴增,何以防御?就不破贼,尚当自完。奈何乘危,不以为惧?事将危矣,惟陛下察之!”帝悟昭言,即诏尚等促出。贼两头并前,官兵一道引去,不时得泄,将军石建、高迁仅得自免。军出旬日,江水暴长。帝曰:“君论此事,何其审也!正使张、陈当之,何以复加。”五年,徙封成都乡侯,拜太常。其年,徙光禄大夫、给事中。从大驾东征,七年还,拜太仆。明帝即位,进爵乐平侯,邑千户,转卫尉。分邑百户,赐一子爵关内侯。

太和四年,行司徒事,六年,拜真。昭上疏陈末流之弊曰:“凡有天下者,莫不贵尚敦朴忠信之士,深疾虚伪不真之人者,以其毁教乱治,败俗伤化也。近魏讽则伏诛建安之末,曹伟则斩戮黄初之始。伏惟前后圣诏,深疾浮伪,欲以破散邪党,常用切齿;而执法之吏皆畏其权势,莫能纠擿,毁坏风俗,侵欲滋甚。窃见当今年少,不复以学问为本,专更以交游为业;国士不以孝悌清脩为首,乃以趋势游利为先。合党连群,互相褒叹,以毁訾为罚戮,用党誉为爵赏,附己者则叹之盈言,不附者则为作瑕衅。至乃相谓'今世何忧不度邪,但求人道不勤,罗之不博耳;又何患其不知己矣,但当吞之以药而柔调耳。'又闻或有使奴客名作在职家人,冒之出入,往来禁奥,交通书疏,有所探问。凡此诸事,皆法之所不取,刑之所不赦,虽讽、伟之罪,无以加也。”帝於是发切诏,斥免诸葛诞、邓飏等。昭年八十一薨,谥曰定侯。子胄嗣。胄历位郡守、九卿。

刘晔字子扬,淮南成德人,汉光武子阜陵王延后也。父普,母脩,产涣及晔。涣九岁,晔七岁,而母病困。临终,戒涣、晔以“普之侍人,有谄害之性。身死之后,惧必乱家。汝长大能除之,则吾无恨矣。”晔年十三,谓兄涣曰:“亡母之言,可以行矣。”涣曰:“那可尔!”晔即入室杀侍者,径出拜墓。舍内大惊,白普。普怒,遣人追晔。晔还拜谢曰:“亡母顾命之言,敢受不请擅行之罚。”普心异之,遂不责也。汝南许劭名知人,避地扬州,称晔有佐世之才。

扬士多轻侠狡桀,有郑宝、张多、许乾之属,各拥部曲。宝最骁果,才力过人,一方所惮。欲驱略百姓越赴江表,以晔高族名人,欲强逼晔使唱导此谋。晔时年二十馀,心内忧之,而未有缘。会太祖遣使诣州,有所案问。晔往见,为论事势,要将与归,驻止数日。宝果从数百人赍牛酒来候使,晔令家僮将其众坐中门外,为设酒饭;与宝於内宴饮。密勒健儿,令因行觞而斫宝。宝性不甘酒,视候甚明,觞者不敢发。晔因自引取佩刀斫杀宝,斩其首以令其军,云:“曹公有令,敢有动者,与宝同罪。”众皆惊怖,走还营。营有督将精兵数千,惧其为乱,晔即乘宝马,将家僮数人,诣宝营门,呼其渠帅,喻以祸福,皆叩头开门内晔。晔抚慰安怀,咸悉悦服,推晔为主。晔睹汉室渐微,己为支属,不欲拥兵,遂委其部曲与庐江太守刘勋。勋怪其故,晔曰:“宝无法制,其众素以钞略为利,仆宿无资,而整齐之,必怀怨难久,故相与耳。”时勋兵强于江、淮之间。孙策恶之,遣使卑辞厚币,以书说勋曰:“上缭宗民,数欺下国,忿之有年矣。击之,路不便,愿因大国伐之。上缭甚实,得之可以富国,请出兵为外援。”勋信之,又得策珠宝、葛越,喜悦。外内尽贺,而晔独否。勋问其故,对曰:“上缭虽小,城坚池深,攻难守易,不可旬日而举,则兵疲於外,而国内虚。策乘虚而袭我,则后不能独守。是将军进屈於敌,退无所归。若军必出,祸今至矣。”勋不从。兴兵伐上缭,策果袭其后。勋穷踧,遂奔太祖。

太祖至寿春,时庐江界有山贼陈策,众数万人,临险而守。先时遣偏将致诛,莫能禽克。太祖问群下,可伐与不?咸云:“山峻高而谿谷深隘,守易攻难;又无之不足为损,得之不足为益。”晔曰:“策等小竖,因乱赴险,遂相依为强耳,非有爵命威信相伏也。往者偏将资轻,而中国未夷,故策敢据险以守。今天下略定,后伏先诛。夫畏死趋赏,愚知所同,故广武君为韩信画策,谓其威名足以先声后实而服邻国也。岂况明公之德,东征西怨,先开赏募,大兵临之,令宣之日,军门启而虏自溃矣。”太祖笑曰:“卿言近之!”遂遣猛将在前,大军在后,至则克策,如晔所度。太祖还,辟晔为司空仓曹掾。

太祖征张鲁,转晔为主簿。既至汉中,山峻难登,军食颇乏。太祖曰:“此妖妄之国耳,何能为有无?吾军少食,不如速还。”便自引归,令晔督后诸军,使以次出。晔策鲁可克,加粮道不继,虽出,军犹不能皆全,驰白太祖:“不如致攻。“遂进兵,多出弩以射其营。鲁奔走,汉中遂平。晔进曰:“明公以步卒五千,将诛董卓,北破袁绍,南征刘表,九州百郡,十并其八,威震天下,势慑海外。今举汉中,蜀人望风,破胆失守,推此而前,蜀可传檄而定。刘备,人杰也,有度而迟,得蜀日浅,蜀人未恃也。今破汉中,蜀人震恐,其势自倾。以公之神明,因其倾而压之,无不克也。若小缓之,诸葛亮明於治而为相,关羽、张飞勇冠三军而为将,蜀民既定,据险守要,则不可犯矣。今不取,必为后忧。”太祖不从,傅子曰:居七日,蜀降者说:“蜀中一日数十惊,备虽斩之而不能安也。”太祖延问晔曰:“今尚可击不?”晔曰:“今已小定,未可击也。”大军遂还。晔自汉中还,为行军长史,兼领军。延康元年,蜀将孟达率众降。达有容止才观,文帝甚器爱之,使达为新城太守,加散骑常侍。晔以为“达有苟得之心,而恃才好术,必不能感恩怀义。新城与吴、蜀接连,若有变态,为国生患。”文帝竟不易,后达终于叛败。傅子曰:初,太祖时,魏讽有重名,自卿相以下皆倾心交之。其后孟达去刘备归文帝,论者多称有乐毅之量。晔一见讽、达而皆云必反,卒如其言。

黄初元年,以晔为侍中,赐爵关内侯。诏问群臣令料刘备当为关羽出报吴不。众议咸云:“蜀,小国耳,名将唯羽。羽死军破,国内忧惧,无缘复出。”晔独曰:“蜀虽狭弱,而备之谋欲以威武自强,势必用众以示其有馀。且关羽与备,义为君臣,恩犹父子;羽死不能为兴军报敌,於终始之分不足。”后备果出兵击吴。吴悉国应之,而遣使称藩。朝臣皆贺,独晔曰:“吴绝在江、汉之表,无内臣之心久矣。陛下虽齐德有虞,然丑虏之性,未有所感。因难求臣,必难信也。彼必外迫内困,然后发此使耳,可因其穷,袭而取之。夫一日纵敌,数世之患,不可不察也。”备军败退,吴礼敬转废,帝欲兴众伐之,晔以为“彼新得志,上下齐心,而阻带江湖,必难仓卒。”帝不听。五年,幸广陵泗口,命荆、扬州诸军并进。会群臣,问:“权当自来不?”咸曰:“陛下亲征,权恐怖,必举国而应。又不敢以大众委之臣下,必自将而来。”晔曰:“彼谓陛下欲以万乘之重牵己,而超越江湖者在於别将,必勒兵待事,未有进退也。”大驾停住积日,权果不至,帝乃旋师。云:“卿策之是也。当念为吾灭二贼,不可但知其情而已。”

明帝即位,进爵东亭侯,邑三百户。诏曰:“尊严祖考,所以崇孝表行也;追本敬始,所以笃教流化也。是以成汤、文、武,实造商、周,诗、书之义,追尊稷、契,歌颂有娀、姜嫄之事,明盛德之源流,受命所由兴也。自我魏室之承天序,既发迹於高皇、太皇帝,而功隆于武皇、文皇帝。至于高皇之父处士君,潜脩德让,行动神明,斯乃乾坤所福飨,光灵所从来也。而精神幽远,号称罔记,非所谓崇孝重本也。其令公卿已下,会议号谥。”晔议曰:“圣帝孝孙之欲褒崇先祖,诚无量已。然亲疏之数,远近之降,盖有礼纪,所以割断私情,克成公法,为万世式也。周王所以上祖后稷者,以其佐唐有功,名在祀典故也。至於汉氏之初,追谥之义,不过其父。上比周室,则大魏发迹自高皇始;下论汉氏,则追谥之礼不及其祖。此诚往代之成法,当今之明义也。陛下孝思中发,诚无已已,然君举必书,所以慎於礼制也。以为追尊之义,宜齐高皇而已。”尚书卫臻与晔议同,事遂施行。辽东太守公孙渊夺叔父位,擅自立,遣使表状。晔以为公孙氏汉时所用,遂世官相承,水则由海,陆则阻山,故胡夷绝远难制,而世权日久。今若不诛,后必生患。若怀贰阻兵,然后致诛,於事为难。不如因其新立,有党有仇,先其不意,以兵临之,开设赏募,可不劳师而定也。后渊竟反。

晔在朝,略不交接时人。或问其故,晔答曰:“魏室即阼尚新,智者知命,俗或未咸。仆在汉为支叶,於魏备腹心,寡偶少徒,於宜未失也。”太和六年,以疾拜太中大夫。有间,为大鸿胪,在位二年逊位,复为太中大夫,薨。谥曰景侯。子宇嗣。少子陶,亦高才而薄行,官至平原太守。

蒋济字子通,楚国平阿人也。仕郡计吏、州别驾。建安十三年,孙权率众围合肥。时大军征荆州,遇疾疫,唯遣将军张喜单将千骑,过领汝南兵以解围,颇复疾疫。济乃密白刺史伪得喜书,云步骑四万已到雩娄,遣主簿迎喜。三部使赍书语城中守将,一部得入城,二部为贼所得。权信之,遽烧围走,城用得全。明年使於谯,太祖问济曰:“昔孤与袁本初对官渡,徙燕、白马民,民不得走,贼亦不敢钞。今欲徙淮南民,何如?”济对曰:“是时兵弱贼强,不徙必失之。自破袁绍,北拔柳城,南向江、汉,荆州交臂,威震天下,民无他志。然百姓怀土,实不乐徙,惧必不安。”太祖不从,而江、淮间十馀万众,皆惊走吴。后济使诣邺,太祖迎见大笑曰:“本但欲使避贼,乃更驱尽之。”拜济丹阳太守。大军南征还,以温恢为扬州刺史,济为别驾。令曰:“季子为臣,吴宜有君。今君还州,吾无忧矣。”民有诬告济为谋叛主率者,太祖闻之,指前令与左将军于禁、沛相封仁等曰:“蒋济宁有此事!有此事,吾为不知人也。此必愚民乐乱,妄引之耳。”促理出之。辟为丞相主簿西曹属。令曰:“舜举皋陶,不仁者远;臧否得中,望于贤属矣。”关羽围樊、襄阳。太祖以汉帝在许,近贼,欲徙都。司马宣王及济说太祖曰:“于禁等为水所没,非战攻之失,於国家大计未足有损。刘备、孙权,外亲内疏,关羽得志,权必不愿也。可遣人劝蹑其后,许割江南以封权,则樊围自解。”太祖如其言。权闻之,即引兵西袭公安、江陵。羽遂见禽。

文帝即王位,转为相国长史。及践阼,出为东中郎将。济请留,诏曰:“高祖歌曰‘安得猛士守四方’!天下未宁,要须良臣以镇边境。如其无事,乃还鸣玉,未为后也。”济上万机论,帝善之。入为散骑常侍。时有诏,诏征南将军夏侯尚曰:“卿腹心重将,特当任使。恩施足死,惠爱可怀。作威作福,杀人活人。”尚以示济。济既至,帝问曰;“卿所闻见天下风教何如?“济对曰:“未有他善,但见亡国之语耳。”帝忿然作色而问其故,济具以答,因曰:“夫‘作威作福’,书之明诫。‘天子无戏言’,古人所慎。惟陛下察之!”於是帝意解,遣追取前诏。黄初三年,与大司马曹仁征吴,济别袭羡谿。仁欲攻濡须洲中,济曰:“贼据西岸,列船上流,而兵入洲中,是为自内地狱,危亡之道也。”仁不从,果败。仁薨,复以济为东中郎将,代领其兵。诏曰:“卿兼资文武,志节慷慨,常有超越江湖吞吴会之志,故复授将率之任。”顷之,徵为尚书。车驾幸广陵,济表水道难通,又上三州论以讽帝。帝不从,於是战船数千皆滞不得行。议者欲就留兵屯田,济以为东近湖,北临淮,若水盛时,贼易为寇,不可安屯。帝从之,车驾即发。还到精湖,水稍尽,尽留船付济。船本历適数百里中,济更凿地作四五道,蹴船令聚;豫作土豚遏断湖水,皆引后船,一时开遏入淮中。帝还洛阳,谓济曰:“事不可不晓。吾前决谓分半烧船于山阳池中,卿於后致之,略与吾俱至谯。又每得所陈,实入吾意。自今讨贼计画,善思论之。”

明帝即位,赐爵关内侯。大司马曹休帅军向皖,济表以为“深入虏地,与权精兵对,而朱然等在上流,乘休后,臣未见其利也。”军至皖,吴出兵安陆,济又上疏曰:“今贼示形於西,必欲并兵图东,宜急诏诸军往救之。”会休军已败,尽弃器仗辎重退还。吴欲塞夹石,遇救兵至,是以官军得不没。迁为中护军。时中书监、令号为专任,济上疏曰:“大臣太重者国危,左右太亲者身蔽,古之至戒也。往者大臣秉事,外内扇动。陛下卓然自览万机,莫不祗肃。夫大臣非不忠也,然威权在下,则众心慢上,势之常也。陛下既已察之於大臣,愿无忘於左右。左右忠正远虑,未必贤於大臣,至於便辟取合,或能工之。今外所言,辄云中书,虽使恭慎不敢外交,但有此名,犹惑世俗。况实握事要,日在目前,傥因疲倦之间有所割制,众臣见其能推移於事,即亦因时而向之。一有此端,因当内设自完,以此众语,私招所交,为之内援。若此,臧否毁誉,必有所兴,功负赏罚,必有所易;直道而上者或壅,曲附左右者反达。因微而入,缘形而出,意所狎信,不复猜觉。此宜圣智所当早闻,外以经意,则形际自见。或恐朝臣畏言不合而受左右之怨,莫適以闻。臣窃亮陛下潜神默思,公听并观,若事有未尽於理而物有未周於用,将改曲易调,远与黄、唐角功,近昭武、文之迹,岂近习而已哉!然人君犹不可悉天下事以適己明,当有所付。三官任一臣,非周公旦之忠,又非管夷吾之公,则有弄机败官之弊。当今柱石之士虽少,至于行称一州,智效一官,忠信竭命,各奉其职,可并驱策,不使圣明之朝有专吏之名也。”诏曰:“夫骨鲠之臣,人主之所仗也。济才兼文武,服勤尽节,每军国大事,辄有奏议,忠诚奋发,吾甚壮之。”就迁为护军将军,加散骑常侍

景初中,外勤征役,内务宫室,怨旷者多,而年谷饥俭。济上疏曰:“陛下方当恢崇前绪,光济遗业,诚未得高枕而治也。今虽有十二州,至于民数,不过汉时一大郡。二贼未诛,宿兵边陲,且耕且战,怨旷积年。宗庙宫室,百事草创,农桑者少,衣食者多,今其所急,唯当息耗百姓,不至甚弊。弊攰之民,傥有水旱,百万之众,不为国用。凡使民必须农隙,不夺其时。夫欲大兴功之君,先料其民力而燠休之。句践养胎以待用,昭王恤病以雪仇,故能以弱燕服强齐,羸越灭劲吴。今二敌不攻不灭,不事即侵,当身不除,百世之责也。以陛下圣明神武之略,舍其缓者,专心讨贼,臣以为无难矣。又欢娱之耽,害于精爽;神太用则竭,形太劳则弊。愿大简贤妙,足以充'百斯男'者。其冗散未齿,且悉分出,务在清静”诏曰:“微护军,吾弗闻斯言也。”

齐王即位,徙为领军将军,进爵昌陵亭侯,迁太尉。初,侍中高堂隆论郊祀事,以魏为舜后,推舜配天。济以为舜本姓妫,其苗曰田,非曹之先,著文以追诘隆。是时,曹爽专政,丁谧、邓飏等轻改法度。会有日蚀变,诏群臣问其得失,济上疏曰:“昔大舜佐治,戒在比周;周公辅政,慎于其朋;齐侯问灾,晏婴对以布惠;鲁君问异,臧孙答以缓役。应天塞变,乃实人事。今二贼未灭,将士暴露已数十年,男女怨旷,百姓贫苦。夫为国法度,惟命世大才,乃能张其纲维以垂于后,岂中下之吏所宜改易哉?终无益于治,適足伤民,望宜使文武之臣各守其职,率以清平,则和气祥瑞可感而致也。”以随太傅司马宣王屯洛水浮桥,诛曹爽等,进封都乡侯,邑七百户。济上疏曰:“臣忝宠上司,而爽敢苞藏祸心,此臣之无任也。太傅奋独断之策,陛下明其忠节,罪人伏诛,社稷之福也。夫封宠庆赏,必加有功。今论谋则臣不先知,语战则非臣所率,而上失其制,下受其弊。臣备宰司,民所具瞻,诚恐冒赏之渐自此而兴,推让之风由此而废。”固辞,不许。子秀嗣。秀薨,子凯嗣。咸熙中,开建五等,以济著勋前朝,改封凯为下蔡子。

刘放字子弃,涿郡人,汉广阳顺王子西乡侯宏后也。历郡纲纪,举孝廉。遭世大乱,时渔阳王松据其土,放往依之。太祖克冀州,放说松曰:“往者董卓作逆,英雄并起,阻兵擅命,人自封殖,惟曹公能拔拯危乱,翼戴天子,奉辞伐罪,所向必克。以二袁之强,守则淮南冰消,战则官渡大败;乘胜席卷,将清河朔,威刑既合,大势以见。速至者渐福,后服者先亡,此乃不俟终日驰骛之时也。昔黥布弃南面之尊,仗剑归汉,诚识废兴之理,审去就之分也。将军宜投身委命,厚自结纳。”松然之。会太祖讨袁谭於南皮,以书招松,松举雍奴、泉州、安次以附之。放为松答太祖书,其文甚丽。太祖既善之,又闻其说,由是遂辟放。建安十年,与松俱至。太祖大悦,谓放曰:“昔班彪依窦融而有河西之功,今一何相似也!”乃以放参司空军事,历主簿记室,出为郃阳、祋祤、赞令。

魏国既建,与太原孙资俱为秘书郎。先是,资亦历县令,参丞相军事。文帝即位,放、资转为左右丞。数月,放徙为令。黄初初,改秘书为中书,以放为监,资为令,各加给事中;放赐爵关内侯,资为关中侯,遂掌机密。三年,放进爵魏寿亭侯,资关内侯。明帝即位,尤见宠任,同加散骑常侍;进放爵西乡侯,资乐阳亭侯。太和末,吴遣将周贺浮海诣辽东,招诱公孙渊。帝欲邀讨之,朝议多以为不可。惟资决行策,果大破之,进爵左乡侯。放善为书檄,三祖诏命有所招喻,多放所为。青龙初,孙权与诸葛亮连和,欲俱出为寇。边候得权书,放乃改易其辞,往往换其本文而傅合之,与征东将军满宠,若欲归化,封以示亮。亮腾与吴大将步骘等,骘等以见权。权惧亮自疑,深自解说。是岁,俱加侍中、光禄大夫。景初二年,辽东平定,以参谋之功,各进爵,封本县,放方城侯,资中都侯。

其年,帝寝疾,欲以燕王宇为大将军,及领军将军夏侯献、武卫将军曹爽、屯骑校尉曹肇、骁骑将军秦朗共辅政。宇性恭良,陈诚固辞。帝引见放、资,入卧内,问曰:“燕王正尔为?”放、资对曰:“燕王实自知不堪大任故耳。”帝曰:“曹爽可代宇不?“放、资因赞成之。又深陈宜速召太尉司马宣王,以纲维皇室。帝纳其言,即以黄纸授放作诏。放、资既出,帝意复变,诏止宣王勿使来。寻更见放、资曰:“我自召太尉,而曹肇等反使吾止之,几败吾事!”命更为诏,帝独召爽与放、资俱受诏命,遂免宇、献、肇、朗官。太尉亦至,登床受诏,然后帝崩。齐王即位,以放、资决定大谋,增邑三百,放并前千一百,资千户;封爱子一人亭侯,次子骑都尉,馀子皆郎中。正始元年,更加放左光禄大夫,资右光禄大夫,金印紫绶,仪同三司。六年,放转骠骑,资卫将军,领监、令如故。七年,复封子一人亭侯,各年老逊位,以列侯朝朔望,位特进。曹爽诛后,复以资为侍中,领中书令。嘉平二年,放薨,谥曰敬侯。子正嗣。资复逊位归第,就拜骠骑将军,转侍中,特进如故。三年薨,谥曰贞侯。子宏嗣。

放才计优资,而自脩不如也。放、资既善承顺主上,又未尝显言得失,抑辛毗而助王思,以是获讥於世。然时因群臣谏诤,扶赞其义,并时密陈损益,不专导谀言云。及咸熙中,开建五等,以放、资著勋前朝,改封正方城子,宏离石子。

评曰:程昱、郭嘉、董昭、刘晔、蒋济才策谋略,世之奇士,虽清治德业,殊於荀攸,而筹画所料,是其伦也。刘放文翰,孙资勤慎,并管喉舌,权闻当时,雅亮非体,是故讥谀之声,每过其实矣。

\part{魏书十五}
\chapter{刘司马梁张温贾传第十五}

刘馥字元颖,沛国相人也。避乱扬州,建安初,说袁术将戚寄、秦翊,使率众与俱诣太祖。太祖悦之,司徒辟为掾。后孙策所置庐江太守李述攻杀扬州刺史严象。庐江梅乾、雷绪、陈兰等聚众数万在江、淮间,郡县残破。太祖方有袁绍之难,谓馥可任以东南之事,遂表为扬州刺史。

馥既受命,单马造合肥空城,建立州治,南怀绪等,皆安集之,贡献相继。数年中恩化大行,百姓乐其政,流民越江山而归者以万数。於是聚诸生,立学校,广屯田,兴治芍陂及茄陂、七门、吴塘诸堨以溉稻田,官民有畜。又高为城垒,多积木石,编作草苫数千万枚,益贮鱼膏数千斛,为战守备。

建安十三年卒。孙权率十万众攻围合肥城百馀日,时天连雨,城欲崩,於是以苫蓑覆之,夜然脂照城外,视贼所作而为备,贼以破走。扬州士民益追思之,以为虽董安于之守晋阳,不能过也。及陂塘之利,至今为用。

馥子靖,黄初中从黄门侍郎迁庐江太守,诏曰:“卿父昔为彼州,今卿复据此郡,可谓克负荷者也。”转在河内,迁尚书,赐爵关内侯,出为河南尹。散骑常侍应璩书与靖曰:“入作纳言,出临京任。富民之术,日引月长。藩落高峻,绝穿窬之心。五种别出,远水火之灾。农器必具,无失时之阙。蚕麦有苫备之用,无雨湿之虞。封符指期,无流连之吏。鳏寡孤独,蒙廪振之实。加之以明擿幽微,重之以秉宪不挠;有司供承王命,百里垂拱仰办。虽昔赵、张、三王之治,未足以方也。”靖为政类如此。初虽如碎密,终於百姓便之,有馥遗风。母丧去官,后为大司农卫尉,进封广陆亭侯,邑三百户。上疏陈儒训之本曰:“夫学者,治乱之轨仪,圣人之大教也。自黄初以来,崇立太学二十馀年,而寡有成者,盖由博士选轻,诸生避役,高门子弟,耻非其伦,故无学者。虽有其名而无其人,虽设其教而无其功。宜高选博士,取行为人表,经任人师者,掌教国子。依遵古法,使二千石以上子孙,年从十五,皆入太学。明制黜陟荣辱之路;其经明行修者,则进之以崇德;荒教废业者,则退之以惩恶;举善而教不能则劝,浮华交游,不禁自息矣。阐弘大化,以绥未宾;六合承风,远人来格。此圣人之教,致治之本也。”后迁镇北将军,假节都督河北诸军事。靖以为“经常之大法,莫善於守防,使民夷有别”。遂开拓边守,屯据险要。又修广戾陵渠大堨,水溉灌蓟南北,三更种稻,边民利之。嘉平六年薨,追赠征北将军,进封建成乡侯,谥曰景侯。子熙嗣。

司马朗字伯达,河内温人也。九岁,人有道其父字者,朗曰:“慢人亲者,不敬其亲者也。”客谢之。十二,试经为童子郎,监试者以其身体壮大,疑朗匿年,劾问。朗曰:“朗之内外,累世长大,朗虽稚弱,无仰高之风,损年以求早成,非志所为也。”监试者异之。后关东兵起,故冀州刺史李邵家居野王,近山险,欲徙居温。朗谓邵曰:“唇齿之喻,岂唯虞、虢,温与野王即是也;今去彼而居此,是为避朝亡之期耳。且君,国人之望也,今寇未至而先徙,带山之县必骇,是摇动民之心而开奸宄之原也,窃为郡内忧之。”邵不从。边山之民果乱,内徙,或为寇钞。

是时董卓迁天子都长安,卓因留洛阳。朗父防为治书御史,当徙西,以四方云扰,乃遣朗将家属还本县。或有告朗欲逃亡者,执以诣卓,卓谓朗曰:“卿与吾亡儿同岁,几大相负!”朗因曰:“明公以高世之德,遭阳九之会,清除群秽,广举贤士,此诚虚心垂虑,将兴至治也。威德以隆,功业以著,而兵难日起,州郡鼎沸,郊境之内,民不安业,捐弃居产,流亡藏窜。虽四关设禁,重加刑戮,犹不绝息,此朗之所以於邑也。愿明公监观往事,少加三思,即荣名并於日月,伊、周不足侔也。”卓曰:“吾亦悟之,卿言有意!”

朗知卓必亡,恐见留,即散财物以赂遗卓用事者,求归乡里。到谓父老曰;“董卓悖逆,为天下所仇,此忠臣义士奋发之时也。郡与京都境壤相接,洛东有成皋,北界大河,天下兴义兵者若未得进,其势必停於此。此乃四分五裂战争之地,难以自安,不如及道路尚通,举宗东到黎阳。黎阳有营兵,赵威孙乡里旧婚,为监营谒者,统兵马,足以为主。若后有变,徐复观望未晚也。”父老恋旧,莫有从者,惟同县赵咨,将家属俱与朗往焉。后数月,关东诸州郡起兵,众数十万,皆集荥阳及河内。诸将不能相一,纵兵钞掠,民人死者且半。久之,关东兵散,太祖与吕布相持於濮阳,朗乃将家还温。时岁大饥,人相食,朗收恤宗族,教训诸弟,不为衰世解业。

年二十二,太祖辟为司空掾属,除成皋令,以病去,复为堂阳长。其治务宽惠,不行鞭杖,而民不犯禁。先时,民有徙充都内者,后县调当作船,徙民恐其不办,乃相率私还助之,其见爱如此。迁元城令,入为丞相主簿。朗以为天下土崩之势,由秦灭五等之制,而郡国无蒐狩习战之备故也。今虽五等未可复行,可令州郡并置兵,外备四夷,内威不轨,於策为长。又以为宜复井田。往者以民各有累世之业,难中夺之,是以至今。今承大乱之后,民人分散,土业无主,皆为公田,宜及此时复之。议虽未施行,然州郡领兵,朗本意也。迁兖州刺史,政化大行,百姓称之。虽在军旅,常粗衣恶食,俭以率下。雅好人伦典籍,乡人李觌等盛得名誉,朗常显贬下之;后觌等败,时人服焉。锺繇、王粲著论云:“非圣人不能致太平。”朗以为“伊、颜之徒虽非圣人,使得数世相承,太平可致”。建安二十二年,与夏侯惇、臧霸等征吴。到居巢,军士大疫,朗躬巡视,致医药。遇疾卒,时年四十七。遗命布衣幅巾,敛以时服,州人追思之。明帝即位,封朗子遗昌武亭侯,邑百户。朗弟孚又以子望继朗后。遗薨,望子洪嗣。

初朗所与俱徙赵咨,官至太常,为世好士。

梁习字子虞,陈郡柘人也,为郡纲纪。太祖为司空,辟召为漳长,累转乘氏、海西、下邳令,所在有治名。还为西曹令史,迁为属。并土新附,习以别部司马领并州刺史。时承高幹荒乱之馀,胡狄在界,张雄跋扈,吏民亡叛,入其部落;兵家拥众,作为寇害,更相扇动,往往棋跱。习到官,诱谕招纳,皆礼召其豪右,稍稍荐举,使诣幕府;豪右已尽,乃次发诸丁强以为义从;又因大军出征,分请以为勇力。吏兵已去之后,稍移其家,前后送邺,凡数万口;其不从命者,兴兵致讨,斩首千数,降附者万计。单于恭顺,名王稽颡,部曲服事供职,同於编户。边境肃清,百姓布野,勤劝农桑,令行禁止。贡达名士,咸显於世,语在常林传。太祖嘉之,赐爵关内侯,更拜为真。长老称咏,以为自所闻识,刺史未有及习者。建安十八年,州并属冀州,更拜议郎、西部都督从事,统属冀州,总故部曲。又使於上党取大材供邺宫室。习表置屯田都尉二人,领客六百夫,於道次耕种菽粟,以给人牛之费。后单于入侍,西北无虞,习之绩也。文帝践阼,复置并州,复为刺史,进封申门亭侯,邑百户;政治常为天下最。太和二年,徵拜大司农。习在州二十馀年,而居处贫穷,无方面珍物,明帝异之,礼赐甚厚。四年,薨,子施嗣。

初,济阴王思与习俱为西曹令史。思因直日白事,失太祖指。太祖大怒,教召主者,将加重辟。时思近出,习代往对,已被收执矣,思乃驰还,自陈己罪,罪应受死。太祖叹习之不言,思之识分,曰:“何意吾军中有二义士乎?”后同时擢为刺史,思领豫州。思亦能吏,然苛碎无大体,官至九卿,封列侯。

张既字德容,冯翊高陵人也。年十六,为郡小吏。后历右职,举孝廉,不行。太祖为司空,辟,未至。举茂才,除新丰令,治为三辅第一。袁尚拒太祖於黎阳,遣所置河东太守郭援、并州刺史高幹及匈奴单于取平阳,发使西与关中诸将合从。司隶校尉锺繇遣既说将军马腾等,既为言利害,腾等从之。腾遣子超将兵万馀人,与繇会击幹、援,大破之,斩援首。幹及单于皆降。其后幹复举并州反。河内张晟众万馀人无所属,寇崤、渑间,河东卫固、弘农张琰各起兵以应之。太祖以既为议郎,参繇军事,使西徵诸将马腾等,皆引兵会击晟等,破之。斩琰、固首,幹奔荆州。封既武始亭侯。太祖将征荆州,而腾等分据关中。太祖复遣既喻腾等,令释部曲求还。腾已许之而更犹豫,既恐为变,乃移诸县促储偫,二千石郊迎。腾不得已,发东。太祖表腾为卫尉,子超为将军,统其众。后超反,既从太祖破超於华阴,西定关右。以既为京兆尹,招怀流民,兴复县邑,百姓怀之。魏国既建,为尚书,出为雍州刺史。太祖谓既曰:“还君本州,可谓衣绣昼行矣。”从征张鲁,别从散关入讨叛氐,收其麦以给军食。鲁降,既说太祖拔汉中民数万户以实长安及三辅。其后与曹洪破吴兰於下辩,又与夏侯渊讨宋建,别攻临洮、狄道,平之。是时,太祖徙民以充河北,陇西、天水、南安民相恐动,扰扰不安,既假三郡人为将吏者休课,使治屋宅,作水碓,民心遂安。太祖将拔汉中守,恐刘备北取武都氐以逼关中,问既。既曰:“可劝使北出就谷以避贼,前至者厚其宠赏,则先者知利,后必慕之。”太祖从其策,乃自到汉中引出诸军,令既之武都,徙氐五万馀落出居扶风、天水界。

是时,武威颜俊、张掖和鸾、酒泉黄华、西平麹演等并举郡反,自号将军,更相攻击。俊遣使送母及子诣太祖为质,求助。太祖问既,既曰:“俊等外假国威,内生傲悖,计定势足,后即反耳。今方事定蜀,且宜两存而斗之,犹卞庄子之刺虎,坐收其毙也。”太祖曰:“善。”岁馀,鸾遂杀俊,武威王祕又杀鸾。是时不置凉州,自三辅拒西域,皆属雍州。文帝即王位,初置凉州,以安定太守邹岐为刺史。张掖张进执郡守举兵拒岐,黄华、麹演各逐故太守,举兵以应之。既进兵为护羌校尉苏则声势,故则得以有功。既进爵都乡侯。凉州卢水胡伊健妓妾、治元多等反,河西大扰。帝忧之,曰:“非既莫能安凉州。”乃召邹岐,以既代之。诏曰:“昔贾复请击郾贼,光武笑曰:'执金吾击郾,吾复何忧?'卿谋略过人,今则其时。以便宜从事,勿复先请。”遣护军夏侯儒、将军费曜等继其后。既至金城,欲渡河,诸将守以为“兵少道险,未可深入”。既曰:“道虽险,非井陉之隘,夷狄乌合,无左车之计,今武威危急,赴之宜速。“遂渡河。贼七千馀骑逆拒军於鹯阴口,既扬声军由鹯阴,乃潜由且次出至武威。胡以为神,引还显美。既已据武威,曜乃至,儒等犹未达。既劳赐将士,欲进军击胡。诸将皆曰:“士卒疲倦,虏众气锐,难与争锋。”既曰:“今军无见粮,当因敌为资。若虏见兵合,退依深山,追之则道险穷饿,兵还则出候寇钞。如此,兵不得解,所谓'一日纵敌,患在数世'也。“遂前军显美。胡骑数千,因大风欲放火烧营,将士皆恐。既夜藏精卒三千人为伏,使参军成公英督千馀骑挑战,敕使阳退。胡果争奔之,因发伏截其后,首尾进击,大破之,斩首获生以万数。帝甚悦,诏曰:“卿逾河历险,以劳击逸,以寡胜众,功过南仲,勤逾吉甫。此勋非但破胡,乃永宁河右,使吾长无西顾之念矣。”徙封西乡侯,增邑二百,并前四百户。

酒泉苏衡反,与羌豪邻戴及丁令胡万馀骑攻边县。既与夏侯儒击破之,衡及邻戴等皆降。遂上疏请与儒治左城,筑鄣塞,置烽侯、邸阁以备胡。西羌恐,率众二万馀落降。其后西平麹光等杀其郡守,诸将欲击之,既曰:“唯光等造反,郡人未必悉同。若便以军临之,吏民羌胡必谓国家不别是非,更使皆相持著,此为虎傅翼也。光等欲以羌胡为援,今先使羌胡钞击,重其赏募,所虏获者皆以畀之。外沮其势,内离其交,必不战而定。”乃檄告谕诸羌,为光等所诖误者原之;能斩贼帅送首者当加封赏。於是光部党斩送光首,其馀咸安堵如故。

既临二州十馀年,政惠著闻,其所礼辟扶风庞延、天水杨阜、安定胡遵、酒泉庞淯、敦煌张恭、周生烈等,终皆有名位。黄初四年薨。诏曰:“昔荀桓子立勋翟土,晋侯赏以千室之邑;冯异输力汉朝,光武封其二子。故凉州刺史张既,能容民畜众,使群羌归土,可谓国之良臣。不幸薨陨,朕甚愍之,其赐小子翁归爵关内侯。”明帝即位,追谥曰肃侯。子缉嗣。

缉以中书郎稍迁东莞太守。嘉平中,女为皇后,徵拜光禄大夫,位特进,封妻向为安城乡君。缉与中书令李丰同谋,诛。语在夏侯玄传。

温恢字曼基,太原祁人也。父恕,为涿郡太守,卒。恢年十五,送丧还归乡里,内足於财。恢曰:“世方乱,安以富为?”一朝尽散,振施宗族。州里高之,比之郇越。举孝廉,为廪丘长,鄢陵、广川令,彭城、鲁相,所在见称。入为丞相主簿,出为扬州刺史。太祖曰:“甚欲使卿在亲近,顾以为不如此州事大。故书云:‘股肱良哉!庶事康哉!’得无当得蒋济为治中邪?“时济见为丹杨太守,乃遣济还州。又语张辽、乐进等曰:“扬州刺史晓达军事,动静与共咨议。”

建安二十四年,孙权攻合肥,是时诸州皆屯戍。恢谓兖州刺史裴潜曰:“此间虽有贼,不足忧,而畏征南方有变。今水生而子孝县军,无有远备。关羽骁锐,乘利而进,必将为患。”於是有樊城之事。诏书召潜及豫州刺史吕贡等,潜等缓之。恢密语潜曰:“此必襄阳之急欲赴之也。所以不为急会者,不欲惊动远众。一二日必有密书促卿进道,张辽等又将被召。辽等素知王意,后召前至,卿受其责矣!”潜受其言,置辎重,更为轻装速发,果被促令。辽等寻各见召,如恢所策。

文帝践阼,以恢为侍中,出为魏郡太守。数年,迁凉州刺史,持节领护羌校尉。道病卒,时年四十五。诏曰:“恢有柱石之质,服事先帝,功勤明著。及为朕执事,忠於王室,故授之以万里之任,任之以一方之事。如何不遂,吾甚愍之!”赐恢子生爵关内侯。生早卒,爵绝。

恢卒后,汝南孟建为凉州刺史,有治名,官至征东将军。

贾逵字梁道,河东襄陵人也。自为儿童,戏弄常设部伍,祖父习异之,曰:“汝大必为将率。”口授兵法数万言。初为郡吏,守绛邑长。郭援之攻河东,所经城邑皆下,逵坚守,援攻之不拔,乃召单于并军急攻之。城将溃,绛父老与援要,不害逵。绛人既溃,援闻逵名,欲使为将,以兵劫之,逵不动。左右引逵使叩头,逵叱之曰:“安有国家长吏为贼叩头!”援怒,将斩之。绛吏民闻将杀逵,皆乘城呼曰:“负要杀我贤君,宁俱死耳!”左右义逵,多为请,遂得免。初,逵过皮氏,曰:“争地先据者胜。”及围急,知不免,乃使人间行送印绶归郡,且曰“急据皮氏”。援既并绛众,将进兵。逵恐其先得皮氏,乃以他计疑援谋人祝奥,援由是留七日。郡从逵言,故得无败。

后举茂才,除渑池令。高幹之反,张琰将举兵以应之。逵不知其谋,往见琰。闻变起,欲还,恐见执,乃为琰画计,如与同谋者,琰信之。时县寄治蠡城,城堑不固,逵从琰求兵脩城。诸欲为乱者皆不隐其谋,故逵得尽诛之。遂脩城拒琰。琰败,逵以丧祖父去官,司徒辟为掾,以议郎参司隶军事。太祖征马超,至弘农,曰“此西道之要”,以逵领弘农太守。召见计事,大悦之,谓左右曰:“使天下二千石悉如贾逵,吾何忧?”其后发兵,逵疑屯田都尉藏亡民。都尉自以不属郡,言语不顺。逵怒,收之,数以罪,挝折脚,坐免。然太祖心善逵,以为丞相主簿。太祖征刘备,先遣逵至斜谷观形势。道逢水衡,载囚人数十车,逵以军事急,辄竟重者一人,皆放其馀。太祖善之,拜谏议大夫,与夏侯尚并掌军计。太祖崩洛阳,逵典丧事。时鄢陵侯彰行越骑将军,从长安来赴,问逵先生玺绶所在。逵正色曰:“太子在邺,国有储副。先王玺绶,非君侯所宜问也。“遂奉梓宫还邺。

文帝即王位,以邺县户数万在都下,多不法,乃以逵为邺令。月馀,迁魏郡太守。大军出征,复为丞相主簿祭酒。逵尝坐人为罪,王曰:“叔向犹十世宥之,况逵功德亲在其身乎?”从至黎阳,津渡者乱行,逵斩之,乃整。至谯,以逵为豫州刺史。是时天下初复,州郡多不摄。逵曰:“州本以御史出监诸郡,以六条诏书察长吏二千石已下,故其状皆言严能鹰扬有督察之才,不言安静宽仁有恺悌之德也。今长吏慢法,盗贼公行,州知而不纠,天下复何取正乎?”兵曹从事受前刺史假,逵到官数月,乃还;考竟其二千石以下阿纵不如法者,皆举奏免之。帝曰:“逵真刺史矣。”布告天下,当以豫州为法。赐爵关内侯。

州南与吴接,逵明斥候,缮甲兵,为守战之备,贼不敢犯。外修军旅,内治民事,遏鄢、汝,造新陂,又断山溜长谿水,造小弋阳陂,又通运渠二百馀里,所谓贾侯渠者也。黄初中,与诸将并征吴,破吕范於洞浦,进封阳里亭侯,加建威将军。明帝即位,增邑二百户,并前四百户。时孙权在东关,当豫州南,去江四百馀里。每出兵为寇,辄西从江夏,东从庐江。国家征伐,亦由淮、沔。是时州军在项,汝南、弋阳诸郡,守境而已。权无北方之虞,东西有急,并军相救,故常少败。逵以为宜开直道临江,若权自守,则二方无救;若二方无救,则东关可取。乃移屯潦口,陈攻取之计,帝善之。

吴将张婴、王崇率众降。太和二年,帝使逵督前将军满宠、东莞太守胡质等四军,从西阳直向东关,曹休从皖,司马宣王从江陵。逵至五将山,休更表贼有请降者,求深入应之。诏宣王驻军,逵东与休合进。逵度贼无东关之备,必并军於皖;休深入与贼战,必败。乃部署诸将,水陆并进,行二百里,得生贼,言休战败,权遣兵断夹石。诸将不知所出,或欲待后军。逵曰:“休兵败於外,路绝於内,进不能战,退不得还,安危之机,不及终日。贼以军无后继,故至此;今疾进,出其不意,此所谓先人以夺其心也,贼见吾兵必走。若待后军,贼已断险,兵虽多何益!”乃兼道进军,多设旗鼓为疑兵,贼见逵军,遂退。逵据夹石,以兵粮给休,休军乃振。初,逵与休不善。黄初中,文帝欲假逵节,休曰:“逵性刚,素侮易诸将,不可为督。”帝乃止。及夹石之败,微逵,休军几无救也。

会病笃,谓左右曰:“受国厚恩,恨不斩孙权以下见先帝。丧事一不得有所脩作。”薨,谥曰肃侯。子充嗣。豫州吏民追思之,为刻石立祠。青龙中,帝东征,乘辇入逵祠,诏曰:“昨过项,见贾逵碑像,念之怆然。古人有言,患名之不立,不患年之不长。逵存有忠勋,没而见思,可谓死而不朽者矣。其布告天下,以劝将来。”充,咸熙中为中护军。

评曰:自汉季以来,刺史总统诸郡,赋政于外,非若曩时司察之而已。太祖创基,迄终魏业,此皆其流称誉有名实者也。咸精达事机,威恩兼著,故能肃齐万里,见述于后也。

\part{魏书十六}
\chapter{任苏杜郑仓传第十六}

任峻字伯达,河南中牟人也。汉末扰乱,关东皆震。中牟令杨原愁恐,欲弃官走。峻说原曰:“董卓首乱,天下莫不侧目,然而未有先发者,非无其心也,势未敢耳。明府若能唱之,必有和者。”原曰:“为之奈何?”峻曰:“今关东有十馀县,能胜兵者不减万人,若权行河南尹事,总而用之,无不济矣。”原从其计,以峻为主簿。峻乃为原表行尹事,使诸县坚守,遂发兵。会太祖起关东,入中牟界,众不知所从,峻独与同郡张奋议,举郡以归太祖。峻又别收宗族及宾客家兵数百人,愿从太祖。太祖大悦,表峻为骑都尉,妻以从妹,甚见亲信。太祖每征伐,峻常居守以给军。是时岁饥旱,军食不足,羽林监颍川枣祗建置屯田,太祖以峻为典农中郎将,募百姓屯田於许下,得谷百万斛,郡国列置田官,数年中所在积粟,仓廪皆满。官渡之战,太祖使峻典军器粮运。贼数寇钞绝粮道,乃使千乘为一部,十道方行,为复陈以营卫之,贼不敢近。军国之饶,起於枣祗而成於峻。太祖以峻功高,乃表封为都亭侯,邑三百户,迁长水校尉。

峻宽厚有度而见事理,每有所陈,太祖多善之。於饥荒之际,收恤朋友孤遗,中外贫宗,周急继乏,信义见称。建安九年薨,太祖流涕者久之。子先嗣。先薨,无子,国除。文帝追录功臣,谥峻曰成侯。复以峻中子览为关内侯。

苏则字文师,扶风武功人也。少以学行闻,举孝廉茂才,辟公府,皆不就。起家为酒泉太守,转安定、武都,所在有威名。太祖征张鲁,过其郡,见则悦之,使为军导。鲁破,则绥定下辩诸氐,通河西道,徙为金城太守。是时丧乱之后,吏民流散饥穷,户口损耗,则抚循之甚谨。外招怀羌胡,得其牛羊,以养贫老。与民分粮而食,旬月之间,流民皆归,得数千家。乃明为禁令,有干犯者辄戮,其从教者必赏。亲自教民耕种,其岁大丰收,由是归附者日多。李越以陇西反,则率羌胡围越,越即请服。太祖崩,西平麹演叛,称护羌校尉。则勒兵讨之。演恐,乞降。文帝以其功,加则护羌校尉,赐爵关内侯。

后演复结旁郡为乱,张掖张进执太守杜通,酒泉黄华不受太守辛机,进、华皆自称太守以应之。又武威三种胡并寇钞,道路断绝。武威太守毌丘兴告急於则。时雍、凉诸豪皆驱略羌胡以从进等,郡人咸以为进不可当。又将军郝昭、魏平先是各屯守金城,亦受诏不得西度。则乃见郡中大吏及昭等与羌豪帅谋曰:“今贼虽盛,然皆新合,或有胁从,未必同心;因衅击之,善恶必离,离而归我,我增而彼损矣。既获益众之实,且有倍气之势,率以进讨,破之必矣。若待大军,旷日持久,善人无归,必合於恶,善恶既合,势难卒离。虽有诏命,违而合权,专之可也。”於是昭等从之,乃发兵救武威,降其三种胡,与兴击进於张掖。演闻之,将步骑三千迎则,辞来助军,而实欲为变。则诱与相见,因斩之,出以徇军,其党皆散走。则遂与诸军围张掖,破之,斩进及其支党,众皆降。演军败,华惧,出所执乞降,河西平。乃还金城。进封都亭侯,邑三百户。

徵拜侍中,与董昭同寮。昭尝枕则膝卧,则推下之,曰:“苏则之膝,非佞人之枕也。”初,则及临菑侯植闻魏氏代汉,皆发服悲哭,文帝闻植如此,而不闻则也。帝在洛阳,尝从容言曰:“吾应天而禅,而闻有哭者,何也?”则谓为见问,须髯悉张,欲正论以对。侍中傅巽掐则曰:“不谓卿也。”於是乃止。文帝问则曰:“前破酒泉、张掖,西域通使,敦煌献径寸大珠,可复求市益得不?”则对曰:“若陛下化洽中国,德流沙漠,即不求自至;求而得之,不足贵也。”帝默然。后则从行猎,槎桎拔,失鹿,帝大怒,踞胡床拔刀,悉收督吏,将斩之。则稽首曰:“臣闻古之圣王不以禽兽害人,今陛下方隆唐尧之化,而以猎戏多杀群吏,愚臣以为不可。敢以死请!”帝曰:“卿,直臣也。”遂皆赦之。然以此见惮。黄初四年,左迁东平相。未至,道病薨,谥曰刚侯。子怡嗣。怡薨,无子,弟愉袭封。愉,咸熙中为尚书。

杜畿字伯侯,京兆杜陵人也。少孤,继母苦之,以孝闻。年二十,为郡功曹,守郑县令。县囚系数百人,畿亲临狱,裁其轻重,尽决遣之,虽未悉当,郡中奇其年少而有大意也。举孝廉,除汉中府丞。会天下乱,遂弃官客荆州,建安中乃还。荀彧进之太祖,太祖以畿为司空司直,迁护羌校尉,使持节,领西平太守。

太祖既定河北,而高幹举并州反。时河东太守王邑被徵,河东人卫固、范先外以请邑为名,而内实与幹通谋。太祖谓荀彧曰:“关西诸将,恃险与马,征必为乱。张晟寇殽、渑间,南通刘表,固等因之,吾恐其为害深。河东被山带河,四邻多变,当今天下之要地也。君为我举萧何、寇恂以镇之。”彧曰:“杜畿其人也。”於是追拜畿为河东太守。固等使兵数千人绝陕津,畿至不得渡。太祖遣夏侯惇讨之,未至。或谓畿曰:“宜须大兵。”畿曰:“河东有三万户,非皆欲为乱也。今兵迫之急,欲为善者无主,必惧而听於固。固等势专,必以死战。讨之不胜,四邻应之,天下之变未息也;讨之而胜,是残一郡之民也。且固等未显绝王命,外以请故君为名,必不害新君。吾单车直往,出其不意。固为人多计而无断,必伪受吾。吾得居郡一月,以计縻之,足矣。“遂诡道从郖津度。范先欲杀畿以威众。且观畿去就,於门下斩杀主簿已下三十馀人,畿举动自若。於是固曰:“杀之无损,徒有恶名;且制之在我。”遂奉之。畿谓卫固、范先曰:“卫、范,河东之望也,吾仰成而已。然君臣有定义,成败同之,大事当共平议。”以固为都督,行丞事,领功曹;将校吏兵三千馀人,皆范先督之。固等喜,虽阳事畿,不以为意。固欲大发兵,畿患之,说固曰:“夫欲为非常之事,不可动众心。今大发兵,众必扰,不如徐以赀募兵。”固以为然,从之,遂为赀调发,数十日乃定,诸将贪多应募而少遣兵。又入喻固等曰:“人情顾家,诸将掾吏,可分遣休息,急缓召之不难。”固等恶逆众心,又从之。於是善人在外,阴为己援;恶人分散,各还其家,则众离矣。会白骑攻东垣,高幹入濩泽,上党诸县杀长吏,弘农执郡守,固等密调兵未至。畿知诸县附己,因出,单将数十骑,赴张辟拒守,吏民多举城助畿者,比数十日,得四千馀人。固等与幹、晟共攻畿,不下,略诸县,无所得。会大兵至,幹、晟败,固等伏诛,其馀党与皆赦之,使复其居业。

是时天下郡县皆残破,河东最先定,少耗减。畿治之,崇宽惠,与民无为。民尝辞讼,有相告者,畿亲见为陈大义,遣令归谛思之,若意有所不尽,更来诣府。乡邑父老自相责怒曰:“有君如此,奈何不从其教?“自是少有辞讼。班下属县,举孝子、贞妇、顺孙,复其繇役,随时慰勉之。渐课民畜牸牛、草马,下逮鸡豚犬豕,皆有章程。百姓勤农,家家丰实。畿乃曰:“民富矣,不可不教也。”於是冬月修戎讲武,又开学宫,亲自执经教授,郡中化之。

韩遂、马超之叛也,弘农、冯翊多举县邑以应之。河东虽与贼接,民无异心。太祖西征至蒲阪,与贼夹渭为军,军食一仰河东。及贼破,馀畜二十馀万斛。太祖下令曰:“河东太守杜畿,孔子所谓'禹,吾无间然矣'。增秩中二千石。”太祖征汉中,遣五千人运,运者自率勉曰:“人生有一死,不可负我府君。”终无一人逃亡,其得人心如此。魏国既建,以畿为尚书。事平,更有令曰:“昔萧何定关中,寇恂平河内,卿有其功,间将授卿以纳言之职;顾念河东吾股肱郡,充实之所,足以制天下,故且烦卿卧镇之。”畿在河东十六年,常为天下最。

文帝即王位,赐爵关内侯。徵为尚书。及践阼,进封丰乐亭侯。邑百户,守司隶校尉。帝征吴,以畿为尚书仆射,统留事。其后帝幸许昌,畿复居守。受诏作御楼船,於陶河试船,遇风没。帝为之流涕。诏曰:“昔冥勤其官而水死,稷勤百谷而山死。故尚书仆射杜畿,於孟津试船,遂至覆没,忠之至也。朕甚愍焉。”追赠太仆,谥曰戴侯。子恕嗣。

恕字务伯,太和中为散骑黄门侍郎。恕推诚以质,不治饰,少无名誉。及在朝,不结交援,专心向公。每政有得失,常引纲维以正言,於是侍中辛毗等器重之。

时公卿以下大议损益,恕以为“古之刺史,奉宣六条,以清静为名,威风著称,今可勿令领兵,以专民事。“俄而镇北将军吕昭又领冀州,乃上疏曰:

帝王之道,莫尚乎安民;安民之术,在於丰财。丰财者,务本而节用也。方今二贼未灭,戎车亟驾,此自熊虎之士展力之秋也。然搢绅之儒,横加荣慕,搤腕抗论,以孙、吴为首。州郡牧守,咸共忽恤民之术,脩将率之事。农桑之民,竞干戈之业,不可谓务本。帑藏岁虚而制度岁广,民力岁衰而赋役岁兴,不可谓节用。今大魏奄有十州之地,而承丧乱之弊,计其户口不如往昔一州之民,然而二方僣逆,北虏未宾,三边遘难,绕天略匝;所以统一州之民,经营九州之地,其为艰难,譬策羸马以取道里,岂可不加意爱惜其力哉?以武皇帝之节俭,府藏充实,犹不能十州拥兵;郡且二十也。今荆、扬、青、徐、幽、并、雍、凉缘边诸州皆有兵矣,其所恃内充府库外制四夷者,惟兖、豫、司、冀而已。臣前以州郡典兵,则专心军功,不勤民事,宜别置将守,以尽治理之务;而陛下复以冀州宠秩吕昭。冀州户口最多,田多垦辟,又有桑枣之饶,国家徵求之府,诚不当复任以兵事也。若以北方当须镇守,自可专置大将以镇安之。计所置吏士之费,与兼官无异。然昭於人才尚复易;中朝苟乏人,兼才者势不独多。以此推之,知国家以人择官,不为官择人也。官得其人,则政平讼理;政平故民富贵,讼理故囹圄空虚。陛下践阼,天下断狱百数十人,岁岁增多,至五百馀人矣。民不益多,法不益峻。以此推之,非政教陵迟,牧守不称之明效欤?往年牛死,通率天下十能损二;麦不半收,秋种未下。若二贼游魂於疆埸,飞刍輓粟,千里不及。究此之术,岂在强兵乎?武士劲卒愈多,愈多愈病耳。夫天下犹人之体,腹心充实,四支虽病,终无大患;今兖、豫、司、冀亦天下之腹心也。是以愚臣慺慺,实愿四州之牧守,独脩务本之业,以堪四支之重。然孤论难持,犯欲难成,众怨难积,疑似难分,故累载不为明主所察。凡言此者,类皆疏贱;疏贱之言,实未易听。若使善策必出於亲贵,亲贵固不犯四难以求忠爱,此古今之所常患也。

时又大议考课之制,以考内外众官。恕以为用不尽其人,虽才且无益,所存非所务,所务非世要。上疏曰:

书称“明试以功,三考黜陟”,诚帝王之盛制。使有能者当其官,有功者受其禄,譬犹乌获之举千钧,良、乐之选骥足也。虽历六代而考绩之法不著,关七圣而课试之文不垂,臣诚以为其法可粗依,其详难备举故也。语曰:“世有乱人而无乱法。”若使法可专任,则唐、虞可不须稷、契之佐,殷、周无贵伊、吕之辅矣。今奏考功者,陈周、汉之法为,缀京房之本旨,可谓明考课之要矣。於以崇揖让之风,兴济济之治,臣以为未尽善也。其欲使州郡考士,必由四科,皆有事效,然后察举,试辟公府,为亲民长吏,转以功次补郡守者,或就增秩赐爵,此最考课之急务也。臣以为便当显其身,用其言,使具为课州郡之法,法具施行,立必信之赏,施必行之罚。至於公卿及内职大臣,亦当俱以其职考课之也。

古之三公,坐而论道,内职大臣,纳言补阙,无善不纪,无过不举。且天下至大,万机至众,诚非一明所能遍照。故君为元首,臣作股肱,明其一体相须而成也。是以古人称廊庙之材,非一木之支;帝王之业,非一士之略。由是言之,焉有大臣守职辨课可以致雍熙者哉!且布衣之交,犹有务信誓而蹈水火,感知己而披肝胆,徇声名而立节义者;况於束带立朝,致位卿相,所务者非特匹夫之信,所感者非徒知己之惠,所徇者岂声名而已乎!

诸蒙宠禄受重任者,不徒欲举明主於唐、虞之上而已;身亦欲厕稷、契之列。是以古人不患於念治之心不尽,患於自任之意不足,此诚人主使之然也。唐、虞之君,委任稷、契、夔、龙而责成功,及其罪也,殛鲧而放四凶。今大臣亲奉明诏,给事目下,其有夙夜在公,恪勤特立,当官不挠贵势,执平不阿所私,危言危行以处朝廷者,自明主所察也。若尸禄以为高,拱默以为智,当官苟在於免负,立朝不忘於容身,絜行逊言以处朝廷者,亦明主所察也。诚使容身保位,无放退之辜,而尽节在公,抱见疑之势,公义不脩而私议成俗,虽仲尼为谋,犹不能尽一才,又况於世俗之人乎!今之学者,师商、韩而上法术,竞以儒家为迂阔,不周世用,此最风俗之流弊,创业者之所致慎也。后考课竟不行。

乐安廉昭以才能拔擢,颇好言事。恕上疏极谏曰:

伏见尚书郎廉昭奏左丞曹璠以罚当关不依诏,坐判问。又云“诸当坐者别奏“。尚书令陈矫自奏不敢辞罚,亦不敢以处重为恭,意至恳恻。臣窃悯然为朝廷惜之!夫圣人不择世而兴,不易民而治,然而生必有贤智之佐者,盖进之以道,率之以礼故也。古之帝王之所以能辅世长民者,莫不远得百姓之欢心,近尽群臣之智力。诚使今朝任职之臣皆天下之选,而不能尽其力,不可谓能使人;若非天下之选,亦不可谓能官人。陛下忧劳万机,或亲灯火,而庶事不康,刑禁日弛,岂非股肱不称之明效欤?原其所由,非独臣有不尽忠,亦主有不能使。百里奚愚於虞而智於秦,豫让苟容中行而著节智伯,斯则古人之明验矣。今臣言一朝皆不忠,是诬一朝也;然其事类,可推而得。陛下感帑藏之不充实,而军事未息,至乃断四时之赋衣,薄御府之私谷,帅由圣意,举朝称明,与闻政事密勿大臣,宁有恳恳忧此者乎?

骑都尉王才、幸乐人孟思所为不法,振动京都,而其罪状发於小吏,公卿大臣初无一言。自陛下践阼以来,司隶校尉、御史中丞宁有举纲维以督奸宄,使朝廷肃然者邪?若陛下以为今世无良才,朝廷乏贤佐,岂可追望稷、契之遐踪,坐待来世之俊乂乎!今之所谓贤者,尽有大官而享厚禄矣,然而奉上之节未立,向公之心不一者,委任之责不专,而俗多忌讳故也。臣以为忠臣不必亲,亲臣不必忠。何者?以其居无嫌之地而事得自尽也。今有疏者毁人不实其所毁,而必曰私报所憎,誉人不实其所誉,而必曰私爱所亲,左右或因之以进憎爱之说。非独毁誉有之,政事损益,亦皆有嫌。陛下当思所以阐广朝臣之心,笃厉有道之节,使之自同古人,望与竹帛耳。反使如廉昭者扰乱其间,臣惧大臣遂将容身保位,坐观得失,为来世戒也!

昔周公戒鲁侯曰“无使大臣怨乎不以”,不言贤愚,明皆当世用也。尧数舜之功,称去四凶,不言大小,有罪则去也。今者朝臣不自以为不能,以陛下为不任也;不自以为不智,以陛下为不问也。陛下何不遵周公之所以用,大舜之所以去?使侍中、尚书坐则侍帷幄,行则从华辇,亲对诏问,所陈必达,则群臣之行,能否皆可得而知;忠能者进,闇劣者退,谁敢依违而不自尽?以陛下之圣明,亲与群臣论议政事,使群臣人得自尽,人自以为亲,人思所以报,贤愚能否,在陛下之所用。以此治事,何事不办?以此建功,何功不成?每有军事,诏书常曰:“谁当忧此者邪?吾当自忧耳。”近诏又曰:“忧公忘私者必不然,但先公后私即自办也。”伏读明诏,乃知圣思究尽下情,然亦怪陛下不治其本而忧其末也。人之能否,实有本性,虽臣亦以为朝臣不尽称职也。明主之用人也,使能者不敢遗其力,而不能者不得处非其任。选举非其人,未必为有罪也;举朝共容非其人,乃为怪耳。陛下知其不尽力也,而代之忧其职,知其不能也,而教之治其事,岂徒主劳而臣逸哉?虽圣贤并世,终不能以此为治也。

陛下又患台阁禁令之不密,人事请属之不绝,听伊尹作迎客出入之制,选司徒更恶吏以守寺门;威禁由之,实未得为禁之本也。昔汉安帝时,少府窦嘉辟廷尉郭躬无罪之兄子,犹见举奏,章劾纷纷。近司隶校尉孔羡辟大将军狂悖之弟,而有司嘿尔,望风希指,甚於受属。选举不以实,人事之大者也。嘉有亲戚之宠,躬非社稷重臣,犹尚如此;以今况古,陛下自不督必行之罚以绝阿党之原耳。伊尹之制,与恶吏守门,非治世之具也。使臣之言少蒙察纳,何患於奸不削灭,而养若昭等乎!

夫纠擿奸宄,忠事也,然而世憎小人行之者,以其不顾道理而苟求容进也。若陛下不复考其终始,必以违众忤世为奉公,密行白人为尽节,焉有通人大才而更不能为此邪?诚顾道理而弗为耳。使天下皆背道而趋利,则人主之所最病者,陛下将何乐焉,胡不绝其萌乎!夫先意承旨以求容美,率皆天下浅薄无行义者,其意务在於適人主之心而已,非欲治天下安百姓也。陛下何不试变业而示之,彼岂执其所守以违圣意哉?夫人臣得人主之心,安业也;处尊显之官,荣事也;食千锺之禄,厚实也。人臣虽愚,未有不乐此而喜干迕者也,迫於道,自强耳。诚以为陛下当怜而佑之,少委任焉,如何反录昭等倾侧之意,而忽若人者乎?今者外有伺隙之寇,内有贫旷之民,陛下当大计天下之损益,政事之得失,诚不可以怠也。

恕在朝八年,其论议亢直,皆此类也。

出为弘农太守,数岁转赵相,以疾去官。起家为河东太守,岁馀,迁淮北都督护军,复以疾去。恕所在,务存大体而已,其树惠爱,益得百姓欢心,不及於畿。顷之,拜御史中丞。恕在朝廷,以不得当世之和,故屡在外任。复出为幽州刺史,加建威将军,使持节,护乌丸校尉。时征北将军程喜屯蓟,尚书袁侃等戒恕曰:“程申伯处先帝之世,倾田国让於青州。足下今俱杖节,使共屯一城,宜深有以待之。”而恕不以为意。至官未期,有鲜卑大人儿,不由关塞,径将数十骑诣州,州斩所从来小子一人,无表言上。喜於是劾奏恕,下廷尉,当死。以父畿勤事水死,免为庶人,徙章武郡,是岁嘉平元年。恕倜傥任意,而思不防患,终致此败。

初,恕从赵郡还,陈留阮武亦从清河太守徵,俱自薄廷尉。谓恕曰:“相观才性可以由公道而持之不厉,器能可以处大官而求之不顺,才学可以述古今而志之不一,此所谓有其才而无其用。今向间暇,可试潜思,成一家言。”在章武,遂著体论八节。又著兴性论一篇,盖兴於为己也。四年,卒於徙所。

甘露二年,河东乐详年九十馀,上书讼畿之遗绩,朝廷感焉。诏封恕子预为丰乐亭侯,邑百户。

恕奏议论駮皆可观,掇其切世大事著于篇。

郑浑字文公,河南开封人也。高祖父众,众父兴,皆为名儒。浑兄泰,与荀攸等谋诛董卓,为扬州刺史,卒。浑将泰小子袤避难淮南,袁术宾礼甚厚。浑知术必败。时华歆为豫章太守,素与泰善,浑乃渡江投歆。太祖闻其笃行,召为掾,复迁下蔡长、邵陵令。天下未定,民皆剽轻,不念产殖;其生子无以相活,率皆不举。浑所在夺其渔猎之具,课使耕桑,又兼开稻田,重去子之法。民初畏罪,后稍丰给,无不举赡;所育男女,多以郑为字。辟为丞相掾属,迁左冯翊。

时梁兴等略吏民五千馀家为寇钞,诸县不能御,皆恐惧,寄治郡下。议者悉以为当移就险,浑曰:“兴等破散,窜在山阻。虽有随者,率胁从耳。今当广开降路,宣喻恩信。而保险自守,此示弱也。”乃聚敛吏民,治城郭,为守御之备。遂发民逐贼,明赏罚,与要誓,其所得获,十以七赏。百姓大悦,皆愿捕贼,多得妇女、财物。贼之失妻子者,皆还求降。浑责其得他妇女,然后还其妻子,於是转相寇盗,党与离散。又遣吏民有恩信者,分布山谷告喻,出者相继。乃使诸县长吏各还本治以安集之。兴等惧,将馀众聚鄜城。太祖使夏侯渊就助郡击之,浑率吏民前登,斩兴及其支党。又贼靳富等,胁将夏阳长、邵陵令并其吏民入硙山,浑复讨击破富等,获二县长吏,将其所略还。及赵青龙者,杀左内史程休,浑闻,遣壮士就枭其首。前后归附四千馀家,由是山贼皆平,民安产业。转为上党太守。

太祖征汉中,以浑为京兆尹。浑以百姓新集,为制移居之法,使兼複者与单轻者相伍,温信者与孤老为比,勤稼穑,明禁令,以发奸者。由是民安於农,而盗贼止息。及大军入汉中,运转军粮为最。又遣民田汉中,无逃亡者。太祖益嘉之,复入为丞相掾。文帝即位,为侍御史,加驸马都尉,迁阳平、沛郡二太守。郡界下湿,患水涝,百姓饥乏。浑於萧、相二县界,兴陂遏,开稻田。郡人皆以为不便,浑曰:“地势洿下,宜溉灌,终有鱼稻经久之利,此丰民之本也。”遂躬率吏民,兴立功夫,一冬间皆成。比年大收,顷亩岁增,租入倍常。民赖其利,刻石颂之,号曰郑陂。转为山阳、魏郡太守,其治放此。又以郡下百姓,苦乏材木,乃课树榆为篱,并益树五果;榆皆成藩,五果丰实。入魏郡界,村落齐整如一,民得财足用饶。明帝闻之,下诏称述,布告天下,迁将作大匠。浑清素在公,妻子不免於饥寒。及卒,以子崇为郎中。

仓慈字孝仁,淮南人也。始为郡吏。建安中,太祖开募屯田於淮南,以慈为绥集都尉。黄初末,为长安令,清约有方,吏民畏而爱之。太和中,迁敦煌太守。郡在西陲,以丧乱隔绝,旷无太守二十岁。大姓雄张,遂以为俗。前太守尹奉等,循故而已,无所匡革。慈到,抑挫权右,抚恤贫羸,甚得其理。旧大族田地有馀,而小民无立锥之土;慈皆随口割赋,稍稍使毕其本直。先是属城狱讼众猥,县不能决,多集治下;慈躬往省阅,料简轻重,自非殊死,但鞭杖遣之,一岁决刑曾不满十人。又常日西域杂胡欲来贡献,而诸豪族多逆断绝;既与贸迁,欺诈侮易,多不得分明。胡常怨望,慈皆劳之。欲诣洛者,为封过所,欲从郡还者,官为平取,辄以府见物与共交市,使吏民护送道路,由是民夷翕然称其德惠。数年卒官,吏民悲感如丧亲戚,图画其形,思其遗像。及西域诸胡闻慈死,悉共会聚於戊己校尉及长吏治下发哀,或有以刀画面,以明血诚,又为立祠,遥共祠之。

自太祖迄于咸熙,魏郡太守陈国吴瓘、清河太守乐安任燠、京兆太守济北颜斐、弘农太守太原令狐邵、济南相鲁国孔乂,或哀矜折狱,或推诚惠爱,或治身清白,或擿奸发伏,咸为良二千石。

评曰:任峻始兴义兵,以归太祖,辟土殖谷,仓庾盈溢,庸绩致矣。苏则威以平乱,既政事之良,又矫矫刚直,风烈足称。杜畿宽猛克济,惠以康民。郑浑、仓慈,恤理有方。抑皆魏代之名守乎!恕屡陈时政,经论治体,盖有可观焉。

\part{魏书十七}
\chapter{张乐于张徐传第十七}

张辽字文远,雁门马邑人也。本聂壹之后,以避怨变姓。少为郡吏。汉末,并州刺史丁原以辽武力过人,召为从事,使将兵诣京都。何进遣诣河北募兵,得千馀人。还,进败,以兵属董卓。卓败,以兵属吕布,迁骑都尉。布为李傕所败,从布东奔徐州,领鲁相,时年二十八。太祖破吕布於下邳,辽将其众降,拜中郎将,赐爵关内侯。数有战功,迁裨将军。袁绍破,别遣辽定鲁国诸县。与夏侯渊围昌豨於东海,数月粮尽,议引军还,辽谓渊曰:“数日已来,每行诸围,豨辄属目视辽。又其射矢更稀,此必豨计犹豫,故不力战。辽欲挑与语,傥可诱也?”乃使谓豨曰:“公有命,使辽传之。”豨果下与辽语,辽为说“太祖神武,方以德怀四方,先附者受大赏”。豨乃许降。辽遂单身上三公山,入豨家,拜妻子。豨欢喜,随诣太祖。太祖遣豨还,责辽曰:“此非大将法也。”辽谢曰:“以明公威信著於四海,辽奉圣旨,豨必不敢害故也。”从讨袁谭、袁尚於黎阳,有功,行中坚将军。从攻尚於邺,尚坚守不下。太祖还许,使辽与乐进拔阴安,徙其民河南。复从攻邺,邺破,辽别徇赵国、常山,招降缘山诸贼及黑山孙轻等。从攻袁谭,谭破,别将徇海滨,破辽东贼柳毅等。还邺,太祖自出迎辽,引共载,以辽为荡寇将军。复别击荆州,定江夏诸县,还屯临颍,封都亭侯。从征袁尚於柳城,卒与虏遇,辽劝太祖战,气甚奋,太祖壮之,自以所持麾授辽。遂击,大破之,斩单于蹋顿。

时荆州未定,复遣辽屯长社。临发,军中有谋反者,夜惊乱起火,一军尽扰。辽谓左右曰:“勿动。是不一营尽反,必有造变者,欲以动乱人耳。”乃令军中,其不反者安坐。辽将亲兵数十人,中陈而立。有顷定,即得首谋者杀之。陈兰、梅成以氐六县叛,太祖遣于禁、臧霸等讨成,辽督张郃、牛盖等讨兰。成伪降禁,禁还。成遂将其众就兰,转入灊山。灊中有天柱山,高峻二十馀里,道险狭,步径裁通,兰等壁其上。辽欲进,诸将曰:“兵少道险,难用深入。”辽曰:“此所谓一与一,勇者得前耳。“遂进到山下安营,攻之,斩兰、成首,尽虏其众。太祖论诸将功,曰:“登天山,履峻险,以取兰、成,荡寇功也。”增邑,假节。

太祖既征孙权还,使辽与乐进、李典等将七千馀人屯合肥。太祖征张鲁,教与护军薛悌,署函边曰“贼至乃发”。俄而权率十万众围合肥,乃共发教,教曰:“若孙权至者,张、李将军出战;乐将军守,护军勿得与战。”诸将皆疑。辽曰;“公远征在外,比救至,彼破我必矣。是以教指及其未合逆击之,折其盛势,以安众心,然后可守也。成败之机,在此一战,诸君何疑?”李典亦与辽同。於是辽夜募敢从之士,得八百人,椎牛飨将士,明日大战。平旦,辽被甲持戟,先登陷陈,杀数十人,斩二将,大呼自名,冲垒入,至权麾下。权大惊,众不知所为,走登高冢,以长戟自守。辽叱权下战,权不敢动,望见辽所将众少,乃聚围辽数重。辽左右麾围,直前急击,围开,辽将麾下数十人得出,馀众号呼曰:“将军弃我乎!”辽复还突围,拔出馀众。权人马皆披靡,无敢当者。自旦战至日中,吴人夺气,还修守备,众心乃安,诸将咸服。权守合肥十馀日,城不可拔,乃引退。辽率诸军追击,几复获权。太祖大壮辽,拜征东将军。建安二十一年,太祖复征孙权,到合肥,循行辽战处,叹息者良久。乃增辽兵,多留诸军,徙屯居巢。

关羽围曹仁於樊,会权称藩,召辽及诸军悉还救仁。辽未至,徐晃已破关羽,仁围解。辽与太祖会摩陂。辽军至,太祖乘辇出劳之,还屯陈郡。文帝即王位,转前将军。分封兄汎及一子列侯。孙权复叛,遣辽还屯合肥,进辽爵都乡侯。给辽母舆车,及兵马送辽家诣屯,敕辽母至,导从出迎。所督诸军将吏皆罗拜道侧,观者荣之。文帝践阼,封晋阳侯,增邑千户,并前二千六百户。黄初二年,辽朝洛阳宫,文帝引辽会建始殿,亲问破吴意状。帝叹息顾左右曰:“此亦古之召虎也。”为起第舍,又特为辽母作殿,以辽所从破吴军应募步卒,皆为虎贲。孙权复称藩。辽还屯雍丘,得疾。帝遣侍中刘晔将太医视疾,虎贲问消息,道路相属。疾未瘳,帝迎辽就行在所,车驾亲临,执其手,赐以御衣,太官日送御食。疾小差,还屯。孙权复叛,帝遣辽乘舟,与曹休至海陵,临江。权甚惮焉,敕诸将:“张辽虽病,不可当也,慎之!”是岁,辽与诸将破权将吕范。辽病笃,遂薨于江都。帝为流涕,谥曰刚侯。子虎嗣。六年,帝追念辽、典在合肥之功,诏曰:“合肥之役,辽、典以步卒八百,破贼十万,自古用兵,未之有也。使贼至今夺气,可谓国之爪牙矣。其分辽、典邑各百户,赐一子爵关内侯。”虎为偏将军,薨。子统嗣。

乐进字文谦,阳平卫国人也。容貌短小,以胆烈从太祖,为帐下吏。遣还本郡募兵,得千馀人,还为军假司马、陷陈都尉。从击吕布於濮阳,张超於雍丘,桥蕤於苦,皆先登有功,封广昌亭侯。从征张绣於安众,围吕布於下邳,破别将,击眭固於射犬,攻刘备於沛,皆破之,拜讨寇校尉。渡河攻获嘉,还,从击袁绍於官渡,力战,斩绍将淳于琼。从击谭、尚於黎阳,斩其大将严敬,行游击将军。别击黄巾,破之,定乐安郡。从围邺,邺定,从击袁谭於南皮,先登,入谭东门。谭败,别攻雍奴,破之。建安十一年,太祖表汉帝,称进及于禁、张辽曰:“武力既弘,计略周备,质忠性一,守执节义,每临战攻,常为督率,奋强突固,无坚不陷,自援枹鼓,手不知倦。又遣别征,统御师旅,抚众则和,奉令无犯,当敌制决,靡有遗失。论功纪用,宜各显宠。”於是禁为虎威;进,折冲;辽,荡寇将军。

进别征高幹,从北道入上党,回出其后。幹等还守壶关,连战斩首。幹坚守未下,会太祖自征之,乃拔。太祖征管承,军淳于,遣进与李典击之。承破走,逃入海岛,海滨平,荆州未服,遣屯阳翟。后从平荆州,留屯襄阳,击关羽、苏非等,皆走之,南郡诸郡山谷蛮夷诣进降。又讨刘备临沮长杜普、旌阳长梁大,皆大破之。后从征孙权,假进节。太祖还,留进与张辽、李典屯合肥,增邑五百,并前凡千二百户。以进数有功,分五百户,封一子列侯;进迁右将军。建安二十三年薨,谥曰威侯。子綝嗣。綝果毅有父风,官至扬州刺史。诸葛诞反,掩袭杀綝,诏悼惜之,追赠卫尉,谥曰愍侯。子肇嗣。

于禁字文则,泰山钜平人也。黄巾起,鲍信招合徒众,禁附从焉。及太祖领兖州,禁与其党俱诣为都伯,属将军王朗。朗异之,荐禁才任大将军。太祖召见与语,拜军司马,使将兵诣徐州,攻广戚,拔之,拜陷陈都尉。从讨吕布於濮阳,别破布二营於城南,又别将破高雅於须昌。从攻寿张、定陶、离狐,围张超於雍丘,皆拔之。从征黄巾刘辟、黄邵等,屯版梁,邵等夜袭太祖营,禁帅麾下击破之,斩邵等,尽降其众。迁平虏校尉。从围桥蕤於苦,斩蕤等四将。从至宛,降张绣。绣复叛,太祖与战不利,军败,还舞阴。是时军乱,各间行求太祖,禁独勒所将数百人,且战且引,虽有死伤不相离。虏追稍缓,禁徐整行队,鸣鼓而还。未至太祖所,道见十馀人被创裸走,禁问其故,曰:“为青州兵所劫。”初,黄巾降,号青州兵,太祖宽之,故敢因缘为略。禁怒,令其众曰:“青州兵同属曹公,而还为贼乎!”乃讨之,数之以罪。青州兵遽走诣太祖自诉。禁既至,先立营垒,不时谒太祖。或谓禁:“青州兵已诉君矣,宜促诣公辨之。”禁曰:“今贼在后,追至无时,不先为备,何以待敌?且公聪明,谮诉何缘!”徐凿堑安营讫,乃入谒,具陈其状。太祖悦,谓禁曰:“淯水之难,吾其急也,将军在乱能整,讨暴坚垒,有不可动之节,虽古名将,何以加之”於是录禁前后功,封益寿亭侯。复从攻张绣於穰,禽吕布於下邳,别与史涣、曹仁攻眭固於射犬,破斩之。

太祖初征袁绍,绍兵盛,禁原为先登。太祖壮之,乃遣步卒二千人,使禁将,守延津以拒绍,太祖引军还官渡。刘备以徐州叛,太祖东征之。绍攻禁,禁坚守,绍不能拔。复与乐进等将步骑五千,击绍别营,从延津西南缘河至汲、获嘉二县,焚烧保聚三十馀屯,斩首获生各数千,降绍将何茂、王摩等二十馀人。太祖复使禁别将屯原武,击绍别营於杜氏津,破之。迁裨将军,后从还官渡。太祖与绍连营,起土山相对。绍射营中,士卒多死伤,军中惧。禁督守土山,力战,气益奋。绍破,迁偏将军。冀州平。昌豨复叛,遣禁征之。禁急进攻豨;豨与禁有旧,诣禁降。诸将皆以为豨已降,当送诣太祖,禁曰:“诸君不知公常令乎!围而后降者不赦。夫奉法行令,事上之节也。豨虽旧友,禁可失节乎!”自临与豨决,陨涕而斩之。是时太祖军淳于,闻而叹曰:“豨降不诣吾而归禁,岂非命耶!”益重禁。东海平,拜禁虎威将军。后与臧霸等攻梅成,张辽、张郃等讨陈兰。禁到,成举众三千馀人降。既降复叛,其众奔兰。辽等与兰相持,军食少,禁运粮前后相属,辽遂斩兰、成。增邑二百户,并前千二百户。是时,禁与张辽、乐进、张郃、徐晃俱为名将,太祖每征伐,咸递行为军锋,还为后拒;而禁持军严整,得贼财物,无所私入,由是赏赐特重。然以法御下,不甚得士众心。太祖常恨朱灵,欲夺其营。以禁有威重,遣禁将数十骑,赍令书,径诣灵营夺其军,灵及其部众莫敢动;乃以灵为禁部下督,众皆震服,其见惮如此。迁左将军,假节钺,分邑五百户,封一子列侯。

建安二十四年,太祖在长安,使曹仁讨关羽於樊,又遣禁助仁。秋,大霖雨,汉水溢,平地水数丈,禁等七军皆没。禁与诸将登高望水,无所回避,羽乘大船就攻禁等,禁遂降,惟庞德不屈节而死。太祖闻之,哀叹者久之,曰:“吾知禁三十年,何意临危处难,反不如庞德邪!”会孙权禽羽,获其众,禁复在吴。文帝践阼,权称藩,遣禁还。帝引见禁,须发皓白,形容憔悴,泣涕顿首。帝慰谕以荀林父、孟明视故事,拜为安远将军。欲遣使吴,先令北诣邺谒高陵。帝使豫於陵屋画关羽战克、庞德愤怒、禁降服之状。禁见,惭恚发病薨。子圭嗣封益寿亭侯。谥禁曰厉侯。

张郃字俊乂,河间鄚人也。汉末应募讨黄巾,为军司马,属韩馥。馥败,以兵归袁绍。绍以郃为校尉,使拒公孙瓒。瓒破,郃功多,迁宁国中郎将。太祖与袁绍相拒於官渡,绍遣将淳于琼等督运屯乌巢,太祖自将急击之。郃说绍曰:“曹公兵精,往必破琼等;琼等破,则将军事去矣,宜急引兵救之。”郭图曰:“郃计非也。不如攻其本营,势必还,此为不救而自解也。”郃曰:“曹公营固,攻之必不拔,若琼等见禽,吾属尽为虏矣。”绍但遣轻骑救琼,而以重兵攻太祖营,不能下。太祖果破琼等,绍军溃。图惭,又更谮郃曰:“郃快军败,出言不逊。”郃惧,乃归太祖。

太祖得郃甚喜,谓曰:“昔子胥不早寤,自使身危,岂若微子去殷、韩信归汉邪?”拜郃偏将军,封都亭侯。授以众,从攻邺,拔之。又从击袁谭於渤海,别将军围雍奴,大破之。从讨柳城,与张辽俱为军锋,以功迁平狄将军。别征东莱,讨管承,又与张辽讨陈兰、梅成等,破之。从破马超、韩遂於渭南。围安定,降杨秋。与夏侯渊讨鄜贼梁兴及武都氐。又破马超,平宋建。太祖征张鲁,先遣郃督诸军讨兴和氐王窦茂。太祖从散关入汉中,又先遣郃督步卒五千於前通路。至阳平,鲁降,太祖还,留郃与夏侯渊等守汉中,拒刘备。郃别督诸军,降巴东、巴西二郡,徙其民於汉中。进军宕渠,为备将张飞所拒,引还南郑。拜荡寇将军。刘备屯阳平,郃屯广石。备以精卒万馀,分为十部,夜急攻郃。郃率亲兵搏战,备不能克。其后备於走马谷烧都围,渊救火,从他道与备相遇,交战,短兵接刃。渊遂没,郃还阳平。当是时,新失元帅,恐为备所乘,三军皆失色。渊司马郭淮乃令众曰:“张将军,国家名将,刘备所惮;今日事急,非张将军不能安也。”遂推郃为军主。郃出,勒兵安陈,诸将皆受郃节度,众心乃定。太祖在长安,遣使假郃节。太祖遂自至汉中,刘备保高山不敢战。太祖乃引出汉中诸军,郃还屯陈仓。

文帝即王位,以郃为左将军,进爵都乡侯。及践阼,进封鄚侯。诏郃与曹真讨安定卢水胡及东羌,召郃与真并朝许宫,遣南与夏侯尚击江陵。郃别督诸军渡江,取洲上屯坞。明帝即位,遣南屯荆州,与司马宣王击孙权别将刘阿等,追至祁口,交战,破之。诸葛亮出祁山。加郃位特进,遣督诸军,拒亮将马谡於街亭。谡依阻南山,不下据城。郃绝其汲道,击,大破之。南安、天水、安定郡反应亮,郃皆破平之。诏曰:“贼亮以巴蜀之众,当虓虎之师。将军被坚执锐,所向克定,朕甚嘉之。益邑千户,并前四千三百户。”司马宣王治水军於荆州,欲顺沔入江伐吴,诏郃督关中诸军往受节度。至荆州,会冬水浅,大船不得行,乃还屯方城。诸葛亮复出,急攻陈仓,帝驿马召郃到京都。帝自幸河南城,置酒送郃,遣南北军士三万及分遣武卫、虎贲使卫郃,因问郃曰:“迟将军到,亮得无已得陈仓乎!”郃知亮县军无谷,不能久攻,对曰:“比臣未到,亮已走矣;屈指计亮粮不至十日。”郃晨夜进至南郑,亮退。诏郃还京都,拜征西车骑将军。

郃识变数,善处营陈,料战势地形,无不如计,自诸葛亮皆惮之。郃虽武将而爱乐儒士,尝荐同乡卑湛经明行修,诏曰:“昔祭遵为将,奏置五经大夫,居军中,与诸生雅歌投壶。今将军外勒戎旅,内存国朝。朕嘉将军之意,今擢湛为博士。”

诸葛亮复出祁山,诏郃督诸将西至略阳,亮还保祁山,郃追至木门,与亮军交战,飞矢中郃右膝,薨,谥曰壮侯。子雄嗣。郃前后征伐有功,明帝分郃户,封郃四子列侯。赐小子爵关内侯。

徐晃字公明,河东杨人也。为郡吏,从车骑将军杨奉讨贼有功,拜骑都尉。李傕、郭汜之乱长安也,晃说奉,令与天子还洛阳,奉从其计。天子渡河至安邑,封晃都亭侯。及到洛阳,韩暹、董承日争斗,晃说奉令归太祖;奉欲从之,后悔。太祖讨奉於梁,晃遂归太祖。

太祖授晃兵,使击卷、原武贼,破之,拜裨将军。从征吕布,别降布将赵庶、李邹等。与史涣斩眭固於河内。从破刘备,又从破颜良,拔白马,进至延津,破文丑,拜偏将军。与曹洪击〈氵隱〉强贼祝臂,破之,又与史涣击袁绍运车於故市,功最多,封都亭侯。太祖既围邺,破邯郸,易阳令韩范伪以城降而拒守,太祖遣晃攻之。晃至,飞矢城中,为陈成败。范悔,晃辄降之。既而言於太祖曰:“二袁未破,诸城未下者倾耳而听,今日灭易阳,明日皆以死守,恐河北无定时也。愿公降易阳以示诸城,则莫不望风。”太祖善之。别讨毛城,设伏兵掩击,破三屯。从破袁谭於南皮,讨平原叛贼,克之。从征蹋顿,拜横野将军。从征荆州,别屯樊,讨中庐、临沮、宜城贼。又与满宠讨关羽於汉津,与曹仁击周瑜於江陵。十五年,讨太原反者,围大陵,拔之,斩贼帅商曜。韩遂、马超等反关右,遣晃屯汾阴以抚河东,赐牛酒,令上先人墓。太祖至潼关,恐不得渡,召问晃。晃曰:“公盛兵於此,而贼不复别守蒲阪,知其无谋也。今假臣精兵渡蒲坂津,为军先置,以截其里,贼可擒也。”太祖曰:“善。”使晃以步骑四千人渡津。作堑栅未成,贼梁兴夜将步骑五千馀人攻晃,晃击走之,太祖军得渡。遂破超等,使晃与夏侯渊平隃麋、汧诸氐,与太祖会安定。太祖还邺,使晃与夏侯渊平鄜、夏阳馀贼,斩梁兴,降三千馀户。从征张鲁。别遣晃讨攻椟、仇夷诸山氐,皆降之。迁平寇将军。解将军张顺围。击贼陈福等三十馀屯,皆破之。

太祖还邺,留晃与夏侯渊拒刘备於阳平。备遣陈式等十馀营绝马鸣阁道,晃别征破之,贼自投山谷,多死者。太祖闻,甚喜,假晃节,令曰:“此阁道,汉中之险要咽喉也。刘备欲断绝外内,以取汉中。将军一举,克夺贼计,善之善者也。”太祖遂自至阳平,引出汉中诸军。复遣晃助曹仁讨关羽,屯宛。会汉水暴隘,于禁等没。羽围仁於樊,又围将军吕常於襄阳。晃所将多新卒,以羽难与争锋,遂前至阳陵陂屯。太祖复还,遣将军徐商、吕建等诣晃,令曰:“须兵马集至,乃俱前。”贼屯偃城。晃到,诡道作都堑,示欲截其后,贼烧屯走。晃得偃城,两面连营,稍前,去贼围三丈所,未攻。太祖前后遣殷署、朱盖等凡十二营诣晃。贼围头有屯,又别屯四冢。晃扬声当攻围头屯,而密攻四冢。羽见四冢欲坏,自将步骑五千出战,晃击之,退走,遂追陷与俱入围,破之,或自投沔水死。太祖令曰:“贼围堑鹿角十重,将军致战全胜,遂陷贼围,多斩首虏。吾用兵三十馀年,及所闻古之善用兵者,未有长驱径入敌围者也。且樊、襄阳之在围,过於莒、即墨,将军之功,逾孙武、穰苴。”晃振旅还摩陂,太祖迎晃七里,置酒大会。太祖举卮酒劝晃,且劳之曰:“全樊、襄阳,将军之功也。”时诸军皆集,太祖案行诸营,士卒咸离陈观,而晃军营整齐,将士驻陈不动。太祖叹曰:“徐将军可谓有周亚夫之风矣。”

文帝即王位,以晃为右将军,进封逯乡侯。及践阼,进封杨侯。与夏侯尚讨刘备於上庸,破之。以晃镇阳平,徙封阳平侯。明帝即位,拒吴将诸葛瑾於襄阳。增邑二百,并前三千一百户。病笃,遗令敛以时服。

性俭约畏慎,将军常远斥候,先为不可胜,然后战,追奔争利,士不暇食。常叹曰:“古人患不遭明君,今幸遇之,常以功自效,何用私誉为!”终不广交援。太和元年薨,谥曰壮侯。子盖嗣。盖薨,子霸嗣。明帝分晃户,封晃子孙二人列侯。

初,清河朱灵为袁绍将。太祖之征陶谦,绍使灵督三营助太祖,战有功。绍所遣诸将各罢归,灵曰:“灵观人多矣,无若曹公者,此乃真明主也。今已遇,复何之?”遂留不去。所将士卒慕之,皆随灵留。灵后遂为好将,名亚晃等,至后将军,封高唐亭侯。

评曰:太祖建兹武功,而时之良将,五子为先。于禁最号毅重,然弗克其终。张郃以巧变为称,乐进以骁果显名,而鉴其行事,未副所闻。或注记有遗漏,未如张辽、徐晃之备详也。

\part{魏书十八}

\chapter{二李臧文吕许典二庞阎传第十八}

李典字曼成,山阳钜野人也。典从父乾,有雄气,合宾客数千家在乘氏。初平中,以众随太祖,破黄巾於寿张,又从击袁术,征徐州。吕布之乱,太祖遣乾还乘氏,慰劳诸县。布别驾薛兰、治中李封招乾,欲俱叛,乾不听,遂杀乾。太祖使乾子整将乾兵,与诸将击兰、封。兰、封破,从平兖州诸县有功,稍迁青州刺史。整卒,典徙颍阴令,为中郎将,将整军,迁离狐太守。

时太祖与袁绍相拒官渡,典率宗族及部曲输谷帛供军。绍破,以典为裨将军,屯安民。太祖击谭、尚於黎阳,使典与程昱等以船运军粮。会尚遣魏郡太守高蕃将兵屯河上,绝水道,太祖敕典、昱:“若船不得过,下从陆道。”典与诸将议曰:“蕃军少甲而恃水,有懈怠之心,击之必克。军不内御;苟利国家,专之可也,宜亟击之。”昱亦以为然。遂北渡河,攻蕃,破之,水道得通。刘表使刘备北侵,至叶,太祖遣典从夏侯惇拒之。备一旦烧屯去,惇率诸军追击之,典曰:“贼无故退,疑必有伏。南道狭窄,草木深,不可追也。”惇不听,与于禁追之,典留守。惇等果入贼伏里,战不利,典往救,备望见救至,乃散退。从围邺,邺定,与乐进围高幹於壶关,击管承於长广,皆破之。迁捕虏将军,封都亭侯。典宗族部曲三千馀家,居乘氏,自请愿徙诣魏郡。太祖笑曰:“卿欲慕耿纯邪?”典谢曰:“典驽怯功微,而爵宠过厚,诚宜举宗陈力;加以征伐未息,宜实郊遂之内,以制四方,非慕纯也。”遂徙部曲宗族万三千馀口居邺。太祖嘉之,迁破虏将军。与张辽、乐进屯合肥,孙权率众围之,辽欲奉教出战。进、典、辽皆素不睦,辽恐其不从,典慨然曰:“此国家大事,顾君计何如耳,吾可以私憾而忘公义乎!”乃率众与辽破走权。增邑百户,并前三百户。

典好学问,贵儒雅,不与诸将争功。敬贤士大夫,恂恂若不及,军中称其长者。年三十六薨,子祯嗣。文帝践阼,追念合肥之功,增祯邑百户,赐典一子爵关内侯,邑百户;谥典曰愍侯。

李通字文达,江夏平春人也。以侠闻於江、汝之间。与其郡人陈恭共起兵於朗陵,众多归之。时有周直者,众二千馀家,与恭、通外和内违。通欲图杀直而恭难之。通知恭无断,乃独定策,与直克会,酒酣杀直。众人大扰,通率恭诛其党帅,尽并其营。后恭妻弟陈郃,杀恭而据其众。通攻破郃军,斩郃首以祭恭墓。又生禽黄巾大帅吴霸而降其属。遭岁大饥,通倾家振施,与士分糟糠,皆争为用,由是盗贼不敢犯。

建安初,通举众诣太祖於许。拜通振威中郎将,屯汝南西界。太祖讨张绣,刘表遣兵以助绣,太祖军不利。通将兵夜诣太祖,太祖得以复战,通为先登,大破绣军。拜裨将军,封建功侯。分汝南二县,以通为阳安都尉。通妻伯父犯法,朗陵长赵俨收治,致之大辟。是时杀生之柄,决於牧守,通妻子号泣以请其命。通曰:“方与曹公戮力,义不以私废公。”嘉俨执宪不阿,与为亲交。太祖与袁绍相拒於官渡。绍遣使拜通征南将军,刘表亦阴招之,通皆拒焉。通亲戚部曲流涕曰:“今孤危独守,以失大援,亡可立而待也,不如亟从绍。”通按剑以叱之曰:“曹公明哲,必定天下。绍虽强盛,而任使无方,终为之虏耳。吾以死不贰。”即斩绍使,送印绶诣太祖。又击郡贼瞿恭、江宫、沈成等,皆破残其众,送其首,遂定淮、汝之地。改封都亭侯,拜汝南太守。时贼张赤等五千馀家聚桃山,通攻破之。刘备与周瑜围曹仁於江陵,别遣关羽绝北道。通率众击之,下马拔鹿角入围,且战且前,以迎仁军,勇冠诸将。通道得病薨,时年四十二。追增邑二百户,并前四百户。文帝践阼,谥曰刚侯。诏曰:“昔袁绍之难,自许、蔡以南,人怀异心。通秉义不顾,使携贰率服,朕甚嘉之。不幸早薨,子基虽已袭爵,未足酬其庸勋。基兄绪,前屯樊城,又有功,世笃其劳,其以基为奉义中郎将,绪平虏中郎将,以宠异焉。”

臧霸字宣高,泰山华人也。父戒,为县狱掾,据法不听太守欲所私杀。太守大怒,令收戒诣府,时送者百馀人。霸年十八,将客数十人径於费西山中要夺之,送者莫敢动,因与父俱亡命东海,由是以勇壮闻。黄巾起,霸从陶谦击破之,拜骑都尉。遂收兵於徐州,与孙观、吴敦、尹礼等并聚众,霸为帅,屯於开阳。太祖之讨吕布也,霸等将兵助布。既禽布,霸自匿。太祖募索得霸,见而悦之,使霸招吴敦、尹礼、孙观、观兄康等,皆诣太祖。太祖以霸为琅邪相,敦利城、礼东莞、观北海、康城阳太守,割青、徐二州,委之於霸。太祖之在兖州,以徐翕、毛晖为将。兖州乱,翕、晖皆叛。后兖州定,翕、晖亡命投霸。太祖语刘备,令语霸送二人首。霸谓备曰:“霸所以能自立者,以不为此也。霸受公生全之恩,不敢违命。然王霸之君可以义告,愿将军为之辞。”备以霸言白太祖,太祖叹息,谓霸曰:“此古人之事而君能行之,孤之愿也。”乃皆以翕、晖为郡守。时太祖方与袁绍相拒,而霸数以精兵入青州,故太祖得专事绍,不以东方为念。太祖破袁谭於南皮,霸等会贺。霸因求遣子弟及诸将父兄家属诣邺,太祖曰:“诸君忠孝,岂复在是!昔萧何遣子弟入侍,而高祖不拒,耿纯焚室舆榇以从,而光武不逆,吾将何以易之哉!”东州扰攘,霸等执义征暴,清定海岱,功莫大焉,皆封列侯。霸为都亭侯,加威虏将军。又与于禁讨昌豨,与夏侯渊讨黄巾馀贼徐和等,有功,迁徐州刺史。沛国武周为下邳令,霸敬异周,身诣令舍。部从事总詷不法,周得其罪,便收考竟,霸益以善周。从讨孙权,先登,再入巢湖,攻居巢,破之。张辽之讨陈兰,霸别遣至皖,讨吴将韩当,使权不得救兰。当遣兵逆霸,霸与战於逢龙,当复遣兵邀霸於夹石,与战破之,还屯舒。权遣数万人乘船屯舒口,分兵救兰,闻霸军在舒,遁还。霸夜追之,比明,行百馀里,邀贼前后击之。贼窘急,不得上船,赴水者甚众。由是贼不得救兰,辽遂破之。霸从讨孙权於濡须口,与张辽为前锋,行遇霖雨,大军先及,水遂长,贼船稍进,将士皆不安。辽欲去,霸止之曰:“公明於利钝,宁肯捐吾等邪?”明日果有令。辽至,以语太祖。太祖善之,拜扬威将军,假节。后权乞降,太祖还,留霸与夏侯惇等屯居巢。

文帝即王位,迁镇东将军,进爵武安乡侯,都督青州诸军事。及践阼,进封开阳侯,徙封良成侯。与曹休讨吴贼,破吕范於洞浦,徵为执金吾,位特进。每有军事,帝常咨访焉。明帝即位,增邑五百,并前三千五百户。薨,谥曰威侯。子艾嗣。艾官至青州刺史、少府。艾薨,谥曰恭侯。子权嗣。霸前后有功,封子三人列侯,赐一人爵关内侯。

而孙观亦至青州刺史,假节,从太祖讨孙权,战被创,薨。子毓嗣,亦至青州刺史。

文聘字仲业,南阳宛人也,为刘表大将,使御北方。表死,其子琮立。太祖征荆州,琮举州降,呼聘欲与俱,聘曰:“聘不能全州,当待罪而已。“太祖济汉,聘乃诣太祖,太祖问曰:“来何迟邪?”聘曰:“先日不能辅弼刘荆州以奉国家,荆州虽没,常愿据守汉川,保全土境,生不负於孤弱,死无愧於地下,而计不得已,以至於此。实怀悲惭,无颜早见耳。”遂欷歔流涕。太祖为之怆然,曰:“仲业,卿真忠臣也。”厚礼待之。授聘兵,使与曹纯追讨刘备於长阪。太祖先定荆州,江夏与吴接,民心不安,乃以聘为江夏太守,使典北兵,委以边事,赐爵关内侯。与乐进讨关羽於寻口,有功,进封延寿亭侯,加讨逆将军。又攻羽辎重於汉津,烧其船於荆城。文帝践阼,进爵长安乡侯,假节。与夏侯尚围江陵,使聘别屯沔口,止石梵,自当一队,御贼有功,迁后将军,封新野侯。孙权以五万众自围聘於石阳,甚急,聘坚守不动,权住二十馀日乃解去。聘追击破之。增邑五百户,并前千九百户。

聘在江夏数十年,有威恩,名震敌国,贼不敢侵。分聘户邑封聘子岱为列侯,又赐聘从子厚爵关内侯。聘薨,谥曰壮侯。岱又先亡,聘养子休嗣。卒,子武嗣。

嘉平中,谯郡桓禺为江夏太守,清俭有威惠,名亚於聘。

吕虔字子恪,任城人也。太祖在兖州,闻虔有胆策,以为从事,将家兵守湖陆。襄贲校尉杜松部民炅母等作乱,与昌豨通。太祖以虔代松。虔到,招诱炅母渠率及同恶数十人,赐酒食。简壮士伏其侧,虔察炅母等皆醉,使伏兵尽格杀之。抚其馀众,群贼乃平。太祖以虔领泰山太守。郡接山海,世乱,闻民人多藏窜。袁绍所置中郎将郭祖、公孙犊等数十辈,保山为寇,百姓苦之。虔将家兵到郡,开恩信,祖等党属皆降服,诸山中亡匿者尽出安土业。简其强者补战士,泰山由是遂有精兵,冠名州郡。济南黄巾徐和等,所在劫长吏,攻城邑。虔引兵与夏侯渊会击之,前后数十战,斩首获生数千人。太祖使督青州诸郡兵以讨东莱群贼李条等,有功。太祖令曰:“夫有其志,必成其事,盖烈士之所徇也。卿在郡以来,禽奸讨暴,百姓获安,躬蹈矢石,所征辄克。昔寇恂立名於汝、颍,耿弇建策於青、兖,古今一也。”举茂才,加骑都尉,典郡如故。虔在泰山十数年,甚有威惠。文帝即王位,加裨将军,封益寿亭侯,迁徐州刺史,加威虏将军。请琅邪王祥为别驾,民事一以委之,世多其能任贤。讨利城叛贼,斩获有功。明帝即位,徙封万年亭侯,增邑二百,并前六百户。虔薨,子翻嗣。翻薨,子桂嗣。

\begin{yuanwen}
许褚字仲康,谯国谯人也。长八尺馀,腰大十围,容貌雄毅,勇力绝人。汉末,聚少年及宗族数千家,共坚壁以御寇。时汝南葛陂贼万馀人攻褚壁,褚众少不敌,力战疲极。兵矢尽,乃令壁中男女,聚治石如杅斗者置四隅。褚飞石掷之,所值皆摧碎。贼不敢进。粮乏,伪与贼和,以牛与贼易食,贼来取牛,牛辄奔还。褚乃出陈前,一手逆曳牛尾,行百馀步。贼众惊,遂不敢取牛而走。由是淮、汝、陈、梁间,闻皆畏惮之。
\end{yuanwen}

\begin{yuanwen}
太祖徇淮、汝,褚以众归太祖。太祖见而壮之曰:“此吾樊哙也。”

即日拜都尉,引入宿卫。诸从褚侠客,皆以为虎士。从征张绣,先登,斩首万计,迁校尉。从讨袁绍於官渡。

时常从士徐他等谋为逆,以褚常侍左右,惮之不敢发。伺褚休下日,他等怀刀入。褚至下舍心动,即还侍。他等不知,入帐见褚,大惊愕。他色变,褚觉之,即击杀他等。太祖益亲信之,出入同行,不离左右。从围邺,力战有功,赐爵关内侯。
\end{yuanwen}

\begin{yuanwen}
从讨韩遂、马超於潼关。太祖将北渡,临济河,先渡兵,独与褚及虎士百馀人留南岸断后。超将步骑万馀人,来奔太祖军,矢下如雨。褚白太祖,贼来多,今兵渡已尽,宜去,乃扶太祖上船。贼战急,军争济,船重欲没。褚斩攀船者,左手举马鞍蔽太祖。船工为流矢所中死,褚右手并溯\footnote{s\`u}船,仅乃得渡。是日,微褚几危。

其后太祖与遂、超等单马会语,左右皆不得从,唯将褚。超负其力,阴欲前突太祖,素闻褚勇,疑从骑是褚。乃问太祖曰:“公有虎侯者安在?”

太祖顾指褚,褚瞋目盻之。超不敢动,乃各罢。后数日会战,大破超等,褚身斩首级,迁武卫中郎将。武卫之号,自此始也。军中以褚力如虎而痴,故号曰虎痴;是以超问虎侯,至今天下称焉,皆谓其姓名也。
\end{yuanwen}

\begin{yuanwen}
褚性谨慎奉法,质重少言。曹仁自荆州来朝谒,太祖未出,入与褚相见於殿外。仁呼褚入便坐语,褚曰:“王将出。”便还入殿,仁意恨之。

或以责褚曰:“征南宗室重臣,降意呼君,君何故辞?”

褚曰:“彼虽亲重,外藩也。褚备内臣,众谈足矣,入室何私乎?”

太祖闻,愈爱待之,迁中坚将军。太祖崩,褚号泣呕血。文帝践阼,进封万岁亭侯,迁武卫将军,都督中军宿卫禁兵,甚亲近焉。

初,褚所将为虎士者从征伐,太祖以为皆壮士也,同日拜为将,其后以功为将军封侯者数十人,都尉、校尉百馀人,皆剑客也。

太和中,帝思褚忠孝,下诏褒赞,复赐褚子孙二人爵关内侯。
\end{yuanwen}

明帝即位,进封牟乡侯,邑七百户,赐子爵一人关内侯。褚薨,谥曰壮侯。子仪嗣。褚兄定,亦以军功封为振威将军,都督徼道虎贲。

仪为锺会所杀。泰始初,子综嗣。

\begin{yuanwen}
典韦,陈留己吾人也。形貌魁梧,旅力\footnote{体力,力气。}过人,有志节任侠。襄邑刘氏与睢阳李永为雠(仇),韦为报之。永故富春长,备卫甚谨。韦乘车载鸡酒,伪为候者,门开,怀匕首入杀永,并杀其妻,徐出,取车上刀戟,步去。永居近巿,一巿尽骇。追者数百,莫敢近。行四五里,遇其伴,转战得脱。由是为豪杰所识。

初平中,张邈举义兵,韦为士,属司马赵宠。牙门旗长大,人莫能胜,韦一手建之,宠异其才力。后属夏侯惇,数斩首有功,拜司马。太祖讨吕布於濮阳。布有别屯在濮阳西四五十里,太祖夜袭,比明破之。未及还,会布救兵至,三面掉战。时布身自搏战,自旦至日昳数十合,相持急。太祖募陷陈,韦先占,将应募者数十人,皆重衣两铠,弃楯,但持长矛撩戟。时西面又急,韦进当之,贼弓弩乱发,矢至如雨,韦不视,谓等人曰:“虏来十步,乃白之。”

等人曰:“十步矣。”

又曰:“五步乃白。”

等人惧,疾言“虏至矣”!韦手持十馀戟,大呼起,所抵无不应手倒者。布众退。

会日暮,太祖乃得引去。拜韦都尉,引置左右,将亲兵数百人,常绕大帐。韦既壮武,其所将皆选卒,每战斗,常先登陷陈。迁为校尉。性忠至谨重,常昼立侍终日,夜宿帐左右,稀归私寝。好酒食,饮啖兼人,每赐食於前,大饮长歠\footnote{chu\`o,通“啜”,饮、喝的意思。},左右相属,数人益乃供,太祖壮之。韦好持大双戟与长刀等,军中为之语曰:“帐下壮士有典君,提一双戟八十斤。”
\end{yuanwen}

\begin{yuanwen}
太祖征荆州,至宛,张绣迎降。太祖甚悦,延绣及其将帅,置酒高会。太祖行酒,韦持大斧立后,刃径尺,太祖所至之前,韦辄举斧目之。竟酒,绣及其将帅莫敢仰视。

后十馀日,绣反,袭太祖营,太祖出战不利,轻骑引去。韦战於门中,贼不得入。兵遂散从他门并入。时韦校尚有十馀人,皆殊死战,无不一当十。贼前后至稍多,韦以长戟左右击之,一叉入,辄十馀矛摧。左右死伤者略尽。韦被数十创,短兵接战,贼前搏之。韦双挟两贼击杀之,馀贼不敢前。韦复前突贼,杀数人,创重发,瞋目大骂而死。贼乃敢前,取其头,传观之,覆军就视其躯。

太祖退住舞阴,闻韦死,为流涕,募间取其丧,亲自临哭之,遣归葬襄邑,拜子满为郎中。车驾每过,常祠以中牢。太祖思韦,拜满为司马,引自近。文帝即王位,以满为都尉,赐爵关内侯。
\end{yuanwen}


庞德字令明,南安狟道人也。少为郡吏州从事。初平中,从马腾击反羌叛氐。数有功,稍迁至校尉。建安中,太祖讨袁谭、尚於黎阳,谭遣郭援、高幹等略取河东,太祖使锺繇率关中诸将讨之。德随腾子超拒援、幹於平阳,德为军锋,进攻援、幹,大破之,亲斩援首。拜中郎将,封都亭侯。后张白骑叛於弘农,德复随腾征之,破白骑於两殽间。每战,常陷陈卻敌,勇冠腾军。后腾徵为卫尉,德留属超。太祖破超於渭南,德随超亡入汉阳,保冀城。后复随超奔汉中,从张鲁。太祖定汉中,德随众降。太祖素闻其骁勇,拜立义将军,封关门亭侯,邑三百户。

侯音、卫开等以宛叛,德将所领与曹仁共攻拔宛,斩音、开,遂南屯樊,讨关羽。樊下诸将以德兄在汉中,颇疑之。德常曰:“我受国恩,义在效死。我欲身自击羽。今年我不杀羽,羽当杀我。”后亲与羽交战,射羽中额。时德常乘白马,羽军谓之白马将军,皆惮之。仁使德屯樊北十里,会天霖雨十馀日,汉水暴溢,樊下平地五六丈,德与诸将避水上堤。羽乘船攻之,以大船四面射堤上。德被甲持弓,箭不虚发。将军董衡、部曲将董超等欲降,德皆收斩之。自平旦力战至日过中,羽攻益急,矢尽,短兵接战。德谓督将成何曰:“吾闻良将不怯死以苟免,烈士不毁节以求生,今日,我死日也。”战益怒,气愈壮,而水浸盛,吏士皆降。德与麾下将一人,五伯二人,弯弓傅矢,乘小船欲还仁营。水盛船覆,失弓矢,独抱船覆水中,为羽所得,立而不跪。羽谓曰:“卿兄在汉中,我欲以卿为将,不早降何为?”德骂羽曰:“竖子,何谓降也!魏王带甲百万,威振天下。汝刘备庸才耳,岂能敌邪!我宁为国家鬼,不为贼将也。”遂为羽所杀。太祖闻而悲之,为之流涕,封其二子为列侯。文帝即王位,乃遣使就德墓赐谥,策曰:“昔先轸丧元,王蠋绝脰,陨身徇节,前代美之。惟侯式昭果毅,蹈难成名,声溢当时,义高在昔,寡人愍焉,谥曰壮侯”又赐子会等四人爵关内侯,邑各百户。会勇烈有父风,官至中尉将军,封列侯。

庞淯字子异,酒泉表氏人也。初以凉州从事守破羌长,会武威太守张猛反,杀刺史邯郸商,猛令曰:“敢有临商丧,死不赦。”淯闻之,弃官,昼夜奔走,号哭丧所讫,诣猛门,衷匕首,欲因见以杀猛。猛知其义士,敕遣不杀,由是以忠烈闻。太守徐揖请为主簿。后郡人黄昂反,围城。淯弃妻子,夜逾城出围,告急於张掖、敦煌二郡。初疑未肯发兵,淯欲伏剑,二郡感其义,遂为兴兵。军未至而郡城邑已陷,揖死。淯乃收敛揖丧,送还本郡,行服三年乃还。太祖闻之,辟为掾属。文帝践阼,拜驸马都尉,迁西海太守,赐爵关内侯。后徵拜中散大夫,薨,子曾嗣。

初,淯外祖父赵安为同县李寿所杀,淯舅兄弟三人同时病死,寿家喜。淯母娥自伤父雠不报,乃帏车袖剑,白日刺寿於都亭前,讫,徐诣县,颜色不变,曰:“父雠己报,请受戮。“禄福长尹嘉解印绶纵娥,娥不肯去,遂强载还家。会赦得免,州郡叹贵,刊石表闾。

阎温字伯俭,天水西城人也。以凉州别驾守上邽令。马超走奔上邽,郡人任养等举众迎之。温止之,不能禁,乃驰还州。超复围州所治冀城甚急,州乃遣温密出,告急於夏侯渊。贼围数重,温夜从水中潜出。明日,贼见其迹,遣人追遮之,於显亲界得温,执还诣超。超解其缚,谓曰:“今成败可见,足下为孤城请救而执於人手,义何所施?若从吾言,反谓城中,东方无救,此转祸为福之计也。不然,今为戮矣。”温伪许之,超乃载温诣城下。温向城大呼曰:“大军不过三日至,勉之!”城中皆泣,称万岁。超怒数之曰:“足下不为命计邪?”温不应。时超攻城久不下,故徐诱温,冀其改意。复谓温曰:“城中故人,有欲与吾同者不?”温又不应。遂切责之,温曰:“夫事君有死无贰,而卿乃欲令长者出不义之言,吾岂苟生者乎?”超遂杀之。

先是,河右扰乱,隔绝不通,敦煌太守马艾卒官,府又无丞。功曹张恭素有学行,郡人推行长史事,恩信甚著,乃遣子就东诣太祖,请太守。时酒泉黄华、张掖张进各据其郡,欲与恭并势。就至酒泉,为华所拘执,劫以白刃。就终不回,私与恭疏曰:“大人率厉敦煌,忠义显然,岂以就在困厄之中而替之哉?昔乐羊食子,李通覆家,经国之臣,宁怀妻孥邪?今大军垂至,但当促兵以掎之耳;愿不以下流之爱,使就有恨於黄壤也。”恭即遣从弟华攻酒泉沙头、乾齐二县。恭又连兵寻继华后,以为首尾之援。别遣铁骑二百,迎吏官属,东缘酒泉北塞,径出张掖北河,逢迎太守尹奉。於是张进须黄华之助;华欲救进,西顾恭兵,恐急击其后,遂诣金城太守苏则降。就竟平安。奉得之官。黄初二年,下诏褒扬,赐恭爵关内侯,拜西域戊己校尉。数岁徵还,将授以侍臣之位,而以子就代焉。恭至敦煌,固辞疾笃。太和中卒,赠执金吾。就后为金城太守,父子著称於西州。

评曰:李典贵尚儒雅,义忘私隙,美矣。李通、臧霸、文聘、吕虔镇卫州郡,并著威惠。许褚、典韦折冲左右,抑亦汉之樊哙也。庞德授命叱敌,有周苛之节。庞淯不惮伏剑,而诚感邻国。阎温向城大呼,齐解、路之烈焉。

\part{魏书十九}
\chapter{任城陈萧王传第十九}

任城威王彰,字子文。少善射御,膂力过人,手格猛兽,不避险阻。数从征伐,志意慷慨。太祖尝抑之曰:“汝不念读书慕圣道,而好乘汗马击剑,此一夫之用,何足贵也!”课彰读诗、书,彰谓左右曰:“丈夫一为卫、霍,将十万骑驰沙漠,驱戎狄,立功建号耳,何能作博士邪?”太祖尝问诸子所好,使各言其志。彰曰:“好为将。”太祖曰:“为将柰何?”对曰:“被坚执锐,临难不顾,为士卒先;赏必行,罚必信。”太祖大笑。建安二十一年,封鄢陵侯。

二十三年,代郡乌丸反,以彰为北中郎将,行骁骑将军。临发,太祖戒彰曰:“居家为父子,受事为君臣,动以王法从事,尔其戒之!”彰北征,入涿郡界,叛胡数千骑卒至。时兵马未集,唯有步卒千人,骑数百匹。用田豫计,固守要隙,虏乃退散。彰追之,身自搏战,射胡骑,应弦而倒者前后相属。战过半日,彰铠中数箭,意气益厉,乘胜逐北,至于桑乾,去代二百馀里。长史诸将皆以为新涉远,士马疲顿,又受节度,不得过代,不可深进,违令轻敌。彰曰:“率师而行,唯利所在,何节度乎?胡走未远,追之必破。从令纵敌,非良将也。”遂上马,令军中:“后出者斩。”一日一夜与虏相及,击,大破之,斩首获生以千数。彰乃倍常科大赐将士,将士无不悦喜。时鲜卑大人轲比能将数万骑观望强弱,见彰力战,所向皆破,乃请服。北方悉平。时太祖在长安,召彰诣行在所。彰自代过邺,太子谓彰曰:“卿新有功,今西见上,宜勿自伐,应对常若不足者。“彰到,如太子言,归功诸将。太祖喜,持彰须曰:“黄须儿竟大奇也!”

太祖东还,以彰行越骑将军,留长安。太祖至洛阳,得疾,驿召彰,未至,太祖崩。文帝即王位,彰与诸侯就国。诏曰:“先王之道,庸勋亲亲,并建母弟,开国承家,故能藩屏大宗,御侮厌难。彰前受命北伐,清定朔土,厥功茂焉。增邑五千,并前万户。”黄初二年,进爵为公。三年,立为任城王。四年,朝京都,疾薨于邸,谥曰威。至葬,赐銮辂、龙旂,虎贲百人,如汉东平王故事。子楷嗣,徙封中牟。五年,改封任城县。太和六年,复改封任城国,食五县二千五百户。青龙三年,楷坐私遣官属诣中尚方作禁物,削县二千户。正始七年,徙封济南,三千户。正元、景元初,连增邑,凡四千四百户。

陈思王植字子建。年十岁馀,诵读诗、论及辞赋数十万言,善属文。太祖尝视其文,谓植曰:“汝倩人邪?”植跪曰:“言出为论,下笔成章,顾当面试,柰何倩人?”时邺铜爵台新城,太祖悉将诸子登台,使各为赋。植援笔立成,可观,太祖甚异之。性简易,不治威仪。舆马服饰,不尚华丽。每进见难问,应声而对,特见宠爱。建安十六年,封平原侯。十九年,徙封临菑侯。太祖征孙权,使植留守邺,戒之曰:“吾昔为顿邱令,年二十三。思此时所行,无悔於今。今汝年亦二十三矣,可不勉与!”植既以才见异,而丁仪、丁廙、杨脩等为之羽翼。太祖狐疑,几为太子者数矣。而植任性而行,不自彫励,饮酒不节。文帝御之以术,矫情自饰,宫人左右,并为之说,故遂定为嗣。二十二年,增置邑五千,并前万户。植尝乘车行驰道中,开司马门出。太祖大怒,公车令坐死。由是重诸侯科禁,而植宠日衰。太祖既虑终始之变,以杨脩颇有才策,而又袁氏之甥也,於是以罪诛脩。植益内不自安。二十四年,曹仁为关羽所围。太祖以植为南中郎将,行征虏将军,欲遣救仁,呼有所敕戒。植醉不能受命,於是悔而罢之。

文帝即王位,诛丁仪、丁廙并其男口。植与诸侯并就国。黄初二年,监国谒者灌均希指,奏“植醉酒悖慢,劫胁使者”。有司请治罪,帝以太后故,贬爵安乡侯。其年改封鄄城侯。三年,立为鄄城王,邑二千五百户。

四年,徙封雍丘王。其年,朝京都。上疏曰:

臣自抱衅归藩,刻肌刻骨,追思罪戾,昼分而食,夜分而寝。诚以天罔不可重离,圣恩难可再恃。窃感相鼠之篇,无礼遄死之义,形影相吊,五情愧赧。以罪弃生,则违古贤“夕改”之劝,忍活苟全,则犯诗人“胡颜”之讥。伏惟陛下德象天地,恩隆父母,施畅春风,泽如时雨。是以不别荆棘者,庆云之惠也;七子均养者,尸鸠之仁也;舍罪责功者,明君之举也;矜愚爱能者,慈父之恩也:是以愚臣徘徊於恩泽而不能自弃者也。

前奉诏书,臣等绝朝,心离志绝,自分黄耇无复执珪之望。不图圣诏猥垂齿召,至止之日,驰心辇毂。僻处西馆,未奉阙廷,踊跃之怀,瞻望反仄。谨拜表献诗二篇,其辞曰:“於穆显考,时惟武皇,受命于天,宁济四方。朱旗所拂,九土披攘,玄化滂流,荒服来王。超商越周,与唐比踪。笃生我皇,奕世载聪,武则肃烈,文则时雍,受禅炎汉,临君万邦。万邦既化,率由旧则;广命懿亲,以藩王国。帝曰尔侯,君兹青土,奄有海滨,方周于鲁,车服有辉,旗章有叙,济济隽乂,我弼我辅。伊予小子,恃宠骄盈,举挂时网,动乱国经。作藩作屏,先轨是堕,傲我皇使,犯我朝仪。国有典刑,我削我绌,将寘于理,元凶是率。明明天子,时笃同类,不忍我刑,暴之朝肆,违彼执宪,哀予小子。改封兖邑,于河之滨,股肱弗置,有君无臣,荒淫之阙,谁弼予身?茕茕仆夫,于彼冀方,嗟予小子,乃罹斯殃。赫赫天子,恩不遗物,冠我玄冕,要我朱绂。朱绂光大,使我荣华,剖符授玉,王爵是加。仰齿金玺,俯执圣策,皇恩过隆,祗承怵惕。咨我小子,顽凶是婴,逝惭陵墓,存愧阙廷。匪敢傲德,实恩是恃,威灵改加,足以没齿。昊天罔极,性命不图,常惧颠沛,抱罪黄垆。愿蒙矢石,建旗东岳,庶立豪氂,微功自赎。危躯授命,知足免戾,甘赴江、湘,奋戈吴、越。天启其衷,得会京畿,迟奉圣颜,如渴如饥。心之云慕,怆矣其悲,天高听卑,皇肯照微!“又曰:“肃承明诏,应会皇都,星陈夙驾,秣马脂车。命彼掌徒,肃我征旅,朝发鸾台,夕宿兰渚。芒芒原隰,祁祁士女,经彼公田,乐我稷黍。爰有樛木,重阴匪息;虽有糇粮,饥不遑食。望城不过,面邑匪游,仆夫警策,平路是由。玄驷蔼蔼,扬镳氵剽沫;流风翼衡,轻云承盖。涉涧之滨,缘山之隈,遵彼河浒,黄阪是阶。西济关谷,或降或升;騑骖倦路,再寝再兴。将朝圣皇,匪敢晏宁;弭节长骛,指日遄征。前驱举燧,后乘抗旌;轮不辍运,鸾无废声。爰暨帝室,税此西墉;嘉诏未赐,朝觐莫从。仰瞻城阈,俯惟阙廷;长怀永慕,忧心如酲。”

帝嘉其辞义,优诏答勉之。

六年,帝东征,还过雍丘,幸植宫,增户五百。太和元年,徙封浚仪。二年,复还雍丘。植常自愤怨,抱利器而无所施,上疏求自试曰:

臣闻士之生世,入则事父,出则事君;事父尚於荣亲,事君贵於兴国。故慈父不能爱无益之子,仁君不能畜无用之臣。夫论德而授官者,成功之君也;量能而受爵者,毕命之臣也。故君无虚授,臣无虚受;虚授谓之谬举,虚受谓之尸禄,诗之“素餐”所由作也。昔二虢不辞两国之任,其德厚也;旦、奭不让燕、鲁之封,其功大也。今臣蒙国重恩,三世于今矣。正值陛下升平之际,沐浴圣泽,潜润德教,可谓厚幸矣。而窃位东藩,爵在上列,身被轻暖,口厌百味,目极华靡,耳倦丝竹者,爵重禄厚之所致也。退念古之授爵禄者,有异於此,皆以功勤济国,辅主惠民。今臣无德可述,无功可纪,若此终年无益国朝,将挂风人“彼其”之讥。是以上惭玄冕,俯愧朱绂。

方今天下一统,九州晏如,而顾西有违命之蜀,东有不臣之吴,使边境未得脱甲,谋士未得高枕者,诚欲混同宇内以致太和也。故启灭有扈而夏功昭,成克商、奄而周德著。今陛下以圣明统世,将欲卒文、武之功,继成、康之隆,简贤授能,以方叔、召虎之臣镇御四境,为国爪牙者,可谓当矣。然而高鸟未挂於轻缴,渊鱼未县於钩饵者,恐钓射之术或未尽也。昔耿弇不俟光武,亟击张步,言不以贼遗於君父。故车右伏剑於鸣毂,雍门刎首於齐境,若此二士,岂恶生而尚死哉?诚忿其慢主而陵君也。夫君之宠臣,欲以除患兴利;臣之事君,必以杀身靖乱,以功报主也。昔贾谊弱冠,求试属国,请系单于之颈而制其命;终军以妙年使越,欲得长缨缨其王,羁致北阙。此二臣,岂好为夸主而耀世哉?志或郁结,欲逞其才力,输能於明君也。昔汉武为霍去病治第,辞曰:“匈奴未灭,臣无以家为!”夫忧国忘家,捐躯济难,忠臣之志也。今臣居外,非不厚也,而寝不安席,食不遑味者,伏以二方未克为念。

伏见先武皇帝武臣宿将,年耆即世者有闻矣。虽贤不乏世,宿将旧卒,犹习战陈,窃不自量,志在效命,庶立毛发之功,以报所受之恩。若使陛下出不世之诏,效臣锥刀之用,使得西属大将军,当一校之队,若东属大司马,统偏舟之任,必乘危蹈险,骋舟奋骊,突刃触锋,为士卒先。虽未能禽权馘亮,庶将虏其雄率,歼其丑类,必效须臾之捷,以灭终身之愧,使名挂史笔,事列朝策。虽身分蜀境,首县吴阙,犹生之年也。如微才弗试,没世无闻,徒荣其躯而丰其体,生无益於事,死无损於数,虚荷上位而忝重禄,禽息鸟视,终於白首,此徒圈牢之养物,非臣之所志也。流闻东军失备,师徒小衄,辍食弃餐,奋袂攘衽,抚剑东顾,而心已驰於吴会矣。

臣昔从先武皇帝南极赤岸,东临沧海,西望玉门,北出玄塞,伏见所以行军用兵之势,可谓神妙矣。故兵者不可豫言,临难而制变者也。志欲自效於明时,立功於圣世。每览史籍,观古忠臣义士,出一朝之命,以徇国家之难,身虽屠裂,而功铭著於鼎锺,名称垂於竹帛,未尝不拊心而叹息也。臣闻明主使臣,不废有罪。故奔北败军之将用,秦、鲁以成其功;绝缨盗马之臣赦,楚、赵以济其难。臣窃感先帝早崩,威王弃世,臣独何人,以堪长久!常恐先朝露,填沟壑,坟土未乾,而身名并灭。臣闻骐骥长鸣,则伯乐照其能;卢狗悲号,则韩国知其才。是以效之齐、楚之路,以逞千里之任;试之狡兔之捷,以验搏噬之用。今臣志狗马之微功,窃自惟度,终无伯乐、韩国之举,是以於邑而窃自痛者也。

夫临搏而企竦,闻乐而窃抃者,或有赏音而识道也。昔毛遂,赵之陪隶,犹假锥囊之喻,以寤主立功,何况巍巍大魏多士之朝,而无慷慨死难之臣乎!夫自衒自媒者,士女之丑行也。干时求进者,道家之明忌也。而臣敢陈闻於陛下者,诚与国分形同气,忧患共之者也。冀以尘雾之微补益山海,荧烛末光增辉日月,是以敢冒其丑而献其忠。

三年,徙封东阿。五年,复上疏求存问亲戚,因致其意曰:

臣闻天称其高者,以无不覆;地称其广者,以无不载;日月称其明者,以无不照;江海称其大者,以无不容。故孔子曰:“大哉尧之为君!惟天为大,惟尧则之。”夫天德之於万物,可谓弘广矣。盖尧之为教,先亲后疏,自近及远。其传曰:“克明峻德,以亲九族;九族既睦,平章百姓。”及周之文王亦崇厥化,其诗曰:“刑于寡妻,至于兄弟,以御于家邦。”是以雍雍穆穆。风人咏之。昔周公吊管、蔡之不咸,广封懿亲以藩屏王室,传曰:“周之宗盟,异姓为后。”诚骨肉之恩爽而不离,亲亲之义实在敦固,未有义而后其君,仁而遗其亲者也。

伏惟陛下资帝唐钦明之德,体文王翼翼之仁,惠洽椒房,恩昭九族,群后百寮,番休递上,执政不废於公朝,下情得展於私室,亲理之路通,庆吊之情展,诚可谓恕己治人,推惠施恩者矣。至於臣者,人道绝绪,禁锢明时,臣窃自伤也。不敢过望交气类,脩人事,叙人伦。近且婚媾不通,兄弟乖绝,吉凶之问塞,庆吊之礼废,恩纪之违,甚於路人,隔阂之异,殊於胡越。今臣以一切之制,永无朝觐之望,至於注心皇极,结情紫闼,神明知之矣。然天实为之,谓之何哉!退唯诸王常有戚戚具尔之心,愿陛下沛然垂诏,使诸国庆问,四节得展,以叙骨肉之欢恩。全怡怡之笃义。妃妾之家,膏沐之遗,岁得再通,齐义於贵宗,等惠於百司,如此,则古人之所叹,风雅之所咏,复存於圣世矣。

臣伏自惟省,无锥刀之用。及观陛下之所拔授,若以臣为异姓,窃自料度,不后於朝士矣。若得辞远游,戴武弁,解朱组,佩青绂,驸马、奉车,趣得一号,安宅京室,执鞭珥笔,出从华盖,入侍辇毂,承答圣问,拾遗左右,乃臣丹诚之至愿,不离於梦想者也。远慕鹿鸣君臣之宴,中咏常棣匪他之诫,下思伐木友生之义,终怀蓼莪罔极之哀;每四节之会,块然独处,左右惟仆隶,所对惟妻子,高谈无所与陈,发义无所与展,未尝不闻乐而拊心,临觞而叹息也。臣伏以为犬马之诚不能动人,譬人之诚不能动天。崩城、陨霜,臣初信之,以臣心况,徒虚语耳。若葵藿之倾叶,太阳虽不为之回光,然向之者诚也。窃自比於葵藿,若降天地之施,垂三光之明者,实在陛下。

臣闻文子曰:“不为福始,不为祸先。”今之否隔,友于同忧,而臣独倡言者,窃不愿於圣世使有不蒙施之物。有不蒙施之物,必有惨毒之怀,故柏舟有“天只“之怨,谷风有“弃予”之叹。故伊尹耻其君不为尧舜,孟子曰:“不以舜之所以事尧事其君者,不敬其君者也。”臣之愚蔽,固非虞、伊,至於欲使陛下崇光被时雍之美,宣缉熙章明之德者,是臣慺慺之诚,窃所独守,实怀鹤立企伫之心。敢复陈闻者,冀陛下傥发天聪而垂神听也。

诏报曰:“盖教化所由,各有隆弊,非皆善始而恶终也,事使之然。故夫忠厚仁极草木,则行苇之诗作;恩泽衰薄,不亲九族,则角弓之章刺。今令诸国兄弟,情理简怠,妃妾之家,膏沐疏略,朕纵不能敦而睦之,王援古喻义备悉矣,何言精诚不足以感通哉?夫明贵贱,崇亲亲,礼贤良,顺少长,国之纲纪,本无禁固诸国通问之诏也,矫枉过正,下吏惧谴,以至於此耳。已敕有司,如王所诉。”

植复上疏陈审举之义,曰:

臣闻天地协气而万物生,君臣合德而庶政成;五帝之世非皆智,三季之末非皆愚,用与不用,知与不知也。既时有举贤之名,而无得贤之实,必各援其类而进矣。谚曰:“相门有相,将门有将。”夫相者,文德昭者也;将者,武功烈者也。文德昭,则可以匡国朝,致雍熙,稷、契、夔、龙是也;武功烈,则所以征不庭,威四夷,南仲、方叔是矣。昔伊尹之为媵臣,至贱也,吕尚之处屠钓,至陋也,及其见举於汤武、周文,诚道合志同,玄谟神通,岂复假近习之荐,因左右之介哉。书曰:“有不世之君,必能用不世之臣;用不世之臣,必能立不世之功。”殷周二王是矣。若夫龌龊近步,遵常守故,安足为陛下言哉?故阴阳不和,三光不畅,官旷无人,庶政不整者,三司之责也。疆埸骚动,方隅内侵,没军丧众,干戈不息者,边将之忧也。岂可虚荷国宠而不称其任哉?故任益隆者负益重,位益高者责益深,书称“无旷庶官”,诗有“职思其忧”,此其义也。

陛下体天真之淑圣,登神机以继统,冀闻康哉之歌,偃武行文之美。而数年以来,水旱不时,民困衣食,师徒之发,岁岁增调,加东有覆败之军,西有殪没之将,至使蚌蛤浮翔於淮、泗,鼲鼬讙哗於林木。臣每念之,未尝不辍食而挥餐,临觞而搤腕矣。昔汉文发代,疑朝有变,宋昌曰:“内有朱虚、东牟之亲,外有齐、楚、淮南、琅邪,此则磐石之宗,愿王勿疑。”臣伏惟陛下远览姬文二虢之援,中虑周成召、毕之辅,下存宋昌磐石之固。昔骐骥之於吴阪,可谓困矣,及其伯乐相之,孙邮御之,形体不劳而坐取千里。盖伯乐善御马,明君善御臣;伯乐驰千里,明君致太平;诚任贤使能之明效也。若朝司惟良,万机内理,武将行师,方难克弭。陛下可得雍容都城,何事劳动銮驾,暴露於边境哉?

臣闻羊质虎皮,见草则悦,见豺则战,忘其皮之虎也。今置将不良,有似於此。故语曰:“患为之者不知,知之者不得为也。”昔乐毅奔赵,心不忘燕;廉颇在楚,思为赵将。臣生乎乱,长乎军,又数承教于武皇帝,伏见行师用兵之要,不必取孙、吴而闇与之合。窃揆之於心,常愿得一奉朝觐,排金门,蹈玉陛,列有职之臣,赐须臾之问,使臣得一散所怀,摅舒蕴积,死不恨矣。

被鸿胪所下发士息书,期会甚急。又闻豹尾已建,戎轩骛驾,陛下将复劳玉躬,扰挂神思。臣诚竦息,不遑宁处。愿得策马执鞭,首当尘露,撮风后之奇,接孙、吴之要,追慕卜商起予左右,效命先驱,毕命轮毂,虽无大益,冀有小补。然天高听远,情不上通,徒独望青云而拊心,仰高天而叹息耳。屈平曰:“国有骥而不知乘,焉皇皇而更索!”昔管、蔡放诛,周、召作弼;叔鱼陷刑,叔向匡国。三监之衅,臣自当之;二南之辅,求必不远。华宗贵族,藩王之中,必有应斯举者。故传曰:“无周公之亲,不得行周公之事。”唯陛下少留意焉。

近者汉氏广建藩王,丰则连城数十,约则飨食祖祭而已,未若姬周之树国,五等之品制之。若扶苏之谏始皇,淳于越之难周青臣,可谓知时变矣。夫能使天下倾耳注目者,当权者是矣,故谋能移主,威能慑下。豪右执政,不在亲戚;权之所在,虽疏必重,势之所去,虽亲必轻,盖取齐者田族,非吕宗也。分晋者赵、魏,非姬姓也。唯陛下察之。苟吉专其位,凶离其患者,异姓之臣也。欲国之安,祈家之贵,存共其荣,没同其祸者,公族之臣也。今反公族疏而异姓亲,臣窃惑焉。

臣闻孟子曰:“君子穷则独善其身,达则兼善天下。”今臣与陛下践冰履炭,登山浮涧,寒温燥湿,高下共之,岂得离陛下哉?不胜愤懑,拜表陈情。若有不合,乞且藏之书府,不便灭弃,臣死之后,事或可思。若有豪釐少挂圣意者,乞出之朝堂,使夫博古之士,纠臣表之不合义者。如是,则臣愿足矣。

帝辄优文答报。

其年冬,诏诸王朝六年正月。其二月,以陈四县封植为陈王,邑三千五百户。植每欲求别见独谈,论及时政,幸冀试用,终不能得。既还,怅然绝望。时法制,待藩国既自峻迫,寮属皆贾竖下才,兵人给其残老,大数不过二百人。又植以前过,事事复减半,十一年中而三徙都,常汲汲无欢,遂发疾薨,时年四十一。遗令薄葬。以小子志,保家之主也,欲立之。初,植登鱼山,临东阿,喟然有终焉之心,遂营为墓。子志嗣,徙封济北王。景初中诏曰:“陈思王昔虽有过失,既克己慎行,以补前阙,且自少至终,篇籍不离於手,诚难能也。其收黄初中诸奏植罪状,公卿已下议尚书、秘书、中书三府、大鸿胪者皆削除之。撰录植前后所著赋颂诗铭杂论凡百馀篇,副藏内外。”志累增邑,并前九百九十户。

萧怀王熊,早薨。黄初二追封谥萧怀公。太和三年,又追封爵为王。青龙二年,子哀王炳嗣,食邑二千五百户。六年薨,无子,国除。

评曰:任城武艺壮猛,有将领之气。陈思文才富艳,足以自通后叶,然不能克让远防,终致携隙。传曰“楚则失之矣,而齐亦未为得也”,其此之谓欤!

\part{魏书二十}
\chapter{武文世王公传第二十}

武皇帝二十五男:卞皇后生文皇帝、任城威王彰、陈思王植、萧怀王熊,刘夫人生丰愍王昂、相殇王铄,环夫人生邓哀王冲、彭城王据、燕王宇,杜夫人生沛穆王林、中山恭王衮,秦夫人生济阳怀王玹、陈留恭王峻,尹夫人生范阳闵王矩,王昭仪生赵王幹,孙姬生临邑殇公子上、楚王彪、刚殇公子勤,李姬生谷城殇公子乘、郿戴公子整、灵殇公子京,周姬生樊安公均,刘姬生广宗殇公子棘,宋姬生东平灵王徽,赵姬生乐陵王茂。

丰愍王昂字子脩。弱冠举孝廉。随太祖南征,为张绣所害。无子。黄初二年追封,谥曰丰悼公。三年,以樊安公均子琬奉昂后,封中都公。其年徙封长子公。五年,追加昂号曰丰悼王。太和三年改昂谥曰愍王。嘉平六年,以琬袭昂爵为丰王。正元、景元中,累增邑,并前二千七百户。琬薨,谥曰恭王。子廉嗣。

相殇王铄,早薨,太和三年追封谥。青龙元年,子愍王潜嗣,其年薨。二年,子怀王偃嗣,邑二千五百户,四年薨。无子,国除。正元二年,以乐陵王茂子阳都乡公竦继铄后。

邓哀王冲字仓舒。少聪察岐嶷,生五六岁,智意所及,有若成人之智。时孙权曾致巨象,太祖欲知其斤重,访之群下,咸莫能出其理。冲曰:“置象大船之上,而刻其水痕所至,称物以载之,则校可知矣。”太祖大悦,即施行焉。时军国多事,用刑严重。太祖马鞍在库,而为鼠所齧,库吏惧必死,议欲面缚首罪,犹惧不免。冲谓曰:“待三日中,然后自归。”冲於是以刀穿单衣,如鼠齧者,谬为失意,貌有愁色。太祖问之,冲对曰:“世俗以为鼠齧衣者,其主不吉。今单衣见齧,是以忧戚。”太祖曰:“此妄言耳,无所苦也。”俄而库吏以齧鞍闻,太祖笑曰:“儿衣在侧,尚齧,况鞍县柱乎?”一无所问。冲仁爱识达,皆此类也。凡应罪戮,而为冲微所辨理,赖以济宥者,前后数十。太祖数对群臣称述,有欲传后意。年十三,建安十三年疾病,太祖亲为请命。及亡,哀甚。文帝宽喻太祖,太祖曰:“此我之不幸,而汝曹之幸也。”言则流涕,为聘甄氏亡女与合葬,赠骑都尉印绶,命宛侯据子琮奉冲后。二十二年,封琮为邓侯。黄初二年,追赠谥冲曰邓哀侯,又追加号为公。三年,进琮爵,徙封冠军公。四年,徙封己氏公。太和五年,加冲号曰邓哀王。景初元年,琮坐於中尚方作禁物,削户三百,贬爵为都乡侯。三年,复为己氏公。正始七年,转封平阳公。景初、正元、景元中,累增邑,并前千九百户。

彭城王据,建安十六年封范阳侯。二十二年,徙封宛侯。黄初二年,进爵为公。三年,为章陵王,其年徙封义阳。文帝以南方下湿,又以环太妃彭城人,徙封彭城。又徙封济阴。五年,诏曰:“先王建国,随时而制。汉祖增秦所置郡,至光武以天下损耗,并省郡县。以今比之,益不及焉。其改封诸王,皆为县王。“据改封定陶县。太和六年,改封诸王,皆以郡为国,据复封彭城。景初元年,据坐私遣人诣中尚方作禁物,削县二千户。三年,复所削户邑。正元、景元中累增邑,并前四千六百户。

燕王宇字彭祖。建安十六年,封都乡侯。二十二年,改封鲁阳侯。黄初二年,进爵为公。三年,为下邳王。五年,改封单父县。太和六年,改封燕王。明帝少与宇同止,常爱异之。及即位,宠赐与诸王殊。青龙三年,徵入朝。景初元年,还邺。二年夏,复徵诣京都。冬十二月,明帝疾笃,拜宇为大将军,属以后事。受署四日,宇深固让;帝意亦变,遂免宇官。三年夏,还邺。景初、正元、景元中,累增邑,并前五千五百户。常道乡公奂,宇之子,入继大宗。

沛穆王林,建安十六年封饶阳侯。二十二年,徙封谯。黄初二年,进爵为公。三年,为谯王。五年,改封谯县。七年,徙封鄄城。太和六年,改封沛。景初、正元、景元中,累增邑,并前四千七百户。林薨,子纬嗣。

中山恭王衮,建安二十一年封平乡侯。少好学,年十馀岁能属文。每读书,文学左右常恐以精力为病,数谏止之,然性所乐,不能废也。二十二年,徙封东乡侯,其年又改封赞侯。黄初二年,进爵为公,官属皆贺,衮曰:“夫生深宫之中,不知稼穑之艰难,多骄逸之失。诸贤既庆其休,宜辅其阙。”每兄弟游娱,衮独覃思经典。文学防辅相与言曰:“受诏察公举错,有过当奏,及有善,亦宜以闻,不可匿其美也。“遂共表称陈衮美。衮闻之,大惊惧,责让文学曰:“脩身自守,常人之行耳,而诸君乃以上闻,是適所以增其负累也。且如有善,何患不闻,而遽共如是,是非益我者。”其戒慎如此。三年,为北海王。其年,黄龙见邺西漳水,衮上书赞颂。诏赐黄金十斤,诏曰:“昔唐叔归禾,东平献颂,斯皆骨肉赞美,以彰懿亲。王研精坟典,耽味道真,文雅焕炳,朕甚嘉之。王其克慎明德,以终令闻。”四年,改封赞王。七年,徙封濮阳。太和二年就国,尚约俭,教敕妃妾纺绩织纴,习为家人之事。五年冬,入朝。六年,改封中山。

初,衮来朝,犯京都禁。青龙元年,有司奏衮。诏曰:“王素敬慎,邂逅至此,其以议亲之典议之。”有司固执。诏削县二,户七百五十。衮忧惧,戒敕官属愈谨。帝嘉其意,二年,复所削县。三年秋,衮得疾病,诏遣太医视疾,殿中、虎贲赍手诏、赐珍膳相属,又遣太妃、沛王林并就省疾。衮疾困,敕令官属曰:“吾寡德忝宠,大命将尽。吾既好俭,而圣朝著终诰之制,为天下法。吾气绝之日,自殡及葬,务奉诏书。昔卫大夫蘧瑗葬濮阳,吾望其墓,常想其遗风,愿讬贤灵以弊发齿,营吾兆域,必往从之。礼:男子不卒妇人之手。亟以时成东堂。”堂成,名之曰遂志之堂,舆疾往居之。又令世子曰:“汝幼少,未闻义方,早为人君,但知乐,不知苦;不知苦,必将以骄奢为失也。接大臣,务以礼。虽非大臣,老者犹宜答拜。事兄以敬,恤弟以慈;兄弟有不良之行,当造膝谏之。谏之不从,流涕喻之;喻之不改,乃白其母。若犹不改,当以奏闻,并辞国土。与其守宠罹祸,不若贫贱全身也。此亦谓大罪恶耳,其微过细故,当掩覆之。嗟尔小子,慎脩乃身,奉圣朝以忠贞,事太妃以孝敬。闺闱之内,奉令於太妃;阃阈之外,受教於沛王。无怠乃心,以慰予灵。”其年薨。诏沛王林留讫葬,使大鸿胪持节典护丧事,宗正吊祭,赠赗甚厚。凡所著文章二万馀言,才不及陈思王而好与之侔。子孚嗣。景初、正元、景元中,累增邑,并前三千四百户。

济阳怀王玹,建安十六年封西乡侯。早薨,无子。二十年,以沛王林子赞袭玹爵邑,早薨,无子。文帝复以赞弟壹绍玹后。黄初二年,改封济阳侯。四年,进爵为公。太和四年,追进玹爵,谥曰怀公。六年,又进号曰怀王,追谥赞曰西乡哀侯。壹薨,谥曰悼公。子恒嗣。景初、正元、景元中,累增邑,并前千九百户。

陈留恭王峻字子安,建安二十一年封郿侯。二十二年,徙封襄邑。黄初二年,进爵为公。三年,为陈留王。五年,改封襄邑县。太和六年,又封陈留。甘露四年薨。子澳嗣。景初、正元、景元中,累增邑,并前四千七百户。

范阳闵王矩,早薨,无子。建安二十二年,以樊安公均子敏奉矩后,封临晋侯。黄初三年,追封谥矩为范阳闵公。五年,改封敏范阳王。七年,徙封句阳,太和六年,追进矩号曰范阳闵王,改封敏琅邪王。景初、正元、景元中,累增邑,并前三千四百户。敏薨,谥曰原王。子焜嗣。

赵王幹,建安二十年封高平亭侯。二十二年,徙封赖亭侯。其年改封弘农侯。黄初二年,进爵,徙封燕公。三年,为河间王。五年,改封乐城县。七年,徙封钜鹿。太和六年,改封赵王。幹母有宠於太祖。及文帝为嗣,幹母有力。文帝临崩,有遗诏,是以明帝常加恩意。青龙二年,私通宾客,为有司所奏,赐幹玺书诫诲之,曰:“易称'开国承家,小人勿用',诗著'大车惟尘'之诫。自太祖受命创业,深睹治乱之源,鉴存亡之机,初封诸侯,训以恭慎之至言,辅以天下之端士,常称马援之遗诫,重诸侯宾客交通之禁,乃使与犯妖恶同。夫岂以此薄骨肉哉?徒欲使子弟无过失之愆,士民无伤害之悔耳。高祖践阼,祗慎万机,申著诸侯不朝之令。朕感诗人常棣之作,嘉采菽之义,亦缘诏文曰'若有诏得诣京都',故命诸王以朝聘之礼。而楚、中山并犯交通之禁,赵宗、戴捷咸伏其辜。近东平王复使属官殴寿张吏,有司举奏,朕裁削县。今有司以曹纂、王乔等因九族时节,集会王家,或非其时,皆违禁防。朕惟王幼少有恭顺之素,加受先帝顾命,欲崇恩礼,延乎后嗣,况近在王之身乎?且自非圣人,孰能无过?已诏有司宥王之失。古人有言:'戒慎乎其所不睹,恐惧乎其所弗闻,莫见乎隐,莫显乎微,故君子慎其独焉。'叔父兹率先圣之典,以纂乃先帝之遗命,战战兢兢,靖恭厥位,称朕意焉。”景初、正元、景元中,累增邑,并前五千户。

临邑殇公子上,早薨。太和五年,追封谥。无后。

楚王彪字朱虎。建安二十一年,封寿春侯。黄初二年,进爵,徙封汝阳公。三年,封弋阳王。其年徙封吴王。五年,改封寿春县。七年,徙封白马。太和五年冬,朝京都。六年,改封楚。初,彪来朝,犯禁,青龙元年,为有司所奏,诏削县三,户千五百。二年,大赦,复所削县。景初三年,增户五百,并前三千户。嘉平元年,兖州刺史令狐愚与太尉王凌谋迎彪都许昌。语在凌传。乃遣傅及侍御史就国案验,收治诸相连及者。廷尉请徵彪治罪。於是依汉燕王旦故事,使兼廷尉大鸿胪持节赐彪玺书切责之,使自图焉。彪乃自杀。妃及诸子皆免为庶人,徙平原。彪之官属以下及监国谒者,坐知情无辅导之义,皆伏诛。国除为淮南郡。正元元年诏曰:“故楚王彪,背国附奸,身死嗣替,虽自取之,犹哀矜焉。夫含垢藏疾,亲亲之道也,其封彪世子嘉为常山真定王。”景元元年,增邑,并前二千五百户。

刚殇公子勤,早薨。太和五年追封谥。无后。

谷城殇公子乘,早薨。太和五年追封谥。无后。

郿戴公子整,奉从叔父郎中绍后。建安二十二年,封郿侯。二十三年薨。无子。黄初二年追进爵,谥曰戴公。以彭城王据子范奉整后。三年,封平氏侯。四年,徙封成武。太和三年,进爵为公。青龙三年薨。谥曰悼公。无后。四年,诏以范弟东安乡公阐为郿公,奉整后。正元、景元中、累增邑,并前千八百户。

灵殇公子京,早薨。太和五年追封谥。无后。

樊安公均,奉叔父蓟恭公彬后。建安二十二年,封樊侯。二十四年薨。子抗嗣。黄初二年,追进公爵,谥曰安公。三年,徙封抗蓟公。四年,徙封屯留公。景初元年薨,谥曰定公。子谌嗣。景初、正元、景元中,累增邑,并前千九百户。

广宗殇公子棘,早薨。太和五年追封谥。无后。

东平灵王徽,奉叔公朗陵哀侯玉后。建安二十二年,封历城侯。黄初二年,进爵为公。三年,为庐江王。四年,徙封寿张王。五年,改封寿张县。太和六年,改封东平。青龙二年,徽使官属挝寿张县吏,为有司所奏。诏削县一,户五百。其年复所削县。正始三年薨。子翕嗣。景初、正元、景元中,累增邑,并前三千四百户。

乐陵王茂,建安二十二年封万岁亭侯。二十三年,改封平舆侯。黄初三年,进爵,徙封乘氏公。七年,徙封中丘。茂性傲佷,少无宠於太祖。及文帝世,又独不王。太和元年,徙封聊城公,其年为王。诏曰:“昔象之为虐至甚,而大舜犹侯之有庳。近汉氏淮南、阜陵,皆为乱臣逆子,而犹或及身而复国,或至子而锡土。有虞建之於上古,汉文、明、章行之乎前代,斯皆敦叙亲亲之厚义也。聊城公茂少不闲礼教,长不务善道。先帝以为古之立诸侯也,皆命贤者,故姬姓有未必侯者,是以独不王茂。太皇太后数以为言。如闻茂顷来少知悔昔之非,欲脩善将来。君子与其进,不保其往也。今封茂为聊城王,以慰太皇太后下流之念。”六年,改封曲阳王。正始三年,东平灵王薨,茂称嗌痛,不肯发哀,居处出入自若。有司奏除国土,诏削县一,户五百。五年,徙封乐陵,诏以茂租奉少,诸子多,复所削户,又增户七百。嘉平、正元、景元中,累增邑,并前五千户。

文皇帝九男:甄氏皇后生明帝,李贵人生赞哀王协,潘淑媛生北海悼王蕤,朱淑媛生东武阳怀王鉴,仇昭仪生东海定王霖,徐姬生元城哀王礼,苏姬生邯郸怀王邕,张姬生清河悼王贡,宋姬生广平哀王俨。

赞哀王协,早薨。太和五年追封谥曰经殇公。青龙二年,更追改号谥。三年,子殇王寻嗣。景初三年,增户五百,并前三千户。正始九年薨。无子。国除。

北海悼王蕤,黄初七年,明帝即位,立为阳平县王。太和六年,改封北海。青龙元年薨。二年,以琅邪王子赞奉蕤后,封昌乡公。景初二年,立为饶安王。正始七年,徙封文安。正元、景元中,累增邑,并前三千五百户。

东武阳怀王鉴,黄初六年立。其年薨。青龙三年赐谥。无子。国除。

东海定王霖,黄初三年立为河东王。六年,改封馆陶县。明帝即位,以先帝遗意,爱宠霖异於诸国。而霖性粗暴,闺门之内,婢妾之间,多所残害。太和六年,改封东海。嘉平元年薨。子启嗣。景初、正元、景元中,累增邑,并前六千二百户。高贵乡公髦,霖之子也,入继大宗。

元城哀王礼,黄初二年封秦公,以京兆郡为国。三年,改为京兆王。六年,改封元城王。太和三年薨。五年,以任城王楷子悌嗣礼后。六年,改封梁王。景初、正元、景元中,累增邑,并前四千五百户。

邯郸怀王邕,黄初二年封淮南公,以九江郡为国。三年,进为淮南王。四年,改封陈。六年,改封邯郸。太和三年薨。五年,以任城王楷子温嗣邕后。六年,改封鲁阳。景初、正元、景元中,累增邑,并前四千四百户。

清河悼王贡,黄初三年封。四年薨。无子。国除。

广平哀王俨,黄初三年封。四年薨。无子。国除。

评曰:魏氏王公,既徒有国土之名,而无社稷之实,又禁防壅隔,同於囹圄;位号靡定,大小岁易;骨肉之恩乖,常棣之义废。为法之弊,一至于此乎!

\part{魏书二十一}
\chapter{王卫二刘傅传第二十一}



王粲字仲宣,山阳高平人也。曾祖父龚,祖父畅,皆为汉三公。父谦,为大将军何进长史。进以谦名公之胄,欲与为婚,见其二子,使择焉。谦弗许。以疾免,卒于家。

献帝西迁,粲徙长安,左中郎将蔡邕见而奇之。时邕才学显著,贵重朝廷,常车骑填巷,宾客盈坐。闻粲在门,倒屣迎之。粲至,年既幼弱,容状短小,一坐尽惊。邕曰:“此王公孙也,有异才,吾不如也。吾家书籍文章,尽当与之。”年十七,司徒辟,诏除黄门侍郎,以西京扰乱,皆不就。乃之荆州依刘表。表以粲貌寝而体弱通侻,不甚重也。表卒,粲劝表子琮,令归太祖。太祖辟为丞相掾,赐爵关内侯。太祖置酒汉滨,粲奉觞贺曰:“方今袁绍起河北,仗大众,志兼天下,然好贤而不能用,故奇士去之。刘表雍容荆楚,坐观时变,自以为西伯可规。士之避乱荆州者,皆海内之俊杰也;表不知所任,故国危而无辅。明公定冀州之日,下车即缮其甲卒,收其豪杰而用之,以横行天下;及平江、汉,引其贤俊而置之列位,使海内回心,望风而愿治,文武并用,英雄毕力,此三王之举也。”后迁军谋祭酒。魏国既建,拜侍中。博物多识,问无不对。时旧仪废弛,兴造制度,粲恒典之。

初,粲与人共行,读道边碑,人问曰:“卿能闇诵乎?”曰:“能。”因使背而诵之,不失一字。观人围棋,局坏,粲为覆之。棋者不信,以帊盖局,使更以他局为之。用相比校,不误一道。其强记默识如此。性善算,作算术,略尽其理。善属文,举笔便成,无所改定,时人常以为宿构;然正复精意覃思,亦不能加也。著诗、赋、论、议垂六十篇。建安二十一年,从征吴。二十二年春,道病卒,时年四十一。粲二子,为魏讽所引,诛。后绝。

始文帝为五官将,及平原侯植皆好文学。粲与北海徐幹字伟长、广陵陈琳字孔璋、陈留阮瑀字元瑜、汝南应玚字德琏、东平刘桢字公幹并见友善。

幹为司空军谋祭酒掾属,五官将文学。

琳前为何进主簿。进欲诛诸宦官,太后不听,进乃召四方猛将,并使引兵向京城,欲以劫恐太后。琳谏进曰:“易称'即鹿无虞'。谚有'掩目捕雀'。夫微物尚不可欺以得志,况国之大事,其可以诈立乎?今将军总皇威,握兵要,龙骧虎步,高下在心;以此行事,无异於鼓洪炉以燎毛发。但当速发雷霆,行权立断,违经合道,天人顺之;而反释其利器,更徵於他。大兵合聚,强者为雄,所谓倒持干戈,授人以柄;功必不成,祇为乱阶。”进不纳其言,竟以取祸。琳避难冀州,袁绍使典文章。袁氏败,琳归太祖。太祖谓曰:“卿昔为本初移书,但可罪状孤而已,恶恶止其身,何乃上及父祖邪?”琳谢罪,太祖爱其才而不咎。

瑀少受学於蔡邕。建安中都护曹洪欲使掌书记,瑀终不为屈。太祖并以琳、瑀为司空军谋祭酒,管记室,军国书檄,多琳、瑀所作也。琳徙门下督,瑀为仓曹掾属。

玚、桢各被太祖辟为丞相掾属。玚转为平原侯庶子,后为五官将文学。桢以不敬被刑,刑竟署吏。咸著文赋数十篇。

瑀以十七年卒。幹、琳、玚、桢二十二年卒。文帝书与元城令吴质曰:“昔年疾疫,亲故多离其灾,徐、陈、应、刘,一时俱逝。观古今文人,类不护细行,鲜能以名节自立。而伟长独怀文抱质,恬淡寡欲,有箕山之志,可谓彬彬君子矣。著中论二十馀篇,辞义典雅,足传于后。德琏常斐然有述作意,其才学足以著书,美志不遂,良可痛惜!孔璋章表殊健,微为繁富。公幹有逸气,但未遒耳。元瑜书记翩翩,致足乐也。仲宣独自善於辞赋,惜其体弱,不起其文;至於所善,古人无以远过也。昔伯牙绝弦於锺期,仲尼覆醢于子路,痛知音之难遇,伤门人之莫逮也。诸子但为未及古人,自一时之俊也。”

自颍川邯郸淳、繁钦、陈留路粹、沛国丁仪、丁廙、弘农杨脩、河内荀纬等,亦有文采,而不在此七人之例。

玚弟璩,璩子贞,咸以文章显。璩官至侍中。贞咸熙中参相国军事。

瑀子籍,才藻艳逸,而倜傥放荡,行己寡欲,以庄周为模则。官至步兵校尉。

时又有谯郡嵇康,文辞壮丽,好言老、庄,而尚奇任侠。至景元中,坐事诛。

景初中,下邳桓威出自孤微,年十八而著浑舆经,依道以见意。从齐国门下书佐、司徒署吏,后为安成令。

吴质,济阴人,以文才为文帝所善,官至振威将军,假节都督河北诸军事,封列侯。

卫觊字伯儒,河东安邑人也。少夙成,以才学称。太祖辟为司空掾属,除茂陵令、尚书郎。太祖征袁绍,而刘表为绍援,关中诸将又中立。益州牧刘璋与表有隙,觊以治书侍御史使益州,令璋下兵以缀表军。至长安,道路不通,觊不得进,遂留镇关中。时四方大有还民,关中诸将多引为部曲,觊书与荀彧曰:“关中膏腴之地,顷遭荒乱,人民流入荆州者十万馀家,闻本土安宁,皆企望思归。而归者无以自业,诸将各竞招怀,以为部曲。郡县贫弱,不能与争,兵家遂强。一旦变动,必有后忧。夫盐,国之大宝也,自乱来散放,宜如旧置使者监卖,以其直益巿犁牛。若有归民,以供给之。勤耕积粟,以丰殖关中。远民闻之,必日夜竞还。又使司隶校尉留治关中以为之主,则诸将日削,官民日盛,此强本弱敌之利也。”彧以白太祖。太祖从之,始遣谒者仆射监盐官,司隶校尉治弘农。关中服从,乃白召觊还,稍迁尚书。魏国既建,拜侍中,与王粲并典制度。文帝即王位,徙为尚书。顷之,还汉朝为侍郎,劝赞禅代之义,为文诰之诏。文帝践阼,复为尚书,封阳吉亭侯。

明帝即位,进封閺乡侯,三百户。觊奏曰:“九章之律,自古所传,断定刑罪,其意微妙。百里长吏,皆宜知律。刑法者,国家之所贵重,而私议之所轻贱;狱吏者,百姓之所县命,而选用者之所卑下。王政之弊,未必不由此也。请置律博士,转相教授。“事遂施行。时百姓凋匮而役务方殷,觊上疏曰:“夫变情厉性,强所不能,人臣言之既不易,人主受之又艰难。且人之所乐者富贵显荣也,所恶者贫贱死亡也,然此四者,君上之所制也,君爱之则富贵显荣,君恶之则贫贱死亡;顺指者爱所由来,逆意者恶所从至也。故人臣皆争顺指而避逆意,非破家为国,杀身成君者,谁能犯颜色,触忌讳,建一言,开一说哉?陛下留意察之,则臣下之情可见矣。今议者多好悦耳,其言政治则比陛下於尧舜,其言征伐则比二虏於貍鼠。臣以为不然。昔汉文之时,诸侯强大,贾谊累息以为至危。况今四海之内,分而为三,群士陈力,各为其主。其来降者,未肯言舍邪就正,咸称迫於困急,是与六国分治,无以为异也。当今千里无烟,遗民困苦,陛下不善留意,将遂凋弊不可复振。礼,天子之器必有金玉之饰,饮食之肴必有八珍之味,至於凶荒,则彻膳降服。然则奢俭之节,必视世之丰约也。武皇帝之时,后宫食不过一肉,衣不用锦绣,茵蓐不缘饰,器物无丹漆,用能平定天下,遗福子孙。此皆陛下之所亲览也。当今之务,宜君臣上下,并用筹策,计校府库,量入为出。深思句践滋民之术,由恐不及,而尚方所造金银之物,渐更增广,工役不辍,侈靡日崇,帑藏日竭。昔汉武信求神仙之道,谓当得云表之露以餐玉屑,故立仙掌以承高露。陛下通明,每所非笑。汉武有求於露,而由尚见非,陛下无求於露而空设之;不益於好而糜费功夫,诚皆圣虑所宜裁制也。”觊历汉、魏,时献忠言,率如此。

受诏典著作,又为魏官仪,凡所撰述数十篇。好古文、鸟篆、隶草,无所不善。建安末,尚书右丞河南潘勖,黄初时,散骑常侍河内王象,亦与觊并以文章显。觊薨,谥曰敬侯。子瓘嗣。瓘咸熙中为镇西将军。

刘廙字恭嗣,南阳安众人也。年十岁,戏於讲堂上,颍川司马德操拊其头曰:“孺子,孺子,'黄中通理',宁自知不?”廙兄望之,有名於世,荆州牧刘表辟为从事。而其友二人,皆以谗毁,为表所诛。望之又以正谏不合,投传告归。廙谓望之曰:“赵杀鸣、犊,仲尼回轮。今兄既不能法柳下惠和光同尘於内,则宜模范蠡迁化於外。坐而自绝於时,殆不可也!”望之不从,寻复见害。廙惧,奔扬州,遂归太祖。太祖辟为丞相掾属,转五官将文学。文帝器之,命廙通草书。廙答书曰:“初以尊卑有逾,礼之常分也。是以贪守区区之节,不敢脩草。必如严命,诚知劳谦之素,不贵殊异若彼之高,而惇白屋如斯之好,苟使郭隗不轻於燕,九九不忽於齐,乐毅自至,霸业以隆。亏匹夫之节,成巍巍之美,虽愚不敏,何敢以辞?”魏国初建,为黄门侍郎。

太祖在长安,欲亲征蜀,廙上疏曰:“圣人不以智轻俗,王者不以人废言。故能成功於千载者,必以近察远,智周於独断者,不耻於下问,亦欲博采必尽於众也。且韦弦非能言之物,而圣贤引以自匡。臣才智闇浅,愿自比於韦弦。昔乐毅能以弱燕破大齐,而不能以轻兵定即墨者,夫自为计者虽弱必固,欲自溃者虽强必败也。自殿下起军以来,三十馀年,敌无不破,强无不服。今以海内之兵,百胜之威,而孙权负险於吴,刘备不宾於蜀。夫夷狄之臣,不当冀州之卒,权、备之籍,不比袁绍之业,然本初以亡,而二寇未捷,非闇弱於今而智武於昔也。斯自为计者,与欲自溃者异势耳。故文王伐崇,三驾不下,归而脩德,然后服之。秦为诸侯,所征必服,及兼天下,东向称帝,匹夫大呼而社稷用隳。是力毙於外,而不恤民於内也。臣恐边寇非六国之敌,而世不乏才,土崩之势,此不可不察也。天下有重得,有重失:势可得而我勤之,此重得也;势不可得而我勤之,此重失也。於今之计,莫若料四方之险,择要害之处而守之,选天下之甲卒,随方面而岁更焉。殿下可高枕於广夏,潜思於治国;广农桑,事从节约,脩之旬年,则国富民安矣。”太祖遂进前而报廙曰:“非但君当知臣,臣亦当知君。今欲使吾坐行西伯之德,恐非其人也。”

魏讽反,廙弟伟为讽所引,当相坐诛。太祖令曰:“叔向不坐弟虎,古之制也。”特原不问,徙署丞相仓曹属。廙上疏谢曰:“臣罪应倾宗,祸应覆族。遭乾坤之灵,值时来之运,扬汤止沸,使不燋烂;起烟於寒灰之上,生华於已枯之木。物不答施於天地,子不谢生於父母,可以死效,难用笔陈。”廙著书数十篇,及与丁仪共论刑礼,皆传於世。文帝即王位,为侍中,赐爵关内侯。黄初二年卒。无子。帝以弟子阜嗣。

刘劭字孔才,广平邯郸人也。建安中,为计吏,诣许。太史上言:“正旦当日蚀。“劭时在尚书令荀彧所,坐者数十人,或云当废朝,或云宜卻会。劭曰:“梓慎、裨灶,古之良史,犹占水火,错失天时。礼记曰诸侯旅见天子,及门不得终礼者四,日蚀在一。然则圣人垂制,不为变异豫废朝礼者,或灾消异伏,或推术谬误也。”彧善其言。敕朝会如旧,日亦不蚀。

御史大夫郗虑辟劭,会虑免,拜太子舍人,迁秘书郎。黄初中,为尚书郎、散骑侍郎。受诏集五经群书,以类相从,作皇览。明帝即位,出为陈留太守,敦崇教化,百姓称之。徵拜骑都尉,与议郎庾嶷、荀诜等定科令,作新律十八篇,著律略论。迁散骑常侍。时闻公孙渊受孙权燕王之号,议者欲留渊计吏,遣兵讨之,劭以为“昔袁尚兄弟归渊父康,康斩送其首,是渊先世之效忠也。又所闻虚实,未可审知。古者要荒未服,脩德而不征,重劳民也。宜加宽贷,使有以自新。”后渊果斩送权使张弥等首。劭尝作赵都赋,明帝美之,诏劭作许都、洛都赋。时外兴军旅,内营宫室,劭作二赋,皆讽谏焉。

青龙中,吴围合肥,时东方吏士皆分休,征东将军满宠表请中军兵,并召休将士,须集击之。劭议以为“贼众新至,心专气锐。宠以少人自战其地,若便进击,不必能制。宠求待兵,未有所失也。以为可先遣步兵五千,精骑三千,军前发,扬声进道,震曜形势。骑到合肥,疏其行队,多其旌鼓,曜兵城下,引出贼后,拟其归路,要其粮道。贼闻大军来,骑断其后,必震怖遁走,不战自破贼矣。”帝从之。兵比至合肥,贼果退还。

时诏书博求众贤。散骑侍郎夏侯惠荐劭曰:“伏见常侍刘劭,深忠笃思,体周於数,凡所错综,源流弘远,是以群才大小,咸取所同而斟酌焉。故性实之士服其平和良正,清静之人慕其玄虚退让,文学之士嘉其推步详密,法理之士明其分数精比,意思之士知其沈深笃固,文章之士爱其著论属辞,制度之士贵其化略较要,策谋之士赞其明思通微,凡此诸论,皆取適己所长而举其支流者也。臣数听其清谈,览其笃论,渐渍历年,服膺弥久,实为朝廷奇其器量。以为若此人者,宜辅翼机事,纳谋帏幄,当与国道俱隆,非世俗所常有也。惟陛下垂优游之听,使劭承清闲之欢,得自尽於前,则德音上通,辉耀日新矣。”

景初中,受诏作都官考课。劭上疏曰:“百官考课,王政之大较,然而历代弗务,是以治典阙而未补,能否混而相蒙。陛下以上圣之宏略,愍王纲之弛颓,神虑内鉴,明诏外发。臣奉恩旷然,得以启蒙,辄作都官考课七十二条,又作说略一篇。臣学寡识浅,诚不足以宣畅圣旨,著定典制。”又以为宜制礼作乐,以移风俗,著乐论十四篇,事成未上,会明帝崩,不施行。正始中,执经讲学,赐爵关内侯。凡所撰述,法论、人物志之类百馀篇。卒,追赠光禄勋。子琳嗣。

劭同时东海缪袭亦有才学,多所述叙,官至尚书、光禄勋。

袭友人山阳仲长统,汉末为尚书郎,早卒。著昌言,词佳可观省。

散骑常侍陈留苏林、光禄大夫京兆韦诞、陈郡太守任城孙该、郎中令河东杜挚等亦著文赋,颇传於世。

傅嘏字兰石,北地泥阳人,傅介子之后也。伯父巽,黄初中为侍中尚书。嘏弱冠知名,司空陈群辟为掾。时散骑常侍刘劭作考课法,事下三府。嘏难劭论曰:“盖闻帝制宏深,圣道奥远,苟非其才,则道不虚行,神而明之,存乎其人。暨乎王略亏颓而旷载罔缀,微言既没,六籍泯玷。何则?道弘致远而众才莫晞也。案劭考课论,虽欲寻前代黜陟之文,然其制度略以阙亡。礼之存者,惟有周典,外建侯伯,藩屏九服,内立列司,筦齐六职,土有恒贡,官有定则,百揆均任,四民殊业,故考绩可理而黜陟易通也。大魏继百王之末,承秦、汉之烈,制度之流,靡所脩采。自建安以来,至于青龙,神武拨乱,肇基皇祚,扫除凶逆,芟夷遗寇,旌旗卷舒,日不暇给。及经邦治戎,权法并用,百官群司,军国通任,随时之宜,以应政机。以古施今,事杂义殊,难得而通也。所以然者,制宜经远,或不切近,法应时务,不足垂后。夫建官均职,清理民物,所以立本也;循名考实,纠励成规,所以治末也。本纲末举而造制未呈,国略不崇而考课是先,惧不足以料贤愚之分,精幽明之理也。昔先王之择才,必本行於州闾,讲道於庠序,行具而谓之贤,道脩则谓之能。乡老献贤能于王,王拜受之,举其贤者,出使长之,科其能者,入使治之,此先王收才之义也。方今九州之民,爰及京城,未有六乡之举,其选才之职,专任吏部。案品状则实才未必当,任薄伐则德行未为叙,如此则殿最之课,未尽人才。述综王度,敷赞国式,体深义广,难得而详也。”

正始初,除尚书郎,迁黄门侍郎。时曹爽秉政,何晏为吏部尚书,嘏谓爽弟羲曰:“何平叔外静而内銛巧,好利,不念务本。吾恐必先惑子兄弟,仁人将远,而朝政废矣。”晏等遂与嘏不平,因微事以免嘏官。起家拜荥阳太守,不行。太傅司马宣王请为从事中郎。曹爽诛,为河南尹,迁尚书。嘏常以为”秦始罢侯置守,设官分职,不与古同。汉、魏因循,以至于今。然儒生学士,咸欲错综以三代之礼,礼弘致远,不应时务,事与制违,名实未附,故历代而不至於治者,盖由是也。欲大改定官制,依古正本,今遇帝室多难,未能革易”。

时论者议欲自伐吴,三征献策各不同。诏以访嘏,嘏对曰:“昔夫差陵齐胜晋,威行中国,终祸姑苏;齐闵兼土拓境,辟地千里,身蹈颠覆。有始不必善终,古之明效也。孙权自破关羽并荆州之后,志盈欲满,凶宄以极,是以宣文侯深建宏图大举之策。今权以死,讬孤於诸葛恪。若矫权苛暴,蠲其虐政,民免酷烈,偷安新惠,外内齐虑,有同舟之惧,虽不能终自保完,犹足以延期挺命於深江之外矣。而议者或欲汎舟径济,横行江表;或欲四道并进,攻其城垒;或欲大佃疆埸,观衅而动:诚皆取贼之常计也。然自治兵以来,出入三载,非掩袭之军也。贼之为寇,几六十年矣,君臣伪立,吉凶共患,又丧其元帅,上下忧危,设令列船津要,坚城据险,横行之计,其殆难捷。惟进军大佃,最差完牢。兵出民表,寇钞不犯;坐食积谷,不烦运士;乘衅讨袭,无远劳费:此军之急务也。昔樊哙愿以十万之众,横行匈奴,季布面折其短。今欲越长江,涉虏庭,亦向时之喻也。未若明法练士,错计於全胜之地,振长策以御敌之馀烬,斯必然之数也。”后吴大将诸葛恪新破东关,乘胜扬声欲向青、徐,朝廷将为之备。嘏议以为”淮海非贼轻行之路,又昔孙权遣兵入海,漂浪沉溺,略无孑遗,恪岂敢倾根竭本,寄命洪流,以徼乾没乎?恪不过遣偏率小将素习水军者,乘海溯淮,示动青、徐,恪自并兵来向淮南耳”。后恪果图新城,不克而归。

嘏常论才性同异,锺会集而论之。嘉平末,赐爵关内侯。高贵乡公即尊位,进封武乡亭侯。正元二年春,毌丘俭、文钦作乱。或以司马景王不宜自行,可遣太尉孚往,惟嘏及王肃劝之。景王遂行。以嘏守尚书仆射,俱东。俭、钦破败,嘏有谋焉。及景王薨,嘏与司马文王径还洛阳,文王遂以辅政。语在锺会传。会由是有自矜色,嘏戒之曰:“子志大其量,而勋业难为也,可不慎哉!”嘏以功进封阳乡侯,增邑六百户,并前千二百户。是岁薨,时年四十七,追赠太常,谥曰元侯。祗嗣。咸熙中开建五等,以嘏著勋前朝,改封祗泾原子。

评曰:昔文帝、陈王以公子之尊,博好文采,同声相应,才士并出,惟粲等六人最见名目。而粲特处常伯之官,兴一代之制,然其冲虚德宇,未若徐幹之粹也。卫觊亦以多识典故,相时王之式。刘劭该览学籍,文质周洽。刘廙以清鉴著,傅嘏用才达显云。


\part{魏书二十二}
\chapter{桓二陈徐卫卢传第二十二}

桓阶字伯绪,长沙临湘人也。仕郡功曹。太守孙坚举阶孝廉,除尚书郎。父丧还乡里。会坚击刘表战死,阶冒难诣表乞坚丧,表义而与之。后太祖与袁绍相拒於官渡,表举州以应绍。阶说其太守张羡曰:“夫举事而不本於义,未有不败者也。故齐桓率诸侯以尊周,晋文逐叔带以纳王。今袁氏反此,而刘牧应之,取祸之道也。明府必欲立功明义,全福远祸,不宜与之同也。”羡曰:“然则何向而可?”阶曰:“曹公虽弱,仗义而起,救朝廷之危,奉王命而讨有罪,孰敢不服?今若举四郡保三江以待其来,而为之内应,不亦可乎!”羡曰:“善。”乃举长沙及旁三郡以拒表,遣使诣太祖。太祖大悦。会绍与太祖连战,军未得南。而表急攻羡,羡病死。城陷,阶遂自匿。久之,刘表辟为从事祭酒,欲妻以妻妹蔡氏。阶自陈已结婚,拒而不受,因辞疾告退。

太祖定荆州,闻其为张羡谋也,异之,辟为丞相掾主簿,迁赵郡太守。魏国初建,为虎贲中郎将侍中。时太子未定,而临菑侯植有宠。阶数陈文帝德优齿长,宜为储副,公规密谏,前后恳至。又毛玠、徐奕以刚蹇少党,而为西曹掾丁仪所不善,仪屡言其短,赖阶左右以自全保。其将顺匡救,多此类也。迁尚书,典选举。曹仁为关羽所围,太祖遣徐晃救之,不解。太祖欲自南征,以问群下。群下皆谓:“王不亟行,今败矣。”阶独曰:“大王以仁等为足以料事势不也?”曰:“能。”“大王恐二人遗力邪?”曰:“不”“然则何为自往?”曰:“吾恐虏众多,而晃等势不便耳。“阶曰:“今仁等处重围之中而守死无贰者,诚以大王远为之势也。夫居万死之地,必有死争之心;内怀死争,外有强救,大王案六军以示馀力,何忧於败而欲自往?”太祖善其言,驻军於摩陂。贼遂退。

文帝践阼,迁尚书令,封高乡亭侯,加侍中。阶疾病,帝自临省,谓曰:“吾方讬六尺之孤,寄天下之命於卿。勉之!”徙封安乐乡侯,邑六百户,又赐阶三子爵关内侯,祐以嗣子不封,病卒,又追赠关内侯。后阶疾笃,遣使者即拜太常,薨,帝为之流涕,谥曰贞侯。子嘉嗣。以阶弟纂为散骑侍郎,赐爵关内侯。嘉尚升迁亭公主。会嘉平中,以乐安太守与吴战於东关,军败,没,谥曰壮侯。子翊嗣。

陈群字长文,颍川许昌也。祖父寔,父纪,叔父谌,皆有盛名。群为儿时,寔常奇异之,谓宗人父老曰:“此儿必兴吾宗。”鲁国孔融高才倨傲,年在纪、群之间,先与纪友,后与群交,更为纪拜,由是显名。刘备临豫州,辟群为别驾。时陶谦病死,徐州迎备,备欲往,群说备曰:“袁术尚强,今东,必与之争。吕布若袭将军之后,将军虽得徐州,事必无成。“备遂东,与袁术战。布果袭下邳,遣兵助术,大破备军,备恨不用群言。举茂才,除柘令,不行,随纪避难徐州。属吕布破,太祖辟群为司空西曹掾属。时有荐乐安王模、下邳周逵者,太祖辟之。群封还教,以为模、逵秽德,终必败,太祖不听。后模、逵皆坐奸宄诛,太祖以谢群。群荐广陵陈矫、丹阳戴乾,太祖皆用之。后吴人叛,乾忠义死难,矫遂为名臣,世以群为知人。除萧、赞、长平令,父卒去官。后以司徒掾举高第,为治书侍御史,转参丞相军事。魏国既建,迁为御史中丞。

时太祖议复肉刑,令曰:“安得通理君子达於古今者,使平斯事乎!昔陈鸿胪以为死刑有可加於仁恩者,正谓此也。御史中丞能申其父之论乎?”群对曰:“臣父纪以为汉除肉刑而增加笞,本兴仁恻而死者更众,所谓名轻而实重者也。名轻则易犯,实重则伤民。书曰:'惟敬五刑,以成三德。'易著劓、刖、灭趾之法,所以辅政助教,惩恶息杀也。且杀人偿死,合於古制;至於伤人,或残毁其体而裁翦毛发,非其理也。若用古刑,使淫者下蚕室,盗者刖其足,则永无淫放穿窬之奸矣。夫三千之属,虽未可悉复,若斯数者,时之所患,宜先施用。汉律所杀殊死之罪,仁所不及也,其馀逮死者,可以刑杀。如此,则所刑之与所生足以相贸矣。今以笞死之法易不杀之刑,是重人支体而轻人躯命也。”时锺繇与群议同,王朗及议者多以为未可行。太祖深善繇、群言,以军事未罢,顾众议,故且寝。

群转为侍中,领丞相东西曹掾。在朝无適无莫,雅杖名义,不以非道假人。文帝在东宫,深敬器焉,待以交友之礼,常叹曰:“自吾有回,门人日以亲。”及即王位,封群昌武亭侯,徙为尚书。制九品官人之法,群所建也。及践阼,迁尚书仆射,加侍中,徙尚书令,进爵颖乡侯。帝征孙权,至广陵,使群领中领军。帝还,假节,都督水军。还许昌,以群为镇军大将军,领中护军,录尚书事。帝寝疾,群与曹真、司马宣王等并受遗诏辅政。明帝即位,进封颍阴侯,增邑五百,并前千三百户,与征东大将军曹休、中军大将军曹真、抚军大将军司马宣王并开府。顷之,为司空,故录尚书事。

是时,帝初莅政,群上疏曰:“诗称'仪刑文王,万邦作孚';又曰'刑于寡妻,至于兄弟,以御于家邦'。道自近始,而化洽於天下。自丧乱已来,干戈未戢,百姓不识王教之本,惧其陵迟巳甚。陛下当盛魏之隆,荷二祖之业,天下想望至治,唯有以崇德布化,惠恤黎庶,则兆民幸甚。夫臣下雷同,是非相蔽,国之大患也。若不和睦则有雠党,有雠党则毁誉无端,毁誉无端则真伪失实,不可不深防备,有以绝其源流。”太和中,曹真表欲数道伐蜀,从斜谷入。群以为”太祖昔到阳平攻张鲁,多收豆麦以益军粮,鲁未下而食犹乏。今既无所因,且斜谷阻险,难以进退,转运必见钞截,多留兵守要,则损战士,不可不熟虑也“。帝从群议。真复表从子午道。群又陈其不便,并言军事用度之计。诏以群议下真,真据之遂行。会霖雨积日,群又以为宜诏真还,帝从之。

后皇女淑薨,追封谥平原懿公主。群上疏曰:“长短有命,存亡有分。故圣人制礼,或抑或致,以求厥中。防墓有不脩之俭,嬴、博有不归之魂。夫大人动合天地,垂之无穷,又大德不逾闲,动为师表故也。八岁下殇,礼所不备,况未期月,而以成人礼送之,加为制服,举朝素衣,朝夕哭临,自古已来,未有此比。而乃复自往视陵,亲临祖载。愿陛下抑割无益有损之事,但悉听群臣送葬,乞车驾不行,此万国之至望也。闻车驾欲幸摩陂,实到许昌,二宫上下,皆悉俱东,举朝大小,莫不惊怪。或言欲以避衰,或言欲於便处移殿舍,或不知何故。臣以为吉凶有命,祸福由人,移徙求安,则亦无益。若必当移避,缮治金墉城西宫,及孟津别宫,皆可权时分止。可无举宫暴露野次,废损盛节蚕农之要。又贼地闻之,以为大衰,加所烦费,不可计量。且吉士贤人,当盛衰,处安危,秉道信命,非徙其家以宁,乡邑从其风化,无恐惧之心。况乃帝王万国之主,静则天下安,动则天下扰;行止动静,岂可轻脱哉?”帝不听。

青龙中,营治宫室,百姓失农时。群上疏曰:“禹承唐、虞之盛,犹卑宫室而恶衣服,况今丧乱之后,人民至少,比汉文、景之时,不过一大郡。加边境有事,将士劳苦,若有水旱之患,国家之深忧也。且吴、蜀未灭,社稷不安。宜及其未动,讲武劝农,有以待之。今舍此急而先宫室,臣惧百姓遂困,将何以应敌?昔刘备自成都至白水,多作传舍,兴费人役,太祖知其疲民也。今中国劳力,亦吴、蜀之所愿。此安危之机也,惟陛下虑之。“帝答曰:“王者宫室,亦宜并立。灭贼之后,但当罢守耳,岂可复兴役邪?是故君之职,萧何之大略也。”群又曰:“昔汉祖唯与项羽争天下,羽已灭,宫室烧焚,是以萧何建武库、太仓,皆是要急,然犹非其壮丽。今二虏未平,诚不宜与古同也。夫人之所欲,莫不有辞,况乃天王,莫之敢违。前欲坏武库,谓不可不坏也;后欲置之,谓不可不置也。若必作之,固非臣下辞言所屈;若少留神,卓然回意,亦非臣下之所及也。汉明帝欲起德阳殿,锺离意谏,即用其言,后乃复作之;殿成,谓群臣曰:'锺离尚书在,不得成此殿也。'夫王者岂惮一臣,盖为百姓也。今臣曾不能少凝圣听,不及意远矣。“帝於是有所减省。

初,太祖时,刘廙坐弟与魏讽谋反,当诛。群言之太祖,太祖曰:“廙,名臣也,吾亦欲赦之。”乃复位。廙深德群,群曰:“夫议刑为国,非为私也;且自明主之意,吾何知焉?”其弘博不伐,皆此类也。青龙四年薨,谥曰靖侯。子泰嗣。帝追思群功德,分群户邑,封一子列侯。

泰字玄伯。青龙中,除散骑侍郎。正始中,徙游击将军,为并州刺史,加振威将军,使持节,护匈奴中郎将,怀柔夷民,甚有威惠。京邑贵人多寄宝货,因泰市奴婢,泰皆挂之於壁,不发其封,及徵为尚书,悉以还之。嘉平初,代郭淮为雍州刺史,加奋威将军。蜀大将军姜维率众依麹山筑二城,使牙门将句安、李歆等守之,聚羌胡质任等寇偪诸郡。征西将军郭淮与泰谋所以御之,泰曰:“麹城虽固,去蜀险远,当须运粮。羌夷患维劳役,必未肯附。今围而取之,可不血刃而拔其城;虽其有救,山道阻险,非行兵之地也。”淮从泰计,使泰率讨蜀护军徐质、南安太守邓艾等进兵围之,断其运道及城外流水。安等挑战,不许,将士困窘,分粮聚雪以稽日月。维果来救,出自牛头山,与泰相对。泰曰:“兵法贵在不战而屈人。今绝牛头,维无反道,则我之禽也。”敕诸军各坚垒勿与战,遣使白淮,欲自南渡白水,循水而东,使淮趣牛头,截其还路,可并取维,不惟安等而已。淮善其策,进率诸军军洮水。维惧,遁走,安等孤县,遂皆降。

淮薨,泰代为征西将军,假节都督雍、凉诸军事。后年,雍州刺史王经白泰,云姜维、夏侯霸欲三道向祁山、石营、金城,求进兵为翅,使凉州军至枹罕,讨蜀护军向祁山。泰量贼势终不能三道,且兵势恶分,凉州未宜越境,报经:“审其定问,知所趣向,须东西势合乃进。“时维等将数万人至枹罕,趣狄道。泰敕经进屯狄道,须军到,乃规取之。泰进军陈仓。会经所统诸军於故关与贼战不利,经辄渡洮。泰以经不坚据狄道,必有他变。并遣五营在前,泰率诸军继之。经巳与维战,大败,以万馀人还保狄道城,馀皆奔散。维乘胜围狄道。泰军上邽,分兵守要,晨夜进前。邓艾、胡奋、王秘亦到,即与艾、秘等分为三军,进到陇西。艾等以为“王经精卒破衄於西,贼众大盛,乘胜之兵既不可当,而将军以乌合之卒,继败军之后,将士失气,陇右倾荡。古人有言:'蝮蛇螫手,壮士解其腕。'孙子曰:'兵有所不击,地有所不守。'盖小有所失而大有所全故也。今陇右之害,过於蝮蛇,狄道之地,非徒不守之谓。姜维之兵,是所辟之锋。不如割险自保,观衅待弊,然后进救,此计之得者也。“泰曰:“姜维提轻兵深入,正欲与我争锋原野,求一战之利。王经当高壁深垒,挫其锐气。今乃与战,使贼得计,走破王经,封之狄道。若维以战克之威,进兵东向,据栎阳积谷之实,放兵收降,招纳羌、胡,东争关、陇,传檄四郡,此我之所恶也。而维以乘胜之兵,挫峻城之下,锐气之卒,屈力致命,攻守势殊,客主不同。兵书云'脩橹轒榅,三月乃成,拒堙三月而后已'。诚非轻军远入,维之诡谋仓卒所办。县军远侨,粮谷不继,是我速进破贼之时也,所谓疾雷不及掩耳,自然之势也。洮水带其表,维等在其内,今乘高据势,临其项领,不战必走。寇不可纵,围不可久,君等何言如此?“遂进军度高城岭,潜行,夜至狄道东南高山上,多举烽火,鸣鼓角。狄道城中将士见救者至,皆愤踊。维始谓官救兵当须众集乃发,而卒闻已至,谓有奇变宿谋,上下震惧。自军之发陇西也,以山道深险,贼必设伏。泰诡从南道,维果三日施伏。定军潜行,卒出其南。维乃缘山突至,泰与交战,维退还。凉州军从金城南至沃干阪。泰与经共密期,当共向其还路,维等闻之,遂遁,城中将士得出。经叹曰:“粮不至旬,向不应机,举城屠裂,覆丧一州矣。”泰慰劳将士,前后遣还,更差军守,并治城垒,还屯上邽。

初,泰闻经见围,以州军将士素皆一心,加得保城,非维所能卒倾。表上进军晨夜速到还。众议以经奔北,城不足自固,维若断凉州之道,兼四郡民夷,据关、陇之险,敢能没经军而屠陇右。宜须大兵四集,乃致攻讨。大将军司马文王曰:“昔诸葛亮常有此志,卒亦不能。事大谋远,非维所任也。且城非仓卒所拔,而粮少为急,征西速救,得上策矣。”泰每以一方有事,辄以虚声扰动天下,故希简白上事,驿书不过六百里。司马文王语荀顗曰:“玄伯沈勇能断,荷方伯之重,救将陷之城,而不求益兵,又希简上事,必能办贼故也。都督大将,不当尔邪!”

后徵泰为尚书右仆射,典选举,加侍中光禄大夫。吴大将孙峻出淮、泗。以泰为镇军将军,假节都督淮北诸军事,诏徐州监军已下受泰节度。峻退,军还,转为左仆射。诸葛诞作乱寿春,司马文王率六军军丘头,泰总署行台。司马景王、文王皆与泰亲友,及沛国武陔亦与泰善。文王问陔曰:“玄伯何如其父司空也?”陔曰:“通雅博畅,能以天下声教为己任者,不如也;明统简至,立功立事,过之。”泰前后以功增邑二千六百户,赐子弟一人亭侯,二人关内侯。景元元年薨,追赠司空。谥曰穆侯。子恂嗣。恂薨,无嗣,弟温绍封。咸熙中开建五等,以泰著勋前朝,改封温为慎子。

陈矫字季弼,广陵东阳人也。避乱江东及东城,辞孙策、袁术之命,还本郡。太守陈登请为功曹,使矫诣许,谓曰:“许下论议,待吾不足;足下相为观察,还以见诲。”矫还曰:“闻远近之论,颇谓明府骄而自矜。”登曰:“夫闺门雍穆,有德有行,吾敬陈元方兄弟;渊清玉絜,有礼有法,吾敬华子鱼;清脩疾恶,有识有义,吾敬赵元达;博闻强记,奇逸卓荦,吾敬孔文举;雄姿杰出,有王霸之略,吾敬刘玄德。所敬如此,何骄之有!馀子琐琐,亦焉足录哉?”登雅意如此,而深敬友矫。

郡为孙权所围於匡奇,登令矫求救於太祖。矫说太祖曰:“鄙郡虽小,形便之国也,若蒙救援,使为外藩,则吴人剉谋,徐方永安,武声远震,仁爱滂流,未从之国,望风景附,崇德养威,此王业也。”太祖奇矫,欲留之。矫辞曰:“本国倒悬,本奔走告急,纵无申胥之效,敢忘弘演之义乎?”太祖乃遣赴救。吴军既退,登多设间伏,勒兵追奔,大破之。

太祖辟矫为司空掾属,除相令,征南长史,彭城、乐陵太守,魏郡西部都尉。曲周民父病,以牛祷,县结正弃市。矫曰:“此孝子也。“表赦之。迁魏郡太守。时系囚千数,至有历年,矫以为周有三典之制,汉约三章之法,今惜轻重之理,而忽久系之患,可谓谬矣。悉自览罪状,一时论决。大军东征,入为丞相长史。军还,复为魏郡,转西曹属。从征汉中,还为尚书。行前未到邺,太祖崩洛阳,群臣拘常,以为太子即位,当须诏命。矫曰:“王薨于外,天下惶惧。太子宜割哀即位,以系远近之望。且又爱子在侧,彼此生变,则社稷危矣。”即具官备礼,一日皆办。明旦,以王后令,策太子即位,大赦荡然。文帝曰:“陈季弼临大节,明略过人,信一时之俊杰也。“帝既践阼,转署吏部,封高陵亭侯,迁尚书令。明帝即位,进爵东乡侯,邑六百户。车驾尝卒至尚书门,矫跪问帝曰:“陛下欲何之?”帝曰:“欲案行文书耳。“矫曰:“此自臣职分,非陛下所宜临也。若臣不称其职,则请就黜退。陛下宜还。”帝惭,回车而反。其亮直如此。加侍中光禄大夫,迁司徒。景初元年薨,谥曰贞侯。

子本嗣,历位郡守、九卿。所在操纲领,举大体,能使群下自尽。有统御之才,不亲小事,不读法律而得廷尉之称,优於司马岐等,精练文理。迁镇北将军,假节都督河北诸军事。薨,子粲嗣。本弟骞,咸熙中为车骑将军。

初,矫为郡功曹,使过泰山。泰山太守东郡薛悌异之,结为亲友。戏谓矫曰:“以郡吏而交二千石,邻国君屈从陪臣游,不亦可乎!”悌后为魏郡及尚书令,皆承代矫云。

徐宣字宝坚,广陵海西人也。避乱江东,又辞孙策之命,还本郡。与陈矫并为纲纪,二人齐名而私好不协,然俱见器於太守陈登,与登并心於太祖。海西、淮浦二县民作乱,都尉卫弥、令梁习夜奔宣家,密送免之。太祖遣督军扈质来讨贼,以兵少不进。宣潜见责之,示以形势,质乃进破贼。太祖辟为司空掾属,除东缗、发干令,迁齐郡太守,入为门下督,从到寿春。会马超作乱,大军西征,太祖见官属曰:“今当远征,而此方未定,以为后忧,宜得清公大德以镇统之。”乃以宣为左护军,留统诸军。还,为丞相东曹掾,出为魏郡太守。太祖崩洛阳,群臣入殿中发哀。或言可易诸城守,用谯、沛人。宣厉声曰:“今者远近一统,人怀效节,何必谯、沛,而沮宿卫者心。”文帝闻曰:“所谓社稷之臣也。”帝既践阼,为御史中丞,赐爵关内侯,徙城门校尉,旬月迁司隶校尉,转散骑常侍。从至广陵,六军乘舟,风浪暴起,帝船回倒,宣病在后,陵波而前,群寮莫先至者。帝壮之,迁尚书。

明帝即位,封津阳亭侯,邑二百户。中领军桓范荐宣曰:“臣闻帝王用人,度世授才。争夺之时,以策略为先,分定之后,以忠义为首。故晋文行舅犯之计而赏雍季之言,高祖用陈平之智而讬后於周勃也。窃见尚书徐宣,体忠厚之行,秉直亮之性;清雅特立,不拘世俗;确然难动,有社稷之节;历位州郡,所在称职。今仆射缺,宣行掌后事;腹心任重,莫宜宣者。“帝遂以宣为左仆射,后加侍中光禄大夫。车驾幸许昌,总统留事。帝还,主者奏呈文书。诏曰:“吾省与仆射何异?”竟不视。尚方令坐猥见考竟,宣上疏陈威刑大过,又谏作宫殿穷尽民力,帝皆手诏嘉纳。宣曰:“七十有县车之礼,今已六十八,可以去矣。”乃固辞疾逊位,帝终不许。青龙四年薨,遗令布衣疏巾,敛以时服。诏曰:“宣体履至实,直内方外,历在三朝,公亮正色,有讬孤寄命之节,可谓柱石臣也。常欲倚以台辅,未及登之,惜乎大命不永!其追赠车骑将军,葬如公礼。”谥曰贞侯。子钦嗣。

卫臻字公振,陈留襄邑人也。父兹,有大节,不应三公之辟。太祖之初至陈留,兹曰:“平天下者,必此人也。”太祖亦异之,数诣兹议大事。从讨董卓,战于荥阳而卒。太祖每涉郡境,辄遣使祠焉。夏侯惇为陈留太守,举臻计吏,命妇出宴,臻以为“末世之俗,非礼之正“。惇怒,执臻,既而赦之。后为汉黄门侍郎。东郡朱越谋反,引臻。太祖令曰:“孤与卿君同共举事,加钦令问。始闻越言,固自不信。及得荀令君书,具亮忠诚。”会奉诏命,聘贵人于魏,因表留臻参丞相军事。追录臻父旧勋,赐爵关内侯,转为户曹掾。文帝即王位,为散骑常侍。及践阼,封安国亭侯。时群臣并颂魏德,多抑损前朝。臻独明禅授之义,称扬汉美。帝数目臻曰:“天下之珍,当与山阳共之。”迁尚书,转侍中吏部尚书。帝幸广陵,行中领军,从。征军大将军曹休表得降贼辞,“孙权已在濡须口”。臻曰:“权恃长江,未敢抗衡,此必畏怖伪辞耳。“考核降者,果守将诈所作也。

明帝即位,进封康乡侯,后转为右仆射,典选举,如前加侍中。中护军蒋济遗臻书曰:“汉祖遇亡虏为上将,周武拔渔父为太师;布衣厮养,可登王公,何必守文,试而后用?”臻答曰:“古人遗智慧而任度量,须考绩而加黜陟;今子同牧野於成、康,喻断蛇文、景,好不经之举,开拔奇之津,将使天下驰骋而起矣。”诸葛亮寇天水,臻奏:“宜遣奇兵入散关,绝其粮道。”乃以臻为征蜀将军,假节督诸军事,到长安,亮退。还,复职,加光禄大夫。是时,帝方隆意於殿舍,臻数切谏。及殿中监擅收兰台令史,臻奏案之。诏曰:“殿舍不成,吾所留心,卿推之何?”臻上疏曰:“古制侵官之法,非恶其勤事也,诚以所益者小,所堕者大也。臣每察校事,类皆如此,惧群司将遂越职,以至陵迟矣。”亮又出斜谷;征南上:“朱然等军已过荆城。”臻曰:“然,吴之骁将,必下从权,且为势以缀征南耳。”权果召然入居巢,进攻合肥。帝欲自东征,臻曰:“权外示应亮,内实观望。且合肥城固,不足为虑。车驾可无亲征,以省六军之费。”帝到寻阳而权竟退。

幽州刺史毌丘俭上疏曰:“陛下即位已来,未有可书。吴、蜀恃险,未可卒平,聊可以此方无用之士克定辽东。”臻曰:“俭所陈皆战国细术,非王者之事也。吴频岁称兵,寇乱边境,而犹案甲养士,未果寻致讨者,诚以百姓疲劳故也。且渊生长海表,相承三世,外抚戎夷,内脩战射,而俭欲以偏军长驱,朝至夕卷,知其妄矣。”俭行军遂不利。

臻迁为司空,徙司徒。正始中,进爵长垣侯,邑千户,封一子列侯。初,太祖久不立太子,而方奇贵临菑侯。丁仪等为之羽翼,劝臻自结,臻以大义拒之。及文帝即位,东海王霖有宠,帝问臻:“平原侯何如?”臻称明德美而终不言。曹爽辅政,使夏侯玄宣指,欲引臻入守尚书令,及为弟求婚,皆不许。固乞逊位。诏曰:“昔干木偃息,义压强秦;留侯颐神,不忘楚事。谠言嘉谋,望不吝焉。”赐宅一区,位特进,秩如三司。薨,追赠太尉,谥曰敬侯。子烈嗣,咸熙中为光禄勋。

卢毓字子家,涿郡涿人也。父植,有名於世。毓十岁而孤,遇本州乱,二兄死难。当袁绍、公孙瓒交兵,幽冀饥荒,养寡嫂孤兄子,以学行见称。文帝为五官将,召毓署门下贼曹。崔琰举为冀州主簿。时天下草创,多逋逃,故重士亡法,罪及妻子。亡士妻白等,始適夫家数日,未与夫相见,大理奏弃巿。毓驳之曰:“夫女子之情,以接见而恩生,成妇而义重。故诗云'未见君子,我心伤悲;亦既见止,我心则夷'。又礼'未庙见之妇而死,归葬女氏之党,以未成妇也'。今白等生有未见之悲,死有非妇之痛,而吏议欲肆之大辟,则若同牢合卺之后,罪何所加?且记曰'附从轻',言附人之罪,以轻者为比也。又书云'与其杀不辜,宁失不经',恐过重也。苟以白等皆受礼聘,已入门庭,刑之为可,杀之为重。”太祖曰:“毓执之是也。又引经典有意,使孤叹息。”由是为丞相法曹议令史,转西曹议令史。

魏国既建,为吏部郎。文帝践阼,徙黄门侍郎,出为济阴相,梁、谯二郡太守。帝以谯旧乡,故大徙民充之,以为屯田。而谯土地墝瘠,百姓穷困,毓愍之,上表徙民於梁国就沃衍,失帝意。虽听毓所表,心犹恨之,遂左迁毓,使将徙民为睢阳典农校尉。毓心在利民,躬自临视,择居美田,百姓赖之。迁安平、广平太守,所在有惠化。

青龙二年,入为侍中。先是,散骑常侍刘劭受诏定律,未就。毓上论古今科律之意,以为法宜一正,不宜有两端,使奸吏得容情。及侍中高堂隆数以宫室事切谏,帝不悦,毓进曰:“臣闻君明则臣直,古之圣王恐不闻其过,故有敢谏之鼓。近臣尽规,此乃臣等所以不及隆。隆诸生,名为狂直,陛下宜容之。”在职三年,多所驳争。诏曰:“官人秩才,圣帝所难,必须良佐,进可替否。侍中毓禀性贞固,心平体正,可谓明试有功,不懈于位者也。其以毓为吏部尚书。“使毓自选代,曰:“得如卿者乃可。”毓举常侍郑冲,帝曰:“文和,吾自知之,更举吾所未闻者。”乃举阮武、孙邕,帝於是用邕。

前此诸葛诞、邓飏等驰名誉,有四聪八达之诮,帝疾之。时举中书郎,诏曰:“得其人与否,在卢生耳。选举莫取有名,名如画地作饼,不可啖也。”毓对曰:“名不足以致异人,而可以得常士。常士畏教慕善,然后有名,非所当疾也。愚臣既不足以识异人,又主者正以循名案常为职,但当有以验其后。故古者敷奏以言,明试以功。今考绩之法废,而以毁誉相进退,故真伪浑杂,虚实相蒙。”帝纳其言,即诏作考课法。会司徒缺,毓举处士管宁,帝不能用。更问其次,毓对曰:“敦笃至行,则太中大夫韩暨;亮直清方,则司隶校尉崔林;贞固纯粹,则太常常林。”帝乃用暨。毓於人及选举,先举性行,而后言才。黄门李丰尝以问毓,毓曰:“才所以为善也,故大才成大善,小才成小善。今称之有才而不能为善,是才不中器也。”丰等服其言。

齐王即位,赐爵关内侯。时曹爽秉权,将树其党,徙毓仆射,以侍中何晏代毓。顷之,出毓为廷尉,司隶毕轨又枉奏免官,众论多讼之,乃以毓为光禄勋。爽等见收,太傅司马宣王使毓行司隶校尉,治其狱。复为吏部尚书,加奉车都尉,封高乐亭侯,转为仆射,故典选举,加光禄大夫。高贵乡公即位,进封大梁乡侯。封一子亭侯。毌丘俭作乱,大将军司马景王出征,毓纲纪后事,加侍中。正元三年,疾病,逊位。迁为司空,固推骠骑将军王昶、光禄大夫王观、司隶校尉王祥。诏使使者即授印绶,进爵封容城侯,邑二千三百户。甘露二年薨,谥曰成侯。孙藩嗣。毓子钦、珽,咸熙中钦为尚书,珽泰山太守。

评曰:桓阶识睹成败,才周当世。陈群动仗名义,有清流雅望;泰弘济简至,允克堂构矣。魏世事统台阁,重内轻外,故八座尚书,即古六卿之任也。陈、徐、卫、卢,久居斯位,矫、宣刚断骨鲠,臻、毓规鉴清理,咸不忝厥职云。

\part{魏书二十三}
\chapter{和常杨杜赵裴传第二十三}

和洽字阳士,汝南西平人也。举孝廉,大将军辟,皆不就。袁绍在冀州,遣使迎汝南士大夫。洽独以“冀州土平民强,英桀所利,四战之地。本初乘资,虽能强大,然雄豪方起,全未可必也。荆州刘表无他远志,爱人乐士,土地险阻,山夷民弱,易依倚也”。遂与亲旧俱南从表,表以上客待之。洽曰:“所以不从本初,辟争地也。昏世之主,不可黩近,久而阽危,必有谗慝间其中者。”遂南度武陵。

太祖定荆州,辟为丞相掾属。时毛玠、崔琰并以忠清幹事,其选用先尚俭节。洽言曰:“天下大器,在位与人,不可以一节检也。俭素过中,自以处身则可,以此节格物,所失或多。今朝廷之议,吏有著新衣、乘好车者,谓之不清;长吏过营,形容不饰,衣裘敝坏者,谓之廉洁。至令士大夫故汙辱其衣,藏其舆服;朝府大吏,或自挈壶餐以入官寺。夫立教观俗,贵处中庸,为可继也。今崇一概难堪之行以检殊涂,勉而为之,必有疲瘁。古之大教,务在通人情而已。凡激诡之行,则容隐伪矣。”

魏国既建,为侍中,后有白毛玠谤毁太祖,太祖见近臣,怒甚。洽陈玠素行有本,求案实其事。罢朝,太祖令曰:“今言事者白玠不但谤吾也,乃复为崔琰觖望。此损君臣恩义,妄为死友怨叹,殆不可忍也。昔萧、曹与高祖并起微贱,致功立勋。高祖每在屈笮,二相恭顺,臣道益彰,所以祚及后世也。和侍中比求实之,所以不听,欲重参之耳。”洽对曰:“如言事者言,玠罪过深重,非天地所覆载。臣非敢曲理玠以枉大伦也,以玠出群吏之中,特见拔擢,显在首职,历年荷宠,刚直忠公,为众所惮,不宜有此。然人情难保,要宜考覈,两验其实。今圣恩垂含垢之仁,不忍致之于理,更使曲直之分不明,疑自近始。”太祖曰:“所以不考,欲两全玠及言事者耳。“洽对曰:“玠信有谤上之言,当肆之巿朝;若玠无此,言事者加诬大臣以误主听;二者不加检覈,臣窃不安。”太祖曰:“方有军事,安可受人言便考之邪?狐射姑刺阳处父於朝,此为君之诫也。”

太祖克张鲁,洽陈便宜以时拔军徙民,可省置守之费。太祖未纳,其后竟徙民弃汉中。出为郎中令。文帝践阼,为光禄勋,封安城亭侯。明帝即位,进封西陵乡侯,邑二百户。

太和中,散骑常侍高堂隆奏:“时风不至,而有休废之气,必有司不勤职事以失天常也。”诏书谦虚引咎,博谘异同。洽以为“民稀耕少,浮食者多。国以民为本,民以谷为命。故费一时之农,则失育命之本。是以先王务蠲烦费,以专耕农。自春夏以来,民穷於役,农业有废,百姓嚣然,时风不至,未必不由此也。消复之术,莫大於节俭。太祖建立洪业,奉师徒之费,供军赏之用,吏士丰於资食,仓府衍於谷帛,由不饰无用之宫,绝浮华之费,方今之要,固在息省劳烦之役,损除他馀之务,以为军戎之储。三边守御,宜在备豫。料贼虚实,蓄士养众,算庙胜之策,明攻取之谋,详询众庶以求厥中。若谋不素定,轻弱小敌,军人数举,举而无庸,所谓'悦武无震',古人之诫也。”

转为太常,清贫守约,至卖田宅以自给。明帝闻之,加赐谷帛。薨,谥曰简侯。子离嗣。离弟逌,才爽开济,官至廷尉书。

洽同郡许混者,许劭子也。清醇有鉴识,明帝时为尚书。

常林字伯槐,河内温人也。年七岁,有父党造门,问林:“伯先在否?汝何不拜!”林曰:“虽当下客,临子字父,何拜之有?”於是咸共嘉之。太守王匡起兵讨董卓,遣诸生於属县微伺吏民罪负,便收之,考责钱谷赎罪, 稽迟则夷灭宗族,以崇威严。林叔父挝客,为诸生所白,匡怒收治。举宗惶怖,不知所责多少,惧系者不救。林往见匡同县胡母彪曰:“王府君以文武高才,临吾鄙郡。鄙郡表里山河,土广民殷,又多贤能,惟所择用。今主上幼冲,贼臣虎据,华夏震栗,雄才奋用之秋也。若欲诛天下之贼,扶王室之微,智者望风,应之若响,克乱在和,何征不捷。苟无恩德,任失其人,覆亡将至,何暇匡翼朝廷,崇立功名乎?君其藏之!”因说叔父见拘之意。彪即书责匡,匡原林叔父。林乃避地上党,耕种山阿。当时旱蝗,林独丰收,尽呼比邻,升斗分之。依故河间太守陈延壁。陈、冯二姓,旧族冠冕。张杨利其妇女,贪其资货。林率其宗族,为之策谋。见围六十馀日,卒全堡壁。

并州刺史高幹表为骑都尉,林辞不受。后刺史梁习荐州界名士林及杨俊、王凌、王象、荀纬,太祖皆以为县长。林宰南和,治化有成,超迁博陵太守、幽州刺史,所在有绩。文帝为五官将,林为功曹。太祖西征,田银、苏伯反,幽、冀扇动。文帝欲亲自讨之,林曰:“昔忝博陵,又在幽州,贼之形势,可料度也。北方吏民,乐安厌乱,服化已久,守善者多。银、伯犬羊相聚,智小谋大,不能为害。方今大军在远,外有强敌,将军为天下之镇也,轻动远举,虽克不武。”文帝从之,遣将往伐,应时克灭。

出为平原太守、魏郡东部都尉,入为丞相东曹属。魏国既建,拜尚书。文帝践阼,迁少府,封乐阳亭侯,转大司农。明帝即位,进封高阳乡侯,徙光禄勋太常。晋宣王以林乡邑耆德,每为之拜。或谓林曰:“司马公贵重,君宜止之。”林曰:“司马公自欲敦长幼之叙,为后生之法。贵非吾之所畏,拜非吾之所制也。”言者踧而退。时论以林节操清峻,欲致之公辅,而林遂称疾笃。拜光禄大夫。年八十三,薨,追赠骠骑将军,葬如公礼,谥曰贞侯。子峕嗣,为泰山太守,坐法诛。峕弟静绍封。

杨俊字季才,河内获嘉人也。受学陈留边让,让器异之。俊以兵乱方起,而河内处四达之衢,必为战场,乃扶持老弱诣京、密山间,同行者百馀家。俊振济贫乏,通共有无。宗族知故为人所略作奴仆者凡六家,俊皆倾财赎之。司马宣王年十六七,与俊相遇,俊曰:“此非常之人也。”又司马朗早有声名,其族兄芝,众未之知,惟俊言曰:“芝虽夙望不及朗,实理但有优耳。”俊转避地并州。本郡王象,少孤特,为人仆隶,年十七八,见使牧羊而私读书,因被箠楚。俊嘉其才质,即赎象著家,聘娶立屋,然后与别。

太祖除俊曲梁长,入为丞相掾属,举茂才,安陵令,迁南阳太守。宣德教,立学校,吏民称之。徙为征南军师。魏国既建,迁中尉。太祖征汉中,魏讽反於邺。俊自劾诣行在所。俊以身方罪免,笺辞太子。太子不悦,曰:“杨中尉便去,何太高远邪!”遂被书左迁平原太守。文帝践阼,复在南阳。时王象为散骑常侍,荐俊曰:“伏见南阳太守杨俊,秉纯粹之茂质,履忠肃之弘量,体仁足以育物,笃实足以动众,克长后进,惠训不倦,外宽内直,仁而有断。自初弹冠,所历垂化,再守南阳,恩德流著,殊邻异党,襁负而至。今境守清静,无所展其智能,宜还本朝,宣力辇毂,熙帝之载。”

俊自少及长,以人伦自任。同郡审固、陈留卫恂本皆出自兵伍,俊资拔奖致,咸作佳士;后固历位郡守,恂御史、县令,其明鉴行义多此类也。初,临菑侯与俊善,太祖適嗣未定,密访群司。俊虽并论文帝、临菑才分所长,不適有所据当,然称临菑犹美,文帝常以恨之。黄初三年,车驾至宛,以巿不丰乐,发怒收俊。尚书仆射司马宣王、常侍王象、荀纬请俊,叩头流血,帝不许。俊曰:“吾知罪矣。”遂自杀。众冤痛之。

杜袭字子绪,颍川定陵人也。曾祖父安,祖父根,著名前世。袭避乱荆州,刘表待以宾礼。同郡繁钦数见奇於表,袭喻之曰:“吾所以与子俱来者,徒欲龙蟠幽薮,待时凤翔。岂谓刘牧当为拨乱之主,而规长者委身哉?子若见能不已,非吾徒也。吾其与子绝矣!”钦慨然曰:“请敬受命。”袭遂南適长沙。

建安初,太祖迎天子都许。袭逃还乡里,太祖以为西鄂长。县滨南境,寇贼纵横。时长吏皆敛民保城郭,不得农业。野荒民困,仓庾空虚。袭自知恩结於民,乃遣老弱各分散就田业,留丁强备守,吏民欢悦。会荆州出步骑万人来攻城,袭乃悉召县吏民任拒守者五十馀人,与之要誓。其亲戚在外欲自营护者,恣听遣出;皆叩头愿致死。於是身执矢石,率与戮力。吏民感恩,咸为用命。临陈斩数百级,而袭众死者三十馀人,其馀十八人尽被创,贼得入城。袭帅伤痍吏民决围得出,死丧略尽,而无反背者。遂收散民,徙至摩陂营,吏民慕而从之如归。

司隶锺繇表拜议郎参军事。荀彧又荐袭,太祖以为丞相军祭酒。魏国既建,为侍中,与王粲、和洽并用。粲强识博闻,故太祖游观出入,多得骖乘,至其见敬不及洽、袭。袭尝独见,至于夜半。粲性躁竞,起坐曰:“不知公对杜袭道何等也?”洽笑答曰:“天下事岂有尽邪?卿昼侍可矣,悒悒於此,欲兼之乎!”后袭领丞相长史,随太祖到汉中讨张鲁。太祖还,拜袭驸马都尉,留督汉中军事。绥怀开导,百姓自乐出徙洛、邺者,八万馀口。夏侯渊为刘备所没,军丧元帅,将士失色。袭与张郃、郭淮纠摄诸军事,权宜以郃为督,以一众心,三军遂定。太祖东还,当选留府长史,镇守长安,主者所选多不当,太祖令曰:“释骐骥而不乘,焉皇皇而更索?”遂以袭为留府长史,驻关中。

时将军许攸拥部曲,不附太祖而有慢言。太祖大怒,先欲伐之。群臣多谏:“可招怀攸,共讨强敌。”太祖横刀於膝,作色不听。袭入欲谏,太祖逆谓之曰:“吾计以定,卿勿复言。”袭曰:“若殿下计是邪,臣方助殿下成之;若殿下计非邪,虽成宜改之。殿下逆臣,令勿言之,何待下之不阐乎?“太祖曰:“许攸慢吾,如何可置乎?”袭曰:“殿下谓许攸何如人邪?”太祖曰:“凡人也。”袭曰:“夫惟贤知贤,惟圣知圣,凡人安能知非凡人邪?方今豺狼当路而狐狸是先,人将谓殿下避强攻弱,进不为勇,退不为仁。臣闻千钧之弩不为鼷鼠发机,万石之锺不以莛撞起音,今区区之许攸,何足以劳神武哉?”太祖曰:“善。”遂厚抚攸,攸即归服。时夏侯尚昵於太子,情好至密。袭谓尚非益友,不足殊待,以闻太祖。文帝初甚不悦,后乃追思。语在尚传。其柔而不犯,皆此类也。

文帝即王位,赐爵关内侯。及践阼,为督军粮御史,封武平亭侯,更为督军粮执法,入为尚书。明帝即位,进封平阳乡侯。诸葛亮出秦川,大将军曹真督诸军拒亮,徙袭为大将军军师,分邑百户赐兄基爵关内侯。真薨,司马宣王代之,袭复为军师,增邑三百,并前五百五十户。以疾徵还,拜太中大夫。薨,追赠少府,谥曰定侯。子会嗣。

赵俨字伯然,颍川阳翟人也。避乱荆州,与杜袭、繁钦通财同计,合为一家。太祖始迎献帝都许,俨谓钦曰:“曹镇东应期命世,必能匡济华夏,吾知归矣。“建安二年,年二十七,遂扶持老弱诣太祖,太祖以俨为朗陵长。县多豪猾,无所畏忌。俨取其尤甚者,收缚案验,皆得死罪。俨既囚之,乃表府解放,自是威恩并著。时袁绍举兵南侵,遣使招诱豫州诸郡,诸郡多受其命。惟阳安郡不动,而都尉李通急录户调。俨见通曰:“方今天下未集,诸郡并叛,怀附者复收其绵绢,小人乐乱,能无遗恨!且远近多虞,不可不详也。”通曰:“绍与大将军相持甚急,左右郡县背叛乃尔。若绵绢不调送,观听者必谓我顾望,有所须待也。”俨曰:“诚亦如君虑;然当权其轻重,小缓调,当为君释此患。”乃书与荀彧曰:“今阳安郡当送绵绢,道路艰阻,必致寇害。百姓困穷,邻城并叛,易用倾荡,乃一方安危之机也。且此郡人执守忠节,在险不贰。微善必赏,则为义者劝。善为国者,藏之於民。以为国家宜垂慰抚,所敛绵绢,皆俾还之。”彧报曰:“辄白曹公,公文下郡,绵绢悉以还民。”上下欢喜,郡内遂安。

入为司空掾属主簿。时于禁屯颍阴,乐进屯阳翟,张辽屯长社,诸将任气,多共不协;使俨并参三军,每事训喻,遂相亲睦。太祖征荆州,以俨领章陵太守,徙都督护军,护于禁、张辽、张郃、朱灵、李典、路招、冯楷七军。复为丞相主簿,迁扶风太守。太祖徙出故韩遂、马超等兵五千馀人,使平难将军殷署等督领,以俨为关中护军,尽统诸军。羌虏数来寇害,俨率署等追到新平,大破之。屯田客吕并自称将军,聚党据陈仓,俨复率署等攻之,贼即破灭。

时被书差千二百兵往助汉中守,署督送之。行者卒与室家别,皆有忧色。署发后一日,俨虑其有变,乃自追至斜谷口,人人慰劳,又深戒署。还宿雍州刺史张既舍。署军复前四十里,兵果叛乱,未知署吉凶。而俨自随步骑百五十人,皆与叛者同部曲,或婚姻,得此问,各惊,被甲持兵,不复自安。俨欲还,既等以为“今本营党已扰乱,一身赴之无益,可须定问”。俨曰:“虽疑本营与叛者同谋,要当闻行者变,乃发之。又有欲善不能自定,宜及犹豫,促抚宁之。且为之元帅,既不能安辑,身受祸难,命也。“遂去。行三十里止,放马息,尽呼所从人,喻以成败,慰励恳切。皆慷慨曰:“死生当随护军,不敢有二。”前到诸营,各召料简诸奸结叛者八百馀人,散在原野,惟取其造谋魁率治之,馀一不问。郡县所收送,皆放遣,乃即相率还降。俨密白:“宜遣将诣大营,请旧兵镇守关中。“太祖遣将军刘柱将二千人,当须到乃发遣,而事露,诸营大骇,不可安喻。俨谓诸将曰:“旧兵既少,东兵未到,是以诸营图为邪谋。若或成变,为难不测。因其狐疑,当令早决。”遂宣言当差留新兵之温厚者千人镇守关中,其馀悉遣东。便见主者,内诸营兵名籍,案累重,立差别之。留者意定,与俨同心。其当去者亦不敢动,俨一日尽遣上道,因使所留千人,分布罗落之。东兵寻至,乃复胁喻,并徙千人,令相及共东,凡所全致二万馀口。

关羽围征南将军曹仁於樊。俨以议郎参仁军事南行,舆平寇将军徐晃俱前。既到,羽围仁遂坚,馀救兵未到。晃所督不足解围,而诸将呵责晃促救。俨谓诸将曰:“今贼围素固,水潦犹盛。我徒卒单少,而仁隔绝不得同力,此举適所以弊内外耳。当今不若前军偪围,遣谍通仁,使知外救,以励将士。计北军不过十日,尚足坚守。然后表里俱发,破贼必矣。如有缓救之戮,余为诸军当之。”诸将皆喜,便作地道,箭飞书与仁,消息数通,北军亦至,并势大战。羽军既退,舟船犹据沔水,襄阳隔绝不通,而孙权袭取羽辎重,羽闻之,即走南还。仁会诸将议,咸曰:“今因羽危惧,必可追禽也。”俨曰:“权邀羽连兵之难,欲掩制其后,顾羽还救,恐我承其两疲,故顺辞求效,乘衅因变,以观利钝耳。今羽已孤迸,更宜存之以为权害。若深入追北,权则改虞於彼,将生患於我矣。王必以此为深虑。”仁乃解严。太祖闻羽走,恐诸将追之,果疾敕仁,如俨所策。

文帝即王位,为侍中。顷之,拜驸马都尉,领河东太守,典农中郎将。黄初三年,赐爵关内侯。孙权寇边,征东大将军曹休统五州军御之,徵俨为军师。权众退,军还,封宜土亭侯,转为度支中郎将,迁尚书。从征吴,到广陵,复留为征东军师。明帝即位,进封都乡侯,邑六百户,监荆州诸军事,假节。会疾,不行,复为尚书,出监豫州诸军事,转大司马军师,入为大司农。齐王即位,以俨监雍、凉诸军事,假节,转征蜀将军,又迁征西将军,都督雍、凉。正始四年,老疾求还,徵为骠骑将军,迁司空。薨,谥曰穆侯。子亭嗣。初,俨与同郡辛毗、陈群、杜袭并知名,号曰辛、陈、杜、赵云。

裴潜字文行,河东闻喜人也。避乱荆州,刘表待以宾礼。潜私谓所亲王粲、司马芝曰:“刘牧非霸王之才,乃欲西伯自处,其败无日矣。”遂南適长沙。太祖定荆州,以潜参丞相军事,出历三县令,入为仓曹属。太祖问潜曰:“卿前与刘备俱在荆州,卿以备才略何如?“潜曰:“使居中国,能乱人而不能为治也。若乘间守险,足以为一方主。”

时代郡大乱,以潜为代郡太守。乌丸王及其大人,凡三人,各自称单于,专制郡事。前太守莫能治正,太祖欲授潜精兵以镇讨之。潜辞曰:“代郡户口殷众,士马控弦,动有万数。单于自知放横日久,内不自安。今多将兵往,必惧而拒境,少将则不见惮。宜以计谋图之,不可以兵威迫也。”遂单车之郡。单于惊喜。潜抚之以静,单于以下脱帽稽颡,悉还前后所掠妇女、器械、财物。潜案诛郡中大吏与单于为表里者郝温、郭端等十馀人,北边大震,百姓归心。在代三年,还为丞相理曹掾,太祖褒称治代之功,潜曰:“潜於百姓虽宽,於诸胡为峻。今计者必以潜为理过严,而事加宽惠;彼素骄恣,过宽必弛,既弛又将摄之以法,此讼争所由生也。以势料之,代必复叛。”於是太祖深悔还潜之速。后数十日,三单于反问至,乃遣鄢陵侯彰为骁骑将军征之。

潜出为沛国相,迁兖州刺史。太祖次摩陂,叹其军陈齐整,特加赏赐。文帝践阼,入为散骑常侍。出为魏郡、颍川典农中郎将,奏通贡举,比之郡国,由是农官进仕路泰。迁荆州刺史,赐爵关内侯。明帝即位,入为尚书。出为河南尹,转太尉军师、大司农,封清阳亭侯,邑二百户。入为尚书令,奏正分职,料简名实,出事使断官府者百五十馀条。丧父去官,拜光禄大夫。正始五年薨,追赠太常,谥曰贞侯。子秀嗣。遗令俭葬,墓中惟置一坐,瓦器数枚,其馀一无所设。秀,咸熙中为尚书仆射。

评曰:和洽清和幹理,常林素业纯固,杨俊人伦行义,杜袭温粹识统,赵俨刚毅有度,裴潜平恒贞幹,皆一世之美士也。至林能不系心於三司,以大夫告老,美矣哉!

\part{魏书二十四}
\chapter{韩崔高孙王传第二十四}

韩暨字公至,南阳堵阳人也。同县豪右陈茂,谮暨父兄,畿至大辟。暨阳不以为言,庸赁积资,阴结死士,遂追呼寻禽茂,以首祭父墓,由是显名。举孝廉,司空辟,皆不就。乃变名姓,隐居避乱鲁阳山中。山民合党,欲行寇掠。暨散家财以供牛酒,请其渠帅,为陈安危。山民化之,终不为害。避袁术命召,徙居山都之山。荆州牧刘表礼辟,遂遁逃,南居孱陵界,所在见敬爱,而表深恨之。暨惧,应命,除宜城长。

太祖平荆州,辟为丞相士曹属。后选乐陵太守,徙监冶谒者。旧时冶作马排,每一熟石用马百匹;更作人排,又费功力;暨乃因长流为水排,计其利益,三倍於前。在职七年,器用充实。制书褒叹,就加司金都尉,班亚九卿。文帝践阼,封宜城亭侯。黄初七年,迁太常,进封南乡亭侯,邑二百户。

时新都洛阳,制度未备,而宗庙主祏皆在邺都。暨奏请迎邺四庙神主,建立洛阳庙,四时蒸尝,亲奉粢盛。崇明正礼,废去淫祀,多所匡正。在官八年,以疾逊位。景初二年春,诏曰:“太中大夫韩暨,澡身浴德,志节高絜,年逾八十,守道弥固,可谓纯笃,老而益劭者也。其以暨为司徒。”夏四月薨,遗令敛以时服,葬为土藏。谥曰恭侯。子肇嗣。肇薨,子邦嗣。

崔林字德儒,清河东武城人也。少时晚成,宗族莫知,惟从兄琰异之。太祖定冀州,召除邬长,贫无车马,单步之官。太祖征壶关,问长吏德政最者,并州刺史张陟以林对,於是擢为冀州主簿,徙署别驾、丞相掾属。魏国既建,稍迁御史中丞。

文帝践阼,拜尚书,出为幽州刺史。北中郎将吴质统河北军事,涿郡太守王雄谓林别驾曰:“吴中郎将,上所亲重,国之贵臣也。仗节统事,州郡莫不奉笺致敬,而崔使君初不与相闻。若以边塞不脩斩卿,使君宁能护卿邪?”别驾具以白林,林曰:“刺史视去此州如脱屣,宁当相累邪?此州与胡虏接,宜镇之以静,扰之则动其逆心,特为国家生北顾忧,以此为寄。”在官一期,寇窃寝息;犹以不事上司,左迁河间太守,清论多为林怨也。

迁大鸿胪。龟兹王遣侍子来朝,朝廷嘉其远至,褒赏其王甚厚。馀国各遣子来朝,间使连属,林恐所遣或非真的,权取疏属贾胡,因通使命,利得印绶,而道路护送,所损滋多。劳所养之民,资无益之事,为夷狄所笑,此曩时之所患也。乃移书敦煌喻指,并录前世待遇诸国丰约故事,使有恒常。明帝即位,赐爵关内侯,转光禄勋、司隶校尉。属郡皆罢非法除过员吏。林为政推诚,简存大体,是以去后每辄见思。

散骑常侍刘劭作考课论,制下百僚。林议曰:“案周官考课,其文备矣,自康王以下,遂以陵迟,此即考课之法存乎其人也。及汉之季,其失岂在乎佐吏之职不密哉?方今军旅,或猥或卒,备之以科条,申之以内外,增减无常,固难一矣。且万目不张举其纲,众毛不整振其领。皋陶仕虞,伊尹臣殷,不仁者远。五帝三王未必如一,而各以治乱。易曰:'易简,而天下之理得矣。'太祖随宜设辟,以遗来今,不患不法古也。以为今之制度,不为疏阔,惟在守一勿失而已。若朝臣能任仲山甫之重,式是百辟,则孰敢不肃?”

景初元年,司徒、司空并缺,散骑侍郎孟康荐林曰:“夫宰相者,天下之所瞻效,诚宜得秉忠履正本德仗义之士,足为海内所师表者。窃见司隶校尉崔林,禀自然之正性,体高雅之弘量。论其所长以比古人,忠直不回则史鱼之俦,清俭守约则季文之匹也。牧守州郡,所在而治,及为外司,万里肃齐,诚台辅之妙器,衮职之良才也。”后年遂为司空,封安阳亭侯,邑六百户。三公封列侯,自林始也。顷之,又进封安阳乡侯。

鲁相上言:“汉旧立孔子庙,褒成侯岁时奉祠,辟雍行礼,必祭先师,王家出谷,春秋祭祀。今宗圣侯奉嗣,未有命祭之礼,宜给牲牢,长吏奉祀,尊为贵神。“制三府议,博士傅祗以春秋传言立在祀典,则孔子是也。宗圣適足继绝世,章盛德耳。至於显立言,崇明德,则宜如鲁相所上。林议以为“宗圣侯亦以王命祀,不为未有命也。周武王封黄帝、尧、舜之后,及立三恪,禹、汤之世,不列于时,复特命他官祭也。今周公已上,达於三皇,忽焉不祀,而其礼经亦存其言。今独祀孔子者,以世近故也。以大夫之后,特受无疆之祀,礼过古帝,义逾汤、武,可谓崇明报德矣,无复重祀於非族也。”

明帝又分林邑,封一子列侯。正始五年薨,谥曰孝侯。子述嗣。

高柔字文惠,陈留圉人也。父靖,为蜀郡都尉。柔留乡里,谓邑中曰:“今者英雄并起,陈留四战之地也。曹将军虽据兖州,本有四方之图,未得安坐守也。而张府君先得志於陈留,吾恐变乘间作也,欲与诸君避之。”众人皆以张邈与太祖善,柔又年少,不然其言。柔从兄幹,袁绍甥也,在河北呼柔,柔举宗从之。会靖卒於西州,时道路艰涩,兵寇纵横,而柔冒艰险诣蜀迎丧,辛苦荼毒,无所不尝,三年乃还。

太祖平袁氏,以柔为县长。县中素闻其名,奸吏数人,皆自引去。柔教曰:“昔邴吉临政,吏尝有非,犹尚容之。况此诸吏,於吾未有失乎!其召复之。”咸还,皆自励,咸为佳吏。高幹既降,顷之以并州叛。柔自归太祖,太祖欲因事诛之,以为刺奸令史。处法允当,狱无留滞,辟为丞相仓曹属。太祖欲遣锺繇等讨张鲁,柔谏,以为今猥遣大兵,西有韩遂、马超,谓为己举,将相扇动作逆,宜先招集三辅,三辅苟平,汉中可传檄而定也。繇入关,遂、超等果反。

魏国初建,为尚书郎。转拜丞相理曹掾,令曰:“夫治定之化,以礼为首。拨乱之政,以刑为先。是以舜流四凶族,皋陶作士。汉祖除秦苛法,萧何定律。掾清识平当,明于宪典,勉恤之哉!”鼓吹宋金等在合肥亡逃。旧法,军征士亡,考竟其妻子。太祖患犹不息,更重其刑。金有母妻及二弟皆给官,主者奏尽杀之。柔启曰:“士卒亡军,诚在可疾,然窃闻其中时有悔者。愚谓乃宜贷其妻子,一可使贼中不信,二可使诱其还心。正如前科,固已绝其意望,而猥复重之,柔恐自今在军之士,见一人亡逃,诛将及己,亦且相随而走,不可复得杀也。此重刑非所以止亡,乃所以益走耳。”太祖曰:“善。”即止不杀金母、弟,蒙活者甚众。

迁为颍川太守,复还为法曹掾。时置校事卢洪、赵达等,使察群下,柔谏曰:“设官分职,各有所司。今置校事,既非居上信下之旨。又达等数以憎爱擅作威福,宜检治之。”太祖曰:“卿知达等,恐不如吾也。要能刺举而辨众事,使贤人君子为之,则不能也。昔叔孙通用群盗,良有以也。”达等后奸利发,太祖杀之以谢於柔。

文帝践阼,以柔为治书侍御史,赐爵关内侯,转加治书执法。民间数有诽谤妖言,帝疾之,有妖言辄杀,而赏告者。柔上疏曰:“今妖言者必戮,告之者辄赏。既使过误无反善之路,又将开凶狡之群相诬罔之渐,诚非所以息奸省讼,缉熙治道也。昔周公作诰,称殷之祖宗,咸不顾小人之怨。在汉太宗,亦除妖言诽谤之令。臣愚以为宜除妖谤赏告之法,以隆天父养物之仁。”帝不即从,而相诬告者滋甚。帝乃下诏:“敢以诽谤相告者,以所告者罪罪之。”於是遂绝。校事刘慈等,自黄初初数年之间,举吏民奸罪以万数,柔皆请惩虚实;其馀小小挂法者,不过罚金。四年,迁为廷尉。

魏初,三公无事,又希与朝政。柔上疏曰:“天地以四时成功,元首以辅弼兴治;成汤仗阿衡之佐,文、武凭旦、望之力,逮至汉初,萧、曹之俦并以元勋代作心膂,此皆明王圣主任臣於上,贤相良辅股肱於下也。今公辅之臣,皆国之栋梁,民所具瞻,而置之三事,不使知政,遂各偃息养高,鲜有进纳,诚非朝廷崇用大臣之义,大臣献可替否之谓也。古者刑政有疑,辄议於槐棘之下。自今之后,朝有疑议及刑狱大事,宜数以咨访三公。三公朝朔望之日,又可特延入,讲论得失,博尽事情,庶有裨起天听,弘益大化。”帝嘉纳焉。

帝以宿嫌,欲枉法诛治书执法鲍勋,而柔固执不从诏命。帝怒甚,遂召柔诣台;遣使者承指至廷尉考竟勋,勋死乃遣柔还寺。

明帝即位,封柔延寿亭侯。时博士执经,柔上疏曰:“臣闻遵道重学,圣人洪训;褒文崇儒,帝者明义。昔汉末陵迟,礼乐崩坏,雄战虎争,以战陈为务,遂使儒林之群,幽隐而不显。太祖初兴,愍其如此,在於拨乱之际,并使郡县立教学之官。高祖即位,遂阐其业,兴复辟雍,州立课试,於是天下之士,复闻庠序之教,亲俎豆之礼焉。陛下临政,允迪叡哲,敷弘大猷,光济先轨,虽夏启之承基,周成之继业,诚无以加也。然今博士皆经明行脩,一国清选,而使迁除限不过长,惧非所以崇显儒术,帅励怠惰也。孔子称'举善而教不能则劝',故楚礼申公,学士锐精,汉隆卓茂,搢绅竞慕。臣以为博士者,道之渊薮,六艺所宗,宜随学行优劣,待以不次之位。敦崇道教,以劝学者,於化为弘。”帝纳之。

后大兴殿舍,百姓劳役;广采众女,充盈后宫;后宫皇子连夭,继嗣未育。柔上疏曰:“二虏狡猾,潜自讲肄,谋动干戈,未图束手;宜畜养将士,缮治甲兵,以逸待之。而顷兴造殿舍,上下劳扰;若使吴、蜀知人虚实,通谋并势,复俱送死,甚不易也。昔汉文惜十家之资,不营小台之娱;去病虑匈奴之害,不遑治第之事。况今所损者非惟百金之费,所忧者非徒北狄之患乎?可粗成见所营立,以充朝宴之仪。乞罢作者,使得就农。二方平定,复可徐兴。昔轩辕以二十五子,传祚弥远;周室以姬国四十,历年滋多。陛下聪达,穷理尽性,而顷皇子连多夭逝,熊罴之祥又未感应。群下之心,莫不悒戚。周礼,天子后妃以下百二十人,嫔嫱之仪,既以盛矣。窃闻后庭之数,或复过之,圣嗣不昌,殆能由此。臣愚以为可妙简淑媛,以备内官之数,其馀尽遣还家。且以育精养神,专静为宝。如此,则螽斯之徵,可庶而致矣。”帝报曰:“知卿忠允,乃心王室,辄克昌言;他复以闻。“

时猎法甚峻。宜阳典农刘龟窃於禁内射兔,其功曹张京诣校事言之。帝匿京名,收龟付狱。柔表请告者名,帝大怒曰:“刘龟当死,乃敢猎吾禁地。送龟廷尉,廷尉便当考掠,何复请告者主名,吾岂妄收龟邪?”柔曰:“廷尉,天下之平也,安得以至尊喜怒而毁法乎?”重复为奏,辞指深切。帝意寤,乃下京名。即还讯,各当其罪。

时制,吏遭大丧者,百日后皆给役。有司徒吏解弘遭父丧,后有军事,受敕当行,以疾病为辞。诏怒曰:“汝非曾、闵,何言毁邪?”促收考竟。柔见弘信甚羸劣,奏陈其事,宜加宽贷。帝乃诏曰:“孝哉弘也!其原之。”

初,公孙渊兄晃,为叔父恭任内侍,先渊未反,数陈其变。及渊谋逆,帝不忍巿斩,欲就狱杀之。柔上疏曰:“书称'用罪伐厥死,用德彰厥善',此王制之明典也。晃及妻子,叛逆之类,诚应枭县,勿使遗育。而臣窃闻晃先数自归,陈渊祸萌,虽为凶族,原心可恕。夫仲尼亮司马牛之忧,祁奚明叔向之过,在昔之美义也。臣以为晃信有言,宜贷其死;苟自无言,便当巿斩。今进不赦其命,退不彰其罪,闭著囹圄,使自引分,四方观国,或疑此举也。”帝不听,竟遣使赍金屑饮晃及其妻子,赐以棺、衣,殡敛於宅

是时,杀禁地鹿者身死,财产没官,有能觉告者厚加赏赐。柔上疏曰:“圣王之御世,莫不以广农为务,俭用为资。夫农广则谷积,用俭则财畜,畜财积谷而有忧患之虞者,未之有也。古者,一夫不耕,或为之饥;一妇不织,或为之寒。中间已来,百姓供给众役,亲田者既减,加顷复有猎禁,群鹿犯暴,残食生苗,处处为害,所伤不赀。民虽障防,力不能御。至如荥阳左右,周数百里,岁略不收,元元之命,实可矜伤。方今天下生财者甚少,而麋鹿之损者甚多。卒有兵戎之役,凶年之灾,将无以待之。惟陛下览先圣之所念,愍稼穑之艰难,宽放民间,使得捕鹿,遂除其禁,则众庶久济,莫不悦豫矣。“

顷之,护军营士窦礼近出不还。营以为亡,表言逐捕,没其妻盈及男女为官奴婢。盈连至州府,称冤自讼,莫有省者。乃辞诣廷尉。柔问曰:“汝何以知夫不亡?”盈垂泣对曰:“夫少单特,养一老妪为母,事甚恭谨,又哀儿女,抚视不离,非是轻狡不顾室家者也。”柔重问曰:“汝夫不与人有怨雠乎?”对曰:“夫良善,与人无雠。”又曰:“汝夫不与人交钱财乎?“对曰:“尝出钱与同营士焦子文,求不得。”时子文適坐小事系狱,柔乃见子文,问所坐。言次,曰:“汝颇曾举人钱不?“子文曰:“自以单贫,初不敢举人钱物也。”柔察子文色动,遂曰:“汝昔举窦礼钱,何言不邪?”子文怪知事露,应对不次。柔曰:“汝已杀礼,便宜早服。”子文於是叩头,具首杀礼本末,埋藏处所。柔便遣吏卒,承子文辞往掘礼,即得其尸。诏书复盈母子为平民。班下天下,以礼为戒。

在官二十三年,转为太常,旬日迁司空,后徙司徒。太傅司马宣王奏免曹爽,皇太后诏召柔假节行大将军事,据爽营。太傅谓柔曰:“君为周勃矣。”爽诛,进封万岁乡侯。高贵乡公即位,进封安国侯,转为太尉。常道乡公即位,增邑并前四千,前后封二子亭侯。景元四年,年九十薨,谥曰元侯。孙浑嗣。咸熙中,开建五等,以柔等著勋前朝,改封浑昌陆子。

孙礼字德达,涿郡容城人也。太祖平幽州,召为司空军谋掾。初丧乱时,礼与母相失,同郡马台求得礼母,礼推家财尽以与台。台后坐法当死,礼私导令逾狱自首,既而曰:“臣无逃亡之义。”径诣刺奸主簿温恢。恢嘉之,具白太祖,各减死一等。

后除河间郡丞,稍迁荥阳都尉。鲁山中贼数百人,保固险阻,为民作害;乃徙礼为鲁相。礼至官,出俸谷,发吏民,募首级,招纳降附,使还为间,应时平泰。历山阳、平原、平昌、琅邪太守。从大司马曹休征吴於夹石,礼谏以为不可深入,不从而败。迁阳平太守,入为尚书。

明帝方修宫室,而节气不和,天下少谷。礼固争,罢役,诏曰:“敬纳谠言,促遣民作。”时李惠监作,复奏留一月,有所成讫。礼径至作所,不复重奏,称诏罢民,帝奇其意而不责也。

帝猎於大石山,虎趋乘舆,礼便投鞭下马,欲奋剑斫虎,诏令礼上马。明帝临崩之时,以曹爽为大将军,宜得良佐,於床下受遗诏,拜礼大将军长史,加散骑常侍。礼亮直不挠,爽弗便也,以为扬州刺史,加伏波将军,赐爵关内侯。吴大将全琮帅数万众来侵寇,时州兵休使,在者无几。礼躬勒卫兵御之,战於芍陂,自旦及暮,将士死伤过半。礼犯蹈白刃,马被数创,手秉枹鼓,奋不顾身,贼众乃退。诏书慰劳,赐绢七百匹。礼为死事者设祀哭临,哀号发心,皆以绢付亡者家,无以入身。

徵拜少府,出为荆州刺史,迁冀州牧。太傅司马宣王谓礼曰:“今清河、平原争界八年,更二刺史,靡能决之;虞、芮待文王而了,宜善令分明。”礼曰:“讼者据墟墓为验,听者以先老为正,而老者不可加以榎楚,又墟墓或迁就高敞,或徙避仇雠。如今所闻,虽皋陶犹将为难。若欲使必也无讼,当以烈祖初封平原时图决之。何必推古问故,以益辞讼?昔成王以桐叶戏叔虞,周公便以封之。今图藏在天府,便可於坐上断也,岂待到州乎?”宣王曰:“是也。当别下图。“礼到,案图宜属平原。而曹爽信清河言,下书云:“图不可用,当参异同。”礼上疏曰:“管仲霸者之佐,其器又小,犹能夺伯氏骈邑,使没齿无怨言。臣受牧伯之任,奉圣朝明图,验地著之界,界实以王翁河为限;而鄃以马丹候为验,诈以鸣犊河为界。假虚讼诉,疑误台阁。窃闻众口铄金,浮石沈木,三人成巿虎,慈母投其杼。今二郡争界八年,一朝决之者,缘有解书图画,可得寻案擿校也。平原在两河,向东上,其间有爵堤,爵堤在高唐西南,所争地在高唐西北,相去二十馀里,可谓长叹息流涕者也。案解与图奏而鄃不受诏,此臣软弱不胜其任,臣亦何颜尸禄素餐。”辄束带著履,驾车待放。爽见礼奏,大怒。劾礼怨望,结刑五岁。在家期年,众人多以为言,除城门校尉。

时匈奴王刘靖部众强盛,而鲜卑数寇边,乃以礼为并州刺史,加振武将军,使持节,护匈奴中郎将。往见太傅司马宣王,有忿色而无言。宣王曰:“卿得并州,少邪?恚理分界失分乎?今当远别,何不欢也!”礼曰:“何明公言之乖细也!礼虽不德,岂以官位往事为意邪?本谓明公齐踪伊、吕,匡辅魏室,上报明帝之讬,下建万世之勋。今社稷将危,天下凶凶,此礼之所以不悦也。“因涕泣横流。宣王曰:“且止,忍不可忍。”爽诛后,入为司隶校尉,凡临七郡五州,皆有威信。迁司空,封大利亭侯,邑一百户。礼与卢毓同郡时辈,而情好不睦。为人虽互有长短,然名位略齐云。嘉平二年薨,谥曰景侯。孙元嗣。

王观字伟台,东郡廪丘人也。少孤贫励志,太祖召为丞相文学掾,出为高唐、阳泉、酂、任令,所在称治。文帝践阼,入为尚书郎、廷尉监,出为南阳、涿郡太守。涿北接鲜卑,数有寇盗,观令边民十家已上,屯居,筑京候。时或有不愿者,观乃假遣朝吏,使归助子弟,不与期会,但敕事讫各还。於是吏民相率不督自劝,旬日之中,一时俱成。守御有备,寇钞以息。明帝即位,下诏书使郡县条为剧、中、平者。主者欲言郡为中平,观教曰:“此郡滨近外虏,数有寇害,云何不为剧邪?”主者曰:“若郡为外剧,恐於明府有任子。”观曰:“夫君者,所以为民也。今郡在外剧,则於役条当有降差。岂可为太守之私而负一郡之民乎?”遂言为外剧郡,后送任子诣邺。时观但有一子而又幼弱。其公心如此。观治身清素,帅下以俭,僚属承风,莫不自励。

明帝幸许昌,召观为治书侍御史,典行台狱。时多有仓卒喜怒,而观不阿意顺指。太尉司马宣王请观为从事中郎,迁为尚书,出为河南尹,徙少府。大将军曹爽使材官张达斫家屋材,及诸私用之物,观闻知,皆录夺以没官。少府统三尚方御府内藏玩弄之宝,爽等奢放,多有干求,惮观守法,乃徙为太仆。司马宣王诛爽,使观行中领军,据爽弟羲营,赐爵关内侯,复为尚书,加驸马都尉。高贵乡公即位,封中乡亭侯。顷之,加光禄大夫,转为右仆射。常道乡公即位,进封阳乡侯,增邑千户,并前二千五百户。迁司空,固辞,不许,遣使即第拜授。就官数日,上送印绶,辄自舆归里舍。薨于家,遗令藏足容棺,不设明器,不封不树。谥曰肃侯。子悝嗣。咸熙中,开建五等,以观著勋前朝,改封悝胶东子。

评曰:韩暨处以静居行化,出以任职流称;崔林简朴知能;高柔明於法理;孙礼刚断伉厉;王观清劲贞白:咸克致公辅。及暨年过八十,起家就列;柔保官二十年,元老终位:比之徐邈、常林,於兹为疚矣。

\part{魏书二十五}
\chapter{辛毗杨阜高堂隆传第二十五}

辛毗字佐治,颍川阳翟人也。其先建武中,自陇西东迁。毗随兄评从袁绍。太祖为司空,辟毗,毗不得应命。及袁尚攻兄谭於平原,谭使毗诣太祖求和。太祖将征荆州,次于西平。毗见太祖致谭意,太祖大悦。后数日,更欲先平荆州,使谭、尚自相弊。他日置酒,毗望太祖色,知有变,以语郭嘉。嘉白太祖,太祖谓毗曰:“谭可信?尚必可克不?”毗对曰:“明公无问信与诈也,直当论其势耳。袁氏本兄弟相伐,非谓他人能间其间,乃谓天下可定於己也。今一旦求救於明公,此可知也。显甫见显思困而不能取,此力竭也。兵革败於外,谋臣诛於内,兄弟谗阋,国分为二;连年战伐,而介胄生虮虱,加以旱蝗,饥馑并臻,国无囷仓,行无裹粮,天灾应於上,人事困於下,民无愚智,皆知土崩瓦解,此乃天亡尚之时也。兵法称有石城汤池带甲百万而无粟者,不能守也。今往攻邺,尚不还救,即不能自守。还救,即谭踵其后。以明公之威,应困穷之敌,击疲弊之寇,无异迅风之振秋叶矣。天以袁尚与明公,明公不取而伐荆州。荆州丰乐,国未有衅。仲虺有言:'取乱侮亡。'方今二袁不务远略而内相图,可谓乱矣;居者无食,行者无粮,可谓亡矣。朝不谋夕,民命靡继,而不绥之,欲待他年;他年或登,又自知亡而改脩厥德,失所以用兵之要矣。今因其请救而抚之,利莫大焉。且四方之寇,莫大於河北;河北平,则六军盛而天下震。”太祖曰:“善。”乃许谭平,次于黎阳。明年攻邺,克之,表毗为议郎。

久之,太祖遣都护曹洪平下辩,使毗与曹休参之,令曰:“昔高祖贪财好色,而良、平匡其过失。今佐治、文烈忧不轻矣。”军还,为丞相长史。

文帝践阼,迁侍中,赐爵关内侯。时议改正朔。毗以魏氏遵舜、禹之统,应天顺民;至於汤、武,以战伐定天下,乃改正朔。孔子曰“行夏之时”,左氏传曰“夏数为得天正”,何必期於相反。帝善而从之。

帝欲徙冀州士家十万户实河南。时连蝗民饥,群司以为不可,而帝意甚盛。毗与朝臣俱求见,帝知其欲谏,作色以见之,皆莫敢言。毗曰:“陛下欲徙士家,其计安出?”帝曰:“卿谓我徙之非邪?“毗曰:“诚以为非也。“帝曰:“吾不与卿共议也。”毗曰:“陛下不以臣不肖,置之左右,厕之谋议之官,安得不与臣议邪!臣所言非私也,乃社稷之虑也,安得怒臣!”帝不答,起入内;毗随而引其裾,帝遂奋衣不还,良久乃出,曰:“佐治,卿持我何太急邪?”毗曰:“今徙,既失民心,又无以食也。“帝遂徙其半。尝从帝射雉,帝曰:“射雉乐哉”毗曰:“於陛下甚乐,而於群下甚苦。”帝默然,后遂为之稀出。

上军大将军曹真征朱然于江陵,毗行军师。还,封广平亭侯。帝欲大兴军征吴,毗谏曰:“吴、楚之民,险而难御,道隆后服,道洿先叛,自古患之,非徒今也。今陛下祚有海内,夫不宾者,其能久乎?昔尉佗称帝,子阳僣号,历年未几,或臣或诛。何则,违逆之道不久全,而大德无所不服也。方今天下新定,土广民稀。夫庙算而后出军,犹临事而惧,况今庙算有阙而欲用之,臣诚未见其利也。先帝屡起锐师,临江而旋。今六军不增於故,而复循之,此未易也。今日之计,莫若脩范蠡之养民,法管仲之寄政,则充国之屯田,明仲尼之怀远;十年之中,强壮未老,童龀胜战,兆民知义,将士思奋,然后用之,则役不再举矣。“帝曰:“如卿意,更当以虏遗子孙邪?”毗对曰:“昔周文王以纣遗武王,唯知时也。苟时未可,容得已乎!”帝竟伐吴,至江而还。

明帝即位,进封颍乡侯,邑三百户。时中书监刘放、令孙资见信於主,制断时政,大臣莫不交好,而毗不与往来。毗子敞谏曰:“今刘、孙用事,众皆影附,大人宜小降意,和光同尘;不然必有谤言。”毗正色曰:“主上虽未称聪明,不为闇劣。吾之立身,自有本未。就与刘、孙不平,不过令吾不作三公而已,何危害之有?焉有大丈夫欲为公而毁其高节者邪”冗从仆射毕轨表言:“尚书仆射王思精勤旧吏,忠亮计略不如辛毗,毗宜代思。“帝以访放、资,放、资对曰:“陛下用思者,诚欲取其效力,不贵虚名也。毗实亮直,然性刚而专,圣虑所当深察也。”遂不用。出为卫尉。

帝方脩殿舍,百姓劳役,毗上疏曰:“窃闻诸葛亮讲武治兵,而孙权巿马辽东,量其意指,似欲相左右。备豫不虞,古之善政,而今者宫室大兴,加连年谷麦不收。诗云:'民亦劳止,迄可小康,惠此中国,以绥四方。'唯陛下为社稷计。”帝报曰:“二虏未灭而治宫室,直谏者立名之时也。夫王者之都,当及民劳兼办,使后世无所复增,是萧何为汉规摹之略也。今卿为魏重臣,亦宜解其大归。”帝又欲平北芒,令於其上作台观,则见孟津。毗谏曰:“天地之性,高高下下,今而反之,既非其理;加以损费人功,民不堪役。且若九河盈溢,洪水为害,而丘陵皆夷,将何以御之?”帝乃止。

青龙二年,诸葛亮率众出渭南。先是,大将军司马宣王数请与亮战,明帝终不听。是岁恐不能禁,乃以毗为大将军军师,使持节。六军皆肃,准毗节度,莫敢犯违。亮卒,复还为卫尉。薨,谥曰肃侯。子敞嗣,咸熙中为河内太守。

杨阜字义山,天水冀人也。以州从事为牧韦端使诣许,拜安定长史。阜还,关右诸将问袁、曹胜败孰在,阜曰:“袁公宽而不断,好谋而少决;不断则无威,少决则失后事,今虽强,终不能成大业。曹公有雄才远略,决机无疑,法一而兵精,能用度外之人,所任各尽其力,必能济大事者也。”长史非其好,遂去官。而端徵为太仆,其子康代为刺史,辟阜为别驾。察孝廉,辟丞相府,州表留参军事。

马超之战败渭南也,走保诸戎。太祖追至安定,而苏伯反河间,将引军东还。阜时奉使,言於太祖曰:“超有信、布之勇,甚得羌、胡心,西州畏之。若大军还,不严为之备,陇上诸郡非国家之有也。”太祖善之,而军还仓卒,为备不周。超率诸戎渠帅以击陇上郡县,陇上郡县皆应之,惟冀城奉州郡以固守。超尽兼陇右之众,而张鲁又遣大将杨昂以助之,凡万馀人,攻城。阜率国士大夫及宗族子弟胜兵者千馀人,使从弟岳於城上作偃月营,与超接战,自正月至八月拒守而救兵不至。州遣别驾阎温循水潜出求救,为超所杀,於是刺史、太守失色,始有降超之计。阜流涕谏曰:“阜等率父兄子弟以义相励,有死无二;田单之守,不固於此也。弃垂成之功,陷不义之名,阜以死守之。”遂号哭。刺史、太守卒遣人请和,开城门迎超。超入,拘岳於冀,使杨昂杀刺史、太守。

阜内有报超之志,而未得其便。顷之,阜以丧妻求葬假。阜外兄姜叙屯历城。阜少长叙家,见叙母及叙,说前在冀中时事,歔欷悲甚。叙曰:“何为乃尔?”阜曰:“守城不能完,君亡不能死,亦何面目以视息於天下!马超背父叛君,虐杀州将,岂独阜之忧责,一州士大夫皆蒙其耻。君拥兵专制而无讨贼心,此赵盾所以书弑君也。超强而无义,多衅易图耳。”叙母慨然,敕叙从阜计。计定,外与乡人姜隐、赵昂、尹奉、姚琼、孔信、武都人李俊、王灵结谋,定讨超约,使从弟谟至冀语岳,并结安定梁宽、南安赵衢、庞恭等。约誓既明,十七年九月,与叙起兵於卤城。超闻阜等兵起,自将出。而衢、宽等解岳,闭冀城门,讨超妻子。超袭历城,得叙母。叙母骂之曰:“汝背父之逆子,杀君之桀贼,天地岂久容汝,而不早死,敢以面目视人乎!”超怒,杀之。阜与超战,身被五创,宗族昆弟死者七人。超遂南奔张鲁。

陇右平定,太祖封讨超之功,侯者十一人,赐阜爵关内侯。阜让曰:“阜君存无扞难之功,君亡无死节之效,於义当绌,於法当诛;超又不死,无宜苟荷爵禄。”太祖报曰:“君与群贤共建大功,西土之人以为美谈。子贡辞赏,仲尼谓之止善。君其剖心以顺国命。姜叙之母,劝叙早发,明智乃尔,虽杨敞之妻盖不过此。贤哉,贤哉!良史记录,必不坠於地矣。

太祖征汉中,以阜为益州刺史。还,拜金城太守,未发,转武都太守。郡滨蜀汉,阜请依龚遂故事,安之而已。会刘备遣张飞、马超等从沮道趣下辩,而氐雷定等七部万馀落反应之。太祖遣都护曹洪御超等,超等退还。洪置酒大会,令女倡著罗縠之衣,蹋鼓,一坐皆笑。阜厉声责洪曰:“男女之别,国之大节,何有於广坐之中裸女人形体!虽桀、纣之乱,不甚於此。”遂奋衣辞出。洪立罢女乐,请阜还坐,肃然惮焉。

及刘备取汉中以逼下辩,太祖以武都孤远,欲移之,恐吏民恋土。阜威信素著,前后徙民、氏,使居京兆、扶风、天水界者万馀户,徙郡小槐里,百姓襁负而随之。为政举大纲而已,下不忍欺也。文帝问侍中刘晔等:“武都太守何如人也?”皆称阜有公辅之节。未及用,会帝崩。在郡十馀年,徵拜城门校尉。

阜常见明帝著绣衤冒,被缥绫半褎,阜问帝曰:“此於礼何法服也?”帝默然不答,自是不法服不以见阜。

迁将作大匠。时初治宫室,发美女以充后庭,数出入弋猎。秋,大雨震电,多杀鸟雀。阜上疏曰:“臣闻明主在上,群下尽辞。尧、舜圣德,求非索谏;大禹勤功,务卑宫室;成汤遭旱,归咎责己;周文刑於寡妻,以御家邦;汉文躬行节俭,身衣弋纟弟:此皆能昭令问,贻厥孙谋者也。伏惟陛下奉武皇帝开拓之大业,守文皇帝克终之元绪,诚宜思齐往古圣贤之善治,总观季世放荡之恶政。所谓善治者,务俭约、重民力也;所谓恶政者,从心恣欲,触情而发也。惟陛下稽古世代之初所以明赫,及季世所以衰弱至于泯灭,近览汉末之变,足以动心诫惧矣。曩使桓、灵不废高祖之法,文、景之恭俭,太祖虽有神武,於何所施其能邪?而陛下何由处斯尊哉?今吴、蜀未定,军旅在外,愿陛下动则三思,虑而后行,重慎出入,以往鉴来,言之若轻,成败甚重。顷者天雨,又多卒暴,雷电非常,至杀鸟雀。天地神明,以王者为子也,政有不当,则见灾谴。克己内讼,圣人所记。惟陛下虑患无形之外,慎萌纤微之初,法汉孝文出惠帝美人,令得自嫁;顷所调送小女。远闻不令,宜为后图。诸所缮治,务从约节。书曰:'九族既睦,协和万国。'事思厥宜,以从中道,精心计谋,省息费用。吴、蜀以定,尔乃上安下乐,九亲熙熙。如此以往,祖考心欢,尧舜其犹病诸。今宜开大信於天下,以安众庶,以示远人。”时雍丘王植怨於不齿,藩国至亲,法禁峻密,故阜又陈九族之义焉。诏报曰:“间得密表,先陈往古明王圣主,以讽闇政,切至之辞,款诚笃实。退思补过,将顺匡救,备至悉矣。览思苦言,吾甚嘉之。”

后迁少府。是时大司马曹真伐蜀,遇雨不进。阜上疏曰:“昔文王有赤乌之符,而犹日昃不暇食;武王白鱼入舟,君臣变色。而动得吉瑞,犹尚忧惧,况有灾异而不战竦者哉?今吴、蜀未平,而天屡降变,陛下宜深有以专精应答,侧席而坐,思示远以德,绥迩以俭。间者诸军始进,便有天雨之患,稽阂山险,以积日矣。转运之劳,担负之苦,所费以多,若有不继,必违本图。传曰:'见可而进,知难而退,军之善政也。'徒使六军困於山谷之间,进无所略,退又不得,非主兵之道也。武王还师,殷卒以亡,知天期也。今年凶民饥,宜发明诏损膳减服,技巧珍玩之物,皆可罢之。昔邵信臣为少府於无事之世,而奏罢浮食;今者军用不足,益宜节度。”帝即召诸军还。

后诏大议政治之不便於民者,阜议以为:“致治在於任贤,兴国在於务农。若舍贤而任所私,此忘治之甚者也。广开宫馆,高为台榭,以妨民务,此害农之甚者也。百工不敦其器,而竞作奇巧,以合上欲,此伤本之甚者也。孔子曰:'苛政甚於猛虎。'今守功文俗之吏,为政不通治体,苟好烦苛,此乱民之甚者也。当今之急,宜去四甚,并诏公卿郡国,举贤良方正敦朴之士而选用之,此亦求贤之一端也。

阜又上疏欲省宫人诸不见幸者,乃召御府吏问后宫人数。吏守旧令,对曰:“禁密,不得宣露。”阜怒,杖吏一百,数之曰:“国家不与九卿为密,反与小吏为密乎”帝闻而愈敬惮阜。

帝爱女淑,未期而夭,帝痛之甚,追封平原公主,立庙洛阳,葬於南陵。将自临送,阜上疏曰:“文皇帝、武宣皇后崩,陛下皆不送葬,所以重社稷、备不虞也。何至孩抱之赤子而可送葬也哉?”帝不从。

帝既新作许宫,又营洛阳宫殿观阁。阜上疏曰:“尧尚茅茨而万国安其居,禹卑宫室而天下乐其业;及至殷、周,或堂崇三尺,度以九筵耳。古之圣帝明王,未有极宫室之高丽以彫弊百姓之财力者也。桀作璇室、象廊,纣为倾宫、鹿台,以丧其社稷,楚灵以筑章华而身受其祸;秦始皇作阿房而殃及其子,天下叛之,二世而灭。夫不度万民之力,以从耳目之欲,未有不亡者也。陛下当以尧、舜、禹、汤、文、武为法则,夏桀、殷纣、楚灵、秦皇为深诫。高高在上,实监后德。慎守天位,以承祖考,巍巍大业,犹恐失之。不夙夜敬止,允恭恤民,而乃自暇自逸,惟宫台是侈是饰,必有颠覆危亡之祸。易曰:'丰其屋,蔀其家,闚其户,阒其无人。'王者以天下为家,言丰屋之祸,至於家无人也。方今二虏合从,谋危宗庙,十万之军,东西奔赴,边境无一日之娱。农夫废业,民有饥色。陛下不以是为忧,而营作宫室,无有已时。使国亡而臣可以独存,臣又不言也;君作元首,臣为股肱,存亡一体,得失同之。孝经曰:'天子有争臣七人,虽无道不失其天下。'臣虽驽怯,敢忘争臣之义?言不切至,不足以感寤陛下。陛下不察臣言,恐皇祖烈考之祚,将坠于地。使臣身死有补万一,则死之日,犹生之年也。谨叩棺沐浴,伏俟重诛。”奏御,天子感其忠言,手笔诏答。每朝廷会议,阜常侃然以天下为己任。数谏争,不听,乃屡乞逊位,未许。会卒,家无馀财。孙豹嗣。

高堂隆字升平,泰山平阳人,鲁高堂生后也。少为诸生,泰山太守薛悌命为督邮。郡督军与悌争论,名悌而呵之。隆按剑叱督军曰:“昔鲁定见侮,仲尼历阶;赵弹秦筝,相如进缶。临臣名君,义之所讨也。”督军失色,悌惊起止之。后去吏,避地济南。

建安十八年,太祖召为丞相军议掾,后为历城侯徽文学,转为相。徽遭太祖丧,不哀,反游猎驰骋;隆以义正谏,甚得辅导之节。黄初中,为堂阳长,以选为平原王傅。王即尊位,是为明帝。以隆为给事中、博士、驸马都尉。帝初践阼,群臣或以为宜飨会,隆曰:“唐、虞有遏密之哀,高宗有不言之思,是以至德雍熙,光于四海。”以为不宜为会,帝敬纳之。迁陈留太守。犊民酉牧,年七十馀,有至行,举为计曹掾;帝嘉之,特除郎中以显焉。徵隆为散骑常侍,赐爵关内侯。

青龙中,大治殿舍,西取长安大钟。隆上疏曰;“昔周景王不仪刑文、武之明德,忽公旦之圣制,既铸大钱,又作大钟,单穆公谏而弗听,泠州鸠对而弗从,遂迷不反,周德以衰,良史记焉,以为永鉴。然今之小人,好说秦、汉之奢靡以荡圣心,求取亡国不度之器,劳役费损,以伤德政,非所以兴礼乐之和,保神明之休也。”是日,帝幸上方,隆与卞兰从。帝以隆表授兰,使难隆曰:“兴衰在政,乐何为也?化之不明,岂钟之罪?”隆曰:“夫礼乐者,为治之大本也。故箫韶九成,凤皇来仪,雷鼓六变,天神以降,政是以平,刑是以错,和之至也。新声发响,商辛以陨,大钟既铸,周景以弊,存亡之机,恒由斯作,安在废兴之不阶也?君举必书,古之道也,作而不法,何以示后?圣王乐闻其阙,故有箴规之道;忠臣愿竭其节,故有匪躬之义也。”帝称善。

迁侍中,犹领太史令。崇华殿灾,诏问隆:“此何咎?於礼,宁有祈禳之义乎?”隆对曰:“夫灾变之发,皆所以明教诫也,惟率礼脩德,可以胜之。易传曰:'上不俭,下不节,孽火烧其室。'又曰:'君高其台,天火为灾。'此人君苟饰宫室,不知百姓空竭,故天应之以旱,火从高殿起也。上天降鉴,故谴告陛下;陛下宜增崇人道,以答天意。昔太戊有桑谷生於朝,武丁有雊雉登於鼎,皆闻灾恐惧,侧身脩德,三年之后,远夷朝贡,故号曰中宗、高宗。此则前代之明鉴也。今案旧占,灾火之发,皆以台榭宫室为诫。然今宫室之所以充广者,实由宫人猥多之故。宜简择留其淑懿,如周之制,罢省其馀。此则祖己之所以训高宗,高宗之所以享远号也。”诏问隆:“吾闻汉武帝时,柏梁灾,而大起宫殿以厌之,其义云何?”隆对曰:“臣闻西京柏梁既灾,越巫陈方,建章是经,以厌火祥;乃夷越之巫所为,非圣贤之明训也。五行志曰:'柏梁灾,其后有江充巫蛊卫太子事。'如志之言,越巫建章无所厌也。孔子曰:'灾者脩类应行,精祲相感,以戒人君。'是以圣主睹灾责躬,退而脩德,以消复之。今宜罢散民役。宫室之制,务从约节,内足以待风雨,外足以讲礼仪。清埽所灾之处,不敢於此有所立作,莆、嘉禾必生此地,以报陛下虔恭之德。岂可疲民之力,竭民之财!实非所以致符瑞而怀远人也。”帝遂复崇华殿,时郡国有九龙见,故改曰九龙殿。

陵霄阙始构,有鹊巢其上,帝以问隆,对曰:“诗云'维鹊有巢,维鸠居之'。今兴宫室,起陵霄阙,而鹊巢之,此宫室未成身不得居之象也。天意若曰,宫室未成,将有他姓制御之,斯乃上天之戒也。夫天道无亲,惟与善人,不可不深防,不可不深虑。夏、商之季,皆继体也,不钦承上天之明命,惟谗谄是从,废德適欲,故其亡也忽焉。太戊、武丁,睹灾竦惧,祗承天戒,故其兴也勃焉。今若休罢百役,俭以足用,增崇德政,动遵帝则,除普天之所患,兴兆民之所利,三王可四,五帝可六,岂惟殷宗转祸为福而已哉!臣备腹心,苟可以繁祉圣躬,安存社稷,臣虽灰身破族,犹生之年也。岂惮忤逆之灾,而令陛下不闻至言乎?”於是帝改容动色。

是岁,有星孛于大辰。隆上疏曰:“凡帝王徙都立邑,皆先定天地社稷之位,敬恭以奉之。将营宫室,则宗庙为先,厩库为次,居室为后。今圜丘、方泽、南北郊、明堂、社稷,神位未定,宗庙之制又未如礼,而崇饰居室,士民失业。外人咸云宫人之用,与兴戎军国之费,所尽略齐。民不堪命,皆有怨怒。书曰'天聪明自我民聪明,天明畏自我民明威',舆人作颂,则向以五福,民怒吁嗟,则威以六极,言天之赏罚,随民言,顺民心也。是以临政务在安民为先,然后稽古之化,格于上下,自古及今,未尝不然也。夫采椽卑宫,唐、虞、大禹之所以垂皇风也;玉台琼室,夏癸、商辛之所以犯昊天也。今之宫室,实违礼度,乃更建立九龙,华饰过前。天彗章灼,始起於房心,犯帝坐而干紫微,此乃皇天子爱陛下,是以发教戒之象,始卒皆於尊位,殷勤郑重,欲必觉寤陛下;斯乃慈父恳切之训,宜崇孝子祗耸之礼,以率先先下,以昭示后昆,不宜有忽,以重天怒。”

时军国多事,用法深重。隆上疏曰:“夫拓迹垂统,必俟圣明,辅世匡治,亦须良佐,用能庶绩其凝而品物康乂也。夫移风易俗,宣明道化,使四表同风,回首面内,德教光熙,九服慕义,固非俗吏之所能也。今有司务纠刑书,不本大道,是以刑用而不措,俗弊而不敦。宜崇礼乐,班叙明堂,修三雍、大射、养老,营建郊庙,尊儒士,举逸民,表章制度,改正朔,易服色,布恺悌,尚俭素,然后备礼封禅,归功天地,使雅颂之声盈于六合,缉熙之化混于后嗣。斯盖至治之美事,不朽之贵业也。然九域之内,可揖让而治,尚何忧哉!不正其本而救其末,譬犹棼丝,非政理也。可命群公卿士通儒,造具其事,以为典式。”隆又以为改正朔,易服色,殊徽号,异器械,自古帝王所以神明其政,变民耳目,故三春称王,明三统也。於是敷演旧章,奏而改焉。帝从其议,改青龙五年春三月为景初元年孟夏四月,服色尚黄,牺牲用白,从地正也。

迁光禄勋。帝愈增崇宫殿,彫饰观阁,凿太行之石英,采谷城之文石,起景阳山於芳林之园,建昭阳殿於太极之北,铸作黄龙凤皇奇伟之兽,饰金墉、陵云台、陵霄阙。百役繁兴,作者万数,公卿以下至于学生,莫不展力,帝乃躬自掘土以率之。而辽东不朝。悼皇后崩。天作淫雨,冀州水出,漂没民物。隆上疏切谏曰:

盖“天地之大德曰生,圣人之大宝曰位;何以守位?曰仁;何以聚人?曰财”。然则士民者,乃国家之镇也;谷帛者,乃士民之命也。谷帛非造化不育,非人力不成。是以帝耕以劝农,后桑以成服,所以昭事上帝,告虔报施也。昔在伊唐,世值阳九厄运之会,洪水滔天,使鲧治之,绩用不成,乃举文命,随山刊木,前后历年二十二载。灾眚之甚,莫过於彼,力役之兴,莫久於此。尧、舜君臣,南面而已。禹敷九州,庶士庸勋,各有等差,君子小人,物有服章。今无若时之急,而使公卿大夫并与厮徒共供事役,闻之四夷,非嘉声也,垂之竹帛,非令名也。是以有国有家者,近取诸身,远取诸物,妪煦养育,故称“恺悌君子,民之父母”。今上下劳役,疾病凶荒,耕稼者寡,饥馑荐臻,无以卒岁;宜加愍恤,以救其困。

臣观在昔书籍所载,天人之际,未有不应也。是以古先哲王,畏上天之明命,循阴阳之逆顺,矜矜业业,惟恐有违。然后治道用兴,德与神符,灾异既发,惧而脩政,未有不延期流祚者也。爰及末叶,闇君荒主,不崇先王之令轨,不纳正士之直言,以遂其情志,恬忽变戒,未有不寻践祸难,至於颠覆者也。

天道既著,请以人道论之。夫六情五性,同在於人,嗜欲廉贞,各居其一。及其动也,交争于心。欲强质弱,则纵滥不禁;精诚不制,则放溢无极。夫情之所在,非好则美,而美好之集,非人力不成,非谷帛不立。情苟无极,则人不堪其劳,物不充其求。劳求并至,将起祸乱。故不割情,无以相供。仲尼云:“人无远虑,必有近忧。”由此观之,礼义之制,非苟拘分,将以远害而兴治也。

今吴、蜀二贼,非徒白地小虏、聚邑之寇,乃据险乘流,跨有士众,僣号称帝,欲与中国争衡。今若有人来告,权、禅并脩德政,复履清俭,轻省租赋,不治玩好,动咨耆贤,事遵礼度。陛下闻之,岂不惕然恶其如此,以为难卒讨灭,而为国忧乎?若使告者曰,彼二贼并为无道,崇侈无度,役其士民,重其徵赋,下不堪命,吁嗟日甚。陛下闻之,岂不勃然忿其困我无辜之民,而欲速加之诛,其次,岂不幸彼疲弊而取之不难乎?苟如此,则可易心而度,事义之数亦不远矣。

且秦始皇不筑道德之基,而筑阿房之宫,不忧萧墙之变,而脩长城之役。当其君臣为此计也,亦欲立万世之业,使子孙长有天下,岂意一朝匹夫大呼,而天下倾覆哉?故臣以为使先代之君知其所行必将至於败,则弗为之矣。是以亡国之主自谓不亡,然后至於亡;贤圣之君自谓将亡,然后至於不亡。昔汉文帝称为贤主,躬行约俭,惠下养民,而贾谊方之,以为天下倒悬,可为痛哭者一,可为流涕者二,可为长叹息者三。况今天下彫弊,民无儋石之储,国无终年之畜,外有强敌,六军暴边,内兴土功,州郡骚动,若有寇警,则臣惧版筑之士不能投命虏庭矣。

又,将吏奉禄,稍见折减,方之於昔,五分居一;诸受休者又绝廪赐,不应输者今皆出半:此为官入兼多於旧,其所出与参少於昔。而度支经用,更每不足,牛肉小赋,前后相继。反而推之,凡此诸费,必有所在。且夫禄赐谷帛,人主所以惠养吏民而为之司命者也,若今有废,是夺其命矣。既得之而又失之,此生怨之府也。周礼,大府掌九赋之财,以给九式之用,入有其分,出有其所,不相干乘而用各足。各足之后,乃以式贡之馀,供王玩好。又上用财,必考于司会。今陛下所与共坐廊庙治天下者,非三司九列,则台阁近臣,皆腹心造膝,宜在无讳。若见丰省而不敢以告,从命奔走,惟恐不胜,是则具臣,非鲠辅也。昔李斯教秦二世曰:“为人主而不恣睢,命之曰天下桎梏。”二世用之,秦国以覆,斯亦灭族。是以史迁议其不正谏,而为世诫。

书奏,帝览焉,谓中书监、令曰:“观隆此奏,使朕惧哉!”

隆疾笃,口占上疏曰:

曾子有疾,孟敬子问之。曾子曰:“鸟之将死,其鸣也哀;人之将死,其言也善。”臣寝疾病,有增无损,常惧奄忽,忠款不昭。臣之丹诚,岂惟曾子,愿陛下少垂省览!涣然改往事之过谬,勃然兴来事之渊塞,使神人向应,殊方慕义,四灵效珍,玉衡曜精,则三王可迈,五帝可越,非徒继体守文而已也。

臣常疾世主莫不思绍尧、舜、汤、武之治,而蹈踵桀、纣、幽、厉之迹,莫不蚩笑季世惑乱亡国之主,而不登践虞、夏、殷、周之轨。悲夫!以若所为,求若所致,犹缘木求鱼,煎水作冰,其不可得,明矣。寻观三代之有天下也,圣贤相承,历载数百,尺土莫非其有,一民莫非其臣,万国咸宁,九有有截;鹿台之金,巨桥之粟,无所用之,仍旧南面,夫何为哉!然癸、辛之徒,恃其旅力,知足以拒谏,才足以饰非,谄谀是尚,台观是崇,淫乐是好,倡优是说,作靡靡之乐,安濮上之音。上天不蠲,眷然回顾,宗国为墟,下夷子隶,纣县白旗,桀放条;天子之尊,汤、武有之,岂伊异人,皆明王之胄也。且当六国之时,天下殷炽,秦既兼之,不脩圣道,乃构阿房之宫,筑长城之守,矜夸中国,威服百蛮,天下震竦,道路以目;自谓本枝百叶,永垂洪晖,岂寤二世而灭,社稷崩圮哉?近汉孝武乘文、景之福,外攘夷狄,内兴宫殿,十馀年间,天下嚣然。乃信越巫,怼天迁怒,起建章之宫,千门万户,卒致江充妖蛊之变,至於宫室乖离,父子相残,殃咎之毒,祸流数世。

臣观黄初之际,天兆其戒,异类之鸟,育长燕巢,口爪胸赤,此魏室之大异也,宜防鹰扬之臣於萧墙之内。可选诸王,使君国典兵,往往棋跱,镇抚皇畿,翼亮帝室。昔周之东迁,晋、郑是依,汉吕之乱,实赖朱虚,斯盖前代之明鉴。夫皇天无亲,惟德是辅。民咏德政,则延期过历,下有怨叹,掇录授能。由此观之,天下之天下,非独陛下之天下也。臣百疾所锺,气力稍微,辄自舆出,归还里舍,若遂沈沦,魂而有知,结草以报。

诏曰:“生廉追伯夷,直过史鱼,执心坚白,謇謇匪躬,如何微疾未除,退身里舍?禹以守节,疾笃而济愈。生其强饭专精以自持。”隆卒,遗令薄葬,敛以时服。

初,太和中,中护军蒋济上疏曰“宜遵古封禅”。诏曰:“闻济斯言,使吾汗出流足。”事寝历岁,后遂议脩之,使隆撰其礼仪。帝闻隆没,叹息曰:“天不欲成吾事,高堂生舍我亡也。”子琛嗣爵。

始,景初中,帝以苏林、秦静等并老,恐无能传业者。乃诏曰:“昔先圣既没,而其遗言馀教,著於六艺。六艺之文,礼又为急,弗可斯须离者也。末俗背本,所由来久。故闵子讥原伯之不学,荀卿丑秦世之坑儒,儒学既废,则风化曷由兴哉?方今宿生巨儒,并各年高,教训之道,孰为其继?昔伏生将老,汉文帝嗣以晁错;谷梁寡畴,宣帝承以十郎。其科郎吏高才解经义者三十人,从光禄勋隆、散骑常侍林、博士静,分受四经三礼,主者具为设课试之法。夏侯胜有言:'士病不明经术,经术苟明,其取青紫如俯拾地芥耳。'今学者有能究极经道,则爵禄荣宠,不期而至。可不勉哉!”数年,隆等皆卒,学者遂废。

初,任城栈潜,太祖世历县令,尝督守邺城。时文帝为太子,耽乐田猎,晨出夜还。潜谏曰:“王公设险以固其国,都城禁卫,用戒不虞。大雅云:'宗子维城,无俾城坏。'又曰:'犹之未远,是用大谏。'若逸于游田,晨出昬归,以一日从禽之娱,而忘无垠之衅,愚窃惑之。”太子不悦,然自后游出差简。黄初中,文帝将立郭贵嫔为皇后,潜上疏谏,语在后妃传。明帝时,众役并兴,戚属疏斥,潜上疏曰:“天生蒸民而树之君,所以覆焘群生,熙育兆庶,故方制四海匪为天子,裂土分疆匪为诸侯也。始自三皇,爰暨唐、虞,咸以博济加于天下,醇德以洽,黎元赖之。三王既微,降逮于汉,治日益少,丧乱弘多,自时厥后,亦罔克乂。太祖濬哲神武,芟除暴乱,克复王纲,以开帝业。文帝受天明命,廓恢皇基,践阼七载,每事未遑。陛下圣德,纂承洪绪,宜崇晏晏,与民休息。而方隅匪宁,征夫远戍,有事海外,县旌万里,六军骚动,水陆转运,百姓舍业,日费千金。大兴殿舍,功作万计,徂来之松,刊山穷谷,怪石珷玞,浮于河、淮。都圻之内,尽为甸服,当供槁秸铚粟之调,而为苑囿择禽之府,盛林莽之秽,丰鹿兔之薮;伤害农功,地繁茨棘,灾疫流行,民物大溃,上减和气,嘉禾不植。臣闻文王作丰,经始勿亟,百姓子来,不日而成。灵沼、灵囿,与民共之。今宫观崇侈,彫镂极妙,忘有虞之总期,思殷辛之琼室,禁地千里,举足投网,丽拟阿房,役百乾谿,臣恐民力彫尽,下不堪命也。昔秦据殽函以制六合,自以德高三皇,功兼五帝,欲号谥至万叶,而二世颠覆,愿为黔首,由枝幹既扤,本实先拔也。盖圣王之御世也,克明俊德,庸勋亲;俊乂在官,则功业可隆,亲亲显用,则安危同忧;深根固本,并为幹翼,虽历盛衰,内外有辅。昔成王幼冲,未能莅政,周、吕、召、毕,并在左右;今既无卫侯、康叔之监,分陕所任,又非旦、奭。东宫未建,天下无副。愿陛下留心关塞,永保无极,则海内幸甚。”后为燕中尉,辞疾不就,卒。

评曰:辛毗、杨阜,刚亮公直,正谏匪躬,亚乎汲黯之高风焉。高堂隆学业,志在匡君,因变陈戒,发於恳诚,忠矣哉!及至必改正朔,俾魏祖虞,所谓意过其通者欤!

\part{魏书二十六}
\chapter{满田牵郭传第二十六}

满宠字伯宁,山阳昌邑人也。年十八,为郡督邮。时郡内李朔等各拥部曲,害于平民,太守使宠纠焉。朔等请罪,不复钞略。守高平令。县人张苞为郡督邮,贪秽受取,干乱吏政。宠因其来在传舍,率吏卒出收之,诘责所犯,即日考竟,遂弃官归。

太祖临兖州,辟为从事。及为大将军,辟署西曹属,为许令。时曹洪宗室亲贵,有宾客在界,数犯法,宠收治之。洪书报宠,宠不听。洪白太祖,太祖召许主者。宠知将欲原,乃速杀之。太祖喜曰:“当事不当尔邪?”故太尉杨彪收付县狱,尚书令荀彧、少府孔融等并属宠:“但当受辞,勿加考掠。”宠一无所报,考讯如法。数日,求见太祖,言之曰:“杨彪考讯无他辞语。当杀者宜先彰其罪;此人有名海内,若罪不明,必大失民望,窃为明公惜之。”太祖即日赦出彪。初,彧、融闻考掠彪,皆怒,及因此得了,更善宠。

时袁绍盛於河朔,而汝南绍之本郡,门生宾客布在诸县,拥兵拒守。太祖忧之,以宠为汝南太守。宠募其服从者五百人,率攻下二十馀壁,诱其未降渠帅,於坐上杀十馀人,一时皆平。得户二万,兵二千人,令就田业。

建安十三年,从太祖征荆州。大军还,留宠行奋威将军,屯当阳。孙权数扰东陲,复召宠还为汝南太守,赐爵关内侯。关羽围襄阳,宠助征南将军曹仁屯樊城拒之,而左将军于禁等军以霖雨水长为羽所没。羽急攻樊城,樊城得水,往往崩坏,众皆失色。或谓仁曰:“今日之危,非力所支。可及羽围未合,乘轻船夜走,虽失城,尚可全身。”宠曰:“山水速疾,冀其不久。闻羽遣别将已在郏下,自许以南,百姓扰扰,羽所以不敢遂进者,恐吾军掎其后耳。今若遁去,洪河以南,非复国家有也;君宜待之。”仁曰:“善。”宠乃沈白马,与军人盟誓。会徐晃等救至,宠力战有功,羽遂退。进封安昌亭侯。文帝即王位,迁扬武将军。破吴於江陵有功,更拜伏波将军,屯新野。大军南征,到精湖,宠帅诸军在前,与贼隔水相对。宠敕诸将曰:“今夕风甚猛,贼必来烧军,宜为其备。”诸军皆警。夜半,贼果遣十部伏夜来烧,宠掩击破之,进封南乡侯。黄初三年,假宠节钺。五年,拜前将军。明帝即位,进封昌邑侯。太和二年,领豫州刺史。三年春,降人称吴大严,扬声欲诣江北猎,孙权欲自出。宠度其必袭西阳而为之备,权闻之,退还。秋,使曹休从庐江南入合肥,令宠向夏口。宠上疏曰:“曹休虽明果而希用兵,今所从道,背湖旁江,易进难退,此兵之洼地也。若入无强口,宜深为之备。“宠表未报,休遂深入。贼果从无强口断夹石,要休还路。休战不利,退走。会朱灵等从后来断道,与贼相遇。贼惊走,休军乃得还。是岁休薨,宠以前将军代都督扬州诸军事。汝南兵民恋慕,大小相率,奔随道路,不可禁止。护军表上,欲杀其为首者。诏使宠将亲兵千人自随,其馀一无所问。四年,拜宠征东将军。其冬,孙权扬声欲至合肥,宠表召兖、豫诸军,皆集。贼寻退还,被诏罢兵。宠以为今贼大举而还,非本意也,此必欲伪退以罢吾兵,而倒还乘虚,掩不备也,表不罢兵。后十馀日,权果更来,到合肥城,不克而还。其明年,吴将孙布遣人诣扬州求降,辞云:“道远不能自致,乞兵见迎。”刺史王凌腾布书,请兵马迎之。宠以为必诈,不与兵,而为凌作报书曰:“知识邪正,欲避祸就顺,去暴归道,甚相嘉尚。今欲遣兵相迎,然计兵少则不足相卫,多则事必远闻。且先密计以成本志,临时节度其宜。“宠会被书当入朝,敕留府长史:“若凌欲往迎,勿与兵也。”凌於后索兵不得,乃单遣一督将步骑七百人往迎之。布夜掩击,督将迸走,死伤过半。初,宠与凌共事不平,凌支党毁宠疲老悖谬,故明帝召之。既至,体气康强,见而遣还。宠屡表求留,诏报曰:“昔廉颇强食,马援据鞍,今君未老而自谓已老,何与廉、马之相背邪?其思安边境,惠此中国。”

明年,吴将陆逊向庐江,论者以为宜速赴之。宠曰:“庐江虽小,将劲兵精,守则经时。又贼舍船二百里来,后尾空县,尚欲诱致,今宜听其遂进,但恐走不可及耳。”整军趋杨宜口。贼闻大兵东下,即夜遁。时权岁有来计。青龙元年,宠上疏曰:“合肥城南临江湖,北远寿春,贼攻围之,得据水为势;官兵救之,当先破贼大辈,然后围乃得解。贼往甚易,而兵往救之甚难,宜移城内之兵,其西三十里,有奇险可依,更立城以固守,此为引贼平地而掎其归路,於计为便。“护军将军蒋济议,以为:“既示天下以弱,且望贼烟火而坏城,此为未攻而自拔。一至於此,劫略无限,必以淮北为守。”帝未许。宠重表曰:“孙子言,兵者,诡道也。故能而示之以弱不能,骄之以利,示之以慑。此为形实不必相应也。又曰'善动敌者形之'。今贼未至而移城卻内,此所谓形而诱之也。引贼远水,择利而动,举得於外,则福生於内矣。”尚书赵咨以宠策为长,诏遂报听。其年,权自出,欲围新城,以其远水,积二十日不敢下船。宠谓诸将曰:“权得吾移城,必於其众中有自大之言,今大举来欲要一切之功,虽不敢至,必当上岸耀兵以示有馀。”乃潜遣步骑六千,伏肥城隐处以待之。权果上岸耀兵,宠伏军卒起击之,斩首数百,或有赴水死者。明年,权自将号十万,至合肥新城。宠驰往赴,募壮士数十人,折松为炬,灌以麻油,从上风放火,烧贼攻具,射杀权弟子孙泰。贼於是引退。三年春,权遣兵数千家佃於江北。至八月,宠以为田向收熟,男女布野,其屯卫兵去城远者数百里,可掩击也。遣长吏督三军循江东下,摧破诸屯,焚烧谷物而还。诏美之,因以所获尽为将士赏。

景初二年,以宠年老徵还,迁为太尉。宠不治产业,家无馀财。诏曰:“君典兵在外,专心忧公,有行父、祭遵之风。赐田十顷,谷五百斛,钱二十万,以明清忠俭约之节焉。”宠前后增邑,凡九千六百户,封子孙二人亭侯。正始三年薨,谥曰景侯。子伟嗣。伟以格度知名,官至卫尉。

田豫字国让,渔阳雍奴人也。刘备之奔公孙瓒也,豫时年少,自讬於备,备甚奇之。备为豫州刺史,豫以母老求归,备涕泣与别,曰:“恨不与君共成大事也。”

公孙瓒使豫守东州令,瓒将王门叛瓒,为袁绍将万馀人来攻。众惧欲降。豫登城谓门曰:“卿为公孙所厚而去,意有所不得已也;今还作贼,乃知卿乱人耳。夫挈瓶之智,守不假器,吾既受之矣,何不急攻乎?”门惭而退。瓒虽知豫有权谋而不能任也。瓒败而鲜于辅为国人所推,行太守事,素善豫,以为长史。时雄杰并起,辅莫知所从。豫谓辅曰:“终能定天下者,必曹氏也。宜速归命,无后祸期。”辅从其计,用受封宠。太祖召豫为丞相军谋掾,除颍阴、朗陵令,迁弋阳太守,所在有治。

鄢陵侯彰征代郡,以豫为相。军次易北,虏伏骑击之,军人扰乱,莫知所为。豫因地形,回车结圜陈,弓弩持满於内,疑兵塞其隙。胡不能进,散去。追击,大破之,遂前平代,皆豫策也。

迁南阳太守。先时,郡人侯音反,众数千人在山中为群盗,大为郡患。前太守收其党与五百馀人,表奏皆当死。豫悉见诸系囚,慰谕,开其自新之路,一时破械遣之。诸囚皆叩头,愿自效,即相告语,群贼一朝解散,郡内清静。具以状上,太祖善之。

文帝初,北狄强盛,侵扰边塞,乃使豫持节护乌丸校尉,牵招、解俊并护鲜卑。自高柳以东,濊貊以西,鲜卑数十部,比能、弥加、素利割地统御,各有分界;乃共要誓,皆不得以马与中国市。豫以戎狄为一,非中国之利,乃先构离之,使自为雠敌,互相攻伐。素利违盟,出马千匹与官,为比能所攻,求救於豫。豫恐遂相兼并,为害滋深,宜救善讨恶,示信众狄。单将锐卒,深入虏庭,胡人众多,钞军前后,断截归路。豫乃进军,去虏十馀里结屯营,多聚牛马粪然之,从他道引去。胡见烟火不绝,以为尚在,去,行数十里乃知之。追豫到马城,围之十重,豫密严,使司马建旌旗,鸣鼓吹,将步骑从南门出,胡人皆属目往赴之。豫将精锐自北门出,鼓噪而起,两头俱发,出虏不意,虏众散乱,皆弃弓马步走,追讨二十馀里,僵尸蔽地。又乌丸王骨进桀黠不恭,豫因出塞案行,单将麾下百馀骑入进部。进逆拜,遂使左右斩进,显其罪恶以令众。众皆怖慑不敢动,便以进弟代进。自是胡人破胆,威震沙漠。山贼高艾,众数千人,寇钞,为幽、冀害,豫诱使鲜卑素利部斩艾,传首京都。封豫长乐亭侯。为校尉九年,其御夷狄,恒摧抑兼并,乖散强猾。凡逋亡奸宄,为胡作计不利官者,豫皆构刺搅离,使凶邪之谋不遂,聚居之类不安。事业未究,而幽州刺史王雄支党欲令雄领乌丸校尉,毁豫乱边,为国生事。遂转豫为汝南太守,加殄夷将军。

太和末,公孙渊以辽东叛,帝欲征之而难其人,中领军杨暨举豫应选。乃使豫以本官督青州诸军,假节,往讨之。会吴贼遣使与渊相结,帝以贼众多,又以渡海,诏豫使罢军。豫度贼船垂还,岁晚风急,必畏漂浪,东随无岸,当赴成山。成山无藏船之处,辄便循海,案行地势,及诸山岛,徼截险要,列兵屯守。自入成山,登汉武之观。贼还,果遇恶风,船皆触山沈没,波荡著岸,无所蒙窜,尽虏其众。初,诸将皆笑於空地待贼,及贼破,竞欲与谋,求入海钩取浪船。豫惧穷虏死战,皆不听。初,豫以太守督青州,青州刺史程喜内怀不服,军事之际,多相违错。喜知帝宝爱明珠,乃密上:“豫虽有战功而禁令宽弛,所得器仗珠金甚多,放散皆不纳官。”由是功不见列。

后孙权号十万众攻新城,征东将军满宠欲率诸军救之。豫曰:“贼悉众大举,非徒投射小利,欲质新城以致大军耳。宜听使攻城,挫其锐气,不当与争锋也。城不可拔,众必罢怠;罢怠然后击之,可大克也。若贼见计,必不攻城,势将自走。若便进兵,適入其计。又大军相向,当使难知,不当使自画也。”豫辄上状,天子从之。会贼遁走。后吴复来寇,豫往拒之,贼即退。诸军夜惊,云:“贼复来!”豫卧不起,令众“敢动者斩”。有顷,竟无贼。

景初末,增邑三百,并前五百户。正始初,迁使持节护匈奴中郎将,加振威将军,领并州刺史。外胡闻其威名,相率来献。州界宁肃,百姓怀之。徵为卫尉。屡乞逊位,太傅司马宣王以为豫克壮,书喻未听。豫书答曰:“年过七十而以居位,譬犹钟鸣漏尽而夜行不休,是罪人也。”遂固称疾笃。拜太中大夫,食卿禄。年八十二薨。子彭祖嗣。

豫清俭约素,赏赐皆散之将士。每胡、狄私遗,悉簿藏官,不入家;家常贫匮。虽殊类,咸高豫节。嘉平六年,下诏褒扬,赐其家钱谷。语在徐邈传。

牵招字子经,安平观津人也。年十馀岁,诣同县乐隐受学。后隐为车骑将军何苗长史,招随卒业。值京都乱,苗、隐见害,招俱与隐门生史路等触蹈锋刃,共殡敛隐尸,送丧还归。道遇寇钞,路等皆悉散走。贼欲斫棺取钉,招垂泪请赦。贼义之,乃释而去。由此显名。

冀州牧袁绍辟为督军从事,兼领乌丸突骑。绍舍人犯令,招先斩乃白,绍奇其意而不见罪也。绍卒,又事绍子尚。建安九年,太祖围邺。尚遣招至上党,督致军粮。未还,尚破走,到中山。时尚外兄高幹为并州刺史,招以并州左有恒山之险,右有大河之固,带甲五万,北阻强胡,劝幹迎尚,并力观变。幹既不能,而阴欲害招。招闻之,间行而去,道隔不得追尚,遂东诣太祖。太祖领冀州,辟为从事。

太祖将讨袁谭,而柳城乌丸欲出骑助谭。太祖以招尝领乌丸,遣诣柳城。到,值峭王严,以五千骑当遣诣谭。又辽东太守公孙康自称平州牧,遣使韩忠赍单于印绶往假峭王。峭王大会群长,忠亦在坐。峭王问招:“昔袁公言受天子之命,假我为单于;今曹公复言当更白天子,假我真单于;辽东复持印绶来。如此,谁当为正?”招答曰:“昔袁公承制,得有所拜假;中间违错,天子命曹公代之,言当白天子,更假真单于,是也。辽东下郡,何得擅称拜假也?”忠曰:“我辽东在沧海之东,拥兵百万,又有扶馀、濊貊之用;当今之势,强者为右,曹操独何得为是也?”招呵忠曰:“曹公允恭明哲,翼戴天子,伐叛柔服,宁静四海,汝君臣顽嚚,今恃险远,背违王命,欲擅拜假,侮弄神器,方当屠戮,何敢慢易咎毁大人?”便捉忠头顿筑,拔刀欲斩之。峭王惊怖,徒跣抱招,以救请忠,左右失色。招乃还坐,为峭王等说成败之效,祸福所归,皆下席跪伏,敬受敕教,便辞辽东之使,罢所严骑。

太祖灭谭於南皮,署招军谋掾,从讨乌丸。至柳城,拜护乌丸校尉。还邺,辽东送袁尚首,县在马市,招睹之悲感,设祭头下。太祖义之,举为茂才。从平汉中,太祖还,留招为中护军。事罢,还邺,拜平虏校尉,将兵督青、徐州郡诸军事,击东莱贼,斩其渠率,东土宁静。

文帝践阼,拜招使持节护鲜卑校尉,屯昌平。是时,边民流散山泽,又亡叛在鲜卑中者,处有千数。招广布恩信,招诱降附。建义中郎将公孙集等,率将部曲,咸各归命;使还本郡。又怀来鲜卑素利、弥加等十馀万落,皆令款塞。

大军欲征吴,召招还,至,值军罢,拜右中郎将,出为雁门太守。郡在边陲,虽有候望之备,而寇钞不断。招既教民战陈,又表复乌丸五百馀家租调,使备鞍马,远遣侦候。虏每犯塞,勒兵逆击,来辄摧破,於是吏民胆气日锐,荒野无虞。又构间离散,使虏更相猜疑。鲜卑大人步度根、泄归泥等与轲比能为隙,将部落三万馀家诣郡附塞。敕令还击比能,杀比能弟苴罗侯,及叛乌丸归义侯王同、王寄等,大结怨雠。是以招自出,率将归泥等讨比能於云中故郡,大破之。招通河西鲜卑附头等十馀万家,缮治陉北故上馆城,置屯戍以镇内外,夷虏大小,莫不归心,诸叛亡虽亲戚不敢藏匿,咸悉收送。於是野居晏闭,寇贼静息。招乃简选有才识者,诣太学受业,还相授教,数年中庠序大兴。郡所治广武,井水咸苦,民皆担辇远汲流水,往返七里。招准望地势,因山陵之宜,凿原开渠,注水城内,民赖其益。

明帝即位,赐爵关内侯。太和二年,护乌丸校尉田豫出塞,为轲比能所围於故马邑城,移招求救。招即整勒兵马,欲赴救豫。并州以常宪禁招,招以为节将见围,不可拘於吏议,自表辄行。又并驰布羽檄,称陈形势,云当西北掩取虏家,然后东行,会诛虏身。檄到,豫军踊跃。又遗一通於虏蹊要,虏即恐怖,种类离散。军到故平城,便皆溃走。比能复大合骑来,到故平州塞北。招潜行扑讨,大斩首级。招以蜀虏诸葛亮数出,而比能狡猾,能相交通,表为防备,议者以为县远,未之信也。会亮时在祁山,果遣使连结比能。比能至故北地石城,与相首尾。帝乃诏招,使从便宜讨之。时比能已还漠南,招与刺史毕轨议曰:“胡虏迁徙无常。若劳师远追,则迟速不相及。若欲潜袭,则山溪艰险,资粮转运,难以密办。可使守新兴、雁门二牙门,出屯陉北,外以镇抚,内令兵田,储畜资粮,秋冬马肥,州郡兵合,乘衅征讨,计必全克。”未及施行,会病卒。招在郡十二年,威风远振。其治边之称,次于田豫,百姓追思之。而渔阳傅容在雁门有名绩,继招后,在辽东又有事功云。

招子嘉嗣。次子弘,亦猛毅有招风,以陇西太守随邓艾伐蜀有功,咸熙中为振威护军。嘉与晋司徒李胤同母,早卒。

郭淮字伯济,太原阳曲人也。建安中举孝廉,除平原府丞。文帝为五官将,召淮署为门下贼曹,转为丞相兵曹议令史,从征汉中。太祖还,留征西将军夏侯渊拒刘备,以淮为渊司马。渊与备战,淮时有疾不出。渊遇害,军中扰扰,淮收散卒,推荡寇将军张郃为军主,诸营乃定。其明日,备欲渡汉水来攻。诸将议众寡不敌,备便乘胜,欲依水为陈以拒之。淮曰:“此示弱而不足挫敌,非算也。不如远水为陈,引而致之,半济而后击,备可破也。”既陈,备疑不渡,淮遂坚守,示无还心。以状闻,太祖善之,假郃节,复以淮为司马。文帝即王位,赐爵关内侯,转为镇西长史。又行征羌护军,护左将军张郃、冠军将军杨秋讨山贼郑甘、卢水叛胡,皆破平之。关中始定,民得安业。

黄初元年,奉使贺文帝践阼,而道路得疾,故计远近为稽留。及群臣欢会,帝正色责之曰:“昔禹会诸侯於涂山,防风后至,便行大戮。今溥天同庆而卿最留迟,何也?”淮对曰:“臣闻五帝先教导民以德,夏后政衰,始用刑辟。今臣遭唐虞之世,是以自知免於防风之诛也。”帝悦之,擢领雍州刺史,封射阳亭侯,五年为真。安定羌大帅辟虒反,讨破降之。每羌、胡来降,淮辄先使人推问其亲理,男女多少,年岁长幼;及见,一二知其款曲,讯问周至,咸称神明。

太和二年,蜀相诸葛亮出祁山,遣将军马谡至街亭,高详屯列柳城。张郃击谡,淮攻详营,皆破之。又破陇西名羌唐虒於枹罕,加建威将军。五年,蜀出卤城。是时,陇右无谷,议欲关中大运,淮以威恩抚循羌、胡,家使出谷,平其输调,军食用足,转扬武将军。青龙二年,诸葛亮出斜谷,并田于兰坑。是时司马宣王屯渭南;淮策亮必争北原,宜先据之,议者多谓不然。淮曰:“若亮跨渭登原,连兵北山,隔绝陇道,摇荡民、夷,此非国之利也。”宣王善之,淮遂屯北原。堑垒未成,蜀兵大至,淮逆击之。后数日,亮盛兵西行,诸将皆谓欲攻西围,淮独以为此见形於西,欲使官兵重应之,必攻阳遂耳。其夜果攻阳遂,有备不得上。

正始元年,蜀将姜维出陇西。淮遂进军,追至强中,维退,遂讨羌迷当等,按抚柔氐三千馀落,拔徙以实关中。迁左将军。凉州休屠胡梁元碧等,率种落二千馀家附雍州。淮奏请使居安定之高平,为民保障,其后因置西州都尉。转拜前将军,领州如故。

五年,夏侯玄伐蜀,淮督诸军为前锋。淮度势不利,辄拔军出,故不大败。还假淮节。八年,陇西、南安、金城、西平诸羌饿何、烧戈、伐同、蛾遮塞等相结叛乱,攻围城邑,南招蜀兵,凉州名胡治无戴复叛应之。讨蜀护军夏侯霸督诸军屯为翅。淮军始到狄道,议者佥谓宜先讨定枹罕,内平恶羌,外折贼谋。淮策维必来攻霸,遂入沨中,转南迎霸。维果攻为翅,会淮军適至,维遁退。进讨叛羌,斩饿何、烧戈,降服者万馀落。九年,遮塞等屯河关、白土故城,据河拒军。淮见形上流,密於下渡兵据白土城,击,大破之。治无戴围武威,家属留在西海。淮进军趋西海,欲掩取其累重,会无戴折还,与战於龙夷之北,破走之。令居恶虏在石头山之西,当大道止,断绝王使。淮还过讨,大破之。姜维出石营,从强川,乃西迎治无戴,留阴平太守廖化於成重山筑城,敛破羌保质。淮欲分兵取之。诸将以维众西接强胡,化以据险,分军两持,兵势转弱,进不制维,退不拔化,非计也,不如合而俱西,及胡、蜀未接,绝其内外,此伐交之兵也。淮曰:“今往取化,出贼不意,维必狼顾。比维自致,足以定化,且使维疲於奔命。兵不远西,而胡交自离,此一举而两全之策也。”乃别遣夏侯霸等追维於沓中,淮自率诸军就攻化等。维果驰还救化,皆如淮计。进封都乡侯。

嘉平元年,迁征西将军,都督雍、凉诸军事。是岁,与雍州刺史陈泰协策,降蜀牙门将句安等於翅上。二年,诏曰:“昔汉川之役,几至倾覆。淮临危济难,功书王府。在关右三十馀年,外征寇虏,内绥民夷。比岁以来,摧破廖化,禽虏句安,功绩显著,朕甚嘉之。今以淮为车骑将军、仪同三司,持节、都督如故。”进封阳曲侯,邑凡二千七百八十户,分三百户,封一子亭侯。正元二年薨,追赠大将军,谥曰贞侯。子统嗣。统官至荆州刺史,薨。子正嗣。咸熙中,开建五等,以淮著勋前朝,改封汾阳子。

评曰:满宠立志刚毅,勇而有谋。田豫居身清白,规略明练。牵招秉义壮烈,威绩显著。郭淮方策精详,垂问秦、雍。而豫位止小州,招终於郡守,未尽其用也。

\part{魏书二十七}
\chapter{徐胡二王传第二十七}

徐邈字景山,燕国蓟人也。太祖平河朔,召为丞相军谋掾,试守奉高令,入为东曹议令史。魏国初建,为尚书郎。时科禁酒,而邈私饮至於沈醉。校事赵达问以曹事,邈曰:“中圣人。”达白之太祖,太祖甚怒。度辽将军鲜于辅进曰:“平日醉客谓酒清者为圣人,浊者为贤人,邈性脩慎,偶醉言耳。”竟坐得免刑。后领陇西太守,转为南安。文帝践阼,历谯相,平阳、安平太守,颍川典农中郎将,所在著称,赐爵关内侯。车驾幸许昌,问邈曰:“颇复中圣人不?”邈对曰:“昔子反毙於谷阳,御叔罚於饮酒,臣嗜同二子,不能自惩,时复中之。然宿瘤以丑见传,而臣以醉见识。”帝大笑,顾左右曰:“名不虚立。”迁抚军大将军军师。

明帝以凉州绝远,南接蜀寇,以邈为凉州刺史,使持节领护羌校尉。至,值诸葛亮出祁山,陇右三郡反,邈辄遣参军及金城太守等击南安贼,破之。河右少雨,常苦乏谷,邈上脩武威、酒泉盐池以收虏谷,又广开水田,募贫民佃之,家家丰足,仓库盈溢。乃支度州界军用之馀,以市金帛犬马,通供中国之费。以渐收敛民间私仗,藏之府库。然后率以仁义,立学明训,禁厚葬,断淫祀,进善黜恶,风化大行,百姓归心焉。西域流通,荒戎入贡,皆邈勋也。讨叛羌柯吾有功,封都亭侯,邑三百户,加建威将军。邈与羌、胡从事,不问小过;若犯大罪,先告部帅,使知,应死者乃斩以徇,是以信服畏威。赏赐皆散与将士,无入家者,妻子衣食不充;天子闻而嘉之,随时供给其家。弹邪绳枉,州界肃清。

正始元年,还为大司农。迁为司隶校尉,百寮敬惮之。公事去官。后为光禄大夫,数岁即拜司空,邈叹曰:“三公论道之官,无其人则缺,岂可以老病忝之哉?”遂固辞不受。嘉平元年,年七十八,以大夫薨于家,用公礼葬,谥曰穆侯。子武嗣。六年,朝廷追思清节之士,诏曰:“夫显贤表德,圣王所重;举善而教,仲尼所美。故司空徐邈、征东将军胡质、卫尉田豫皆服职前朝,历事四世,出统戎马,入赞庶政,忠清在公,忧国忘私,不营产业,身没之后,家无馀财,朕甚嘉之。其赐邈等家谷二千斛,钱三十万,布告天下。”邈同郡韩观曼游,有鉴识器幹,与邈齐名,而在孙礼、卢毓先,为豫州刺史,甚有治功,卒官。卢钦著书,称邈曰:“徐公志高行絜,才博气猛。其施之也,高而不狷,絜而不介,博而守约,猛而能宽。圣人以清为难,而徐公之所易也。“或问钦:“徐公当武帝之时,人以为通,自在凉州及还京师,人以为介,何也?”钦答曰;”往者毛孝先、崔季珪等用事,贵清素之士,于时皆变易车服以求名高,而徐公不改其常,故人以为通。比来天下奢靡,转相仿效,而徐公雅尚自若,不与俗同,故前日之通,乃今日之介也。是世人之无常,而徐公之有常也。”

胡质字文德,楚国寿春人也。少与蒋济、朱绩俱知名於江、淮间,仕州郡。蒋济为别驾,使见太祖。太祖问曰:“胡通达,长者也,宁有子孙不?”济曰:“有子曰质,规模大略不及於父,至於精良综事过之。”太祖即召质为顿丘令。县民郭政通於从妹,杀其夫程他,郡吏冯谅系狱为证。政与妹皆耐掠隐抵,谅不胜痛,自诬,当反其罪。质至官,察其情色,更详其事,检验具服。

入为丞相东曹议令史,州请为治中。将军张辽与其护军武周有隙。辽见刺史温恢求请质,质辞以疾。辽出谓质曰:“仆委意於君,何以相辜如此?”质曰:“古人之交也,取多知其不贪,奔北知其不怯,闻流言而不信,故可终也。武伯南身为雅士,往者将军称之不容於口,今以睚眦之恨,乃成嫌隙。况质才薄,岂能终好?是以不愿也。”辽感言,复与周平。

太祖辟为丞相属。黄初中,徙吏部郎,为常山太守,迁任东莞。士卢显为人所杀,质曰:“此士无雠而有少妻,所以死乎!”悉见其比居年少,书吏李若见问而色动,遂穷诘情状。若即自首,罪人斯得。每军功赏赐,皆散之於众,无入家者。在郡九年,吏民便安,将士用命。

迁荆州刺史,加振威将军,赐爵关内侯。吴大将朱然围樊城,质轻军赴之。议者皆以为贼盛不可迫,质曰:“樊城卑下,兵少,故当进军为之外援;不然,危矣。”遂勒兵临围,城中乃安。迁征东将军,假节都督青、徐诸军事。广农积谷,有兼年之储,置东征台,且佃且守。又通渠诸郡,利舟楫,严设备以待敌。海边无事。

性沉实内察,不以其节检物,所在见思。嘉平二年薨,家无馀财,惟有赐衣书箧而已。军师以闻,追进封阳陵亭侯,邑百户,谥曰贞侯。子威嗣。六年,诏书褒述质清行,赐其家钱谷。语在徐邈传。威,咸熙中官至徐州刺史,有殊绩,历三郡守,所在有名。卒於安定。

王昶字文舒,太原晋阳人也。少与同郡王凌俱知名。凌年长,昶兄事之。文帝在东宫,昶为太子文学,迁中庶子。文帝践阼,徙散骑侍郎,为洛阳典农。时都畿树木成林,昶斫开荒莱,勤劝百姓,垦田特多。迁兖州刺史。明帝即位,加扬烈将军,赐爵关内侯。昶虽在外任,心存朝廷,以为魏承秦、汉之弊,法制苛碎,不大釐改国典以准先王之风,而望治化复兴,不可得也。乃著治论,略依古制而合於时务者二十馀篇,又著兵书十馀篇,言奇正之用,青龙中奏之。

其为兄子及子作名字,皆依谦实,以见其意,故兄子默字处静,沈字处道,其子浑字玄冲,深字道冲。遂书戒之曰:

夫人为子之道,莫大於宝身全行,以显父母。此三者人知其善,而或危身破家,陷于灭亡之祸者,何也?由所祖习非其道也。夫孝敬仁义,百行之首,行之而立,身之本也。孝敬则宗族安之,仁义则乡党重之,此行成於内,名著于外者矣。人若不笃於至行,而背本逐末,以陷浮华焉,以成朋党焉;浮华则有虚伪之累,朋党则有彼此之患。此二者之戒,昭然著明,而循覆车滋众,逐末弥甚,皆由惑当时之誉,昧目前之利故也。夫富贵声名,人情所乐,而君子或得而不处,何也?恶不由其道耳。患人知进而不知退,知欲而不知足,故有困辱之累,悔吝之咎。语曰:“如不知足,则失所欲。”故知足之足常足矣。览往事之成败,察将来之吉凶,未有干名要利,欲而不厌,而能保世持家,永全福禄者也。欲使汝曹立身行己,遵儒者之教,履道家之言,故以玄默冲虚为名,欲使汝曹顾名思义,不敢违越也。古者盘杅有铭,几杖有诫,俯仰察焉,用无过行;况在己名,可不戒之哉!夫物速成则疾亡,晚就则善终。朝华之草,夕而零落;松柏之茂,隆寒不衰。是以大雅君子恶速成,戒阙党也。若范匄对秦客而武子击之,折其委笄,恶其掩人也。夫人有善鲜不自伐,有能者寡不自矜;伐则掩人,矜则陵人。掩人者人亦掩之,陵人者人亦陵之。故三郤为戮于晋,王叔负罪於周,不惟矜善自伐好争之咎乎?故君子不自称,非以让人,恶其盖人也。夫能屈以为伸,让以为得,弱以为强,鲜不遂矣。夫毁誉,爱恶之原而祸福之机也,是以圣人慎之。孔子曰:“吾之於人,谁毁谁誉;如有所誉,必有所试。”又曰:“子贡方人。赐也贤乎哉,我则不暇。”以圣人之德,犹尚如此,况庸庸之徒而轻毁誉哉?

昔伏波将军马援戒其兄子,言:“闻人之恶,当如闻父母之名;耳可得而闻,口不可得而言也。”斯戒至矣。人或毁己,当退而求之於身。若己有可毁之行,则彼言当矣;若己无可毁之行,则彼言妄矣。当则无怨于彼,妄则无害於身,又何反报焉?且闻人毁己而忿者,恶丑声之加人也,人报者滋甚,不如默而自脩己也。谚曰:“救寒莫如重裘,止谤莫如自脩。”斯言信矣。若与是非之士,凶险之人,近犹不可,况与对校乎?其害深矣。夫虚伪之人,言不根道,行不顾言,其为浮浅较可识别;而世人惑焉,犹不检之以言行也。近济阴魏讽、山阳曹伟皆以倾邪败没,荧惑当世,挟持奸慝,驱动后生。虽刑於鈇钺,大为炯戒,然所汙染,固以众矣。可不慎与!

若夫山林之士,夷、叔之伦,甘长饥於首阳,安赴火於绵山,虽可以激贪励俗,然圣人不可为,吾亦不愿也。今汝先人世有冠冕,惟仁义为名,守慎为称,孝悌於闺门,务学於师友。吾与时人从事,虽出处不同,然各有所取。颍川郭伯益,好尚通达,敏而有知。其为人弘旷不足,轻贵有馀;得其人重之如山,不得其人忽之如草。吾以所知亲之昵之,不愿儿子为之。北海徐伟长,不治名高,不求苟得,澹然自守,惟道是务。其有所是非,则讬古人以见其意,当时无所褒贬。吾敬之重之,愿儿子师之。东平刘公幹,博学有高才,诚节有大意,然性行不均,少所拘忌,得失足以相补。吾爱之重之,不愿儿子慕之。乐安任昭先,淳粹履道,内敏外恕,推逊恭让,处不避洿,怯而义勇,在朝忘身。吾友之善之,愿儿子遵之。若引而伸之,触类而长之,汝其庶几举一隅耳。及其用财先九族,其施舍务周急,其出入存故老,其论议贵无贬,其进仕尚忠节,其取人务实道,其处世戒骄淫,其贫贱慎无戚,其进退念合宜,其行事加九思,如此而已。吾复何忧哉?

青龙四年,诏“欲得有才智文章,谋虑渊深,料远若近,视昧而察,筹不虚运,策弗徒发,端一小心,清脩密静,乾乾不解,志尚在公者,无限年齿,勿拘贵贱,卿校已上各举一人”。太尉司马宣王以昶应选。正始中,转在徐州,封武观亭侯,迁征南将军,假节都督荆、豫诸军事。昶以为国有常众,战无常胜;地有常险,守无常势。今屯宛,去襄阳三百馀里,诸军散屯,船在宣池,有急不足相赴,乃表徙治新野,习水军于二州,广农垦殖,仓谷盈积。

嘉平初,太傅司马宣王既诛曹爽,乃奏博问大臣得失。昶陈治略五事:其一,欲崇道笃学,抑绝浮华,使国子入太学而脩庠序;其二,欲用考试,考试犹准绳也,未有舍准绳而意正曲直,废黜陟而空论能否也;其三,欲令居官者久於其职,有治绩则就增位赐爵;其四,欲约官实禄,励以廉耻,不使与百姓争利;其五,欲绝侈靡,务崇节俭,令衣服有章,上下有叙,储谷畜帛,反民於朴。诏书褒赞。因使撰百官考课事,昶以为唐虞虽有黜陟之文,而考课之法不垂。周制冢宰之职,大计群吏之治而诛赏,又无校比之制。由此言之,圣主明於任贤,略举黜陟之体,以委达官之长,而总其统纪,故能否可得而知也。其大指如此。

二年,昶奏:“孙权流放良臣,適庶分争,可乘衅而制吴、蜀;白帝、夷陵之间,黔、巫、秭归、房陵皆在江北,民夷与新城郡接,可袭取也。”乃遣新城太守州泰袭巫、秭归、房陵,荆州刺史王基诣夷陵,昶诣江陵,两岸引竹縆为桥,渡水击之。贼奔南岸,凿七道并来攻。於是昶使积弩同时俱发,贼大将施绩夜遁入江陵城,追斩数百级。昶欲引致平地与合战,乃先遣五军案大道发还,使贼望见以喜之,以所获铠马甲首,驰环城以怒之,设伏兵以待之。绩果追军,与战,克之。绩遁走,斩其将锺离茂、许旻,收其甲首旗鼓珍宝器仗,振旅而还。王基、州泰皆有功。於是迁昶征南大将军、仪同三司,进封京陵侯。毌丘俭、文钦作乱,引兵拒俭、钦有功,封二子亭侯、关内侯,进位骠骑将军。诸葛诞反,昶据夹石以逼江陵,持施绩、全熙使不得东。诞既诛,诏曰:“昔孙膑佐赵,直凑大梁。西兵骤进,亦所以成东征之势也。”增邑千户,并前四千七百户,迁司空,持节、都督如故。甘露四年薨,谥曰穆侯。子浑嗣,咸熙中为越骑校尉。

王基字伯舆,东莱曲城人也。少孤,与叔父翁居。翁抚养甚笃,基亦以孝称。年十七,郡召为吏,非其好也,遂去,入琅邪界游学。黄初中,察孝廉,除郎中。是时青土初定,刺史王凌特表请基为别驾,后召为秘书郎,凌复请还。顷之,司徒王朗辟基,凌不遣。朗书劾州曰:“凡家臣之良,则升于公辅,公臣之良,则入于王职,是故古者侯伯有贡士之礼。今州取宿卫之臣,留秘阁之吏,所希闻也。”凌犹不遣。凌流称青土,盖亦由基协和之辅也。大将军司马宣王辟基,未至,擢为中书侍郎。

明帝盛脩宫室,百姓劳瘁。基上疏曰:“臣闻古人以水喻民,曰‘水所以载舟,亦所以覆舟’。故在民上者,不可以不戒惧。夫民逸则虑易,苦则思难,是以先王居之以约俭,俾不至於生患。昔颜渊云东野子之御,马力尽矣而求进不已,是以知其将败。今事役劳苦,男女离旷,愿陛下深察东野之弊,留意舟水之喻,息奔驷於未尽,节力役於未困。昔汉有天下,至孝文时唯有同姓诸侯,而贾谊忧之曰:‘置火积薪之下而寝其上,因谓之安也。’今寇贼未殄,猛将拥兵,检之则无以应敌,久之则难以遗后,当盛明之世,不务以除患,若子孙不竞,社稷之忧也。使贾谊复起,必深切于曩时矣。”

散骑常侍王肃著诸经传解及论定朝仪,改易郑玄旧说,而基据持玄义,常与抗衡。迁安平太守,公事去官。大将军曹爽请为从事中郎,出为安丰太守。郡接吴寇,为政清严有威惠,明设防备,敌不敢犯。加讨寇将军。吴尝大发众集建业,扬声欲入攻扬州,刺史诸葛诞使基策之。基曰:“昔孙权再至合肥,一至江夏,其后全琮出庐江,朱然寇襄阳,皆无功而还。今陆逊等已死,而权年老,内无贤嗣,中无谋主。权自出则惧内衅卒起,痈疽发溃;遣将则旧将已尽,新将未信。此不过欲补定支党,还自保护耳。”后权竟不能出。时曹爽专柄,风化陵迟,基著时要论以切世事。以疾徵还,起家为河南尹,未拜,爽伏诛,基尝为爽官属,随例罢。

其年为尚书,出为荆州刺史,加扬烈将军,随征南王昶击吴。基别袭步协於夷陵,协闭门自守。基示以攻形,而实分兵取雄父邸阁,收米三十馀万斛,虏安北将军谭正,纳降数千口。於是移其降民,置夷陵县。赐爵关内侯。基又表城上昶,徙江夏治之,以偪夏口,由是贼不敢轻越江。明制度,整军农,兼脩学校,南方称之。时朝廷议欲伐吴,诏基量进趣之宜。基对曰:“夫兵动而无功,则威名折於外,财用穷於内,故必全而后用也。若不资通川聚粮水战之备,则虽积兵江内,无必渡之势矣。今江陵有沮、漳二水,溉灌膏腴之田以千数。安陆左右,陂池沃衍。若水陆并农,以实军资,然后引兵诣江陵、夷陵,分据夏口,顺沮、漳,资水浮谷而下。贼知官兵有经久之势,则拒天诛者意沮,而向王化者益固。然后率合蛮夷以攻其内,精卒劲兵以讨其外,则夏口以上必拔,而江外之郡不守。如此,吴、蜀之交绝,交绝而吴禽矣。不然,兵出之利,未可必矣。”於是遂止。

司马景王新统政,基书戒之曰:“天下至广,万机至猥,诚不可不矜矜业业,坐而待旦也。夫志正则众邪不生,心静则众事不躁,思虑审定则教令不烦,亲用忠良则远近协服。故知和远在身,定众在心。许允、傅嘏、袁侃、崔赞皆一时正士,有直质而无流心,可与同政事者也。”景王纳其言。

高贵乡公即尊位,进封常乐亭侯。毌丘俭、文钦作乱,以基为行监军、假节,统许昌军,適与景王会於许昌。景王曰:“君筹俭等何如?”基曰:“淮南之逆,非吏民思乱也,俭等诳胁迫惧,畏目下之戮,是以尚群聚耳。若大兵临偪,必土崩瓦解,俭、钦之首,不终朝而县於军门矣。”景王曰:“善。”乃令基居军前。议者咸以俭、钦慓悍,难与争锋。诏基停驻。基以为:“俭等举军足以深入,而久不进者,是其诈伪已露,众心疑沮也。今不张示威形以副民望,而停军高垒,有似畏懦,非用兵之势也。若或虏略民人,又州郡兵家为贼所得者,更怀离心;俭等所迫胁者,自顾罪重,不敢复还,此为错兵无用之地,而成奸宄之源。吴寇因之,则淮南非国家之有,谯、沛、汝、豫危而不安,此计之大失也。军宜速进据南顿,南顿有大邸阁,计足军人四十日粮。保坚城,因积谷,先人有夺人之心,此平贼之要也。”基屡请,乃听进据〈氵隱〉水。既至,复言曰:“兵闻拙速,未睹工迟之久。方今外有强寇,内有叛臣,若不时决,则事之深浅未可测也。议者多欲将军持重。将军持重是也,停军不进非也。持重非不行之谓也,进而不可犯耳。今据坚城,保壁垒,以积实资虏,县运军粮,甚非计也。”景王欲须诸军集到,犹尚未许。基曰:“将在军,君令有所不受。彼得则利,我得亦利,是谓争城,南顿是也。”遂辄进据南顿,俭等从项亦争欲往,发十馀里,闻基先到,复还保项。时兖州刺史邓艾屯乐嘉,俭使文钦将兵袭艾。基知其势分,进兵偪项,俭众遂败。钦等已平,迁镇南将军,都督豫州诸军事,领豫州刺史,进封安乐乡侯。上疏求分户二百,赐叔父子乔爵关内侯,以报叔父拊育之德。有诏特听。

诸葛诞反,基以本官行镇东将军,都督扬、豫诸军事。时大军在项,以贼兵精,诏基敛军坚垒。基累启求进讨。会吴遣朱异来救诞,军於安城。基又被诏引诸军转据北山,基谓诸将曰:“今围垒转固,兵马向集,但当精脩守备以待越逸,而更移兵守险,使得放纵,虽有智者不能善后矣。”遂守便宜上疏曰:“今与贼家对敌,当不动如山。若迁移依险,人心摇荡,於势大损。诸军并据深沟高垒,众心皆定,不可倾动,此御兵之要也。”书奏,报听。大将军司马文王进屯丘头,分部围守,各有所统。基督城东城南二十六军,文王敕军吏入镇南部界,一不得有所遣。城中食尽,昼夜攻垒,基辄拒击,破之。寿春既拔,文王与基书曰:“初议者云云,求移者甚众,时未临履,亦谓宜然。将军深算利害,独秉固志,上违诏命,下拒众议,终至制敌禽贼,虽古人所述,不是过也。”文王欲遣诸将轻兵深入,招迎唐咨等子弟,因衅有荡覆吴之势。基谏曰:“昔诸葛恪乘东关之胜,竭江表之兵,以围新城,城既不拔,而众死者太半。姜维因洮上之利,轻兵深入,粮饷不继,军覆上邽。夫大捷之后,上下轻敌,轻敌则虑难不深。今贼新败於外,又内患未弭,是其脩备设虑之时也。且兵出逾年,人有归志,今俘馘十万,罪人斯得,自历代征伐,未有全兵独克如今之盛者也。武皇帝克袁绍於官渡,自以所获已多,不复追奔,惧挫威也。”文王乃止。以淮南初定,转基为征东将军,都督扬州诸军事,进封东武侯。基上疏固让,归功参佐,由是长史司马等七人皆侯。

是岁,基母卒,诏秘其凶问,迎基父豹丧合葬洛阳,追赠豹北海太守。甘露四年,转为征南将军,都督荆州诸军事。常道乡公即尊位,增邑千户,并前五千七百户。前后封子二人亭侯、关内侯。

景元二年,襄阳太守表吴贼邓由等欲来归化,基被诏,当因此震荡江表。基疑其诈,驰驿陈状。且曰:“嘉平以来,累有内难,当今之务,在于镇安社稷,绥宁百姓,未宜动众以求外利。”文王报书曰:“凡处事者,多曲相从顺,鲜能确然共尽理实。诚感忠爱,每见规示,辄敬依来指。”后由等竟不降。

是岁基薨,追赠司空,谥曰景侯。子徽嗣,早卒。咸熙中,开建五等,以基著勋前朝,改封基孙廙,而以东武馀邑赐一子爵关内侯。晋室践阼,下诏曰:“故司空王基既著德立勋,又治身清素,不营产业,久在重任,家无私积,可谓身没行显,足用励俗者也。其以奴婢二人赐其家。”

评曰:徐邈清尚弘通,胡质素业贞粹,王昶开济识度,王基学行坚白,皆掌统方任,垂称著绩。可谓国之良臣,时之彦士矣。

\part{魏书二十八}
\chapter{王毌丘诸葛邓锺传第二十八}

王凌字彦云,太原祁人也。叔父允,为汉司徒,诛董卓。卓将李傕、郭汜等为卓报仇,入长安,杀允,尽害其家。凌及兄晨,时年皆少,逾城得脱,亡命归乡里。凌举孝廉,为发干长,稍迁至中山太守,所在有治,太祖辟为丞相掾属。

文帝践阼,拜散骑常侍,出为兖州刺史,与张辽等至广陵讨孙权。临江,夜大风,吴将吕范等船漂至北岸。凌与诸将逆击,捕斩首虏,获舟船,有功,封宜城亭侯,加建武将军,转在青州。是时海滨乘丧乱之后,法度未整。凌布政施教,赏善罚恶,甚有纲纪,百姓称之,不容於口。后从曹休征吴,与贼遇於夹石,休军失利,凌力战决围,休得免难。仍徙为扬、豫州刺史,咸得军民之欢心。始至豫州,旌先贤之后,求未显之士,各有条教,意义甚美。初,凌与司马朗、贾逵友善,及临兖、豫,继其名迹。正始初,为征东将军,假节都督扬州诸军事。二年,吴大将全琮数万众寇芍陂,凌率诸军逆讨,与贼争塘,力战连日,贼退走。进封南乡侯,邑千三百五十户,迁车骑将军、仪同三司。

是时,凌外甥令狐愚以才能为兖州刺史,屯平阿。舅甥并典兵,专淮南之重。凌就迁为司空。司马宣王既诛曹爽,进凌为太尉,假节钺。凌、愚密协计,谓齐王不任天位,楚王彪长而才,欲迎立彪都许昌。嘉平元年九月,愚遣将张式至白马,与彪相问往来。凌又遣舍人劳精诣洛阳,语子广。广言:“废立大事,勿为祸先。”其十一月,愚复遣式诣彪,未还,会愚病死。二年,荧惑守南斗,凌谓:“斗中有星,当有暴贵者。”三年春,吴贼塞涂水。凌欲因此发,大严诸军,表求讨贼;诏报不听。凌阴谋滋甚,遣将军杨弘以废立事告兖州刺史黄华,华、弘连名以白太傅司马宣王。宣王将中军乘水道讨凌,先下赦赦凌罪,又将尚书广东,使为书喻凌,大军掩至百尺逼凌。凌自知势穷,乃乘船单出迎宣王,遣掾王彧谢罪,送印绶、节钺。军到丘头,凌面缚水次。宣王承诏遣主簿解缚反服,见凌,慰劳之,还印绶、节钺,遣步骑六百人送还京都。凌至项,饮药死。宣王遂至寿春。张式等皆自首,乃穷治其事。彪赐死,诸相连者悉夷三族。朝议咸以为春秋之义,齐崔杼、郑归生皆加追戮,陈尸斫棺,载在方策。凌、愚罪宜如旧典。乃发凌、愚冢,剖棺,暴尸於所近市三日,烧其印绶、朝服,亲土埋之。进弘、华爵为乡侯。广有志尚学行,死时年四十馀。

毌丘俭字仲恭,河东闻喜人也。父兴,黄初中为武威太守,伐叛柔服,开通河右,名次金城太守苏则。讨贼张进及讨叛胡有功,封高阳乡侯。入为将作大匠。俭袭父爵,为平原侯文学。明帝即位,为尚书郎,迁羽林监。以东宫之旧,甚见亲待。出为洛阳典农。时取农民以治宫室,俭上疏曰:“臣愚以为天下所急除者二贼,所急务者衣食。诚使二贼不灭,士民饥冻,虽崇美宫室,犹无益也。”迁荆州刺史。

青龙中,帝图讨辽东,以俭有幹策,徙为幽州刺史,加度辽将军,使持节,护乌丸校尉。率幽州诸军至襄平,屯辽隧。右北平乌丸单于寇娄敦、辽西乌丸都督率众王护留等,昔随袁尚奔辽东者,率众五千馀人降。寇娄敦遣弟阿罗槃等诣阙朝贡,封其渠率二十馀人为侯、王,赐舆马缯彩各有差。公孙渊逆与俭战,不利,引还。明年,帝遣太尉司马宣王统中军及俭等众数万讨渊,定辽东。俭以功进封安邑侯,食邑三千九百户。

正始中,俭以高句骊数侵叛,督诸军步骑万人出玄菟,从诸道讨之。句骊王宫将步骑二万人,进军沸流水上,大战梁口,宫连破走。俭遂束马县车,以登丸都,屠句骊所都,斩获首虏以千数。句骊沛者名得来,数谏宫,宫不从其言。得来叹曰:“立见此地将生蓬蒿。”遂不食而死,举国贤之。俭令诸军不坏其墓,不伐其树,得其妻子,皆放遣之。宫单将妻子逃窜。俭引军还。六年,复征之,宫遂奔买沟。俭遣玄菟太守王颀追之,过沃沮千有馀里,至肃慎氏南界,刻石纪功,刊丸都之山,铭不耐之城。诸所诛纳八千馀口,论功受赏,侯者百馀人。穿山溉灌,民赖其利。

迁左将军,假节监豫州诸军事,领豫州刺史,转为镇南将军。诸葛诞战于东关,不利,乃令诞、俭对换。诞为镇南,都督豫州。俭为镇东,都督杨州。吴太傅诸葛恪围合肥新城,俭与文钦御之,太尉司马孚督中军东解围,恪退还。

初,俭与夏侯玄、李丰等厚善。扬州刺史前将军文钦,曹爽之邑人也,骁果粗猛,数有战功,好增虏获,以徼宠赏,多不见许,怨恨日甚。俭以计厚待钦,情好欢洽。钦亦感戴,投心无贰。正元二年正月,有彗星数十丈,西北竟天,起于吴、楚之分。俭、钦喜,以为己祥。遂矫太后诏,罪状大将军司马景王,移诸郡国,举兵反。迫胁淮南将守诸别屯者,及吏民大小,皆入寿春城,为坛於城西,歃血称兵为盟,分老弱守城,俭、钦自将五六万众渡淮,西至项。俭坚守,钦在外为游兵。

大将军统中外军讨之,别使诸葛诞督豫州诸军从安风津拟寿春,征东将军胡遵督青、徐诸军出于谯、宋之间,绝其归路。大将军屯汝阳,使监军王基督前锋诸军据南顿以待之。今诸军皆坚壁勿与战。俭、钦进不得斗,退恐寿春见袭,不得归,计穷不知所为。淮南将士,家皆在北,众心沮散,降者相属,惟淮南新附农民为之用。大将军遣兖州刺史邓艾督泰山诸军万馀人至乐嘉,示弱以诱之,大将军寻自洙至。钦不知,果夜来欲袭艾等,会明,见大军兵马盛,乃引还。大将军纵骁骑追击,大破之,钦遁走。是日,俭闻钦战败,恐惧夜走,众溃。比至慎县,左右人兵稍弃俭去,俭独与小弟秀及孙重藏水边草中。安风津都尉部民张属就射杀俭,传首京都。属封侯。秀、重走入吴。将士诸为俭、钦所迫胁者,悉归降。

俭子甸为治书侍御史,先时知俭谋将发,私出将家属逃走新安灵山上。别攻下之,夷俭三族。

钦亡入吴,吴以钦为都护、假节、镇北大将军、幽州牧、谯侯。

诸葛诞字公休,琅邪阳都人,诸葛丰后也。初以尚书郎为荥阳令,入为吏部郎。人有所属讬,辄显其言而承用之,后有当否,则公议其得失以为褒贬,自是群僚莫不慎其所举。累迁御史中丞尚书,与夏侯玄、邓飏等相善,收名朝廷,京都翕然。言事者以诞、飏等脩浮华,合虚誉,渐不可长。明帝恶之,免诞官。会帝崩,正始初,玄等并在职。复以诞为御史中丞尚书,出为扬州刺史,加昭武将军。

王凌之阴谋也,太傅司马宣王潜军东伐,以诞为镇东将军、假节都督扬州诸军事,封山阳亭侯。诸葛恪兴东关,遣诞督诸军讨之,与战,不利。还,徙为镇南将军。

后毌丘俭、文钦反,遣使诣诞,招呼豫州士民。诞斩其使,露布天下,令知俭、钦凶逆。大将军司马景王东征,使诞督豫州诸军,渡安风津向寿春。俭、钦之破也,诞先至寿春。寿春中十馀万口,闻俭、钦败,恐诛,悉破城门出,流迸山泽,或散走入吴。以诞久在淮南,乃复以为镇东大将军、仪同三司、都督扬州。吴大将孙峻、吕据、留赞等闻淮南乱,会文钦往,乃帅众将钦径至寿春;时诞诸军已至,城不可攻,乃走。诞遣将军蒋班追击之,斩赞,传首,收其印节。进封高平侯,邑三千五百户,转为征东大将军。

诞既与玄、飏等至亲,又王凌、毌丘俭累见夷灭,惧不自安,倾帑藏振施以结众心,厚养亲附及扬州轻侠者数千人为死士。甘露元年冬,吴贼欲向徐堨,计诞所督兵马足以待之,而复请十万众守寿春,又求临淮筑城以备寇,内欲保有淮南。朝廷微知诞有自疑心,以诞旧臣,欲入度之。二年五月,徵为司空。诞被诏书,愈恐,遂反。召会诸将,自出攻扬州刺史乐綝,杀之。敛淮南及淮北郡县屯田口十馀万官兵,扬州新附胜兵者四五万人,聚谷足一年食,闭城自守。遣长史吴纲将小子靓至吴请救。吴人大喜,遣将全怿、全端、唐咨、王祚等,率三万众,密与文钦俱来应诞。以诞为左都护、假节、大司徒、骠骑将军、青州牧、寿春侯。是时镇南将军王基始至,督诸军围寿春,未合。咨、钦等从城东北,因山乘险,得将其众突入城。

六月,车驾东征,至项。大将军司马文王督中外诸军二十六万众,临淮讨之。大将军屯丘头。使基及安东将军陈骞等四面合围,表里再重,堑垒甚峻。又使监军石苞、兖州刺史州泰等,简锐卒为游军,备外寇。钦等数出犯围,逆击走之。吴将朱异再以大众来迎诞等,渡黎浆水,泰等逆与战,每摧其锋。孙綝以异战不进,怒而杀之。城中食转少,外救不至,众无所恃。将军蒋班、焦彝,皆诞爪牙计事者也,弃诞,逾城自归大将军。大将军乃使反间,以奇变说全怿等,怿等率众数千人开门来出。城中震惧,不知所为。

三年正月,诞、钦、咨等大为攻具,昼夜五六日攻南围,欲决围而出。围上诸军,临高以发石车火箭逆烧破其攻具,弩矢及石雨下,死伤者蔽地,血流盈堑。复还入城,城内食转竭,降出者数万口。钦欲尽出北方人,省食,与吴人坚守,诞不听,由是争恨。钦素与诞有隙,徒以计合,事急愈相疑。钦见诞计事,诞遂杀钦。钦子鸯及虎将兵在小城中,闻钦死,勒兵驰赴之,众不为用。鸯、虎单走,逾城出,自归大将军。军吏请诛之,大将军令曰:“钦之罪不容诛,其子固应当戮,然鸯、虎以穷归命,且城未拔,杀之是坚其心也。”乃赦鸯、虎,使将兵数百骑驰巡城,呼语城内云:“文钦之子犹不见杀,其馀何惧?”表鸯、虎为将军,各赐爵关内侯。城内喜且扰,又日饥困,诞、咨等智力穷。大将军乃自临围,四面进兵,同时鼓噪登城,城内无敢动者。诞窘急,单乘马,将其麾下突小城门出。大将军司马胡奋部兵逆击,斩诞,传首,夷三族。诞麾下数百人,坐不降见斩,皆曰:“为诸葛公死,不恨。”其得人心如此。唐咨、王祚及诸裨将皆面缚降,吴兵万众,器仗军实山积。

初围寿春,议者多欲急攻之,大将军以为:“城固而众多,攻之必力屈,若有外寇,表里受敌,此危道也。今三叛相聚於孤城之中,天其或者将使同就戮,吾当以全策縻之,可坐而制也。”诞以二年五月反,三年二月破灭。六军按甲,深沟高垒,而诞自困,竟不烦攻而克。及破寿春,议者又以为淮南仍为叛逆,吴兵室家在江南,不可纵,宜悉坑之。大将军以为古之用兵,全国为上,戮其元恶而已。吴兵就得亡还,適可以示中国之弘耳。一无所杀,分布三河近郡以安处之。

唐咨本利城人。黄初中,利城郡反,杀太守徐箕,推咨为主。文帝遣诸军讨破之,咨走入海,遂亡至吴,官至左将军,封侯、持节。诞、钦屠戮,咨亦生禽,三叛皆获,天下快焉。拜咨安远将军,其馀裨将咸假号位,吴众悦服。江东感之,皆不诛其家。其淮南将吏士民诸为诞所胁略者,惟诛其首逆,馀皆赦之。听鸯、虎收敛钦丧,给其车牛,致葬旧墓。

邓艾字士载,义阳棘阳人也。少孤,太祖破荆州,徙汝南,为农民养犊。年十二,随母至颍川,读故太丘长陈寔碑文,言“文为世范,行为士则”,艾遂自名范,字士则。后宗族有与同者,故改焉。为都尉学士,以口吃,不得作幹佐。为稻田守丛草吏。同郡吏父怜其家贫,资给甚厚,艾初不称谢。每见高山大泽,辄规度指画军营处所,时人多笑焉。后为典农纲纪,上计吏,因使见太尉司马宣王。宣王奇之,辟之为掾,迁尚书郎。

时欲广田畜谷,为灭贼资,使艾行陈、项已东至寿春。艾以为“田良水少,不足以尽地利,宜开河渠,可以引水浇溉,大积军粮,又通运漕之道。”乃著济河论以喻其指。又以为“昔破黄巾,因为屯田,积谷于许都以制四方。今三隅已定,事在淮南,每大军征举,运兵过半,功费巨亿,以为大役。陈、蔡之间,土下田良,可省许昌左右诸稻田,并水东下。令淮北屯二万人,淮南三万人,十二分休,常有四万人,且田且守。水丰常收三倍於西,计除众费,岁完五百万斛以为军资。六七年间,可积三千万斛於淮上,此则十万之众五年食也。以此乘吴,无往而不克矣。”宣王善之,事皆施行。正始二年,乃开广漕渠,每东南有事,大军兴众,汎舟而下,达于江、淮,资食有储而无水害,艾所建也。

出参征西军事,迁南安太守。嘉平元年,与征西将军郭淮拒蜀偏将军姜维。维退,淮因西击羌。艾曰:“贼去未远,或能复还,宜分诸军以备不虞。”於是留艾屯白水北。三日,维遣廖化自白水南向艾结营。艾谓诸将曰:“维今卒还,吾军人少,法当来渡而不作桥。此维使化持吾,令不得还。维必自东袭取洮城。”洮城在水北,去艾屯六十里。艾即夜潜军径到,维果来渡,而艾先至据城,得以不败。赐爵关内侯,加讨寇将军,后迁城阳太守。

是时并州右贤王刘豹并为一部,艾上言曰:“戎狄兽心,不以义亲,强则侵暴,弱则内附,故周宣有玁狁之寇,汉祖有平城之围。每匈奴一盛,为前代重患。自单于在外,莫能牵制长卑。诱而致之,使来入侍。由是羌夷失统,合散无主。以单于在内,万里顺轨。今单于之尊日疏,外土之威浸重,则胡虏不可不深备也。闻刘豹部有叛胡,可因叛割为二国,以分其势。去卑功显前朝,而子不继业,宜加其子显号,使居雁门。离国弱寇,追录旧勋,此御边长计也。”又陈:“羌胡与民同处者,宜以渐出之,使居民表崇廉耻之教,塞奸宄之路。”大将军司马景王新辅政,多纳用焉。迁汝南太守,至则寻求昔所厚己吏父,久已死,遣吏祭之,重遗其母,举其子与计吏。艾所在,荒野开辟,军民并丰。

诸葛恪围合肥新城,不克,退归。艾言景王曰:“孙权已没,大臣未附,吴名宗大族,皆有部曲,阻兵仗势,足以建命。恪新秉国政,而内无其主,不念抚恤上下以立根基,竞於外事,虐用其民,悉国之众,顿於坚城,死者万数,载祸而归,此恪获罪之日也。昔子胥、吴起、商鞅、乐毅皆见任时君,主没而败。况恪才非四贤,而不虑大患,其亡可待也。”恪归,果见诛。迁兖州刺史,加振威将军。上言曰:“国之所急,惟农与战,国富则兵强,兵强则战胜。然农者,胜之本也。孔子曰'足食足兵',食在兵前也。上无设爵之劝,则下无财畜之功。今使考绩之赏,在於积粟富民,则交游之路绝,浮华之原塞矣。”

高贵乡公即尊位,进封方城亭侯。毌丘俭作乱,遣健步赍书,欲疑惑大众,艾斩之,兼道进军,先趣乐嘉城,作浮桥。司马景王至,遂据之。文钦以后大军破败於城下,艾追之至丘头。钦奔吴。吴大将军孙峻等号十万众,将渡江,镇东将军诸葛诞遣艾据肥阳,艾以与贼势相远,非要害之地,辄移屯附亭,遣泰山太守诸葛绪等于黎浆拒战,遂走之。其年徵拜长水校尉。以破钦等功,进封方城乡侯,行安西将军。解雍州刺史王经围於狄道,姜维退驻锺提,乃以艾为安西将军,假节、领护东羌校尉。议者多以为维力已竭,未能更出。艾曰:“洮西之败,非小失也;破军杀将,仓廪空虚,百姓流离,几於危亡。今以策言之,彼有乘胜之势,我有虚弱之实,一也。彼上下相习,五兵犀利,我将易兵新,器杖未复,二也。彼以船行,吾以陆军,劳逸不同,三也。狄道、陇西、南安、祁山,各当有守,彼专为一,我分为四,四也。从南安、陇西,因食羌谷,若趣祁山,熟麦千顷,为之县饵,五也。贼有黠数,其来必矣。”顷之,维果向祁山,闻艾已有备,乃回从董亭趣南安,艾据武城山以相持。维与艾争险,不克,其夜,渡渭东行,缘山趣上邽,艾与战於段谷,大破之。甘露元年诏曰:“逆贼姜维连年狡黠,民夷骚动,西土不宁。艾筹画有方,忠勇奋发,斩将十数,馘首千计;国威震於巴、蜀,武声扬於江、岷。今以艾为镇西将军、都督陇右诸军事,进封邓侯。分五百户封子忠为亭侯。“二年,拒姜维于长城,维退还。迁征西将军,前后增邑凡六千六百户。景元三年,又破维于侯和,维卻保沓中。四年秋,诏诸军征蜀,大将军司马文王皆指授节度,使艾与维相缀连;雍州刺史诸葛绪要维,令不得归。艾遣天水太守王颀等直攻维营,陇西太守牵弘等邀其前,金城太守杨欣等诣甘松。维闻锺会诸军已入汉中,引退还。欣等追蹑於强川口,大战,维败走。闻雍州已塞道,屯桥头,从孔函谷入北道,欲出雍州后。诸葛绪闻之,卻还三十里。维入北道三十馀里,闻绪军卻,寻还,从桥头过,绪趣截维,较一日不及。维遂东引,还守剑阁。锺会攻维未能克。艾上言:“今贼摧折,宜遂乘之,从阴平由邪径经汉德阳亭趣涪,出剑阁西百里,去成都三百馀里,奇兵冲其腹心。剑阁之守必还赴涪,则会方轨而进;剑阁之军不还,则应涪之兵寡矣。军志有之曰:'攻其无备,出其不意。'今掩其空虚,破之必矣。”

冬十月,艾自阴平道行无人之地七百馀里,凿山通道,造作桥阁。山高谷深,至为艰险,又粮运将匮,频於危殆。艾以毡自裹,推转而下。将士皆攀木缘崖,鱼贯而进。先登至江由,蜀守将马邈降。蜀卫将军诸葛瞻自涪还绵竹,列陈待艾。艾遣子惠唐亭侯忠等出其右,司马师纂等出其左。忠、纂战不利,并退还,曰:“贼未可击。”艾怒曰:“存亡之分,在此一举,何不可之有?”乃叱忠、纂等,将斩之。忠、纂驰还更战,大破之,斩瞻及尚书张遵等首,进军到雒。刘禅遣使奉皇帝玺绶,为笺诣艾请降。

艾至成都,禅率太子诸王及群臣六十馀人面缚舆榇诣军门,艾执节解缚焚榇,受而宥之。检御将士,无所虏略,绥纳降附,使复旧业,蜀人称焉。辄依邓禹故事,承制拜禅行骠骑将军,太子奉车、诸王驸马都尉。蜀群司各随高下拜为王官,或领艾官属。以师纂领益州刺史,陇西太守牵弘等领蜀中诸郡。使於绵竹筑台以为京观,用彰战功。士卒死事者,皆与蜀兵同共埋藏。艾深自矜伐,谓蜀士大夫曰:“诸君赖遭某,故得有今日耳。若遇吴汉之徒,已殄灭矣。”又曰:“姜维自一时雄儿也,与某相值,故穷耳。”有识者笑之。

十二月,诏曰:“艾曜威奋武,深入虏庭,斩将搴旗,枭其鲸鲵,使僣号之主,稽首系颈,历世逋诛,一朝而平。兵不逾时,战不终日,云彻席卷,荡定巴蜀。虽白起破强楚,韩信克劲赵,吴汉禽子阳,亚夫灭七国,计功论美,不足比勋也。其以艾为太尉,增邑二万户,封子二人亭侯,各食邑千户。”艾言司马文王曰:“兵有先声而后实者,今因平蜀之势以乘吴,吴人震恐,席卷之时也。然大举之后,将士疲劳,不可便用,且徐缓之;留陇右兵二万人,蜀兵二万人,煮盐兴冶,为军农要用,并作舟船,豫顺流之事,然后发使告以利害,吴必归化,可不征而定也。今宜厚刘禅以致孙休,安士民以来远人,若便送禅於京都,吴以为流徙,则於向化之心不劝。宜权停留,须来年秋冬,比尔吴亦足平。以为可封禅为扶风王,锡其资财,供其左右。郡有董卓坞,为之宫舍。爵其子为公侯,食郡内县,以显归命之宠。开广陵、城阳以待吴人,则畏威怀德,望风而从矣。”文王使监军卫瓘喻艾:“事当须报,不宜辄行。”艾重言曰:“衔命征行,奉指授之策,元恶既服;至于承制拜假,以安初附,谓合权宜。今蜀举众归命,地尽南海,东接吴会,宜早镇定。若待国命,往复道途,延引日月。春秋之义,大夫出疆,有可以安社稷,利国家,专之可也。今吴未宾;势与蜀连,不可拘常以失事机。兵法,进不求名,退不避罪,艾虽无古人之节,终不自嫌以损于国也。“锺会、胡烈、师纂等皆白艾所作悖逆,变衅以结。诏书槛车徵艾。

艾父子既囚,锺会至成都,先送艾,然后作乱。会已死,艾本营将士追出艾槛车,迎还。瓘遣田续等讨艾,遇於绵竹西,斩之。子忠与艾俱死,馀子在洛阳者悉诛,徙艾妻子及孙於西域。

初,艾当伐蜀,梦坐山上而有流水,以问殄虏护军爰邵。邵曰:“按易卦,山上有水曰蹇。蹇繇曰:'蹇利西南,不利东北。'孔子曰:'蹇利西南,往有功也;不利东北,其道穷也。'往必克蜀,殆不还乎!”艾怃然不乐。

泰始元年,晋室践阼,诏曰:“昔太尉王凌谋废齐王,而王竟不足以守位。征西将军邓艾,矜功失节,实应大辟。然被书之日,罢遣人众,束手受罪,比于求生遂为恶者,诚复不同。今大赦得还,若无子孙者听使立后,令祭祀不绝。“三年,议郎段灼上疏理艾曰:“艾心怀至忠而荷反逆之名,平定巴蜀而受夷灭之诛,臣窃悼之。惜哉,言艾之反也!艾性刚急,轻犯雅俗,不能协同朋类,故莫肯理之。臣敢言艾不反之状。昔姜维有断陇右之志,艾脩治备守,积谷强兵。值岁凶旱,艾为区种,身被乌衣,手执耒耜,以率将士。上下相感,莫不尽力。艾持节守边,所统万数,而不难仆虏之劳,士民之役,非执节忠勤,孰能若此?故落门、段谷之战,以少击多,摧破强贼。先帝知其可任,委艾庙胜,授以长策。艾受命忘身,束马县车,自投死地,勇气陵云,士众乘势,使刘禅君臣面缚,叉手屈膝。艾功名以成,当书之竹帛,传祚万世。七十老公,反欲何求!艾诚恃养育之恩,心不自疑,矫命承制,权安社稷;虽违常科,有合古义,原心定罪,本在可论。锺会忌艾威名,构成其事。忠而受诛,信而见疑,头县马巿,诸子并斩,见之者垂泣,闻之者叹息。陛下龙兴,阐弘大度,释诸嫌忌,受诛之家,不拘叙用。昔秦民怜白起之无罪,吴人伤子胥之冤酷,皆为立祠。今天下民人为艾悼心痛恨,亦犹是也。臣以为艾身首分离,捐弃草土,宜收尸丧,还其田宅。以平蜀之功,绍封其孙,使阖棺定谥,死无馀恨。赦冤魂于黄泉,收信义于后世,葬一人而天下慕其行,埋一魂而天下归其义,所为者寡而悦者众矣。”九年,诏曰:“艾有功勋,受罪不逃刑,而子孙为民隶,朕常愍之。其以嫡孙朗为郎中。”

艾在西时,修治障塞,筑起城坞。泰始中,羌虏大叛,频杀刺史,凉州道断。吏民安全者,皆保艾所筑坞焉。

艾州里时辈南阳州泰,亦好立功业,善用兵,官至征虏将军、假节都督江南诸军事。景元二年薨,追赠卫将军,谥曰壮侯。

锺会字士季,颍川长社人,太傅繇小子也。少敏惠夙成。中护军蒋济著论,谓“观其眸子,足以知人。”会年五岁,繇遣见济,济甚异之,曰:“非常人也。”及壮,有才数技艺,而博学精练名理,以夜续昼,由是获声誉。正始中,以为秘书郎,迁尚书中书侍郎。高贵乡公即尊位,赐爵关内侯。

毌丘俭作乱,大将军司马景王东征,会从,典知密事,卫将军司马文王为大军后继。景王薨於许昌,文王总统六军,会谋谟帷幄。时中诏敕尚书傅嘏,以东南新定,权留卫将军屯许昌为内外之援,令嘏率诸军还。会与嘏谋,使嘏表上,辄与卫将军俱发,还到雒水南屯住。於是朝廷拜文王为大将军、辅政,会迁黄门侍郎,封东武亭侯,邑三百户。

甘露二年,徵诸葛诞为司空,时会丧宁在家,策诞必不从命,驰白文王。文王以事已施行,不复追改。及诞反,车驾住项,文王至寿春,会复从行。

初,吴大将全琮,孙权之婚亲重臣也,琮子怿、孙静、从子端、翩、缉等,皆将兵来救诞。怿兄子辉、仪留建业,与其家内争讼,携其母,将部曲数十家渡江,自归文王。会建策,密为辉、仪作书,使辉、仪所亲信赍入城告怿等,说吴中怒怿等不能拔寿春,欲尽诛诸将家,故逃来归命。怿等恐惧,遂将所领开东城门出降,皆蒙封宠,城中由是乖离。寿春之破,会谋居多,亲待日隆,时人谓之子房。军还,迁为太仆,固辞不就。以中郎在大将军府管记室事,为腹心之任。以讨诸葛诞功,进爵陈侯,屡让不受。诏曰:“会典综军事,参同计策,料敌制胜,有谋谟之勋,而推宠固让,辞指款实,前后累重,志不可夺。夫成功不处,古人所重,其听会所执,以成其美。”迁司隶校尉。虽在外司,时政损益,当世与夺,无不综典。嵇康等见诛,皆会谋也。

文王以蜀大将姜维屡扰边陲,料蜀国小民疲,资力单竭,欲大举图蜀。惟会亦以为蜀可取,豫共筹度地形,考论事势。景元三年冬,以会为镇西将军、假节都督关中诸军事。文王敕青、徐、兖、豫、荆、扬诸州,并使作船,又令唐咨作浮海大船,外为将伐吴者。四年秋,乃下诏使邓艾、诸葛绪各统诸军三万馀人,艾趣甘松、沓中连缀维,绪趣武街、桥头绝维归路。会统十馀万众,分从斜谷、骆谷入。先命牙门将许仪在前治道,会在后行,而桥穿,马足陷,於是斩仪。仪者,许褚之子,有功王室,犹不原贷。诸军闻之,莫不震竦。蜀令诸围皆不得战,退还汉、乐二城守。魏兴太守刘钦趣子午谷,诸军数道平行,至汉中。蜀监军王含守乐城,护军蒋斌守汉城,兵各五千。会使护军荀恺、前将军李辅各统万人,恺围汉城,辅围乐城。会径过,西出阳安口,遣人祭诸葛亮之墓。使护军胡烈等行前,攻破关城,得库藏积谷。姜维自沓中还,至阴平,合集士众,欲赴关城。未到,闻其已破,退趣白水,与蜀将张翼、廖化等合守剑阁拒会。会移檄蜀将吏士民曰:

往者汉祚衰微,率土分崩,生民之命,几于泯灭。太祖武皇帝神武圣哲,拨乱反正,拯其将坠,造我区夏。高祖文皇帝应天顺民,受命践阼。烈祖明皇帝奕世重光,恢拓洪业。然江山之外,异政殊俗,率土齐民未蒙王化,此三祖所以顾怀遗恨也。今主上圣德钦明,绍隆前绪,宰辅忠肃明允,劬劳王室,布政垂惠而万邦协和,施德百蛮而肃慎致贡。悼彼巴蜀,独为匪民,愍此百姓,劳役未已。是以命授六师,龚行天罚,征西、雍州、镇西诸军,五道并进。古之行军,以仁为本,以义治之;王者之师,有征无战;故虞舜舞干戚而服有苗,周武有散财、发廪、表闾之义。今镇西奉辞衔命,摄统戎重,庶弘文告之训,以济元元之命,非欲穷武极战,以快一朝之政,故略陈安危之要,其敬听话言。

益州先主以命世英才,兴兵朔野,困踬冀、徐之郊,制命绍、布之手,太祖拯而济之,与隆大好。中更背违,弃同即异,诸葛孔明仍规秦川,姜伯约屡出陇右,劳动我边境,侵扰我氐、羌,方国家多故,未遑修九伐之征也。今边境乂清,方内无事,畜力待时,并兵一向,而巴蜀一州之众,分张守备,难以御天下之师。段谷、侯和沮伤之气,难以敌堂堂之陈。比年以来,曾无宁岁,征夫勤瘁,难以当子来之民。此皆诸贤所亲见也。蜀相壮见禽於秦,公孙述授首于汉,九州之险,是非一姓。此皆诸贤所备闻也。明者见危于无形,智者规祸于未萌,是以微子去商,长为周宾,陈平背项,立功于汉。岂晏安鸩毒,怀禄而不变哉?今国朝隆天覆之恩,宰辅弘宽恕之德,先惠后诛,好生恶杀。往者吴将孙壹举众内附,位为上司,宠秩殊异。文钦、唐咨为国大害,叛主仇贼,还为戎首。咨困逼禽获,钦二子还降,皆将军、封侯;咨与闻国事。壹等穷踧归命,犹加盛宠,况巴蜀贤知见机而作者哉!诚能深鉴成败,邈然高蹈,投迹微子之踪,错身陈平之轨,则福同古人,庆流来裔,百姓士民,安堵旧业,农不易亩,巿不回肆,去累卵之危,就永安之福,岂不美与!若偷安旦夕,迷而不反,大兵一发,玉石皆碎,虽欲悔之,亦无及已。其详择利害,自求多福,各具宣布,咸使闻知。

邓艾追姜维到阴平,简选精锐,欲从汉德阳入江由、左儋道诣绵竹,趣成都,与诸葛绪共行。绪以本受节度邀姜维,西行非本诏,遂进军前向白水,与会合。会遣将军田章等从剑阁西,径出江由。未至百里,章先破蜀伏兵三校,艾使章先登。遂长驱而前。会与绪军向剑阁,会欲专军势,密白绪畏懦不进,槛车徵还。军悉属会,进攻剑阁,不克,引退,蜀军保险拒守。艾遂至绵竹,大战,斩诸葛瞻。维等闻瞻巳破,率其众东入于巴。会乃进军至涪,遣胡烈、田续、庞会等追维。艾进军向成都,刘禅诣艾降,遣使敕维等令降于会。维至广汉郪县,令兵悉放器仗,送节传於胡烈,便从东道诣会降。会上言曰:“贼姜维、张翼、廖化、董厥等逃死遁走,欲趣成都。臣辄遣司马夏侯咸、护军胡烈等,经从剑阁,出新都、大渡截其前,参军爰青彡、将军句安等蹑其后,参军皇甫闿、将军王买等从涪南出冲其腹,臣据涪县为东西势援。维等所统步骑四五万人,擐甲厉兵,塞川填谷,数百里中首尾相继,凭恃其众,方轨而西。臣敕咸、闿等令分兵据势,广张罗罔,南杜走吴之道,西塞成都之路,北绝越逸之径,四面云集,首尾并进,蹊路断绝,走伏无地。臣又手书申喻,开示生路,群寇困逼,知命穷数尽,解甲投戈,面缚委质,印绶万数,资器山积。昔舜舞干戚,有苗自服;牧野之师,商旅倒戈:有征无战,帝王之盛业。全国为上,破国次之;全军为上,破军次之:用兵之令典。陛下圣德,侔踪前代,翼辅忠明,齐轨公旦,仁育群生,义征不譓,殊俗向化,无思不服,师不逾时,兵不血刃,万里同风,九州共贯。臣辄奉宣诏命,导扬恩化,复其社稷,安其闾伍,舍其赋调,弛其征役,训之德礼以移其风,示之轨仪以易其俗,百姓欣欣,人怀逸豫,后来其苏,义无以过。”会于是禁检士众不得钞略,虚己诱纳,以接蜀之群司,与维情好欢甚。十二月诏曰:“会所向摧弊,前无强敌,缄制众城,罔罗迸逸。蜀之豪帅,面缚归命,谋无遗策,举无废功。凡所降诛,动以万计,全胜独克,有征无战。拓平西夏,方隅清晏。其以会为司徒,进封县侯,增邑万户。封子二人亭侯,邑各千户。”

会内有异志,因邓艾承制专事,密白艾有反状,於是诏书槛车徵艾。司马文王惧艾或不从命,敕会并进军成都,监军卫瓘在会前行,以文王手笔令宣喻艾军,艾军皆释仗,遂收艾入槛车。会所惮惟艾,艾既禽而会寻至,独统大众,威震西土。自谓功名盖世,不可复为人下,加猛将锐卒皆在己手,遂谋反。欲使姜维等皆将蜀兵出斜谷,会自将大众随其后。既至长安,令骑士从陆道,步兵从水道顺流浮渭入河,以为五日可到孟津,与骑会洛阳,一旦天下可定也。会得文王书云:“恐邓艾或不就徵,今遣中护军贾充将步骑万人径入斜谷,屯乐城,吾自将十万屯长安,相见在近。”会得书,惊呼所亲语之曰:“但取邓艾,相国知我能独办之;今来大重,必觉我异矣,便当速发。事成,可得天下;不成,退保蜀汉,不失作刘备也。我自淮南以来,画无遗策,四海所共知也。我欲持此安归乎!”会以五年正月十五日至,其明日,悉请护军、郡守、牙门骑督以上及蜀之故官,为太后发丧于蜀朝堂。矫太后遗诏,使会起兵废文王,皆班示坐上人,使下议讫,书版署置,更使所亲信代领诸军。所请群官,悉闭著益州诸曹屋中,城门宫门皆闭,严兵围守。会帐下督丘建本属胡烈,烈荐之文王,会请以自随,任爱之。建愍烈独坐,启会,使听内一亲兵出取饮食,诸牙门随例各内一人。烈绐语亲兵及疏与其子曰:“丘建密说消息,会已作大坑,白棓数千,欲悉呼外兵入,人赐白〈巾臽〉,拜为散将,以次棓杀坑中。”诸牙门亲兵亦咸说此语,一夜传相告,皆遍。或谓会:“可尽杀牙门骑督以上。”会犹豫未决。十八日日中,烈军兵与烈儿雷鼓出门,诸军兵不期皆鼓噪出,曾无督促之者,而争先赴城。时方给与姜维铠杖,白外有匈匈声,似失火,有顷,白兵走向城。会惊,谓维曰:“兵来似欲作恶,当云何?”维曰:“但当击之耳。“会遣兵悉杀所闭诸牙门郡守,内人共举机以柱门,兵斫门,不能破。斯须,门外倚梯登城,或烧城屋,蚁附乱进,矢下如雨,牙门、郡守各缘屋出,与其卒兵相得。姜维率会左右战,手杀五六人,众既格斩维,争赴杀会。会时年四十,将士死者数百人。

初,艾为太尉,会为司徒,皆持节、都督诸军如故,咸未受命而毙。会兄毓,以四年冬薨,会竟未知问。会兄子邕,随会与俱死,会所养兄子毅及峻、辿等下狱,当伏诛。司马文王表天子下诏曰:“峻等祖父繇,三祖之世,极位台司,佐命立勋,飨食庙庭。父毓,历职内外,幹事有绩。昔楚思子文之治,不灭斗氏之祀。晋录成宣之忠,用存赵氏之后。以会、邕之罪,而绝繇、毓之类,吾有愍然!峻、辿兄弟特原,有官爵者如故。惟毅及邕息伏法。”或曰,毓曾密启司马文王,言会挟术难保,不可专任,故宥峻等云。

初,文王欲遣会伐蜀,西曹属邵悌求见曰:“今遣锺会率十馀万众伐蜀,愚谓会单身无重任,不若使馀人行。”文王笑曰:“我宁当复不知此耶?蜀为天下作患,使民不得安息,我今伐之如指掌耳,而众人皆言蜀不可伐。夫人心豫怯则智勇并竭,智勇并竭而强使之,適为敌禽耳。惟锺会与人意同,今遣会伐蜀,必可灭蜀。灭蜀之后,就如卿所虑,当何所能一办耶?凡败军之将不可以语勇,亡国之大夫不可与图存,心胆以破故也。若蜀以破,遗民震恐,不足与图事;中国将士各自思归,不肯与同也。若作恶,祗自灭族耳。卿不须忧此,慎莫使人闻也。”及会白邓艾不轨,文王将西,悌复曰:“锺会所统,五六倍于邓艾,但可敕会取艾,不足自行。”文王曰:“卿忘前时所言邪,而更云可不须行乎?虽尔,此言不可宣也。我要自当以信义待人,但人不当负我,我岂可先人生心哉!近日贾护军问我,言:‘颇疑锺会不?’我答言:'如今遣卿行,宁可复疑卿邪?'贾亦无以易我语也。我到长安,则自了矣。“军至长安,会果已死,咸如所策。

会尝论易无互体、才性同异。及会死后,于会家得书二十篇,名曰道论,而实刑名家也,其文似会。初,会弱冠与山阳王弼并知名。弼好论儒道,辞才逸辩,注易及老子,为尚书郎,年二十馀卒。

评曰:王凌风节格尚,毌丘俭才识拔幹,诸葛诞严毅威重,锺会精练策数,咸以显名,致兹荣任,而皆心大志迂,不虑祸难,变如发机,宗族涂地,岂不谬惑邪!邓艾矫然强壮,立功立事,然闇于防患,咎败旋至,岂远知乎诸葛恪而不能近自见,此盖古人所谓目论者也。

\part{魏书二十九}
\chapter{方技传第二十九}

华佗字元化,沛国谯人也,一名旉。游学徐土,兼通数经。沛相陈珪举孝廉,太尉黄琬辟,皆不就。晓养性之术,时人以为年且百岁而貌有壮容。又精方药,其疗疾,合汤不过数种,心解分剂,不复称量,煮熟便饮,语其节度,舍去辄愈。若当灸,不过一两处,每处不过七八壮,病亦应除。若当针,亦不过一两处,下针言“当引某许,若至,语人”。病者言“巳到”,应便拔针,病亦行差。若病结积在内,针药所不能及,当须刳割者,便饮其麻沸散,须臾便如醉死无所知,因破取。病若在肠中,便断肠湔洗,缝腹膏摩,四五日差,不痛,人亦不自寤,一月之间,即平复矣。

故甘陵相夫人有娠六月,腹痛不安,佗视脉,曰:“胎已死矣。”使人手摸知所在,在左则男,在右则女。人云“在左”,於是为汤下之,果下男形,即愈。

县吏尹世苦四支烦,口中乾,不欲闻人声,小便不利。佗曰:“试作热食,得汗则愈;不汗,后三日死。”即作热食而不汗出,佗曰:“藏气已绝於内,当啼泣而绝。”果如佗言。

府吏倪寻、李延共止,俱头痛身热,所苦正同。佗曰:“寻当下之,延当发汗。”或难其异,佗曰:“寻外实,延内实,故治之宜殊。”即各与药,明旦并起。

盐渎严昕与数人共候佗,適至,佗谓昕曰:“君身中佳否?”昕曰:“自如常。”佗曰:“君有急病见於面,莫多饮酒。”坐毕归,行数里,昕卒头眩堕车,人扶将还,载归家,中宿死。

故督邮顿子献得病已差,诣佗视脉,曰:“尚虚,未得复,勿为劳事,御内即死。临死,当吐舌数寸。”其妻闻其病除,从百馀里来省之,止宿交接,中间三日发病,一如佗言。

督邮徐毅得病,佗往省之。毅谓佗曰:“昨使医曹吏刘租针胃管讫,便苦欬嗽,欲卧不安。”佗曰:“刺不得胃管,误中肝也,食当日减,五日不救。”遂如佗言。

东阳陈叔山小男二岁得疾,下利常先啼,日以羸困。问佗,佗曰:“其母怀躯,阳气内养,乳中虚冷,儿得母寒,故令不时愈。”佗与四物女宛丸,十日即除。

彭城夫人夜之厕,虿螫其手,呻呼无赖。佗令温汤近热,渍手其中,卒可得寐,但旁人数为易汤,汤令暖之,其旦即愈。

军吏梅平得病,除名还家,家居广陵,未至二百里,止亲人舍。有顷,佗偶至主人许,主人令佗视平,佗谓平曰:“君早见我,可不至此。今疾已结,促去可得与家相见,五日卒。”应时归,如佗所刻。

佗行道,见一人病咽塞,嗜食而不得下,家人车载欲往就医。佗闻其呻吟,驻车往视,语之曰:“向来道边有卖饼家蒜齑大酢,从取三升饮之,病自当去。”即如佗言,立吐蛇一枚,县车边,欲造佗。佗尚未还,小儿戏门前,逆见,自相谓曰:“似逢我公,车边病是也。”疾者前入坐,见佗北壁县此蛇辈约以十数。

又有一郡守病,佗以为其人盛怒则差,乃多受其货而不加治,无何弃去,留书骂之。郡守果大怒,令人追捉杀佗。郡守子知之,属使勿逐。守瞋恚既甚,吐黑血数升而愈。

又有一士大夫不快,佗云:“君病深,当破腹取。然君寿亦不过十年,病不能杀君,忍病十岁,寿俱当尽,不足故自刳裂。”士大夫不耐痛痒,必欲除之。佗遂下手,所患寻差,十年竟死。

广陵太守陈登得病,胸中烦懑,面赤不食。佗脉之曰:“府君胃中有虫数升,欲成内疽,食腥物所为也。”即作汤二升,先服一升,斯须尽服之。食顷,吐出三升许虫,赤头皆动,半身是生鱼脍也,所苦便愈。佗曰:“此病后三期当发,遇良医乃可济救。”依期果发动,时佗不在,如言而死。

太祖闻而召佗,佗常在左右。太祖苦头风,每发,心乱目眩,佗针鬲,随手而差。

李将军妻病甚,呼佗视脉,曰:“伤娠而胎不去。”将军言:“闻实伤娠,胎已去矣。”佗曰:“案脉,胎未去也。”将军以为不然。佗舍去,妇稍小差。百馀日复动,更呼佗,佗曰:“此脉故事有胎。前当生两儿,一儿先出,血出甚多,后儿不及生。母不自觉,旁人亦不寤,不复迎,遂不得生。胎死,血脉不复归,必燥著母脊,故使多脊痛。今当与汤,并针一处,此死胎必出。”汤针既加,妇痛急如欲生者。佗曰:“此死胎久枯,不能自出,宜使人探之。”果得一死男,手足完具,色黑,长可尺所。

佗之绝技,凡此类也。然本作士人,以医见业,意常自悔,后太祖亲理,得病笃重,使佗专视。佗曰:“此近难济,恒事攻治,可延岁月。”佗久远家思归,因曰:“当得家书,方欲暂还耳。“到家,辞以妻病,数乞期不反。太祖累书呼,又敕郡县发遣。佗恃能厌食事,犹不上道。太祖大怒,使人往检。若妻信病,赐小豆四十斛,宽假限日;若其虚诈,便收送之。於是传付许狱,考验首服。荀彧请曰:“佗术实工,人命所县,宜含宥之。”太祖曰:“不忧,天下当无此鼠辈耶?”遂考竟佗。佗临死,出一卷书与狱吏,曰:“此可以活人。”吏畏法不受,佗亦不强,索火烧之。佗死后,太祖头风未除。太祖曰:“佗能愈此。小人养吾病,欲以自重,然吾不杀此子,亦终当不为我断此根原耳。”及后爱子仓舒病因,太祖叹曰:“吾悔杀华佗,令此儿强死也。”

初,军吏李成苦欬嗽,昼夜不寤,时吐脓血,以问佗。佗言:“君病肠臃,欬之所吐,非从肺来也。与君散两钱,当吐二升馀脓血讫,快自养,一月可小起,好自将爱,一年便健。十八岁当一小发,服此散,亦行复差。若不得此药,故当死。”复与两钱散,成得药去。五六岁,亲中人有病如成者,谓成曰:“卿今强健,我欲死,何忍无急去药,以待不祥?先持贷我,我差,为卿从华佗更索。”成与之。已故到谯,適值佗见收,怱怱不忍从求。后十八岁,成病竟发,无药可服,以至於死。

广陵吴普、彭城樊阿皆从佗学。普依准佗治,多所全济。佗语普曰:“人体欲得劳动,但不当使极尔。动摇则谷气得消,血脉流通,病不得生,譬犹户枢不朽是也。是以古之仙者为导引之事,熊颈鸱顾,引輓腰体,动诸关节,以求难老。吾有一术,名五禽之戏,一曰虎,二曰鹿,三曰熊,四曰猿,五曰鸟,亦以除疾,并利蹄足,以当导引。体中不快,起作一禽之戏,沾濡汗出,因上著粉,身体轻便,腹中欲食。”普施行之,年九十馀,耳目聪明,齿牙完坚。阿善针术。凡医咸言背及胸藏之间不可妄针,针之不过四分,而阿针背入一二寸,巨阙胸藏针下五六寸,而病辄皆瘳。阿从佗求可服食益於人者,佗授以漆叶青黏散。漆叶屑一升,青黏屑十四两,以是为率,言久服去三虫,利五藏,轻体,使人头不白。阿从其言,寿百馀岁。漆叶处所而有,青黏生於丰、沛、彭城及朝歌云。

杜夔字公良,河南人也。以知音为雅乐郎,中平五年,疾去官。州郡司徒礼辟,以世乱奔荆州。荆州牧刘表令与孟曜为汉主合雅乐,乐备,表欲庭观之,夔谏曰:“今将军号【不】为天子合乐,而庭作之,无乃不可乎!”表纳其言而止。后表子琮降太祖,太祖以夔为军谋祭酒,参太乐事,因令创制雅乐。

夔善钟律,聪思过人,丝竹八音,靡所不能,惟歌舞非所长。时散郎邓静、尹齐善咏雅乐,歌师尹胡能歌宗庙郊祀之曲,舞师冯肃、服养晓知先代诸舞,夔总统研精,远考诸经,近采故事,教习讲肄,备作乐器,绍复先代古乐,皆自夔始也。

黄初中,为太乐令、协律都尉。汉铸钟工柴玉巧有意思,形器之中,多所造作,亦为时贵人见知。夔令玉铸铜钟,其声均清浊多不如法,数毁改作。玉甚厌之,谓夔清浊任意,颇拒捍夔。夔、玉更相白於太祖,太祖取所铸钟,杂错更试,然后知夔为精而玉之妄也,於是罪玉及诸子,皆为养马士。文帝爱待玉,又尝令夔与〈马真〉等於宾客之中吹笙鼓琴,夔有难色,由是帝意不阅。后因他事系夔,使〈马真〉等就学,夔自谓所习者雅,仕宦有本,意犹不满,遂黜免以卒。

弟子河南邵登、张泰、桑馥,各至太乐丞,下邳陈颃司律中郎将。自左延年等虽妙於音,咸善郑声,其好古存正莫及夔。

朱建平,沛国人也。善相术,於闾巷之间,效验非一。太祖为魏公,闻之,召为郎。文帝为五官将,坐上会客三十馀人,文帝问己年寿,又令遍相众宾。建平曰:“将军当寿八十,至四十时当有小厄,愿谨护之。”谓夏侯威曰:“君四十九位为州牧,而当有厄,厄若得过,可年至七十,致位公辅。”谓应璩曰:“君六十二位为常伯,而当有厄,先此一年,当独见一白狗,而旁人不见也。”谓曹彪曰:“君据藩国,至五十七当厄於兵,宜善防之。”

初,颍川荀攸、锺繇相与亲善。攸先亡,子幼。繇经纪其门户,欲嫁其妾。与人书曰:“吾与公达曾共使朱建平相,建平曰:'荀君虽少,然当以后事付锺君。'吾时啁之曰:'惟当嫁卿阿骛耳。'何意此子竟早陨没,戏言遂验乎!今欲嫁阿骛,使得善处。追思建平之妙,虽唐举、许负何以复加也!”

文帝黄初七年,年四十,病困,谓左右曰:“建平所言八十,谓昼夜也,吾其决矣。”顷之,果崩。夏侯威为兖州刺史,年四十九,十二月上旬得疾,念建平之言,自分必死,豫作遗令及送丧之备,咸使素办。至下旬转差,垂以平复。三十日日昃,请纪纲大吏设酒,曰:“吾所苦渐平,明日鸡鸣,年便五十,建平之戒,真必过矣。”威罢客之后,合瞑疾动,夜半遂卒。璩六十一为侍中,直省内,欻见白狗,问之众人,悉无见者。於是数聚会,并急游观田里,饮宴自娱,过期一年,六十三卒。曹彪封楚王,年五十七,坐与王凌通谋,赐死。凡说此辈,无不如言,不能具详,故粗记数事。惟相司空王昶、征北将军程喜、中领军王肃有蹉跌云。肃年六十二,疾笃,众医并以为不愈。肃夫人问以遣言,肃云:“建平相我逾七十,位至三公,今皆未也,将何虑乎!”而肃竟卒。

建平又善相马。文帝将出,取马外入,建平道遇之,语曰:“此马之相,今日死矣。”帝将乘马,马恶衣香,惊咬文帝膝,帝大怒,即便杀之。建平黄初中卒。

周宣字孔和,乐安人也。为郡吏。太守杨沛梦人曰:“八月一日曹公当至,必与君杖,饮以药酒。”使宣占之。是时黄巾贼起,宣对曰:“夫杖起弱者,药治人病,八月一日,贼必除灭。”至期,贼果破。

后东平刘桢梦蛇四足,穴居门中,使宣占之,宣曰:“此为国梦,非君家之事也。当杀女子而作贼者。”顷之,女贼郑、姜遂俱夷讨,以蛇女子之祥,足非蛇之所宜故也。

文帝问宣曰:“吾梦殿屋两瓦堕地,化为双鸳鸯,此何谓也?“宣对曰:“后宫当有暴死者。”帝曰:“吾诈卿耳!”宣对曰:“夫梦者意耳,苟以形言,便占吉凶。”言未毕,而黄门令奏宫人相杀。无几,帝复问曰:“我昨夜梦青气自地属天。”宣对曰:“天下当有贵女子冤死。“是时,帝已遣使赐甄后玺书,闻宣言而悔之,遣人追使者不及。帝复问曰:“吾梦摩钱文,欲令灭而更愈明,此何谓邪?”宣怅然不对。帝重问之,宣对曰:“此自陛下家事,虽意欲尔而太后不听,是以文欲灭而明耳。”时帝欲治弟植之罪,偪於太后,但加贬爵。以宣为中郎,属太史。

尝有问宣曰:“吾昨夜梦见刍狗,其占何也?”宣答曰:“君欲得美食耳!”有顷,出行,果遇丰膳。后又问宣曰:“昨夜复梦见刍狗,何也?“宣曰:“君欲堕车折脚,宜戒慎之。“顷之,果如宣言。后又问宣:“昨夜复梦见刍狗,何也?“宣曰:“君家失火,当善护之。”俄遂火起。语宣曰:“前后三时,皆不梦也。聊试君耳,何以皆验邪?”宣对曰:“此神灵动君使言,故与真梦无异也。”又问宣曰:“三梦刍狗而其占不同,何也?”宣曰:“刍狗者,祭神之物。故君始梦,当得馀食也。祭祀既讫,则刍狗为车所轹,故中梦当堕车折脚也。刍狗既车轹之后,必载以为樵,故后梦忧失火也。”宣之叙梦,凡此类也。十中八九,世以比建平之相矣。其馀效故不次列。明帝末卒。

管辂字公明,平原人也。容貌粗丑,无威仪而嗜酒,饮食言戏,不择非类,故人多爱之而不敬也。

父为利漕,利漕民郭恩兄弟三人,皆得躄疾,使辂筮其所由。辂曰:“卦中有君本墓,墓中有女鬼,非君伯母,当叔母也。昔饥荒之世,当有利其数升米者,排著井中,啧啧有声,推一大石,下破其头,孤魂冤痛,自诉於天。”於是恩涕泣服罪。

广平刘奉林妇病困,已买棺器。时正月也,使辂占,曰:“命在八月辛卯日日中之时。”林谓必不然,而妇渐差,至秋发动,一如辂言。

辂往见安平太守王基,基令作卦,辂曰:“当有贱妇人,生一男儿,堕地便走入灶中死。又床上当有一大蛇衔笔,小大共视,须臾去之也。又乌来入室中,与燕共斗,燕死,乌去。有此三怪。”基大惊,问其吉凶。辂曰:“直客舍久远,魑魅魍魉为怪耳。儿生便走,非能自走,直宋无忌之妖将其入灶也。大蛇衔笔,直老书佐耳。乌与燕斗,直老铃下耳。今卦中见象而不见其凶,知非妖咎之徵,自无所忧也。”后卒无患。

时信都令家妇女惊恐,更互疾病,使辂筮之。辂曰:“君北堂西头,有两死男子,一男持矛,一男持弓箭,头在壁内,脚在壁外。持矛者主刺头,故头重痛不得举也。持弓箭者主射胸腹,故心中县痛不得饮食也。昼则浮游,夜来病人,故使惊恐也。“於是掘徙骸骨,家中皆愈。

清河王经去官还家,辂与相见。经曰:“近有一怪,大不喜之,欲烦作卦。”卦成,辂曰:“爻吉,不为怪也。君夜在堂户前,有一流光如燕爵者,入君怀中,殷殷有声,内神不安,解衣彷徉,招呼妇人,觅索馀光。”经大笑曰:“实如君言。”辂曰:“吉,迁官之徵也,其应行至。”顷之,经为江夏太守。

辂又至郭恩家,有飞鸠来在梁头,鸣甚悲。辂曰:“当有老公从东方来,携豚一头,酒一壶。主人虽喜,当有小故。”明日果有客,如所占。恩使客节酒、戒肉、慎火,而射鸡作食,箭从树间激中数岁女子手,流血惊怖。

辂至安德令刘长仁家,有鸣鹊来在閤屋上,其声甚急。辂曰:“鹊言东北有妇昨杀夫,牵引西家人夫离娄,候不过日在虞渊之际,告者至矣。”到时,果有东北同伍民来告,邻妇手杀其夫,诈言西家人与夫有嫌,来杀我婿。

辂至列人典农王弘直许,有飘风高三尺馀,从申上来,在庭中幢幢回转,息以复起,良久乃止。直以问辂,辂曰:“东方当有马吏至,恐父哭子,如何!”明日胶东吏到,直子果亡。直问其故,辂曰:“其日乙卯,则长子之候也。木落於申,斗建申,申破寅,死丧之候也。日加午而风发,则马之候也。离为文章,则吏之候也。申未为虎,虎为大人,则父之候也。”有雄雉飞来,登直内铃柱头,直大以不安,令辂作卦,辂曰:“到五月必迁。”时三月也,至期,直果为勃海太守。

馆陶令诸葛原迁新兴太守,辂往祖饯之,宾客并会。原自起取燕卵、蜂窠、蜘蛛著器中,使射覆。卦成,辂曰:“第一物,含气须变,依乎宇堂,雄雌以形,翅翼舒张,此燕卵也。第二物,家室倒县,门户众多,藏精育毒,得秋乃化,此蜂窠也。第三物,觳觫长足,吐丝成罗,寻网求食,利在昬夜,此蜘蛛也。”举坐惊喜。

辂族兄孝国,居在斥丘,辂往从之,与二客会。客去后,辂谓孝国曰:“此二人天庭及口耳之间同有凶气,异变俱起,双魂无宅,流魂于海,骨归于家,少许时当并死也。”复数十日,二人饮酒醉,夜共载车,牛惊下道入漳河中,皆即溺死也。

当此之时,辂之邻里,外户不闭,无相偷窃者。清河太守华表,召辂为文学掾。安平赵孔曜荐辂於冀州刺史裴徽曰:“辂雅性宽大,与世无忌,仰观天文则同妙甘公、石申,俯览周易则齐思季主。今明使君方垂神幽薮,留精九皋,辂宜蒙阴和之应,得及羽仪之时。”徽於是辟为文学从事,引与相见,大善友之。徙部钜鹿,迁治中别驾。

初应州召,与弟季儒共载,至武城西,自卦吉凶,语儒云:“当在故城中见三貍,尔者乃显。”前到河西故城角,正见三貍共踞城侧,兄弟并喜。正始九年举秀才。

十二月二十八日,吏部尚书何晏请之,邓飏在晏许。晏谓辂曰:“闻君著爻神妙,试为作一卦,知位当至三公不?”又问:“连梦见青蝇数十头,来在鼻上,驱之不肯去,有何意故?”辂曰:“夫飞鸮,天下贱鸟,及其在林食椹,则怀我好音,况辂心非草木,敢不尽忠?昔元、凯之弼重华,宣惠慈和,周公之翼成王,坐而待旦,故能流光六合,万国咸宁。此乃履道休应。非卜筮之所明也。今君侯位重山岳,势若雷电,而怀德者鲜,畏威者众,殆非小心翼翼多福之仁。又鼻者艮,此天中之山,高而不危,所以长守贵也。今青蝇臭恶,而集之焉。位峻者颠,轻豪者亡,不可不思害盈之数,盛衰之期。是故山在地中曰谦,雷在天上曰壮;谦则裒多益寡,壮则非礼不履。未有损己而不光大,行非而不伤败。愿君侯上追文王六爻之旨,下思尼父彖象之义,然后三公可决,青蝇可驱也。”飏曰:“此老生之常谭。”辂答曰:“夫老生者见不生,常谭者见不谭。”晏曰:“过岁更当相见。”辂还邑舍,具以此言语舅氏,舅氏责辂言太切至。辂曰;“与死人语,何所畏邪?”舅大怒,谓辂狂悖。岁朝,西北大风,尘埃蔽天,十馀日,闻晏、飏皆诛,然后舅氏乃服。

始辂过魏郡太守锺毓,共论易义,辂因言“卜可知君生死之日。”毓使筮其生日月,如言无蹉跌。毓大愕然,曰:“君可畏也。死以付天,不以付君。“遂不复筮。毓问辂:“天下当太平否?”辂曰:“方今四九天飞,利见大人,神武升建,王道文明,何忧不平?”毓未解辂言,无几,曹爽等诛,乃觉寤云。

平原太守刘邠取印囊及山鸡毛著器中,使筮。辂曰:“内方外圆,五色成文,含宝守信,出则有章,此印囊也。高岳岩岩,有鸟朱身,羽翼玄黄,鸣不失晨,此山鸡毛也。”邠曰:“此郡官舍,连有变怪,使人恐怖,其理何由?”辂曰:“或因汉末之乱,兵马扰攘,军尸流血,汙染丘山,故因昬夕,多有怪形也。明府道德高妙,自天祐之,愿安百禄,以光休宠。”

清河令徐季龙使人行猎,令辂筮其所得。辂曰:“当获小兽,复非食禽,虽有爪牙,微而不强,虽有文章,蔚而不明,非虎非雉,其名曰狸。”猎人暮归,果如辂言。季龙取十三种物,著大箧中,使辂射。云:“器中藉藉有十三种物。”先说鸡子,后道蚕蛹,遂一一名之,惟以梳为枇耳。

辂随军西行,过毌丘俭墓下,倚树哀吟,精神不乐。人问其故,辂曰:“林木虽茂,无形可久;碑诔虽美,无后可守。玄武藏头,苍龙无足,白虎衔尸,朱雀悲哭,四危以备,法当灭族。不过二载,其应至矣。”卒如其言。后得休,过清河倪太守。时天旱,倪问辂雨期,辂曰:“今夕当雨。”是日旸燥,昼无形似,府丞及令在坐,咸谓不然。到鼓一中,星月皆没,风云并起,竟成快雨。於是倪盛脩主人礼,共为欢乐。

正元二年,弟辰谓辂曰:“大将军待君意厚,冀当富贵乎?”辂长叹曰:“吾自知有分直耳,然天与我才明,不与我年寿,恐四十七八间,不见女嫁儿娶妇也。若得免此,欲作洛阳令,可使路不拾遣,枹鼓不鸣。但恐至太山治鬼,不得治生人,如何!”辰问其故,辂曰:“吾额上无生骨,眼中无守精,鼻无梁柱,脚无天根,背无三甲,腹无三壬,此皆不寿之验。又吾本命在寅,加月食夜生。天有常数,不可得讳,但人不知耳。吾前后相当死者过百人,略无错也。“是岁八月,为少府丞。明年二月卒,年四十八。

评曰:华佗之医诊,杜夔之声乐,朱建平之相术,周宣之相梦,管辂之术筮,诚皆玄妙之殊巧,非常之绝技矣。昔史迁著扁鹊、仓公、日者之传,所以广异闻而表奇事也。故存录云尔。

\part{魏书三十}
\chapter{乌丸鲜卑东夷传第三十}

《书》载“蛮夷猾夏”,《诗》称“猃狁孔炽”,久矣其为中国患也。

秦、汉以来,匈奴久为边害。孝武虽外事四夷,东平两越、朝鲜,西讨贰师、大宛,开邛苲、夜郎之道,然皆在荒服之外,不能为中国轻重。而匈奴最逼于诸夏,胡骑南侵则三边受敌,是以屡遣卫、霍之将,深入北伐,穷追单于,夺其饶衍之地。后遂保塞称藩,世以衰弱。

建安中,呼厨泉南单于入朝,遂留内待,使右贤王抚其国,而匈奴折节,过于汉旧。然乌丸、鲜卑稍更强盛,亦因汉末之乱,中国多事,不遑外讨,故得擅(汉)[漠]南之地,寇暴城邑,杀略人民,北边仍受其困。会袁绍兼河北,乃抚有三郡乌丸,宠其名王而收其精骑。

其后尚、熙又逃于蹋顿。蹋顿又骁武,边长老皆比之冒顿,恃其阻远,敢受亡命,以雄百蛮。太祖潜师北伐,出其不意,一战而定之,夷狄慑服,威振朔土。遂引乌丸之众服从征讨,而边民得用安息。后鲜卑大人轲比能复制御群狄,尽收匈奴故地,自云中、五原以东抵辽水,皆为鲜卑庭。数犯塞寇边,幽、并苦之。田豫有马城之围,毕轨有陉北之败。青龙中,帝乃听王雄,遣剑客刺之。然后种落离散互相侵伐,强者远遁,弱者请服。由是边陲差安,(汉)[漠]南少事,虽时颇钞盗,不能复相扇动矣。乌丸、鲜卑即古所谓东胡也。其习俗前事,撰汉记者己录而载之矣。故但举汉末魏初以来,以备四夷之变云。

汉末,辽西乌丸大人丘力居,众五千余落,上谷乌丸大人难楼,众九千余落,各称王,而辽东属国乌丸大人苏仆延,众千余落,自称峭王,右北平乌丸大人乌延。众八百余落,自称汗鲁王,皆有计策勇健。中山太守张纯叛人丘力居众中,自号弥天安定王,为三郡乌丸元帅,寇略青、徐、幽、冀四州,杀略吏民。灵帝末,以刘虞为幽州牧,募胡斩纯首,北州乃定。后丘力居死。子楼班年小,从子蹋顿有武略,代立,总摄三王部,众皆从其教令。袁绍与公孙瓒连战不决,蹋顿遣使诣绍求和亲,助绍击瓒,破之。绍矫制赐蹋顿、难峭王、汗鲁王印绶,皆以为单于。后楼班大,峭王率其部众奉楼班为单于,蹋顿为王。然蹋顿多画计策。广阳阎柔,少没乌九、鲜卑中,为其种所归信。柔乃因鲜卑众,杀乌丸校尉邢举代之,绍因宠慰以安北边。后袁尚败奔蹋顿,凭其势,复图冀州。

会太祖平河北,柔帅鲜卑、乌丸归附,遂因以柔为校尉,犹持汉使节,治广宁如旧。建安十一年,太祖自征蹋顿于柳城,潜军诡道,未至百余里,虏乃觉。尚与蹋顿将众逆战于凡城,兵马甚盛。太祖登高望虏陈,柳军未进,观其小动,乃击破其众,临陈斩蹋顿首,死者被野。速附丸、楼班、乌延等走辽东,辽东悉斩,传送其首。其余遗迸皆降。及幽州、并州柔所统乌丸万余落,悉徙其族居中国,帅从其侯王大人种众与征伐。由是三郡乌丸为天下名骑。

鲜卑步度根既立,众稍衰弱。中兄扶罗韩亦别拥众数万为大人。建安中太祖定幽州,步度根与轲比能等因乌丸校尉阎柔上贡献。后代郡乌丸能臣氐等叛,求属扶罗韩,扶罗韩将万余骑迎之。到桑干,氐等议,以为扶罗韩部威禁宽缓,恐不见济,更遣人呼轲比能。比能即将万余骑到,当共盟誓。比能便于会上杀扶罗韩,扶罗韩子泄归泥及部众悉属比能。比能自以杀归泥父,特又善遇之。步度根由是怨比能。文帝践阼,田豫为乌丸校尉,持节并护鲜卑,屯昌平;步度根遣使献马,帝拜为王。后数与轲比能更相攻击,步度根部众稍寡弱,将其众万余落保太原、雁门郡。步度根乃使人招呼泄归泥曰:“汝父为比能所杀,不念报仇,反属怨家。今虽厚待汝,是欲杀汝计也。不如还我,我与汝是骨肉至亲,岂与仇等?”由是归泥将其部落逃归步度根,比能追之弗及。至黄初五年,步度根诣阙贡献,厚加赏赐,是后一心守边,不为寇害,而轲比能众遂强盛。明帝即位,务欲绥和戎狄,以息征伐,羁縻两部而已。至青龙元年,比能诱步度根深结和亲,于是步度根将泄归泥及部众悉保比能,寇钞并州,杀略吏民。帝遣骁骑将军秦朗征之,归泥叛比能,将其部众降,拜归义王,赐幢麾、曲盖、鼓吹,居并州如故。步度根为比能所杀。

轲比能本小种鲜卑,以勇健,断法平端。不贪财物,众推以为大人。部落近塞,自袁绍据河北,中国人多亡叛归之,教作兵器铠楯,颇学文字。故其勒御部众,拟则中国,出入弋猎,建立旌麾,以鼓节为进退。建实中,因阎柔上贡献。太祖西征关中,田银反河间,比能将三千余骑随柔击破银。后代郡乌丸反,比能复助为寇害,太祖以鄢陵侯彰为骁骑将军,北征,大破之。比能走出塞,后复通贡献。延康初,比能遣使献马。文帝亦立比能为附义王。

黄初二年,比能出诸魏人在鲜卑者五百余家,还居代郡。明年,比能帅部落大人小于代郡乌丸修武卢等三千余骑,驱牛马七万余口交市,遣魏人千余家居上谷。后与东部鲜卑大人素利及步度根三部争斗,更相攻击。田豫和合,使不得相侵。五年,比能复击素利,豫帅轻骑径进掎其后。比能使别小帅琐奴拒豫,豫进讨,破走之,由是怀贰。乃与辅国将军鲜于辅书曰:“夷狄不识文字,故校尉阎柔保我于天子。我与素利为仇,往年攻击之,而田校尉助素利。我临陈使琐奴往,闻使君来,即便引军退。步度根数数抄盗,又杀我弟,而诬我以抄盗。我夷狄虽不知礼义,兄弟子孙受天子印绶,牛马尚知美水草,况我有人心邪!将军当保明我于天子。”辅得书以闻,帝复使豫招纳安慰。比能众遂强盛,控弦十余万骑。每钞略得财物,均平分付,一决目前,终无所私,故得众死力,余部大人皆敬惮之,然犹未能及檀石槐也。

太和二年,豫遣译夏舍诣比能女婿郁筑鞬部舍为鞬所杀。其秋,豫将西部鲜卑蒲头、泄归泥出塞讨郁筑鞬,大破之。还至马城,比能自将三万骑围豫七日。上谷太守阎志,柔之弟也,素为鲜卑所倍。志往解喻,即解围去。后幽州刺史王雄并领校尉,抚以恩信。

比能数款塞,诣州奉贡献。至青龙元年,比能诱纳步度根,使叛并州,与结和亲,自勒万骑迎其累重于陉北。并州刺史毕轨遣将军苏尚、董弼等击之,比能遣子将骑与尚等会战于楼烦,临陈害尚、弼。至三年中,雄遣勇士韩龙刺杀比能,更立其弟。素利、弥加、厥机皆为大人,在辽西、右北平、渔阳塞外,道远初不为边患,然其种众多于比能。建安中,因阎柔上贡献,通市,太祖皆表宠以为王。厥机死,又立其子沙末汗为亲汉王。

延康初,又各遣使献马。文帝立素利、弥加为归义王。素利与比能更相攻击。太和二年,素利死。子小,以弟成律归为王,代摄其众。

《书》称:东渐于海,西被于流沙。其九服之制,可得而言也。然荒域之外,重译而至,非足迹车轨所及,未有知其国俗殊方者也。自虞暨周,西戎有白环之献,东夷有肃慎之贡,皆旷世而至,其遐远也如此。及汉氏遣张骞使西域,穷河源,经历诸园,遂置都护以总领之,然后西域之事具存,故汉宫得详载焉。魏兴,西域虽不能尽至,其大国龟兹、于寘、康居、乌孙、疏勒、月氏、鄯善、车师之属。无岁不奉朝贡,略如汉氏故事。而公孙渊仍父祖三世有辽东,天子为其绝域,委以海外之事,遂隔断东夷,不得通于诸夏。景初中,大兴师旅,诛渊,又潜军浮海,收乐浪、带方之郡,而后海表谧然,东夷屈服。其后高句丽背叛,又遣偏师致讨,穷追极远,逾乌丸、骨都,过沃沮,践肃慎之庭,东临大海。长老说有异面之人,近日之所出,遂周观诸国,采其法俗,小大区别,各有名号,可得详纪。虽夷狄之邦,而俎豆之象存。中国失礼,求之四夷,犹信。故撰次其国,列其同异,以接前史之所未备焉。

夫余在长城之北,去玄菟千里。南与高句丽,东与挹娄,西与鲜卑接,北有弱水,方可二千里。户八万。其民土着,有宫室、仓库、牢狱。多山陵、广泽,于东夷之域最平敝。土地宜五谷,不生五果。其人粗大,性强勇谨厚,不寇钞。国有君王,皆以六畜名官,有马加、牛加、猪加、狗加、大使、大使者、使者。邑落有豪民,名下户皆为奴仆。诸加别主四出,道大者主数千家,小者数百家。食饮皆用俎豆。会同、拜爵、洗爵,揖让升降。以殷正月祭天,国中大会,连日饮食歌舞,名曰迎鼓,于是时断刑狱,解囚徒。在国衣尚白,白布大袂,袍、裤,履革鞜。出国则尚缯绣锦罽,大人加狐狸、狖白、黑貂之裘,以金银饰帽。译人传辞,皆跪,手据地窃语。用刑严急,杀人者死,没其家人为奴婢。窃盗一责十二。男女淫,妇人妒,皆杀之。尤僧妒,已杀,尸之国南山上,至腐烂。女家欲得,输牛马乃与之。兄死妻嫂,与匈奴同俗。其国善养牲,出名马、赤玉、貂狖、美珠。珠大者如酸枣。以弓矢刀矛为兵,家家自有铠仗。国之耆老自说古之亡人。作城栅皆员,有似牢狱。行道昼夜无老幼皆歌,通日声不绝。有军事亦祭天,杀牛观蹄以占吉凶,蹄解者为凶,合者为吉。有敌,诸加自战,下户俱担粮饮食之。其死,夏月皆用冰。杀人殉葬。多者百数。厚葬,有椁无棺。

夫余本属玄菟。汉末,公孙度雄张海东,威服外夷,夫余王尉仇台更属辽东。时句丽、鲜卑强,度以夫余在二虏之间,妻以宗女。尉仇台死,简位居立。无适子,有孽子麻余。位居死,诸加共立麻余。牛加兄子名位居,为大使,轻财善施,国人附之,岁岁遣使诣京都贡献。

正始中,幽州刺史毋丘俭讨句丽,遣玄菟太守王颀诣夫余,位居遣大加郊迎,供军粮。季父牛加有二心,位居杀季父父子,籍没财物,遣使簿敛送官。旧夫余俗,水旱不调,五谷不熟,辄归咎于王,或言当易,或言当杀。麻余死,其子依虑年六岁,立以为王。汉时,夫余王葬用玉匣。常豫以付玄菟郡,王死则迎取以葬。公孙渊伏诛,玄菟库犹有玉匣一具。今夫余库有玉璧、珪、瓒数代之物,传世以为宝,耆老言先代之所赐也。其印文言“濊王之印”,国有故城名濊城,盖本濊貊之地,而夫余王其中,自谓“亡人”,抑有似也。

高句丽在辽东之东千里。南与朝鲜、濊貊,东与沃沮,北与夫余接。都于丸都之下,方可二千里,户三万。多大山深谷,无原泽。随山谷以为居,食涧水。无良田,虽力佃作,不足以实口腹。其俗节食,好治宫室,于所居之左右立大屋,祭鬼神,又祠灵星、社稷。其人性凶急,喜寇抄。其国有王,其官有相加、对卢、沛者、古雏加、主簿、优台丞、使者、皂衣先人,尊卑各有等级。

东夷旧语以为以为夫余别种,言语诸事,多与夫余同,其性气、衣服有异。本有五族,有渭奴都、绝奴部、顺奴部、灌奴部、桂娄部。本涓奴部为王,稍微弱,今桂娄部代之。汉时赐鼓吹技人,常从玄菟郡受朝服,衣帻,高句丽令主其名籍。后稍骄恣,不复诣郡,于东界筑小城,置朝服衣帻其中,岁时来取之,今胡犹名此城为帻沟溇。沟溇者,句丽名城也。其置官,有对卢则不置沛者,有沛者则不置对卢。王之宗族,其大加皆称古雏加。涓奴部本国主,今虽不为王,适统大人,得称古雏加,亦得立宗庙,祠灵星、社稷。绝奴部世与王婚,加古雏之号。诸大加亦自置使者、皂衣先人,名皆远于王,如卿大夫之家臣,会同坐起,不得与王家使者、阜衣先人同列。其国中大家不佃作,坐食者万余口,下户远担米粮鱼监供给之。其民喜歌舞,国中邑落,暮夜男女群聚,相就歌戏。无大仓库,家家自有小仓,名之为桴京。其人洁清自喜,善藏酿。跪拜申一脚,与夫余异,行步皆走。以十月祭天,国中大会,名曰东盟。其公会衣服皆锦绣金银以自饰。大加、主簿头着帻,如帻而无余,其小加着折风,形如弁。其国东有大穴,名隧穴,十月国中大会;迎隧神还于国东上祭之,置木隧于神坐。无牢狱,有罪诸加评议,便杀之,没人妻子为奴婢。其俗作婚姻,言语已定。女家作小屋于大屋后,名婿屋,婿暮至女家户外,自名跪拜,乞得就女宿,如是者再三,女父母乃听使就小屋中宿,傍顿钱帛,至生子已长大,乃将妇归家。其俗淫。男女已嫁娶,便稍作送终之衣。厚葬,金银财币,尽于送死,积石为封,列种松柏。其马皆小,便登山。国人有气力,习战斗,沃沮、东濊皆属焉。又有小水貊。句丽作国,依大水而居。西安平县北有有小水,南流人海,句丽别种依小水作国,因名之为小水貊,出好弓,所谓貊弓是也。

王莽初发高句丽兵以伐胡,不欲行,强迫遣之,皆亡出塞为寇盗。辽西大尹田谭追击之,为所杀。州郡县归咎于句丽侯騊.严尤奏言:“貊人犯法,罪不起于騊,且宜安慰,今猥被之大罪,恐其遂反。”莽不听,诏尤击之。尤诱期句丽侯騊至而斩之,传送其首诣长安。莽大悦,布告天下,更名高句丽为下句丽。当此时为侯国,汉光武帝八年,高句丽王遣使朝贡,始见称王。

至殇、安之间,句丽王宫数寇辽东,更属玄菟。辽东太守蔡风、玄菟太守姚光以宫为二郡害,兴师伐之。宫诈降请和,二郡不进。宫密遣军攻玄菟,焚烧候城,入辽隧,杀吏民。后宫复犯辽东,蔡风轻将吏士追讨之,军败没。宫死,于伯固立。顺、桓之间,复犯辽东,寇新安、居乡,又攻西安平,于道上杀带方令,略得乐浪太守妻子。灵帝建宁二年,玄菟太守耿临讨之,斩首虏数百级,伯固降,属辽东。(嘉)[熹]平中,伯固乞属玄菟。公孙度之雄海东也,伯固遣大加优居、主簿然人等助度击富山贼,破之。伯固死,有二子,长子拔奇,小于伊夷模。拔奇不肖,国人便共立伊夷模为王。自伯固时,数寇辽东,又受亡胡五百余家。建安中,公孙康出军击之,破其国,焚烧邑落。拔奇怒为兄而不得立,与渭奴加各将下户三万余口诣康降,还住沸流水。降胡亦叛伊夷模,伊夷模更作新国,今日所在是也。拔奇遂往辽东,有子留句丽国,今古雏加驳位居是也。其后复击玄菟,玄菟与辽东合击,大破之。

伊夷模无子,淫灌奴部,生子名位宫。伊夷模死,立以为王,今句丽王宫是也。其曾祖名宫,生能开目视,其国人恶之。及长大,果凶虐,数寇抄,国见残破。今王生堕地,亦能开目视人,句丽呼相似为位,似其祖,故名之为位宫。位宫有力勇,便鞍马,善猎射。景初二年,太尉司马宣王率众讨公孙渊,宫遣主簿大加将数千人助军。正始三年,宫寇西安平,其五年,为幽州刺史毋丘俭所破。语在《俭传》。

东沃沮在高句丽盖马大山之东,滨大海而居。其地形东北狭,西南长,可千里,北与挹娄、夫余,南与濊貊接。户五千,无大君王,世世邑落,各有长帅。其言语与句丽大同,时时小异。汉初,燕亡人卫满王朝鲜,时沃沮皆属焉。

汉武帝元封二年,伐朝鲜,杀满孙右渠,分其地为四郡,以沃沮城为玄菟郡。后为夷貊所侵,徙句丽西北,今所谓玄菟故府是也。沃沮还属乐浪。汉以土地广远,在单单大领之东,分置东部都尉,治不耐城,别主领东七县,时沃沮亦皆为县。汉光武六年,省边郡,都尉由此罢。其后皆以其县中渠帅为县侯,不耐,华丽,沃沮诸县皆为侯国。

夷狄更相攻伐,唯不耐濊侯至今犹置功曹、主簿诸曹,皆濊民作之。沃沮诸邑落渠帅,皆自称三老,则故县国之制也。国小,迫于大国之间,遂臣属句丽。句丽复置其中大人为使者,使相主领,又使大加统责其租税,貊布、鱼、盐、海中食物,千里担负致之,又送其美女以为婢妾,遇之如奴仆。其土地肥美,背山向海,宜五谷,善田种。人性质直强勇,少牛马,便持矛步战。食饮居处,衣服礼节,有似句丽。其葬作大木椁,长十余丈,开一头作户。新死者皆假埋之,才使覆形,皮肉尽,乃取骨置椁中。

举家皆共一椁,刻木如生形,随死者为数。又有瓦(钅历),置米其中,编县之于椁户边。

毋丘俭讨句丽,句丽王宫奔沃沮,遂近师击之。沃沮邑落皆破之,斩获首虏三千余级,宫奔北沃沮。北沃沮一名置沟娄,去南沃沮八百余里,其俗南北皆同,与挹娄接。

挹娄喜乘船寇钞,北沃沮畏之,夏月恒在山岩深穴中为守备,冬月冰冻,船道不通,乃下居村落。王颀别遣追讨宫,尽其东界。问其耆老:“海东复有人不?”耆老言国人常乘船捕鱼,遭风见吹数十日,东得一岛,上有人,言语不相晓,其俗常以七月取童女沉海。又言有一国亦在海中,纯女无男。又说得一布衣,从海中浮出,其身如中国人衣,其两袖长三丈。又得一破船,随波在海岸边,有一人项中复有面,生得之,与语不相通,不食而死。其域皆沃沮东大海中。

挹娄在夫余东北千余里,滨大海。南与北沃沮接,未知其北所极。其土地多山险。

其人形似夫余。言语不与夫余、句丽同。有五谷、牛、马、麻布。人多勇力,无大君长,邑落各有大人。处山林之间,常穴居,大家深九梯,以多为好。土气寒,剧于夫余。其俗好养猪,食其肉,衣其皮。冬以猪膏涂身,厚数分,以御风寒。夏则裸袒,以尺布隐其前后,以蔽形体。其人不洁,作溷在中央,人围其表居,其弓长四尺,力如弩,矢用楛,长尺八寸,青石为镞,古之肃慎氏之国也。

善射,射人者皆入因。矢施毒,人中皆死。出赤玉、好貂,今所谓挹娄貂是也。自汉已来,臣属夫余,夫余责其租赋重,以黄初中叛之。夫余数伐之,其人众虽少,所在山险,邻国人畏其弓矢,卒不能服也。其国便乘船寇盗,邻国患之。东夷饮食类皆用俎豆,唯挹娄不,法俗最无纲纪也。

濊南与辰韩,北与高句丽、沃沮接,东穷大海,今朝鲜之东皆其地也。户二万。昔箕子既适朝鲜,作八条之教以教之。无门户之闭而民不为盗。其后四十余世,朝鲜侯淮僭号称王。陈胜等起,天下叛秦,燕、齐、赵民避地朝鲜数万口。燕人卫满,魋结夷服,复来王之。汉武帝伐灭朝鲜,分其地为四郡。自是之后,胡、汉稍别。无大君长,自汉已来,其官有侯邑君、三老,统主下户。其耆老旧自渭与句丽同种。其人性愿悫,少嗜欲,有廉耻,不请句丽。言语法俗大抵与句丽同,衣服有异。男女衣皆着曲领,男子系银花广数寸以为饰。自单单大山领以西属乐浪,自领以东七县,都尉主之,皆以为民。后省都尉,封其渠帅为侯,今不耐濊皆其种也。汉末更属句丽。

其俗重山川,山川各有部分,不得妄想涉入。同姓不婚。多忌讳,疾病死亡辄捐弃旧宅,更作新居。有麻布,蚕桑作绵,晓候星宿,豫知年岁丰约。不以珠玉为宝。常用十月节祭天,昼夜饮酒歌舞,名之为舞天。又祭虎以为神。其邑落相侵犯,辄相罚责生口牛马,名之为责祸。杀人者偿死。少寇盗。作矛长三丈,或数人共持之,能步战。乐浪擅弓出其地。其海出班鱼皮,土地饶文豹,又出果下马,汉桓时献之。

正始六年,乐浪太守刘茂、带方太守弓遵以领东濊属句丽,兴师伐之,不耐侯等举邑降。其八年,诣阙朝贡,诏更拜不耐濊王。居处杂在民间,四时诣郡朝谒。二郡有军征赋调,供给役使,遇之如民。韩在带方之南,东西以海为限,南与倭接。

方可四千里。有三种,一曰马韩,二曰辰韩,三日弁韩。辰韩者,古之辰国也。马韩在西。其民土著,种植,知蚕桑,作绵布。各有长帅,大者自名为臣智,其次为邑借,散在山海间,无城郭。有爰襄国、牟水国、桑外国、小石索国、大石索国、优休牟涿国、臣濆沽国、伯济国、速卢不斯国、日华国、古诞者国、古离国、怒蓝国、月支国、咨离牟卢国、素谓干国、古爰国、莫卢国、卑弥国、占离卑园、臣衅国、支侵国、狗卢国、卑离国、监奚卑离国、古蒲国、致利鞠国、冉路国、儿林国、驷卢国、内卑离国、感奚国、万卢国、辟卑离因、日斯乌旦国、一离国、不弥国、支半国、狗素国、捷卢国、牟卢卑离国、臣苏涂国、莫卢国、古腊国、临素半国、臣云新国、如来卑离国、楚山涂卑离国、一难国、狗奚国、不云国,不斯濆邪国、爰池国、干马国、楚离国、凡五十余国。

大国万余家,小国数千家,总十余万户。辰王治月支国。臣智或加优呼臣云遣支报安邪踧支濆臣离儿不例拘邪秦支廉之号。其官有魏率善、邑君、归义侯、中即将、都尉、伯长。侯准既僭号称王。为燕亡人卫满所攻夺,将其左右宫人走人海。居韩地,自号韩王。其后绝灭,今韩人犹有奉其祭祀者。汉时属乐浪郡,四时朝谒。桓、灵之末,韩濊强盛,郡县不能制,民多流入韩国。

建安中,公孙康分屯有县以南荒地为带方郡,遣公孙模、张敞等收集遗民,兴兵伐韩濊,旧民稍出,是后倭、韩遂属带方。景初中,明帝密遣带方太守刘昕、乐浪太守鲜于嗣越海定二郡,诸韩国臣智加赐邑郡印绶,其次与邑长。其俗好衣帻,下户诣郡朝谒,皆假衣帻,自服印绶衣帻千余有人。部从事吴林以乐浪本统韩国,分割辰韩八国以与乐浪,吏译转有异同,臣智激韩忿,攻带方郡崎离营。时太守弓遵、乐浪太守刘茂兴兵伐之,遵战死,二郡遂灭韩。

其俗少纲纪,国邑虽有主帅,邑落杂居,不能善相制御。无跪拜之礼。居处作草屋土室,形如冢,其户在上,举家共在中,无长幼男女之别。其葬有椁无棺,不知乘牛马,牛马尽于送死。以璎珠为财宝。或以缀衣为饰,或以县颈垂耳,不以金银锦绣为珍。其人性强勇,魁头露紒。如灵兵,衣布袍,足履革(足乔)蹋。其国中有所为及官家使筑城郭,诸年少勇健者,皆凿脊皮,以大绳贯之,又以丈许木锸之,通日欢呼作力,不以为痛,既以劝作,且以为健。常以五月下种讫,祭鬼神,群聚歌舞,饮酒昼夜无休。其舞,数十—人惧起相随,踏地低昂,手足相应,节奏有似铎舞。

十月农功毕,亦复如之。信鬼神,园邑各立一人主祭天神,名之天君。又诸国各有别邑,名之为苏涂。立大木,悬铃鼓,事鬼神。诸亡逃至其中,皆不还之,好作贼。其立苏涂之义,有似浮屠,而所行善恶有异。其北方近郡诸国差晓礼俗,其远处直如囚徒奴婢相聚。无他珍宝。禽兽草木略与中国同。出大栗,大如梨。又出细尾鸡。其尾皆长五尺余。其男子时时有文身。又有州胡在马韩之西海中大岛上,其人差短小,言语不与韩同,皆髡头如鲜卑,但衣韦,好养牛及猪。其衣有上无下,略如裸势。乘船往来,市买韩中。

辰韩在马韩之东,其耆老传世,自言古之亡入避秦役来适韩国,马韩割其东界地与之。有城栅。其言语不与马韩同,名国为邦,弓为弧,贼为寇,行酒为行觞。相呼皆为徒,有似秦人,非但燕、齐之名物也。名乐浪人为阿残;东方人名我为阿,谓乐浪人本其残余人。今有名之为秦韩者。始有六国,稍分为十二国。

弁辰亦十二国,又有诸小别邑,备有渠帅,大者名臣智。其次有险侧,次有樊濊,次有杀奚,次有邑借。有已柢国、不斯国、弁辰弥离弥弥冻国、并辰接涂国、勤耆国、难弥离冻国、弁辰古资弥冻国、弁辰古淳是国、冉奚国、弁辰半路国、弁辰乐奴园、军弥国弁军弥国、弁辰弥乌邪马国、如湛国、弁辰甘路国、户路国、州鲜国、马延国、弁辰狗邪国,弁辰走漕马国、弁辰安邪国、马延国、弁辰渎卢国、斯卢国、优由园、弁、辰韩合二十四国,大国四五千家,小国六七百家,总四五万户。其十二国属辰王。

辰王常用马韩人作之,世世相继。辰王不得自立为王。土地肥美,宜种五谷及稻,晓蚕桑,作缣布,乘驾牛马。嫁娶礼俗,男女有别。以大鸟羽送死,其意欲使死者飞扬。国出铁,韩、濊、倭皆从取之。诸市买皆用铁,如中国用钱,又以供给二郡。俗喜歌舞饮酒。有瑟,其形似筑,弹之亦有音曲。儿生,便以石厌其头,欲其褊。今辰韩人皆褊头。男女近倭,亦文身。便步战,兵仗与马韩同。其俗,行者相逢,皆住让路弁辰与辰韩杂居,亦有城郭。衣服居处与辰韩同。言语法俗相似,祠祭鬼神有异,施灶皆在户西。其渎卢国与倭接界。十二国亦有王,其人形皆大。衣服洁清,长发。亦作广幅细布。法俗特严峻。

倭人在带方东南大海之中,依山岛为国邑。旧百余国,汉时有朝见者,今使译所通三十国。从郡至倭,循海岸水行,历韩国,乍南乍东,到其北岸狗邪韩国,七千余里,始度一海,千余里至对马国。其大官曰卑狗,副曰卑奴母离。所居绝岛,方可四百余里,土地山险,多深林,道路如禽鹿径。有千余户,无良田,食海物自活,乘船南北市籴。

又南渡一海千余里,名曰渤海。至一大国,官亦曰卑狗,副曰卑奴母离。方可三百里,多竹木丛林,有三千许家,差有田地,耕田犹不足食,亦南北市籴。又渡一海,千余里至末卢国,有四千余户,滨山海居,草木茂盛,行不见前人。好捕鱼鳆,水无深浅,皆沉没取之。

东南陆行五百里,到伊都国,官曰尔支,副曰泄漠觚、柄渠觚。有千余户,世有王,皆统属女王国,郡使往来常所驻。东南至奴国百里,官曰兕马觚、副曰卑奴母离,有二万余户。东行至不弥国百里,官曰多模,副曰卑奴母离,有千余家。南至投马国,水行二十日,官曰弥弥,副曰弥弥那利,可五万余户。南至邪马壹国,女王之所都,水行十日,陆行一月。官有伊支马,次曰弥马升,次曰弥马获支,次曰奴佳鞮,可七万余户。

自女王国以北,其户数道里可得略载,其余旁国远绝,不可得详。次有斯马国,次有已百支奴国,次有伊邪国,次有都支国,次有弥奴国,次有好古都国,次有不呼国,次有姐奴国,次有对苏国,次有苏奴国,次有呼邑园,次有华奴苏奴国,次有鬼国,次有为吾国,次有鬼奴国,次有邪马国,次有躬臣国,次有巴利国,次有支惟国,次有乌奴国,次有奴国,此女王境界所尽。其南有狗奴国,男子为王,其官有狗古智卑狗,不属女王。

自郡至女王国万二千余里。

男子无大小皆黥面文身。自古以来,其使诣中国,皆自称大夫。夏后少康之子封于会稽,断发文身以避蛟龙之害,今倭水人好沉没捕鱼蛤,文身亦以厌大鱼水禽,后稍以为饰。诸国文身各异,或左或右,或大或小,尊卑有差。计其道里,当在会稽、东冶之东。其风俗不淫,男子皆露紒,以木绵招头。其衣横幅,但结束相连,略无缝。妇人被发屈紒,作衣如单被,穿其中央,贯头衣之。种禾稻、纻麻、蚕桑、缉绩,出细纻、缣绵。其地无牛马虎豹羊鹊。兵用矛、楯、木弓。木弓短下长上,竹箭或铁镞或骨簇,所有无与儋耳、朱崖同。倭地温暖,冬夏食生莱,皆徒跣。有屋室,父母兄弟卧息异处,以朱丹徐其身体,如中国用粉也。食饮用笾豆,手食。其死,有棺无椁,封土作冢。

始死停丧十余日,当时不食肉,丧主哭泣,他人就歌舞饮酒。已葬,举家诣水中澡浴,以如练沐。其行来渡海诣中国,恒使一人不梳头,不去虮虱,衣服垢污,不食肉,不近妇人,如丧人,名之为持衰。若行者吉善,共顿其生口财物。若有疾病,遭暴害,便欲杀之,谓其持衰不谨。出真珠、青玉。其山有丹,其木有柟、杼、豫樟、杼枥、橿、乌号、枫香,其竹筱竿、桃支。有姜、桔、椒、蓑荷,不知以为滋味。有猕猴、黑雉。

其俗举事行来,有所云为,辄灼骨而卜,以占吉凶,先告所卜,其辞如令龟法,视火坼占兆。其会同坐起,父子男女无别,人性嗜酒。见大人所敬,但搏手以当跪拜。其人寿考,或百年,或八九十年。其俗,国大人皆四五妇,下户或二三妇。妇人不淫,不妨忌。

不盗窃,少诤讼。其犯法,轻者没其妻子,重者灭其门户。及宗族尊卑,备有差序,足相臣服。收租赋。有邸阁。国国有市,交易有无,使大倭监之。自女王国以北,特置一大率,检察诸国,诸国畏惮之。常治伊都国,于国中合如刺史。王遣有诣京都、带方郡、诸韩国,及郡使倭国,皆临津搜露,传送文书赐遣之物诣女王,不得差错。下户与大人相逢道路,逡巡入草。传辞说事,或蹲或跪,两手据地,为之恭敬。对应声曰噫,比如然诺。

其国本亦以男子为王,住七八十年,倭国乱,相攻伐历年,乃共立一女子为王。名曰卑弥呼,事鬼道能惑众,年已长大,无夫婿,有男弟佐治国。自为王以来,少有见者。

以婢千人自侍,唯有男子一人给饮食,传辞出入。居处宫室楼观,城栅严设,常有人持兵守卫。女王国东渡海千余里,复有国,皆倭种。又有侏儒国在其南。人长三四尺,去女王四千余里。又有裸国、黑齿国复在其东南,船行一年可至。参问倭地,绝在海中洲岛之上,或绝或连,阂旋可五千余里。

景初二年六月。倭女王遣大夫难升米等诣郡,求诣天于朝献,太守刘夏遣吏将送诣京都。其年十二月,诏沼书报倭女王曰:“制诏亲魏倭王卑弥呼:带方守刘夏遣使送汝大夫难升米、次使都市牛利奉汝所献男生口四人,女生口六人,班布二匹二丈,以到。

汝所在逾远,乃遣使贡献,是汝之忠孝,我甚哀汝。今以汝为亲魏倭王,假金印紫绶,装封付带方太守假授汝。其绥抚种人,勉为孝顺。汝来使难升米,牛利涉远,道路勒劳,今以难升米为率善中郎将,牛利为率善校尉,假银印青绶,引见劳赐遣还。今以绛地交龙锦五匹、绛地约粟罽十张、蒨绛五十匹、绀青五十匹,答汝所献贡直。又特赐汝绀地句文绵三匹、细班华罽五张、白绢五十匹、金八两、五尺刀二口、铜镜百枚、真珠、铅丹各五十厅。皆将封付难升米、牛利还到录受。悉可以示汝国中人,使知国家哀汝,故郑重赐汝好物也。“

正始元年,太守弓遵遣建中校尉梯俊等奉诏书印绶诣倭国,拜假倭王。并赍诏赐金、帛、锦罽、刀、镜、采物,倭王因使上表答谢恩诏。其四年,倭王复遣使大夫伊声耆、掖邪狗等八人,上献生口、倭锦、绛青缣、绵衣、帛布、丹木、?狗?、短弓矢。掖邪狗等壹拜率善中郎将印绶。其六年,诏赐倭难升米黄幢,付郡假授。其八年,太守王颀到官。倭女王卑弥呼与狗奴国男王卑弥弓呼素不和,遣倭载斯、乌越等诣郡说相攻击状。

遣塞曹掾史张政等因赍诏书、黄幢,拜假难升米为檄告喻之。卑弥呼以死,大作冢,径百余步,徇葬者奴婢百余人。更立男王,国中不服,更相诛杀,当时杀千余人。复立卑弥呼宗女壹与,年十三为王,国中遂定。政等以檄告喻壹与,壹与遣倭大夫率善中郎将掖邪狗等二十人送政等还,因诣台,献上男女生口三十人,贡白殊五千孔,青大句珠二枚,异文杂锦二十匹。

评曰:《史》、《汉》着朝鲜、两越,东京撰录西羌。魏世匈奴遂衰,更有乌丸、鲜卑,爰及东夷,使译时通,记述随事,岂常也哉!

\part{蜀书一}
\chapter{刘二牧传第一}

刘焉字君郎,江夏竟陵人也,汉鲁恭王之后裔,章帝元和中徙封竟陵,支庶家焉。

焉少仕州郡,以宗室拜中郎,后以师祝公丧去官。居阳城山,积学教授,举贤良方正,辟司徒府,历雒阳令、冀州刺史、南阳太守、宗正、太常。焉睹灵帝政治衰缺,王室多故,乃建议言:“刺史、太守,货赂为官,割剥百姓,以致离叛。可选清名重臣以为牧伯,镇安方夏。”焉内求交址牧,欲避世难。议未即行,侍中广汉董扶私谓焉曰:“京师将乱,益州分野有天子气。”焉闻扶言,意更在益州。会益州刺史郤俭赋敛烦扰,谣言远闻,而并州杀刺史张壹,凉州杀刺史耿鄙,焉谋得施。出为监军使者,领益州牧,封阳城侯,当收俭治罪;扶亦求为蜀郡西部属国都尉,及太仓令(会)巴西赵韪去官,俱随焉。

是时益州逆贼马相、赵祗等于绵竹县自号黄巾,合聚疲役之民,一二日中得数千人,先杀绵竹令李升,吏民翕集,合万余人,便前破雒县,攻益州杀俭,又到蜀郡、犍为,旬月之间,破坏三郡。相自称天子,众以万数。州从事贾龙(素)领[家]兵数百人在犍为东界,摄敛吏民,得千余人,攻相等,数日破走,州界清静。龙乃选吏卒迎焉。焉徙治绵竹,抚纳离叛,务行宽惠,阴图异计。张鲁母始以鬼道,又有少容,常往来焉家,故焉遣鲁为督义司马,住汉中,断绝谷阁,杀害汉使。焉上书言米贼断道,不得复通,又托他事杀州中豪强王咸、李权等十余人,以立威刑。犍为太守任岐及贾龙由此反攻焉,焉击杀岐、龙。

焉意渐盛,造作乘舆车具千乘。荆州牧刘表表上焉有似子夏在西河疑圣人之论。时焉子范为左中郎将,诞治书御史,璋为奉车都尉,皆从献帝在长安,惟(小)[叔]子别部司马瑁素随焉。献帝使璋晓谕焉,焉留璋不遣。时征西将军马腾屯郿而反,焉及范与腾通谋,引兵袭长安。范谋泄,奔槐里,腾败,退还凉州,范应时见杀,于是收诞行刑。议郎河南庞羲与焉通家,乃募将焉诸孙入蜀。时焉被天火烧城,车具荡尽,延及民家。焉徙治成都,既痛其子,又感祆灾,兴平元年,痈疽发背而卒。州大吏赵韪等贪璋温仁,共上璋为益州刺史,诏书因以为监军使者,领益州牧,以韪为征东中郎将,率众击刘表。

璋,字季玉,既袭焉位。而张鲁稍骄恣,不承顺璋,璋杀鲁母及弟,遂为仇敌。璋累遣庞羲等攻鲁,[数为]所破。鲁部曲多在巴西,故以羲为巴西太守,领兵御鲁。后羲与璋情好携隙,赵韪称兵内向,众散见杀,皆由璋明断少而外言入故也。璋闻曹公征荆州,已定汉中,遣河内阴溥致敬于曹公。加璋振威将军,兄瑁平寇将军。瑁狂疾物故。璋复遣别驾从事蜀郡张肃送叟兵三百人并杂御物于曹公,曹公拜肃为广汉太守。璋复遣别驾张松诣曹公,曹公时已定荆州,走先主,不复存录松,松以此怨。会曹公军不利赤壁,兼以疫死。松还,疵毁曹公,劝璋自绝,因说璋曰:“刘豫州,使君之肺腑,可与交通。”璋皆然之,遣法正连好先主,寻又令正及孟达送兵数千助先主守御,正遂还。后松复说璋曰:“今州中诸将庞羲、李异等皆恃功骄豪,欲有外意,不得豫州,则敌攻其外,民攻其内,必败之道也。”璋又从之,遣法正请先主。璋主簿黄权陈其利害,从事广汉王累自倒县于州门以谏,璋一无所纳,敕在所供奉先主,先主入境如归。先主至江州,北由垫江水诣涪。去成都三百六十里,是岁建安十六年也。璋率步骑三万余人,车乘帐幔,精光耀目,往就与会先主所将将士。更相之适,欢饮百余日。璋资给先主,使讨张鲁,然后分别。

明年,先主至葭萌,还兵南向,所在皆克。十九年,进围成都数十日,城中尚有精兵三万人,谷帛支一年,吏民咸欲死战。璋言:“父子在州二十余年,无恩德以加百姓。百姓攻战三年,肌膏草野者,以璋故也,何心能安!”遂开城出降,群下莫不流涕。先主迁璋于南郡公安,尽归其财物及故佩振威将军印绶。孙权杀关羽,取荆州,以璋为益州牧,驻秭归。璋卒,南中豪率雍闿据益郡反,附于吴。权复以璋子阐为益州刺史,处交、益界首。丞相诸葛亮平南土,阐还吴,为御史中丞。初,璋长子循妻,庞羲女也。先主定蜀,羲为左将军司马,璋时从羲启留循,先主以为奉车中郎将。是以璋二子之后,分在吴、蜀。

评曰:昔魏豹闻许负之言则纳薄姬于室,刘歆见图谶之文则名字改易,终于不免其身,而庆钟二主。此则神明不可虚要,天命不可妄冀,必然之验也。而刘焉闻董扶之辞则心存益土,听相者之言则求婚吴氏,遽造舆服,图窃神器,其惑甚矣。璋才非人雄,而据土乱世,负乘致寇,自然之理,其见夺取,非不幸也。

\part{蜀书二}

\chapter{先主传第二}

\begin{yuanwen}
先主姓刘,讳备,字玄德,涿郡涿县人,汉景帝子中山靖王胜之后也。胜子贞,元狩六年封涿县陆城亭侯。坐酎金失侯,因家焉。先主祖雄,父弘,世仕州郡。雄举孝廉,官至东郡范令。
\end{yuanwen}

\begin{yuanwen}
先主少孤,与母贩履织席为业。舍东南角篱上有桑树生高五丈馀,遥望见童童如小车盖,往来者皆怪此树非凡,或谓当出贵人。先主少时,与宗中诸小儿於树下戏,言:“吾必当乘此羽葆盖车。”

叔父子敬谓曰:“汝勿妄语,灭吾门也!”
\end{yuanwen}

\begin{yuanwen}
年十五,母使行学,与同宗刘德然、辽西公孙瓒俱事故九江太守同郡卢植。德然父元起常资给先主,与德然等。元起妻曰:“各自一家,何能常尔邪!”

起曰:“吾宗中有此儿,非常人也。”

而瓒深与先主相友。瓒年长,先主以兄事之。先主不甚乐读书,喜狗马、音乐、美衣服。身长七尺五寸,垂手下膝,顾自见其耳。少语言,善下人,喜怒不形於色。好交结豪侠,年少争附之。中山大商张世平、苏双等赀\footnote{z\=i,资助。}累千金,贩马周旋於涿郡,见而异之,乃多与之金财。先主由是得用合徒众。
\end{yuanwen}

\begin{yuanwen}
灵帝末,黄巾起,州郡各举义兵,先主率其属从校尉邹靖讨黄巾贼有功,除安喜尉。督邮以公事到县,先主求谒,不通,直入缚督邮,杖二百,解绶系其颈着马枊,五葬反。弃官亡命。顷之,大将军何进遣都尉毌丘毅诣丹杨募兵,先主与俱行,至下邳遇贼,力战有功,除为下密丞。复去官。后为高唐尉,迁为令。为贼所破,往奔中郎将公孙瓒,瓒表为别部司马,使与青州刺史田楷以拒冀州牧袁绍。数有战功,试守平原令,后领平原相。郡民刘平素轻先主,耻为之下,使客刺之。客不忍刺,语之而去。其得人心如此。
\end{yuanwen}

\begin{yuanwen}
袁绍攻公孙瓒,先主与田楷东屯齐。曹公征徐州,徐州牧陶谦遣使告急於田楷,楷与先主俱救之。时先主自有兵千馀人及幽州乌丸杂胡骑,又略得饥民数千人。

既到,谦以丹杨兵四千益先主,先主遂去楷归谦。谦表先主为豫州刺史,屯小沛。谦病笃,谓别驾麋竺曰:“非刘备不能安此州也。”
\end{yuanwen}

\begin{yuanwen}
谦死,竺率州人迎先主,先主未敢当。下邳陈登谓先主曰:“今汉室陵迟,海内倾覆,立功立事,在於今日。彼州殷富,户口百万,欲屈使君抚临州事。”

先主曰:“袁公路近在寿春,此君四世五公,海内所归,君可以州与之。”

登曰:“公路骄豪,非治乱之主。今欲为使君合步骑十万,上可以匡主济民,成五霸之业,下可以割地守境,书功於竹帛。若使君不见听许,登亦未敢听使君也。”

北海相孔融谓先主曰:“袁公路岂忧国忘家者邪?冢中枯骨,何足介意。今日之事,百姓与能,天与不取,悔不可追。”

先主遂领徐州。
\end{yuanwen}

\begin{yuanwen}
袁术来攻先主,先主拒之於盱眙\footnote{x\=u y\'i}、淮阴(阳)。曹公表先主为镇东将军,封宜城亭侯,是岁建安元年也。

先主与术相持经月,吕布乘虚袭下邳。下邳守将曹豹反,间迎布。布虏先主妻子,先主转军海西。杨奉、韩暹寇徐、扬间,先主邀击,尽斩之。先主求和於吕布,布还其妻子。先主遣关羽守下邳。
\end{yuanwen}

\begin{yuanwen}
先主还小沛,复合兵得万馀人。吕布恶之,自出兵攻先主,先主败走归曹公。曹公厚遇之,以为豫州牧。将至沛收散卒,给其军粮,益与兵使东击布。布遣高顺攻之,曹公遣夏侯惇往,不能救,为顺所败,复虏先主妻子送布。曹公自出东征,助先主围布於下邳,生禽布。先主复得妻子,从曹公还许。表先主为左将军,礼之愈重,出则同舆,坐则同席。袁术欲经徐州北就袁绍,曹公遣先主督朱灵、路招要击术。未至,术病死。
\end{yuanwen}

\begin{yuanwen}
先主未出时,献帝舅车骑将军董承辞受帝衣带中密诏,当诛曹公。先主未发。是时曹公从容谓先主曰:“今天下英雄,唯使君与操耳。本初之徒,不足数也。”

先主方食,失匕箸。遂与承及长水校尉种辑、将军吴子兰、王子服等同谋。会见使,未发。事觉,承等皆伏诛。
\end{yuanwen}

\begin{yuanwen}
先主据下邳。灵等还,先主乃杀徐州刺史车胄,留关羽守下邳,而身还小沛。东海昌霸反,郡县多叛曹公为先主,众数万人,遣孙乾与袁绍连和,曹公遣刘岱、王忠击之,不克。

五年,曹公东征先主,先主败绩。曹公尽收其众,虏先主妻子,并禽关羽以归。
\end{yuanwen}

\begin{yuanwen}
先主走青州。青州刺史袁谭,先主故茂才也,将步骑迎先主。先主随谭到平原,谭驰使白绍。绍遣将道路奉迎,身去邺二百里,与先主相见。驻月馀日,所失亡士卒稍稍来集。曹公与袁绍相拒於官渡,汝南黄巾刘辟等叛曹公应绍。绍遣先主将兵与辟等略许下。关羽亡归先主。曹公遣曹仁将兵击先主,先主还绍军,阴欲离绍,乃说绍南连荆州牧刘表。绍遣先主将本兵复至汝南,与贼龚都等合,众数千人。曹公遣蔡阳击之,为先主所杀。
\end{yuanwen}

\begin{yuanwen}
曹公既破绍,自南击先主。先主遣麋竺、孙乾与刘表相闻,表自郊迎,以上宾礼待之,益其兵,使屯新野。荆州豪杰归先主者日益多,表疑其心,阴御之。使拒夏侯惇、于禁等於博望。久之,先主设伏兵,一旦自烧屯伪遁,惇等追之,为伏兵所破。
\end{yuanwen}

\begin{yuanwen}
十二年,曹公北征乌丸,先主说表袭许,表不能用。曹公南征表,会表卒,子琮代立,遣使请降。先主屯樊,不知曹公卒至,至宛乃闻之,遂将其众去。过襄阳,诸葛亮说先主攻琮,荆州可有。先主曰:“吾不忍也。”乃驻马呼琮,琮惧不能起。琮左右及荆州人多归先主。

此到当阳,众十馀万,辎重数千两,日行十馀里,别遣关羽乘船数百艘,使会江陵。或谓先主曰:“宜速行保江陵,今虽拥大众,被甲者少,若曹公兵至,何以拒之?”

先主曰:“夫济大事必以人为本,今人归吾,吾何忍弃去!”
\end{yuanwen}

\begin{yuanwen}
曹公以江陵有军实,恐先主据之,乃释辎重,轻军到襄阳。闻先主已过,曹公将精骑五千急追之,一日一夜行三百馀里,及於当阳之长坂。先主弃妻子,与诸葛亮、张飞、赵云等数十骑走,曹公大获其人众辎重。先主斜趋汉津,適与羽船会,得济沔,遇表长子江夏太守琦众万馀人,与俱到夏口。先主遣诸葛亮自结於孙权,权遣周瑜、程普等水军数万,与先主并力,与曹公战於赤壁,大破之,焚其舟船。先主与吴军水陆并进,追到南郡,时又疾疫,北军多死,曹公引归。
\end{yuanwen}

\begin{yuanwen}
先主表琦为荆州刺史,又南征四郡。武陵太守金旋、长沙太守韩玄、桂阳太守赵范、零陵太守刘度皆降。庐江雷绪率部曲数万口稽颡。琦病死,群下推先主为荆州牧,治公安。权稍畏之,进妹固好。先主至京见权,绸缪恩纪。权遣使云欲共取蜀,或以为宜报听许,吴终不能越荆有蜀,蜀地可为己有。荆州主簿殷观进曰:“若为吴先驱,进未能克蜀,退为吴所乘,即事去矣。今但可然赞其伐蜀,而自说新据诸郡,未可兴动,吴必不敢越我而独取蜀。如此进退之计,可以收吴、蜀之利。”

先主从之,权果辍计。迁观为别驾从事。
\end{yuanwen}

\begin{yuanwen}
十六年,益州牧刘璋遥闻曹公将遣锺繇等向汉中讨张鲁,内怀恐惧。别驾从事蜀郡张松说璋曰:“曹公兵强无敌於天下,若因张鲁之资以取蜀土,谁能御之者乎?”

璋曰:“吾固忧之而未有计。”

松曰:“刘豫州,使君之宗室而曹公之深雠也,善用兵,若使之讨鲁,鲁必破。鲁破,则益州强,曹公虽来,无能为也。”

璋然之,遣法正将四千人迎先主,前后赂遗以巨亿计。正因陈益州可取之策。先主留诸葛亮、关羽等据荆州,将步卒数万人入益州。至涪,璋自出迎,相见甚欢。张松令法正白先主,及谋臣庞统进说,便可於会所袭璋。

先主曰:“此大事也,不可仓卒。”

璋推先主行大司马,领司隶校尉;先主亦推璋行镇西大将军,领益州牧。璋增先主兵,使击张鲁,又令督白水军。先主并军三万馀人,车甲器械资货甚盛。

是岁,璋还成都。先主北到葭萌,未即讨鲁,厚树恩德,以收众心。
\end{yuanwen}

\begin{yuanwen}
明年,曹公征孙权,权呼先主自救。先主遣使告璋曰:“曹公征吴,吴忧危急。孙氏与孤本为唇齿,又乐进在青泥与关羽相拒,今不往救羽,进必大克,转侵州界,其忧有甚於鲁。鲁自守之贼,不足虑也。”

乃从璋求万兵及资实,欲以东行。璋但许兵四千,其馀皆给半。张松书与先主及法正曰:“今大事垂可立,如何释此去乎!”

松兄广汉太守肃,惧祸逮己,白璋发其谋。於是璋收斩松,嫌隙始构矣。璋敕关戍诸将文书勿复关通先主。先主大怒,召璋白水军督杨怀,责以无礼,斩之。乃使黄忠、卓膺勒兵向璋。先主径至关中,质诸将并士卒妻子,引兵与忠、膺等进到涪,据其城。璋遣刘璝、冷苞、张任、邓贤等拒先主於涪,皆破败,退保绵竹。璋复遣李严督绵竹诸军,严率众降先主。先主军益强,分遣诸将平下属县,诸葛亮、张飞、赵云等将兵泝流定白帝、江州、江阳,惟关羽留镇荆州。先主进军围雒;时璋子循守城,被攻且一年。
\end{yuanwen}

十九年夏,雒城破,进围成都数十日,璋出降。蜀中殷盛丰乐,先主置酒大飨士卒,取蜀城中金银分赐将士,还其谷帛。先主复领益州牧,诸葛亮为股肱,法正为谋主,关羽、张飞、马超为爪牙,许靖、麋竺、简雍为宾友。及董和、黄权、李严等本璋之所授用也,吴壹、费观等又璋之婚亲也,彭羕又璋之所排摈也,刘巴者宿昔之所忌恨也,皆处之显任,尽其器能。有志之士,无不竞劝。

二十年,孙权以先主已得益州,使使报欲得荆州。先主言:“须得凉州,当以荆州相与。”权忿之,乃遣吕蒙袭夺长沙、零陵、桂阳三郡。先主引兵五万下公安,令关羽入益阳。是岁,曹公定汉中,张鲁遁走巴西。先主闻之,与权连和,分荆州、江夏、长沙、桂阳东属,南郡、零陵、武陵西属,引军还江州。遣黄权将兵迎张鲁,张鲁已降曹公。曹公使夏侯渊、张郃屯汉中,数数犯暴巴界。先主令张飞进兵宕渠,与郃等战於瓦口,破郃等,郃收兵还南郑。先主亦还成都。

二十三年,先主率诸将进兵汉中。分遣将军吴兰、雷铜等入武都,皆为曹公军所没。先主次于阳平关,与渊、郃等相拒。

\begin{yuanwen}
二十四年春,自阳平南渡沔水,缘山稍前,於定军兴势作营。渊将兵来争其地。先主命黄忠乘高鼓噪攻之,大破渊军,斩渊及曹公所署益州刺史赵颙等。曹公自长安举众南征。先主遥策之曰:“曹公虽来,无能为也,我必有汉川矣。”

及曹公至,先主敛众拒险,终不交锋,积月不拔,亡者日多。

夏,曹公果引军还,先主遂有汉中。遣刘封、孟达、李平等攻申耽於上庸。
\end{yuanwen}

秋,群下上先主为汉中王,表於汉帝曰:“平西将军都亭侯臣马超、左将军长史领镇军将军臣许靖、营司马臣庞羲、议曹从事中郎军议中郎将臣射援、军师将军臣诸葛亮、荡寇将军汉寿亭侯臣关羽、征虏将军新亭侯臣张飞、征西将军臣黄忠、镇远将军臣赖恭、扬武将军臣法正、兴业将军臣李严等一百二十人上言曰:昔唐尧至圣而四凶在朝,周成仁贤而四国作难,高后称制而诸吕窃命,孝昭幼冲而上官逆谋,皆冯世宠,藉履国权,穷凶极乱,社稷几危。非大舜、周公、朱虚、博陆,则不能流放禽讨,安危定倾。伏惟陛下诞姿圣德,统理万邦,而遭厄运不造之艰。董卓首难,荡覆京畿,曹操阶祸,窃执天衡;皇后太子,鸩杀见害,剥乱天下,残毁民物。久令陛下蒙尘忧厄,幽处虚邑。人神无主,遏绝王命,厌昧皇极,欲盗神器。左将军领司隶校尉豫、荆、益三州牧宜城亭侯备,受朝爵秩,念在输力,以殉国难。睹其机兆,赫然愤发,与车骑将军董承同谋诛操,将安国家,克宁旧都。会承机事不密,令操游魂得遂长恶,残泯海内。臣等每惧王室大有阎乐之祸,小有定安之变,夙夜惴惴,战栗累息。昔在虞书,敦序九族,周监二代,封建同姓,诗著其义,历载长久。汉兴之初,割裂疆土,尊王子弟,是以卒折诸吕之难,而成太宗之基。臣等以备肺腑枝叶,宗子藩翰,心存国家,念在弭乱。自操破於汉中,海内英雄望风蚁附,而爵号不显,九锡未加,非所以镇卫社稷,光昭万世也。奉辞在外,礼命断绝。昔河西太守梁统等值汉中兴,限於山河,位同权均,不能相率,咸推窦融以为元帅,卒立效绩,摧破隗嚣。今社稷之难,急於陇、蜀。操外吞天下,内残群寮,朝廷有萧墙之危,而御侮未建,可为寒心。臣等辄依旧典,封备汉中王,拜大司马,董齐六军,纠合同盟,扫灭凶逆。以汉中、巴、蜀、广汉、犍为为国,所署置依汉初诸侯王故典。夫权宜之制,苟利社稷,专之可也。然后功成事立,臣等退伏矫罪,虽死无恨。”遂於沔阳设坛场,陈兵列众,群臣陪位,读奏讫,御王冠於先主。

先主上言汉帝曰:“臣以具臣之才,荷上将之任,董督三军,奉辞於外,不得扫除寇难,靖匡王室,久使陛下圣教陵迟,六合之内,否而未泰,惟忧反侧,疢如疾首。曩者董卓造为乱阶,自是之后,群凶纵横,残剥海内。赖陛下圣德威灵,人神同应,或忠义奋讨,或上天降罚,暴逆并殪,以渐冰消。惟独曹操,久未枭除,侵擅国权,恣心极乱。臣昔与车骑将军董承图谋讨操,机事不密,承见陷害,臣播越失据,忠义不果。遂得使操穷凶极逆,主后戮杀,皇子鸩害。虽纠合同盟,念在奋力,懦弱不武,历年未效。常恐殒没,孤负国恩,寤寐永叹,夕惕若厉。今臣群寮以为在昔虞书敦叙九族,庶明励翼,五帝损益,此道不废。周监二代,并建诸姬,实赖晋、郑夹辅之福。高祖龙兴,尊王子弟,大启九国,卒斩诸吕,以安大宗。今操恶直丑正,寔繁有徒,包藏祸心,篡盗已显。既宗室微弱,帝族无位,斟酌古式,依假权宜,上臣大司马汉中王。臣伏自三省,受国厚恩,荷任一方,陈力未效,所获已过,不宜复忝高位以重罪谤。群寮见逼,迫臣以义。臣退惟寇贼不枭,国难未已,宗庙倾危,社稷将坠,成臣忧责碎首之负。若应权通变,以宁靖圣朝,虽赴水火,所不得辞,敢虑常宜,以防后悔。辄顺众议,拜受印玺,以崇国威。仰惟爵号,位高宠厚,俯思报效,忧深责重,惊怖累息,如临于谷。尽力输诚,奖厉六师,率齐群义,应天顺时,扑讨凶逆,以宁社稷,以报万分,谨拜章因驿上还所假左将军、宜城亭侯印绶。”於是还治成都。拔魏延为都督,镇汉中。时关羽攻曹公将曹仁,禽于禁於樊。俄而孙权袭杀羽,取荆州。

\begin{yuanwen}
二十五年,魏文帝称尊号,改年曰黄初。或传闻汉帝见害,先主乃发丧制服,追谥曰孝愍皇帝。
\end{yuanwen}

是后在所并言众瑞,日月相属,故议郎阳泉侯刘豹、青衣侯向举、偏将军张裔、黄权、大司马属殷纯、益州别驾从事赵莋、治中从事杨洪、从事祭酒何宗、议曹从事杜琼、劝学从事张爽、尹默、谯周等上言:“臣闻河图、洛书,五经谶、纬,孔子所甄,验应自远。谨案洛书甄曜度曰:'赤三日德昌,九世会备,合为帝际。'洛书宝号命曰:'天度帝道备称皇,以统握契,百成不败。'洛书录运期曰:'九侯七杰争命民炊骸,道路籍籍履人头,谁使主者玄且来。'孝经钩命决录曰:'帝三建九会备。'臣父群未亡时,言西南数有黄气,直立数丈,见来积年,时时有景云祥风,从璿玑下来应之,此为异瑞。又二十二年中,数有气如旗,从西竟东,中天而行,图、书曰'必有天子出其方'。加是年太白、荧惑、填星,常从岁星相追。近汉初兴,五星从岁星谋;岁星主义,汉位在西,义之上方,故汉法常以岁星候人主。当有圣主起於此州,以致中兴。时许帝尚存,故群下不敢漏言。顷者荧惑复追岁星,见在胃昴毕;昴毕为天纲,经曰'帝星处之,众邪消亡'。圣讳豫睹,推揆期验,符合数至,若此非一。臣闻圣王先天而天不违,后天而奉天时,故应际而生,与神合契。愿大王应天顺民,速即洪业,以宁海内。”

太傅许靖、安汉将军糜竺、军师将军诸葛亮、太常赖恭、光禄勋黄柱、少府王谋等上言:“曹丕篡弑,湮灭汉室,窃据神器,劫迫忠良,酷烈无道。人鬼忿毒,咸思刘氏。今上无天子,海内惶惶,靡所式仰。群下前后上书者八百馀人,咸称述符瑞,图、谶明徵。间黄龙见武阳赤水,九日乃去。孝经援神契曰'德至渊泉则黄龙见',龙者,君之象也。易乾九五'飞龙在天',大王当龙升,登帝位也。又前关羽围樊、襄阳,襄阳男子张嘉、王休献玉玺,玺潜汉水,伏於渊泉,晖景烛耀,灵光彻天。夫汉者,高祖本所起定天下之国号也,大王袭先帝轨迹,亦兴於汉中也。今天子玉玺神光先见,玺出襄阳,汉水之末,明大王承其下流,授与大王以天子之位,瑞命符应,非人力所致。昔周有乌鱼之瑞,咸曰休哉。二祖受命,图、书先著,以为徵验。今上天告祥,群儒英俊,并进河、洛,孔子谶、记,咸悉具至。伏惟大王出自孝景皇帝中山靖王之胄,本支百世,乾祇降祚,圣姿硕茂,神武在躬,仁覆积德,爱人好士,是以四方归心焉。考省灵图,启发谶、纬,神明之表,名讳昭著。宜即帝位,以纂二祖,绍嗣昭穆,天下幸甚。臣等谨与博士许慈、议郎孟光,建立礼仪,择令辰,上尊号。“即皇帝位於成都武担之南。为文曰:“惟建安二十六年四月丙午,皇帝备敢用玄牡,昭告皇天上帝后土神祇:汉有天下,历数无疆。曩者王莽篡盗,光武皇帝震怒致诛,社稷复存。今曹操阻兵安忍,戮杀主后,滔天泯夏,罔顾天显。操子丕,载其凶逆,窃居神器。群臣将士以为社稷堕废,备宜脩之,嗣武二祖,龚行天罚。备惟否德,惧忝帝位。询于庶民,外及蛮夷君长,佥曰'天命不可以不答,祖业不可以久替,四海不可以无主'。率土式望,在备一人。备畏天明命,又惧汉阼将湮于地,谨择元日,与百寮登坛,受皇帝玺绶。脩燔瘗,告类于天神,惟神飨祚于汉家,永绥四海!”

\begin{yuanwen}
章武元年夏四月,大赦,改年。以诸葛亮为丞相,许靖为司徒。置百官,立宗庙,祫祭高皇帝以下。
\end{yuanwen}

\begin{yuanwen}
五月,立皇后吴氏,子禅为皇太子。

六月,以子永为鲁王,理为梁王。车骑将军张飞为其左右所害。

初,先主忿孙权之袭关羽,将东征,秋七月,遂帅诸军伐吴。孙权遣书请和,先主盛怒不许,吴将陆议、李异、刘阿等屯巫、秭归;将军吴班、冯习自巫攻破异等,军次秭归,武陵五谿蛮夷遣使请兵。
\end{yuanwen}

\begin{yuanwen}
二年春正月,先主军还秭归,将军吴班、陈式水军屯夷陵,夹江东西岸。

二月,先主自秭归率诸将进军,缘山截岭,於夷道猇\footnote{text}亭驻营,自佷山通武陵,遣侍中马良安慰五谿蛮夷,咸相率响应。镇北将军黄权督江北诸军,与吴军相拒於夷陵道。

夏六月,黄气见自秭归十馀里中,广数十丈。后十馀日,陆议大破先主军於猇亭,将军冯习、张南等皆没。先主自猇亭还秭归,收合离散兵,遂弃船舫,由步道还鱼复,改鱼复县曰永安。吴遣将军李异、刘阿等踵蹑先主军,屯驻南山。

秋八月,收兵还巫。司徒许靖卒。

冬十月,诏丞相亮营南北郊於成都。孙权闻先主住白帝,甚惧,遣使请和。先主许之,遣太中大夫宗玮报命。

冬十二月,汉嘉太守黄元闻先主疾不豫,举兵拒守。
\end{yuanwen}

\begin{yuanwen}
三年春二月,丞相亮自成都到永安。

三月,黄元进兵攻临邛县。遣将军陈曶讨元,元军败,顺流下江,为其亲兵所缚,生致成都,斩之。先主病笃,讬孤於丞相亮,尚书令李严为副。

夏四月癸巳,先主殂于永安宫,时年六十三。
\end{yuanwen}

亮上言於后主曰:“伏惟大行皇帝迈仁树德,覆焘无疆,昊天不吊,寝疾弥留,今月二十四日奄忽升遐,臣妾号咷,若丧考妣。乃顾遗诏,事惟大宗,动容损益;百寮发哀,满三日除服,到葬期复如礼;其郡国太守、相、都尉、县令长,三日便除服。臣亮亲受敕戒,震畏神灵,不敢有违。臣请宣下奉行。”

\begin{yuanwen}
五月,梓宫自永安还成都,谥曰昭烈皇帝。

秋,八月,葬惠陵。
\end{yuanwen}

\begin{yuanwen}

\end{yuanwen}

评曰:先主之弘毅宽厚,知人待士,盖有高祖之风,英雄之器焉。及其举国讬孤於诸葛亮,而心神无贰,诚君臣之至公,古今之盛轨也。机权幹略,不逮魏武,是以基宇亦狭。然折而不挠,终不为下者,抑揆彼之量必不容己,非唯竞利,且以避害云尔。

\part{蜀书三}
\chapter{后主传第三}

后主讳禅,字公嗣,先主子也。建安二十四年,先主为汉中王,立为王太子。及即尊号,册曰:“惟章武元年五月辛巳,皇帝若曰:太子禅,朕遭汉运艰难,贼臣篡盗,社稷无主,格人群正,以天明命,朕继大统。今以禅为皇太子,以承宗庙,祗肃社稷。使使持节丞相亮授印缓,敬听师傅,行一物而三善皆得焉,可不勉与!”三年夏四月,先主殂于永安宫。五月,后主袭位於成都,时年十七。尊皇后曰皇太后。大赦,改元。是岁魏黄初四年也。

建兴元年夏,牂牁太守朱褒拥郡反。先是,益州郡有大姓雍闿反,流太守张裔於吴,据郡不宾,越隽夷王高定亦背叛。是岁,立皇后张氏。遣尚书郎邓芝固好於吴,吴王孙权与蜀和亲使聘,是岁通好。

二年春,务农殖谷,闭关息民。

三年春三月,丞相亮南征四郡,四郡皆平。改益州郡为建宁郡,分建宁、永昌郡为云南郡,又分建宁、牂牁为兴古郡。十二月,亮还成都。

四年春,都护李严自永安还住江州,筑大城。

五年春,丞相亮出屯汉中,营沔北阳平石马。

六年春,亮出攻祁山,不克。冬,复出散关,围陈仓,粮尽退。魏将王双率军追亮,亮与战,破之,斩双,还汉中。

七年春,亮遣陈式攻武都、阴平,遂克定二郡。冬,亮徙府营於南山下原上,筑汉、乐二城。是岁,孙权称帝,与蜀约盟,共交分天下。

八年秋,魏使司马懿由西城,张郃由子午,曹真由斜谷,欲攻汉中。丞相亮待之於城固、赤阪,大雨道绝,真等皆还。是岁,魏延破魏雍州刺史郭淮于阳谿。徙鲁王永为甘陵王。梁王理为安平王,皆以鲁、梁在吴分界故也。

九年春二月,亮复出军围祁山,始以木牛运。魏司马懿、张郃救祁山。夏六月,亮粮尽退军,郃追至青封,与亮交战,被箭死。秋八月,都护李平废徙梓潼郡。

十年,亮休士劝农於黄沙,作流马木牛毕,教兵讲武。

十一年冬,亮使诸军运米,集於斜谷口,治斜谷邸阁。是岁,南夷刘胄反,将军马忠破平之。

十二年春二月,亮由斜谷出,始以流马运。秋八月,亮卒于渭滨。征西大将军魏延与丞相长史杨仪争权不和,举兵相攻,延败走;斩延首,仪率诸军还成都。大赦。以左将军吴壹为车骑将军,假节督汉中。以丞相留府长史蒋琬为尚书令,总统国事。

十三年春正月,中军师杨仪废徙汉嘉郡。夏四月,进蒋琬位为大将军。

十四年夏四月,后主至湔,登观阪,看汶水之流,旬日还成都。徙武都氐王苻健及氐民四百馀户於广都。

十五年夏六月,皇后张氏薨。

延熙元年春正月,立皇后张氏。大赦,改元。立子璿为太子,子瑶为安定王。冬十一月,大将军蒋琬出屯汉中。

二年春三月,进蒋琬位为大司马。

三年春,使越隽太守张嶷平定越隽郡。

四年冬十月,尚书令费祎至汉中,与蒋琬谘论事计,岁尽还。

五年春正月,监军姜维督偏军,自汉中还屯涪县。

六年冬十月,大司马蒋琬自汉中还,住涪。十一月,大赦。以尚书令费祎为大将军。

七年闰月,魏大将军曹爽、夏侯玄等向汉中,镇北大将军王平拒兴势围,大将军费祎督诸军往赴救,魏军退。夏四月,安平王理卒。秋九月,祎还成都。

八年秋八月,皇太后薨。十二月,大将军费祎至汉中,行围守。

九年夏六月,费祎还成都。秋,大赦。冬十一月,大司马蒋琬卒。

十年,凉州胡王白虎文、治无戴等率众降,卫将军姜维迎逆安抚,居之于繁县。是岁,汶山平康夷反,维往讨,破平之。

十一年夏五月,大将军费祎出屯汉中。秋,涪陵属国民夷反,车骑将军邓芝往讨,皆破平之。

十二年春正月,魏诛大将军曹爽等,右将军夏侯霸来降。夏四月,大赦。秋,卫将军姜维出攻雍州,不克而还。将军句安、李韶降魏。

十三年,姜维复出西平,不克而还。

十四年夏,大将军费祎还成都。冬,复北驻汉寿。大赦。

十五年,吴王孙权薨。立子琮为西河王。

十六年春正月,大将军费祎为魏降人郭脩所杀于汉寿。夏四月,卫将军姜维复率众围南安,不克而还。

十七年春正月,姜维还成都。大赦。夏六月,维复率众出陇西。冬,拔狄道、河关、临洮县民,居于绵竹、繁县。

十八年春,姜维还成都。夏,复率诸军出狄道,与魏雍州刺史王经战于洮西,大破之。经退保狄道城,维卻住锺题。

十九年春,进姜维位为大将军,督戎马,与镇西将军胡济期会上邽,济失誓不至。秋八月,维为魏大将军邓艾所破于上邽。维退军还成都。是岁,立子瓒为新平王。大赦。

二十年,闻魏大将军诸葛诞据寿春以叛,姜维复率众出骆谷,至芒水。是岁大赦。

景耀元年,姜维还成都。史官言景星见,於是大赦,改年。宦人黄皓始专政。吴大将军孙綝废其主亮,立琅邪王休。

二年夏六月,立子谌为北地王,恂为新兴王,虔为上党王。

三年秋九月,追谥故将军关羽、张飞、马超、庞统、黄忠。

四年春三月,追谥故将军赵云。冬十月,大赦。

五年春正月,西河王琮卒。是岁,姜维复率众出侯和,为邓艾所破,还住沓中。

六年夏,魏大兴徒众,命征西将军邓艾、镇西将军锺会、雍州刺史诸葛绪数道并攻。於是遣左右车骑将军张翼、廖化、辅国大将军董厥等拒之。大赦。改元为炎兴。冬,邓艾破卫将军诸葛瞻於绵竹。用光禄大夫谯周策,降於艾,奉书曰:"限分江、汉,遇值深远,阶缘蜀土,斗绝一隅,干运犯冒,渐苒历载,遂与京畿攸隔万里。每惟黄初中,文皇帝命虎牙将军鲜于辅,宣温密之诏,申三好之恩,开示门户,大义炳然,而否德暗弱,窃贪遗绪,俯仰累纪,未率大教。天威既震,人鬼归能之数,怖骇王师,神武所次,敢不革面,顺以从命!辄敕群帅投戈释甲,官府帑藏一无所毁。百姓布野,馀粮栖亩,以俟后来之惠,全元元之命。伏惟大魏布德施化,宰辅伊、周,含覆藏疾。谨遣私署侍中张绍、光禄大夫谯周、驸马都尉邓良奉赍印缓,请命告诚,敬输忠款,存亡敕赐,惟所裁之。舆榇在近,不复缕陈。"是日,北地王谌伤国之亡,先杀妻子,次以自杀。绍、良与艾相遇於雒县。艾得书,大喜,即报书,遣绍、良先还。艾至城北,后主舆榇自缚,诣军垒门。艾解缚焚榇,延请相见。因承制拜后主为骠骑将军。诸围守悉被后主敕,然后降下。艾使后主止其故宫,身往造焉。资严未发,明年春正月,艾见收。锺会自涪至成都作乱。会既死,蜀中军众钞略,死丧狼籍,数日乃安集。

后主举家东迁,既至洛阳,策命之曰:"惟景元五年三月丁亥。皇帝临轩,使太常嘉命刘禅为安乐县公。於戏,其进听朕命!盖统天载物,以咸宁为大,光宅天下,以时雍为盛。故孕育群生者,君人之道也,乃顺承天者,坤元之义也。上下交畅,然后万物协和,庶类获乂。乃者汉氏失统,六合震扰。我太祖承运龙兴,弘济八极,是用应天顺民,抚有区夏。于时乃考因群杰虎争,九服不静,乘间阻远,保据庸蜀,遂使西隅殊封,方外壅隔。自是以来,干戈不戢,元元之民,不得保安其性,几将五纪。朕永惟祖考遗志,思在绥缉四海,率土同轨,故爰整六师,耀威梁、益。公恢崇德度,深秉大正,不惮屈身委质,以爱民全国为贵,降心回虑,应机豹变,履言思顺,以享左右无疆之休,岂不远欤!朕嘉与君公长飨显禄,用考咨前训,开国胙土,率遵旧典,钖兹玄牡,苴以白茅,永为魏藩辅,往钦哉!公其祗服朕命,克广德心,以终乃显烈。"食邑万户,赐绢万匹,奴婢百人,他物称是。子孙为三都尉封侯者五十馀人。尚书令樊建、侍中张绍、光禄大夫谯周、秘书令郤正、殿中督张通并封列侯。公泰始七年薨於洛阳。

评曰:后主任贤相则为循理之君,惑阉竖则为昬闇之后,传曰“素丝无常,唯所染之”,信矣哉!礼,国君继体,逾年改元,而章武之三年,则革称建兴,考之古义,体理为违。又国不置史,注记无官,是以行事多遗,灾异靡书。诸葛亮虽达於为政,凡此之类,犹有未周焉。然经载十二而年名不易,军旅屡兴而赦不妄下,不亦卓乎!自亮没后,兹制渐亏,优劣著矣。

\part{蜀书四}
\chapter{二主妃子传第四}

先主甘皇后,沛人也。先主临豫州,住小沛,纳以为妾。先主数丧嫡室,常摄内事。随先主於荆州,产后主。值曹公军至,追及先主於当阳长阪,于时困偪,弃后及后主,赖赵云保护,得免於难。后卒,葬于南郡。章武二年,追谥皇思夫人,迁葬於蜀,未至而先主殂陨。丞相亮上言:“皇思夫人履行脩仁,淑慎其身。大行皇帝昔在上将,嫔妃作合,载育圣躬,大命不融。大行皇帝存时,笃义垂恩,念皇思夫人神柩在远飘飖,特遣使者奉迎。会大行皇帝崩,今皇思夫人神柩以到,又梓宫在道,园陵将成,安厝有期。臣辄与太常臣赖恭等议:礼记曰:‘立爱自亲始,教民孝也;立敬自长始,教民顺也。’不忘其亲,所由生也。春秋之义,母以子贵。昔高皇帝追尊太上昭灵夫人为昭灵皇后,孝和皇帝改葬其母梁贵人,尊号曰恭怀皇后,孝愍皇帝亦改葬其母王夫人,尊号曰灵怀皇后。今皇思夫人宜有尊号,以慰寒泉之思,辄与恭等案谥法,宜曰昭烈皇后。诗曰:‘谷则异室,死则同穴。’故昭烈皇后宜与大行皇帝合葬,臣请太尉告宗庙,布露天下,具礼仪别奏。“制曰可。

先主穆皇后,陈留人也。兄吴壹,少孤,壹父素与刘焉有旧,是以举家随焉入蜀。焉有异志,而闻善相者相后当大贵。焉时将子瑁自随,遂为瑁纳后。瑁死,后寡居。先主既定益州,而孙夫人还吴,群下劝先主聘后,先主疑与瑁同族,法正进曰:“论其亲疏,何与晋文之於子圉乎?”於是纳后为夫人。建安二十四年,立为汉中王后。章武元年夏五月,策曰:“朕承天命,奉至尊,临万国。今以后为皇后,遣使持节丞相亮授玺绶,承宗庙,母天下,皇后其敬之哉!“建兴元年五月,后主即位,尊后为皇太后,称长乐宫。壹官至车骑将军,封县侯。延熙八年,后薨,合葬惠陵。

后主敬哀皇后,车骑将军张飞长女也。章武元年,纳为太子妃。建兴元年,立为皇后。十五年薨,葬南陵。

后主张皇后,前后敬哀之妹也。建兴十五年,入为贵人。延熙元年春正月,策曰:“朕统承大业,君临天下,奉郊庙社稷。今以贵人为皇后,使行丞相事左将军向朗持节授玺绶。勉脩中馈,恪肃禋祀,皇后其敬之哉!”咸熙元年,随后主迁于洛阳。

刘永字公寿,先主子,后主庶弟也。章武元年六月,使司徒靖立永为鲁王,策曰:“小子永,受兹青土。朕承天序,继统大业,遵脩稽古,建尔国家,封于东土,奄有龟蒙,世为藩辅。呜呼,恭朕之诏!惟彼鲁邦,一变適道,风化存焉。人之好德,世兹懿美。王其秉心率礼,绥尔士民,是飨是宜,其戒之哉!”建兴八年,改封为甘陵王。初,永憎宦人黄皓,皓既信任用事,谮构永于后主,后主稍疏外永,至不得朝见者十馀年。咸熙元年,永东迁洛阳,拜奉车都尉,封为乡侯。

刘理字奉孝,亦后主庶弟也,与永异母。章武元年六月,使司徒靖立理为梁王,策曰:“小子理,朕统承汉序,祗顺天命,遵脩典秩,建尔于东,为汉藩辅。惟彼梁土,畿甸之邦,民狎教化,易导以礼。往悉乃心,怀保黎庶,以永尔国,王其敬之哉!”建兴八年,改封理为安平王。延熙七年卒,谥曰悼王。子哀王胤嗣,十九年卒。子殇王承嗣,二十年卒。景耀四年诏曰:“安平王,先帝所命。三世早夭,国嗣颓绝,朕用伤悼。其以武邑侯辑袭王位。”辑,理子也,咸熙元年,东迁洛阳,拜奉车都尉,封乡侯。

后主太子璿,字文衡。母王贵人,本敬哀张皇后侍人也。延熙元年正月策曰:“在昔帝王,继体立嗣,副贰国统,古今常道。今以璿为皇太子,昭显祖宗之威,命使行丞相事左将军朗持节授印绶。其勉脩茂质,祗恪道义,谘询典礼,敬友师傅,斟酌众善,翼成尔德,可不务脩以自勖哉!”时年十五。景耀六年冬,蜀亡。咸熙元年正月,锺会作乱於成都,璿为乱兵所害。

评曰:易称有夫妇然后有父子,夫人伦之始,恩纪之隆,莫尚於此矣。是故纪录,以究一国之体焉。

\part{蜀书五}

\chapter{诸葛亮传第五}

\begin{yuanwen}
诸葛亮字孔明,琅琊阳都人也。汉司隶校尉诸葛丰后也。父珪,字君贡,汉末为太山郡丞。亮早孤,从父玄为袁术所署豫章太守,玄将亮及亮弟均之官。会汉朝更选朱皓代玄。玄素与荆州牧刘表有旧,往依之。玄卒,亮躬耕陇亩,好为《梁父吟》。身长八尺,每自比於管仲、乐毅,时人莫之许也。惟博陵崔州平、颍川徐庶元直与亮友善,谓为信然。
\end{yuanwen}

\begin{yuanwen}
时先主屯新野。徐庶见先主,先主器之,谓先主曰:“诸葛孔明者,卧龙也,将军岂愿见之乎?”

先主曰:“君与俱来。”

庶曰:“此人可就见,不可屈致也。将军宜枉驾顾之。”

由是先主遂诣亮,凡三往,乃见。
\end{yuanwen}

\begin{yuanwen}
因屏人曰:“汉室倾颓,奸臣窃命,主上蒙尘。孤不度德量力,欲信大义於天下,而智术浅短,遂用猖蹶,至于今日。然志犹未已,君谓计将安出?”

亮答曰:“自董卓已来,豪杰并起,跨州连郡者不可胜数。曹操比於袁绍,则名微而众寡,然操遂能克绍,以弱为强者,非惟天时,抑亦人谋也。今操已拥百万之众,挟天子而令诸侯,此诚不可与争锋。孙权据有江东,已历三世,国险而民附,贤能为之用,此可以为援而不可图也。荆州北据汉、沔,利尽南海,东连吴会,西通巴、蜀,此用武之国,而其主不能守,此殆天所以资将军,将军岂有意乎?益州险塞,沃野千里,天府之土,高祖因之以成帝业。刘璋闇弱,张鲁在北,民殷国富而不知存恤,智能之士思得明君。将军既帝室之胄,信义著於四海,总揽英雄,思贤如渴,若跨有荆、益,保其岩阻,西和诸戎,南抚夷越,外结好孙权,内脩政理;天下有变,则命一上将将荆州之军以向宛、洛,将军身率益州之众出於秦川,百姓孰敢不箪食壶浆以迎将军者乎?诚如是,则霸业可成,汉室可兴矣。”

先主曰:“善!”於是与亮情好日密。

关羽、张飞等不悦,先主解之曰:“孤之有孔明,犹鱼之有水也。愿诸君勿复言。”羽、飞乃止。
\end{yuanwen}

\begin{yuanwen}
刘表长子琦,亦深器亮。表受后妻之言,爱少子琮,不悦於琦。琦每欲与亮谋自安之术,亮辄拒塞,未与处画。琦乃将亮游观后园,共上高楼,饮宴之间,令人去梯,因谓亮曰:“今日上不至天,下不至地,言出子口,入於吾耳,可以言未?”

亮答曰:“君不见申生在内而危,重耳在外而安乎?”

琦意感悟,阴规出计。会黄祖死,得出,遂为江夏太守。俄而表卒,琮闻曹公来征,遣使请降。先主在樊闻之,率其众南行,亮与徐庶并从,为曹公所追破,获庶母。

庶辞先主而指其心曰:“本欲与将军共图王霸之业者,以此方寸之地也。今已失老母,方寸乱矣,无益於事,请从此别。”遂诣曹公。
\end{yuanwen}

\begin{yuanwen}
先主至於夏口,亮曰:“事急矣,请奉命求救於孙将军。”

时权拥军在柴桑,观望成败,亮说权曰:“海内大乱,将军起兵据有江东,刘豫州亦收众汉南,与曹操并争天下。今操芟夷大难,略已平矣,遂破荆州,威震四海。英雄无所用武,故豫州遁逃至此。将军量力而处之:若能以吴、越之众与中国抗衡,不如早与之绝;若不能当,何不案兵束甲,北面而事之!今将军外托服从之名,而内怀犹豫之计,事急而不断,祸至无日矣!”

权曰:“苟如君言,刘豫州何不遂事之乎?”

亮曰:“田横,齐之壮士耳,犹守义不辱,况刘豫州王室之胄,英才盖世,众士慕仰,若水之归海,若事之不济,此乃天也,安能复为之下乎!”

权勃然曰:“吾不能举全吴之地,十万之众,受制於人。吾计决矣!非刘豫州莫可以当曹操者,然豫州新败之后,安能抗此难乎?”

亮曰:“豫州军虽败於长阪,今战士还者及关羽水军精甲万人,刘琦合江夏战士亦不下万人。曹操之众,远来疲弊,闻追豫州,轻骑一日一夜行三百馀里,此所谓'强弩之末,势不能穿鲁缟'者也。故兵法忌之,曰'必蹶上将军'。且北方之人,不习水战;又荆州之民附操者,逼兵势耳,非心服也。今将军诚能命猛将统兵数万,与豫州协规同力,破操军必矣。操军破,必北还,如此则荆、吴之势强,鼎足之形成矣。成败之机,在於今日。”

权大悦,即遣周瑜、程普、鲁肃等水军三万,随亮诣先主,并力拒曹公。曹公败於赤壁,引军归邺。先主遂收江南,以亮为军师中郎将,使督零陵、桂阳、长沙三郡,调其赋税,以充军实。
\end{yuanwen}

\begin{yuanwen}
建安十六年,益州牧刘璋遣法正迎先主,使击张鲁。亮与关羽镇荆州。先主自葭萌还攻璋,亮与张飞、赵云等率众溯江,分定郡县,与先主共围成都。成都平,以亮为军师将军,署左将军府事。先主外出,亮常镇守成都,足食足兵。
\end{yuanwen}

\begin{yuanwen}
二十六年,群下劝先主称尊号,先主未许,亮说曰:“昔吴汉、耿弇等初劝世祖即帝位,世祖辞让,前后数四,耿纯进言曰,'天下英雄喁喁,冀有所望。如不从议者,士大夫各归求主,无为从公也。'世祖感纯言深至,遂然诺之。今曹氏篡汉,天下无主,大王刘氏苗族,绍世而起,今即帝位,乃其宜也。士大夫随大王久勤苦者,亦欲望尺寸之功如纯言耳。”

先主於是即帝位,策亮为丞相曰:“朕遭家不造,奉承大统,兢兢业业,不敢康宁,思靖百姓,惧未能绥。於戏!丞相亮其悉朕意,无怠辅朕之阙,助宣重光,以照明天下,君其勖哉!”

亮以丞相(录)尚书事,假节。张飞卒后,领司隶校尉。
\end{yuanwen}

\begin{yuanwen}
章武三年春,先主於永安病笃,召亮於成都,属以后事,谓亮曰:“君才十倍曹丕,必能安国,终定大事。若嗣子可辅,辅之;如其不才,君可自取。”

亮涕泣曰:“臣敢竭股肱之力,效忠贞之节,继之以死!”

先主又为诏敕后主曰:“汝与丞相从事,事之如父。”
\end{yuanwen}

\begin{yuanwen}
建兴元年,封亮武乡侯,开府治事。顷之,又领益州牧。政事无巨细,咸决於亮。南中诸郡,并皆叛乱,亮以新遭大丧,故未便加兵,且遣使聘吴,因结和亲,遂为与国。
\end{yuanwen}

\begin{yuanwen}
三年春,亮率众南征,其秋悉平。军资所出,国以富饶,乃治戎讲武,以俟大举。
\end{yuanwen}

五年,率诸军北驻汉中,临发,上疏曰:

先帝创业未半而中道崩殂,今天下三分,益州疲弊,此诚危急存亡之秋也。然侍卫之臣不懈於内,忠志之士忘身於外者,盖追先帝之殊遇,欲报之於陛下也。诚宜开张圣听,以光先帝遗德,恢弘志士之气,不宜妄自菲薄,引喻失义,以塞忠谏之路也。宫中府中俱为一体,陟罚臧否,不宜异同。若有作奸犯科及为忠善者,宜付有司论其刑赏,以昭陛下平明之理,不宜偏私,使内外异法也。侍中、侍郎郭攸之、费祎、董允等,此皆良实,志虑忠纯,是以先帝简拔以遗陛下。愚以为宫中之事,事无大小,悉以咨之,然后施行,必能裨补阙漏,有所广益。将军向宠,性行淑均,晓畅军事,试用於昔日,先帝称之曰能,是以众议举宠为督。愚以为营中之事,悉以咨之,必能使行阵和睦,优劣得所。亲贤臣,远小人,此先汉所以兴隆也;亲小人,远贤臣,此后汉所以倾颓也。先帝在时,每与臣论此事,未尝不叹息痛恨於桓、灵也。侍中、尚书、长史、参军,此悉贞良死节之臣,愿陛下亲之信之,则汉室之隆,可计日而待也。

臣本布衣,躬耕於南阳,苟全性命於乱世,不求闻达於诸侯。先帝不以臣卑鄙,猥自枉屈,三顾臣於草庐之中,谘臣以当世之事,由是感激,遂许先帝以驱驰。后值倾覆,受任於败军之际,奉命於危难之间,尔来二十有一年矣。先帝知臣谨慎,故临崩寄臣以大事也。受命以来,夙夜忧叹,恐讬付不效,以伤先帝之明,故五月渡泸,深入不毛。今南方已定,兵甲已足,当奖率三军,北定中原,庶竭驽钝,攘除奸凶,兴复汉室,还于旧都。此臣所以报先帝,而忠陛下之职分也。

至於斟酌损益,进尽忠言,则攸之、祎、允之任也。愿陛下讬臣以讨贼兴复之效;不效,则治臣之罪,以告先帝之灵。若无兴德之言,则责攸之、祎、允等之慢,以彰其咎。陛下亦宜自谋,以谘诹善道,察纳雅言,深追先帝遗诏。臣不胜受恩感激,今当远离,临表涕零,不知所言。

遂行,屯于沔阳。

\begin{yuanwen}
六年春,扬声由斜谷道取郿,使赵云、邓芝为疑军,据箕谷,魏大将军曹真举众拒之。亮身率诸军攻祁山,戎陈整齐,赏罚肃而号令明,南安、天水、安定三郡叛魏应亮,关中响震。

魏明帝西镇长安,命张郃拒亮,亮使马谡督诸军在前,与郃战于街亭。谡违亮节度,举动失宜,大为郃所破。

亮拔西县千馀家,还于汉中,戮谡以谢众。上疏曰:“臣以弱才,叨窃非据,亲秉旄钺以厉三军,不能训章明法,临事而惧,至有街亭违命之阙,箕谷不戒之失,咎皆在臣授任无方。臣明不知人,恤事多闇,《春秋》责帅,臣职是当。请自贬三等,以督厥咎。”於是以亮为右将军,行丞相事,所总统如前。
\end{yuanwen}

\begin{yuanwen}
冬,亮复出散关,围陈仓,曹真拒之,亮粮尽而还。魏将军王双率骑追亮,亮与战,破之,斩双。

七年,亮遣陈式攻武都、阴平。魏雍州刺史郭淮率众欲击式,亮自出至建威,淮退还,遂平二郡。

诏策亮曰:“街亭之役,咎由马谡,而君引愆,深自贬抑,重违君意,听顺所守。前年耀师,馘斩王双;今岁爰征,郭淮遁走;降集氐、羌,兴复二郡,威镇凶暴,功勋显然。方今天下骚扰,元恶未枭,君受大任,幹国之重,而久自挹损,非所以光扬洪烈矣。今复君丞相,君其勿辞。”
\end{yuanwen}

\begin{yuanwen}
九年,亮复出祁山,以木牛运,粮尽退军,与魏将张郃交战,射杀郃。

十二年春,亮悉大众由斜谷出,以流马运,据武功五丈原,与司马宣王对於渭南。亮每患粮不继,使己志不申,是以分兵屯田,为久驻之基。耕者杂於渭滨居民之间,而百姓安堵,军无私焉。相持百馀日。其年八月,亮疾病,卒于军,时年五十四。及军退,宣王案行其营垒处所,曰:“天下奇才也!”
\end{yuanwen}

\begin{yuanwen}
亮遗命葬汉中定军山,因山为坟,冢足容棺,敛以时服,不须器物。
\end{yuanwen}

诏策曰:“惟君体资文武,明叡笃诚,受遗讬孤,匡辅朕躬,继绝兴微,志存靖乱;爰整六师,无岁不征,神武赫然,威镇八荒,将建殊功於季汉,参伊、周之巨勋。如何不吊,事临垂克,遘疾陨丧!朕用伤悼,肝心若裂。夫崇德序功,纪行命谥,所以光昭将来,刊载不朽。今使使持节左中郎将杜琼,赠君丞相武乡侯印绶,谥君为忠武侯。魂而有灵,嘉兹宠荣。呜呼哀哉!呜呼哀哉!”

初,亮自表后主曰:“成都有桑八百株,薄田十五顷,子弟衣食,自有馀饶。至於臣在外任,无别调度,随身衣食,悉仰於官,不别治生,以长尺寸。若臣死之日,不使内有馀帛,外有赢财,以负陛下。”及卒,如其所言。

亮性长於巧思,损益连弩,木牛流马,皆出其意;推演兵法,作八陈图,咸得其要云。亮言教书奏多可观,别为一集。

景耀六年春,诏为亮立庙于沔阳。秋,魏镇西将军钟会,至汉川,祭亮之庙,令军士不得于亮墓所左右刍牧樵采。亮弟均,官至长水校尉。亮子瞻,嗣爵。

诸葛氏集目录:开府作牧第一权制第二南征第三北出第四计算第五训厉第六综核上第七综核下第八杂言上第九杂言第十贵和第十一兵要第十二传运第十三与孙权书第十四与诸葛谨书第十五与孟达书第十六废李平第十七法检上第十八法检下第十九科令上第二十科令下第二十一军令上第二十二军令中第二十三军令下第二十四右二十四篇,凡十万四千一百一十二字。

臣寿等言:臣前在著作郎,侍中领中书监济北侯臣荀勖、中书令关内侯臣和峤奏,使臣定故蜀丞相诸葛亮故事。亮毗佐危国,负阻不宾,然犹存录其言,耻善有遗,诚是大晋光明至德,泽被无疆,自古以来,未之有伦也。辄删除复重,随类相从,凡为二十四篇,篇名如右。

亮少有逸群之才,英霸之器,身长八尺,容貌甚伟,时人异焉。遭汉末扰乱,随叔父玄避难荆州,躬耕于野,不求闻达。时左将军刘备以亮有殊量,乃三顾亮於草庐之中;亮深谓备雄姿杰出,遂解带写诚,厚相结纳。及魏武帝南征荆州,刘琮举州委质,而备失势众寡,无立锥之地。亮时年二十七,乃建奇策,身使孙权,求援吴会。权既宿服仰备,又睹亮奇雅,甚敬重之,即遣兵三万人以助备。备得用与武帝交战,大破其军,乘胜克捷,江南悉平。后备又西取益州。益州既定,以亮为军师将军。备称尊号,拜亮为丞相,录尚书事。及备殂没,嗣子幼弱,事无巨细,亮皆专之。於是外连东吴,内平南越,立法施度,整理戎旅,工械技巧,物究其极,科教严明,赏罚必信,无恶不惩,无善不显,至於吏不容奸,人怀自厉,道不拾遗,强不侵弱,风化肃然也。

当此之时,亮之素志,进欲龙骧虎视,苞括四海,退欲跨陵边疆,震荡宇内。又自以为无身之日,则未有能蹈涉中原、抗衡上国者,是以用兵不戢,屡耀其武。然亮才,於治戎为长,奇谋为短,理民之幹,优於将略。而所与对敌,或值人杰,加众寡不侔,攻守异体,故虽连年动众,未能有克。昔萧何荐韩信,管仲举王子城父,皆忖己之长,未能兼有故也。亮之器能政理,抑亦管、萧之亚匹也,而时之名将无城父、韩信,故使功业陵迟,大义不及邪?盖天命有归,不可以智力争也。

青龙二年春,亮帅众出武功,分兵屯田,为久驻之基。其秋病卒,黎庶追思,以为口实。至今梁、益之民,咨述亮者,言犹在耳,虽甘棠之咏召公,郑人之歌子产,无以远譬也。孟轲有云:“以逸道使民,虽劳不怨;以生道杀人,虽死不忿。”信矣!论者或怪亮文彩不艳,而过於丁宁周至。臣愚以为咎繇大贤也,周公圣人也,考之尚书,咎繇之谟略而雅,周公之诰烦而悉。何则?咎繇与舜、禹共谈,周公与群下矢誓故也。亮所与言,尽众人凡士,故其文指不得及远也。然其声教遗言,皆经事综物,公诚之心,形于文墨,足以知其人之意理,而有补於当世。

伏惟陛下迈踪古圣,荡然无忌,故虽敌国诽谤之言,咸肆其辞而无所革讳,所以明大通之道也。谨录写上诣著作。臣寿诚惶诚恐,顿首顿首,死罪死罪。泰始十年二月一日癸巳,平阳侯相臣陈寿上。

乔字伯松,亮兄瑾之第二子也,本字仲慎。与兄元逊俱有名於时,论者以为乔才不及兄,而性业过之。初,亮未有子,求乔为嗣,瑾启孙权遣乔来西,亮以乔为己適子,故易其字焉。拜为驸马都尉,随亮至汉中。年二十五,建兴六年卒。子攀,官至行护军翊武将军,亦早卒。诸葛恪见诛於吴,子孙皆尽,而亮自有胄裔,故攀还复为瑾后。

瞻字思远。建兴十二年,亮出武功,与兄瑾书曰:“瞻今已八岁,聪慧可爱,嫌其早成,恐不为重器耳。“年十七,尚公主,拜骑都尉。其明年为羽林中郎将,屡迁射声校尉、侍中、尚书仆射,加军师将军。瞻工书画,强识念,蜀人追思亮,咸爱其才敏。每朝廷有一善政佳事,虽非瞻所建倡,百姓皆传相告曰:“葛侯之所为也。“是以美声溢誉,有过其实。景耀四年,为行都护卫将军,与辅国大将军南乡侯董厥并平尚书事。六年冬,魏征西将军邓艾伐蜀,自阴平由景谷道旁入。瞻督诸军至涪停住,前锋破,退还,住绵竹。艾遣书诱瞻曰:“若降者必表为琅邪王。”瞻怒,斩艾使。遂战,大败,临陈死,时年三十七。众皆离散,艾长驱至成都。瞻长子尚,与瞻俱没。次子京及攀子显等,咸熙元年内移河东。

董厥者,丞相亮时为府令史,亮称之曰:“董令史,良士也。吾每与之言,思慎宜適。“徙为主簿。亮卒后,稍迁至尚书仆射,代陈祗为尚书令,迁大将军,平台事,而义阳樊建代焉。延熙十四年,以校尉使吴,值孙权病笃,不自见建。权问诸葛恪曰:“樊建何如宗预也?”恪对曰:“才识不及预,而雅性过之。”后为侍中,守尚书令。自瞻、厥、建统事,姜维常征伐在外,宦人黄皓窃弄机柄,咸共将护,无能匡矫,然建特不与皓和好往来。蜀破之明年春,厥、建俱诣京都,同为相国参军,其秋并兼散骑常侍,使蜀慰劳。

评曰:诸葛亮之为相国也,抚百姓,示仪轨,约官职,从权制,开诚心,布公道;尽忠益时者虽雠必赏,犯法怠慢者虽亲必罚,服罪输情者虽重必释,游辞巧饰者虽轻必戮;善无微而不赏,恶无纤而不贬;庶事精练,物理其本,循名责实,虚伪不齿;终於邦域之内,咸畏而爱之,刑政虽峻而无怨者,以其用心平而劝戒明也。可谓识治之良才,管、萧之亚匹矣。然连年动众,未能成功,盖应变将略,非其所长欤!

\part{蜀书六}

\chapter{关张马黄赵传第六}

\begin{yuanwen}
关羽字云长,本字长生,河东解人也。亡命奔涿郡。先主於乡里合徒众,而羽与张飞为之御侮。先主为平原相,以羽、飞为别部司马,分统部曲。先主与二人寝则同床,恩若兄弟。而稠人广坐,侍立终日,随先主周旋,不避艰险。先主之袭杀徐州刺史车胄,使羽守下邳城,行太守事,而身还小沛。
\end{yuanwen}

\begin{yuanwen}
建安五年,曹公东征,先主奔袁绍。曹公禽羽以归,拜为偏将军,礼之甚厚。绍遣大将颜良攻东郡太守刘延於白马,曹公使张辽及羽为先锋击之。羽望见良麾盖,策马刺良於万众之中,斩其首还,绍诸将莫能当者,遂解白马围。曹公即表封羽为汉寿亭侯。
\end{yuanwen}

\begin{yuanwen}
初,曹公壮羽为人,而察其心神无久留之意,谓张辽曰:“卿试以情问之。”

既而辽以问羽,羽叹曰:“吾极知曹公待我厚,然吾受刘将军厚恩,誓以共死,不可背之。吾终不留,吾要当立效以报曹公乃去。”

辽以羽言报曹公,曹公义之。及羽杀颜良,曹公知其必去,重加赏赐。羽尽封其所赐,拜书告辞,而奔先主於袁军。左右欲追之,曹公曰:“彼各为其主,勿追也。”
\end{yuanwen}

\begin{yuanwen}
从先主就刘表。表卒,曹公定荆州,先主自樊将南渡江,别遣羽乘船数百艘会江陵。曹公追至当阳长阪,先主斜趣汉津,適与羽船相值,共至夏口。孙权遣兵佐先主拒曹公,曹公引军退归。先主收江南诸郡,乃封拜元勋,以羽为襄阳太守、荡寇将军,驻江北。先主西定益州,拜羽董督荆州事。羽闻马超来降,旧非故人,羽书与诸葛亮,问超人才可谁比类。亮知羽护前,乃答之曰:“孟起兼资文武,雄烈过人,一世之杰,黥、彭之徒,当与益德并驱争先,犹未及髯之绝伦逸群也。”

羽美须髯\footnote{r\'an,胡须。},故亮谓之髯。羽省书大悦,以示宾客。
\end{yuanwen}

\begin{yuanwen}
羽尝为流矢所中,贯其左臂,后创虽愈,每至阴雨,骨常疼痛,医曰:“矢镞\footnote{z\'u,箭头。}有毒,毒入于骨,当破臂作创,刮骨去毒,然后此患乃除耳。”

羽便伸臂令医劈之。时羽適请诸将饮食相对,臂血流离,盈於盘器,而羽割炙引酒,言笑自若。
\end{yuanwen}

\begin{yuanwen}
二十四年,先主为汉中王,拜羽为前将军,假节钺。是岁,羽率众攻曹仁於樊。曹公遣于禁助仁。

秋,大霖雨,汉水汎(泛)溢,禁所督七军皆没。禁降羽,羽又斩将军庞德。梁、郏、陆浑群盗或遥受羽印号,为之支党,羽威震华夏。曹公议徙许都以避其锐,司马宣王、蒋济以为关羽得志,孙权必不愿也。可遣人劝权蹑其后,许割江南以封权,则樊围自解。曹公从之。

先是,权遣使为子索羽女,羽骂辱其使,不许婚,权大怒。又南郡太守麋芳在江陵,将军士仁屯公安,素皆嫌羽轻己。自羽之出军,芳、仁供给军资,不悉相救。羽言“还当治之”,芳、仁咸怀惧不安。於是权阴诱芳、仁,芳、仁使人迎权。而曹公遣徐晃救曹仁,羽不能克,引军退还。权已据江陵,尽虏羽士众妻子,羽军遂散。权遣将逆击羽,斩羽及子平于临沮。
\end{yuanwen}

\begin{yuanwen}
追谥羽曰壮缪侯。子兴嗣。(关)兴字安国,少有令问,丞相诸葛亮深器异之。弱冠为侍中、中监军,数岁卒。子统嗣,尚公主,官至虎贲中郎将。卒,无子,以兴庶子彝续封。
\end{yuanwen}

\begin{yuanwen}
张飞字益德,涿郡人也,少与关羽俱事先主。羽年长数岁,飞兄事之。先主从曹公破吕布,随还许,曹公拜飞为中郎将。先主背曹公依袁绍、刘表。表卒,曹公入荆州,先主奔江南。曹公追之,一日一夜,及於当阳(之)长阪。先主闻曹公卒至,弃妻子走,使飞将二十骑拒后。飞据水断桥,瞋目横矛曰:“身是张益德也,可来共决死!”

敌皆无敢近者,故遂得免。先主既定江南,以飞为宜都太守、征虏将军,封新亭侯,后转在南郡。先主入益州,还攻刘璋,飞与诸葛亮等泝流而上,分定郡县。至江州,破璋将巴郡太守严颜,生获颜。飞呵颜曰:“大军至,何以不降而敢拒战?”

颜答曰:“卿等无状,侵夺我州,我州但有断头将军,无有降将军也。”

飞怒,令左右牵去斫\footnote{zhu\'o,砍头。}头,颜色不变,曰:“斫头便斫头,何为怒邪!”

飞壮而释之,引为宾客。飞所过战克,与先主会于成都。益州既平,赐诸葛亮、法正、飞及关羽金各五百斤,银千斤,钱五千万,锦千匹,其馀颁赐各有差,以飞领巴西太守。
\end{yuanwen}

\begin{yuanwen}
曹公破张鲁,留夏侯渊、张郃守汉川。郃别督诸军下巴西,欲徙其民於汉中,进军宕渠、蒙头、盪石,与飞相拒五十馀日。飞率精卒万馀人,从他道邀郃军交战,山道迮狭,前后不得相救,飞遂破郃。郃弃马缘山,独与麾下十馀人从间道退,引军还南郑,巴土获安。先主为汉中王,拜飞为右将军、假节。
\end{yuanwen}

\begin{yuanwen}
章武元年,迁车骑将军,领司隶校尉,进封西乡侯,策曰:“朕承天序,嗣奉洪业,除残靖乱,未烛厥理。今寇虏作害,民被荼毒,思汉之士,延颈鹤望。朕用怛然,坐不安席,食不甘味,整军诰誓,将行天罚。以君忠毅,侔踪召虎,名宣遐迩,故特显命,高墉进爵,兼司于京。其诞将天威,柔服以德,伐叛以刑,称朕意焉。《诗》不云乎,'匪疚匪棘,王国来极。肇敏戎功,用锡尔祉'。可不勉欤!”
\end{yuanwen}

\begin{yuanwen}
初,飞雄壮威猛,亚於关羽,魏谋臣程昱等咸称羽、飞万人之敌也。羽善待卒伍而骄於士大夫,飞爱敬君子而不恤小人。先主常戒之曰:“卿刑杀既过差,又日鞭挝健儿,而令在左右,此取祸之道也。”

飞犹不悛。先主伐吴,飞当率兵万人,自阆中会江州。临发,其帐下将张达、范强杀飞,持其首,顺流而奔孙权。飞营都督表报先主,先主闻飞都督之有表也,曰:“噫!飞死矣。”

追谥飞曰桓侯。长子苞,早夭。次子绍嗣,官至侍中尚书仆射。苞子遵为尚书,随诸葛瞻於绵竹,与邓艾战,死。
\end{yuanwen}

\begin{yuanwen}
马超字孟起,扶风茂陵人也。父腾,灵帝末与边章、韩遂等俱起事於西州。

初平三年,遂、腾率众诣长安。汉朝以遂为镇西将军,遣还金城,腾为征西将军,遣屯郿。后腾袭长安,败走,退还凉州。司隶校尉锺繇镇关中,移书遂、腾,为陈祸福。腾遣超随繇讨郭援、高幹於平阳,超将庞德亲斩援首。后腾与韩遂不和,求还京畿。於是徵为卫尉,以超为偏将军,封都亭侯,领腾部曲。
\end{yuanwen}

\begin{yuanwen}
超既统众,遂与韩遂合从,及杨秋、李堪、成宜等相结,进军至潼关。曹公与遂、超单马会语,超负其多力,阴欲突前捉曹公,曹公左右将许褚瞋目盻之,超乃不敢动。曹公用贾诩谋,离间超、遂,更相猜疑,军以大败。超走保诸戎,曹公追至安定,会北方有事,引军东还。杨阜说曹公曰:“超有信、布之勇,甚得羌、胡心。若大军还,不严为其备,陇上诸郡非国家之有也。”

超果率诸戎以击陇上郡县,陇上郡县皆应之,杀凉州刺史韦康,据冀城,有其众。超自称征西将军,领并州牧,督凉州军事。康故吏民杨阜、姜叙、梁宽、赵衢等,合谋击超。阜、叙起於卤城,超出攻之,不能下;宽、衢闭冀城门,超不得入。进退狼狈,乃奔汉中依张鲁。鲁不足与计事,内怀於邑,闻先主围刘璋於成都,密书请降。
\end{yuanwen}

\begin{yuanwen}
先主遣人迎超,超将兵径到城下。城中震怖,璋即稽首,以超为平西将军,督临沮,因为前都亭侯。先主为汉中王,拜超为左将军,假节。

章武元年,迁骠骑将军,领凉州牧,进封斄乡侯,策曰:“朕以不德,获继至尊,奉承宗庙。曹操父子,世载其罪,朕用惨怛,疢如疾首。海内怨愤,归正反本,暨于氐、羌率服,獯鬻\footnote{x\=un y\`u,指代匈奴。}慕义。以君信著北土,威武并昭,是以委任授君,抗飏虓虎,兼董万里,求民之瘼。其明宣朝化,怀保远迩,肃慎赏罚,以笃汉祜,以对于天下。”

二年卒,时年四十七。临没上疏曰:“臣门宗二百馀口,为孟德所诛略尽,惟有从弟岱,当为微宗血食之继,深讬(托)陛下,馀无复言。”

追谥超曰威侯,子承嗣。岱位至平北将军,进爵陈仓侯。超女配安平王理。
\end{yuanwen}

\begin{yuanwen}
黄忠字汉升,南阳人也。荆州牧刘表以为中郎将,与表从子磐共守长沙攸县。及曹公克荆州,假行裨将军,仍就故任,统属长沙太守韩玄。先主南定诸郡,忠遂委质,随从入蜀。自葭萌受任,还攻刘璋,忠常先登陷陈,勇毅冠三军。益州既定,拜为讨虏将军。
\end{yuanwen}

\begin{yuanwen}
建安二十四年,於汉中定军山击夏侯渊。渊众甚精,忠推锋必进,劝率士卒,金鼓振天,欢声动谷,一战斩渊,渊军大败。迁征西将军。

是岁,先主为汉中王,欲用忠为后将军,诸葛亮说先主曰:“忠之名望,素非关、马之伦也,而今便令同列。马、张在近,亲见其功,尚可喻指;关遥闻之,恐必不悦,得无不可乎!”

先主曰:“吾自当解之。”遂与羽等齐位,赐爵关内侯。明年卒,追谥刚侯。子叙,早没,无后。
\end{yuanwen}

\begin{yuanwen}
赵云字子龙,常山真定人也。本属公孙瓒,瓒遣先主为田楷拒袁绍,云遂随从,为先主主骑。及先主为曹公所追於当阳长阪,弃妻子南走,云身抱弱子,即后主也,保护甘夫人,即后主母也,皆得免难。迁为牙门将军。先主入蜀,云留荆州。
\end{yuanwen}

\begin{yuanwen}
先主自葭萌还攻刘璋,召诸葛亮。亮率云与张飞等俱溯(泝)江西上,平定郡县。至江州,分遣云从外水上江阳,与亮会于成都。成都既定,以云为翊军将军。

建兴元年,为中护军、征南将军,封永昌亭侯,迁镇东将军。

五年,随诸葛亮驻汉中。

明年,亮出军,扬声由斜谷道,曹真遣大众当之。亮令云与邓芝往拒,而身攻祁山。云、芝兵弱敌强,失利於箕谷,然敛众固守,不至大败。军退,贬为镇军将军。

七年卒,追谥顺平侯。
\end{yuanwen}

\begin{yuanwen}
初,先主时,惟法正见谥;后主时,诸葛亮功德盖世,蒋琬、费祎荷国之重,亦见谥;陈祗宠待,特加殊奖,夏侯霸远来归国,故复得谥;於是关羽、张飞、马超、庞统、黄忠及云乃追谥,时论以为荣。云子统嗣,官至虎贲中郎,督行领军。次子广,牙门将,随姜维沓中,临陈战死。
\end{yuanwen}

评曰:关羽、张飞皆称万人之敌,为世虎臣。羽报效曹公,飞义释严颜,并有国士之风。然羽刚而自矜,飞暴而无恩,以短取败,理数之常也。马超阻戎负勇,以覆其族,惜哉!能因穷致泰,不犹愈乎!黄忠、赵云强挚壮猛,并作爪牙,其灌、滕之徒欤?

\part{蜀书七}
\chapter{庞统法正传第七}

庞统字士元,襄阳人也。少时朴钝,未有识者。颍川司马徽清雅有知人鉴,统弱冠往见徽,徽采桑於树上,坐统在树下,共语自昼至夜。徽甚异之,称统当南州士之冠冕,由是渐显。后郡命为功曹。性好人伦,勤於长养。每所称述,多过其才,时人怪而问之,统答曰:“当今天下大乱,雅道陵迟,善人少而恶人多。方欲兴风俗,长道业,不美其谭即声名不足慕企,不足慕企而为善者少矣。今拔十失五,犹得其半,而可以崇迈世教,使有志者自励,不亦可乎?”吴将周瑜助先主取荆州,因领南郡太守。瑜卒,统送丧至吴,吴人多闻其名。及当西还,并会昌门,陆勣、顾劭、全琮皆往。统曰:“陆子可谓驽马有逸足之力,顾子可谓驽牛能负重致远也。”谓全琮曰:“卿好施慕名,有似汝南樊子昭。虽智力不多,亦一时之佳也。”绩、劭谓统曰:“使天下太平,当与卿共料四海之士。“深与统相结而还。

先主领荆州,统以从事守耒阳令,在县不治,免官。吴将鲁肃遗先主书曰:“庞士元非百里才也,使处治中、别驾之任,始当展其骥足耳。”诸葛亮亦言之於先主,先主见与善谭,大器之,以为治中从事。亲待亚於诸葛亮,遂与亮并为军师中郎将。亮留镇荆州。统随从入蜀。

益州牧刘璋与先主会涪,统进策曰:“今因此会,便可执之,则将军无用兵之劳而坐定一州也。”先主曰:“初入他国,恩信未著,此不可也。”璋既还成都,先主当为璋北征汉中,统复说曰:“阴选精兵,昼夜兼道,径袭成都;璋既不武,又素无预备,大军卒至,一举便定,此上计也。杨怀、高沛,璋之名将,各仗强兵,据守关头,闻数有笺谏璋,使发遣将军还荆州。将军未至,遣与相闻,说荆州有急,欲还救之,并使装束,外作归形;此二子既服将军英名,又喜将军之去,计必乘轻骑来见,将军因此执之,进取其兵,乃向成都,此中计也。退还白帝,连引荆州,徐还图之,此下计也。若沈吟不去,将致大因,不可久矣。”先主然其中计,即斩怀、沛,还向成都,所过辄克。於涪大会,置酒作乐,谓统曰:“今日之会,可谓乐矣。”统曰:“伐人之国而以为欢,非仁者之兵也。”先主醉,怒曰:“武王伐纣,前歌后舞,非仁者邪?卿言不当,宜速起出!”於是统逡巡引退。先主寻悔,请还。统复故位,初不顾谢,饮食自若。先主谓曰:“向者之论,阿谁为失?”统对曰:“君臣俱失。”先主大笑,宴乐如初。

进围雒县,统率众攻城,为流矢所中,卒,时年三十六。先主痛惜,言则流涕。拜统父议郎,迁谏议大夫,诸葛亮亲为之拜。追赐统爵关内侯,谥曰靖侯。统子宏,字巨师,刚简有臧否,轻傲尚书令陈只,为只所抑,卒於涪陵太守。统弟林,以荆州治中从事参镇北将军黄权征吴,值军败,随权入魏,魏封列侯,至钜鹿太守。

法正字孝直,扶风郿人也。祖父真,有清节高名。建安初,天下饥荒,正与同郡孟达俱入蜀依刘璋,久之为新都令,后召署军议校尉。既不任用,又为其州邑俱侨客者所谤无行,志意不得。益州别驾张松与正相善,忖璋不足与有为,常窃叹息。松於荆州见曹公还,劝璋绝曹公而自结先主。璋曰:“谁可使者?”松乃举正,正辞让,不得已而往。正既还,为松称说先主有雄略,密谋协规,愿共戴奉,而未有缘。后因璋闻曹公欲遣将征张鲁之有惧心也,松遂说璋宜迎先主,使之讨鲁,复令正衔命。正既宣旨,阴献策於先主曰:“以明将军之英才,乘刘牧之懦弱;张松,州之股肱,以响应於内;然后资益州之殷富,冯天府之险阻,以此成业,犹反掌也。”先主然之,溯江而西,与璋会涪。北至葭萌,南还取璋。

郑度说璋曰:“左将军县军袭我,兵不满万,士众未附,野谷是资,军无辎重。其计莫若尽驱巴西、梓潼民内涪水以西,其仓廪野谷,一皆烧除,高垒深沟,静以待之。彼至,请战,勿许,久无所资,不过百日,必将自走。走而击之,则必禽耳。”先主闻而恶之,以问正。正曰:“终不能用,无可忧也。“璋果如正言,谓其群下曰:“吾闻拒敌以安民,未闻动民以避敌也。”於是黜度,不用其计。及军围雒城,正笺与璋曰:“正受性无术,盟好违损,惧左右不明本末,必并归咎,蒙耻没身,辱及执事,是以损身於外,不敢反命。恐圣听秽恶其声,故中间不有笺敬,顾念宿遇,瞻望悢々。然惟前后披露腹心,自从始初以至於终,实不藏情,有所不尽,但愚闇策薄,精诚不感,以致於此耳。今国事已危,祸害在速,虽捐放於外,言足憎尤,犹贪极所怀,以尽馀忠。明将军本心,正之所知也,实为区区不欲失左将军之意,而卒至於是者,左右不达英雄从事之道,谓可违信黩誓,而以意气相致,日月相迁,趋求顺耳悦目,随阿遂指,不图远虑为国深计故也。事变既成,又不量强弱之势,以为左将军县远之众,粮谷无储,欲得以多击少,旷日相持。而从关至此,所历辄破,离宫别屯,日自零落。雒下虽有万兵,皆坏陈之卒,破军之将,若欲争一旦之战,则兵将势力,实不相当。各欲远期计粮者,今此营守已固,谷米已积,而明将军土地日削,百姓日困,敌对遂多,所供远旷。愚意计之,谓必先竭,将不复以持久也。空尔相守,犹不相堪,今张益德数万之众,已定巴东,入犍为界,分平资中、德阳,三道并侵,将何以御之?本为明将军计者,必谓此军县远无粮,馈运不及,兵少无继。今荆州道通,众数十倍,加孙车骑遣弟及李异、甘宁等为其后继。若争客主之势,以土地相胜者,今此全有巴东,广汉、犍为,过半已定,巴西一郡,复非明将军之有也。计益州所仰惟蜀,蜀亦破坏;三分亡二,吏民疲困,思为乱者十户而八;若敌远则百姓不能堪役,敌近则一旦易主矣。广汉诸县,是明比也。又鱼复与关头实为益州福祸之门,今二门悉开,坚城皆下,诸军并破,兵将俱尽,而敌家数道并进,已入心腹,坐守都、雒,存亡之势,昭然可见。斯乃大略,其外较耳,其馀屈曲,难以辞极也。以正下愚,犹知此事不可复成,况明将军左右明智用谋之士,岂当不见此数哉?旦夕偷幸,求容取媚,不虑远图,莫肯尽心献良计耳。若事穷势迫,将各索生,求济门户,展转反覆,与今计异,不为明将军尽死难也。而尊门犹当受其忧。正虽获不忠之谤,然心自谓不负圣德,顾惟分义,实窃痛心。左将军从本举来,旧心依依,实无薄意。愚以为可图变化,以保尊门。”

十九年,进围成都,璋蜀郡太守许靖将逾城降,事觉,不果。璋以危亡在近,故不诛靖。璋既稽服,先主以此薄靖不用也。正说曰:“天下有获虚誉而无其实者,许靖是也。然今主公始创大业,天下之人不可户说,靖之浮称,播流四海,若其不礼,天下之人以是谓主公为贱贤也。宜加敬重,以眩远近,追昔燕王之待郭隗。”先主於是乃厚待靖。以正为蜀郡太守、扬武将军,外统都畿,内为谋主。一餐之德,睚眦之怨,无不报复,擅杀毁伤己者数人。或谓诸葛亮曰:“法正於蜀郡太纵横,将军宜启主公,抑其威福。”亮答曰:“主公之在公安也,北畏曹公之强,东惮孙权之逼,近则惧孙夫人生变於肘腋之下;当斯之时,进退狼跋,法孝直为之辅翼,令翻然翱翔,不可复制,如何禁止法正使不得行其意邪!”初,孙权以妹妻先主,妹才捷刚猛,有诸兄之风,侍婢百馀人,皆亲执刀侍立,先主每入,衷心常凛凛;亮又知先主雅爱信正,故言如此。

二十二年,正说先主曰:“曹操一举而降张鲁,定汉中,不因此势以图巴、蜀,而留夏侯渊、张郃屯守,身遽北还,此非其智不逮而力不足也,必将内有忧偪故耳。今策渊、郃才略,不胜国之将帅,举众往讨,则必可克。克之之日,广农积谷,观衅伺隙,上可以倾覆寇敌,尊奖王室,中可以蚕食雍、凉,广拓境土,下可以固守要害,为持久之计。此盖天以与我,时不可失也。”先主善其策,乃率诸将进兵汉中,正亦从行。二十四年,先主自阳平南渡沔水,缘山稍前,於定军、兴势作营。渊将兵来争其地。正曰:“可击矣。”先主命黄忠乘高鼓噪攻之,大破渊军,渊等授首。曹公西征,闻正之策,曰:“吾故知玄德不办有此,必为人所教也。”

先主立为汉中王,以正为尚书令、护军将军。明年卒,时年四十五。先主为之流涕者累日。谥曰翼侯。赐子邈爵关内侯,官至奉车都尉、汉阳太守。诸葛亮与正,虽好尚不同,以公义相取。亮每奇正智术。先主既即尊号,将东征孙权以复关羽之耻,群臣多谏,一不从。章武二年,大军败绩,还住白帝。亮叹曰:“法孝直若在,则能制主上,令不东行;就复东行,必不倾危矣。”

评曰:庞统雅好人流,经学思谋,于时荆、楚谓之高俊。法正著见成败,有奇画策算,然不以德素称也。拟之魏臣,统其荀彧之仲叔,正其程、郭之俦俪邪?

\part{蜀书八}
\chapter{许麋孙简伊秦传第八}

许靖字文休,汝南平舆人。少与从弟劭俱知名,并有人伦臧否之称,而私情不协。劭为郡功曹,排摈靖不得齿叙,以马磨自给。颍川刘翊为汝南太守,乃举靖计吏,察孝廉,除尚书郎,典选举。灵帝崩,董卓秉政,以汉阳周毖为吏部尚书,与靖共谋议,进退天下之士,沙汰秽浊,显拔幽滞。进用颍川荀爽、韩融、陈纪等为公、卿、郡守,拜尚书韩馥为冀州牧,侍中刘岱为兖州刺史,颍川张咨为南阳太守,陈留孔伷为豫州刺史,东郡张邈为陈留太守,而迁靖巴郡太守,不就,补御史中丞。馥等到官,各举兵还向京都,欲以诛卓。卓怒毖曰:“诸君言当拔用善士,卓从君计,不欲违天下人心。而诸君所用人,至官之日,还来相图。卓何用相负!”叱毖令出,於外斩之。靖从兄陈相玚,又与伷合规,靖惧诛,奔伷。伷卒,依扬州刺史陈祎。祎死,吴郡都尉许贡、会稽太守王朗素与靖有旧,故往保焉。靖收恤亲里,经纪振赡,出於仁厚。

孙策东渡江,皆走交州以避其难,靖身坐岸边,先载附从,疏亲悉发,乃从后去,当时见者莫不叹息。既至交阯,交阯太守士燮厚加敬待。陈国袁徽以寄寓交州,徽与尚书令荀彧书曰:“许文休英才伟士,智略足以计事。自流宕已来,与群士相随,每有患急,常先人后己,与九族中外同其饥寒。其纪纲同类,仁恕恻隐,皆有效事,不能复一二陈之耳。”钜鹿张翔衔王命使交部,乘势募靖,欲与誓要,靖拒而不许。靖与曹公书曰:

世路戎夷,祸乱遂合,驽怯偷生,自窜蛮貊,成阔十年,吉凶礼废。昔在会稽,得所贻书,辞旨款密,久要不忘。迫於袁术方命圮族,扇动群逆,津涂四塞,虽县心北风,欲行靡由。正礼师退,术兵前进,会稽倾覆,景兴失据,三江五湖,皆为虏庭。临时困厄,无所控告。便与袁沛、邓子孝等浮涉沧海,南至交州。经历东瓯、闽、越之国,行经万里,不见汉地,漂薄风波,绝粮茹草,饥殍荐臻,死者大半。既济南海,与领守儿孝德相见,知足下忠义奋发,整饬元戎,西迎大驾,巡省中岳。承此休问,且悲且憙,即与袁沛及徐元贤复共严装,欲北上荆州。会苍梧诸县夷、越蜂起,州府倾覆,道路阻绝,元贤被害,老弱并杀。靖寻循渚岸五千馀里,复遇疾疠,伯母陨命,并及群从,自诸妻子,一时略尽。复相扶侍,前到此郡,计为兵害及病亡者,十遗一二。生民之艰,辛苦之甚,岂可具陈哉!惧卒颠仆,永为亡虏,忧瘁惨惨,忘寝与食。欲附奉朝贡使,自获济通,归死阙庭,而荆州水陆无津,交部驿使断绝。欲上益州,复有峻防,故官长吏,一不得入。前令交阯太守士威彦,深相分讬於益州兄弟,又靖亦自与书,辛苦恳恻,而复寂寞,未有报应。虽仰瞻光灵,延颈企踵,何由假翼自致哉?

知圣主允明,显授足下专征之任,凡诸逆节,多所诛讨,想力竞者一心,顺从者同规矣。又张子云昔在京师,志匡王室,今虽临荒域,不得参与本朝,亦国家之藩镇,足下之外援也。若荆、楚平和,王泽南至,足下忽有声命於子云,勤见保属,令得假途由荆州出,不然,当复相绍介於益州兄弟,使相纳受。倘天假其年,人缓其祸,得归死国家,解逋逃之负,泯躯九泉,将复何恨!若时有险易,事有利钝,人命无常,陨没不达者,则永衔罪责,入於裔土矣。

昔营邱翼周,杖钺专征,博陆佐汉,虎贲警跸。汉书霍光传曰:“光出都肄郎羽林,道上称警跸。”未详虎贲所出也。今日足下扶危持倾,为国柱石,秉师望之任,兼霍光之重。五侯九伯,制御在手,自古及今,人臣之尊未有及足下者也。夫爵高者忧深,禄厚者责重,足下据爵高之任,当责重之地,言出於口,即为赏罚,意之所存,便为祸福。行之得道,即社稷用宁;行之失道,即四方散乱。国家安危,在於足下;百姓之命,县於执事。自华及夷,颙颙注望。足下任此,岂可不远览载籍废兴之由,荣辱之机,弃忘旧恶,宽和群司,审量五材,为官择人?苟得其人,虽雠必举;苟非其人,虽亲不授。以宁社稷,以济下民,事立功成,则系音於管弦,勒勋於金石,愿君勉之!为国自重,为民自爱。”翔恨靖之不自纳,搜索靖所寄书疏,尽投之于水。

后刘璋遂使使招靖,靖来入蜀。璋以靖为巴郡、广汉太守。南阳宋仲子於荆州与蜀郡太守王商书曰:“文休倜傥瑰玮,有当世之具,足下当以为指南。”建安十六年,转在蜀郡。十九年,先主克蜀,以靖为左将军长史。先主为汉中王,靖为太傅。及即尊号,策靖曰:“朕获奉洪业,君临万国,夙宵惶惶,惧不能绥。百姓不亲,五品不逊,汝作司徒,其敬敷五教,在宽。君其勖哉!秉德无怠,称朕意焉。”

靖虽年逾七十,爱乐人物,诱纳后进,清谈不倦。丞相诸葛亮皆为之拜。章武二年卒。子钦,先靖夭没。钦子游,景耀中为尚书。始靖兄事颍川陈纪,与陈郡袁涣、平原华歆、东海王朗等亲善,歆、朗及纪子群,魏初为公辅大臣,咸与靖书,申陈旧好,情义款至,文多故不载。

麋竺字子仲,东海朐人也。祖世货殖,僮客万人,赀产钜亿。后徐州牧陶谦辟为别驾从事。谦卒,竺奉谦遗命,迎先主於小沛。建安元年,吕布乘先主之出拒袁术,袭下邳,虏先主妻子。先主转军广陵海西,竺於是进妹於先主为夫人,奴客二千,金银货币以助军资;于时困匮,赖此复振。后曹公表竺领嬴郡太守,竺弟芳为彭城相,皆去官,随先主周旋。先主将適荆州,遣竺先与刘表相闻,以竺为左将军从事中郎。益州既平,拜为安汉将军,班在军师将军之右。竺雍容敦雅,而幹翮非所长。是以待之以上宾之礼,未尝有所统御。然赏赐优宠,无与为比。

芳为南郡太守,与关羽共事,而私好携贰,叛迎孙权,羽因覆败。竺面缚请罪,先主慰谕以兄弟罪不相及,崇待如初。竺惭恚发病,岁馀卒。子威,官至虎贲中郎将。威子照,虎骑监。自竺至照,皆便弓马,善射御云。

孙乾字公祐,北海人也。先主领徐州,辟为从事,后随从周旋。先主之背曹公,遣乾自结袁绍,将適荆州,乾又与麋竺俱使刘表,皆如意指。后表与袁尚书,说其兄弟分争之变,曰:“每与刘左将军、孙公祐共论此事,未尝不痛心入骨,相为悲伤也。”其见重如此。先主定益州,乾自从事中郎为秉忠将军,见礼次麋竺,与简雍同等。顷之,卒。

简雍字宪和,涿郡人也。少与先主有旧,随从周旋。先主至荆州,雍与麋竺、孙乾同为从事中郎,常为谈客,往来使命。先主入益州,刘璋见雍,甚爱之。后先主围成都,遣雍往说璋,璋遂与雍同舆而载,出城归命。先主拜雍为昭德将军。优游风议,性简傲跌宕,在先主坐席,犹箕踞倾倚,威仪不肃,自纵適;诸葛亮已下则独擅一榻,项枕卧语,无所为屈。时天旱禁酒,酿者有刑。吏於人家索得酿具,论者欲令与作酒者同罚。雍与先主游观,见一男女行道,谓先主曰:“彼人欲行淫,何以不缚?”先主曰:“卿何以知之?”雍对曰:“彼有其具,与欲酿者同”先主大笑,而原欲酿者。雍之滑稽,皆此类也。

伊籍字机伯,山阳人。少依邑人镇南将军刘表。先主之在荆州,籍常往来自讬。表卒,遂随先主南渡江,从入益州。益州既定,以籍为左将军从事中郎,见待亚於简雍、孙乾等。遣东使於吴,孙权闻其才辩,欲逆折以辞。籍適入拜,权曰:“劳事无道之君乎?”籍既对曰:“一拜一起,未足为劳”籍之机捷,类皆如此,权甚异之。后迁昭文将军,与诸葛亮、法正、刘巴、李严共造蜀科;蜀科之制,由此五人焉。

秦宓字子敕,广汉绵竹人也。少有才学,州郡辟命,辄称疾不往。奏记州牧刘焉,荐儒士任定祖曰:“昔百里、蹇叔以耆艾而定策,甘罗、子奇以童冠而立功,故书美黄发,而易称颜渊,固知选士用能,不拘长幼,明矣。乃者以来,海内察举,率多英隽而遗旧齿,众论不齐,异同相半,此乃承平之翔步,非乱世之急务也。夫欲救危抚乱,脩己以安人,则宜卓荦超伦,与时殊趣,震惊邻国,骇动四方,上当天心,下合人意;天人既和,内省不疚,虽遭凶乱,何忧何惧!昔楚叶公好龙,神龙下之,好伪彻天,何况於真?今处士任安,仁义直道,流名四远,如令见察,则一州斯服。昔汤举伊尹,不仁者远,何武贡二龚,双名竹帛,故贪寻常之高而忽万仞之嵩,乐面前之饰而忘天下之誉,斯诚往古之所重慎也。甫欲凿石索玉,剖蚌求珠,今乃随、和炳然,有如皎日,复何疑哉!诚知昼不操烛,日有馀光,但愚情区区,贪陈所见”

刘璋时,宓同郡王商为治中从事,与宓书曰:“贫贱困苦,亦何时可以终身!卞和衒玉以耀世,宜一来,与州尊相见。”宓答书曰:“昔尧优许由,非不弘也,洗其两耳;楚聘庄周,非不广也,执竿不顾。易曰'确乎其不可拔',夫何衒之有?且以国君之贤,子为良辅,不以是时建萧、张之策,未足为智也。仆得曝背乎陇亩之中,诵颜氏之箪瓢,咏原宪之蓬户,时翱翔於林泽,与沮、溺之等俦,听玄猿之悲吟,察鹤鸣於九皋,安身为乐,无忧为福,处空虚之名,居不灵之龟,知我者希,则我贵矣。斯乃仆得志之秋也,何困苦之戚焉!”后商为严君平、李弘立祠,宓与书曰:“疾病伏匿,甫知足下为严、李立祠,可谓厚党勤类者也。观严文章,冠冒天下,由、夷逸操,山岳不移,使扬子不叹,固自昭明。如李仲元不遭法言,令名必沦,其无虎豹之文故也,可谓攀龙附凤者矣。如扬子云潜心著述,有补於世,泥蟠不滓,行参圣师,于今海内,谈咏厥辞。邦有斯人,以耀四远,怪子替兹,不立祠堂。蜀本无学士,文翁遣相如东受七经,还教吏民,於是蜀学比於齐、鲁。故地里志曰:‘文翁倡其教,相如为之师。’汉家得士,盛於其世;仲舒之徒,不达封禅,相如制其礼。夫能制礼造乐,移风易俗,非礼所秩有益於世者乎!虽有王孙之累,犹孔子大齐桓之霸,公羊贤叔术之让。仆亦善长卿之化,宜立祠堂,速定其铭”

先是,李权从宓借战国策,宓曰:“战国从横,用之何为?”权曰:“仲尼、严平,会聚众书,以成春秋、指归之文,故海以合流为大,君子以博识为弘。”宓报曰:“书非史记周图,仲尼不采;道非虚无自然,严平不演。海以受淤,岁一荡清;君子博识,非礼不视。今战国反覆仪、秦之术,杀人自生,亡人自存,经之所疾。故孔子发愤作春秋,大乎居正,复制孝经,广陈德行。杜渐防萌,预有所抑,是以老氏绝祸於未萌,岂不信邪!成汤大圣,睹野鱼而有猎逐之失,定公贤者,见女乐而弃朝事,若此辈类,焉可胜陈。道家法曰:‘不见所欲,使心不乱。’是故天地贞观,日月贞明;其直如矢,君子所履。洪范记灾,发於言貌,何战国之谲权乎哉!”

或谓宓曰:“足下欲自比於巢、许、四皓,何故扬文藻见瑰颖乎?”宓答曰:“仆文不能尽言,言不能尽意,何文藻之有扬乎!昔孔子三见哀公,言成七卷,事盖有不可嘿嘿也。接舆行且歌,论家以光篇;渔父咏沧浪,贤者以耀章。此二人者,非有欲於时者也。夫虎生而文炳,凤生而五色,岂以五采自饰画哉?天性自然也。盖河、洛由文兴,六经由文起,君子懿文德,采藻其何伤!以仆之愚,犹耻革子成之误,况贤於己者乎!”

先主既定益州,广汉太守夏侯纂请宓为师友祭酒,领五官掾,称曰仲父。宓称疾,卧在第舍,纂将功曹古朴、主簿王普,厨膳即宓第宴谈,宓卧如故。纂问朴曰:“至於贵州养生之具,实绝馀州矣,不知士人何如馀州也?”朴对曰:“乃自先汉以来,其爵位者或不如馀州耳,至於著作为世师式,不负於馀州也。严君平见黄、老作指归,扬雄见易作太玄,见论语作法言,司马相如为武帝制封禅之文,于今天下所共闻也。”纂曰:“仲父何如?”宓以簿击颊,曰:“愿明府勿以仲父之言假於小草,民请为明府陈其本纪。蜀有汶阜之山,江出其腹,帝以会昌,神以建福,故能沃野千里。淮、济四渎,江为其首,此其一也。禹生石纽,今之汶山郡是也。昔尧遭洪水,鲧所不治,禹疏江决河,东注于海,为民除害,生民已来功莫先者,此其二也。天帝布治房心,决政参伐,参伐则益州分野,三皇乘祗车出谷口,今之斜谷是也。此便鄙州之阡陌,明府以雅意论之,何若於天下乎?”於是纂逡巡无以复答。

益州辟宓为从事祭酒。先主既称尊号,将东征吴,宓陈天时必无其利,坐下狱幽闭,然后贷出。建兴二年,丞相亮领益州牧,选宓迎为别驾,寻拜左中郎将、长水校尉。吴遣使张温来聘,百官皆往饯焉。众人皆集而宓未往,亮累遣使促之,温曰:“彼何人也?”亮曰:“益州学士也。”及至,温问曰:“君学乎?”宓曰:“五尺童子皆学,何必小人!”温复问曰:“天有头乎?”宓曰:“有之。“温曰:“在何方也?“宓曰:“在西方。诗曰:'乃眷西顾。'以此推之,头在西方。“温曰:“天有耳乎?”宓曰:“天处高而听卑,诗云:'鹤鸣于九皋,声闻于天。'若其无耳,何以听之?”温曰:“天有足乎?”宓曰:“有。诗云:'天步艰难,之子不犹。'若其无足,何以步之?“温曰:“天有姓乎?“宓曰:“有。”温曰:“何姓?”宓曰:“姓刘。”温曰:“何以知之?”答曰:“天子姓刘,故以此知之。”温曰:“日生於东乎?”宓曰:“虽生于东而没於西。”答问如响,应声而出,於是温大敬服。宓之文辩,皆此类也。迁大司农,四年卒。初宓见帝系之文,五帝皆同一族,宓辨其不然之本。又论皇帝王霸豢龙之说,甚有通理。谯允南少时数往谘访,纪录其言於春秋然否论,文多故不载。

评曰:许靖夙有名誉,既以笃厚为称,又以人物为意,虽行事举动,未悉允当,蒋济以为“大较廊庙器“也。麋竺、孙乾、简雍、伊籍,皆雍容风议,见礼於世。秦宓始慕肥遯之高,而无若愚之实。然专对有馀,文藻壮美,可谓一时之才士矣。

\part{蜀书九}
\chapter{董刘马陈董吕传第九}

董和字幼宰,南郡枝江人也,其先本巴郡江州人。汉末,和率宗族西迁,益州牧刘璋以为牛鞞、江原长、成都令。蜀土富实,时俗奢侈,货殖之家,侯服玉食,婚姻葬送,倾家竭产。和躬率以俭,恶衣蔬食,防遏逾僣,为之轨制,所在皆移风变善,畏而不犯。然县界豪强惮和严法,说璋转和为巴东属国都尉。吏民老弱相携乞留和者数千人,璋听留二年,还迁益州太守,其清约如前。与蛮夷从事,务推诚心,南土爱而信之。

先主定蜀,徵和为掌军中郎将,与军师将军诸葛亮并署左将军大司马府事,献可替否,共为欢交。自和居官食禄,外牧殊域,内幹机衡,二十馀年,死之日家无儋石之财。亮后为丞相,教与群下曰:“夫参署者,集众思广忠益也。若远小嫌,难相违覆,旷阙损矣。违覆而得中,犹弃弊蹻而获珠玉。然人心苦不能尽,惟徐元直处兹不惑,又董幼宰参署七年,事有不至,至于十反,来相启告。苟能慕元直之十一,幼宰之殷勤,有忠於国,则亮可少过矣。“又曰:“昔初交州平,屡闻得失,后交元直,勤见启诲,前参事於幼宰,每言则尽,后从事於伟度,数有谏止;虽姿性鄙暗,不能悉纳,然与此四子终始好合,亦足以明其不疑於直言也。“其追思和如此。

刘巴字子初,零陵烝阳人也。少知名,荆州牧刘表连辟,及举茂才,皆不就。表卒,曹公征荆州。先主奔江南,荆、楚群士从之如云,而巴北诣曹公。曹公辟为掾,使招纳长沙、零陵、桂阳。会先主略有三郡,巴不得反使,遂远適交阯,先主深以为恨。

巴复从交阯至蜀。俄而先主定益州,巴辞谢罪负,先主不责。而诸葛孔明数称荐之,先主辟为左将军西曹掾。建安二十四年,先主为汉中王,巴为尚书,后代法正为尚书令。躬履清俭,不治产业,又自以归附非素,惧见猜嫌,恭默守静,退无私交,非公事不言。先主称尊号,昭告于皇天上帝后土神祇,凡诸文诰策命,皆巴所作也。章武二年卒。卒后,魏尚书仆射陈群与丞相诸葛亮书,问巴消息,称曰刘君子初,甚敬重焉。

马良字季常,襄阳宜城人也。兄弟五人,并有才名,乡里为之谚曰:“马氏五常,白眉最良。“良眉中有白毛,故以称之。先主领荆州,辟为从事。及先主入蜀,诸葛亮亦从后往,良留荆州,与亮书曰:“闻雒城已拔,此天祚也。尊兄应期赞世,配业光国,魄兆见矣。夫变用雅虑,审贵垂明,於以简才,宜適其时。若乃和光悦远,迈德天壤,使时闲於听,世服於道,齐高妙之音正郑、卫之声,并利於事,无相夺伦,此乃管弦之至,牙、旷之调也。虽非锺期,敢不击节!“先主辟良为左将军掾。

后遣使吴,良谓亮曰:“今衔国命,协穆二家,幸为良介於孙将军。“亮曰:“君试自为文。“良即为草曰:“寡君遣掾马良通聘继好,以绍昆吾、豕韦之勋。其人吉士,荆楚之令,鲜於造次之华,而有克终之美,愿降心存纳,以慰将命。“权敬待之。

先主称尊号,以良为侍中。及东征吴,遣良入武陵招纳五溪蛮夷,蛮夷渠帅皆受印号,咸如意指。会先主败绩於夷陵,良亦遇害。先主拜良子秉为骑都尉。

良弟谡,字幼常,以荆州从事随先主入蜀,除绵竹成都令、越隽太守。才器过人,好论军计,丞相诸葛亮深加器异。先主临薨谓亮曰:“马谡言过其实,不可大用,君其察之!“亮犹谓不然,以谡为参军,每引见谈论,自昼达夜。

建兴六年,亮出军向祁山,时有宿将魏延、吴壹等,论者皆言以为宜令为先锋,而亮违众拔谡,统大众在前,与魏将张郃战于街亭,为郃所破,士卒离散。亮进无所据,退军还汉中。谡下狱物故,亮为之流涕。良死时年三十六,谡年三十九。

陈震字孝起,南阳人也。先主领荆州牧,辟为从事,部诸郡,随先主入蜀。蜀既定,为蜀郡北部都尉,因易郡名,为汶山太守,转在犍为。建兴三年,入拜尚书,迁尚书令,奉命使吴。七年,孙权称尊号,以震为卫尉,贺权践阼,诸葛亮与兄瑾书曰:“孝起忠纯之性,老而益笃,及其赞述东西,欢乐和合,有可贵者。“震入吴界,移关候曰:“东之与西,驿使往来,冠盖相望,申盟初好,日新其事。东尊应保圣祚,告燎受符,剖判土宇,天下响应,各有所归。於此时也,以同心讨贼,则何寇不灭哉!西朝君臣,引领欣赖。震以不才,得充下使,奉聘叙好,践界踊跃,入则如归。献子適鲁,犯其山讳,春秋讥之。望必启告,使行人睦焉。即日张旍诰众,各自约誓。顺流漂疾,国典异制,惧或有违,幸必斟诲,示其所宜。“震到武昌,孙权与震升坛歃盟,交分天下:以徐、豫、幽、青属吴,并、凉、冀、兖属蜀,其司州之土,以函谷关为界。震还,封城阳亭侯。九年,都护李平坐诬罔废;诸葛亮与长史蒋琬、侍中董允书曰:“孝起前临至吴,为吾说正方腹中有鳞甲,乡党以为不可近。吾以为鳞甲者但不当犯之耳,不图复有苏、张之事出於不意。可使孝起知之。“十三年,震卒。子济嗣。

董允字休昭,掌军中郎将和之子也。先主立太子,允以选为舍人,徙洗马。后主袭位,迁黄门侍郎。丞相亮将北征,住汉中,虑后主富於春秋,朱紫难别,以允秉心公亮,欲任以宫省之事。上疏曰:“侍中郭攸之、费祎、侍郎董允等,先帝简拔以遗陛下,至於斟酌规益,进尽忠言,则其任也。愚以为宫中之事,事无大小,悉以咨之,必能裨补阙漏,有所广益。若无兴德之言,则戮允等以彰其慢。“亮寻请祎为参军,允迁为侍中,领虎贲中郎将,统宿卫亲兵。攸之性素和顺,备员而已。献纳之任,允皆专之矣。允处事为防制,甚尽匡救之理。后主常欲采择以充后宫,允以为古者天子后妃之数不过十二,今嫔嫱已具,不宜增益,终执不听。后主益严惮之。尚书令蒋琬领益州刺史,上疏以让费祎及允,又表“允内侍历年,翼赞王室,宜赐爵土以褒勋劳。“允固辞不受。后主渐长大,爱宦人黄皓。皓便辟佞慧,欲自容入。允常上则正色匡主,下则数责於皓。皓畏允,不敢为非。终允之世,皓位不过黄门丞。

允尝与尚书令费祎、中典军胡济等共期游宴,严驾已办,而郎中襄阳董恢诣允脩敬。恢年少官微,见允停出,逡巡求去,允不许,曰:“本所以出者,欲与同好游谈也,今君已自屈,方展阔积,舍此之谈,就彼之宴,非所谓也。“乃命解骖,祎等罢驾不行。其守正下士,凡此类也。延熙六年,加辅国将军。七年,以侍中守尚书令,为大将军费祎副贰。九年,卒。

陈祗代允为侍中,与黄皓互相表里,皓始预政事。祗死后,皓从黄门令为中常侍、奉车都尉,操弄威柄,终至覆国。蜀人无不追思允。及邓艾至蜀,闻皓奸险,收闭,将杀之,而皓厚赂艾左右,得免。

祗字奉宗,汝南人,许靖兄之外孙也。少孤,长於靖家。弱冠知名,稍迁至选曹郎,矜厉有威容。多技艺,挟数术,费祎甚异之,故超继允内侍。吕乂卒,祗又以侍中守尚书令,加镇军将军,大将军姜维虽班在祗上,常率众在外,希亲朝政。祗上承主指,下接阉竖,深见信爱,权重於维。景耀元年卒,后主痛惜,发言流涕,乃下诏曰:“祗统职一纪,柔嘉惟则,幹肃有章,和义利物,庶绩允明。命不融远,朕用悼焉。夫存有令问,则亡加美谥,谥曰忠侯。“赐子粲爵关内侯,拔次子裕为黄门侍郎。自祗之有宠,后主追怨允日深,谓为自轻,由祗媚兹一人,皓构间浸润故耳。允孙宏,晋巴西太守。

吕乂字季阳,南阳人也。父常,送故将【军】刘焉入蜀,值王路隔塞,遂不得还。乂少孤,好读书鼓琴。初,先主定益州,置盐府校尉,较盐铁之利,后校尉王连请乂及南阳杜祺、南乡刘幹等并为典曹都尉。乂迁新都、绵竹令,乃心隐恤,百姓称之,为一州诸城之首。迁巴西太守。丞相诸葛亮连年出军,调发诸郡,多不相救,乂募取兵五千人诣亮,慰喻检制,无逃窜者。徙为汉中太守,兼领督农,供继军粮。亮卒,累迁广汉、蜀郡太守。蜀郡一都之会,户口众多,又亮卒之后,士伍亡命,更相重冒,奸巧非一。乂到官,为之防禁,开喻劝导,数年之中,漏脱自出者万馀口。后入为尚书,代董允为尚书令,众事无留,门无停宾。乂历职内外,治身俭约,谦靖少言,为政简而不烦,号为清能;然持法刻深,好用文俗吏,故居大官,名声损於郡县。延熙十四年卒。子辰,景耀中为成都令。辰弟雅,谒者。雅清厉有文才,著格论十五篇。

杜祺历郡守监军大将军司马,刘幹官至巴西太守,皆与乂亲善,亦有当时之称,而俭素守法,不及於乂。

评曰:董和蹈羔羊之素,刘巴履清尚之节,马良贞实,称为令士,陈震忠恪,老而益笃,董允匡主,义形於色,皆蜀臣之良矣。吕乂临郡则垂称,处朝则被损,亦黄、薛之流亚矣。

\part{蜀书十}
\chapter{刘彭廖李刘魏杨传第十}

刘封者,本罗侯寇氏之子,长沙刘氏之甥也。先主至荆州,以未有继嗣,养封为子。及先主入蜀,自葭萌还攻刘璋,时封年二十馀,有武艺,气力过人,将兵俱与诸葛亮、张飞等溯流西上,所在战克。益州既定,以封为副军中郎将。

初,刘璋遣扶风孟达副法正,各将兵二千人,使迎先主,先主因令达并领其众,留屯江陵。蜀平后,以达为宜都太守。建安二十四年,命达从秭归北攻房陵,房陵太守蒯祺为达兵所害。达将进攻上庸,先主阴恐达难独任,乃遣封自汉中乘沔水下统达军,与达会上庸。上庸太守申耽举众降,遣妻子及宗族诣成都。先主加耽征北将军,领上庸太守员乡侯如故,以耽弟仪为建信将军、西城太守,迁封为副军将军。自关羽围樊城、襄阳,连呼封、达,令发兵自助。封、达辞以山郡初附,未可动摇,不承羽命。会羽覆败,先主恨之。又封与达忿争不和,封寻夺达鼓吹。达既惧罪,又忿恚封,遂表辞先主,率所领降魏。魏文帝善达之姿才容观,以为散骑常侍、建武将军,封平阳亭侯。合房陵、上庸、西城三郡为新城郡,以达领新城太守。遣征南将军夏侯尚、右将军徐晃与达共袭封。达与封书曰:

古人有言:‘不间亲,新不加旧。’此谓上明下直,谗慝不行也。若乃权君谲主,贤父慈亲,犹有忠臣蹈功以罹祸,孝子抱仁以陷难,种、商、白起、孝己、伯奇,皆其类也。其所以然,非骨肉好离,亲亲乐患也。或有恩移爱易,亦有谗间其间,虽忠臣不能移之於君,孝子不能变之於父者也。势利所加,改亲为雠,况非亲亲乎!故申生、卫伋、御寇、楚建禀受形之气,当嗣立之正,而犹如此。今足下与汉中王,道路之人耳,亲非骨血而据势权,义非君臣而处上位,征则有偏任之威,居则有副军之号,远近所闻也。自立阿斗为太子已来,有识之人相为寒心。如使申生从子舆之言,必为太伯;卫伋听其弟之谋,无彰父之讥也。且小白出奔,入而为霸;重耳逾垣,卒以克复。自古有之,非独今也。

夫智贵免祸,明尚夙达,仆揆汉中王虑定於内,疑生於外矣;虑定则心固,疑生则心惧,乱祸之兴作,未曾不由废立之间也。私怨人情,不能不见,恐左右必有以间於汉中王矣。然则疑成怨闻,其发若践机耳。今足下在远,尚可假息一时;若大军遂进,足下失据而还,窃相为危之。昔微子去殷,智果别族,违难背祸,犹皆如斯。今足下弃父母而为人后,非礼也;知祸将至而留之,非智也;见正不从而疑之,非义也。自号为丈夫,为此三者,何所贵乎?以足下之才,弃身来东,继嗣罗侯,不为背亲也;北面事君,以正纲纪,不为弃旧也;怒不致乱,以免危亡,不为徒行也。加陛下新受禅命,虚心侧席,以德怀远,若足下翻然内向,非但与仆为伦,受三百户封,继统罗国而已,当更剖符大邦,为始封之君。陛下大军,金鼓以震,当转都宛、邓;若二敌不平,军无还期。足下宜因此时早定良计。易有‘利见大人’,诗有‘自求多福’,行矣。今足下勉之,无使狐突闭门不出。封不从达言。

申仪叛封,封破走还成都。申耽降魏,魏假耽怀集将军,徙居南阳,仪魏兴太守,封员乡侯,屯洵口。封既至,先主责封之侵陵达,又不救羽。诸葛亮虑封刚猛,易世之后终难制御,劝先主因此除之。於是赐封死,使自裁。封叹曰:“恨不用孟子度之言!”先主为之流涕。达本字子敬,避先主叔父敬,改之。

彭羕字永年,广汉人。身长八尺,容貌甚伟。姿性骄傲,多所轻忽,惟敬同郡秦子敕,荐之於太守许靖曰:“昔高宗梦傅说,周文求吕尚,爰及汉祖,纳食其於布衣,此乃帝王之所以倡业垂统,缉熙厥功也。今明府稽古皇极,允执神灵,体公刘之德,行勿翦之惠,清庙之作於是乎始,褒贬之义於是乎兴,然而六翮未之备也。伏见处士绵竹秦宓,膺山甫之德,履隽生之直,枕石漱流,吟咏缊袍,偃息於仁义之途,恬淡於浩然之域,高概节行,守真不亏,虽古人潜遁,蔑以加旃。若明府能招致此人,必有忠谠落落之誉,丰功厚利,建迹立勋,然后纪功於王府,飞声於来世,不亦美哉!”

羕仕州,不过书佐,后又为众人所谤毁於州牧刘璋,璋髡钳羕为徒隶。会先主入蜀,溯流北行。羕欲纳说先主,乃往见庞统。统与羕非故人,又適有宾客,羕径上统床卧,谓统曰:“须客罢当与卿善谈。”统客既罢,往就羕坐,羕又先责统食,然后共语,因留信宿,至于经日。统大善之,而法正宿自知羕,遂并致之先主。先主亦以为奇,数令羕宣传军事,指授诸将,奉使称意,识遇日加。成都既定,先主领益州牧,拔羕为治中从事。羕起徒步,一朝处州人之上,形色嚣然,自矜得遇滋甚。诸葛亮虽外接待羕,而内不能善。屡密言先主,羕心大志广,难可保安。先主既敬信亮,加察羕行事,意以稍疏,左迁羕为江阳太守。

羕闻当远出,私情不悦,往诣马超。超问羕曰:“卿才具秀拔,主公相待至重,谓卿当与孔明、孝直诸人齐足并驱,宁当外授小郡,失人本望乎?“羕曰:“老革荒悖,可复道邪!”又谓超曰:“卿为其外,我为其内,天下不足定也。”超羁旅归国,常怀危惧,闻羕言大惊,默然不答。羕退,具表羕辞,於是收羕付有司。

羕於狱中与诸葛亮书曰:“仆昔有事於诸侯,以为曹操暴虐,孙权无道,振威闇弱,其惟主公有霸王之器,可与兴业致治,故乃翻然有轻举之志。会公来西,仆因法孝直自衒鬻,庞统斟酌其间,遂得诣公於葭萌,指掌而谭,论治世之务,讲霸王之义,建取益州之策,公亦宿虑明定,即相然赞,遂举事焉。仆於故州不免凡庸,忧於罪罔,得遭风云激矢之中,求君得君,志行名显,从布衣之中擢为国士,盗窃茂才。分子之厚,谁复过此。羕一朝狂悖,自求菹醢,为不忠不义之鬼乎!先民有言,左手据天下之图,右手刎咽喉,愚夫不为也。况仆颇别菽麦者哉!所以有怨望意者,不自度量,苟以为首兴事业,而有投江阳之论,不解主公之意,意卒感激,颇以被酒,侻失'老'语。此仆之下愚薄虑所致,主公实未老也。且夫立业,岂在老少,西伯九十,宁有衰志,负我慈父,罪有百死。至於内外之言,欲使孟起立功北州,戮力主公,共讨曹操耳,宁敢有他志邪?孟起说之是也,但不分别其间,痛人心耳。昔每与庞统共相誓约,庶讬足下末踪,尽心於主公之业,追名古人,载勋竹帛。统不幸而死,仆败以取祸。自我堕之,将复谁怨!足下,当世伊、吕也,宜善与主公计事,济其大猷。天明地察,神祇有灵,复何言哉!贵使足下明仆本心耳。行矣努力,自爱,自爱!”羕竟诛死,时年三十七。

廖立字公渊,武陵临沅人。先主领荆州牧,辟为从事,年未三十,擢为长沙太守。先主入蜀,诸葛亮镇荆土,孙权遣使通好於亮,因问士人皆谁相经纬者,亮答曰:“庞统、廖立,楚之良才,当赞兴世业者也。”建安二十年,权遣吕蒙奄袭南三郡,立脱身走,自归先主。先主素识待之,不深责也,以为巴郡太守。二十四年,先主为汉中王,徵立为侍中。后主袭位,徙长水校尉。

立本意,自谓才名宜为诸葛亮之贰,而更游散在李严等下,常怀怏怏。后丞相掾李邵、蒋琬至,立计曰:“军当远出,卿诸人好谛其事。昔先帝不取汉中,走与吴人争南三郡,卒以三郡与吴人,徒劳役吏士,无益而还。既亡汉中,使夏侯渊、张郃深入于巴,几丧一州。后至汉中,使关侯身死无孑遗,上庸覆败,徒失一方。是羽怙恃勇名,作军无法,直以意突耳,故前后数丧师众也。如向朗、文恭,凡俗之人耳。恭作治中无纲纪;朗昔奉马良兄弟,谓为圣人,今作长史,素能合道。中郎郭演长,从人者耳,不足与经大事,而作侍中。今弱世也,欲任此三人,为不然也。王连流俗,苟作掊克,使百姓疲弊,以致今日。”邵、琬具白其言於诸葛亮。亮表立曰:“长水校尉廖立,坐自贵大,臧否群士,公言国家不任贤达而任俗吏,又言万人率者皆小子也;诽谤先帝,疵毁众臣。人有言国家兵众简练,部伍分明者,立举头视屋,愤咤作色曰:‘足言!’凡如是者不可胜数。羊之乱群,犹能为害,况立讬在大位,中人以下识真伪邪?”於是废立为民,徙汶山郡。立躬率妻子耕殖自守,闻诸葛亮卒,垂泣叹曰:“吾终为左衽矣!”后监军姜维率偏军经汶山、诣立,称立意气不衰,言论自若。立遂终徙所。妻子还蜀。

李严字正方,南阳人也。少为郡职吏,以才幹称。荆州牧刘表使历诸郡县。曹公入荆州时,严宰秭归,遂西诣蜀,刘璋以为成都令,复有能名。建安十八年,署严为护军,拒先主於绵竹。严率众降先主,先主拜严裨将军。成都既定,为犍为太守、兴业将军。二十三年,盗贼马秦、高胜等起事於郪,合聚部伍数万人,到资中县。时先主在汉中,严不更发兵,但率将郡士五千人讨之,斩秦、胜等首。枝党星散,悉复民籍。又越巂夷率高定遣军围新道县,严驰往赴救,贼皆破走。加辅汉将军,领郡如故。章武二年,先主徵严诣永安宫,拜尚书令。三年,先主疾病,严与诸葛亮并受遗诏辅少主;以严为中都护,统内外军事,留镇永安。建兴元年,封都乡侯,假节,加光禄勋。四年,转为前将军。以诸葛亮欲出军汉中,严当知后事,移屯江州,留护军陈到驻永安,皆统属严。严与孟达书曰:“吾与孔明俱受寄讬,忧深责重,思得良伴。”亮亦与达书曰:“部分如流,趋舍罔滞,正方性也。”其见贵重如此。八年,迁骠骑将军。以曹真欲三道向汉川,亮命严将二万人赴汉中。亮表严子丰为江州都督督军,典严后事。亮以明年当出军,命严以中都护署府事。严改名为平。

九年春,亮军祁山,平催督运事。秋夏之际,值天霖雨,运粮不继,平遣参军狐忠、督军成藩喻指,呼亮来还;亮承以退军。平闻军退,乃更阳惊,说“军粮饶足,何以便归“!欲以解己不办之责,显亮不进之愆也。又表后主,说“军伪退,欲以诱贼与战”。亮具出其前后手笔书疏本末,平违错章灼。平辞穷情竭,首谢罪负。於是亮表平曰:“自先帝崩后,平所在治家,尚为小惠,安身求名,无忧国之事。臣当北出,欲得平兵以镇汉中,平穷难纵横,无有来意,而求以五郡为巴州刺史。去年臣欲西征,欲令平主督汉中,平说司马懿等开府辟召。臣知平鄙情,欲因行之际偪臣取利也,是以表平子丰督主江州,隆崇其遇,以取一时之务。平至之日,都委诸事,群臣上下皆怪臣待平之厚也。正以大事未定,汉室倾危,伐平之短,莫若褒之。然谓平情在於荣利而已,不意平心颠倒乃尔。若事稽留,将致祸败,是臣不敏,言多增咎。”乃废平为民,徙梓潼郡。十二年,平闻亮卒,发病死。平常冀亮当自补复,策后人不能,故以激愤也。丰官至朱提太守。

刘琰字威硕,鲁国人也。先主在豫州,辟为从事,以其宗姓,有风流,善谈论,厚亲待之,遂随从周旋,常为宾客。先主定益州,以琰为固陵太守。后主立,封都乡侯,班位每亚李严,为卫尉中军师后将军,迁车骑将军。然不豫国政,但领兵千馀,随丞相亮讽议而已。车服饮食,号为侈靡,侍婢数十,皆能为声乐,又悉教诵读鲁灵光殿赋。建兴十年,与前军师魏延不和,言语虚诞,亮责让之。琰与亮笺谢曰:“琰禀性空虚,本薄操行,加有酒荒之病,自先帝以来,纷纭之论,殆将倾覆。颇蒙明公本其一心在国,原其身中秽垢,扶持全济,致其禄位,以至今日。间者迷醉,言有违错,慈恩含忍,不致之于理,使得全完,保育性命。虽必克己责躬,改过投死,以誓神灵;无所用命,则靡寄颜。”於是亮遣琰还成都,官位如故。

琰失志慌惚。十二年正月,琰妻胡氏入贺太后,太后令特留胡氏,经月乃出。胡氏有美色,琰疑其与后主有私,呼五百挝胡,至於以履搏面,而后弃遣。胡具以告言琰,琰坐下狱。有司议曰:“卒非挝妻之人,面非受履之地。”琰竟弃市。自是大臣妻母朝庆遂绝。

魏延字文长,义阳人也。以部曲随先主入蜀,数有战功,迁牙门将军。先主为汉中王,迁治成都,当得重将以镇汉川,众论以为必在张飞,飞亦以心自许。先主乃拔延为督汉中镇远将军,领汉中太守,一军尽惊。先主大会群臣,问延曰:“今委卿以重任,卿居之欲云何?”延对曰:“若曹操举天下而来,请为大王拒之;偏将十万之众至,请为大王吞之。“先主称善,众咸壮其言。先主践尊号,进拜镇北将军。建兴元年,封都亭侯。五年,诸葛亮驻汉中,更以延为督前部,领丞相司马、凉州刺史,八年,使延西入羌中,魏后将军费瑶、雍州刺史郭淮与延战于阳谿,延大破淮等,迁为前军师征西大将军,假节,进封南郑侯。

延每随亮出,辄欲请兵万人,与亮异道会于潼关,如韩信故事,亮制而不许。延常谓亮为怯,叹恨己才用之不尽。延既善养士卒,勇猛过人,又性矜高,当时皆避下之。唯杨仪不假借延,延以为至忿,有如水火。十二年,亮出北谷口,延为前锋。出亮营十里,延梦头上生角,以问占梦赵直,直诈延曰:“夫麒麟有角而不用,此不战而贼欲自破之象也。”退而告人曰:“角之为字,刀下用也;头上用刀,其凶甚矣。”

秋,亮病困,密与长史杨仪、司马费祎、护军姜维等作身殁之后退军节度,令延断后,姜维次之;若延或不从命,军便自发。亮適卒,秘不发丧,仪令祎往揣延意指。延曰:“丞相虽亡,吾自见在。府亲官属便可将丧还葬,吾自当率诸军击贼,云何以一人死废天下之事邪?且魏延何人,当为杨仪所部勒,作断后将乎!”因与祎共作行留部分,令祎手书与己连名,告下诸将。祎绐延曰:“当为君还解杨长史,长史文吏,稀更军事,必不违命也。”祎出门驰马而去,延寻悔,追之已不及矣。延遣人觇仪等,遂使欲案亮成规,诸营相次引军还。延大怒,【才】搀仪未发,率所领径先南归,所过烧绝阁道。延、仪各相表叛逆,一日之中,羽檄交至。后主以问侍中董允、留府长史蒋琬,琬、允咸保仪疑延。仪等槎山通道,昼夜兼行,亦继延后。延先至,据南谷口,遣兵逆击仪等,仪等令何平在前御延。平叱延先登曰:“公亡,身尚未寒,汝辈何敢乃尔!”延士众知曲在延,莫为用命,军皆散。延独与其子数人逃亡,奔汉中。仪遣马岱追斩之,致首於仪,仪起自踏之,曰:“庸奴!复能作恶不”遂夷延三族。初,蒋琬率宿卫诸营赴难北行,行数十里,延死问至,乃旋。原延意不北降魏而南还者,但欲除杀仪等。平日诸将素不同,冀时论必当以代亮。本指如此。不便背叛。

杨仪字威公,襄阳人也。建安中,为荆州刺史傅群主簿,背群而诣襄阳太守关羽。羽命为功曹,遣奉使西诣先主。先主与语论军国计策,政治得失,大悦之,因辟为左将军兵曹掾。及先主为汉中王,拔仪为尚书。先主称尊号,东征吴,仪与尚书令刘巴不睦,左迁遥署弘农太守。建兴三年,丞相亮以为参军,署府事,将南行。五年,随亮汉中。八年,迁长史,加绥军将军。亮数出军,仪常规画分部,筹度粮谷,不稽思虑,斯须便了。军戎节度,取办於仪。亮深惜仪之才幹,凭魏延之骁勇,常恨二人之不平,不忍有所偏废也。十二年,随亮出屯谷口。亮卒于敌场。仪既领军还,又诛讨延,自以为功勋至大,宜当代亮秉政,呼都尉赵正以周易筮之,卦得家人,默然不悦。而亮平生密指,以仪性狷狭,意在蒋琬,琬遂为尚书令、益州刺史。仪至,拜为中军师,无所统领,从容而已。

初,仪为先主尚书,琬为尚书郎,后虽俱为丞相参军长史,仪每从行,当其劳剧,自惟年宦先琬,才能逾之,於是怨愤形于声色,叹咤之音发於五内。时人畏其言语不节,莫敢从也,惟后军师费祎往慰省之。仪对祎恨望,前后云云,又语祎曰:“往者丞相亡没之际,吾若举军以就魏氏,处世宁当落度如此邪!令人追悔不可复及。”祎密表其言。十三年,废仪为民,徙汉嘉郡。仪至徙所,复上书诽谤,辞指激切,遂下郡收仪。仪自杀,其妻子还蜀。

评曰:刘封处嫌疑之地,而思防不足以自卫。彭羕、廖立以才拔进,李严以幹局达,魏延以勇略任,杨仪以当官显,刘琰旧仕,并咸贵重。览其举措,迹其规矩,招祸取咎,无不自己也。

\part{蜀书十一}
\chapter{霍王向张杨费传第十一}

霍峻字仲邈,南郡枝江人也。兄笃,於乡里合部曲数百人。笃卒,荆州牧刘表令峻摄其众。表卒,峻率众归先主,先主以峻为中郎将。先主自葭萌南还袭刘璋,留峻守葭萌城。张鲁遣将杨帛诱峻,求共守城,峻曰:“小人头可得,城不可得。”帛乃退去。后璋将扶禁、向存等帅万馀人由阆水上,攻围峻,且一年,不能下。峻城中兵才数百人,伺其怠隙,选精锐出击,大破之,即斩存首。先主定蜀,嘉峻之功,乃分广汉为梓潼郡,以峻为梓潼太守、裨将军。在官三年,年四十卒,还葬成都。先主甚悼惜,乃诏诸葛亮曰:“峻既佳士,加有功於国,欲行酹。”遂亲率群僚临会吊祭,因留宿墓上,当时荣之。

子弋,字绍先,先主末年为太子舍人。后主践阼,除谒者。丞相诸葛亮北驻汉中,请为记室,使与子乔共周旋游处。亮卒,为黄门侍郎。后主立太子璿,以弋为中庶子,璿好骑射,出入无度,弋援引古义,尽言规谏,甚得切磋之体。后为参军庲降屯副贰都督,又转护军,统事如前。时永昌郡夷獠恃险不宾,数为寇害,乃以弋领永昌太守,率偏军讨之,遂斩其豪帅,破坏邑落,郡界宁静。迁监军翊军将军,领建宁太守,还统南郡事。景耀六年,进号安南将军。是岁,蜀并于魏。弋与巴东领军襄阳罗宪各保全一方,举以内附,咸因仍前任,宠待有加。

王连字文仪,南阳人也。刘璋时入蜀,为梓潼令。先主起事葭萌,进军来南,连闭城不降,先主义之,不强偪也。及成都既平,以连为什邡令,转在广都,所居有绩。迁司盐校尉,较盐铁之利,利入甚多,有裨国用,於是简取良才以为官属,若吕乂、杜祺、刘幹等,终皆至大官,自连所拔也。迁蜀郡太守、兴业将军,领盐府如故。建兴元年,拜屯骑校尉,领丞相长史,封平阳亭侯。时南方诸郡不宾,诸葛亮将自征之,连谏以为“此不毛之地,疫疠之乡,不宜以一国之望,冒险而行”。亮虑诸将才不及己,意欲必往,而连言辄恳至,故停留者久之。会连卒。子山嗣,官至江阳太守。

向朗字巨达,襄阳宜城人也。荆州牧刘表以为临沮长。表卒,归先主。先主定江南,使朗督秭归、夷道、巫、夷陵四县军民事。蜀既平,以朗为巴西太守,顷之转任牂牁,又徙房陵。后主践阼,为步兵校尉,代王连领丞相长史。丞相亮南征,朗留统后事。五年,随亮汉中。朗素与马谡善,谡逃亡,朗知情不举,亮恨之,免官还成都。数年,为光禄勋,亮卒后徒左将军,追论旧功,封显明亭侯,位特进。初,朗少时虽涉猎文学,然不治素检,以吏能见称。自去长史,优游无事垂三十乃更潜心典籍,孜孜不倦。年逾八十,犹手自校书,刊定谬误,积聚篇卷,於时最多。开门接宾,诱纳后进,但讲论古义,不干时事,以是见称。上自执政,下及童冠,皆敬重焉。延熙十年卒。子条嗣,景耀中为御史中丞。

朗兄子宠,先主时为牙门将。秭归之败,宠营特完。建兴元年封都亭侯,后为中部督,典宿卫兵。诸葛亮当北行,表与后主曰:“将军向宠,性行淑均,晓畅军事,试用於昔,先帝称之曰能,是以众论举宠为督。愚以为营中之事,悉以咨之,必能使行陈和睦,优劣得所也。”迁中领军。延熙三年,征汉嘉蛮夷,遇害。宠弟充,历射声校尉尚书。

张裔字君嗣,蜀郡成都人也。治公羊春秋,博涉史、汉。汝南许文休入蜀,谓裔幹理敏捷,是中夏锺元常之伦也。刘璋时,举孝廉,为鱼复长,还州署从事,领帐下司马。张飞自荆州由垫江入,璋授裔兵,拒张飞於德阳陌下,军败,还成都。为璋奉使诣先主,先主许以礼其君而安其人也,裔还,城门乃开。先主以裔为巴郡太守,还为司金中郎将,典作农战之器。先是,益州郡杀太守正昂,耆率雍闿恩信著於南土,使命周旋,远通孙权。乃以裔为益州太守,径往至郡。闿遂趑趄不宾,假鬼教曰:“张府君如瓠壶,外虽泽而内实粗,不足杀,令缚与吴。”於是遂送裔於权。

会先主薨,诸葛亮遣邓芝使吴,亮令芝言次可从权请裔。裔自至吴数年,流徙伏匿,权未之知也,故许芝遣裔。裔临发,权乃引见,问裔曰:“蜀卓氏寡女,亡奔司马相如,贵土风俗何以乃尔乎?”裔对曰:“愚以卓氏之寡女,犹贤於买臣之妻。”权又谓裔曰:“君还,必用事西朝,终不作田父於闾里也,将何以报我?”裔对曰:“裔负罪而归,将委命有司。若蒙徼倖得全首领,五十八已前父母之年也,自此已后大王之赐也。”权言笑欢悦,有器裔之色。裔出閤,深悔不能阳愚,即便就船,倍道兼行。权果追之,裔已入永安界数十里,追者不能及。

既至蜀,丞相亮以为参军,署府事,又领益州治中从事。亮出驻汉中,裔以射声校尉领留府长吏,常称曰:“公赏不遗远,罚不阿近,爵不可以无功取,刑不可以贵势免,此贤愚之所以佥忘其身者也。”其明年,北诣亮谘事,送者数百,车乘盈路,裔还书与所亲曰:“近者涉道,昼夜接宾,不得宁息,人自敬丞相长史,男子张君嗣附之,疲倦欲死。”其谈啁流速,皆此类也。少与犍为杨恭友善,恭早死,遗孤未数岁,裔迎留,与分屋而居,事恭母如母。恭之子息长大,为之娶妇,买田宅产业,使立门户。抚恤故旧,振赡衰宗,行义甚至。加辅汉将军,领长史如故。建兴八年卒。子毣嗣,历三郡守监军。毣弟都,太子中庶子。

杨洪字季休,犍为武阳人也。刘璋时历部诸郡。先主定蜀,太守李严命为功曹。严欲徙郡治舍,洪固谏不听,遂辞功曹,请退。严欲荐洪於州,为蜀部从事。先主争汉中,急书发兵,军师将军诸葛亮以问洪,洪曰:“汉中则益州咽喉,存亡之机会,若无汉中则无蜀矣,此家门之祸也。方今之事,男子当战,女子当运,发兵何疑?”时蜀郡太守法正从先主北行,亮於是表洪领蜀郡太守,众事皆办,遂使即真。顷之,转为益州治中从事。

先主既称尊号,征吴不克,还住永安。汉嘉太守黄元素为诸葛亮所不善,闻先主疾病,惧有后患,举郡反,烧临邛城。时亮东行省疾,成都单虚,是以元益无所惮。洪即启太子,遣其亲兵,使将军陈曶、郑绰讨元。众议以为元若不能围成都,当由越巂据南中,洪曰:“元素性凶暴,无他恩信,何能办此?不过乘水东下,冀主上平安,面缚归死;如其有异,奔吴求活耳。敕曶、绰但於南安峡口遮即便得矣。”曶、绰承洪言,果生获元。洪建兴元年赐爵关内侯,复为蜀郡太守、忠节将军,后为越骑校尉,领郡如故。

五年,丞相亮北住汉中,欲用张裔为留府长史,问洪何如?洪对曰:“裔天姿明察,长於治剧,才诚堪之,然性不公平,恐不可专任,不如留向朗。朗情伪差少,裔随从目下,效其器能,於事两善。”初,裔少与洪亲善。裔流放在吴,洪临裔郡,裔子郁给郡吏,微过受罚,不特原假。裔后还闻之,深以为恨,与洪情好有损。及洪见亮出,至裔许,具说所言。裔答洪曰:“公留我了矣,明府不能止。”时人或疑洪意自欲作长史,或疑洪知裔自嫌,不愿裔处要职,典后事也。后裔与司盐校尉岑述不和,至于忿恨。亮与裔书曰:“君昔在【柏】陌下,营坏,吾之用心,食不知味;后流迸南海,相为悲叹,寝不安席;及其来还,委付大任,同奖王室,自以为与君古之石交也。石交之道,举雠以相益,割骨肉以相明,犹不相谢也,况吾但委意於元俭,而君不能忍邪?”论者由是明洪无私。

洪少不好学问,而忠清款亮,忧公如家,事继母至孝。六年卒官。始洪为李严功曹,严未去至犍为而洪已为蜀郡。洪迎门下书佐何祗,有才策功幹,举郡吏,数年为广汉太守,时洪亦尚在蜀郡。是以西土咸服诸葛亮能尽时人之器用也。

费诗字公举,犍为南安人也。刘璋时为绵竹令,先主攻绵竹时,诗先举城降。成都既定,先主领益州牧,以诗为督军从事,出为牂牁太守,还为州前部司马。先主为汉中王,遣诗拜关羽为前将军,羽闻黄忠为后将军,羽怒曰:“大丈夫终不与老兵同列!”不肯受拜。诗谓羽曰:“夫立王业者,所用非一。昔萧、曹与高祖少小亲旧,而陈、韩亡命后至,论其班列,韩最居上,未闻萧、曹以此为怨。今汉王以一时之功,隆崇於汉升,然意之轻重,宁当与君侯齐乎!且王与君侯,譬犹一体,同休等戚,祸福共之,愚为君侯,不宜计官号之高下,爵禄之多少为意也。仆一介之使,衔命之人,君侯不受拜,如是便还,但相为惜此举动,恐有后悔耳!”羽大感悟,遽即受拜。

后群臣议欲推汉中王称尊号,诗上疏曰:“殿下以曹操父子偪主篡位,故乃羁旅万里,纠合士众,将以讨贼。今大敌未克,而先自立,恐人心疑惑。昔高祖与楚约,先破秦者王。及屠咸阳,获子婴,犹怀推让,况今殿下未出门庭,便欲自立邪!愚臣诚不为殿下取也。”由是忤指,左迁部永昌从事。建兴三年,随诸葛亮南行,归至汉阳县,降人李鸿来诣亮,亮见鸿,时蒋琬与诗在坐。鸿曰:“间过孟达许,適见王冲从南来,言往者达之去就,明公切齿,欲诛达妻子,赖先主不听耳。达曰:‘诸葛亮见顾有本末,终不尔也。’尽不信冲言,委仰明公,无复已已。”亮谓琬、诗曰:“还都当有书与子度相闻。”诗进曰:“孟达小子,昔事振威不忠,后又背叛先主,反覆之人,何足与书邪!”亮默然不答。亮欲诱达以为外援,竟与达书曰:“往年南征,岁末乃还,適与李鸿会於汉阳,承知消息,慨然永叹,以存足下平素之志,岂徒空讬名荣,贵为乖离乎!呜呼孟子,斯实刘封侵陵足下,以伤先主待士之义。又鸿道王冲造作虚语,云足下量度吾心,不受冲说。寻表明之言,追平生之好,依依东望,故遣有书。”达得亮书,数相交通,辞欲叛魏。魏遣司马宣王征之,即斩灭达。亮亦以达无款诚之心,故不救助也。蒋琬秉政,以诗为谏议大夫,卒於家。

王冲者,广汉人也。为牙门将,统属江州督李严。为严所疾,惧罪降魏。魏以冲为乐陵太守。

评曰:霍峻孤城不倾,王连固节不移,向朗好学不倦,张裔肤敏应机,杨洪乃心忠公,费诗率意而言,皆有可纪焉。以先主之广济,诸葛之准绳,诗吐直言,犹用陵迟,况庸后乎哉!

\part{蜀书十二}
\chapter{杜周杜许孟来尹李谯郤传第十二}

杜微字国辅,梓潼涪人也。少受学於广汉任安。刘璋辟为从事,以疾去官。及先主定蜀,微常称聋,闭门不出。建兴二年,丞相亮领益州牧,选迎皆妙简旧德,以秦宓为别驾,五梁为功曹,微为主簿。微固辞,轝而致之。既致,亮引见微,微自陈谢。高以微不闻人语,於坐上与书曰:“服闻德行,饥渴历时,清浊异流,无缘咨覯。王元泰、李伯仁、王文仪、杨季休、丁君幹、李永南兄弟、文仲宝等,每叹高志,未见如旧。猥以空虚,统领贵州,德薄任重,惨惨忧虑。朝廷今年始十八,天姿仁敏,爱德下士。天下之人思慕汉室,欲与君因天顺民,辅此明主,以隆季兴之功,著勋於竹帛也。以谓贤愚不相为谋,故自割绝,守劳而已,不图自屈也。”微自乞老病求归,亮又与书答曰:“曹丕篡弑,自立为帝,是犹土龙刍狗之有名也。欲与群贤因其邪伪,以正道灭之。怪君未有相诲,便欲求还於山野。丕又大兴劳役,以向吴、楚。今因丕多务,且以闭境勤农,育养民物,并治甲兵,以待其挫,然后伐之,可使兵不战民不劳而天下定也。君但当以德辅时耳,不责君军事,何为汲汲欲求去乎!”其敬微如此。拜为谏议大夫,以从其志。

五梁者,字德山,犍为南安人也,以儒学节操称。从议郎迁谏议大夫、五官中郎将。

周群字仲直,巴西阆中人也。父舒,字叔布,少学术於广汉杨厚,名亚董扶、任安。数被徵,终不诣。时人有问:“春秋谶曰代汉者当涂高,此何谓也?”舒曰:“当涂高者,魏也。”乡党学者私传其语。群少受学於舒,专心候业。於庭中作小楼,家富多奴,常令奴更直於楼上视天灾,才见一气,即白群,群自上楼观之,不避晨夜。故凡有气候,无不见之者,是以所言多中。州牧刘璋,辟以为师友从事。先主定蜀,署儒林校尉。先主欲与曹公争汉中,问群,群对曰:“当得其地,不得其民也。若出偏军,必不利,当戒慎之!”时州后部司马蜀郡张裕亦晓占候,而天才过群,谏先主曰:“不可争汉中,军必不利。”先主竟不用裕言,果得地而不得民也。遣将军吴兰、雷铜等入武都,皆没不还,悉如群言。於是举群茂才。

裕又私语人曰:“岁在庚子,天下当易代,刘氏祚尽矣。主公得益州,九年之后,寅卯之间当失之。”人密白其言。初,先主与刘璋会涪时,裕为璋从事,侍坐。其人饶须,先主嘲之曰:“昔吾居涿县,特多毛姓,东西南北皆诸毛也,涿令称曰‘诸毛绕涿居乎’!”裕即答曰:“昔有作上党潞长,迁为涿令者,去官还家,时人与书,欲署潞则失涿,欲署涿则失潞,乃署曰‘潞涿君’。”先主无须,故裕以此及之。先主常衔其不逊,加忿其漏言,乃显裕谏争汉中不验,下狱,将诛之。诸葛亮表请其罪,先主答曰:“芳兰生门,不得不鉏。”裕遂弃市。后魏氏之立,先主之薨,皆如裕所刻。又晓相术,每举镜视面,自知刑死,未尝不扑之於地也。

群卒,子巨颇传其术。

杜琼字伯瑜,蜀郡成都人也。少受学於任安,精究安术。刘璋时辟为从事。先主定益州,领牧,以琼为议曹从事。后主践阼,拜谏议大夫,迁左中郎将、大鸿胪、太常。为人静默少言,阖门自守,不与世事。蒋琬、费祎等皆器重之。虽学业入深,初不视天文有所论说。后进通儒谯周常问其意,琼答曰:“欲明此术甚难,须当身视,识其形色,不可信人也。晨夜苦剧,然后知之,复忧漏泄,不如不知,是以不复视也。”周因问曰:“昔周徵君以为当涂高者魏也,其义何也?“琼答曰:“魏,阙名也,当涂而高,圣人取类而言耳。”又问周曰:“宁复有所怪邪?”周曰:“未达也。“琼又曰:“古者名官职不言曹;始自汉已来,名官尽言曹,使言属曹,卒言侍曹,此殆天意也。”琼年八十馀,延熙十三年卒。著韩诗章句十馀万言,不教诸子,内学无传业者。周缘琼言,乃触类而长之曰:“春秋传著晋穆侯名太子曰仇,弟曰成师。师服曰:‘异哉君之名子也!嘉耦曰妃,怨偶曰仇,今君名太子曰仇,弟曰成师,始兆乱矣,兄其替乎?’其后果如服言。及汉灵帝名二子曰史侯、董侯,既立为帝,后皆免为诸侯,与师服言相似也。先主讳备,其训具也,后主讳禅,其训授也,如言刘已具矣,当授与人也;意者甚於穆侯、灵帝之名子。”后宦人黄皓弄权於内,景耀五年,宫中大树无故自折,周深忧之,无所与言,乃书柱曰:“众而大,期之会,具而授,若何复?”言曹者众也,魏者大也,众而大,天下其当会也。具而授,如何复有立者乎?蜀既亡,咸以周言为验。周曰:“此虽己所推寻,然有所因,由杜君之辞而广之耳,殊无神思独至之异也。”

许慈字仁笃,南阳人也。师事刘熙,善郑氏学,治易、尚书、三礼、毛诗、论语。建安中,与许靖等俱自交州入蜀。时又有魏郡胡潜,字公兴,不知其所以在益土。潜虽学不沾洽,然卓荦强识,祖宗制度之仪,丧纪五服之数,皆指掌画地,举手可采。先主定蜀,承丧乱历纪,学业衰废,乃鸠合典籍,沙汰众学,慈、潜并为学士,与孟光、来敏等典掌旧文。值庶事草创,动多疑议,慈、潜更相克伐,谤讟忿争,形於声色;书籍有无,不相通借,时寻楚挞,以相震攇。其矜己妒彼,乃至於此。先主愍其若斯,群僚大会,使倡家假为二子之容。效其讼阋之状,酒酣乐作,以为嬉戏,初以辞义相难,终以刀杖相屈,用感切之。潜先没,慈后主世稍迁至大长秋,卒。子勋传其业,复为博士。

孟光字孝裕,河南洛阳人,汉太尉孟郁之族。灵帝末为讲部吏。献帝迁都长安,遂逃入蜀,刘焉父子待以客礼。博物识古,无书不览,尤锐意三史,长於汉家旧典。好公羊春秋而讥呵左氏,每与来敏争此二义,光常譊々讙咋。先主定益州,拜为议郎,与许慈等并掌制度。后主践阼,为符节令、屯骑校尉、长乐少府,迁大司农。延熙九年秋,大赦,光於众中责大将军费祎曰:“夫赦者,偏枯之物,非明世所宜有也。衰弊穷极,必不得已,然后乃可权而行之耳。今主上仁贤,百僚称职,有何旦夕之危,倒悬之急,而数施非常之恩,以惠奸宄之恶乎?又鹰隼始击,而更原宥有罪,上犯天时,下违人理。老夫耄朽,不达治体,窃谓斯法难以经久,岂具瞻之高美,所望於明德哉!”祎但顾谢踧踖而已。光之指摘痛痒,多如是类,故执政重臣,心不能悦,爵位不登;每直言无所回避,为代所嫌。太常广汉镡承、光禄勋河东裴俊等,年资皆在光后,而登据上列,处光之右,盖以此也。

后进文士秘书郎郤正数从光谘访,光问正太子所习读并其情性好尚,正答曰:“奉亲虔恭,夙夜匪懈,有古世子之风;接待群僚,举动出於仁恕。”光曰:“如君所道,皆家户所有耳;吾今所问,欲知其权略智调何如也。”正曰:“世子之道,在於承志竭欢,既不得妄有所施为,且智调藏於胸怀,权略应时而发,此之有无,焉可豫设也?”光解正慎宜,不为放谈,乃曰:“吾好直言,无所回避,每弹射利病,为世人所讥嫌;省君意亦不甚好吾言,然语有次。今天下未定,智意为先,智意虽有自然,然亦可力强致也。此储君读书,宁当效吾等竭力博识以待访问,如傅士探策讲试以求爵位邪!当务其急者。”正深谓光言为然。后光坐事免官,年九十馀卒。

来敏字敬达,义阳新野人,来歙之后也。父艳,为汉司空。汉末大乱,敏随姊奔荆州,姊夫黄琬是刘璋祖母之侄,故璋遣迎琬妻,敏遂俱与姊入蜀,常为璋宾客。涉猎书籍,善左氏春秋,尤精於仓、雅训诂,好是正文字。先主定益州,署敏典学校尉,及立太子,以为家令。后主践阼,为虎贲中郎将。丞相亮住汉中,请为军祭酒、辅军将军,坐事去职。亮卒后,还成都为大长秋,又免,后累迁为光禄大夫,复坐过黜。前后数贬削,皆以语言不节,举动违常也。时孟光亦以枢机不慎,议论于时,然犹愈於敏,俱以其耆宿学士见礼於世。而敏荆楚名族,东宫旧臣,特加优待,是故废而复起。后以敏为执慎将军,欲令以官重自警戒也。年九十七,景耀中卒。子忠,亦博览经学,有敏风,与尚书向充等并能协赞大将军姜维。维善之,以为参军。

尹默字思潜,梓潼涪人也。益部多贵今文而不崇章句,默知其不博,乃远游荆州,从司马德操、宋仲子等受古学。皆通诸经史,又专精於左氏春秋,自刘歆条例,郑众、贾逵父子、陈元、服虔注说,咸略诵述,不复按本。先主定益州,领牧,以为劝学从事,及立太子,以默为仆,以左氏传授后主。后主践阼,拜谏议大夫。丞相亮住汉中,请为军祭酒。亮卒,还成都,拜太中大夫,卒。子宗传其业,为博士。

李譔字钦仲,梓潼涪人也。父仁,字德贤,与同县尹默俱游荆州,从司马徽、宋忠等学。撰具传其业,又从默讲论义理,五经、诸子,无不该览,加博好技艺,算术、卜数、医药、弓弩、机械之巧,皆致思焉。始为州书佐、尚书令史。延熙元年,后主立太子,以撰为庶子,迁为仆。转中散中大夫、右中郎将,犹侍太子。太子爱其多知,甚悦之。然体轻脱,好戏啁,故世不能重也。著古文易、尚书、毛诗、三礼、左氏传、太玄指归,皆依准贾、马,异於郑玄。与王氏殊隔,初不见其所述,而意归多同。景耀中卒。时又有汉中陈术,字申伯,亦博学多闻,著释问七篇、益部耆旧传及志,位历三郡太守。

谯周字允南,巴西西充国人也。父〈山并〉,字荣始,治尚书,兼通诸经及图、纬。州郡辟请,皆不应,州就假师友从事。周幼孤,与母兄同居。既长,耽古笃学,家贫未尝问产业,诵读典籍,欣然独笑,以忘寝食。研精六经,尤善书札。颇晓天文,而不以留意;诸子文章非心所存,不悉遍视也。身长八尺,体貌素朴,性推诚不饰,无造次辩论之才,然潜识内敏。

建兴中,丞相亮领益州牧,命周为劝学从事。亮卒於敌庭,周在家闻问,即便奔赴,寻有诏书禁断,惟周以速行得达。大将军蒋琬领刺史,徙为典学从事,总州之学者。

后主立太子,以周为仆,转家令。时后主颇出游观,增广声乐。周上疏谏曰:“昔王莽之败,豪杰并起,跨州据郡,欲弄神器,於是贤才智士思望所归,未必以其势之广狭,惟其德之薄厚也。是故於时更始、公孙述及诸有大众者多已广大,然莫不快情恣欲,怠於为善,游猎饮食,不恤民物。世祖初入河北,冯异等劝之曰:'当行人所不能为。'遂务理冤狱,节俭饮食,动遵法度,故北州歌叹,声布四远。於是邓禹自南阳追之,吴汉、寇恂未识世祖,遥闻德行,遂以权计举渔阳、上谷突骑迎于广阿。其馀望风慕德者邳肜、耿纯、刘植之徒,至于舆病赍棺,礻强负而至者,不可胜数,故能以弱为强,屠王郎,吞铜马,折赤眉而成帝业也。及在洛阳,尝欲小出,车驾已御,銚期谏曰:‘天下未宁,臣诚不愿陛下细行数出。’即时还车。及征隗嚣,颍川盗起,世祖还洛阳,但遣寇恂往,恂曰:‘颍川以陛下远征,故奸猾起叛,未知陛下还,恐不时降;陛下自临,颍川贼必即降。'遂至颍川,竟如恂言。故非急务,欲小出不敢,至於急务,欲自安不为,故帝者之欲善也如此!故传曰'百姓不徒附',诚以德先之也。今汉遭厄运,天下三分,雄哲之士思望之时也。陛下天姿至孝,丧逾三年,言及陨涕,虽曾闵不过也。敬贤任才,使之尽力,有逾成康。故国内和一,大小戮力,臣所不能陈。然臣不胜大愿,愿复广人所不能者。夫輓大重者,其用力苦不众,拔大艰者,其善术苦不广,且承事宗庙者,非徒求福祐,所以率民尊上也。至於四时之祀,或有不临,池苑之观,或有仍出,臣之愚滞,私不自安。夫忧责在身者,不暇尽乐,先帝之志,堂构未成,诚非尽乐之时。愿省减乐官、后宫所增造,但奉脩先帝所施,下为子孙节俭之教。”徙为中散大夫,犹侍太子。

于时军旅数出,百姓彫瘁,周与尚书令陈祗论其利害,退而书之,谓之仇国论。其辞曰:“因馀之国小,而肇建之国大,并争於世而为仇敌。因馀之国有高贤卿者,问於伏愚子曰:'今国事未定,上下劳心,往古之事,能以弱胜强者,其术何如?'伏愚子曰:‘吾闻之,处大无患者恒多慢,处小有忧者恒思善;多慢则生乱,思善则生治,理之常也。故周文养民,以少取多,勾践恤众,以弱毙强,此其术也。’贤卿曰:‘曩者项强汉弱,相与战争,无日宁息,然项羽与汉约分鸿沟为界,各欲归息民;张良以为民志既定,则难动也,寻帅追羽,终毙项氏,岂必由文王之事乎?肇建之国方有疾疢,我因其隙,陷其边陲,觊增其疾而毙之也。’伏愚子曰:‘当殷、周之际,王侯世尊,君臣久固,民习所专;深根者难拔,据固者难迁。当此之时,虽汉祖安能杖剑鞭马而取天下乎?当秦罢侯置守之后,民疲秦役,天下土崩,或岁改主,或月易公,鸟惊兽骇,莫知所从,於是豪强并争,虎裂狼分,疾博者获多,迟后者见吞。今我与肇建皆传国易世矣,既非秦末鼎沸之时,实有六国并据之势,故可为文王,难为汉祖。夫民疲劳则骚扰之兆生,上慢下暴则瓦解之形起。谚曰:“射幸数跌,不如审发。”是故智者不为小利移目,不为意似改步,时可而后动,数合而后举,故汤、武之师不再战而克,诚重民劳而度时审也。如遂极武黩征,土崩势生,不幸遇难,虽有智者将不能谋之矣。若乃奇变纵横,出入无间,冲波截辙,超谷越山,不由舟楫而济盟津者,我愚子也,实所不及。’”

后迁光禄大夫,位亚九列。周虽不与政事,以儒行见礼,时访大议,辄据经以对,而后生好事者亦咨问所疑焉。

景耀六年冬,魏大将军邓艾克江由,长驱而前。而蜀本谓敌不便至,不作城守调度,及闻艾已入阴平,百姓扰扰,皆迸山野,不可禁制。后主使群臣会议,计无所出。或以为蜀之与吴,本为和国,宜可奔吴;或以为南中七郡,阻险斗绝,易以自守,宜可奔南。惟周以为:“自古已来,无寄他国为天子者也,今若入吴,固当臣服。且政理不殊,则大能吞小,此数之自然也。由此言之,则魏能并吴,吴不能并魏明矣。等为小称臣,孰与为大,再辱之耻,何与一辱?且若欲奔南,则当早为之计,然后可果;今大敌以近,祸败将及,群小之心,无一可保?恐发足之日,其变不测,何至南之有乎!”群臣或难周曰:“今艾以不远,恐不受降,如之何?”周曰:“方今东吴未宾,事势不得不受,受之之后,不得不礼。若陛下降魏,魏不裂土以封陛下者,周请身诣京都,以古义争之。”众人无以易周之理。

后主犹疑於入南,周上疏曰:“或说陛下以北兵深入,有欲適南之计,臣愚以为不安。何者?南方远夷之地,平常无所供为,犹数反叛,自丞相亮南征,兵势偪之,穷乃幸从。是后供出官赋,取以给兵,以为愁怨,此患国之人也。今以穷迫,欲往依恃,恐必复反叛,一也。北兵之来,非但取蜀而已,若奔南方,必因人势衰,及时赴追,二也。若至南方,外当拒敌,内供服御,费用张广,他无所取,耗损诸夷必甚,甚必速叛,三也。昔王郎以邯郸僣号,时世祖在信都,畏偪於郎,欲弃还关中。邳肜谏曰:'明公西还,则邯郸城民不肯捐父母,背城主,而千里送公,其亡叛可必也。'世祖从之,遂破邯郸。今北兵至,陛下南行,诚恐邳肜之言复信於今,四也。愿陛下早为之图,可获爵土;若遂適南,势穷乃服,其祸必深。易曰:'亢之为言,知得而不知丧,知存而不知亡;知得失存亡而不失其正者,其惟圣人乎!'言圣人知命而不苟必也。故尧、舜以子不善,知天有授,而求授人;子虽不肖,祸尚未萌,而迎授与人,况祸以至乎!故微子以殷王之昆,面缚衔璧而归武王,岂所乐哉,不得已也。”於是遂从周策。刘氏无虞,一邦蒙赖,周之谋也。

时晋文王为魏相国,以周有全国之功,封阳城亭侯。又下书辟周,周发至汉中,困疾不进。咸熙二年夏,巴郡文立从洛阳还蜀,过见周。周语次,因书版示立曰:“典午忽兮,月酉没兮。”典午者谓司马也,月酉者谓八月也,至八月而文王果崩。晋室践阼,累下诏所在发遣周。周遂舆疾诣洛,泰始三年至。以疾不起,就拜骑都尉,周乃自陈无功而封,求还爵土,皆不听许。

五年,予尝为本郡中正,清定事讫,求休还家,往与周别。周语予曰:“昔孔子七十二、刘向、扬雄七十一而没,今吾年过七十,庶慕孔子遗风,可与刘、扬同轨,恐不出后岁,必便长逝,不复相见矣。”疑周以术知之,假此而言也。六年秋,为散骑常侍,疾笃不拜,至冬卒。凡所著述,撰定法训、五经论、古史考之属百馀篇。周三子,熙、贤、同。少子同颇好周业,亦以忠笃质素为行,举孝廉,除锡令、东宫洗马,召不就。

郤正字令先,河南偃师人也。祖父俭,灵帝末为益州刺史,为盗贼所杀。会天下大乱,故正父揖因留蜀。揖为将军孟达营都督,随达降魏,为中书令史。正本名纂。少以父死母嫁,单茕只立,而安贫好学,博览坟籍。弱冠能属文,入为秘书吏,转为令史,迁郎,至令。性澹於荣利,而尤耽意文章,自司马、王、扬、班、傅、张、蔡之俦遗文篇赋,及当世美书善论,益部有者,则钻凿推求,略皆寓目。自在内职,与宦人黄皓比屋周旋,经三十年,皓从微至贵,操弄威权,正既不为皓所爱,亦不为皓所憎,是以官不过六百石,而免於忧患。

依则先儒,假文见意,号曰释讥,其文继於崔骃达旨。其辞曰:

或有讥余者曰:'闻之前记,夫事与时并,名与功偕,然则名之与事,前哲之急务也。是故创制作范,匪时不立,流称垂名,匪功不记,名必须功而乃显,事亦俟时以行止,身没名灭,君子所耻。是以达人研道,探赜索微,观天运之符表,考人事之盛衰,辩者驰说,智者应机,谋夫演略,武士奋威,云合雾集,风激电飞,量时揆宜,用取世资,小屈大申,存公忽私,虽尺枉而寻直,终扬光以发辉也。今三方鼎跱,九有未乂,悠悠四海,婴丁祸败,嗟道义之沈塞,愍生民之颠沛,此诚圣贤拯救之秋,烈士树功之会也。吾子以高朗之才,珪璋之质,兼览博闚,留心道术,无远不致,无幽不悉;挺身取命,幹兹奥秘,踌躇紫闼,喉舌是执,九考不移,有入无出,究古今之真伪,计时务之得失。虽时献一策,偶进一言,释彼官责,慰此素飧,固未能输竭忠款,尽沥胸肝,排方入直,惠彼黎元,俾吾徒草鄙并有闻焉也。盍亦绥衡缓辔,回轨易涂,舆安驾肆,思马斯徂,审厉揭以投济,要夷庚之赫怃,播秋兰以芳世,副吾徒之披图,不亦盛与!’

余闻而叹曰:“呜呼,有若云乎邪!夫人心不同,实若其面,子虽光丽,既美且艳,管闚筐举,守厥所见,未可以言八纮之形埒,信万事之精练也。

或人率尔,抑而扬衡曰:‘是何言与!是何言与!’

余应之曰:'虞帝以面从为戒,孔圣以悦己为尤,若子之言,良我所思,将为吾子论而释之。昔在鸿荒,蒙昧肇初,三皇应箓,五帝承符,爰暨夏、商,前典攸书。姬衰道缺,霸者翼扶,嬴氏惨虐,吞嚼八区,於是从横云起,狙诈如星,奇邪蜂动,智故萌生;或饰真以雠伪,或挟邪以干荣,或诡道以要上,或鬻技以自矜;背正崇邪,弃直就佞,忠无定分,义无常经。故鞅法穷而慝作,斯义败而奸成,吕门大而宗灭,韩辩立而身刑。夫何故哉?利回其心,宠耀其目,赫赫龙章,铄铄车服,媮幸苟得,如反如仄,淫邪荒迷,恣睢自极,和鸾未调而身在辕侧,庭宁未践而栋折榱覆。天收其精,地缩其泽,人吊其躬,鬼芟其额。初升高冈,终陨幽壑,朝含荣润,夕为枯魄。是以贤人君子,深图远虑,畏彼咎戾,超然高举,宁曳尾於涂中,秽浊世之休誉。彼岂轻主慢民,而忽於时务哉?盖易著行止之戒,诗有靖恭之叹,乃神之听之而道使之然也。

自我大汉,应天顺民,政治之隆,皓若阳春,俯宪坤典,仰式乾文,播皇泽以熙世,扬茂化之醲醇,君臣履度,各守厥真;下垂询纳之弘,下有匡救之责,士无虚华之宠,民有一行之迹,粲乎亹亹,尚此忠益。然而道有隆窳,物有兴废,有声有寂,有光有翳。朱阳否於素秋,玄阴抑於孟春,羲和逝而望舒系,运气匿而耀灵陈。冲、质不永,桓、灵坠败,英雄云布,豪杰盖世,家挟殊议,人怀异计,故从横者欻披其胸,狙诈者暂吐其舌也。

今天纲已缀,德树西邻,丕显祖之宏规,縻好爵於士人,兴五教以训俗,丰九德以济民,肃明祀以礿祭,几皇道以辅真。虽跱者未一,伪者未分,圣人垂戒,盖均无贪;故君臣协美於朝,黎庶欣戴於野,动若重规,静若叠矩。济济伟彦,元凯之伦也,有过必知,颜子之仁也,侃侃庶政,冉、季之治也,鹰杨鸷腾,伊、望之事也;总群俊之上略,含薛氏之三计,敷张、陈之秘策,故力征以勤世,援华英而不遑,岂暇脩枯箨於榛秽哉!

然吾不才,在朝累纪,讬身所天,心焉是恃。乐沧海之广深,叹嵩岳之高跱,闻仲尼之赞商,感乡校之益己,彼平仲之和羹,亦进可而替否;故蒙冒瞽说,时有攸献,譬遒人之有采于市闾,游童之吟咏乎疆畔,庶以增广福祥,输力规谏。若其合也,则以闇协明,进应灵符;如其违也,自我常分,退守己愚。进退任数,不矫不诬,循性乐天,夫何恨诸?此其所以既入不出,有而若无者也。狭屈氏之常醒,浊渔父之必醉,溷柳季之卑辱,褊夷叔之高怼。合不以得,违不以失,得不克诎,失不惨悸;不乐前以顾轩,不就后以虑轾,不鬻誉以干泽,不辞愆以忌绌。何责之释?何飧之恤?何方之排?何直之入?九考不移,固其所执也。

方今朝士山积,髦俊成群,犹鳞介之潜乎巨海,毛羽之集乎邓林,游禽逝不为之鲜,浮鲂臻不为之殷。且阳灵幽於唐叶,阴精应於商时,阳盱请而洪灾息,桑林祷而甘泽滋。行止有道,启塞有期。我师遗训,不怨不尤,委命恭己,我又何辞?辞穷路单,将反初节,综坟典之流芳,寻孔氏之遗艺,缀微辞以存道,宪先轨而投制,韪叔肸之优游,美疏氏之遐逝,收止足以言归,汎皓然以容裔,欣环堵以恬娱,免咎悔於斯世,顾兹心之未泰,惧末涂之泥滞,仍求激而增愤,肆中怀以告誓。昔九方考精於至贵,秦牙沈思於殊形;薛烛察宝以飞誉,瓠梁讬弦以流声;齐隶拊髀以济文,楚客潜寇以保荆;雍门援琴而挟说,韩哀秉辔而驰名;卢敖翱翔乎玄阙,若士竦身于云清。余实不能齐技於数子,故乃静然守已而自宁。’

景耀六年,后主从谯周之计,遣使请降于邓艾,其书,正所造也。明年正月,锺会作乱成都,后主东迁洛阳,时扰攘仓卒,蜀之大臣无翼从者,惟正及殿中督汝南张通,舍妻子单身随侍。后主赖正相导宜適,举动无阙,乃慨然叹息,恨知正之晚。时论嘉之。赐爵关内侯。泰始中,除安阳令,迁巴西太守。泰始八年诏曰:“正昔在成都,颠沛守义,不违忠节,及见受用,尽心幹事,有治理之绩,其以正为巴西太守。”咸宁四年卒。凡所著述诗论赋之属,垂百篇。

评曰:杜微脩身隐静,不役当世,庶几夷、皓之。周群占天有徵,杜琼沈默慎密,诸生之纯也。许、孟、来、李,博涉多闻,尹默精于左氏,虽不以德业为称,信皆一时之学士。谯周词理渊通,为世硕儒,有董、扬之规,郤正文辞灿烂,有张、蔡之风,加其行止,君子有取焉。二子处晋事少,在蜀事多,故著于篇。

\part{蜀书十三}
\chapter{黄李吕马王张传第十三}

黄权字公衡,巴西阆中人也。少为郡吏,州牧刘璋召为主簿。时别驾张松建议,宜迎先主,使伐张鲁。权谏曰:“左将军有骁名,今请到,欲以部曲遇之,则不满其心,欲以宾客礼待,则一国不容二君。若客有泰山之安,则主有累卵之危。可但闭境,以待河清。”璋不听,竟遣使迎先主,出权为广汉长。及先主袭取益州,将帅分下郡县,郡县望风景附,权闭城坚守,须刘璋稽服,乃诣降先主。先主假权偏将军。及曹公破张鲁,鲁走入巴中,权进曰:“若失汉中,则三巴不振,此为割蜀之股臂也。”於是先主以权为护军,率诸将迎鲁。鲁已还南郑,北降曹公,然卒破杜濩、朴胡,杀夏侯渊,据汉中,皆权本谋也。

先主为汉中王,犹领益州牧,以权为治中从事。及称尊号,将东伐吴,权谏曰:“吴人悍战,又水军顺流,进易退难,臣请为先驱以尝寇,陛下宜为后镇。”先主不从,以权为镇北将军,督江北军以防魏师;先主自在江南。及吴将军陆议乘流断围,南军败绩,先主引退。而道隔绝,权不得还,故率将所领降于魏。有司执法,白收权妻子。先主曰:“孤负黄权,权不负孤也。”待之如初。

魏文帝谓权曰:“君舍逆效顺,欲追踪陈、韩邪?”权对曰:“臣过受刘主殊遇,降吴不可,还蜀无路,是以归命。且败军之将,免死为幸,何古人之可慕也”文帝善之,拜为镇南将军,封育阳侯,加侍中,使之陪乘。蜀降人或云诛权妻子,权知其虚言,未便发丧,后得审问,果如所言。及先主薨问至,魏群臣咸贺而权独否。文帝察权有局量,欲试惊之,遣左右诏权,未至之间,累催相属,马使奔驰,交错於道,官属侍从莫不碎魄,而权举止颜色自若。后领益州刺史,徙占河南。大将军司马宣王深器之,问权曰:“蜀中有卿辈几人?”权笑而答曰:“不图明公见顾之重也!”宣王与诸葛亮书曰:“黄公衡,快士也,每坐起叹述足下,不去口实。”景初三年,蜀延熙二年,权迁车骑将军、仪同三司。明年卒,谥曰景侯。子邕嗣。邕无子,绝。

权留蜀子崇,为尚书郎,随卫将军诸葛瞻拒邓艾。到涪县,瞻盘桓未进,崇屡劝瞻宜速行据险,无令敌得入平地。瞻犹与未纳,崇至于流涕。会艾长驱而前,瞻卻战至绵竹,崇帅厉军士,期於必死,临陈见杀。

李恢字德昂,建宁俞元人也。仕郡督邮,姑夫爨习为建伶令,有违犯之事,恢坐习免官。太守董和以习方土大姓,寝而不许。后贡恢于州,涉道未至,闻先主自葭萌还攻刘璋。恢知璋之必败,先主必成,乃讬名郡使,北诣先主,遇於绵竹。先主嘉之,从至雒城,遣恢至汉中交好马超,超遂从命。成都既定,先主领益州牧,以恢为功曹书佐主簿。后为亡虏所诬,引恢谋反,有司执送,先主明其不然,更迁恢为别驾从事。章武元年,庲降都督邓方卒,先主问恢:“谁可代者?”恢对曰:“人之才能,各有长短,故孔子曰'其使人也器之。且夫明主在上,则臣下尽情,是以先零之役,赵充国曰'莫若老臣。臣窃不自揆,惟陛下察之。“先主笑曰:“孤之本意,亦已在卿矣。”遂以恢为庲降都督,使持节领交州刺史,住平夷县。

先主薨,高定恣睢於越巂,雍闿跋扈於建宁,朱褒反叛於牂牁。丞相亮南征,先由越巂,而恢案道向建宁。诸县大相纠合,围恢军於昆明。时恢众少敌倍,又未得亮声息,绐谓南人曰:“官军粮尽,欲规退还,吾中间久斥乡里,乃今得旋,不能复北,欲还与汝等同计谋,故以诚相告。”南人信之,故围守怠缓。於是恢出击,大破之,追奔逐北,南至槃江,东接牂牁,与亮声势相连。南土平定,恢军功居多,封汉兴亭侯,加安汉将军。后军还,南夷复叛,杀害守将。恢身往扑讨,鉏尽恶类,徙其豪帅于成都,赋出叟、濮耕牛战马金银犀革,充继军资,于时费用不乏。

建兴七年,以交州属吴,解恢刺史。更领建宁太守,以还居本郡。徙居汉中,九年卒。子遗嗣。恢弟子球,羽林右部督,随诸葛瞻拒邓艾,临陈授命,死于绵竹。

吕凯字季平、永昌不韦人也。仕郡五官掾功曹。时雍闿等闻先主薨於永安,骄黠滋甚。都护李严与闿书六纸,解喻利害,闿但答一纸曰:“盖闻天无二日,土无二王,今天下鼎立,正朔有三,是以远人惶惑,不知所归也。”其桀慢如此。闿又降於吴,吴遥署闿为永昌太守。永昌既在益州郡之西,道路壅塞,与蜀隔绝,而郡太守改易,凯与府丞蜀郡王伉帅厉吏民,闭境拒闿。闿数移檄永昌,称说云云。凯答檄曰:“天降丧乱,奸雄乘衅,天下切齿,万国悲悼,臣妾大小,莫不思竭筋力,肝脑涂地,以除国难。伏惟将军世受汉恩,以为当躬聚党众,率先启行,上以报国家,下不负先人,书功竹帛,遗名千载。何期臣仆吴越,背本就末乎?昔舜勤民事,陨于苍梧,书籍嘉之,流声无穷。崩于江浦,何足可悲!文、武受命,成王乃平。先帝龙兴,海内望风,宰臣聪睿,自天降康。而将军不睹盛衰之纪,成败之符,譬如野火在原,蹈履河冰,火灭冰泮,将何所依附?曩者将军先君雍侯,造怨而封,窦融知兴,归志世祖,皆流名后叶,世歌其美。今诸葛丞相英才挺出,深睹未萌,受遗讬孤,翊赞季兴,与众无忌,录功忘瑕。将军若能翻然改图,易迹更步,古人不难追,鄙土何足宰哉!盖闻楚国不恭,齐桓是责,夫差僣号,晋人不长,况臣於非主,谁肯归之邪?窃惟古义,臣无越境之交,是以前后有来无往。重承告示,发愤忘食,故略陈所怀,惟将军察焉。”凯威恩内著,为郡中所信,故能全其节。

及丞相亮南征讨闿,既发在道,而闿已为高定部曲所杀。亮至南,上表曰:“永昌郡吏吕凯、府丞王伉等,执忠绝域,十有馀年,雍闿、高定偪其东北,而凯等守义不与交通。臣不意永昌风俗敦直乃尔!”以凯为云南太守,封阳迁亭侯。会为叛夷所害,子祥嗣。而王伉亦封亭侯,为永昌太守。

马忠字德信,巴西阆中人也。少养外家,姓狐,名笃,后乃复姓,改名忠。为郡吏,建安末举孝廉,除汉昌长。先主东征,败绩猇亭,巴西太守阎芝发诸县兵五千人以补遗阙,遣忠送往。先主已还永安,见忠与语,谓尚书令刘巴曰:“虽亡黄权,复得狐笃,此为世不乏贤也。”建兴元年,丞相亮开府,以忠为门下督。三年,亮入南,拜忠牂牁太守。郡丞朱褒反。叛乱之后,忠抚育恤理,甚有威惠。八年,召为丞相参军,副长史蒋琬署留府事。又领州治中从事。明年,亮出祁山,忠诣亮所,经营戎事。军还,督将军张嶷等讨汶山郡叛羌。十一年,南夷豪帅刘胄反,扰乱诸郡。徵庲降都督张翼还,以忠代翼。忠遂斩胄,平南土。加忠监军奋威将军,封博阳亭侯。初,建宁郡杀太守正昂,缚太守张裔於吴,故都督常驻平夷县。至忠,乃移治味县,处民夷之间。又越巂郡亦久失土地,忠率将太守张嶷开复旧郡,由此就加安南将军,进封彭乡亭侯。延熙五年还朝,因至汉中,见大司马蒋琬,宣传诏旨,加拜镇南大将军。七年春,大将军费祎北御魏敌,留忠成都,平尚书事。祎还,忠乃归南。十二年卒,子脩嗣。

忠为人宽济有度量,但诙啁大笑,忿怒不形於色。然处事能断,威恩并立,是以蛮夷畏而爱之。及卒,莫不自致丧庭,流涕尽哀,为之立庙祀,迄今犹在。

张表,时名士,清望逾忠。阎宇,宿有功幹,於事精勤。继踵在忠后,其威风称绩,皆不及忠。

王平字子均,巴西宕渠人也。本养外家何氏,后复姓王。随杜濩、朴胡诣洛阳,假校尉,从曹公征汉中,因降先主,拜牙门将、裨将军。建兴六年,属参军马谡先锋。谡舍水上山,举措烦扰,平连规谏谡,谡不能用,大败於街亭。众尽星散,惟平所领千人,鸣鼓自持,魏将张郃疑其伏兵,不往偪也。於是平徐徐收合诸营遗迸,率将士而还。丞相亮既诛马谡及将军张休、李盛,夺将军黄袭等兵,平特见崇显,加拜参军,统五部兼当营事,进位讨寇将军,封亭侯。九年,亮围祁山,平别守南围。魏大将军司马宣王攻亮,张郃攻平,平坚守不动,郃不能克。十二年,亮卒於武功,军退还,魏延作乱,一战而败,平之功也。迁后典军、安汉将军,副车骑将军吴壹住汉中,又领汉中太守。十五年,进封安汉侯,代壹督汉中。延熙元年,大将军蒋琬住沔阳,平更为前护军,署琬府事。六年,琬还住涪,拜平前监军、镇北大将军,统汉中。

七年春,魏大将军曹爽率步骑十馀万向汉川,前锋已在骆谷。时汉中守兵不满三万,诸将大惊。或曰:“今力不足以拒敌,听当固守汉、乐二城,遇贼令入,比尔间,涪军足得救关。”平曰:“不然。汉中去涪垂千里。贼若得关,便为祸也。今宜先遣刘护军、杜参军据兴势,平为后拒;若贼分向黄金,平率千人下自临之,比尔间,涪军行至,此计之上也。”惟护军刘敏与平意同,即便施行。涪诸军及大将军费祎自成都相继而至,魏军退还,如平本策。是时,邓芝在东,马忠在南,平在北境,咸著名迹。

平生长戎旅,手不能书,其所识不过十字,而口授作书,皆有意理。使人读史、汉诸纪传,听之,备知其大义,往往论说不失其指。遵履法度,言不戏谑,从朝至夕,端坐彻日,忄画无武将之体,然性狭侵疑,为人自轻,以此为损焉。十一年卒,子训嗣。

初,平同郡汉昌句扶忠勇宽厚,数有战功,功名爵位亚平,官至左将军,封宕渠侯。

张嶷字伯岐,巴郡南充国人也。弱冠为县功曹。先主定蜀之际,山寇攻县,县长捐家逃亡,嶷冒白刃,携负夫人,夫人得免。由是显名,州召为从事。时郡内士人龚禄、姚伷位二千石,当世有声名,皆与嶷友善。建兴五年,丞相亮北住汉中,广汉、绵竹山贼张慕等钞盗军资,劫掠吏民,嶷以都尉将兵讨之。嶷度其鸟散,难以战禽,乃诈与和亲,克期置酒。酒酣,嶷身率左右,因斩慕等五十馀级,渠帅悉殄。寻其馀类,旬日清泰。后得疾病困笃,家素贫匮,广汉太守蜀郡何祗,名为通厚,嶷宿与疏阔,乃自轝诣祗,讬以治疾。祗倾财医疗,数年除愈。其党道信义皆此类也。拜为牙门将,属马忠,北讨汶山叛羌,南平四郡蛮夷,辄有筹画战克之功。十四年,武都氐王苻健请降,遣将军张尉往迎,过期不到,大将军蒋琬深以为念。嶷平之曰:“苻健求附款至,必无他变,素闻健弟狡黠,又夷狄不能同功,将有乖离,是以稽留耳。”数日,问至,健弟果将四百户就魏,独健来从。

初,越巂郡自丞相亮讨高定之后,叟夷数反,杀太守龚禄、焦璜,是后太守不敢之郡,只住安上县,去郡八百馀里,其郡徒有名而已。时论欲复旧郡,除嶷为越巂太守,嶷将所领往之郡,诱以恩信,蛮夷皆服,颇来降附。北徼捉马最骁劲,不承节度,嶷乃往讨,生缚其帅魏狼,又解纵告喻,使招怀馀类。表拜狼为邑侯,种落三千馀户皆安土供职。诸种闻之,多渐降服,嶷以功赐爵关内侯。

苏祁邑君冬逢、逢弟隗渠等,已降复反。嶷诛逢。逢妻,旄牛王女,嶷以计原之。而渠逃入西徼。渠刚猛捷悍,为诸种深所畏惮,遣所亲二人诈降嶷,实取消息。嶷觉之,许以重赏,使为反间,二人遂合谋杀渠。渠死,诸种皆安。又斯都耆帅李求承,昔手杀龚禄,嶷求募捕得,数其宿恶而诛之。

始嶷以郡郛宇颓坏,更筑小坞。在官三年,徙还故郡,缮治城郭,夷种男女莫不致力。

定莋、台登、卑水三县去郡三百馀里,旧出盐铁及漆,而夷徼久自固食。嶷率所领夺取,署长吏焉。嶷之到定莋,定莋率豪狼岑,槃木王舅,甚为蛮夷所信任,忿嶷自侵,不自来诣。嶷使壮士数十直往收致,挞而杀之,持尸还种,厚加赏赐,喻以狼岑之恶,且曰:“无得妄动,动即殄矣!”种类咸面缚谢过。嶷杀牛飨宴,重申恩信,遂获盐铁,器用周赡。

汉嘉郡界旄牛夷种类四千馀户,其率狼路,欲为姑婿冬逢报怨,遣叔父离将逢众相度形势。嶷逆遣亲近赍牛酒劳赐,又令离逆逢妻宣畅意旨。离既受赐,并见其姊,姊弟欢悦,悉率所领将诣嶷,嶷厚加赏待,遣还。旄牛由是辄不为患。

郡有旧道,经旄牛中至成都,既平且近;自旄牛绝道,已百馀年,更由安上,既险且远。嶷遣左右赍货币赐路,重令路姑喻意,路乃率兄弟妻子悉诣嶷,嶷与盟誓,开通旧道,千里肃清,复古亭驿。奏封路为旄牛〈田句〉毗王,遣使将路朝贡。后主於是加嶷怃戎将军,领郡如故。

嶷初见费祎为大将军,恣性汎爱,待信新附太过,嶷书戒之曰:“昔岑彭率师,来歙杖节,咸见害於刺客,今明将军位尊权重,宜鉴前事,少以为警”后祎果为魏降人郭脩所害。

吴太傅诸葛恪以初破魏军,大兴兵众以图攻取。侍中诸葛瞻,丞相亮之子,恪从弟也,嶷与书曰:“东主初崩,帝实幼弱,太傅受寄讬之重,亦何容易!亲以周公之才,犹有管、蔡流言之变,霍光受任,亦有燕、盖、上官逆乱之谋,赖成、昭之明,以免斯难耳。昔每闻东主杀生赏罚,不任下人,又今以垂没之命,卒召太傅,属以后事,诚实可虑。加吴、楚剽急,乃昔所记,而太傅离少主,履敌庭,恐非良计长算之术也。虽云东家纲纪肃然,上下辑睦,百有一失,非明者之虑邪?取古则今,今则古也,自非郎君进忠言於太傅,谁复有尽言者也!旋军广农,务行德惠,数年之中,东西并举,实为不晚,愿深采察。”恪竟以此夷族。嶷识见多如是类。

在郡十五年,邦域安穆。屡乞求还,乃徵诣成都。民夷恋慕,扶毂泣涕,过旄牛邑,邑君襁负来迎,及追寻至蜀郡界,其督相率随嶷朝贡者百馀人。嶷至,拜荡寇将军,慷慨壮烈,士人咸多贵之,然放荡少礼,人亦以此讥焉,是岁延熙十七年也。魏狄道长李简密书请降,卫将军姜维率嶷等因简之资以出陇西。既到狄道,简悉率城中吏民出迎军。军前与魏将徐质交锋,嶷临陈陨身,然其所杀伤亦过倍。既亡,封长子瑛西乡侯,次子护雄袭爵。南土越巂民夷闻嶷死,无不悲泣,为嶷立庙,四时水旱辄祀之。

评曰:黄权弘雅思量,李恢公亮志业,吕凯守节不回,马忠扰而能毅,王平忠勇而严整,张嶷识断明果,咸以所长,显名发迹,遇其时也。

\part{蜀书十四}
\chapter{蒋琬费祎姜维传第十四}

蒋琬字公琰、零陵湘乡人也。弱冠与外弟泉陵刘敏俱知名。琬以州书佐随先主入蜀,除广都长。先主尝因游观奄至广都,见琬众事不理,时又沈醉,先主大怒,将加罪戮。军师将军诸葛亮请曰:“蒋琬,社稷之器,非百里之才也。其为政以安民为本,不以脩饰为先,愿主公重加察之。”先主雅敬亮,乃不加罪,仓卒但免官而已。琬见推之后,夜梦有一牛头在门前,流血滂沱,意甚恶之,呼问占梦赵直。直曰:“夫见血者,事分明也。牛角及鼻,‘公’字之象,君位必当至公,大吉之徵也。”顷之,为什邡令。先主为汉中王,琬入为尚书郎。建兴元年,丞相亮开府,辟琬为东曹掾。举茂才,琬固让刘邕、阴化、庞延、廖淳,亮教答曰:“思惟背亲舍德,以殄百姓,众人既不隐於心,实又使远近不解其义,是以君宜显其功举,以明此选之清重也。”迁为参军。五年,亮住汉中,琬与长史张裔统留府事。八年,代裔为长史,加抚军将军。亮数外出,琬常足食足兵以相供给。亮每言:“公琰讬志忠雅,当与吾共赞王业者也。”密表后主曰:“臣若不幸,后事宜以付琬。”

亮卒,以琬为尚书令,俄而加行都护,假节,领益州刺史,迁大将军,录尚书事,封安阳亭侯。时新丧元帅,远近危悚。琬出类拔萃,处群僚之右,既无戚容,又无喜色,神守举止,有如平日,由是众望渐服,延熙元年,诏琬曰:“寇难未弭,曹叡骄凶,辽东三郡苦其暴虐,遂相纠结,与之离隔。叡大兴众役,还相攻伐。曩秦之亡,胜、广首难,今有此变,斯乃天时。君其治严,总帅诸军屯住汉中,须吴举动,东西掎角,以乘其衅。”又命琬开府,明年就加为大司马。

东曹掾杨戏素性简略,琬与言论,时不应答。或欲构戏於琬曰:“公与戏语而不见应,戏之慢上,不亦甚乎!”琬曰:“人心不同,各如其面;面从后言,古人之所诫也。戏欲赞吾是耶,则非其本心,欲反吾言,则显吾之非,是以默然,是戏之快也。”又督农杨敏曾毁琬曰:“作事愦愦,诚非及前人。”或以白琬,主者请推治敏,琬曰:“吾实不如前人,无可推也。”主者重据听不推,则乞问其愦愦之状。琬曰:“苟其不如,则事不当理,事不当理,则愦愦矣。复何问邪?”后敏坐事系狱,众人犹惧其必死,琬心无適莫,得免重罪。其好恶存道,皆此类也。

琬以为昔诸葛亮数闚秦川,道险运艰,竟不能克,不若乘水东下。乃多作舟船,欲由汉,沔袭魏兴、上庸。会旧疾连动,未时得行。而众论咸谓如不克捷,还路甚难,非长策也。於是遣尚书令费祎、中监军姜维等喻指。琬承命上疏曰:“芟秽弭难,臣职是掌。自臣奉辞汉中,已经六年,臣既闇弱,加婴疾疢,规方无成,夙夜忧惨。今魏跨带九州,根蒂滋蔓,平除未易。若东西并力,首尾掎角,虽未能速得如志,且当分裂蚕食,先摧其支党。然吴期二三,连不克果,俯仰惟艰,实忘寝食。辄与费祎等议,以凉州胡塞之要,进退有资,贼之所惜;且羌、胡乃心思汉如渴,又昔偏军入羌,郭淮破走,算其长短,以为事首,宜以姜维为凉州刺史。若维征行,衔持河右,臣当帅军为维镇继。今涪水陆四通,惟急是应,若东北有虞,赴之不难。”由是琬遂还住涪。疾转增剧,至九年卒,谥曰恭。

子斌嗣,为绥武将军、汉城护军。魏大将军锺会至汉城,与斌书曰:“巴蜀贤智文武之士多矣。至於足下、诸葛思远,譬诸草木,吾气类也。桑梓之敬,古今所敦。西到,欲奉瞻尊大君公侯墓,当洒扫坟茔,奉祠致敬。愿告其所在!“斌答书曰:“知惟臭味意眷之隆,雅讬通流,未拒来谓也。亡考昔遭疾疢,亡於涪县,卜云其吉,遂安厝之。知君西迈,乃欲屈驾脩敬坟墓。视予犹父,颜子之仁也,闻命感怆,以增情思。”会得斌书报,嘉叹意义,及至涪,如其书云。

后主既降邓艾,斌诣会於涪,待以交友之礼。随会至成都,为乱兵所杀。斌弟显,为太子仆,会亦爱其才学,与斌同时死。

刘敏,左护军、扬威将军,与镇北大将军王平俱镇汉中。魏遣大将军曹爽袭蜀时,议者或谓但可守城,不出拒敌,必自引退。敏以为男女布野,农谷栖亩,若听敌入,则大事去矣。遂帅所领与平据兴势,多张旗帜,弥亘百馀里。会大将军费祎从成都至,魏军即退,敏以功封云亭侯。

费祎字文伟,江夏鄳人也。少孤,依族父伯仁。伯仁姑,益州牧刘璋之母也。璋遣使迎仁,仁将祎游学入蜀。会先主定蜀,祎遂留益土,与汝南许叔龙、南郡董允齐名。时许靖丧子,允与祎欲共会其葬所。允白父和请车,和遣开后鹿车给之。允有难载之色,祎便从前先上。及至丧所,诸葛亮及诸贵人悉集,车乘甚鲜,允犹神色未泰,而祎晏然自若。持车人还,和问之,知其如此,乃谓允曰:“吾常疑汝於文伟优劣未别也,而今而后,吾意了矣。”

先主立太子,祎与允俱为舍人,迁庶子。后主践位,为黄门侍郎。丞相亮南征还,群寮於数十里逢迎,年位多在祎右,而亮特命祎同载,由是众人莫不易观。亮以初从南归,以祎为昭信校尉使吴。孙权性既滑稽,嘲啁无方,诸葛恪、羊衟等才博果辩,论难锋至,祎辞顺义笃,据理以答,终不能屈。权甚器之,谓祎曰:“君天下淑德,必当股肱蜀朝,恐不能数来也。”还,迁为侍中。亮北住汉中,请祎为参军。以奉使称旨,频烦至吴。建兴八年,转为中护军,后又为司马。值军师魏延与长史杨仪相憎恶,每至并坐争论,延或举刃拟仪,仪泣涕横集。祎常入其坐间,谏喻分别,终亮之世,各尽延、仪之用者,祎匡救之力也。亮卒,祎为后军师。顷之,代蒋琬为尚书令。琬自汉中还涪,祎迁大将军,录尚书事。

延熙七年,魏军次于兴势,假祎节,率众往御之。光禄大夫来敏至祎许别,求共围棋。于时羽檄交驰。人马擐甲,严驾已讫,祎与敏留意对戏,色无厌倦。敏曰:“向聊观试君耳!君信可人,必能办贼者也。”祎至,敌遂退,封成乡侯。琬固让州职,祎复领益州刺史。祎当国功名,略与琬比。十一年,出住汉中。自琬及祎,虽自身在外,庆赏刑威,皆遥先谘断,然后乃行,其推任如此。后十四年夏,还成都,成都望气者云都邑无宰相位,故冬复北屯汉寿。延熙十五年,命祎开府。十六年岁首大会,魏降人郭脩在坐。祎欢饮沈醉,为脩手刃所害,谥曰敬侯。子承嗣,为黄门侍郎。承弟恭,尚公主。祎长女配太子璿为妃。

姜维字伯约,天水冀人也。少孤,与母居。好郑氏学。仕郡上计掾,州辟为从事。以父冏昔为郡功曹,值羌、戎叛乱,身卫郡将,没於战场,赐维官中郎,参本郡军事。建兴六年,丞相诸葛亮军向祁山,时天水太守適出案行,维及功曹梁绪、主簿尹赏、主记梁虔等从行。太守闻蜀军垂至,而诸县响应,疑维等皆有异心,於是夜亡保上邽。维等觉太守去,追迟,至城门,城门已闭,不纳。维等相率还冀,冀亦不入维。维等乃俱诣诸葛亮。会马谡败於街亭,亮拔将西县千馀家及维等还,故维遂与母相失。亮辟维为仓曹掾,加奉义将军,封当阳亭侯,时年二十七。亮与留府长史张裔、参军蒋琬书曰:“姜伯约忠勤时事,思虑精密,考其所有,永南、季常诸人不如也。其人,凉州上士也。”又曰:“须先教中虎步兵五六千人。姜伯约甚敏於军事,既有胆义,深解兵意。此人心存汉室,而才兼於人,毕教军事,当遣诣宫,觐见主上。”后迁中监军征西将军。

十二年,亮卒,维还成都,为右监军辅汉将军,统诸军,进封平襄侯。延熙元年,随大将军蒋琬住汉中。琬既迁大司马,以维为司马,数率偏军西入。六年,迁镇西大将军,领凉州刺史。十年,迁卫将军,与大将军费祎共录尚书事。是岁,汶山平康夷反,维率众讨定之。又出陇西、南安、金城界,与魏大将军郭淮、夏侯霸等战於洮西。胡王治无戴等举部落降,维将还安处之。十二年,假维节,复出西平,不克而还。维自以练西方风俗,兼负其才武,欲诱诸羌、胡以为羽翼,谓自陇以西可断而有也。每欲兴军大举,费祎常裁制不从,与其兵不过万人。

十六年春,祎卒。夏,维率数万人出石营,经董亭,围南安,魏雍州刺史陈泰解围至洛门,维粮尽退还。明年,加督中外军事。复出陇西,守狄道长李简举城降。进围襄武,与魏将徐质交锋,斩首破敌,魏军败退。维乘胜多所降下,拔河关、狄道、临洮三县民还,后十八年,复与车骑将军夏侯霸等俱出狄道,大破魏雍州刺史王经於洮西,经众死者数万人。经退保狄道城,维围之。魏征西将军陈泰进兵解围,维卻住锺题。

十九年春,就迁维为大将军。更整勒戎马,与镇西大将军胡济期会上邽,济失誓不至,故维为魏大将邓艾所破於段谷,星散流离,死者甚众。众庶由是怨讟,而陇已西亦骚动不宁,维谢过引负,求自贬削。为后将军,行大将军事。

二十年,魏征东大将军诸葛诞反於淮南,分关中兵东下。维欲乘虚向秦川,复率数万人出骆谷,径至沈岭。时长城积谷甚多而守兵乃少,闻维方到,众皆惶惧。魏大将军司马望拒之,邓艾亦自陇右,皆军于长城。维前住芒水,皆倚山为营。望、艾傍渭坚围,维数下挑战,望、艾不应。景耀元年,维闻诞破败,乃还成都。复拜大将军。

初,先主留魏延镇汉中,皆实兵诸围以御外敌,敌若来攻,使不得入。及兴势之役,王平捍拒曹爽,皆承此制。维建议,以为错守诸围,虽合周易“重门”之义,然適可御敌,不获大利。不若使闻敌至,诸围皆敛兵聚谷,退就汉、乐二城,使敌不得入平,且重关镇守以捍之。有事之日,令游军并进以伺其虚。敌攻关不克,野无散谷,千里县粮,自然疲乏。引退之日,然后诸城并出,与游军并力搏之,此殄敌之术也。於是令督汉中胡济卻住汉寿,监军王含守乐城,护军蒋斌守汉城,又於西安、建威、武卫、石门、武城、建昌、临远皆立围守。

五年,维率众出汉、侯和,为邓艾所破,还住沓中。维本羁旅讬国,累年攻战,功绩不立,而宦官黄皓等弄权於内,右大将军阎宇与皓协比,而皓阴欲废维树宇。维亦疑之。故自危惧,不复还成都。六年,维表后主:“闻锺会治兵关中,欲规进取,宜并遣张翼、廖化督诸军分护阳安关口、阴平桥头以防未然。”皓徵信鬼巫,谓敌终不自致,启后主寝其事,而群臣不知。及锺会将向骆谷,邓艾将入沓中,然后乃遣右车骑廖化诣沓中为维援,左车骑张翼、辅国大将军董厥等诣阳安关口以为诸围外助。比至阴平,闻魏将诸葛绪向建威,故住待之。月馀,维为邓艾所摧,还住阴平。锺会攻围汉、乐二城,遣别将进攻关口,蒋舒开城出降,傅佥格斗而死。会攻乐城,不能克,闻关口已下,长驱而前。翼、厥甫至汉寿,维、化亦舍阴平而退,適与翼、厥合,皆退保剑阁以拒会。会与维书曰:“公侯以文武之德,怀迈世之略,功济巴、汉,声畅华夏,远近莫不归名。每惟畴昔,尝同大化,吴札、郑乔,能喻斯好。”维不答书,列营守险。会不能克,粮运县远,将议还归。

而邓艾自阴平由景谷道傍入,遂破诸葛瞻於绵竹。后主请降於艾,艾前据成都。维等初闻瞻破,或闻后主欲固守成都,或闻欲东入吴,或闻欲南入建宁,於是引军由广汉、郪道以审虚实。寻被后主敕令,乃投戈放甲,诣会於涪军前,将士咸怒,拔刀砍石。

会厚待维等,皆权还其印号节盖。会与维出则同轝,坐则同席,谓长史杜预曰:“以伯约比中土名士,公休、太初不能胜也。”会既构邓艾,艾槛车徵,因将维等诣成都,自称益州牧以叛。欲授维兵五万人,使为前驱。魏将士愤怒,杀会及维,维妻子皆伏诛。

郤正著论论维曰:“姜伯约据上将之重,处群臣之右,宅舍弊薄,资财无馀,侧室无妾媵之亵,后庭无声乐之娱,衣服取供,舆马取备,饮食节制,不奢不约,官给费用,随手消尽;察其所以然者,非以激贪厉浊,抑情自割也,直谓如是为足,不在多求。凡人之谈,常誉成毁败,扶高抑下,咸以姜维投厝无所,身死宗灭,以是贬削,不复料擿,异乎春秋褒贬之义矣。如姜维之乐学不倦,清素节约,自一时之仪表也。”

维昔所俱至蜀,梁绪官至大鸿胪,尹赏执金吾,梁虔大长秋,皆先蜀亡没。

评曰:蒋琬方整有威重,费祎宽济而博爱,咸承诸葛之成规,因循而不革,是以边境无虞,邦家和一,然犹未尽治小之宜,居静之理也。姜维粗有文武,志立功名,而玩众黩旅,明断不周,终致陨毙。老子有云:“治大国者犹烹小鲜。”况於区区蕞尔,而可屡扰乎哉?

\part{蜀书十五}
\chapter{邓张宗杨传第十五}

邓芝字伯苗,义阳新野人,汉司徒禹之后也。汉末入蜀,未见知待。时益州从事张裕善相,芝往从之,裕谓芝曰:“君年过七十,位至大将军,封侯。”芝闻巴西太守庞羲好士,往依焉。先主定益州,芝为郫邸阁督。先主出至郫,与语,大奇之,擢为郫令,迁广汉太守。所在清严有治绩,入为尚书。

先主薨於永安。先是,吴王孙权请和,先主累遣宋玮、费祎等与相报答。丞相诸葛亮深虑权闻先主殂陨,恐有异计,未知所如。芝见亮曰:“今主上幼弱,初在位,宜遣大使重申吴好。”亮答之曰:“吾思之久矣,未得其人耳,今日始得之。”芝问其人为谁?亮曰:“即使君也。”乃遣芝脩好於权。权果狐疑,不时见芝,芝乃自表请见权曰:“臣今来亦欲为吴,非但为蜀也。“权乃见之,语芝曰:“孤诚愿与蜀和亲,然恐蜀主幼弱,国小势偪,为魏所乘,不自保全,以此犹豫耳。”芝对曰:“吴、蜀二国四州之地,大王命世之英,诸葛亮亦一时之杰也。蜀有重险之固,吴有三江之阻,合此二长,共为唇齿,进可并兼天下,退可鼎足而立,此理之自然也。大王今若委质於魏,魏必上望大王之入朝,下求太子之内侍,若不从命,则奉辞伐叛,蜀必顺流见可而进,如此,江南之地非复大王之有也。”权默然良久曰:“君言是也。”遂自绝魏,与蜀连和,遣张温报聘於蜀。蜀复令芝重往,权谓芝曰:“若天下太平,二主分治,不亦乐乎!”芝对曰:“夫天无二日,土无二王,如并魏之后,大王未深识天命者也,君各茂其德,臣各尽其忠,将提枹鼓,则战争方始耳。“权大笑曰:“君之诚款,乃当尔邪!”权与亮书曰:“丁厷掞张,阴化不尽;和合二国,唯有邓芝。”及亮北住汉中,以芝为中监军、扬武将军。亮卒,迁前军师前将军,领衮州刺史,封阳武亭侯,顷之为督江州。权数与芝相闻,馈遗优渥。延熙六年,就迁为车骑将军,后假节。十一年,涪陵国人杀都尉反叛,芝率军征讨,即枭其渠帅,百姓安堵,十四年卒。

芝为将军二十馀年,赏罚明断,善恤卒伍。身之衣食资仰於官,不苟素俭,然终不治私产,妻子不免饥寒,死之日家无馀财。性刚简,不饰意气,不得士类之和。於时人少所敬贵,唯器异姜维云。子良,袭爵,景耀中为尚书左选郎,晋朝广汉太守。

张翼字伯恭,犍为武阳人也。高祖父司空浩,曾祖父广陵太守纲,皆有名迹。先主定益州,领牧,翼为书佐。建安末,举孝廉,为江阳长,徙涪陵令,迁梓潼太守,累迁至广汉、蜀郡太守。建兴九年,为庲降都督、绥南中郎将。翼性持法严,不得殊俗之欢心。耆率刘胄背叛作乱,翼举兵讨胄。胄未破,会被徵当还,群下咸以为宜便驰骑即罪,翼曰:“不然。吾以蛮夷蠢动,不称职故还耳,然代人未至,吾方临战场,当运粮积谷,为灭贼之资,岂可以黜退之故而废公家之务乎?“於是统摄不懈,代到乃发。马忠因其成基以破殄胄,丞相亮闻而善之。亮出武功,以翼为前军都督,领扶风太守。亮卒,拜前领军,追论讨刘胄功,赐爵关内侯。延熙元年,入为尚书,稍迁督建威,假节,进封都亭侯,征西大将军。

十八年,与卫将军姜维俱还成都。维议复出军,唯翼廷争,以为国小民劳,不宜黩武。维不听,将翼等行,进翼位镇南大将军。维至狄道,大破魏雍州刺史王经,经众死於洮水者以万计。翼曰:“可止矣,不宜复进,进或毁此大功。”维大怒。曰:“为蛇画足。”维竟围经於狄道,城不能克。自翼建异论,维心与翼不善,然常牵率同行,翼亦不得已而往。景耀二年,迁左车骑将军,领冀州刺史。六年,与维咸在剑阁,共诣降锺会于涪。明年正月,随会至成都,为乱兵所杀。

宗预字德艳,南阳安众人也。建安中,随张飞入蜀。建兴初,丞相亮以为主簿,迁参军右中郎将。及亮卒,吴虑魏或承衰取蜀,增巴丘守兵万人,一欲以为救援,二欲以事分割也。蜀闻之,亦益永安之守,以防非常。预将命使吴,孙权问预曰:“东之与西,譬犹一家,而闻西更增白帝之守,何也?”预对曰:“臣以为东益巴丘之戍,西增白帝之守,皆事势宜然,俱不足以相问也。”权大笑,嘉其抗直,甚爱待之,见敬亚於邓芝、费祎。迁为侍中,徙尚书。延熙十年,为屯骑校尉。时车骑将军邓芝自江州还,来朝,谓预曰:“礼,六十不服戎,而卿甫受兵,何也?”预答曰:“卿七十不还兵,我六十何为不受邪?”芝性刚简,自大将军费祎等皆避下之,而预独不为屈。预复东聘吴,孙权捉预手,涕泣而别曰:“君每衔命结二国之好。今君年长,孤亦衰老,恐不复相见!”遗预大珠一斛,乃还。迁后将军,督永安,就拜征西大将军,赐爵关内侯。景耀元年,以疾徵还成都。后为镇军大将军,领兖州刺史。时都护诸葛瞻初统朝事,廖化过预,欲与预共诣瞻许。预曰:“吾等年逾七十,所窃已过,但少一死耳,何求於年少辈而屑屑造门邪?”遂不往。

廖化字元俭,本名淳,襄阳人也。为前将军关羽主簿,羽败,属吴。思归先主,乃诈死,时人谓为信然,因携持老母昼夜西行。会先主东征,遇於秭归。先主大悦,以化为宜都太守。先主薨,为丞相参军,后为督广武,稍迁至右车骑将军,假节,领并州刺史,封中乡侯,以果烈称。官位与张翼齐,而在宗预之右。

咸熙元年春,化、预俱内徙洛阳,道病卒。

杨戏字文然,犍为武阳人也。少与巴西程祁公弘、巴郡杨汰季儒、蜀郡张表伯达并知名。戏每推祁以为冠首,丞相亮深识之。戏年二十馀,从州书佐为督军从事,职典刑狱,论法决疑,号为平当,府辟为属主簿。亮卒,为尚书右选部郎,刺史蒋琬请为治中从事史。琬以大将军开府,又辟为东曹掾,迁南中郎参军,副贰庲降都督,领建宁太守。以疾徵还成都,拜护军监军,出领梓潼太守,入为射声校尉,所在清约不烦。延熙二十年,随大将军姜维出军至芒水。戏素心不服维,酒后言笑,每有傲弄之辞。维外宽内忌,意不能堪,军还,有司承旨奏戏,免为庶人。后景耀四年卒。

戏性虽简惰省略,未尝以甘言加人,过情接物。书符指事,希有盈纸。然笃於旧故,居诚存厚。与巴西韩俨、黎韬童幼相亲厚,后俨痼疾废顿,韬无行见捐,戏经纪振恤,恩好如初。又时人谓谯周无当世才,少归敬者,唯戏重之,尝称曰:“吾等后世,终自不如此长儿也。”有识以此贵戏。

张表有威仪风观,始名位与戏齐,后至尚书,督庲降后将军,先戏没。祁、汰各早死。

戏以延熙四年著季汉辅臣赞,其所颂述,今多载于蜀书,是以记之於左。自此之后卒者,则不追谥,故或有应见称纪而不在乎篇者也。其戏之所赞而今不作传者,余皆注疏本末於其辞下,可以觕知其仿佛云尔。

昔文王歌德,武王歌兴,夫命世之主,树身行道,非唯一时,亦由开基植绪,光于来世者也。自我中汉之末,王纲弃柄,雄豪并起,役殷难结,生人涂地。於是世主感而虑之,初自燕、代则仁声洽著,行自齐、鲁则英风播流,寄业荆、郢则臣主归心,顾援吴、越则贤愚赖风,奋威巴、蜀则万里肃震,厉师庸、汉则元寇敛迹,故能承高祖之始兆,复皇汉之宗祀也。然而奸凶怼险,天征未加,犹孟津之翔师,复须战於鸣条也。天禄有终,奄忽不豫。虽摄归一统,万国合从者,当时俊乂扶携翼戴,明德之所怀致也,盖济济有可观焉。遂乃并述休风,动于后听。其辞曰:

皇帝遗植,爰滋八方,别自中山,灵精是锺,顺期挺生,杰起龙骧。始于燕、代,伯豫君荆,吴、越凭赖,望风请盟,挟巴跨蜀,庸汉以并。乾坤复秩,宗祀惟宁,蹑基履迹,播德芳声。华夏思美,西伯其音,开庆来世,历载攸兴。──赞昭烈皇帝

忠武英高,献策江滨,攀吴连蜀,权我世真。受遗阿衡,整武齐文,敷陈德教,理物移风,贤愚竞心,佥忘其身。诞静邦内,四裔以绥,屡临敌庭,实耀其威,研精大国,恨於未夷。──赞诸葛丞相

司徒清风,是咨是臧,识爱人伦,孔音锵锵。──赞许司徒

关、张赳赳,出身匡世,扶翼携上,雄壮虎烈。藩屏左右,翻飞电发,济于艰难,赞主洪业,侔迹韩、耿,齐声双德。交待无礼,并致奸慝,悼惟轻虑,陨身匡国。──赞关云长、张益德

骠骑奋起,连横合从,首事三秦,保据河、潼。宗计於朝,或异或同,敌以乘衅,家破军亡。乖道反德,讬凤攀龙。──赞马孟起

翼侯良谋,料世兴衰,委质于主,是训是谘,暂思经算,睹事知机。──赞法孝直

军师美至,雅气晔晔,致命明主,忠情发臆,惟此义宗,亡身报德。──赞庞士元

将军敦壮,摧峰登难,立功立事,于时之幹。──赞黄汉升

掌军清节,亢然恒常,谠言惟司,民思其纲。──赞董幼宰

安远强志,允休允烈,轻财果壮,当难不惑,以少御多,殊方保业。──赞邓孔山

孔山名方,南郡人也。以荆州从事随先主入蜀。蜀既定,为犍为属国都尉,因易郡名,为朱提太守,选为安远将军、庲降都督,住南昌县。章武二年卒。失其行事,故不为传。

扬威才幹,欷歔文武,当官理任,衎衎辩举,图殖财施,有义有叙。──赞费宾伯

宾伯名观,江夏鄳人也。刘璋母,观之族姑,璋又以女妻观。观建安十八年参李严军,拒先主於绵竹,与严俱降,先主既定益州,拜为裨将军,后为巴郡太守、江州都督,建兴元年封都亭侯,加振威将军。观为人善於交接。都护李严性自矜高,护军辅匡等年位与严相次,而严不与亲亵;观年少严二十馀岁,而与严通狎如时辈云。年三十七卒。失其行事,故不为传。

屯骑主旧,固节不移,既就初命,尽心世规,军资所恃,是辨是裨。──赞王文仪

尚书清尚,敕行整身,抗志存义,味览典文,倚其高风,好侔古人。──赞刘子初

安汉雍容,或婚或宾,见礼当时,是谓循臣。──赞麋子仲

少府修慎,鸿胪明真,谏议隐行,儒林天文。宣班大化,或首或林。──赞王元泰、何彦英、杜辅国、周仲直

王元泰名谋,汉嘉人也。有容止操行。刘璋时,为巴郡太守,还为州治中从事。先主定益州,领牧,以为别驾。先主为汉中王,用荆楚宿士零陵赖恭为太常,南阳黄柱为光禄勋,谋为少府;建兴初,赐爵关内侯,后代赖恭为太常。恭、柱、谋皆失其行事,故不为传。恭子厷,为丞相西曹令史,随诸葛亮於汉中,早夭,亮甚惜之,与留府长史参军张裔、蒋琬书曰:“令史失赖厷,掾属丧杨颙,为朝中损益多矣。“颙亦荆州人也。后大将军蒋琬问张休曰:“汉嘉前辈有王元泰,今谁继者?“休对曰:“至於元泰,州里无继,况鄙郡乎!“其见重如此。

何彦英名宗,蜀郡郫人也。事广汉任安学,精究安术,与杜琼同师而名问过之。刘璋时,为犍为太守。先主定益州,领牧,辟为从事祭酒。后援引图、谶,劝先主即尊号。践阼之后,迁为大鸿胪。建兴中卒。失其行事,故不为传。子双,字汉偶。滑稽谈笑,有淳于髡、东方朔之风。为双柏长。早卒。

车骑高劲,惟其泛爱,以弱制强,不陷危坠。──赞吴子远

子远名壹,陈留人也。随刘焉入蜀。刘璋时,为中郎将,将兵拒先主於涪,诣降。先主定益州,以壹为护军讨逆将军,纳壹妹为夫人。章武元年,为关中都督。建兴八年,与魏延入南安界,破魏将费瑶,徙亭侯,进封高阳乡侯,迁左将军。十二年,丞相亮卒,以壹督汉中,车骑将军,假节,领雍州刺史,进封济阳侯。十五年卒。失其行事,故不为传。壹族弟班,字元雄,大将军何进官属吴匡之子也。以豪侠称,官位常与壹相亚。先主时,为领军。后主世,稍迁至骠骑将军,假节,封绵竹侯。

安汉宰南,奋击旧乡,翦除芜秽,惟刑以张,广迁蛮、濮,国用用强。──赞李德昂

辅汉惟聪,既机且惠,因言远思,切问近对,赞时休美,和我业世。──赞张君嗣

镇北敏思,筹画有方,导师禳秽,遂事成章。偏任东隅,末命不祥,哀悲本志,放流殊疆。──赞黄公衡

越骑惟忠,厉志自祗,职于内外,念公忘私。──赞杨季休

征南厚重,征西忠克,统时选士,猛将之烈。──赞赵子龙、陈叔至

叔至名到,汝南人也。自豫州随先主,名位常亚赵云,俱以忠勇称。建兴初,官至永安都督、征西将军,封亭侯。

镇南粗强,监军尚笃,并豫戎任,任自封裔。──赞辅元弼、刘南和

辅元弼名匡,襄阳人也。随先主入蜀。益州既定,为巴郡太守。建兴中,徙镇南,为右将军,封中乡侯。

刘南和名邕,义阳人也。随先主入蜀。益州既定,为江阳太守。建兴中,稍迁至监军后将军,赐爵关内侯,卒。子式嗣。少子武,有文,与樊建齐名,官亦至尚书。

司农性才,敷述允章,藻丽辞理,斐斐有光。──赞秦子敕

正方受遗,豫闻后纲,不陈不佥,造此异端,斥逐当时,任业以丧。──赞李正方

文长刚粗,临难受命,折冲外御,镇保国境。不协不和,忘节言乱,疾终惜始,实惟厥性。──赞魏文长

威公狷狭,取异众人;闲则及理,逼则伤侵,舍顺入凶,大易之云。──赞杨威公

季常良实,文经勤类,士元言规,处仁闻计,孔休、文祥,或才或臧,播播述志,楚之兰芳。──赞马季常、卫文经、韩士元、张处仁、殷孔林、习文祥

文经、士元,皆失其名实、行事、郡县。处仁本名存,南阳人也。以荆州从事随先主入蜀,南次至雒,以为广汉太守。存素不服庞统,统中矢卒,先主发言嘉叹,存曰:“统虽尽忠可惜,然违大雅之义。“先主怒曰:“统杀身成仁,更为非也?“免存官。顷之,病卒。失其行事,故不为传。

孔休名观,为荆州主簿别驾从事,见先主传。失其郡县。文祥名祯,襄阳人也。随先主入蜀,历雒、郫令,广汉太守。失其行事。子忠,官至尚书郎。

国山休风,永南耽思;盛衡、承伯,言藏言时;孙德果锐,伟南笃常;德绪、义强,志壮气刚。济济脩志,蜀之芬香。──赞王国山、李永南、马盛衡、马承伯、李孙德、李伟南,龚德绪、王义强

国山名甫,广汉郪人也。好人流言议。刘璋时,为州书佐。先主定蜀后,为绵竹令,还为荆州议曹从事。随先主征吴,军败於秭归,遇害。子祐,有父风,官至尚书右选郎。

永南名邵,广汉郪人也。先主定蜀后,为州书佐部从事。建兴元年,丞相亮辟为西曹掾。亮南征,留邵为治中从事,是岁卒。  盛衡名勋,承伯名齐,皆巴西阆中人也。勋,刘璋时为州书佐,先主定蜀,辟为左将军属,后转州别驾从事,卒。齐为太守张飞功曹。飞贡之先主,为尚书郎。建兴中,从事丞相掾,迁广汉太守,复为参军。亮卒,为尚书。勋、齐皆以才幹自显见;归信於州党,不如姚伷。伷字子绪,亦阆中人。先主定益州后,为功曹书佐。建兴元年,为广汉太守。丞相亮北驻汉中,辟为掾。并进文武之士,亮称曰:“忠益者莫大於进人,进人者各务其所尚;今姚掾并存刚柔,以广文武之用,可谓博雅矣,愿诸掾各希此事,以属其望。“迁为参军。亮卒,稍迁为尚书仆射。时人服其真诚笃粹。延熙五年卒,在作赞之后。

孙德名福,梓潼涪人也。先主定益州后,为书佐、西充国长、成都令。建兴元年,徙巴西太守,为江州督、杨威将军,入为尚书仆射,封平阳亭侯。延熙初,大将军蒋琬出征汉中,福以前监军领司马,卒。

伟南名朝,永南兄。郡功曹,举孝廉,临邛令,入为别驾从事。随先主东征吴,章武二年卒於永安。 德绪名禄,巴西安汉人也。先主定益州,为郡从事牙门将。建兴三年,为越巂太守,随丞相亮南征,为蛮夷所害,时年三十一。弟衡,景耀中为领军。义强名士,广汉郪人,国山从兄也。从先主入蜀后,举孝廉,为符节长,迁牙门将,出为宕渠太守,徙在犍为。会丞相亮南征,转为益州太守,将南行,为蛮夷所害。休元轻寇,损时致害,文进奋身,同此颠沛,患生一人,至於弘大。──赞冯休元、张文进  休元名习,南郡人。随先主入蜀。先主东征吴,习为领军,统诸军,大败於猇亭。

文进名南,亦自荆州随先主入蜀,领兵从先主征吴,与习俱死。时又有义阳傅肜,先主退军,断后拒战,兵人死尽,吴将语肜令降,肜骂曰:“吴狗!何有汉将军降者!”遂战死。拜子佥为左中郎,后为关中都督,景耀六年,又临危授命。论者嘉其父子奕世忠义。

江阳刚烈,立节明君,兵合遇寇,不屈其身,单夫只役,陨命於军。──赞程季然

季然名畿,巴西阆中人也。刘璋时为汉昌长。县有賨人,种类刚猛,昔高祖以定关中。巴西太守庞羲以天下扰乱,郡宜有武卫,颇招合部曲。有谗於璋,说羲欲叛者,璋阴疑之。羲闻,甚惧,将谋自守,遣畿子郁宣旨,索兵自助。畿报曰:“郡合部曲,本不为叛,虽有交构,要在尽诚;若必以惧,遂怀异志,非畿之所闻。“并敕郁曰:“我受州恩,当为州牧尽节。汝为郡吏,当为太守效力,不得以吾故有异志也。“羲使人告畿曰:“尔子在郡,不从太守,家将及祸!“畿曰:“昔乐羊为将,饮子之羹,非父子无恩,大义然也。今虽复羹子,吾必饮之。“羲知畿必不为己,厚陈谢於璋以致无咎。璋闻之,迁畿江阳太守。先主领益州牧,辟为从事祭酒。后随先主征吴,遇大军败绩,溯江而还,或告之曰:“后追已至,解船轻去,乃可以免。“畿曰:“吾在军,未曾为敌走,况从天子而见危哉!“追人遂及畿船,畿身执戟战,敌船有覆者。众大至,共击之,乃死。

公弘后生,卓尔奇精,夭命二十,悼恨未呈。──赞程公弘

公弘,名祁,季然之子也。

古之奔臣,礼有来偪,怨兴司官,不顾大德。靡有匡救,倍成奔北,自绝于人,作笑二国。──赞糜芳、士仁、郝普、潘濬

糜芳字子方,东海人也,为南郡太守。士仁字君义,广阳人也,为将军,住公安,统属关羽;与羽有隙,叛迎孙权。郝普字子太,义阳人。先主自荆州入蜀,以普为零陵太守。为吴将吕蒙所谲,开城诣蒙。潘濬字承明,武陵人也。先主入蜀,以为荆州治中,典留州事,亦与关羽不穆。孙权袭羽,遂入吴。普至廷尉,濬至太常,封侯。

评曰:邓芝坚贞简亮,临官忘家,张翼亢姜维之锐,宗预御孙权之严,咸有可称。杨戏商略,意在不群,然智度有短,殆罹世难云。

\part{吴书一}
\chapter{孙破虏讨逆传第一}

孙坚字文台,吴郡富春人,盖孙武之后也。少为县吏。年十七,与父共载船至钱唐,会海贼胡玉等从匏里上掠取贾人财物,方於岸上分之,行旅皆住,船不敢进。坚谓父曰:“此贼可击,请讨之。”父曰:“非尔所图也。”坚行操刀上岸,以手东西指麾,若分部人兵以罗遮贼状。贼望见,以为官兵捕之,即委财物散走。坚追,斩得一级以还;父大惊。由是显闻,府召署假尉。会稽妖贼许昌起於句章,自称阳明皇帝,与其子韶扇动诸县,众以万数。坚以郡司马募召精勇,得千馀人,与州郡合讨破之。是岁,熹平元年也。刺史臧旻列上功状,诏书除坚盐渎丞,数岁徙盱眙丞,又徙下邳丞。

中平元年,黄巾贼帅张角起于魏郡,讬有神灵,遣八使以善道教化天下,而潜相连结,自称黄天泰平。三月甲子,三十六方一旦俱发,天下响应,燔烧郡县,杀害长吏。汉遣车骑将军皇甫嵩、中郎将朱俊将兵讨击之。俊表请坚为佐军司马,乡里少年随在下邳者皆愿从。坚又募诸商旅及淮、泗精兵,合千许人,与俊并力奋击,所向无前。汝、颍贼困迫,走保宛城。坚身当一面,登城先入,众乃蚁附,遂大破之。俊具以状闻上,拜坚别部司马。

边章、韩遂作乱凉州。中郎将董卓拒讨无功。中平三年,遣司空张温行车骑将军,西讨章等。温表请坚与参军事,屯长安。温以诏书召卓,卓良久乃诣温。温责让卓,卓应对不顺。坚时在坐,前耳语谓温曰:“卓不怖罪而鸱张大语,宜以召不时至,陈军法斩之。”温曰:“卓素著威名於陇蜀之间,今日杀之,西行无依。”坚曰:“明公亲率王兵,威震天下,何赖於卓?观卓所言,不假明公,轻上无礼,一罪也。章、遂跋扈经年,当以时进讨,而卓云未可,沮军疑众,二罪也。卓受任无功,应召稽留,而轩昂自高,三罪也。古之名将,仗钺临众,未有不断斩以示威者也,是以穰苴斩庄贾,魏绛戮杨干。今明公垂意於卓,不即加诛,亏损威刑,於是在矣。”温不忍发举,乃曰:“君且还,卓将疑人。”坚因起出。章、遂闻大兵向至,党众离散,皆乞降。军还,议者以军未临敌,不断功赏,然闻坚数卓三罪,劝温斩之,无不叹息。拜坚议郎。时长沙贼区星自称将军,众万馀人,攻围城邑,乃以坚为长沙太守。到郡亲率将士,施设方略,旬月之间,克破星等。周朝、郭石亦帅徒众起於零、桂,与星相应。遂越境寻讨,三郡肃然。汉朝录前后功,封坚乌程侯。

灵帝崩,卓擅朝政,横恣京城。诸州郡并兴义兵,欲以讨卓。坚亦举兵。荆州刺史王叡素遇坚无礼,坚过杀之。比至南阳,众数万人。南阳太守张咨闻军至,晏然自若。坚以牛酒礼咨,咨明日亦答诣坚。酒酣,长沙主簿入白坚:“前移南阳,而道路不治,军资不具,请收主簿推问意故。”咨大惧欲去,兵陈四周不得出。有顷,主簿复入白坚:“南阳太守稽停义兵,使贼不时讨,请收出案军法从事。”便牵咨於军门斩之。郡中震栗,无求不获。前到鲁阳,与袁术相见。术表坚行破虏将军,领豫州刺史。遂治兵於鲁阳城。当进军讨卓,遣长史公仇称将兵从事还州督促军粮。施帐幔於城东门外,祖道送称,官属并会。卓遣步骑数万人逆坚,轻骑数十先到。坚方行酒谈笑,敕部曲整顿行陈,无得妄动。后骑渐益,坚徐罢坐,导引入城,乃谓左右曰:“向坚所以不即起者,恐兵相蹈籍,诸君不得入耳。”卓兵见坚士众甚整,不敢攻城,乃引还。坚移屯梁东,大为卓军所攻,坚与数十骑溃围而出。坚常著赤罽帻,乃脱帻令亲近将祖茂著之。卓骑争逐茂,故坚从间道得免。茂困迫,下马,以帻冠冢间烧柱,因伏草中。卓骑望见,围绕数重,定近觉是柱,乃去。坚复相收兵,合战於阳人,大破卓军,枭其都督华雄等。是时,或间坚於术,术怀疑,不运军粮。阳人去鲁阳百馀里,坚夜驰见术,画地计校,曰:“所以出身不顾,上为国家讨贼,下慰将军家门之私雠。坚与卓非有骨肉之怨也,而将军受谮润之言,还相嫌疑!”术踧唶,即调发军粮。坚还屯。卓惮坚猛壮,乃遣将军李傕等来求和亲,今坚列疏子弟任刺史、郡守者,许表用之。坚曰:“卓逆天无道,荡覆王室,今不夷汝三族,县示四海,则吾死不瞑目,岂将与乃和亲邪?”复进军大谷,拒雒九十里。卓寻徙都西入关,焚烧雒邑。坚乃前入至雒,脩诸陵,平塞卓所发掘。讫,引军还,住鲁阳。

初平三年,术使坚征荆州,击刘表。表遣黄祖逆於樊、邓之间。坚击破之,追渡汉水,遂围襄阳,单马行岘山,为祖军士所射杀。兄子贲,帅将士众就术,术复表贲为豫州刺史。

坚四子:策、权、翊、匡。权既称尊号,谥坚曰武烈皇帝。

策字伯符。坚初兴义兵,策将母徙居舒,与周瑜相友,收合士大夫,江、淮间人咸向之。坚薨,还葬曲阿。已乃渡江居江都。

徐州牧陶谦深忌策。策舅吴景,时为丹杨太守,策乃载母徙曲阿,与吕范、孙河俱就景,因缘召募得数百人。兴平元年,从袁术。术甚奇之,以坚部曲还策。太傅马日磾杖节安集关东,在寿春以礼辟策,表拜怀义校尉,术大将乔蕤、张勋皆倾心敬焉。术常叹曰:“使术有子如孙郎,死复何恨!”策骑士有罪,逃入术营,隐於内厩。策指使人就斩之,讫,诣术谢。术曰:“兵人好叛,当共疾之,何为谢也?“由是军中益畏惮之。术初许策为九江太守,已而更用丹杨陈纪。后术欲攻徐州,从庐江太守陆康求米三万斛。康不与,术大怒。策昔曾诣康,康不见,使主簿接之。策尝衔恨。术遣策攻康,谓曰:“前错用陈纪,每恨本意不遂。今若得康,庐江真卿有也。“策攻康,拔之,术复用其故吏刘勋为太守,策益失望。先是,刘繇为扬州刺史,州旧治寿春。寿春,术已据之,繇乃渡江治曲阿。时吴景尚在丹杨,策从兄贲又为丹杨都尉,繇至,皆迫逐之。景、贲退舍历阳。繇遣樊能、于麋【陈】东屯横江津,张英屯当利口,以距术。术自用故吏琅邪惠衢为扬州刺史,更以景为督军中郎将,与贲共将兵击英等,连年不克。策乃说术,乞助景等平定江东。术表策为折冲校尉,行殄寇将军,兵财千馀,骑数十匹,宾客愿从者数百人。比至历阳,众五六千。策母先自曲阿徙於历阳,策又徙母阜陵,渡江转斗,所向皆破,莫敢当其锋,而军令整肃,百姓怀之。

策为人,美姿颜,好笑语,性阔达听受,善於用人,是以士民见者,莫不尽心,乐为致死。刘繇弃军遁逃,诸郡守皆捐城郭奔走。吴人严白虎等众各万馀人,处处屯聚。吴景等欲先击破虎等,乃至会稽。策曰:“虎等群盗,非有大志,此成禽耳。“遂引兵渡浙江,据会稽,屠东冶,乃攻破虎等。尽更置长吏,策自领会稽太守,复以吴景为丹杨太守,以孙贲为豫章太守;分豫章为庐陵郡,以贲弟辅为庐陵太守,丹杨朱治为吴郡太守。彭城张昭、广陵张纮、秦松、陈端等为谋主。时袁术僣号,策以书责而绝之。曹公表策为讨逆将军,封为吴侯。后术死,长史杨弘、大将张勋等将其众欲就策,庐江太守刘勋要击,悉虏之,收其珍宝以归。策闻之,伪与勋好盟。勋新得术众,时豫章上缭宗民万馀家在江东,策劝勋攻取之。勋既行,策轻军晨夜袭拔庐江,勋众尽降,勋独与麾下数百人自归曹公。是时袁绍方强,而策并江东,曹公力未能逞,且欲抚之。乃以弟女配策小弟匡,又为子章取贲女,皆礼辟策弟权、翊,又命扬州刺史严象举权茂才。

建安五年,曹公与袁绍相拒於官渡,策阴欲袭许,迎汉帝,密治兵,部署诸将。未发,会为故吴郡太守许贡客所杀。先是,策杀贡,贡小子与客亡匿江边。策单骑出,卒与客遇,客击伤策。创甚,请张昭等谓曰:“中国方乱,夫以吴、越之众,三江之固,足以观成败。公等善相吾弟!”呼权佩以印绶,谓曰:“举江东之众,决机於两阵之间,与天下争衡,卿不如我;举贤任能,各尽其心,以保江东,我不如卿。”至夜卒,时年二十六。

权称尊号,追谥策曰长沙桓王,封子绍为吴侯,后改封上虞侯。绍卒,子奉嗣。孙皓时,讹言谓奉当立,诛死。

评曰:孙坚勇挚刚毅,孤微发迹,导温戮卓,山陵杜塞,有忠壮之烈。策英气杰济,猛锐冠世,览奇取异,志陵中夏。然皆轻佻果躁,陨身致败。且割据江东,策之基兆也,而权尊崇未至,子止侯爵,於义俭矣。

\part{吴书二}
\chapter{吴主传第二}

\begin{yuanwen}
孙权字仲谋。兄策既定诸郡,时权年十五,以为阳羡长。郡察孝廉,州举茂才,行奉义校尉。汉以策远脩职贡,遣使者刘琬加锡命。琬语人曰:“吾观孙氏兄弟虽各才秀明达,然皆禄祚不终,惟中弟孝廉,形貌奇伟,骨体不恒,有大贵之表,年又最寿,尔试识之。”
\end{yuanwen}

\begin{yuanwen}
建安四年,从策征庐江太守刘勋。勋破,进讨黄祖於沙羡。
\end{yuanwen}

\begin{yuanwen}
五年,策薨,以事授权,权哭未及息。策长史张昭谓权曰:“孝廉,此宁哭时邪?且周公立法而伯禽不师,非欲违父,时不得行也。况今奸宄竞逐,豺狼满道,乃欲哀亲戚,顾礼制,是犹开门而揖盗,未可以为仁也。”

乃改易权服,扶令上马,使出巡军。是时惟有会稽、吴郡、丹杨、豫章、庐陵,然深险之地犹未尽从,而天下英豪布在州郡,宾旅寄寓之士以安危去就为意,未有君臣之固。张昭、周瑜等谓权可与共成大业,故委心而服事焉。曹公表权为讨虏将军,领会稽太守,屯吴,使丞之郡行文书事。待张昭以师傅之礼,而周瑜、程普、吕范等为将率。招延俊秀,聘求名士,鲁肃、诸葛瑾等始为宾客。分部诸将,镇抚山越,讨不从命。
\end{yuanwen}

\begin{yuanwen}
七年,权母吴氏薨。
\end{yuanwen}

\begin{yuanwen}
八年,权西伐黄祖,破其舟军,惟城未克,而山寇复动。还过豫章,使吕范平鄱阳,程普讨乐安,太史慈领海昏,韩当、周泰、吕蒙等为剧县令长。
\end{yuanwen}

\begin{yuanwen}
九年,权弟丹杨太守翊为左右所害,以从兄瑜代翊。
\end{yuanwen}

\begin{yuanwen}
十年,权使贺齐讨上饶,分为建平县。

十二年,西征黄祖,虏其人民而还。
\end{yuanwen}

\begin{yuanwen}
十三年春,权复征黄祖,祖先遣舟兵拒军,都尉吕蒙破其前锋,而凌统、董袭等尽锐攻之,遂屠其城。祖挺身亡走,骑士冯则追枭其首,虏其男女数万口。

是岁,使贺齐讨黟、歙,分歙为始新、新定、犁阳、休阳县,以六县为新都郡。荆州牧刘表死,鲁肃乞奉命吊表二子,且以观变。肃未到,而曹公已临其境,表子琮举众以降。刘备欲南济江,肃与相见,因传权旨,为陈成败。备进住夏口,使诸葛亮诣权,权遣周瑜、程普等行。是时曹公新得表众,形势甚盛,诸议者皆望风畏惧,多劝权迎之。惟瑜、肃执拒之议(仪),意与权同。

瑜、普为左右督,各领万人,与备俱进,遇於赤壁,大破曹公军。公烧其馀船引退,士卒饥疫,死者大半。备、瑜等复追至南郡,曹公遂北还,留曹仁、徐晃於江陵,使乐进守襄阳。时甘宁在夷陵,为仁党所围,用吕蒙计,留凌统以拒仁,以其半救宁,军以胜反。权自率众围合肥,使张昭攻九江之当涂。昭兵不利,权攻城逾月不能下。曹公自荆州还,遣张喜将骑赴合肥。未至,权退。
\end{yuanwen}

\begin{yuanwen}

\end{yuanwen}

\begin{yuanwen}

\end{yuanwen}

\begin{yuanwen}

\end{yuanwen}

\begin{yuanwen}

\end{yuanwen}
十四年,瑜、仁相守岁馀,所杀伤甚众。仁委城走。权以瑜为南郡太守。刘备表权行车骑将军,领徐州牧。备领荆州牧,屯公安。

十五年,分豫章为鄱阳郡;分长沙为汉昌郡,以鲁肃为太守,屯陆口。

十六年,权徙治秣陵。明年,城石头,改秣陵为建业。闻曹公将来侵,作濡须坞。

十八年正月,曹公攻濡须,权与相拒月馀。曹公望权军,叹其齐肃,乃退。初,曹公恐江滨郡县为权所略,徵令内移。民转相惊,自庐江、九江、蕲春、广陵户十馀万皆东渡江,江西遂虚,合肥以南惟有皖城。

\begin{yuanwen}
十九年五月,权征皖城。闰月,克之,获庐江太守朱光及参军董和,男女数万口。

是岁刘备定蜀。权以备已得益州,令诸葛瑾从求荆州诸郡。备不许,曰:“吾方图凉州,凉州定,乃尽以荆州与吴耳。”

权曰:“此假而不反,而欲以虚辞引岁。”

遂置南三郡长吏,关羽尽逐之。权大怒,乃遣吕蒙督鲜于丹、徐忠、孙规等兵二万取长沙、零陵、桂阳三郡,使鲁肃以万人屯巴丘以御关羽。权住陆口,为诸军节度。蒙到,二郡皆服,惟零陵太守郝普未下。会备到公安,使关羽将三万兵至益阳,权乃召蒙等使还助肃。蒙使人诱普,普降,尽得三郡将守,因引军还,与孙皎、潘璋并鲁肃兵并进,拒羽於益阳。未战,会曹公入汉中,备惧失益州,使使求和。权令诸葛瑾报,更寻盟好,遂分荆州长沙、江夏、桂阳以东属权,南郡、零陵、武陵以西属备。备归,而曹公已还。权反自陆口,遂征合肥。合肥未下,彻(撤)军还。兵皆就路,权与凌统、甘宁等在津北为魏将张辽所袭,统等以死扞(捍)权,权乘骏马越津桥得去。
\end{yuanwen}

\begin{yuanwen}

\end{yuanwen}

二十一年冬,曹公次于居巢,遂攻濡须。

二十二年春,权令都尉徐详诣曹公请降,公报使脩好,誓重结婚。

\begin{yuanwen}
二十三年十月,权将如吴,亲乘马射虎於庱\footnote{ch\v{e}ng}亭。马为虎所伤,权投以双戟,虎卻废,常从张世击以戈,获之。
\end{yuanwen}

\begin{yuanwen}
二十四年,关羽围曹仁於襄阳,曹公遣左将军于禁救之。会汉水暴起,羽以舟兵尽虏禁等步骑三万送江陵,惟城未拔。权内惮羽,外欲以为己功,笺与曹公,乞以讨羽自效。曹公且欲使羽与权相持以斗之,驿传权书,使曹仁以弩射示羽。羽犹豫不能去。

闰月,权征羽,先遣吕蒙袭公安,获将军士仁。蒙到南郡,南郡太守麋芳以城降。蒙据江陵,抚其老弱,释于禁之囚。陆逊别取宜都,获秭归、枝江、夷道,还屯夷陵,守峡口以备蜀。关羽还当阳,西保麦城。权使诱之。羽伪降,立幡旗为象人於城上,因遁走,兵皆解散,尚十馀骑。权先使朱然、潘璋断其径路。

十二月,璋司马马忠获羽及其子平、都督赵累等於章乡,遂定荆州。是岁大疫,尽除荆州民租税。曹公表权为骠骑将军,假节领荆州牧,封南昌侯。权遣校尉梁寓奉贡于汉,及令王惇市马,又遣朱光等归。
\end{yuanwen}

\begin{yuanwen}
二十五年春正月,曹公薨,太子丕代为丞相魏王,改年为延康。

秋,魏将梅敷使张俭求见抚纳。南阳阴、酂、筑阳、山都、中卢(庐)五县民五千家来附。

冬,魏嗣王称尊号,改元为黄初。
\end{yuanwen}

\begin{yuanwen}
二年四月,刘备称帝於蜀。权自公安都鄂,改名武昌,以武昌、下雉、寻阳、阳新、柴桑、沙羡六县为武昌郡。

五月,建业言甘露降。
\end{yuanwen}

\begin{yuanwen}

\end{yuanwen}
八月,城武昌,下令诸将曰:“夫存不忘亡,安必虑危,古之善教。昔隽不疑汉之名臣,於安平之世而刀剑不离於身,盖君子之於武备,不可以已。况今处身疆畔,豺狼交接,而可轻忽不思变难哉?顷闻诸将出入,各尚谦约,不从人兵,甚非备虑爱身之谓。夫保己遗名,以安君亲,孰与危辱?宜深警戒,务崇其大,副孤意焉。“自魏文帝践阼,权使命称藩,及遣于禁等还。十一月,策命权曰:“盖圣王之法,以德设爵,以功制禄;劳大者禄厚,德盛者礼丰。故叔旦有夹辅之勋,太公有鹰扬之功,并启土宇,并受备物,所以表章元功,殊异贤哲也。近汉高祖受命之初,分裂膏腴以王八姓,斯则前世之懿事,后王之元龟也。朕以不德,承运革命,君临万国,秉统天机,思齐先代,坐而待旦。惟君天资忠亮,命世作佐,深睹历数,达见废兴,远遣行人,浮于潜汉。望风影附,抗疏称藩,兼纳纤絺南方之贡,普遣诸将来还本朝,忠肃内发,款诚外昭,信著金石,义盖山河,朕甚嘉焉。今封君为吴王,使使持节太常高平侯贞,授君玺绶策书、金虎符第一至第五、左竹使符第一至第十,以大将军使持节督交州,领荆州牧事,锡君青土,苴以白茅,对扬朕命,以尹东夏。其上故骠骑将军南昌侯印绶符策。今又加君九锡,其敬听后命。以君绥安东南,纲纪江外,民夷安业,无或携贰,是用锡君大辂、戎辂各一,玄牡二驷。君务财劝农,仓库盈积,是用锡君衮冕之服,赤舄副焉。君化民以德,礼教兴行,是用锡君轩县之乐。君宣导休风,怀柔百越,是用锡君朱户以居。君运其才谋,官方任贤,是用锡君纳陛以登。君忠勇并奋,清除奸慝,是用锡君虎贲之士百人。君振威陵迈,宣力荆南,枭灭凶丑,罪人斯得,是用锡君鈇钺各一。君文和於内,武信於外,是用锡君彤弓一、彤矢百、玈弓十、玈矢千。君以忠肃为基,恭俭为德,是用锡君秬鬯一卣,圭瓒副焉。钦哉!敬敷训典,以服朕命,以勖相我国家,永终尔显烈。”

\begin{yuanwen}
是岁,刘备帅(师)军来伐,至巫山、秭归,使使诱导武陵蛮夷,假与印传,许之封赏。於是诸县及五谿民皆反为蜀。权以陆逊为督,督朱然、潘璋等以拒之。遣都尉赵咨使魏。魏帝问曰:“吴王何等主也?”

咨对曰:“聪明仁智,雄略之主也。”

帝问其状,咨曰:“纳鲁肃於凡品,是其聪也;拔吕蒙於行陈,是其明也;获于禁而不害,是其仁也;取荆州而兵不血刃,是其智也;据三州虎视於天下,是其雄也;屈身於陛下,是其略也。”

帝欲封权子登,权以登年幼,上书辞封,重遣西曹掾沈珩陈谢,并献方物。立登为王太子。
\end{yuanwen}

\begin{yuanwen}

\end{yuanwen}\begin{yuanwen}

\end{yuanwen}\begin{yuanwen}

\end{yuanwen}\begin{yuanwen}

\end{yuanwen}\begin{yuanwen}

\end{yuanwen}\begin{yuanwen}

\end{yuanwen}\begin{yuanwen}

\end{yuanwen}\begin{yuanwen}

\end{yuanwen}\begin{yuanwen}

\end{yuanwen}\begin{yuanwen}

\end{yuanwen}\begin{yuanwen}

\end{yuanwen}\begin{yuanwen}

\end{yuanwen}\begin{yuanwen}

\end{yuanwen}\begin{yuanwen}

\end{yuanwen}\begin{yuanwen}

\end{yuanwen}\begin{yuanwen}

\end{yuanwen}\begin{yuanwen}

\end{yuanwen}\begin{yuanwen}

\end{yuanwen}\begin{yuanwen}

\end{yuanwen}\begin{yuanwen}

\end{yuanwen}



黄武元年春正月,陆逊部将军宋谦等攻蜀五屯,皆破之,斩其将。三月,鄱阳言黄龙见。蜀军分据险地,前后五十馀营,逊随轻重以兵应拒,自正月至闰月,大破之,临陈所斩及投兵降首数万人。刘备奔走,仅以身免。

初,权外讬事魏,而诚心不款。魏欲遣侍中辛毗、尚书桓阶往与盟誓,并徵任子,权辞让不受。秋九月,魏乃命曹休、张辽、臧霸出洞口,曹仁出濡须,曹真、夏侯尚、张郃、徐晃围南郡。权遣吕范等督五军,以舟军拒休等,诸葛瑾、潘璋、杨粲救南郡,朱桓以濡须督拒仁。时扬、越蛮夷多未平集,内难未弭,故权卑辞上书,求自改厉,”若罪在难除,必不见置,当奉还土地民人,乞寄命交州,以终馀年。”文帝报曰:“君生於扰攘之际,本有从横之志,降身奉国,以享兹祚。自君策名已来,贡献盈路。讨备之功,国朝仰成。埋而掘之,古人之所耻。朕之与君,大义已定,岂乐劳师远临江汉?廊庙之议,王者所不得专;三公上君过失,皆有本末。朕以不明,虽有曾母投杼之疑,犹冀言者不信,以为国福。故先遣使者犒劳,又遣尚书、侍中践脩前言,以定任子。君遂设辞,不欲使进,议者怪之。又前都尉浩周劝君遣子,乃实朝臣交谋,以此卜君,君果有辞,外引隗嚣遣子不终,内喻窦融守忠而已。世殊时异,人各有心。浩周之还,口陈指麾,益令议者发明众嫌,终始之本,无所据仗,故遂俯仰从群臣议。今省上事,款诚深至,心用慨然,凄怆动容。即日下诏,敕诸军但深沟高垒,不得妄进。若君必效忠节,以解疑议,登身朝到,夕召兵还。此言之诚,有如大江!”权遂改年,临江拒守。冬十一月,大风,范等兵溺死者数千,馀军还江南。曹休使臧霸以轻船五百、敢死万人袭攻徐陵,烧攻城车,杀略数千人。将军全琮、徐盛追斩魏将尹卢,杀获数百。十二月,权使太中大夫郑泉聘刘备于白帝,始复通也。然犹与魏文帝相往来,至后年乃绝。是岁改夷陵为西陵。

二年春正月,曹真分军据江陵中州。是月,城江夏山。改四分,用乾象历。三月,曹仁遣将军常彫等,以兵五千,乘油船,晨渡濡须中州。仁子泰因引军急攻朱桓,桓兵拒之,遣将军严圭等击破彫等。是月,魏军皆退。夏四月,权群臣劝即尊号,权不许。刘备薨于白帝。五月,曲阿言甘露降。先是戏口守将晋宗杀将王直,以众叛如魏,魏以为蕲春太守,数犯边境。六月,权令将军贺齐督糜芳、刘邵等袭蕲春,邵等生虏宗。冬十一月,蜀使中郎将邓芝来聘。

三年夏,遣辅义中郎将张温聘于蜀。秋八月,赦死罪。九月,魏文帝出广陵,望大江,曰“彼有人焉,未可图也”,乃还。

四年夏五月,丞相孙邵卒。六月,以太常顾雍为丞相。皖口言木连理。冬十二月,鄱阳贼彭绮自称将军,攻没诸县,众数万人。是岁地连震。

五年春,令曰:“军兴日久,民离农畔,父子夫妇,不听相恤,孤甚愍之。今北虏缩窜,方外无事,其下州郡,有以宽息。”是时陆逊以所在少谷,表令诸将增广农亩。权报曰:“甚善。今孤父子亲自受田,车中八牛以为四耦,虽未及古人,亦欲与众均等其劳也。”秋七月,权闻魏文帝崩,征江夏,围石阳,不克而还。苍梧言凤皇见。分三郡恶地十县置东安郡,以全琮为太守,平讨山越。冬十月,陆逊陈便宜,劝以施德缓刑,宽赋息调。又云:“忠谠之言,不能极陈,求容小臣,数以利闻。”权报曰:“夫法令之设,欲以遏恶防邪,儆戒未然也,焉得不有刑罚以威小人乎?此为先令后诛,不欲使有犯者耳。君以为太重者,孤亦何利其然,但不得已而为之耳。今承来意,当重谘谋,务从其可。且近臣有尽规之谏,亲戚有补察之箴,所以匡君正主明忠信也。书载'予违汝弼,汝无面从’,孤岂不乐忠言以自裨补邪?而云“不敢极陈”,何得为忠谠哉?若小臣之中,有可纳用者,宁得以人废言而不采择乎?但谄媚取容,虽闇亦所明识也。至於发调者,徒以天下未定,事以众济。若徒守江东,脩崇宽政,兵自足用,复用多为?顾坐自守可陋耳。若不豫调,恐临时未可便用也。又孤与君分义特异,荣戚实同,来表云不敢随众容身苟免,此实甘心所望於君也。”於是令有司尽写科条,使郎中褚逢赍以就逊及诸葛瑾,意所不安,令损益之。是岁,分交州置广州,俄复旧。

六年春正月,诸将获彭绮。闰月,韩当子综以其众降魏。

七年春三月,封子虑为建昌侯。罢东安郡。夏五月,鄱阳太守周鲂伪叛,诱魏将曹休。秋八月,权至皖口,使将军陆逊督诸将大破休於石亭。大司马吕范卒。是岁,改合浦为珠官郡。

黄龙元年春,公卿百司皆劝权正尊号。夏四月,夏口、武昌并言黄龙、凤凰见。丙申,南郊即皇帝位,是日大赦,改年。追尊父破虏将军坚为武烈皇帝,母吴氏为武烈皇后,兄讨逆将军策为长沙桓王。吴王太子登为皇太子。将吏皆进爵加赏。初,兴平中,吴中童谣曰:“黄金车,班兰耳,闿昌门,出天子。”五月,使校尉张刚、管笃之辽东。六月,蜀遣卫尉陈震庆权践位。权乃参分天下,豫、青、徐、幽属吴,兖、冀、并、凉属蜀。其司州之土,以函谷关为界,造为盟曰:“天降丧乱,皇纲失叙,逆臣乘衅,劫夺国柄,始於董卓,终於曹操,穷凶极恶,以覆四海,至令九州幅裂,普天无统,民神痛怨,靡所戾止。及操子丕,桀逆遗丑,荐作奸回,偷取天位,而叡么麽,寻丕凶迹,阻兵盗土,未伏厥诛。昔共工乱象而高辛行师,三苗干度而虞舜征焉。今日灭叡,禽其徒党,非汉与吴,将复谁任?夫讨恶翦暴,必声其罪,宜先分制,夺其土地,使士民之心,各知所归。是以春秋晋侯伐卫,先分其田以畀宋人,斯其义也。且古建大事,必先盟誓,故周礼有司盟之官,尚书有告誓之文,汉之与吴,虽信由中,然分土裂境,宜有盟约。诸葛丞相德威远著,翼戴本国,典戎在外,信感阴阳,诚动天地,重复结盟,广诚约誓,使东西士民咸共闻知。故立坛杀牲,昭告神明,再歃加书,副之天府。天高听下,灵威棐谌,司慎司盟,群神群祀,莫不临之。自今日汉、吴既盟之后,戮力一心,同讨魏贼,救危恤患,分灾共庆,好恶齐之,无或携贰。若有害汉,则吴伐之;若有害吴,则汉伐之。各守分土,无相侵犯。传之后叶,克终若始。凡百之约。皆如载书。信言不艳,实居于好。有渝此盟,创祸先乱,违贰不协,慆慢天命,明神上帝是讨是督,山川百神是纠是殛,俾坠其师,无克祚国。于尔大神,其明鉴之!”秋九月,权迁都建业,因故府不改馆,徵上大将军陆逊辅太子登,掌武昌留事。

二年春正月,魏作合肥新城。诏立都讲祭酒,以教学诸子。遣将军卫温、诸葛直将甲士万人浮海求夷洲及亶洲。亶洲在海中,长老传言秦始皇帝遣方士徐福将童男童女数千人入海,求蓬莱神山及仙药,止此洲不还。世相承有数万家,其上人民,时有至会稽货布,会稽东县人海行,亦有遭风流移至亶洲者。所在绝远,卒不可得至,但得夷洲数千人还。

三年春二月,遣太常潘濬率众五万讨武陵蛮夷。卫温、诸葛直皆以违诏无功,下狱诛。夏,有野蚕成茧,大如卵。由拳野稻自生,改为禾兴县。中郎将孙布诈降以诱魏将王凌,凌以军迎布。冬十月,权以大兵潜伏於阜陵俟之,凌觉而走。会稽南始平言嘉禾生。十二月丁卯,大赦,改明年元也。

嘉禾元年春正月,建昌侯虑卒。三月,遣将军周贺、校尉裴潜乘海之辽东。秋九月,魏将田豫要击,斩贺于成山。冬十月,魏辽东太守公孙渊遣校尉宿舒、阆中令孙综称藩於权,并献貂马。权大悦,加渊爵位。

二年春正月,诏曰:“朕以不德,肇受元命,夙夜兢兢,不遑假寝。思平世难,救济黎庶,上答神祗,下慰民望。是以眷眷,勤求俊杰,将与戮力,共定海内,苟在同心,与之偕老。今使持节督幽州领青州牧辽东太守燕王,久胁贼虏,隔在一方,虽乃心於国,其路靡缘。今因天命,远遣二使,款诚显露,章表殷勤,朕之得此,何喜如之!虽汤遇伊尹,周获吕望,世祖未定而得河右,方之今日,岂复是过?普天一统,於是定矣。书不云乎,‘一人有庆,兆民赖之’。其大赦天下,与之更始,其明下州郡,咸使闻知。特下燕国,奉宣诏恩,令普天率土备闻斯庆。三月,遣舒、综还,使太常张弥、执金吾许晏、将军贺达等将兵万人,金宝珍货,九锡备物,乘海授渊。举朝大臣,自丞相雍已下皆谏,以为渊未可信,而宠待太厚,但可遣吏兵数百护送舒、综,权终不听。渊果斩弥等,送其首于魏,没其兵资。权大怒,欲自征渊,尚书仆射薛综等切谏乃止。是岁,权向合肥新城,遣将军全琮征六安,皆不克还。

三年春正月,诏曰:“兵久不辍,民困於役,岁或不登。其宽诸逋,勿复督课。”夏五月,权遣陆逊、诸葛瑾等屯江夏、沔口,孙韶、张承等向广陵、淮阳,权率大众围合肥新城。是时蜀相诸葛亮出武功,权谓魏明帝不能远出,而帝遣兵助司马宣王拒亮,自率水军东征。未至寿春,权退还,孙韶亦罢。秋八月,以诸葛恪为丹杨太守,讨山越。九月朔,陨霜伤谷。冬十一月,太常潘濬平武陵蛮夷,事毕,还武昌。诏复曲阿为云阳,丹徒为武进。庐陵贼李桓、罗厉等为乱。

四年夏,遣吕岱讨桓等。秋七月,有雹。魏使以马求易珠玑、翡翠、玳瑁,权曰:“此皆孤所不用,而可得马,何苦而不听其交易?”

五年春,铸大钱,一当五百。诏使吏民输铜,计铜畀直。设盗铸之科。二月,武昌言甘露降於礼宾殿。辅吴将军张昭卒。中郎将吾粲获李桓,将军唐咨获罗厉等。自十月不雨,至於夏。冬十月,彗星见于东方。鄱阳贼彭旦等为乱。

六年春正月,诏曰:“夫三年之丧,天下之达制,人情之极痛也;贤者割哀以从礼,不肖者勉而致之。世治道泰,上下无事,君子不夺人情,故三年不逮孝子之门。至於有事,则杀礼以从宜,要绖而处事。故圣人制法,有礼无时则不行。遭丧不奔非古也,盖随时之宜,以义断恩也。前故设科,长吏在官,当须交代,而故犯之,虽随纠坐,犹已废旷。方事之殷,国家多难,凡在官司,宜各尽节,先公后私,而不恭承,甚非谓也。中外群僚,其更平议,务令得中,详为节度。“顾谭议,以为奔丧立科,轻则不足以禁孝子之情,重则本非应死之罪,虽严刑益设,违夺必少。若偶有犯者,加其刑则恩所不忍,有减则法废不行。愚以为长吏在远,苟不告语,势不得知。比选代之间,若有传者,必加大辟,则长吏无废职之负,孝子无犯重之刑。”将军胡综议,以为“丧纪之礼,虽有典制,苟无其时,所不得行。方今戎事军国异容,而长吏遭丧,知有科禁,公敢干突,苟念闻忧不奔之耻,不计为臣犯禁之罪,此由科防本轻所致。忠节在国,孝道立家,出身为臣,焉得兼之?故为忠臣不得为孝子。宜定科文,示以大辟,若故违犯,有罪无赦。以杀止杀,行之一人,其后必绝。”丞相雍奏从大辟。其后吴令孟宗丧母奔赴,已而自拘於武昌以听刑。陆逊陈其素行,因为之请,权乃减宗一等,后不得以为比,因此遂绝。二月,陆逊讨彭旦等,其年,皆破之。冬十月,遣卫将军全琮袭六安,不克。诸葛恪平山越事毕,北屯庐江。

赤乌元年春,铸当千大钱。夏,吕岱讨庐陵贼,毕,还陆口。秋八月,武昌言麒麟见。有司奏言麒麟者太平之应,宜改年号。诏曰:“间者赤乌集於殿前,朕所亲见,若神灵以为嘉祥者,改年宜以赤乌为元。”群臣奏曰:“昔武王伐纣,有赤乌之祥,君臣观之,遂有天下,圣人书策载述最详者,以为近事既嘉,亲见又明也。”於是改年。步夫人卒,追赠皇后。初,权信任校事吕壹,壹性苛惨,用法深刻。太子登数谏,权不纳,大臣由是莫敢言。后壹奸罪发露伏诛,权引咎责躬,乃使中书郎袁礼告谢诸大将,因问时事所当损益。礼还,复有诏责数诸葛瑾、步骘、朱然、吕岱等曰:“袁礼还,云与子瑜、子山、义封、定公相见,并以时事当有所先后,各自以不掌民事,不肯便有所陈,悉推之伯言、承明。伯言、承明见礼,泣涕恳恻,辞旨辛苦,至乃怀执危怖,有不自安之心。闻此怅然,深自刻怪。何者?夫惟圣人能无过行,明者能自见耳。人之举措,何能悉中,独当己有以伤拒众意,忽不自觉,故诸君有嫌难耳;不尔,何缘乃至於此乎?自孤兴军五十年,所役赋凡百皆出於民。天下未定,孽类犹存,士民勤苦,诚所贯知。然劳百姓,事不得已耳。与诸君从事,自少至长,发有二色,以谓表里足以明露,公私分计,足用相保。尽言直谏,所望诸君;拾遗补阙,孤亦望之。昔卫武公年过志壮,勤求辅弼,每独叹责。且布衣韦带,相与交结,分成好合,尚污垢不异。今日诸君与孤从事,虽君臣义存,犹谓骨肉不复是过。荣福喜戚,相与共之。忠不匿情,智无遗计,事统是非,诸君岂得从容而已哉!同船济水,将谁与易?齐桓诸侯之霸者耳,有善管子未尝不叹,有过未尝不谏,谏而不得,终谏不止。今孤自省无桓公之德,而诸君谏诤未出於口,仍执嫌难。以此言之,孤於齐桓良优,未知诸君於管子何如耳?久不相见,因事当笑。共定大业,整齐天下,当复有谁?凡百事要所当损益,乐闻异计,匡所不逮。”

二年春三月,遣使者羊衟、郑胄、将军孙怡之辽东,击魏守将张持、高虑等,虏得男女。零陵言甘露降。夏五月,城沙羡。冬十月,将军蒋秘南讨夷贼。秘所领都督廖式杀临贺太守严纲等,自称平南将军,与弟潜共攻零陵、桂阳,及摇动交州、苍梧、郁林诸郡,众数万人。遣将军吕岱、唐咨讨之,岁馀皆破。

三年春正月,诏曰:“盖君非民不立,民非谷不生。顷者以来,民多征役,岁又水旱,年谷有损,而吏或不良,侵夺民时,以致饥困。自今以来,督军郡守,其谨察非法,当农桑时,以役事扰民者,举正以闻。”夏四月,大赦,诏诸郡县治城郭,起谯楼,穿堑发渠,以备盗贼。冬十一月,民饥,诏开仓廪以赈贫穷。

四年春正月,大雪,平地深三尺,鸟兽死者大半。夏四月,遣卫将军全琮略淮南,决芍陂,烧安城邸阁,收其人民。威北将军诸葛恪攻六安。琮与魏将王凌战于芍陂,中郎将秦晃等十馀人战死。车骑将军朱然围樊,大将军诸葛瑾取柤中。五月,太子登卒。是月,魏太傅司马宣王救樊。六月,军还。闰月,大将军瑾卒。秋八月,陆逊城邾。

五年春正月,立子和为太子,大赦,改禾兴为嘉兴。百官奏立皇后及四王,诏曰:“今天下未定,民物劳瘁,且有功者或未录,饥寒者尚未恤,猥割土壤以丰子弟,崇爵位以宠妃妾,孤甚不取。其释此议。”三月,海盐县言黄龙见。夏四月,禁进献御,减太官膳。秋七月,遣将军聂友、校尉陆凯以兵三万讨珠崖、儋耳。是岁大疫,有司又奏立后及诸王。八月,立子霸为鲁王。

六年春正月,新都言白虎见。诸葛恪征六安,破魏将谢顺营,收其民人。冬十一月,丞相顾雍卒。十二月,扶南王范旃遣使献乐人及方物。是岁,司马宣王率军入舒,诸葛恪自皖迁于柴桑。

七年春正月,以上大将军陆逊为丞相。秋,宛陵言嘉禾生。是岁,步骘、朱然等各上疏云:“自蜀还者,咸言欲背盟与魏交通,多作舟船,缮治城郭。又蒋琬守汉中,闻司马懿南向,不出兵乘虚以掎角之,反委汉中,还近成都。事已彰灼,无所复疑,宜为之备。”权揆其不然,曰:“吾待蜀不薄,聘享盟誓,无所负之,何以致此?又司马懿前来入舒,旬日便退,蜀在万里,何知缓急而便出兵乎?昔魏欲入汉川,此间始严,亦未举动,会闻魏还而止,蜀宁可复以此有疑邪?又人家治国,舟船城郭,何得不护?今此间治军,宁复欲以御蜀邪?人言苦不可信,朕为诸君破家保之。”蜀竟自无谋,如权所筹。

八年春二月,丞相陆逊卒。夏,雷霆犯宫门柱,又击南津大桥楹。茶陵县鸿水溢出,流漂居民二百馀家。秋七月,将军马茂等图逆,夷三族。八月,大赦。遣校尉陈勋将屯田及作士三万人凿句容中道,自小其至云阳西城,通会市,作邸阁。

九年春二月,车骑将军朱然征魏柤中,斩获千馀。夏四月,武昌言甘露降。秋九月,以骠骑将军步骘为丞相,车骑将军朱然为左大司马,卫将军全琮为右大司马,镇南将军吕岱为上大将军,威北将军诸葛恪为大将军。

十年春正月,右大司马全琮卒。二月,权適南宫。三月,改作太初宫,诸将及州郡皆义作。夏五月,丞相步骘卒。冬十月,赦死罪。

十一年春正月,朱然城江陵。二月,地仍震。三月,宫成。夏四月,雨雹,云阳言黄龙见。五月,鄱阳言白虎仁。诏曰:“古者圣王积行累善,脩身行道,以有天下,故符瑞应之,所以表德也。朕以不明,何以臻兹?书云'虽休勿休',公卿百司,其勉脩所职,以匡不逮。”

十二年春三月,左大司马朱然卒。四月,有两乌衔鹊堕东馆。丙寅,骠骑将军朱据领丞相,燎鹊以祭。

十三年夏五月,日至,荧惑入南斗,秋七月,犯魁第二星而东。八月,丹杨、句容及故鄣、宁国诸山崩,鸿水溢。诏原逋责,给贷种食。废太子和,处故鄣。鲁王霸赐死。冬十月,魏将文钦伪叛以诱朱异,权遣吕据就异以迎钦。异等持重,钦不敢进。十一月,立子亮为太子。遣军十万,作堂邑涂塘以淹北道。十二月,魏大将军王昶围南郡,荆州刺史王基攻西陵,遣将军戴烈、陆凯往拒之,皆引还。是岁,神人授书,告以改年、立后。

太元元年夏五月,立皇后潘氏,大赦,改年。初临海罗阳县有神,自称王表。周旋民间,语言饮食,与人无异,然不见其形。又有一婢,名纺绩。是月,遣中书郎李崇赍辅国将军罗阳王印绶迎表。表随崇俱出,与崇及所在郡守令长谈论,崇等无以易。所历山川,辄遣婢与其神相闻。秋七月,崇与表至,权於苍龙门外为立第舍,数使近臣赍酒食往。表说水旱小事,往往有验。秋八月朔,大风,江海涌溢,平地深八尺,吴高陵松柏斯拔,郡城南门飞落。冬十一月,大赦。权祭南郊还,寝疾。十二月,驿徵大将军恪,拜为太子太傅。诏省徭役,减征赋,除民所患苦。

二年春正月,立故太子和为南阳王,居长沙;子奋为齐王,居武昌;子休为琅邪王,居虎林。二月,大赦,改元为神凤。皇后潘氏薨。诸将吏数诣王表请福,表亡去。夏四月,权薨,时年七十一,谥曰大皇帝。秋七月,葬蒋陵。

评曰:孙权屈身忍辱,任才尚计,有勾践之奇,英人之杰矣。故能自擅江表,成鼎峙之业。然性多嫌忌,果於杀戮,暨臻末年,弥以滋甚。至于谗说殄行,胤嗣废毙,岂所谓贻厥孙谋以燕翼子者哉?其后叶陵迟,遂致覆国,未必不由此也。

\part{吴书三}
\chapter{三嗣主传第三}

孙亮字子明,权少子也。权春秋高,而亮最少,故尤留意。姊全公主尝谮太子和子母,心不自安,因倚权意,欲豫自结,数称述全尚女,劝为亮纳。

赤乌十三年,和废,权遂立亮为太子,以全氏为妃。

太元元年夏,亮母潘氏立为皇后。冬,权寝疾,徵大将军诸葛恪为太子太傅,会稽太守滕胤为太常,并受诏辅太子。明年四月,权薨,太子即尊号,大赦,改元。是岁,於魏嘉平四年也。

建兴元年闰月,以恪为帝太傅,胤为卫将军领尚书事,上大将军吕岱为大司马,诸文武在位皆进爵班赏,官{宀儿}加等。冬十月,太傅恪率军遏巢湖,城东兴,使将军全端守西城,都尉留略守东城。十二月朔丙申,大风雷电,魏使将军诸葛诞、胡遵等步骑七万围东兴,将军王昶攻南郡,毌丘俭向武昌。甲寅,恪以大兵赴敌。戊午,兵及东兴,交战,大破魏军,杀将军韩综、桓嘉等。是月,雷雨,天灾武昌端门;改作端门,又灾内殿。

二年春正月丙寅,立皇后全氏,大赦。庚午,王昶等皆退。二月,军还自东兴,大行封赏。三月,恪率军伐魏。夏四月,围新城,大疫,兵卒死者大半。秋八月,恪引军还。冬十月,大飨。武卫将军孙峻伏兵杀恪於殿堂。大赦。以峻为丞相,封富春侯。十一月,有大鸟五见于春申,改明年元。

五凤元年夏,大水。秋,吴侯英谋杀峻,觉,英自杀。冬十一月,星茀于斗、牛。

二年春正月,魏镇东大将军毌丘俭、前将军文钦以淮南之众西入,战于乐嘉。闰月壬辰,峻及骠骑将军吕据、左将军留赞率兵袭寿春,军及东兴,闻钦等败。壬寅,兵进于橐皋,钦诣峻降,淮南馀众数万口来奔。魏诸葛诞入寿春,峻引军还。二月,及魏将军曹珍遇于高亭,交战,珍败绩。留赞为诞别将蒋班所败于菰陂,赞及将军孙楞、蒋脩等皆遇害。三月,使镇南将军朱异袭安丰,不克。秋七月,将军孙仪、张怡、林恂等谋杀峻,发觉,仪自杀,恂等伏辜。阳羡离里山大石自立。使卫尉冯朝城广陵,拜将军吴穰为广陵太守,留略为东海太守。是岁大旱。十二月,作太庙。以冯朝为监军使者,督徐州诸军事,民饥,军士怨畔。

太平元年春二月朔,建业火。峻用征北大将军文钦计,将征魏。八月,先遣钦及骠骑将军吕据、车骑将军刘纂、镇南将军朱异、前将军唐咨军自江都入淮、泗。九月丁亥,峻卒,以从弟偏将军綝为侍中、武卫将军,领中外诸军事,召还据等。据闻綝代峻,大怒。己丑,大司马吕岱卒。壬辰,太白犯南斗。据、钦、咨等表荐卫将军滕胤为丞相,綝不听。癸卯,更以胤为大司马,代吕岱驻武昌。据引兵还,欲讨綝。綝遣使以诏书告喻钦、咨等,使取据。冬十月丁未,遣孙宪及丁奉、施宽等以舟兵逆据於江都,遣将军刘丞督步骑攻胤。胤兵败夷灭。己酉,大赦,改年。辛亥,获吕据於新州。十一月,以綝为大将军、假节,封永宁侯。孙宪与将军王惇谋杀綝,事觉,綝杀惇,迫宪令自杀。十二月,使五官中郎将刁玄告乱于蜀。

二年春二月甲寅,大雨,震电。乙卯,雪,大寒。以长沙东部为湘东郡,西部为衡阳郡,会稽东部为临海郡,豫章东部为临川郡。夏四月,亮临正殿,大赦,始亲政事。綝所表奏,多见难问,又科兵子弟年十八已下十五已上,得三千馀人,选大将子弟年少有勇力者为之将帅。亮曰:“吾立此军,欲与之俱长。”日於苑中习焉。

五月,魏征东大将军诸葛诞以淮南之众保寿春城,遣将军朱成称臣上疏,又遣子靓、长史吴纲诸牙门子弟为质。六月,使文钦、唐咨、全端等步骑三万救诞。朱异自虎林率众袭夏口,夏口督孙壹奔魏。秋七月,綝率众救寿春,次于镬里,朱异至自夏口,綝使异为前部督,与丁奉等将介士五万解围。八月,会稽南部反,杀都尉。鄱阳、新都民为乱,廷尉丁密、步兵校尉郑胄、将军锺离牧率军讨之。朱异以军士乏食引还,綝大怒,九月朔己巳,杀异於镬里。辛未,綝自镬里还建业。甲申,大赦。十一月,全绪子祎、仪以其母奔魏。十二月,全端、怿等自寿春城诣司马文王。

三年春正月,诸葛诞杀文钦。三月,司马文王克寿春,诞及左右战死,将吏已下皆降。秋七月,封故齐王奋为章安侯。诏州郡伐宫材。自八月沈阴不雨四十馀日。亮以綝专恣,与太常全尚,将军刘丞谋诛綝。九月戊午,綝以兵取尚,遣弟恩攻杀丞於苍龙门外,召大臣会宫门,黜亮为会稽王,时年十六。

孙休字子烈,权第六子。年十三,从中书郎射慈、郎中盛冲受学。太元二年正月,封琅邪王,居虎林。四月,权薨,休弟亮承统,诸葛恪秉政,不欲诸王在滨江兵马之地,徙休於丹杨郡。太守李衡数以事侵休,休上书乞徙他郡,诏徙会稽。居数岁,梦乘龙上天,顾不见尾,觉而异之。孙亮废,己未,孙綝使宗正孙楷与中书郎董朝迎休。休初闻问,意疑,楷、朝具述綝等所以奉迎本意,留一日二夜,遂发。十月戊寅,行至曲阿,有老公干休叩头曰:“事久变生,天下喁喁,愿陛下速行。”休善之,是日进及布塞亭。武卫将军恩行丞相事,率百僚以乘舆法驾迎於永昌亭,筑宫,以武帐为便殿,设御座。己卯,休至,望便殿止住,使孙楷先见恩。楷还,休乘辇进,群臣再拜称臣。休升便殿,谦不即御坐,止东厢。户曹尚书前即阶下赞奏,丞相奉玺符。休三让,群臣三请。休曰:“将相诸侯咸推寡人,寡人敢不承受玺符。”群臣以次奉引,休就乘舆,百官陪位,綝以兵千人迎於半野,拜于道侧,休下车答拜。即日,御正殿,大赦,改元。是岁,於魏甘露三年也。

永安元年冬十月壬午,诏曰:“夫褒德赏功,古今通义。其以大将军綝为丞相、荆州牧,增食五县。武卫将军恩为御史大夫、卫将军、中军督,封县侯。威远将军据为右将军、县侯。偏将军幹杂号将军、亭侯。长水校尉张布辅导勤劳,以布为辅义将军,封永康侯。董朝亲迎,封为乡侯。”又诏曰:“丹阳太守李衡,以往事之嫌,自拘有司。夫射钩斩袪,在君为君,遣衡还郡,勿令自疑。”己丑,封孙皓为乌程侯,皓弟德钱唐侯,谦永安侯。

十一月甲午,风四转五复,蒙雾连日。綝一门五侯皆典禁兵,权倾人主,有所陈述,敬而不违,於是益恣。休恐其有变,数加赏赐。丙申,诏曰:“大将军忠款内发,首建大计以安社稷,卿士内外,咸赞其议,并有勋劳。昔霍光定计,百僚同心,无复是过。亟案前日与议定策告庙人名,依故事应加爵位者,促施行之。“戊戌,诏曰:“大将军掌中外诸军事,事统烦多,其加卫将军御史大夫恩侍中,与大将军分省诸事。“壬子,诏曰:“诸吏家有五人三人兼重为役,父兄在都,子弟给郡县吏,既出限米,军出又从,至於家事无经护者,朕甚愍之。其有五人三人为役,听其父兄所欲留,为留一人,除其米限,军出不从。”又曰:“诸将吏奉迎陪位在永昌亭者,皆加位一级。”顷之,休闻綝逆谋,阴与张布图计。十二月戊辰腊,百僚朝贺,公卿升殿,诏武士缚綝,即日伏诛。己巳,诏以左将军张布讨奸臣,加布为中军督,封布弟惇为都亭侯,给兵三百人,惇弟恂为校尉。

诏曰:“古者建国,教学为先,所以道世治性,为时养器也。自建兴以来,时事多故,吏民颇以目前趋务,去本就末,不循古道。夫所尚不惇,则伤化败俗。其案古置学官,立五经博士,核取应选,加其宠禄,科见吏之中及将吏子弟有志好者,各令就业。一岁课试,差其品第,加以位赏。使见之者乐其荣,闻之者羡其誉。以敦王化,以隆风俗。”

二年春正月,震电。三月,备九卿官,诏曰:“朕以不德,讬于王公之上,夙夜战战,忘寝与食。今欲偃武修文,以崇大化。推此之道,当由士民之赡,必须农桑。管子有言:‘仓廪实,知礼节;衣食足,知荣辱。’夫一夫不耕,有受其饥,一妇不织,有受其寒;饥寒并至而民不为非者,未之有也。自顷年已来,州郡吏民及诸营兵,多违此业,皆浮船长江,贾作上下,良田渐废,见谷日少,欲求大定,岂可得哉?亦由租入过重,农人利薄,使之然乎!今欲广开田业,轻其赋税,差科强羸,课其田亩,务令优均,官私得所,使家给户赡,足相供养,则爱身重命,不犯科法,然后刑罚不用,风俗可整。以群僚之忠贤,若尽心於时,虽太古盛化,未可卒致,汉文升平,庶几可及。及之则臣主俱荣,不及则损削侵辱,何可从容俯仰而已?诸卿尚书,可共咨度,务取便佳。田桑已至,不可后时。事定施行,称朕意焉。”

三年春三月,西陵言赤乌见。秋,用都尉严密议,作浦里塘。会稽郡谣言王亮当还为天子,而亮宫人告亮使巫祷祠,有恶言。有司以闻,黜为候官侯,遣之国。道自杀,卫送者伏罪。以会稽南部为建安郡,分宜都置建平郡。

四年夏五月,大雨,水泉涌溢。秋八月,遣光禄大夫周奕、石伟巡行风俗,察将吏清浊,民所疾苦,为黜陟之诏。九月,布山言白龙见。是岁,安吴民陈焦死,埋之,六日更生,穿土中出。

五年春二月,白虎门北楼灾。秋七月,始新言黄龙见。八月壬午,大雨震电,水泉涌溢。乙酉,立皇后朱氏。戊子,立子{雨單}为太子,大赦。冬十月,以卫将军濮阳兴为丞相,廷尉丁密、光禄勋孟宗为左右御史大夫。休以丞相兴及左将军张布有旧恩,委之以事,布典宫省,兴关军国。休锐意於典籍,欲毕览百家之言,尤好射雉,春夏之间常晨出夜还,唯此时舍书。休欲与博士祭酒韦曜、博士盛冲讲论道艺,曜、冲素皆切直,布恐入侍,发其阴失,令己不得专,因妄饰说以拒遏之。休答曰:“孤之涉学,群书略遍,所见不少也;其明君闇王,奸臣贼子,古今贤愚成败之事,无不览也。今曜等入,但欲与论讲书耳,不为从曜等始更受学也。纵复如此,亦何所损?君特当以曜等恐道臣下奸变之事,以此不欲令入耳。如此之事,孤已自备之,不须曜等然后乃解也。此都无所损,君意特有所忌故耳。“布得诏陈谢,重自序述,又言惧妨政事。休答曰:“书籍之事,患人不好,好之无伤也。此无所为非,而君以为不宜,是以孤有所及耳。政务学业,其流各异,不相妨也。不图君今日在事,更行此於孤也,良所不取。”布拜表叩头,休答曰:“聊相开悟耳,何至叩头乎!如君之忠诚,远近所知。往者所以相感,今日之巍巍也。诗云:‘靡不有初,鲜克有终。’终之实难,君其终之。“初休为王时,布为左右将督,素见信爱,及至践阼,厚加宠待,专擅国势,多行无礼,自嫌瑕短,惧曜、冲言之,故尤患忌。休虽解此旨,心不能悦,更恐其疑惧,竟如布意,废其讲业,不复使冲等入。是岁使察战到交阯调孔爵、大猪。

六年夏四月,泉陵言黄龙见。五月,交阯郡吏吕兴等反,杀太守孙谞。谞先是科郡上手工千馀人送建业,而察战至,恐复见取,故兴等因此扇动兵民,招诱诸夷也。冬十月,蜀以魏见伐来告。癸未,建业石头小城火,烧西南百八十丈。甲申,使大将军丁奉督诸军向魏寿春,将军留平别诣施绩於南郡,议兵所向,将军丁封、孙异如沔中,皆救蜀。蜀主刘禅降魏问至,然后罢。吕兴既杀孙谞,使使如魏,请太守及兵。丞相兴建取屯田万人以为兵。分武陵为天门郡。

七年春正月,大赦。二月,镇军将军陆抗、抚军将军步协、征西将军留平、建平太守盛曼,率众围蜀巴东守将罗宪。夏四月,魏将新附督王稚浮海入句章,略长吏赀财及男女二百馀口。将军孙越徼得一船,获三十人。秋七月,海贼破海盐,杀司盐校尉骆秀。使中书郎刘川发兵庐陵。豫章民张节等为乱,众万馀人。魏使将军胡烈步骑二万侵西陵,以救罗宪,陆抗等引军退。复分交州置广州。壬午,大赦。癸未,休薨,时年三十,谥曰景皇帝。

孙皓字元宗,权孙,和子也,一名彭祖,字皓宗。孙休立,封皓为乌程侯,遣就国。西湖民景养相皓当大贵,皓阴喜而不敢泄。休薨,是时蜀初亡,而交阯携叛,国内震惧,贪得长君。左典军万彧昔为乌程令,与皓相善,称皓才识明断,是长沙桓王之畴也,又加之好学,奉遵法度,屡言之於丞相濮阳兴、左将军张布。兴、布说休妃太后朱,欲以皓为嗣。朱曰:“我寡妇人,安知社稷之虑,苟吴国无损,宗庙有赖可矣。”於是遂迎立皓,时年二十三。改元,大赦。是岁,於魏咸熙元年也。

元兴元年八月,以上大将军施绩、大将军丁奉为左右大司马,张布为骠骑将军,加侍中,诸增位班赏,一皆如旧。九月,贬太后为景皇后,追谥父和曰文皇帝,尊母何为太后。十月,封休太子{雨單}为豫章王,次子汝南王,次子梁王,次子陈王,立皇后滕氏。皓既得志,粗暴骄盈,多忌讳,好酒色,大小失望。兴、布窃悔之。或以谮皓,十一月,诛兴、布。十二月,孙休葬定陵。封后父滕牧为高密侯,舅何洪等三人皆列侯。是岁,魏置交阯太守之郡。晋文帝为魏相国,遣昔吴寿春城降将徐绍、孙彧衔命赍书,陈事势利害,以申喻皓。

甘露元年三月,皓遣使随绍、彧报书曰:“知以高世之才,处宰辅之任,渐导之功,勤亦至矣。孤以不德,阶承统绪,思与贤良共济世道,而以壅隔未有所缘,嘉意允著,深用依依。今遣光禄大夫纪陟、五官中郎将弘璆宣明至怀。”绍行到濡须,召还杀之,徙其家属建安,始有白绍称美中国者故也。夏四月,蒋陵言甘露降,於是改年大赦。秋七月,皓逼杀景后朱氏,亡不在正殿,於苑中小屋治丧,众知其非疾病,莫不痛切。又送休四子於吴小城,寻复追杀大者二人。九月,从西陵督步阐表,徙都武昌,御史大夫丁固、右将军诸葛靓镇建业。陟、璆至洛,遇晋文帝崩,十一月,乃遣还。皓至武昌,又大赦。以零陵南部为始安郡,桂阳南部为始兴郡。十二月,晋受禅。

宝鼎元年正月,遣大鸿胪张俨、五官中郎将丁忠吊祭晋文帝。及还,俨道病死。忠说皓曰:“北方守战之具不设,弋阳可袭而取。”皓访群臣,镇西大将军陆凯曰:“夫兵不得已而用之耳,且三国鼎立已来,更相侵伐,无岁宁居。今强敌新并巴蜀,有兼土之实,而遣使求亲,欲息兵役,不可谓其求援於我。今敌形势方强,而欲徼幸求胜,未见其利也。“车骑将军刘纂曰:“天生五才,谁能去兵?谲诈相雄,有自来矣。若其有阙,庸可弃乎?宜遣间谍,以观其势。”皓阴纳纂言,且以蜀新平,故不行,然遂自绝。八月,所在言得大鼎,於是改年,大赦。以陆凯为左丞相,常侍万彧为右丞相。冬十月,永安山贼施但等聚众数千人,劫皓庶弟永安侯谦出乌程,取孙和陵上鼓吹曲盖。比至建业,众万馀人。丁固、诸葛靓逆之於牛屯,大战,但等败走。获谦,谦自杀。分会稽为东阳郡,分吴、丹杨为吴兴郡。以零陵北部为邵陵郡。十二月,皓还都建业,卫将军滕牧留镇武昌。

二年春,大赦。右丞相万彧上镇巴丘。夏六月,起显明宫,冬十二月,皓移居之。是岁,分豫章、庐陵、长沙为安成郡。

三年春二月,以左右御史大夫丁固、孟仁为司徒、司空。秋九月,皓出东关,丁奉至合肥。是岁,遣交州刺史刘俊、前部督脩则等入击交阯,为晋将毛炅等所破,皆死,兵散还合浦。

建衡元年春正月,立子瑾为太子,及淮阳、东平王。冬十月,改年,大赦。十一月,左丞相陆凯卒。遣监军虞汜、威南将军薛珝、苍梧太守陶璜由荆州,监军李勖、督军徐存从建安海道,皆就合浦击交阯。

二年春。万彧还建业。李勖以建安道不通利,杀导将冯斐,引军还。三月,天火烧万馀家,死者七百人。夏四月,左大司马施绩卒。殿中列将何定曰:“少府李勖枉杀冯斐,擅彻军退还。”勖及徐存家属皆伏诛。秋九月,何定将兵五千人上夏口猎。都督孙秀奔晋。是岁大赦。

三年春正月晦,皓举大众出华里,皓母及妃妾皆行,东观令华覈等固争,乃还。是岁,汜、璜破交阯,禽杀晋所置守将,九真、日南皆还属。大赦,分交阯为新昌郡。诸将破扶严,置武平郡。以武昌督范慎为太尉。右大司马丁奉、司空孟仁卒。西苑言凤凰集,改明年元。

凤皇元年秋八月,徵西陵督步阐。阐不应,据城降晋。遣乐乡都督陆抗围取阐,阐众悉降。阐及同计数十人皆夷三族。大赦。是岁右丞相万彧被谴忧死,徙其子弟於庐陵。何定奸秽发闻,伏诛。皓以其恶似张布,追改定名为布。

二年春三月,以陆抗为大司马。司徒丁固卒。秋九月,改封淮阳为鲁,东平为齐,又封陈留、章陵等九王,凡十一王,王给三千兵。大赦。皓爱妾或使人至市劫夺百姓财物,司市中郎将陈声,素皓幸臣也,恃皓宠遇,绳之以法。妾以愬皓,皓大怒,假他事烧锯断声头,投其身於四望之下。是岁,太尉范慎卒。

三年,会稽妖言章安侯奋当为天子。临海太守奚熙与会稽太守郭诞书,非论国政。诞但白熙书,不白妖言,送付建安作船。遣三郡督何植收熙,熙发兵自卫,断绝海道。熙部曲杀熙,送首建业,夷三族。秋七月,遣使者二十五人分至州郡,科出亡叛。大司马陆抗卒。自改年及是岁,连大疫。分郁林为桂林郡。

天册元年,吴郡言掘地得银,长一尺,广三分,刻上有年月字,於是大赦,改年。

天玺元年,吴郡言临平湖自汉末草秽壅塞,今更开通。长老相传,此湖塞,天下乱,此湖开,天下平。又於湖边得石函,中有小石,青白色,长四寸,广二寸馀,刻上作皇帝字,於是改年,大赦。会稽太守车浚、湘东太守张咏不出算缗,就在所斩之,徇首诸郡。秋八月,京下督孙楷降晋。鄱阳言历阳山石文理成字,凡二十,云“楚九州渚,吴九州都,扬州士,作天子,四世治,太平始”。又吴兴阳羡山有空石,长十馀丈,名曰石室,在所表为大瑞。乃遣兼司徒董朝、兼太常周处至阳羡县,封襌国山。改明年元,大赦,以协石文。

天纪元年夏,夏口督孙慎出江夏、汝南,烧略居民。初,驺子张俶多所谮白,累迁为司直中郎将,封侯,甚见宠爱,是岁奸情发闻,伏诛。

二年秋七月,立成纪、宣威等十一王,王给三千兵,大赦。

三年夏,郭马反。马本合浦太守脩允部曲督。允转桂林太守,疾病,住广州,先遣马将五百兵至郡安抚诸夷。允死,兵当分给,马等累世旧军,不乐离别。皓时又科实广州户口,马与部曲将何典、王族、吴述、殷兴等因此恐动兵民,合聚人众,攻杀广州督虞授。马自号都督交、广二州诸军事、安南将军,兴广州刺史,述南海太守。典攻苍梧,族攻始兴。八月,以军师张悌为丞相,牛渚都督何植为司徒。执金吾滕循为司空,未拜,转镇南将军,假节领广州牧,率万人从东道讨马,与族遇于始兴,未得前。马杀南海太守刘略,逐广州刺史徐旗。皓又遣徐陵督陶濬将七千人从西道,命交州牧陶璜部伍所领及合浦、郁林诸郡兵,当与东西军共击马。

有鬼目菜生工人黄耇家,依缘枣树,长丈馀,茎广四寸,厚三分。又有买菜生工人吴平家,高四尺,厚三分,如枇杷形,上广尺八寸,下茎广五寸,两边生叶绿色。东观案图,名鬼目作芝草,买菜作平虑草,遂以耇为侍芝郎,平为平虑郎,皆银印青绶。

冬,晋命镇东大将军司马伷向涂中,安东将军王浑、扬州刺史周浚向牛渚,建威将军王戎向武昌,平南将军胡奋向夏口,镇南将军杜预向江陵,龙骧将军王濬、广武将军唐彬浮江东下,太尉贾充为大都督,量宜处要,尽军势之中。陶濬至武昌,闻北军大出,停驻不前。

初,皓每宴会群臣,无不咸令沈醉。置黄门郎十人,特不与酒,侍立终日,为司过之吏。宴罢之后,各奏其阙失,迕视之咎,谬言之愆,罔有不举。大者即加威刑,小者辄以为罪。后宫数千,而采择无已。又激水入宫,宫人有不合意者,辄杀流之。或剥人之面,或凿人之眼。岑昬险谀贵幸,致位九列,好兴功役,众所患苦。是以上下离心,莫为皓尽力,盖积恶已极,不复堪命故也。

四年春,立中山、代等十一王,大赦。濬、彬所至,则土崩瓦解,靡有御者。预又斩江陵督伍延,浑复斩丞相张悌、丹杨太守沈莹等,所在战克。

三月丙寅,殿中亲近数百人叩头请皓杀岑昬,皓惶愦从之。

戊辰,陶濬从武昌还,即引见,问水军消息,对曰:“蜀船皆小,今得二万兵,乘大船战,自足击之。”於是合众,授濬节钺。明日当发,其夜众悉逃走。而王濬顺流将至,司马伷、王浑皆临近境。皓用光禄勋薛莹、中书令胡冲等计,分遣使奉书於濬、伷、浑曰:“昔汉室失统,九州分裂,先人因时,略有江南,遂分阻山川,与魏乖隔。今大晋龙兴,德覆四海。闇劣偷安,未喻天命。至于今者,猥烦六军,衡盖路次,远临江渚,举国震惶,假息漏刻。敢缘天朝含弘光大,谨遣私署太常张夔等奉所佩印绶,委质请命,惟垂信纳,以济元元。”

壬申,王濬最先到,於是受皓之降,解缚焚榇,延请相见。伷以皓致印绶於己,遣使送皓。皓举家西迁,以太康元年五月丁亥集于京邑。四月甲申,诏曰:“孙皓穷迫归降,前诏待之以不死,今皓垂至,意犹愍之,其赐号为归命侯。进给衣服车乘,田三十顷,岁给谷五千斛,钱五十万,绢五百匹,绵五百斤。”皓太子瑾拜中郎,诸子为王者,拜郎中。五年,皓死于洛阳。

评曰:孙亮童孺而无贤辅,其替位不终,必然之势也。休以旧爱宿恩,任用兴、布,不能拔进良才,改弦易张,虽志善好学,何益救乱乎?又使既废之亮不得其死,友于之义薄矣。皓之淫刑所滥,陨毙流黜者,盖不可胜数。是以群下人人惴恐,皆日日以冀,朝不谋夕。其荧惑、巫祝,交致祥瑞,以为至急。昔舜、禹躬稼,至圣之德,犹或矢誓众臣,予违女弼,或拜昌言,常若不及。况皓凶顽,肆行残暴,忠谏者诛,谗谀者进,虐用其民,穷淫极侈,宜腰首分离,以谢百姓。既蒙不死之诏,复加归命之宠,岂非旷荡之恩,过厚之泽也哉!

\part{吴书四}
\chapter{刘繇太史慈士燮传第四}

刘繇字正礼,东莱牟平人也。齐孝王少子封牟平侯,子孙家焉。繇伯父宠,为汉太尉。繇兄岱,字公山,历位侍中,兖州刺史。

繇年十九,从父韪为贼所劫质,繇篡取以归,由是显名。举孝廉,为郎中,除下邑长。时郡守以贵戚讬之,遂弃官去。州辟部济南,济南相中常侍子,贪秽不循,繇奏免之。平原陶丘洪荐繇,欲令举茂才。刺史曰:“前年举公山,奈何复举正礼乎?”洪曰:“若明使君用公山於前,擢正礼於后,所谓御二龙於长涂,骋骐骥於千里,不亦可乎!”会辟司空掾,除侍御史,不就。避乱淮浦,诏书以为扬州刺史。时袁术在淮南,繇畏惮,不敢之州。欲南渡江,吴景、孙贲迎置曲阿。术图为僣逆,攻没诸郡县。繇遣樊能、张英屯江边以拒之。以景、贲术所授用,乃迫逐使去。於是术乃自置扬州刺史,与景、贲并力攻英、能等,岁馀不下。汉命加繇为牧,振武将军,众数万人,孙策东渡,破英、能等。繇奔丹徒,遂溯江南保豫章,驻彭泽。笮融先至,杀太守朱皓,入居郡中。繇进讨融,为融所破,更复招合属县,攻破融。融败走入山,为民所杀,繇寻病卒,时年四十二。

笮融者,丹杨人,初聚众数百,往依徐州牧陶谦。谦使督广陵、彭城运漕,遂放纵擅杀,坐断三郡委输以自入。乃大起浮图祠,以铜为人,黄金涂身,衣以锦采,垂铜槃九重,下为重楼阁道,可容三千馀人,悉课读佛经,令界内及旁郡人有好佛者听受道,复其他役以招致之,由此远近前后至者五千馀人户。每浴佛,多设酒饭,布席於路,经数十里,民人来观及就食且万人,费以巨亿计。曹公攻陶谦,徐土骚动,融将男女万口,马三千匹,走广陵,广陵太守赵昱待以宾礼。先是,彭城相薛礼为陶谦所偪,屯秣陵。融利广陵之众,因酒酣杀昱,放兵大略,因载而去。过杀礼,然后杀皓。

后策西伐江夏,还过豫章,收载繇丧,善遇其家。王朗遗策书曰:“刘正礼昔初临州,未能自达,实赖尊门为之先后,用能济江成治,有所处定。践境之礼,感分结意,情在终始。后以袁氏之嫌,稍更乖剌。更以同盟,还为雠敌,原其本心,实非所乐。康宁之后,常愿渝平更成,复践宿好。一尔分离,款意不昭,奄然殂陨,可为伤恨!知敦以厉薄,德以报怨,收骨育孤,哀亡愍存,捐既往之猜,保六尺之讬,诚深恩重分,美名厚实也。昔鲁人虽有齐怨,不废丧纪,春秋善之,谓之得礼,诚良史之所宜藉,乡校之所叹闻。正礼元子,致有志操,想必有以殊异。威盛刑行,施之以恩,不亦优哉!”

繇长子基,字敬舆,年十四,居繇丧尽礼,故吏馈饷,皆无所受。姿容美好,孙权爱敬之。权为骠骑将军,辟东曹掾,拜辅义校尉、建忠中郎将。权为吴王,迁基大农。权尝宴饮,骑都尉虞翻醉酒犯忤,权欲杀之,威怒甚盛,由基谏争,翻以得免。权大暑时,尝於船中宴饮,於船楼上值雷雨,权以盖自覆,又命覆基,馀人不得也。其见待如此。徙郎中令。权称尊号,改为光禄勋,分平尚书事。年四十九卒。后权为子霸纳基女,赐第一区,四时宠赐,与全、张比。基二弟,铄、尚,皆骑都尉。

太史慈字子义,东莱黄人也。少好学,仕郡奏曹史。会郡与州有隙,曲直未分,以先闻者为善。时州章已去,郡守恐后之,求可使者。慈年二十一,以选行,晨夜取道,到洛阳,诣公车门,见州吏始欲求通。慈问曰:“君欲通章邪?“吏曰:“然。”问:“章安在?”曰:“车上。”慈曰:“章题署得无误邪?取来视之。”吏殊不知其东莱人也,因为取章。慈已先怀刀,便截败之。吏踊跃大呼,言“人坏我章”!慈将至车间,与语曰:“向使君不以章相与,吾亦无因得败之,是为吉凶祸福等耳,吾不独受此罪。岂若默然俱出去,可以存易亡,无事俱就刑辟。”吏言:“君为郡败吾章,已得如意,欲复亡为?”慈答曰:“初受郡遣,但来视章通与未耳。吾用意太过,乃相败章。今还,亦恐以此见谴怒,故俱欲去尔。”吏然慈言,即日俱去。慈既与出城,因遁还通郡章。州家闻之,更遣吏通章,有司以格章之故不复见理,州受其短。由是知名,而为州家所疾,恐受其祸,乃避之辽东。

北海相孔融闻而奇之,数遣人讯问其母,并致饷遗。时融以黄巾寇暴,出屯都昌,为贼管亥所围。慈从辽东还,母谓慈曰:“汝与孔北海未尝相见,至汝行后,赡恤殷勤,过於故旧,今为贼所围,汝宜赴之。”慈留三日,单步径至都昌。时围尚未密,夜伺间隙,得入见融,因求兵出斫贼。融不听,欲待外救。未有至者,而围日偪。融欲告急平原相刘备,城中人无由得出,慈自请求行。融曰:“今贼围甚密,众人皆言不可,卿意虽壮,无乃实难乎?”慈对曰:“昔府君倾意於老母,老母感遇,遣慈赴府君之急,固以慈有可取,而来必有益也。今众人言不可,慈亦言不可,岂府君爱顾之义,老母遣慈之意邪?事已急矣,愿府君无疑。”融乃然之。於是严行蓐食,须明,便带鞬摄弓上马,将两骑自随,各作一的持之,开门直出。外围下左右人并惊骇,兵马互出。慈引马至城下堑内,植所持的各一,出射之,射之毕,径入门。明晨复如此,围下人或起或卧,慈复植的,射之毕,复入门。明晨复出如此,无复起者,於是下鞭马直突围中驰去。比贼觉知,慈行已过,又射杀数人,皆应弦而倒,故无敢追者。遂到平原,说备曰:“慈,东莱之鄙人也,与孔北海亲非骨肉,比非乡党,特以名志相好,有分灾共患之义。今管亥暴乱,北海被围,孤穷无援,危在旦夕。以君有仁义之名,能救人之急,故北海区区,延颈恃仰,使慈冒白刃,突重围,从万死之中自讬於君,惟君所以存之。”备敛容答曰:“孔北海知世间有刘备邪!”即遣精兵三千人随慈。贼闻兵至,解围散走。融既得济,益奇贵慈,曰:“卿吾之少友也。”事毕,还启其母,母曰:“我喜汝有以报孔北海也。”

扬州刺史刘繇与慈同郡,慈自辽东还,未与相见,暂渡江到曲阿见繇,未去,会孙策至。或劝繇可以慈为大将军,繇曰:“我若用子义,许子将不当笑我邪?”但使慈侦视轻重。时独与一骑卒遇策。策从骑十三,皆韩当、宋谦、黄盖辈也。慈便前斗,正与策对。策刺慈马,而揽得慈项上手戟,慈亦得策兜鍪。会两家兵骑并各来赴,於是解散。

慈当与繇俱奔豫章,而遁於芜湖,亡入山中,称丹杨太守。是时,策已平定宣城以东,惟泾以西六县未服。慈因进住泾县,立屯府,大为山越所附。策躬自攻讨,遂见囚执。策即解缚,捉其手曰:“宁识神亭时邪?若卿尔时得我云何?”慈曰:“未可量也。”策大笑曰:“今日之事,当与卿共之。”即署门下督,还吴授兵,拜折冲中郎将。后刘繇亡於豫章,士众万馀人未有所附,策命慈往抚安焉。左右皆曰:“慈必北去不还。“策曰:“子义舍我,当复与谁?”饯送昌门,把腕别曰:“何时能还?”答曰:“不过六十日。”果如期而反。

刘表从子磐,骁勇,数为寇於艾、西安诸县。策於是分海昬、建昌左右六县,以慈为建昌都尉,治海昬,并督诸将拒磐。磐绝迹不复为寇。

慈长七尺七寸,美须髯,猿臂善射,弦不虚发。尝从策讨麻保贼,贼於屯里缘楼上行詈,以手持楼棼,慈引弓射之,矢贯手著棼,围外万人莫不称善。其妙如此。曹公闻其名,遗慈书,以箧封之,发省无所道,而但贮当归。孙权统事,以慈能制磐,遂委南方之事。年四十一,建安十一年卒。子享,官至越骑校尉。

士燮字威彦,苍梧广信人也。其先本鲁国汶阳人,至王莽之乱,避地交州。六世至燮父赐,桓帝时为日南太守。燮少游学京师,事颍川刘子奇,治左氏春秋。察孝廉,补尚书郎,公事免官。父赐丧阕后,举茂才,除巫令,迁交阯太守。

弟壹,初为郡督邮。刺史丁宫徵还京都,壹侍送勤恪,宫感之,临别谓曰:“刺史若待罪三事,当相辟也。”后宫为司徒,辟壹。比至,宫已免,黄琬代为司徒,甚礼遇壹。董卓作乱,壹亡归乡里。交州刺史朱符为夷贼所杀,州郡扰乱。燮乃表壹领合浦太守,次弟徐闻令〈黄有〉领九真太守,〈黄有〉弟武,领南海太守。

燮体器宽厚,谦虚下士,中国士人往依避难者以百数。耽玩春秋,为之注解。陈国袁徽与尚书令荀彧书曰:“交阯士府君既学问优博,又达於从政,处大乱之中,保全一郡,二十馀年疆埸无事,民不失业,羁旅之徒,皆蒙其庆,虽窦融保河西,曷以加之?官事小阕,辄玩习书传,春秋左氏传尤简练精微,吾数以咨问传中诸疑,皆有师说,意思甚密。又尚书兼通古今,大义详备。闻京师古今之学,是非忿争,今欲条左氏、尚书长义上之。”其见称如此。

燮兄弟并为列郡,雄长一州,偏在万里,威尊无上。出入鸣锺磬,备具威仪,笳箫鼓吹,车骑满道,胡人夹毂焚烧香者常有数十。妻妾乘辎軿,子弟从兵骑,当时贵重,震服百蛮,尉他不足逾也。武先病没。

朱符死后,汉遣张津为交州刺史,津后又为其将区景所杀,而荆州牧刘表遣零陵赖恭代津。是时苍梧太守史璜死,表又遣吴巨代之,与恭俱至。汉闻张津死,赐燮玺书曰:“交州绝域,南带江海,上恩不宣,下义壅隔,知逆贼刘表又遣赖恭闚看南土,今以燮为绥南中郎将,董督七郡,领交阯太守如故。”后燮遣吏张旻奉贡诣京都,是时天下丧乱,道路断绝,而燮不废贡职,特复下诏拜安远将军,封龙度亭侯。

后巨与恭相失,举兵逐恭,恭走还零陵。建安十五年,孙权遣步骘为交州刺史。骘到,燮率兄弟奉承节度。而吴巨怀异心,骘斩之。权加燮为左将军。建安末年,燮遣子廞入质,权以为武昌太守,燮、壹诸子在南者,皆拜中郎将。燮又诱导益州豪姓雍闿等,率郡人民使遥东附,权益嘉之,迁卫将军,封龙编侯,弟壹偏将军,都乡侯。燮每遣使诣权,致杂香细葛,辄以千数,明珠、大贝、流离、翡翠、玳瑁、犀、象之珍,奇物异果,蕉、邪、龙眼之属,无岁不至。壹时贡马凡数百匹。权辄为书,厚加宠赐,以答慰之。燮在郡四十馀岁,黄武五年,年九十卒。

权以交阯县远,乃分合浦以北为广州,吕岱为刺史;交阯以南为交州,戴良为刺史。又遣陈时代燮为交阯太守。岱留南海,良与时俱前行到合浦,而燮子徽自署交阯太守,发宗兵拒良。良留合浦。交阯桓邻,燮举吏也,叩头谏徽使迎良,徽怒,笞杀邻。邻兄治子发又合宗兵击徽,徽闭门城守,治等攻之数月不能下,乃约和亲,各罢兵还。而吕岱被诏诛徽,自广州将兵昼夜驰入,过合浦,与良俱前。壹子中郎将匡与岱有旧,岱署匡师友从事,先移书交阯,告喻祸福,又遣匡见徽,说令服罪,虽失郡守,保无他忧。岱寻匡后至,徽兄祗,弟幹、颂等六人肉袒奉迎。岱谢令复服,前至郡下。明旦早施帐幔,请徽兄弟以次入,宾客满坐。岱起,拥节读诏书,数徽罪过,左右因反缚以出,即皆伏诛,传首诣武昌。壹、〈黄有〉、匡后出,权原其罪,及燮质子廞,皆免为庶人。数岁,壹、〈黄有〉坐法诛。廞病卒,无子,妻寡居,诏在所月给俸米,赐钱四十万。

评曰:刘繇藻厉名行,好尚臧否,至於扰攘之时,据万里之土,非其长也。太史慈信义笃烈,有古人之分。士燮作守南越,优游终世,至子不慎,自贻凶咎,盖庸才玩富贵而恃阻险,使之然也。

\part{吴书五}
\chapter{妃嫔传第五}

孙破虏吴夫人,吴主权母也。本吴人,徙钱唐,早失父母,与弟景居。孙坚闻其才貌,欲娶之。吴氏亲戚嫌坚轻狡,将拒焉,坚甚以惭恨。夫人谓亲戚曰:“何爱一女以取祸乎?如有不遇,命也。”於是遂许为婚,生四男一女。

景常随坚征伐有功,拜骑都尉。袁术上景领丹杨太守,讨故太守周昕,遂据其郡。孙策与孙河、吕范依景,合众共讨泾县山贼祖郎,郎败走。会为刘繇所迫,景复北依术,术以为督军中郎将,与孙贲共讨樊能、于麋於横江,又击笮融、薛礼於秣陵。时策被创牛渚,降贼复反,景攻讨,尽禽之。从讨刘繇,繇奔豫章,策遣景、贲到寿春报术。术方与刘备争徐州,以景为广陵太守。术后僣号,策以书喻术,术不纳,便绝江津,不与通,使人告景。景即委郡东归,策复以景为丹杨太守。汉遣议郎王誧衔命南行,表景为扬武将军,领郡如故。

及权少年统业,夫人助治军国,甚有补益。建安七年,临薨,引见张昭等,属以后事,合葬高陵。

八年,景卒官,子奋授兵为将,封新亭侯,卒。子安嗣,安坐党鲁王霸死。奋弟祺嗣,封都亭侯,卒。子纂嗣。纂妻即滕胤女也,胤被诛,并遇害。

吴主权谢夫人,会稽山阴人也。父煚,汉尚书郎、徐令。权母吴,为权聘以为妃,爱幸有宠。后权纳姑孙徐氏,欲令谢下之,谢不肯,由是失志,早卒。后十馀年,弟承拜五官郎中,稍迁长沙东部都尉、武陵太守,撰后汉书百馀卷。

吴主权徐夫人,吴郡富春人也。祖父真,与权父坚相亲,坚以妹妻真,生琨。琨少仕州郡,汉末扰乱,去吏,随坚征伐有功,拜偏将军。坚薨,随孙策讨樊能、于麋等於横江,击张英於当利口,而船少,欲驻军更求。琨母时在军中,谓琨曰:“恐州家多发水军来逆人,则不利矣,如何可驻邪?宜伐芦苇以为泭,佐船渡军。”琨具启策,策即行之,众悉俱济,遂破英,击走笮融、刘繇,事业克定。策表琨领丹杨太守,会吴景委广陵来东,复为丹杨守,琨以督军中郎将领兵,从破庐江太守李术,封广德侯,迁平虏将军。后从讨黄祖,中流矢卒。

琨生夫人,初適同郡陆尚。尚卒,权为讨虏将军在吴,聘以为妃,使母养子登。后权迁移,以夫人妒忌,废处吴。积十馀年,权为吴王及即尊号,登为太子,群臣请立夫人为后,权意在步氏,卒不许。后以疾卒。兄矫,嗣父琨侯,讨平山越,拜偏将军,先夫人卒,无子。弟祚袭封,亦以战功至芜湖督、平魏将军。

吴主权步夫人,临淮淮阴人也,与丞相骘同族。汉末,其母携将徙庐江,庐江为孙策所破,皆东渡江,以美丽得幸於权,宠冠后庭。生二女,长曰鲁班,字大虎,前配周瑜子循,后配全琮;少曰鲁育,字小虎,前配朱据,后配刘纂。

夫人性不妒忌,多所推进,故久见爱待。权为王及帝,意欲以为后,而群臣议在徐氏,权依违者十馀年,然宫内皆称皇后,亲戚上疏称中宫。及薨,臣下缘权指,请追正名号,乃赠印绶,策命曰:“惟赤乌元年闰月戊子,皇帝曰:呜呼皇后,惟后佐命,共承天地。虔恭夙夜,与朕均劳。内教脩整,礼义不愆。宽容慈惠,有淑懿之德。民臣县望,远近归心。朕以世难未夷,大统未一,缘后雅志,每怀谦损。是以于时未授名号,亦必谓后降年有永,永与朕躬对扬天休。不寤奄忽,大命近止。朕恨本意不早昭显,伤后殂逝,不终天禄。愍悼之至,痛于厥心。今使使持节丞相醴陵侯雍,奉策授号,配食先后。魂而有灵,嘉其宠荣。呜呼哀哉!”葬於蒋陵。

吴主权王夫人,琅邪人也。夫人以选入宫,黄武中得幸,生和,宠次步氏。步氏薨后,和立为太子,权将立夫人为后,而全公主素憎夫人,稍稍谮毁。及权寝疾,言有喜色,由是权深责怒,以忧死。和子皓立,追尊夫人曰大懿皇后,封三弟皆列侯。

吴主权王夫人,南阳人也,以选入宫,嘉禾中得幸,生休。及和为太子,和母贵重,诸姬有宠者,皆出居外。夫人出公安,卒,因葬焉。休即位,遣使追尊曰敬怀皇后,改葬敬陵。王氏无后,封同母弟文雍为亭侯。

吴主权潘夫人,会稽句章人也。父为吏,坐法死。夫人与姊俱输织室,权见而异之,召充后宫。得幸有娠,梦有以龙头授己者,己以蔽膝受之,遂生亮。赤乌十三年,亮立为太子,请出嫁夫人之姊,权听许之。明年,立夫人为皇后。性险妒容媚,自始至卒,谮害袁夫人等甚众。权不豫,夫人使问中书令孙弘吕后专制故事。侍疾疲劳,因以羸疾,诸宫人伺其昬卧,共缢杀之,讬言中恶。后事泄,坐死者六七人。权寻薨,合葬蒋陵。孙亮即位,以夫人姊婿谭绍为骑都尉,授兵。亮废,绍与家属送本郡庐陵。

孙亮全夫人,全尚女也。从祖母公主爱之,每进见辄与俱。及潘夫人母子有宠,全主自以与孙和母有隙,乃劝权为潘氏男亮纳夫人,亮遂为嗣。夫人立为皇后,以尚为城门校尉,封都亭侯,代滕胤为太常、卫将军,进封永平侯,录尚书事。时全氏侯有五人,并典兵马,其馀为侍郎、骑都尉,宿卫左右,自吴兴,外戚贵盛莫及。及魏大将诸葛诞以寿春来附,而全怿、全端、全祎、全仪等并因此际降魏,全熙谋泄见杀,由是诸全衰弱。会孙綝废亮为会稽王,后又黜为候官侯,夫人随之国,居候官,尚将家属徙零陵,追见杀。

孙休朱夫人,朱据女,休姊公主所生也。赤乌末,权为休纳以为妃。休为琅邪王,随居丹阳。建兴中,孙峻专政,公族皆患之。全尚妻即峻姊,故惟全主祐焉。初,孙和为太子时,全主谮害王夫人,欲废太子,立鲁王,朱主不听,由是有隙。五凤中,孙仪谋杀峻,事觉被诛。全主因言朱主与仪同谋,峻枉杀朱主。休惧,遣夫人还建业,执手泣别。既至,峻遣还休。太平中,孙亮知朱主为全主所害,问朱主死意?全主惧曰:“我实不知,皆据二子熊、损所白。”亮杀熊、损。损妻是峻妹也,孙綝益忌亮,遂废亮,立休。永安五年,立夫人为皇后。休卒,群臣尊夫人为皇太后。孙皓即位月馀,贬为景皇后,称安定宫。甘露元年七月,见逼薨,合葬定陵。

孙和何姬,丹杨句容人也。父遂,本骑士。孙权尝游幸诸营,而姬观於道中,权望见异之,命宦者召入,以赐子和。生男,权喜,名之曰彭祖,即皓也。太子和既废,后为南阳王,居长沙。孙亮即位,孙峻辅政。峻素媚事全主,全主与和母有隙,遂劝峻徙和居新都,遣使赐死,嫡妃张氏亦自杀。何姬曰:“若皆从死,谁当养孤?”遂拊育皓,及其三弟。皓即位,尊和为昭献皇帝,何姬为昭献皇后,称升平宫,月馀,进为皇太后。封弟洪永平侯,蒋溧阳侯,植宣城侯。洪卒,子邈嗣,为武陵监军,为晋所杀。植官至大司徒。吴末昬乱,何氏骄僣,子弟横放,百姓患之。故民讹言”皓久死,立者何氏子”云。

孙皓滕夫人,故太常胤之族女也。胤夷灭,夫人父牧,以疏远徙边郡。孙休即位,大赦,得还,以牧为五官中郎。皓既封乌程侯,聘牧女为妃。皓即位,立为皇后,封牧高密侯,拜卫将军,录尚书事。后朝士以牧尊戚,颇推令谏争。而夫人宠渐衰,皓滋不悦,皓母何恒左右之。又太史言,於运历,后不可易,皓信巫觋,故得不废,常供养升平宫。牧见遣居苍梧郡,虽爵位不夺,其实裔也,遂道路忧死。长秋官僚,备员而已,受朝贺表疏如故。而皓内诸宠姬,佩皇后玺绂者多矣。天纪四年,随皓迁于洛阳。

评曰:易称“正家而天下定”。诗云:“刑于寡妻,至于兄弟,以御于家邦。”诚哉,是言也!远观齐桓,近察孙权,皆有识士之明,杰人之志,而嫡庶不分,闺庭错乱,遗笑古今,殃流后嗣。由是论之,惟以道义为心、平一为主者,然后克免斯累邪!

\part{吴书六}
\chapter{宗室传第六}

孙静字幼台,坚季弟也。坚始举事,静纠合乡曲及宗室五六百人以为保障,众咸附焉。策破刘繇,定诸县,进攻会稽,遣人请静,静将家属与策会于钱唐。是时太守王朗拒策於固陵,策数度水战,不能克。静说策曰:“朗负阻城守,难可卒拔。查渎南去此数十里,而道之要径也,宜从彼据其内,所谓攻其无备、出其不意者也。吾当自帅众为军前队,破之必矣。“策曰:“善。“乃诈令军中曰:“顷连雨水浊,兵饮之多腹痛,令促具罂缶数百口澄水。“至昏暮,罗以然火诳朗,便分军夜投查渎道,袭高迁屯。朗大惊,遣故丹杨太守周昕等帅兵前战。策破昕等,斩之,遂定会稽。表拜静为奋武校尉,欲授之重任,静恋坟墓宗族,不乐出仕,求留镇守。策从之。权统事,就迁昭义中郎将,终於家。有五子,暠、瑜、皎、奂、谦。暠三子:绰、超、恭。超为偏将军。恭生峻。绰生綝。

瑜字仲异,以恭义校尉始领兵众。是时宾客诸将多江西人,瑜虚心绥抚,得其欢心。建安九年,领丹杨太守,为众所附,至万馀人。加绥远将军。十一年,与周瑜共讨麻、保二屯,破之。后从权拒曹公於濡须,权欲交战,瑜说权持重,权不从,军果无功。迁奋威将军,领郡如故,自溧阳徙屯牛渚。瑜以永安人饶助为襄安长,无锡人颜连为居巢长,使招纳庐江二郡,各得降附。济阴人马普笃学好古,瑜厚礼之,使二府将吏子弟数百人就受业,遂立学官,临飨讲肄。是时诸将皆以军务为事,而瑜好乐坟典,虽在戎旅,诵声不绝。年三十九,建安二十年卒。瑜五子:弥、熙、耀、曼、纮。曼至将军,封侯。

孙皎字叔朗,始拜护军校尉,领众二千馀人。是时曹公数出濡须,皎每赴拒,号为精锐。迁都护征虏将军,代程普督夏口。黄盖及兄瑜卒,又并其军。赐沙羡、云杜、南新市、竟陵为奉邑,自置长吏。轻财能施,善於交结,与诸葛瑾至厚,委庐江刘靖以得失,江夏李允以众事,广陵吴硕、河南张梁以军旅,而倾心亲待,莫不自尽。皎尝遣兵候获魏边将吏美女以进皎,皎更其衣服送还之,下令曰:“今所诛者曹氏,其百姓何罪?自今以往,不得击其老弱。“由是江淮间多归附者。尝以小故与甘宁忿争,或以谏宁,宁曰:“臣子一例,征虏虽公子,何可专行侮人邪!吾值明主,但当输效力命,以报所天,诚不能随俗屈曲矣。“权闻之,以书让皎曰:“自吾与北方为敌,中间十年,初时相持年小,今者且三十矣。孔子言'三十而立',非但谓五经也。授卿以精兵,委卿以大任,都护诸将於千里之外,欲使如楚任昭奚恤,扬威於北境,非徒相使逞私志而已。近闻卿与甘兴霸饮,因酒发作,侵陵其人,其人求属吕蒙督中。此人虽粗豪,有不如人意时,然其较略大丈夫也。吾亲之者,非私之也。我亲爱之,卿疏憎之;卿所为每与吾违,其可久乎?夫居敬而行简,可以临民;爱人多容,可以得众。二者尚不能知,安可董督在远,御寇济难乎?卿行长大,特受重任,上有远方瞻望之视,下有部曲朝夕从事,何可恣意有盛怒邪?人谁无过,贵其能改,宜追前愆,深自咎责。今故烦诸葛子瑜重宣吾意。临书摧怆,心悲泪下。“皎得书,上疏陈谢,遂与宁结厚。后吕蒙当袭南郡,权欲令皎与蒙为左右部大督,蒙说权曰:“若至尊以征虏能,宜用之;以蒙能,宜用蒙。昔周瑜、程普为左右部督,共攻江陵,虽事决於瑜,普自恃久将,且俱是督,遂共不睦,几败国事,此目前之戒也。“权寤,谢蒙曰:“以卿为大督,命皎为后继。“禽关羽,定荆州,皎有力焉。建安二十四年卒。权追录其功,封子胤为丹杨侯。胤卒,无子。弟晞嗣,领兵,有罪自杀,国除。弟咨、弥、仪皆将军,封侯。咨羽林督,仪无难督。咨为滕胤所杀,仪为孙峻所害。

孙奂字季明。兄皎既卒,代统其众,以扬武中郎将领江夏太守。在事一年,遵皎旧迹,礼刘靖、李允、吴硕、张梁及江夏闾举等,并纳其善。奂讷於造次而敏於当官,军民称之。黄武五年,权攻石阳,奂以地主,使所部将军鲜于丹帅五千人先断淮道,自帅吴硕、张梁五千人为军前锋,降高城,得三将。大军引还,权诏使在前往,驾过其军,见奂军陈整齐,权叹曰:“初吾忧其迟钝,今治军,诸将少能及者,吾无忧矣。“拜扬威将军,封沙羡侯。吴硕、张梁皆裨将军,赐爵关内侯。奂亦爱乐儒生,复命部曲子弟就业,后仕进朝廷者数十人。年四十,嘉禾三年卒。子承嗣,以昭武中郎将代统兵,领郡。赤乌六年卒,无子,封承庶弟壹奉奂后,袭业为将。孙峻之诛诸葛恪也,壹与全熙、施绩攻恪弟公安督融,融自杀。壹从镇南迁镇军,假节督夏口。及孙綝诛滕胤、吕据,据、胤皆壹之妹夫也,壹弟封又知胤、据谋,自杀。綝遣朱异潜袭壹。异至武昌,壹知其攻己,率部曲千馀口过将胤妻奔魏。魏以壹为车骑将军、仪同三司,封吴侯,以故主芳贵人邢氏妻之。邢美色妒忌,下不堪命,遂共杀壹及邢氏。壹入魏三年死。

孙贲字伯阳。父羌字圣台,坚同产兄也。贲早失二亲,弟辅婴孩,贲自赡育,友爱甚笃。为郡督邮守长。坚於长沙举义兵,贲去吏从征伐。坚薨,贲摄帅馀众,扶送灵柩。后袁术徙寿春,贲又依之。术从兄绍用会稽周昂为九江太守,绍与术不协,术遣贲攻破昂於阴陵。术表贲领豫州刺史,转丹杨都尉,行征虏将军,讨平山越。为扬州刺史刘繇所迫逐,因将士众还住历阳。顷之,术复使贲与吴景共击樊能、张英等,未能拔。及策东渡,助贲、景破英、能等,遂进击刘繇。繇走豫章。策遣贲、景还寿春报术,值术僣号,署置百官,除贲九江太守。贲不就,弃妻孥还江南。时策已平吴、会二郡,贲与策征庐江太守刘勋、江夏太守黄祖,军旋,闻繇病死,过定豫章,上贲领太守,后封都亭侯。建安十三年,使者刘隐奉诏拜贲为征虏将军,领郡如故。在官十一年卒。子邻嗣。

邻年九岁,代领豫章,进封都乡侯。在郡垂二十年,讨平叛贼,功绩脩理。召还武昌,为绕帐督。时太常潘濬掌荆州事,重安长陈留舒燮有罪下狱,濬尝失燮,欲寘之於法。论者多为有言,濬犹不释。邻谓濬曰:“舒伯膺兄弟争死,海内义之,以为美谭,仲膺又有奉国旧意。今君杀其子弟,若天下一统,青盖北巡,中州士人必问仲膺继嗣,答者云潘承明杀燮,於事何如?“濬意即解,燮用得济。邻迁夏口沔中督、威远将军,所居任职。赤乌十二年卒。子苗嗣。苗弟旅及叔父安、熙、绩,皆历列位。

孙辅字国仪,贲弟也,以扬武校尉佐孙策平三郡。策讨丹杨七县,使辅西屯历阳以拒袁术,并招诱馀民,鸠合遗散。又从策讨陵阳,生得祖郎等。策西袭庐江太守刘勋,辅随从,身先士卒,有功。策立辅为庐陵太守,抚定属城,分置长吏。迁平南将军,假节领交州刺史。遣使与曹公相闻,事觉,权幽系之。数岁卒。子兴、昭、伟、昕,皆历列位。

孙翊字叔弼,权弟也,骁悍果烈,有兄策风。太守朱治举孝廉,司空辟。建安八年,以偏将军领丹杨太守,时年二十。后卒为左右边鸿所杀,鸿亦即诛。

子松为射声校尉、都乡侯。黄龙三年卒。蜀丞相诸葛亮与兄瑾书曰:“既受东朝厚遇,依依於子弟。又子乔良器,为之恻怆。见其所与亮器物,感用流涕。“其悼松如此,由亮养子乔咨述故云。

孙匡字季佐,翊弟也。举孝廉茂才,未试用,卒,时年二十馀。子泰,曹氏之甥也,为长水校尉。嘉禾三年,从权围新城,中流矢死。泰子秀为前将军、夏口督。秀公室至亲,握兵在外,皓意不能平。建衡二年,皓遣何定将五千人至夏口猎。先是,民间佥言秀当见图,而定远猎,秀遂惊,夜将妻子亲兵数百人奔晋。晋以秀为骠骑将军、仪同三司,封会稽公。

孙韶字公礼。伯父河,字伯海,本姓俞氏,亦吴人也。孙策爱之,赐姓为孙,列之属籍。后为将军,屯京城。

初,孙权杀吴郡太守盛宪,宪故孝廉妫览、戴员亡匿山中,孙翊为丹杨,皆礼致之。览为大都督督兵,员为郡丞。及翊遇害,河驰赴宛陵,责怒览、员,以不能全权,令使奸变得施。二人议曰:“伯海与将军疏远,而责我乃耳。讨虏若来,吾属无遗矣。“遂杀河,使人北迎扬州刺史刘馥,令住历阳,以丹杨应之。会翊帐下徐元、孙高、傅婴等杀览、员。

韶年十七,收河馀众,缮治京城,起楼橹,脩器备以御敌。权闻乱,从椒丘还,过定丹杨,引军归吴。夜至京城下营,试攻惊之,兵皆乘城传檄备警,讙声动地,颇射外人,权使晓喻乃止。明日见韶,甚器之,即拜承烈校尉,统河部曲,食曲阿、丹徒二县,自置长吏,一如河旧。后为广陵太守、偏将军。权为吴王,迁扬威将军,封建德侯。权称尊号,为镇北将军。韶为边将数十年,善养士卒,得其死力。常以警疆埸远斥候为务,先知动静而为之备,故鲜有负败。青、徐、汝、沛颇来归附,淮南滨江屯候皆彻兵远徙,徐、泗、江、淮之地,不居者各数百里。自权西征,还都武昌,韶不进见者十馀年。权还建业,乃得朝觐。权问青、徐诸屯要害,远近人马众寡,魏将帅姓名,尽具识之,有问咸对。身长八尺,仪貌都雅。权欢悦曰:“吾久不见公礼,不图进益乃尔。“加领幽州牧、假节。赤乌四年卒。子越嗣,至右将军。越兄楷武卫大将军、临成侯,代越为京下督。楷弟异至领军将军,奕宗正卿,恢武陵太守。天玺元年,徵楷为宫下镇骠骑将军。初永安贼施但等劫皓弟谦,袭建业,或白楷二端不即赴讨者,皓数遣诘楷。楷常惶怖,而卒被召,遂将妻子亲兵数百人归晋,晋以为车骑将军,封丹杨侯。

孙桓字叔武,河之子也。年二十五,拜安东中郎将,与陆逊共拒刘备。备军众甚盛,弥山盈谷,桓投刀奋命,与逊戮力,备遂败走。桓斩上夔道,截其径要。备逾山越险,仅乃得免,忿恚叹曰:“吾昔初至京城,桓尚小儿,而今迫孤乃至此也!“桓以功拜建武将军,封丹徒侯,下督牛渚,作横江坞,会卒。

评曰:夫亲亲恩义,古今之常。宗子维城,诗人所称。况此诸孙,或赞兴初基,或镇据边陲,克堪厥任,不忝其荣者乎!故详著云。

\part{吴书七}
\chapter{张顾诸葛步传第七}

张昭字子布,彭城人也。少好学,善隶书,从白侯子安受左氏春秋,博览众书,与琅邪赵昱、东海王朗俱发名友善。弱冠察孝廉,不就,与朗共论旧君讳事,州里才士陈琳等皆称善之。刺史陶谦举茂才,不应,谦以为轻己,遂见拘执。昱倾身营救,方以得免。汉末大乱,徐方士民多避难扬土,昭皆南渡江。孙策创业,命昭为长史、抚军中郎将,升堂拜母,如比肩之旧,文武之事,一以委昭。昭每得北方士大夫书疏,专归美於昭,昭欲嘿而不宣则惧有私,宣之则恐非宜,进退不安。策闻之,欢笑曰:“昔管仲相齐,一则仲父,二则仲父,而桓公为霸者宗。今子布贤,我能用之,其功名独不在我乎!”

策临亡,以弟权讬昭,昭率群僚立而辅之。上表汉室,下移属城,中外将校,各令奉职。权悲感未视事,昭谓权曰:“夫为人后者,贵能负荷先轨,克昌堂构,以成勋业也。方今天下鼎沸,群盗满山,孝廉何得寝伏哀戚,肆匹夫之情哉?”乃身自扶权上马,陈兵而出,然后众心知有所归。昭复为权长史,授任如前。后刘备表权行车骑将军,昭为军师。权每田猎,常乘马射虎,虎常突前攀持马鞍。昭变色而前曰:“将军何有当尔?夫为人君者,谓能驾御英雄,驱使群贤,岂谓驰逐於原野,校勇於猛兽者乎?如有一旦之患,奈天下笑何?”权谢昭曰:“年少虑事不远,以此惭君。”然犹不能已,乃作射虎车,为方目,间不置盖,一人为御,自於中射之。时有逸群之兽,辄复犯车,而权每手击以为乐。昭虽谏争,常笑而不答。魏黄初二年,遣使者邢贞拜权为吴王。贞入门,不下车。昭谓贞曰:“夫礼无不敬,故法无不行。而君敢自尊大,岂以江南寡弱,无方寸之刃故乎!”贞即遽下车。拜昭为绥远将军,封由拳侯。权於武昌,临钓台,饮酒大醉。权使人以水洒群臣曰:“今日酣饮,惟醉堕台中,乃当止耳。”昭正色不言,出外车中坐。权遣人呼昭还,谓曰:“为共作乐耳,公何为怒乎?”昭对曰:“昔纣为糟丘酒池长夜之饮,当时亦以为乐,不以为恶也。”权默然,有惭色,遂罢酒。初,权当置丞相,众议归昭。权曰:“方今多事,职统者责重,非所以优之也。”后孙邵卒,百寮复举昭,权曰:“孤岂为子布有爱乎?领丞相事烦,而此公性刚,所言不从,怨咎将兴,非所以益之也。”乃用顾雍。

权既称尊号,昭以老病,上还官位及所统领。更拜辅吴将军,班亚三司,改封娄侯,食邑万户。在里宅无事,乃著春秋左氏传解及论语注。权尝问卫尉严峻:“宁念小时所闇书不?”峻因诵孝经“仲尼居”。昭曰:“严畯鄙生,臣请为陛下诵之。”乃诵“君子之事上”,咸以昭为知所诵。

昭每朝见,辞气壮厉,义形於色,曾以直言逆旨,中不进见。后蜀使来,称蜀德美,而群臣莫拒,权叹曰:“使张公在坐,彼不折则废,安复自夸乎?”明日,遣中使劳问,因请见昭。昭避席谢,权跪止之。昭坐定,仰曰:“昔太后、桓王不以老臣属陛下,而以陛下属老臣,是以思尽臣节,以报厚恩,使泯没之后,有可称述,而意虑浅短,违逆盛旨,自分幽沦,长弃沟壑,不图复蒙引见,得奉帷幄。然臣愚心所以事国,志在忠益,毕命而已。若乃变心易虑,以偷荣取容,此臣所不能也”权辞谢焉。

权以公孙渊称藩,遣张弥、许晏至辽东拜渊为燕王,昭谏曰:“渊背魏惧讨,远来求援,非本志也。若渊改图,欲自明於魏,两使不反,不亦取笑於天下乎?”权与相反覆,昭意弥切。权不能堪,案刀而怒曰:“吴国士人入宫则拜孤,出宫则拜君,孤之敬君,亦为至矣,而数於众中折孤,孤尝恐失计。”昭熟视权曰:“臣虽知言不用,每竭愚忠者,诚以太后临崩,呼老臣於床下,遗诏顾命之言故在耳。”因涕泣横流。权掷刀致地,与昭对泣。然卒遣弥、晏往。昭忿言之不用,称疾不朝。权恨之,土塞其门,昭又於内以土封之。渊果杀弥、晏。权数慰谢昭,昭固不起,权因出过其门呼昭,昭辞疾笃。权烧其门,欲以恐之,昭更闭户。权使人灭火,住门良久,昭诸子共扶昭起,权载以还宫,深自克责。昭不得已,然后朝会。

昭容貌矜严,有威风,权常曰:“孤与张公言,不敢妄也。”举邦惮之。年八十一,嘉禾五年卒。遗令幅巾素棺,敛以时服。权素服临吊,谥曰文侯。长子承已自封侯,少子休袭爵。

昭弟子奋年二十,造作攻城大攻车,为步骘所荐。昭不愿曰:“汝年尚少,何为自委於军旅乎?”奋对曰:“昔童汪死难,子奇治阿,奋实不才耳,於年不为少也。”遂领兵为将军,连有功效,至半州都督,封乐乡亭侯。

承字仲嗣,少以才学知名,与诸葛瑾、步骘、严畯相友善。权为骠骑将军,辟西曹掾,出为长沙西部都尉。讨平山寇,得精兵万五千人。后为濡须都督、奋威将军,封都乡侯,领部曲五千人,承为人壮毅忠谠,能甄识人物,拔彭城蔡款、南阳谢景於孤微童幼,后并为国士,款至卫尉,景豫章太守。又诸葛恪年少时,众人奇其英才,承言终败诸葛氏者元逊也。勤於长进,笃於物类,凡在庶几之流,无不造门。年六十七,赤乌七年卒,谥曰定侯。子震嗣。初,承丧妻,昭欲为索诸葛瑾女,承以相与有好,难之,权闻而劝焉,遂为婿。生女,权为子和纳之。权数令和脩敬於承,执子婿之礼。震诸葛恪诛时亦死。

休字叔嗣,弱冠与诸葛恪、顾谭等俱为太子登僚友,以汉书授登。从中庶子转为右弼都尉。权常游猎,迨暮乃归,休上疏谏戒,权大善之,以示於昭。及登卒后,为侍中,拜羽林都督,平三典军事,迁扬武将军。为鲁王霸友党所谮,与顾谭、承俱以芍陂论功事,休、承与典军陈恂通情,诈增其伐,并徙交州。中书令孙弘佞伪险诐,休素所忿,弘因是谮诉,下诏书赐休死,时年四十一。

顾雍字元叹,吴郡吴人也。蔡伯喈从朔方还,尝避怨於吴,雍从学琴书。州郡表荐,弱冠为合肥长,后转在娄、曲阿、上虞,皆有治迹。孙权领会稽太守,不之郡,以雍为丞,行太守事,讨除寇贼,郡界宁静,吏民归服。数年,入为左司马。权为吴王,累迁大理奉常,领尚书令,封阳遂乡侯,拜侯还寺,而家人不知,后闻乃惊。

黄武四年,迎母於吴。既至,权临贺之,亲拜其母於庭,公卿大臣毕会,后太子又往庆焉。雍为人不饮酒,寡言语,举动时当。权尝叹曰:“顾君不言,言必有中。”至饮宴欢乐之际,左右恐有酒失而雍必见之,是以不敢肆情。权亦曰:“顾公在坐,使人不乐。”其见惮如此。是岁,改为太常,进封醴陵侯,代孙邵为丞相,平尚书事。其所选用文武将吏各随能所任,心无適莫。时访逮民间,及政职所宜,辄密以闻。若见纳用,则归之於上,不用,终不宣泄。权以此重之。然於公朝有所陈及,辞色虽顺而所执者正。权尝咨问得失,张昭因陈听采闻,颇以法令太稠,刑罚微重,宜有所蠲损。权默然,顾问雍曰:“君以为何如?”雍对曰:“臣之所闻,亦如昭所陈。”於是权乃议狱轻刑。久之,吕壹、秦博为中书,典校诸官府及州郡文书。壹等因此渐作威福,遂造作榷酤障管之利,举罪纠奸,纤介必闻,重以深案丑诬,毁短大臣,排陷无辜,雍等皆见举白,用被谴让。后壹奸罪发露,收系廷尉。雍往断狱,壹以囚见,雍和颜色,问其辞状,临出,又谓壹曰:“君意得无欲有所道?”壹叩头无言。时尚书郎怀叙面詈辱壹,雍责叙曰:“官有正法,何至於此!”

雍为相十九年,年七十六,赤乌六年卒。初疾微时,权令医赵泉视之,拜其少子济为骑都尉。雍闻,悲曰:“泉善别死生,吾必不起,故上欲及吾目见济拜也。”权素服临吊,谥曰肃侯。长子邵早卒,次子裕有笃疾,少子济嗣,无后,绝。永安元年,诏曰:“故丞相雍,至德忠贤,辅国以礼,而侯统废绝,朕甚愍之。其以雍次子裕袭爵为醴陵侯,以明著旧勋。”

邵字孝则,博览书传,好乐人伦。少与舅陆绩齐名,而陆逊、张敦、卜静等皆亚焉。自州郡庶几及四方人士,往来相见,或言议而去,或结厚而别,风声流闻,远近称之。权妻以策女。年二十七,起家为豫章太守。下车祀先贤徐孺子之墓,优待其后;禁其淫祀非礼之祭者。小吏资质佳者,辄令就学,择其先进,擢置右职,举善以教,风化大行。初,钱唐丁谞出於役伍,阳羡张秉生於庶民,乌程吴粲、云阳殷礼起乎微贱,邵皆拔而友之,为立声誉。秉遭大丧,亲为制服结绖。邵当之豫章,发在近路,值秉疾病,时送者百数,邵辞宾客曰:“张仲节有疾,苦不能来别,恨不见之,暂还与诀,诸君少时相待。”其留心下士,惟善所在,皆此类也。谞至典军中郎,秉云阳太守,礼零陵太守,粲太子少傅。世以邵为知人。在郡五年,卒官,子谭、承云。

谭字子默,弱冠与诸葛恪等为太子四友,从中庶子转辅正都尉。赤乌中,代恪为左节度。每省簿书,未尝下筹,徒屈指心计,尽发疑谬,下吏以此服之。加奉车都尉。薛综为选曹尚书,固让谭曰:“谭心精体密,贯道达微,才照人物,德允众望,诚非愚臣所可越先。”后遂代综。祖父雍卒数月,拜太常,代雍平尚书事。是时鲁王霸有盛宠,与太子和齐衡,谭上疏曰:“臣闻有国有家者,必明嫡庶之端,异尊卑之礼,使高下有差,阶级逾邈,如此则骨肉之恩生,觊觎之望绝。昔贾谊陈治安之计,论诸侯之势,以为势重,虽亲必有逆节之累,势轻,虽疏必有保全之祚。故淮南亲弟,不终飨国,失之於势重也;吴芮疏臣,传祚长沙,得之於势轻也。昔汉文帝使慎夫人与皇后同席,袁盎退夫人之座,帝有怒色,及盎辨上下之仪,陈人彘之戒,帝既悦怿,夫人亦悟。今臣所陈,非有所偏,诚欲以安太子而便鲁王也。”由是霸与谭有隙。时长公主婿卫将军全琮子寄为霸宾客,寄素倾邪,谭所不纳。先是,谭弟承与张休俱北征寿春,全琮时为大都督,与魏将王凌战於芍陂,军不利,魏兵乘胜陷没五营将秦晃军,休、承奋击之。遂驻魏师。时琮群子绪、端亦并为将,因敌既住,乃进击之,凌军用退。时论功行赏,以为驻敌之功大,退敌之功小,休、承并为杂号将军,绪、端偏裨而已。寄父子益恨,共构会谭。谭坐徙交州,幽而发愤,著新言二十篇。其知难篇盖以自悼伤也。见流二年,年四十二,卒於交阯。

承字子直,嘉禾中与舅陆瑁俱以礼徵。权赐丞相雍书曰:“贵孙子直,令问休休,至与相见,过於所闻,为君嘉之。”拜骑都尉,领羽林兵。后为吴郡西部都尉,与诸葛恪等共平山越,别得精兵八千人,还屯军章坑,拜昭义中郎将,入为侍中。芍陂之役,拜奋威将军,出领京下督。数年,与兄谭、张休等俱徙交州,年三十七卒。

诸葛瑾字子瑜,琅邪阳都人也。汉末避乱江东。值孙策卒,孙权姊婿曲阿弘咨见而异之,荐之於权,与鲁肃等并见宾待,后为权长史,转中司马。建安二十年,权遣瑾使蜀通好刘备,与其弟亮俱公会相见,退无私面。

与权谈说谏喻,未尝切愕,微见风彩,粗陈指归,如有未合,则舍而及他,徐复讬事造端,以物类相求,於是权意往往而释。吴郡太守朱治,权举将也,权曾有以望之,而素加敬,难自诘让,忿忿不解。瑾揣知其故,而不敢显陈,乃乞以意私自问,遂於权前为书,泛论物理,因以己心遥往忖度之。毕,以呈权,权喜,笑曰:“孤意解矣。颜氏之德,使人加亲,岂谓此邪?”权又怪校尉殷模,罪至不测。群下多为之言,权怒益甚,与相反覆,惟瑾默然,权曰:“子瑜何独不言?”瑾避席曰:“瑾与殷模等遭本州倾覆,生类殄尽。弃坟墓,携老弱,披草莱,归圣化,在流隶之中,蒙生成之福,不能躬相督厉,陈答万一,至令模孤负恩惠,自陷罪戾。臣谢过不暇,诚不敢有言。”权闻之怆然,乃曰:“特为君赦之。”

后从讨关羽,封宣城侯,以绥南将军代吕蒙领南郡太守,住公安。刘备东伐吴,吴王求和,瑾与备笺曰:“奄闻旗鼓来至白帝,或恐议臣以吴王侵取此州,危害关羽,怨深祸大,不宜答和,此用心於小,未留意於大者也。试为陛下论其轻重,及其大小。陛下若抑威损忿,蹔省瑾言者,计可立决,不复咨之於群后也。陛下以关羽之亲何如先帝?荆州大小孰与海内?俱应仇疾,谁当先后?若审此数,易於反掌。”时或言瑾别遣亲人与备相闻,权曰:“孤与子瑜有死生不易之誓,子瑜之不负孤,犹孤之不负子瑜也。”黄武元年,迁左将军,督公安,假节,封宛陵侯。

虞翻以狂直流徙,惟瑾屡为之说。翻与所亲书曰:“诸葛敦仁,则天活物,比蒙清论,有以保分。恶积罪深,见忌殷重,虽有祁老之救,德无羊舌,解释难冀也。”

瑾为人有容貌思度,于时服其弘雅。权亦重之,大事咨访。又别咨瑾曰:“近得伯言表,以为曹丕已死,毒乱之民,当望旌瓦解,而更静然。闻皆选用忠良,宽刑罚,布恩惠,薄赋省役,以悦民心,其患更深於操时。孤以为不然。操之所行,其惟杀伐小为过差,及离间人骨肉,以为酷耳。至於御将,自古少有。丕之於操,万不及也。今叡之不如丕,犹丕不如操也。其所以务崇小惠,必以其父新死,自度衰微,恐困苦之民一朝崩沮,故强屈曲以求民心,欲以自安住耳,宁是兴隆之渐邪!闻任陈长文、曹子丹辈,或文人诸生,或宗室戚臣,宁能御雄才虎将以制天下乎?夫威柄不专,则其事乖错,如昔张耳、陈馀,非不敦睦,至於秉势,自还相贼,乃事理使然也。又长文之徒,昔所以能守善者,以操笮其头,畏操威严,故竭心尽意,不敢为非耳。逮丕继业,年已长大,承操之后,以恩情加之,用能感义。今叡幼弱,随人东西,此曹等辈,必当因此弄巧行态,阿党比周,各助所附。如此之日,奸谗并起,更相陷怼,转成嫌贰。一尔已往,群下争利,主幼不御,其为败也焉得久乎?所以知其然者,自古至今,安有四五人把持刑柄,而不离刺转相蹄齧者也!强当陵弱,弱当求援,此乱亡之道也。子瑜,卿但侧耳听之,伯言常长於计校,恐此一事小短也。”

权称尊号,拜大将军、左都护,领豫州牧。及吕壹诛,权又有诏切磋瑾等,语在权传。瑾辄因事以答,辞顺理正。瑾子恪,名盛当世,权深器异之;然瑾常嫌之,谓非保家之子,每以忧戚。赤乌四年,年六十八卒,遗命令素棺敛以时服,事从省约。恪已自封侯,故弟融袭爵,摄兵业驻公安,部曲吏士亲附之。疆外无事,秋冬则射猎讲武,春夏则延宾高会,休吏假卒,或不远千里而造焉。每会辄历问宾客,各言其能,乃合榻促席,量敌选对,或有博弈,或有摴蒱,投壶弓弹,部别类分,於是甘果继进,清酒徐行,融周流观览,终日不倦。融父兄质素,虽在军旅,身无采饰;而融锦罽文绣,独为奢绮。孙权薨,徙奋威将军。后恪征淮南,假融节,令引军入沔,以击西兵。恪既诛,遣无难督施宽就将军施绩、孙壹、全熙等取融。融卒闻兵士至,惶惧犹豫,不能决计,兵到围城,饮药而死,三子皆伏诛。

步骘字子山,临淮淮阴人也。世乱,避难江东,单身穷困,与广陵卫旌同年相善,俱以种瓜自给,昼勤四体,夜诵经传。

会稽焦征羌,郡之豪族,人客放纵。骘与旌求食其地,惧为所侵,乃共脩刺奉瓜,以献征羌。征羌方在内卧,驻之移时,旌欲委去,骘止之曰:“本所以来,畏其强也;而今舍去,欲以为高,祗结怨耳。“良久,征羌开牖见之,身隐几坐帐中,设席致地,坐骘、旌於牖外,旌愈耻之,骘辞色自若。征羌作食,身享大案,殽膳重沓,以小盘饭与骘、旌,惟菜茹而已。旌不能食,骘极饭致饱乃辞出。旌怒骘曰:“何能忍此?”骘曰:“吾等贫贱,是以主人以贫贱遇之,固其宜也,当何所耻?”

孙权为讨虏将军,召骘为主记,除海盐长,还辟车骑将军东曹掾。建安十五年,出领鄱阳太守。岁中,徙交州刺史、立武中郎将,领武射吏千人,便道南行。明年,追拜使持节、征南中郎将。刘表所置苍梧太守吴巨阴怀异心,外附内违。骘降意怀诱,请与相见,因斩徇之,威声大震。士燮兄弟,相率供命,南土之宾,自此始也。益州大姓雍闿等杀蜀所署太守正昂,与燮相闻,求欲内附。骘因承制遣使宣恩抚纳,由是加拜平戎将军,封广信侯。

延康元年,权遣吕岱代骘,骘将交州义士万人出长沙。会刘备东下,武陵蛮夷蠢动,权遂命骘上益阳。备既败绩,而零、桂诸郡犹相惊扰,处处阻兵;骘周旋征讨,皆平之。黄武二年,迁右将军左护军,改封临湘侯。五年,假节,徙屯沤口。

权称尊号,拜骠骑将军,领冀州牧。是岁,都督西陵,代陆逊抚二境,顷以冀州在蜀分,解牧职。时权太子登驻武昌,爱人好善,与骘书曰:“夫贤人君子,所以兴隆大化,佐理时务者也。受性闇蔽,不达道数,虽实区区欲尽心於明德,归分於君子,至於远近士人,先后之宜,犹或缅焉,未之能详。传曰:'爱之能勿劳乎?忠焉能勿诲乎?'斯其义也,岂非所望於君子哉!”骘於是条于时事业在荆州界者,诸葛瑾、陆逊、朱然、程普、潘濬、裴玄、夏侯承、卫旌、李肃、周条、石幹十一人,甄别行状,因上疏奖劝曰:“臣闻人君不亲小事,百官有司各任其职。故舜命九贤,则无所用心,弹五弦之琴,咏南风之诗,不下堂庙而天下治也。齐桓用管仲,被发载车,齐国既治,又致匡合。近汉高祖揽三杰以兴帝业,西楚失雄俊以丧成功。汲黯在朝,淮南寝谋;郅都守边,匈奴窜迹。故贤人所在,折冲万里,信国家之利器,崇替之所由也。方今王化未被於汉北,河、洛之滨尚有僣逆之丑,诚揽英雄拔俊任贤之时也。愿明太子重以轻意,则天下幸甚。”

后中书吕壹典校文书,多所纠举,骘上疏曰:“伏闻诸典校擿抉细微,吹毛求瑕,重案深诬,辄欲陷人以成威福;无罪无辜,横受大刑,是以使民跼天蹐地,谁不战栗?昔之狱官,惟贤是任,故皋陶作士,吕侯赎刑,张、于廷尉,民无冤枉,休泰之祚,实由此兴。今之小臣,动与古异,狱以贿成,轻忽人命,归咎于上,为国速怨。夫一人吁嗟,王道为亏,甚可仇疾。明德慎罚,哲人惟刑,书传所美。自今蔽狱,都下则宜谘顾雍,武昌则陆逊、潘濬,平心专意,务在得情,骘党神明,受罪何恨?”又曰:“天子父天母地,故宫室百官,动法列宿。若施政令,钦顺时节,官得其人,则阴阳和平,七曜循度。至於今日,官寮多阙,虽有大臣,复不信任,如此天地焉得无变?故频年枯旱,亢阳之应也。又嘉禾六年五月十四日,赤乌二年正月一日及二十七日,地皆震动。地阴类,臣之象,阴气盛故动,臣下专政之故也。夫天地见异,所以警悟人主,可不深思其意哉!”又曰:“丞相顾雍、上大将军陆逊、太常潘濬,忧深责重,志在谒诚,夙夜兢兢,寝食不宁,念欲安国利民,建久长之计,可谓心膂股肱,社稷之臣矣。宜各委任,不使他官监其所司,责其成效,课其负殿。此三臣者,思虑不到则已,岂敢专擅威福欺负所天乎?“又曰:“县赏以显善,设刑以威奸,任贤而使能,审明於法术,则何功而不成,何事而不辨,何听而不闻,何视而不睹哉?若今郡守百里,皆各得其人,共相经纬,如是,庶政岂不康哉?窃闻诸县并有备吏,吏多民烦,俗以之弊。但小人因缘衔命,不务奉公而作威福,无益视听,更为民害,愚以为可一切罢省。”权亦觉梧,遂诛吕壹。骘前后荐达屈滞,救解患难,书数十上。权虽不能悉纳,然时采其言,多蒙济赖。

赤乌九年,代陆逊为丞相,犹诲育门生,手不释书,被服居处有如儒生。然门内妻妾服饰奢绮,颇以此见讥。在西陵二十年,邻敌敬其威信。性宽弘得众,喜怒不形於声色,而外内肃然。

十年卒,子协嗣,统骘所领,加抚军将军。协卒,子玑嗣侯。协弟阐,继业为西陵督,加昭武将军,封西亭侯。凤皇元年,召为绕帐督。阐累世在西陵,卒被徵命,自以失职,又惧有谗祸,於是据城降晋。遣玑与弟璿诣洛阳为任,晋以阐为都督西陵诸军事、卫将军、仪同三司,加侍中,假节领交州牧,封宜都公;玑监江陵诸军事、左将军,加散骑常侍,领庐陵太守,改封江陵侯;璿给事中、宣威将军,封都乡侯。命车骑将军羊祜、荆州刺史杨肇往赴救阐。孙皓使陆抗西行,祜等遁退。抗陷城,斩阐等,步氏泯灭,惟璿绍祀。

颍川周昭著书称步骘及严畯等曰:“古今贤士大夫所以失名丧身倾家害国者,其由非一也,然要其大归,总其常患,四者而已。急论议一也,争名势二也,重朋党三也,务欲速四也。急论议则伤人,争名势则败友,重朋党则蔽主,务欲速则失德,此四者不除,未有能全也。当世君子能不然者,亦比有之,岂独古人乎!然论其绝异,未若顾豫章、诸葛使君、步丞相、严卫尉、张奋威之为美也。论语言'夫子恂恂然善诱人',又曰'成人之美,不成人之恶',豫章有之矣。'望之俨然,即之也温,听其言也厉',使君体之矣。'恭而安,威而不猛',丞相履之矣。学不求禄,心无苟得,卫尉、奋威蹈之矣。此五君者,虽德实有差,轻重不同,至於趣舍大检,不犯四者,俱一揆也。昔丁谞出於孤家,吾粲由於牧竖,豫章扬其善,以并陆、全之列,是以人无幽滞而风俗厚焉。使君、丞相、卫尉三君,昔以布衣俱相友善,诸论者因各叙其优劣。初,先卫尉,次丞相,而后有使君也;其后并事明主,经营世务,出处之才有不同,先后之名须反其初,此世常人决勤薄也。至於三君分好,卒无亏损,岂非古人交哉!又鲁横江昔杖万兵,屯据陆口,当世之美业也,能与不能,孰不愿焉?而横江既亡,卫尉应其选,自以才非将帅,深辞固让,终於不就。后徙九列,迁典八座,荣不足以自曜,禄不足以自奉。至於二君,皆位为上将,穷富极贵。卫尉既无求欲,二君又不称荐,各守所志,保其名好。孔子曰:'君子矜而不争,群而不党。'斯有风矣。又奋威之名,亦三君之次也,当一方之戍,受上将之任,与使君、丞相不异也。然历国事,论功劳,实有先后,故爵位之荣殊焉。而奋威将处此,决能明其部分,心无失道之欲,事无充诎之求,每升朝堂,循礼而动,辞气謇謇,罔不惟忠。叔嗣虽亲贵,言忧其败,蔡文至虽疏贱,谈称其贤。女配太子,受礼若吊,慷忾之趋,惟笃人物,成败得失,皆如所虑,可谓守道见机,好古之士也。若乃经国家,当军旅,於驰骛之际,立霸王之功,此五者未为过人。至其纯粹履道,求不苟得,升降当世,保全名行,邈然绝俗,实有所师。故粗论其事,以示后之君子。”周昭者字恭远,与韦曜、薛莹、华覈并述吴书,后为中书郎,坐事下狱,覈表救之,孙休不听,遂伏法云。

评曰:张昭受遗辅佐,功勋克举,忠謇方直,动不为己;而以严见惮,以高见外,既不处宰相,又不登师保,从容闾巷,养老而已,以此明权之不及策也。顾雍依杖素业,而将之智局,故能究极荣位。诸葛瑾、步骘并以德度规检见器当世,张承、顾邵虚心长者,好尚人物,周昭之论,称之甚美,故详录焉。谭献纳在公,有忠贞之节。休、承脩志,咸庶为善。爱恶相攻,流播南裔,哀哉!

\part{吴书八}
\chapter{张严程阚薛传第八}

张纮字子纲,广陵人。游学京都,还本郡,举茂才,公府辟,皆不就,避难江东。孙策创业,遂委质焉。表为正议校尉,从讨丹杨。策身临行陈,纮谏曰:“夫主将乃筹谟之所自出,三军之所系命也,不宜轻脱,自敌小寇。愿麾下重天授之姿,副四海之望,无令国内上下危惧。”

建安四年,策遣纮奉章至许宫,留为侍御史。少府孔融等皆与亲善。曹公闻策薨,欲因丧伐吴。纮谏,以为乘人之丧,既非古义,若其不克,成雠弃好,不如因而厚之。曹公从其言,即表权为讨虏将军,领会稽太守。曹公欲令纮辅权内附,出纮为会稽东部都尉。

后权以纮为长史,从征合肥。权率轻骑将往突敌,纮谏曰:“夫兵者凶器,战者危事也。今麾下恃盛壮之气,忽强暴之虏,三军之众,莫不寒心,虽斩将搴旗,威震敌场,此乃偏将之任,非主将之宜也。愿抑贲、育之勇,怀霸王之计。”权纳纮言而止。既还,明年将复出军,纮又谏曰:“自古帝王受命之君,虽有皇灵佐於上,文德播於下,亦赖武功以昭其勋。然而贵於时动,乃后为威耳。今麾下值四百之厄,有扶危之功,宜且隐息师徒,广开播殖,任贤使能,务崇宽惠,顺天命以行诛,可不劳而定也。”於是遂止不行。纮建计宜出都秣陵,权从之。令还吴迎家,道病卒。临困,授子靖留笺曰:“自古有国有家者,咸欲脩德政以比隆盛世,至於其治,多不馨香。非无忠臣贤佐,闇於治体也,由主不胜其情,弗能用耳。夫人情惮难而趋易,好同而恶异,与治道相反。传曰‘从善如登,从恶如崩’,言善之难也。人君承奕世之基,据自然之势,操八柄之威,甘易同之欢,无假取於人;而忠臣挟难进之术,吐逆耳之言,其不合也,不亦宜乎!离则有衅,巧辩缘间,眩於小忠,恋於恩爱,贤愚杂错,长幼失叙,其所由来,情乱之也。故明君悟之,求贤如饥渴,受谏而不厌,抑情损欲,以义割恩,上无偏谬之授,下无希冀之望。宜加三思,含垢藏疾,以成仁覆之大。”时年六十卒。权省书流涕。

纮著诗赋铭诔十馀篇。子玄,官至南郡太守、尚书。玄子尚,孙皓时为侍郎,以言语辩捷见知,擢为侍中、中书令。皓使尚鼓琴,尚对曰:“素不能。”敕使学之。后晏言次说琴之精妙,尚因道“晋平公使师旷作清角,旷言吾君德簿,不足以听之。”皓意谓尚以斯喻己,不悦。后积他事下狱,皆追以此为诘,送建安作船。久之,又就加诛。

初,纮同郡秦松字文表,陈端字子正,并与纮见待於孙策,参与谋谟。各早卒。

严畯字曼才,彭城人也。少耽学,善诗、书、三礼,又好说文。避乱江东,与诸葛瑾、步骘齐名友善。性质直纯厚,其於人物,忠告善道,志存补益。张昭进之於孙权,权以为骑都尉、从事中郎。及横江将军鲁肃卒,权以畯代肃,督兵万人,镇据陆口。众人咸为畯喜,畯前后固辞:“朴素书生,不闲军事,非才而据,咎悔必至。”发言慷慨,至於流涕,权乃听焉。世嘉其能以实让。权为吴王,及称尊号,畯尝为卫尉,使至蜀,蜀相诸葛亮深善之。不畜禄赐,皆散之亲戚知故,家常不充。广陵刘颖与畯有旧,颖精学家巷,权闻徵之,以疾不就。其弟略为零陵太守,卒官,颖往赴丧,权知其诈病,急驿收录。畯亦驰语颖,使还谢权。权怒废畯,而颖得免罪。久之,以畯为尚书令,后卒。

畯著孝经传、潮水论,又与裴玄、张承论管仲、季路,皆传於世。玄字彦黄,下邳人也,亦有学行,官至太中大夫。问子钦齐桓、晋文、夷、惠四人优劣,钦答所见,与玄相反覆,各有文理。钦与太子登游处,登称其翰采。

程秉字德枢,汝南南顿人也。逮事郑玄,后避乱交州,与刘熙考论大义,遂博通五经。士燮命为长史。权闻其名儒,以礼徵,秉既到,拜太子太傅。黄武四年,权为太子登娉周瑜女,秉守太常,迎妃於吴,权亲幸秉船,深见优礼。既还,秉从容进说登曰:“婚姻人伦之始,王教之基,是以圣王重之,所以率先众庶,风化天下,故诗美关雎,以为称首。愿太子尊礼教於闺房,存周南之所咏,则道化隆於上,颂声作於下矣。”登笑曰:“将顺其美,匡救其恶,诚所赖於傅君也”

病卒官。著周易摘、尚书駮、论语弼,凡三万馀言。秉为傅时,率更令河南徵崇亦笃学立行云。

阚泽字德润,会稽山阴人也。家世农夫,至泽好学,居贫无资,常为人佣书,以供纸笔,所写既毕,诵读亦遍。追师论讲,究览群籍,兼通历数,由是显名。察孝廉,除钱唐长,迁郴令。孙权为骠骑将军,辟补西曹掾;及称尊号,以泽为尚书。嘉禾中,为中书令,加侍中。赤乌五年,拜太子太傅,领中书如故。

泽以经传文多,难得尽用,乃斟酌诸家,刊约礼文及诸注说以授二宫,为制行出入及见宾仪,又著乾象历注以正时日。每朝廷大议,经典所疑,辄谘访之。以儒学勤劳,封都乡侯。性谦恭笃慎,宫府小吏,呼召对问,皆为抗礼。人有非短,口未尝及,容貌似不足者,然所闻少穷。权尝问:“书传篇赋,何者为美?”泽欲讽喻以明治乱,因对贾谊过秦论最善,权览读焉。初,以吕壹奸罪发闻,有司穷治,奏以大辟,或以为宜加焚裂,用彰元恶。权以访泽,泽曰:“盛明之世,不宜复有此刑。”权从之。又诸官司有所患疾,欲增重科防,以检御臣下,泽每曰“宜依礼、律“,其和而有正,皆此类也。六年冬卒,权痛惜感悼,食不进者数日。

泽州里先辈丹杨唐固亦修身积学,称为儒者,著国语、公羊、谷梁传注,讲授常数十人。权为吴王,拜固议郎,自陆逊、张温、骆统等皆拜之。黄武四年为尚书仆射,卒。

薛综字敬文,沛郡竹邑人也。少依族人避地交州,从刘熙学。士燮既附孙权,召综为五官中郎将,除合浦、交阯太守。时交土始开,刺史吕岱率师讨伐,综与俱行,越海南征,及到九真。事毕还都,守谒者仆射。西使张奉於权前列尚书阚泽姓名以嘲泽,泽不能答。综下行酒,因劝酒曰:“蜀者何也?有犬为独,无犬为蜀,横目苟身,虫入其腹。”奉曰:“不当复列君吴邪?”综应声曰:“无口为天,有口为吴,君临万邦,天子之都。”於是众坐喜笑,而奉无以对。其枢机敏捷,皆此类也。

吕岱从交州召出,综惧继岱者非其人,上疏曰:“昔帝舜南巡,卒於苍梧。秦置桂林、南海、象郡,然则四国之内属也,有自来矣。赵佗起番禺,怀服百越之君,珠官之南是也。汉武帝诛吕嘉,开九郡,设交阯刺史以镇监之。山川长远,习俗不齐,言语同异,重译乃通,民如禽兽,长幼无别,椎结徒跣,贯头左衽,长吏之设,虽有若无。自斯以来,颇徙中国罪人杂居其间,稍使学书,粗知言语,使驿往来,观见礼化。及后锡光为交阯,任延为九真太守,乃教其耕犁,使之冠履;为设媒官,始知聘娶;建立学校,导之经义。由此已降,四百馀年,颇有似类。自臣昔客始至之时,珠崖除州县嫁娶,皆须八月引户,人民集会之时,男女自相可適,乃为夫妻,父母不能止。交阯糜泠、九真都庞二县,皆兄死弟妻其嫂,世以此为俗,长吏恣听,不能禁制。日南郡男女倮体,不以为羞。由此言之,可谓虫豸,有靦面目耳。然而土广人众,阻险毒害,易以为乱,难使从治。县官羁縻,示令威服,田户之租赋,裁取供办,贵致远珍名珠、香药、象牙、犀角、玳瑁、珊瑚、琉璃、鹦鹉、翡翠、孔雀、奇物、充备宝玩,不必仰其赋入,以益中国也。然在九甸之外,长吏之选,类不精覈。汉时法宽,多自放恣,故数反违法。珠崖之废,起於长吏睹其好发,髡取为髲。及臣所见,南海黄盖为日南太守,下车以供设不丰,挝杀主簿,仍见驱逐。九真太守儋萌为妻父周京作主人,并请大吏,酒酣作乐,功曹番歆起舞属京,京不肯起,歆犹迫强,萌忿杖歆,亡於郡内。歆弟苗帅众攻府,毒矢射萌,萌至物故。交阯太守士燮遣兵致讨,卒不能克。又故刺史会稽朱符,多以乡人虞褒、刘彦之徒分作长吏,侵虐百姓,强赋於民,黄鱼一枚收稻一斛,百姓怨叛,山贼并出,攻州突郡。符走入海,流离丧亡。次得南阳张津,与荆州牧刘表为隙,兵弱敌强,岁岁兴军,诸将厌患,去留自在。津小检摄,威武不足,为所陵侮,遂至杀没。后得零陵赖恭,先辈仁谨,不晓时事。表又遣长沙吴巨为苍梧太守。巨武夫轻悍,不为恭所服,辄相怨恨,逐出恭,求步骘。是时津故将夷廖、钱博之徒尚多,骘以次鉏治,纲纪適定,会仍召出。吕岱既至,有士氏之变。越军南征,平讨之日,改置长吏,章明王纲,威加万里,大小承风。由此言之,绥边抚裔,实有其人。牧伯之任,既宜清能,荒流之表,祸福尤甚。今日交州虽名粗定,尚有高凉宿贼;其南海、苍梧、郁林、珠官四郡界未绥,依作寇盗,专为亡叛逋逃之薮。若岱不复南,新刺史宜得精密,检摄八郡,方略智计,能稍稍以渐治高凉者,假其威宠,借之形势,责其成效,庶几可补复。如但中人,近守常法,无奇数异术者,则群恶日滋,久远成害。故国之安危,在於所任,不可不察也。窃惧朝廷忽轻其选,故敢竭愚情,以广圣思。”

黄龙三年,建昌侯虑为镇军大将军,屯半州,以综为长史,外掌众事,内授书籍。虑卒,入守贼曹尚书,迁尚书仆射。时公孙渊降而复叛,权盛怒,欲自亲征。综上疏谏曰:“夫帝王者,万国之元首,天下之所系命也。是以居则重门击柝以戒不虞,行则清道案节以养威严,盖所以存万安之福,镇四海之心。昔孔子疾时,讬乘桴浮海之语,季由斯喜,拒以无所取才。汉元帝欲御楼船,薛广德请刎颈以血染车。何则?水火之险至危,非帝王所宜涉也。谚曰:‘千金之子,坐不垂堂。’况万乘之尊乎?今辽东戎貊小国,无城池之固,备御之术,器械铢钝,犬羊无政,往必禽克,诚如明诏。然其方土寒确,谷稼不殖,民习鞍马,转徙无常。卒闻大军之至,自度不敌,鸟惊兽骇,长驱奔窜,一人匹马,不可得见,虽获空地,守之无益,此不可一也。加又洪流滉瀁,有成山之难,海行无常,风波难免,倏忽之间,人船异势。虽有尧舜之德,智无所施,贲育之勇,力不得设,此不可二也。加以郁雾冥其上,咸水蒸其下,善生流肿,转相洿染,凡行海者,稀无斯患,此不可三也。天生神圣,显以符瑞,当乘平丧乱,康此民物;嘉祥日集,海内垂定,逆虏凶虐,灭亡在近。中国一平,辽东自毙,但当拱手以待耳。今乃违必然之图,寻至危之阻,忽九州之固,肆一朝之忿,既非社稷之重计,又开辟以来所未尝有,斯诚群僚所以倾身侧息,食不甘味,寝不安席者也。惟陛下抑雷霆之威,忍赫斯之怒,遵乘桥之安,远履冰之险,则臣子赖祉,天下幸甚。“时群臣多谏,权遂不行。

正月乙未,权敕综祝祖不得用常文,综承诏,卒造文义,信辞粲烂。权曰:“复为两头,使满三也。”综复再祝,辞令皆新,众咸称善。赤乌三年,徙选曹尚书。五年,为太子少傅,领选职如故。六年春,卒。凡所著诗赋难论数万言,名曰私载,又定五宗图述、二京解,皆传於世。

子珝,宫至威南将军,征交阯还,道病死。珝弟莹,字道言,初为秘府中书郎,孙休即位,为散骑中常侍。数年,以病去官。孙皓初,为左执法,迁选曹尚书,及立太子,又领少傅。建衡三年,皓追叹莹父综遗文,且命莹继作。莹献诗曰:“惟臣之先,昔仕于汉,奕世绵绵,颇涉台观。暨臣父综,遭时之难,卯金失御,邦家毁乱。適兹乐土,庶存孑遗,天启其心,东南是归。厥初流隶,困于蛮垂。大皇开基,恩德远施。特蒙招命。拯擢泥汙,释放巾褐,受职剖符。作守合浦,在海之隅,迁入京辇,遂升机枢。枯瘁更荣,绝统复纪,自微而显,非愿之始。亦惟宠遇,心存足止。重值文皇,建号东宫,乃作少傅,光华益隆。明明圣嗣,至德谦崇,礼遇兼加,惟渥惟丰。哀哀先臣,念竭其忠,洪恩未报,委世以终。嗟臣蔑贱,惟昆及弟,幸生幸育,讬综遗体。过庭既训,顽蔽难启。堂构弗克,志存耦耕。岂悟圣朝,仁泽流盈。追录先臣,愍其无成,是济是拔,被以殊荣。珝忝千里,受命南征,旌旗备物,金革杨声。及臣斯陋,实闇实微,既显前轨,人物之机;复傅东宫,继世荷辉,才不逮先,是忝是违。乾德博好,文雅是贵,追悼亡臣,冀存遗类。如何愚胤,曾无仿佛!瞻彼旧宠,顾此顽虚,孰能忍愧,臣实与居。夙夜反侧,克心自论,父子兄弟,累世蒙恩,死惟结草,生誓杀身,虽则灰陨,无报万分。”

是岁,何定建议凿圣谿以通江淮,皓令莹督万人往,遂以多盘石难施功,罢还,出为武昌左部督。后定被诛,皓追圣谿事,下莹狱,徙广州。右国史华覈上疏曰:“臣闻五帝三王皆立史官,叙录功美,垂之无穷。汉时司马迁、班固,咸命世大才,所撰精妙,与六经俱传。大吴受命,建国南土。大皇帝末年,命太史令丁孚、郎中项峻始撰吴书。孚、峻俱非史才,其所撰作,不足纪录。至少帝时,更差韦曜、周昭、薛莹、梁广及臣五人,访求往事,所共撰立,备有本末。昭、广先亡,曜负恩蹈罪,莹出为将,复以过徙,其书遂委滞,迄今未撰奏。臣愚浅才劣,適可为莹等记注而已,若使撰合,必袭孚、峻之迹,惧坠大皇帝之元功,损当世之盛美。莹涉学既博,文章尤妙,同寮之中,莹为冠首。今者见吏,虽多经学,记述之才,如莹者少,是以慺慺为国惜之。实欲使卒垂成之功,编於前史之末。奏上之后,退填沟壑,无所复恨。”皓遂召莹还,为左国史。顷之,选曹尚书同郡缪祎以执意不移,为群小所疾,左迁衡阳太守。既拜,又追以职事见诘责,拜表陈谢。因过诣莹,复为人所白,云祎不惧罪,多将宾客会聚莹许。乃收祎下狱,徙桂阳,莹还广州。未至,召莹还,复职。是时法政多谬,举措烦苛,莹每上便宜,陈缓刑简役,以济育百姓,事或施行。迁光禄勋。天纪四年,晋军征皓,皓奉书於司马伷、王浑、王濬请降,其文,莹所造也。莹既至洛阳,特先见叙,为散骑常侍,答问处当,皆有条理。太康三年卒。著书八篇,名曰新议。

评曰:张纮文理意正,为世令器,孙策待之亚於张昭,诚有以也。严、程、阚生,一时儒林也。至畯辞荣济旧,不亦长者乎!薛综学识规纳,为吴良臣。及莹纂蹈,允有先风,然於暴酷之朝,屡登显列,君子殆诸。

\part{吴书九}

\chapter{周瑜鲁肃吕蒙传第九}


周瑜字公瑾,庐江舒人也。从祖父景,景子忠,皆为汉太尉。父异,洛阳令。瑜长壮有姿貌。初,孙坚与义兵讨董卓,徙家于舒。坚子策兴瑜同年,独相友善,瑜推道南大宅以舍策,升堂拜母,有无通共。瑜从父尚为丹杨太守,瑜往省之。会策将东渡,到历阳,驰书报瑜,瑜将兵迎策。策大喜曰:“吾得卿。谐也。”遂从攻横江、当利,皆拔之。乃渡江击秣陵,破笮融、薛礼。转下湖孰、江乘,进入曲阿。刘繇奔走,而策之众已数万矣。因谓瑜曰:“吾以此众取吴会平山越已足。卿还镇丹杨。”瑜还。顷之,袁术遣从弟胤代尚为太守,而瑜与尚俱还寿春。术欲以瑜为将,瑜观术终无所成,故求为居巢长,欲假涂东归,术听之。遂自居巢还吴。是岁,建安三年也。策亲自迎瑜,授建威中郎将,即与兵二千人,骑五十匹。瑜时年二十四,吴中皆呼为周郎。以瑜恩信著于庐江,出备牛渚,后领春谷长。顷之,策欲取荆州,以瑜为中护军,领导江夏太守,从攻皖,拔之。时得桥公两女,皆国色也。策自纳大桥,瑜纳小桥。复近寻阳,破刘勋,讨江夏,还定豫章、庐陵,留镇巴丘。

五年,策薨。权统事。瑜将兵赴丧,遂留吴,以中护军与长史张昭共掌众事。十一年,督孙瑜等讨麻、保二屯,枭其渠帅,囚俘万余口,还备(官亭)。江夏太守黄祖遣将邓龙将兵数千人入柴桑,瑜追讨击,生虏龙送吴。十三年春,权讨江夏,瑜为前部大督。其年九月,曹公入荆州,刘琮举众降,曹公得其水军,船步兵数十万,将士闻之皆恐。权延见群下,问以计策。议者咸曰:“曹公豺虎也,然托名汉相,挟天子以征四方,动以朝廷为辞,今日拒之,事更不顺,且将军大势可以拒操者,长江也。今操得荆州,奄有其地。刘表治水军,蒙冲斗舰,乃以千数,操悉浮以沿江,兼有步兵,水陆俱下。此为长江之险,已与我共之矣。而势力众寡,又不可论。愚谓大计不如迎之。”瑜曰:“不然。操虽托名汉相,其实汉贼也。将军以神武雄才,兼仗父兄之烈,割据江东,地方数千里,兵精足用,英雄乐业,尚当横行天下,为汉家除残去秽。况操自送死,而可迎之耶?请为将军筹之:今使北土已安,操无内忧,能旷日持久,来争疆场,又能与我校胜负于船楫,(可)乎?今北土既未平安,加马超、韩遂尚在关西,为操后患。且舍鞍马,仗舟揖,与吴越争衡,本非中国所长。又今盛寒,马无藁草。驱中国士众远涉江湖之间,不习水土,必生疾病。此数四者,用兵之患也,而操皆冒行之。将军擒操,宜在今日。瑜请得精兵三万人,进住夏口,保为将军破之。”权曰:“老贼欲废汉自立久矣,陡忌二袁、吕布、刘表与孤耳。今数雄已灭,惟孤尚存,孤与老贼,势不两立。君言当击,甚与孤台,此天以君授孤也。

时刘备为曹公所破,欲引南渡江。与鲁肃遇于当阳,遂共图计,因进住夏口,遣诸葛亮诣权。权遂遣瑜及程普等与备并力逆曹公,遇于赤壁。时曹公军众已有疾病,初一交战,公军败退,引次江北。瑜等在南岸。瑜部将黄盖曰:“今寇众我寡,难与持久。然观操军船舰,首尾相接,可烧而走也。”乃取蒙冲斗舰数十艘,实以薪草,膏油灌其中。裹以帷幕,上建牙旗,先书报曹公,欺以欲降。又豫备走舸,各系大船后,因引次俱前。曹公军吏士皆延颈观望,指言盖降。盖放诸船,同时发火。时风盛猛,悉延烧岸上营落。顷之。烟炎张天,人马烧溺死者甚众,军遂败退,还保南郡。备与瑜等复共追。曹公留曹仁等守江陵城。径自北归。

瑜与程普又进南郡,与仁相对,各隔大江。兵未交锋,瑜即遣甘宁前据夷陵。仁分兵骑别攻围宁。宁告急于瑜。瑜用吕蒙计,留淩统以守其后,身与蒙上救宁。宁围既解,乃渡屯北岸,克期大战。瑜亲跨马擽陈,会流矢中右胁,疮甚,便还。后仁闻瑜卧未起,勒兵就陈。瑜乃自兴,案行军营,激扬吏士,仁由是遂退。

权拜瑜偏将军,领南郡太守。以下隽、汉昌、刘阳、州陵为奉邑,屯据江陵。刘备以左将军领荆州牧,治公安,备诣京见权,瑜上疏曰:“刘备以枭雄之姿,而有关羽、张飞熊虎之将,必非久屈为人用者。愚谓大计宜徙备置吴,盛为筑宫室,多其美女玩好,以娱其耳目,分此二人,各置一方,使如瑜者得挟与攻战,大事可定也。今猥割土地以资业之,聚此三人,俱在疆场,恐蛟龙得云雨,终非池中物也。”权以曹公在北方,当广揽英雄,又恐备难卒制,故不纳。是时刘璋为益州牧。外有张鲁寇侵,瑜乃诣京见权曰:“今曹操新折衄,方忧在腹心,未能与将军连兵相事也。乞与奋威俱进取蜀,得蜀而并张鲁,因留奋威固守其地,好与马超结援。瑜还与将军据襄阳以蹙操,北方可图也。”权许之。瑜还江陵为行装,而道于马丘病卒,时年三十六。权素服举哀。感动左右。丧当还吴,又迎之芜湖,众事费度,一为供给。后著令曰:“故将军周瑜、程普,其有人客,皆不得问。”初瑜见友于策,太妃又使权以兄奉之。是时权位为将军,诸将宾客为礼尚简,而瑜独先尽敬,便执臣节。性度恢廓,大率为得人,惟与程普不睦。

\begin{yuanwen}
瑜少精意于音乐。虽三爵之后,其有阙误。瑜必知之,知之必顾,故时人谣曰:“曲有误,周郎顾。”
\end{yuanwen}

瑜两男一女,女配太子登。男循尚公主,拜骑都尉,有瑜风,早卒。循弟胤,初拜兴业都尉。妻以宗女,授兵千人,屯公安。黄龙元年,封都乡侯,后以罪徙庐陵郡。赤乌二年,诸葛瑾、步骘连名上疏曰:“故将军周瑜子胤,昔蒙粉饰,受封为将,不能养之以福,思立功效,至纵情欲,招速罪辟。臣窃以瑜昔见宠任,入作心膂,出为爪牙,衔命出征,身当矢石,尽节用命,视死如归。故能摧曹操于乌林,走曹仁于郢都,扬国威德,华夏是震,蠢尔蛮荆,莫不宾服。虽周之方叔,汉之信、布,诚无以尚也。夫折冲扦难之臣,自古帝王莫不贵重,故汉高帝封爵之誓曰‘使黄河如带,太山如砺,国以永存,爰及苗裔’。申以丹书,重以盟诅,藏于宗庙,传于无穷,欲使功臣之后,世世相踵,非徒子孙,乃关苗裔,报德明功,勤勤恳恳,如此之至,欲以劝戒后人,用命之臣,死而无悔也。况于瑜身没未久,而其子胤降为匹夫,益可悼伤。窃惟陛下钦明稽古,隆于兴继,为胤归诉,乞丐余罪,还兵复爵,使失旦之鸡,复得一鸣。抱罪之臣,展其后效。”权答曰:“腹心旧勋,与孤协事,公瑾有之,诚所不忘。昔胤年少,初无功劳,横受精兵,爵以侯将,盖念公瑾以及于胤也。而胤恃此,酗淫自恣,前后告喻,曾无悛改。孤于公瑾,义犹二君,乐胤成就,岂有已哉?迫胤罪恶,未宜便还,且欲苦之,使自知耳。今二君勤勤援引汉高河山之誓,孤用恧然。虽德非其畴,犹欲庶几,事亦如尔,故未顺旨。以公瑾之子,而二君在中间,苟使能改,亦何患乎!”瑾、骘表比上,朱然及全琮亦俱陈乞,权乃许之。会胤病死。

瑜兄子峻,亦以瑜元功为偏将军,领吏士千人。峻卒,全琮表峻子护为将。权曰:“昔走曹操,拓有荆州,皆是公瑾,常不忘之。初闻峻亡,仍欲用护,闻护性行危险,用之适为作祸,故便止之。孤念公瑾,岂有已乎?”

\begin{yuanwen}
鲁肃字子敬,临淮东城人也。生而失父,与祖母居。家富于财,性好施与。尔时天下已乱,肃不治家事,大散财货,摽卖田地,以赈穷弊结士为务,甚得乡邑欢心。
\end{yuanwen}

\begin{yuanwen}
周瑜为居巢长,将数百人故过候肃,并求资粮。肃家有两囷米,各三千斛。肃乃指一囷与周瑜,瑜益知其奇也。遂相亲结,定侨、札之分。袁术闻其名,就署东城长。肃见术无纲纪,不足与立事,乃携老弱将轻侠少年百余人,南到居巢就瑜。瑜之东渡,因与同行,留家曲阿。会祖母亡,还葬东城。
\end{yuanwen}

\begin{yuanwen}
刘子扬与肃友善,遗肃书曰:“方今天下豪杰并起,吾子姿才,尤宜今日。急还迎老母,无事滞于东城。近郑宝者,今在巢湖,拥众万余,处地肥饶,庐江间人多依就之,况吾徒乎?观其形势,又可博集,时不可失,足下速之。”

肃答然其计。葬毕还曲阿,欲北行。会瑜已徙肃母到吴,肃具以状语瑜。时孙策已薨,权尚住吴,瑜谓肃曰:“昔马援答光武云‘当今之世,非但君择臣,臣亦择君’。今主人亲贤贵士,纳奇录异,且吾闻先哲秘论,承运代刘氏者,必兴于东南,推步事势,当其历数,终构帝基,以协天符,是烈士攀龙附凤驰骛之秋。吾方达此,足下不须以子扬之言介意也。”

肃从其言。瑜因荐肃才宜佐时,当广求其比,以成功业,不可令去也。
\end{yuanwen}

\begin{yuanwen}
权即见肃,与语甚悦之。众宾罢退,肃亦辞出,乃独引肃还,合榻对饮。因密议曰:“今汉室倾危,四方云扰,孤承父兄余业,思有桓文之功。君既惠顾,何以佐之?”

肃对曰:“昔高帝区区欲尊事义帝而不获者,以项羽为害也。今之曹操,犹昔项羽,将军何由得为桓文乎?肃窃料之,汉室不可复兴,曹操不可卒除。为将军计,惟有鼎足江东,以观天下之衅。规模如此,亦自无嫌。何者?北方诚多务也。因其多务,剿除黄祖,进伐刘表,竟长江所极,据而有之,然后建号帝王以图天下,此高帝之业也。”

权曰:“今尽力一方,冀以辅汉耳,此言非所及也。”

张昭非肃谦下不足,颇訾毁之,云肃年少粗疏,未可用。极不以介意,益贵重之,赐肃母衣服帏帐,居处杂物,富拟其旧。
\end{yuanwen}

\begin{yuanwen}
刘表死,肃进说曰:“夫荆楚与国邻接,水流顺北,外带江汉,内阻山陵,有金城之固,沃野万里,士民殷富,若据而有之,此帝王之资也。今表新亡,二子素不辑睦,军中诸将,各有彼此。加刘备天下枭雄,与操有隙,寄寓于表,表恶其能而不能用也。若备与彼协心,上下齐同,则宜抚安,与结盟好:如有离违,宜别图之,以济大事。肃请得奉命吊表二子,并慰劳其军中用事者,及说备使抚表众,同心一意,共治曹操,备必喜而从命。如其克谐,天下可定也。今不速往,恐为操所先。”权即遣肃行。
\end{yuanwen}

\begin{yuanwen}
到夏口,闻曹公已向荆州,晨夜兼道。比至南郡,而表子琮已降曹公,备惶遽奔走,欲南渡江。肃径迎之,到当阳长阪,与备会,宣腾权旨,及陈江东强固,劝备与权并力。备甚欢悦。时诸葛亮与备相随,肃谓亮曰“我子瑜友也”,即共定交。备遂到夏口,遣亮使权,肃亦反命。
\end{yuanwen}

\begin{yuanwen}
会权得曹公欲东之问,与诸将议,皆劝权迎之,而肃独不言。权起更衣,肃追于宇下,权知其意,执肃手曰:“卿欲何言?”

肃对曰:“向察众人之议,专欲误将军,不足与图大事。今肃可迎操耳,如将军,不可也。何以言之?今肃迎操,操当以肃还付乡党。品其名位,犹不失下曹从事,乘犊车、从吏卒、交游士林、累官故不失州郡也。将军迎操,欲安所归?愿早定大计,莫用众人之议也。”

权叹息曰:“此诸人持议,甚失孤望;今卿廓开大计,正与孤同,此天以卿赐我也。”
\end{yuanwen}

\begin{yuanwen}
时周瑜受使至鄱阳,肃劝追召瑜还。遂任瑜以行事,以肃为赞军校尉,助画方略。曹公破走,肃即先还,权大请诸将迎肃。肃将入閤拜,权起礼之,因谓曰:“子敬,孤持鞍下马相迎,足以显卿未?”

肃趋近曰:“未也。”

众人闻之,无不愕然。就坐,徐举鞭言曰:“愿至尊威德加乎四海,总括九州,克成帝业,更以安车软轮征肃,始当显耳。”权抚掌欢笑。
\end{yuanwen}

\begin{yuanwen}
后备诣京见权,求都督荆州,惟肃劝权借之,共拒曹公。曹公闻权以土地业备,方作书,落笔于地。
\end{yuanwen}

\begin{yuanwen}
周瑜病困,上疏曰:“当今天下,方有事役,是瑜乃心夙夜所忧,愿至尊先虑未然,然后康乐。今既与曹操为敌,刘备近在公安,边境密迩,百姓未附,宜得良将以镇抚之。鲁肃智略足任,乞以代瑜。瑜陨踣之日,所怀尽矣。”

即拜肃奋武校尉,代瑜领兵。瑜士众四千余人。奉邑四县,皆属焉。令程普领南郡太守。肃初住江陵,后下屯陆口,威恩大行,众增万余人,拜汉昌太守、偏将军。十九年,从权破皖城,转横江将军。
\end{yuanwen}

\begin{yuanwen}
先是,益州牧刘璋纲维颓弛。周瑜、甘宁并劝权取蜀,权以咨备,备内欲自规,仍伪报曰:“备与璋托为宗室,冀凭英灵,以匡汉朝。今璋得罪左右,备独竦惧,非所敢闻,愿加宽贷。若不获请,备当放发归于山林。”

后备西图璋,留关羽守。权曰:“猾虏乃敢挟诈!”

及羽与肃邻界,数生狐疑,疆埸纷错,肃常以欢好抚之。备既定益州,权求长沙、零、桂,备不承旨,权遣吕蒙率众进取。备闻,自还公安,遣羽争三郡。肃住益阳,与羽相拒。肃邀羽相见,各驻兵马百步上,但诸将军单刀俱会。肃因责数羽曰:“国家区区本以土地借卿家者,卿家军败远来,无以为资故也。今已得益州,既无奉还之意,但求三郡,又不从命。”

语未究竟,坐有一人曰:“夫土地者,惟德所在耳,何常之有!”

肃厉声呵之,辞色甚切。羽操刀起谓曰:“此自国家事,是人何知!”目使之去。备遂割湘水为界,于是罢军。
\end{yuanwen}

\begin{yuanwen}
肃年四十六,建安二十二年卒。权为举哀,又临其葬。诸葛亮亦为发哀。权称尊号,临坛,顾谓公卿曰:“昔鲁子敬尝道此,可谓明于事势矣。”
\end{yuanwen}

肃遣腹子淑既壮,濡须督张承谓终当到至。永安中,为昭武将军、都亭侯、武昌督。建衡中,假节,迁夏口督。所在严整,有方干。凤皇三年卒。子睦袭爵,领兵马。

吕蒙字子明,汝南富陂人也。少南渡,依姊夫邓当。当为孙策将,数讨山越。蒙年十五六,窃随当击贼,当顾见大惊,呵叱不能禁止。归以告蒙母,母恚欲罚之,蒙曰:“贫贱难可居,脱误有功,富贵可致。旦不探虎穴,安得虎子?”母哀而舍之。时当职吏以蒙年小轻之,曰:“彼坚子何能为?此欲以肉喂虎耳。”他日与蒙会,又蚩辱之。蒙大怒,引刀杀吏,出走,逃邑子郑长家。出因校尉袁雄自首,承间为言,策召见奇之,引置左右。数岁,邓当死,张昭荐蒙代当,拜别部司马。权统事,料诸小将兵少而用薄者,欲并合之。蒙阴赊贳,为兵作绛衣行滕,及简日,陈列赫然,兵人练习,权见之大悦,增其兵。从讨丹杨,所向有功,拜平北都尉,领广德长。从征黄祖,祖令都督陈就逆以水军出战。蒙勒前锋,亲枭就首,将士乘胜,进攻其城。祖闻就死,委城走,兵追禽之。权曰:“事之克,由陈就先获也。”以蒙为横野中郎将,赐钱千万。

是岁,又与周瑜、程普等西破曹公于乌林,围曹仁于南郡。益州将袭肃举军来附,瑜表以肃兵益蒙,蒙盛称肃有胆用。且慕化远来,于义宜益不宜夺也。权善其言,还肃兵。瑜使甘宁前据夷陵,曹仁分众围宁,宁困急,使使请救。诸将以兵少不足分,蒙谓瑜、普曰:“留凌公绩,蒙与君行,解围释急,势亦不久,蒙保公绩能十日守也。”又说瑜分遣三百人柴断险道,贼走可得其马。瑜从之。军到夷陵,即日交战,所杀过半。敌夜遁去,行遇柴道,骑皆舍马步走。兵追蹙击,获马三百匹,方船载还。于是将士形势自倍,乃渡江立屯,与相攻击,曹仁退走。遂据南郡,抚定荆州。还,拜偏将军,领寻阳令。

鲁肃代周瑜,当之陆口,过蒙屯下。肃意尚轻蒙,或说肃曰:“吕将军功名日显,不可以故意待也,君宜顾之。”遂往诣蒙。酒酣,蒙问肃曰:“君受重任,与关羽为邻,将何计略以备不虞?”肃造次应曰:“临时施宜。”蒙曰:“今东西虽为一家,而关羽实熊虎也,计安可不豫定?”因为肃画五策。肃于是越席就之,拊其背曰:“吕子明,吾不知卿才略所及乃至于此也。”遂拜蒙母,结友而别。时蒙与成当、宋定、徐顾屯次比近,三将死,子弟幼弱,权悉以兵并蒙。蒙固辞,陈启顾等皆勤劳国事,子弟虽小,不可废也。书三上,权乃听。蒙于是又为择师,使辅导之,其操心率如此。

魏使庐江谢奇为蕲春典农,屯皖田乡,数为边寇。蒙使人诱之,不从,则伺隙袭击,奇遂缩退,其部伍孙子才、宋豪等,皆携负老弱,诣蒙降。后从权拒曹公于濡须,数近奇计,又劝权夹水口立坞,所以备御甚精,曹公不能下而退。

曹公遣朱光为庐江太守,屯皖,大开稻田,又令间人招诱鄱阳贼帅,使作内应。蒙曰:“皖田肥美,若一收孰,彼众必增,如是数岁,操态见矣,宜早除之。”乃具陈其状。于是权亲征皖,引见诸将,问以计策。蒙乃荐甘宁为升城督,督攻在前,蒙以精锐继之。侵晨进攻,蒙手执枹鼓,士卒皆腾踊自升,食时破之。既而张辽至夹石,闻城已拔,乃退。权嘉其功,即拜庐江太守,所得人马皆分与之,别赐寻阳屯田六百户,官属三十人。蒙还寻阳,未期而卢陵贼起,诸将讨击不能禽,权曰:“鸷鸟累百,不如一鹗。”复令蒙讨之。蒙至,诛其首恶,余皆释放,复为平民。

是时刘备令关羽镇守,专有荆士,权命蒙西取长沙、零、桂三郡。蒙移书二郡,望风归服,惟零陵太守郝普城守不降。而备自蜀亲至公安,遣羽争三郡。权时住陆口,使鲁肃将万人屯益阳拒羽,而飞书召蒙,使舍零陵,急还助肃。初,蒙既定长沙,当之零陵,过酃,载南阳邓玄之,玄之者郝普之旧也,欲令诱普。及被书当还,蒙秘之。夜召诸将,授以方略,晨当攻城。顾谓玄之曰:“郝子太闻世间有忠义事,亦欲为之,而不知时也。左将军在汉中,为夏侯渊所围。关羽在南郡,今至尊身自临之。近者破樊本屯,救酃,逆为孙规所破。此皆目前之事,君所亲见也。彼方首尾倒悬,救死不给,岂有余力复营此哉?今吾士卒精锐,人思致命。至尊遣兵,相继于道。今子太以旦夕之命,待不可望之救。犹牛蹄中鱼,冀赖江汉,其不可恃亦明矣。若子太必能一士卒之心,保孤城之守,尚能稽延旦夕,以待所归者,可也。今吾计力度虑,而以攻此,曾不移日,而城必破,城破之后,身死何益于事,而令百岁老母,戴白受诛,岂不痛哉?度此家不得外问,谓援可恃,故至于此耳。君可见之,为陈祸福。”玄之见普,具宣蒙意,普惧而听之。玄之先出报蒙:“普寻后当至。”蒙豫敕四将,各选百人,普出,便入守城门。须臾普出,蒙迎执其手,与俱下船。语毕,出书示之。因拊手大笑。普见书,知备在公安,而羽在益阳,惭恨入地。蒙留(孙河),委以后事,即日引军赴益阳。刘备请盟,权乃归普等。割湘水,以零陵还之。以寻阳、阳新为蒙奉邑。

师还,遂征合肥,既撤兵,为张辽等所袭,蒙与淩统以死扦卫。后曹公又大出濡须,权以蒙为督,据前所立坞,置强弩万张于其上,以拒曹公。曹公前锋屯未就,蒙攻破之,曹公引退。拜蒙左护军、虎威将军。

鲁肃卒,蒙西屯陆口,肃军人马万余尽以属蒙。又拜汉昌太守,食下隽、刘阳、汉昌、州陵。与关羽分土接境,知羽骁雄,有并兼心,且居国上流,其势难久。初,鲁肃等以为曹公尚存,祸难始构,宜相辅协,与之同仇,不可失也。蒙乃密陈计策曰:“今令征虏守南郡,潘璋住白帝,蒋钦将游兵万人循江上下,应敌所在,蒙为国家前据襄阳,如此,何忧于操,何赖于羽?且羽君臣,矜其诈力,所在反复,不可以腹心待也。今羽所以未便东向者,以至尊圣明,蒙等尚存也。今不于强壮时图之,一日僵仆,欲复陈力,其可得邪?”权深纳其策,又聊复与论取徐州意。蒙对曰:“今操远在河北,新破诸袁,抚集幽、冀,未暇东顾。徐土守兵,闻不足言,往自可克。然地势陆通,骁骑所聘,至尊今日得徐州,操后旬必来争,虽以七八万人守之,犹当怀忧。不如取羽,全据长江,形势益张。”权尤以此言为当。及蒙代肃,初至陆口,外倍修恩厚,与羽结好。

后羽讨樊,留兵将备公安、南郡。蒙上疏曰:“羽讨樊而多留备兵,必恐蒙图其后故也。蒙常有病,乞分士众还建业,以治疾为名。羽闻之,必撤备兵,尽赴襄阳。大军浮江,昼夜驰上,袭其空虚,则南郡可下,而羽可擒也。”遂称病笃,权乃露檄召蒙还,阴与图计。羽果信之,稍撤兵以赴樊。魏使于禁救樊,羽尽擒禁等,人马数万,托以粮乏,擅取湘关米。权闻之,遂行。先遣蒙在前。蒙至寻阳,尽伏其精兵舳舻中,使白衣摇橹,作商贾人服,昼夜兼行,至羽所置江边屯候,尽收缚之,是故羽不闻知。遂到南郡,士仁、麋芳皆降。蒙入据城,尽得羽及将士家属,皆抚慰,约令军中不得干历人家,有所求取。蒙麾下士,是汝南人,取民家一笠,以覆官铠,官铠虽公,蒙犹以为犯军令,不可以乡里故而废法,遂垂涕斩之。于是军中震慄,道不拾遗。蒙旦暮使亲近存恤耆老,问所不足,疾病者给医药,饥寒者赐衣粮。羽府藏财宝,皆封闭以待权至。羽还,在道路,数使人与蒙相闻,蒙辄厚遇其使,周游城中,家家致问,或手书示信。羽人还,私相参讯,咸知家门无恙,见待过于平时,故羽吏士无斗心。会权寻至,羽自知孤穷,乃走麦城,西至漳乡,众皆委羽而降。权使朱然、潘璋断其径路,即父子俱获,荆州遂定。

以蒙为南郡太守,封孱陵候,赐钱一亿,黄金五百斤。蒙固辞金钱,权不许。封爵未下。会蒙疾发,权时在公安,迎置内殿。所以治护者万方,募封内有能愈蒙疾者,赐千金。时有针加,权为之惨戚,欲数见其颜色,又恐劳动,常穿壁瞻之,见小能下食则喜,顾左右言笑,不然则咄唶,夜不能寐。病中瘳,为下赦令,群臣毕贺。后更增笃,权自临视,命道士于星辰下为之请命。年四十二,遂卒于内殿。时权哀痛甚,为之降损。蒙未死时,所得金宝诸赐尽付府藏,敕主者命绝之日皆上还,丧事务约。权闻之,益以悲感。

蒙少不修书传,每陈大事,常口占为笺疏。常以部曲事为江夏太守蔡遗所白,蒙无恨意。及豫章太守顾邵卒,权问所用,蒙因荐遗奉职佳吏,权笑曰:“君欲为祁奚耶?”于是用之。甘宁粗暴好杀,既常失蒙意,又时违权令,权怒之,蒙辄陈请:“天下未定,斗将如宁难得,宜容忍之。”权遂厚宁,卒得其用。蒙子霸袭爵,与守冢三百家,复田五十顷。霸卒,兄琮袭候。琮卒,弟睦嗣。

孙权与陆逊论周瑜、鲁肃及蒙曰:“公瑾雄烈,胆略兼人,遂破孟德,开拓荆州,邈焉难继,君今继之。公瑾昔要子敬来东,致达于孤,孤与宴语,便及大略帝王之业,此一快也。后孟德因获刘琮之势,张言方率数十万众水步俱下。孤普请诸将,咨问所宜,无适先对,至子布、文表,俱言宜遣使修檄迎之,子敬即驳言不可,劝孤急呼公瑾,付任以众,逆而击之,此二快也。且其决计策意,出张、苏远矣。后虽劝吾借玄德地,是其一短,不足以损其二长也。周公不求备于一人,故孤忘其短而贵其长,常以比方邓禹也。又子明少时,孤谓不辞剧易,果敢有胆而已。及身长大,学问开益,筹略奇至,可以次于公瑾,但言议英发不及之耳。图取关羽,胜于子敬。子敬答孤书云:‘帝王之起,皆有驱除,羽不足忌。’此子敬内不能办,外为大言耳,孤亦恕之,不苟责也。然其作军屯营,不失令行禁止,部界无废负,路无拾遗,其法亦美也。”

评曰:曹公乘汉相之资,挟天子而扫群桀,新荡荆城,仗威东夏,于时议者莫不疑贰。周瑜、鲁肃建独断之明出众人之表,实奇才也。吕蒙勇而有谋,断识军计,谲郝普,禽关羽,最其妙者。初虽轻果妄杀,终于克己,有国士之量,岂徒武将而已乎!孙权之论,优劣允当,故载录焉。

\part{吴书十}
\chapter{程黄韩蒋周陈董甘凌徐潘丁传第十}

程普字德谋,右北平土垠人也。初为州郡吏,有容貌计略,善於应对。从孙坚征伐,讨黄巾於宛、邓,破董卓於阳人,攻城野战,身被创夷。

坚薨,复随孙策在淮南,从攻庐江,拔之,还俱东渡。策到横江、当利,破张英、于麋等,转下秣陵、湖孰、句容、曲阿,普皆有功,增兵二千,骑五十匹。进破乌程、石木、波门、陵传、馀杭,普功为多。策入会稽,以普为吴郡都尉,治钱唐。后徙丹杨都尉,居石城。复讨宣城、泾、安吴、陵阳、春谷诸贼,皆破之。策尝攻祖郎,大为所围,普与一骑共蔽扞策,驱马疾呼,以矛突贼,贼披,策因随出。后拜荡寇中郎将,领零陵太守,从讨刘勋於寻阳,进攻黄祖於沙羡,还镇石城。

策薨,与张昭等共辅孙权,遂周旋三郡,平讨不服。又从征江夏,还过豫章,别讨乐安。乐安平定,代太史慈备海昬,与周瑜为左右督,破曹公於乌林,又进攻南郡,走曹仁。拜裨将军,领江夏太守,治沙羡,食四县。

先出诸将,普最年长,时人皆呼程公。性好施与,喜士大夫。周瑜卒,代领南郡太守。权分荆州与刘备,普复还领江夏,迁荡寇将军,卒。权称尊号,追论普功,封子咨为亭侯。

黄盖字公覆,零陵泉陵人也。初为郡吏,察孝廉,辟公府。孙坚举义兵,盖从之。坚南破山贼,北走董卓,拜盖别部司马。坚薨,盖随策及权,擐甲周旋,蹈刃屠城。

诸山越不宾,有寇难之县,辄用盖为守长。石城县吏,特难检御,盖乃署两掾,分主诸曹。教曰:“令长不德,徒以武功为官,不以文吏为称。今贼寇未平,有军旅之务,一以文书委付两掾,当检摄诸曹,纠擿谬误。两掾所署,事入诺出,若有奸欺,终不加以鞭杖,宜各尽心,无为众先。“初皆怖威,夙夜恭职;久之,吏以盖不视文书,渐容人事。盖亦嫌外懈怠,时有所省,各得两掾不奉法数事。乃悉请诸掾吏,赐酒食,因出事诘问。两掾辞屈,皆叩头谢罪。盖曰:“前已相敕,终不以鞭杖相加,非相欺也。“遂杀之。县中震栗。后转春谷长,寻阳令。凡守九县,所在平定。迁丹杨都尉,抑强扶弱,山越怀附。

盖姿貌严毅,善於养众,每所征讨,士卒皆争为先。建安中,随周瑜拒曹公於赤壁,建策火攻,语在瑜传。拜武锋中郎将。武陵蛮夷反乱,攻守城邑,乃以盖领太守。时郡兵才五百人,自以不敌,因开城门,贼半入,乃击之,斩首数百,馀皆奔走,尽归邑落。诛讨魁帅,附从者赦之。自春讫夏,寇乱尽平,诸幽邃巴、醴、由、诞邑侯君长,皆改操易节,奉礼请见,郡境遂清。后长沙益阳县为山贼所攻,盖又平讨。加偏将军,病卒于官。

盖当官决断,事无留滞,国人思之。及权践阼,追论其功,赐子柄爵关内侯。

韩当字义公,辽西令支人也。以便弓马,有膂力,幸於孙坚,从征伐周旋,数犯危难,陷敌擒虏,为别部司马。及孙策东渡,从讨三郡,迁先登校尉,授兵二千,骑五十匹。从征刘勋,破黄祖,还讨鄱阳,领乐安长,山越畏服。后以中郎将与周瑜等拒破曹公,又与吕蒙袭取南郡,迁偏将军,领永昌太守。宜都之役,与陆逊、朱然等共攻蜀军於涿乡,大破之,徙威烈将军,封都亭侯。曹真攻南郡,当保东南。在外为帅,厉将士同心固守,又敬望督司,奉遵法令,权善之。黄武二年,封石城侯,迁昭武将军,领冠军太守,后又加都督之号。将敢死及解烦兵万人,讨丹杨贼,破之。会病卒,子综袭侯领兵。

其年,权征石阳,以综有忧,使守武昌,而综淫乱不轨。权虽以父故不问,综内怀惧,载父丧,将母家属部曲男女数千人奔魏。魏以为将军,封广阳侯。数犯边境,杀害人民,权常切齿。东兴之役,综为前锋,军败身死,诸葛恪斩送其首,以白权庙。

蒋钦字公奕,九江寿春人也。孙策之袭袁术,钦随从给事。及策东渡,拜别部司马,授兵。与策周旋,平定三郡,又从定豫章。调授葛阳尉,历三县长,讨平盗贼,迁西部都尉。会稽冶贼吕合、秦狼等为乱,钦将兵讨击,遂禽合、狼,五县平定,徙讨越中郎将,以经拘、昭阳为奉邑。贺齐讨黟贼,钦督万兵,与齐并力,黟贼平定。从征合肥,魏将张辽袭权於津北,钦力战有功,迁荡寇将军,领濡须督。后召还都,拜右护军,典领辞讼。

权尝入其堂内,母疏帐缥被,妻妾布裙。权叹其在贵守约,即敕御府为母作锦被,改易帷帐,妻妾衣服悉皆锦绣。

初,钦屯宣城,尝讨豫章贼。芜湖令徐盛收钦屯吏,表斩之,权以钦在远不许,盛由是自嫌於钦。曹公出濡须,钦与吕蒙持诸军节度。盛常畏钦因事害己,而钦每称其善。盛既服德,论者美焉。

权讨关羽,钦督水军入沔,还,道病卒。权素服举哀,以芜湖民二百户、田二百顷,给钦妻子。子壹封宣城侯,领兵拒刘备有功,还赴南郡,与魏交战,临陈卒。壹无子,弟休领兵,后有罪失业。

周泰字幼平,九江下蔡人也。与蒋钦随孙策为左右,服事恭敬,数战有功。策入会稽,署别部司马,授兵。权爱其为人,请以自给。策讨六县山贼,权住宣城,使士自卫,不能千人,意尚忽略,不治围落,而山贼数千人卒至。权始得上马,而贼锋刃已交於左右,或斫中马鞍,众莫能自定。惟泰奋激,投身卫权,胆气倍人,左右由泰并能就战。贼既解散,身被十二创,良久乃苏。是日无泰,权几危殆。策深德之,补春谷长。后从攻皖,及讨江夏,还过豫章,复补宜春长,所在皆食其征赋。

从讨黄祖有功。后与周瑜、程普拒曹公於赤壁,攻曹仁於南郡。荆州平定,将兵屯岑。曹公出濡须,泰复赴击,曹公退,留督濡须,拜平虏将军。时朱然、徐盛等皆在所部,并不伏也,权特为案行至濡须坞,因会诸将,大为酣乐。权自行酒到泰前,命泰解衣,权手自指其创痕,问以所起。泰辄记昔战斗处以对,毕,使复服,欢宴极夜。其明日,遣使者授以御盖。於是盛等乃伏。

后权破关羽,欲进图蜀,拜泰汉中太守、奋威将军,封陵阳侯。黄武中卒。

子邵以骑都尉领兵。曹仁出濡须,战有功,又从攻破曹休,进位裨将军,黄龙二年卒。弟承领兵袭侯。

陈武字子烈,庐江松滋人。孙策在寿春,武往脩谒,时年十八,长七尺七寸,因从渡江,征讨有功,拜别部司马。策破刘勋,多得庐江人,料其精锐,乃以武为督,所向无前。及权统事,转督五校。仁厚好施,乡里远方客多依讬之。尤为权所亲爱,数至其家。累有功劳,进位偏将军。建安二十年,从击合肥,奋命战死。权哀之,自临其葬。

子脩有武风,年十九,权召见奖厉,拜别部司马,授兵五百人。时诸新兵多有逃叛,而脩抚循得意,不失一人。权奇之,拜为校尉。建安末,追录功臣后,封脩都亭侯,为解烦督。黄龙元年卒。

弟表,字文奥,武庶子也,少知名,与诸葛恪、顾谭、张休等并侍东宫,皆共亲友。尚书暨艳亦与表善,后艳遇罪,时人咸自营护,信厚言薄,表独不然,士以此重之。从太子中庶子,拜翼正都尉。兄脩亡后,表母不肯事脩母,表谓其母曰:“兄不幸早亡,表统家事,当奉嫡母。母若能为表屈情,承顺嫡母者,是至愿也;若母不能,直当出别居耳。“表於大义公正如此。由是二母感寤雍穆。表以父死敌场,求用为将,领兵五百人。表欲得战士之力,倾意接待,士皆爱附,乐为用命。时有盗官物者,疑无难士施明。明素壮悍,收考极毒,惟死无辞,廷尉以闻。权以表能得健儿之心,诏以明付表,使自以意求其情实。表便破械沐浴,易其衣服,厚设酒食,欢以诱之。明乃首服,具列支党。表以状闻。权奇之,欲全其名,特为赦明,诛戮其党。迁表为无难右部督,封都亭侯,以继旧爵。表皆陈让,乞以传脩子延,权不许。嘉禾三年,诸葛恪领丹杨太守,讨平山越,以表领新安都尉,与恪参势。初,表所受赐复人得二百家,在会稽新安县。表简视其人,皆堪好兵,乃上疏陈让,乞以还官,充足精锐。诏曰:“先将军有功於国,国家以此报之,卿何得辞焉?“表乃称曰:“今除国贼,报父之仇,以人为本。空枉此劲锐以为僮仆,非表志也。“皆辄料取以充部伍。所在以闻,权甚嘉之。下郡县,料正户羸民以补其处。表在官三年,广开降纳,得兵万馀人。事捷当出,会鄱阳民吴遽等为乱,攻没城郭,属县摇动,表便越界赴讨,遽以破败,遂降。陆逊拜表偏将军,进封都乡侯,北屯章坑。年三十四卒。家财尽於养士,死之日,妻子露立,太子登为起屋宅。子敖年十七,拜别部司马,授兵四百人。敖卒,脩子延复为司马代敖。延弟永,将军,封侯。始施明感表,自变行为善,遂成健将,致位将军。

董袭字元代,会稽馀姚人,长八尺,武力过人。孙策入郡,袭迎於高迁亭,策见而伟之,到署门下贼曹。时山阴宿贼黄龙罗、周勃聚党数千人,策自出讨,袭身斩罗、勃首,还拜别部司马,授兵数千,迁扬武都尉。从策攻皖,又讨刘勋於寻阳,伐黄祖於江夏。

策薨,权年少,初统事,太妃忧之,引见张昭及袭等,问江东可保安否,袭对曰:“江东地势,有山川之固,而讨逆明府,恩德在民。讨虏承基,大小用命,张昭秉众事,袭等为爪牙,此地利人和之时也,万无所忧。“众皆壮其言。

鄱阳贼彭虎等众数万人,袭与凌统、步骘、蒋钦各别分讨。袭所向辄破,虎等望见旌旗,便散走,旬日尽平,拜威越校尉,迁偏将军。

建安十三年,权讨黄祖。祖横两蒙冲挟守沔口,以栟闾大绁系石为碇,上有千人,以弩交射,飞矢雨下,军不得前。袭与凌统俱为前部,各将敢死百人,人被两铠,乘大舸船,突入蒙冲里。袭身以刀断两绁,蒙冲乃横流,大兵遂进。祖便开门走,兵追斩之。明日大会,权举觞属袭曰:“今日之会,断绁之功也。“

曹公出濡须,袭从权赴之,使袭督五楼船住濡须口。夜卒暴风,五楼船倾覆,左右散走舸,乞使袭出。袭怒曰:“受将军任,在此备贼,何等委去也,敢复言此者斩!“於是莫敢干。其夜船败,袭死。权改服临殡,供给甚厚。

甘宁字兴霸,巴郡临江人也。少有气力,好游侠,招合轻薄少年,为之渠帅;群聚相随,挟持弓弩,负毦带铃,民闻铃声,即知是宁。人与相逢,及属城长吏,接待隆厚者乃与交欢;不尔,即放所将夺其资货,於长吏界中有所贼害,作其发负,至二十馀年。止不攻劫,颇读诸子,乃往依刘表,因居南阳,不见进用,后转托黄祖,祖又以凡人畜之。

於是归吴。周瑜、吕蒙皆共荐达,孙权加异,同於旧臣。宁陈计曰:“今汉祚日微,曹操弥忄乔,终为篡盗。南荆之地。山陵形便,江川流通,诚是国之西势也。宁已观刘表,虑既不远,儿子又劣,非能承业传基者也。至尊当早规之,不可后操。图之之计,宜先取黄祖。祖今年老,昏耄已甚,财谷并乏,左右欺弄,务於货利,侵求吏士,吏士心怨,舟船战具,顿废不脩,怠於耕农,军无法伍。至尊今往,其破可必。一破祖军,鼓行而西,西据楚关,大势弥广,即可渐规巴蜀。“权深纳之。张昭时在坐,难曰:“吴下业业,若军果行,恐必致乱。“宁谓昭曰:“国家以萧何之任付君,君居守而忧乱,奚以希慕古人乎?“权举酒属宁曰:“兴霸,今年行讨,如此酒矣,决以付卿。卿但当勉建方略,令必克祖,则卿之功,何嫌张长史之言乎。“权遂西,果禽祖,尽获其士众。遂授宁兵,屯当口。

后随周瑜拒破曹公於乌林。攻曹仁於南郡,未拔,宁建计先径进取夷陵,往即得其城,因入守之。时手下有数百兵,并所新得,仅满千人。曹仁乃令五六千人围宁。宁受攻累日,敌设高楼,雨射城中,士众皆惧,惟宁谈笑自若。遣使报瑜,瑜用吕蒙计,帅诸将解围。后随鲁肃镇益阳,拒关羽。羽号有三万人,自择选锐士五千人,投县上流十馀里浅濑,云欲夜涉渡。肃与诸将议。宁时有三百兵,乃曰:“可复以五百人益吾,吾往对之,保羽闻吾欬唾,不敢涉水,涉水即是吾禽。“肃便选千兵益宁,宁乃夜往。羽闻之,住不渡,而结柴营,今遂名此处为关羽濑。权嘉宁功,拜西陵太守,领阳新、下雉两县。

后从攻皖,为升城督。宁手持练,身缘城,为吏士先,卒破获朱光。计功,吕蒙为最。宁次之,拜折冲将军。

后曹公出濡须,宁为前部督,受敕出斫敌前营。权特赐米酒众殽,宁乃料赐手下百馀人食。食毕,宁先以银碗酌酒,自饮两碗,乃酌与其都督。都督伏,不肯时持。宁引白削置膝上,呵谓之曰:“卿见知於至尊,孰与甘宁?甘宁尚不惜死,卿何以独惜死乎?“都督见宁色厉,即起拜持酒,通酌兵各一银碗。至二更时,衔枚出斫敌。敌惊动,遂退。宁益贵重,增兵二千人。

宁虽粗猛好杀,然开爽有计略,轻财敬士,能厚养健儿,健儿亦乐为用命。建安二十年,从攻合肥,会疫疾,军旅皆已引出,唯车下虎士千馀人,并吕蒙、蒋钦、凌统及宁,从权逍遥津北。张辽觇望知之,即将步骑奄至。宁引弓射敌,与统等死战。宁厉声问鼓吹何以不作,壮气毅然,权尤嘉之。

宁厨下儿曾有过,走投吕蒙。蒙恐宁杀之,故不即还。后宁赍礼礼蒙母,临当与升堂,乃出厨下儿还宁。宁许蒙不杀。斯须还船,缚置桑树,自挽弓射杀之。毕,敕船人更增舸缆,解衣卧船中。蒙大怒,击鼓会兵,欲就船攻宁。宁闻之,故卧不起。蒙母徒跣出谏蒙曰:“至尊待汝如骨肉,属汝以大事,何有以私怒而欲攻杀甘宁?宁死之日,纵至尊不问,汝是为臣下非法。“蒙素至孝,闻母言,即豁然意释,自至宁船,笑呼之曰:“兴霸,老母待卿食,急上!“宁涕泣歔欷曰:“负卿。“与蒙俱还见母,欢宴竟日。

宁卒,权痛惜之。子瑰,以罪徙会稽,无几死。

凌统字公绩,吴郡馀杭人也。父操,轻侠有胆气,孙策初兴,每从征伐,常冠军履锋。守永平长,平治山越,奸猾敛手,迁破贼校尉。及权统军,从讨江夏。入夏口,先登,破其前锋,轻舟独进,中流矢死。

统年十五,左右多称述者,权亦以操死国事,拜统别部司马,行破贼都尉,使摄父兵。后从击山贼,权破保屯先还,馀麻屯万人,统与督张异等留攻围之,克日当攻。先期,统与督陈勤会饮酒,勤刚勇任气,因督祭酒,陵轹一坐,举罚不以其道。统疾其侮慢,面折不为用。勤怒詈统,及其父操,统流涕不答,众因罢出。勤乘酒凶悖,又於道路辱统。统不忍,引刀斫勤,数日乃死。及当攻屯,统曰:“非死无以谢罪。“乃率厉士卒,身当矢石,所攻一面,应时披坏,诸将乘胜,遂大破之。还,自拘於军正。权壮其果毅,使得以功赎罪。

后权复征江夏,统为前锋,与所厚健儿数十人共乘一船,常去大兵数十里。行入右江,斩黄祖将张硕,尽获船人。还以白权,引军兼道,水陆并集。时吕蒙败其水军,而统先搏其城,於是大获。权以统为承烈都尉,与周瑜等拒破曹公於乌林,遂攻曹仁,迁为校尉。虽在军旅,亲贤接士,轻财重义,有国士之风。

又从破皖,拜荡寇中郎将,领沛相。与吕蒙等西取三郡,反自益阳,从往合肥,为右部督。时权彻军,前部已发,魏将张辽等奄至津北。权使追还前兵,兵去已远,势不相及,统率亲近三百人陷围,扶扞权出。敌已毁桥,桥之属者两版,权策马驱驰,统复还战,左右尽死,身亦被创,所杀数十人,度权已免,乃还。桥败路绝,统被甲潜行。权既御船,见之惊喜。统痛亲近无反者,悲不自胜。权引袂拭之,谓曰:“公绩,亡者已矣,苟使卿在,何患无人?“拜偏将军,倍给本兵。

时有荐同郡盛暹於权者,以为梗概大节,有过於统,权曰:“且令如统足矣。“后召暹夜至,时统已卧,闻之,摄衣出门,执其手以入。其爱善不害如此。

统以山中人尚多壮悍,可以威恩诱也,权令东占且讨之,命敕属城,凡统所求,皆先给后闻。统素爱士,士亦慕焉。得精兵万馀人,过本县,步入寺门,见长吏怀三版,恭敬尽礼,亲旧故人,恩意益隆。事毕当出,会病卒,时年四十九。权闻之,拊床起坐,哀不能自止,数日减膳,言及流涕,使张承为作铭诔。

二子烈、封,年各数岁,权内养於宫,爱待与诸子同,宾客进见,呼示之曰:“此吾虎子也。“及八九岁,令葛光教之读书,十日一令乘马,追录统功,封烈亭侯,还其故兵。后烈有罪免,封复袭爵领兵。

徐盛字文向,琅邪莒人也。遭乱,客居吴,以勇气闻。孙权统事,以为别部司马,授兵五百人,守柴桑长,拒黄祖。祖子射,尝率数千人下攻盛。盛时吏士不满二百,与相拒击,伤射吏士千馀人。已乃开门出战,大破之。射遂绝迹不复为寇。权以为校尉、芜湖令。复讨临城南阿山贼有功,徙中郎将,督校兵。

曹公出濡须,从权御之。魏尝大出横江,盛与诸将俱赴讨。时乘蒙冲,遇迅风,船落敌岸下,诸将恐惧,未有出者,盛独将兵,上突斫敌,敌披退走,有所伤杀,风止便还,权大壮之。

及权为魏称藩,魏使邢贞拜权为吴王。权出都亭候贞,贞有骄色,张昭既怒,而盛忿愤,顾谓同列曰:“盛等不能奋身出命,为国家并许洛,吞巴蜀,而令吾君与贞盟,不亦辱乎!“因涕泣横流。贞闻之,谓其旅曰:“江东将相如此,非久下人者也。“     ,

后迁建武将军,封都亭侯,领庐江太守,赐临城县为奉邑。刘备次西陵,盛攻取诸屯,所向有功。曹休出洞口,盛与吕范、全琮渡江拒守。遭大风,船人多丧,盛收馀兵,与休夹江。休使兵将就船攻盛,盛以少御多,敌不能克,各引军退。迁安东将军,封芜湖侯。

后魏文帝大出,有渡江之志,盛建计从建业筑围,作薄落,围上设假楼,江中浮船。诸将以为无益,盛不听,固立之。文帝到广陵,望围愕然,弥漫数百里,而江水盛长,便引军退。诸将乃伏。

黄武中卒。子楷,袭爵领兵。

潘璋字文珪,东郡发干人也。孙权为阳羡长,始往随权。性博荡嗜酒,居贫,好赊酤,债家至门,辄言后豪富相还。权奇爱之,因使召募,得百馀人,遂以为将。讨山贼有功,署别部司马。后为吴大巿刺奸,盗贼断绝,由是知名,迁豫章西安长。刘表在荆州,民数被寇,自璋在事,寇不入境。比县建昌起为贼乱,转领建昌,加武猛校尉,讨治恶民,旬月尽平,召合遗散,得八百人,将还建业。

合肥之役,张辽奄至,诸将不备,陈武斗死,宋谦、徐盛皆披走,璋身次在后,便驰进,横马斩谦、盛兵走者二人,兵皆还战。权甚壮之,拜偏将军,遂领百校,屯半州。

权征关羽,璋与朱然断羽走道,到临沮,住夹石。璋部下司马马忠禽羽,并羽子平、都督赵累等。权即分宜都巫、秭归二县为固陵郡,拜璋为太守、振威将军,封溧阳侯。甘宁卒,又并其军。刘备出夷陵,璋与陆逊并力拒之,璋部下斩备护军冯习等,所杀伤甚众,拜平北将军、襄阳太守。

魏将夏侯尚等围南郡,分前部三万人作浮桥,渡百里洲上,诸葛瑾、杨粲并会兵赴救,未知所出,而魏兵日渡不绝。璋曰:“魏势始盛,江水又浅,未可与战。“便将所领,到魏上流五十里,伐苇数百万束,缚作大筏,欲顺流放火,烧败浮桥。作筏適毕,伺水长当下,尚便引退。璋下备陆口。权称尊号,拜右将军。

璋为人粗猛,禁令肃然,好立功业,所领兵马不过数千,而其所在常如万人。征伐止顿,便立军巿,他军所无,皆仰取足。然性奢泰,末年弥甚,服物僣拟。吏兵富者,或杀取其财物,数不奉法。监司举奏,权惜其功而辄原不问。嘉禾三年卒。子平,以无行徙会稽。璋妻居建业,赐田宅,复客五十家。

丁奉字承渊,庐江安丰人也。少以骁勇为小将,属甘宁、陆逊、潘璋等。数随征伐,战斗常冠军。每斩将搴旗,身被创夷。稍迁偏将军。孙亮即位,为冠军将军,封都亭侯。

魏遣诸葛诞、胡遵等攻东兴,诸葛恪率军拒之。诸将皆曰:“敌闻太傅自来,上岸必遁走。“奉独曰:“不然。彼动其境内,悉许、洛兵大举而来,必有成规,岂虚还哉?无恃敌之不至,恃吾有以胜之。“及恪上岸,奉与将军唐咨、吕据、留赞等,俱从山西上。奉曰:“今诸军行迟,若敌据便地,则难与争锋矣。“乃辟诸军使下道,帅麾下三千人径进。时北风,奉举帆二日至,遂据徐塘。天寒雪,敌诸将置酒高会,奉见其前部兵少,相谓曰:“取封侯爵赏,正在今日!“乃使兵解铠著胄,持短兵。敌人从而笑焉,不为设备。奉纵兵斫之,大破敌前屯。会据等至,魏军遂溃。迁灭寇将军,进封都乡侯。

魏将文钦来降,以奉为虎威将军,从孙峻至寿春迎之,与敌追军战於高亭。奉跨马持矛,突入其陈中,斩首数百,获其军器。进封安丰侯。

太平二年,魏大将军诸葛诞据寿春来降,魏人围之。遣朱异、唐咨等往救,复使奉与黎斐解围。奉为先登,屯於黎浆,力战有功,拜左将军。

孙休即位,与张布谋,欲诛孙綝,布曰:“丁奉虽不能吏书,而计略过人,能断大事。“休召奉告曰:“綝秉国威,将行不轨,欲与将军诛之。“奉曰:“丞相兄弟友党甚盛,恐人心不同,不可卒制,可因腊会,有陛下兵以诛之也。“休纳其计,因会请綝,奉与张布目左右斩之。迁大将军,加左右都护。永安三年,假节领徐州牧。六年,魏伐蜀,奉率诸军向寿春,为救蜀之势。蜀亡,军还。

休薨,奉与丞相濮阳兴等从万彧之言,共迎立孙皓,迁右大司马左军师。宝鼎三年,皓命奉与诸葛靓攻合肥。奉与晋大将石苞书,构而间之,苞以徵还。建衡元年,奉复帅众治徐塘,因攻晋谷阳。谷阳民知之,引去,奉无所获。皓怒,斩奉导军。三年,卒。奉贵而有功,渐以骄矜,或有毁之者,皓追以前出军事,徙奉家於临川。奉弟封,官至后将军,先奉死。

评曰:凡此诸将,皆江表之虎臣,孙氏之所厚待也。以潘璋之不脩,权能忘过记功,其保据东南,宜哉!陈表将家支庶,而与胄子名人比翼齐衡,拔萃出类,不亦美乎!

\part{吴书十一}
\chapter{朱治朱然吕范朱桓传第十一}

 朱治字君理,丹杨故鄣人也。初为县吏,后察孝廉,州辟从事,随孙坚征伐。中平五年,拜司马,从讨长沙、零、桂等三郡贼周朝、苏马等,有功,坚表治行都尉。从破董卓於阳人,入洛阳。表治行督军校尉,特将步骑,东助徐州牧陶谦讨黄巾。

会坚薨,治扶翼策,依就袁术。后知术政德不立,乃劝策还平江东。时太傅马日磾在寿春,辟治为掾,迁吴郡都尉。是时吴景已在丹杨,而策为术攻庐江,於是刘繇恐为袁、孙所并,遂构嫌隙。而策家门尽在州下,治乃使人於曲阿迎太妃及权兄弟,所以供奉辅护,甚有恩纪。治从钱唐欲进到吴,吴郡太守许贡拒之於由拳,治与战,大破之。贡南就山贼严白虎,治遂入郡,领太守事。策既走刘繇,东定会稽。

权年十五,治举为孝廉。后策薨,治与张昭等共尊奉权。建安七年,权表治为吴郡太守,行扶义将军,割娄、由拳、无锡、毗陵为奉邑,置长吏。征讨夷越,佐定东南,禽截黄巾馀类陈败、万秉等。黄武元年,封毗陵侯,领郡如故。二年,拜安国将军,金印紫绶,徙封故鄣。

权历位上将,及为吴王,治每进见,权常亲迎,执版交拜,飨宴赠赐,恩敬特隆,至从行吏,皆得奉贽私觌,其见异如此。

初,权弟翊,性峭急,喜怒快意,治数责数,谕以道义。权从兄豫章太守贲,女为曹公子妇,及曹公破荆州,威震南土,贲畏惧,欲遣子入质。治闻之,求往见贲,为陈安危,贲由此遂止。

权常叹治忧勤王事。性俭约,虽在富贵,车服惟供事。权优异之,自令督军御史典属城文书,治领四县租税而已。然公族子弟及吴四姓多出仕郡,郡吏常以千数,治率数年一遣诣王府,所遣数百人,每岁时献御,权答报过厚。是时丹杨深地,频有奸叛,亦以年向老,思恋土风,自表屯故鄣,镇抚山越。诸父老故人,莫不诣门,治皆引进,与共饮宴,乡党以为荣。在故鄣岁馀,还吴。黄武三年卒,在郡三十一年,年六十九。

子才,素为校尉领兵,既嗣父爵,迁偏将军。才弟纪,权以策女妻之,亦以校尉领兵。纪弟纬、万岁,皆早夭。才子琬,袭爵为将,至镇西将军。

朱然字义封,治姊子也,本姓施氏。初治未有子,然年十三,乃启策乞以为嗣。策命丹杨郡以羊酒召然,然到吴,策优以礼贺。

然尝与权同学书,结恩爱。至权统事,以然为馀姚长,时年十九。后迁山阴令,加折冲校尉,督五县。权奇其能,分丹杨为临川郡,然为太守,授兵二千人。会山贼盛起,然平讨,旬月而定。曹公出濡须,然备大坞及三关屯,拜偏将军。建安二十四年,从讨关羽,别与潘璋到临沮禽羽,迁昭武将军,封西安乡侯。

虎威将军吕蒙病笃,权问曰:“卿如不起,谁可代者?”蒙对曰:“朱然胆守有馀,愚以为可任。”蒙卒,权假然节,镇江陵。黄武元年,刘备举兵攻宜都,然督五千人与陆逊并力拒备。然别攻破备前锋,断其后道,备遂破走。拜征北将军,封永安侯。

魏遣曹真、夏侯尚、张郃等攻江陵,魏文帝自住宛,为其势援,连屯围城。权遣将军孙盛督万人备州上,立围坞,为然外救。郃渡兵攻盛,盛不能拒,即时卻退,郃据州上围守,然中外断绝。权遣潘璋、杨粲等解围而围不解。时然城中兵多肿病,堪战者裁五千人。真等起土山,凿地道,立楼橹临城,弓矢雨注,将士皆失色,然晏如而无恐意,方厉吏士,伺间隙攻破两屯。魏攻围然凡六月日,未退。江陵令姚泰领兵备城北门,见外兵盛,城中人少,谷食欲尽,因与敌交通,谋为内应。垂发,事觉,然治戮泰。尚等不能克,乃彻攻退还。由是然名震於敌国,改封当阳侯。

六年,权自率众攻石阳,及至旋师,潘璋断后。夜出错乱,敌追击璋,璋不能禁。然即还住拒敌,使前船得引极远,徐乃后发。黄龙元年,拜车骑将军、右护军,领兖州牧。顷之,以兖州在蜀分,解牧职。

嘉禾三年,权与蜀克期大举,权自向新城,然与全琮各受斧钺,为左右督。会吏士疾病,故未攻而退。

赤乌五年,征柤中,魏将蒲忠、胡质各将数千人,忠要遮险隘,图断然后,质为忠继援。时然所督兵将先四出,闻问不暇收合,便将帐下见兵八百人逆掩。忠战不利,质等皆退。九年,复征柤中,魏将李兴等闻然深入,率步骑六千断然后道,然夜出逆之,军以胜反。先是,归义马茂怀奸,觉诛,权深忿之。然临行上疏曰:“马茂小子,敢负恩养。臣今奉天威,事蒙克捷,欲令所获,震耀远近,方舟塞江,使足可观,以解上下之忿。惟陛下识臣先言,责臣后效。”权时抑表不出。然既献捷,群臣上贺,权乃举酒作乐,而出然表曰:“此家前初有表,孤以为难必,今果如其言,可谓明於见事也。”遣使拜然为左大司马、右军师。

然长不盈七尺,气候分明,内行脩絜,其所文采,惟施军器,馀皆质素。终日钦钦,常在战场,临急胆定,尤过绝人,虽世无事,每朝夕严鼓,兵在营者,咸行装就队,以此玩敌,使不知所备,故出辄有功。诸葛瑾子融、步骘子协,虽各袭任,权特复使然总为大督。又陆逊亦卒,功臣名将存者惟然,莫与比隆。寝疾二年,后渐增笃,权昼为减膳,夜为不寐,中使医药口食之物,相望於道。然每遣使表疾病消息,权辄召见,口自问讯,入赐酒食,出送布帛。自创业功臣疾病,权意之所锺,吕蒙、凌统最重,然其次矣。年六十八,赤乌十二年卒,权素服举哀,为之感恸。子绩嗣。

绩字公绪,以父任为郎,后拜建忠都尉。叔父才卒,绩领其兵,随太常潘濬讨五溪,以胆力称。迁偏将军营下督,领盗贼事,持法不倾。鲁王霸注意交绩,尝至其廨,就之坐,欲与结好,绩下地住立,辞而不当。然卒,绩袭业,拜平魏将军,乐乡督。明年,魏征南将军王昶率众攻江陵城,不克而退。绩与奋威将军诸葛融书曰:“昶远来疲困,马无所食,力屈而走,此天助也。今追之力少,可引兵相继,吾欲破之於前,足下乘之於后,岂一人之功哉,宜同断金之义。”融答许绩。绩便引兵及昶於纪南,纪南去城三十里,绩先战胜而融不进,绩后失利。权深嘉绩,盛责怒融,融兄大将军恪贵重,故融得不废。初绩与恪、融不平,及此事变,为隙益甚。建兴元年,迁镇东将军。二年春,恪向新城,要绩并力,而留置半州,使融兼其任。冬,恪、融被害,绩复还乐乡,假节。太平二年,拜骠骑将军。孙綝秉政,大臣疑贰,绩恐吴必扰乱,而中国乘衅,乃密书结蜀,使为并兼之虑。蜀遣右将军阎宇将兵五千,增白帝守,以须绩之后命。永安初,迁上大将军、都护督,自巴丘上迄西陵。元兴元年,就拜左大司马。初,然为治行丧竟,乞复本姓,权不许,绩以五凤中表还为施氏,建衡二年卒。

吕范字子衡,汝南细阳人也。少为县吏,有容观姿貌。邑人刘氏,家富女美,范求之。女母嫌,欲勿与,刘氏曰:“观吕子衡宁当久贫者邪?”遂与之婚。后避乱寿春,孙策见而异之,范遂自委昵,将私客百人归策。时太妃在江都,策遣范迎之。徐州牧陶谦谓范为袁氏觇候,讽县掠考范,范亲客健儿篡取以归。时唯范与孙河常从策,跋涉辛苦,危难不避,策亦亲戚待之,每与升堂,饮宴於太妃前。

后从策攻破庐江,还俱东渡,到横江、当利,破张英、于麋,下小丹杨、湖孰,领湖孰相。策定秣陵、曲阿,收笮融、刘繇馀众,增范兵二千,骑五十匹。后领宛陵令,讨破丹杨贼,还吴,迁都督。

是时下邳陈瑀自号吴郡太守,住海西,与强族严白虎交通。策自将讨虎,别遣范与徐逸攻瑀於海西,枭其大将陈牧。又从攻祖郎於陵阳,太史慈於勇里。七县平定,拜征虏中郎将,征江夏,还平鄱阳。

策薨,奔丧于吴。后权复征江夏,范与张昭留守。

曹公至赤壁,与周瑜等俱拒破之,拜裨将军,领彭泽太守,以彭泽、柴桑、历阳为奉邑。刘备诣京见权,范密请留备。后迁平南将军,屯柴桑。

权讨关羽,过范馆,谓曰:“昔早从卿言,无此劳也。今当上取之,卿为我守建业。”权破羽还,都武昌,拜范建威将军,封宛陵侯,领丹杨太守,治建业,督扶州以下至海,转以溧阳、怀安、宁国为奉邑。

曹休、张辽、臧霸等来伐,范督徐盛、全琮、孙韶等,以舟师拒休等於洞口。迁前将军,假节,改封南昌侯。时遭大风,船人覆溺,死者数千,还军,拜扬州牧。

性好威仪,州民如陆逊、全琮及贵公子,皆脩敬虔肃,不敢轻脱。其居处服饰,於时奢靡,然勤事奉法,故权悦其忠,不怪其侈。

初策使范典主财计,权时年少,私从有求,范必关白,不敢专许,当时以此见望。权守阳羡长,有所私用,策或料覆,功曹周谷辄为傅著簿书,使无谴问。权临时悦之,及后统事,以范忠诚,厚见信任,以谷能欺更簿书,不用也。

黄武七年,范迁大司马,印绶未下,疾卒。权素服举哀,遣使者追赠印绶。及还都建业,权过范墓呼曰:“子衡!”言及流涕,祀以太牢。

范长子先卒,次子据嗣。据字世议,以父任为郎,后范寝疾,拜副军校尉,佐领军事。范卒,迁安军中郎将。数讨山贼,诸深恶剧地,所击皆破。随太常潘濬讨五谿,复有功。朱然攻樊,据与朱异破城外围,还拜偏将军,入补马闲右部督,迁越骑校尉。太元元年,大风,江水溢流,渐淹城门,权使视水,独见据使人取大船以备害。权嘉之,拜荡魏将军。权寝疾,以据为太子右部督。太子即位,拜右将军。魏出东兴,据赴讨有功。明年,孙峻杀诸葛恪,迁据为骠骑将军,平西宫事。五凤二年,假节,与峻等袭寿春,还遇魏将曹珍,破之於高亭。太平元年,帅师侵魏,未及淮,闻孙峻死,以从弟綝自代,据大怒,引军还,欲废綝。綝闻之,使中书奉诏,诏文钦、刘纂、唐咨等使取据,又遣从兄宪以都下兵逆据於江都。左右劝据降魏,据曰:“耻为叛臣。”遂自杀。夷三族。

朱桓字休穆,吴郡吴人也。孙权为将军,桓给事幕府,除馀姚长。往遇疫疠,谷食荒贵,桓分部良吏,隐亲医药,飧粥相继,士民感戴之。迁荡寇校尉,授兵二千人,使部伍吴、会二郡,鸠合遗散,期年之间,得万馀人。后丹杨、鄱阳山贼蜂起,攻没城郭,杀略长吏,处处屯聚。桓督领诸将,周旋赴讨,应皆平定。稍迁裨将军,封新城亭侯。

后代周泰为濡须督。黄武元年,魏使大司马曹仁步骑数万向濡须,仁欲以兵袭取州上,伪先扬声,欲东攻羡溪。桓分兵将赴羡溪,既发,卒得仁进军拒濡须七十里问。桓遣使追还羡溪兵,兵未到而仁奄至。时桓手下及所部兵,在者五千人,诸将业业,各有惧心,桓喻之曰:“凡两军交对,胜负在将,不在众寡。诸君闻曹仁用兵行师,孰与桓邪?兵法所以称客倍而主人半者,谓俱在平原,无城池之守,又谓士众勇怯齐等故耳。今人既非智勇,加其士卒甚怯,又千里步涉,人马罢困,桓与诸军,共据高城,南临大江,北背山陵,以逸待劳,为主制客,此百战百胜之势也。虽曹丕自来,尚不足忧,况仁等邪!”桓因偃旗鼓,外示虚弱,以诱致仁。仁果遣其子泰攻濡须城,分遣将军常雕督诸葛虔、王双等,乘油船别袭中洲。中洲者,部曲妻子所在也。仁自将万人留橐皋,复为泰等后拒。桓部兵将攻取油船,或别击雕等,桓等身自拒泰,烧营而退,遂枭雕,生虏双,送武昌,临陈斩溺,死者千馀。权嘉桓功,封嘉兴侯,迁奋武将军,领彭城相。

黄武七年,鄱阳太守周鲂谲诱魏大司马曹休,休将步骑十万至皖城以迎鲂。时陆逊为元帅,全琮与桓为左右督,各督三万人击休。休知见欺,当引军还,自负众盛,邀於一战。桓进计曰:“休本以亲戚见任,非智勇名将也。今战必败,败必走,走当由夹石、挂车,此两道皆险厄,若以万兵柴路,则彼众可尽,而休可生虏,臣请将所部以断之。若蒙天威,得以休自效,便可乘胜长驱,进取寿春,割有淮南,以规许、洛,此万世一时,不可失也。”权先与陆逊议,逊以为不可,故计不施行。

黄龙元年,拜桓前将军,领青州牧,假节。嘉禾六年,魏庐江主簿吕习请大兵自迎,欲开门为应。桓与卫将军全琮俱以师迎。既至,事露,军当引还。城外有溪水,去城一里所,广三十馀丈,深者八九尺,浅者半之,诸军勒兵渡去,桓自断后。时庐江太守李膺整严兵骑,欲须诸军半渡,因迫击之。及见桓节盖在后,卒不敢出,其见惮如此。

是时全琮为督,权又令偏将军胡综宣传诏命,参与军事。琮以军出无获,议欲部分诸将,有所掩袭。桓素气高,耻见部伍,乃往见琮,问行意,感激发怒,与琮校计。琮欲自解,因曰:“上自令胡综为督,综意以为宜尔。”桓愈恚恨,还乃使人呼综。综至军门,桓出迎之,顾谓左右曰:“我纵手,汝等各自去。“有一人旁出,语综使还。桓出,不见综,知左右所为,因斫杀之。桓佐军进谏,刺杀佐军,遂讬狂发,诣建业治病。权惜其功能,故不罪。使子异摄领部曲,令医视护,数月复遣还中洲。权自出祖送,谓曰:“今寇虏尚存,王涂未一,孤当与君共定天下,欲令君督五万人专当一面,以图进取,想君疾未复发也。”桓曰:“天授陛下圣姿,当君临四海,猥重任臣,以除奸逆,臣疾当自愈。”

桓性护前,耻为人下,每临敌交战,节度不得自由,辄嗔恚愤激。然轻财贵义,兼以强识,与人一面,数十年不忘,部曲万口,妻子尽识之。爱养吏士,赡护六亲,俸禄产业,皆与共分。及桓疾困,举营忧戚。年六十二,赤乌元年卒。吏士男女,无不号慕。又家无馀财,权赐盐五千斛以周丧事。子异嗣。

异字季文,以父任除郎,后拜骑都尉,代桓领兵。赤乌四年,随朱然攻魏樊城,建计破其外围,还拜偏将军。魏庐江太守文钦营住六安,多设屯砦,置诸道要,以招诱亡叛,为边寇害。异乃身率其手下二千人,掩破钦七屯,斩首数百,迁扬武将军。权与论攻战,辞对称意。权谓异从父骠骑将军据曰:“本知季文胆定,见之复过所闻。”十三年,文钦诈降,密书与异,欲令自迎。异表呈钦书,因陈其伪,不可便迎。权诏曰:“方今北土未一,钦云欲归命,宜且迎之。若嫌其有谲者,但当设计网以罗之,盛重兵以防之耳。”乃遣吕据督二万人,与异并力,至北界,钦果不降。建兴元年,迁镇南将军。是岁魏遣胡遵、诸葛诞等出东兴,异督水军攻浮梁,坏之,魏军大破。太平二年,假节,为大都督,救寿春围,不解。还军,为孙綝所枉害。

评曰:朱治、吕范以旧臣任用,朱然、朱桓以勇烈著闻,吕据、朱异、施绩咸有将领之才,克绍堂构。若范、桓之越隘,得以吉终,至於据、异无此之尤而反罹殃者,所遇之时殊也。

\part{吴书十二}
\chapter{虞陆张骆陆吾朱传第十二}

虞翻字仲翔,会稽馀姚人也,太守王朗命为功曹。孙策征会稽,翻时遭父丧,衰绖诣府门,朗欲就之,翻乃脱衰入见,劝朗避策。朗不能用,拒战败绩,亡走浮海。翻追随营护,到东部候官,候官长闭城不受,翻往说之,然后见纳。朗谓翻曰:“卿有老母,可以还矣。”翻既归,策复命为功曹,待以交友之礼,身诣翻第。

策好驰骋游猎,翻谏曰:“明府用乌集之众,驱散附之士,皆得其死力,虽汉高帝不及也。至於轻出微行,从官不暇严,吏卒常苦之。夫君人者不重则不威,故白龙鱼服,困於豫且,白蛇自放,刘季害之,愿少留意。“策曰:“君言是也。然时有所思,端坐悒悒,有裨谌草创之计,是以行耳。”

翻出为富春长。策薨,诸长吏并欲出赴丧,翻曰:“恐邻县山民或有奸变,远委城郭,必致不虞。”因留制服行丧。诸县皆效之,咸以安宁。后翻州举茂才,汉召为侍御史,曹公为司空辟,皆不就。

翻与少府孔融书,并示以所著易注。融答书曰:“闻延陵之理乐,睹吾子之治易,乃知东南之美者,非徒会稽之竹箭也。又观象云物,察应寒温,原其祸福,与神合契,可谓探赜穷通者也。”会稽东部都尉张纮又与融书曰:“虞仲翔前颇为论者所侵,美宝为质,彫摩益光,不足以损。”

孙权以为骑都尉。翻数犯颜谏争,权不能悦,又性不协俗,多见谤毁,坐徙丹杨泾县。吕蒙图取关羽,称疾还建业,以翻兼知医术,请以自随,亦欲因此令翻得释也。后蒙举军西上,南郡太守麋芳开城出降。蒙未据郡城而作乐沙上,翻谓蒙曰:“今区区一心者麋将军也,城中之人岂可尽信,何不急入城持其管籥乎?”蒙即从之。时城中有伏计,赖翻谋不行。关羽既败,权使翻筮之,得兑下坎上,节,五爻变之临,翻曰:“不出二日,必当断头。”果如翻言。权曰:“卿不及伏羲,可与东方朔为比矣。”

魏将于禁为羽所获,系在城中,权至释之,请与相见。他日,权乘马出,引禁并行,翻呵禁曰:“尔降虏,何敢与吾君齐马首乎!”欲抗鞭击禁,权呵止之。后权于楼船会群臣饮,禁闻乐流涕,翻又曰:“汝欲以伪求免邪?”权怅然不平。

权既为吴王,欢宴之末,自起行酒,翻伏地阳醉,不持。权去,翻起坐。权於是大怒,手剑欲击之,侍坐者莫不惶遽,惟大农刘基起抱权谏曰:“大王以三爵之后杀善士,虽翻有罪,天下孰知之?且大王以能容贤畜众,故海内望风,今一朝弃之,可乎?”权曰:“曹孟德尚杀孔文举,孤於虞翻何有哉?”基曰:“孟德轻害士人,天下非之。大王躬行德义,欲与尧、舜比隆,何得自喻於彼乎?”翻由是得免。权因敕左右,自今酒后言杀,皆不得杀。

翻尝乘船行,与麋芳相逢,芳船上人多欲令翻自避,先驱曰:“避将军船!”翻厉声曰:“失忠与信,何以事君?倾人二城,而称将军,可乎?”芳阖户不应而遽避之。后翻乘车行,又经芳营门,吏闭门,车不得过。翻复怒曰:“当闭反开,当开反闭,岂得事宜邪?”芳闻之,有惭色。

翻性疏直,数有酒失。权与张昭论及神仙,翻指昭曰:“彼皆死人,而语神仙,世岂有仙人邪!”权积怒非一,遂徙翻交州。虽处罪放,而讲学不倦,门徒常数百人。又为老子、论语、国语训注,皆传於世。

初,山阴丁览,太末徐陵,或在县吏之中,或众所未识,翻一见之,便与友善,终成显名。

在南十馀年,年七十卒。归葬旧墓,妻子得还。

翻有十一子,第四子汜最知名,永安初,从选曹郎为散骑中常侍,后为监军使者,讨扶严,病卒。汜弟忠,宜都太守;耸,越骑校尉,累迁廷尉,湘东、河间太守;昺,廷尉尚书,济阴太守。

陆绩字公纪,吴郡吴人也。父康,汉末为庐江太守。绩年六岁,於九江见袁术。术出橘,绩怀三枚,去,拜辞堕地,术谓曰:“陆郎作宾客而怀橘乎?”绩跪答曰:“欲归遗母。”术大奇之。孙策在吴,张昭、张纮、秦松为上宾,共论四海未泰,须当用武治而平之,绩年少末坐,遥大声言曰:“昔管夷吾相齐桓公,九合诸侯,一匡天下,不用兵车。孔子曰:‘远人不服,则脩文德以来之。’今论者不务道德怀取之术,而惟尚武,绩虽童蒙,窃所未安也。”昭等异焉。

绩容貌雄壮,博学多识,星历算数无不该览。虞翻旧齿名盛,庞统荆州令士,年亦差长,皆与绩友善。孙权统事,辟为奏曹掾,以直道见惮,出为郁林太守,加偏将军,给兵二千人。绩既有躄疾,又意存儒雅,非其志也。虽有军事,著述不废,作浑天图,注易释玄,皆传於世。豫自知亡日,乃为辞曰:“有汉志士吴郡陆绩,幼敦诗、书,长玩礼、易,受命南征,遘疾逼厄,遭命不永,呜呼悲隔!”又曰:“从今已去,六十年之外,车同轨,书同文,恨不及见也。”年三十二卒。长子宏,会稽南部都尉,次子叡,长水校尉。

张温字惠恕,吴郡吴人也。父允,以轻财重士,名显州郡,为孙权东曹掾,卒。温少脩节操,容貌奇伟。权闻之,以问公卿曰:“温当今与谁为比?”大农刘基曰:“可与全琮为辈。”太常顾雍曰:“基未详其为人也。温当今无辈。”权曰:“如是,张允不死也。”徵到延见,文辞占对,观者倾竦,权改容加礼。罢出,张昭执其手曰:“老夫讬意,君宜明之。”拜议郎、选曹尚书,徙太子太傅,甚见信重。

时年三十二,以辅义中郎将使蜀。权谓温曰:“卿不宜远出,恐诸葛孔明不知吾所以与曹氏通意,故屈卿行。若山越都除,便欲大构於丕。行人之义,受命不受辞也。”温对曰:“臣入无腹心之规,出无专对之用,惧无张老延誉之功,又无子产陈事之效。然诸葛亮达见计数,必知神虑屈申之宜,加受朝廷天覆之惠,推亮之心,必无疑贰。”温至蜀,诣阙拜章曰:“昔高宗以谅闇昌殷祚於再兴,成王以幼冲隆周德於太平,功冒溥天,声贯罔极。今陛下以聪明之姿,等契往古,总百揆於良佐,参列精之炳耀,遐迩望风,莫不欣赖。吴国勤任旅力,清澄江浒,愿与有道平一宇内,委心协规,有如河水,军事凶烦,使役乏少,是以忍鄙倍之羞,使下臣温通致情好。陛下敦崇礼义,未便耻忽。臣自远境,及即近郊,频蒙劳来,恩诏辄加,以荣自惧,悚怛若惊。谨奉所赍函书一封。”蜀甚贵其才。还,顷之,使入豫章部伍出兵,事业未究。

权既阴衔温称美蜀政,又嫌其声名大盛,众庶炫惑,恐终不为己用,思有以中伤之,会暨艳事起,遂因此发举。艳字子休,亦吴郡人也,温引致之,以为选曹郎,至尚书。艳性狷厉,好为清议,见时郎署混浊淆杂,多非其人,欲臧否区别,贤愚异贯。弹射百僚,覈选三署,率皆贬高就下,降损数等,其守故者十未能一,其居位贪鄙,志节汙卑者,皆以为军吏,置营府以处之。而怨愤之声积,浸润之谮行矣。竞言艳及选曹郎徐彪,专用私情,爱憎不由公理,艳、彪皆坐自杀。温宿与艳、彪同意,数交书疏,闻问往还,即罪温。权幽之有司,下令曰:“昔令召张温,虚己待之,既至显授,有过旧臣,何图凶丑,专挟异心。昔暨艳父兄,附于恶逆,寡人无忌,故进而任之,欲观艳何如。察其中间,形态果见。而温与之结连死生,艳所进退,皆温所为头角,更相表里,共为腹背,非温之党,即就疵瑕,为之生论。又前任温董督三郡,指捴吏客及残馀兵,时恐有事,欲令速归,故授棨戟,奖以威柄。乃便到豫章,表讨宿恶,寡人信受其言,特以绕帐、帐下、解烦兵五千人付之。后闻曹丕自出淮、泗,故豫敕温有急便出,而温悉内诸将,布於深山,被命不至。赖丕自退,不然,已往岂可深计。又殷礼者,本占候召,而温先后乞将到蜀,扇扬异国,为之谭论。又礼之还,当亲本职,而令守尚书户曹郎,如此署置,在温而已。又温语贾原,当荐卿作御史,语蒋康,当用卿代贾原,专衒贾国恩,为己形势。揆其奸心,无所不为。不忍暴於巿朝,今斥还本郡,以给厮吏。呜呼温也,免罪为幸!”

将军骆统表理温曰:“伏惟殿下,天生明德,神启圣心,招髦秀於四方,置俊乂於宫朝。多士既受普笃之恩,张温又蒙最隆之施。而温自招罪谴,孤负荣遇,念其如此,诚可悲疚。然臣周旋之间,为国观听,深知其状,故密陈其理。温实心无他情,事无逆迹,但年纪尚少,镇重尚浅,而戴赫烈之宠,体卓伟之才,亢臧否之谭,效褒贬之议。於是务势者妒其宠,争名者嫉其才,玄默者非其谭,瑕衅者讳其议,此臣下所当详辨,明朝所当究察也。昔贾谊,至忠之臣也,汉文,大明之君也,然而绛、灌一言,贾谊远退。何者?疾之者深,谮之者巧也。然而误闻天下,失彰於后世,故孔子曰'为君难,为臣不易'也。温虽智非从横,武非虓虎,然其弘雅之素,英秀之德,文章之采,论议之辨,卓跞冠群,炜晔曜世,世人未有及之者也。故论温才即可惜,言罪则可恕。若忍威烈以赦盛德,宥贤才以敦大业,固明朝之休光,四方之丽观也。国家之於暨艳,不内之忌族,犹等之平民,是故先见用於朱治,次见举於众人,中见任於明朝,亦见交於温也。君臣之义,义之最重,朋友之交,交之最轻者也。国家不嫌於艳为最重之义,是以温亦不嫌与艳为最轻之交也。时世宠之於上,温窃亲之於下也。夫宿恶之民,放逸山险,则为劲寇,将置平土,则为健兵,故温念在欲取宿恶,以除劲寇之害,而增健兵之锐也。但自错落,功不副言。然计其送兵,以比许晏,数之多少,温不减之,用之强羸,温不下之,至於迟速,温不后之,故得及秋冬之月,赴有警之期,不敢忘恩而遗力也。温之到蜀,共誉殷礼,虽臣无境外之交,亦有可原也。境外之交,谓无君命而私相从,非国事而阴相闻者也;若以命行,既脩君好,因叙己情,亦使臣之道也。故孔子使邻国,则有私觌之礼;季子聘诸夏,亦有燕谭之义也。古人有言,欲知其君,观其所使,见其下之明明,知其上之赫赫。温若誉礼,能使彼叹之,诚所以昭我臣之多良,明使之得其人,显国美於异境,扬君命於他邦。是以晋赵文子之盟于宋也,称随会於屈建;楚王孙圉之使于晋也,誉左史於赵鞅。亦向他国之辅,而叹本邦之臣,经传美之以光国,而不讥之以外交也。王靖内不忧时,外不趋事,温弹之不私,推之不假,於是与靖遂为大怨,此其尽节之明验也。靖兵众之势,幹任之用,皆胜於贾原、蒋康,温尚不容私以安於靖,岂敢卖恩以协原、康邪?又原在职不勤,当事不堪,温数对以丑色,弹以急声;若其诚欲卖恩作乱,则亦不必贪原也。凡此数者,校之於事既不合,参之於众亦不验。臣窃念人君虽有圣哲之姿,非常之智,然以一人之身,御兆民之众,从层宫之内,瞰四国之外,照群下之情,求万机之理,犹未易周也,固当听察群下之言,以广聪明之烈。今者人非温既殷勤,臣是温又契阔,辞则俱巧,意则俱至,各自言欲为国,谁其言欲为私,仓卒之间,犹难即别。然以殿下之聪叡,察讲论之曲直,若潜神留思,纤粗研核,情何嫌而不宣,事何昧而不昭哉?温非亲臣,臣非爱温者也。昔之君子,皆抑私忿,以增君明。彼独行之於前,臣耻废之於后,故遂发宿怀於今日,纳愚言於圣听,实尽心於明朝,非有念於温身也。”权终不纳。

后六年,温病卒。二弟祗、白,亦有才名,与温俱废。

骆统字公绪,会稽乌伤人也。父俊,官至陈相,为袁术所害。统母改適,为华歆小妻,统时八岁,遂与亲客归会稽。其母送之,拜辞上车,面而不顾,其母泣涕於后。御者曰:“夫人犹在也。”统曰:“不欲增母思,故不顾耳。”事適母甚谨。时饥荒,乡里及远方客多有困乏,统为之饮食衰少。其姊仁爱有行,寡归无子,见统甚哀之,数问其故。统曰:“士大夫糟糠不足,我何心独饱!”姊曰:“诚如是,何不告我,而自苦若此?”乃自以私粟与统,又以告母,母亦贤之,遂使分施,由是显名。

孙权以将军领会稽太守,统年二十,试为乌程相,民户过万,咸叹其惠理。权嘉之,召为功曹,行骑都尉,妻以从兄辅女。统志在补察,苟所闻见,夕不待旦。常劝权以尊贤接士,勤求损益,飨赐之日,可人人别进,问其燥湿,加以密意,诱谕使言,察其志趣,令皆感恩戴义,怀欲报之心。权纳用焉。出为建忠中郎将,领武射吏三千人。及凌统死,复领其兵。

是时徵役繁数,重以疫疠,民户损耗,统上疏曰:“臣闻君国者,以据疆土为强富,制威福为尊贵,曜德义为荣显,永世胤为丰祚。然财须民生,强赖民力,威恃民势,福由民殖,德俟民茂,义以民行,六者既备,然后应天受祚,保族宜邦。书曰:‘众非后无能胥以宁,后非众无以辟四方。’推是言之,则民以君安,君以民济,不易之道也。今强敌未殄,海内未乂,三军有无已之役,江境有不释之备,徵赋调数,由来积纪,加以殃疫死丧之灾,郡县荒虚,田畴芜旷,听闻属城,民户浸寡,又多残老,少有丁夫,闻此之日,心若焚燎。思寻所由,小民无知,既有安土重迁之性,且又前后出为兵者,生则困苦无有温饱,死则委弃骸骨不反,是以尤用恋本畏远,同之於死。每有徵发,羸谨居家重累者先见输送。小有财货,倾居行赂,不顾穷尽。轻剽者则迸入险阻,党就群恶。百姓虚竭,嗷然愁扰,愁扰则不营业,不营业则致穷困,致穷困则不乐生,故口腹急,则奸心动而携叛多也。又闻民间,非居处小能自供,生产儿子,多不起养;屯田贫兵,亦多弃子。天则生之,而父母杀之,既惧干逆和气,感动阴阳。且惟殿下开基建国,乃无穷之业也,强邻大敌非造次所灭,疆埸常守非期月之戍,而兵民减耗,后生不育,非所以历远年,致成功也。夫国之有民,犹水之有舟,停则以安,扰则以危,愚而不可欺,弱而不可胜,是以圣王重焉,祸福由之,故与民消息,观时制政。方今长吏亲民之职,惟以办具为能,取过目前之急,少复以恩惠为治,副称殿下天覆之仁,勤恤之德者。官民政俗,日以彫弊,渐以陵迟,势不可久。夫治疾及其未笃,除患贵其未深,愿殿下少以万机馀间,留神思省,补复荒虚,深图远计,育残馀之民,阜人财之用,参曜三光,等崇天地。臣统之大愿,足以死而不朽矣。”权感统言,深加意焉。

以随陆逊破蜀军於宜都,迁偏将军。黄武初,曹仁攻濡须,使别将常雕等袭中洲,统与严圭共拒破之,封新阳亭侯,后为濡须督。数陈便宜,前后书数十上,所言皆善,文多故不悉载。尤以占募在民间长恶败俗,生离叛之心,急宜绝置,权与相反覆,终遂行之。年三十六,黄武七年卒。

陆瑁字子璋,丞相逊弟也。少好学笃义。陈国陈融、陈留濮阳逸、沛郡蒋纂、广陵袁迪等,皆单贫有志,就瑁游处,瑁割少分甘,与同丰约。及同郡徐原,爰居会稽,素不相识,临死遗书,讬以孤弱,瑁为起立坟墓,收导其子。又瑁从父绩早亡,二男一女,皆数岁以还,瑁迎摄养,至长乃别。州郡辟举,皆不就。

时尚书暨艳盛明臧否,差断三署,颇扬人闇昧之失,以显其谪。瑁与书曰:“夫圣人嘉善矜愚,忘过记功,以成美化。加今王业始建,将一大统,此乃汉高弃瑕录用之时也,若令善恶异流,贵汝颍月旦之评,诚可以厉俗明教,然恐未易行也。宜远模仲尼之汎爱,中则郭泰之弘济,近有益於大道也。”艳不能行,卒以致败。

嘉禾元年,公车徵瑁,拜议郎、选曹尚书。孙权忿公孙渊之巧诈反覆,欲亲征之,瑁上疏谏曰:“臣闻圣王之御远夷,羁縻而已,不常保有,故古者制地,谓之荒服,言慌惚无常,不可保也。今渊东夷小丑,屏在海隅,虽讬人面,与禽兽无异。国家所为不爱货宝远以加之者,非嘉其德义也,诚欲诱纳愚弄,以规其马耳。渊之骄黠,恃远负命,此乃荒貊常态,岂足深怪?昔汉诸帝亦尝锐意以事外夷,驰使散货,充满西域,虽时有恭从,然其使人见害,财货并没,不可胜数。今陛下不忍悁悁之忿,欲越巨海,身践其土,群臣愚议,窃谓不安。何者?北寇与国,壤地连接,苟有间隙,应机而至。夫所以越海求马,曲意於渊者,为赴目前之急,除腹心之疾也,而更弃本追末,捐近治远,忿以改规,激以动众,斯乃猾虏所愿闻,非大吴之至计也。又兵家之术,以功役相疲,劳逸相待,得失之间,所觉辄多。且沓渚去渊,道里尚远,今到其岸,兵势三分,使强者进取,次当守船,又次运粮,行人虽多,难得悉用;加以单步负粮,经远深入,贼地多马,邀截无常。若渊狙诈,与北未绝,动众之日,唇齿相济。若实孑然无所凭赖,其畏怖远迸,或难卒灭。使天诛稽於朔野,山虏承间而起,恐非万安之长虑也。”权未许。

瑁重上疏曰:“夫兵革者,固前代所以诛暴乱,威四夷也,然其役皆在奸雄已除,天下无事,从容庙堂之上,以馀议议之耳。至于中夏鼎沸,九域槃互之时,率须深根固本,爱力惜费,务自休养,以待邻敌之阙,未有正於此时,舍近治远,以疲军旅者也。昔尉佗叛逆,僣号称帝,于时天下乂安,百姓殷阜,带甲之数,粮食之积,可谓多矣,然汉文犹以远征不易,重兴师旅,告喻而已。今凶桀未殄,疆埸犹警,虽蚩尤、鬼方之乱,故当以缓急差之,未宜以渊为先。愿陛下抑威住计,暂宁六师,潜神嘿规,以为后图,天下幸甚。”权再览瑁书,嘉其词理端切,遂不行。

初,瑁同郡闻人敏见待国邑,优於宗脩,惟瑁以为不然,后果如其言。

赤乌二年,瑁卒。子喜亦涉文籍,好人伦,孙皓时为选曹尚书。

吾粲字孔休,吴郡乌程人也。孙河为县长,粲为小吏,河深奇之。河后为将军,得自选长吏,表粲为曲阿丞,迁为长史,治有名迹。虽起孤微,与同郡陆逊、卜静等比肩齐声矣。孙权为车骑将军,召为主簿,出为山阴令,还为参军校尉。

黄武元年,与吕范、贺齐等俱以舟师拒魏将曹休於洞口。值天大风,诸船绠绁断绝,漂没著岸,为魏军所获,或覆没沈溺,其大船尚存者,水中生人皆攀缘号呼,他吏士恐船倾没,皆以戈矛撞击不受。粲与黄渊独令船人以承取之,左右以为船重必败,粲曰:“船败,当俱死耳!人穷,奈何弃之。”粲、渊所活者百馀人。

还,迁会稽太守,召处士谢谭为功曹,谭以疾不诣,粲教曰:“夫应龙以屈伸为神,凤皇以嘉鸣为贵,何必隐形於天外,潜鳞於重渊者哉?”粲募合人众,拜昭义中郎将,与吕岱讨平山越,入为屯骑校尉、少府,迁太子太傅。遭二宫之变,抗言执正,明嫡庶之分,欲使鲁王霸出驻夏口,遣杨竺不得令在都邑。又数以消息语陆逊,逊时驻武昌,连表谏争。由此为霸、竺等所谮害,下狱诛。

朱据字子范,吴郡吴人也。有姿貌膂力,又能论难。黄武初,徵拜五官郎中,补侍御史。是时选曹尚书暨艳,疾贪汙在位,欲沙汰之。据以为天下未定,宜以功覆过,弃瑕取用,举清厉浊,足以沮劝,若一时贬黜,惧有后咎。艳不听,卒败。

权咨嗟将率,发愤叹息,追思吕蒙、张温,以为据才兼文武,可以继之,自是拜建义校尉,领兵屯湖孰。黄龙元年,权迁都建业,徵据尚公主,拜左将军,封云阳侯。谦虚接士,轻财好施,禄赐虽丰而常不足用。嘉禾中,始铸大钱,一当五百。后据部曲应受三万缗,工王遂诈而受之,典校吕壹疑据实取,考问主者,死於杖下,据哀其无辜,厚棺敛之。壹又表据吏为据隐,故厚其殡。权数责问据,据无以自明,藉草待罪。数月,典军吏刘助觉,言王遂所取,权大感寤,曰:“朱据见枉,况吏民乎?”乃穷治壹罪,赏助百万。

赤乌九年,迁骠骑将军。遭二宫构争,据拥护太子,言则恳至,义形于色,守之以死,遂左迁新都郡丞。未到,中书令孙弘谮润据,因权寝疾,弘为昭书追赐死,时年五十七。孙亮时,二子熊、损各复领兵,为全公主所谮,皆死。永安中,追录前功,以熊子宣袭爵云阳侯,尚公主。孙皓时,宣至骠骑将军。

评曰:虞翻古之狂直,固难免乎末世,然权不能容,非旷宇也。陆绩之於扬玄,是仲尼之左丘明,老聃之严周矣;以瑚琏之器,而作守南越,不亦贼夫人欤!张温才藻俊茂,而智防未备,用致艰患。骆统抗明大义,辞切理至,值权方闭不开。陆瑁笃义规谏,君子有称焉。吾粲、朱据遭罹屯蹇,以正丧身,悲夫!

\part{吴书十三}
\chapter{陆逊传第十三}

陆逊字伯言,吴郡吴人也。本名议,世江东大族。逊少孤,随从祖庐江太守康在官。袁术与康有隙,将攻康,康遣逊及亲戚还吴。逊年长於康子绩数岁,为之纲纪门户。

孙权为将军,逊年二十一,始仕幕府,历东西曹令史,出为海昌屯田都尉,并领县事。县连年亢旱,逊开仓谷以振贫民,劝督农桑,百姓蒙赖。时吴、会稽、丹杨多有伏匿,逊陈便宜,乞与募焉。会稽山贼大帅潘临,旧为所在毒害,历年不禽。逊以手下召兵,讨治深险,所向皆服,部曲已有二千馀人。鄱阳贼帅尤突作乱,复往讨之,拜定威校尉,军屯利浦。

权以兄策女配逊,数访世务,逊建议曰:“方今英雄棋跱,财狼闚望,克敌宁乱,非众不济。而山寇旧恶,依阻深地。夫腹心未平,难以图远,可大部伍,取其精锐。”权纳其策,以为帐下右部督。会丹杨贼帅费栈受曹公印绶,扇动山越,为作内应,权遣逊讨栈。栈支党多而往兵少,逊乃益施牙幢,分布鼓角,夜潜山谷间,鼓噪而前,应时破散。遂部伍东三郡,强者为兵,羸者补户,得精卒数万人,宿恶荡除,所过肃清,还屯芜湖。

会稽太守淳于式表逊枉取民人,愁扰所在。逊后诣都,言次,称式佳吏,权曰:“式白君而君荐之,何也?”逊对曰:“式意欲养民,是以白逊。若逊复毁式以乱圣听,不可长也。”权曰:“此诚长者之事,顾人不能为耳。”

吕蒙称疾诣建业,逊往见之,谓曰:“关羽接境,如何远下,后不当可忧也?”蒙曰:“诚如来言,然我病笃。”逊曰:“羽矜其骁气,陵轹於人。始有大功,意骄志逸,但务北进,未嫌於我,有相闻病,必益无备。今出其不意,自可禽制。下见至尊,宜好为计。”蒙曰:“羽素勇猛,既难为敌,且已据荆州,恩信大行,兼始有功,胆势益盛,未易图也。”蒙至都,权问:“谁可代卿者?”蒙对曰:“陆逊意思深长,才堪负重,观其规虑,终可大任。而未有远名,非羽所忌,无复是过。若用之,当令外自韬隐,内察形便,然后可克。”权乃召逊,拜偏将车右部督代蒙。

逊至陆口,书与羽曰:“前承观衅而动,以律行师,小举大克,一何巍巍!敌国败绩,利在同盟,闻庆拊节,想遂席卷,共奖王纲。近以不敏,受任来西,延慕光尘,思禀良规。”又曰:“于禁等见获,遐迩欣叹,以为将军之勋足以长世,虽昔晋文城濮之师,淮阴拔赵之略,蔑以尚兹。闻徐晃等少骑驻旌,闚望麾葆。操猾虏也,忿不思难,恐潜增众,以逞其心。虽云师老,犹有骁悍。且战捷之后,常苦轻敌,古人杖术,军胜弥警,愿将军广为方计,以全独克。仆书生疏迟,忝所不堪,喜邻威德,乐自倾尽,虽未合策,犹可怀也。傥明注仰,有以察之。”羽览逊书,有谦下自讬之意,意大安,无复所嫌。逊具启形状,陈其可禽之要。权乃潜军而上,使逊与吕蒙为前部,至即克公安、南郡。逊径进,领宜都太守,拜抚边将军,封华亭侯。备宜都太守樊友委郡走,诸城长吏及蛮夷君长皆降。逊请金银铜印,以假授初附。是岁建安二十四年十一月也。

逊遣将军李异、谢旌等将三千人,攻蜀将詹晏、陈凤。异将水军,旌将步兵,断绝险要,即破晏等,生降得凤。又攻房陵太守邓辅、南乡太守郭睦,大破之。秭归大姓文布、邓凯等合夷兵数千人,首尾西方。逊复部旌讨破布、凯。布、凯脱走,蜀以为将。逊令人诱之,布帅众还降。前后斩获招纳,凡数万计。权以逊为右护军、镇西将军,进封娄侯。

时荆州士人新还,仕进或未得所,逊上疏曰:“昔汉高受命,招延英异,光武中兴,群俊毕至,苟可以熙隆道教者,未必远近。今荆州始定,人物未达,臣愚慺慺,乞普加覆载抽拔之恩,令并获自进,然后四海延颈,思归大化。”权敬纳其言。

黄武元年,刘备率大众来向西界,权命逊为大都督、假节,督朱然、潘璋、宋谦、韩当、徐盛、鲜于丹、孙桓等五万人拒之。备从巫峡、建平连围至夷陵界,立数十屯,以金锦爵赏诱动诸夷,使将军冯习为大督,张南为前部,辅匡、赵融、廖淳、傅肜等各为别督,先遣吴班将数千人於平地立营,欲以挑战。诸将皆欲击之,逊曰:“此必有谲,且观之。”备知其计不可,乃引伏兵八千,从谷中出。逊曰:“所以不听诸君击班者,揣之必有巧故也。“逊上疏曰:“夷陵要害,国之关限,虽为易得,亦复易失。失之非徒损一郡之地,荆州可忧。今日争之,当令必谐。备干天常,不守窟穴,而敢自送。臣虽不材,凭奉威灵,以顺讨逆,破坏在近。寻备前后行军,多败少成,推此论之,不足为戚。臣初嫌之,水陆俱进,今反舍船就步,处处结营,察其布置,必无他变。伏愿至尊高枕,不以为念也。”诸将并曰:“攻备当在初,今乃令入五六百里,相衔持经七八月,其诸要害皆以固守,击之必无利矣。”逊曰:“备是猾虏,更尝事多,其军始集,思虑精专,未可干也。今住已久,不得我便,兵疲意沮,计不复生,犄角此寇,正在今日。”乃先攻一营,不利。诸将皆曰:“空杀兵耳。”逊曰:“吾已晓破之之术。”乃敕各持一把茅,以火攻拔之。一尔势成,通率诸军同时俱攻,斩张南、冯习及胡王沙摩柯等首,破其四十馀营。备将杜路、刘宁等穷逼请降。备升马鞍山,陈兵自绕。逊督促诸军四面蹙之,土崩瓦解,死者万数。备因夜遁,驿人自担烧铙铠断后,仅得入白帝城。其舟船器械,水步军资,一时略尽,尸骸漂流,塞江而下。备大惭恚,曰:“吾乃为逊所折辱,岂非天邪!”

初,孙桓别讨备前锋於夷道,为备所围,求救於逊。逊曰:“未可。”诸将曰:“孙安东公族,见围已困,奈何不救?”逊曰:“安东得士众心,城牢粮足,无可忧也。待吾计展,欲不救安东,安东自解。”及方略大施,备果奔溃。桓后见逊曰:“前实怨不见救,定至今日,乃知调度自有方耳。”

当御备时,诸将军或是孙策时旧将,或公室贵戚,各自矜恃,不相听从。逊案剑曰:“刘备天下知名,曹操所惮,今在境界,此强对也。诸君并荷国恩,当相辑睦,共翦此虏,上报所受,而不相顺,非所谓也。仆虽书生,受命主上。国家所以屈诸君使相承望者,以仆有尺寸可称,能忍辱负重故也。各任其事,岂复得辞!军令有常,不可犯矣。”及至破备,计多出逊,诸将乃服。权闻之,曰:“君何以初不启诸将违节度者邪?”逊对曰:“受恩深重,任过其才。又此诸将或任腹心,或堪爪牙,或是功臣,皆国家所当与共克定大事者。臣虽驽懦,窃慕相如、寇恂相下之义,以济国事。”权大笑称善,加拜逊辅国将军,领荆州牧,即改封江陵侯。

又备既住白帝,徐盛、潘璋、宋谦等各竞表言备必可禽,乞复攻之。权以问逊,逊与朱然、骆统以为曹丕大合士众,外讬助国讨备,内实有奸心,谨决计辄还。无几,魏军果出,三方受敌也。

备寻病亡,子禅袭位,诸葛亮秉政,与权连和。时事所宜,权辄令逊语亮,并刻权印,以置逊所。权每与禅、亮书,常过示逊,轻重可否,有所不安,便令改定,以印封行之。

七年,权使鄱阳太守周鲂谲魏大司马曹休,休果举众入皖,乃召逊假黄钺,为大都督,逆休。休既觉知,耻见欺诱,自恃兵马精多,遂交战。逊自为中部,令朱桓、全琮为左右翼,三道俱进,果冲休伏兵,因驱走之,追亡逐北,径至夹石,斩获万馀,牛马骡驴车乘万两,军资器械略尽。休还,疽发背死。诸军振旅过武昌,权令左右以御盖覆逊,入出殿门,凡所赐逊,皆御物上珍,於时莫与为比。遣还西陵。

黄龙元年,拜上大将军、右都护。是岁,权东巡建业,留太子、皇子及尚书九官,徵逊辅太子,并掌荆州及豫章三郡事,董督军国。时建昌侯虑於堂前作斗鸭栏,颇施小巧,逊正色曰:“君侯宜勤览经典以自新益,用此何为?”虑即时毁彻之。射声校尉松於公子中最亲,戏兵不整,逊对之髡其职吏。南阳谢景善刘廙先刑后礼之论,逊呵景曰:“礼之长於刑久矣,廙以细辩而诡先圣之教,皆非也。君今侍东宫,宜遵仁义以彰德音,若彼之谈,不须讲也。”

逊虽身在外,乃心於国,上疏陈时事曰:“臣以为科法严峻,下犯者多。顷年以来,将吏罹罪,虽不慎可责,然天下未一,当图进取,小宜恩贷,以安下情。且世务日兴,良能为先,自非奸秽入身,难忍之过,乞复显用,展其力效。此乃圣王忘过记功,以成王业。昔汉高舍陈平之愆,用其奇略,终建勋祚,功垂千载。夫峻法严刑,非帝王之隆业;有罚无恕,非怀远之弘规也。”

权欲遣偏师取夷州及朱崖,皆以谘逊,逊上疏曰:“臣愚以为四海未定,当须民力,以济时务。今兵兴历年,见众损减,陛下忧劳圣虑,忘寝与食,将远规夷州,以定大事,臣反覆思惟,未见其利,万里袭取,风波难测,民易水土,必致疾疫,今驱见众,经涉不毛,欲益更损,欲利反害。又珠崖绝险,民犹禽兽,得其民不足济事,无其兵不足亏众。今江东见众,自足图事,但当畜力而后动耳。昔桓王创基,兵不一旅,而开大业。陛下承运,拓定江表。臣闻治乱讨逆,须兵为威,农桑衣食,民之本业,而干戈未戢,民有饥寒。臣愚以为宜育养士民,宽其租赋,众克在和,义以劝勇,则河渭可平,九有一统矣。”权遂征夷州,得不补失。

及公孙渊背盟,权欲往征,逊上疏曰:“渊凭险恃固,拘留大使,名马不献,实可雠忿。蛮夷猾夏,未染王化,鸟窜荒裔,拒逆王师,至令陛下爰赫斯怒,欲劳万乘汎轻越海,不虑其危而涉不测。方今天下云扰,群雄虎争,英豪踊跃,张声大视。陛下以神武之姿,诞膺期运,破操乌林,败备西陵,禽羽荆州,斯三虏者当世雄杰,皆摧其锋。圣化所绥,万里草偃,方荡平华夏,总一大猷。今不忍小忿,而发雷霆之怒,违垂堂之戒,轻万乘之重,此臣之所惑也。臣闻志行万里者,不中道而辍足;图四海者,匪怀细以害大。强寇在境,荒服未庭,陛下乘桴远征,必致闚,慼至而忧,悔之无及。若使大事时捷,则渊不讨自服;今乃远惜辽东众之与马,奈何独欲捐江东万安之本业而不惜乎?乞息六师,以威大虏,早定中夏,垂耀将来。”权用纳焉。

嘉禾五年,权北征,使逊与诸葛瑾攻襄阳。逊遣亲人韩扁赍表奉报,还,遇敌於沔中,钞逻得扁。瑾闻之甚惧,书与逊云:“大驾已旋,贼得韩扁,具知吾阔狭。且水乾,宜当急去。”逊未答,方催人种葑豆,与诸将弈棋射戏如常。瑾曰:“伯言多智略,其当有以。”自来见逊,逊曰:“贼知大驾以旋,无所复慼,得专力於吾。又已守要害之处,兵将意动,且当自定以安之,施设变术,然后出耳。今便示退,贼当谓吾怖,仍来相蹙,必败之势也。”乃密与瑾立计,令瑾督舟船,逊悉上兵马,以向襄阳城。敌素惮逊,遽还赴城。瑾便引船出,逊徐整部伍,张拓声势,步趋船,敌不敢干。军到白围,讬言住猎,潜遣将军周峻、张梁等击江夏新市、安陆、石阳,石阳市盛,峻等奄至,人皆捐物入城。城门噎不得关,敌乃自斫杀己民,然后得阖。斩首获生,凡千馀人。其所生得,皆加营护,不令兵士干扰侵侮。将家属来者,使就料视。若亡其妻子者,即给衣粮,厚加慰劳,发遣令还,或有感慕相携而归者。邻境怀之,江夏功曹赵濯、弋阳备将裴生及夷王梅颐等,并帅支党来附逊。逊倾财帛,周赡经恤。

又魏江夏太守逯式兼领兵马,颇作边害,而与北旧将文聘子休宿不协。逊闻其然,即假作答式书云:“得报恳恻,知与休久结嫌隙,势不两存,欲来归附,辄以密呈来书表闻,撰众相迎。宜潜速严,更示定期。”以书置界上,式兵得书以见式,式惶惧,遂自送妻子还洛。由是吏士不复亲附,遂以免罢。

六年,中郎将周祗乞於鄱阳召募,事下问逊。逊以为此郡民易动难安,不可与召,恐致贼寇。而祗固陈取之,郡民吴遽等果作贼杀祗,攻没诸县。豫章、庐陵宿恶民,并应遽为寇。逊自闻,辄讨即破,遽等相率降,逊料得精兵八千馀人,三郡平。

时中书典校吕壹,窃弄权柄,擅作威福,逊与太常潘濬同心忧之,言至流涕。后权诛壹,深以自责,语在权传。

时谢渊、谢厷等各陈便宜,欲兴利改作,以事下逊。逊议曰:“国以民为本,强由民力,财由民出。夫民殷国弱,民瘠国强者,未之有也。故为国者,得民则治,失之则乱,若不受利,而令尽用立效,亦为难也。是以诗叹'宜民宜人,受禄于天'。乞垂圣恩,宁济百姓,数年之间,国用少丰,然后更图。”

赤乌七年,代顾雍为丞相,诏曰:“朕以不德,应期践运,王涂未一,奸宄充路,夙夜战惧,不惶鉴寐。惟君天资聪叡,明德显融,统任上将,匡国弭难。夫有超世之功者,必应光大之宠;怀文武之才者,必荷社稷之重。昔伊尹隆汤,吕尚翼周,内外之任,君实兼之。今以君为丞相,使使持节守太常傅常授印绶。君其茂昭明德,脩乃懿绩,敬服王命,绥靖四方。於乎!总司三事,以训群寮,可不敬与,君其勖之!其州牧都护领武昌事如故。”

先是,二宫并阙,中外职司,多遣子弟给侍。全琮报逊,逊以为子弟苟有才,不忧不用,不宜私出以要荣利;若其不佳,终为取祸。且闻二宫势敌,必有彼此,此古人之厚忌也。琮子寄,果阿附鲁王,轻为交构。逊书与琮曰:“卿不师日䃅,而宿留阿寄,终为足下门户致祸矣。”琮既不纳,更以致隙。及太子有不安之议,逊上疏陈:“太子正统,宜有盘石之固,鲁王藩臣,当使宠秩有差,彼此得所,上下获安。谨叩头流血以闻。”书三四上,及求诣都,欲口论適庶之分,以匡得失。既不听许,而逊外生顾谭、顾承、姚信,并以亲附太子,枉见流徙。太子太傅吾粲坐数与逊交书,下狱死。权累遣中使责让逊,逊愤恚致卒,时年六十三,家无馀财。

初,暨艳造营府之论,逊谏戒之,以为必祸。又谓诸葛恪曰:“在我前者,吾必奉之同升;在我下者,则扶持之。今观君气陵其上,意蔑乎下,非安德之基也。”又广陵杨竺少获声名,而逊谓之终败,劝竺兄穆令与别族。其先睹如此。长子延早夭,次子抗袭爵。孙休时,追谥逊曰昭侯。

抗字幼节,孙策外孙也。逊卒时,年二十,拜建武校尉,领逊众五千人,送葬东还,诣都谢恩。孙权以杨竺所白逊二十事问抗,禁绝宾客,中使临诘,抗无所顾问,事事条答,权意渐解。赤乌九年,迁立节中郎将,与诸葛恪换屯柴桑。抗临去,皆更缮完城围,葺其墙屋,居庐桑果,不得妄败。恪入屯,俨然若新。而恪柴桑故屯,颇有毁坏,深以为惭。太元元年,就都治病。病差当还,权涕泣与别,谓曰:“吾前听用谗言,与汝父大义不笃,以此负汝。前后所问,一焚灭之,莫令人见也。”建兴元年,拜奋威将军。太平二年,魏将诸葛诞举寿春降,拜抗为柴桑督,赴寿春,破魏牙门将偏将军,迁征北将军。永安二年,拜镇军将军,都督西陵,自关羽至白帝。三年,假节。孙皓即位,加镇军大将军,领益州牧。建衡二年,大司马施绩卒,拜抗都督信陵、西陵、夷道、乐乡,公安诸军事,治乐乡。

抗闻都下政令多阙,忧深虑远,乃上疏曰:“臣闻德均则众者胜寡,力侔则安者制危,盖六国所以兼并於强秦,西楚所以北面於汉高也。今敌跨制九服,非徒关右之地;割据九州,岂但鸿沟以西而已。国家外无连国之援,内非西楚之强,庶政陵迟,黎民未乂,而议者所恃,徒以长川峻山,限带封域,此乃守国之末事,非智者之所先也。臣每远惟战国存亡之符,近览刘氏倾覆之衅,考之典籍,验之行事,中夜抚枕,临餐忘食。昔匈奴未灭,去病辞馆;汉道未纯,贾生哀泣。况臣王室之出,世荷光宠,身名否泰,与国同慼,死生契阔,义无苟且,夙夜忧怛,念至情惨。夫事君之义犯而勿欺,人臣之节匪躬是殉,谨陈时宜十七条如左。”十七条失本,故不载。

时何定弄权,阉官预政;抗上疏曰:“臣闻开国承家,小人勿用,靖谮庸回,唐书攸戒,是以雅人所以怨刺,仲尼所以叹息也。春秋已来,爰及秦、汉,倾覆之衅,未有不由斯者也。小人不明理道,所见既浅,虽使竭情尽节,犹不足任,况其奸心素笃,而憎爱移易哉?苟患失之,无所不至。今委以聪明之任,假以专制之威,而冀雍熙之声作,肃清之化立,不可得也。方今见吏,殊才虽少,然或冠冕之胄,少渐道教,或清苦自立,资能足用,自可随才授职,抑黜群小,然后俗化可清,庶政无秽也。”

凤皇元年,西陵督步阐据城以叛,遣使降晋。抗闻之,日部分诸军,令将军左奕、吾彦、蔡贡等径赴西陵,敕军营更筑严围,自赤谿至故市,内以围阐,外以御寇,昼夜催切,如敌以至,众甚苦之。诸将咸谏曰:“今及三军之锐,亟以攻阐,比晋救至,阐必可拔。何事於围,而以弊士民之力乎?”抗曰:“此城处势既固,粮谷又足,且所缮修备御之具,皆抗所宿规。今反身攻之,既非可卒克,且北救必至,至而无备,表里受难,何以御之?”诸将咸欲攻阐,抗每不许。宜都太守雷谭言至恳切,抗欲服众,听令一攻。攻果无利,围备始合。晋车骑将军羊祜率师向江陵,诸将咸以抗不宜上,抗曰:“江陵城固兵足,无所忧患。假令敌没江陵,必不能守,所损者小。如使西陵槃结,则南山群夷皆当扰动,则所忧虑,难可竟言也。吾宁弃江陵而赴西陵,况江陵牢固乎?”初,江陵平衍,道路通利,抗敕江陵督张咸作大堰遏水,渐渍平中,以绝寇叛。祜欲因所遏水,浮船运粮,扬声将破堰以通步车。抗闻,使咸亟破之。诸将皆惑,屡谏不听。祜至当阳,闻堰败,乃改船以车运,大费损功力。晋巴东监军徐胤率水军诣建平,荆州刺史杨肇至西陵。抗令张咸固守其城;公安督孙遵巡南岸御祜;水军督留虑、镇西将军朱琬拒胤;身率三军,凭围对肇。将军朱乔、营都督俞赞亡诣肇。抗曰:“赞军中旧吏,知吾虚实者,吾常虑夷兵素不简练,若敌攻围,必先此处。”即夜易夷民,皆以旧将充之。明日,肇果攻故夷兵处,抗命旋军击之,矢石雨下,肇众伤死者相属。肇至经月,计屈夜遁。抗欲追之,而虑阐畜力项领,伺视间隙,兵不足分,於是但鸣鼓戒众,若将追者。肇众凶惧,悉解甲挺走,抗使轻兵蹑之,肇大破败,祜等皆引军还。抗遂陷西陵城,诛夷阐族及其大将吏,自此以下,所请赦者数万口。脩治城围,东还乐乡,貌无矜色,谦冲如常,故得将士欢心。

加拜都护。闻武昌左部督薛莹徵下狱,抗上疏曰:“夫俊乂者,国家之良宝,社稷之贵资,庶政所以伦叙,四门所以穆清也。故大司农楼玄、散骑中常侍王蕃、少府李勖,皆当世秀颖,一时显器,既蒙初宠,从容列位,而并旋受诛殛,或圮族替祀,或投弃荒裔。盖周礼有赦贤之辟,春秋有宥善之义,书曰:'与其杀不辜,宁失不经。'而蕃等罪名未定,大辟以加,心经忠义,身被极刑,岂不痛哉!且已死之刑,固无所识,至乃焚烁流漂,弃之水滨,惧非先王之正典,或甫侯之所戒也。是以百姓哀耸,士民同慼。蕃、勖永已,悔亦靡及,诚望陛下赦召玄出,而顷闻薛莹卒见逮录。莹父综纳言先帝,傅弼文皇,及莹承基,内厉名行,今之所坐,罪在可宥。臣惧有司未详其事,如复诛戮,益失民望,乞垂天恩,原赦莹罪,哀矜庶狱,清澄刑网,则天下幸甚!”

时师旅仍动,百姓疲弊,抗上疏曰:“臣闻易贵随时,传美观衅,故有夏多罪而殷汤用师,纣作淫虐而周武授钺。苟无其时,玉台有忧伤之虑,孟津有反旆之军。今不务富国强兵,力农畜谷,使文武之才效展其用,百揆之署无旷厥职,明黜陟以厉庶尹,审刑罚以示劝沮,训诸司以德,而抚百姓以仁,然后顺天乘运,席卷宇内,而听诸将徇名,穷兵黩武,动费万计,士卒彫瘁,寇不为衰,而我已大病矣!今争帝王之资,而昧十百之利,此人臣之奸便,非国家之良策也。昔齐鲁三战,鲁人再克而亡不旋踵。何则?大小之势异也。况今师所克获,不补所丧哉?且阻兵无众,古之明鉴,诚宜蹔息进取小规,以畜士民之力,观衅伺隙,庶无悔吝。”

二年春,就拜大司马、荆州牧。三年夏,疾病,上疏曰:“西陵、建平,国之蕃表,既处下流,受敌二境。若敌汎舟顺流,舳舻千里,星奔电迈,俄然行至,非可恃援他部以救倒县也。此乃社稷安危之机,非徒封疆侵陵小害也。臣父逊昔在西垂陈言,以为西陵国之西门,虽云易守,亦复易失。若有不守,非但失一郡,则荆州非吴有也。如其有虞,当倾国争之。臣往在西陵,得涉逊迹,前乞精兵三万,而主者循常,未肯差赴。自步阐以后,益更损耗。今臣所统千里,受敌四处,外御强对,内怀百蛮,而上下见兵财有数万,羸弊日久,难以待变。臣愚以为诸王幼冲,未统国事,可且立傅相,辅导贤姿,无用兵马,以妨要务。又黄门竖宦,开立占募,兵民怨役,逋逃入占。乞特诏简阅,一切料出,以补疆埸受敌常处,使臣所部足满八万,省息众务,信其赏罚,虽韩、白复生,无所展巧。若兵不增,此制不改,而欲克谐大事,此臣之所深慼也。若臣死之后,乞以西方为属。愿陛下思览臣言,则臣死且不朽。”

秋遂卒,子晏嗣。晏及弟景、玄、机、云、分领抗兵。晏为裨将军、夷道监。天纪四年,晋军伐吴,龙骧将军王濬顺流东下,所至辄克,终如抗虑。景字士仁,以尚公主拜骑都尉,封毗陵侯,既领抗兵,拜偏将军、中夏督,澡身好学,著书数十篇也。二月壬戌,晏为王濬别军所杀。癸亥,景亦遇害,时年三十一。景妻,孙皓適妹,与景俱张承外孙也。

评曰:刘备天下称雄,一世所惮,陆逊春秋方壮,威名未著,摧而克之,罔不如志。予既奇逊之谋略,又叹权之识才,所以济大事也。及逊忠诚恳至,忧国亡身,庶几社稷之臣矣。抗贞亮筹幹,咸有父风,奕世载美,具体而微,可谓克构者哉!

\part{吴书十四}
\chapter{吴主五子传第十四}

孙登字子高,权长子也。魏黄初二年,以权为吴王,拜登东中郎将,封万户侯,登辞疾不受。是岁,立登为太子,选置师傅,铨简秀士,以为宾友,於是诸葛恪、张休、顾谭、陈表等以选入,侍讲诗书,出从骑射。权欲登读汉书,习知近代之事,以张昭有师法,重烦劳之,乃令休从昭受读,还以授登。登待接寮属,略用布衣之礼,与恪、休、谭等或同舆而载,或共帐而寐。太傅张温言於权曰:“夫中庶子官最亲密,切问近对,宜用隽德。”於是乃用表等为中庶子。后又以庶子礼拘,复令整巾侍坐。黄龙元年,权称尊号,立为皇太子,以恪为左辅,休右弼,谭为辅正,表为翼正都尉,是为四友,而谢景、范慎、刁玄、羊衟等皆为宾客,於是东宫号为多士。

权迁都建业,徵上大将军陆逊辅登镇武昌,领宫府留事。登或射猎,当由径道,常远避良田,不践苗稼,至所顿息,又择空间之地,其不欲烦民如此。尝乘马出,有弹丸过,左右求之。有一人操弹佩丸,咸以为是,辞对不服,从者欲捶之,登不听,使求过丸,比之非类,乃见释。又失盛水金马盂,觉得其主,左右所为,不忍致罚,呼责数之,长遣归家,敕亲近勿言。后弟虑卒,权为之降损,登昼夜兼行,到赖乡,自闻,即时召见。见权悲泣,因谏曰:“虑寝疾不起,此乃命也。方今朔土未一,四海喁喁,天戴陛下,而以下流之念,减损大官殽馔,过於礼制,臣窃忧惶。”权纳其言,为之加膳。住十馀日,欲遣西还,深自陈乞,以久离定省,子道有阙,又陈陆逊忠勤,无所顾忧,权遂留焉。嘉禾三年,权征新城,使登居守,总知留事。时年谷不丰,颇有盗贼,乃表定科令,所以防御,甚得止奸之要。

初,登所生庶贱,徐夫人少有母养之恩,后徐氏以妒废处吴,而步夫人最宠。步氏有赐,登不敢辞,拜受而已。徐氏使至,所赐衣服,必沐浴服之。登将拜太子,辞曰:“本立而道生,欲立太子,宜先立后。”权曰:“卿母安在?”对曰:“在吴。“权默然。

立凡二十一年,年三十三卒。临终,上疏曰:“臣以无状,婴抱笃疾,自省微劣,惧卒陨毙。臣不自惜,念当委离供养,埋胔后土,长不复奉望宫省,朝觐日月,生无益於国,死贻陛下重慼,以此为哽结耳。臣闻死生有命,长短自天,周晋、颜回有上智之才,而尚夭折,况臣愚陋,年过其寿,生为国嗣,没享荣祚,於臣已多,亦何悲恨哉!方今大事未定,逋寇未讨,万国喁喁,系命陛下,危者望安,乱者仰治。愿陛下弃忘臣身,割下流之恩,修黄老之术,笃养神光,加羞珍膳,广开神明之虑,以定无穷之业,则率土幸赖,臣死无恨也。皇子和仁孝聪哲,德行清茂,宜早建置,以系民望。诸葛恪才略博达,器任佐时。张休、顾谭、谢景,皆通敏有识断,入宜委腹心,出可为爪牙。范慎、华融矫矫壮节,有国士之风。羊衟辩捷,有专对之材。刁玄优弘,志履道真。裴钦博记,翰采足用。蒋脩、虞翻,志节分明。凡此诸臣,或宜廊庙,或任将帅,皆练时事,明习法令,守信固义,有不可夺之志。此皆陛下日月所照,选置臣官,得与从事,备知情素,敢以陈闻。臣重惟当今方外多虞,师旅未休,当厉六军,以图进取。军以人为众,众以财为宝,窃闻郡县颇有荒残,民物凋弊,奸乱萌生,是以法令繁滋,刑辟重切。臣闻为政听民,律令与时推移,诚宜与将相大臣详择时宜,博采众议,宽刑轻赋,均息力役,以顺民望。陆逊忠勤於时,出身忧国,謇謇在公,有匪躬之节。诸葛瑾、步骘、朱然、全琮、朱据、吕岱、吾粲、阚泽、严畯、张承、孙怡忠於为国,通达治体。可令陈上便宜,蠲除苛烦,爱养士马,抚循百姓。五年之外,十年之内,远者归复,近者尽力,兵不血刃,而大事可定也。臣闻'鸟之将死其鸣也哀,人之将死其言也善',故子囊临终,遗言戒时,君子以为忠,岂况臣登,其能已乎?愿陛下留意听采,臣虽死之日,犹生之年也。”既绝而后书闻,权益以摧感,言则陨涕。是岁,赤乌四年也。谢景时为豫章太守,不胜哀情,弃官奔赴,拜表自劾。权曰:“君与太子从事,异於他吏。“使中使慰劳,听复本职,发遣还郡。谥登曰宣太子。”

子璠、希,皆早卒,次子英,封吴侯。五凤元年,英以大将军孙峻擅权,谋诛峻,事觉自杀,国除。

谢景者字叔发,南阳宛人。在郡有治迹,吏民称之,以为前有顾劭,其次即景。数年卒官。

孙虑字子智,登弟也。少敏惠有才艺,权器爱之。黄武七年,封建昌侯。后二年,丞相雍等奏虑性聪体达,所尚日新,比方近汉,宜进爵称王,权未许。久之,尚书仆射存上疏曰:“帝王之兴,莫不褒崇至亲,以光群后,故鲁卫於周,宠冠诸侯,高帝五王,封列于汉,所以藩屏本朝,为国镇卫。建昌侯虑禀性聪敏,才兼文武,於古典制,宜正名号。陛下谦光,未肯如旧,群寮大小,咸用於邑。方今奸寇恣睢,金鼓未弭,腹心爪牙,惟亲与贤。辄与丞相雍等议,咸以虑宜为镇军大将军,授任偏方,以光大业。”权乃许之,於是假节开府,治半州。虑以皇子之尊,富於春秋,远近嫌其不能留意。及至临事,遵奉法度,敬纳师友,过於众望。年二十,嘉禾元年卒。无子,国除。

孙和字子孝,虑弟也。少以母王有宠见爱,年十四,为置宫卫,使中书令阚泽教以书艺。好学下士,甚见称述。赤乌五年,立为太子,时年十九。阚泽为太傅,薛综为少傅,而蔡颖、张纯、封俌、严维等皆从容侍从。

是时有司颇以条书问事,和以为奸妄之人,将因事错意,以生祸心,不可长也,表宜绝之。又都督刘宝白庶子丁晏,晏亦白宝,和谓晏曰:“文武在事,当能几人,因隙构薄,图相危害,岂有福哉?”遂两释之,使之从厚。常言当世士人宜讲脩术学,校习射御,以周世务,而但交游博弈以妨事业,非进取之谓。后群寮侍宴,言及博弈,以为妨事费日而无益於用,劳精损思而终无所成,非所以进德脩业,积累功绪者也。且志士爱日惜力,君子慕其大者,高山景行,耻非其次。夫以天地长久,而人居其间,有白驹过隙之喻,年齿一暮,荣华不再。凡所患者,在於人情所不能绝,诚能绝无益之欲以奉德义之涂,弃不急之务以脩功业之基,其於名行,岂不善哉?夫人情犹不能无嬉娱,嬉娱之好,亦在於饮宴琴书射御之间,何必博弈,然后为欢。乃命侍坐者八人,各著论以矫之。於是中庶子韦曜退而论奏,和以示宾客。时蔡颖好弈,直事在署者颇斅焉,故以此讽之。

是后王夫人与全公主有隙。权尝寝疾,和祠祭於庙,和妃叔父张休居近庙,邀和过所居。全公主使人觇视,因言太子不在庙中,专就妃家计议;又言王夫人见上寝疾,有喜色。权由是发怒,夫人忧死,而和宠稍损,惧於废黜。鲁王霸觊觎滋甚,陆逊、吾粲、顾谭等数陈適庶之义,理不可夺,全寄、杨竺为鲁王霸支党,谮愬日兴。粲遂下狱诛,谭徙交州。权沈吟者历年,后遂幽闭和。於是骠骑将军朱据、尚书仆射屈晃率诸将吏泥头自缚,连日诣阙请和。权登白爵观见,甚恶之,敕据、晃等无事忩々。权欲废和立亮,无难督陈正、五营督陈象上书,称引晋献公杀申生,立奚齐,晋国扰乱,又据、晃固谏不止。权大怒,族诛正、象,据、晃牵入殿,杖一百,竟徙和於故鄣,群司坐谏诛放者十数。众咸冤之。

太元二年正月,封和为南阳王,遣之长沙。四月,权薨,诸葛恪秉政。恪即和妃张之舅也。妃使黄门陈迁之建业上疏中宫,并致问於恪。临去,恪谓迁曰:“为我达妃,期当使胜他人。”此言颇泄。又恪有徙都意,使治武昌宫,民间或言欲迎和。及恪被诛,孙峻因此夺和玺绶,徙新都,又遣使者赐死。和与妃张辞别,张曰:“吉凶当相随,终不独生活也。”亦自杀,举邦伤焉。

孙休立,封和子皓为乌程侯,自新都之本国。休薨,皓即阼,其年追谥父和曰文皇帝,改葬明陵,置园邑二百家,令、丞奉守。后年正月,又分吴郡、丹杨九县为吴兴郡,治乌程,置太守,四时奉祠。有司奏言,宜立庙京邑。宝鼎二年七月,使守大匠薛珝营立寝堂,号曰清庙。十二月,遣守丞相孟仁、太常姚信等备官僚中军步骑二千人,以灵舆法驾,东迎神於明陵。皓引见仁,亲拜送於庭。灵舆当至,使丞相陆凯奉三牲祭於近郊,皓於金城外露宿。明日,望拜於东门之外。其翌日,拜庙荐祭,歔欷悲感。比七日三祭,倡技昼夜娱乐。有司奏言“祭不欲数,数则黩,宜以礼断情”,然后止。

孙霸字子威,和弟也。和为太子。霸为鲁王,宠爱崇特,与和无殊。顷之,和、霸不穆之声闻於权耳,权禁断往来,假以精学。督军使者羊衟上疏曰:“臣闻古之有天下者,皆先显别適庶,封建子弟,所以尊重祖宗,为国藩表也。二宫拜授,海内称宜,斯乃大吴兴隆之基。顷闻二宫并绝宾客,远近悚然,大小失望。窃从下风,听采众论,咸谓二宫智达英茂,自正名建号,於今三年,德行内著,美称外昭,西北二隅,久所服闻。谓陛下当副顺遐迩所以归德,勤命二宫宾延四远,使异国闻声,思为臣妾。今既未垂意於此,而发明诏,省夺备卫,抑绝宾客,使四方礼敬,不复得通,虽实陛下敦尚古义,欲令二宫专志於学,不复顾虑观听小宜,期於温故博物而已,然非臣下倾企喁喁之至愿也。或谓二宫不遵典式,此臣所以寝息不宁。就如所嫌,犹宜补察,密加斟酌,不使远近得容异言。臣惧积疑成谤,久将宣流,而西北二隅,去国不远,异同之语,易以闻达。闻达之日,声论当兴,将谓二宫有不顺之愆,不审陛下何以解之?若无以解异国,则亦无以释境内。境内守疑,异国兴谤,非所以育巍巍,镇社稷也。愿陛下早发优诏,使二宫周旋礼命如初,则天清地晏,万国幸甚矣。”

时全寄、吴安、孙奇、杨竺等阴共附霸,图危太子。谮毁既行,太子以败,霸亦赐死。流竺尸于江,兄穆以数谏戒竺,得免大辟,犹徙南州。霸赐死后,又诛寄、安、奇等,咸以党霸构和故也。

霸二子,基、壹。五凤中,封基为吴侯,壹宛陵侯。基侍孙亮在内,太平二年,盗乘御马,收付狱。亮问侍中刁玄曰:“盗乘御马罪云何?”玄对曰:“科应死。然鲁王早终,惟陛下哀原之。”亮曰:“法者,天下所共,何得阿以亲亲故邪?当思惟可以释此者,奈何以情相迫乎?”玄曰:“旧赦有大小,或天下,亦有千里、五百里赦,随意所及。“亮曰:“解人不当尔邪!”乃赦宫中,基以得免。孙皓即位,追和、霸旧隙,削基、壹爵土,与祖母谢姬俱徙会稽乌伤县。

孙奋字子扬,霸弟也,母曰仲姬。太元二年,立为齐王,居武昌。权薨,太傅诸葛恪不欲诸王处江滨兵马之地,徙奋於豫章。奋怒,不从命,又数越法度。恪上笺谏曰:“帝王之尊,与天同位,是以家天下,臣父兄,四海之内,皆为臣妾。仇雠有善,不得不举,亲戚有恶,不得不诛,所以承天理物,先国后身,盖圣人立制,百代不易之道也。昔汉初兴,多王子弟,至於太强,辄为不轨,上则几危社稷,下则骨肉相残,其后惩戒,以为大讳。自光武以来,诸王有制,惟得自娱於宫内,不得临民,干与政事,其与交通,皆有重禁,遂以全安,各保福祚。此则前世得失之验也。近袁绍、刘表各有国土,土地非狭,人众非弱,以適庶不分,遂灭其宗祀。此乃天下愚智,所共嗟痛。大行皇帝览古戒今,防芽遏萌,虑於千载。是以寝疾之日,分遣诸王,各早就国,诏策殷勤,科禁严峻,其所戒敕,无所不至,诚欲上安宗庙,下全诸王,使百世相承,无凶国害家之悔也。大王宜上惟太伯顺父之志,中念河间献王、东海王强恭敬之节,下当裁抑骄恣荒乱以为警戒。而闻顷至武昌以来,多违诏敕,不拘制度,擅发诸将兵治护宫室。又左右常从有罪过者,当以表闻,公付有司,而擅私杀,事不明白。大司马吕岱亲受先帝诏敕,辅导大王,既不承用其言,令怀忧怖。华锜先帝近臣,忠良正直,其所陈道,当纳用之,而闻怒锜,有收缚之语。又中书杨融,亲受诏敕,所当恭肃,云'正自不听禁,当如我何'?闻此之日,大小惊怪,莫不寒心。里语曰:'明镜所以照形,古事所以知今。'大王宜深以鲁王为戒,改易其行,战战兢兢,尽敬朝廷,如此则无求不得。若弃忘先帝法教,怀轻慢之心,臣下宁负大王,不敢负先帝遗诏,宁为大王所怨疾,岂敢忘尊主之威,而令诏敕不行於藩臣邪?此古今正义,大王所照知也。夫福来有由,祸来有渐,渐生不忧,将不可悔。向使鲁王早纳忠直之言,怀惊惧之虑,享祚无穷,岂有灭亡之祸哉?夫良药苦口,惟疾者能甘之。忠言逆耳,惟达者能受之。今者恪等慺慺欲为大王除危殆於萌芽,广福庆之基原,是以不自知言至,愿蒙三思。

奋得笺惧,遂移南昌,游猎弥甚,官属不堪命。及恪诛,奋下住芜湖,欲至建业观变。傅相谢慈等谏奋,奋杀之。坐废为庶人,徙章安县。太平三年,封为章安侯。

建衡二年,孙皓左夫人王氏卒。皓哀念过甚,朝夕哭临,数月不出,由是民间或谓皓死,讹言奋与上虞侯奉当有立者。奋母仲姬墓在豫章,豫章太守张俊疑其或然,扫除坟茔。皓闻之,车裂俊,夷三族,诛奋及其五子,国除。

评曰:孙登居心所存,足为茂美之德。虑、和并有好善之姿,规自砥砺,或短命早终,或不得其死,哀哉!霸以庶干適,奋不遵轨度,固取危亡之道也。然奋之诛夷,横遇飞祸矣。

\part{吴书十五}
\chapter{贺全吕周锺离传第十五}

贺齐字公苗,会稽山阴人也。少为郡吏,守剡长。县吏斯从轻侠为奸,齐欲治之。

主簿谏曰:“从,县大族,山越所附,今日治之,明日寇至。”齐闻大怒,便立斩众。

从族党遂相纠合,众千余人,举兵攻县。齐率吏民,开城门突击,大破之,威震山越。

后太末、丰浦民反,转守太末长,诛恶养善,期用尽平。建安元年,孙策临郡,察齐孝廉。时王朗奔东冶,侯官长商升为朗起兵。策遣永宁长韩晏领南部都尉,将兵讨升,以齐为永宁长。晏为升所败,齐又代晏领都尉事。升畏齐威名,遣使乞盟。齐因告喻,为陈祸福,升遂送上印绶,出舍求降。贼帅张雅、詹强等不愿升降,反共杀升,雅称无上将军,强称会稽太守。贼盛兵少,未足以讨,齐住军息兵。雅与女婿何雄争势两乖,齐令越人因事交构。遂至疑隙,阻兵相图。齐乃进讨,一战大破雅,强党震惧,率众出降。

侯官既平。而建安、汉兴、南平复乱,齐进兵建安,立都尉府,是岁八年也。郡发属县五千兵,各使本县长将之,皆受齐节度。贼洪明、洪进、苑御、吴免、华当等五人,率各万户,连屯汉兴,吴五六千户别屯大潭。邹临六千户别屯盖竹,大潭同出馀汗。军讨汉兴,经馀汗。齐以为贼众兵少,深入无继,恐为所断,令松阳长丁蕃留备余汗。蕃本与齐邻城,耻见部伍,辞不肯留。齐乃斩蕃,于是军中震栗。无不用命。遂分兵留备,进讨明等,连大破之。临陈斩明,其免、当、进、御皆降。转击盖竹,军向大潭,三将又降。凡讨治斩首六千级,名帅尽擒。复立县邑,料出兵万人,拜为平东校尉。十年,转讨上饶,分以为建平县。

十三年,迁威武中郎将,讨丹阳黟、歙。时武强、叶乡、东阳、丰浦四乡先降,齐表言以叶乡为始新县。而歙贼帅金奇万户屯安勒山,毛甘万户屯乌聊山,黟帅陈仆,祖山等二万户屯林历山。林历山四面壁立,高数十丈,径路危狭,不容刀楯,贼临高下石,不可得攻。军住经日,将吏患之。齐身出周行,观视形便,阴募轻捷士,为作铁弋,密于隐险贼所不备处,以戈拓斩山为缘道,夜令潜上,乃多县布以援下人,得上百数人,四面流布,俱鸣鼓角,齐勒兵待之。贼夜闻鼓声四合,谓大军悉已得上,惊惧惑乱,不知所为,守路备险者,皆走还依众。大军因是得上,大破仆等,其余皆降,凡斩首七千。

齐复表分歙为新定;黎阳、休阳。并黟、歙凡六县。权遂割为新都郡,齐为太守,立府于始新,加偏将军。

十六年,吴郡余杭民郎稚合宗起贼,复数千人,齐出讨之。即复破稚,表言分余杭为临水县。被命诣所在,及当还郡,权出祖道,作乐舞象。赐齐軿车骏马,罢坐住驾,使齐就车。齐辞不敢,权使左右扶齐上车,令导吏卒兵骑,如在郡仪。权望之笑曰:“人当努力,非积行累勤,此不可得。”去百余步乃旋。

十八年,豫章东部民彭材、李玉、王海等起为贼乱,众万余人。齐讨平之,诛其首恶,余皆降服。拣其精健为兵,次为县户。迁奋武将军。二十年,从权征合肥。时城中出战,徐盛被创失矛,齐引兵拒击,得盛所失。二十一年,鄱阳民尤突受曹公印绶,化民为贼,陵阳、始安、泾县皆与突相应。齐与陆逊讨破突,斩首数千,余党震服,丹杨三县皆降,料得精兵八千人。拜安东将军,封山阴侯,出镇江上,督扶州以上至皖。

黄武初,魏使曹休来伐。齐以道远后至,因住新市为拒。会洞口诸军遭风流溺,所亡中分,将士失色,赖齐未济,偏军独全,诸将倚以为势。

齐性奢绮,尤好军事,兵甲器械极为精好,所乘船雕刻丹镂,青盖绛襜,干橹戈矛,葩爪文画,弓弩矢箭,咸取上材,蒙冲斗舰之属,望之若山。休等惮之,遂引军还。迁后将军,假节领徐州牧。

初,晋宗为戏口将,以众叛如魏,还为蕲春太守,图袭安乐,取其保质。权以为耻忿,因军初罢,六月盛夏、出其不意,诏齐督麋芳、鲜于丹等袭蕲春,遂生虏宗。后四年卒,子达及弟景皆有令名,为佳将。

全琮字子璜,吴郡钱唐人也。父柔,汉灵帝时举孝廉。补尚书郎右丞,董卓之乱,弃官归。州辟别驾从事,诏书就拜会稽东部都尉。孙策到吴,柔举兵先附,策表柔为丹杨都尉。孙权为车骑将军,以柔为长史,徙桂阳太守。柔尝使琮赍米数千斛到吴,有所市易。琮至,皆散用,空船而还。柔大怒,琮顿首曰:“愚以所市非急,而士大夫方有倒县之患,故便振赡,不及启报。”柔更以奇之。是时中州士人避乱而南,依琮居者以百数,琮倾家给济,与共有无,遂显名远近。后权以为奋威校尉,授兵数千人,使讨山越。因开募召,得精兵万余人,出屯牛渚,稍迁偏将军。

建安二十四年,刘备将关羽围樊、襄阳,琮上疏陈羽可讨之计,权时已与吕蒙阴仪袭之,恐事泄,故寝琮表不答。及禽羽,权置酒公安。顾谓琮曰:“君前陈此,孤虽不相答,今日之捷,抑亦君之功也。”于是封阳华亭侯。

黄武元年,魏以舟军大出洞口,权使吕范督诸将拒之,军营相望。敌数以轻船抄击,琮常带甲仗兵,伺候不休。顷之,敌数千人出江中,琮击破之,枭其将军尹卢。迁琮绥南将军,进封钱唐侯。四年,假节领九江太守。七年,权到皖,使琮与辅国将军陆逊击曹休,破之于石亭。是时丹杨、吴、会山民复为寇贼,攻没属县,权分三郡险地为东安郡,琮领太守。至,明赏罚。招诱降附,数年中,得万余人。权召琮还牛渚,罢东安郡。

黄龙元年,迁卫将军、左护军、徐州牧,尚公主。

嘉禾二年,督步骑五万征六安,六安民皆散走,诸将欲分兵捕之。琮曰:“夫乘危侥幸,举不百全者,非国家大体也。今分兵捕民,得失相半,岂可谓全哉?纵有所获,犹不足以弱敌而副国望也。如或邂逅,亏损非小,与其获罪,琮宁以身受之。不敢徼功以负国也。”

赤乌九年,迁右大司马、左军师。为人恭顺,善于承颜纳规,言辞未尝切迕。初,权将围珠崖及夷州,皆先问琮。琮曰:“以圣朝之威,何向而不克?然殊方异域,隔绝障海,水土气毒,自古有之,兵入民出,必生疾病,转相污染,往者惧不能反,所获何可多致?猥亏江岸之兵,以冀万一之利,愚臣犹所不安。”权不听。军行经岁,士众疾疫死者十有八九,权深悔之。后言次及之,琮对曰:“当是时,群臣有不谏者,臣以为不忠。”琮既亲重,宗族子弟并蒙宠贵,赐累千金,然犹谦虚接士,貌无骄色。十二年卒,子怿嗣。后袭业领兵,救诸葛诞于寿春,出城先降,魏以为平东将军,封临湘侯。

怿兄子祎、仪、静等亦降魏,皆历郡守列侯。

吕岱字定公,广陵海陵人也,为郡县吏,避乱南渡。孙权统事,岱诣幕府,出守吴丞。权亲断诸县仓库及囚系,长丞皆见,岱处法应问,甚称权意,召署录事,出补余姚长,召募精健,得千余人。会稽东冶五县贼吕合、秦狼等为乱,权以岱为督军校尉,与将军蒋钦等将兵讨之,遂擒合、狼,五县平定,拜昭信中郎将。

建安二十年,督孙茂等十将从取长沙三郡。又安成、攸、永新、茶陵四县吏共入阴山城,合众拒岱,岱攻围,即降,三郡克定。权留岱镇长沙。安成长吴砀及中郎将袁龙等首尾关羽,复为反乱。砀据攸县,龙在醴陵。权遣横江将军鲁肃攻攸,砀得突走。岱攻醴陵,遂禽斩龙。迁庐陵太守。

延康元年,代步骘为交州刺史。到州,高凉贼帅钱博乞降,岱因承制,以博为高凉西部都尉。又郁林夷贼攻围郡县,岱讨破之。是时桂阳、浈阳贼王金合众于南海界上,首乱为害,权又诏岱讨之,生缚金,传送诣都,斩首获生凡万余人。迁安南将军,假节,封都乡侯。

交址太守士燮卒,权以燮子徽为安远将军,领九真太守,以校尉陈时代燮。岱表分海南三郡为交州,以将军戴良为刺史,海东四郡为广州,岱自为刺史。遣良与时南入,而徽不承命,举兵戍海口以拒良等。岱于是上疏请讨徽罪,督兵三千人晨夜浮海。或谓岱曰:“徽藉累世之恩,为一州所附,未易轻也。”岱曰:“今徽虽怀逆计,未虞吾之卒至,若我潜军轻举,掩其无备,破之必也。稽留不速,使得生心,婴城固守,七郡百蛮,云合响应,虽有智者,谁能图之?”遂行,过合浦,与良俱进。徽闻岱至,果大震怖,不知所出,即率兄弟六人肉袒迎岱。岱皆斩送其首,徽大将甘醴,桓治等率吏民攻岱,岱奋击大破之,进封番禺侯。于是除广州,复为交州如故。岱既定交州,复进讨九真,斩获以万数。又遣从事南宣国化,暨徼外扶南、林邑、堂明诸王,各遣使奉贡。权嘉其功,进拜镇南将军。

黄龙三年,以南土清定,召岱还屯长沙沤口。会武陵蛮夷蠢动,岱与太常潘浚共讨定之。嘉禾三年,权令岱领潘璋士众,屯陆口,后徙蒲圻。四年,庐陵贼李桓、路合、会稽东冶贼随春、南海贼罗厉等一时并起。权复诏岱督刘纂、唐咨等分部讨击,春即时首降,岱拜春偏将军,使领其众,遂为列将,桓、厉等皆见斩获,传首诣都。

权诏岱曰:“厉负险作乱,自致枭首;桓凶狡反复,已降复叛。前后讨伐,历年不禽,非君规略,谁能枭之?忠武之节,于是益者。元恶既除,大小震慑,其余细类,扫地族矣。自今巳去,国家永无南顾之虞,三郡晏然,无怵惕之惊。又得恶民以供赋役,重用叹息。赏不逾月,国之常典,制度所宜,君其裁之。”

潘浚卒,岱代浚领荆州文书,与陆逊井在武昌,故督蒲圻。顷之,廖式作乱。攻围城邑,零陵、苍梧、郁林诸郡骚扰,岱自表辄行,星夜兼路。权遣使追拜岱交州牧,及遣诸将唐咨等骆驿相继,攻讨一年破之,斩式及遣诸所伪署临贺太守费杨等,并其支党,郡县悉平,复还武昌。

时年已八十,然体素精勤,躬亲王事。奋威将军张承与岱书曰:“昔旦奭翼同,《二南》作歌,今则足下与陆子也。忠勤相先,劳谦相让,功以权成,化与道合,君子叹其德,小人悦其美。加以文书鞅掌,宾客终日,罢不舍事,劳不言倦。又知上马辄自超乘,不由跨蹑,如此足下过廉颇也。何其事事快也。《周易》有之,礼言恭,德言盛,足下何有尽此美耶!”及陆逊卒,诸葛恪代逊,权乃分武昌为两部,岱督右部,自武昌上至蒲圻。迁上大将军,拜子凯副军校尉,监兵蒲圻,孙亮即位,拜大司马。岱清身奉公,所在可述。初在交州,历年不饷家,妻子饥乏。权闻之叹息,以让群臣曰:“吕岱出身万里,为国勤事,家门内困,而孤不早知。股肱耳目,其责安在?”于是加赐钱米布绢,岁有常限。

始,岱亲近吴郡徐原,慷慨有才志,岱知其可成,赐巾褠,与共言论,后遂荐拔,官至侍御史。原性忠壮,好直言,岱时有得失,原辄谏诤,又公论之,人或以告岱,岱叹曰:“是我所以贵德渊者也。”及原死,岱哭之甚哀。曰:“德渊,吕岱之益友,今不幸,岱复于何闻过?”谈者美之。太平元年,年九十六卒,子凯嗣。遣令殡以素棺,疏巾布褠,葬送之制,务从俭约,凯皆奉行之。

周鲂字子鱼,吴郡阳羡人也。少好学,举孝廉,为宁国长,转在怀安。钱唐大帅彭式等蚁聚为寇,以鲂为钱唐侯相,旬月之间,斩式首及其支党,迁丹杨西部都尉。黄武中。鄱阳大帅彭绮作乱,攻没属城,乃以鲂为鄱阳太守,与胡综戮力攻讨,遂生禽绮,送诣武昌,加昭义校尉。被命密求山中旧族名帅为北敌所闻知者,令谲挑魏大司马扬州牧曹休。鲂答,恐民帅小丑不足仗任,事或漏泄,不能致休,乞遣亲人赍笺七条以诱休。

其一曰:“鲂以千载饶幸,得备州民,远隔江川,敬恪未显,瞻望云景,天实为之。

精诚微薄,名位不昭,虽怀焦渴,曷缘见明?狐死首丘,人情恋本,而逼所制,奉觌礼违。每独矫首西顾,未尝不寤寐劳叹,展转反侧也。今因隙穴之际,得陈宿昔之志,非神启之,岂能致此!不胜翘企,万里托命。谨遣亲人董岑、邵南等托叛奉笺。时事变故,列于别纸,惟明公君侯垂日月之光,照远民之趣,永令归命者有所戴赖。“

其二曰:“鲂远在边隅,江汜分绝;恩泽教化,未蒙抚及,而于山谷之间,遥陈所怀,惧以大义,未见信纳。夫物有感激,计因变生,古今同揆;鲂仕东典郡,始愿已获,铭心立报,永矣无贰。岂图顷者中被横谴,祸在漏刻,危于投卵,进有离合去就之宜,退有诬罔枉死之咎,虽志行轻微,存没一节,顾非其所,能不怅然!敢缘古人,因知所归,拳拳输情,陈露肝膈。乞降春天之润,哀拯其急,不复猜疑,绝其委命。事之宣泄。

受罪不测,一则伤慈损计,二则杜绝向化者心,惟明使君远览前世,矜而愍之,留神所质,速赐秘报。鲂当候望举动,俟须响应。“

其三曰:“鲂所代故太守广陵王靖,往者亦以郡民为变,以见谴责,靖勤自陈释,而终不解,因立密计,欲北归命,不幸事露,诛及婴孩。鲂既目见靖事,且观东主一所非薄,婳不复厚,虽或暂舍,终见翦除。今又令鲂领郡者,是欲责后效,必杀鲂之趣也。

虽尚视息,忧惕焦灼,未知躯命,竟在何时。人居世间,犹白驹过隙,而常抱危怖,其可言乎!惟当陈愚,重自披尽,惧以卑贱,未能采纳。愿明使君少垂详察,忖度其言。

今此郡民,虽外名降首,而故在山草,看伺空隙,欲复为乱。为乱之日,鲂命讫矣。东主顷者潜部分诸将,图欲北进。吕范、孙韶等入淮,全琮、朱桓趋合肥,诸葛瑾、步骘、朱然到襄阳。陆议、潘璋等讨梅敷。东主中营自掩石阳,别遣从孙奂治安陆城,修立邸阁,辇赀运粮,以为军储,又命诸葛亮进指关西,江边诸将无复在者,才留三千所兵守武昌耳。若明使君以万兵从皖南首江渚,鲂便从此率厉吏民,以为内应。此方诸郡,前后举事,垂成而败者,由无外援使其然耳;若北军临境,传檄属城,思咏之民,谁不企踵?愿明使君上观天时,下察人事,中参蓍龟,则足昭往言之不虚也。“

其四曰:“所遣董岑、邵南少长家门。亲之信之,有如儿子。是以特令赍笺,托叛为辞,目语心计,不宣唇齿,骨肉至亲,无有知者。又已敕之,到州当言往降,欲北叛来者得传之也。鲂建此计,任之于天,若其济也,则有生全之福。邂逅泄漏,则受夷灭之祸。常中夜仰天,告誓星辰。精成之微,岂能上感,然事急孤穷,惟天是诉耳。遣使之日,载生载死,形存气亡,魄爽恍惚。私恐使君未深保明,岑、南二人可留其一,以为后信。一赍教还,教还故当言悔叛还首。东主有常科,悔叛还者,皆自原罪。如是彼此俱塞,永无端原。县命西望,涕笔俱下。”

其五曰:“鄱阳之民,实多愚劲,帅之赴役。未即应人,倡之为变,闻声响拚.今虽降首,盘节未解,山栖草藏,乱心犹存。而今东主图兴大众,举国悉出。江边空旷,屯坞虚损,惟有诸刺奸耳。若因是际而骚动此民,一旦可得更会,然要恃外援,表里机互,不尔以往,无所成也。今使君若从皖道进住江上,鲂当从南对岸历口为应。若未径到江岸,可住百里上。令此间民知北军在彼,即自善也。此间民非苦饥寒而甘兵寇,苦于征讨,乐得北属,但穷困举事,不时见应,寻受其祸耳。如使石阳及青、徐诸军首尾相衔,牵缀往兵,使不得速退者,则善之善也。鲂生在江、淮,长于时事,见其便利,百举百捷,时不再来,敢布腹心。”

其六曰:“东主致恨前者不拔石阳,今此后举,大合新兵,并使潘浚发夷民,人数甚多,闻豫设科条,当以新羸兵置前,好兵在后,攻城之日。云欲以羸兵填堑,使即时破,虽未能然,是事大趣也。私恐石阳城小,不能久留往兵,明使君速垂救济,试宜疾密。王靖之变,其鉴不远。今鲂归命,非复在天,正在明使君耳。若见救以往,则功可必成,如见救不时,则与靖等同祸。前彭绮时,闻旗麾在逢龙,此郡民大小欢喜,并思立效,若留一月日间,事当大成,恨去电速,东得增众专力讨绮,绮始败耳。愿使君深察此言。”

其七曰:“今举大事,自非爵号无以劝之,乞请将军、侯印各五十纽,郎将印百纽,校尉、都尉印各二百纽,得以假授诸魁帅,奖厉其志,并乞请幢麾数十,以为表帜,使山兵吏民,目瞻见之,知去就之分己决,承引所救画定。又彼此降叛,日月有人,阔狭之间,辄得闻知。今之大事,事宜神密,若省鲂笺,乞加隐秘。伏知智度有常,防虑必深,鲂怀忧震灼,启事蒸仍,乞未罪怪。”

鲂因别为密表曰:“方北有逋寇,固阻河洛,久稽王诛,自擅朔土,臣曾不能吐奇举善。上以光赞洪化,下以输展万一,忧心如捣,假寐忘寝。圣朝天覆,含臣无效,猥发优命。敕臣以前诱致贼休,恨不如计,令于郡界求山谷魁帅为北贼所闻知者。令与北通。臣伏思惟,喜怖交集。窃恐此人不可卒得,假使得之,惧不可信,不如令臣谲休,于计为便。此臣得以经年之冀愿。逢值千载之一会,辄自督竭,竭尽顽蔽,撰立笺草以诳诱休者,如别纸。臣知无古人单复之术,加卒奉大略,伀蒙狼狈,惧以轻愚,忝负特施。豫怀忧灼。臣闻唐尧先天而天弗违,博询刍荛,以成盛勋。朝廷神谟,欲必致休于步度之中,灵赞圣规,休必自送,使六军囊括,虏无孑遗,威风电迈,天下幸甚。谨拜表以闻,并呈笺草,惧于浅局,追用悚息。”被报施行。休果信鲂,帅步骑十万,辎重满道,径来入皖。鲂亦合众,随陆逊横截休,休幅裂瓦解,斩获万计。

鲂初建密计时,频有郎官奉诏诘问诸事,鲂乃诣部郡门下,因下发谢,故休闻之,不复疑虑。事捷军旋,权大会诸将欢宴,酒酣。谓鲂曰:“君下发载义,成孤大事,君之功名,当书之竹帛。”加裨将军,赐爵关内侯。贼帅董嗣负阻劫钞,豫章、临川并受其害。吾粲、唐咨尝以三千兵攻守,连月不能拔。鲂表乞罢兵,得以便宜从事。鲂遣间谍,授以方策,诱狙杀嗣。嗣弟怖惧,诣武昌降于陆逊,乞出平地,自改为善,由是数郡无复忧惕。

鲂在郡十三年卒,赏善罚恶。威恩并行。子处,亦有文武材干,天纪中为东观今、无难督。钟离牧字子干,会稽山阴人,汉鲁相意七世孙也。少爰居永兴,躬自垦田,种稻二十余亩。临熟,县民有识认之。牧曰:“本以田荒,故垦之耳。”遂以稻与县人。

县长闻之,召民系狱,欲绳以法,牧为之请。长曰:“君慕承宫,自行义事,仆为民主,当以法率下,何得寝公宪而从君邪?”牧曰:“此是郡界,缘君意顾,故来暂住。今以少稻而杀此民,何心复留?”遂出装,还山阴,长自往止之,为释系民。民惭惧,率妻子舂所取稻得六十斛米,送还牧,牧闭门不受。民输置道旁,莫有取者。牧由此发名。

赤乌五年,从郎中补太子辅义都尉,迁南海太守。还为丞相长史,转司直,迁中书令。会建安、鄱阳、新都三郡山民作乱,出牧为监军使者,讨平之。贼帅黄乱、常俱筹出其部伍,以充兵役。封秦亭侯,拜越骑校尉。

永安六年,蜀并于魏,武陵五溪夷与蜀接界。时论惧叛乱,乃以牧为平魏将军,领武陵太守,往之郡。魏遣汉葭县长郭纯试守武陵太守,率涪陵民入蜀迁陵界,屯于赤沙,诱致诸夷邑君,或起应纯,又进攻酉阳县,郡中震惧。牧问朝吏曰:“西蜀倾覆,边境见侵,何以御之?”皆对曰:“今二县山险,诸夷阻兵,不可以军惊扰,惊扰则诸夷盘结。宜以渐安,可遣恩信吏宣教慰劳。”牧曰:“不然。外境内侵,诳诱人民,当及其根柢未深而扑取之,此救火贵速之势也。”敕外趣严,掾史沮仪者便行军法。抚夷将军尚说牧曰:“昔潘太常督兵五万,然后以讨五溪夷耳。是时刘氏连和,诸夷率化,今既无往日之援,而郭纯已据迁陵,而明府以三千兵深入,尚未见其利也。”牧曰:“非常之事,何得循旧?”即率所领,晨夜进道,缘山险行,垂二千里,从塞上,斩恶民怀异心者魁帅百余人及其支党凡于余级,纯等散,五溪平。迁公安督、扬武将军,封都乡侯,徙濡须督。复以前将军假节,领武陵太守。卒官。家无余财,士民思之。子袆嗣,代领兵。

评曰:山越好为叛乱,难安易动,是以孙权不遑外御,卑词魏氏。凡此诸臣,皆克宁内难,绥静邦域者也。吕岱清恪在公;周鲂谲略多奇;钟离牧蹈长者之规;全琮有当世之才,贵重于时,然不检奸子,获讥毁名云。

\part{吴书十六}
\chapter{潘濬陆凯传第十六}

潘濬字承明,武陵汉寿人也。弱冠从宋仲子受学。年未三十,荆州牧刘表辟为部江夏从事。时沙羡长赃秽不脩,濬按杀之,一郡震竦。后为湘乡令,治甚有名。刘备领荆州,以濬为治中从事。备入蜀,留典州事。

孙权杀关羽,并荆土,拜濬辅军中郎将,授以兵。迁奋威将军,封常迁亭侯。权称尊号,拜为少府,进封刘阳侯,迁太常。五谿蛮夷叛乱盘结,权假濬节,督诸军讨之。信赏必行,法不可干,斩首获生,盖以万数,自是群蛮衰弱,一方宁静。

先是,濬与陆逊俱驻武昌,共掌留事,还复故。时校事吕壹操弄威柄,奏按丞相顾雍、左将军朱据等,皆见禁止。黄门侍郎谢厷语次问壹:“顾公事何如?”壹答:“不能佳。”厷又问:“若此公免退,谁当代之?”壹未答厷,厷曰:“得无潘太常得之乎?”壹良久曰:“君语近之也。”厷谓曰:“潘太常常切齿於君,但道远无因耳。今日代顾公,恐明日便击君矣。”壹大惧,遂解散雍事。濬求朝,诣建业,欲尽辞极谏。至,闻太子登已数言之而不见从,濬乃大请百寮,欲因会手刃杀壹,以身当之,为国除患。壹密闻知,称疾不行。濬每进见,无不陈壹之奸险也。由此壹宠渐衰,后遂诛戮。权引咎责躬,因诮让大臣,语在权传。

赤乌二年,濬卒,子翥嗣。濬女配建昌侯孙虑。

陆凯字敬风,吴郡吴人,丞相逊族子也。黄武初为永兴、诸暨长,所在有治迹,拜建武都尉,领兵。虽统军众,手不释书。好太玄,论演其意,以筮辄验。赤乌中,除儋耳太守,讨朱崖,斩获有功,迁为建武校尉。五凤二年,讨山贼陈毖於零陵,斩毖克捷,拜巴丘督、偏将军,封都乡侯,转为武昌右部督。与诸将共赴寿春,还,累迁荡魏、绥远将军。孙休即位,拜征北将军,假节领豫州牧。孙皓立,迁镇西大将军,都督巴丘,领荆州牧,进封嘉兴侯。孙皓与晋平,使者丁忠自北还,说皓弋阳可袭,凯谏止,语在皓传。宝鼎元年,迁左丞相。

皓性不好人视己,群臣侍见,皆莫敢迕。凯说皓曰:“夫君臣无不相识之道,若卒有不虞,不知所赴。”皓听凯自视。

皓徙都武昌,扬土百姓溯流供给,以为患苦,又政事多谬,黎元穷匮。凯上疏曰:

臣闻有道之君,以乐乐民;无道之君,以乐乐身。乐民者,其乐弥长;乐身者,不乐而亡。夫民者,国之根也,诚宜重其食,爱其命。民安则君安,民乐则君乐。自顷年以来,君威伤於桀纣,君明闇於奸雄,君惠闭於群孽。无灾而民命尽,无为而国财空,辜无罪,赏无功,使君有谬误之愆,天为作妖。而诸公卿媚上以求爱,困民以求饶,导君於不义,败政於淫俗,臣窃为痛心。今邻国交好,四边无事,当务息役养士,实其廪库,以待天时。而更倾动天心,骚扰万姓,使民不安,大小呼嗟,此非保国养民之术也。

臣闻吉凶在天,犹影之在形,响之在声也,形动则影动,形止则影止,此分数乃有所系,非在口之所进退也。昔秦所以亡天下者,但坐赏轻而罚重,政刑错乱,民力尽於奢侈,目眩於美色,志浊於财宝,邪臣在位,贤哲隐藏,百姓业业,天下苦之,是以遂有覆巢破卵之忧。汉所以强者,躬行诚信,听谏纳贤,惠及负薪,躬请岩穴,广采博察,以成其谋。此往事之明证也。

近者汉之衰末,三家鼎立,曹失纲纪,晋有其政。又益州危险,兵多精强,闭门固守,可保万世,而刘氏与夺乖错,赏罚失所,君恣意於奢侈,民力竭於不急,是以为晋所伐,君臣见虏。此目前之明验也。

臣闇於大理,文不及义,智慧浅劣,无复冀望,窃为陛下惜天下耳。臣谨奏耳目所闻见,百姓所为烦苛,刑政所为错乱,愿陛下息大功,损百役,务宽荡,忽苛政。

又武昌土地,实危险而塉确,非王都安国养民之处,船泊则沈漂,陵居则峻危,且童谣言:“宁饮建业水,不食武昌鱼;宁还建业死,不止武昌居。”臣闻翼星为变,荧惑作妖,童谣之言,生於天心,乃以安居而比死,足明天意,知民所苦也。

臣闻国无三年之储,谓之非国,而今无一年之畜,此臣下之责也。而诸公卿位处人上,禄延子孙,曾无致命之节,匡救之术,苟进小利於君,以求容媚,荼毒百姓,不为君计也。自从孙弘造义兵以来,耕种既废,所在无复输入,而分一家父子异役,廪食日张,畜积日耗,民有离散之怨,国有露根之渐,而莫之恤也。民力困穷,鬻卖儿子,调赋相仍,日以疲极,所在长吏,不加隐括,加有监官,既不爱民,务行威势,所在骚扰,更为烦苛,民苦二端,财力再耗,此为无益而有损也。愿陛下一息此辈,矜哀孤弱,以镇抚百姓之心。此犹鱼鳖得免毒螫之渊,鸟兽得离罗网之纲,四方之民襁负而至矣。如此,民可得保,先王之国存焉。

臣闻五音令人耳不聪,五色令人目不明,此无益於政,有损於事者也。自昔先帝时,后宫列女,及诸织络,数不满百,米有畜积,货财有馀。先帝崩后,幼、景在位,更改奢侈,不蹈先迹。伏闻织络及诸徒坐,乃有千数,计其所长,不足为国财,然坐食官廪,岁岁相承,此为无益,愿陛下料出赋嫁,给与无妻者。如此,上应天心,下合地意,天下幸甚。

臣闻殷汤取士於商贾,齐桓取士於车辕,周武取士於负薪,大汉取士於奴仆。明王圣主取士以贤,不拘卑贱,故其功德洋溢,名流竹素,非求颜色而取好服、捷口、容悦者也。臣伏见当今内宠之臣,位非其人,任非其量,不能辅国匡时,群党相扶,害忠隐贤。愿陛下简文武之臣,各勤其官,州牧督将,藩镇方外,公卿尚书,务脩仁化,上助陛下,下拯黎民,各尽其忠,拾遗万一,则康哉之歌作,刑错之理清。愿陛下留神思臣愚言。

时殿上列将何定佞巧便辟,贵幸任事,凯面责定曰:“卿见前后事主不忠,倾乱国政,宁有得以寿终者邪!何以专为佞邪,秽尘天听?宜自改厉。不然,方见卿有不测之祸矣。”定大恨凯,思中伤之,凯终不以为意,乃心公家,义形於色,表疏皆指事不饰,忠恳内发。

建衡元年,疾病,皓遣中书令董朝问所欲言,凯陈:“何定不可任用,宜授外任,不宜委以国事。奚熙小吏,建起浦里田,欲复严密故迹,亦不可听。姚信、楼玄、贺卲、张悌、郭逴、薛莹、滕脩及族弟喜、抗,或清白忠勤,或姿才卓茂,皆社稷之桢幹,国家之良辅,愿陛下重留神思,访以时务,各尽其忠,拾遗万一。”遂卒,时年七十二。

子祎,初为黄门侍郎,出领部曲,拜偏将军。凯亡后,入为太子中庶子。右国史华覈表荐祎曰:“祎体质方刚,器幹强固,董率之才,鲁肃不过。及被召当下,径还赴都,道由武昌,曾不回顾,器械军资,一无所取,在戎果毅,临财有节。夫夏口,贼之冲要,宜选名将以镇戍之,臣窃思惟,莫善於祎。”

初,皓常衔凯数犯颜忤旨,加何定谮构非一,既以重臣,难绳以法,又陆抗时为大将在疆埸,故以计容忍。抗卒后,竟徙凯家於建安。

或曰宝鼎元年十二月,凯与大司马丁奉、御史大夫丁固谋,因皓谒庙,欲废皓立孙休子。时左将军留平领兵先驱,故密语平,平拒而不许,誓以不泄,是以所图不果。太史郎陈苗奏皓久阴不雨,风气回逆,将有阴谋,皓深警惧云。

予连从荆、扬来者得凯所谏皓二十事,博问吴人,多云不闻凯有此表。又按其文殊甚切直,恐非皓之所能容忍也。或以为凯藏之箧笥,未敢宣行,病困,皓遣董朝省问欲言,因以付之。虚实难明,故不著于篇,然爱其指擿皓事,足为后戒,故钞列于凯传左云。

皓遣亲近赵钦口诏报凯前表曰:“孤动必遵先帝,有何不平?君所谏非也。又建业宫不利,故避之,而西宫室宇摧朽,须谋移都,何以不可徙乎?”凯上疏曰:

臣窃见陛下执政以来,阴阳不调,五星失晷,职司不忠,奸党相扶,是陛下不遵先帝之所致。夫王者之兴,受之於天,脩之由德,岂在宫乎?而陛下不谘之公辅,便盛意驱驰,六军流离悲惧,逆犯天地,天地以灾,童歌其谣。纵令陛下一身得安,百姓愁劳,何以用治?此不遵先帝一也。

臣闻有国以贤为本,夏杀龙逢,殷获伊挚,斯前世之明效,今日之师表也。中常侍王蕃黄中通理,处朝忠謇,斯社稷之重镇,大吴之龙逢也,而陛下忿其苦辞,恶其直对,枭之殿堂,尸骸暴弃。邦内伤心,有识悲悼,咸以吴国夫差复存。先帝亲贤,陛下反之,是陛下不遵先帝二也。

臣闻宰相国之柱也,不可不强,是故汉有萧、曹之佐,先帝有顾、步之相。而万彧琐才凡庸之质,昔从家隶,超步紫闼,於彧已丰,於器已溢,而陛下爱其细介,不访大趣,荣以尊辅,越尚旧臣。贤良愤惋,智士赫咤,是不遵先帝三也。

先帝爱民过於婴孩,民无妻者以妾妻之,见单衣者以帛给之,枯骨不收而取埋之。而陛下反之,是不遵先帝四也。

昔桀纣灭由妖妇,幽厉乱在嬖妾,先帝鉴之,以为身戒,故左右不置淫邪之色,后房无旷积之女。今中宫万数,不备嫔嫱,外多鳏夫,女吟於中。风雨逆度,正由此起,是不遵先帝五也。

先帝忧劳万机,犹惧有失。陛下临阼以来,游戏后宫,眩惑妇女,乃令庶事多旷,下吏容奸,是不遵先帝六也。

先帝笃尚朴素,服不纯丽,宫无高台,物不彫饰,故国富民充,奸盗不作。而陛下徵调州郡,竭民财力,土被玄黄,宫有朱紫,是不遵先帝七也。

先帝外仗顾、陆、朱、张,内近胡综、薛综,是以庶绩雍熙,邦内清肃。今者外非其任,内非其人,陈声、曹辅,斗筲小吏,先帝之所弃,而陛下幸之,是不遵先帝八也。

先帝每宴见群臣,抑损醇醲,臣下终日无失慢之尤,百寮庶尹,并展所陈。而陛下拘以视瞻之敬,惧以不尽之酒。夫酒以成礼,过则败德,此无异商辛长夜之饮也,是不遵先帝九也。

昔汉之桓、灵,亲近宦竖,大失民心。今高通、詹廉、羊度,黄门小人,而陛下赏以重爵,权以战兵。若江渚有难,烽燧互起,则度等之武不能御侮明也,是不遵先帝十也。

今宫女旷积,而黄门复走州郡,条牒民女,有钱则舍,无钱则取,怨呼道路,母子死诀,是不遵先帝十一也。

先帝在时,亦养诸王太子,若取乳母,其夫复役,赐与钱财,给其资粮,时遣归来,视其弱息。今则不然,夫妇生离,夫故作役,儿从后死,家为空户,是不遵先帝十二也。

先帝叹曰:“国以民为本,民以食为天,衣其次也,三者,孤存之於心。”今则不然,农桑并废,是不遵先帝十三也。

先帝简士,不拘卑贱,任之乡闾,效之於事,举者不虚,受者不妄。今则不然,浮华者登,朋党者进,是不遵先帝十四也。

先帝战士,不给他役,使春惟知农,秋惟收稻,江渚有事,责其死效。今之战士,供给众役,廪赐不赡,是不遵先帝十五也。

夫赏以劝功,罚以禁邪,赏罚不中,则士民散失。今江边将士,死不见哀,劳不见赏,是不遵先帝十六也。

今在所监司,已为烦猥,兼有内使,扰乱其中,一民十吏,何以堪命?昔景帝时,交阯反乱,实由兹起,是为遵景帝之阙,不遵先帝十七也。

夫校事,吏民之仇也。先帝末年,虽有吕壹、钱钦,寻皆诛夷,以谢百姓。今复张立校曹,纵吏言事,是不遵先帝十八也。

先帝时,居官者咸久於其位,然后考绩黜陟。今州县职司,或莅政无几,便徵召迁转,迎新送旧,纷纭道路,伤财害民,於是为甚,是不遵先帝十九也。

先帝每察竟解之奏,当留心推按,是以狱无冤囚,死者吞声。今则违之,是不遵先帝二十也。

若臣言可录,藏之盟府;如其虚妄,治臣之罪。愿陛下留意。

胤字敬宗,凯弟也。始为御史、尚书选曹郎,太子和闻其名,待以殊礼。会全寄、杨竺等阿附鲁王霸,与和分争,阴相谮构,胤坐收下狱,楚毒备至,终无他辞。

后为衡阳督军都尉。赤乌十一年,交阯九真夷贼攻没城邑,交部骚动。以胤为交州刺史、安南校尉。胤入南界,喻以恩信,务崇招纳,高凉渠帅黄吴等支党三千馀家皆出降。引军而南,重宣至诚,遗以财币。贼帅百馀人,民五万馀家,深幽不羁,莫不稽颡,交域清泰。就加安南将军。复讨苍梧建陵贼,破之,前后出兵八千馀人,以充军用。

永安元年,徵为西陵督,封都亭侯,后转在虎林。中书丞华覈表荐胤曰:“胤天姿聪朗,才通行絜,昔历选曹,遗迹可纪。还在交州,奉宣朝恩,流民归附,海隅肃清。苍梧、南海,岁有暴风瘴气之害,风则折木,飞砂转石,气则雾郁,飞鸟不经。自胤至州,风气绝息,商旅平行,民无疾疫,田稼丰稔。州治临海,海流秋咸,胤又畜水,民得甘食。惠风横被,化感人神,遂凭天威,招合遗散。至被诏书当出,民感其恩,以忘恋土,负老携幼,甘心景从,众无携贰,不烦兵卫。自诸将合众,皆胁之以威,未有如胤结以恩信者也。衔命在州,十有馀年,宾带殊俗,宝玩所生,而内无粉黛附珠之妾,家无文甲犀象之珍,方之今臣,实难多得。宜在辇毂,股肱王室,以赞唐虞康哉之颂。江边任轻,不尽其才,虎林选督,堪之者众。若召还都,宠以上司,则天工毕脩,庶绩咸熙矣。”

胤卒,子式嗣,为柴桑督、扬武将军。天策元年,与从兄祎俱徙建安。天纪二年,召还建业,复将军、侯。

评曰:潘濬公清割断,陆凯忠壮质直,皆节梗梗,有大丈夫格业。胤身絜事济,著称南土,可谓良牧矣。

\part{吴书十七}
\chapter{是仪胡综传第十七}

是仪字子羽,北海营陵人也。本姓氏,初为县吏,后仕郡,郡相孔融嘲仪,言“氏”字“民”无上,可改为“是”,乃遂改焉。后依刘繇,避乱江东。繇军败,仪徙会稽。

孙权承摄大业,优文徵仪。到见亲任,专典机密,拜骑都尉。

吕蒙图袭关羽,权以问仪,仪善其计,劝权听之。从讨羽,拜忠义校尉。仪陈谢,权令曰:“孤虽非赵简子,卿安得不自屈为周舍邪?”

既定荆州,都武昌,拜裨将军,后封都亭侯,守侍中。欲复授兵,仪自以非材,固辞不受。黄武中,遣仪之皖就将军刘邵,欲诱致曹休。休到,大破之,迁偏将军,入阙省尚书事,外总平诸官,兼领辞讼,又令教诸公子书学。

大驾东迁,太子登留镇武昌,使仪辅太子。太子敬之,事先谘询,然后施行。进封都乡侯。后从太子还建业,复拜侍中、中执法,平诸官事、领辞讼如旧。典校郎吕壹诬白故江夏太守刁嘉谤讪国政,权怒,收嘉系狱,悉验问。时同坐人皆怖畏壹,并言闻之,仪独云无闻。於是见穷诘累日,诏旨转厉,群臣为之屏息。仪对曰:“今刀锯已在臣颈,臣何敢为嘉隐讳,自取夷灭,为不忠之鬼!顾以闻知当有本末。”据实答问,辞不倾移。权遂舍之,嘉亦得免。

蜀相诸葛亮卒,权垂心西州,遣仪使蜀申固盟好。奉使称意,后拜尚书仆射。

南、鲁二宫初立,仪以本职领鲁王傅。仪嫌二宫相近切,乃上疏曰:“臣窃以鲁王天挺懿德,兼资文武,当今之宜,宜镇四方,为国藩辅。宣扬德美,广耀威灵,乃国家之良规,海内所瞻望。但臣言辞鄙野,不能究尽其意。愚以二宫宜有降杀,正上下之序,明教化之本。”书三四上。为傅尽忠,动辄规谏;事上勤,与人恭。

不治产业,不受施惠,为屋舍财足自容。邻家有起大宅者,权出望见,问起大室者谁,左右对曰:“似是仪家也。”权曰:“仪俭,必非也。”问果他家。其见知信如此。

服不精细,食不重膳,拯赡贫困,家无储畜。权闻之,幸仪舍,求视蔬饭,亲尝之,对之叹息,即增俸赐,益田宅。仪累辞让,以恩为戚。

时时有所进达,未尝言人之短。权常责仪以不言事,无所是非,仪对曰:“圣主在上,臣下守职,惧於不称,实不敢以愚管之言,上干天听。”

事国数十年,未尝有过。吕壹历白将相大臣,或一人以罪闻者数四,独无以白仪。权叹曰:“使人尽如是仪,当安用科法为?”

及寝疾,遗令素棺,敛以时服,务从省约,年八十一卒。

胡综字伟则,汝南固始人也。少孤,母将避难江东。孙策领会稽太守,综年十四,为门下循行,留吴与孙权共读书。策薨,权为讨虏将军,以综为金曹从事,从讨黄祖,拜鄂长。权为车骑将军,都京,召综还,为书部,与是仪、徐详俱典军国密事。刘备下白帝,权以见兵少,使综料诸县,得六千人,立解烦两部,详领左部、综领右部督。吴将晋宗叛归魏,魏以宗为蕲春太守,去江数百里,数为寇害。权使综与贺齐轻行掩袭,生虏得宗,加建武中郎将。魏拜权为吴王,封综、仪、详皆为亭侯。

黄武八年夏,黄龙见夏口,於是权称尊号,因瑞改元。又作黄龙大牙,常在中军,诸军进退,视其所向,命综作赋曰:

乾坤肇立,三才是生。狼弧垂象,实惟兵精。圣人观法,是效是营,始作器械,爰求厥成。黄、农创代,拓定皇基,上顺天心,下息民灾。高辛诛共,舜征有苗,启有甘师,汤有鸣条。周之牧野,汉之垓下,靡不由兵,克定厥绪。明明大吴,实天生德,神武是经,惟皇之极。乃自在昔,黄、虞是祖,越历五代,继世在下。应期受命,发迹南土,将恢大繇,革我区夏。乃律天时,制为神军,取象太一,五将三门;疾则如电,迟则如云,进止有度,约而不烦。四灵既布,黄龙处中,周制日月,实曰太常,桀然特立,六军所望。仙人在上,鉴观四方,神实使之,为国休祥。军欲转向,黄龙先移,金鼓不鸣,寂然变施,闇谟若神,可谓秘奇。在昔周室,赤乌衔书,今也大吴,黄龙吐符。合契河洛,动与道俱,天赞人和,佥曰惟休。

蜀闻权践阼,遣使重申前好。综为盟文,文义甚美,语在权传。

权下都建业,详、综并为侍中,进封乡侯,兼左右领军。时魏降人或云魏都督河北振威将军吴质,颇见猜疑,综乃伪为质作降文三条:

其一曰:“天纲弛绝,四海分崩,群生憔悴,士人播越,兵寇所加,邑无居民,风尘烟火,往往而处,自三代以来,大乱之极,未有若今时者也。臣质志薄,处时无方,系於土壤,不能翻飞,遂为曹氏执事戎役,远处河朔,天衢隔绝,虽望风慕义,思讬大命,愧无因缘,得展其志。每往来者,窃听风化,伏知陛下齐德乾坤,同明日月,神武之姿,受之自然,敷演皇极,流化万里,自江以南,户受覆焘。英雄俊杰,上达之士,莫不心歌腹咏,乐在归附者也。今年六月末,奉闻吉日,龙兴践阼,恢弘大繇,整理天纲,将使遗民,睹见定主。昔武王伐殷,殷民倒戈;高祖诛项,四面楚歌。方之今日,未足以喻。臣质不胜昊天至愿,谨遣所亲同郡黄定恭行奉表,乃讬降叛,间关求达,其欲所陈,载列于左。”

其二曰:“昔伊尹去夏入商,陈平委楚归汉,书功竹帛,遗名后世,世主不谓之背诞者,以为知天命也。臣昔为曹氏所见交接,外讬君臣,内如骨肉,恩义绸缪,有合无离,遂受偏方之任,总河北之军。当此之时,志望高大,永与曹氏同死俱生,惟恐功之不建,事之不成耳。及曹氏之亡,后嗣继立,幼冲统政,谗言弥兴。同侪者以势相害,异趣者得间其言,而臣受性简略,素不下人,视彼数子,意实迫之,此亦臣之过也。遂为邪议所见构会,招致猜疑,诬臣欲叛。虽识真者保明其心,世乱谗胜,馀嫌犹在,常惧一旦横受无辜,忧心孔疚,如履冰炭。昔乐毅为燕昭王立功於齐,惠王即位,疑夺其任,遂去燕之赵,休烈不亏。彼岂欲二三其德,盖畏功名不建,而惧祸之将及也。昔遣魏郡周光以贾贩为名,讬叛南诣,宣达密计。时以仓卒,未敢便有章表,使光口传而已。以为天下大归可见,天意所在,非吴复谁?此方之民,思为臣妾,延颈举踵,惟恐兵来之迟耳。若使圣恩少加信纳,当以河北承望王师,款心赤实,天日是鉴。而光去经年,不闻咳唾,未审此意竟得达不?瞻望长叹,日月以几,鲁望高子,何足以喻!又臣今日见待稍薄,苍蝇之声,绵绵不绝,必受此祸,迟速事耳。臣私度陛下未垂明慰者,必以臣质贯穿仁义之道,不行若此之事,谓光所传,多虚少实,或谓此中有他消息,不知臣质构谗见疑,恐受大害也。且臣质若有罪之日,自当奔赴鼎镬,束身待罪,此盖人臣之宜也。今日无罪,横见谮毁,将有商鞅、白起之祸。寻惟事势,去亦宜也。死而弗义,不去何为!乐毅之出,吴起之走,君子伤其不遇,未有非之者也。愿陛下推古况今,不疑怪於臣质也。又念人臣获罪,当如伍员奉己自效,不当徼幸因事为利。然今与古,厥势不同,南北悠远,江湖隔绝,自不举事,何得济免!是以忘志士之节,而思立功之义也。且臣质又以曹氏之嗣,非天命所在,政弱刑乱,柄夺於臣,诸将专威於外,各自为政,莫或同心,士卒衰耗,帑藏空虚,纲纪毁废,上下并昬,想前后数得降叛,具闻此问。兼弱攻昧,宜应天时,此实陛下进取之秋,是以区区敢献其计。今若内兵淮、泗,据有下邳,荆、扬二州,闻声响应,臣从河北席卷而南,形势一连,根牙永固。关西之兵系於所卫,青、徐二州不敢彻守,许、洛馀兵众不满万,谁能来东与陛下争者?此诚千载一会之期,可不深思而熟计乎!及臣所在,既自多马,加以羌胡常以三四月中美草时,驱马来出,隐度今者,可得三千馀匹。陛下出军,当投此时,多将骑士来就马耳。此皆先定所一二知。凡两军不能相究虚实,今此间实羸,易可克定,陛下举动,应者必多。上定洪业,使普天一统,下令臣质建非常之功,此乃天也。若不见纳,此亦天也。愿陛下思之,不复多陈。”

其三曰:“昔许子远舍袁就曹,规画计较,应见纳受,遂破袁军,以定曹业。向使曹氏不信子远,怀疑犹豫,不决於心,则今天下袁氏有也。愿陛下思之。间闻界上将阎浮、赵楫欲归大化,唱和不速,以取破亡。今臣款款,远授其命,若复怀疑,不时举动,令臣孤绝,受此厚祸,即恐天下雄夫烈士欲立功者,不敢复讬命陛下矣。愿陛下思之。皇天后土,实闻其言。”此文既流行,而质已入为侍中矣。

二年,青州人隐蕃归吴,上书曰:“臣闻纣为无道,微子先出;高祖宽明,陈平先入。臣年二十二,委弃封域,归命有道,赖蒙天灵,得自全致。臣至止有日,而主者同之降人,未见精别,使臣微言妙旨,不得上达。於邑三叹,曷惟其已。谨诣阙拜章,乞蒙引见。”权即召入。蕃谢答问,及陈时务,甚有辞观。综时侍坐,权问何如,综对曰:“蕃上书,大语有似东方朔,巧捷诡辩有似祢衡,而才皆不及。”权又问可堪何官,综对曰:“未可以治民,且试以都辇小职。”权以蕃盛论刑狱,用为廷尉监。左将军朱据、廷尉郝普称蕃有王佐之才,普尤与之亲善,常怨叹其屈。后蕃谋叛,事觉伏诛,普见责自杀。据禁止,历时乃解。拜综偏将军,兼左执法,领辞讼。辽东之事,辅吴将军张昭以谏权言辞切至,权亦大怒,其和协彼此,使之无隙,综有力焉。

性嗜酒,酒后欢呼极意,或推引杯觞,搏击左右。权爱其才,弗之责也。

凡自权统事,诸文诰策命,邻国书符,略皆综之所造也。初以内外多事,特立科,长吏遭丧,皆不得去,而数有犯者。权患之,使朝臣下议。综议以为宜定科文,示以大辟,行之一人,其后必绝。遂用综言,由是奔丧乃断。

赤乌六年卒,子冲嗣。冲平和有文幹,天纪中为中书令。

徐详者字子明,吴郡乌程人也,先综死。

评曰:是仪、徐详、胡综,皆孙权之时幹兴事业者也。仪清恪贞素,详数通使命,综文采才用,各见信任,辟之广夏,其榱椽之佐乎!

\part{吴书十八}
\chapter{吴范刘惇赵达传第十八}

吴范字文则,会稽上虞人也。以治历数,知风气,闻於郡中。举有道,诣京都,世乱不行。会孙权起於东南,范委身服事,每有灾祥,辄推数言状,其术多效,遂以显名。

初,权在吴,欲讨黄祖,范曰:“今兹少利,不如明年。明年戊子,荆州刘表亦身死国亡。”权遂征祖,卒不能克。明年,军出,行及寻阳,范见风气,因诣船贺,催兵急行,至即破祖,祖得夜亡。权恐失之,范曰:“未远,必生禽祖。”至五更中,果得之。刘表竟死,荆州分割。

及壬辰岁,范又白言:“岁在甲午,刘备当得益州。”后吕岱从蜀还,遇之白帝,说备部众离落,死亡且半,事必不克。权以难范,范曰:“臣所言者天道也,而岱所见者人事耳。”备卒得蜀。

权与吕蒙谋袭关羽,议之近臣,多曰不可。权以问范,范曰:“得之。”后羽在麦城,使使请降。权问范曰:“竟当降否?”范曰:“彼有走气,言降诈耳。”权使潘璋邀其径路,觇候者还,白羽已去。范曰:“虽去不免。”问其期,曰:“明日日中。”权立表下漏以待之。及中不至,权问其故,范曰:“时尚未正中也。”顷之,有风动帷,范拊手曰:“羽至矣。”须臾,外称万岁,传言得羽。

后权与魏为好,范曰:“以风气言之,彼以貌来,其实有谋,宜为之备。”刘备盛兵西陵,范曰:“后当和亲。”终皆如言。其占验明审如此。

权以范为骑都尉,领太史令,数从访问,欲知其决。范秘惜其术,不以至要语权。权由是恨之。

初,权为将军时,范尝白言江南有王气,亥子之间有大福庆。权曰:“若终如言,以君为侯。”及立为吴王,范时侍宴,曰:“昔在吴中,尝言此事,大王识之邪?”权曰:“有之。”因呼左右,以侯绶带范。范知权欲以厌当前言,辄手推不受。及后论功行封,以范为都亭侯。诏临当出,权恚其爱道於己也,削除其名。

范为人刚直,颇好自称,然与亲故交接有终始。素与魏滕同邑相善。滕尝有罪,权责怒甚严,敢有谏者死,范谓滕曰:“与汝偕死。”滕曰:“死而无益,何用死为?“范曰:“安能虑此,坐观汝邪?”乃髡头自缚诣门下,使铃下以闻。铃下不敢,曰:“必死,不敢白。”范曰:“汝有子邪?”曰:“有。”曰:“使汝为吴范死,子以属我。”铃下曰:“诺。”乃排閤入。言未卒,权大怒,欲便投以戟。逡巡走出,范因突入,叩头流血,言与涕并。良久,权意释,乃免滕。滕见范谢曰:“父母能生长我,不能免我於死。丈夫相知,如汝足矣,何用多为!”

黄武五年,范病卒。长子先死,少子尚幼,於是业绝。权追思之,募三州有能举知术数如吴范、赵达者,封千户侯,卒无所得。

刘惇字子仁,平原人也。遭乱避地,客游庐陵,事孙辅。以明天官达占数显於南土。每有水旱寇贼,皆先时处期,无不中者。辅异焉,以为军师,军中咸敬事之,号曰神明。

建安中,孙权在豫章,时有星变,以问惇,惇曰:“灾在丹杨。”权曰:“何如?”曰:“客胜主人,到某日当得问。”是时边鸿作乱,卒如惇言。

惇於诸术皆善,尤明太乙,皆能推演其事,穷尽要妙,著书百馀篇,名儒刁玄称以为奇。惇亦宝爱其术,不以告人,故世莫得而明也。

赵达,河南人也。少从汉侍中单甫受学,用思精密,谓东南有王者气,可以避难,故脱身渡江。治九宫一算之术,究其微旨,是以能应机立成,对问若神,至计飞蝗,射隐伏,无不中效。或难达曰:“飞者固不可校,谁知其然,此殆妄耳。”达使其人取小豆数斗,播之席上,立处其数,验覆果信。尝过知故,知故为之具食。食毕,谓曰:“仓卒乏酒,又无嘉肴,无以叙意,如何?”达因取盘中只箸,再三从横之,乃言:“卿东壁下有美酒一斛,又有鹿肉三斤,何以辞无?“时坐有他宾,内得主人情,主人惭曰:“以卿善射有无,欲相试耳,竟效如此。”遂出酒酣饮。又有书简上作千万数,著空仓中封之,令达算之。达处如数,云:“但有名无实。”其精微若是。

达宝惜其术,自阚泽、殷礼皆名儒善士,亲屈节就学,达秘而不告。太史丞公孙滕少师事达,勤苦累年,达许教之者有年数矣,临当喻语而辄复止。滕他日赍酒具,候颜色,拜跪而请,达曰:“吾先人得此术,欲图为帝王师,至仕来三世,不过太史郎,诚不欲复传之。且此术微妙,头乘尾除,一算之法,父子不相语。然以子笃好不倦,今真以相授矣。”饮酒数行,达起取素书两卷,大如手指,达曰:“当写读此,则自解也。吾久废,不复省之,今欲思论一过,数日当以相与。”滕如期往,至乃阳求索书,惊言失之,云:“女婿昨来,必是渠所窃。”遂从此绝。

初孙权行师征伐,每令达有所推步,皆如其言。权问其法,达终不语,由此见薄,禄位不至。

达常笑谓诸星气风术者曰:“当回算帷幕,不出户牖以知天道,而反昼夜暴露以望气祥,不亦难乎!”间居无为,引算自校,乃叹曰:“吾算讫尽某年月日,其终矣。”达妻数见达效,闻而哭泣。达欲弭妻意,乃更步算,言:“向者谬误耳,尚未也。”后如期死。权闻达有书,求之不得,乃录问其女,及发棺无所得,法术绝焉。

评曰:三子各於其术精矣,其用思妙矣,然君子等役心神,宜於大者远者,是以有识之士,舍彼而取此也。

\part{吴书十九}
\chapter{诸葛滕二孙濮阳传第十九}

诸葛恪字元逊,瑾长子也。少知名。弱冠拜骑都尉,与顾谭、张休等侍太子登讲论道艺,并为宾友。从中庶子转为左辅都尉。恪父瑾面长似驴。孙权大会群臣,使人牵一驴入,长检其面,题曰诸葛子瑜。恪跪曰:“乞请竺益两字。因听与笔。恪绩其下曰:”之驴。“举座欢笑,乃以驴赐恪。他日复见,权问恪曰:”卿父与叔父孰贤?“对曰:”臣父为优。“权问其故。对曰:”臣父知所事,叔父不知,以是为优。“权又大噱。

命恪行酒,至张昭前,昭先有酒色,不肯饮。曰:“此非养老之礼也。”权曰:“卿其能令张公辞屈,乃当饮之耳。”恪难昭曰:“昔师尚父九十,秉旄仗钺,犹未告老也。

今军旅之事,将军在后,酒食之事,将军在先,何谓不养老也?“昭卒无辞,遂为尽爵。

后蜀好,群臣并会,权谓使曰:“此诸葛恪雅使至骑乘,还告丞相,为致好马。”恪因下谢,权曰:“马未至面谢何也?”恪对曰:“夫蜀者陛下之外厩,今有恩诏,马必至也,安敢不谢?”恪之才捷,皆此类也。权甚异之,欲试以事,令守节度。节度掌军粮谷,文书繁猥,非其好也。

恪以丹杨山险,民多果劲,虽前发兵,徒得外县平民而已。其余深远,莫能禽尽,屡自求乞为官出之。三年可得甲士四万。众议咸以“丹杨地势险阻,与吴郡、会稽、新都、鄱阳四郡邻接,周旋数千里,山谷万重,其幽邃民人,未尝人城邑,对长吏,皆仗兵野逸,白首于林莽。逋亡宿恶,咸共逃窜。山出铜铁,自铸甲兵。俗好武习战,高尚气力,其升山赴险,抵突丛棘。若鱼之走渊,猨狖之腾木也。时观间隙,出为寇盗,每致兵征伐,寻其窟藏。其战则蜂至,败则鸟窜,自前世以来,不能羁也”。皆以为难。

恪父瑾闻之,亦以事终不逮,叹曰:“恪不大兴吾家,将大赤吾族也。”恪盛陈其必捷。

权拜恪抚赵将军,领丹杨太守,授棨戟武骑三百。拜毕,命恪备威仪,作鼓吹,导引归家,时年三十二。恪到府,乃移书四部属城长空。令各保其疆界,明立部伍,其从化平民,悉令屯居。乃分内诸将,罗兵幽阻,但缮藩篱,不与交锋,候其谷稼将熟,辄纵兵芟刈,使无遗种。旧谷既尽,新田不收,平民屯居,略无所入,于是山民饥穷,渐出降首。恪乃复敕下曰:“山民去恶从化,皆当抚慰,徙出外县,不得嫌疑,有所执拘。”

臼阳长胡伉得降民周遗,遗旧恶民,困迫暂出,内图叛逆,伉缚送言府。恪以伉违教,遂斩以徇,以状表上。民闻伉坐执人被戮,知官惟欲出之而已,于是老幼相携而出,岁期,人数皆如本规。恪自领万人,余分给诸将。

权嘉其功,遣尚书仆射薛综劳军。综先移恪等曰:“山越恃阻,不宾历世,缓则首鼠,急则狼顾。皇帝赫然,命将西征,神策内授,武师外震。兵不染锷,甲不沾汗。元恶既枭,种党归义,荡涤山薮,献戎十万。野无遗寇,邑罔残奸。既扫凶慝,又充军用。

藜莜稂莠,化为善草。魑魅魍魉,更成虎土。虽实国家威灵之所加,亦信无帅临履之所致也。虽《诗》美执讯,《易》嘉折首,周之方、召,汉之卫、霍,岂足以谈?功轶古人,勋超前世。主上欢然,遥用叹息。感《四牡》之遗典,思饮至之旧章。故遣中台近官,迎致稿赐,以旌茂功,以慰劬劳“拜恪威北将军,封都乡侯。恪乞率众佃庐江皖口,因轻兵袭舒,掩得其民而还。复远遣斥候,观相径要,欲图寿春,权以为不可。

赤乌中,魏司马宣王谋欲攻恪。权方发兵应之,望气者以为不利,于是徒恪屯于柴桑。与丞相陆逊书曰:“杨敬叔传述清论,以为方今人物凋尽,守德业者不能复几,宜相左右。更为辅车,上熙国事,下相珍惜。又疾世俗好相谤毁,使已成之器,中有损累。

将进之徒,意不欢笑,闻此喟然,诚独击节。愚以为君子不求备于一人,自孔氏门徒大数三干,其见者七十二人。至于子张、子路、子贡等七十之徒,亚圣之德,然犹各有所短,师辟由喭,赐不受命,岂况下此而无所阙?且仲尼不以数予之不备而引以为友,不以人所短弃其所长也。加以当今取士,宜宽于往古,何者?时务从横,而善人单少,国家职司,常苦不充。苟令性不邪恶,志在陈力,便可奖就,骋其所任。若于小小宜适,私行不足,皆宜阔略,不足缕责。“且士诚不可纤论苛克,苛克则彼贤圣犹将不全,况其出入者邪?故曰以道望人则难,以人望人则易,贤愚可知。

自汉末以来,中国土大夫如许子将辈,所以更相谤讪,或至为祸,原其本起。非为大仇,惟坐克己不能尽如礼,而责人专以正义。夫己不如礼,则人不服。责人以正义,则人不堪。内不服其行,外不堪其责,则不得不相怨。相怨一生,则小人得容其间。得容其间,则三至之言,浸润之谮,纷错交至。虽使至明至亲者处之,犹难以自定。况已为隙,且未能明者乎?是故张、陈至于血刃,萧、朱不终其好,本由于此而已。夫不舍小过,纤微相责,久乃至于家户为怨,一国无复全行之士也。“恪知逊以此嫌己,故遂广其理而赞其旨也。会逊卒,恪迁大将军,假节,驻武昌,代逊领荆州事。

久之,权不豫,而太子少,乃征恪以大将军领太子太傅,中书令孙弘领少傅。权疾困,召恪、弘及太常滕胤、将军吕据、侍中孙峻,属以后事。

翌日,权薨。弘素与恪不平,惧为恪所治,秘权死问,欲矫诏除恪。峻以告恪,恪请弘咨事,于坐中诛之,乃发丧制服。与弟公安督融书曰:“今月十六日乙未,大行皇帝委弃万国,群下大小,莫不伤悼。至吾父子兄弟,井受殊恩,非徒凡庸之隶,是以悲恸,肝心圮裂。皇太子以丁酉践酋号,哀喜交并,不知所措。吾身受顾命,辅相幼主,窃自揆度;才非博陆而受姬公负图之托,惧忝丞相辅汉之效;恐损先帝委付之明,是以忧惭惶惶,所虑万端。且民恶其上,动见瞻观,何时易哉?今以顽钝之姿,处保傅之位,艰多智寡,任重谋浅,谁为唇齿?近汉之世,燕、盖交遘,有上官之变,以身值此,何敢怡豫邪?又弟所在,与贼犬牙相错,当于今时整顿军具,率厉将士,警备过常,念出万死,无顾一生,以报朝廷,无忝尔先。又诸将备守各有境界,犹恐贼虏闻讳,恣睢寇窃。边邑诸曹,已别下约敕,所部督将,不得妄委所戍,径来奔赴。虽怀怆但不忍之心,公义夺私,伯禽服戎,若苟违戾,非徒小故。以亲正疏,古人明戒也。”

恪更拜太傅。于是罢视听,息校官,原逋责,除关税,事崇恩泽,众莫不悦。恪每出入,百姓延颈思见其状。

初,权黄龙元年迁都建业。二年筑东兴堤遏湖水。后征淮南,败,以内船,由是废不复修。恪以建兴元年十月会众于东兴,更作大堤,左右结山侠筑两城,各留千人,使全端、留略守之,引军而还。魏以吴军入其疆土,耻于受侮,命大将胡遵、诸葛诞等率众七万,欲攻围两坞,图坏堤遏。恪兴军四万,晨夜赴救。遵等敕其诸军作浮桥度,陈于堤上,分兵攻两城。城在高峻,不可卒拔。恪遣将军留赞、吕据、唐咨、丁奉为前部。

时天寒雪,魏诸将会饮,见赞等兵少,而解置铠甲,不持矛戟。但兜鍪刀楯,倮身缘遏,大笑之,不即严兵。兵得上,便鼓噪乱斫。魏军惊扰散走,争渡浮桥,桥坏绝,自投于水,更相蹈藉。乐安太守恒嘉等同时并没,死者数万。故叛将韩综为魏前军督,亦斩之。

获车乘牛马驴骡各数千,资器山积,振旅而归。进封恪阳都侯,加荆扬州牧,督中外诸军事,赐金一百斤,马二百匹,缯布各万匹。

恪遂有轻敌之心,以十二月战克,明年春,复欲出军。诸大臣以为数出罢劳,同辞谏恪,恪不听。中散大夫蒋延或以固争,扶出。恪乃着论谕众意曰:“夫天无二日,土无二王,王者不务兼并天下而欲垂祚后世,古今未之有也。昔战国之时,诸候自恃兵强地广,互有救援,谓此足以传世,人莫能危。恣情从怀,惮于劳苦,使秦渐得自大,遂以并之,此既然矣。近者刘景升在荆州,有众十万,财谷如山。不及曹操尚微,与之力竞,坐观其强大,吞灭诸袁,北方都定之后,操率三十万众来向荆州,当时虽有吞智者,不能复为画计,于是景升儿子,交臂请降,遂为囚虏。凡敌国欲相吞,即仇雦欲相除也,有仇而长之,祸不在己,则在后人,不可不为远虑也。昔伍子胥曰:”越十年生聚,十年教训,二十年之外,吴其为沼乎!‘夫差自恃强大,闻此邈然,是以诛子胥而无备越之心,至于临败悔之,岂有及乎?越小于吴,尚为吴祸,况其强大者邪?昔秦但得关西耳,尚以并吞六国,今贼皆得秦、赵、韩、魏、燕齐九州之地,地悉戎马之乡,士林之薮。

今以魏比古之秦,土地数倍;以吴与蜀比古穴国,不能半之。然所以能敌之,但以操时兵众于今适尽,而后生者未悉长大,正是贼衰少未盛之时。加司马懿先诛王淩,续自陨毙,其子幼弱,而专彼大任,虽有智计之士,未得施用。当今伐之,是其厄会。圣人急于趋时,诚谓今日。若顺众人之情,怀偷安之计,以为长江之险可以传世;不论魏之终始,而以今日遂轻其后。此吾所以长叹息者也。自本以来,务在产育,今者贼民岁月繁滋,但以尚小,未可得用耳。

若复十数年后,其众必倍于今,而国家劲兵之地,皆已空尽,唯有此见众可以定事。

若不早用之,端坐使老,复十数年,略当损半,而见子弟数不足言。若贼众一倍,而我兵损半,虽复使伊、管图之,未可如何。今不达远虑者,必以此言为迂。夫祸难未至而豫忧虑,此固众人之所迂也。及于难至,然后顿颡,虽有智者,又不能图。此乃古今所病,非独一时。昔吴始以伍员为迂,故难至而不可救。刘景升不能虑十年之后,故无以治其子孙。今恪无具臣之才,而受大吴萧、霍之任,智与众同思不经远,若不及今日为国斥境,俯仰年老,而仇敌更强。欲刎颈谢责,宁有补邪?今闻众人或以百姓尚贫,欲务闲息,此不右其虑其大危而其小勤者也。昔汉祖幸已自有三秦之地,何不闭关守险以自娱乐,空出攻楚,身被创痍,介胄生虮虱,将士厌困苦,岂甘锋刃而忘安宁哉?虑于长久不得两存者耳!每览荆邯说公孙述以进取之图,近风家叔父表陈与贼争竞之计,未尝不喟然叹息也。夙夜反侧,所虑如此,故聊疏愚言,以达二三君子之末。若一朝陨殁志画不立,贵令来世知我所忧,可思于后,“众皆以恪此论欲必为之辞,然莫敢复难。

丹杨太守聂友素与恪善。书谏恪曰:“大行皇帝本有遏东关之计,计未施行。今公辅赞大业,成先帝之志。寇远自送,将士凭赖威德,出身用命,一旦有非常之功,岂非宗庙神灵社稷之福邪!宜且案兵养锐,观衅而动。今乘此势欲复大出,天时未可。而苟任盛意,私心以为不安。”恪题论后,为书答友曰:“足下虽有自然之理,然未见大数。

熟省此论,可以开悟矣。“于是违众出军,大发州郡二十万众,百姓骚动,始失人心。

恪意欲曜威淮南,驱略民人。而诸将或难之曰:“今引军深入,疆场之民,必相率远遁,恐兵劳而功少,不如止围新城。新城困,救必至,至而图之,乃可大获。”恪从其计,回军还围新城。攻守连月,城不拔。士卒疲劳,因暑饮水,泄下、流肿,病者大半,死伤涂地。诸营吏日白病者多,恪以为作,欲斩之,自是莫敢言。恪内惟失计,而耻城不下,忿形于色。将军朱异有所是非,恪怒,立夺其兵。都尉蔡林数陈军计,恪不能用,策马奔魏。魏知战士罢病,乃进救兵。恪引军而去。士卒伤病,流曳道路,或顿仆坑壑,或见略获,存记忿痛,大小呼嗟。而恪宴然自若。出住江渚一月,图起田于浔阳,诏召相衔,徐乃旋师。由此众庶失望,而怨黩兴矣。

秋八月军还,陈兵导从,归入府馆。即召中书令孙嘿,厉声谓曰:“卿等何敢妄数作诏?”嘿惶惧辞出,因病还家。恪征行之后,曹所奏署令长职司,一罢更选,愈治威严,多所罪责,当进见者无不竦息。又改易宿卫,用其亲近。复敕兵严,欲向责、徐。

孙峻因民之多怨,众之所嫌,构恪欲为变,与亮谋,置酒请恪。恪将见之夜,精爽扰动,通夕不寐。明将盥漱,闻水腥臭,侍者授衣,衣服亦臭。恪怪其故,易衣易水,其臭如初,意惆怅不悦。严毕趋出,犬衔引其衣,恪曰:“犬不欲我行乎?”还坐,顷刻乃复起,犬又衔其衣,恪令从者逐犬,遂升车。

初,恪将征淮南,有孝子着缞衣入其阁中,从者白之,令外诘问,孝子曰:“不自觉入。”时中外守备,亦悉不见,众皆异之。出行之后,所坐厅事屋栋中折。自新城出住东兴,有白虹见其船,还拜蒋陵,白虹复绕其车。及将见,驻车宫门,峻已伏兵于帷中,恐恪不时入,事泄,自出见恪曰:“使君若尊体不安,自可须后,峻当具白主上。”

欲以尝知恪。恪答曰:“当自力入。”散骑常侍张约、朱恩等密书与恪曰:“今日张设非常,疑有他故。”恪省书而去。未出路门,逢太常滕胤,恪曰:“卒腹痛,不任人。”

胤不知峻阴计,谓恪曰:“君自行旋未见,今上置酒请君,君已至门,宜当力进。”恪踌躇而还,剑履上殿。谢亮,还坐。设酒,恪疑未饮,峻因曰:“使君病未善平,当有常服药酒,自可取之。”恪意乃安,别饮所赍酒。酒数行,亮还内,峻起如厕,解长衣,着短服,出曰:“有诏收诸葛恪!”恪惊起,拔剑未得,而峻刀交下。张约从旁斫峻,裁伤左手,峻应手所约断右臂。武卫之士皆趋上殿,峻云:“所取者恪也,今已死。”

悉令复刃,乃除地更饮。

先是,童谣曰:“诸葛恪,芦苇单衣蔑钩落,于何相求成子阁。”成子阁者,反语石子冈也。建业面有长陵,名曰石子冈,葬者依焉。钩落者,校饰革带,世谓之钩络带。

恪果以苇席裹其身而篾束其腰,投之于此冈。恪长子绰,骑都尉,以交关鲁王事,权遣付恪,令更教诲,恪鸩杀之。中子竦,长水校尉。少子建,步兵校尉。闻恪诛,车载其母而走。峻遣骑督承追斩竦于白都。建得渡江,欲北走魏,行数千里,为追兵所逮。恪外甥都乡侯张震及常侍朱恩等,皆夷三族。

初,竦数谏恪,恪不从,常忧惧祸。及亡,临淮臧均表乞收葬恪曰:“臣闻震雷电激,不崇一朝,大风冲发,杀有极日。然犹继以云雨,因以润物,是则天地之威,不可经日浃辰,帝王之怒,不宜讫情尽意,臣以狂愚,不知忌讳,敢冒破灭之罪,以邀风雨之会。伏念故太傅诸葛恪得承祖考风流之烈,伯叔诸父遭汉祚尽,九州鼎立,分托三方,并履忠勤,熙隆世业。爰及于恪,生长王国,陶育圣化,致名英伟,服事累纪,祸心未萌,先帝委以伊、周之任,属以万机之事。恪索性刚履,矜己陵人,不能敬守神器,穆静帮内,兴功暴师,未期三出,虚耗士民,空竭府藏,专擅国宪,废易由意,假刑劫众,大小屏息。侍中武卫将军都乡候俱受先帝嘱寄之诏,见其奸虐,日月滋甚,将恐荡摇宇宙,倾危社稷,奋其威怒,精贯昊天,计虑先于神明,智勇百于荆、聂,躬持白刃,枭恪殿堂,勋超朱虚,功越东牟。国之元害,一朝大除,驰首徇示,六军喜踊,日月增光,风尘不动,斯实宗庙之神灵,天人之同验也。今恪父子三首,悬市积日,观者数万,詈声成风。国之大刑,无所不震,长老孩幼,无不华见。人情之于品物,乐极则哀生,见恪贵盛,世莫与贰,身处台辅,中间历年,今之诛夷,无异禽兽,观讫情反,能不憯然!

且已死之人,与土壤同域,凿掘斫刺,无所复加。愿圣朝稽则乾坤,怒不极旬,使其乡邑若故吏民收以士伍之服,惠以三寸之棺。昔项籍受殡葬之施,韩信获收敛之恩,斯则汉高发神明之誉也。惟陛下敦三皇之仁,垂哀矜之心,使国泽加于辜戮之骸,复受不已之恩,于以扬声遐方,沮劝天下,岂不弘哉!昔栾布矫命彭越,臣窃恨之,不先请主上,而专名以肆情,其得不诛,实为幸耳。今臣不敢章宣愚情以露天恩,谨伏手书,冒昧陈闻,乞圣朝哀察。“于是亮、峻听恪故吏敛葬,遂求之于石子冈。

始恪退军还,聂友知其将败。书与滕胤曰:“当人强盛,河山可拔,一朝羸缩,人情万端,言之悲叹。”恪诛后,孙峻忌友。欲以为郁林太守,友发病忧死。友字文悌,豫章人也。

滕胤字承嗣,北海剧人也。伯父耽,父胄,与刘繇州里通家。以世扰乱,渡江依繇。

孙权为车骑将军,拜耽右司马,以宽厚称,早卒,无嗣。胄善属文,权待以宾礼,军国书疏,常令损益润色之,亦不幸短命。权为吴王,迫录旧恩,封胤都亭侯。少有节操,美容仪。弱冠尚公主。年三十,起家为丹杨太守,徙吴郡、会稽,所在见称。太元元年,权寝疾,诣都,留为太常;与诺葛恪等俱受遗诏辅政。孙亮即位,加卫将军。

恪将悉众伐魏。胤谏恪曰:“君以丧代之际,受伊、霍之托,入安本朝,出摧强敌,名声振于海内,天下莫不震动,万姓之心,冀得蒙君而息。今猥以劳役之后,兴师出征,民疲力屈,远主有备。若攻城不克,野略无获,是丧前劳而招后责也。不如案甲息师,观隙而动。且兵者大事,事以众济,众苟不悦,君独安之?”恪曰:“诸云不可者,皆不见计算,怀居苟安者也,而子复以为然,吾何望焉?夫以曹劳暗劣,而政在私门,彼之臣民,固有离心。今吾因国家之资,借战胜之威,则何往而不克哉!”以胤为都下督,掌统留事。胤白日接宾客,夜省文书,或通晓不寐。

孙峻字子远,孙坚弟静之曾孙也。静生皓,皓生恭,为散骑侍郎。恭生峻。少便弓马,精果胆决。孙权末,徙武卫都尉,为侍中。权临薨,受遗辅政,领武卫将军,故典宿卫,封都乡侯。既诛诸葛恪,迁丞相大将军,督中外诸军事、假节,进封富春侯。滕胤以恪子竦妻父辞位。峻曰:“鲧、禹罪不相及,滕侯何为?”峻、胤虽内不沾洽,而外相包容,进胤爵高密侯,共事如前。峻素无重名,骄矜险害,多所刑杀,百姓嚣然。

又奸乱宫人,与公主鲁班私通。五凤元年,吴侯英谋杀峻,英事泄死。

二年,魏毋丘俭、文钦以众叛,与魏人战于乐嘉,峻帅骠骑将军吕据、左将军留赞袭寿春,会钦败降,军还。是岁,蜀使来聘,将军孙仪、孙邵綝恂等欲因会杀峻。事泄,仪等自杀,死者数十入,并及公主鲁育。

峻欲城广陵,朝臣知其不可城,而畏之莫敢言。唯滕胤谏止,不从,而功竟不就。

其明年,文钦说峻征魏,峻使钦与吕据、车骑刘纂、镇南朱异、前将军唐咨自江都人淮、泗,以图青、徐。峻与胤至石头,因饯之,领从者百许人入据营。据御军齐整,峻恶之,称心痛去。遂梦为诸葛恪所击,恐惧发病死,时年三十八,以后事付綝。

孙綝字子通,与峻同祖。綝父绰为安民都尉。綝始为偏将军,及峻死,为待中武卫将军,领中外诸军事,代知朝政。吕据闻之大恐,与诸督将连名,共表荐滕胤为丞相,綝以胤为大司马,代吕岱驻武昌。据引兵还,使人报胤,欲共废綝.綝闻之,遣从兄虑将兵逆据于江都,使中使敕文钦、刘纂、唐咨等合众击据,遣侍中左将军华融、中书丞丁晏告胤取据,并喻胤宜速去意,胤自以祸及,因留融、晏,勒兵自卫,召典军扬崇、将军孙咨,告以綝为乱,迫融等使有书难綝.綝不听,表言胤反,许将军刘丞以封爵,使率兵骑急攻围胤。胤又劫融等使诈诏发兵。融等不从,胤皆杀之。胤颜色不变,谈笑若常。或劝胤引兵至苍龙门,“将士见公出,必皆委綝就公”。时夜已半,胤恃与据期。

又难举兵向富,乃约令部曲,说吕侯以在近道,故皆为胤尽死,无离散者。时大风,比晓,据不至。綝兵大会,遂杀及将士数十人,夷胤三族。

綝迁大将军,假节,封永宁侯,负贵倨傲,多行无礼。初,峻从弟虑与诸葛恪之谋,峻厚之,至右将军、无难督,授节盖,平九官事。綝遇虑薄于峻时,虑怒,与将军王惇谋杀綝.綝杀惇。虑服药死。

魏大将军诸葛诞举寿春叛,保城请降。吴遣文钦、唐咨、全端、全怿等三万人救之。

魏镇南将军王基围入诞。钦等突围城。魏悉中外军二十余万增诞之围。朱异帅三万人屯安丰城,为文钦势。魏兖州刺史州泰据异于阳渊,异败退,为泰所追,死伤二干人。林于是大发率出屯镬里,复遣异率将军丁奉、黎斐等五万人攻魏,留辎重于都陆。异屯黎浆,遣将军任度、张震等慕勇敢六千人,于屯西六里为浮桥夜渡,筑偃月垒。为魏监军石苞及州泰所破,军却退就高。异复作车箱围趣五木城。苞、泰攻异,异败归,而魏太山太守胡烈以奇兵五千诡道袭都陆,尽焚异资粮。綝授兵三万人使异死战,异不从,綝斩之于镬里,而遣弟恩救。会诞败引还。綝既不能拔出诞,而丧败士众,自戮名将,莫不怨之。

綝以孙亮始亲政事,多所难问,甚惧。还建业,称疾不朝。筑室干朱雀桥南,使弟威远将军据入苍龙宿卫,弟武卫将军恩、偏将军干、长水校尉闿分屯诸营,欲以专朝自固。亮内嫌綝,乃推鲁育见杀本末,责怒虎林督朱熊、熊弟外部督朱损不匡正孙峻,乃令丁奉杀熊于虎林,杀损于建业。綝入谏不从,亮遂与公主鲁班、太常全尚、将军刘承议诛綝.亮妃,綝从姊女也,以其谋告綝.綝率众夜袭全尚,遣弟恩杀刘承于苍龙门外,遂围宫。使光禄勋盂宗告庙废亮,召群司仪曰:“少帝荒病昏乱,不可以处大位,承宗庙,以告先帝废之。诸君若有不同者,下异议。”皆震怖。曰:“唯将军令。”綝遣中书郎李祟夺亮玺绶,以亮罪状班告远近。尚书桓彝不肯署名,綝怒杀之。

典军施正劝綝征立琅邪王休,綝从之。遣宗正楷奉书于休曰:“綝以酶才,见授大任,不能辅导陛下。顷月以来,多所造立。亲近刘承,悦于美色;发吏民妇女,料其好者,留于宫内,取兵弟十八已下三千余人,习之苑中,连日续夜,大小呼嗟,败坏藏中矛戈五千余枚,以作戏具。朱据先帝旧臣,子男熊、损皆承父之基,以忠议自立,昔杀小主。自是大主所创,帝不复精其本未,便杀熊、损,谏不见用。诸下莫不侧息。帝于宫中作小船三百余艘,成以金银,师工昼夜不息。太常全尚,累世受恩,不能督诸宗亲,而全端等委城就魏。尚位过重,曾无一言以谏陛下,而与敌往来,使传国消息,惧必倾危社稷。推案旧典,运集大王,辄以今月二十七日擒尚斩承。以帝为会稽王,遣楷牵迎。

百寮喁喁。立任道侧。“

綝遣将军孙耽送亮之国,徙尚于零陵,迁公主于豫章。綝意弥溢,侮慢民神,遂烧大桥头伍子胥庙,又坏浮屠祠,斩道人。休既即位,称草莽臣。诣阙上书曰:“臣伏自省,才非干国,因缘肺腑,位极人臣,伤锦败驾,罪负彰露,寻愆惟阙,夙夜忧惧。臣闻天命棐谌,必就有德,是以幽、厉失度,阂宣中兴,陛下圣德,纂承大统,宜得良辅;以协雍熙,虽尧之盛,犹求稷契之佐;以协明圣之德。古人有言:”陈力就列,不能者止。‘臣虽自展竭,无益庶政,谨上印绶节钺,退还田里,以避贤路。“休引见慰喻。

又下诏曰:“朕以不德,守藩于外,值兹际会,群公卿士,暨于朕躬,以奉宗庙。朕用抚然,若涉渊冰。大将军忠计内发,扶危定倾,安康社稷,功勋赫然。昔汉孝宣践阼,霍光尊显,褒德赏功,古今之通义也。其以大将军为丞相、荆州牧,食五县。”恩为御史大夫、卫将军,据右将军。皆县侯。干杂号将军、亭侯。闿亦封亭侯。綝一门五侯,皆典禁兵,权倾人主,自吴国朝臣未尝有也。

綝奉牛酒诣休,休不受,赍诣左将军张布。酒酣,出怨言曰:“彻废少主时,多劝吾自为之者。吾以陛下贤明,故迎之。帝非我不立,今上礼见拒,是与凡臣无异,当复改图耳。”布以言闻休,休衔之。巩其有变,数加赏赐,又复加恩侍中,与綝分省文书。

或有告綝怀怨侮上欲图反者,休执以付綝,綝杀之。由是愈惧,因孟宗求出屯武昌,休许焉,尽敕所督中营精兵万余人,皆令装载,所取武库兵器,咸令给与。将军魏邈说休曰“綝居外必有变”,武卫士施朔又告“綝欲反有征”休密问张布,布与丁奉谋于会杀綝.永安元年十二月丁卯,建业中谣言明会有变。綝闻之,不悦。夜大风发木扬沙,綝益恐。戊辰腊会,綝称疾。休强起之,使者十余辈。綝不得已,将人,众止焉。綝曰:“国家屡有命,不可辞。可豫整兵,令府内起火,因是可得速还。”遂入,寻而火起,綝求出,休曰:“外兵自多,不足烦丞相也。”綝起离席,奉、布目左右缚之。綝叩首曰:“愿徙交州。”休曰:“卿何以不徙滕胤、吕据?”綝复曰:“愿没为官奴。”休曰:“何不以胤、据为奴乎!”遂斩之。以綝首令其众曰:“诸与綝同谋皆赦。”放仗者五千人。闿乘船欲北降,追杀之。夷三族。发孙峻棺,取其印绶,綝其木而埋之,以杀鲁育等故也。

綝死时年二十八。休耽与峻、綝同族,特除其属籍,称之曰故峻、故綝云。休又下诏曰:“诸葛恪、滕胤、吕据盖以无罪为峻、綝兄弟所见残害,可为痛心,促皆改葬,各为祭奠。其罹恪等事见远徙者,一切召还。

濮阳兴字子元,陈留人也。父逸,汉末避乱江东,官至长沙太守。兴少有士名,孙权时除上虞令,稍迁至尚书左曹,以五官中郎将使蜀,还为会稽太守。时琅邪王休居会稽,兴深与相结。及休即位,征兴为太常卫将军、平军国事,封外黄侯。

永安三年,都尉严密建丹杨湖田,作浦里塘。诏百官会议,咸以为用功多而田不保成,唯兴以为可成。遂会诸兵民就作,功佣之费不可胜数,士卒死亡,或自贼杀,百姓大怨之。兴迁为丞相,与休宠臣左将军张共布相表里,邦内失望。七年七月,休薨。左典军万彧素与乌程侯孙皓善,乃劝兴、布,于是兴、布废休适子而迎立皓。皓既践阼,加兴侍中,领青州牧。俄彧谮兴、布追悔前事。十一年朔入朝,皓因收兴、布,徙广州,道追杀之,夷三族。

评曰:“诸葛恪才气干略,邦人所称,然骄且吝,周公无观,况在于恪?矜己陵人,能无败乎!若躬行所与陆逊及弟融之书,则悔吝不至,何尤祸之有哉?滕胤厉修士操,遵蹈规矩,而孙峻之时犹保其贵,必危之理也。峻、綝凶竖盈溢,固无足论者。濮阳兴身居宰辅,虑不经国,协张布之邪,纳万彧之说,诛夷其宣矣。

\part{吴书二十}
\chapter{王楼贺韦华传第二十}

王蕃字永元,庐江人也。博览多闻,兼通术艺。始为尚书郎,去官。孙休即位,与贺邵、薛莹、虞汜俱为散骑中常侍,皆加驸马都尉。时论清之。遣使至蜀,蜀人称焉,还为夏口监军。

孙皓初。复入为常侍,与万彧同官。彧与皓有旧,俗士挟侵,谓蕃自轻。又中书丞陈声,皓之嬖臣,数谮毁蕃。蕃体气高亮,不能承颜顺指;时或迕意,积以见责。

甘露二年,丁忠使晋还,皓大会群臣,蕃沉醉顿伏。皓疑而不悦,举蕃出外。顷之请还,酒亦不解。蕃性有威严,行止自若,皓大怒,呵左右于殿下斩之。卫将军滕牧、征西将军留平请,不能得。

丞相陆凯上疏曰:“常侍王蕃黄中通理,知天知物,处朝忠蹇,斯社稷之重镇,大吴之龙逢也。昔事景皇,纳言左右,景皇钦嘉,叹为异伦。而陛下忿其苦辞,恶其直对,枭之殿堂,尸骸暴弃,邦内伤心,有识悲悼。”其痛蕃如此。蕃死时年三十九,皓徙蕃家属广州。二弟着、延皆作佳器,郭马起事,不为马用,见害。

楼玄字承先,沛郡蕲人也。孙休时为监农御史。孙皓即位,与王蕃、郭逴、万彧俱为散骑中常侍,出为会稽太守,入为大司农。旧禁中主者自用亲近人作之,彧陈亲密近职宜用好人,皓因敕有司,求忠清之士,以应其选,遂用玄为宫下镇禁中候,主殿中事,玄从九卿持刀侍卫,正身率众,奉法而行,应对切直,数迕皓意,渐见责怒。后人诬白玄与贺邵相逢,驻共耳语大笑,谤讪政事,遂被诏诘责,送付广州。东观令华核上疏曰:“臣窃以治国之体,其犹治家。主田野者,皆宜良信。又宜得一人总其条目,为作维纲,众事乃理。

《论语》曰:“无为而治者其舜也与!恭己正南面而己。‘言所任得其人,放优游而自逸也。今海内未定,天下多事,事无大小,皆当关闻,动经御坐,劳损圣虑。陛下既垂意博古,综极艺文,加勤心好道,随节致气,宜得闲静以展神思,呼翕清淳,与天同极。臣夙夜思惟,诸吏之中,任干之事,足委丈者,无胜于楼玄。玄清忠奉公,冠冕当世,众服其操,无与争先。失清者则心平而意直,忠者惟正道而履之,如玄之性,终始可保,乞陛下赦玄前愆,使得自新,擢之宰司,责其后效。使为官择人,随才授任,则舜之恭己,近亦可得。”皓话玄名声,复徙玄及子据,付交址将张奕,使以战自效,阴别敕奕令杀之。据到交址,病死。玄一身随亦讨贼,持刀步涉,见亦辄拜,亦未忍杀。

会亦暴卒,玄殡敛亦,于器中见敕书,还便自杀。

贺邵字兴伯,会稽山阴人也,孙休即位,从中郎为期骑中常侍,出为吴郡太守。孙皓时,入为左典军,迁中书令,领太子太傅。皓凶暴骄矜,政事日弊。邵上疏谏曰:“古之圣王,所以潜处重闱之内而知万里之情,垂拱衽席之上,明照八极之际者,任贤之功也。陛下以至德淑姿,统承皇业,宜率身履道,恭奉神器,旌贤表善,以康庶政。

自顷年以来,朝列纷错,真伪相贸,上下空任,文武旷位,外无山岳之镇,内无拾遗之臣。佞谀之徒拊冀天飞,干弄朝威,盗窃荣利,而忠良排坠,信臣被害。是以正士摧方,而庸臣苟媚,先意承旨,各希时趣。人执反理之评,士吐诡道之论,遂使清流变浊,忠臣结舌。陛下处九天之上,隐百重之室,言出风靡,令行景从,亲洽宠媚之臣,日闻顺意之辞,将谓此辈实贤,而天下已平也。臣心所不安,敢不以闻。

臣闻兴国之君乐闻其过,荒乱之主乐闻其誉。闻其过者过日消而福臻,闻其誉者誉日损而祸至。是以古之人君,捐让以进贤,虚己以求过,譬天位于乘犇,以虎尾为警戒。

至于陛下,严刑法以禁直辞,黜善士以逆谏臣,眩耀毁誉之实,沉沦近习之言。昔高宗思佐,梦寐得贤,而陛下求之如忘,忽之如遗。故常侍王蕃忠恪在公,才任辅弼,以醉酒之间加之大戮。近鸿胪葛奚,先帝旧臣,偶有逆迕,昏醉之言耳,三爵之后,礼所不讳,陛下猥发雷霆,谓之轻慢,饮之醇酒,中毒陨命。自是之后,海内悼心,朝臣失图,仕者以退为幸,居者以出为福,诚非所以保光洪绪,臣隆道化也。

“又何定本趋走小人,仆隶之下,身无锱铢之行,能无鹰犬之用,而陛下爱其佞媚,假其威柄,使定恃宠放恣,自擅威福,口正国议,手弄天机,上亏日月之明,下塞君子之路。夫小人求人,必进奸利,定间妄兴事役,发江边戍兵以驱麋鹿,结置山陵,芟夷林莽,殚其九野之兽,聚于重围之内,上无益时之分,下有损耗之费。而兵士疲于运送,人力竭于驱逐,老弱饥冻,大小怨叹。臣窃观天变,自比年以来阴阳错谬,四时逆节,日食地震;中夏陨霜,参之典籍,皆阴气陵阳,小人弄势之所致也。臣尝览书传,验诸行事,灾祥之应,所为寒栗。昔高宗修己以消鼎雉之异,宋景崇德以退荧惑之变。愿陛下上惧皇天谴告之诮,下追二君攘灾之道,远览前代任贤之功,近寤今日谬授之失,清澄朝位,旌叙俊乂,放退佞邪,抑夺奸势。如是之辈,一匆复用,广延淹滞,容受直辞,祗承乾指,敬奉先业,则大化光敷,天人望塞也。

《传》曰:“国之兴也,视民如赤子。其亡也,以民为草芥。”陛下昔韬神光,潜德东夏,以圣哲茂姿,龙飞应天,四海延颈,八方拭目,以成康之化必隆于旦夕也。自登位以来,法禁转苛,赋调益繁。中宫内竖,分布州郡,横兴事役,竞造奸利。百姓罹杼轴之困,黎民罢无已之求,老幼饥寒,家户莱色,而所在长吏,迫畏罪负,严法峻刑,苦民求办。是以人力不堪,家户离散,呼嗟之声,感伤和气。又江边戍兵,远当以拓土广境,近当以守界备难,宜特优育,以待有事,而征发赋调,烟至云集,衣不全裋褐,食不瞻朝夕,出当锋镝之难,入抱无聊之戚。是以父子相弃,叛者成行。愿陛下宽赋除烦,振恤穷乏,省诸不急,荡禁约法,则海内乐业,大化普洽。夫民者国之本,食者民之命也,今国无一年之储。家无经月之畜,而后宫之中坐食者万有余人。内有离旷之怨,外有损耗之费。使库廪空于无用,士民饥于糟糠。

又北敌注目,伺国盛衰,陛下不恃己之威德;而怙敌之不来,忽四海之困穷,而轻虏之不为难,诚非长策庙胜之要也。昔大皇帝勤身苦体,创基南夏,割据江山,拓士万里,虽承天赞,实由人力也。余庆遗祚,至于陛下,陛下宜勉崇德器,以光前烈。爱民养士,保全先轨,何可忽显祖之功勤,轻难得之大业。忘天下之不振,替兴衰之巨变哉?

臣闻否泰无常,吉凶由人,长江限不可久恃,苟我不守,一苇可航也。昔秦建皇帝之号,据殽函之阻,德化不修,法政苛酷,毒流生民,忠臣杜口,是以一夫大呼,社稷倾覆。

近刘氏据三关之险,守重山之固,可谓金城石室,万世之业,任授失贤,一朝丧没,君臣系颈,共为羁仆。此当世之明鉴,目前之炯戒也。愿陛下远考前事,近览世变,丰基强本,割情从道,则成康之治兴,而圣祖之祚隆矣。书奏,皓深恨之。邵奉公贞正,亲近所惮。乃共谮邵与楼玄谤毁国事,俱被诘责。玄见送南州,邵原复职。后邵中恶风,口不能言,去职数月,皓疑其托疾,收付酒藏,掠考千所,邵卒无一语,竟见杀害,家属徙临海。并下诏诛玄子孙,是岁天册元年也,邵年四十九。

韦曜字弘嗣,吴郡云阳人也。少好学,能属文,从丞相掾除西安令,还为尚书郎,迁太子中庶子。时蔡颖亦在东宫,性好博奕。太子和以为无益,命曜论之。其辞曰:“盖闻君子耻当年而功不立,疾设世而名不称,故曰‘学如不及,犹恐失之’。是以古之志士,悼年齿之流迈而惧名称之不立也,故逸精厉操,晨兴夜寐,不遑宁息,经之以岁月,累之以日力,若宁越之勤,董生之笃,渐渍德义之渊,栖迟道艺之域。且以西伯之圣,姬公之才,犹有日昃待旦之劳,故能隆兴周道,垂名亿载,况在臣庶,而可以已乎?历观古今功名之士,皆有累积殊异之迹,劳身苦体,契阔勤思,平居不堕其业,穷困不易其素,是以卜式立志于耕牧,而黄霸受道于囹圄,终有荣显之福,以成不朽之名。

故山甫勤于夙夜,而吴汉不离公门,岂有游惰哉?

“今世之人多不务经术,好玩博奕,废事弃业,忘寝与食,穷日尽明,继以脂烛。

当其临局交争,雌雄未决,专精锐意,心劳体倦,人事旷而不修,宾旅阙而不接,虽有太牢之馔,《韶》、《夏》之乐,不暇存也。至或赌及衣物,徙棋易行,廉耻之意弛,而忿戾之色发,然其所志不出一枰之上,所务不过方罫之间,胜敌无封爵之赏,获地无兼土之实,技非六艺,用非经国。立身者不阶其术,征选者不由其道。求之于战陈,则非孙、吴之伦也。考之于道艺,则非孔氏之门也;以变诈为务,则非忠信之士也;以劫杀为名,则非仁者之意也;而空妨日废业,终无补益。是何异设木而击之,置石而投之哉!且君子之居室也勤身以致养,其在朝也竭命以纳忠,临事且犹旰食,而何博奕之足耽?

夫然,故孝友之行立,贞纯之名彰也。

“方今大吴受命,海内未平,圣朝乾乾,务在得人,勇略之士则受熊虎之任,儒雅之徒则处龙凤之署,百行兼苞,文武并骛,博选良才,旌简髦俊。设程式之科,垂金爵之赏,诚千载之嘉会,百世之良遇也,当世之士,宜勉思至道,爱功惜力,以佐明时,使名书史籍,勋在盟府,乃君子之上务,当今之先急也。

“夫一木之枰孰与方国之封?枯棋三百孰与万人之将?兖龙之服,金石之乐,足以兼棋局而贸博弈矣。假令世士移博奕之力而用之于诗书,是有颜、闵之志也。用之于智计,是有良、平之思也。用之于资货,是有猗顿之富也;用之于射御,是有将帅之备也。如此则功名立而鄙贱远矣。

和废后,为黄门侍郎。孙亮即位,诸葛恪辅政,表曜为太史令,撰《吴书》,华核、薛莹等皆与参同,孙休践阼,为中书郎、博士祭酒。命曜依刘向故事,校定众书。又欲延曜侍讲,而左将军张布近习宠幸,事行多玷,惮曜侍讲儒士,又性精确,惧以古今警戒休意,固争不可。休深恨布,语在《休传》。然曜竟止不入。孙皓即位,封高陵亭候,迁中书仆射,职省,为侍中,常领左国史。时所在承指数言瑞应。皓以问曜,曜答曰:“此人家筐箧中物耳。”又皓欲为父和作纪,曜执以和不登帝位,宜名为传。如是者非一,渐见责怒。曜益忧惧,自陈衰老,求去侍、史二官,乞欲成所造书,以从业别有所付,皓终不听。时有疾病,医药监护,持之愈急。皓每飨宴,无不竟日,坐席无能否率以七升为限,虽不悉入口,皆浇灌取尽。曜素饮酒不过二升,初见礼异时,常为裁减,或密赐茶荈以当酒,至于宠衰,更见逼强,辄以为罪。又于酒后使侍臣难折公卿,以嘲弄侵克发摘私短以为欢。时有衍过,或误犯皓讳,辄见收缚,至于诛戮。曜以为外相毁伤,内长尤恨,使不济济,非佳事也,故但示难问经义言论而已。皓以为不承用诏命,意不忠尽,遂积前后嫌忿,收曜付狱,是岁凤皇二年也。

曜因狱吏上辞曰:“囚荷恩见哀,无与为比,曾无芒氂有以上报,孤辱恩宠,自陷极罪。念当灰灭。长弃黄泉,愚情(忄娄)(忄娄),窃有所怀,贪令上闻。囚昔见世间有古历注,其所记载既多虚无,在书籍者亦复错谬。囚寻按传记,考合异同,采摭耳目所及。以作《洞纪》,纪自庖牺,至于秦、汉,凡为三卷,当起黄武以来,别作一卷,事尚未成。又见刘熙所作《释名》,信多佳者,然物类众多,难得详究。故时有得失,而爵位之事,又有非是。愚以官爵,今之所急,不宜乘误。囚自忘至微,又作《官职训》及《辩释名》各一卷,欲表上之。新写始毕,会以无状,幽囚特命,泯没之日,恨不上闻。谨以先死列状,乞上言秘府,于外料取,呈内以闻。迫惧浅蔽,不合天听,抱怖雀息,乞垂哀省。”

曜冀以此求免,而皓更怪其书之垢故,又以诘曜。曜对曰:“囚撰此书,实欲表上,惧有误谬,数数省读,不觉点污。被问寒战,形气呐吃,谨追辞叩头五百下,两手自搏。”而华核连上疏救曜曰:“曜运值千载,特蒙哀识,以其儒学,得与史官,貂蝉内侍,承答天问,圣朝仁笃,慎终追远,迎神之际,垂涕敕曜。曜愚惑不达。不能敷宣陛下大舜之美,而拘击史官,使圣趣不叙,至行不彰,实曜愚蔽当死之罪,然臣(忄娄)

(忄娄),见曜自少勤学,虽老不倦,探综坟典,温故知新,及意所经识古今行事,外吏之中少过曜者。昔李陵为汉将,军败不还而降匈奴,司马迁不加疾恶,为陵游说,汉武帝以迁有良史之才,欲使毕成所撰,忍不加诛,书卒成立,垂之无穷。今曜在吴,亦汉之史迁也。伏见前后符瑞彰着。神指天应,继出累见,一统之期,庶不复久。

事乎之后,当观时设制,三王不相因礼,五帝不相沿乐,质文殊涂,损益异体,宜得辈依准古义,有所改立。汉氏承秦,则有叔孙通定一代之仪,曜之才学亦汉通之次也。

又《吴书》虽已有头角,叙赞未述。昔班固作《汉书》,文辞典雅,后刘珍,刘毅等作《汉记》,远不及固,叙传尤劣。今年《吴书》当垂千载,编次诸吏,后之才士论次善恶,非得良才如曜者,实不可使阙不朽之书。如臣顽蔽,诚非其人。

曜年已七十,余数无几,乞赦其一等之罪,为终身徒,使成书业,水足传未,垂之百世。谨通进表,叩头百下。“皓不许,遂诛曜,徙百家零陵。子隆,亦有文学也。

华核字永先,吴郡武进人也。始为上虞尉、曲农都尉,以文学入为秘府郎,迁中书丞。蜀为魏所并,核诣宫门发表曰:“间闻贼众蚁聚向西境,西境艰险,谓当无虞。定闻陆抗表至,成都不守,臣主播越,社稷倾覆。昔卫为翟所灭而桓公存之,今道里长远,不可救振,失委附之土,弃贡献之国,臣以草芥,窃怀不宁。陛下圣仁,恩泽远抚,卒闻如此,必垂哀悼。臣不胜忡怅之情,谨拜表以闻。”

孙皓即位,封除陵亭候。实鼎二年,皓更营新宫,制度弘广,饰以珠玉,所费甚多。

是时盛夏兴工,农守并废,核上疏谏曰:“臣闻汉文之世,九州晏然,秦民喜去惨毒之苛政,归刘氏之宽仁,省役约法,与之更始,分王子弟以藩汉室,当此之时,皆以为泰山之安,无穷之基之也。至于贾谊,独以为可痛哭及流涕者三,可为长叹息者六,乃曰当今之势何异抱火积薪之下而寝其上,火末及然而谓之安。其后变乱,皆如其言。臣虽下愚,不识大伦,窃以囊时之事,揆今之势。

谊曰复数年间,诸王方刚,汉之傅相称疾罢归,欲以此为治,虽尧、舜不能安。今大敌据九州之地,有大半之众,习攻战之余术,乘戎刀之旧势,欲与中国争相吞之计,其犹楚汉势不两立,非徒汉之诸王淮南,济北而已。谊之所欲痛哭,比今为缓,抱火卧薪之喻,于今而急。大皇帝览前代之如彼,察今势之如此,故广开农桑之业,积不訾之储,恤民重役,务养战士,是以大小感恩,各思竭命。斯运未至,早弃万国,自是之后,强臣专政,上诡天时,下违从议,忘安存之本,邀一时之利,数兴军旅,倾竭府藏,兵劳民困,无时获安。今之存者乃创夷之遗众,哀苦之余及耳。遂使军盗空匮,仓廪不实,布帛之赐,寒暑不周,重以失业,家户不赡。而北积谷养民,专心向东,无复他警。蜀为西藩,土地险固,加承先主统御之术,谓其守御足以长久,不图一朝奄至倾覆!唇亡齿寒,古人所惧。交州诸郡,国之南土,交址、九真二郡已没,日南孤危,存亡难保,合浦以北,民皆摇动。因连避役,多有离叛,而备戍减少,威镇转轻,常恐呼吸复有变故。昔海虏窥窬东县,多得离民,地习海行,狃于往年,钞盗无日,今胸背有嫌,首尾多难,乃国朝之厄会也。诚宜住建立之役,先备豫之计,勉垦殖之业,为饥乏之救。惟恐农时将过,东作向晚,有事之日,整严未办。若舍此急,尽力功作,卒有风尘不虞之变。当委版筑之役,应烽燧之急,驱怨苦之众,赴自刃之难,此乃大敌所因为资也。如但固守,旷日持久,则军粮必乏,不待接刃,而战士已困矣。

昔太戊之时,桑谷生庭,惧而修德,怪消殷兴。荧惑守心,宋以为灾,景公下从瞽史之言,而荧惑退舍,景公延年。夫修德于身而感异类,言发于口通神明。臣以愚蔽,误忝近署,不能冀宣仁泽以感灵祗,仰惭俯愧,无所投处。退伏思惟,荣惑桑谷之异,天示二主,至如他余锱介之妖;近是门庭小神所为,验之天地,无有他变,而征样符瑞前后屡臻,明珠既觌,白雀继见,万亿之祚,实灵所挺。以九域为宅,天下为家,不与编户之民转徙同也。又今之宫室,先帝所营。卜土立基,非为不祥。又杨市土地与宫连接,若大功毕竟,舆驾迁住,门行之神,皆当转移,犹恐长,久未必胜旧。屡迁不少,留则有嫌,此乃愚臣所以夙夜为忧灼也。臣省《月令》,季夏之月,不可以兴土功,不可以会诸侯,不可以起兵动众,举大事必有大殃。今虽诸侯不会,诸侯之军与会无异。

六月戊己,土行正王,既不可犯,加又农月,时不可失。昔鲁隐公夏城中丘,《春秋》书之,垂为后戒。今筑宫为长世之洪基,而犯天地之大禁,袭《春秋》之所书,废敬授之上务,臣以愚管,窃所未安。

又恐所召离民,或有不至,讨之则废役兴事,不讨则日月滋慢。若悉并到,大众聚会,希无疾病。且人心安则念善,苦则怨叛。江南精兵,北土所难,欲以十卒当东一人。

天下未定,深可忧惜之。如此宫成,死叛五千,则北军之众更增五万,着到万人,则倍益十万,病者有死亡之损,叛者传不善之语,此乃大敌所以欢喜也。今当角力中原,以定强弱,正于际会。彼益我损,加以劳困,此乃雄夫智士所以深忧。

臣闻先王治国无三年之储,曰国非其国,安宁之世戒备如此。况敌强大而忽农忘畜。

今虽颇种殖,间者大水沉没,其余存者当须耘获。而长吏怖期,上方诸郡,身涉山林,尽力伐材,废农弃务;士民妻孥羸小,垦殖又薄;若有水旱则永无所获。州郡见米,当待有事,冗食之众,仰官供济。若上下空乏,运漕不供,而北敌犯疆,使周、召更生,良、平复出,不能为陛下计明矣。臣闻君明者臣忠,主圣者臣直,是以(忄娄)(忄娄),昧犯天威,乞垂哀省。

书奏,皓不纳。后迁东观令,领右国吏,核上疏辞让。皓答曰:“得表,以东观儒林之府,当讲校文艺,处定疑难,汉时皆名学硕儒乃任其职,乞更选英贤。闻之,以卿研精坟典,博览多闻,可谓悦礼乐敦诗书者也。当飞翰骋藻,光赞时事,以越扬、班、张、蔡之畴,怪乃谦光,厚自菲薄,宜勉备所职,以迈先贤,勿复纷纷。”

时仓廪无储,世俗滋侈,核上疏曰:“今冠虏充斥,征伐未已,居无积年之储,出无敌之畜,此乃有国者所宣深忧也。夫财谷所生,皆出于民,趋时务农,国之上急。而都下诸官,所掌别异,各自下调,不计民力,辄与近期。长吏畏罪,昼夜催民,委舍佃事,遑赴会日,定送到都,或蕴积不用,而徒使百姓消力失时。到秋收月,督其限入,夺其播殖之时,而责定送其今年之税,如有逋悬,则籍没财物,故家户贫困,衣食不足。

宜暂息众役,专心农桑。古人称一夫不耕,或受其饥。一女不织,或受其寒。是以先王治国,惟农是务。军兴以来,已向百栽,农人废南亩之务,女工停机杼之业。推此揆之,则蔬食而长饥,薄衣而履冰者,固不少矣。臣闻主之所求于民者二,民之所望于主者三。

二谓求其为己劳也,求其为己死也。三谓讥者能食之,劳者能息之,有功者能赏之。民以致其二事而主失其三望者,则怨心生而功不建,今帑藏不实,民劳役猥,主之二求已备,民之三望未报。且饥者不待美馔而后饱,寒者不俟狐貉而后温,为味者口之奇,文绣者身之饰也。今事多而役繁,民贫而俗奢,百工作无用之器,妇人为绮靡之饰,不勤麻枲,并乡黼黻,转相仿效,耻独无有。兵民之家,犹复遂俗,内无儋石之储,而出有绫绮之服,至于富贾商贩之家,重以金银,奢恣尤甚。天下未平,百姓不赡,宜一生民之原,丰谷帛之业。而弃功于浮华之巧,妨日于侈靡之事,上无尊卑等级之差,下有耗财物力之损。今吏士之家,少无子女,多者三四,少者一二,通令户有一女,十万家则十万人,人织绩一岁一束,则十万束矣。使四疆之内同心戮力,数年之间,布帛必积。

恣民五色,惟所服用,但禁绮绣无益之饰。且美貌者不待华采以祟好,艳姿者不待文绮以致爱,五采之饰,足以丽矣。若极粉黛,穷盛服,未必无丑妇。废华采,去文绣,未必无美人也。若实如论,有之无益废之无损者,何爱而不斩禁以充府藏之急乎?此救乏之上务,富国之本业也,使管、晏复生,无以易此。汉之文、景,承平继统,天下已定,四方无虞,犹以雕文之妨农事,锦绣之害女红,开富国之利,杜饥寒之本。况今六台分乖;豺狼充路;兵不离疆;甲不解带。而可以不广生财之原,充府藏之积哉?“

皓以核年老,敕令草表,核不敢。又敕作草文,停立待之。核为文曰:“咨核小臣,草芥凡庸。遭眷值圣,受恩特隆。越从朽壤,蝉蜕朝中。熙光紫闼;青璅是凭。毖挹清露,沐浴凯风。效无丝氂,负阙山祟。滋润含垢,恩贷累重。秽质被荣,局命得融。欲报罔极,委之皇穹。圣恩雨注,哀弃其尤。猥命草对,润被下愚。不敢达敕,惧速罪诛。

冒承诏命,魂逝形留。“核前后陈便宜,及贡荐良能,解释罪过,书百余上,皆有补益,文多不悉载。天册元年以微谴免,数岁卒。曜、核所论事章疏,咸传于世也。

评曰:薛莹称王蕃器量绰异,弘博多通。楼玄清白节操,才理条畅;贺邵厉志高洁,机理清要。韦曜笃学好古,博见群籍,有记述之才。胡冲以为玄、邵、蕃一时清妙,略无优劣,必不得已,玄宜在先,邵当次之。华核文赋之才,有过于曜,而典诰不及也。

矛观核数献良规,期于自尽,庶几忠臣矣。然此数子,处无妄之世而有名位,强死其理,得免为幸耳。

\backmatter
\chapter{生字表}

颙 y\'ong  颍 yǐng  宄 gu\v{i}

\end{document}