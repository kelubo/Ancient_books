% 三国志
% 三国志.tex

\documentclass[12pt,UTF8]{ctexbook}

% 设置纸张信息。
\usepackage[a4paper,twoside]{geometry}
\geometry{
	left=25mm,
	right=25mm,
	bottom=25.4mm,
	bindingoffset=10mm
}

% 设置字体,并解决显示难检字问题。
\xeCJKsetup{AutoFallBack=true}
\setCJKmainfont{SimSun}[BoldFont=SimHei, ItalicFont=KaiTi, FallBack=SimSun-ExtB]

% 目录 chapter 级别加点(.)。
\usepackage{titletoc}
\titlecontents{chapter}[0pt]{\vspace{3mm}\bf\addvspace{2pt}\filright}{\contentspush{\thecontentslabel\hspace{0.8em}}}{}{\titlerule*[8pt]{.}\contentspage}

% 设置 part 和 chapter 标题格式。
\ctexset{
	part/name= {卷,},
	part/number={\chinese{part}},
	chapter/name={},
	chapter/number={}
}

% 设置古文原文格式。
\newenvironment{yuanwen}{\bfseries\zihao{4}}

% 设置署名格式。
\newenvironment{shuming}{\hfill\bfseries\zihao{4}}

% 注脚每页重新编号,避免编号过大。
\usepackage[perpage]{footmisc}

\title{\heiti\zihao{0} 三国志}
\author{}
\date{}

\begin{document}

\maketitle
\tableofcontents

\frontmatter
\chapter{前言、序言}

《三国志》是记述东汉未年到西晋统一间这段历史的一部史学名著,它记述的主要是魏、蜀、吴三国的历史,故称《三国志》。《三国志》历来备受推崇,它与《史记》、《汉书》、《后汉书》合称前四史,而前四史被公认为是二十四史中成就最高的四部史书。《后汉书》的作者范晔是南朝刘宋人,成书时间晚于《三国志》一百多年,《三国志》实际是二十四史中紧承《史记》、《汉书》的第三部史著。

作者陈寿,字承祚,蜀汉巴西郡安汉县(今四川南充)人,生于蜀汉建兴十一年(233),卒于西晋元康七年(297),年六十五岁。陈寿从小好学,“聪慧敏识,属文富艳”,曾从蜀汉的著名史学家谯周学习,研读《尚书》、《春秋》等经史著作,尤精于《史记》、《汉书》,这为他后来撰写《三国志》打下了很好的基础。蜀汉被灭后,陈寿以亡国之臣的身份入魏。他历仕蜀汉、曹魏、西晋三朝,曾任东观秘书郎、散骑黄门侍郎、著作郎、平阳侯相、治书侍御史等职,一生仕途坎坷,官位不显,所以《华阳国志·陈寿传》说:“位望不充其才,当时冤之。”

陈寿的著述很多,撰有《益部耆旧传》十篇、《古国志》五十篇,又编纂《诸葛亮集》、《魏名臣奏事》、《汉名臣奏事》等书。吴平之后,陈寿始“鸠合三国史,著魏、吴、蜀三书六十五卷,号《三国志》”。《三国志》写成后,曾在士大夫间流传,立刻就得到了人们的认可,“时人称其善叙事,有良史之才”。当时人夏侯湛正在修《魏书》,见到陈寿的《三国志》,便将自己的书毁掉,不再继续撰写。朝廷重臣张华非常欣赏陈寿,认为他的史才要超过《史记》的作者司马迁和《汉书》的作者班固,准备将修《晋书》的事情托付给他。陈寿去世后,尚书郎范觏等人上书说:“故治书侍御史陈寿作《三国志》,词多劝戒,明乎得失,有益风化。虽文艳不如相如,而质直过之,愿垂采纳。”于是晋惠帝下诏,令河南尹、洛阳令到陈寿家抄录《三国志》。《三国志》得到官方的认可,正式流传。

《三国志》之所以备受推崇,是因为它有许多突出的优点。

《三国志》六十五卷,其中《魏书》三十卷、《蜀书》十五卷、《吴书》二十卷,是一部纪传体史书。在陈寿之前,司马迁的《史记》贯穿古今,是第一部纪传体史书;班固的《汉书》,则是一部纪传体的断代史。为同时并立的三国修史,是陈寿面对的新问题,于是他另辟蹊径,分作《魏书》、《蜀书》、《吴书》为三国各自修史,然后合为一书,总其名为《三国志》,非常巧妙地解决了这个难题。这充分显示了陈寿的史才,同时也使《三国志》成为二十四史中别具一格的史书。

陈寿修《三国志》,眼光未局限于三国,而是上接汉末,为群雄董卓、袁绍、袁术、吕布等人立传。这是很有见地的做法,因为讲三国的历史离不开汉末的历史,离不开当时的这些风云人物,如果缺了汉末群雄在历史大舞台上的表演,轰轰烈烈的三国历史将会大为减色。陈寿虽分为魏、蜀、吴三国修史,但以《魏书》为主,即所谓以魏为正统。魏的君主依帝王例立本纪,蜀、吴的君主则降低一格,分别立传。另外,在谴词用字、人物的称谓上也体现出这种区别,如称魏君主为帝,蜀君主称先主、后主,吴则称吴主或径称其名等。以魏为正统,是陈寿颇受非议的地方,但这却是陈寿的无奈之举。陈寿修《三国志》是在吴平之后,大体上是当代人修当代史。西晋统治者司马氏是取代魏登上帝位的,只有承认魏的正统地位,才可以证明取代者司马氏的正统。而实际上从修史体例来说,他是将蜀、吴二国当做与魏并列的王朝来处理的,蜀、吴二国君主的传记,都是以本纪的规格来写,即以编年为序来记述传主的言行事迹,并以此为纲来记述一朝的政治、经济、文化等方方面面的重大事件。其实这正说明了陈寿的苦心,由此说来,人们对他的这种非议是不足一驳的。

在《三国志》的撰写上,陈寿取材谨严,剪裁得当,坚持以求实的态度修史。陈寿修《三国志》,可供他选用参考的资料,有魏王沈的《魏书》、鱼豢的《魏略》,吴韦昭的《吴书》等,还有他自己掌握的蜀国资料。他在史料的选用取舍上非常用心,皆再三审慎斟酌后才予采用。清代学者赵翼在批评陈寿的同时,也不得不承认他“剪裁斟酌处,亦自有下笔不苟者,参订他书而后知其矜慎也”,并列举大量例证说明他在资料使用上的剪裁得当。

《三国志》叙事生动简洁,语言洗练干净,评论中肯得当,毫无繁冗之词。也有学者认为《三国志》质朴有余,文采不足。当然,如果与《史记》那样的极品相比,《三国志》整体上的文采确是略逊一筹,然而在具体章节上,却不乏精彩的描写。比如赤壁之战,陈寿将相关史料分别放在《魏书·武帝纪》、《蜀书·先主传》、《蜀书·诸葛亮传》、《吴书·吴主传》、《吴书·周瑜传》、《吴书·鲁肃传》等几个纪传中,通过各有侧重的描写,把赤壁之战渲染得有声有色,尤其是其中吴蜀双方、各自的君臣之间的对话,非常传神。所以宋代大史学家司马光修《资治通鉴》时,对于这段历史,基本采用了《三国志》的记述,有些地方如《蜀书·诸葛亮传》所载诸葛亮与孙权的对话,几乎全文照录,这也从侧面证明了陈寿文字功夫的精到。又他在纪传后面的评论,不仅颇具文采,而且大都贴切公允,寓意深刻,堪称点睛之笔。比如看了陈寿对曹操精彩的评论,就会发现后人对曹操所谓的翻案,并没有太大的意义,陈寿对曹操早已作了非常恰当的评价。

《三国志》也有一些不足之处:

1.有纪传而无志。志是专门记载政治、经济、天文、地理、礼乐等典章制度的,它是对当时社会生活的全面反映,是后人了解认识这一时期历史所凭依的资料。在《三国志》前边的《史记》、《汉书》,都有这方面的内容。《三国志》无志,可能有社会动乱不已、资料不足等多种原因,但一部完整的纪传体史书无志,不能不说是一个大的缺憾。

2.叙述过于简略。《三国志》叙事简洁是它的优点,但对于史书来说,还是要给后人尽量多地留下有价值的资料。这一点陈寿做得有些不够,一些重要历史人物的传记资料很少,与传主的身份地位很不相称。如关羽、张飞、赵云,乃是蜀汉的开国元勋,而《蜀书·关羽传》仅一千二百余字,《蜀书·张飞传》仅八百余字,《蜀书·赵云传》仅四百余字。又如徐幹、陈琳、应场、阮瑀等建安时期的著名文人,皆未立专传,只附记在《魏书·王粲传》中,而且除记陈琳事有三四百言外,其余皆寥寥数语。这其中也有资料不足的原因,《蜀书·后主传》的评论说:“国不置史,注记无官,是以行事多遗,灾异靡书。”不设史官,当然记录下来的资料就不会太多。

3.对曹氏、司马氏等统治者隐恶溢美,曲笔回护。这是陈寿最受垢病的地方。中国的修史传统,讲究直书实录,不隐恶,不扬善,给后人留下信史。唐代史学家刘知已批评陈寿对司马氏篡权弑主事未置一词,不肯如实记录。清代学者赵翼《甘二史札记》有“《三国志》多回护”一条,专论曲笔回护事。考之史实,这确实不是虚词。但所谓曲笔回护,是专制时代无法完全避免的。陈寿作为晋人,让他直指甚至揭露当朝者的丑恶行径是不现实的,有时甚至还要为他们粉饰,对他来说,这也是无奈之举。而实际上陈寿在对统治者有所回护的同时,也对他们作了一些隐讳的批评,如在《魏书·文帝纪》的评论中,他就对魏文帝的心胸狭隘进行了批评。而且同样是曲笔,要看是有意还是无奈,还要看程度的多少。考察《三国志》,毕竟直笔实录的多,曲笔回护的少,整体上是实录。

4.此外,还有人批评陈寿借修史谋取私利和发泄私愤,古今学者对此作了驳正,都是不足凭信的虚言。

《三国志》一个与众不同的特点是它的注。陈寿去世一百多年后,随着有关三国的史料的逐渐出现,南朝宋文帝刘义隆认为《三国志》过于简略,命时任中书侍郎的裴松之为之作注。裴松之字世期,河东郡闻喜县(今山西闻喜)人,在他的祖父时,裴家迁居江南。他从小好学,“八岁学通《论语》、《毛诗》。博览坟籍,立身简素”,著述甚丰。他的儿子裴骃也是著名的史学家,曾撰后来被称为《史记》三家注的《史记集解》一书。裴松之领命后,“鸠集传记,增广异闻”,于刘宋元嘉六年(429)将《三国志注》完成。书成奏上,宋文帝非常满意,赞扬说:“此为不朽矣。”

裴松之的注不同于传统的注,重点不在于对语言、名物、制度的考证解释,而在于对史事的补缺、备异、惩妄、论辩。其中补缺、备异,是资料的补充,是对于所引用资料的归纳与整理。惩妄、论辩,则是对于所引用资料的考证与批评。这就是说,他不仅仅是简单的搜集罗列资料,而且要经过校勘考证,提出自己的观点。有学者考证,裴松之引用的书达二百一十种,其中大部分已经亡佚,而他所引史事,大多首尾俱全,未加删节,这就为后人留下了大量珍贵的资料。所以《四库全书总目》说裴注“转相引据者,反多于陈寿本书焉”。比如关羽、赵云二人传记的简略问题,经裴注得到了很好的解决。《蜀书·关羽传》补充约一千字,其中有关羽喜读《左传》、许田射猎等事。《蜀书·赵云传》补充一千四百余字,使人物形象更加丰满,其中有赵云截江夺阿斗等事。这些大量的极其有用的资料,为后人阅读理解这部史学名著提供了很大的帮助,同时也为后人创作《三国演义》提供了素材。

对于裴松之的注,人们也有非议之词,比较突出的是认为他引用资料过于芜杂烦琐。《四库全书总目》就说裴松之“往往嗜奇爱博,颇伤芜杂”,这种说法有一定的道理。至于有的学者批评他“注之所载,皆寿之弃余”,就不够客观了,他所引用的资料,大部分出于陈寿同时人或后人的著作,陈寿根本没有见到它们的可能。而有些可能被批评者引以为据的东西,裴松之的初衷并不全是为补充资料,一定程度上是为了表达自己的史学观点。比如著名的“空城计”的故事,它最早见于西晋郭冲的《诸葛亮五事》,根据史实,是毫无根据的妄说。所以陈寿未予采纳。裴松之虽然引用,但应该是出于惩妄的目的。所以他在引用的同时,对这种说法的荒谬以及停于史实,都作了批评,认为“此书举引皆虚”。而这些虚妄的材料本身,以后人的眼光来看也不是毫无用处,它被后人加工借用,成为《三国演义》中非常精彩的一个章节。
总之,裴松之的注价值极高,它已经与《三国志》成为一体,读《三国志》必须要读裴注。清代学者钱大昭甚至认为,裴松之依据他所掌握的新材料,完全可以自成一史,是因为他自己谦虚,才附于《三国志》下作为注而存在。

由于具有巨大的史学价值和文学价值,《三国志》不仅在史学史上占有崇高的地位,对后世的社会文化也产生了极大的影响。《三国志》很早就流传到海外,有各种文字的版本,海外有众多的学者在研究《三国志》。大约在元末明初,罗贯中依据《三国志》创作《三国演义》,把作为正史的《三国志》通俗化为小说,它最初的名字就叫作《三国志通俗演义》。书中大多数的人物及故事情节都可以在《三国志》及裴注中找到根据或线索,材料的主要来源就是《三国志》及裴注。由于《三国演义》的推波助澜,三国故事在中国可谓家喻户晓,而来源于三国故事的人物、典故、语言等等,更是早已深入到人们的生活当中。各种文艺形式的创作也大量的取材于三国故事,比如作为中国国粹的京剧,甚至专有成套且具相当规模的三国戏。三国故事在日本及东南亚等地广为流传,在西方也有很大的影响。有关三国故事的方方面面早已形成为丰富多彩的三国文化,而三国文化的普及反过来也促进了人们对于《三国志》的了解。就社会影响及普及性而言,《三国志》在中国的史书中是绝无仅有的。总体而言,绚丽多姿的三国文化滥觞于陈寿的《三国志》,而由裴松之的广采博注助扬其波,至《三国演义》的流传,则蔚为大观而成江河了。

《三国志》问世以来,除裴松之外,为它作注的代不乏人,尤其是清代的一些大学者,在这方面更是下了很大的功夫。近代以来,卢弼的《三国志集解》是一部集历代研究成果之大成的著作,是非常详尽的注释本。上世纪八十年代由著名学者缪钺先生主编、中华书局出版的《三国志选注》,是一部注释精审的选本。近年以来,有不少注译本问世,也都各具特色。

\mainmatter

\part{魏书一}

\chapter{武帝纪第一}

《武帝纪》是《三国志》的第一篇,记述的是魏武帝曹操的事迹。陈寿全面而详尽地记述了曹操不平凡的一生。曹操是非常有名的历史人物,他以他的雄才大略,叱咤风云数十年,统一了我国北方,同时也奠定了曹魏王朝的基业。曹操将汉献帝控制在手中后,令由己出,已经“三分天下有其二”,这也是他被后人称为奸雄的原因,但他自己并没有废掉汉帝自立。他的儿子曹不代汉后,追尊他为武皇帝。曹操是大乱世造就的大英雄,他能在群雄中脱颖而出,成就霸业,是因为他具备成就大事的条件。他目光远大,有治平天下的雄心壮志,有百折不挠的勇气和顽强的毅力,有过人的胆识和谋略,有清醒的头脑和恢弘的气度。他精通兵法,用兵如神,善于发现和使用人才。他壮心不已,一生几乎都在征战之中,直至在征途中去世。曹操又是个文采风流、多才多艺的人。他喜欢读书,在军中三十余年,手不释卷,“昼则讲武策,夜则思经传”。他“登高必赋,及造新诗,被之管弦,皆成乐章”,开创了慷慨悲凉的一代诗风,他的文章清峻整洁,他是建安文学的代表作家。他善书法,可与当时的书法名家张芝、张昶相媲美;善围棋,能和当时的高手山子道、王九真一较高低。陈寿说他是“非常之人,超世之杰”,确实不是过誉之词。

\begin{yuanwen}
太祖武皇帝\footnote{曹操的儿子曹丕称帝后,追尊曹操为武皇帝,定庙号为太祖。},沛国\footnote{王国名,今安徽濉溪西北。}谯\footnote{县名,今安徽亳县。}人也,姓曹,讳\footnote{古代对帝王及尊长不能直呼其名,故称讳,以示尊重。}操,字孟德,汉相国\footnote{丞相的尊称。}参\footnote{曹参,西汉大臣,曾任相国。}之后。\footnote{曹瞒传曰:太祖一名吉利,小字阿瞒。王沈魏书曰:其先出于黄帝。当高阳世,陆终之子曰安,是为曹姓。周武王克殷,存先世之后,封曹侠于邾。春秋之世,与于盟会,逮至战国,为楚所灭。子孙分流,或家于沛。汉高祖之起,曹参以功封平阳侯,世袭爵土,绝而复绍,至今适嗣国于容城。} 桓帝\footnote{汉恒帝刘志,公元146--147年在位。}世,曹腾为中常侍\footnote{宦官名,掌管传达诏令及宫中文书,权力很大。}大长秋\footnote{官名,东汉时多由宦官担任,为皇后近侍,掌管传达皇后旨意及宫中事务。},封费亭侯\footnote{封爵名。侯爵的一种。东汉封侯,根据功劳大小分为县、乡、亭侯,各享有数目不同的食邑。}。\footnote{司马彪续汉书曰:腾父节,字符伟,素以仁厚称。邻人有亡豕者,与节豕相类,诣门认之,节不与争;后所亡豕自还其家,豕主人大惭,送所认豕,并辞谢节,节笑而受之。由是乡党贵叹焉。长子伯兴,次子仲兴,次子叔兴。腾字季兴,少除黄门从官。永宁元年,邓太后诏黄门令选中黄门从官年少温谨者配皇太子书,腾应其选。太子特亲爱腾,饮食赏赐与众有异。顺帝即位,为小黄门,迁至中常侍大长秋。在省闼三十余年,历事四帝,未尝有过。好进达贤能,终无所毁伤。其所称荐,若陈留虞放、边韶、南阳延固、张温、弘农张奂、颍川堂溪典等,皆致位公卿,而不伐其善。蜀郡太守因计吏修敬于腾,益州刺史种暠于函谷关搜得其笺,上太守,并奏腾内臣外交,所不当为,请免官治罪。帝曰:“笺自外来,腾书不出,非其罪也。”乃寝暠奏。腾不以介意,常称叹暠,以为暠得事上之节。暠后为司徒,语人曰:“今日为公,乃曹常侍恩也。”腾之行事,皆此类也。桓帝即位,以腾先帝旧臣,忠孝彰着,封费亭侯,加位特进。太和三年,追尊腾曰高皇帝。}养子嵩嗣\footnote{继承。},官至太尉\footnote{官名。三公之一,掌管全国军事,为全国最高军事长官。},莫能审\footnote{了解,明白。}其生出本末。\footnote{续汉书曰:嵩字巨高。质性敦慎,所在忠孝。为司隶校尉,灵帝擢拜大司农、大鸿胪,代崔烈为太尉。黄初元年,追尊嵩曰太皇帝。吴人作曹瞒传及郭颁世语并云:嵩,夏侯氏之子,夏侯惇之叔父。太祖于惇为从父兄弟。}嵩生太祖。
\end{yuanwen}

太祖武皇帝,沛国谯县人,姓曹名操,字孟德,是汉相国曹参的后代。汉桓帝世,曹腾任中常侍大长秋,封费亭侯,曹腾的养子曹嵩继承了曹腾的爵位,官至太尉,没人能搞清楚他的来历。曹嵩生下太祖。

\begin{yuanwen}
太祖少机警,有权数,而任侠放荡,不治行业\footnote{品行学业。},故世人未之奇也;\footnote{曹瞒传云:太祖少好飞鹰走狗,游荡无度,其叔父数言之于嵩。太祖患之,后逢叔父于路,乃阳败面喎口;叔父怪而问其故,太祖曰:“卒中恶风。”叔父以告嵩。嵩惊愕,呼太祖,太祖口貌如故。嵩问曰:“叔父言汝中风,已差乎?”太祖曰:“初不中风,但失爱于叔父,故见罔耳。”嵩乃疑焉。自后叔父有所告,嵩终不复信,太祖于是益得肆意矣。}惟梁国\footnote{王国名,今河南商丘南。}桥玄\footnote{东汉大臣,官至太尉,以知人名世。}、南阳\footnote{郡名。今河南南阳。}何颙\footnote{y\'ong东汉大臣,曾与王允等人谋诛董卓,后忧愤而死。}异焉\footnote{text}。玄谓太祖曰:“天下将乱,非命世之才\footnote{安邦济世的人才。}不能济也,能安之者,其在君乎!”\footnote{魏书曰:太尉桥玄,世名知人,鷪太祖而异之,曰:“吾见天下名士多矣,未有若君者也!君善自持。吾老矣!愿以妻子为托。”由是声名益重。续汉书曰:玄字公祖,严明有才略,长于人物。张璠汉纪曰:玄历位中外,以刚断称,谦俭下士,不以王爵私亲。光和中为太尉,以久病策罢,拜太中大夫,卒,家贫乏产业,柩无所殡。当世以此称为名臣。世语曰:玄谓太祖曰:“君未有名,可交许子将。”太祖乃造子将,子将纳焉,由是知名。孙盛异同杂语云:太祖尝私入中常侍张让室,让觉之;乃舞手戟于庭,踰垣而出。才武绝人,莫之能害。博览髃书,特好兵法,抄集诸家兵法,名曰接要,又注孙武十三篇,皆传于世。尝问许子将:“我何如人?”子将不答。固问之,子将曰:“子治世之能臣,乱世之奸雄。”太祖大笑。}
\end{yuanwen}

太祖从小机警,富有权数计谋,行侠仗义,放荡不羁,不大注意培养自己的品行学业,所以当时人没很看重他;只有梁国人桥玄、南阳人何颙对他非常赏识。桥玄对他说:“天下将要大乱,非安邦定国之才不能挽救,能安定天下的,大概就是您了!”

\begin{yuanwen}
年二十,举孝廉\footnote{汉代察举科目之一,被举之人须孝顺父母,行为廉洁。察举为汉代选举制度的一种。}为郎\footnote{郎官,汉代中郎、侍郎、郎中等官的通称。掌管皇帝侍从宿卫,并可参议朝政。},除洛阳\footnote{东汉都城,在今河南洛阳东。}北部尉\footnote{官名。汉代县设县尉,为县令长之副手,掌管纠察盗贼,维护地方治安。县尉大县二人,小县一人。洛阳为东汉都城,设尉不止一人,故有北部尉。},迁\footnote{升职。}顿丘\footnote{县名。今河南清丰西南。}令\footnote{县令,官名。汉代县之主官,万户以上县称县令,万户以下县称县长。},\footnote{曹瞒传曰:太祖初入尉廨,缮治四门。造五色棒,县门左右各十余枚,有犯禁,不避豪强,皆棒杀之。后数月,灵帝爱幸小黄门蹇硕叔父夜行,即杀之。京师敛夡,莫敢犯者。近习宠臣咸疾之,然不能伤,于是共称荐之,故迁为顿丘令。}征\footnote{征聘。汉代选举制度之一,由朝廷直接征聘。}拜\footnote{任命。}议郎\footnote{官名。郎官的一种,属光禄勋,掌管参议朝政。}。\footnote{魏书曰:太祖从妹夫□强侯宋奇被诛,从坐免官。后以能明古学,复征拜议郎。先是大将军窦武、太傅陈蕃谋诛阉官,反为所害。太祖上书陈武等正直而见陷害,奸邪盈朝,善人壅塞,其言甚切;灵帝不能用。是后诏书敕三府:举奏州县政理无效,民为作谣言者免罢之。三公倾邪,皆希世见诏用,货赂并行,强者为怨,不见举奏,弱者守道,多被陷毁。太祖疾之。是岁以灾异博问得失,因此复上书切谏,说三公所举奏专回避贵戚之意。奏上,天子感悟,以示三府责让之,诸以谣言征者皆拜议郎。是后政教日乱,豪猾益炽,多所摧毁;太祖知不可匡正,遂不复献言。}
\end{yuanwen}

二十岁时,太祖被举荐为孝廉,入朝任郎官,被任命为洛阳北部尉,升任顿丘县令,朝廷又征召他入朝,任命他为议郎。

\begin{yuanwen}
光和\footnote{汉灵帝 178--184}末,黄巾\footnote{黄巾军,东汉末年农民起义军。}起。拜骑都尉\footnote{官名。掌管皇帝侍从宿卫,属光禄勋,位次低于将军。},讨颍川\footnote{郡名。今河南禹县。}贼\footnote{指黄巾军,是统治者对农民起义军的诬蔑称呼。}。迁为济南\footnote{王国名。今山东济南东。}相\footnote{官名。汉代王国由朝廷委派国相一人,掌管王国政务,地位同于郡太守。},国有十余县,长吏\footnote{指县令长。}多阿附贵戚,赃污狼藉,于是奏免其八;禁断淫祀\footnote{指未经官方准许的祭祀。},奸宄\footnote{违法乱纪之徒。gu\v{i}}逃窜,郡界\footnote{指济南国境内。国与郡为同级行政单位,封王于其地则为国,未封则为郡,故以郡称国。}肃然。\footnote{魏书曰:长吏受取贪饕,依倚贵势,历前相不见举;闻太祖至,咸皆举免,小大震怖,奸宄遁逃,窜入他郡。政教大行,一郡清平。初,城阳景王刘章以有功于汉,故其国为立祠,青州诸郡转相仿效,济南尤盛,至六百余祠。贾人或假二千石舆服导从作倡乐,奢侈日甚,民坐贫穷,历世长吏无敢禁绝者。太祖到,皆毁坏祠屋,止绝官吏民不得祠祀。及至秉政,遂除奸邪鬼神之事,世之淫祀由此遂绝。}久之,征还为东郡\footnote{郡名。今河南濮阳西南。}太守\footnote{官名。汉代郡之主官,又称郡守。};不就,称疾归乡里。\footnote{魏书曰:于是权臣专朝,贵戚横恣。太祖不能违道取容。数数干忤,恐为家祸,遂乞留宿韂。拜议郎,常托疾病,辄告归乡里;筑室城外,春夏习读书传,秋冬弋猎,以自娱乐。}
\end{yuanwen}

光和末年,黄巾军起事。太祖被任命为骑都尉,率军攻打颍川黄巾军。升任济南相,济南国下辖十余县,各县的县令长大多巴结讨好权贵,贪赃枉法,声名狼藉,太祖于是上奏朝廷,罢免了其中八个县的县令长;又在境内禁绝淫祀,违法乱纪之徒纷纷逃窜,境内安定,秩序井然。过了很久,太祖被朝廷征还,任为东郡太守;他没有去赴任,称病返回家乡。

\begin{yuanwen}
顷之,冀州刺史王芬、南阳许攸、沛国周旌等连结豪杰,谋废灵帝,立合肥侯,以告太祖,太祖拒之。芬等遂败。\footnote{司马彪九州春秋曰:于是陈蕃子逸与术士平原襄楷会于芬坐,楷曰:“天文不利宦者,黄门、常侍*(贵)**[真]*族灭矣。”逸喜。芬曰:“若然者,芬愿驱除。”于是与攸等结谋。}

灵帝欲北巡河间旧宅,芬等谋因此作难,上书言黑山贼攻劫郡县,求得起兵。会北方有赤气,东西竟天,太史上言“当有阴谋,不宜北行”,帝乃止。敕芬罢兵,俄而征之。芬惧,自杀。

魏书载太祖拒芬辞曰:“夫废立之事,天下之至不祥也。古人有权成败、计轻重而行之者,伊尹、霍光是也。伊尹怀至忠之诚,据宰臣之势,处官司之上,故进退废置,计从事立。及至霍光受托国之任,藉宗臣之位,内因太后秉
政之重,外有髃卿同欲之势,昌邑即位日浅,未有贵宠,朝乏谠臣,议出密近,故计行如转圜,事成如摧朽。今诸君徒见曩者之易,未鷪当今之难。诸君自度,结众连党,何若七国?

合肥之贵,孰若吴、楚?而造作非常,欲望必克,不亦危乎!”
\end{yuanwen}

\begin{yuanwen}
金城\footnote{郡名。今甘肃永靖西北。}边章、韩遂杀刺史\footnote{官名。西汉分全国为十三部(州),设置刺史,以诏书六条察问郡县,本属监察官,东汉后权力加大,成为郡之上的一级行政长官。此处指凉州刺史耿鄙,金城为凉州辖郡,凉州治今甘肃张家川回族自治区。}郡守\footnote{指金城太守陈懿。}以叛,众十余万,天下骚动。征太祖为典军校尉\footnote{官名。东汉末,灵帝时为加强军事力量,设置西园八校尉,典军校尉即其中之一。}。会\footnote{恰巧,赶上。}灵帝\footnote{汉灵帝刘宏,公元168--189年在位。}崩\footnote{皇帝去世。},太子\footnote{汉少帝刘辩,公元189年在位,后被董卓废黜杀掉。}即位,太后\footnote{何太后,少帝的生母,后被董卓杀掉。}临朝。大将军\footnote{官名。东汉将军的最高称号,位在三公之上,掌管统兵征伐并执掌朝政,权力极大,不常置。}何进\footnote{何太后的异母兄长,官至大将军,灵帝死后专断朝政,与袁绍等人谋诛宦官,事泄被杀。}与袁绍\footnote{东汉末群雄之一。}谋诛宦官,太后不听。进乃召董卓\footnote{东汉末年祸国权臣。},欲以胁太后,\footnote{魏书曰:太祖闻而笑之曰:“阉竖之官,古今宜有,但世主不当假之权宠,使至于此。既治其罪,当诛元恶,一狱吏足矣,何必纷纷召外将乎?欲尽诛之,事必宣露,吾见其败也。”}卓未至而进见杀。卓到,废帝为弘农王而立献帝\footnote{东汉末代皇帝刘协,少帝刘辩的异母弟,公元189--220年在位,在位时东汉已名存实亡。},京都大乱。卓表\footnote{上表举荐。}太祖为骁骑校尉\footnote{官名。统领京师禁卫军,掌管京师宿卫。},欲与计事。太祖乃变易姓名,间行\footnote{取小路悄悄走。}东归。\footnote{魏曰:太祖以卓终必覆败,遂不就拜,逃归乡里。从数骑过故人成皋吕伯奢;伯奢不在,其子与宾客共劫太祖,取马及物,太祖手刃击杀数人。世语曰:太祖过伯奢。伯奢出行,五子皆在,备宾主礼。太祖自以背卓命,疑其图己,手剑夜杀八人而去。孙盛杂记曰:太祖闻其食器声,以为图己,遂夜杀之。既而凄怆曰:“宁我负人,毋人负我!”遂行。}出关\footnote{虎牢关,今河南荥阳汜水镇。},过中牟\footnote{县名。今河南中牟东。},为亭长\footnote{官名。汉代乡间十里为一亭,设亭长一人,负责当地治安等事务。}所疑,执\footnote{捉拿,拘捕。}诣县,邑中或窃识之,为请得解。\footnote{世语曰:中牟疑是亡人,见拘于县。时掾亦已被卓书;唯功曹心知是太祖,以世方乱,不宜拘天下雄鉨,因白令释之。}卓遂杀太后及弘农王。太祖至陈留\footnote{县名。今河南开封东南。},散家财,合义兵,将以诛卓。冬十二月,始起兵于己吾\footnote{县名。今河南宁陵西南。},\footnote{世语曰:陈留孝廉韂兹以家财资太祖,使起兵,众有五千人。}是岁中平\footnote{汉灵帝年号(184--189)}六年也。
\end{yuanwen}

金城人边章、韩遂杀掉凉州刺史和金城太守叛乱,聚集部众十余万人,天下惊扰不安。朝廷征召太祖入朝出任典军校尉。正好赶上汉灵帝去世,太子刘辩即位,何太后临朝听政。大将军何进与袁绍谋划诛杀宦官,何太后不肯听从。何进便征召董卓率军进京,想以此胁迫何太后答应,董卓还没有到达,而何进已先被杀害。董卓到京城后,废黜少帝刘辩为弘农王,另立献帝刘协,京城大乱。董卓上表举荐太祖为骁骑校尉,想和他商议政事。太祖于是改换姓名,从小路向东悄悄返回家乡。出虎牢关,途经中牟时,一个亭长对太祖产生怀疑,把他抓起来送到县城。县城里有人暗中认出了他,便为他向县里求情,结果太祖被释放。这时董卓在京城杀掉何太后和弘农王。太祖到达陈留,散出家中的财物,聚集义兵,准备凭借这支军队消灭董卓。冬十二月,太祖开始在已吾起兵举义,这一年是中平六年。

\begin{yuanwen}
初平\footnote{汉献帝年号(190--193)。}元年春正月,后将军\footnote{官名。与前、左、右将军为四将军,掌统兵征伐,地位很高。}袁术\footnote{东汉末群雄之一。}、冀州\footnote{州名。今河北临漳西南。}牧\footnote{州牧,官名。汉代各州或置州牧,或置刺史,二者职位相同而州牧稍重。}韩馥、\footnote{英雄记曰:馥字文节,颍川人。为御史中丞。董卓举为冀州牧。于时冀州民人殷盛,兵粮优足。袁绍之在勃海,馥恐其兴兵,遣数部从事守之,不得动摇。东郡太守桥瑁诈作京师三公移书与州郡,陈卓罪恶,云“见逼迫,无以自救,企望义兵,解国患难。”馥得移,请诸从事问曰:“今当助袁氏邪,助董卓邪?”治中从事刘子惠曰:“今兴兵为国,何谓袁、董!”馥自知言短而有惭色。子惠复言:“兵者凶事,不可为首;今宜往视他州,有发动者,然后和之。冀州于他州不为弱也,他人功未有在冀州之右者也。”馥然之。馥乃作书与绍,道卓之恶,听其举兵。}豫州\footnote{州名。今安徽亳县。}刺史孔伷\footnote{zh\`ou}、\footnote{英雄记曰:伷字公绪,陈留人。张璠汉纪载郑泰说卓云:“孔公绪能清谈高论,嘘枯吹生。”}兖州\footnote{州名。今山东金乡西北。}刺史刘岱、\footnote{岱,刘繇之兄,事见吴志。}河内\footnote{郡名。今河南武陟西南。}太守王匡\footnote{英雄记曰:匡字公节,泰山人。轻财好施,以任侠闻。辟大将军何进府进符使,匡于徐州发强弩五百西诣京师。会进败,匡还州里。起家,拜河内太守。谢承后汉书曰:匡少与蔡邕善。其年为卓军所败,走还泰山,收集劲勇得数千人,欲与张邈合。匡先杀执金吾胡母班。班亲属不胜愤怒,与太祖并势,共杀匡。}、勃海\footnote{郡名。今河北南皮东北。}太守袁绍、陈留太守张邈、东郡太守桥瑁\footnote{英雄记曰:瑁字符伟,玄族子。先为兖州刺史,甚有威惠。}、山阳\footnote{郡名。今山东金乡西北,为兖州治所。}太守袁遗\footnote{遗字伯业,绍从兄。为长安令。河间张超尝荐遗于太尉朱鉨,称遗“有冠世之懿,干时之量。其忠允亮直,固天所纵;若乃包罗载籍,管综百氏,登高能赋,鷪物知名,求之今日,邈焉靡俦。”事在超集。英雄记曰:绍后用遗为扬州刺史,为袁术所败。太祖称“长大而能勤学者,惟吾与袁伯业耳。”语在文帝典论。}、济北\footnote{王国名。今山东长清南。}相鲍信\footnote{信事见子勋传。}同时俱起兵,众各数万,推绍为盟主。太祖行\footnote{代理。}奋武将军\footnote{官名。汉杂号将军,掌统兵征伐。汉代杂号将军名目繁多,不常置,因事临时而设,战事结束则罢。}。
\end{yuanwen}

初平元年春正月,后将军袁术、冀州牧韩馥、豫州刺史孔伷、兖州刺史刘岱、河内太守王匡、勃海太守袁绍、陈留太守张邈、东郡太守桥瑁、山阳太守袁遗、济北相鲍信同时起兵讨伐董卓,各有部众数万,推举袁绍为盟主。太祖代理奋武将军。

\begin{yuanwen}
二月,卓闻兵起,乃徙天子都长安\footnote{西汉都城,在今陕西长安西北。}。卓留屯洛阳,遂焚宫室。是时绍屯河内,邈、岱、瑁、遗屯酸枣\footnote{县名。今河南延津西南。},术屯南阳,伷屯颍川,馥在邺\footnote{县名。今河北临漳西南。}。卓兵强,绍等莫敢先进。太祖曰:“举义兵以诛暴乱,大众已合,诸君何疑?向使\footnote{假使。}董卓闻山东\footnote{崤山以东地区。}兵起,倚王室之重,据二周\footnote{指战国时西周、东周两个小国。西周位于今河南洛阳西,东周位于今河南洛阳东。}之险,东向以临天下;虽以无道行之,犹足为患。今焚烧宫室,劫迁天子,海内震动,不知所归,此天亡之时也。一战而天下定矣,不可失也。”

遂引兵西,将据成皋\footnote{县名。今河南荥阳西北。}。邈遣将卫兹分兵随太祖。到荥阳\footnote{县名。今河南荥阳东北。}汴水,遇卓将徐荣,与战不利,士卒死伤甚多。太祖为流矢所中,所乘马被创,从弟\footnote{堂弟。}洪以马与太祖,得夜遁去。荣见太祖所将兵少,力战尽日,谓酸枣未易攻也,亦引兵还。
\end{yuanwen}

这年二月,董卓闻知袁绍等人起兵,便将献帝迁徙到长安,以长安为都城。董卓自己率军留驻洛阳,于是纵火焚烧洛阳的皇官。当时袁绍驻军河内,张邈、刘岱、桥瑁、袁遗等人驻军酸枣,袁术驻军南阳,孔伷驻军颖川,韩馥驻军邺县。董卓兵力强盛,袁绍等没人敢率先进军。太祖说:“我们起义兵来诛灭暴乱,大军已经聚合,诸位还迟疑什么?假使董卓闻知山东起兵,依仗皇家的权势威望,据守洛阳东西两边的险要之地,向东进军以控制天下,那即使他的行为不合道义,也还是足以成为祸患。现在他却焚烧皇官,劫持迁徙天子海内震动,人们不知道去归附谁,这是上天灭亡他的好时机啊。只须一战就可安定天下,这个好机会不能失掉。”

便自己领兵西进,准备占据成皋。张邈分出一部分将士交与手下将领卫兹统领,派遣他跟随太祖进军。军至荥阳汴水,遇到董卓的将领徐荣,太祖与之交战不利,土卒死伤甚多。太祖被流箭射中,所乘战马受伤,他的堂弟曹洪把自己的战马给他乘骑,他才得以乘夜逃走。徐荣见太祖所率领的军队数量很少,却能够力战一整天,认为酸枣不容易攻下,便也率军返回。

\begin{yuanwen}
太祖到酸枣,诸军兵十余万,日置酒高会,不图进取。太祖责让\footnote{责备。}之,因为谋曰:“诸君听吾计,使勃海\footnote{指袁绍。}引河内之众临孟津\footnote{黄河渡口。位于今河南孟县西南,设有关隘。},酸枣诸将守成皋,据敖仓\footnote{大粮仓名。位于今河南郑州西北。},塞轘\footnote{hu\'an}辕\footnote{关名。位于今河南偃师东南。}、太谷\footnote{关名。位于今河南洛阳东南。},全制其险;使袁将军率南阳之军军丹\footnote{丹水县,今河南淅川西。}、析\footnote{县名。今河南西峡。},入武关\footnote{关名。位于今陕西丹凤东南。},以震三辅\footnote{西汉武帝时,以京兆尹、右扶风、左冯翊分管京师长安附件地区,称三辅。}:皆高垒深壁,勿与战,益为疑兵,示天下形势,以顺诛逆,可立定也。今兵以义动,持疑而不进,失天下之望,窃\footnote{敬词,私下。}为诸君耻之!”邈等不能用。
\end{yuanwen}

太祖回到酸枣,那里的各路军队十余万人,天天置办酒宴聚会作乐,不谋划进攻董卓的大事。太祖责备他们,并趁便为他们谋划说:“诸位请听我的计策,让勃海太守领河内的军队迫近孟津,酸枣的诸位将军镇守成皋,占据敖仓,阻断轘辕关和太古关,完全控制险要之地;让袁术将军率领南阳的将士进军丹水、析县,进人武关,以威慑三辅地区:各路军队都高筑壁垒,深挖堑壕,不与敌人交战,多设疑兵,显示出有利于我军的天下大势,以正义讨伐叛逆,可以立刻平定。现在我们以正义起兵,却心存疑虑不敢前进,令天下人大失所望,我私下为诸位感到羞耻!”张邈等人没能采纳太祖的计划。

\begin{yuanwen}
太祖兵少,乃与夏侯惇等诣扬州募兵,刺史陈温、丹杨太守周昕与兵四千余人。还到龙亢,士卒多叛。\footnote{魏书曰:兵谋叛,夜烧太祖帐,太祖手剑杀数十人,余皆披靡,乃得出营;其不叛者五百余人。刘岱与桥瑁相恶,岱杀瑁,以王肱领东郡太守。}至铚、建平,复收兵得千余人,进屯河内。

袁绍与韩馥谋立幽州牧刘虞为帝,太祖拒之。[一]绍又尝得一玉印,于太祖坐中举向其肘,太祖由是笑而恶焉。[二]
注[一]魏书载太祖答绍曰:“董卓之罪,暴于四海,吾等合大众、兴义兵而远近莫不响应,此以义动故也。今幼主微弱,制于奸臣,未有昌邑亡国之衅,而一旦改易,天下其孰安之?
诸君北面,我自西向。”
注[二]魏书曰:太祖大笑曰:“吾不听汝也。”绍复使人说太祖曰:“今袁公势盛兵强,二子已长,天下髃英,孰踰于此?”太祖不应。由是益不直绍,图诛灭之。
二年春,绍、馥遂立虞为帝,虞终不敢当。
夏四月,卓还长安。
秋七月,袁绍胁韩馥,取冀州。
黑山贼于毒、白绕、眭固等*眭,申随反。*十余万众略魏郡、东郡,王肱不能御,太祖引兵入东郡,击白绕于濮阳,破之。袁绍因表太祖为东郡太守,治东武阳。

三年春,太祖军顿丘,毒等攻东武阳。太祖乃引兵西入山,攻毒等本屯。\footnote{魏书曰:诸将皆以为当还自救。太祖曰:“孙膑救赵而攻魏,耿弇欲走西安攻临菑。使贼闻我西而还,武阳自解也;不还,我能败其本屯,虏不能拔武阳必矣。”遂乃行。}毒闻之,弃武阳还。太祖要击眭\footnote{su\=i}固,又击匈奴於夫罗于内黄,皆大破之。\footnote{魏书曰:于夫罗者,南单于子也。中平中,发匈奴兵,于夫罗率以助汉。会本国反,杀南单于,于夫罗遂将其众留中国。因天下挠乱,与西河白波贼合,破太原、河内,抄略诸郡为寇。}

夏四月,司徒王允与吕布共杀卓。卓将李傕\footnote{ju\'e}、郭汜\footnote{s\`i}等杀允攻布,布败,东出武关。傕等擅朝政。
\end{yuanwen}

\begin{yuanwen}
青州\footnote{州名。今山东淄博东北。}黄巾众百万入兖州,杀任城\footnote{王国名。今山东济宁。}相郑遂,转入东平\footnote{王国名。今山东东平东。}。刘岱欲击之,鲍信谏曰:“今贼众百万,百姓皆震恐,士卒无斗志,不可敌也。观贼众群辈相随,军无辎重,唯以钞略为资,今不若畜士众之力,先为固守。彼欲战不得,攻又不能,其势必离散,后选精锐,据其要害,击之可破也。”岱不从,遂与战,果为所杀。\footnote{世语曰:岱既死,陈宫谓太祖曰:“州今无主,而王命断绝,宫请说州中,明府寻往牧之,资之以收天下,此霸王之业也。”宫说别驾、治中曰:“今天下分裂而州无主;曹东郡,命世之才也,若迎以牧州,必宁生民。”鲍信等亦谓之然。}信乃与州吏万潜等至东郡迎太祖领\footnote{兼任。}兖州牧。遂进兵击黄巾于寿张\footnote{县名。今山东东平西南。}东。信力战斗死,仅而破之。\footnote{魏书曰:太祖将步骑千余人,行视战地,卒抵贼营,战不利,死者数百人,引还。贼寻前进。黄巾为贼久,数乘胜,兵皆精悍。太祖旧兵少,新兵不习练,举军皆惧。太祖被甲婴冑,亲巡将士,明劝赏罚,众乃复奋,承闲讨击,贼稍折退。贼乃移书太祖曰:“昔在济南,毁坏神坛,其道乃与中黄太乙同,似若知道,今更迷惑。汉行已尽,黄家当立。天之大运,非君才力所能存也。”太祖见檄书,呵骂之,数开示降路;遂设奇伏,昼夜会战,战辄禽获,贼乃退走。}购求\footnote{悬赏寻求。}信丧\footnote{遗体。}不得,众乃刻木如信形状,祭而哭焉。追黄巾至济北。乞降。冬\footnote{初平三年(192)冬。},受降卒三十余万,男女百余万口,收其精锐者,号为青州兵。
\end{yuanwen}

青州黄巾军上百万人进入充州,杀掉任城相郑遂,转头进入东平。刘岱想进兵攻打他们,鲍信劝速说:“现在贼众有百万人,百姓全都震惊惶恐,士卒没有斗志,抵挡不住他们,我看贼众有老小家属跟随,军中没有辎重,只靠劫掠来供给军用,现在不如积蓄将士们的力量,先据城固守。敌人想要交战没有机会,攻城又攻不下来,势必会分崩离散,到那时我们挑选精锐,占据要害之地,再发动进攻,就可以将他们打败了。”刘岱不肯听从,与黄巾军交战,果然兵败被杀。鲍信便与州吏万潜等人到东郡迎接太祖,请他兼任兖州牧。太祖于是进军,在寿张东与黄巾军展开激战,鲍信临阵战死,太祖军竭尽全力才勉强把黄巾军打败。悬赏寻找鲍信的遗体,没能找到,于是大家用木头刻成鲍信的像,哭着祭奠他。太祖追击黄巾军到济北。黄巾军请求投降。这年冬天,收纳黄巾军降卒三十余万,男女人口一百余万,挑选出其中的精锐士卒组成军队,号称青州兵。

\begin{yuanwen}
袁术与绍有隙,术求援于公孙瓒,瓒使刘备屯高唐,单经屯平原,陶谦屯发干,以逼绍。太祖与绍会击,皆破之。

四年春,军鄄城。荆州牧刘表断术粮道,术引军入陈留,屯封丘,黑山余贼及於夫罗等佐之。

术使将刘详屯匡亭。太祖击详,术救之,与战,大破之。术退保封丘,遂围之,未合,术走襄邑,追到太寿,决渠水灌城。走宁陵,又追之,走九江。夏,太祖还军定陶。

下邳阙宣聚众数千人,自称天子;徐州牧陶谦与共举兵,取泰山华、费,略任城。秋,太祖征陶谦,下十余城,谦守城不敢出。

是岁,孙策受袁术使渡江,数年闲遂有江东。

兴平元年春,太祖自徐州还,初,太祖父嵩,去官后还谯,董卓之乱,避难琅邪,为陶谦所害,故太祖志在复仇东伐。\footnote{世语曰:嵩在泰山华县。太祖令泰山太守应劭送家诣兖州,劭兵未至,陶谦密遣数千骑掩捕。嵩家以为劭迎,不设备。谦兵至,杀太祖弟德于门中。嵩惧,穿后垣,先出其妾,妾肥,不时得出;嵩逃于厕,与妾俱被害,阖门皆死。劭惧,弃官赴袁绍。后太祖定冀州,劭时已死。韦曜吴书曰:太祖迎嵩,辎重百余两。陶谦遣都尉张闿将骑二百韂送,闿于泰山华、费间杀嵩,取财物,因奔淮南。太祖归咎于陶谦,故伐之。}夏,使荀彧、程昱守鄄城,复征陶谦,拔五城,遂略地至东海。还过郯,谦将曹豹与刘备屯郯东,要太祖。太祖击破之,遂攻拔襄贲,所过多所残戮。\footnote{孙盛曰:夫伐罪吊民,古之令轨;罪谦之由,而残其属部,过矣。}

会张邈与陈宫叛迎吕布,郡县皆应。荀彧、程昱保鄄城,范、东阿二县固守,太祖乃引军还。

布到,攻鄄城不能下,西屯濮阳。太祖曰:“布一旦得一州,不能据东平,断亢父、泰山之道,乘险要我,而乃屯濮阳,吾知其无能为也。”遂进军攻之。

布出兵战,先以骑犯青州兵。

青州兵奔,太祖陈\footnote{通“阵”,指军阵。}乱,驰突火出,坠马,烧左手掌。司马楼异扶太祖上马,遂引去。\footnote{袁暐献帝春秋曰:太祖围濮阳,濮阳大姓田氏为反闲,太祖得入城。烧其东门,示无反意。及战,军败。布骑得太祖而不知是,问曰:“曹操何在?”太祖曰:“乘黄马走者是也。”布骑乃释太祖而追黄马者。门火犹盛,太祖突火而出。}未至营止,诸将未与太祖相见,皆怖。太祖乃自力劳军,令军中促为攻具,进复攻之,与布相守百余日。蝗虫起,百姓大饿,布粮食亦尽,各引去。

秋九月,太祖还鄄城。布到乘氏,为其县人李进所破,东屯山阳。于是绍使人说太祖,欲连和。太祖新失兖州,军食尽,将许之。程昱止太祖,太祖从之。冬十月,太祖至东阿。

是岁谷一斛\footnote{h\'u,一种计量单位,一斛本为十斗,后来改为五斗。}五十余万钱,人相食,乃罢吏兵新募者。陶谦死,刘备代之。

二年春,袭定陶。济阴太守吴资保南城,未拔。会吕布至,又击破之。夏,布将薛兰、李封屯钜野,太祖攻之,布救兰,兰败,布走,遂斩兰等。布复从东缗\footnote{m\'in}与陈宫将万余人来战,时太祖兵少,设伏,纵奇兵击,大破之。\footnote{魏书曰:于是兵皆出取麦,在者不能千人,屯营不固。太祖乃令妇人守陴,悉兵拒之。屯西有大堤,其南树木幽深。布疑有伏,乃相谓曰:“曹操多谲,勿入伏中。”引军屯南十余里。明日复来,太祖隐兵堤里,出半兵堤外。布益进,乃令轻兵挑战,既合,伏兵乃悉乘堤,步骑并进,大破之,获其鼓车,追至其营而还。}布夜走,太祖复攻,拔定陶,分兵平诸县。布东奔刘备,张邈从布,使其弟超将家属保雍丘。秋八月,围雍丘。冬十月,天子拜太祖兖州牧。

十二月,雍丘溃,超自杀。夷邈三族。邈诣袁术请救,为其众所杀,兖州平,遂东略陈地。

是岁,长安乱,天子东迁,败于曹阳,渡河幸安邑。
\end{yuanwen}

\begin{yuanwen}
建安\footnote{汉献帝年号(196--220)。}元年春正月,太祖军临武平\footnote{县名。今河南鹿邑。},袁术所置陈相\footnote{陈国国相。陈国治今河南淮阳。}袁嗣降。太祖将迎天子\footnote{当时天子汉献帝久经颠沛流离,暂居于安邑县(今山西夏县西北)。曹操准备将汉献帝接到自己身边,以取得政治上的优势。},诸将或疑,荀彧、程昱\footnote{都是曹操的重要谋士。}劝之,乃遣曹洪将兵西迎。卫将军\footnote{官名。统领京师禁卫军,掌管京师及宫廷宿卫,权位稍逊于三公。}董承与袁术将苌\footnote{ch\'ang}奴拒险,洪不得进。
\end{yuanwen}

建安元年春正月,太祖军至武平,袁术设置的陈国相袁嗣投降。太祖准备去迎接汉献帝,众将中有人怀疑这件事,荀彧、程昱二人则极力赞成,太祖于是派遣曹洪率军西进迎接献帝。卫将军董承与袁术将领苌奴凭险阻挡,曹洪无法前进。

\begin{yuanwen}
汝南、颍川黄巾何仪、刘辟、黄邵、何曼等,众各数万,初应袁术,又附孙坚。二月,太祖进军讨破之,斩辟、邵等,仪及其众皆降。
\end{yuanwen}

\begin{yuanwen}
天子拜太祖建德将军,夏六月,迁镇东将军,封费亭侯。秋七月,杨奉、韩暹\footnote{xi\=an}\footnote{都是东汉末军阀混战中的群雄。}以天子还洛阳,\footnote{献帝春秋曰:天子初至洛阳,幸城西故中常侍赵忠宅。使张杨缮治宫室,名殿曰扬安殿,八月,帝乃迁居。}奉别屯梁\footnote{县名。今河南临汝西。}。太祖遂至洛阳,卫京都,暹遁走\footnote{逃走。走,跑,也特指逃跑。}。天子假\footnote{借,此处为授予之意。}太祖节\footnote{符节,皇帝授予臣子行使权力的凭证。}钺\footnote{yu\`e,古代兵器,状如大斧,常作为仪仗表示王权,这时又称黄钺。大臣假黄钺者,表示代表皇帝征讨四方,有权指挥全国军队。},录尚书事\footnote{总领尚书台事务。东汉尚书台是实际的行政中枢,录尚书事者无所不总,实即独揽朝政,后录尚书事成固定官职。}。\footnote{献帝纪曰:又领司隶校尉。}洛阳残破,董昭等劝太祖都许\footnote{县名。今河南许昌东。}。九月,车驾出轘辕而东,以太祖为大将军,封武平侯\footnote{封于武平县的县侯。}。自天子西迁,朝廷日乱,至是宗庙\footnote{皇帝祭祀祖先的处所。}社\footnote{祭祀土神的祭坛。}稷\footnote{祭祀谷神的祭坛。}制度始立。\footnote{张璠汉纪曰:初,天子败于曹阳,欲浮河东下。侍中太史令王立曰:“自去春太白犯镇星于牛斗,过天津,荧惑又逆行守北河,不可犯也。”由是天子遂不北渡河,将自轵关东出。立又谓宗正刘艾曰:“前太白守天关,与荧惑会;金火交会,革命之象也。汉祚终矣,晋、魏必有兴者。”立后数言于帝曰:“天命有去就,五行不常盛,代火者土也,承汉者魏也,能安天下者,曹姓也,唯委任曹氏而已。”公闻之,使人语立曰:“知公忠于朝廷,然天道深远,幸勿多言。”}
\end{yuanwen}

汉献帝任命太祖为建德将军。夏六月,太祖升任镇东将军,封费亭侯。秋七月,杨奉、韩暹护送献帝返回洛阳,杨奉另率一军驻扎梁县。太祖于是进军洛阳,捍卫京师,韩暹逃走。献帝授予太祖符节黄钺,使太祖录尚书事。洛阳已经残破不堪,董昭等人便劝太祖将都城迁至许县。九月,献帝出辍辕关向东进发,任命太祖为大将军,封武平侯。自从献帝西迁长安以来,朝廷上下日益混乱,到这时,宗庙社稷等祭祀制度才开始建立起来,

\begin{yuanwen}
天子之东也,奉自梁欲要\footnote{y\=ao,半路拦截。}之,不及。冬十月,公征奉,奉南奔袁术,遂攻其梁屯,拔之。于是以袁绍为太尉,绍耻班\footnote{等级,位次。}在公下,不肯受。公乃固辞,以大将军让绍。天子拜公司空\footnote{官名。掌管全国的土木建筑及水利工程,为三公之一。},行车骑将军\footnote{官名。统领京师禁卫军,掌管京师及官延宿卫,地位很高,位次同于三公。}。是岁用枣祗、韩浩\footnote{都是曹操手下的重要将领,二人建屯田之议。}等议,始兴屯田\footnote{指曹操建立的强制百姓屯田制度。屯田制度的成功推行,保证了曹操的军粮供应,为他统一北方莫定了物质基础。}。\footnote{魏书曰:自遭荒乱,率乏粮谷。诸军并起,无终岁之计,饥则寇略,饱则弃余,瓦解流离,无敌自破者不可胜数。袁绍之在河北,军人仰食桑椹。袁术在江、淮,取给蒲蠃。民人相食,州里萧条。公曰:“夫定国之术,在于强兵足食,秦人以急农兼天下,孝武以屯田定西域,此先代之良式也。”是岁乃募民屯田许下,得谷百万斛。于是州郡例置田官,所在积谷。征伐四方,无运粮之劳,遂兼灭髃贼,克平天下。}
\end{yuanwen}

在献帝东迁许县的时候,杨奉从梁县进军准备拦截,但能赶上。冬十月,曹公进军征讨杨奉,杨奉向南去投奔袁

\begin{yuanwen}
吕布袭刘备,取下邳。备来奔。程昱说公曰:“观刘备有雄才而甚得众心,终不为人下,不如早图之。”

公曰:“方今收英雄时也,杀一人而失天下之心,不可。”

张济自关中走南阳。济死,从子绣领其众。二年春正月,公到宛。张绣降,既而悔之,复反。

公与战,军败,为流矢所中,长子昂、弟子安民遇害。\footnote{魏书曰:公所乘马名绝影,为流矢所中,伤颊及足,并中公右臂。世语曰:昂不能骑,进马于公,公故免,而昂遇害。}公乃引兵还舞阴,绣将骑来钞,公击破之。绣奔穰,与刘表合。公谓诸将曰:“吾降张绣等,失不便取其质,以至于此。吾知所以败。诸卿观之,自今已后不复败矣。”遂还许。\footnote{世语曰:旧制,三公领兵入见,皆交戟叉颈而前。初,公将讨张绣,入觐天子,时始复此制。公自此不复朝见。}

袁术欲称帝于淮南,使人告吕布。布收其使,上其书。术怒,攻布,为布所破。秋九月,术侵陈,公东征之。术闻公自来,弃军走,留其将桥蕤\footnote{ru\'i}、李丰、梁纲、乐就;公到,击破蕤等,皆斩之。术走渡淮。公还许。

公之自舞阴还也,南阳、章陵诸县复叛为绣,公遣曹洪击之,不利,还屯叶,数为绣、表所侵。冬十一月,公自南征,至宛。\footnote{魏书曰:临淯水,祠亡将士,歔欷流涕,众皆感恸。}表将邓济据湖阳。攻拔之,生擒济,湖阳降。攻舞阴,下之。

三年春正月,公还许,初置军师祭酒。三月,公围张绣于穰。夏五月,刘表遣兵救绣,以绝军后。\footnote{献帝春秋曰:袁绍叛卒诣公云:“田丰使绍早袭许,若挟天子以令诸侯,四海可指麾而定。”公乃解绣围。}公将引还,绣兵来追,公军不得进,连营稍前。公与荀彧书曰:“贼来追吾,虽日行数里,吾策之,到安众,破绣必矣。”

到安众,绣与表兵合守险,公军前后受敌。公乃夜凿险为地道,悉过辎重,设奇兵。会明,贼谓公为遁也,悉军来追。乃纵奇兵步骑夹攻,大破之。

秋七月,公还许。荀彧问公:“前以策贼必破,何也?”

公曰:“虏遏吾归师,而与吾死地战,吾是以知胜矣。”

吕布复为袁术使高顺攻刘备,公遣夏侯惇救之,不利。备为顺所败。九月,公东征布。冬十月,屠彭城,获其相侯谐。进至下邳,布自将骑逆击。大破之,获其骁将成廉。追至城下,布恐,欲降。陈宫等沮其计,求救于术,劝布出战,战又败,乃还固守,攻之不下。时公连战,士卒罢,欲还,用荀攸、郭嘉计,遂决泗、沂水以灌城。月余,布将宋宪、魏续等执陈宫,举城降,生禽布、宫,皆杀之。太山臧霸、孙观、吴敦、尹礼、昌豨\footnote{x\=i}各聚众。布之破刘备也,霸等悉从布。布败,获霸等,公厚纳待,遂割青、徐二州附于海以委焉,分琅邪、东海、北海为城阳、利城、昌虑郡。

初,公为兖州,以东平毕谌为别驾。张邈之叛也,邈劫谌母弟妻子;公谢遣之,曰:“卿老母在彼,可去。”谌顿首无二心,公嘉之,为之流涕。既出,遂亡归。及布破,谌生得,众为谌惧,公曰:“夫人孝于其亲者,岂不亦忠于君乎!吾所求也。”以为鲁相。[一]
注[一]魏书曰:袁绍宿与故太尉杨彪、大长秋梁绍、少府孔融有隙,欲使公以他过诛之。公曰:“当今天下土崩瓦解,雄豪并起,辅相君长,人怀怏怏,各有自为之心,此上下相疑之秋也,虽以无嫌待之,犹惧未信;如有所除,则谁不自危?且夫起布衣,在尘垢之间,为庸人之所陵陷,可胜怨乎!高祖赦雍齿之雠而髃情以安,如何忘之?”绍以为公外托公义,内实离异,深怀怨望。臣松之以为杨彪亦曾为魏武所困,几至于死,孔融竟不免于诛灭,岂所谓先行其言而后从之哉!非知之难,其在行之,信矣。
四年春二月,公还至昌邑。张杨将杨丑杀杨,眭固又杀丑,以其众属袁绍,屯射犬。夏四月,进军临河,使史涣、曹仁渡河击之。固使杨故长史薛洪、河内太守缪尚留守,自将兵北迎绍求救,与涣、仁相遇犬城。交战,大破之,斩固。公遂济河,围射犬。洪、尚率众降,封为列侯,还军敖仓。以魏种为河内太守,属以河北事。
初,公举种孝廉。兖州叛,公曰:“唯魏种且不弃孤也。”及闻种走,公怒曰:“种不南走越、北走胡,不置汝也!”既下射犬,生禽种,公曰:“唯其才也!”释其缚而用之。

是时袁绍既并公孙瓒,兼四州之地,众十余万,将进军攻许。诸将以为不可敌,公曰:“吾知绍之为人,志大而智小,色厉而胆薄,忌克而少威,兵多而分画不明,将骄而政令不一,土地虽广,粮食虽丰,适足以为吾奉也。”秋八月,公进军黎阳,使臧霸等入青州破齐、北海、东安,留于禁屯河上。九月,公还许,分兵守官渡。冬十一月,张绣率众降,封列侯。

十二月,公军官渡。

袁术自败于陈,稍困,袁谭自青州遣迎之。术欲从下邳北过,公遣刘备、朱灵要之。会术病死。程昱、郭嘉闻公遣备,言于公曰:“刘备不可纵。”公悔,追之不及。备之未东也,阴与董承等谋反,至下邳,遂杀徐州刺史车冑,举兵屯沛。遣刘岱、王忠击之,不克。\footnote{献帝春秋曰:备谓岱等曰:“使汝百人来,其无如我何;曹公自来,未可知耳!”魏武故事曰:岱字公山,沛国人。以司空长史从征伐有功,封列侯。魏略曰:王忠,扶风人,少为亭长。三辅乱,忠饥乏噉人,随辈南向武关。值娄子伯为荆州遣迎北方客人;忠不欲去,因率等仵逆击之,夺其兵,聚众千余人以归公。拜忠中郎将,从征讨。五官将知忠尝噉人,因从驾出行,令俳取頉间髑髅系着忠马鞍,以为欢笑。庐江太守刘勋率众降,封为列侯。}

五年春正月,董承等谋泄,皆伏诛。公将自东征备,诸将皆曰:“与公争天下者,袁绍也。今绍方来而弃之东,绍乘人后,若何?”

公曰:“夫刘备,人杰也,今不击,必为后患。\footnote{孙盛魏氏春秋云:答诸将曰:“刘备,人杰也,将生忧寡人。”	臣松之以为史之记言,既多润色,故前载所述有非实者矣,后之作者又生意改之,于失实也,不亦弥远乎!凡孙盛制书,多用左氏以易旧文,如此者非一。嗟乎,后之学者将何取信哉?且魏武方以天下励志,而用夫差分死之言,尤非其类。}袁绍虽有大志,而见事迟,必不动也。”

郭嘉亦劝公,遂东击备,破之,生禽其将夏侯博。

备走奔绍,获其妻子。备将关羽屯下邳,复进攻之,羽降。昌豨叛为备,又攻破之。公还官渡,绍卒不出。

二月,绍遣郭图、淳于琼、颜良攻东郡太守刘延于白马,绍引兵至黎阳,将渡河。夏四月,公北救延。荀攸说公曰:“今兵少不敌,分其势乃可。公到延津,若将渡兵向其后者,绍必西应之,然后轻兵袭白马,掩其不备,颜良可禽也。”公从之。绍闻兵渡,即分兵西应之。

公乃引军兼行趣白马,未至十余里,良大惊,来逆战。使张辽、关羽前登,击破,斩良。遂解白马围,徙其民,循河而西。绍于是渡河追公军,至延津南。公勒兵驻营南阪下,使登垒望之,曰:“可五六百骑。”有顷,复白:“骑稍多,步兵不可胜数。”公曰:“勿复白。”

乃令骑解鞍放马。是时,白马辎重就道。诸将以为敌骑多,不如还保营。荀攸曰:“此所以饵敌,如何去之!”绍骑将文丑与刘备将五六千骑前后至。诸将复白:“可上马。”公曰:“未也。”有顷,骑至稍多,或分趣辎重。公曰:“可矣。”乃皆上马。时骑不满六百,遂纵兵击,大破之,斩丑。良、丑皆绍名将也,再战,悉禽,绍军大震。公还军官渡。绍进保阳武。关羽亡归刘备。

八月,绍连营稍前,依沙塠为屯,东西数十里。公亦分营与相当,合战不利。\footnote{羽凿齿汉晋春秋曰:许攸说绍曰:“公无与操相攻也。急分诸军持之,而径从他道迎天子,则事立济矣。”绍不从,曰:“吾要当先围取之。”攸怒。}时公兵不满万,伤者十二三。\footnote{臣松之以为魏武初起兵,已有众五千,自后百战百胜,败者十二三而已矣。但一破黄巾,受降卒三十余万,余所吞并,不可悉纪;虽征战损伤,未应如此之少也。夫结营相守,异于摧锋决战。本纪云:“绍众十余万,屯营东西数十里。”魏太祖虽机变无方,略不世出,安有以数千之兵,而得逾时相抗者哉?以理而言,窃谓不然。绍为屯数十里,公能分营与相当,此兵不得甚少,一也。绍若有十倍之众,理应当悉力围守,使出入断绝,而公使徐晃等击其运车,公又自出击淳于琼等,扬旌往还,曾无抵阂,明绍力不能制,是不得甚少,二也。诸书皆云公坑绍众八万,或云七万。夫八万人奔散,非八千人所能缚,而绍之大众皆拱手就戮,何缘力能制之?是不得甚少,三也。将记述者欲以少见奇,非其实录也。按钟繇传云:“公与绍相持,繇为司隶,送马二千余匹以给军。”本纪及世语并云公时有骑六百余匹,繇马为安在哉?}绍复进临官渡,起土山地道。公亦于内作之,以相应。绍射营中,矢如雨下,行者皆蒙楯,众大惧。时公粮少,与荀彧书,议欲还许。彧以为:“绍悉众聚官渡,欲与公决胜败。公以至弱当至强,若不能制,必为所乘,是天下之大机也。且绍,布衣之雄耳,能聚人而不能用。夫以公之神武明哲而辅以大顺,何向而不济!”公从之。

孙策闻公与绍相持,乃谋袭许,未发,为刺客所杀。

汝南降贼刘辟等叛应绍,略许下。绍使刘备助辟,公使曹仁击破之。备走,遂破辟屯。

袁绍运谷车数千乘至,公用荀攸计,遣徐晃、史涣邀击,大破之,尽烧其车。公与绍相拒连月,虽比战斩将,然众少粮尽,士卒疲乏。公谓运者曰:“却十五日为汝破绍,不复劳汝矣。”

冬十月,绍遣车运谷,使淳于琼等五人将兵万余人送之,宿绍营北四十里。绍谋臣许攸贪财,绍不能足,来奔,因说公击琼等。左右疑之,荀攸、贾诩劝公。公乃留曹洪守,自将步骑五千人夜往,会明至。琼等望见公兵少,出陈门外。公急击之,琼退保营,遂攻之。绍遣骑救琼。左右或言“贼骑稍近,请分兵拒之”。公怒曰:“贼在背后,乃白!”士卒皆殊死战,大破琼等,皆斩之。\footnote{曹瞒传曰:公闻攸来,跣出迎之,抚掌笑曰:“*(子卿远)**[子远,卿]*来,吾事济矣!”既入坐,谓公曰:“袁氏军盛,何以待之?今有几粮乎?”公曰:“尚可支一岁。”攸曰:“无是,更言之!”又曰:“可支半岁。”攸曰:“足下不欲破袁氏邪,何言之不实也!”公曰:“向言戏之耳。其实可一月,为之柰何?”攸曰:“公孤军独守,外无救援而粮谷已尽,此危急之日也。今袁氏辎重有万余乘,在故市、乌巢,屯军无严备;今以轻兵袭之,不意而至,燔其积聚,不过三日,袁氏自败也。”公大喜,乃选精锐步骑,皆用袁军旗帜,衔枚缚马口,夜从间道出,人抱束薪,所历道有问者,语之曰:“袁公恐曹操钞略后军,遣兵以益备。”闻者信以为然,皆自若。既至,围屯,大放火,营中惊乱。大破之,尽燔其粮谷宝货,斩督将眭元进、骑督韩莒子、吕威璜、赵叡等首,割得将军淳于仲简鼻,未死,杀士卒千余人,皆取鼻,牛马割唇舌,以示绍军。将士皆怛惧。时有夜得仲简,将以诣麾下,公谓曰:“何为如是?”仲简曰:“胜负自天,何用为问乎!”公意欲不杀。许攸曰:“明旦鉴于镜,此益不忘人。”乃杀之。}绍初闻公之击琼,谓长子谭曰:“就彼攻琼等,吾攻拔其营,彼固无所归矣!”乃使张郃、高览攻曹洪。郃等闻琼破,遂来降。绍众大溃,绍及谭弃军走,渡河。

追之不及,尽收其辎重图书珍宝,虏其众。\footnote{献帝起居注曰:公上言“大将军邺侯袁绍前与冀州牧韩馥立故大司马刘虞,刻作金玺,遣故任长毕瑜诣虞,为说命录之数。又绍与臣书云:‘可都鄄城,当有所立。’擅铸金银印,孝廉计吏,皆往诣绍。从弟济阴太守□与绍书云:‘今海内丧败,天意实在我家,神应有征,当在尊兄。南兄臣下欲使即位,南兄言,以年则北兄长,以位则北兄重。便欲送玺,会曹操断道。’绍宗族累世受国重恩,而凶逆无道,乃至于此。辄勒兵马,与战官渡,乘圣朝之威,得斩绍大将淳于琼等八人首,遂大破溃。绍与子谭轻身迸走。凡斩首七万余级,辎重财物巨亿。”}公收绍书中,得许下及军中人书,皆焚之。\footnote{魏氏春秋曰:公云:“当绍之强,孤犹不能自保,而况众人乎!”}冀州诸郡多举城邑降者。

初,桓帝时有黄星见于楚、宋之分,辽东殷馗*馗,古逵字,见三苍。*善天文,言后五十岁当有真人起于梁、沛之间,其锋不可当。至是凡五十年,而公破绍,天下莫敌矣。

六年夏四月,扬兵河上,击绍仓亭军,破之。绍归,复收散卒,攻定诸叛郡县。九月,公还许。绍之未破也,使刘备略汝南,汝南贼共都等应之。遣蔡扬击都,不利,为都所破。公南征备。备闻公自行,走奔刘表,都等皆散。

七年春正月,公军谯,令曰:“吾起义兵,为天下除暴乱。旧土人民,死丧略尽,国中终日行,不见所识,使吾凄怆伤怀。其举义兵已来,将士绝无后者,求其亲戚以后之,授土田,官给耕牛,置学师以教之。为存者立庙,使祀其先人,魂而有灵,吾百年之后何恨哉!”遂至浚仪,治睢阳渠,遣使以太牢祀桥玄。[一]进军官渡。
注[一]褒赏令载公祀文曰:“故太尉桥公,诞敷明德,泛爱博容。国念明训,士思令谟。灵幽体翳,邈哉晞矣!吾以幼年,逮升堂室,特以顽鄙之姿,为大君子所纳。增荣益观,皆由奖助,犹仲尼称不如颜渊,李生之厚叹贾复。士死知己,怀此无忘。又承从容约誓之言:‘殂逝之后,路有经由,不以斗酒只鸡过相沃酹,车过三步,腹痛勿怪!’虽临时戏笑之言,非至亲之笃好,胡肯为此辞乎?匪谓灵忿,能诒己疾,怀旧惟顾,念之凄怆。奉命东征,屯次乡里,北望贵土,乃心陵墓。裁致薄奠,公其尚飨!”

绍自军破后,发病欧\footnote{通“呕”,吐的意思。}血,夏五月死。小子尚代,谭自号车骑将军,屯黎阳。

秋九月,公征之,连战。谭、尚数败退,固守。

八年春三月,攻其郭,乃出战,击,大破之,谭、尚夜遁。

夏四月,进军邺。

五月还许,留贾信屯黎阳。

己酉,令曰:“司马法‘将军死绥’,[一]故赵括之母,乞不坐括。是古之将者,军破于外,而家受罪于内也。自命将征行,但赏功而不罚罪,非国典也。其令诸将出征,败军者抵罪,失利者免官爵。”[二]
注[一]魏书曰:绥,却也。有前一尺,无却一寸。
注[二]魏书载庚申令曰:“议者或以军吏虽有功能,德行不足堪任郡国之选,所谓‘可与适道,未可与权’。管仲曰:‘使贤者食于能则上尊,□士食于功则卒轻于死,二者设于国则天下治。’未闻无能之人,不□之士,并受禄赏,而可以立功兴国者也。故明君不官无功之臣,不赏不战之士;治平尚德行,有事赏功能。论者之言,一似管窥虎欤!”
秋七月,令曰:“丧乱已来,十有五年,后生者不见仁义礼让之风,吾甚伤之。其令郡国各修文学,县满五百户置校官,选其乡之俊造而教学之,庶几先王之道不废,而有以益于天下。”

八月,公征刘表,军西平。公之去邺而南也,谭、尚争冀州,谭为尚所败,走保平原。尚攻之急,谭遣辛毗乞降请救。诸将皆疑,荀攸劝公许之,\footnote{魏书曰:公云:“我攻吕布,表不为寇,官渡之役,不救袁绍,此自守之贼也,宜为后图。谭、尚狡猾,当乘其乱。纵谭挟诈,不终束手,使我破尚,偏收其地,利自多矣。”乃许之。}公乃引军还。

冬十月,到黎阳,为子整与谭结婚。\footnote{臣松之案:绍死至此,过周五月耳。谭虽出后其伯,不为绍服三年,而于再儙之内以行吉礼,悖矣。魏武或以权宜与之约言;今云结婚,未必便以此年成礼。}尚闻公北,乃释平原还邺。东平吕旷、吕翔叛尚,屯阳平,率其众降,封为列侯。\footnote{魏书曰:谭之围解,阴以将军印绶假旷。旷受印送之,公曰:“我固知谭之有小计也。欲使我攻尚,得以其闲略民聚众,尚之破,可得自强以乘我弊也。尚破我盛,何弊之乘乎?”}

九年春正月,济河,遏淇水入白沟以通粮道。

二月,尚复攻谭,留苏由、审配守邺。公进军到洹水,由降。既至,攻邺,为土山、地道。武安长尹楷屯毛城,通上党粮道。

夏四月,留曹洪攻邺,公自将击楷,破之而还。尚将沮鹄\footnote{沮音菹,河朔闲今犹有此姓。鹄,沮授子也。}守邯郸,又击拔之。易阳令韩范、涉长梁岐举县降,赐爵关内侯。

五月,毁土山、地道,作围堑,决漳水灌城;城中饿死者过半。

秋七月,尚还救邺,诸将皆以为“此归师,人自为战,不如避之”。公曰:“尚从大道来,当避之;若循西山来者,此成禽耳。”尚果循西山来,临滏水为营。\footnote{曹瞒传曰:遣候者数部前后参之,皆曰“定从西道,已在邯郸”。公大喜,会诸将曰:“孤已得冀州,诸君知之乎?”皆曰:“不知。”公曰:“诸君方见不久也。”}夜遣兵犯围,公逆击破走之,遂围其营。未合,尚惧,遣故豫州刺史阴夔\footnote{ku\'i}及陈琳乞降,公不许,为围益急。尚夜遁,保祁山,追击之。其将马延、张顗\footnote{y\v{i}}等临陈降,众大溃,尚走中山。尽获其辎重,得尚印绶节钺,使尚降人示其家,城中崩沮。八月,审配兄子荣夜开所守城东门内兵。配逆战,败,生禽配,斩之,邺定。公临祀绍墓,哭之流涕;慰劳绍妻,还其家人宝物,赐杂缯絮,廪食之。\footnote{孙盛云:昔者先王之为诛赏也,将以惩恶劝善,永彰鉴戒。绍因世艰危,遂怀逆谋,上议神器,下干国纪。荐社污宅,古之制也,而乃尽哀于逆臣之頉,加恩于饕餮之室,为政之道,于斯踬矣。夫匿怨友人,前哲所耻,税骖旧馆,义无虚涕,苟道乖好绝,何哭之有!昔汉高失之于项氏,魏武遵谬于此举,岂非百虑之一失也。}

初,绍与公共起兵,绍问公曰:“若事不辑,则方面何所可据?”公曰:“足下意以为何如?”
绍曰:“吾南据河,北阻燕、代,兼戎狄之众,南向以争天下,庶可以济乎?”公曰:“吾任天下之智力,以道御之,无所不可。”[一]
注[一]傅子曰:太祖又云:“汤、武之王,岂同土哉?若以险固为资,则不能应机而变化也。”
九月,令曰:“河北罹袁氏之难,其令无出今年租赋!”重豪强兼并之法,百姓喜悦。[一]天子以公领冀州牧,公让还兖州。
注[一]魏书载公令曰:“有国有家者,不患寡而患不均,不患贫而患不安。袁氏之治也,使豪强擅恣,亲戚兼并;下民贫弱,代出租赋,衒鬻家财,不足应命;审配宗族,至乃藏匿罪人,为逋逃主。欲望百姓亲附,甲兵强盛,岂可得邪!其收田租亩四升,户出绢二匹、绵二斤而已,他不得擅兴发。郡国守相明检察之,无令强民有所隐藏,而弱民兼赋也。”

公之围邺也,谭略取甘陵、安平、勃海、河间。尚败,还中山。谭攻之,尚奔故安,遂并其众。公遗谭书,责以负约,与之绝婚,女还,然后进军。谭惧,拔平原,走保南皮。

十二月,公入平原,略定诸县。

十年春正月,攻谭,破之,斩谭,诛其妻子,冀州平。\footnote{魏书曰:公攻谭,旦及日中不决;公乃自执桴鼓,士卒咸奋,应时破陷。}下令曰:“其与袁氏同恶者,与之更始。”令民不得复私仇,禁厚葬,皆一之于法。是月,袁熙大将焦触、张南等叛攻熙、尚,熙、尚奔三郡乌丸。触等举其县降,封为列侯。初讨谭时,民亡椎冰,\footnote{臣松之以为讨谭时,川渠水冻,使民椎冰以通船,民惮役而亡。}令不得降。

顷之,亡民有诣门首者,公谓曰:“听汝则违令,杀汝则诛首,归深自藏,无为吏所获。”

民垂泣而去;后竟捕得。

夏四月,黑山贼张燕率其众十余万降,封为列侯。故安赵犊、霍奴等杀幽州刺史、涿郡太守。
三郡乌丸攻鲜于辅于犷平。[一]秋八月,公征之,斩犊等,乃渡潞河救犷平,乌丸奔走出塞。
注[一]续汉书郡国志曰:犷平,县名,属渔阳郡。
九月,令曰:“阿党比周,先圣所疾也。闻冀州俗,父子异部,更相毁誉。昔直不疑无兄,世人谓之盗嫂;第五伯鱼三娶孤女,谓之挝妇翁;王凤擅权,谷永比之申伯,王商忠议,张匡谓之左道:此皆以白为黑,欺天罔君者也。吾欲整齐风俗,四者不除,吾以为羞。”冬十月,公还邺。

初,袁绍以甥高干领并州牧,公之拔邺,干降,遂以为刺史。干闻公讨乌丸,乃以州叛,执上党太守,举兵守壶关口。遣乐进、李典击之,干还守壶关城。十一年春正月,公征干。干闻之,乃留其别将守城,走入匈奴,求救于单于,单于不受。公围壶关三月,拔之。干遂走荆州,上洛都尉王琰捕斩之。
秋八月,公东征海贼管承,至淳于,遣乐进、李典击破之,承走入海岛。割东海之襄贲、郯、戚以益琅邪,省昌虑郡。[一]
注[一]魏书载十月乙亥令曰:“夫治世御众,建立辅弼,诫在面从,诗称‘听用我谋,庶无大悔’,斯实君臣恳恳之求也。吾充重任,每惧失中,频年已来,不闻嘉谋,岂吾开延不勤之咎邪?自今以后,诸掾属治中、别驾,常以月旦各言其失,吾将览焉。”
三郡乌丸承天下乱,破幽州,略有汉民合十余万户。袁绍皆立其酋豪为单于,以家人子为己女,妻焉。辽西单于蹋顿尤强,为绍所厚,故尚兄弟归之,数入塞为害。公将征之,凿渠,自呼扨入泒水,*泒音孤。*名平虏渠;又从泃河口*泃音句。*凿入潞河,名泉州渠,以通海。
十二月春二月,公自淳于还邺。丁酋,令曰:“吾起义兵诛暴乱,于今十九年,所征必克,岂吾功哉?乃贤士大夫之力也。天下虽未悉定,吾当要与贤士大夫共定之;而专飨其劳,吾何以安焉!其促定功行封。”于是大封功臣二十余人,皆为列侯,其余各以次受封,及复死事之孤,轻重各有差。[一]
注[一]魏书载公令曰:“昔赵奢、窦婴之为将也,受赐千金,一朝散之,故能济成大功,永世流声。吾读其文,未尝不慕其为人也。与诸将士大夫共从戎事,幸赖贤人不爱其谋,髃士不遗其力,是夷险平乱,而吾得窃大赏,户邑三万。追思窦婴散金之义,今分所受租与诸将掾属及故戍于陈、蔡者,庶以畴答众劳,不擅大惠也。宜差死事之孤,以租谷及之。若年殷用足,租奉毕入,将大与众人悉共飨之。”
将北征三郡乌丸,诸将皆曰:“袁尚,亡虏耳,夷狄贪而无亲,岂能为尚用?今深入征之,刘备必说刘表以袭许。万一为变,事不可悔。”惟郭嘉策表必不能任备,劝公行。夏五用,至无终。秋七月,大水,傍海道不通,田畴请为乡导,公从之。引军出卢龙塞,塞外道绝不通,乃堑山堙谷五百余里,经白檀,历平冈,涉鲜卑庭,东指柳城。未至二百里,虏乃知之。
尚、熙与蹋顿、辽西单于楼班、右北平单于能臣抵之等将数万骑逆军。八月,登白狼山,卒与虏遇,众甚盛。公车重在后,被甲者少,左右皆惧。公登高,望虏陈不整,乃纵兵击之,使张辽为先锋,虏众大崩,斩蹋顿及名王已下,胡、汉降者二十余万口。辽东单于速仆丸及辽西、北平诸豪,弃其种人,与尚、熙奔辽东,众尚有数千骑。初,辽东太守公孙康恃远不服。及公破乌丸,或说公遂征之,尚兄弟可禽也。公曰:“吾方使康斩送尚、熙首,不烦兵矣。”九月,公引兵自柳城还,[一]康即斩尚、熙及速仆丸等,传其首。诸将或问:“公还而康斩送尚、熙,何也?”公曰:“彼素畏尚等,吾急之则并力,缓之则自相图,其势然也。”十一月至易水,代郡乌丸行单于普富卢、上郡乌丸行单于那楼将其名王来贺。
注[一]曹瞒传曰:时寒且旱,二百里无复水,军又乏食,杀马数千匹以为粮,凿地入三十余丈乃得水。既还,科问前谏者,众莫知其故,人人皆惧。公皆厚赏之,曰:“孤前行,乘危以徼幸,虽得之,天所佐也,故不可以为常。诸君之谏,万安之计,是以相赏,后勿难言之。”
十三年春正月,公还邺,作玄武池以肄舟师。[一]汉罢三公官,置丞相、御史大夫。夏六月,以公为丞相。[二]
注[一]肄,以四反。三苍曰:“肄,习也。”
注[二]献帝起居注曰:使太常徐璆即授印绶。御史大夫不领中丞,置长史一人。先贤行状曰:
璆字*(孟平)**[孟玉]*,广陵人。少履清爽,立朝正色。历任城、汝南、东海三郡,所在化行。被征当还,为袁术所劫。术僭号,欲授以上公之位,璆终不为屈。术死后,璆得术玺,致之汉朝,拜韂尉太常;公为丞相,以位让璆焉。
秋七月,公南征刘表。八月,表卒,其子琮代,屯襄阳,刘备屯樊。九月,公到新野,琮遂降,备走夏口。公进军江陵,下令荆州吏民,与之更始。乃论荆州服从之功,侯者十五人,以刘表大将文聘为江夏太守,使统本兵,引用荆州名士韩嵩、邓义等。[一]益州牧刘璋始受征役,遣兵给军。十二月,孙权为备攻合肥。公自江陵征备,至巴丘,遣张□救合肥。权闻□至,乃走。公至赤壁,与备战,不利。于是大疫,吏士多死者,乃引军还。备遂有荆州、江南诸郡。[二]
注[一]韂恒四体书势序曰:上谷王次仲善隶书,始为楷法。至灵帝好书,世多能者。而师宜官为最,甚矜其能,每书,辄削焚其札。梁鹄乃益为版而饮之酒,候其醉而窃其札,鹄卒以攻书至选部尚书。于是公欲为洛阳令,鹄以为北部尉。鹄后依刘表。及荆州平,公募求鹄,鹄惧,自缚诣门,署军假司马,使在秘书,以*(勤)**[勒]*书自效。公尝悬着帐中,及以钉壁玩之,谓胜宜官。鹄字孟黄,安定人。魏宫殿题署,皆鹄书也。皇甫谧逸士传曰:汝南王鉨,字子文,少为范滂、许章所识,与南阳岑晊善。公之为布衣,特爱鉨;鉨亦称公有治世之具。及袁绍与弟术丧母,归葬汝南,鉨与公会之,会者三万人。公于外密语鉨曰:“天下将乱,为乱魁者必此二人也。欲济天下,为百姓请命,不先诛此二子,乱今作矣。”鉨曰:
“如卿之言,济天下者,舍卿复谁?”相对而笑。鉨为人外静而内明,不应州郡三府之命。
公车征,不到,避地居武陵,归鉨者一百余家。帝之都许,复征为尚书,又不就。刘表见绍强,阴与绍通,鉨谓表曰:“曹公,天下之雄也,必能兴霸道,继桓、文之功者也。今乃释近而就远,如有一朝之急,遥望漠北之救,不亦难乎!”表不从。鉨年六十四,以寿终于武陵,公闻而哀伤。及平荆州,自临江迎丧,改葬于江陵,表为先贤也。
注[二]山阳公载记曰:公船舰为备所烧,引军从华容道步归,遇泥泞,道不通,天又大风,悉使羸兵负草填之,骑乃得过。羸兵为人马所蹈藉,陷泥中,死者甚众。军既得出,公大喜,诸将问之,公曰:“刘备,吾俦也。但得计少晚;向使早放火,吾徒无类矣。”备寻亦放火而无所及。孙盛异同评曰:按吴志,刘备先破公军,然后权攻合肥,而此记云权先攻合肥,后有赤壁之事。二者不同,吴志为是。
十四年春三月,军至谯,作轻舟,治水军。秋七月,自涡入淮,出肥水,军合肥。辛未,令曰:“自顷已来,军数征行,或遇疫气,吏士死亡不归,家室怨旷,百姓流离,而仁者岂乐之哉?不得已也。其令死者家无基业不能自存者,县官勿绝廪,长吏存恤抚循,以称吾意。”
置扬州郡县长吏,开芍陂屯田。十二月,军还谯。
十五年春,下令曰:“自古受命及中兴之君,曷尝不得贤人君子与之共治天下者乎!及其得贤也,曾不出闾巷,岂幸相遇哉?上之人不求之耳。今天下尚未定,此特求贤之急时也。‘孟公绰为赵、魏老则优,不可以为滕、薛大夫’。若必廉士而后可用,则齐桓其何以霸世!今天下得无有被褐怀玉而钓于渭滨者乎?又得无盗嫂受金而未遇无知者乎?二三子其佐我明扬仄陋,唯才是举,吾得而用之。”冬,作铜雀台。[一]
注[一]魏武故事载公十二月己亥令曰:“孤始举孝廉,年少,自以本非岩穴知名之士,恐为海内人之所见凡愚,欲为一郡守,好作政教,以建立名誉,使世士明知之;故在济南,始除残去秽,平心选举,违迕诸常侍。以为强豪所忿,恐致家祸,故以病还。去官之后,年纪尚少,顾视同岁中,年有五十,未名为老,内自图之,从此却去二十年,待天下清,乃与同岁中始举者等耳。故以四时归乡里,于谯东五十里筑精舍,欲秋夏读书,冬春射猎,求底下之地,欲以泥水自蔽,绝宾客往来之望,然不能得如意。后征为都尉,迁典军校尉,意遂更欲为国家讨贼立功,欲望封侯作征西将军,然后题墓道言‘汉故征西将军曹侯之墓’,此其志也。而遭值董卓之难,兴举义兵。是时合兵能多得耳,然常自损,不欲多之;所以然者,多兵意盛,与强敌争,倘更为祸始。故汴水之战数千,后还到扬州更募,亦复不过三千人,此其本志有限也。后领兖州,破降黄巾三十万众。又袁术僭号于九江,下皆称臣,名门曰建号门,衣被皆为天子之制,两妇预争为皇后。志计已定,人有劝术使遂即帝位,露布天下,答言‘曹公尚在,未可也’。后孤讨禽其四将,获其人众,遂使术穷亡解沮,发病而死。及至袁绍据河北,兵势强盛,孤自度势,实不敌之,但计投死为国,以义灭身,足垂于后。幸而破绍,枭其二子。又刘表自以为宗室,包藏奸心,乍前乍却,以观世事,据有当州,孤复定之,遂平天下。身为宰相,人臣之贵已极,意望已过矣。
今孤言此,若为自大,欲人言尽,故无讳耳。设使国家无有孤,不知当几人称帝,几人称王。
或者人见孤强盛,又性不信天命之事,恐私心相评,言有不逊之志,妄相忖度,每用耿耿。
齐桓、晋文所以垂称至今日者,以其兵势广大,犹能奉事周室也。论语云‘三分天下有其二,以服事殷,周之德可谓至德矣’,夫能以大事小也。昔乐毅走赵,赵王欲与之图燕,乐毅伏而垂泣,对曰:‘臣事昭王,犹事天王;臣若获戾,放在他国,没世然后已,不忍谋赵之徒隶,况燕后嗣乎!’胡亥之杀蒙恬也,恬曰:‘自吾先人及至子孙,积信于秦三世矣;今臣将兵三十余万,其势足以背叛,然自知必死而守义者,不敢辱先人之教以忘先王也。’孤每读此二人书,未尝不怆然流涕也。孤祖父以至孤身,皆当亲重之任,可谓见信者矣,以及*(子植)**[子桓]*兄弟,过于三世矣。孤非徒对诸君说此也,常以语妻妾,皆令深知此意。孤谓之言:‘顾我万年之后,汝曹皆当出嫁,欲令传道我心,使他人皆知之。’孤此言皆肝鬲之要也。所以勤勤恳恳□心腹者,见周公有金縢之书以自明,恐人不信之故。然欲孤便尔委捐所典兵众以还执事,归就武平侯国,实不可也。何者?诚恐己离兵为人所祸也。既为子孙计,又己败则国家倾危,是以不得慕虚名而处实祸,此所不得为也。前朝恩封三子为侯,固辞不受,今更欲受之,非欲复以为荣,欲以为外援,为万安计。孤闻介推之避晋封。申胥之逃楚赏,未尝不舍书而叹,有以自省也。奉国威灵,仗钺征伐,推弱以克强,处小而禽大,意之所图,动无违事,心之所虑,何向不济,遂荡平天下,不辱主命,可谓天助汉室,非人力也。然封兼四县,食户三万,何德堪之!江湖未静,不可让位;至于邑土,可得而辞。今上还阳夏、柘、苦三县户二万,但食武平万户,且以分损谤议,少减孤之责也。”
十六年春正月,[一]天子命公世子丕为五官中郎将,置官属,为丞相副。太原商曜等以大陵叛,遣夏侯渊、徐晃围破之。张鲁据汉中,三月,遣钟繇讨之。公使渊等出河东与繇会。
注[一]魏书曰:庚辰,天子报:减户五千,分所让三县万五千封三子,植为平原侯,据为范阳侯,豹为饶阳侯,食邑各五千户。
是时关中诸将疑繇欲自袭,马超遂与韩遂、杨秋、李堪、成宜等叛。遣曹仁讨之。超等屯潼关,公敕诸将:“关西兵精悍,坚壁勿与战。”秋七月,公西征,[一]与超等夹关而军。公急持之,而潜遣徐晃、朱灵等夜渡蒲阪津,据河西为营。公自潼关北渡,未济,超赴船急战。
校尉丁斐因放牛马以饵贼,贼乱取牛马,公乃得渡,[二]循河为甬道而南。贼退,拒渭口,公乃多设疑兵,潜以舟载兵入渭,为浮桥,夜,分兵结营于渭南。贼夜攻营,伏兵击破之。
超等屯渭南,遣信求割河以西请和,公不许。九月,进军渡渭。[三]超等数挑战,又不许;
固请割地,求送任子,公用贾诩计,伪许之。韩遂请与公相见,公与遂父同岁孝廉,又与遂同时侪辈,于是交马语移时,不及军事,但说京都旧故,拊手欢笑。既罢,超等问遂:“公何言?”
遂曰:“无所言也。”超等疑之。[四]他日,公又与遂书,多所点窜,如遂改定者;超等愈疑遂。公乃与克日会战,先以轻兵挑之,战良久,乃纵虎骑夹击,大破之,斩成宜、李堪等。
遂、超等走凉州,杨秋奔安定,关中平。诸将或问公曰:“初,贼守潼关,渭北道缺,不从河东击冯翊而反守潼关,引日而后北渡,何也?”公曰:“贼守潼关,若吾入河东,贼必引守诸津,则西河未可渡,吾故盛兵向潼关;贼悉众南守,西河之备虚,故二将得擅取西河;
然后引军北渡,贼不能与吾争西河者,以有二将之军也。连车树栅,为甬道而南,[五]既为不可胜,且以示弱。渡渭为坚垒,虏至不出,所以骄之也;故贼不为营垒而求割地。吾顺言许之,所以从其意,使自安而不为备,因畜士卒之力,一旦击之,所谓疾雷不及掩耳,兵之变化,固非一道也。”始,贼每一部到,公辄有喜色。贼破之后,诸将问其故。公答曰:“关中长远,若贼各依险阻,征之,不一二年不可定也。今皆来集,其众虽多,莫相归服,军无适主,一举可灭,为功差易,吾是以喜。”
注[一]魏书曰:议者多言“关西兵强,习长矛,非精选前锋,则不可以当也”。公谓诸将曰:
“战在我,非在贼也。贼虽习长矛,将使不得以刺,诸君但观之耳。”
注[二]曹瞒传曰:公将过河,前队适渡,超等奄至,公犹坐胡黙不起。张合等见事急,共引公入船。河水急,比渡,流四五里,超等骑追射之,矢下如雨。诸将见军败,不知公所在,皆惶惧,至见,乃悲喜,或流涕。公大笑曰:“今日几为小贼所困乎!”
注[三]曹瞒传曰:时公军每渡渭,辄为超骑所冲突,营不得立,地又多沙,不可筑垒。娄子伯说公曰:“今天寒,可起沙为城,以水灌之,可一夜而成。”公从之,乃多作缣囊以运水,夜渡兵作城,比明,城立,由是公军尽得渡渭。或疑于时九月,水未应冻。臣松之按魏书:
公军八月至潼关,闰月北渡河,则其年闰八月也,至此容可大寒邪!
注[四]魏书曰:公后日复与遂等会语,诸将曰:“公与虏交语,不宜轻脱,可为木行马以为防遏。”公然之。贼将见公,悉于马上拜,秦、胡观者,前后重沓,公笑谓贼曰:“汝欲观曹公邪?亦犹人也,非有四目两口,但多智耳!”胡前后大观。又列铁骑五千为十重陈,精光耀日,贼益震惧。
注[五]臣松之案:汉高祖二年,与楚战荥阳京、索之间,筑甬道属河以取敖仓粟。应劭曰:
“恐敌钞辎重,故筑垣墙如街巷也。”今魏武不筑垣墙,但连车树栅以扞两面。
冬十月,军自长安北征杨秋,围安定。秋降,复其爵位,使留抚其民人。[一]十二月,自安定还,留夏侯渊屯长安。
注[一]魏略曰:杨秋,黄初中迁讨寇将军,位特进,封临泾侯,以寿终。
十七年春正月,公还邺。天子命公赞拜不名,入朝不趋,剑履上殿,如萧何故事。马超余众梁兴等屯蓝田,使夏侯渊击平之。割河内之荡阴、朝歌、林虑,东郡之卫国、顿丘、东武阳、发干,钜鹿之廮陶、曲周、南和,广平之任城,赵之襄国、邯郸、易阳以益魏郡。
冬十月,公征孙权。
十八年春正月,进军濡须口,攻破权江西营,获权都督公孙阳,乃引军还。诏书并十四州,复为九州。夏四月,至邺。
五月丙申,天子使御史大夫郗虑持节策命公为魏公[一]曰: Us朕以不德,少遭愍凶,越在西土,迁于唐、韂。当此之时,若缀旒然,[二]宗庙乏祀,社稷无位;髃凶觊觎,分裂诸夏,率土之民,朕无获焉,即我高祖之命将坠于地。朕用夙兴假寐,震悼于厥心,曰“惟祖惟父,股肱先正,[三]其孰能恤朕躬”?乃诱天衷,诞育丞相,保乂我皇家,弘济于艰难,朕实赖之。今将授君典礼,其敬听朕命。 Us昔者董卓初兴国难,髃后释位以谋王室,[四]君则摄进,首启戎行,此君之忠于本朝也。后及黄巾反易天常,侵我三州,延及平民,君又翦之以宁东夏,此又君之功也。韩暹、杨奉专用威命,君则致讨,克黜其难,遂迁许都,造我京畿,设官兆祀,不失旧物,天地鬼神于是获乂,此又君之功也。袁术僭逆,肆于淮南,慑惮君灵,用丕显谋,蕲阳之役,桥蕤授首,棱威南迈,术以陨溃,此又君之功也。回戈东征,吕布就戮,乘辕将返,张杨殂毙,眭固伏罪,张绣稽服,此又君之功也。袁绍逆乱天常,谋危社稷,凭恃其众,称兵内侮,当此之时,王师寡弱,天下寒心,莫有固志,君执大节,精贯白日,奋其武怒,运其神策,致届官渡,大歼丑类,[五]俾我国家拯于危坠,此又君之功也。济师洪河,拓定四州,袁谭、高干,咸枭其首,海盗奔迸,黑山顺轨,此又君之功也。乌丸三种,崇乱二世,袁尚因之,逼据塞北,束马县车,一征而灭,此又君之功也。刘表背诞,不供贡职,王师首路,威风先逝,百城八郡,交臂屈膝,此又君之功也。马超、成宜,同恶相济,滨据河、潼,求逞所欲,殄之渭南,献馘万计,遂定边境,抚和戎狄,此又君之功也。鲜卑、丁零,重译而至,*(单于)**[箄于]*、白屋,请吏率职,此又君之功也。君有定天下之功,重之以明德,班□海内,宣美风俗,旁施勤教,恤慎刑狱,吏无苛政,民无怀慝;敦崇帝族,表继绝世,旧德前功,罔不咸秩;虽伊尹格于皇天,周公光于四海,方之蔑如也。
Us朕闻先王并建明德,胙之以土,分之以民,崇其宠章,备其礼物,所以藩韂王室,左右厥世也。
其在周成,管、蔡不静,惩难念功,乃使邵康公赐齐太公履,东至于海,西至于河,南至于穆陵,北至于无棣,五侯九伯,实得征之,世祚太师,以表东海;爰及襄王,亦有楚人不供王职,又命晋文登为侯伯,锡以二辂、虎贲、鈇钺、秬鬯、弓矢,大启南阳,世作盟主。故周室之不坏,繄二国是赖。今君称丕显德,明保朕躬,奉答天命,导扬弘烈,缓爰九域,莫不率俾,[六]功高于伊、周,而赏卑于齐、晋,朕甚恧焉。朕以眇眇之身,托于兆民之上,永思厥艰,若涉渊冰,非君攸济,朕无任焉。今以冀州之河东、河内、魏郡、赵国、中山、常山、钜鹿、安平、甘陵、平原凡十郡,封君为魏公。锡君玄土,苴以白茅;爰契尔龟,用建頉社。昔在周室,毕公、毛公入为卿佐,周、邵师保出为二伯,外内之任,君实宜之,其以丞相领冀州牧如故。又加君九锡,其敬听朕命。以君经纬礼律,为民轨仪,使安职业,无或迁志,是用锡君大辂、戎辂各一,玄牡二驷。君劝分务本,穑人昏作,[七]粟帛滞积,大业惟兴,是用锡君衮冕之服,赤舄副焉。君敦尚谦让,俾民兴行,少长有礼,上下咸和,是用锡君轩县之乐,六佾之舞。君翼宣风化,爰发四方,远人革面,华夏充实,是用锡君朱户以居。君研其明哲,思帝所难,官才任贤,髃善必举,是用锡君纳陛以登。君秉国之钧,正色处中,纤毫之恶,靡不抑退,是用锡君虎贲之士三百人。君纠虔天刑,章厥有罪,[八]犯关干纪,莫不诛殛,是用锡君鈇钺各一。君龙骧虎视,旁眺八维,掩讨逆节,折冲四海,是用锡君彤弓一,彤矢百,玈弓十,玈矢千。君以温恭为基,孝友为德,明允笃诚,感于朕思,是用锡君秬鬯一卣,珪瓒副焉。魏国置丞相已下髃卿百寮,皆如汉初诸侯王之制。往钦哉,敬服朕命!简恤尔众,时亮庶功,用终尔显德,对扬我高祖之休命![九]
注[一]续汉书曰:虑字鸿豫,山阳高平人。少受业于郑玄,建安初为侍中。虞溥江表传曰:
献帝尝特见虑及少府孔融,问融曰:“鸿豫何所优长?”融曰:“可与适道,未可与权。”虑举笏曰:
“融昔宰北海,政散民流,其权安在也!”遂与融互相长短,以至不睦。公以书和解之。虑从光禄勋迁为大夫。
注[二]公羊传曰:“君若赘旒然。”何休云:“赘犹缀也。旒,旗旒也。以旒譬者,言为下所执持东西也。”
注[三]文侯之命曰:“亦惟先正。”郑玄云:“先正,先臣。谓公卿大夫也。”
注[四]左氏传曰:“诸侯释位以闲王政。”服虔曰:“言诸侯释其私政而佐王室。”
注[五]诗曰:“致天之届,于牧之野。”郑玄云:“届,极也。”鸿范曰:“鲧则殛死。”
注[六]盘庚曰:“绥爰有众。”郑玄曰:“爰,于也,安隐于其众也。”君奭曰:“海隅出日,罔不率俾。”率,循也。俾,使也。四海之隅,日出所照,无不循度而可使也。
注[七]盘庚曰:“堕农自安,不昏作劳。”郑玄云:“昏,勉也。”
注[八]“纠虔天刑”语出国语,韦昭注曰:“纠,察也。虔,敬也。刑,法也。”
注[九]后汉尚书左丞潘勖之辞也。勖字符茂,陈留中牟人。魏书载公令曰:“夫受九锡,广开土宇,周公其人也。汉之异姓八王者,与高祖俱起布衣,□定王业,其功至大,吾何可比之?”前后三让。于是中军师*(王)*陆树亭侯荀攸、前军师东武亭侯钟繇、左军师凉茂、右军师毛玠、平虏将军华乡侯刘勋、建武将军清苑亭侯刘若、伏波将军高安侯夏侯惇、扬武将军都亭侯王忠、奋威将军乐乡侯刘展、建忠将军昌乡亭侯鲜于辅、奋武将军安国亭侯程昱、太中大夫都乡侯贾诩、军师祭酒千秋亭侯董昭、都亭侯薛洪、南乡亭侯董蒙、关内侯王粲、傅巽、祭酒王选、袁涣、王朗、张承、任藩、杜袭、中护军国明亭侯曹洪、中领军万岁亭侯韩浩、行骁骑将军安平亭侯曹仁、领护军将军王图、长史万潜、谢奂、袁霸等劝进曰:“自古三代,胙臣以土,受命中兴,封秩辅佐,皆所以褒功赏德,为国藩韂也。往者天下崩乱,髃凶豪起,颠越跋扈之险,不可忍言。明公奋身出命以徇其难,诛二袁篡盗之逆,灭黄巾贼乱之类,殄夷首逆,芟拨荒秽,沐浴霜露二十余年,书契已来,未有若此功者。昔周公承文、武之夡,受已成之业,高枕墨笔,拱揖髃后,商、奄之勤,不过二年,吕望因三分有二之形,据八百诸侯之势,暂把旄钺,一时指麾,然皆大启土宇,跨州兼国。周公八子,并为侯伯,白牡骍刚,郊祀天地,典策备物,拟则王室,荣章宠盛如此之弘也。逮至汉兴,佐命之臣,张耳、吴芮,其功至薄,亦连城开地,南面称孤。此皆明君达主行之于上,贤臣圣宰受之于下,三代令典,汉帝明制。今比劳则周、吕逸,计功则张、吴微,论制则齐、鲁重,言地则长沙多;然则魏国之封,九锡之荣,况于旧赏,犹怀玉而被褐也。且列侯诸将,幸攀龙骥,得窃微劳,佩紫怀黄,盖以百数,亦将因此传之万世,而明公独辞赏于上,将使其下怀不自安,上违圣朝欢心,下失冠带至望,忘辅弼之大业,信匹夫之细行,攸等所大惧也。”于是公敕外为章,但受魏郡。攸等复曰:“伏见魏国初封,圣朝发虑,稽谋髃寮,然后策命;而明公久违上指,不即大礼。今既虔奉诏命,副顺众望,又欲辞多当少,让九受一,是犹汉朝之赏不行,而攸等之请未许也。昔齐、鲁之封,奄有东海,疆域井赋,四百万家,基隆业广,易以立功,故能成翼戴之勋,立一匡之绩。今魏国虽有十郡之名,犹减于曲阜,计其户数,不能参半,以藩韂王室,立垣树屏,犹未足也。且圣上览亡秦无辅之祸,惩曩日震荡之艰,托建忠贤,废坠是为,愿明公恭承帝命,无或拒违。”公乃受命。魏略载公上书谢曰:“臣蒙先帝厚恩,致位郎署,受性疲怠,意望毕足,非敢希望高位,庶几显达。会董卓作乱,义当死难,故敢奋身出命,摧锋率众,遂值千载之运,奉役目下。当二袁炎沸侵侮之际,陛下与臣寒心同忧,顾瞻京师,进受猛敌,常恐君臣俱陷虎口,诚不自意能全首领。赖祖宗灵佑,丑类夷灭,得使微臣窃名其间。陛下加恩,授以上相,封爵宠禄,丰大弘厚,生平之愿,实不望也。口与心计,幸且待罪,保持列侯,遗付子孙,自托圣世,永无忧责。不意陛下乃发盛意,开国备锡,以贶愚臣,地比齐、鲁,礼同藩王,非臣无功所宜膺据。归情上闻,不蒙听许,严诏切至,诚使臣心俯仰逼迫。伏自惟省,列在大臣,命制王室,身非己有,岂敢自私,遂其愚意,亦将黜退,令就初服。今奉疆土,备数藩翰,非敢远期,虑有后世;至于父子相誓终身,灰躯尽命,报塞厚恩。天威在颜,悚惧受诏。”
秋七月,始建魏社稷宗庙。天子聘公三女为贵人,少者待年于国。[一]九月,作金虎台,凿渠引漳水入白沟以通河。冬十月,分魏郡为东西部,置都尉。十一月,初置尚书、侍中、六卿。[二]
注[一]献帝起居注曰:使使持节行太常大司农安阳亭侯王邑,赍璧、帛、玄纁、绢五万匹之邺纳聘,介者五人,皆以议郎行大夫事,副介一人。
注[二]魏氏春秋曰:以荀攸为尚书令,凉茂为仆射,毛玠、崔琰、常林、徐奕、何夔为尚书,王粲、杜袭、韂觊、和洽为侍中。
马超在汉阳,复因羌、胡为害,氐王千万叛应超,屯兴国。使夏侯渊讨之。
十九年春正月,始耕籍田。南安赵衢、汉阳尹奉等讨超,枭其妻子,超奔汉中。韩遂徙金城,入氐王千万部,率羌、胡万余骑与夏侯渊战,击,大破之,遂走西平。渊与诸将攻兴国,屠之。省安东、永阳郡。
安定太守□丘兴将之官,公戒之曰:“羌,胡欲与中国通,自当遣人来,慎勿遣人往。善人难得,必将教羌、胡妄有所请求,因欲以自利;不从便为失异俗意,从之则无益事。”兴至,遣校尉范陵至羌中,陵果教羌,使自请为属国都尉。公曰:“吾预知当尔,非圣也,但更事多耳。”[一]
注[一]献帝起居注曰:使行太常事大司农安阳亭侯王邑与宗正刘艾,皆持节,介者五人,赍束帛驷马,及给事黄门侍郎、掖庭丞、中常侍二人,迎二贵人于魏公国。二月癸亥,又于魏公宗庙授二贵人印绶。甲子,诣魏公宫延秋门,迎贵人升车。魏遣郎中令、少府、博士、御府乘黄厩令、丞相掾属侍送贵人。癸酉,二贵人至洧仓中,遣侍中丹将冗从虎贲前后骆驿往迎之。乙亥,二贵人入宫,御史大夫、中二千石将大夫、议郎会殿中,魏国二卿及侍中、中郎二人,与汉公卿并升殿宴。
三月,天子使魏公位在诸侯王上,改授金玺,赤绂、远游冠。[一]
注[一]献帝起居注曰:使左中郎将杨宣、亭侯裴茂持节、印授之。
秋七月,公征孙权。[一]
注[一]九州春秋曰:参军傅干谏曰:“治天下之大具有二,文与武也;用武则先威,用文则先德,威德足以相济,而后王道备矣。往者天下大乱,上下失序,明公用武攘之,十平其九。
今未承王命者,吴与蜀也,吴有长江之险,蜀有崇山之阻,难以威服,易以德怀。愚以为可且按甲寝兵,息军养士,分土定封,论功行赏,若此则内外之心固,有功者劝,而天下知制矣。然后渐兴学校,以导其善性而长其义节。公神武震于四海,若修文以济之,则普天之下,无思不服矣。今举十万之众,顿之长江之滨,若贼负固深藏,则士马不能逞其能,奇变无所用其权,则大威有屈而敌心未能服矣。唯明公思虞舜舞干戚之义,全威养德,以道制胜。”公不从,军遂无功。
干字彦材,北地人,终于丞相仓曹属。有子曰玄。
初,陇西宋建自称河首平汉王,聚众枹罕,改元,置百官,三十余年。遣夏侯渊自兴国讨之。
冬十月,屠枹罕,斩建,凉州平。
公自合肥还。
十一月,汉皇后伏氏坐昔与父故屯骑校尉完书,云帝以董承被诛怨恨公,辞甚丑恶,发闻,后废黜死,兄弟皆伏法。[一]
注[一]曹瞒传曰:公遣华歆勒兵入宫收后,后闭户匿壁中。歆坏户发壁,牵后出。帝时与御史大夫郗虑坐,后被发徒跣过,执帝手曰:“不能复相活邪?”帝曰:“我亦不自知命在何时也。”帝谓虑曰:“郗公,天下宁有是邪!”遂将后杀之,完及宗族死者数百人。
十二月,公至孟津。天子命公置旄头,宫殿设钟虡。乙未,令曰:“夫有行之士未必能进取,进取之士未必能有行也。陈平岂笃行,苏秦岂守信邪?而陈平定汉业,苏秦济弱燕。由此言之,士有偏短,庸可废乎!有司明思此义,则士无遗滞,官无废业矣。”又曰:“夫刑,百姓之命也,而军中典狱者或非其人,而任以三军死生之事,吾甚惧之。其选明达法理者,使持典刑。”于是置理曹掾属。
二十年春正月,天子立公中女为皇后。省云中、定襄、五原、朔方郡,郡置一县领其民,合以为新兴郡。
三月,公西征张鲁,至陈仓,将自武都入氐;氐人塞道,先遣张合、朱灵等攻破之。夏四月,公自陈仓以出散关,至河池。氐王窦茂众万余人,恃险不服,五月,公攻屠之。西平、金城诸将曲演、蒋石等共斩送韩遂首。[一]秋七月,公至阳平。张鲁使弟韂与将杨昂等据阳平关,横山筑城十余里,攻之不能拔,乃引军还。贼见大军退,其守备解散。公乃密遣解□、高祚等乘险夜袭,大破之,斩其将杨任,进攻韂,韂等夜遁,鲁溃奔巴中。公军入南郑,尽得鲁府库珍宝。[二]巴、汉皆降。复汉宁郡为汉中;分汉中之安阳、西城为西城郡,置太守;分锡、上庸郡,置都尉。
注[一]典略曰:遂字文约,始与同郡边章俱著名西州。章为督军从事。遂奉计诣京师,何进宿闻其名,特与相见,遂说进使诛诸阉人,进不从,乃求归。会凉州宋扬、北宫玉等反,举章、遂为主,章寻病卒,遂为扬等所劫,不得已,遂阻兵为乱,积三十二年,至是乃死,年七十余矣。刘艾灵帝纪曰:章,一名*(元)**[允]*。
注[二]魏书曰:军自武都山行千里,升降险阻,军人劳苦;公于是大飨,莫不忘其劳。
八月,孙权围合肥,张辽、李典击破之。
九月,巴七姓夷王朴胡、賨邑侯杜濩举巴夷、賨民来附,[一]于是分巴郡,以胡为巴东太守,濩为巴西太守,皆封列侯。天子命公承制封拜诸侯守相。[二]
注[一]孙盛曰:朴音浮。濩音户。
注[二]孔衍汉魏春秋曰:天子以公典任于外,临事之赏,或宜速疾,乃命公得承制封拜诸侯守相,诏曰:“夫军之大事,在兹赏罚,劝善惩恶,宜不旋时,故司马法曰‘赏不逾日’者,欲民速鷪为善之利也。昔在中兴,邓禹入关,承制拜军祭酒李文为河东太守,来歙又承制拜高峻为通路将军,察其本传,皆非先请,明临事刻印也,斯则世祖神明,权达损益,盖所用速示威怀而着鸿勋也。其春秋之义,大夫出疆,有专命之事,苟所以利社稷安国家而已。况君秉任二伯,师尹九有,实征夷夏,军行藩甸之外,失得在于斯须之间,停赏俟诏以滞世务,固非朕之所图也。自今已后,临事所甄,当加宠号者,其便刻印章假授,咸使忠义得相銟励,勿有疑焉。”
冬十月,始置名号侯至五大夫,与旧列侯、关内侯凡六等,以赏军功。[一]
注[一]魏书曰:置名号侯爵十八级,关中侯爵十七级,皆金印紫绶;又置关内外侯十六级,铜印龟纽墨绶;五大夫十五级,铜印环纽,亦墨绶,皆不食租,与旧列侯关内侯凡六等。臣松之以为今之虚封盖自此始。
十一月,鲁自巴中将其余众降。封鲁及五子皆为列侯。刘备袭刘璋,取益州,遂据巴中;遣张合击之。
十二月,公自南郑还,留夏侯渊屯汉中。[一]
注[一]是行也,侍中王粲作五言诗以美其事曰:“从军有苦乐,但问所从谁。所从神且武,安得久劳师?相公征关右,赫怒振天威,一举灭獯虏,再举服羌夷,西收边地贼,忽若俯拾遗。陈赏越山岳,酒肉踰川坻,军中多饶饫,人马皆溢肥,徒行兼乘还,空出有余资。拓土三千里,往反速如飞,歌舞入邺城,所愿获无违。”
二十一年春二月,公还邺。[一]三月壬寅,公亲耕籍田。[二]夏五月,天子进公爵为魏王。
[三]代郡乌丸行单于普富卢与其侯王来朝。天子命王女为公主,食汤沐邑。秋七月,匈奴南单于呼厨泉将其名王来朝,待以客礼,遂留魏,使右贤王去卑监其国。八月,以大理钟繇为相国。[四]
注[一]魏书曰:辛未,有司以太牢告至,策勋于庙,甲午始春祠,令曰:“议者以为祠庙上殿当解履。吾受锡命,带剑不解履上殿。今有事于庙而解履,是尊先公而替王命,敬父祖而简君主,故吾不敢解履上殿也。又临祭就洗,以手拟水而不盥。夫盥以洁为敬,未闻拟*(向)**[而]*不盥之礼,且‘祭神如神在’,故吾亲受水而盥也。又降神礼讫,下阶就幕而立,须奏乐毕竟,似若不*(愆)**[衎]*烈祖,迟祭*(不)*速讫也,故吾坐俟乐阕送神乃起也。
受胙纳*(神)**[袖]*,以授侍中,此为敬恭不终实也,古者亲执祭事,故吾亲纳于*(神)**[袖]*,终抱而归也。仲尼曰‘虽违众,吾从下’,诚哉斯言也。”
注[二]魏书曰:有司奏:“四时讲武于农隙。汉承秦制,三时不讲,唯十月都试车马,幸长水南门,会五营士为八陈进退,名曰乘之。今金革未偃,士民素习,自今已后,可无四时讲武,但以立秋择吉日大朝车骑,号曰治兵,上合礼名,下承汉制。”奏可。
注[三]献帝传载诏曰:“自古帝王,虽号称相变,爵等不同,至乎褒崇元勋,建立功德,光启氏姓,延于子孙,庶姓之与亲,岂有殊焉。昔我圣祖受命,□业肇基,造我区夏,鉴古今之制,通爵等之差,尽封山川以立藩屏,使异姓亲戚,并列土地,据国而王,所以保乂天命,安固万嗣。历世承平,臣主无事。世祖中兴而时有难易,是以旷年数百,无异姓诸侯王之位。
朕以不德,继序弘业,遭率土分崩,髃凶纵毒,自西徂东,辛苦卑约。当此之际,唯恐溺入于难,以羞先帝之圣德。赖皇天之灵,俾君秉义奋身,震迅神武,捍朕于艰难,获保宗庙,华夏遗民,含气之伦,莫不蒙焉。君勤过稷、禹,忠侔伊、周,而掩之以谦让,守之以弥恭,是以往者初开魏国,锡君土宇,惧君之违命,虑君之固辞,故且怀志屈意,封君为上公,欲以钦顺高义,须俟勋绩。韩遂、宋建,南结巴、蜀,髃逆合从,图危社稷,君复命将,龙骧虎奋,枭其元首,屠其窟栖。暨至西征,阳平之役,亲擐甲冑,深入险阻,芟夷蝥贼,殄其凶丑,荡定西陲,悬旌万里,声教远振,宁我区夏。盖唐、虞之盛,三后树功,文、武之兴,旦、奭作辅,二祖成业,英豪佐命;夫以圣哲之君,事为己任,犹锡土班瑞以报功臣,岂有如朕寡德,仗君以济,而赏典不丰,将何以答神只慰万方哉?今进君爵为魏王,使使持节行御史大夫、宗正刘艾奉策玺玄土之社,苴以白茅,金虎符第一至第五,竹使符第一至十。君其正王位,以丞相领冀州牧如故。其上魏公玺绶符册。敬服朕命,简恤尔众,克绥庶绩,以扬我祖宗之休命。”魏王上书三辞,诏三报不许。又手诏曰:“大圣以功德为高美,以忠和为典训,故□业垂名,使百世可希,行道制义,使力行可效,是以勋烈无穷,休光茂着。稷、契载元首之聪明,周、邵因文、武之智用,虽经营庶官,仰叹俯思,其对岂有若君者哉?朕惟古人之功,美之如彼,思君忠勤之绩,茂之如此,是以每将镂符析瑞,陈礼命册,寤寐慨然,自忘守文之不德焉。今君重违朕命,固辞恳切,非所以称朕心而训后世也。其抑志撙节,勿复固辞。”四体书势序曰:梁鹄以公为北部尉。
曹瞒传曰:为尚书右丞司马建公所举。及公为王,召建公到邺,与欢饮,谓建公曰:“孤今日可复作尉否?”建公曰:“昔举大王时,适可作尉耳。”王大笑。建公名防,司马宣王之父。臣松之案司马彪序传,建公不为右丞,疑此不然,而王隐晋书云赵王篡位,欲尊祖为帝,博士马平议称京兆府君昔举魏武帝为北部尉,贼不犯界,如此则为有征。
注[四]魏书曰:始置奉常宗正官。
冬十月,治兵,[一]遂征孙权,十一月至谯。
注[一]魏书曰:王亲执金鼓以令进退。
二十二年春正月,王军居巢,二月,进军屯江西郝溪。权在濡须口筑城拒守,遂逼攻之,权退走。三月,王引军还,留夏侯惇、曹仁、张辽等屯居巢。
夏四月,天子命王设天子旌旗,出入称警跸。五月,作泮宫。六月,以军师华歆为御史大夫。
[一]冬十月,天子命王冕十有二旒,乘金根车,驾六马,设五时副车,以五官中郎将丕为魏太子。
注[一]魏书曰:初置韂尉官。秋八月,令曰:“昔伊挚、傅说出于贱人,管仲,桓公贼也,皆用之以兴。萧何、曹参,县吏也,韩、陈平负污辱之名,有见笑之耻,卒能成就王业,声着千载。吴起贪将,杀妻自信,散金求官,母死不归,然在魏,奏人不敢东向,在楚则三晋不敢南谋。今天下得无有至德之人放在民间,及果勇不顾,临敌力战;若文俗之吏,高才异质,或堪为将守;负污辱之名,见笑之行,或不仁不孝而有治国用兵之术:其各举所知,勿有所遗。”
刘备遣张飞、马超、吴兰等屯下辩;遣曹洪拒之。
二十三年春正月,汉太医令吉本与少府耿纪、司直韦晃等反,攻许,烧丞相长史王必营,[一]必与颍川典农中郎将严匡讨斩之。[二]
注[一]魏武故事载令曰:“领长史王必,是吾披荆棘时吏也。忠能勤事,心如铁石,国之良吏也。蹉跌久未辟之,舍骐骥而弗乘,焉遑遑而更求哉?故教辟之,已署所宜,便以领长史统事如故。”
注[二]三辅决录注曰:时有京兆金祎字德祎,自以世为汉臣,自日磾讨莽何罗,忠诚显著,名节累叶。鷪汉祚将移,谓可季兴,乃喟然发愤,遂与耿纪、韦晃、吉本、本子邈、邈弟穆等结谋。纪字季行,少有美名,为丞相掾,王甚敬异之,迁侍中,守少府。邈字文然,穆字思然,以祎慷慨有日磾之风,又与王必善,因以闲之,若杀必,欲挟天子以攻魏,南援刘备。
时关羽强盛,而王在邺,留必典兵督许中事。文然等率杂人及家僮千余人夜烧门攻必,祎遣人为内应,射必中肩。必不知攻者为谁,以素与祎善,走投祎,夜唤德祎,祎家不知是必,谓为文然等,错应曰:“王长史已死乎?卿曹事立矣!”必乃更他路奔。一曰:必欲投祎,其帐下督谓必曰:“今日事竟知谁门而投入乎?”扶必奔南城。会天明,必犹在,文然等众散,故败。后十余日,必竟以创死。献帝春秋曰:收纪、晃等,将斩之,纪呼魏王名曰:“恨吾不自生意,竟为髃儿所误耳!”晃顿首搏颊,以至于死。山阳公载记曰:王闻王必死,盛怒,召汉百官诣邺,令救火者左,不救火者右。众人以为救火者必无罪,皆附左;王以为“不救火者非助乱,救火乃实贼也”。皆杀之。
曹洪破吴兰,斩其将任夔等。三月,张飞、马超走汉中,阴平氐强端斩吴兰,传其首。
夏四月,代郡、上谷乌丸无臣氐等叛,遣鄢陵侯彰讨破之。[一]
注[一]魏书载王令曰:“去冬天降疫疠,民有凋伤,军兴于外,垦田损少,吾甚忧之。其令吏民男女:女年七十已上无夫子,若年十二已下无父母兄弟,及目无所见,手不能作,足不能行,而无妻子父兄产业者,廪食终身。幼者至十二止,贫穷不能自赡者,随口给贷。老耄须待养者,年九十已上,复不事,家一人。”
六月,令曰:“古之葬者,必居瘠薄之地。其规西门豹祠西原上为寿陵,因高为基,不封不树。周礼頉人掌公墓之地,凡诸侯居左右以前,卿大夫居后,汉制亦谓之陪陵。其公卿大臣列将有功者,宜陪寿陵,其广为兆域,使足兼容。”
秋七月,治兵,遂西征刘备,九月,至长安。
冬十月,宛守将侯音等反,执南阳太守,劫略吏民,保宛。初,曹仁讨关羽,屯樊城,是月使仁围宛。
二十四年春正月,仁屠宛,斩音。[一]
注[一]曹瞒传曰:是时南阳闲苦繇役,音于是执太守*(东里箧)**[东里衮]*,与吏民共反,与关羽连和。南阳功曹宗子卿往说音曰:“足下顺民心,举大事,远近莫不望风;然执郡将,逆而无益,何不遣之。吾与子共暞力,比曹公军来,关羽兵亦至矣。”音从之,即释遣太守。
子卿因夜踰城亡出,遂与太守收余民围音,会曹仁军至,共灭之。
夏侯渊与刘备战于阳平,为备所杀。三月,王自长安出斜谷,军遮要以临汉中,遂至阳平。
备因险拒守。[一]
注[一]九州春秋曰:时王欲还,出令曰“鸡肋”,官属不知所谓。主簿杨修便自严装,人惊问修:“何以知之?”修曰:“夫鸡肋,弃之如可惜,食之无所得,以比汉中,知王欲还也。”
夏五月,引军还长安。
秋七月,以夫人卞氏为王后。遣于禁助曹仁击关羽。八月,汉水溢,灌禁军,军没,羽获禁,遂围仁。使徐晃救之。
九月,相国钟繇坐西曹掾魏讽反免。[一]
注[一]世语曰:讽字子京,沛人,有惑众才,倾动邺都,钟繇由是辟焉。大军未反,讽潜结徒党,又与长乐韂尉陈祎谋袭邺。未及期,祎惧,告之太子,诛讽,坐死者数十人。王昶家诫曰“济阴魏讽”,而此云沛人,未详。
冬十月,军还洛阳。[一]孙权遣使上书,以讨关羽自效。王自洛阳南征羽,未至,晃攻羽,破之,羽走,仁围解。王军摩陂。[二]
注[一]曹瞒传曰:王更修治北部尉廨,令过于旧。
注[二]魏略曰:孙权上书称臣,称说天命。王以权书示外曰:“是儿欲踞吾着炉火上邪!”
侍中陈髃、尚书桓阶奏曰:“汉自安帝已来,政去公室,国统数绝,至于今者,唯有名号,尺土一民,皆非汉有,期运久已尽,历数久已终,非适今日也。是以桓、灵之间,诸明图纬者,皆言‘汉行气尽,黄家当兴’。
殿下应期,十分天下而有其九,以服事汉,髃生注望,,遐迩怨叹,是故孙权在远称臣,此天人之应,异气齐声。臣愚以为虞、夏不以谦辞,殷、周不吝诛放,畏天知命,无所与让也。”
魏氏春秋曰:夏侯惇谓王曰:“天下咸知汉祚已尽,异代方起。自古已来,能除民害为百姓所归者,即民主也。今殿下即戎三十余年,功德着于黎庶,为天下所依归,应天顺民,复何疑哉!”王曰:“‘施于有政,是亦为政’。若天命在吾,吾为周文王矣。”曹瞒传及世语并云桓阶劝王正位,夏侯惇以为宜先灭蜀,蜀亡则吴服,二方既定,然后遵舜、禹之轨,王从之。及至王薨,惇追恨前言,发病卒。孙盛评曰:夏侯惇耻为汉官,求受魏印,桓阶方惇,有义直之节;考其传记,世语为妄矣。
二十五年春正月,至洛阳。权击斩羽,传其首。
庚子,王崩于洛阳,年六十六。[一]遗令曰:“天下尚未安定,未得遵古也。葬毕,皆除服。
其将兵屯戍者,皆不得离屯部。有司各率乃职。敛以时服,无藏金玉珍宝。”谥曰武王。二月丁卯,葬高陵。[二]
注[一]世语曰:太祖自汉中至洛阳,起建始殿,伐濯龙祠而树血出。曹瞒传曰:王使工苏越徙美梨,掘之,根伤尽出血。越白状,王躬自视而恶之,以为不祥,还遂寝疾。
注[二]魏书曰:太祖自统御海内,芟夷髃丑,其行军用师,大较依孙、吴之法,而因事设奇,谲敌制胜,变化如神。自作兵书十万余言,诸将征伐,皆以新书从事。临事又手为节度,从令者克捷,违教者负败。与虏对陈,意思安闲,如不欲战,然及至决机乘胜,气势盈溢,故每战必克,军无幸胜。知人善察,难眩以伪,拔于禁、乐进于行陈之间,取张辽、徐晃于亡虏之内,皆佐命立功,列为名将;其余拔出细微,登为牧守者,不可胜数。是以□造大业,文武并施,御军三十余年,手不舍书,昼则讲武策,夜则思经传,登高必赋,及造新诗,被之管弦,皆成乐章。才力绝人,手射飞鸟,躬禽猛兽,尝于南皮一日射雉获六十三头。及造作宫室,缮治器械,无不为之法则,皆尽其意。雅性节俭,不好华丽,后宫衣不锦绣,侍御履不二采,帷帐屏风,坏则补纳,茵蓐取温,无有缘饰。攻城拔邑,得美丽之物,则悉以赐有功,勋劳宜赏,不吝千金,无功望施,分毫不与,四方献御,与髃下共之。常以送终之制,袭称之数,繁而无益,俗又过之,故预自制终亡衣服,四箧而已。傅子曰:太祖愍嫁取之奢僭,公女适人,皆以皁帐,从婢不过十人。张华博物志曰:汉世,安平崔瑗、瑗子寔、弘农张芝、芝弟昶并善草书,而太祖亚之。桓谭、蔡邕善音乐,冯翊山子道、王九真、郭凯等善围澙,太祖皆与埒能。又好养性法,亦解方药,招引方术之士,庐江左慈、谯郡华佗、甘陵甘始、阳城蜔俭无不毕至,又习啖野葛至一尺,亦得少多饮鸩酒。傅子曰:汉末王公,多委王服,以幅巾为雅,是以袁绍、*(崔豹)**[崔钧]*之徒,虽为将帅,皆着缣巾。魏太祖以天下凶荒,资财乏匮,拟古皮弁,裁缣帛以为帢,合于简易随时之义,以色别其贵贱,于今施行,可谓军容,非国容也。曹瞒传曰:太祖为人佻易无威重,好音乐,倡优在侧,常以日达夕。被服轻绡,身自佩小鞶囊,以盛手巾细物,时或冠帢帽以见宾客。每与人谈论,戏弄言诵,尽无所隐,及欢悦大笑,至以头没杯案中,肴膳皆沾污巾帻,其轻易如此。然持法峻刻,诸将有计画胜出己者,随以法诛之,及故人旧怨,亦皆无余。其所刑杀,辄对之垂涕嗟痛之,终无所活。初,袁忠为沛相,尝欲以法治太祖,沛国桓邵亦轻之,及在兖州,陈留边让言议颇侵太祖,太祖杀让,族其家,忠、邵俱避难交州,太祖遣使就太守士燮尽族之。桓邵得出首,拜谢于庭中,太祖谓曰:“跪可解死邪!”遂杀之。常出军,行经麦中,令“士卒无败麦,犯者死”。骑士皆下马,付麦以相持,于是太祖马腾入麦中,□主簿议罪;主簿对以春秋之义,罚不加于尊。太祖曰:“制法而自犯之,何以帅下?然孤为军帅,不可自杀,请自刑。”因援剑割发以置地。又有幸姬常从昼寝,枕之卧,告之曰:“须臾觉我。”姬见太祖卧安,未即寤,及自觉,棒杀之。常讨贼,廪谷不足,私谓主者曰:“如何?”主者曰:“可以小斛以足之。”太祖曰:“善。”后军中言太祖欺众,太祖谓主者曰:
“特当借君死以厌众,不然事不解。”乃斩之,取首题徇曰:“行小斛,盗官谷,斩之军门。”
其酷虐变诈,皆此类也。
评曰:汉末,天下大乱,雄豪并起,而袁绍虎摉四州,强盛莫敌。太祖运筹演谋,鞭挞宇内,閴申、商之法术,该韩、白之奇策,官方授材,各因其器,矫情任算,不念旧恶,终能总御皇机,克成洪业者,惟其明略最优也。抑可谓非常之人,超世之杰矣。 
\end{yuanwen}

\part{魏书二}

\chapter{文帝纪第二}
\begin{yuanwen}
文皇帝讳丕,字子桓,武帝太子也。中平四年冬,生于谯。[一]建安十六年,为五官中郎将、副丞相。二十二年,立为魏太子。[二]太祖崩,嗣位为丞相、魏王。[三]尊王后曰王太后。
改建安二十五年为延康元年。
注[一]魏书曰:帝生时,有云气青色而圜如车盖当其上,终日,望气者以为至贵之证,非人臣之气。年八岁,能属文。有逸才,遂博贯古今经传诸子百家之书。善骑射,好击剑。举茂才,不行。献帝起居注曰:建安十*(五)**[三]*年,为司徒赵温所辟。太祖表“温辟臣子弟,选举故不以实”。使侍中守光禄勋郗虑持节奉策免温官。
注[二]魏略曰:太祖不时立太子,太子自疑。是时有高元吕者,善相人,乃呼问之,对曰:
“其贵乃不可言。”问:“寿几何?”元吕曰:“其寿,至四十当有小苦,过是无忧也。”后无几而立为王太子,至年四十而薨。
注[三]袁宏汉纪载汉帝诏曰:“魏太子丕:昔皇天授乃显考以翼我皇家,遂攘除髃凶,拓定九州,弘功茂绩,光于宇宙,朕用垂拱负扆二十有余载。天不慭遗一老,永保余一人,早世潜神,哀悼伤切。丕奕世宣明,宜秉文武,绍熙前绪。
今使使持节御史大夫华歆奉策诏授丕丞相印绶、魏王玺绂,领冀州牧。方今外有遗虏,遐夷未宾,旗鼓犹在边境,干戈不得韬刃,斯乃播扬洪烈,立功垂名之秋也。岂得修谅闇之礼,究曾、闵之志哉?其敬服朕命,抑弭忧怀,旁祗厥绪,时亮庶功,以称朕意。于戏,可不勉与!”
元年二月[一]王戌,以大中大夫贾诩为太尉,御史大夫华歆为相国,大理王朗为御史大夫。
置散骑常侍、侍郎各四人,其宦人为官者不得过诸署令;为金策着令,藏之石室。
注[一]魏书载庚戌令曰:“关津所以通商旅,池苑所以御灾荒,设禁重税,非所以便民;其除池□之禁,轻关津之税,皆复什一。”辛亥,赐诸侯王将相已下将粟万斛,帛千匹,金银各有差等。遣使者循行郡国,有违理掊克暴虐者,举其罪。
初,汉熹平五年,黄龙见谯,光禄大夫桥玄问太史令单扬:“此何祥也?”扬曰:“其国后当有王者兴,不及五十年,亦当复见。天事恒象,此其应也。”内黄殷登默而记之。至四十五年,登尚在。三月,黄龙见谯,登闻之曰:“单扬之言,其验兹乎!”[一]
注[一]魏书曰:王召见登,谓之曰:“昔成风闻楚丘之繇而敬事季友,邓晨信少公之言而自纳光武。登以笃老,服膺占术,记识天道,岂有是乎!”赐登谷三百斛,遣归家。
已卯,以前将军夏侯惇为大将军。濊貊、扶余单于、焉耆、于阗王皆各遣使奉献。[一]
注[一]魏书曰:丙戌,令史官奏修重、黎、羲、和之职,钦若昊天,历象日月星辰以奉天时。
臣松之案:魏书有是言而不闻其职也。丁亥令曰:“故尚书仆射毛玠、奉常王修、凉茂、郎中令袁涣、少府谢奂、万潜、中尉徐奕、国渊等,皆忠直在朝,履蹈仁义,并早即世,而子孙陵迟,恻然愍之,其皆拜子男为郎中。”
夏四月丁巳,饶安县言白雉见。[一]庚午,大将军夏侯惇薨。[二]
注[一]魏书曰:赐饶安田租,勃海郡百户牛酒,大酺三日;太常以太牢祠宗庙。
注[二]魏书曰:王素服幸邺东城门发哀。孙盛曰:在礼,天子哭同姓于宗庙门之外。哭于城门,失其所也。
五月戊寅,天子命王追尊皇祖太尉曰太王,夫人丁氏曰太王后,封王子叡为武德侯。[一]是月,冯翊山贼郑甘、王照率众降,皆封列侯。[二]
注[一]魏略曰:以侍中郑称为武德侯傅,令曰:“龙渊、太阿出昆吾之金,和氏之璧由井里之田;砻之以砥砺,错之以他山,故能致连城之价,为命世之宝。学亦人之砥砺也。称笃学大儒,勉以经学辅侯,宜旦夕入侍,曜明其志。”
注[二]魏书曰:初,郑甘、王照及卢水胡率其属来降,王得降书以示朝曰:“前欲有令吾讨鲜卑者,吾不从而降;又有欲使吾及今秋讨卢水胡者,吾不听,今又降。昔魏武侯一谋而当,有自得之色,见讥李悝。吾今说此,非自是也,徒以为坐而降之,其功大于动兵革也。”
酒泉黄华、张掖张进等各执太守以叛。金城太守苏则讨进,斩之。华降。[一]
注[一]华后为兖州刺史,见王凌传。
六月辛亥,治兵于东郊,[一]庚午,遂南征。[二]
注[一]魏书曰:公卿相仪,王御华盖,视金鼓之节。
注[二]魏略曰:王将出征,度支中郎将新平霍性上疏谏曰:“臣闻文王与纣之事,是时天下括囊无咎,凡百君子,莫肯用讯。今大王体则乾坤,广开四聪,使贤愚各建所规。伏惟先王功无与比,而今能言之类,不称为德。故圣人曰‘得百姓之欢心’。兵书曰‘战,危事也’是以六国力战,强秦承弊,豳王不争,周道用兴。愚谓大王且当委重本朝而守其雌,抗威虎卧,功业可成。而今□基,便复起兵,兵者凶器,必有凶扰,扰则思乱,乱出不意。臣谓此危,危于累卵。昔夏启隐神三年,易有‘不远而复’,论有‘不惮改’。诚愿大王揆古察今,深谋远虑,与三事大夫算其长短。臣沐浴先王之遇,又初改政,复受重任,虽知言触龙鳞,阿谀近福,窃感所诵,危而不持。”奏通,帝怒,遣刺奸就考,竟杀之。既而悔之,追原不及。
秋七月庚辰,令曰:“轩辕有明台之议,放勋有衢室之问,皆所以广询于下也。[一]百官有司,其务以职尽规谏,将率陈军法,朝士明制度,牧守申政事,缙绅考六艺,吾将兼览焉。”
注[一]管子曰:黄帝立明台之议者,上观于兵也;尧有衢室之问者,下听于民也;舜有告善之旌,而主不蔽也;禹立建鼓于朝,而备诉讼也;汤有总街之廷,以观民非也;武王有灵台之囿,而贤者进也:此古圣帝明王所以有而勿失,得而勿忘也。
孙权遣使奉献。蜀将孟达率众降。武都氐王杨仆率种人内附,居汉阳郡。[一]
注[一]魏略载王自手笔令曰:“*(吾)**[日]*前遣使宣国威灵,而达即来。吾惟春秋褒仪父,即封拜达,使还领新城太守。近复有扶老携幼首向王化者。吾闻夙沙之民自缚其君以归神农,豳国之众襁负其子而入丰、镐,斯岂驱略迫胁之所致哉?乃风化动其情而仁义感其衷,欢心内发使之然也。以此而推,西南将万里无外,权、备将与谁守死乎?”
甲午,军次于谯,大飨六军及谯父老百姓于邑东。[一]八月,石邑县言凤皇集。
注[一]魏书曰:设伎乐百戏,令曰:“先王皆乐其所生,礼不忘其本。谯,霸王之邦,真人本出,其复谯租税二年。”三老吏民上寿,日夕而罢。丙申,亲祠谯陵。孙盛曰:昔者先王之以孝治天下也,内节天性,外施四海,存尽其敬,亡极其哀,思慕谅闇,寄政頉宰,故曰“三年之丧,自天子达于庶人”;夫然,故在三之义惇,臣子之恩笃,雍熙之化隆,经国之道固,圣人之所以通天地,厚人伦,显至教,敦风俗,斯万世不易之典,百王服膺之制也。
是故丧礼素冠,郐人着庶见之讥,宰予降儙,仲尼发不仁之叹,子颓忘戚,君子以为乐祸,鲁侯易服,春秋知其不终,岂不以坠至痛之诚心,丧哀乐之大节者哉?故虽三季之末,七雄之弊,犹未有废缞斩于旬朔之间,释麻杖于反哭之日者也。逮于汉文,变易古制,人道之纪,一旦而废,缞素夺于至尊,四海散其遏密,义感阙于髃后,大化坠于君亲;虽心存贬约,虑在经纶,至于树德垂声,崇化变俗,固以道薄于当年,风颓于百代矣。且武王载主而牧野不陈,晋襄墨缞而三帅为俘,应务济功,服其焉害?魏王既追汉制,替其大礼,处莫重之哀而设飨宴之乐,居贻厥之始而坠王化之基,及至受禅,显纳二女,忘其至恤以诬先圣之典,天心丧矣,将何以终!是以知王龄之不遐,卜世之期促也。
冬十*(一)*月癸卯,令曰:“诸将征伐,士卒死亡者或未收敛,吾甚哀之;其告郡国给槥椟殡敛,*槥音韂。*送致其家,官为设祭。”[一]丙午,行至曲蠡。
注[一]汉书高祖八月令曰:“士卒从军死,为槥。”应劭曰:“槥,小棺也,今谓之椟。”应璩百一诗曰:“槥车在道路,征夫不得休。”陆机大墓赋曰:“观细木而闷迟,鷪洪椟而念槥。”
汉帝以众望在魏,乃召髃公卿士,[一]告祠高庙。使兼御史大夫张音持节奉玺绶禅位,册曰:
“咨尔魏王:昔者帝尧禅位于虞舜,舜亦以命禹,天命不于常,惟归有德。汉道陵迟,世失其序,降及朕躬,大乱兹昏,髃凶肆逆,宇内颠覆。赖武王神武,拯兹难于四方,惟清区夏,以保绥我宗庙,岂予一人获乂,俾九服实受其赐。今王钦承前绪,光于乃德,恢文武之大业,昭尔考之弘烈。皇灵降瑞,人神告征,诞惟亮采,师锡朕命,佥曰尔度克协于虞舜,用率我唐典,敬逊尔位。于戏!天之历数在尔躬,允执其中,天禄永终;君其祗顺大礼,飨兹万国,以肃承天命。”[二]乃为坛于繁阳。庚午,王升坛即阼,百官陪位。事讫,降坛,视燎成礼而反。改延康为黄初,大赦。[三]
注[一]袁宏汉纪载汉帝诏曰:“朕在位三十有二载,遭天下荡覆,幸赖祖宗之灵,危而复存。
然仰瞻天文,俯察民心,炎精之数既终,行运在乎曹氏。是以前王既树神武之绩,今王又光曜明德以应其期,是历数昭明,信可知矣。夫大道之行,天下为公,选贤与能,故唐尧不私于厥子,而名播于无穷。朕羡而慕焉,今其追踵尧典,禅位于魏王。”
注[二]献帝传载禅代众事曰:左中郎将李伏表魏王曰:“昔先王初建魏国,在境外者闻之未审,皆以为拜王。武都李庶、姜合羁旅汉中,谓臣曰:‘必为魏公,未便王也。定天下者,魏公子桓,神之所命,当合符谶,以应天人之位。’臣以合辞语镇南将军张鲁,鲁亦问合知书所出?合曰:‘孔子玉版也。天子历数,虽百世可知。’是后月余,有亡人来,写得册文,卒如合辞。合长于内学,关右知名。鲁虽有怀国之心,沉溺异道变化,不果寤合之言。后密与臣议策质,国人不协,或欲西通,鲁即怒曰:‘宁为魏公奴,不为刘备上客也。’言发恻痛,诚有由然。合先迎王师,往岁病亡于邺。自臣在朝,每为所亲宣说此意,时未有宜,弗敢显言。殿下即位初年,祯祥众瑞,日月而至,有命自天,昭然着见。然圣德洞达,符表豫明,实乾坤挺庆,万国作孚。臣每庆贺,欲言合验;事君尽礼,人以为谄。况臣名行秽贱,入朝日浅,言为罪尤,自抑而已。今洪泽被四表,灵恩格天地,海内翕习,殊方归服,兆应并集,以扬休命,始终允臧。臣不胜喜舞,谨具表通。”王令曰:“以示外。薄德之人,何能致此,未敢当也;斯诚先王至德通于神明,固非人力也。” Ui魏王侍中刘廙、辛毗、刘晔、尚书令桓阶、尚书陈矫、陈髃、给事黄门侍郎王毖、董遇等言:“臣伏读左中郎将李伏上事,考图纬之言,以效神明之应,稽之古代,未有不然者也。故尧称历数在躬,璇玑以明天道;周武未战而赤乌衔书;汉祖未兆而神母告符;孝宣仄微,字成木叶;光武布衣,名已勒谶。是天之所命以着圣哲,非有言语之声,芬芳之臭,可得而知也,徒县象以示人,微物以效意耳。
自汉德之衰,渐染数世,桓、灵之末,皇极不建,暨于大乱,二十余年。天之不泯,诞生明圣,以济其难,是以符谶先着,以彰至德。殿下践阼未儙,而灵象变于上,髃瑞应于下,四方不羁之民,归心向义,唯惧在后,虽典籍所传,未若今之盛也。臣妾远近,莫不凫藻。”
王令曰:“儣牛之驳似虎,莠之幼似禾,事有似是而非者,今日是已。鷪斯言事,良重吾不德。”于是尚书仆射宣告官寮,咸使闻知。 Ui辛亥,太史丞许芝条魏代汉见谶纬于魏王曰:
“易传曰:‘圣人受命而王,黄龙以戊己日见。’七月四日戊寅,黄龙见,此帝王受命之符瑞最着明者也。又曰:‘初六,履霜,阴始凝也。’又有积虫大穴天子之宫,厥咎然,今蝗虫见,应之也。又曰:‘圣人以德亲比天下,仁恩洽普,厥应麒麟以戊己日至,厥应圣人受命。’又曰:‘圣人清净行中正,贤人福至民从命,厥应麒麟来。’春秋汉含孳曰:‘汉以魏,魏以征。’春秋玉版谶曰:‘代赤者魏公子。’春秋佐助期曰:‘汉以许昌失天下。’故白马令李云上事曰:‘许昌气见于当涂高,当涂高者当昌于许。’当涂高者,魏也;象魏者,两观阙是也;当道而高大者魏。魏当代汉。今魏基昌于许,汉征绝于许,乃今效见,如李云之言,许昌相应也。佐助期又曰:‘汉以蒙孙亡。’说者以蒙孙汉二十四帝,童蒙愚昏,以弱亡。或以杂文为蒙其孙当失天下,以为汉帝非正嗣,少时为董侯,名不正,蒙乱之荒惑,其子孙以弱亡。孝经中黄谶曰:‘日载东,绝火光。不横一,圣聪明。四百之外,易姓而王。天下归功,致太平,居八甲;共礼乐,正万民,嘉乐家和杂。’此魏王之姓讳,着见图谶。易运期谶曰:‘言居东,西有午,两日并光日居下。其为主,反为辅。五八四十,黄气受,真人出。’言午,许字。
两日,昌字。汉当以许亡,魏当以许昌。今际会之期在许,是其大效也。易运期又曰:‘鬼在山,禾女连,王天下。’臣闻帝王者,五行之精;易姓之符,代兴之会,以七百二十年为一轨。有德者过之,至于八百,无德者不及,至四百载。是以周家八百六十七年,夏家四百数十年,汉行夏正,迄今四百二十六岁。又高祖受命,数虽起乙未,然其兆征始于获麟。获麟以来七百余年,天之历数将以尽终。帝王之兴,不常一姓。太微中,黄帝坐常明,而赤帝坐常不见,以为黄家兴而赤家衰,凶亡之渐。自是以来四十余年,又荧惑失色不明十有余年。
建安十年,彗星先除紫微,二十三年,复扫太微。新天子气见东南以来,二十三年,白虹贯日,月蚀荧惑,比年己亥、壬子、丙午日蚀,皆水灭火之象也。殿下即位,初践阼,德配天地,行合神明,恩泽盈溢,广被四表,格于上下。是以黄龙数见,凤皇仍翔,麒麟皆臻,白虎效仁,前后献见于郊甸;甘露醴泉,奇兽神物,众瑞并出。斯皆帝王受命易姓之符也。昔黄帝受命,风后受河图;舜、禹有天下,凤皇翔,洛出书;汤之王,白鸟为符;文王为西伯,赤鸟衔丹书;武王伐殷,白鱼升舟;高祖始起,白蛇为征。巨迹瑞应,皆为圣人兴。观汉前后之大灾,今兹之符瑞,察图谶之期运,揆河洛之所甄,未若今大魏之最美也。夫得岁星者,道始兴。昔武王伐殷,岁在鹑火,有周之分野也。高祖入秦,五星聚东井,有汉之分野也。今兹岁星在大梁,有魏之分野也。而天之瑞应,并集来臻,四方归附,襁负而至,兆民欣戴,咸乐嘉庆。春秋大传曰:‘周公何以不之鲁?盖以为虽有继体守文之君,不害圣人受命而王。’周公反政,尸子以为孔子非之,以为周公不圣,不为兆民也。京房作易传曰:‘凡为王者,恶者去之,弱者夺之。易姓改代,天命应常,人谋鬼谋,百姓与能。’伏惟殿下体尧舜之盛明,膺七百之禅代,当汤武之期运,值天命之移受,河洛所表,图谶所载,昭然明白,天下学士所共见也。臣职在史官,考符察征,图谶效见,际会之期,谨以上闻。”王令曰:“昔周文三分天下有其二,以服事殷,仲尼叹其至德;公旦履天子之籍,听天下之断,终然复子明辟,书美其人。吾虽德不及二圣,敢忘高山景行之义哉?若夫唐尧、舜、禹之迹,皆以圣质茂德处之,故能上和灵只,下宁万姓,流称今日。今吾德至薄也,人至鄙也,遭遇际会,幸承先王余业,恩未被四海,泽未及天下,虽倾仓竭府以振魏国百姓,犹寒者未尽暖,饥者未尽饱。夙夜忧惧,弗敢遑宁,庶欲保全发齿,长守今日,以没于地,以全魏国,下见先王,以塞负荷之责。望狭志局,守此而已;虽屡蒙祥瑞,当之战惶,五色无主。若芝之言,岂所闻乎?心栗手悼,书不成字,辞不宣心。吾闲作诗曰:‘丧乱悠悠过纪,白骨纵横万里,哀哀下民靡恃,吾将佐时整理,复子明辟致仕。’庶欲守此辞以自终,卒不虚言也。宜宣示远近,使昭赤心。”于是侍中辛毗、刘晔、散骑常侍傅巽、韂臻、尚书令桓阶、尚书陈矫、陈髃、给事中博士骑都尉苏林、董巴等奏曰:“伏见太史丞许芝上魏国受命之符;令书恳切,允执谦让,虽舜、禹、汤、文,义无以过。然古先哲王所以受天命而不辞者,诚急遵皇天之意,副兆民之望,弗得已也。且易曰:‘观乎天文以察时变,观乎人文以化成天下。’又曰:
‘天垂象,见吉凶,圣人则之;河出图,洛出书,圣人效之。’以为天文因人而变,至于河洛之书,着于洪范,则殷、周效而用之矣。斯言,诚帝王之明符,天道之大要也。是以由德应录者代兴于前,失道数尽者迭废于后,传讥苌弘欲支天之所坏,而说蔡墨‘雷乘干’之说,明神器之存亡,非人力所能建也。今汉室衰替,帝纲堕坠,天子之诏,歇灭无闻,皇天将舍旧而命新,百姓既去汉而为魏,昭然着明,是可知也。先王拨乱平世,将建洪基;至于殿下,以至德当历数之运,即位以来,天应人事,粲然大备,神灵图籍,兼仍往古,休征嘉兆,跨越前代;是芝所取中黄、运期姓纬之谶,斯文乃着于前世,与汉并见。由是言之,天命久矣,非殿下所得而拒之也。神明之意,候望禋享,兆民颙颙,咸注嘉愿,惟殿下览图籍之明文,急天下之公义,辄宣令外内,布告州郡,使知符命着明,而殿下谦虚之意。”令曰:“下四方以明孤款心,是也。至于览余辞,岂余所谓哉?宁所堪哉?诸卿指论,未若孤自料之审也。夫虚谈谬称,鄙薄所弗当也。且闻比来东征,经郡县,历屯田,百姓面有饥色,衣或短褐不完,罪皆在孤;
是以上惭众瑞,下愧士民。由斯言之,德尚未堪偏王,何言帝者也!宜止息此议,无重吾不德,使逝之后,不愧后之君子。” Ui癸丑,宣告髃寮。督军御史中丞司马懿、侍御史郑浑、羊秘、鲍勋、武周等言:“令如左。伏读太史丞许芝上符命事,臣等闻有唐世衰,天命在虞,虞氏世衰,天命在夏;然则天地之灵,历数之运,去就之符,惟德所在。故孔子曰:‘凤鸟不至,河不出图,吾已矣夫!’今汉室衰,自安、和、冲、质以来,国统屡绝,桓、灵荒淫,禄去公室,此乃天命去就,非一朝一夕,其所由来久矣。殿下践阼,至德广被,格于上下,天人感应,符瑞并臻,考之旧史,未有若今日之盛。夫大人者,先天而天弗违,后天而奉天时,天时已至而犹谦让者,舜、禹所不为也,故生民蒙救济之惠,髃类受育长之施。今八方颙颙,大小注望,皇天乃眷,神人同谋,十分而九以委质,义过周文,所谓过恭也。臣妾上下,伏所不安。”令曰:“世之所不足者道义也,所有余者苟妄也;常人之性,贱所不足,贵所有余,故曰‘不患无位,患所以立’。孤虽寡德,庶自免于常人之贵。夫‘石可破而不可夺坚,丹可磨而不可夺赤’。丹石微物,尚保斯质,况吾托士人之末列,曾受教于君子哉?且于陵仲子以仁为富,柏成子高以义为贵,鲍焦感子贡之言,弃其蔬而槁死,薪者讥季札失辞,皆委重而弗视。吾独何人?昔周武,大圣也,使叔旦盟胶鬲于四内,使召公约微子于共头,故伯夷、叔齐相与笑之曰:‘昔神农氏之有天下,不以人之坏自成,不以人之卑自高。’以为周之伐殷以暴也。
吾德非周武而义惭夷、齐,庶欲远苟妄之失道,立丹石之不夺,迈于陵之所富,蹈柏成之所贵,执鲍焦之贞至,遵薪者之清节。故曰:‘三军可夺帅,匹夫不可夺志。’吾之斯志,岂可夺哉?” Ui乙卯,册诏魏王禅代天下曰:“惟延康元年十月乙卯,皇帝曰,咨尔魏王:
夫命运否泰,依德升降,三代卜年,着于春秋,是以天命不于常,帝王不一姓,由来尚矣。
汉道陵迟,为日已久,安、顺已降,世失其序,冲、质短祚,三世无嗣,皇纲肇亏,帝典颓沮。暨于朕躬,天降之灾,遭无妄厄运之会,值炎精幽昧之期。变兴辇毂,祸由阉宦。董卓乘衅,恶甚浇、豷,劫迁省御,*(太仆)**[火扑]*宫庙,遂使九州幅裂,强敌虎争,华夏鼎沸,蝮蛇塞路。当斯之时,尺土非复汉有,一夫岂复朕民?幸赖武王德膺符运,奋扬神武,芟夷凶暴,清定区夏,保乂皇家。今王缵承前绪,至德光昭,御衡不迷,布德优远,声教被四海,仁风扇鬼区,是以四方效珍,人神响应,天之历数实在尔躬。昔虞舜有大功二十,而放勋禅以天下;大禹有疏导之绩,而重华禅以帝位。汉承尧运,有传圣之义,加顺灵只,绍天明命,厘降二女,以嫔于魏。使使持节行御史大夫事太常音,奉皇帝玺绶,王其永君万国,敬御天威,允执其中,天禄永终,敬之哉?”于是尚书令桓阶等奏曰:“汉氏以天子位禅之陛下,陛下以圣明之德,历数之序,承汉之禅,允当天心。夫天命弗可得辞,兆民之望弗可得违,臣请会列侯诸将、髃臣陪隶,发玺书,顺天命,具礼仪列奏。”令曰:“当议孤终不当承之意而已。犹猎,还方有令。”
Ui尚书令等又奏曰:“昔尧、舜禅于文祖,至汉氏,以师征受命,畏天之威,不敢怠遑,便即位行在所之地。今当受禅代之命,宜会百寮髃司,六军之士,皆在行位,使咸鷪天命。
营中促狭,可于平敞之处设坛场,奉答休命。臣辄与侍中常侍会议礼仪,太史官择吉日讫,复奏。”令曰:“吾殊不敢当之,外亦何豫事也!” Ui侍中刘廙、常侍韂臻等奏议曰:“汉氏遵唐尧公天下之议,陛下以圣德膺历数之运,天人同欢,靡不得所,宜顺灵符,速践皇阼。
问太史丞许芝,今月十七日己未直成,可受禅命,辄治坛场之处,所当施行别奏。”令曰;
“属出见外,便设坛场,斯何谓乎?今当辞让不受诏也。但于帐前发玺书,威仪如常,且天寒,罢作坛士使归。”既发玺书,王令曰:“当奉还玺绶为让章。吾岂奉此诏承此贶邪?昔尧让天下于许由、子州支甫,舜亦让于善卷、石户之农、北人无择,或退而耕颍之阳,或辞以幽忧之疾,或远入山林,莫知其处,或携子入海,终身不反,或以为辱,自投深渊;且颜烛惧太朴之不完,守知足之明分,王子搜乐丹穴之潜处,被熏而不出,柳下惠不以三公之贵易其介,曾参不以晋、楚之富易其仁:斯九士者,咸高节而尚义,轻富而贱贵,故书名千载,于今称焉。求仁得仁,仁岂在远?孤独何为不如哉?义有蹈东海而逝,不奉汉朝之诏也。亟为上章还玺绶,宣之天下,使咸闻焉。”己未,宣告髃僚,下魏,又下天下。 Ui辅国将军清苑侯刘若等百二十人上书曰:“伏读令书,深执克让,圣意恳恻,至诚外昭,臣等有所不安。何者?石户、北人,匹夫狂狷,行不合义,事不经见者,是以史迁谓之不然,诚非圣明所当希慕。且有虞不逆放勋之禅,夏禹亦无辞位之语,故传曰:‘舜陟帝位,若固有之。’斯诚圣人知天命不可逆,历数弗可辞也。伏惟陛下应干符运,至德发闻,升昭于天,是三灵降瑞,人神以和,休征杂沓,万国响应,虽欲勿用,将焉避之?而固执谦虚,违天逆众,慕匹夫之微分,背上圣之所蹈,违经谶之明文,信百氏之穿凿,非所以奉答天命,光慰众望也。
臣等昧死以请,辄整顿坛场,至吉日受命,如前奏,分别写令宣下。”王令曰:“昔柏成子高辞夏禹而匿野,颜阖辞鲁币而远迹,夫以王者之重,诸侯之贵,而二子忽之,何则?其节高也。故烈士徇荣名,义夫高贞介,虽蔬食瓢饮,乐在其中。是以仲尼师王骀,而子产嘉申徒。今诸卿皆孤股肱腹心,足以明孤,而今咸若斯,则诸卿游于形骸之内,而孤求为形骸之外,其不相知,未足多怪。亟为上章还玺绶,勿复纷纷也。”
Ui辅国将军等一百二十人又奏曰:“臣闻符命不虚见,众心不可违,故孔子曰:‘周公其为不圣乎?以天下让。是天地日月轻去万物也。’是以舜向天下,不拜而受命。今火德气尽,炎上数终,帝迁明德,祚隆大魏。符瑞昭鴋,受命既固,光天之下,神人同应,虽有虞仪凤,成周跃鱼,方今之事,未足以喻。而陛下违天命以饰小行,逆人心以守私志,上忤皇穹眷命之旨,中忘圣人达节之数,下孤人臣翘首之望,非所以扬圣道之高衢,乘无穷之懿勋也。臣等闻事君有献可替否之道,奉上有逆鳞固争之义,臣等敢以死请。”令曰:“夫古圣王之治也,至德合乾坤,惠泽均造化,礼教优乎昆虫,仁恩洽乎草木,日月所照,戴天履地含气有生之类,靡不被服清风,沐浴玄德;
是以金革不起,苛慝不作,风雨应节,祯祥触类而见。今百姓寒者未暖,饥者未饱,□者未室,寡者未嫁;权、备尚存,未可舞以干戚,方将整以齐斧;戎役未息于外,士民未安于内,耳未闻康哉之歌,目未鷪击壤之戏,婴儿未可托于高巢,余粮未可以宿于田亩:人事未备,至于此也。夜未曜景星,治未通真人,河未出龙马,山未出象车,蓂荚未植阶庭,萐莆未生庖厨,王母未献白环,渠搜未见珍裘:灵瑞未效,又如彼也。昔东户季子、容成、大庭、轩辕、赫胥之君,咸得以此就功勒名。今诸卿独不可少假孤精心竭虑,以和天人,以格至理,使彼众事备,髃瑞效,然后安乃议此乎,何遽相愧相迫之如是也?速为让章,上还玺绶,无重吾不德也。” Ui侍中刘廙等奏曰:“伏惟陛下以大圣之纯懿,当天命之历数,观天象则符瑞着明,考图纬则文义焕炳,察人事则四海齐心,稽前代则异世同归;而固拒禅命,未践尊位,圣意恳恻,臣等敢不奉诏?辄具章遣使者。”奉令曰:“泰伯三以天下让,人无得而称焉,仲尼叹其至德,孤独何人?” Ui庚申,魏王上书曰:“皇帝陛下:奉被今月乙卯玺书,伏听册命,五内惊震,精爽散越,不知所处。臣前上还相位,退守藩国,圣恩听许。臣虽无古人量德度身自定之志,保己存性,实其私愿。不寤陛下猥损过谬之命,发不世之诏,以加无德之臣。且闻尧禅重华,举其克谐之德,舜授文命,采其齐圣之美,犹下咨四岳,上观璇玑。今臣德非虞、夏,行非二君,而承历数之谘,应选授之命,内自揆抚,无德以称。且许由匹夫,犹拒帝位,善卷布衣,而逆虞诏。臣虽鄙蔽,敢忘守节以当大命,不胜至愿。谨拜章陈情,使行相国永寿少府粪土臣毛宗奏,并上玺绶。” Ui辛酉,给事中博士苏林、董巴上表曰:“天有十二次以为分野,王公之国,各有所属,周在鹑火,魏在大梁。岁星行历十二次国,天子受命,诸侯以封。周文王始受命,岁在鹑火,至武王伐纣十三年,岁星复在鹑火,故春秋传曰:‘武王伐纣,岁在鹑火;岁之所在,即我有周之分野也。’昔光和七年,岁在大梁,武王始受命,*(为)**[于]*时将讨黄巾。是岁改年为中平元年。建安元年,岁复在大梁,始拜大将军。十三年复在大梁,始拜丞相。今二十五年,岁复在大梁,陛下受命。此魏得岁与周文王受命相应。今年青龙在庚子,诗推度灾曰:‘庚者更也,子者滋也,圣命天下治。’又曰:‘王者布德于子,治成于丑。’此言今年天更命圣人制治天下,布德于民也。魏以改制天下,与*(时)**[诗]*协矣。
颛顼受命,岁在豕韦,韂居其地,亦在豕韦,故春秋传曰:‘韂,颛顼之墟也。’今十月斗之建,则颛顼受命之分也,始魏以十月受禅,此同符始祖受命之验也。魏之氏族,出自颛顼,与舜同祖,见于春秋世家。舜以土德承尧之火,今魏亦以土德承汉之火,于行运,会于尧舜授受之次。臣闻天之去就,固有常分,圣人当之,昭然不疑,故尧捐骨肉而禅有虞,终无□色,舜发陇亩而君天下,若固有之,其相受授,闲不替漏;天下已传矣,所以急天命,天下不可一日无君也。今汉期运已终,妖异绝之已审,阶下受天之命,符瑞告征,丁宁详悉,反复备至,虽言语相喻,无以代此。今既发诏书,玺绶未御,固执谦让,上逆天命,下违民望。臣谨案古之典籍,参以图纬,魏之行运及天道所在,即尊之验,在于今年此月,昭晰分明。唯阶下迁思易虑,以时即位,显告天帝而告天下,然后改正朔,易服色,正大号,天下幸甚。”令曰:“凡斯皆宜圣德,故曰:‘苟非其人,道不虚行。’天瑞虽彰,须德而光;吾德薄之人,胡足以当之?今让,冀见听许,外内咸使闻知。” Ui壬戌,册诏曰:“皇帝问魏王言:遣宗奉庚申书到,所称引,闻之。朕惟汉家世踰二十,年过四百,运周数终,行祚已讫,天心已移,兆民望绝,天之所废,有自来矣。今大命有所厎止,神器当归圣德,违众不顺,逆天不祥。王其体有虞之盛德,应历数之嘉会,是以祯祥告符,图谶表录,神人同应,受命咸宜。朕畏上帝,致位于王;天不可违,众不可拂。且重华不逆尧命,大禹不辞舜位,若夫由、卷匹夫,不载圣籍,固非皇材帝器所当称慕。今使音奉皇帝玺绶,王其陟帝位,无逆朕命,以祗奉天心焉。” Ui于是尚书令桓阶等奉曰:“今汉使音奉玺书到,臣等以为天命不可稽,神器不可渎。周武中流有白鱼之应,不待师期而大号已建,舜受大麓,桑荫未移而已陟帝位,皆所以祗承天命,若此之速也。故无固让之义,不以守节为贵,必道信于神灵,符合于天地而已。易曰:‘其受命如响,无有远近幽深,遂知来物,非天下之至赜,其孰能与于此?’今陛下应期运之数,为皇天所子,而复稽滞于辞让,低回于大号,非所以则天地之道,副万国之望。臣等敢以死请,辄敕有司修治坛场,择吉日,受禅命,发玺绶。”令曰:“冀三让而不见听,何汲汲于斯乎?” Ui甲子,魏王上书曰:“奉今月壬戌玺书,重被圣命,伏听册告,肝胆战悸,不知所措。天下神器,禅代重事,故尧将禅舜,纳于大麓,舜之命禹,玄圭告功;烈风不迷,九州攸平,询事考言,然后乃命,而犹执谦让于德不嗣。况臣顽固,质非二圣,乃应天统,受终明诏;敢守微节,归志箕山,不胜大愿。谨拜表陈情,使并奉上玺绶。” Ui侍中刘廙等奏曰:“臣等闻圣帝不违时,明主不逆人,故易称通天下之志,断天下之疑。伏惟陛下体有虞之上圣,承土德之行运,当亢阳明夷之会,应汉氏祚终之数,合契皇极,同符两仪。是以圣瑞表征,天下同应,历运去就,深切着明;论之天命,无所与议,比之时宜,无所与争。故受命之期,时清日晏,曜灵施光,休气云蒸。是乃天道悦怿,民心欣戴,而仍见闭拒,于礼何居?且髃生不可一日无主,神器不可以斯须无统,故臣有违君以成业,下有矫上以立事,臣等敢不重以死请。”王令曰:“天下重器,王者正统,以圣德当之,犹有惧心,吾何人哉?且公卿未至乏主,斯岂小事,且宜以待固让之后,乃当更议其可耳。” Ui丁卯,册诏魏王曰:“天讫汉祚,辰象着明,朕祗天命,致位于王,仍陈历数于诏册,喻符运于翰墨;神器不可以辞拒,皇位不可以谦让,稽于天命,至于再三。
且四海不可以一日旷主,万机不可以斯须无统,故建大业者不拘小节,知天命者不系细物,是以舜受大业之命而无逊让之辞,圣人达节,不亦远乎!今使音奉皇帝玺绶,王其钦承,以答天下向应之望焉。” Ui相国华歆、太尉贾诩、御史大夫王朗及九卿上言曰:“臣等被召到,伏见太史丞许芝、左中郎将李伏所上图谶、符命,侍中刘廙等宣□众心,人灵同谋。又汉朝知陛下圣化通于神明,圣德参于虞、夏,因瑞应之备至,听历数之所在,遂献玺绶,固让尊号。能言之伦,莫不抃舞,河图、洛书,天命瑞应,人事协于天时,民言协于天□。而陛下性秉劳谦,体尚克让,明诏恳切,未肯听许,臣妾小人,莫不伊邑。臣等闻自古及今,有天下者不常在乎一姓;考以德势,则盛衰在乎强弱,论以终始,则废兴在乎期运。唐、虞历数,不在厥子而在舜、禹。舜、禹虽怀克让之意迫,髃后执玉帛而朝之,兆民怀欣戴而归之,率土扬歌谣而咏之,故其守节之拘,不可得而常处,达节之权,不可得而久避;是以或逊位而不□,或受禅而不辞,不□者未必厌皇宠,不辞者未必渴帝祚,各迫天命而不得以已。既禅之后,则唐氏之子为宾于有虞,虞氏之冑为客于夏代,然则禅代之义,非独受之者实应天福,授之者亦与有余庆焉。汉自章、和之后,世多变故,稍以陵迟,洎乎孝灵,不恒其心,虐贤害仁,聚敛无度,政在嬖竖,视民如绚,遂令上天震怒,百姓从风如归;当时则四海鼎沸,既没则祸发宫庭,宠势并竭,帝室遂卑,若在帝舜之末节,犹择圣代而授之,荆人抱玉璞,犹思良工而刊之,况汉国既往,莫之能匡,推器移君,委之圣哲,固其宜也。汉朝委质,既愿礼禅之速定也,天祚率土,必将有主;主率土者,非陛下其孰能任之?所谓论德无与为比,考功无推让矣。天命不可久稽,民望不不可久违,臣等慺慺,不胜大愿。伏请陛下割撝谦之志,修受禅之礼,副人神之意,慰外内之愿。”令曰:“以德则孤不足,以时则戎虏未灭。若以髃贤之灵,得保首领,终君魏国,于孤足矣。若孤者,胡足以辱四海?至乎天瑞人事,皆先王圣德遗庆,孤何有焉?是以未敢闻命。” Ui己巳,魏王上书曰:“臣闻舜有宾于四门之勋,乃受禅于陶唐,禹有存国七百之功,乃承禄于有虞。臣以蒙蔽,德非二圣,猥当天统,不敢闻命。敢屡抗疏,略陈私愿,庶章通紫庭,得全微节,情达宸极,永守本志。而音重复衔命,申制诏臣,臣实战惕,不发玺书,而音迫于严诏,不敢复命。愿陛下驰传骋驿,召音还台。不胜至诚,谨使宗奉书。” Ui相国歆、太尉诩、御史大夫朗及九卿奏曰:“臣等伏读诏书,于邑益甚。臣等闻易称圣人奉天时,论语云君子畏天命,天命有去就,然后帝者有禅代。是以唐之禅虞,命在尔躬,虞之顺唐,谓之受终;尧知天命去己,故不得不禅舜,舜知历数在躬,故不敢不受;不得不禅,奉天时也,不敢不受,畏天命也。汉朝虽承季末陵迟之余,犹务奉天命以则尧之道,是以愿禅帝位而归二女。而陛下正于大魏受命之初,抑虞、夏之达节,尚延陵之让退,而所枉者大,所直者小,所详者轻,所略者重,中人凡士犹为陛下陋之。没者有灵,则重华必忿愤于苍梧之神墓,大禹必郁悒于会稽之山阴,武王必不悦于*(商)**[高]*陵之玄宫矣。是以臣等敢以死请。且汉政在阉宦,禄去帝室七世矣,遂集矢石于其宫殿,而二京为之丘墟。当是之时,四海荡覆,天下分崩,武王亲衣甲而冠冑,沐雨而栉风,为民请命,则活万国,为世拨乱,则致升平,鸠民而立长,筑宫而置吏,元元无过,罔于前业,而始有造于华夏。陛下即位,光昭文德,以翊武功,勤恤民隐,视之如伤,惧者宁之,劳者息之,寒者以暖,饥者以充,远人以*(恩复)**[德服]*,寇敌以恩降,迈恩种德,光被四表;稽古笃睦,茂于放勋,网漏吞舟,弘乎周文。是以布政未儙,人神并和,皇天则降甘露而臻四灵,后土则挺芝草而吐醴泉,虎豹鹿兔,皆素其色,雉鸠燕雀,亦白其羽,连理之木,同心之瓜,五采之鱼,珍祥瑞物,杂嗠于其间者,无不毕备。古人有言:‘微禹,吾其鱼乎!’微大魏,则臣等之白骨交横于旷野矣。伏省髃臣外内前后章奏,所以陈□陛下之符命者,莫不条河洛之图书,据天地之瑞应,因汉朝之款诚,宣万方之景附,可谓信矣*(省)**[着]*矣;三王无以及,五帝无以加。民命之悬于魏*[邦,民心之系于魏]*政,三十有余年矣,此乃千世时至之会,万载一遇之秋;达节广度,宜昭于斯际,拘牵小节,不施于此时。久稽天命,罪在臣等。辄营坛场,具礼仪,择吉日,昭告昊天上帝,秩髃神之礼,须禋祭毕,会髃寮于朝堂,议年号、正朔、服色当施行,上。”复令曰:“昔者大舜饭糗茹草,将终身焉,斯则孤之前志也。及至承尧禅,被*(珍)**[袗]*裘,妻二女,若固有之,斯则顺天命也。髃公卿士诚以天命不可拒,民望不可违,孤亦曷以辞焉?” Ui庚午,册诏魏王曰:“昔尧以配天之德,秉六合之重,犹鷪历运之数,移于有虞,委让帝位,忽如遗迹。今天既讫我汉命,乃眷北顾,帝皇之业,实在大魏。朕守空名以窃古义,顾视前事,犹有惭德,而王逊让至于三四,朕用惧焉。夫不辞万乘之位者,知命达节之数也,虞、夏之君,处之不疑,故勋烈垂于万载,美名传于无穷。
今遣守尚书令侍中*(顗)**[觊]*喻,王其速陟帝位,以顺天人之心,副朕之大愿。”
Ui于是尚书令桓阶等奏曰:“今汉氏之命已四至,而陛下前后固辞,臣等伏以为上帝之临圣德,期运之隆大魏,斯岂数载?传称周之有天下,非甲子之朝,殷之去帝位,非牧野之日也,故诗序商汤,追本玄王之至,述姬周,上录后稷之生,是以受命既固,厥德不回。汉氏衰废,行次已绝,三辰垂其征,史官着其验,耆老记先古之占,百姓协歌谣之声。陛下应天受禅,当速即坛场,柴燎上帝,诚不宜久停神器,拒亿兆之愿。臣辄下太史令择元辰,今月二十九日,可登坛受命,请诏王公髃卿,具条礼仪别奏。”令曰:“可。”
注[三]献帝传曰:辛未,魏王登坛受禅,公卿、列侯、诸将、匈奴单于、四夷朝者数万人陪位,燎祭天地、五岳、四渎,曰:“皇帝臣丕敢用玄牡昭告于皇皇后帝:汉历世二十有四,践年四百二十有六,四海困穷,三纲不立,五纬错行,灵祥并见,推术数者,虑之古道,咸以为天之历数,运终兹世,凡诸嘉祥民神之意,比昭有汉数终之极,魏家受命之符。汉主以神器宜授于臣,宪章有虞,致位于丕。丕震畏天命,虽休勿休。髃公庶尹六事之人,外及将士,洎于蛮夷君长,佥曰:‘天命不可以辞拒,神器不可以久旷,髃臣不可以无主,万几不可以无统。’丕祗承皇象,敢不钦承。卜之守龟,兆有大横,筮之三易,兆有革兆,谨择元日,与髃寮登坛受帝玺绶,告类于尔大神;唯尔有神,尚飨永吉,兆民之望,祚于有魏世享。”
遂制诏三公:“上古之始有君也,必崇恩化以美风俗,然百姓顺教而刑辟厝焉。今朕承帝王之绪,其以延康元年为黄初元年,议改正朔,易服色,殊徽号,同律度量,承土行,大赦天下;自殊死以下,诸不当得赦,皆赦除之。” Ui魏氏春秋曰:帝升坛礼毕,顾谓髃臣曰:
“舜、禹之事,吾知之矣。” Ui干窦搜神记曰:宋大夫邢史子臣明于天道,周敬王之三十七年,景公问曰:“天道其何祥?”对曰:“后五*(十)*年五月丁亥,臣将死;死后五年五月丁卯,吴将亡;亡后五年,君将终;终后四百年,邾王天下。”
俄而皆如其言。所云邾王天下者,谓魏之兴也。邾,曹姓,魏亦曹姓,皆邾之后。其年数则错,未知邢史失其数邪,将年代久远,注记者传而有谬也?
黄初元年十一月癸酉,以河内之山阳邑万户奉汉帝为山阳公,行汉正朔,以天子之礼郊祭,上书不称臣,京都有事于太庙,致胙;封公之四子为列侯。追尊皇祖太王曰太皇帝,考武王曰武皇帝,尊王太后曰皇太后。赐男子爵人一级,为父后及孝悌力田人二级。以汉诸侯王为崇德侯,列侯为关中侯。以颍阴之繁阳亭为繁昌县。封爵增位各有差。改相国为司徒,御史大夫为司空,奉常为太常,郎中令为光禄勋,大理为廷尉,大农为大司农。郡国县邑,多所改易。更授匈奴南单于呼厨泉魏玺绶,赐青盖车、乘舆、宝剑、玉玦。十二月,初营洛阳宫,戊午幸洛阳。[一]
注[一]臣松之案:诸书记是时帝居北宫,以建始殿朝髃臣,门曰承明,陈思王植诗曰“谒帝承明庐”是也。至明帝时,始于汉南宫崇德殿处起太极、昭阳诸殿。魏书曰:以夏数为得天,故即用夏正,而服色尚黄。魏略曰:诏以汉火行也,火忌水,故“洛”去“水”而加“佳”。
魏于行次为土,土,水之牡也,水得土而乃流,土得水而柔,故除“佳”加“水”,变“雒”为“洛”。
是岁,长水校尉戴陵谏不宜数行弋猎,帝大怒;陵减死罪一等。
二年春正月,郊祀天地、明堂。甲戌,校猎至原陵,遣使者以太牢祠汉世祖。乙亥,朝日于东郊。[一]初令郡国口满十万者,岁察孝廉一人;其有秀异,无拘户口。辛巳,分三公户邑,封子弟各一人为列侯。壬午,复颍川郡一年田租。[二]改许县为许昌县。以魏郡东部为阳平郡,西部为广平郡。[三]
注[一]臣松之以为礼天子以春分朝日,秋分夕月;寻此年正月郊祀,有月无日,乙亥朝日,则有日无月,盖文之脱也。案明帝朝日夕月,皆如礼文,故知此纪为误者也。
注[二]魏书载诏曰:“颍川,先帝所由起兵征伐也。官渡之役,四方瓦解,远近顾望,而此郡守义,丁壮荷戈,老弱负粮。昔汉祖以秦中为国本,光武恃河内为王基,今朕复于此登坛受禅,天以此郡翼成大魏。”
注[三]魏略曰:改长安、谯、许昌、邺、洛阳为五都;立石表,西界宜阳,北循太行,东北界阳平,南循鲁阳,东界郯,为中都之地。令天下听内徙,复五年,后又增其复。
诏曰:“昔仲尼资大圣之才,怀帝王之器,当衰周之末,无受命之运,在鲁、韂之朝,教化乎洙、泗之上,凄凄焉,遑遑焉,欲屈己以存道,贬身以救世。于时王公终莫能用之,乃退考五代之礼,修素王之事,因鲁史而制春秋,就太师而正雅颂,俾千载之后,莫不宗其文以述作,仰其圣以成谋,咨!可谓命世之大圣,亿载之师表者也。遭天下大乱,百祀堕坏,旧居之庙,毁而不修,褒成之后,绝而莫继,阙里不闻讲颂之声,四时不鷪蒸尝之位,斯岂所谓崇礼报功,盛德百世必祀者哉!其以议郎孔羡为宗圣侯,邑百户,奉孔子祀。”令鲁郡修起旧庙,置百户吏卒以守韂之,又于其外广为室屋以居学者。
*(春)*三月,加辽东太守公孙恭为车骑将军。初复五铢钱。夏四月,以车骑将军曹仁为大将军。五月,郑甘复叛,遣曹仁讨斩之。六月庚子,初祀五岳四渎,咸秩髃祀。[一]丁卯,夫人甄氏卒。戊辰晦,日有食之,有司奏免太尉,诏曰:“灾异之作,以谴元首,而归过股肱,岂禹、汤罪己之义乎?其令百官各虔厥职,后有天地之眚,勿复劾三公。”
注[一]魏书:甲辰,以京师宗庙未成,帝亲祠武皇帝于建始殿,躬执馈奠,如家人之礼。
秋八月,孙权遣使奉章,并遣于禁等还。丁巳,使太常邢贞持节拜权为大将军,封吴王,加九锡。冬十月,授杨彪光禄大夫。[一]以谷贵,罢五铢钱。[二]己卯,以大将军曹仁为大司马。十二月,行东巡。是岁筑陵云台。
注[一]魏书曰:己亥,公卿朝朔旦,并引故汉太尉杨彪,待以客礼,诏曰:“夫先王制几杖之赐,所以宾礼黄耇褒崇元老也。昔孔光、卓茂皆以淑德高年,受兹嘉锡。公故汉宰臣,乃祖已来,世著名节,年过七十,行不踰矩,可谓老成人矣,所宜宠异以章旧德。其赐公延年杖及冯几;谒请之日,便使杖入,又可使着鹿皮冠。”彪辞让不听,竟着布单衣、皮弁以见。
续汉书曰:彪见汉祚将终,自以累世为三公,耻为魏臣,遂称足挛,不复行。积十余年,帝即王位,欲以为太尉,令近臣宣旨。彪辞曰:“尝以汉朝为三公,值世衰乱,不能立尺寸之益,若复为魏臣,于国之选,亦不为荣也。”帝不夺其意。黄初四年,诏拜光禄大夫,秩中二千石,朝见位次三公,如孔光故事。彪上章固让,帝不听,又为门施行马,致吏卒,以优崇之。年八十四,以六年薨。子修,事见陈思王传。
注[二]魏书曰:十一月辛未,镇西将军曹真命众将及州郡兵讨破叛胡治元多、卢水、封赏等,斩首五万余级,获生口十万,羊一百一十一万口,牛八万,河西遂平。帝初闻胡决水灌显美,谓左右诸将曰:“昔隗嚣灌略阳,而光武因其疲弊,进兵灭之。今胡决水灌显美,其事正相似,破胡事今至不久。”旬日,破胡告檄到,上大笑曰:“吾策之于帷幕之内,诸将奋击于万里之外,其相应若合符节。前后战克获虏,未有如此也。”
三年春正月丙寅朔,日有蚀之。庚午,行幸许昌宫。诏曰:“今之计、*(考)**[孝]*,古之贡士也;十室之邑,必有忠信,若限年然后取士,是吕尚、周晋不显于前世也。其令郡国所选,勿拘老幼;儒通经术,吏达文法,到皆试用。有司纠故不以实者。”[一]
注[一]魏书曰:癸亥,孙权上书,说:“刘备支党四万人,马二三千匹,出秭归,请往扫扑,以克捷为效。”帝报曰:“昔隗嚣之弊,祸发栒邑,子阳之禽,变起扞关,将军其亢厉威武,勉蹈奇功,以称吾意。”
二月,鄯善、龟兹、于阗王各遣使奉献,诏曰:“西戎即□,氐、羌来王,诗、书美之。顷者西域外夷并款塞内附,[一]其遣使者抚劳之。”是后西域遂通,置戊己校尉。
注[一]应劭汉书注曰:款,叩也;皆叩塞门来服从。
三月乙丑,立齐公叡为平原王,帝弟鄢陵公彰等十一人皆为王。初制封王之庶子为乡公,嗣王之庶子为亭侯,公之庶子为亭伯。甲戌,立皇子霖为河东王。甲午,行幸襄邑。夏四月戊申,立鄄城侯植为鄄城王。癸亥,行还许昌宫。五月,以荆、扬、江表八郡为荆州,孙权领牧故也;荆州江北诸郡为郢州。
闰月,孙权破刘备于夷陵。初,帝闻备兵东下,与权交战,树栅连营七百余里,谓髃臣曰:
“备不晓兵,岂有七百里营可以拒敌者乎!‘苞原隰险阻而为军者为敌所禽’,此兵忌也。
孙权上事今至矣。”后七日,破备书到。
秋七月,冀州大蝗,民饥,使尚书杜畿持节开仓廪以振之。八月,蜀大将黄权率众降。[一]
注[一]魏书曰:权及领南郡太守史合等三百一十八人,诣荆州刺史奉上所假印绶、棨戟、幢麾、牙门、鼓车。权等诣行在所,帝置酒设乐,引见于承光殿。权、合等人人前自陈,帝为论说军旅成败去就之分,诸将无不喜悦。赐权金帛、车马、衣裘、帷帐、妻妾,下及偏裨皆有差。拜权为侍中镇南将军,封列侯,即日召使骖乘;及封史合等四十二人皆为列侯,为将军郎将百余人。
九月甲午,诏曰:“夫妇人与政,乱之本也。自今以后,髃臣不得奏事太后,后族之家不得当辅政之任,又不得横受茅土之爵;以此诏传后世,若有背违,天下共诛之。”[一]庚子,立皇后郭氏。赐天下男子爵人二级;□寡笃癃及贫不能自存者赐谷。
注[一]孙盛曰:夫经国营治,必凭俊箉之辅,贤达令德,必居参乱之任,故虽周室之盛,有妇人与焉。然则坤道承天,南面罔二,三从之礼,谓之至顺,至于号令自天子出,奏事专行,非古义也。昔在申、吕,实匡有周。苟以天下为心,惟德是杖,则亲簄之授,至公一也,何至后族而必斥远之哉?二汉之季世,王道陵迟,故令外戚凭宠,职为乱阶。*(于)*此自时昏道丧,运祚将移,纵无王、吕之难,岂乏田、赵之祸乎?而后世观其若此,深怀酖毒之戒也。
至于魏文,遂发一概之诏,可谓有识之爽言,非帝者之宏议。
冬十月甲子,表首阳山东为寿陵,作终制曰:“礼,国君即位为椑,*椑音扶历反。*存不忘亡也。[一]昔尧葬谷林,通树之,禹葬会稽,农不易亩,[二]故葬于山林,则合乎山林。封树之制,非上古也,吾无取焉。寿陵因山为体,无为封树,无立寝殿,造园邑,通神道。夫葬也者,藏也,欲人之不得见也。骨无痛痒之知,頉非栖神之宅,礼不墓祭,欲存亡之不黩也,为棺椁足以朽骨,衣衾足以朽肉而已。故吾营此丘墟不食之地,欲使易代之后不知其处。
无施苇炭,无藏金银铜铁,一以瓦器,合古涂车、刍灵之义。棺但漆际会三过,饭含无以珠玉,无施珠襦玉匣,诸愚俗所为也。季孙以玙璠敛,孔子历级而救之,譬之暴骸中原。宋公厚葬,君子谓华元、乐莒不臣,以为弃君于恶。汉文帝之不发,霸陵无求也;光武之掘,原陵封树也。霸陵之完,功在释之;原陵之掘,罪在明帝。是释之忠以利君,明帝爱以害亲也。
忠臣孝子,宜思仲尼、丘明、释之之言,鉴华元、乐莒、明帝之戒,存于所以安君定亲,使魂灵万载无危,斯则贤圣之忠孝矣。自古及今,未有不亡之国,亦无不掘之墓也。丧乱以来,汉氏诸陵无不发掘,至乃烧取玉匣金缕,骸骨并尽,是焚如之刑,岂不重痛哉!祸由乎厚葬封树。‘桑、霍为我戒’,不亦明乎?其皇后及贵人以下,不随王之国者,有终没皆葬涧西,前又以表其处矣。盖舜葬苍梧,二妃不从,延陵葬子,远在嬴、博,魂而有灵,无不之也,一涧之闲,不足为远。若违今诏,妄有所变改造施,吾为戮尸地下,戮而重戮,死而重死。臣子为蔑死君父,不忠不孝,使死者有知,将不福汝。其以此诏藏之宗庙,副在尚书、秘书、三府。”
注[一]臣松之按:礼,天子诸侯之棺,各有重数;棺之亲身者曰椑。
注[二]吕氏春秋:尧葬于谷林,通树之;舜葬于纪,市廛不变其肆;禹葬会稽,不变人徒。
是月,孙权复叛。复郢州为荆州。帝自许昌南征,诸军兵并进,权临江拒守。十一月辛丑,行幸宛。庚申晦,日有食之。是岁,穿灵芝池。
四年春正月,诏曰:“丧乱以来,兵革未戢,天下之人,互相残杀。今海内初定,敢有私复雠者皆族之。”筑南巡台于宛。三月丙申,行自宛还洛阳宫。癸卯,月犯心中央大星。[一]丁未,大司马曹仁薨。是月大疫。
注[一]魏书载丙午诏曰:“孙权残害民物,朕以寇不可长,故分命猛将三道并征。今征东诸军与权党吕范等水战,则斩首四万,获船万艘。大司马据守濡须,其所禽获亦以万数。中军、征南,攻围江陵,左将军张合等舳舻直渡,击其南渚,贼赴水溺死者数千人,又为地道攻城,城中外雀鼠不得出入,此几上肉耳!而贼中疠气疾病,夹江涂地,恐相染污。昔周武伐殷,旋师孟津,汉祖征隗嚣,还军高平,皆知天时而度贼情也。且成汤解三面之网,天下归仁。今开江陵之围,以缓成死之禽。且休力役,罢省繇戍,畜养士民,咸使安息。”
夏五月,有鹈鹕鸟集灵芝池,诏曰:“此诗人所谓污泽也。曹诗‘刺恭公远君子而近小人’,今岂有贤智之士处于下位乎?否则斯鸟何为而至?其博举天下鉨德茂才、独行君子,以答曹人之刺。”[一]
注[一]魏书曰:辛酉,有司奏造二庙,立太皇帝庙,大长秋特进侯与高祖合祭,亲尽以次毁;
特立武皇帝庙,四时享祀,为魏太祖,万载不毁也。
六月甲戌,任城王彰薨于京都。甲申,太尉贾诩薨。太白昼见。是月大雨,伊、洛溢流,杀人民,坏庐宅。[一]秋八月丁卯,以廷尉钟繇为太尉。[二]辛未,校猎于荥阳,遂东巡。论征孙权功,诸将已下进爵增户各有差。九月甲辰,行幸许昌宫。[三]
注[一]魏书曰:七月乙未,大军当出,使太常以特牛一告祠于郊。臣松之按:魏郊祀奏中,尚书卢毓议祀厉殃事云:“具牺牲祭器,如前后师出告郊之礼。”如此,则魏氏出师,皆告郊也。
注[二]魏书曰:有司奏改汉氏宗庙安世乐曰正世乐,嘉至乐曰迎灵乐,武德乐曰武颂乐,昭容乐曰昭业乐,云*(翻)**[翘]*舞曰凤翔舞,育命舞曰灵应舞,武德舞曰武颂舞,文*(昭)**[始]*舞曰大*(昭)**[韶]*舞,五行舞曰大武舞。
注[三]魏书曰:十二月丙寅,赐山阳公夫人汤沐邑,公女曼为长乐郡公主,食邑各五百户。
是冬,甘露降芳林园。臣松之按:芳林园即今华林园,齐王芳即位,改为华林。
五年春正月,初令谋反大逆乃得相告,其余皆勿听治;敢妄相告,以其罪罪之。三月,行自许昌还洛阳宫。夏四月,立太学,制五经课试之法,置春秋谷梁博士。五月,有司以公卿朝朔望日,因奏疑事,听断大政,论辨得失。秋七月,行东巡,幸许昌宫。八月,为水军,亲御龙舟,循蔡、颍,浮淮,幸寿春。扬州界将吏士民,犯五岁刑已下,皆原除之。九月,遂至广陵,赦青、徐二州,改易诸将守。冬十月乙卯,太白昼见。行还许昌宫。[一]十一月庚寅,以冀州饥,遣使者开仓廪振之。戊申晦,日有食之。
注[一]魏书载癸酉诏曰:“近之不绥,何远之怀?今事多而民少,上下相弊以文法,百姓无所措其手足。昔太山之哭者,以为苛政甚于猛虎,吾备儒者之风,服圣人之遗教,岂可以目翫其辞,行违其诫者哉?广议轻刑,以惠百姓。”
十二月,诏曰:“先王制礼,所以昭孝事祖,大则郊社,其次宗庙,三辰五行,名山大川,非此族也,不在祀典。叔世衰乱,崇信巫史,至乃宫殿之内,户牖之闲,无不沃酹,甚矣其惑也。自今,其敢设非祀之祭,巫祝之言,皆以执左道论,着于令典。”是岁穿天渊池。
六年春二月,遣使者循行许昌以东尽沛郡,问民所疾苦,贫者振贷之。[一]三月,行幸召陵,通讨虏渠。乙巳,还许昌宫。并州刺史梁习讨鲜卑轲比能,大破之。辛未,帝为舟师东征。五月戊申,幸谯。壬戌,荧惑入太微。
注[一]魏略载诏曰:“昔轩辕建四面之号,周武称‘予有乱臣十人’,斯盖先圣所以体国君民,亮成天工,多贤为贵也。今内有公卿以镇京师,外设牧伯以监四方,至于元戎出征,则军中宜有柱石之贤帅,辎重所在,又宜有镇守之重臣,然后车驾可以周行天下,无内外之虑。吾今当征贼,欲守之积年。其以尚书令颍乡侯陈髃为镇军大将军,尚书仆射西乡侯司马懿为抚军大将军。若吾临江授诸将方略,则抚军当留许昌,督后诸军,录后台文书事;镇军随车驾,当董督众军,录行尚书事;皆假节鼓吹,给中军兵骑六百人。吾欲去江数里,筑宫室,往来其中,见贼可击之形,便出奇兵击之;若或未可,则当舒六军以游猎,飨赐军士。”
六月,利成郡兵蔡方等以郡反,杀太守徐质。遣屯骑校尉任福、步兵校尉段昭与青州刺史讨平之;其见胁略及亡命者,皆赦其罪。
秋七月,立皇子鉴为东武阳王。八月,帝遂以舟师自谯循涡入淮,从陆道幸徐。九月,筑东巡台。冬十月,行幸广陵故城,临江观兵,戎卒十余万,旌旗数百里。[一]是岁大寒,水道冰,舟不得入江,乃引还。十一月,东武阳王鉴薨。十二月,行自谯过梁,遣使以太牢祀故汉太尉桥玄。
注[一]魏书载帝于马上为诗曰:“观兵临江水,水流何汤汤!戈矛成山林,玄甲耀日光。猛将怀暴怒,胆气正从横。谁云江水广,一苇可以航,不战屈敌虏,戢兵称贤良。古公宅岐邑,实始翦殷商。孟献营虎牢,郑人惧稽颡。充国务耕植,先零自破亡。兴农淮、泗间,筑室都徐方。量宜运权略,六军咸悦康;岂如东山诗,悠悠多忧伤。”
七年春正月,将幸许昌,许昌城南门无故自崩,帝心恶之,遂不入。壬子,行还洛阳宫。三月,筑九华台。夏五月丙辰,帝疾笃,召中军大将军曹真、镇军大将军陈髃、征东大将军曹休、抚军大将军司马宣王,并受遗诏辅嗣主。遣后宫淑媛、昭仪已下归其家。丁巳,帝崩于嘉福殿,时年四十。[一]六月戊寅,葬首阳陵。自殡及葬,皆以终制从事。[二]
注[一]魏书曰:殡于崇华前殿。
注[二]魏氏春秋曰:明帝将送葬,曹真、陈髃、王朗等以暑热固谏,乃止。孙盛曰:夫窀穸之事,孝子之极痛也,人伦之道,于斯莫重。故天子七月而葬,同轨毕至。夫以义感之情,犹尽临隧之哀,况乎天性发中,敦礼者重之哉!魏氏之德,仍世不基矣。昔华元厚葬,君子以为弃君于恶,髃等之谏,弃孰甚焉!鄄城侯植为诔曰:“惟黄初七年五月七日,大行皇帝崩,呜呼哀哉!于时天震地骇,崩山陨霜,阳精薄景,五纬错行,百姓呼嗟,万国悲伤,若丧考妣,*(恩过慕)**[思慕过]*唐,擗踊郊野,仰想穹苍,佥曰何辜,早世殒丧,呜呼哀哉!
悲夫大行,忽焉光灭,永弃万国,云往雨绝。承问荒忽,惛懵哽咽,袖锋抽刃,叹自僵毙,追慕三良,甘心同穴。感惟南风,惟以郁滞,终于偕没,指景自誓。考诸先记,寻之哲言,生若浮寄,唯德可论,朝闻夕逝,孔志所存。皇虽一没,天禄永延,何以述德?表之素旃。
何以咏功?宣之管弦。乃作诔曰:皓皓太素,两仪始分,中和产物,肇有人伦,爰暨三皇,实秉道真,降逮五帝,继以懿纯,三代制作,踵武立勋。季嗣不维,网漏于秦,崩乐灭学,儒坑礼焚,二世而歼,汉氏乃因,弗求古训,嬴政是遵,王纲帝典,阒尔无闻。末光幽昧,道究运迁,乾坤回历,简圣授贤,乃眷大行,属以黎元。龙飞启祚,合契上玄,五行定纪,改号革年,明明赫赫,受命于天。
仁风偃物,德以礼宣;祥惟圣质,嶷在幼妍。庶几六典,学不过庭,潜心无罔,抗志青冥。
才秀藻朗,如玉之莹,听察无向,瞻鷪未形。其刚如金,其贞如琼,如冰之洁,如砥之平。
爵公无私,戮违无轻,心镜万机,揽照下情。思良股肱,嘉昔伊、吕,搜扬侧陋,举汤代禹;
拔才岩穴,取士蓬户,唯德是萦,弗拘祢祖。宅土之表,道义是图,弗营厥险,六合是虞。
齐契共遵,下以纯民,恢拓规矩,克绍前人。科条品制,曪贬以因。乘殷之辂,行夏之辰。
金根黄屋,翠葆龙鳞,绋冕崇丽,衡紞维新,尊肃礼容,瞩之若神。方牧妙举,钦于恤民,虎将荷节,镇彼四邻;朱旗所剿,九壤被震,畴克不若?孰敢不臣?县旌海表,万里无尘。
虏备凶彻,鸟殪江岷,权若涸鱼,干腊矫鳞,肃慎纳贡,越裳效珍,条支绝域,侍子内宾。
德侪先皇,功侔太古。上灵降瑞,黄初叔祜:河龙洛龟,凌波游下;平钧应绳,神鸾翔舞;
数荚阶除,系风扇暑;皓兽素禽,飞走郊野;神钟宝鼎,形自旧土;云英甘露,瀸涂被宇;
灵芝冒沼,朱华荫渚。回回凯风,祁祁甘雨,稼穑丰登,我稷我黍。家佩惠君,户蒙慈父。
图致太和,洽德全义。将登介山,先皇作俪。镌石纪勋,兼录众瑞,方隆封禅,归功天地,宾礼百灵,勋命视规,望祭四岳,燎封奉柴,肃于南郊,宗祀上帝。三牲既供,夏禘秋尝,元侯佐祭,献璧奉璋。鸾舆幽蔼,龙旗太常,爰迄太庙,钟鼓锽锽,颂德咏功,八佾锵锵。
皇祖既飨,烈考来享,神具醉止,降兹福祥。天地震荡,大行康之;三辰暗昧,大行光之;
皇纮绝维,大行纲之;神器莫统,大行当之;礼乐废弛,大行张之;仁义陆沉,大行扬之;
潜龙隐凤,大行翔之;疏狄遐康,大行匡之。在位七载,元功仍举,将永太和,绝迹三五,宜作物师,长为神主,寿终金石,等算东父,如何奄忽,摧身后土,俾我□□,靡瞻靡顾。
嗟嗟皇穹,胡宁忍务?呜呼哀哉!明监吉凶,体远存亡,深垂典制,申之嗣皇。圣上虔奉,是顺是将,乃□玄宇,基为首阳,拟夡谷林,追尧慕唐,合山同陵,不树不疆,涂车刍灵,珠玉靡藏。百神警侍,来宾幽堂,耕禽田兽,望魂之翔。于是,俟大隧之致功兮,练元辰之淑祯,潜华体于梓宫兮,冯正殿以居灵。顾望嗣之号咷兮,存临者之悲声,悼晏驾之既修兮,感容车之速征。浮飞魂于轻霄兮,就黄墟以灭形,背三光之昭晰兮,归玄宅之冥冥。嗟一往之不反兮,痛閟闼之长扃。咨远臣之眇眇兮,感凶讳以怛惊,心孤绝而靡告兮,纷流涕而交颈。思恩荣以横奔兮,阂阙塞之峣峥,顾衰绖以轻举兮,迫关防之我婴。欲高飞而遥憩兮,惮天网之远经,遥投骨于山足兮,报恩养于下庭。慨拊心而自悼兮,惧施重而命轻,嗟微驱之是效兮,甘九死而忘生,几司命之役籍兮,先黄发而陨零,天盖高而察卑兮,冀神明之我听。独郁伊而莫愬兮,追顾景而怜形,奏斯文以写思兮,结翰墨以敷诚。呜呼哀哉!”
初,帝好文学,以著述为务,自所勒成垂百篇。又使诸儒撰集经传,随类相从,凡千余篇,号曰皇览。[一]
注[一]魏书曰:帝初在东宫,疫疠大起,时人雕伤,帝深感叹,与素所敬者大理王朗书曰:
“生有七尺之形,死唯一棺之土,唯立德扬名,可以不朽,其次莫如着篇籍。疫疠数起,士人雕落,余独何人,能全其寿?”故论撰所着典论、诗赋,盖百余篇,集诸儒于肃城门内,讲论大义,侃侃无倦。常嘉汉文帝之为君,宽仁玄默,务欲以德化民,有贤圣之风。时文学诸儒,或以为孝文虽贤,其于聪明,通达国体,不如贾谊。帝由是着太宗论曰:“昔有苗不宾,重华舞以干戚,尉佗称帝,孝文抚以恩德,吴王不朝,锡之几杖以抚其意,而天下赖安;
乃弘三章之教,恺悌之化,欲使曩时累息之民,得阔步高谈,无危惧之心。若贾谊之才敏,筹画国政,特贤臣之器,管、晏之姿,岂若孝文大人之量哉?”三年之中,以孙权不服,复颁太宗论于天下,明示不愿征伐也。他日又从容言曰:“顾我亦有所不取于汉文帝者三:杀薄昭;幸邓通;慎夫人衣不曳地,集上书囊为帐帷。以为汉文俭而无法,舅后之家,但当养育以恩而不当假借以权,既触罪法,又不得不害矣。”其欲秉持中道,以为帝王仪表者如此。胡冲吴历曰:帝以素书所着典论及诗赋饷孙权,又以纸写一通与张昭。
评曰:文帝天资文藻,下笔成章,博闻强识,才蓺兼该;[一]若加之旷大之度,励以公平之诚,迈志存道,克广德心,则古之贤主,何远之有哉!
注[一]典论帝自□曰:初平之元,董卓杀主鸩后,荡覆王室。是时四海既困中平之政,兼恶卓之凶逆,家家思乱,人人自危。山东牧守,咸以春秋之义,“韂人讨州吁于濮”,言人人皆得讨贼。于是大兴义兵,名豪大侠,富室强族,飘扬云会,万里相赴;兖、豫之师战于荥阳,河内之甲军于孟津。卓遂迁大驾,西都长安。而山东大者连郡国,中者婴城邑,小者聚阡陌,以还相吞灭。会黄巾盛于海、岱,山寇暴于并、冀,乘胜转攻,席卷而南,乡邑望烟而奔,城郭鷪尘而溃,百姓死亡,暴骨如莽。余时年五岁,上以世方扰乱,教余学射,六岁而知射,又教余骑马,八岁而能骑射矣。以时之多故,每征,余常从。建安初,上南征荆州,至宛,张绣降。旬日而反,亡兄孝廉子修、从兄安民遇害。时余年十岁,乘马得脱。夫文武之道,各随时而用,生于中平之季,长于戎旅之间,是以少好弓马,于今不衰;逐禽辄十里,驰射常百步,日多体健,心每不厌。建安十年,始定冀州,濊、貊贡良弓,燕、代献名马。时岁之暮春,勾芒司节,和风扇物,弓燥手柔,草浅兽肥,与族兄子丹猎于邺西,终日手获□鹿九,雉兔三十。后军南征次曲蠡,尚书令荀彧奉使犒军,见余谈论之末,彧言:“闻君善左右射,此实难能。”余言:“执事未鷪夫项发口纵,俯马蹄而仰月支也。”彧喜笑曰:“乃尔!”
余曰:“埒有常径,的有常所,虽每发辄中,非至妙也。若驰平原,赴丰草,要狡兽,截轻禽,使弓不虚弯,所中必洞,斯则妙矣。”时军祭酒张京在坐,顾彧拊手曰“善”。余又学击剑,阅师多矣,四方之法各异,唯京师为善。桓、灵之间,有虎贲王越善斯术,称于京师。河南史阿言昔与越游,具得其法,余从阿学之精熟。尝与平虏将军刘勋、奋威将军邓展等共饮,宿闻展善有手臂,晓五兵,又称其能空手入白刃。余与论剑良久,谓言将军法非也,余顾尝好之,又得善术,因求与余对。时酒酣耳热,方食芊蔗,便以为杖,下殿数交,三中其臂,左右大笑。展意不平,求更为之。余言吾法急属,难相中面,故齐臂耳。展言愿复一交,余知其欲突以取交中也,因伪深进,展果寻前,余却脚鄛,正截其颡,坐中惊视。余还坐,笑曰:“昔阳庆使淳于意去其故方,更授以秘术,今余亦愿邓将军捐弃故伎,更受要道也。”一坐尽欢。夫事不可自谓己长,余少晓持复,自谓无对;俗名双戟为坐铁室,镶楯为蔽木户;后从陈国袁敏学,以单攻复,每为若神,对家不知所出,先日若逢敏于狭路,直决耳!余于他戏弄之事少所喜,唯弹澙略尽其巧,少为之赋。昔京师先工有马合乡侯、东方安世、张公子,常恨不得与彼数子者对。上雅好诗书文籍,虽在军旅,手不释卷,每每定省从容,常言人少好学则思专,长则善忘,长大而能勤学者,唯吾与袁伯业耳。
余是以少诵诗、论,及长而备历五经、四部,史、汉、诸子百家之言,靡不毕览。博物志曰:
帝善弹澙,能用手巾角。时有一书生,又能低头以所冠着葛巾角撇澙。 
\end{yuanwen}

\part{魏书三}

\chapter{明帝纪第三}

\begin{yuanwen}
明皇帝讳叡,字符仲,文帝太子也。生而太祖爱之,常令在左右。[一]年十五,封武德侯,黄初二年为齐公,三年为平原王。以其母诛,故未建为嗣。[二]七年夏五月,帝病笃,乃立为皇太子。丁巳,即皇帝位,大赦。尊皇太后曰太皇太后,皇后曰皇太后。诸臣封爵各有差。
[三]癸未,追谥母甄夫人曰文昭皇后。壬辰,立皇弟蕤为阳平王。
注[一]魏书曰:帝生数岁而有岐嶷之姿,武皇帝异之,曰:“我基于尔三世矣。”每朝宴会同,与侍中近臣并列帷幄。好学多识,特留意于法理。
注[二]魏略曰:文帝以郭后无子,诏使子养帝。帝以母不以道终,意甚不平。后不获已,乃敬事郭后,旦夕因长御问起居,郭后亦自以无子,遂加慈爱。文帝始以帝不悦,有意欲以他姬子京兆王为嗣,故久不拜太子。魏末传曰:帝常从文帝猎,见子母鹿。文帝射杀鹿母,使帝射鹿子,帝不从,曰:“陛下已杀其母,臣不忍复杀其子。”因涕泣。文帝即放弓箭,以此深奇之,而树立之意定。
注[三]世语曰:帝与朝士素不接,即位之后,髃下想闻风采。居数日,独见侍中刘晔,语尽日。众人侧听,晔既出,问“何如”?晔曰:“秦始皇、汉孝武之俦,才具微不及耳。”
八月,孙权攻江夏郡,太守文聘坚守。朝议欲发兵救之,帝曰:“权习水战,所以敢下船陆攻者,几掩不备也。今已与聘相持,夫攻守势倍,终不敢久也。”先时遣治书侍御史荀禹慰劳边方,禹到,于江夏发所经县兵及所从步骑千人乘山举火,权退走。
辛巳,立皇子冏为清河王。吴将诸葛瑾、张霸等寇襄阳,抚军大将军司马宣王讨破之,斩霸,征东大将军曹休又破其别将于寻阳。论功行赏各有差。冬十月,清河王冏薨。十二月,以太尉钟繇为太傅,征东大将军曹休为大司马,中军大将军曹真为大将军,司徒华歆为太尉,司空王朗为司徒,镇军大将军陈髃为司空,抚军大将军司马宣王为骠骑大将军。
太和元年春正月,郊祀武皇帝以配天,宗祀文皇帝于明堂以配上帝。分江夏南部,置江夏南部都尉。西平曲英反,杀临羌令、西都长,遣将军郝昭、鹿盘讨斩之。二月辛未,帝耕于籍田。辛巳,立文昭皇后寝庙于邺。丁亥,朝日于东郊。夏四月乙亥,行五铢钱。甲申,初营宗庙。秋八月,夕月于西郊。冬十月丙寅,治兵于东郊。焉耆王遣子入侍。十一月,立皇后毛氏。赐天下男子爵人二级,□寡孤独不能自存者赐谷。十二月,封后父毛嘉为列侯。新城太守孟达反,诏骠骑将军司马宣王讨之。[一]
注[一]三辅决录曰:伯郎,凉州人,名不令休。其注曰:伯郎姓孟,名他,扶风人。灵帝时。
中常侍张让专朝政,让监奴典护家事。他仕不遂,乃尽以家财赂监奴,与共结亲,积年家业为之破尽。众奴皆惭,问他所欲,他曰:“欲得卿曹拜耳。”奴被恩久,皆许诺。时宾客求见让者,门下车常数百乘,或累日不得通。他最后到,众奴伺其至,皆迎车而拜,径将他车独入。众人悉惊,谓他与让善,争以珍物遗他。他得之,尽以赂让,让大喜。他又以蒲桃酒一斛遗让,即拜凉州刺史。
他生达,少入蜀。其处蜀事夡在刘封传。魏略曰:达以延康元年率部曲四千余家归魏。文帝时初即王位,既宿知有达,闻其来,甚悦,令贵臣有识察者往观之,还曰“将帅之才也”,或曰“卿相之器也”,王益钦达。逆与达书曰:“近日有命,未足达旨,何者?昔伊挚背商而归周,百里去虞而入秦,乐毅感鸱夷以蝉蜕,王遵识逆顺以去就,皆审兴废之符效,知成败之必然,故丹青画其形容,良史载其功勋。闻卿姿度纯茂,器量优绝,当骋能明时,收名传记。今者翻然濯鳞清流,甚相嘉乐,虚心西望,依依若旧,下笔属辞,欢心从之。昔虞卿入赵,再见取相,陈平就汉,一觐参乘,孤今于卿,情过于往,故致所御马物以昭忠爱。”又曰:“今者海内清定,万里一统,三垂无边尘之警,中夏无狗吠之虞,以是弛罔阔禁,与世无疑,保官空虚,初无*(资)**[质]*任。卿来相就,当明孤意,慎勿令家人缤纷道路,以亲骇簄也。若卿欲来相见,且当先安部曲,有所保固,然后徐徐轻骑来东。”达既至谯,进见闲雅,才辩过人,众莫不属目。又王近出,乘小辇,执达手,抚其背戏之曰:“卿得无为刘备刺客邪?”遂与同载。又加拜散骑常侍,领新城太守,委以西南之任。时众臣或以为待之太猥,又不宜委以方任。王闻之曰:“吾保其无他,亦譬以蒿箭射蒿中耳。”达既为文帝所宠,又与桓阶、夏侯尚亲善,及文帝崩,时桓、尚皆卒,达自以羁旅久在疆埸,心不自安。
诸葛亮闻之,阴欲诱达,数书招之,达与相报答。魏兴太守申仪与达有隙,密表达与蜀潜通,帝未之信也。司马宣王遣参军梁几察之,又劝其入朝。达惊惧,遂反。
干宝晋纪曰:达初入新城,登白马塞,叹曰:“刘封、申耽,据金城千里而失之乎!”
二年春正月,宣王攻破新城,斩达,传其首。[一]分新城之上庸、武陵、巫县为上庸郡,锡县为锡郡。
注[一]魏略曰:宣王诱达将李辅及达甥邓贤,贤等开门纳军。达被围旬有六日而败,焚其首于洛阳四达之衢。
蜀大将诸葛亮寇边,天水、南安、安定三郡吏民叛应亮。[一]遣大将军曹真都督关右,并进兵。右将军张合击亮于街亭,大破之。亮败走,三郡平。丁未,行幸长安。[二]夏四月丁酉,还洛阳宫。[三]赦系囚非殊死以下。乙巳,论讨亮功,封爵增邑各有差。五月,大旱。六月,诏曰:“尊儒贵学,王教之本也。自顷儒官或非其人,将何以宣明圣道?其高选博士,才任侍中常侍者。申敕郡国,贡士以经学为先。”秋九月,曹休率诸军至皖,与吴将陆议战于石亭,败绩。乙酉,立皇子穆为繁阳王。庚子,大司马曹休薨。冬十月,诏公卿近臣举良将各一人。十一月,司徒王朗薨。十二月,诸葛亮围陈仓,曹真遣将军费曜等拒之。[四]辽东太守公孙恭兄子渊,劫夺恭位,遂以渊领辽东太守。
注[一]魏书曰:是时朝臣未知计所出,帝曰:“亮阻山为固,今者自来,既合兵书致人之术;
且亮贪三郡,知进而不知退,今因此时,破亮必也。”乃部勒兵马步骑五万拒亮。
注[二]魏略载帝露布天下并班告益州曰:“刘备背恩,自窜巴蜀。诸葛亮弃父母之国,阿残贼之党,神人被毒,恶积身灭。亮外慕立孤之名,而内贪专擅之实。刘升之兄弟守空城而己。亮又侮易益土,虐用其民,是以利狼、宕渠、高定、青羌莫不瓦解,为亮仇敌。而亮反裘负薪,里尽毛殚,刖趾适屦,刻肌伤骨,反更称说,自以为能。行兵于井底,游步于牛蹄。自朕即位,三边无事,犹哀怜天下数遭兵革,且欲养四海之耆老,长后生之孤幼,先移风于礼乐,次讲武于农隙,置亮画外,未以为虞。而亮怀李熊愚勇之*(智)*[志],不思荆邯度德之戒,驱略吏民,盗利祁山。
王师方振,胆破气夺,马谡、高祥,望旗奔败。虎臣逐北,蹈尸涉血,亮也小子,震惊朕师。
猛锐踊跃,咸思长驱。朕惟率土莫非王臣,师之所处,荆棘生焉,不欲使千室之邑忠信贞良,与夫淫昏之党,共受涂炭。故先开示,以昭国诚,勉思变化,无滞乱邦。巴蜀将吏士民诸为亮所劫迫,公卿已下皆听束手。”
注[三]魏略曰:是时斗言,云帝已崩,从驾髃臣迎立雍丘王植。京师自卞太后髃公尽惧。及帝还,皆私察颜色。卞太后悲喜,欲推始言者,帝曰:“天下皆言,将何所推?”
注[四]魏略曰:先是,使将军郝昭筑陈仓城;会亮至,围昭,不能拔。昭字伯道,太原人,为人雄壮,少入军为部曲督,数有战功,为杂号将军,遂镇守河西十余年,民夷畏服。亮围陈仓,使昭乡人靳详于城外遥说之,昭于楼上应详曰:“魏家科法,卿所练也;我之为人,卿所知也。我受国恩多而门户重,卿无可言者,但有必死耳。卿还谢诸葛,便可攻也。”详以昭语告亮,亮又使详重说昭,言人兵不敌,无为空自破灭。昭谓详曰:“前言已定矣。我识卿耳,箭不识也。”详乃去。亮自以有众数万,而昭兵纔千余人,又度东救未能便到,乃进兵攻昭,起云梯冲车以临城。昭于是以火箭逆射其云梯,梯然,梯上人皆烧死。昭又以绳连石磨压其冲车,冲车折。亮乃更为井阑百尺以射城中,以土丸填堑,欲直攀城,昭又于内筑重墙。亮又为地突,欲踊出于城里,昭又于城内穿地横截之。昼夜相攻拒二十余日,亮无计,救至,引退。诏嘉昭善守,赐爵列侯。及还,帝引见慰劳之,顾谓中书令孙资曰:“卿乡里乃有尔曹快人,为将灼如此,朕复何忧乎?”仍欲大用之。会病亡,遗令戒其子凯曰:“吾为将,知将不可为也。吾数发冢,取其木以为攻战具,又知厚葬无益于死者也。汝必敛以时服。且人生有处所耳,死复何在耶?今去本墓远,东西南北,在汝而已。”
三年夏四月,元城王礼薨。六月癸卯,繁阳王穆薨。戊申,追尊高祖大长秋曰高皇帝,夫人吴氏曰高皇后。
秋七月,诏曰:“礼,王后无嗣,择建支子以继大宗,则当纂正统而奉公义,何得复顾私亲哉!汉宣继昭帝后,加悼考以皇号;哀帝以外藩援立,而董宏等称引亡秦,惑误时朝,既尊恭皇,立庙京都,又宠藩妾,使比长信,□昭穆于前殿,并四位于东宫,僭差无度,人神弗佑,而非罪师丹忠正之谏,用致丁、傅焚如之祸。自是之后,相踵行之。昔鲁文逆祀,罪由夏父;宋国非度,讥在华元。其令公卿有司,深以前世行事为戒。后嗣万一有由诸侯入奉大统,则当明为人后之义;敢为佞邪导谀时君,妄建非正之号以干正统,谓考为皇,称妣为后,则股肱大臣,诛之无赦。其书之金策,藏之宗庙,着于令典。”
冬十月,改平望观曰听讼观。帝常言“狱者,天下之性命也”,每断大狱,常幸观临听之。
初,洛阳宗庙未成,神主在邺庙。十一月,庙始成,使太常韩暨持节迎高皇帝、太皇帝、武帝、文帝神主于邺,十二月己丑至,奉安神主于庙。[一]
注[一]臣松之按:黄初四年,有司奏立二庙,太皇帝大长秋与文帝之高祖共一庙,特立武帝庙,百世不毁。今此无高祖神主,盖以亲尽毁也。此则魏初唯立亲庙,祀四室而已。至景初元年,始定七庙之制。孙盛曰:事亡犹存,祭如神在,迎迁神主,正斯宜矣。
癸卯,大月氏王波调遣使奉献,以调为亲魏大月氏王。
四年春二月壬午,诏曰:“世之质文,随教而变。兵乱以来,经学废绝,后生进趣,不由典谟。岂训导未洽,将进用者不以德显乎?其郎吏学通一经,才任牧民,博士课试,擢其高第者,亟用;其浮华不务道本者,皆罢退之。”戊子,诏太傅三公:以文帝典论刻石,立于庙门之外。癸巳,以大将军曹真为大司马,骠骑将军司马宣王为大将军,辽东太守公孙渊为车骑将军。夏四月,太傅钟繇薨。六月戊子,太皇太后崩。丙申,省上庸郡。秋七月,武宣卞后祔葬于高陵。诏大司马曹真、大将军司马宣王伐蜀。八月辛巳,行东巡,遣使者以特牛祠中岳。[一]乙未,幸许昌宫。九月,大雨,伊、洛、河、汉水溢,诏真等班师。冬十月乙卯,行还洛阳宫。庚申,令:“罪非殊死听赎各有差。”十一月,太白犯岁星。十二月辛未,改葬文昭甄后于朝阳陵。丙寅,诏公卿举贤良。
注[一]魏书曰:行过繁昌,使执金吾臧霸行太尉事,以特牛祠受禅坛。
臣松之按:汉纪章帝元和三年,诏高邑县祠即位坛,五成陌,比腊祠门户。此虽前代已行故事,然为坛以祀天,而坛非神也,今无事于上帝,而致祀于虚坛,求之义典,未详所据。
五年春正月,帝耕于籍田。三月,大司马曹真薨。诸葛亮寇天水,诏大将军司马宣王拒之。
自去冬十月至此月不雨,辛巳,大雩。夏四月,鲜卑附义王轲比能率其种人及丁零大人儿禅诣幽州贡名马。复置护匈奴中郎将。秋七月丙子,以亮退走,封爵增位各有差。[一]乙酉,皇子殷生,大赦。
注[一]魏书曰:初,亮出,议者以为亮军无辎重,粮必不继,不击自破,无为劳兵;或欲自芟上邽左右生麦以夺贼食,帝皆不从。前后遣兵增宣王军,又敕使护麦。宣王与亮相持,赖得此麦以为军粮。
八月,诏曰:“古者诸侯朝聘,所以敦睦亲亲协和万国也。先帝着令,不欲使诸王在京都者,谓幼主在位,母后摄政,防微以渐,关诸盛衰也。朕惟不见诸王十有二载,悠悠之怀,能不兴思!其令诸王及宗室公侯各将适子一人朝。后有少主、母后在宫者,自如先帝令,申明着于令。”冬十一月乙酉,月犯轩辕大星。戊戌晦,日有蚀之。十二月甲辰,月犯镇星。戊午,太尉华歆薨。
六年春二月,诏曰:“古之帝王,封建诸侯,所以藩屏王室也。诗不云乎,‘怀德维宁,宗子维城’。秦、汉继周,或强或弱,俱失厥中。大魏创业,诸王开国,随时之宜,未有定制,非所以永为后法也。其改封诸侯王,皆以郡为国。”三月癸酉,行东巡,所过存问高年□寡孤独,赐谷帛。乙亥,月犯轩辕大星。夏四月壬寅,行幸许昌宫。甲子,初进新果于庙。五月,皇子殷薨,追封谥安平哀王。秋七月,以韂尉董昭为司徒。九月,行幸摩陂,治许昌宫,起景福、承光殿。冬十月,殄夷将军田豫帅众讨吴将周贺于成山,杀贺。十一月丙寅,太白昼见。有星孛于翼,近太微上将星。庚寅,陈思王植薨。十二月,行还许昌宫。
青龙元年春正月甲申,青龙见郏之摩陂井中。二月丁酉,幸摩陂观龙,于是改年;改摩陂为龙陂,赐男子爵人二级,□寡孤独无出今年租赋。三月甲子,诏公卿举贤良笃行之士各一人。
夏五月壬申,诏祀故大将军夏侯惇、大司马曹仁、车骑将军程昱于太祖庙庭。[一]戊寅,北海王蕤薨。闰月庚寅朔,日有蚀之。丁酉,改封宗室女非诸王女皆为邑主。诏诸郡国山川不在祠典者勿祠。六月,洛阳宫鞠室灾。
注[一]魏书载诏曰:“昔先王之礼,于功臣存则显其爵禄,没则祭于大蒸,故汉氏功臣,祀于庙庭。大魏元功之臣功勋优着,终始休明者,其皆依礼祀之。”于是以惇等配飨。
保塞鲜卑大人步度根与叛鲜卑大人轲比能私通,并州刺史毕轨表,辄出军以外威比能,内镇步度根。帝省表曰:“步度根以为比能所诱,有自疑心。今轨出军,适使二部惊合为一,何所威镇乎?”促敕轨,以出军者慎勿越塞过句注也。比诏书到,轨以进军屯阴馆,遣将军苏尚、董弼追鲜卑。比能遣子将千余骑迎步度根部落,与尚、弼相遇,战于楼烦,二将*[败]*没。步度根部落皆叛出塞,与比能合寇边。遣骁骑将军秦朗将中军讨之,虏乃走漠北。
秋九月,安定保塞匈奴大人胡薄居姿职等叛,司马宣王遣将军胡遵等追讨,破降之。
冬十月,步度根部落大人戴胡阿狼泥等诣并州降,朗引军还。[一]
注[一]魏氏春秋曰:朗字符明,新兴人。献帝传曰:朗父名宜禄,为吕布使诣袁术,术妻以汉宗室女。其前妻杜氏留下邳。布之被围,关羽屡请于太祖,求以杜氏为妻,太祖疑其有色,及城陷,太祖见之,乃自纳之。宜禄归降,以为铚长。及刘备走小沛,张飞随之,过谓宜禄曰:“人取汝妻,而为之长,乃蚩蚩若是邪!随我去乎?”宜禄从之数里,悔欲还,飞杀之。
朗随母氏畜于公宫,太祖甚爱之,每坐席,谓宾客曰:“世有人爱假子如孤者乎?”魏略曰:
朗游遨诸侯间,历武、文之世而无尤也。及明帝即位,授以内官,为骁骑将军、给事中,每车驾出入,朗常随从。时明帝喜发举,数有以轻微而致大辟者,朗终不能有所谏止,又未尝进一善人,帝亦以是亲爱;每顾问之,多呼其小字阿稣,数加赏赐,为起大第于京城中。四方虽知朗无能为益,犹以附近至尊,多赂遗之,富均公侯。世语曰:朗子秀,劲厉能直言,为晋武帝博士。魏略以朗与孔桂俱在佞幸篇。桂字叔林,天水人也。建安初,数为将军杨秋使诣太祖,太祖表拜骑都尉。桂性便辟,晓博弈、□鞠,故太祖爱之,每在左右,出入随从。
桂察太祖意,喜乐之时,因言次曲有所陈,事多见从,数得赏赐,人多馈遗,桂由此侯服玉食。太祖既爱桂,五官将及诸侯亦皆亲之。其后桂见太祖久不立太子,而有意于临菑侯,因更亲附临菑侯而简于五官将,将甚衔之。及太祖薨,文帝即王位,未及致其罪。黄初元年,随例转拜驸马都尉。而桂私受西域货赂,许为人事。事发,有诏收问,遂杀之。鱼豢曰:为上者不虚授,处下者不虚受,然后外无伐檀之叹,内无尸素之刺,雍熙之美着,太平之律显矣。而佞幸之徒,但姑息人主,至乃无德而荣,无功而禄,如是焉得不使中正日朘,倾邪滋多乎!以武皇帝之慎赏,明皇帝之持法,而犹有若此等人,而况下斯者乎?
十二月,公孙渊斩送孙权所遣使张弥、许晏首,以渊为大司马乐浪公。[一]
注[一]世语曰:并州刺史毕轨送汉故度辽将军范明友鲜卑奴,年三百五十岁,言语饮食如常人。奴云:“霍显,光后小妻。明友妻,光前妻女。”博物志曰:时京邑有一人,失其姓名,食啖兼十许人,遂肥不能动。其父曾作远方长吏,官徙送彼县,令故义传供食之;一二年中,一乡中辄为之俭。傅子曰:时太原发頉破棺,棺中有一生妇人,将出与语,生人也。送之京师,问其本事,不知也。视其頉上树木可三十岁,不知此妇人三十岁常生于地中邪?将一朝欻生,偶与发頉者会也?
二年春二月乙未,太白犯荧惑。癸酉,诏曰:“鞭作官刑,所以纠慢怠也,而顷多以无辜死。
其减鞭杖之制,着于令。”三月庚寅,山阳公薨,帝素服发哀,遣使持节典护丧事。己酉,大赦。夏四月,大疫。崇华殿灾。丙寅,诏有司以太牢告祠文帝庙。追谥山阳公为汉孝献皇帝,葬以汉礼。[一]
注[一]献帝传曰:帝变服,率髃臣哭之,使使持节行司徒太常和洽吊祭,又使持节行大司空大司农崔林监护丧事。诏曰:“盖五帝之事尚矣,仲尼盛称尧、舜巍巍荡荡之功者,以为禅代乃大圣之懿事也。山阳公深识天禄永终之运,禅位文皇帝以顺天命。先帝命公行汉正朔,郊天祀祖以天子之礼,言事不称臣,此舜事尧之义也。昔放勋殂落,四海如丧考妣,遏密八音,明丧葬之礼同于王者也。今有司奏丧礼比诸侯王,此岂古之遗制而先帝之至意哉?今谥公汉孝献皇帝。”使太尉具以一太牢告祠文帝庙,曰:“叡闻夫礼也者,反本修古,不忘厥初,是以先代之君,尊尊亲亲,咸有尚焉。今山阳公寝疾弃国,有司建言丧纪之礼视诸侯王。
叡惟山阳公昔知天命永终于己,深观历数允在圣躬,传祚禅位,尊我民主,斯乃陶唐懿德之事也。黄初受终,命公于国行汉正朔,郊天祀祖礼乐制度率乃汉旧,斯亦舜、禹明堂之义也。
上考遂初,皇极攸建,允熙克让,莫朗于兹。盖子以继志嗣训为孝,臣以配命钦述为忠,故诗称‘匪棘其犹,聿追来孝’,书曰‘前人受命,兹不忘大功’。叡敢不奉承徽典,以昭皇考之神灵。今追谥山阳公曰孝献皇帝,册赠玺绂。命司徒、司空持节吊祭护丧,光禄、大鸿胪为副,将作大匠、复土将军营成陵墓,及置百官髃吏,车旗服章丧葬礼仪,一如汉氏故事;
丧葬所供髃官之费,皆仰大司农。立其后嗣为山阳公,以通三统,永为魏宾。”于是赠册曰:
“呜呼,昔皇天降戾于汉,俾逆臣董卓,播厥凶虐,焚灭京都,劫迁大驾。于时六合云扰,奸雄熛起。帝自西京,徂唯求定,臻兹洛邑。畴咨圣贤,聿改乘辕,又迁许昌,武皇帝是依。
岁在玄枵,皇师肇征,迄于鹑尾,十有八载,髃寇歼殄,九域咸乂。惟帝念功,祚兹魏国,大启土宇。爰及文皇帝,齐圣广渊,仁声旁流,柔远能迩,殊俗向义,干精承祚,坤灵吐曜,稽极玉衡,允膺历数,度于轨仪,克厌帝心。乃仰钦七政,俯察五典,弗采四岳之谋,不俟师锡之举,幽赞神明,承天禅位。祚*(建)**[逮]*朕躬,统承洪业。盖闻昔帝尧,元恺既举,凶族未流,登舜百揆,然后百揆时序,内平外成,授位明堂,退终天禄,故能冠德百王,表功嵩岳。自往迄今,弥历七代,岁暨三千,而大运来复,庸命厎绩,纂我民主,作建皇极。念重光,绍咸池,继韶夏,超群后之遐踪,邈商、周之惭德,可谓高朗令终,昭明洪烈之懿盛者矣。非夫汉、魏与天地合德,与四时合信,动和民神,格于上下,其孰能至于此乎?朕惟孝献享年不永,钦若顾命,考之典谟,恭述皇考先灵遗意,阐崇弘谥,奉成圣美,以章希世同符之隆,以传亿载不朽之荣。魂而有灵,嘉兹弘休。呜呼哀哉!”八月壬申,葬于山阳国,陵曰禅陵,置园邑。葬之日,帝制锡衰弁绖,哭之恸。适孙桂氏乡侯康,嗣立为山阳公。
是月,诸葛亮出斜谷,屯渭南,司马宣王率诸军拒之。诏宣王:“但坚壁拒守以挫其锋,彼进不得志,退无与战,久停则粮尽,虏略无所获,则必走矣。走而追之,以逸待劳,全胜之道也。”[一]
注[一]魏氏春秋曰:亮既屡遣使交书,又致巾帼妇人之饰,以怒宣王。宣王将出战,辛毗杖节奉诏,勒宣王及军吏已下,乃止。宣王见亮使,唯问其寝食及其事之烦简,不问戎事。使对曰:“诸葛公夙兴夜寐,罚二十已上,皆亲览焉;所啖食不过数升。”宣王曰:“亮体毙矣,其能久乎?”
五月,太白昼见。孙权入居巢湖口,向合肥新城,又遣将陆议、孙韶各将万余人入淮、沔。
六月,征东将军满宠进军拒之。宠欲拔新城守,致贼寿春,帝不听,曰:“昔汉光武遣兵县据略阳,终以破隗嚣,先帝东置合肥,南守襄阳,西固祁山,贼来辄破于三城之下者,地有所必争也。纵权攻新城,必不能拔。敕诸将坚守,吾将自往征之,比至,恐权走也。”秋七月壬寅,帝亲御龙舟东征,权攻新城,将军张颖等拒守力战,帝军未至数百里,权遁走,议、韶等亦退。髃臣以为大将军方与诸葛亮相持未解,车驾可西幸长安。帝曰:“权走,亮胆破,大将军以制之,吾无忧矣。”遂进军幸寿春,录诸将功,封赏各有差。八月己未,大曜兵,飨六军,遣使者持节犒劳合肥、寿春诸军。辛巳,行还许昌宫。
司马宣王与亮相持,连围积日,亮数挑战,宣王坚垒不应。会亮卒,其军退还。
冬十月乙丑,月犯镇星及轩辕。戊寅,月犯太白。十一月,京都地震,从东南来,隐隐有声,摇动屋瓦。十二月,诏有司删定大辟,减死罪。
三年春正月戊子,以大将军司马宣王为太尉。己亥,复置朔方郡。京都大疫。丁巳,皇太后崩。乙亥,陨石于寿光县。三月庚寅,葬文德郭后,营陵于首阳陵涧西,如终制。[一]
注[一]顾恺之启蒙注曰:魏时人有开周王頉者,得殉葬女子,经数日而有气,数月而能语;
年可二十。送诣京师,郭太后爱养之。十余年,太后崩,哀思哭泣,一年余而死。
是时,大治洛阳宫,起昭阳、太极殿,筑总章观。百姓失农时,直臣杨阜、高堂隆等各数切谏,虽不能听,常优容之。[一]
注[一]魏略曰:是年起太极诸殿,筑总章观,高十余丈,建翔凤于其上;又于芳林园中起陂池,楫棹越歌;又于列殿之北,立八坊,诸才人以次序处其中,贵人夫人以上,转南附焉,其秩石拟百官之数。帝常游宴在内,乃选女子知书可付信者六人,以为女尚书,使典省外奏事,处当画可,自贵人以下至尚保,及给掖庭洒扫,习伎歌者,各有千数。通引谷水过九龙殿前,为玉井绮栏,蟾蜍含受,神龙吐出。使博士马均作司南车,水转百戏。岁首建巨兽,鱼龙曼延,弄马倒骑,备如汉西京之制,筑阊阖诸门阙外罘罳。太子舍人张茂以吴、蜀数动,诸将出征,而帝盛兴宫室,留意于玩饰,赐与无度,帑藏空竭;又录夺士女前已嫁为吏民妻者,还以配士,既听以生口自赎,又简选其有姿色者内之掖庭,乃上书谏曰:“臣伏见诏书,诸士女嫁非士者,一切录夺,以配战士,斯诚权时之宜,然非大化之善者也。臣请论之。陛下,天之子也,百姓吏民,亦陛下之子也。礼,赐君子小人不同日,所以殊贵贱也。吏属君子,士为小人,今夺彼以与此,亦无以异于夺兄之妻妻弟也,于父母之恩偏矣。又诏书听得以生口年纪、颜色与妻相当者自代,故富者则倾家尽产,贫者举假贷贳,贵买生口以赎其妻;县官以配士为名而实内之掖庭,其丑恶者乃出与士。得妇者未必有欢心,而失妻者必有忧色,或穷或愁,皆不得志。夫君有天下而不得万姓之欢心者,寭不危殆。且军师在外数千万人,一日之费非徒千金,举天下之赋以奉此役,犹将不给,况复有宫庭非员无录之女,椒房母后之家,赏赐横兴,内外交引,其费半军。昔汉武帝好神仙,信方士,掘地为海,封土为山,赖是时天下为一,莫敢与争者耳。自衰乱以来,四五十载,马不舍鞍,士不释甲,每一交战,血流丹野,创痍号痛之声,于今未已。犹强寇在疆,图危魏室。陛下不兢兢业业,念崇节约,思所以安天下者,而乃奢靡是务,中尚方纯作玩弄之物,炫耀后园,建承露之盘,斯诚快耳目之观,然亦足以骋寇绚之心矣。惜乎,舍尧舜之节俭,而为汉武之侈事,臣窃为陛下不取也。愿陛下沛然下诏,万几之事有无益而有损者悉除去之,以所除无益之费,厚赐将士父母妻子之饥寒者,问民所疾而除其所恶,实仓廪,缮甲兵,恪恭以临天下。如是,吴贼面缚,蜀虏舆榇,不待诛而自服,太平之路可计日而待也。陛下可无劳神思于海表,军师高枕,战士备员。今髃公皆结舌,而臣所以不敢不献瞽言者,臣昔上要言,散骑奏臣书,以听谏篇为善,诏曰:‘是也’,擢臣为太子舍人;
且臣作书讥为人臣不能谏诤,今有可谏之事而臣不谏,此为作书虚妄而不能言也。臣年五十,常恐至死无以报国,是以投躯没命,冒昧以闻,惟陛下裁察。”书通,上顾左右曰:“张茂恃乡里故也。”以事付散骑而已。茂字彦林,沛人。
秋七月,洛阳崇华殿灾,八月庚午,立皇子芳为齐王,询为秦王。丁巳,行还洛阳宫。命有司复崇华,改名九龙殿。冬十月己酉,中山王兖薨。壬申,太白昼见。十一月丁酉,行幸许昌宫。[一]
注[一]魏氏春秋曰:是岁张掖郡删丹县金山玄川溢涌,宝石负图,状象灵龟,广一丈六尺,长一丈七尺一寸,围五丈八寸,立于川西。有石马七,其一仙人骑之,其一羁绊,其五有形而不善成。有玉匣关盖于前,上有玉字,玉玦二,璜一。麒麟在东,凤鸟在南,白虎在西,牺牛在北,马自中布列四面,色皆苍白。其南有五字,曰“上上三天王”;又曰“述大金,大讨曹,金但取之,金立中,大金马一匹在中,大*(告)**[吉]*开寿,此马甲寅述水”。凡“中”字六,“金”字十;又有若八卦及列宿孛彗之象焉。世语曰:又有一鸡象。搜神记曰:
初,汉元、成之世,先识之士有言曰,魏年有和,当有开石于西三千余里,系五马,文曰“大讨曹”。及魏之初兴也,张掖之柳谷,有开石焉,始见于建安,形成于黄初,文备于太和,周围七寻,中高一仞,苍质素章,龙马、麟鹿、凤皇、仙人之象,粲然咸着,此一事者,魏、晋代兴之符也。至晋泰始三年,张掖太守焦胜上言,以留郡本国图校今石文,文字多少不同,谨具图上。按其文有五马象,其一有人平上帻,执戟而乘之,其一有若马形而不成,其字有“金”,有“中”,有“大司马”,有“王”,有“大吉”,有“正”,有“开寿”,其一成行,曰“金当取之”。汉晋春秋曰:氐池县大柳谷口夜激波涌溢,其声如雷,晓而有苍石立水中,长一丈六尺,高八尺,白石画之,为十三马,一牛,一鸟,八卦玉玦之象,皆隆起,其文曰“大讨曹,适水中,甲寅”。帝恶其“讨”也,使凿去为“计”,以苍石窒之,宿昔而白石满焉。至晋初,其文愈明,马象皆焕彻如玉焉。
四年春二月,太白复昼见,月犯太白,又犯轩辕一星,入太微而出。夏四月,置崇文观,征善属文者以充之。五月乙卯,司徒董昭薨。丁巳,肃慎氏献楛矢。
六月壬申,诏曰:“有虞氏画象而民弗犯,周人刑错而不用。朕从百王之末,追望上世之风,邈乎何相去之远?法令滋章,犯者弥多,刑罚愈众,而奸不可止。往者按大辟之条,多所蠲除,思济生民之命,此朕之至意也。而郡国毙狱,一岁之中尚过数百,岂朕训导不醇,俾民轻罪,将苛法犹存,为之陷藊乎?有司其议狱缓死,务从宽简,及乞恩者,或辞未出而狱以报断,非所以究理尽情也。其令廷尉及天下狱官,诸有死罪具狱以定,非谋反及手杀人,亟语其亲治,有乞恩者,使与奏当文书俱上,朕将思所以全之。其布告天下,使明朕意。”
秋七月,高句骊王宫斩送孙权使胡韂等首,诣幽州。甲寅,太白犯轩辕大星。冬十月己卯,行还洛阳宫。甲申,有星孛于大辰,乙酉,又孛于东方。十一月己亥,彗星见,犯宦者天纪星。十二月癸巳,司空陈髃薨。乙未,行幸许昌宫。
景初元年春正月壬辰,山茌县言黄龙见。*茌音仕狸反。*于是有司奏,以为魏得地统,宜以建丑之月为正。三月,定历改年为孟夏四月。[一]服色尚黄,牺牲用白,戎事乘黑首白马,建大赤之旗,朝会建大白之旗。[二]改太和历曰景初历。其春夏秋冬孟仲季月虽与正岁不同,至于郊祀、迎气、礿祠、蒸尝、巡狩、搜田、分至启闭、班宣时令、中气早晚、敬授民事,皆以正岁斗建为历数之序。
注[一]魏书曰:初,文皇帝即位,以受禅于汉,因循汉正朔弗改。帝在东宫着论,以为五帝三王虽同气共祖,礼不相袭,正朔自宜改变,以明受命之运。及即位,优游者久之,史官复着言宜改,乃诏三公、特进、九卿、中郎将、大夫、博士、议郎、千石、六百石博议,议者或不同。帝据古典,甲子诏曰:“夫太极运三辰五星于上,元气转三统五行于下,登降周旋,终则又始。故仲尼作春秋,于三微之月,每月称王,以明三正迭相为首。今推三统之次,魏得地统,当以建丑之月为正月。考之髃艺,厥义章矣。其改青龙五年三月为景初元年四月。”
注[二]臣松之按:魏为土行,故服色尚黄。行殷之时,以建丑为正,故牺牲旗旗一用殷礼。
礼记云:“夏后氏尚黑,故戎事乘骊,牲用玄;殷人尚白,戎事乘翰,牲用白;周人尚赤,戎事乘騵,牲用骍。”郑玄云:“夏后氏以建寅为正,物生色黑;殷以建丑为正,物牙色白;
周以建子为正,物萌色赤。翰,白色马也,易曰‘白马翰如’。”周礼巾车职“建大赤以朝”,大白以即戎,此则周以正色之旗以朝,先代之旗即戎。今魏用殷礼,变周之制,故建大白以朝,大赤即戎。
五月己巳,行还洛阳宫。己丑,大赦。六月戊申,京都地震。己亥,以尚书令陈矫为司徒,尚书*(左)**[右]*仆射韂臻为司空。丁未,分魏兴之魏阳、锡郡之安富、上庸为上庸郡。省锡郡,以锡县属魏兴郡。
有司奏:武皇帝拨乱反正,为魏太祖,乐用武始之舞。文皇帝应天受命,为魏高祖,乐用咸熙之舞。帝制作兴治,为魏烈祖,乐用章*(武)**[斌]*之舞。三祖之庙,万世不毁。其余四庙,亲尽迭毁,如周后稷、文、武庙祧之制。[一]
注[一]孙盛曰:夫谥以表行,庙以存容,皆于既没然后着焉,所以原始要终,以示百世也。
未有当年而逆制祖宗,未终而豫自尊显。昔华乐以厚敛致讥,周人以豫凶违礼,魏之髃司,于是乎失正。
秋七月丁卯,司徒陈矫薨。孙权遣将朱然等二万人围江夏郡,荆州刺史胡质等击之,然退走。
初,权遣使浮海与高句骊通,欲袭辽东。遣幽州刺史□丘俭率诸军及鲜卑、乌丸屯辽东南界,玺书征公孙渊。渊发兵反,俭进军讨之,会连雨十日,辽水大涨,诏俭引军还。右北平乌丸单于寇娄敦、辽西乌丸都督王护留等居辽东,率部众随俭内附。己卯,诏辽东将吏士民为渊所胁略不得降者,一切赦之。辛卯,太白昼见。渊自俭还,遂自立为燕王,置百官,称绍汉元年。
诏青、兖、幽、冀四州大作海船。九月,冀、兖、徐、豫四州民遇水,遣侍御史循行没溺死亡及失财产者,在所开仓振救之。庚辰,皇后毛氏卒。冬十月丁未,月犯荧惑。癸丑,葬悼毛后于愍陵。乙卯,营洛阳南委粟山为圜丘。[一]十二月壬子冬至,始祀。丁巳,分襄阳临沮、宜城、旍阳、邔*邔音其己反。*四县,置襄阳南部都尉。己未,有司奏文昭皇后立庙京都。分襄阳郡之鄀叶县属义阳郡。[二]
注[一]魏书载诏曰:“盖帝王受命,莫不恭承天地以章神明,尊祀世统以昭功德,故先代之典既着,则禘郊祖宗之制备也。昔汉氏之初,承秦灭学之后,采摭残缺,以备郊祀,自甘泉后土、雍宫五畤,神只兆位,多不见经,是以制度无常,一彼一此,四百余年,废无禘祀。
古代之所更立者,遂有阙焉。曹氏系世,出自有虞氏,今祀圜丘,以始祖帝舜配,号圜丘曰皇皇帝天;方丘所祭曰皇皇后地,以舜妃伊氏配;天郊所祭曰皇天之神,以太祖武皇帝配;
地郊所祭曰皇地之只,以武宣后配;宗祀皇考高祖文皇帝于明堂,以配上帝。”至晋泰始二年,并圜丘、方丘二至之祀于南北郊。
注[二]魏略曰:是岁,徙长安诸钟懬、骆驼、铜人、承露盘。盘折,铜人重不可致,留于霸城。大发铜铸作铜人二,号曰翁仲,列坐于司马门外。又铸黄龙、凤皇各一,龙高四丈,凤高三丈余,置内殿前。起土山于芳林园西北陬,使公卿髃僚皆负土成山,树松竹杂木善草于其上,捕山禽杂兽置其中。汉晋春秋曰:帝徙盘,盘折,声闻数十里,金狄或泣,因留霸城。
魏略载司徒军议掾河东董寻上书谏曰:“臣闻古之直士,尽言于国,不避死亡。故周昌比高祖于桀、纣,刘辅譬赵后于人婢。天生忠直,虽白刃沸汤,往而不顾者,诚为时主爱惜天下也。建安以来,野战死亡,或门殚户尽,虽有存者,遗孤老弱。若今宫室狭小,当广大之,犹宜随时,不妨农务,况乃作无益之物,黄龙、凤皇,九龙、承露盘,土山、渊池,此皆圣明之所不兴也,其功参倍于殿舍。三公九卿侍中尚书,天下至德,皆知非道而不敢言者,以陛下春秋方刚,心畏雷霆。今陛下既尊群臣,显以冠冕,被以文绣,载以华舆,所以异于小人;而使穿方举土,面目垢黑,沾体涂足,衣冠了鸟,毁国之光以崇无益,甚非谓也。孔子曰:‘君使臣以礼,臣事君以忠。’无忠无礼,国何以立!故有君不君,臣不臣,上下不通,心怀郁结,使阴阳不和,灾害屡降,凶恶之徒,因间而起,谁当为陛下尽言事者乎?又谁当干万乘以死为戏乎?臣知言出必死,而臣自比于牛之一毛,生既无益,死亦何损?秉笔流涕,心与世辞。臣有八子,臣死之后,累陛下矣!”
将奏,沐浴。既通,帝曰:“董寻不畏死邪!”主者奏收寻,有诏勿问。后为贝丘令,清省得民心。
二年春正月,诏太尉司马宣王帅众讨辽东。[一]
注[一]干窦晋纪曰:帝问宣王:“度公孙渊将何计以待君?”宣王对曰:“渊弃城预走,上计也;据辽水拒大军,其次也;坐守襄平,此为成禽耳。”帝曰:“然则三者何出?”对曰:“唯明智审量彼我,乃预有所割弃,此既非渊所及,又谓今往县远,不能持久,必先拒辽水,后守也。”帝曰:“住还几日?”对曰:“往百日,攻百日;还百日,以六十日为休息,如此,一年足矣。”魏名臣奏载散骑常侍何曾表曰:“臣闻先王制法,必于全慎,故建官授任,则置假辅,陈师命将,则立监贰,宣命遣使,则设介副,临敌交刃,则参御右,盖以尽谋思之功,防安危之变也。是以在险当难,则权足相济,陨缺不预,则才足相代,其为固防,至深至远。及至汉氏,亦循旧章。韩信伐赵,张耳为贰;马援讨越,刘隆副军。前世之夡,着在篇志。今懿奉辞诛罪,步骑数万,道路回阻,四千余里,虽假天威,有征无战,寇或潜遁,消散日月,命无常期。人非金石,远虑详备,诚宜有副。今北边诸将及懿所督,皆为僚属,名位不殊,素无定分,卒有变急,不相镇摄。存不忘亡,圣达所戒,宜选大臣名将威重宿著者,盛其礼秩,遣诣懿军,进同谋略,退为副佐。虽有万一不虞之灾,军主有储,则无患矣。”□丘俭志记云,时以俭为宣王副也。
二月癸卯,以大中大夫韩暨为司徒。癸丑,月犯心距星,又犯心中央大星。夏四月庚子,司徒韩暨薨。壬寅,分沛国萧、相、竹邑、符离、蕲、铚、龙亢、山桑、洨、虹*洨音胡交反。
虹音绛。*十县为汝阴郡。宋县、陈郡苦县皆属谯郡。以沛、杼秋、公丘、彭城丰国、广戚,并五县为沛王国。庚戌,大赦。五月乙亥,月犯心距星,又犯中央大星。[一]六月,省渔阳郡之狐奴县,复置安乐县。
注[一]魏书载戊子诏曰:“昔汉高祖创业,光武中兴,谋除残暴,功昭四海,而坟陵崩颓,童儿牧竖践蹈其上,非大魏尊崇所承代之意也。其表高祖、光武陵四面百步,不得使民耕牧樵采。”
秋八月,烧当羌王芒中、注诣等叛,凉州刺史率诸郡攻讨,斩注诣首。癸丑,有彗星见张宿。
[一]
注[一]汉晋春秋曰:史官言于帝曰:“此周之分野也,洛邑恶之。”于是大修禳祷之术以厌焉。魏书曰:九月,蜀阴平太守廖惇反,攻守善羌侯宕蕈营。雍州刺史郭淮遣广魏太守王赟、南安太守游奕将兵讨惇。淮上书:“赟、奕等分兵夹山东西,围落贼表,破在旦夕。”帝曰:
“兵势恶离。”促诏淮敕奕诸别营非要处者,还令据便地。诏敕未到,奕军为惇所破;赟为流矢所中死。
丙寅,司马宣王围公孙渊于襄平,大破之,传渊首于京都,海东诸郡平。冬十一月,录讨渊功,太尉宣王以下增邑封爵各有差。初,帝议遣宣王讨渊,发卒四万人。议臣皆以为四万兵多,役费难供。帝曰:“四千里征伐,虽云用奇,亦当任力,不当稍计役费。”遂以四万人行。及宣王至辽东,霖雨不得时攻,髃臣或以为渊未可卒破,宜诏宣王还。帝曰:“司马懿临危制变,擒渊可计日待也。”卒皆如所策。
壬午,以司空韂臻为司徒,司隶校尉崔林为司空。闰月,月犯心中央大星。十二月乙丑,帝寝疾不豫。辛巳,立皇后。赐天下男子爵人二级,□寡孤独谷。以燕王宇为大将军,甲申免,以武韂将军曹爽代之。[一]
注[一]汉晋春秋曰:帝以燕王宇为大将军,使与领军将军夏侯献、武韂将军曹爽、屯骑校尉曹肇、骁骑将军秦朗等对辅政。中书监刘放、令孙资久专权宠,为朗等素所不善,惧有后害,阴图间之,而宇常在帝侧,故未得有言。甲申,帝气微,宇下殿呼曹肇有所议,未还,而帝少闲,惟曹爽独在。放知之,呼资与谋。资曰:“不可动也。”放曰:“俱入鼎镬,何不可之有?”乃突前见帝,垂泣曰:“陛下气微,若有不讳,将以天下付谁?”帝曰:“卿不闻用燕王耶?”放曰:“陛下忘先帝诏敕,藩王不得辅政。且陛下方病,而曹肇、秦朗等便与才人侍疾者言戏。燕王拥兵南面,不听臣等入,此即竖刁、赵高也。今皇太子幼弱,未能统政,外有强暴之寇,内有劳怨之民,陛下不远虑存亡,而近系恩旧。委祖宗之业,付二三凡士,寝疾数日,外内壅隔,社稷危殆,而己不知,此臣等所以痛心也。”帝得放言,大怒曰:“谁可任者?”放、资乃举爽代宇,又白“宜诏司马宣王使相参”,帝从之。放、资出,曹肇入,泣涕固谏,帝使肇敕停。肇出户,放、资趋而往,复说止帝,帝又从其言。放曰:“宜为手诏。”
帝曰:“我困笃,不能。”放即上黙,执帝手强作之,遂赍出,大言曰:“有诏免燕王宇等官,不得停省中。”于是宇、肇、献、朗相与泣而归第。
初,青龙三年中,寿春农民妻自言为天神所下,命为登女,当营韂帝室,蠲邪纳福。饮人以水,及以洗疮,或多愈者。于是立馆后宫,下诏称扬,甚见优宠。及帝疾,饮水无验,于是杀焉。
三年春正月丁亥,太尉宣王还至河内,帝驿马召到,引入卧内,执其手谓曰:“吾疾甚,以后事属君,君其与爽辅少子。吾得见君,无所恨!”宣王顿首流涕。[一]即日,帝崩于嘉福殿,[二]时年三十六。[三]癸丑,葬高平陵。[四]
注[一]魏略曰:帝既从刘放计,召司马宣王,自力为诏,既封,顾呼宫中常所给使者曰:“辟邪来!汝持我此诏授太尉也。”辟邪驰去。先是,燕王为帝画计,以为关中事重,宜便道遣宣王从河内西还,事以施行。宣王得前诏,斯须复得后手笔,疑京师有变,乃驰到,入见帝。
劳问讫,乃召齐、秦二王以示宣王,别指齐王谓宣王曰:“此是也,君谛视之,勿误也!”
又教齐王令前抱宣王颈。魏氏春秋曰:时太子芳年八岁,秦王九岁,在于御侧。帝执宣王手,目太子曰:“死乃复可忍,朕忍死待君,君其与爽辅此。”宣王曰:“陛下不见先帝属臣以陛下乎?”
注[二]魏书曰:殡于九龙前殿。
注[三]臣松之按:魏武以建安九年八月定邺,文帝始纳甄后,明帝应以十年生,计至此年正月,整三十四年耳。时改正朔,以故年十二月为今年正月,可强名三十五年,不得三十六也。
注[四]魏书曰:帝容止可观,望之俨然。自在东宫,不交朝臣,不问政事,唯潜思书籍而已。

即位之后,褒礼大臣,料简功能,真伪不得相贸,务绝浮华谮毁之端,行师动众,论决大事,谋臣将相,咸服帝之大略。性特强识,虽左右小臣官簿性行,名迹所履,及其父兄子弟,一经耳目,终不遗忘。含垢藏疾,容受直言,听受吏民士庶上书,一月之中至数十百封,虽文辞鄙陋,犹览省究竟,意无厌倦。孙监曰:闻之长老,魏明帝天姿秀出,立发垂地,口吃少言,而沉毅好断。初,诸公受遗辅导,帝皆以方任处之,政自己出。而优礼大臣,开容善直,虽犯颜极谏,无所摧戮,其君人之量如此之伟也。然不思建德垂风,不固维城之基,至使大权偏据,社稷无韂,悲夫!
评曰:明帝沉毅断识,任心而行,盖有君人之至概焉。于时百姓雕弊,四海分崩,不先聿修显祖,阐拓洪基,而遽追秦皇、汉武,宫馆是营,格之远猷,其殆疾乎!
\end{yuanwen}

\part{魏书四}

\chapter{三少帝纪第四}

\part{魏书五}

\chapter{后妃传第五}

\part{魏书六}
\chapter{董二袁刘传第六}

\part{魏书七}
\chapter{吕布*(张邈)*臧洪传第七}

\part{魏书八}
\chapter{二公孙陶四张传第八}

\part{魏书九}
\chapter{诸夏侯曹传第九}

\part{魏书十}
\chapter{荀彧荀攸贾诩传第十}

\part{魏书十一}
\chapter{袁张凉国田王邴管传第十一}

\part{魏书十二}
\chapter{崔毛徐何邢鲍司马传第十二}

\part{魏书十三}
\chapter{锺繇华歆王朗传第十三}

\part{魏书十四}
\chapter{程郭董刘蒋刘传第十四}

\part{魏书十五}
\chapter{刘司马梁张温贾传第十五}

\part{魏书十六}
\chapter{任苏杜郑仓传第十六}

\part{魏书十七}
\chapter{张乐于张徐传第十七}

\part{魏书十八}
\chapter{二李臧文吕许典二庞阎传第十八}

\part{魏书十九}
\chapter{任城陈萧王传第十九}

\part{魏书二十}
\chapter{武文世王公传第二十}

\part{魏书二十一}
\chapter{王卫二刘傅传第二十一}

\part{魏书二十二}
\chapter{桓二陈徐卫卢传第二十二}

\part{魏书二十三}
\chapter{和常杨杜赵裴传第二十三}

\part{魏书二十四}
\chapter{韩崔高孙王传第二十四}

\part{魏书二十五}
\chapter{辛毗杨阜高堂隆传第二十五}

\part{魏书二十六}
\chapter{满田牵郭传第二十六}

\part{魏书二十七}
\chapter{徐胡二王传第二十七}

\part{魏书二十八}
\chapter{王毌丘诸葛邓锺传第二十八}

\part{魏书二十九}
\chapter{方技传第二十九}

\part{魏书三十}
\chapter{乌丸鲜卑东夷传第三十}

\part{蜀书一}
\chapter{刘二牧传第一}

\part{蜀书二}
\chapter{先主传第二}

\part{蜀书三}
\chapter{后主传第三}

\part{蜀书四}
\chapter{二主妃子传第四}

\part{蜀书五}
\chapter{诸葛亮传第五}

\part{蜀书六}
\chapter{关张马黄赵传第六}

\part{蜀书七}
\chapter{庞统法正传第七}

\part{蜀书八}
\chapter{许麋孙简伊秦传第八}

\part{蜀书九}
\chapter{董刘马陈董吕传第九}

\part{蜀书十}
\chapter{刘彭廖李刘魏杨传第十}

\part{蜀书十一}
\chapter{霍王向张杨费传第十一}

\part{蜀书十二}
\chapter{杜周杜许孟来尹李谯郤传第十二}

\part{蜀书十三}
\chapter{黄李吕马王张传第十三}

\part{蜀书十四}
\chapter{蒋琬费祎姜维传第十四}

\part{蜀书十五}
\chapter{邓张宗杨传第十五}

\part{吴书一}
\chapter{孙破虏讨逆传第一}

\part{吴书二}
\chapter{吴主传第二}

\part{吴书三}
\chapter{三嗣主传第三}

\part{吴书四}
\chapter{刘繇太史慈士燮传第四}

\part{吴书五}
\chapter{妃嫔传第五}

\part{吴书六}
\chapter{宗室传第六}

\part{吴书七}
\chapter{张顾诸葛步传第七}

\part{吴书八}
\chapter{张严程阚薛传第八}

\part{吴书九}

\chapter{周瑜鲁肃吕蒙传第九}


周瑜字公瑾,庐江舒人也。从祖父景,景子忠,皆为汉太尉。父异,洛阳令。瑜长壮有姿貌。初,孙坚与义兵讨董卓,徙家于舒。坚子策兴瑜同年,独相友善,瑜推道南大宅以舍策,升堂拜母,有无通共。瑜从父尚为丹杨太守,瑜往省之。会策将东渡,到历阳,驰书报瑜,瑜将兵迎策。策大喜曰:“吾得卿。谐也。”遂从攻横江、当利,皆拔之。乃渡江击秣陵,破笮融、薛礼。转下湖孰、江乘,进入曲阿。刘繇奔走,而策之众已数万矣。因谓瑜曰:“吾以此众取吴会平山越已足。卿还镇丹杨。”瑜还。顷之,袁术遣从弟胤代尚为太守,而瑜与尚俱还寿春。术欲以瑜为将,瑜观术终无所成,故求为居巢长,欲假涂东归,术听之。遂自居巢还吴。是岁,建安三年也。策亲自迎瑜,授建威中郎将,即与兵二千人,骑五十匹。瑜时年二十四,吴中皆呼为周郎。以瑜恩信著于庐江,出备牛渚,后领春谷长。顷之,策欲取荆州,以瑜为中护军,领导江夏太守,从攻皖,拔之。时得桥公两女,皆国色也。策自纳大桥,瑜纳小桥。复近寻阳,破刘勋,讨江夏,还定豫章、庐陵,留镇巴丘。

五年,策薨。权统事。瑜将兵赴丧,遂留吴,以中护军与长史张昭共掌众事。十一年,督孙瑜等讨麻、保二屯,枭其渠帅,囚俘万余口,还备(官亭)。江夏太守黄祖遣将邓龙将兵数千人入柴桑,瑜追讨击,生虏龙送吴。十三年春,权讨江夏,瑜为前部大督。其年九月,曹公入荆州,刘琮举众降,曹公得其水军,船步兵数十万,将士闻之皆恐。权延见群下,问以计策。议者咸曰:“曹公豺虎也,然托名汉相,挟天子以征四方,动以朝廷为辞,今日拒之,事更不顺,且将军大势可以拒操者,长江也。今操得荆州,奄有其地。刘表治水军,蒙冲斗舰,乃以千数,操悉浮以沿江,兼有步兵,水陆俱下。此为长江之险,已与我共之矣。而势力众寡,又不可论。愚谓大计不如迎之。”瑜曰:“不然。操虽托名汉相,其实汉贼也。将军以神武雄才,兼仗父兄之烈,割据江东,地方数千里,兵精足用,英雄乐业,尚当横行天下,为汉家除残去秽。况操自送死,而可迎之耶?请为将军筹之:今使北土已安,操无内忧,能旷日持久,来争疆场,又能与我校胜负于船楫,(可)乎?今北土既未平安,加马超、韩遂尚在关西,为操后患。且舍鞍马,仗舟揖,与吴越争衡,本非中国所长。又今盛寒,马无藁草。驱中国士众远涉江湖之间,不习水土,必生疾病。此数四者,用兵之患也,而操皆冒行之。将军擒操,宜在今日。瑜请得精兵三万人,进住夏口,保为将军破之。”权曰:“老贼欲废汉自立久矣,陡忌二袁、吕布、刘表与孤耳。今数雄已灭,惟孤尚存,孤与老贼,势不两立。君言当击,甚与孤台,此天以君授孤也。

时刘备为曹公所破,欲引南渡江。与鲁肃遇于当阳,遂共图计,因进住夏口,遣诸葛亮诣权。权遂遣瑜及程普等与备并力逆曹公,遇于赤壁。时曹公军众已有疾病,初一交战,公军败退,引次江北。瑜等在南岸。瑜部将黄盖曰:“今寇众我寡,难与持久。然观操军船舰,首尾相接,可烧而走也。”乃取蒙冲斗舰数十艘,实以薪草,膏油灌其中。裹以帷幕,上建牙旗,先书报曹公,欺以欲降。又豫备走舸,各系大船后,因引次俱前。曹公军吏士皆延颈观望,指言盖降。盖放诸船,同时发火。时风盛猛,悉延烧岸上营落。顷之。烟炎张天,人马烧溺死者甚众,军遂败退,还保南郡。备与瑜等复共追。曹公留曹仁等守江陵城。径自北归。

瑜与程普又进南郡,与仁相对,各隔大江。兵未交锋,瑜即遣甘宁前据夷陵。仁分兵骑别攻围宁。宁告急于瑜。瑜用吕蒙计,留淩统以守其后,身与蒙上救宁。宁围既解,乃渡屯北岸,克期大战。瑜亲跨马擽陈,会流矢中右胁,疮甚,便还。后仁闻瑜卧未起,勒兵就陈。瑜乃自兴,案行军营,激扬吏士,仁由是遂退。

权拜瑜偏将军,领南郡太守。以下隽、汉昌、刘阳、州陵为奉邑,屯据江陵。刘备以左将军领荆州牧,治公安,备诣京见权,瑜上疏曰:“刘备以枭雄之姿,而有关羽、张飞熊虎之将,必非久屈为人用者。愚谓大计宜徙备置吴,盛为筑宫室,多其美女玩好,以娱其耳目,分此二人,各置一方,使如瑜者得挟与攻战,大事可定也。今猥割土地以资业之,聚此三人,俱在疆场,恐蛟龙得云雨,终非池中物也。”权以曹公在北方,当广揽英雄,又恐备难卒制,故不纳。是时刘璋为益州牧。外有张鲁寇侵,瑜乃诣京见权曰:“今曹操新折衄,方忧在腹心,未能与将军连兵相事也。乞与奋威俱进取蜀,得蜀而并张鲁,因留奋威固守其地,好与马超结援。瑜还与将军据襄阳以蹙操,北方可图也。”权许之。瑜还江陵为行装,而道于马丘病卒,时年三十六。权素服举哀。感动左右。丧当还吴,又迎之芜湖,众事费度,一为供给。后著令曰:“故将军周瑜、程普,其有人客,皆不得问。”初瑜见友于策,太妃又使权以兄奉之。是时权位为将军,诸将宾客为礼尚简,而瑜独先尽敬,便执臣节。性度恢廓,大率为得人,惟与程普不睦。

\begin{yuanwen}
瑜少精意于音乐。虽三爵之后,其有阙误。瑜必知之,知之必顾,故时人谣曰:“曲有误,周郎顾。”
\end{yuanwen}

瑜两男一女,女配太子登。男循尚公主,拜骑都尉,有瑜风,早卒。循弟胤,初拜兴业都尉。妻以宗女,授兵千人,屯公安。黄龙元年,封都乡侯,后以罪徙庐陵郡。赤乌二年,诸葛瑾、步骘连名上疏曰:“故将军周瑜子胤,昔蒙粉饰,受封为将,不能养之以福,思立功效,至纵情欲,招速罪辟。臣窃以瑜昔见宠任,入作心膂,出为爪牙,衔命出征,身当矢石,尽节用命,视死如归。故能摧曹操于乌林,走曹仁于郢都,扬国威德,华夏是震,蠢尔蛮荆,莫不宾服。虽周之方叔,汉之信、布,诚无以尚也。夫折冲扦难之臣,自古帝王莫不贵重,故汉高帝封爵之誓曰‘使黄河如带,太山如砺,国以永存,爰及苗裔’。申以丹书,重以盟诅,藏于宗庙,传于无穷,欲使功臣之后,世世相踵,非徒子孙,乃关苗裔,报德明功,勤勤恳恳,如此之至,欲以劝戒后人,用命之臣,死而无悔也。况于瑜身没未久,而其子胤降为匹夫,益可悼伤。窃惟陛下钦明稽古,隆于兴继,为胤归诉,乞丐余罪,还兵复爵,使失旦之鸡,复得一鸣。抱罪之臣,展其后效。”权答曰:“腹心旧勋,与孤协事,公瑾有之,诚所不忘。昔胤年少,初无功劳,横受精兵,爵以侯将,盖念公瑾以及于胤也。而胤恃此,酗淫自恣,前后告喻,曾无悛改。孤于公瑾,义犹二君,乐胤成就,岂有已哉?迫胤罪恶,未宜便还,且欲苦之,使自知耳。今二君勤勤援引汉高河山之誓,孤用恧然。虽德非其畴,犹欲庶几,事亦如尔,故未顺旨。以公瑾之子,而二君在中间,苟使能改,亦何患乎!”瑾、骘表比上,朱然及全琮亦俱陈乞,权乃许之。会胤病死。

瑜兄子峻,亦以瑜元功为偏将军,领吏士千人。峻卒,全琮表峻子护为将。权曰:“昔走曹操,拓有荆州,皆是公瑾,常不忘之。初闻峻亡,仍欲用护,闻护性行危险,用之适为作祸,故便止之。孤念公瑾,岂有已乎?”

\begin{yuanwen}
鲁肃字子敬,临淮东城人也。生而失父,与祖母居。家富于财,性好施与,尔时天下已乱,肃不治家事,大散财货,摽卖田地,以赈穷弊结士为务,甚得乡邑欢心。

周瑜为居巢长,将数百人故过候肃,并求资粮。肃家有两囷米,各三千斛。肃乃指一囷与周瑜,瑜益知其奇也。遂相亲结,定侨、札之分。袁术闻其名,就署东城长。肃见术无纲纪,不足与立事,乃携老弱将轻侠少年百余人,南到居巢就瑜。瑜之东渡,因与同行,留家曲阿。会祖母亡,还葬东城。
\end{yuanwen}

\begin{yuanwen}
刘子扬与肃友善,遗肃书曰:“方今天下豪杰并起,吾子姿才,尤宜今日。急还迎老母,无事滞于东城。近郑宝者,今在巢湖,拥众万余,处地肥饶,庐江间人多依就之,况吾徒乎?观其形势,又可博集,时不可失,足下速之。”

肃答然其计。葬毕还曲阿,欲北行。会瑜已徙肃母到吴,肃具以状语瑜。时孙策已薨,权尚住吴,瑜谓肃曰:“昔马援答光武云‘当今之世,非但君择臣,臣亦择君’。今主人亲贤贵士,纳奇录异,且吾闻先哲秘论,承运代刘氏者,必兴于东南,推步事势,当其历数,终构帝基,以协天符,是烈士攀龙附凤驰骛之秋。吾方达此,足下不须以子扬之言介意也。”

肃从其言。瑜因荐肃才宜佐时,当广求其比,以成功业,不可令去也。
\end{yuanwen}

\begin{yuanwen}
权即见肃,与语甚悦之。众宾罢退,肃亦辞出,乃独引肃还,合榻对饮。因密议曰:“今汉室倾危,四方云扰,孤承父兄余业,思有桓文之功。君既惠顾,何以佐之?”

肃对曰:“昔高帝区区欲尊事义帝而不获者,以项羽为害也。今之曹操,犹昔项羽,将军何由得为桓文乎?肃窃料之,汉室不可复兴,曹操不可卒除。为将军计,惟有鼎足江东,以观天下之衅。规模如此,亦自无嫌。何者?北方诚多务也。因其多务,剿除黄祖,进伐刘表,竟长江所极,据而有之,然后建号帝王以图天下,此高帝之业也。”

权曰:“今尽力一方,冀以辅汉耳,此言非所及也。”

张昭非肃谦下不足,颇訾毁之,云肃年少粗疏,未可用。极不以介意,益贵重之,赐肃母衣服帏帐,居处杂物,富拟其旧。
\end{yuanwen}

\begin{yuanwen}
刘表死,肃进说曰:“夫荆楚与国邻接,水流顺北,外带江汉,内阻山陵,有金城之固,沃野万里,士民殷富,若据而有之,此帝王之资也。今表新亡,二子素不辑睦,军中诸将,各有彼此。加刘备天下枭雄,与操有隙,寄寓于表,表恶其能而不能用也。若备与彼协心,上下齐同,则宜抚安,与结盟好:如有离违,宜别图之,以济大事。肃请得奉命吊表二子,并慰劳其军中用事者,及说备使抚表众,同心一意,共治曹操,备必喜而从命。如其克谐,天下可定也。今不速往,恐为操所先。”权即遣肃行。
\end{yuanwen}

\begin{yuanwen}
到夏口,闻曹公已向荆州,晨夜兼道。比至南郡,而表子琮已降曹公,备惶遽奔走,欲南渡江。肃径迎之,到当阳长阪,与备会,宣腾权旨,及陈江东强固,劝备与权并力。备甚欢悦。时诸葛亮与备相随,肃谓亮曰“我子瑜友也”,即共定交。备遂到夏口,遣亮使权,肃亦反命。
\end{yuanwen}

\begin{yuanwen}
会权得曹公欲东之问,与诸将议,皆劝权迎之,而肃独不言。权起更衣,肃追于宇下,权知其意,执肃手曰:“卿欲何言?”

肃对曰:“向察众人之议,专欲误将军,不足与图大事。今肃可迎操耳,如将军,不可也。何以言之?今肃迎操,操当以肃还付乡党。品其名位,犹不失下曹从事,乘犊车、从吏卒、交游士林、累官故不失州郡也。将军迎操,欲安所归?愿早定大计,莫用众人之议也。”

权叹息曰:“此诸人持议,甚失孤望;今卿廓开大计,正与孤同,此天以卿赐我也。”
\end{yuanwen}

\begin{yuanwen}
时周瑜受使至鄱阳,肃劝追召瑜还。遂任瑜以行事,以肃为赞军校尉,助画方略。曹公破走,肃即先还,权大请诸将迎肃。肃将入閤拜,权起礼之,因谓曰:“子敬,孤持鞍下马相迎,足以显卿未?”

肃趋近曰:“未也。”

众人闻之,无不愕然。就坐,徐举鞭言曰:“愿至尊威德加乎四海,总括九州,克成帝业,更以安车软轮征肃,始当显耳。”权抚掌欢笑。
\end{yuanwen}

\begin{yuanwen}
后备诣京见权,求都督荆州,惟肃劝权借之,共拒曹公。曹公闻权以土地业备,方作书,落笔于地。
\end{yuanwen}

\begin{yuanwen}
周瑜病困,上疏曰:“当今天下,方有事役,是瑜乃心夙夜所忧,愿至尊先虑未然,然后康乐。今既与曹操为敌,刘备近在公安,边境密迩,百姓未附,宜得良将以镇抚之。鲁肃智略足任,乞以代瑜。瑜陨踣之日,所怀尽矣。”

即拜肃奋武校尉,代瑜领兵。瑜士众四千余人。奉邑四县,皆属焉。令程普领南郡太守。肃初住江陵,后下屯陆口,威恩大行,众增万余人,拜汉昌太守、偏将军。十九年,从权破皖城,转横江将军。
\end{yuanwen}

\begin{yuanwen}
先是,益州牧刘璋纲维颓弛。周瑜、甘宁并劝权取蜀,权以咨备,备内欲自规,仍伪报曰:“备与璋托为宗室,冀凭英灵,以匡汉朝。今璋得罪左右,备独竦惧,非所敢闻,愿加宽贷。若不获请,备当放发归于山林。”

后备西图璋,留关羽守。权曰:“猾虏乃敢挟诈!”

及羽与肃邻界,数生狐疑,疆埸纷错,肃常以欢好抚之。备既定益州,权求长沙、零、桂,备不承旨,权遣吕蒙率众进取。备闻,自还公安,遣羽争三郡。肃住益阳,与羽相拒。肃邀羽相见,各驻兵马百步上,但诸将军单刀俱会。肃因责数羽曰:“国家区区本以土地借卿家者,卿家军败远来,无以为资故也。今已得益州,既无奉还之意,但求三郡,又不从命。”

语未究竟,坐有一人曰:“夫土地者,惟德所在耳,何常之有!”

肃厉声呵之,辞色甚切。羽操刀起谓曰:“此自国家事,是人何知!”目使之去。备遂割湘水为界,于是罢军。
\end{yuanwen}

\begin{yuanwen}
肃年四十六,建安二十二年卒。权为举哀,又临其葬。诸葛亮亦为发哀。权称尊号,临坛,顾谓公卿曰:“昔鲁子敬尝道此,可谓明于事势矣。”
\end{yuanwen}

肃遣腹子淑既壮,濡须督张承谓终当到至。永安中,为昭武将军、都亭侯、武昌督。建衡中,假节,迁夏口督。所在严整,有方干。凤皇三年卒。子睦袭爵,领兵马。

吕蒙字子明,汝南富陂人也。少南渡,依姊夫邓当。当为孙策将,数讨山越。蒙年十五六,窃随当击贼,当顾见大惊,呵叱不能禁止。归以告蒙母,母恚欲罚之,蒙曰:“贫贱难可居,脱误有功,富贵可致。旦不探虎穴,安得虎子?”母哀而舍之。时当职吏以蒙年小轻之,曰:“彼坚子何能为?此欲以肉喂虎耳。”他日与蒙会,又蚩辱之。蒙大怒,引刀杀吏,出走,逃邑子郑长家。出因校尉袁雄自首,承间为言,策召见奇之,引置左右。数岁,邓当死,张昭荐蒙代当,拜别部司马。权统事,料诸小将兵少而用薄者,欲并合之。蒙阴赊贳,为兵作绛衣行滕,及简日,陈列赫然,兵人练习,权见之大悦,增其兵。从讨丹杨,所向有功,拜平北都尉,领广德长。从征黄祖,祖令都督陈就逆以水军出战。蒙勒前锋,亲枭就首,将士乘胜,进攻其城。祖闻就死,委城走,兵追禽之。权曰:“事之克,由陈就先获也。”以蒙为横野中郎将,赐钱千万。

是岁,又与周瑜、程普等西破曹公于乌林,围曹仁于南郡。益州将袭肃举军来附,瑜表以肃兵益蒙,蒙盛称肃有胆用。且慕化远来,于义宜益不宜夺也。权善其言,还肃兵。瑜使甘宁前据夷陵,曹仁分众围宁,宁困急,使使请救。诸将以兵少不足分,蒙谓瑜、普曰:“留凌公绩,蒙与君行,解围释急,势亦不久,蒙保公绩能十日守也。”又说瑜分遣三百人柴断险道,贼走可得其马。瑜从之。军到夷陵,即日交战,所杀过半。敌夜遁去,行遇柴道,骑皆舍马步走。兵追蹙击,获马三百匹,方船载还。于是将士形势自倍,乃渡江立屯,与相攻击,曹仁退走。遂据南郡,抚定荆州。还,拜偏将军,领寻阳令。

鲁肃代周瑜,当之陆口,过蒙屯下。肃意尚轻蒙,或说肃曰:“吕将军功名日显,不可以故意待也,君宜顾之。”遂往诣蒙。酒酣,蒙问肃曰:“君受重任,与关羽为邻,将何计略以备不虞?”肃造次应曰:“临时施宜。”蒙曰:“今东西虽为一家,而关羽实熊虎也,计安可不豫定?”因为肃画五策。肃于是越席就之,拊其背曰:“吕子明,吾不知卿才略所及乃至于此也。”遂拜蒙母,结友而别。时蒙与成当、宋定、徐顾屯次比近,三将死,子弟幼弱,权悉以兵并蒙。蒙固辞,陈启顾等皆勤劳国事,子弟虽小,不可废也。书三上,权乃听。蒙于是又为择师,使辅导之,其操心率如此。

魏使庐江谢奇为蕲春典农,屯皖田乡,数为边寇。蒙使人诱之,不从,则伺隙袭击,奇遂缩退,其部伍孙子才、宋豪等,皆携负老弱,诣蒙降。后从权拒曹公于濡须,数近奇计,又劝权夹水口立坞,所以备御甚精,曹公不能下而退。

曹公遣朱光为庐江太守,屯皖,大开稻田,又令间人招诱鄱阳贼帅,使作内应。蒙曰:“皖田肥美,若一收孰,彼众必增,如是数岁,操态见矣,宜早除之。”乃具陈其状。于是权亲征皖,引见诸将,问以计策。蒙乃荐甘宁为升城督,督攻在前,蒙以精锐继之。侵晨进攻,蒙手执枹鼓,士卒皆腾踊自升,食时破之。既而张辽至夹石,闻城已拔,乃退。权嘉其功,即拜庐江太守,所得人马皆分与之,别赐寻阳屯田六百户,官属三十人。蒙还寻阳,未期而卢陵贼起,诸将讨击不能禽,权曰:“鸷鸟累百,不如一鹗。”复令蒙讨之。蒙至,诛其首恶,余皆释放,复为平民。

是时刘备令关羽镇守,专有荆士,权命蒙西取长沙、零、桂三郡。蒙移书二郡,望风归服,惟零陵太守郝普城守不降。而备自蜀亲至公安,遣羽争三郡。权时住陆口,使鲁肃将万人屯益阳拒羽,而飞书召蒙,使舍零陵,急还助肃。初,蒙既定长沙,当之零陵,过酃,载南阳邓玄之,玄之者郝普之旧也,欲令诱普。及被书当还,蒙秘之。夜召诸将,授以方略,晨当攻城。顾谓玄之曰:“郝子太闻世间有忠义事,亦欲为之,而不知时也。左将军在汉中,为夏侯渊所围。关羽在南郡,今至尊身自临之。近者破樊本屯,救酃,逆为孙规所破。此皆目前之事,君所亲见也。彼方首尾倒悬,救死不给,岂有余力复营此哉?今吾士卒精锐,人思致命。至尊遣兵,相继于道。今子太以旦夕之命,待不可望之救。犹牛蹄中鱼,冀赖江汉,其不可恃亦明矣。若子太必能一士卒之心,保孤城之守,尚能稽延旦夕,以待所归者,可也。今吾计力度虑,而以攻此,曾不移日,而城必破,城破之后,身死何益于事,而令百岁老母,戴白受诛,岂不痛哉?度此家不得外问,谓援可恃,故至于此耳。君可见之,为陈祸福。”玄之见普,具宣蒙意,普惧而听之。玄之先出报蒙:“普寻后当至。”蒙豫敕四将,各选百人,普出,便入守城门。须臾普出,蒙迎执其手,与俱下船。语毕,出书示之。因拊手大笑。普见书,知备在公安,而羽在益阳,惭恨入地。蒙留(孙河),委以后事,即日引军赴益阳。刘备请盟,权乃归普等。割湘水,以零陵还之。以寻阳、阳新为蒙奉邑。

师还,遂征合肥,既撤兵,为张辽等所袭,蒙与淩统以死扦卫。后曹公又大出濡须,权以蒙为督,据前所立坞,置强弩万张于其上,以拒曹公。曹公前锋屯未就,蒙攻破之,曹公引退。拜蒙左护军、虎威将军。

鲁肃卒,蒙西屯陆口,肃军人马万余尽以属蒙。又拜汉昌太守,食下隽、刘阳、汉昌、州陵。与关羽分土接境,知羽骁雄,有并兼心,且居国上流,其势难久。初,鲁肃等以为曹公尚存,祸难始构,宜相辅协,与之同仇,不可失也。蒙乃密陈计策曰:“今令征虏守南郡,潘璋住白帝,蒋钦将游兵万人循江上下,应敌所在,蒙为国家前据襄阳,如此,何忧于操,何赖于羽?且羽君臣,矜其诈力,所在反复,不可以腹心待也。今羽所以未便东向者,以至尊圣明,蒙等尚存也。今不于强壮时图之,一日僵仆,欲复陈力,其可得邪?”权深纳其策,又聊复与论取徐州意。蒙对曰:“今操远在河北,新破诸袁,抚集幽、冀,未暇东顾。徐土守兵,闻不足言,往自可克。然地势陆通,骁骑所聘,至尊今日得徐州,操后旬必来争,虽以七八万人守之,犹当怀忧。不如取羽,全据长江,形势益张。”权尤以此言为当。及蒙代肃,初至陆口,外倍修恩厚,与羽结好。

后羽讨樊,留兵将备公安、南郡。蒙上疏曰:“羽讨樊而多留备兵,必恐蒙图其后故也。蒙常有病,乞分士众还建业,以治疾为名。羽闻之,必撤备兵,尽赴襄阳。大军浮江,昼夜驰上,袭其空虚,则南郡可下,而羽可擒也。”遂称病笃,权乃露檄召蒙还,阴与图计。羽果信之,稍撤兵以赴樊。魏使于禁救樊,羽尽擒禁等,人马数万,托以粮乏,擅取湘关米。权闻之,遂行。先遣蒙在前。蒙至寻阳,尽伏其精兵舳舻中,使白衣摇橹,作商贾人服,昼夜兼行,至羽所置江边屯候,尽收缚之,是故羽不闻知。遂到南郡,士仁、麋芳皆降。蒙入据城,尽得羽及将士家属,皆抚慰,约令军中不得干历人家,有所求取。蒙麾下士,是汝南人,取民家一笠,以覆官铠,官铠虽公,蒙犹以为犯军令,不可以乡里故而废法,遂垂涕斩之。于是军中震慄,道不拾遗。蒙旦暮使亲近存恤耆老,问所不足,疾病者给医药,饥寒者赐衣粮。羽府藏财宝,皆封闭以待权至。羽还,在道路,数使人与蒙相闻,蒙辄厚遇其使,周游城中,家家致问,或手书示信。羽人还,私相参讯,咸知家门无恙,见待过于平时,故羽吏士无斗心。会权寻至,羽自知孤穷,乃走麦城,西至漳乡,众皆委羽而降。权使朱然、潘璋断其径路,即父子俱获,荆州遂定。

以蒙为南郡太守,封孱陵候,赐钱一亿,黄金五百斤。蒙固辞金钱,权不许。封爵未下。会蒙疾发,权时在公安,迎置内殿。所以治护者万方,募封内有能愈蒙疾者,赐千金。时有针加,权为之惨戚,欲数见其颜色,又恐劳动,常穿壁瞻之,见小能下食则喜,顾左右言笑,不然则咄唶,夜不能寐。病中瘳,为下赦令,群臣毕贺。后更增笃,权自临视,命道士于星辰下为之请命。年四十二,遂卒于内殿。时权哀痛甚,为之降损。蒙未死时,所得金宝诸赐尽付府藏,敕主者命绝之日皆上还,丧事务约。权闻之,益以悲感。

蒙少不修书传,每陈大事,常口占为笺疏。常以部曲事为江夏太守蔡遗所白,蒙无恨意。及豫章太守顾邵卒,权问所用,蒙因荐遗奉职佳吏,权笑曰:“君欲为祁奚耶?”于是用之。甘宁粗暴好杀,既常失蒙意,又时违权令,权怒之,蒙辄陈请:“天下未定,斗将如宁难得,宜容忍之。”权遂厚宁,卒得其用。蒙子霸袭爵,与守冢三百家,复田五十顷。霸卒,兄琮袭候。琮卒,弟睦嗣。

孙权与陆逊论周瑜、鲁肃及蒙曰:“公瑾雄烈,胆略兼人,遂破孟德,开拓荆州,邈焉难继,君今继之。公瑾昔要子敬来东,致达于孤,孤与宴语,便及大略帝王之业,此一快也。后孟德因获刘琮之势,张言方率数十万众水步俱下。孤普请诸将,咨问所宜,无适先对,至子布、文表,俱言宜遣使修檄迎之,子敬即驳言不可,劝孤急呼公瑾,付任以众,逆而击之,此二快也。且其决计策意,出张、苏远矣。后虽劝吾借玄德地,是其一短,不足以损其二长也。周公不求备于一人,故孤忘其短而贵其长,常以比方邓禹也。又子明少时,孤谓不辞剧易,果敢有胆而已。及身长大,学问开益,筹略奇至,可以次于公瑾,但言议英发不及之耳。图取关羽,胜于子敬。子敬答孤书云:‘帝王之起,皆有驱除,羽不足忌。’此子敬内不能办,外为大言耳,孤亦恕之,不苟责也。然其作军屯营,不失令行禁止,部界无废负,路无拾遗,其法亦美也。”

评曰:曹公乘汉相之资,挟天子而扫群桀,新荡荆城,仗威东夏,于时议者莫不疑贰。周瑜、鲁肃建独断之明出众人之表,实奇才也。吕蒙勇而有谋,断识军计,谲郝普,禽关羽,最其妙者。初虽轻果妄杀,终于克己,有国士之量,岂徒武将而已乎!孙权之论,优劣允当,故载录焉。

\part{吴书十}
\chapter{程黄韩蒋周陈董甘凌徐潘丁传第十}

\part{吴书十一}
\chapter{朱治朱然吕范朱桓传第十一}

\part{吴书十二}
\chapter{虞陆张骆陆吾朱传第十二}

\part{吴书十三}
\chapter{陆逊传第十三}

\part{吴书十四}
\chapter{吴主五子传第十四}

\part{吴书十五}
\chapter{贺全吕周锺离传第十五}

\part{吴书十六}
\chapter{潘濬陆凯传第十六}

\part{吴书十七}
\chapter{是仪胡综传第十七}

\part{吴书十八}
\chapter{吴范刘惇赵达传第十八}

\part{吴书十九}
\chapter{诸葛滕二孙濮阳传第十九}

\part{吴书二十}
\chapter{王楼贺韦华传第二十}


\end{document}