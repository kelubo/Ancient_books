% 史记
% 史记.tex

\documentclass[12pt,UTF8]{ctexbook}

% 设置纸张信息。
\usepackage[a4paper,twoside]{geometry}
\geometry{
	left=25mm,
	right=25mm,
	bottom=25.4mm,
	bindingoffset=10mm
}

% 设置字体,并解决显示难检字问题。
\xeCJKsetup{AutoFallBack=true}
\setCJKmainfont{SimSun}[BoldFont=SimHei, ItalicFont=KaiTi, FallBack=SimSun-ExtB]

% 目录 chapter 级别加点(.)。
\usepackage{titletoc}
\titlecontents{chapter}[0pt]{\vspace{3mm}\bf\addvspace{2pt}\filright}{\contentspush{\thecontentslabel\hspace{0.8em}}}{}{\titlerule*[8pt]{.}\contentspage}

% 设置 part 和 chapter 标题格式。
\ctexset{
	part/name= {},
	part/number={},
	chapter/name={},
	chapter/number={}
}

% 设置古文原文格式。
\newenvironment{yuanwen}{\bfseries\zihao{4}}

% 设置署名格式。
\newenvironment{shuming}{\hfill\bfseries\zihao{4}}

% 注脚每页重新编号,避免编号过大。
\usepackage[perpage]{footmisc}

\title{\heiti\zihao{0} 史记}
\author{司马迁}
\date{}

\begin{document}

\maketitle
\tableofcontents

\frontmatter
\chapter{前言}

1.原文
2.注释
3.翻译

《史记》的作者是西汉人司马迁。司马迁,字子长,今陕西韩城市人,生于公元前145年,卒于公元前90年左右。他的生卒年大体和汉武帝相始终。

到司马迁生活的时代,中国已经有了三千多年的历史发展。这个历史发展,先是经历了传说中黄帝、颛顼、帝喾、尧、舜远古五帝的阶段。在这个阶段的开始,黄帝先后战胜了炎帝、蚩尤,被诸侯们拥立为天子,并逐步建立起具有原始国家性质的政权机构与社会职能,从而开启了中华大地上天下和合的文化传承,黄帝因此也就成了中华民族的人文始祖。此后,夏、商、周一统王朝的相继出现,延续着中华后代子孙均从黄帝所出的总体趋势,和合一体的文化传承在巩固和发展。特别是周王朝的制礼作乐,分封诸候,有力地加强了中央施政的威权,整个社会也进入了繁荣祥和而为后来孔子所向往赞叹的新境界。然而分封的出现,正好形成历史进程中的第一大变局,使西周之后紧接着出现的春秋、战国阶段,爆发出诸侯割据、群雄混战的乱象,其结果却正为中原文化向南传播,从而促使地方经济发展、区域文化繁荣提供了极好际遇。政权下移,竞争激烈,士林阶层的热情涌现,学术思想的相互碰撞,分裂危难中蕴藏着无穷生机。也正是在这个阶段,一个强悍坚定、充满自信的游牧民族从西方兴起,在与东方六国的争斗中逐步获得胜利,最终以其深刻的社会变革所获取的强大经济力量,支撑着强暴的战争手段,在数百年后又一次使分裂了的中国重归统一。

崭新的专制主义中央集权王朝的出现,形成了社会前进的新波峰。但是好景不长,秦王朝穷奢极欲,劳役繁重,不关心社会经济发展以安抚民众,结果付出了沉重代价,仅仅十五年,它就被波澜壮阔的农民起义推翻,中国历史由此进入短暂的秦楚之际的阶段。秦楚之际,风云变幻,经过八年,形成历史进程中统一后出现强烈社会动荡的第二大变局。然后是无土而王的刘邦建汉,经昌后、文、景采取道家无为措施,至武帝之初,已有七十余年,社会财富大量增加,国力充实,中国社会以此进入强盛阶段。汉武帝乘这个趋势内兴制度,外攘四夷,独尊儒术,远通西域,专制主义中央集权达到了历史上空前强大的规模。司马迁亲历了当时的灿烂与辉煌,感到非常兴奋。他决心以自己深邃的历史观察力,将这三千年的史事变化记录下来,并借以歌颂当下的伟大成就,他写的《史记》达到了这一要求。时代造就了司马迁,而《史记》就是历史发展到汉武帝时期所呈现的巨大文化成就。

司马迁对自己出身于史官世家感到非常自豪,《自序》中将家庭宗系的远祖追溯到了唐虞之际的重黎氏,就是强调家学传统的厚重。从小开始,司马迁在他父亲司马谈的教导和培育下,为将来能从事历史著述进行了各方面的准备。他十岁就读了古文,长大后还在京城听了董仲舒讲《公羊春秋》,听孔安国讲古文《尚书》。他父亲崇尚黄老,所著《论六家之要指》亦对他的学术理念产生了很深的影响。加上小时候先在家中“耕牧河山之阳”,二十岁后就开始壮游全国,了解各地的民情风俗并考察文物遗迹、访古问故,后来在汉武帝身边任职时又有仕宦之旅,司马迁一生的行踪遍及全国各地,这些都为他后来的写作创造了非常有利的条件。司马迁接续他父亲司马谈的职务担任太史令之后,在朝廷的工作还比较顺利,他思想上也是诚恳尽心的。殊不知遭受到李陵案的株连竟被下狱受腐刑,一场横祸给了他致命的打击。但他终于从这场灾难中走了出来,坚定了自己写史的信念。人生的转折使他的思想更加现实,更加深沉了。在这些条件的影响下,他终于完成了卓绝千古而不朽的《史记》的写作。

秉承时代的造就,司马迁写《史记》发挥了极大的创造功能。从全书体例、涉篇立意、多方取材、文章结构、叙事手法、评论艺术,直至书中每一个历史人物的处理,无一不贯穿着独具匠心的设计。《史记》主要的学术成就有如下几方面。

第一,创造了纪传体史书体例。《史记》有五体,本纪以事势主宰者的在位为线索,统览其事迹为全书的纲纪。表以世、年、月为体,记述形式简明扼要,纵横经纬以观天下的发展大势;诸序则搬其大义,辅本纪以明全书之纲。书以记国之大事,重经国要务而轻典制记载,亦有辅纲的意义,亦为后世史书典制体之滥觞。世家以“拱辰共毂”为要义,录分封而极重天下一统,于诸侯、世尊、重臣、国戚内容有别,然强本弱末之势显见。列传以“立功名”为人生价值,强调其人行事的原则风范、主观能动性及对国家民族进步所能做出的诸多贡献。本纪、表、书、世家、列传五体主要以人物活动为社会历史的主体,互相呼应,紧密连接,形成巨大的运载能量,支撑成一座具有辉煌文化价值的著述大厦,当是可堪赞叹的奇迹。

第二,撰述成一部百科全书式的通史。在《史记》以前,中国有《尚书》、《诗经》、《春秋》、《左传》、《国语》、《战国策》、《世本》、《楚汉春秋》等涉及史事的书,但它们有的甚至都不能称为完整意义上的历史记载。只有《史记》能将自上古传说时代的黄帝,直至汉武帝太初前后的史事,分别出阶段而又延续不断地记载下来,中国才出现了第一部通史。这件事本身的意义极为重大,它奠定了中国之通史传统的基础,而其后来的发展,使中国成为世界文明古国中唯一具有保持长久历史记载的国家,司马迁《史记》开创之功极巨。《史记》全书记载的内容非常广泛全面,它涉及政治、经济、军事、文化、天文、地理、水利、交通、民情、风俗等纷繁复杂的社会事务,它及于社会各阶层如天子、国君、重臣、士大夫,直至农、工、商、虞,还包括各类社会职业如日者、龟策、游侠、刺客,乃或守门屠夫、鸡鸣狗盗之徒等等。总之,社会生活的方方面面,无一不在它记载的视野之中。《史记》的记载着重于人,通过记人来表述国家世事的发展变化,并宣扬在其中所表现的优秀的民族精神及文化品格,还以其包罗万象的记载,阐发出关于国家治理及社会变化的诸多经验、活动法则,亦是人类智慧发展的全面展现。

第三,表现出著史的“实录”原则。这首先说明,作为史书,《史记》的记事总体是准确的。司马迁写史,他自己提出“考信于六”、“折中于夫子”、“总之不离古文者近是”等的要求,就是要掌握取材的可靠性。司马迁写史,依靠着典籍、文书、档案、访古问故以及亲身闻见的资料,加以比较研究以判定事实真伪后,才据以下笔记事的,所以《史记》才获得了“信史”的称誉。遇到材料矛盾说法不一时,司马迁采取“疑则传疑”、“疑者阙焉”的态度,不武断做结论,以避免记载论事的失误。司马迁写史,被后世论定为“良史之才”,就是因为他能“善序事理,辩而不华,质而不俚,其文直,其事核,不虚美,不隐恶”。这当中不仅要具备准确记载史事的品质,还要做到对诸端事务的论判,一定要以客观标准为基础,以当时公认的政治道德原则为准绳,仔细衡量后做出结论,而不能因为个人的情感爱好而违反历史发展的人类正义良知,歪曲编造,护恶献媚。这同时表明,作为史家,是肩负着极其重大的社会责任的。以此,《史记》作为一面旗帜,在历史上树立起了中国史家著史的“实录”家风。

第四,提出了撰史的著述宗旨,规定为“究天人之际,通古今之变,成一家之言”、“稽其成败兴坏之理”。

究天人之际,是说要考究“天”与社会人事的关系。《史记》中的“天”,包含有三方面的意思。一是自然的“天”,指日月星辰、雷电风雨之类,《天官书》是专门讲这方面的科学内容的。司马迁是当时有名的天文学家,他还亲身参与了太初历的制定。一是具有人格神的“天”,汉武帝时代采取董仲舒的《公羊春秋》和阴阳五行学说相结合的指导思想,宣传君权神授的说法来强化皇权,但《史记》的记述对这方面的某些说法表示怀疑,《伯夷列传》就直接提出:“傥所谓天道,是邪非邪!”一是在人们的主观意识之外但实际对社会客观事物的发展无形中起着推动作用的事势,大的方面说是“理”、规律一类。《史记》中这方面的表述是很多的,如《魏世家》的论赞语说:“天方令秦平海内,其业未成,魏虽得阿衡之佐,易益乎?”这个“天”,就是指天下事势,可见司马迁的见解是很高明的。

通古今之变,是说要考察古今历史及事件的变化。考察的方法首先是“原始察终”:凡事要搞清它的原委,来龙去脉,它和其他事物的关系以及它的最后结局,这也叫“综其终始”,是研究通变的基础。其次是“见盛观衰”:当一个事物发展到鼎盛的时候,要注意它内在可能存在的衰败迹象,这样才可以对造成这种事态的各种趋势做出合理的解释。再次是“承敝易变”:当一个事物已经衰败退出历史舞台,要弄清楚它衰败的原因,就应采取措施使事物发生变化,使之重新兴盛起来。通观古今的事势来考察历史的变化,无疑是一种较为科学的历史观,是对社会历史发展认识中的积极主动精神,有利于较为客观全面地揭示社会和人生规律的正确原则。

成一家之言,是说司马迁要提出自己对历史分析的独立看法,它有别于当时的学术诸家的观点。司马谈的《论六家之要指》提出了学术上“家”的概念,并在思想上基于汉初黄老政治的成就而全面肯定道家。司马迁生活的武帝时代是独尊儒术。面对现实社会的诸多矛盾,他既不可全然赞成道家,又不愿整体地顺服于儒家,于是他企图采取客观态度,对历史与现实的发展提出契合于历史学家良知意识的,故而会有异于独尊一统的独特看法,在认识上自成一家。这表明司马迁的著史,已将史学推向了区别于学术各家而独立发展的轨道,而其成就影响所及,致使西汉末期刘向、刘歆父子在学术文献分类上有经、史、子、集的裁断。故此司马迁在历史著述上“成一家之言”的提出,不仅在史学上,而且在中国整个学术发展史上的贡献也是居功至伟的。

稽其成败兴坏之理,是说研究历史,要探求社会变化过程中的法则,以便总结经验教训,从而提高人们的认识以警示后来。《史记》全书五十二万六千五百字,内容广泛丰富,但归根结底就是一句话,它是论“治道”的书,也就是说,大的方面讲如何治理好一代朝廷、一个国家,小的方面说是如何治理好一方社会、一人之身。综合起来是要运用考察所认识的法则,促使人们的行动能更好地获得事业的成功。《高祖功臣侯者年表序》直接提出:“居今之世,志古之道,所以自镜也,未必尽同。帝王者各殊礼而异务,要以成功为统纪,岂可乎?”这是具体明确了历史记述的目的,与《太史公自序》中所说的“述往事,思来者”结合起来看,表明司马迁精心撰就的《史记》,已经具有非常明确的社会功能,这成为中国史学的又一优秀传统。

第五,具有极高的文学成就。《史记》不仅是一部信史,而且是一部优秀的文学作品。历代学者有评论认为西汉文人都会写文章,但文章写得最好的两个人,司马相如之外就是司马迁,所以《史记》无论是作为史学著作还是文学著作,在历史上都是颇负盛名的。毛泽东在《为人民服务》一文中曾经称司马迁是“中国古代的文学家”。鲁迅评价《史记》为“史家之绝唱,无韵之离骚”,也恰好是肯定了《史记》在史学、文学两方面的贡献。《史记》的文体是散文,多属传记文学,无论是写人还是叙事,都秉承时代的气息,具有一种恢宏的气势。文章视野开阔,议论洒脱,布局宏大,运笔奔泻,细加品读似觉司马迁胸中有百万雄兵,挥戈呼号而无能阻遏。《史记》的重要特点是通过写人而来写史,但是它又把这历史中人塑造成一个个活生生的典型形象,从而在历史上树立起许多英雄人物的丰碑。基于自身具有的悲愤情结,司马迁对其中一些英雄的忠君爱国却惨遭不幸,才华横溢却命运多舛的人生结局,给予了深切同情。《史记》写人除了笔法细腻,镂雕人微,设计灵巧,动静自如之外,尤其善于刻画人物的心理状态和表现人物的丰富情感,在揭示人物内心世界的同时,借以展现社会风情的各色面貌,以见其历史发展的绚丽璀璨。《史记》叙事极具语言艺术,其所资之材料虽来源不同,却均能熔铸为一篇而不现斧凿之迹。叙述文字生动流畅,评论语词深刻犀利,文如其人,论如其事,仔细读来,常能使人身临其境并能获得穿透心灵般的世事领略和人生感受。《史记》叙事在表彰人物及其功名时,能有效地展现其所述历史时期内人类的诸多谋略与最高的文化成就,因而《史记》亦是中国古代三千年的智慧集结作品,于启迪后代,促进民族思想的进步,同样极具价值。

司马迁《史记》所获得的这些伟大成就,使它成为中华文化发展史上的瑰宝,对后来中国史学、文学的不断进步繁荣,做出了不可灭的贡献。

《史记》是司马谈、司马迁父子的私家著史,撰就后只将副本送往京师,而正本由自家藏之“名山”,直到司马迁的外孙杨恽的时候才开始向社会传播。《史记》在传播过程中缺失了10篇,今见《史记》有后人补作的痕迹。《史记》文字难读,传播中有人相继为它作注。南北朝时宋人裴骃作《史记集解》,唐时司马贞作《史记索隐》、张守节作《史记正义》,最为有名,号称“三家注”。三家注本独立成书,至宋时始将它们合刻于所注《史记》的正文下方,以便阅读。清人梁玉绳作《史记志疑》,考论详确,被认为是“三家注”之后的又一家,颇受重视。日本人泷川资言作《史记会注考证》,流传甚广。今人韩兆琦有《史记笺证》,是最新的汇注之作,可资参阅。

《史记》的古本残卷现存最早的当数北宋刊本《史记集解》,现藏中国国家图书馆。通行的旧本,现在最早的有南宋黄善夫刻本,经商务印书馆影印,收于《百衲本二十四史》,现知亦已另行标点出版。另外,还有明嘉靖、万历间南北监刻的《二十一史》本、毛氏汲古阁刻的《十七史》本以及清乾隆四年(1739)武英殿《二十四史》附考证本。其中,1986年12月上海古籍出版社及上海书店将武英殿本编辑影印为《二十五史》(加上《清史稿》)全十二册本,多有流行。今见《史记》最普通的本子,是中华书局于1959年9月出版的点校本,它是一个时期中最精善、最流行、最实用的本子,今此本已于2014年8月出版修订本。

《史记》有十二本纪,十表,八书,三十世家,七十列传。共一百三十篇。

《史记》有五帝(黄帝、颛顼、帝喾、尧、舜)、夏、殷、周、秦、秦始皇、项羽、高祖、吕太后、孝文、孝景、今上(武帝)十二篇本纪。司马迁在《太史公自序》中说:“罔罗天下放失旧闻,王迹所兴,原始察终,见盛观衰论考之行事,略推三代,录秦汉,上记轩辕,下至于兹,著十二本纪,既科条之矣。”是说要略三代,详秦汉,就王迹所兴的问题,来通观古今的变化。所以,《史记》的本纪,实质上乃是要就历史的事势发展,来总览天下的大局,作为一种“科条”,起到全书纲领的作用,并不单纯地强调记帝王、天子。刘知幾沿用裴松之的说法,“天子称本纪,诸侯曰世家”,还说本纪是“唯记天子一人”、“系日月以成岁时,书君上以显国统”之类,是以《汉书》之后的正统观念来范围《史记》,并不完全符合司马迁创设本纪这一体裁的原意。这是任何一个研读《史记》的人首先应该加以明确的。

《史记》是由司马迁撰写的中国第一部纪传体通史。记载了上自上古传说中的黄帝时代,下至汉武帝元狩元年(前122年)间共3000多年的历史(哲学、政治、经济、军事等)。

最初没有固定书名,或称《太史公书》,或称《太史公传》,也省称《太史公》。

“史记”本是古代史书通称,从三国时期开始,“史记”由史书的通称逐渐成为《太史公书》的专称。

《史记》与后来的《汉书》(班固)、《后汉书》(范晔、司马彪)、《三国志》(陈寿)合称“前四史”。刘向等人认为此书“善序事理,辩而不华,质而不俚”。

与司马光的《资治通鉴》并称“史学双璧”。

全书略于先秦,详于秦汉,有本纪十二篇,表十篇,书八篇,世家三十篇,列传七十篇,共一百三十篇。

本纪,实际上就是帝王的传记。同时,本纪也是全书的总纲,用编年体的方法记事。在本纪的写作中,司马迁采取了详今略远的办法,时代愈远愈略,愈近愈详。本纪从传说中的黄帝开始写起,是因为黄帝是中华民族的始祖。司马迁还将项羽列入本纪,一是秦汉间几年“政由羽出”,一是推崇其人格。

表,用表格的形式记述重大事件,纲举目张,以简御繁,一目了然,便于观览、检索。

书,记载历代朝章国典,以明古今制度沿革的专章。班固《汉书》改称志,以后成为通例。书的修撰,为研究各种专门史提供了丰富的资料。

世家,记载诸侯王国之事。因为诸侯开国承家,子孙世袭,所以他们的传记叫作世家。司马迁把孔子和陈涉也列入世家,是一种例外。孔子虽非王侯,但却是传承三代文化的宗主,更何况汉武帝时儒学独尊,孔子是儒学的创始人,将之列入世家也反映了思想领域的现实情况。至于陈涉,他是第一个起义亡秦的领导者,而且是三代以来以平民起兵而反残暴统治的第一人,而亡秦的侯王原来又多是他的部下。司马迁将之列入世家,把他的功业和商汤推翻夏桀,武王伐纣,孔子作《春秋》相比,将他写成震撼秦帝国统治、叱咤风云的伟大历史英雄,反映了作者进步的历史观。

列传是记载帝王、诸侯以外的各种历史人物的,有单传,有合传,有类传。单传是一人一传,如《商君列传》、《李斯列传》等。合传是记二人以上,如《管晏列传》、《老子韩非列传》等。类传是以类相从,把同一类人物的活动,归到一个传内,如《儒林列传》、《循吏列传》、《刺客列传》等。司马迁把当时我国四周少数民族的历史情况,也用类传的形式记载下来,如《匈奴列传》、《西南夷列传》、《大宛列传》等,这就为研究我国古代少数民族的历史,提供了重要的史料来源。

\mainmatter

\part{卷一}

\chapter{五帝本纪第一}

杨慎:《五帝纪》亦非太史公极笔。

归震川:史公究是秦汉时人作,始皇项羽本纪其事雄伟,笔力与之称,五帝三王本纪便时见其陋,然古书存者盖亦少矣。

记述了我国古代神话传说中的五个圣明的帝王,即黄帝、颛顼、帝喾、尧、舜。主要取材于《世本》、《尚书》、《大戴礼记》。有关黄帝的传说,春秋、战国以至西汉有不少,但因为“荒诞离奇”,与真正的人类历史距离太远,孔子、孟子都不怎么讲。司马迁把他从众多神话人物中选出来,又择取了一些比较“可信”的材料,将之作为《史记》的开端。随后,他将颛顼、帝喾、尧、舜、禹、汤、文武,春秋战国时期的中原诸国、秦、楚、吴、越,以及周边的匈奴、东越、南越等都说成是黄帝的子孙,这就为中国人确定了“始祖”,同时又确定了华夏与周边各民族的同胞兄弟关系。

尧、舜被儒家称为“圣人”,见之于儒家著作的说法较多,尤其尧、舜“禅让”的故事更被后世传为美谈。司马迁之写尧、舜两位古代帝王,从中寄托了自己的政治理想,并使之与秦、汉以来的专制政治形成对照,其用意是显而易见的,尧、舜无疑是《史记》中最使司马迁尊崇的大公无私的理想帝王。

一篇文章开头最难写,一部书,开篇也最难下笔。现代人写文章开篇有序,可以说明本意。古人写书,序在最后,这样,第一篇写出来,摆在人们的面前就能表明著者的主要作意。所以《五帝本纪》表示了贯穿于《史记》全书的几个方面的思想。

司马迁写《史记》究竟应该从哪里开始?司马迁是知道传说中在黄帝以前还有神农氏、伏羲氏、无怀氏和泰帝。但是他毅然从黄帝开始,黄帝以前则付之不闻不问。黄帝是个传说人物,他就妥贴地将黄帝只写成人而不是神,更废弃三头六臂、蛇身人首、人身牛首等说法,说黄帝是一个地地道道、非常聪明能干的人,有生也有死,没有长生不死,更没有乘龙上天。黄帝管理的社会,有行政机构,设置了官吏,制订了历法,组织了人们的生产生活。总之,从黄帝开始的中国历史,是质朴真实的人类社会史,不是什么神灵的再现,也将一切荒诞怪异的诬妄之说排除在历史之外,这样处理,在当时是具有重要政治意义的。

黄帝的成功是依靠修德。黄帝是在反对暴虐侵凌的基础上受到诸部落拥戴的,是在战败了炎帝、蚩尤以后由于诸侯的归顺而成为天子的,不是按照上帝的意志受天命而来统治人民的。因此,黄帝时最初国家政权的建立归之于“德”,以后世系传递的禅让亦依以“德”,五帝尤其是尧、舜都具有高尚的德性。文篇的最后提出“以章明德”连起以后诸篇。司马迁是将五帝的治理归结到儒家理想的德治,以表示中国历史开始时的事势发展。黄帝有妻儿子女。他的后代谁德性好谁就嗣位为天子,其孙颛顼接替了他,他的曾孙又接替了颛顼,帝喾是颛顼的族子,帝喾死,大儿子挚不善,放勋即位是为尧,帝舜又是黄帝的八世孙。在中国境内这个统系的形成,说明了整个中原地区的“氏姓之所自出”。其他篇目中凡记述诸侯国或少数民族的氏姓来源时,司马迁都与黄帝的这个统系联结起来,总的是说明了当时中国境内各族是有一个共同的祖先,他就是黄帝。尽管地区有南北,文化有深浅,大家都是异支同源,这对形成民族的凝聚力很有积极意义。

《史记》第一篇很注意表述国境的四界。黄帝时就有东西南北的所到处,颛顼“北至于曲陵,南至于交,西至于流沙,东至于蟠木。动静之物小大之神,日月所照,莫不砥属”,帝喾也是“日月所照,风雨所至,莫不从服”。总之当时所知的世界都在中国政治文化的范围之中,这表现了民族的宏伟气魄,也显示了司马迁具有高远的眼光。而且,黄帝时是“万国和”,帝喾是“溉执中而遍天下”,尧“合和万国”、“四海之内,咸戴帝之功”。这是在国家概念的面前,强调历史的统一趋势以及协调和合的政治精神,其影响也是深远的。
实际上,历史学家常常是以当代政治的基本模式去观察、研究和说明古代社会的。司马迁对五帝的叙述,就是天下已经归于一统形成为专制主义中央集权时,将远古意向化的一种表现。但反过来,对古代社会的研究有力地说明了现在,支持了现实的政治,这正是本篇作用之所在。

本篇论赞(“太史公曰”)是《史记》中出现的第一篇评论,历来为学者所瞩目。司马迁之编定古史,均是以古代典籍和自己的亲身考察作为材料依据的,在“百家言黄帝,其文不雅驯”的情况下,他依据的基本原则是“总之不离古文者近是”。司马迁认为依据什么样的材料和如何编撰成书,都是需要以一定的思想为指导的,这当中富于深刻的哲理,而他自己的作史意图是不容易被人所理解的。因此他慎重提出“非好学深思,心知其意,固难为浅见寡闻者道也”。这是明确地宣布作史应具有不同的主张和是非观念。司马迁要为自己矢志追求的目标而不懈地奋斗。

\begin{yuanwen}
黄帝\footnote{我国古代传说时期的原始社会的帝王,为中原各族的共同祖先。}者,少典\footnote{传说中的有熊氏部落,黄帝、炎帝均为其子孙。}之子\footnote{后代。},姓公孙\footnote{据传黄帝原姓公孙,后因长于姬水,改姓姬。},名曰轩辕\footnote{一说为古地名,在今河南新郑县西北。另一说,梁玉绳《史记志疑》认为,轩辕之丘是因黄帝得名,不是黄帝从轩辕之丘得名。}。生而神灵\footnote{据传黄帝母附宝二十四月而生黄帝于寿丘(今山东曲阜市东北),头额如太阳,眉宇如龙骨。},弱\footnote{指初生不久的婴儿。又特指出生七十天以内的小孩。}而能言,幼而徇齐\footnote{一说通“迅疾”,机敏之意。或说通“迅给”,指口才锋利。},长而敦\footnote{忠厚诚实。}敏,成而聪明。
\end{yuanwen}

黄帝,少典氏的后代,姓公孙,名叫轩辕。刚出生就表现出神奇灵异的一面,很小的时候就会说话了,年幼时就思维敏捷,稍大一些则纯朴勤勉,成年以后能够明辨是非。

黄帝,是少典族的子孙,姓公孙,名叫轩辕。他生下来就显出神灵,七十天内就能说话,幼小的时候就很机智,长大后敦厚机敏,成年后闻见广博对事明辩。

\begin{yuanwen}
轩辕之时,神农氏\footnote{传说中的古代帝王之一,因他教民耕种,故称。一说神农即炎帝。}世衰。诸侯相侵伐,暴虐百姓,而神农氏弗能征。于是轩辕乃习用干戈\footnote{泛指武器。干是盾牌,戈是一种长柄横刃兵器。},以征不享\footnote{不朝贡的部族。享,进献。},诸侯咸\footnote{都。}来宾从\footnote{服从,归顺。}。而蚩尤\footnote{传说中九黎部族首领。}最为暴,莫能伐。炎帝欲侵陵\footnote{陵藉,欺侮。}诸侯,诸侯咸归轩辕。轩辕乃修德振兵,治五气\footnote{一说为五行之气。一说指晴、雨、冷、热、风五种气候。},蓺\footnote{y\`i,种植。}五种\footnote{即黍、稷、稻、麦、菽五种作物。},抚万民,度四方\footnote{丈量四方的土地。du\'o},教熊、罴\footnote{p\'i,人熊、马熊。}、貔\footnote{p\'i,似虎之猛兽。}、貅\footnote{xi\=u,传说中的猛兽名。}、貙\footnote{ch\=u,虎属猛兽,似狸而大。}、虎\footnote{都是猛兽名。《索隐》认为这些猛兽经过训练可以作战。《正义》认为这些猛兽名是用来给军队命名的,借以威吓敌人。六种猛兽可能是六个氏族的图腾。},以与炎帝战于阪泉\footnote{古地名,在今河北涿鹿县东南。b\v{a}n}之野。三战,然后得其志。蚩尤作乱,不用帝命。于是黄帝乃征师诸侯,与蚩尤战于涿鹿\footnote{古地名,在今河北涿鹿县东南。zhuō}之野,遂禽\footnote{通“擒”,抓捕。}杀蚩尤。而诸侯咸尊轩辕为天子,代神农氏,是为黄帝。天下有不顺者,黄帝从而征之,平者去之\footnote{使之离开。},披\footnote{分开、劈开。}山通道,未尝宁居。
\end{yuanwen}

轩辕的时候,正处在神农氏衰落的时代。诸侯之间相互攻伐,残害百姓,而神农氏没有能力征讨。于是轩辕就训练士兵使用武器,来征讨不来朝贡的人,诸侯就都俯首称臣了。可是蚩尤非常残暴,谁也无法征服他。炎帝想要侵犯诸侯,诸侯就都归顺了轩辕。于是轩辕推广德行,整顿军队,调和五行,种植五谷,安抚百姓,测量土地,训练像熊罴、貔貅、貙虎一样凶猛的士兵,率领他们在阪泉的郊野和炎帝交战。经过三次交战,黄帝的军队获得胜利。这时蚩尤作乱,不听黄帝的命令。于是黄帝就征调诸侯的军队,在涿鹿的郊野和蚩尤交战,最后擒获并杀死蚩尤。这时诸侯都尊奉轩辕为天子,取代神农氏,这就是黄帝。天下有不顺从的,黄帝就去征讨,直到平定才离开。从此黄帝开山通路,从来没有安稳休息的时候。

轩辕的时候,神农氏的后代道德衰薄。各地方的诸侯互相侵犯攻伐,残害百姓,但是神农氏没有能力征讨他们。在这种情况下轩辕就时常动用军事力量,去征讨诸侯中不来朝享的人,四方诸侯因此都来称臣归服。但是蚩尤最为残暴,还没有谁能去征讨他。炎帝也想侵犯凌辱诸侯,于是四方诸侯都来归附轩辕。轩辕修治德政并整肃军旅,顺应四时五方的自然气象,种植黍、稷、菽麦、稻等农作物,抚慰千千万万的民众,丈量

陈仁锡:《五帝纪》有谓非太史公极笔者,以少跌宕处耳,正惟无跌宕,乃太史公极用意之文,其叙次征诛、揖让、朝觐、会同,圣人经世大典,严整慎重,一切齐谐怪语之书,不能轻入,所以为史中之经,而高古质邃尤为可喜。

\begin{yuanwen}
东至于海,登丸山\footnote{山名,在今山东临朐县东北。},及岱宗\footnote{泰山。人称泰山为五岳之首,四岳所宗,泰山又名“岱”,故称。}。西至于空桐\footnote{崆峒,山名,在今甘肃省平凉市西。},登鸡头\footnote{空桐山主峰。}。南至于江\footnote{长江。},登熊\footnote{熊耳山,在今湖南益阳市西。}、湘\footnote{湘山,又名君山、洞庭山,在今湖南岳阳洞庭湖中。}。北逐荤粥\footnote{x\=un y\`u,古代北方民族,秦汉以后称“匈奴”。},合符\footnote{会诸侯以验证符节。}釜山\footnote{山名,一说在今河北怀来东,一说在今河北保定市徐水区西,又说在今河南偃师、灵宝等地。},而邑\footnote{都邑,作动词,建立都邑。}于涿鹿之阿。迁徙往来无常处,以师兵为营卫。官名皆以云命,为云师\footnote{《史记集解》引应劭曰:“黄帝受命,有云端,故以云记事也。春官为青云,夏官为缙云,秋官为白云,冬官为黑云,中官为黄云。”又引张晏曰:“黄帝有景云之应,因以名师与官。”}。置左右大监\footnote{相传黄帝所置监察官,监察各地、各部落。},监于万国。万国和,而鬼神、山川、封禅\footnote{古代帝王登名山,封地为坛曰封,扫地而祭曰禅sh\`an,用以祭祀天地,庆祝成功和太平。}与为多\footnote{指规模大。}焉。获宝鼎\footnote{古多以鼎为王朝相传之重器,故称为宝。},迎日推筴\footnote{根据神策占卜之征兆结果来推算未来日、月、朔、望等情况。筴,同“策”,亦叫神策,卜筮所用之蓍草。}。举风后、力牧、常先、大鸿\footnote{四人均为传说中的黄帝之大臣,其中以风后为相,力牧为将。}以治民。顺天地之纪\footnote{规律。},幽明之占,死生之说,存亡之难\footnote{不容易,意指道理。}。时\footnote{按季节。一说通“莳”,栽种。}播百谷草木,淳化\footnote{驯化。}鸟兽虫蛾,旁罗\footnote{广泛地罗列、观察。}日月星辰,水波\footnote{如水力一样摇动。波,通“播”,簸扬,摇动。}土石金玉,劳勤\footnote{烦劳勤苦。}心力\footnote{心思与能力。}耳目,节用水火材物。有土德之瑞\footnote{祥瑞,吉利的征兆。},故号黄帝。
\end{yuanwen}

东面到大海,登上丸山,一直到泰山。西面到空桐山,登上鸡头山。南面到长江,登上熊山、湘山。在北面驱逐荤粥,来到釜山与诸侯合验符契,并在涿鹿山坳处建立都邑。黄帝四处迁徙,没有固定的住处,派军队为自己宿卫。官职都是用云来命名,称为云师。设置左右大监,监理各国。从此天下各国安定,因此祭祀鬼神、山川、封禅的事情,在黄帝时最多。黄帝得到宝鼎,观测太阳,推算历法。他任用风后、力牧、常先、大鸿来治理百姓。黄帝顺应天地四时的规律,推测阴阳气候的变化,论说生死的道理,分析存亡的原因。他按时节栽种谷物和草木,驯化鸟兽和昆虫,包罗日月星辰,泽及土石金玉,劳烦身心耳目,节约各种器物。有土德的祥瑞,因此号为黄帝。

\begin{yuanwen}
黄帝二十五子,其得姓\footnote{表明部落系统的称号,后发展成为独立的氏族。}者十四人。

黄帝居轩辕之丘,而娶于西陵\footnote{当时传说中的部落名。}之女,是为嫘祖\footnote{léi}。嫘祖为黄帝正妃,生二子,其后皆有天下:其一曰玄嚣\footnote{xiāo},是为青阳,青阳降居\footnote{让帝子为诸侯。降,下也。}江水;其二曰昌意,降居若水\footnote{古水名,即今四川境内的雅砻江。}。昌意娶蜀山氏\footnote{传说中的部落名。}女,曰昌仆,生高阳,高阳有圣德焉。黄帝崩,葬桥山\footnote{山名,又名子午山,在今陕西黄陵县北。}。其孙昌意之子高阳立,是为帝颛顼\footnote{zhuān xū}也。
\end{yuanwen}

黄帝有二十五个儿子,其中获得姓氏的有十四个人。

黄帝住在轩辕山,娶西陵氏的女子为妻,就是嫘祖。嫘祖是黄帝的正妃,生了两个儿子,他们的后代都曾经统治天下:第一个叫玄嚣,也就是青阳,他在江水立国;第二个叫昌意,他在若水立国。昌意娶蜀山氏的女子为妻,名叫昌仆,生下高阳。高阳是个具有至高道德的人。黄帝去世后,被埋葬在桥山。他的孙子、昌意的儿子高阳继承帝位,这就是帝颛顼。

\begin{yuanwen}
帝颛顼高阳者,黄帝之孙而昌意之子也。静渊\footnote{宁静深远。}以有谋,疏通\footnote{疏旷通达。}而知事;养材\footnote{生产物质财富。材,《大戴礼》作“财”。}以任地\footnote{开发利用土地。},载时\footnote{按照天象变化规律行事。}以象天,依鬼神以制义\footnote{指制定各种尊卑的义理。},治气\footnote{指四时五行之气。}以教化,絜(洁)诚\footnote{洁心诚意。}以祭祀。北至于幽陵\footnote{幽州,今河北省北部、辽宁省西南一带。},南至于交阯\footnote{亦作“交趾”,古地名,泛指今广东、广西的大部分和越南北部、中部一带。zhǐ},西至于流沙\footnote{今内蒙腾格里沙漠一带。},东至于蟠木\footnote{即扶桑,古指日出之处。}。动静之物,大小之神\footnote{国中的大小神祗。大神指五岳和四渎(江、河、淮、济)之神,小神指小山小河之神。},日月所照,莫不砥属\footnote{平服归顺。砥,dǐ,平定。}。

帝颛顼生子曰穷蝉。颛顼崩,而玄嚣之孙高辛立,是为帝喾\footnote{kù}。
\end{yuanwen}

帝颛顼高阳,他是黄帝的孙子,昌意的儿子。他深沉而有谋略,通达而明事理;他让人耕作时充分利用土地,做事时顺应自然规律,依从鬼神来制定礼仪,调和五行来教化百姓,虔诚恭敬地进行祭祀。他向北到达幽陵,向南到达交阯,向西到达流沙,向东到达蟠木。事物不论动静,神灵不论大小,只要是日月能够照到的地方,没有不归顺他的。

帝颛顼生的儿子叫穷蝉。颛顼去世以后,玄嚣的孙子高辛继承帝位,这就是帝喾。

\begin{yuanwen}
帝喾高辛者,黄帝之曾孙也。高辛父曰蟜极\footnote{jiǎo},蟜极父曰玄嚣,玄嚣父曰黄帝。自玄嚣与蟜极皆不得在位,至高辛即帝位。高辛于颛顼为族子\footnote{堂侄。}。

高辛生而神灵,自言其名。普施利物,不于其身。聪以知远,明以察微。顺天之义,知民之急。仁而威,惠而信,修身而天下服。取地之财而节用之,抚教万民而利诲之,历日月\footnote{根据日月运行情况制订历法。}而迎送\footnote{指推算弦、望、晦、朔。}之,明鬼神而敬事\footnote{侍奉。}之。其色\footnote{仪表,神态。}郁郁\footnote{犹“穆穆”,肃穆的样子。},其德嶷嶷\footnote{yí,高尚,杰出。}。其动也时\footnote{合乎时宜。},其服也士\footnote{一般人士。}。帝喾溉\footnote{灌溉。一说通“概”,本义指量粮食时用来刮平升斗的木板,引申义为公平。}执中\footnote{持平,公正。}而遍天下,日月所照,风雨所至,莫不从服。

帝喾娶陈锋氏\footnote{又作“陈丰氏”,传说中的部落名,其女子曰庆都。}女,生放勋。娶娵訾氏\footnote{jū z\=i,传说中的部落名,其女子曰常仪。}女,生挚。帝喾崩,而挚代立。帝挚立,不善,崩,而弟放勋立,是为帝尧。
\end{yuanwen}

帝喾高辛,是黄帝的曾孙。高辛的父亲叫蟜极,蟜极的父亲叫玄嚣,玄嚣的父亲就是黄帝。从玄嚣到蟜极都没能继承帝位,到高辛时才登上帝位。高辛是颛顼的族侄。

高辛生下来就表现出神奇灵异的一面,能够说出自己的名字。他广施恩惠,不为自己着想。凭借聪慧来探知远处之事,凭借明智来洞察细微之情。他顺应上天的道义,了解百姓的疾苦。仁德而不失威严,慈惠而遵守诚信,修养身心而使天下人都归服。他获取土地的财富而节约使用,安抚教化百姓而用利益引导他们,推算日月的运行规律并迎来送往,明白鬼神的祭祀礼仪并虔诚供奉。他仪表堂堂,道德高尚。他的举动符合时宜,他的服饰就像士人。帝喾秉持公正的举措遍及天下,日月所能够照到的地方,风雨所能够降临的地方,没有不服从他的。

帝喾娶陈锋氏的女子为妻,生下放勋;娶娵訾氏的女子为妻,生下挚。帝喾去世以后,挚继承帝位。帝挚在位期间,不行善政,他的弟弟放勋登上帝位,这就是帝尧。

郭嵩焘:案史公赞称‘百家言黄帝,其文不雅驯’,《五帝纪》但叙其德而不详其事,以事之著见于百家者,皆非雅驯者也。伏羲之蛇首人身,神农之人身牛首,皆其类也。杨升庵谓‘《五帝纪》非史公极笔’,固也。然史公意在雅驯而已,太古荒遐,传闻缪悠,史公于此为有断制。

\begin{yuanwen}
帝尧者,放勋\footnote{帝号曰“尧”,名“放勋”,国号曰“陶唐”。}。其仁如天\footnote{《史记索隐》:“如天之函养也。”},其知\footnote{通“智”。}如神。就之如日\footnote{《史记索隐》:“如日之照临,人咸依就之,若葵藿倾心以向日也。”},望之如云\footnote{《史记索隐》:“如云之覆渥,言德化广大而浸润生人,人咸仰望之,故曰如百谷之仰膏雨也。”}。富而不骄,贵而不舒\footnote{放纵,恣意而行。怠慢,松懈。}。黄收\footnote{冕名,其色黄,故称“黄收”。}纯衣\footnote{即“缁(z\=i)衣”,黑衣。纯,读曰“缁”。《史记集解》引郑玄曰:“纯衣,士之祭服。”},彤\footnote{t\'ong,赤红色。}车乘\footnote{四马拉一车。}白马。能明驯德\footnote{顺天应人的美德。驯,同“顺”。},以亲九族\footnote{泛指自己的宗族和外戚。}。九族既睦,便章\footnote{也作“辨章”,辨别彰明,治理的意思。}百姓\footnote{这里指百官。}。百姓昭明\footnote{指各自的权利、职责、义务分明。},合和万国\footnote{指天下和谐一致。}。
\end{yuanwen}

帝尧,名放勋。他的仁德有如苍天,覆盖大地。他的智慧有如神灵,无所不晓。人们对他的归附,如同葵花向阳。人们对他的企盼,有如大旱之望云雨。他富有而不骄奢,他尊贵而不放纵。他戴着黄色的帽子,穿着黑色的衣裳,坐着红色的车子,拉车的都是白马。他有顺天应人的美德,能使自己的九族亲善。九族亲善后,便进一步治理朝廷百官。等到朝廷百官的职分明确且又各司其职,再进一步使天下万国都变得融洽和睦。

帝尧名叫放勋。他的仁德像天空一样浩大,他的智慧像神灵一样高深。人们像追逐太阳一样跟随着他,像遥望云彩一样仰慕着他。帝尧富有却不骄纵,尊贵却不傲慢。他头戴黄色冕冠,身穿黑色朝服,乘坐红色车驾,用四匹白马拉着。他能发扬高尚的品德,让九族亲密和睦。九族已经和睦,再去考察百官。百官的善恶已经彰明,再去协调各国的关系。

\begin{yuanwen}
乃命羲、和\footnote{羲氏、和氏的并称,均为重黎氏后代。尧命羲仲、羲叔、和仲、和叔分驻四方,观天象,制历法。},敬顺昊\footnote{hào,深远广大。}天,数\footnote{历数。}法\footnote{准则,规律。}日月星辰,敬授民时。

分命\footnote{分派,派出。}羲仲,居郁夷\footnote{亦作“蜗夷”,今山东半岛一带。},曰旸\footnote{y\'ang}谷\footnote{也作“汤谷”,相传为日出之处。}。敬道\footnote{同“导”,引导。}日出,便程\footnote{分别次第,使做事有步骤。辨别日程、时令。便,通“辨”,别。}东作\footnote{春天的农事活动。}。日\footnote{指春分。}中\footnote{正中。},星鸟\footnote{有歧义。鸟星,即“七星”,也单称为“星”,是二十八宿中的东方七宿之一。(朱雀,即二十八宿中的南方七宿。)},以殷\footnote{正,正定。}中春。其民析\footnote{分散,分散到田野上进行农业劳动。},鸟兽字\footnote{乳也,谓产子、哺乳。}微\footnote{同“尾”,交尾。}\footnote{也作“孳尾”,交配繁衍。}。申命\footnote{任命时予以告诫。申,重、又。}羲叔,居南交\footnote{南方的交趾。}。便程南为\footnote{夏季的劝农事务。},敬致\footnote{敬行教化以求得功效。指太阳回归。}。日永\footnote{指夏至,这一天白昼最长。永,长。},星火\footnote{也称“大火”、“商星”,即心宿,是二十八宿中的东方七宿之一。夏至傍晚时刻,此星出现在正南方面。},以正中夏。其民因\footnote{就,指老弱到田中帮助丁壮务农。},鸟兽希革\footnote{指夏季炎热,鸟兽皮上毛羽稀少。希,同“稀”。革,兽皮。}。申命和仲,居西土,曰昧\footnote{m\`ei}谷\footnote{神话中的日落之处。《史记集解》引孔安国曰:“日入于谷而天下冥,故曰昧谷。此居治西方之官,掌秋天之政也。”}。敬道日入,便程西成\footnote{秋收之事务。}。夜中\footnote{指秋分,白昼与黑夜相等。},星虚\footnote{虚宿,二十八宿中北方七宿之一。},以正中秋。其民夷易\footnote{平和、快乐的样子,言其为秋收而喜悦也。},鸟兽毛毨\footnote{xi\v{a}n,(鸟兽新生的毛)齐整。}。申命和叔,居北方,曰幽都\footnote{北方的阴气聚集之地。旧称日没于此,万象阴暗,故称此。}。便在\footnote{意同“便程”。}伏\footnote{储藏。}物。日短\footnote{指冬至。这天白昼最短。},星昴\footnote{星名,是二十八宿中的北方七宿之一。昴m\v{a}o宿,二十八宿中西方七宿之一。},以正中冬。其民燠\footnote{y\`u,热,暖。此指保暖之衣,或曰保暖之室。},鸟兽氄\footnote{r\v{o}ng,(毛)细而软。同“茸”。}毛。岁三百六十六日,以闰月正四时\footnote{调正四季。}。信\footnote{同“申”,申明条例、申明纪律。诚,切实。}饬\footnote{ch\`i,约束,整顿。}百官,众功\footnote{事业。}皆兴。
\end{yuanwen}

帝尧任命羲氏、和氏,让他遵循上天的法则,观察日月星辰运行的规律,制定历法,教导百姓按照时令从事生产。

他命令羲仲,居住在郁夷,又叫旸谷。他恭敬地迎接太阳升起,有步骤地测定日出的准确时刻。白天和夜晚等长那天是春分,朱雀出现在正南方时,确定为仲春。这个时候,百姓分散到田野中耕种,鸟兽交尾生育。帝尧又命令羲叔,居住在南交,有步骤地观察太阳向南运行的规律,恭敬地等待太阳的回归。白天最长的那天是夏至,心宿出现在正南方时,确定为仲夏。这个时候,百姓都搬到高处,鸟兽的毛还不够丰满。帝尧又命令和仲,居住在西方,名叫昧谷。他恭敬地送别太阳,有步骤地观测日落的情况。夜晚和白天等长的那天是秋分,虚宿出现在正南方时,确定为仲秋。这个时候,百姓迁往平原,鸟兽刚长出整齐的毛。帝尧命令和叔,居住在北方,名叫幽都,有步骤地安排储藏的事情。白天最短的那天是冬至,昴宿出现在正南方时,确定为仲冬。这个时候,百姓进入温暖的室内,鸟兽的毛变得浓密。一年中有三百六十六天,用设置闰月的方法调整四季。帝尧告诫百官,各种事务都顺利进行。

\begin{yuanwen}
尧曰:“谁可顺\footnote{循,继承。}此事?”

放齐曰:“嗣子丹朱开明。”

尧曰:“吁\footnote{xū}!顽凶\footnote{既愚顽又凶狠。或曰“凶”同“讼”,争讼。},不用。”

尧又曰:“谁可者?”

讙\footnote{hu\=an}兜\footnote{尧的大臣,为后文所称的“四凶”之一。}曰:“共工\footnote{尧的大臣,水官,为后文所称的“四凶”之一,与“怒触不周山,天柱折,地维绝”的共工非一人。}旁\footnote{同“普”。}聚布功,可用。”

尧曰:“共工善言,其用\footnote{行事。}僻\footnote{邪恶。},似恭漫天,不可。”

尧又曰:“嗟,四岳\footnote{四方的诸侯之长。},汤汤\footnote{sh\=ang 。水势浩大的样子。}洪水滔天,浩浩怀\footnote{包围。}山襄\footnote{上,意即淹没。}陵,下民其忧,有能使治者?”

皆曰鲧\footnote{gǔn,尧臣,禹的父亲。}可。

尧曰:“鲧负\footnote{背,违。}命毁族\footnote{类,同伙。},不可。”

岳曰:“异哉,试不可用而已。”

尧于是听岳用鲧。九岁,功用\footnote{因}不成。
\end{yuanwen}

尧说:“有谁能够治理好国家的大事?”

放齐说:“您的儿子丹朱通达事理。”

尧说:“唉!丹朱凶顽,不能用他。”

尧又说:“还有谁能够继承我的帝位呢?”

讙兜说:“共工能够广泛地聚集民众,很有功效,可以让他来继承帝位。”

尧说:“共工说话很好听,但是做起事情却违背正道,表面上恭敬,实际上罪恶滔天,不能用他。”

尧又说:“啊!四岳,波涛滚滚的洪水漫天而来,浩浩荡荡地围绕群山,冲上丘陵,百姓都非常担心,有谁能整治洪水呢?”

众人都说鲧可以胜任。

尧说:“鲧违反命令,伤害族人,不能用他。”

四岳说:“不会吧!可以先让他试试,如果不行再罢免他。”

于是尧听从了四岳的建议,任用鲧治水。鲧治水九年,没有成功。

\begin{yuanwen}
尧曰:“嗟!四岳,朕在位七十载,汝能庸命\footnote{指顺应天命。庸,同“用”。},践朕位?”

岳应曰:“鄙德\footnote{犹言“德鄙”,品德不高。鄙,粗野。}忝\footnote{ti\v{a}n,辱,辱没。不配。}帝位。”

尧曰:“悉举贵戚及疏远隐匿者。”

众皆言于尧曰:“有矜\footnote{gu\=an,同“鳏”,老而无妻。}在民间,曰虞舜。”

尧曰:“然,朕闻之。其何如?”

岳曰:“盲者子。父顽\footnote{心不则德义之经为顽。},母嚚\footnote{y\'in,奸诈。口不道忠信之言为嚚。愚昧。},弟傲\footnote{舜之弟名“象”,为人狂傲。},能和以孝,烝烝\footnote{zh\=eng,温厚善良的样子。}治\footnote{劝导使其自治。},不至奸\footnote{干,抵触,冒犯。}。”
\end{yuanwen}

尧说:“啊!四岳,我在位已经七十年了,你们哪一位能够顺应天命,接替我的职位呢?”四岳回答说:“我们的德行低微,不配登上帝位。”尧说:“你们尽管举荐贵族宗亲或旁支隐居的人。”众人都对尧说:“在民间有一个还没娶妻的人,他叫虞舜。”尧说:“是的,这个人我听说过。他是个怎样的人呢?”四岳说:“他是盲人的儿子。他的父亲凶顽,母亲愚昧,弟弟狂傲,他却能用孝行使家庭和睦,用美德治理家业,让家人远离邪恶。”

\begin{yuanwen}
尧曰:“吾其试哉。”

于是尧妻之二女\footnote{娥皇、女英。},观其德于二女。舜饬\footnote{训教,告诫。}下二女于妫汭\footnote{gu\=i ru\`i,妫水入黄河的河口,舜的老家之所在,在今山西永济境内。汭:河流拐弯处。},如妇礼。尧善之,乃使舜慎和\footnote{谨慎地制订并付诸实行。}五典\footnote{也称“五常”,指“父子有亲,君臣有义,夫妇有别,长幼有序,朋友有信。”},五典能从。乃遍入百官,百官时\footnote{是,因此。}序。宾\footnote{用如动词,迎宾,礼宾。}于四门,四门穆穆\footnote{喜悦,心服的样子。},诸侯远方宾客皆敬。尧使舜入山林川泽,暴风雷雨,舜行不迷。尧以为圣,召舜曰:“女\footnote{通“汝”,你。}谋事至而言可绩\footnote{考查。},三年矣。女登帝位。”

舜让于德\footnote{推让说自己的德行不够。}不怿\footnote{y\`i,不乐,因感力不胜任。}。正月上日\footnote{正月初一。上日,朔日。或曰“上日”谓上旬吉日。},舜受终\footnote{本意应该是指“接受禅让”,但这里实际是指接受“摄政”之权。}于文祖\footnote{此指文祖之庙。}。文祖者,尧大祖\footnote{即太祖。大同“太”。}也。
\end{yuanwen}

尧说:“我试试吧!”

于是尧把自己的两个女儿嫁给了舜,通过舜对待妻子的态度来观察他的德行。舜在妫水弯曲处迎娶尧的两个女儿,让她们遵守为人妇的礼节。尧对舜的做法非常满意,就让舜谨慎地完善各种伦理教化,人们都能遵守教化。于是舜用这些教化来整顿百官,百官也能遵守法纪。舜在四门接待宾客,四门一派庄严肃穆的景象,四方诸侯和远方宾客都非常恭敬。尧派舜进入山林川泽,当暴风雷雨来临的时候,舜仍然前进而不会迷失方向。尧认为舜很圣明,召见舜说:“你考虑问题周密,说过的话都能办到,已经三年了。现在请你登上帝位。”

舜认为自己的德行还不能胜任,并且感到不安。正月初一日,舜在文祖庙接受禅让。文祖,就是尧的太祖。

\begin{yuanwen}
于是帝尧老\footnote{即今之所谓“退位”。},命舜摄行天子之政,以观天命。
\end{yuanwen}

从此帝尧退位,让舜代行天子之政,以此观察上天的反应。
这时帝尧年事已高,命令舜代替他行使天子的职权,来观察他是否顺应天命。

\begin{yuanwen}
舜乃在璇玑玉衡\footnote{指北斗七星。},以齐七政\footnote{整顿七种政务。七政,指下文中的祭祀、班瑞、东巡、南巡、西巡、北巡、归至祖庙七项政事。}。遂类\footnote{通“禷”,一种祭天礼。}于上帝,禋\footnote{y\=in,一种祭天礼,烧柴升烟,向天祈福。}于六宗\footnote{指天地和四季。},望于山川,辩\footnote{通“遍”,普遍地祭祀。}于群神。揖\footnote{通“辑”,聚敛。}五瑞,择吉月日,见四岳诸牧,班\footnote{同“颁”,分赐,颁发。}瑞。岁二月,东巡狩,至于岱宗,祡\footnote{同“柴”,指烧柴祭天。},望秩于山川。遂见东方君长,合时月正日,同律度量衡,修五礼\footnote{指五等爵位的朝聘礼仪。}、五玉\footnote{即前文中的“五瑞”。}、三帛、二生\footnote{活的羊羔和大雁。}、一死\footnote{死的野雉。}为挚\footnote{通“贽”,礼物。},如五器,卒乃复。五月,南巡狩;八月,西巡狩;十一月,北巡狩:皆如初。归,至于祖祢\footnote{mí}庙,用特牛礼。五岁一巡狩,群后四朝。遍告以言,明试以功,车服以庸。肇\footnote{zhào}十有二州,决川。象以典刑,流宥\footnote{yòu}五刑,鞭作官刑,扑作教刑,金作赎刑。眚\footnote{shěng}灾过赦;怙终贼刑。钦哉,钦哉,惟刑之静哉!
\end{yuanwen}

于是舜根据北斗七星的运行规律,来整顿七项政事。舜就祭祀天帝,礼敬六宗,望祭名山大川,遍祭天下诸神。舜收聚五种瑞玉,选择吉利的日子,会见四岳和诸侯,并向他们颁赐瑞玉。这年二月,舜到东方去巡视,到达泰山,在那里烧柴祭天,望祭名山大川。于是他会见东方各国的君长,调和校正历法,共同制定音律、长度、容积、重量的标准,颁行五等爵位的朝聘礼仪、五种瑞玉的形制、三种彩色丝帛、二种活牲、一种死禽作为礼物。那五种玉器,在礼毕后全部还给诸侯。五月,舜到南方去巡视;八月,他又到西方去巡视;十一月,到北方去巡视,都遵照最初的礼仪。回来以后,他就到宗庙,用一头牛做祭品。以后他每隔五年到各地巡视一次,各地君长每隔四年前来朝见一次。舜向天下各地宣明政令,根据政绩公开进行考核,把车马服饰赏赐给他们。舜设立了十二州,疏导了河流。他在器物上画着五种刑罚的图像,用流放代替五刑,以鞭打为做官不治的刑罚,以杖笞为不遵教化的刑罚,以罚金为赎罪的刑罚。犯了错误可以赦免,姑息养奸就要给予严惩。小心啊,小心啊,在使用刑罚时,可要十分慎重啊!

\begin{yuanwen}
讙兜进言共工,尧曰:“不可。”而试之工师\footnote{主管土木建筑的官员。},共工果淫辟\footnote{骄纵,邪恶。}。四岳举鲧治鸿水,尧以为不可,岳强请试之,试之而无功,故百姓不便。三苗\footnote{古代的少数民族名,生活在今湖南一带,其种不一,故称“三苗”。}在江淮、荆州数为乱。于是舜归而言于帝,请流\footnote{迁,发配。}共工于幽陵\footnote{北部边裔的都城,约当今之北京密云。},以变北狄\footnote{使其逐渐同化北方的少数民族,也就是起一种抵御北方民族入侵的作用。};放讙兜于崇山\footnote{具体方位不祥,约当今之越南北部一带。},以变南蛮\footnote{泛指南方的少数民族。};迁三苗于三危\footnote{山名,在今甘肃敦煌东南。},以变西戎\footnote{泛指西部的少数民族。};殛\footnote{j\'i,诛。这里是“流放远方”的意思。}鲧于羽山\footnote{东部边地的山名,约在今山东临沂一带。},以变东夷\footnote{泛指东部地区的少数民族。}:四罪\footnote{被治罪。}而天下咸服。
\end{yuanwen}

讙兜推荐共工,尧说:“不能用他。”可是试着让他做工师,结果共工果然放纵作恶。四岳推荐鲧治理洪水,尧也认为他不可用,四岳坚持请求试用鲧,但是试用以后没有收到成效,所以百姓还是生活艰难。三苗部族在江淮、荆州一带经常作乱。这时舜回来向帝尧报告,请求将共工流放到幽陵,使其与北狄融合;将讙兜驱赶到崇山,使其与南蛮融合;将三苗迁徙到三危,使其与西戎融合;将鲧放逐到羽山,使其与东夷融合:四凶服罪使天下人都臣服于舜。

\begin{yuanwen}
尧立七十年得舜,二十年而老,令舜摄行天子之政,荐之于天。尧辟位\footnote{避位,退位。辟,同“避”。}凡二十八年而崩\footnote{据文意,是舜“摄政”二十八年,尧始崩。此与后文所述不同,详见后。}。百姓悲哀,如丧父母。三年,四方莫举乐,以思尧。尧知子丹朱之不肖\footnote{不贤,不成才。},不足授天下,于是乃权授舜\footnote{此以封建社会的制度推测远古。权,变通。}。授舜,则天下得其利而丹朱病;授丹朱,则天下病而丹朱得其利。

尧曰:“终不以天下之病而利一人。”,而卒授舜以天下。\footnote{“授舜”至“终不以天下”六句不见于古书,乃史公所自增,可见其社会理想。}

尧崩,三年之丧毕,舜让辟\footnote{动词连用,让位于人而己回避之。}丹朱于南河之南。诸侯朝觐\footnote{指诸侯进京朝见天子。春见曰“朝”,秋见曰“觐j\`in”}者不之丹朱而之舜,狱讼者不之丹朱而之舜,讴歌者不讴歌丹朱而讴歌舜。

舜曰:“天也。”夫而后之中国\footnote{由“南河之南”进入京城。中国,一国之中心,即首都。}践天子位焉,是为帝舜。
\end{yuanwen}

尧在位七十年才得到舜,二十年后年老退位,让舜代替自己行使天子的职权,向上天举荐他。尧让位二十八年后去世了。百姓都非常难过,就像失去了父母一样。尧死后的三年内,各地都不曾演奏音乐,以此表达哀思。尧知道儿子丹朱不贤能,无法承担治理天下的重任,于是先把帝位传给了舜。尧将帝位传给舜,就会使天下人获利,而让丹朱一人痛苦;将帝位传给丹朱,就会使天下人受苦,而只让丹朱一人获利。

尧说“终究不能让天下人受苦而使一人获利”,而终于将天下传给了舜。

帝尧去世,在其三年丧期结束后,舜将帝位让给丹朱后躲避到南河以南。结果诸侯都不去朝见丹朱而来朝见舜,诉讼的人也不去找丹朱而来找舜,歌颂的人不歌颂丹朱而歌颂舜。

舜说“这是天命”,然后到中原正式登上天子之位,这就是帝舜。

\begin{yuanwen}
虞舜者,名曰重华。重华父曰瞽\footnote{gǔ}叟,瞽叟父曰桥牛,桥牛父曰句望,句望父曰敬康,敬康父曰穷蝉,穷蝉父曰帝颛顼,颛顼父曰昌意:以至舜七世矣。自从穷蝉以至帝舜,皆微为庶人。

舜父瞽叟盲,而舜母死,瞽叟更娶妻而生象,象傲。瞽叟爱后妻子,常欲杀舜,舜避逃;及有小过,则受罪。舜事父及后母与弟,日以笃谨,匪\footnote{没有,不。}有解\footnote{同“懈”,怠慢。}。
\end{yuanwen}

虞舜名叫重华。重华的父亲是瞽叟,瞽叟的父亲是桥牛,而桥牛的父亲叫句望,句望的父亲是敬康,敬康的父亲是穷蝉,穷蝉的父亲是颛顼,颛顼的父亲是昌意:到舜已经有七代了。从穷蝉到帝舜,都是卑微的平民。

舜的父亲瞽叟是个盲人,而舜的母亲去世了,瞽叟续娶妻子生下象,象为人狂傲。瞽叟宠爱后妻的儿子,经常想要杀害舜,但是舜都逃脱了;舜一旦犯了小的过失,就会遭到父亲的惩罚。他却仍然恭顺地对待父亲以及后母和弟弟,他每天对家人都很真诚,没有松懈怠慢的时候。

\begin{yuanwen}
舜,冀州之人也。舜耕历山,渔雷泽,陶河滨,作什器\footnote{各种生活、劳动用品。}于寿丘,就时\footnote{犹逐时,乘时射利,即经商。}于负夏。舜父瞽叟顽,母嚚,弟象傲,皆欲杀舜。舜顺适\footnote{顺从。}不失子道,兄弟孝慈。欲杀,不可得;即求,尝\footnote{同“常”。}在侧。
\end{yuanwen}

舜是冀州人。他在历山耕种,在雷泽捕鱼,在黄河边烧制陶器,在寿丘制作各种日用器物,在负夏经商。舜的父亲瞽叟凶顽,母亲愚昧,弟弟象狂傲,他们都想将舜杀死。舜却恭顺地对待他们而不失为人子之道,孝顺父母,友爱兄弟。父母兄弟虽然想杀他,却始终不能得逞;他们有求于舜,舜又会经常陪伴左右。

\begin{yuanwen}
舜年二十以孝闻。三十而帝尧问可用者,四岳咸荐虞舜,曰可。于是尧乃以二女妻舜以观其内,使九男与处以观其外。舜居妫汭,内行\footnote{在家族以内的行为表现与其处理事务的能力。}弥谨。尧二女不敢以贵骄事舜亲戚\footnote{这里指公婆。},甚有妇道。尧九男皆益笃。舜耕历山,历山之人皆让畔\footnote{田界。};渔雷泽,雷泽上\footnote{应作“之”。}人皆让居;陶河滨,河滨器皆不苦窳\footnote{yǔ,粗劣,易坏。}。一年而所居成聚\footnote{村落。},二年成邑\footnote{市镇。},三年成都\footnote{都城。}。尧乃赐舜絺衣\footnote{ch\=i,细葛布做的衣裳,在当时很贵重。},与琴,为筑仓\footnote{粮仓。}廪\footnote{上有篷顶的粮仓。},予牛羊。瞽叟尚复欲杀之,使舜上涂廪\footnote{用泥抹粮仓上的屋顶。},瞽叟从下纵火焚廪。舜乃以两笠自扞\footnote{hàn,同“捍”,防护。}而下,去,得不死。后瞽叟又使舜穿井,舜穿井为匿\footnote{藏,不使人知。}空\footnote{孔。}旁出\footnote{从旁边通向地面。}。舜既入深,瞽叟与象共下土实井,舜从匿空出,去。瞽叟、象喜,以舜为已死。

象曰:“本谋\footnote{主谋。}者象。”

象与其父母分。于是曰:“舜妻尧二女,与琴,象取之。牛羊仓廪予父母。”

象乃止\footnote{这里指住。}舜宫\footnote{屋舍。}居,鼓其琴。舜往见之。象鄂\footnote{通“愕”,吃惊。}不怿\footnote{y\`i。这里是尴尬的样子。},曰:“我思舜正郁陶\footnote{伤心痛苦的样子。}!”

舜曰:“然,尔其庶\footnote{可以,够味。}矣!”

舜复事瞽叟爱弟弥谨。于是尧乃试舜五典百官,皆治。
\end{yuanwen}

舜二十岁的时候凭借孝德而天下闻名。三十岁的时候,帝尧询问可以重用的人,四岳都推荐虞舜,说他可用。于是帝尧将两个女儿嫁给舜来观察他治理家庭的情况,派九个儿子和舜相处来观察他待人接物的情况。舜居住在妫水弯曲处,他在家中表现得非常恭谨。尧的两个女儿也不敢凭借高贵的出身而以傲慢的态度对待舜的亲戚,谨守为人妇之道。尧的九个儿子也更加纯朴厚道了。舜在历山种田的时候,历山的百姓都相互谦让田界;舜在雷泽捕鱼的时候,雷泽的百姓都相互谦让位子;舜在黄河边烧制陶器的时候,黄河边出产的陶器没有一件是粗制滥造的。舜所居住的地方一年就会形成村落,两年就会变为城邑,三年就会发展成都市。于是尧赏赐给舜细葛布做成的衣服,还赠给他琴,为他建造粮仓,送给他牛羊。舜的父亲瞽叟还想谋害舜,就让舜到粮仓顶部涂泥,然后瞽叟就从下面放火烧粮仓。舜就用两个斗笠保护自己从粮仓上面跳下来,逃走了,得以保住性命。瞽叟又让舜去挖井,舜在挖井的时候挖了一个通向外面的隐蔽通道。当舜挖到深处时,瞽叟和象一起用土把井口填实,舜就从隐蔽的通道逃出来,逃走了。瞽叟、象很高兴,他们以为舜已经死了。

象说:“这个主意本来是我想到的。”

象和父母分舜的财产,于是说:“舜的妻子,也就是尧的两个女儿,和琴,都归我了。牛羊和粮仓留给父母。”

于是象跑到舜的居室住了下来,还弹着舜的琴。舜回到自己的住处看到象,象大惊失色,说:“我思念你正感到伤心难过呢!”

舜说:“是的,你对我差不多就是这样了!”

舜仍然侍奉瞽叟、友爱兄弟,而且更加勤谨了。于是尧试着让舜负责推行五种教化和管理百官,舜都做得很好。

\begin{yuanwen}
昔高阳氏有才子\footnote{成材的人。}八人,世得其利,谓之“八恺”。高辛氏有才子八人,世谓之“八元”。此十六族者,世济\footnote{达到,成就。}其美,不陨\footnote{落。}其名。至于尧,尧未能举。舜举八恺,使主后土\footnote{即指土,大地。},以揆\footnote{ku\'i,观察,忖度。这里指治理。}百事,莫不时序\footnote{顺承。}。举八元,使布五教\footnote{即前所谓“五常”。}于四方,父义,母慈,兄友,弟恭,子孝,内\footnote{诸夏。}平外\footnote{夷狄。}成。
\end{yuanwen}

从前高阳氏有八个德才兼备的儿子,世人都得到他们的恩惠,称他们为“八恺”。高辛氏也有八个德才兼备的儿子,世人称他们为“八元”。这十六个家族的人,世代保持着祖先的美好品德,从来没有做过毁损祖先声誉的事情。到了尧的时候,尧没有能够任用他们。舜举荐八恺,让他们主管农业,负责各项政务,没有一件事不合时令。舜又举荐八元,让他们到四方推行五种教化,于是父亲恩义,母亲慈爱,哥哥友善,弟弟恭敬,儿子孝顺,家庭和睦,邻里真诚。

\begin{yuanwen}
昔帝鸿氏\footnote{指黄帝之族。}有不才子,掩义隐贼\footnote{掩蔽仁义,包庇奸贼。“掩”亦可训为“袭击”。},好行凶慝\footnote{凶邪。t\`e},天下谓之“浑沌\footnote{即讙兜。}”。少皞氏\footnote{也作“少昊”。h\`ao}有不才子,毁信恶忠,崇饰恶言,天下谓之“穷奇”。颛顼氏有不才子,不可教训,不知话言\footnote{谓善言。},天下谓之“梼杌\footnote{táo wù}”。此三族世忧之。至于尧,尧未能去。缙云氏\footnote{姜姓,炎帝之苗裔。j\`in}有不才子,贪于饮食,冒\footnote{没,其他皆所不顾。}于货贿,天下谓之“饕餮\footnote{tāo tiè}”。天下恶之,比之三凶。舜宾于四门,乃流四凶族,迁于四裔,以御螭魅\footnote{chī},于是四门辟\footnote{四门大开,言其太平无事之状。},言毋\footnote{通“无”。}凶人也。
\end{yuanwen}

从前帝鸿氏有一个不成器的儿子,不施仁义,阴险狠毒,喜欢行凶作恶,天下人都称他为“浑沌”。少皞氏也有一个不成器的儿子,诋毁诚实的人,厌恶忠直的人,粉饰邪恶的言论,天下人都称他为“穷奇”。颛顼氏也有一个不成器的儿子,他不接受教诲,不听取好话,天下人都称他为“梼杌”。这三个家族的人,世代令人感到忧虑。到了尧的时候,尧也没能除掉他们。缙云氏有一个不成器的儿子,贪恋饮食,贪求财物,天下人都称他为“饕餮”。天下人都非常憎恶他,把他和上述三个凶恶的人相提并论。舜在四门接待宾客的时候,就将这四个凶恶的家族流放了,把他们迁徙到最偏远的四个地方,让他们去抵御妖魔鬼怪。于是四门通畅,人们都说没有凶恶的人了。

\begin{yuanwen}
舜入于大麓\footnote{山脚。这里指深山。},烈风雷雨不迷,尧乃知舜之足授天下。尧老,使舜摄行天子政,巡狩。舜得举用事二十年,而尧使摄政。摄政八年而尧崩\footnote{据前文,是舜居首辅二十年后,乃摄政;摄政二十八年后,尧始崩。而今又谓“摄政八年而尧崩”,前后不一。}。三年丧毕,让丹朱,天下归舜。而禹\footnote{鯀之子,因治水有功,受舜禅让为帝。}、皋陶\footnote{g\=ao y\'ao,舜时掌刑狱的大臣。}、契\footnote{舜时掌教化的官,商朝的祖先。}、后稷\footnote{名弃,舜时掌管农事的官,周朝的的祖先。j\`i}、伯夷\footnote{舜时掌礼的官,与周初之饿死首阳山者同名。}、夔\footnote{ku\'i,舜时主乐的官。}、龙\footnote{舜时的谏官。}、倕\footnote{chuí,舜时主管建筑的官。}、益\footnote{也称“伯益”、“伯翳”、“大业”,秦国的祖先。}、彭祖,自尧时而皆举用,未有分职。于是舜乃至于文祖,谋于四岳,辟四门,明通四方耳目。命十二牧\footnote{十二州的州长。}论帝德\footnote{弘扬帝尧之德。论,阐发,光大。},行厚德,远佞人\footnote{n\`ing,以甜言蜜语取悦于人者。},则蛮夷率服\footnote{相率来归顺。}。
\end{yuanwen}

舜来到高山密林中,在烈风雷雨里也没有迷失方向,尧这时知道舜是能够托付天下的人。尧年老的时候,让舜代替他行使天子的职权,巡视天下。舜被举用当政二十年后,尧让舜代行天子职权。舜代行天子职权八年后,尧去世了。三年丧期结束后,舜把帝位让给尧的儿子丹朱,天下人却都归附舜。而禹、皋陶、契、后稷、伯夷、夔、龙、倕、益、彭祖,在尧的时候就被任用,只是没有得到相应的封邑和职务。于是舜来到文祖庙,和四岳商议,打开四门,让四方言路畅通,任命十二州的长官评论天子的德行,广施恩德,远离小人,偏远地区的部族就能够臣服。

梁启超:“带有神话性的(人物),纵然伟大,不应作传。譬如黄帝很伟大,但不见得真有其人。太史公作《五帝本纪》,亦作得恍惚迷离。不过说他(生而神明,弱而能言,幼而徇齐,长而敦敏,成而聪明)这些话,很像词章家的点缀堆砌,一点不踏实,其余的传说,资料尽管丰富,但绝对靠不住。纵然不抹杀,亦应怀疑。”

\begin{yuanwen}
舜谓四岳曰:“有能奋庸美尧之事者,使居官相事?”

皆曰:“伯禹为司空,可美帝功。”

舜曰:“嗟,然!禹,汝平水土,维是勉哉。”

禹拜稽首,让于稷、契与皋陶。

舜曰:“然,往矣。”

舜曰:“弃,黎民始饥,汝后稷播时百谷。”

舜曰:“契,百姓不亲,五品不驯,汝为司徒,而敬敷五教,在宽\footnote{宽厚。一说宽即缓,意思是要慢慢地进行。}。”

舜曰:“皋陶,蛮夷猾夏,寇贼奸轨\footnote{通“宄”,在外作恶。},汝作士,五刑有服,五服三就\footnote{指行刑之处。};五流有度,五度三居\footnote{指流放之处。}:维明能信。”

舜曰:“谁能驯予工?”

皆曰垂\footnote{即上文中的“倕”。}可。于是以垂为共工\footnote{官名,非上文中流于幽陵者。}。

舜曰:“谁能驯予上下草木鸟兽?”

皆曰益可。

于是以益为朕虞。益拜稽首,让于诸臣朱虎、熊罴。

舜曰:“往矣,汝谐。”遂以朱虎、熊罴为佐。

舜曰:“嗟!四岳,有能典朕三礼?”

皆曰伯夷可。

舜曰:“嗟!伯夷,以汝为秩宗,夙夜维敬,直哉维静絜。”

伯夷让夔、龙。

舜曰:“然。以夔为典乐,教稚子,直而温,宽而栗\footnote{让人敬畏。},刚而毋虐,简而毋傲;诗言意,歌长言,声依永,律和声,八音能谐,毋相夺伦,神人以和。”

夔曰:“於!予击石拊石,百兽率舞。”

舜曰:“龙,朕畏忌谗说殄伪\footnote{灭绝道德的行为。殄,tiǎn。伪,通“为”。},振惊朕众,命汝为纳言,夙夜出入朕命,惟信。”

舜曰:“嗟!女二十有二人,敬哉,惟时相天事。”

三岁一考功,三考绌陟\footnote{贬降或升迁。绌,通“黜”。陟,zhì。},远近众功咸兴。分北\footnote{分离,分解。“北”,同“背”。}三苗。
\end{yuanwen}

舜对四岳说:“有谁能够奋发努力将帝尧的事业发扬光大,让我授予官职处理政事吗?”

众人都说:“让伯禹担任司空,就可以将帝尧的事业发扬光大。”

舜说:“啊!是的。禹,你去治理水土,一定要努力去做啊!”

禹跪拜叩头,让位给稷、契和皋陶。

舜说:“好了,去吧。”

舜又说:“弃,民众开始忍受饥饿,由你负责农业,教百姓种植各种谷物。”

舜说:“契,百姓不能和睦相处,五种伦理教化不能顺利推行,你来担任司徒,去恭敬地推行五种教化,为政要宽厚。”

舜说:“皋陶,偏远地区的部族骚扰中原,残暴的贼寇经常作乱,你来担任士,据犯人的罪行使用五种刑罚,按照罪行的轻重在三个地方行刑。流放也要根据罪行的轻重分为五种,流放地点分为三个等级:公正严明才能令人信服。”

舜说:“谁能帮助我管理好百工事务?”

众人都说垂能胜任。于是舜任命垂为共工。

舜说:“谁能管理好各地的草木鸟兽?”

众人都说益能胜任。于是舜任命益为虞官。

益跪拜叩头,让位给大臣朱虎、熊罴。

舜说:“去吧,你能做好。”于是舜让朱虎、熊罴做益的助手。

舜说:“啊!四岳,有谁能为我主持三大祭礼?”

众人都说伯夷能胜任。

舜说:“啊!伯夷,我任命你为秩宗,从早到晚虔诚恭谨,要正直而清明啊!”

伯夷让位给夔、龙。

舜说:“好。让夔主管音乐,教育少年,要正直又温和,宽宏又谨慎,刚强但不暴虐,简约但不傲慢;诗用来表达思想,歌用来延长诗中语言的情感,音调根据歌唱而决定,韵律应和音调。八种乐器的声音协调一致,不相互失去次序,神和人都可以安宁和乐。”

夔说:“啊!我有节奏地敲打石磬,各类鸟兽都跟着起舞。”

舜说:“龙,我非常忌讳谗言与暴行,惊扰我的臣民,我任命你为纳言,从早到晚宣读我的旨意,一定要诚实守信。”

舜说:“啊!你们二十二个人,一定要恭谨啊!时刻想着接受上天的命令并帮助上天治理臣民。”

从此舜每隔三年就考核一次政绩,通过三次考核来决定罢免或升迁,因此远近的各项事务都发展起来。舜还分化瓦解了三苗部族。

\begin{yuanwen}
此二十二人咸成厥功:皋陶为大理\footnote{官名,全国最高的司法官。},平,民各伏\footnote{通“服”,谓被定罪者皆内心服气。}得其实;伯夷主礼,上下咸让;倕主工师,百工致功;益主虞,山泽辟;弃主稷,百谷时茂;契主司徒,百姓亲和;龙主宾客\footnote{龙为“纳言”,求见舜者必须首先通过龙,故曰龙主宾客”。},远人至;十二牧行而九州\footnote{华夏原称“九州”,其长宫也只有“九牧”;后又增三州为“十二州”,故其长官也就成了“十二牧”。此处“十二”与九"错落使用。}莫敢辟\footnote{邪悉。}违\footnote{抗命。};唯禹之功为大,披\footnote{通“劈”。}九山\footnote{极言为泄导洪水所开凿的山岭之多。},通九泽,决\footnote{疏通。}九河,定九州,各以其职\footnote{责任。也就是按照本州地形与物产应向朝延进献的贡品。}来贡,不失厥宜。方五千里,至于荒服\footnote{此指当时整个华夏的疆域。古称自天子王畿向四周辐射,五百里甸服,五百里侯服,五百里绥服,五百里要服,五百里荒服”。按直径计算,即五千里。}。南抚交阯\footnote{也作“交趾”,其首府即今越南河内市。}、北发\footnote{即“北向户”指广东、广西南部之北回归线以南,窗户向北开的地方。门户朝北开,指南方极远之地。},西戎、析枝、渠廋\footnote{sōu}、氐、羌\footnote{“西”下省“抚”字。戎、析枝、渠度、氐、羌,都是西部的少数民族名,大约生活在今陕西西部、四川西北部与甘肃、青海一带地区。},北山戎、发、息慎\footnote{text},东长、鸟夷\footnote{乌夷也作“岛夷”。这些指当时东部大海中的岛国名。}\footnote{“北”下亦省“抚”字。山戎、发、息慎,都是当时东北地区的少数民族名。},四海之内咸\footnote{全,都。}戴\footnote{拥戴,拥护。}帝舜之功。于是禹乃兴《九招》之乐,致异物,凤皇来翔。天下明德\footnote{兼指崇高的道德与圣明的政治。}皆自虞帝始\footnote{“于是”句:“禹”字疑当作“夔”,叙禹于诸臣之后者,以禹功最大。而太乐之作,所以告成功,故又叙夔于再之后。《九招》,同“《九韶》”,相传为舜时所作的古乐名。}。
\end{yuanwen}

这二十二个人都有出色的政绩:皋陶担任司法官,断案公平,百姓都非常信服他实事求是;伯夷掌管礼仪,尊卑贵贱都谦恭礼让;垂管理工匠,百工都能做好工作;益担任虞官,山林湖泽都得到开发利用;弃管理农业,各种谷物都生长茂盛;契担任司徒,百姓亲密和睦;龙负责接待宾客,远方的部族前来归附;十二个长官外出巡视,九州的百姓没有敢躲避和违抗的;只有禹的功劳最大,他开辟了九座大山,疏通了九个湖泊,治理了九条江河,划定了九州疆界,各州都按照规定前来进贡,没有违反规定的。疆域有五千里,一直伸延到荒服。他在南方安抚了交阯、门户朝北开的地区,在西方安抚了戎、析枝、渠廋、氐、羌,在北方安抚了山戎、发、息慎,在东方安抚了长夷、鸟夷,四海之内都歌颂帝舜的功德。于是禹创作《九招》乐曲,致使灵异之物降临,凤凰飞翔到此。天下的圣明之德都是从虞帝时代开始的。

\begin{yuanwen}
舜年二十以孝闻,年三十尧举之,年五十摄行天子事,年五十八尧崩\footnote{此与舜纪前文“弹得举用事二十年,而尧使摄政,摄政八年面尧崩”的说法相同;与尧纪所谓“尧立七十年得舜,二十年而老,令舜摄行天子之政,荐之于天,尧辟位凡二十八年而崩”的说法不同。},年六十一代尧践帝位。践帝位三十九年,南巡狩,崩于苍梧\footnote{汉都名,郡治广信、即今广西梧州。}之野。葬于江南九疑\footnote{山名,在今湖南宁远南,因山有九峰皆相似。故称“九疑”。},是为零陵\footnote{汉郡名,郡治在今广西兴安北,九疑山正处于当时苍梧郡与零陵郡的交界处。}。
\end{yuanwen}

舜在二十岁的时候凭借孝行而闻名天下,三十岁的时候被尧任用,五十岁的时候代行天子职事,五十八岁的时候尧去世,六十一岁的时候接替尧登上帝位。舜登上帝位三十九年,到南方巡视,在苍梧的郊野去世。他被人们安葬在长江以南的九嶷山,也就是零陵。

\begin{yuanwen}
舜之践帝位,载天子旗,往朝父瞽叟,夔夔\footnote{和顺恭敬的样子。}唯谨,如子道。封弟象为诸侯。舜子商均亦不肖,舜乃豫\footnote{通“预”,事先。}荐禹于天。十七年而崩。三年丧毕,禹亦乃让舜子,如舜让尧子。诸侯归之,然后禹践天子位。尧子丹朱,舜子商均,皆有疆土,以奉先祀。服其服,礼乐如之。以客见天子,天子弗臣,示不敢专也。
\end{yuanwen}

舜登上帝位以后,他的车子上竖立着天子的旗帜,乘车去朝见他的父亲瞽叟,态度非常谦恭顺从,遵守为人子之道。舜封他的弟弟象为诸侯。舜的儿子商均也是没有才能,舜就在自己去世前向天帝推荐了禹。十七年以后,舜去世了。三年丧期结束时,禹把帝位让给了舜的儿子,就像舜将帝位让给尧的儿子一样。诸侯都归顺禹,然后禹才登上天子之位。尧的儿子丹朱,舜的儿子商均,都有自己的封地,用来供奉他们的祖先。他们穿着祖先的服饰,使用祖先的礼乐。他们用宾客之礼朝见天子,天子不把他们视为臣下,以此表明自己不敢独占天下。

\begin{yuanwen}
自黄帝至舜、禹,皆同姓而异其国号,以章明德。故黄帝为有熊,帝颛顼为高阳,帝喾为高辛,帝尧为陶唐,帝舜为有虞。帝禹为夏后而别氏,姓姒氏\footnote{姒,sì。姓是共同血缘关系的标记,氏则是姓的分支,在宗法制下,同一姓按照血缘亲疏,会分为许多不同的氏。战国时期,宗法制瓦解,秦汉以后,姓氏合一。所以司马迁混淆了姓和氏的概念,《史记》中才多次出现“姓某氏”的说法。}。契为商,姓子氏。弃为周,姓姬氏。
\end{yuanwen}

从黄帝到舜、禹,他们都是同姓,而使用不同的国号,以此显示各自的德行。因此黄帝称有熊,帝颛顼称高阳,帝喾称高辛,帝尧称陶唐,帝舜称有虞。帝禹称夏后却改变了姓氏,姓姒氏。契的后裔建立了商,姓子氏。弃的后裔建立了周,姓姬氏。

\begin{yuanwen}
太史公\footnote{司马迁的自称。司马迁曾任太史令,“太史公曰”以下的文字是他的论赞,为一篇的结语,其内容或为发表议论,或为说明立意,或为补充史实。}曰:学者多称五帝,尚\footnote{久远。}矣。然《尚书》独载尧以来;而百家言黄帝,其文不雅驯,荐绅\footnote{缙绅,指官员。}先生难言之。孔子所传宰予问《五帝德》及《帝系姓》,儒者或不传。余尝西至空桐,北过涿鹿,东渐于海,南浮江淮矣,至长老皆各往往称黄帝、尧、舜之处,风教固殊焉,总之不离古文者近是。予观《春秋》、《国语》,其发明《五帝德》、《帝系姓》章矣,顾弟\footnote{同“第”,但,只。}弗深考,其所表见皆不虚。《书》缺有间矣,其轶乃时时见于他说。非好学深思,心知其意,固难为浅见寡闻道也。余并论次,择其言尤雅者,故著为本纪书首。
\end{yuanwen}

太史公说:学者中有很多人在称颂五帝,可是太久远了。然而《尚书》中只记载了尧之后的事情;而各家讲述黄帝事迹,文字也不够典雅纯正,官员和儒生也很难说清楚。孔子所传授的宰予问《五帝德》和《帝系姓》,有些儒生也不传习。我曾经向西到达空桐山,向北经过涿鹿,向东前往海边,向南渡过长江、淮水,所到之处的老人经常赞颂黄帝、尧、舜,各地的风俗教化又有很大区别,总之和古文的记载不相悖的说法还是比较可信的。我阅读《春秋》、《国语》,其中有些章节阐明了《五帝德》、《帝系姓》,只是我没有进行深入的研究,其中所表述的观点都不是虚妄的。《尚书》有残缺已经很长时间了,其中散佚的部分经常能在别的著作中看到。不是好学深思,心领神会的人,自然难以对见识浅陋的谈论这些问题。我综合各家的著述,选择文辞最为典雅的写成这篇文字,所以置于本纪的首篇。

\part{卷二}

\chapter{夏本纪第二}

本篇记述了夏朝的历史,多取材于《尚书》。早在先秦时期,人们就把夏、商、周三王与五帝相提并论,将其奉为上古圣王。尽管并没有太多的考古资料可以证明夏朝的存在,其相关传说却在中国历史上具有重要的意义,对后世帝王统绪、王朝更替、治国方略等理论均有深远影响。

阎若璩:“《齐乘》以大清为古济水,以小清为刘豫所道,非也。以《水经注》、《元和志》、《寰宇记》考之,济最南,漯在中,河最北。今小清所经,自历城以东,章邱、邹平、高苑、博兴诸县,皆古济所行。而大清所行,自历城以上,至东阿,固皆济水故道。而自历城东北,如济阳、齐东、青城诸县,则皆古漯所行。蒲台以北,则古河所经。盖唐宋时,河行漯川,其后大清兼行河、漯二川;其小清所行,则断为济水故道也。”

\begin{yuanwen}
夏禹,名曰文命。禹之父曰鲧,鲧之父曰帝颛顼,颛顼之父曰昌意,昌意之父曰黄帝。禹者,黄帝之玄孙而帝颛顼之孙也。禹之曾大父昌意及父鲧皆不得在帝位,为人臣。
\end{yuanwen}

夏禹,名叫文命。他的父亲是鲧,鲧的父亲是帝颛顼,颛顼的父亲是昌意,昌意的父亲就是黄帝。禹是黄帝的玄孙,又是颛顼的孙子。在禹的曾祖父昌意和父亲鲧都没能登上帝位,只是做了臣子。

\begin{yuanwen}
当帝尧之时,鸿水滔天,浩浩怀山襄陵,下民其忧。尧求能治水者,群臣四岳皆曰鲧可。

尧曰:“鲧为人负命毁族,不可。”

四岳曰:“等\footnote{比较。}之未有贤于鲧者,愿帝试之。”

于是尧听四岳,用鲧治水。九年而水不息,功用不成。于是帝尧乃求人,更得舜。舜登用,摄行天子之政,巡狩。行视鲧之治水无状,乃殛\footnote{流放远方。}鲧于羽山以死。天下皆以舜之诛为是。于是舜举鲧子禹,而使续鲧之业。
\end{yuanwen}

在尧帝的时候,洪水波涛满天,浩浩荡荡地包围了高山,淹没了丘陵,百姓深受洪水的困扰。尧访求可以治理洪水的人,群臣和四岳都说鲧能够胜任。

尧说:“鲧这个人,违背天命、败坏家族,不能用他。”

四岳说:“相比之下鲧是最贤能的了,希望您试一试。”

于是尧采纳了四岳的建议,任用鲧治理洪水。过了九年洪水也没有退去,鲧没有取得任何成效。于是帝尧才另外寻找人才,又得到了舜。舜被任用,代行天子职事,巡视天下。舜在巡行中发现鲧在治水方面一事无成,于是把他流放到羽山而死在那里。天下人都认为舜惩罚鲧的做法是正确的。这时舜举荐鲧的儿子禹,并且让他继承鲧的事业。

\begin{yuanwen}
尧崩,帝舜问四岳曰:“有能成美尧之事者使居官?”

皆曰:“伯禹为司空,可成美尧之功。”

舜曰:“嗟,然!”

命禹:“女平水土,维是勉之。”

禹拜稽首,让于契、后稷、皋陶。

舜曰:“女其往视尔事矣。”
\end{yuanwen}

尧去世以后,帝舜问四岳说:“有谁能够更好地完成尧的事业并担任官职呢?”

众人都说:“让伯禹担任司空,就可以更好地完成尧的事业。”

帝舜说:“啊!是这样。”

舜任命禹说:“你去平治水土,一定要努力去做。”

禹跪拜叩头,并让位给契、后稷、皋陶等人。

舜说:“你还是去做你的工作吧。”

\begin{yuanwen}
禹为人敏给\footnote{敏捷。}克勤,其德不违,其仁可亲,其言可信:声为律,身为度,称以出\footnote{指人们以禹之声为音律的标准,以禹之身为尺度的标准。};亹亹\footnote{wěi,勤勉的样子。}穆穆\footnote{庄严的样子。},为纲为纪。
\end{yuanwen}

禹这个人,敏捷而勤恳;他品德端正,仁爱平和,说话诚信;他的声音符合音律,他的行为符合法,他所表现出的美德被人们视为规范准则;他勤勉庄严,为人们制定纲纪。

\begin{yuanwen}
禹乃遂与益、后稷奉帝命,命诸侯百姓兴人徒以傅\footnote{《尚书》作“敷”,是划分的意思,指分治九州土地。}土,行山表木\footnote{伐木为路标。},定高山大川。禹伤先人父鲧功之不成受诛,乃劳身焦思,居外十三年,过家门不敢入。薄衣食,致孝于鬼神。卑宫室,致费于沟淢\footnote{xù}。陆行乘车,水行乘船,泥行乘橇,山行乘檋\footnote{jū,底部有钉齿的鞋,一说为类似滑竿的交通工具。}。左准绳,右规矩,载四时,以开九州,通九道,陂九泽,度九山。令益予众庶稻,可种卑湿。命后稷予众庶难得之食。食少,调有余相给,以均诸侯。禹乃行相地宜所有以贡,及山川之便利。
\end{yuanwen}

禹于是和伯益、后稷一同奉行帝舜的命令,派诸侯和百官征发民夫来分别整治天下的水土,根据山的走向砍伐树木制作路标,来测定高山大川。禹为先父鲧治水无功遭受惩罚而感到悲伤,于是辛苦劳作,努力思考,在外十三年,即使路过家门也不敢进去。禹的吃穿非常简朴,但是他向祖先神灵所进献的祭品却很丰厚;禹的住宅十分简陋,但是他对修渠挖沟所耗费的财力却不吝惜。他在平地行走的时候乘坐车驾,在水路行进的时候乘坐船只,在泥地行走的时候使用木橇,在山间攀爬的时候就穿上底部有齿的鞋。禹随身带着测平直的准平与绳墨,以及画图式的圆规与直尺,装载测定时令的仪器,用来开辟九州的土地,打通九州的道路,修筑九州的堤坝,测量九州的山岳。禹命令益给予民众稻种,教导民众在低洼潮湿的地方种植。他又命令后稷给予民众应急充饥的食物。在缺少食物的地方,禹就从富裕的地方调来食物补充不足的地方,使各地的食物储备得以均衡。禹还根据各地的不同情况定下进献贡物的标准,以及高山大川的交通状况。

张尔岐:“《夏本纪》云,禹伤父功之不成受诛,乃劳身焦思,居外十三年,过家门不入云云。写出圣人仁孝、沉痛、精诚,直至地平天成止了,干蛊一事勿论,功能二字不足言,即悲天悯人,犹是圣人安常处顺之事,非所以论禹也。”

\begin{yuanwen}
禹行自冀州始。冀州:既载壶口,治梁及岐。既修太原,至于岳阳。覃怀致功,致于衡、漳。其土白壤\footnote{柔土。}。赋上上错\footnote{指冀州某些地区可缴纳上中等赋税。},田中中。常、卫既从,大陆既为。鸟夷皮服。夹右碣石,入于海\footnote{据《尚书·禹贡》,应为“河”。}。
\end{yuanwen}

禹的巡行从冀州开始。冀州:壶口的工程已经完成,又治理了梁山和岐山。太原修整以后,一直到太岳山以南。覃怀的工程也完成了,一直到衡水、漳水一带。这里的土壤是白色柔土。赋税为上上等,某些地区也可缴纳上中等,田地为中中等,常水、卫水沿着河道入海,大陆泽附近的土地可以用于耕作。鸟夷进贡皮革衣服,从西边靠近碣石山的地方进入黄河。

\begin{yuanwen}
济、河维沇州\footnote{即兖州。}:九河既道\footnote{同“导”,疏通。},雷夏既泽,雍、沮会同,桑土既蚕,于是民得下丘居土。其土黑坟\footnote{沃土。},草繇\footnote{茂盛。}木条\footnote{高大。}。田中下,赋贞\footnote{应为“下下”之讹。},作十有三年乃同。其贡漆、丝,其篚\footnote{fěi,圆形竹筐。}织文。浮于济、漯\footnote{luò},通于河。
\end{yuanwen}

济水、黄河之间是兖州:黄河下游的九条支流都已经被疏导,雷夏蓄水成为湖泽,雍水、沮水会合流入雷夏泽,适合种桑的土地开始养蚕,于是民众下山搬到平地生活。这里的土壤是黑色沃土,生长着茂盛的草木。田地为中下等,赋税为下下等,经过十三年的耕作才能赶上其他地区。这里的贡物是漆和丝,有美丽花纹的丝织品装在圆筐中,进贡时乘船经过济水、漯水,一直通往黄河。

\begin{yuanwen}
海、岱维青州:堣\footnote{yú}夷既略,潍、淄其道。其土白坟,海滨广潟\footnote{xì,盐碱地。},厥田斥卤。田上下,赋中上。厥贡盐絺,海物维错,岱畎\footnote{山谷。}丝、枲\footnote{xǐ,麻。}、铅\footnote{指锡。}、松、怪石,莱夷为牧,其篚酓\footnote{yǎn,山桑树。}丝。浮于汶,通于济。
\end{yuanwen}

大海、泰山之间是青州:堣夷已经得到治理,潍水、淄水得以疏通。这里的土壤是白色沃土,海滨一带有大面积盐碱地,这里的田地多含盐卤。田地为上下等,赋税为中上等。这里的贡物是食盐和细葛布,也进贡海产品,泰山的山谷里出产丝、麻、锡、松木、奇石,莱夷可以放牧,山桑蚕丝装在圆筐中,进贡时在汶水乘船,一直通往济水。

\begin{yuanwen}
海、岱及淮维徐州:淮、沂其治,蒙、羽其蓺。大野既都\footnote{通“潴”,水停聚的地方。},东原厎\footnote{dǐ}平。其土赤埴坟\footnote{粘土。},草木渐包\footnote{通“苞”,草木丛生。}。其田上中,赋中中。贡维土五色,羽畎\footnote{quǎn}夏狄\footnote{通“翟”,长尾野鸡。},峄\footnote{yí}阳孤桐,泗滨浮磬,淮夷蠙珠臮\footnote{同“暨”,及,与。}鱼,其篚玄纤缟\footnote{gǎo}。浮于淮、泗,通于河。
\end{yuanwen}

大海、泰山和淮水之间是徐州:淮水、沂水已经得到治理,蒙山、羽山已经可以耕种。大野蓄水成为湖泽,东原也得以整治。这里的土壤是红色粘土,逐渐草木繁盛。田地为上中等,赋税为中中等。这里的贡物是五色土,羽山的山谷里出产大雉鸡,峄山以南出产孤桐木,泗水之滨出产浮磬,淮夷进贡珍珠和鱼,黑绸和白绢装在圆筐中,在淮水、泗水乘船,一直通往黄河。

\begin{yuanwen}
淮海维扬州:彭蠡既都,阳鸟\footnote{候鸟。}所居。三江既入,震泽致定。竹箭\footnote{泛指各种竹子。}既布。其草惟夭,其木惟乔,其土涂泥。田下下,赋下上上杂。贡金三品\footnote{指金、银、铜三种贵金属。},瑶、琨、竹箭,齿、革、羽、旄,岛夷卉服\footnote{蓑衣。},其篚织贝,其包橘、柚锡贡。均江海,通淮、泗。
\end{yuanwen}

淮水、大海之间是扬州:彭蠡已经蓄水成为湖泽,成为候鸟的栖息之地。三江已经流入大海,震泽也得以整治。这里生长着茂密的竹林。青草繁茂,树木高大,这里的土壤是潮湿的泥土。田地为下下等,赋税为下上等,有时为中下等。这里的贡物是三种贵金属,以及瑶石、琨石、各种竹子、象牙、兽皮、鸟羽、旄牛尾,岛夷进贡蓑衣,丝织品和海贝装在圆筐中,把橘子、柚子包裹好用来进贡,经由长江入海,通往淮水、泗水。荆山和衡山以南是荆州:长江、汉水流入大海。长江下游的众多支流有了固定的河道,沱水、涔水已经疏通,云梦泽也得以治理。这里的土壤是潮湿的泥土。田地为下中等,赋税为上下等。这里的贡物是鸟羽、旄牛尾、象牙、兽皮、三种贵金属,和椿木、柘木、桧木、柏木,以及砺石、砥石、砮石、朱砂,还有竹笋、簬竹、楛木。各国都进贡当地的特产,把菁茅装在匣子里包裹起来,黑红丝绸和丝绳珠串装在圆筐中,还有长江下游支流地区进贡的大龟,在长江、沱水、涔水、汉水乘船,经过雒水,到达南河。

\begin{yuanwen}
荆及衡阳维荆州:江、汉朝宗于海。九江甚中,沱、涔已道,云土梦\footnote{云梦,古湖泽,一说即洞庭湖。}为治。其土涂泥。田下中,赋上下。贡羽、旄、齿、革、金三品,杶\footnote{chūn}、榦\footnote{gàn}、栝\footnote{guā}、柏,砺、砥、砮\footnote{石箭镞。}、丹,维箘\footnote{jùn,竹笋。}、簬\footnote{lù}、楛\footnote{hù}。三国致贡其名,包匦\footnote{guǐ,匣子。}菁茅,其篚玄纁\footnote{xūn}玑组\footnote{丝绳穿的珠串。},九江入赐大龟,浮于江、沱、涔、(于)汉,逾于雒,至于南河。
\end{yuanwen}



\begin{yuanwen}
荆、河惟\footnote{通“维”。}豫州:伊、雒、瀍\footnote{chán}、涧既入于河,荥播既都,道荷泽,被\footnote{通“陂”,指筑堤防。}明都。其土壤,下土坟垆\footnote{黑色的硬土。}。田中上,赋杂上中。贡漆、丝、絺、纻,其篚纤絮,锡贡磬错\footnote{打磨玉磬的石头。}。浮于雒,达于河。
\end{yuanwen}

荆山、黄河之间是豫州:伊水、雒水、瀍水、涧水都已经汇入黄河,荥播已经蓄水成为湖泽,当水丰沛的时候,疏导菏泽的水,在明都泽修筑堤坝。这里的土壤是柔土,低洼处是黑色的硬土。田地为中上等,赋税为上中等,有的时候为上上等。这里的贡物是漆、丝、细葛布、苎麻,细丝絮装在圆筐中,还进贡打磨玉磬的石头,在雒水乘船,抵达黄河。

\begin{yuanwen}
华阳黑水惟梁州:汶、嶓\footnote{bō}既(蓺)艺,沱、涔既道,蔡、蒙旅平,和夷厎绩。其土青骊\footnote{青黑色的松散土壤。}。田下上,赋下中三错。贡璆\footnote{qiú}、铁、银、镂\footnote{硬铁。}、砮、磬,熊、罴、狐、貍、织皮\footnote{毛织物。}。西倾因桓是来,浮于潜,逾于沔,入于渭,乱\footnote{横渡。}于河。
\end{yuanwen}

华山以南、黑水之间是梁州:汶山、嶓山已经可以耕种了,沱水、涔水已经被疏导,蔡山、蒙山已经得以整治,和夷也安居乐业。这里的土壤是青黑色的松软泥土。田地为下上等,赋税为下中等,处在下上等到下下等之间。这里的贡物有璆玉、铁、银、硬铁、砮石、玉磬,以及熊、罴、狐、狸、毛纺织品,经由西倾山沿着桓水前来,在潜水乘船,越过沔水,进入渭水,横渡黄河。

\begin{yuanwen}
黑水西河惟雍州:弱水既西,泾属渭汭. 漆、沮既从,沣水所同。荆、岐已旅,终南、敦物至于鸟鼠。原隰\footnote{xí}厎绩,至于都野。三危既度\footnote{通“宅”,可居住。},三苗大序。其土黄壤。田上上,赋中下。贡璆、琳、琅玕\footnote{gān}. 浮于积石,至于龙门西河,会于渭汭. 织皮昆仑、析支、渠搜,西戎即序。
\end{yuanwen}

黑水、西河之间是雍州:弱水已经向西流去,泾水汇入渭水。漆水、沮水相继汇入洛水,沣水也汇入渭水。荆山、岐山已经得以治理,以及终南山、敦物山直到鸟鼠山。高平到低洼地带的整治都已经竣工,直到都野泽。三危山已经可以居住,三苗人生活安定。这里的土壤是黄色柔土。田地为上上等,赋税为中下等。这里的贡物是璆玉、琳玉、琅玕,在积石山进入黄河乘船,抵达龙门山西河,在泾水和渭水的交汇处集合。昆仑山、析支山、渠搜山进贡毛纺织品,西戎于是安居乐业。

\begin{yuanwen}
道九山:汧\footnote{qiān,通“岍”,山名。}及岐至于荆山,逾于河;壶口、雷首至于太岳;砥柱、析城至于王屋;太行、常山至于碣石,入于海;西倾、朱圉、鸟鼠至于太华;熊耳、外方、桐柏至于负尾;道嶓冢,至于荆山;内方至于大别;汶山之阳至衡山,过九江,至于敷浅原。
\end{yuanwen}

禹开通了九条山路:从汧山、岐山直到荆山,跨过黄河;从壶口山、雷首山直到太岳山;从砥柱山、析城山直到王屋山;从太行山、常山直到碣石山,进入大海;从西倾山、朱圉山、鸟鼠山直到太华山;从熊耳山、外方山、桐柏山直到负尾山;从嶓冢山直到荆山;从内方山直到大别山;从汶山以南直到衡山,经过九江,直到敷浅原。

\begin{yuanwen}
道九川:弱水至于合黎,余波入于流沙。道黑水,至于三危,入于南海。道河积石,至于龙门,南至华阴,东至砥柱,又东至于盟津,东过雒汭,至于大邳,北过降水,至于大陆,北播为九河,同为逆河\footnote{黄河入海处,因海潮逆流而得名。},入于海。嶓冢道漾(瀁),东流为汉,又东为苍浪之水,过三澨\footnote{shì},入于大别,南入于江,东汇泽为彭蠡,东为北江,入于海。汶山道江,东别为沱,又东至于醴,过九江,至于东陵,东迆北会于汇\footnote{通“淮”,淮水。},东为中江,入于海。道沇\footnote{yǎn}水,东为济,入于河,泆\footnote{同“溢”。}为荥,东出陶丘北,又东至于荷,又东北会于汶,又东北入于海。道淮自桐柏,东会于泗、沂,东入于海。道渭自鸟鼠同穴,东会于沣,又东北至于泾,东过漆、沮,入于河。道雒自熊耳,东北会于涧、瀍,又东会于伊,东北入于河。
\end{yuanwen}

禹疏导了九条大河:将弱水疏导至合黎,下游流入沙漠;疏导黑水,经过三危山,最后注入南海;在积石山疏导黄河,一直到龙门山,向南到华山以北,向东到砥柱山,再向东到盟津,向东经过雒水入河口,一直到大邳山,向北经过降水,一直到大陆泽,向北分为九条支流,又汇合为逆河,最后注大海;在嶓冢山疏导瀁水,向东流是汉水,再向东流是苍浪水,经过三澨水,进入大别山,向南流入长江,向东汇入彭蠡泽,向东流是北江,最后注入大海;在汶山疏导长江,向东分出支流是沱水,再向东到醴水,经过九江,一直到东陵,向东倾斜流向北方汇入淮水,向东流是中江,最后注入大海;疏导沇水,向东流是济水,流入黄河,溢出聚为荥播泽,向东到陶丘以北,再向东抵达菏泽,又向东北汇入汶水,最后向东北注入大海;从桐柏山疏导淮水,向东和泗水、沂水汇合,最后向东注入大海;从鸟鼠山疏导渭水,向东和沣水汇合,再向东北流到泾水,向东流经漆水、沮水,流入黄河;从熊耳山疏导雒水,向东北和涧水、瀍水汇合,又向东与伊水汇合,向东北流入黄河。

\begin{yuanwen}
于是九州攸同,四奥\footnote{四方之内。奥,同“墺”,可定居之地。}既居,九山刊旅,九川涤原\footnote{疏通水源。原,同“源”。},九泽既陂\footnote{bēi},四海会同。六府甚修,众土交正,致慎财赋,咸则三壤成赋。中国赐土姓:“祗台\footnote{台,我,天子自称。}德先,不距\footnote{违。}朕行。”
\end{yuanwen}

这时九州统一,四境之内都可以居住,九州的大山得到整治,九州的河道已经疏通,九州的湖泽附近也修筑了堤坝,四海之内的诸侯都来朝贡。六府的物资管理得当,各处的土地都能正确勘测,根据各地情况谨慎征收赋税,都按照土壤的不同等级来征收。中央赐给诸侯土地、姓氏:“恭敬中央以德业为先,不要违背我的一贯作风。”

\begin{yuanwen}
令天子之国以外五百里甸服:百里赋纳总\footnote{连杆带穗的禾。},二百里纳铚\footnote{zhì,禾穗。},三百里纳秸服\footnote{去掉杆和芒的谷物。},四百里粟,五百里米。甸服外五百里侯服:百里采\footnote{为天子服役。},二百里任国\footnote{管理国家事务。},三百里诸侯\footnote{听候天子之命。侯,通“候”。}。侯服外五百里绥服:三百里揆文教,二百里奋武卫。绥服外五百里要服:三百里夷\footnote{指奉行天子常法。},二百里蔡\footnote{指只奉行天子刑法。}。要服外五百里荒服:三百里蛮\footnote{指以文德招抚。},二百里流\footnote{指自由流动,即不强制朝贡。}。
\end{yuanwen}

禹规定在天子国都以外五百里的地方称为甸服:一百里以内要缴纳整株庄稼,二百里以内要缴纳带穗的庄稼,三百里以内要缴纳去掉杆和芒的谷物,四百里以内要缴纳粟米,五百里以内要缴纳稻米。甸服以外五百里的地方称为侯服:一百里以内要为天子服役,二百里以内要管理国家事务,三百里以内要听候天子之命。侯服以外五百里的地方称为绥服:三百里以内要推行文教,二百里以内要奋力保卫王室。绥服以外五百里的地方称为要服:三百里以内要奉行天子常法,二百里以内只奉行天子刑法。要服以外五百里的地方称为荒服:三百里以内要以文德招抚,二百里以内的民众可以自由流动。

\begin{yuanwen}
东渐于海,西被于流沙,朔、南暨:声教讫\footnote{尽。}于四海。于是帝锡禹玄圭,以告成功于天下。天下于是太平治。
\end{yuanwen}

国土向东延伸到大海,向西覆盖到沙漠,从北方到南方,天子的声威教化可以传遍四海。于是舜帝赏赐禹玄圭,用来向全天下人宣布治水的成功。天下从此太平安定了。

\begin{yuanwen}
皋陶作士以理民。帝舜朝,禹、伯夷、皋陶相与语帝前。皋陶述其谋曰:“信其道德,谋明辅和。”

禹曰:“然,如何?”

皋陶曰:“於!慎其身修,思长,敦序九族,众明高翼,近可远在已。”

禹拜美言,曰:“然。”

皋陶曰:“於!在知人,在安民。”

禹曰:“吁!皆若是,惟帝其难之。知人则智,能官人;能安民则惠,黎民怀之。能知能惠,何忧乎驩兜,何迁乎有苗,何畏乎巧言善色佞人?”

皋陶曰:“然,於!亦行有九德,亦言其有德。”

乃言曰:“始事事,宽而栗\footnote{威严。},柔而立\footnote{坚定。},愿而共\footnote{同“恭”。},治而敬,扰\footnote{顺服。}而毅,直而温,简而谦,刚而实,强而义,章其有常吉哉。日宣三德,蚤\footnote{通“早”。}夜翊\footnote{yì}明有家\footnote{卿大夫的封地。}。日严振敬六德,亮采有国\footnote{诸侯的封地。}。翕\footnote{xī}受普施,九德咸事,俊乂在官,百吏肃谨。毋教邪淫奇谋。非其人居其官,是谓乱天事。天讨有罪,五刑五用哉。吾言厎可行乎?”

禹曰:“女言致可绩行。”

皋陶曰:“余未有知,思赞道哉。”
\end{yuanwen}

皋陶担任司法官来治理人民。帝舜上朝,禹、伯夷、皋陶一同上前和帝舜说话。皋陶讲述他的主张说:“要注重道德修养,长远谋划,共同辅佐天子。”

禹说:“正是这样,但是该怎么做呢?”

皋陶说:“啊!要谨慎地约束自身,使族人宽厚顺从,让众贤士努力辅佐,由近及远,从自身做起。”

禹下拜并赞赏他的话,说:“正是这样。”

皋陶又说:“啊!这取决于善用人才,取决于安定百姓。”

禹说:“唉!如果都是这样,只怕连帝尧也会感到困难。善用人才就是智慧,可以任用这样的人做官;能够安定百姓就是仁惠,这样民众才会心怀感激。能够做到善用人才,又可以施恩惠于百姓,还担心什么驩兜,还流放什么三苗,还畏惧什么巧言令色的坏人呢?”

皋陶说:“是的,啊!通常人有九种品德,听人说话也要看他的品德。”

于是他说:“开始做事时,宽厚而威严,柔和而坚定,诚实而恭敬,善治而谨慎,顺服而刚毅,正直而温和,平易而廉洁,果断而踏实,强直而道义,能够彰显这些德行就可以把事情做好。每天能够表现出三种品德,从早到晚勤勉遵行,就可以被封为卿大夫。每天严谨恭敬地表现出六种品德,用来辅佐天子,就可以被封为诸侯。全部接受并普遍施行,就可以为具备九种品德的人安排职事,让才能出众的人得到官位,百官能够恭肃谨慎。不要宣扬骄奢淫逸和施行阴谋诡计,不称职的人在官位上,这就是扰乱天命。上天要讨伐那些有罪的人,会按照五刑分别执行啊!我说的这些话可以成功地实行吗?”

禹说:“你说的话完全可以成功地实行。”

皋陶说:“我不聪明,只想着治国之道啊!”

\begin{yuanwen}
帝舜谓禹曰:“女亦昌言。”

禹拜曰:“於,予何言!予思日孳孳\footnote{同“孜孜”,勤勉不懈的样子。}。”

皋陶难禹曰:“何谓孳孳?”

禹曰:“鸿水滔天,浩浩怀山襄陵,下民皆服\footnote{承受,忍受。}于水。予陆行乘车,水行乘舟,泥行乘橇,山行乘檋,行山刊木。与益予众庶稻鲜食。以决九川致四海,浚畎浍\footnote{huì}致之川。与稷予众庶难得之食。食少,调有余补不足,徙居。众民乃定,万国为治。”

皋陶曰:“然,此而美也。”
\end{yuanwen}

帝舜对禹说:“你也说一说好的建议。”

禹下拜说:“啊,我该说什么呢?每天我只想着勤勤恳恳地做好自己的工作。”

皋陶追问禹说:“怎样才是勤勤恳恳地做好工作呢?”

禹说:“洪水波涛漫天,浩浩荡荡地淹没高山,冲上丘陵,天下民众正遭受着洪水的威胁。我在平地行走时乘坐车驾,在水路行进时乘坐船只,在泥地行走时使用木橇,在山间攀爬时穿上底部有齿的鞋,沿着山势伐木设立路标。我和益为百姓送去稻粮和新鲜的肉食。我将九条大河疏导而引入四方大海,把田间沟渠疏浚而引入各大江河。我和稷为百姓送去应急的食物。在食物匮乏的时候,我们就从富裕的地方调来粮食补充不足的地方,或者迁徙民众。百姓于是安定下来,各国也太平无事。”

皋陶说:“正是这样,这是美好的功绩。”

\begin{yuanwen}
禹曰:“於,帝!慎乃在位,安尔止。辅德,天下大应。清意以昭待上帝命,天其重命用休。”

帝曰:“吁,臣哉,臣哉!臣作朕股肱\footnote{gōng}耳目。予欲左右有民,女辅之。余欲观古人之象,日月星辰,作文绣服色,女明之。予欲闻六律五声八音,来始滑,以出入五言,女听。予即辟,女匡拂\footnote{通“弼”,辅佐。}予。女无面谀,退而谤予。敬四辅臣。诸众谗嬖臣,君德诚施皆清矣。”

禹曰:“然。帝即不时,布同善恶则毋功。”
\end{yuanwen}

禹说:“啊!帝舜,身居帝位一定要谨慎,把你的心态摆正。用仁德辅政,天下人就都会拥护你。等待有识之士明白地奉行天帝的命令,上天就会降下祥瑞。”

帝舜说:“啊!大臣啊,大臣啊!大臣应该做我的手足耳目。我想帮助天下民众,你们就要辅助我;我想要观察古人衣服上的纹饰,把日月星辰等图案,绣到衣服上,你们就要明辨等级;我要倾听六律、五声、八音,观察治乱兴衰,来采纳各方的意见,你们就要前去听取。如果我有做错的地方,你们要匡正辅佐我。你们不要当着面阿谀,退朝后又诽谤我。我敬重各辅臣。那些喜欢进谗言邀宠幸的奸臣,只要能够推行德政,他们就会被全部清除了。”

禹回答说:“正是这样。如果您不这么做,使善恶并存,那么您就不会取得功业了。”

\begin{yuanwen}
帝曰:“毋若丹朱傲,维\footnote{通“唯”,只。}慢游是好,毋水行舟,朋淫于家,用绝其世。予不能顺是。”

禹曰:“予娶涂山,辛壬癸甲\footnote{代指四天。},生启予不子\footnote{指没有回家看儿子。},以故能成水土功。辅成五服,至于五千里,州十二师,外薄四海,咸建五长,各道有功,苗顽不即功,帝其念哉。”

帝曰:“道吾德,乃女功序之也。”
\end{yuanwen}

帝舜说:“不要像丹朱那样狂傲,只喜欢放纵游玩,没有水也要乘船,在家纵情声色,导致没有继承帝位。我不能像他那样。”

禹说:“我娶涂山氏的女子为妻,四天后就离开家,她生下启我也没有回家看儿子,所以能够成功治理洪水。我开辟疆土划分五服,拓展到五千里外,设置十二州的长官,政教向外宣导至临近四海的地方,任命五个诸侯之长,让他们按照各自的方法取得成绩。三苗凶顽而没有功绩,您一定要记住啊!”

帝舜说:“我用仁德治理天下,让他们归顺是你的功劳。”

\begin{yuanwen}
皋陶于是敬禹之德,令民皆则禹。不如言,刑从之。舜德大明。
\end{yuanwen}

皋陶从此更加敬重禹的品德,他命令百姓都以禹为准则。如果有人不听从命令,他就对其处以刑罚。于是帝舜的德业更加昌明。

高燮:「自古创业之功,莫高于大禹,而中兴之功莫盛于少康,太史公述《夏本纪》载禹治水一事独详,是也。」

\begin{yuanwen}
于是夔行乐,祖考至,群后相让,鸟兽翔舞,《箫韶\footnote{sháo}》九成\footnote{演奏九次而礼成。},凤凰来仪,百兽率舞,百官信谐。帝用此作歌曰:“陟天之命,维时维几。”乃歌曰:“股肱喜哉,元首起哉,百工熙哉!”

皋陶拜手稽首扬言曰:“念哉,率为兴事,慎乃宪,敬哉!”乃更为歌曰:“元首明哉,股肱良哉,庶事康哉!”(舜)又歌曰:“元首丛脞\footnote{琐碎之事。脞,cuǒ。}哉,股肱惰哉,万事堕哉!”

帝拜曰:“然,往钦哉!”于是天下皆宗禹之明度数声乐,为山川神主。
\end{yuanwen}

这时夔开始演奏乐曲,祖先神灵全部降临,诸侯相互礼让,鸟兽翩翩起舞,《箫韶》演奏九次而礼成,凤凰也成双飞来,百兽共同舞蹈,百官协调一致。帝舜因此作歌唱道:“奉行天命,实行德政,顺应天时,谨微慎行。”接着他又唱道:“大臣高兴啊,君主努力啊,百官事业兴盛啊!”

皋陶拱手行礼叩头大声说:“一定要记住啊,带头干好事业,谨慎对待法令,一定要恭敬啊!”

帝舜于是接着唱道:“君主圣明啊,大臣贤良啊,所有的事情都安乐啊!”他又唱道:“君主计较小事啊,大臣为政怠惰啊,所有的事情都败坏啊!”

帝舜下拜说:“是这样,都去努力做事吧!”这时天下人都以禹所公示的度量和音律为准则,尊奉他为山川神主。

\begin{yuanwen}
帝舜荐禹于天,为嗣。十七年而帝舜崩。三年丧毕,禹辞辟\footnote{同“避”。}舜之子商均于阳城。天下诸侯皆去商均而朝禹。禹于是遂即天子位,南面朝天下,国号曰夏后,姓姒氏。
\end{yuanwen}

帝舜向上天推荐禹,立他为继承人。十七年后帝舜去世。三年丧期结束,禹躲避到阳城而将天子之位让给舜的儿子商均。天下诸侯都离开商均而去朝见禹。禹于是登上天子之位,面向南方治理天下,国号为夏后,姓姒氏。

\begin{yuanwen}
帝禹立而举皋陶荐之,且授政焉,而皋陶卒。封皋陶之后于英、六,或在许。而后举益,任之政。
\end{yuanwen}

帝禹即位后向上天举荐皋陶,并且把政事交给他,可是皋陶死了。禹把英、六两地封给皋陶的后代,也有人被封在许。之后禹任用益,让他处理政事。

\begin{yuanwen}
十年,帝禹东巡狩,至于会稽而崩。以天下授益。三年之丧毕,益让帝禹之子启,而辟居箕山之阳。禹子启贤,天下属意焉。及禹崩,虽授益,益之佐禹日浅,天下未洽。故诸侯皆去益而朝启,曰“吾君帝禹之子也。”于是启遂即天子之位,是为夏后帝启。
\end{yuanwen}

在位第十年的时候,帝禹到东方巡视,在会稽去世,把天下让给了益。三年丧期结束,益把帝位让给了帝禹的儿子启,自己躲避到箕山以南。禹的儿子启很有才能,天下人都希望他可以做天子。等到禹去世的时候,虽然把天下让给了益,但是益辅佐禹的时间还很短,天下人还不信服。所以天下诸侯都离开益而去朝见启,说“我君是帝禹的儿子”。于是启登上天子之位,这就是夏后帝启。

\begin{yuanwen}
夏后帝启,禹之子,其母涂山氏之女也。
\end{yuanwen}

夏后帝启,是禹的儿子,他的母亲是涂山氏的女子。

\begin{yuanwen}
有扈\footnote{hù}氏不服,启伐之,大战于甘。将战,作《甘誓》,乃召六卿\footnote{天子六军的主帅。}申之。启曰:“嗟!六事之人,予誓告女:有扈氏威侮五行\footnote{这里指天命。},怠弃三正\footnote{三政,指正德、利用、厚生。},天用剿绝其命。今予维共行天之罚。左\footnote{战车载三人,左持弓箭,右持矛戈,中间的人驾车。}不攻于左,右不攻于右,女不共命。御非其马之政,女不共命。用命,赏于祖;不用命,僇\footnote{lù,通“戮”,杀。}于社,予则帑僇\footnote{刑罚连及子女。也作“孥戮”。}女。”

遂灭有扈氏。天下咸朝。
\end{yuanwen}

有扈氏不服从启的命令,启率兵讨伐,在甘激战。开战前,启作《甘誓》,于是召集六军主帅训诫。启说:“啊!六军的将士们,我向你们誓师:有扈氏蔑视天命,荒废政事,上天要断绝他们的命数。现在我恭敬地奉行上天的责罚。车左不能完成车左的任务,车右不能完成车右的任务,你们就是不遵守命令。御手不能驾驭战车,也是不遵守命令。遵守命令的,就在祖庙里给予奖赏;不遵守命令的,就在社坛前将其杀死,还要诛灭你们的子女。”

最终启灭掉有扈氏。天下诸侯都来朝见启。

\begin{yuanwen}
夏后帝启崩,子帝太康立。帝太康失国,昆弟五人,须于洛汭,作《五子之歌》。
\end{yuanwen}

夏后帝启去世,他的儿子帝太康继位。帝太康失掉国家,他的五个兄弟逃到洛水转弯处等待太康归来,作《五子之歌》。

\begin{yuanwen}
太康崩,弟中康立,是为帝中康。帝中康时,羲、和湎淫,废时乱日。胤往征之,作《胤征》。
\end{yuanwen}

太康去世,他的弟弟中康继位,这就是帝中康。帝中康在位的时候,掌管天地四时的羲氏、和氏沉湎酒色,荒废职事,扰乱时日。胤奉命前往征讨他们,作《胤征》。

\begin{yuanwen}
中康崩,子帝相立。帝相崩,子帝少康立\footnote{据《索隐》《正义》考证,有穷氏的后羿驱逐太康而当政,到帝相时篡位,后来寒浞杀后羿当政,至少康时才复国。司马迁并未采用这些史料。}。帝少康崩,子帝予立。帝予崩,子帝槐立。帝槐崩,子帝芒立。帝芒崩,子帝泄立。帝泄崩,子帝不降立。帝不降崩,弟帝扃\footnote{jiōng}立。帝扃崩,子帝廑\footnote{jǐn}立。帝廑崩,立帝不降之子孔甲,是为帝孔甲。
\end{yuanwen}

中康去世,他的儿子帝相继位。帝相去世,他的儿子帝少康继位。帝少康去世,他的儿子帝予继位。帝予去世,他的儿子帝槐继位。帝槐去世,他的儿子帝芒继位。帝芒去世,他的儿子帝泄继位。帝泄去世,他的儿子帝不降继位。帝不降去世,他的弟弟帝扃继位。帝扃去世,他的儿子帝廑继位。帝廑去世,立帝不降的儿子孔甲,这就是帝孔甲。

\begin{yuanwen}
帝孔甲立,好方\footnote{通“仿”,仿效。}鬼神,事淫乱。夏后氏德衰,诸侯畔之。天降龙二,有雌雄,孔甲不能食,未得豢\footnote{huàn}龙氏。陶唐\footnote{帝尧的后裔。}既衰,其后有刘累,学扰龙于豢龙氏,以事孔甲。孔甲赐之姓曰御龙氏,受豕韦之后。龙一雌死,以食夏后。夏后使求,惧而迁去。
\end{yuanwen}

帝孔甲即位以后,喜欢仿效鬼神,放纵淫乱。夏后氏的命数日渐衰败,诸侯纷纷叛离。这时天上降下两条龙,一雌一雄,孔甲不知道如何饲养,找不到善于养龙的豢龙氏。当时陶唐氏已经衰落,其后代有一个叫刘累的人,曾经向豢龙氏学过养龙的本领,以此事奉孔甲。孔甲赐他姓御龙氏,封他为豕韦国的君主。后来那条雌龙死了,刘累把龙肉给孔甲吃。夏帝想要看那两条龙的时候,刘累因恐惧而逃走了。

\begin{yuanwen}
孔甲崩,子帝皋立。帝皋崩,子帝发立。帝发崩,子帝履癸立,是为桀。\end{yuanwen}

孔甲去世,他的儿子帝皋继位。帝皋去世,他的儿子帝发继位。帝发去世,他的儿子帝履癸继位,这就是桀。

\begin{yuanwen}
帝桀之时,自孔甲以来而诸侯多畔夏,桀不务德而武伤百姓,百姓弗堪。乃召汤而囚之夏台,已而释之。汤修德,诸侯皆归汤,汤遂率兵(以)伐夏桀。桀走鸣条,遂放\footnote{流放。}而死。桀谓人曰:“吾悔不遂杀汤于夏台,使至此。”

汤乃践天子位,代夏朝天下。汤封夏之后,至周封于杞也。
\end{yuanwen}

帝桀在位的时候,由于自孔甲以来诸侯多有背叛夏朝的,桀不崇尚德政而用武力去镇压诸侯、伤害百族,百族都无法忍受。于是桀征召汤,并把他囚禁在夏台,不久又把他放了。汤能够修持德政,天下诸侯都归附汤,于是汤率兵讨伐夏桀,桀逃到鸣条,最终被流放而死。桀对人说:“我真后悔没在夏台杀死汤,才落得如此下场。”

于是汤登上天子之位,取代夏朝拥有天下。汤赐予夏朝后代封地,到周朝时将其封在杞国。

\begin{yuanwen}
太史公曰:禹为姒姓,其后分封,用国为姓,故有夏后氏、有扈氏、有男氏、斟寻氏、彤城氏、褒氏、费氏、杞氏、缯\footnote{zèng}氏、辛氏、冥氏、斟(氏)戈氏。孔子正夏时,学者多传《夏小正》云。自虞、夏时,贡赋备矣。或言禹会诸侯江南,计功而崩,固(因)葬焉,命曰会稽。会稽者,会计\footnote{指天子对诸侯论功行赏。}也。
\end{yuanwen}

太史公说:禹为姒姓,他的后代被分封,就以国号为姓,因此有夏后氏、有扈氏、有男氏、斟寻氏、彤城氏、褒氏、费氏、杞氏、缯氏、辛氏、冥氏、斟戈氏。孔子校正夏朝历法,很多学者传习《夏小正》的说法。从虞、夏时开始,贡赋制度已经完备了。有人认为禹曾经在江南召集诸侯,在考核诸侯功绩的时候去世了,因此就葬在那里,人们就把当地命名为会稽。会稽,就是会计。

\part{卷三}
\chapter{殷本纪第三}

本篇记述了商朝的历史,多取材于《尚书》《诗经》。商朝前期曾多次迁都,直到盘庚迁殷才安定下来。周人习惯称商朝为“殷”,汉朝承袭了这一称呼。相传成汤灭夏是历史上第一次改朝换代,与后来的武王伐纣并称“革命”,都具有划时代的意义,备受儒家推崇。

茅坤:「伊尹与成汤同起伐(商),《夏本纪》所载汤至太戊且七世矣,而尹之子陟乃为相,岂得年寿悬绝若此!可见史迁所述帝系世本,未有不足凭者。」

\begin{yuanwen}
殷契,母曰简狄,有娀\footnote{sōng}氏之女,为帝喾次妃。三人行浴,见玄鸟堕其卵,简狄取吞之,因孕生契。契长而佐禹治水有功。帝舜乃命契曰:“百姓不亲,五品不训\footnote{指父义、母慈、兄友、弟恭、子孝五种伦理教化。},汝为司徒而敬敷五教,五教在宽。”

封于商,赐姓子氏。契兴于唐、虞、大禹之际,功业著于百姓,百姓以平。
\end{yuanwen}

殷契的母亲名叫简狄,她是有娀氏的女子,是帝喾的次妃。包括她在内的三个女子外出洗澡,看见一只玄鸟生蛋落下来,简狄就捡起来吃了,因而怀孕生下契。契长大以后辅佐禹治水有功。帝舜于是任命契说:“百姓不相亲睦,教化不能和顺,你担任司徒,要恭谨地推行五种伦理教化,要以宽厚为根本。”

舜把商地封给他,赐姓子氏。契兴起于唐尧、虞舜、大禹的时代,在百姓心中功勋卓著,百姓因此生活安定。

\begin{yuanwen}
契卒,子昭明立。昭明卒,子相土立。相土卒,子昌若立。昌若卒,子曹圉\footnote{yǔ}立。曹圉卒,子冥立。冥卒,子振立。振卒,子微立。微卒,子报丁立。报丁卒,子报乙立。报乙卒,子报丙立。报丙卒,子主壬立。主壬卒,子主癸立。主癸卒,子天乙立,是为成汤。
\end{yuanwen}

契死后,他的儿子昭明继位。昭明死后,他的儿子相土继位。相土死后,他的儿子昌若继位。昌若死后,他的儿子曹圉继位。曹圉死后,他的儿子冥继位。冥死后,他的儿子振继位。振死后,他的儿子微继位。微死后,他的儿子报丁继位。报丁死后,他的儿子报乙继位。报乙死后,他的儿子报丙继位。报丙死后,他的儿子主壬继位。主壬死后,他的儿子主癸继位。主癸死后,他的儿子太乙继位,这就是成汤。

\begin{yuanwen}
成汤,自契至汤八迁。汤始居亳\footnote{bó},从先王居,作《帝诰》。	
\end{yuanwen}

成汤,从契到汤一共迁都八次。到汤在位时才定都于亳,这是先王居住过的地方,汤作《帝诰》。

\begin{yuanwen}
汤征诸侯。葛伯不祀,汤始伐之。汤曰:“予有言:人视水见形,视民知治不\footnote{同“否”。}。”

伊尹曰:“明哉!言能听,道乃进。君国子民,为善者皆在王官。勉哉,勉哉!”

汤曰:“汝不能敬命,予大罚殛之,无有攸\footnote{同“所”。}赦。”作《汤征》。
\end{yuanwen}

汤征伐诸侯。葛伯不按时祭祀,汤开始征讨他。汤说:“我说过这样的话:人在水中能看到自己的形象,观察百姓就能知道自己的治理是否得当。”

伊尹说:“圣明啊!能听取别人的意见,道德才会进步。治理国家,管理人民,做好事的人就都在帝王身边做官。努力吧,努力吧!”

汤说:“如果你们不听从我的命令,我就要严厉地惩罚你们,不会宽赦。”汤作《汤征》。

\begin{yuanwen}
伊尹名阿衡。阿衡欲奸\footnote{gān,通“干”,求见。}汤而无由,乃为有莘氏媵臣,负鼎俎,以滋味说汤,致于王道。或曰,伊尹处士\footnote{隐居不仕的贤人。},汤使人聘迎之,五反然后肯往从汤,言素王\footnote{有帝王之德而无帝王之位的人。}及九主\footnote{指三皇、五帝、夏禹。}之事。汤举任以国政。伊尹去汤适夏。既丑有夏,复归于亳。入自北门,遇女鸠、女房,作《女鸠》、《女房》。	
\end{yuanwen}

伊尹的官名是阿衡。阿衡想要求见汤却找不到门路,于是成为有莘氏女子陪嫁的奴隶,背着厨具,凭借谈论烹调的道理来说服汤,最终成就王道。有人说,伊尹是一位隐士,汤派人聘请迎接他,往返五次他才肯前往跟随汤,讲述素王和九主的事情。汤任用他来管理国政。伊尹离开汤去夏都。他看到夏朝的丑恶,于是又回到亳。他从北门进入城中,遇到汤的大臣女鸠、女房,于是作《女鸠》、《女房》。

\begin{yuanwen}
汤出,见野张网四面,祝曰:“自天下四方皆入吾网。”

汤曰:“嘻,尽之矣!”乃去其三面,祝曰:“欲左,左。欲右,右。不用命,乃入吾网。”

诸侯闻之,曰:“汤德至矣,及禽兽。”
\end{yuanwen}

汤外出,看见野外有猎人在四个方向设网打猎,并且祷告说:“愿天下四方来的猎物都进入我的网中。”

汤说:“唉,那样就把鸟兽都捕尽了!”于是撤去三个方向的网,并且祷告说:“想要去左边的,向左走。想要去右边的,向右走。不听从命令的,才会进入我的网中。”

诸侯听说以后,都说:“汤的德行已经达到最高境界了,连鸟兽也能享受他的恩惠。”

\begin{yuanwen}
当是时,夏桀为虐政淫荒,而诸侯昆吾氏为乱。汤乃兴师率诸侯,伊尹从汤,汤自把钺\footnote{yuè}以伐昆吾,遂伐桀。

汤曰:“格女众庶,来,女悉听朕言。匪\footnote{同“非”。}台小子\footnote{帝王的自谦之辞。}敢行举乱,有夏多罪,予维闻女众言,夏氏有罪。予畏上帝,不敢不正\footnote{通“征”。}。今夏多罪,天命殛之。今女有众,女曰:‘我君不恤我众,舍我啬事\footnote{农事。啬,通“穑”,收割谷物。}而割政\footnote{暴政。割,hài,通“害”。}’。女其曰:‘有罪,其奈何?’夏王率止众力,率夺夏国。众有率怠不和,曰:‘是日何时丧?予与女皆亡!’夏德若兹,今朕必往。尔尚\footnote{通“倘”,如果。}及予一人\footnote{帝王的自称。}致天之罚,予其大理女。女毋不信,朕不食言。女不从誓言,予则帑僇\footnote{刑罚连及子女。也作“孥戮”。帑,nú。}女,无有攸赦。”

以告令师,作《汤誓》。于是汤曰:“吾甚武”,号曰武王。
\end{yuanwen}

在这个时候,夏桀施行暴政而荒淫无道,同时诸侯昆吾氏趁机作乱。于是汤发兵率领诸侯前去讨伐,伊尹跟随汤,汤亲自手持斧钺来攻打昆吾氏,接着又去攻打夏桀。

汤说:“你们这些人,来,都听我说。不是我胆敢起来叛乱,夏朝的罪恶深重,我听你们也说,夏朝有罪。我畏惧天帝,不敢不去征伐。现在夏桀的罪恶深重,这是上天要诛杀他。现在你们这些人,都说:‘我们的君王不体恤我们众人,废弃农事而推行暴政。’你们说:‘夏朝有罪,又能怎么样呢?’夏王竭尽民力为他一人服务,掠夺整个夏国的财富。有些民众都懈怠而不愿服从,说:‘这个太阳什么时候才会灭亡?我们宁愿和你同归于尽!’夏朝的命数已经衰败到这个地步了,现在我必须前往征伐。你们跟随我执行上天的惩罚,我就会重赏你们。你们不要不相信,我不会说话不守信用。如果你们不服从誓言,我就诛杀你们和你们子女,绝不宽赦。”

汤用这些话号令全军,作《汤誓》。于是汤说:“我十分勇武”,号称武王。

郝敬:「孔甲好鬼而淫乱,无足称者。」

\begin{yuanwen}
桀败于有娀之虚\footnote{同“墟”,旧址。},桀奔于鸣条,夏师败绩。汤遂伐三嵕,俘厥宝玉,义伯、仲伯作《典宝\footnote{《尚书》佚篇。《五帝本纪》、《夏本纪》、《殷本纪》、《周本纪》大量引用《尚书》文字,而《典宝》、《夏社》、《明居》等篇目不见于今本《尚书》,《汤诰》篇目虽见于伪古文《尚书》,文字却大不相同。可见《史记》在保存古代文献方面的重要贡献。}》。汤既胜夏,欲迁其社,不可,作《夏社》。伊尹报。于是诸侯毕服,汤乃践天子位,平定海内。
\end{yuanwen}

桀在有娀氏的故地被打败,逃到鸣条,夏军战败。汤于是攻打三嵕,获得了那里的宝玉,义伯、仲伯因此作《典宝》。汤战胜夏桀以后,就想要迁移社神,没能成功,于是作《夏社》。伊尹向汤报告。这时诸侯都归顺了汤,汤于是登上天子之位,平定四海之内。

\begin{yuanwen}
汤归至于泰卷陶,中垒(仲虺huǐ)作诰。既绌\footnote{通“黜”,废止,废弃。}夏命,还亳,作《汤诰》:“维三月,王自至于东郊。告诸侯群后:‘毋不有功于民,勤力乃事。予乃大罚殛女,毋予怨。’曰:‘古禹、皋陶久劳于外,其有功乎民,民乃有安。东为江,北为济,西为河,南为淮,四渎已修,万民乃有居。后稷降播,农殖百谷。三公咸有功于民,故后有立。昔蚩尤与其大夫作乱百姓,帝乃弗予,有状。先王言不可不勉。’曰:‘不道,毋之在国,女毋我怨。’”以令诸侯。伊尹作《咸有一德》,咎单作《明居》。
\end{yuanwen}

汤在返回的途中来到泰卷陶,大臣仲虺作了一篇诰。汤已经推翻了夏朝,回到亳,作《汤诰》:“在三月,王亲自来到东郊。他告诫天下诸侯说:‘不要在对民众毫无功绩,努力做好你们的事情。否则我就要严厉地惩罚你们,你们不要怨恨我。’他又说:‘古时候的禹、皋陶长年在外面操劳,他们对民众有功绩,民众才得以安定。他们在东方治理长江,在北方治理济水,在西方治理黄河,在南方治理淮水,这四条大河治理好以后,天下百姓才有了居住的地方。后稷教导人民耕种,人们才知道种植各种庄稼。他们三位都对民众有功绩,所以他们的后代才能建立国家。从前蚩尤和他的大臣为害百姓,天帝不保佑他,这些都是真实的事情。先代圣王的话不可以不用来勉励自己。’他又说:‘为政无道,就不让他治理国家,你们也不要怨恨我。’”他以此命令诸侯。伊尹作《咸有一德》,咎单作《明居》。

\begin{yuanwen}
汤乃改正朔\footnote{指历法。},易服色,上\footnote{同“尚”,崇尚。}白,朝会以昼。
\end{yuanwen}

汤于是变更历法,改易服饰的颜色,崇尚白色,群臣在白天朝见天子。

\begin{yuanwen}
汤崩,太子太丁未立而卒,于是乃立太丁之弟外丙,是为帝外丙。帝外丙即位三年,崩,立外丙之弟中壬,是为帝中壬。帝中壬即位四年,崩,伊尹乃立太丁之子太甲。太甲,成汤敌\footnote{通“嫡”。}长孙也,是为帝太甲。帝太甲元年,伊尹作《伊训》,作《肆命》,作《徂后》。
\end{yuanwen}

汤去世以后,太子太丁还没有继位就死去了,于是就立太丁的弟弟外丙为帝,这就是帝外丙。帝外丙在位三年,去世以后,立外丙的弟弟中壬为帝,这就是帝中壬。帝中壬在位四年,去世以后,伊尹于是立太丁的儿子太甲为帝。太甲是成汤的嫡长孙,他就是帝太甲。帝太甲元年,伊尹作《伊训》,作《肆命》,作《徂后》。

\begin{yuanwen}
帝太甲既立三年,不明,暴虐,不遵汤法,乱德,于是伊尹放之于桐宫。三年,伊尹摄行政当国,以朝诸侯。
\end{yuanwen}

帝太甲在位三年,政治不清明,为人很残暴,不遵守汤制定的法度,道德败坏,于是伊尹将他流放到桐宫。在太甲流放的三年中,伊尹代行天子职权处理国政,接受诸侯的朝见。

\begin{yuanwen}
帝太甲居桐宫三年,悔过自责,反\footnote{同“返”,归向。}善,于是伊尹乃迎帝太甲而授之政\footnote{这种说法出自《尚书》《孟子》等儒家典籍,而《竹书纪年》则记载“仲壬崩,伊尹放太甲于桐,乃自立也”,七年后,太甲逃出流放地,杀死伊尹,夺回帝位,把伊尹的封地平分为他的两个儿子伊陟、伊奋。}。帝太甲修德,诸侯咸归殷\footnote{盘庚迁都于殷以前,国号不应称殷。“殷”为周人对商朝的称呼,司马迁作《殷本纪》将各时期的商朝统称为“殷”。},百姓以宁。伊尹嘉之,乃作《太甲训》三篇,褒帝太甲,称太宗。
\end{yuanwen}

帝太甲在桐宫居住三年,忏悔过错,并责备自己,开始归于善道,于是伊尹把帝太甲迎接回来,并将国政交还给他。帝太甲修整德行,诸侯都来归顺殷商,百姓因此安宁。伊尹赞赏他的做法,于是作《太甲训》三篇,用来褒扬帝太甲,尊他为太宗。

\begin{yuanwen}
太宗崩,子沃丁立。帝沃丁之时,伊尹卒。既葬伊尹于亳,咎单遂训伊尹事,作《沃丁》。
\end{yuanwen}

太宗去世以后,他的儿子沃丁继位。帝沃丁在位的时候,伊尹去世。伊尹被安葬在亳以后,咎单于是想用伊尹的事迹训诫后人,作《沃丁》。

\begin{yuanwen}
沃丁崩,弟太庚立,是为帝太庚。帝太庚崩,子帝小甲立。帝小甲崩,弟雍己立,是为帝雍己。殷道衰,诸侯或不至。
\end{yuanwen}

沃丁去世以后,他的弟弟太庚继位,他就是帝太庚。帝太庚去世以后,他的儿子帝小甲继位。帝小甲去世以后,他的弟弟雍己继位,他就是帝雍己。这时殷商国势衰败,有的诸侯不来朝见了。

\begin{yuanwen}
帝雍己崩,弟太戊立,是为帝太戊。帝太戊立伊陟为相。亳有祥桑谷(榖)共生于朝,一暮大拱。帝太戊惧,问伊陟。伊陟曰:“臣闻妖不胜德,帝之政其有阙\footnote{同“缺”,缺点、过失。}与?帝其修德。”

太戊从之,而祥桑枯死而去。伊陟赞言于巫咸。巫咸治王家有成,作《咸艾》,作《太戊》。帝太戊赞伊陟于庙,言弗臣,伊陟让,作《原命》。殷复兴,诸侯归之,故称中宗。
\end{yuanwen}

帝雍己去世以后,他的弟弟太戊继位,这就是帝太戊。帝太戊任命伊陟为相。亳都的朝堂有桑树和榖树生长在一起的怪异现象,一个晚上就长成双手合抱的围度。帝太戊很害怕,就询问伊陟。伊陟说:“我听说怪异的事物无法战胜美好的德行,难道是您的政令还有什么缺点吗?您应该努力修整德行。”

太戊听从了他的建议,因而那棵怪桑树就枯死消失了。伊陟对巫咸赞美并讲述了这件事。巫咸治理王室也非常有成绩,于是作《咸艾》,作《太戊》。帝太戊在宗庙称赞伊陟,言辞中不把他视为臣子,伊陟辞让,作《原命》。殷商重新振兴起来,诸侯都来归顺,因此太戊被尊称为中宗。

\begin{yuanwen}
中宗崩,子帝中丁立。帝中丁迁于隞\footnote{áo}。河亶\footnote{dàn}甲居相。祖乙迁于邢。帝中丁崩,弟外壬立,是为帝外壬。仲丁\footnote{即前文中的“中丁”。仲,甲骨文、金文作“中”。}《书》阙不具。帝外壬崩,弟河亶甲立,是为帝河亶甲。河亶甲时,殷复衰。
\end{yuanwen}

中宗去世以后,他的儿子帝中丁继位。帝中丁将都城迁到隞。河亶甲又迁都到相。祖丁迁都到邢。帝中丁去世以后,他的弟弟外壬继位,这就是帝外壬。仲丁的记载在《尚书》中残缺不全。帝外壬去世以后,他的弟弟河亶甲继位,这就是帝河亶甲。河亶甲在位期间,殷商再次衰败。

\begin{yuanwen}
河亶甲崩,子帝祖乙立。帝祖乙立,殷复兴。巫贤任职。

祖乙崩,子帝祖辛立。帝祖辛崩,弟沃甲立,是为帝沃甲。帝沃甲崩,立沃甲兄祖辛之子祖丁,是为帝祖丁。帝祖丁崩,立弟沃甲之子南庚,是为帝南庚。帝南庚崩,立帝祖丁之子阳甲,是为帝阳甲。帝阳甲之时,殷衰。
\end{yuanwen}

河亶甲去世以后,他的儿子帝祖乙继位。帝祖乙在位时,殷商又兴盛起来,巫贤当权。

祖乙去世以后,他的儿子帝祖辛继位。帝祖辛去世以后,他的弟弟沃甲继位,这就是帝沃甲。帝沃甲去世以后,立沃甲的哥哥祖辛的儿子祖丁为帝,这就是帝祖丁。帝祖丁去世以后,立沃甲的儿子南庚为帝,这就是帝南庚。帝南庚去世以后,立帝祖丁的儿子阳甲为帝,这就是帝阳甲。帝阳甲在位期间,殷商又衰落了。

\begin{yuanwen}
自中丁以来,废適\footnote{適:通“嫡”。}而更立诸弟子,弟子或争相代立,比九世乱,于是诸侯莫朝。
\end{yuanwen}

自中丁以来,经常废黜嫡子而改立兄弟及其儿子,有的时候兄弟及其儿子相互争夺帝位,连续九世都非常混乱,于是诸侯都不来朝见。

\begin{yuanwen}
帝阳甲崩,弟盘庚立,是为帝盘庚。帝盘庚之时,殷已都河北,盘庚渡河南,复居成汤之故居,乃五迁,无定处。殷民咨\footnote{嗟叹。}胥\footnote{全部,与“皆”意思相同。}皆怨,不欲徙。盘庚乃告谕诸侯大臣曰:“昔高后成汤与尔之先祖俱定天下,法则可修。舍而弗勉,何以成德!”乃遂涉河南,治亳\footnote{《尚书》作亳殷,即今河南安阳殷墟,现在位于黄河以北,而成汤定都之亳(今河南省商丘市)在黄河以南。从武乙时“殷复去亳,徙河北”的记载来看,司马迁将盘庚之亳误认为成汤之亳了。},行汤之政,然后百姓由宁,殷道复兴。诸侯来朝,以其遵成汤之德也。
\end{yuanwen}

帝阳甲去世以后,他的弟弟盘庚继位,这就是帝盘庚。帝盘庚在位的时候,殷商已经定都黄河以北,盘庚渡过黄河到南岸,重新在成汤的故居定都,至此一共迁都五次,没有固定的地方。殷商民众都叹息怨恨,不想再迁徙。盘庚于是告谕诸侯和大臣说:“从前伟大的先帝成汤和你们的先祖一起安定天下,他的法则应该遵守。舍弃他的法则而不去努力,怎么能够成就功德呢!”于是他渡过黄河来到南岸,修治亳,实行成汤时的政令,此后百姓得以安宁,殷商的国势重新振兴。诸侯都来朝见,因为盘庚遵守成汤时的德政。

\begin{yuanwen}
帝盘庚崩,弟小辛立,是为帝小辛。帝小辛立,殷复衰。百姓思盘庚,乃作《盘庚》三篇。帝小辛崩,弟小乙立,是为帝小乙。
\end{yuanwen}

帝盘庚去世以后,他的弟弟小辛继位,这就是帝小辛。帝小辛在位时,殷商又衰落了。百姓都思念盘庚,于是作《盘庚》三篇。帝小辛去世以后,他的弟弟小乙继位,这就是帝小乙。

\begin{yuanwen}
帝小乙崩,子帝武丁立。帝武丁即位,思复兴殷,而未得其佐。三年不言,政事决定于冢宰,以观国风。武丁夜梦得圣人,名曰说\footnote{yuè}。以梦所见视群臣百吏,皆非也。于是乃使百工营求之野,得说于傅险中。是时说为胥靡\footnote{服劳役的奴隶或刑徒。},筑于傅险。见于武丁,武丁曰是也。得而与之语,果圣人,举以为相,殷国大治。故遂以傅险姓之,号曰傅说。
\end{yuanwen}

帝小乙去世以后,他的儿子帝武丁继位。帝武丁在位时,想要重新振兴殷商,却没有找到合适的助手。他三年没有发布政令,政事都由冢宰决断,他借此机会观察国内的情况。武丁在一天夜里梦见一位圣人,名叫说。他根据梦中看到的样子观察群臣百官,都不是那个人。于是他派百官到民间去寻找,终于在傅险找到了说。当时说是个服劳役的奴隶,在傅险筑城。说被带去见武丁,武丁说就是这个人。武丁和说谈论,发现他果然是个圣人,就任用他为相,殷商于是太平安定。于是就以傅险为他的姓氏,称为傅说。

\begin{yuanwen}
帝武丁祭成汤,明日,有飞雉登鼎耳而呴\footnote{同“雊”,野鸡鸣叫。},武丁惧。祖己曰:“王勿忧,先修政事。”

祖己乃训王曰:“唯天监下典厥义,降年有永有不永,非天夭民,中绝其命。民有不若德,不听罪,天既附命正厥德,乃曰其奈何。鸣呼!王嗣敬民,罔非天继,常祀毋礼于弃道。”

武丁修政行德,天下咸驩,殷道复兴。
\end{yuanwen}

帝武丁祭祀成汤,第二天,有雉鸡飞来站在鼎耳上鸣叫,武丁非常害怕。祖己说:“大王不要担忧,还是先修整好政务。”

祖己于是就对王训诫说:“上天考察天下民众主要看他们的行为是否符合道义,上天赐予的寿命有的长有的不长,并不是上天想要让某些人短命,而是他们自己中途断绝了自己的命数。有的人不遵守道德,不服从判罚,上天已经赐予命数来纠正他们的德行,这个时候才说怎么办。唉!大王继承王位,恭敬对待民众,他们无不是上天的子孙,要经常祭祀,不要失礼而废弃天道。”

武丁修整政事,施行仁德,天下人都很欢欣,殷商的国势再次兴盛起来。

杨慎:「武丁在民间,已知说之贤矣。一旦欲举而加之臣民之上,人未必帖然以听也,故征之于梦焉。盖商俗质而信鬼,因民之所信而导之,是圣人所以成务之几也。」

\begin{yuanwen}
帝武丁崩,子帝祖庚立。祖己嘉武丁之以祥雉为德,立其庙为高宗,遂作《高宗肜\footnote{róng}日》及《训》。
\end{yuanwen}

帝武丁去世以后,他的儿子帝祖庚继位。祖己赞赏武丁因雉鸡飞来的怪异现象而推行德政,为他建立宗庙,称其为高宗,于是作《高宗肜日》和《高宗之训》。

\begin{yuanwen}
帝祖庚崩,弟祖甲立,是为帝甲。帝甲淫乱,殷复衰。
\end{yuanwen}

帝祖庚去世以后,他的弟弟祖甲继位,这就是帝甲。帝甲荒淫悖乱,殷商再次衰落。

\begin{yuanwen}
帝甲崩,子帝廪辛立。帝廪辛崩,弟庚丁立,是为帝庚丁。帝庚丁崩,子帝武乙立。殷复去亳,徙河北。
\end{yuanwen}

帝甲去世以后,他的儿子帝廪辛继位。帝廪辛去世以后,他的弟弟庚丁继位,这就是帝庚丁。帝庚丁去世以后,他的儿子帝武乙继位。殷商又离开亳,迁徙到黄河以北。

\begin{yuanwen}
帝武乙无道,为偶人,谓之天神。与之博,令人为行。天神不胜,乃僇辱之。为革囊,盛血,昂(卬)\footnote{同“仰”。}而射之,命曰“射天”。

武乙猎于河渭之闲,暴雷,武乙震死。子帝太丁立。帝太丁崩,子帝乙立。帝乙立,殷益衰。
\end{yuanwen}

帝武乙统治无道,制作木偶人,称其为天神。他和这个木偶人赌博,让别人代替他玩。天神不能赢,就侮辱它。他又制作皮囊,在里面盛满血,仰头朝它射箭,称为“射天”。

武乙在黄河与渭水之间狩猎,天上突然打雷,将武乙震死了。他的儿子帝太丁继位。帝太丁去世以后,他的儿子帝乙继位。帝乙在位时,殷商的国势更加衰落了。

\begin{yuanwen}
帝乙长子曰微子启,启母贱,不得嗣。少子辛,辛母正后,辛为嗣。帝乙崩,子辛立,是为帝辛,天下谓之纣。
\end{yuanwen}

帝乙的长子叫微子启,启的母亲地位低贱,因此他没有资格继承王位。帝乙的小儿子叫辛,辛的母亲是正后,因此辛得以继承王位。帝乙去世以后,他的儿子辛继位,这就是帝辛,天下人称他为纣。

\begin{yuanwen}
帝纣资辨\footnote{同“辩”,有口才。}捷疾,闻见甚敏;材力过人,手格猛兽;知足以距\footnote{同“拒”,拒绝。}谏,言足以饰非;矜人臣以能,高天下以声,以为皆出己之下。好酒淫乐,嬖\footnote{bì}于妇人。爱妲己,妲己之言是从。于是使师涓作新淫声,北里之舞,靡靡之乐。厚赋税以实鹿台之钱,而盈钜桥之粟。益收狗马奇物,充仞\footnote{满。}宫室。益广沙丘苑台,多取野兽蜚鸟\footnote{即飞鸟,“蜚”同“飞”。}置其中。慢于鬼神。大聚(冣)乐戏于沙丘,以酒为池,县(悬)肉为林,使男女倮\footnote{同“裸”。}相逐其间,为长夜之饮。
\end{yuanwen}

帝纣能言善辩,思维敏捷,耳聪目明;他的身高力大,超过常人,可以徒手和猛兽搏斗;他的智慧足以拒绝臣下的劝谏,口才足以粉饰自己的错误;他用才能向大臣夸耀,用声威压制天下,认为所有人都不如自己。他喜欢喝酒,骄奢淫逸,宠爱女人。他尤其宠爱妲己,妲己说的话他都听从。于是他让乐师涓创作放荡的乐曲,北里的舞蹈,淫靡的音乐。他增加赋税来充实聚集在鹿台的钱财,又在钜桥的粮仓装满了粮食。他广泛地搜寻狗马和奇珍异宝,使其充盈宫室。他还扩建沙丘的苑囿和亭台,捕捉很多野兽飞禽投放在里面。他对鬼神的祭祀很怠慢。他在沙丘聚集很多歌舞艺人,蓄酒为池塘,悬肉为树林,让男女裸体相互追逐,通宵达旦地饮酒取乐。

\begin{yuanwen}
百姓怨望而诸侯有畔者,于是纣乃重刑辟,有炮格之法。以西伯昌、九侯、鄂侯为三公。九侯有好女,入之纣。九侯女不憙\footnote{同“喜”。}淫,纣怒,杀之,而醢\footnote{hǎi,将人剁成肉酱的酷刑。}九侯。鄂侯争之强,辨之疾,并脯鄂侯。西伯昌闻之,窃叹。崇侯虎知之,以告纣,纣囚西伯羑\footnote{yǒu}里。西伯之臣闳\footnote{hóng}夭之徒,求美女、奇物、善马以献纣,纣乃赦西伯。西伯出而献洛西之地,以请除炮格之刑。纣乃许之,赐弓矢斧钺,使得征伐,为西伯。而用费中为政。费中善谀,好利,殷人弗亲。纣又用恶来。恶来善毁谗,诸侯以此益疏。
\end{yuanwen}

百姓怨恨的同时,诸侯也有反叛的,于是纣加重刑法,创制炮烙的酷刑。纣任命西伯昌、九侯、鄂侯为三公。九侯有个女儿很漂亮,就把她献给了纣。九侯的女儿不喜欢淫乱,纣非常愤怒,就将其杀死,还把九侯剁成肉酱。鄂侯坚决劝谏,激烈争辩,纣把鄂侯杀死制成了肉干。西伯昌听说以后,私下里叹息。崇侯虎知道了这件事,向纣报告,纣就把西伯囚禁在羑里。西伯的大臣闳夭等人,搜求美女、珍宝、良马来献给纣,纣因此赦免了西伯。西伯出狱后进献洛水以西的一块土地,请求纣王废除炮烙的刑罚。纣答应了,还赏赐给他弓箭和斧钺,让他拥有征伐的大权,任命他为西方诸侯的首领。纣任用费中主持政务。费中善于谄媚,贪图财利,殷商的民众都不亲睦。纣又任用恶来。恶来喜欢诋毁别人,诸侯因此更加疏远纣。

\begin{yuanwen}
西伯归,乃阴修德行善,诸侯多叛纣而往归西伯。西伯滋大,纣由是稍失权重。王子比干谏,弗听。商容贤者,百姓爱之,纣废之。及西伯伐饥国,灭之,纣之臣祖伊闻之而咎周,恐,奔告纣曰:“天既讫我殷命,假人\footnote{《尚书》作“格人”,即至德之人。}元龟,无敢知吉,非先王不相我后人,维王淫虐用自绝,故天弃我,不有安食,不虞知天性,不迪率典。今我民罔不欲丧,曰‘天曷不降威,大命胡不至’?今王其奈何?”

纣曰:“我生不有命在天乎!”

祖伊反,曰:“纣不可谏矣。”

西伯既卒,周武王之东伐,至盟津,诸侯叛殷会周者八百。诸侯皆曰:“纣可伐矣。”

武王曰:“尔未知天命。”乃复归。
\end{yuanwen}

西伯回国以后,暗中修德政,行善事,有很多诸侯叛离纣王归附西伯。西伯的势力渐渐壮大,纣的权利渐渐丧失。王子比干劝谏,纣不听从。商容是一个贤德的人,百姓都喜爱他,纣却把他罢免了。等到西伯讨伐饥国时,将其灭掉,纣的大臣祖伊听说以后就怨恨周国,又很害怕,跑去告诉纣说:“上天已经断绝了我殷商的国运,至德之人用大龟占卜,都不再显示吉兆,这不是先王不帮助我们后人,只是因为大王荒淫暴虐,自绝于天,因此上天抛弃了我们,让我们无法安稳地生活,不预先知道天命,就不能按照常法行事。现在我们的人民没有不希望我们灭亡的,他们都说:‘上天为什么还不降下威严,天命为什么还不到来?’现在大王准备怎么办呢?”

纣说:“我生下来不就是有天命护佑吗!”

祖伊回去,说:“纣已经不能接受劝谏了。”

西伯去世以后,周武王向东方征讨,来到盟津,诸侯背叛殷商与周军结盟的有八百个。诸侯都说:“可以讨伐纣了。”

武王说:“你们不知道天命。”于是他回去了。

\begin{yuanwen}
纣愈淫乱不止。微子数谏不听,乃与大师、少师谋,遂去。比干曰:“为人臣者,不得不以死争\footnote{同“诤”,谏诤。}。”乃强谏纣。

纣怒曰:“吾闻圣人心有七窍。”剖比干,观其心。

箕子惧,乃详\footnote{通“佯”,假装。}狂为奴,纣又囚之。殷之大师、少师乃持其祭乐器奔周。周武王于是遂率诸侯伐纣。纣亦发兵距之牧野。甲子日,纣兵败。纣走,入登鹿台,衣其宝玉衣,赴火而死。周武王遂斩纣头,县\footnote{xuán,同“悬”。}之[大]白旗。杀妲己。释箕子之囚,封\footnote{培土。}比干之墓,表商容之闾。封纣子武庚、禄父,以续殷祀,令修行盘庚之政。殷民大说。于是周武王为天子。其后世贬帝号,号为王。而封殷后为诸侯,属\footnote{归属。}周。
\end{yuanwen}

纣的淫乱越来越严重了。微子多次劝谏,纣都不听,于是微子和太师、少师商量以后,终于离开纣。比干说:“做人臣子的,不能不用性命来劝谏。”于是他极力劝谏。

纣生气地说:“我听说圣人的心有七窍。”他下令剖开比干的胸膛,看他的心。

箕子非常恐惧,假装发疯做了奴隶,纣又把他囚禁起来。殷商的太师、少师就带着祭器和乐器逃到周国去了。周武王于是率领诸侯带兵前去讨伐纣。纣也发兵在牧野抵抗周军。甲子日,纣的士兵大败。纣逃走,登上鹿台,穿上缀有宝玉的衣服,跳进火中自焚而死。周武王于是斩下纣的头,挂在白旗上。他下令杀死妲己。他释放了箕子,在比干的坟墓培土,在商容的家门进行表彰。他又分封纣的儿子武庚禄父,让他来继续奉祀殷商的先人,命令他遵行盘庚时的政令。殷商民众非常高兴。于是周武王成为天子。因为殷商后人贬降帝号,称为王。周武王封殷商的后人为诸侯,隶属于周。

\begin{yuanwen}
周武王崩,武庚与管叔、蔡叔作乱,成王命周公诛之,而立微子于宋,以续殷后焉。
\end{yuanwen}

周武王去世以后,武庚和管叔、蔡叔作乱,成王命令周公将其平定,立微子为宋君,来延续殷商的后代。

\begin{yuanwen}
太史公曰:余以《颂》次契之事,自成汤以来,采于《书》、《诗》。契为子姓,其后分封,以国为姓,有殷氏、来氏、宋氏、空桐氏、稚氏、北殷氏、目夷氏。孔子曰,殷路车为善,而色尚白。
\end{yuanwen}

太史公说:我根据《商颂》来编次契的事迹,自成汤以下的史料,都出自《尚书》《诗经》。契为子姓,他的后代受分封,以国名为姓,有殷氏、来氏、宋氏、空桐氏、稚氏、北殷氏、目夷氏。孔子说,殷人的车子很好,并且崇尚白色。

\part{卷四}
\chapter{周本纪第四}

《周本纪》是以周朝帝王为纲领的整个周民族与周王朝的编年史。

本篇记述了周朝的历史。周人先祖的事迹,多取材于《尚书》《诗经》,始祖后稷为农官,可知周人重视农业,对后世历代的农本思想影响很大。周武王定都镐京,史称西周,以分封、宗法、礼乐等制度维系天子的共主地位。周平王迁都洛邑,史称东周。春秋以后,王室衰微,礼崩乐坏,诸侯混战不已。到战国时期,周王室分裂为两个小国,先后被秦国所灭。

周民族的发展史经历夏朝、殷朝共千余年,至商末强大起来,雄踞西方。至周文王,吞并了四周小国,为日后武王灭商奠定了基础。武王继位后,在姜太公、周公、召公等一大批贤才得辅佐下,于公元前1046年率领许多同盟力量共同伐纣,灭亡了殷朝,建立了周朝。

关于西周史,《诗经》、《尚书》、《逸周书》、《国语》等文献中有比较丰富的史料可供依据,而战国以来还没有一种比较系统的西周历史,所以司马迁便对西周部分作了比较详细的铺陈,以一个“德”字贯穿西周史的始终,其中蕴含着令人警省的教训。春秋时期的周史主要依据《春秋》、《左传》而作。当时政治舞台的主宰者已由天子转为诸侯中的霸主,太史公在撰写这一时期的周史时突出了王道衰微的内容。战国史料被秦始皇焚烧殆尽,供司马迁取材的只有《战国策》与诸子中涉及的一些材料,而这些材料的真实性也成问题,再加上战国后期已经小得极其可怜的周国又分裂成东西两部分,以至于司马迁连这两个小国诸侯的名字与世系都无法说清了。

\begin{yuanwen}
周后稷,名弃。其母有邰氏女,曰姜原。姜原为帝喾元妃。姜原出野,见巨人迹,心忻\footnote{xīn}然说\footnote{同“悦”,喜悦,高兴。},欲践之,践之而身动如孕者。居期而生子,以为不祥,弃之隘巷,马牛过者皆辟不践;徙置之林中,適会山林多人,迁之;而弃渠中冰上,飞鸟以其翼覆荐之。姜原以为神,遂收养长之。初欲弃之,因名曰弃。
\end{yuanwen}

周朝的始祖后稷,名叫弃。他的母亲是有邰氏的女子,名叫姜原。姜原是帝喾的正妃。姜原外出到郊野,看见一个巨人的脚印,心里很高兴,就想去踩它,踩上去就感觉体内有胎动,就像怀孕一样。过了一年,她生下一个孩子,认为不吉利,就把孩子扔到小巷中,路过的马牛却绕过去不踩他;她把孩子扔在山林中,正赶上山林中人很多,又转到别处;她将孩子扔在水渠的冰面上,有飞鸟用翅膀盖在他身上,垫在他身下。姜原认为这个孩子很神奇,就把他抱回来抚养长大。由于最开始想遗弃这个孩子,因此给他取名叫弃。

\begin{yuanwen}
弃为儿时,屹如巨人之志。其游戏,好种树麻、菽,麻、菽美。及为成人,遂好耕农,相地之宜,宜穀者稼穑焉,民皆法则之。帝尧闻之,举弃为农师,天下得其利,有功。帝舜曰:“弃,黎民始饥,尔后稷播时百穀。”封弃于邰,号曰后稷,别姓姬氏。后稷之兴,在陶唐、虞、夏之际,皆有令德。
\end{yuanwen}

弃还是个孩子的时候,就像伟大人物一样有高远的志向。他做游戏的时候,喜欢种植麻、豆,他种的麻、豆都长得很茂盛。等他长大以后,就喜好耕作,观察土地的情况,选择合适的土地种植庄稼,百姓都仿效他。帝尧听说以后,就任用弃为农师,天下人都因此获利,他立下了功劳。帝舜说:“弃,百姓最初忍受饥饿,你担任后稷来播种各种谷物。”于是他被封在邰,号称后稷,另外以姬为姓。后稷的兴起,在唐尧、虞舜、夏禹的时代,这一家族都有美好的德行。

\begin{yuanwen}
后稷卒,子不窋\footnote{zhú}立。不窋末年,夏后氏政衰,去稷不务,不窋以失其官而饹(奔)戎狄之间。不窋卒,子鞠立。鞠卒,子公刘立。公刘虽在戎狄之间,复脩后稷之业,务耕种,行地宜,自漆、沮度渭,取材用,行者有资,居者有畜积,民赖其庆。百姓怀之,多徙而保归焉。周道之兴自此始,故诗人歌乐思其德。公刘卒,子庆节立,国于豳\footnote{bīn}。
\end{yuanwen}

后稷去世以后,他的儿子不窋继位。不窋晚年,夏后氏的国政衰败,舍弃后稷之官而不再重视农业,不窋因为失去官职而逃往戎狄地区。不窋去世以后,他的儿子鞠继位。鞠去世以后,他的儿子公刘继位。公刘虽然在戎狄地区生活,但是他重新经营后稷的基业,致力于耕种,根据土地的实际情况决定种植的农作物,从漆水、沮水渡过渭水,砍伐木材而使用,让出行的人有资财,在家的人有积蓄,人民依靠他而生活快乐。百姓感激他,很多人迁徙来归顺他。周朝治国之道的兴盛就是从这个时候开始的,所以诗人创作诗歌和乐曲来追思他的美德。公刘去世以后,他的儿子庆节继位,在豳建国。

\begin{yuanwen}
庆节卒,子皇仆立。皇仆卒,子差弗立。差弗卒,子毁隃\footnote{yú}立。毁隃卒,子公非立。公非卒,子高圉立。高圉卒,子亚圉立。亚圉卒,子公叔祖类立。公叔祖类卒,子古公亶父立。古公亶父复脩后稷、公刘之业,积德行义,国人皆戴之。薰育\footnote{也作“荤粥”,古代北方民族,秦汉以后称“匈奴”。}戎狄攻之,欲得财物,予之。已复攻,欲得地与民。民皆怒,欲战。古公曰:“有民立君,将以利之。今戎狄所为攻战,以吾地与民。民之在我,与其在彼,何异?民欲以我故战,杀人父子而君之,予不忍为。”乃与私属遂去豳,度漆、沮,逾梁山,止于岐下。

豳人举国扶老携弱,尽复归古公于岐下。及他旁国闻古公仁,亦多归之。于是古公乃贬戎狄之俗,而营筑城郭室屋,而邑别居之。作五官\footnote{指司徒、司空、司马、司士、司寇。}有司。民皆歌乐之,颂其德。
\end{yuanwen}

庆节去世以后,他的儿子皇仆继位,皇仆去世以后,他的儿子差弗继位。差弗去世以后,他的儿子毁隃继位。毁隃去世以后,他的儿子公非继位。公非去世以后,他的儿子高圉继位。高圉去世以后,他的儿子亚圉继位。亚圉去世以后,他的儿子公叔祖类继位。公叔祖类去世以后,他的儿子古公亶父继位。古公亶父重新经营后稷、公刘的基业,积累德行,广施仁义,国人都拥戴他。薰育等戎狄部族前来攻打,想得到财物,古公就把财物给了他们。薰育不久又来攻打,想得到土地和人民。人民都非常愤怒,想要迎战。古公说:“民众拥立君长,是要让他们获利。现在戎狄前来攻打的原因,是要得到土地和人民。人民被我统治,和被他们统治,有什么区别呢?人民是因为我的缘故才去打仗,靠牺牲别人的父亲和孩子来换取君长之位,我不忍心。”于是古公和他的部众离开豳,渡过漆水、沮水,翻过梁山,在岐山脚下定居。

豳人一起扶老携幼,全部来到岐山脚下,重新归附古公。附近国家的人民听说古公的仁慈,也有很多前来归顺的。于是古公摒弃了戎狄的习俗,建造城郭和房屋,并且设置邑落分别居住。设立五官的衙署。人民都为他创作诗歌和乐曲,赞颂他的美德。

\begin{yuanwen}
古公有长子曰太伯,次曰虞仲。太姜生少子季历,季历娶太任,皆贤妇人,生昌,有圣瑞。古公曰:“我世当有兴者,其在昌乎?”

长子太伯、虞仲知古公欲立季历以传昌,乃二人亡如荆蛮,文身断发,以让季历。
\end{yuanwen}

古公的大儿子叫太伯,二儿子叫虞仲。太姜生的小儿子叫季历,季历娶太任,太姜和太任都是贤惠的妻子,太任生下昌,有圣明的征兆。古公说:“我的后代一定会有成大事的人,也许就是昌吧?”

大儿子太伯、二儿子虞仲知道古公想要立季历来传位给昌,于是两人就逃亡到荆蛮地区,身体刺上花纹,剪断头发,就是为了让位给季历。

朱熹:「司马迁云,文王之治岐,耕者九一,仕者世禄,皆是降阴德,以分纣之天下,不知文王之心诚于为民者若此!」陈仁锡:「为善是圣人性内事,若显为善,则开积善累德之谗,故『阴行善』三字,太史公写出至圣一段明夷艰贞苦心处,不是后来阴谋。」

\begin{yuanwen}
古公卒,季历立,是为公季。公季脩古公遗道,笃于行义,诸侯顺之。
\end{yuanwen}

古公去世以后,季历继位,这就是公季。公季修持古公留下的治国之道,坚定地推行仁义,诸侯都归顺他。

\begin{yuanwen}
公季卒,子昌立,是为西伯。西伯曰文王,遵后稷、公刘之业,则古公、公季之法,笃仁,敬老,慈少。礼下贤者,日中不暇食以待士,士以此多归之。伯夷、叔齐在孤竹,闻西伯善养老,盍往归之。太颠、闳夭、散宜生、鬻\footnote{yù}子、辛甲大夫之徒皆往归之。
\end{yuanwen}

公季去世以后,他的儿子昌继位,这就是西伯。西伯被称为文王,继承后稷、公刘的事业,遵循古公、公季的法则,笃行仁义,尊敬老人,慈爱幼小。他对贤能的人以礼相待,每天接待士人,到中午还来不及吃饭,很多士人因此来归顺他。伯夷、叔齐在孤竹国,听说西伯尊敬赡养老人,就一同前往归顺。太颠、闳夭、散宜生、鬻子、辛甲大夫等人也都归顺了他。

\begin{yuanwen}
崇侯虎谮\footnote{zèn}西伯于殷纣曰:“西伯积善累德,诸侯皆乡\footnote{xiàng,同“向”,向往,归顺。}之,将不利于帝。”

帝纣乃囚西伯于羑里。闳夭之徒患之。乃求有莘氏美女,骊戎之文马,有熊九驷\footnote{驾四匹马的车子。},他奇怪物,因殷嬖臣费仲而献之纣。

纣大说,曰:“此一物足以释西伯,况其多乎!”乃赦西伯,赐之弓矢斧钺,使西伯得征伐。曰:“谮西伯者,崇侯虎也。”

西伯乃献洛西之地,以请纣去砲烙之刑。纣许之。
\end{yuanwen}

崇侯虎在殷纣的面前陷害西伯说:“西伯行善积德,诸侯都归顺他,将会对您不利。”

帝纣于是把西伯囚禁在羑里。闳夭等人都非常担心,就去寻求有莘氏的美女,骊戎的彩色骏马,有熊的九辆马车,以及其他奇珍异宝,通过殷商的宠臣费仲而献给纣。

纣很高兴,说:“这些东西中的任何一件都足以让我释放西伯,何况还有这么多呢!”于是他赦免了西伯,并赏赐给他弓箭和斧钺,让他拥有征伐的大权。纣说:“陷害西伯的人,是崇侯虎。”

西伯于是献出洛水以西的土地,并请求纣废除炮烙的刑罚。纣应允了。

\begin{yuanwen}
西伯阴行善,诸侯皆来决平。于是虞、芮\footnote{ruì}之人有狱不能决,乃如周。入界,耕者皆让畔,民俗皆让长。虞、芮之人未见西伯,皆惭,相谓曰:“吾所争,周人所耻,何往为,祇取辱耳。”遂还,俱让而去。

诸侯闻之,曰“西伯盖受命之君”。
\end{yuanwen}

西伯暗中做好事,诸侯之间的矛盾都请他来裁决。当时虞国、芮国有人发生纠纷无法解决,他们就来到周国。当他们进入周国境内的时候,看到耕种的人彼此谦让田界,民间都把谦让长者当成风俗。虞国、芮国的人还没见到西伯,就已经觉得惭愧,相互说:“我们所争执的,正是周人所鄙视的,我们还去干什么呢?只是自取羞辱罢了。”于是他们返回,彼此谦让而去。

诸侯听说以后,都说“西伯大概是接受天命的君主”。

\begin{yuanwen}
明年,伐犬戎。

明年,伐密须。

明年,败耆国。殷之祖伊闻之,惧,以告帝纣。

纣曰:“不有天命乎?是何能为!”

明年,伐邘。

明年,伐崇侯虎。而作丰邑,自岐下而徙都丰。

明年,西伯崩,太子发立,是为武王。
\end{yuanwen}

第二年,西伯讨伐犬戎。

又一年,西伯讨伐密须。

再一年,西伯打败耆国。殷商的祖伊听说以后,非常恐惧,他把这些事情都告诉了帝纣。

纣说:“不是有天命帮助我吗?他又能怎样呢!”

第二年,西伯讨伐邘国。

下一年,西伯讨伐崇侯虎。同时他建造了丰邑,从岐山脚下迁都到丰邑。

又过了一年,西伯去世,太子发即位,这就是武王。

\begin{yuanwen}
西伯盖即位五十年。其囚羑里,盖益《易》之八卦为六十四卦。诗人道西伯,盖受命之年称王而断虞、芮之讼。后十年而崩,谥为文王\footnote{《逸周书·谥法解》提到周公制谥,而近代以来,王国维等学者据金文考释认定,周文王、周武王是其生前的称号,并非死后的谥号,谥号制度是在西周中后期产生,春秋时期通行的。}。改法度,制正朔矣。追尊古公为太王,公季为王季。盖王瑞自太王兴。
\end{yuanwen}

西伯在位大概五十年。当他被囚禁在羑里的时候,把《易》的八卦增为六十四卦。诗人称赞西伯,大概是在西伯接受天命称王的那一年裁决了虞、芮两国的纷争。十年以后,西伯去世,谥号是文王。他修改了法令,制定了历法。他追尊古公为太王,公季为王季。大概称王的祥瑞是从太王开始的。

\begin{yuanwen}
武王即位,太公望为师\footnote{text},周公旦为辅,召\footnote{shào}公、毕公之徒左右王\footnote{text},师修文王绪业\footnote{text}。
\end{yuanwen}

武王即位,太公望担任太师,周公旦是武王的助手,召公、毕公等人辅佐武王,继承文王的遗业。

\begin{yuanwen}
九年\footnote{text},武王上祭于毕\footnote{text}。东观兵\footnote{text},至于盟津。为文王木主,载以车中军\footnote{text}。武王自称太子发,言奉文王以伐\footnote{text},不敢自专。乃告司马、司徒、司空、诸节\footnote{text}:“齐栗\footnote{text},信哉!予无知,以先祖有德臣,小子受先功\footnote{text},毕立赏罚,以定其功。”

遂兴师。师尚父号曰:“总尔众庶\footnote{text},与尔舟楫,后至者斩。”

武王渡河,中流,白鱼跃入王舟中,武王俯取以祭。既渡,有火自上复于下\footnote{text},至于王屋\footnote{text},流为乌,其色赤,其声魄云。是时,诸侯不期而会盟津者八百。

诸侯皆曰:“纣可伐矣。”

武王曰:“女未知天命,未可也。”乃还师归。
\end{yuanwen}

九年,武王在毕地祭祀文王。然后他前往东方检阅部队,一直到达盟津。为文王制作了牌位,用车子装载,供奉在部队中。武王自称为太子发,表示是遵从文王的命令进行征伐,不敢独断专行。于是他告诫司马、司徒、司空、诸节:“大家要严肃恭敬,要诚实!我不聪明,因为先祖留下了有品德的大臣,我继承祖先的功业,才制定了奖赏和惩罚的措施,来巩固祖先的功业。”

终于发兵讨伐。太师尚父号令说:“集合你们众人,交给你们船只,迟到的人斩首!”

武王渡过黄河,船行到河中的时候,有一条白色的鱼跳进武王的船中,武王俯身拾起鱼用来祭祀。过河以后,有一团火从天而降,正落在武王的屋顶上,最后变成一只乌鸦,它的颜色是红的,降落时发出轰隆隆的响声。这时候,有八百诸侯不约而同地来到盟津会盟。

诸侯都说:“可以讨伐纣了。”

武王说:“你们不知道天命,现在还不行。”于是班师返回。

\begin{yuanwen}
居二年,闻纣昏乱暴虐滋甚,杀王子比干,囚箕子。太师疵、少师彊抱其乐器而奔周\footnote{text}。于是武王遍告诸侯曰:“殷有重罪,不可以不毕伐\footnote{text}。”

乃遵文王,遂率戎车三百乘\footnote{text},虎贲三千人\footnote{text},甲士四万五千人,以东伐纣。

十一年十二月戊午\footnote{text},师毕渡盟津,诸侯咸会。曰:“孳孳无怠!”

武王乃作《太誓\footnote{司马迁所摘录的是西汉时通行的今文《尚书》中的文字,后来失传,不同于今天所看到的伪古文《尚书》中的《泰誓》三篇。}》,告于众庶:“今殷王纣乃用其妇人之言,自绝于天,毁坏其三正\footnote{text},离逷\footnote{同“逖”,远。}其王父母弟\footnote{text},乃断弃其先祖之乐,乃为淫声,用变乱正声\footnote{text},怡说妇人。故今予发维共\footnote{通“恭”。}行天罚\footnote{text},勉哉夫子\footnote{text},不可再,不可三!”
\end{yuanwen}

过了两年,武王听说纣王昏庸暴虐比从前更厉害了,他杀死了王子比干,囚禁了箕子。太师疵、少师彊各自抱着乐器前来投奔周。这时武王才宣告诸侯说:“殷王已经犯下了大罪,不可再不去讨伐了。”

于是大家遵从文王的遗志,并率领三百辆战车,虎贲三千人,穿着甲胄的战士四万五千人,向东进攻,讨伐纣。

十一年十二月的戊午日,讨伐纣的军队全部渡过盟津,诸侯都到齐了。武王说:“一定要勤勤恳恳,不能懈怠呀!”

武王于是作《太誓》,并向所有人宣告:“现在殷王纣竟然听信妻妾的话,他这是自己断绝天命,违背了日、月、北斗的运行,疏远了自己的同胞兄弟,他废弃了先祖的音乐,并胡乱采用淫乱的音乐从而窜改了典雅的音乐,这样做就是为了让他的妻妾们高兴。所以今天我要恭敬地替上天执行惩罚。努力吧,战士们,这样的事情不会出现第二次,更不会有第三次!”

\begin{yuanwen}
二月甲子昧爽\footnote{text},武王朝至于商郊牧野\footnote{text},乃誓。武王左杖黄钺\footnote{text},右秉白旄\footnote{text},以麾\footnote{text}。曰:“远矣,西土之人!”

武王曰:“嗟!我有国冢君\footnote{text},司徒、司马、司空,亚旅、师氏\footnote{text},千夫长、百夫长\footnote{text},及庸、蜀、羌、髳\footnote{máo}、微、纑、彭、濮人,称尔戈,比尔干,立尔矛,予其誓\footnote{text}。”

王曰:“古人有言‘牝\footnote{pìn}鸡无晨。牝鸡之晨,惟家之索\footnote{尽,这里有破败的意思。}’。今殷王纣维妇人言是用,自弃其先祖肆祀不答\footnote{text};昬\footnote{通“泯”,蔑。}弃其家国\footnote{text},遗其王父母弟不用,乃维四方之多罪逋逃是崇是长\footnote{text},是信是使,俾暴虐于百姓,以奸轨\footnote{通“宄”,内乱。}于商国\footnote{text}。今予发维共行天之罚。今日之事,不过六步七步,乃止齐焉\footnote{text},夫子勉哉!不过于四伐五伐六伐七伐,乃止齐焉,勉哉夫子!尚桓桓\footnote{text},如虎如罴,如豺如离\footnote{同“螭”,传说中一种似龙的动物。},于商郊,不御克奔,以役西土,勉哉夫子!尔所不勉,其于尔身有戮。”

誓已,诸侯兵会者车四千乘,陈师牧野。
\end{yuanwen}

二月甲子日凌晨,武王很早就来到商的郊外牧野,然后誓师。武王左手拿着黄色的钺,右手举着用白色的旄牛尾做成装饰的旗,并以此来指挥。武王说:“一路辛苦了,西方来的将士们!”

武王接着说:“啊!我的友邦的君主们,司徒、司马、司空、亚旅、师氏、千夫长、百夫长,还有庸人、蜀人、羌人、髳人、微人、纑人、彭人、濮人,请高举你们的戈,排齐你们的盾,竖起你们的矛,让我们一起宣誓。”

武王说:“古人常说‘母鸡是不报晓的。假如母鸡报晓,就一定会倾家荡产’。现在殷王纣只要是他的爱妾所说的他就听,自己放弃了祖先的祭祀不再过问,抛弃国家大政,不任用自己的同祖兄弟,反而召集四方各国犯罪逃亡的人,推崇他们,看重他们,信任他们,任用他们,放纵他们在百姓中横行霸道,结果令商朝的政事被奸佞、狡诈的人控制。今天,我要恭敬地执行上天的惩罚。今天我们去讨伐纣,每前进六七步,一定要停顿下来整顿队伍,努力吧,战士们!每次刺击一定不要超过四下、五下、六下、七下,就必须停顿整齐队伍,努力吧,将士们!希望大家要勇猛威武,就像虎、罴、豺、螭那样,我们和商朝的军队在商的郊野作战,不要攻打前来投降的人,让他们听从我们这些从西方来的人的命令,努力吧,将士们!如果你们有谁不努力,我就会将他处死。”

誓师完毕,前来会合的诸侯军队,战车有四千辆,共同在牧野列阵。

\begin{yuanwen}
帝纣闻武王来,亦发兵七十万人距\footnote{同“拒”,抵抗。}武王。武王使师尚父与百夫致师\footnote{text},以大卒驰帝纣师\footnote{text}。纣师虽众,皆无战之心,心欲武王亟入。纣师皆倒兵以战,以开武王。武王驰之,纣兵皆崩畔\footnote{通“叛”,背叛。}纣。纣走,反入登于鹿台之上\footnote{text},蒙衣其殊(珠)玉,自燔于火而死。武王持大白旗以麾诸侯,诸侯毕拜武王,武王乃揖诸侯,诸侯毕从。武王至商国\footnote{text},商国百姓咸待于郊\footnote{text}。于是武王使群臣告语商百姓曰:“上天降休\footnote{text}!”

商人皆再拜稽首,武王亦答拜\footnote{text}。遂入,至纣死所。武王自射之,三发而后下车,以轻剑击之\footnote{text},以黄钺斩纣头,县\footnote{同“悬”,悬挂。}大白之旗\footnote{text}。已而至纣之嬖妾二女,二女皆经自杀。武王又射三发,击以剑,斩以玄钺,县其头小白之旗。武王已乃出复军。
\end{yuanwen}

帝纣听说武王前来讨伐自己,他也发兵七十万用来抵抗武王。武王派师尚父带领百名士兵前去挑战,而他自己带领大部分士兵攻击纣的军队。纣的军队虽然人数很多,但是士兵都不愿意作战,大家都盼着武王的军队赶快攻打进来。纣的士兵都掉转武器反而攻打纣,为武王的军队开路。武王坐着战车冲进城,纣的军队马上就溃散了,背叛了纣。纣逃跑,退回城中登上鹿台,他穿上自己的宝玉衣服,投火自焚而死。武王举着大白旗指挥诸侯,诸侯都拜见武王,武王也向诸侯们回礼,诸侯都听从武王的号令。武王进入商的都城,商朝的百姓都等在郊外迎接。于是武王命令群臣告诉商朝的百姓说:“上天赐福给你们!”

商朝的百姓都对着武王再拜稽首,武王也恭敬地还礼。然后,武王进入城中,到了纣王自焚的地方。武王亲自用箭射他,射了三箭以后下车,然后又用很轻的剑刺纣王的尸体,最后用黄色的钺砍下纣的头,挂到大白旗上。然后,武王又来到纣的两个宠妾那里,这时,那两个宠妾都已经上吊自杀了。武王又对着她们的尸体射了三箭,并用剑刺她们的尸体,最后用黑色的钺砍下她们的头,悬挂在小白旗上。做完这一切,武王出城,回到军中。

\begin{yuanwen}
其明日,除道,修社及商纣宫\footnote{text}。及期,百夫荷罕旗以先驱\footnote{text}。武王弟叔振铎奉陈常车\footnote{text},周公旦把大钺,毕公把小钺\footnote{text},以夹武王\footnote{text}。散宜生、太颠、闳夭皆执剑以卫武王。既入,立于社南大卒之左\footnote{text},左右毕从。毛叔郑奉明水\footnote{text},卫康叔封布兹\footnote{text},召公奭赞采\footnote{text},师尚父牵牲\footnote{text}。尹佚筴(策)\footnote{cè}祝曰:“殷之末孙季纣,殄废先王明德\footnote{text},侮蔑神祇不祀,昏暴商邑百姓,其章显闻于天皇上帝\footnote{text}。”

于是武王再拜稽首,曰:“膺更大命,革殷,受天明命\footnote{text}。”

武王又再拜稽首,乃出。
\end{yuanwen}

等到第二天,武王命人清除道路,修整社庙以及商纣的宫室。到了约好的日期,一百名士兵扛着“罕旗”走在最前面。武王的弟弟振铎护卫,并摆开了“常车”,周公旦手里握着大钺,毕公手里握着小钺,两人分别站在武王两边。散宜生、太颠、闳夭都手拿佩剑护卫着武王。武王进入城中,站到社庙南面的大卒的左边,左右的大臣都跟从他。毛叔郑手捧明水,卫康叔封铺好了草席,召公奭帮着拿好彩帛,师尚父牵着祭牲。尹佚大声朗读着竹简上的祭文:“殷朝末代子孙季纣,废弃了先王圣明的美德,侮辱蔑视神明,不去祭祀,对商朝的百姓实行暴政,他的这些做法在皇天上帝面前都已经表现得清清楚楚。”

于是武王再拜稽首,并说道:“我承受上天的大命,革除殷朝的弊政,是接受了上天所降下的光明之命。”

武王又一次再拜稽首,然后离开城中。

\begin{yuanwen}
封商纣子禄父殷之馀民。武王为殷初定未集,乃使其弟管叔鲜、蔡叔度相禄父治殷。已而命召公释箕子之囚。命毕公释百姓之囚,表商容之闾。命南宫括散鹿台之财,发钜桥之粟,以振\footnote{同“赈”,赈济、救济。}贫弱萌\footnote{通“氓”,外来的百姓,也泛指老百姓。}隶。命南宫括、史佚展九鼎保玉。命闳夭封比干之墓。命宗祝享祠于军。乃罢兵西归。行狩,记政事,作《武成》。封诸侯,班赐\footnote{分赐。班,同“颁”,颁发。}宗彝,作《分殷之器物》。武王追思先圣王,乃襃封神农之后于焦,黄帝之后于祝,帝尧之后于蓟,帝舜之后于陈,大禹之后于杞。于是封功臣谋士,而师尚父为首封。封尚父于营丘,曰齐。封弟周公旦于曲阜,曰鲁。封召公奭于燕。封弟叔鲜于管,弟叔度于蔡。馀各以次受封。
\end{yuanwen}

武王把殷商遗民都封给商纣的儿子禄父。武王因为天下刚刚平定下来,还没有安定,于是他派他的弟弟管叔鲜、蔡叔度帮助禄父治理殷商遗民。然后武王又命令召公将箕子释放出狱。命令毕公将囚禁在狱中的百姓释放,旌表了商容的闾门来表彰他的德行。命令南宫括将聚集在鹿台的钱财及巨桥的粮食都分发给百姓,用来赈济穷苦的人民。武王命令南宫括、史佚搬走殷人的九鼎和宝玉。命令闳夭为比干的坟墓培土修缮。命令宗祝在军队中祭祀阵亡将士的亡灵。然后武王撤兵又回到西方。武王巡视各个诸侯的领地,记录他们的治理事情,作《武成》。武王封诸侯,将殷商宗庙的祭器分别赏赐给他们,并作《分殷之器物》。武王追思怀念从前的圣王,并对神农的后代进行嘉奖,把他们封在焦,把黄帝的后代封在祝,把帝尧的后代封在蓟,把帝舜的后代封在陈,把大禹的后代封在杞。然后,武王又分封了功臣谋士,其中,师尚父是第一个接受封赏的。武王将尚父封在营丘,称为齐。封他的弟周公旦在曲阜,称为鲁。将召公奭封在燕。分封他的弟弟叔鲜在管,封他的弟弟叔度在蔡。其他的人也都依次受封。

\begin{yuanwen}
武王征九牧之君,登豳之阜,以望商邑。武王至于周,自夜不寐。周公旦即王所,曰:“曷为不寐?”

王曰:“告女:维天不飨\footnote{同“享”,鬼神享用祭品叫飨。}殷,自发未生于今六十年,麋鹿在牧,蜚鸿满野。天不享殷,乃今有成。维天建殷,其登名民三百六十夫,不显亦不宾\footnote{通“摈”,遗弃,排斥。}灭,以至今。我未定天保,何暇寐!”

王曰:“定天保,依天室,悉求夫恶,贬从殷王受。日夜劳来定我西土,我维显服,及德方明。自洛汭延于伊汭,居易毋固,其有夏之居。我南望三涂,北望岳鄙\footnote{指太行山一带。},顾詹有河,粤詹雒、伊,毋远天室。”

营周居于雒邑而后去。纵马于华山之阳,放牛于桃林之虚;偃干戈,振兵释旅,示天下不复用也。
\end{yuanwen}

武王召集九州的首领,登上了豳的高地,遥望着商的都城。武王再回到周,他在晚上总是无法入睡。周公旦来到武王的住处,问道:“为什么您无法入睡呢?”

武王说:“告诉你吧:上天不享受殷朝的祭品,从我还没出生一直到现在已经有六十年了,麋鹿在牧野出没,到处都是害虫。正因为上天不再享受殷朝的享祭,所以我今天才获得了成功。上天建立殷朝,任用有才能的人三百六十人,虽然没有做出明显的政绩,但是也让殷朝的国政一直存在,一直持续到今天。我还没有真正受到上天的保佑,哪里有空闲睡觉呢!”

武王说:“如果想真正得到上天的庇佑,就要住在靠近天帝的居室,将作恶的人全部找出来,惩罚他们,就像对待殷王那样。我要日夜勤勉地努力,从而令我西方的国土安定,我还要做好各种事情,直到我们的德教能够在四方都得以显现。从洛水拐弯处一直到伊水的拐弯处,地势平坦没有险阻,人们在这里生活,这里是夏人活动的地方。我向南望见三涂山,向北面望到太行山一带,回首又看到黄河,又观察了雒水、伊水,那是离天帝不远的地方。”

武王命人在雒邑营建周城,然后离开。武王还让人把战马都放养在华山以南,把作战时拉车的牛放养在桃林一带,将武器放下,整顿军队,解除武装,向天下人表示不再用兵。

\begin{yuanwen}
武王已克殷,后二年,问箕子殷所以亡。箕子不忍言殷恶,以存亡国宜告。武王亦丑,故问以天道。
\end{yuanwen}

武王已经战胜了殷商,过了两年,他向箕子询问殷商灭亡的原因。箕子不忍心说殷商的坏话,于是只和武王谈论了治国存亡的道理。武王也感到很惭愧,因此他只向箕子询问了天道。

\begin{yuanwen}
武王病。天下未集,群公惧,穆卜,周公乃祓\footnote{fú}斋,自为质,欲代武王,武王有瘳\footnote{chōu}。后而崩,太子诵代立,是为成王。
\end{yuanwen}

武王生病。天下还没有安定,所有的大臣都非常恐惧,他们就进行“穆卜”。于是周公斋戒沐浴,祷告上天,愿意替武王承受痛苦,武王的病渐渐好转。后来武王去世,太子诵继承王位,这就是成王。

\begin{yuanwen}
成王少,周初定天下,周公恐诸侯畔周,公乃摄行政当国。管叔、蔡叔群弟疑周公,与武庚作乱,畔周。周公奉成王命,伐诛武庚、管叔,放蔡叔。以微子开\footnote{此处避汉景帝刘启讳。}代殷后,国于宋。颇收殷馀民,以封武王少弟封为卫康叔。晋唐叔得嘉穀\footnote{指双穗禾。},献之成王,成王以归\footnote{通“馈”,赠送。}周公于兵所。周公受禾东土,鲁\footnote{鱼味道鲜美,引申为赞美。}天子之命。初,管、蔡畔周,周公讨之,三年而毕定,故初作《大诰》,次作《微子之命》,次《归禾》,次《嘉禾》,次《康诰》、《酒诰》、《梓材》,其事在周公之篇\footnote{指《鲁周公世家》。}。周公行政七年,成王长,周公反\footnote{同“返”,这里是交还的意思。}政成王,北面就群臣之位。
\end{yuanwen}

成王年少,周朝刚刚平定天下,周公害怕诸侯背叛周朝,于是他代行天子职权主持国政。管叔、蔡叔等兄弟怀疑周公,他们勾结武庚一起叛乱,背叛了周朝。周公遵奉成王的命令,讨伐武庚、管叔,流放蔡叔。周公用微子启代替武庚为殷商的后嗣,建立宋国。周公又收聚很多殷商遗民,分封武王最小的弟弟封为卫康叔。晋唐叔得到双穗禾,并把它献给成王,成王把谷穗送到周公驻兵的地方。周公在东方接受双穗禾,赞美天子的命令。起初,管叔、蔡叔反叛周朝,周公前去征讨,经过三年终于平定了叛乱,因此先写下《大诰》,然后又作《微子之命》,之后作《归禾》、《嘉禾》,之后作《康诰》、《酒诰》、《梓材》,相关的事情都记载在周公那一篇中。周公代行国政七年,成王终于长大成人,周公将政权交还给成王,自己又回到群臣的行列中。

\begin{yuanwen}
成王在丰,使召公复营洛邑,如武王之意。周公复卜申视,卒营筑,居九鼎焉。曰:“此天下之中,四方入贡道里均。”作《召诰》、《洛诰》。成王既迁殷遗民,周公以王命告,作《多士》、《无佚》。召公为保,周公为师,东伐淮夷,残奄,迁其君薄姑。成王自奄归,在宗周\footnote{指周都丰、镐一带。},作《多方》。既绌殷命,袭淮夷,归在丰,作《周官》。兴正礼乐,度制于是改,而民和睦,颂声兴。成王既伐东夷,息慎来贺,王赐荣伯作《贿息慎之命》。
\end{yuanwen}

成王住在丰邑,命令召公重新经营洛邑,这是为了遵从武王的遗愿。周公又占卜勘察地形,最后筑城,并把九鼎放在那里。他说:“这里是天下的中心,各处到这里进贡的路程都是相等的。”他又作《召诰》、《洛诰》。成王将殷商遗民迁徙过去以后,周公告诉他们成王的命令,作《多士》、《无佚》。召公担任太保,周公担任太师,他们征伐东方的淮夷,灭掉奄国,将奄国的国君迁徙到薄姑。成王从奄国返回以后,在宗周,作《多方》。周朝断绝殷商的命数,攻打淮夷以后,成王返回丰邑,作《周官》。成王制定并推广礼乐,法令和制度于是被修改,百姓相互亲睦,兴起称颂之声。成王征讨东夷以后,息慎前来祝贺,成王命荣伯作《贿息慎之命》。

\begin{yuanwen}
成王将崩,惧太子钊之不任,乃命召公、毕公率诸侯以相太子而立之。成王既崩,二公率诸侯,以太子钊见于先王庙,申告以文王、武王之所以为王业之不易,务在节俭,毋多欲,以笃信临之,作《顾命》。太子钊遂立,是为康王。康王即位,遍告诸侯,宣告以文武之业以申之,作《康诰》。故成康之际,天下安宁,刑错\footnote{同“措”,放置,搁放。}四十馀年不用。康王命作策毕公分居里,成周郊,作《毕命》。
\end{yuanwen}

成王临终前,担心太子钊无法胜任,于是命令召公、毕公率领诸侯一起辅佐太子治理国家。成王去世以后,召公、毕公率领诸侯,带着太子钊前去谒见先王的宗庙,并不断告诉他文王、武王创立基业的不易,让他务求节俭,不能有太多的欲望,要凭借笃厚诚实来治理天下,作《顾命》。太子钊于是登上王位,这就是康王。康王即位后,通告天下诸侯,不断用文王和武王的功业训诫他们,作《康诰》。因此在成王和康王统治期间,天下安定太平,有四十多年没有动用刑罚。康王发布策命毕公的文诰,命令他分治殷商遗民,安定周都的郊野,作《毕命》。

\begin{yuanwen}
康王卒,子昭王瑕立。昭王之时,王道微缺。昭王南巡狩\footnote{本义是天子巡视诸侯国。所谓的春秋笔法主张“为尊者讳”,因此帝王被迫离开国都、死于外地,也讳称“巡狩”。周昭王南征楚蛮,全军覆没而死,所以史官称其“南巡狩不返”。}不返,卒于江上。其卒不赴告,讳之也。立昭王子满,是为穆王。穆王即位,春秋已五十矣。王道衰微,穆王闵文武之道缺,乃命伯臩,申诫太仆国之政,作《臩命》。复宁。
\end{yuanwen}

康王去世以后,他的儿子昭王瑕继位。昭王在位的时候,先王之道稍有缺漏。昭王到南方巡视没能返回,在长江边去世了。昭王去世以后没有通告诸侯,因为忌讳这件事。昭王的儿子满被立为天子,这就是穆王。穆王在位时,已经五十岁了。当时先王之道已经衰败,穆王为文王和武王的法度缺失而伤心,于是任命伯臩为太仆,告诫他治国为政的道理,作《臩命》。天下重新安宁。

柳宗元:「康公之母诚贤耶?则宜以淫荒失度命其子,焉用惧之以数!且以德大而后堪,则纳三女之奔者,德果何如?若曰勿受之则可矣,教子而媚王以女,非正也。」

\begin{yuanwen}
穆王将征犬戎,祭公谋父谏曰:“不可。先王燿德不观兵。夫兵戢\footnote{jí}而时动,动则威,观则玩,玩则无震。是故周文公\footnote{周公旦,谥号“文”。}之颂曰:‘载戢干戈,载櫜\footnote{gāo}弓矢,我求懿德,肆于时夏,允王保之。’先王之于民也,茂\footnote{通“懋”,勉力,尽力。}正其德而厚其性,阜其财求而利其器用,明利害之乡,以文脩之,使之务利而辟害,怀德而畏威,故能保世以滋大。昔我先王世后稷以服事虞、夏。及夏之衰也,弃稷不务,我先王不窋用失其官,而自窜于戎狄之间。不敢怠业,时序其德,遵脩其绪,脩其训典,朝夕恪勤,守以敦笃,奉以忠信。奕世载德,不忝前人。至于文王、武王,昭前之光明而加之以慈和,事神保民,无不欣喜。商王帝辛大恶于民,庶民不忍,䜣载\footnote{通“欣戴”,高兴地拥戴。}武王,以致戎于商牧。是故先王非务武也,劝(勤)恤民隐而除其害也。夫先王之制,邦内甸服,邦外侯服,侯卫宾服,夷蛮要服,戎翟荒服。甸服者祭\footnote{指献给天子父、祖的祭品。},侯服者祀\footnote{指献给天子曾祖、高祖的祭品。},宾服者享\footnote{指献给天子远祖的祭品。},要服者贡,荒服者王\footnote{指朝见天子。}。日祭,月祀,时享,岁贡,终王。先王之顺祀\footnote{训诫。顺,通“训”。祀,应为衍文。}也,有不祭则脩意,有不祀则脩言,有不享则脩文,有不贡则脩名,有不王则脩德,序成而有不至则脩刑。于是有刑不祭,伐不祀,征不享,让不贡,告\footnote{告谕。}不王。于是有刑罚之辟,有攻伐之兵,有征讨之备,有威让之命,有文告之辞。布令陈辞而有不至,则增脩于德,无勤民于远。是以近无不听,远无不服。今自大毕、伯士之终也,犬戎氏以其职来王,天子曰‘予必以不享征之,且观之兵’,无乃废先王之训,而王几顿乎?吾闻犬戎树敦,率旧德而守终纯固,其有以御我矣。”王遂征之,得四白狼、四白鹿以归。自是荒服者不至。
\end{yuanwen}

穆王将要征伐犬戎,祭公谋父劝谏说:“不可以。先王显耀德行而不炫耀武力。军队偃旗息鼓而在必要的时候才出动,出动就要显示威力,炫耀武力就会亵渎军队,亵渎军队就不能震慑敌人。因此赞颂周文公的颂歌说:‘收起干和戈,藏好弓与箭,我想要美德,传遍全天下,王业永保全。’先王对待百姓,会尽力使他们品德端正,并且使他们性情纯厚,让他们财产增多,让他们器用便利,明示利害所在,用文德教化,令他们追求福利,并且远离灾害,心中怀有仁德而恐惧威刑,因此才能保住先王的事业来不断壮大。从前我们的先王世代担任后稷来事奉虞舜、夏禹。夏朝的命数衰败的时候,废弃后稷一职,不再重视农业,我们的先王不窋丢掉官职,而流落到戎狄地区。他对农业不敢懈怠,经常宣传后稷的美德,遵循和继承他的事业,修持他的教化和法度,从早到晚恭谨勤奋,始终用忠实诚信的态度来奉行。后来人们世代遵行这种美德,不敢辱没先人。到文王、武王时期,发扬先人美德而加以慈爱,祭祀神灵,保护百姓,天下人没有不高兴的。商王帝辛对人民犯下滔天大罪,人民无法忍受,高兴地拥戴武王,于是在商郊牧野开战。所以先王不崇尚武力,而是勤恳地体恤百姓的疾苦,并且为民除害。遵照先王的制度,邦畿以内是甸服,邦畿以外是侯服,诸侯守卫之地是宾服,蛮夷所在之地是要服,戎翟所在之地是荒服。甸服的诸侯要向天子的父、祖进献祭品,侯服的诸侯要向天子的曾祖、高祖进献祭品,宾服的诸侯要向天子的远祖进献祭品,要服的诸侯要定期进献贡物,荒服的诸侯要来朝见天子。甸服诸侯进献的祭品每天一次,侯服诸侯进献的祭品每月一次,宾服诸侯进献的祭品每季一次,要服诸侯进献的贡物每年一次,荒服诸侯朝见天子终身一次。在先王的遗训中,甸服诸侯不进献祭品就要反省思想,侯服诸侯不进献祭品就要反省言辞,宾服诸侯不进献祭品就要修定制度,要服诸侯不进献贡物就要修正名分,荒服诸侯不朝见天子就要反思德行,依次做到却仍然不能使诸侯尽职就要施以刑罚。于是王室可以惩罚甸服不进献祭品的诸侯,攻打侯服不进献祭品的诸侯,征讨宾服不进献祭品的诸侯,责备要服不进献贡物的诸侯,告谕荒服不朝见天子的诸侯。因此有关于刑罚的规定,有用于攻伐的军队,有用于征讨的武器,有施于责让的命令,有写于文告的言辞。下达文辞命令却仍然不能使诸侯尽职,就要进一步反思德行,不要使人民疲劳进行远征。所以邻近的诸侯没有不听命的,远方的国家没有不服从的。现在自从大毕、伯士去世以后,犬戎氏遵守其职责前来朝见天子,天子说‘我一定要按照不进献祭品的罪名征讨他们,还要向他们炫耀武力’,这无异于抛弃了先王的训诫,而大王就要陷入困境了吧?我听说犬戎氏宣扬敦厚的风气,都遵照前人的德行,并且能够一直坚持,他们就有足够抵御我们的力量了。”穆王最终还是发兵征讨了,抓获四只白狼、四只白鹿返回。从此荒服的国家都不来朝见天子了。

\begin{yuanwen}
诸侯有不睦者,甫侯言于王,作脩刑辟。王曰:“吁,来!有国有土,告汝祥刑。在今尔安百姓,何择非其人,何敬非其刑,何居非其宜与?两造具备,师听五辞\footnote{五刑的法律条文。}。五辞简信,正于五刑。五刑不简,正于五罚\footnote{五种情况的罚金。}。五罚不服,正于五过\footnote{指五种可以宽赦的罪过。}。五过之疵,官狱内狱,阅实其罪,惟钧其过。五刑之疑有赦,五罚之疑有赦,其审克之。简信有众,惟讯有稽。无简不疑,共严天威。黥辟\footnote{又称墨刑,在犯人脸上刺字的刑罚。}疑赦,其罚百率\footnote{《尚书》作“锾”,重量单位,六两为一锾。},阅实其罪。劓\footnote{yì}辟疑赦,其罚倍洒\footnote{加倍。},阅实其罪。膑辟疑赦,其罚倍差\footnote{不到两倍。},阅实其罪。宫辟疑赦,其罚五百率,阅实其罪。大辟\footnote{死刑。}疑赦,其罚千率,阅实其罪。墨罚之属千,劓罚之属千,膑罚之属五百,宫罚之属三百,大辟之罚其属二百,五刑之属三千。”命曰《甫刑\footnote{即《尚书·吕刑》。}》。
\end{yuanwen}

诸侯中有不和睦的,甫侯把这种情况告诉穆王,穆王设置各种刑法。穆王说:“啊,到我这里来!有封国有土地的诸侯,我要告诉你们完善的刑法。现在你们要安定百姓,为什么选择不称职的人做官?为什么敬重不合理的刑法?为什么用在不恰当的地方?原告与被告都在,士师应该根据五刑的法律条文来审验。通过审验与事实相符,就用五刑来定罪。如果五刑不合适,就用五罚来定罪。如果五罚也不合适,就用五过来定罪。按照五过来判定罪行有缺陷,司法官会遇到畏惧高官和接受说情的情况,要详细调查案情,使罪名和过失相当。按照五刑治罪有疑问而需要赦免的,以及按照五罚治罪有疑问而需要赦免的,就要认真审查。收集证据应该顺从民众,审讯也要有证据。假如没有充分的证据就不要怀疑,一定要严肃上天的威信。应该处以黥刑有疑问而需要赦免的,罚金为一百率,查明罪状。应该处以劓刑有疑问而需要赦免的,罚金是前者的两倍,也要查明罪状。应该处以膑刑有疑问而需要赦免的,罚金不到前者的两倍,也要查明罪状。应该处以宫刑有疑问而需要赦免的,罚金是五百率,要查明罪状。应该处以死有疑问而需要赦免的,其罚金为一千率,也要查明罪状。关于墨刑的条文有一千条,关于劓刑的条文也有一千条,关于膑刑的条文有五百条,关于宫刑的条文有三百条,关于死刑的条文有二百条,五刑的条文一共有三千条。”这套刑法被称为《甫刑》。

\begin{yuanwen}
穆王立五十五年,崩,子共王繄\footnote{yī}扈立。共王游于泾上,密康公从,有三女饹(奔)之。其母曰:“必致之王。夫兽三为群,人三为众,女三为粲。王田不取群,公行不下众\footnote{因众人而下车。},王御不参一族。夫粲,美之物也。众以美物归\footnote{通“馈”,赠予,送给。}女,而何德以堪之?王犹不堪,况尔之小丑乎!小丑备物,终必亡。”

康公不献,一年,共王灭密。共王崩,子懿王囏\footnote{jiān}畑立。懿王之时,王室遂衰,诗人作刺。
\end{yuanwen}

穆王在位五十五年,去世以后,他的儿子共王繄扈继位。共王在泾水边游玩,密康公跟在身边,有三个女子前来投奔他。密康公的母亲对他说:“一定要把她们献给天子。兽有三只以上称为群,人有三个以上称为众,女子有三人以上称为粲。天子打猎不捕获三只以上的兽,诸侯出行不因见到三个以上的人下车,天子的妃嫔属于同一宗族的不超过三个人。所谓粲,就是美好的事物。众人都把美好的东西送给你,可是你有什么德行去享用这些东西呢?连天子都不配享用,何况是你这样的小人物呢!小人物如果拥有以上的东西,最终一定会灭亡。”

密康公不愿意献给共王,一年以后,共王灭掉密国。共王去世以后,他的儿子懿王囏继位。懿王在位的时候,王室就衰败了,诗人写诗讽刺。

\begin{yuanwen}
懿王崩,共王弟辟方立,是为孝王。孝王崩,诸侯复立懿王太子燮\footnote{xiè},是为夷王。
\end{yuanwen}

懿王去世以后,共王的弟弟辟方继位,这就是孝王。孝王去世以后,诸侯又拥立懿王的太子燮继位,这就是夷王。

\begin{yuanwen}
夷王崩,子厉王胡立。厉王即位三十年,好利,近荣夷公。大夫芮良夫谏厉王曰:“王室其将卑乎?夫荣公好专利而不知大难。夫利,百物之所生也,天地之所载也,而有专之,其害多矣。天地百物皆将取焉,何可专也?所怒甚多,(而)不备大难。以是教王,王其能久乎?夫王人者,将导利而布之上下者也。使神人百物无不得极,犹日怵惕惧怨之来也。故《颂》曰‘思文后稷,克配彼天,立我蒸\footnote{同“丞”,众,众多。}民,莫匪\footnote{同“非”,不。}尔极\footnote{等于说“极尔”,意思是把你当作榜样。}’。《大雅》曰‘陈锡载周’。是不布利而惧难乎,故能载周以至于今。今王学专利,其可乎?匹夫专利,犹谓之盗,王而行之,其归鲜矣。荣公若用,周必败也。”

厉王不听,卒以荣公为卿士,用事。
\end{yuanwen}

夷王去世以后,他的儿子厉王胡继位。厉王在位三十年,贪图财利,亲近荣夷公。大夫芮良夫劝谏厉王说:“王室难道要衰落了吗?荣夷公爱好聚敛财利却不知道大难就要来临了。财利,是万物所产生的,是天地所承载的,却想要据为己有,害处实在太多了。天地万物都想获取,怎么能够据为己有呢?他所触怒的人太多了,又不防备大难。他用这些来教导大王,大王怎么能够长久地统治呢?统治百姓的人,应该疏导财利而将其从上到下分布出去。即使对世间万物都做到极致,仍然要每天提心吊胆,担忧引起不满。因此《颂》说:‘追念后稷有文德,功绩卓越配上天,天下万民得安定,无不以你为榜样。’《大雅》中写道:‘广施恩泽,成就周邦。’这不就是说要布施财利而畏惧灾难吗?所以才使周朝延续至今天。现在大王学的是聚敛财利,怎么可以呢?一个普通人聚敛财利,尚且被称为盗贼,身为天子却要这样做,愿意归附他的人一定很少了。荣公如果得到重用,周朝一定会衰败。”

厉王不听他的建议,最后还是任命荣夷公为卿士,让他管理国家大事。

\begin{yuanwen}
王行暴虐侈傲,国人谤王。召公\footnote{名虎,谥号“穆”。}谏曰:“民不堪命矣。”

王怒,得卫巫,使监谤者,以告则杀之。其谤鲜矣,诸侯不朝。三十四年,王益严,国人莫敢言,道路以目。厉王喜,告召公曰:“吾能弭谤矣,乃不敢言。”

召公曰:“是鄣\footnote{zhāng}之也。防民之口,甚于防水。水壅而溃,伤人必多,民亦如之。是故为水者决之使导,为民者宣之使言。故天子听政,使公卿至于列士献诗,瞽献曲,史献书,师箴\footnote{zhēn},瞍\footnote{sǒu,没有眼珠的盲人,也是乐师。}赋,矇\footnote{有眼珠的盲人,也是乐师。}诵,百工谏,庶人传语,近臣尽规,亲戚补察,瞽史教诲,耆艾脩(修)之,而后王斟酌焉,是以事行而不悖。民之有口也,犹土之有山川也,财用于是乎出:犹其有原隰衍沃也,衣食于是乎生。口之宣言也,善败于是乎兴。行善而备败,所以产财用衣食者也。夫民虑之于心而宣之于口,成而行之。若壅其口,其与能几何?”

王不听。于是国莫敢出言,三年,乃相与畔,袭厉王。厉王出奔于彘。
\end{yuanwen}

厉王施行暴政,奢侈傲慢,国人批评他。召公劝谏说:“人民不能忍受了。”

厉王非常愤怒,找到一个卫国的巫师,让他监视批评他的人,报告他之后就处死。结果批评厉王的人少了,诸侯也不来朝见了。三十四年(前844年),厉王对百姓的控制更加严厉,国人不敢随便说话,走在路上相互使眼色。厉王非常高兴,他告诉召公说:“我可以平息人们的批评了,他们不敢说话。”

召公说:“这只是把他们的嘴堵住了。堵住百姓的嘴,要比堵住洪水还要危险。洪水被堵住会冲破堤坝,伤害的人一定很多,而堵住百姓的嘴也是一样的道理。因此治理洪水的人应该打通淤塞对其加以疏导,治理百姓的人应该开放言路让他们发表议论。因此天子处理政事,应该让公卿乃至列士都献上诗歌,让盲乐师献上乐曲,让史官献上图书,让太师进行规诫,让无眼珠的盲乐师演唱诗歌,让有眼珠的盲乐师朗诵规诫,让百工都来劝谏,让平民相互议论,让近臣尽心规劝,让亲戚弥补过失,让盲乐师和史官一起教诲,让年老的贤人进行训诫,最后由大王仔细思考,这样政事才能够施行而不违背情理。人民有嘴,就好像土地有山川一样,财用都是从那里产生的;就像平原有沃土一样,衣食都是从那里产生的。让百姓把话都讲出来,治理国家的成败就可以看出来。做好事并防备出现坏事,这就像产生财用和衣食的道理一样。百姓心中所思虑的,就是嘴上所说的,这样才能把事情办好,如果堵住百姓的嘴,又怎么能长久呢?”

厉王不听从劝谏。于是国人都不敢说话,过了三年,人们一起发动叛乱,袭击厉王。厉王出逃到彘。

钱福:「昔齐妃笑跛而郤克师兴,赵妾笑躄而平原客散,幽王举火戏诸侯,以发褒姒之笑,而诸侯畔。自古妇人一笑虽微,而贻无穷之祸,人岂可以笑为轻而不致谨哉!」

\begin{yuanwen}
厉王太子静匿召公之家,国人闻之,乃围之。召公曰:“昔吾骤谏王,王不从,以及此难也。今杀王太子,王其以我为雠(仇)而懟\footnote{duì}怒乎?夫事君者,险而不雠懟,怨而不怒,况事王乎!”乃以其子代王太子,太子竟得脱。
\end{yuanwen}

厉王的太子静躲藏在召公家中,国人听说以后,就包围了召公家。召公说:“以前我多次劝谏大王,但是大王不听从,才有了今天的灾难。现在我要是杀死大王的太子,大王也许以为我是因为记仇而泄愤吧?事奉君主,身处危难也不应该记仇,心有怨气也不应该发泄,何况是事奉天子呢!”于是他用自己的儿子代替厉王的太子,太子最后才得以逃脱。

\begin{yuanwen}
召公、周公\footnote{名已失传,谥号“定”。}二相行政,号曰“共和”。共和十四年,厉王死于彘。太子静长于召公家,二相乃共立之为王,是为宣王。宣王即位,二相辅之,脩政,法文、武、成、康之遗风,诸侯复宗周。十二年,鲁武公来朝。
\end{yuanwen}

召公、周公两位辅臣共同执政,号称“共和”。共和十四年(前828年),厉王死在彘。太子静在召公的家里长大,两位辅臣于是共同拥立他成为王,这就是宣王。宣王在位时,两位辅臣辅佐他,修整政治,效法文王、武王、成王、康王的遗风,诸侯重新以周朝为宗主。十二年(前816年),鲁武公前来朝见。

\begin{yuanwen}
宣王不脩籍\footnote{指帝王亲耕,以示重视农业。}于千亩,虢\footnote{guó}文公谏曰不可,王弗听。三十九年,战于千亩,王师败绩于姜氏之戎。
\end{yuanwen}

宣王不在千亩亲耕,虢文公劝谏说不可以,宣王不听从。三十九年(前789年),在千亩作战,王室的军队被姜氏之戎打败。

\begin{yuanwen}
宣王既亡南国之师,乃料民于太原。仲山甫谏曰:“民不可料也。”宣王不听,卒料民。
\end{yuanwen}

宣王丧失征伐南方的军队以后,就在太原统计户口。仲山甫劝谏说:“户口不可以由天子直接进行统计。”宣王不听从,最终还是统计了户口。

\begin{yuanwen}
四十六年,宣王崩,子幽王宫湦\footnote{shēng}立。幽王二年,西周\footnote{指丰、镐一带。}三川\footnote{指泾、渭、洛三条河流。}皆震。伯阳甫曰:“周将亡矣。夫天地之气,不失其序;若过其序,民乱之也。阳伏而不能出,阴迫而不能蒸,于是有地震。今三川实震,是阳失其所而填阴\footnote{为阴气所镇伏。填,通“镇”。}也。阳失而在阴,原\footnote{同“源”,水源。}必塞;原塞,国必亡。夫水土演而民用也。土无所演,民乏财用,不亡何待!昔伊、洛竭而夏亡,河竭而商亡。今周德若二代之季矣,其川原又塞,塞必竭。夫国必依山川,山崩川竭,亡国之徵也。川竭必山崩。若国亡不过十年,数之纪也。天之所弃,不过其纪。”是岁也,三川竭,岐山崩。
\end{yuanwen}

四十六年(前782年),宣王去世,他的儿子幽王宫湦继位。幽王二年(前780年),周都附近的三川地区发生地震。伯阳甫说:“周朝将要灭亡了。天地二气,不能失去次序;假如失去应有的次序,民众就会被扰乱。阳气潜伏而不能出来,被阴气压制而不能蒸腾,于是发生地震。现在三川地区发生地震,是因为阳气失掉位次而被阴气镇伏。阳气失去位次而被阴气镇伏,水源就一定会堵塞;水源被堵塞,国家就一定会灭亡。水土通畅才会使百姓得到财用。假如地下的水流不能通畅,人民就会缺乏财用,国家不灭亡还能等到什么时候呢?从前伊水、洛水枯竭而夏朝灭亡,黄河枯竭而商朝灭亡。现在周朝的德运就像两朝的末期一样了,河川的水源已经被堵塞,被堵塞就一定会枯竭。建立国都就一定要依从高山大河,高山崩塌,大河枯竭,这是亡国的征兆。大河枯竭一定会引起高山崩塌。如果亡国就不会超过十年,因为数是以十进位的。上天所抛弃的,不会超过这个数字。”这一年,三川枯竭,岐山崩塌。

\begin{yuanwen}
三年,幽王嬖爱襃姒。襃姒生子伯服,幽王欲废太子。太子母申侯女,而为后。后幽王得襃姒,爱之,欲废申后,并去太子宜臼,以襃姒为后,以伯服为太子。周太史伯阳读史记\footnote{泛指史书。}曰:“周亡矣。”昔自夏后氏之衰也,有二神龙止于夏帝庭而言曰:“余,襃之二君。”夏帝卜杀之与去之与止之,莫吉。卜请其漦\footnote{chí,指龙的唾液。}而藏之,乃吉。于是布币而策告之,龙亡而漦在,椟而去之。夏亡,传此器殷。殷亡,又传此器周。比三代,莫敢发之,至厉王之末,发而观之。漦流于庭,不可除。厉王使妇人裸而譟(噪)之。漦化为玄鼋\footnote{yuán,黑色的蜥蜴。},以入王后宫。后宫之童妾既龀\footnote{乳牙脱落,这里指七八岁的年龄。}而遭之,既笄\footnote{jī,女子成人礼,这里指十五岁的年龄。}而孕,无夫而生子,惧而弃之。宣王之时童女谣曰:“檿\footnote{yǎn}弧箕服\footnote{箭袋。},实亡周国。”于是宣王闻之,有夫妇卖是器者,宣王使执而戮之。逃于道,而见乡者后宫童妾所弃妖子出于路者,闻其夜啼,哀而收之,夫妇遂亡,饹于襃。襃人有罪,请入童妾所弃女子者于王以赎罪。弃女子出于襃,是为襃姒。当幽王三年,王之后宫见而爱之,生子伯服,竟废申后及太子,以襃姒为后,伯服为太子。太史伯阳曰:“祸成矣,无可奈何!”
\end{yuanwen}

三年(前779年),幽王宠爱褒姒。褒姒生下儿子伯服,幽王想废黜太子。太子的母亲是申侯的女儿,也是王后。后来幽王得到褒姒,十分喜爱她,想废黜申后,一并废掉太子宜臼,立褒姒为王后,立伯服为太子。周太史伯阳阅读史书说:“周朝就要亡国了。”从前夏后氏衰败时,有两条神龙降落在夏帝的庭院中说:“我们,是褒国的两位君主。”夏帝为杀掉它们或赶走它们或留下它们而占卜,都没有得到吉兆。卜者请求把龙的唾液收藏起来,于是得到吉兆。于是夏帝准备好丝帛,并且书写简策祷告,龙飞走以后留下了唾液,夏帝命人将其盛在木匣中收藏好。夏朝灭亡以后,这件器物传给了殷商。殷商灭亡以后,这件器物又传给了周朝。连续三个朝代,都没有人敢打开它。到厉王在位的末期,才打开观看。龙的唾液流到庭院中,不能清除。厉王让女人赤裸着在庭院中大声呼喊。龙的唾液变为黑色的蜥蜴,进入厉王的后宫。后宫中有一个七八岁的小奴婢遇到了蜥蜴,在十五岁时怀孕了,没有丈夫却生下孩子,她因害怕而抛弃了孩子。宣王在位的时候有一个童女唱歌谣说:“檿木做弧弓,箕木做箭袋,周朝不久将灭亡。”这时宣王听到了,遇到一对夫妇在卖这两样器物,宣王命人把他们抓起来杀掉。他们在逃跑的路上,看见之前被后宫小奴婢遗弃在路边的孩子,他们听到孩子在夜里哭泣,因为哀怜而将其收养。夫妇二人于是逃走,跑到褒国。褒国人犯了罪,就请将小奴婢遗弃在路边的妖异女子献给天子来赎罪。这个被遗弃的女子来自褒国,这就是褒姒。幽王在位三年的时候,他来到后宫看见褒姒而非常喜爱她,就和她生下儿子伯服,竟然要废黜申后和太子,立褒姒为王后,立伯服为太子。太史伯阳说:“灾祸已经形成了,没有什么办法了!”

杨慎:「周三十七王,八百六十七年。然自武王灭殷,至幽王,二百五十七年耳。而昭王时王道已微,懿王时王道遂衰,昭王南巡不返,厉王死于彘,此二百五十七年中,变故多矣,东迁以后不足言也。夫莫盛于周,而治日之少如此,有国者其慎之。」

\begin{yuanwen}
襃姒不好笑,幽王欲其笑万方,故不笑。幽王为烽燧\footnote{边防预警的信号,白天放烟为烽,夜晚举火为燧。}大鼓,有寇至则举烽火。诸侯悉至,至而无寇,襃姒乃大笑。幽王说之,为数举烽火。其后不信,诸侯益亦不至。	
\end{yuanwen}

褒姒不爱笑,幽王就想尽各种办法逗她笑,可是她还是不笑。幽王在各地设有烽燧和大鼓,有敌人进犯就点起烽火。有一次幽王点燃烽火,诸侯都赶过来,却没有看到敌人,褒姒于是大笑。幽王很高兴,就为褒姒多次点燃烽火。后来诸侯都不再相信幽王,看到烽火也不来了。

\begin{yuanwen}
幽王以虢石父为卿,用事,国人皆怨。石父为人佞巧善谀好利,王用之。又废申后,去太子也。申侯怒,与缯、西夷犬戎攻幽王。幽王举烽火徵兵,兵莫至。遂杀幽王骊山下,虏襃姒,尽取周赂而去。于是诸侯乃即申侯而共立故幽王太子宜臼,是为平王,以奉周祀。
\end{yuanwen}

幽王任命虢石父为卿士,管理国政,国人都有怨言。虢石父为人巧言令色,擅长阿谀,贪图财利,幽王却重用他。幽王又废黜了申后,赶走了太子。申侯很生气,联合缯国、西夷的犬戎攻打幽王。幽王点燃烽火征召诸侯的军队,诸侯的军队却都没有赶来。于是他们在骊山下把幽王杀死了,掳走褒姒,把周朝的财物洗劫一空才离开。于是诸侯都到申侯那里,并且共同拥立幽王以前的太子宜臼为天子,这就是平王,由他来继承周朝的祭祀。

\begin{yuanwen}
平王立,东迁于雒邑,辟戎寇。平王之时,周室衰微,诸侯强并弱,齐、楚、秦、晋始大,政由方伯\footnote{霸主,即诸侯盟主。}。
\end{yuanwen}

平王继位以后,把都城向东迁到雒邑,躲避西戎的敌人。平王在位的时候,周王室日渐衰败,诸侯中强大的吞并弱小的,齐、楚、秦、晋等国开始壮大,政令出自称霸的诸侯。

\begin{yuanwen}
四十九年,鲁隐公即位。
\end{yuanwen}

四十九年(前722年),鲁隐公即位。

\begin{yuanwen}
五十一年,平王崩,太子洩父蚤\footnote{通“早”。}死,立其子林,是为桓王。桓王,平王孙也。
\end{yuanwen}

五十一年(前720年),平王去世,太子洩父很早就死了,立他的儿子林为王,这就是桓王。桓王,是平王的孙子。

\begin{yuanwen}
桓王三年,郑庄公朝,桓王不礼。五年,郑怨,与鲁易许田。许田,天子之用事太山\footnote{泰山。}田也。八年,鲁杀隐公,立桓公。十三年,伐郑,郑射伤桓王,桓王去归。
\end{yuanwen}

桓王三年(前717年),郑庄公前来朝见,桓王没有以礼相待。五年(前715年),郑国有怨恨,与鲁国交换许田。许田,是天子用来祭祀泰山的土地。八年(前712年),鲁国杀死隐公,拥立桓公。十三年(前707年),王室讨伐郑国,郑军射伤了桓王,桓王逃回。

\begin{yuanwen}
二十三年,桓王崩,子庄王佗立。庄王四年,周公黑肩欲杀庄王而立王子克。辛伯告王,王杀周公。王子克饹(奔)燕。


\end{yuanwen}

二十三年(前697年),桓王去世,他的儿子庄王佗继位。庄王四年(前693年),周公黑肩想要杀死庄王改立王子克。辛伯向庄王报告,庄王杀死周公黑肩。王子克逃到燕国。

\begin{yuanwen}
十五年,庄王崩,子釐王胡齐立。釐王三年,齐桓公始霸。
\end{yuanwen}

十五年(前682年),庄王去世,他的儿子釐王胡齐继位。釐王三年(前679年),齐桓公开始确立霸主的地位。

\begin{yuanwen}
五年,釐王崩,子惠王阆立。惠王二年。初,庄王嬖姬姚,生子穨\footnote{tuí},穨有宠。及惠王即位,夺其大臣园以为囿,故大夫边伯等五人作乱,谋召燕、卫师,伐惠王。惠王饹(奔)温,已居郑之栎。立釐王弟穨为王。乐及遍舞,郑、虢君怒。四年,郑与虢君伐杀王穨,复入惠王。惠王十年,赐齐桓公为伯。
\end{yuanwen}

五年(前677年),釐王去世,他的儿子惠王阆继位。惠王二年(前675年)。当初,庄王的宠妾姚姓女子,生下儿子穨,穨得到宠爱。等到惠王在位,夺取大臣的园林作为自己的苑囿,所以大夫边伯等五个人发动叛乱,谋划召集燕国、卫国的军队,共同讨伐惠王。惠王逃到温邑,不久又住在郑国的栎邑。叛乱者拥立釐王的弟弟穨为王,并演奏全套的舞乐,郑、虢两国的国君大怒。四年(前673年),郑国和虢国的国君讨伐并杀死王穨,重新迎回惠王。惠王十年(前667年),赐齐桓公为霸主。

\begin{yuanwen}
二十五年,惠王崩,子襄王郑立。襄王母蚤死,后母曰惠后。惠后生叔带,有宠于惠王,襄王畏之。三年,叔带与戎、翟谋伐襄王,襄王欲诛叔带,叔带饹(奔)齐。

齐桓公使管仲平戎于周,使隰朋平戎于晋。王以上卿礼管仲。管仲辞曰:“臣贱有司也,有天子之二守国、高在。若节春秋来承王命,何以礼焉?陪臣\footnote{相隔一级的臣。诸侯是天子的臣,卿大夫是诸侯的臣,而卿大夫之下又有家臣,因此卿大夫是天子的陪臣,其家臣是诸侯的陪臣。《论语·季氏》所说的“陪臣执国命”,指的是鲁国季氏的家臣阳虎执掌国政之事。}敢辞。”

王曰:“舅氏\footnote{对异姓诸侯的尊称。周天子与异姓诸侯通婚,因此尊称其为舅。对于同姓诸侯,周天子则尊称伯父、叔父。《尚书·文侯之命》中,周平王称晋文侯(字义和)为“父义和”。},余嘉乃勋,毋逆朕命。”

管仲卒受下卿之礼而还。

九年,齐桓公卒。

十二年,叔带复归于周。
\end{yuanwen}

二十五年(前652年),惠王去世,他的儿子襄王郑继位。襄王的母亲很早就去世了,后母就是惠后。惠后生下叔带,得到惠王的宠爱,襄王为此担忧。三年(前649年),叔带和戎、翟共同谋划讨伐襄王,襄王想要杀掉叔带,叔带逃到了齐国。

齐桓公派管仲调解戎人和周朝的矛盾,派隰朋调解戎人和晋国的矛盾。襄王用上卿的礼仪招待管仲。管仲拒绝说:“我是地位卑微的官员,现在有天子任命的两位上卿国氏、高氏在。如果用春秋两季的朝聘礼仪来领受王命,我又将用什么礼仪回应呢?陪臣冒昧地请求拒绝这一礼节。”

襄王说:“身为舅家的人,我要嘉奖你的功勋,不要违背我的命令。”

管仲最终只接受下卿的礼仪并返回。

九年(前643年),齐桓公去世。

十二年(前640年),叔带重新回到周都。

\begin{yuanwen}
十三年,郑伐滑,王使游孙、伯服请滑,郑人囚之。郑文公怨惠王之入不与厉公爵,又怨襄王之与卫滑,故囚伯服。王怒,将以翟伐郑。富辰谏曰:“凡我周之东徙,晋、郑焉依。子穨之乱,又郑之由定,今以小怨弃之!”王不听。

十五年,王降翟师以伐郑。王德翟人,将以其女为后。富辰谏曰:“平、桓、庄、惠皆受郑劳,王弃亲亲翟,不可从。”王不听。

十六年,王绌翟后,翟人来诛,杀谭伯。富辰曰:“吾数谏不从。如是不出,王以我为懟乎?”乃以其属死之。
\end{yuanwen}

十三年(前639年),郑国讨伐滑国,襄王派游孙、伯服为滑国求情,郑人将他们囚禁起来。郑文公埋怨惠王复国却没有赐给郑厉公酒爵,又埋怨襄王把滑国交给卫国,因此把伯服囚禁起来。襄王大怒,准备召集翟人讨伐郑国。富辰劝谏说:“但凡我们周朝向东迁都,都是依靠晋国、郑国的力量。王子穨发动叛乱,也是依靠郑国的力量才平定的,现在怎么能因为小怨恨而抛弃郑国呢!”襄王不听。

十五年(前637年),襄王命令翟人的军队讨伐郑国。襄王很感谢翟人,将要立该国女子为王后。富辰又劝谏说:“平王、桓王、庄王、惠王都得到过郑国的帮助,大王抛弃亲近的人而去亲近翟人,不可以这样做。”襄王不听。

十六年(前636年),襄王废黜翟后,翟人前来问罪,杀死谭伯。富辰说:“我多次劝谏大王都不听,如果我不出战,大王一定认为我在怨恨他吧?”于是他率领自己的部众战死。

\begin{yuanwen}
初,惠后欲立王子带,故以党开\footnote{迎接,引申为做内应。}翟人,翟人遂入周。襄王出饹(奔)郑,郑居王于氾。子带立为王,取\footnote{同“娶”。}襄王所绌翟后与居温。

十七年,襄王告急于晋,晋文公纳王而诛叔带。襄王乃赐晋文公珪鬯\footnote{秬鬯,一种用郁金草和黑黍酿造的香酒,用于祭祀。}弓矢,为伯,以河内地与晋。二十年,晋文公召襄王,襄王会之河阳、践土,诸侯毕朝,书讳曰“天王狩于河阳”。
\end{yuanwen}

当初,惠后想要立王子带,因此派党羽为翟人做内应,翟人于是攻进周都。襄王逃到郑国,郑国把襄王安顿在氾邑。王子带被立为王,娶襄王所废黜的翟后为妻,共同住在温邑。

十七年(前635年),襄王向晋国求救,晋文公接纳襄王并杀死叔带。襄王于是赏赐晋文公玉珪、秬鬯、弓箭,任命他为霸主,把河内的土地赏赐给晋国。二十年(前632年),晋文公召来襄王,襄王在河阳、践土与他见面,诸侯都来朝见,史书讳称“天子到河阳巡视”。

\begin{yuanwen}
二十四年,晋文公卒。
\end{yuanwen}

二十四年(前628年),晋文公去世。

\begin{yuanwen}
三十一年,秦穆公卒。	
\end{yuanwen}

三十一年(前621年),秦穆公去世。

\begin{yuanwen}
三十二\footnote{应为三十三年。}年,襄王崩,子顷王壬臣立。顷王六年,崩,子匡王班立。匡王六年,崩,弟瑜立,是为定王。
\end{yuanwen}

三十三年(前619年),襄王去世,他的儿子顷王壬臣继位。顷王在位六年,就去世了,他的儿子匡王班继位。匡王在位六年,也去世了,他的弟弟瑜继位,这就是定王。

\begin{yuanwen}
定王元年,楚庄王伐陆浑之戎,次洛,使人问九鼎。王使王孙满应设以辞,楚兵乃去。

十年,楚庄王围郑,郑伯降,已而复之。

十六年,楚庄王卒。	
\end{yuanwen}

定王元年(前606年),楚庄王讨伐陆浑之戎,驻扎在洛邑,派人询问九鼎的大小和轻重。定王派王孙满以言辞对答,楚国军队才撤退。

十年(前597年),楚庄王包围了郑都,郑伯投降,不久又复国。

十六年(前591年),楚庄王去世。
	
\begin{yuanwen}
二十一年,定王崩,子简王夷立。

简王十三年,晋杀其君厉公,迎子周于周,立为悼公。
\end{yuanwen}

二十一年(前586年),定王去世,他的儿子简王夷继位。

简王十三年(前573年),晋国杀死其国君厉公,从周都迎接子周,立他为悼公。

\begin{yuanwen}
十四年,简王崩,子灵王泄心立。

灵王二十四年,齐崔杼\footnote{zhù}弑其君庄公。

二十七年,灵王崩,子景王贵立。

景王十八年,后太子圣而\footnote{圣而,据《左传》,应为“寿”。}蚤卒。

二十年\footnote{应为二十五年。},景王爱子朝,欲立之,会崩,子丐之党与争立,国人立长子猛为王,子朝攻杀猛。猛为悼王。晋人攻子朝而立丐,是为敬王。
\end{yuanwen}

十四年(前572年),简王去世,他的儿子灵王泄心继位。

灵王二十四年(前548年),齐国的崔杼杀死其国君庄公。

二十七年(前545年),灵王去世,他的儿子景王贵继位。

景王十八年(前527年),王后、太子寿很早就去世了。

二十五年(前520年),景王喜爱儿子朝,想要立他为太子,却在这时去世了,他的儿子丐及其党羽和王子朝争夺王位,国人拥立长子猛为王,王子朝于是攻杀猛。猛就是悼王。晋人攻打王子朝改立丐为王,他就是敬王。

\begin{yuanwen}
敬王元年,晋人入敬王,子朝自立,敬王不得入,居泽。

四年,晋率诸侯入敬王于周,子朝为臣,诸侯城周。

十六年,子朝之徒复作乱,敬王饹(奔)于晋。

十七年,晋定公遂入敬王于周。
\end{yuanwen}

敬王元年(前519年),晋人送敬王回周都,王子朝已经自立为王,敬王无法回去,就住在泽邑。

四年(前516年),晋国率领诸侯送敬王回周都,王子朝成为臣子,诸侯修筑周都的城墙。

十六年(前504年),王子朝的党羽再次发动叛乱,敬王逃到晋国。

十七年(前503年),晋定公最终将敬王送回周都。

\begin{yuanwen}
三十九年,齐田常\footnote{本名恒,《史记》避汉文帝刘恒讳改其名为常。}杀其君简公。
\end{yuanwen}

三十九年(前481年),齐国的田常杀死其国君简公。
	
\begin{yuanwen}
四十一年,楚灭陈。孔子卒。
\end{yuanwen}

四十一年(前479年),楚国灭掉陈国。孔子去世。

\begin{yuanwen}
四十二年\footnote{应为四十三年。},敬王崩,子元王仁立。

元王八年,崩,子定王\footnote{又称贞王或贞定王。}介立。
\end{yuanwen}

四十三年(前477年),敬王去世,他的儿子元王仁继位。

元王在位八年,去世以后,他的儿子定王介继位。

\begin{yuanwen}
定王十六年,三晋\footnote{指晋国赵、魏、韩三卿。}灭智伯,分有其地。
\end{yuanwen}

定王十六年(前453年),赵、魏、韩三家灭掉智伯,瓜分他的土地。

\begin{yuanwen}
二十八年,定王崩,长子去疾立,是为哀王。

哀王立三月,弟叔袭杀哀王而自立,是为思王。

思王立五月,少弟嵬\footnote{wéi}攻杀思王而自立,是为考王。此三王皆定王之子。
\end{yuanwen}

二十八年(前441年),定王去世,他的大儿子去疾继位,这就是哀王。

哀王在位三个月,他的弟弟叔袭杀死哀王而自立,这就是思王。

思王在位五个月,他的少弟嵬攻打并杀死思王而自立,这就是考王。这三王都是定王的儿子。

\begin{yuanwen}
考王十五年,崩,子威烈王午立。
\end{yuanwen}

考王十五年(前426年),去世,他的儿子威烈王午继位。

\begin{yuanwen}
考王封其弟于河南,是为桓公,以续周公之官职。桓公卒,子威公代立。威公卒,子惠公代立,乃封其少子于巩以奉王,号东周惠公。
\end{yuanwen}

考王封他的一个弟弟在河南,这就是西周桓公,让他继承周公的官职。桓公去世,他的儿子威公继位。威公去世,他的儿子惠公继位,于是封小儿子在巩邑事奉天子,号称东周惠公。

\begin{yuanwen}
威烈王二十三年,九鼎震。命韩、魏、赵为诸侯。
\end{yuanwen}

威烈王二十三年(前403年),九鼎所在的都城发生地震。威烈王策命韩、魏、赵为诸侯。

\begin{yuanwen}
二十四年,崩,子安王骄立。是岁盗杀楚声王。
\end{yuanwen}

二十四年(前402年),威烈王去世,他的儿子安王骄继位。这一年有盗贼杀死楚声王。

\begin{yuanwen}
安王立二十六年,崩,子烈王喜立。

烈王二年,周太史儋见秦献公曰:“始周与秦国合而别,别五百载复合,合十七岁而霸王者出焉。”
\end{yuanwen}

安王在位二十六年,去世,他的儿子烈王喜继位。

烈王二年(前374年),周太史儋见秦献公说:“最初周朝和秦国合一而后来分开,分开五百年以后又会合在一起,合在一起十七年后会有霸主出现。”

\begin{yuanwen}
十年\footnote{应为七年。},烈王崩,弟扁立,是为显王。

显王五年,贺秦献公,献公称伯。九年,致文武胙\footnote{zuò}于秦孝公。

二十五年,秦会诸侯于周。

二十六年,周致伯于秦孝公。

三十三年,贺秦惠王。

三十五年,致文武胙于秦惠王。

四十四年,秦惠王称王。其后诸侯皆为王。
\end{yuanwen}

七年(前369年),烈王去世,他的弟弟扁继位,这就是显王。

显王五年(前364年),祝贺秦献公,献公称霸。

九年(前360年),显王把文王和武王的祭肉送给秦孝公。

二十五年(前344年),秦国在周都会见诸侯。

二十六年(前343年),周朝赐给秦孝公霸主的称号。

三十三年(前336年),祝贺秦惠王。

三十五年(前334年),显王把文王和武王的祭肉送给秦惠王。

四十四年(前325年),秦惠王称王。从此以后诸侯都称王。

\begin{yuanwen}
四十八年,显王崩,子慎靓王定立。慎靓王立六年,崩,子赧王延立。王赧时东西周分治。王赧徙都西周。
\end{yuanwen}

四十八年(前321年),显王去世,他的儿子慎靓王定继位。慎靓王在位六年,去世,他的儿子赧王延继位。赧王在位的时候东西二周各自为政。赧王将都城迁到西周。

\begin{yuanwen}
西周武公之共太子死,有五庶子,毋適立。司马翦谓楚王曰:“不如以地资公子咎,为请太子。”

左成曰:“不可。周不听,是公之知困而交疏于周也。不如请周君孰欲立,以微告翦,翦请令楚之以地。”果立公子咎为太子。
\end{yuanwen}

西周武公的共太子去世,他有五个庶子,没有适合立为太子的。司马翦对楚王说:“不如用土地来资助公子咎,为他请求太子之位。”

左成说:“不可以。周国不会听从,这样您就知道自己处于困境,并且和周国的关系会变得疏远了。不如去问周君想要立谁为太子,然后悄悄告诉给司马翦,司马翦再请楚国资助土地。”西周君果然立公子咎为太子。

\begin{yuanwen}
八年,秦攻宜阳,楚救之。而楚以周为秦故,将伐之。苏代为周说楚王曰:“何以周为秦之祸也?言周之为秦甚于楚者,欲令周入秦也,故谓‘周秦’也。周知其不可解,必入于秦,此为秦取周之精者也。为王计者,周于秦因善之,不于秦亦言善之,以疏之于秦。周绝于秦,必入于郢\footnote{楚都,代指楚国。}矣。”
\end{yuanwen}

八年(前307年),秦国攻打宜阳,楚国发兵营救。而楚国因为周国帮助秦国的缘故,将要攻打周国。苏代为周国劝楚王说:“为什么认为周国帮助秦国是祸患呢?声称周国帮助秦国比为楚国出力更多的人,是想让周国加入秦国的阵营,所以有‘周国帮助秦国’的说法。周国知道自己无法解脱,就一定会加入秦国的阵营,这正是帮助秦国攻取周国的妙计。我为大王考虑,周国为秦出力要善待,不为秦国出力也要善待,这样才能让周国和秦国疏远。周国和秦国绝交,就一定会加入楚国的阵营了。”

\begin{yuanwen}
秦借道两周之间,将以伐韩,周恐借之畏于韩,不借畏于秦。史厌谓周君曰:“何不令人谓韩公叔曰‘秦之敢绝周而伐韩者,信东周也。公何不与周地,发质使之楚’?秦必疑楚不信周,是韩不伐也。又谓秦曰‘韩强与周地,将以疑周于秦也,周不敢不受’。秦必无辞而令周不受,是受地于韩而听于秦。”
\end{yuanwen}

秦国在东西二周之间借道,将要讨伐韩国,东周担心借道会得罪韩国,不借道又会得罪秦国。史厌对东周君说:“为什么不派人对韩国的公叔说:‘秦国敢从东周借道去攻打韩,是因为相信东周。您为什么不给东周一些土地,派人质去楚国呢?’这样秦国就一定会怀疑楚国,不再相信东周,这样就不会攻打韩了。您再派人对秦人说:‘韩国强行给东周一些土地,将要用这样的办法使秦国怀疑东周,东周不敢不接受。’秦国一定没有让东周不接受土地的说辞,这样既得到了韩的土地,又听从了秦国的命令。”

\begin{yuanwen}
秦召西周君,西周君恶往,故令人谓韩王曰:“秦召西周君,将以使攻王之南阳也,王何不出兵于南阳?周君将以为辞于秦。周君不入秦,秦必不敢逾河而攻南阳矣。”
\end{yuanwen}

秦王召见西周君,西周君不愿意去,因此派人对韩王说:“秦国召见西周君,想要派他去攻打大王的南阳,大王为什么不出兵到南阳呢?西周君将会以此为不去秦国的说辞。西周君不去秦国,秦国一定不敢越过黄河去攻打南阳了。”

\begin{yuanwen}
东周与西周战,韩救西周。或为东周说韩王曰:“西周故天子之国,多名器重宝。王案兵毋出,可以德东周,而西周之宝必可以尽矣。”
\end{yuanwen}

东周和西周交战,韩国救援西周。有人为东周劝韩王说:“西周原来是天子的都城,有很多名贵的器物和珍宝。大王按兵不动,可以使东周感恩,而西周的珍宝也可以全部得到了。”

\begin{yuanwen}
王赧谓成君。楚围雍氏,韩徵甲与粟于东周,东周君恐,召苏代而告之。代曰:“君何患于是。臣能使韩毋徵甲与粟于周,又能为君得高都。”

周君曰:“子苟能,请以国听子。”

代见韩相国曰:“楚围雍氏,期三月也,今五月不能拔,是楚病也。今相国乃徵甲与粟于周,是告楚病也。”

韩相国曰:“善。使者已行矣。”

(五)代曰:“何不与周高都?”

韩相国大怒曰:“吾毋徵甲与粟于周亦已多矣,何故与周高都也?”

代曰:“与周高都,是周折而入于韩也,秦闻之必大怒忿周,即不通周使,是以弊高都得完周也。曷为不与?”

相国曰:“善。”果与周高都。
\end{yuanwen}

赧王被称为成君。楚国围攻雍氏,韩国在东周征用甲胄和粮食,东周君非常恐慌,就召见苏代并告诉他这些事情。苏代说:“您为什么要为这件事担忧呢?我可以让韩国不在东周征用甲胄和粮食,又能让您得到高都。”

东周君说:“您如果能做到这些,我就率领全国的人听命于您。”

苏代见韩国的相国说:“楚国围攻雍氏,约定三个月攻下,现在五个月也没能攻取,楚国为此而担心。现在相国竟然在东周征用甲胄和粮食,这是在把自己的担心告诉楚国。”

韩相国说:“很好。但是使者已经出发了。”

苏代说:“为什么不把高都送给东周?”

韩相国大怒说:“我不在东周征用甲胄和粮食已经做出很大让步了,为什么还要把高都送给东周呢?”

苏代说:“送给东周高都,这样东周就会转而加入韩国的阵营,秦国听说以后一定对东周非常愤恨,就不会和东周互通使节,这是用破败的高都来换取完整的东周。为什么不给呢?”

韩相国说:“很好。”果然把高都给了东周。

吴汝纶:「周事兴于仁义,亡于积弱,自成康以前叙其盛,由积善累仁;自幽厉以后叙其衰,即以政由方伯,摄起强侯行政,以为卒亡于秦作势。上下八百年始末,具于一篇之中,体势闳远。」

\begin{yuanwen}
三十四年,苏厉谓周君\footnote{据《战国策》,此为西周君。}曰:“秦破韩、魏,扑师武,北取赵蔺\footnote{lìn}、离石者,皆白起也。是善用兵,又有天命。今又将兵出塞攻梁,梁破则周危矣。君何不令人说白起乎?曰‘楚有养由基者,善射者也。去柳叶百步而射之,百发而百中之。左右观者数千人,皆曰善射。有一夫立其旁,曰“善,可教射矣”。养由基怒,释弓搤\footnote{è,同“扼”,握。}剑,曰“客安能教我射乎?”客曰“非吾能教子支左诎右\footnote{左右持弓,右手拉弦。}也。夫去柳叶百步而射之,百发而百中之,不以善息,少焉气衰力倦,弓拨矢钩,一发不中者,百发尽息”。今破韩、魏,扑师武,北取赵蔺、离石者,公之功多矣。今又将兵出塞,过两周,倍\footnote{通“背”。}韩,攻梁,一举不得,前功尽弃。公不如称病而无出’。”
\end{yuanwen}

三十四年(前281年),苏厉对西周君说:“秦国打败韩、魏两国的军队,杀死师武,在北面取得赵国的蔺、离石等地,都是依靠白起。这个人善于用兵,又有天命相助。现在他又要率兵出塞去攻打大梁,如果大梁被攻破,那么周都也危险了。您为什么不派人去劝说白起呢?说:“楚国有一个名叫养由基的人,他非常擅长射箭。射击距离百步的柳叶,百发百中。每次他身边都有几千人围观,都说他擅长射箭。有一个男子站在他的身旁,说:”很好,可以让我教你射箭了。”养由基大怒,放下弓,握住剑,说:‘客人凭什么教我射箭呢?’客人说:‘不是我能教您左手持弓、右手拉弦的技术。距离百步的柳叶,百发百中,不妥善休整,不久就会筋疲力尽,无法瞄准,只要有一发不能射中,就会前功尽弃。’现在打败韩、魏两国的军队,杀死师武,在北面取得赵国的蔺、离石,您的功劳已经很多了。现在您又要带兵出塞,经过东西二周,背靠韩国,去攻打大梁,只要有一战不能取胜,就会前功尽弃。您不如称病而不出战。”

\begin{yuanwen}
四十二年,秦破华阳约\footnote{险隘。}。马犯谓周君曰:“请令梁\footnote{指魏国。}城周。”

乃谓梁王曰:“周王病若死,则犯必死矣。犯请以九鼎自入于王,王受九鼎而图\footnote{谋。}犯。”

梁王曰:“善。”遂与之卒,言戍周。

因谓秦王曰:“梁非戍周也,将伐周也。王试出兵境以观之。”秦果出兵。

又谓梁王曰:“周王病甚矣,犯请后可而复之。今王使卒之周,诸侯皆生心,后举事且不信。不若令卒为周城,以匿事端。”

梁王曰:“善。”遂使城周。
\end{yuanwen}

四十二年(前273年),秦国攻破了华阳险隘。马犯对西周君说:“请让魏国为周国修筑城墙。”

于是他对魏王说:“周王如果因为担忧而死去,那么我也一定会死了。我请求亲自把九鼎献给大王,大王接受了九鼎就要考虑我所说的事情。”

魏王说:“好。”于是他给马犯士兵,说是去戍守周都。

马犯趁机又对秦王说:“魏国不是要戍守周都,而是要攻打周都。大王可以试着派兵出境去观察他们的动向。”秦国果然出兵了。

马犯又对魏王说:“周王的病情已经非常严重了,我所请求的事情以后再说。现在大王派兵到周都,诸侯都会心生疑虑,以后再有行动也不会被信任了。不如让士兵为周国修筑城墙,来掩盖事情的初衷。”

魏王说:“好。”于是他派兵修筑周都的城墙。

\begin{yuanwen}
四十五年,周君之秦,客谓周冣\footnote{zuì}曰:“公不若誉\footnote{称赞。}秦王之孝,因以应为太后养地\footnote{又称“汤沐邑”,是贵族的私人领地,以其赋税供食宿、斋戒、沐浴使用,但没有行政权,不属于正式的封地。},秦王必喜,是公有秦交。交善,周君必以为公功。交恶,劝周君入秦者必有罪矣。”

秦攻周,而周冣谓秦王曰:“为王计者不攻周。攻周,实不足以利,声畏天下。天下以声畏秦,必东合于齐。兵弊于周。合天下于齐,则秦不王矣。天下欲弊秦,劝王攻周。秦与天下弊,则令不行矣。”
\end{yuanwen}

四十五年(前270年),西周君到秦国,一个宾客对周冣说:“您不如去赞美秦王的孝心,趁机将应邑送给太后做她的养地,秦王一定会非常高兴,这样您就和秦国有了交情。关系好,西周君一定会认为是您的功劳;关系不好,劝说西周君来秦国的人就一定会获罪了。”

秦国攻打周都,而周冣对秦王说:“我为大王考虑,还是不要攻打周都。攻打周都,实际上并不能获得足够的利益,却会让全天下害怕秦国的威名,全天下因为这种威名而害怕秦国,就一定会联合齐国。由于攻打周都而消耗兵力,让全天下都与齐国联合,那么秦国就不能够称王了。全天下都想消耗秦国的实力,就会劝大王攻打周都。秦国顺应全天下的想法而被削弱,那么号令就很难通行了。”

\begin{yuanwen}
五十八年,三晋距秦。周令其相国之秦,以秦之轻也,还其行。客谓相国曰:“秦之轻重未可知也。秦欲知三国之情。公不如急见秦王曰‘请为王听东方之变’,秦王必重公。重公,是秦重周,周以取秦也;齐重,则固有周聚\footnote{即周冣。}以收齐。是周常不失重国之交也。”

秦信周,发兵攻三晋。
\end{yuanwen}

五十八年(前257年),赵、魏、韩三国共同抵抗秦国。周国派其相国到秦国,由于秦国轻视周国,因此又返了回来。有一个宾客对相国说:“秦国的态度还无法知道。秦国很想知道三国的情况。您不如马上去见秦王对他说:‘请让我为大王去刺探东方各国的情况。’秦王就一定会重视您。重视您,就相当于秦国重视周国,周国因此可以和秦国亲近;齐国重视周国,那么自有周冣来和齐国亲善。这样周国就可以永远和大国保持交情了。”

秦国信任周国,发兵攻打赵、魏、韩三国。

\begin{yuanwen}
五十九年,秦取韩阳城、负黍,西周恐,倍秦,与诸侯约从\footnote{zòng,同“纵”,合纵,即联合抗秦。},将天下锐师\footnote{精锐之师。}出伊阙攻秦,令秦无得通阳城。秦昭王怒,使将军摎\footnote{jiū}攻西周。西周君奔秦,顿首受罪,尽献其邑三十六,口三万。秦受其献,归其君于周。
\end{yuanwen}

五十九年(前256年),秦国夺取了韩国的阳城、负黍,西周非常恐慌,背叛了秦国,和诸侯联合,发动天下的精锐士兵从伊阙山出兵攻打秦国,使秦兵无法到达阳城。秦昭王大怒,派将军摎攻打西周。西周君逃到秦国,叩首认罪,将治下三十六座城邑,人口三万全部献上。秦国接受了西周君的进献,把他放回周地。

\begin{yuanwen}
周君、王赧卒,周民遂东亡。秦取九鼎宝器,而迁西周公于𢠸\footnote{d\`an}狐。后七岁,秦庄襄王灭东周。东西周皆入于秦,周既不祀。
\end{yuanwen}

西周君、赧王去世,周地的民众都向东方逃去。秦国得到九鼎等宝物,把西周君迁到𢠸狐。七年以后,秦庄襄王灭掉东周。东西二周全部被秦国吞并,从此周朝的祭祀断绝。

\begin{yuanwen}
太史公曰:学者皆称周伐纣,居洛邑,综其实不然。武王营之,成王使召公卜居,居九鼎焉,而周复都丰、镐。至犬戎败幽王,周乃东徙于洛邑。所谓“周公葬于毕”,毕在镐东南杜中。秦灭周。汉兴九十有馀载,天子将封\footnote{帝王在泰山祭天的典礼。}泰山,东巡狩至河南,求周苗裔,封其后嘉三十里地,号曰周子南君,比列侯,以奉其先祭祀。
\end{yuanwen}

太史公说:学者都说周讨伐纣,把都城建在洛邑,综上所述,其实并不是这样。武王营建洛邑,成王派召公以占卜来选择地点,把九鼎安放在那里,可是周王室仍然定都丰、镐。到犬戎打败幽王,周王室才将都城东迁到洛邑。人们所说的“周公葬在毕”,毕在镐京东南的杜中。秦国灭掉周朝。汉朝建立九十多年以来,天子将要在泰山祭天,向东巡视到河南,访求周朝的后裔,封其后人姬嘉三十里的土地,封号是周子南君,级别相当于列侯,来主持他祖先的祭祀。

\begin{yuanwen}
后稷居邰,太王作周。丹开雀录,火降乌流。三分既有,八百不谋。苍兕誓众,白鱼入舟。太师抱乐,箕子拘囚。成康之日,政简刑措。南巡不还,西服莫附。共和之后,王室多故。檿弧兴谣,龙漦作蠹。穨带荏祸,实倾周祚。
\end{yuanwen}

\part{卷五}

\chapter{秦本纪第五}

归有光:「《秦本纪》方成一篇文字,秦以前本纪,旧史皆亡,故多凑合。秦虽暴乱,而史职不废,太史公当时有所本也。……又《史记》五帝三代本纪零碎,《秦纪》便好,盖秦原有史,故文字佳。《赵世家》文字周详,亦赵有史,其他想无全书故也。」

本篇记述了秦国从兴起到发展壮大的历史过程,主要取材于已经失传的史书《秦记》。秦国自襄公开始为诸侯,穆公时与晋国争霸,其后一度衰落,到献公、孝公时重新振兴,到昭襄王时完全显露吞并天下的气势,秦始皇即位后终于完成统一大业。由于秦国灭东西二周,最终兼并天下,因此司马迁以其为正统,将秦国列入本纪。

\begin{yuanwen}
秦之先,帝颛顼之苗裔孙曰女(脩)修。女脩织,玄鸟\footnote{《殷本纪》引述《诗经》中“天命玄鸟”的殷人起源传说,秦人起源也有类似传说,综合文献资料和甘肃清水李崖遗址等地的最新考古发现来推测,秦人可能源于一支西迁的东夷民族。}陨卵,女脩吞之,生子大业。大业取少典之子,曰女华。女华生大费,与禹平水土。已成,帝锡\footnote{赐。}玄圭。禹受曰:“非予能成,亦大费为辅。”

帝舜曰:“咨尔费,赞禹功,其赐尔皂游\footnote{旌旗的旒苏,代指旌旗。}。尔后嗣将大出。”乃妻之姚姓之玉女。大费拜受,佐舜调驯鸟兽,鸟兽多驯服,是为柏翳\footnote{也作“伯益”。}。舜赐姓嬴氏。
\end{yuanwen}

秦人的祖先,是帝颛顼的后裔叫女脩。女脩纺织的时候,有一只玄鸟生蛋落下来,女脩就把它吃了,生下儿子大业。大业娶少典的女儿为妻,名叫女华。女华生下大费,和禹一起治理水土。治水成功以后,帝舜赐给禹玄圭。禹接受说:“治水不是靠我一个人就能成功的,也得力于大费的辅佐。”

帝舜说:“大费,我要告诉你,你帮助禹建立功勋,我要赐给你黑色的旌旗。你的后世子孙将会兴旺。”于是帝舜把一个姚姓的美女嫁给大费。大费叩拜并接受赏赐,从此辅佐舜驯养鸟兽,鸟兽大多被他所驯服,他就是柏翳。舜赐姓嬴氏。

\begin{yuanwen}
大费生子二人:一曰大廉,实鸟俗氏;二曰若木,实费氏。其\footnote{指继承费氏的若木。}玄孙曰费昌,子孙或在中国,或在夷狄。费昌当夏桀之时,去夏归商,为汤御,以败桀于鸣条。大廉玄孙曰孟戏、中衍,鸟身人言。帝太戊\footnote{即商中宗。}闻而卜之使御,吉,遂致使御而妻之。自太戊以下,中衍之后,遂世有功,以佐殷国,故嬴姓多显,遂为诸侯。
\end{yuanwen}

大费生了两个儿子:一个叫大廉,是鸟俗氏;另一个叫若木,是费氏。若木的玄孙名叫费昌,有的子孙在中原,有的在夷狄。费昌正处在夏桀统治的时期,于是他离开夏投奔商,为汤驾车,在鸣条打败了桀。大廉的玄孙名叫孟戏、中衍,长着鸟的身体,说着人的语言。帝太戊听说了,经过占卜让他为自己驾车,占卜的结果是吉,于是让他驾车,并为其娶妻。自太戊以下诸帝,中衍的子孙,世代都有功劳,以此辅佐殷商,所以嬴姓有很多人身居显位,最终成为诸侯。

\begin{yuanwen}
其玄孙曰中潏\footnote{yù},在西戎,保西垂\footnote{边疆。}。生蜚廉。蜚廉生恶来。恶来有力,蜚廉善走,父子俱以材力事殷纣。周武王之伐纣,并杀恶来。是时蜚廉为纣石\footnote{应作“使”,出使。}北方,还,无所报,为坛霍太山而报,得石棺,铭曰“帝令处父\footnote{蜚廉的字。}不与殷乱,赐尔石棺以华氏\footnote{光大氏族。}”。死,遂葬于霍太山。蜚廉复有子曰季胜。季胜生孟增。孟增幸于周成王,是为宅皋狼。皋狼生衡父,衡父生造父。造父以善御\footnote{驾车。}幸于周缪王\footnote{即周穆王。},得骥、温骊、骅骝(駵)、騄耳之驷,西巡狩,乐而忘归。徐偃王作乱,造父为缪王御,长驱归周,一日千里以救乱。缪王以赵城封造父,造父族由此为赵氏。自蜚廉生季胜已下五世至造父,别居赵。赵衰其后也。恶来革者,蜚廉子也,蚤死。有子曰女防。女防生旁皋,旁皋生太几,太几生大骆,大骆生非子。以造父之宠,皆蒙赵城,姓赵氏\footnote{《秦始皇本纪》称秦始皇生于赵国邯郸,“及生,名为政,姓赵氏”,使人误认为其以出生地为氏。据《秦本纪》可知,秦国公族为赵氏,始于造父,并非只有秦始皇一人以赵为氏。}。
\end{yuanwen}

中衍的玄孙名叫中潏,生活在西戎地区,保卫西部边疆。他生了蜚廉,蜚廉又生了恶来。恶来很有力气,蜚廉擅长跑步,父子都凭借自己的才能事奉殷纣。周武王伐纣的时候,一并杀死恶来。这时蜚廉正为纣出使北方,他返回后,无处禀报,就在霍太山筑起祭坛向纣禀报,他找到一个石棺,上面的铭文是“天帝让处父不参与殷商的战乱,赐给你石棺来光大氏族”。蜚廉死后,他就被安葬在霍太山。蜚廉还有一个儿子名叫季胜。季胜生下孟增。孟增受到周成王的重用,这就是宅皋狼。宅皋狼生下衡父,衡父生下造父。造父因为擅长驾车而受到周缪王的重用,得到赤骥、温丽、骅駵、騄耳等骏马,于是驾车到西方去巡视,高兴得忘记返回。徐偃王发动叛乱,造父为周缪王驾车,快速从远方赶回周都,一天走一千里路来挽救乱局。周缪王把赵城封给造父,造父的族人从此为赵氏。自蜚廉生下季胜以后五代人一直到造父,才分出住在赵城。赵衰就是他的后人。恶来革,也是蜚廉的儿子,很早就死了。他有一个儿子名叫女防。女防生下旁皋,旁皋生下太几,太几生下大骆,大骆生下非子。因为造父受到周王室的宠幸,都得以住在赵城,姓赵氏。

\begin{yuanwen}
非子居犬丘,好马及畜,善养息之。犬丘人言之周孝王,孝王召使主马于汧\footnote{qiān}渭之闲(间),马大蕃息\footnote{繁衍。}。孝王欲以为大骆適\footnote{通“嫡”。}嗣。申侯之女为大骆妻,生子成为適。申侯乃言孝王曰:“昔我先郦山之女,为戎胥轩妻,生中潏\footnote{shù},以亲故归周,保西垂,西垂以其故和睦。今我复与大骆妻,生適子成。申骆重婚\footnote{两次联姻。},西戎皆服,所以为王。王其图之。”

于是孝王曰:“昔伯翳为舜主畜,畜多息,故有土,赐姓嬴。今其后世亦为朕息马,朕其分土为附庸\footnote{附属于大国的小国,等级低于诸侯。}。”邑之秦,使复续嬴氏祀,号曰秦嬴。亦不废申侯之女子为骆適者,以和西戎。
\end{yuanwen}

非子居住在犬丘,喜好马匹和其他牲畜,他擅长饲养繁育牲畜。犬丘人告诉周孝王这件事,周孝王就召见非子,让他在汧水和渭水之间主管饲养马匹,马匹大量繁殖。周孝王想让非子做大骆的继承人。申侯的女儿嫁给大骆为妻,生下儿子成是大骆的嫡子。申侯于是向周孝王进言说:“以前我的祖先娶骊山氏的女子为妻,生下的女儿嫁给戎胥轩为妻,生下中潏,因为亲近的缘故归顺了周朝,保卫西部边疆,西部边疆因此和睦。现在我再次把女儿嫁给大骆,生下嫡子成。申氏和大骆两次联姻,都让西戎归服王室,大王才得以做天子。大王请慎重考虑。”

于是周孝王说:“以前柏翳为舜主管牲畜,牲畜大量繁殖,所以得到封地,获赐嬴姓。现在他的后代也为我驯养马匹,我要分封土地让他做附庸。”周孝王就以秦为封邑赐给非子,让他再次延续嬴姓的祭祀,号称秦嬴。周王室也没有废除申侯的女儿所生大骆的嫡子,以此与西戎和睦相处。

\begin{yuanwen}
秦嬴生秦侯。秦侯立十年,卒。生公伯。公伯立三年,卒。生秦仲。
\end{yuanwen}

秦嬴生下秦侯。秦侯在位十年,去世。秦侯生下公伯。公伯在位三年,去世。公伯生下秦仲。

\begin{yuanwen}
秦仲立三年,周厉王无道,诸侯或叛之。西戎反王室,灭犬丘大骆之族。周宣王即位,乃以秦仲为大夫,诛西戎。西戎杀秦仲。秦仲立二十三年,死于戎。有子五人,其长者曰庄公。周宣王乃召庄公昆弟五人,与兵七千人,使伐西戎,破之。于是复予秦仲后,及其先大骆地犬丘并有之,为西垂大夫。
\end{yuanwen}

秦仲在位三年,周厉王不行正道,有些诸侯背叛王室。西戎反叛王室,灭掉犬丘大骆的家族。周宣王即位以后,就任命秦仲为大夫,讨伐西戎。西戎杀死秦仲。秦仲在位一共二十三年,最后死在西戎。秦仲有五个儿子,最年长的是庄公。周宣王于是召见庄公兄弟五人,给他们士兵七千人,让他们去讨伐西戎,最后打败西戎。于是周宣王再次赏赐秦仲的后人,把当初大骆的领地犬丘一并赏赐给他们,任命庄公为西垂大夫。

\begin{yuanwen}
庄公居其故西犬丘,生子三人,其长男世父。世父曰:“戎杀我大父仲,我非杀戎王则不敢入邑。”遂将击戎,让其弟襄公。襄公为太子。庄公立四十四年,卒,太子襄公代立。襄公元年,以女弟\footnote{妹妹。}缪嬴为丰王妻。

襄公二年,戎围犬丘,世父击之,为戎人所虏。岁余,复归世父。

七年春,周幽王用褒姒废太子,立褒姒子为適,数欺诸侯,诸侯叛之。西戎犬戎与申侯伐周,杀幽王郦山下。而秦襄公将兵救周,战甚力,有功。周避犬戎难,东徙雒邑,襄公以兵送周平王。平王封襄公为诸侯,赐之岐以西之地。曰:“戎无道,侵夺我岐、丰之地,秦能攻逐戎,即有其地。”与誓,封爵之。襄公于是始国,与诸侯通使聘享之礼,乃用駵驹、黄牛、羝羊各三,祠上帝西畤。

十二年,伐戎而至岐,卒。生文公。
\end{yuanwen}

庄公居住在祖先故地西犬丘,生了三个儿子,最年长的叫世父。世父说:“西戎杀死我的祖父秦仲,我除非杀死西戎王,否则不敢进入封邑。”于是他将要去攻打西戎,让位给他的弟弟襄公。襄公成为太子。庄公在位四十四年,去世,太子襄公继位。襄公元年(前777年),襄公把自己的妹妹缪嬴嫁给丰王。

襄公二年(前776年),西戎包围犬丘,世父攻打敌军,被西戎人所俘获。一年多以后,西戎放回世父。襄公七年(前771年)春,周幽王因为宠爱褒姒而废掉太子,改立褒姒的儿子为嫡嗣,多次欺骗诸侯,诸侯背叛了他。西戎的犬戎和申侯讨伐周朝,在骊山下杀死了周幽王。当时秦襄公率领军队救援周王室,作战时出力很多,立下大功。周王室躲避犬戎的祸难,向东迁都到雒邑,襄公率领军队护送周平王。周平王封襄公为诸侯,赏赐给他岐山以西的土地,说:“西戎不行正道,侵占我岐山、丰水一带的土地,秦国如果可以攻打并驱逐西戎,就可以拥有这片土地。”他立下誓言,赐给襄公封地和爵位。襄公从此建立国家,开始和诸侯互通使者,互致聘礼,于是用駵驹、黄牛、羝羊各三只,在西畤祭祀天帝。

十二年(前766年),襄公讨伐西戎时来到岐山,去世。襄公生文公。

\begin{yuanwen}
文公元年,居西垂宫。三年,文公以兵七百人东猎。四年,至汧渭之会,曰:“昔周邑我先秦嬴于此,后卒获为诸侯。”乃卜居之,占曰吉,即营邑之。

十年,初为鄜畤,用三牢\footnote{祭祀或宴饮时牛、羊、猪齐备称三牢,又称太牢。羊、猪各一,称少牢,用于级别低于太牢的祭祀。牢,圈养牲畜的栏,代指祭祀用的牺牲。}。

十三年,初有史以纪事,民多化者。

十六年,文公以兵伐戎,戎败走。于是文公遂收周余民有之,地至岐,岐以东献之周。

十九年,得陈宝。

二十年,法初有三族之罪。

二十七年,伐南山大梓,丰大特。

四十八年,文公太子卒,赐谥为竫\footnote{也作“静”。}公。竫公之长子为太子,是文公孙也。

五十年,文公卒,葬西山。竫公子立,是为宁\footnote{应作“宪”。}公。
\end{yuanwen}

文公元年(前765年),居住在西部边疆的宫中。

三年(前763年),文公率领士兵七百人到东方狩猎。

四年(前762年),文公来到汧水和渭水的交会处,说:“以前周朝把这里赐给我的祖先秦嬴为封邑,后来我国终于得到诸侯的封号。”于是他占卜选择地点,得到吉利的结果,就在这里营建都邑。

十年(前756年),开始修筑鄜畤,用三牢祭祀。

十三年(前753年),开始有史官记录国家大事,很多人民得到教化。

十六年(前750年),文公率领士兵讨伐西戎,西戎人战败逃跑。至此文公终于把周都一带的遗民全部收入秦国,势力范围扩大到岐山,把岐山以东的土地献给周王室。

十九年(前747年),文公在陈仓得到宝物。

二十年(前746年),法律中开始出现诛灭三族的条文。

二十七年(前739年),秦国砍伐南山的大梓树,宰杀丰水中的大公牛。

四十八年(前718年),文公的太子去世,赐谥号为竫公。竫公的长子被立为太子,就是文公的孙子。

五十年(前716年),文公去世,被安葬在西山。竫公的儿子继位,这就是宪公。

\begin{yuanwen}
宁公二年,公徙居平阳。遣兵伐荡社。

三年,与亳战,亳王奔戎,遂灭荡社。

四年,鲁公子翚弑\footnote{臣杀君,子杀父。}其君隐公。

十二年,伐荡氏,取之。

宁公生十岁立,立十二年卒,葬西山。生子三人,长男武公为太子。武公弟德公,同母鲁姬子。生出子\footnote{“生”前脱“王姬”二字。王姬,周天子的女儿。出子,《十二诸侯年表》作“出公”。战国时另有秦惠公之子谥出公。}。宁公卒,大庶长弗忌、威垒、三父废太子而立出子为君。

出子六年,三父等复共令人贼杀\footnote{杀害。}出子。出子生五岁立,立六年卒。三父等乃复立故太子武公。
\end{yuanwen}

宪公二年(前714年),宪公迁都到平阳。他派兵征伐荡社。

三年(前713年),宪公和亳人作战,亳王逃到西戎地区,秦军于是灭掉荡社。

四年(前712年),鲁国的公子翚杀死其国君隐公。

十二年(前704年),秦国征伐荡氏,攻取那里。

宪公十岁时被立为国君,在位十二年去世,安葬在西山。他有三个儿子,长子武公是太子。武公和他的弟弟德公,都是鲁姬的儿子。王姬生了出子。宪公去世以后,大庶长弗忌、威垒、三父废黜太子,改立出子为国君。

出子六年(前698年),三父等人又派人杀死出子。出子五岁时被立为国君,在位六年去世。三父等人于是重新拥立原来的太子武公为国君。

吴汝纶:「此篇为秦有天下作势,通篇趋重末段。有以善御主与分封,见无他功德,襄公得周地,缪公与晋争强,孝公以后与六国争强,皆所以力争天下之渐也。」

\begin{yuanwen}
武公元年,伐彭戏氏,至于华山下,居平阳封宫。

三年,诛三父等而夷三族,以其杀出子也。郑高渠眯杀其君昭公。

十年,伐邽、冀戎,初县\footnote{设立县。}之。

十一年,初县杜、郑。灭小虢。
\end{yuanwen}

武公元年(前697年),秦军讨伐彭戏氏,攻至华山下,武公居住在平阳封宫。

三年(前695年),武公诛杀三父等人,并且诛灭他们的三族,因为他们曾经杀死出子。郑国的高渠眯杀死其国君昭公。

十年(前688年),秦军讨伐邽、冀等地戎人,开始设立县。

十一年(前687年),秦国开始在杜、郑设立县。秦军灭掉小虢。

\begin{yuanwen}
十三年,齐人管至父、连称等杀其君襄公而立公孙无知。晋灭霍、魏、耿。齐雍廪杀无知、管至父等而立齐桓公。齐、晋为强国。
\end{yuanwen}

十三年(前685年),齐国人管至父、连称等人杀死其国君襄公,改立公孙无知。晋国灭掉霍、魏、耿三国。齐国的雍廪杀死公孙无知、管至父等人,改立齐桓公。齐国、晋国成为强大的国家。

\begin{yuanwen}
十九年,晋曲沃\footnote{晋国旁支曲沃氏,见《晋世家》。}始为晋侯。齐桓公伯\footnote{霸,诸侯盟主。}于鄄。
\end{yuanwen}

十九年(前679年),晋国的曲沃氏开始成为晋侯。齐桓公在鄄邑称霸。

\begin{yuanwen}
二十年,武公卒,葬雍平阳。初以人从死,从死者六十六人。有子一人,名曰白,白不立,封平阳。立其弟德公。
\end{yuanwen}

二十年(前678年),武公去世,被安葬在雍邑的平阳。秦国开始用活人殉葬,殉葬的有六十六人。武公有一个儿子,名叫白,白没有继位,被封在平阳。武公的弟弟德公被立为国君。

\begin{yuanwen}
德公元年,初居雍城大郑宫。以牺三百牢\footnote{一百副太牢,即牛、羊、猪各三百。}祠鄜畤。卜居雍。后子孙饮马于河。梁伯、芮伯来朝。

二年,初伏,以狗御蛊。德公生三十三岁而立,立二年卒。生子三人:长子宣公,中子成公,少子穆公。长子宣公立。
\end{yuanwen}

德公元年(前677年),德公开始居住在雍邑的大郑宫。用一百副太牢在鄜畤举行祭祀。通过占卜在雍邑选择地点,得到后世子孙在黄河边饮马的结果。梁伯、芮伯前来朝见。

二年(前676年),秦国开始举行进入伏日的祭祀,用杀狗的办法驱除毒热之气。德公三十三岁时被立为国君,在位二年去世。他有三个儿子:大儿子是宣公,二儿子是成公,小儿子是穆公。长子宣公被立为国君。

\begin{yuanwen}
宣公元年,卫、燕伐周,出惠王,立王子穨。

三年,郑伯、虢叔杀子穨而入惠王。

四年,作密畤。与晋战河阳,胜之。

十二年,宣公卒。生子九人,莫立,立其弟成公。
\end{yuanwen}

宣公元年(前675年),卫国、燕国联合攻伐周朝,驱逐周惠王,王子穨被立为天子。

三年(前673年),郑伯、虢叔杀死王子穨,并且迎接周惠王进入周都。

四年(前672年),秦国修建密畤。秦军和晋军在河阳交战,战胜对方。

十二年(前664年),宣公去世。宣公有九个儿子,没有一人被立为国君,他的弟弟成公被立为国君。

\begin{yuanwen}
成公元年,梁伯、芮伯来朝。齐桓公伐山戎,次于孤竹。
\end{yuanwen}

成公元年(前663年),梁伯、芮伯前来朝见。齐桓公讨伐山戎,驻扎在孤竹。

\begin{yuanwen}
成公立四年卒。子七人,莫立,立其弟缪公\footnote{即穆公。}。
\end{yuanwen}

成公在位四年去世。他有七个儿子,没有一人被立为国君,他的弟弟缪公被立为国君。

\begin{yuanwen}
缪公任好\footnote{缪公的名。}元年,自将伐茅津,胜之。

四年,迎妇于晋,晋太子申生姊也。其岁,齐桓公伐楚,至邵陵。
\end{yuanwen}

穆公任好元年(前659年),亲自率领军队攻打茅津,取得胜利。

四年(前656年),他到晋国迎娶夫人,夫人是晋国太子申生的姐姐。这一年,齐桓公讨伐楚国,攻至邵陵。

\begin{yuanwen}
五年,晋献公灭虞、虢,虏虞君与其大夫百里傒,以璧马赂\footnote{赠送财物。}于虞故也。既虏百里傒,以为秦缪公夫人媵于秦。百里傒亡秦走宛,楚鄙人执之。缪公闻百里傒贤,欲重赎之,恐楚人不与,乃使人谓楚曰:“吾媵臣百里傒在焉,请以五羖\footnote{黑色公羊。}羊皮赎之。”。

楚人遂许与之。当是时,百里傒年已七十余。缪公释其囚,与语国事。

谢曰:“臣亡国之臣,何足问!”

缪公曰:“虞君不用子,故亡,非子罪也。”固问,语三日,缪公大说,授之国政,号曰五羖大夫。

百里傒让曰:“臣不及臣友蹇叔,蹇叔贤而世莫知。臣常游困于齐而乞食铚人,蹇叔收臣。臣因而欲事齐君无知,蹇叔止臣,臣得脱齐难,遂之周。周王子穨好牛,臣以养牛干\footnote{求见。}之。及穨欲用臣,蹇叔止臣,臣去,得不诛。事虞君,蹇叔止臣。臣知虞君不用臣,臣诚私利禄爵,且留。再用其言,得脱,一不用,及虞君难:是以知其贤。”

于是缪公使人厚币迎蹇叔,以为上大夫。
\end{yuanwen}

五年(前655年),晋献公灭掉虞、虢两国,俘获虞君和他的大夫百里傒,这就是晋国把璧玉和骏马赠给虞君的原因。百里傒已经被俘获,晋国就以他为秦缪公夫人陪嫁的奴隶送到了秦国。百里傒从秦国逃到宛,被楚国守边的士兵捉住。缪公听说百里傒贤能,想要用重金把他赎回,又害怕楚国人不同意,于是派人对楚国人说:“我国陪嫁的奴隶百里傒在楚国,请让我国用五张黑羊皮把他赎回来。”

楚国人就同意把百里傒交给秦国了。在这个时候,百里傒已经七十多岁了。缪公将他释放,和他探讨治理国家的问题。

百里傒辞谢说:“我是亡国的臣子,有什么值得探讨的呢!”

缪公说:“虞君不重用您,因此亡国,这不是您的罪过。”缪公执意向他请教,于是两个人探讨了三天,缪公非常高兴,就把国政交给他处理,号称五羖大夫。

百里傒推让说:“我比不上我的朋友蹇叔,蹇叔贤能而世人却不知道。我曾在齐国游历,因困顿向铚邑人乞讨食物,蹇叔收留了我。我当时想去事奉齐君无知,蹇叔阻止了我,我才躲过了齐国的灾难,于是我来到周王室。周朝的王子穨非常喜欢牛,我就通过养牛来求见他。等到王子穨想要任用我的时候,蹇叔又阻止了我,我离开周都,得以不被诛杀。我去事奉虞君,蹇叔还是阻止我。虽然我知道虞君不会重用我,但是我实在贪恋财利、俸禄和爵位,暂且留在那里。我两次听从蹇叔的劝告,两次脱险,只有一次没听从他的劝告,就遇到虞君的灾难,所以我知道他是贤能的人。”

于是缪公派人带着丰厚的礼物去请蹇叔,任命他为上大夫。

\begin{yuanwen}
秋,缪公自将伐晋,战于河曲。晋骊姬作乱,太子申生死新城,重耳、夷吾出奔。
\end{yuanwen}

秋季,缪公亲自率领军队讨伐晋国,在河曲和晋军交战。这时,晋国的骊姬制造祸乱,太子申生死在新城,公子重耳、夷吾逃亡在外。

\begin{yuanwen}
九年,齐桓公会诸侯于葵丘。
\end{yuanwen}

九年(前651年),齐桓公在葵丘大会诸侯。

\begin{yuanwen}
晋献公卒。立骊姬子奚齐,其臣里克杀奚齐。荀息立卓子,克又杀卓子及荀息。夷吾使人请秦,求入晋。于是缪公许之,使百里傒将兵送夷吾。夷吾谓曰:“诚得立,请割晋之河西八城与秦。”

及至,已立,而使丕郑谢秦,背约不与河西城,而杀里克。丕郑闻之,恐,因与缪公谋曰:“晋人不欲夷吾,实欲重耳。今背秦约而杀里克,皆吕甥、郄芮之计也。愿君以利急召吕、郄,吕、郄至,则更入重耳便。”

缪公许之,使人与丕郑归,召吕、郄。吕、郄等疑丕郑有间,乃言夷吾杀丕郑。丕郑子丕豹奔秦,说缪公曰:“晋君无道,百姓不亲,可伐也。”

缪公曰:“百姓苟不便,何故能诛其大臣?能诛其大臣,此其调也。”不听,而阴用豹。
\end{yuanwen}

晋献公去世。国人立骊姬的儿子傒齐为国君,晋国的大臣里克杀死傒齐。荀息立卓子为国君,里克又杀死卓子和荀息。公子夷吾派人到秦国请求帮助,希望回到晋国继位。于是缪公答应了他,派百里傒率领士兵护送夷吾。夷吾对秦的国君说:“如果我真能继位,请求割让晋国河西的八座城邑送给秦国。”

等到他返回晋国,已经继位,就派丕郑到秦国道谢,却违背约定不给秦国河西的八座城邑,并且杀死里克。丕郑听说以后,非常害怕,趁机和秦穆公谋划说:“晋国人不想让夷吾做国君,其实希望重耳回国继位。现在夷吾违背约定,还杀死里克,这都是吕甥、郤芮的计策。希望您用财利召见吕、郤二人,吕、郤二人来了,再护送重耳回晋国就方便了。”

缪公同意了,派人护送丕郑回到晋国,召见吕、郤二人。吕、郤等人怀疑丕郑有离间之言,于是建议夷吾杀死丕郑。丕郑的儿子丕豹逃到秦国,劝缪公说:“晋君不行正道,百姓不亲近他,您可以讨伐晋国。”

缪公说:“百姓如果不亲近他,他为什么能诛杀大臣呢?既然他能诛杀大臣,这就说明他是可以协调国内各种势力。”他没有听从,却暗自重用丕豹。

\begin{yuanwen}
十二年,齐管仲、隰朋死。
\end{yuanwen}

十二年(前648年),齐国的管仲、隰朋死去。

\begin{yuanwen}
晋旱,来请粟。丕豹说缪公勿与,因其饥而伐之。缪公问公孙支,支曰:“饥\footnote{指农作物歉收。}穰\footnote{指丰收。}更事耳,不可不与。”

问百里傒,傒曰:“夷吾得罪于君,其百姓何罪?”

于是用百里傒、公孙支言,卒与之粟。以船漕\footnote{通过水路运粮。}车转,自雍相望至绛。
\end{yuanwen}

晋国发生旱灾,前来请求救济粮食。丕豹劝说缪公不给晋国粮食,利用晋国发生饥荒的时机去讨伐。缪公询问公孙支,公孙支说:“荒年和丰年更替到来罢了,不可以不给他们粮食。”

缪公又问百里傒,百里傒说:“夷吾得罪了您,晋国的百姓有什么罪过呢?”

于是缪公采纳了百里傒和公孙支的建议,最终给了晋国粮食。用船只和车辆运输,从雍邑到绛邑的路上连续不断。

\begin{yuanwen}
十四年,秦饥,请粟于晋。晋君谋之群臣。虢射曰:“因其饥伐之,可有大功。”晋君从之。

十五年,兴兵将攻秦。缪公发兵,使丕豹将,自往击之。

九月壬戌,与晋惠公夷吾合战于韩地。晋君弃其军,与秦争利,还\footnote{通“旋”,盘旋。}而马騺\footnote{马难以起步。}。缪公与麾下驰追之,不能得晋君,反为晋军所围。晋击缪公,缪公伤。于是岐下食善马者三百人驰冒晋军,晋军解围,遂脱缪公而反生得晋君。

初,缪公亡善马,岐下野人\footnote{居住在郊野的人。}共得而食之者三百余人,吏逐得,欲法之\footnote{将他们法办。}。

缪公曰:“君子不以畜产害人。吾闻食善马肉不饮酒,伤人。”乃皆赐酒而赦之。

三百人者闻秦击晋,皆求从,从而见缪公窘,亦皆推锋争死,以报食马之德。于是缪公虏晋君以归,令于国:“齐\footnote{zhāi}宿,吾将以晋君祠上帝。”

周天子闻之,曰“晋我同姓”,为请晋君。夷吾姊亦为缪公夫人,夫人闻之,乃衰\footnote{cuī}绖跣,曰:“妾兄弟不能相救,以辱君命。”

缪公曰:“我得晋君以为功,今天子为请,夫人是忧。”乃与晋君盟,许归之,更舍上舍\footnote{上等房舍。},而馈之七牢。

十一月,归晋君夷吾,夷吾献其河西地,使太子圉为质于秦。秦妻子圉以宗女。是时秦地东至河。
\end{yuanwen}

十四年(前646年),秦国发生饥荒,向晋国借粮。晋君和群臣商量。虢射说:“利用秦国发生饥荒的时机去讨伐,可以建立大功。”晋君采纳了他的建议。

十五年(前645年),晋国征发士兵将要攻打秦国。缪公也调发军队,让丕豹率领,缪公亲自前往迎战。

九月壬戌日,秦军和晋惠公夷吾在韩地交战。晋君抛下自己的军队,独自和秦军争胜,在转弯的时候拉车的马难以起步。缪公和部下驱车追赶,没有捉到晋君,反而被晋国的军队所围困。晋国士兵攻击缪公,缪公受伤。这时岐山下曾经偷吃缪公良马的三百个人驱车冲向晋军,晋军的包围圈解除了,最终使缪公脱险,并且活捉了晋君。

当初,缪公丢失了一匹良马,生活在岐山下的三百多个郊野之人得到这匹马,并且一同把它的肉吃了,官吏最终捉住这些人,想要按照法律惩处他们。

缪公说:“君子不会因为牲畜杀人。我听说吃了良马的肉不喝酒,对人身体有害。”于是他赏赐给这些人酒喝,并且赦免了他们。

这三百个人听说秦国要攻打晋国,都请求跟随军队出征,因此看到缪公处境困窘,也都不顾生死地争相冲锋,来报答缪公宽恕他们偷吃马肉的恩德。

于是缪公俘获晋君返回,号令全国:“斋戒沐浴,我将要用晋君来祭祀天帝。”

周天子听说这件事,说“晋君和我是同姓”,为晋君求情。夷吾的姐姐是缪公的夫人,她听说这件事,就穿上丧服,光着脚,说:“我的兄弟有难,我不能相救,辱没了您的命令。”

缪公说:“我俘虏了晋君,以此为功绩,现在有天子为他求情,夫人也因此担忧。”于是他和晋君订立盟约,同意把他放回去,还让他住在上等的馆舍,用七副太牢的规格招待他。

十一月,秦国放晋君回国,夷吾把河西的土地献给秦国,派太子圉到秦国做人质。秦国把一名宗室女子嫁给了太子圉。这时秦国的领土已经向东扩展到黄河流域。

\begin{yuanwen}
十八年,齐桓公卒。
\end{yuanwen}

十八年(前642年),齐桓公去世。

\begin{yuanwen}
二十年,秦灭梁、芮。
\end{yuanwen}

二十年(前640年),秦国灭掉梁、芮两国。

\begin{yuanwen}
二十二年,晋公子圉闻晋君病,曰:“梁,我母家也,而秦灭之。我兄弟多,即君百岁后,秦必留我,而晋轻,亦更立他子。”子圉乃亡归晋。

二十三年,晋惠公卒,子圉立为君。秦怨圉亡去,乃迎晋公子重耳于楚,而妻以故子圉妻。重耳初谢,后乃受。缪公益礼厚遇之。

二十四年春,秦使人告晋大臣,欲入重耳。晋许之,于是使人送重耳。

二月,重耳立为晋君,是为文公。文公使人杀子圉。子圉是为怀公。
\end{yuanwen}

二十二年(前638年),晋国太子圉听说晋君生病,说:“梁国,是我母亲的娘家,秦国却将其灭掉了。我的兄弟很多,在晋君去世以后,秦国一定会扣留我,而晋国会轻视我,也将改立其他的公子为国君。”太子圉于是逃回晋国。

二十三年(前637年),晋惠公去世,太子圉被立为国君。秦国怨恨太子圉逃回晋国,于是从楚国迎接晋国公子重耳,并且把原来太子圉的妻子嫁给他。重耳最初拒绝,后来还是接受了。缪公用更隆重的礼节厚待重耳。

二十四年(前636年)的春天,秦国派人告诉晋国的大臣,想要护送公子重耳回到晋国。晋国同意了,于是缪公派人护送重耳回到晋国。

二月,重耳被拥立为国君,这就是晋文公。晋文公命人杀死子圉。子圉就是晋怀公。

\begin{yuanwen}
其秋,周襄王弟带以翟伐王,王出居郑。

二十五年,周王使人告难于晋、秦。秦缪公将兵助晋文公入襄王,杀王弟带。

二十八年,晋文公败楚于城濮。

三十年,缪公助晋文公围郑。郑使人言缪公曰:“亡郑厚晋,于晋而得矣,而秦未有利。晋之强,秦之忧也。”缪公乃罢兵归。晋亦罢。

三十二年冬,晋文公卒。
\end{yuanwen}

当年秋季,周襄王的弟弟带勾结翟人攻打周襄王,周襄王出逃到郑国。

二十五年(前635年),周襄王派人到晋国、秦国求救。秦缪公率领士兵帮助晋文公护送周襄王进入国都,杀死周襄王的弟弟带。

二十八年(前632年),晋文公在城濮打败楚国。

三十年(前630年),缪公帮助晋文公围困郑国。郑国派人劝秦穆公说:“灭掉郑国只能对晋国有利,对于晋国是很大的收获,而对于秦国却没有好处。晋国的强大,就是秦国的忧患。”缪公于是撤兵回国。晋国也撤兵了。

三十二年(前628年)冬季,晋文公去世。

\begin{yuanwen}
郑人有卖郑于秦曰:“我主其城门,郑可袭也。”

缪公问蹇叔、百里傒,对曰:“径\footnote{穿行。}数国千里而袭人,希有得利者。且人卖郑,庸知我国人不有以我情告郑者乎?不可。”

缪公曰:“子不知也,吾已决矣。”遂发兵,使百里傒子孟明视,蹇叔子西乞术及白乙丙将兵。

行日,百里傒、蹇叔二人哭之。

缪公闻,怒曰:“孤\footnote{先秦帝王、诸侯常用的自称之一。《老子》说:“人之所恶,惟孤寡不穀,而王公以为称。”孤,指年幼丧父之人,王公多为丧父后继位,所以自称“孤”。寡人,指寡德之人。不谷,指不善之人。都是自谦之辞。}发兵而子沮\footnote{败坏士气。}哭吾军,何也?”

二老曰:“臣非敢沮君军。军行,臣子与往;臣老,迟还恐不相见,故哭耳。”

二老退,谓其子曰:“汝军即败,必于肴\footnote{xiáo,指崤山。}阨矣。”

三十三年春,秦兵遂东,更\footnote{经过。}晋地,过周北门。周王孙满曰:“秦师无礼,不败何待!”

兵至滑,郑贩卖贾人弦高,持十二牛将卖之周,见秦兵,恐死虏,因献其牛,曰:“闻大国将诛郑,郑君谨修守御备,使臣以牛十二劳军士。”

秦三将军相谓曰:“将袭郑,郑今已觉之,往无及已。”灭滑。滑,晋之边邑也。
\end{yuanwen}

郑国有个人把郑国出卖给秦国说:“我主管郑国的城门,可以偷袭郑国。”

缪公询问蹇叔、百里傒,两个人回答说:“穿行多个国家,经过上千里路去偷袭别人,很少有获得成功的。况且有人可以出卖郑国,怎么知道我国没有人把我们的事情通报给郑国呢?不可以去。”

缪公说:“二位不知道内情,我已经决定了。”于是他调发士兵,派百里傒的儿子孟明视、蹇叔的儿子西乞术和白乙丙带兵。

军队出发那天,百里傒、蹇叔两个人为将士哭泣。

缪公听到后,生气地说:“我调发士兵,可是你们却在这里大哭来败坏我军的士气,这是为什么?”

两位老臣说:“我们不敢败坏您军队的士气。大军就要出发了,我们的儿子一同去作战;我们老了,他们回来晚恐怕就不能相见了,所以痛哭罢了。”

两位老臣退下,对他们的儿子说:“你们的军队将会战败,一定是在崤山的险要地带了。”

三十三年(前627年)的春季,秦国的军队于是向东进发,经过晋国的领地,经过周都的北门。周朝的王孙满说:“秦国的军队不讲礼法,不失败还等什么呢?”

秦国的军队来到滑国,遇到贩运货物的郑国商人弦高,他正赶着十二头牛打算到周都去贩卖,看到秦国的士兵,害怕被抓住杀死,就把牛献给秦军,说:“听说大国将惩罚郑国,郑君已经谨慎地做好了防备,派我献上十二头牛来犒劳将士。”

秦国的三位将军相互商量说:“我们将要偷袭郑国,郑国现在已经察觉了,我们赶过去也没有机会了。”于是秦军灭掉滑国。滑国,是晋国边境的一个小国。

\begin{yuanwen}
当是时,晋文公丧尚未葬。太子襄公怒曰:“秦侮我孤,因丧破我滑。” 遂墨衰绖\footnote{把丧服染成黑色。军中不宜穿丧服,不穿丧服又违背孝道,所以把丧服染成黑色,是一种变通的做法。},发兵遮秦兵于殽,击之,大破秦军,无一人得脱者。虏秦三将以归。

文公夫人\footnote{晋襄公嫡母,并非生母。},秦女也,为秦三囚将请曰:“缪公\footnote{“缪”是秦君任好死后的谥号,此处称“缪公”是后世史官的追述。从下文可知,此处应称“我君”。}之怨此三人入于骨髓,愿令此三人归,令我君得自快烹之。”

晋君许之,归秦三将。三将至,缪公素服郊迎,向三人哭曰:“孤以不用百里傒、蹇叔言以辱三子,三子何罪乎?子其悉心雪耻,毋怠。”遂复三人官秩\footnote{官职和俸禄。}如故,愈益厚之。
\end{yuanwen}

在这个时候,晋文公的遗体还没有安葬。太子襄公生气地说:“秦国欺负我这个孤儿,利用国内有丧事的时机攻破我边境的滑国。”于是他把丧服染成黑色,调发士兵在崤山截击秦军,发起攻击,大败秦军,秦国的士兵没有一个人得以逃脱。晋国俘虏秦国的三位将军后返回。

晋文公的夫人,是秦国女子,她为被俘虏的三位秦军将领求情说:“我君怨恨这三个人已经深入骨髓,希望把这三个人放回秦国,让我君亲自烹杀他们来解恨。”

晋君同意了,于是释放三位将领回国。三位将军回到秦国,缪公就穿着白色的衣服到郊外迎接,面对三位将军哭着说:“我因为没有听从百里傒、蹇叔的话而让三位蒙受耻辱,三位哪里有罪过呢?你们准备洗雪耻辱,不要懈怠。”于是他恢复三个人的官职和俸禄,更加厚待他们。

\begin{yuanwen}
三十四年,楚太子商臣\footnote{即楚穆王。}弑其父成王代立。缪公于是复使孟明视等将兵伐晋,战于彭衙。秦不利,引兵归。
\end{yuanwen}

三十四年(前626年),楚国的太子商臣杀死他的父亲成王,自立为楚王。缪公在这一年再次命令孟明视等人带兵攻打晋国,在彭衙和晋军交战。秦军没有获胜,领兵撤回。

\begin{yuanwen}
戎王使由余于秦。由余,其先晋人也,亡入戎,能晋言。闻缪公贤,故使由余观秦。

秦缪公示以宫室、积聚。由余曰:“使鬼为之,则劳神矣。使人为之,亦苦民矣。”

缪公怪之,问曰:“中国以诗书、礼乐、法度为政,然尚时乱,今戎夷无此,何以为治,不亦难乎?”

由余笑曰:“此乃中国所以乱也。夫自上圣黄帝作为礼乐、法度,身以先之,仅以小治。及其后世,日以骄淫。阻法度之威,以责督\footnote{责罚监督。}于下,下罢\footnote{pí}极则以仁义怨望于上,上下交争怨而相篡弑,至于灭宗,皆以此类也。夫戎夷不然。上含淳德以遇其下,下怀忠信以事其上,一国之政犹一身之治,不知所以治,此真圣人之治也。”

于是缪公退而问内史廖曰:“孤闻邻国有圣人,敌国之忧也。今由余贤,寡人之害,将奈之何?”

内史廖曰:“戎王处辟匿,未闻中国之声。君试遗其女乐,以夺其志;为由余请,以疏其间;留而莫遣,以失其期。戎王怪之,必疑由余。君臣有间,乃可虏也。且戎王好乐,必怠于政。”

缪公曰:“善。”因与由余曲席\footnote{坐席相连,指不分上下。}而坐,传器而食,问其地形与其兵势尽詧\footnote{同“察”,详细了解。},而后令内史廖以女乐二八\footnote{十六。}遗戎王。戎王受而说之,终年不还。于是秦乃归由余。由余数谏不听,缪公又数使人间要由余,由余遂去降秦。缪公以客礼礼之,问伐戎之形。
\end{yuanwen}

戎王派由余出使秦国。由余,他的祖先是晋国人,逃到西戎,会说晋国的语言。戎王听说缪公贤德,因此派由余到秦国去考察。

秦缪公让由余参观了秦国的宫室、储备。由余说:“如果让鬼神来打造,他们也会觉得劳心了;如果让人去制作,百姓也会很辛苦了。”

缪公感到奇怪,问道:“中原用诗书、礼乐、法度来治理国家,然而还是经常有混乱局面,现在戎夷没有这些,用什么治理国家,不也是很困难吗?”

由余笑着说:“这就是中原各国出现混乱局面的原因。自从上古的圣人黄帝制定了礼乐、法度,他率先遵行,只能实现小规模的安定局面。等到后世,逐渐骄奢淫逸。为政者自己阻碍了法度的威严,却用法度责罚监督下级,下级疲困到极点就会怨恨上级不行仁义,上下相互争斗怨恨,并且篡位弑君,甚至诛灭宗族,都是这样的情况。而戎夷就不会这样。上级能够用淳厚的美德来对待下级,下级能够用忠诚和信用来事奉上级,一个国家的政治就像一个人的修养,尽管不知道是如何治理的,但这才是真正的圣人之治。”

于是缪公退朝并询问内史廖说:“我听说邻国有圣人,就是敌国的忧患。现在由余的贤能,就是我的祸害,应该怎么办呢?”

内史廖回答说:“戎王身处偏僻闭塞的地方,从没听过中原的音乐。您可以尝试送他一些歌舞艺人,以此削弱他的心志;再为由余向戎王请求,来疏远他们的关系;我们留住他不让他回去,来拖延他回国的期限。戎王怪罪由余,就一定会怀疑他。君臣之间产生嫌隙,我们就可以俘虏戎王了。况且戎王喜好音乐,一定荒废政务。”

缪公说:“很好。”他就和由余席垫相连而坐,共用餐具而食,向他询问西戎的地理形势和兵力配备而全部详细了解,然后派内史廖把十六名歌舞艺人送给戎王。戎王接受后很高兴,一整年也不理朝政。这时秦国才让由余回去。由余多次劝谏,戎王都不听从,缪公又多次派人暗中邀请由余,由余终于归降了秦国。缪公用宾客的礼节招待由余,向他询问讨伐西戎的事情。

\begin{yuanwen}
三十六年,缪公复益厚孟明等,使将兵伐晋,渡河焚船,大败晋人,取王官及鄗,以报殽之役。晋人皆城守不敢出。于是缪公乃自茅津渡河,封殽中尸,为发丧,哭之三日。乃誓于军曰:“嗟,士卒!听无譁,余誓告汝。古之人谋黄髪番番\footnote{指老人。番,pó,通“皤”,白色。},则无所过。”

以申思不用蹇叔、百里傒之谋,故作此誓,令后世以记余过。君子闻之,皆为垂涕,曰:“嗟乎!秦缪公之与人周也,卒得孟明之庆。”
\end{yuanwen}

三十六年(前624年),缪公更加厚待孟明视等人,派他们带兵攻打晋国,渡过黄河就烧毁船只,大败晋军,夺取王官和鄗邑,以此报崤山战败之仇。晋人都固守在城中不敢出来。于是缪公亲自从茅津渡过黄河,安葬了崤山中的尸体,为他们举办丧事,痛哭致哀三天。于是他在军中发布誓词说:“啊,将士们!听着,不要喧哗,我要向你们发布誓词。古时候的人总是会向白发的老人请教,那样就不会犯错。”

他以此申明反思没有听从蹇叔、百里傒建议的错误,所以发布这篇誓词,让后世记住他的错误。君子听说以后,都为此流下眼泪,说:“啊!秦缪公在用人方面考虑得非常周到,最终得到了孟明视取胜的喜讯。”

\begin{yuanwen}
三十七年,秦用由余谋伐戎王,益国十二,开地千里,遂霸西戎。天子使召公过贺缪公以金鼓\footnote{用金属制成的钲,因其外形像鼓,故得名金鼓。}。
\end{yuanwen}

三十七年(前623年),秦国采用由余的计策讨伐戎王,增加了十二国的土地,开拓了一千里的疆域,终于称霸西戎。周天子派召公过带着金鼓向缪公道贺。

\begin{yuanwen}
三十九年,缪公卒,葬雍。从死者百七十七人,秦之良臣子舆氏三人名曰奄息、仲行、鍼虎,亦在从死之中。秦人哀\footnote{同情。}之,为作歌《黄鸟》之诗。君子曰:“秦缪公广地益国,东服强晋,西霸戎夷,然不为诸侯盟主,亦宜哉。死而弃民,收其良臣而从死。且先王崩,尚犹遗德垂法,况夺之善人良臣百姓所哀者乎?是以知秦不能复东征也。”缪公子四十人,其太子嵤代立,是为康公。
\end{yuanwen}

三十九年(前621年),缪公去世,被安葬在雍邑。殉葬的人有一百七十七人,秦国良臣子舆氏的三个人,名叫奄息、仲行、鍼虎,也在殉葬之列。秦国人同情他们,为他们作诗歌《黄鸟》。君子说:“秦缪公开疆拓土,在东方制服强大的晋国,在西方称霸戎夷地区,然而没有成为诸侯盟主,也是合理的啊!他死后就抛弃臣民,让他的良臣殉葬。而且古代圣王去世,还能留下美德和法度,怎么能强迫百姓所同情的好人和良臣去殉葬呢?因此可以知道秦国不会再向东征讨了。”缪公有四十个儿子,太子嵤继位,这就是康公。

\begin{yuanwen}
康公元年。往岁缪公之卒,晋襄公亦卒;襄公之弟名雍,秦出也,在秦。晋赵盾欲立之,使随会来迎雍,秦以兵送至令狐。晋立襄公子\footnote{名夷皋,即晋灵公。}而反击秦师,秦师败,随会来奔。

二年,秦伐晋,取武城,报令狐之役。

四年,晋伐秦,取少梁。

六年,秦伐晋,取羁马。战于河曲,大败晋军。晋人患\footnote{忧虑。}随会在秦为乱,乃使魏雠馀详\footnote{通“佯”,假装。}反,合谋会,诈而得会,会遂归晋。

康公立十二年卒,子共公立。
\end{yuanwen}

康公元年(前620年),前一年缪公去世的时候,晋襄公也去世了,晋襄公的弟弟名叫雍,他的母亲是秦国人,他住在秦国。晋国的赵盾想要立他为国君,派随会到秦国迎接雍,秦国派兵护送他到令狐。晋国却立襄公的儿子为国君,反而派兵攻击秦军,秦军战败,随会逃到秦国。

二年(前619年),秦国讨伐晋国,攻取武城,报令狐战败之仇。

四年(前617年),晋国讨伐秦国,攻取少梁。

六年(前615年),秦国讨伐晋国,攻取羁马。秦军与对方在河曲交战,大败晋军。晋国人担心随会在秦国扰乱晋国,于是派魏雠馀假装反叛晋国,和随会合谋,使诈而劫持了随会,随会于是回到晋国。

康公在位十二年去世,他的儿子共公继位。

\begin{yuanwen}
共公二年,晋赵穿弑其君灵公。

三年,楚庄王强,北兵至雒,问周鼎\footnote{楚庄王问王孙满九鼎的大小和轻重,表明其取代周王室号令天下的野心,此时详见《周本纪》、《楚世家》和《左传·宣公三年》。周鼎,相传大禹铸九鼎,象征九州,传至周朝,为王权的象征。}。

共公立五年卒,子桓公立。
\end{yuanwen}

共公二年(前607年),晋国的赵穿杀死其国君灵公。

三年(前606年),楚庄王势力强大,带兵北上来到雒邑,询问九鼎的大小和重量。

共公在位五年去世,他的儿子桓公继位。

\begin{yuanwen}
桓公三年,晋败我一将。

十年,楚庄王服郑,北败晋兵于河上。当是之时,楚霸,为会盟合诸侯。

二十四年,晋厉公初立,与秦桓公夹河而盟。归而秦倍盟,与翟合谋击晋。

二十六年,晋率诸侯伐秦,秦军败走,追至泾而还。

桓公立二十七年卒,子景公立。
\end{yuanwen}

桓公三年(前601年),晋国打败秦国的一位将领。

十年(前594年),楚庄王征服郑国,带兵北上在黄河边打败晋军。在这个时候,楚国称霸,举行会盟召集诸侯。

二十四年(前580年),晋厉公刚继位,和秦桓公隔着黄河订立盟约。桓公返回后秦国就背弃了盟约,和翟人合谋攻打晋国。

二十六年(前578年),晋国带领诸侯攻打秦国,秦军战败逃跑,诸侯的军队追到泾水才返回。

桓公在位二十七年去世,他的儿子景公继位。

\begin{yuanwen}
景公四年,晋栾书弑其君厉公。

十五年,救郑,败晋兵于栎。是时晋悼公为盟主。

十八年,晋悼公强,数会诸侯,率以伐秦,败秦军。秦军走,晋兵追之,遂渡泾,至棫林而还。

二十七年,景公如晋,与平公盟,已而背之。

三十六年,楚公子围弑其君而自立,是为灵王。景公母弟后子鍼有宠,景公母弟富,或谮\footnote{说坏话。}之,恐诛,乃奔晋,车重千乘。

晋平公曰:“后子富如此,何以自亡?”

对曰:“秦公无道,畏诛,欲待其后世乃归。” 

三十九年,楚灵王强,会诸侯于申,为盟主,杀齐庆封。景公立四十年卒,子哀公立。后子复来归秦。
\end{yuanwen}

景公四年(前573年),晋国的栾书杀死其国君厉公。

十五年(前562年),秦国援救郑国,在栎邑将晋军打败。当时晋悼公是诸侯盟主。

十八年(前559年),晋悼公的势力强大,多次和诸侯会盟,率领军队讨伐秦国,打败秦军。秦军战败逃跑,晋军追击,一直渡过泾水,追到棫林才带兵撤回。

二十七年(前550年),景公到晋国,和晋平公订立盟约,不久又背弃了盟约。

三十六年(前541年),楚国的公子围杀死其国君而自立,这就是楚灵王。景公的同母弟后子鍼深受宠信,景公的同母弟富有,就有人说他的坏话,他害怕被诛杀,于是逃到晋国,他的资财装满了一千辆车。

晋平公说:“后子这样富有,为什么要逃亡呢?”

他回答说:“秦君不行正道,我害怕被诛杀,我想等他去世以后再返回秦国。”

三十九年(前538年),楚灵王的势力强大,和诸侯在申地会盟,成为盟主,杀死齐国的庆封。景公在位四十年去世,他的儿子哀公继位。后子鍼重新回到秦国。

\begin{yuanwen}
哀公八年,楚公子弃疾弑灵王而自立,是为平王。

十一年,楚平王来求秦女为太子建妻。至国,女好而自娶之。

十五年,楚平王欲诛建,建亡。伍子胥奔吴。晋公室卑而六卿强,欲内相攻,是以久秦晋不相攻。

三十一年,吴王阖闾与伍子胥伐楚,楚王\footnote{指楚昭王。}亡奔随,吴遂入郢。楚大夫申包胥来告急,七日不食,日夜哭泣。于是秦乃发五百乘救楚,败吴师。吴师归,楚昭王乃得复入郢。

哀公立三十六年卒。太子夷公,夷公蚤死,不得立,立夷公子,是为惠公。
\end{yuanwen}

哀公八年(前529年),楚国的公子弃疾杀死灵王而自立,他就是楚平王。

哀公十一年(前526年),楚平王使人前来求娶秦国女子为太子建的妻子。接回楚国,平王看到秦女长得漂亮,就自己娶了她。

十五年(前522年),楚平王想要诛杀太子建,太子建逃跑。伍子胥逃到吴国。晋国公室衰落而六卿势力强大,他们想要在内部相互争权,所以很长时间秦国和晋国都没有发生战争。

三十一年(前506年),吴王阖闾和伍子胥讨伐楚国,楚昭王逃到随国,吴军于是进入郢。楚国的大夫申包胥来到秦国求救,他七天不吃东西,从早到晚哭泣。于是秦国就调发五百辆兵车去救援楚国,打败吴军。吴军撤退以后,楚昭王才得以重新回到郢。

哀公在位三十六年去世。太子是夷公,夷公死得很早,没能继位,夷公的儿子被立为国君,这就是秦惠公。

\begin{yuanwen}
惠公元年,孔子行鲁相事。

五年,晋卿中行、范氏反晋,晋使智氏、赵简子攻之,范、中行氏亡奔齐。

惠公立十年卒,子悼公立。
\end{yuanwen}

惠公元年(前500年),孔子出任鲁国的傧相。

五年(前496年),晋国六卿中的中行氏、范氏反叛晋国,晋国派智氏、赵简子去攻打两家,范氏、中行氏逃到齐国。

惠公在位十年去世,他的儿子悼公继位。

\begin{yuanwen}
悼公二年,齐臣田乞弑其君孺子,立其兄阳生,是为悼公。

六年,吴败齐师。齐人弑悼公,立其子简公。

九年,晋定公与吴王夫差盟,争长于黄池,卒先吴。吴强,陵中国。

十二年,齐田常\footnote{本名恒,《史记》避汉文帝刘恒讳改其名为常。}弑简公,立其弟平公,常相之。

十三年,楚灭陈。秦悼公立十四年卒,子厉共公立。孔子以悼公十二年卒。
\end{yuanwen}

悼公二年(前489年),齐国大臣田乞杀死其国君孺子,立孺子的哥哥阳生为国君,这就是齐悼公。

六年(前485年),吴军打败齐军。齐国人杀死其国君齐悼公,立他的儿子简公为国君。

九年(前482年),晋定公和吴王夫差举行会盟,在黄池为盟主之位而争执,最后吴国排在前面。吴国势力强大,经常欺凌中原各国。

十二年(前479年),齐国的田常杀死简公,立简公的弟弟平公为国君,田常辅佐他。

十三年(前478年),楚国灭掉陈国。秦悼公在位十四年去世,他的儿子厉共公继位。孔子在秦悼公十二年去世。

\begin{yuanwen}
厉共公二年,蜀人来赂。

十六年,堑\footnote{防御用的壕沟。这里作动词。}河旁。以兵二万伐大荔,取其王城。

二十一年,初县频阳。晋取武成。

二十四年,晋乱,杀智伯,分其国与赵、韩、魏。

二十五年,智开与邑人来奔。

三十三年,伐义渠,虏其王。

三十四年,日食。厉共公卒,子躁公立。

躁公二年,南郑反。

十三年,义渠来伐,至渭南。

十四年,躁公卒,立其弟怀公。
\end{yuanwen}

厉共公二年(前475年),蜀国人来到秦国进献礼物。

十六年(前461年),在黄河边挖堑壕。秦国派士兵两万人攻打大荔戎,攻取大荔王城。

二十一年(前456年),秦国开始在频阳设县。晋国攻取武城。

二十四年(前453年),晋国发生内乱,智伯被杀,赵、韩、魏三家瓜分他的封地。

二十五年(前452年),智开和他封地的属民前来投奔。

三十三年(前444年),秦国攻打义渠戎,俘获义渠王。

三十四年(前443年),发生日食。厉共公去世,他的儿子躁公继位。

躁公二年(前441年),南郑反叛。

十三年(前430年),义渠戎攻打秦国,攻到渭水以南。

十四年(前429年),躁公去世,他的弟弟怀公继位。

\begin{yuanwen}
怀公四年,庶长晁与大臣围怀公,怀公自杀。怀公太子曰昭子,蚤死,大臣乃立太子昭子之子,是为灵公。灵公,怀公孙也。
\end{yuanwen}

怀公四年(前425年),庶长晁和大臣围攻打怀公,怀公自杀。怀公的太子叫昭子,很早就死了,大臣于是立昭子的儿子为国君,这就是灵公。灵公,是怀公的孙子。

\begin{yuanwen}
灵公六年,晋城少梁,秦击之。

十三年\footnote{秦灵公共在位十年。},城籍姑。灵公卒,子献公不得立,立灵公季父悼子,是为简公。简公,昭子之弟而怀公子也。
\end{yuanwen}

秦灵公六年(前419年),晋国在少梁筑城,秦国攻打那里。

十年(前415年),秦国在籍姑筑城。灵公去世,他的儿子献公没有被立为国君,灵公的叔父悼子继位,这就是简公。简公是昭子的弟弟,也是怀公的儿子。

\begin{yuanwen}
简公六年,令吏初带剑。堑洛。城重泉。十六年\footnote{秦简公共在位十五年。}卒,子惠公立。
\end{yuanwen}

简公六年(前409年),秦国开始命令官吏带剑。在洛水挖堑壕。在重泉筑城。十五年(前400年),简公去世,他的儿子惠公继位。

\begin{yuanwen}
惠公十二年,子出子生。

十三年,伐蜀,取南郑。惠公卒,出子立。
\end{yuanwen}

惠公十二年(前388年),他的儿子出子出生。

十三年(前387年),秦国讨伐蜀国,攻取南郑。惠公去世,他的儿子出子继位。

\begin{yuanwen}
出子二年,庶长改迎灵公之子献公于河西而立之。杀出子及其母,沉之渊旁。秦以往者数易君,君臣乖乱\footnote{变乱,混乱。},故晋复强,夺秦河西地。
\end{yuanwen}

出子二年(前385年),庶长改在河西迎接灵公的儿子献公并立他为国君。献公杀死出子和他的母亲,把尸体沉到深渊中。秦国此前多次更换国君,君臣关系紧张混乱,因此晋国的势力再次强大,夺取秦国河西的土地。

\begin{yuanwen}
献公元年,止从死。

二年,城栎阳。

四年正月庚寅,孝公生。

十一年,周太史儋见献公曰:“周故与秦国合而别,别五百岁复合,合十七岁而霸王出。”

十六年,桃冬花。

十八年,雨金栎阳。

二十一年,与晋\footnote{指魏国,下同。}战于石门,斩首六万,天子贺以黼黻\footnote{绣有华美花纹的礼服。}。

二十三年,与魏晋战少梁,虏其将公孙痤。

二十四年,献公卒\footnote{按《六国年表》,秦献公死于少梁之战当年。先秦有新君继位当年改元和逾年改元两种方法,且因诸侯混战而无定制,故《史记》关于战国时期的纪年多有谬误和矛盾之处。},子孝公立,年已二十一岁矣。
\end{yuanwen}

献公元年(前384年),废止殉葬的制度。

二年(前383年),秦国在栎阳筑城。

四年(前381年)正月庚寅日,孝公出生。

十一年(前374年),周朝的太史儋拜见秦献公说:“周朝原来和秦国合一而后来分开,分开五百年以后又会合在一起,合在一起十七年后会有霸主出现。”

十六年(前369年),桃树在冬季开花。

十八年(前367年),栎阳降下金雨。

二十一年(前364年),秦军和晋军在石门交战,斩首六万人,周天子赠送黼黻之服表示祝贺。

二十三年(前362年),秦军和魏军在少梁交战,俘获魏国的将领公孙痤。

二十四年(前361年),献公去世,他的儿子孝公继位,孝公已经二十一岁了。

\begin{yuanwen}
孝公元年,河山\footnote{黄河和崤山。}以东强国六,与齐威、楚宣、魏惠、燕悼\footnote{又称燕文侯。}、韩哀、赵成侯并。淮泗之间小国十余。楚、魏与秦接界。魏筑长城,自郑滨洛以北,有上郡。楚自汉中,南有巴、黔中。周室微,诸侯力政,争相并。秦僻在雍州,不与中国诸侯之会盟,夷翟遇之。孝公于是布惠,振孤寡,招战士,明功赏。下令国中曰:“昔我缪公自岐雍之间,修德行武,东平晋乱,以河为界,西霸戎翟,广地千里,天子致伯,诸侯毕贺,为后世开业,甚光美。会往者厉、躁、简公、出子之不宁,国家内忧,未遑外事,三晋\footnote{指赵、魏、韩三国。}攻夺我先君河西地,诸侯卑秦、丑莫大焉。献公即位,镇抚边境,徙治栎阳,且欲东伐,复缪公之故地,修缪公之政令。寡人思念先君之意,常痛于心。宾客群臣有能出奇计强秦\footnote{使秦强盛。}者,吾且尊官\footnote{提升他的官职。},与之分土。”于是乃出兵东围陕城,西斩戎之獂王。

卫鞅闻是令下,西入秦,因景监求见孝公。
\end{yuanwen}

孝公元年(前361年),黄河和崤山以东的强国有六个,孝公和齐威王、楚宣王、魏惠王、燕悼侯、韩哀侯、赵成侯并称七雄。在淮水和泗水之间还有小国十几个。楚国、魏国和秦国的领土相接。魏国修筑长城,从郑县沿洛水向北延伸,占有上郡。楚国从汉中向南占有巴郡、黔中郡。周王室衰微,诸侯凭借武力对外征伐,争相兼并。秦国地处偏远的雍州,不参与中原诸侯的会盟,被视为夷狄国家。孝公于是施行恩惠,赈济孤寡,招募士兵,严明奖赏。他在全国颁布命令说:“当年我的祖先缪公在岐山和雍州之间,修持文德,建立武功,在东方平定晋国的内乱,以黄河为国界,在西方称霸戎狄地区,扩展疆土一千里,周天子任命他为方伯,诸侯都来庆贺,他为后世开创的基业,是光辉美好的。可是后来的厉公、躁公、简公、出子在位时政治不安宁,国家内部存在忧患,没有时间顾及外部的事务,赵、魏、韩三国攻取了我先君的河西疆土,诸侯鄙视秦国,没有比这更大的耻辱了。献公即位以后,镇守安抚边境,迁都到栎阳,并且想要向东征伐,收复缪公时的土地,重修缪公时的政令。每当我想到先君的遗愿,总是感到心痛。宾客和群臣中,如果有人能够进献妙计使秦国强大,我就提升他的官职,分封给他土地。”于是他派兵向东围困陕城,向西斩杀西戎的獂王。

卫鞅听说秦国发布这一命令以后,就西行来到秦国,通过景监求见孝公。

\begin{yuanwen}
二年,天子致胙。

三年,卫鞅说孝公变法修刑,内务耕稼,外劝战死\footnote{主要指二十等军功爵制度。新法规定,士兵斩获敌人首级,可以得到爵位,无军功者不得继承官爵。二十等爵是:一公士,二上造,三簪袅,四不更,五大夫,六官大夫,七公大夫,八公乘,九五大夫,十左庶长,十一右庶长,十二左更,十三中更,十四右更,十五少良造,十六大良造,十七驷车庶长,十八大庶长,十九关内侯,二十彻侯。}之赏罚,孝公善之。甘龙、杜挚等弗然,相与争之。卒用鞅法,百姓苦之;居三年,百姓便之。乃拜鞅为左庶长。其事在商君语中。
\end{yuanwen}

二年(前360年),周天子送给孝公祭肉。

三年(前359年),卫鞅劝说孝公推行变法,修整刑罚,对内大力发展农业,对外用相应的赏罚鼓励作战效死,孝公非常赞赏。甘龙、杜挚等人认为不好,一同和卫鞅争论。最终孝公还是任用卫鞅推行变法,百姓为此感到苦恼;过了三年,百姓就认为很好了。于是孝公任命卫鞅为左庶长。他的详细事迹在《商君列传》中。

\begin{yuanwen}
七年,与魏惠王会杜平。

八年,与魏战元里,有功。

十年,卫鞅为大良造,将兵围魏安邑,降之。

十二年,作为咸阳,筑冀阙\footnote{宫门外的观楼。},秦徙都之。并诸小乡聚,集为大县,县一令,四十一县。为田开阡陌\footnote{纵横交错的田界。}。东地渡洛。

十四年,初为赋。

十九年,天子致伯。

二十年,诸侯毕贺。秦使公子少官率师会诸侯逢泽,朝天子。
\end{yuanwen}

七年(前355年),孝公和魏惠王在杜平会面。

八年(前354年),秦军和魏军在元里交战,建立战功。

十年(前352年),卫鞅升任大良造,率领士兵围攻魏国的安邑,迫使守军投降。

十二年(前350年),秦国修筑咸阳城,在宫门外建造观楼,秦国迁都到那里。卫鞅合并小村落,集结为大县,每个县设一名县令,全国共设四十一个县。打破田亩间的界限。秦国的疆土向东扩展到洛水以东。

十四年(前348年),秦国开始征收赋税。

十九年(前343年),周天子任命孝公为方伯。

二十年(前342年),诸侯都来庆贺。秦国派公子少官带兵到逢泽和诸侯举行会盟,朝见周天子。

\begin{yuanwen}
二十一年,齐败魏马陵。

二十二年,卫鞅击魏,虏魏公子昂。封鞅为列侯,号商君。

二十四年,与晋战雁门,虏其将魏错。
\end{yuanwen}

二十一年(前341年),齐军在马陵打败魏军。

二十二年(前340年),卫鞅带兵攻打魏国,俘虏魏国的公子卬。孝公封卫鞅为列侯,封号是商君。

二十四年(前338年),秦军和魏军在雁门交战,俘虏对方的将领魏错。

\begin{yuanwen}
孝公卒,子惠文君立。是岁,诛卫鞅。鞅之初为秦施法,法不行,太子犯禁。鞅曰:“法之不行,自于贵戚。君必欲行法,先于太子。太子不可黥\footnote{在犯人脸上刺字的刑罚。},黥其傅师。”于是法大用,秦人治。及孝公卒,太子立,宗室多怨鞅,鞅亡,因以为反,而卒车裂以徇\footnote{示众。}秦国。
\end{yuanwen}

秦孝公去世,他的儿子惠文君继位。这一年,秦国诛杀卫鞅。卫鞅最初在秦国开展变法,法令不能通行,太子违犯禁令。卫鞅说:“法令无法推行的原因,正是由于贵戚的阻挠。您一定要推行变法的话,就先惩治太子。对太子无法动用黥刑,就对他的师傅动用黥刑。”于是法令很快得以推行,秦国民众被治理得很好。等到孝公去世以后,太子继位,宗室很多人怨恨卫鞅,卫鞅逃亡,太子趁机为他定下反叛的罪名,最终对他施以车裂之刑,在秦国示众。

\begin{yuanwen}
惠文君元年,楚、韩、赵、蜀人来朝。

二年,天子贺。

三年,王冠\footnote{男子成人礼,在二十岁左右举行。}。

四年,天子致文武胙。齐、魏为王。
\end{yuanwen}

惠文君元年(前337年),楚国、韩国、赵国、蜀国派人前来朝见。

二年(前336年),周天子派人前来祝贺。

三年(前335年),惠文王举行冠礼。

四年(前334年),周天子送来祭祀文王和武王的牲肉。齐君、魏君称王。

\begin{yuanwen}
五年,阴晋人犀首\footnote{指魏国人公孙衍。}为大良造。

六年,魏纳阴晋,阴晋更名宁秦。

七年,公子昂与魏战,虏其将龙贾,斩首八万。

八年,魏纳河西地。

九年,渡河,取汾阴、皮氏。与魏王会应。围焦,降之。

十年,张仪相秦。魏纳上郡十五县。

十一年,县义渠\footnote{秦昭襄王三十五年(前272年),秦国灭义渠,而此时不可能在义渠设县。}。归魏焦、曲沃。义渠君为臣。更名少梁曰夏阳。

十二年,初腊\footnote{腊祭,即冬至后第三个戌日祭祀众神。}。

十三年四月戊午,魏君为王\footnote{此条记载有明确的年月日,次年又改元,可知此处应为“秦君为王”。魏君称王在秦惠文君四年,并非十三年。},韩亦为王。使张仪伐取陕,出其人与魏。
\end{yuanwen}

五年(前333年),阴晋人犀首任大良造。

六年(前332年),魏国将阴晋割让给秦国,阴晋改名为宁秦。

七年(前331年),公子卬带兵和魏军交战,俘虏魏国的将领龙贾,斩首八万人。

八年(前330年),魏国将河西的土地割让给秦国。

九年(前329年),秦军渡过黄河,攻取汾阴、皮氏。惠文君和魏王在应邑会面。秦军包围焦邑,迫使其守军投降。

十年(前328年),张仪担任秦国的相国。魏国割让上郡的十五个县。

十一年(前327),秦国在义渠设置县。将焦邑、曲沃归还魏国。义渠君向秦国称臣。将少梁改名为夏阳。

十二年(前326年),开始举行腊祭。

十三年(前325年)四月戊午日,惠文君称王,韩君也称王。秦国派张仪攻取陕县,把当地民众驱逐到魏国。

\begin{yuanwen}
十四年,更为元年。

二年,张仪与齐、楚大臣会啮桑。

三年,韩、魏太子来朝。张仪相魏。

五年,王游至北河。

七年,乐池相秦。韩、赵、魏、燕、齐帅匈奴共攻秦。秦使庶长疾\footnote{即樗里疾,庶长是其官爵。}与战修鱼,虏其将申差,败赵公子渴、韩太子奂,斩首八万二千。

八年,张仪复相秦。

九年,司马错伐蜀,灭之。伐取赵中都、西阳。

十年,韩太子苍来质。伐取韩石章。伐败赵将泥。伐取义渠二十五城。

十一年,樗里疾攻魏焦,降之。败韩岸门,斩首万,其将犀首走。公子通封于蜀。燕君让其臣子之。

十二年,王与梁王会临晋。庶长疾攻赵,虏赵将庄。张仪相楚。

十三年,庶长章\footnote{即魏章。}击楚于丹阳,虏其将屈匄\footnote{gài},斩首八万;又攻楚汉中,取地六百里,置汉中郡。楚围雍氏,秦使庶长疾助韩而东攻齐,到满助魏攻燕。

十四年,伐楚,取召陵。丹、犁臣,蜀相壮\footnote{陈壮,又名庄。}杀蜀侯来降。
\end{yuanwen}

十四年(前324年),改为元年。

二年(前323年),张仪和齐国、楚国的大臣在啮桑会面。

三年(前322年),韩国、魏国的太子前来朝见。张仪担任魏国的相国。

五年(前321年),惠文王巡游到北河。

七年(前319年),乐池担任秦国的相国。韩国、赵国、魏国、燕国、齐国率领匈奴军队联合攻打秦国。秦国派庶长樗里疾和敌军在修鱼交战,俘虏敌将申差,打败赵国的公子渴、韩国的太子奂,斩首八万二千人。

八年(前318年),张仪再次担任秦国的相国。

九年(前317年),司马错攻打蜀国,将其灭掉。秦军攻取赵国的中都、西阳。

十年(前316年),韩国的太子苍前来做人质。秦军攻取韩国的石章。打败赵国的将领泥。攻取义渠的二十五座城。

十一年(前315年),樗里疾攻打魏国的焦邑,迫使其守军投降。在岸门打败韩军,斩首一万人,敌将犀首逃跑。公子通被封为蜀侯。燕君让位给他的大臣子之。

十二年(前314年),惠文王和梁王在临晋会面。庶长樗里疾攻打赵国,俘虏赵将庄。张仪担任楚国的相国。

十三年(前313年),庶长魏章在丹阳攻打楚国,俘虏敌将屈匄,斩首八万人;又攻打楚国的汉中,夺取土地六百里,设置汉中郡。楚国包围雍氏,秦国派庶长樗里疾援助韩国而向东攻打齐国,派到满援助魏国攻打燕国。

十四年(前312年),秦国讨伐楚国,攻取召陵。丹戎、犁戎向秦国称臣。蜀相陈壮杀死蜀侯前来归降。

\begin{yuanwen}
惠王卒,子武王立。韩、魏、齐、楚、越皆宾从\footnote{服从,归顺。}。
\end{yuanwen}

惠王去世,他的儿子武王继位。韩国、魏国、齐国、楚国、越国都服从秦国。

\begin{yuanwen}
武王元年,与魏襄王会临晋。诛蜀相壮。张仪、魏章皆东出之魏。伐义渠、丹、犁。

二年,初置丞相,樗里疾、甘茂为左右丞相。张仪死于魏。

三年,与韩襄王会临晋外。南公揭卒,樗里疾相韩。武王谓甘茂曰:“寡人欲容车\footnote{送葬时运载死者衣冠、画像的车。}通三川,窥周室,死不恨\footnote{遗憾。}矣。”其秋,使甘茂、庶长封伐宜阳。

四年,拔宜阳,斩首六万。涉河,城武遂。魏太子来朝。武王有力好戏,力士任鄙、乌获、孟说皆至大官。王与孟说举鼎,绝膑。

八月,武王死。族\footnote{灭族。}孟说。武王取魏女为后,无子。立异母弟,是为昭襄王。昭襄母楚人,姓芈氏,号宣太后。武王死时,昭襄王为质于燕,燕人送归,得立。
\end{yuanwen}

武王元年(前310年),和魏惠王在临晋会面。秦军诛杀蜀相陈壮。张仪、魏章都离开秦国向东前往魏国。秦国讨伐义渠、丹、犁。

二年(前309年),秦国开始设置丞相,樗里疾、甘茂分别担任左、右丞相。张仪死在魏国。

三年(前308年),武王和韩襄王在临晋城外会面。南公揭去世,樗里疾担任韩国的相国。武王对甘茂说:“我想要为自己的丧车在三川地区打通一条路,到周王室看一看,就算死去也不会有遗憾了。”当年秋季,武王派甘茂、庶长封攻打宜阳。

四年(前307年),秦军攻下宜阳,斩首六万人。渡过黄河,修筑武遂城。魏国的太子前来朝见。武王很有力气,爱好与人比赛,力士任鄙、乌获、孟说都当上了大官。武王和孟说在洛阳举鼎,砸断了腿。

八月,武王去世。秦国将孟说灭族。武王娶魏国女子为王后,没有生下子嗣。立异母弟为王,这就是昭襄王。昭襄王的母亲是楚国人,姓芈氏,号宣太后。武王死的时候,昭襄王在燕国做人质,燕国人将他护送回国,得以继位。

\begin{yuanwen}
昭襄王元年,严君疾\footnote{即樗里疾,封严君。}为相。甘茂出之魏。

二年,彗星见。庶长壮与大臣、诸侯、公子为逆,皆诛,及惠文后皆不得良死。悼武王后出归魏。

三年,王冠。与楚王会黄棘,与楚上庸。

四年,取蒲阪。彗星见。

五年,魏王来朝应亭,复与魏蒲阪。

六年,蜀侯煇反,司马错定蜀。庶长奂伐楚,斩首二万。泾阳君质于齐。日食,昼晦。

七年,拔新城。樗里子卒。

八年,使将军芈戎攻楚,取新市。齐使章子,魏使公孙喜,韩使暴鸢共攻楚方城,取唐眛。赵破中山,其君亡,竟死齐。魏公子劲、韩公子长为诸侯\footnote{这里指封君,战国封君制度是春秋时期分封卿大夫的延续。秦国二十等军功爵,最高者为彻侯,汉武帝以后改称列侯,当时诸侯国内部分封制度与此大体相近,封号为某侯、某君、某公。}。

九年,孟尝君薛文\footnote{孟尝君为齐国田氏,封薛公,又称薛氏。}来相秦。奂攻楚,取八城,杀其将景快。

十年,楚怀王入朝秦,秦留之。薛文以金受免。楼缓为丞相。

十一年,齐、韩、魏、赵、宋、中山五国\footnote{应为“六国”。}共攻秦,至盐氏而还。秦与韩、魏河北及封陵以和。彗星见。楚怀王走之赵,赵不受,还之秦,即死,归葬。

十二年,楼缓免,穰侯魏冉为相。予楚粟五万石。
\end{yuanwen}

昭襄王元年(前306年),严君疾担任相国。甘茂离开秦国前往魏国。

二年(前305年),彗星出现。庶长壮与大臣、诸侯、公子叛乱,都被诛杀,连累惠文后都不得善终。悼武王后离开秦国回到魏国。

三年(前304年),昭襄王举行冠礼。昭襄王和楚王在黄棘会面,将上庸还给楚国。

四年(前303年),秦国攻取蒲阪。彗星出现。

五年(前302年),魏王到应亭前来朝见,昭襄王把蒲阪归还魏国。

六年(前301年),蜀侯煇反叛,司马错平定蜀国。庶长奂攻打楚国,斩首两万人。泾阳君到齐国做人质。出现日食,白天变得非常昏暗。

七年(前300年),秦国攻取新城。樗里子去世。

八年(前299年),秦国派将军芈戎攻打楚国,夺取新市。齐国派章子,魏国派孙喜,韩国派暴鸢,联合攻打楚国的方城,夺取唐眜。赵国攻破中山,中山君逃亡,最后死在齐国。魏国的公子劲、韩国的公子长被封为侯。

九年(前298年),孟尝君薛文来到秦国担任丞相。庶长奂攻打楚国,夺取八座城,杀死楚将景快。

十年(前297年),楚怀王来到秦国朝见,秦国将他扣留。薛文因为金受的缘故被免去官职。楼缓担任丞相。

十一年(前296年),齐、韩、魏、赵、宋、中山六国联合攻打秦国,攻到盐氏才返回。秦国将黄河以北的土地以及封陵归还韩国、魏国来求和。彗星出现。楚怀王逃到赵国,赵国不收留,他又回到秦国,不久死去,尸体被运回楚国安葬。

十二年(前295年),楼缓被免去官职,穰侯魏冉担任丞相。秦国送给楚国五万石粮食。

\begin{yuanwen}
十三年,向寿伐韩,取武始。左更白起攻新城。五大夫礼出亡奔魏。任鄙为汉中守。

十四年,左更白起攻韩、魏于伊阙,斩首二十四万,虏公孙喜,拔五城。

十五年,大良造白起攻魏,取垣,复予之。攻楚,取宛。

十六年,左更错取轵及邓。冉免,封公子市宛,公子悝邓,魏冉陶,为诸侯。

十七年,城阳君入朝,及东周君来朝。秦以垣为蒲阪、皮氏。王之宜阳。

十八年,错攻垣、河雍,决桥取之。

十九年,王为西帝,齐为东帝,皆复去之。吕礼来自归。齐破宋,宋王在魏,死温。任鄙卒。

二十年,王之汉中,又之上郡、北河。

二十一年,错攻魏河内。魏献安邑,秦出其人,募徙河东赐爵,赦罪人迁之。泾阳君封宛。

二十二年,蒙武伐齐。河东为九县。与楚王会宛。与赵王会中阳。
\end{yuanwen}

十三年(前294年),向寿攻打韩国,夺取武始。左更白起攻打新城。五大夫吕礼离开秦国逃到魏国。任鄙担任汉中郡守。

十四年(前293年),左更白起在伊阙攻打韩、魏两国军队,斩首二十四万人,俘虏公孙喜,夺取五座城。

十五年(前292年),大良造白起攻打魏国,夺取垣邑,又归还了。秦国攻打楚国,夺取宛邑。

十六年(前291年),左更司马错攻取轵邑和邓邑。魏冉被免去官职,秦国封公子巿于宛,封公子悝于邓,封魏冉于陶,都受封为侯。

十七年(前290年),城阳君前来朝见,东周君也前来朝见。秦国用垣县换取蒲阪、皮氏。昭襄王来到宜阳。

十八年(前289年),司马错攻打垣县、河雍,拆掉桥梁后攻取两地。

十九年(前288年),昭襄王称西帝,齐王称东帝,不久又都去掉帝号。吕礼回到秦国。齐国攻破宋国,宋王逃到魏国,死在温县。任鄙去世。

二十年(前287年),昭襄王来到汉中,又到上郡、北河。

二十一年(前286年),司马错攻打河内。魏国献出安邑,秦国把当地居民驱赶出去,从河东招募迁徙军民,赐给他们爵位,赦免罪犯迁到安邑。泾阳君被封于宛。

二十二年(前285年),蒙武讨伐齐国。秦国在河东设置九个县。昭襄王和楚王在宛县会面,和赵王在中阳会面。

\begin{yuanwen}
二十三年,尉斯离与三晋、燕伐齐,破之济西。王与魏王会宜阳,与韩王会新城。

二十四年,与楚王会鄢,又会穰。秦取魏安城,至大梁,燕、赵救之,秦军去。魏冉免相。

二十五年,拔赵二城。与韩王会新城,与魏王会新明邑。

二十六年,赦罪人迁之穰。侯冉复相。

二十七年,错攻楚。赦罪人迁之南阳。白起攻赵,取代光狼城。又使司马错发陇西,因蜀攻楚黔中,拔之。

二十八年,大良造白起攻楚,取鄢、邓,赦罪人迁之。

二十九年,大良造白起攻楚,取郢为南郡,楚王走。周君\footnote{不知是西周君还是东周君。}来。王与楚王会襄陵。白起为武安君。

三十年,蜀守若伐楚,取巫郡,及江南为黔中郡。

三十一年,白起伐魏,取两城。楚人反我江南。

三十二年,相穰侯攻魏,至大梁,破暴鸢,斩首四万,鸢走,魏入三县请和。

三十三年,客卿胡阳攻魏卷、蔡阳、长社,取之。击芒卯华阳,破之,斩首十五万。魏入南阳以和。

三十四年,秦与魏、韩上庸地为一郡\footnote{此句语意不通,历史学家杨宽认为,“秦把所占韩、魏的南阳和楚的上庸地合建为郡”。},南阳免臣\footnote{免罪的奴隶。}迁居之。

三十五年,佐韩、魏、楚伐燕。初置南阳郡。

三十六年,客卿灶攻齐,取刚、寿,予穰侯。

三十八年,中更胡阳攻赵阏与,不能取。
\end{yuanwen}

二十三年(前284年),国尉斯离和韩、魏、赵、燕等国的军队讨伐齐国,在济水以西打败齐军。昭襄王和魏王在宜阳会面,和韩王在新城会面。

二十四年(前283年),昭襄王和楚王在鄢会面,又在穰会面。秦国攻取了魏国的安城,进军至大梁,燕国、赵国派兵救援,秦军撤走。魏冉被免去丞相之职。

二十五年(前282年),秦国攻下赵国的两座城。昭襄王和韩王在新城会面,和魏王在新明邑会面。

二十六年(前281年),秦国赦免罪犯迁到穰邑。穰侯魏冉再次担任丞相。

二十七年(前280年),司马错攻打楚国。秦国赦免罪犯迁到南阳。白起攻打赵国,夺取光狼城。秦国又派司马错征发陇西的士兵,途经蜀郡去攻打楚国的黔中,攻下那里。

二十八年(前279年),大良造白起攻打楚国,夺取鄢、邓两邑,秦国赦免罪犯迁到那里。

二十九年(前278年),大良造白起攻打楚国,攻取郢设为南郡,楚王逃跑。周君前来。昭襄王和楚王在襄陵会面,白起被封为武安君。

三十年(前277年),蜀郡守张若攻打楚国,夺取巫郡,占领江南地区设置黔中郡。

三十一年(前276年),白起攻打魏国,夺取两座城。楚国人在江南地区反叛。

三十二年(前275年),丞相穰侯攻打魏国,到达大梁,打败暴鸢,斩首四万人,暴鸢逃跑,魏国割让三个县以求和。

三十三年(前274年),客卿胡阳攻打魏国的卷邑、蔡阳、长社,夺取三地。秦国在华阳进攻芒卯,打败魏军,斩首十五万人。魏国割让南阳以求和。

三十四年(前273年),秦国将所占韩国、魏国的土地与上庸合建为一个郡,把南阳免罪的奴隶迁到那里。

三十五年(前272年),秦国帮助韩国、魏国、楚国攻打燕国。秦国开始设置南阳郡。

三十六年(前271年),客卿灶攻打齐国,夺取刚、寿两邑,昭襄王将其赐予穰侯。

三十八年(前269年),中更胡阳攻打赵国的阏与,不能夺取。

\begin{yuanwen}
四十年,悼太子死魏,归葬芷阳。

四十一年夏,攻魏,取邢丘、怀。

四十二年,安国君为太子。十月,宣太后薨,葬芷阳郦山。九月,穰侯出之陶。

四十三年,武安君白起攻韩,拔九城,斩首五万。

四十四年,攻韩南阳,取之。

四十五年,五大夫贲攻韩,取十城。叶阳君悝出之国\footnote{前往封地。公子悝的封地在邓县。},未至而死。

四十七年,秦攻韩上党,上党降赵,秦因攻赵,赵发兵击秦,相距。秦使武安君白起击,大破赵于长平,四十余万尽杀之。

四十八年十月\footnote{此处时间有误。昭襄王四十八年以十月开始,其后有正月,又有十月,四十九年则以正月开始。秦始皇以十月为岁首,而昭襄王时仍以正月为岁首,可能致使后世修史时产生错误,《史记》则如实抄录。},韩献垣雍。秦军分为三军\footnote{《白起王翦列传》记载“秦分军为二”,分别由王龁、司马梗率领。可知另有一军由白起带领返回秦国。}。武安君归。王龁将伐赵武安、皮牢,拔之。司马梗北定太原,尽有韩上党。正月,兵罢,复守上党。其十月,五大夫陵攻赵邯郸。

四十九年正月,益发卒佐陵。陵战不善,免,王龁代将。其十月,将军张唐攻魏,为蔡尉捐\footnote{弃,丢失。}弗守,还斩之。

五十年十月,武安君白起有罪,为士伍\footnote{士兵。},迁阴密。张唐攻郑\footnote{郑为韩国都城,此处应为“邺”。},拔之。

十二月,益发卒军汾城旁。武安君白起有罪,死。龁攻邯郸,不拔,去,还奔汾军二月余。攻晋军,斩首六千,晋楚流死河二万人。攻汾城,即从唐拔宁新中,宁新中更名安阳。初作河桥。
\end{yuanwen}

四十年(前267年),悼太子死在魏国,尸体被运回秦国葬在芷阳。

四十一年(前266年)夏季,秦国攻打魏国,夺取邢丘、怀邑。

四十二年(前265年),昭襄王立安国君为太子。十月,宣太后去世,葬在芷阳的郦山。九月,穰侯离开都城前往陶邑。

四十三年(前264年),武安君白起攻打韩国,攻下九座城,斩首五万人。

四十四年(前263年),白起攻打韩国的南阳,夺取那里。

四十五年(前262年),五大夫贲攻打韩国,夺取十座城。叶阳君悝离开都城前往封地,未能到达就死在了路上。

四十七年(前260年),秦国攻打韩国的上党,上党投降赵国,秦国趁机攻打赵国,赵国发兵抗击秦军,双方相持。秦国派武安君白起发动进攻,在长平大破赵军,四十多万名士兵全部被杀。

四十八年(前259年)十月,韩国割让垣雍县。秦军兵分三路。武安君返回。王龁带兵攻打赵国的武安、皮牢,攻下两地。司马梗向北平定太原,秦国完全占有韩国的上党。正月,秦国收兵,重新在上党集结。当年十月,五大夫王陵攻打赵国的邯郸。

四十九年(前258年)正月,秦国增兵援助王陵。王陵作战不利,被免去职务。王龁代替王陵带兵作战。当年十月,将军张唐攻打魏国,守将蔡尉弃城不守,逃回魏国被斩首。

五十年(前257年)十月,武安君白起犯罪,被降职为士兵,流放到阴密。张唐攻打邺县,攻下那里。

十二月,秦国增兵驻扎在汾城的附近。武安君白起犯罪,赐死。王龁攻打邯郸,不能攻下,于是撤兵,返回投奔驻扎在汾城的部队,休整两个多月。秦军进攻魏军,斩首六千人,魏国、楚国的士兵逃跑时落入黄河而死的有两万人。秦军攻打汾城,此后随张唐夺取宁新中,宁新中改名为安阳。秦国开始在黄河架桥。

\begin{yuanwen}
五十一年,将军摎攻韩,取阳城、负黍,斩首四万。攻赵,取二十余县,首虏\footnote{敌人的首级。}九万。西周君背秦,与诸侯约从,将天下锐兵出伊阙攻秦,令秦毋得通阳城。于是秦使将军摎攻西周。西周君走来自归,顿首受罪,尽献其邑三十六城,口三万。秦王受献,归其君于周。

五十二年,周民东亡,其器九鼎入秦。周初亡\footnote{此时东周国尚存。}。
\end{yuanwen}

五十一年(前256年),将军摎攻打韩国,夺取阳城、负黍,斩首四万人。秦国攻打赵国,夺取二十多个县,斩杀九万人。西周君背叛秦国,和诸侯约定合纵,率领天下的精锐士兵从伊阙山出发攻打秦国,使秦国不能打通前往阳城的道路。于是秦国派将军摎攻打西周。西周君逃跑后亲自来谢罪,叩头请求受罚,献出其全部领地三十六座城,民户三万口。秦王接受西周君的进献,让他回到周都。

五十二年(前255年),西周的百姓向东逃亡,周朝的宝器九鼎被送到秦国。周朝基本灭亡了。

\begin{yuanwen}
五十三年,天下来宾。魏后,秦使摎伐魏,取吴城。韩王入朝,魏委国听令。

五十四年,王郊见上帝于雍。

五十六年秋,昭襄王卒,子孝文王立。尊唐八子\footnote{嫔妃的封号。}为唐太后,而合其葬于先王。韩王衰绖入吊祠\footnote{吊唁祭祀。},诸侯皆使其将相来吊祠,视丧事。
\end{yuanwen}

五十三年(前254年),天下诸侯都来归顺。魏国来晚了,秦国就派摎去攻打魏国,夺取吴城。韩王前来朝见,魏国把政权交出,完全听命于秦国。

五十四年(前253年),昭襄王在雍县南郊举行祭祀天帝的仪式。

五十六年(前251年)秋季,昭襄王去世,他的儿子孝文王继位。他尊唐八子为唐太后,将她和昭襄王合葬在一处。韩王穿着丧服前来吊祭,诸侯都派各自的将相前来吊祭,参加丧礼。

\begin{yuanwen}
孝文王元年,赦罪人,修先王功臣,褒厚亲戚,弛苑囿。孝文王除丧,十月己亥即位,三日辛丑卒,子庄襄王立。
\end{yuanwen}

孝文王元年(前250年),赦免罪人,嘉奖先王时的功臣,厚待宗室的亲属,向民众开放苑囿。孝文王服丧期满后,十月己亥日即位,在第三天辛丑日就去世了,他的儿子庄襄王继位。

\begin{yuanwen}
庄襄王元年,大赦罪人,修先王功臣,施德厚骨肉\footnote{指至亲。}而布惠于民。东周君与诸侯谋秦,秦使相国吕不韦诛之,尽入其国。秦不绝其祀,以阳人地赐周君,奉其祭祀。使蒙骜伐韩,韩献成皋、巩。秦界至大梁,初置三川郡。

二年,使蒙骜攻赵,定太原。

三年,蒙骜攻魏高都、汲,拔之。攻赵榆次、新城、狼孟,取三十七城。

四月日食。王龁攻上党。初置太原郡。魏将无忌\footnote{即信陵君。}率五国兵击秦,秦却于河外。蒙骜败,解而去。

五月丙午,庄襄王卒,子政立,是为秦始皇帝。
\end{yuanwen}

庄襄王元年(前249年),大规模赦免罪人,嘉奖先王时的功臣,施恩德厚待至亲,并且向民众施仁惠。东周君和诸侯图谋攻打秦国,秦国派相国吕不韦前去惩治东周君,完全吞并东周国。秦国人没有断绝周朝的祭祀,将阳人地赏赐给东周君,让他供奉祖先的祭祀。秦国派蒙骜攻打韩国,韩国割让成皋、巩邑。秦国的疆界扩展到大梁,开始设置三川郡。

二年(前248年),庄襄王派蒙骜攻打赵国,平定太原。

三年(前247年),蒙骜攻打魏国的高都、汲邑,攻下两地。又去攻打赵国的榆次、新城、狼孟,夺取三十七座城。

四月,出现日食。王龁攻打上党。秦国开始设置太原郡。魏将无忌率领五国军队抗击秦国,秦军撤退到河外。蒙骜战败,诸侯联军解除包围撤回。

五月丙午日,庄襄王去世,他的儿子政继位,这就是秦始皇帝。

\begin{yuanwen}
秦王政立二十六年,初并天下为三十六郡,号为始皇帝。始皇帝五十一年\footnote{指五十一岁。}而崩,子胡亥立,是为二世皇帝。

三年,诸侯并起叛秦,赵高杀二世,立子婴。子婴立月余,诸侯诛之,遂灭秦。其语在《始皇本纪》中。
\end{yuanwen}

秦王政在位二十六年时,开始兼并天下而设置三十六个郡,号称始皇帝。始皇帝五十一岁时去世,他的儿子胡亥继位,这就是二世皇帝。

三年(前207年),诸侯联合起兵反叛秦朝,赵高杀死二世,立子婴为王。子婴在位一个多月,被诸侯杀死,秦朝最终灭亡。这些事情记录在《始皇本纪》中。

\begin{yuanwen}
太史公曰:秦之先为嬴姓。其后分封,以国为姓\footnote{应为“以国为氏”。徐、江、黄、秦、赵等都是嬴姓国,而氏是姓的分支,所以上述诸侯国同姓而不同氏。秦汉时,姓、氏合一,很多人混淆了二者的概念,南宋史学家郑樵在《通志·氏族略》中说:“虽子长(司马迁)、知几(刘知几)二良史犹昧于此。”},有徐氏、郯氏、莒氏、终黎氏、运奄氏、菟裘氏、将梁氏、黄氏、江氏、修鱼氏、白冥氏、蜚廉氏、秦氏。然秦以其先造父封赵城,为赵氏。
\end{yuanwen}

太史公说:秦国的祖先是嬴姓。其后代被分封在各地,以封国为姓,有徐氏、郯氏、莒氏、终黎氏、运奄氏、菟裘氏、将梁氏、黄氏、江氏、修鱼氏、白冥氏、蜚廉氏、秦氏。然而秦国由于祖先造父被封在赵城,因此为赵氏。

\part{卷六}

\chapter{秦始皇本纪第六}

本篇是秦朝帝王的本纪,记述了秦始皇统一天下到秦朝灭亡的历史。秦始皇被誉为“千古一帝”,他开创了中国历史上的大一统局面,并且推行了统一文字、货币、度量衡等政策,在中华文明发展的历程中有着深远的影响。然而秦朝二世而亡,确实值得深思。司马迁在篇末借贾谊的《过秦论》分析了秦朝兴亡的原因。

记载了秦始皇在其历代祖先积蓄力量的基础上并吞六国,统一天下,第一次建立了中央集权的强大国家的过程,肯定了秦始皇的丰功伟绩。同时也记载了秦始皇称帝后由于缺少历史经验而采取的种种错误做法,尤其是写了秦始皇死后,秦二世以非法手段篡取政权,倒行逆施,终致在两年多的时间里将秦王朝彻底葬送的悲惨教训。作品篇幅很长,叙述极其精彩,是《史记》中篇幅较长的作品之一。

司马迁是将始皇帝作为一个因缺少历史经验而招致失败的悲剧英雄来进行写作的,笔下有无限惋惜之情。

\begin{yuanwen}
秦始皇帝者,秦庄襄王子也。庄襄王为秦质子\footnote{被派到别国做人质的人,多为诸侯之子。质子外交源于春秋,盛于战国,质子成为各国之间表明信用的凭证。}于赵,见吕不韦姬,悦而取之,生始皇。以秦昭王四十八年正月生于邯郸。及生,名为政,姓赵氏。年十三岁,庄襄王死,政代立为秦王。当是之时,秦地已并巴、蜀、汉中,越宛有郢,置南郡矣;北收上郡以东,有河东、太原、上党郡;东至荥阳,灭二周,置三川郡。吕不韦为相,封十万户,号曰文信侯。招致宾客游士,欲以并天下。李斯为舍人\footnote{门客。}。蒙骜、王齮\footnote{yǐ}、麃\footnote{biāo}公等为将军。王年少,初即位,委\footnote{委托。}国事大臣。
\end{yuanwen}

秦始皇帝,是秦庄襄王的儿子。庄襄王身为秦国的质子在赵国,见到吕不韦的姬妾,十分喜欢就娶了她,生下始皇。始皇在秦昭王四十八年(前259年)正月出生于邯郸。等到他出生之后,取名为政,姓赵氏。十三岁的时候,庄襄王去世,赵政被立为秦王。在这个时候,秦国已经吞并巴、蜀、汉中,跨越宛县占有郢都,设置了南郡;秦军向北攻取上郡以东地区,占有河东、太原、上党三郡;向东进军到荥阳,灭掉东西二周,设置了三川郡。吕不韦担任相国,封邑十万户,封号是文信侯。吕不韦招揽宾客游士,想要借此兼并天下。李斯是他的门客,蒙骜、王齮、麃公等人担任将军。秦王年纪还小,刚即位,把国事托付给大臣。

\begin{yuanwen}
晋阳反,元年,将军蒙骜击定之。

二年,麃公将卒攻卷,斩首三万。

三年,蒙骜攻韩,取十三城。王齮死。

十月,将军蒙骜攻魏氏篸(畼chàng)、有诡。岁大饥。

四年,拔篸、有诡。

三月,军罢。秦质子归自赵,赵太子出归国。

十月庚寅,蝗蟲从东方来,蔽天。天下疫。百姓内\footnote{nà,同“纳”,缴纳。}粟千石,拜爵一级。

五年,将军骜攻魏,定酸枣、燕、虚、长平、雍丘、山阳城,皆拔之,取二十城。初置东郡。冬雷。

六年,韩、魏、赵、卫、楚共击秦,取寿陵。秦出兵,五国兵罢。拔卫,迫东郡,其君角率其支属\footnote{亲属。}徙居野王,阻其山以保魏之河内。

七年,彗星先出东方,见北方,五月见西方。将军骜死。以攻龙、孤、庆都,还兵攻汲。彗星复见西方十六日。夏太后死。

八年,王弟长安君成蟜\footnote{jiǎo}将军击赵,反,死屯留,军吏皆斩死,迁其民于临洮。将军壁死,卒屯留、蒲、惣(鶮hú)\footnote{蒲、鶮:一说为名叫蒲的士卒,一说为蒲、二邑,这里采用后种一说法。}反,戮其尸。河鱼大上,轻车重\footnote{大。}马东就食\footnote{到有食物的地方去。}。
\end{yuanwen}

晋阳发生叛乱,元年(前246年),将军蒙骜前往平定。

二年(前245年),麃公带领士兵攻打卷邑,斩首三万人。

三年(前244年),蒙骜攻打韩国,攻取十三座城。王齮死去。

十月,将军蒙骜攻打魏国的畼邑、有诡。这一年发生饥荒。

四年(前243年),蒙骜攻下畼邑、有诡。

三月,蒙骜撤军。秦国的质子从赵国返回,赵国的太子离开秦国返回本国。

十月庚寅日,蝗虫从东方飞来,遮蔽天空。天下瘟疫肆虐。百姓缴纳一千石粟米,就可以授予一级爵位。

五年(前242年),将军蒙骜攻打魏国,平定酸枣、燕邑、虚邑、长平、雍丘、山阳城,全部攻下,夺取二十座城。秦国开始设置东郡。冬季有雷声响起。

六年(前241年),韩、魏、赵、卫、楚五国联合攻打秦国,夺取寿陵。秦国出兵,五国联军撤退。秦国攻下卫国,逼近东郡,卫君角率领他的亲属迁居到野王,凭借山势的险阻来保护魏国的河内地区。

七年(前240年),彗星先出现在东方,又出现在北方,五月出现在西方。蒙骜将军死去,是在攻打龙邑、孤邑、庆都,又回军攻打汲邑的时候死去的。彗星再次出现在西方十六天。夏太后死去。

八年(前239年),秦王的弟弟长安君成蟜带领军队攻打赵国,发动叛乱,死在屯留,他的军吏都被斩杀,秦王把他封地的民众迁徙到临洮。将军壁死去,屯留、蒲邑、鶮邑的士兵叛乱,鞭戮壁的尸体。大量黄河里的鱼被冲到岸上,人们用高头大马拉着轻便的车子赶到东方去寻找食物。

\begin{yuanwen}
嫪\footnote{lào}毐\footnote{ǎi}封为长信侯。予之山阳地,令毐居之。宫室、车马、衣服、苑囿、驰猎恣毐。事无小大皆决于毐。又以河西、太原郡更为毐国。

九年,彗星见,或竟天\footnote{划过天空。}。攻魏垣、蒲阳。

四月,上宿雍。己酉,王冠\footnote{男子成人礼,在二十岁左右举行。},带剑。长信侯毐作乱而觉,矫王御玺及太后玺以发县卒及卫卒、官骑、戎翟君公、舍人,将欲攻蕲\footnote{qí}年宫为乱。王知之,令相国、昌平君、昌文君发卒攻毐。战咸阳,斩首数百,皆拜爵,及宦者皆在战中,亦拜爵一级。毐等败走。即令国中:有生得\footnote{活捉。}毐,赐钱百万;杀之,五十万。

尽得毐等。卫尉竭、内史肆、佐弋竭、中大夫令齐等二十人皆枭首\footnote{斩首示众。}。车裂以徇,灭其宗。及其舍人,轻者为鬼薪\footnote{在宗庙服劳役。}。及夺爵迁蜀四千馀家,家房陵。是月寒冻,有死者。杨端和攻衍氏。彗星见西方,又见北方,从斗以南八十日。

十年,相国吕不韦坐嫪毐免。桓齮为将军。齐、赵来置酒。齐人茅焦说秦王曰:“秦方以天下为事,而大王有迁母太后之名,恐诸侯闻之,由此倍\footnote{通“背”,背叛。}秦也。”秦王乃迎太后于雍而入咸阳,复居甘泉宫。
\end{yuanwen}

嫪毐被封为长信侯。秦王赐给他山阳为封地,让他居住,宫室、车马、服饰、苑囿、游猎任由嫪毐享受,事情无论大小都由嫪毐决断。秦王又把河西、太原二郡改为嫪毐的封地。

九年(前238年),彗星出现,有的划过天空。秦国攻打魏国的垣邑、蒲阳。

四月,秦王留宿雍县。己酉日,秦王举行冠礼,佩带宝剑。长信侯嫪毐叛乱却被发现,伪造秦王的御玺和太后印玺来调发各县的士兵和宫中的侍卫、秦王的骑兵、戎狄的首领、门下的食客,将要攻打蕲年宫发动叛乱。秦王得知这个消息后,命令相国、昌平君、昌文君调发士兵进攻嫪毐。双方在咸阳交战,斩首几百人,立功的将士都赐予爵位,连参加战斗的宦官,也得到一级爵位。嫪毐等人战败逃跑。秦王就在全国下令:“有能生擒嫪毐的人,赏钱一百万;有能杀死嫪毐的人,赏钱五十万。”

因而将嫪毐等人全部抓获。卫尉竭、内史肆、佐弋竭、中大夫令齐等二十人都被斩首示众。秦王又将嫪毐车裂以示众,诛灭他的宗族。连及嫪毐的门客,罪行轻的为宗庙服劳役。有四千多家被剥夺爵位流放到蜀郡,定居房陵。这个月天寒地冻,有人被冻死。杨端和攻打衍氏。彗星出现在西方,又出现在北方,跟随北斗向南运行了八十天。

十年(前237年),相国吕不韦因为嫪毐之乱获罪被罢免。桓齮担任将军。齐国、赵国的使者前来,秦王摆酒款待。齐国人茅焦劝秦王说:“秦国以统一天下为事业,可是大王有流放母太后的恶名,恐怕诸侯听说这件事后,会因此背叛秦国。”秦王于是把太后从雍县迎接回咸阳,让她重新居住在甘泉宫。

\begin{yuanwen}
大索,逐客,李斯上书说,乃止逐客令。李斯因说秦王,请先取韩以恐他国,于是使斯下\footnote{使屈服。}韩。韩王患之。与韩非谋弱秦。大梁人尉缭来,说秦王曰:“以秦之彊,诸侯譬如郡县之君,臣但恐诸侯合从,翕\footnote{闭合,引申为隐秘。}而出不意,此乃智伯\footnote{晋国上卿,被赵、魏、韩所灭。}、夫差\footnote{末代吴王,为越国所灭。}、湣王\footnote{齐王,被燕将乐毅击败。}之所以亡也。愿大王毋爱财物,赂其豪臣,以乱其谋,不过亡三十万金,则诸侯可尽。”

秦王从其计,见尉缭亢礼\footnote{也作“抗礼”,指用平等的礼节相待。},衣服、食饮与缭同。缭曰:“秦王为人,蜂准,长目,挚\footnote{通“鸷”,猛禽。}鸟膺,豺声,少恩而虎狼心,居约\footnote{穷困。}易出人下,得志亦轻食人。我布衣,然见我常身自下我。诚使秦王得志于天下,天下皆为虏矣。不可与久游。”乃亡去。秦王觉,固止,以为秦国尉,卒用其计策。而李斯用事。
\end{yuanwen}

秦国大举搜查,驱逐宾客,李斯上书劝说,秦王才废止逐客令。李斯趁机劝说秦王,请求先攻取韩国来震慑其他国家,于是秦王派李斯迫使韩国屈服。韩王很担心这件事,与韩非商议削弱秦国的办法。大梁人尉缭来到秦国,劝秦王说:“以秦国的强盛,诸侯就像郡县的长官,我只担心诸侯合纵,隐秘而出其不意,这就是智伯、夫差、湣王败亡的原因。希望大王不要吝惜财物,以此贿赂各国有权势的大臣,来破坏他们的计划,这样只不过损失三十万金,却能够将诸侯彻底消灭。”

秦王听从了尉缭的建议,每次接见他的时候都平礼相待,服饰、饮食也与他一样。尉缭说:“秦王这个人,长着高鼻梁,长眼睛,猛禽一样的胸膛,豺狼一样的声音,少恩德而有虎狼一样的心肠,在穷困时不难屈居人下,在得志后也能轻易将人吞食。我是一个平民,然而他接见我时经常居我之下。假如秦王能够如愿统一天下,那么天下人就都变成他的俘虏了。不能和他长期相处。”尉缭于是逃跑了。秦王发现之后,坚决挽留他,任命他为秦国的国尉,最终采用了他的计策。当时李斯在朝中当政。

\begin{yuanwen}
十一年,王翦、桓齮、杨端和攻邺,取九城。王翦攻阏与、橑\footnote{liáo}杨,皆并为一军。翦将十八日,军归斗食\footnote{指军吏的俸禄。}以下,什推二人从军取邺、安阳,桓齮将。

十二年,文信侯不韦死,窃葬。其舍人临者,晋\footnote{三晋,即赵、魏、韩。}人也逐出之;秦人六百石以上夺爵,迁;五百石以下不临,迁,勿夺爵。自今以来,操国事不道如嫪毐、不韦者籍其门,视此。秋,复嫪毐舍人迁蜀者。当是之时,天下大旱,六月至八月乃雨。
\end{yuanwen}

十一年(前236年),王翦、桓齮、杨端和攻打邺县,夺取九座城。王翦攻打阏与、橑杨,将所有士兵合并成一支军队。王翦指挥军队十八天,遣返俸禄一斗以下的军吏,从十个人中选拔两个人跟随军队攻取邺县、安阳,由桓齮率领。

十二年(前235年),文信侯吕不韦死去,私下举行葬礼。他门下前来致哀的食客,是三晋的人就全部驱逐出境;是秦国人,俸禄在六百石以上的官员就剥夺爵位,将他们流放;俸禄在五百石以下的官员,没有前来致哀,只是流放,不剥夺爵位。从此以后,像嫪毐、吕不韦一样执掌国政不行正道的人,都会抄没全家,像这样处理。秋季,赦免被迁徙到蜀郡的嫪毐门客。在这个时候,全国发生严重的旱灾,从六月持续到八月才下雨。

\begin{yuanwen}
十三年,桓齮攻赵平阳,杀赵将扈辄,斩首十万。王之河南。正月,彗星见东方。十月,桓齮攻赵。

十四年,攻赵军于平阳,取宜安,破之,杀其将军。桓齮定平阳、武城。韩非使秦,秦用李斯谋,留非,非死云阳。韩王请为臣。
\end{yuanwen}

十三年(前234年),桓齮攻打赵国的平阳,杀死赵将扈辄,斩首十万人。秦王前往黄河以南。正月,彗星出现在东方。十月,桓齮攻打赵国。

十四年(前233年),在平阳进攻赵国军队,夺取宜安,击败敌军,杀死对方的将军。桓齮平定平阳、武城。韩非出使秦国,秦王采用李斯的计谋,扣留韩非,韩非死在云阳。韩王请求向秦国称臣。

\begin{yuanwen}
十五年,大兴兵,一军至邺,一军至太原,取狼孟。地动。

十六年九月,发卒受地韩南阳假守腾。初令男子书年。魏献地于秦。秦置丽邑。

十七年,内史腾攻韩,得韩王安,尽纳其地,以其地为郡,命曰颍川。地动。华阳太后卒。民大饥。
\end{yuanwen}

十五年(前232年),秦国大举出兵,一路军队来到邺县,一路军队来到太原,攻取狼孟。发生地震。

十六年(前231年)九月,秦国派兵接收韩国的南阳地区,任命腾为代理郡守。秦国开始命令全国男子登记年龄。魏国割让土地给秦国。秦国设置丽邑。

十七年(前230年),内史腾攻打韩国,擒获韩王安,韩王献出全部国土,秦国将韩地设置为郡,命名为颍川。发生地震。华阳太后去世。民间发生严重的饥荒。

\begin{yuanwen}
十八年,大兴兵攻赵,王翦将上地,下井陉,端和将河内,羌瘣\footnote{huì}伐赵,端和围邯郸城。

十九年,王翦、羌瘣尽定取赵地东阳,得赵王。引兵欲攻燕,屯中山。秦王之邯郸,诸尝与王生赵时母家有仇怨,皆坑\footnote{也作“阬”,一说为活埋,一说为把人杀死后将尸首堆积起来。白起坑杀赵军四十万降卒,项羽坑杀秦军二十万降卒,活埋之说显然不合理。春秋时,战争胜利一方就有堆积敌人尸首筑为京官的做法,以此炫耀武功。而秦王政坑杀仇家、诸生,并不是作战之后所为,筑京官之说似乎也不合理。所以此处只译为“坑杀”,以求折中。}之。秦王还,从太原、上郡归。始皇帝母太后崩。赵公子嘉率其宗数百人之代,自立为代王,东与燕合兵,军上谷。大饥。
\end{yuanwen}

十八年(前229年),秦国大举发兵攻打赵国,王翦带领上郡的军队,攻下井陉,杨端和带领河内的军队,羌瘣攻打赵国,杨端和包围邯郸。

十九年(前228年),王翦、羌瘣完全平定并攻取赵国的东阳,擒获赵王。王翦带兵想要攻打燕国,驻扎在中山。秦王来到邯郸,那些曾经和他出生在赵国时母家有仇怨的人,全部被坑杀。秦王返回,经过太原、上郡回到国都。始皇帝母太后去世。赵国的公子嘉带领宗族几百人来到代郡,自立为代王,向东与燕国的军队联合,驻扎在上谷。秦国发生严重的饥荒。

\begin{yuanwen}
二十年,燕太子丹患秦兵至国,恐,使荆轲刺秦王。秦王觉之,体解轲以徇,而使王翦、辛胜攻燕。燕、代发兵击秦军,秦军破燕易水之西。

二十一年,王贲攻荆\footnote{指楚国。秦始皇的父亲庄襄王名楚,秦国史书为避讳,改称楚国为荆国。}。乃益发卒诣王翦军,遂破燕太子军,取燕蓟城,得太子丹之首。燕王东收辽东而王之。王翦谢病老归。新郑反。昌平君徙于郢。大雨雪,深二尺五寸。
\end{yuanwen}

二十年(前227年),燕国的太子丹担心秦国军队打到他的国家,非常恐惧,派荆轲去刺杀秦王。秦王发觉后,将荆轲肢解来示众,并且派王翦、辛胜攻打燕国。燕国、代国出兵迎击秦军,秦军在易水以西打败燕军。

二十一年(前226年),王贲攻打楚。秦王于是增派兵力前往支援王翦的军队,最终打败了燕国太子丹的军队,攻取燕都蓟城,得到太子丹的首级。燕王向东收拢辽东的残部而在那里称王。王翦称病告老回乡。新郑反叛。昌平君被迁到郢。天降大雪,积雪深达二尺五寸。

\begin{yuanwen}
二十二年,王贲攻魏,引河沟灌大梁,大梁城坏,其王请降,尽取其地。

二十三年,秦王复召王翦,彊起之,使将击荆。取陈以南至平舆,虏荆王。秦王游至郢陈\footnote{指陈县,楚国曾定都于此。}。荆将项燕立昌平君为荆王\footnote{据《六国年表》《楚世家》,项燕死于秦王政二十三年,而楚王负刍被俘于二十四年,与《秦本纪》矛盾。},反秦于淮南。

二十四年,王翦、蒙武攻荆,破荆军,昌平君死,项燕遂自杀。
\end{yuanwen}

二十二年(前225年),王贲攻打魏国,引黄河水淹灌大梁城,大梁城墙被水冲毁,魏王请求投降,秦国完全占领魏国的土地。

二十三年(前224年),秦王再次召见王翦,坚持要任用他,派他去攻打楚国。秦军夺取陈县以南一直到平舆的土地,擒获楚王。秦王到郢陈巡游。楚将项燕立昌平君为楚王,在淮水以南反抗秦军。

二十四年(前223年),王翦、蒙武进攻楚国,打败楚军,昌平君死去,项燕最终自杀。

\begin{yuanwen}
二十五年,大兴兵,使王贲将,攻燕辽东,得燕王喜。还攻代,虏代王嘉。王翦遂定荆江南地,降越君,置会稽郡。五月,天下大酺\footnote{pú,聚会宴饮。}。

二十六年,齐王建与其相后胜发兵守其西界,不通秦。秦使将军王贲从燕南攻齐,得齐王建。
\end{yuanwen}

二十五年(前222年),秦国大举发兵,派王贲带兵,攻打燕国的辽东,擒获燕王喜。秦军回转兵锋攻打代国,俘虏代王嘉。王翦最终平定了楚国的江南地区,降服越君,设置会稽郡。五月,秦王下令天下欢聚宴饮。

二十六年(前221年),齐王建和他的相国后胜调发士兵防守齐国的西部边界,不与秦国往来。秦王派将军王贲从燕国以南攻打齐国,擒获齐王建。

\begin{yuanwen}
秦初并天下,令丞相、御史\footnote{text}曰:“异日韩王纳地效玺,请为藩臣,已而倍约,与赵、魏合从畔秦,故兴兵诛之,虏其王。寡人以为善,庶几\footnote{或许。}息兵革。赵王使其相李牧来约盟\footnote{text},故归其质子。已而倍盟,反我太原,故兴兵诛之,得其王。赵公子嘉乃自立为代王,故举兵击灭之。魏王始约服入秦,已而与韩、赵谋袭秦,秦兵吏诛,遂破之。荆王献青阳以西,已而畔约,击我南郡\footnote{text},故发兵诛,得其王,遂定其荆地。燕王昏乱,其太子丹乃阴令荆轲为贼,兵吏诛,灭其国。齐王用后胜计,绝秦使\footnote{text},欲为乱,兵吏诛,虏其王,平齐地。寡人以眇眇\footnote{渺小,自谦之词。眇,miǎo,同“渺”。}之身,兴兵诛暴乱,赖宗庙之灵,六王咸伏其辜,天下大定\footnote{text}。今名号不更\footnote{text},无以称成功,传后世。其议帝号。”
\end{yuanwen}

秦国刚刚兼并天下,秦王对丞相、御史下令说:“昔日韩王进献土地,交出印玺,请求成为秦国的藩臣,不久就违背了约定,与赵国、魏国合纵反叛秦国,所以我发兵讨伐韩国,俘虏韩王。我认为这样做很好,或许可以平息战争。赵王派他的相国李牧前来缔结盟约,因此我归还赵国的质子。不久赵国违背盟约,在我国太原反叛,因此我发兵惩罚赵国,擒获赵王。赵国的公子嘉于是自立为代王,因此我发兵去消灭他。魏王最初约定向秦国臣服,不久又同韩国、赵国谋划袭击秦国,秦国的将士前往惩罚他们,于是将他们打败。楚王割让青阳以西的土地,不久也违背约定,攻打我国的南郡,因此我发兵惩罚楚国,擒获楚王,最终平定楚地。燕王昏庸悖乱,他的太子丹暗中指使荆轲为刺客,我派出将士惩罚他们,灭掉燕国。齐王采用后胜的计谋,断绝与秦国的往来,想要作乱,我派出将士惩罚他们,俘虏齐王,平定齐地。我凭借微不足道的身躯,发兵讨伐暴乱的国家,仰赖祖宗的威灵,六国之王都认罪伏法,天下完全平定了。现在不改变名号,就无法颂扬我所建立的功勋,从而流传后世。群臣商议帝王称号的事情。”

\begin{yuanwen}
丞相绾、御史大夫劫、廷尉斯等皆曰\footnote{text}:“昔者五帝地方千里,其外侯服\footnote{九服之一,指王畿外五百里之地。}、夷服\footnote{九服之一,指王畿外三千五百里之地。},诸侯或朝或否,天子不能制。今陛下兴义兵,诛残贼\footnote{text},平定天下,海内为郡县,法令由一统,自上古以来未尝有,五帝所不及。臣等谨与博士议\footnote{text}曰:‘古有天皇,有地皇,有泰皇,泰皇最贵\footnote{text}。’臣等昧死上尊号,王为‘泰皇’。命为‘制’,令为‘诏’,天子自称曰‘朕’。”
\end{yuanwen}

丞相王绾、御史大夫冯劫、廷尉李斯等人都说:“从前五帝统治方圆千里的地方,外围侯服、夷服等地的诸侯是否前来朝贡,天子不能控制。现在陛下发动正义之师,诛杀乱臣贼子,平定天下,四海之内都成为郡县,法令都由陛下一人发布,自从上古以来也没有这样的事情,是五帝所不能比的。我们谨慎地与博士商议说:‘古时候有天皇,有地皇,有泰皇,泰皇最尊贵。’我们冒死献上尊号,大王称‘泰皇’,教命称‘制’,号令称‘诏’,天子自称‘朕’。”

\begin{yuanwen}
王曰:“去‘泰’,著‘皇’,采上古‘帝’位号,号曰‘皇帝’。他如议。”

制曰:“可\footnote{text}。”追尊庄襄王为太上皇。

制曰:“朕闻太古有号毋谥,中古有号,死而以行为谥。如此,则子议父,臣议君也,甚无谓,朕弗取焉。自今已来,除谥法。朕为始皇帝。后世以计数,二世、三世至于万世,传之无穷。”
\end{yuanwen}

秦王说:“去掉‘泰’字,保留‘皇’字,采用上古时代的‘帝’这一称号,称为‘皇帝’。其他的都按照群臣商议的办。”

制命说:“可以。”始皇追尊庄襄王为太上皇。

制命说:“我听说远古时代有名号而没有谥号,中古时代有名号,死后根据他生前的行为追加谥号。像这样,就是儿子议论父亲,大臣议论君主,很没有意义,我不采取这种做法。从现在开始,废除追加谥号的做法。我是始皇帝,后世用数字计算,二世、三世一直到万世,永远传承下去。”

\begin{yuanwen}
始皇推终始五德\footnote{战国时期形成的阴阳学说以五行为五德,以其生克关系为王朝更替之应。相克关系为金克木、木克土、土克水、水克火、火克金,相生关系为金生水、水生木、木生火、火生土、土生金。阴阳家认为黄帝至虞舜为土德,夏为木德,商为金德,周为火德,秦取代周,因此为水德,崇尚黑色。}之传\footnote{text},以为周得火德,秦代周德,从所不胜\footnote{克。}。方今水德之始\footnote{text},改年始\footnote{text},朝贺\footnote{官员在元旦入朝庆贺。}皆自十月朔。衣服、旄旌、节旗皆上\footnote{同“尚”,崇尚。}黑\footnote{text}。数以六为纪\footnote{五行中水的成数是六。《周易·系辞上》说:“天一,地二;天三,地四;天五,地六;天七,地八;天九,地十。”天一生水,地二生火,天三生木,地四生金,天五生土,以上为生数。地六成水,天七成火,地八成木,天九成金,地十成土,以上为成数。},符、法冠\footnote{御史所戴之冠,又称獬豸冠。}皆六寸\footnote{text},而舆六尺\footnote{text},六尺为步,乘六马。更名河曰德水,以为水德之始。刚毅戾深\footnote{text},事皆决于法,刻削毋仁恩和义,然后合五德之数\footnote{text}。于是急法,久者不赦。
\end{yuanwen}

始皇推演五德终始的次序,认为周朝得到火德,秦朝取代周朝火德,根据火德所不能克制的属性,现在应该是水德的开始,改变每年开始的月份,规定官员入朝庆贺都在十月初一日。服饰、旌旗、符节都崇尚黑色。数字以六为标准,兵符、法冠都是六寸,而舆车宽六尺,以六尺为一步,用六匹马驾车。把黄河改名为德水,以此为水德的开始。始皇刚强坚毅,暴戾严苛,所有的事情都用法令来解决,刻薄而不讲仁爱道义,认为这样才能符合五德终始的规律。于是他急于推行法令,有的罪犯囚禁很长时间也没有被赦免。

\begin{yuanwen}
丞相绾等言:“诸侯初破,燕、齐、荆地远,不为置王,毋以填\footnote{通“镇”,安定。}之。请立诸子,唯上幸许。”

始皇下其议于群臣,群臣皆以为便。廷尉李斯议曰:“周文武所封子弟同姓甚众\footnote{text},然后属疏远\footnote{text},相攻击如仇雠,诸侯更相诛伐,周天子弗能禁止。今海内赖陛下神灵一统,皆为郡县\footnote{text},诸子功臣以公赋税重赏赐之\footnote{text},甚足易制。天下无异意,则安宁之术也。置诸侯不便。”

始皇曰:“天下共苦战斗不休,以有侯王。赖宗庙,天下初定,又复立国,是树兵也,而求其宁息,岂不难哉!廷尉议是。”
\end{yuanwen}

丞相王绾等人进言:“诸侯刚被消灭,燕、齐、楚等地非常偏远,如果不在那里设置藩王,就无法安定当地民众。请求立皇子为王,希望皇帝能够批准。”

始皇把他们的建议下交群臣商议,群臣都觉得很适当。廷尉李斯提议说:“周文王和武王所分封的同姓子弟非常多,可是后代血缘关系变得疏远,彼此攻击好像仇敌,诸侯相互征战,周天子没有办法制止。现在四海之内仰赖陛下的神武威灵得以统一,都设置为郡县,皇子和功臣用国家的赋税来施以重赏,局面很容易控制。天下人没有二心,这就是使国家安定的办法。分封诸侯不妥当。”

始皇说:“天下人都为战乱不止而苦恼,就是因为有诸侯的缘故。仰赖祖先的保佑,我刚刚平定天下,如果再去建立诸侯国,这是给自己树敌,再想要求得安宁,岂不是很难做到吗!廷尉的建议是正确的。”

杨慎:「诸铭直致无华采,颇杂以吏牍,自是秦时一样文字。钟惺曰:诸铭横而大,在工拙之外求之。有韵之文,只如信口序事,妙手。」顾炎武:「按《国语》《吴越春秋》,句践栖会稽后,惟恐国人之不蕃,不复禁其淫佚,其风至六国末犹在。故始皇为之厉禁,著于石刻。繁而不杀,此亦防民正俗之意也。」

\begin{yuanwen}
分天下以为三十六郡\footnote{text},郡置守、尉、监\footnote{text}。更名民曰“黔首\footnote{text}”。大酺。收天下兵\footnote{text},聚之咸阳,销以为钟鐻\footnote{编钟和钟架。鐻,jù。},金人十二,重各千石,置廷宫中。一法度衡石丈尺\footnote{text}。车同轨\footnote{text}。书同文字\footnote{text}。地东至海暨朝鲜,西至临洮、羌中,南至北乡户\footnote{窗户朝北开的地方,指南方极远之处。乡,同“向”。},北据河为塞,并阴山至辽东\footnote{text}。徙天下豪富于咸阳十二万户。诸庙及章台、上林皆在渭南\footnote{text}。秦每破诸侯,写放\footnote{仿照。放,通“仿”。}其宫室\footnote{text},作之咸阳北阪上\footnote{text}。南临渭,自雍门以东至泾、渭,殿屋复道\footnote{楼阁间架设的通道。}周阁相属\footnote{text}。所得诸侯美人钟鼓,以充入之。
\end{yuanwen}

始皇将全国划分为三十六个郡,每个郡设置守、尉、监。将民众改称为“黔首”。下令欢聚宴饮。始皇将全国的兵器都收聚到一起,集中在咸阳,熔化后铸成编钟,以及十二个金属人像,每一个重达一千石,放置在宫廷之中。始皇统一法令制度和度量衡标准,统一车轨的距离,统一用于书写的文字。秦朝的疆域向东到大海和朝鲜,向西到临洮、羌中,向南到窗户朝北开的地方,向北拒守黄河为屏障,从阴山一直到辽东。始皇将国内的十二万户富豪迁徙到咸阳。秦国历代帝王的宗庙和章台、上林都在渭水以南。秦国每消灭一个诸侯国,就会模仿其宫殿样式,在咸阳以北山坡上建造一座,向南临近渭水,从雍门以东到泾水、渭水一带,宫殿复道和环廊连续不绝。始皇从诸侯国抢来的美女、钟鼓,都安置在这里。

\begin{yuanwen}
二十七年,始皇巡陇西、北地,出鸡头山,过回中。焉作信宫\footnote{长信宫}渭南\footnote{text},已更命信宫为极庙,象天极\footnote{text}。自极庙道通郦山\footnote{text},作甘泉前殿\footnote{text}。筑甬道\footnote{两旁有遮蔽物的通道。},自咸阳属之\footnote{text}。是岁,赐爵一级。治驰道\footnote{供天子车马行驶的道路。}。
\end{yuanwen}

二十七年(前220年),始皇巡视陇西、北地,来到鸡头山,经过回中。于是在渭水以南建造了长信宫,不久又将长信宫改名为极庙,象征天上的北极星。从极庙铺设一条通往骊山的道路,又建造了甘泉宫前殿。修筑甬道,使咸阳与此相连。这一年,始皇赏赐天下人爵位一级。修筑驰道。

\begin{yuanwen}
二十八年\footnote{前219年},始皇东行郡县,上邹峄山。立石,与鲁诸儒生议,刻石颂秦德,议封禅望祭山川之事。乃遂上泰山,立石,封,祠祀。下,风雨暴至,休于树下,因封其树为五大夫。禅梁父。刻所立石,其辞曰:

\begin{quotation}
皇帝临位,作制明法,臣下脩饬\footnote{行为端正,不违礼义。}。二十有六年,初并天下,罔不宾服。亲巡远方黎民,登兹泰山,周览东极。从臣思迹,本原\footnote{推究。}事业,祗诵功德。治道运行,诸产得宜,皆有法式。大义休明,垂于后世,顺承勿革。皇帝躬圣,既平天下,不懈于治。夙兴夜寐,建设长利,专隆教诲。训经宣达,远近毕理,咸承圣志。贵贱分明,男女礼顺,慎遵职事。昭隔内外,靡不清净,施于后嗣。化及无穷,遵奉遗诏,永承重戒。
\end{quotation}

\end{yuanwen}

二十八年,始皇向东巡视郡县,登上邹峄山,树立石碑,和鲁郡的儒生们商议,在石碑上镌刻颂扬秦朝功德的文字,还讨论了封禅和望祭名山大川的事情。于是始皇登上泰山,树立石碑,积土为祭坛,祭祀上天。下山的时候,忽然遭遇狂风暴雨,始皇在一棵树下休息,因此他封这棵树为五大夫。始皇在梁父山祭祀大地。在所立的石碑上镌刻文字,上面的碑文说:

\begin{quotation}
皇帝即位,立下制度,严明法令,臣下严谨整饬。二十六年时,刚刚兼并天下,没有人不臣服。亲自巡视远方黎民,登上泰山,便览东方边疆。随行大臣追忆往昔,探求事业本源,恭敬赞颂功德。皇帝治国有道,各项生产安排得当,所有事务都有规范。大义美善明晰,作为后世榜样,沿袭不要改变。皇帝贤明通达,已经平定天下,治国仍然不懈。每天早起晚睡,为国长远谋划,尤其重视教导。宣明政令,各地井然有序,遵循皇帝意志。等级分明,男女有别,谨慎遵行职事。内外不同,无不清静,恩泽施于后代。教化所及,没有穷尽,遵守皇帝诏令,永遵重要告诫。
\end{quotation}

\begin{yuanwen}
于是乃并勃海\footnote{渤海。}以东,过黄、腄,穷成山,登之罘,立石颂秦德焉而去。
\end{yuanwen}

于是沿着渤海向东,经过黄县、腄县,走到成山的尽头,登上之罘山,树立赞颂秦朝功德的石碑后离去。

\begin{yuanwen}
南登琅邪,大乐之,留三月。乃徙黔首三万户琅邪台下,复十二岁。作琅邪台,立石刻,颂秦德,明得意。曰:

\begin{quotation}
维二十八年,皇帝作始。端平法度,万物之纪。以明人事,合同父子。圣智仁义,显白道理。东抚东土,以省卒士。事已大毕,乃临于海。皇帝之功,勤劳本事。上农除末,黔首是富。普天之下,抟\footnote{专一。}心揖\footnote{通“辑”,会集。}志。器械一量,同书文字。日月所照,舟舆所载。皆终其命,莫不得意。应时动事,是维皇帝。匡饬异俗,陵水经地。忧恤黔首,朝夕不懈。除疑定法,咸知所辟。方伯分职,诸治经易。举错必当,莫不如画。皇帝之明,临察四方。尊卑贵贱,不逾次行。奸邪不容,皆务贞良。细大尽力,莫敢怠荒。远迩\footnote{ěr}辟隐,专务肃庄。端直敦忠,事业有常。皇帝之德,存定四极。诛乱除害,兴利致福。节事以时,诸产繁殖。黔首安宁,不用兵革。六亲相保,终无寇贼。驩欣奉教,尽知法式。六合之内,皇帝之土。西涉流沙,南尽北户。东有东海,北过大夏\footnote{指太原。}。人迹所至,无不臣者。功盖五帝,泽及牛马。莫不受德,各安其宇。
\end{quotation}
\end{yuanwen}

始皇向南巡视登上琅邪山,非常喜欢那里,在那里停留三个月。于是他迁徙三万户黔首到琅邪台下,免除他们十二年的赋役。他下令建造琅邪台,树立石碑,歌颂秦朝的功德,表明心意。碑文说:

\begin{quotation}
二十八年,皇帝刚刚登基。制定公正法律,整治万物纲纪。以此明确人事,父子齐心协力。皇帝圣明仁义,明白事物道理。安抚东部地区,以此检阅士卒。巡视完全结束,来到海滨之地。皇帝功勋卓著,操劳国家大事。推行重农抑商,百姓富裕生活。举国上下齐心,一心一意跟随。统一器械度量,还有书写文字。日月所照之地,舟车所至之处,全部听从命令,没有忤逆之意。办事时机适当。整顿不良风俗,跨越千山万水。忧怀体恤百姓,早晚不懈努力。消除各种疑虑,制定法令制度,人人遵纪守法。郡守分工合作,政务简单易行。措施采取得当,全都整齐划一。皇帝神明伟大,明察秋毫四方。尊卑贵贱有别,不得逾越等级。除掉奸邪现象,百姓务必淳良。大事小情尽力,不敢懈怠荒废。不论远处近处,还是偏僻地方,一律肃穆庄重,品德正直忠厚,办事遵守规则。皇帝德泽无边,保佑安定四方。除害惩处暴乱,兴利造福于民。做事依从时令,发展各种产业。黔首安定和睦,不再发动战争。六亲彼此相安,再也没有盗贼。欣然接受教化,通晓法令制度。天下四海之内,尽是皇帝国土。向西连及沙漠,向南门户北开。向东直到东海,向北越过太原。人迹所到之地,无人不来臣服。皇帝功盖五帝,恩泽惠及牲畜。无人不受德化,各自安居乐业。
\end{quotation}

王世贞:「始皇既平六国,欲无不遂,所必不可得者寿耳。故方士以长生不死之说中之,遂遣徐福率童男女入海,为微行以辟恶鬼,幸梁山而捕中人,卒致山鬼持璧,捐馆沙邱。长生不死者,今安在哉!」顾炎武:「始皇崩于沙邱,乃又徙井陉抵九原,然后从直道至咸阳,回绕三四千里而归者,盖始皇先使蒙恬通道,自九原抵甘泉千八百里,若径归咸阳,不果行游,恐人疑揣,故载辒辌而北行。但以欺天下,虽君父之尸臭腐而不顾,亦残忍无人心之极矣。」
	
\begin{yuanwen}
维秦王兼有天下,立名为皇帝,乃抚东土,至于琅邪。列侯武城侯王离、列侯通武侯王贲、伦侯\footnote{爵位低于列侯,没有封邑。}建成侯赵亥、伦侯昌武侯成、伦侯武信侯冯毋择、丞相隗\footnote{wěi}林、丞相王绾、卿李斯、卿王戊、五大夫赵婴、五大夫杨樛从,与议于海上,曰:“古之帝者,地不过千里,诸侯各守其封域,或朝或否,相侵暴乱,残伐不止,犹刻金石,以自为纪。古之五帝三王\footnote{夏禹、商汤、周武王。},知教不同,法度不明,假威鬼神,以欺远方,实不称名,故不久长。其身未殁,诸侯倍叛,法令不行。今皇帝并一海内,以为郡县,天下和平。昭明宗庙,体道行德,尊号大成。群臣相与诵皇帝功德,刻于金石,以为表经\footnote{表率,典范。}。”
\end{yuanwen}

秦王兼并天下,设立皇帝称号,于是巡视东方疆土,来到琅邪郡。列侯武城侯王离、列侯通武侯王贲、伦侯建成侯赵亥、伦侯昌武侯成、伦侯武信侯冯毋择、丞相隗林、丞相王绾、卿李斯、卿王戊、五大夫赵婴、五大夫杨樛随行,与皇帝在海边商议,说:“古时候称帝的人,领地不超过一千里,诸侯分别守护自己的封地,有的朝贡,有的不来,相互侵扰,犯上作乱,残害生民,征战不止,却仍会立下石碑,用来记录自己的功业。古时候的五帝和三王,推行的教化不一致,施用的法令不明确,借助鬼神的威力,来欺骗远方的民众,名不副实,因此国运不长。这些帝王自己还没有死去,诸侯就已经背叛他们了,法令无法通行。现在皇帝统一四海之内,设置为郡县,天下太平安定。将祖先的美德发扬光大,实践正道而推行德化,尊号得以完备。群臣共同颂扬皇帝的功德,铭刻在金石上,以此为后世的表率。”

\begin{yuanwen}
既已,齐人徐市\footnote{fú}等上书,言海中有三神山,名曰蓬莱、方丈、瀛\footnote{yíng}洲,仙人居之。请得斋戒,与童男女求之。于是遣徐市发童男女数千人,入海求仙人。
\end{yuanwen}

事情过后,齐郡人徐巿等人上书,说海中有三座神山,名叫蓬莱、方丈、瀛洲,是仙人居住的地方。徐巿等人请求斋戒沐浴,带领童男童女前往寻找神山。于是始皇派徐巿挑选童男童女几千人,进入海中去寻找仙人。

\begin{yuanwen}
始皇还,过彭城,斋戒祷祠\footnote{泛指祭祀。},欲出周鼎泗水。使千人没水求之,弗得。乃西南渡淮水,之衡山、南郡。浮江,至湘山祠。逢大风,几不得渡。上问博士曰:“湘君何神?”

博士对曰:“闻之,尧女,舜之妻,而葬此。”于是始皇大怒,使刑徒三千人皆伐湘山树,赭\footnote{zhě}其山。上自南郡由武关归。
\end{yuanwen}

始皇返回,经过彭城,在那里斋戒祭祀,想要从泗水中打捞周朝的九鼎。他派出上千人潜入水中寻找,没有找到。于是向西南渡过淮水,前往衡山郡、南郡。泛舟长江,来到湘山祭祀。遭遇大风,几乎不能渡过长江。始皇问博士说:“湘君是什么神仙?”

博士回答说:“我听说,湘君是尧的女儿,舜的妻子,死后埋葬在这里。”于是始皇十分生气,命令服役的罪犯三千人砍光湘山上的树木,露出红褐色的山岩。始皇从南郡经武关回到国都。

\begin{yuanwen}
二十九年,始皇东游。至阳武博狼沙中,为盗所惊\footnote{指张良行刺之事。}。求弗得,乃令天下大索十日。
\end{yuanwen}

二十九年(前218年),始皇到东方巡视。来到阳武博狼沙,被盗贼所惊吓。始皇下令追捕,却没有抓到人,就下令在全国大规模搜查十天。

\begin{yuanwen}
登之罘,刻石。其辞曰:

\begin{quotation}
维二十九年,时在中春,阳和方起。皇帝东游,巡登之罘,临照于海。从臣嘉观,原念休烈,追诵本始。大圣作治,建定法度,显箸\footnote{同“著”,显明。}纲纪。外教诸侯,光\footnote{通“广”。}施文惠,明以义理。六国回辟\footnote{奸回邪僻。},贪戾无厌,虐杀不已。皇帝哀众,遂发讨师,奋扬武德。义诛信行,威燀\footnote{chǎn}旁达,莫不宾服。烹灭彊暴,振救黔首,周定四极。普施明法,经纬天下,永为仪则。大矣哉!宇县之中,承顺圣意。群臣诵功,请刻于石,表垂于常式。
\end{quotation}

其东观曰:

\begin{quotation}
维二十九年,皇帝春游,览省远方。逮于海隅,遂登之罘,昭临朝阳。观望广丽,从臣咸念,原道至明。圣法初兴,清理疆内,外诛暴彊。武威旁畅,振\footnote{同“震”。}动四极,禽灭六王。阐并天下,甾害\footnote{灾害。甾,通“灾”。}绝息,永偃戎兵。皇帝明德,经理宇内,视听不怠。作立大义,昭设备器,咸有章旗。职臣遵分,各知所行,事无嫌疑。黔首改化,远迩同度,临古绝尤。常职既定,后嗣循业,长承圣治。群臣嘉德,祗诵圣烈,请刻之罘。
\end{quotation}

旋,遂之琅邪,道上党入。
\end{yuanwen}

始皇登上之罘山,镌刻石碑。碑文说:

\begin{quotation}
二十九年,仲春时节,天气回暖。皇帝到东方巡视,登上之罘山,面朝大海。随行众臣观看美景,回想皇帝伟业,追念祖先事迹。伟大圣明的皇帝治理国家,建立法令制度,彰明纲常纪律。对外教化诸侯,广泛布施恩泽,使其明白义理。六国奸回邪僻,贪婪乖戾不休,残酷暴虐不止。皇帝同情民众,终于发兵征讨,发扬以武除暴的威德。正义的诛伐,诚信的行动,声威光大远播,没有人不臣服。消灭残暴的强敌,拯救天下的黔首,四海之内得以安定。推行严明的法令,经营治理天下,永远以此为典范。伟大啊!天下万民,遵行皇帝的旨意。群臣歌功颂德,请求将皇帝的功绩刻在石碑上,永为后世的法则。
\end{quotation}

东面台阁的碑文说:

\begin{quotation}
二十九年,皇帝在春季出游,巡视远方。来到海边,登上之罘山,面向朝阳。皇帝远望辽阔秀丽的景色,随行众臣都在追念往事,想到创业初始的圣明。皇帝制定的法律刚刚施行,对内清除奸邪,对外诛伐暴乱。军威远扬,震动四方,擒获六国之王。皇帝兼并天下,消除灾祸,永远平息战乱。皇帝圣明仁德,治理国家大事,视听从不懈怠。创立伟大道义,统一器量标准,使其都有规范。大臣遵守本分,都知道自己的职责,做事没有疑虑。黔首移风易俗,各地制度统一,自古绝无仅有。人们职分已定,后代遵循旧业,永远承袭英明的制度。群臣赞赏皇帝的美德,恭敬地歌颂他的伟大功勋,请求在之罘山上立下石碑。
\end{quotation}

不久,始皇就前往琅邪郡,经上党郡回到国都。

\begin{yuanwen}
三十年,无事。

三十一年十二月,更名腊\footnote{腊祭,即冬至后第三个戌日祭祀众神。}曰“嘉平”。赐黔首里六石米,二羊。始皇为微行咸阳,与武士四人俱,夜出逢盗兰池,见窘,武士击杀盗,关中大索二十日。米石千六百。
\end{yuanwen}

三十年(前217年),没有重大事件。

三十一年(前216年)十二月,把腊祭改名为“嘉平”。赏赐黔首每里六石米,两只羊。始皇微服出行巡视咸阳,有四名武士随行,夜晚出来在兰池遭遇盗贼,处境危险,武士杀死盗贼,于是在关中大规模搜捕二十天。米价达到每石一千六百钱。

\begin{yuanwen}
三十二年,始皇之碣石,使燕人卢生求羡门、高誓。刻碣石门。坏城郭,决通隄防。其辞曰:

\begin{quotation}
遂兴师旅,诛戮无道,为逆灭息。武殄暴逆,文复无罪,庶心咸服。惠论功劳,赏及牛马,恩肥土域。皇帝奋威,德并诸侯,初一泰平。堕\footnote{同“隳”,毁。}坏城郭,决通川防,夷去险阻。地势既定,黎庶\footnote{百姓。}无繇\footnote{徭役。},天下咸抚。男乐其畴,女修其业,事各有序。惠被诸产,久并来田\footnote{本地和外来的农户。},莫不安所。群臣诵烈,请刻此石,垂著仪矩\footnote{法度、规矩。}。
\end{quotation}

因使韩终、侯公、石生求仙人不死之药。始皇巡北边,从上郡入。燕人卢生使入海还,以鬼神事,因奏录图书,曰“亡秦者胡也”。始皇乃使将军蒙恬发兵三十万人北击胡,略取河南地\footnote{指河套以南地区。}。
\end{yuanwen}

三十二年(前215年),始皇前往碣石山,派燕人卢生寻找羡门、高誓两位仙人。在碣石门上镌刻文字。始皇下令毁坏城墙,挖通堤防。铭文说:

\begin{quotation}
调兵遣将,诛伐无道,消灭叛逆。用武力铲除残暴的逆贼,用文德保护无罪的好人,这都是民心所向。赏赐有功劳的人,恩惠遍及牲畜,滋养了境内的土地。皇帝奋发声威,用德化兼并诸侯,首次实现天下的统一安定。摧毁城墙,挖通堤防,铲平险阻。地理形势已经确定,百姓无需再服徭役。男子快乐地种田,女子操持家业,各项事情井然有序。皇帝的恩惠遍及各项产业,本地和外来的农民,没有人不安居乐业。群臣称赞皇帝的功绩,请求在这里镌刻石碑,为后世彰明法度。
\end{quotation}

始皇于是派韩终、侯公、石生去寻找仙人不死的灵药。始皇到北方边境巡视,经由上郡回到国都。燕人卢生从海上返回,述说神鬼之事,趁机献上他所抄录的图书,上面写着“亡秦者胡也”。始皇于是派将军蒙恬调发士兵三十万人向北攻打胡人,攻取了河南之地。

\begin{yuanwen}
三十三年,发诸尝逋亡人\footnote{亡命之人。}、赘\footnote{zhuì}婿、贾人略取陆梁地,为桂林、象郡、南海,以适遣戍。西北斥逐匈奴。自榆中并河以东,属之阴山,以为十四县,城河上为塞。又使蒙恬渡河取高阙、阳山、北假中,筑亭障\footnote{在边塞设置的堡垒。}以逐戎人。徙谪,实之初县。禁不得祠。明星出西方。

三十四年,适治狱吏不直者,筑长城及南越地。
\end{yuanwen}

三十三年(前214年),始皇下令征发曾经逃亡在外的犯人、入赘妻家的男子、商人去攻打陆梁地区,在那里设置桂林郡、象郡、南海郡,遣送罪犯去戍守。在西北驱逐匈奴。从榆中沿黄河向东,一直到阴山,在那里设置四十四个县,在黄河边筑城为要塞。始皇又派蒙恬渡过黄河占领高阙、阳山、北假中,修筑堡垒来驱逐戎人。迁徙被贬谪的犯人,让他们充实刚建立的县。禁止民间祭祀。彗星出现在西方。

三十四年(前213年),有断案不公的官吏,就会被送去修筑长城或戍守南越地区。

\begin{yuanwen}
始皇置酒咸阳宫,博士七十人前为寿。仆射周青臣进颂曰:“他时秦地不过千里,赖陛下神灵明圣,平定海内,放逐蛮夷,日月所照,莫不宾服。以诸侯为郡县,人人自安乐,无战争之患,传之万世。自上古不及陛下威德。”

始皇悦。博士齐人淳于越进曰:“臣闻殷周之王千馀岁,封子弟功臣,自为枝辅。今陛下有海内,而子弟为匹夫,卒有田常\footnote{春秋时齐国权臣,陈国公子完后裔,谥号成,本名恒,汉代避文帝刘恒讳改称其名为常。他当政期间杀齐简公,其曾孙田和最终篡夺齐国君位,史称“田氏代齐”。}、六卿\footnote{春秋时晋国权臣。晋文公时设三军,每军设一将一佐,称六卿。春秋末期,赵、魏、韩、范、智、中行为六卿,经过兼并,赵、魏、韩最终瓜分晋国,史称“三家分晋”。}之臣,无辅拂,何以相救哉?事不师古而能长久者,非所闻也。今青臣又面谀以重陛下之过,非忠臣。”

始皇下其议。丞相李斯曰:“五帝不相复,三代\footnote{夏、商、周。}不相袭,各以治,非其相反,时变异也。今陛下创大业,建万世之功,固非愚儒所知。且越言乃三代之事,何足法也?异时诸侯并争,厚招游学。今天下已定,法令出一,百姓当家则力农工,士则学习法令辟禁。今诸生不师今而学古,以非当世,惑乱黔首。丞相臣斯昧死言:古者天下散乱,莫之能一,是以诸侯并作,语皆道古以害今,饰虚言以乱实,人善其所私学,以非上之所建立。今皇帝并有天下,别黑白而定一尊。私学而相与非法教,人闻令下,则各以其学议之,入则心非,出则巷议,夸主以为名,异取\footnote{也作“异趣”,表达不同的观点。}以为高,率群下以造谤。如此弗禁,则主势降乎上,党与成乎下。禁之便。臣请史官非《秦记》皆烧之。非博士官所职,天下敢有藏《诗》、《书》、百家语者,悉诣守、尉杂烧之。有敢偶语《诗》、《书》者弃市\footnote{在闹市执行死刑,表示与众共弃。}。以古非今者族。吏见知不举者与同罪。令下三十日不烧,黥为城旦\footnote{罚罪犯筑城四年的劳役。}。所不去者,医药、卜筮、种树之书。若欲有学法令,以吏为师。”

制曰:“可。”
\end{yuanwen}

始皇在咸阳宫中摆酒设宴,博士七十人上前祝福始皇长寿。仆射周青臣进献颂辞说:“从前秦国的土地不超过一千里,仰赖陛下的神灵圣明,得以平定四海之内,驱逐戎狄蛮夷,太阳和月亮能够照耀到的地方,没有人不臣服。将诸侯的封国设置为郡县,每个人都安居乐业,没有战争的忧患,这样的伟业可以流传万世。自上古以来都没有人能够赶得上陛下的威德。”

始皇非常高兴。博士齐人淳于越进言说:“我听说殷周称王一千多年,分封子弟和功臣,让他们辅佐王室。现在陛下拥有四海之内的土地,可是王室子弟却是普通人,突然出现田常、六卿一样的乱臣,没有其他人的辅佐,靠谁来拯救呢?做事不效仿古法却能够长久的,我没有听说过。现在周青臣又当面阿谀来加重陛下的过错,不是忠臣。”

始皇将二人的议论下交群臣讨论。丞相李斯说:“五帝不相重复,三代不相沿袭,各按自己的方法治理国家,不是后者要与前者相违背,只是时代不同罢了。现在陛下开创伟大的事业,建立万世的功勋,本来就不是愚蠢的儒生所能够理解的。况且淳于越说的只是三代的事情,有什么值得效仿呢?从前诸侯并立竞争,以优厚的待遇招揽周游列国的学者。现在天下已经平定,法令都陛下统一颁布,百姓在家就应该努力务农做工,士人就应该学习法律回避禁令。现在的儒生们不学习当今的法令却学习古代的制度,以此指责当代的政治,来迷惑黔首。丞相臣李斯冒死直言:古时候天下分离混乱,没有人能够统一号令,所以诸侯并立,都崇尚称道古代来危害当今的学说,用虚伪的言论来掩盖事情的本质,人们赞美自己所偏爱的学说,以此指责陛下所建立的制度。现在皇帝兼并天下,辨别黑白而尊崇陛下一个人。私人办学而共同非议国家的法令,人们听见命令下达后,就用各自所学的知识来议论,入朝时在心里指责,出门候在街巷议论,夸奖主上来博取名声,表达异议来抬高自己,率领众人来造谣生事。像这样却不禁止,就会在上降低君主的威望,在下结党营私。禁止是正确的。我请求将《秦记》以外的史书都烧掉。除身居博士的官职外,天下有人敢私藏《诗》《书》、诸子百家著作,都要上交到郡守、县尉那里烧毁。有人敢偶尔谈论《诗》《书》就斩首示众。用古代的制度来指责当今法令的人灭族。官吏知情不举报就与罪犯同罪。命令下达三十天而不烧掉书籍的人,在脸上刺字,罚为城旦。禁令不需要销毁的,是医药、卜筮、种植方面的书籍。如果有人想要学习法令,可以向官吏学习。”

制命说:“可以。”

\begin{yuanwen}
三十五年,除\footnote{整治,开辟。}道,道九原抵云阳,堑山堙谷,直通之。于是始皇以为咸阳人多,先王之宫廷小:“吾闻周文王都丰,武王都镐,丰、镐之间,帝王之都也。”乃营作朝宫渭南上林苑中。先作前殿阿房,东西五百步,南北五十丈,上可以坐万人,下可以建五丈旗。周驰为阁道,自殿下直抵南山。表南山之颠\footnote{同“巅”,顶。}以为阙。为复道,自阿房渡渭,属之咸阳,以象天极阁道绝汉抵营室\footnote{二十八宿中北方七宿之一。}也。阿房宫未成;成,欲更择令名名之。作宫阿房,故天下谓之阿房宫。隐宫\footnote{一说指宫刑,一说应为“隐官”。服役者七十多万都是受宫刑者,似乎不可能,这里采用后一种说法。隐官,秦汉时收容刑徒以服劳役的官署。}徒刑者七十馀万人,乃分作阿房宫,或作丽山。发北山石椁\footnote{疑似衍文。},乃写\footnote{运输。}蜀、荆地材皆至。关中计宫三百,关外四百馀。于是立石东海上朐界中,以为秦东门。因徙三万家丽邑,五万家云阳,皆复不事十岁。
\end{yuanwen}

三十五年(前212年),始皇下令开通道路,从九原通到云阳,凿开大山,填平深谷,开通一条笔直的大道。这个时候始皇认为咸阳人口众多,先王的宫廷太小:“我听说周文王建都于丰,武王建都于镐,丰、镐一带,正是帝王建都的地方。”于是在渭水以南的上林苑中兴建宫殿。先兴建的是前殿阿房宫,东西长五百步,南北宽五十丈,殿上可以坐一万人,殿下可以竖起五丈高的旗杆。四周环绕可以驾车驰骋的阁道,从殿下一直通到南山。在南山顶修建观楼为标记。在空中架设复道,从阿房宫可以渡过渭水,与咸阳相连,以此象征天极阁道越过银河直通营室。阿房宫还没有完工,完工后,想要另取一个好听的名字。这座宫殿是在阿房建造的,所以天下人称其为阿房宫。隐官服役的罪犯七十多万人,分批去修建阿房宫,或者修建丽山陵墓。从北山采下石头,从蜀地、楚地运来木材,都送到阿房宫。关中的宫殿一共有三百座,关外有四百多座。于是在东海边的朐县境内立下石碑,以此为秦地东方的门户。因而迁徙三万户百姓到丽邑居住,五万户到云阳居住,都免除十年的徭役。

\begin{yuanwen}
卢生说始皇曰:“臣等求芝奇药仙者常弗遇,类物有害之者。方中,人主时为微行以辟恶鬼,恶鬼辟,真人至。人主所居而人臣知之,则害于神。真人者,入水不濡\footnote{沾湿。},入火不爇\footnote{ruò,烧。},陵云气,与天地久长。今上治天下,未能恬倓。原上所居宫毋令人知,然后不死之药殆可得也。”

于是始皇曰:“吾慕真人,自谓‘真人’,不称‘朕’。”乃令咸阳之旁二百里内宫观二百七十复道甬道相连,帷帐、锺鼓、美人充之,各案署不移徙。行所幸,有言其处者,罪死。始皇帝幸梁山宫,从山上见丞相车骑众,弗善也。中人\footnote{阉宦。}或告丞相,丞相后损车骑。

始皇怒曰:“此中人泄吾语。”案问莫服。当是时,诏捕诸时在旁者,皆杀之。自是后莫知行之所在。听事,群臣受决事,悉于咸阳宫。
\end{yuanwen}

卢生劝始皇说:“我们寻找灵药和仙人总是找不到,可能有什么东西妨碍了这件事。仙方上说,人主经常微服出巡可以躲避恶鬼,恶鬼避开了,真人就出现了。人主所居住的地方被人臣知道了,就会妨碍神仙到来。真人,进入水中不会沾湿衣服,进入火中不会烧伤身体,能够腾云驾雾,寿命与天地一样长久。现在皇帝治理天下,还不能做到清静无为。希望皇帝所居住的宫室不要让其他人知道,然后长生不死的灵药大概就能够得到了。”

于是始皇说:“我很羡慕那些真人,以后我就自称‘真人’,不再自称‘朕’了。”于是他下令将咸阳附近二百里以内的二百七十座宫殿都用复道和甬道连接起来,用帷帐、钟鼓、美人充实其中,各自按照登记的位置居处,不得擅自移动。皇帝所到之处,如果有人向外说出地点,就会论罪处死。始皇帝亲临梁山宫,从山上看见丞相的随行车马众多,认为不好。有宦者告知丞相,以后丞相出门就减少了车马的数量。

始皇生气地说:“一定是宦者泄露了我说过的话。”他下令追查,却没有人认罪。在这个时候,始皇下诏将当时在他身旁的人拘捕起来,全都杀掉。从此以后就没有人知道皇帝的行踪了。始皇处理事务,群臣领受命令,都在咸阳宫中进行。

\begin{yuanwen}
侯生、卢生相与谋曰:“始皇为人,天性刚戾自用,起诸侯,并天下,意得欲从,以为自古莫及己。专任狱吏,狱吏得亲幸。博士虽七十人,特备员弗用。丞相诸大臣皆受成事,倚辨\footnote{通“办”,办理,办事。}于上。上乐以刑杀为威,天下畏罪持禄,莫敢尽忠。上不闻过而日骄,下慑伏谩欺以取容。秦法,不得兼方,不验辄死。然候星气者至三百人,皆良士,畏忌讳谀,不敢端言其过。天下之事无小大皆决于上,上至以衡石量书,日夜有呈,不中呈不得休息。贪于权势至如此,未可为求仙药。”于是乃亡去。

始皇闻亡,乃大怒曰:“吾前收天下书不中用者尽去之。悉召文学方术士甚众,欲以兴太平,方士欲练\footnote{同“炼”,熔炼。}以求奇药。今闻韩众去不报,徐市等费以巨万计,终不得药,徒奸利相告日闻。卢生等吾尊赐之甚厚,今乃诽谤我,以重吾不德也。诸生在咸阳者,吾使人廉问,或为訞言以乱黔首。”

于是使御史悉案问诸生,诸生传相告引,乃自除犯禁者四百六十馀人,皆坑之咸阳,使天下知之,以惩后。益发谪徙边。

始皇长子扶苏谏曰:“天下初定,远方黔首未集,诸生皆诵法孔子,今上皆重法绳之,臣恐天下不安。唯上察之。”始皇怒,使扶苏北监蒙恬于上郡。
\end{yuanwen}

侯生、卢生一起商量说:“始皇这个人,生性刚毅残暴,自以为是,以诸侯的身份起兵,兼并天下,事情都称心如意,认为古往今来没有人能比得上他。他只重用断案的官吏,断案的官吏都受到宠幸。虽然博士有七十人,却只是充数的官员而不受重用。丞相和众大臣都是接受制定好的命令,按照他的意志做事情。皇帝喜欢用刑罚杀戮来树立威严,天下的官员都害怕犯罪而想要保住俸禄,没有人敢竭尽忠诚。皇帝听不到自己的过错而日益骄横,群臣害怕威刑而欺骗皇帝来求得安身。秦朝的法律规定,不准兼用两种方术,方术不能应验就处以死刑。然而占候星象云气的有三百多人,都是好人,他们害怕禁忌而委婉地奉承皇帝,不敢直言他的过错。天下的事情无论大小都由皇帝决断,皇帝甚至用称量奏书的重量,从早到晚都有奏书呈上,其间不可以呈上,不批阅完毕就不休息。他贪恋权势到了这种地步,我们不能为他寻找仙药。”

于是他们就逃走了。始皇听说方士逃走了,生气地说:“我以前收集天下的书籍,将没有用处的都销毁了。我又招揽了很多文学方术之士,想要实现天下太平的局面,方士想要炼制灵丹妙药。现在听说韩众等人逃跑不回来复命,徐巿等人耗费巨资,最后也没能找到仙药,每天只有一些小人为谋取私利向我奏报。我对卢生等人待遇优厚,现在他们却诽谤我,来加重我的不仁。那些在咸阳的儒生们,我将派人去审问,也许有人在用妖言迷惑黔首。”

于是始皇派御史审问儒生们,儒生们相互指责检举,始皇就亲自挑选出触犯禁令的儒生四百六十多人,将他们全部在咸阳坑杀,让天下人都知道这件事,以此惩戒后人。始皇又征发更多的人去戍守边境。

始皇的长子扶苏劝谏说:“天下刚刚平定,远方的黔首还没有安定,儒生们都是歌颂和效法孔子的人,当今圣上却用严酷的刑罚来惩治他们,我担心天下人心中不安。希望皇帝明察。”

始皇很生气,派扶苏到北方的上郡去做蒙恬的监军。

\begin{yuanwen}
三十六年,荧惑\footnote{火星。}守心\footnote{二十八宿中东方七宿之一。}。有坠星下东郡,至地为石,黔首或刻其石曰“始皇帝死而地分”。始皇闻之,遣御史逐问,莫服,尽取石旁居人诛之,因燔销其石。始皇不乐,使博士为《仙真人诗》,及行所游天下,传令乐人\footnote{善歌舞的艺人。}歌弦之。

秋,使者从关东夜过华阴平舒道,有人持璧遮使者曰:“为吾遗滈池君。”

因言曰:“今年祖龙\footnote{暗指秦始皇。}死。”

使者问其故,因忽不见,置其璧去。使者奉璧具以闻。始皇默然良久,曰:“山鬼固不过知一岁事也。”

退言曰:“祖龙者,人之先也。”

使御府视璧,乃二十八年行渡江所沉璧也。于是始皇卜之,卦得游徙吉。迁北河、榆中三万家。拜爵一级。
\end{yuanwen}

三十六年(前211年),荧惑星接近心宿。有一颗星在东郡陨落,到地面上变成一块石头,黔首中有人在石头上刻“始皇帝死而地分”的文字。始皇听说后,派御史前去审问,没有人认罪,就把居住在石头附近的居民都抓起来处死,用火销熔这块石头。始皇不高兴,让博士创作《仙真人诗》,记述他出行巡游天下的事件,传令乐工谱曲演唱。

秋季,一位使者从关东来,在夜晚经过华阴平舒,有人拿着玉璧拦住使者说:“替我把它送给滈池君。”

他趁机说:“今年祖龙死。”

使者问他其中的缘故,这个人忽然不见了,只留下他的玉璧。使者捧着玉璧上奏朝廷。始皇沉默了很长时间,说:“山鬼只不过知道一年的事情。”

他退朝后说:“祖龙,是人们的首领。”

始皇派御府来看玉璧,竟然是二十八年出行渡江时落入水中的那一块。于是始皇命人占卜,卦象显示出游迁徙就会吉利。于是他命令迁徙北河、榆中三万户百姓,赏赐爵位一级。

\begin{yuanwen}
三十七年十月癸丑,始皇出游。左丞相斯从,右丞相去疾守。少子胡亥爱慕请从,上许之。

十一月,行至云梦,望祀虞舜于九疑山。浮江下,观籍柯,渡海渚。过丹阳,至钱唐。临浙江,水波恶,乃西百二十里从狭中渡。上会稽,祭大禹,望于南海,而立石刻颂秦德。其文曰:

\begin{quotation}
皇帝休烈,平一宇内,德惠脩长。三十有七年,亲巡天下,周览远方。遂登会稽,宣省\footnote{体察。}习俗,黔首斋庄\footnote{虔诚庄重。}。群臣诵功,本原事迹,追首高明。秦圣临国,始定刑名,显陈旧章。初平法式,审别职任,以立恒常。六王专倍,贪戾慠猛,率众自彊。暴虐恣行,负力而骄,数动甲兵。阴通间使,以事合从,行为辟方\footnote{放纵,胡作非为。辟,同“僻”。方,通“放”。}。内饰诈谋,外来侵边,遂起祸殃。义威诛之,殄\footnote{tiǎn}熄暴悖,乱贼灭亡。圣德广密,六合之中,被泽无疆。皇帝并宇,兼听万事,远近毕清。运理群物,考验事实,各载其名。贵贱并通,善否陈前,靡有隐情。饰省\footnote{粉饰过失。省,通“眚”,过失。}宣义,有子而嫁,倍死不贞。防隔内外,禁止淫泆,男女絜诚。夫为寄豭\footnote{借人用于配种的公猪,比喻与人通奸的男子。豭,jiā。},杀之无罪,男秉义程。妻为逃嫁,子不得母,咸化廉清。大治濯俗,天下承风,蒙被休经。皆遵度轨,和安敦勉,莫不顺令。黔首脩絜,人乐同则,嘉保太平。后敬奉法,常治无极,舆舟不倾。从臣诵烈,请刻此石,光垂休铭\footnote{美好的铭文。}。
\end{quotation}
\end{yuanwen}

三十七年(前210年)十月癸丑日,始皇出外巡游。左丞相李斯跟随,右丞相冯去疾留守。始皇的小儿子胡亥非常羡慕,请求随行,始皇答应了。十一月,始皇一行人走到云梦泽,朝九疑山方向遥祭虞舜。在长江上顺流而下,观览籍柯,渡过海渚。途经丹阳,来到钱唐。面向浙江,波涛汹涌,于是向西行进一百二十里,从狭窄的地方渡过。始皇登上会稽山,祭祀大禹,望祭南海,并且在山上立石碑赞颂秦朝的功德。碑文说:

\begin{quotation}
皇帝建立伟业,平定统一海内,仁德恩惠长久。在位三十七年,亲自巡视天下,遍览遥远之地。此时登上会稽,体察民间风俗,黔首虔诚庄重。群臣歌功颂德,回顾创业始末,追溯英明决策。秦国圣人治国,首创法令刑律,彰明旧有典章。初定制度规范,审核官吏职事,以此建立常法。六王独断专行,贪婪乖戾傲慢,率领民众图强。肆意暴虐行凶,骄纵自以为是,屡次挑起战乱。暗中派遣间谍,谋求合纵抗秦,做法邪僻放纵。对内虚伪狡诈,对外侵扰边境,终于引来灾难。皇帝兴起义兵,征伐平定暴虐,剿灭乱臣贼子。圣德宏大缜密,天下四海之内,蒙受无限恩泽。皇帝兼并天下,处理各项事务,远近井然有序。运筹治理万物,查验事情本质,各自确立名分。无论高低贵贱,不管善恶黑白,丝毫没有隐瞒。假意宣扬道义,生子改嫁他人,背叛亡夫不忠。内外隔离开来,禁止过度放纵,男女洁身自爱。丈夫与人通奸,杀他也不犯罪,男人遵守规范。妻子与人私奔,子不认她为母,风气正直清白。大治移风易俗,天下接受教化,得到良好治理。人人奉公守法,平和安定敦厚,无人不遵法令。黔首修身养性,乐于遵守法制,共同保有太平。后世遵纪守法,以此长治久安,车船不会倾覆。群臣赞颂伟业,请求刻石立碑,铭文永垂后世。
\end{quotation}

\begin{yuanwen}
还过吴,从江乘渡。并海上,北至琅邪。方士徐市等入海求神药,数岁不得,费多,恐谴,乃诈曰:“蓬莱药可得,然常为大鲛\footnote{鲨鱼。}鱼所苦,故不得至,原请善射与俱,见则以连弩射之。”

始皇梦与海神战,如人状。问占梦,博士曰:“水神不可见,以大鱼蛟龙为候。今上祷祠备谨,而有此恶神,当除去,而善神可致。”

乃令入海者赍\footnote{jī}捕巨鱼具,而自以连弩候大鱼出射之。自琅邪北至荣成山,弗见。至之罘,见巨鱼,射杀一鱼。遂并海西。
\end{yuanwen}

始皇返回时途经吴县,乘船渡过长江。沿海岸而上,向北来到琅邪。方士徐巿等人到海中寻找神药,多年过去仍未找到,耗费大量财物,担心受到惩罚,就欺骗始皇说:“蓬莱的神药是可以找到的,然而我总被一条大鲨鱼所困扰,所以不能到达蓬莱,希望派一些擅长射箭的人与我同行,见到大鲨鱼就用连弩射它。”

始皇在梦中与海神交战,海神长着人的相貌。始皇命人占卜此梦,博士说:“水神是看不到的,他以大鱼和蛟龙为征兆。现在陛下虔诚地祷告和祭祀,却梦见这样一个凶神,应当将其除掉,然后善神就能出现了。”

于是始皇命令海上的人携带捕杀大鱼的器具,并且自己带上连弩等待大鱼出现时射它。从琅邪向北行进到荣成山,都没有遇见大鱼。到了之罘山,终于看见大鱼,射死了一条。于是始皇沿海岸西行。

\begin{yuanwen}
至平原津而病。始皇恶言死,群臣莫敢言死事。上病益甚,乃为玺书赐公子扶苏曰:“与丧会咸阳而葬。”

书已封,在中车府令赵高行符玺事所,未授使者。

七月丙寅,始皇崩于沙丘平台。丞相斯为上崩在外,恐诸公子及天下有变,乃秘之,不发丧。棺载辒凉车\footnote{可以供人躺下休息的车子,有帷帐。}中,故幸宦者参乘\footnote{陪乘的人。古代乘车,御者居中,尊者在左,参乘在右。},所至上食。百官奏事如故,宦者辄从辒凉车中可其奏事。独子胡亥、赵高及所幸宦者五六人知上死。赵高故尝教胡亥书及狱律令法事,胡亥私幸之。高乃与公子胡亥、丞相斯阴谋破去始皇所封书赐公子扶苏者,而更诈为丞相斯受始皇遗诏沙丘,立子胡亥为太子。更为书赐公子扶苏、蒙恬,数以罪,(其)赐死。语具在《李斯传》中。行,遂从井陉抵九原。会暑,上辒车臭,乃诏从官令车载一石鲍鱼\footnote{咸鱼。},以乱其臭。
\end{yuanwen}

始皇来到平原津就生病了。他厌恶说死,群臣没有敢提到死的事情。始皇的病情日益加重,于是就写了一封盖有御玺的书信赐给公子扶苏说:“回来护送我的灵柩回咸阳下葬。”

书信已经封好,放在掌管符节印玺的中车府令赵高那里,没有交给使者。

七月丙寅日,始皇在沙丘平台去世。丞相李斯认为皇帝在外地去世,担心众皇子和天下百姓会生出变故,就封锁了消息,不举办丧事。始皇的棺材用辒凉车运载,由原来受到宠幸的宦者在右侧陪乘,随时进献饮食。百官也照常奏报政事,宦者总是从辒凉车里批准他们所奏之事。始皇去世的事情只有公子胡亥、赵高和五六个亲近的宦者知道。赵高过去曾经教胡亥学习写字和刑狱法令之事,胡亥私下里与他很亲近。赵高就与公子胡亥、丞相李斯暗中商议毁掉始皇所封好的赐给公子扶苏的诏书,而谎称丞相李斯在沙丘接受始皇遗诏,立儿子胡亥为太子。他们又另写了诏书赐给公子扶苏、蒙恬,列举他们的罪状,命令他们自杀。这些事情在《李斯列传》中。众人继续前行,最终从井陉来到九原。当时正值暑天,始皇的辒凉车散发出腐臭的气味,李斯以诏书命令随行官员每车载一石咸鱼,用来混淆始皇尸体的气味。

\begin{yuanwen}
行从直道至咸阳,发丧。太子胡亥袭位,为二世皇帝。

九月,葬始皇郦山。始皇初即位,穿治郦山,及并天下,天下徒送诣七十馀万人,穿三泉,下铜\footnote{以铜汁灌地,防止渗水。}而致椁,宫观、百官、奇器、珍怪徙臧满之。令匠作机弩矢,有所穿近者辄射之。以水银为百川江河大海,机相灌输,上具天文,下具地理。以人鱼膏\footnote{一说为娃娃鱼的油脂,一说为鲸鱼的油脂,由于年代久远,缺少可靠的记载,难以断定。}为烛,度不灭者久之。

二世曰:“先帝后宫非有子者,出焉不宜。”皆令从死,死者甚众。

葬既已下,或言工匠为机,臧皆知之,臧重即泄。大事毕,已臧,闭中羡\footnote{yán},下外羡门\footnote{墓道之门。},尽闭工匠臧者,无复出者。树草木以象山。
\end{yuanwen}

众人经由直道回到咸阳,这才公布始皇的死讯。太子胡亥承袭帝位,成为二世皇帝。

九月,在郦山安葬始皇。始皇刚即位时,就开凿郦山修建陵墓。等到他兼并天下,从全国征发七十多万刑徒,挖地至三重泉水的深度,用铜汁灌地为外棺,陵墓内部宫室、官署、奇珍、异宝充斥其中。始皇命令工匠制造机关弓弩,有人挖开陵墓接近墓室就会遭到射击。墓室里用水银做成江河湖海的样子,用机关使其相互灌注流通,上方有日月星辰,下方有山川景观。用人鱼的油脂做成蜡烛,估计可以点燃很长时间。

二世说:“先帝后宫中没有儿子的嫔妃,放出宫外不合适。”他下令让她们都为始皇殉葬,死去的人非常多。

下葬完毕,有人说工匠制造了机关,对里面所藏的珍宝很清楚,藏有珍宝的消息恐怕会泄漏出去。丧事办完以后,把珍宝封藏好,又把墓道的门关闭起来,把工匠和珍宝都关在里面,没有一个人逃出来。又在坟丘上种植草木,使其看上去像山丘一样。

\begin{yuanwen}
二世皇帝元年,年二十一。赵高为郎中令,任用事\footnote{掌握大权。}。

二世下诏,增始皇寝庙牺牲及山川百祀之礼。令群臣议尊始皇庙。群臣皆顿首言曰:“古者天子七庙\footnote{指天子七代祖先之庙。始祖居中,其余按父子顺序分居两侧,左称昭,右称穆。传承数代以后,在位天子只保留与自己世代较近祖宗的庙,除始祖和个别功勋卓著的祖先外,与其世代相隔太远的庙就要毁弃,而将神主移至始祖庙中,始终保持七庙的数量。},诸侯五,大夫三,虽万世世不轶毁。今始皇为极庙,四海之内皆献贡职,增牺牲,礼咸备,毋以加。先王庙或在西雍,或在咸阳。天子仪当独奉酌祠始皇庙。自襄公已下轶毁。所置凡七庙。群臣以礼进祠,以尊始皇庙为帝者祖庙。皇帝复自称‘朕’。”
\end{yuanwen}

二世皇帝元年(前209年),胡亥二十一岁。赵高担任郎中令,受重用执掌政事。

二世下诏书,增加始皇陵寝和宗庙用的牲畜数量,提高对名山大川的祭祀等级,让群臣议论尊崇始皇庙的方法。群臣都叩头进言说:“古时候天子有七座宗庙,诸侯有五座宗庙,大夫有三座宗庙,即使经过万世也不能毁坏。现在始皇庙是级别最高的庙,四海之内都进献贡物,增加祭祀用的牲畜,礼仪都很完备,不需要再增加了。先王庙有的在西雍,有的在咸阳。按照天子的礼仪,陛下应该亲自捧着酒杯去祭拜始皇庙。自襄公以下的宗庙都已毁掉,所设置的一共七座宗庙。群臣按照礼仪进献祭品,以此尊崇始皇庙为皇帝的祖庙。皇帝应该重新自称‘朕’。”

\begin{yuanwen}
二世与赵高谋曰:“朕年少,初即位,黔首未集附。先帝巡行郡县,以示彊,威服海内。今晏然\footnote{安定的样子。}不巡行,即见弱,毋以臣畜天下。”

春,二世东行郡县,李斯从。到碣石,并海,南至会稽,而尽刻始皇所立刻石,石旁著大臣从者名,以章先帝成功盛德焉。

皇帝曰:“金石刻尽始皇帝所为也。今袭号而金石刻辞不称始皇帝,其于久远也如后嗣为之者,不称成功盛德。”

丞相臣斯、臣去疾、御史大夫臣德昧死言:“臣请具刻诏书刻石,因明白矣。臣昧死请。”

制曰:“可。”

遂至辽东而还。
\end{yuanwen}

二世和赵高商议说:“我年龄小,刚登上帝位,黔首还没有归顺。先帝巡视郡县,以此显示强大,用威势让海内民众臣服。现在天下安定而不出去巡游,就是显示弱小,无法让天下人臣服。”

春季,二世向东巡视郡县,李斯随行。到达碣石后,沿海岸行进,向南到达会稽,并且在始皇所立的石碑上都刻了文字,石碑旁还刻有随行大臣的名字,来彰显先帝的丰功伟绩。

皇帝说:“这些石碑都是始皇帝所立的。现在我继承了皇帝的称号,可是这些石碑所刻文字不称始皇帝,等到很久以后看起来就像是后来继位的人所立的,配不上始皇帝的丰功伟绩。”

丞相臣李斯、臣冯去疾、御史大夫臣德冒死进言:“我们请求将诏书全部刻在石碑上,这样就清楚了。我们冒死提出这一请求。”

制命说:“可以。”

二世到达辽东后返回国都。

\begin{yuanwen}
于是二世乃遵用赵高,申法令。乃阴与赵高谋曰:“大臣不服,官吏尚彊,及诸公子必与我争,为之奈何?”

高曰:“臣固原言而未敢也。先帝之大臣,皆天下累世名贵人也,积功劳世以相传久矣。今高素小贱,陛下幸称举,令在上位,管中事。大臣鞅鞅\footnote{同“怏怏”,不满意的样子。},特以貌从臣,其心实不服。今上出,不因此时案郡县守尉有罪者诛之,上以振威天下,下以除去上生平所不可者。今时不师文而决于武力,原陛下遂从时毋疑,即群臣不及谋。明主收举馀民\footnote{遗民,亡国之人。},贱者贵之,贫者富之,远者近之,则上下集而国安矣。”

二世曰:“善。”乃行诛大臣及诸公子,以罪过连逮少近官三郎\footnote{指近侍、郎官。},无得立者,而六公子戮死于杜。公子将闾昆弟三人囚于内宫,议其罪独后。二世使使令将闾曰:“公子不臣,罪当死,吏致法焉。”

将闾曰:“阙廷之礼,吾未尝敢不从宾赞\footnote{司仪。}也;廊庙之位,吾未尝敢失节也;受命应对,吾未尝敢失辞\footnote{说错话。}也。何谓不臣?原闻罪而死。”

使者曰:“臣不得与谋,奉书从事。”

将闾乃仰天大呼天者三,曰:“天乎!吾无罪!”昆弟三人皆流涕拔剑自杀。宗室振恐。群臣谏者以为诽谤,大吏持禄取容,黔首振恐。
\end{yuanwen}

这时二世就采用赵高的建议,申明法令。他私下里和赵高谋划说:“大臣不顺服,官吏的势力还很强大,皇子们一定会与我争夺帝位,怎么办呢?”

赵高说:“我本来就想说却没敢说。先帝的大臣,都是天下世代享有声望的权贵,他们建立功勋世代传承已经很久了。现在我向来身份卑贱,承蒙陛下的信任抬举,才让我身居高位,得以掌管宫中事务。大臣心里都很不满意,特意在表面上很顺从我,他们心里其实不服气。现在陛下外出,不如趁这个时机调查郡县守尉,如果有罪就将他处死,对上能够用声威震慑天下,对下能够铲除陛下平生所不满意的人。当今时代不能采用文治的方法而要采用武力的手段,希望陛下能够顺应时势而不要犹豫,在群臣还没来得及谋划的时候动手。圣明的君主可以收养遗民,使卑贱的人高贵,使贫穷的人富有,使远方的人归顺,那么上下就会和睦,而国家就会安定了。”

二世说:“很好。”于是他诛杀很多大臣和众皇子,因罪株连近侍郎官,没有能够保住官位的人,并且始皇的其他六个儿子都在杜县被处死。公子将闾兄弟三人囚禁在内宫,最后单独审议他们的罪行。二世派使者对将闾说:“公子没有尽到大臣的职责,论罪应当处死,官吏将会施以法律制裁。”

将闾说:“宫廷的礼仪,我从来不敢不服从司仪的指挥;朝廷的位次,我从来不敢不遵守礼节;接受命令回应质询,我从来不敢有言语的差错。为什么说我没有尽到大臣的职责呢?希望让我知道自己所犯下的罪行之后再死去。”

使者说:“我没有参与谋划,只是奉诏事。”

将闾于是仰天大喊三声,说:“苍天啊!我没有罪!”兄弟三人都流着眼泪拔剑自杀。宗室子弟都非常害怕。群臣进谏的就会被认为是诽谤,大臣拿着俸禄以求安身,黔首都很害怕。

\begin{yuanwen}
四月,二世还至咸阳,曰:“先帝为咸阳朝廷小,故营阿房宫为室堂。未就,会上崩,罢其作者,复土郦山。郦山事大毕,今释阿房宫弗就,则是章先帝举事过也。”

复作阿房宫。外抚四夷,如始皇计。尽徵其材士五万人为屯卫咸阳,令教射狗马禽兽。当食者多,度不足,下调郡县转输菽粟刍藁\footnote{刍藁,牲畜吃的草料。},皆令自赍粮食,咸阳三百里内不得食其穀。用法益刻深。
\end{yuanwen}

四月,二世回到咸阳,说:“先帝因为咸阳宫廷狭小,所以兴建阿房宫作为居室和堂屋。宫殿还没有建成,就赶上先帝去世,停止动工,调去郦山建造墓地。郦山的工程已经完毕,现在搁置阿房宫不去完成,就是表明先帝做的事情是错的。”

于是继续修建阿房宫。二世对外安抚四方夷狄,沿用始皇的策略。他征集强壮的士兵五万人来驻守咸阳,教习射箭,加上狗马牲畜,每天都要消耗很多粮食,估计存储不够,就向下令从郡县转运粮食和草料,负责运送的人都需要自带粮食,咸阳三百里以内的百姓不能吃运来的谷物。二世推行的法令更加严苛。

\begin{yuanwen}
七月,戍卒陈胜等反故荆地,为“张楚”。胜自立为楚王,居陈,遣诸将徇地。山东郡县少年苦秦吏,皆杀其守尉令丞反,以应陈涉\footnote{陈胜,字涉。},相立为侯王,合从西乡,名为伐秦,不可胜数也。

谒者\footnote{官名,负责传达政令。}使东方来,以反者闻二世。二世怒,下吏。后使者至,上问,对曰:“群盗,郡守尉方逐捕,今尽得,不足忧。”

上悦。武臣自立为赵王,魏咎为魏王,田儋\footnote{dān}为齐王。沛公起沛。项梁举兵会\footnote{kuài}稽郡。
\end{yuanwen}

七月,戍守边境的士卒陈胜等人在楚国故地反叛,建立“张楚”。陈胜自立为楚王,居住在陈县,派众将领攻取土地。山东郡县的年轻人深受秦朝官吏的折磨,都杀死当地的守尉和令丞反叛,来响应陈涉,相继被封为诸侯王,联合向西进军,以讨伐暴秦为名,人数多得无法计算。

谒者出使东部回来后,将反叛的事情报告给二世。二世大怒,把谒者下交狱吏治罪。后派出的使者回来后,二世问他情况,使者回答说:“这些盗贼,郡里的守尉正在追捕,现在已经全部抓获了,不足以担忧。”

二世很高兴。武臣自立为赵王,魏咎自立为魏王,田儋自立为齐王。沛公在沛县起兵。项梁在会稽起兵。

\begin{yuanwen}
二年冬,陈涉所遣周章等将西至戏,兵数十万。二世大惊,与群臣谋曰:“奈何?”

少府章邯曰:“盗已至,众彊,今发近县不及矣。郦山徒多,请赦之,授兵以击之。”

二世乃大赦天下,使章邯将,击破周章军而走,遂杀章曹阳。二世益遣长史司马欣、董翳\footnote{yì}佐章邯击盗,杀陈胜城父,破项梁定陶,灭魏咎临济。楚地盗名将已死,章邯乃北渡河,击赵王歇等于钜鹿。
\end{yuanwen}

二年(前208年)冬季,陈涉所派遣的周章等将领向西行进到戏水,兵力数十万。二世非常吃惊,和群臣商议说:“怎么办呢?”

少府章邯说:“盗贼已经到来,人多势强,现在调发附近县里的军队已经来不及了。郦山有很多刑徒,请求赦免他们,发给他们兵器来迎击盗贼。”

二世于是大赦天下,派章邯率领刑徒,打退了周章的军队,最终在曹阳杀死了周章。二世又派长史司马欣、董翳协助章邯攻打盗贼,在城父杀死了陈胜,在定陶打败了项梁,在临济消灭了魏咎。楚地盗贼中有名的将领都已经死了,章邯就向北渡过黄河,在钜鹿攻打赵王歇等人。

\begin{yuanwen}
赵高说二世曰:“先帝临制天下久,故群臣不敢为非,进邪说。今陛下富于春秋,初即位,奈何与公卿廷决事?事即有误,示群臣短也。天子称朕\footnote{有征兆的意思。},固不闻声。”

于是二世常居禁中,与高决诸事。其后公卿希得朝见,盗贼益多,而关中卒发东击盗者毋已。
\end{yuanwen}

赵高劝二世说:“先帝统治天下时间很长,因此群臣不敢做坏事,说坏话。现在陛下年轻,刚登上帝位,怎么能和公卿在朝堂上决断事情呢?如果事情有差错,就把自己的短处在群臣面前暴露了。天子自称朕,本来就不能让人轻易听到声音。”

于是二世经常住在宫里,和赵高决断各种政务。从此以后公卿就很少有机会见到皇帝了,盗贼越来越多,朝廷不断从关中调发士卒向东攻打盗贼。

\begin{yuanwen}
右丞相去疾、左丞相斯、将军冯劫进谏曰:“关东群盗并起,秦发兵诛击,所杀亡甚众,然犹不止。盗多,皆以戌\footnote{xū}漕转作事苦,赋税大也。请且止阿房宫作者,减省四边戍转。”

二世曰:“吾闻之韩子\footnote{韩非。引文出自《韩非子·五蠹》。}曰:‘尧舜采椽不刮,茅茨不翦\footnote{同“剪”。},饭土塯\footnote{liù},啜\footnote{chuò}土形\footnote{通“型”,一种瓦器。},虽监门之养,不觳\footnote{薄。}于此。禹凿龙门,通大夏,决河亭\footnote{同“停”,停滞。}水,放之海,身自持筑臿,胫毋毛,臣虏之劳不烈于此矣。’凡所为贵有天下者,得肆意极欲,主重明法,下不敢为非,以制御海内矣。夫虞、夏之主,贵为天子,亲处穷苦之实,以徇百姓,尚何于法?朕尊万乘,毋其实,吾欲造千乘之驾,万乘之属,充吾号名。且先帝起诸侯,兼天下,天下已定,外攘四夷以安边竟\footnote{同“境”。},作宫室以章得意,而君观先帝功业有绪。今朕即位二年之间,群盗并起,君不能禁,又欲罢先帝之所为,是上毋以报先帝,次不为朕尽忠力,何以在位?”

下去疾、斯、劫吏,案责他罪。

去疾、劫曰:“将相不辱。”自杀。

斯卒囚,就五刑。
\end{yuanwen}

右丞相冯去疾、左丞相李斯、将军冯劫进谏说:“关东各路盗贼同时起兵,秦国发兵镇压,已经杀死很多人了,可是叛乱依然没有平息。盗贼太多,正是因为戍守转运和劳役使百姓过于辛苦,赋税过于沉重。请求暂时停止阿房宫的兴建,减少四方的戍守和转运。”

二世说:“我听韩子说:‘尧舜的屋椽不加整治,屋顶的茅草不加修葺,吃饭用土碗,喝水用瓦盆,现在即使是看守城门的人,也不会比这更节俭了。大禹开凿龙门,打通华夏的道路,疏导黄河的积水,将其引入大海,亲自拿着板筑和土锸,磨光了小腿的汗毛,现在奴仆的劳苦也不会比这更厉害了。’那些尊贵而拥有天下的人,应该为所欲为,主要是严明法令,下面的人就不敢做坏事,就能控制四海之内了。像虞、夏的君主,尽管贵为天子,却亲自从事穷苦的工作,为百姓做表率,这样还要法令干什么?我贵为万乘之君,却名不副实,我要制造千乘的车驾,设置万乘的臣属,以此符合的名号。况且先帝出身于诸侯,后来兼并天下,现在天下已经平定,对外能够驱逐四方夷狄来安定边境,对内兴修宫殿来显示得意,而各位都看到了先帝功业的开创过程。现在我即位两年之内,各路盗贼同时起兵,各位不能够制止,又想要废除先帝推行的政策,首先无法报答先帝,其次也不能对我尽忠效力,凭什么身居高位?”

他把冯去疾、李斯、冯劫下交狱吏,追究他们的各种罪行。

冯去疾、冯劫说:“将相不能受到侮辱。”于是他们自杀了。

李斯最后被囚禁,遭受了各种刑罚。

\begin{yuanwen}
三年,章邯等将其卒围钜鹿,楚上将军项羽将楚卒往救钜鹿。冬,赵高为丞相,竟案李斯杀之。夏,章邯等战数却,二世使人让邯,邯恐,使长史欣请事。赵高弗见,又弗信。欣恐,亡去,高使人捕追不及。欣见邯曰:“赵高用事于中,将军有功亦诛,无功亦诛。”

项羽急击秦军,虏王离,邯等遂以兵降诸侯。
\end{yuanwen}

三年(前207年),章邯等人率领他们的士兵包围了钜鹿,楚国的上将军项羽率领楚国的士兵前往救援钜鹿。冬季,赵高担任丞相,最终将李斯审问后杀死。夏季,章邯等人作战屡次败退,二世派人责备章邯,章邯心里很害怕,派长史司马欣向朝廷请示。赵高不接见他,又不信任他。司马欣也很害怕,就逃走了,赵高派人追捕却没有抓到。司马欣见到章邯说:“赵高在朝中独揽大权,将军有功会被处死,无功也会被处死。”

项羽猛攻秦军,俘虏了王离,章邯等人于是带领军队向诸侯投降。

\begin{yuanwen}
八月己亥,赵高欲为乱,恐群臣不听,乃先设验,持鹿献于二世,曰:“马也。”

二世笑曰:“丞相误邪?谓鹿为马。”

问左右,左右或默,或言马以阿顺赵高。或言鹿,高因阴中诸言鹿者以法。后群臣皆畏高。
\end{yuanwen}

八月己亥日,赵高想要作乱,担心群臣不服,就事先做了一个试验,他找来一只鹿献给二世,说:“这是一匹马。”

二世笑着说:“丞相错了吧?把鹿说成是马。”

赵高问身边的大臣,有的沉默不语,有的为了迎合赵高就说是马,也有人说是鹿。赵高趁机把那些说是鹿的人暗中治罪。后来群臣都很畏惧赵高。

\begin{yuanwen}
高前数言“关东盗毋能为也”,及项羽虏秦将王离等钜鹿下而前,章邯等军数却,上书请益助,燕、赵、齐、楚、韩、魏皆立为王,自关以东,大氐\footnote{同“抵”。}尽畔秦吏应诸侯,诸侯咸率其众西乡。沛公将数万人已屠武关,使人私于高,高恐二世怒,诛及其身,乃谢病不朝见。二世梦白虎齧(啮)其左骖马,杀之,心不乐,怪问占梦。

卜曰:“泾水为祟。”

二世乃斋于望夷宫,欲祠泾,沈四白马。使使责让高以盗贼事。

高惧,乃阴与其婿咸阳令阎乐、其弟赵成谋曰:“上不听谏,今事急,欲归祸于吾宗。吾欲易置上,更立公子婴。子婴仁俭,百姓皆载\footnote{同“戴”,拥护。}其言。”

使郎中令为内应,诈为有大贼,令乐召吏发卒,追劫乐母置高舍。遣乐将吏卒千馀人至望夷宫殿门,缚卫令仆射,曰:“贼入此,何不止?”

卫令曰:“周庐设卒甚谨,安得贼敢入宫?”乐遂斩卫令,直将吏入行射,郎宦者大惊,或走或格,格者辄死,死者数十人。郎中令与乐俱入,射上幄坐帏。

二世怒,召左右,左右皆惶扰不斗。旁有宦者一人,侍不敢去。二世入内,谓曰:“公何不蚤告我?乃至于此!”

宦者曰:“臣不敢言,故得全。使臣蚤言,皆已诛,安得至今?”

阎乐前即二世数曰:“足下\footnote{对上级的尊称,后来多用于称呼同辈。}骄恣,诛杀无道,天下共畔足下,足下其自为计。”

二世曰:“丞相可得见否?”

乐曰:“不可。”

二世曰:“吾原得一郡为王。”弗许。

又曰:“原为万户侯。”弗许。

曰:“原与妻子为黔首,比诸公子。”

阎乐曰:“臣受命于丞相,为天下诛足下,足下虽多言,臣不敢报。”麾其兵进。二世自杀。
\end{yuanwen}

赵高以前多次说“关东的盗贼不会有什么作为”,等到项羽在钜鹿俘虏了秦将王离等人后继续前进,章邯等人的军队屡次败退,上书请求增兵援助,燕、赵、齐、楚、韩、魏都自立为王,自函谷关以东,民众大多背叛秦朝的官吏响应诸侯,诸侯都率领各自的军队向西进军。沛公率领几万人已经屠戮武关,派人私下与赵高联系。赵高担心二世发怒,自己受到诛杀,就以生病为借口不去朝见皇帝。

二世梦见一只白虎咬他所乘车驾左边的马,将其咬死,心里很不高兴,他感到奇怪,就去问占卜的人。

占卜的人说:“泾水之神在作祟。”

二世于是在望夷宫斋戒,想要祭祀泾水,将四匹白马沉入水中。他派使者以盗贼的事情指责赵高。

赵高很害怕,就与他的女婿咸阳令阎乐、他的弟弟赵成暗中谋划说:“皇帝不听劝谏,现在事情紧急,想要嫁祸给我们家族。我想要废掉皇帝,改立公子婴。公子婴仁爱节俭,百姓都拥戴他。”

赵高派郎中令做内应,欺骗二世说有大群盗贼打过来,命令阎乐召集官吏征发士兵,挟持了阎乐的母亲安置在赵高的家里。赵高派阎乐带领一千多名士卒来到望夷宫殿门,将卫令仆射捆起来,说:“盗贼已经来到这里,为什么不阻止?”

卫令说:“周庐设置的守备非常严谨,怎么会有盗贼敢闯入宫中呢?”

阎乐于是斩杀卫令,径直率领吏卒边走边射箭,郎官和宦者都非常惊慌,有的逃窜,有的搏斗,搏斗的人都被杀死,死了几十人。郎中令和阎乐一起进入宫中,朝二世坐息的帷帐射箭。

二世大怒,召集身边的侍卫,侍卫都感到害怕而不敢反抗。身边有一个宦者,始终陪着二世不敢离去。

二世逃到内室,对近侍们说:“各位怎么不早告诉我?事情竟然到了这种地步!”

宦者说:“我不敢说,所以才能保全性命。如果我早说,就已经被杀死了,怎么会活到现在?”

阎乐走到二世面前列举他的罪状说:“您骄横放肆,残暴无道,天下人都背叛了您,请您自作打算。”

二世说:“可以让我见丞相吗?”

阎乐说:“不可以。”

二世说:“我希望得到一个郡,在那里做藩王。”阎乐不答应。

二世又说:“我希望做万户侯。”阎乐也不答应。

二世说:“我希望和妻子儿女做黔首,就像众公子一样。”

阎乐说:“我奉丞相的命令,替天下人诛杀您,您虽然说了很多,但是我不敢向丞相报告。”阎乐指挥他的士兵向前。二世自杀而死。

\begin{yuanwen}
阎乐归报赵高,赵高乃悉召诸大臣公子,告以诛二世之状。曰:“秦故王国,始皇君天下,故称帝。今六国复自立,秦地益小,乃以空名为帝,不可。宜为王如故,便。”

立二世之兄子公子婴\footnote{《六国年表》作“二世兄子婴”,《李斯列传》作“乃召始皇弟,授之玺”,关于子婴的身世,史学界众说纷纭。二世为始皇少子,与赵高谋划诛杀诸公子,因此子婴不会是始皇的子孙,而可能出自宗室旁支,且年长于二世。}为秦王。以黔首葬二世杜南宜春苑中。

令子婴斋,当庙见,受王玺。斋五日,子婴与其子二人谋曰:“丞相高杀二世望夷宫,恐群臣诛之,乃详\footnote{yáng,通“佯”,假装。}以义立我。我闻赵高乃与楚约,灭秦宗室而王关中。今使我斋见庙,此欲因庙中杀我。我称病不行,丞相必自来,来则杀之。”

高使人请子婴数辈,子婴不行,高果自往,曰:“宗庙重事,王奈何不行?”

子婴遂刺杀高于斋宫,三族高家以徇咸阳。子婴为秦王四十六日,楚将沛公破秦军入武关,遂至霸上,使人约降子婴。

子婴即系颈以组,白马素车,奉天子玺符,降轵道旁。沛公遂入咸阳,封宫室府库,还军霸上。

居月馀,诸侯兵至,项籍为从长,杀子婴及秦诸公子宗族。遂屠咸阳,烧其宫室,虏其子女,收其珍宝货财,诸侯共分之。灭秦之后,各分其地为三,名曰雍王、塞王、翟王,号曰三秦。项羽为西楚霸王,主命分天下王诸侯,秦竟灭矣。

后五年,天下定于汉。
\end{yuanwen}

阎乐回来向赵高报告,赵高于是召集所有大臣和公子,告诉他们诛杀二世的情况。赵高说:“秦国原本只是一个王国,始皇统治天下,所以号称皇帝。现在六国重新自立为王,秦国土地日益缩小,却空有皇帝名号,这是不可以的。应该像过去那样称王,这样才合适。”

赵高立二世兄长的儿子公子婴为秦王。用黔首的礼仪将二世埋葬在杜县以南的宜春苑中。赵高命令子婴斋戒,到宗庙祭祀祖先,接受秦王的印玺。

斋戒五天后,子婴和他的两个儿子密谋说:“丞相赵高在望夷宫杀死了二世皇帝,担心群臣诛杀他,就假借道义立我为王。我听说赵高竟然和楚国约定,消灭秦国宗室之后在关中自立为王。现在他让我斋戒祭祖,这是想要趁机在宗庙里杀死我。我借口生病不去,丞相一定会亲自来到这里,来的时候我们就杀死他。”

赵高派人请了子婴几次,子婴都不去,赵高果然亲自来了,说:“宗庙祭祀是重要的事情,大王怎么能不去呢?”

子婴就在斋戒的宫室里将赵高刺杀,然后诛灭赵高的三族在咸阳示众。子婴做秦王四十六天,楚将沛公打败秦军进入武关,于是驻扎在霸上,派人与子婴约定投降。

子婴就用丝绳系住脖子,乘坐白马素车,捧着天子的印玺和符节,在轵道旁投降。沛公于是进入咸阳,封闭宫室府库,回到霸上驻扎。

过了一个多月,诸侯的军队到了,项籍是诸侯联军首领,把子婴和秦国众公子的宗族全部杀光。最终他屠戮咸阳城,将宫室烧掉,俘虏当地的年轻女子,劫掠宫中的珍宝财物,和诸侯瓜分了。秦国灭亡以后,其领土被一分为三,封给雍王、塞王、翟王,号称三秦。项羽是西楚霸王,主管分封天下立诸侯王,秦朝终于灭亡了。

五年以后,汉朝统一天下。

\begin{yuanwen}
太史公曰:秦之先伯翳,尝有勋于唐虞之际,受土赐姓。及殷夏之间微散。至周之衰,秦兴,邑于西垂\footnote{同“陲”,边境。}。自缪公以来,稍蚕食诸侯,竟成始皇。始皇自以为功过五帝,地广三王,而羞与之侔。善哉乎贾生\footnote{贾谊。}推言之也!
\end{yuanwen}

太史公说:秦国的祖先伯翳,曾经在唐尧和虞舜之时建立功勋,受封土地获赐姓氏。其宗族到夏朝和商朝时候逐渐衰微。到周朝衰败后,秦国兴起,在西部边境建立城邑。自缪公以来,秦国逐渐蚕食诸侯,最后成就始皇的伟业。始皇认为自己的功绩超过五帝,统治的国土超过三王,以和三王五帝相提并论为羞耻。贾生的论述太好了!

\begin{yuanwen}
曰:秦并兼诸侯山东三十馀郡,缮津关,据险塞,修甲兵而守之。然陈涉以戍卒散乱之众数百,奋臂大呼,不用弓戟之兵,鉏櫌\footnote{同“耰”,用于碎土的农具。}白梃,望屋而食\footnote{指行军不带粮草,就地取食。},横行天下。秦人阻险不守,关梁不阖,长戟不刺,彊弩不射。楚师深入,战于鸿门,曾无籓篱之艰。于是山东大扰,诸侯并起,豪俊相立。秦使章邯将而东征,章邯因以三军之众要市于外,以谋其上。群臣之不信,可见于此矣。子婴立,遂不寤\footnote{同“悟”。}。藉使子婴有庸主之材,仅得中佐,山东虽乱,秦之地可全而有,宗庙之祀未当绝也。
\end{yuanwen}

他说:秦国兼吞诸侯在山东地区的三十多个郡,修缮渡口和关隘,占据险峻的要塞,训练士兵来防守国土。然而陈涉凭借几百个散乱的戍卒,举起双臂大喊一声,不用弓箭矛戟等兵器,只用农具和木棍,就地取食,横行天下。秦人有险要地形却不能固守,有关隘桥梁却不能封锁,有长柄矛戟却不能刺杀,有强弓劲弩却不能射击。楚国的军队深入腹地,在鸿门交战,竟然没有遇到像越过篱笆一样的困难。当时山东大乱,诸侯并起,豪杰俊才相继称王。秦国派遣章邯率军东征,章邯在外利用手中的重兵进行要挟,向朝廷谋取私利。群臣不讲信用,可以从这件事看清楚了。子婴被立为王,最终也没有醒悟。如果子婴具备普通的才能,只要得到中等才能的辅臣,即使山东叛乱,秦国的故地还是可以保全的,宗庙的祭祀也不会断绝。

\begin{yuanwen}
秦地被山带河以为固,四塞之国也。自缪公以来,至于秦王,二十馀君,常为诸侯雄。岂世世贤哉?其势居然也。且天下尝同心并力而攻秦矣。当此之世,贤智并列,良将行其师,贤相通其谋,然困于阻险而不能进,秦乃延入战而为之开关,百万之徒逃北而遂坏。岂勇力智慧不足哉?形不利,势不便也。秦小邑并大城,守险塞而军,高垒毋战,闭关据阨,荷戟而守之。诸侯起于匹夫,以利合,非有素王\footnote{有帝王之德而无帝王之位的人。}之行也。其交未亲,其下未附,名为亡秦,其实利之也。彼见秦阻之难犯也,必退师。安土息民,以待其敝,收弱扶罢\footnote{通“疲”。},以令大国之君,不患不得意于海内。贵为天子,富有天下,而身为禽者,其救败非也。
\end{yuanwen}

秦地背靠群山,有黄河环绕,以此固守,是四面都有屏障的国家。自缪公以来,一直到秦王,二十多个君主,曾经称雄于诸侯。难道秦国世代都有贤君吗?是其地理形势所决定的。何况天下诸侯曾经同心协力攻打秦国了。在这个时候,贤人和智者同时涌现,优秀的将领统率士兵,贤明的宰相进献谋略,然而因为险峻的地形而不能前进,秦国却可以对各国军队敞开关塞使其深入腹地,结果上百万的士兵败退而覆灭。难道是因为六国的武力和智慧不够吗?这是因为地形不利,形势不便。秦国将小邑合并为大城,在险要的地方驻军防守,高筑营垒不去交战,封闭关口拒守险隘,士兵手持长矛戍守。诸侯以平民的身份起兵为了利益而结盟,并不具备帝王的德行。他们之间的交情并不深厚,下属也并非诚心归服,名义上是为了消灭秦国,实际是为了从中获利。当他们看见秦国地势险阻,很难进犯,一定会撤军。如果秦国休养生息,等待诸侯自行衰败,再收养和扶助其贫弱的子民,来向大国之君发号施令,不愁不在四海之内施展抱负。贵为天子,富有天下,自己却被擒获,是因为他挽救败亡的策略错误。

\begin{yuanwen}
秦王足己\footnote{自以为是。}不问,遂过而不变。二世受之,因而不改,暴虐以重祸。子婴孤立无亲,危弱无辅。三主惑而终身不悟,亡,不亦宜乎?当此时也,世非无深虑知化之士也,然所以不敢尽忠拂过者,秦俗多忌讳之禁,忠言未卒于口而身为戮没矣。故使天下之士,倾耳而听,重足而立,拑口而不言。是以三主失道,忠臣不敢谏,智士不敢谋,天下已乱,奸不上闻,岂不哀哉!先王知雍\footnote{通“壅”。}蔽之伤国也,故置公卿、大夫、士,以饰法设刑,而天下治。其彊也,禁暴诛乱而天下服。其弱也,五伯征而诸侯从。其削也,内守外附而社稷存。故秦之盛也,繁法严刑而天下振;及其衰也,百姓怨望而海内畔矣。故周五序\footnote{公、侯、伯、子、男五爵,代指分封诸侯。}得其道,而千馀岁不绝。秦本末并失,故不长久。由此观之,安危之统相去远矣。野谚曰“前事之不忘,后事之师也”。是以君子为国,观之上古,验之当世,参以人事,察盛衰之理,审权势之宜,去就有序,变化有时,故旷日长久而社稷安矣。
\end{yuanwen}

秦王自以为是,不向大臣咨询问题,于是犯下错误也不知道改正。二世继承了这一错误,沿袭而不改悔,凶狠残暴的做法加重了祸患。子婴孤立无援,危殆无助。以上三位君主迷惑而终身没有醒悟,最终亡国,不也是正常的吗?在这个时候,世上并不是没有深谋远虑和明白事理的人,然而人们都不敢竭尽忠心纠正错误的原因,正是由于秦国的风俗中有很多禁忌,好话还没有完全说出口而自己就已经被杀害了。因此全天下的人,都侧耳倾听,叠足站立,闭口不言。所以三位君主违背正道,忠诚的大臣不敢进谏,智慧的谋士不敢献策,天下已经混乱,坏事不向朝廷禀告,难道不可悲吗?古代圣王知道受到蒙蔽会损害国家,所以设置了公卿、大夫、士,来整饬法令,设置刑罚,因而天下太平。国势强盛的时候,能够禁止暴行,讨伐叛乱,使天下人臣服。国势衰弱的时候,五霸四处征讨,使诸侯顺服。国势更加衰微的时候,对内加强守备,对外依附强国,使国家得以残存。所以秦国强盛的时候,用严刑酷法使天下震慑;等到衰弱的时候,就导致百姓怨恨和海内反叛了。所以周朝分封诸侯推行正道,而延续一千多年也没有断绝。秦国把根本和末节都失去了,因此不能长久。由此看来,国家安危的统绪已经背离了。民间有句谚语说“前事不忘,后事之师”。所以君子治理国家,会观察远古的得失,考察当代的所为,参考世间的人情,明白兴衰的道理,考虑变化的形势,懂得取舍,适时改革,所以国家就能够长治久安了。

\begin{yuanwen}
秦孝公据殽\footnote{yáo}函之固,拥雍州之地,君臣固守而窥周室,有席卷天下,包举宇内,囊括四海之意,并吞八荒之心。当是时,商君佐之,内立法度,务耕织,修守战之备,外连衡\footnote{也作“连横”,与“合纵”都是战国时期的主要外交战略。在地理形势上,秦国位于最西方,六国在东方分居南北,东西为横,南北为纵,因此六国联合抗秦称合纵,秦国分化合纵而各个击破六国称连横,并称“纵横”。}而斗诸侯,于是秦人拱手而取西河之外。
\end{yuanwen}

秦孝公占据殽山和函谷关,拥有雍州地区,君臣固守国土而窥探周王室,有席卷天下、征服世界、囊括四海的意图,以及吞并八方的心愿。在那个时候,商君辅佐他,对内设立法度,重视农耕和纺织,以及整修战备,对外连横而与诸侯争斗,这样秦人轻而易举地获得了西河以外的土地。

\begin{yuanwen}
孝公既没,惠王、武王蒙故业,因遗册\footnote{同“策”。},南兼汉中,西举巴、蜀,东割膏腴之地,收要害之郡。诸侯恐惧,会盟而谋弱秦,不爱珍器重宝肥美之地,以致天下之士,合从缔交,相与为一。当是时,齐有孟尝,赵有平原,楚有春申,魏有信陵。此四君者,皆明知而忠信,宽厚而爱人,尊贤重士,约从离衡,并韩、魏、燕、楚、齐、赵、宋、卫、中山之众。于是六国之士有宁越、徐尚、苏秦、杜赫之属为之谋,齐明、周最、陈轸、昭滑、楼缓、翟景、苏厉、乐毅之徒通其意,吴起、孙膑、带佗、兒良、王廖、田忌、廉颇、赵奢之朋制其兵。常以十倍之地,百万之众,叩关而攻秦。秦人开关延敌,九国之师逡\footnote{qūn}巡遁逃而不敢进。秦无亡矢遗镞之费,而天下诸侯已困矣。于是从散约解,争割地而奉秦。秦有馀力而制其敝,追亡逐北,伏尸百万,流血漂卤\footnote{通“橹”,大盾。}。因利乘便,宰割天下,分裂河山,彊国请服,弱国入朝。延及孝文王、庄襄王,享国日浅,国家无事。
\end{yuanwen}

孝公去世以后,惠王、武王继承原有的基业,沿袭遗留的策略,向南兼并汉中,向西攻打巴、蜀,向东割占肥沃的土地,夺取险要的郡县。诸侯感到害怕,结盟而商量削弱秦国的方法,不吝惜奇珍异宝和肥美的土地,来招揽天下的士人,缔结合纵联盟,彼此结合在一起。在这个时候,齐国有孟尝君,赵国有平原君,楚国有春申君,魏国有信陵君。这四位公子,都是英明智慧而忠诚守信的人,为人宽厚而爱护他人,尊敬贤士而重视人才,相约合纵来破坏连横,集合了韩、魏、燕、楚、齐、赵、宋、卫、中山的兵众。当时六国有宁越、徐尚、苏秦、杜赫这样的人出谋划策,有齐明、周最、陈轸、昭滑、楼缓、翟景、苏厉、乐毅这样的人传递意见,有吴起、孙膑、带佗、兒良、王廖、田忌、廉颇、赵奢这样的人统领军队。诸侯经常凭借十倍于秦国的土地,上百万人的兵众,敲击函谷关而进攻秦国。秦人打开关口迎敌,九国的军队徘徊逃遁而不敢前进。秦国没有消耗一兵一卒,就已经让天下诸侯处于困境之中。于是合纵盟约宣告瓦解,诸侯争相割让土地进献给秦国。秦国有足够的力量来利用各国的短处,追赶逃亡的敌人,杀敌百万,流出的鲜血足以使盾牌漂浮起来。秦国借助有利条件,宰割天下,划分山河,强大的国家请求归顺,弱小的国家入朝进贡。延续到孝文王、庄襄王,这两位秦王在位时间短,国家太平无事。

\begin{yuanwen}
及至秦王,续六世之馀烈,振长策而御宇内,吞二周而亡诸侯,履至尊而制六合,执棰拊以鞭笞天下,威振四海。南取百越之地,以为桂林、象郡,百越之君俯首系颈,委命下吏。乃使蒙恬北筑长城而守籓篱,卻匈奴七百馀里,胡人不敢南下而牧马,士不敢弯弓而报怨。于是废先王之道,焚百家之言,以愚黔首。堕名城,杀豪俊,收天下之兵聚之咸阳,销锋铸鐻,以为金人十二,以弱黔首之民。然后斩华为城,因河为津,据亿丈之城,临不测之谿以为固。良将劲弩守要害之处,信臣精卒陈利兵而谁何\footnote{盘查。},天下以定。秦王之心,自以为关中之固,金城千里,子孙帝王万世之业也。
\end{yuanwen}

到秦王在位时,继承六代先王留下的伟业,举起长鞭而统治全国,吞并东西二周而消灭诸侯,登上至尊的帝位而控制天地四方,手持鞭杖来笞打天下,声威震动四海。秦王向南攻取百越地区,设置桂林、象郡,百越的君主都低下头,用绳索系住脖子,把性命交给下级官吏。秦王又派蒙恬到北方修筑长城,以此为防御的屏障,使匈奴退却七百多里,胡人不敢南下牧马,武士不敢挽弓复仇。这时秦王废弃古代圣王的法则,焚烧诸子百家的典籍,以此愚弄百姓。他摧毁名城,杀死豪杰,收聚天下的兵器集中到咸阳,销毁后铸成编钟,以及十二个铜人,以此削弱黔首的反抗力量。然后他劈开华山为城垣,利用黄河为渡口,占据高达亿丈的城池,俯视深不可测的山谷来固守。优秀的将领以强劲的弓弩把守在要害的地方,忠诚的大臣和精锐的士兵手持利刃来盘查过往行人,天下因此安定。秦王的心里,以为关中的坚固,就像千里金城,子孙可以世代继承帝王的功业。

\begin{yuanwen}
秦王既没,馀威振于殊俗。陈涉,瓮牖绳枢\footnote{破瓮为窗户,草绳为门轴,形容家境贫穷。}之子,甿隶之人,而迁徙之徒,才能不及中人,非有仲尼、墨翟之贤,陶硃、猗\footnote{yī}顿之富,蹑足行伍之间,而倔起什伯\footnote{军中以十人为什,百人为伯。}之中,率罢散之卒,将数百之众,而转攻秦。斩木为兵,揭竿为旗,天下云集响应,赢粮而景\footnote{同“影”。}从,山东豪俊遂并起而亡秦族矣。
\end{yuanwen}

秦王死后,余威震慑远方。陈涉,一个穷苦人家的孩子,从事佣耕杂役的人,又是被迁徙的刑徒,才能比不上一个普通人,没有仲尼、墨翟的贤明,也没有陶朱、猗顿的富有,跻身于行伍之间,在士卒之中崛起,率领疲惫散乱的士卒,带着几百个人,转过身攻打秦国。他们砍伐树木制作兵器,高举竹竿作为旗帜。天下人聚集响应,携带着粮食,如影相随,山东豪杰同时起兵而灭掉了秦国宗族。

\begin{yuanwen}
且夫天下非小弱也,雍州之地,殽函之固自若也。陈涉之位,非尊于齐、楚、燕、赵、韩、魏、宋、卫、中山之君;鉏櫌棘矜,非錟\footnote{同“铦”,锋利。}于句戟长铩也;適戍之众,非抗于九国之师;深谋远虑,行军用兵之道,非及乡时之士也。然而成败异变,功业相反也。试使山东之国与陈涉度长絜大,比权量力,则不可同年而语矣。然秦以区区之地,千乘之权,招八州\footnote{天下九州,秦属雍州。}而朝同列,百有馀年矣。然后以六合为家,殽函为宫,一夫作难而七庙堕\footnote{huī},身死人手,为天下笑者,何也?仁义不施而攻守之势异也。
\end{yuanwen}

况且当时秦朝的天下并不弱小,雍州的土地,殽山和函谷关的险固和以前一样。陈涉的地位,并不比齐、楚、燕、赵、韩、魏、宋、卫、中山等国的君主尊贵;粗糙的农具,并不比钩戟长矛锋利;被贬谪到边疆戍守的士卒,并不能与九国的军队抗衡;深谋远虑,和行军用兵的方法,也比不上过去的谋士。然而成败的情况却发生了巨大的变化,所建立的功业也截然相反。使者以山东六国的势力与陈涉比较优劣,无论是权势还是力量,都不能相提并论了。然而秦国凭借一块很小的国土,只有一千辆兵车的力量,让八州诸侯前来朝见与自己地位相同的秦王,这种情况已经持续一百多年了。此后秦国将天地四方视为私产,把殽山和函谷关设为宫殿,一个人起来发难而使宗庙都被毁灭,自己也死在别人手中,被天下人耻笑,这是为什么呢?因为秦国不施行仁义而使进攻和防守的形势发生了变化。

\begin{yuanwen}
秦并海内,兼诸侯,南面称帝,以养四海,天下之士斐然乡\footnote{xiàng}风,若是者何也?曰:近古之无王者久矣。周室卑微,五霸既殁,令不行于天下,是以诸侯力政\footnote{以武力征伐。政,通“征”。},彊侵弱,众暴寡,兵革不休,士民罢敝。今秦南面而王天下,是上有天子也。既元元之民冀得安其性命,莫不虚心而仰上,当此之时,守威定功,安危之本在于此矣。
\end{yuanwen}

秦国吞并四海之内,统一天下诸侯,面向南方称帝,安抚国内百姓,天下的士人都欣然归向教化,产生这样的局面是什么原因呢?因为近古以来已经很长时间没有帝王统一天下了。周朝逐渐衰败,五霸已经去世,天子的政令不能通行于天下,所以诸侯通过武力对外征战,强国侵略弱国,人多欺负人少,战乱不断,百姓疲惫。现在秦王面向南方治理天下,这就在上面有了一个天子。芸芸众生都希望能安居乐业,没有人不虚心敬仰天子。在这个时候,保持威势,巩固基业,就是国家安危的关键了。

\begin{yuanwen}
秦王怀贪鄙之心,行自奋之智,不信功臣,不亲士民,废王道,立私权,禁文书而酷刑法,先诈力而后仁义,以暴虐为天下始。夫并兼者高诈力,安定者贵顺权,此言取与守不同术也。秦离战国而王天下,其道不易,其政不改,是其所以取之守之者(无)异也。孤独而有之,故其亡可立而待。借使秦王计上世之事,并殷周之迹,以制御其政,后虽有淫骄之主而未有倾危之患也。故三王之建天下,名号显美,功业长久。
\end{yuanwen}

秦王怀有贪婪卑鄙的心,运用独断专行的智慧,不信任有功之臣,不亲近士人百姓,废弃先王的法度,树立个人的权威,查禁图书而施行严刑酷法,把权术和武力放在前面,把仁德和道义放在后面,以凶狠残暴为统治天下的序幕。兼并天下的人崇尚权术和武力,安定天下的人主张顺应形势,这就是说取天下与守天下的方法是不同的。秦国结束战国时代而称王于天下,可是治道不变,政令不改,这就是取天下和守天下的方法没有不同。秦王只身一人占有天下,因此灭亡也为期不远了。假使秦王能够参考上古的事情,以及商朝和周朝的历史,来制定治国的政策,后世即使有骄奢淫逸的君主,也不会出现危亡的灾难。因此三王建立天下,名号显扬善美,功业长久流传。

\begin{yuanwen}
今秦二世立,天下莫不引领而观其政。夫寒者利裋\footnote{shù}褐而饥者甘糟穅,天下之嗷嗷,新主之资也。此言劳民之易为仁也。乡使二世有庸主之行,而任忠贤,臣主一心而忧海内之患,缟素\footnote{指丧服。}而正先帝之过,裂地分民以封功臣之后,建国立君以礼天下,虚囹圉而免刑戮,除去收帑\footnote{将罪犯的妻儿贬为官奴的刑罚。}污秽之罪,使各反其乡里,发仓廪,散财币,以振\footnote{同“赈”。}孤独穷困之士,轻赋少事,以佐百姓之急,约法省刑以持其后,使天下之人皆得自新,更节修行,各慎其身,塞万民之望,而以威德与天下,天下集矣。即四海之内,皆讙然各自安乐其处,唯恐有变,虽有狡猾之民,无离上之心,则不轨之臣无以饰其智,而暴乱之奸止矣。
\end{yuanwen}

现在秦二世继位,天下没有人不伸长脖子而观望他的政策。受冻的人渴望粗布短衣而挨饿的人渴望粗茶淡饭,天下人的呼声,就是新皇帝的资本。这就是说对劳苦民众的愿望很容易用仁政来满足。假如二世具有普通君主的德行,并且任用忠臣贤士,君臣同心而忧虑四海之内的愁苦,在守丧期间能够纠正先帝的错误,把土地分给百姓,并且分封功臣的后代,建立封国设置国君来礼敬天下人,空出监狱而免除酷刑,废除罚没官奴和污人身体的刑罚,让罪犯各自返回家乡,打开国库,散发钱财,赈济孤独穷困的人,减轻赋税和徭役,帮助百姓走出困境,减轻刑法来延续前面的政策,使天下的百姓都能重新做人,改善品行,各自谨慎地约束自己,满足万民的愿望,再用威势治理天下,天下就安定了。在四海之内,都欣然安居乐业,只怕发生变乱,即使有狡猾的人,也不会产生背叛皇帝的想法,那么不法的奸臣就不能掩饰他的阴谋,而暴乱的恶行也可以制止了。

\begin{yuanwen}
二世不行此术,而重之以无道,坏宗庙与民,更始作阿房宫,繁刑严诛,吏治刻深,赏罚不当,赋敛无度,天下多事,吏弗能纪,百姓困穷而主弗收恤。然后奸伪并起,而上下相遁,蒙罪者众,刑戮相望于道,而天下苦之。自君卿以下至于众庶,人怀自危之心,亲处穷苦之实,咸不安其位,故易动也。是以陈涉不用汤武之贤,不藉公侯之尊,奋臂于大泽而天下响应者,其民危也。故先王见始终之变,知存亡之机,是以牧民\footnote{统治人民。}之道,务在安之而已。天下虽有逆行之臣,必无响应之助矣。故曰“安民可与行义,而危民易与为非”,此之谓也。贵为天子,富有天下,身不免于戮杀者,正倾非也。是二世之过也。
\end{yuanwen}

二世没有采用这种方法,而是更加暴虐无道,毁坏宗庙,伤害臣民,又开始修建阿房宫,制定严刑酷法,官吏处理政务十分严苛,奖赏和惩罚不适当,赋税和徭役无限制,天下事情繁多,官吏无法处理,百姓贫困而君主不能安抚。此后奸邪诈伪之人同时作乱,全国上下相互隐瞒,获罪的人越来越多,受刑被杀的人堵住了道路,而天下人深受其苦。自君卿以下到平民百姓,每个人都感到自己处境危险,处在穷困苦难之中,都不能安于自己的位置,所以容易动摇。因此陈涉不需要具有商汤和周武王的才能,也不需要具有公侯的地位,只在大泽乡举起手臂,天下人就群起响应了,这是因为民众感到处境危险。所以古代圣王能够洞察事物的变化发展,从中得知国家存亡的关键,因此统治人民的方法,只在于使其安定罢了。天下即使有倒行逆施的臣子,也一定不会得到人民的响应了。所以说“生活安定的民众可以对其推行仁义,而处境危险的民众容易与其为非作歹”,说的就是这个道理。皇帝贵为天子,富有天下,自己却不能免于被杀戮,正是因为挽救危亡的方法不正确。这是二世的错误。

\begin{yuanwen}
襄公立,享国十二年。初为西畤\footnote{zhì}。葬西垂。生文公。

文公立,居西垂宫。五十年死,葬西垂。生静公。

静公不享国而死。生宪公。

宪公\footnote{《秦本纪》误作“宁公”。}享国十二年,居西新邑。死,葬衙。生武公、德公、出子。

出子享国六年,居西陵。庶长弗忌、威累、参父三人,率贼贼出子鄙衍,葬衙。武公立。

武公享国二十年。居平阳封宫。葬宣阳聚东南。三庶长伏其罪。德公立。

德公享国二年。居雍大郑宫。生宣公、成公、缪公。葬阳。初伏,以御蛊\footnote{这里指伤人的热毒邪气。}。

宣公享国十二年。居阳宫。葬阳。初志\footnote{记载。}闰月。

成公享国四年,居雍之宫。葬阳。齐伐山戎、孤竹。

缪公享国三十九年。天子致霸。葬雍。缪公学著人\footnote{门屏间的侍卫。}。生康公。
\end{yuanwen}

襄公即位,在位十二年。开始修建西畤。埋葬在西垂。生下文公。

文公即位,居住在西垂宫。在位五十年去世,埋葬在西垂。生下静公。

静公还没有即位就去世了。生下宪公。

宪公在位十二年,居住在西新邑。死后,埋葬在衙邑。生下武公、德公、出子。

出子在位六年,居住在西陵。庶长弗忌、威累、参父三个人,率领盗贼在鄙衍将出子杀害,埋葬在衙邑。武公继位。

武公在位二十年。居住在平阳封宫。埋葬在宣阳聚东南。三个庶长伏法被诛。德公继位。

德公在位二年。居住在雍邑大郑宫。生下宣公、成公、缪公。埋葬在阳邑。开始设置伏日,来禳除暑热瘟疫。

宣公在位十二年。居住在阳宫。埋葬在阳邑。开始记载闰月。

成公在位四年,居住在雍邑的宫中。埋葬在阳邑。齐国讨伐山戎、孤竹。

缪公在位三十九年。天子给予霸主的名号。埋葬在雍邑。缪公向门屏间的侍卫学习。生下康公。

\begin{yuanwen}
康公享国十二年。居雍高寝。葬竘\footnote{qǔ}社。生共公。

共公享国五年,居雍高寝。葬康公南。生桓公。

桓公享国二十七年。居雍太寝。葬义里丘北。生景公。

景公享国四十年。居雍高寝,葬丘里南。生毕公。

毕公\footnote{《秦本纪》作“哀公”。}享国三十六年。葬车里北。生夷公。

夷公不享国。死,葬左宫。生惠公。

惠公享国十年。葬车里。生悼公。

悼公享国十五年。葬僖公\footnote{即景公。}西。城雍。生剌龚公。

剌龚公\footnote{《秦本纪》作“厉公”。}享国三十四年。葬入里。生躁公、怀公。其十年,彗星见。

躁公享国十四年。居受寝。葬悼公南。其元年,彗星见。
\end{yuanwen}

康公在位十二年。居住在雍邑高寝。埋葬在竘社。生下共公。

共公在位五年。居住在雍邑高寝。埋葬在康公陵墓以南。生下桓公。

桓公在位二十七年。居住在雍邑太寝。埋葬在义里丘以北。生下景公。

景公在位四十年。居住在雍邑高寝。埋葬在丘里以南。生下毕公。

毕公在位三十六年。埋葬在车里以北。生下夷公。

夷公没有即位。去世后,埋葬在左宫。生下惠公。

惠公在位十年。埋葬在车里。生下悼公。

悼公在位十五年。埋葬在景公陵墓以西。在雍邑筑城。生下刺龚公。

刺龚公在位三十四年。埋葬在入里。生下躁公、怀公。刺龚公十年,彗星出现。

躁公在位十四年。居住在受寝。埋葬在悼公陵墓以南。躁公元年,彗星出现。

\begin{yuanwen}
怀公从晋来。享国四年。葬栎\footnote{lì}圉氏。生灵公\footnote{灵公为怀公之孙。}。诸臣围怀公,怀公自杀。

肃灵公\footnote{即灵公。},昭子\footnote{怀公太子,未继位。}子也。居泾阳。享国十年。葬悼公西。生简公。

简公从晋来。享国十五年。葬僖公西。生惠公。其七年。百姓初带剑。

惠公享国十三年。葬陵圉。生出公。

出公享国二年。出公自杀,葬雍。

献公享国二十三年。葬嚣圉。生孝公。

孝公享国二十四年。葬弟圉。生惠文王。其十三年,始都咸阳。

惠文王享国二十七年。葬公陵。生悼武王\footnote{《秦本纪》作“武王”。}。

悼武王享国四年,葬永陵。

昭襄王享国五十六年。葬茝\footnote{zhǐ}阳。生孝文王。
\end{yuanwen}

怀公从晋国返回。在位四年。埋葬在栎圉。生下灵公。群臣围攻怀公,怀公自杀。

肃灵公,是昭子的儿子。居住在泾阳。在位十年。埋葬在悼公陵墓以西。生下简公。

简公从晋国返回。在位十五年。埋葬在景公陵墓以西。生下惠公。简公七年,百姓开始带剑。

惠公在位十三年。埋葬在陵圉。生下出公。

出公在位二年。出公自杀,埋葬在雍邑。

献公在位二十三年。埋葬在嚣圉。生下孝公。

孝公在位二十四年。埋葬在弟圉。生下惠文王。孝公十三年,开始建都咸阳。

惠文王在位二十七年。埋葬在公陵。生下悼武王。

悼武王在位四年。埋葬在永陵。

昭襄王在位五十六年。埋葬在茝阳。生下孝文王。

\begin{yuanwen}
孝文王享国一年。葬寿陵。生庄襄王。

庄襄王享国三年。葬茝阳。生始皇帝。吕不韦相。

献公立七年,初行为市\footnote{集市。}。十年,为户籍相伍。

孝公立十六年。时桃李冬华\footnote{开花。}。

惠文王生十九年而立。立二年,初行钱。有新生婴兒曰“秦且王”。

悼武王生十九年而立。立三年,渭水赤三日。

昭襄王生十九年而立。立四年,初为田开阡陌。

孝文王生五十三年而立。

庄襄王生三十二年而立。立二年,取太原地。庄襄王元年,大赦,脩先王功臣,施德厚骨肉,布惠于民。东周与诸侯谋秦,秦使相国不韦诛之,尽入其国。秦不绝其祀,以阳人地赐周君,奉其祭祀。

始皇享国三十七年。葬郦邑。生二世皇帝。始皇生十三年而立。

二世皇帝享国三年。葬宜春。赵高为丞相安武侯。二世生十二年\footnote{本篇前文称“二世皇帝元年,年二十一”。}而立。

右秦襄公至二世,六百一十岁。
\end{yuanwen}

孝文王在位一年。埋葬在寿陵。生下庄襄王。

庄襄王在位三年。埋葬在茝阳。生下始皇帝。吕不韦为丞相。

献公即位七年,开始设置市场。十年,建立户籍,以五户为一伍。

孝公即位十六年,当时桃树和李树在冬季开花。

惠文王出生后十九年即位。即位二年,开始发行钱币。有一个刚出生的婴儿说“秦国将要称王”。

悼武王出生后十九年即位。即位三年,渭水呈现红色三天。

昭襄王出生后十九年即位。即位四年,开始废除原有田界。

孝文王出生后五十三年即位。

庄襄王出生后三十二年即位。即位二年,攻取太原地区。庄襄王元年,实行大赦,嘉奖先王时期的功臣,施行恩德厚待宗室至亲,对百姓推行仁惠。东周和诸侯图谋秦国,秦国派相国吕不韦将其灭掉,完全吞并其国土。秦国没有断绝周朝的祭祀,把阳人地赐给周君,让他供奉周朝的祭祀。

始皇在位三十七年。埋葬在郦邑。生下二世皇帝。始皇出生后十三年即位。

二世皇帝在位三年。埋葬在宜春。赵高担任丞相,封安武侯。二世出生后二十一年即位。

以上是秦襄公到二世的世系,一共六百一十年。

\begin{yuanwen}
孝明皇帝十七年十月十五日乙丑,班固曰\footnote{永平十七年(74年),汉明帝诏问班固《秦始皇本纪》中司马迁赞语是否得当,班固上表陈述秦亡之过,并且评论贾谊之语。《史记》在流传的过程中,由于各种原因,窜入了其他文字。班固是东汉史学家,其年代比司马迁晚一百多年。除本篇外,《孝武本纪》《外戚世家》《三王世家》《司马相如列传》《酷吏列传》《日者列传》《龟策列传》等篇也有不同程度的窜入文字。}:

周历\footnote{历数,命数。}已移,仁不代母\footnote{五行相生为母子关系,仁者建立的王朝,不应与前一王朝是母子关系。汉武帝时确立周为火德,秦为水德,汉为土德,均为相克关系。西汉末谶纬家又以周为木德,汉为火德,木生火,为母子。班固认为汉朝为仁者所建,不应为周所生。这反映的正是西汉末至东汉初的学术观点,而不同于司马迁时代的思想。}。秦直其位,吕政\footnote{指秦始皇,传言他是吕不韦之子。}残虐。然以诸侯十三,并兼天下,极情纵欲,养育宗亲。三十七年,兵无所不加,制作政令,施于后王。盖得圣人之威,河神授图,据狼、狐,蹈参、伐,佐政驱除,距之称始皇。

始皇既殁,胡亥极愚,郦山未毕,复作阿房,以遂前策。云:“凡所为贵有天下者,肆意极欲,大臣至欲罢先君所为?”诛斯、去疾,任用赵高。痛哉言乎!人头畜鸣\footnote{意思是说出话来像牲畜。}。不威不伐恶,不笃不虚亡,距之不得留,残虐以促期,虽居形便之国,犹不得存。
\end{yuanwen}

孝明皇帝十七年(74年)十月十五日乙丑,班固说:

周朝气数已尽,仁者不与前朝为母子。秦朝处在这个位置,吕政残酷暴虐。然而他能在十三岁的时候,以诸侯的身份,兼并天下,放纵情欲,抚养宗族。他在位三十七年,兵锋无处不在,制定政令,传给以后的帝王。他大概得到了圣人的威势,黄河之神授予他图书,身据狼星、狐星,脚踏参星、伐星,上天帮助他驱除暴乱,最终号称始皇。

始皇去世以后,胡亥非常蠢钝,郦山陵墓还没有完工,又要修建阿房宫,只为完成始皇的遗命,他说:“凡是尊贵而拥有天下的人,都可以随心所欲,大臣怎么能试图废除先帝所做的事情呢?”他诛杀李斯、冯去疾,任用赵高。他的话实在令人心痛啊!他长着人的头,却发出牲畜的叫声。没有威势就不能自夸,罪恶不深重就不会轻易灭亡,最终帝位无法保留,残酷暴虐只能让在位的时间更加短暂。秦国虽然占据了有利地形,却还是不能保全。

\begin{yuanwen}
子婴度次得嗣,冠玉冠,佩华绂\footnote{蔽膝,代指祭服。},车黄屋,从百司,谒七庙。小人乘非位,莫不怳(恍)忽失守,偷安日日,独能长念卻虑,父子作权,近取于户牖之间,竟诛猾臣,为君讨贼。高死之后,宾婚未得尽相劳,餐未及下咽,酒未及濡脣,楚兵已屠关中,真人\footnote{指刘邦。}翔霸上,素车婴组,奉其符玺,以归帝者。郑伯茅旌鸾刀\footnote{《公羊传·宣公十二年》记载,楚庄王伐郑,郑襄公袒露上身,左手持茅旌,右手持鸾刀,亲自出迎谢罪。茅旌,旌旗。鸾刀,祭祀时割牲肉的刀。},严王\footnote{楚庄王。东汉避汉明帝刘庄讳,改庄为严。}退舍。河决不可复壅,鱼烂不可复全。贾谊、司马迁曰:“向使婴有庸主之才,仅得中佐,山东虽乱,秦之地可全而有,宗庙之祀未当绝也。”秦之积衰,天下土崩瓦解,虽有周旦之材,无所复陈其巧,而以责一日之孤\footnote{指新君。},误哉!俗传秦始皇起罪恶,胡亥极,得其理矣。复责小子,云秦地可全,所谓不通时变者也。纪季以酅\footnote{《左传·庄公三年》记载,纪季献出酅邑投降齐国,后来纪国灭亡,酅邑却延续了纪国的祭祀,因此《春秋》对其加以褒扬。酅,xī。},春秋不名\footnote{不称名,以示尊敬。}。吾读《秦纪》,至于子婴车裂赵高,未尝不健其决,怜其志。婴死生之义备矣。
\end{yuanwen}

子婴按照次序得以继位,头戴玉冠,身穿华服,车上有黄色伞盖,百官跟随其后,前去祭拜宗庙。小人得到本不属于自己的位子,没有不恍然若失,偷安度日的,子婴却能够深谋远虑,与儿子商议权宜之计,就在门户之内,终于杀死狡猾的奸臣,为先君除掉了贼人。赵高死后,宾客和姻亲还没有悉数慰劳,饭还没有咽下,酒还没有沾唇,楚兵已经屠戮关中,真人已经降临霸上,子婴乘坐素车白马,用丝绳系住脖子,手捧符节和印玺,来向高皇帝归降。就像郑伯手持茅旌和鸾刀,楚庄王后撤七里一样。黄河决口就不能再堵住,鱼肉腐烂就不能再复原。贾谊、司马迁说:“假如子婴具有普通君主的才能,只得到中等才能的辅臣,山东即使发生叛乱,秦国的土地还是可以保全,宗庙的祭祀也不会断绝。”秦国常年积累的衰败局面,导致天下土崩瓦解,即使有周公旦一样的人才,也无法施展他的才智,却以此责备一个刚即位的新君,错误啊!民间传言罪恶起源于秦始皇,到胡亥时达到登峰,这是有道理的。贾谊、司马迁又责备新君,说秦国的土地本来是可以保全的,这就是人们所说的不懂时势的变化。纪季献出酅邑投降齐国,《春秋》不称其名以示褒扬。我读《秦纪》,读到子婴车裂赵高的时候,没有一次不称赞他的果决,怜悯他的志向。子婴就死生大义而言,已经做得很好了。

\begin{yuanwen}
六国陵替,二周沦亡。并一天下,号为始皇。阿房云构,金狄成行。南游勒石,东瞰浮梁。滈池见遗,沙丘告丧。二世矫制,赵高是与。诈因指鹿,灾生噬虎。子婴见推,恩报君父。下乏中佐,上乃庸主。欲振穨纲,云谁克补。
\end{yuanwen}

\part{卷七}

\chapter{项羽本纪第七}

司马迁以无限饱满的热情歌颂了项羽在灭秦过程中所建立的丰功伟绩,充分地肯定了他的历史作用;而对于项羽在楚汉战争中由于政治思想落后,政策方略错误,以及他个人性格上的种种缺点所导致的最终失败,则寄予了极大的惋惜与同情。有人仅取一端,或扬之为千古英雄,或抑之为桀、纣再世,亦可谓偏颇之极。司马迁的叙述全面,评价准确。作品所展示的重大历史场面的复杂性与深刻性,所描绘的人际关系与种种细节的深沉的历史感,都是前所未见的文献资料。

本篇是西楚霸王项羽的本纪,记述了他一生的功过,涉及年代与《高祖本纪》多有重合,而记事更加详细,文字更加生动。尽管项羽没有称帝,却威震天下,分封诸侯,在秦汉之间承上启下,因此司马迁将他列入本纪。

\begin{yuanwen}
项籍者,下相人也,字羽。初起时,年二十四。其季父\footnote{小叔父。季是兄弟排行中最小的。}项梁,梁父即楚将项燕,为秦将王翦\footnote{始皇前期的名将。}所戮者也。项氏世世为楚将,封于项,故姓项氏。
\end{yuanwen}

项籍是下相人,字羽。他刚起兵的时候只有二十四岁。他的叔父名叫项梁,项梁的父亲就是楚将项燕,被秦将王翦杀死的那个人。项氏世代担任楚将,被封在项邑,因此姓项氏。

\begin{yuanwen}
项籍少时,学书不成,去学剑,又不成。项梁怒之。籍曰:“书,足以记名姓而已。剑,一人敌,不足学。学万人敌。”

于是项梁乃教籍兵法,籍大喜,略知其意,又不肯竟学。项梁尝有栎阳逮,乃请蕲狱掾曹咎书抵栎阳狱掾司马欣\footnote{text},以故\footnote{因此。}事得已。项梁杀人,与籍避仇于吴中\footnote{text}。吴中贤士大夫皆出项梁下。每吴中有大繇役及丧\footnote{text},项梁常为主办,阴以兵法部勒宾客及子弟,以是知其能。秦始皇帝游会稽\footnote{text},渡浙江\footnote{text},梁与籍俱观。

籍曰:“彼可取而代也。”

梁掩其口,曰:“毋妄言,族矣!”梁以此奇籍。

籍长八尺馀\footnote{text},力能扛鼎\footnote{text},才气过人\footnote{text},虽吴中子弟皆已惮籍矣。
\end{yuanwen}

项籍年少的时候,学写字不成,去学击剑,又没学成。项梁对他很生气。项籍说:“写字可以记姓名就足够了。击剑只能与一个人搏斗,这些都不值得学,我想学与一万人对抗的。”

于是项梁就教项籍兵法,项籍十分高兴,大致了解兵法的主旨后,又不愿意学完。项梁曾经在栎阳被通缉,就请蕲县狱掾曹咎写信给栎阳狱掾司马欣,因此事情得以了结。项梁杀人,和项籍到吴中躲避仇家。吴中贤能的士大夫都比不上项梁。每当吴中有大规模的徭役和丧事的时候,经常由项梁主持办理,他暗中用兵法部署调度宾客和子弟,所以能够了解他们的才能。秦始皇帝巡游会稽,渡过浙江,项梁和项籍一同前去观看。

项籍说:“我可以取代他。”

项梁捂住他的嘴,说:“不许乱说,否则就要灭族了!”项梁因此觉得项籍非同寻常。

项籍身高八尺有余,很有力气,能够举起大鼎,他的才能勇气超过常人,就连吴中子弟也都畏惧他了。

\begin{yuanwen}
秦二世元年七月\footnote{text},陈胜(涉)等起大泽中\footnote{text}。其九月,会稽守通谓梁曰:“江西皆反\footnote{text},此亦天亡秦之时也。吾闻先即制人,后则为人所制。吾欲发兵,使公及桓楚将。”

是时桓楚亡在泽中。梁曰:“桓楚亡,人莫知其处,独籍知之耳。”

梁乃出,诫籍持剑居外待。梁复入,与守坐,曰:“请召籍,使受命召桓楚。”

守曰:“诺。”

梁召籍入。须臾,梁眴\footnote{shùn,使眼色。}籍曰:“可行矣!”

于是籍遂拔剑斩守头。项梁持守头,佩其印绶。门下大惊,扰乱,籍所击杀数十百人。一府中皆慴(慑)伏,莫敢起。梁乃召故所知豪吏,谕以所为起大事,遂举吴中兵。

使人收下县,得精兵八千人。梁部署吴中豪杰为校尉、候、司马。有一人不得用,自言于梁。梁曰:“前时某丧使公主某事,不能办,以此不任用公。”众乃皆伏。于是梁为会稽守,籍为裨将,徇下县。
\end{yuanwen}

秦二世元年(前209年)七月的时候,陈涉等人在大泽乡起事。当年九月,会稽郡守殷通对项梁说:“长江以西地区都造反了,这也是上天要灭亡秦朝的时机。我听说先行动就能控制别人,后行动则被别人所控制。我想要发兵,请您和桓楚带领。”

当时桓楚逃亡到泽中。项梁说:“桓楚逃亡在外,没有人知道他的下落,只有项籍知道罢了。”

项梁走出来,告诉项籍拿着剑在外面等候。项梁又走进去,与郡守坐在一起,说:“请召见项籍,让他接受命令召回桓楚。”

郡守说:“好。”

项梁召唤项籍进来。没过多久,项梁对项籍使眼色说:“可以行动了!”

于是项籍就拔出剑砍下郡守的头。项梁拿着郡守的头,身佩他的官印。门下的侍从大惊,陷入混乱,项籍杀死了几十上百人。全府的人都被项籍所震慑,没有人敢反抗。项梁于是召集原来他所认识的有权势的官吏,告知他们所要做的大事,于是他调发吴中的士兵。

他派人收编会稽郡下辖各县的士卒,得到精锐士兵八千人。项梁安排吴中的豪杰担任校尉、军候、司马。有一个人没有得到任用,亲自去对项梁说。项梁说:“前不久有一件丧事让您主办,您没有能力办好,所以没有任用您。”众人于是都很佩服项梁。这时项梁担任会稽郡守,项籍为裨将,攻取下辖各县。

\begin{yuanwen}
广陵人召\footnote{shào}平于是为陈王\footnote{陈胜。}徇广陵,未能下。闻陈王败走,秦兵又且至,乃渡江矫陈王命,拜梁为楚王上柱国。曰:“江东已定,急引兵西击秦。”

项梁乃以八千人渡江而西。闻陈婴已下东阳,使使欲与连和俱西。

陈婴者,故东阳令史,居县中,素信谨,称为长者。东阳少年杀其令,相聚数千人,欲置长,无適用,乃请陈婴。婴谢不能,遂彊立婴为长,县中从者得二万人。少年欲立婴便为王,异军苍头特起。

陈婴母谓婴曰:“自我为汝家妇,未尝闻汝先古之有贵者。今暴得大名,不祥。不如有所属,事成犹得封侯,事败易以亡,非世所指名也。”

婴乃不敢为王。谓其军吏曰:“项氏世世将家,有名于楚。今欲举大事,将非其人,不可。我倚名族,亡秦必矣。”

于是众从其言,以兵属项梁。项梁渡淮,黥布、蒲将军亦以兵属焉。凡六七万人,军下邳。
\end{yuanwen}

广陵人召平当时为陈王攻取广陵,没有能够攻下。他听说陈王战败逃跑,秦兵又快要赶到了,就渡过长江假传陈王的命令,任命项梁为楚王的上柱国。召平说:“江东已经平定,赶快带兵西进攻打秦军。”

项梁就派八千人渡过长江向西进发。他听说陈婴已经攻下东阳,就派使者想要与陈婴联合西进。

陈婴这个人,原来是东阳令史,居住在县里,向来诚实谨慎,被称为忠厚之人。东阳少年杀死了他们的县令,相互聚集起几千人,想要选一个首领,没有找到合适的人,就请陈婴来做首领。陈婴推辞却又无法拒绝,就被迫成为首领,县里追随他的人有二万人。少年想要推举陈婴称王,独树一帜以青巾裹头。

陈婴的母亲对陈婴说:“自从我做了你家的媳妇,从来没听说过你的祖先有谁地位显赫。现在你突然得到显赫的名号,不是好兆头。不如归附别人,事成还可以封侯,失败也容易逃脱,不会被世人指名道姓。”

陈婴于是不敢称王。他对军吏说:“项氏世代为将领,在楚地非常有声望。现在想要做大事,没有称职的将领,是不行的。我们依靠名门大族,一定能让秦朝灭亡了。”

于是众人都听从他的话,率领军队归附项梁。项梁渡过淮水,黥布、蒲将军也率领军队归附。项梁的军队一共有六七万人,驻扎在下邳。

\begin{yuanwen}
当是时,秦嘉已立景驹为楚王,军彭城东,欲距项梁。项梁谓军吏曰:“陈王先首事,战不利,未闻所在。今秦嘉倍陈王而立景驹,逆无道。”

乃进兵击秦嘉。秦嘉军败走,追之至胡陵。嘉还战一日,嘉死,军降。景驹走死梁地。项梁已并秦嘉军,军胡陵,将引军而西。章邯军至栗,项梁使别将硃鸡石、馀樊君与战。馀樊君死。硃鸡石军败,亡走胡陵。项梁乃引兵入薛,诛鸡石。项梁前使项羽别攻襄城,襄城坚守不下。已拔,皆阬\footnote{同“坑”,坑杀。}之。还报项梁。项梁闻陈王定死,召诸别将会薛计事。此时沛公亦起沛,往焉。
\end{yuanwen}

在这个时候,秦嘉已经立景驹为楚王,驻扎在彭城以东,想要与项梁相对抗。项梁对军吏说:“陈王首先起事,作战不利,下落不明。现在秦嘉背叛陈王而立景驹为王,这是大逆不道的举动。”

项梁于是带兵攻打秦嘉。秦嘉的军队战败逃跑,项梁追击到胡陵。秦嘉回军交战一天,秦嘉战死,士兵投降。景驹逃到梁地死去。项梁已经兼并了秦嘉的军队,驻扎在胡陵,将要率领军队西进。章邯的军队到达栗县,项梁派别将朱鸡石、馀樊君与他交战。馀樊君战死。朱鸡石的军队战败,逃到胡陵。项梁于是带兵进入薛县,处死朱鸡石。项梁此前派项羽率领另一支军队攻打襄城,襄城坚守不能攻下。项羽攻下襄城,将城中军民全部坑杀,返回报告项梁。项梁听说陈王确实死了,召集各路将领到薛县议事。这时沛公也在沛县起兵,前往薛县。

\begin{yuanwen}
居鄛\footnote{cháo}人范增,年七十,素居家,好奇计,往说项梁曰:“陈胜败固当。夫秦灭六国,楚最无罪。自怀王入秦不反,楚人怜之至今,故楚南公曰‘楚虽三户,亡秦必楚’也。今陈胜首事,不立楚后而自立,其势不长。今君起江东,楚蜂午\footnote{蜂起,纷然并起。}之将皆争附君者,以君世世楚将,为能复立楚之后也。”

于是项梁然其言,乃求楚怀王孙心\footnote{楚怀王熊槐三十年(前299年)与秦昭襄王会盟被劫持,三年后死在秦国,当时其少子公子兰也已成年,而且距熊心被立为楚王有近百年之久,所以熊心与楚怀王不太可能只相隔两代,但一定是其嫡系后裔。}民间,为人牧羊,立以为楚怀王,从民所望也。陈婴为楚上柱国,封五县,与怀王都盱\footnote{xū}台\footnote{yí}。项梁自号为武信君。
\end{yuanwen}

居鄛人范增,当时已经七十岁了,一直住在家里,喜好奇谋,前去劝项梁说:“陈胜的败亡本来就是注定的。秦国灭掉六国,楚国是最无辜的。自从怀王去秦国没有返回,楚人至今还很同情他。因此楚南公说‘楚国即使只剩三户人家,灭掉秦国的也一定是楚国人’。现在陈胜首先起事,不立楚国后裔却自立为王,他的势头一定不会持久。现在您在江东起兵,楚地将领纷然并起,都争相归附您,正是因为项氏世代担任楚将,认为您能够再立楚国后裔为王。”

这时项梁认为范增的话很有道理,就在民间找到楚怀王的后代熊心,他正在给人放羊,项梁立他为楚怀王,顺从人民的愿望。陈婴担任楚国的上柱国,封地为五个县,和怀王在盱台建都。项梁自称为武信君。

\begin{yuanwen}
居数月,引兵攻亢父,与齐田荣、司马龙且军救东阿,大破秦军于东阿。田荣即引兵归,逐其王假。假亡走楚。假相田角亡走赵。角弟田间故齐将,居赵不敢归。田荣立田儋子市为齐王。项梁已破东阿下军,遂追秦军。数使使趣齐兵,欲与俱西。

田荣曰:“楚杀田假,赵杀田角、田间,乃发兵。”

项梁曰:“田假为与国\footnote{盟国。}之王,穷来从我,不忍杀之。”

赵亦不杀田角、田间以市于齐。齐遂不肯发兵助楚。项梁使沛公及项羽别攻城阳,屠之。西破秦军濮阳东,秦兵收入濮阳。沛公、项羽乃攻定陶。定陶未下,去,西略地至雝\footnote{yōng}丘,大破秦军,斩李由。还攻外黄,外黄未下。
\end{yuanwen}

过了几个月,项梁领兵攻打亢父,与齐将田荣、司马龙且的军队援救东阿,在东阿大败秦军。田荣当时领兵返回,驱逐齐王田假。田假逃到楚国。田假的相国田角逃到赵国。田角的弟弟田间原本是齐将,留在赵国不敢回去。田荣立田儋的儿子田巿为齐王。项梁已经打败东阿地区的秦军,于是追击秦军。他多次派使者到齐国,想要让齐军与自己一同西进。

田荣说:“如果楚国杀死田假,赵国杀死田角、田间,我就出兵。”

项梁说:“田假是盟国的王,困窘时来投靠我国,我不忍心杀死他。”

赵国也不杀田角、田间,以此与齐国做交易。齐国始终不肯发兵帮助楚国。项梁派沛公和项羽率领另外一支军队攻打城阳,屠戮那里。楚军向西在濮阳以东打败秦军,秦军收兵退到濮阳城内。沛公、项羽于是攻打定陶。定陶还没有攻下,楚军就离开了,向西攻取土地,到达雝丘,大败秦军,斩杀李由。回军攻打外黄,没有攻下。

\begin{yuanwen}
项梁起东阿,西,至定陶,再破秦军,项羽等又斩李由,益轻秦,有骄色。宋义乃谏项梁曰:“战胜而将骄卒惰者败。今卒少惰矣,秦兵日益,臣为君畏之。”

项梁弗听。乃使宋义使于齐。道遇齐使者高陵君显,曰:“公将见武信君乎?”

曰:“然。”

曰:“臣论武信君军必败。公徐行即免死,疾行则及祸。”

秦果悉起兵益章邯,击楚军,大破之定陶,项梁死。沛公、项羽去外黄攻陈留,陈留坚守不能下。沛公、项羽相与谋曰:“今项梁军破,士卒恐。”

乃与吕臣军俱引兵而东。吕臣军彭城东,项羽军彭城西,沛公军砀。
\end{yuanwen}

项梁在东阿发兵,向西进军,临近定陶,再次打败秦军,项羽等人又斩杀了李由,使项梁越来越轻敌,露出骄傲的神色。宋义于是劝谏项梁说:“打了胜仗而将领和士兵骄傲懈怠的军队一定会失败。现在士兵有些懈怠了,秦兵一天比一天多,我为您感到担心。”

项梁不听。他就派宋义出使齐国。宋义在路上遇到齐国使者高陵君田显,说:“您要去见武信君吗?”

高陵君说:“是的。”

宋义说:“我认为武信君的军队一定会失败。您慢些去就能避免一死,走得太快则大难临头。”

秦国果然发动全部兵力增援章邯,攻打楚军,在定陶大破楚军,项梁战死。沛公、项羽离开外黄攻打陈留,陈留坚守不能攻下。沛公、项羽相互商量说:“现在项梁的军队已经被打败,士兵恐慌。”

于是他们和吕臣的军队一起领兵向东进发。吕臣驻扎在彭城以东,项羽驻扎在彭城以西,沛公驻扎在砀县。

\begin{yuanwen}
章邯\footnote{hán}已破项梁军,则以为楚地兵不足忧,乃渡河击赵\footnote{text},大破之。当此时,赵歇为王,陈馀为将,张耳为相,皆走入钜鹿城\footnote{text}。章邯令王离、涉间围钜鹿\footnote{text},章邯军其南,筑甬道\footnote{两旁有遮蔽物的通道。}而输之粟。陈馀为将,将卒数万人而军钜鹿之北,此所谓河北之军也\footnote{text}。
\end{yuanwen}

章邯击败项梁的军队以后,就认为楚地的士兵不值得忧虑,于是渡过黄河去攻打赵国,大破赵军。在这个时候,赵歇是赵王,陈馀是将军,张耳是相国,都逃进钜鹿城。章邯命令王离、涉间围攻钜鹿,章邯驻扎在钜鹿以南,修筑甬道而运输粮食。陈馀担任将军率领几万人而驻扎在钜鹿以北,这就是人们所说的河北之军。

\begin{yuanwen}
楚兵已破于定陶,怀王恐,从盱台之彭城,并项羽、吕臣军自将之。以吕臣为司徒,以其父吕青为令尹。以沛公为砀郡长,封为武安侯,将砀郡兵。
\end{yuanwen}

楚军在定陶被打败以后,楚怀王很恐慌,从盱台来到彭城,合并项羽、吕臣的军队亲自率领。他任命吕臣为司徒,任命吕臣的父亲吕青为令尹。他又任命沛公为砀郡长,封为武安侯,率领砀郡的士兵。

\begin{yuanwen}
初,宋义所遇齐使者高陵君显在楚军,见楚王\footnote{text}曰:“宋义论武信君之军必败,居数日,军果败。兵未战而先见败征,此可谓知兵矣。”

王召宋义与计事而大说之,因置以为上将军\footnote{text},项羽为鲁公,为次将,范增为末将\footnote{text},救赵。诸别将皆属宋义\footnote{text},号为卿子冠军\footnote{text}。行至安阳\footnote{text},留四十六日不进。

项羽曰:“吾闻秦军围赵王钜鹿,疾引兵渡河,楚击其外,赵应其内,破秦军必矣。”

宋义曰:“不然。夫搏牛之虻不可以破虮\footnote{jǐ}虱\footnote{shī}。今秦攻赵,战胜则兵罢\footnote{text},我承其敝;不胜,则我引兵鼓行而西\footnote{text},必举秦矣。故不如先斗秦、赵。夫被坚执锐,义不如公;坐而运策,公不如义。”

因下令军中曰:“猛如虎,很\footnote{违逆,执拗。}如羊\footnote{text},贪如狼,彊不可使者,皆斩之。”乃遣其子宋襄相齐,身送之至无盐\footnote{text},饮酒高会\footnote{text}。

天寒大雨,士卒冻饥。项羽曰:“将戮力\footnote{合力,并力。}而攻秦\footnote{text},久留不行。今岁饥民贫,士卒食芋菽\footnote{shū},军无见粮,乃饮酒高会,不引兵渡河因赵食,与赵并力攻秦,乃曰‘承其敝’。夫以秦之彊,攻新造之赵\footnote{text},其势必举赵。赵举而秦强,何敝之承!且国兵新破,王坐不安席,埽\footnote{sào}境内而专属于将军,国家安危,在此一举。今不恤士卒而徇其私\footnote{text},非社稷之臣\footnote{text}。”

项羽晨朝上将军宋义,即其帐中斩宋义头,出令军中曰:“宋义与齐谋反楚,楚王阴令羽诛之。”

当是时,诸将皆慑服,莫敢枝梧\footnote{抗拒。}。皆曰:“首立楚者,将军家也。今将军诛乱\footnote{text}。”

乃相与共立羽为假上将军\footnote{text}。使人追宋义子,及之齐,杀之。使桓楚报命于怀王。怀王因使项羽为上将军\footnote{text},当阳君、蒲将军皆属项羽\footnote{text}。
\end{yuanwen}

当初,宋义所遇到的齐国使者高陵君田显在楚国军中,见到楚怀王说:“宋义认为武信君的军队一定会失败,过了几天,他的军队果然失败了。军队还没有作战却预先见识失败的征兆,这可以说是精通兵法了。”

楚怀王召见宋义与他议事,因而非常赏识他,就任命他为上将军,项羽为鲁公,担任次将,范增担任末将,前去援救赵国。各路将领都隶属于宋义,宋义号称卿子冠军。大军行进到安阳,停留四十六天没有前进。

项羽说:“我听说秦军把赵王围困在钜鹿,应该赶快渡河,楚军在外面进攻,赵军在里面响应,就一定能打败秦军了。”

宋义说:“不是这样。可以叮咬牛的虻却不能对付虮虱。现在秦军攻打赵军,取胜也会全军疲惫,我们可以利用对方的弱点;秦军不能取胜,我们就领兵击鼓西进,就一定能够灭掉秦国了。因此不如先让秦、赵两军相斗。身披甲胄,手执利器,冲锋杀敌,我不如您;静观形势,谋划策略,您不如我。”

宋义于是向军中下令说:“像虎一样凶猛,像羊一样执拗,像狼一样贪婪,倔强不听从指挥的人,一律斩首!”

宋义于是派他的儿子宋襄去辅助齐国,亲自送他到无盐,设酒宴大会宾客。

当时天气寒冷下大雨,士兵受冻挨饿。项羽说:“将要合力攻打秦军,却长时间停留在这里不前进。现在遇到饥荒,百姓贫困,士卒只能吃芋头和豆子,军中没有存粮,宋将军却设酒宴大会宾客,不领兵渡河在赵国寻找食物,和赵军合力攻打秦军,却说‘利用对方的弱点’。凭借秦国的强大,进攻刚建立的赵国,一定会灭掉赵国。赵国被灭掉而秦国变强大,还有什么弱点可以利用呢!况且我国的军队刚被打败,怀王坐不安稳,将国内全部兵力都交给宋将军,国家的安危,就在这一战。现在他不体恤士兵,却谋取私利,不是国家的忠臣。”

项羽在清晨拜见上将军宋义,就在上将军的帐幕中砍下宋义的头,走出来向军中下令说:“宋义和齐国谋划反对楚国,楚怀王暗中命令我诛杀他。”

在这个时候,众将领都被震慑,没有人敢抗拒。众人都说:“创建楚国的,是将军一家。现在将军诛杀了叛乱之人。”

将士共同推举项羽为代理上将军。项羽派人追赶宋义的儿子,追到齐国,将他杀死。项羽派桓楚向怀王报告。怀王就任命项羽为上将军,当阳君、蒲将军都隶属于项羽。

\begin{yuanwen}
项羽已杀卿子冠军,威震楚国,名闻诸侯。乃遣当阳君、蒲将军将卒二万渡河,救钜鹿。战少利,陈馀复请兵。项羽乃悉引兵渡河,皆沉船,破釜甑\footnote{煮炊食物的器具。甑,zèng 。},烧庐舍,持三日粮,以示士卒必死,无一还心。于是至则围王离,与秦军遇,九战,绝其甬道,大破之\footnote{text},杀苏角,虏王离。涉间不降楚,自烧杀。

当是时,楚兵冠诸侯。诸侯军救钜鹿下者十馀壁\footnote{text},莫敢纵兵。及楚击秦,诸将皆从壁上观。楚战士无不一以当十,楚兵呼声动天,诸侯军无不人人惴恐\footnote{text}。于是已破秦军,项羽召见诸侯将,入辕门\footnote{将帅军营的大门。},无不膝行而前,莫敢仰视。项羽由是始为诸侯上将军,诸侯皆属焉。
\end{yuanwen}

项羽杀死卿子冠军以后,声威震慑楚国,名望传遍诸侯。他就派当阳君、蒲将军率领士兵二万人渡过黄河,援救钜鹿。取得一些胜利,陈馀又向项羽请求援兵。项羽就率领全军渡河,他们凿沉船只,砸破炊具,烧毁营舍,携带三天的口粮,以此表示要拼死决战,没有活着回来的打算。于是军队刚到就围住了王离,与秦军相遇,九次交战,截断对方的甬道,大破敌军,杀死苏角,俘获王离。涉间不向楚军投降,自焚而死。

在这个时候,楚军勇冠诸侯。诸侯派兵援救钜鹿的有十几处营寨,没有人敢轻易出战。等到楚军攻打秦军的时候,各路将领都站在营垒上观望。楚军战士都以一敌十,楚兵的呐喊声震撼天际,诸侯军中没有人不胆战心惊。于是楚军打败秦军以后,项羽召见诸侯将领,进入辕门的时候,都跪在地上爬行,没有人敢仰视项羽。项羽从此开始成为诸侯联军的上将军,诸侯都听命于他。

陈仁锡:「白起杀降,后世兵家者流亦恶之。秦亡之亟,起与有力焉。籍方欲诛无道秦,乃坑降卒至二十余万,其斩刈之惨,复一秦耳,何以慰斯民之望哉!」杨慎:「将飞者翼伏,将奋者足跼,将噬者爪缩。惟鸿门之不争,故垓下莫能与之争。」

\begin{yuanwen}
章邯军棘原\footnote{text},项羽军漳南\footnote{text},相持未战。秦军数卻,二世使人让章邯。章邯恐,使长史欣请事。至咸阳,留司马门\footnote{皇宫的外门。}三日,赵高不见,有不信之心。长史欣恐,还走其军,不敢出故道。赵高果使人追之,不及。欣至军,报曰:“赵高用事于中,下无可为者。今战能胜,高必疾妒吾功;战不能胜,不免于死。原将军孰计之。”

陈馀亦遗章邯书曰:“白起为秦将,南征鄢、郢,北阬马服\footnote{指赵括。赵奢封马服君,其子赵括袭爵。},攻城略地,不可胜计,而竟赐死。蒙恬为秦将,北逐戎人,开榆中地数千里,竟斩阳周。何者?功多,秦不能尽封,因以法诛之。今将军为秦将三岁矣,所亡失以十万数,而诸侯并起滋益多。彼赵高素谀日久,今事急,亦恐二世诛之,故欲以法诛将军以塞责\footnote{敷衍,推卸责任。},使人更代将军以脱其祸。夫将军居外久,多内卻\footnote{内部的仇怨。},有功亦诛,无功亦诛。且天之亡秦,无愚智皆知之。今将军内不能直谏,外为亡国将,孤特独立而欲常存,岂不哀哉!将军何不还兵与诸侯为从,约共攻秦,分王其地,南面称孤;此孰与身伏鈇(斧)质,妻子为僇乎?”

章邯狐疑\footnote{犹豫。},阴使候始成使项羽,欲约。约未成,项羽使蒲将军日夜引兵度三户\footnote{text},军漳南\footnote{text},与秦战,再破之。项羽悉引兵击秦军汙水上\footnote{text},大破之。
\end{yuanwen}

章邯的军队驻扎在棘原,项羽的军队驻扎在漳水以南,两军对峙,还没有交战。秦军多次退却,二世派人责备章邯。章邯很害怕,派长史司马欣向朝廷请示。来到咸阳,司马欣停留在司马门三天,赵高不接见他,有不信任的意思。长史司马欣很害怕,逃回军中,不敢走原路。赵高果然派人去追赶他,没有追上。司马欣回到军中,向章邯报告说:“赵高在朝中当政,下面没有人敢擅自行动。现在作战能够取胜,赵高一定会嫉妒我们的功劳;作战不能够取胜,一定免不了一死。希望将军仔细考虑。”

陈馀也写信给章邯说:“白起担任秦将,向南征讨鄢、郢,向北坑杀马服君,攻城略地,战功不可计数,可是最后被赐一死。蒙恬担任秦将,向北驱逐戎狄,开辟榆中几千里的疆土,最后在阳周被斩杀。这是为什么呢?因为他们功劳太多,秦国不能封赏,就论罪诛杀他们。现在将军担任秦将三年了,所损失的士兵数以十万计,而诸侯同时起兵,并且越来越多。那个赵高平时借助谄媚当权很久了,现在形势危急,他也担心二世诛杀他,所以想要论罪诛杀将军来推卸责任,派人取代将军来摆脱灾祸。将军在外征战的时间太久,与朝中的人多有仇怨,有功会被诛杀,无功也会被诛杀。况且上天将要灭亡秦国,无论是愚钝的人还是聪明的人都知道这个道理。现在将军在内不能直言进谏,在外是个亡国之将,孤立无援却想要长久存活,难道不可悲吗!将军为什么不回军与诸侯联合,相约共同攻打秦国,分割秦国的土地称王,面向南方称孤道寡?这与自己俯身被诛,妻儿惨遭屠戮相比,哪个更好呢?”

章邯犹豫不决,暗中派军候始成出使项羽军中,想要约定反秦。盟约还没有缔结,项羽就派蒲将军日夜兼程领兵渡过三户津,驻扎在漳水以南,与秦军交战,再次打败秦军。项羽率领全军将士在汙水进攻秦军,大破秦军。

\begin{yuanwen}
章邯使人见项羽,欲约。项羽召军吏谋曰:“粮少,欲听其约。”

军吏皆曰:“善。”

项羽乃与期\footnote{约定日期。}洹水南殷虚上。已盟,章邯见项羽而流涕,为言赵高。项羽乃立章邯为雍王,置楚军中\footnote{text}。使长史欣为上将军,将秦军为前行。
\end{yuanwen}

章邯派人去见项羽,想要缔结盟约。项羽召集军吏商量说:“军中粮少,我想要批准与他们缔结盟约。”

军吏都说:“好。”

项羽就与章邯约定日期在洹水以南的殷虚见面。缔结盟约以后,章邯见到项羽而流下眼泪,对他说赵高的坏话。项羽就立章邯为雍王,将他安置在楚国的军中。项羽任命长史司马欣为上将军,率领秦军在前面行进。

\begin{yuanwen}
到新安\footnote{text}。诸侯吏卒异时故繇使屯戍过秦中\footnote{text},秦中吏卒遇之多无状\footnote{没有礼貌。},及秦军降诸侯,诸侯吏卒乘胜多奴虏使之,轻折辱秦吏卒\footnote{text}。秦吏卒多窃言曰:“章将军等诈吾属降诸侯,今能入关破秦\footnote{text},大善;即不能\footnote{text},诸侯虏吾属而东,秦必尽诛吾父母妻子。”

诸侯微闻其计\footnote{text},以告项羽。项羽乃召黥布、蒲将军计曰:“秦吏卒尚众,其心不服,至关中不听\footnote{text},事必危。不如击杀之,而独与章邯、长史欣、都尉翳入秦\footnote{text}。”

于是楚军夜击阬秦卒二十馀万人新安城南。
\end{yuanwen}

到达新安时,诸侯将士过去曾经服徭役或屯戍边疆而路过秦地,秦地官兵对他们大多很无礼,等到秦军投降诸侯,诸侯将士大多借着战胜的机会像对待奴隶和俘虏一样役使他们,随意羞辱秦军官兵。秦军官兵大多在私下议论说:“章将军等人欺骗我们投降诸侯,现在能够入关攻破秦国,当然很好;假如不能,诸侯俘虏我们到东方,秦国一定会将我们的父母妻儿全部处死。”

各路将领们隐约听说了这些议论,报告项羽。项羽于是召来黥布、蒲将军谋划说:“秦军官兵人数众多,他们心里不服,到达关中还是不听号令,情况就危险了,不如出击杀掉他们,而只与章邯、长史司马欣、都尉董翳进入秦地。”于是楚军在夜里在新安城南将秦军降卒二十多万人全部坑杀。

凌稚隆:「先入关者王之,怀王之约也。沛公先定关中,则当王关中。增乃劝羽杀之,目无怀王矣。他日羽之行弑,增未必不与谋也。」

\begin{yuanwen}
行略定秦地\footnote{text}。函谷关有兵守关\footnote{text},不得入。又闻沛公已破咸阳,项羽大怒,使当阳君等击关。项羽遂入,至于戏西\footnote{text}。沛公军霸上\footnote{text},未得与项羽相见。沛公左司马曹无伤使人言于项羽曰:“沛公欲王关中,使子婴为相\footnote{text},珍宝尽有之。”

项羽大怒,曰:“旦日飨\footnote{text}士卒,为击破沛公军!”

当是时,项羽兵四十万,在新丰鸿门\footnote{text},沛公兵十万,在霸上。范增说项羽曰:“沛公居山东时\footnote{text},贪于财货,好美姬。今入关,财物无所取,妇女无所幸,此其志不在小。吾令人望其气\footnote{观看云气预测吉凶的一种方术。},皆为龙虎,成五采,此天子气也。急击勿失。”
\end{yuanwen}

诸侯进军攻取秦地。函谷关有重兵把守,军队不能进去。项羽又听说沛公已经攻破咸阳,非常生气,派当阳君等人攻打函谷关。项羽于是进入关中,来到戏水以西。沛公驻扎在霸上,没能和项羽相见。沛公的左司马曹无伤派人对项羽说:“沛公想要在关中称王,任命子婴为相国,把全部奇珍异宝据为己有。”

项羽非常生气,说:“明天早晨犒劳士兵,为了打败沛公的军队!”

在这个时候,项羽的士兵有四十万人,驻扎在新丰鸿门,沛公的士兵只有十万人,驻扎在霸上。范增劝项羽说:“沛公在山东的时候,贪图财货,喜爱美女。现在进入关中,不收取财物,不亲近妇女,由此看来他的志向一定不小。我命人观望他头上的云气,都是龙虎之形,呈现五种颜色,这是天子的云气。赶快进攻,不要失去机会。”

\begin{yuanwen}
楚左尹项伯\footnote{text}者,项羽季父也,素善留侯\footnote{张良在汉朝建立以后的封号。}张良。张良是时从沛公,项伯乃夜驰之沛公军,私见张良,具告以事,欲呼张良与俱去。曰:“毋从俱死也。”

张良曰:“臣为韩王\footnote{韩成,最后被项羽所杀。}送沛公\footnote{text},沛公今事有急,亡去不义,不可不语。”

良乃入,具告沛公。沛公大惊,曰:“为之奈何?”

张良曰:“谁为大王为此计者?”

曰:“鲰生\footnote{肤浅之人。鲰,zōu。}说\footnote{text}我曰,‘距关,毋内诸侯\footnote{text},秦地可尽王也’。故听之。”

良曰:“料大王士卒足以当项王\footnote{此时项羽尚未称王。}乎?”

沛公默然,曰:“固不如也,且为之奈何?”

张良曰:“请往谓项伯,言沛公不敢背项王也。”

沛公曰:“君安与项伯有故?”

张良曰:“秦时与臣游,项伯杀人,臣活之。今事有急,故幸来告良。”

沛公曰:“孰与君少长?”

良曰:“长于臣。”

沛公曰:“君为我呼入,吾得兄事之。”

张良出,要项伯。项伯即入见沛公。沛公奉卮酒为寿,约为婚姻,曰:“吾入关,秋豪不敢有所近,籍吏民\footnote{text},封府库,而待将军。所以遣将守关者,备他盗之出入与非常也\footnote{text}。日夜望将军至,岂敢反乎!原伯\footnote{据《高祖功臣侯者年表》,项伯名缠,伯是字。古人在社交场合互称其字以示尊敬,直呼其名则为不敬。}具言臣之不敢倍德\footnote{text}也。”

项伯许诺。谓沛公曰:“旦日不可不蚤自来谢项王\footnote{text}。”

沛公曰:“诺。”

于是项伯复夜去,至军中,具以沛公言报项王,因言曰:“沛公不先破关中,公岂敢入乎?今人有大功而击之,不义也,不如因善遇之。”项王许诺。
\end{yuanwen}

楚国左尹项伯,是项羽的叔父,他和留侯张良一向关系很好。张良当时跟随沛公,项伯就在夜里骑马到沛公的军中,私下会见张良,把情况全部告知,想要叫张良和他一起离开。项伯说:“不要跟他们一起死。”

张良说:“我为韩王护送沛公,沛公现在有急难,我逃走是不仁义的,不能不告诉他。”

张良于是进去,把情况全部告知沛公。沛公大惊,说:“怎么办呢?”

张良说:“谁给大王出的这个主意?”

沛公说:“一个小人劝我说:‘拒守函谷关,不要放诸侯进来,就可以在完全占有秦地而称王。’因此我听信了他的话。”

张良说:“大王估计自己的士兵能够抵挡项王吗?”

沛公沉默片刻,说:“当然不能抵挡,那又能怎么办呢?”

张良说:“请让我去告诉项伯,就说沛公不敢背叛项王。”

沛公说:“您怎么会与项伯有交情呢?”

张良说:“秦朝的时候,项伯和我有交往,他杀了人,我救了他一命。现在情况危急,所以幸亏他来告诉我。”

沛公说:“项伯与您相比,谁的年纪大?”

张良说:“他比我年纪大。”

沛公说:“您替我把他叫进来,我要以兄长的礼节待他。”

张良出去,邀请项伯。项伯就进去见沛公。沛公捧着酒杯祝福项伯长寿,相约为儿女亲家,说:“我进入关中,财物分毫不敢接受,将官吏民众登记在册,封存府库,并且等待将军的到来。我派将领把守关隘的原因,是为了防备别的盗贼出入以及发生意外。我从早到晚盼望将军到来,怎么敢反叛呢!请项伯兄向将军详细说明我不敢背弃恩德。”

项伯答应了,对沛公说:“明天不能不早点过来亲自向项王道歉。”

沛公说:“好。”

于是项伯又连夜离去,回到军中,把沛公的话全部报知项王,随即对他说:“沛公不先攻破关中,您怎么能轻易进来呢?现在人家立下大功却要攻打他,是不道义的,不如趁机善待他。”项王答应了。

\begin{yuanwen}
沛公旦日从百馀骑来见项王,至鸿门,谢曰:“臣与将军戮力而攻秦,将军战河北,臣战河南,然不自意能先入关破秦\footnote{text},得复见将军于此。今者有小人之言,令将军与臣有隙\footnote{裂痕,引申为矛盾。}。”

项王曰:“此沛公左司马曹无伤言之,不然,籍何以至此。”

项王即日因留沛公与饮。项王、项伯东乡坐\footnote{xiàng,同“向”。}。亚父南乡坐\footnote{text}。亚父者,范增也。沛公北乡坐,张良西乡侍。

范增数目项王,举所佩玉玦\footnote{有缺口的玉环,表示决绝。}以示之者三\footnote{text},项王默然不应。范增起,出召项庄\footnote{text},谓曰:“君王为人不忍,若\footnote{代词,你。}入前为寿\footnote{text},寿毕,请以剑舞,因击沛公于坐,杀之。不者,若属皆且为所虏。”

庄则入为寿,寿毕,曰:“君王与沛公饮,军中无以为乐,请以剑舞。”

项王曰:“诺。”

项庄拔剑起舞,项伯亦拔剑起舞,常以身翼蔽沛公\footnote{text},庄不得击。
\end{yuanwen}

沛公在第二天带领一百多骑兵来见项王,到达鸿门,谢罪说:“我和将军合力而攻打秦国,将军在河北作战,我在河南作战,然而我自己也没想到能够先入关打败秦军,得以在这里再次见到将军。现在有小人进谗言,使将军和我产生矛盾。”

项王说:“这是沛公的左司马曹无伤说的,不是这样的话,我怎么会来到这里?”

项王当天就留下沛公一同宴饮。项王、项伯朝东而坐。亚父朝南而坐。亚父,就是范增。沛公朝北而坐,张良朝西陪坐。

范增多次对项王使眼色,多次举起自己所佩戴的玉玦向项王示意,项王沉默不回应。范增起身,出去找来项庄,对他说:“君王为人不够狠心,你进去上前敬酒,敬酒之后,请求舞剑,趁机袭击座位上的沛公,把他杀死。不这样的话,你们都将被他所擒获。”

项庄就进去敬酒,敬酒之后,说:“君王和沛公饮酒,军中没有什么可以取乐,请让我舞剑。”

项王说:“好。”

项庄拔剑起舞,项伯也拔剑起舞,总是用身体保护沛公,项庄没有机会袭击沛公。

陈仁锡:「人臣之义无私交,项伯不忠,为子房用,因为沛公用,以致养虎遗患,蹙灭项氏,罪无容逭矣。然虽项伯之卖国,亦由羽之疏慢其亲臣,而沛公之能得人心也。」又曰:「项伯之贰于羽,以羽之王章邯也。项梁,羽季父,伯亦羽季父。邯杀梁而羽王邯,伯之心始寒矣。纪首曰:『项氏世世为楚将,封于项,故姓项氏。』末云『皆项氏,赐姓刘氏』,罪项王也,罪项王之堕其宗也。」

\begin{yuanwen}
于是张良至军门,见樊哙\footnote{kuài}。樊哙曰:“今日之事何如?”

良曰:“甚急。今者项庄拔剑舞,其意常在沛公也。”

哙曰:“此迫矣,臣请入,与之同命\footnote{text}。”哙即带剑拥盾入军门\footnote{text}。

交戟之卫士欲止不内,樊哙侧其盾以撞,卫士仆地,哙遂入。披帷西乡立,瞋目\footnote{瞪眼。瞋,chēn。}视项王,头发上指,目眦尽裂。项王按剑而跽\footnote{双膝跪地,上身挺直。古人席地而坐,臀部坐在小腿上。“跽”指半起身,可以拱手表示敬意。项羽“按剑而趿”,是表示警惕,随时可以起身格斗。}曰:“客何为者?”

张良曰:“沛公之参乘\footnote{陪乘的人。古代乘车,御者居中,尊者在左,参乘在右。}樊哙者也。”

项王曰:“壮士,赐之卮酒。”则与斗卮酒。

哙拜谢,起,立而饮之。

项王曰:“赐之彘肩\footnote{猪前腿。彘,zhì。}。”则与一生彘肩。樊哙覆其盾于地,加彘肩上,拔剑切而啖\footnote{dàn}之。

项王曰:“壮士,能复饮乎?”

樊哙曰:“臣死且不避,卮酒安足辞!夫秦王有虎狼之心,杀人如不能举\footnote{text},刑人如恐不胜\footnote{text},天下皆叛之。怀王与诸将约曰,‘先破秦入咸阳者王之’。今沛公先破秦入咸阳,豪毛不敢有所近,封闭宫室,还军霸上,以待大王来。故遣将守关者,备他盗出入与非常也。劳苦而功高如此,未有封侯之赏,而听细\footnote{text}说,欲诛有功之人。此亡秦之续耳,窃为大王不取也。”

项王未有以应,曰:“坐。”

樊哙从良坐。坐须臾,沛公起如厕,因招樊哙出。
\end{yuanwen}

这时张良来到军门,见到樊哙。樊哙说:“今天的情况怎么样?”

张良说:“非常危急。现在项庄拔剑起舞,用意在于刺杀沛公。”

樊哙说:“这太紧急了,请让我进去,与他们拼命!”樊哙立即带剑持盾进入军门。

交戟守门的卫士想要阻拦不让他进去,樊哙侧过他的盾撞过去,卫士倒在地上,樊哙于是进入大帐,他揭开帷幕朝西站立,瞪着眼睛看着项王,头发向上竖起,眼眶都要破裂了。项王立即伸手握住宝剑,挺直了上身说:“客人是干什么的?”

张良说:“这是沛公的参乘樊哙。”

项王说:“壮士,赐给他一杯酒。”侍卫就给他盛满一斗酒的大杯。

樊哙拜谢后,起身,站着一饮而尽。

项王说:“赐给他猪腿。”侍卫就给一只生猪腿。樊哙将盾放倒在地上,把猪腿放在盾上,拔出剑来切肉吃。

项王说:“壮士,还能再饮酒吗?”

樊哙说:“我连死都不怕,又怎么会拒绝一杯酒呢!那秦王有虎狼之心,杀人就像担心杀不尽一样,用刑就像担心不够重一样,天下人都背叛了他。怀王和众将领约定说:‘先打败秦军进入咸阳的人就可以在关中称王。’现在沛公率先打败秦军进入咸阳,财物分毫不敢接受,还封闭宫室,回到霸上驻扎,以此等待大王到来。派将士把守关口的原因,是防备别的盗贼出入和发生意外。像这样劳苦而功高,不但没有封侯的奖赏,而且听信闲言碎语,想要诛杀有功的人。这只是在重走灭亡的秦朝的老路罢了,我私下里认为大王的做法不可取。”

项王不能回应,只是说:“请坐。”

樊哙在张良旁边坐下。坐了不久,沛公起身上厕所,顺便叫樊哙出来。

\begin{yuanwen}
沛公已出,项王使都尉陈平召沛公\footnote{text}。沛公曰:“今者出,未辞也,为之奈何?”

樊哙曰:“大行不顾细谨,大礼不辞小让。如今人方为刀俎\footnote{刀和砧板。},我为鱼肉,何辞为?”于是遂去。乃令张良留谢。

良问曰:“大王\footnote{刘邦当时也尚未称王。}来何操?”

曰:“我持白璧一双,欲献项王;玉斗一双,欲与亚父,会其怒,不敢献。公为我献之。”

张良曰:“谨诺。”

当是时,项王军在鸿门下,沛公军在霸上,相去四十里。沛公则置车骑\footnote{text},脱身独骑,与樊哙、夏侯婴、靳强、纪信等四人持剑盾步走,从郦山下\footnote{text},道芷阳间行\footnote{text}。沛公谓张良曰:“从此道至吾军,不过二十里耳。度我至军中,公乃入。”

沛公已去,间至军中\footnote{text},张良入谢,曰:“沛公不胜杯杓\footnote{sháo},不能辞。谨使臣良奉白璧一双,再拜献大王足下\footnote{对上级的尊称,后来多用于称呼同辈。};玉斗一双,再拜奉大将军\footnote{指范增。}足下。”

项王曰:“沛公安在?”

良曰:“闻大王有意督过之\footnote{text},脱身独去,已至军矣。”

项王则受璧,置之坐上。亚父受玉斗,置之地,拔剑撞而破之,曰:“唉!竖子\footnote{奴仆,多用于骂人。}不足与谋。夺项王天下者,必沛公也,吾属今为之虏矣。”

沛公至军,立诛杀曹无伤。
\end{yuanwen}

沛公出去以后,项王派都尉陈平叫沛公回来。沛公说:“我们现在出去,没有辞行,怎么办呢?”

樊哙说:“做大事就不要顾及小的礼数,持大节就不要回避小的责备。现在别人是刀和砧板,我们是待宰割的鱼和肉,还告辞做什么?”于是离去。沛公临走时命令张良留下道歉。

张良问道:“大王来时带了什么?”

沛公说:“我带来一双白璧,想献给项王,一双玉斗,想送给亚父,正赶上他们生气,没敢进献。您替我献给他们。”

张良说:“遵命。”

在这个时候,项王的军队在鸿门下,沛公的军队在霸上,距离四十里。沛公则丢下车骑,独自骑马逃脱,樊哙、夏侯婴、靳强、纪信等四人带剑持盾跑步,从郦山下,经芷阳走小路。沛公对张良说:“从这条路到我的军营,只有二十里路罢了。估计我已经回到军中,您再进去。”

沛公离去以后,走小路回到军中。张良进去道歉,说:“沛公不胜酒力,不能亲自辞行。特意让我奉上白璧一双,拜献大王;玉斗一双,拜送大将军。”

项王说:“沛公在哪里?”

张良说:“听说大王有责备他的意思,就独自脱身而去,现在已经回到军中了。”

项王则接过玉璧,放在坐席上。亚父接过玉斗,扔在地上,拔剑将其砍碎,说:“唉!这小子不足以谋大事。夺取项王天下的,一定是沛公,我们都要被他俘虏了。”

沛公回到军中,立刻诛杀曹无伤。

\begin{yuanwen}
居数日,项羽引兵西屠咸阳,杀秦降王子婴,烧秦宫室,火三月不灭;收其货宝妇女而东。人或说项王曰:“关中阻山河四塞,地肥饶,可都以霸。”

项王见秦宫室皆以烧残破,又心怀思欲东归,曰:“富贵不归故乡,如衣绣夜行,谁知之者!”

说者曰:“人言楚人沐猴\footnote{猕猴。}而冠耳\footnote{text},果然。”

项王闻之,烹\footnote{用大锅煮人的酷刑。}说者。
\end{yuanwen}

过了几天,项羽领兵西进屠戮咸阳,杀死已经投降的秦王子婴,烧毁秦朝的宫室,大火三个月都没有熄灭,并且收缴财宝、劫掠妇女向东而去。有人劝阻项王说:“关中有群山和黄河为险阻,四面都是关塞,土地肥沃富饶,可以定都来建立霸业。”

项王看见秦朝的宫室都被烧得残破不堪,心中又思念故乡而想要回到东方,就说:“得到富贵却不回到故乡,就像穿着绣花的衣服在夜里走路,有谁会知道呢!”

劝项王的人说:“人们都说楚国人只是猕猴戴帽子罢了,真是这样。”

项王听到这句话,将这个人烹杀。

\begin{yuanwen}
项王使人致命怀王\footnote{text}。怀王曰:“如约。”乃尊怀王为义帝。

项王欲自王,先王诸将相。谓曰:“天下初发难时,假立诸侯后以伐秦\footnote{text}。然身被坚执锐首事,暴露于野三年,灭秦定天下者,皆将相诸君与籍之力也。义帝虽\footnote{仅,只。}无功,故当分其地而王之。”

诸将皆曰:“善。”乃分天下,立诸将为侯王。

项王、范增疑沛公之有天下,业已讲解,又恶负约,恐诸侯叛之,乃阴谋\footnote{text}曰:“巴、蜀道险\footnote{text},秦之迁人皆居蜀。”

乃曰:“巴、蜀亦关中地也\footnote{text}。”

故立沛公为汉王,王巴、蜀、汉中\footnote{text},都南郑。而三分关中,王秦降将以距塞汉王。
\end{yuanwen}

项王派人向怀王请示。怀王说:“按照约定办。”

项羽于是尊怀王为义帝。项王想自己称王,就先分封众将相。他对众人说:“天下人刚开始起事的时候,暂时拥立诸侯的后裔为王来讨伐秦朝。然而亲自身穿铠甲,手执利器,率先起事,三年来风餐露宿,消灭秦朝平定天下,依靠的都是各位将相和我的力量。只是义帝没有功劳,所以应当分出他的土地封给各位为王。”

众将领都说:“好。”项王于是分封天下,立众将领为王侯。

项王、范增怀疑沛公有夺取天下的想法,却已经与他和解,又不想违背约定,担心诸侯背叛他,就在暗中商量说:“巴、蜀道路艰险,被秦朝贬谪的罪人都居住在蜀地。”

于是说:“巴、蜀也是关中地区。”

所以项羽封沛公为汉王,统治巴、蜀、汉中三个郡,定都南郑。而把关中一分为三,封给秦朝的降将来阻挡汉王。

\begin{yuanwen}
项王乃立章邯为雍王,王咸阳以西,都废丘。长史欣者,故为栎阳狱掾\footnote{yuàn},尝有德于项梁;都尉董翳者,本劝章邯降楚。故立司马欣为塞王,王咸阳以东至河,都栎阳;立董翳为翟王,王上郡,都高奴。徙魏王豹为西魏王,王河东,都平阳。瑕丘申阳者,张耳嬖臣也,先下河南,迎楚河上,故立申阳为河南王,都雒阳。韩王成因故都,都阳翟。赵将司马卬\footnote{áng}定河内,数有功,故立卬为殷王,王河内,都朝歌。徙赵王歇为代王。赵相张耳素贤,又从入关,故立耳为常山\footnote{即恒山。《史记》避汉文帝刘恒讳,改恒为常。}王,王赵地,都襄国。当阳君黥布为楚将,常冠军,故立布为九江王,都六。鄱\footnote{也作“番”。}君吴芮率百越佐诸侯,又从入关,故立芮为衡山王,都邾\footnote{zhū}。义帝柱国共敖将兵击南郡,功多,因立敖为临江王,都江陵。徙燕王韩广为辽东王。燕将臧荼从楚救赵,因从入关,故立荼为燕王,都蓟。徙齐王田市为胶东王。齐将田都从共救赵,因从入关,故立都为齐王,都临菑。故秦所灭齐王建孙田安,项羽方渡河救赵,田安下济北数城,引其兵降项羽,故立安为济北王,都博阳。田荣者,数负项梁,又不肯将兵从楚击秦,以故不封。成安君陈馀弃将印去,不从入关,然素闻其贤,有功于赵,闻其在南皮,故因环封三县。番君将梅鋗\footnote{xuān}功多,故封十万户侯。项王自立为西楚霸王\footnote{text},王九郡,都彭城\footnote{text}。
\end{yuanwen}

项王于是封章邯为雍王,统治咸阳以西,定都废丘。长史司马欣,以前担任过栎阳狱掾,曾经对项梁有恩德。都尉董翳,是最初劝说章邯投降楚国的人。因此项王封司马欣为塞王,统治咸阳以东到黄河边,定都栎阳。封董翳为翟王,统治上郡,定都高奴。改封魏王豹为西魏王,统治河东地区,定都平阳。瑕丘人申阳,是张耳的宠臣,先攻下河南,在黄河边迎接楚军,因此项王封申阳为河南王,定都雒阳。韩王成在旧都建国,定都阳翟。赵将司马卬平定河内,屡次建立战功,因此封为殷王,统治河内地区,定都朝歌。改封赵王歇为代王。赵相张耳向来贤能,又追随项王进入关中,因此封张耳为常山王,统治赵地,定都襄国。当阳君黥布担任楚将,经常勇冠全军,因此封黥布为九江王,定都六县。鄱君吴芮率领百越协助诸侯,又跟随项王进入关中,因此封吴芮为衡山王,定都邾县。义帝的柱国共敖率领士兵攻打南郡,建立很多战功,因此封共敖为临江王,定都江陵。改封燕王韩广为辽东王。燕将臧荼曾经追随楚军援救赵军,因而跟随项王进入关中,所以封臧荼为燕王,定都蓟县。改封齐王田巿为胶东王。齐将田都曾经跟随楚军共同援救赵国,因而跟随项王进入关中,所以封田都为齐王,定都临菑。过去被秦朝所灭掉的齐王建的孙子田安,在项羽渡河援救赵国的时候,攻下济水以北的几座城,带领他的军队投降项羽,因此封田安为济北王,定都博阳。田荣,多次辜负项梁,又不愿带领军队跟随楚军攻打秦军,所以不封。成安君陈馀丢下将印离去,没有跟随项王进入关中,然而项王向来听说他很贤能,对赵国有功,听说他在南皮,因此顺便把南皮周边的三个县封给他。番君的将领梅鋗战功很多,因此封为十万户侯。项王自立为西楚霸王,统治九个郡,定都彭城。

\begin{yuanwen}
汉之元年\footnote{刘邦称王建国而尚未即帝位,因此不称高祖元年。刘邦称帝在称王的第五年,也没有改元。}四月\footnote{text},诸侯罢戏下\footnote{一说为麾下,一说为戏水边。},各就国。

项王出之国,使人徙义帝\footnote{text},曰:“古之帝者地方千里,必居上游。”

乃使使徙义帝长沙郴\footnote{chēn}县,趣义帝行。其群臣稍稍背叛之,乃阴令衡山、临江王击杀之江中\footnote{text}。
\end{yuanwen}

汉元年(前206年)四月,诸侯从戏下撤离,各自回到封国。

项王也出关回到封国,派人迁徙义帝,说:“古时候的帝王拥有方圆千里的土地,一定要居住在上游。”

于是他派使者将义帝迁往长沙郡的郴县。项王催促义帝早些动身,群臣逐渐产生了背叛之心,于是项王暗中命令衡山王、临江王在长江上杀死义帝。

\begin{yuanwen}
韩王成无军功,项王不使之国,与俱至彭城,废以为侯,已又杀之。臧荼\footnote{tú}之国,因逐韩广之辽东,广弗听,荼击杀广无终,并王其地。
\end{yuanwen}

韩王成没有军功,项王不让他回到封国,带着他一起来到彭城,废为侯爵,不久又将他杀死。臧荼回到封国,趁机将韩广驱逐到辽东,韩广不服从,臧荼就在无终杀死韩广,吞并了他的封地。

\begin{yuanwen}
田荣闻项羽徙齐王巿\footnote{fú}胶东,而立齐将田都为齐王,乃大怒,不肯遣齐王之胶东,因以齐反,迎击田都。田都走楚。齐王巿畏项王,乃亡之胶东就国。田荣怒,追击杀之即墨。荣因自立为齐王,而西杀击济北王田安,并王三齐\footnote{齐、胶东、济北三郡。}。荣与彭越将军印,令反梁地。陈馀阴使张同、夏说说齐王田荣曰:“项羽为天下宰,不平。今尽王故王于丑地,而王其群臣诸将善地,逐其故主赵王,乃北居代,馀以为不可。闻大王起兵,且不听不义,原大王资馀兵,请以击常山,以复赵王,请以国为扞\footnote{hàn}蔽。”

齐王许之,因遣兵之赵。陈馀悉发三县兵,与齐并力击常山,大破之。张耳走归汉。陈馀迎故赵王歇于代,反之赵。赵王因立陈馀为代王。
\end{yuanwen}

田荣听说项羽改封齐王田巿为胶东王,而立齐将田都为齐王,于是非常生气,不想让齐王去胶东,趁机在齐国反叛,迎战田都。田都逃往楚国。齐王田巿惧怕项王,就逃到胶东的封国。田荣很生气,派兵追击,在即墨将他杀死。田荣因此自立为齐王,并且向西进攻杀死济北王田安,吞并三齐地区。田荣授予彭越将军印,让他在梁地反叛。陈馀暗中派张同、夏说劝齐王田荣说:“项羽是天下的主宰,分封诸侯很不公平。现在将原来的六国之王都封在不好的地方,而将他的群臣诸将都封在好的地方,还驱逐他原来的君主,赵王于是居住在北方的代郡,我认为不能这样。我听说大王已经起兵,而且不服从不道义的命令,希望大王资助我军队,请让我攻打常山,恢复赵王的地位,请求将赵国作为齐国的屏障。”齐王答应了,顺势派兵去赵国。陈馀调发三个县的全部士兵,与齐军合力攻打常山,大破常山军。张耳逃走归附汉王。陈馀去代郡迎接原来的赵王歇,使他返回赵国。赵王顺势封陈馀为代王。

\begin{yuanwen}
是时,汉还定三秦\footnote{指雍、塞、翟三国。}。项羽闻汉王皆已并关中,且东,齐、赵叛之,大怒。乃以故吴令郑昌为韩王,以距汉。令萧公角等击彭越。彭越败萧公角等。汉使张良徇韩,乃遗项王书曰:“汉王失职\footnote{失去职权。},欲得关中,如约即止,不敢东。”

又以齐、梁反书遗项王曰:“齐欲与赵并灭楚。”

楚以此故无西意,而北击齐。徵兵九江王布。布称疾不往,使将将数千人行。项王由此怨布也。
\end{yuanwen}

这个时候,汉王回军平定三秦地区。项羽听说汉王已经吞并整个关中地区,将要向东进发,齐、赵两国也背叛了他,感到非常生气。于是项羽封前吴县令郑昌为韩王,来阻挡汉军,命令萧公角等人攻打彭越。彭越打败萧公角等人,汉王派张良攻取韩地,张良就写信给项王说:“汉王失去了应得的官爵,想要得到关中,实现约定就会停止进军,不敢向东进军。”

张良又将齐、梁两国的反叛文书送给项王说:“齐国想要联合赵国一起消灭楚国。”

楚军因为这个缘故,没有向西进军的想法,反而向北攻打齐国。项王向九江王黥布征调士兵。黥布称病不去,只派将领率几千人前去。项王由此开始怨恨黥布。

\begin{yuanwen}
汉之二年冬,项羽遂北至城阳,田荣亦将兵会战。田荣不胜,走至平原,平原民杀之。遂北烧夷齐城郭室屋,皆阬田荣降卒,系虏其老弱妇女。徇齐至北海,多所残灭。齐人相聚而叛之。于是田荣弟田横收齐亡卒得数万人,反城阳。项王因留,连战未能下。
\end{yuanwen}

汉二年(前205年)冬季,项羽向北行进到城阳,田荣也率领军队来到城阳与其交战。田荣不能取胜,逃到平原,平原的百姓将他杀死。项羽于是向北进军烧毁了齐国的城郭和房屋,把田荣的降卒全部坑杀,掳掠齐国的老弱和妇女。楚军从齐国攻城略地到北海郡,所到之处大多被摧毁。齐国人聚集在一起反叛项羽。于是田荣的弟弟田横收拢齐军逃亡的士卒得到几万人,在城阳反叛。项王因此留下作战,连续交战几次也没有攻下城阳。

\begin{yuanwen}
春,汉王部五诸侯兵\footnote{text},凡五十六万人,东伐楚。项王闻之,即令诸将击齐,而自以精兵三万人南从鲁出胡陵\footnote{text}。

四月,汉皆已入彭城,收其货宝美人,日置酒高会。项王乃西从萧,晨击汉军而东\footnote{text},至彭城,日中,大破汉军。汉军皆走,相随入穀(谷)、泗水\footnote{text},杀汉卒十馀万人。汉卒皆南走山,楚又追击至灵壁东睢水上\footnote{text}。汉军却,为楚所挤,多杀,汉卒十馀万人皆入睢水,睢水为之不流。围汉王三匝\footnote{zā}。于是大风从西北而起,折木发屋,扬沙石,窈冥\footnote{阴暗的样子。}昼晦\footnote{text},逢迎楚军\footnote{text}。楚军大乱,坏散,而汉王乃得与数十骑遁去。欲过沛,收家室而西;楚亦使人追之沛,取汉王家,家皆亡,不与汉王相见。汉王道逢得孝惠、鲁元\footnote{text},乃载行。楚骑追汉王,汉王急,推堕孝惠、鲁元车下,滕公\footnote{夏侯婴。}常下收载之\footnote{text}。如是者三。曰:“虽急不可以驱,奈何弃之?”

于是遂得脱。求太公、吕后\footnote{text},不相遇。审食\footnote{yì}其\footnote{jī}从太公、吕后间行,求汉王,反遇楚军。楚军遂与归,报项王,项王常置军中。
\end{yuanwen}

春季,汉王调遣五路诸侯的军队,共计五十六万人,一起向东讨伐楚国。项王听说后,立即命令众将领攻打齐国,而自己则率领精锐士兵三万人向南从鲁县越过胡陵。

四月,汉军已经全部进入彭城,收缴和劫掠项羽的财宝和美女,每天设宴豪饮。项王向西经过萧县,在清晨攻打汉军向东进发,到达彭城,中午,大破汉军。汉军将士都逃跑了,相继逃入穀水、泗水,楚军杀死汉军士兵十多万人。汉军士兵都向南方的山里逃跑,楚军又追到灵壁以东的睢水边。汉军败退,受到楚军的逼迫,大多被杀死。汉军士兵十多万人都跌进睢水中,堵塞了睢水。楚军将汉王包围了三重。这时从西北刮起一阵大风,折断了树木,掀翻了房屋,扬起了沙石,白天阴暗得像傍晚一样,狂风刮向楚军。楚军大乱,队形散乱,而汉王才得以和几十名骑兵逃出去。汉王想要途经沛县,带上家眷向西撤退。楚军也派人追到沛县,抄截汉王的家眷。汉王的家眷都逃跑了,没有见到汉王。汉王在路上遇到孝惠帝、鲁元公主,就用车载着他们走。楚军的骑兵追赶汉王,汉王很着急,就将孝惠帝、鲁元公主推下车,滕公总是下车将他们重新抱上来,这样反复多次。滕公说:“事情虽然危急,无法让车走得更快,怎么能丢弃他们呢?”

汉王终于得以逃脱。汉王寻找太公、吕后而没有找到。审食其与太公、吕后走小路,也在寻找汉王,反而遭遇楚军。楚军就带他们返回,报告项王,项王将他们一直留在军中。

\begin{yuanwen}
是时吕后兄周吕侯\footnote{吕泽。}为汉将兵居下邑,汉王间往从之,稍稍收其士卒。至荥阳,诸败军皆会,萧何亦发关中老弱未傅\footnote{未载入服役簿籍中的人。老弱者不必服役,年龄未满二十三岁为弱,超过五十六岁为老。}悉诣荥阳,复大振。楚起于彭城,常乘胜逐北,与汉战荥阳南京、索间,汉败楚,楚以故不能过荥阳而西。
\end{yuanwen}

这时吕后的兄长周吕侯,担任汉将驻扎在下邑,汉王走小路来到他的军营里,逐渐收拢逃散的士兵。来到荥阳,各路败军都聚集在一起,萧何也征发关中没有编入服役簿籍的老弱之人,全部送到荥阳,汉军的士气再次振奋。楚军从彭城出发,经常乘胜追击败兵,与汉军在荥阳以南的京、索两邑之间交战,汉军打败楚军,楚军因此无法越过荥阳向西挺进。

\begin{yuanwen}
项王之救彭城,追汉王至荥阳,田横亦得收齐,立田荣子广为齐王。汉王之败彭城,诸侯皆复与楚而背汉。汉军荥阳,筑甬道属之河,以取敖仓粟。
\end{yuanwen}

项王援救彭城时,追击汉王到荥阳,田横也趁机收复了齐国,立田荣的儿子田广为齐王。汉王在彭城战败时,诸侯又都投靠楚国而背叛汉国。汉军驻扎在荥阳,修筑甬道与黄河相连,来运送敖仓的粮食。

\begin{yuanwen}
汉之三年,项王数侵夺汉甬道,汉王食乏,恐,请和,割荥阳以西为汉。

项王欲听之。历阳侯范增曰:“汉易与耳,今释弗取,后必悔之。”

项王乃与范增急围荥阳。汉王患之,乃用陈平计间项王。项王使者来,为太牢具\footnote{猪、牛、羊齐备的盛宴。},举欲进之。见使者,详惊愕曰:“吾以为亚父使者,乃反项王使者。”

更持去,以恶食食项王使者。使者归报项王,项王乃疑范增与汉有私,稍夺之权。范增大怒,曰:“天下事大定矣,君王自为之。原赐骸\footnote{hái}骨归卒伍。\footnote{赐骸骨归卒伍:告老还乡为平民。官员献身国家,所以年老辞官称“赐骸骨”、“乞骸骨”,请求君主赐还枯骨以回乡安葬。古代编户以五户为“伍”,三百户为“卒”,“卒伍”代指乡里。}”

项王许之。行未至彭城,疽发背而死。
\end{yuanwen}

汉三年(前204年),项王屡次侵扰汉军的甬道夺取粮食,汉王缺少粮食,感到害怕,请求和解,要求把荥阳以西割让给汉国。

项王想要答应这个条件。历阳侯范增说:“汉军很容易对付了,现在放弃而不获取,以后一定会后悔。”

项王就和范增猛攻荥阳。汉王为此担忧,就采用陈平的计策离间项王君臣。项王的使者到来,为他准备丰盛的饭食,准备好将要进献。送饭食的人看到使者,假装惊讶地说:“我以为是亚父的使者,竟然是项王的使者!”

他就把饭食撤下去,换上粗劣的饭食给项王的使者吃。使者回来报告项王,项王于是怀疑范增与汉军私下勾结,逐渐剥夺了他的权力。范增十分生气,说:“天下的形势已经确定了,请君王好自为之。请求准许我告老还乡。”

项王答应了。范增还没有走到彭城,就因背上长毒疮而死去。

陈仁锡:「固陵之追,籍已兵罢食尽,犹能大破汉军,则刘非项敌明矣。然未几而三将会兵,卒能破楚,则汉之用人与籍之自用,其相去何翅什伯?详书信、越、英布会兵之由,所以见汉之擒籍,卒赖此三人力也。」

\begin{yuanwen}
汉将纪信说汉王曰:“事已急矣,请为王诳\footnote{欺骗。}楚为王,王可以间出。”

于是汉王夜出女子荥阳东门被甲二千人,楚兵四面击之。纪信乘黄屋车,傅左纛\footnote{dào,帝王车马上的羽毛饰物。},曰:“城中食尽,汉王降。”

楚军皆呼万岁。汉王亦与数十骑从城西门出,走成皋\footnote{gāo}。项王见纪信,问:“汉王安在?”

曰:“汉王已出矣。”项王烧杀纪信。
\end{yuanwen}

汉将纪信劝汉王说:“情况已经非常危急了,请让我假扮成大王去骗骗楚军,大王就可以趁机逃出去了。”

于是汉王在夜里从荥阳东门放出两千名身披铠甲的女子,楚军从四面围攻她们。纪信乘坐有黄色伞盖的车子,左侧的马头上插着羽毛饰物,说:“城中的粮食吃光了,汉王出城投降。”

楚军将士都高呼万岁。这时汉王和几十名骑兵从西门出城,逃往成皋。项王见到纪信,问他:“汉王在哪里?”

纪信说:“汉王已经出城了。”项王烧死了纪信。

\begin{yuanwen}
汉王使御史大夫周苛、枞公、魏豹守荥阳。周苛、枞公谋曰:“反国之王,难与守城。”乃共杀魏豹。

楚下荥阳城,生得周苛。项王谓周苛曰:“为我将,我以公为上将军,封三万户。”

周苛骂曰:“若不趣降汉,汉今虏若,若非汉敌也。”

项王怒,烹周苛,井杀枞公。
\end{yuanwen}

汉王派御史大夫周苛、枞公、魏豹驻守荥阳。周苛、枞公商议说:“魏豹是反叛之国的王,与他一起很难守住城池。”于是二人一起杀死了魏豹。

楚军攻下荥阳城,活捉了周苛。项王对周苛说:“做我的将领,我任命您为上将军,封三万户食邑。”

周苛大骂说:“你不赶快投降汉王,汉王就要俘虏你了,你不是汉王的对手!”

项王很生气,烹杀周苛,一并杀死枞公。

\begin{yuanwen}
汉王之出荥阳,南走宛、叶,得九江王布,行收兵,复入保成皋。
\end{yuanwen}

汉王逃出荥阳以后,向南经过宛县、叶县,招降九江王黥布,一路上收拢逃散的士兵,再次进入成皋固守。

\begin{yuanwen}
汉之四年,项王进兵围成皋。汉王逃,独与滕公出成皋北门,渡河走脩武,从张耳、韩信军。诸将稍稍得出成皋,从汉王。楚遂拔成皋,欲西。汉使兵距之巩,令其不得西。
\end{yuanwen}

汉四年(前203年),项王进军围攻成皋。汉王逃走,独自一人与滕公出成皋北门,渡过黄河逃往脩武,来到张耳、韩信的军营。众将领陆续逃出成皋,前去追随汉王。楚军终于攻下成皋,想要向西进军。汉王派兵在巩县抗拒,使其无法西进。

\begin{yuanwen}
是时,彭越渡河击楚东阿,杀楚将军薛公。项王乃自东击彭越。汉王得淮阴侯\footnote{韩信在汉朝建立后的封号。}兵,欲渡河南。郑忠说汉王,乃止壁河内。使刘贾将兵佐彭越,烧楚积聚。项王东击破之,走彭越。汉王则引兵渡河,复取成皋,军广武,就敖仓食。项王已定东海来,西,与汉俱临广武而军,相守数月。
\end{yuanwen}

这时,彭越渡过黄河在东阿攻打楚军,杀死了楚国的将军薛公。项王于是亲自向东攻打彭越。汉王得到淮阴侯的军队,想要渡过黄河南下。郑忠劝阻汉王,汉王才留在河内筑起营垒。汉王派刘贾率领军队协助彭越,烧毁楚军聚集的物资。项王向东打败汉军,彭越逃走了。汉王则率军渡过黄河,重新夺取成皋,驻扎在广武,吃敖仓的粮食。项王平定东海以后,向西进发,与汉军都在广武驻扎,两军相持几个月。

\begin{yuanwen}
当此时,彭越数反梁地,绝楚粮食,项王患之。为高俎\footnote{一说为盛肉的四脚方盘,后来演化为砧板,一说为行军时装在车上用于瞭望的高台。},置太公其上,告汉王曰:“今不急下,吾烹太公。”

汉王曰:“吾与项羽俱北面受命怀王,曰‘约为兄弟’,吾翁即若翁,必欲烹而翁,则幸分我一杯羹\footnote{text}。”

项王怒,欲杀之。项伯曰:“天下事未可知,且为天下者不顾家,虽杀之无益,只益祸耳\footnote{text}。”项王从之。
\end{yuanwen}

在这个时候,彭越在梁地多次反击楚军,切断了楚军运送粮食的路线,项王为此很担心。他准备好高俎,将太公放在上面,警告汉王说:“现在不赶快投降,我就烹杀太公。”

汉王说:“我和你都面朝北方听命于怀王,说过‘结为兄弟’的话,我的父亲就是你的父亲,你一定要烹杀自己的父亲,就请分给我一杯肉羹。”

项王很生气,想要烹杀太公。项伯说:“天下的形势还不明朗,况且打天下的人不会顾及家人,就算杀了他也没有好处,只会增添祸患罢了。”项王听从了项伯的劝告。

\begin{yuanwen}
楚汉久相持未决,丁壮苦军旅,老弱罢转漕\footnote{text}。项王谓汉王曰:“天下匈匈\footnote{纷乱的样子。}数岁者\footnote{text},徒以吾两人耳,原与汉王挑战决雌雄,毋徒苦天下之民父子为也。”

汉王笑谢曰:“吾宁斗智,不能斗力。”

项王令壮士出挑战。汉有善骑射者楼烦\footnote{text},楚挑战三合\footnote{text},楼烦辄射杀之。项王大怒,乃自被甲持戟挑战。楼烦欲射之,项王瞋目叱之,楼烦目不敢视,手不敢发,遂走还入壁,不敢复出。汉王使人间问之,乃项王也。汉王大惊。于是项王乃即汉王相与临广武间\footnote{jiàn,山涧。}而语\footnote{text}。汉王数之\footnote{text},项王怒,欲一战。汉王不听,项王伏弩射中汉王\footnote{text}。汉王伤,走入成皋\footnote{text}。
\end{yuanwen}

楚、汉两军长期相持不分胜负,丁壮为行军作战而劳苦,老弱因转运物资而疲惫。项王对汉王说:“天下纷乱多年,只是因为我们两个人罢了,我愿向汉王挑战来一决胜负,不要再让天下的百姓白受苦了。”

汉王笑着推辞说:“我宁愿比智慧,也不会比力气。”

项王命令壮士出去挑战。汉军有个擅长骑马射箭的人叫楼烦,楚军接连挑战三次,楼烦都把前来挑战的壮士射死了。项王非常生气,就亲自披上铠甲拿起长戟出来挑战。楼烦想要射击,项王瞪着眼睛呵叱,楼烦眼不敢正视,手不敢放箭,就跑进营垒,再也不敢出来了。汉王派人暗中打听,才知道是项王。汉王大惊。于是项王靠近汉王的军营,和他隔着广武涧谈话。汉王列举项王的罪状,项王很生气,想要决战。汉王不同意,项王军中埋伏好的弓弩手射中了汉王。汉王受伤,逃回成皋。

\begin{yuanwen}
项王闻淮阴侯已举河北,破齐、赵,且欲击楚,乃使龙且往击之。淮阴侯与战,骑将灌婴击之,大破楚军,杀龙且。韩信因自立为齐王。项王闻龙且军破,则恐,使盱台人武濊涉往说淮阴侯。淮阴侯弗听。是时,彭越复反,下梁地,绝楚粮。项王乃谓海春侯大司马曹咎等曰:“谨守成皋,则汉欲挑战,慎勿与战,毋令得东而已。我十五日必诛彭越,定梁地,复从将军。”乃东,行击陈留、外黄。

外黄不下。数日,已降,项王怒,悉令男子年十五已上诣城东,欲阬之。

外黄令舍人兒年十三,往说项王曰:“彭越彊劫外黄,外黄恐,故且降,待大王。大王至,又皆阬之,百姓岂有归心?从此以东,梁地十馀城皆恐,莫肯下矣。”

项王然其言,乃赦外黄当阬者。东至睢阳,闻之皆争下项王。
\end{yuanwen}

项王听说淮阴侯已经攻下河北地区,打败齐、赵两国军队,并且想要攻打楚军,就派龙且前去迎战。淮阴侯与龙且交战,骑兵将领灌婴攻击龙且,大破楚军,杀死龙且。韩信趁机自立为齐王。项王听说龙且的军队被打败,感到害怕,派盱台人武涉前去劝说淮阴侯。淮阴侯不听从。这时,彭越再次反叛楚军,攻下梁地,断绝了楚军的粮道。项王就对海春侯大司马曹咎等人说:“谨慎地防守成皋,即使汉军想要挑战,也千万不要与其交战,只要不让他们向东进发就行了。我十五天以内一定会杀掉彭越,平定梁地,再回来与将军会合。”于是项王向东进军,一路上攻打陈留、外黄。

外黄不能攻下。过了几天,外黄终于投降,项王非常生气,命令将十五岁以上的男子全部送到城东,想要将他们坑杀。

外黄县令门客的儿子只有十三岁,他前去劝项王说:“彭越武力强迫外黄和他反叛,外黄百姓都很害怕,所以暂时投降彭越,等待大王到来。现在大王到来,又要将他们全部坑杀,百姓怎么会有归顺的想法呢?从这里往东,梁地的十几座城都会感到害怕,再也没有肯投降的了。”

项王认为他的话很有道理,就赦免了外黄在坑杀之列的人。楚军向东行进到睢阳,百姓听说后,都争相投降项王。

\begin{yuanwen}
汉果数挑楚军战,楚军不出。使人辱之,五六日,大司马怒,渡兵汜水。士卒半渡,汉击之,大破楚军,尽得楚国货赂。大司马咎、长史翳、塞王欣\footnote{塞王司马欣曾任长史,董翳曾任都尉,不应称长史。按《高祖本纪》《汉书》都没有“翳塞王”三字,可知在汜水自杀者为曹咎、司马欣二人,董翳不在其中。}皆自刭汜水上。大司马咎者,故蕲狱掾,长史欣亦故栎阳狱吏,两人尝有德于项梁,是以项王信任之。当是时,项王在睢阳,闻海春侯军败,则引兵还。汉军方围锺离眛于荥阳东,项王至,汉军畏楚,尽走险阻。
\end{yuanwen}

汉军果然屡次向楚军挑战,楚军不出去交战。汉军派人在阵前辱骂,连续五六天,大司马非常生气,让士兵渡过汜水。士兵刚渡过一半,汉军发起进攻,大破楚军,缴获了楚军的全部物资。大司马曹咎、长史司马欣都在汜水边自刎而死。大司马曹咎,原本是蕲县狱掾,长史司马欣原本也是栎阳狱吏,两人曾经对项梁有恩德,所以项王信任他们。在这个时候,项王正在睢阳,听说海春侯的军队战败,就领兵返回。汉军正在荥阳围攻钟离眛,项王赶到,汉军害怕楚军,就全部撤到地势险峻之处了。

\begin{yuanwen}
是时,汉兵盛食多,项王兵罢食绝\footnote{text}。汉遣陆贾说项王,请太公,项王弗听。汉王复使侯公往说项王,项王乃与汉约,中分天下,割鸿沟以西者为汉\footnote{text},鸿沟而东者为楚。项王许之,即归汉王父母妻子。军皆呼万岁。

汉王乃封侯公为平国君。匿弗肯复见。曰:“此天下辩士,所居倾国,故号为平国君。”

项王已约,乃引兵解而东归\footnote{text}。
\end{yuanwen}

这时,汉军兵多粮足,项王兵疲粮绝。汉王派陆贾劝说项王,请求释放太公,项王不答应。汉王又派侯公前去劝说项王,项王就和汉王约定,平分天下,把鸿沟以西的土地划归汉国,鸿沟以东的土地属于楚国。项王同意了,立即送回汉王的父母妻儿。汉军都高呼万岁。

汉王于是封侯公为平国君,让他躲藏起来不肯再见面,说:“这个人是天下善辩之士,他所到的国家会因他而倾覆,因此封号为平国君。”

项王已经达成约定,就领兵撤退返回东方。

\begin{yuanwen}
汉欲西归,张良、陈平说曰:“汉有天下太半,而诸侯皆附之。楚兵罢食尽,此天亡楚之时也,不如因其机而遂取之。今释弗击,此所谓‘养虎自遗患’也。”

汉王听之。
\end{yuanwen}

汉王想要向西撤回,张良、陈平劝他说:“汉国占有大半个天下,而且诸侯都归附了我们。楚国兵疲粮尽,这是上天灭亡楚国的时机,不如抓住这个机会而彻底消灭楚国。现在放走敌人不去攻打,这就是人们所说的‘养虎为患’。”

汉王听从了他们的建议。

王世贞:「《垓下歌》正不必以虞兮为嫌悲壮呜咽,与《大风》各自描画帝王兴衰气象。千载而下,惟孟德『山不厌高』、『老骥伏枥』,仲达『天地开闢,日月重光』语差可嗣响。」

\begin{yuanwen}
汉五年,汉王乃追项王至阳夏南,止军,与淮阴侯韩信、建成侯彭越期会而击楚军。至固陵,而信、越之兵不会。楚击汉军,大破之。汉王复入壁,深堑而自守。谓张子房\footnote{张良,字子房。}曰:“诸侯不从约,为之奈何?”

对曰:“楚兵且破,信、越未有分地,其不至固宜。君王能与共分天下,今可立致也。即不能,事未可知也。君王能自陈以东傅海,尽与韩信;睢阳以北至穀城,以与彭越:使各自为战,则楚易败也。”

汉王曰:“善。”于是乃发使者告韩信、彭越曰:“并力击楚。楚破,自陈以东傅海与齐王,睢阳以北至穀城与彭相国。”

使者至,韩信、彭越皆报曰:“请今进兵。”

韩信乃从齐往,刘贾军从寿春并行,屠城父,至垓下。大司马周殷叛楚,以舒屠六,举九江兵,随刘贾、彭越皆会垓下,诣项王。
\end{yuanwen}

汉五年(前202年),汉王追击项王到阳夏以南,军队停下驻扎,与淮阴侯韩信、建成侯彭越约定日期一起进攻楚军。大军行进到固陵,可是韩信、彭越的军队没有赶来会合。楚军攻打汉军,大破汉军。汉王又进入营垒,深挖堑壕固守不出。汉王对张子房说:“诸侯不遵守约定,怎么办呢?”

张良回答说:“楚军就要失败了,韩信、彭越还没有得到封地,他们不来也是符合情理的。君王能和他们共分天下,现在他们立刻就会前来。假如不能这样做,以后的事情就不清楚了。君王能把从陈县以东到海边的土地都封给韩信,把从睢阳以北到穀城的土地都封给彭越,让他们各为自己的封地去作战,那么楚军就很容易打败了。”

汉王说:“好。”于是他就派使者前去告诉韩信、彭越说:“我们合力攻打楚军。楚军被打败后,从陈县以东到海边的土地都封给齐王,从睢阳以北到穀城的土地都封给彭相国。”

使者一到,韩信、彭越都回报说:“请让我们立刻进兵。”

韩信于是从齐国出发,刘贾的军队从寿春出发,与其一起前进,屠戮城父,行进到垓下。大司马周殷反叛楚国,率领舒县的士兵屠戮六县,发动九江的士兵,跟随刘贾、彭越一起在垓下会合,来到项王阵前。

\begin{yuanwen}
项王军壁垓下,兵少食尽\footnote{text},汉军及诸侯兵围之数重。夜闻汉军四面皆楚歌,项王乃大惊曰:“汉皆已得楚乎?是何楚人之多也!”

项王则夜起,饮帐中。有美人名虞\footnote{text},常幸从;骏马名骓\footnote{zhuī},常骑之。于是项王乃悲歌慷慨,自为诗曰:“力拔山兮气盖世,时不利兮骓不逝。骓不逝兮可奈何?虞兮虞兮奈若何!”

歌数阕\footnote{量词,歌曲一首或一段为一阕。},美人和之。项王泣数行下,左右皆泣,莫能仰视。
\end{yuanwen}

项王的军队在垓下修筑营垒,兵少粮尽,汉军和诸侯的军队将其重重包围。夜晚听见四周的汉军都在唱楚地的歌曲,项王于是非常吃惊地说:“汉军已经占领整个楚国了吗?为什么楚国人这么多?”

项王就在夜里起身,来到营帐中饮酒。有一个美人名叫虞姬,因为受宠而经常跟随项王行军;有一匹骏马名叫骓,项王经常骑着它作战。于是项王情绪激愤地唱起悲伤的歌,自己作诗吟唱道:“力能拔山啊,豪气盖世;时运不济啊,骓不能至。骓不能至啊,该怎么办?虞姬啊虞姬,该怎么办!”

歌唱了几段,虞美人应和着。项王流下数行眼泪,身边的侍从都流下眼泪,没有人能抬起头来。

\begin{yuanwen}
于是项王乃上马骑\footnote{text},麾下壮士骑从者八百馀人,直夜溃围南出\footnote{text},驰走。平明,汉军乃觉之,令骑将灌婴以五千骑追之。项王渡淮,骑能属者百馀人耳\footnote{text}。项王至阴陵\footnote{text},迷失道,问一田父,田父绐\footnote{dài,欺骗。}曰“左”。左,乃陷大泽中,以故汉追及之\footnote{text}。项王乃复引兵而东,至东城\footnote{text},乃有二十八骑。汉骑追者数千人。

项王自度不得脱。谓其骑曰:“吾起兵至今八岁矣,身七十馀战\footnote{text},所当者破,所击者服,未尝败北,遂霸有天下。然今卒困于此,此天之亡我,非战之罪也。今日固决死,愿为诸君快战\footnote{text},必三胜之,为诸君溃围,斩将,刈\footnote{yì}旗。令诸君知天亡我,非战之罪也。”

乃分其骑以为四队,四乡\footnote{text}。汉军围之数重。项王谓其骑曰:“吾为公取彼一将。”令四面骑驰下,期山东为三处。于是项王大呼驰下,汉军皆披靡\footnote{text},遂斩汉一将。

是时,赤泉侯\footnote{杨喜在杀死项羽后得到的封号。}为骑将,追项王,项王瞋目而叱之,赤泉侯人马俱惊,辟易\footnote{躲避。}数里,与其骑会为三处。汉军不知项王所在,乃分军为三,复围之。项王乃驰,复斩汉一都尉,杀数十百人,复聚其骑,亡其两骑耳。乃谓其骑曰:“何如?”

骑皆伏\footnote{text}曰:“如大王言。”
\end{yuanwen}

于是项王就骑上马,部下跟随他的精壮骑兵有八百多人,趁着夜色向南冲破包围圈,飞奔而去。天亮以后,汉军才发觉,命令骑兵将领灌婴率领五千名骑兵追击。项王渡过淮水,能够跟上的骑兵只有一百多人了。项王到达阴陵,迷失去了方向,询问一个农夫,农夫欺骗他说:“往左走。”项王向左逃去,却陷入一大片沼泽地。因为这个缘故汉军才追上他们。项王就又领兵向东逃去,到达东城,只剩下二十八名骑兵了。汉军前来追击的骑兵有几千人。

项王估计自己不能脱身了,对他的骑兵说:“我起兵到现在已经八年了,亲身经历过七十多次战斗,抵挡我的都被打败,我进攻的全部臣服,还没有战败过,所以能够称霸天下。然而现在我被困在这里,这是上天想要我灭亡,并不是我作战时犯了错误。今天本来就要决一死战,希望为各位痛快地打一仗,一定会连胜三次,为各位冲破包围圈,斩杀敌将,砍倒敌旗,让各位知道是上天想要我灭亡,并不是我作战时犯了错误。”

项王于是将他的骑兵分成四队,面朝四方。汉军将其重重包围。项王对他的骑兵说:“我为各位取来一个敌将的性命。”项王命令骑兵向四个方向冲出去,约定在山以东分三处会合。于是项王大声呐喊着冲了过去,所到之处的汉军全部倒下,终于斩杀一个汉军将领。

这个时候,赤泉侯担任骑兵将领,追击项王,项王瞪着眼睛向他呵叱,赤泉侯连人带马都很震惊,退避了好几里。项王和他的骑兵在约定的三处会合。汉军不知道项王在哪里,就把军队一分为三,又将项王包围起来。项王于是骑马飞奔,又斩杀一个汉军的都尉,杀死了几十上百人,再把他的骑兵召集到一起,只死掉两名骑兵罢了。项王就对他的骑兵说:“怎么样?”

骑兵都伏在地上说:“就像大王所说的一样。”

\begin{yuanwen}
于是项王乃欲东渡乌江\footnote{text}。乌江亭长舣船\footnote{把船靠岸。}待\footnote{text},谓项王曰:“江东虽小,地方千里,众数十万人,亦足王也。愿大王急渡。今独臣有船,汉军至,无以渡。”

项王笑曰:“天之亡我,我何渡为!且籍与江东子弟八千人渡江而西,今无一人还。纵江东父兄怜而王我,我何面目见之?纵彼不言,籍独不愧于心乎\footnote{text}?”

乃谓亭长曰:“吾知公长者。吾骑此马五岁,所当无敌,尝一日行千里,不忍杀之,以赐公。”

乃令骑皆下马步行,持短兵接战。独籍所杀汉军数百人。项王身亦被十馀创。顾见汉骑司马吕马童,曰:“若非吾故人乎?”

马童面\footnote{通“偭”,背对。}之,指王翳\footnote{text}曰:“此项王也。”

项王乃曰:“吾闻汉购\footnote{悬赏。}我头千金,邑万户,吾为若德。”乃自刎而死\footnote{text}。

王翳取其头,馀骑相蹂践争项王,相杀者数十人。最其后,郎中骑杨喜,骑司马吕马童,郎中吕胜、杨武各得其一体\footnote{text}。五人共会其体,皆是。故分其地为五:封吕马童为中水侯,封王翳为杜衍侯,封杨喜为赤泉侯,封杨武为吴防侯,封吕胜为涅\footnote{niè}阳侯。
\end{yuanwen}

于是项王想要向东渡过乌江。乌江亭长将船靠在岸边等待,对项王说:“江东虽然很小,但是土地也有方圆千里,民众也有几十万人,足以在那里称王。希望大王赶快渡江。现在只有我这一条船,汉军来到这里,也没有办法渡江。”

项王笑着说:“上天要我灭亡,我渡江又能怎么样呢!况且我和江东子弟八千人渡江向西进军,现在没有一个人回来,即使江东的父兄可怜我,让我在那里称王,我又有什么脸面去见他们呢?即使他们不说什么,我自己难道心里不愧疚吗?”

于是他对亭长说:“我知道您是个忠厚的人。我骑这匹马五年了,它所向无敌,曾经一天行进千里,我不忍心杀死它,把它赐给您了。”

他命令骑兵都下马步行,手持短兵器迎战。项籍独自杀死汉军几百人。项王自己也受伤十多处。他回头看见汉军的骑司马吕马童,说:“你不是我的老朋友吗?”

吕马童转过身,指给王翳说:“这就是项王。”

项王于是说:“我听说汉王悬赏用一千金、一万户封邑来得到我的头,我成全你。”于是他自刎而死。

王翳割下他的头,其他骑兵相互踩踏着争抢项王的尸体,自相残杀死了几十人。最后,郎中骑杨喜,骑司马吕马童,郎中吕胜、杨武,各自得到了项王的一部分肢体。五个人把肢体合起来,都是项王的。所以汉王把悬赏的封地一分为五:封吕马童为中水侯,封王翳为杜衍侯,封杨喜为赤泉侯,封杨武为吴防侯,封吕胜为涅阳侯。

\begin{yuanwen}
项王已死,楚地皆降汉,独鲁不下。汉乃引天下兵欲屠之。为其守礼义,为主死节\footnote{text},乃持项王头视鲁\footnote{text},鲁父兄乃降。始,楚怀王初封项籍为鲁公,及其死,鲁最后下,故以鲁公礼葬项王穀(谷)城。汉王为发哀,泣之而去。
\end{yuanwen}

项王死后,楚地都投降汉王,只有鲁县不投降。汉王于是率领天下军队想要屠戮鲁县,因为当地人持守礼义,愿意为君主守节而死,就派人拿出项王的头给鲁县的人看,鲁县的父兄才投降。最初,楚怀王封项籍为鲁公,等到他死后,鲁县又最后投降,因此用鲁公的礼仪将他埋葬在穀城。汉王为其致哀,哭泣后离去。

\begin{yuanwen}
诸项氏枝属,汉王皆不诛。乃封项伯为射阳侯。桃侯、平皋侯、玄武侯皆项氏,赐姓刘。
\end{yuanwen}

项氏的各支属,汉王都没有诛杀。汉王封项伯为射阳侯。桃侯、平皋侯、玄武侯都是项氏,赐姓刘。

\begin{yuanwen}
太史公曰:吾闻之周生\footnote{西汉儒生,其名不详。}曰“舜目盖重瞳子\footnote{眼睛有两个瞳孔。}”,又闻项羽亦重瞳子。羽岂其苗裔邪?何兴之暴也\footnote{text}!夫秦失其政,陈涉首难,豪杰蜂起,相与并争,不可胜数。然羽非有尺寸\footnote{即使很小的封地也没有。},乘势起陇亩\footnote{田垄,代指民间。}之中,三年遂将五诸侯灭秦,分裂天下,而封王侯,政由羽出,号为“霸王”,位虽不终,近古以来未尝有也\footnote{text}。及羽背关怀楚\footnote{text},放逐义帝而自立,怨王侯叛己,难矣。自矜功伐,奋其私智而不师古,谓霸王之业,欲以力征经营天下,五年卒亡其国,身死东城\footnote{text},尚不觉寤而不自责,过矣。乃引“天亡我,非用兵之罪也”,岂不谬哉!
\end{yuanwen}

太史公说:我听周生说“舜的眼睛大概有两个瞳孔”,又听说项羽也有两个瞳孔。项羽难道是舜的后裔吗?不然为什么会兴起得如此迅速呢?秦朝政治失策,陈涉率先起事,天下豪杰纷然响应,相互兼并争夺,不可胜数。然而项羽连很小的封地也没有,借助时势在民间发迹,三年后,就率领五路诸侯灭掉秦朝,分割天下,封立王侯,政令由他颁布,号称“霸王”,地位虽然没有保持到最后,但是近代以来也没有过这样的事情。等到项羽放弃关中思念楚地,放逐义帝自立为王,抱怨王侯背叛自己,情况已经很艰难了。项羽居功自傲,凭借个人才智而不仿效古法,认为自己建立了霸王基业,想要用武力治理天下,五年后就失去了他的国家,自己也死在东城,却还没有觉悟,不自我责备,这就是他的错了。他竟然说“上天想要我灭亡,不是我用兵时犯了错误”,难道不荒谬吗!

\begin{yuanwen}
(亡秦鹿走,伪楚狐鸣。云郁沛谷,剑挺吴城。勋开鲁甸,势合砀兵。卿子无罪,亚父推诚。始救赵歇,终诛子婴。违约王汉,背关怀楚。常迁上游,臣迫故主。灵壁大振,成皋久拒。战非无功,天实不与。嗟彼盖代,卒为凶竖。)
\end{yuanwen}

\part{卷八}
\chapter{高祖本纪第八}

本篇是汉高帝刘邦的本纪,记述了他一生的主要经历,尤其对他从出生到起兵时的种种异象作了渲染,以证明汉朝得天下的合法性。高帝的为人尽管颇受争议,不过他也算是一位乱世雄主。

吕思勉:「赤帝子之说,则又因高祖为沛公旗帜皆赤而附会,未必与行序有关。《史记》本纪言旗帜皆赤,由所杀蛇白帝子,杀者赤帝子,疑出后人增窜,非谈、迁原文也。」

\begin{yuanwen}
高祖,沛丰邑中阳里人,姓刘氏,字季\footnote{text}。父曰太公\footnote{text},母曰刘媪\footnote{ǎo}\footnote{太公、媪分别为对老年男女的敬称,并非刘邦父母之名。刘邦父母不参与政治活动,又因为避讳,所以文献中没有留下他们的名字。}。其先刘媪尝息大泽之陂\footnote{水边。},梦与神遇。是时雷电晦冥\footnote{text},太公往视,则见蛟龙于其上。已而有身,遂产高祖\footnote{text}。
\end{yuanwen}

高祖是沛县丰邑中阳里人,姓刘氏,字季。他的父亲叫太公,母亲叫刘媪。早先刘媪曾在大湖的岸边休息,在梦里和水神相遇。当时电闪雷鸣,天色昏暗,太公前去察看,就发现一条蛟龙趴在刘媪身上。不久她就怀有身孕,于是生下了高祖。

\begin{yuanwen}
高祖为人,隆准\footnote{高鼻梁。}而龙颜\footnote{额头丰满。},美须髯\footnote{rán},左股有七十二黑子。仁而爱人,喜施,意豁如也\footnote{text}。常有大度,不事家人生产作业\footnote{text}。及壮,试为吏,为泗水亭长,廷中吏无所不狎侮\footnote{不庄重的亲近。}。好酒及色。常从王媪、武负\footnote{姓武的老妇。负,通“媍”。}贳\footnote{shì,赊欠。}酒,醉卧,武负、王媪见其上常有龙,怪之。高祖每酤\footnote{gū}留饮,酒雠\footnote{出售。}数倍\footnote{text}。及见怪,岁竟,此两家常折券弃责\footnote{zhài,同“债”。}。
\end{yuanwen}

高祖的相貌,有着挺拔的鼻梁和丰满的额头,留着漂亮的胡须,左大腿长着七十二颗黑痣。他仁厚爱人,喜欢施舍,胸襟广阔。他时常有远大的抱负,不愿从事普通百姓的生产劳作。等到壮年之时,被试用做过小吏,担任泗水亭长,廷中的官吏没有人不和他熟识。他喜好饮酒和女色。他经常到王媪、武负的酒肆赊酒,醉了就躺下,武负、王媪看到他的上方经常有一条龙,感到奇怪。高祖每次来到酒肆中喝酒,酒的销量都要比平时多出几倍。等到她们发觉了这些奇怪的现象,年终的时候,这两家酒肆经常毁掉债券免除他的酒钱。

\begin{yuanwen}
高祖常繇\footnote{服徭役。}咸阳\footnote{text},纵观\footnote{text},观秦皇帝,喟\footnote{kuì}然太息\footnote{text}曰:“嗟乎,大丈夫当如此也!”
\end{yuanwen}

高祖曾经前往咸阳去服徭役,随意观看,见到秦始皇帝,就长叹一声说:“唉,大丈夫就应该是这个样子!”

\begin{yuanwen}
单父人吕公善沛令,避仇从之客\footnote{text},因家沛焉。沛中豪桀吏闻令有重客\footnote{text},皆往贺。萧何为主吏\footnote{text},主进\footnote{接收客人进献的礼物。},令诸大夫\footnote{text}曰:“进不满千钱,坐之堂下。”

高祖为亭长,素易\footnote{轻视。}诸吏\footnote{text},乃绐\footnote{欺骗。}为谒\footnote{拜访别人时所用的名帖,上面书写姓名等个人信息,以及拜访原因等说明文字,多为木片、竹片制成。}曰“贺钱万”\footnote{text},实不持一钱。谒入,吕公大惊,起,迎之门。吕公者,好相人,见高祖状貌,因重敬之,引入坐。萧何曰:“刘季固多大言,少成事。”

高祖因狎侮诸客,遂坐上坐,无所诎\footnote{谦让。}。酒阑\footnote{尽,结束。},吕公因目固留高祖\footnote{text}。高祖竟酒,后。吕公曰:“臣少好相人\footnote{text},相人多矣,无如季相,愿季自爱\footnote{text}。臣有息女\footnote{亲生女儿。},愿为季箕帚妾\footnote{text}。”

酒罢,吕媪怒吕公曰:“公始常欲奇此女\footnote{text},与贵人。沛令善公,求之不与,何自妄许与刘季?”

吕公曰:“此非儿女子所知也\footnote{text}。”卒与刘季。

吕公女乃吕后也,生孝惠帝、鲁元公主\footnote{text}。
\end{yuanwen}

单父人吕公和沛县令关系很好,为躲避仇家而投奔县令做宾客,就迁居到沛县。沛县的豪杰和官吏听说县令家来了贵客,都前往道贺。萧何是主吏,负责接收礼物,他对各位贵客说:“进献的礼钱不到一千钱的,坐在堂下。”

高祖担任亭长,一向看不起众官吏,就欺骗性地在名贴上写“贺礼一万钱”,实际上他没有带一个钱。名帖递进去后,吕公很惊讶,起身,到门口去迎接。吕公这个人,喜好为人相面,见到高祖的相貌,就非常敬重他,带他到堂上坐下。萧何说:“刘季向来爱说大话,很少做成事。”

高祖因而戏弄堂上众宾客,就自己坐到上座,毫不谦让。酒席结束,吕公趁机使眼色要求高祖留下来。高祖喝完酒,故意最后走。吕公说:“我年轻的时候就喜好为人相面,看过的人也有很多了,没有像你刘季这样的,希望你能自重。我有个亲生女儿,愿意许配给你做个打扫屋子的妻妾。”

酒席散后,吕媪生气地对吕公说:“您当初总是想让女儿与众不同,把她许配给贵人。沛县令对您很好,求娶女儿却不答应,为什么擅自把她许配给刘季?”

吕公说:“这不是妇孺之辈所能够理解的。”最终他还是将女儿许配给了刘季。

吕公的女儿就是吕后,生下了孝惠帝、鲁元公主。

\begin{yuanwen}
高祖为亭长时,常告归之田。吕后与两子居田中耨\footnote{nòu,锄草。},有一老父过请饮,吕后因餔\footnote{请吃饭。}之。老父相吕后曰:“夫人天下贵人。”

令相两子,见孝惠,曰:“夫人所以贵者,乃此男也。”

相鲁元,亦皆贵。老父已去,高祖適从旁舍来,吕后具言客有过,相我子母皆大贵。高祖问,曰:“未远。”

乃追及,问老父。老父曰:“乡者\footnote{刚才。乡,同“向”。}夫人婴儿皆似君,君相贵不可言。”

高祖乃谢曰:“诚如父言,不敢忘德。”及高祖贵,遂不知老父处。

高祖为亭长,乃以竹皮为冠,令求盗\footnote{负责缉拿盗贼的亭卒。}之薛治之,时时冠之,及贵常冠,所谓“刘氏冠”乃是也。
\end{yuanwen}

高祖做亭长时,经常请假回到田里。吕后和两个孩子在田里锄草,有一个老人路过讨水喝,于是吕后请他吃了一顿饭。老人为吕后相面说:“夫人是天下的贵人。”

吕后让他为两个孩子相面。老人见到孝惠帝,说:“夫人能够富贵的原因,就在于这个孩子。”

他为鲁元公主相面,也说是富贵之人。老人离开以后,高祖恰好从旁边的屋子过来,吕后就说有客人路过这里,为母子相面后说都是富贵之人。高祖就问老人在哪里,吕后说:“还没有走远。”

高祖就追上去,询问老人。老人说:“刚才看过夫人和孩子的面相都像您一样富贵,您的面相富贵不可言说。”

高祖就道谢说:“真的像老人家所说的那样,我绝不会忘记恩德。”等到高祖富贵以后,已经不知道老人在哪里了。

高祖担任亭长,就用竹皮制作冠帽,命令求盗前往薛县制作,他经常戴在头上,等到富贵时也经常戴着,就是人们所说的“刘氏冠”。

\begin{yuanwen}
高祖以亭长为县送徒郦山\footnote{text},徒多道亡\footnote{text}。自度比至皆亡之\footnote{text},到丰西泽中,止饮,夜乃解纵所送徒\footnote{text}。曰:“公等皆去,吾亦从此逝\footnote{text}矣!”

徒中壮士愿从者十馀人。高祖被酒\footnote{text},夜径泽中\footnote{text},令一人行前。行前者还报曰:“前有大蛇当径,愿还。”

高祖醉,曰:“壮士行,何畏!”乃前,拔剑击斩蛇,蛇遂分为两,径开。

行数里,醉,因卧。后人来至蛇所,有一老妪夜哭。人问何哭,妪曰:“人杀吾子,故哭之。”

人曰:“妪子何为见杀?”

妪曰:“吾子,白帝子\footnote{text}也,化为蛇,当道,今为赤帝子斩之\footnote{text},故哭。”

人乃以妪为不诚,欲笞\footnote{ch\=i}(告\footnote{有版本为告,因其不诚而告发,未免小题大做。《汉书·高帝纪》作“苦”,即让老妇吃点苦头,比较合理。《汉书·高帝纪》材料主要取自《史记·高祖本纪》,所以此处从《汉书》。})之,妪因忽不见。后人至,高祖觉。后人告高祖,高祖乃心独喜,自负\footnote{text}。诸从者日益畏之。
\end{yuanwen}

高祖担任亭长时为县里押送刑徒前往郦山,很多刑徒在路上逃走了。他心里估计到郦山时就都逃光了,走到丰邑的沼泽,停下来饮酒,在夜里把他所押送的刑徒全部释放。高祖说:“各位都走吧,从此以后我也不回去了!”

刑徒里有十几个壮士愿意跟随他。高祖带着酒意,在夜里径直穿过沼泽,让一个人在前面探路。走在前面探路的人回来报告说:“前方有一条大蛇挡在路上,我想回去。”

高祖喝醉了,说:“壮士走路,怕什么!”于是他上前,拔剑砍蛇。蛇就被砍成两段,道路通畅了。

走了几里,高祖因为醉酒,就地躺下。后面的人走到大蛇出现的地方,见到一个老妇在夜里哭泣。人们就问她,老妇说:“有人杀死了我的儿子,因此我在这里哭。”

人们就说:“老妇的儿子为什么被杀?”

老妇说:“我的儿子,其实是白帝的儿子,变化为蛇,挡在路上,今天被赤帝的儿子砍为两段,因此我在这里哭。”

人们都认为老妇不诚实,想要让她吃点苦头,老妇却忽然不见了。后面的人赶上来,高祖醒了。他们告诉高祖这些事,高祖心里暗自高兴,自以为了不起。众追随者越来越敬畏他。

\begin{yuanwen}
秦始皇帝常曰“东南有天子气”,于是因东游以厌\footnote{镇压。}之。高祖即自疑,亡匿,隐于芒、砀\footnote{dàng}山泽岩石之间。吕后与人俱求,常得之。高祖怪问之。吕后曰:“季所居上常有云气,故从往常得季\footnote{text}。”高祖心喜。沛中子弟或闻之,多欲附者矣。
\end{yuanwen}

秦始皇帝经常说“东南方有天子之气”,当时就巡游东方来镇压那里的天子之气。高祖怀疑和自己有关,就逃走藏匿起来,隐居在芒砀山的湖泽与岩石之间。吕后和别人一同寻找,总是能找到他。高祖感到奇怪而问吕后原因。吕后说:“你所在之处的上方总有云气聚集,所以跟着云气走就会找到你。”高祖心里很高兴。沛县的子弟有人听说了这件事,大多想要追随他。

顾颉刚:「在新的五德终始系统中,汉之不得不为火,正如在旧的系统中汉之不得不为土一样。黄龙见于成纪的符瑞,是公孙臣主张汉为土德之后才出现的。那么,神母夜号的符瑞,自然应当待刘向父子发明了汉为火德的主张之后才出现,可以无疑也。」

\begin{yuanwen}
秦二世元年秋\footnote{text},陈胜等起蕲\footnote{qí},至陈而王\footnote{text},号为“张楚\footnote{text}”。诸郡县皆多杀其长吏以应陈涉\footnote{陈胜,字涉。}。沛令恐,欲以沛应涉。掾\footnote{曹参任狱掾,因此“掾”字应在“曹参”前。}、主吏萧何、曹参乃曰:“君为秦吏,今欲背之,率沛子弟,恐不听。愿君召诸亡在外者,可得数百人,因劫众,众不敢不听。”乃令樊哙召刘季。刘季之众已数十百人\footnote{text}矣。
\end{yuanwen}

秦二世元年(前209年)秋季,陈胜等人在蕲县起兵,来到陈县后自立为王,号称“张楚”。很多郡县的百姓杀死当地长官来响应陈涉。沛县令很害怕,想要在沛县起兵响应陈涉。主吏萧何、狱掾曹参于是说:“您担任秦朝的官吏,现在却想要背叛秦朝,率领沛县的子弟起兵,恐怕他们不会听从。希望您召回那些逃亡在外的人,能够得到几百人,利用他们来胁迫众人,众人不敢不听从。”于是县令就命令樊哙去召回刘季。刘季的部众已经有几十上百人了。

\begin{yuanwen}
于是樊哙从刘季来。沛令后悔,恐其有变,乃闭城城守,欲诛萧、曹。萧、曹恐,逾城保刘季\footnote{text}。刘季乃书帛射城上,谓沛父老曰:“天下苦秦久矣。今父老虽为沛令守,诸侯并起,今屠沛\footnote{text}。沛今共诛令,择子弟可立者立之,以应诸侯,则家室完。不然,父子俱屠,无为\footnote{徒劳,指白送命。}也。”

父老乃率子弟共杀沛令,开城门迎刘季,欲以为沛令。刘季曰:“天下方扰,诸侯并起,今置将不善,壹败涂地。吾非敢自爱,恐能薄,不能完父兄子弟。此大事,愿更相推择可者\footnote{text}。”

萧、曹等皆文吏,自爱\footnote{text},恐事不就,后秦种族\footnote{灭族。}其家,尽让刘季。诸父老皆曰:“平生所闻刘季诸珍怪\footnote{text},当贵,且卜筮之,莫如刘季最吉。”

于是刘季数让。众莫敢为,乃立季为沛公。祠黄帝\footnote{text},祭蚩尤于沛庭\footnote{text},而衅\footnote{杀牲以血祭新制的器物。}鼓旗\footnote{text},帜皆赤。由所杀蛇白帝子,杀者赤帝子,故上\footnote{同“尚”,崇尚。}赤。于是少年豪吏如萧、曹、樊哙等皆为收沛子弟二三千人,攻胡陵、方与,还守丰。
\end{yuanwen}

这时樊哙跟随刘季回来。沛县令后悔了,担心发生变故,就关闭城门防守,想要诛杀萧何、曹参。萧何、曹参很害怕,就翻越城墙投靠刘季自保。刘季在丝帛上写了一封信射入城中,告知沛县的父老说:“天下人深受秦朝暴政的折磨已经很久了。现在父老虽然为沛县令守城,但是诸侯都起兵反抗秦朝,就要屠戮沛县了。现在沛县的父老一起诛杀县令,选出可以担当首领的子弟推举他,来响应诸侯,身家性命才能保全。否则,父子都会被屠杀,那是白送命。”

父老就率领子弟一起杀死了沛县令,打开城门迎进刘季,想要让他做沛县令。刘季说:“天下正陷入混乱,诸侯都起兵反抗秦朝,现在选出的将领不称职,就一败涂地了。我不敢吝惜自己的性命,只怕自己才能浅薄,不能保全父兄子弟的性命。这是大事,希望另选一位合适的人。”

萧何、曹参等人都是文吏,吝惜自己的性命,害怕事情不成功,过后秦朝会将他们灭族,都推举刘季。众父老都说:“我们平时就常听说刘季的那些奇异之事,刘季必当得到富贵,况且经过占卜,没有人比刘季更吉利。”

于是刘季多次谦让。众人都不敢担当,就推举刘季为沛公。众人在沛县官署的庭院祭祀黄帝、蚩尤,把牲畜的血涂在鼓旗上,旗帜都染成红色。由于所杀的蛇是白帝之子,杀蛇的人是赤帝之子,因此崇尚红色。这时沛县少年和有权势的官吏如萧何、曹参、樊哙等人都替沛公召集子弟,得到二三千人,一起攻打胡陵、方与,退守丰邑。

\begin{yuanwen}
秦二世二年,陈涉之将周章军西至戏而还。燕、赵、齐、魏皆自立为王。项氏起吴。秦泗川监平\footnote{泗川郡御史,名叫平,其姓不详。御史监察一郡政务,因此又称监。据史学界对西安市北郊秦宫遗址出土的封泥研究得出结论,泗川郡应为四川郡,《汉书》更是讹为泗水郡。汉武帝元鼎四年(前113年)分东海郡三万户封刘商为泗水王,光武帝建武二年(26年)封刘歙为泗水王,刘歙死后废国设郡。可知四川与泗水不是同一郡。川、水二字篆、隶字体相近,可能是讹误的主要原因。}将兵围丰,二日,出与战,破之。命雍齿守丰,引兵之薛。泗州(川)守壮败于薛,走至戚,沛公左司马得泗川守壮,杀之。沛公还军亢父,至方与,未战。陈王使魏人周市略地。周市使人谓雍齿曰:“丰,故梁徙也。今魏地已定者数十城。齿今下魏,魏以齿为侯守丰。不下,且屠丰。”

雍齿雅不欲属沛公,及魏招之,即反为魏守丰。沛公引兵攻丰,不能取。沛公病,还之沛。沛公怨雍齿与丰子弟叛之,闻东阳甯(宁)君、秦嘉立景驹为假王,在留,乃往从之,欲请兵以攻丰。是时秦将章邯从陈,别将司马枿将兵北定楚地,屠相,至砀。东阳甯君、沛公引兵西,与战萧西,不利。还收兵聚留,引兵攻砀,三日乃取砀。因收砀兵,得五六千人。攻下邑,拔之。还军丰。
\end{yuanwen}

秦二世二年(前208年),陈涉的将领周章带领军队向西来到戏水后返回。燕、赵、齐、魏都自立为王。项氏在吴地起兵。秦朝泗川郡御史平率领士兵包围丰邑,过了两天,沛公出城与其交战,打败了秦军。沛公命令雍齿镇守丰邑,自己领兵前往薛县。泗川郡守壮在薛县被打败,逃往戚县。沛公的左司马擒获泗川郡守壮,将他杀死。沛公回军亢父,来到方与,没有作战。陈王派魏人周巿攻取土地。周巿派人对雍齿说:“丰邑,大梁人曾迁徙到这里。现在魏地已经平定几十座城邑。你雍齿投降魏王,魏王就会封你为侯来镇守丰邑。要是不投降,我就要屠戮丰邑。”

雍齿向来不情愿跟随沛公,等到魏国招降他,立即反叛为魏国镇守丰邑。沛公领兵攻打丰邑,不能攻取。沛公生病了,返回沛县。沛公怨恨雍齿和丰邑的子弟背叛他,听说东阳宁君、秦嘉立景驹为代理楚王,在留县,就前去投靠他们,想要借兵去攻打丰邑。当时秦将章邯追击陈王,别将司马 率领士兵向北平定楚地,屠戮相县,来到砀县。东阳宁君、沛公领兵向西进发,与其在萧县以西交战,不能取胜。沛公回军收拢残兵聚集在留县,领兵攻打砀县,三天就攻占了砀县。他顺势收编了砀县的士兵,得到五六千人。攻打下邑,夺取那里。回军丰邑。

\begin{yuanwen}
闻项梁在薛,从骑百馀往见之。
\end{yuanwen}

刘邦听说项梁已经到了薛县,就带着一百多个随从去拜见他。

\begin{yuanwen}
项梁益沛公卒五千人,五大夫\footnote{秦二十等爵的第九级。}将十人。沛公还,引兵攻丰。

从项梁月馀,项羽已拔襄城还。项梁尽召别将居薛。闻陈王定死,因立楚后怀王孙心为楚王,治盱\footnote{xū}台\footnote{yí}。项梁号武信君。居数月,北攻亢父,救东阿,破秦军。齐军归,楚独追北,使沛公、项羽别攻城阳,屠之。军濮阳之东,与秦军战,破之。

秦军复振,守濮阳,环水\footnote{掘开堤坝以水环城。}。楚军去而攻定陶,定陶未下。沛公与项羽西略地至雍丘之下,与秦军战,大破之,斩李由。还攻外黄,外黄未下。
\end{yuanwen}

项梁增援士兵五千人给沛公,爵位为五大夫的将领有十人。沛公返回,领兵攻打丰邑。

沛公跟随项梁一个多月,项羽已经攻下襄城返回了。项梁召集各路将领来到薛县。听到陈王的确死了,就立楚国后代楚怀王裔孙熊心为楚王,定都盱台。项梁自称武信君。过了几个月,楚军向北进攻亢父,援救东阿,打败秦军。齐军返回齐国,楚军独自追击败军,派沛公、项羽另外率领军队攻打城阳,屠戮那里。沛公、项羽驻扎在濮阳以东,和秦军交战,打败了敌人。

秦军重新振作,坚守濮阳,以水环绕城池。楚军离去,转而攻打定陶,定陶没有攻下。沛公和项羽向西攻取土地到雍丘城下,和秦军交战,大破秦军,斩杀李由。回军攻打外黄,外黄没有攻下。

\begin{yuanwen}
项梁再破秦军\footnote{text},有骄色。宋义谏,不听。秦益章邯兵,夜衔枚\footnote{偷袭时衔木棍于口中,以防士兵说话。}击项梁\footnote{text},大破之定陶,项梁死。沛公与项羽方攻陈留,闻项梁死,引兵与吕将军俱东\footnote{text}。吕臣军彭城东,项羽军彭城西,沛公军砀。
\end{yuanwen}

项梁再次击败秦军,露出骄傲的神色。宋义劝诫,他不听。秦朝派兵增援章邯,在夜里偷袭项梁,在定陶大破项梁,项梁战死。沛公和项羽正在攻打陈留,听说项梁战死,就领兵与吕将军一同向东撤退。吕臣驻扎在彭城以东,项羽驻扎在彭城以西,沛公驻扎在砀县。

\begin{yuanwen}
章邯已破项梁军,则以为楚地兵不足忧,乃渡河,北击赵,大破之。当是之时,赵歇为王,秦将王离围之钜鹿城,此所谓河北之军也。
\end{yuanwen}

章邯打败项梁的军队以后,就认为楚地的军队不值得忧虑了,于是渡过黄河,向北攻打赵国,大败赵军。在这个时候,赵歇是赵王,秦将王离把他围困在钜鹿城,这就是人们所说的河北之军。

\begin{yuanwen}
秦二世三年\footnote{text},楚怀王见项梁军破\footnote{text},恐,徙盱眙(台)都彭城\footnote{text},并吕臣、项羽军自将之\footnote{text}。以沛公为砀郡长,封为武安侯,将砀郡兵。封项羽为长安侯,号为鲁公。吕臣为司徒,其父吕青为令尹。
\end{yuanwen}

秦二世三年(前207年),楚怀王见项梁的军队战败,很害怕,迁离盱台定都彭城,合并吕臣、项羽的军队亲自率领。他任命沛公为砀郡长,封为武安侯,率领砀郡的士兵。封项羽为长安侯,号称鲁公。吕臣担任司徒,他的父亲吕青担任令尹。

\begin{yuanwen}
赵数请救,怀王乃以宋义为上将军\footnote{text},项羽为次将,范增为末将,北救赵。令沛公西略地入关\footnote{text}。与诸将约,先入定关中者王\footnote{text}之。
\end{yuanwen}

赵国屡次请求救援,怀王就任命宋义为上将军,任命项羽为次将,任命范增为末将,向北进发救援赵国。他命令沛公向西攻取土地进入关中。他和众将领约定,先进入关中的人在关中称王。

\begin{yuanwen}
当是时,秦兵强,常乘胜逐北,诸将莫利先入关\footnote{}。独项羽怨秦破项梁军,奋\footnote{text},愿与沛公西入关。怀王诸老将皆曰:“项羽为人僄悍猾贼\footnote{凶悍残忍。}。项羽尝攻襄城,襄城无遗类,皆阬(坑)\footnote{text}之,诸所过无不残灭。且楚数进取\footnote{text},前陈王、项梁皆败。不如更遣长者扶义而西\footnote{text},告谕秦父兄。秦父兄苦其主久矣,今诚得长者往,毋侵暴,宜可下。今项羽僄悍,(今)不可遣。独沛公素宽大长者,可遣。”

卒不许项羽,而遣沛公西略地,收陈王、项梁散卒。乃道砀至成阳,与杠里秦军夹壁\footnote{对垒。},破秦二军。楚军出兵击王离\footnote{text},大破之。
\end{yuanwen}

在这个时候,秦军强大,经常乘胜追击败军,众将领没有人认为先入关中有利。只有项羽怨恨秦军打败项梁的军队,心情激愤,愿意和沛公向西进入关中。怀王的众老将都说:“项羽这个人凶悍残忍。项羽曾攻打襄城,襄城没有留下一个活口,都被坑杀,所到之处没有不被摧毁的。况且楚军多次进兵攻取土地,此前陈王、项梁都失败了。不如另派一个宽厚的人秉持正义向西进军,告谕秦地父老兄弟这些道理。秦地的父老兄弟被他们的君主折磨很久了,现在真能找到一个宽厚的人前去,不欺凌施暴,应该可以攻下关中。现在项羽十分凶悍,不能派他去。只有沛公一向是个宽厚的人,可以派他去。”

最后怀王没有答应项羽去,而是派沛公向西攻取土地,收拢陈王、项梁逃散的士兵。大军经过砀县抵达成阳,和杠里的秦军对垒,打败两支秦军。楚军派兵攻打王离,大破敌军。

\begin{yuanwen}
沛公引兵西,遇彭越昌邑,因与俱攻秦军,战不利。还至栗,遇刚武侯,夺其军,可四千馀人,并之。与魏将皇欣、魏申徒\footnote{官名,即司徒。}武蒲之军并攻昌邑,昌邑未拔。西过高阳。郦食\footnote{yì}其\footnote{jī}为监门\footnote{text},曰:“诸将过此者多,吾视沛公大人长者。”

乃求见,说沛公。沛公方踞床\footnote{叉开腿坐在床上。床是坐具,形似方凳,正式的坐姿是跪坐其上。踞即两腿叉开而坐,形如簸箕,所以又称箕踞,是一种不庄重的坐姿,因此郦食其说“不宜踞见长者”。jù。},使两女子洗足。郦生不拜,长揖\footnote{text},曰:“足下必欲诛无道秦,不宜踞见长者\footnote{text}。”

于是沛公起,摄衣谢\footnote{text}之,延上坐\footnote{text}。食其说沛公袭陈留,得秦积粟。
\end{yuanwen}

\begin{yuanwen}
乃以郦食其为广野君,郦商为将,将陈留兵,与偕攻开封\footnote{当时称启封,《史记》避汉景帝刘启讳,改启为开。},开封未拔。西与秦将杨熊战白马,又战曲遇东,大破之。杨熊走之荥阳,二世使使者斩以徇。南攻颍\footnote{yǐng}阳,屠之。因张良遂略韩地轘\footnote{huán}辕。
\end{yuanwen}

沛公领兵向西进发,在昌邑遇到彭越,顺势和他一同进攻秦军,作战没有取胜。回军来到栗县,遇到刚武侯,夺取他的军队,差不多有四千多人,两支军队合并。沛公和魏将皇欣、魏司徒武蒲的军队一起进攻昌邑,昌邑没有攻下。向西经过高阳。郦食其负责看守城门,说:“从这路过的将领有很多,我看沛公是个宽厚的大人物。”

于是他请求拜见劝说沛公。沛公正叉开腿坐在床上,让两个女子为他洗脚。郦生没有下拜,只是拱手高举行礼,说:“您一定想要诛灭无道的秦朝,就不应该叉开腿坐着接见长者。”

这时沛公站起来,整理衣装向他道歉,请他坐上座。郦食其劝说沛公袭击陈留,得到秦朝囤积的粮食。

沛公于是封郦食其为广野君,任命郦商为将军,率领陈留的军队,与他一起攻打开封,开封没有攻下。沛公向西和秦将杨熊在白马交战,又在曲遇以东交战,大破秦军。杨熊逃到荥阳,秦二世派使者将他斩首示众。沛公向南进攻颍阳,屠戮那里。依靠张良终于攻下韩地的轘辕关。

\begin{yuanwen}
当是时,赵别将司马卬方欲渡河入关,沛公乃北攻平阴,绝河津\footnote{text}。南,战雒阳东,军不利,还至阳城,收军中马骑\footnote{text},与南阳守齮战犨东,破之。略南阳郡,南阳守齮走,保城守宛\footnote{text}。沛公引兵过而西。张良谏曰:“沛公虽欲急入关,秦兵尚众,距险。今不下宛,宛从后击,强秦在前,此危道也。”

于是沛公乃夜引兵从他道还,更旗帜\footnote{text},黎明,围宛城三匝\footnote{text}。南阳守欲自刭,其舍人陈恢\footnote{text}曰:“死未晚也。”

乃逾城见沛公,曰:“臣闻足下约,先入咸阳者王之。今足下留守宛。宛,大郡之都\footnote{南阳郡治所在宛县。}也,连城数十,人民众,积蓄多,吏人自以为降必死,故皆坚守乘城。今足下尽日止攻,士死伤者必多;引兵去宛,宛必随足下后,足下前则失咸阳之约,后又有强宛之患。为足下计,莫若约降\footnote{text},封其守,因使止守,引其甲卒与之西。诸城未下者,闻声争开门而待,足下通行无所累\footnote{text}。”

沛公曰:“善。”乃以宛守为殷侯,封陈恢千户。引兵西,无不下者。
\end{yuanwen}

在这个时候,赵国别将司马卬正要渡过黄河进入关中,沛公向北进攻平阴,切断黄河渡口。向南进发,在雒阳以东交战,不能取胜,回军到阳城,集合军队中的战马和骑兵,与南阳郡守齮在犨县以东交战,击败敌军。攻取南阳郡的土地,南阳郡守齮逃跑,退守到宛县。沛公领兵绕过宛县向西进发。张良劝谏说:“沛公虽然想要尽快攻入关中,但是秦军人数还有很多,据守险要地带。现在不攻下宛县,宛县从后面进攻,强大的秦军挡在前方,这是危险的做法。”

于是沛公就在夜里领兵从另一条路返回,变换了旗帜,天快亮时,把宛县城围了三重。南阳郡守想要自杀。他的门客陈恢说:“等我回来再死也不算晚。”

他就翻越城墙去见沛公,说:“我听说您有约定,先进入咸阳的人可以在关中称王。现在您却留下围攻宛县。宛县,是大郡的治所,相连的城邑有几十座,人口众多,积蓄充足,官吏和百姓都认为投降一定会死,因此都登上城墙顽强防守。现在您整天留在这里攻城,士卒死伤的一定有很多;领兵离开宛县,宛县的军队一定会跟在您的后面,您向前就不能履行先入咸阳的约定,向后又有强大的宛县为祸患。我替您考虑,不如约请宛县守军投降,封赏南阳郡守,顺便留他在这里守城,您率领他的士卒向西进发。那些没有被攻下的城邑,听说这个消息,都会争相打开城门等您进城,您就能通行无所牵挂了。”

沛公说:“好。”于是他封南阳郡守为殷侯,封给陈恢一千户食邑。沛公领兵向西进发,没有不能攻下的城邑。

\begin{yuanwen}
至丹水,高武侯鰓、襄侯王陵降西陵。还攻胡阳,遇番君\footnote{吴芮。番,pó。}别将梅鋗,与皆\footnote{同“偕”。},降析、郦。遣魏人甯昌使秦,使者未来。是时章邯已以军降项羽于赵矣。
\end{yuanwen}

抵达丹水,高武侯戚鳃、襄侯王陵在西陵投降。回军进攻胡阳,遇到番君的别将梅鋗,和他一起,降服了析县、郦县。沛公派魏人宁昌出使秦朝,使者没有回来。当时章邯已经率领全军在赵地向项羽投降了。

\begin{yuanwen}
初,项羽与宋义北救赵,及项羽杀宋义,代为上将军,诸将黥布皆属,破秦将王离军,降章邯,诸侯皆附。

及赵高已杀二世\footnote{text},使人来,欲约分王关中。沛公以为诈,乃用张良计\footnote{text},使郦生\footnote{郦食其。}、陆贾往说秦将\footnote{text},啖\footnote{用利益引诱。}以利\footnote{text},因袭攻武关\footnote{text},破之。又与秦军战于蓝田南\footnote{text},益张疑兵旗帜\footnote{text},诸所过毋得掠卤\footnote{通“虏”。},秦人熹\footnote{xǐ},秦军解\footnote{text},因大破之。又战其北,大破之。乘胜,遂破之\footnote{text}。
\end{yuanwen}

当初,项羽和宋义向北援救赵国,等到项羽杀死宋义,取代他为上将军,众将领黥布等人都隶属于他,打败秦将王离的军队,降服章邯,诸侯都归附项羽。

等到赵高杀死秦二世之后,派人前来,想要约定在关中分地称王。沛公认为有诈,就采用张良的计策,派郦生、陆贾前去劝说秦军将领,用利益引诱他们,顺势袭击武关,将其攻破。又在蓝田以南和秦军交战,增设疑兵和旗帜,所有路过的地方都不准掳掠,秦地的百姓都很高兴,秦军因此瓦解,沛公趁机大破秦军。又在蓝田以北交战,大败秦军。沛公乘胜追击,终于打败秦军。

\begin{yuanwen}
汉元年十月\footnote{当时刘邦尚未称汉王。},沛公兵遂先诸侯至霸上\footnote{text}。秦王子婴素车白马,系颈以组\footnote{text},封皇帝玺符节\footnote{text},降轵道旁。诸将或言诛秦王。沛公曰:“始怀王遣我,固以能宽容;且人已服降,又杀之,不祥。”

乃以秦王属吏\footnote{text}。遂西入咸阳,欲止宫休舍。樊哙、张良谏,乃封秦重宝财物府库,还军霸上\footnote{text}。召诸县父老豪桀曰:“父老苦秦苛法久矣,诽谤者族,偶语者弃巿\footnote{在街市处死罪犯,令众人唾弃之。}。吾与诸侯约,先入关者王之,吾当王关中。与父老约法三章耳:杀人者死,伤人及盗抵罪\footnote{text}。馀悉除去秦法\footnote{text}。诸吏人皆案堵\footnote{安居。}如故\footnote{text}。凡吾所以来,为父老除害,非有所侵暴,无恐!且吾所以还军霸上,待诸侯至而定约束耳。”

乃使人与秦吏行县乡邑,告谕之。秦人大喜,争持牛羊酒食献飨军士\footnote{text}。沛公又让不受,曰:“仓粟多,非乏,不欲费人。”

人又益喜,唯恐沛公不为秦王\footnote{text}。
\end{yuanwen}

汉元年(前206年)十月,沛公的军队最终先于各路诸侯到达霸上。秦王子婴乘坐素车白马,用丝绳系着脖子,封好皇帝印玺和符节,在轵道旁投降。众将领中有的人建议诛杀秦王。沛公说:“最初怀王派我来,本来就是因为我能宽容。况且人家已经投降了,还要杀掉人家,这样做不吉利。”

他就将秦王交给官吏,向西进入咸阳。沛公想要留在宫中歇息,樊哙、张良劝谏,他才封存秦宫的珍宝财物及府库,撤回霸上驻扎。沛公召集各县的父老及豪杰说:“各位父老深受秦朝的严苛法令折磨已经很久了,批评朝政的要灭族,相聚交谈的要弃市。我和诸侯约定,先进入关中的人在这里称王,我应当在关中称王。我和父老约法三章:杀人的处死,伤人和劫掠的根据罪行来治罪。其余的秦朝法令全部废除。众官吏和百姓都像往常一样安居乐业。我来到这里的原因,就是要为父老铲除祸害,不会有欺凌施暴的行为,不要害怕!况且我撤回霸上驻扎,只是要等诸侯前来共同制定法令罢了。”

于是他派人和秦地的官吏巡视县乡邑,告谕百姓。秦地百姓十分高兴,争相送来牛羊酒食犒劳将士。沛公又推辞不接受,说:“仓库的粮食充足,不缺少食物,不想让你们破费。”

民众更加高兴,只怕沛公不做秦王。

\begin{yuanwen}
或说沛公曰:“秦富十倍天下,地形强。今闻章邯降项羽,项羽乃号为雍王,王关中\footnote{text}。今则来\footnote{text},沛公恐不得有此。可急使兵守函谷关,无内\footnote{nà}诸侯军,稍征关中兵以自益,距\footnote{text}之。”

沛公然其计,从之。

十一月中,项羽果率诸侯兵西,欲入关,关门闭。闻沛公已定关中,大怒,使黥布等攻破函谷关。

十二月中,遂至戏\footnote{text}。沛公左司马曹无伤闻项王怒,欲攻沛公,使人言项羽曰:“沛公欲王关中,令子婴为相,珍宝尽有之。”

欲以求封。亚父\footnote{范增。}劝项羽击沛公\footnote{text}。方飨士,旦日合战。是时项羽兵四十万,号百万。沛公兵十万,号二十万,力不敌。会项伯欲活张良\footnote{text},夜往见良,因以文谕项羽\footnote{text},项羽乃止。沛公从百馀骑,驱之鸿门\footnote{text},见谢项羽。项羽曰:“此沛公左司马曹无伤言之。不然,籍何以生此!”

沛公以樊哙、张良故,得解归。归,立诛曹无伤。
\end{yuanwen}

有人劝沛公说:“秦地的富足是天下的十倍,地形有优势。现在听说章邯已经投降项羽,项羽就封他为雍王,统治关中。现在他们就要过来,沛公恐怕无法保住这里了。应该赶快派兵守住函谷关,不要让诸侯的军队进入,逐步征发关中的士兵来增强实力,抗拒他们。”

沛公认为他的建议很有道理,就采纳了。

十一月中旬,项羽果然率领诸侯的军队向西进发,想要进入函谷关,关门却紧闭。项羽听说沛公已经平定关中,非常生气,派黥布等人攻破函谷关。

十二月中旬,项羽抵达戏水。沛公的左司马曹无伤听说项王很生气,想要进攻沛公,就派人对项羽说:“沛公想要在关中称王,让子婴做丞相,把珍宝据为己有。”

曹无伤想要借此求得封赏。亚父劝说项羽攻打沛公。项羽正在犒劳士卒,打算第二天和沛公交战。当时项羽的兵力有四十万,号称一百万。沛公的兵力是十万,号称二十万,兵力不能匹敌。正逢项伯想救张良,在夜里去见他,沛公趁机通过项伯对项羽讲道理,项羽才作罢。沛公带着一百多名骑兵,乘车来到鸿门,拜见项羽谢罪。项羽说:“这是沛公的左司马曹无伤对我说的。否则,我怎会做这样的事呢!”

沛公由于樊哙、张良的缘故,才得以脱身归来。回到军中,他立即诛杀了曹无伤。

\begin{yuanwen}
项羽遂西,屠烧咸阳秦宫室,所过无不残破。秦人大失望,然恐,不敢不服耳\footnote{text}。
\end{yuanwen}

项羽于是向西进发,屠戮焚烧咸阳城和秦朝宫室,所经过的地方没有不被摧毁的。秦地的百姓非常失望,然而因为害怕项羽,只是不敢不服从罢了。

\begin{yuanwen}
项羽使人还报怀王。怀王曰:“如约\footnote{text}。”

项羽怨怀王不肯令与沛公俱西入关,而北救赵,后天下约。乃曰:“怀王者,吾家项梁所立耳,非有功伐,何以得主约!本定天下,诸将及籍也。”

乃详尊怀王为义帝,实不用其命。
\end{yuanwen}

项羽派人回去报告怀王。怀王说:“按照约定办。”

项羽怨恨怀王不肯让他和沛公一同西进入关,却派他北上援救赵国,耽误了天下诸侯的约定,就说:“怀王,只是我叔父项梁所立的罢了,没有功劳,凭什么能够主持约定呢!本来平定天下的,就是各位将领和我。”

于是他假意尊崇楚怀王为义帝,其实并不服从他的命令。

\begin{yuanwen}
正月\footnote{text},项羽自立为西楚霸王,王梁、楚地九郡,都彭城。负约,更立沛公为汉王,王巴、蜀、汉中,都南郑。三分关中,立秦三将:章邯为雍王,都废丘;司马欣为塞王,都栎阳;董翳为翟王,都高奴。楚将瑕丘申阳为河南王,都洛阳。赵将司马卬为殷王,都朝歌。赵王歇徙王代。赵相张耳为常山\footnote{即恒山。《史记》避汉文帝刘恒讳,改恒为常。}王,都襄国。当阳君黥布为九江王,都六。怀王柱国共敖为临江王,都江陵。番君吴芮\footnote{ruì}为衡山王,都邾。燕将臧荼为燕王,都蓟。故燕王韩广徙王辽东。广不听,臧荼攻杀之无终。封成安君陈馀河间三县,居南皮。封梅鋗十万户。
\end{yuanwen}

正月,项羽自立为西楚霸王,统治梁地、楚地的九个郡,定都彭城。他违背约定,改立沛公为汉王,统治巴、蜀、汉中,定都南郑。关中一分为三,封给秦朝的三个降将:章邯为雍王,定都废丘;司马欣为塞王,定都栎阳;董翳为翟王,定都高奴。楚将瑕丘人申阳为河南王,定都洛阳。赵将司马卬为殷王,定都朝歌。赵王歇改封代王。赵将张耳为常山王,定都襄国。当阳君黥布为九江王,定都六县。怀王的柱国共敖为临江王,定都江陵。番君吴芮为衡山王,定都邾县。燕将臧荼为燕王,定都蓟县。前燕王韩广改封辽东王。韩广没有服从,臧荼就进攻韩广,在无终将其杀死。封给成安君陈馀河间的三个县为食邑,让他居住在南皮。封给梅鋗十万户为食邑。

\begin{yuanwen}
四月,兵罢戏下\footnote{text},诸侯各就国。汉王之国,项王使卒三万人从\footnote{text},楚与诸侯之慕从者数万人,从杜南入蚀中。去辄烧绝栈道\footnote{在崖壁上凿孔以木头搭建的通道。},以备诸侯盗兵袭之,亦示项羽无东意。至南郑,诸将及士卒多道亡归,士卒皆歌思东归。

韩信说汉王曰:“项羽王诸将之有功者,而王独居南郑,是迁\footnote{text}也。军吏士卒皆山东\footnote{text}之人也,日夜跂\footnote{qì,踮脚。}而望归\footnote{text}。及其锋而用之,可以有大功。天下已定,人皆自宁,不可复用。不如决策东乡\footnote{text},争权天下。”
\end{yuanwen}

四月,各路大军在戏下撤离,诸侯分别回到自己的封国。汉王前往封国,项王派士兵三万人随行,楚国和诸侯的将士因为仰慕汉王而追随的多达几万人,沿杜县往南进入蚀中。汉军离开之后就烧毁了沿途的栈道,来防备诸侯和盗贼偷袭,也向项羽表示没有东进之意。抵达南郑,众将领和士卒有许多人在途中逃回东方,士卒都唱着歌想要回到东方。

韩信劝汉王说:“项羽封有功的将领为王,可是大王却被封到南郑,这是流放。军吏和士兵都是山东的人,日夜踮脚盼望回到家乡。利用他们此时的锐气,可以建立大功。天下平定以后,人们都安下心来,就再也不能利用了。不如下令东进,和诸侯争夺天下。”

\begin{yuanwen}
项羽出关,使人徙义帝。曰:“古之帝者地方千里,必居上游。”

乃使使徙义帝长沙郴县,趣\footnote{cù,通“促”,催促。}义帝行,群臣稍倍叛之,乃阴令衡山王、临江王击之,杀义帝江南。项羽怨田荣,立齐将田都为齐王。田荣怒,因自立为齐王,杀田都而反楚;予彭越将军印,令反梁地。楚令萧公角击彭越,彭越大破之。陈馀怨项羽之弗王己也,令夏说说田荣,请兵击张耳。齐予陈馀兵,击破常山王张耳,张耳亡归汉。迎赵王歇于代,复立为赵王。赵王因立陈馀为代王。项羽大怒,北击齐。
\end{yuanwen}

项羽出函谷关,派人去迁徙义帝,说:“古时候的帝王统治方圆千里的土地,一定会在上游居住。”

他就派使者将义帝迁徙到长沙的郴县,催促义帝动身,群臣逐渐产生了背叛之心,于是暗中命令衡山王、临江王袭击义帝,将其杀死在江南。项羽怨恨田荣,立齐将田都为齐王。田荣很生气,趁机自立为齐王,杀死田都而背叛楚国,授予彭越将军印,命令他在梁地反叛。楚国命令萧公角进攻彭越,彭越大破萧公角。陈馀怨恨项羽没有封自己为王,命令夏说劝说田荣,借兵进攻张耳。齐国借兵给陈馀,打败了常山王张耳,张耳逃跑归顺汉王。陈馀从代郡迎回赵王歇,重新立为赵王。赵王顺势立陈馀为代王。项羽非常生气,出兵向北进攻齐国。

\begin{yuanwen}
八月,汉王用韩信之计,从故道还\footnote{text},袭雍王章邯。邯迎击汉陈仓,雍兵败,还走,止战好畤;又复败,走废丘,汉王遂定雍地。东至咸阳,引兵围雍王废丘\footnote{text},而遣诸将略定陇西、北地、上郡\footnote{text}。令将军薛欧、王吸出武关,因王陵兵南阳\footnote{text},以迎太公、吕后于沛。楚闻之,发兵距之阳夏,不得前。令故吴令郑昌为韩王,距汉兵。
\end{yuanwen}

八月,汉王采取韩信的计策,从原路返回关中,袭击雍王章邯。章邯在陈仓迎击汉军,雍军战败,撤退;在好畤停下交战,再次战败,逃往废丘。汉王最终平定雍地。大军进发到咸阳,领兵将雍王围困在废丘,并且派将领攻取陇西、北地、上郡。汉王命令将军薛欧、王吸从武关出发,凭借王陵驻守南阳的军队,去沛县迎接太公、吕后。楚国听说后,发兵在阳夏阻拦,汉军无法前进。项羽封原吴县令郑昌为韩王,阻挡汉军。

\begin{yuanwen}
二年,汉王东略地\footnote{text},塞王欣、翟王翳、河南王申阳皆降。韩王昌不听,使韩信击破之。于是置陇西、北地、上郡、渭南、河上、中地郡;关外置河南郡。更立韩太尉信\footnote{与淮阴侯韩信同姓名,称韩王信以区别。}为韩王。诸将以万人若以一郡降者,封万户。缮治河上塞。诸故秦苑囿\footnote{yòu}园池,皆令人得田之。

正月,虏雍王弟章平。大赦罪人。
\end{yuanwen}

二年(前205年),汉王向东攻取城邑,塞王司马欣、翟王董翳、河南王申阳全部投降。韩王郑昌不愿臣服,汉王就派韩信将他击败。于是汉王设置陇西、北地、上郡、渭南、河上、中地等郡,在关外设置河南郡。重新任命韩国太尉韩信为韩王。众将领率领一万人或是一郡归降的,就封赏一万户食邑。汉军修整河上的要塞。那些原属秦朝的园林和池塘,都分给百姓来耕种。

正月,汉军俘获雍王的弟弟章平。汉王大赦罪人。

\begin{yuanwen}
汉王之出关至陕,抚关外父老,还,张耳来见,汉王厚遇之。
\end{yuanwen}

汉王出函谷关来到陕县时,安抚关外的百姓,返回后,张耳前来拜见,汉王待他很优厚。

\begin{yuanwen}
二月,令除秦社稷\footnote{土神庙和谷神庙。我国古代以农为本,历代王朝均以农立国,重视对后土、后稷的祭祀,将社、稷立于国都的中心地带,象征国家政权。},更立汉社稷\footnote{更立社稷表示一个新王朝的开始。可见刘邦规模宏远。}。
\end{yuanwen}

二月,汉王命令废弃秦社稷,改立汉社稷。

\begin{yuanwen}
三月,汉王从临晋渡,魏王豹将兵从。下河内,虏殷王,置河内郡。南渡平阴津,至雒阳。新城三老\footnote{官名,掌管教化,每乡设一人。}董公遮说汉王以义帝死故\footnote{text}。汉王闻之,袒而大哭\footnote{text},遂为义帝发丧,临三日\footnote{text}。发使者告诸侯曰:“天下共立义帝,北面事之。今项羽放杀义帝于江南\footnote{text},大逆无道。寡人亲为发丧,诸侯皆缟素\footnote{text}。悉发关内兵,收三河士\footnote{text},南浮江、汉以下,愿从诸侯王击楚之杀义帝者\footnote{text}。”
\end{yuanwen}

三月,汉王在临晋渡过黄河,魏王豹率领军队跟随。汉军攻下河内,擒获殷王,设置河内郡。汉军向南从平阴津渡河,抵达雒阳。新城三老董公拦住汉王告知义帝的死因。汉王听说后,袒露上身大哭,于是为义帝举行丧礼,致哀三天。他派使者通告诸侯说:“天下诸侯共同拥立义帝,面向北方事奉他。现在项羽流放义帝,在江南杀害他,这是大逆不道。我亲自为义帝举行丧礼,诸侯都要身穿丧服。调发关内的军队,集结三河的士兵,向南渡过长江和汉水,愿意追随诸侯王讨伐楚国杀害义帝的人。”

\begin{yuanwen}
是时项王北击齐,田荣与战城阳。田荣败,走平原,平原民杀之。齐皆降楚。楚因焚烧其城郭,系虏其子女。齐人叛之。田荣弟横立荣子广为齐王,齐王反楚城阳。项羽虽闻汉东,既已连齐兵\footnote{text},欲遂破之而击汉。汉王以故得劫五诸侯兵\footnote{text},遂入彭城。项羽闻之,乃引兵去齐,从鲁出胡陵,至萧,与汉大战彭城灵壁东睢水上,大破汉军,多杀士卒,睢水为之不流。乃取汉王父母妻子于沛,置之军中以为质。当是时,诸侯见楚强汉败,还皆去汉复为楚。塞王欣亡入楚。
\end{yuanwen}

当时项王向北攻打齐国,田荣与其在城阳交战。田荣战败,逃往平原,被平原的百姓杀死。齐地全部投降楚国。楚军趁机烧毁齐国的城郭,掳掠他们的子女。齐人又背叛了楚国。田荣的弟弟田横立田荣的儿子田广为齐王,齐王在城阳反叛楚国。项羽虽然听说汉军向东进发,但是和齐军连续交战以后,就想彻底打败齐军再迎战汉军。汉王因为这个缘故得以劫持五路诸侯的军队,终于进入彭城。项羽听说后,就领兵离开齐国,经鲁县从胡陵出发,到达萧县,和汉军在彭城灵璧以东的睢水边大战,大破汉军,杀死众多士兵,睢水因为尸体的堵塞而无法流通。楚军就在沛县抓走了汉王的父母妻儿,安置在军中做人质。在这个时候,诸侯见楚军强盛而汉军的败退,都背离汉王而重新归附楚国。塞王司马欣逃到楚国。

\begin{yuanwen}
吕后兄周吕侯为汉将兵,居下邑\footnote{text}。汉王从之,稍收士卒,军砀\footnote{text}。汉王乃西过梁地,至虞\footnote{text}。使谒者\footnote{官名,掌管传达政令、接待宾客之事。}随何之九江王布所\footnote{text},曰:“公能令布举兵叛楚,项羽必留击之。得留数月,吾取天下必矣。”

随何往说九江王布,布果背楚\footnote{text}。楚使龙且往击之。
\end{yuanwen}

吕后的哥哥周吕侯为汉军统率一支军队,在下邑驻扎。汉王来投奔他,逐渐收拢逃散的士兵,驻扎在砀县。汉王于是向西路过梁地,来到虞县,派谒者随何前往九江王黥布的住所,汉王说:“您能够让黥布起兵背叛楚国,项羽一定会留下进攻九江。能够拖延他几个月,我就一定能夺取天下了。”

随何前去劝说九江王黥布,黥布果然背叛楚国。楚国派龙且前去进攻九江。

\begin{yuanwen}
汉王之败彭城而西,行使人求家室,家室亦亡,不相得。败后乃独得孝惠。

六月,立为太子,大赦罪人。令太子守栎阳,诸侯子在关中者皆集栎阳为卫。引水灌废丘,废丘降,章邯自杀。更名废丘为槐里。于是令祠官祀天地、四方上帝、山川,以时祀之。兴关内卒乘塞。

是时九江王布与龙且战,不胜,与随何间行归汉。汉王稍收士卒,与诸将及关中卒益出,是以兵大振荥阳,破楚京、索间。
\end{yuanwen}

汉王在彭城战败而向西撤退时,在路上派人去找寻家眷,家眷也逃走了,没能遇见。战败之后只找到了孝惠帝。

六月,立他为太子,大赦罪人。汉王命令太子驻守栎阳,诸侯的儿子在关中的都聚集在栎阳防守。汉军引水淹灌废丘,废丘投降,章邯自杀。汉王将废丘改名为槐里。这时他命令祠官祭祀天地、四方、上帝、山川,以后按季节祭祀。汉军征调关内的士兵去防守边塞。

当时九江王黥布和龙且交战,不能取胜,和随何走小路归附汉国。汉王逐渐收拢逃散的士兵,众将领和关中的士兵出去增援,于是荥阳士气振奋,在京、索两邑之间打败了楚军。

\begin{yuanwen}
三年,魏王豹谒归视亲疾,至即绝河津,反为楚。汉王使郦生说豹,豹不听。汉王遣将军韩信击,大破之,虏豹。遂定魏地,置三郡,曰河东、太原、上党。汉王乃令张耳与韩信遂东下井陉击赵,斩陈馀、赵王歇。其明年,立张耳为赵王。
\end{yuanwen}

三年(前204年),魏王豹请求回家探视生病的父母,到达魏国就切断黄河渡口,反叛汉国归附楚国。汉王派郦生劝说魏豹,魏豹不听从。汉王派将军韩信攻打魏国,大败魏军,擒获魏豹。汉军终于平定魏地,设置三个郡,分别为河东、太原、上党。汉王于是命令张耳和韩信向东经由井陉攻打赵国,斩杀陈馀、赵王歇。第二年,封张耳为赵王。

\begin{yuanwen}
汉王军荥阳南,筑甬道属之河\footnote{text},以取敖仓\footnote{text}。与项羽相距岁馀\footnote{text}。项羽数侵夺汉甬道,汉军乏食,遂围汉王。汉王请和,割荥阳以西者为汉。项王不听。汉王患之,乃用陈平之计,予陈平金四万斤,以间疏楚君臣。于是项羽乃疑亚父。亚父是时劝项羽遂下荥阳\footnote{text},及其见疑,乃怒,辞老\footnote{text},愿赐骸骨归卒伍\footnote{指告老还乡。},未至彭城而死。
\end{yuanwen}

汉王驻扎在荥阳以南,修筑甬道和黄河连通,来运取敖仓的粮食。汉军和项羽相持了一年多。项羽多次袭扰汉军的甬道,汉军缺乏食物,项羽最终包围了汉王。汉王请求和解,提出将荥阳以西划归汉国的条件。项王不答应。汉王为此很担心,就采用陈平的计策,给陈平金四万斤,来离间楚国的君臣关系。于是项羽开始怀疑亚父。亚父当时劝说项羽趁机攻下荥阳,等到他被怀疑,就感到很生气,以年老为理由,请求辞官回家,还没到彭城就去世了。

\begin{yuanwen}
汉军绝食,乃夜出女子东门二千馀人,被甲\footnote{text},楚因四面击之。将军纪信乃乘王驾,诈为汉王,诳楚\footnote{text},楚皆呼万岁,之城东观,以故汉王得与数十骑出西门遁。令御史大夫周苛、魏豹、枞公守荥阳。诸将卒不能从者,尽在城中。周苛、枞公相谓曰:“反国之王,难与守城。”因杀魏豹。
\end{yuanwen}

汉军断粮,就在夜里从东门放出两千多个女子,身披铠甲,楚军于是从四面围攻。将军纪信就乘坐着汉王的车驾,假扮成汉王,欺骗楚军。楚国士兵都高喊万岁,去城东观看,因为这个缘故汉王得以和几十名骑兵从西门逃出去。汉王命令御史大夫周苛、魏豹、枞公驻守荥阳。众将领及士卒无法随行的,都留在城中。周苛、枞公相互商量说:“魏豹是反叛之国的王,难以和他一起守城。”于是他们杀死了魏豹。

\begin{yuanwen}
汉王之出荥阳入关,收兵欲复东。袁生说汉王曰:“汉与楚相距荥阳数岁,汉常困。原君王出武关,项羽必引兵南走,王深壁,令荥阳成皋间且得休。使韩信等辑\footnote{平定。}河北赵地,连燕齐,君王乃复走荥阳,未晚也。如此,则楚所备者多,力分,汉得休,复与之战,破楚必矣。”

汉王从其计,出军宛叶间,与黥布行收兵。
\end{yuanwen}

汉王出荥阳进入关中时,收拢残兵想要再次向东进发。袁生劝汉王说:“汉军和楚军在荥阳相持已经几年了,汉军总是处于困境。希望君王从武关出击,项羽一定会领兵向南逃跑,君王凭借深沟高垒,让荥阳、成皋之间的士兵暂时得以休整。派韩信等人去平定黄河以北的赵地,与燕、齐两国联合,君王再去荥阳,也不算晚。像这样,楚军所要防备的地方就很多,兵力分散,汉军得以休整,再和楚军交战,就一定能够击败楚军了。”

汉王采纳了他的计策,出兵驻扎在宛县、叶县之间,和黥布在沿途收拢残兵。

\begin{yuanwen}
项羽闻汉王在宛,果引兵南。汉王坚壁不与战。是时彭越渡睢水\footnote{text},与项声、薛公战下邳\footnote{text},彭越大破楚军。项羽乃引兵东击彭越,汉王亦引兵北军成皋。项羽已破走彭越,闻汉王复军成皋,乃复引兵西,拔荥阳,诛周苛、枞公,而虏韩王信,遂围成皋。
\end{yuanwen}

项羽听说汉王在宛县,果然领兵向南进发。汉王坚守营垒不与楚军交战。当时彭越渡过睢水,和项声、薛公在下邳交战,彭越大破楚军。项羽就领兵向东进攻彭越。汉王也领兵向北驻扎在成皋。项羽击退彭越之后,听说汉王重新在成皋驻扎,就再次领兵向西进发,攻下荥阳,杀死周苛、枞公,并且擒获韩王信,最终包围成皋。

\begin{yuanwen}
汉王跳\footnote{通“逃”。},独与滕公共车出成皋玉门,北渡河,驰宿脩武\footnote{text}。自称使者,晨驰入张耳、韩信壁\footnote{text},而夺之军。乃使张耳北益收兵赵地,使韩信东击齐。汉王得韩信军,则复振。引兵临河,南飨军小脩武南,欲复战。郎中郑忠乃说止汉王,使高垒深堑,勿与战。汉王听其计,使卢绾、刘贾将卒二万人,骑数百,渡白马津,入楚地,与彭越复击破楚军燕郭西,遂复下梁地十馀城。
\end{yuanwen}

汉王逃走,独自和滕公一起乘车从成皋玉门出去,向北渡过黄河,驱车到脩武住宿。汉王自称使者,早上驱车进入张耳、韩信的营中,并且夺取他们的兵权,于是命令张耳向北招募赵地的士兵增援,命令韩信向东进攻齐国。汉王得到韩信的军队,士气重新振奋。他率领军队来到黄河边,向南在小脩武以南犒劳士卒,想要再次作战。郎中郑忠于是劝阻汉王,让他凭借高垒深沟,不要与项羽交战。汉王采纳了他的计策,派卢绾、刘贾率领士兵两万人,骑兵几百人,从白马津渡河,进入楚地,和彭越在燕县外城以西再次击败楚军,最终重新攻下梁地的十多座城邑。

朱子:「项籍垓下之败,实被韩信布得阵好,是以一败而竟毙。不特此耳,自韩信左取燕、齐、赵、魏,右取九江王布,收大司马殷周,而羽渐困于中,而手足日翦,则不待垓下之败,而其大势盖已不胜汉矣。」

\begin{yuanwen}
淮阴\footnote{指韩信。}已受命东,未渡平原。汉王使郦生往说齐王田广,广叛楚,与汉和,共击项羽。韩信用蒯通\footnote{本名彻,避汉武帝刘彻讳,改彻为通。}计,遂袭破齐。齐王烹郦生,东走高密。项羽闻韩信已举河北兵破齐、赵,且欲击楚,则使龙且、周兰往击之。韩信与战,骑将灌婴击,大破楚军,杀龙且。齐王广(饹|奔)彭越。当此时,彭越将兵居梁地,往来苦楚兵,绝其粮食。
\end{yuanwen}

淮阴侯接到命令后向东进发,还在平原没有渡河。汉王派郦生前去劝说齐王田广,田广反叛楚国,和汉国和解,一同进攻项羽。韩信采纳蒯通的计策,最终攻破齐国。齐王烹杀郦生,向东逃往高密。项羽听说韩信已经率领黄河以北的士兵攻破齐、赵两国,并且想要进攻楚国,就派龙且、周兰前往阻击。韩信和楚交战,骑兵将领灌婴出击,大破楚军,杀死龙且。齐王田广投奔彭越。在这个时候,彭越率领士兵驻扎在梁地,往来袭扰楚军,断绝他们运送粮食的通道。

\begin{yuanwen}
四年,项羽乃谓海春侯大司马曹咎曰:“谨守成皋。若汉挑战,慎勿与战,无令得东而已。我十五日必定梁地,复从将军。”

乃行击陈留、外黄、睢阳,下之。汉果数挑楚军,楚军不出,使人辱之五六日,大司马怒,度兵汜水。士卒半渡,汉击之,大破楚军,尽得楚国金玉货赂。大司马咎、长史欣皆自刭汜\footnote{sì}水上。项羽至睢阳,闻海春侯破,乃引兵还。汉军方围锺离眛于荥阳东,项羽至,尽走险阻。

韩信已破齐,使人言曰:“齐边楚,权轻,不为假王\footnote{代理的王。},恐不能安齐。”

汉王欲攻之。留侯曰:“不如因而立之,使自为守。”乃遣张良操印绶立韩信为齐王。

项羽闻龙且军破,则恐,使盱台人武涉往说韩信。韩信不听。
\end{yuanwen}

四年(前203年),项羽就对海春侯大司马曹咎说:“谨慎地守卫成皋。如果汉军挑战,千万不要和他们交战,只是不让他们向东进发就可以了。我十五天内一定会平定梁地,再和将军会合。”

他就出兵去攻打陈留、外黄、睢阳,将其攻下。汉军果然多次挑战楚军,楚军不出战,汉军派人接连辱骂楚军五六天,大司马很生气,让士兵渡过汜水。士兵渡过一半时,汉军发起进攻,大破楚军,缴获楚国的全部金玉财物。大司马曹咎、长史司马欣都在汜水边上自刎。项羽来到睢阳,听说海春侯被打败,就率领军队返回。汉军当正在荥阳以东围攻钟离眛,项羽赶到,汉军全部撤退到险要地带。

韩信攻破齐国以后,派人告诉汉王说:“齐国临近楚国,权力太小,不做代理的国王,恐怕无法安定齐国。”

汉王想进攻韩信。留侯说:“不如趁机封他为齐王,让他为自己驻守齐国。”汉王就派张良带着印绶封韩信为齐王。

项羽听说龙且的军队被打败,就很害怕,派盱台人武涉前往劝说韩信。韩信不听。

\begin{yuanwen}
楚、汉久相持未决\footnote{text},丁壮苦军旅,老弱罢转(饷|馕)。汉王、项羽相与临广武之间\footnote{山涧。}而语\footnote{text}。项羽欲与汉王独身挑战。汉王数项羽曰:“始与项羽俱受命怀王,曰先入定关中者王之,项羽负约,王我于蜀汉,罪一。项羽矫\footnote{假传命令。}杀卿子冠军而自尊,罪二。项羽已救赵,当还报,而擅劫诸侯兵入关,罪三。怀王约入秦无暴掠,项羽烧秦宫室,掘始皇帝冢,私收其财物,罪四。又彊杀秦降王子婴,罪五。诈阬(坑)秦子弟新安二十万,王其将,罪六。项羽皆王诸将善地,而徙逐故主,令臣下争叛逆,罪七。项羽出逐义帝彭城,自都之,夺韩王地,并王梁楚,多自予,罪八。项羽使人阴弑义帝江南,罪九。夫为人臣而弑其主,杀已降,为政不平,主约不信,天下所不容,大逆无道,罪十也。吾以义兵从诸侯诛残贼,使刑馀\footnote{受过刑罚。}罪人击杀项羽,何苦乃与公\footnote{据文意及刘邦的性格分析,似乎应作“与乃公”。乃公,意思是你父亲,是一种傲慢的自称。如果作“乃与公”解释,虽然也说得通,但是以刘邦尊称项羽为“公”,似乎不合情理。}挑战!”

项羽大怒,伏弩射中汉王。汉王伤匈,乃扪足曰:“虏中吾指\footnote{text}!”

汉王病创卧,张良强请汉王起行劳军,以安士卒,毋令楚乘胜于汉。汉王出行军\footnote{text},病甚,因驰入成皋。
\end{yuanwen}

楚、汉长期相持不分胜负,丁壮为行军作战而劳苦,老弱因转运物资而疲惫。汉王、项羽一起在广武涧的两边交谈。项羽想要和汉王单独挑战。汉王列举项羽的罪过说:“最初我与项羽一起领受怀王的命令,说好谁先进入平定关中就可以在那里称王。项羽违背约定,封我为蜀、汉的王,此为第一条罪状。项羽假传怀王的命令杀死卿子冠军,并且自立为上将军,此为第二条罪状。项羽援救赵国以后,应当回去报告,却擅自劫持诸侯的军队进入关中,此为第三条罪状。怀王约定进入秦地不能施暴劫掠,项羽却烧毁秦宫室,挖掘始皇帝的坟墓,私下聚敛其财物,此为第四条罪状。项羽又强行杀死投降的秦王子婴,此为第五条罪状。项羽以欺骗手段在新安坑杀秦地子弟二十万人,封他们的将领为王,此为第六条罪状。项羽把自己的将领都封在好地方为王,却赶走当地原来的君主,让臣下争相叛逆,此为第七条罪状。项羽把义帝从彭城赶走,自己在那里建都,夺走韩王的封地,一并统治梁、楚两地,分给自己更多的土地,此为第八条罪状。项羽派人在江南暗杀义帝,此为第九条罪状。身为别人的臣子却杀害自己的君主,诛杀已经投降的人,处理政事不能公平,主持约定不能守信,天下人都不能容忍,大逆不道,此为第十条罪状。我率领正义之师跟随诸侯来诛杀残暴的贼人,派受过刑罚的罪人杀掉项羽,为什么向老子挑战!”

项羽非常生气,埋伏的弓弩手射中了汉王。汉王伤到胸部,却捂着脚说:“贼人射到了我的脚趾!”

汉王受伤卧床,张良请汉王强行起来巡视慰劳士卒,来稳定军心,不让楚军利用胜势攻打汉军。汉王出来巡视军中,伤势加重,因而驱车进入成皋。

\begin{yuanwen}
病愈,西入关,至栎阳,存问\footnote{慰问。}父老,置酒,枭故塞王欣头栎阳市。留四日,复如军,军广武。关中兵益出。

当此时,彭越将兵居梁地,往来苦楚兵,绝其粮食。田横往从之。
\end{yuanwen}

汉王伤愈,向西进入关中,抵达栎阳,慰问当地百姓,摆酒设宴,在栎阳街市将前塞王司马欣斩首示众。过了四天,汉王再次回到军中,驻扎在广武。关中的士兵出发增援。

在这个时候,彭越率领士兵驻扎在梁地,往来袭扰楚军,断绝他们运送粮食的通道。田横前去归附彭越。

\begin{yuanwen}
项羽数击彭越等,齐王信又进击楚。项羽恐,乃与汉王约,中分天下,割鸿沟而西者为汉,鸿沟而东者为楚。项王归汉王父母妻子,军中皆呼万岁\footnote{text},乃归而别去\footnote{text}。
\end{yuanwen}

项羽多次进攻彭越等人,齐王韩信又进军攻打楚军。项羽很害怕,就和汉王约定,平分天下,划分鸿沟以西的土地归汉国,鸿沟以东的土地归楚国。项王归还汉王的父母妻儿,汉军都高喊万岁,于是撤退离去。

\begin{yuanwen}
项羽解而东归。汉王欲引而西归,用留侯、陈平计\footnote{text},乃进兵追项羽,至阳夏南止军,与齐王信、建成侯彭越期会而击楚军\footnote{text}。至固陵\footnote{text},不会\footnote{text}。楚击汉军,大破之。汉王复入壁,深堑而守之。用张良计\footnote{text},于是韩信、彭越皆往。及刘贾入楚地,围寿春,汉王败固陵,乃使使者召大司马周殷,举九江兵而迎武王\footnote{指黥布。},行屠城父,随刘贾、齐梁诸侯皆大会垓下\footnote{text}。立武王布为淮南王。
\end{yuanwen}

项羽解除包围向东返回。汉王想要领兵向西返回,采用留侯、陈平的计策,于是出兵追击项羽,到阳夏以南停留驻扎,和齐王韩信、建成侯彭越约定时间会合后攻打楚军。汉王来到固陵,韩信、彭越没有来会合。楚军进攻汉军,大破汉军。汉王再次进入营垒,深挖堑壕来防守。汉王采用张良的计策,这时韩信、彭越都赶来会合。等到刘贾进入楚地,围困寿春,汉王在固陵战败,才派出使者召见大司马周殷,发动九江的全部兵力去迎接武王,在进军途中屠戮城父,跟随刘贾、齐国和梁国的诸侯都在垓下会合。汉王封武王为淮南王。

\begin{yuanwen}
五年\footnote{text},高祖与诸侯兵共击楚军,与项羽决胜垓下。淮阴侯将三十万自当之,孔将军居左,费将军居右\footnote{text},皇帝\footnote{刘邦在汉五年称帝,而此时尚未称帝。}在后,绛侯\footnote{周勃在汉朝建立后的封号。}、柴将军在皇帝后。项羽之卒可十万。淮阴先合\footnote{text},不利,却\footnote{text}。孔将军、费将军纵\footnote{text},楚兵不利。淮阴侯复乘之\footnote{text},大败垓\footnote{gài}下\footnote{text}。项羽卒闻汉军之楚歌,以为汉尽得楚地,项羽乃败而走,是以兵大败。使骑将灌婴追杀项羽东城\footnote{text},斩首八万,遂略定楚地。鲁为楚坚守不下\footnote{text},汉王引诸侯兵北,示鲁父老项羽头,鲁乃降。遂以鲁公号葬项羽穀城。还至定陶,驰入齐王壁,夺其军。
\end{yuanwen}

五年(前202年),高祖和诸侯的军队一起进攻楚军,和项羽在垓下决战。淮阴侯率领三十万人独自在正面抵挡,孔将军在左翼,费将军在右翼,皇帝在后方,绛侯、柴将军在皇帝之后。项羽的士兵大约有十万人。淮阴侯率先和项羽交战,不能取胜,撤退。孔将军、费将军出击,楚军不能取胜,淮阴侯又趁机进攻,在垓下大败项羽。项羽的士兵听到汉军士兵唱楚歌,以为汉军已经全部占领楚地,项羽于是战败逃走,所以楚军大败。汉王派骑兵将领灌婴在东城追杀项羽,斩首八万人,终于平定楚地。鲁县人为楚国坚守而无法攻下。汉王率领诸侯的军队向北进发,向鲁县的百姓展示项羽的头,鲁县人才投降。于是汉王以鲁公的封号将项羽埋葬在穀城。汉王回军到定陶,驱车进入齐王的营垒,夺取了他的兵权。

\begin{yuanwen}
正月,诸侯及将相相与共请尊汉王为皇帝。汉王曰:“吾闻帝贤者有也,空言虚语,非所守也,吾不敢当帝位。”

群臣皆曰:“大王起微细,诛暴逆,平定四海,有功者辄裂地而封为王侯。大王不尊号,皆疑不信。臣等以死守之。”

汉王三让,不得已,曰:“诸君必以为便,便国家。”

甲午\footnote{text},乃即皇帝位氾水之阳\footnote{text}。
\end{yuanwen}

正月,诸侯和将相共同请求尊崇汉王为皇帝。汉王说:“我听说皇帝的尊号是贤德之人才能拥有的,不切实际的名号,不是这样的人所持守的,我不敢担当皇帝的尊位。”

群臣都说:“大王出身寒微,诛除暴逆,平定天下,有功的人就分土地而封为王侯。大王不称皇帝的尊号,大家都会对自己的封号产生疑虑。我们冒死请求大王称皇帝。”

汉王多次辞让,没有办法,说:“各位一定是认为这样做有利,对国家有利。”

甲午日,汉王在氾水以北登上皇帝之位。

\begin{yuanwen}
皇帝曰义帝无后。齐王韩信习楚风俗,徙为楚王,都下邳。立建成侯彭越为梁王,都定陶。故韩王信为韩王,都阳翟。徙衡山王吴芮为长沙王,都临湘。番君之将梅鋗有功,从入武关,故德番君。淮南王布、燕王臧荼、赵王敖\footnote{张耳死后,其子张敖继位。}皆如故。

天下大定。高祖都雒阳,诸侯皆臣属。故临江王驩\footnote{共敖死后,其子共驩继位。}为项羽叛汉,令卢绾、刘贾围之,不下。数月而降,杀之雒阳。

五月,兵皆罢归家。诸侯子在关中者复之十二岁,其归者复之六岁,食之一岁。
\end{yuanwen}

皇帝说义帝没有后代,齐王韩信熟悉楚地的风俗,就改封他为楚王,定都下邳。封建成侯彭越为梁王,定都定陶。原韩王信为韩王,定都阳翟。改封衡山王吴芮为长沙王,定都临湘。番君的将领梅鋗有战功,跟随皇帝进入武关,所以皇帝感激番君。淮南王黥布、燕王臧荼、赵王张敖封地和以前一样。

天下大致平定。高祖定都雒阳,诸侯都是他的臣属。原临江王共驩为项羽反叛汉王,高祖命令卢绾、刘贾包围临江,不能攻下。几个月后将其降服,在雒阳杀死共驩。

五月,士兵都回到家乡。诸侯的子弟留在关中的都免除徭役十二年,回到家乡的士兵则免除六年,发给他们一年的粮食。

\begin{yuanwen}
高祖置酒雒阳南宫。高祖曰:“列侯\footnote{秦二十等爵最高者为彻侯,后来避汉武帝刘彻讳,改称通侯、列侯。}诸将无敢隐朕,皆言其情。吾所以有天下者何?项氏之所以失天下者何?”

高起、王陵对曰:“陛下慢而侮人,项羽仁而爱人。然陛下使人攻城略地,所降下者因以予之,与天下同利也。项羽妒贤嫉能,有功者害之\footnote{text},贤者疑之;战胜而不予人功,得地而不予人利,此所以失天下也。”

高祖曰:“公知其一,未知其二。夫运筹策帷帐\footnote{指将帅的军帐。}之中\footnote{text},决胜于千里之外,吾不如子房;镇国家,抚百姓,给馈饷(餽馕)\footnote{text},不绝粮道,吾不如萧何;连百万之军,战必胜,攻必取,吾不如韩信。此三者,皆人杰也,吾能用之,此吾所以取天下也。项羽有一范增而不能用,此其所以为我擒也\footnote{text}。”
\end{yuanwen}

高祖在雒阳的南宫摆酒设宴。高祖说:“列侯众将不要隐瞒我,都说真心话。我能取得天下的原因是什么呢?项氏失掉天下的原因是什么呢?”

高起、王陵回答说:“陛下傲慢而喜欢侮辱人,项羽仁厚而愿意爱护人。然而陛下派人攻打城邑夺取土地,所降服和攻下的地方就封给他们,和天下人同享利益。项羽却嫉妒贤能之士,有功的人就加以陷害,贤能的人就加以怀疑,获取胜利却不计别人的功劳,夺得土地也不给别人好处,这就是他失掉天下的原因。”

高祖说:“各位只知道其中一个方面,却不知道另一个方面。要说在帷帐中筹划策略,在千里外决战制胜,我比不上子房。镇守国家,抚慰百姓,供给粮饷,保证粮道不被阻断,我比不上萧何。统帅百万大军,作战必胜,进攻必克,我比不上韩信。他们三个人,都是人中的俊杰,我却能够任用他们,这才是我能够得到天下的原因。项羽只有一个范增却不能任用,这就是他被我擒获的原因。”

\begin{yuanwen}
高祖欲长都雒阳,齐人刘敬\footnote{娄敬,赐姓刘。}说,乃留侯劝上入都关中,高祖是日驾,入都关中\footnote{text}。

六月,大赦天下。
\end{yuanwen}

高祖想要长期定都雒阳,齐国人刘敬劝说高祖,留侯也劝说高祖进入关中建都,高祖当天起驾,进入关中建都。

六月,大赦天下。

\begin{yuanwen}
十月,燕王臧荼反,攻下代地。高祖自将击之,得燕王臧荼。即立太尉卢绾为燕王。使丞相哙将兵攻代。

其秋,利几反,高祖自将兵击之,利几走。利几者,项氏之将。项氏败,利几为陈公,不随项羽,亡降高祖,高祖侯之颍川。高祖至雒阳,举通侯籍召之,而利几恐,故反。
\end{yuanwen}

十月,燕王臧荼反叛,攻下代地。高祖亲自率领军队攻打叛军,擒获燕王臧荼,随即封太尉卢绾为燕王。他派丞相樊哙率领军队攻打代郡。

当年秋季,利几反叛,高祖亲自率领军队攻打叛军,利几逃走。利几,是项氏的将领。项氏失败,利几自称陈公,没有跟随项羽,逃出来投降高祖,高祖封他为颍川侯。高祖来到雒阳,根据名籍召见列侯,利几感到害怕,所以反叛。

\begin{yuanwen}
六年,高祖五日一朝太公,如家人父子礼。太公家令说太公曰:“天无二日,土无二王。今高祖\footnote{皇帝生前不应以庙号、谥号相称,此处是后世史官的追述。从上下文可知,此处应称“上”或“皇帝”。}虽子,人主也;太公虽父,人臣也。奈何令人主拜人臣!如此,则威重不行。”

后高祖朝,太公拥篲\footnote{扫帚。},迎门卻行。高祖大惊,下扶太公。

太公曰:“帝,人主也,奈何以我乱天下法!”

于是高祖乃尊太公为太上皇。心善家令言,赐金五百斤。
\end{yuanwen}

六年(前201年),高祖每隔五天朝见一次太公,像普通百姓一样行父子之礼。太公的家令劝太公说:“天上没有两个太阳,地上也没有两个君主。现在皇帝尽管是儿子,却也是君主;太公尽管是父亲,却也是臣下。怎么可以让君主跪拜臣下呢!像这样,君主的威严和尊贵就无法彰显。”

后来高祖朝见太公,太公就抱着扫帚,迎在门口,向后倒退着走。高祖非常惊讶,下车搀扶太公。

太公说:“皇帝是君主,怎么可以因为我而扰乱天下的法纪呢!”

于是高祖就尊崇太公为太上皇。高祖心中赞赏家令的话,赐给他金五百斤。

\begin{yuanwen}
十二月,人有上变事告楚王信谋反,上问左右,左右争欲击之。用陈平计,乃伪游云梦,会诸侯于陈,楚王信迎,即因执之。是日,大赦天下。田肯贺,因说高祖曰:“陛下得韩信,又治秦中。秦,形胜\footnote{地形条件优越。}之国,带河山之险,县隔千里,持戟百万,秦得百二\footnote{百倍。}焉。地埶(执)便利,其以下兵于诸侯,譬犹居高屋之上建瓴\footnote{líng,水瓶。}水也。夫齐,东有琅邪、即墨之饶,南有泰山之固,西有浊河之限,北有勃海\footnote{即渤海。}之利。地方二千里,持戟百万,县隔千里之外,齐得十二焉。故此东西秦也。非亲子弟,莫可使王齐矣。”

高祖曰:“善。”赐黄金五百斤。
\end{yuanwen}

十二月,有人上书告发楚王韩信阴谋反叛的事情,高祖询问身边的大臣,大臣们争着想要去征讨韩信。高祖采纳陈平的计策,假装巡游云梦泽,在陈县会见诸侯,楚王韩信前往迎接,就趁机将他擒获。当天,高祖大赦天下。田肯前来祝贺,趁机劝高祖说:“陛下擒获韩信,又定都秦地。秦地,是地理形势优越的地方,有四周环绕山河的险阻,与诸侯远隔上千里,手持长戟的武士达上百万,秦地的优势强于别处百倍。地理形势有利,从这里出兵到诸侯国,就像站在高大的房屋上往下倾倒瓶中的水一样。齐国,东面有琅邪、即墨的丰饶,南面有泰山的险固,西面有浊河的屏障,北面有渤海的物产。土地方圆二千里,手持长戟的武士上百万,与诸侯远隔上千里,齐国的优势强于别处十倍。因此这两个地方可以算是东西秦。不是陛下的亲近子弟,就不要封他为齐王了。”

高祖说:“好。”赐给他金五百斤。

\begin{yuanwen}
后十馀日,封韩信为淮阴侯,分其地为二国。高祖曰将军刘贾数有功,以为荆王,王淮东。弟交为楚王,王淮西。子肥为齐王,王七十馀城,民能齐言者皆属齐。乃论功,与诸列侯剖符\footnote{天子封侯、授官时剖开竹符,赐予大臣一半,自留一半,以此为信物。}行封。徙韩王信太原。
\end{yuanwen}

十几天以后,高祖封韩信为淮阴侯,将他原来的封地分为两国。高祖说将军刘贾多次立下战功,封他为荆王,统治淮水以东。封弟弟刘交为楚王,统治淮水以西。封儿子刘肥为齐王,统治七十多座城邑,说齐语的民众都归属齐国。高祖于是评定功劳,和众列侯剖符分封。他将韩王信迁徙到太原。

\begin{yuanwen}
七年,匈奴攻韩王信马邑,信因与谋反太原。白土曼丘臣、王黄立故赵将赵利为王以反,高祖自往击之。会天寒,士卒堕指\footnote{冻掉手指。}者什二三,遂至平城。匈奴围我平城,七日而后罢去。令樊哙止定代地。立兄刘仲为代王。
\end{yuanwen}

七年(前200年),匈奴来到马邑进攻韩王信,韩王信趁机和匈奴谋划在太原反叛。白土人曼丘臣、王黄立原赵将赵利为王来反叛,高祖亲自前去讨伐叛军。正逢天气寒冷,士兵冻掉手指的有十分之二三,终于来到平城。匈奴在平城围攻我军,七天后才撤离。高祖命令樊哙留下平定代地,封哥哥刘仲为代王。

\begin{yuanwen}
二月,高祖自平城过赵、雒阳,至长安。长乐宫成,丞相已下徙治长安。
\end{yuanwen}

二月,高祖从平城经赵国、雒阳,抵达长安。长乐宫建成,丞相以下官员迁到长安办公。

\begin{yuanwen}
八年,高祖东击韩王信馀反寇于东垣。
\end{yuanwen}

八年(前199年),高祖向东进发到东垣攻打韩王信的残余叛贼。

\begin{yuanwen}
萧丞相营作未央宫,立东阙、北阙、前殿、武库、太仓。高祖还,见宫阙壮甚,怒,谓萧何曰:“天下匈匈\footnote{纷乱的样子。}苦战数岁,成败未可知,是何治宫室过度也?”

萧何曰:“天下方未定,故可因遂就宫室。且夫天子四海为家,非壮丽无以重威,且无令后世有以加也。”

高祖乃说。

高祖之东垣,过柏人,赵相贯高等谋弑高祖,高祖心动,因不留。代王刘仲弃国亡,自归雒阳,废以为合阳侯。
\end{yuanwen}

萧丞相修建未央宫,建造了东阙、北阙、前殿、武库、太仓。高祖返回,见到宫阙非常壮丽,很生气,对萧何说:“天下纷乱不安,人们被战事困扰多年,事业的成败还不明确,怎么能把宫室修建得这样奢华呢?”

萧何说:“天下还没有安定,因此可以趁机修建宫室。况且天子以四海为家,宫室不壮丽就无法彰显尊贵和威严,而且也可以让后世不能再扩建。”

高祖这才感到高兴。

高祖前往东垣,路过柏人,赵相贯高等人要谋杀高祖,高祖感到心惊,因此没有在柏人停留。代王刘仲丢弃封国逃跑,私自回到雒阳,被贬为合阳侯。

\begin{yuanwen}
九年,赵相贯高等事发觉,夷三族。废赵王敖为宣平侯。是岁,徙贵族楚昭、屈、景、怀、齐田氏关中。
\end{yuanwen}

九年(前198年),赵相贯高等人的阴谋被发觉,诛灭三族。高祖贬赵王张敖为宣平侯。这一年,迁徙楚国贵族昭氏、屈氏、景氏、怀氏,齐国贵族田氏到关中。

\begin{yuanwen}
未央宫成。高祖大朝诸侯群臣,置酒未央前殿。高祖奉玉卮,起为太上皇寿,曰:“始大人\footnote{对父母的尊称。}常以臣无赖\footnote{没出息。},不能治产业,不如仲力。今某之业所就孰与仲多?”

殿上群臣皆呼万岁,大笑为乐。
\end{yuanwen}

未央宫建成。高祖隆重地会见诸侯和群臣,在未央宫前殿摆酒设宴。高祖捧着玉杯,站起来祝福太上皇长寿,说:“最初父亲大人总是认为我没出息,不能经营产业,不如二哥努力。现在我的事业和二哥比,谁的更多呢?”

殿上的群臣都高喊万岁,大笑着取乐。

\begin{yuanwen}
十年十月,淮南王黥布、梁王彭越、燕王卢绾、荆王刘贾、楚王刘交、齐王刘肥、长沙王吴芮皆来朝长乐宫。

春、夏无事。
\end{yuanwen}

十年(前197年)十月,淮南王黥布、梁王彭越、燕王卢绾、荆王刘贾、楚王刘交、齐王刘肥、长沙王吴芮都来到长乐宫朝见高祖。

春季、夏季没有重大事件。

\begin{yuanwen}
七月\footnote{汉初沿袭秦朝历法,以夏历(农历)十月为岁首,到第二年九月为一年,因此七月在十月之后。},太上皇崩栎阳宫。楚王、梁王皆来送葬。赦栎阳囚。更命郦邑曰新丰。

八月,赵相国陈豨\footnote{xī}反代地。上曰:“豨尝为吾使,甚有信。代地吾所急也,故封豨为列侯,以相国守代,今乃与王黄等劫掠代地!代地吏民非有罪也。其赦代吏民。”

九月,上自东往击之。至邯郸,上喜曰:“豨不南据邯郸而阻漳水,吾知其无能为也。”

闻豨将皆故贾人也,上曰:“吾知所以与之。”

乃多以金啗(啖)豨将,豨将多降者。
\end{yuanwen}

七月,太上皇在栎阳宫去世。楚王、梁王都前来送葬。高祖赦免栎阳的囚犯。郦邑改名为新丰。

八月,赵相国陈豨在代地反叛。高祖说:“陈豨曾经做过我的使者,很守信用。代地是我很看重的地方,所以封陈豨为列侯,以相国的名义驻守代地,现在竟然和王黄等人劫掠代地!代地的官吏和百姓并没有罪,赦免代地的官吏和百姓。”

九月,高祖亲自向东攻打陈豨。来到邯郸,高祖高兴地说:“陈豨不向南占据邯郸,并且以漳水为屏障,我就知道他不能成功了。”

听说陈豨的部将都是以前的商人,高祖说:“我知道对付这些人的办法了。”

他就用金钱引诱陈豨的部将,陈豨有很多部将投降了。

\begin{yuanwen}
十一年,高祖在邯郸诛豨等未毕,豨将侯敞将万馀人游行\footnote{迂回作战。},王黄军曲逆,张春渡河击聊城。汉使将军郭蒙与齐将击,大破之。太尉周勃道太原入,定代地。至马邑,马邑不下,即攻残之。

豨将赵利守东垣,高祖攻之,不下。月馀,卒骂高祖,高祖怒。城降,令出骂者斩之,不骂者原之。于是乃分赵山北,立子恆(恒)以为代王,都晋阳。
\end{yuanwen}

十一年(前196年),高祖在邯郸还没有将陈豨等人完全诛灭,陈豨的部将侯敞率领一万多人迂回作战,王黄驻扎在曲逆,张春渡过黄河攻打聊城。汉军派将军郭蒙和齐国将领出击,大败叛军。太尉周勃取道太原进军,平定了代地。来到马邑,马邑不投降,周勃就将其摧毁。

陈豨的部将赵利驻守东垣,高祖攻打东垣,不能攻下。一个多月后,叛军士卒辱骂高祖,高祖非常生气。东垣投降时,高祖命令交出辱骂自己的人,将其斩首,没有辱骂的就原谅了他们。因此划出赵国常山以北的土地,封儿子刘恒为代王,定都晋阳。

\begin{yuanwen}
春,淮阴侯韩信谋反关中\footnote{text},夷三族。
\end{yuanwen}

春季,淮阴侯韩信在关中阴谋反叛,诛灭三族。

\begin{yuanwen}
夏,梁王彭越谋反,废迁蜀;复欲反,遂夷三族\footnote{text}。立子恢为梁王,子友为淮阳王\footnote{text}。
\end{yuanwen}

夏季,梁王彭越阴谋反叛,废去封号迁徙蜀郡;他又想反叛,就将其诛灭三族。高祖封儿子刘恢为梁王,儿子刘友为淮阳王。

\begin{yuanwen}
秋七月,淮南王黥布反\footnote{text},东并荆王刘贾地,北渡淮,楚王交走入薛。高祖自往击之,立子长为淮南王。
\end{yuanwen}

秋季七月,淮南王黥布反叛,向东吞并荆王刘贾的封地,向北渡过淮水,楚王刘交逃到薛县。高祖亲自前去攻打他。封儿子刘长为淮南王。

\begin{yuanwen}
十二年,十月,高祖已击布军会甀\footnote{zhuì},布走,令别将追之。
\end{yuanwen}

十二年(前195年)十月,高祖在会甀攻打黥布的军队后,黥布逃走,高祖命令别将去追击。

\begin{yuanwen}
高祖还归,过沛,留。置酒沛宫\footnote{text},悉召故人父老子弟纵酒。发沛中儿得百二十人,教之歌。酒酣,高祖击筑,自为歌诗曰:“大风起兮云飞扬,威加海内兮归故乡,安得猛士兮守四方\footnote{text}!”

令儿皆和习之。高祖乃起舞,慷慨伤怀,泣数行下。谓沛父兄曰:“游子悲故乡。吾虽都关中,万岁后\footnote{对帝王之死的讳称。}吾魂魄犹乐思沛\footnote{text}。且朕自沛公以诛暴逆,遂有天下,其以沛为朕汤沐邑\footnote{以赋税提供食宿、斋戒、沐浴的私人领地。},复其民\footnote{text},世世无有所与。”

沛父兄诸母\footnote{众女性长辈。}故人日乐饮极欢,道旧故为笑乐。十馀日,高祖欲去,沛父兄固请留高祖。

高祖曰:“吾人众多,父兄不能给。”乃去。

沛中空县皆之邑西献\footnote{text}。高祖复留止,张饮三日\footnote{text}。沛父兄皆顿首曰:“沛幸得复,丰未复\footnote{text},唯陛下哀怜之。”

高祖曰:“丰吾所生长,极不忘耳,吾特为其以雍齿故反我为魏\footnote{text}。”

沛父兄固请,乃并复丰,比沛。于是拜沛侯刘濞为吴王。
\end{yuanwen}

高祖撤军返回,经过沛县,留下来。他在沛宫摆酒设宴,将昔日的朋友和父老子弟都召来纵情饮酒,征发沛县的儿童一百二十人,教他们唱歌。酒喝到畅快时,高祖击筑,亲自作诗唱道:“大风吹起来啊云彩飞扬,声威遍及海内啊回到故乡,如何得到勇士啊守卫四方!”

他命令那些儿童都跟着学唱。高祖于是起身跳舞,情绪激动而伤感,眼泪留下数行。他对沛县的百姓说:“远游之人思念故乡。我虽然定都关中,去世以后我的魂魄还是愿意回到沛县。况且我以沛公的身份诛灭暴逆,终于得到天下,以沛县为我的汤沐邑,免除沛县民众的赋役,世代都不用再征缴。”

沛县的男女长辈和朋友每天快乐饮酒,叙说旧事取笑为乐。过了十几天,高祖想要离开,沛县的百姓执意挽留高祖。

高祖说:“我的随行人员太多,父兄供养不起。”

于是离去。沛县百姓倾城而出,都来到城西进献酒食。高祖又留下来,设帷帐宴饮三天。沛县百姓都叩头说:“沛县有幸得以免除赋役,丰邑还没有免除。希望陛下哀怜丰邑百姓。”

高祖说:“丰邑是我生长的地方,我绝不能忘记,我只因为当地人跟随雍齿背叛我投靠魏国才不愿免除他们的赋役。”

沛县百姓执意请求,高祖才一并免除丰邑的赋役,与沛县相同。这时高祖封沛侯刘濞为吴王。

刘基:「汉兴,一扫衰周之文敝而还诸朴。《丰沛之歌》,雄伟不饰,移风易尚之机,实肇于此。」王世贞:「《大风》三言,气笼宇宙,张千古帝王赤帜。高帝哉!」

\begin{yuanwen}
汉将别击布军洮水南北\footnote{text},皆大破之,追得斩布鄱阳\footnote{text}。

樊哙别将兵定代,斩陈豨当城。
\end{yuanwen}

汉将在洮水南北两路分别追击黥布的军队,都大破叛军,在鄱阳追上黥布而将其斩首。

樊哙另外率领一支军队平定了代地,在当城将陈豨斩首。

\begin{yuanwen}
十一月,高祖自布军至长安。

十二月,高祖曰:“秦始皇帝、楚隐王陈涉、魏安釐王、齐缗王、赵悼襄王皆绝无后,予守冢各十家,秦皇帝二十家,魏公子无忌五家。”

赦代地吏民为陈豨、赵利所劫掠者,皆赦之。陈豨降将言豨反时,燕王卢绾使人之豨所,与阴谋。上使辟阳侯\footnote{审食其。}迎绾,绾称病。辟阳侯归,具言绾反有端矣。

二月,使樊哙、周勃将兵击燕王绾,赦燕吏民与反者。立皇子建为燕王。
\end{yuanwen}

十一月,高祖从讨伐黥布的前线返回长安。

十二月,高祖说:“秦始皇帝、楚隐王陈涉、魏安釐王、齐缗王、赵悼襄王都断绝祭祀没有后代,分别派守墓人十户,秦始皇帝为二十户,魏公子无忌为五户。”

赦免代地被陈豨、赵利所胁迫的官吏及百姓,全部赦免。陈豨的降将说陈豨在反叛时,燕王卢绾派人来到陈豨的住所,和他暗中谋划。高祖派辟阳侯去迎接卢绾,卢绾称病不来。辟阳侯返回,详细报告卢绾已经显现反叛的端绪了。

二月,高祖派樊哙、周勃率领军队进攻燕王卢绾,赦免燕国参与反叛的官员和百姓。封皇子刘建为燕王。

\begin{yuanwen}
高祖击布时,为流矢所中,行道病。病甚,吕后迎良医,医入见。高祖问医,医曰:“病可治\footnote{text}。”

于是高祖嫚骂之曰:“吾以布衣提三尺剑取天下,此非天命乎?命乃在天,虽扁鹊何益!”

遂不使治病,赐金五十斤罢之。

已而吕后问:“陛下百岁后,萧相国即死,令谁代之?”

上曰:“曹参可。”

问其次,上曰:“王陵可。然陵少戆\footnote{有些鲁莽。},陈平可以助之。陈平智有馀,然难以独任。周勃重厚少文\footnote{text},然安刘氏者必勃也,可令为太尉\footnote{text}。”

吕后复问其次,上曰:“此后亦非而所知也。”

卢绾与数千骑居塞下候伺,幸上病愈自入谢。
\end{yuanwen}

高祖攻打黥布时,被流箭所射中,行军途中生病。病情严重,吕后迎接良医。医生进宫拜见,高祖询问医生,医生说:“病能够治好。”

于是高祖谩骂医生说:“我以平民的身份手持三尺长剑夺取天下,这不是天命吗?性命由上天注定,即使扁鹊重生,又能有什么用处!”

高祖于是没有让医生治病,赏赐给他金五十斤,让他走了。

不久吕后问:“陛下去世以后,萧相国也死了,让谁去接替他?”

高祖说:“曹参可以。”

又问其次,高祖说:“王陵可以。然而王陵有些鲁莽,陈平可以协助他。陈平智慧有余,然而难以独自担当重任。周勃沉稳而缺少文采,然而能够安定刘氏的人一定是周勃,可以让他做太尉。”

吕后再问其次,高祖说:“这以后的事情也不是你所能知道的。”

卢绾率领几千名骑兵在边塞打听消息,希望在高祖病愈时亲自进宫谢罪。

程子:「凡读史不徒要记事迹,须要识治乱兴废存亡之理。如读《高帝一纪》,便须识得汉家四百年终始治乱当何如,是亦学也;观帝宽大长者,能用三杰,则知汉家所以立四百年基业;观伪游云梦,则知诸侯次第而畔;观萧何系狱,则知汉臣多不保终。」

\begin{yuanwen}
四月甲辰\footnote{text},高祖崩长乐宫。四日不发丧。吕后与审食其谋曰:“诸将与帝为编户民,今北面为臣,此常怏怏\footnote{不满意的样子。},今乃事少主,非尽族是,天下不安。”

人或闻之,语郦将军。郦将军往见审食其,曰:“吾闻帝已崩,四日不发丧,欲诛诸将。诚如此,天下危矣。陈平、灌婴将十万守荥阳,樊哙、周勃将二十万定燕、代,此闻帝崩,诸将皆诛,必连兵还乡以攻关中。大臣内叛,诸侯外反,亡可翘足而待也。”

审食其入言之,乃以丁未发丧,大赦天下。

卢绾闻高祖崩,遂亡入匈奴。
\end{yuanwen}

四月甲辰日,高祖在长乐宫去世。过了四天也不公布死讯。吕后与审食其商量说:“众将领和皇帝都出身编户平民,现在面朝北方向皇帝称臣,心里经常感到不满,现在又事奉年少的君主,不把他们全部灭族,天下就不会安定。”

有人听说这个消息,就对郦将军说了。郦将军前去见审食其,说:“我听说皇帝已经去世,四天仍不公布死讯,还想杀死众将领。如果真的这样做,天下就危险了。陈平、灌婴率领十万人驻守荥阳,樊哙、周勃率领二十万人平定燕国、代国,这时他们听说皇帝去世,众将领都被杀掉,一定会联合起来回军向关中进发。大臣在内部叛乱,诸侯在外面造反,灭亡可以踮脚等待了。”

审食其进宫转述这些话,才在丁未日公布死讯,大赦天下。

卢绾听说高祖去世,就逃到匈奴。

\begin{yuanwen}
丙寅,葬。己巳,立太子,至太上皇庙。群臣皆曰:“高祖起微细,拨乱世反之正,平定天下,为汉太祖,功最高。”

上尊号为高皇帝。太子袭号为皇帝,孝惠帝也。令郡国诸侯各立高祖庙,以岁时祠。
\end{yuanwen}

丙寅日,安葬高祖。己巳日,立太子为皇帝,来到太上皇庙。群臣都说:“高祖出身寒微,平定乱世使其重返正道,平定天下,是汉朝的太祖,功劳最高。”

上尊号为高皇帝。太子继承皇帝称号,这就是孝惠帝。朝廷命令郡国诸侯分别建立高祖庙,每年按照季节进行祭祀。

\begin{yuanwen}
及孝惠五年,思高祖之悲乐沛,以沛宫为高祖原庙。高祖所教歌儿百二十人,皆令为吹乐,后有缺,辄补之。
\end{yuanwen}

等到孝惠帝五年(前190年),想起高祖回到沛县时悲喜交加的情景,就将沛宫改建为高祖的原庙。高祖所教唱歌的儿童一百二十人,都让他们在高祖庙里吹奏音乐,以后有缺额,就立即补充。

\begin{yuanwen}
高帝八男:长庶齐悼惠王肥;次孝惠,吕后子;次戚夫人子赵隐王如意;次代王恆(恒),已立为孝文帝,薄太后子;次梁王恢,吕太后时徙为赵共王;次淮阳王友,吕太后时徙为赵幽王;次淮南厉王长;次燕王建。
\end{yuanwen}

高帝有八个儿子:庶长子齐悼惠王刘肥;其次是孝惠帝,吕后的儿子;其次是戚夫人的儿子赵隐王刘如意;其次是代王刘恒,后来被立为孝文帝,是薄太后的儿子;其次是梁王刘恢,吕太后时改封他为赵共王;其次是淮阳王刘友,吕太后时改封他为赵幽王;其次是淮南厉王刘长;其次是燕王刘建。

\begin{yuanwen}
太史公曰:夏之政忠\footnote{质朴。}。忠之敝,小人以野,故殷人承之以敬\footnote{崇信鬼神。}。敬之敝,小人以鬼,故周人承之以文\footnote{文采,指礼乐制度。}。文之敝,小人以僿\footnote{sài,不诚恳。},故救僿莫若以忠。三王之道若循环,终而复始。周秦之间,可谓文敝矣。秦政不改,反酷刑法,岂不缪乎?故汉兴,承敝易变,使人不倦,得天统矣。朝以十月。车服,黄屋左纛\footnote{dào,帝王车马上的羽毛饰物。}。葬长陵。
\end{yuanwen}

太史公说:夏朝时的政治质朴。质朴的缺点,就是平民粗野无礼,因此殷商以虔敬来取代它。虔敬的缺点,就是平民迷信鬼神,因此周朝以文采来取代它。文采的缺点,就是平民不够诚恳,因此补救不诚恳没有比质朴更好的了。三王的治国之道就像循环,终而复始。周朝到秦朝之间,可以说是文采的缺点暴露的时代。秦朝的政令没有改变这种情况,反而使用严刑酷法,难道不荒谬吗?因此汉朝兴起,面对过去的缺点改变了政策,使人们不感到疲倦,于是得到上天的统绪了。每年十月群臣诸侯进宫朝见。车马服饰的制度,皇帝车驾有黄色伞盖,左侧的马头插上羽毛饰物。高祖被安葬在长陵。

\begin{yuanwen}
高祖初起,始自徒中。言从泗上,即号沛公。啸命豪杰,奋发材雄。彤云郁砀,素灵告丰。龙变星聚,蛇分径空。项氏主命,负约弃功。王我巴蜀,实愤于衷。三秦既北,五兵遂东。氾水即位,咸阳筑宫。威加四海,还歌大风。
\end{yuanwen}

\part{卷九}
\chapter{吕太后本纪第九}

司马光:「为人子者,父母有过则几,谏不听则号泣随之。安有守高祖之业,为天下之主,不忍母之残酷,遂弃国家而不恤,纵酒色以伤生?若惠帝者,可谓笃于小仁,而未知大义也。」郭嵩焘:「案此《本纪》中明言『孝惠日饮,为淫乐,不听政』,是惠帝初立后,吕后专杀自恣,政由己出,固已久矣。史公不为惠帝立纪,以纪实也。」

本篇是高皇后吕雉的本纪,记述了从汉惠帝到她称制期间的重大事件。吕后尽管没有称帝,其权位却与帝王毫无二致,被《史记》列入本纪。虽然吕后的秉政引发诸吕之乱,但是司马迁仍然不得不肯定她在治理国家方面的杰出贡献。

\begin{yuanwen}
吕太后者,高祖微时\footnote{贫贱的时候。}妃也,生孝惠帝、女鲁元太后。及高祖为汉王,得定陶戚姬,爱幸,生赵隐王如意。孝惠为人仁弱,高祖以为不类我,常欲废太子,立戚姬子如意,如意类我。戚姬幸,常从上之关东,日夜啼泣,欲立其子代太子。吕后年长,常留守,希\footnote{同“稀”。}见上,益疏。如意立为赵王后,几代太子者数矣,赖大臣争\footnote{谏诤,劝阻。}之,及留侯策,太子得毋废。
\end{yuanwen}

吕太后,是高祖贫贱时的妻子,她生下孝惠帝、女儿鲁元太后。等到汉高祖成为汉王的时候,得到定陶人戚姬,很宠爱她,生下赵隐王刘如意。孝惠帝为人仁慈软弱,高祖认为他不像自己,总是想要废掉太子,立戚姬的儿子刘如意,因为刘如意像自己。戚姬受到宠幸,经常跟随高祖前往关东,她日夜哭泣,希望高祖立她的儿子来取代太子。吕后年纪大了,经常留守都城,很少见到高祖,夫妻关系更加疏远。刘如意被封为赵王以后,多次差点取代了太子,幸亏大臣劝阻,再加上留侯的计策,太子才没有被废掉。

\begin{yuanwen}
吕后为人刚毅,佐高祖定天下,所诛大臣多吕后力。吕后兄二人,皆为将。长兄周吕侯\footnote{吕泽。}死事,封其子吕台为郦侯,子产为交侯;次兄吕释之为建成侯。
\end{yuanwen}

吕后为人刚强坚毅,辅佐高祖平定天下,诛杀大臣大多是吕后的计划。吕后有两个哥哥,都担任将领。大哥周吕侯为国而死,高祖封他的儿子吕台为郦侯,另一个儿子吕产为交侯;二哥吕释之被封为建成侯。

\begin{yuanwen}
高祖十二年四月甲辰,崩长乐宫,太子袭号为帝。是时高祖八子:长男肥,孝惠兄也,异母,肥为齐王;馀皆孝惠弟,戚姬子如意为赵王,薄夫人子恆为代王,诸姬子子恢为梁王,子友为淮阳王,子长为淮南王,子建为燕王。高祖弟交为楚王,兄子濞为吴王。非刘氏功臣番君吴芮子臣为长沙王。
\end{yuanwen}

高祖十二年(前195年)四月甲辰日,高祖在长乐宫去世,太子继承皇帝称号。当时高祖有八个儿子:长子刘肥,是孝惠帝的哥哥,不是同母所生,被封为齐王;其余都是孝惠帝的弟弟,戚姬的儿子刘如意被封为赵王,薄夫人的儿子刘恒被封为代王,其他姬妾的儿子刘恢被封为梁王,刘友被封为淮阳王,刘长被封为淮南王,刘建被封为燕王。高祖的弟弟刘交被封为楚王,哥哥的儿子刘濞被封为吴王。不属于刘氏的功臣番君吴芮的儿子吴臣被封为长沙王。

\begin{yuanwen}
吕后最怨戚夫人及其子赵王,乃令永巷\footnote{宫中囚禁嫔妃、宫女的地方。}囚戚夫人,而召赵王。使者三反,赵相建平侯周昌谓使者曰:“高帝属臣赵王,赵王年少。窃闻太后怨戚夫人,欲召赵王并诛之,臣不敢遣王。王且亦病,不能奉诏。”

吕后大怒,乃使人召赵相。赵相徵至长安,乃使人复召赵王。王来,未到。孝惠帝慈仁,知太后怒,自迎赵王霸上,与入宫,自挟与赵王起居饮食。太后欲杀之,不得间。
\end{yuanwen}

吕后最怨恨戚夫人和她的儿子赵王,于是下令将戚夫人囚禁在永巷,并且召见赵王。使者往返多次,赵相建平侯周昌对使者说:“高帝把赵王托付给我,现在赵王年纪小。我私下里听说太后怨恨戚夫人,想要征召赵王一并诛杀,我不敢送走赵王。况且赵王也生病了,不能遵奉诏令。”

吕后非常生气,就派人召见赵相。赵相应征来到长安,吕后就派人再去召见赵王。赵王来了,还没有到达。孝惠帝慈爱仁厚,知道太后生气了,就亲自到霸上迎接赵王,和他一起进宫,亲自带着赵王一同坐卧饮食。太后想要杀死赵王,却找不到机会。

\begin{yuanwen}
孝惠元年十二月,帝晨出射。赵王少,不能蚤起。太后闻其独居,使人持酖饮之。犁明\footnote{黎明。},孝惠还,赵王已死。于是乃徙淮阳王友为赵王。
\end{yuanwen}

孝惠帝元年(前194年)十二月,孝惠帝早晨外出射猎。赵王年纪小,不能早起。太后听说他独自在宫中,就派人带上毒酒给他喝。黎明时分,孝惠帝回来,赵王已经死了。于是朝廷改封淮阳王刘友为赵王。

\begin{yuanwen}
夏,诏赐郦侯父追谥为令武侯。
\end{yuanwen}

夏季,孝惠帝下诏令追谥郦侯的父亲为令武侯。

\begin{yuanwen}
太后遂断戚夫人手足,去眼,煇\footnote{同“熏”。}耳,饮瘖\footnote{哑。}药,使居厕中,命曰“人彘”。居数日,乃召孝惠帝观人彘。孝惠见,问,乃知其戚夫人,乃大哭,因病,岁馀不能起。使人请太后曰:“此非人所为。臣为太后子,终不能治天下。”

孝惠以此日饮为淫乐,不听政,故有病也。
\end{yuanwen}

太后终于砍断戚夫人的手脚,挖去双眼,熏聋双耳,灌下哑药,把她扔到厕所里,称她为“人猪”。过了几天,她让孝惠帝前去观看人猪。孝惠帝看到后,问别人,才知道是戚夫人,于是大哭,因此生病,卧床一年多不能起来。他派人向太后请示说:“这不是人能做出来的事情。我是太后的儿子,终究还是不能治理天下。”

孝惠帝因此每天放纵地饮酒取乐,不处理朝政,所以身患疾病。

\begin{yuanwen}
二年,楚元王、齐悼惠王皆来朝。

十月,孝惠与齐王燕饮太后前,孝惠以为齐王兄,置上坐,如家人之礼。太后怒,乃令酌两卮酖,置前,令齐王起为寿\footnote{敬酒祝福长寿。}。齐王起,孝惠亦起,取卮欲俱为寿。太后乃恐,自起泛\footnote{倾覆,打翻。}孝惠卮。齐王怪之,因不敢饮,详醉去。问,知其酖,齐王恐,自以为不得脱长安,忧。齐内史士说王曰:“太后独有孝惠与鲁元公主。今王有七十馀城,而公主乃食数城。王诚以一郡上太后,为公主汤沐邑\footnote{以赋税提供食宿、斋戒、沐浴的私人领地。},太后必喜,王必无忧。”

于是齐王乃上城阳之郡,尊公主为王太后\footnote{鲁元公主的丈夫是赵王张敖,因贯高行刺事件废为侯,其子张偃在吕后时封鲁王,所以称公主为王太后。}。吕后喜,许之。乃置酒齐邸,乐饮,罢,归齐王。
\end{yuanwen}

二年(前193年),楚元王、齐悼惠王都来朝见。

十月,孝惠帝和齐王在太后面前设宴饮酒,孝惠帝认为齐王是哥哥,就安排他坐在上首,按照普通百姓家的礼节。太后很生气,就命人倒了两杯毒酒,放在前面,命令齐王起身为自己敬酒。齐王起身,孝惠帝也起身,取过一个酒杯想要和齐王一起敬酒。太后这才感到害怕,亲自起身打翻了孝惠帝的杯子。齐王感到奇怪,因而没敢喝这杯酒,假装醉酒离开了。后来经过询问,才知道是毒酒,齐王很害怕,认为自己无法从长安逃脱,为此忧虑。齐国内史士劝齐王说:“太后只有孝惠帝和鲁元公主两个孩子。现在大王有七十多座城邑,而公主只有几座城为食邑。假如大王将一个郡献给太后,作为公主的汤沐邑,太后一定会很高兴,大王就一定没有忧虑了。”

于是齐王将城阳郡献给公主,尊崇公主为王太后。吕后非常高兴,答应了齐王的请求,就在齐王的官邸摆酒设宴,畅快饮酒,宴会结束后,就让齐王回去了。

\begin{yuanwen}
三年,方筑长安城,四年就半,五年六年城就。诸侯来会。十月,朝贺。
\end{yuanwen}

三年(前192年),开始修筑长安城,四年(前191年)完成一半,经过五年(前190年)和六年(前189年)最终完工。诸侯都来聚会。十月,诸侯入朝庆贺。

\begin{yuanwen}
七年秋八月戊寅,孝惠帝崩。发丧,太后哭,泣不下。留侯子张辟彊为侍中,年十五,谓丞相曰:“太后独有孝惠,今崩,哭不悲,君知其解\footnote{原因。}乎?”

丞相曰:“何解?”

辟彊曰:“帝毋壮子,太后畏君等。君今请拜吕台、吕产、吕禄为将,将兵居南北军,及诸吕皆入宫,居中用事,如此则太后心安,君等幸得脱祸矣。”

丞相乃如辟彊计。太后说\footnote{yuè},其哭乃哀。吕氏权由此起。乃大赦天下。

九月辛丑,葬。太子即位为帝,谒高庙。
\end{yuanwen}

七年(前188年)秋季八月戊寅日,孝惠帝去世。举行丧礼时,太后哭了,却没有流泪。留侯的儿子张辟彊担任侍中,年仅十五岁,对丞相说:“太后只有孝惠帝一个儿子,现在去世了,却哭得不悲伤,您知道其中的原因吗?”

丞相说:“是什么原因?”

张辟彊说:“皇帝没有成年的儿子,太后惧怕各位。您现在请太后任命吕台、吕产、吕禄为将军,统领南北军,等到吕氏都进入宫中,在朝廷掌握大权,这样太后才会心安,各位就能侥幸摆脱灾祸了。”

丞相就按照张辟彊的计策去做了。太后很高兴,她哭得才悲伤起来。吕氏的权势由此建立。于是朝廷大赦天下。

九月辛丑日,安葬孝惠帝。太子即位为皇帝,拜谒高祖庙。

\begin{yuanwen}
元年\footnote{汉惠帝死后,吕太后开始称制,史称高皇后元年(“高”为谥号),因此她废立皇帝而不改元。},号令一出太后。

太后称制\footnote{代行皇帝职权。},议欲立诸吕为王,问右丞相王陵。王陵曰:“高帝刑白马盟曰‘非刘氏而王,天下共击之’。今王吕氏,非约也。”

太后不说。问左丞相陈平、绛侯周勃。勃等对曰:“高帝定天下,王子弟,今太后称制,王昆弟诸吕,无所不可。”

太后喜,罢朝。王陵让陈平、绛侯曰:“始与高帝(喋|啑)\footnote{同“歃”。}血盟,诸君不在邪?今高帝崩,太后女主,欲王吕氏,诸君从欲阿意背约,何面目见高帝地下?”

陈平、绛侯曰:“于今面折\footnote{当面指责别人。}廷争,臣不如君;夫全社稷,定刘氏之后,君亦不如臣。”

王陵无以应之。

十一月,太后欲废王陵,乃拜为帝太傅,夺之相权。王陵遂病免归。乃以左丞相平为右丞相,以辟阳侯审食其为左丞相。左丞相不治事,令监宫中,如郎中令。食其故得幸太后,常用事,公卿皆因而决事。乃追尊郦侯父为悼武王,欲以王诸吕为渐。
\end{yuanwen}

元年(前187年),号令都由太后发出。

太后代行皇帝职权,想要商议封吕氏子弟为王,询问右丞相王陵。王陵说:“高帝杀白马盟誓说:‘不是刘氏子弟而称王的,天下人一起攻打他。’现在封吕氏子弟为王,违背了高帝的盟约。”

太后很不高兴。她又询问左丞相陈平、绛侯周勃。周勃等人回答说:“高帝平定天下,封子弟为王,现在太后代行皇帝职权,封弟兄和吕氏子弟为王,没什么不可以的。”

太后很高兴,退朝。王陵责备陈平、绛侯说:“当初和高帝歃血为盟,各位不在场吗?现在高帝去世了,太后以女人的身份当权,想要封吕氏子弟为王,各位纵使想要阿谀逢迎违背盟约,又有什么颜面去地下见高帝呢?”

陈平、绛侯说:“今天在当面指责别人,在朝堂上直言进谏,我们不如您;保全国家社稷,安定刘氏后裔,您又不如我们。”

王陵没有办法回应他们。

十一月,太后想要罢免王陵,就任命他为皇帝的太傅,剥夺了他的丞相职权。王陵于是称病免职回家。吕后就任命左丞相陈平为右丞相,任命辟阳侯审食其为左丞相。左丞相不处理政务,只是监督宫中的事情,就像郎中令一样。审食其因此深受太后宠幸,经常参与朝政,公卿都依靠他来决断政事。这时太后追尊郦侯的父亲为悼武王,想要以此为封吕氏子弟为王的开端。

\begin{yuanwen}
四月,太后欲侯诸吕,乃先封高祖之功臣郎中令无择为博城侯。鲁元公主薨,赐谥为鲁元太后。子偃为鲁王。鲁王父,宣平侯张敖也。封齐悼惠王子章为硃虚侯,以吕禄女妻之。齐丞相寿为平定侯。少府延为梧侯。乃封吕种为沛侯,吕平为扶柳侯,张买为南宫侯。
\end{yuanwen}

四月,太后想要封吕氏子弟为侯,就先封高祖的功臣郎中令冯无择为博城侯。鲁元公主去世,赐谥号为鲁元太后。鲁元公主的儿子张偃被封为鲁王。鲁王的父亲,就是宣平侯张敖。太后封齐悼惠王的儿子刘章为朱虚侯,把吕禄的女儿嫁给他为妻。齐丞相齐寿被封为平定侯。少府阳成延被封为梧侯。于是封吕种为沛侯,封吕平为扶柳侯,封张买为南宫侯。

\begin{yuanwen}
太后欲王吕氏,先立孝惠后宫子彊为淮阳王,子不疑为常山王,子山为襄城侯,子朝为轵侯,子武为壶关侯。太后风\footnote{通“讽”,婉言劝说,暗示。}大臣,大臣请立郦侯吕台为吕王,太后许之。建成康侯释之卒,嗣子有罪,废,立其弟吕禄为胡陵侯,续康侯后。
\end{yuanwen}

太后想要封吕氏子弟为王,先封孝惠帝后宫所生的儿子刘彊为淮阳王,刘不疑为常山王,刘山为襄城侯,刘朝为轵侯,刘武为壶关侯。太后暗示大臣,于是大臣请求封郦侯吕台为吕王,太后批准了。建成康侯吕释之去世,嗣子有罪,被废黜,太后封他的弟弟吕禄为胡陵侯,来延续康侯的后代。

\begin{yuanwen}
二年,常山王薨,以其弟襄城侯山为常山王,更名义。

十一月,吕王台薨,谥为肃王,太子嘉代立为王。
\end{yuanwen}

二年(前186年),常山王去世,封他的弟弟襄城侯刘山为常山王,改名刘义。

十一月,吕王吕台去世,谥号为肃王,太子吕嘉继位为王。

\begin{yuanwen}
三年,无事。

四年,封吕嬃\footnote{xū}为临光侯,吕他为俞侯,吕更始为赘其侯,吕忿为吕城侯,及诸侯丞相五人。
\end{yuanwen}

三年(前185年),没有重大事件。

四年(前184年),封吕媭为临光侯,封吕他为俞侯,封吕更始为赘其侯,封吕忿为吕城侯,同时封赏诸侯和丞相五人。

\begin{yuanwen}
宣平侯女为孝惠皇后时,无子,详\footnote{通“佯”,假装。}为有身,取美人子名之,杀其母,立所名子为太子。孝惠崩,太子立为帝。帝壮,或闻其母死,非真皇后子,乃出言曰:“后安能杀吾母而名我?我未壮,壮即为变。”

太后闻而患之,恐其为乱,乃幽之永卷中,言帝病甚,左右莫得见。太后曰:“凡有天下治为万民命者,盖之如天,容之如地,上有欢心以安百姓,百姓欣然以事其上,欢欣交通而天下治。今皇帝病久不已,乃失惑惛乱,不能继嗣奉宗庙祭祀,不可属天下,其代之。”

群臣皆顿首言:“皇太后为天下齐民计,所以安宗庙社稷甚深,群臣顿首奉诏。”

帝废位,太后幽杀之。

五月丙辰,立常山王义为帝,更名曰弘。不称元年者,以太后制天下事也。以轵\footnote{zhǐ}侯朝为常山王。置太尉官,绛侯勃为太尉。
\end{yuanwen}

宣平侯的女儿做孝惠帝的皇后时,没有生下儿子,就假装怀孕,抱来一个美人的儿子做自己的儿子,将孩子的母亲杀死,立这个收养的孩子为太子。孝惠帝去世后,太子被立为皇帝。皇帝长大后,从别处听说自己的母亲已经死了,自己并不是皇后的亲生儿子,就说出这样的话:“皇后怎么能杀死我的母亲而让我做她的儿子呢?我还没有长大,等我长大后一定要作乱。”

太后听说后为此担忧,害怕他作乱,就将他幽禁在永巷中,声称皇帝病重,身边的侍臣也不能见到皇帝。太后说:“凡是拥有天下治理万民的人,应该像苍天一样覆盖众生,像大地一样容纳万物,皇帝怀有欢愉之心来安定百姓,百姓就会高兴地事奉皇帝,欢愉和高兴相互交融,天下就能太平。现在皇帝久病不愈,已经迷惑昏乱,不能继承主持宗庙祭祀的重任,不能让他来治理天下,应该找人来取代他。”

群臣都叩头说:“皇太后为天下民众考虑,为安定国家社稷而长远规划,我们叩头遵奉诏令。”

皇帝被废黜,太后将他暗杀。

五月丙辰日,立常山王刘义为皇帝,改名刘弘。皇帝即位不称元年,是因为太后代行皇帝职权处理天下大事。封轵侯刘朝为常山王。设置太尉一职,任命绛侯周勃为太尉。

\begin{yuanwen}
五年八月,淮阳王薨,以弟壶关侯武为淮阳王。

六年十月,太后曰吕王嘉居处骄恣,废之,以肃王台弟吕产为吕王。夏,赦天下。封齐悼惠王子兴居为东牟侯。
\end{yuanwen}

五年(前183年)八月,淮阳王去世,封他的弟弟壶关侯刘武为淮阳王。

六年(前182年)十月,太后认为吕王吕嘉平时骄横放纵,将他废黜,封肃王吕台的弟弟吕产为吕王。夏季,大赦天下。封刘悼惠王的儿子刘兴居为东牟侯。

\begin{yuanwen}
七年正月,太后召赵王友。友以诸吕女为(受)后,弗爱,爱他姬,诸吕女妒,怒去,谗之于太后,诬以罪过,曰:“吕氏安得王!太后百岁后,吾必击之”。

太后怒,以故召赵王。赵王至,置邸不见,令卫围守之,弗与食。其群臣或窃馈,辄捕论之,赵王饿,乃歌曰:“诸吕用事兮刘氏危,迫胁王侯兮彊授我妃。我妃既妒兮诬我以恶,谗女乱国兮上曾不寤。我无忠臣兮何故弃国?自决中野兮苍天举直!于嗟不可悔兮宁蚤自财\footnote{通“裁”。}。为王而饿死兮谁者怜之!吕氏绝理兮(讬/托)天报仇。”

丁丑,赵王幽死,以民礼葬之长安民冢次。

己丑,日食,昼晦。太后恶之,心不乐,乃谓左右曰:“此为我也。”
\end{yuanwen}

七年(前181年)正月,太后召见赵王刘友。刘友娶吕氏的女子为王后,却不喜欢她,宠爱别的姬妾。这个吕氏的女子心怀嫉妒,生气地离去,在太后面前说赵王的坏话,诬告他有罪说:“吕氏怎么能封王!太后去世以后,我一定会除掉他们。”

太后很生气,因为这个缘故召见赵王。赵王到来,被安置在府邸而不接见,命令卫士将他围困起来,不给他食物。赵王的群臣中有人偷偷给他送饭吃,被发现就逮捕论罪。赵王感到饥饿,就唱道:“吕氏当权啊刘氏危急,胁迫王侯啊强令我娶妻。王妃嫉妒啊以恶名相诬,妒妇乱国啊皇帝不醒悟。没有忠臣啊为何离封国?野外自尽啊苍天伸冤屈!唉,无法后悔啊宁愿早自裁。封王却饿死啊谁能怜惜!吕氏无理啊托上天报仇。”

丁丑日,赵王被幽禁而死,以平民的礼仪将他埋葬在长安百姓的墓地。

己丑日,发生日食,白天很昏暗。太后很厌恶,心里不高兴,就对身边的侍臣说:“这是因为我的缘故。”

\begin{yuanwen}
二月,徙梁王恢为赵王。吕王产徙为梁王,梁王不之国,为帝太傅。立皇子平昌侯太为吕王。更名梁曰吕,吕曰济川。太后女弟\footnote{妹。}吕嬃有女为营陵侯刘泽妻,泽为大将军。太后王诸吕,恐即崩后刘将军为害,乃以刘泽为琅邪王,以慰其心。

梁王恢之徙王赵,心怀不乐。太后以吕产女为赵王后。王后从官皆诸吕,擅权,微伺赵王,赵王不得自恣。王有所爱姬,王后使人酖杀之。王乃为歌诗四章,令乐人歌之。王悲,六月即自杀。太后闻之,以为王用妇人弃宗庙礼,废其嗣。

宣平侯张敖卒,以子偃为鲁王,敖赐谥为鲁元王。
\end{yuanwen}

二月,太后改封梁王刘恢为赵王。吕王吕产改封为梁王,梁王不去封国,担任皇帝的太傅。太后封皇子昌平侯刘太为吕王。梁国改名为吕国,吕国改名为济川国。太后的妹妹吕媭有一个女儿是营陵侯刘泽的妻子,刘泽担任大将军。太后封吕氏子弟为王,担心自己去世后刘将军作乱,就封刘泽为琅邪王,来安抚他。

梁王刘恢改封赵王,心里很不高兴。太后将吕产的女儿嫁给赵王为王后。王后的侍从官都是吕氏子弟,他们独断专行,暗中监视赵王,赵王不能自作主张。赵王有一个他所宠爱的姬妾,王后命人将她毒死了。赵王于是创作了诗歌四章,命令乐工演唱。赵王很悲伤,六月就自杀了。太后听说后,认为赵王为了女人背弃宗庙礼法,废黜他的嗣子的继承权。

宣平侯张敖去世,因他的儿子张偃是鲁王,所以追赐张敖谥号为鲁元王。

\begin{yuanwen}
秋,太后使使告代王,欲徙王赵。代王谢,原守代边。

太傅产、丞相平等言,武信侯吕禄上侯,位次第一,请立为赵王。太后许之,追尊禄父康侯为赵昭王。

九月,燕灵王建薨,有美人子,太后使人杀之,无后,国除。
\end{yuanwen}

秋季,太后派使者告诉代王,想要改封他为赵王。代王谢绝了,希望在代国戍守边疆。

太傅吕产、丞相陈平等人进言,武信侯吕禄是上等侯爵,在列侯中位列第一,请求封他为赵王。太后批准了,追尊吕禄的父亲建成康侯为赵昭王。

九月,燕灵王刘建去世,他的美人生有一个儿子,太后派把这个孩子杀死,导致他没有子嗣,封国被削除。

\begin{yuanwen}
八年十月,立吕肃王子东平侯吕通为燕王,封通弟吕庄为东平侯。

三月中,吕后祓\footnote{一种禳灾祈福的祭祀。},还过轵道,见物如苍犬,据\footnote{《汉书·五行志》、《前汉纪·高后纪》等书作“橶”或“撠”,有刺击、抓住的意思,可以理解为扑咬。}高后掖\footnote{通“腋”。},忽弗复见。卜之,云赵王如意为祟。高后遂病掖伤。

高后为外孙鲁元王偃\footnote{张敖谥元王,此处“元”为衍字。}年少,蚤失父母,孤弱,乃封张敖前姬两子,侈为新都侯,寿为乐昌侯,以辅鲁元王偃。及封中大谒者张释为建陵侯,吕荣为祝兹侯。诸中宦者令丞皆为关内侯,食邑五百户。

七月中,高后病甚,乃令赵王吕禄为上将军,军北军;吕王产居南军。吕太后诫产、禄曰:“高帝已定天下,与大臣约,曰‘非刘氏王者,天下共击之’。今吕氏王,大臣弗平。我即崩,帝年少,大臣恐为变。必据兵卫宫,慎毋送丧,毋为人所制。”

辛巳,高后崩,遗诏赐诸侯王各千金,将相、列侯、郎吏皆以秩赐金。大赦天下。以吕王产为相国,以吕禄女为帝后。

高后已葬,以左丞相审食其为帝太傅。
\end{yuanwen}

八年(前180年)十月,太后封吕肃王的儿子东平侯吕通为燕王,封吕通的弟弟吕庄为东平侯。

三月中旬,吕后举行祓祭,回来经过轵道,看见一个形似黑狗的东西,扑咬她的腋下,忽然又不见了。她找人占卜,说是赵王刘如意的鬼魂作祟。高后于是因为腋下的伤而生病。

高后因为外孙鲁王张偃年纪小,很早就失去父母,孤立无助,就分封张敖以前姬妾的两个儿子,张侈为新都侯,张寿为乐昌侯,以此辅佐鲁王张偃。又封中大谒者张释为建陵侯,吕荣为祝兹侯。宫中担任令和丞的宦者都被封为关内侯,每个人食邑五百户。七月中旬,高后病情严重,于是她任命赵王吕禄为上将军,统领北军;吕王吕产统领南军。吕太后告诫吕产、吕禄说:“高帝平定天下以后,和大臣约定,说:‘不是刘氏子弟而称王的,天下人一起攻打他。’现在吕氏封王,大臣心中不满。我就要死了,皇帝年纪小,大臣恐怕会作乱。你们一定要掌管军队保卫皇宫,千万不要给我送丧,不要被别人所控制。”

八月辛巳日,高后去世,留下诏书赏赐诸侯王每个人一千金,将相、列侯、郎官都按照品秩赏赐金钱。大赦天下。高后任命吕王吕产为相国,立吕禄的女儿为皇后。

高后被安葬以后,朝廷任命左丞相审食为皇帝的太傅。

\begin{yuanwen}
(硃/朱)虚侯刘章有气力,东牟侯兴居其弟也。皆齐哀王\footnote{刘襄。}弟,居长安。当是时,诸吕用事擅权,欲为乱,畏高帝故大臣绛、灌等,未敢发。硃虚侯妇,吕禄女,阴知其谋。恐见诛,乃阴令人告其兄齐王,欲令发兵西,诛诸吕而立。硃虚侯欲从中与大臣为应。齐王欲发兵,其相弗听。

八月丙午,齐王欲使人诛相,相召平乃反,举兵欲围王,王因杀其相,遂发兵东,诈夺琅邪王兵,并将之而西。语在齐王语\footnote{指《齐悼惠王世家》。}中。
\end{yuanwen}

朱虚侯刘章有勇力,东牟侯刘兴居是他的弟弟,他们都是齐哀王的弟弟,居住在长安。在这个时候,吕氏当政专权,想要作乱,畏惧高帝的旧臣绛侯、灌婴等人,还没有敢行动。朱虚侯的妻子,是吕禄的女儿,他在暗中打听到他们的阴谋。他担心自己被诛杀,就暗中派人告诉他的哥哥齐王,想要让他发兵西进,诛灭吕氏而自立为帝。朱虚侯想要在宫中和大臣做内应。齐王想要发兵,他的相国不服从命令。

八月丙午日,齐王想要派人诛杀相国,相国召平竟然反叛,举兵想要围攻齐王,齐王趁机杀死了他的相国,终于发兵东进,用计谋夺取了琅邪王的军队,合并两支军队向西进发。相关记载在齐王的传记中。

\begin{yuanwen}
齐王乃遗诸侯王书曰:“高帝平定天下,王诸子弟,悼惠王王齐。悼惠王薨,孝惠帝使留侯良立臣为齐王。孝惠崩,高后用事,春秋高,听诸吕,擅废帝更立,又比杀三赵王,灭梁、赵、燕以王诸吕,分齐为四。忠臣进谏,上惑乱弗听。今高后崩,而帝春秋富,未能治天下,固恃大臣诸侯。而诸吕又擅自尊官,聚兵严威,劫列侯忠臣,矫制以令天下,宗庙所以危。寡人\footnote{王侯的自称。}率兵入诛不当为王者。”

汉闻之,相国吕产等乃遣颍阴侯灌婴将兵击之。灌婴至荥阳,乃谋曰:“诸吕权兵关中,欲危刘氏而自立。今我破齐还报,此益吕氏之资也。”

乃留屯荥阳,使使谕齐王及诸侯,与连和,以待吕氏变,共诛之。齐王闻之,乃还兵西界待约。
\end{yuanwen}

齐王于是写信给诸侯王说:“高帝平定天下,封子弟们为王,悼惠王被封为齐王。悼惠王去世以后,孝惠帝派留侯张良立我为齐王。孝惠帝去世,高后当政,她年纪大了,听信吕氏子弟的话,擅自废立皇帝,又接连杀死三位赵王,灭掉梁国、赵国、燕国来封吕氏子弟为王,齐国也被一分为四。忠臣进言劝谏,高后迷惑昏乱不能采纳。现在高后去世,而皇帝还年轻,不能治理天下,当然要依靠大臣和诸侯的力量。可是吕氏擅自任命自家人为高官,聚集军队树立威严,胁迫列侯和忠臣,假传制命来号令天下,宗庙因此陷入危亡。我要率领军队进宫诛杀那些不应当被封为王的人。”

汉朝廷得知这一消息,相国吕产等人就派颍阴侯灌婴率领军队去迎战。灌婴来到荥阳,就与人谋划说:“吕氏控制了关中的军队,想要危害刘氏而自立为皇帝。现在我打败齐军回去复命,这是在增加吕氏作乱的资本。”

于是他留在荥阳驻扎,派使者告知齐王和诸侯,和他们结盟,来等待吕氏作乱,共同将其诛灭。齐王听说后,就撤军回到齐国的西部边界,按照约定等待时机。

\begin{yuanwen}
吕禄、吕产欲发乱关中,内惮绛侯、硃虚等,外畏齐、楚兵,又恐灌婴畔之,欲待灌婴兵与齐合而发,犹豫未决。当是时,济川王太、淮阳王武、常山王朝名为少帝弟,及鲁元王吕后外孙,皆年少未之国,居长安。赵王禄、梁王产\footnote{梁国已经改名为吕国。}各将兵居南北军,皆吕氏之人。列侯群臣莫自坚其命。
\end{yuanwen}

吕禄、吕产想要在关中发动叛乱,对内忌惮绛侯、朱虚侯等人,对外畏惧齐国、楚国的军队,又担心灌婴背叛他们,想要等到灌婴的军队与齐国的军队交战再行动,因此犹豫不决。在这个时候,济川王刘太、淮阳王刘武、常山王刘朝名义上是少帝的弟弟,以及吕后的外孙鲁王,都因为年纪小而没有前往封国,居住在长安。赵王吕禄、吕王吕产各自率领军队驻扎在南北军,他们都是吕氏的人。列侯和群臣没有人坚信自己能够保全性命。

\begin{yuanwen}
太尉绛侯勃不得入军中主兵。曲周侯郦商老病,其子寄与吕禄善。绛侯乃与丞相陈平谋,使人劫郦商。令其子寄往绐\footnote{欺骗。}说吕禄曰:“高帝与吕后共定天下,刘氏所立九王,吕氏所立三王,皆大臣之议,事已布告诸侯,诸侯皆以为宜。今太后崩,帝少,而足下佩赵王印,不急之国守籓,乃为上将,将兵留此,为大臣诸侯所疑。足下何不归印,以兵属太尉?请梁王归相国印,与大臣盟而之国,齐兵必罢,大臣得安,足下高枕而王千里,此万世之利也。”

吕禄信然其计,欲归将印,以兵属太尉。使人报吕产及诸吕老人,或以为便,或曰不便,计犹豫未有所决。吕禄信郦寄,时与出游猎。过其姑吕嬃,嬃大怒,曰:“若为将而弃军,吕氏今无处矣。”

乃悉出珠玉宝器散堂下,曰:“毋为他人守也”

左丞相食其免。
\end{yuanwen}

太尉绛侯周勃无法进入军中掌管兵权。曲周侯郦商年老多病,他的儿子郦寄和吕禄关系很好。绛侯就和丞相陈平谋划,派人劫持郦商,命令他的儿子郦寄前去欺骗吕禄说:“高帝和吕后一起平定天下,刘氏有九人被封为王,吕氏有三人被封为王,都是大臣商议决定的,事情已经通告诸侯,诸侯都认为很妥当。现在太后去世了,皇帝年少,而您佩带赵王的印玺,不赶快回到封国守卫国土,却担任上将军,率领军队留在这里,被大臣和诸侯所猜疑。您为什么不交还将军印,把军队交给太尉呢?也请吕王交还他的相国印,和大臣盟誓后回到封国,这样齐国的军队一定会撤退,大臣也能够安心,您就可以高枕无忧地统治方圆千里的国土了,这是传承万世的好事。”

吕禄相信了他的建议,想要交还将军印,把军队交给太尉。他派人报告吕产和吕氏的老人,有的人认为这样做可行,有的人认为不可行,正在犹豫而尚未决定。吕禄信任郦寄,经常和他外出狩猎。有一次经过姑母吕媭家,吕媭非常生气,说:“你身为将军却要放弃军队,吕氏现在没有安身的地方了。”

于是她拿出全部珠玉宝器散放在堂下,说:“不要为别人看守这些东西了。”

左丞相审食其被罢免。

\begin{yuanwen}
八月庚申旦,平阳侯窋行御史大夫事,见相国产计事。郎中令贾寿使从齐来,因数产曰:“王不蚤之国,今虽欲行,尚可得邪?”

具以灌婴与齐、楚合从,欲诛诸吕告产,乃趣\footnote{cù}产急入宫。平阳侯颇闻其语,乃驰告丞相、太尉。太尉欲入北军,不得入。襄平侯通尚\footnote{掌管。}符节。乃令持节矫内\footnote{nà}太尉北军。太尉复令郦寄与典客刘揭先说吕禄曰:“帝使太尉守北军,欲足下之国,急归将印辞去,不然,祸且起。”

吕禄以为郦兄\footnote{郦寄,字况。兄,kuàng,通“况”。}不欺己,遂解印属典客,而以兵授太尉。太尉将之入军门,行令军中曰:“为吕氏右襢(袒),为刘氏左襢(袒)。”

军中皆左襢为刘氏。太尉行至,将军吕禄亦已解上将印去,太尉遂将北军。
\end{yuanwen}

八月庚申日清晨,代行御史大夫职事的平阳侯曹窋,拜见相国吕产议事。郎中令贾寿出使齐国归来,趁机责备吕产说:“大王不早些回到封国,现在即使想去,还能去吗?”

贾寿把灌婴与齐国、楚国结盟,想要诛灭吕氏的事情告诉吕产,于是催促他赶快进宫。平阳侯大致听到了这些话,于是驱车报告丞相、太尉。太尉想要进入北军,不能进去。襄平侯纪通主管符节。太尉于是命令襄平侯手持符节假传诏令让太尉进入北军。太尉又命令郦寄和典客刘揭先去劝吕禄说:“皇帝派太尉掌管北军,想要您回到封国,赶快交还将军印辞别,不这样做,将会引起灾祸。”

吕禄认为郦兄不会欺骗自己,于是解下将军印交给典客,又把军队交给太尉。太尉率领这支军队进入军门,在军中发布命令说:“支持吕氏的袒露右臂,支持刘氏的袒露左臂。”

军中都袒露左臂支持刘氏。太尉来到军营,将军吕禄也已经交出将军印离去了,太尉终于得以率领北军。

\begin{yuanwen}
然尚有南军。平阳侯闻之,以吕产谋告丞相平,丞相平乃召硃虚侯佐太尉。太尉令硃虚侯监军门。令平阳侯告卫尉:“毋入相国产殿门。”

吕产不知吕禄已去北军,乃入未央宫,欲为乱,殿门弗得入,裴回\footnote{同“徘徊”。}往来。平阳侯恐弗胜,驰语太尉。太尉尚恐不胜诸吕,未敢讼言\footnote{公开说。}诛之,乃遣硃虚侯谓曰:“急入宫卫帝。”

硃虚侯请卒,太尉予卒千馀人。入未央宫门,遂见产廷中。日餔\footnote{傍晚。}时,遂击产。产走,天风大起,以故其从官乱,莫敢斗。逐产,杀之郎中府吏厕中。
\end{yuanwen}

然而还有南军没有被控制。平阳侯听说后,将吕产的阴谋告知丞相陈平,丞相陈平于是召来朱虚侯辅佐太尉。太尉命令朱虚侯把守军门。他命令平阳侯告诉卫尉:“不要让相国吕产进入殿门。”

吕产不知道吕禄已经离开北军,竟然进入未央宫,想要作乱,却不能进入殿门,只在殿门前徘徊。平阳侯担心不能取胜,驱车禀告太尉。太尉也担心不能战胜吕氏,不敢公开说要诛灭吕产,于是派朱虚侯对吕产说:“赶快进宫保护皇帝。”

朱虚侯请求派兵,太尉给他士兵一千多人。进入未央宫门的时候,看见吕产已经在朝堂上了。傍晚时分,朱虚侯就向吕产发起进攻。吕产逃跑,天空刮起大风,因为这个缘故,他的侍从官慌乱,没有人敢抵抗。朱虚侯追赶吕产,在郎中府吏的厕所里将他杀死。

\begin{yuanwen}
硃虚侯已杀产,帝命谒者持节劳硃虚侯。硃虚侯欲夺节信,谒者不肯,硃虚侯则从与载,因节信驰走,斩长乐卫尉吕更始。还,驰入北军,报太尉。太尉起,拜贺硃虚侯曰:“所患独吕产,今已诛,天下定矣。”

遂遣人分部悉捕诸吕男女,无少长皆斩之。辛酉,捕斩吕禄,而笞杀吕嬃。使人诛燕王吕通,而废鲁王偃。壬戌,以帝太傅食其复为左丞相。戊辰,徙济川王王梁,立赵幽王子遂为赵王。遣硃虚侯章以诛诸吕氏事告齐王,令罢兵。灌婴兵亦罢荥阳而归。
\end{yuanwen}

朱虚侯杀死吕产之后,皇帝命令谒者带着符节前去慰劳朱虚侯。朱虚侯夺取符节,谒者不肯交出,朱虚侯就坐上他的车,利用符节驱车飞奔,将长乐宫卫尉吕更始斩首。返回的时候,朱虚侯驱车进入北军,向太尉报告。太尉起身,下拜向朱虚侯庆贺说:“我们所担心的只是吕产,现在吕产已经被诛杀,天下安定了。”

于是他派人分别将吕氏男女全部逮捕,无论老少全部斩首。九月辛酉日,将吕禄逮捕斩首,将吕媭笞打而死。派人诛杀燕王吕通,并且废黜鲁王张偃。九月壬戌日,朝廷任命皇帝的太傅审食其为左丞相。九月戊辰日,改封济川王为梁王,封赵幽王的儿子刘遂为赵王。派朱虚侯刘章将诛灭吕氏的事情告知齐王,让他撤兵。灌婴的军队也从荥阳撤回。

\begin{yuanwen}
诸大臣相与阴谋曰:“少帝及梁、淮阳、常山王,皆非真孝惠子也。吕后以计诈名他人子,杀其母,养后宫,令孝惠子之,立以为后,及诸王,以彊吕氏。今皆已夷灭诸吕,而置所立,即长用事,吾属无类\footnote{没有幸存者。}矣。不如视诸王最贤者立之。”

或言:“齐悼惠王高帝长子,今其適子\footnote{dí,嫡子。}为齐王,推本言之,高帝適长孙,可立也”。

大臣皆曰:“吕氏以外家\footnote{外戚,指帝王的母族、妻族。}恶而几危宗庙,乱功臣。今齐王母家驷\footnote{齐哀王的舅舅。驷,sì},驷钧,恶人也。即立齐王,则复为吕氏。”

欲立淮南王,以为少,母家又恶。乃曰:“代王方今高帝见子,最长,仁孝宽厚。太后家薄氏谨良。且立长故顺,以仁孝闻于天下,便。”

乃相与共阴使人召代王。代王使人辞谢。再反,然后乘六乘传\footnote{六匹马拉的驿车。}。后九月\footnote{闰九月。}晦日\footnote{每个月的最后一天,即二十九日或三十日。}己酉,至长安,舍代邸。大臣皆往谒,奉天子玺上代王,共尊立为天子。代王数让,群臣固请,然后听。
\end{yuanwen}

众大臣相互暗中商议说:“少帝和梁王、淮阳王、常山王,都不是孝惠帝的亲生儿子。吕后使用诡计把别人的儿子当作孝惠帝的儿子,杀死这些孩子的母亲,将他们养育在后宫,让孝惠帝把他们视为自己的儿子,立为后嗣,以及封为诸王,来增强吕氏的实力。现在吕氏已经被全部消灭,却留下他们所立的皇帝,等到他长大当权以后,我们就没有机会活下去了。不如从诸王中挑选一个最贤明的人立为皇帝。”

有人说:“齐悼惠王是高帝的长子,现在他的嫡子为齐王,从血缘关系来说,他是高帝的嫡长孙,可以立为皇帝。”

大臣都说:“吕氏凭借外戚的身份作恶而几乎危害国家,他们迫害功臣。现在齐王母族的驷钧,是个恶人。假如立齐王为皇帝,就会重现吕氏之乱。”

想要立淮南王为皇帝,众人认为他年纪小,母族也很凶恶。于是众人说:“代王是高帝在世的儿子中,年纪最大的,他仁慈孝顺而宽容厚道。太后的家族薄氏也谨慎善良。况且立年长的儿子为皇帝本来就名正言顺,代王又凭借仁慈孝顺闻名于天下,立他为皇帝最妥当。”

于是众人就暗中派人召来代王。代王派人推辞。使者第二次去,此后代王乘坐六匹马拉的驿车出发。闰九月己酉晦日,代王来到长安,居住在代国官邸。大臣都前往拜见,向代王进献天子印玺,共同立代王为天子。代王多次推让,群臣坚持请求,最后代王同意了。

\begin{yuanwen}
东牟侯兴居曰:“诛吕氏吾无功,请得除宫。”

乃与太仆汝阴侯滕公\footnote{夏侯婴。}入宫,前谓少帝曰:“足下非刘氏,不当立。”

乃顾麾\footnote{指挥。}左右执戟者掊\footnote{通“踣”,放倒。}兵罢去。有数人不肯去兵,宦者令张泽谕告,亦去兵。滕公乃召乘舆车载少帝出。

少帝曰:“欲将我安之乎?”

滕公曰:“出就舍。”

舍少府。乃奉天子法驾,迎代王于邸,报曰:“宫谨除。”

代王即夕入未央宫。有谒者十人持戟卫端门,曰:“天子在也,足下何为者而入?”

代王乃谓太尉。太尉往谕,谒者十人皆掊兵而去。代王遂入而听政。夜,有司分部诛灭梁、淮阳、常山王及少帝于邸。

代王立为天子。二十三年崩,谥为孝文皇帝。
\end{yuanwen}

东牟侯刘兴居说:“诛灭吕氏我没有立功,请求为陛下清理宫廷。”

于是他和太仆汝阴侯滕公进入宫中,上前对少帝说:“您不是刘氏的后代,不应当做皇帝。”

于是他回头指挥持戟的侍卫放下武器离开。有几个人不肯放下武器,宦者令张泽上前告谕,他们也放下了武器。滕公于是叫来车驾载着少帝离开皇宫。

少帝说:“你们想把我安置在哪里?”

滕公说:“出宫住在官署。”

少帝被安置在少府居住。于是群臣随天子车驾,去官邸迎接代王,报告说:“宫廷已经清理过了。”

代王在当天晚上进入未央宫。有谒者十人持戟守卫端门,说:“天子在这里,您为什么要进去?”

代王就转告太尉。太尉上前告谕,十名谒者都放下武器走开了。代王终于进入宫中处理政事。夜里,百官分别到各官署诛杀梁王、淮阳王、常山王和少帝。

代王被立为天子,在位二十三年去世,谥号为孝文皇帝。

\begin{yuanwen}
太史公曰:孝惠皇帝、高后之时,黎民得离战国之苦,君臣俱欲休息乎无为,故惠帝垂拱\footnote{垂衣拱手,形容圣人无为而治。},高后女主称制,政不出房户,天下晏然\footnote{安定的样子。}。刑罚罕用,罪人是希。民务稼穑,衣食滋殖。
\end{yuanwen}

太史公说:孝惠皇帝、高后的时候,百姓得以摆脱战乱纷争的苦难,君臣都想要借助清静无为来休养生息,所以惠帝垂衣拱手统治天下,高后以女人的身份代行皇帝职权,政令不出宫门,就使天下安定太平。很少使用刑罚,犯罪的人很少。民众勤于耕作,衣食越发丰足。

\begin{yuanwen}
高祖犹微,吕氏作妃。及正轩掖,潜用福威。志怀安忍,性挟猜疑。置鸩齐悼,残彘戚姬。孝惠崩殒,其哭不悲。诸吕用事,天下示私。大臣菹醢,支孽芟夷。祸盈斯验,苍狗为菑。
\end{yuanwen}

\part{卷十}
\chapter{孝文本纪第十}

本篇是汉文帝刘恒的本纪,记述了他一生的主要事迹。文帝在诸吕之乱后被大臣拥立,心中始终惊惧不安,然而他精于统治术,不但稳定了政局,而且延续了汉朝建国以来的休养生息政策,被誉为一代明君。

\begin{yuanwen}
孝文皇帝,高祖中子也。高祖十一年春,已破陈豨军,定代地,立为代王,都中都。太后薄氏子。即位十七年,高后八年七月,高后崩。九月,诸吕吕产等欲为乱,以危刘氏,大臣共诛之,谋召立代王,事在吕后语\footnote{指《吕太后本纪》。}中。
\end{yuanwen}

孝文皇帝,是高祖排行居中的儿子。高祖十一年(前196年)春季,打败陈豨的军队后,平定代地,封文帝为代王,定都中都。他是太后薄氏的儿子。代王即位十七年,即高后八年,七月,高后去世。九月,吕氏子弟吕产等人想要作乱,来危害刘氏,大臣共同将其诛灭,商议召代王来,立他为皇帝,相关记载在吕后的传记中。

\begin{yuanwen}
丞相陈平、太尉周勃等使人迎代王。代王问左右郎中令张武等。张武等议曰:“汉大臣皆故高帝时大将,习兵,多谋诈,此其属意非止此也,特畏高帝、吕太后威耳。今已诛诸吕,新啑血\footnote{喋血。}京师,此以迎大王为名,实不可信。原大王称疾毋往,以观其变。”

中尉宋昌进曰:“群臣之议皆非也。夫秦失其政,诸侯豪桀并起,人人自以为得之者以万数,然卒践天子之位者,刘氏也,天下绝望,一矣。高帝封王子弟,地犬牙相制,此所谓盘石之宗也,天下服其彊,二矣。汉兴,除秦苛政,约法令,施德惠,人人自安,难动摇,三矣。夫以吕太后之严,立诸吕为三王,擅权专制,然而太尉以一节入北军,一呼士皆左袒,为刘氏,叛诸吕,卒以灭之。此乃天授,非人力也。今大臣虽欲为变,百姓弗为使,其党宁能专一邪?方今内有硃虚、东牟之亲,外畏吴、楚、淮南、琅邪、齐、代之彊。方今高帝子独淮南王与大王,大王又长,贤圣仁孝,闻于天下,故大臣因天下之心而欲迎立大王,大王勿疑也。”

代王报太后计之,犹与\footnote{同“犹豫”。}未定。卜之龟,卦兆得大横\footnote{烧灼龟甲后出现横纹。}。占曰:“大横庚庚\footnote{纹理横布的样子。},余为天王,夏启以光\footnote{夏启的事业发扬光大。夏启为大禹之子,开创了父传子家天下的制度,这里用来暗示代王继承汉高帝的皇位合理合法。}。”

代王曰:“寡人固已为王矣,又何王?”

卜人曰:“所谓天王者乃天子。”

于是代王乃遣太后弟薄昭往见绛\footnote{jiàng}侯,绛侯等具为昭言所以迎立王意。薄昭还报曰:“信矣,毋可疑者。”

代王乃笑谓宋昌曰:“果如公言。”

乃命宋昌参乘\footnote{乘车时居右负责警卫的人。},张武等六人乘传诣长安。至高陵休止,而使宋昌先驰之长安观变。
\end{yuanwen}

丞相陈平、太尉周勃等人派人迎接代王。代王询问身边的近臣郎中令张武等人。张武等人建议说:“汉朝廷的大臣都是当初高帝时的大将,熟悉兵事,多有权谋,他们的用意恐怕不止于此,只是因为畏惧高帝、吕太后的威势罢了。现在他们已经诛灭吕氏,血洗京城,他们名义上是迎接大王,实际上不可轻信。希望大王称病不要前往,以此观察事态的变化。”

中尉宋昌进言说:“群臣的建议都是错误的。当初秦朝政治混乱,诸侯豪杰纷然起事,自认为能够得到天下的人有数以万计,然而最终登上天子之位的,是刘氏,天下豪杰已经对争夺皇帝之位不抱希望,这是第一点。高帝分封刘氏子弟为王,封国的疆界犬牙交错相互制约,这就是人们所说的像磐石一样坚固的宗族,天下人因为其强大而臣服,这是第二点。汉朝建立以后,废除了秦朝严苛的政令,精简法令,施以恩惠,人们都安居乐业,难以动摇,这是第三点。凭借吕太后的威严,吕氏已经有三人封王,在朝中独断专权,然而太尉持符节进入北军,呼喊一声就使士兵都袒露左臂,支持刘氏,背叛吕氏,最终消灭吕氏。这是上天授予的神威,不是人力所能做到的。现在大臣即使想要作乱,百姓也不会为他们所驱使,他们的党羽难道能齐心协力吗?现在内有朱虚侯、东牟侯这样的亲属,对外畏惧吴王、楚王、淮南王、琅邪王、齐王、代王的强大。现在高帝的儿子只有淮南王和大王,大王又年长,贤能圣明而仁慈孝顺,闻名于天下,所以大臣顺应天下人心而想要迎接大王立为皇帝,大王不要怀疑。”

代王报告太后与其商议,还是犹豫不决。于是用龟甲来占卜,得到大横的卦象。卜辞说:“纹络粗横,我为天王,夏启事业,散发光芒。”

代王说:“我本来已经是王了,还要做什么王?”

占卜的人说:“这里所说的天王就是天子。”

于是代王派太后的弟弟薄昭前去拜见绛侯,绛侯等人把迎立代王的意图如实告知薄昭。薄昭回来报告说:“可以相信了,不用怀疑了。”

代王于是笑着对宋昌说:“果然如您所说。”

他命令宋昌为参乘,张武等六人也乘坐驿车前往长安。走到高陵停下休息,代王派宋昌先驱车去长安观察局势的变化。

\begin{yuanwen}
昌至渭桥,丞相以下皆迎。宋昌还报。代王驰至渭桥,群臣拜谒称臣。代王下车拜。太尉勃进曰:“原请间\footnote{指在空闲时通报事情,不方便公布于众。}言。”

宋昌曰:“所言公,公言之。所言私,王者不受私。”

太尉乃跪上天子玺符。代王谢曰:“至代邸而议之。”

遂驰入代邸。群臣从至。丞相陈平、太尉周勃、大将军陈武、御史大夫张苍、宗正刘郢、硃虚侯刘章、东牟侯刘兴居、典客刘揭皆再拜言曰:“子弘等皆非孝惠帝子,不当奉宗庙。臣谨请(与)阴安侯、列侯顷王后\footnote{汉高帝的大嫂、二嫂。刘邦称帝前,长兄刘伯去世,于是追封武哀侯,吕后时晋封武哀王,封其妻为阴安侯。刘邦称帝后,封二哥刘仲为代王,因匈奴进犯时弃城逃跑,被废为侯,死后其子刘濞封吴王,追谥他为顷王,因此其妻为顷王后。顷王后前有“列侯”二字,一说是衍字,一说是刘仲妻也像刘伯妻一样曾受封列侯。}与琅邪王、宗室、大臣、列侯、吏二千石议曰:‘大王高帝长子,宜为高帝嗣。’原大王即天子位。”

代王曰:“奉高帝宗庙,重事也。寡人不佞\footnote{没有才能。},不足以称宗庙。原请楚王计宜者,寡人不敢当。”

群臣皆伏固请。代王西乡让者三,南乡让者再。丞相平等皆曰:“臣伏计之,大王奉高帝宗庙最宜称,虽天下诸侯万民以为宜。臣等为宗庙社稷计,不敢忽。原大王幸听臣等。臣谨奉天子玺符再拜上。”

代王曰:“宗室、将相、王、列侯以为莫宜寡人,寡人不敢辞。”遂即天子位。
\end{yuanwen}

宋昌来到渭桥,丞相以下官员都来迎接。宋昌返回报告。代王驱车来到渭桥,群臣都来拜见称臣。代王下车答拜。太尉周勃上前说:“我请求私下向大王奏事。”

宋昌说:“所说的是公事,就公开说。所说的是私事,帝王不受理私事。”

太尉于是跪下进献天子的印玺和符节。代王辞谢说:“到代国官邸再商议这件事。”

于是他驱车进入代国官邸。群臣都跟随而来。丞相陈平、太尉周勃、大将军陈武、御史大夫张苍、宗正刘郢、朱虚侯刘章、东牟侯刘兴居、典客刘揭都两次下拜后进言说:“刘弘等人都不是孝惠帝的儿子,不应当主持宗庙祭祀。我们恭请阴安侯、列侯顷王后和琅琊王、宗室、大臣、列侯、二千石以上的官吏商议说:‘大王是高帝最年长的儿子,应该做高帝的继承人。’希望大王登上天子之位。”

代王说:“尊奉高帝宗庙,事关重大。我没有才能,不足以尊奉宗庙。希望请求楚王重新考虑,我不敢担当。”

群臣都跪拜坚决请求。代王面朝西方谦让三次,又面朝南方谦让两次。丞相陈平等人都说:“我们经过商议,认为大王尊奉高帝宗庙是最适宜的,即使让天下诸侯和百姓来考虑,他们也认为适宜。我们为国家社稷考虑,不敢轻率决定。希望大王听从我们的建议。我们再次下拜敬献天子玺印和符节。”

代王说:“宗室、将相、诸王、列侯都认为没有人比我更适宜,我也不敢再推辞了。”于是他登上天子之位。

\begin{yuanwen}
群臣以礼次侍。乃使太仆婴与东牟侯兴居清宫,奉天子法驾,迎于代邸。皇帝即日夕入未央宫。乃夜拜宋昌为卫将军,镇抚南北军。以张武为郎中令,行殿中。还坐前殿。于是夜下诏书曰:“间者诸吕用事擅权,谋为大逆,欲以危刘氏宗庙,赖将相、列侯、宗室、大臣诛之,皆伏其辜。朕初即位,其赦天下,赐民爵一级,女子百户牛酒,酺\footnote{聚会宴饮。}五日。”
\end{yuanwen}

群臣按照礼仪依次侍立。于是派太仆夏侯婴和东牟侯刘兴居清理宫廷,奉迎天子车驾,去代国官邸迎接皇帝。皇帝在当天晚上进入未央宫。于是他在夜里任命宋昌为卫将军,安抚南北军。任命张武为郎中令,在殿中巡逻。皇帝回到前殿坐下,在当天夜里下诏书说:“此前吕氏当政专权,阴谋叛逆,想要危害刘氏宗庙,仰仗将相、列侯、宗室、大臣将其诛灭,使他们服罪受到惩罚。我刚即位,将要大赦天下,赏赐民众每人提升爵位一级,女子以一百户为单位赏赐酒肉,百姓聚会宴饮五天。”

\begin{yuanwen}
孝文皇帝元年十月庚戌,徙立故琅邪王泽为燕王。
\end{yuanwen}

孝文皇帝元年(前179年)十月庚戌日,改封原琅邪王刘泽为燕王。

\begin{yuanwen}
辛亥,皇帝即阼\footnote{登上宗庙东阶,象征皇帝正式即位。},谒高庙。右丞相平徙为左丞相,太尉勃为右丞相,大将军灌婴为太尉。诸吕所夺齐楚故地,皆复与之。
\end{yuanwen}

辛亥日,皇帝正式即位,拜谒高庙。右丞相陈平改任左丞相,太尉周勃担任右丞相,大将军灌婴担任太尉。吕氏所侵夺的原属齐国、楚国的土地,都归还齐王、楚王。

\begin{yuanwen}
壬子,遣车骑将军薄昭迎皇太后于代。皇帝曰:“吕产自置为相国,吕禄为上将军,擅矫遣灌将军婴将兵击齐,欲代刘氏,婴留荥阳弗击,与诸侯合谋以诛吕氏。吕产欲为不善,丞相陈平与太尉周勃谋夺吕产等军。硃虚侯刘章首先捕吕产等。太尉身率襄平侯通持节承诏入北军。典客刘揭身夺赵王吕禄印。益封太尉勃万户,赐金五千斤。丞相陈平、灌将军婴邑各三千户,金二千斤。硃虚侯刘章、襄平侯通、东牟侯刘兴居邑各二千户,金千斤。封典客揭为阳信侯,赐金千斤。”
\end{yuanwen}

壬子日,文帝派车骑将军薄昭去代国迎接皇太后。皇帝说:“吕产自任相国,吕禄担任上将军,擅自假传诏令派灌将军率领军队攻打齐国,想要取代刘氏,灌婴留守荥阳没有进攻齐国,和诸侯共同谋划来诛灭吕氏。吕产想要作乱,丞相陈平和太尉周勃谋划夺取吕产等人的兵权。朱虚侯刘章率先逮捕吕产等人。太尉亲自率领襄平侯纪通持符节奉诏书进入北军。典客刘揭亲自夺取赵王吕禄的将军印。加封太尉周勃食邑一万户,赏金五千斤。加封丞相陈平、将军灌婴食邑各三千户,赏金两千斤。加封朱虚侯刘章、襄平侯纪通、东牟侯刘兴居食邑各二千户,赏金一千斤。封典客刘揭为阳信侯,赏金一千斤。”

\begin{yuanwen}
十二月,上曰:“法者,治之正\footnote{原则,标准。}也,所以禁暴而率善人也。今犯法已论,而使毋罪之父母妻子同产坐之,及为收帑\footnote{将罪犯的子女罚为官奴。},朕甚不取。其议之。”

有司皆曰:“民不能自治,故为法以禁之。相坐坐收,所以累其心,使重犯法,所从来远矣。如故便。”

上曰:“朕闻法正则民悫\footnote{忠厚。},罪当则民从。且夫牧民\footnote{治理民众。}而导之善者,吏也。其既不能导,又以不正之法罪之,是反害于民为暴者也。何以禁之?朕未见其便,其孰计之。”

有司皆曰:“陛下加大惠,德甚盛,非臣等所及也。请奉诏书,除收帑诸相坐律令。”
\end{yuanwen}

十二月,文帝说:“法令,是治理国家的准则,可以用来禁止暴行,并且劝人向善。现在犯法的人定罪后,却要连累他们无罪的父母妻儿和兄弟一同受罚,还要罚没其子女为官奴,我认为这种做法很不可取。群臣讨论一下。”

有关部门的官员都说:“民众不能自行治理,所以制定法令来约束他们。受连累的家属也要被逮捕,就是要使人们心有牵挂,使他们慎重看待触犯法律的事情,这种做法由来已久了。还是遵照原来的做法更适宜。”

文帝说:“我听说法令公正则百姓忠厚,量刑得当则百姓顺服。况且治理百姓而劝人向善的人,是官吏。如果官吏既不能劝导百姓,又使用不公正的法令来处罚他们,这是反而对百姓有害而逼迫他们去做坏事。用什么来禁止呢?我没有看出来为什么适宜,群臣再详细考虑一下。”

有关部门的官员都说:“陛下的恩惠浩大,功德隆盛,不是我们所能想到的。我们遵奉诏书,废除将无罪的家属罚为官奴的法令。”

\begin{yuanwen}
正月,有司言曰:“蚤\footnote{通“早”。}建太子,所以尊宗庙。请立太子。”

上曰:“朕既不德,上帝神明未歆享\footnote{神灵享用祭品。},天下人民未有嗛志\footnote{满意。}。今纵不能博求天下贤圣有德之人而禅天下焉,而曰豫\footnote{通“预”。}建太子,是重吾不德也。谓天下何?其安之。”

有司曰:“豫建太子,所以重宗庙社稷,不忘天下也。”

上曰:“楚王,季父也,春秋高,阅天下之义理多矣,明于国家之大体。吴王于朕,兄也,惠仁以好德。淮南王,弟也,秉德以陪朕。岂为不豫哉!诸侯王、宗室昆弟、有功臣,多贤及有德义者,若举有德以陪朕之不能终,是社稷之灵,天下之福也。今不选举焉,而曰必子,人其以朕为忘贤有德者而专于子,非所以忧天下也。朕甚不取也。”

有司皆固请曰:“古者殷周有国,治安皆千馀岁,古之有天下者莫长焉,用此道也。立嗣必子,所从来远矣。高帝亲率士大夫,始平天下,建诸侯,为帝者太祖。诸侯王及列侯始受国者皆亦为其国祖。子孙继嗣,世世弗绝,天下之大义也,故高帝设之以抚海内。今释宜建而更选于诸侯及宗室,非高帝之志也。更议不宜。子某\footnote{据《汉书》,应为启,即汉景帝。}最长,纯厚慈仁,请建以为太子。”

上乃许之。因赐天下民当代父后者爵各一级。封将军薄昭为轵侯。
\end{yuanwen}

正月,有关部门的官员进言说:“早立太子,以此尊奉宗庙。请求立太子。”

文帝说:“我德行浅薄,天帝神明还没有享用祭品,天下人民还没有感到满意。现在我纵然不能遍寻天下贤能圣明有德的人而把天下禅让给他,却说要预先立太子,这会使我的德行更加浅薄。我怎么对天下人说?这件事暂时搁置。”

官员说:“预先立太子,就是要尊崇宗庙和社稷,表示不忘天下人民。”

文帝说:“楚王,是我的叔父,他年纪大,懂得很多天下的道理了,明白国家的大体。吴王,是我的兄长,慈惠仁爱而喜好道德。淮南王,是我的弟弟,持守道德来辅佐我。他们难道不是预先确立的继承人吗?诸侯王、宗室兄弟、有功之臣,很多贤能和有德义的人,如果推举有德之人来辅佐我完成未竟的事业,这是社稷的好事,是天下的福分。现在不推举他们,而说一定要立自己的儿子,人们就会认为我忘记了贤能有德的人,而只想着自己的儿子,不是为天下人分忧的做法。我认为这样做很不可取。”

官员都坚决请求说:“古时候殷商和周朝建立国家,长治久安一千多年,古时候得天下的朝代没有比这更长久的了,因为这两个朝代的帝王所选的继承人一定是自己的儿子,这是由来已久的了。高帝亲自率领文臣武将,刚平定天下,就建立诸侯国,成为本朝帝王的太祖。最初接受封国的诸侯王和列侯也是各国的始祖。子孙继承嗣位,世代不会断绝,这是天下的大义,所以高帝设立这一制度来安定海内人心。现在放弃应当遵守的,却另选诸侯和宗室人,这不是高帝的本意。再议论也不合适。皇子刘启最年长,纯厚仁慈,请求立他为太子。”

文帝于是同意了。于是赏赐天下百姓中应当做父亲嗣子的人各晋升爵位一级。封将军薄昭为轵侯。

\begin{yuanwen}
三月,有司请立皇后。薄太后曰:“诸侯皆同姓,立太子母为皇后。”

皇后姓窦氏。上为立后故,赐天下鳏寡孤独\footnote{年老无妻为鳏,年老无夫为寡,年幼丧父为孤,年老无子为独。}穷困及年八十已上孤儿九岁已下布帛米肉各有数。上从代来,初即位,施德惠天下,填抚诸侯四夷皆洽驩,乃循从代来功臣。上曰:“方大臣之诛诸吕迎朕,朕狐疑\footnote{犹豫不决。},皆止朕,唯中尉宋昌劝朕,朕以得保奉宗庙。已尊昌为卫将军,其封昌为壮武侯。诸从朕六人,官皆至九卿。”
\end{yuanwen}

三月,有关部门的官员请求立皇后。薄太后说:“诸侯都是同姓,立太子的母亲为皇后。”

皇后姓窦氏。文帝因为册立皇后的缘故,赐给天下鳏寡孤独贫穷困苦的人以及八十岁以上的老人和九岁以下的孤儿布帛米肉各有定额。文帝从代国来到长安,刚即位,就对天下百姓施行恩惠,安抚诸侯和四方部族使其和睦欢欣,于是依次封赏从代国来的功臣。文帝说:“大臣诛灭吕氏迎接我来即位时,我犹豫不决,群臣都阻止我,只有中尉宋昌鼓励我,我才得以尊奉宗庙。我已经提拔宋昌为卫将军,现在封他为壮武侯。另外随我前来的六个人,官职都提拔为九卿。”

\begin{yuanwen}
上曰:“列侯从高帝入蜀、汉中者六十八人皆益封各三百户,故吏二千石以上从高帝颍川守尊等十人食邑六百户,淮阳守申徒嘉等十人五百户,卫尉定等十人四百户。封淮南王舅父赵兼为周阳侯,齐王舅父驷钧为清郭侯。”

秋,封故常山丞相蔡兼为樊侯。
\end{yuanwen}

文帝说:“跟随高帝去蜀郡、汉中的列侯六十八个人加封食邑各三百户,过去追随高帝的二千石以上官吏颍川郡守刘尊等十个人封赏食邑各六百户,淮阳郡守申徒嘉等十个人封赏食邑各五百户,卫尉定等十个人封赏食邑各四百户。封淮南王的舅父赵兼为周阳侯,封齐王的舅父驷钧为清郭侯。”

秋季,文帝封原常山国丞相蔡兼为樊侯。

\begin{yuanwen}
人或说右丞相曰:“君本诛诸吕,迎代王,今又矜其功,受上赏,处尊位,祸且及身。”

右丞相勃乃谢病免罢,左丞相平专为丞相。
\end{yuanwen}

有人劝右丞相说:“您本来诛杀吕氏,迎接代王,现在又自夸功高,受到最高的赏赐,居处尊显的地位,将要大难临头。”

右丞相周勃于是称病免职,由左丞相陈平独自担任丞相。

\begin{yuanwen}
二年十月,丞相平卒,复以绛侯勃为丞相。上曰:“朕闻古者诸侯建国千馀,各守其地,以时入贡,民不劳苦,上下驩欣,靡有遗德\footnote{未受重用的贤人。}。今列侯多居长安,邑远,吏卒给输费苦,而列侯亦无由教驯其民。其令列侯之国,为吏及诏所止者,遣太子。”
\end{yuanwen}

二年(前178年)十月,丞相陈平去世,朝廷重新任命绛侯周勃为丞相。文帝说:“我听说古时候的诸侯建立了一千多个国家,他们各自守卫封地,按时入朝进贡,百姓不会劳苦,上下欢欣喜悦,没有未受重用的贤人。现在列侯大多居住在长安,远离封邑,官吏和士卒供给运输当地的赋税有损耗又辛苦,而列侯也无法教导各自的民众。命令列侯回到各自的封国,在朝中担任官职以及有诏令准许留下的,要派太子回去。”

\begin{yuanwen}
十一月晦,日有食之。十二月望\footnote{望日,即每个月的十五日。},日又食。上曰:“朕闻之,天生蒸\footnote{通“烝”,众多。}民,为之置君以养治之。人主不德,布政不均,则天示之以菑\footnote{zāi,通“灾”。},以诫不治。乃十一月晦,日有食之,適\footnote{zhè,通“谪”,惩罚。}见于天,菑孰大焉!朕获保宗庙,以微眇之身讬(托)于兆民君王之上,天下治乱,在朕一人,唯二三执政犹吾股肱也。朕下不能理育群生,上以累三光\footnote{又称三辰,指日、月、星。}之明,其不德大矣。令至,其悉思朕之过失,及知见思之所不及,匄\footnote{乞,求。}以告朕。及举贤良方正能直言极谏者,以匡朕之不逮。因各饬其任职,务省繇费以便民。朕既不能远德,故(憪)然\footnote{戒惧不安的样子。}念外人之有非,是以设备未息。今纵不能罢边屯戍,而又饬兵厚卫,其罢卫将军军。太仆见马遗财\footnote{通“才”。}足,馀皆以给传置\footnote{驿站。}。”
\end{yuanwen}

十一月晦日,发生了日食。十二月望日,又发生了日食。文帝说:“我听说,上天生育万民,为他们设置君主来抚养和管理他们。如果君主不仁德,施政不公平,上天就会显现灾异,来告诫他天下治理得不好。于是在十一月晦日,发生了日食,上天显现对我的责罚,没有比这更严重的灾难了!我能够保全宗庙,凭借渺小的身躯被万民和诸侯抬举在上,天下的安定与混乱,取决于我一个人,众位执掌国政的大臣就像我的左膀右臂。我对下不能治理和养育众生,对上又拖累三辰的光辉,我的不仁德实在太严重了。接到诏令后,众位大臣要认真思考我的过失,以及我见闻思虑所不能及的方面,请告诉我。还要推选贤良方正能够直言进谏的人,来纠正我的过失。各级官吏趁机各自整饬职事,务求减省徭役和费用来方便百姓。我已经不能将恩德布施远方,所以心里忧虑不安,担心远方之人为非作歹,因此从来没有停止边防事务。现在纵然不能撤除边塞的屯戍,却又命令增加军队来保护宫廷,应该撤销卫将军的军队。太仆现有马匹只留下够用的就可以了,其余的都交给驿站使用。”

\begin{yuanwen}
正月,上曰:“农,天下之本,其开籍田\footnote{天子亲自耕种的田。},朕亲率耕,以给宗庙粢盛\footnote{盛放在礼器内供祭祀用的谷物。}。”
\end{yuanwen}

正月,文帝说:“农业,是天下的根本,现在开垦籍田,我亲自带头耕种,来供给宗庙祭祀所用的谷物。”

\begin{yuanwen}
三月,有司请立皇子为诸侯王。上曰:“赵幽王幽死,朕甚怜之,已立其长子遂为赵王。遂弟辟彊及齐悼惠王子硃虚侯章、东牟侯兴居有功,可王。”

乃立赵幽王少子辟彊为河间王,以齐剧郡\footnote{政务繁忙的郡,即大郡。}立硃虚侯为城阳王,立东牟侯为济北王,皇子武为代王,子参为太原王,子揖为梁王。
\end{yuanwen}

三月,有关部门的官员请求封皇子为诸侯王。文帝说:“赵幽王被幽禁而死,我非常怜悯他,已经立他的长子刘遂为赵王。刘遂的弟弟刘辟彊以及齐悼惠王的儿子朱虚侯刘章、东牟侯刘兴居有功,也可以封王。”

于是封赵幽王的小儿子刘辟彊为河间王,用齐国的大郡封朱虚侯为城阳王,封东牟侯为济北王,封皇子刘武为代王,封皇子刘参为太原王,封皇子刘揖为梁王。

\begin{yuanwen}
上曰:“古之治天下,朝有进善之旌,诽谤之木,所以通治道而来谏者。今法有诽谤妖言之罪,是使众臣不敢尽情,而上无由闻过失也。将何以来远方之贤良?其除之。民或祝诅上以相约结而后相谩\footnote{欺骗。},吏以为大逆,其有他言,而吏又以为诽谤。此细民之愚无知抵死,朕甚不取。自今以来,有犯此者勿听治。”
\end{yuanwen}

文帝说:“古时候治理天下,朝廷设置进献善言的旌旗,批评朝政的木柱,以此保持治国之道的畅通,并且鼓励直言进谏。现在法令中有诽谤朝政和妖言惑众的罪名,这使群臣不敢畅所欲言,并且皇帝也无法得知自己的过失。这样怎么能招来远方的贤良之士呢?应当废除这一法令。百姓当中有人诅咒皇帝而约定相互隐瞒,后来却相互欺骗,官吏认为这是大逆不道,再有别的言论,官吏又认为是诽谤朝政。因为小民的愚昧无知而将其判处死刑,我认为非常不可取。从今以后,有犯这类罪行的不要受理治罪。”

\begin{yuanwen}
九月,初与郡国守相\footnote{即郡守国相。西汉郡国并行,一郡之长为太守,国分王国、侯国两级,王国与郡平级,侯国与县平级,诸侯国设相治民,级别与郡守、县令相同。}为铜虎符、竹使符。
\end{yuanwen}

九月,朝廷开始授予郡国守相铜虎符、竹使符。

\begin{yuanwen}
三年十月丁酉晦,日有食之。

十一月,上曰:“前日诏遣列侯之国,或辞未行。丞相朕之所重,其为朕率列侯之国。”

绛侯勃免丞相就国,以太尉颍阴侯婴为丞相。罢太尉官,属丞相。

四月,城阳王章薨。淮南王长与从者魏敬杀辟阳侯审食其。
\end{yuanwen}

三年(前177年)十月丁酉晦日,发生了日食。

十一月,文帝说:“此前下诏让列侯回到封国,有的人找借口不走。丞相是我所器重的人,希望丞相为我率领列侯回到封国。”

绛侯周勃免去丞相官职回到封国。文帝任命太尉颍阴侯灌婴为丞相。撤销太尉一职,把太尉职权归于丞相。

四月,城阳王刘章去世。淮南王刘长和随从魏敬杀死辟阳侯审食其。

\begin{yuanwen}
五月,匈奴入北地,居河南\footnote{指河套以南地区。}为寇。帝初幸甘泉。

六月,帝曰:“汉与匈奴约为昆弟,毋使害边境,所以输遗匈奴甚厚。今右贤王离其国,将众居河南降地,非常故,往来近塞,捕杀吏卒,驱保塞蛮夷,令不得居其故,陵轹\footnote{lì}边吏,入盗,甚敖\footnote{通“傲”,狂傲。}无道,非约也。其发边吏骑八万五千诣高奴,遣丞相颍阴侯灌婴击匈奴。”

匈奴去,发中尉材官\footnote{秦汉时始置的一种地方预备兵种。}属卫将军,军长安。
\end{yuanwen}

五月,匈奴进入北地郡,占据河南地区进行劫掠。文帝初次来到甘泉宫。

六月,文帝说:“汉朝和匈奴结为兄弟,是为了不让他们侵害边境,所以送给匈奴的财物非常丰厚。现在右贤王离开他的国土,率领部众占据已经归属汉朝的河南地区,不是正常的状态,往来逼近边塞,逮捕杀害官吏和士卒,驱逐保卫边塞的蛮夷,不让他们居住在原来的地方,欺凌边防官吏,进犯劫掠,非常傲慢,不守正道,破坏盟约。现在调发守边的官吏和骑兵八万五千人前往高奴,派丞相颍阴侯灌婴率领军队攻打匈奴。”

匈奴离去,文帝调发中尉材官归属卫将军,驻扎在长安。

\begin{yuanwen}
辛卯,帝自甘泉之高奴,因幸太原,见故群臣,皆赐之。举功行赏,诸民里赐牛酒。复晋阳、中都民三岁。留游太原十馀日。

济北王兴居闻帝之代,欲往击胡,乃反,发兵欲袭荥阳。于是诏罢丞相兵,遣棘蒲侯陈武为大将军,将十万往击之。祁侯贺为将军,军荥阳。

七月辛亥,帝自太原至长安。乃诏有司曰:“济北王背德反上,诖误\footnote{连累。}吏民,为大逆。济北吏民兵未至先自定,及以军地邑降者,皆赦之,复官爵。与王兴居去来,亦赦之。”

八月,破济北军,虏其王。赦济北诸吏民与王反者。
\end{yuanwen}

辛卯日,文帝从甘泉宫前往高奴,顺便来到太原,接见原来代国的群臣,都给予赏赐。根据功劳施行奖赏,赐给百姓酒肉。免除晋阳、中都百姓三年的赋役。文帝在太原巡游十几天。

济北王刘兴居听说文帝前往代国,想要出击匈奴,于是反叛,派军队想要袭击荥阳。于是文帝下诏撤回丞相的军队,派棘蒲侯陈武为大将军,率领十万大军前去讨伐叛军。祁侯缯贺担任将军,驻扎在荥阳。

七月辛亥日,文帝从太原回到长安,就下诏命令有关部门的官员说:“济北王违背恩德,背叛皇帝,连累属下官吏和百姓,这是大逆不道。济北国的官吏和百姓在大军没有到达时自行归顺,以及率领军队或献出城邑投降的人,一律赦免,恢复官职和爵位。追随济北王刘兴居而后来归顺的人,也赦免。”

八月,汉军打败济北军,俘虏济北王。文帝赦免济北国跟随济北王反叛的官吏和百姓。

\begin{yuanwen}
六年,有司言淮南王长废先帝法,不听天子诏,居处毋度,出入拟于天子,擅为法令,与棘蒲侯太子奇谋反,遣人使闽越及匈奴,发其兵,欲以危宗庙社稷。群臣议,皆曰“长当弃市\footnote{在闹市执行死刑,表示与众共弃。}。”

帝不忍致法于王,赦其罪,废勿王。群臣请处王蜀严道、邛\footnote{qióng}都,帝许之。长未到处所,行病死,上怜之。后十六年,追尊淮南王长谥为厉王,立其子三人为淮南王、衡山王、庐江王。
\end{yuanwen}

六年(前174年),有关部门的官员报告淮南王刘长废弃先帝的法令,不听天子的诏令,他所居处的宫室规格超过限度,出入的排场近似天子,擅自制定法令,与棘蒲侯的太子陈奇谋划反叛,派人出使闽越和匈奴,调发当地的军队,想要危害宗庙社稷。群臣经过商议,都说:“应当把刘长当众处死。”

文帝不忍心对淮南王用刑,赦免了他的死罪,废黜了他的王位。群臣请求把淮南王流放到蜀郡的严道、邛都一带,文帝批准了。刘长没有到达流放地,就病死在路上,文帝很怜悯他。后来到十六年(前164年),追尊刘长为淮南王,谥号为厉王,封他的三个儿子为淮南王、衡山王、庐江王。

\begin{yuanwen}
十三年夏,上曰:“盖闻天道祸自怨起而福繇德兴。百官之非,宜由朕躬。今祕祝\footnote{官名,负责为官府祈祝。祕,mì。}之官移过于下,以彰吾之不德,朕甚不取。其除之。”
\end{yuanwen}

十三年(前167年)夏季,文帝说:“我听说灾祸从怨恨产生,福运由仁德兴起,这样符合天道。百官的过错,应该由我来承担责任。现在祕祝官把过错都推给下级,因此彰显了我的不仁德,我认为很不可取。废弃这一官职。”

\begin{yuanwen}
五月,齐太仓令\footnote{官名,主管国家粮食储备。}淳于公有罪当刑,诏狱逮徙系长安。太仓公无男,有女五人。太仓公将行会逮,骂其女曰:“生子不生男,有缓急非有益也!”

其少女缇\footnote{tí}萦\footnote{yíng}自伤泣,乃随其父至长安,上书曰:“妾父为吏,齐中皆称其廉平,今坐法当刑。妾伤夫死者不可复生,刑者不可复属,虽复欲改过自新,其道无由也。妾原没入为官婢,赎父刑罪,使得自新。”

书奏天子,天子怜悲其意,乃下诏曰:“盖闻有虞氏之时,画衣冠异章服以为僇\footnote{lù,羞辱,引申为惩罚。},而民不犯。何则?至治也。今法有肉刑三\footnote{指黥面、劓鼻、刖足。},而奸不止,其咎安在?非乃朕德薄而教不明欤?吾甚自愧。故夫驯道不纯而愚民陷焉。《诗\footnote{指《诗经·大雅·泂酌》。}》曰‘恺悌\footnote{平易和乐。}君子,民之父母’。今人有过,教未施而刑加焉?或欲改行为善而道毋由也。朕甚怜之。夫刑至断支体,刻肌肤,终身不息,何其楚痛而不德也,岂称为民父母之意哉!其除肉刑。”
\end{yuanwen}

五月,齐国太仓令淳于公犯罪应当受刑,朝廷下诏将他逮捕押解到长安拘禁。太仓公没有儿子,只有五个女儿。太仓公将要被押走时,骂他的女儿说:“生孩子而没有儿子,遇到紧急情况就一点用处都没有。”

他的小女儿缇萦独自伤心地哭泣,并且跟随父亲来到长安,向朝廷上书说:“我父亲做官,齐国人都称赞他廉洁公正,现在犯法应当受刑。我悲伤的是,死去的人不能再活过来,受刑的人肢体不能复合,即使想要改过自新,也没有别的办法了。我愿意被罚为官奴,来为父亲赎罪,让他能够改过自新。”

缇萦的上书送到天子那里,天子怜悯她的孝心,就下诏说:“听说在有虞氏的时候,在罪犯的衣帽上画相应刑罚的图案,使其区别于普通人的服饰,以此惩罚他们,可是民众不会违犯法令。这是为什么呢?因为当时是太平盛世。现在的法令中有三种肉刑,可是不能禁止人们犯法,其中的过错在哪里呢?不就是因为我的德行浅薄而教化不明吗?我为自己感到惭愧,所以训导方法不完善导致愚昧小民陷入不法的境地。《诗》说:‘和乐君子,为民父母。’现在人们犯了罪,还没有进行教化就先施以刑罚,即使有人想要改过向善也没有机会了。我非常怜悯他们。刑罚导致肢体残缺,在肌肤上刺字,终身不能恢复,多么令人痛苦而且不道德,怎么能符合为人父母的心意呢!应该废除肉刑。”

\begin{yuanwen}
上曰:“农,天下之本,务莫大焉。今勤身从事而有租税之赋,是为本末\footnote{农为本,工商为末。}者毋以异,其于劝农之道未备。其除田之租税。”
\end{yuanwen}

文帝说:“农业,是天下的根本,没有比这更重要的事务了。现在人们辛勤劳作却要交纳租税,这就使从事本业和末业的人没有区别了,这是由于鼓励农耕的方法还不完备。应该减免农田的租税。”

\begin{yuanwen}
十四年冬,匈奴谋入边为寇,攻朝塞,杀北地都尉卬。上乃遣三将军军陇西、北地、上郡,中尉周舍为卫将军,郎中令张武为车骑将军,军渭北,车千乘,骑卒十万。帝亲自劳军,勒兵申教令,赐军吏卒。帝欲自将击匈奴,群臣谏,皆不听。皇太后固要帝,帝乃止。于是以东阳侯张相如为大将军,成侯赤为内史,栾布为将军,击匈奴。匈奴遁走。
\end{yuanwen}

十四年(前166年)冬季,匈奴谋划进入边境劫掠,攻打朝塞,杀死北地都尉孙卬。文帝于是派三位将军分别驻扎在陇西、北地、上郡,任命中尉周舍为卫将军,郎中令张武为车骑将军,驻扎在渭水以北,拥有战车一千辆,骑兵十万人。文帝亲自慰劳士卒,整饬军队申明训令,赏赐全军将士。文帝想要亲自率领士兵攻打匈奴,群臣劝阻,文帝一概不听。皇太后坚决阻止文帝,文帝才作罢。于是他任命东阳侯张相如为大将军,成侯董赤为内史,栾布为将军,出兵攻打匈奴。匈奴逃跑了。

\begin{yuanwen}
春,上曰:“朕获执牺牲珪币以事上帝宗庙,十四年于今,历日(绵)长,以不敏不明而久抚临天下,朕甚自愧。其广增诸祀墠场\footnote{祭祀用的坛场。墠,shàn。}珪币。昔先王远施不求其报,望祀不祈其福,右贤左戚\footnote{右为上,左为下。},先民后己,至明之极也。今吾闻祠官祝釐\footnote{通“禧”,福运,吉祥。},皆归福朕躬,不为百姓,朕甚愧之。夫以朕不德,而躬享独美其福,百姓不与焉,是重吾不德。其令祠官致敬,毋有所祈。”
\end{yuanwen}

春季,文帝说:“我得以用牺牲和玉帛来祭祀天帝和宗庙,到现在已经十四年,时间算是很长久了,我不勤敏不明智却能长期治理天下,为此深感惭愧。应该扩建坛场,增设祭品。过去先王远施恩惠而不求回报,遥祭山川而不为祈福,尊崇贤才,不重亲戚,先为民众考虑,后为自己考虑,圣明到了极点。现在我听说主管祭祀的官吏向神灵祈福,都归于我一个人,却不为百姓考虑,我为此深感惭愧。像我这样德行浅薄的人,却独享神灵的福祉,百姓却不能享受,这就使我的德行更加浅薄。现在命令主管祭祀的官吏向神表达敬意,不要为我祈求。”

\begin{yuanwen}
是时北平侯张苍为丞相,方明律历。鲁人公孙臣上书陈终始传五德事,言方今土德时,土德应黄龙见,当改正朔服色制度。天子下其事与丞相议。丞相推以为今水德,始明正十月上黑事,以为其言非是,请罢之。
\end{yuanwen}

当时北平侯张苍担任丞相,正在制定乐律和历法。鲁郡人公孙臣上书讲述五德终始的学说,提出现在是土德,土德与黄龙出现相应验,应当更改历法和服饰等制度。天子把这件事下交丞相与群臣讨论。丞相研究后认为当今是水德,开始规定以十月为岁首和服饰崇尚黑色,认为公孙臣的学说不正确,请求不要采纳。

\begin{yuanwen}
十五年,黄龙见成纪,天子乃复召鲁公孙臣,以为博士,申明土德事。于是上乃下诏曰:“有异物之神见于成纪,无害于民,岁以有年。朕亲郊祀上帝诸神。礼官议,毋讳以劳朕。”

有司礼官皆曰:“古者天子夏躬亲礼祀上帝于郊,故曰郊。”

于是天子始幸雍,郊见五帝,以孟夏四月答礼焉。赵人新垣平以望气\footnote{观看云气预测吉凶的一种方术。}见,因说上设立渭阳五庙。欲出周鼎,当有玉英\footnote{玉的精华。}见。
\end{yuanwen}

十五年(前165年),有黄龙出现在成纪,天子于是再次召见鲁郡的公孙臣,任命他为博士,让他阐明土德的相关学说。于是文帝下达诏书说:“有怪异的神物出现在成纪,没有伤害百姓,今年的收成很好。我要亲自到郊外去祭祀天帝和众神。掌管礼仪的官员商议一下,不要担心我劳累。”

掌管礼仪的官员都说:“古时候天子在夏季亲自到郊外按照礼仪祭祀天帝,所以称为郊祭。”

于是文帝第一次亲临雍县,郊祭五帝,在孟夏四月答谢上天降下祥瑞。赵国人新垣平因为善于望气之术受到召见,趁机劝说文帝在渭阳修建五帝庙,想要使周朝的九鼎重新出现,还会有玉石的精华出现。

\begin{yuanwen}
十六年,上亲郊见渭阳五帝庙,亦以夏答礼而尚赤。
\end{yuanwen}

十六年(前164年),文帝亲自到渭阳的五帝庙举行郊祭,也在夏季答谢上天降下祥瑞,而服饰崇尚红色。

\begin{yuanwen}
十七年,得玉杯,刻曰“人主延寿”。于是天子始更为元年\footnote{即改元。最初没有年号,仅以帝王在位时间纪年,新君即位后改称元年。战国时期,魏惠王、秦惠文王称王时改元,开创了因重大事件改元的先例。汉文帝因“人主延寿”祥瑞改元,为年号的产生提供了理论依据。汉景帝时则两次改元。汉武帝十八年,狩猎时捕获一只独角兽,于是第二年称元狩元年,是为史上第一个年号。},令天下大酺。其岁,新垣平事觉,夷三族。
\end{yuanwen}

十七年(前163年),文帝得到一个玉杯,上面刻有“人主延寿”。于是文帝开始把当年改为元年,下令天下百姓聚会宴饮。当年,新垣平的阴谋败露,诛灭三族。

\begin{yuanwen}
后二年,上曰:“朕既不明,不能远德,是以使方外之国或不宁息。夫四荒之外不安其生,封畿之内勤劳不处,二者之咎,皆自于朕之德薄而不能远达也。间者累年,匈奴并暴边境,多杀吏民,边臣兵吏又不能谕吾内志,以重吾不德也。夫久结难连兵,中外\footnote{中央和地方,泛指天下。}之国将何以自宁?今朕夙兴夜寐,勤劳天下,忧苦万民,为之怛惕不安,未尝一日忘于心,故遣使者冠盖相望\footnote{指冠帽和车盖相连,形容使者往来不断。},结轶\footnote{通“辙”,车轮压出的痕迹。}于道,以谕朕意于单于\footnote{匈奴首领的称号,此时为老上单于,名稽粥。}。今单于反古之道,计社稷之安,便万民之利,亲与朕俱弃细过,偕之大道,结兄弟之义,以全天下元元之民。和亲已定,始于今年。”
\end{yuanwen}

后元二年(前162年),文帝说:“我不够明智,不能把恩德推广到远方,所以使境外的有些国家不能安定和睦。四方边远地区的百姓不能安居乐业,境内的百姓辛勤劳苦不得休息,这两方面的过错,都是由于我的德行浅薄不能泽及远方。最近连续几年,匈奴都来祸害边境,杀死很多官吏和百姓,守边的官兵又不能告谕匈奴我的想法,使我的德行更加浅薄。这样长期结怨交战,天下各国将凭借什么各自安宁呢?现在我早起晚睡,为天下大事操劳,为亿万百姓忧虑,心里惶恐不安,不曾有一天忘记这件事情。所以我派出的使者往来不断,车马络绎不绝,来告谕单于我的想法。现在单于回到以往友好相处的道路上来,考虑国家的安宁,为民众谋求利益,亲自和我忘却小过节,共同走在大道上,结下兄弟般的情谊,以保全天下亿万百姓。已经决定和亲,从今年开始。”

\begin{yuanwen}
后六年冬,匈奴三万人入上郡,三万人入云中。以中大夫令勉为车骑将军,军飞狐;故楚相苏意为将军,军句注;将军张武屯北地;河内守周亚夫为将军,居细柳;宗正刘礼为将军,居霸上;祝兹侯\footnote{据《汉书·文帝纪》,此人为徐厉。据《史记·惠景间侯者年表》,徐厉的封号应为松兹侯。吕后封吕荣为祝兹侯,诸吕之乱平定后,祝兹侯国废为县,直到汉武帝时才封刘延年为祝兹侯,因此汉文帝时无祝兹侯。}军棘门:以备胡。数月,胡人去,亦罢。
\end{yuanwen}

后元六年(前158年)冬季,匈奴三万人进入上郡,三万人进入云中。文帝任命中大夫令勉为车骑将军,驻扎在飞狐;任命原楚国相苏意为将军,驻扎在句注;命令将军张武屯守北地;任命河内郡守周亚夫为将军,驻扎在细柳;任命宗正刘礼为将军,驻军霸上;命令松兹侯驻扎在棘门,以此防备胡人。几个月后,胡人撤离,各路军队也撤退。

\begin{yuanwen}
天下旱,蝗。帝加惠:“令诸侯毋入贡,弛山泽,减诸服御狗马,损郎吏员,发仓庾以振贫民,民得卖爵。”
\end{yuanwen}

天下大旱,蝗虫成灾。文帝施加恩惠:“命令诸侯不要入朝进贡,开放山林湖泽,减省服饰车驾和狗马等物,裁减郎官的名额,打开粮仓来赈济贫苦百姓,允许民间买卖爵位。”

\begin{yuanwen}
孝文帝从代来,即位二十三年,宫室、苑囿、狗马、服御无所增益,有不便,辄弛以利民。尝欲作露台,召匠计之,直百金。上曰:“百金中民十家之产,吾奉先帝宫室,常恐羞之,何以台为!”

上常衣绨\footnote{tí,一种粗厚光滑的丝织品。}衣,所幸慎夫人,令衣不得曳地,帏帐不得文绣,以示敦朴,为天下先。治霸陵\footnote{汉文帝的陵墓。}皆以瓦器,不得以金银铜锡为饰,不治坟,欲为省,毋烦民。南越王尉佗\footnote{赵佗,原为秦朝南海郡尉。}自立为武帝,然上召贵尉佗兄弟,以德报之,佗遂去帝称臣。与匈奴和亲,匈奴背约入盗,然令边备守,不发兵深入,恶烦苦百姓。吴王诈病不朝,就赐几杖\footnote{凭几和手杖,供老年人倚靠和扶持的器物。}。群臣如袁盎等称说虽切,常假借\footnote{宽容。}用之。群臣如张武等受赂遗金钱,觉,上乃发御府金钱赐之,以愧其心,弗下吏。专务以德化民,是以海内殷富,兴于礼义。
\end{yuanwen}

孝文帝从代国来到长安,在位二十三年,宫室、苑囿、狗马、服饰、车驾都没有增加过,有对百姓不便的禁令,就废止来便利百姓。文帝曾经想要修建露台,召来工匠计算费用,价值一百金。文帝说:“一百金相当于十户中等人家的产业,我奉守先帝的宫室,经常担心使先帝蒙羞,还修建露台做什么呢?”

文帝平时穿着粗丝衣服,他所宠爱的慎夫人,也不准穿拖地的衣服,宫室的帷帐不准刺绣花纹,以此表示敦厚俭朴,为天下做表率。文帝修建霸陵,随葬品都用瓦器,不准使用金银铜锡为装饰,不修筑高大的坟丘,想要节省财物,不去烦扰百姓。南越王赵佗自立为武帝,然而文帝却征召他的兄弟赐予爵禄,用恩德回报他,赵佗于是取消帝号向汉朝称臣。汉朝与匈奴和亲,匈奴违背约定入境劫掠,然而文帝只是命令边塞加强守备,不出兵深入匈奴境内,担心会烦扰百姓。吴王谎称生病不来朝见,文帝顺便赐给他凭几和手杖。群臣之中如袁盎等人进谏的言辞虽然激烈,但是文帝经常宽容地采纳他们的建议。群臣之中如张武等人接受别人馈赠的金钱,被发现后,文帝就把宫中府库里的金钱赏赐给他们,以此使他们内心感到羞愧,而不交给官吏治罪。文帝致力于用道德教化臣民,因此四海之内殷实富足,礼仪道义盛行于世。

顾炎武:「史言文帝治霸陵皆以瓦器,不以金银铜锡为饰,刘向亦以孝文薄葬。然考《张汤传》,武帝时已有盗发孝文葬钱者;而晋建兴中盗发汉霸、杜二陵,多获珍宝。盖自春秋以来,厚葬之俗,虽以孝文之明达俭约,犹不能尽除,而史所书未必尽实录也。」

\begin{yuanwen}
后七年六月己亥,帝崩于未央宫。遗诏曰:“朕闻盖天下万物之萌生,靡不有死。死者天地之理,物之自然者,奚(傒)可甚哀。当今之时,世咸嘉生而恶死,厚葬以破业,重服\footnote{过度地服丧。}以伤生,吾甚不取。且朕既不德,无以佐百姓;今崩,又使重服久临,以离寒暑之数,哀人之父子,伤长幼之志,损其饮食,绝鬼神之祭祀,以重吾不德也,谓天下何!朕获保宗庙,以眇眇之身讬于天下君王之上,二十有馀年矣。赖天地之灵,社稷之福,方内安宁,靡有兵革。朕既不敏,常畏过行,以羞先帝之遗德;维年之久长,惧于不终。今乃幸以天年,得复供养于高庙。朕之不明与嘉之,其奚(傒)哀悲之有!其令天下吏民,令到出临三日,皆释服\footnote{脱去丧服。}。毋禁取妇、嫁女祠祀、饮酒、食肉者。自当给丧事服临者,皆无践\footnote{通“跣”,赤足。}。绖带\footnote{束腰的麻带。}无过三寸,毋布车及兵器,毋发民男女哭临宫殿。宫殿中当临者,皆以旦夕各十五举声,礼毕罢。非旦夕临时,禁毋得擅哭。已下,服大红\footnote{大功,丧服的一种,以熟麻制成,较小功粗。下文的“小红”即“小功”。古代丧服按照血缘亲疏关系,分为齐缞、斩缞、大功、小功、缌麻五种,齐缞最重,缌麻最轻。红,gōng。}十五日,小红十四日,纤\footnote{细麻布丧服,即缌麻。}七日,释服。佗不在令中者,皆以此令比率从事。布告天下,使明知朕意。霸陵山川因其故,毋有所改。归夫人以下至少使。”

令中尉亚夫为车骑将军,属国悍为将屯将军,郎中令武为复土将军,发近县见卒万六千人,发内史卒万五千人,藏郭\footnote{开凿墓穴。}、穿\footnote{挖土。}、复土\footnote{下葬后以土覆盖棺椁。}属将军武。
\end{yuanwen}

后元七年(前157年)六月己亥日,文帝在未央宫去世。遗诏说:“我听说大概天下万物的萌发生长,没有不死亡的。死亡是天地之间的常理,是世间万物的自然规律,有什么值得过分悲哀的呢!当今时代,世人都好生而恶死,死后却要厚葬来败光家产,过度服丧来伤害身体,我认为很不可取。况且我德行浅薄,无法扶助百姓。现在去世了,又让人们过度服丧长期致哀,来遭受严寒酷暑的折磨,使别人家的父子为我悲哀,伤害长幼的心灵,减少他们的饮食,禁止对鬼神的祭祀,使我的德行更加浅薄,怎么向天下人交代呢!我能够奉守宗庙,凭借渺小的身躯被天下诸侯王抬举在上,已经有二十多年了。仰仗天地的威灵,社稷的福祉,才使得国内安宁,没有战乱。我不够勤敏,经常担心行为有过失,使先帝遗留下的美德蒙受耻辱,时间长了,害怕不能善终。现在竟然有幸享尽天年,得以再被供奉在高庙里。我不贤明却能有善终,有什么可悲哀的呢!现在命令天下的官吏和百姓,命令下达后出门致哀三天,然后都脱去丧服。不要禁止娶妻、嫁女、祭祀、饮酒、吃肉。本来就应当参加丧礼身穿丧服致哀的人,都不要赤脚。丧服的麻带宽度不要超过三寸,不要陈设车驾和兵器,不要发动民间男女来宫中痛哭致哀。宫中应当致哀的人,都在每天早晚各哭十五声,行礼完毕就停止。不是早晚应当致哀的时刻,不准擅自哭泣。下葬之后,应当穿大功服丧九个月的只穿十五天,应当穿小功服丧五个月的只穿十四天,应当穿缌麻服丧三个月的只穿七天,然后脱去丧服。其他不在规定范围内的,都参照这一诏令办理。将诏令布告天下,使天下人明白我的心意。霸陵周围的山川要保持原样,不要有所改变。后宫夫人以下到少使的嫔妃都遣散回家。”

朝廷任命中尉周亚夫为车骑将军,典属国徐悍为将屯将军,郎中令张武为复土将军,征发长安附近各县现役士卒一万六千人,又征发内史士卒一万五千人,归将军张武指挥,负责凿穴、挖土、填土。

\begin{yuanwen}
乙巳,群臣皆顿首上尊号曰孝文皇帝。

太子即位于高庙。丁未,袭号曰皇帝。
\end{yuanwen}

六月乙巳日,群臣都叩头为皇帝上尊号为孝文皇帝。

太子在高庙即位。丁未日,继承皇帝称号。

\begin{yuanwen}
孝景皇帝元年十月,制诏御史:“盖闻古者祖有功而宗有德,制礼乐各有由。闻歌者,所以发德也;舞者,所以明功也。高庙酎\footnote{zhòu,祭祀时进献醇酒。},奏《武德》、《文始》、《五行》之舞。孝惠庙酎,奏《文始》、《五行》之舞。孝文皇帝临天下,通关梁,不异远方。除诽谤,去肉刑,赏赐长老,收恤孤独,以育群生。减嗜欲,不受献,不私其利也。罪人不帑,不诛无罪。除宫刑,出美人,重绝人之世。朕既不敏,不能识。此皆上古之所不及,而孝文皇帝亲行之。德厚侔\footnote{móu}天地,利泽施四海,靡不获福焉。明象乎日月,而庙乐不称。朕甚惧焉。其为孝文皇帝庙为《昭德》之舞,以明休德。然后祖宗之功德著于竹帛,施于万世,永永无穷,朕甚嘉之。其与丞相、列侯、中二千石、礼官具为礼仪奏。”

丞相臣嘉等言:“陛下永思孝道,立《昭德》之舞以明孝文皇帝之盛德。皆臣嘉等愚所不及。臣谨议:世功莫大于高皇帝,德莫盛于孝文皇帝,高皇庙宜为帝者太祖之庙,孝文皇帝庙宜为帝者太宗之庙。天子宜世世献祖宗之庙。郡国诸侯宜各为孝文皇帝立太宗之庙。诸侯王列侯使者侍祠天子,岁献祖宗之庙。请著之竹帛,宣布天下。”

制曰:“可。”
\end{yuanwen}

孝景皇帝元年(前156年)十月,制诏御史说:“我听说古时候有功的帝王称为祖,有德的帝王称为宗,制定礼仪和音乐各有理由。我听说歌是用来颂扬德行的,舞是用来显示功绩的。在高庙献酒时,表演《武德》《文始》《五行》乐舞。在孝惠庙献酒时,表演《文始》《五行》乐舞。孝文皇帝治理天下,开放了关塞桥梁,使远处没有差别。他废除诽谤罪,取消肉刑,赏赐老年人,收养孤儿独老,以此养育众生。他减少嗜欲,不接受贡物,不私自占有利益。罪犯的妻子儿女不受牵连,不诛罚无罪的人。废除宫刑,放出后宫美人,重视断绝后嗣的问题。我不勤敏,不能明白。这些事情都是上古帝王所不能做到的,而孝文皇帝却亲自实行了。他功德显赫与天地平齐,恩惠广施而遍及四海,没有人不曾享受福祉。他的光辉宛如日月,可是宗庙祭祀所采用的乐舞并不相称,我很恐惧。应该为孝文皇帝庙创作《昭德》乐舞,来彰明他的美德。然后将祖宗的功德记载在史册上,流传万世,永无尽头,我很赞成这种做法。这件事交给丞相、列侯、中二千石的官吏、掌管礼仪的官吏共同拟定礼仪奏上。”

丞相申徒嘉等人说:“陛下始终考虑孝道,创作《昭德》乐舞来彰显孝文皇帝的盛大功德,都是我们由于愚昧而想不到的。我们恭敬地建议:世间的功绩没有人超过高皇帝,德行没有人超过孝文皇帝,高皇帝庙应为本朝帝王的太祖庙,孝文皇帝庙应为本朝帝王的太宗庙。天子应世代祭祀太祖庙和太宗庙。郡国诸侯应为孝文皇帝建立太宗庙。诸侯王和列侯派使者陪同天子祭祀,每年向太祖庙和太宗庙进献祭品。请把这些规定记录下来,向天下人公布。”

制命说:“可以。”

\begin{yuanwen}
太史公曰:孔子言\footnote{语出《论语·子路》。}“必世\footnote{三十年为一世。}然后仁。善人之治国百年,亦可以胜残去杀”。诚哉是言!汉兴,至孝文四十有馀载,德至盛也。廪廪\footnote{逐渐。}乡\footnote{xiàng}改正服、封禅矣,谦让未成于今。呜呼,岂不仁哉!
\end{yuanwen}

太史公说:孔子说:“一定要经过三十年才能广施仁德。善人治理国家一百年,也可以战胜残暴,废除刑杀了。”这句话实在是有道理啊!汉朝建立以来,到孝文皇帝有四十多年,德政已经兴盛,逐渐接近修改历法服色、举行封禅这一目标了,只是由于谦让才至今没有完成。啊,这难道不是仁德吗!

\begin{yuanwen}
孝文在代,兆遇大横。宋昌建册,绛侯奉迎。南面而让,天下归诚。务农先籍,布德偃兵。除帑削谤,政简刑清。綈衣率俗,露台罢营。法宽张武,狱恤缇萦。霸陵如故,千年颂声。
\end{yuanwen}

\part{卷十一}
\chapter{孝景本纪第十一}

《汉书》载:「景帝五年,遣公主嫁匈奴单于;六年,皇后薄氏废;中二年,改磔曰弃市,勿复磔;六年,灭笞法,定箠令。此皆大事,不得不书者,《史记》皆略之。」

本篇是汉景帝刘启的本纪,记述了他在位期间的大事。景帝继位后,基本延续了文帝时的政策,同时又根据时势的变化强化了朝廷的权力,为武帝的大一统奠定了基础。

\begin{yuanwen}
孝景皇帝者,孝文之中子也。母窦太后。孝文在代时,前后\footnote{文帝为代王时的王后。}有三男,及窦太后得幸,前后死,及三子更\footnote{相继。}死,故孝景得立。
\end{yuanwen}

孝景皇帝,是孝文皇帝排在中间的儿子。他的母亲是窦太后。孝文皇帝在代国之时,前任王后曾生了三个儿子,到了窦太后得宠时,前任王后去世,她的三个儿子也陆续死去,因此孝景皇帝得以嗣位。

\begin{yuanwen}
元年四月乙卯,赦天下。乙巳,赐民爵一级。

五月,除田半租,为孝文立太宗庙。令群臣无朝贺。匈奴入代,与约和亲。
\end{yuanwen}

景帝元年(前156年)四月乙卯日,大赦天下。乙巳日,赐百姓每人一级爵位。

五月,减免全国农户一半的田租,为孝文皇帝建立了太宗庙。要求大臣们不要为自己即位一事上朝拜贺。允许匈奴进入代地,与匈奴约定和亲。

\begin{yuanwen}
二年春,封故相国萧何孙系为武陵侯。男子二十而得傅。

四月壬午,孝文太后\footnote{文帝生母薄太后。}崩。广川、长沙王皆之国。丞相申屠嘉卒。

八月,以御史大夫开封侯陶青为丞相。彗星出东北。

秋,衡山雨雹,大者五寸,深者二尺。荧惑逆行,守\footnote{逼近。}北辰。月出北辰间。岁星逆行天廷中。置南陵及内史、祋\footnote{duì}祤\footnote{yǔ}为县。
\end{yuanwen}

二年(前155年)春季,册封从前的相国萧何之孙萧系为武陵侯。规定男子到了二十岁时要记录在册。

四月壬午日,孝文太后去世。广川王、长沙王都前往自己的封国。丞相申屠嘉去世。

八月,任命御史大夫开封侯陶青为丞相。彗星在东北方出现。

秋季,衡山降下冰雹,最大的冰雹直径可达五寸,最深的地方可达二尺。火星反向运行,逼近北极星。月亮出现在北极星的位置。木星在天庭中逆向运行。皇帝将南陵和内史、祋祤设置成县。

\begin{yuanwen}
三年正月乙巳,赦天下。长星\footnote{彗星。}出西方。天火燔雒阳东宫大殿城室。吴王濞、楚王戊、赵王遂、胶西王卬、济南王辟光、菑川王贤、胶东王雄渠反,发兵西乡。天子为诛晁错,遣袁盎谕告,不止,遂西围梁。上乃遣大将军窦婴、太尉周亚夫将兵诛之。

六月乙亥。赦亡军及楚元王子蓺等与谋反者。封大将军窦婴为魏其侯。立楚元王子平陆侯礼为楚王。立皇子端为胶西王,子胜为中山王。徙济北王志为菑川王,淮阳王馀为鲁王,汝南王非为江都王。齐王将庐、燕王嘉皆薨\footnote{hōng}。
\end{yuanwen}

三年(前154年)正月乙巳日,大赦天下。有一颗长尾慧星出现在西方。天火烧毁了雒阳东宫的大殿和城楼。吴王刘濞、楚王刘戊、赵王刘遂、胶西王刘卬、济南王刘辟光、淄川王刘贤、胶东王刘雄渠发动叛乱,起兵向西方进发。皇帝为安抚反叛的各王而杀掉了晁错,派袁盎通告七国,但七国仍不罢兵,继续向西包围了梁国。皇帝因此派大将军窦婴和太尉周亚夫率军讨伐,消灭了他们。

六月乙亥日,皇帝赦免了战败逃跑的士卒和楚元王的儿子刘蓺等参与谋反的人。册封大将军窦婴为魏其侯。立楚元王的儿子平陆侯刘礼为楚王。立皇子刘端为胶西王,皇子刘胜为中山王。改封济北王刘志为淄川王,淮阳王刘馀为鲁王,汝南王刘非为江都王。齐王刘将庐、燕王刘嘉都去世了。

\begin{yuanwen}
四年夏,立太子\footnote{指栗姬所生的刘荣。}。立皇子彻\footnote{刘彻,即后来的汉武帝,妃王氏所生。}为胶东王。

六月甲戌,赦天下。

后九月\footnote{闰九月。},更以易阳为阳陵。复置津关,用传\footnote{通行的证件。}出入。

冬,以赵国为邯郸郡。
\end{yuanwen}

四年(前153年)夏季,立皇太子。立皇子刘彻为胶东王。

六月甲戌日,大赦天下。

闰九月,把弋阳改名为阳陵。重新在渡口和关口设置关卡,凭借证件出入。

冬季,撤消赵国设置为邯郸郡。

\begin{yuanwen}
五年三月,作阳陵、渭桥。

五月,募徙阳陵,予钱二十万。江都大暴风从西方来,坏城十二丈。

丁卯,封长公主子蟜为隆虑侯。徙广川王为赵王。
\end{yuanwen}

五年(前152年)三月,修筑阳陵和渭桥。

五月,拨钱二十万,招募各地百姓迁往阳陵。江都地区遭到西边刮来的大风暴侵袭,被大风毁坏的城墙达十二丈。

丁卯日,皇帝封他的姐姐长公主的儿子为隆虑侯。改封广川王为赵王。

\begin{yuanwen}
六年春,封中尉绾为建陵侯,江都丞相嘉为建平侯,陇西太守浑邪为平曲侯,赵丞相嘉为江陵侯,故将军布为鄃\footnote{shū}侯。梁楚二王皆薨。

后九月,伐驰道树,殖兰池。
\end{yuanwen}

六年(前151年)春季,封中尉卫绾为建陵侯,江都国丞相程嘉为建平侯,陇西郡太守公孙浑邪为平曲侯,赵国丞相苏嘉为江陵侯,从前的将军栾布为鄃侯。梁王、楚王都去世了。

闰九月,砍掉了驰道两旁的树木,填平了兰池。

\begin{yuanwen}
七年冬,废栗太子为临江王。

十(一)月晦,日有食之。

春,免徒隶作阳陵者。丞相青\footnote{陶青。}免。

二月乙巳,以太尉条侯周亚夫为丞相。

四月乙巳,立胶东王太后为皇后。丁巳,立胶东王为太子。名彻。
\end{yuanwen}

七年(前150年)冬季,废除栗太子刘荣,封他为临江王。

十一月晦日,发生了日食。

春季,赦免了修筑阳陵的囚犯和奴隶。丞相陶青被免职。

二月乙巳日,任命太尉条侯周亚夫为丞相。

四月乙巳日,册立胶东王的母亲为皇后。丁巳日,册立胶东王为太子。太子名彻。

\begin{yuanwen}
中元年,封故御史大夫周苛\footnote{刘邦部将,后被项羽所杀。}孙平为绳侯,故御史大夫周昌\footnote{周苛堂弟,汉初功臣。}孙左车为安阳侯。

四月乙巳,赦天下,赐爵一级。除禁锢。地动。衡山、原都雨雹,大者尺八寸。
\end{yuanwen}

中元元年(前149年),封前御史大夫周苛的孙子周平为绳侯,前御史大夫周昌的孙子周左车为安阳侯。

四月乙巳日,大赦天下,赐百姓每人一级爵位。废除不准商人、入赘女婿做官和不准犯过罪的官吏重新做官的禁令。发生地震。衡山、原都地区下了冰雹,最大的直径可达一尺八寸。

\begin{yuanwen}
中二年二月,匈奴入燕,遂不和亲。

三月,召临江王来。即死中尉府中。

夏,立皇子越为广川王,子寄为胶东王。封四侯。

九月甲戌,日食。
\end{yuanwen}

中元二年(前148年)二月,匈奴侵入燕地,朝廷因此拒绝与匈奴和亲。

三月,下令召临江王刘荣来都城问罪。他很快就死在了中尉郅都的府第中。

夏季,册立皇子刘越为广川王,皇子刘寄为胶东王。分封了四个列侯。

九月甲戌日,发生了日食。

\begin{yuanwen}
中三年冬,罢诸侯御史中丞。

春,匈奴王二人率其徒来降,皆封为列侯。立皇子方乘为清河王。

三月,彗星出西北。丞相周亚夫免,以御史大夫桃侯刘舍为丞相。

四月,地动。

九月戊戌晦,日食。军东都门外。
\end{yuanwen}

中元三年(前147年)冬季,废除了诸侯王国中御史中丞的官职。

春季,有两位匈奴王率领自己的部众前来归降,都被封为列侯。册立皇子刘方乘为清河王。

三月,彗星出现在天空的西北方。丞相周亚夫被免职,任命御史大夫桃侯刘舍为丞相。

四月,发生地震。

九月戊戌晦日,发生了日食。派军队驻扎在京城的东都门外。

\begin{yuanwen}
中四年三月,置德阳宫。大蝗。秋,赦徒作阳陵者。
\end{yuanwen}

中元四年(前146年)三月,修建德阳宫。发生了大蝗灾。

秋季,赦免建造阳陵的囚犯。

\begin{yuanwen}
中五年夏,立皇子舜为常山王。封十侯。

六月丁巳,赦天下,赐爵一级。天下大潦\footnote{同“涝”,雨多成灾。}。更命诸侯丞相曰相。秋,地动。
\end{yuanwen}

中元五年(前145年)夏季,册立皇子刘舜为常山王。分封了十个人为侯。

六月丁巳日,大赦天下,赐予百姓每人一级爵位。全国范围内发生严重涝灾。将诸侯国的丞相改称为相。秋季,发生地震。

\begin{yuanwen}
中六年二月己卯,行幸雍,郊见五帝。

三月,雨雹。

四月,梁孝王、城阳共王、汝南王皆薨。立梁孝王子明为济川王,子彭离为济东王,子定为山阳王,子不识为济阴王。梁分为五。封四侯。更命廷尉为大理,将作少府为将作大匠,主爵中尉为都尉,长信詹事为长信少府,将行为大长秋,大行为行人,奉常为太常,典客为大行,治粟内史为大农。以大内为二千石,置左右内官,属大内。

七月辛亥,日食。

八月,匈奴入上郡。
\end{yuanwen}

中元六年(前144年)二月己卯日,皇帝来到雍县,郊祀五帝庙。三月,下了冰雹。

四月,梁孝王、城阳共王和汝南王都去世了。分别立梁孝王的儿子刘明为济川王,刘彭离为济东王,刘定为山阳王,刘不识为济阴王。把原来的梁国一分为五。封四人为列侯。把廷尉这个官职改名为大理,将作少府改名为将作大匠,主爵中尉则改名为都尉,长信詹事改名为长信少府,将行改名为大长秋,大行改名为行人,奉常改名为太常,典客改名为大行,治粟内史改名为大农。将主管大内仓库的大内定为二千石级别的官员,设置左、右内官,使其隶属于大内。

七月辛亥日,发生了日食。

八月,匈奴入侵上郡。

\begin{yuanwen}

后元年冬,更命中大夫令为卫尉。

三月丁酉,赦天下,赐爵一级,中二千石、诸侯相爵右庶长。

四月,大酺\footnote{聚饮。}。

五月丙戌,地动,其蚤食时复动。上庸地动二十二日,坏城垣。

七月乙巳,日食。丞相刘舍免。

八月壬辰,以御史大夫绾\footnote{wǎn}为丞相,封建陵侯。
\end{yuanwen}

后元元年(前143年)冬季,将中大夫令改名为卫尉。

三月丁酉日,大赦天下,赐予百姓每人一级爵位,赐给中中二千石的官员和诸侯国的相以右庶长的爵位。

四月,下令允许百姓聚会饮酒。

五月丙戌日,发生地震,早饭时又震。上庸县地震持续二十二天,城墙被震坏。

七月乙巳日,发生了日食。丞相刘舍被免职。

八月壬辰日,任命御史大夫卫绾为丞相,册封为建陵侯。

\begin{yuanwen}
后二年正月,地一日三动。郅将军击匈奴。酺五日。令内史郡不得食马粟,没入县官。令徒隶衣七緵\footnote{zòng}布。止马舂。为岁不登,禁天下食不造岁。省列侯遣之国。

三月,匈奴入雁门。

十月,租长陵田。大旱。衡山国、河东、云中郡民疫。
\end{yuanwen}

后元二年(前142年)正月,一天之内接连发生三次地震。郅都将军率军出击匈奴。下令允许百姓聚会饮酒五日。诏令内史和各郡不准用粮食喂马,违者将其马匹收归官府。规定罪犯和奴隶只准穿粗劣的衣服。严禁用马舂米。由于当年粮食歉收,诏令天下百姓不准在一年内就将当年收获的口粮吃完。削减驻京的列侯,让他们都回到自己的封地。

三月,匈奴侵入雁门郡。

十月,把高祖陵墓长陵周围的耕地租给农民耕种。发生大旱灾。衡山国、河东郡和云中郡都有瘟疫流行。

\begin{yuanwen}
后三年十月,日月皆食赤五日。

十二月晦,䨓\footnote{léi}袴。日如紫。五星逆行守太微。月贯天廷中。

正月甲寅,皇太子冠。

甲子,孝景皇帝崩。遗诏赐诸侯王以下至民为父后\footnote{父亲的继承人。}爵一级,天下户百钱。出宫人\footnote{宫女。}归其家,复无所与。太子即位,是为孝武皇帝。

三月,封皇太后弟蚡\footnote{fén}为武安侯,弟胜为周阳侯。置阳陵。
\end{yuanwen}

后元三年(前141年)十月,发生了日食和月食,太阳和月亮连续五天呈现为红色。

十二月晦日,发生雷震。太阳变成紫色。五大行星向相反的方向运行,逼近太微垣。月亮横着穿过天庭。

正月甲寅日,皇太子刘彻举行加冠典礼。

甲子日,孝景皇帝去世。遗诏赐给诸侯王以下一直到百姓中父亲的继承人每人一级爵位,全国百姓每户一百钱。将宫女放回老家,免除了他们终身的赋税和徭役。太子即位为帝,就是孝武皇帝。

三月,册封皇太后的弟弟田蚡为武安侯,田胜为周阳侯。并把景帝的灵柩安置在阳陵。

\begin{yuanwen}
太史公曰:汉兴,孝文施大德,天下怀安,至孝景,不复忧异姓,而晁错刻削诸侯,遂使七国俱起,合从而西乡\footnote{同“西向”。},以诸侯太盛,而错为之不以渐也。及主父偃\footnote{姓主父,名偃,汉武帝时大臣。}言之,而诸侯以弱,卒以安。安危之机,岂不以谋哉?
\end{yuanwen}

太史公说:汉兴之后,孝文皇帝广施恩德,天下百姓得以怀恩而安宁。到了孝景皇帝时,已经无需再忧虑异姓诸侯王的反叛了。但是晁错严厉地削夺同姓诸侯王的封地,因此使得吴、楚七国一同起兵反叛,联合向西攻打朝廷。这是因为诸侯王势力还太过强大,而晁错又没有采取渐渐削弱的方法。到了主父偃,建议准许诸侯王分封自己的子弟为侯,从而使诸侯王逐渐被削弱,国家终于安定下来。所以说,国家安危的关键,难道不得依靠谋略吗?

\begin{yuanwen}
景帝即位,因脩静默。勉人于农,率下以德。制度斯创,礼法可则。一朝吴楚,乍起凶慝。提局成衅,拒轮致惑。晁错虽诛,梁城未克。条侯出将,追奔逐北。坐见枭黥,立翦牟贼。如何太尉,后卒下狱。惜哉明君,斯功不录!
\end{yuanwen}

\part{卷十二}

\chapter{孝武本纪第十二}

本篇是汉武帝刘彻的本纪,记述了这位集雄主与暴君于一身的帝王一生中经历的重大事件。武帝尊儒术、改正朔、征匈奴、平南越、行平准,可谓政绩卓越,然而他又穷奢极欲、穷兵黩武,晚年的暴政加重了民众的负担,巫蛊之祸更是导致宫廷大乱。《史记·今上本纪》失传,今本《孝武本纪》是后人据《封禅书》等材料抄改而成。

\begin{yuanwen}
孝武皇帝者,孝景中子也。母曰王太后。孝景四年,以皇子为胶东王。孝景七年,栗太子\footnote{名荣,景帝长子,生母为栗姬。}废为临江王,以胶东王为太子。孝景十六年崩,太子即位,为孝武皇帝。孝武皇帝初即位,尤敬鬼神之祀。
\end{yuanwen}

孝武皇帝,是孝景帝排行居中的儿子。他的母亲是王太后。孝景帝四年(前153年),他以皇子的身份被封为胶东王。孝景在位的第七年,栗太子被废黜为临江王,立胶东王为太子。景帝在位十六年去世,太子即位,就是孝武皇帝。孝武皇帝刚刚即位,就尤其注重对鬼神的祭祀。

\begin{yuanwen}
元年\footnote{指建元元年。},汉兴已六十馀岁矣,天下乂安,荐绅之属皆望天子封禅改正度也。而上乡儒术,招贤良,赵绾、王臧等以文学为公卿,欲议古立明堂城南,以朝诸侯。草巡狩封禅改历服色事未就。会窦太后治黄老言,不好儒术,使人微得\footnote{私下查明。}赵绾等奸利事,召案绾、臧,绾、臧自杀,诸所兴为者皆废。
\end{yuanwen}

建元元年(前140年),汉朝已经建立有六十多年了,天下安定太平,朝廷中的大臣们都企盼天子能够举行封禅仪式,修正各种制度。而且天子也十分崇信儒家学说,就招纳四方贤良方正之人。赵绾、王臧等人凭借他们文章的博学而被封为公卿,他们都想建议天子遵照古制在城南筑造一座宣明政教的明堂,并在此朝会诸侯。他们草拟了天子出巡狩猎、封禅及改换历法服色制度等的计划,但并未完成。当时窦太后崇尚信奉黄帝及老子的道家之言,不喜儒家之术,派人私下调察赵绾等人作奸犯科的行为,传讯赵绾、王臧,赵绾、王臧自杀,他们所提议兴办的诸多举措都被废弃。

\begin{yuanwen}
后六年,窦太后崩。其明年,上徵文学之士公孙弘等。
\end{yuanwen}

六年后,窦太后去世。第二年,皇帝征召了学识渊博之士公孙弘等人。

\begin{yuanwen}
明年,上初至雍,郊见五畤。后常三岁一郊。是时上求神君,舍之上林中氾(蹏)氏观。神君者,长陵女子,以子死悲哀,故见神于先后宛若。宛若祠之其室,民多往祠。平原君\footnote{即臧儿,武帝外祖母,王太后之母。}往祠,其后子孙以尊显。及武帝即位,则厚礼置祠之内中,闻其言,不见其人云。
\end{yuanwen}

又过了一年,皇帝第一次来到雍县,在五畤进行郊祀。后来经常每三年举行一次郊祀。当时皇帝求到一个神君,并把她供奉在上林苑中的蹏氏观。那个神君其实是长陵的一个女子,由于儿子夭折,结果悲哀而死,她死后就显灵在她的妯娌宛若身上。宛若在自己家里供奉她,后来很多人都去她家祭祀。皇帝的外祖母平原君也曾去祭祀过她,于是她的后代都获得了尊贵的地位,声名日盛。等到武帝即位后,就以很隆重的礼仪把她安置在后宫供奉,能够听到神君的说话声,但却看不见她本人。

\begin{yuanwen}
是时而李少君亦以祠灶、穀道、卻老\footnote{延缓衰老。}方见上,上尊之。少君者,故深泽侯入以主方。匿其年及所生长,常自谓七十,能使物,卻老。其游以方遍诸侯。无妻子。人闻其能使物及不死,更馈遗之,常馀金钱帛衣食。人皆以为不治产业而饶给,又不知其何所人,愈信,争事之。少君资好方,善为巧发奇中。尝从武安侯饮,坐中有年九十馀老人,少君乃言与其大父游射处,老人为儿时从其大父行,识其处,一坐尽惊。少君见上,上有故铜器,问少君。少君曰:“此器齐桓公十年陈于柏寝。”

已而案其刻,果齐桓公器。一宫尽骇,以少君为神,数百岁人也。
\end{yuanwen}

当时李少君也凭借会祭灶致福、辟谷不食、长生不老的方术觐见皇帝,皇帝很尊敬他。李少君这个人是由已经去世的深泽侯推荐来管理方术之事的。他隐瞒年龄和自己生长的地方,常常自称已经七十岁了,能驱使神灵,长生不老。他凭借方术游遍了诸侯各国。他并无妻子和儿女。人们听说他能驱使神灵,还能长生不老,就馈赠财物给他,所以就有很多金钱、丝织品、衣服和食物。人们都认为他没有经营产业却十分富有,再加上不知他是哪里的人,因此更加相信他,都争着去侍奉他。李少君天性就喜欢方术,善于猜测,说中人的一些隐私。他曾经和武安侯一起宴饮,在座的有一位九十多岁的老人,李少君就说起曾和老人的祖父在一起游玩射猎之地,老人小的时候曾跟在祖父身边,知道那些地方,结果在座的宾客都非常惊讶。少君由此觐见皇帝,皇帝有一件古老的铜器,向少君询问。少君回答说:“这件铜器曾经在齐桓公十年时摆设在柏寝台。”

不久检查铜器上刻的铭文,果然是齐桓公时期的器物。整个宫里的人都十分惊骇,把少君当成了神,是有几百岁的人。

\begin{yuanwen}
少君言于上曰:“祠灶则致物,致物而丹沙可化为黄金,黄金成以为饮食器则益寿,益寿而海中蓬莱仙者可见,见之以封禅则不死,黄帝是也。臣尝游海上,见安期生\footnote{人称千岁翁,安丘先生,是楚汉时期一个具有神秘色彩的人物。},食臣枣,大如瓜。安期生仙者,通蓬莱中,合则见人,不合则隐。”

于是天子始亲祠灶,而遣方士入海求蓬莱安期生之属,而事化丹沙诸药齐为黄金矣。
\end{yuanwen}

李少君曾对皇帝说:“祭祀灶神就能招致神灵,招致神灵后朱砂就能化成黄金,黄金炼成后就能够以它铸造饮食器具,使用这种器具就能益寿延年,延长寿命就能够见到东海蓬莱岛的仙人,见到蓬莱仙人后再举行封禅大典就能够长生不死了,黄帝就是这样做的。我曾游历在海上,见过安期生,他给了我一个枣吃,那枣大得像瓜。安期生是个仙人,与蓬莱岛的人们往来,符合他脾气的,他就相见,不符合的他就藏起来不见。”

于是天子开始亲自祭祀灶神,并派方士去东海求访安期生那样的仙人,同时开始试着用丹砂等药剂炼制黄金。

\begin{yuanwen}
居久之,李少君病死。天子以为化去不死也,而使黄锤史宽舒受其方。求蓬莱安期生莫能得,而海上燕齐怪迂之方士多相效,更言神事矣。
\end{yuanwen}

过了很长时间,李少君生病死去。皇帝认为他是成仙而不是死去了,就让东莱郡属吏宽舒学习他的方术。访求蓬莱仙人安期生并没能成功,而沿海一带的燕、齐地区很多荒唐迂腐的方士却都效仿李少君,争相谈论神仙之事。

\begin{yuanwen}
亳人薄诱忌奏祠泰一方,曰:“天神贵者泰一,泰一佐\footnote{辅佐者。}曰五帝。古者天子以春秋祭泰一东南郊,用太牢具,七日,为坛开八通之鬼道。”

于是天子令太祝立其祠长安东南郊,常奉祠如忌方。其后人有上书,言“古者天子三年一用太牢具祠神三一:天一,地一,泰一”。

天子许之,令太祝领祠之忌泰一坛上,如其方。后人复有上书,言“古者天子常以春秋解祠,祠黄帝用一枭破镜;冥羊用羊;祠马行用一青牡马;泰一、皋山山君、地长用牛;武夷君用乾鱼;阴阳使者以一牛”。

令祠官领之如其方,而祠于忌泰一坛旁。
\end{yuanwen}

亳县人薄诱忌向朝廷上奏了祭祀泰一神的方法,他说:“天神中最尊贵的就属泰一神,泰一的辅佐神是五帝。古时天子会在春秋两季到东南方的郊外祭祀泰一神,用牛、羊、猪三牲的太牢祭祀七天,并且筑造祭坛,在八方设置代鬼神行走的通道。”

于是皇帝就命令太祝在长安城的东南郊建造泰一神祠,并常常依照薄诱忌提供的方法供奉、祭祀。后来有人上书说:“古时天子每隔三年一次,用牛、羊、猪三牲的太牢祭祀三一之神,也就是:天一神、地一神和泰一神。”

皇帝答应了,就命令太祝带领群臣祭祀,在泰一神的神坛同时供奉三一之神,严格遵照上书人提到的方法进行。后来再次有人上书说:“古时天子常常于春秋两季祭祀以解罪求福,祭祀黄帝要用一枭和一破镜;祭祀羊神要用羊;祭祀马行神要用一匹青色雄马;祭祀泰一神、皋山山君和地长神要用牛;祭祀武夷君则用干鱼;祭祀阴阳使者神要用一头牛。”

皇帝就命令祠官依照上书人所给的方法办理,并在泰一神的神坛边上建了一座神祠。

\begin{yuanwen}
其后,天子苑\footnote{即上林苑。}有白鹿,以其皮为币,以发瑞应,造白金焉。
\end{yuanwen}

后来,天子的上林苑中生养有白鹿,有人说用白鹿的皮制成皮币,可以引发吉祥的征兆,就又制造了一些白金。

\begin{yuanwen}
其明年,郊雍,获一角兽,若麃然。有司曰:“陛下肃祗\footnote{恭敬。}郊祀,上帝报享,锡一角兽,盖麟云。”

于是以荐五畤,畤加一牛以燎。赐诸侯白金,以风符应合于天地。
\end{yuanwen}

到了第二年,皇帝前往雍县举行郊祀,捕获了一头独角的兽类,外形就像一头麃子。相关官员说:“陛下您恭敬地举行郊祀,上天的帝君为了报答您的供奉,就赐了这头独角兽给您,这应该就是麒麟。”

如此就用它来祭祀五帝,而且又加了一头牛来进行焚柴祭天的燎祭。赐予诸侯白金,以此来暗示他们这种吉祥的先兆是符合天地之意的。

\begin{yuanwen}
于是济北王以为天子且封禅,乃上书献泰山及其旁邑。天子受之,更以他县偿之。常山王有罪,迁,天子封其弟于真定,以续先王祀,而以常山为郡。然后五岳皆在天子之郡。
\end{yuanwen}

由此济北王认为皇帝将要举行封禅典礼了,就上奏把泰山及其附近城邑的封禅用地献给皇帝。皇帝接受了,另外赐给他一些其他县邑作为补偿。常山王有罪,被流放,皇帝就将他的弟弟封在真定,以此延续对其先人的祭祀,并把常山设置为郡。这样以后,五岳就都在天子直接管辖的郡内了。

\begin{yuanwen}
其明年,齐人少翁以鬼神方见上。上有所幸王夫人,夫人卒,少翁以方术盖夜致\footnote{在夜里招来。}王夫人及灶鬼之貌云,天子自帷中望见焉。于是乃拜少翁为文成将军,赏赐甚多,以客礼礼之。

文成言曰:“上即欲与神通,宫室被服不象神,神物不至。”

乃作画云气车,及各以胜日驾车辟恶鬼。又作甘泉宫,中为台室,画天、地、泰一诸神,而置祭具以致天神。居岁馀,其方益衰,神不至。乃为帛书以饭牛,详弗知也,言此牛腹中有奇。杀而视之,得书,书言其怪,天子疑之。有识其手书\footnote{手迹。},问之人,果伪书。于是诛文成将军而隐之。
\end{yuanwen}

第二年的时候,齐人少翁凭借精通鬼神之道得以觐见皇帝。皇帝十分宠爱的王夫人去世了,少翁就使用方术在夜里招来王夫人和灶神的形貌,天子在帷幕中就能望见。于是就封少翁为文成将军,赏赐他很多东西,并用待宾客的礼仪对待他。

文成将军说道:“皇上倘若想和神相交往,但宫室、被服等生活用品却根本不像神所用的,神灵就不会降临。”

因此就制造了画有云气的车子,并在不同日子里分别驾驶不同颜色的车子祛除恶鬼。又建造了甘泉宫,并在宫中修建了高台宫室,室中画有天、地和泰一等众多神灵,并摆设了祭祀用具,想要以此招来天神。又过了一年多,他的方术越来越不灵验,神仙一直都没有来。文成将军就写了一些字在一块帛上,让牛吃掉,并装作自己毫不知情,对别人说这头牛的肚子很奇怪。杀了牛再看,得到了那块写有字的帛,上面写的话十分奇怪,皇帝对这件事有所怀疑。有人认出那是文成将军的字迹,问有关的人,果然是少翁伪造的书帛。于是皇帝杀掉了文成将军,并隐瞒了这件事。

\begin{yuanwen}
其后则又作柏梁\footnote{柏梁台,传说是用“香柏”所造,高数十丈,在当时未央宫北的桂宫内。}、铜柱、承露仙人掌之属矣。
\end{yuanwen}

在此之后,又建造了柏梁台、铜柱和承露仙人掌之类的建筑。

\begin{yuanwen}
文成死明年,天子病鼎湖\footnote{宫名。}甚,巫医无所不致,不愈。游水发根乃言曰:“上郡有巫,病而鬼下之。”

上召置祠之甘泉。及病,使人问神君。神君言曰:“天子毋忧病。病少愈,强与我会甘泉。”

于是病愈,遂幸甘泉,病良已。大赦天下,置寿宫神君。神君最贵者太一,其佐曰大禁、司命之属,皆从之。非可得见,闻其音,与人言等。时去时来,来则风肃然也。居室帷中。时昼言,然常以夜。天子祓,然后入。因巫为主人,关饮食。所欲者言行下。又置寿宫、北宫,张羽旗,设供具,以礼神君。神君所言,上使人受书其言,命之曰“画法”。其所语,世俗之所知也,毋绝殊者,而天子独喜。其事(祕/秘),世莫知也。
\end{yuanwen}

文成将军被处死的第二年,皇帝在鼎湖宫病得很严重,巫医们想尽了办法,皇帝的病也没有治好。有个叫游水发根的人上奏说:“上郡有个巫师,他有次生病鬼神就附上了他的身体。”

皇帝召来巫师并供奉在甘泉宫。等到巫师生病之时,就派人问上了巫师身的神君。神君说:“天子不用担心自己的病情。您的病不久就会痊愈,请您到时振作精神和我在甘泉宫相会。”

于是皇帝的病好了很多,就亲自去甘泉宫祭祀,病就彻底好了。皇帝大赦天下,将神君安置在寿宫中。神君里面最尊贵的就是太一神,他的辅佐神包括大禁、司命这一类神仙,这些神仙都跟随着他。众神君的样子是看不见的,但可以听到他们的声音,和人的声音相同。神君们有时离开有时回来,回来的时候就会有沙沙的风声。他们都居住在室内的帷帐中,有时白天会说话,但常常是在夜里说话。皇帝要先斋戒,之后才能进入室内。皇帝把巫师当成这里的主人,让他给神君送饮食。倘若神君想要说话,也是先告诉巫师,由巫师向下传达。皇帝又建造了寿宫、北宫,张挂羽旗,摆设祭器,以此礼敬神君。神君说过的话,皇帝都让人记录下来,为它命名为“画法”。但其实神君说过的话,也是世俗之人都知晓的,并无任何特别之处,但皇帝却特别喜欢。这些事都很隐秘,不为世人所知。

\begin{yuanwen}
其后三年,有司言元宜以天瑞命,不宜以一二数。一元曰建元,二元以长星曰元光,三元以郊得一角兽曰元狩云。
\end{yuanwen}

三年后,相关官员说年号应该根据上天所赐的吉祥征兆而命名,不应该用一年、二年的顺序来制定。第一个纪元年号应该为建元,第二个纪元因为有种名叫长星的彗星出现,所以应该为元光,第三个纪元因为在郊外祭祀时捕获了独角兽,所以应为元狩。

\begin{yuanwen}
其明年冬,天子郊雍,议曰:“今上帝朕亲郊,而后土毋祀,则礼不答\footnote{合。}也。”

有司与太史公\footnote{指司马迁之父司马谈。}、祠官宽舒等议:“天地牲角茧栗。今陛下亲祀后土,后土宜于泽中圜丘为五坛,坛一黄犊太牢具,已祠尽瘗,而从祠衣上黄。”

于是天子遂东,始立后土祠汾阴脽上,如宽舒等议。上亲望拜,如上帝礼。礼毕,天子遂至荥阳而还。过雒阳,下诏曰:“三代邈绝\footnote{久远。},远矣难存。其以三十里地封周后为周子南君,以奉先王祀焉。”

是岁,天子始巡郡县,侵寻于泰山矣。
\end{yuanwen}

第二年冬季,皇帝在雍县郊祀时与人们商议说:“现在天帝由我亲自祭拜,但还未祭拜地神后土,这样与礼数不合。”

相关官员就和太史令司马谈、祠官宽舒等人商量说:“祭拜天地要用角像蚕茧或板栗大小的小牛。如今陛下想要亲身祭祀后土,祭祀后土应当在大泽里的圆丘上建造五个祭坛,每个祭坛都要用一头黄牛犊加一猪一羊做为供品,祭祀结束,牲品全部都要埋掉,而且陪从祭祀之人也要身穿黄色衣服。”

于是皇帝就向东行,第一次在汾阴丘上修建了一座后土的神祠,是按照宽舒等人的建议修建的。皇帝亲自望空跪拜地神,与祭祀天帝用的礼仪一样。祭礼完毕后,皇帝便去了荥阳,然后回到长安。途中路过雒阳,颁布诏书说:“夏、商、周三个朝代距离现在已经很远了,太过遥远的后代也难以存留,应该划出三十里的地方赐给周王的后代,封其为周子南君,以便他们祭祀祖先。”

这一年,天子开始巡游各郡县,最后渐渐行至泰山。

\begin{yuanwen}
其春,乐成侯上书言栾大。栾大,胶东宫人,故尝与文成将军同师,已而为胶东王尚方。而乐成侯\footnote{丁义,刘邦功臣丁礼曾孙。}姊为康王后,毋子。康王死,他姬子立为王。而康后有淫行,与王不相中,相危以法。康后闻文成已死,而欲自媚于上,乃遣栾大因乐成侯求见言方。天子既诛文成,后悔恨其早死,惜其方不尽,及见栾大,大悦。大为人长美,言多方略,而敢为大言,处之不疑。大言曰:“臣尝往来海中,见安期、羡门之属。顾以为臣贱,不信臣。又以为康王诸侯耳,不足予方。臣数言康王,康王又不用臣。臣之师曰:‘黄金可成,而河决可塞,不死之药可得,仙人可致也。’臣恐效文成,则方士皆掩口,恶敢言方哉!”

上曰:“文成食马肝死\footnote{汉时传言食马肝可使人致死。}耳。子诚能脩其方,我何爱乎!”

大曰:“臣师非有求人,人者求之。陛下必欲致之,则贵其使者,令有亲属,以客礼待之,勿卑,使各佩其信印,乃可使通言于神人。神人尚肯邪不邪。致尊其使,然后可致也。”

于是上使先验小方,斗旗,旗自相触击。
\end{yuanwen}

这一年春季,乐成侯上书举荐栾大。栾大是胶东王刘寄宫里掌管日常生活事务的宫人,之前曾和文成将军在同一个师傅那里学艺,后来他就成了胶东王管理配制药品的官员。当时乐成侯的姐姐就是胶东康王的王后,没有儿子。康王去世,别的姬妾的儿子被立为嗣王。而康后有过淫乱的行为,与新王不是很合得来,两人互相想尽办法加害对方。康后听说文成将军已经死去,就想要讨好皇帝,于是派栾大借着乐成侯的推荐觐见皇帝谈论方术。武帝虽然杀了文成将军,也后悔这么早就让他死了,叹惜自己没有让他献出全部方术。等到见了栾大,皇帝十分高兴。栾大这个人身材高大,相貌英俊,说话很讲究策略,而且敢于说大话,处事非常果断。他夸口说:“我曾经在海中来来往往,遇到过安期生、羡门高这些仙人。但他们都以为我地位很低,并不相信我。又认为康王仅仅是个诸侯而已,不配给他那些神仙方术。我多次向康王进言,康王却并不任用我。我的老师说过:‘黄金能够炼成,黄河决口也能堵住,长生不死的药物也可求得,神仙也能招来。’我现在唯恐会如文成一般遭临杀身之祸,那么所有的方士都会闭口不言了,不敢再提到方术了!”

皇帝说:“文成是不小心吃了马肝而死的。您如果真的能得到神仙的方术,我又有什么是舍不得的呢!”

栾大说:“我的老师从来都不会求人,都是别人去求他。陛下如果一定要招来我的老师,那就要先让我地位尊贵,让我有自己的家室,用款待宾客的礼仪善待我,不可鄙视,让我佩戴各种印信,这样我才能传达神仙的话。但神仙究竟会不会来却并不一定。总之,必须要让您的使者尊贵起来,这样之后才可能请来神仙。”

于是皇帝让他表演个小方术,检验他的效果,他表演的是斗棋,让棋子在棋盘上自行相互撞击。

\begin{yuanwen}
是时上方忧河决,而黄金不就,乃拜大为五利将军。居月馀,得四金印,佩天士将军、地土将军、大通将军、天道将军印。制诏御史:“昔禹疏九江,决四渎\footnote{指黄河、长江、淮水和济水。}。间者河溢皋陆,(堤/隄)繇不息。朕临天下二十有八年,天若遗朕士而大通焉。《乾》称‘蜚龙’,‘鸿渐于般’,意庶几与焉。其以二千户封地士将军大为乐通侯。”

赐列侯甲第\footnote{甲等府第。},僮千人。乘舆斥车马帷帐器物以充其家。又以卫长公主\footnote{皇后卫子夫所生的长女。}妻之,赍金万斤,更名其邑曰当利公主。天子亲如五利之第。使者存问所给,连属于道。自大主\footnote{名嫖,窦太后之女,武帝的姑姑。}将相以下,皆置酒其家,献遗之。于是天子又刻玉印曰“天道将军”,使使衣羽衣,夜立白茅上,五利将军亦衣羽衣,立白茅上受印,以示弗臣也。而佩“天道”者,且为天子道天神也。于是五利常夜祠其家,欲以下神。神未至而百鬼集矣,然颇能使之。其后治装行,东入海,求其师云。大见数月,佩六印,贵振天下,而海上燕齐之间,莫不搤捥而自言有禁方\footnote{秘方。},能神仙矣。
\end{yuanwen}

当时皇帝正在为黄河决口的事情忧心,而且黄金也没有炼成,就封栾大为五利将军。才一个多月时间,栾大已经得到了四枚金印,身上佩有天士将军、地士将军、大通将军和天道将军等印。皇帝给御史下诏说:“从前大禹疏导九江,开通长江、黄河、淮河、济水四渎。最近,河水溢出,淹没了陆地,筑堤的劳役接二连三。我统治天下已有二十八年了,上天倘若要送给我方士,那应该就是大通将军了。《乾卦》称之为“飞龙”,《渐卦》上也提到“鸿雁”,这应该是对我们君臣相得益彰的称许吧!应该用二千户的地方封地士将军栾大为乐通侯。”

赐给栾大甲等的宅第及奴仆千人,并把皇帝用不到的车马和帷帐等器物赐给栾大,摆满了他的新居。并把卫长公主嫁给他做妻子,又给卫长公主万斤黄金做陪嫁,将她的封号改为当利公主。皇帝还亲自来到五利将军的府第,使者们也都前往慰问,他们赠送的物品充盈道路。自皇帝的姑姑大长公主一直到将相之下的人,都到栾大家中摆设酒宴,并献赠礼物给他。于是皇帝还刻了一枚印,上写“天道将军”,并让使者拿着玉印,身披鸟羽制作的衣服,在夜里站在白茅草上,五利将军同样身穿鸟羽制作的衣服,站在白茅草上领受玉印,以表示皇帝并非把受印者当作自己的臣下。并且佩戴着“天道”之印,这是要替天子引导天神降临。由此五利常常趁夜在家中祭祀,想要通过这种方法求神仙降临。但神仙并没有来,种种恶鬼却都在这里聚集起来,但五利将军很会驱使这些鬼。不久他就整好行装,向东前往海上,相传是去找寻他的老师。栾大仅仅被引见几个月,就佩了六枚大印,尊贵足以震惊天下,燕、齐沿海地区的那些方士们见到他都兴奋地握住他的手腕,言说自己拥有秘方,能够招来神仙。

\begin{yuanwen}
其夏六月中,汾阴巫锦为民祠魏脽后土营旁,见地如钩状,掊视\footnote{用手扒开观看。}得鼎。鼎大异于众鼎,文镂毋款识,怪之,言吏。吏告河东太守胜,胜以闻。天子使使验问巫锦得鼎无奸诈,乃以礼祠,迎鼎至甘泉,从行,上荐之。至中山,晏温,有黄云盖焉。有麃过,上自射之,因以祭云。至长安,公卿大夫皆议请尊宝鼎。天子曰:“间者河溢,岁数不登,故巡祭后土,祈为百姓育穀。今年丰庑未有报,鼎曷为出哉?”

有司皆曰:“闻昔大帝兴神鼎一,一者一统,天地万物所系终也。黄帝作宝鼎三,象天地人也。禹收九牧之金,铸九鼎,皆尝鬺\footnote{煮。}烹上帝鬼神。遭圣则兴,迁于夏商。周德衰,宋之社亡,鼎乃沦伏而不见。《颂》云‘自堂徂基,自羊徂牛;鼐\footnote{nài}鼎及鼒,不虞不骜,胡考之休’。今鼎至甘泉,光润龙变,承休无疆。合兹中山,有黄白云降盖,若兽为符,路弓乘矢,集获坛下,报祠大飨。惟受命而帝者心知其意而合德焉。鼎宜见于祖祢,藏于帝廷,以合明应。”

制曰:“可。”
\end{yuanwen}

这一年夏季的六月里,汾阴有一个叫锦的巫师为民众在魏脽后土祠旁祭祀,见一处地面形状像弯钩一样,就用手扒开土看,结果得到了一只鼎。这只鼎和普通的鼎有很大不同,上面雕刻着花纹却并无文字。巫师感到奇怪,就把这件事报告给了当地官吏。当地的官吏又把这件事报告给了河东太守胜,胜上报朝廷。皇帝派使者来查验巫师锦得鼎的经过,确认并无诈伪,这才按照礼仪进行祭祀,将鼎请回了甘泉宫,皇帝和鼎同行,准备将它呈献给天帝。行至中山时,发现天空中云气缭绕,有黄云在上覆盖。这时一头麃子跑了过去,皇帝亲自用箭把它射死,并用它来祭云。到了长安,公卿大夫们都认为应该尊奉宝鼎。皇帝说:“最近黄河泛滥,连年收成不好,因此我才出巡前往各郡县祭祀后土,祈求百姓能够得到丰收。今年五谷收成丰茂,却并未举行祭祀酬拜地神,这鼎怎么会出现呢?”

相关的官员就说:“相传很久以前太帝太昊伏羲氏曾经制造了一只神鼎,以示天下统一,让天地万物都归终在神鼎里。黄帝一共造了三只宝鼎,寓意是天、地、人。夏禹收集九州的铜,铸造了九只宝鼎,都曾被用来烹煮祭祀上帝及鬼神的牲畜。每逢遇到圣主鼎就会出现,宝鼎从夏朝传到商朝。周朝末年德行败坏,宋国用来祭祀土神的社坛被摧毁,结果鼎就沦没隐伏不见了。《周颂》中说:‘自堂到阶来回地走,有羊有牛准备齐全,大鼎小鼎检查完毕,喧哗不要骄傲不要,健康长寿多福多禄。’如今鼎已经被迎到了甘泉宫,它的外表光彩无比,变化神奇灵动,这预示着大汉必将得到无尽的吉祥。这正合行到中山时,黄白祥云在上覆盖,麃子吉兽在下逢迎的祥瑞征兆,还有您获得的神坛下的大弓和四箭,这都是您在太庙中祭祀远近先祖神主所得到的回报啊。只有那些承受天命做皇帝的人才能通晓上天的旨意且合于天德。因此宝鼎应当进献给祖先,保存在天帝的宫廷,这才与之前的种种吉兆相合。”

皇帝下诏说:“可以。”

\begin{yuanwen}
入海求蓬莱者,言蓬莱不远,而不能至者,殆\footnote{可能,表推测。}不见其气。上乃遣望气佐侯\footnote{协助观测。}其气云。
\end{yuanwen}

前往海上找寻蓬莱仙山的人们,都说蓬莱山不是很远,而未能到那里的人,可能是因为他们无法看到仙山的云气。于是皇帝派擅长望气的官员协助观测云气。

\begin{yuanwen}
其秋,上幸雍,且郊。或曰:“五帝,泰一之佐也。宜立泰一而上亲郊之”。

上疑未定。齐人公孙卿\footnote{齐地方士。}曰:“今年得宝鼎,其冬辛巳朔旦冬至,与黄帝时等。”

卿有札书曰:“黄帝得宝鼎宛朐,问于鬼臾区。区对曰:‘帝得宝鼎神筴,是岁己酉朔旦冬至,得天之纪,终而复始。’于是黄帝迎日推筴,后率二十岁得朔旦冬至,凡二十推,三百八十年。黄帝仙登于天。”

卿因所忠\footnote{武帝的侍臣。}欲奏之。所忠视其书不经,疑其妄书,谢曰:“宝鼎事已决矣,尚何以为!”

卿因嬖人奏之。上大说,召问卿。对曰:“受此书申功,申功已死。”

上曰:“申功何人也?”

卿曰:“申功,齐人也。与安期生通,受黄帝言,无书,独有此鼎书。曰‘汉兴复当黄帝之时。汉之圣者在高祖之孙且曾孙也。宝鼎出而与神通,封禅。封禅七十二王,唯黄帝得上泰山封’。申功曰:‘汉主亦当上封,上封则能仙登天矣。黄帝时万诸侯,而神灵之封居七千。天下名山八,而三在蛮夷,五在中国。中国华山、首山、太室、泰山、东莱,此五山黄帝之所常游,与神会。黄帝且战且学仙。患百姓非其道,乃断斩非鬼神者。百馀岁然后得与神通。黄帝郊雍上帝,宿三月。鬼臾区号大鸿,死葬雍,故鸿冢是也。其后(于)黄帝接万灵明廷。明廷者,甘泉也。所谓寒门者,谷口也。黄帝采首山铜,铸鼎(于)荆山下。鼎既成,有龙垂胡(髯/珣)下迎黄帝。黄帝上骑,群臣后宫从上龙七十馀人,(龙)乃上去。馀小臣不得上,乃悉持龙(髯/珣),龙(髯/珣)拔,堕黄帝之弓。百姓仰望黄帝既上天,乃抱其弓与龙胡(髯/珣)号。故后世因名其处曰鼎湖,其弓曰乌号。’”

于是天子曰:“嗟乎!吾诚得如黄帝,吾视去妻子如脱(鵕/躧\footnote{通“屣”,鞋。})耳。”乃拜卿为郎,东使候神于太室。
\end{yuanwen}

这一年秋季,皇上来到雍县,准备举行郊祀。有人说:“五帝不过是泰一神的辅佐神而已,应该建造泰一神坛,并且由皇上亲自进行郊祀。”

皇帝犹豫着未做决定。齐地的公孙卿说:“今年获得宝鼎,今年冬天的辛巳日正好是朔日初一,当天早晨恰逢冬至,这与黄帝得到宝鼎的时间一致。”

公孙卿藏有的一本木简书上说:“黄帝在宛朐县得到宝鼎,并询问鬼臾区。鬼臾区回答说:‘黄帝您获得了宝鼎和占卜所用的神策,就在己酉年的朔日冬至,这与天道历数很符合,天道历数总是周而复始、不断循环的。’由此黄帝靠观察太阳的运行来演算历法,从那后大概每隔二十年就会出现一个朔日早晨交冬至,这样他一共做了二十次推算,共计三百八十年,黄帝成仙升天。”

公孙卿想要通过所忠将这件事上奏皇上。所忠见他的书荒诞不经,怀疑这是本伪造的书,就推辞说:“宝鼎的事已经有了定论,还上奏能怎么样!”

公孙卿又借助皇帝宠信的人上奏此事。皇帝十分高兴,就召公孙卿来询问。公孙卿就回答说:“是从申功那里得到的这本书,但他已死。”

皇帝就问道:“申功是谁?”

公孙卿说:“申功是齐地的人。他和安期生交往过,并接受了黄帝的教诲,并未留下别的什么书,只有这部和鼎相关的书。书中提到:‘汉代的兴盛之期应当与黄帝时一样。汉代的圣君,将会出现在高祖皇帝的孙子或者曾孙中。宝鼎出现就一定能与神仙相通,应当进行封禅。从古至今,有七十二位君王举行过封禅,只有黄帝可以登泰山封禅。’申功说:‘汉代的皇帝也应当登泰山封禅,登上泰山封禅后就能成仙升天。黄帝时期有近万个诸侯国,而为了祭祀神灵而建的诸侯国就有七千。全天下的名山一共有八座,其中的三座位于蛮夷地区,五座位于中原地区。中原地区有华山、首山、太室山、泰山和东莱山,这五座山都是黄帝经常出游之处,他在那里与神仙相会。黄帝时而作战时而修习仙道。他担忧百姓非议他所修习的仙道,就果断杀掉了那些诽谤鬼神的人。这样一直持续了一百多年,才和神仙相通。黄帝曾经在雍县郊祀天帝,住了三个月。鬼臾区的绰号叫大鸿,他死后就被葬在了雍县,也就是鸿冢。从那以后黄帝在明廷接待众神。明廷就是如今的甘泉山。而所说的寒门就是如今的谷口。黄帝发掘了首山的铜矿,并在荆山下铸鼎。鼎铸成后,有一条脖子下面垂着两腮长须的龙从天而下迎接黄帝。黄帝骑到了龙背上,群臣和后宫嫔妃一起跟着骑上龙背的有七十多人,龙这才飞着离开。其余小臣无法上去,都紧紧抓着龙须,龙须都被拉断了,掉下了黄帝的弓。百姓们仰着头看着黄帝飞升上天,就都怀抱他的弓和掉下的龙须哭喊不已。因此后世就把这个地方称为鼎湖,称那张弓为乌号。’”

于是皇帝说:“唉!要是我真的能和黄帝一般,我就会把离开妻儿看作像脱掉鞋子一样了。”就任命公孙卿为郎官,派他向东前往太室山迎候神仙。

\begin{yuanwen}
上遂郊雍,至陇西,西登空桐,幸甘泉。令祠官宽舒等具泰一祠坛,坛放\footnote{仿照。}薄忌泰一坛,坛三垓\footnote{通“陔”,层。}。五帝坛环居其下,各如其方,黄帝西南,除八通鬼道。泰一所用,如雍一畤物,而加醴枣脯之属,杀一犛牛以为俎豆牢具。而五帝独有俎豆醴进。其下四方地,为餟食群神从者及北斗云。已祠,胙馀皆燎之。其牛色白,鹿居其中,彘在鹿中,水而洎之。祭日以牛,祭月以羊彘特。泰一祝宰则衣紫及绣。五帝各如其色,日赤,月白。
\end{yuanwen}

皇帝于是前往雍县郊祭,然后又到了陇西,向西登临崆峒山,最后回到了甘泉宫。命令祠官宽舒等人设置好泰一神的祭坛,祭坛仿照薄诱忌所说的泰一坛修建,共分三层。五帝的祭坛都环绕在泰一坛的下面,都各自按照其所在的方位排列,黄帝坛位居西南方,修建了八条供鬼神来去的通道。泰一坛所用的祭品,都和雍县的祭祀并无二致,而且添加了甜酒、枣果和干肉等物,还杀掉了一头犛牛当作祭器中的牲品。但五帝坛就只有牛羊等牲品及甜酒,并无犛牛。祭坛下的四面,用酒活地并祭祀随从的群神和北斗星。祭祀结束,用来祭祀的祭品都被烧掉。祭祀使用的牛是白色的,祭祀时要把一头鹿塞入牛的腹腔,再把一头猪塞入鹿的腹腔,然后把祭品浸泡在水中。祭日神要用一头牛,祭月神则要用一只羊或一头猪。祭祀泰一神的祝官要身穿紫色绣衣,祭祀五帝的祝官的礼服颜色都要严格遵照五帝所属的颜色,祭日神要身穿红衣,祭月神则身穿白衣。

\begin{yuanwen}
十一月辛巳朔旦冬至,昧爽,天子始郊拜泰一。朝朝日,夕夕月,则揖;而见泰一如雍礼。其赞飨曰:“天始以宝鼎神筴授皇帝,朔而又朔,终而复始,皇帝敬拜见焉。”

而衣上黄。其祠列火满坛,坛旁烹炊具。有司云:“祠上有光焉。”

公卿言:“皇帝始郊见泰一云阳,有司奉瑄玉嘉牲荐飨。是夜有美光,及昼,黄气上属天。”

太史公、祠官宽舒等曰:“神灵之休\footnote{美。},祐福兆祥,宜因此地光域立泰畤坛以明应。令太祝领,秋及腊间祠。三岁天子一郊见。”
\end{yuanwen}

十一月辛巳朔日早晨是冬至,这天拂晓时,皇帝就开始在郊外拜祭泰一神。早晨拜祭日神,傍晚拜祭月神,全都是拱着手十分严肃地祭拜;而祭祀泰一神则严格遵照雍县的郊祀礼仪。祭神祝颂中说:“上天已经开始把宝鼎和神策赠给了皇帝,一个朔旦又一个朔旦,循环往复,永无止息。皇帝就在这里恭敬地拜见天神。”

祭祀用的礼服为黄色。祭祀的时候坛上布满了火炬,坛旁则摆放着烹煮所用的器具。相关官员说:“祠坛之上有光现出。”

公卿大臣们都说:“皇帝最开始是在云阳宫举行郊祀,拜祭泰一神,相关官员都手捧直径六寸的大璧瑄玉和毛色纯美膘肥体瘦的祭祀牲畜献给神灵享用。那天夜里出现了美丽的光彩,等到了白天,有黄色的云气不断上升,直到与天相接。”

太史公和祠官宽舒等都说:“神灵灵光的照耀,是佑助福禄的吉兆,应当就在神光照射的地方修建泰畤坛,以此宣扬上天的瑞应。让太祝主持这件事,在每年的秋天和腊月间进行祭祀,天子每三年亲自郊祭一次。”

\begin{yuanwen}
其秋,为伐南越,告祷泰一,以牡荆画幡日月北斗登龙,以象天一三星,为泰一锋,名曰“灵旗”。为兵祷\footnote{战前祈祷。},则太史奉以指所伐国。而五利将军使不敢入海,之泰山祠。上使人微随验,实无所见。五利妄言见其师,其方尽,多不雠。上乃诛五利。
\end{yuanwen}

这一年秋季,为了讨伐南越而拜祭泰一神,用牡荆制成幡旗竿,旗上则画着日、月、北斗及腾空而起的龙,以此来象征天一三星,作为泰一神的先锋旗,命名为“灵旗”。战前祈祷时,则让太史官手举灵旗指着被伐国的方向。但是五利将军虽是使者却不敢入海求仙,只去了泰山祭拜。皇上派人暗中跟踪他,发现他其实上什么都没有见到。五利将军撒谎说他见到了自己的老师,实际上他的方术都已用尽,多半都没有应验。皇上于是杀掉了五利将军。

\begin{yuanwen}
其冬,公孙卿候神河南,见仙人迹缑氏城上,有物若雉,往来城上。天子亲幸缑氏城视迹。问卿:“得毋效文成、五利乎?”

卿曰:“仙者非有求人主,人主求之。其道非少宽假,神不来。言神事,事如迂诞,积以岁乃可致。”

于是郡国各除道\footnote{整饬道路。},缮治宫观名山神祠所,以望幸矣。
\end{yuanwen}

这一年冬季,公孙卿在河南恭候神仙,说在缑氏城上曾见过仙人的脚印,还有一只和山鸡相似的神物在城上来回飞着。皇帝亲自来到缑氏城观看脚印,问公孙卿说:“你应该不会效仿文成和五利欺骗我吧?”

公孙卿回答说:“仙人并不会有求于皇帝,但皇帝却有求于仙人。求仙之道,倘若不能把时间放宽一点,神仙并不会来。说起求神这样的事,似乎很是迂腐怪诞,但也只有积年累月才能等来神仙啊。”

因此各个郡国都整饬道路,整修宫殿观台及名山上的神祠,以盼望皇帝驾临。

\begin{yuanwen}
其年,既灭南越,上有嬖臣李延年\footnote{西汉音乐家,武帝宠妃李夫人之兄,贰师将军李广利之弟。}以好音见。上善之,下公卿议,曰:“民间祠尚有鼓舞之乐,今郊祠而无乐,岂称乎?”

公卿曰:“古者祀天地皆有乐,而神祇可得而礼。”

或曰:“泰帝使素女鼓五十弦瑟,悲,帝禁不止,故破其瑟为二十五弦。”

于是塞南越,祷祠泰一、后土,始用乐舞,益召歌儿,作二十五弦及箜篌瑟自此起。
\end{yuanwen}

这一年,已经灭掉了南越,有个皇帝的宠臣李延年因为擅长音乐而得以觐见皇帝。皇帝对他很好,就让公卿大臣们商议这件事,说:“在民间祭祀时还伴有鼓、舞及音乐,现在我进行郊祀却并无音乐,这难道能说是相称吗?”

公卿们说:“古时祭祀天地时都伴有音乐,由此天神和地神才来享受人们的祭祀。”

有的人说:“泰帝令女神素女弹奏有五十根弦的瑟,声音悲切,泰帝让她停下,但她却无法自止,因此就剖开了她的瑟改成二十五弦。”

于是在为庆祝平定南越而酬祭泰一、后土神时,开始使用音乐和歌舞,增加了歌手的人数,制作二十五弦的瑟及箜篌就是从这时开始的。

\begin{yuanwen}
其来年冬,上议曰:“古者先振兵泽旅,然后封禅。”

乃遂北巡朔方,勒兵十馀万,还祭黄帝冢桥山,泽兵须如。上曰:“吾闻黄帝不死,今有冢,何也?”

或对曰:“黄帝已仙上天,群臣葬其衣冠。”

即至甘泉,为且用事泰山,先类祠泰一。
\end{yuanwen}

第二年冬季,皇帝提议说:“古时的帝王都要先停止用武,其后才能举行封禅大典。”

于是皇帝就向北巡视朔方,统率军队十余万,回来的时候就在桥山的黄帝陵前祭祀,到须如时解散了军队。皇帝就说:“我曾听到过黄帝并没死的消息,可是而今却有他的陵墓,这是为什么?”

有人回答说:“黄帝在成仙上天之后,余下的众臣就将他的衣服和帽子葬在了这里。”

后来皇帝来到甘泉宫,为了要登泰山进行封禅,就用同样的礼仪先祭祀泰一神。

\begin{yuanwen}
自得宝鼎,上与公卿诸生议封禅。封禅用希旷绝\footnote{时隔久远,荒废绝灭。},莫知其仪礼,而群儒采封禅《尚书》、《周官》、《王制》之望祀射牛事。

齐人丁公年九十馀,曰:“封者,合不死之名也。秦皇帝不得上封。陛下必欲上,稍上即无风雨,遂上封矣。”

上于是乃令诸儒习射牛,草封禅仪。数年,至且行。天子既闻公孙卿及方士之言,黄帝以上封禅,皆致怪物与神通,欲放黄帝以尝接神仙人蓬莱士,高世比德于九皇\footnote{或为方士妄言,无从考据。},而颇采儒术以文之。群儒既以不能辩明封禅事,又牵拘于《诗》、《书》古文而不敢骋。上为封祠器示群儒,群儒或曰“不与古同”,徐偃\footnote{西汉经学大师申培公的弟子,此时为博士。}又曰“太常诸生行礼不如鲁善”,周霸属图封事,于是上绌偃、霸,尽罢诸儒弗用。
\end{yuanwen}

自从获得宝鼎,皇帝就和公卿大臣及儒生们商议封禅之事。因为封禅大典很长时间没有举行了,几近失传,没有人知道具体的礼仪。儒生们都提议采用《尚书》、《周礼》、《王制》中所记载的天子射牛、望祀等仪式来举行大典。

齐人丁公已经有九十多岁了,他说:“‘封’应该就是长生不死的意思。秦始皇没有能够登泰山举行封禅之礼。陛下一定要上去,只要登到稍高一些的地方就不再有风雨阻挡了,于是就可登上泰山举行封禅之礼了。”

皇帝因此就让儒生们修习射牛之礼,并草拟了封禅所用的礼仪。过了几年,封禅典礼就要举行了。皇帝已经听到了公孙卿与方士的话,说是在黄帝之前的帝王进行封禅之礼时,都会招来一些能与神仙相通的怪异之物,就打算仿照黄帝那时迎接仙人蓬莱士之法,通过这样来超乎世俗,和九皇的德行相媲美,而且还采用很多儒术来修饰。众儒生既无法明辨封禅的具体礼数,又深受《诗》、《书》等古文典籍的束缚,不敢全力施展才华。皇帝将封禅所用的祭器带给儒生们看,众儒生中有人说“这和古代的并不相同”,徐偃也说“太常祠官们所行的礼仪并没有古代鲁国的好”,周霸聚集了群儒以筹划这次的封禅事宜,因此皇帝贬退了徐偃和周霸,罢黜这些儒生一概不用。

\begin{yuanwen}
三月,遂东幸缑氏,礼登中岳太室。从官在山下闻若有言“万岁”云。问上,上不言;问下,下不言。于是以三百户封太室奉祠,命曰崇高邑。东上泰山,山之草木叶未生,乃令人上石立之泰山颠。
\end{yuanwen}

三月,皇帝向东来到缑氏县,登上了嵩山的太室山举行祭祀。随行的官员在山脚下似乎听到有人在喊“万岁”。问山上的人,山上的人都说没有喊过;问山下的人,山下的人也都说并未喊过。因此皇帝就封给太室山三百户以方便其祭祀,并为其命名为崇高邑。向东登上泰山,山中的草木还未长出叶子,就命人将石碑运送上山,立在泰山的顶峰。

\begin{yuanwen}
上遂东巡海上,行礼祠八神。齐人之上疏言神怪奇方者以万数,然无验者。乃益发船,令言海中神山者数千人求蓬莱神人。公孙卿持节常先行候名山,至东莱,言夜见一人,长数丈,就之则不见,见其迹甚大,类禽兽云。群臣有言见一老父牵狗,言“吾欲见巨公\footnote{指皇帝。}”,已忽不见。上既见大迹,未信,及群臣有言老父,则大以为仙人也。宿留海上,与方士传车及间使\footnote{秘使。}求仙人以千数。
\end{yuanwen}

皇帝接着向东巡视海上,举行典礼以祭祀天主、地主、兵主、阴主、阳主、月主、日主和四时主八神。齐人上奏讲说神仙精灵及奇异方术有近万人之多,但都不灵验。因此皇帝就增派船只,命令那些上奏讲说海上神山的几千个人都去求访蓬莱仙人。公孙卿手持着符节,先前往各山等候神仙,到了东莱,就说他在夜里曾看到一个人,身高数丈,等他靠近后就不见了。看到他的脚印非常大,很像禽兽的脚印。大臣中有人说曾看到一位牵着狗的老人,说“我很想见天子”,一转眼又不见了。皇帝已经听说了有人看到大脚印,但并未相信,直到群臣里有人提到老人的事,才真的认为那个老人就是神仙。于是他就留住在海边,并给方士驿车,派了几千名秘使前去求访仙人。

\begin{yuanwen}
四月,还至奉高。上念诸儒及方士言封禅人人殊,不经,难施行。天子至梁父,礼祠地主。

乙卯,令侍中儒者皮弁荐绅,射牛行事。封泰山下东方,如郊祠泰一之礼。封\footnote{祭坛。}广丈二尺,高九尺,其下则有玉牒书,书祕。礼毕,天子独与侍中奉车子侯上泰山,亦有封。其事皆禁。

明日,下阴道。

丙辰,禅泰山下阯东北肃然山,如祭后土礼。天子皆亲拜见,衣上黄而尽用乐焉。江淮间一茅三脊为神藉。五色土益杂封。纵远方奇兽蜚禽及白雉诸物,颇以加祠。兕旄牛犀象之属弗用。皆至泰山然后去。封禅祠,其夜若有光,昼有白云起封中。
\end{yuanwen}

四月,皇帝回到奉高。皇帝考虑到儒生及方士们所提到的封禅礼仪并不一样,古书上也并无记载,确实很难施行。天子又前往梁父山,举行典礼祭祀地神。

乙卯日,命令侍中官儒生戴着白鹿皮帽,穿上插笏官服,举行射牛仪式。在泰山脚下的东方筑坛祭祀,并严格遵照祭拜泰一神的礼仪进行。那个祭天用的坛宽为一丈二,高九尺,坛下则放置着封禅的文书,文书中的内容很是隐秘。祭礼结束,皇帝独自与侍中奉车都尉霍子侯登临泰山,也举行了一番祭天仪式。这些事情都是绝密的,不得外露。

第二天,沿着山北坡的道路下山。

丙辰日,又在泰山脚下东北方的肃然山举行了祭地仪式,参照的是祭祀后土的礼仪。这些封禅,天子都亲身祭拜天神、地神,身上穿着黄色的礼服而且全都使用了音乐。使用从江淮一带采来的三棱灵茅作为神垫,用象征五方的五色泥土杂糅起来筑成祭坛。当时还放出了一些远方奇异的飞禽走兽及白毛野鸡之等的动物,很是增加了礼仪的庄重气氛。但那些兕牛、旄牛、犀牛、大象这类的动物都没有用到。皇帝和随从的大臣们都是先来到泰山,之后再离开的。在举行封禅的这段时间,每天夜里都似乎有亮光现出,白天时有白云从祭坛中升起。

\begin{yuanwen}
天子从封禅还,坐明堂,群臣更上寿。于是制诏御史:“朕以眇眇之身承至尊,兢兢焉惧弗任。维德菲薄,不明于礼乐。脩祀泰一,若有象景光,箓(屑)如有望,依依震于怪物,欲止不敢,遂登封泰山,至于梁父,而后禅肃然。自新,嘉与士大夫更始,赐民百户牛一酒十石,加年八十孤寡布帛二匹。复\footnote{免除赋税。}博、奉高、蛇丘、历城,毋出今年租税。其赦天下,如乙卯赦令。行所过毋有复作。事在二年前,皆勿听治\footnote{盘查审理。}。”

又下诏曰:“古者天子五载一巡狩,用事泰山,诸侯有朝宿地。其令诸侯各治邸泰山下。”
\end{yuanwen}

皇帝从封禅之处归来,坐在明堂之上,臣子们都陆续上来祝贺。于是皇帝给御史颁诏说:“我以自己的渺小之身承担了皇帝的至尊之位,小心谨慎深恐无法胜任。我的德行浅薄,也不太懂礼乐。在祭祀泰一神之时,天空似乎出现祥瑞之光,我心里十分不安,好像看到了什么,心里也因这怪异的景象而被深深触动,虽然想停止却不敢,之后终于得以登泰山祭天。等到了梁父山,之后在肃然山修整场所祭地。我想要完善自己,很高兴和士大夫们一同重新开始。赐给平民每百户人家一头牛、十石酒,另外还给八十岁以上的老人和孤寡老人再加布帛二匹。博县、奉高、蛇丘和历城这些地区都不用缴纳当年的租税。大赦天下,与乙卯年的赦免相同。我在巡行时所经过的地方,都不能再加重百姓的麻烦。倘若犯罪发生在两年前,都不再追究。”

接着又下诏说:“古时候天子每隔五年就会出巡一次,前往泰山进行祭祀,来朝拜的诸侯们都有各自的住所。应让各诸侯都在泰山脚下修筑官邸。”

\begin{yuanwen}
天子既已封禅泰山,无风雨菑,而方士更言蓬莱诸神山若将可得,于是上欣然庶几\footnote{也许。}遇之,乃复东至海上望,冀遇蓬莱焉。奉车子侯暴病,一日死。上乃遂去,并海上,北至碣石,巡自辽西,历北边至九原。

五月,返至甘泉。有司言宝鼎出为元鼎,以今年为元封元年。
\end{yuanwen}

天子在泰山的封禅典礼结束后,并没有遇到风雨干扰,方士们也说蓬莱这些神山似乎也将找到了。于是皇帝非常高兴,认为自己也许会遇到神仙,就再次东行来到海边眺望,希望遇到蓬莱仙人。奉车都尉霍子侯忽然生了重病,一天时间就死掉了。皇帝这才离开,沿着海岸,向北一直到达碣石,之后从辽西开始巡游,经过北方边境一直抵达九原县。

五月,皇帝回到了甘泉宫。相关官员就提议将宝鼎被发现的那一年的年号定为“元鼎”,由于今年的泰山封禅大典,年号就定为“元封”,以今年为元封元年。

\begin{yuanwen}
其秋,有星茀于东井。后十馀日,有星茀于三能。望气王朔言:“候独见其星出如瓠,食顷复入焉。”

有司言曰:“陛下建汉家封禅,天其报德星云(嘒)”
\end{yuanwen}

这一年秋季,有彗星在东井宿天区出现,光芒璀璨。过了十几天,又有彗星在三台宿天区出现,光芒依然璀璨。有个能望气的叫王朔的人说:“我当时正在观测,就见那颗星出现时它的形状就如同葫芦瓜,一顿饭的工夫就又不见了。”

相关官员说:“陛下创制了汉家的封禅礼制,上天就让代表吉祥的德星现出以报答您。”

\begin{yuanwen}
其来年冬,郊雍五帝,还,拜祝祠泰一。赞飨曰:“德星昭衍\footnote{光明广远。},厥维休祥。寿星仍出,渊耀光明。信星昭见,皇帝敬拜泰祝之飨。”
\end{yuanwen}

第二年冬季,皇帝前往雍县祭祀五帝,返回后又祭拜了泰一神。祝辞中说:“德星光明照四方,预示美好吉祥。寿星也出现,光辉远播四方。信星闪烁降福,皇帝将祭品敬献给各位神灵。”

\begin{yuanwen}
其春,公孙卿言见神人东莱山,若云“见天子”。天子于是幸缑氏城,拜卿为中大夫。遂至东莱,宿留之数日,毋所见,见大人迹。复遣方士求神怪采芝药以千数。是岁旱。于是天子既出毋名,乃祷万里沙,过祠泰山。还至瓠子,自临塞决河,留二日,(沈/沉)祠\footnote{将祭品沉入水中,祭祀河神。}而去。使二卿将卒塞决河,河徙二渠,复禹之故迹焉。
\end{yuanwen}

这一年春季,公孙卿说他曾在东莱山遇到了仙人,那仙人似乎在说“要见天子”。皇帝因此来到缑氏城,任命公孙卿为中大夫。接着前往东莱,又在那里留住了几天,但什么都没看见,只发现了大脚印。皇帝再次派出几千方士前去寻访神仙奇物,采寻灵芝仙药。这一年天时干旱。于是皇帝便再无出巡的合理理由,就去万里沙祈雨,经过泰山时进行了祭祀。返回时到瓠子口,亲自前往堵塞黄河决口的地方,在这里留了两天,并沉下白马以祭拜河神,然后才离开。皇帝命令两位大臣带兵堵住决口,为黄河疏凿了两条河渠,让它再现了当年大禹治水后的样子。

\begin{yuanwen}
是时既灭南越,越人勇之乃言“越人俗信鬼,而其祠皆见鬼,数有效。昔东瓯王敬鬼,寿至百六十岁。后世谩怠,故衰秏”。

乃令越巫立越祝祠,安台无坛\footnote{指只有祭台,并无祭坛。},亦祠天神上帝百鬼,而以鸡卜。上信之,越祠鸡卜始用焉。
\end{yuanwen}

当时已经消灭了南越,越人勇之对皇上说“越人有信鬼的习俗,而且每次祭祀都能见到鬼,总是很灵验。过去东瓯王礼敬鬼神,活到一百六十岁的高寿。后代的子孙怠慢鬼,因此渐渐衰微。”

皇帝因此命令越地的巫师修建越祠,只设置祭台却没有建立祭坛,用以拜祭天神上帝百鬼,并且用鸡骨占卜。因为皇帝相信,越祠和鸡卜的方法从那时开始逐渐得以流行。

\begin{yuanwen}
公孙卿曰:“仙人可见,而上往常遽\footnote{骤,匆忙。},以故不见。今陛下可为观,如缑氏城,置脯枣,神人宜可致。且仙人好楼居。”

于是上令长安则作蜚廉桂观,甘泉则作益延寿观,使卿持节设具而候神人,乃作通天台,置祠具其下,将招来神仙之属。于是甘泉更置前殿,始广诸宫室。

夏,有芝生殿防内中。天子为塞河,兴通天台,若有光云,乃下诏曰:“甘泉防生芝九茎,赦天下,毋有复作。”
\end{yuanwen}

公孙卿说:“仙人是可以见到的,但皇上每次去求访仙人时总是很匆忙,所以才无法见到仙人。如今陛下可以筑造一座台阁,像缑氏城那样,置办干肉枣果一类的祭品,就应该能招来神仙。况且仙人都喜爱住楼阁。”

于是皇帝就让人在长安修建蜚廉观和桂观,在甘泉宫筑造益延寿观,让公孙卿手拿符节摆设好祭品,迎接仙人。接着又修建了通天台,在台下摆设祭祀用的各种礼器,期望能够招来仙人。由此又在甘泉宫修建了前殿,并开始扩建宫室。

夏季,有灵芝草生长在宫殿中。皇帝由于黄河决口被堵住,修建了通天台,似乎看到了亮光,于是下诏说:“甘泉宫殿房内生出了九茎灵芝,大赦天下,再不役兵扰民。”

\begin{yuanwen}
其明年,伐朝鲜。

夏,旱。公孙卿曰:“黄帝时封则天旱,乾封三年。”

上乃下诏曰:“天旱,意乾封乎?其令天下尊祠灵星焉。”
\end{yuanwen}

过了一年,征伐朝鲜。

夏季,干旱。公孙卿说:“黄帝的时候举行了封禅大典就会干旱,为了封坛的土尽快晾干,需要大旱三年。”

于是皇上就下诏说:“天时干旱,是为了晒干封坛的土吧?命令全国都要祭祀掌管农业的灵星。”

\begin{yuanwen}
其明年,上郊雍,通回中道,巡之。春,至鸣泽\footnote{沼泽名,所在不详。},从西河归。
\end{yuanwen}

又过了一年,皇帝前往雍县郊祀,在通过回中的道路时去那里巡视。春季,来到鸣泽,然后经过西河返回。

\begin{yuanwen}
其明年冬,上巡南郡,至江陵而东。登礼潜之天柱山,号曰南岳。浮江,自寻阳出枞阳,过彭蠡,祀其名山川。北至琅邪,并海上。四月中,至奉高脩封焉。
\end{yuanwen}

来年冬季,皇帝前往南郡巡察,到江陵后向东。登上了潜县的天柱山,称这座山为南岳。乘船沿江而下,从寻阳一直穿过枞阳,路过彭蠡,沿江祭祀经过的名山大川。向北一直到达琅邪郡,然后沿海北上。四月中旬,来到奉高县,举行了封禅大典。

\begin{yuanwen}
初,天子封泰山,泰山东北阯古时有明堂处,处险不敞。上欲治明堂奉高旁,未晓其制度。济南人公(玉/玊)带上黄帝时《明堂图》。《明堂图》中有一殿,四面无壁,以茅盖,通水,圜宫垣为(複/复)道,上有楼,从西南入,命曰昆仑,天子从之入,以拜祠上帝焉。于是上令奉高作明堂汶上,如带图。及五年脩封,则祠泰一、五帝于明堂上坐,令高皇帝祠坐对之。祠后土于下房,以二十太牢。天子从昆仑道入,始拜明堂如郊礼。礼毕,燎堂下。而上又上泰山,有祕祠\footnote{秘密祭祀。}其颠。而泰山下祠五帝,各如其方,黄帝并赤帝,而有司侍祠焉。泰山上举火,下悉应之。
\end{yuanwen}

起初,皇帝来到泰山举行封禅大典时,泰山脚的东北方向发现有古代的明堂,所在的地势险峻又不宽敞。皇帝打算在奉高县附近建造明堂,却无法得知明堂的形制及尺度。济南人公玊带敬献黄帝时期的《明堂图》。《明堂图》中绘制有一座殿堂,四周没有墙壁,使用茅草制成盖顶,四周有水沟相通,绕着宫墙的周围还建有天桥,殿上建有楼,自西南方向延伸进殿堂,被称作昆仑道,天子就从这里走进殿堂,去那里祭祀上帝。因此皇帝下令依照公玊带的图样在奉高的汶上修建明堂。等到五年后再来这里进行封禅时,就把泰一神和五帝的神位都放在明堂上座举行祭祀,并令高皇帝的灵位和他们相对。在下房拜祭后土神,使用了牛、羊、猪各二十头。皇帝自昆仑道进入,开始在明堂祭拜,一如郊祀的礼仪。祭拜结束,在堂下把祭品烧掉。之后,皇帝再次登上泰山,在山顶举行秘密祭祀。在泰山脚下拜祭五帝之时,都依照他们各自所在的方位,只有黄帝和赤帝是在一起的,这时有相关官员相陪。在泰山上点着火,山下也都点火相呼应。

\begin{yuanwen}
其后二岁,十一月甲子朔旦冬至,推历者以本统。天子亲至泰山,以十一月甲子朔旦冬至日祠上帝明堂,每脩封禅。其赞飨曰:“天增授皇帝泰元神(筴/策),周而复始。皇帝敬拜泰一。”

东至海上,考入海及方士求神者,莫验,然益遣,冀遇之。
\end{yuanwen}

又过了两年,十一月甲子朔日,早晨与冬至相交,那些推算历法的人认为应该以这一天作为新历法的起点才是正统。天子亲自前往泰山,就在十一月甲子朔日的早晨来到明堂拜祭上帝,但未举行封禅典礼。祝辞中说:“上天赐予皇帝泰元神策,周而复始。皇帝在此崇敬地祭拜泰一神。”

皇帝又东行来到海边,考察前往海上求仙的人及众方士,毫无结果,然而还是派更多的使者去海上求仙,渴望与神仙相遇。

\begin{yuanwen}
十一月乙酉,柏梁灾\footnote{火灾。}。

十二月甲午朔,上亲禅高里\footnote{泰山下的小山。},祠后土。临渤海,将以望祠蓬莱之属,冀至殊庭焉。
\end{yuanwen}

十一月乙酉日,柏梁台发生了火灾。

十二月甲午朔日,皇帝亲自前往高里山拜祭后土神。又来到渤海边,遥望并祭祀蓬莱那样的仙山,渴望能够前往仙人所在的异域。

\begin{yuanwen}
上还,以柏梁灾故,朝受计甘泉。公孙卿曰:“黄帝就青灵台,十二日烧,黄帝乃治明庭。明庭,甘泉也。”

方士多言古帝王有都甘泉者。其后天子又朝诸侯甘泉,甘泉作诸侯邸。勇之乃曰:“越俗有火(灾/烖),复起屋必以大,用胜服之。”

于是作建章宫,度为千门万户。前殿度高未央,其东则凤阙,高二十馀丈。其西则唐中\footnote{宫殿名。},数十里虎圈。其北治大池,渐台高二十馀丈,名曰泰液池,中有蓬莱、方丈、瀛洲、壶梁\footnote{海中仙山名。},象海中神山龟鱼之属。其南有玉堂、璧门、大鸟之属。乃立神明台、井幹楼,度五十馀丈,辇道相属焉。
\end{yuanwen}

皇帝回到京城后,因为柏梁台被烧毁,就在甘泉宫上朝接纳各郡国报上来的文簿。公孙卿说:“黄帝筑建青灵台,十二天后就被烧毁了,于是黄帝又修造了明庭。明庭就是现在的甘泉宫。”

方士中有很多人也说古时帝王也有建都在甘泉的。那以后天子又在甘泉宫接见诸侯,还在甘泉为诸侯建造官邸。勇之说:“越地有这样的风俗,火灾后再盖的房子一定要比烧掉的大,用来镇住火灾。”

于是天子就修建了建章宫,规模有千门万户那么大。它的前殿高过了未央宫,东面就是凤阙,高达二十多丈。它的西面是唐中苑,有数十里宽的虎圈。北面则修造了大水池,池中的渐台高二十多丈,池的名字叫泰液池,池中建有蓬莱、方丈、瀛洲和壶梁四座山,都仿照海中仙山,还有一些用石头雕的成龟鱼之类的东西。它的南面有玉堂、璧门和神鸟之类的雕像。还修建了神明台、井幹楼,都有五十多丈高,楼台之间连接着车道。

\begin{yuanwen}
夏,汉改历,以正月为岁首,而色上黄,官名更印章以五字。因为太初元年。是岁,西伐大宛。蝗大起。丁夫人、雒阳虞初等以方祠诅匈奴、大宛焉。
\end{yuanwen}

夏季,汉朝改用新历法,以夏历的正月当作一年的首月,官服改为黄色,每个官名的官印都改为五个字,因而就把年号改为太初元年。这一年,向西讨伐大宛。发生了大规模的蝗灾。丁夫人及雒阳虞初等人使用方术进行祭祀,祈祷神明降灾给匈奴、大宛。

\begin{yuanwen}
其明年,有司言雍五畤无牢熟具,芬芳不备。乃命祠官进畤犊牢具,五色食所胜,而以木禺马代驹焉。独五帝用驹,行亲郊用驹。及诸名山川用驹者,悉以木禺马代。行过,乃用驹。他礼如故。
\end{yuanwen}

第二年,相关官员说,在雍县祭祀五畤时没有使用煮熟的牲畜等祭品,也没有现出芬芳的香味。于是皇帝下令让祠官将牛犊制成的熟牲祭品分别献给五畤,按照五行生克的原则选用牲牢的毛色,并用木偶马替代壮马充当祭品。只在祭祀五帝之时才选用壮马,皇帝亲自去郊祀时才用壮马。祭祀名山大川时原本要用壮马的,也全部用木偶马替换。皇帝出巡经过时祭祀要用壮马。其他祭祀和原来一样。

\begin{yuanwen}
其明年,东巡海上,考神仙之属,未有验者。方士有言“黄帝时为五城十二楼,以候神人于执期\footnote{地名,为方士捏造。},命曰迎年”。

上许作之如方,名曰明年。上亲礼祠上帝,衣上黄焉。
\end{yuanwen}

又过了一年,天子向东巡视海边,考察神仙一类的事情,没有灵验的。有的方士说“黄帝时期曾修造了五城十二楼,目的就是便于在执期这地方迎候神仙,并称作迎年”。

皇上同意按他的说法修建五城十二楼,并称其为明年。皇帝还亲自前往行礼拜祭上帝,穿着黄色的衣服。

\begin{yuanwen}
公玊带曰:“黄帝时虽封泰山,然风后、封钜、岐伯令黄帝封东泰山,禅凡山合符,然后不死焉。”

天子既令设祠具,至东泰山\footnote{即今沂山。},东泰山卑小,不称其声,乃令祠官礼之,而不封禅焉。其后令带奉祠候神物。

夏,遂还泰山,脩五年之礼如前,而加禅祠石闾\footnote{山名。}。石闾者,在泰山下阯南方,方士多言此仙人之闾也,故上亲禅焉。
\end{yuanwen}

公玊带说:“黄帝时期虽然在泰山举行过封禅大典,但风后、封钜、岐伯这些人还要求黄帝前往东泰山筑坛祭天,来到凡山辟场祭地,以求与符瑞相合,之后才可以长生。”

天子就下令筹备祭品,然后来到东泰山,看到东泰山矮小,和它的名声并不相符,就命令祠官进行祭祀,但不举行封禅大典。后来又命令公玊带去那里供奉、祭祀并迎候神灵。

夏季,皇帝才回到泰山,和之前一样举行每五年一次的封禅大典,还增加了在石闾山祭地的礼仪。石闾位于泰山脚的南方,方士中有许多人说那里是仙人居住的地方,因此皇上亲自前去祭祀。

\begin{yuanwen}
其后五年,复至泰山脩封,还过祭常山。
\end{yuanwen}

之后五年,皇帝再次来到泰山举行封禅大典,归来途中还祭祀了常山。

\begin{yuanwen}
今天子所兴祠,泰一、后土,三年亲郊祠,建汉家封禅,五年一脩封。薄忌泰一及三一、冥羊、马行、赤星\footnote{即前文中的“灵星”。},五,宽舒之祠官以岁时致礼。凡六祠,皆太祝领之。至如八神诸神,明年、凡山他名祠,行过则祀,去则已。方士所兴祠,各自主,其人终则已,祠官弗主。他祠皆如其故。今上封禅,其后十二岁而还,遍于五岳、四渎矣。而方士之候祠神人,入海求蓬莱,终无有验。而公孙卿之候神者,犹以大人迹为解,无其效。天子益怠厌方士之怪迂语矣,然终羁縻弗绝,冀遇其真。自此之后,方士言祠神者弥众,然其效可睹矣。
\end{yuanwen}

如今天子所定下的祭祀,泰一与后土,都是每隔三年就亲自郊祀一次,建立起汉室的封禅制度,每隔五年举行一次封禅典礼。薄诱忌还奏请修建了泰一祠和三一、冥羊、马行、赤星等共五座神祠,由宽舒属下的祠官每年按时前往祭祀。再加上后土祠,一共六座神祠,统一由太祝管理。至于八神中的各神,还有明年、凡山这些地方的有名的神祠,都在天子经过之时祭祀,离开的话就算了。方士们所修造的神祠,都由他们各自祭祀,等到人死了,祭祀也就结束了,祠官不再掌管祭祀。其他的神祠都按照最初的规定办理。自皇帝举行封禅典礼起,一共经过了十二年,曾祭祀过的神灵遍布五岳四渎。而方士们迎候拜祭神仙,前往海上求访蓬莱仙山,始终没有结果。公孙卿这些等候神仙的方士,依旧以巨人脚印做借口来辩解,却并没有效果。天子越来越厌倦方士们的荒唐之语了,却始终控制着他们,并没断绝与他们往来,总还渴求能遇到方术奏效的人。自那之后,方士们议论祭神之说的更多了,但效果是人们都能看到的。

\begin{yuanwen}
太史公曰:余从巡祭天地诸神名山川而封禅焉。入寿宫侍祠神语,究观\footnote{仔细观察。}方士祠官之言,于是退而论次自古以来用事于鬼神者,具见其表里。后有君子,得以览焉。至若俎豆珪币之详,献酬之礼,则有司存\footnote{在相关官员那里有具体规定。}焉。
\end{yuanwen}

太史公说:我跟随皇帝一同出巡,拜祭天地众神及名山大川,参加了封禅典礼。我也进入过寿宫陪皇帝祭祀,听到过祝官的祷词,细看并研究过方士与祠官们的言辞,于是归来后按照顺序论述从古至今的祭祀鬼神的活动,将这些活动里里外外展现出来。后世的君子们,应该能从这里看到当时的情形。至于和祭祀有关的所用的俎豆等礼器、玉帛的详细情况,还有拜祭酬神的礼仪,在相关官员那里都有具体的规定。

\begin{yuanwen}
孝武纂极,四海承平。志尚奢丽,尤敬神明。坛开八道,接通五城。朝亲五利,夕拜文成。祭非祀典,巡乖卜征。登嵩勒岱,望景传声。迎年祀日,改历定正。疲秏中土,事彼边兵。日不暇给,人无聊生。俯观嬴政,几欲齐衡。
\end{yuanwen}

\part{卷十三}

\chapter{三代世表第一}

陈仁锡:「史之为表也,一经一纬,从行曰经,横行曰纬。《三代世表》以帝王世为经,其属为纬,然世属长短参差不齐,以殷之属十七君,与夏之世十七王大略相当,至周之属十七君,与夏殷之世四十七王,则是周一君而当夏殷三王也。今本《世表》,殷周二属牵连横书,殷属终帝槐之世,周属终帝芒之世,自帝泄至帝辛有世而无属。又自成王以后,周之世与列侯之属长短亦不相当,皆由后人抄录之讹也。」

本篇主要列举了自黄帝到三代天下一统这一时期的历史,从这篇表中,能够看出黄帝对后世产生的深远影响。

\begin{yuanwen}
太史公曰:五帝、三代之记,尚矣。自殷以前诸侯不可得而谱,周以来乃颇可著。孔子因史文次\footnote{编次,编排。}《春秋》,纪元年,正时日月,盖\footnote{通“盍”,何。}其详哉。至于序《尚书》则略,无年月;或颇有,然多阙,不可录。故疑则传疑,盖其慎也。
\end{yuanwen}

太史公说:关于五帝和三代的记载,已经很久远了。在殷代之前诸侯国的历史都很难编列出来,从周以来的史事才稍稍能够著录。孔子依照文史材料编排了《春秋》,就是用鲁公的纪元年数,整合事件年月,已经相当详尽了。倒是依次编纂《尚书》时却简去了年月;有的尽管有年月,却多有缺漏,难以著录。所以他这种虽有疑问却把疑问传下来的态度,是非常慎重的啊!

\begin{yuanwen}
余读谍\footnote{通“牒”。}记,黄帝以来皆有年数。稽其历谱谍、终始五德之传\footnote{五德终始学说认为王朝更替是按金、木、水、火、土五行的规律进行的,一个王朝受其中一行的支配,这就是“德”。},古文咸不同,乖异。夫子之弗论次其年月,岂虚哉!于是以《五帝系谍》、《尚书》集世纪黄帝以来讫共和为《世表》。
\end{yuanwen}

我读到谍记,自黄帝而下都记载着年份。在考察那些年历谱谍及五德转次相承的各种情况方面,古文的记载却并不相同,甚至会有很大差异。孔夫子并没有谈论这些年月,怎么可能没有原因呢?因此我就依照《五帝系谍》及《尚书》上记载的从黄帝到共和时代的世系,著成了《三代世表》。

\begin{yuanwen}
(表略)
\end{yuanwen}

\begin{yuanwen}
[补]张夫子\footnote{名长安,字幼君,是汉元帝、成帝时人。事迹见《汉书·儒林传》。}问褚先生\footnote{名少孙,汉元帝、成帝时博士,曾增补《史记》,《史记》中的“禇先生曰”都是他的补作。}曰:“《诗》言契、后稷皆无父而生。今案\footnote{考察。}诸传记咸言有父,父皆黄帝子也,得无与《诗》谬乎?”
\end{yuanwen}

张夫子曾经问褚先生说:“《诗经》记载契和后稷并无父亲就生了下来。现在考察一下传记,都表明他们有父亲,他们的父亲都是黄帝的子孙。难道这种说法与《诗经》所载并不违背吗?”

\begin{yuanwen}
褚先生曰:“不然。《诗》言契生于卵,后稷人迹者,欲见其有天命精诚之意耳。鬼神不能自成,须人而生,奈何无父而生乎!一言有父,一言无父,信以传信,疑以传疑,故两言之。尧知契、稷皆贤人,天之所生,故封之契七十里,后十馀世至汤,王\footnote{称王,为王。}天下。尧知后稷子孙之后王也,故益封之百里,其后世且千岁,至文王而有天下。《诗传》曰:“汤之先为契,无父而生。契母与姊妹浴于玄丘水,有燕衔卵堕之,契母得,故含之,误吞之,即生契。契生而贤,尧立为司徒,姓之曰子氏。子者兹;兹,益大也。诗人美而颂之曰“殷社芒芒,天命玄鸟,降而生商”。商者质,殷号也。文王之先为后稷,后稷亦无父而生。后稷母为姜嫄,出见大人迹而履践\footnote{履行实践。}之,知于身,则生后稷。姜嫄以为无父,贱而弃之道中,牛羊避不践也。抱之山中,山者养之。又捐之大泽,鸟覆席食之。姜嫄怪之,于是知其天子,乃取长之。尧知其贤才,立以为大农,姓之曰姬氏。姬者,本也。诗人美而颂之曰“厥初生民”,深修益成,而道后稷之始也。”孔子曰:“昔者尧命契为子氏,为有汤也。命后稷为姬氏,为有文王也。大王命季历,明天瑞也。太伯之吴,遂生源也。”天命难言,非圣人莫能见。舜、禹、契、后稷皆黄帝子孙也。黄帝策天命而治天下,德泽深后世,故其子孙皆复立为天子,是天之报有德也。人不知,以为氾从布衣匹夫起耳。夫布衣匹夫安能无故而起王天下乎?其有天命然。”
\end{yuanwen}

褚先生回答说:“不是这样的。《诗经》记载,契在卵中出生,后稷则因人迹而出生,这都是为了表明他们的天命精诚的意思。鬼与神都无法自己生成,必须要经由人才能产生,怎么能说是契和后稷没有父亲就生下来了呢?有种说法是他们是有父亲的,另一种说法就觉得他们并无父亲。信的人就传播信的观点,怀疑的人就传下怀疑之处,所以才有两种说法。尧相信契和后稷都是贤明之人,为上天所生,所以就赐给契七十里的封地,之后又经历了十多代才传到了汤,最后君临天下。尧也知晓后稷的子孙未来会成就帝业,所以就加赐后稷一百里封地,他的后代又经过了近千年,到了周文王时期才能拥有天下。《诗传》上说:‘汤的祖先就是契,契并无父亲却生下来了。契的母亲与她的姐妹们来到玄丘水洗澡,有只燕子口中衔着的卵掉了下来,契的母亲拿到了卵,本来用嘴含着,却不小心吞了下去,后来就生下了契。契天生就十分贤良,尧任命他为司徒,赐给他的姓氏为子氏。子其实就是“兹”;兹有渐渐强大之意。诗人赞美并称颂他说“殷的土地广阔无边!上天降下玄鸟,所以才产生了商”。商的意思是“质”,质就成了殷的美号。周文王的祖先就是后稷,后稷也并无父亲就生了下来。后稷的母亲是姜嫄,出门时见到了巨人的足迹,她踩在上面,就觉得怀孕了,后来就生下了后稷。姜嫄因为儿子并无父亲,就很轻视他,把他丢在路上,牛羊全都躲开而不去踩踏他。姜嫄把他丢进山里,山里的人却喂养了他。姜嫄又把他丢进大泽之中,鸟都覆盖着他,遮挡并喂养他。姜嫄因此而觉得疑惑,认为他是上天之子,就把他带回来抚养成人。尧知道他是个贤能的人才,就任命他为大农,赐给他的姓氏为姬氏。姬,就是“本”。诗人赞美并称颂他说“自从有生民之后”,辛勤修炼而成就日长,这就成了后稷称道的开始。’孔子说:‘过去尧赐给契子氏,是为了汤的出现。赐后稷姓氏为姬氏,是为了文王的出现。大王将季历任命为自己的继承者,这也表明了上天降下的祥瑞。太伯来到吴地,因此才有了周王朝代代传递的源本。’上天的命兆很难言说,除了圣人谁也无法得知。舜、禹、契和后稷等人都是黄帝的子孙。黄帝承受上天的旨意而统治天下,他的德行对后世影响深远。因此他的子孙才能都被再次任命为天子,这是上天对有德行的人的报答。人们都不知道这个道理,觉得帝王全都是从普通百姓中出现的。普通百姓怎能平白无故地兴起并君临天下呢?这都是天命使然。”

\begin{yuanwen}
“黄帝后世何王天下之久远邪?”
\end{yuanwen}

“黄帝的后代怎么会统治天下那么长时间呢?”

\begin{yuanwen}
曰:“《传》云天下之君王为万夫之黔首\footnote{百姓。}请赎民之命者帝,有福万世。黄帝是也。五政明则修礼义,因天时举兵征伐而利者王,有福千世。蜀王,黄帝后世也,至今在汉西南五千里,常来朝降,输献于汉,非以其先之有德,泽流后世邪?行道德岂可以忽乎哉!人君王者举而观之。汉大将军霍子孟名光者,亦黄帝后世也。此可为博闻远见者言,固难为浅闻者说也。何以言之?古诸侯以国为姓。霍者,国名也。武王封弟叔处于霍,后世晋献公灭霍公,后世为庶民,往来居平阳。平阳在河东,河东晋地,分为卫国。以《诗》言之,亦可为周世。周起后稷,后稷无父而生。以三代世传言之,后稷有父名高辛;高辛,黄帝曾孙。《黄帝终始传》曰:“汉兴百有馀年,有人不短不长,出白燕之乡,持天下之政,时有婴儿主,(欲/却)行车。”霍将军者,本居平阳白燕。臣为郎时,与方士考功会旗亭下,为臣言。岂不伟哉!”
\end{yuanwen}

褚先生说:“《经传》上记载:天下的君主就是所有百姓的元首,是祈祷上天延续百姓生命的人,他们就是帝,有福泽能够万世流传。黄帝就是这种人。五政清明而兴修礼义,遵照天时举兵征讨而取得胜利的就可称王,有福泽能够千世流传。蜀王,就是黄帝的后代,今天在汉西南五千里之外的地方,却常常来朝拜汉天子,献出贡物给汉朝,这怎会不是由于他的祖先有德行,德泽能够流传后世的结果呢?怎么可以忽视修练德行呢!君主和王者都应该以此来省察自己。汉朝的大将军霍子孟,名光,他也是黄帝的后人。这样的事只能与那些博闻远见的人谈论,本就不能对那些浅陋的人说。为什么要这样说呢?古时诸侯总是用国作为姓氏。霍氏,就是一个国家的名称。周武王曾经将他的弟弟叔处封在霍地,到了后来,晋献公消灭了霍公,霍公的后人就成了平民,在平阳周围往来居住。平阳处于河东地区,河东是晋国的领土,后来才析分成了卫国。根据《诗经》记载,也可以说成是周的后代。周兴起于后稷,后稷并无父亲却出生了。根据三代世传而言,后稷应该有个父亲叫高辛;高辛,就是黄帝的曾孙。《黄帝终始传》记载:‘汉朝已经兴起了一百多年,有个人既不高也不矮,从白燕之乡而来,掌握着天下的政事。那时还有个幼小的皇帝,这个人能让幼主的辇车倒着行走。’霍将军,就是最初居住在平阳白燕乡的那个人。臣在做郎官之时,曾经与方士们考功时在旗亭下相会,方士和我说起过霍将军的情况。难道他还不伟大吗?”

\begin{yuanwen}
高辛之胤,大启祯祥。脩己吞薏,石纽兴王。天命玄鸟,简秋生商。姜嫄履迹,祚流岐昌。俱膺历运,互有兴亡。风馀周召,刑措成康。出彘之后,诸侯日彊。
\end{yuanwen}

\part{卷十四}
\chapter{十二诸侯年表第二}

司马贞:「按:篇言十二,实叙十三者,贱夷狄不数吴,又霸在后故也。不数而叙之者,阖闾霸盟上国故也。」茅坤:「古雅可诵。孔子作《春秋》,而太史公得因之以表十二诸侯本末盛衰之迹也。」王国维:「太史公作《十二诸侯年表》,实为《春秋》《国语》作目录,故云『为成学治古文者要删』。」

本表所列举的事件起自周公、召公共和执政,结束于孔子去世,表现了当时诸侯专政、五霸盛衰的春秋时代的历史,社会由大一统逐渐走向分裂。

\begin{yuanwen}
太史公读《春秋历谱谍》,至周厉王,未尝不废书而叹也。曰:呜呼,师挚见之矣!纣为象箸而箕子唏\footnote{哀叹。}。周道缺,诗人本之衽席\footnote{卧席,引申为寝居之所,指男女房中之事。衽,寝卧之席。},《关雎》作。仁义陵迟,《鹿鸣》刺焉。及至厉王,以恶闻其过,公卿惧诛而祸作,厉王遂奔于彘,乱自京师始,而共和行政焉。是后或力政,彊乘弱,兴师不请天子。然挟王室之义,以讨伐为会盟主,政由五伯,诸侯恣行,淫侈不轨,贼臣(絪/篡)子滋起矣。齐、晋、秦、楚其在成周微甚,封或百里或五十里。晋阻三河,齐负东海,楚介江淮,秦因雍州之固,四海迭兴,更为伯主,文武所(襃/褒)大封,皆威而服焉。是以孔子明王道,干七十馀君,莫能用,故西观周室,论史记旧闻,兴于鲁而次《春秋》,上记隐,下至哀之获麟,约其辞文,去其烦重,以制义法,王道备,人事浃\footnote{周全。}。七十子之徒口受其传指,为有所刺讥(襃/褒)讳挹\footnote{通“抑”。}损之文辞不可以书见也。鲁君子左丘明惧弟子人人异端,各安其意,失其真,故因孔子史记具论其语,成《左氏春秋》。铎椒为楚威王(传/傅),为王不能尽观《春秋》,采取成败,卒四十章,为《铎氏微》。赵孝成王时,其相虞卿上采《春秋》,下观近势,亦著八篇,为《虞氏春秋》。吕不韦者,秦庄襄王相,亦上观尚古,删拾《春秋》,集六国时事,以为八览、六论、十二纪,为《吕氏春秋》。及如荀卿、孟子、公孙固、韩非之徒,各往往捃摭\footnote{摘取。}春秋之文以著书,不可胜纪。汉相张苍历谱五德,上大夫董仲舒推《春秋》义,颇著文焉。
\end{yuanwen}

太史公在读《春秋历谱谍》时,每当读到周厉王之时,总会放下书并发出叹息。说:唉!鲁国的太师挚可真是有真知灼见啊!商纣王制作象牙筷子,而箕子因为他的奢侈而悲叹。周朝的政治尚有缺陷,诗人以男女爱情生活为基础,作出了《关雎》。仁义日渐衰败,诗人创作《鹿鸣》诗以讽刺。到了周厉王,因为厌恶听到别人对他指出过失,那时的公卿都害怕遭到杀害而不敢直谏,祸乱由此而发生,结果厉王逃奔到了彘,祸乱是从京师开始的,后来朝廷由周公、召公联合起来执政。从那以后,不同的诸侯国彼此开始武力征伐,强盛的国家欺压弱小的国家,征讨也不再请示周天子。但这些诸侯国却都假借着朝廷的名义讨伐别的国家,争夺盟主之位,政令都操纵在五霸手中,诸侯任意横行,奢侈淫乱,不守法度,作乱篡权的臣子渐渐兴起了。齐、晋、秦、楚等国家在西周时都十分弱小,他们有的封地百里,有的只有五十里。晋国有三河相阻,齐国背负着东海,楚国则划长江、淮河为界,秦国所依靠的是雍州的险固,这四个国家依次兴起,相继做过霸主,文王、武王曾经褒封过的大国,都因为他们的声威而顺服。因此孔子为了彰明帝王之道,曾经为七十多个国君做事,都没能得到重用。所以孔子就向西去观阅周室的图籍,论述历史旧闻,以鲁国的历史记录为基础编成《春秋》,上起鲁隐公元年,向下一直到鲁哀公时的获麟之年。他简略了史籍所用的文辞,删去了那些繁琐冗杂的地方,以此来制定义理和法度,进而让王道更加完备,让人事得以通彻。孔子的七十多个徒弟都接受了他亲口传授的意旨,因为其中包含着批评、规劝、褒扬、隐讳、贬抑、损伤之类的文辞,所以不能从书面上表达出来。鲁国的君子左丘明担忧孔子的徒弟们都各持异端,各执己见,丧失掉《春秋》的最初面目,因此他以孔子编的历史为基础,详细阐述了他的观点,写出了《左氏春秋》。铎椒是楚威王的师傅,因为楚王无法阅读全部的《春秋》,他就选取了其中讲述成败的经验教训部分,写出了四十章,命名为《铎氏微》。赵孝成王之时,他的宰辅虞卿上采《春秋》的记载,下观当今时代的形势,也写出了八篇著作,也就是《虞氏春秋》。吕不韦,是秦国庄襄王的丞相,他也查阅了远古史迹,删减或是选用《春秋》的内容,汇集了六国的时事,写出了八览、六论、十二纪,称之为《吕氏春秋》。而到了荀卿、孟子、公孙固及韩非这些人,也都时常各自采取《春秋》的文义来著书立说,实在是难以全部记述下来。汉朝的丞相张苍采用历谱的格式写出了《终始五德传》,上大夫董仲舒推究《春秋》所包含的微言大义,也很是写出了一些文章。

郭嵩焘:「案厉王奔于彘而君臣之分渎,纪纲坏失,周道遂不可振,史公谱《三代世表》讫于共和,而用为诸侯年表之原始,盖有深意存焉,《年表》序所以专致慨于厉王也。其谱十二诸侯,归本孔子《春秋》。史公之著《史记》自以为继《春秋》而作,以明著书之旨也,而因采取诸子之名为《春秋》者论次之,赞语著明诸家得失,以自证其上拟《春秋》之义,言微而旨远矣。」

\begin{yuanwen}
太史公曰:儒者断其义,驰说者骋其辞,不务综其终始;历人取其年月,数家隆于神运,谱谍独记世谥,其辞略,欲一观诸要难。于是谱十二诸侯,自共和讫孔子,表见《春秋》、《国语》学者所讥盛衰大指著于篇,为成学治古文者要删焉。
\end{yuanwen}

太史公说:对于《春秋》来说,儒者所用只是断章取义,游说家发挥的只是它的辞令,并不重视综合其始终;钻研历法之人只看重《春秋》的年月,研究命运的人只看到《春秋》的神运,整饬谱谍的人只借鉴《春秋》的世系,所用的言词都十分简略,我们想一睹《春秋》的要旨都相当困难。因此我就将春秋时期的十二个诸侯国编成系谱,上起共和下至孔子,用年表的形式把钻研《春秋》、《国语》的学者所得到的盛衰意旨详述在本篇中,为那些立志成学治古文的人提举纲要,删去芜杂。

\part{卷十五}
\chapter{六国年表第三}

陈仁锡:「十二国、六国《年表》以年为经,以国为纬,盖参用《春秋》书法而纪其大事,与本纪世家相表里者也。」

「表名六国,实纪七雄,亦犹《十二诸侯年表》实载十三国也。」

「魏表附晋、卫,韩表附郑,楚表附鲁、蔡,齐表附宋,何也?以晋、卫、郑、鲁、蔡、宋诸国为魏、韩、楚、齐所灭,故附纪焉,亦以终十二诸侯事也。」

本表起自周元王,到秦二世结束,大致展现了战国时代纵横捭阖的历史,同时还较为突出地展现了秦国由微弱到强大,从一统天下到最终灭亡的整个过程。

\begin{yuanwen}
太史公读《秦记》,至犬戎败幽王,周东徙洛邑,秦襄公始封为诸侯,作西畤\footnote{神祠。}用事上帝,僭端见矣。《礼》曰:“天子祭天地,诸侯祭其域内名山大川。”今秦杂戎翟之俗,先暴戾,后仁义,位在籓臣而胪于郊祀,君子惧焉。及文公逾陇,攘夷狄,尊陈宝,营岐雍之间,而穆公脩政,东竟至河,则与齐桓、晋文中国侯伯侔矣。是后陪臣执政,大夫世禄,六卿擅晋权,征伐会盟,威重于诸侯。及田常杀简公而相齐国,诸侯晏然弗讨,海内争于战功矣。三国终之卒分晋,田和亦灭齐而有之,六国之盛自此始。务在彊兵并敌,谋诈用而从衡短长之说起。矫称蜂出,誓盟不信,虽置质剖符犹不能约束也。秦始小国僻远,诸夏宾之,比于戎翟,至献公之后常雄诸侯。论秦之德义不如鲁卫之暴戾者,量秦之兵不如三晋之彊也,然卒并天下,非必险固便形势利也,盖若天所助焉。
\end{yuanwen}

太史公阅览《秦记》,看到犬戎击败周幽王,周平王向东迁徙到洛邑,秦襄公因为护送平王有功而被封为诸侯,修造了西畤以侍奉天帝,就觉得这有越位犯上的痕迹。《礼记》中说:“天子拜祭天地,诸侯拜祭他们封地中的名山大川。”如今秦国掺杂有戎狄的习俗,以暴戾为根本,将仁义放在末节,虽然位处藩臣却列席天子的郊祭。有识之土都十分忧惧。到了秦文公,他越过陇坂向东进发,打败了夷狄,尊奉陈宝为神物,在岐山与雍州之间营造基业。穆公修明国政,秦国边境向东一直到黄河,这就和齐桓公、晋文公这些中原霸主并驾齐驱了。从那以后,陪臣们掌管着国政,大夫们世世代代享有俸禄;六卿掌管晋国的政权,利用征伐或者会盟,他们的权势压倒了别的诸侯。等到田常杀掉了齐简公,成为齐国的丞相,诸侯们都十分漠然,并没有前往讨伐,海内都在争相忙于战功了。终于韩、赵、魏三国瓜分掉了晋国,田和也灭掉齐国并占有了它,六国的强盛就是在这时开端的。不同的诸侯国都以加强军事力量、吞并敌国为己任,阴谋诡计所用至极,纵横短长的说法纷纷出现。假借王命的诸侯像蜂一样拥出,盟誓也不再遵守,虽然各国都互送人质、剖析符契,却都无法约束。秦国最初只是个小国,地处边远,中原的诸侯国都排斥它,甚至将它等同于西方的戎翟看待,到了献公以后却常常在诸侯中称雄。说起来,秦国的德义之行,也比不上鲁卫的暴戾之行更加合乎德义,估计秦国的兵力不如三晋的强大,可最终秦国却兼并了天下,这并不是秦国靠着地理位置的险固,以及形势的便利,而似乎是上天的帮助。

\begin{yuanwen}
或曰“东方物所始生,西方物之成孰”。夫作事者必于东南,收功实者常于西北。故禹兴于西羌,汤起于亳,周之王也以丰镐伐殷,秦之帝用雍州兴,汉之兴自蜀汉。
\end{yuanwen}

有人说“东方是所有事物的开端之地,西方为所有事物的成熟所在”。首先起事的人一定在东南,而得到实际的却常常在西北。因此大禹兴起在西羌,商汤兴起于亳,周朝取得王道是借助于丰镐之地征讨殷纣,秦王能够称帝是依靠雍州,汉朝的兴盛源自蜀汉。

牛运震:「战国七雄,独秦最强,六国皆为秦所并。又六国时事多见于《秦纪》,故《年表》总论以《秦纪》发端,以《秦纪》收结,中间以秦取天下为主,而以六国事夹说带叙,归于权变诈谋,以为俗变议卑,亦有可采,殆有痛于中为是不得已之论,然而世变可睹矣。」

\begin{yuanwen}
秦既得意,烧天下《诗》、《书》,诸侯史记尤甚,为其有所刺讥也。《诗》、《书》所以复见者,多藏人家,而史记独藏周室,以故灭。惜哉,惜哉!独有《秦记》,又不载日月,其文略不具。然战国之权变亦有可颇采者,何必上古?秦取天下多暴,然世异变,成功大。传曰“法后王”,何也?以其近己而俗变相类,议卑而易行也。学者牵于所闻,见秦在帝位日浅,不察其终始,因举而笑之,不敢道,此与以耳食无异?悲夫!
\end{yuanwen}

秦国已经得到天下,就烧毁了天下的《诗经》和《尚书》,对别的诸侯国的历史记载尤其毁灭得厉害,因为其他诸侯国的史书中多有嘲讽秦国的言语。《诗经》和《尚书》之所以还能再看到,是因为这些书多藏在私人家中,而记载历史的典籍单独藏在周室,因此都被烧掉了。多么可惜,多么可惜啊!唯独《秦记》得以保存,却偏偏又没有记载日月,所用的文辞也简略不全。但是在战国时期,权变之法也都有许多可以选取的地方,何必都要遵循上古的规矩呢?秦能够得到天下多采取残暴的手段,但社会在不断变化,秦朝能因时代变迁而变法图强,所以成就很大。经传中说“要师法后王”,这是怎么回事呢?这是由于后王的施政和我们这个时代相似,而风俗演变也与我们类似,他们的议论虽然卑下却简单易行。一般学者受自己的见闻牵制过多,见到秦朝在帝位的日子并不多,不去钻研它的终始,就因此而讥笑它,不敢称道秦朝,这和轻信别人的话有什么分别呢?多么可悲呀!

\begin{yuanwen}
余于是因《秦记》,踵\footnote{紧跟。}《春秋》之后,起周元王,表六国时事,讫二世,凡二百七十年,著诸所闻兴坏之端。后有君子,以览观焉。
\end{yuanwen}

我因此就依照《秦记》的记载,紧跟孔子的《春秋》,起自周元王,列出了六国时事,一直到秦二世,一共二百七十年,写出了我听到过的所有和治乱兴废相关的缘由,以供后世的有识之士观览。

\part{卷十六}
\chapter{秦楚之际月表第四}

本表起自陈涉发难,终于刘邦称帝,表现了秦末社会动荡,天下诸侯并起,从推翻暴秦的统治到楚汉相争,最后终于走向汉家一统天下的这段历史。

\begin{yuanwen}
太史公读秦楚之际\footnote{楚是指项羽所建立的西楚政权,秦楚之际指秦二世胡亥至楚霸王项羽灭亡这一历史时期。},曰:初作难,发于陈涉;虐戾灭秦,自项氏;拨乱诛暴,平定海内,卒践\footnote{登。}帝祚\footnote{帝位。},成于汉家。五年之间,号令三嬗\footnote{变。}。自生民以来,未始有受命若斯之亟\footnote{迅速。}也。
\end{yuanwen}

太史公读到秦楚之际的历史,认为:最先起义反秦的是陈涉;依靠残暴的手段使秦朝灭亡的是项羽;拨乱反正,诛杀强暴,平定四方,最后登上帝位,成就这项功业的是汉家。在五年的时间内,发号施令的人就换了三次。自从有了人类以来,还从未有接受天命像这么快的。

\begin{yuanwen}
昔虞、夏之兴,积善累功数十年,德洽百姓,摄行政事,考之于天,然后在位。汤、武之王,乃由契、后稷脩仁行义十馀世。不期而会孟津八百诸侯,犹以为未可,其后乃放弑。秦起襄公,章于文、缪,献、孝之后,稍以蚕食六国,百有馀载,至始皇乃能并冠带之伦。以德若彼,用力如此,盖一统若斯之难也。
\end{yuanwen}

过去虞、夏的兴起,开创帝业的君主们都是积累善行和功绩好几十年,他们的功德恩泽百姓,他们在执行政事的时候都接受了上天的考验,然后才正式登上帝位。商汤和周武王之所以能称王,是由于从他们的祖先契和后稷开始,就修行仁义长达十几代。周武王讨伐商纣时八百诸侯没有通过邀约就在孟津会盟了,但周武王还是认为推翻商纣的条件并不成熟。等到时机成熟后,商汤才流放了夏桀,周武王才杀掉了商纣。秦国的兴起始于襄公,文公、穆公时期逐渐彰显,到了献公、孝公以后,才渐渐地蚕食关东六国,经过一百多年,到了秦始皇时期才兼并了六国。像虞、夏、商、周那样以恩德泽润百姓,像秦国这样使用强力兼并诸侯,都是经历了长期的努力才能成功的,原来统一天下是如此的艰难啊!

\begin{yuanwen}
秦既称帝,患兵革不休,以有诸侯也,于是无尺土之封,堕坏名城,销锋镝,鉏豪桀,维万世之安。然王迹之兴,起于闾巷\footnote{民间。},合从讨伐,轶\footnote{超过。}于三代,(乡/鄉)秦之禁,適足以资贤者为驱除难耳。故愤发其所为天下雄,安在“无土不王”?此乃传之所谓大圣乎?岂非天哉,岂非天哉!非大圣孰能当此受命而帝者乎?
\end{yuanwen}

秦始皇称帝之后,担心战乱不能休止,是因为有诸侯的存在,因此他连一尺土地也没有分封,毁坏了有名的城池,销毁了天下的兵器,剪除了天下的豪强,他希望靠这些手段来维持万世的安定。然而帝王之业的勃兴却起于民间,各地的豪杰都联合起来攻秦,其浩大的声势超过了夏、商、周三代。过去秦朝不封诸侯、毁坏名城、销毁兵器等制度措施恰好帮助贤者排除了他们推翻秦朝的困难。所以,高祖以一个普通百姓的身份愤发而起,成为了天下的雄略之主,怎么能说“没有封土便不能成王呢”?他不就是书传上称道的大圣人吗?这难道不是天意吗,难道不是天意吗!如果不是大圣人,怎能够在这豪杰并起的时期承受天命而称帝呢?

\part{卷十七}
\chapter{汉兴以来诸侯王年表第五}

本表起自高祖元年,终至武帝太初四年,展现了汉兴百年之间各诸侯王的废立及分削情况。

\begin{yuanwen}
太史公曰:殷以前尚\footnote{久远。}矣。周封五等:公,侯,伯,子,男。然封伯禽、康叔于鲁、卫,地各四百里,亲亲之义,襃\footnote{褒奖。}有德也;太公于齐,兼五侯地,尊勤劳也。武王、成、康所封数百,而同姓五十五,地上不过百里,下三十里,以辅卫王室。管、蔡、康叔、曹、郑,或过或损。厉、幽之后,王室缺,侯伯彊国兴焉,天子微,弗能正。非德不纯,形势弱也。
\end{yuanwen}

太史公认为:殷朝以前的历史已经很久远了。周朝时期的爵位有五个等级:公、侯、伯、子、男。然而伯禽在鲁地、康叔在卫地,受封的封地各四百里,这是为了表示亲近亲属的情谊,褒扬建立了功德的人;在齐地分封太公吕望,让他兼有五个诸侯的封地,这是对他为国家操劳而表示尊重。周武王、成王、康王时受封的诸侯有数百个,而其中同姓的诸侯就有五十五个,他们的封地最大的不超过百里,最小的不少于三十里,要依靠他们来辅佐和捍卫王室。后来,管、蔡、康叔、曹、郑各诸侯的封地,有的超出了规定,有的遭到了削减。厉王、幽王之后,王室亏损,那些侯、伯却变为强国逐渐兴起,而周天子的势力微弱,无力拨正。这并不是周王室的品德不好,主要是由于势力已经衰败了。

\begin{yuanwen}
汉兴,序二等。高祖末年,非刘氏而王者,若无功上所不置而侯者,天下共诛之。高祖子弟同姓为王者九国,(虽/唯)独长沙异姓,而功臣侯者百有馀人。自雁门、太原以东至辽阳,为燕、代国;常山以南,大行左转,度河、济,阿、甄以东薄海,为齐、赵国;自陈以西,南至九疑,东带江、淮、穀、泗,薄会稽,为梁、楚、淮南、长沙国:皆外接于胡、越。而内地北距山以东尽诸侯地,大者或五六郡,连城数十,置百官宫观,僭于天子。汉独有三河、东郡、颍川、南阳,自江陵以西至蜀,北自云中至陇西,与内史凡十五郡,而公主列侯颇食邑其中。何者?天下初定,骨肉同姓少,故广彊庶孽,以镇抚四海,用承卫天子也。
\end{yuanwen}

汉朝兴起之后,将爵位分为两等,功大的封为王,功小的封为侯。在高祖末期,曾立下誓约,不是刘氏宗室而封王的,或者没有功劳、不是皇帝所封置而封侯的,天下的人都可以共同讨伐他们。那时高祖的子弟和同姓宗室封王的有九个,只有长沙王是异姓王,而有功劳的臣子被封为侯的有一百多位。自雁门、太原以东至辽阳一带属于燕国、代国;自常山以南,由太行山向东,过黄河、济水,从东阿、甄城以东一直延伸到海边,属于齐国、赵国;自陈县以西,南至九疑山,往东长江、淮河、穀水、泗水一带,临近会稽山,属于梁国、楚国、淮南国、长沙国。这些诸侯国都与匈奴和越族相接。而内地往北至崤山以东,都属于诸侯的封地,有些大的诸侯拥有五到六个郡,几十座城池相连,他们独立设置的百官和宫观已经超过了规定,几乎与天子相同。当时,汉王室仅仅占有三河、东郡、颍川、南阳,自江陵以西到蜀郡,北自云中到陇西,再加上内史,共有十五个郡。而公主的封地多数都在这些地区。为什么呢?天下才初步平定,能封王封侯的同姓人少,因此要广泛增强汉室子弟的力量,用来镇抚天下,以及辅佐和捍卫汉朝廷。

\begin{yuanwen}
汉定百年之间,亲属益疏,诸侯或骄奢,忕\footnote{习惯。}邪臣计谋为淫乱,大者叛逆,小者不轨于法,以危其命,殒身亡国。天子观于上古,然后加惠,使诸侯得推恩分子弟国邑,故齐分为七,赵分为六,梁分为五,淮南分三,及天子支庶子为王,王子支庶为侯,百有馀焉。吴楚时,前后诸侯或以適\footnote{zhé,通“谪”。}削地,是以燕、代无北边郡,吴、淮南、长沙无南边郡,齐、赵、梁、楚支郡名山陂海咸纳于汉。诸侯稍微,大国不过十馀城,小侯不过数十里,上足以奉贡职,下足以供养祭祀,以蕃辅京师。而汉郡八九十,形错诸侯间,犬牙相临,秉其(阨戹)塞地利,强本干,弱枝叶之势,尊卑明而万事各得其所矣。
\end{yuanwen}

汉朝平定天下不过百年的时间,亲属之间日益疏远,一些诸侯骄奢淫逸,习惯了奸佞之臣的诱惑,暗中策划叛乱的行动,动作大的成为了叛党,动作小的不遵循国家的法度,因而危及到自身的性命,最后导致身死国亡。天子考察了上古政治的得失,于是采取了增加诸侯恩惠的方法,促使他们把自己的恩泽推广出去,将国邑分给自己的子弟,所以齐被分为了七个国家,赵被分为了六个国家,梁被分为了五个国家,淮南被分为了三个国家,到了天子的旁支庶子被封王、王子的旁支庶子被封侯的,就达到了一百多个。吴楚动乱的时期,前后有些诸侯由于受到谪罚而减削封地,所以燕、代没有北部边郡,吴、淮南、长沙没有南部边郡,齐、赵、梁、楚的支郡和有名的山、海都由中央政府直接管辖。诸侯们的势力逐渐被削弱,大的封国只拥有十几个城,小的封侯只占有几十里地,使他们向上足以进献贡赋,对下足以供奉祭祀,用以辅卫京师。而汉中央政府所设置的郡共有八九十个,错综地分布在诸侯国之间,就像犬牙一样相互交错地衔接,朝廷掌握着险要的关塞和有利的地势,从而增强了中央政府的力量,减弱了诸侯国的势力,使尊卑关系明确了,所有的事情也就各得其所了。

\begin{yuanwen}
臣迁谨记高祖以来至太初诸侯,谱其下益损之时,令时世得览。形势虽彊,要之以仁义为本。
\end{yuanwen}

作为臣子,我司马迁慎重地记录了自高祖以来到太初时期的各诸侯国的史实,写下了他们后来的势力加强或削弱的情况,使后人得以阅览,从而从中得到教训。现在朝廷形势即使很强盛,但仍要将仁义作为治国的根本。

\part{卷十八}
\chapter{高祖功臣侯年表第六}

陈仁锡:「高祖功臣侯者一百四十三,至文帝之世,存者一百二十五,至武帝时,存者七十一,当时之守先典待旧勋,孰得孰失,皆可失矣。」

方苞:「汉武以列侯莫求从军,坐酎金失侯者百余人。迁不敢斥言其过,故微词以见义,言古之道,笃于仁义以安勋旧,而今任法刻削,不同于古,帝王疏礼异务,各以自就其功绪,岂可混而一之乎?刺武帝用一切之法以侵夺群下,而成其南诛北讨之功也。」

本表通过对高祖时期功臣侯者命运的关注,意在总结其失侯的原因,从一个侧面对汉朝统治者及有功之臣进行警示。

\begin{yuanwen}
太史公曰:古者人臣功有五品,以德立宗庙定社稷曰勋,以言曰劳,用力曰功,明其等曰伐,积日曰阅。封爵之誓曰:“使河如带\footnote{衣带。},泰山若厉\footnote{磨刀石。}。国以永宁,爰\footnote{于是。}及苗裔。”始未尝不欲固其根本,而枝叶稍陵夷\footnote{衰微。}衰微也。
\end{yuanwen}

太史公说:古代臣子的功勋分为五等,凭借德行建立基业、安定国家的,称之为“勋”;凭借进言立功的,称之为“劳”;凭借武力立功的,称之为“功”;表明功劳等级的,称之为“伐”;计算他们资历长短的,称之为“阅”。封爵的誓言上说:“即使黄河像衣带一样细了,泰山像磨刀石一样小了,也要使封国永远安宁,恩泽延续到子孙后代。”当初开始封爵,未尝不是想巩固国家的根本,但到了后来,它的枝叶却逐渐衰败、微弱下去了。

\begin{yuanwen}
余读高祖侯功臣,察其首封,所以失之者,曰:异哉新闻!《书》曰“协和万国”,迁于夏商,或数千岁。盖周封八百,幽厉之后,见于《春秋》。《尚书》有唐虞之侯伯,历三代千有馀载,自全以蕃卫天子,岂非笃于仁义,奉上法哉?汉兴,功臣受封者百有馀人。天下初定,故大城名都散亡,户口可得而数者十二三,是以大侯不过万家,小者五六百户。后数世,民咸归乡里,户益息,萧、曹、绛、灌之属或至四万,小侯自倍,富厚如之。子孙骄溢,忘其先,淫嬖。至太初百年之间,见侯五,馀皆坐法陨命亡国,秏矣。罔亦少密焉,然皆身无兢兢于当世之禁云。
\end{yuanwen}

我读了高祖对功臣封侯的记载,考察当初分封的爵位到后来之所以丧失的情况,认为:实际情况与传闻截然相反!《尚书·尧典》上说“要使万国的诸侯协调和睦”,这“万国的诸侯”变迁不断,到夏、商时期,已经有千年之久了。周初时期曾分封了八百个诸侯,到了幽王、厉王之后的诸侯的史迹,出现在《春秋》上。《尚书》上所记载的唐尧、虞舜所封的侯伯,历经夏、商、周三代,已经有一千多年了,他们还能保全自己的封爵,并能够作为屏藩来保卫周天子,这难道不是由于他们能够坚守仁义,遵循周天子的法规的缘故吗?汉朝在建立之后,受封的功臣有一百多人。当时,天下才初步安定,原本大城名都中的人口大多逃亡在外,留存下来的只有十分之二三。所以,大的诸侯的封邑不超过一万户,小的诸侯的封邑只有五六百户。过了几代之后,老百姓重新返回乡里,户口逐渐繁衍,萧何、曹参、周勃、灌婴这些人的后代,有的封邑的人口的数量竟能达到四万,甚至小侯的封户也成倍地增长,因而他们的财富也迅速汇集起来。但是,他们的子孙后代却因此而骄奢自满,忘记了他们的祖先在创业时的艰难,专干一些邪恶放荡的事情。到武帝太初之时,只过了百年的时间,保存下来的侯爵只剩下五个,其余的都因犯法而丧命亡国,全都耗费殆尽了。国家的法网日渐严谨是其中一个原因,但也是因为他们没有认认真真地对待当世的禁令。

朱东润:「读史之功,莫甚于读表,其所得往往有出于纪传世家以外者。余读《高祖功臣侯年表》,知诸侯功状,史迁皆有所本以资移录,其间存亡攻守之故,迁亦未能尽悉。」

\begin{yuanwen}
居今之世,志古之道,所以自镜也,未必尽同。帝王者各殊礼而异务,要以成功为统纪,岂可绲\footnote{同“混”。}乎?观所以得尊宠及所以废辱,亦当世得失之林也,何必旧闻?于是谨其终始,表其文,颇有所不尽本末;著其明,疑者阙之。后有君子,欲推而列之,得以览焉。
\end{yuanwen}

生活在当今之世,要记住古人处世的道理,是因为可以引以为鉴,当然,古今的做法未必全都相同。作为帝王的人要各自有不同的礼法和政务,最重要的是应该以成功的经验作为准则,哪能混同一致呢?分析诸侯王能够得到尊重、宠幸或遭受废弃、屈辱的原因,也是在当今时代能够取得成功或遇到失败的道理所在,何必专门去寻找古代的传闻呢?于是,我便仔细地考察了诸侯王被废立的始末,以表格的形式列出,并附以文字说明,但其中也有一些没能够详尽地记录下来。我只记述了那些清楚明确的材料,那些有疑问的地方就没有记录。如果后世有人想推究和阐述高祖分封功臣的原委,这个表就可以提供参考。

\begin{yuanwen}
(表略)

圣贤影响,风云潜契。高祖应箓,功臣命世。起沛入秦,凭谋仗计。纪勋书爵,河盟山誓。萧曹轻重,绛灌权势。咸就封国,或萌罪戾。仁贤者祀,昏虐者替。永监前修,良惭固蒂。
\end{yuanwen}

\part{卷十九}
\chapter{惠景间侯者年表第七}

本表展示了惠景间九十三个侯国的存亡兴废情况。

\begin{yuanwen}
太史公读列封至便侯\footnote{即吴浅,于孝惠帝元年受封为便侯。便,县名,在今湖南永兴。}曰:有以也夫!长沙王者,(着/著)令甲\footnote{汉代法令以甲、乙、丙、丁的次序编排。令甲就是法令中的第一篇。},称其忠焉。昔高祖定天下,功臣非同姓疆土而王者八国。至孝惠(帝)时,唯独长沙全,禅五世,以无嗣绝,竟无过,为籓守职,信矣。故其泽流枝庶,毋功而侯者数人。及孝惠讫孝景间五十载,追修\footnote{清理前事。}高祖时遗功臣,及从代来,吴楚之劳,诸侯子(弟)若肺腑,外国归义\footnote{归化。},封者九十有余。咸表始终,当世仁义成功之著者也。
\end{yuanwen}

太史公读有关列侯分封的资料,读到便侯时说到:所有的事情都是有原因的!长沙王被封为诸侯王,记录在法令的第一篇中,他的忠诚受到了赞扬。当初高祖平定天下,功臣之中不是皇室同姓宗亲而裂土受封的诸侯王有八个。到孝惠帝时,只剩下长沙王保全了他的封国,接连传承五世,最后因为没有后嗣才绝封,自始自终都没有犯什么过错,作为朝廷的藩守,尽职尽责,事情果真如此。所以他的德业能使旁系子孙都能享受到恩惠,没有建立功勋而受封为侯的就有数人。从孝惠帝之初到孝景帝之末,期间五十多年,追录高祖时遗漏未封的功臣,以及追随孝文帝从代国入继大统的旧臣,孝景帝时期在平定吴楚七国之乱的战事中立下了显著功劳的将相官员,身为皇室骨肉至亲的诸侯子弟,前来投顺的外邦异族的首领等等,先后受封为侯的有九十多人。现在把他们受封传承的情况列表记载如下,这些都是当代有仁义、建大功的人物中比较突出的。

\begin{yuanwen}
(表略)
\end{yuanwen}

\begin{yuanwen}
惠景之际,天下已平。诸吕构祸,吴楚连兵。条侯出讨,壮武奉迎。薄窦恩泽,张赵忠贞。本枝分荫,肺腑归诚。新市死事,建陵勋荣。咸开青社,俱受丹旌。旋窥甲令,吴便有声。
\end{yuanwen}

\part{卷二十}
\chapter{建元以来侯者年表第八}

本表所表现的主要是那些征伐匈奴、东越、南越等四夷的功臣。

《史记索隐》:“七十二国,太史公之旧;余四十五国,褚先生补也。”

\begin{yuanwen}
太史公曰:匈奴绝和亲\footnote{汉初与匈奴实行和亲政策,然文、景时仍不断侵犯,武帝以汉兴后积累的国力,对匈奴大举用兵,以保卫中原,扩充疆土。},攻当路塞\footnote{直当交通要道上的关塞。};闽越\footnote{居住在今浙江、福建交界地区的一支越族。}擅伐\footnote{擅自出兵讨伐。指其举动违背汉朝中央朝廷的意愿。},东瓯\footnote{居住在今浙江东南沿海一带的一支越族。ōu}请降\footnote{指建元三年(前138)闽越进攻东瓯,汉廷派兵往救,时东瓯请求内徙,而将他们安置在江淮间事。}。二夷\footnote{指匈奴与闽越。}交侵,当盛汉之隆,以此知功臣受封侔\footnote{齐等,相当。}于祖考\footnote{祖指大父,考指父亲。}矣。何者?自《诗》、《书》称三代“戎狄是膺\footnote{y\=ing},荆荼\footnote{舒。}是徵\footnote{ch\'eng}\footnote{语见《诗经·鲁颂·閟宫》。戎,指西戎。狄,指北狄。膺,击。荆,楚的别名。荼,同“舒”,楚的属国。徵,通“惩”,惩戒,惩罚。}”,齐桓越燕伐山戎\footnote{居住在今河北东北部一带的部族,或称北戎。},武灵王以区区赵服单于\footnote{匈奴最高首领的称号。},秦缪用百里\footnote{百里奚,曾辅佐秦穆公称霸。}霸西戎\footnote{西方戎族。},吴楚之君以诸侯役百越\footnote{秦汉以前即已广泛分布在长江中下游以南的越族,因其部落众多,故称百越或百粤。}。况乃以中国一统,明天子在上,兼文武,席卷四海,内辑\footnote{和协,安集。}亿万之众,岂以晏然\footnote{安适,平静。}不为边境征伐哉!自是后,遂出师北讨强胡\footnote{指匈奴。},南诛\footnote{讨伐。}劲越,将卒\footnote{将帅。卒,当作“率”,通“帅”。}以次封矣。
\end{yuanwen}

太史公说:匈奴断绝和亲后,攻打边境上的险要关塞;闽越凭借武力,擅自攻打东瓯,逼迫东瓯请降。这两支外夷交相不断地侵扰边境,即使是在汉朝最兴隆的时期也是如此,由此可以推知功臣受到分封的数量应该很多,当与高祖开国时相等。为什么会这样呢?自从《诗经》《尚书》说夏、商、周三代“抵挡抗击北方的戎狄,讨伐处罚南方的荆舒”以来,齐桓公曾经越过燕国去攻打山戎,赵武灵王依靠小小的赵国就降服了匈奴单于,秦穆公凭借百里傒称霸西戎,吴、楚两国的国君凭着诸侯的身份却能驱使百越。何况是一个统一的国家,上有圣明的天子统治,又同时具备文武之才,平定四海,使国内亿万百姓都团结和睦、安居乐业,面对外夷的侵扰怎能够安然无事,而不去经营边疆、出兵征伐呢?从此以后,汉朝就出兵北方,讨伐凶狠的匈奴,又出兵南方,消灭了强大的南越,将士们最终也都依据功劳受到了封赏。

\begin{yuanwen}
(表略)
\end{yuanwen}

\begin{yuanwen}
后进好事儒者褚先生\footnote{褚少孙,是汉代元帝、成帝间的一个博士。}曰:太史公记事尽于孝武之事,故复修记孝昭以来功臣侯者,编于左方,令后好事者得览观成败长短绝世之适,得以自戒焉。当世之君子,行权\footnote{变通,不守成规,见机行事。}合变,度\footnote{揣摩,揣测。}时施宜,希世\footnote{迎合世俗。}用事,以建功有土封侯,立名当世,岂不盛哉!
\end{yuanwen}

晚辈儒士褚先生说:太史公在本篇的记事截止到孝武帝,因此我又撰写记录了孝昭帝之后功臣封侯的情况,编在下方,这是为了让后世的晚辈能够看到那些功臣们、传世久远者的经验,以及那些失败者、很短时间内就将爵位断送了的教训,使自己引以为戒。当世的君子,能够不守成规而见机行事,能够审时度势而随机应变,能够迎合世俗而受到重用,从而建功立业,封地封侯,在当世扬名,好不兴盛至极啊!

\begin{yuanwen}
观其持满\footnote{谨慎地保住兴旺发达的家业。}守成\footnote{守住已经获得的成就。}之道,皆不谦让,骄蹇\footnote{骄傲自大。}争权,喜扬声誉,知进不知退,终以杀身灭国。以三得之\footnote{指上文“行权合变,度时施宜,希世用事”。},及身\footnote{在自己这一辈。}失之,不能传功于后世,令恩德流子孙,岂不悲哉!
\end{yuanwen}

然而观察他们谨慎地保住兴旺的家业和取得的成就的表现,一点也不谦虚谨慎,而是骄傲自大,争权夺力,喜欢宣扬自己的名声,只知前进,不留任何退路,最终都被杀,封国也随即灭绝。依靠以上三种途径获得的侯位,在自己这一代就丢掉了,没能将功业传给子孙后人,让子孙们也能沐浴恩德,这不是很大的悲哀吗?

\begin{yuanwen}
夫龙雒侯曾为前将军,世俗顺善,厚重谨信,不与政事,退让爱人。其先起于晋六卿之世。有土君国以来,为王侯,子孙相承不绝,历年经世,以至于今,凡百余岁,岂可与功臣及身失之者同日而语之哉?悲夫,后世其诫之!
\end{yuanwen}

龙雒侯韩曾出任前将军一职时,他能够顺应世俗,做善事,忠厚、谨慎,不参与政事,遇事谦和退让,爱护别人。他的先祖原本是春秋时晋国的六卿之一。自从拥有封地成为诸侯以来,子孙相继为王为侯,子子孙孙世代相传,不曾断绝,经过了多少年岁和世代,直到今天,算起来已经有一百多年了,怎么能够和在自己这一辈就失去爵位、封地的功臣们相提并论呢?实在是悲哀啊,后世的人们要引以为戒!

\begin{yuanwen}
孝武之代,天下多虞。南讨瓯越,北击单于。长平鞠旅,冠军前驱。术阳衔璧,临蔡破禺。博陆上宰,平津巨儒。金章且佩,紫绶行纡。昭帝已后,勋宠不殊。惜哉绝笔,褚氏补诸。
\end{yuanwen}

\part{卷二十一}
\chapter{建元已来王子侯者年表第九}

陈仁锡:「『盛哉天子之德』即所谓『推私恩』也,其说盖自主父偃发之。元光侯者七,元朔侯者一百二十七元狩侯者二十五,元鼎侯者三,当时之分封诸侯子弟施行次第,皆可知矣。」

“王子侯”,即本表中所列王侯皆因王子而侯,也就是这些人并无功可言。本表共列出了武帝所封的王子侯一百六十二人。

\begin{yuanwen}
制诏\footnote{以制书诏告御史。制,制诏,制书。《秦始皇本纪》:“命为‘制’,令为‘诏’。”此诏用为动词,意为告令。}御史\footnote{御史大夫之省称。皇帝的诏令由御史大夫转知丞相、九卿执行。}:“诸侯王或欲推私恩\footnote{将自己的恩德推广及于他人。恩,恩德,恩惠。汉初诸侯王死后,由嫡子继承其土地爵位,而其他的儿子没有土地分封。汉武帝采纳主父偃的建议,令诸侯王把土地分封给其余众子为侯国,这样一个诸侯国被分为若干个小诸侯国,实力大为削弱。}分子弟邑者,令各条上,朕且临定\footnote{亲自确定。}其号名。”
\end{yuanwen}

皇帝发布命令诏告御史大夫说:“诸侯王有打算推恩于私亲而把封地分给子弟的,让他们各自把意见呈上,我将亲自制定这些子弟的封号。”

\begin{yuanwen}
太史公曰:盛哉,天子之德!一人有庆,天下赖之\footnote{语出《尚书·吕刑》:“一人有庆,兆民赖之。”一人,指天子。庆,善。赖,利,谓天下亦得其利。}。
\end{yuanwen}

太史公说:“天子的恩德,真伟大啊!皇帝有可贺之事,天下人都能承蒙恩泽。”
太史公说:“广阔兴盛呀,天子的德泽!一个人有善行,天下的人都获得了利益。”

\begin{yuanwen}
(表略)
\end{yuanwen}


\begin{yuanwen}
汉世之初,矫枉过正。欲大本枝,先封同姓。建元已后,籓翰克盛。主父上言,推恩下令。长沙济北,中山赵敬。分邑广封。振振在咏。扞城御侮,晔晔辉映。百足不僵,一人有庆。
\end{yuanwen}

\part{卷二十二}

\chapter{汉兴以来将相名臣年表第十}

本表起自高祖元年,终于成帝鸿嘉元年。相、将为国家安定之关键,内要靠相,外要靠将,对于巩固大汉王朝的天下有重要意义。本表所展现的就是汉兴百年间国家所经的大事及诸将相的变化情况,可看作是整个汉代社会历史的一个缩影。

\begin{yuanwen}
高祖初起,啸命群雄。天下未定,王我汉中。三杰既得,六奇献功。章邯已破,萧何筑宫。周勃厚重,朱虚至忠。陈平作相,条侯总戎。丙魏立志,汤尧饰躬。天汉之后,表述非功。
\end{yuanwen}

\part{卷二十三}

\chapter{礼书第一}

本篇介绍了礼仪制度的相关问题。崇尚礼乐是华夏民族有别于周边民族的显著特征,相传周公为礼乐的创制者。礼仪作为一种维系社会各阶层等级秩序的制度,在中华文明的发展进程中起到了相当重要的作用。春秋战国时期礼崩乐坏,孔子提出了恢复周礼的主张,并成为儒家思想的重要内容。汉武帝提倡儒术,受时代风尚的影响,司马迁将《礼书》列为八书之首。本篇大量引用先秦儒家典籍中的文字,司马迁在期间穿插议论,层层深入,反复强调礼仪对社会生活的重要作用。

\begin{yuanwen}
太史公曰:洋洋\footnote{众多、盛美的样子。}美德乎!宰制万物,役使群众,岂人力也哉?余至大行\footnote{秦官名,主管礼仪。}礼官,观三代损益,乃知缘人情而制礼,依人性而作仪,其所由来尚矣。
\end{yuanwen}

太史公说:盛大的美德啊!主宰制约着世间万物,驱使支配着天下民众,难道是人力所能决定的吗?我到大行礼官的官署,考察夏、商、周三代礼制增减的情况,才知道根据人之常情来制定礼数,依据人的本性来制定仪式,这种情况由来已久了。

\begin{yuanwen}
人道经纬万端,规矩无所不贯,诱进以仁义,束缚以刑罚,故德厚者位尊,禄重者宠荣,所以总一海内而整齐万民也。人体安驾乘,为之金舆\footnote{古代车由三大部分组成,上为车盖,中间的车身为舆,下为车轮。此处泛指车。}错衡以繁其饰;目好五色,为之黼\footnote{fǔ}黻\footnote{白与黑相交谓之黼,黑与青相交谓之黻。}文章\footnote{青与赤相交谓之文,赤与白相交谓之章。}以表其能;耳乐钟磬,为之调谐八音以荡其心;口甘五味,为之庶羞\footnote{泛指美味食品。羞,肉类食品。}酸咸以致其美;情好珍善,为之琢磨圭璧以通其意。故大路越席\footnote{蒲草席。},皮弁布裳\footnote{以白鹿皮做冠,白缯布为裳,是天子朝服之一。},硃弦洞越,大羹\footnote{又作“泰羹”,不加盐的肉汤。}玄酒\footnote{水。水色黑,太古时用于祭祀。},所以防其淫侈,救其彫敝。是以君臣朝廷尊卑贵贱之序,下及黎庶车舆衣服宫室饮食嫁娶丧祭之分,事有宜適,物有节文。仲尼曰:“禘\footnote{四时祭名。}自既灌\footnote{祭祀中的一个步骤,指以酒灌地。}而往者,吾不欲观之矣。”
\end{yuanwen}

为人之道千头万绪,规则礼法无处不在,用仁义来诱导人们上进,用刑罚来约束人们,所以道德高尚的人地位尊贵,俸禄丰厚的人备受尊宠,以此来统一四海而使民众步调一致。人的身体乘车马感到舒适,就会在车身上装饰黄金,在横木上嵌错花纹,加上繁琐的装饰;眼睛喜欢看五彩缤纷的颜色,就会在服装上刺绣花纹和图案,使外表形态更美好;耳朵乐于听钟磐演奏的音乐,就会调和八种乐器的音律,来激荡内心的情感;口舌愿意品尝美味佳肴,就会调制出各种可口的滋味,来各尽其美;性情喜好珍贵美好之物,就会雕琢打磨精致的玉器,用来顺应自己的心意。因此天子的大辂车用蒲草做席子,鹿皮冠帽配布衣裳,红弦的琴瑟底部有孔洞,用清肉汤和淡水酒祭祀,这是因为要防止过于奢侈,改正刻意铺张的弊端。因此上至君臣朝廷尊卑贵贱的秩序,下至平民百姓的车马、服饰、起居、饮食、嫁娶、丧祭等,都有严格的等级区分,凡事都有适当的标准,万物都有节度。孔子说:“鲁国举行的天子禘祭之礼,从以酒灌地后,我就不想看下去了。”

\begin{yuanwen}
周衰,礼废乐坏,大小相逾,管仲之家,兼备三归。循法守正者见侮于世,奢溢僭差者谓之显荣。自子夏,门人之高弟也,犹云“出见纷华盛丽而说,入闻夫子之道而乐,二者心战,未能自决”,而况中庸以下,渐渍于失教,被服于成俗乎?孔子曰“必也正名”,于卫所居不合。仲尼没后,受业之徒沈湮而不举,或适齐、楚,或入河海,岂不痛哉!
\end{yuanwen}

周朝衰落以后,礼仪荒废了,乐舞失传了,各等级相互僭越,以至管仲家中娶三姓妇女为妻。遵循法规、坚守正道的人被欺侮,奢侈淫乱、僭越等级被视为显贵荣耀。像子夏这样的孔门高徒尚且说“出门看见纷繁华丽之物而感到喜悦,回来听到老师讲授大道而感到快乐,这两种令人高兴之事在内心斗争,不能决定取舍”,何况那些资质中等以下的人,逐渐失去教化,长期被世俗所浸染呢?孔子说“一定要辨正名分”,他在卫国居住时看到了许多不合礼法的事。孔子死后,那些受业门人湮没在世俗中而不能振作,有的人前往齐国、楚国,有的人迁居黄河、东海之滨,怎么能不令人痛心呢!

\begin{yuanwen}
至秦有天下,悉内\footnote{通“纳”,收纳。}六国礼仪,采择其善,虽不合圣制,其尊君抑臣,朝廷济济,依古以来。至于高祖,光有四海,叔孙通颇有所增益减损,大抵皆袭秦故。自天子称号下至佐僚及宫室官名,少所变改。孝文即位,有司议欲定仪礼,孝文好道家之学,以为繁礼饰貌,无益于治,躬化\footnote{亲自施行。}谓何耳,故罢去之。孝景时,御史大夫晁错明于世务刑名,数干谏孝景曰:“诸侯籓辅,臣子一例,古今之制也。今大国专治异政,不禀京师,恐不可传后。”孝景用其计,而六国畔逆,以错首名,天子诛错以解难。事在《袁盎》语中。是后官者养交安禄而已,莫敢复议。
\end{yuanwen}

到秦始皇统一天下以后,全部接纳了东方六国的礼仪,选择其中好的方面加以采纳,虽然不合于古代圣王的制度,却尊崇君主、抑制臣僚,朝廷上下井然有序,依照了古代沿袭下来的礼仪。到了汉高祖时期,光复四海,叔孙通对礼仪稍加增加或删减,大体上都是沿袭的秦朝原有的礼制。从天子的称号到左右辅佐的官僚以及宫室和官职的名称,很少有改变的地方。孝文帝即位以后,有关部门商议要重新制定礼仪,那时孝文帝喜好道家无为学说,认为用繁琐的礼仪只能粉饰外表,对治理天下没有帮助,通过身体力行来教化别人有什么意义呢!因此废止了这件事。孝景帝时期,御史大夫晁错精通谋身治世之术和刑法名实之学,多次求见景帝进谏说:“诸侯王是天子屏障和辅助,与臣僚同列于朝廷,这是从古到今通用的制度。现在大的诸侯国割据一方,施行与朝廷相异的政策,不到京城汇报情况,恐怕不能再让这些封国继续传给诸侯王的后代了。”孝景帝采纳了他的计策,六个诸侯国反叛了朝廷,把诛杀晁错放在起兵理由的第一条,皇帝杀死了晁错来解除危难。这件事记载在《袁盎晁错列传》里。从此以后,做官的人只热衷于培养交情、安享俸禄罢了,不敢再议论礼制了。

\begin{yuanwen}
今上即位,招致儒术之士,令共定仪,十馀年不就。或言古者太平,万民和喜,瑞应辨至,乃采风俗,定制作。上闻之,制诏御史曰:“盖受命而王,各有所由兴,殊路而同归,谓因民而作,追俗为制也。议者咸称太古,百姓何望?汉亦一家之事,典法不传,谓子孙何?化\footnote{教化。}隆者闳博,治浅者褊狭,可不勉与!”乃以太初之元改正朔,易服色,封太山\footnote{即泰山。},定宗庙百官之仪,以为典常,垂之于后云。
\end{yuanwen}

当今皇上即位以后,招揽通晓儒学的士人,命令他们共同制定礼仪,十多年也没有做成。有的人说,古时候天下太平,万民和睦安乐,祥瑞应验之事层出不穷,于是采集各地风情民俗,制定礼仪。皇上听到这些话,诏令御史大夫说:“大凡接受天命而称王的人,各自都有兴起的原因,走的是不同的道路,结果却是一样的,是因为顺应民众的实际情况来制定礼仪,遵循民众的风俗习惯来建立制度。议论的人都说礼仪制度太久远了,今天的百姓还能看到什么呢?汉朝也是刘氏一家的事业,如果典章法令不能流传下去,又能对子孙后代说些什么呢?教化深远崇高的,对后世的影响也恢宏广博;治民目光浅薄的,在人们心目中的地位也偏颇狭隘,怎么能不以此自勉呢!”于是从太初元年起,改定历法,变更服饰颜色,在泰山举行封禅大典,确定宗庙祭祀和百官朝贺的礼仪,以此作为朝廷常用的典章制度,流传给后人作为典范。

\begin{yuanwen}
礼由人起。人生有欲,欲而不得则不能无忿,忿而无度量则争,争则乱。先王恶其乱,故制礼义以养人之欲,给人之求,使欲不穷于物,物不屈于欲,二者相待而长,是礼之所起也。故礼者养也。稻粱五味,所以养口也;椒\footnote{香料名。}兰芬茝\footnote{即白芷,一种香草。},所以养鼻也;钟鼓管弦,所以养耳也;刻镂文章,所以养目也;疏房床笫几席,所以养体也:故礼者养也。
\end{yuanwen}

礼制因人而产生。人们生来就有欲望,有欲望却得不到满足就不可能没有怨恨,怨恨毫无限度时就会引发争斗,有争斗就产生祸乱。古代圣王讨厌这种混乱的局面,所以制定礼仪道义来滋养人的欲望,满足人的需求,使欲望不会因为物质不足而受到限制,物质也不会因为欲望的增长而显得匮乏,两者相互协调而能长久,这就是礼制产生的原因。因此礼仪就是一种调养之法。稻米、高粱、五味食品,是用来调养口舌之欲的;香料、鲜花、芳草,是用来调养嗅觉之欲的;钟鼓、管弦演奏的音乐,是用来调养听觉之欲的;精雕细刻的花纹,是用来调养视觉之欲的;宽敞的房间和床榻、几案、坐席,是用来调养身体之欲的:所以礼仪就是一种调养之法。

\begin{yuanwen}
君子既得其养,又好其辨也。所谓辨者,贵贱有等,长少有差,贫富轻重皆有称也。故天子大路越席,所以养体也;侧载臭\footnote{气味,这里指香气。}茝,所以养鼻也;前有错衡,所以养目也;和鸾\footnote{车马铃。}之声,步中《武》、《象》,骤中《韶》、《濩》,所以养耳也;龙(旂/旗)九斿\footnote{同“旒”。古代旌旗边缘上悬垂的装饰品。},所以养信也;寝兕\footnote{以犀牛皮为席。}持虎\footnote{以猛兽皮文饰倚较及伏轼。},鲛韅弥龙\footnote{复盖着龙纹。},所以养威也。故大路之马,必信至教顺,然后乘之,所以养安也。孰知夫出死要节\footnote{希求名节。}之所以养生也。孰知夫轻费用之所以养财也,孰知夫恭敬辞让之所以养安也,孰知夫礼义文理\footnote{即礼。文是礼之饰,理是礼之质。}之所以养情也。
\end{yuanwen}

君子的欲望已经得到了调养,又愿意受礼仪区别的制约。所谓区别,是指高贵的人和低贱的人要讲等级,年长的人和年幼的人要有差别,贫穷的人、富有的人、卑贱的人、尊贵的人都有与身分相称的待遇。所以天子乘大辂车用蒲草做席子,用来调养身体之欲的;车子两侧摆放香草,用来调养嗅觉之欲;车子前面有嵌错花纹的横木,用来调养视觉之欲;和铃与鸾铃的声响,慢走时与《武》《象》的节奏相符,快走时与《韶》《濩》的节奏相符,用来调养听觉之欲;龙旗的边缘装饰有九旒,用来培养天子诚信之心;车轮装饰着伏卧犀牛的形状,伏轼上装饰有虎皮,马肚带以鲨鱼皮制成,车轭雕刻成龙的样子,用来培养天子威严之气。所以大辂车的马匹,一定驯化到温顺的程度,然后才可以驾驭它,用来保证天子人身安全。谁能知道人们出生入死追求名节正是为了保养身体呢?哪个知道人轻视费用就是为了聚集财富呢?哪个知道恭敬辞让就是为了保证平安呢?谁知道学习礼仪、仁义、文章、道理正是为了调养性情呢?

\begin{yuanwen}
人苟生之为见,若者必死;苟利之为见,若者必害;怠惰之为安,若者必危;情胜之为安,若者必灭。故圣人一之于礼义,则两得之矣;一之于情性,则两失之矣。故儒者将使人两得之者也,墨者将使人两失之者也。是儒墨之分。
\end{yuanwen}

人以苟且偷生为目的,就必定死亡;以苟且获利为追求,就必定受害;以懒惰懈怠为安逸,就必定危险;以冲动固执为安分,就必定毁灭。所以圣人用礼仪和道义来规范一切,那么趋利避害两方面就都得到了;若一概任情尽性,那么两方面就都会失掉。所以儒家要使人两方面都得到,墨家要使人两方面都失掉。这就是儒、墨两家的区别。

\begin{yuanwen}
治辨之极也,彊固之本也,威行之道也,功名之总也。王公由之,所以一天下,臣诸侯也;弗由之,所以捐社稷也。故坚革利兵不足以为胜,高城深池不足以为固,严令繁刑不足以为威。由其道则行,不由其道则废。楚人鲛革犀兕,所以为甲,坚如金石;宛之钜铁施\footnote{鍦,即矛。},钻如蜂虿\footnote{蜂虿之尾。虿,蝎类毒虫。},轻利剽遬\footnote{剽悍迅速。遬,同“速”。},卒如熛风\footnote{疾速如飞。}。然而兵殆于垂涉,唐昧死焉;庄蹻起,楚分而为四参。是岂无坚革利兵哉?其所以统之者非其道故也。汝颍以为险,江汉以为池,阻之以邓林,缘之以方城。然而秦师至鄢郢,举若振槁。是岂无固塞险阻哉?其所以统之者非其道故也。纣剖比干,囚箕子,为砲格,刑杀无辜,时臣下懔然,莫必其命。然而周师至,而令不行乎下,不能用其民。是岂令不严,刑不(鷟/陖)\footnote{同“峻”。}哉?其所以统之者非其道故也。
\end{yuanwen}

礼是圣人治理天下的最高准则,国家强盛巩固的根本方式,君主威严顺利施行的最佳途径,人们成就功业声名的总体纲要。王公遵循礼治,就可以凭借它统一天下,使诸侯臣服;不遵循礼治,就会失掉国家。所以坚硬的铠甲和锋利的兵器不能保证取得胜利,高耸的城墙和深深的护城河不能保证防守坚固,严苛的法令和繁杂的刑罚不能保证树立权威。遵循这一道理就能通行无阻,不遵循这一道理就将失败。楚国人用鲨鱼皮、犀牛皮制作铠甲,坚硬得如同金属和石头;用宛城的大铁矛进攻,钻刺时就像蜜蜂和蝎子的尾巴一样,锋利而飘忽,士兵像暴风一样迅猛。然而楚国的军队兵败垂涉,大将唐昧死在了那里;庄蹻起兵,楚国分裂成四国。这难道是因为没有坚硬的铠甲、锋利的武器吗?是由于统治手段没有遵循礼治的缘故。楚国以汝水和颍水为天险,以长江和汉水为护城河,以邓林阻挡敌人,以长城为疆界。然而秦国的军队到达了鄢、郢二都,发兵打仗就像摇动枯树那样容易。这难道是因为没有坚固的要塞和险要的阻隘吗?是由于统治手段没有遵循礼治的缘故。商纣王剖了比干的心,囚禁了箕子,创制了炮烙的酷刑,用来杀害无辜的人,当时的臣民战战兢兢,没有一个人能确保自己的性命。然而周国的军队一到,商纣王的命令在下面就无法通行了,他也无法驱使民众为其效力了。这难道是因为政令不严,刑罚不重吗?是由于统治没有礼治之道的缘故。

\begin{yuanwen}
古者之兵,戈矛弓矢而已,然而敌国不待试而诎。城郭不集,沟池不掘,固塞不树,机变不张,然而国晏然不畏外而固者,无他故焉,明道而均分之,时使\footnote{役使以时,如同说使民以时。}而诚爱之,则下应之如景响。有不由命者,然后俟之以刑,则民知罪矣。故刑一人而天下服。罪人不尤其上,知罪之在己也。是故刑罚省而威行如流,无他故焉,由其道故也。故由其道则行,不由其道则废。古者帝尧之治天下也,盖杀一人刑二人而天下治。传曰:“威厉而不试,刑措而不用”。
\end{yuanwen}

古代的兵器,不过是戈、矛、弓、箭这几种罢了,然而敌国却不等尝试使用时就屈服了。城郭不需要集结兵力戍守,护城河不需要挖掘加深,坚固的要塞不需要增建,智谋权变不需要采用,然而国家却平安无事不畏惧外敌而且防守稳固,在这里没有其他缘故,只因为能明白礼义之道并且能公平地处理人际关系,按时令役使民众并且能真心地爱护他们,那么臣民就会像影子一样跟随他。有不服从命令的人,然后再用刑罚对待他,那么民众就知道哪些是犯罪行为了。所以处罚一个人而能使天下的人都顺服。犯罪的人不责怪官长,是因为他们知道错在自己。因此刑罚少而权威仍然像流水一样通达于臣下,在这里没有其他缘故,只因为是遵循礼治之道的缘故。所以遵循这一道理就通行,不遵循这一道理就失败。古代帝王尧在治理天下时,大概只是杀了一个人,处罚了两个人,天下就安定太平了。书上说:“威严虽然猛厉但是不可轻易采用,刑罚虽然设置但是不可轻易施行。”

\begin{yuanwen}
天地者,生之本也;先祖者,类之本也;君师者,治之本也。无天地恶生?无先祖恶出?无君师恶治?三者偏亡,则无安人。故礼,上事天,下事地,尊先祖而隆君师,是礼之三本也。
\end{yuanwen}

苍天大地,是生命的本源;先人祖宗,是家族的本源;君主师长,是国家安定的本源。没有苍天大地怎么能有生命?没有先人祖宗怎么能有后代?没有君主师长怎么能有国家的安定?这三个方面偏废了,就没有安居乐业之人了。因此礼仪就是对上敬奉苍天,对下敬奉大地,同时尊敬先人祖宗并且尊崇君主师长,这是礼仪的三项根本原则。

\begin{yuanwen}
故王者天太祖,诸侯不敢怀,大夫士有常宗\footnote{《礼记·丧服小记》说:“别子为祖,继别为宗。”大夫、士虽各自为祖,但自有常宗,不能越宗而祭。},所以辨贵贱。贵贱治,得之本也。郊畴乎天子,社至乎诸侯,函及士大夫,所以辨尊者事尊,卑者事卑,宜钜者钜,宜小者小。故有天下者事七世,有一国者事五世,有五乘之地者事三世,有三乘之地者事二世,有特牲而食者\footnote{特牲即牛,有一牛而耕食者为庶人。}不得立宗庙,所以辨积厚者流泽广,积薄者流泽狭也。
\end{yuanwen}

所以称王之人把上天与始祖相配加以祭祀,诸侯不敢怀有这个想法,大夫和士也各自有支脉的先祖,这样是为了区分高贵与卑贱。高贵的人和卑贱的人得以安定,这样就得到了礼治的根本。郊外祭天的仪式只有天子才有资格主持,土地神的祭祀由诸侯主持,也包括士大夫在内,这样是为了规定尊贵的人敬奉尊贵的神灵,卑贱的人敬奉卑贱的神灵,适合做大事的人去做大事,适合做小事的人去做小事。所以统治天下的天子可以供奉七代祖先的牌位,统治一国的诸侯可以供奉五代祖先的牌位,拥有五乘土地的大夫可以供奉三代祖先的牌位,拥有三乘士地的士人可以供奉两代祖先的牌位,普通百姓不允许设立宗庙,这样是为了区别道德高深的人恩泽流布广阔,道德浅薄的人恩泽流布狭窄。

\begin{yuanwen}
大飨上玄尊\footnote{盛玄酒的酒樽,又称水尊,这里指玄酒。},俎\footnote{木制有足的器皿,祭祀时盛祭品。}上腥鱼,先大羹,贵食饮之本也。大飨上玄尊而用薄酒,食先黍稷而饭稻粱,祭哜\footnote{品尝。}先大羹而饱庶羞,贵本而亲用也。贵本之谓文,亲用之谓理,两者合而成文,以归太一,是谓大隆。故尊之上玄尊也,俎之上腥鱼也,豆之先大羹,一也。利爵弗啐也,成事俎弗尝也,三侑之弗食也,大昏\footnote{同“婚”。}之未废齐\footnote{祭神。废,通“发”。齐,同“斋”。}也,大庙之未内尸也,始绝之未小敛\footnote{同“殓”。},一也。大路之素帱也,郊之麻絻,丧服之先散麻,一也。三年哭之不反也,《清庙》之歌一倡而三叹,县一钟尚拊膈,(硃/朱)弦而通越,一也。
\end{yuanwen}

大祭祀飨神,崇尚玄水,俎实崇尚生鱼,先进献清肉汤,表示重视饮食的本源。大祭祀飨神崇尚玄酒而饮用薄酒,食物崇尚黄米和小米而吃稻米和高粱做成的米饭,祭尸时先上清肉汤而饱腹的是美味佳肴,表示既重视饮食的本源又兼顾实际效用。重视本源是为了展现祭祀的形式,就像事物的外在纹饰;兼顾实用是为了表达祭祀的理念,就像事物的内在脉络,这两点相互结合而构成礼仪的外在形式,归结为天地的本源,这就是最隆重的典礼。因此樽酒崇尚玄酒,俎实崇尚上生鱼,豆羹崇尚不加调料的清肉汤,道理是一样的。佐食的人不小口喝酒,一饮而尽;卒哭之祭有献无酢,参加祭祀的人除尸之外,不品尝俎实;祝与佐食的人劝尸用饭,三劝之后,礼数已成,尸停止用饭,婚礼在祭神之前,祭祀时迎尸进太庙之前,亲人刚去世还没有入殓之前,道理是一样的。大辂车搭配白丝帷帐,郊外祭天时戴麻布帽子,穿着丧服时腰间散扎着麻带,道理是一样的。三年守孝期间痛哭不注重腔调,《清庙》之歌由一个人领唱并且有三个人咏叹,敲打悬挂的大钟时会撞击木制的钟格,装有红色丝弦的琴瑟下有孔洞,道理是一样的。

\begin{yuanwen}
凡礼始乎脱,成乎文,终乎税\footnote{有所舍弃。}。故至备,情文俱尽;其次,情文代胜;其下,复情以归太一。天地以合,日月以明,四时以序,星辰以行,江河以流,万物以昌,好恶以节,喜怒以当。以为下则顺,以为上则明。
\end{yuanwen}

大凡礼仪都是从疏脱开始,加上修饰而完成,最后有所取舍地流传下去。所以最完备的礼仪,应该是内在情感和外在形式全都具备;其次,是内在情感与外在形式交替占上风;再次,是内在情感完全归结为天地的本源。天地根据礼仪之道相互协调,日月根据礼仪之道彰显光明,四季根据礼仪之道井然有序,星辰根据礼仪之道周行不殆,江河根据礼仪之道奔流不止,万物根据礼仪之道繁荣昌盛,好恶根据礼仪之道克制调节,喜怒根据礼仪之道处置恰当。用礼仪治理臣下就会和顺,用礼仪规范君上就会英明。

\begin{yuanwen}
太史公曰:至矣哉!立隆以为极,而天下莫之能益损也。本末相顺,终始相应,至文有以辨\footnote{指能辨别贵贱等级尊卑。},至察有以说。天下从之者治,不从者乱;从之者安,不从者危。小人不能则也。
\end{yuanwen}

太史公说:真完美啊!创立完备的礼仪作为调和人际关系的最高准则,而且天下没有人能够增加或删减。礼仪的根本与分支相互承接,结尾与开始相互照应,文采飞扬又能彰显差别,明察秋毫又能使人和悦。天下人遵从礼治就太平,不遵从礼治就混乱;遵从礼治就安定,不遵从礼治就危险。平民百姓无法以此为法则。

\begin{yuanwen}
礼之貌诚深矣,坚白同异\footnote{战国时辩论的两个著名命题。坚白论认为坚石非石,白马非马。同异论认为万物毕同毕异。}之察,入焉而弱。其貌诚大矣,擅作典制褊\footnote{biǎn}陋之说,入焉而望。其貌诚高矣,暴慢恣睢,轻俗以为高之属,入焉而队\footnote{同“坠”。}。故绳诚陈,则不可欺以曲直;衡诚县,则不可欺以轻重;规矩诚错,则不可欺以方员;君子审礼,则不可欺以诈伪。故绳者,直之至也;衡者,平之至也;规矩者,方员之至也;礼者,人道之极也。然而不法礼者不足礼,谓之无方之民;法礼足礼,谓之有方之士。礼之中,能思索,谓之能虑;能虑勿易,谓之能固。能虑能固,加好之焉,圣矣。天者,高之极也;地者,下之极也;日月者,明之极也;无穷者,广大之极也;圣人者,道之极也。
\end{yuanwen}

礼仪看上去确实很深奥了,关于“坚白”与“同异”的辩论,进入礼仪的范畴就显得苍白无力了。它看上去确实很宏大了,擅自创设典章制度、发表偏激浅陋言论的人,进入礼仪的范畴就只能望而却步了。它看上去确实很高远了,那些残暴、傲慢、放纵而以蔑视礼俗为了不起的人,进入礼仪的范畴就变得渺小卑微了。所以墨绳画好以后,就不能再以曲直相欺骗了;秤杆持平以后,就不能再以轻重相欺骗了;圆规直尺摆好位置以后,就不能再以方圆相欺骗了;君子认真地审视礼法,就不能再以诈伪相欺骗了。所以墨绳是直线的最高标准;秤杆是轻重的最高标准;圆规直尺是方圆的最高标准;礼仪是为人处事的最高标准。然而那些不遵从礼仪的人不值得以礼相待,称之为不法之民;遵从礼仪的人应该以礼相待,称之为守法之士。做事合于礼法,能够思考探究,称之为勤于思考;能考虑问题而不随便改变观点,称之为持之以恒。既能考虑问题又能坚持道义,再加上内心真正喜欢这样做,那就是最伟大的圣人了。苍天,是高远的极点;大地,是低下的极点;日月,是光明的极点;无边的宇宙,是广大的极点;圣人,是遵循礼仪之道的极点。

\begin{yuanwen}
以财物为用,以贵贱为文,以多少为异,以隆杀\footnote{隆盛、减杀。}为要。文貌繁,情欲省,礼之隆也;文貌省,情欲繁,礼之杀也;文貌情欲相为内外表里,并行而杂,礼之中流也。君子上致其隆,下尽其杀,而中处其中。步骤驰骋广骛不外,是以君子之性守宫庭\footnote{这里是中的意思,就是常居中道,不偏不倚。}也。人域是域,士君子也。外是,民也。于是中焉,房皇周浃,曲得其次序,圣人也。故厚者,礼之积也;大者,礼之广也;高者,礼之隆也;明者,礼之尽也。
\end{yuanwen}

礼仪的应用表现在财物的使用上,文彩表现在贵贱的等级上,差异表现在礼节的繁简上,要领表现在礼遇的厚薄上。外在文彩看上去纷繁,性情欲望能够有所收敛,这是礼仪隆重的表现;外在文彩看上去简约,性情欲望却十分繁盛,这是礼仪消糜的表现;外在文彩和性情欲望互为里表,二者有机地结合在一起,这就是最合乎礼仪的表现。君子向上能以大礼彰显隆重的威仪,向下能以小礼展现简约的本性,又能以中礼做得恰到好处。轻重缓急都不超出礼仪的范畴,所以君子的本性就是持守中庸之道。人生在世做事不超越礼义的界限,这就是兼具学识与道德的士君子。做事超出礼义的界限,这就是普通人了。在这些人中间,徘徊周旋,对错都能依照礼仪的程序去做,这就是圣人。所以道德高尚的人,是积累礼义的结果;人格伟大的人,是广施礼义的结果;地位崇高的人,是重视礼义的结果;精明智慧的人,是严格按照礼义做事的结果。

\begin{yuanwen}
礼因人心,非从天下。合诚饰貌,救弊兴雅。以制黎甿,以事宗社。情文可重,丰杀难假。仲尼坐树,孙通蕝野。圣人作教,罔不由者。
\end{yuanwen}

\part{卷二十四}
\chapter{乐书第二}

本篇介绍了大量古代音乐知识,也保存了大量先秦时期关于音乐的论述。本篇论点是“乐者乐也”,即音乐就是快乐,同时又提出不可以听淫靡的音乐,所以听音乐的目的是为了快乐,但不单纯是为了享乐。篇中大量引用了《礼记·乐记》中的文字,也有后人增补的文字,所以总体上显得有些杂乱。乐是礼乐文明的重要内容,也是儒家推行教化的重要手段,所以《乐书》被列于八书第二篇。

\begin{yuanwen}
太史公曰:余每读《虞书》,至于君臣相敕\footnote{告诫,勉励。},维是几安,而股肱不良,万事堕\footnote{huī,同“隳”,毁坏。}坏,未尝不流涕也。成王作颂,推己惩艾\footnote{通“刈”,创伤、苦痛。},悲彼家难\footnote{国家的难事。这里指文王囚羑里,武王伐纣。},可不谓战战恐惧,善守善终哉?君子不为约则修德,满则弃礼,佚\footnote{通“逸”,安逸。}能思初,安能惟始,沐浴膏泽\footnote{肉之肥者为膏。泽亦指膏。}而歌咏勤苦,非大德谁能如斯!传曰“治定功成,礼乐乃兴”。海内人道益深,其德益至,所乐者益异。满而不损则溢,盈而不持则倾。凡作乐者,所以节乐。君子以谦退为礼,以损减为乐,乐其如此也。以为州异国殊,情习不同,故博采风俗,协比声律,以补短移化,助流政教。天子躬于明堂临观,而万民咸荡涤邪秽,斟酌饱满,以饰厥性。故云《雅》、《颂》之音理而民正,嘄噭之声兴而士奋,郑卫之曲动而心淫。及其调和谐合,鸟兽尽感,而况怀五常\footnote{人的五种常行,指父义、母慈、兄友、弟恭、子孝。},含好恶,自然之势也?
\end{yuanwen}

太史公说:我每次读《虞书》,看到描述君臣相互告诫,因此天下稍微得到安定,然而左右的辅佐之臣不够贤良,所有的事业都毁坏的记载时,没有一次不感动得落泪。周成王作《周颂》,推究自己所受的创伤,为国家所遭遇的祸难而感到悲哀,怎可说他不是小心谨慎、善始善终的帝王呢?君子不因为穷困而修养道德,志得意满就背弃礼义,应该在安逸的时候想到当初创业的艰难,在安定的时候想到创始时的艰难,沐浴在富裕之中应该歌颂勤奋的品质,不是具有崇高道德的人谁能像这样啊!书上说“统治安定大功告成,礼乐制度于是产生”。天下为人之道越深奥,帝王的品德越崇高,他所认为欢乐的事情就越不同于常人。水满了却不减少就会流出来,时机成熟了而不能把握就会跌倒。大凡创作音乐的原因,是为了节制欢乐。君子把谦虚退让视为礼节,把削减欲望视为乐事,音乐产生的意义大概就像这样。由于地域不同,人的性情习俗也不同,所以广博地采集各地民风习俗,与声律相协调,用来弥补不足和移风易俗,帮助推行政令与教化。天子亲自在明堂观看乐舞,可以使民众洗清内心的邪恶与污秽,调和饱满无缺的人性,用来整饬其性情。所以说《雅》《颂》的音乐奏起,民众就会持守正道;高亢嘹亮的音乐演奏起来,战士就会振奋精神;郑国、卫国的靡靡之音响起,人们就意乱情迷了。等到音调和谐以后演奏,飞鸟走兽也都会为之动容,更何况是心怀五常、明辨爱憎的人了,这就是自然之势吧?

\begin{yuanwen}
治道亏缺而郑音兴起,封君世辟,名显邻州,争以相高。自仲尼不能与齐优遂容于鲁,虽退正乐以诱世,作五章以剌时,犹莫之化。陵迟\footnote{迟缓的样子。}以至六国,流沔沈佚,遂往不返,卒于丧身灭宗,并国于秦。
\end{yuanwen}

治国之道匮乏,郑国淫邪的音乐就流行起来,封国之君世袭之主,在邻近州郡名声显扬,却争着欣赏互比高低。自从孔子无法与齐国俳优在鲁国并存,即使隐退后整理雅正的音乐来诱导世人,创作五章来讥刺时政,还是没有办法感化世人。慢慢地到了六国时代,人们随波逐流而沉湎于逸乐,于是一去不回头了,最后自身败亡,宗庙被毁,国家被秦国吞并了。

\begin{yuanwen}
秦二世尤以为娱。丞相李斯进谏曰:“放弃《诗》、《书》,极意声色,祖伊所以惧也;轻积细过,恣心长夜,纣所以亡也。”

赵高曰:“五帝、三王乐各殊名,示不相袭。上自朝廷,下至人民,得以接欢喜,合殷勤,非此和说不通,解泽不流,亦各一世之化,度时之乐,何必华山之騄耳\footnote{古代骏马名,周穆王八骏之一。}而后行远乎?”二世然之。
\end{yuanwen}

秦二世更是把音乐当成娱乐。丞相李斯进谏说:“放弃《诗》、《书》,把心思花在享受声色上,这正是古代贤臣祖伊担心的事情;忽视细小过错的积累,在漫漫长夜里放纵心意,这正是暴君纣王灭亡的原因。”

赵高说:“五帝和夏、商、周的开国之君各自崇尚不同的音乐,用来表示礼法不相承袭。上自朝廷,下至人民,能够相处得见面时欢欣喜悦,相处得情意深厚,除了这种音乐就不能使和谐愉悦的感情得以沟通,也不能使上级布施的恩泽得以流行,这也都是各自时代所具有的风尚,王者会根据时代的不同来选择音乐,为什么一定要有华山的騄耳宝马然后才去走远路呢?”秦二世认为他说得很对。

\begin{yuanwen}
高祖过沛诗《三侯之章》,令小儿歌之。高祖崩,令沛得以四时歌(鳷/舞)宗庙。孝惠、孝文、孝景无所增更,于乐府\footnote{古代管理音乐的官署。}习常肄旧而已。
\end{yuanwen}

高祖路过故乡沛县时作了《三侯之章》的诗歌,让小孩学着唱。高祖逝世时,朝廷命令沛县一年四季都要在祭祀宗庙时以此诗为歌舞乐曲。孝惠帝、孝文帝、孝景帝在位时都没有任何增加和更改,在乐府里也只是像往常一样演练罢了。

\begin{yuanwen}
至今上即位,作《十九章》,令侍中李延年次序其声,拜为协律都尉。通一经之士不能独知其辞,皆集会五经家,相与共讲习读之,乃能通知其意,多《尔雅》之文。
\end{yuanwen}

到当今皇上即位以后,作《郊祀歌十九章》,让侍中李延年按次序谱曲,任命他为协律都尉。仅通晓一部经书的学者不能独自解释诗歌的含意,朝廷把通晓五经的学者全部召集起来,相互研究、诵读,才能明白其中的意思,其中大多是出自《尔雅》的文字。

\begin{yuanwen}
汉家常以正月上辛\footnote{每月中的第一个辛日。}祠太一甘泉,以昏时夜祠,到明而终。常有流星经于祠坛上。使僮男僮女七十人俱歌。春歌《青阳》,夏歌《(朱/硃)明》,秋歌《西(暤/昚)》,冬歌《玄冥》。世多有,故不论。
\end{yuanwen}

汉朝通常在正月上旬辛日在甘泉宫祭祀太一之神,从黄昏时分起在夜间开始祭祀,到天亮以后结束,祭祀的时候经常有流星划过祭坛的上空。安排童男童女七十人一起唱歌。春天唱《青阳》,夏天唱《朱明》,秋天唱《西暤》,冬天唱《玄冥》。这些诗歌世间多有,所以不在这里讨论。

\begin{yuanwen}
又尝得神马渥洼水中,复次以为《太一之歌》。(歌)曲曰:“太一贡兮天马下,(霑/沾)赤汗兮沫\footnote{洗脸。}流赭。骋容与兮跇万里\footnote{超越,跨越。},今安匹兮龙为友。”

后伐大宛得千里马,马名蒲梢,次作以为歌。歌诗曰:“天马来兮从西极,经万里兮归有德。承灵威兮降外国,涉流沙兮四夷服。”

中尉汲黯进曰:“凡王者作乐,上以承祖宗,下以化兆民。今陛下得马,诗以为歌,协于宗庙,先帝百姓岂能知其音邪?”

上默然不说。丞相公孙弘曰:“黯诽谤圣制,当族\footnote{刑罚名,即灭族之刑。}。”
\end{yuanwen}

皇上曾在渥洼水中捕获一匹神马,又配曲为《太一之歌》。歌词说:“太一神的恩赐啊,天马降临人间,身上沾满红色的汗啊,嘴里流出褐色的唾沫。从容驰骋啊,一跃过万里,现在什么能与之匹敌啊,龙是它的朋友。”

后来征伐大宛得到了千里马,马名叫“蒲梢”,配曲成为歌。歌词说:“天马来啊,从遥远的西方,经历万里啊,归顺有德的皇帝。承蒙神威啊,降服外国,跨越流沙啊,四方蛮夷宾服。”

中尉汲黯进谏说:“大凡称王的人创作音乐,对上用来继承祖宗的恩德,对下用来教化亿万臣民。现在陛下得到几匹马,作诗用来歌唱,并且用于宗庙祭祀,先帝和百姓怎么能理解这种音乐呢?”

皇上沉默不高兴。丞相公孙弘说:“汲黯诽谤皇上的作品,应当处以灭族之刑。”

\begin{yuanwen}
凡音之起,由人心生也。人心之动,物使之然也。感于物而动,故形于声;声相应,故生变;变成方,谓之音;比\footnote{随着、顺着。}音而乐之,及干戚羽旄\footnote{干指盾牌,戚指斧,二者是周武王所制《武》舞中舞人所执的器具;羽指雄性山鸡尾,旄指旄牛尾,二者是文舞中舞人所执的器具。},谓之乐也。乐者,音之所由生也,其本在人心感于物也。是故其哀心感者,其声噍以杀\footnote{促急而迅速减弱。};其乐心感者,其声啴以缓\footnote{宽缓。};其喜心感者,其声发以散;其怒心感者,其声粗以厉;其敬心感者,其声直以廉\footnote{边、角分明,绝无邪曲的意思。};其爱心感者,其声和以柔。六者非性也,感于物而后动,是故先王慎所以感之。故礼以导其志,乐以和其声,政以壹其行,刑以防其奸。礼乐刑政,其极一也,所以同民心而出治道也。
\end{yuanwen}

大凡音律的产生,来自于人心的变动。人心的变动,是外物造成的。人心有感于物而变动,所以形成声音;声音相互呼应,所以产生变化;变化形成一定的规律,就叫音律;编排音律用来娱乐,再借助武器、旗帜来表演,就叫音乐。音乐,是由音律生成的,它的根源在于人心有感于物。所以被物所感而生哀痛心情时,其声急促而且迅速减弱;心生欢乐时,其声舒慢而宽缓;心生喜悦时,其声发扬而且轻散;心生愤怒时,其声粗猛严厉;心生敬意时,其声正直清亮;心生爱意时,其声柔和动听。这六种声音并不发自人的本性,而是内心有感于物后产生的,因此古代圣王对外物的影响格外慎重。所以礼仪用来引导人们的志趣,音乐用来调和人们的声音,政令用来统一人们的行为,刑罚用来防范人们的奸邪。礼仪、音乐、刑罚、政令,它们的终极目标是一样的,都是用来统一民众的思想,从而实现天下大治。

\begin{yuanwen}
凡音者,生人心者也。情动于中,故形于声,声成文谓之音。是故治世之音安以乐,其正和;乱世之音怨以怒,其正乖;亡国之音哀以思,其民困。声音之道,与正通矣。宫为君,商为臣,角为民,徵为事,羽为物。五者不乱,则无(怗/惉)懘之音矣。宫乱则荒,其君骄;商乱则(搥/捶),其臣坏;角乱则忧,其民怨;徵乱则哀,其事勤;羽乱则危,其财匮。五者皆乱,迭相陵,谓之慢。如此则国之灭亡无日矣。郑卫之音,乱世之音也,比于慢矣。桑间濮上之音,亡国之音也,其政散,其民流,诬上行私而不可止。
\end{yuanwen}

张守节:「天有日月星辰,地有山陵河海,岁有万物成熟,国有圣贤宫观周域官僚,人有言语衣服体貌端修,咸谓之乐。」

大凡音律,是从人的内心中产生的。感情在内心震荡,因此形成声音,声音符合一定的韵律就叫音律。所以太平之世的音乐充满了安逸和快乐,其政治和谐;动荡之世的音乐充满了怨恨和愤怒,其政治混乱;将亡之国的音乐充满了悲哀和忧愁,其民众困顿。声乐音律的道理,和政治是相通的。宫声好比君主,商声好比大臣,角声好比人民,徵声好比政事,羽声好比器物。宫、商、角、徵、羽五声不紊乱,就不会出现不和谐的音调。宫声紊乱音律就荒废,那么君主必定骄纵;商声紊乱音律就邪僻,那么大臣必定败坏;角声紊乱音律就忧郁,那么人民必定怨恨;徵声紊乱音律就悲哀,那么政事必定繁重;羽声紊乱音律就高危,那么财货必定匮乏。五声全都紊乱,相互交替着占据主旋律,就叫无礼。像这样的话,国家的灭亡就时日无多了。郑国、卫国的淫邪之音,是动荡之世的音乐,接近于无礼的地步了。桑间、濮上的靡靡之音,是将亡之国的音乐,其政治涣散,其百姓流亡,下级欺骗上级,徇私舞弊的行为不可制止。

\begin{yuanwen}
凡音者,生于人心者也;乐者,通于伦理者也。是故知声而不知音者,禽兽是也;知音而不知乐者,众庶\footnote{普通百姓。}是也。唯君子为能知乐。是故审声以知音,审音以知乐,审乐以知政,而治道备矣。是故不知声者不可与言音,不知音者不可与言乐。知乐则几于礼矣。礼乐皆得,谓之有德。德者得也。是故乐之隆,非极音也;食飨\footnote{祭祀结束后把飨神的食物供宾客享用。飨,通“享”。}之礼,非极味也。清庙之瑟,硃弦而疏越,一倡而三叹,有遗音者矣。大飨之礼,尚玄酒而俎腥鱼,大羹\footnote{肉汁羹。}不和,有遗味者矣。是故先王之制礼乐也,非以极口腹耳目之欲也,将以教民平好恶而反人道之正也。
\end{yuanwen}

大凡音律,是从人的内心中形成的;音乐,与伦理道德是相通的。因此只知道听声音而不知道调和音律的,是禽兽;只知道听音律而不知道创作音乐的,是庶民。只有君子能通晓音乐。因此通过审察声音来理解音律,通过审察音律来理解音乐,通过审察音乐来理解政治,这样治国的方法就完备了。所以不知道声音的不可以跟他讲音律,不知道音律的不可以跟他讲音乐,理解音乐就接近于通晓礼仪了。礼仪和音乐都具备,就叫有德。仁德就是获得。因此音乐的隆盛,不在于把奏乐的规模发挥到极致;宴享礼的隆盛,不在于把食物的美味发挥到极致。演奏《清庙》的瑟,外表是红色的丝弦,底部有两个通气洞,一个人领唱三个人咏叹,音乐得以流传百世。大飨的礼仪崇尚玄酒,俎板上盛放生鱼,肉汤不加五味调料,原味得以流传百世。所以古代圣王创制礼仪和音乐,并不是为了满足口腹耳目之欲的,是要用来教导民众判断善恶的,从而返回为人的正道上来。

\begin{yuanwen}
人生而静,天之性也;感于物而动,性之颂\footnote{容貌、外表。}也。物至知\footnote{同“智”,智慧。}知,然后好恶形焉。好恶无节于内,知诱于外,不能反己,天理灭矣。夫物之感人无穷,而人之好恶无节,则是物至而人化物也。人化物也者,灭天理而穷人欲者也。于是有悖逆诈伪之心,有淫佚作乱之事。是故彊者胁弱,众者暴寡,知者诈愚,勇者苦怯,疾病不养,老幼孤寡不得其所,此大乱之道也。是故先王制礼乐,人为之节:衰麻\footnote{丧服。此处指有关衰麻的礼仪制度。}哭泣,所以节丧纪也;钟鼓干戚,所以和安乐也;婚姻冠笄,所以别男女也;射乡食飨,所以正交接也。礼节民心,乐和民声,政以行之,刑以防之。礼乐刑政四达而不悖,则王道备矣。
\end{yuanwen}

人生下来的时候是宁静的,这是上天赋予的本性;内心受到外物刺激而有所振动,这是本性的外在表现。外物出现而心智感知,然后喜好和厌恶的情感就会表现出来。喜好和厌恶的情感在内心毫无节制,心智就会被外物诱惑,不能回归自己最初的心境,理性就随之泯灭了。外物对人的感动是没有穷尽的,而人的喜好和厌恶毫无节制,那么外物出现人就会被同化。人被外物同化,就会通过泯灭天理来满足个人欲望。于是就产生了悖逆和诈伪的念头,也会做出淫靡佚乐和犯上作乱的事情。因此强大的胁迫弱小的,众多的欺凌寡少的,聪明的诈骗愚昧的,勇猛的摧残怯懦的,生病的人得不到照顾,老幼孤寡得不到应有的安置,这是导致天下大乱的做法。所以古代圣王创制礼乐,人们因为它而节制欲望。表达哀思时披麻哭泣,是为了节制丧事;敲打钟鼓、挥动盾牌和斧头的舞蹈,是为了调和安乐;婚姻之事和成人仪式上的冠笄之礼,是为了区别男女;乡射和宴饮之礼,是为了规范社交。礼仪节制民众的思想,音乐调和民众的愿望,政令推动礼乐的施行,刑罚防止礼乐被破坏。礼仪、音乐、刑罚、政令四个方面通达而不相抵触,那么王者的治国方略就完备了。

\begin{yuanwen}
乐者为同,礼者为异。同则相亲,异则相敬。乐胜则流,礼胜则离。合情饰貌\footnote{整饬行为、外貌,使其等级分明。}者,礼乐之事也。礼义立,则贵贱等矣;乐文同,则上下和矣;好恶著,则贤不肖别矣;刑禁暴,爵举贤,则政均矣。仁以爱之,义以正之,如此则民治\footnote{天下大治。}行矣。
\end{yuanwen}

音乐是为了求同,礼仪是为了求异。求同人们就会相互亲爱,求异人们就会相互尊敬。音乐的形式太过就会使人目无尊长,从而随波逐流;礼仪的形式太过就会使人无情无义,从而众叛亲离。调合内心情感、修饰外在容貌的做法,是符合礼乐制度的事情。礼仪和道义确立了,那么高贵和低贱的等级就会明了;音乐与曲调和谐了,那么上级与下级的关系就会和顺;喜好和厌恶分明了,那么贤人和庸人就能得以区别;刑罚禁止暴行,封爵推举贤能,那么政治就会彰显公正。用仁慈去关爱百姓,用道义来规范百姓,像这样治理民众的方法就得以施行了。

\begin{yuanwen}

\end{yuanwen}\begin{yuanwen}
	
\end{yuanwen}\begin{yuanwen}
	
\end{yuanwen}\begin{yuanwen}
	
\end{yuanwen}\begin{yuanwen}
	
\end{yuanwen}\begin{yuanwen}
	
\end{yuanwen}\begin{yuanwen}
	
\end{yuanwen}\begin{yuanwen}
	
\end{yuanwen}


\begin{yuanwen}



乐由中出,礼自外作。乐由中出,故静;礼自外作,故文。大乐必易,大礼必简。乐至则无怨,礼至则不争。揖让而治天下者,礼乐之谓也。暴民不作,诸侯宾服,兵革不试,五刑不用,百姓无患,天子不怒,如此则乐达矣。合父子之亲,明长幼之序,以敬四海之内。天子如此,则礼行矣。

大乐与天地同和,大礼与天地同节。和,故百物不失;节,故祀天祭地。明则有礼乐,幽则有鬼神,如此则四海之内合敬同爱矣。礼者,殊事合敬者也;乐者,异文合爱者也。礼乐之情同,故明王以相沿也。故事与时并,名与功偕。故钟鼓管磬羽籥干戚,乐之器也;诎信俯仰级兆舒疾,乐之文也。簠簋俎豆制度文章,礼之器也;升降上下周旋裼袭,礼之文也。故知礼乐之情者能作,识礼乐之文者能术。作者之谓圣,术者之谓明。明圣者,术作之谓也。

乐者,天地之和也;礼者,天地之序也。和,故百物皆化;序,故群物皆别。乐由天作,礼以地制。过制则乱,过作则暴。明于天地,然后能兴礼乐也。论伦无患,乐之情也;欣喜驩爱,乐之也。中正无邪,礼之质也;庄敬恭顺,礼之制也。若夫礼乐之施于金石,越于声音,用于宗庙社稷,事于山川鬼神,则此所以与民同也。

王者功成作乐,治定制礼。其功大者其乐备,其治辨者其礼具。干戚之舞,非备乐也;亨孰而祀,非达礼也。五帝殊时,不相沿乐;三王异世,不相袭礼。乐极则忧,礼粗则偏矣。及夫敦乐而无忧,礼备而不偏者,其唯大圣乎?天高地下,万物散殊,而礼制行也;流而不息,合同而化,而乐兴也。春作夏长,仁也;秋敛冬藏,义也。仁近于乐,义近于礼。乐者敦和,率神而从天;礼者辨宜,居鬼而从地。故圣人作乐以应天,作礼以配地。礼乐明备,天地官矣。

天尊地卑,君臣定矣。高卑已陈,贵贱位矣。动静有常,小大殊矣。方以类聚,物以群分,则性命不同矣。在天成象,在地成形,如此则礼者天地之别也。地气上隮,天气下降,阴阳相摩,天地相荡,鼓之以雷霆,奋之以风雨,动之以四时,暖之以日月,而百化兴焉,如此则乐者天地之和也。

化不时则不生,男女无别则乱登,此天地之情也。及夫礼乐之极乎天而蟠乎地,行乎阴阳而通乎鬼神,穷高极远而测深厚,乐著太始而礼居成物。著不息者天也,著不动者地也。一动一静者,天地之间也。故圣人曰“礼云乐云”。

昔者舜作五弦之琴,以歌南风;夔始作乐,以赏诸侯。故天子之为乐也,以赏诸侯之有德者也。德盛而教尊,五穀时孰,然后赏之以乐。故其治民劳者,其舞行级远;其治民佚者,其舞行级短。故观其舞而知其德,闻其谥而知其行。大章,章之也;咸池,备也;韶,继也;夏,大也;殷周之乐尽也

天地之道,寒暑不时则疾,风雨不节则饥。教者,民之寒暑也,教不时则伤世。事者,民之风雨也,事不节则无功。然则先王之为乐也,以法治也,善则行象德矣。夫豢豕为酒,非以为祸也;而狱讼益烦,则酒之流生祸也。是故先王因为酒礼,一献之礼,宾主百拜,终日饮酒而不得醉焉,此先王之所以备酒祸也。故酒食者,所以合欢也。

乐者,所以象德也;礼者,所以闭淫也。是故先王有大事,必有礼以哀之;有大福,必有礼以乐之:哀乐之分,皆以礼终。

乐也者,施也;礼也者,报也。乐,乐其所自生;而礼,反其所自始。乐章德,礼报情反始也。所谓大路者,天子之舆也;龙旂九旒,天子之旌也;青黑缘者,天子之葆龟也;从之以牛羊之群,则所以赠诸侯也。

乐也者,情之不可变者也;礼也者,理之不可易者也。乐统同,礼别异,礼乐之说贯乎人情矣。穷本知变,乐之情也;著诚去伪,礼之经也。礼乐顺天地之诚,达神明之德,降兴上下之神,而凝是精粗之体,领父子君臣之节。

是故大人举礼乐,则天地将为昭焉。天地欣合,阴阳相得,煦妪覆育万物,然后草木茂,区萌达,羽翮奋,角生,蛰蟲昭稣,羽者妪伏,毛者孕鬻,胎生者不殰而卵生者不殈,则乐之道归焉耳。

乐者,非谓黄锺大吕弦歌干扬也,乐之末节也,故童者舞之;布筵席,陈樽俎,列笾豆,以升降为礼者,礼之末节也,故有司掌之。乐师辩乎声诗,故北面而弦;宗祝辩乎宗庙之礼,故后尸;商祝辩乎丧礼,故后主人。是故德成而上,成而下;行成而先,事成而后。是故先王有上有下,有先有后,然后可以有制于天下也。

乐者,圣人之所乐也,而可以善民心。其感人深,其风移俗易,故先王著其教焉。

夫人有血气心知之性,而无哀乐喜怒之常,应感起物而动,然后心术形焉。是故志微焦衰之音作,而民思忧;啴缓慢易繁文简节之音作,而民康乐;粗厉猛起奋末广贲之音作,而民刚毅;廉直经正庄诚之音作,而民肃敬;宽裕肉好顺成和动之音作,而民慈爱;流辟邪散狄成涤滥之音作,而民淫乱。

是故先王本之情性,稽之度数,制之礼义,合生气之和,道五常之行,使之阳而不散,阴而不密,刚气不怒,柔气不慑,四暢交于中而发作于外,皆安其位而不相夺也。然后立之学等,广其节奏,省其文采,以绳德厚也。类小大之称,比终始之序,以象事行,使亲疏贵贱长幼男女之理皆形见于乐:故曰“乐观其深矣”。

土敝则草木不长,水烦则鱼鳖不大,气衰则生物不育,世乱则礼废而乐淫。是故其声哀而不庄,乐而不安,慢易以犯节,流湎以忘本。广则容奸。狭则思欲,感涤荡之气而灭平和之德,是以君子贱之也。

凡奸声感人而逆气应之,逆气成象而淫乐兴焉。正声感人而顺气应之,顺气成象而和乐兴焉。倡和有应,回邪曲直各归其分,而万物之理以类相动也。

是故君子反情以和其志,比类以成其行。奸声乱色不留聪明,淫乐废礼不接于心术,惰慢邪辟之气不设于身体,使耳目鼻口心知百体皆由顺正,以行其义。然后发以声音,文以琴瑟,动以干戚,饰以羽旄,从以箫管,奋至德之光,动四气之和,以著万物之理。是故清明象天,广大象地,终始象四时,周旋象风雨;五色成文而不乱,八风从律而不奸,百度得数而有常;小大相成,终始相生,倡和清浊,代相为经。故乐行而伦清,耳目聪明,血气和平,移风易俗,天下皆宁。故曰“乐者乐也”。君子乐得其道,小人乐得其欲。以道制欲,则乐而不乱;以欲忘道,则惑而不乐。是故君子反情以和其志,广乐以成其教,乐行而民乡方,可以观德矣。

德者,性之端也;乐者,德之华也;金石丝竹,乐之器也。诗,言其志也;歌,咏其声也;舞,动其容也:三者本乎心,然后乐气从之。是故情深而文明,气盛而化神,和顺积中而英华发外,唯乐不可以为伪。

乐者,心之动也;声者,乐之象也;文采节奏,声之饰也。君子动其本,乐其象,然后治其饰。是故先鼓以警戒,三步以见方,再始以著往,复乱以饬归,奋疾而不拔,极幽而不隐。独乐其志,不厌其道;备举其道,不私其欲。是以情见而义立,乐终而德尊;君子以好善,小人以息过:故曰“生民之道,乐为大焉”。

君子曰:礼乐不可以斯须去身。致乐以治心,则易直子谅之心油然生矣。易直子谅之心生则乐,乐则安,安则久,久则天,天则神。天则不言而信,神则不怒而威。致乐,以治心者也;致礼,以治躬者也。治躬则庄敬,庄敬则严威。心中斯须不和不乐,而鄙诈之心入之矣;外貌斯须不庄不敬,而慢易之心入之矣。故乐也者,动于内者也;礼也者,动于外者也。乐极和,礼极顺。内和而外顺,则民瞻其颜色而弗与争也,望其容貌而民不生易慢焉。德煇动乎内而民莫不承听,理发乎外而民莫不承顺,故曰“知礼乐之道,举而错之天下无难矣”。

乐也者,动于内者也;礼也者,动于外者也。故礼主其谦,乐主其盈。礼谦而进,以进为文;乐盈而反,以反为文。礼谦而不进,则销;乐盈而不反,则放。故礼有报而乐有反。礼得其报则乐,乐得其反则安。礼之报,乐之反,其义一也。

夫乐者乐也,人情之所不能免也。乐必发诸声音,形于动静,人道也。声音动静,性术之变,尽于此矣。故人不能无乐,乐不能无形。形而不为道,不能无乱。先王恶其乱,故制雅颂之声以道之,使其声足以乐而不流,使其文足以纶而不息,使其曲直繁省廉肉节奏,足以感动人之善心而已矣,不使放心邪气得接焉,是先王立乐之方也。是故乐在宗庙之中,君臣上下同听之,则莫不和敬;在族长乡里之中,长幼同听之,则莫不和顺;在闺门之内,父子兄弟同听之,则莫不和亲。故乐者,审一以定和,比物以饰节,节奏合以成文,所以合和父子君臣,附亲万民也,是先王立乐之方也。故听其雅颂之声,志意得广焉;执其干戚,习其俯仰诎信,容貌得庄焉;行其缀兆,要其节奏,行列得正焉,进退得齐焉。故乐者天地之齐,中和之纪,人情之所不能免也。

夫乐者,先王之所以饰喜也;军旅鈇钺者,先王之所以饰怒也。故先王之喜怒皆得其齐矣。喜则天下和之,怒则暴乱者畏之。先王之道礼乐可谓盛矣。

魏文侯问于子夏曰:“吾端冕而听古乐则唯恐卧,听郑卫之音则不知倦。敢问古乐之如彼,何也?新乐之如此,何也?”

子夏答曰:“今夫古乐,进旅而退旅,和正以广,弦匏笙簧合守拊鼓,始奏以文,止乱以武,治乱以相,讯疾以雅。君子于是语,于是道古,修身及家,平均天下:此古乐之发也。今夫新乐,进俯退俯,奸声以淫,溺而不止,及优侏儒,
杂子女,不知父子。乐终不可以语,不可以道古:此新乐之发也。今君之所问者乐也,所好者音也。夫乐之与音,相近而不同。”

文侯曰:“敢问如何?”

子夏答曰:“夫古者天地顺而四时当,民有德而五穀昌,疾疢不作而无祅祥,此之谓大当。然后圣人作为父子君臣以为之纪纲,纪纲既正,天下大定,天下大定,然后正六律,和五声,弦歌诗颂,此之谓德音,德音之谓乐。诗曰:‘莫其德音,其德克明,克明克类,克长克君。王此大邦,克顺克俾。俾于文王,其德靡悔。既受帝祉,施于孙子。’此之谓也。今君之所好者,其溺音与?”

文侯曰:“敢问溺音者何从出也?”

子夏答曰:“郑音好滥淫志,宋音燕女溺志,卫音趣数烦志,齐音骜辟骄志,四者皆淫于色而害于德,是以祭祀不用也。诗曰:‘肃雍和鸣,先祖是听。’夫肃肃,敬也;雍雍,和也。夫敬以和,何事不行?为人君者,谨其所好恶而已矣。君好之则臣为之,上行之则民从之。诗曰:‘诱民孔易’,此之谓也。然后圣人作为鞉鼓椌楬埙篪,此六者,德音之音也。然后钟磬竽瑟以和之,干戚旄狄以舞之。此所以祭先王之庙也,所以献酬酳酢也,所以官序贵贱各得其宜也,此所以示后世有尊卑长幼序也。钟声铿,铿以立号,号以立横,横以立武。君子听钟声则思武臣。石声硜,硜以立别,别以致死。君子听磬声则思死封疆之臣。丝声哀,哀以立廉,廉以立志。君子听琴瑟之声则思志义之臣。竹声滥,滥以立会,会以聚众。君子听竽笙箫管之声则思畜聚之臣。鼓鼙之声讙,讙以立动,动以进众。君子听鼓鼙之声则思将帅之臣。君子之听音,非听其铿鎗而已也,彼亦有所合之也。”

宾牟贾侍坐于孔子,孔子与之言,及乐,曰:“夫武之备戒之已久,何也?”

答曰:“病不得其众也。”

“永叹之,淫液之,何也?”

答曰:“恐不逮事也。”

“发扬蹈厉之已蚤,何也?”

答曰:“及时事也。”

“武坐致右宪左,何也?”

答曰:“非武坐也。”

“声淫及商,何也?”

答曰:“非武音也。”

子曰:“若非武音,则何音也?”

答曰:“有司失其传也。如非有司失其传,则武王之志荒矣。”

子曰:“唯丘之闻诸苌弘,亦若吾子之言是也。”

宾牟贾起,免席而请曰:“夫武之备戒之已久,则既闻命矣。敢问迟之迟而又久,何也?”

子曰:“居,吾语汝。夫乐者,象成者也。总干而山立,武王之事也;发扬蹈厉,太公之志也;武乱皆坐,周召之治也。且夫武,始而北出,再成而灭商,三成而南,四成而南国是疆,五成而分陕,周公左,召公右,六成复缀,以崇天子,夹振之而四伐,盛威于中国也。分夹而进,事蚤济也。久立于缀,以待诸侯之至也。且夫女独未闻牧野之语乎?武王克殷反商,未及下车,而封黄帝之后于蓟,封帝尧之后于祝,封帝舜之后于陈;下车而封夏后氏之后于杞,封殷之后于宋,封王子比干之墓,释箕子之囚,使之行商容而复其位。庶民弛政,庶士倍禄。济河而西,马散华山之阳而弗复乘;牛散桃林之野而不复服;车甲弢而藏之府库而弗复用;倒载干戈,苞之以虎皮;将率之士,使为诸侯,名之曰‘建櫜’:然后天下知武王之不复用兵也。散军而郊射,左射貍首,右射驺虞,而贯革之射息也;裨冕搢笏,而虎贲之士税剑也;祀乎明堂,而民知孝;朝觐,然后诸侯知所以臣;耕藉,然后诸侯知所以敬:五者天下之大教也。食三老五更于太学,天子袒而割牲,执酱而馈,执爵而酳,冕而总干,所以教诸侯之悌也。若此,则周道四达,礼乐交通,则夫武之迟久,不亦宜乎?”

子贡见师乙而问焉,曰:“赐闻声歌各有宜也,如赐者宜何歌也?”

师乙曰:“乙,贱工也,何足以问所宜。请诵其所闻,而吾子自执焉。宽而静,柔而正者宜歌颂;广大而静,疏达而信者宜歌大雅;恭俭而好礼者宜歌小雅;正直清廉而谦者宜歌风;肆直而慈爱者宜歌商;温良而能断者宜歌齐。夫歌者,直己而陈德;动己而天地应焉,四时和焉,星辰理焉,万物育焉。故商者,五帝之遗声也,商人志之,故谓之商;齐者,三代之遗声也,齐人志之,故谓之齐。明乎商之诗者,临事而屡断;明乎齐之诗者,见利而让也。临事而屡断,勇也;见利而让,义也。有勇有义,非歌孰能保此?故歌者,上如抗,下如队,曲如折,止如木,居中矩,句中钩,累累乎殷如贯珠。故歌之为言也,长言之也。说之,故言之;言之不足,故长言之;长言之不足,故嗟叹之;嗟叹之不足,故不知手之舞之足之蹈之。”子贡问乐。

凡音由于人心,天之与人有以相通,如景之象形,响之应声。故为善者天报之以福,为恶者天与之以殃,其自然者也。

故舜弹五弦之琴,歌南风之诗而天下治;纣为朝歌北鄙之音,身死国亡。舜之道何弘也?纣之道何隘也?夫南风之诗者生长之音也,舜乐好之,乐与天地同意,得万国之驩心,故天下治也。夫朝歌者不时也,北者败也,鄙者陋也,纣乐好之,与万国殊心,诸侯不附,百姓不亲,天下畔之,故身死国亡。

而卫灵公之时,将之晋,至于濮水之上舍。夜半时闻鼓琴声,问左右,皆对曰“不闻”。乃召师涓曰:“吾闻鼓琴音,问左右,皆不闻。其状似鬼神,为我听而写之。”师涓曰:“诺。”因端坐援琴,听而写之。明日,曰:“臣得之矣,然未习也,请宿习之。”灵公曰:“可。”因复宿。明日,报曰:“习矣。”即去之晋,见晋平公。平公置酒于施惠之台。酒酣,灵公曰:“今者来,闻新声,请奏之。”平公曰:“可。”即令师涓坐师旷旁,援琴鼓之。未终,师旷抚而止之曰:“此亡国之声也,不可遂。”平公曰:“何道出?”师旷曰:“师延所作也。与纣为靡靡之乐,武王伐纣,师延东走,自投濮水之中,故闻此声必于濮水之上,先闻此声者国削。”平公曰:“寡人所好者音也,原遂闻之。”师涓鼓而终之。

平公曰:“音无此最悲乎?”师旷曰:“有。”平公曰:“可得闻乎?”师旷曰:“君德义薄,不可以听之。”平公曰:“寡人所好者音也,原闻之。”师旷不得已,援琴而鼓之。一奏之,有玄鹤二八集乎廊门;再奏之,延颈而鸣,舒翼而舞。

平公大喜,起而为师旷寿。反坐,问曰:“音无此最悲乎?”师旷曰:“有。昔者黄帝以大合鬼神,今君德义薄,不足以听之,听之将败。”平公曰:“寡人老矣,所好者音也,原遂闻之。”师旷不得已,援琴而鼓之。一奏之,有白云从西北起;再奏之,大风至而雨随之,飞廊瓦,左右皆奔走。平公恐惧,伏于廊屋之间。晋国大旱,赤地三年。

听者或吉或凶。夫乐不可妄兴也。

太史公曰:夫上古明王举乐者,非以娱心自乐,快意恣欲,将欲为治也。正教者皆始于音,音正而行正。故音乐者,所以动荡血脉,通流精神而和正心也。故宫动脾而和正圣,商动肺而和正义,角动肝而和正仁,徵动心而和正礼,羽动肾而和正智。故乐所以内辅正心而外异贵贱也;上以事宗庙,下以变化黎庶也。琴长八尺一寸,正度也。弦大者为宫,而居中央,君也。商张右傍,其馀大小相次,不失其次序,则君臣之位正矣。故闻宫音,使人温舒而广大;闻商音,使人方正而好义;闻角音,使人恻隐而爱人;闻徵音,使人乐善而好施;闻羽音,使人整齐而好礼。夫礼由外入,乐自内出。故君子不可须臾离礼,须臾离礼则暴慢之行穷外;不可须臾离乐,须臾离乐则奸邪之行穷内。故乐音者,君子之所养义也。夫古者,天子诸侯听钟磬未尝离于庭,卿大夫听琴瑟之音未尝离于前,所以养行义而防淫佚也。夫淫佚生于无礼,故圣王使人耳闻雅颂之音,目视威仪之礼,足行恭敬之容,口言仁义之道。故君子终日言而邪辟无由入也。

乐之所兴,在乎防欲。陶心暢志,舞手蹈足。舜曰箫韶,融称属续。审音知政,观风变俗。端如贯珠,清同叩玉。洋洋盈耳,咸英馀曲。
\end{yuanwen}

\chapter{律书}

\begin{yuanwen}
王者制事立法,物度轨则,壹禀于六律,六律为万事根本焉。

其于兵械尤所重,故云“望敌知吉凶,闻声效胜负”,百王不易之道也。

武王伐纣,吹律听声,推孟春以至于季冬,杀气相并,而音尚宫。同声相从,物之自然,何足怪哉?

兵者,圣人所以讨彊暴,平乱世,夷险阻,救危殆。自含戴角之兽见犯则校,而况于人怀好恶喜怒之气?喜则爱心生,怒则毒螫加,情性之理也。

昔黄帝有涿鹿之战,以定火灾;颛顼有共工之陈,以平水害;成汤有南巢之伐,以殄夏乱。递兴递废,胜者用事,所受于天也。

自是之后,名士迭兴,晋用咎犯,而齐用王子,吴用孙武,申明军约,赏罚必信,卒伯诸侯,兼列邦土,虽不及三代之诰誓,然身宠君尊,当世显扬,可不谓荣焉?岂与世儒闇于大较,不权轻重,猥云德化,不当用兵,大至君辱失守,小乃侵犯削弱,遂执不移等哉!故教笞不可废于家,刑罚不可捐于国,诛伐不可偃于天下,用之有巧拙,行之有逆顺耳。

夏桀、殷纣手搏豺狼,足追四马,勇非微也;百战克胜,诸侯慑服,权非轻也。秦二世宿军无用之地,连兵于边陲,力非弱也;结怨匈奴,絓祸于越,势非寡也。及其威尽势极,闾巷之人为敌国,咎生穷武之不知足,甘得之心不息也。

高祖有天下,三边外畔;大国之王虽称蕃辅,臣节未尽。会高祖厌苦军事,亦有萧、张之谋,故偃武一休息,羁縻不备。

历至孝文即位,将军陈武等议曰:“南越、朝鲜自全秦时内属为臣子,后且拥兵阻戹,选蠕观望。高祖时天下新定,人民小安,未可复兴兵。今陛下仁惠抚百姓,恩泽加海内,宜及士民乐用,征讨逆党,以一封疆。”孝文曰:“朕能任衣冠,念不到此。会吕氏之乱,功臣宗室共不羞耻,误居正位,常战战栗栗,恐事之不终。且兵凶器,虽克所原,动亦秏病,谓百姓远方何?又先帝知劳民不可烦,故不以为意。朕岂自谓能?今匈奴内侵,军吏无功,边民父子荷兵日久,朕常为动心伤痛,无日忘之。今未能销距,原且坚边设候,结和通使,休宁北陲,为功多矣。且无议军。”故百姓无内外之繇,得息肩于田亩,天下殷富,粟至十馀钱,鸣鸡吠狗,烟火万里,可谓和乐者乎!

太史公曰:文帝时,会天下新去汤火,人民乐业,因其欲然,能不扰乱,故百姓遂安。自年六七十翁亦未尝至市井,游敖嬉戏如小兒状。孔子所称有德君子者邪!

书曰“七正”,二十八舍。律历,天所以通五行八正之气,天所以成孰万物也。舍者,日月所舍。舍者,舒气也。

不周风居西北,主杀生。东壁居不周风东,主辟生气而东之。至于营室。营室者,主营胎阳气而产之。东至于危。危,垝也。言阳气之垝,故曰危。十月也,律中应锺。应锺者,阳气之应,不用事也。其于十二子为亥。亥者,该也。言阳气藏于下,故该也。

广莫风居北方。广莫者,言阳气在下,阴莫阳广大也,故曰广莫。东至于虚。虚者,能实能虚,言阳气冬则宛藏于虚,日冬至则一阴下藏,一阳上舒,故曰虚。东至于须女。言万物变动其所,阴阳气未相离,尚相胥也,故曰须女。十一月也,律中黄锺。黄锺者,阳气踵黄泉而出也。其于十二子为子。子者,滋也;滋者,言万物滋于下也。其于十母为壬癸。壬之为言任也,言阳气任养万物于下也。癸之为言揆也,言万物可揆度,故曰癸。东至牵牛。牵牛者,言阳气牵引万物出之也。牛者,冒也,言地虽冻,能冒而生也。牛者,耕植种万物也。东至于建星。建星者,建诸生也。十二月也,律中大吕。大吕者。其于十二子为丑。

条风居东北,主出万物。条之言条治万物而出之,故曰条风。南至于箕。箕者,言万物根棋,故曰箕。正月也,律中泰蔟。泰蔟者,言万物蔟生也,故曰泰蔟。其于十二子为寅。寅言万物始生螾然也,故曰寅。南至于尾,言万物始生如尾也。南至于心,言万物始生有华心也。南至于房。房者,言万物门户也,至于门则出矣。

明庶风居东方。明庶者,明众物尽出也。二月也,律中夹锺。夹锺者,言阴阳相夹厕也。其于十二子为卯。卯之为言茂也,言万物茂也。其于十母为甲乙。甲者,言万物剖符甲而出也;乙者,言万物生轧轧也。南至于氐者。氐者,言万物皆至也。南至于亢。亢者,言万物亢见也。南至于角。角者,言万物皆有枝格如角也。三月也,律中姑洗。姑洗者,言万物洗生。其于十二子为辰。辰者,言万物之蜄也。

清明风居东南维,主风吹万物而西之。轸。轸者,言万物益大而轸轸然。西至于翼。翼者,言万物皆有羽翼也。四月也,律中中吕。中吕者,言万物尽旅而西行也。其于十二子为巳。巳者,言阳气之已尽也。西至于七星。七星者,阳数成于七,故曰七星。西至于张。张者,言万物皆张也。西至于注。注者,言万物之始衰,阳气下注,故曰注。五月也,律中蕤宾。蕤宾者,言阴气幼少,故曰蕤;痿阳不用事,故曰宾。

景风居南方。景者,言阳气道竟,故曰景风。其于十二子为午。午者,阴阳交,故曰午。其于十母为丙丁。丙者,言阳道著明,故曰丙;丁者,言万物之丁壮也,故曰丁。西至于弧。弧者,言万物之吴落且就死也。西至于狼。狼者,言万物可度量,断万物,故曰狼。

凉风居西南维,主地。地者,沈夺万物气也。六月也,律中林锺。林锺者,言万物就死气林林然。其于十二子为未。未者,言万物皆成,有滋味也。北至于罚。罚者,言万物气夺可伐也。北至于参。参言万物可参也,故曰参。七月也,律中夷则。夷则,言阴气之贼万物也。其于十二子为申。申者,言阴用事,申贼万物,故曰申。北至于浊。浊者,触也,言万物皆触死也,故曰浊。北至于留。留者,言阳气之稽留也,故曰留。八月也,律中南吕。南吕者,言阳气之旅入藏也。其于十二子为酉。酉者,万物之老也,故曰酉。

阊阖风居西方。阊者,倡也;阖者,藏也。言阳气道万物,阖黄泉也。其于十母为庚辛。庚者,言阴气庚万物,故曰庚;辛者,言万物之辛生,故曰辛。北至于胃。胃者,言阳气就藏,皆胃胃也。北至于娄。娄者,呼万物且内之也。北至于奎。奎者,主毒螫杀万物也,奎而藏之。九月也,律中无射。无射者,阴气盛用事,阳气无馀也,故曰无射。其于十二子为戌。戌者,言万物尽灭,故曰戌。律数:九九八十一以为宫。三分去一,五十四以为徵。三分益一,七十二以为商。三分去一,四十八以为羽。三分益一,六十四以为角。黄锺长八寸七分一,宫。大吕长七寸五分三分。太蔟长七寸分二,角。夹锺长六寸分三分一。姑洗长六寸分四,羽。仲吕长五寸九分三分二,徵。蕤宾长五寸六分三分。林锺长五寸分四,角。夷则长五寸三分二,商。南吕长四寸分八,徵。无射长四寸四分三分二。应锺长四寸二分三分二,羽。生锺分:子一分。丑三分二。寅九分八。卯二十七分十六。辰八十一分六十四。巳二百四十三分一百二十八。午七百二十九分五百一十二。未二千一百八十七分一千二十四。申六千五百六十一分四千九十六。酉一万九千六百八十三分八千一百九十二。戌五万九千四十九分三万二千七百六十八。亥十七万七千一百四十七分六万五千五百三十六。

生黄锺术曰:以下生者,倍其实,三其法。以上生者,四其实,三其法。上九,商八,羽七,角六,宫五,徵九。置一而九三之以为法。实如法,得长一寸。凡得九寸,命曰“黄锺之宫”。故曰音始于宫,穷于角;数始于一,终于十,成于三;气始于冬至,周而复生。

神生于无,形成于有,形然后数,形而成声,故曰神使气,气就形。形理如类有可类。或未形而未类,或同形而同类,类而可班,类而可识。圣人知天地识之别,故从有以至未有,以得细若气,微若声。然圣人因神而存之,虽妙必效情,核其华道者明矣。非有圣心以乘聪明,孰能存天地之神而成形之情哉?神者,物受之而不能知其去来,故圣人畏而欲存之。唯欲存之,神之亦存。其欲存之者,故莫贵焉。

太史公曰:旋玑玉衡以齐七政,即天地二十八宿。十母,十二子,锺律调自上古。建律运历造日度,可据而度也。合符节,通道德,即从斯之谓也。

自昔轩后,爰命伶纶。雄雌是听,厚薄伊均。以调气候,以轨星辰。军容取节,乐器斯因。自微知著,测化穷神。大哉虚受,含养生人。
\end{yuanwen}

\part{卷二十六}
\chapter{历书第四}

\begin{yuanwen}
昔自在古,历建正作于孟春。于时冰泮发蛰,百草奋兴,秭鳺先滜。物乃岁具,生于东,次顺四时,卒于冬分。时鸡三号,卒明。抚十二节,卒于丑。日月成,故明也。明者孟也,幽者幼也,幽明者雌雄也。雌雄代兴,而顺至正之统也。日归于西,起明于东;月归于东,起明于西。正不率天,又不由人,则凡事易坏而难成矣。

王者易姓受命,必慎始初,改正朔,易服色,推本天元,顺承厥意。

太史公曰:神农以前尚矣。盖黄帝考定星历,建立五行,起消息,正闰馀,于是有天地神祇物类之官,是谓五官。各司其序,不相乱也。民是以能有信,神是以能有明德。民神异业,敬而不渎,故神降之嘉生,民以物享,灾祸不生,所求不匮。

少昚氏之衰也,九黎乱德,民神杂扰,不可放物,祸菑荐至,莫尽其气。颛顼受之,乃命南正重司天以属神,命火正黎司地以属民,使复旧常,无相侵渎。其后三苗服九黎之德,故二官咸废所职,而闰馀乖次,孟陬殄灭,摄提无纪,历数失序。尧复遂重黎之后,不忘旧者,使复典之,而立羲和之官。明时正度,则阴阳调,风雨节,茂气至,民无夭疫。年耆禅舜,申戒文祖,云“天之历数在尔躬”。舜亦以命禹。由是观之,王者所重也。

夏正以正月,殷正以十二月,周正以十一月。盖三王之正若循环,穷则反本。天下有道,则不失纪序;无道,则正朔不行于诸侯。

幽、厉之后,周室微,陪臣执政,史不记时,君不告朔,故畴人子弟分散,或在诸夏,或在夷狄,是以其禨祥废而不统。周襄王二十六年闰三月,而春秋非之。先王之正时也,履端于始,举正于中,归邪于终。履端于始,序则不愆;举正于中,民则不惑;归邪于终,事则不悖。

其后战国并争,在于彊国禽敌,救急解纷而已,岂遑念斯哉!是时独有邹衍,明于五德之传,而散消息之分,以显诸侯。而亦因秦灭六国,兵戎极烦,又升至尊之日浅,未暇遑也。而亦颇推五胜,而自以为获水德之瑞,更名河曰“德水”,而正以十月,色上黑。然历度闰馀,未能睹其真也。

汉兴,高祖曰“北畤待我而起”,亦自以为获水德之瑞。虽明习历及张苍等,咸以为然。是时天下初定,方纲纪大基,高后女主,皆未遑,故袭秦正朔服色。

至孝文时,鲁人公孙臣以终始五德上书,言“汉得土德,宜更元,改正朔,易服色。当有瑞,瑞黄龙见”。事下丞相张苍,张苍亦学律历,以为非是,罢之。其后黄龙见成纪,张苍自黜,所欲论著不成。而新垣平以望气见,颇言正历服色事,贵幸,后作乱,故孝文帝废不复问。

至今上即位,招致方士唐都,分其天部;而巴落下闳运算转历,然后日辰之度与夏正同。乃改元,更官号,封泰山。因诏御史曰:“乃者,有司言星度之未定也,广延宣问,以理星度,未能詹也。盖闻昔者黄帝合而不死,名察度验,定清浊,起五部,建气物分数。然盖尚矣。书缺乐弛,朕甚闵焉。朕唯未能循明也,绩日分,率应水德之胜。今日顺夏至,黄钟为宫,林钟为徵,太蔟为商,南吕为羽,姑洗为角。自是以后,气复正,羽声复清,名复正变,以至子日当冬至,则阴阳离合之道行焉。十一月甲子朔旦冬至已詹,其更以七年为太初元年。年名‘焉逢摄提格’,月名‘毕聚’,日得甲子,夜半朔旦冬至。”

◎历术甲子篇

太初元年,岁名“焉逢摄提格”,月名“毕聚”,日得甲子,夜半朔旦冬至。

正北

十二无大馀,无小馀;无大馀,无小馀;

焉逢摄提格太初元年。

十二

大馀五十四,小馀三百四十八;大馀五,小馀八;

端蒙单阏二年。

闰十三

大馀四十八,小馀六百九十六;大馀十,小馀十六;

游兆执徐三年。

十二

大馀十二,小馀六百三;大馀十五,小馀二十四;

彊梧大荒落四年。

十二

大馀七,小馀十一;大馀二十一,无小馀;

徒维敦牂天汉元年。

闰十三

大馀一,小馀三百五十九;大馀二十六,小馀八;

祝犁协洽二年。

十二

大馀二十五,小馀二百六十六;大馀三十一,小馀十六;

商横涒滩三年。

十二

大馀十九,小馀六百一十四;大馀三十六,小馀二十四;

昭阳作鄂四年。

闰十三

大馀十四,小馀二十二;大馀四十二,无小馀;横艾淹茂太始元年。

十二

大馀三十七,小馀八百六十九;大馀四十七,小馀八;

尚章大渊献二年。

闰十三

大馀三十二,小馀二百七十七;大馀五十二,小馀一十六;

焉逢困敦三年。

十二

大馀五十六,小馀一百八十四;大馀五十七,小馀二十四;

端蒙赤奋若四年。

十二

大馀五十,小馀五百三十二;大馀三,无小馀;

游兆摄提格征和元年。

闰十三

大馀四十四,小馀八百八十;大馀八,小馀八;

彊梧单阏二年。

十二

大馀八,小馀七百八十七;大馀十三,小馀十六;

徒维执徐三年。

十二

大馀三,小馀一百九十五;大馀十八,小馀二十四;

祝犁大芒落四年。

闰十三

大馀五十七,小馀五百四十三;大馀二十四,无小馀;

商横敦牂后元元年。

十二

大馀二十一,小馀四百五十;大馀二十九,小馀八;

昭阳汁洽二年。

闰十三

大馀十五,小馀七百九十八;大馀三十四,小馀十六;

横艾涒滩始元元年。

正西十二

大馀三十九,小馀七百五;大馀三十九,小馀二十四;

尚章作噩二年。

十二

大馀三十四,小馀一百一十三;大馀四十五,无小馀;

焉逢淹茂三年。

闰十三

大馀二十八,小馀四百六十一;大馀五十,小馀八;

端蒙大渊献四年。

十二

大馀五十二,小馀三百六十八;大馀五十五,小馀十六;

游兆困敦五年。

十二

大馀四十六,小馀七百一十六;无大馀,小馀二十四;

彊梧赤奋若六年。

闰十三

大馀四十一,小馀一百二十四;大馀六,无小馀;

徒维摄提格元凤元年。

十二

大馀五,小馀三十一;大馀十一,小馀八;

祝犁单阏二年。

十二

大馀五十九,小馀三百七十九;大馀十六,小馀十六;

商横执徐三年。

闰十三

大馀五十三,小馀七百二十七;大馀二十一,小馀二十四;

昭阳大荒落四年。

十二

大馀十七,小馀六百三十四;大馀二十七,无小馀;

横艾敦牂五年。

闰十三

大馀十二,小馀四十二;大馀三十二,小馀八;

尚章汁洽六年。

十二

大馀三十五,小馀八百八十九;大馀三十七,小馀十六;

焉逢涒滩元平元年

十二

大馀三十,小馀二百九十七;大馀四十二,小馀二十四;

端蒙作噩本始元年。

闰十三

大馀二十四,小馀六百四十五;大馀四十八,无小馀;

游兆阉茂二年。

十二

大馀四十八,小馀五百五十二;大馀五十三,小馀八;

彊梧大渊献三年。

十二

大馀四十二,小馀九百;大馀五十八,小馀十六;徒维困敦四年。

闰十三

大馀三十七,小馀三百八;大馀三,小馀二十四;

祝犁赤奋若地节元年。

十二

大馀一,小馀二百一十五;大馀九,无小馀;

商横摄提格二年。

闰十三

大馀五十五,小馀五百六十三;大馀十四,小馀八;

昭阳单阏三年。

正南十二

大馀十九,小馀四百七十;大馀十九,小馀十六;

横艾执徐四年。

十二

大馀十三,小馀八百一十八;大馀二十四,小馀二十四;

尚章大荒落元康元年。

闰十三

大馀八,小馀二百二十六;大馀三十,无小馀;

焉逢敦牂二年。

十二

大馀三十二,小馀一百三十三;大馀三十五,小馀八;

端蒙协洽三年。

十二

大馀二十六,小馀四百八十一;大馀四十,小馀十六;

游兆涒滩四年。

闰十三

大馀二十,小馀八百二十九;大馀四十五,小馀二十四;

彊梧作噩神雀元年。

十二

大馀四十四,小馀七百三十六;大馀五十一,无小馀;

徒维淹茂二年。

十二

大馀三十九,小馀一百四十四;大馀五十六,小馀八;

祝犁大渊献三年。

闰十三

大馀三十三,小馀四百九十二;大馀一,小馀十六;

商横困敦四年。

十二

大馀五十七,小馀三百九十九;大馀六,小馀二十四;

昭阳赤奋若五凤元年。

闰十三

大馀五十一,小馀七百四十七;大馀十二,无小馀;

横艾摄提格二年。

十二

大馀十五,小馀六百五十四;大馀十七,小馀八;

尚章单阏三年。

十二

大馀十,小馀六十二;大馀二十二,小馀十六;

焉逢执徐四年。

闰十三

大馀四,小馀四百一十;大馀二十七,小馀二十四;

端蒙大荒落甘露元年。

十二

大馀二十八,小馀三百一十七;大馀三十三,无小馀;

游兆敦牂二年。

十二

大馀二十二,小馀六百六十五;大馀三十八,小馀八;

彊梧协洽三年。

闰十三

大馀十七,小馀七十三;大馀四十三,小馀十六;

徒维涒滩四年。

十二

大馀四十,小馀九百二十;大馀四十八,小馀二十四;

祝犁作噩黄龙元年。

闰十三

大馀三十五,小馀三百二十八;大馀五十四,无小馀;

商横淹茂初元元年。

正东十二

大馀五十九,小馀二百三十五;大馀五十九,小馀八;

昭阳大渊献二年。

十二

大馀五十三,小馀五百八十三;大馀四,小馀十六;

横艾困敦三年。

闰十三

大馀四十七,小馀九百三十一;大馀九,小馀二十四;

尚章赤奋若四年。

十二

大馀十一,小馀八百三十八;大馀十五,无小馀;

焉逢摄提格五年。

十二

大馀六,小馀二百四十六;大馀二十,小馀八;

端蒙单阏永光元年。

闰十三

无大馀,小馀五百九十四;大馀二十五,小馀十六;

游兆执徐二年。

十二

大馀二十四,小馀五百一;大馀三十,小馀二十四;

彊梧大荒落三年。

十二

大馀十八,小馀八百四十九;大馀三十六,无小馀;

徒维敦牂四年。

闰十三

大馀十三,小馀二百五十七;大馀四十一,小馀八;

祝犁协洽五年。

十二

大馀三十七,小馀一百六十四;大馀四十六,小馀十六;

商横涒滩建昭元年。

闰十三

大馀三十一,小馀五百一十二;大馀五十一,小馀二十四;

昭阳作噩二年。

十二

大馀五十五,小馀四百一十九;大馀五十七,无小馀;

横艾阉茂三年。

十二

大馀四十九,小馀七百六十七;大馀二,小馀八;

尚章大渊献四年。

闰十三

大馀四十四,小馀一百七十五;大馀七,小馀十六;

焉逢困敦五年。

十二

大馀八,小馀八十二;大馀十二,小馀二十四;

端蒙赤奋若竟宁元年。

十二

大馀二,小馀四百三十;大馀十八,无小馀;

游兆摄提格建始元年。

闰十三

大馀五十六,小馀七百七十八;大馀二十三,小馀八;

彊梧单阏二年。

十二

大馀二十,小馀六百八十五;大馀二十八,小馀十六;

徒维执徐三年。

闰十三

大馀十五,小馀九十三;大馀三十三,小馀二十四;

祝犁大荒落四年。

右历书:大馀者,日也。小馀者,月也。端蒙者,年名也。支:丑名赤奋若,寅名摄提格。干:丙名游兆。正北,冬至加子时;正西,加酉时;正南,加午时;正东,加卯时。

历数之兴,其来尚矣。重黎是司,容成斯纪。推步天象,消息母子。五胜轮环,三正互起。孟陬贞岁,畴人顺轨。敬授之方,履端为美。
\end{yuanwen}

\part{卷二十七}
\chapter{天官书第五}

陈仁锡:「《天官书》独无序何也?岂后世缺之耶?其中文字无限奇古,极多变化。篇末有太史公论『自初生民以来』至『天官备矣』一章,盖本书之首序,而错简在后耳。」

《天官书》与《历书》一样,也不在《史记》十篇亡书之数,必为太史公原著(也有人以为是“妄人录《汉书·天文志》而成”,可不论),当然不排斥其中有错简以及后人窜入的成分,这是古书难以避免的情况,但不影响它的本来价值。《天官书》的内容大概可七章,一为经星,分作五宫记述三垣二十八宿等恒星;二为五纬,记木、火、土、金、水五行星;三为二曜,记日与月;四为异星;五为云气;六为候岁;七为总论。

\begin{yuanwen}
中宫天极星,其一明者,太一常居也;旁三星三公,或曰子属。后句四星,末大星正妃,馀三星后宫之属也。环之匡卫\footnote{环绕护卫。}十二星,籓臣。皆曰紫宫。
\end{yuanwen}

中宫正中央的一颗星星称为天极星,比它周围的星星都要明亮,常居于正北不动,所以称它为太一,是天帝的意思;旁边的三颗星称为三公,也有人称为太子、庶子。天极星的后面是形如钩状的勾星四颗,其中最后一颗较为明亮的星称为正妃,其余三颗星为后宫的嫔妃之类。像护卫一样环绕护卫着天极星的十二颗星的,是藩臣。它们合起来就称为紫宫。

\begin{yuanwen}
前列直斗口三星,随北端兑,若见若不,曰阴德,或曰天一。紫宫左三星曰天枪,右五星曰天棓,后六星绝汉抵营室,曰阁道。
\end{yuanwen}


\begin{yuanwen}\end{yuanwen}\begin{yuanwen}\end{yuanwen}\begin{yuanwen}\end{yuanwen}\begin{yuanwen}\end{yuanwen}\begin{yuanwen}\end{yuanwen}\begin{yuanwen}\end{yuanwen}\begin{yuanwen}\end{yuanwen}\begin{yuanwen}\end{yuanwen}\begin{yuanwen}\end{yuanwen}\begin{yuanwen}\end{yuanwen}\begin{yuanwen}\end{yuanwen}\begin{yuanwen}\end{yuanwen}
\begin{yuanwen}\end{yuanwen}\begin{yuanwen}\end{yuanwen}\begin{yuanwen}\end{yuanwen}\begin{yuanwen}\end{yuanwen}\begin{yuanwen}\end{yuanwen}\begin{yuanwen}\end{yuanwen}\begin{yuanwen}\end{yuanwen}
\begin{yuanwen}\end{yuanwen}\begin{yuanwen}\end{yuanwen}\begin{yuanwen}\end{yuanwen}\begin{yuanwen}\end{yuanwen}\begin{yuanwen}\end{yuanwen}\begin{yuanwen}\end{yuanwen}\begin{yuanwen}\end{yuanwen}\begin{yuanwen}\end{yuanwen}\begin{yuanwen}\end{yuanwen}\begin{yuanwen}\end{yuanwen}\begin{yuanwen}\end{yuanwen}\begin{yuanwen}\end{yuanwen}\begin{yuanwen}\end{yuanwen}\begin{yuanwen}\end{yuanwen}\begin{yuanwen}\end{yuanwen}\begin{yuanwen}\end{yuanwen}\begin{yuanwen}\end{yuanwen}\begin{yuanwen}\end{yuanwen}\begin{yuanwen}\end{yuanwen}\begin{yuanwen}\end{yuanwen}\begin{yuanwen}\end{yuanwen}\begin{yuanwen}\end{yuanwen}\begin{yuanwen}\end{yuanwen}\begin{yuanwen}\end{yuanwen}\begin{yuanwen}\end{yuanwen}\begin{yuanwen}\end{yuanwen}

\begin{yuanwen}




北斗七星,所谓“旋、玑、玉衡以齐七政”。杓携龙角,衡殷南斗,魁枕参首。用昏建者杓;杓,自华以西南。夜半建者衡;衡,殷中州河、济之间。平旦建者魁;魁,海岱以东北也。斗为帝车,运于中央,临制四乡。分阴阳,建四时,均五行,移节度,定诸纪,皆系于斗。

斗魁戴匡六星曰文昌宫:一曰上将,二曰次将,三曰贵相,四曰司命,五曰司中,六曰司禄。在斗魁中,贵人之牢。魁下六星,两两相比者,名曰三能。三能色齐,君臣和;不齐,为乖戾。辅星明近,辅臣亲彊;斥小,疏弱。

杓端有两星:一内为矛,招摇;一外为盾,天锋。有句圜十五星,属杓,曰贱人之牢。其牢中星实则囚多,虚则开出。

天一、枪、棓、矛、盾动摇,角大,兵起。

东宫苍龙,房、心。心为明堂,大星天王,前后星子属。不欲直,直则天王失计。房为府,曰天驷。其阴,右骖。旁有两星曰衿;北一星曰辖。东北曲十二星曰旗。旗中四星天市;中六星曰市楼。市中星众者实;其虚则秏。房南众星曰骑官。

左角,李;右角,将。大角者,天王帝廷。其两旁各有三星,鼎足句之,曰摄提。摄提者,直斗杓所指,以建时节,故曰“摄提格”。亢为疏庙,主疾。其南北两大星,曰南门。氐为天根,主疫。

尾为九子,曰君臣;斥绝,不和。箕为敖客,曰口舌。

火犯守角,则有战。房、心,王者恶之也。

南宫硃鸟,权、衡。衡,太微,三光之廷。匡卫十二星,籓臣:西,将;东,相;南四星,执法;中,端门;门左右,掖门。门内六星,诸侯。其内五星,五帝坐。后聚一十五星,蔚然,曰郎位;傍一大星,将位也。月、五星顺入,轨道,司其出,所守,天子所诛也。其逆入,若不轨道,以所犯命之;中坐,成形,皆群下从谋也。金、火尤甚。廷籓西有隋星五,曰少微,士大夫。权,轩辕。轩辕,黄龙体。前大星,女主象;旁小星,御者后宫属。月、五星守犯者,如衡占。

东井为水事。其西曲星曰钺。钺北,北河;南,南河;两河、天阙间为关梁。舆鬼,鬼祠事;中白者为质。火守南北河,兵起,穀不登。故德成衡,观成潢,伤成钺,祸成井,诛成质。

柳为鸟注,主木草。七星,颈,为员官。主急事。张,素,为厨,主觞客。翼为羽翮,主远客。

轸为车,主风。其旁有一小星,曰长沙,星星不欲明;明与四星等,若五星入轸中,兵大起。轸南众星曰天库楼;库有五车。车星角若益众,及不具,无处车马。

西宫咸池,曰天五潢。五潢,五帝车舍。火入,旱;金,兵;水,水。中有三柱;柱不具,兵起。

奎曰封豕,为沟渎。娄为聚众。胃为天仓。其南众星曰廥积。

昴曰髦头,胡星也,为白衣会。毕曰罕车,为边兵,主弋猎。其大星旁小星为附耳。附耳摇动,有谗乱臣在侧。昴、毕间为天街。其阴,阴国;阳,阳国。

参为白虎。三星直者,是为衡石。下有三星,兑,曰罚,为斩艾事。其外四星,左右肩股也。小三星隅置,曰觜觿,为虎首,主葆旅事。其南有四星,曰天厕。厕下一星,曰天矢。矢黄则吉青、白、黑,凶。其西有句曲九星,三处罗:一曰天旗,二曰天苑,三曰九游。其东有大星曰狼。狼角变色,多盗贼。下有四星曰弧,直狼。狼比地有大星,曰南极老人。老人见,治安;不见,兵起。常以秋分时候之于南郊。

附耳入毕中,兵起。

北宫玄武,虚、危。危为盖屋;虚为哭泣之事。

其南有众星,曰羽林天军。军西为垒,或曰钺。旁有一大星为北落。北落若微亡,军星动角益希,及五星犯北落,入军,军起。火、金、水尤甚:火,军忧;水,患;木、土,军吉。危东六星,两两相比,曰司空。

营室为清庙,曰离宫、阁道。汉中四星,曰天驷。旁一星,曰王良。王良策马,车骑满野。旁有八星,绝汉,曰天潢。天潢旁,江星。江星动,人涉水。

杵、臼四星,在危南。匏瓜,有青黑星守之,鱼盐贵。

南斗为庙,其北建星。建星者,旗也。牵牛为牺牲。其北河鼓。河鼓大星,上将;左右,左右将。婺女,其北织女。织女,天女孙也。

察日、月之行以揆岁星顺逆。曰东方木,主春,日甲乙。义失者,罚出岁星。岁星赢缩,以其舍命国。所在国不可伐,可以罚人。其趋舍而前曰赢,退舍曰缩。赢,其国有兵不复;缩,其国有忧,将亡,国倾败。其所在,五星皆从而聚于一舍,其下之国可以义致天下。

以摄提格岁:岁阴左行在寅,岁星右转居丑。正月,与斗、牵牛晨出东方,名曰监德。色苍苍有光。其失次,有应见柳。岁早,水;晚,旱。

岁星出,东行十二度,百日而止,反逆行;逆行八度,百日,复东行。岁行三十度十六分度之七,率日行十二分度之一,十二岁而周天。出常东方,以晨;入于西方,用昏。

单阏岁:岁阴在卯,星居子。以二月与婺女、虚、危晨出,曰降入。大有光。其失次,有应见张。其岁大水。

执徐岁:岁阴在辰,星居亥。以三月与营室、东壁晨出,曰青章。青青甚章。其失次;有应见轸。岁早,旱;晚,水。

大荒骆岁:岁阴在巳,星居戌。以四月与奎、娄晨出,曰跰踵。熊熊赤色,有光。其失次,有应见亢。

敦牂岁:岁阴在午,星居酉。以五月与胃、昴、毕晨出,曰开明。炎炎有光。偃兵;唯利公王,不利治兵。其失次,有应见房。岁早,旱;晚,水。

叶洽岁:岁阴在未,星居申。以六月与觜觿、参晨出,曰长列。昭昭有光。利行兵。其失次,有应见箕。

涒滩岁:岁阴在申,星居未。以七月与东井、舆鬼晨出,曰大音。昭昭白。其失次,有应见牵牛。

作鄂岁:岁阴在酉,星居午。以八月与柳、七星、张晨出,曰长王。作作有芒。国其昌,熟穀。其失次,有应见危。有旱而昌,有女丧,民疾。

阉茂岁:岁阴在戌,星居巳。以九月与翼、轸晨出,曰天睢。白色大明。其失次,有应见东壁。岁水,女丧。

大渊献岁:岁阴在亥,星居辰。以十月与角、亢晨出,曰大章。苍苍然,星若跃而阴出旦,是谓“正平”。起师旅,其率必武;其国有德,将有四海。其失次,有应见娄。

困敦岁:岁阴在子,星居卯。以十一月与氐、房、心晨出,曰天泉。玄色甚明。江池其昌,不利起兵。其失次,有应昴。

赤奋若岁:岁阴在丑,星居寅,以十二月与尾、箕晨出,曰天皓。黫然黑色甚明。其失次,有应见参。

当居不居,居之又左右摇,未当去去之,与他星会,其国凶。所居久,国有德厚。其角动,乍小乍大,若色数变,人主有忧。

其失次舍以下,进而东北,三月生天棓,长四丈,末兑,进而东南,三月生彗星,长二丈,类彗。退而西北,三月生天欃,长四丈,末兑。退而西南,三月生天枪,长数丈,两头兑。谨视其所见之国,不可举事用兵。其出如浮如沈,其国有土功;如沈如浮,其野亡。色赤而有角,其所居国昌。迎角而战者,不胜。星色赤黄而沈,所居野大穰。色青白而赤灰,所居野有忧。岁星入月,其野有逐相;与太白斗,其野有破军。

岁星一曰摄提,曰重华,曰应星,曰纪星。营室为清庙,岁星庙也。

察刚气以处荧惑。曰南方火,主夏,日丙、丁。礼失,罚出荧惑,荧惑失行是也。出则有兵,入则兵散。以其舍命国。荧惑为勃乱,残贼、疾、丧、饥、兵。反道二舍以上,居之,三月有殃,五月受兵,七月半亡地,九月太半亡地。因与俱出入,国绝祀。居之,殃还至,虽大当小;久而至,当小反大。其南为丈夫,北为女子丧。若角动绕环之,及乍前乍后,左右,殃益大。与他星斗,光相逮,为害;不相逮,不害。五星皆从而聚于一舍,其下国可以礼致天下。

法,出东行十六舍而止;逆行二舍;六旬,复东行,自所止数十舍,十月而入西方;伏行五月,出东方。其出西方曰“反明”,主命者恶之。东行急,一日行一度半。

其行东、西、南、北疾也。兵各聚其下;用战,顺之胜,逆之败。荧惑从太白,军忧;离之,军卻。出太白阴,有分军;行其阳,有偏将战。当其行,太白逮之,破军杀将。其入守犯太微、轩辕、营室,主命恶之。心为明堂,荧惑庙也。谨候此。

历斗之会以定填星之位。曰中央土,主季夏,日戊、己,黄帝,主德,女主象也。岁填一宿,其所居国吉。未当居而居,若已去而复还,还居之,其国得土,不乃得女。若当居而不居,既已居之,又西东去,其国失土,不乃失女,不可举事用兵。其居久,其国福厚;易,福薄。

其一名曰地侯,主岁。岁行十度百十二分度之五,日行二十八分度之一,二十八岁周天。其所居,五星皆从而聚于一舍,其下之国,可重致天下。礼、德、义、杀、刑尽失,而填星乃为之动摇。

赢,为王不宁;其缩,有军不复。填星,其色黄,九芒,音曰黄锺宫。其失次上二三宿曰赢,有主命不成,不乃大水。失次下二三宿曰缩,有后戚,其岁不复,不乃天裂若地动。

斗为文太室,填星庙,天子之星也。

木星与土合,为内乱。饥,主勿用战,败;水则变谋而更事;火为旱;金为白衣会若水。金在南曰牝牡,年穀熟,金在北,岁偏无。火与水合为焠,与金合为铄,为丧,皆不可举事,用兵大败。土为忧,主孽卿;大饥,战败,为北军,军困,举事大败。土与水合,穰而拥阏,有覆军,其国不可举事。出,亡地;入,得地。金为疾,为内兵,亡地。三星若合,其宿地国外内有兵与丧,改立公王。四星合,兵丧并起,君子忧,小人流。五星合,是为易行,有德,受庆,改立大人,掩有四方,子孙蕃昌;无德,受殃若亡。五星皆大,其事亦大;皆小,事亦小。

蚤出者为赢,赢者为客。晚出者为缩,缩者为主人。必有天应见于杓星。同舍为合。相陵为斗,七寸以内必之矣。

五星色白圜,为丧旱;赤圜,则中不平,为兵;青圜,为忧水;黑圜,为疾,多死;黄圜,则吉。赤角犯我城,黄角地之争,白角哭泣之声,青角有兵忧,黑角则水。意,行穷兵之所终。五星同色,天下偃兵,百姓宁昌。春风秋雨,冬寒夏暑,动摇常以此。

填星出百二十日而逆西行,西行百二十日反东行。见三百三十日而入,入三十日复出东方。太岁在甲寅,镇星在东壁,故在营室。

察日行以处位太白。曰西方,秋,日庚、辛,主杀。杀失者,罚出太白。太白失行,以其舍命国。其出行十八舍二百四十日而入。入东方,伏行十一舍百三十日;其入西方,伏行三舍十六日而出。当出不出,当入不入,是谓失舍,不有破军,必有国君之篡。

其纪上元,以摄提格之岁,与营室晨出东方,至角而入;与营室夕出西方,至角而入;与角晨出,入毕;与角夕出,入毕;与毕晨出,入箕;与毕夕出,入箕;与箕晨出,入柳;与箕夕出,入柳;与柳晨出,入营室;与柳夕出,入营室。凡出入东西各五,为八岁,二百二十日,复与营室晨出东方。其大率,岁一周天。其始出东方,行迟,率日半度,一百二十日,必逆行一二舍;上极而反,东行,行日一度半,一百二十日入。其庳,近日,曰明星,柔;高,远日,曰大嚣,刚。其始出西,行疾,率日一度半,百二十日;上极而行迟,日半度,百二十日,旦入,必逆行一二舍而入。其庳,近日,曰大白,柔;高,远日,曰大相,刚。出以辰、戌,入以丑、未。

当出不出,未当入而入,天下偃兵,兵在外,入。未当出而出,当入而不入,下起兵,有破国。其当期出也,其国昌。其出东为东,入东为北方;出西为西,入西为南方。所居久,其乡利;,其乡凶。

出西至东,正西国吉。出东至西,正东国吉。其出不经天;经天,天下革政。

小以角动,兵起。始出大,后小,兵弱;出小,后大,兵强。出高,用兵深吉,浅凶;庳,浅吉,深凶。日方南金居其南,日方北金居其北,曰赢,侯王不宁,用兵进吉退凶。日方南金居其北,日方北金居其南,曰缩,侯王有忧,用兵退吉进凶。用兵象太白:太白行疾,疾行;迟,迟行。角,敢战。动摇躁,躁。圜以静,静。顺角所指,吉;反之,皆凶。出则出兵,入则入兵。赤角,有战;白角,有丧;黑圜角,忧,有水事;青圜小角,忧,有木事;黄圜和角,有土事,有年。其已出三日而复,有微入,入三日乃复盛出,是谓耎,其下国有军败将北。其已入三日又复微出,出三日而复盛入,其下国有忧;师有粮食兵革,遗人用之;卒虽众,将为人虏。其出西失行,外国败;其出东失行,中国败。其色大圜黄滜,可为好事;其圜大赤,兵盛不战。

太白白,比狼;赤,比心;黄,比参左肩;苍,比参右肩;黑,比奎大星。五星皆从太白而聚乎一舍,其下之国可以兵从天下。居实,有得也;居虚,无得也。行胜色,色胜位,有位胜无位,有色胜无色,行得尽胜之。出而留桑榆间,疾其下国。上而疾,未尽其日,过参天,疾其对国。上复下,下复上,有反将。其入月,将僇。金、木星合,光,其下战不合,兵虽起而不斗;合相毁,野有破军。出西方,昏而出阴,阴兵彊;暮食出,小弱;夜半出,中弱;鸡鸣出,大弱:是谓阴陷于阳。其在东方,乘明而出阳,阳兵之彊,鸡鸣出,小弱;夜半出,中弱;昏出,大弱:是谓阳陷于阴。太白伏也,以出兵,兵有殃。其出卯南,南胜北方;出卯北,北胜南方;正在卯,东国利。出酉北,北胜南方;出酉南,南胜北方;正在酉,西国胜。

其与列星相犯,小战;五星,大战。其相犯,太白出其南,南国败;出其北,北国败。行疾,武;不行,文。色白五芒,出蚤为月蚀,晚为天夭及彗星,将发其国。出东为德,举事左之迎之,吉。出西为刑,举事右之背之,吉。反之皆凶。太白光见景,战胜。昼见而经天,是谓争明,彊国弱,小国彊,女主昌。

亢为疏庙,太白庙也。太白,大臣也,其号上公。其他名殷星、太正、营星、观星、宫星、明星、大衰、大泽、终星、大相、天浩、序星、月纬。大司马位谨候此。

察日辰之会,以治辰星之位。曰北方水,太阴之精,主冬,日壬、癸。刑失者,罚出辰星,以其宿命国。

是正四时:仲春春分,夕出郊奎、娄、胃东五舍,为齐;仲夏夏至,夕出郊东井、舆鬼、柳东七舍,为楚;仲秋秋分,夕出郊角、亢、氐、房东四舍,为汉;仲冬冬至,晨出郊东方,与尾、箕、斗、牵牛俱西,为中国。其出入常以辰、戌、丑、未。

其蚤,为月蚀;晚,为彗星及天夭。其时宜效不效为失,追兵在外不战。一时不出,其时不和;四时不出,天下大饥。其当效而出也,色白为旱,黄为五穀熟,赤为兵,黑为水。出东方,大而白,有兵于外,解。常在东方,其赤,中国胜;其西而赤,外国利。无兵于外而赤,兵起。其与太白俱出东方,皆赤而角,外国大败,中国胜;其与太白俱出西方,皆赤而角,外国利。五星分天之中,积于东方,中国利;积于西方,外国用者利。五星皆从辰星而聚于一舍,其所舍之国可以法致天下。辰星不出,太白为客;其出,太白为主。出而与太白不相从,野虽有军,不战。出东方,太白出西方;若出西方,太白出东方,为格,野虽有兵不战。失其时而出,为当寒反温,当温反寒。当出不出,是谓击卒,兵大起。其入太白中而上出,破军杀将,客军胜;下出,客亡地。辰星来抵太白,太白不去,将死。正旗上出,破军杀将,客胜;下出,客亡地。视旗所指,以命破军。其绕环太白,若与斗,大战,客胜。兔过太白,间可咸剑,小战,客胜。兔居太白前,军罢;出太白左,小战;摩太白,有数万人战,主人吏死;出太白右,去三尺,军急约战。青角,兵忧;黑角,水。赤行穷兵之所终。

兔七命,曰小正、辰星、天欃、安周星、细爽、栖星、钩星。其色黄而小,出而易处,天下之文变而不善矣。兔五色,青圜忧,白圜丧,赤圜中不平,黑圜吉。赤角犯我城,黄角地之争,白角号泣之声。

其出东方,行四舍四十八日,其数二十日,而反入于东方;其出西方,行四舍四十八日,其数二十日,而反入于西方。其一候之营室、角、毕、箕、柳。出房、心间,地动。

辰星之色:春,青黄;夏,赤白;秋,青白,而岁熟;冬,黄而不明。即变其色,其时不昌。春不见,大风,秋则不实。夏不见,有六十日之旱,月蚀。秋不见,有兵,春则不生。冬不见,阴雨六十日,有流邑,夏则不长。

角、亢、氐,兗州。房、心,豫州。尾、箕,幽州。斗,江、湖。牵牛、婺女,杨州。虚、危,青州。营室至东壁,并州。奎、娄、胃,徐州。昴、毕,冀州。觜觿、参,益州。东井、舆鬼,雍州。柳、七星、张,三河。翼、轸,荆州。

七星为员官,辰星庙,蛮夷星也。

两军相当,日晕;晕等,力钧;厚长大,有胜;薄短小,无胜。重抱大破无。抱为和,背不和,为分离相去。直为自立,立侯王;杀将。负且戴,有喜。围在中,中胜;在外,外胜。青外赤中,以和相去;赤外青中,以恶相去。气晕先至而后去,居军胜。先至先去,前利后病;后至后去,前病后利;后至先去,前后皆病,居军不胜。见而去,其发疾,虽胜无功。见半日以上,功大。白虹屈短,上下兑,有者下大流血。日晕制胜,近期三十日,远期六十日。

其食,食所不利;复生,生所利;而食益尽,为主位。以其直及日所宿,加以日时,用命其国也。

月行中道,安宁和平。阴间,多水,阴事。外北三尺,阴星。北三尺,太阴,大水,兵。阳间,骄恣。阳星,多暴狱。太阳,大旱丧也。角、天门,十月为四月,十一月为五月,十二月为六月,水发,近三尺,远五尺。犯四辅,辅臣诛。行南北河,以阴阳言,旱水兵丧。

月蚀岁星,其宿地,饥若亡。荧惑也乱,填星也下犯上,太白也彊国以战败,辰星也女乱。大角,主命者恶之;心,则为内贼乱也;列星,其宿地忧。

月食始日,五月者六,六月者五,五月复六,六月者一,而五月者五,凡百一十三月而复始。故月蚀,常也;日蚀,为不臧也。甲、乙,四海之外,日月不占。丙、丁,江、淮、海岱也。戊、己,中州、河、济也。庚、辛,华山以西。壬、癸,恆山以北。日蚀,国君;月蚀,将相当之。

国皇星,大而赤,状类南极。所出,其下起兵,兵彊;其冲不利。

昭明星,大而白,无角,乍上乍下。所出国,起兵,多变。

五残星,出正东东方之野。其星状类辰星,去地可六丈。

大贼星,出正南南方之野。星去地可六丈,大而赤,数动,有光。

司危星,出正西西方之野。星去地可六丈,大而白,类太白。

狱汉星,出正北北方之野。星去地可六丈,大而赤,数动,察之中青。此四野星所出,出非其方,其下有兵,冲不利。

四填星,所出四隅,去地可四丈。

地维咸光,亦出四隅,去地可三丈,若月始出。所见,下有乱;乱者亡,有德者昌。

烛星,状如太白,其出也不行。见则灭。所烛者,城邑乱。

如星非星,如云非云,命曰归邪。归邪出,必有归国者。

星者,金之散气,本曰火。星众,国吉;少则凶。

汉者,亦金之散气,其本曰水。汉,星多,多水,少则旱,其大经也。

天鼓,有音如雷非雷,音在地而下及地。其所往者,兵发其下。

天狗,状如大奔星,有声,其下止地,类狗。所堕及,望之如火光炎炎冲天。其下圜如数顷田处,上兑者则有黄色,千里破军杀将。

格泽星者,如炎火之状。黄白,起地而上。下大,上兑。其见也,不种而穫;不有土功,必有大害。

蚩尤之旗,类彗而后曲,象旗。见则王者征伐四方。

旬始,出于北斗旁,状如雄鸡。其怒,青黑,象伏鳖。

枉矢,类大流星,行而仓黑,望之如有毛羽然。

长庚,如一匹布著天。此星见,兵起。

星坠至地,则石也。河、济之间,时有坠星。

天精而见景星。景星者,德星也。其状无常,常出于有道之国。

凡望云气,仰而望之,三四百里;平望,在桑榆上,千馀二千里;登高而望之,下属地者三千里。云气有兽居上者,胜。

自华以南,气下黑上赤。嵩高、三河之郊,气正赤。恆山之北,气下黑下青。勃、碣、海、岱之间,气皆黑。江、淮之间,气皆白。

徒气白。土功气黄。车气乍高乍下,往往而聚。骑气卑而布。卒气抟。前卑而后高者,疾;前方而后高者,兑;后兑而卑者,卻。其气平者其行徐。前高而后卑者,不止而反。气相遇者,卑胜高,兑胜方。气来卑而循车通者,不过三四日,去之五六里见。气来高七八尺者,不过五六日,去之十馀里见。气来高丈馀二丈者,不过三四十日,去之五六十里见。

稍云精白者,其将悍,其士怯。其大根而前绝远者,当战。青白,其前低者,战胜;其前赤而仰者,战不胜。阵云如立垣。杼云类杼。轴云抟两端兑。杓云如绳者,居前互天,其半半天。其
者类阙旗故。钩云句曲。诸此云见,以五色合占。而泽抟密,其见动人,乃有占;兵必起,合
其直。

王朔所候,决于日旁。日旁云气,人主象。皆如其形以占。

故北夷之气如群畜穹闾,南夷之气类舟船幡旗。大水处,败军场,破国之虚,下有积钱,金宝之上,皆有气,不可不察。海旁蜄气象楼台;广野气成宫阙然。云气各象其山川人民所聚积。

故候息秏者,入国邑,视封疆田畴之正治,城郭室屋门户之润泽,次至车服畜产精华。实息者,吉;虚秏者,凶。

若烟非烟,若云非云,郁郁纷纷,萧索轮囷,是谓卿云。卿云,喜气也。若雾非雾,衣冠而不濡,见则其域被甲而趋。

雷电、虾虹、辟历、夜明者,阳气之动者也,春夏则发,秋冬则藏,故候者无不司之。

天开县物,地动坼绝。山崩及徙,川塞谿垘;水澹地长,见象。城郭门闾,闺臬枯;宫庙邸第,人民所次。谣俗车服,观民饮食。五穀草木,观其所属。仓府厩库,四通之路。六畜禽兽,所产去就;鱼鳖鸟鼠,观其所处。鬼哭若呼,其人逢俉。化言,诚然。

凡候岁美恶,谨候岁始。岁始或冬至日,产气始萌。腊明日,人众卒岁,一会饮食,发阳气,故曰初岁。正月旦,王者岁首;立春日,四时之始也。四始者,候之日。

而汉魏鲜集腊明正月旦决八风。风从南方来,大旱;西南,小旱;西方,有兵;西北,戎菽为,小雨,趣兵;北方,为中岁;东北,为上岁;东方,大水;东南,民有疾疫,岁恶。故八风各与其冲对,课多者为胜。多胜少,久胜亟,疾胜徐。旦至食,为麦;食至日昳,为稷;昳至餔,为黍;餔至下餔,为菽;下餔至日入,为麻。欲终日有云,有风,有日。日当其时者,深而多实;无云有风日,当其时,浅而多实;有云风,无日,当其时,深而少实;有日,无云,不风,当其时者稼有败。如食顷,小败;熟五斗米顷,大败。则风复起,有云,其稼复起。各以其时用云色占种所宜。其雨雪若寒,岁恶。

是日光明,听都邑人民之声。声宫,则岁善,吉;商,则有兵;徵,旱;羽,水;角,岁恶。

或从正月旦比数雨。率日食一升,至七升而极;过之,不占。数至十二日,日直其月,占水旱。为其环千里内占,则为天下候,竟正月。月所离列宿,日、风、云,占其国。然必察太岁所在。在金,穰;水,毁;木,饥;火,旱。此其大经也。

正月上甲,风从东方,宜蚕;风从西方,若旦黄云,恶。

冬至短极,县土炭,炭动,鹿解角,兰根出,泉水跃,略以知日至,要决晷景。岁星所在,五穀逢昌。其对为冲,岁乃有殃。

太史公曰:自初生民以来,世主曷尝不历日月星辰?及至五家、三代,绍而明之,内冠带,外夷狄,分中国为十有二州,仰则观象于天,俯则法类于地。天则有日月,地则有阴阳。天有五星,地有五行。天则有列宿,地则有州域。三光者,阴阳之精,气本在地,而圣人统理之。

幽厉以往,尚矣。所见天变,皆国殊窟穴,家占物怪,以合时应,其文图籍禨祥不法。是以孔子论六经,纪异而说不书。至天道命,不传;传其人,不待告;告非其人,虽言不著。

昔之传天数者:高辛之前,重、黎;于唐、虞,羲、和;有夏,昆吾;殷商,巫咸;周室,史佚、苌弘;于宋,子韦;郑则裨灶;在齐,甘公;楚,唐眛;赵,尹皋;魏,石申。

夫天运,三十岁一小变,百年中变,五百载大变;三大变一纪,三纪而大备:此其大数也。为国者必贵三五。上下各千岁,然后天人之际续备。

太史公推古天变,未有可考于今者。盖略以春秋二百四十二年之间,日蚀三十六,彗星三见,宋襄公时星陨如雨。天子微,诸侯力政,五伯代兴,更为主命,自是之后,众暴寡,大并小。秦、楚、吴、越,夷狄也,为彊伯。田氏篡齐,三家分晋,并为战国。争于攻取,兵革更起,城邑数屠,因以饥馑疾疫焦苦,臣主共忧患,其察礻几祥候星气尤急。近世十二诸侯七国相王,言从衡者继踵,而皋、唐、甘、石因时务论其书传,故其占验凌杂米盐。

二十八舍主十二州,斗秉兼之,所从来久矣。秦之疆也,候在太白,占于狼、弧。吴、楚之疆,候在荧惑,占于鸟衡。燕、齐之疆,候在辰星,占于虚、危。宋、郑之疆,候在岁星,占于房、心。晋之疆,亦候在辰星,占于参罚。

及秦并吞三晋、燕、代,自河山以南者中国。中国于四海内则在东南,为阳;阳则日、岁星、荧惑、填星;占于街南,毕主之。其西北则胡、貉、月氏诸衣旃裘引弓之民,为阴;阴则月、太白、辰星;占于街北,昴主之。故中国山川东北流,其维,首在陇、蜀,尾没于勃、碣。是以秦、晋好用兵,复占太白,太白主中国;而胡、貉数侵掠,独占辰星,辰星出入躁疾,常主夷狄:其大经也。此更为客主人。荧惑为孛,外则理兵,内则理政。故曰“虽有明天子,必视荧惑所在”。诸侯更彊,时菑异记,无可录者。

秦始皇之时,十五年彗星四见,久者八十日,长或竟天。其后秦遂以兵灭六王,并中国,外攘四夷,死人如乱麻,因以张楚并起,三十年之间兵相骀藉,不可胜数。自蚩尤以来,未尝若斯也。

项羽救钜鹿,枉矢西流,山东遂合从诸侯,西坑秦人,诛屠咸阳。

汉之兴,五星聚于东井。平城之围,月晕参、毕七重。诸吕作乱,日蚀,昼晦。吴楚七国叛逆,彗星数丈,天狗过梁野;及兵起,遂伏尸流血其下。元光、元狩,蚩尤之旗再见,长则半天。其后京师师四出,诛夷狄者数十年,而伐胡尤甚。越之亡,荧惑守斗;朝鲜之拔,星茀于河戍;兵征大宛,星茀招摇:此其荦荦大者。若至委曲小变,不可胜道。由是观之,未有不先形见而应随之者也。

夫自汉之为天数者,星则唐都,气则王朔,占岁则魏鲜。故甘、石历五星法,唯独荧惑有反逆行;逆行所守,及他星逆行,日月薄蚀,皆以为占。

余观史记,考行事,百年之中,五星无出而不反逆行,反逆行,尝盛大而变色;日月薄蚀,行南北有时:此其大度也。故紫宫、房心、权衡、咸池、虚危列宿部星,此天之五官坐位也,为经,不移徙,大小有差,阔狭有常。水、火、金、木、填星,此五星者,天之五佐,为纬,见伏有时,所过行赢缩有度。

日变脩德,月变省刑,星变结和。凡天变,过度乃占。国君彊大,有德者昌;羽小,饰诈者亡。太上脩德,其次脩政,其次脩救,其次脩禳,正下无之。夫常星之变希见,而三光之占亟用。日月晕適,云风,此天之客气,其发见亦有大运。然其与政事俯仰,最近人之符。此五者,天之感动。为天数者,必通三五。终始古今,深观时变,察其精粗,则天官备矣。

苍帝行德,天门为之开。赤帝行德,天牢为之空。黄帝行德,天夭为之起。风从西北来,必以庚、辛。一秋中,五至,大赦;三至,小赦。白帝行德,以正月二十日、二十一日,月晕围,常大赦载,谓有太阳也。一曰:白帝行德,毕、昴为之围。围三暮,德乃成;不三暮,及围不合,德不成。二曰:以辰围,不出其旬。黑帝行德,天关为之动。天行德,天子更立年;不德,风雨破石。三能、三衡者,天廷也。客星出天廷,有奇令。

在天成象,有同影响。观文察变,其来自往。天官既书,太史攸掌。云物必记,星辰可仰。盈缩匪,应验无爽。至哉玄监,云谁欲!
\end{yuanwen}

\chapter{封禅书}

\begin{yuanwen}
自古受命帝王,曷尝不封禅?盖有无其应而用事者矣,未有睹符瑞见而不臻乎泰山者也。虽受命而功不至,至梁父矣而德不洽,洽矣而日有不暇给,是以即事用希。传曰:“三年不为礼,礼必废;三年不为乐,乐必坏。”每世之隆,则封禅答焉,及衰而息。厥旷远者千有馀载,近者数百载,故其仪阙然堙灭,其详不可得而记闻云。

尚书曰,舜在璇玑玉衡,以齐七政。遂类于上帝,禋于六宗,望山川,遍群神。辑五瑞,择吉月日,见四岳诸牧,还瑞。岁二月,东巡狩,至于岱宗。岱宗,泰山也。柴,望秩于山川。遂觐东后。东后者,诸侯也。合时月正日,同律度量衡,修五礼,五玉三帛二生一死贽。五月,巡狩至南岳。南岳,衡山也。八月,巡狩至西岳。西岳,华山也。十一月,巡狩至北岳。北岳,恆山也。皆如岱宗之礼。中岳,嵩高也。五载一巡狩。

禹遵之。后十四世,至帝孔甲,淫德好神,神渎,二龙去之。其后三世,汤伐桀,欲迁夏社,不可,作夏社。后八世,至帝太戊,有桑穀生于廷,一暮大拱,惧。伊陟曰:“妖不胜德。”太戊修德,桑穀死。伊陟赞巫咸,巫咸之兴自此始。后十四世,帝武丁得傅说为相,殷复兴焉,称高宗。有雉登鼎耳雊,武丁惧。祖己曰:“修德。”武丁从之,位以永宁。后五世,帝武乙慢神而震死。后三世,帝纣淫乱,武王伐之。由此观之,始未尝不肃祗,后稍怠慢也。

周官曰,冬日至,祀天于南郊,迎长日之至;夏日至,祭地祗。皆用乐舞,而神乃可得而礼也。天子祭天下名山大川,五岳视三公,四渎视诸侯,诸侯祭其疆内名山大川。四渎者,江、河、淮、济也。天子曰明堂、辟雍,诸侯曰泮宫。

周公既相成王,郊祀后稷以配天,宗祀文王于明堂以配上帝。自禹兴而修社祀,后稷稼穑,故有稷祠,郊社所从来尚矣。

自周克殷后十四世,世益衰,礼乐废,诸侯恣行,而幽王为犬戎所败,周东徙雒邑。秦襄公攻戎救周,始列为诸侯。秦襄公既侯,居西垂,自以为主少昚之神,作西畤,祠白帝,其牲用
驹黄牛羝羊各一云。其后十六年,秦文公东猎汧渭之间,卜居之而吉。文公梦黄蛇自天下属地,其口止于鄜衍。文公问史敦,敦曰:“此上帝之徵,君其祠之。”于是作鄜畤,用三牲郊祭白帝焉。

自未作鄜畤也,而雍旁故有吴阳武畤,雍东有好畤,皆废无祠。或曰:“自古以雍州积高,神明之隩,故立畤郊上帝,诸神祠皆聚云。盖黄帝时尝用事,虽晚周亦郊焉。”其语不经见,缙绅者不道。

作鄜畤后九年,文公获若石云,于陈仓北阪城祠之。其神或岁不至,或岁数来,来也常以夜,光辉若流星,从东南来集于祠城,则若雄鸡,其声殷云,野鸡夜雊。以一牢祠,命曰陈宝。

作鄜畤后七十八年,秦德公既立,卜居雍,“后子孙饮马于河”,遂都雍。雍之诸祠自此兴。用三百牢于鄜畤。作伏祠。磔狗邑四门,以御蛊菑。

德公立二年卒。其后年,秦宣公作密畤于渭南,祭青帝。

其后十四年,秦缪公立,病卧五日不寤;寤,乃言梦见上帝,上帝命缪公平晋乱。史书而记藏之府。而后世皆曰秦缪公上天。

秦缪公即位九年,齐桓公既霸,会诸侯于葵丘,而欲封禅。管仲曰:“古者封泰山禅梁父者七十二家,而夷吾所记者十有二焉。昔无怀氏封泰山,禅云云;虙羲封泰山,禅云云;神农封泰山,禅云云;炎帝封泰山,禅云云;黄帝封泰山,禅亭亭;颛顼封泰山,禅云云;帝幹封泰山,禅云云;尧封泰山,禅云云;舜封泰山,禅云云;禹封泰山,禅会稽;汤封泰山,禅云云;周成王封泰山,禅社首:皆受命然后得封禅。”桓公曰:“寡人北伐山戎,过孤竹;西伐大夏,涉流沙,束马悬车,上卑耳之山;南伐至召陵,登熊耳山以望江汉。兵车之会三,而乘车之会六,九合诸侯,一匡天下,诸侯莫违我。昔三代受命,亦何以异乎?”于是管仲睹桓公不可穷以辞,因设之以事,曰:“古之封禅,鄗上之黍,北里之禾,所以为盛;江淮之间,一茅三脊,所以为藉也。东海致比目之鱼,西海致比翼之鸟,然后物有不召而自至者十有五焉。今凤皇麒麟不来,嘉穀不生,而蓬蒿藜莠茂,鸱枭数至,而欲封禅,毋乃不可乎?”于是桓公乃止。是岁,秦缪公内晋君夷吾。其后三置晋国之君,平其乱。缪公立三十九年而卒。

其后百有馀年,而孔子论述六,传略言易姓而王,封泰山禅乎梁父者七十馀王矣,其俎豆之礼不章,盖难言之。或问禘之说,孔子曰:“不知。知禘之说,其于天下也视其掌。”诗云纣在位,文王受命,政不及泰山。武王克殷二年,天下未宁而崩。爰周德之洽维成王,成王之封禅则近之矣。及后陪臣执政,季氏旅于泰山,仲尼讥之。

是时苌弘以方事周灵王,诸侯莫朝周,周力少,苌弘乃明鬼神事,设射貍首。貍首者,诸侯之不来者。依物怪欲以致诸侯。诸侯不从,而晋人执杀苌弘。周人之言方怪者自苌弘。

其后百馀年,秦灵公作吴阳上畤,祭黄帝;作下畤,祭炎帝。

后四十八年,周太史儋见秦献公曰:“秦始与周合,合而离,五百岁当复合,合十七年而霸王出焉。”栎阳雨金,秦献公自以为得金瑞,故作畦畤栎阳而祀白帝。

其后百二十岁而秦灭周,周之九鼎入于秦。或曰宋太丘社亡,而鼎没于泗水彭城下。

其后百一十五年而秦并天下。

秦始皇既并天下而帝,或曰:“黄帝得土德,黄龙地螾见。夏得木德,青龙止于郊,草木暢茂。殷得金德,银自山溢。周得火德,有赤乌之符。今秦变周,水德之时。昔秦文公出猎,获黑龙,此其水德之瑞。”于是秦更命河曰“德水”,以冬十月为年首,色上黑,度以六为名,音上大吕,事统上法。

即帝位三年,东巡郡县,祠驺峄山,颂秦功业。于是徵从齐鲁之儒生博士七十人,至乎泰山下。诸儒生或议曰:“古者封禅为蒲车,恶伤山之土石草木;埽地而祭,席用菹秸,言其易遵也。”始皇闻此议各乖异,难施用,由此绌儒生。而遂除车道,上自泰山阳至巅,立石颂秦始皇帝德,明其得封也。从阴道下,禅于梁父。其礼颇采太祝之祀雍上帝所用,而封藏皆祕之,世不得而记也。

始皇之上泰山,中阪遇暴风雨,休于大树下。诸儒生既绌,不得与用于封事之礼,闻始皇遇风雨,则讥之。

于是始皇遂东游海上,行礼祠名山大川及八神,求仙人羡门之属。八神将自古而有之,或曰太公以来作之。齐所以为齐,以天齐也。其祀绝莫知起时。八神:一曰天主,祠天齐。天齐渊水,居临菑南郊山下者。二曰地主,祠泰山梁父。盖天好阴,祠之必于高山之下,小山之上,命曰“畤”;地贵阳,祭之必于泽中圜丘云。三曰兵主,祠蚩尤。蚩尤在东平陆监乡,齐之西境也。四曰阴主,祠三山。五曰阳主,祠之罘。六曰月主,祠之莱山。皆在齐北,并勃海。七曰日主,祠成山。成山斗入海,最居齐东北隅,以迎日出云。八曰四时主,祠琅邪。琅邪在齐东方,盖岁之所始。皆各用一牢具祠,而巫祝所损益,珪币杂异焉。

自齐威、宣之时,驺子之徒论著终始五德之运,及秦帝而齐人奏之,故始皇采用之。而宋毋忌、正伯侨、充尚、羡门高最后皆燕人,为方仙道,形解销化,依于鬼神之事。驺衍以阴阳主运显于诸侯,而燕齐海上之方士传其术不能通,然则怪迂阿谀苟合之徒自此兴,不可胜数也。

自威、宣、燕昭使人入海求蓬莱、方丈、瀛洲。此三神山者,其傅在勃海中,去人不远;患且至,则船风引而去。盖尝有至者,诸仙人及不死之药皆在焉。其物禽兽尽白,而黄金银为宫阙。未至,望之如云;及到,三神山反居水下。临之,风辄引去,终莫能至云。世主莫不甘心焉。及至秦始皇并天下,至海上,则方士言之不可胜数。始皇自以为至海上而恐不及矣,使人乃赍童男女入海求之。船交海中,皆以风为解,曰未能至,望见之焉。其明年,始皇复游海上,至琅邪,过恆山,从上党归。后三年,游碣石,考入海方士,从上郡归。后五年,始皇南至湘山,遂登会稽,并海上,冀遇海中三神山之奇药。不得,还至沙丘崩。

二世元年,东巡碣石,并海南,历泰山,至会稽,皆礼祠之,而刻勒始皇所立石书旁,以章始皇之功德。其秋,诸侯畔秦。三年而二世弑死。

始皇封禅之后十二岁,秦亡。诸儒生疾秦焚诗书,诛僇文学,百姓怨其法,天下畔之,皆讹曰:“始皇上泰山,为暴风雨所击,不得封禅。”此岂所谓无其德而用事者邪?

昔三代之皆在河洛之间,故嵩高为中岳,而四岳各如其方,四渎咸在山东。至秦称帝,都咸阳,则五岳、四渎皆并在东方。自五帝以至秦,轶兴轶衰,名山大川或在诸侯,或在天子,其礼损益世殊,不可胜记。及秦并天下,令祠官所常奉天地名山大川鬼神可得而序也。

于是自殽以东,名山五,大川祠二。曰太室。太室,嵩高也。恆山,泰山,会稽,湘山。水曰济,曰淮。春以脯酒为岁祠,因泮冻,秋涸冻,冬塞祷祠。其牲用牛犊各一,牢具珪币各异。

自华以西,名山七,名川四。曰华山,薄山。薄山者,衰山也。岳山,岐山,吴岳,鸿冢,渎山。渎山,蜀之汶山。水曰河,祠临晋;沔,祠汉中;湫渊,祠朝;江水,祠蜀。亦春秋泮涸祷塞,如东方名山川;而牲牛犊牢具珪币各异。而四大冢鸿、岐、吴、岳,皆有尝禾。

陈宝节来祠。其河加有尝醪。此皆在雍州之域,近天子之都,故加车一乘,
驹四。

霸、产、长水、沣、涝、泾、渭皆非大川,以近咸阳,尽得比山川祠,而无诸加。

汧、洛二渊,鸣泽、蒲山、岳鞚山之属,为小山川,亦皆岁祷塞泮涸祠,礼不必同。

而雍有日、月、参、辰、南北斗、荧惑、太白、岁星、填星、、二十八宿、风伯、雨师、四海、九臣、十四臣、诸布、诸严、诸逑之属,百有馀庙。西亦有数十祠。于湖有周天子祠。于下邽有天神。沣、滈有昭明、天子辟池。于、亳有三社主之祠、寿星祠;而雍菅庙亦有杜主。杜主,故周之右将军,其在秦中,最小鬼之神者。各以岁时奉祠。

唯雍四畤上帝为尊,其光景动人民唯陈宝。故雍四畤,春以为岁祷,因泮冻,秋涸冻,冬塞祠,五月尝驹,及四仲之月月祠,陈宝节来一祠。春夏用骍,秋冬用
。畤驹四匹,木禺龙栾车一驷,木禺车马一驷,各如其帝色。黄犊羔各四,珪币各有数,皆生瘗埋,无俎豆之具。三年一郊。秦以冬十月为岁首,故常以十月上宿郊见,通权火,拜于咸阳之旁,而衣上白,其用如经祠云。西畤、畦畤,祠如其故,上不亲往。

诸此祠皆太祝常主,以岁时奉祠之。至如他名山川诸鬼及八神之属,上过则祠,去则已。郡县远方神祠者,民各自奉祠,不领于天子之祝官。祝官有祕祝,即有菑祥,辄祝祠移过于下。

汉兴,高祖之微时,尝杀大蛇。有物曰:“蛇,白帝子也,而杀者赤帝子。”高祖初起,祷丰枌榆社。徇沛,为沛公,则祠蚩尤,衅鼓旗。遂以十月至灞上,与诸侯平咸阳,立为汉王。因以十月为年首,而色上赤。

二年,东击项籍而还入关,问:“故秦时上帝祠何帝也?”对曰:“四帝,有白、青、黄、赤帝之祠。”高祖曰:“吾闻天有五帝,而有四,何也?”莫知其说。于是高祖曰:“吾知之矣,乃待我而具五也。”乃立黑帝祠,命曰北畤。有司进祠,上不亲往。悉召故秦祝官,复置太祝、太宰,如其故仪礼。因令县为公社。下诏曰:“吾甚重祠而敬祭。今上帝之祭及山川诸神当祠者,各以其时礼祠之如故。”

后四岁,天下已定,诏御史,令丰谨治枌榆社,常以四时春以羊彘祠之。令祝官立蚩尤之祠于长安。长安置祠祝官、女巫。其梁巫,祠天、地、天社、天水、房中、堂上之属;晋巫,祠五帝、东君、云中、司命、巫社、巫祠、族人、先炊之属;秦巫,祠社主、巫保、族累之属;荆巫,祠堂下、巫先、司命、施糜之属;九天巫,祠九天:皆以岁时祠宫中。其河巫祠河于临晋,而南山巫祠南山秦中。秦中者,二世皇帝。各有时。

其后二岁,或曰周兴而邑邰,立后稷之祠,至今血食天下。于是高祖制诏御史:“其令郡国县立灵星祠,常以岁时祠以牛。”

高祖十年春,有司请令县常以春月及腊祠社稷以羊豕,民里社各自财以祠。制曰:“可。”

其后十八年,孝文帝即位。即位十三年,下诏曰:“今祕祝移过于下,朕甚不取。自今除之。”

始名山大川在诸侯,诸侯祝各自奉祠,天子官不领。及齐、淮南国废,令太祝尽以岁时致礼如故。

是岁,制曰:“朕即位十三年于今,赖宗庙之灵,社稷之福,方内艾安,民人靡疾。间者比年登,朕之不德,何以飨此?皆上帝诸神之赐也。盖闻古者飨其德必报其功,欲有增诸神祠。有司议增雍五畤路车各一乘,驾被具;西畤畦畤禺车各一乘,禺马四匹,驾被具;其河、湫、汉水加玉各二;及诸祠,各增广坛场,珪币俎豆以差加之。而祝釐者归福于朕,百姓不与焉。自今祝致敬,毋有所祈。”

鲁人公孙臣上书曰:“始秦得水德,今汉受之,推终始传,则汉当土德,土德之应黄龙见。宜改正朔,易服色,色上黄。”是时丞相张苍好律历,以为汉乃水德之始,故河决金隄,其符也。年始冬十月,色外黑内赤,与德相应。如公孙臣言,非也。罢之。后三岁,黄龙见成纪。文帝乃召公孙臣,拜为博士,与诸生草改历服色事。其夏,下诏曰:“异物之神见于成纪,无害于民,岁以有年。朕祈郊上帝诸神,礼官议,无讳以劳朕。”有司皆曰“古者天子夏亲郊,祀上帝于郊,故曰郊”。于是夏四月,文帝始郊见雍五畤祠,衣皆上赤。

其明年,赵人新垣平以望气见上,言“长安东北有神气,成五采,若人冠纟免焉。或曰东北神明之舍,西方神明之墓也。天瑞下,宜立祠上帝,以合符应”。于是作渭阳五帝庙,同宇,帝一殿,面各五门,各如其帝色。祠所用及仪亦如雍五畤。

夏四月,文帝亲拜霸渭之会,以郊见渭阳五帝。五帝庙南临渭,北穿蒲池沟水,权火举而祠,若光煇然属天焉。于是贵平上大夫,赐累千金。而使博士诸生刺六经中作王制,谋议巡狩封禅事。

文帝出长门,若见五人于道北,遂因其直北立五帝坛,祠以五牢具。

其明年,新垣平使人持玉杯,上书阙下献之。平言上曰:“阙下有宝玉气来者。”已视之,果有献玉杯者,刻曰“人主延寿”。平又言“臣候日再中”。居顷之,日卻复中。于是始更以十七年为元年,令天下大酺。

平言曰:“周鼎亡在泗水中,今河溢通泗,臣望东北汾阴直有金宝气,意周鼎其出乎?兆见不迎则不至。”于是上使使治庙汾阴南,临河,欲祠出周鼎。

人有上书告新垣平所言气神事皆诈也。下平吏治,诛夷新垣平。自是之后,文帝怠于改正朔服色神明之事,而渭阳、长门五帝使祠官领,以时致礼,不往焉。

明年,匈奴数入边,兴兵守御。后岁少不登。

数年而孝景即位。十六年,祠官各以岁时祠如故,无有所兴,至今天子。

今天子初即位,尤敬鬼神之祀。

元年,汉兴已六十馀岁矣,天下艾安,搢绅之属皆望天子封禅改正度也,而上乡儒术,招贤良,赵绾、王臧等以文学为公卿,欲议古立明堂城南,以朝诸侯。草巡狩封禅改历服色事未就。会窦太后治黄老言,不好儒术,使人微伺得赵绾等奸利事,召案绾、臧,绾、臧自杀,诸所兴为皆废。

后六年,窦太后崩。其明年,徵文学之士公孙弘等。

明年,今上初至雍,郊见五畤。后常三岁一郊。是时上求神君,舍之上林中氾氏观。神君者,长陵女子,以子死,见神于先后宛若。宛若祠之其室,民多往祠。平原君往祠,其后子孙以尊显。及今上即位,则厚礼置祠之内中。闻其言,不见其人云。

是时李少君亦以祠灶、穀道、卻老方见上,上尊之。少君者,故深泽侯舍人,主方。匿其年及其生长,常自谓七十,能使物,卻老。其游以方遍诸侯。无妻子。人闻其能使物及不死,更馈遗之,常馀金钱衣食。人皆以为不治生业而饶给,又不知其何所人,愈信,争事之。少君资好方,善为巧发奇中。尝从武安侯饮,坐中有九十馀老人,少君乃言与其大父游射处,老人为兒时从其大父,识其处,一坐尽惊。少君见上,上有故铜器,问少君。少君曰:“此器齐桓公十年陈于柏寝。”已而案其刻,果齐桓公器。一宫尽骇,以为少君神,数百岁人也。少君言上曰:“祠灶则致物,致物而丹沙可化为黄金,黄金成以为饮食器则益寿,益寿而海中蓬莱仙者乃可见,见之以封禅则不死,黄帝是也。臣尝游海上,见安期生,安期生食巨枣,大如瓜。安期生仙者,通蓬莱中,合则见人,不合则隐。”于是天子始亲祠灶,遣方士入海求蓬莱安期生之属,而事化丹沙诸药齐为黄金矣。

居久之,李少君病死。天子以为化去不死,而使黄锤史宽舒受其方。求蓬莱安期生莫能得,而海上燕齐怪迂之方士多更来言神事矣。

亳人谬忌奏祠太一方,曰:“天神贵者太一,太一佐曰五帝。古者天子以春秋祭太一东南郊,用太牢,七日,为坛开八通之鬼道。”于是天子令太祝立其祠长安东南郊,常奉祠如忌方。其后人有上书,言“古者天子三年壹用太牢祠神三一:天一、地一、太一”。天子许之,令太祝领祠之于忌太一坛上,如其方。后人复有上书,言“古者天子常以春解祠,祠黄帝用一枭破镜;冥羊用羊祠;马行用一青牡马;太一、泽山君地长用牛;武夷君用乾鱼;阴阳使者以一牛”。令祠官领之如其方,而祠于忌太一坛旁。

其后,天子苑有白鹿,以其皮为币,以发瑞应,造白金焉。

其明年,郊雍,获一角兽,若麃然。有司曰:“陛下肃祗郊祀,上帝报享,锡一角兽,盖麟云。”于是以荐五畤,畤加一牛以燎。锡诸侯白金,风符应合于天也。

于是济北王以为天子且封禅,乃上书献太山及其旁邑,天子以他县偿之。常山王有罪,迁,天子封其弟于真定,以续先王祀,而以常山为郡,然后五岳皆在天子之。

其明年,齐人少翁以鬼神方见上。上有所幸王夫人,夫人卒,少翁以方盖夜致王夫人及灶鬼之貌云,天子自帷中望见焉。于是乃拜少翁为文成将军,赏赐甚多,以客礼礼之。文成言曰:“上即欲与神通,宫室被服非象神,神物不至。”乃作画云气车,及各以胜日驾车辟恶鬼。又作甘泉宫,中为台室,画天、地、太一诸鬼神,而置祭具以致天神。居岁馀,其方益衰,神不至。乃为帛书以饭牛,详不知,言曰此牛腹中有奇。杀视得书,书言甚怪。天子识其手书,问其人,果是伪书,于是诛文成将军,隐之。

其后则又作柏梁、铜柱、承露仙人掌之属矣。

文成死明年,天子病鼎湖甚,巫医无所不致,不愈。游水发根言上郡有巫,病而鬼神下之。上召置祠之甘泉。及病,使人问神君。神君言曰:“天子无忧病。病少愈,彊与我会甘泉。”于是病愈,遂起,幸甘泉,病良已。大赦,置寿宫神君。寿宫神君最贵者太一,其佐曰大禁、司命之属,皆从之。非可得见,闻其言,言与人音等。时去时来,来则风肃然。居室帷中。时昼言,然常以夜。天子祓,然后入。因巫为主人,关饮食。所以言,行下。又置寿宫、北宫,张羽旗,设供具,以礼神君。神君所言,上使人受书其言,命之曰“画法”。其所语,世俗之所知也,无绝殊者,而天子心独喜。其事祕,世莫知也。

其后三年,有司言元宜以天瑞命,不宜以一二数。一元曰“建”,二元以长星曰“光”,三元以郊得一角兽曰“狩”云。

其明年冬,天子郊雍,议曰:“今上帝朕亲郊,而后土无祀,则礼不答也。”有司与太史公、祠官宽舒议:“天地牲角茧栗。今陛下亲祠后土,后土宜于泽中圜丘为五坛,坛一黄犊太牢具,已祠尽瘗,而从祠衣上黄。”于是天子遂东,始立后土祠汾阴脽丘,如宽舒等议。上亲望拜,如上帝礼。礼毕,天子遂至荥阳而还。过雒阳,下诏曰:“三代邈绝,远矣难存。其以三十里地封周后为周子南君,以奉其先祀焉。”是岁,天子始巡郡县,侵寻于泰山矣。

其春,乐成侯上书言栾大。栾大,胶东宫人,故尝与文成将军同师,已而为胶东王尚方。而乐成侯姊为康王后,无子。康王死,他姬子立为王。而康后有淫行,与王不相中,相危以法。康后闻文成已死,而欲自媚于上,乃遣栾大因乐成侯求见言方。天子既诛文成,后悔其蚤死,惜其方不尽,及见栾大,大说。大为人长美,言多方略,而敢为大言处之不疑。大言曰:“臣常往来海中,见安期、羡门之属。顾以臣为贱,不信臣。又以为康王诸侯耳,不足与方。臣数言康王,康王又不用臣。臣之师曰:‘黄金可成,而河决可塞,不死之药可得,仙人可致也。’然臣恐效文成,则方士皆奄口,恶敢言方哉!”上曰:“文成食马肝死耳。子诚能脩其方,我何爱乎!”大曰:“臣师非有求人,人者求之。陛下必欲致之,则贵其使者,令有亲属,以客礼待之,勿卑,使各佩其信印,乃可使通言于神人。神人尚肯邪不邪。致尊其使,然后可致也。”于是上使验小方,斗棋,棋自相触击。

是时上方忧河决,而黄金不就,乃拜大为五利将军。居月馀,得四印,佩天士将军、地士将军、大通将军印。制诏御史:“昔禹疏九江,决四渎。间者河溢皋陆,隄繇不息。朕临天下二十有八年,天若遗朕士而大通焉。乾称‘蜚龙’,‘鸿渐于般’,朕意庶几与焉。其以二千户封地士将军大为乐通侯。”赐列侯甲第,僮千人。乘轝斥车马帷幄器物以充其家。又以卫长公主妻之,赍金万斤,更命其邑曰当利公主。天子亲如五利之第。使者存问供给,相属于道。自大主将相以下,皆置酒其家,献遗之。于是天子又刻玉印曰“天道将军”,使使衣羽衣,夜立白茅上,五利将军亦衣羽衣,夜立白茅上受印,以示不臣也。而佩“天道”者,且为天子道天神也。于是五利常夜祠其家,欲以下神。神未至而百鬼集矣,然颇能使之。其后装治行,东入海,求其师云。大见数月,佩六印,贵震天下,而海上燕齐之间,莫不搤捥而自言有禁方,能神仙矣。

其夏六月中,汾阴巫锦为民祠魏脽后土营旁,见地如钩状,掊视得鼎。鼎大异于众鼎,文镂无款识,怪之,言吏。吏告河东太守胜,胜以闻。天子使使验问巫得鼎无奸诈,乃以礼祠,迎鼎至甘泉,从行,上荐之。至中山,曣翚,有黄云盖焉。有麃过,上自射之,因以祭云。至长安,公卿大夫皆议请尊宝鼎。天子曰:“间者河溢,岁数不登,故巡祭后土,祈为百姓育穀。今岁丰庑未报,鼎曷为出哉?”有司皆曰:“闻昔泰帝兴神鼎一,一者壹统,天地万物所系终也。黄帝作宝鼎三,象天地人。禹收九牧之金,铸九鼎。皆尝亨
上帝鬼神。遭圣则兴,鼎迁于夏商。周德衰,宋之社亡,鼎乃沦没,伏而不见。颂云‘自堂徂基,自羊徂牛;鼐鼎及鼒,不吴不骜,胡考之休’。今鼎至甘泉,光润龙变,承休无疆。合兹中山,有黄白云降盖,若兽为符,路弓乘矢,集获坛下,报祠大享。唯受命而帝者心知其意而合德焉。鼎宜见于祖祢,藏于帝廷,以合明应。”制曰:“可。”

入海求蓬莱者,言蓬莱不远,而不能至者,殆不见其气。上乃遣望气佐候其气云。

其秋,上幸雍,且郊。或曰“五帝,太一之佐也,宜立太一而上亲郊之”。上疑未定。齐人公孙卿曰:“今年得宝鼎,其冬辛巳朔旦冬至,与黄帝时等。”卿有札书曰:“黄帝得宝鼎宛朐,问于鬼臾区。鬼臾区对曰:‘帝得宝鼎神策,是岁己酉朔旦冬至,得天之纪,终而复始。’于是黄帝迎日推策,后率二十岁复朔旦冬至,凡二十推,三百八十年,黄帝仙登于天。”卿因所忠欲奏之。所忠视其书不经,疑其妄书,谢曰:“宝鼎事已决矣,尚何以为!”卿因嬖人奏之。上大说,乃召问卿。对曰:“受此书申公,申公已死。”上曰:“申公何人也?”卿曰:“申公,齐人。与安期生通,受黄帝言,无书,独有此鼎书。曰‘汉兴复当黄帝之时’。曰‘汉之圣者在高祖之孙且曾孙也。宝鼎出而与神通,封禅。封禅七十二王,唯黄帝得上泰山封’。申公曰:‘汉主亦当上封,上封能仙登天矣。黄帝时万诸侯,而神灵之封居七千。天下名山八,而三在蛮夷,五在中国。中国华山、首山、太室、泰山、东莱,此五山黄帝之所常游,与神会。黄帝且战且学仙。患百姓非其道者,乃断斩非鬼神者。百馀岁然后得与神通。黄帝郊雍上帝,宿三月。鬼臾区号大鸿,死葬雍,故鸿冢是也。其后黄帝接万灵明廷。明廷者,甘泉也。所谓寒门者,谷口也。黄帝采首山铜,铸鼎于荆山下。鼎既成,有龙垂胡珣下迎黄帝。黄帝上骑,群臣后宫从上者七十馀人,龙乃上去。馀小臣不得上,乃悉持龙珣,龙珣拔,堕,堕黄帝之弓。百姓仰望黄帝既上天,乃抱其弓与胡珣号,故后世因名其处曰鼎湖,其弓曰乌号。’”于是天子曰:“嗟乎!吾诚得如黄帝,吾视去妻子如脱鵕耳。”乃拜卿为郎,东使候神于太室。

上遂郊雍,至陇西,西登崆峒,幸甘泉。令祠官宽舒等具太一祠坛,祠坛放薄忌太一坛,坛三垓。五帝坛环居其下,各如其方,黄帝西南,除八通鬼道。太一,其所用如雍一畤物,而加醴枣脯之属,杀一貍牛以为俎豆牢具。而五帝独有俎豆醴进。其下四方地,为醊食群神从者及北斗云。已祠,胙馀皆燎之。其牛色白,鹿居其中,彘在鹿中,水而洎之。祭日以牛,祭月以羊彘特。太一祝宰则衣紫及绣。五帝各如其色,日赤,月白。

十一月辛巳朔旦冬至,昧爽,天字始郊拜太一。朝朝日,夕夕月,则揖;而见太一如雍郊礼。其赞飨曰:“天始以宝鼎神策授皇帝,朔而又朔,终而复始,皇帝敬拜见焉。”而衣上黄。其祠列火满坛,坛旁亨炊具。有司云“祠上有光焉”。公卿言“皇帝始郊见太一云阳,有司奉瑄玉嘉牲荐飨。是夜有美光,及昼,黄气上属天”。太史公、祠官宽舒等曰:“神灵之休,祐福兆祥,宜因此地光域立太畤坛以明应。令太祝领,秋及腊间祠。三岁天子一郊见。”

其秋,为伐南越,告祷太一。以牡荆画幡日月北斗登龙,以象太一三星,为太一锋,命曰“灵旗”。为兵祷,则太史奉以指所伐国。而五利将军使不敢入海,之泰山祠。上使人随验,实毋所见。五利妄言见其师,其方尽,多不雠。上乃诛五利。

其冬,公孙卿候神河南,言见仙人迹缑氏城上,有物如雉,往来城上。天子亲幸缑氏城视迹。问卿:“得毋效文成、五利乎?”卿曰:“仙者非有求人主,人主者求之。其道非少宽假,神不来。言神事,事如迂诞,积以岁乃可致也。”于是郡国各除道,缮治宫观名山神祠所,以望幸。

其春,既灭南越,上有嬖臣李延年以好音见。上善之,下公卿议,曰:“民间祠尚有鼓舞乐,今郊祀而无乐,岂称乎?”公卿曰:“古者祠天地皆有乐,而神祇可得而礼。”或曰:“太帝使素女鼓五十弦瑟,悲,帝禁不止,故破其瑟为二十五弦。”于是塞南越,祷祠太一、后土,始用乐舞,益召歌兒,作二十五弦及空侯琴瑟自此起。

其来年冬,上议曰:“古者先振兵泽旅,然后封禅。”乃遂北巡朔方,勒兵十馀万,还祭黄帝冢桥山,释兵须如。上曰:“吾闻黄帝不死,今有冢,何也?”或对曰:“黄帝已仙上天,群臣葬其衣冠。”既至甘泉,为且用事泰山,先类祠太一。

自得宝鼎,上与公卿诸生议封禅。封禅用希旷绝,莫知其仪礼,而群儒采封禅尚书、周官、王制之望祀射牛事。齐人丁公年九十馀,曰:“封禅者,合不死之名也。秦皇帝不得上封,陛下必欲上,稍上即无风雨,遂上封矣。”上于是乃令诸儒习射牛,草封禅仪。数年,至且行。天子既闻公孙卿及方士之言,黄帝以上封禅,皆致怪物与神通,欲放黄帝以上接神仙人蓬莱士,高世比德于九皇,而颇采儒术以文之。群儒既已不能辨明封禅事,又牵拘于诗书古文而不能骋。上为封禅祠器示群儒,群儒或曰“不与古同”,徐偃又曰“太常诸生行礼不如鲁善”,周霸属图封禅事,于是上绌偃、霸,而尽罢诸儒不用。

三月,遂东幸缑氏,礼登中岳太室。从官在山下闻若有言“万岁”云。问上,上不言;问下,下不言。于是以三百户封太室奉祠,命曰崇高邑。东上泰山,泰山之草木叶未生,乃令人上石立之泰山巅。

上遂东巡海上,行礼祠八神。齐人之上疏言神怪奇方者以万数,然无验者。乃益发船,令言海中神山者数千人求蓬莱神人。公孙卿持节常先行候名山,至东莱,言夜见大人,长数丈,就之则不见,见其迹甚大,类禽兽云。群臣有言见一老父牵狗,言“吾欲见巨公”,已忽不见。上即见大迹,未信,及群臣有言老父,则大以为仙人也。宿留海上,予方士传车及间使求仙人以千数。

四月,还至奉高。上念诸儒及方士言封禅人人殊,不经,难施行。天子至梁父,礼祠地主。乙卯,令侍中儒者皮弁荐绅,射牛行事。封泰山下东方,如郊祠太一之礼。封广丈二尺,高九尺,其下则有玉牒书,书祕。礼毕,天子独与侍中奉车子侯上泰山,亦有封。其事皆禁。明日,下阴道。丙辰,禅泰山下阯东北肃然山,如祭后土礼。天子皆亲拜见,衣上黄而尽用乐焉。江淮间一茅三脊为神藉。五色土益杂封。纵远方奇兽蜚禽及白雉诸物,颇以加礼。兕牛犀象之属不用。皆至泰山祭后土。封禅祠;其夜若有光,昼有白云起封中。

天子从禅还,坐明堂,群臣更上寿。于是制诏御史:“朕以眇眇之身承至尊,兢兢焉惧不任。维德菲薄,不明于礼乐。脩祠太一,若有象景光,箓如有望,震于怪物,欲止不敢,遂登封太山,至于梁父,而后禅肃然。自新,嘉与士大夫更始,赐民百户牛一酒十石,加年八十孤寡布帛二匹。复博、奉高、蛇丘、历城,无出今年租税。其大赦天下,如乙卯赦令。行所过毋有复作。事在二年前,皆勿听治。”又下诏曰:“古者天子五载一巡狩,用事泰山,诸侯有朝宿地。其令诸侯各治邸泰山下。”

天子既已封泰山,无风雨灾,而方士更言蓬莱诸神若将可得,于是上欣然庶几遇之,乃复东至海上望,冀遇蓬莱焉。奉车子侯暴病,一日死。上乃遂去,并海上,北至碣石,巡自辽西,历北边至九原。五月,反至甘泉。有司言宝鼎出为元鼎,以今年为元封元年。

其秋,有星茀于东井。后十馀日,有星茀于三能。望气王朔言:“候独见填星出如瓜,食顷复入焉。”有司皆曰:“陛下建汉家封禅,天其报德星云。”

其来年冬,郊雍五帝。还,拜祝祠太一。赞飨曰:“德星昭衍,厥维休祥。寿星仍出,渊耀光明。信星昭见,皇帝敬拜太祝之享。”

其春,公孙卿言见神人东莱山,若云“欲见天子”。天子于是幸缑氏城,拜卿为中大夫。遂至东莱,宿留之数日,无所见,见大人迹云。复遣方士求神怪采芝药以千数。是岁旱。于是天子既出无名,乃祷万里沙,过祠泰山。还至瓠子,自临塞决河,留二日,沈祠而去。使二卿将卒塞决河,徙二渠,复禹之故迹焉。

是时既灭两越,越人勇之乃言“越人俗鬼,而其祠皆见鬼,数有效。昔东瓯王敬鬼,寿百六十岁。后世怠慢,故衰秏”。乃令越巫立越祝祠,安台无坛,亦祠天神上帝百鬼,而以鸡卜。上信之,越祠鸡卜始用。

公孙卿曰:“仙人可见,而上往常遽,以故不见。今陛下可为观,如缑城,置脯枣,神人宜可致也。且仙人好楼居。”于是上令长安则作蜚廉桂观,甘泉则作益延寿观,使卿持节设具而候神人。乃作通天茎台,置祠具其下,将招来仙神人之属。于是甘泉更置前殿,始广诸宫室。夏,有芝生殿房内中。天子为塞河,兴通天台,若见有光云,乃下诏:“甘泉房中生芝九茎,赦天下,毋有复作。”

其明年,伐朝鲜。夏,旱。公孙卿曰:“黄帝时封则天旱,乾封三年。”上乃下诏曰:“天旱,意乾封乎?其令天下尊祠灵星焉。”

其明年,上郊雍,通回中道,巡之。春,至鸣泽,从西河归。

其明年冬,上巡南郡,至江陵而东。登礼灊之天柱山,号曰南岳。浮江,自寻阳出枞阳,过彭蠡,礼其名山川。北至琅邪,并海上。四月中,至奉高脩封焉。

初,天子封泰山,泰山东北阯古时有明堂处,处险不敞。上欲治明堂奉高旁,未晓其制度。济南人公
带上黄帝时明堂图。明堂图中有一殿,四面无壁,以茅盖,通水,圜宫垣为衤复道,上有楼,从西南入,命曰昆仑,天子从之入,以拜祠上帝焉。于是上令奉高作明堂汶上,如带图。及五年脩封,则祠太一、五帝于明堂上坐,令高皇帝祠坐对之。祠后土于下房,以二十太牢。天子从昆仑道入,始拜明堂如郊礼。礼毕,燎堂下。而上又上泰山,自有祕祠其巅。而泰山下祠五帝,各如其方,黄帝并赤帝,而有司侍祠焉。山上举火,下悉应之。

其后二岁,十一月甲子朔旦冬至,推历者以本统。天子亲至泰山,以十一月甲子朔旦冬至日祠上帝明堂,毋脩封禅。其赞飨曰:“天增授皇帝太元神策,周而复始。皇帝敬拜太一。”东至海上,考入海及方士求神者,莫验,然益遣,冀遇之。

十一月乙酉,柏梁灾。十二月甲午朔,上亲禅高里,祠后土。临勃海,将以望祀蓬莱之属,冀至殊廷焉。

上还,以柏梁灾故,朝受计甘泉。公孙卿曰:“黄帝就青灵台,十二日烧,黄帝乃治明廷。明廷,甘泉也。”方士多言古帝王有都甘泉者。其后天子又朝诸侯甘泉,甘泉作诸侯邸。勇之乃曰:“越俗有火灾,复起屋必以大,用胜服之。”于是作建章宫,度为千门万户。前殿度高未央。其东则凤阙,高二十馀丈。其西则唐中,数十里虎圈。其北治大池,渐台高二十馀丈,命曰太液池,中有蓬莱、方丈、瀛洲、壶梁,象海中神山龟鱼之属。其南有玉堂、璧门、大鸟之属。乃立神明台、井幹楼,度五十丈,辇道相属焉。

夏,汉改历,以正月为岁首,而色上黄,官名更印章以五字,为太初元年。是岁,西伐大宛。蝗大起。丁夫人、雒阳虞初等以方祠诅匈奴、大宛焉。

其明年,有司上言雍五畤无牢熟具,芬芳不备。乃令祠官进畤犊牢具,色食所胜,而以木禺马代驹焉。独五月尝驹,行亲郊用驹。及诸名山川用驹者,悉以木禺马代。行过,乃用驹。他礼如故。

其明年,东巡海上,考神仙之属,未有验者。方士有言“黄帝时为五城十二楼,以候神人于执期,命曰迎年”。上许作之如方,命曰明年。上亲礼祠上帝焉。

公
带曰:“黄帝时虽封泰山,然风后、封巨、岐伯令黄帝封东泰山,禅凡山,合符,然后不死焉。”天子既令设祠具,至东泰山,泰山卑小,不称其声,乃令祠官礼之,而不封禅焉。其后令带奉祠候神物。夏,遂还泰山,脩五年之礼如前,而加以禅祠石闾。石闾者,在泰山下阯南方,方士多言此仙人之闾也,故上亲禅焉。

其后五年,复至泰山脩封。还过祭恆山。

今天子所兴祠,太一、后土,三年亲郊祠,建汉家封禅,五年一脩封。薄忌太一及三一、冥羊、马行、赤星,五,宽舒之祠官以岁时致礼。凡六祠,皆太祝领之。至如八神诸神,明年、凡山他名祠,行过则祠,行去则已。方士所兴祠,各自主,其人终则已,祠官不主。他祠皆如其故。今上封禅,其后十二岁而还,遍于五岳、四渎矣。而方士之候祠神人,入海求蓬莱,终无有验。而公孙卿之候神者,犹以大人之迹为解,无有效。天子益怠厌方士之怪迂语矣,然羁縻不绝,冀遇其真。自此之后,方士言神祠者弥众,然其效可睹矣。

太史公曰:余从巡祭天地诸神名山川而封禅焉。入寿宫侍祠神语,究观方士祠官之意,于是退而论次自古以来用事于鬼神者,具见其表里。后有君子,得以览焉。若至俎豆珪币之详,献酬之礼,则有司存。

礼载“升中”,书称“肆类”。古今盛典,皇王能事。登封报天,降禅除地。飞英腾实,金泥石记。汉承遗绪,斯道不坠。仙闾、肃然,扬休勒志。
\end{yuanwen}

\part{卷二十九}
\chapter{河渠书第七}

顾炎武:「今河津县北三十里有瓜谷山堰,贞观十年筑;东南二十三里有十石垆渠,二十三年县令公孙恕凿,溉田良沃,亩收十石;西二十一里有马鞍坞渠,亦恕所凿,有龙门仓,开元二年置,所以贮渠田之入,转搬至京以省关中之漕。此即番係之策,乃汉行之不利,唐行之而利。故知天下无不可举之功,存乎其人而已。谓后人之事,必不能过前人者,诬也。」

《河渠书》不是对现有河渠做静态描述,如分别记述某水系有某支流,发源某处,流经某地,沿途有何地形、地物、掌故,入于某川、某河、某海,等等,而是司马迁以极大的热情和兴趣对许多成功的事实和经验做了详细记述,同时他还怀着满腔郁愤,对于豪门的阻挠、气数等迷信思想的干扰做了揭露,从而对汉代弊政进行了无情的鞭挞。

\begin{yuanwen}
《夏书》曰:禹抑洪水十三年,过家不入门。陆行载\footnote{相当于“则”。}车,水行载舟,泥行蹈毳\footnote{同“橇”。},山行即桥。以别九州,随山浚川,任土作贡。通九道,陂\footnote{障塞,壅遏。}九泽,度九山。然河(灾)菑衍溢\footnote{外流。},害中国也尤甚。唯是为务。故道河自积石历龙门,南到华阴,东下砥柱,及孟津、雒汭,至于大邳。于是禹以为河所从来者高,水湍悍,难以行平地,数为败,乃厮\footnote{同“斯”,有析、劈意。}二渠以引其河。北载之高地,过降水,至于大陆,播为九河,同为逆河,入于勃海。九川既疏,九泽既洒,诸夏艾安,功施于三代。
\end{yuanwen}

《尚书·夏书》上说:大禹治理洪水的十三年中,经过自己的家门口而不进去。在陆地上行走要坐车,在水中行走要坐船,在泥泞中行走要踩木橇,在山路中行走则乘轿。天下被划分为九个州,顺着山势疏浚河流,根据土地的肥沃程度制定贡赋等级。开通九州的道路,堵塞九州的湖泽,估量九州的山地的物产。但是黄河泛滥成灾,对中原地区的损害尤其严重。所以集中力量治理黄河。于是疏导黄河,引导河水从积石山开始,经过龙门,南行到华阴县,由此折而东下,经过砥柱山和洛州河阳县的孟津、雒汭,一直到达大邳山。大禹认为大邳以上黄河流经的地区地势很高,水流湍急凶猛,很难在平地通行,多次冲毁河堤,造成水灾,于是开凿两道河渠将黄河分流成两条,引黄河东流直接入海,其正流往北通过高地,经过降水,到达大陆泽,分拨成九条支流,然后又汇合成逆河,流入渤海。九州的河川既然已经疏通,九州的湖泽已经分泄,华夏诸国得以安宁,大禹治水的功绩一直延续到夏、商、周三代。

\begin{yuanwen}
自是之后,荥阳下引河东南为鸿沟,以通宋、郑、陈、蔡、曹、卫,与济、汝、淮、泗会。于楚,西方则通渠汉水、云梦之野,东方则通沟江淮之间。于吴,则通渠三江、五湖。于齐,则通菑济之间。于蜀,蜀守冰凿离碓,辟沫水之害,穿二江成都之中。此渠皆可行舟,有馀则用溉騑,百姓飨其利。至于所过,往往引其水益用溉田畴之渠,以万亿计,然莫足数也。

西门豹引漳水溉鄴,以富魏之河内。

而韩闻秦之好兴事,欲罢之,毋令东伐,乃使水工郑国间说秦,令凿泾水自中山西邸瓠口为渠,并北山东注洛三百馀里,欲以溉田。中作而觉,秦欲杀郑国。郑国曰:“始臣为间,然渠成亦秦之利也。”秦以为然,卒使就渠。渠就,用注填阏之水,溉泽卤之地四万馀顷,收皆亩一钟。于是关中为沃野,无凶年,秦以富彊,卒并诸侯,因命曰郑国渠。

汉兴三十九年,孝文时河决酸枣,东溃金隄,于是东郡大兴卒塞之。

其后四十有馀年,今天子元光之中,而河决于瓠子,东南注钜野,通于淮、泗。于是天子使汲黯、郑当时兴人徒塞之,辄复坏。是时武安侯田蚡为丞相,其奉邑食鄃。鄃居河北,河决而南则鄃无水菑,邑收多。蚡言于上曰:“江河之决皆天事,未易以人力为彊塞,塞之未必应天。”而望气用数者亦以为然。于是天子久之不事复塞也。

是时郑当时为大农,言曰:“异时关东漕粟从渭中上,度六月而罢,而漕水道九百馀里,时有难处。引渭穿渠起长安,并南山下,至河三百馀里,径,易漕,度可令三月罢;而渠下民田万馀顷,又可得以溉田:此损漕省卒,而益肥关中之地,得穀。”天子以为然,令齐人水工徐伯表,悉发卒数万人穿漕渠,三岁而通。通,以漕,大便利。其后漕稍多,而渠下之民颇得以溉田矣。

其后河东守番系言:“漕从山东西,岁百馀万石,更砥柱之限,败亡甚多,而亦烦费。穿渠引汾溉皮氏、汾阴下,引河溉汾阴、蒲坂下,度可得五千顷。五千顷故尽河壖弃地,民茭牧其中耳,今溉田之,度可得穀二百万石以上。穀从渭上,与关中无异,而砥柱之东可无复漕。”天子以为然,发卒数万人作渠田。数岁,河移徙,渠不利,则田者不能偿种。久之,河东渠田废,予越人,令少府以为稍入。

其后人有上书欲通襃斜道及漕事,下御史大夫张汤。汤问其事,因言:“抵蜀从故道,故道多阪,回远。今穿襃斜道,少阪,近四百里;而襃水通沔,斜水通渭,皆可以行船漕。漕从南阳上沔入襃,襃之绝水至斜,间百馀里,以车转,从斜下下渭。如此,汉中之穀可致,山东从沔无限,便于砥柱之漕。且襃斜材木竹箭之饶,拟于巴蜀。”天子以为然,拜汤子卬为汉中守,发数万人作襃斜道五百馀里。道果便近,而水湍石,不可漕。

其后庄熊罴言:“临晋民原穿洛以溉重泉以东万馀顷故卤地。诚得水,可令亩十石。”于是为发卒万馀人穿渠,自徵引洛水至商颜山下。岸善崩,乃凿井,深者四十馀丈。往往为井,井下相通行水。水穨以绝商颜,东至山岭十馀里间。井渠之生自此始。穿渠得龙骨,故名曰龙首渠。作之十馀岁,渠颇通,犹未得其饶。

自河决瓠子后二十馀岁,岁因以数不登,而梁楚之地尤甚。天子既封禅巡祭山川,其明年,旱,乾封少雨。天子乃使汲仁、郭昌发卒数万人塞瓠子决。于是天子已用事万里沙,则还自临决河,沈白马玉璧于河,令群臣从官自将军已下皆负薪窴决河。是时东郡烧草,以故薪柴少,而下淇园之竹以为楗。

天子既临河决,悼功之不成,乃作歌曰:“瓠子决兮将柰何?皓皓旰旰兮闾殚为河!殚为河兮地不得宁,功无已时兮吾山平。吾山平兮钜野溢,鱼沸郁兮柏冬日。延道弛兮离常流,蛟龙骋兮方远游。归旧川兮神哉沛,不封禅兮安知外!为我谓河伯兮何不仁,泛滥不止兮愁吾人?齧桑浮兮淮、泗满,久不反兮水维缓。”一曰:“河汤汤兮激潺湲,北渡污兮浚流难。搴长茭兮沈美玉,河伯许兮薪不属。薪不属兮卫人罪,烧萧条兮噫乎何以御水!穨林竹兮楗石菑,宣房塞兮万福来。”于是卒塞瓠子,筑宫其上,名曰宣房宫。而道河北行二渠,复禹旧迹,而梁、楚之地复宁,无水灾。

自是之后,用事者争言水利。朔方、西河、河西、酒泉皆引河及川谷以溉田;而关中辅渠、灵轵引堵水;汝南、九江引淮;东海引钜定;泰山下引汶水:皆穿渠为溉田,各万馀顷。佗小渠披山通道者,不可胜言。然其著者在宣房。

太史公曰:余南登庐山,观禹疏九江,遂至于会稽太湟,上姑苏,望五湖;东闚洛汭、大邳,迎河,行淮、泗、济、漯洛渠;西瞻蜀之岷山及离碓;北自龙门至于朔方。曰:甚哉,水之为利害也!余从负薪塞宣房,悲瓠子之诗而作河渠书。

水之利害,自古而然。禹疏沟洫,随山濬川。爰洎后世,非无圣贤。鸿沟既划,龙骨斯穿。填阏攸垦,黎蒸有年。宣房在咏,梁楚获全。
\end{yuanwen}

\chapter{平准书}

\begin{yuanwen}
汉兴,接秦之弊,丈夫从军旅,老弱转粮饟,作业剧而财匮,自天子不能具钧驷,而将相或乘牛车,齐民无藏盖。于是为秦钱重难用,更令民铸钱,一黄金一斤,约法省禁。而不轨逐利之民,蓄积馀业以稽市物,物踊腾粜,米至石万钱,马一匹则百金。

天下已平,高祖乃令贾人不得衣丝乘车,重租税以困辱之。孝惠、高后时,为天下初定,复弛商贾之律,然市井之子孙亦不得仕宦为吏。量吏禄,度官用,以赋于民。而山川园池市井租税之入,自天子以至于封君汤沐邑,皆各为私奉养焉,不领于天下之经费。漕转山东粟,以给中都官,岁不过数十万石。

至孝文时,荚钱益多,轻,乃更铸四铢钱,其文为“半两”,令民纵得自铸钱。故吴诸侯也,以即山铸钱,富埒天子,其后卒以叛逆。邓通,大夫也,以铸钱财过王者。故吴、邓氏钱布天下,而铸钱之禁生焉。

匈奴数侵盗北边,屯戍者多,边粟不足给食当食者。于是募民能输及转粟于边者拜爵,爵得至大庶长。

孝景时,上郡以西旱,亦复脩卖爵令,而贱其价以招民;及徒复作,得输粟县官以除罪。益造苑马以广用,而宫室列观舆马益增脩矣。

至今上即位数岁,汉兴七十馀年之间,国家无事,非遇水旱之灾,民则人给家足,都鄙廪庾皆满,而府库馀货财。京师之钱累巨万,贯朽而不可校。太仓之粟陈陈相因,充溢露积于外,至腐败不可食。众庶街巷有马,阡陌之间成群,而乘字牝者儐而不得聚会。守闾阎者食粱肉,为吏者长子孙,居官者以为姓号。故人人自爱而重犯法,先行义而后绌耻辱焉。当此之时,网疏而民富,役财骄溢,或至兼并豪党之徒,以武断于乡曲。宗室有土公卿大夫以下,争于奢侈,室庐舆服僭于上,无限度。物盛而衰,固其变也。

自是之后,严助、硃买臣等招来东瓯,事两越,江淮之间萧然烦费矣。唐蒙、司马相如开路西南夷,凿山通道千馀里,以广巴蜀,巴蜀之民罢焉。彭吴贾灭朝鲜,置沧海之郡,则燕齐之间靡然发动。及王恢设谋马邑,匈奴绝和亲,侵扰北边,兵连而不解,天下苦其劳,而干戈日滋。行者赍,居者送,中外骚扰而相奉,百姓抏弊以巧法,财赂衰秏而不赡。入物者补官,出货者除罪,选举陵迟,廉耻相冒,武力进用,法严令具。兴利之臣自此始也。

其后汉将岁以数万骑出击胡,及车骑将军
青取匈奴河南地,筑朔方。当是时,汉通西南夷道,作者数万人,千里负担馈粮,率十馀锺致一石,散币于邛僰以集之。数岁道不通,蛮夷因以数攻,吏发兵诛之。悉巴蜀租赋不足以更之,乃募豪民田南夷,入粟县官,而内受钱于都内。东至沧海之郡,人徒之费拟于南夷。又兴十万馀人筑卫朔方,转漕甚辽远,自山东咸被其劳,费数十百巨万,府库益虚。乃募民能入奴婢得以终身复,为郎增秩,及入羊为郎,始于此。

其后四年,而汉遣大将将六将军,军十馀万,击右贤王,获首虏万五千级。明年,大将军将六将军仍再出击胡,得首虏万九千级。捕斩首虏之士受赐黄金二十馀万斤,虏数万人皆得厚赏,衣食仰给县官;而汉军之士马死者十馀万,兵甲之财转漕之费不与焉。于是大农陈藏钱经秏,赋税既竭,犹不足以奉战士。有司言:“天子曰‘朕闻五帝之教不相复而治,禹汤之法不同道而王,所由殊路,而建德一也。北边未安,朕甚悼之。日者,大将军攻匈奴,斩首虏万九千级,留蹛无所食。议令民得买爵及赎禁锢免减罪’。请置赏官,命曰武功爵。级十七万,凡直三十馀万金。诸买武功爵官首者试补吏,先除;千夫如五大夫;其有罪又减二等;爵得至乐卿:以显军功。”军功多用越等,大者封侯卿大夫,小者郎吏。吏道杂而多端,则官职秏废。

自公孙弘以春秋之义绳臣下取汉相,张汤用唆文决理为廷尉,于是见知之法生,而废格沮诽穷治之狱用矣。其明年,淮南、衡山、江都王谋反迹见,而公卿寻端治之,竟其党与,而坐死者数万人,长吏益惨急而法令明察。

当是之时,招尊方正贤良文学之士,或至公卿大夫。公孙弘以汉相,布被,食不重味,为天下先。然无益于俗,稍骛于功利矣。

其明年,骠骑仍再出击胡,获首四万。其秋,浑邪王率数万之众来降,于是汉发车二万乘迎之。既至,受赏,赐及有功之士。是岁费凡百馀巨万。

初,先是往十馀岁河决观,梁楚之地固已数困,而缘河之郡隄塞河,辄决坏,费不可胜计。其后番系欲省底柱之漕,穿汾、河渠以为溉田,作者数万人;郑当时为渭漕渠回远,凿直渠自长安至华阴,作者数万人;朔方亦穿渠,作者数万人:各历二三期,功未就,费亦各巨万十数。

天子为伐胡,盛养马,马之来食长安者数万匹,卒牵掌者关中不足,乃调旁近郡。而胡降者皆衣食县官,县官不给,天子乃损膳,解乘舆驷,出御府禁藏以赡之。

其明年,山东被水菑,民多饥乏,于是天子遣使者虚郡国仓廥以振贫民。犹不足,又募豪富人相贷假。尚不能相救,乃徙贫民于关以西,及充朔方以南新秦中,七十馀万口,衣食皆仰给县官。数岁,假予产业,使者分部护之,冠盖相望。其费以亿计,不可胜数。于是县官大空。

而富商大贾或蹛财役贫,转毂百数,废居居邑,封君皆低首仰给。冶铸煮盐,财或累万金,而不佐国家之急,黎民重困。于是天子与公卿议,更钱造币以赡用,而摧浮淫并兼之徒。是时禁苑有白鹿而少府多银锡。自孝文更造四铢钱,至是岁四十馀年,从建元以来,用少,县官往往即多铜山而铸钱,民亦间盗铸钱,不可胜数。钱益多而轻,物益少而贵。有司言曰:“古者皮币,诸侯以聘享。金有三等,黄金为上,白金为中,赤金为下。今半两钱法重四铢,而奸或盗摩钱里取鋊,钱益轻薄而物贵,则远方用币烦费不省。”乃以白鹿皮方尺,缘以藻缋,为皮币,直四十万。王侯宗室朝觐聘享,必以皮币荐璧,然后得行。

又造银锡为白金。以为填用莫如龙,地用莫如马,人用莫如龟,故白金三品:其一曰重八两,圜之,其文龙,名曰“白选”,直三千;二曰以重差小,方之,其文马,直五百;三曰复小,撱之,其文龟,直三百。令县官销半两钱,更铸三铢钱,文如其重。盗铸诸金钱罪皆死,而吏民之盗铸白金者不可胜数。

于是以东郭咸阳、孔仅为大农丞,领盐铁事;桑弘羊以计算用事,侍中。咸阳,齐之大煮盐,孔仅,南阳大冶,皆致生累千金,故郑当时进言之。弘羊,雒阳贾人子,以心计,年十三侍中。故三人言利事析秋豪矣。

法既益严,吏多废免。兵革数动,民多买复及五大夫,徵发之士益鲜。于是除千夫五大夫为吏,不欲者出马;故吏皆適令伐棘上林,作昆明池。

其明年,大将军、骠骑大出击胡,得首虏八九万级,赏赐五十万金,汉军马死者十馀万匹,转漕车甲之费不与焉。是时财匮,战士颇不得禄矣。

有司言三铢钱轻,易奸诈,乃更请诸郡国铸五铢钱,周郭其下,令不可磨取鋊焉。

大农上盐铁丞孔仅、咸阳言:“山海,天地之藏也,皆宜属少府,陛下不私,以属大农佐赋。原募民自给费,因官器作煮盐,官与牢盆。浮食奇民欲擅管山海之货,以致富羡,役利细民。其沮事之议,不可胜听。敢私铸铁器煮盐者,釱左趾,没入其器物。郡不出铁者,置小铁官,便属在所县。”使孔仅、东郭咸阳乘传举行天下盐铁,作官府,除故盐铁家富者为吏。吏道益杂,不选,而多贾人矣。

商贾以币之变,多积货逐利。于是公卿言:“郡国颇被菑害,贫民无产业者,募徙广饶之地。陛下损膳省用,出禁钱以振元元,宽贷赋,而民不齐出于南亩,商贾滋众。贫者畜积无有,皆仰县官。异时算轺车贾人缗钱皆有差,请算如故。诸贾人末作贳贷卖买,居邑稽诸物,及商以取利者,虽无市籍,各以其物自占,率缗钱二千而一算。诸作有租及铸,率缗钱四千一算。非吏比者三老、北边骑士,轺车以一算;商贾人轺车二算;船五丈以上一算。匿不自占,占不悉,戍边一岁,没入缗钱。有能告者,以其半畀之。贾人有市籍者,及其家属,皆无得籍名田,以便农。敢犯令,没入田僮。”

天子乃思卜式之言,召拜式为中郎,爵左庶长,赐田十顷,布告天下,使明知之。

初,卜式者,河南人也,以田畜为事。亲死,式有少弟,弟壮,式脱身出分,独取畜羊百馀,田宅财物尽予弟。式入山牧十馀岁,羊致千馀头,买田宅。而其弟尽破其业,式辄复分予弟者数矣。是时汉方数使将击匈奴,卜式上书,原输家之半县官助边。天子使使问式:“欲官乎?”式曰:“臣少牧,不习仕宦,不原也。”使问曰:“家岂有冤,欲言事乎?”式曰:“臣生与人无分争。式邑人贫者贷之,不善者教顺之,所居人皆从式,式何故见冤于人!无所欲言也。”使者曰:“苟如此,子何欲而然?”式曰:“天子诛匈奴,愚以为贤者宜死节于边,有财者宜输委,如此而匈奴可灭也。”使者具其言入以闻。天子以语丞相弘。弘曰:“此非人情。不轨之臣,不可以为化而乱法,原陛下勿许。”于是上久不报式,数岁,乃罢式。式归,复田牧。岁馀,会军数出,浑邪王等降,县官费众,仓府空。其明年,贫民大徙,皆仰给县官,无以尽赡。卜式持钱二十万予河南守,以给徙民。河南上富人助贫人者籍,天子见卜式名,识之,曰“是固前而欲输其家半助边”,乃赐式外繇四百人。式又尽复予县官。是时富豪皆争匿财,唯式尤欲输之助费。天子于是以式终长者,故尊显以风百姓。

初,式不原为郎。上曰:“吾有羊上林中,欲令子牧之。”式乃拜为郎,布衣屩而牧羊。岁馀,羊肥息。上过见其羊,善之。式曰:“非独羊也,治民亦犹是也。以时起居;恶者辄斥去,毋令败群。”上以式为奇,拜为缑氏令试之,缑氏便之。迁为成皋令,将漕最。上以为式朴忠,拜为齐王太傅。

而孔仅之使天下铸作器,三年中拜为大农,列于九卿。而桑弘羊为大农丞,筦诸会计事,稍稍置均输以通货物矣。

始令吏得入穀补官,郎至六百石。

自造白金五铢钱后五岁,赦吏民之坐盗铸金钱死者数十万人。其不发觉相杀者,不可胜计。赦自出者百馀万人。然不能半自出,天下大抵无虑皆铸金钱矣。犯者众,吏不能尽诛取,于是遣博士褚大、徐偃等分曹循行郡国,举兼并之徒守相为者。而御史大夫张汤方隆贵用事,减宣、杜周等为中丞,义纵、尹齐、王温舒等用惨急刻深为九卿,而直指夏兰之属始出矣。

而大农颜异诛。初,异为济南亭长,以廉直稍迁至九卿。上与张汤既造白鹿皮币,问异。异曰:“今王侯朝贺以苍璧,直数千,而其皮荐反四十万,本末不相称。”天子不说。张汤又与异有卻,及有人告异以它议,事下张汤治异。异与客语,客语初令下有不便者,

异不应,微反脣。汤奏当异九卿见令不便,不入言而腹诽,论死。自是之后,有腹诽之法,而公卿大夫多谄谀取容矣。

天子既下缗钱令而尊卜式,百姓终莫分财佐县官,于是告缗钱纵矣。

郡国多柬铸钱,钱多轻,而公卿请令京师铸锺官赤侧,一当五,赋官用非赤侧不得行。白金稍贱,民不宝用,县官以令禁之,无益。岁馀,白金终废不行。

是岁也,张汤死而民不思。

其后二岁,赤侧钱贱,民巧法用之,不便,又废。于是悉禁郡国无铸钱,专令上林三官铸。钱既多,而令天下非三官钱不得行,诸郡国所前铸钱皆废销之,输其铜三官。而民之铸钱益少,计其费不能相当,唯真工大奸乃盗为之。

卜式相齐,而杨可告缗遍天下,中家以上大抵皆遇告。杜周治之,狱少反者。乃分遣御史廷尉正监分曹往,即治郡国缗钱,得民财物以亿计,奴婢以千万数,田大县数百顷,小县百馀顷,宅亦如之。于是商贾中家以上大率破,民偷甘食好衣,不事畜藏之产业,而县官有盐铁缗钱之故,用益饶矣。

益广关,置左右辅。

初,大农筦盐铁官布多,置水衡,欲以主盐铁;及杨可告缗钱,上林财物众,乃令水衡主上林。上林既充满,益广。是时越欲与汉用船战逐,乃大修昆明池,列观环之。治楼船,高十馀丈,旗帜加其上,甚壮。于是天子感之,乃作柏梁台,高数十丈。宫室之修,由此日丽。

乃分缗钱诸官,而水衡、少府、大农、太仆各置农官,往往即郡县比没入田田之。其没入奴婢,分诸苑养狗马禽兽,及与诸官。诸官益杂置多,徒奴婢众,而下河漕度四百万石,及官自籴乃足。

所忠言:“世家子弟富人或斗鸡走狗马,弋猎博戏,乱齐民。”乃徵诸犯令,相引数千人,命曰“株送徒”。入财者得补郎,郎选衰矣。

是时山东被河菑,及岁不登数年,人或相食,方一二千里。天子怜之,诏曰:“江南火耕水耨,令饥民得流就食江淮间,欲留,留处。”遣使冠盖相属于道,护之,下巴蜀粟以振之。

其明年,天子始巡郡国。东度河,河东守不意行至,不辨,自杀。行西逾陇,陇西守以行往卒,天子从官不得食,陇西守自杀。于是上北出萧关,从数万骑,猎新秦中,以勒边兵而归。新秦中或千里无亭徼,于是诛北地太守以下,而令民得畜牧边县,官假马母,三岁而归,及息什一,以除告缗,用充仞新秦中。

既得宝鼎,立后土、太一祠,公卿议封禅事,而天下郡国皆豫治道桥,缮故宫,及当驰道县,县治官储,设供具,而望以待幸。

其明年,南越反,西羌侵边为桀。于是天子为山东不赡,赦天下,因南方楼船卒二十馀万人击南越,数万人发三河以西骑击西羌,又数万人度河筑令居。初置张掖、酒泉郡,而上郡、朔方、西河、河西开田官,斥塞卒六十万人戍田之。中国缮道餽粮,远者三千,近者千馀里,皆仰给大农。边兵不足,乃发武库工官兵器以赡之。车骑马乏绝,县官钱少,买马难得,乃著令,令封君以下至三百石以上吏,以差出牝马天下亭,亭有畜牸马,岁课息。

齐相卜式上书曰:“臣闻主忧臣辱。南越反,臣原父子与齐习船者往死之。”天子下诏曰:“卜式虽躬耕牧,不以为利,有馀辄助县官之用。今天下不幸有急,而式奋原父子死之,虽未战,可谓义形于内。赐爵关内侯,金六十斤,田十顷。”布告天下,天下莫应。列侯以百数,皆莫求从军击羌、越。至酎,少府省金,而列侯坐酎金失侯者百馀人。乃拜式为御史大夫。

式既在位,见郡国多不便县官作盐铁,铁器苦恶,贾贵,或彊令民卖买之。而船有算,商者少,物贵,乃因孔仅言船算事。上由是不悦卜式。

汉连兵三岁,诛羌,灭南越,番禺以西至蜀南者置初郡十七,且以其故俗治,毋赋税。南阳、汉中以往郡,各以地比给初郡吏卒奉食币物,传车马被具。而初郡时时小反,杀吏,汉发南方吏卒往诛之,间岁万馀人,费皆仰给大农。大农以均输调盐铁助赋,故能赡之。然兵所过县,为以訾给毋乏而已,不敢言擅赋法矣。

其明年,元封元年,卜式贬秩为太子太傅。而桑弘羊为治粟都尉,领大农,尽代仅筦天下盐铁。弘羊以诸官各自巿,相与争,物故腾跃,而天下赋输或不偿其僦费,乃请置大农部丞数十人,分部主郡国,各往往县置均输盐铁官,令远方各以其物贵时商贾所转贩者为赋,而相灌输。置平准于京师,都受天下委输。召工官治车诸器,皆仰给大农。大农之诸官尽笼天下之货物,贵即卖之,贱则买之。如此,富商大贾无所牟大利,则反本,而万物不得腾踊。故抑天下物,名曰“平准”。天子以为然,许之。于是天子北至朔方,东到太山,巡海上,并北边以归。所过赏赐,用帛百馀万匹,钱金以巨万计,皆取足大农。

弘羊又请令吏得入粟补官,及罪人赎罪。令民能入粟甘泉各有差,以复终身,不告缗。他郡各输急处,而诸农各致粟,山东漕益岁六百万石。一岁之中,太仓、甘泉仓满。边馀穀诸物均输帛五百万匹。民不益赋而天下用饶。于是弘羊赐爵左庶长,黄金再百斤焉。

是岁小旱,上令官求雨,卜式言曰:“县官当食租衣税而已,今弘羊令吏坐市列肆,贩物求利。亨弘羊,天乃雨。”

太史公曰:农工商交易之路通,而龟贝金钱刀布之币兴焉。所从来久远,自高辛氏之前尚矣,靡得而记云。故书道唐虞之际,诗述殷周之世,安宁则长庠序,先本绌末,以礼义防于利;事变多故而亦反是。是以物盛则衰,时极而转,一质一文,终始之变也。禹贡九州,各因其土地所宜,人民所多少而纳职焉。汤武承弊易变,使民不倦,各兢兢所以为治,而稍陵迟衰微。齐桓公用管仲之谋,通轻重之权,徼山海之业,以朝诸侯,用区区之齐显成霸名。魏用李克,尽地力,为彊君。自是以后,天下争于战国,贵诈力而贱仁义,先富有而后推让。故庶人之富者或累巨万,而贫者或不厌糟糠;有国彊者或并群小以臣诸侯,而弱国或绝祀而灭世。以至于秦,卒并海内。虞夏之币,金为三品,或黄,或白,或赤;或钱,或布,或刀,或龟贝。及至秦,中一国之币为等,黄金以溢名,为上币;铜钱识曰半两,重如其文,为下币。而珠玉、龟贝、银锡之属为器饰宝藏,不为币。然各随时而轻重无常。于是外攘夷狄,内兴功业,海内之士力耕不足粮饟,女子纺绩不足衣服。古者尝竭天下之资财以奉其上,犹自以为不足也。无异故云,事势之流,相激使然,曷足怪焉。

平准之立,通货天下。既入县官,或振华夏。其名刀布,其文龙马。增算告缗,裒多益寡。弘羊心计,卜式长者。都内充殷,取赡郊野。
\end{yuanwen}

\part{卷三十一}
\chapter{吴太伯世家第一}

苏辙:「季子事吴九十余年,观其挂剑于墓,不以死倍其心,葬子赢博,不以恩累其志,引兵避楚,不以名害其德,盖其所以养之者至矣,虽禄之天下将不受,况吴乎?彼其所养者诚重也。」王世贞:「季札智人也,馀眜卒而犹让,彼见僚以贪愎躁勇之性,光以狡悍忍妒之资,未尝一日忘王位也。札欲以礼息争,而不能以义割恩,而不忍,故熟计而舍之,非得已也。」

\begin{yuanwen}
吴太伯,太伯弟仲雍,皆周太王之子,而王季历\footnote{周文王之父。}之兄也。季历贤,而有圣子昌,太王欲立季历以及昌,于是太佰、仲雍二人乃奔荆蛮,文\footnote{通“纹”。}身断发,示不可用,以避季历。季历果立,是为王季,而昌为文王。太伯之饹荆蛮,自号句吴。荆蛮义之,从而归之千馀家,立为吴太伯。
\end{yuanwen}

吴太伯和太伯的弟弟仲雍,都是周太王古公亶父的儿子,周王季历的哥哥。季历很贤明,有个有圣德之名的儿子昌,太王想要立季历为王,将来再传位给姬昌,于是太伯、仲雍二人便去了荆蛮之地,在身上刺上花纹,剪短自己的头发,表示自己已经不能继承王位,以此来避让季历。季历果然被立为王,就是王季,后来姬昌成为文王。太伯到了荆蛮之地,自称句吴。荆蛮之地的人认为他很仁义,所以有一千多家归附他,拥立他做了吴太伯。

\begin{yuanwen}
太伯卒,无子,弟仲雍立,是为吴仲雍。仲雍卒,子季简立。季简卒,子叔达立。叔达卒,子周章立。是时周武王克殷,求太伯、仲雍之后,得周章。周章已君吴,因而封之。乃封周章弟虞仲于周之北故夏虚,是为虞仲,列为诸侯。
\end{yuanwen}

太伯去世,没有儿子,他的弟弟仲雍继位,就是吴仲雍。仲雍去世,他的儿子季简继位。季简去世后,他的儿子叔达继位。叔达去世后,他的儿子周章继位。这时周武王灭掉殷商,寻找太伯、仲雍的后代,找到周章。周章这时已经是吴国的君主了,于是周武王就正式将吴地封给他。又将周章的弟弟虞仲封在周室北边夏都的旧址,就是虞仲,位于诸侯的行列。

\begin{yuanwen}
周章卒,子熊遂立,熊遂卒,子柯相立。柯相卒,子彊鸠夷立。彊鸠夷卒,子馀桥疑吾立。馀桥疑吾卒,子柯卢立。柯卢卒,子周繇立。周繇卒,子屈羽立。屈羽卒,子夷吾立。夷吾卒,子禽处立。禽处卒,子转立。转卒,子颇高立。颇高卒,子句卑立。是时晋献公灭周北虞公,以开晋伐虢也。句卑卒,子去齐立。去齐卒,子寿梦立。寿梦立而吴始益大,称王。
\end{yuanwen}

周章去世,他的儿子熊遂继位。熊遂去世,他的儿子柯相继位。柯相去世,他的儿子彊鸠夷继位。彊鸠夷去世,他的儿子馀桥疑吾继位。馀桥疑吾去世,他的儿子柯卢继位。柯卢去世,他的儿子周繇继位。周繇去世,他的儿子屈羽继位。屈羽去世,他的儿子夷吾继位。夷吾去世,他的儿子禽处继位。禽处去世,他的儿子转继位。转去世,他的儿子颇高继位。颇高去世,他的儿子句卑继位。这时候晋献公灭掉了周室北边的虞公,因为虞公借道给晋国,让晋军灭掉了虢国。句卑去世,他的儿子去齐继位。去齐去世,他的儿子寿梦继位。寿梦继位以后吴国开始强大,寿梦以王自称。

\begin{yuanwen}
自太伯作\footnote{创立。}吴,五世而武王克殷,封其后为二:其一虞,在中国;其一吴,在夷蛮。十二世而晋灭中国之虞。中国之虞灭二世,而夷蛮之吴兴。大凡从太伯至寿梦十九世。
\end{yuanwen}

自从太伯创建吴国,传了五代时周武王灭掉殷商,封太伯的后代为两个诸侯国:一个是虞国,在中原地区;一个是吴国,在蛮夷地区。传到十二代的时候,晋国将中原的虞国灭掉。中原的虞国被灭以后又过了两代,蛮夷地区的吴国开始强盛。总之,从太伯到寿梦一共传了十九代。

\begin{yuanwen}
王寿梦二年,楚之亡大夫申公巫臣怨楚将子反而饹(奔)晋,自晋使吴,教吴用兵乘车,令其子为吴行人,吴于是始通于中国。吴伐楚。十六年,楚共王\footnote{名审,楚庄王之子。}伐吴,至衡山。
\end{yuanwen}

吴王寿梦二年(前584年),逃亡在外的楚国大夫申公巫臣因为怨恨楚国的将领子反而投奔了晋国,他从晋国出使到了吴国,教给吴国人用兵和驾战车的方法,又让他的儿子狐庸担任吴国主管接待国宾的行人官,吴国于是开始和中原各国交往。后来,吴国讨伐楚国。十六年,楚共王讨伐吴国,兵至衡山。

\begin{yuanwen}
二十五年,王寿梦卒。寿梦有子四人,长曰诸樊,次曰馀祭,次曰馀眛,次曰季札。季札贤,而寿梦欲立之,季札让不可,于是乃立长子诸樊,摄行事当国。
\end{yuanwen}

二十五年(前561年),吴王寿梦去世。寿梦有四个儿子,长子名叫诸樊,次子名叫馀祭,三子叫馀眜,四子名叫季札。季札贤能,寿梦想立他为嗣子,但是季札谦逊不受,于是寿梦就立了长子诸樊为嗣,代理国家政务。

\begin{yuanwen}
王诸樊元年,诸樊已除丧,让位季札。季札谢曰:“曹宣公之卒也,诸侯与曹人不义曹君,将立子臧,子臧去之,以成曹君,君子曰‘能守节矣’。君义嗣,谁敢干君!有国,非吾节也。札虽不材,原附于子臧之义。”

吴人固立季札,季札弃其室而耕,乃舍之。秋,吴伐楚,楚败我师。四年,晋平公初立。
\end{yuanwen}

吴王诸樊元年(前560年),诸樊服丧期满后除去丧服,要将王位让给季札。季札推辞说:“曹宣公去世的时候,诸侯和曹国人都认为新曹君曹成公、曹宣公的弟弟负刍杀太子夺位是不义的行为,准备拥立子臧为曹君,子臧却逃走了,好成全负刍,君子称赞子臧‘能够严守节操’。你是合法的继承人,谁敢干涉你?做国君不符合我这种操行的人的意愿。我季札虽然没什么才能,但是我甘愿仿效子臧的节操。”

吴国人坚决要立季札,于是季札离开了王室,像百姓一样耕地为业,吴国人只好放弃拥立他。秋季,吴国讨伐楚国,被楚军打败。四年(前557年),晋平公继位。

\begin{yuanwen}
十三年,王诸樊卒。有命授弟馀祭,欲传以次,必致国于季札而止,以称先王寿梦之意,且嘉季札之义,兄弟皆欲致国,令以渐至焉。季札封于延陵,故号曰延陵季子。
\end{yuanwen}

十三年(前548年),吴王诸樊去世,他留下遗言将王位传给弟弟馀祭,希望这样依次往下传,一定要将传王位给季札才算完结,好实现父王寿梦的心愿,他还称赞季札让位的节操,希望自己的兄弟都能这样将王位依次相传,以使君位逐渐传到季札手中。季札被封在延陵,所以号为延陵季子。

\begin{yuanwen}
王馀祭三年,齐相庆封有罪,自齐来饹(奔)吴。吴予庆封硃(朱)方之县,以为奉邑\footnote{即俸邑,采邑。},以女妻之,富于在齐。
\end{yuanwen}

吴王馀祭三年(前545年),齐相庆封犯了罪,从齐国逃到吴国。吴王将朱方县封给庆封作为他的俸邑,将宗室的女儿嫁给他为妻,让他比在齐国时更富有。

\begin{yuanwen}
四年,吴使季札聘\footnote{访问。}于鲁,请观周乐。为歌《周南》、《召南》。曰:“美哉,始基之矣,犹未也。然勤而不怨。”

歌《邶》、《鄘》、《卫》。曰:“美哉,渊乎,忧而不困者也。吾闻卫康叔\footnote{姬姓,名封,周武王同母弟,卫国第一任国君。}、武公\footnote{卫武公,姬姓,名和,卫釐侯之子,卫国第十一任国君。}之德如是,是其《卫风》乎?”

歌《王》。曰:“美哉,思而不惧,其周之东乎?”

歌《郑》。曰:“其细已甚,民不堪也,是其先亡乎?”

歌《齐》。曰:“美哉,泱泱乎大风也哉。表东海者,其太公乎?国未可量也。”

歌《豳\footnote{bīn}》。曰:“美哉,荡荡乎,乐而不淫,其周公之东乎?”

歌《秦》。曰:“此之谓夏声。夫能夏则大,大之至也,其周之旧乎?”

歌《魏》。曰:“美哉,沨沨\footnote{舒缓平和的样子。}乎,大而宽,俭而易,行以德辅,此则盟主也。”

歌《唐》。曰:“思深哉,其有陶唐氏之遗风乎?不然,何忧之远也?非令德之后,谁能若是!”

歌《陈》。曰:“国无主,其能久乎?”

自《郐》以下,无讥焉。

歌《小雅》。曰:“美哉,思而不贰,怨而不言,其周德之衰乎?犹有先王之遗民也。”

歌《大雅》。曰:“广哉,熙熙乎,曲而有直体,其文王之德乎?”

歌《颂》。曰:“至矣哉,直而不倨,曲而不诎,近而不偪,远而不携,迁而不淫,复而不厌,哀而不愁,乐而不荒,用而不匮,广而不宣,施而不费,取而不贪,处而不厎,行而不流。五声和,八风平,节有度,守有序,盛德之所同也。”
\end{yuanwen}

四年(前544年),吴王派季札出访鲁国,季札请求观赏鲁国保留的周室礼乐。鲁国的乐师为季札演唱《周南》《召南》。季札说:“歌曲真优美啊,起始处奠定了坚实的基础,但是仍然没有达到完美的程度,然而却唱出了百姓勤劳于王事而没有怨言的感情。”

乐师又演唱了《邶风》《鄘风》《卫风》。季札说:“歌曲真优美啊,曲调深沉,具有忧患意识又不使人困惑。我听说卫康叔、武公的德行就是这样的,这大概就是《卫风》吧?”

乐师又演唱了《王风》。季札说:“歌曲真优美啊,深思却没有惧意,这大概是周王室东迁之后的乐曲吧?”

乐师又演唱了《郑风》。季札说:“这首曲子的情调过于细弱,反映了百姓不堪忍受烦琐的政令,这大概是郑国要灭亡的先兆吧?”

乐师演唱了《齐风》。季札说道:“乐曲真美好啊,气势恢宏有大国的风度。像那宽阔的东海一样,这大概是姜太公的遗风吧?这个国家的前途不可限量啊。”

乐师演唱了《豳风》。季札说:“歌曲真优美啊,气势宽宏浩荡,欢快而又不过度,这大概是反映周公东征的乐曲吧?”

乐师演唱了《秦风》。季札说:“这就是所谓的华夏的声音。能够追慕华夏就会弘大,是弘大的极致,这大概是周王室旧地的乐曲吧?”

乐师演唱了《魏风》。季札说:“乐曲真美好啊,乐声婉转悠扬,气势弘大宽广,节约而易行,再以德教为辅佐,可以做盟主了。”

乐师演唱了《唐风》。季札说:“思虑深远啊,这有陶唐氏的遗风吧?不然,为何忧患意识如此沉重呢?如果不是具备盛德的先王之后,谁能如此!”

乐师又演唱了《陈风》。季札说:“一个国家没有贤明的君主,还能长远吗?”

接下来乐师演唱的是《郐风》,从这以下,季札就再没有给出评论。

乐师演唱了《小雅》。季札说:“乐曲真优美啊,有忧患意识而没有二心,心存怨恨但是没有说出,这可能就是周德衰微的表现吧?但是还有些先王遗民风俗的留存啊。”

乐师又演唱了《大雅》。季札说:“这首乐曲真是宽广弘大啊,融洽和乐,婉转柔缓而又有正直刚强之处,这反映的可能就是周文王的大德吧?”

乐师又演唱了《颂》。季札说:“优美至极啊,曲调刚劲而不傲慢,婉转而不卑屈,紧密却不紧迫,舒缓但不散漫,节奏多变又不混乱,曲调反复而不厌烦,哀伤而不忧愁,欢快而不荒谬,广用智慧而不匮乏,宽宏而不张扬,施惠但不浪费,求取但不贪婪,乐停而余音仍在回荡,音变动而不随波逐流。五声和诣,八音协调,节奏遵循法度,旋律符合规则,与盛德之人一样。”

\begin{yuanwen}
见舞《象箾\footnote{shuò}》、《南籥》者,曰:“美哉,犹有感。”

见舞《大武》,曰:“美哉,周之盛也其若此乎?”

见舞《韶护》者,曰:“圣人之弘也,犹有惭德,圣人之难也!”

见舞《大夏》,曰:“美哉,勤而不德!非禹其谁能及之?”

见舞《招箾》,曰:“德至矣哉,大矣,如天之无不焘\footnote{覆盖。}也,如地之无不载也,虽甚盛德,无以加矣。观止矣,若有他乐,吾不敢观。”
\end{yuanwen}

季札又观赏了《象箾》、《南籥》舞,然后说:“舞蹈真优美啊,但是还有少许遣憾之处。”

季札又观赏了《大武》舞,说:“舞蹈真美妙啊,周朝的盛德大概就是如此吧?”

他又观赏了《韶护》舞,说:“圣人那么伟大,还为自己的德行感到惭愧,可见成为圣人实在很难呀!”

季札又观赏了《大夏》舞,说:“舞蹈真美妙啊,勤劳于民事而不居功自傲!除了大禹还有谁能做到?”

他又观赏了《招箾》舞,说道:“美德修行至此可以说是极点了,真是太伟大了,就像上天一样没有覆盖不到的地方,就像大地一样没有承载不到之物,盛德已经达到极点,再无以复加了。观赏礼乐心满意足了,即使有其他礼乐,我也不敢再要求观赏了。”

\begin{yuanwen}
去鲁,遂使齐。说晏平仲\footnote{晏婴,字平仲,春秋时期齐国政治家。}曰:“子速纳邑与政。无邑无政,乃免于难。齐国之政将有所归;未得所归,难未息也。”

故晏子因陈桓子以纳政与邑,是以免于栾高之难。
\end{yuanwen}

季札离开鲁国,又出使到齐国。他劝晏平仲说:“您赶快交出封地和权力。没有了封地和政权,才能免除灾难。齐国的政权将要易手,在不能确定归入谁手之时,灾难是不会停止的。”

因此晏子通过陈桓子将自己的封地和权力都交了出去,所以他才在栾施、高强两大豪门的权力斗争中幸免于难。

\begin{yuanwen}
去齐,使于郑。见子产,如旧交。谓子产曰:“郑之执政侈,难将至矣,政必及子。子为政,慎以礼。不然,郑国将败。”

去郑,適卫。说蘧\footnote{qú}瑗、史狗、史鰌、公子荆、公叔发、公子朝曰:“卫多君子,未有患也。”
\end{yuanwen}

季札离开齐国,出使到郑国。他见到子产,就像见到老朋友一样。他对子产说:“郑国的执政者荒淫奢侈,灾难即将到来,政权一定会落入您的手中。一旦您主政,要谨慎地以礼治国。否则郑国就会衰败。”

季札离开郑国,前往卫国,他对蘧援、史狗、史鰌、公子荆、公叔发、公子朝说道:“卫国有很多的君子,没有祸患。”

\begin{yuanwen}
自卫如晋,将舍于宿,闻锺声,曰:“异哉!吾闻之,辩而不德,必加于戮。夫子获罪于君以在此,惧犹不足,而又可以畔乎?夫子\footnote{指孙林父。}之在此,犹燕之巢于幕也。君在殡而可以乐乎?”遂去之。

文子闻之,终身不听琴瑟。
\end{yuanwen}

季札从卫国前往晋国,准备在宿地住下时听到了钟声,他说道:“真奇怪啊!我听说,有辩才却不修德行,一定会招来杀身之祸。孙林父得罪了君王还住在这里,害怕还来不及,怎么还在敲钟奏乐?孙林父住在这里,就好像燕子在幕布上筑巢一样非常危险。君王还没有安葬,可以敲钟奏乐吗?”于是离开这里。

孙林父听到了季札的话,一直到去世都没有再听过奏乐。

\begin{yuanwen}
適晋,说赵文子、韩宣子、魏献子曰:“晋国其萃于三家乎!”

将去,谓叔向\footnote{姓羊舌,名肸,晋国大夫。}曰:“吾子勉之!君侈而多良,大夫皆富,政将在三家。吾子直,必思自免于难。”
\end{yuanwen}

季札前往晋国,对赵文子、韩宣子、魏献子说:“晋国的国政将要落到你们三家手中了吧!”

他将要离开晋国,对叔向说道:“您一定要努力啊!晋国国君奢侈无度但是还有很多良臣,大夫们都很富有,政权将落到那三家手中。您为人正直,一定要想办法让自己免遭祸患。”

\begin{yuanwen}
季札之初使,北过徐君。徐君好季札剑,口弗敢言。季札心知之,为使上国,未献。还至徐,徐君已死,于是乃解其宝剑,系之徐君冢树而去。从者曰:“徐君已死,尚谁予乎?”

季子曰:“不然。始吾心已许之,岂以死倍吾心哉!”
\end{yuanwen}

季札刚从吴国出来出使的时候,北上的途中曾经去拜访徐国的国君。徐君很喜欢季札的宝剑,但是口中没敢说出来。季札心里知道了,但是因为出使中原各国要用到宝剑,所以就没有将宝剑献给徐君。返回时又到了徐国,徐君已经去世了,季札于是便解下宝剑,挂在徐君墓旁的树上后转身而去。随行的人说道:“徐君已经去世了,宝剑还给谁呢?”

季札说道:“不对。当初我已经在心里想好了要将宝剑送给他,怎么能因他已经去世就违背我的心意呢!”

\begin{yuanwen}
七年,楚公子围弑其王夹敖而代立,是为灵王。

十年,楚灵王会诸侯而以伐吴之硃方,以诛齐庆封。吴亦攻楚,取三邑而去。

十一年,楚伐吴,至雩娄。

十二年,楚复来伐,次于乾谿,楚师败走。
\end{yuanwen}

七年(前541年),楚国公子围杀死楚王夹敖自立为王,就是楚灵王。

十年(前538年),楚灵王会盟诸侯攻打吴国的朱方县,想要杀掉齐国原来的相国庆封。吴国也出兵攻打楚国,攻取了楚国的三座城池后撤退。

十一年(前537年),楚国出兵讨伐吴国,一直打到雩娄。

十二年(前536年),楚国又一次讨伐吴国,在乾谿驻军,后来战败撤走。

\begin{yuanwen}
十七年,王馀祭卒,弟馀眛立。王馀眛二年,楚公子弃疾弑其君灵王代立焉。
\end{yuanwen}

十七年(前531年),吴王馀祭去世,他的弟弟馀眜继位。吴王馀眜二年(公元前529年),楚国公子弃疾杀死楚灵王,自立为楚王。

\begin{yuanwen}
四年,王馀眛卒,欲授弟季札。季札让,逃去。于是吴人曰:“先王有命,兄卒弟代立,必致季子。季子今逃位,则王馀眛后立。今卒,其子当代。”乃立王馀眛之子僚为王。
\end{yuanwen}

四年(前527年),吴王馀眜去世,想将王位传给弟弟季札。季札推让并逃走。于是吴国人说:“先王有遗命,哥哥去世由弟弟继位,一定要将王位传给季子。季子如今逃走不接受王位,那么馀眜就是兄弟中最后一个继位的君王。如今他去世了,他的儿子应当继立。”于是馀眜的儿子僚便被立为吴王。

\begin{yuanwen}
王僚二年,公子光伐楚,败而亡王舟\footnote{吴国先王所乘之舟,名馀皇。}。光惧,袭楚,复得王舟而还。
\end{yuanwen}

吴王僚二年(前525年),公子光讨伐楚国,打了败仗并丢失了先王的座船。他非常害怕,偷袭楚军,夺回先王的座船后回到吴国。

\begin{yuanwen}
五年,楚之亡臣伍子胥来饹(奔),公子光客之。公子光者,王诸樊之子也。常以为吾父兄弟四人,当传至季子。季子即不受国,光父先立。即不传季子,光当立。阴纳贤士,欲以袭王僚。
\end{yuanwen}

五年(前522年),楚国逃亡在外的大臣伍子胥来到吴国,公子光以宾客之礼节接待他。公子光是吴王诸樊的儿子。他一直认为:我的父辈兄弟四人,王位应当传给季子。季子既然不接受国家,我的父亲又是最先继位为王的。既然不传给季子,就应当由我继承王位。所以他在暗中招纳贤士,想要袭杀王僚。

\begin{yuanwen}
八年,吴使公子光伐楚,败楚师,迎楚故太子建母于居巢以归。因北伐,败陈、蔡之师。

九年,公子光伐楚,拔居巢、锺离。初,楚边邑卑梁氏之处女与吴边邑之女争桑,二女家怒相灭,两国边邑长闻之,怒而相攻,灭吴之边邑。吴王怒,故遂伐楚,取两都而去。
\end{yuanwen}

八年(前519年),吴国派公子光攻打楚国,战胜楚军,将楚国前太子建的母亲从居巢接回吴国。随后乘胜北伐,打败陈国、蔡国的军队。

九年(前518年),公子光讨伐楚国,攻占了居巢、钟离。起初,楚国边城上一位卑梁氏的少女和吴国边城的妇女因为采桑之事而发生争执,两位女子的家人都很气愤,互相攻击杀戮,两国边城的长官听说了这件事,因气愤而互相攻打,吴国的边城被灭掉。吴王知道以后勃然大怒,便出兵攻打楚国,占领了居巢、钟离这两座城邑后撤退。

\begin{yuanwen}
伍子胥之初奔吴,说吴王僚以伐楚之利。公子光曰:“胥之父兄为僇于楚,欲自报其仇耳。未见其利。”

于是伍员知光有他志,乃求勇士专诸,见之光。光喜,乃客伍子胥。子胥退而耕于野,以待专诸之事。
\end{yuanwen}

伍子胥刚到吴国的时候,向吴王僚陈述进攻楚国的好处。公子光说道:“伍子胥的父亲兄长被楚王所害,他是想为自己报私仇而已。我没有看到攻打楚国对吴国有什么好处。”

于是伍子胥看出公子光心怀异志,便找到了一位名叫专诸的勇士并引荐他见公子光。公子光很高兴,对伍子胥以礼相待。伍子胥退隐退到乡间耕作,等待专诸采取行动。

\begin{yuanwen}
十二年冬,楚平王卒。

十三年春,吴欲因楚丧而伐之,使公子盖馀、烛庸以兵围楚之六、灊。使季札于晋,以观诸侯之变。楚发兵绝吴兵后,吴兵不得还。于是吴公子光曰:“此时不可失也。”

告专诸曰:“不索何获!我真王嗣,当立,吾欲求之。季子虽至,不吾废也。”

专诸曰:“王僚可杀也。母老子弱,而两公子将兵攻楚,楚绝其路。方今吴外困于楚,而内空无骨鲠之臣,是无柰我何。”

光曰:“我身,子之身也。”

四月丙子,光伏甲士于窟室,而谒\footnote{请。}王僚饮。王僚使兵陈于道,自王宫至光之家,门阶户席,皆王僚之亲也,人夹持铍\footnote{两刃剑。}。公子光详\footnote{通“佯”,假装。}为足疾,入于窟室,使专诸置匕首于炙鱼之中以进食。手匕首刺王僚,铍交于匈\footnote{同“胸”。},遂弑王僚。公子光竟代立为王,是为吴王阖\footnote{hé}庐。阖庐乃以专诸子为卿。
\end{yuanwen}

十二年(前515年)冬季,楚平王去世。

十三年(前514年)春季,吴国想趁楚国国丧的机会攻打楚国,于是便派公子盖馀、烛庸率军围攻楚国的六、灊。又派季札前往晋国,观察诸侯国的反应。楚国发兵截断吴军后路,吴军无法撤军。于是吴国公子光说:“这个机会不可错过啊。”

他对专诸说:“不动手什么都得不到!我才是真正的王位继承人,应当继位,我要夺回王位。即便季子回来了,也不能废掉我。”

专诸说:“可以杀掉王僚。现在在国内只有他的老母幼子,两公子率领军队攻打楚国,楚军已经截断了他们的后路。如今吴国在外正被楚军围困,在内又没有刚直忠正的臣子,没有谁能阻挡我们了。”

公子光说道:“我的身体,就是你的身体,我们福祸与共。”

四月丙子日,公子光在自己家中的地下室埋伏了士兵,邀请王僚前来饮酒。王僚让士兵排列在沿途的路旁,从王宫到公子光的家,外门、台阶、内门、座位,都有王僚的亲兵,各个人都手持利刃。公子光假装脚疼,躲进了地下室,命专诸将匕首藏在烤鱼的腹中,再将鱼送给王僚。专诸用手抽出匕首刺向王僚,尽管他被王僚的亲兵刺中了胸膛,还是杀死了王僚。公子光最终自立为王,就是吴王阖庐。阖庐任命专诸的儿子做了卿。

\begin{yuanwen}
季子至,曰:“苟先君无废祀,民人无废主,社稷有奉,乃吾君也。吾敢谁怨乎?哀死事生,以待天命。非我生乱,立者从之,先人之道也。”

复命,哭僚墓,复位而待。吴公子烛庸、盖馀二人将兵遇围于楚者,闻公子光弑王僚自立,乃以其兵降楚,楚封之于舒。
\end{yuanwen}

季子回到吴国,说道:“只要对先王的祭祀没有废绝,人民不会没有君主,社稷的神得到供奉,那么他就是我的君王。我还敢责怪谁呢?我只有哀悼死去的,侍奉活着的,以等待天命的安排。祸乱并非由我而生,谁被立为君王我就顺从谁,这是先人的原则啊。”

季子去了王僚的墓,报告自己出使晋国的经过,又哭祭一番,然后回到自己的岗位等待阖庐安排。被楚军包围的吴公子烛庸、盖馀两人,听说公子光杀了王僚自立为王,就率军投降了楚国,楚国将舒地封给他们。

\begin{yuanwen}
王阖庐元年,举伍子胥为行人而与谋国事。楚诛伯州犁,其孙伯嚭\footnote{pǐ}亡奔吴,吴以为大夫。
\end{yuanwen}

吴王阖庐元年(前514年),提拔伍子胥担任行人官,和他谋划国家大事。楚国杀了伯州犁,伯州犁的孙子伯嚭逃到了吴国,吴国任命他为大夫。

\begin{yuanwen}
三年,吴王阖庐与子胥、伯嚭将兵伐楚,拔舒,杀吴亡将二公子。光谋欲入郢,将军孙武曰:“民劳,未可,待之。”

四年,伐楚,取六与灊。

五年,伐越,败之。

六年,楚使子常囊瓦伐吴。迎而击之,大败楚军于豫章,取楚之居巢而还。
\end{yuanwen}

三年(前512年),吴王阖庐和伍子胥、伯嚭率军进攻楚国,攻克舒邑,杀死吴国的亡将烛庸、盖馀两公子。吴王想趁机攻入郢都,将军孙武说道:“现在百姓过于疲劳,不可进军,还是等待机会吧。”

四年(前511年),讨伐楚国,攻占了六、灊两座城池。

五年(前510年),讨伐越国,打败了越军。

六年(前509年),楚国派子常率军攻打吴国。吴国出兵迎战,在豫章大败楚军,夺取了楚国的居巢后回国。

\begin{yuanwen}
九年,吴王阖庐请伍子胥、孙武曰:“始子之言郢未可入,今果如何?”

二子对曰:“楚将子常贪,而唐、蔡皆怨之。王必欲大伐,必得唐、蔡乃可。”

阖庐从之,悉兴师,与唐、蔡西伐楚,至于汉水。楚亦发兵拒吴,夹水陈。吴王阖庐弟夫槩\footnote{gài}欲战,阖庐弗许。夫槩曰:“王已属\footnote{嘱托,托付。}臣兵,兵以利为上,尚何待焉?”

遂以其部五千人袭冒楚,楚兵大败,走。于是吴王遂纵兵追之。比\footnote{及,等到。}至郢,五战,楚五败。楚昭王\footnote{名珍,楚平王之子。}亡出郢,奔郧。郧公弟欲弑昭王,昭王与郧公饹(奔)随。而吴兵遂入郢。子胥、伯嚭鞭平王之尸以报父雠。
\end{yuanwen}

九年(前506年),吴王阖庐恭敬地问伍子胥、孙武道:“当初你们说郢都不可以进入,如今如何?”

两人回答道:“楚国将领子常贪婪,唐国、蔡国都很怨恨他。如果君王一定要大举攻打,那就一定要得到唐国、蔡国的援助才可以。”

阖庐采纳了他们的意见,出动吴国所有的军队,与唐国、蔡国的军队一起向西攻打楚国,到达汉水的边上。楚国也派出军队阻击吴军,两军隔着汉水列阵。吴王阖庐的弟弟夫槩想要作战,阖庐不同意。夫槩说:“现在君王已经将军队交由我指挥,用兵以抓住战机为上策,不动手还等什么呢?”

于是,夫槩率领他的部下五千人突袭楚军,楚军大败而逃。于是吴王纵兵追击。一直追到郢都,两军五次交战,楚军五次被打败。楚昭王逃离郢都,逃奔郧县。郧公的弟弟想要杀死昭王,昭王和郧公便逃到了随国。吴军于是进入郢都。伍子胥、伯嚭用鞭子抽打楚平王的尸体以报其杀父之仇。

\begin{yuanwen}
十年春,越闻吴王之在郢,国空,乃伐吴。吴使别兵击越。楚告急秦,秦遣兵救楚击吴,吴师败。阖庐弟夫槩见秦越交败吴,吴王留楚不去,夫槩亡归吴而自立为吴王。阖庐闻之,乃引兵归,攻夫槩。夫槩败奔楚。楚昭王乃得以九月复入郢,而封夫槩于堂谿,为堂谿氏。

十一年,吴王使太子夫差伐楚,取番。楚恐而去郢徙鄀。
\end{yuanwen}

十年(前505年)春季,越王听说吴王阖庐在楚国的郢都,国内空虚,于是讨伐吴国。吴国派另外一支军队抗击越军。楚国向秦国告急,秦国派兵救援楚国,攻打吴军,吴军被打败。阖庐的弟弟夫槩见秦军、越军先后打败吴军,吴王还留在楚国不回吴国,便逃回吴国自立为吴王。阖庐知道以后,这才率军回国,攻打夫槩。夫槩战败后逃奔楚国。楚昭王这才得以在九月回到了郢都,将夫槩封在堂谿,称为堂谿氏。

十一年(前504年),吴王阖庐派太子夫差讨伐楚国,攻取了番邑。楚国恐惧,就将国都从郢迁到了鄀。

\begin{yuanwen}
十五年,孔子相鲁。
\end{yuanwen}

十五年(前500年),孔子在鲁定公与齐景公于夹谷相会时担任傧相。

\begin{yuanwen}
十九年夏,吴伐越,越王句践迎击之槜李。越使死士挑战,三行造吴师,呼,自刭。吴师观之,越因伐吴,败之姑苏,伤吴王阖庐指\footnote{这里指“脚趾”。},军卻七里。吴王病伤而死。阖庐使立太子夫差,谓曰:“尔而忘句践杀汝父乎?”

对曰:“不敢!”三年,乃报越。
\end{yuanwen}

十九年(前496年)夏季,吴国派兵攻打越国,越王句践在槜李迎战吴军。越王派出敢死队向吴军挑战,敢死队排成三行冲向吴军,他们高声呼喊着,最后全都自刎而死。吴军士兵争相观看,越军抓住这个机会,在姑苏打败了吴军,伤到了吴王阖庐的脚趾,吴军后退七里。后来吴王阖庐创伤发作而死,他遗命立太子夫差为王,对夫差说:“你能忘了句践杀死你父亲的事吗?”

夫差回答道:“不敢!”三年以后,夫差报复了越国。

\begin{yuanwen}
王夫差元年,以大夫伯嚭为太宰。习战射,常以报越为志。

二年,吴王悉精兵以伐越,败之夫椒,报姑苏也。越王句践乃以甲兵五千人栖于会稽,使大夫种\footnote{即文种,春秋时期谋略家。}因吴太宰嚭而行成,请委国为臣妾。吴王将许之,伍子胥谏曰:“昔有过氏杀斟灌以伐斟寻,灭夏后帝相。帝相之妃后缗方娠,逃于有仍而生少康。少康为有仍牧正。有过又欲杀少康,少康奔有虞。有虞思夏德,于是妻之以二女而邑之于纶,有田一成,有众一旅。后遂收夏众,抚其官职。使人诱之,遂灭有过氏,复禹之绩,祀夏配天,不失旧物。今吴不如有过之彊,而句践大于少康。今不因此而灭之,又将宽之,不亦难乎!且句践为人能辛苦,今不灭,后必悔之。”

吴王不听,听太宰嚭,卒许越平,与盟而罢兵去。
\end{yuanwen}

吴王夫差元年(前495年),任命大夫伯嚭为太宰。吴王坚持训练军队,始终以向越国报仇为志。

二年(前494年),吴王夫差派出吴国所有的精兵攻打越国,在夫椒战胜越军,报了姑苏战败的仇。越王句践带领五千士兵退守会稽,派大夫文种通过吴国的太宰伯嚭向吴国求和,越国愿意整个国家都做吴国的奴仆。吴王夫差准备同意,伍子胥进谏说:“从前有过氏杀了斟灌后又去进攻斟寻,灭掉了夏的君王帝相。帝相的妃子后缗当时有孕在身,逃到了有仍国,生下了少康。后来少康做了有仍的牧正官。有过氏又想杀掉少康,少康又逃到了有虞国。有虞氏念及夏的恩德,就将自己的两个女儿嫁给少康,又将他封在纶邑,辖有方圆十里的土地,拥有五百名部下。后来,少康招纳夏的遗民,整顿官制。又派人打入有过氏内部对其进行引诱,终于灭掉了有过氏,再现了夏禹的功绩,祭祀时以夏祖配享上天,没有失去夏原有的河山。现在吴国不如有过氏强盛,但是句践的势力却比少康强大。如果现在不趁这个机会灭掉他,却要宽容他,这是为将来埋下祸患!况且句践这个人能够忍受艰辛苦难,如今不灭掉他,以后肯定会后悔。”

吴王不肯听从,只听信太宰伯嚭的话,最终答应了越国的求和,与越国签订盟约后撤兵回国。

\begin{yuanwen}
七年,吴王夫差闻齐景公死而大臣争宠,新君弱,乃兴师北伐齐。子胥谏曰:“越王句践食不重味,衣不重采,吊死问疾,且欲有所用其众。此人不死,必为吴患。今越在腹心疾而王不先,而务齐,不亦谬乎!”

吴王不听,遂北伐齐,败齐师于艾陵。至缯,召鲁哀公而徵百牢。季康子使子贡以周礼说太宰嚭,乃得止。因留略地\footnote{开拓领地。}于齐鲁之南。

九年,为驺伐鲁,至,与鲁盟乃去。

十年,因伐齐而归。

十一年,复北伐齐。
\end{yuanwen}

七年(前489年),吴王夫差听说齐景公去世而大臣们争权,新立的国君幼小势弱,于是出兵北上征讨齐国。伍子胥劝谏他说:“越王句践吃饭时不会有第二道菜肴,穿的衣服不会有两种颜色,吊唁死去的人,慰问生病的人,他这是想使用他的民众。这个人不死,将来一定会成为吴国的祸患。如今越国是我们的心腹大患,君王不先除掉越国,反倒去讨伐齐国,不是很荒谬吗!”

吴王没有听从伍子胥的劝告,于是北上进攻齐国,在艾陵打败了齐军。到了缯邑,召见鲁哀公,要求鲁国以百牢之礼招待吴王夫差。季康子派子贡列举周礼劝说太宰伯嚭,吴国这才不再要求。吴王就此留下来略取齐国、鲁国两国南部的土地。

九年(前487年),吴王又为了驺国征讨鲁国,抵达鲁国后,和鲁国订了盟约才撤兵而去。

十年(前486年),趁势征讨齐国后班师回国。

十一年(前485年),再次北上征讨齐国。

\begin{yuanwen}
越王句践率其众以朝吴,厚献遗之,吴王喜。唯子胥惧,曰:“是弃吴也。”

谏曰:“越在腹心,今得志于齐,犹石田,无所用。且《盘庚之诰》有‘颠越勿遗,商之以兴’。”

吴王不听,使子胥于齐,子胥属其子于齐鲍氏,还报吴王。吴王闻之,大怒,赐子胥属镂之剑以死。

将死,曰:“树吾墓上以梓,令可为器。抉吾眼置之吴东门,以观越之灭吴也。”
\end{yuanwen}

越王句践率领部下朝见吴王,奉上丰厚的礼物,吴王非常高兴。只有伍子胥忧心忡忡,他说道:“吴国这是要被丢弃了啊。”

他劝谏吴王道:“越国近在腹心,现在我国在齐国大获成功,但是好像得到了一块石头田地,没什么用途。况且《盘庚之诰》中有‘对于悖逆的坏家伙一定要斩草除根,才有商朝的兴旺’的话。”

吴王并不听从他的劝告,派伍子胥出使齐国,伍子胥将自己的儿子托付给齐国的大夫鲍氏,然后回国向吴王报告出使的情况。吴王听说他的做法后大怒,赐给伍子胥属镂宝剑命其自杀。

伍子胥在死之前说道:“在我的墓上种上梓树,长大后让它可以做棺材。将我的眼睛挖出来放在吴国的东城门之上,让我看着越国灭掉吴国。”

\begin{yuanwen}
齐鲍氏弑齐悼公。吴王闻之,哭于军门外三日,乃从海上攻齐。齐人败吴,吴王乃引兵归。
\end{yuanwen}

齐国的大夫鲍氏杀死齐悼公。吴王知道后,在军门外痛哭三天,便率军从海上进攻齐国。齐军打败吴军,吴王只好引败兵回国。

\begin{yuanwen}
十三年,吴召鲁、卫之君会于橐皋。
\end{yuanwen}

十三年(前483年),吴王征召鲁国、卫国的君王在橐皋会盟。

\begin{yuanwen}
十四年春,吴王北会诸侯于黄池,欲霸中国以全周室。

六月丙子,越王句践伐吴。

乙酉,越五千人与吴战。

丙戌,虏吴太子友。

丁亥,入吴。

吴人告败于王夫差,夫差恶其闻也。或泄其语,吴王怒,斩七人于幕下。

七月辛丑,吴王与晋定公\footnote{名午,晋顷公之子。}争长。吴王曰:“于周室我为长。”

晋定公曰:“于姬姓我为伯。”

赵鞅怒,将伐吴,乃长晋定公。吴王已盟,与晋别,欲伐宋。太宰嚭曰:“可胜而不能居也。”

乃引兵归国。国亡太子,内空,王居外久,士皆罢\footnote{通“疲”。}敝,于是乃使厚币以与越平。
\end{yuanwen}

十四年(前482年)春季,吴王北上在黄池会盟诸侯,想要称霸中原从而保全周室。

六月丙子日,越王句践攻打吴国。

乙酉日,五千越军和吴军交战。

丙戌日,越军俘获吴国的太子友。

丁亥日,越军进入吴国境内。

吴国人将战败的消息上报给吴王夫差,夫差不想让诸侯听说这个坏消息。但有人泄漏了消息,吴王大怒,在军营幕下杀了七个人。

七月辛丑日,吴王和晋定公争夺盟主。吴王说道:“在周王室宗族当中,我祖先的辈份最长。”

晋定公说:“姬姓诸侯当中,晋国从文公以来长期居于霸主之位。”

晋国的赵鞅非常生气,要出兵讨伐吴国,这才让晋定公作了盟主。吴王和诸侯签订盟约以后,和晋定公告别,想要攻打宋国。太宰伯嚭说道:“即便打败了宋国,也无法将其占有。”

吴王这才率领吴军回国。这时吴国太子被俘,国内空虚,吴王长年在外征战,士兵都已疲惫不堪,于是便派使者带上重金去和越国讲和。

\begin{yuanwen}
十五年,齐田常杀简公。
\end{yuanwen}

十五年(前481年),齐国田常杀死了齐简公。

\begin{yuanwen}
十八年,越益彊。越王句践率兵复伐败吴师于笠泽。楚灭陈。
\end{yuanwen}

十八年(前478年),越国更加强大。越王句践率领军队再次进攻吴国,在笠泽打败了吴军。楚国灭掉了陈国。

\begin{yuanwen}
二十年,越王句践复伐吴。

二十一年,遂围吴。

二十三年十一月丁卯,越败吴。越王句践欲迁吴王夫差于甬东,予百家居之。吴王曰:“孤老矣,不能事君王也。吾悔不用子胥之言,自令陷此。”遂自刭死。

越王灭吴,诛太宰嚭,以为不忠,而归。
\end{yuanwen}

二十年(前476年),越王句践再次进攻吴国。

二十一年(前475年),越军包围了吴国的都城。

二十三年(前473年)十一月丁卯日,越军打败了吴军。越王句践想把吴王夫差流放到甬东,给他一百户民家的地域让他居住。吴王说:“我已经老了,不能再侍奉君王了。我很后悔当初没有听伍子胥的话,让自己落到今天这个地步。”于是自杀而死。

越王灭掉吴国,诛杀了太宰伯嚭,越王认为他不忠于国家,然后率领军队回到越国。

\begin{yuanwen}
太史公曰:孔子言“太伯可谓至德矣,三以天下让,民无得而称焉”。余读《春秋》古文,乃知中国之虞与荆蛮句吴兄弟也。延陵季子之仁心,慕义无穷,见微而知清浊。呜呼,又何其闳览博物君子也!
\end{yuanwen}

太史公说:孔子说的“吴太伯可称得上德行完美啊,三次以天下相让,百姓都找不到赞美他的言词”。我读《春秋》古文,才知道中原的虞国和荆蛮的吴国原来是兄弟之邦啊。延陵季子的仁爱善心,仰慕道义追求终生,观察细微便知清浊。啊,又是一位多么见多识广、博学多知的君子啊!

\begin{yuanwen}
太伯作吴,高让雄图。周章受国,别封于虞。寿梦初霸,始用兵车。三子递立,延陵不居。光既篡位,是称阖闾。王僚见杀,贼由专诸。夫差轻越,取败姑苏。甬东之耻,空惭伍胥。
\end{yuanwen}


\part{卷三十二}
\chapter{齐太公世家第二}

茅坤:「太史公叙太公始为阴谋处,兵家者言也,非是。」

\begin{yuanwen}
太公望吕尚\footnote{姜姓,名尚,其祖先曾封于吕,故以吕为氏。}者,东海上人。其先祖尝为四岳,佐禹平水土甚有功。虞夏之际封于吕,或封于申,姓姜氏。夏商之时,申、吕或封枝庶子孙,或为庶人,尚其后苗裔也。本姓姜氏,从其封姓,故曰吕尚。
\end{yuanwen}

太公望吕尚是东海边上的人。他的先祖曾经是“四岳”之一,辅佐大禹治水,立下很大的功劳。虞舜、夏禹的时候被封在吕,还有的被封在申,姜姓。夏、商的时候,申、吕这两个地方有的被封给旁支的子孙,有的则成为平民,吕尚是他们的后代。姜尚本来姓姜,用他的封邑做姓氏,所以被称为吕尚。

\begin{yuanwen}
吕尚盖尝穷困,年老矣,以渔钓奸\footnote{通“干”,干谒,进见。}周西伯。西伯将出猎,卜之,曰“所获非龙非彨\footnote{通“螭”,无角的龙。},非虎非罴;所获霸王之辅”。

于是周西伯猎,果遇太公于渭之阳,与语大说,曰:“自吾先君太公曰‘当有圣人适周,周以兴’。子真是邪?吾太公望子久矣。”

故号之曰“太公望”,载与俱归,立为师。
\end{yuanwen}

吕尚曾经非常穷困,到了年老的时候,通过钓鱼求见周西伯侯昌。西伯当时准备外出打猎,临行前占卜,占卜的结果说:“所得到的不是龙不是螭,不是虎也不是熊,而是能够成就霸业的辅佐者。”

于是西伯侯便外出打猎,果然在渭水北岸遇到了太公,和他交谈后非常高兴,说道:“早就听我国先君太公说‘会有圣人到周国来,周国会得以兴盛’。您就是这个人吧?我们的太公盼望您已经很久啦。”

所以便称吕尚为“太公望”,和他一起乘车回去,尊为军师。

\begin{yuanwen}
或曰,太公博闻,尝事纣。纣无道,去之。游说诸侯,无所遇,而卒西归周西伯。或曰,吕尚处士,隐海滨。周西伯拘羑里,散宜生、闳夭素知而招吕尚。吕尚亦曰“吾闻西伯贤,又善养老,盍往焉”。三人者为西伯求美女奇物,献之于纣,以赎西伯。西伯得以出,反国。言吕尚所以事周虽异,然要之为文武师。
\end{yuanwen}

还有人说:太公博学多闻,曾经辅佐商纣王。因为纣王暴虐无道,太公就离开了。他到各国游说诸侯,但是一直没有遇到赏识他的人,最后向西归附周西伯侯。还有人说,吕尚是一个德才兼备却隐居而不仕的人,隐居在海边。周西伯侯被纣王囚禁在羑里,散宜生和闳夭原来就知道吕尚这个人,于是便请他出山。吕尚也这样说:“我听说西伯贤德,又尊重奉养老年人,何不到他那里去呢?”于是三人便替西伯搜罗美女和稀奇的宝物献给纣王,从而赎回西伯。西伯这才得以被释放,回到自己的国家。关于吕尚是如何归周的虽然有不一样的说法,然而核心都是说他成为了周文王、武王的师辅之臣。

\begin{yuanwen}
周西伯昌之脱羑里归,与吕尚阴谋修德以倾商政,其事多兵权与奇计,故后世之言兵及周之阴权皆宗太公为本谋。周西伯政平,及断虞芮之讼,而诗人称西伯受命曰文王。伐崇、密须、犬夷,大作丰邑。天下三分,其二归周者,太公之谋计居多。
\end{yuanwen}

周西伯昌从羑里脱身回国以后,与吕尚暗中谋划施行德政,以推翻商朝的统治,其中很多是用兵的权谋和奇计,所以后世之人论及用兵之道以及周暗中施用的权谋,都推崇太公是其中主要的策划者。周西伯侯昌执政公平,到他裁决了虞、芮两国的争端,诗人称西伯是承受天命的文王。文王征讨崇国、密须、犬夷,大规模地建设丰邑。当时的天下分成三部份,其中两部分都归为周国所有,这其中太公的谋略占了大部分。

\begin{yuanwen}
文王崩,武王即位。九年,欲修文王业,东伐以观诸侯集否。师行,师尚父左杖黄钺,右把白旄以誓,曰:“苍兕\footnote{sì}苍兕,总尔众庶,与尔舟楫,后至者斩!”

遂至盟津\footnote{亦名“孟津”,黄河渡口,在今河南省孟津县东。}。诸侯不期而会者八百诸侯。诸侯皆曰:“纣可伐也。”

武王曰:“未可。”

还师,与太公作此《太誓》。
\end{yuanwen}

文王去世后,武王继位。九年,武王要继续文王的事业,准备东征,以观察诸侯是否前来会合。军队临出发时,被尊为“师尚父”的吕尚左手拿着装饰着黄金的大斧,右手握着装饰着白牦牛尾的军旗誓师,他说道:“苍兕啊苍兕,集合你的民众,聚合船只,迟到者斩!”

于是大军到了盟津,事先没有约定就主动前来的诸侯有八百家之多。诸侯都这样说:“可以征讨纣王了。”

武王说道:“还不可以。”

武王又率领军队回国,和太公一起写了《太誓》。

\begin{yuanwen}
居二年,纣杀王子比干\footnote{纣王的叔父。},囚箕子\footnote{纣王的伯父。}。武王将伐纣,卜,龟兆不吉,风雨暴至。群公尽惧,唯太公强之劝武王,武王于是遂行。

十一年正月甲子,誓于牧野,伐商纣。纣师败绩。纣反走,登鹿台,遂追斩纣。明日,武王立于社,群公奉明水,卫康叔封布采席,师尚父牵牲,史佚策祝,以告神讨纣之罪。散鹿台之钱,发钜桥之粟,以振贫民。封比干墓,释箕子囚。迁九鼎,脩周政,与天下更始。师尚父谋居多。
\end{yuanwen}

过了两年,纣王杀死王子比干,囚禁了箕子。武王准备征讨纣王,用龟甲占卜,卦象不吉利,这时又突然风雨大作。大臣们都很惊惧,只有太公坚决劝说武王就此出兵,武王于是率军出征。

十一年正月甲子日,武王在牧野誓师,讨伐商纣。商纣的军队战败。纣王逃回朝歌,登上鹿台,于是追兵杀了纣王。第二天,武王立于社坛前,大臣们手捧净水,卫康叔封铺好彩席,师尚父吕尚牵来祭祀的牲畜,史佚宣读祭神的策书祝文,敬告神明声讨纣王的罪行。散发了纣王放在鹿台的钱财,打开钜桥的粮仓,以赈济贫穷的百姓。又封高比干的坟墓,释放被囚禁的箕子。将九只宝鼎迁到周,修明周朝的政事,和天下的人民一起开创新的时代。在这些举措中,出自师尚父的谋略占多数。

\begin{yuanwen}
于是武王已平商而王天下,封师尚父于齐营丘。东就国,道宿行迟。逆旅之人曰:“吾闻时难得而易失。客寝甚安,殆非就国者也。”

太公闻之,夜衣而行,犁明至国。莱侯来伐,与之争营丘。营丘边莱。莱人,夷也,会纣之乱而周初定,未能集远方,是以与太公争国。
\end{yuanwen}

这时武王已经平定商纣,称王于天下,便将师尚父封在了齐国的营丘。师尚父东行前往自己的封国,时住时行,前进得很慢。客舍的人对他说:“我听说时机很难得到却容易失去。客人您睡得非常安稳,实在不像是一个要去就任封国的人。”

太公听后,穿上衣服连夜而行,天亮就到了自己的封国。莱侯前来攻打,和太公争夺营丘。营丘在莱国的边境。莱人是夷族,趁着纣王无道而周朝刚刚平定天下,还没有安抚边远地区的时机,和太公争夺这片土地。

\begin{yuanwen}
太公至国,脩政,因其俗,简其礼,通商工之业,便鱼盐之利,而人民多归齐,齐为大国。及周成王少时,管蔡作乱,淮夷畔周,乃使召康公命太公曰:“东至海,西至河,南至穆陵,北至无棣,五侯九伯,实得征之。”

齐由此得征伐,为大国。都营丘。
\end{yuanwen}

太公到了自己的封国以后,修明政治,顺应风俗,简化礼仪,开放商工各业,发展渔业盐业的优势,所以很多百姓都归附齐国,齐国也成为了大国。到了周成王幼年即位之时,管叔、蔡叔作乱,淮夷也背叛了周朝,朝廷就派召康公授命太公说道:“东边到大海,西边到黄河,南边到穆陵,北边到无棣,这里的各等诸侯、各地长官,如有作乱,你都可以征讨。”

齐国因此有征伐之权,成为大国,定都在营丘。

\begin{yuanwen}
盖太公之卒百有馀年,子丁公吕伋立。丁公卒,子乙公得立。乙公卒,子癸公慈母立。癸公卒,子哀公不辰立。
\end{yuanwen}

太公去世时大约已有一百多岁,他的儿子丁公吕伋继位。丁公去世,他的儿子乙公得继位。乙公去世,他的儿子癸公慈母继位。癸公去世,他的儿子哀公不辰继位。

\begin{yuanwen}
哀公时,纪侯谮之周,周烹哀公而立其弟静,是为胡公。胡公徙都薄姑,而当周夷王之时。
\end{yuanwen}

哀公在位时,纪侯在周王面前诽谤他,周王将哀公用大鼎烹死,立哀公的异母弟静为国君,就是胡公。后来胡公将都城迁到薄姑,这是周夷王时候的事。

\begin{yuanwen}
哀公之同母少弟山怨胡公,乃与其党率营丘人袭攻杀胡公而自立,是为献公。献公元年,尽逐胡公子,因徙薄姑都,治临菑。
\end{yuanwen}

哀公同母的小弟弟姜山很怨恨胡公,便和同党率领营丘人偷袭并杀死胡公后自立,他就是献公。献公元年,将胡公的儿子全部驱逐,又趁机从薄姑将都城迁走,迁到临淄。

\begin{yuanwen}
九年,献公卒,子武公寿立。武公九年,周厉王\footnote{名胡,夷王之子。}出奔,居彘。

十年,王室乱,大臣行政,号曰“共和”。

二十四年,周宣王初立。
\end{yuanwen}

九年,献公去世,他的儿子武公寿继位。武公九年,周厉王出逃,住在彘地。

十年(前841年),周王室发生叛乱,大臣们代行政事,号称“共和”。

二十四年(前827年),周宣王继位。

\begin{yuanwen}
二十六年,武公卒,子厉公无忌立。厉公暴虐,故胡公子复入齐,齐人欲立之,乃与攻杀厉公。胡公子亦战死。齐人乃立厉公子赤为君,是为文公,而诛杀厉公者七十人。
\end{yuanwen}

二十六年(前825年),武公去世,他的儿子厉公无忌继位。厉公残忍暴虐,原来的胡公之子重返齐国,齐国人想拥立他为君,便和他一起进攻并杀掉厉公。胡公的儿子也战死了。齐国人便拥立厉公的儿子赤为国君,就是文公,他诛杀了参与攻杀厉公的七十个人。

\begin{yuanwen}
文公十二年卒,子成公脱立。成公九年卒,子庄公购立。
\end{yuanwen}

文公在位十二年后去世,他的儿子成公脱继位。成公在位九年后去世,他的儿子庄公购继位。

\begin{yuanwen}
庄公二十四年,犬戎杀幽王,周东徙雒。秦始列为诸侯。五十六年,晋弑其君昭侯。
\end{yuanwen}

庄公二十四年(前771年),犬戎杀死了周幽王,周王室将都城东迁到雒邑。秦国这时开始位列诸侯。五十六年(前739年),晋国人杀了他们的国君昭侯。

\begin{yuanwen}
六十四年,庄公卒,子釐公禄甫立。
\end{yuanwen}

六十四年(前731年),庄公去世,他的儿子釐公禄甫继位。

\begin{yuanwen}
釐公九年,鲁隐公初立。十九年,鲁桓公弑其兄隐公而自立为君。
\end{yuanwen}

釐公九年(前722年),鲁隐公即位。十九年(前712年),鲁桓公杀死他的哥哥隐公后自立为君。

\begin{yuanwen}
二十五年,北戎伐齐。郑使太子忽来救齐,齐欲妻之。忽曰:“郑小齐大,非我敌。”遂辞之。
\end{yuanwen}

二十五年(前706年),北戎攻打齐国。郑侯派太子忽前来救援,齐侯想将自己的女儿嫁给他为妻。太子忽说道:“郑国小齐国大,我配不上您的女儿。”于是谢绝了。

\begin{yuanwen}
三十二年,釐公同母弟夷仲年死。其子曰公孙无知,釐公爱之,令其秩服奉养比太子。
\end{yuanwen}

三十二年(前699年),釐公的同母弟夷仲年去世,他的儿子名叫公孙无知,釐公非常喜欢他,让他在俸禄、器物服饰、饮食等方面的待遇和太子一样。

\begin{yuanwen}
三十三年,釐公卒,太子诸兒立,是为襄公。
\end{yuanwen}

三十三年(前698年),釐公去世,太子诸兒继位,就是襄公。

\begin{yuanwen}
襄公元年,始为太子时,尝与无知斗,及立,绌无知秩服,无知怨。
\end{yuanwen}

襄公元年(前697年),襄公起初做太子时曾与无知争斗,等到襄公即位后,便降低了无知的待遇,无知十分怨恨。

\begin{yuanwen}
四年,鲁桓公与夫人如齐。齐襄公故尝私通鲁夫人。鲁夫人者,襄公女弟也,自釐公时嫁为鲁桓公妇,及桓公来而襄公复通焉。鲁桓公知之,怒夫人,夫人以告齐襄公。齐襄公与鲁君饮,醉之,使力士彭生抱上鲁君车,因拉杀鲁桓公,桓公下车则死矣。鲁人以为让,而齐襄公杀彭生以谢鲁。
\end{yuanwen}

四年(前694年),鲁桓公和夫人来到齐国。齐襄公从前曾与鲁夫人私通。鲁夫人是襄公的妹妹,在釐公的时候嫁给了鲁桓公为妻,等到鲁桓公来到齐国,襄公又和她勾搭成奸。鲁桓公知道这件事以后,对夫人十分愤怒,夫人将此事告诉了齐襄公。齐襄公就和鲁桓公喝酒,将他灌醉,命大力士彭生将他抱上他的车,将其杀死,桓公下车时就已经死了。鲁国人为了这件事向齐国发难,齐襄公就杀死彭生向鲁国谢罪。

\begin{yuanwen}
八年,伐纪,纪迁去其邑。
\end{yuanwen}

八年(前690年),齐国讨伐纪国,纪国被齐国吞并。

\begin{yuanwen}
十二年,初,襄公使连称、管至父\footnote{皆为齐国大夫。}戍葵丘,瓜时而往,及瓜而代。往戍一岁,卒瓜时而公弗为发代。或为请代,公弗许。故此二人怒,因公孙无知谋作乱。连称有从妹在公宫,无宠,使之间襄公,曰“事成以女\footnote{通“汝”。}为无知夫人”。

冬十二月,襄公游姑棼,遂猎沛丘。见彘,从者曰“彭生”。公怒,射之,彘人立而啼。公惧,坠车伤足,失屦\footnote{jù}。反而鞭主屦者茀三百。茀出宫。而无知、连称、管至父等闻公伤,乃遂率其众袭宫。逢主屦茀,茀曰:“且无入惊宫,惊宫未易入也。”

无知弗信,茀示之创,乃信之。待宫外,令茀先入。茀先入,即匿襄公户间。良久,无知等恐,遂入宫。茀反与宫中及公之幸臣攻无知等,不胜,皆死。无知入宫,求公不得。或见人足于户间,发视,乃襄公,遂弑之,而无知自立为齐君。
\end{yuanwen}

十二年(前686年),早先齐襄公派连称、管至父驻守在葵丘,瓜熟的时候前去,第二年瓜熟的时候派人接替他们。连称、管至父两个人到那里驻守了一年,第二年瓜熟时期已过,襄公却并没有派人替换他们。有人在襄公面前请求派人替换他们,襄公没有同意。因此这两个人很生气,便和公孙无知勾结叛乱。连称有一个堂妹在襄公的宫中不受宠爱,连称就让她暗中监视襄公的行动,说“事成之后,让你做公孙无知的夫人”。

冬季十二月,襄公去姑棼游玩,于是到沛丘打猎。他看见一只野猪,随从的人说道“彭生”。襄公很生气,用箭射它,那只野猪便像人一样站立并啼哭。襄公非常恐惧,从车上摔了下来,跌伤了脚,丢了鞋子。襄公返回后将伺候他穿鞋的人名叫茀的打了三百鞭。茀出宫离去。而无知、连称、管至父等人听说襄公受伤了,就率领部众攻打襄公的宫殿。正好遇到伺候襄公穿鞋的侍者茀,茀说道:“先不要进去惊动宫里的人,惊动了就不容易进去了。”

无知并不相信,茀让他看自己身上的鞭伤,无知这才相信。他们便在宫外等候,让茀先进去。茀先进了宫,将襄公藏在了门后面。过了很长时间,无知等人有些害怕,便闯入宫中。茀反而和宫中卫士还有襄公宠幸的大臣一起攻打无知等人,没有获胜,全部被杀死。无知闯入宫中,四处寻找襄公却没有找到。这时有人发现门下面露出的人脚,推开门一看,果然是襄公,就将他杀了,无知便自立为齐君。

\begin{yuanwen}
桓公元年春,齐君无知游于雍林。雍林人尝有怨无知,及其往游,雍林人袭杀无知,告齐大夫曰:“无知弑襄公自立,臣谨行诛。唯大夫更立公子之当立者,唯命是听。”
\end{yuanwen}

桓公元年(前685年)春季,齐君无知来到雍林游玩。雍林人曾对无知心怀怨恨,等到他来雍林游玩,便袭杀了他,随后对齐国的大夫们说:“无知杀害襄公自立,我等已经将其处死了。希望大夫们在公子中重新拥立应当继位的人,我们一定会听从他的命令。”

\begin{yuanwen}
初,襄公之醉杀鲁桓公,通其夫人,杀诛数不当,淫于妇人,数欺大臣,群弟恐祸及,故次弟纠奔鲁。其母鲁女也。管仲\footnote{名夷吾,字仲,春秋时期政治家、军事家。}、召忽傅之。次弟小白奔莒,鲍叔\footnote{鲍叔牙,齐国大夫。}傅之。小白母,卫女也,有宠于釐公。小白自少好善大夫高傒。及雍林人杀无知,议立君,高、国先阴召小白于莒。鲁闻无知死,亦发兵送公子纠,而使管仲别将兵遮莒道,射中小白带钩。小白详死,管仲使人驰报鲁。鲁送纠者行益迟,六日至齐,则小白已入,高傒立之,是为桓公。
\end{yuanwen}

当初,齐襄公灌醉并杀害了鲁桓公,和鲁桓公的夫人通奸,数次杀罚不当,沉溺于女色,又多次欺辱大臣,他的几个弟弟唯恐祸及自身,因此襄公的二弟公子纠逃奔鲁国。他的母亲是鲁国国君的女儿,管仲和召忽辅佐他。襄公的三弟公子小白逃奔莒国,鲍叔牙辅佐他。小白的母亲是卫国国君的女儿,深受齐釐公的宠爱。小白自幼就和大夫高傒友善。等到雍林人将公孙无知杀死以后,大臣们商议另立新君之时,高氏、国氏两家大族率先暗中召唤在莒国的公子小白回国继位。而鲁国听说无知死了,也派兵护送公子纠回国继位,同时又命管仲单独率领一队士兵堵在莒国回齐国的大路上,管仲一箭射中了小白腰部的衣带钩。小白趁机装死,管仲误以为小白已死,便派人迅速向鲁国报告了这个消息。于是鲁国送公子纠的队伍走得更慢了,走了六天才到达齐国,这时才发现公子小白早已回了齐国,被高傒拥立为君,就是桓公。

\begin{yuanwen}
桓公之中钩,详死以误管仲,已而载温车中驰行,亦有高、国内应,故得先入立,发兵距鲁。

秋,与鲁战于乾时,鲁兵败走,齐兵掩绝鲁归道。齐遗鲁书曰:“子纠兄弟,弗忍诛,请鲁自杀之。召忽、管仲雠也,请得而甘心醢之。不然,将围鲁。”

鲁人患之,遂杀子纠于笙渎。召忽自杀,管仲请囚。桓公之立,发兵攻鲁,心欲杀管仲。鲍叔牙曰:“臣幸得从君,君竟以立。君之尊,臣无以增君。君将治齐,即高傒与叔牙足也。君且欲霸王,非管夷吾不可。夷吾所居国国重,不可失也。”

于是桓公从之。乃详为召管仲欲甘心,实欲用之。管仲知之,故请往。鲍叔牙迎受管仲,及堂阜而脱桎梏,斋祓而见桓公。桓公厚礼以为大夫,任政。
\end{yuanwen}

当时桓公被射中带钩,装死骗过了管仲,随后便藏在辒车中疾驰回国,又有高氏、国氏作为内应,因此得以抢先一步回到齐都被立为君,随后他发兵抵御鲁军。

秋季,齐军和鲁军在乾时大战,鲁军大败而逃,齐军掩杀,截断鲁军归路。齐桓公给鲁庄公写信说道:“公子纠是我的兄弟,我不忍心杀他,请鲁国自行将他杀死。召忽和管仲是我的仇人,请把他们交给我,我要将他们剁成肉酱才解恨。否则我将包围鲁国。”

鲁国人非常害怕,就在笙渎杀掉公子纠。召忽自杀,管仲甘愿被囚禁起来。桓公刚即位便出兵进攻鲁国,就是想杀死管仲。鲍叔牙却劝他道:“我幸运地跟随您,您终于登上了国君之位。您的地位尊贵,我已经无法再让您提高。您将要治理齐国,那么有高傒和我两个人就可以了。如果您还想称霸天下,那就非得有管夷吾不可。夷吾所在的国家会强大,不可失去这样的人才啊。”

于是桓公听从了鲍叔牙的建议,所以他才故意说将管仲要回来杀掉,其实是想要任用他。管仲也知道这一点,所以才请求将自己遣送回国。鲍叔牙出迎接收了管仲,一到堂阜邑就将他的镣铐全都卸了下来,在他沐浴祭祀以后,又带他去见了桓公。桓公对其厚礼相待,任命他为大夫,主掌齐国的政务。

\begin{yuanwen}
桓公既得管仲,与鲍叔、隰朋\footnote{齐国大夫。}、高傒修齐国政,连五家之兵,(设)伸轻重鱼盐之利,以赡贫穷,禄贤能,齐人皆说。
\end{yuanwen}

桓公得到了管仲,和鲍叔牙、隰朋、高傒等人一起整顿齐国的政事,推行以五家为基层单位的兵役制度,确立铸造货币、发展捕鱼煮盐等业的税收制度,用得来的钱赈济贫困的人,优待贤能的人,齐国的百姓都很高兴。

\begin{yuanwen}
二年,伐灭郯,郯子奔莒。初,桓公亡时,过郯,郯无礼,故伐之。
\end{yuanwen}

桓公二年(前684年),齐国征讨并灭掉了郯国,郯国的国君逃奔莒国。早年桓公逃亡的时候,曾经到过郯国,郯国的国君对桓公无礼,因此讨伐它。

\begin{yuanwen}
五年,伐鲁,鲁将师败。鲁庄公请献遂邑以平,桓公许,与鲁会柯而盟。鲁将盟,曹沬以匕首劫桓公于坛上,曰:“反鲁之侵地!”

桓公许之。已而曹沬去匕首,北面就臣位。桓公后悔,欲无与鲁地而杀曹沬。管仲曰:“夫劫许之而倍信杀之,愈一小快耳,而弃信于诸侯,失天下之援,不可。”

于是遂与曹沬三败所亡地于鲁。诸侯闻之,皆信齐而欲附焉。七年,诸侯会桓公于甄\footnote{zhēn},而桓公于是始霸焉。
\end{yuanwen}

五年(前681年),齐国讨伐鲁国,鲁国的军队战败。鲁庄公请求献出遂邑讲和,桓公答应了,与鲁庄公在柯地会盟。当鲁庄公将要和齐桓公订立盟约时,曹沫手持匕首在祭坛上挟持了齐桓公,喝道:“归还侵占鲁国的土地!”

齐桓公只好答应。曹沫便放下匕首,退回到原来北面的臣子位置上。桓公又后悔了,想不归还鲁国的土地并杀死曹沫。管仲说道:“被劫持时已经答应退回,又反悔失信并杀人,贪求一时小的痛快,而失信于诸侯,就会失去天下的支持,这样做不行。”

于是就把曹沫三次战败所丢失的土地全都还给了鲁国。诸侯听说了这件事,都相信齐国而想要归附。七年(前679年),各国诸侯与桓公在甄地会盟,桓公从此开始称霸。

\begin{yuanwen}
十四年,陈厉公子完,号敬仲,来奔齐。齐桓公欲以为卿,让;于是以为工正。田成子常\footnote{田成子,名恒,汉时避文帝刘恒讳,称其为“常”。}之祖也。
\end{yuanwen}

十四年(前672年),陈厉公的儿子田完,号敬仲,前来投奔齐国。齐桓公打算让他做卿,但是他推辞了,于是让他担任主管百工的工正。田完就是田成子常的祖先。

\begin{yuanwen}
二十三年,山戎伐燕,燕告急于齐。齐桓公救燕,遂伐山戎,至于孤竹而还。燕庄公遂送桓公入齐境。桓公曰:“非天子,诸侯相送不出境,吾不可以无礼于燕。”

于是分沟割燕君所至与燕,命燕君复修召公之政,纳贡于周,如成康之时。诸侯闻之,皆从齐。
\end{yuanwen}

二十三年(前663年),山戎攻打燕国,燕国向齐国告急。齐桓公援救燕国,便出兵讨伐山戎,一直打到孤竹才班师回国。燕庄公于是将桓公一直送到齐国境内。桓公说道:“除了天子,诸侯送行不可走出国境,我不能对燕国失礼。”

于是就地挖沟为界,将燕君走到的地方都划给了燕国,他让燕君重修召公的德政,向周王室进贡,就像当年周成王、康王的时候一样。诸侯听说了这件事,都表示服从齐国。

\begin{yuanwen}
二十七年,鲁湣公母曰哀姜,桓公女弟也。哀姜淫于鲁公子庆父,庆父弑湣公,哀姜欲立庆父,鲁人更立釐公。桓公召哀姜,杀之。
\end{yuanwen}

二十七年(前659年),鲁湣公的母亲名为哀姜,是齐桓公的妹妹。哀姜和鲁国公子庆父私通,庆父杀死湣公,哀姜想让庆父继位,但是鲁国人却另立了釐公。桓公召回哀姜,将她杀掉。

\begin{yuanwen}
二十八年,卫文公\footnote{名毁,戴公之子。}有狄乱,告急于齐。齐率诸侯城楚丘而立卫君。
\end{yuanwen}

二十八年(前658年),卫文公遭到狄人的侵扰,向齐国告急。齐国率领诸侯在楚丘修筑城池,扶立卫君。

\begin{yuanwen}
二十九年,桓公与夫人蔡姬戏船中。蔡姬习水,荡公,公惧,止之,不止,出船,怒,归蔡姬,弗绝。蔡亦怒,嫁其女。桓公闻而怒,兴师往伐。
\end{yuanwen}

二十九年(前657年),桓公和夫人蔡姬乘船嬉戏。蔡姬很熟悉水性,所以故意摇晃游船,桓公很害怕,连忙制止她,但是蔡氏却不停止,桓公从船中出来,非常生气,将蔡姬送回了蔡国,但是没有断绝婚姻关系。蔡国的国君也很生气,就将女儿蔡姬嫁给了别人。桓公听说以后勃然大怒,出兵前往讨伐。

\begin{yuanwen}
三十年春,齐桓公率诸侯伐蔡,蔡溃。遂伐楚。楚成王兴师问曰:“何故涉吾地?”

管仲对曰:“昔召康公命我先君太公曰:“五侯九伯,若实征之,以夹辅周室。”赐我先君履,东至海,西至河,南至穆陵,北至无棣\footnote{dì}。楚贡包茅\footnote{成束的菁茅。包,束。茅,菁茅,楚国产的一种茅草,主要用于缩酒祭祀。}不入,王祭不具,是以来责。昭王南征不复,是以来问。”

楚王曰:“贡之不入,有之,寡人罪也,敢不共乎!昭王之出不复,君其问之水滨。”

齐师进次于陉。夏,楚王使屈完将兵扞齐,齐师退次召陵。桓公矜屈完以其众。屈完曰:“君以道则可;若不,则楚方城以为城,江、汉以为沟,君安能进乎?”乃与屈完盟而去。

过陈,陈袁涛涂诈齐,令出东方,觉。秋,齐伐陈。是岁,晋杀太子申生。
\end{yuanwen}

三十年(前656年)春季,齐桓公率领诸侯征讨蔡国,蔡国被击溃。于是桓公又向南征讨楚国。楚成王领兵迎敌,问道:“为什么侵入我国国土?”

管仲答道:“从前召康公曾授命我先君太公说:‘各等诸侯,各地官员,你都有权征讨,以此辅佐周王室。’赐给我们先君势力范围东到海滨,西到黄河,南到穆陵,北到无棣。楚国应该进贡的包茅没有进献,导致天子的祭祀不得完备,因此前来问责。从前周昭王南征没有回去,因此前来问罪。”

楚王说道:“包茅没有进贡,确有其事,这是我的罪过,我以后再也不敢不供应了!至于周昭王南巡而没有回去,您应当到汉水边去问罪。”

齐军继续行进到陉地驻扎。夏季,楚王派大将屈完率军抵抗齐军,齐军后退到召陵驻扎。桓公会见屈完,向他炫耀齐军的强大。屈完说道:“您以道德服人是可以的;否则,楚国就要以方城为城墙,以长江、汉江为护城河,您如何能攻进来呢?”桓公这才和屈完订立盟约,然后便领兵撤退了。

经过陈国时,陈国的大夫袁涛涂欺骗齐军,让齐军绕道从不经过陈国但是很难行的东路走,被齐桓公察觉。秋季,齐国征讨陈国。这一年,晋献公杀了太子申生。

\begin{yuanwen}
三十五年夏,会诸侯于葵丘。周襄王使宰孔赐桓公文武胙、彤弓矢、大路,命无拜。桓公欲许之,管仲曰“不可”,乃下拜受赐。

秋,复会诸侯于葵丘,益有骄色。周使宰孔会。诸侯颇有叛者。晋侯\footnote{指晋献公。}病,后,遇宰孔。宰孔曰:“齐侯骄矣,弟无行。”从之。

是岁,晋献公卒,里克杀奚齐、卓子,秦穆公以夫人入公子夷吾为晋君。桓公于是讨晋乱,至高梁,使隰朋立晋君,还。
\end{yuanwen}

三十五年(前651年)夏季,齐桓公在葵丘会盟诸侯。周襄王派宰孔赐给齐桓公祭过文王武王的祭肉、朱红色的弓箭、大车,命令桓公在接受赏赐时不必行跪拜大礼。桓公想要接受,管仲说“不可以”,于是齐桓公便按照礼数下拜接受天子的赏赐。

秋季,齐桓公又在葵丘与诸侯会盟,这时的他愈发带有骄傲的神色。周王室派出宰孔参加了盟会。诸侯中也渐渐有人反叛。晋侯生病所以出发迟了,在路上他遇到了宰孔。宰孔对他说道:“齐侯非常骄傲,你不要去了。”晋侯听从了他的话。

这一年,晋献公去世,大夫里克杀死傒齐、卓子,秦穆公因为他的夫人是晋公子夷吾的姐姐,将公子夷吾送回晋国,拥立为国君。桓公于是征讨晋国的叛乱,兵至高梁,派隰朋立夷吾为晋君后返回。

\begin{yuanwen}
是时周室微,唯齐、楚、秦、晋为彊。晋初与会,献公死,国内乱。秦穆公辟远,不与中国会盟。楚成王初收荆蛮有之,夷狄自置。唯独齐为中国会盟,而桓公能宣其德,故诸侯宾会。于是桓公称曰:“寡人南伐至召陵,望熊山;北伐山戎、离枝、孤竹;西伐大夏,涉流沙;束马悬车登太行,至卑耳山而还。诸侯莫违寡人。寡人兵车之会三,乘车之会六,九合诸侯,一匡天下。昔三代受命,有何以异于此乎?吾欲封泰山,禅梁父。”

管仲固谏,不听;乃说桓公以远方珍怪物至乃得封,桓公乃止。
\end{yuanwen}

这时周王室衰微,只有齐国、楚国、秦国、晋国比较强大。晋国晋献公刚参加完盟会不久就去世了,国内政局混乱。秦穆公地处偏远,不参加中原各国的会盟。楚成王刚刚占据了荆蛮一带,以夷狄之邦自居。只有齐国召集主持中原各国的会盟,而桓公也确实能够宣扬德行,所以其他诸侯也都服从于齐国,来参加会盟。于是齐桓公宣称:“我向南征讨至召陵,瞭望熊山;向北讨伐山戎、离枝、孤竹;向西征伐大夏,远涉流沙;裹了马脚,挂牢战车,登上太行山,直至卑耳山才返回。诸侯没有违背我的。我召集军事盟会有三次,主持和平盟会有六次,会合诸侯九次,安定周王室一次。从前夏、商、周三代承受天命之时,和我现在又有什么区别?我想去泰山祭天,到梁父山祭地。”

管仲坚持劝阻,桓公并不听从;管仲又劝桓公说等得到了远方的奇珍异宝后才可以到泰山封禅祭祀,桓公这才作罢。

\begin{yuanwen}
三十八年,周襄王弟带与戎、翟合谋伐周,齐使管仲平戎于周。周欲以上卿礼管仲,管仲顿首曰:“臣陪臣,安敢!”

三让,乃受下卿礼以见。

三十九年,周襄王弟带来奔齐。齐使仲孙请王,为带谢。襄王怒,弗听。
\end{yuanwen}

三十八年(前648年),周襄王的弟弟带勾结戎人、翟人一起进攻周王室,齐国派管仲为周王室和戎人讲和。周襄王想以上卿的礼遇对待管仲,管仲叩头说:“我只是诸侯的臣子,怎么敢如此!”

他再三谦让,才接受了以下卿的礼仪朝见周襄王。

三十九年(前645年),周襄王的弟弟带前来投奔齐国。齐桓公派仲孙去向周王求情,为带请罪。襄王十分生气,拒绝答应。

\begin{yuanwen}
四十一年,秦穆公虏晋惠公,复归之。是岁,管仲、隰朋皆卒。管仲病,桓公问曰:“群臣谁可相者?”

管仲曰:“知臣莫如君。”

公曰:“易牙如何?”

对曰:“杀子以適君,非人情,不可。”

公曰:“开方如何?”

对曰:“倍亲以適君,非人情,难近。”

公曰:“竖刁如何?”

对曰:“自宫以適君,非人情,难亲。”

管仲死,而桓公不用管仲言,卒近用三子,三子专权。
\end{yuanwen}

四十一年(前645年),秦穆公俘虏了晋惠公,不久又放他回国。这一年,管仲、隰朋都去世了。管仲病重之时,桓公问他道:“众臣中有谁可以为相?”

管仲说道:“最了解臣下的莫过于君主。”

桓公说:“易牙怎么样?”

管仲答道:“他杀掉自己的儿子迎合国君,不近人情,不可亲信。”

桓公又问:“开方怎么样?”

管仲答道:“他背叛自己的亲人以迎合君主,不近人情,难以亲信。”

桓公说:“竖刁怎么样?”

管仲答道:“他自行阉割以迎合君主,不近人情,难以亲信。”

管仲死后,桓公却没有听从管仲的劝告,最终还是宠信了这三个人,这三个人专揽齐国的大权。

\begin{yuanwen}
四十二年,戎伐周,周告急于齐,齐令诸侯各发卒戍周。是岁,晋公子重耳来,桓公妻之。
\end{yuanwen}

四十二年(前644年),戎人进攻周王室,周王派人向齐国告急。齐国命令诸侯各自出兵拱卫周王室。这一年,晋国公子重耳流亡到了齐国,齐桓公将齐国宗室的女儿嫁给他为妻。

\begin{yuanwen}
四十三年。初,齐桓公之夫人三:曰王姬、徐姬、蔡姬,皆无子。桓公好内,多内宠,如夫人者六人,长卫姬,生无诡;少卫姬,生惠公元;郑姬,生孝公昭;葛嬴,生昭公潘;密姬,生懿公商人;宋华子,生公子雍。桓公与管仲属\footnote{通“嘱”,托付。}孝公于宋襄公,以为太子。雍巫有宠于卫共姬,因宦者竖刀以厚献于桓公,亦有宠,桓公许之立无诡。管仲卒,五公子皆求立。

冬十月乙亥,齐桓公卒。易牙入,与竖刀因内宠杀群吏,而立公子无诡为君。太子昭奔宋。
\end{yuanwen}

四十三年(前643年),齐桓公原先有三位夫人,分别是王姬、徐姬、蔡姬,她们都没有生下儿子。桓公好女色,有很多宠爱的姬妾,侧室有六位,分别是长卫姬,她生了无诡;少卫姬,她生了惠公元;郑姬,她生了孝公昭;葛嬴,她生了昭公潘;密姬,她生了懿公商人;宋华子,她生了公子雍。桓公和管仲将孝公托付给宋襄公,立他为太子。雍巫受到卫共姬的宠幸,又通过宦官竖刁向桓公奉上厚礼,也受到桓公的宠幸,桓公许诺立无诡为太子。管仲去世以后,五位公子都谋求被立为太子。

这年冬季乙亥日,齐桓公去世。易牙入宫,和竖刁一起勾结宫中的宠臣,杀害了很多反对他们的大臣,拥立公子无诡为君。太子昭逃奔宋国。

\begin{yuanwen}
桓公病,五公子各树党争立。及桓公卒,遂相攻,以故宫中空,莫敢棺。桓公尸在床上六十七日,尸蟲出于户。

十二月乙亥,无诡立,乃棺赴。辛巳夜,敛殡。
\end{yuanwen}

齐桓公病重的时候,五位公子各自拉帮结派,争相成为太子。等到桓公去世,便相互攻击,所以宫中空空荡荡,没有人敢为齐桓公装殓入棺。桓公的尸体停在床上一直有六十七天,尸体上生的蛆虫都爬到了门外。

十二月乙亥日,无诡继位,这才将棺材运到宫中,发出讣告。辛巳日夜晚,才举行了入殓和出殡的丧礼。

\begin{yuanwen}
桓公十有馀子,要\footnote{总计。}其后立者五人:无诡立三月死,无谥;次孝公;次昭公;次懿公;次惠公。

孝公元年三月,宋襄公率诸侯兵送齐太子昭而伐齐。齐人恐,杀其君无诡。齐人将立太子昭,四公子之徒攻太子,太子走宋,宋遂与齐人四公子战。

五月,宋败齐四公子师而立太子昭,是为齐孝公。宋以桓公与管仲属之太子,故来征之。以乱故,八月乃葬齐桓公。
\end{yuanwen}

桓公有十多个儿子,总计有五人做过国君:公子无诡继位后,仅过了三个月就死了,没有谥号;接下来是孝公;再接下来是昭公;随后是懿公;最后是惠公。

孝公元年(前642年)三月,宋襄公率领诸侯护送齐国太子昭回国并攻打齐国。齐国人非常害怕,杀了他们的国君无诡。齐国人将要拥立太子昭为君,其余四位公子的党羽攻击太子,太子逃走去了宋国,宋军于是和齐国四公子的军队交战。

五月,宋军打败了齐国四公子的军队,拥立太子昭为君,就是齐孝公。宋君是因为桓公和管仲曾经将太子托付给他,所以前来讨伐作乱的四公子。因为政局混乱,所以一直到八月才将齐桓公安葬。

\begin{yuanwen}
六年春,齐伐宋,以其不同盟于齐也。

夏,宋襄公卒。

七年,晋文公立。
\end{yuanwen}

六年(前637年)春季,齐国讨伐宋国,因为宋襄公拒不参加在齐国举行的诸侯会盟。

这年夏季,宋襄公去世。

七年(前636年),晋文公即位。

\begin{yuanwen}
十年,孝公卒,孝公弟潘因卫公子开方杀孝公子而立潘,是为昭公。昭公,桓公子也,其母曰葛嬴。
\end{yuanwen}

十年(前633年),齐孝公去世,孝公的弟弟潘通过卫公子开方杀死孝公的儿子后自立为国君,就是昭公。昭公是齐桓公之子,他的母亲叫葛嬴。

\begin{yuanwen}
昭公元年,晋文公败楚于城濮,而会诸侯践土,朝周,天子使晋称伯\footnote{通“霸”。}。

六年,翟侵齐。晋文公卒。秦兵败于殽。

十二年,秦穆公卒。
\end{yuanwen}

昭公元年(前632年),晋文公在城濮打败了楚国,又在践土和各国诸侯会盟,朝见周天子,周天子让晋文公做了诸侯的霸主。

六年(前627年),翟人入侵齐国。晋文公去世。秦军在崤大败。

十二年(前621年),秦穆公去世。

\begin{yuanwen}
十九年五月,昭公卒,子舍立为齐君。舍之母无宠于昭公,国人莫畏。昭公之弟商人以桓公死争立而不得,阴交贤士,附爱百姓,百姓说。及昭公卒,子舍立,孤弱,即与众十月即墓上弑齐君舍,而商人自立,是为懿公。懿公,桓公子也,其母曰密姬。
\end{yuanwen}

十九年(前614年)五月,昭公去世,他的儿子舍继位。舍的母亲不受昭公宠爱,所以齐国人都不怕他。昭公的弟弟商人当年在桓公去世时争夺君位失败,暗中结交贤能的人士,关爱百姓,所以百姓都很喜欢他。等到昭公去世,他的儿子舍继位,势单力薄,商人便在十月的时候率领部众趁齐君舍为昭公扫墓的机会将其杀害,商人自立为国君,就是懿公。懿公是桓公的儿子,他的母亲叫密姬。

\begin{yuanwen}
懿公四年春,初,懿公为公子时,与丙戎之父猎,争获不胜,及即位,断丙戎父足,而使丙戎仆。庸职之妻好,公内之宫,使庸职骖乘。

五月,懿公游于申池,二人浴,戏。职曰:“断足子!”戎曰:“夺妻者!”二人俱病此言,乃怨。谋与公游竹中,二人弑懿公车上,弃竹中而亡去。
\end{yuanwen}

懿公四年(前609年)春季,懿公起初做公子时,有一次和丙戎的父亲一起打猎,争夺猎物时没有得胜。等到他继位后,便下令砍断了丙戎父亲的脚,又让丙戎做自己的仆人。庸职的妻子很漂亮,懿公就将她纳入宫中,还让庸职做自己的陪乘。

五月,懿公去申池游玩,丙戎和庸职两人一起洗澡时互相开玩笑。庸职喊道:“断脚人的儿子!”丙戎回应:“妻子被夺走的丈夫!”两人都因对方的话感到耻辱,于是都怨恨懿公。于是他们便怂恿懿公到竹林中去游玩,趁机在车上杀死懿公,将尸体丢在竹林中逃走。

\begin{yuanwen}
懿公之立,骄,民不附。齐人废其子而迎公子元于卫,立之,是为惠公。惠公,桓公子也。其母卫女,曰少卫姬,避齐乱,故在卫。

惠公二年,长翟来,王子城父攻杀之,埋之于北门。晋赵穿弑其君灵公。

十年,惠公卒,子顷公无野立。初,崔杼\footnote{又称崔子、崔武子,齐国大夫。}有宠于惠公,惠公卒,高、国畏其(逼/偪)也,逐之,崔杼奔卫。

顷公元年,楚庄王彊,伐陈;二年,围郑,郑伯\footnote{指郑襄公。}降,已复国郑伯。
\end{yuanwen}

懿公即位后,为人骄横,所以百姓都不归附他。齐国人废黜了他的儿子,从卫国迎回公子元,拥立为国君,就是惠公。惠公是桓公的儿子,他的母亲是卫国的女子,叫少卫姬,他为了躲避齐国的内乱,一直住在卫国。

惠公二年(前607年),长翟人进犯,大夫王子城父杀了他们的头领荣如,将其埋在了北门附近。这一年,晋国的赵穿杀死他的国君晋灵公。

十年(前599年),惠公去世,他的儿子顷公无野继位。起初,崔杼受到惠公的宠爱,惠公去世以后,高氏、国氏担心被他胁迫,就赶走了他,崔杼逃奔卫国。

顷公元年(前598年),楚庄王强大起来,讨伐陈国;二年(前597年),楚国包围了郑国,郑伯投降,不久又让郑伯复国。

\begin{yuanwen}
六年春,晋使郤克于齐,齐使夫人帷中而观之。郤克上,夫人笑之。郤克曰:“不是报,不复涉河!”

归,请伐齐,晋侯弗许。齐使至晋,郤克执齐使者四人河内,杀之。

八年。晋伐齐,齐以公子彊质晋,晋兵去。

十年春,齐伐鲁、卫。鲁、卫大夫如晋请师,皆因郤克。晋使郤克以车八百乘为中军将,士燮将上军,栾书将下军,以救鲁、卫,伐齐。

六月壬申,与齐侯兵合靡笄下。癸酉,陈于(鞌/鞍)。逄丑父为齐顷公右。顷公曰:“驰之,破晋军会食。”

射伤郤克,流血至履。克欲还入壁\footnote{营垒。},其御曰:“我始入,再伤,不敢言疾,恐惧士卒,原子忍之。”

遂复战。战,齐急,丑父恐齐侯得,乃易处,顷公为右,车絓\footnote{通“挂”,绊住。}于木而止。晋小将韩厥伏齐侯车前,曰“寡君使臣救鲁、卫”,戏之。

丑父使顷公下取饮,因得亡,脱去,入其军。晋郤克欲杀丑父。丑父曰:“代君死而见僇,后人臣无忠其君者矣。”

克舍之,丑父遂得亡归齐。于是晋军追齐至马陵。齐侯请以宝器谢,不听;必得笑克者萧桐叔子,令齐东亩。对曰:“叔子,齐君母。齐君母亦犹晋君母,子安置之?且子以义伐而以暴为后,其可乎?”

于是乃许,令反鲁、卫之侵地。
\end{yuanwen}

六年(前593年)春季,晋国派郤克出使齐国,齐侯让母亲在帷帐中偷看。郤克上殿,母夫人看到他是个驼子,便哈哈大笑。郤克说道:“不报这个仇,我誓不再过黄河!”

郤克回国后,请求讨伐齐国,晋君没有同意。齐国的使者到了晋国,卻克在河内抓住齐国使者中的四个人,杀死了他们。

八年(前591年),晋国讨伐齐国,齐国将公子强送到晋国做了人质,晋军撤兵。

十年(前589年)春季,齐国讨伐鲁国、卫国。鲁国、卫国的大夫到晋国求援,都是通过郤克。晋国派郤克为中军主将,率领战车八百辆,命士燮统率上军,栾书统率下军,前去援救鲁国和卫国,攻打齐国。

六月壬申日,晋军与齐军在靡笄山下相遇。癸酉日,两军在鞌地列开阵势。逢丑父为齐顷公担任右护卫。顷公说道:“迅速向前冲,打败晋军后会餐。”

齐军射伤了郤克,血都流到了他的鞋上。郤克想撤退回到营垒中,为他驾车的人说:“我刚进入阵地,已经负伤两处了,并不敢说自己受伤,因为怕士兵们恐慌。希望您能忍耐一下。”

于是郤克又投入了战斗。交战当中齐军形势危急,逄丑父担心齐侯被晋军俘虏,所以就和齐侯换了位置,顷公站在了右边,战车被树木绊住停下。这时晋国的一员小将韩厥匍匐在齐侯车子的前面,对齐侯说“敝国国君派我来救鲁国、卫国”,他这是在讥讽齐侯。

逄丑父让顷公下车取水喝,顷公得以逃走脱身,回到齐军阵中。晋国的郤克想杀掉逄丑父。逄丑父说道:“我替国君去死却被杀戮,从此以后再也不会有忠于国君的臣子了。”

郤克于是放了他,逄丑父这才得以逃回齐国军中。于是晋军追赶齐军到马陵。齐侯请求献上宝器谢罪,晋军不答应,一定要得到嘲笑郤克的萧桐叔子,还要求齐国将田垄和干道都改成东西向的。齐人回答道:“叔子是齐国国君的母亲。齐君的母亲就如晋君的母亲一样,您得到她打算如何安置?况且您以正义的名义征伐我国,最后却施以暴行,难道可以这样做吗?”

晋国这才同意了齐人的要求,让齐国归还侵占的鲁国、卫国的土地。

\begin{yuanwen}
十一年,晋初置六卿,赏(鞍/鞌)之功。齐顷公朝晋,欲尊王晋景公,晋景公不敢受,乃归。归而顷公弛苑囿,薄赋敛,振孤问疾,虚积聚以救民,民亦大说。厚礼诸侯。竟顷公卒,百姓附,诸侯不犯。
\end{yuanwen}

十一年(前588年),晋国开始设置六卿,奖赏鞌地一战中立功的大臣。齐顷公朝见晋君,想用见王者的礼节晋见晋景公,晋景公不敢接受,于是齐顷公就回国了。回国后,顷公就开放园林,降低赋税,赈济孤寡,慰问病者,拿出国家的积蓄救济百姓,百姓们都非常高兴。他又以厚礼对待诸侯。一直到顷公去世,百姓都很亲附,诸侯也都不来进犯。

\begin{yuanwen}
十七年,顷公卒,子灵公环立。
\end{yuanwen}

十七年(前582年),齐顷公去世,他的儿子灵公环继位。

\begin{yuanwen}
灵公九年,晋栾书弑其君厉公。

十年,晋悼公伐齐,齐令公子光质晋。

十九年,立子光为太子,高厚傅之,令会诸侯盟于锺离。

二十七年,晋使中行献子伐齐。齐师败,灵公走入临菑。晏婴止灵公,灵公弗从。曰:“君亦无勇矣!”

晋兵遂围临菑,临菑城守不敢出,晋焚郭中而去。
\end{yuanwen}

灵公九年(前573年),晋国的栾书杀了他的国君晋厉公。

十年(前572年),晋悼公讨伐齐国,齐侯让公子光到晋国当人质。

十九年(前563年),齐侯立公子光为太子,高厚辅佐他,让他到钟离会盟各国诸侯。

二十七年(前555年),晋国派中行献子攻打齐国。齐军战败,灵公逃到了临淄。晏婴曾劝阻灵公不可逃走,灵公并未听从。晏婴说道:“您也太没有勇气了!”

晋军于是包围了临淄,临淄城的军民坚守城池不敢出战,晋军焚毁外城后撤兵而去。

\begin{yuanwen}
二十八年,初,灵公取鲁女,生子光,以为太子。仲姬,戎姬。戎姬嬖,仲姬生子牙,属之戎姬。戎姬请以为太子,公许之。仲姬曰:“不可。光之立,列于诸侯矣,今无故废之,君必悔之。”

公曰:“在我耳。”

遂东太子光,使高厚傅牙为太子。灵公疾,崔杼迎故太子光而立之,是为庄公。庄公杀戎姬。

五月壬辰,灵公卒,庄公即位,执太子牙于句窦之丘,杀之。

八月,崔杼杀高厚。晋闻齐乱,伐齐,至高唐。
\end{yuanwen}

二十八年(前554年),起初,灵公娶了鲁国的女子,生下了公子光,又立他为太子。灵公身边还有仲姬和戎姬。戎姬很得灵公宠爱,仲姬生下了公子牙,并将他托付给戎姬。戎姬向灵公请求将公子牙立为太子,灵公答应了。仲姬却说:“不行。公子光已经被立为太子,他还参加了诸侯会盟,如今无故废掉他,将来您一定会后悔。”

灵公说:“我做决定。”

于是将太子光流放到齐国东部,让高厚辅佐公子牙,立他为太子。灵公病重,崔杼迎回原来的太子光,拥立他为国君,就是庄公。庄公杀了戎姬。

五月壬辰日,灵公去世,庄公继位,在句窦丘抓获了太子牙,并杀了他。

八月,崔杼杀了高厚。晋国听说了齐国的内乱,便趁机出兵征讨,军队一直行进到高唐。

\begin{yuanwen}
庄公三年,晋大夫栾盈奔齐,庄公厚客待之。晏婴、田文子谏,公弗听。

四年,齐庄公使栾盈间入晋曲沃为内应,以兵随之,上太行,入孟门。栾盈败,齐兵还,取朝歌。
\end{yuanwen}

庄公三年(前551年),晋国大夫栾盈逃到了齐国,庄公以隆重的客礼接待了他。晏婴、田文子极力劝阻,庄公并未听从。

四年(前550年),齐庄公让栾盈潜入晋国曲沃作为内应,又派军队跟随,登上太行山,进入孟门关。栾盈事败,齐军撤还,夺取了晋国的朝歌。

\begin{yuanwen}
六年,初,棠公妻好,棠公死,崔杼取之。庄公通之,数如崔氏,以崔杼之冠赐人。待者曰:“不可。”

崔杼怒,因其伐晋,欲与晋合谋袭齐而不得间。庄公尝笞宦者贾举,贾举复侍,为崔杼间公以报怨。

五月,莒子\footnote{莒国的国君。}朝齐,齐以甲戌飨之。崔杼称病不视事。

乙亥,公问崔杼病,遂从崔杼妻。崔杼妻入室,与崔杼自闭户不出,公拥柱而歌。宦者贾举遮公从官而入,闭门,崔杼之徒持兵从中起。公登台而请解,不许;请盟,不许;请自杀于庙,不许。皆曰:“君之臣杼疾病,不能听命。近于公宫。陪臣争趣\footnote{通“趋”。}有淫者,不知二命。”

公逾墙,射中公股,公反坠,遂弑之。晏婴立崔杼门外,曰:“君为社稷死则死之,为社稷亡则亡之。若为己死己亡,非其私暱\footnote{nì},谁敢任之!”

门开而入,枕公尸而哭,三踊而出。人谓崔杼:“必杀之。”

崔杼曰:“民之望也,舍之得民。”
\end{yuanwen}

六年(前548年),起初,棠公的妻子很漂亮,棠公去世以后,崔杼就娶她为妻。庄公跟她私通,多次去崔家,还将崔杼的帽子赐给别人。侍者说:“不可以这样。”

崔杼十分愤怒,想趁着庄公攻打晋国的机会,与晋国合谋袭击齐国,但是一直没有机会。庄公曾经鞭打过宦官贾举,贾举还在服侍他,替崔杼暗中窥探庄公的行动,从而进行报复。

五月,莒子朝见齐君,齐庄公在甲戌日设宴款待他。崔杼以患病为由不理政事。

乙亥日,庄公来探望崔杼的病情,便追逐崔杼的妻子。崔杼的妻子走进内室,和崔杼一起闭门不出,庄公便靠着庭柱唱起歌来。宦官贾举拦住庄公的随从,自己走了进来,关闭大门,崔杼的党羽手持兵器从里面冲了出来。庄公登上高台请求和解,被拒绝;请求盟誓订约,也被拒绝;请求到祖庙中自杀,仍然被拒绝。他们都说:“您的臣子崔杼病重,无法亲自来听候您的命令。这里离公宫很近。我们这些陪臣只知道奋勇捉拿淫色之徒,不会听从崔杼以外其他人的命令。”

庄公翻墙逃跑,追赶的人用弓箭射中了庄公的大腿,庄公翻身跌落,众人于是杀了他。晏婴站在崔杼家大门外,说道:“君主为国家而死,臣子便随他而死;为国家而逃亡,臣子便随他一起逃亡。如果君主为了自己而死或者逃亡,那么除了他的亲信,其他人何必陪着呢!”

大门打开,晏婴走了进去,头枕庄公尸体放声痛哭,向上跳三次以示哀痛,然后便走了出去。有人对崔杼说:“一定要杀了他。”

崔杼说:“他是众望所归的人,放了他可以赢得民心。”

\begin{yuanwen}
丁丑,崔杼立庄公异母弟杵臼,是为景公。景公母,鲁叔孙宣伯女也。景公立,以崔杼为右相,庆封为左相。二相恐乱起,乃与国人盟曰:“不与崔庆者死!”

晏子仰天曰:“婴所不(获)唯忠于君利社稷者是从!”

不肯盟。庆封欲杀晏子,崔杼曰:“忠臣也,舍之。”

齐太史书曰:“崔杼弑庄公。”

崔杼杀之。其弟复书,崔杼复杀之。少弟复书,崔杼乃舍之。
\end{yuanwen}

丁丑日,崔杼拥立庄公的异母弟杵臼为国君,就是景公。景公的母亲是鲁国叔孙宣伯的女儿。景公继位后,任命崔杼为右相,庆封为左相。这两位宰相担心国内发生动乱,就和国都的人盟誓说:“不和崔杼、庆封合作的人处死!”

晏子仰面向天说道:“我晏婴只依从那些忠于君主、利于国家的人!”

他便不肯盟誓。庆封想杀死晏子,崔杼说道:“他是忠臣啊,放了他吧。”

齐国太史写道:“崔杼杀死庄公。”

崔杼将他杀掉。太史的弟弟还是这样写,崔杼又将他的弟弟杀掉。太史的小弟弟还是这样写,崔杼才放过他。

\begin{yuanwen}
景公元年,初,崔杼生子成及彊,其母死,取东郭女,生明。东郭女使其前夫子无咎与其弟偃相崔氏。成有罪,二相急治之,立明为太子。成请老于崔,崔杼许之,二相弗听,曰:“崔,宗邑,不可。”

成、彊怒,告庆封。庆封与崔杼有郤,欲其败也。成、彊杀无咎、偃于崔杼家,家皆奔亡。崔杼怒,无人,使一宦者御,见庆封。庆封曰:“请为子诛之。”

使崔杼仇卢蒲嫳攻崔氏,杀成、彊,尽灭崔氏,崔杼妇自杀。崔杼毋归,亦自杀。庆封为相国,专权。
\end{yuanwen}

景公元年(前547年),起初,崔杼生了儿子崔成和崔彊,他们的母亲去世以后,崔杼娶了东郭家的女儿,又生下了崔明。东郭家之女让她和前夫的儿子棠无咎和她的弟弟东郭偃辅佐崔杼。崔成犯了罪,无咎和东郭偃对他严加惩治,又立崔明为太子。崔成请求让他终老于崔邑,崔杼同意了,但是那两位辅佐的人却不答应,他们说道:“崔邑,是宗庙所在之地,此事不可。”

崔成、崔彊十分愤怒,将这件事告诉了庆封。庆封与崔杼本来就有矛盾,希望崔家衰败。崔成、崔彊在崔杼家杀了无咎和东郭偃,家中的人四散奔逃。崔杼大怒,但是身边却没有人,便派一个宦官驾车去见庆封。庆封说道:“请让我为你杀了他们。”

于是庆封便派崔杼的仇人卢蒲嫳攻打崔家,杀死崔成和崔彊,又将崔家满门杀光,崔杼的妻子自杀。崔杼无家可归,也自杀了。庆封成为相国,独揽齐国大权。

\begin{yuanwen}
三年十月,庆封出猎。初,庆封已杀崔杼,益骄,嗜酒好猎,不听政令。庆舍\footnote{庆封之子。}用政,已有内郤。田文子谓桓子曰:“乱将作。”

田、鲍、高、栾氏相与谋庆氏。庆舍发甲围庆封宫,四家徒共击破之。庆封还,不得入,奔鲁。齐人让鲁,封奔吴。吴与之(硃/朱)方,聚其族而居之,富于在齐。其秋,齐人徙葬庄公,僇崔杼尸于市以说众。
\end{yuanwen}

三年(前545年)十月,庆封外出打猎。起初,庆封杀死崔杼后变得更加骄横,喜欢喝酒,爱好打猎,不理政务。他的儿子庆舍掌握政权,不久他们父子之间便产生了矛盾。田文子对桓子说道:“内乱将要发生。”

田氏、鲍氏、高氏、栾氏一起谋划对付庆氏。庆舍派甲兵包围了庆封的官邸,那四家的部属合力攻破了庆封的家。庆封回来以后进不了家,于是逃奔鲁国。齐国人谴责鲁国,庆封又逃奔吴国。吴国将他封在朱方,聚集他的族人住在那里,比在齐国的时候还富有。这一年秋季,齐国人迁葬庄公,在街头将崔杼戮尸,以博取民众的欢心。

\begin{yuanwen}
九年,景公使晏婴之晋,与叔向私语曰:“齐政卒归田氏。田氏虽无大德,以公权私,有德于民,民爱之。”

十二年,景公如晋,见平公,欲与伐燕。

十八年,公复如晋,见昭公。

三十一年,鲁昭公辟季氏难,奔齐。齐欲以千社封之,子家\footnote{姬姓,东门氏,名归父,字子家,鲁庄公玄孙,又称公孙归父。}止昭公,昭公乃请齐伐鲁,取郓以居昭公。
\end{yuanwen}

九年(前539年),景公派晏婴前往晋国,晏婴私下对叔向说:“齐国的政权最后将会归于田氏。田氏虽然并没有什么大的功德,但是借公权行私惠,对百姓有恩,百姓拥戴他们。”

十二年(前536年),景公前往晋国,会见晋平公,想和晋国一起讨伐燕国。

十八年(前530年),景公再次前往晋国,会见了晋昭公。

二十六年(前522年),景公在鲁国都城的郊外打猎,并顺路去了鲁都,和晏婴一起询问鲁国的礼制。

三十一年(前517年),鲁昭公为躲避季氏的迫害,逃到了齐国。齐景公想将二万五千民户封给他,子家劝鲁昭公推辞,昭公就请求齐国出兵讨伐鲁国,攻取了郓邑给鲁昭公居住。

\begin{yuanwen}
三十二年,彗星见。景公坐柏寝,叹曰:“堂堂!谁有此乎?”

群臣皆泣,晏子笑,公怒。晏子曰:“臣笑群臣谀甚。”

景公曰:“彗星出东北,当齐分野,寡人以为忧。”

晏子曰:“君高台深池,赋敛如弗得,刑罚恐弗胜,茀星将出,彗星何惧乎?”

公曰:“可禳否?”

晏子曰:“使神可祝而来,亦可禳而去也。百姓苦怨以万数,而君令一人禳之,安能胜众口乎?”

是时景公好治宫室,聚狗马,奢侈,厚赋重刑,故晏子以此谏之。
\end{yuanwen}

三十二年(前516年),彗星出现。景公坐在柏寝台上,叹息道:“多么富丽堂皇!它会归谁所有呢?”

大臣们都哭了,只有晏婴笑了,景公大怒。晏婴说:“我嘲笑的是大臣们过于阿谀奉承。”

景公说:“彗星出现在东北方向,正对应齐国的分野,寡人为此感到忧虑。”

晏婴说:“国君修筑高台深池,赋税唯恐收不到手,刑罚唯恐不够严重,连妖星都将出现,彗星又有什么可怕的呢?”

景公说:“可以通过祈祷来消除灾害吗?”

晏婴说:“如果通过祈祷可以将神灵招来,也就可以通过祈祷让神离开。百姓愁苦怨恨的数以万计,而您让一个人去祈祷消灾,怎么能胜过众人的诅咒呢?”

当时景公正喜欢修建宫室,聚养犬马,生活奢侈,重赋税,苛刑罚,因此晏婴用这些话来劝谏他。

\begin{yuanwen}
四十二年,吴王阖闾伐楚,入郢。
\end{yuanwen}

四十二年(前506年),吴王阖闾讨伐楚国,攻入楚国国都郢。

\begin{yuanwen}
四十七年,鲁阳虎\footnote{字货,鲁国季孙氏的家臣。}攻其君,不胜,奔齐,请齐伐鲁。鲍子谏景公,乃囚阳虎。阳虎得亡,奔晋。
\end{yuanwen}

四十七年(前501年),鲁国的阳虎攻打他的国君,没能取胜,逃奔齐国,请求齐国攻打鲁国。鲍子劝谏景公不可,于是囚禁了阳虎。阳虎找到机会逃走,投奔了晋国。

\begin{yuanwen}
四十八年,与鲁定公好会夹谷。犁鉏曰:“孔丘知礼而怯,请令莱人为乐,因执鲁君,可得志。”

景公害孔丘相鲁,惧其霸,故从犁鉏之计。方会,进莱乐,孔子历阶上,使有司执莱人斩之,以礼让景公。景公惭,乃归鲁侵地以谢,而罢去。是岁,晏婴卒。
\end{yuanwen}

四十八年(前500年),齐景公和鲁定公在夹谷进行了一次和平友好的会晤。齐臣犂鉏说:“孔丘懂得礼制却胆小,请让莱人奏乐,我们趁这个机会抓住鲁君,便可以达到我们的目的了。”

景公忌惮孔丘辅佐鲁君,担心鲁国会强大并称霸,所以就听从了犂鉏的计谋。齐鲁两位国君正在会晤之时,齐人进献上莱夷音乐,孔子这时一步一个台阶登上了坛台,命令官员抓住莱人并斩首,随后又引用礼制谴责景公。景公很惭愧,便归还所侵占的鲁国土地以谢罪,然后就离开了。这一年,晏婴去世。

\begin{yuanwen}
五十五年,范、中行反其君于晋,晋攻之急,来请粟。田乞\footnote{妫姓,田氏,名乞,田桓子田无宇之子,齐国权臣。}欲为乱,树党于逆臣,说景公曰:“范、中行数有德于齐,不可不救。”及使乞救而输之粟。
\end{yuanwen}

五十五年(前493年),范氏、中行氏在晋国发动叛乱,晋国国君猛烈地攻打他们,他们就派人到齐国来要求借粮。田乞想要作乱,想结交叛臣以壮大自己的党羽,就劝景公说:“范氏、中行氏多次对齐国有恩,不可不救。”景公便派田乞前往援救,还送去了粮食。

\begin{yuanwen}
五十八年夏,景公夫人燕姬適子死。景公宠妾芮姬生子荼,荼少,其母贱,无行,诸大夫恐其为嗣,乃言愿择诸子长贤者为太子。景公老,恶言嗣事,又爱荼母,欲立之,惮发之口,乃谓诸大夫曰:“为乐耳,国何患无君乎?”

秋,景公病,命国惠子、高昭子立少子荼为太子,逐群公子,迁之莱。景公卒,太子荼立,是为晏孺子。

冬,未葬,而群公子畏诛,皆出亡。荼诸异母兄公子寿、驹、黔奔卫,公子驵、阳生奔鲁。莱人歌之曰:“景公死乎弗与埋,三军事乎弗与谋,师乎师乎,胡党之乎?”
\end{yuanwen}

五十八年(前490年)夏季,景公夫人燕姬所生的嫡子死了。景公的爱妾芮姬生有儿子荼,荼年幼,他的母亲出身卑微,品行还不好,大夫们担心他成为嗣子,就向景公进言,希望从诸子中选择一位年长而又贤德的人立为太子。景公年事已高,厌恶谈论立嗣的事,他又宠爱荼的母亲,就想立荼为太子,但是又难于启齿,就对大臣们说:“还是好好享乐吧,国家还怕没有君主吗?”

秋季,景公病重,命令国惠子、高昭子立小儿子荼为太子,驱逐其他公子,把他们都迁徙到莱邑。景公去世,太子荼被拥立为君,就是晏孺子。

冬季,景公的尸体还没有安葬,诸位公子担心被杀,都逃到了国外。荼的各个异母兄长公子寿、公子驹、公子黔逃奔卫国,公子驵、公子阳生逃奔鲁国。莱邑人唱道:“景公死了不得参与埋葬,三军大事不得参与谋划,公子们的部属们啊,哪里才是他们的安身之所呢?”

\begin{yuanwen}
晏孺子元年春,田乞伪事高、国者,每朝,乞骖乘,言曰:“子得君,大夫皆自危,欲谋作乱。”

又谓诸大夫曰:“高昭子可畏,及未发,先之。”大夫从之。

六月,田乞、鲍牧乃与大夫以兵入公宫,攻高昭子。昭子闻之,与国惠子救公。公师败,田乞之徒追之,国惠子奔莒,遂反杀高昭子。晏圉奔鲁。

八月,齐秉意兹。田乞败二相,乃使人之鲁召公子阳生。阳生至齐,私匿田乞家。

十月戊子,田乞请诸大夫曰:“常之母有鱼菽之祭,幸来会饮。”

会饮,田乞盛阳生橐中,置坐中央,发橐出阳生,曰:“此乃齐君矣!”

大夫皆伏谒。将与大夫盟而立之,鲍牧醉,乞诬\footnote{欺骗。}大夫曰:“吾与鲍牧谋共立阳生。”

鲍牧怒曰:“子忘景公之命乎?”

诸大夫相视欲悔,阳生前,顿首曰:“可则立之,否则已。”

鲍牧恐祸起,乃复曰:“皆景公子也,何为不可!”

乃与盟,立阳生,是为悼公。悼公入宫,使人迁晏孺子于骀,杀之幕下,而逐孺子母芮子。芮子故贱而孺子少,故无权,国人轻之。
\end{yuanwen}

晏孺子元年(前489年)春季,田乞假意顺从高氏、国氏,每次朝会,田乞都请求做高氏或国氏的陪乘,并说道:“您得到国君的宠信,大臣们人人自危,想要谋划作乱。”

又对大臣们说:“高昭子这人太可怕了,我们应该趁他还没有发动叛乱,先动手除掉他。”大臣们听从了他的建议。

六月,田乞、鲍牧和群臣们领兵进入宫中,攻打高昭子。高昭子听说后,就和国惠子一起去援救晏孺子。晏孺子的军队战败,田乞的党羽追赶他,国惠子逃奔莒国,他们于是返回杀死了高昭子。晏圉逃奔鲁国。

八月,齐国的秉意兹也前来投奔鲁国。田乞打败高昭子和国惠子后,就派人到鲁国召回公子阳生。阳生回到齐国,藏在田乞家中。十月戊子日,田乞邀请大臣们,说道:“我家常儿的母亲要举办祭礼,已经准备了一些简单的菜肴,希望各位到舍下共饮一杯。”

在酒宴上,田乞将公子阳生装在袋子里,将袋子放在座位中央,他打开袋子,让公子阳生出来,说道:“这就是齐国的国君!”

大臣们都伏地拜谒。田乞准备和诸位大臣盟誓拥立阳生。鲍牧喝醉了,田乞便欺骗大臣们说:“我和鲍牧商量要共同拥立阳生。”

鲍牧生气地说:“您难道忘了景公的遗命吗?”

众位大臣面面相觑,想要反悔,阳生上前,磕头说:“可以的话就立我为君,不可以的话就算了。”

鲍牧担心引出祸患,就说道:“都是景公的儿子,有何不可!”

于是鲍牧就与田乞订盟,拥立阳生为国君,就是悼公。悼公进入宫中,命人把晏孺子迁到骀邑,在帐篷之中杀死了他,又驱逐了晏孺子的母亲芮子。芮子原本出身卑贱,晏孺子年幼,所以并无权力,齐国人都轻视他们。

\begin{yuanwen}
悼公元年,齐伐鲁,取讙、阐。初,阳生亡在鲁,季康子以其妹妻之。及归即位,使迎之。季姬与季鲂侯通,言其情,鲁弗敢与,故齐伐鲁,竟迎季姬。季姬嬖,齐复归鲁侵地。
\end{yuanwen}

悼公元年(前488年),齐国讨伐鲁国,攻取了讙邑、阐邑。起初,阳生逃亡到鲁国的时候,季康子将自己的妹妹嫁给他为妻。等到阳生回国即位,就派人到鲁国接她回来。季姬和季鲂侯私通,并说出了其中的真相,所以鲁国不敢将她送到齐国,因此齐国攻打鲁国,最后终于迎回季姬。季姬受宠,齐国又归还了所侵占的鲁国的土地。

\begin{yuanwen}
鲍子与悼公有郤,不善。

四年,吴、鲁伐齐南方。鲍子弑悼公,赴于吴。吴王夫差哭于军门外三日,将从海入讨齐。齐人败之,吴师乃去。晋赵鞅伐齐,至赖而去。齐人共立悼公子壬,是为简公。
\end{yuanwen}

鲍子和悼公有矛盾,关系不好。

四年(前485年),吴国、鲁国攻打齐国南部。鲍子杀了悼公,向吴国报丧。吴王夫差在军门外哭祭三天,率军从海路攻打齐国。齐国打败吴军,吴国这才撤军。晋国的赵鞅攻打齐国,一直打到赖邑才撤军而走。齐国人一同拥立悼公的儿子壬为国君,就是简公。

\begin{yuanwen}
简公四年春,初,简公与父阳生俱在鲁也,监止有宠焉。及即位,使为政。田成子惮之,骤顾于朝。御鞅言简公曰:“田、监不可并也,君其择焉。”

弗听。子我夕,田逆杀人,逢之,遂捕以入。田氏方睦,使囚病而遗守囚者酒,醉而杀守者,得亡。子我盟诸田于陈宗\footnote{即田氏宗庙。}。初,田豹欲为子我臣,使公孙言豹,豹有丧而止。后卒以为臣,幸于子我。子我谓曰:“吾尽逐田氏而立女,可乎?”

对曰:“我远田氏矣。且其违者不过数人,何尽逐焉!”

遂告田氏。子行曰:“彼得君,弗先,必祸子。”

子行舍于公宫。
\end{yuanwen}

简公四年(前481年)春季,起初,简公和他的父亲阳生都在鲁国,监止受到宠信。等到简公即位,便让他主持国政。田成子对此很忧虑,几次上朝打探情况。仆御之官田鞅对简公说:“田氏与监氏不能并存,您从中选择一位吧。”

简公并不听从。监止晚上上朝,正好遇到田逆杀人,于是把他抓起来关到了狱中。当时的田氏正在和睦亲族,于是就让囚犯田逆装病,又给看守送去酒食,灌醉并杀死了看守的人,田逆得以逃走。监止和田氏族人在田氏宗庙前订盟。起初,田豹想做监止的家臣,就让公孙在监止面前替自己说话,因为田豹家里有丧事而作罢。后来,田豹终于成为监止的家臣,并受到监止的宠信。监止和他说:“我把田氏家族别的人全都赶走,立你做田氏的宗长,怎么样?”

田豹回答说:“我是田氏的远支。况且他们中间和你对抗的只有几个人,何必要全都赶走呢!”

田豹于是将这件事告诉了田氏。田逆说道:“他得到国君的宠信,如果我们不先动手,必然会害您。”

这样子行便住进了简公的宫中。

\begin{yuanwen}
夏五月壬申,成子兄弟四乘如公。子我在幄,出迎之,遂入,闭门。宦者御之,子行杀宦者。公与妇人饮酒于檀台,成子迁诸寝。公执戈将击之,太史子馀曰:“非不利也,将除害也。”

成子出舍于库,闻公犹怒,将出,曰:“何所无君!”

子行拔剑曰:“需\footnote{迟疑。},事之贼也。谁非田宗?所不杀子者有如田宗。”

乃止。子我归,属徒攻闱与大门,皆弗胜,乃出。田氏追之。丰丘人执子我以告,杀之郭关。成子将杀大陆子方,田逆请而免之。以公命取车于道,出雍门\footnote{临淄城城门。}。田豹与之车,弗受,曰:“逆为余请,豹与余车,余有私焉。事子我而有私于其雠,何以见鲁、卫之士?”
\end{yuanwen}

夏季五月壬申日,田成子兄弟坐着四乘车来到简公这里,监止在帐幕中,出来迎接,于是这些人进入,关闭大门。宦官们抵抗他们,子行杀死了宦官。简公这时正和女人在檀台上饮酒,田成子逼他们移到寝宫。简公拿起戈想要刺田成子,太史子馀说:“他们并非要对您不利,而是要为您除害。”

田成子出了宫住在武器库,听说简公怒气未消,就想逃走,他说道:“哪里没有国君!”

子行抽出剑说道:“迟疑最容易坏大事。我们这些姓田的人谁不能做田氏的宗主?您要出走,我如果不杀死您,我就不是田氏族人!”

田成子这才打消了逃走的念头。监止回家后,集合党羽攻打王宫的侧门和正门,都没有成功,就退了出来。田氏族人紧追不舍。丰丘人抓住了监止,前来报告田氏,后来在郭关杀死了他。田成子想要杀掉大陆子方,田逆为他求情才赦免了他。大陆子方以简公的名义下令在路上截了一辆车,出了雍门。田豹给他车,他不接受,说道:“田逆替我求情,田豹又送我车子,那就是我与你私下有所勾结。我辅佐监止却和他的仇人私下有所勾结,还有何颜面去见鲁国、卫国的人?”

\begin{yuanwen}
庚辰,田常执简公于袪州。公曰:“余蚤从御鞅言,不及此。”

甲午,田常弑简公于袪州。田常乃立简公弟骜,是为平公。平公即位,田常相之,专齐之政,割齐安平以东为田氏封邑。
\end{yuanwen}

庚辰日,田常在俆州抓住了简公。简公说:“我如果早就听御者田鞅的话,也不至于有今天的下场。”

甲午日,田常在俆州杀了简公。田常又拥立简公的弟弟骜为国君,就是平公。平公即位,田常辅佐他,独揽齐国的大权,强行划出齐国安平以东的地方为田氏的封邑。

\begin{yuanwen}
平公八年,越灭吴。二十五年卒,子宣公积立。
\end{yuanwen}

平公八年(前473年),越国灭掉了吴国。二十五年(前456年)平公去世,他的儿子宣公积继位。

\begin{yuanwen}
宣公五十一年卒,子康公贷立。田会反廪丘。
\end{yuanwen}

宣公五十一年(前405年),宣公去世,他的儿子康公贷继位。田会在廪丘反叛。

\begin{yuanwen}
康公二年,韩、魏、赵始列为诸侯。十九年,田常曾孙田和始为诸侯,迁康公海滨。
\end{yuanwen}

康公二年(前403年),韩、魏、赵三国开始被列为诸侯。十九年(前386年),田常的曾孙田和开始被封为诸侯,将康公迁到了海滨。

\begin{yuanwen}
二十六年,康公卒,吕氏遂绝其祀。田氏卒有齐国,为齐威王,彊于天下。
\end{yuanwen}

二十六年(前379年),康公去世,吕氏就断绝了祭祀。田氏终于拥有了齐国,到齐威王时,称雄于天下。

\begin{yuanwen}
太史公曰:吾适齐,自泰山属\footnote{延伸。}之琅邪,北被\footnote{到达。}于海,膏壤二千里,其民阔达多匿知,其天性也。以太公之圣,建国本,桓公之盛,修善政,以为诸侯会盟,称伯,不亦宜乎?洋洋哉,固大国之风也!
\end{yuanwen}

太史公说:我来到齐国,从泰山山脉延伸出琅邪山,北面一直到达大海,肥沃的土地有两千里,这里的人民胸怀豁达、多智而又深沉,这是他们的天性。凭借太公的圣明,奠定了建国的基础,到桓公时期达到极盛,推行善政,主持诸侯会盟,号称霸主,不也是理所当然吗?广阔远大啊,确实有大国的风范!

\begin{yuanwen}
太公佐周,实秉阴谋。既表东海,乃居营丘。小白致霸,九合诸侯。及溺内宠,衅锺蟲流。庄公失德,崔杼作仇。陈氏专政,厚货轻收。悼、简遘祸,田、阚非俦。沨沨馀烈,一变何由?
\end{yuanwen}

\part{卷三十三}
\chapter{鲁周公世家第三}

\begin{yuanwen}
周公旦者,周武王弟也。自文王在时,旦为子孝,笃仁,异于群子。及武王即位,旦常辅翼武王,用事居多。

武王九年,东伐至盟津,周公辅行。

十一年,伐纣,至牧野,周公佐武王,作《牧誓》。破殷,入商宫。已杀纣,周公把大钺,召公把小钺,以夹武王,衅社,告纣之罪于天,及殷民。释箕子之囚。封纣子武庚禄父,使管叔、蔡叔傅之,以续殷祀。(遍/徧)封功臣同姓戚者。封周公旦于少昊之虚曲阜,是为鲁公。周公不就封,留佐武王。
\end{yuanwen}

周公旦是周武王的弟弟。文王在世的时候,周公作为儿子就很孝顺,厚道仁爱,不同于其他的儿子。等到武王即位,周公辅佐武王,承担很多国家政务。

武王九年,周军向东征伐到达盟津,周公一直辅佐同行。

十一年,武王讨伐商纣,到达牧野,周公辅助武王,写了《牧誓》。周军大败殷军,进入商朝的王宫。杀死商纣王以后,周公手持大钺,召公手持小钺,护卫在武王左右,以牲血祭祀殷朝的社坛,向天帝和殷朝的百姓宣布纣王的罪状。释放被囚禁的箕子。封立纣王的儿子武庚禄父,派管叔、蔡叔辅佐他,以延续殷商的祭祀。随后又大规模地分封有功之臣,以及周王室的亲族。将周公旦封在少昊之墟曲阜一带,称为鲁公。周公并没有到封地去,而是留下来辅佐武王。

\begin{yuanwen}
武王克殷二年,天下未集\footnote{安定。},武王有疾,不豫,群臣惧,太公、召公乃缪卜。周公曰:“未可以戚我先王。”

周公于是乃自以为质,设三坛,周公北面立,戴璧秉圭,告于太王、王季、文王。史策祝曰:“惟尔元孙王发,勤劳阻疾。若尔三王是有负子之责于天,以旦代王发之身。旦巧能,多材多艺,能事鬼神。乃王发不如旦多材多艺,不能事鬼神。乃命于帝庭,敷佑四方,用能定汝子孙于下地,四方之民罔不敬畏。无坠天之降葆命,我先王亦永有所依归。今我其即命于元龟,尔之许我,我以其璧与圭归,以俟尔命。尔不许我,我乃屏璧与圭。”

周公已令史策告太王、王季、文王,欲代武王发,于是乃即三王而卜。卜人皆曰吉,发书视之,信吉。周公喜,开籥,乃见书遇吉。周公入贺武王曰:“王其无害。旦新受命三王,维长终是图。兹道能念予一人。”

周公藏其策金縢\footnote{téng}匮中,诫守者勿敢言。明日,武王有瘳\footnote{病愈。}。
\end{yuanwen}

武王灭掉殷商的第二年,天下尚未完全安定,武王患病,身体不适,大臣们都很担心,太公、召公虔诚地进行占卜。周公说:“这还不足以感动我们的先王。”

周公于是以自己的身体做抵押,筑起了三个祭坛,周公面北而立,顶着璧,捧着圭,祝告太王、王季、文王三位先祖。祝史代为高诵祷文道:“你们的子孙周王姬发,辛劳成疾。如果是你们三王在天,因为患病需要他来扶持,那么请用我来替代他。我行事灵巧,多才多艺,能够敬奉鬼神。你们的姬发不如我多才多艺,不能侍奉好鬼神。你们在天庭受命,保佑天下四方,因此能让你们在人间的子孙安定,四方的民众无不敬畏。只要不让上天降下的大命中途废弃,我们先王的神灵也永远有归依的地方。现在我要在大龟上接受你们的命令,你们如果同意我的请求,我就将璧与圭献给你们,回去等候你们的命令。如果你们不答应我,那我就要藏起璧和圭。”

周公既已命令史官册告太王、王季、文王,自己想替武王姬发而死,随后就在三王的神位前占卜。占卜的人说卦象都很吉利,打开兆书看,果然很吉利。周公很高兴,打开存放占卜文书的箱子,于是见兆书很吉利。周公进宫向武王道贺说道:“大王不会有什么灾害。我刚刚接受了三王的命令,您可以做长远的计划,上天的意思已经在关照你了。”

周公将册书封藏在柜子里,又告诫保管的人不可言说此事。第二天,武王的身体就好了。

\begin{yuanwen}
其后武王既崩,成王少,在强葆\footnote{通“襁褓”。}之中。周公恐天下闻武王崩而畔,周公乃践阼代成王摄行政当国。管叔及其群弟流言于国曰:“周公将不利于成王。”

周公乃告太公望、召公奭曰:“我之所以弗辟而摄行政者,恐天下畔周,无以告我先王太王、王季、文王。三王之忧劳天下久矣,于今而后成。武王蚤终,成王少,将以成周,我所以为之若此。”

于是卒相成王,而使其子伯禽代就封于鲁。周公戒伯禽曰:“我文王之子,武王之弟,成王之叔父,我于天下亦不贱矣。然我一沐三捉发,一饭三吐哺,起以待士,犹恐失天下之贤人。子之鲁,慎无以国骄人。”
\end{yuanwen}

后来武王驾崩,成王年幼,尚在襁褓当中。周公担心天下人听说武王去世而发动叛乱,便登临天子之位,代替成王处理政务。这时管叔还有他其他的几个弟兄就在国内散布谣言说:“周公将要对成王不利。”

周公便对太公望、召公奭说:“我之所以不回避而代成王摄行国政,是担心天下反叛周室,那便无法向太王、王季、文王等先王交代。三王开创这份基业长期忧劳,直到今天才算成功。武王早逝,成王又年幼,为了完成周的大业,我才这样做。”

于是他始终辅佐成王,而让他的儿子伯禽替自己去鲁国就封。周公告诫伯禽说:“我是文王的儿子,武王的弟弟,成王的叔父,以整个天下而言,我的地位也不算低了。然而我洗一次头三次挽起头发,吃一顿饭三次吐掉口中的食物,好前去接待贤士,即便这样还怕错过了天下的贤才。你到了鲁国,要谨慎而不可因为有了封国而对人骄横。”

\begin{yuanwen}
管、蔡、武庚等果率淮夷\footnote{古时民族名称,居于淮河下游。}而反。周公乃奉成王命,兴师东伐,作《大诰》。遂诛管叔,杀武庚,放蔡叔。收殷馀民,以封康叔于卫,封微子于宋,以奉殷祀。宁淮夷东土,二年而毕定。诸侯咸服宗周。
\end{yuanwen}

管叔、蔡叔和武庚等果然率领淮夷发动叛乱。周公便奉成王的命令,出兵东征,写了《大诰》。于是诛杀了管叔和武庚,放逐了蔡叔。又将殷朝的遗民集中在卫地,将那里封给康叔。将微子封在宋地,以供奉殷商的宗庙祭祀。又用了两年的时间,平定东方的淮夷。天下的诸侯都表示臣服于周朝。

\begin{yuanwen}
天降祉福,唐叔得禾,异母同颖,献之成王,成王命唐叔以(餽/馈)周公于东土,作《(餽/馈)禾》。周公既受命禾,嘉天子命,作《嘉禾》。东土以集,周公归报成王,乃为诗贻王,命之曰《鸱鸮》。王亦未敢训周公。
\end{yuanwen}

上天降福,唐叔在田里发现一株稻禾,异母同穗的,就将它献给成王,成王命令唐叔将其送给正远征东土的周公,还写了一篇《馈禾》。周公既已接受成王命赠的稻禾,赞扬天子之命,写了《嘉禾》。东方平定以后,周公返回禀报成王,并作诗赠送给成王,题为《鸱鸮》。成王也不好责备周公。

\begin{yuanwen}
成王七年二月乙未,王朝步自周,至丰,使太保召公先之雒相土。其三月,周公往营成周雒邑,卜居焉,曰吉,遂国之。
\end{yuanwen}

成王七年二月乙未日,成王从镐京步行来到丰京的文王庙,派太保召公率先前去雒地勘察地形。当年三月,周公前往指挥营建成周雒邑,又占卜此处是否可以建都,占卜结果显示吉利,便决定以此为国都。

\begin{yuanwen}
成王长,能听政。于是周公乃还政于成王,成王临朝。周公之代成王治,南面倍依以朝诸侯。及七年后,还政成王,北面就臣位,(歔歔/匔匔\footnote{gōng})如畏然。
\end{yuanwen}

成王长大了,已能够临朝听政。于是周公就将国政交还成王,由成王亲自临朝处理政务。周公代替成王治理国政,朝见诸侯时,背对屏风,面向南方而立。到七年后还政于成王时,朝向北面,处于臣子之位,完全是一副恭敬谨慎、有所畏惧的样子。

\begin{yuanwen}
初,成王少时,病,周公乃自揃其蚤\footnote{:剪掉自己的指甲。揃:修剪。蚤:通“爪”。}(沈/沉)之河,以祝于神曰:“王少未有识,奸神命者乃旦也。”

亦藏其策于府。成王病有瘳。及成王用事,人或谮\footnote{说坏话。}周公,周公奔楚。成王发府,见周公祷书,乃泣,反周公。
\end{yuanwen}

起初,成王年少时,有一次患病,周公剪下自己的指甲扔到河中,向河神祷告说:“王年幼不懂事,冒犯神灵的是我。”

事后仍将祷告的册文藏在府中。成王的病果然好了。等到成王掌政之时,有人进谗言诬陷周公,周公逃奔楚国。成王打开府库,看到了周公当年祷告的册文,被感动得哭了,立刻请回周公。

\begin{yuanwen}
周公归,恐成王壮,治有所淫佚,乃作《多士》,作《毋逸》。《毋逸》称:“为人父母,为业至长久,子孙骄奢忘之,以亡其家,为人子可不慎乎!故昔在殷王中宗,严恭敬畏天命,自度治民,震惧不敢荒宁,故中宗飨国\footnote{当政,在位。飨:通“享”。}七十五年。其在高宗,久劳于外,为与小人,作其即位,乃有亮闇,三年不言,言乃讙,不敢荒宁,密靖殷国,至于小大无怨,故高宗飨国五十五年。其在祖甲,不义惟王,久为小人于外,知小人之依,能保施小民,不侮鳏\footnote{guān}寡,故祖甲飨国三十三年。”《多士》称曰:“自汤至于帝乙,无不率祀明德,帝无不配天者。在今后嗣王纣,诞淫厥佚,不顾天及民之从也。其民皆可诛。”“文王日中昃\footnote{日西斜。}不暇食,飨国五十年。”作此以诫成王。
\end{yuanwen}

周公回朝以后,担心成王年轻气盛,在治国时有纵欲放荡之处,就写了《多士》,又写了《毋逸》。《毋逸》篇中说:“为人父母者,创业极其长久艰难,子孙却骄奢淫逸,忘了这些,以至于失去家业,为人子的,不可不谨慎啊!从前的殷王中宗,严谨恭敬地对待天命,自律并治理百姓,心中总有畏惧之心,从来不敢荒废国事,所以中宗能够当政七十五年。到了高宗,他长年住在民间,和人民一起劳动。当他即位之时,便有丧事,三年不谈论国事。三年期后,一发表言论就能赢得百姓的拥戴,他治国不敢荒废,一心要让殷国安定,以至无论贵贱老少都不怨恨他,所以高宗在位五十五年。到了祖甲的时候,因认为自己当君王为不合道义之时,所以长期逃在民间,了解百姓都依赖什么,能够保护百姓并广施恩惠,做到了不欺鳏寡,所以祖甲在位三十三年。”《多士》篇中说:“从商汤到帝乙,无不恭顺祭祀,修明德政,帝王没有不顺从天道的。到如今继承王位的纣王,荒诞淫佚,无视天命及百姓的依从,以至他的臣民都有罪当诛。”“文王每天忙到过了中午没时间吃饭,在位五十年。”周公写了这些篇章来告诫成王。

\begin{yuanwen}
成王在丰,天下已安,周之官政未次序,于是周公作《周官》,官别其宜,作《立政》,以便百姓\footnote{这里指百官。}。百姓说\footnote{通“悦”,高兴。}。
\end{yuanwen}

成王居住在丰都,天下已经安定,但周朝的政府机构尚未完善有序。于是周公撰写了《周官》,对各级官吏的职责范围加以区分,撰写了《立政》,使百官明白为官处理政务的道理。百官都很高兴。

\begin{yuanwen}
周公在丰,病,将没,曰:“必葬我成周,以明吾不敢离成王。”

周公既卒,成王亦让,葬周公于毕,从文王,以明予小子不敢臣周公也。
\end{yuanwen}

周公在丰京,生了病,临去世时说:“我死后一定要将我葬在成周,表明我不敢离开成王。”

周公死后,成王表示谦让,将周公葬在了毕邑,追随文王,以表示他不敢将周公当作他的臣子。

\begin{yuanwen}
周公卒后,秋未(穫/获),暴风雷,禾尽偃\footnote{倒下。},大木尽拔。周国大恐。成王与大夫朝服以开金縢书,王乃得周公所自以为功代武王之说。二公及王乃问史百执事,史百执事曰:“信有,昔周公命我勿敢言。”

成王执书以泣,曰:“自今后其无缪卜乎!昔周公勤劳王家,惟予幼人弗及知。今天动威以彰周公之德,惟朕小子其迎,我国家礼亦宜之。”

王出郊,天乃雨,反风,禾尽起。二公命国人,凡大木所偃,尽起而筑之。岁则大孰\footnote{通“熟”,丰收。}。于是成王乃命鲁得郊祭文王。鲁有天子礼乐者,以襃周公之德也。
\end{yuanwen}

周公去世以后,正当秋季尚未收获,突然间狂风大作,雷电交加,庄稼全部倒伏,大树被连根拔起。周国上下大为惊恐。这时成王便和大夫们穿上礼服,打开金縢之书,便看到了当初周公愿意以自己为人质替武王而死的简书。太公、召公和成王向史官和其他众官员求证,他们回答说:“确有此事,当时周公命令我们不许说出去。”

成王手持简书哭道:“从今以后恐怕再也没有这样虔诚的占卜了。从前周公为王室辛劳,我这个年幼的人还不知道。如今上天显示威严,以此表彰周公的德行,我要亲自去迎接神灵,按我国的礼仪也应该这样做。”

于是,成王出城在城郊举行祭天之礼,天立即下起雨来,风向也倒转了,倒伏的庄稼又都直立起来。太公、召公让人们扶起所有被刮倒的大树,用土将树根培土夯实。这一年全国都获得了大丰收。于是成王命令鲁国国君可以举行郊祭并祭祀文王。鲁国拥有天子的礼乐,就是用来褒奖周公的德行的。

\begin{yuanwen}
周公卒,子伯禽固已前受封,是为鲁公。鲁公伯禽之初受封之鲁,三年而后报政周公。周公曰:“何迟也?”

伯禽曰:“变其俗,革其礼,丧三年然后除之,故迟。”

太公亦封于齐,五月而报政周公。周公曰:“何疾也?”

曰:“吾简其君臣礼,从其俗为也。”

及后闻伯禽报政迟,乃叹曰:“呜呼,鲁后世其北面事齐矣!夫政不简不易,民不有近;平易近民,民必归之。”
\end{yuanwen}

周公去世,他的儿子伯禽在之前已受册封,就是鲁公。鲁公伯禽起初被封到鲁国,三年之后才向周公报告鲁国的政绩。周公说:“为什么报得迟了?”

伯禽说:“改变那里的风俗,变革那里的礼制,丧事要过了三年才能除服,所以迟了。”

当时太公也被封到齐国,五个月以后就向周公报告政务。周公说:“为什么这样快呢?”

太公回答道:“我简化了君臣之间烦琐的礼仪,顺从当地的风俗。”

待到后来有了伯禽报告政事迟缓的事,周公就感叹道:“唉!鲁国将来必定要臣服于齐国了!如果为政不简便易行,人民就不会亲近;统治者平易近人,百姓一定会归附于他。”

\begin{yuanwen}
伯禽即位之后,有管、蔡等反也,淮夷、徐戎亦并兴反。于是伯禽率师伐之于肸\footnote{bì},作《肸誓》,曰:“陈尔甲胄,无敢不善。无敢伤牿\footnote{gù}。马牛其风,臣妾逋逃,勿敢越逐,敬复之。无敢寇攘,逾墙垣。鲁人三郊三隧,歭\footnote{储备。}尔刍茭、糗\footnote{qiǔ}粮、桢榦,无敢不逮。我甲戌筑而征徐戎,无敢不及,有大刑。”作此《肸誓》,遂平徐戎,定鲁。
\end{yuanwen}

伯禽即位以后,发生了管叔和蔡叔等人的叛乱,淮夷、徐戎也同时起兵叛乱。于是伯禽率领军队到肸邑讨伐他们,并作《肸誓》,上面说:“准备好你们的铠甲和头盔,不要有破损不全的。不要伤害拴着的牲畜。如果马牛走失,奴隶逃跑,不许擅自离开队伍前去追赶,要将获得的他人的牛马、奴隶恭敬地归还。不允许掠夺财物,不允许翻墙偷窃。鲁国北、西、南三个方向近郊以及远郊的人,要储备牲畜吃的刍草和人吃的干粮,还有筑墙用的木柱木板,不可以缺少。我将在甲戌日建造工事以征伐徐戎,到时候不许不来,否则就要用严厉的刑罚惩罚。”并写下《肸誓》,于是平定了徐戎,安定了鲁国。

\begin{yuanwen}
鲁公伯禽卒,子考公酋立。考公四年卒,立弟熙,是谓炀公。炀公筑茅阙门。六年卒,子幽公宰立。幽公十四年。幽公弟晞杀幽公而自立,是为魏公。魏公五十年卒,子厉公擢立。厉公三十七年卒,鲁人立其弟具,是为献公。献公三十二年卒,子真公濞立。
\end{yuanwen}

鲁公伯禽去世后,他的儿子考公酋继立。考公在位四年后去世,他的弟弟熙被拥立为国君,就是炀公。炀公修筑了茅阙宫门。在位六年,炀公去世,他的儿子幽公宰继立。幽公十四年,他的弟弟 杀了幽公后自立为国君,就是魏公。魏公在位五十年去世,他的儿子厉公擢继立。厉公在位三十七年去世,鲁人拥立他的弟弟具为国君,就是献公。献公在位三十二年去世,他的儿子真公濞继立。

\begin{yuanwen}
真公十四年,周厉王无道,出奔彘,共和行政。

二十九年,周宣王即位。

三十年,真公卒,弟敖立,是为武公。
\end{yuanwen}

真公十四年,周厉王昏庸无道,出逃投奔彘,国家的政事由周、召二公一起主持。

真公二十九年(前827年),周宣王即位。

三十年(前826年),真公去世,他的弟弟敖继立,就是武公。

\begin{yuanwen}
武公九年春,武公与长子括,少子戏,西朝周宣王。宣王爱戏,欲立戏为鲁太子。周之樊仲山父谏宣王曰:“废长立少,不顺;不顺,必犯王命;犯王命,必诛之:故出令不可不顺也。令之不行,政之不立;行而不顺,民将弃上。夫下事上,少事长,所以为顺。今天子建诸侯,立其少,是教民逆也。若鲁从之,诸侯效之,王命将有所壅;若弗从而诛之,是自诛王命也。诛之亦失,不诛亦失,王其图之。”

宣王弗听,卒立戏为鲁太子。

夏,武公归而卒,戏立,是为懿公。
\end{yuanwen}

武公九年(前817年)春季,武公和他的大儿子括、小儿子戏一起到西方去朝见周宣王。宣王非常宠爱戏,想要立戏为鲁国的太子。周国的大夫樊仲山父劝谏宣王说:“废掉大儿子而立小儿子,这是不符合制度的;不符合制度,就一定会触犯君王的命令;触犯君王的命令,就一定要诛杀他:因此颁布命令不能不符合制度。如果下达的诏令得不到执行,国家的政权也就无法建立;推行的政令不符合法度,百姓就会背弃君王。下级效命上级,年少的服侍年长的,这都是合乎制度的做法。如今天子册封诸侯,立诸侯的小儿子为嗣,这是教百姓做事不符合制度法令。如果鲁国听从命令,其他的诸侯都纷纷效仿,先王的命令就无法实行;如果鲁国不听从您的命令而诛杀他,就相当于您亲自违背了先王的训命。诛杀他是过失,不诛杀他也是过失,君王您要认真考虑这件事。”

宣王并不听从,最终还是立戏为鲁国的太子。

夏季,武公回国以后就去世了,戏被拥立为国君,就是懿公。

\begin{yuanwen}
懿公九年,懿公兄括之子伯御与鲁人攻弑懿公,而立伯御为君。伯御即位十一年,周宣王伐鲁,杀其君伯御,而问鲁公子能道顺诸侯者,以为鲁后。

樊穆仲曰:“鲁懿公弟称,肃恭明神,敬事耆老\footnote{受尊敬的老人。};赋事行刑,必问于遗训而咨于固实\footnote{往事,先例。又作“故事”。};不干所问,不犯所咨。”

宣王曰:“然,能训治其民矣。”

乃立称于夷宫,是为孝公。自是后,诸侯多畔王命。
\end{yuanwen}

懿公九年(前807年),懿公的哥哥括的儿子伯御和鲁国人一起攻打并杀掉懿公,拥立伯御为鲁国的国君。伯御即位十一年时,周宣王讨伐鲁国,杀死了鲁国的国君伯御,在鲁国公子中询问谁能够治理、理顺诸侯,并让他担任鲁国君位的继承人。

樊穆仲说:“鲁懿公的弟弟称,严肃恭谨地事奉鬼神,恭敬地对待长辈与老人;办理事务与执行刑罚的时候,一定会咨询先王的训命,也会根据从前的经验教训来办事;不违背先王的训命,不和过去的经验发生冲突。”

宣王说:“好的,这样他就可以训导并治理好他的百姓了。”

于是在夷宫册立称为鲁国的国君,就是孝公。从此以后,诸侯经常违抗王命。

\begin{yuanwen}
孝公二十五年,诸侯畔周,犬戎杀幽王。秦始列为诸侯。

二十七年,孝公卒,子弗湟立,是为惠公。

惠公三十年,晋人弑其君昭侯。

四十五年,晋人又弑其君孝侯。
\end{yuanwen}

孝公二十五年(前771年),诸侯叛变周王室,犬戎杀掉周幽王。秦开始被列为诸侯。

二十七年(前769年),孝公去世,他的儿子弗湟继立,就是惠公。

惠公三十年(前739年),晋国人杀掉他们的国君昭侯。

四十五年(前724年),晋国人又杀掉他们的国君孝侯。

\begin{yuanwen}
四十六年,惠公卒,长庶子息摄当国,行君事,是为隐公。初,惠公適\footnote{通“嫡”。}夫人无子,公贱妾声子生子息。息长,为娶于宋。宋女至而好,惠公夺而自妻之。生子允。登宋女为夫人,以允为太子。及惠公卒,为允少故,鲁人共令息摄政,不言即位。
\end{yuanwen}

四十六年(前723年),惠公去世,长庶子息代理政事,行使国君的权力,就是隐公。起初,惠公的嫡夫人没有儿子,而惠公的贱妾声子生下儿子息。息长大后,为他从宋国娶了妻子。宋女到了鲁国,惠公见她十分美丽,就自己把她夺了过来。后来宋女生下儿子允。惠公把宋女升为夫人,让允做太子。等到惠公去世,因为允年纪小,鲁国人共同要求息代理国政,没有说让他即位。

\begin{yuanwen}
隐公五年,观渔\footnote{捕鱼。}于棠。

八年,与郑易天子之太山之邑祊及许田,君子讥之。
\end{yuanwen}

隐公五年(前718年),隐公来到棠地观看捕鱼。

八年(前715年),用许田与郑国交换天子赏赐的祭祀泰山的汤沐邑祊,君子讥讽这件事。

杨慎:「老少相让,几于争矣。孔子知鲁道之将微,叹之。太史公观庆父、叔牙之乱务以揖让相尚,而君臣之间,其戾若此,故亦叹之。」

\begin{yuanwen}
十一年冬,公子挥谄谓隐公曰:“百姓便\footnote{认为好。}君,君其遂立。吾请为君杀子允,君以我为相。”

隐公曰:“有先君命。吾为允少,故摄代。今允长矣,吾方营菟\footnote{tù}裘之地而老焉,以授子允政。”

挥惧子允闻而反诛之,乃反谮隐公于子允曰:“隐公欲遂立,去子,子其图之。请为子杀隐公。”

子允许诺。

十一月,隐公祭钟巫\footnote{神名。},齐\footnote{通“斋”。}于社圃,馆于蔿氏。挥使人杀隐公于蔿氏,而立子允为君,是为桓公。
\end{yuanwen}

十一年(前712年)冬季,公子挥向隐公进谗言说:“百姓认为您好,就请您正式继位为国君吧。我请求为您去杀掉太子允,然后您任用我为国相。”

隐公说:“有先王的遗命。我因为允年纪小,因此代理国政。如今允已经长大了,我正想要经营菟裘之地,并到那里养老,将国政交还给子允。”

公子挥畏惧子允听说这件事后会反过来诛杀他,于是返过来在子允面前诬陷隐公说:“隐公想要正式即位,并除掉你,请你尽早做好打算。我请求为你杀掉隐公。”

子允同意了。

十一月,隐公祭祀钟巫神,在社圃园里进行斋戒,住在蒍氏家。公子挥派人在蒍氏家中杀掉隐公,立子允为国君,就是桓公。

\begin{yuanwen}
桓公元年,郑以璧易天子之许田。

二年,以宋之赂鼎入于太庙,君子讥之。

三年,使挥迎妇于齐为夫人。

六年,夫人生子,与桓公同日,故名曰同。同长,为太子。
\end{yuanwen}

桓公元年(前711年),郑国用璧玉和鲁国交换天子赏赐给鲁国国君的许田。

二年(前710年),把宋国贿赂的鼎放到太庙中,君子讥讽这件事。

三年(前709年),桓公派公子挥前往齐国迎娶齐女为夫人。

六年(前706年),夫人生下儿子,和桓公的生日是同一天,因此给孩子取名为同。同长大以后,被立为太子。

\begin{yuanwen}
十六年,会于曹,伐郑,入厉公。
\end{yuanwen}

十六年(前696年),桓公和各诸侯在曹国举行会盟,讨伐郑国,将郑厉公送回国内。

\begin{yuanwen}
十八年春,公将有行,遂与夫人如齐。申繻谏止,公不听,遂如齐。齐襄公通桓公夫人。公怒夫人,夫人以告齐侯。

夏四月丙子,齐襄公飨公,公醉,使公子彭生抱鲁桓公,因命彭生摺\footnote{通“折”。}其胁,公死于车。鲁人告于齐曰:“寡君畏君之威,不敢宁居,来脩好礼。礼成而不反,无所归咎,请得彭生(以)除丑于诸侯。”

齐人杀彭生以说鲁。立太子同,是为庄公。庄公母夫人因留齐,不敢归鲁。
\end{yuanwen}

十八年(前694年)春季,桓公想要出行,便和夫人一起前往齐国。大夫申繻劝谏阻拦,桓公并不听从,于是前往齐国。齐襄公和桓公夫人私通。桓公怨怒夫人,夫人把这件事告诉了齐襄公。

夏季四月丙子日,齐襄公宴请鲁桓公,桓公喝得大醉,齐襄公命令公子彭生将鲁桓公抱到车上,趁机命令彭生打断了桓公的肋骨,桓公死在了车上。鲁国人向齐国的国君提出要求说:“我们的君主畏惧您的威严,不敢安宁地居住,亲自来到贵国修定盟好之礼。礼仪完成人却不能返回,又没有地方可以追究,请让我们得到彭生,以便能在诸侯之间消除这件事所产生的恶劣影响。”

齐国人杀掉彭生来讨好鲁国。鲁国人拥立太子同为国君,就是庄公。庄公的母亲,也就是鲁桓公的夫人因此留在齐国,不敢回到鲁国。

\begin{yuanwen}
庄公五年冬,伐卫,内卫惠公。
\end{yuanwen}

庄公五年(前689年)冬季,鲁国讨伐卫国,将卫惠公送回卫国执政。

\begin{yuanwen}
八年,齐公子纠来奔。

九年,鲁欲内子纠于齐,后桓公,桓公发兵击鲁,鲁急,杀子纠。召忽死。齐告鲁生致管仲。

鲁人施伯曰:“齐欲得管仲,非杀之也,将用之,用之则为鲁患。不如杀,以其尸与之。”

庄公不听,遂囚管仲与齐。齐人相管仲。
\end{yuanwen}

八年(前686年),齐国的公子纠前来投奔鲁国。

九年(前685年),鲁国想要护送公子纠回齐国执政,但是落后于齐桓公小白,齐桓公发兵攻打鲁国,鲁国形势危急,便杀掉公子纠。召忽殉死。齐国告诉鲁国,一定要活捉管仲并把他送到齐国。

鲁国人施伯说:“齐国想要得到管仲,并非要杀掉他,而是要重用他,如果管仲被重用,就会为鲁国带来祸患。不如现在就杀掉他,然后把管仲的尸体送到齐国。”

庄公并不听从,于是囚禁了管仲并送给齐国。齐国任用管仲为相。

\begin{yuanwen}
十三年,鲁庄公与曹沬会齐桓公于柯,曹沬劫齐桓公,求鲁侵地,已盟而释桓公。桓公欲背约,管仲谏,卒归鲁侵地。

十五年,齐桓公始霸。

二十三年,庄公如齐观社。
\end{yuanwen}

十三年(前681年),鲁庄公和曹沬在柯地与齐桓公举行会盟,曹沬劫持齐桓公,要求齐国归还所侵占的鲁国的土地,盟誓结束后释放了桓公。桓公想要背弃盟约,管仲劝谏,最终归还了侵占的鲁国的土地。

十五年(前679年),齐桓公开始在诸侯中称霸。

二十三年(前671年),庄公到齐国观看祭祀社神的活动。

\begin{yuanwen}
三十二年,初,庄公筑台临党氏,见孟女\footnote{党氏的长女。},说而爱之,许立为夫人,割臂以盟。孟女生子斑。斑长,说梁氏女,往观。圉人荦自墙外与梁氏女戏。斑怒,鞭荦\footnote{luò}。

庄公闻之,曰:“荦有力焉,遂杀之,是未可鞭而置也。”

斑未得杀。会庄公有疾。庄公有三弟,长曰庆父,次曰叔牙,次曰季友。庄公取齐女为夫人曰哀姜。哀姜无子。哀姜娣曰叔姜,生子开。庄公无適嗣,爱孟女,欲立其子斑。庄公病,而问嗣于弟叔牙。

叔牙曰:“一继一及\footnote{父死子“继”与兄终弟“及”。},鲁之常也。庆父在,可为嗣,君何忧?”

庄公患叔牙欲立庆父,退而问季友。季友曰:“请以死立斑也。”

庄公曰:“曩\footnote{nǎng}者叔牙欲立庆父,奈何?”

季友以庄公命命牙待于(针/鍼)巫氏,使(针/鍼)季劫饮叔牙以鸩,曰:“饮此则有后奉祀;不然,死且无后。”

牙遂饮鸩而死,鲁立其子为叔孙氏。

八月癸亥,庄公卒,季友竟立子斑为君,如庄公命。侍丧,舍于党氏。
\end{yuanwen}

三十二年(前662年),起初,庄公建造高台的时候曾经到大夫党氏的家中,见到了党氏的女儿孟任,很高兴并且很喜爱她,许诺要立她为夫人,并割破手臂来盟誓。后来孟女生下儿子斑。斑长大后,喜欢上大夫梁氏的女儿,就前去看望她。看见养马官荦在墙外和梁氏的女儿嬉戏。斑非常生气,鞭打荦。

庄公听说这件事以后,说:“荦非常有力气,应该杀掉他,不可以只是以鞭打来处置他。”

斑却未能杀掉荦。这时正赶上庄公生病。庄公有三个弟弟,大弟弟叫庆父,二弟叫叔牙,三弟叫季友。庄公娶了齐国的女子为夫人,名叫哀姜。哀姜没有儿子。哀姜的妹妹名叫叔姜,她为庄公生下儿子开。庄公没有嫡子适合立为继承人,他非常喜爱孟女,就想立她的儿子斑为太子。庄公病重的时候,向弟弟叔牙询问立嗣的事情。

叔牙说:“父死子继,兄终弟及,这是鲁国始终奉行的制度。庆父在,可以作为继承人,君王有什么可担忧的呢?”

庄公担心叔牙想要立庆父为继承人,叔牙出去以后,他又询问季友。季友说:“请让我用我的生命担保来拥立斑为国君。”

庄公说:“刚才叔牙表示想要立庆父,我该怎么办呢?”

于是季友以庄公的命令让叔牙等在鲁大夫鍼巫氏的家里,让鍼季劫持叔牙给他喝下毒酒,并说:“你喝下这杯酒,以后就有后人为你祭祀;否则你死了也不会有人祭祀你。”

叔牙于是喝下毒酒而死,鲁国的国君立叔牙的儿子为叔孙氏。

八月癸亥日,庄公去世,季友最终立斑为国君,遵照庄公的命令行事。斑守丧期间,住在党氏家里。

\begin{yuanwen}
先时庆父与哀姜私通,欲立哀姜娣子开。及庄公卒而季友立斑,十月己未,庆父使圉人荦杀鲁公子斑于党氏。季友饹陈。庆父竟立庄公子开,是为湣公。
\end{yuanwen}

在这之前,庆父曾经与哀姜私通,打算立哀姜妹妹的儿子开为国君。等到庄公去世而季友拥立斑为国君,十月己未日,庆父派养马官荦在党氏家中杀死了鲁公子斑。季友逃奔陈国。庆父终于拥立庄公的儿子开为国君,就是湣公。

\begin{yuanwen}
湣公二年,庆父与哀姜通益甚。哀姜与庆父谋杀湣公而立庆父。庆父使卜齮袭杀湣公于武闱。季友闻之,自陈与湣公弟申如邾,请鲁求内之。鲁人欲诛庆父。
\end{yuanwen}

湣公二年(前660年),庆父与哀姜私通得更加频繁。哀姜和庆父就谋划杀死湣公而立庆父为国君。庆父派卜齮在武闱偷袭并杀死湣公。季友听说了这件事,从陈国和湣公的弟弟申一起前往邾国,请求鲁国接纳他们回国。鲁国人想要杀掉庆父。

\begin{yuanwen}
庆父恐,奔莒。于是季友奉子申入,立之,是为釐公。釐公亦庄公少子。哀姜恐,奔邾。季友以赂如莒求庆父,庆父归,使人杀庆父,庆父请奔,弗听,乃使大夫(奚/傒)斯行哭而往。庆父闻(奚/傒)斯音,乃自杀。齐桓公闻哀姜与庆父乱以危鲁,及召之邾而杀之,以其尸归,戮之鲁。鲁釐公请而葬之。
\end{yuanwen}

庆父十分恐惧,逃奔到莒国。于是季友护卫子申回到鲁国,并拥立他为国君,就是釐公。釐公也是庄公的小儿子。哀姜很恐慌,逃奔邾国。季友贿赂莒国,捉到了庆父,庆父被遣送回鲁国,季友就派人去杀庆父,庆父请求让他逃走,那个杀他的人没有答应,于是季友让大夫傒斯哭着前往。庆父听到傒斯的哭声,便自杀了。齐桓公听说哀姜与庆父淫乱从而危及鲁国,就从邾国把她召来杀掉了,并把她的尸体送回鲁国,在鲁国陈尸示众。鲁釐公请求将哀姜安葬了。

\begin{yuanwen}
季友母陈女,故亡在陈,陈故佐送季友及子申。季友之将生也,父鲁桓公使人卜之,曰:“男也,其名曰‘友’,间于两社,为公室辅。季友亡,则鲁不昌。”

及生,有文在掌曰“友”,遂以名之,号为成季。其后为季氏,庆父后为孟氏也。
\end{yuanwen}

季友的母亲是陈国的女子,因此逃亡到陈国,陈国因此帮助护送季友和子申返回鲁国。季友将要出生的时候,他的父亲鲁桓公曾经让人占卜,卜者说:“是个男孩,他的名字叫‘友’,将来会处于两社之间,成为公室的辅助大臣。季友不在,鲁国就不会昌盛。”

等到季友出生,手掌上有“友”字的纹路,于是就用“友”来命名,号为成季。他的后人是季氏,庆父的后人是孟氏。

\begin{yuanwen}
釐公元年,以汶阳鄪封季友。季友为相。

九年,晋里克杀其君傒齐、卓子。齐桓公率釐公讨晋乱,至高梁而还,立晋惠公。

十七年,齐桓公卒。

二十四年,晋文公即位。

三十三年,釐公卒,子兴立,是为文公。
\end{yuanwen}

釐公元年(前659年),鲁国国君把汶阳、鄪邑封赏给季友。季友成为鲁相。

九年(前651年),晋国的大夫里克杀掉他的国君傒齐和卓子。齐桓公率领釐公讨伐晋国的叛乱,到达晋国的高梁才返回,拥立晋惠公为国君。

十七年(前643年),齐桓公去世。

二十四年(前636年),晋文公即位。

三十三年(前627年),釐公去世,他的儿子兴继立,就是文公。

\begin{yuanwen}
文公元年,楚太子商臣弑其父成王,代立。

三年,文公朝晋襄父。
\end{yuanwen}

文公元年(前626年),楚国的太子商臣杀了他的父亲楚成王,即位为楚王。

三年(前624年),文公前去朝拜晋襄公。

\begin{yuanwen}
十一年十月甲午,鲁败翟\footnote{通“狄”,古代的一个民族。}于咸,获长翟乔如,富父终甥舂\footnote{通“冲”,刺。}其喉以戈,杀之,埋其首于子驹之门,以命宣伯。
\end{yuanwen}

十一年(前616年)十月甲午日,鲁国人在咸地打败了翟人,俘虏了长狄的首领乔如,鲁国的大夫富父终甥用戈刺他的咽喉,杀死了他,然后将他的头颅埋在子驹之门,用“乔如”来命名宣伯。

\begin{yuanwen}
初,宋武公之世,鄋瞒伐宋,司徒皇父\footnote{名充石,字皇父,宋戴公之子。}帅师御之,以败翟于长丘,获长翟缘斯。晋之灭路,获乔如弟棼如。

齐惠公二年,鄋瞒伐齐,齐王子城父获其弟荣如,埋其首于北门。卫人获其季弟简如。鄋瞒由是遂亡。
\end{yuanwen}

起初,宋武公统治时期,鄋瞒讨伐宋国,司徒皇父率领军队抵御,在长丘打败了翟人,俘获了长翟的首领缘斯。晋国灭掉路国,俘获了乔如的弟弟棼如。

齐惠公二年,鄋瞒讨伐齐国,齐国的王子城父抓获了乔如的弟弟荣如,并把他的首级埋在北门。卫国人抓获了鄋瞒最小的弟弟简如。鄋瞒从此就灭亡了。

\begin{yuanwen}
十五年,季文子使于晋。
\end{yuanwen}

十五年(前612年),季文子出使晋国。

\begin{yuanwen}
十八年二月,文公卒。文公有二妃:长妃齐女为哀姜,生子恶及视;次妃敬嬴,嬖爱,生子俀。俀私事襄仲,襄仲欲立之,叔仲曰不可。襄仲请齐惠公,惠公新立,欲亲鲁,许之。

冬十月,襄仲杀子恶及视而立俀,是为宣公。哀姜归齐,哭而过巿,曰:“天乎!襄仲为不道,杀適立庶!”

巿人皆哭,鲁人谓之“哀姜”。鲁由此公室卑,三桓彊。
\end{yuanwen}

十八年(前609年)二月,文公去世。文公有两位妃子:长妃是齐国的女子,名叫哀姜,生下儿子恶和视;次妃叫敬嬴,很得文公宠爱,生下儿子俀。俀私下与襄仲交好,襄仲想要拥立他为国君,叔仲说不可以。襄仲就请齐惠公帮忙,惠公刚刚即位,想要和鲁国亲近,于是便应允了。

冬季十月,襄仲杀掉公子恶和视,立俀为国君,就是宣公。哀姜回到齐国,哭着经过闹市,说道:“天啊!襄仲做出大逆不道之事,杀掉嫡子拥立庶子!”

闹市上的人都哭了,鲁国人称她为“哀姜”。从此鲁国的公室渐渐衰微,三桓的势力变得强大。

\begin{yuanwen}
宣公俀十二年,楚庄王彊,围郑。郑伯降,复国之。
\end{yuanwen}

宣公俀十二年(前597年),楚庄王很强大,他率领军队包围了郑国。郑伯投降,但是楚庄王很快又使他复国。

\begin{yuanwen}
十八年,宣公卒,子成公黑肱立,是为成公。季文子曰:“使我杀適立庶失大援者,襄仲。”

襄仲立宣公,公孙归父有宠。宣公欲去三桓,与晋谋伐三桓。会宣公卒,季文子怨之,归父奔齐。
\end{yuanwen}

十八年(前591年),宣公去世,他的儿子黑肱被立为国君,就是成公。季文子说:“让我们杀掉嫡子拥立庶子从而失去强大外援的,就是襄仲。”

襄仲拥立宣公,他的儿子公孙归父因此受宠。宣公想要除去三桓的势力,曾经和晋国谋划讨伐三桓。不久宣公就去世了,季文子怨恨襄仲,公孙归父逃奔齐国。

\begin{yuanwen}
成公二年春,齐伐取我隆。

夏,公与晋郤克败齐顷公于(鞍/鞌),齐复归我侵地。

四年,成公如晋,晋景公不敬鲁。鲁欲背晋合于楚,或谏,乃不。

十年,成公如晋。晋景公卒,因留成公送葬,鲁讳之。

十五年,始与吴王寿梦会锺离。
\end{yuanwen}

成公二年(前589年)春季,齐国讨伐鲁国,攻取了隆邑。

夏季,成公和晋郤克在鞌地打败了齐顷公,齐国又一次归还所侵占的鲁国的土地。

四年(前587年),成公前往晋国,晋景公对待鲁成公的态度非常不恭敬。鲁成公想要背叛晋国并与楚国缔结的盟约,有人进谏劝阻,于是没有这样做。

十年(前581年),成公前往晋国。晋景公已经去世,因此晋国人把成公留下来送葬,鲁国人认为这是耻辱的事,避讳不说。

十五年(前576年),成公开始和吴王寿梦在钟离举行盟会。

\begin{yuanwen}
十六年,宣伯告晋,欲诛季文子。文子有义,晋人弗许。

十八年,成公卒,子午立,是为襄公。是时襄公三岁也。
\end{yuanwen}

十六年(前575年),宣伯告诉晋国,想要杀掉季文子。季文子很有道义,晋国人没有允许。

十八年(前573年),成公去世,他的儿子午继立,就是襄公。这时襄公只有三岁。

\begin{yuanwen}
襄公元年,晋立悼公。往年冬,晋栾书弑其君厉公。

四年,襄公朝晋。

五年,季文子卒。家无衣帛之妾,厩无食粟之马,府无金玉,以相三君。君子曰:“季文子廉忠矣。”

九年,与晋伐郑。晋悼公冠襄公于卫,季武子从,相\footnote{辅助。}行礼。

十一年,三桓氏分为三军。

十二年,朝晋。

十六年,晋平公即位。

二十一年,朝晋平公。

二十二年,孔丘生。

二十五年,齐崔杼弑其君庄公,立其弟景公。

二十九年,吴延陵季子使鲁,问周乐,尽知其意,鲁人敬焉。

三十一年六月,襄公卒。其九月,太子卒。鲁人立齐归\footnote{胡国女子,襄公妾敬归的妹妹。}之子裯为君,是为昭公。
\end{yuanwen}

襄公元年(前572年),晋国人拥立悼公为国君。前一年冬季,晋国的大夫栾书杀死了他的国君厉公。

四年(前569年),襄公前往晋国朝拜。

五年(前568年),季文子去世。家里没有穿着丝绸的妻妾,马厩中也没有吃粮食的马匹,府库中没有金银珠玉,他连续辅佐过三位国君。君子说:“季文子确实是位廉洁忠诚的人。”

九年(前564年),鲁国和晋国联合讨伐郑国。晋悼公在卫国为襄公举行加冠典礼,季武子跟从襄公,辅助举行冠礼。

十一年(前562年),三桓将鲁国的军队分为三支。

十二年(前561年),鲁襄公前往晋国朝拜。

十六年(前557年),晋平公即位。

二十一年(前552年),鲁襄公再次前去朝见晋平公。

二十二年(前551年),孔丘出生。

二十五年(前548年),齐国的崔杼杀死了他的国君齐庄公,拥立庄公的弟弟景公为国君。

二十九年(前544年),吴国的延陵季子出使鲁国,向他询问周王室的音乐,他完全知道其中的含意,鲁国人非常尊敬他。

三十一年(前542年)六月,鲁襄公去世。这一年的九月,太子也去世了。鲁国人拥立齐归的儿子裯为国君,就是昭公。

\begin{yuanwen}
昭公年十九,犹有童心。穆叔不欲立,曰:“太子死,有母弟可立,不即立长。年钧\footnote{通“均”,相同。}择贤,义钧则卜之。今裯\footnote{dāo}非適嗣,且又居丧意不在戚而有喜色,若果立,必为季氏忧。”

季武子弗听,卒立之。比及葬,三易衰。君子曰:“是不终也。”
\end{yuanwen}

昭公十九岁时,仍然保有一颗童稚之心。穆叔不想拥立他,说:“太子去世以后,还有同母所生的弟弟可以拥立,如果没有同母所生的弟弟,才可以拥立庶长子。年纪如果相仿,就选择贤能的人,如果贤能也一样,就用占卜的方法来决定。如今裯并非嫡系的继承人,而且在守丧时心中毫无哀伤的心意,反而有喜悦的神色,如果真的立他为国君,一定会为季氏带来忧患。”

季武子不肯听从,最终还是立他为国君。等到安葬襄公的时候,昭公竟然三次更换丧服。君子说:“这个人是不会得到善终的。”

\begin{yuanwen}
昭公三年,朝晋至河,晋平公谢还之,鲁耻焉。

四年,楚灵王会诸侯于申,昭公称病不往。

七年,季武子卒。

八年,楚灵王就\footnote{建成。}章华台,召昭公。昭公往贺,赐昭公宝器;已而悔,复诈取之。

十二年,朝晋至河,晋平公谢还之。

十三年,楚公子弃疾弑其君灵王,代立。

十五年,朝晋,晋留之葬晋昭公,鲁耻之。

二十年,齐景公与晏子狩竟,因入鲁问礼。

二十一年,朝晋至河,晋谢还之。
\end{yuanwen}

昭公三年(前539年),鲁国国君到黄河岸边朝见晋君,晋平公谢绝并让他返回,鲁国人为此感到耻辱。

四年(前538年),楚灵王在申地会盟诸侯,昭公声称生病没有前往。

七年(前535年),季武子去世。

八年(534年),楚灵王建好章华台,并召见昭公。昭公前往恭贺,楚灵王赏赐给昭公珍贵的宝器;很快便反悔,又把赏赐给昭公的东西骗了回去。

十二年(前530年),昭公到黄河岸边朝见晋国的国君,晋平公辞谢,并再次请他返回。

十三年(前529年),楚公子弃疾杀了他的国君楚灵王,自立为国君。

十五年(前527年),昭公朝见晋国的国君,晋国留下他并请他为晋昭公送葬,鲁国人为此感到十分耻辱。

二十年(前522年),齐景公和晏子在鲁国的边境狩猎,趁机来到鲁国询问礼制。

二十一年(前521年),鲁昭公到黄河岸边朝见晋君,晋君仍然谢绝,请他返回。

\begin{yuanwen}
二十五年春,(鸲/鸜)鹆来巢。师己曰:“文成之世童谣曰‘(鸲/鸜)鹆来巢,公在乾侯。(鸲/鸜)鹆入处,公在外野’。”
\end{yuanwen}

二十五年(前517年)春季,鸜鹆飞到鲁国来筑巢。鲁国的大夫师己说:“文公和成公统治的时候,有童谣说:‘鸜鹆来筑巢,君主在乾侯。鸜鹆来进窝,君主在野外。’”

\begin{yuanwen}
季氏与郈氏斗鸡,季氏芥鸡羽,郈氏金距。季平子怒而侵郈氏,郈昭伯亦怒平子。臧昭伯之弟会伪谗臧氏,匿季氏,臧昭伯囚季氏人。季平子怒,囚臧氏老。臧、郈氏以难告昭公。昭公九月戊戌伐季氏,遂入。

平子登台请曰:“君以谗不察臣罪,诛之,请迁沂上。”

弗许。请囚于鄪,弗许。请以五乘亡,弗许。

子家驹曰:“君其许之。政自季氏久矣,为徒者众,众将合谋。”

弗听。郈氏曰:“必杀之。”

叔孙氏之臣戾谓其众曰:“无季氏与有,孰利?”

皆曰:“无季氏是无叔孙氏。”

戾曰:“然,救季氏!”遂败公师。

孟懿子闻叔孙氏胜,亦杀郈昭伯。郈昭伯为公使,故孟氏得之。三家共伐公,公遂奔。己亥,公至于齐。

齐景公曰:“请致千社待君。”

子家曰:“弃周公之业而臣于齐,可乎?”

乃止。

子家曰:“齐景公无信,不如早之晋。”

弗从。

叔孙见公还,见平子,平子顿首。初欲迎昭公,孟孙、季孙后悔,乃止。
\end{yuanwen}

季氏和郈氏斗鸡,季氏给鸡套上铁甲,郈氏给鸡套上金属制的爪子。季平子因为斗鸡失败十分愤怒,于是侵占了郈氏的田产,郈昭伯也十分怨恨季平子。臧昭伯的弟弟臧会假装诬陷臧氏,并躲到季氏家中,臧昭伯囚禁了季氏家的人。季平子很愤怒,囚禁了臧氏的家臣宰。臧氏、郈氏把两家遭受的灾难告诉了昭公。于是昭公在九月戊戌日讨伐季氏,并侵入他的宅邑。

平子登上高台请求说:“君王因为谗言而不明察我的罪过,前来诛杀我,请允许我将自己流放到沂水边上。”

昭公不允许。又请求把他囚禁到鄪邑,昭公也不允许。再次请求让他带着五辆车逃亡,昭公仍然不允许。

子家驹说:“君王还是同意他吧。鲁国的国政把持在季氏手中已经很长时间了,他们的党羽非常多,那些人会联合起来对付你的。”

昭公并不听从。郈氏说:“一定要杀了他。”

叔孙氏的家臣戾对他的属下说:“没有季氏和有季氏,哪种情况对我们更加有利呢?”

大家都说:“没有季氏,也就没有了叔孙氏。”

戾又说:“这样,我们就去援救季氏吧!”于是打败了昭公的军队。

孟懿子听说叔孙氏取得了胜利,也追杀郈昭伯。郈昭伯作为昭公的使臣被派往孟氏家,因此孟氏抓到了他。三家联合讨伐昭公,昭公于是逃奔国外。己亥日,昭公逃到齐国。

齐景公对昭公说:“愿意奉送一千社以接待国君。”

子家说:“放弃周公的基业而臣服于齐,可以这样做吗?”

于是停止了这种行为。

子家说:“齐景公这个人不讲信用,不如尽早到晋国去。”

昭公并不听从。叔孙氏到齐国看到昭公后返回鲁国,去见平子,平子叩头。起初,他们想迎回昭公,但是孟孙、季孙后悔,于是没有那么做。

\begin{yuanwen}
二十六年春,齐伐鲁,取郓而居昭公焉。

夏,齐景公将内公,令无受鲁赂。申丰、汝贾\footnote{此二人皆为季氏家臣。}许齐臣高龁、子将粟五千庾\footnote{古代容量单位,一庾等于十六斗。}。

子将言于齐侯曰:“群臣不能事鲁君,有异焉。宋元公\footnote{子姓,名佐,宋平公之子。}为鲁如晋,求内之,道卒。叔孙昭子求内其君,无病而死。不知天弃鲁乎?抑鲁君有罪于鬼神也?原君且待。”齐景公从之。
\end{yuanwen}

二十六年(前516年)春季,齐国讨伐鲁国,夺取了郓城并让昭公住在那里。

夏季,齐景公将要护送昭公回国,命令手下的人不可以接受鲁国人的贿赂。鲁国的大夫申丰、汝贾暗中允诺给齐国的大臣高龁、子将五千庾粮食。

子将对齐侯说:“鲁国的群臣不事奉鲁国的君王,是因为鲁国的君主有怪异的地方。宋元公为了鲁国的国君前往晋国,请求晋国护送鲁国国君回国,在途中死掉了。叔孙昭子请求鲁君回到国内,没有生病便死去了。不知道这是不是上天要抛弃鲁国呢?还是鲁国国君得罪了鬼神呢?希望国君暂时等待一下。”齐景公听从了他的建议。

\begin{yuanwen}
二十八年,昭公如晋,求入。季平子私于晋六卿,六卿受季氏赂,谏晋君,晋君乃止,居昭公乾侯。

二十九年,昭公如郓。齐景公使人赐昭公书,自谓“主君”。昭公耻之,怒而去乾侯。

三十一年,晋欲内昭公,召季平子。平子布衣跣行\footnote{赤脚行走。},因六卿谢罪。六卿为言曰:“晋欲内昭公,众不从。”

晋人止。

三十二年,昭公卒于乾侯。鲁人共立昭公弟宋为君,是为定公。
\end{yuanwen}

二十八年(前514年),鲁昭公前往晋国,请求晋国帮助他回国。季平子和晋国六卿有私交,六卿接受了季氏的贿赂,进谏劝阻晋国国君,晋君于是停止护送鲁君回国,而让昭公住在乾侯。

二十九年(前513年),昭公来到郓城。齐景公派人送给昭公一封信,自称“主君”。昭公感到非常耻辱,一怒之下又回到乾侯。

三十一年(前511年),晋国国君想要护送昭公回国,于是召见季平子。季平子穿着麻布衣,赤脚而行,通过六卿向晋君谢罪。六卿在晋君面前替季平子说:“晋国想要护送昭公回国,鲁国的民众不会依从。”

晋国人这才停止护送昭公回国的行动。

三十二年(前510年),昭公在乾侯去世。鲁国人共同拥立昭公的弟弟宋为国君,就是定公。

\begin{yuanwen}
定公立,赵简子问史墨\footnote{蔡墨,晋国史官。}曰:“季氏亡乎?”

史墨对曰:“不亡。季友有大功于鲁,受鄪为上卿,至于文子、武子,世增其业。鲁文公卒,东门遂杀適立庶,鲁君于是失国政。政在季氏,于今四君矣。民不知君,何以得国!是以为君慎器与名,不可以假人。”
\end{yuanwen}

定公继立,赵简子向史墨询问说:“季氏会灭亡吗?”

史墨回答说:“不会灭亡。季友对鲁国有非常大的功劳,被封在鄪邑,成为上卿,一直到文子、武子,累世扩充了基业。鲁文公去世以后,东门就杀掉嫡子拥立庶子为鲁国国君,鲁国国君于是失去国政大权。国政操纵在季氏手中,到如今已经经历了四位国君。人民不知道他们的国君,国君凭借什么掌握国家的行政大权呢!正因为这样,国君要谨慎地对待国家的车服器物和爵位名号,不能轻易地给予别人。”

\begin{yuanwen}
定公五年,季平子卒。阳虎私怒,囚季桓子\footnote{名斯,季平子之子。},与盟,乃舍之。

七年,齐伐我,取郓,以为鲁阳虎邑以从政。

八年,阳虎欲尽杀三桓適,而更立其所善庶子以代之;载季桓子将杀之,桓子诈而得脱。三桓共攻阳虎,阳虎居阳关。

九年,鲁伐阳虎,阳虎奔齐,已而奔晋赵氏。
\end{yuanwen}

定公五年(前505年),季平子去世。阳虎心怀私愤,囚禁了季桓子,与他订立了盟约,才放了他。

七年(前503年),齐军讨伐鲁国,攻取郓邑,将它作为鲁国阳虎的封邑,并让他参与政事。

八年(前502年),阳虎想要杀掉三桓所有的嫡系继承人,而改立和他关系密切的庶子取代嫡系的继承人;阳虎用车载着季桓子,并趁机要杀掉他,桓子用欺骗的手段得以脱身。三桓联合起来攻打阳虎,阳虎占据阳关。

九年(前501年),鲁国人讨伐阳虎,阳虎逃奔齐国,不久又前去投奔晋国的赵氏。

\begin{yuanwen}
十年,定公与齐景公会于夹谷,孔子行相事。齐欲袭鲁君,孔子以礼历阶,诛齐淫乐,齐侯惧,乃止,归鲁侵地而谢过。

十二年,使仲由毁三桓城,收其甲兵。孟氏不肯堕城,伐之,不克而止。季桓子受齐女乐,孔子去。
\end{yuanwen}

十年(前500年),鲁定公和齐景公在夹谷举行盟会,孔子跟随鲁定公作傧相。齐国人想要偷袭鲁国国君,孔子根据礼仪,一步一阶地快步走上台阶,诛杀了齐国演奏淫靡乐曲的乐师,齐侯畏惧,这才停止了行动,并且归还了侵占的鲁国的土地,向鲁国谢罪。

十二年(前498年),定公派仲由拆毁了三桓都邑的城墙,收缴了他们的武器。孟氏不愿意毁掉城墙,定公派兵讨伐,没有攻克,只好收兵。季桓子接受了齐国赠送的女乐队,孔子便离开了鲁国。

\begin{yuanwen}
十五年,定公卒,子将立,是为哀公。\footnote{哀公以下鲁史纪年历来说法不一。}
\end{yuanwen}

十五年(前495年),定公去世,他的儿子将继立,就是哀公。

\begin{yuanwen}
哀公五年,齐景公卒。

六年,齐田乞弑其君孺子。

七年,吴王夫差彊,伐齐,至缯,徵百牢于鲁。季康子使子贡说吴王及太宰嚭,以礼诎\footnote{通“黜”,驳斥。}之。

吴王曰:“我文身,不足责礼。”乃止。

八年,吴为邹伐鲁,至城下,盟而去。齐伐我,取三邑。

十年,伐齐南边。

十一年,齐伐鲁。季氏用(厓/冉)有有功,思孔子,孔子自卫归鲁。

十四年,齐田常弑其君简公于(袪/俆)州。孔子请伐之,哀公不听。

十五年,使子服景伯、子贡为介\footnote{助手。},(適/适)齐,齐归我侵地。田常初相,欲亲诸侯。

十六年,孔子卒。

二十二年,越王句践灭吴王夫差。

二十七年春,季康子卒。夏,哀公患三桓,将欲因诸侯以劫之,三桓亦患公作难,故君臣多间。公游于陵阪,遇孟武伯于街,曰:“请问余及死乎?”

对曰:“不知也。”

公欲以越伐三桓。

八月,哀公如陉氏。三桓攻公,公奔于卫,去如邹,遂如越。国人迎哀公复归,卒于有山氏。子宁立,是为悼公。
\end{yuanwen}

哀公五年(前490年),齐景公去世。

六年(前489年),齐国的田乞杀了他的国君孺子。

七年(前488年),吴王夫差的势力很强大,派兵讨伐齐国,到了缯地,向鲁国人征发牛、羊、猪各一百头。季康子派子贡前去游说吴王和太宰嚭,依照礼仪驳斥他们。吴王说:“我是断发文身的蛮夷人,不要用中原的礼仪来斥责我。”于是停止对鲁国人的索取。

八年(前487年),吴国为了邹国而讨伐鲁国,一直打到鲁国的城下,双方订立盟约以后离去。齐国讨伐鲁国,攻取了三座城邑。

十年(前485年),鲁国攻打齐国南部的边境。

十一年(前484年),齐国讨伐鲁国。季氏任用冉有建立功勋,想到了孔子,孔子从卫国回到鲁国。

十四年(前481年),齐国的田常在徐州杀了他的国君齐简公。孔子请求鲁国国君发兵前去征讨,鲁哀公没有听从孔子的建议。

十五年(前480年),让子服景伯担任使臣,子贡担任他的助手,前往齐国,齐国归还了侵占的鲁国的土地。田常刚刚成为齐相,想要和诸侯亲近。

十六年(前479年),孔子去世。

二十二年(前473年),越王句践消灭了吴王夫差。

二十七年(前468年)春季,季康子去世。夏季,鲁哀公认为三桓是鲁国的忧患,想要借助诸侯的力量剥夺三桓的权利,三桓也担心哀公会发难,因此君臣间产生了许多隔阂。哀公到陵阪出游,在街上遇见孟武伯,哀公说:“请问我就要死了吗?”

孟武伯回答说:“不知道啊。”

哀公想要借助越人的力量讨伐三桓。

八月,哀公来到陉氏。三桓攻打哀公,哀公逃奔到卫国,又逃到邹国,接着又前往越国。鲁国人又迎接哀公回到鲁国,哀公最后死在有山氏的家里。他的儿子宁即位,就是悼公。

\begin{yuanwen}
悼公之时,三桓胜,鲁如小侯,卑于三桓之家。
\end{yuanwen}

悼公在位时期,三桓的势力占优,鲁国的国君就像一个很小的诸侯,地位低于三桓家族。

\begin{yuanwen}
十三年,三晋灭智伯,分其地有之。
\end{yuanwen}

十三年(前453年),晋国的韩、赵、魏三家消灭了智伯,瓜分了他的土地并占有了晋国。

\begin{yuanwen}
三十七年,悼公卒,子嘉立,是为元公。元公二十一年卒,子显立,是为穆公。穆公三十三年卒,子奋立,是为共公。共公二十二年卒,子屯立,是为康公。康公九年卒,子匽立,是为景公。景公二十九年卒,子叔立,是为平公。是时六国皆称王\footnote{称王的六国为魏、赵、韩、楚、燕、齐。}。
\end{yuanwen}

三十七年(前429年),悼公去世,他的儿子嘉继立,就是元公。元公在位二十一年去世,他的儿子显继立,就是穆公。穆公在位三十三年去世,他的儿子奋继立,就是共公。共公在位二十二年去世,他的儿子屯继立,就是康公。康公在位九年去世,他的儿子匽继立,就是景公。景公在位二十九年去世,他的儿子叔继立,就是平公。这时候六国的国君都已经称王。

\begin{yuanwen}
平公十二年,秦惠王卒。

二十年,平公卒,子贾立,是为文公。

文公元年,楚怀王死于秦。

二十三年,文公卒,子雠立,是为顷公。

顷公二年,秦拔楚之郢,楚顷王东徙于陈。

十九年,楚伐我,取徐州。

二十四年,楚考烈王\footnote{名元,顷襄王之子。}伐灭鲁。顷公亡,迁于下邑,为家人,鲁绝祀。顷公卒于柯。

鲁起周公至顷公,凡三十四世。
\end{yuanwen}

平公十二年(前305年),秦惠王去世。

二十年,平公去世,他的儿子贾即位,就是文公。

文公元年,楚怀王死在秦国。

二十三年,文公去世,他的儿子雠即位,就是顷公。

顷公二年(前271年),秦国拔取楚国的都城郢,楚顷王向东迁都于陈。

十九年(前254年),楚国攻伐鲁国,占领了徐州。

二十四年(前249年),楚考烈王攻打并灭掉鲁国。顷公逃亡,迁居至下邑,成为平民,鲁国的宗庙祭祀至此断绝。顷公死在柯地。

鲁国从周公开始一直到顷公,一共传了三十四代。

\begin{yuanwen}
太史公曰:

余闻孔子称曰“甚矣鲁道之衰也!洙泗之间龂龂\footnote{yín,争辩。}如也”。观庆父及叔牙闵公之际,何其乱也?隐桓之事;襄仲杀適立庶;三家北面为臣,亲攻昭公,昭公以奔。至其揖让之礼则从矣,而行事何其戾\footnote{相反。}也?
\end{yuanwen}

太史公说:

我听说孔子说过“鲁国的礼仪之道衰败得真是太严重了!洙水和泗水之间不停地争吵”。察看庆父和叔牙在湣公时期的行为,是多么混乱啊。隐公和桓公期间的事;襄仲杀嫡立庶;三桓向北称臣,亲自攻打昭公,昭公因此出逃。以至于他们对于传统的揖让礼仪虽然还在遵从,但是做事的时候又是何其的相反呢?

\begin{yuanwen}
武王既没,成王幼孤。周公摄政,负扆据图。及还臣列,北面歔如。元子封鲁,少昊之墟。夹辅王室,系职不渝。降及孝王,穆仲致誉。隐能让国,春秋之初。丘明执简,襃贬备书。
\end{yuanwen}

\part{卷三十四}
\chapter{燕召公世家第四}

\begin{yuanwen}
召\footnote{shào}公奭与周同姓,姓姬氏。周武王之灭纣,封召公于北燕。
\end{yuanwen}

召公奭与周王室同姓,姓姬。周武王伐灭商纣王之后,将召公封在了北燕。

\begin{yuanwen}
其在成王时,召公为三公\footnote{指太师、太傅、太保。召公在周成王时担任太保。}:自陕以西,召公主之;自陕以东,周公主之。成王既幼,周公摄政,当国践祚,召公疑之,作《君奭》。《君奭》不说周公。周公乃称“汤时有伊尹,假于皇天;在太戊时,则有若伊陟、臣扈,假\footnote{《尚书》作“格”,至。}于上帝,巫咸治王家;在祖乙时,则有若巫贤;在武丁时,则有若甘般:率维兹有陈,保乂有殷”。于是召公乃说。
\end{yuanwen}

周成王在位的时候,召公位列三公:自陕县以西,都由召公管理;自陕县以东,由周公治理。成王年纪还小,周公代替他主持朝政,登基行使君王的权力,召公怀疑周公,周公便作《君奭》。《君奭》里表达了召公对周公的不满。周公于是声称“商汤的时候有伊尹,伊尹对国家的治理合乎天道;太戊在位时期,又有像伊陟、臣扈那样的人,他们对国家的治理合乎上帝的旨意,还有巫咸辅佐君王治理王室;祖乙在位期间,有像巫贤那样的人;在武丁的时代,有像甘般那样的人:这些贤臣一般都各在其位,殷朝得到安定与治理”。于是召公才感到高兴。

\begin{yuanwen}
召公之治西方,甚得兆民和。召公巡行乡邑,有棠树,决狱政事其下,自侯伯至庶人各得其所,无失职者。召公卒,而民人思召公之政,怀棠树不敢伐,哥咏之,作《甘棠》之诗。
\end{yuanwen}

召公治理之处在西方,深得人民的爱戴。召公到乡邑去巡视,看到一棵棠梨树,就在那棵树下审判官司、处理政事,从侯爵、伯爵一直到平民百姓都找到各自的位置,没有失职的地方。召公去世以后,人民思念召公时期的政绩,思念棠梨树下召公处理政事的时光,不敢砍伐,并歌颂召公,写下了《甘棠》这首诗篇。

\begin{yuanwen}
自召公已\footnote{通“以”。}下九世至惠侯。燕惠侯当周厉王奔彘,共和之时。

惠侯卒,子釐侯立。是岁,周宣王初即位。

釐侯二十一年,郑桓公初封于郑。

三十六年,釐侯卒,子顷侯立。

顷侯二十年,周幽王淫乱,为犬戎所弑。秦始列为诸侯。

二十四年,顷侯卒,子哀侯立。哀侯二年卒,子郑侯立。郑侯三十六年卒,子缪侯立。

缪侯七年,而鲁隐公元年也。十八年卒,子宣侯立。宣侯十三年卒,子桓侯立。桓侯七年卒,子庄公立。
\end{yuanwen}

自召公以下传到第九代就到了惠侯。燕惠侯处于周厉王逃奔到彘地、共和行政的时代。

惠侯去世以后,他的儿子釐侯继立。这一年,周宣王刚刚继位。

釐侯二十一年,郑桓公刚刚受封到郑国。

三十六年,釐侯去世,他的儿子顷侯继立。

燕顷侯二十年(前771年),周幽王淫乱,被犬戎部族杀掉。秦国开始被列为诸侯。

二十四年(前767年),顷候去世,他的儿子哀侯继立。哀侯在位二年就去世了,他的儿子郑侯继立。郑侯在位三十六年去世,他的儿子缪侯继立。

燕缪侯七年(前722年),就是鲁隐公元年。燕缪侯在位十八年去世,他的儿子宣侯继立。宣侯在位十三年去世,他的儿子桓侯继立。桓侯在位七年去世,他的儿子庄公继立。

\begin{yuanwen}
庄公十二年,齐桓公始霸。

十六年,与宋、卫共伐周惠王,惠王出奔温,立惠王弟穨为周王。

十七年,郑执燕仲父而内\footnote{通“纳”。}惠王于周。

二十七年,山戎来侵我,齐桓公救燕,遂北伐山戎而还。燕君送齐桓公出境,桓公因割燕所至地予燕,使燕共贡天子,如成周时职;使燕复修召公之法。三十三年卒,子襄公立。
\end{yuanwen}

燕庄公十二年(前679年),齐桓公开始在诸侯中称霸。

十六年(前675年),燕国和宋国、卫国一起讨伐周惠王,惠王逃奔到温地,三国拥立惠王的弟弟穨为周王。

十七年(前674年),郑国抓住燕仲父,并护送周惠王返回周国。

二十七年(前664年),山戎部族前来侵犯燕国,齐桓公派兵救援燕国,于是向北讨伐山戎以后率军返回。燕庄公一直把他送出了燕国的边境,齐桓公便把燕庄公到达的地方割给了燕国,让燕庄公和诸侯一起向周王进贡,就像周成王时代那样履行职责;并让燕国重新修整召公时期的法度。庄公在位三十三年去世,他的儿子襄公继立。

\begin{yuanwen}
襄公二十六年,晋文公为践土之会,称伯。

三十一年,秦师败于殽。

三十七年,秦穆公卒。

四十年,襄公卒,桓公立。
\end{yuanwen}

燕襄公二十六年(前632年),晋文公召集诸侯在践土会盟,在诸侯中称霸。

三十一年(前627年),秦军在殽山被晋军打败。

三十七年(前621年),秦穆公去世。

四十年(前618年),燕襄公去世,桓公即位。

\begin{yuanwen}
桓公十六年卒,宣公立。宣公十五年卒,昭公立。昭公十三年卒,武公立。是岁晋灭三郤\footnote{指卻锜、卻犨、卻至。}大夫。
\end{yuanwen}

燕桓公在位十六年去世,宣公即位。宣公在位十五年去世,昭公即位。昭公在位十三年去世,武公即位。这一年,晋国灭掉了三郤大夫。

\begin{yuanwen}
武公十九年卒,文公立。文公六年卒,懿公立。懿公元年,齐崔杼弑其君庄公。四年卒,子惠公立。
\end{yuanwen}

燕武公在位十九年去世,文公即位。文公在位六年去世,懿公即位。懿公元年(前548年),齐国的崔杼杀死他的国君庄公。懿公在位四年去世,他的儿子惠公即位。

\begin{yuanwen}
惠公元年,齐高止来奔。

六年,惠公多宠姬,公欲去诸大夫而立宠姬宋,大夫共诛姬宋,惠公惧,奔齐。

四年,齐高偃如晋,请共伐燕,入其君。晋平公许,与齐伐燕,入惠公。惠公至燕而死。燕立悼公。
\end{yuanwen}

燕惠公元年(前544年),齐国的高止前来投奔燕国。

六年(前539年),惠公有很多宠爱的姬妾,他想要除掉各位大夫从而立一位名叫宋的宠姬,各位大夫联手杀掉了名叫宋的宠姬,惠公心中畏惧,逃奔齐国。

四年以后,齐国派高偃前往晋国,请求共同讨伐燕国,护送惠公返回燕国。晋平公同意了,和齐国一起征讨燕国,护送燕惠公回到燕国。惠公回到燕国以后就死了。燕国人拥立悼公为国君。

\begin{yuanwen}
悼公七年卒,共公立。共公五年卒,平公立。晋公室卑,六卿始彊大。

平公十八年,吴王阖闾破楚入郢。十九年卒,简公立。简公十二年卒,献公立。晋赵鞅围范、中行于朝歌。

献公十二年,齐田常弑其君简公。

十四年,孔子卒。

二十八年,献公卒,孝公立。

\end{yuanwen}

燕悼公在位七年去世,共公即位。共公在位五年去世,平公即位。晋国公室权力日渐卑微,六卿的权力开始强大。

平公十八年(前506年),吴王阖闾攻破楚国,进入郢都。平公在位十九年去世,简公即位。简公在位十二年去世,献公即位。晋国的赵鞅在朝歌包围了范氏、中行氏。

献公十二年(前481年),齐国的田常杀死了他的国君简公。

十四年(前479年),孔子去世。

二十八年(前465年),献公去世,孝公即位。

\begin{yuanwen}
孝公十二年,韩、魏、赵灭知伯,分其地,三晋彊。

十五年,孝公卒,成公立。成公十六年卒,湣公立。湣公三十一年卒,釐公立。是岁,三晋列为诸侯。
\end{yuanwen}

燕孝公十二年(前453年),晋国的韩、魏、赵灭掉了智伯,瓜分了他的土地,三晋从此强大。

十五年(前450年),孝公去世,成公即位。成公在位十六年去世,湣公即位。湣公在位三十一年去世,釐公即位。这一年,韩、赵、魏三晋开始列为诸侯。

\begin{yuanwen}
釐公三十年,伐败齐于林营。釐公卒,桓公立。桓公十一年卒,文公立。是岁,秦献公卒。秦益彊。
\end{yuanwen}

燕釐公三十年,在林营打败了齐国。釐公去世,桓公即位。桓公在位十一年去世,文公即位。这一年,秦献公去世。秦国越发强大。

\begin{yuanwen}
文公十九年,齐威王卒。

二十八年,苏秦始来见,说文公。文公予车马金帛以至赵,赵肃侯用之。因约六国,为从\footnote{通“纵”。}长。秦惠王以其女为燕太子妇。
\end{yuanwen}

燕文公十九年(前343年),齐威王去世。

二十八年(前334年),苏秦第一次来到燕国拜见文公,游说文公。文公给他车马、黄金、布帛,并让他前往赵国,赵肃侯重用了苏秦。于是苏秦联合六国实行合纵政策,他成为纵长。秦惠王把他的女儿嫁给燕国太子为妻。

\begin{yuanwen}
二十九年,文公卒,太子立,是为易王。
\end{yuanwen}

二十九年(前333年),文公去世,太子即位,就是易王。

\begin{yuanwen}
易王初立,齐宣王因燕丧伐我,取十城;苏秦说齐,使复归燕十城。

十年,燕君为王。苏秦与燕文公夫人私通,惧诛,乃说王使齐为反间,欲以乱齐。易王立十二年卒,子燕哙立。
\end{yuanwen}

燕易王刚刚即位,齐宣王趁燕国办丧事的时机攻打燕国,夺取了十座城池;苏秦劝说齐宣王,让齐国归还了占领的燕国的十座城市。

十年(前323年),燕君称王。苏秦和燕文公的夫人通奸,他担心被诛杀,就劝易王派他出使齐国施行反间计,想用这个办法扰乱齐国。易王在位十二年去世,他的儿子燕王哙即位。

\begin{yuanwen}
燕哙既立,齐人杀苏秦。苏秦之在燕,与其相子之为婚,而苏代与子之交。及苏秦死,而齐宣王复用苏代。

燕哙三年,与楚、三晋攻秦,不胜而还。子之相燕,贵重,主断。苏代为齐使于燕,燕王问曰:“齐王奚如?”

对曰:“必不霸。”

燕王曰:“何也?”

对曰:“不信其臣。”

苏代欲以激燕王以尊子之也。于是燕王大信子之。子之因遗苏代百金,而听其所使。
\end{yuanwen}

燕王哙已经继位,齐国人杀了苏秦。苏秦在燕国时,和燕相子之结为姻亲,因而苏秦的弟弟苏代和子之有交往。等到苏秦死了以后,齐宣王又重用了苏代。

燕王哙三年(前318年),燕国和楚、三晋一起攻打秦国,没能取胜便返回了。子之在燕国做丞相,他位尊权重,决断国事。苏代为了齐国出使燕国,燕王哙问他说:“齐王怎么样?”

苏代回答说:“一定不会称霸。”

燕王哙说:“为什么呢?”

苏代回答说:“不能信任他的大臣。”

苏代是想用这样的办法刺激燕王尊重子之。于是燕王非常信任子之。子之因此赠送给苏代一百镒黄金,听凭他使用。

\begin{yuanwen}
鹿毛寿谓燕王:“不如以国让相子之。人之谓尧贤者,以其让天下于许由,许由不受,有让天下之名而实不失天下。今王以国让于子之,子之必不敢受,是王与尧同行也。”

燕王因属国于子之,子之大重。或曰:“禹荐益,已而以启人为吏。及老,而以启人为不足任乎天下,传之于益。已而启与交党攻益,夺之。天下谓禹名传天下于益,已而实令启自取之。今王言属国于子之,而吏无非太子人者,是名属子之而实太子用事也。”

王因收印自三百石吏已(讼/上)而效\footnote{授予。}之子之。子之南面行王事,而哙老不听政,顾\footnote{反而。}为臣,国事皆决于子之。
\end{yuanwen}

鹿毛寿对燕王说:“不如将国家让给丞相子之吧。人们都说尧是贤能的人,是因为他把天下让给许由,许由没有接受,尧拥有禅让天下的美名,实际上也没有失掉天下。如今大王把国家禅让给子之,子之一定不敢接受,这样大王就拥有与尧相同的品德了。”

燕王因此将国家托付给子之,子之的地位就更加尊贵。有人说:“夏禹举荐了益,不久又让启的臣下担任官吏。等到大禹年老的时候,他认为启的臣下没有能力担当治理天下的重任,于是将君位传给了益。不久启和他结交的党徒攻打益,夺得了国君的位置。天下人都说夏禹名义上是把国君的位置传给了益,而实际上是让启自己去夺回。如今大王说是将国家托付给子之,而官吏中没有一个不是太子的亲信,这正是名义上把国家托付给子之,而实际却是让太子掌握权力。”

燕王于是把俸禄在三百石以上的官吏的印信都收缴上来交给子之。子之面向南坐在君位上,行使君王的权力,而燕王哙到了老年的时候就不再处理国家的政事,反而成为臣子,国家的事务完全由子之裁决。

\begin{yuanwen}
三年,国大乱,百姓恫恐\footnote{恐惧。恫,dòng。}。将军市被与太子平谋,将攻子之。诸将谓齐湣王曰:“因而赴之,破燕必矣。”

齐王因令人谓燕太子平曰:“寡人闻太子之义,将废私而立公,饬君臣之义,明父子之位。寡人之国小,不足以为先后。虽然,则唯太子所以令之。”

太子因要党聚众,将军市被围公宫,攻子之,不克。将军市被及百姓反攻太子平,将军市被死,以徇\footnote{示众。}。因搆\footnote{gòu}难数月,死者数万,众人恫恐,百姓离志。

孟轲谓齐王曰:“今伐燕,此文、武之时,不可失也。”

王因令章子将五都之兵,以因北地之众以伐燕。士卒不战,城门不闭,燕君哙死,齐大胜。燕子之亡二年,而燕人共立太子平,是为燕昭王。
\end{yuanwen}

三年以后,燕国大乱,百姓都很恐惧。将军市被和太子平一起谋划,将要攻打子之。齐国的诸位将领对齐湣王说:“我们现在趁燕国发生内乱的时机派军队赶去攻打它,燕国一定会被攻破。”

齐王因而派人对燕太子平说:“我听说太子即将有义举,打算废弃私情,建立公义,整顿君臣之间的关系,辨明父子之间的地位。我的国家非常小,不配跟随在您的左右。虽然是这样,我也愿意听从太子的命令而行事。”

太子于是召集同党、聚合众人。将军市被包围了王宫,攻打子之,没有攻克。将军市被和百姓反过来攻打太子平,将军市被战死,尸体示众。因此造成了燕国国内几个月的动乱,死亡的人有数万,民众都非常恐惧,百姓离心离德。

孟轲对齐王说:“如今我们讨伐燕国,正是周文王、周武王成就事业那样的机遇,不可以错失啊。”

齐王于是派章子率领五都的士兵,凭借北部边境的军队前去攻打燕国。燕国的士兵不出来迎战,城门也不关闭,燕王哙死去,齐国的军队大获全胜。燕国子之死后第二年,燕国人共同拥立了太子平,就是燕昭王。

\begin{yuanwen}
燕昭王于破燕之后即位,卑身厚币以招贤者。谓郭隗曰:“齐因孤之国乱而袭破燕,孤极知燕小力少,不足以报。然诚得贤士以共国,以雪先王之耻,孤之原也。先生视可者,得身事之。”

郭隗曰:“王必欲致士,先从隗始。况贤于隗者,岂远千里哉!”

于是昭王为隗改筑宫而师事之。乐毅自魏往,邹衍自齐往,剧辛自赵往,士争趋燕。燕王吊死问孤,与百姓同甘苦。
\end{yuanwen}

燕昭王在燕国被攻破以后即位,他放低自身并用优厚的待遇招揽贤能的人。

燕昭王对郭隗说:“齐国趁我的国家发生内乱而偷袭并攻破燕国,我非常了解燕国很小、力量薄弱,没有足够的能力报仇。然而如果真的能够得到贤能的人一起治理国家,洗雪先王的耻辱,就是我的心愿。先生如果发现可以一起治理国家的贤人,我一定要亲自侍奉他。”

郭隗说:“大王您一定要招揽贤人,那就先从我郭隗开始吧。况且比我更贤能的人,难道还会嫌路途有千里那么遥远吗!”

于是昭王为郭隗改建了居住的房屋,像对待老师那样侍奉他。乐毅就从魏国前往燕国,邹衍从齐国来到燕国,剧辛从赵国来到燕国,贤士们争着来到燕国。燕昭王吊唁死者,慰问孤儿,和百姓们同甘共苦。

\begin{yuanwen}
二十八年,燕国殷富,士卒乐轶轻战,于是遂以乐毅为上将军,与秦、楚、三晋合谋以伐齐。齐兵败,湣王出亡于外。燕兵独追北,入至临淄,尽取齐宝,烧其宫室宗庙。齐城之不下者,独唯聊、莒、即墨,其馀皆属燕,六岁。

昭王三十三年卒,子惠王立。
\end{yuanwen}

二十八年(前284年),燕国变得殷实和富裕了,士兵也快乐安逸,不怕战争,于是燕国国君就任命乐毅为上将军,与秦国、楚国、韩国、赵国、魏国一起谋划征伐齐国。齐兵战败,齐湣王逃亡到国外。燕军单独追击败逃的士兵,进入齐国的都城临淄,将齐国所有的宝物都掠夺殆尽,放火烧毁了齐国的王宫和宗庙。齐国没有被燕军攻占的城市,只有聊城、莒城、即墨,其余全部归属于燕国,时间长达六年。

燕昭王在位三十三年去世,他的儿子惠王即位。

\begin{yuanwen}
惠王为太子时,与乐毅有隙;及即位,疑毅,使骑劫代将。乐毅亡走赵。齐田单以即墨击败燕军,骑劫死,燕兵引归,齐悉复得其故城。湣王死于莒,乃立其子为襄王。

惠王七年卒。韩、魏、楚共伐燕。燕武成王立。
\end{yuanwen}

燕惠王还是太子的时候,就和乐毅有嫌隙;等到他即位后,便怀疑乐毅,让骑劫代替乐毅为将。乐毅出逃前往赵国。齐国的田单凭借即墨的兵力打败了燕军,骑劫战死,燕军撤回国内,齐国悉数收复了它原有的城池。齐湣王死在莒城,于是齐国人立湣王的儿子为襄王。

惠王在位七年去世,韩国、魏国、楚国共同攻讨伐燕国。燕武成王即位。

\begin{yuanwen}
武成王七年,齐田单伐我,拔中阳。

十三年,秦败赵于长平四十馀万。

十四年,武成王卒,子孝王立。

孝王元年,秦围邯郸者解去。三年卒,子今王\footnote{当今的国王。此处司马迁延用了燕国旧史原文的称呼。}喜立。
\end{yuanwen}

燕武成王七年(前265年),齐国的田单讨伐燕国,拔取中阳。

十三年(前259年),秦国在长平打败赵国四十多万大军。

十四年(前258年),武成王去世,他的儿子孝王即位。

孝王元年(前257年),秦国撤去包围邯郸的军队。孝王在位三年去世,他的儿子当今的君王喜即位。

\begin{yuanwen}
今王喜四年,秦昭王卒。燕王命相栗腹约欢赵,以五百金为赵王酒。

还报燕王曰:“赵王壮者皆死长平,其孤未壮,可伐也。”

王召昌国君乐间问之。对曰:“赵四战之国,其民习兵,不可伐。”

王曰:“吾以五而伐一。”

对曰:“不可。”

燕王怒,群臣皆以为可。卒起二军,车二千乘,栗腹将而攻鄗\footnote{hào},卿秦\footnote{又作庆秦,燕国将领。}攻代。唯独大夫将渠谓燕王曰:“与人通关约交,以五百金饮人之王,使者报而反攻之,不祥,兵无成功。”

燕王不听,自将偏军随之。将渠引燕王绶止之曰:“王必无自往,往无成功。”

王(槅/蹴)之以足。将渠泣曰:“臣非以自为,为王也!”

燕军至宋子,赵使廉颇将,击破栗腹于鄗。(乐乘)破卿秦于代。乐(间/闲)奔赵。廉颇逐之五百馀里,围其国\footnote{指国都。}。燕人请和,赵人不许,必令将渠处和。燕相将渠以处和。赵听将渠,解燕围。
\end{yuanwen}

当今的君王喜四年(前251年),秦昭王去世。燕王命丞相栗腹到赵国去订立友好盟约,栗腹带着五百镒黄金为赵王置酒祝寿。

栗腹回来向燕王报告说:“赵王国中的青壮年都死在了长平之战中,剩下的孤儿还没有成人,我们可以趁机讨伐。”

燕王召见昌国君乐闲向他询问这件事。乐闲回答说:“赵国是四面受敌的国家,国中的百姓都熟悉军事,不可以攻打它。”

燕王说:“我用五个人攻打一个人。”

乐闲说:“不可以。”

燕王大怒,群臣都认为可以攻打赵国。燕王最终调遣两军士兵,二千乘战车,派栗腹领兵攻打鄗邑,派卿秦领兵攻打代地。只有大夫将渠对燕王说:“和人家通好关卡,定好盟约,还用五百镒黄金为人家的君王祝寿,使者回来报告就反过来去攻打,这种做法是不吉利的,进兵不会成功。”

燕王没有听从劝告,亲自率领偏军跟随。将渠拉住燕王的印带阻止他说:“大王一定不要亲自前往,去了也不会获得成功。”

燕王用脚踢开他。将渠哭着说:“臣下并不是为了自己,而是为了大王啊!”

燕军到达宋子,赵国派廉颇统率军队,在鄗邑打败了栗腹。乐乘在代地打败了卿秦。乐闲逃奔到赵国。廉颇一直追赶燕军有五百多里,包围了燕国都城。燕国人请求讲和,赵国人不同意,一定要将渠出面主持议和。燕王于是让将渠担任丞相,主持和议的事情。赵国接受了将渠的请求,解除了对燕国的包围。

\begin{yuanwen}
六年,秦灭东周,置三川郡。

七年,秦拔赵榆次三十七城,秦置大原郡。

九年,秦王政初即位。

十年,赵使廉颇将攻繁阳,拔之。赵孝成王卒,悼襄王立。使乐乘代廉颇,廉颇不听,攻乐乘,乐乘走,廉颇奔大梁。
\end{yuanwen}

六年(前249年),秦国灭掉东周,设置了三川郡。

七年(前248年),秦军攻拔了赵国的榆次等三十七座城,秦国设置了太原郡。

九年(前246年),秦王嬴政刚刚即位。

十年(前245年),赵国派廉颇领兵攻打繁阳,攻克。赵孝成王去世,悼襄王即位。悼襄王派乐乘代替廉颇,廉颇不听从命令,攻打乐乘,乐乘败走,廉颇逃到魏国的大梁。

\begin{yuanwen}
十二年,赵使李牧攻燕,拔武遂、方城。剧辛故居赵,与庞(暖/煖)\footnote{赵国将领。煖,xuān}善,已而亡走燕。燕见赵数困于秦,而廉颇去,令庞(暖/煖)将也,欲因赵(弊/獘)攻之。

问剧辛,辛曰:“庞(暖/煖)易与耳。”

燕使剧辛将击赵,赵使庞(暖/煖)击之,取\footnote{俘获。}燕军二万,杀剧辛。秦拔魏二十城,置东郡。

十九年,秦拔赵之(鄴/邺)九城。赵悼襄王卒。

二十三年,太子丹质于秦,亡归燕。

二十五年,秦虏灭韩王安,置颍川郡。

二十七年,秦虏赵王迁,灭赵。赵公子嘉自立为代王。
\end{yuanwen}

十二年(前243年),赵国派李牧攻打燕国,拔取了武遂、方城。剧辛从前曾居住在赵国,他和庞煖的关系非常好,后来他逃到燕国。燕王看到赵国屡次被秦国围困,并且廉颇也已经离开,就让庞煖统率军队,想趁赵国处于困境的时候攻打它。

燕王询问剧辛,剧辛说:“庞煖容易对付。”

燕王派剧辛带兵攻打赵国,赵国派庞煖迎击,俘获了二万燕军,斩杀了剧辛。秦国拔取魏国的二十座城池,设置了东郡。

十九年(前236年),秦国拔取赵国的邺城等九座城池。赵悼襄王去世。

二十三年(前232年),燕太子丹被送到秦国做人质,他逃回燕国。

二十五年(前230年),秦军俘虏了韩王安,灭掉韩国,设置颍川郡。

二十七年(前228年),秦军俘虏赵王迁,灭掉赵国。赵国的公子嘉自立为代王。

\begin{yuanwen}
燕见秦且灭六国,秦兵临易水,祸且至燕。太子丹阴养壮士二十人,使荆轲献督亢地图于秦,因袭刺秦王。秦王觉,杀轲,使将军王翦击燕。

二十九年,秦攻拔我蓟,燕王亡,徙居辽东,斩丹以献秦。

三十年,秦灭魏。
\end{yuanwen}

燕国看到秦国即将灭亡六国,秦军已经到达易水,灾祸就要降临到燕国。太子丹暗中培养了二十名壮士,派荆轲到秦国进献督亢的地图,趁机刺杀秦王。秦王察觉,杀掉了荆轲,派将军王翦攻打燕国。

二十九年(前226年),秦军攻打并占领了燕国都城蓟城,燕王逃走,迁居至辽东,斩杀了太子丹,并把他的首级献给秦国。

三十年(前225年),秦国灭掉魏国。

\begin{yuanwen}
三十三年,秦拔辽东,虏燕王喜,卒灭燕。是岁,秦将王贲亦虏代王嘉。
\end{yuanwen}

三十三年(前222年),秦军拔取辽东,俘虏了燕王喜,最终灭掉燕国。这一年,秦国将领王贲也俘获了代王嘉。

\begin{yuanwen}
太史公曰:召公奭可谓仁矣!甘棠且思之,况其人乎?燕(外)迫蛮貉\footnote{mò},内措齐、晋,崎岖\footnote{喻处境艰难。}彊国之间,最为弱小,几灭者数矣。然社稷血食者八九百岁,于姬姓独后亡\footnote{按:姬姓诸侯国中最后灭亡的是卫国,不是燕国。},岂非召公之烈邪!	
\end{yuanwen}

太史公说:召公奭可以说很仁义了!那棵棠梨树被人民怀念,更何况是召公本人呢?燕国在外遭受蛮貉的逼迫,在内受到齐、晋大国的夹击,在强国之间艰难生存,势力最弱小,多次几乎被灭亡。然而国家存在了八九百年的时间,在姬姓诸侯中最后灭亡,难道不是召公创下的功绩吗!

\begin{yuanwen}
召伯作相,分陕而治。人惠其德,甘棠是思。庄送霸主,惠罗宠姬。文公从赵,苏秦骋辞。易王初立,齐宣我欺。燕哙无道,禅位子之。昭王待士,思报临菑。督亢不就,卒见芟夷。
\end{yuanwen}

\part{卷三十五}
\chapter{管蔡世家第五}

\begin{yuanwen}
管叔鲜、蔡叔度者,周文王子而武王弟也。武王同母兄弟十人。母曰太姒,文王正妃也。其长子曰伯邑考,次曰武王发,次曰管叔鲜,次曰周公旦,次曰蔡叔度,次曰曹叔振铎,次曰成叔武,次曰霍叔处,次曰康叔封,次曰(厓/冉)季载。(厓/冉)季载最少。同母昆弟十人,唯发、旦贤,左右辅文王,故文王舍伯邑考而以发为太子。及文王崩而发立,是为武王。伯邑考既已前卒矣。
\end{yuanwen}

管叔鲜和蔡叔度,都是周文王的儿子,周武王的弟弟。周武王的同母兄弟一共有十个人。母亲名叫太姒,是文王的正妻。周文王的大儿子叫伯邑考,二儿子叫武王发,三儿子叫管叔鲜,四儿子叫周公旦,五儿子叫蔡叔度,六儿子叫曹叔振铎,七儿子叫成叔武,八儿子叫霍叔处,九儿子叫康叔封,十儿子叫冉季载。冉季载年纪最小。同母的兄弟十个人,只有武王发和周公旦很贤能,在文王的左右辅佐文王,因此文王舍弃伯邑考而立次子发为太子。等到文王驾崩而太子发继立,就是武王。伯邑考已经在这之前去世。

\begin{yuanwen}
武王已克殷纣,平天下,封功臣昆弟。于是封叔鲜于管,封叔度于蔡:二人相纣子武庚禄父\footnote{纣王之子,名禄父,武庚是谥号。},治殷遗民。封叔旦于鲁而相周,为周公。封叔振铎于曹,封叔武于成,封叔处于霍。康叔封、(厓/冉)季载皆少,未得封。
\end{yuanwen}

武王战胜殷纣王,平定天下以后,封赏了功臣和兄弟。于是将管地封给叔鲜,把蔡地封给叔度:让他们二人辅助纣王的儿子武庚禄父,共同治理殷的遗民。武王把鲁地封给叔旦,但是让他留在京师辅助周王,就是周公。武王又把曹地封给叔振铎,把成地封给叔武,把霍地封给叔处。当时康叔封和冉季载都还年少,因此未能受封。

\begin{yuanwen}
武王既崩,成王少,周公旦专王室。管叔、蔡叔疑周公之为不利于成王,乃挟武庚以作乱。周公旦承成王命伐诛武庚,杀管叔,而放蔡叔,迁之,与车十乘,徒七十人从。而分殷馀民为二:其一封微子启于宋,以续殷祀;其一封康叔为卫君,是为卫康叔。封季载于(厓/冉)。(厓/冉)季、康叔皆有驯行\footnote{善行。},于是周公举康叔为周司寇,(厓/冉)季为周司空,以佐成王治,皆有令名\footnote{美名。}于天下。
\end{yuanwen}

武王已经驾崩,成王年纪还小,周公旦独揽国家大权。管叔、蔡叔怀疑周公的行为对成王不利,就挟持武庚发动叛乱。周公旦秉承着成王的命令讨伐并诛杀了武庚,杀了管叔,流放了蔡叔,在流放他的时候,给了他十辆车,七十个随从。把殷的遗民分成两个部分:一部分封给了微子启,在宋地,让他们延续殷代的祭祀;一部分封给康叔,建立卫国,并让他成为卫国的国君,就是卫康叔。还把季载封在了冉地。冉季、康叔都有驯良的品行,于是周公举荐康叔担任周朝的司寇,让冉季担任周朝的司空,让他们辅佐周成王治理国家,美名都传于天下。

\begin{yuanwen}
蔡叔度既迁而死。其子曰胡,胡乃改行,率\footnote{遵循。}德驯善。周公闻之,而举胡以为鲁卿士,鲁国治。于是周公言于成王,复封胡于蔡,以奉蔡叔之祀,是为蔡仲。馀五叔皆就国,无为天子吏者。
\end{yuanwen}

蔡叔度在流放的途中死去。他的儿子名叫胡,胡改变了他父亲在时的行为,遵循德训,实施善行。周公听说以后,便举荐胡担任了鲁国的卿士,鲁国被治理得非常好。于是周公就对成王进言,又把胡封在了蔡地,以供奉蔡叔的祭祀,这就是蔡仲。其他五叔都来到自己的封国,没有担任周天子的官吏。

\begin{yuanwen}
蔡仲卒,子蔡伯荒立。蔡伯荒卒,子宫侯立。宫侯卒,子厉侯立。厉侯卒,子武侯立。武侯之时,周厉王失国,奔彘,共和行政,诸侯多叛周。
\end{yuanwen}

蔡仲去世,他的儿子蔡伯荒继立。蔡伯荒去世,他的儿子宫侯继立。宫侯去世,他的儿子厉侯继立。厉侯去世,他的儿子武侯继立。武侯在位的时候,周厉王失掉了国家,逃到彘地,周王朝实行共和行政,诸侯大多叛变了周朝。

\begin{yuanwen}\end{yuanwen}\begin{yuanwen}\end{yuanwen}\begin{yuanwen}\end{yuanwen}\begin{yuanwen}\end{yuanwen}\begin{yuanwen}\end{yuanwen}\begin{yuanwen}\end{yuanwen}\begin{yuanwen}\end{yuanwen}\begin{yuanwen}\end{yuanwen}\begin{yuanwen}\end{yuanwen}\begin{yuanwen}\end{yuanwen}\begin{yuanwen}\end{yuanwen}\begin{yuanwen}\end{yuanwen}\begin{yuanwen}\end{yuanwen}\begin{yuanwen}\end{yuanwen}\begin{yuanwen}\end{yuanwen}\begin{yuanwen}\end{yuanwen}\begin{yuanwen}\end{yuanwen}\begin{yuanwen}\end{yuanwen}\begin{yuanwen}\end{yuanwen}\begin{yuanwen}\end{yuanwen}\begin{yuanwen}\end{yuanwen}\begin{yuanwen}\end{yuanwen}\begin{yuanwen}\end{yuanwen}\begin{yuanwen}\end{yuanwen}\begin{yuanwen}\end{yuanwen}

\begin{yuanwen}




武侯卒,子夷侯立。夷侯十一年,周宣王即位。二十八年,夷侯卒,子釐侯所事立。

釐侯三十九年,周幽王为犬戎所杀,周室卑而东徙。秦始得列为诸侯。

四十八年,釐侯卒,子共侯兴立。共侯二年卒,子戴侯立。戴侯十年卒,子宣侯措父立。

宣侯二十八年,鲁隐公初立。三十五年,宣侯卒,子桓侯封人立。桓侯三年,鲁弑其君隐公。二十年,桓侯卒,弟哀侯献舞立。

哀侯十一年,初,哀侯娶陈,息侯亦娶陈。息夫人将归,过蔡,蔡侯不敬。息侯怒,请楚文王:“来伐我,我求救于蔡,蔡必来,楚因击之,可以有功。”楚文王从之,虏蔡哀侯以归。哀侯留九岁,死于楚。凡立二十年卒。蔡人立其子肸,是为缪侯。

缪侯以其女弟为齐桓公夫人。十八年,齐桓公与蔡女戏船中,夫人荡舟,桓公止之,不止,公怒,归蔡女而不绝也。蔡侯怒,嫁其弟。齐桓公怒,伐蔡;蔡溃,遂虏缪侯,南至楚邵陵。已而诸侯为蔡谢齐,齐侯归蔡侯。二十九年,缪侯卒,子庄侯甲午立。  庄侯三年,齐桓公卒。十四年,晋文公败楚于城濮。二十年,楚太子商臣弑其父成王代立。二十五年,秦穆公卒。三十三年,楚庄王即位。三十四年,庄侯卒,子文侯申立。

文侯十四年,楚庄王伐陈,杀夏徵舒。十五年,楚围郑,郑降楚,楚复醳之。二十年,文侯卒,子景侯固立。

景侯元年,楚庄王卒。四十九年,景侯为太子般娶妇于楚,而景侯通焉。太子弑景侯而自立,是为灵侯。

灵侯二年,楚公子围弑其王郏敖而自立,为灵王。九年,陈司徒招弑其君哀公。楚使公子弃疾灭陈而有之。十二年,楚灵王以灵侯弑其父,诱蔡灵侯于申,伏甲饮之,醉而杀之,刑其士卒七十人。令公子弃疾围蔡。十一月,灭蔡,使弃疾为蔡公。

楚灭蔡三岁,楚公子弃疾弑其君灵王代立,为平王。平王乃求蔡景侯少子庐,立之,是为平侯。是年,楚亦复立陈。楚平王初立,欲亲诸侯,故复立陈、蔡后。

平侯九年卒,灵侯般之孙东国攻平侯子而自立,是为悼侯。悼侯父曰隐太子友。隐太子友者,灵侯之太子,平侯立而杀隐太子,故平侯卒而隐太子之子东国攻平侯子而代立,是为悼侯。悼侯三年卒,弟昭侯申立。

昭侯十年,朝楚昭王,持美裘二,献其一于昭王而自衣其一。楚相子常欲之,不与。子常谗蔡侯,留之楚三年。蔡侯知之,乃献其裘于子常;子常受之,乃言归蔡侯。蔡侯归而之晋,请与晋伐楚。

十三年春,与卫灵公会邵陵。蔡侯私于周苌弘以求长于卫;卫使史言康叔之功德,乃长卫。夏,为晋灭沈,楚怒,攻蔡。蔡昭侯使其子为质于吴,以共伐楚。冬,与吴王阖闾遂破楚入郢。蔡怨子常,子常恐,奔郑。十四年,吴去而楚昭王复国。十六年,楚令尹为其民泣以谋蔡,蔡昭侯惧。二十六年,孔子如蔡。楚昭王伐蔡,蔡恐,告急于吴。吴为蔡远,约迁以自近,易以相救;昭侯私许,不与大夫计。吴人来救蔡,因迁蔡于州来。二十八年,昭侯将朝于吴,大夫恐其复迁,乃令贼利杀昭侯;已而诛贼利以解过,而立昭侯子朔,是为成侯。

成侯四年,宋灭曹。十年,齐田常弑其君简公。十三年,楚灭陈。十九年,成侯卒,子声侯产立。声侯十五年卒,子元侯立。元侯六年卒,子侯齐立。

侯齐四年,楚惠王灭蔡,蔡侯齐亡,蔡遂绝祀。后陈灭三十三年。

伯邑考,其后不知所封。武王发,其后为周,有本纪言。管叔鲜作乱诛死,无后。周公旦,其后为鲁,有世家言。蔡叔度,其后为蔡,有世家言。曹叔振铎,有后为曹,有世家言。成叔武,其后世无所见。霍叔处,其后晋献公时灭霍。康叔封,其后为卫,有世家言。厓季载,其后世无所见。

太史公曰:管蔡作乱,无足载者。然周武王崩,成王少,天下既疑,赖同母之弟成叔、厓季之属十人为辅拂,是以诸侯卒宗周,故附之世家言。

曹叔振铎者,周武王弟也。武王已克殷纣,封叔振铎于曹。

叔振铎卒,子太伯脾立。太伯卒,子仲君平立。仲君平卒,子宫伯侯立。宫伯侯卒,子孝伯云立。孝伯云卒,子夷伯喜立。

夷伯二十三年,周厉王奔于彘。

三十年卒,弟幽伯彊立。幽伯九年,弟苏杀幽伯代立,是为戴伯。戴伯元年,周宣王已立三岁。三十年,戴伯卒,子惠伯兕立。

惠伯二十五年,周幽王为犬戎所杀,因东徙,益卑,诸侯畔之。秦始列为诸侯。

三十六,惠伯卒,子石甫立,其弟武杀之代立,是为缪公。缪公三年卒,子桓公终生立。

桓公三十五年,鲁隐公立。四十五年,鲁弑其君隐公。四十六年,宋华父督弑其君殇公,及孔父。五十五年,桓公卒,子庄公夕姑立。

庄公二十三年,齐桓公始霸。

三十一年,庄公卒,子釐公夷立。釐公九年卒,子昭公班立。昭公六年,齐桓公败蔡,遂至楚召陵。九年,昭公卒,子共公襄立。

共公十六年,初,晋公子重耳其亡过曹,曹君无礼,欲观其骈胁。釐负羁谏,不听,私善于重耳。二十一年,晋文公重耳伐曹,虏共公以归,令军毋入釐负羁之宗族闾。或说晋文公曰:“昔齐桓公会诸侯,复异姓;今君囚曹君,灭同姓,何以令于诸侯?”晋乃复归共公。

二十五年,晋文公卒。三十五年,共公卒,子文公寿立。文公二十三年卒,子宣公彊立。宣公十七年卒,弟成公负刍立。

成公三年,晋厉公伐曹,虏成公以归,已复释之。五年,晋栾书、中行偃使程滑弑其君厉公。二十三年,成公卒,子武公胜立。武公二十六年,楚公子弃疾弑其君灵王代立。二十七年,武公卒,子平公立。平公四年卒,子悼公午立。是岁,宋、卫、陈、郑皆火。

悼公八年,宋景公立。九年,悼公朝于宋,宋囚之;曹立其弟野,是为声公。悼公死于宋,归葬。

声公五年,平公弟通弑声公代立,是为隐公。隐公四年,声公弟露弑隐公代立,是为靖公。靖公四年卒,子伯阳立。

伯阳三年,国人有梦众君子立于社宫,谋欲亡曹;曹叔振铎止之,请待公孙彊,许之。旦,求之曹,无此人。梦者戒其子曰:“我亡,尔闻公孙彊为政,必去曹,无离曹祸。”及伯阳即位,好田弋之事。六年,曹野人公孙彊亦好田弋,获白雁而献之,且言田弋之说,因访政事。伯阳大说之,有宠,使为司城以听政。梦者之子乃亡去。

公孙彊言霸说于曹伯。十四年,曹伯从之,乃背晋干宋。宋景公伐之,晋人不救。十五年,宋灭曹,执曹伯阳及公孙彊以归而杀之。曹遂绝其祀。

太史公曰:余寻曹共公之不用僖负羁,乃乘轩者三百人,知唯德之不建。及振铎之梦,岂不欲引曹之祀者哉?如公孙彊不脩厥政,叔铎之祀忽诸。

武王之弟,管、蔡及霍。周公居相,流言是作。狼跋致艰,鸱鸮讨恶。胡能改行,克复其爵。献舞执楚,遇息礼薄。穆侯虏齐,荡舟乖谑。曹共轻晋,负羁先觉。伯阳梦社,祚倾振铎。
上一章
\end{yuanwen}

\chapter{陈杞世家}

\begin{yuanwen}
陈胡公满者,虞帝舜之后也。昔舜为庶人时,尧妻之二女,居于妫汭,其后因为氏姓,姓妫氏。舜已崩,传禹天下,而舜子商均为封国。夏后之时,或失或续。至于周武王克殷纣,乃复求舜后,得妫满,封之于陈,以奉帝舜祀,是为胡公。

胡公卒,子申公犀侯立。申公卒,弟相公皋羊立。相公卒,立申公子突,是为孝公。孝公卒,子慎公圉戎立。慎公当周厉王时。慎公卒,子幽公宁立。

幽公十二年,周厉王奔于彘。

二十三年,幽公卒,子釐公孝立。釐公六年,周宣王即位。三十六年,釐公卒,子武公灵立。武公十五年卒,子夷公说立。是岁,周幽王即位。夷公三年卒,弟平公燮立。平公七年,周幽王为犬戎所杀,周东徙。秦始列为诸侯。

二十三年,平公卒,子文公圉立。

文公元年,取蔡女,生子佗。十年,文公卒,长子桓公鲍立。

桓公二十三年,鲁隐公初立。二十六年,卫杀其君州吁。三十三年,鲁弑其君隐公。

三十八年正月甲戌己丑,桓公鲍卒。桓公弟佗,其母蔡女,故蔡人为佗杀五父及桓公太子免而立佗,是为厉公。桓公病而乱作,国人分散,故再赴。

厉公二年,生子敬仲完。周太史过陈,陈厉公使以周易筮之,卦得观之否:“是为观国之光,利用宾于王。此其代陈有国乎?不在此,其在异国?非此其身,在其子孙。若在异国,必姜姓。姜姓,太岳之后。物莫能两大,陈衰,此其昌乎?”

厉公取蔡女,蔡女与蔡人乱,厉公数如蔡淫。七年,厉公所杀桓公太子免之三弟,长曰跃,中曰林,少曰杵臼,共令蔡人诱厉公以好女,与蔡人共杀厉公而立跃,是为利公。利公者,桓公子也。利公立五月卒,立中弟林,是为庄公。庄公七年卒,少弟杵臼立,是为宣公。

宣公三年,楚武王卒,楚始彊。十七年,周惠王娶陈女为后。

二十一年,宣公后有嬖姬生子款,欲立之,乃杀其太子御寇。御寇素爱厉公子完,完惧祸及己,乃奔齐。齐桓公欲使陈完为卿,完曰:“羁旅之臣,幸得免负檐,君之惠也,不敢当高位。”桓公使为工正。齐懿仲欲妻陈敬仲,卜之,占曰:“是谓凤皇于飞,和鸣锵锵。有妫之后,将育于姜。五世其昌,并于正卿。八世之后,莫之与京。”

三十七年,齐桓公伐蔡,蔡败;南侵楚,至召陵,还过陈。陈大夫辕涛涂恶其过陈,诈齐令出东道。东道恶,桓公怒,执陈辕涛涂。是岁,晋献公杀其太子申生。

四十五年,宣公卒,子款立,是为穆公。穆公五年,齐桓公卒。十六年,晋文公败楚师于城濮。是岁,穆公卒,子共公朔立。共公六年,楚太子商臣弑其父成王代立,是为穆王。十一年,秦穆公卒。十八年,共公卒,子灵公平国立。

灵公元年,楚庄王即位。六年,楚伐陈。十年,陈及楚平。

十四年,灵公与其大夫孔宁、仪行父皆通于夏姬,衷其衣以戏于朝。泄冶谏曰:“君臣淫乱,民何效焉?”灵公以告二子,二子请杀泄冶,公弗禁,遂杀泄冶。十五年,灵公与二子饮于夏氏。公戏二子曰:“徵舒似汝。”二子曰:“亦似公。”徵舒怒。灵公罢酒出,徵舒伏弩厩门射杀灵公。孔宁、仪行父皆奔楚,灵公太子午奔晋。徵舒自立为陈侯。徵舒,故陈大夫也。夏姬,御叔之妻,舒之母也。

成公元年冬,楚庄王为夏徵舒杀灵公,率诸侯伐陈。谓陈曰:“无惊,吾诛徵舒而已。”已诛徵舒,因县陈而有之,群臣毕贺。申叔时使于齐来还,独不贺。庄王问其故,对曰:“鄙语有之,牵牛径人田,田主夺之牛。径则有罪矣,夺之牛,不亦甚乎?今王以徵舒为贼弑君,故徵兵诸侯,以义伐之,已而取之,以利其地,则后何以令于天下!是以不贺。”庄王曰:“善。”乃迎陈灵公太子午于晋而立之,复君陈如故,是为成公。孔子读史记至楚复陈,曰:“贤哉楚庄王!轻千乘之国而重一言。”

八年,楚庄王卒。二十九年,陈倍楚盟。三十年,楚共王伐陈。是岁,成公卒,子哀公弱立。楚以陈丧,罢兵去。

哀公三年,楚围陈,复释之。二十八年,楚公子围弑其君郏敖自立,为灵王。

三十四年,初,哀公娶郑,长姬生悼太子师,少姬生偃。二嬖妾,长妾生留,少妾生胜。留有宠哀公,哀公属之其弟司徒招。哀公病,三月,招杀悼太子,立留为太子。哀公怒,欲诛招,招发兵围守哀公,哀公自经杀。招卒立留为陈君。四月,陈使使赴楚。楚灵王闻陈乱,乃杀陈使者,使公子弃疾发兵伐陈,陈君留奔郑。九月,楚围陈。十一月,灭陈。使弃疾为陈公。

招之杀悼太子也,太子之子名吴,出奔晋。晋平公问太史赵曰:“陈遂亡乎?”对曰:“陈,颛顼之族。陈氏得政于齐,乃卒亡。自幕至于瞽瞍,无违命。舜重之以明德。至于遂,世世守之。及胡公,周赐之姓,使祀虞帝。且盛德之后,必百世祀。虞之世未也,其在齐乎?”

楚灵王灭陈五岁,楚公子弃疾弑灵王代立,是为平王。平王初立,欲得和诸侯,乃求故陈悼太子师之子吴,立为陈侯,是为惠公。惠公立,探续哀公卒时年而为元,空籍五岁矣。

十年,陈火。十五年,吴王僚使公子光伐陈,取胡、沈而去。二十八年,吴王阖闾与子胥败楚入郢。是年,惠公卒,子怀公柳立。

怀公元年,吴破楚,在郢,召陈侯。陈侯欲往,大夫曰:“吴新得意;楚王虽亡,与陈有故,不可倍。”怀公乃以疾谢吴。四年,吴复召怀公。怀公恐,如吴。吴怒其前不往,留之,因卒吴。陈乃立怀公之子越,是为湣公。

湣公六年,孔子適陈。吴王夫差伐陈,取三邑而去。十三年,吴复来伐陈,陈告急楚,楚昭王来救,军于城父,吴师去。是年,楚昭王卒于城父。时孔子在陈。十五年,宋灭曹。十六年,吴王夫差伐齐,败之艾陵,使人召陈侯。陈侯恐,如吴。楚伐陈。二十一年,齐田常弑其君简公。二十三年,楚之白公胜杀令尹子西、子綦,袭惠王。叶公攻败白公,白公自杀。

二十四年,楚惠王复国,以兵北伐,杀陈湣公,遂灭陈而有之。是岁,孔子卒。

杞东楼公者,夏后禹之后苗裔也。殷时或封或绝。周武王克殷纣,求禹之后,得东楼公,封之于杞,以奉夏后氏祀。

东楼公生西楼公,西楼公生题公,题公生谋娶公。谋娶公当周厉王时。谋娶公生武公。武公立四十七年卒,子靖公立。靖公二十三年卒,子共公立。共公八年卒,子德公立。德公十八年卒,弟桓公姑容立。桓公十七年卒,子孝公匄立。孝公十七年卒,弟文公益姑立。文公十四年卒,弟平公郁立。平公十八年卒,子悼公成立。悼公十二年卒,子隐公乞立。七月,隐公弟遂弑隐公自立,是为釐公。釐公十九年卒,子湣公维立。湣公十五年,楚惠王灭陈。十六年,湣公弟阏路弑湣公代立,是为哀公。哀公立十年卒,湣公子敕立,是为出公。出公十二年卒,子简公春立。立一年,楚惠王之四十四年,灭杞。杞后陈亡三十四年。杞小微,其事不足称述。

舜之后,周武王封之陈,至楚惠王灭之,有世家言。禹之后,周武王封之杞,楚惠王灭之,有世家言。契之后为殷,殷有本纪言。殷破,周封其后于宋,齐湣王灭之,有世家言。后稷之后为周,秦昭王灭之,有本纪言。皋陶之后,或封英、六,楚穆王灭之,无谱。伯夷之后,至周武王复封于齐,曰太公望,陈氏灭之,有世家言。伯翳之后,至周平王时封为秦,项羽灭之,有本纪言。垂、益、夔、龙,其后不知所封,不见也。右十一人者,皆唐虞之际名有功德臣也;其五人之后皆至帝王,馀乃为显诸侯。滕、薛、驺,夏、殷、周之间封也,小,不足齿列,弗论也。

周武王时,侯伯尚千馀人。及幽、厉之后,诸侯力攻相并。江、黄、胡、沈之属,不可胜数,故弗采著于传。

太史公曰:舜之德可谓至矣!禅位于夏,而后世血食者历三代。及楚灭陈,而田常得政于齐,卒为建国,百世不绝,苗裔兹兹,有土者不乏焉。至禹,于周则杞,微甚,不足数也。楚惠王灭杞,其后越王句践兴。

盛德之祀,必及百世。舜、禹馀烈,陈、杞是继。妫满受封,东楼纂系。阏路篡逆,夏姬淫嬖。二国衰微,或兴或替。前并后虏,皆亡楚惠。句践勃兴,田和吞噬。蝉联血食,岂其苗裔?
\end{yuanwen}

\chapter{卫康叔世家}

\begin{yuanwen}
卫康叔名封,周武王同母少弟也。其次尚有厓季,厓季最少。

武王已克殷纣,复以殷馀民封纣子武庚禄父,比诸侯,以奉其先祀勿绝。为武庚未集,恐其有贼心,武王乃令其弟管叔、蔡叔傅相武庚禄父,以和其民。武王既崩,成王少。周公旦代成王治,当国。管叔、蔡叔疑周公,乃与武庚禄父作乱,欲攻成周。周公旦以成王命兴师伐殷,杀武庚禄父、管叔,放蔡叔,以武庚殷馀民封康叔为卫君,居河、淇间故商墟。

周公旦惧康叔齿少,乃申告康叔曰:“必求殷之贤人君子长者,问其先殷所以兴,所以亡,而务爱民。”告以纣所以亡者以淫于酒,酒之失,妇人是用,故纣之乱自此始。为梓材,示君子可法则。故谓之康诰、酒诰、梓材以命之。康叔之国,既以此命,能和集其民,民大说。

成王长,用事,举康叔为周司寇,赐卫宝祭器,以章有德。

康叔卒,子康伯代立。康伯卒,子考伯立。考伯卒,子嗣伯立。嗣伯卒,子榅伯立。榅伯卒,子靖伯立。靖伯卒,子贞伯立。贞伯卒,子顷侯立。

顷侯厚赂周夷王,夷王命卫为侯。顷侯立十二年卒,子釐侯立。

釐侯十三年,周厉王出饹于彘,共和行政焉。二十八年,周宣王立。

四十二年,釐侯卒,太子共伯馀立为君。共伯弟和有宠于釐侯,多予之赂;和以其赂赂士,以袭攻共伯于墓上,共伯入釐侯羡自杀。卫人因葬之釐侯旁,谥曰共伯,而立和为卫侯,是为武公。

武公即位,修康叔之政,百姓和集。四十二年,犬戎杀周幽王,武公将兵往佐周平戎,甚有功,周平王命武公为公。五十五年,卒,子庄公扬立。

庄公五年,取齐女为夫人,好而无子。又取陈女为夫人,生子,蚤死。陈女女弟亦幸于庄公,而生子完。完母死,庄公令夫人齐女子之,立为太子。庄公有宠妾,生子州吁。十八年,州吁长,好兵,庄公使将。石碏谏庄公曰:“庶子好兵,使将,乱自此起。”不听。二十三年,庄公卒,太子完立,是为桓公。

桓公二年,弟州吁骄奢,桓公绌之,州吁出饹。十三年,郑伯弟段攻其兄,不胜,亡,而州吁求与之友。十六年,州吁收聚卫亡人以袭杀桓公,州吁自立为卫君。为郑伯弟段欲伐郑,请宋、陈、蔡与俱,三国皆许州吁。州吁新立,好兵,弑桓公,卫人皆不爱。石碏乃因桓公母家于陈,详为善州吁。至郑郊,石碏与陈侯共谋,使右宰丑进食,因杀州吁于濮,而迎桓公弟晋于邢而立之,是为宣公。

宣公七年,鲁弑其君隐公。九年,宋督弑其君殇公,及孔父。十年,晋曲沃庄伯弑其君哀侯。

十八年,初,宣公爱夫人夷姜,夷姜生子伋,以为太子,而令右公子傅之。右公子为太子取齐女,未入室,而宣公见所欲为太子妇者好,说而自取之,更为太子取他女。宣公得齐女,生子寿、子朔,令左公子傅之。太子伋母死,宣公正夫人与朔共谗恶太子伋。宣公自以其夺太子妻也,心恶太子,欲废之。及闻其恶,大怒,乃使太子伋于齐而令盗遮界上杀之,与太子白旄,而告界盗见持白旄者杀之。且行,子朔之兄寿,太子异母弟也,知朔之恶太子而君欲杀之,乃谓太子曰:“界盗见太子白旄,即杀太子,太子可毋行。”太子曰:“逆父命求生,不可。”遂行。寿见太子不止,扑盗其沧旄而先驰至界。界盗见其验,即杀之。寿已死,而太子伋又至,谓盗曰:“所当杀乃我也。”盗并杀太子伋,以报宣公。宣公乃以子朔为太子。十九年,宣公卒,太子朔立,是为惠公。

左右公子不平朔之立也,惠公四年,左右公子怨惠公之谗杀前太子伋而代立,乃作乱,攻惠公,立太子伋之弟黔牟为君,惠公饹齐。

卫君黔牟立八年,齐襄公率诸侯奉王命共伐卫,纳卫惠公,诛左右公子。卫君黔牟饹于周,惠公复立。惠公立三年出亡,亡八年复入,与前通年凡十三年矣。

二十五年,惠公怨周之容舍黔牟,与燕伐周。周惠王饹温,卫、燕立惠王弟穨为王。二十九年,郑复纳惠王。三十一年,惠公卒,子懿公赤立。

懿公即位,好鹤,淫乐奢侈。九年,翟伐卫,卫懿公欲发兵,兵或畔。大臣言曰:“君好穀,穀可令击翟。”翟于是遂入,杀懿公。

懿公之立也,百姓大臣皆不服。自懿公父惠公朔之谗杀太子伋代立至于懿公,常欲败之,卒灭惠公之后而更立黔牟之弟昭伯顽之子申为君,是为戴公。

戴公申元年卒。齐桓公以卫数乱,乃率诸侯伐翟,为卫筑楚丘,立戴公弟毁为卫君,初,翟杀懿公也,卫人怜之,思复立宣公前死太子伋之后,伋子又死,而代伋死者子寿又无子。太子伋同母弟二人:其一曰黔牟,黔牟尝代惠公为君,八年复去;其二曰昭伯。昭伯、黔牟皆已前死,故立昭伯子申为戴公。戴公卒,复立其弟毁为文公。

文公初立,轻赋平罪,身自劳,与百姓同苦,以收卫民。

十六年,晋公子重耳过,无礼。十七年,齐桓公卒。二十五年,文公卒,子成公郑立。

成公三年,晋欲假道于卫救宋,成公不许。晋更从南河度,救宋。徵师于卫,卫大夫欲许,成公不肯。大夫元咺攻成公,成公出饹。晋文公重耳伐卫,分其地予宋,讨前过无礼及不救宋患也。卫成公遂出奔陈。二岁,如周求入,与晋文公会。晋使人鸩卫成公,成公私于周主鸩,令薄,得不死。已而周为请晋文公,卒入之卫,而诛元亘,卫君瑕出饹。七年,晋文公卒。十二年,成公朝晋襄公。十四年,秦穆公卒。二十六年,齐邴歜弑其君懿公。三十五年,成公卒,子穆公立。

穆公二年,楚庄王伐陈,杀夏徵舒。三年,楚庄王围郑,郑降,复释之。十一年,孙良夫救鲁伐齐,复得侵地。穆公卒,子定公臧立。定公十二年卒,子献公衎立。

献公十三年,公令师曹教宫妾鼓琴,妾不善,曹笞之。妾以幸恶曹于公,公亦笞曹三百。十八年,献公戒孙文子、甯惠子食,皆往。日旰不召,而去射鸿于囿。二子从之,公不释射服与之言。二子怒,如宿。孙文子子数侍公饮,使师曹歌巧言之卒章。师曹又怒公之尝笞三百,乃歌之,欲以怒孙文子,报卫献公。文子语蘧伯玉,伯玉曰:“臣不知也。”遂攻出献公。献公奔齐,齐置卫献公于聚邑。孙文子、甯惠子共立定公弟秋为卫君,是为殇公。

殇公秋立,封孙文子林父于宿。十二年,甯喜与孙林父争宠相恶,殇公使甯喜攻孙林父。林父饹晋,复求入故卫献公。献公在齐,齐景公闻之,与卫献公如晋求入。晋为伐卫,诱与盟。卫殇公会晋平公,平公执殇公与甯喜而复入卫献公。献公亡在外十二年而入。

献公后元年,诛甯喜。

三年,吴延陵季子使过卫,见蘧伯玉、史,曰:“卫多君子,其国无故。”过宿,孙林父为击磬,曰:“不乐,音大悲,使卫乱乃此矣。”是年,献公卒,子襄公恶立。

襄公六年,楚灵王会诸侯,襄公称病不往。

九年,襄公卒。初,襄公有贱妾,幸之,有身,梦有人谓曰:“我康叔也,令若子必有卫,名而子曰‘元’。”妾怪之,问孔成子。成子曰:“康叔者,卫祖也。”及生子,男也,以告襄公。襄公曰:“天所置也。”名之曰元。襄公夫人无子,于是乃立元为嗣,是为灵公。

灵公五年,朝晋昭公。六年,楚公子弃疾弑灵王自立,为平王。十一年,火。

三十八年,孔子来,禄之如鲁。后有隙,孔子去。后复来。

三十九年,太子蒯聩与灵公夫人南子有恶,欲杀南子。蒯聩与其徒戏阳谋,朝,使杀夫人。戏阳后悔,不果。蒯聩数目之,夫人觉之,惧,呼曰:“太子欲杀我!”灵公怒,太子蒯聩饹宋,已而之晋赵氏。

四十二年春,灵公游于郊,令子郢仆。郢,灵公少子也,字子南。灵公怨太子出饹,谓郢曰:“我将立若为后。”郢对曰:“郢不足以辱社稷,君更图之。”夏,灵公卒,夫人命子郢为太子,曰:“此灵公命也。”郢曰:“亡人太子蒯聩之子辄在也,不敢当。”于是卫乃以辄为君,是为出公。

六月乙酉,赵简子欲入蒯聩,乃令阳虎诈命卫十馀人衰绖归,简子送蒯聩。卫人闻之,发兵击蒯聩。蒯聩不得入,入宿而保,卫人亦罢兵。

出公辄四年,齐田乞弑其君孺子。八年,齐鲍子弑其君悼公。

孔子自陈入卫。九年,孔文子问兵于仲尼,仲尼不对。其后鲁迎仲尼,仲尼反鲁。

十二年,初,孔圉文子取太子蒯聩之姊,生悝。孔氏之竖浑良夫美好,孔文子卒,良夫通于悝母。太子在宿,悝母使良夫于太子。太子与良夫言曰:“苟能入我国,报子以乘轩,免子三死,毋所与。”与之盟,许以悝母为妻。闰月,良夫与太子入,舍孔氏之外圃。昏,二人蒙衣而乘,宦者罗御,如孔氏。孔氏之老栾甯问之,称姻妾以告。遂入,適伯姬氏。既食,悝母杖戈而先,太子与五人介,舆猳从之。伯姬劫悝于厕,彊盟之,遂劫以登台。栾甯将饮酒,炙未熟,闻乱,使告仲由。召护驾乘车,行爵食炙,奉出公辄奔鲁。

仲由将入,遇子羔将出,曰:“门已闭矣。”子路曰:“吾姑至矣。”子羔曰:“不及,莫践其难。”子路曰:“食焉不辟其难。”子羔遂出。子路入,及门,公孙敢阖门,曰:“毋入为也!”子路曰:“是公孙也?求利而逃其难。由不然,利其禄,必救其患。”有使者出,子路乃得入。曰:“太子焉用孔悝?虽杀之,必或继之。”且曰:“太子无勇。若燔台,必舍孔叔。”太子闻之,惧,下石乞、盂黡敌子路,以戈击之,割缨。子路曰:“君子死,冠不免。”结缨而死。孔子闻卫乱,曰:“嗟乎!柴也其来乎?由也其死矣。”孔悝竟立太子蒯聩,是为庄公。

庄公蒯聩者,出公父也,居外,怨大夫莫迎立。元年即位,欲尽诛大臣,曰:“寡人居外久矣,子亦尝闻之乎?”群臣欲作乱,乃止。

二年,鲁孔丘卒。

三年,庄公上城,见戎州。曰:“戎虏何为是?”戎州病之。十月,戎州告赵简子,简子围卫。十一月,庄公出奔,卫人立公子斑师为卫君。齐伐卫,虏斑师,更立公子起为卫君。

卫君起元年,卫石曼尃逐其君起,起奔齐。卫出公辄自齐复归立。初,出公立十二年亡,亡在外四年复入。出公后元年,赏从亡者。立二十一年卒,出公季父黔攻出公子而自立,是为悼公。

悼公五年卒,子敬公弗立。敬公十九年卒,子昭公纠立。是时三晋彊,卫如小侯,属之。

昭公六年,公子亹弑之代立,是为怀公。怀公十一年,公子穨弑怀公而代立,是为慎公。慎公父,公子適;適父,敬公也。慎公四十二年卒,子声公训立。声公十一年卒,子成侯立。

成侯十一年,公孙鞅入秦。十六年,卫更贬号曰侯。

二十九年,成侯卒,子平侯立。平侯八年卒,子嗣君立。

嗣君五年,更贬号曰君,独有濮阳。

四十二年卒,子怀君立。怀君三十一年,朝魏,魏囚杀怀君。魏更立嗣君弟,是为元君。元君为魏婿,故魏立之。元君十四年,秦拔魏东地,秦初置东郡,更徙卫野王县,而并濮阳为东郡。二十五年,元君卒,子君角立。

君角九年,秦并天下,立为始皇帝。二十一年,二世废君角为庶人,卫绝祀。

太史公曰:余读世家言,至于宣公之太子以妇见诛,弟寿争死以相让,此与晋太子申生不敢明骊姬之过同,俱恶伤父之志。然卒死亡,何其悲也!或父子相杀,兄弟相灭,亦独何哉?

司寇受封,梓材有作。成锡厥器,夷加其爵。暨武能脩,从文始约。诗美归燕,传矜石碏。皮冠射鸿,乘轩使穀。宣纵淫嬖,衅生伋、朔。蒯聩得罪,出公行恶。卫祚日衰,失于君角。
\end{yuanwen}

\chapter{宋微子世家}

\begin{yuanwen}
微子开者,殷帝乙之首子而帝纣之庶兄也。纣既立,不明,淫乱于政,微子数谏,纣不听。及祖伊以周西伯昌之修德,灭璿国,惧祸至,以告纣。纣曰:“我生不有命在天乎?是何能为!”于是微子度纣终不可谏,欲死之,及去,未能自决,乃问于太师、少师曰:“殷不有治政,不治四方。我祖遂陈于上,殷既小大好草窃奸宄,卿士师师非度,皆有罪辜,乃无维获,小民乃并兴,相为敌雠。今殷其典丧!若涉水无津涯。殷遂丧,越至于今。”曰:“太师,少师,我其发出往?吾家保于丧?今女无故告予,颠跻,如之何其?”太师若曰:“王子,天笃下菑亡殷国,乃毋畏畏,不用老长。今殷民乃陋淫神祇之祀。今诚得治国,国治身死不恨。为死,终不得治,不如去。”遂亡。

箕子者,纣亲戚也。纣始为象箸,箕子叹曰:“彼为象箸,必为玉桮;为桮,则必思远方珍怪之物而御之矣。舆马宫室之渐自此始,不可振也。”纣为淫泆,箕子谏,不听。人或曰:“可以去矣。”箕子曰:“为人臣谏不听而去,是彰君之恶而自说于民,吾不忍为也。”乃被发详狂而为奴。遂隐而鼓琴以自悲,故传之曰箕子操。

王子比干者,亦纣之亲戚也。见箕子谏不听而为奴,则曰:“君有过而不以死争,则百姓何辜!”乃直言谏纣。纣怒曰:“吾闻圣人之心有七窍,信有诸乎?”乃遂杀王子比干,刳视其心。

微子曰:“父子有骨肉,而臣主以义属。故父有过,子三谏不听,则随而号之;人臣三谏不听,则其义可以去矣。”于是太师、少师乃劝微子去,遂行。

周武王伐纣克殷,微子乃持其祭器造于军门,肉袒面缚,左牵羊,右把茅,膝行而前以告。于是武王乃释微子,复其位如故。

武王封纣子武庚禄父以续殷祀,使管叔、蔡叔傅相之。

武王既克殷,访问箕子。

武王曰:“于乎!维天阴定下民,相和其居,我不知其常伦所序。”

箕子对曰:“在昔鲧堙鸿水,汨陈其五行,帝乃震怒,不从鸿范九等,常伦所斁。鲧则殛死,禹乃嗣兴。天乃锡禹鸿范九等,常伦所序。

“初一曰五行;二曰五事;三曰八政;四曰五纪;五曰皇极;六曰三德;七曰稽疑;八曰庶徵;九曰乡用五福,畏用六极。

“五行:一曰水,二曰火,三曰木,四曰金,五曰土。水曰润下,火曰炎上,木曰曲直,金曰从革,土曰稼穑。润下作咸,炎上作苦,曲直作酸,从革作辛,稼穑作甘。

“五事:一曰貌,二曰言,三曰视,四曰听,五曰思。貌曰恭,言曰从,视曰明,听曰聪,思曰睿。恭作肃,从作治,明作智,聪作谋,睿作圣。

“八政:一曰食,二曰货,三曰祀,四曰司空,五曰司徒,六曰司寇,七曰宾,八曰师。

“五纪:一曰岁,二曰月,三曰日,四曰星辰,五曰历数。

“皇极:皇建其有极,敛时五福,用傅锡其庶民,维时其庶民于女极,锡女保极。凡厥庶民,毋有淫朋,人毋有比德,维皇作极。凡厥庶民,有猷有为有守,女则念之。不协于极,不离于咎,皇则受之。而安而色,曰予所好德,女则锡之福。时人斯其维皇之极。毋侮鳏寡而畏高明。人之有能有为,使羞其行,而国其昌。凡厥正人,既富方穀。女不能使有好于而家,时人斯其辜。于其毋好,女虽锡之福,其作女用咎。毋偏毋颇,遵王之义。毋有作好,遵王之道。毋有作恶,遵王之路。毋偏毋党,王道荡荡。毋党毋偏,王道平平。毋反毋侧,王道正直。会其有极,归其有极。曰王极之傅言,是夷是训,于帝其顺。凡厥庶民,极之傅言,是顺是行,以近天子之光。曰天子作民父母,以为天下王。

“三德:一曰正直,二曰刚克,三曰柔克。平康正直,彊不友刚克,内友柔克,沈渐刚克,高明柔克。维辟作福,维辟作威,维辟玉食。臣有作福作威玉食,其害于而家,凶于而国,人用侧颇辟,民用僭忒。

“稽疑:择建立卜筮人。乃命卜筮,曰雨,曰济,曰涕,曰雾,曰克,曰贞,曰悔,凡七。卜五,占之用二,衍貣。立时人为卜筮,三人占则从二人之言。女则有大疑,谋及女心,谋及卿士,谋及庶人,谋及卜筮。女则从,龟从,筮从,卿士从,庶民从,是之谓大同,而身其康彊,而子孙其逢吉。女则从,龟从,筮从,卿士逆,庶民逆,吉。卿士从,龟从,筮从,女则逆,庶民逆,吉。庶民从,龟从,筮从,女则逆,卿士逆,吉。女则从,龟从,筮逆,卿士逆,庶民逆,作内吉,作外凶。龟筮共违于人,用静吉,用作凶。

“庶徵:曰雨,曰阳,曰奥,曰寒,曰风,曰时。五者来备,各以其序,庶草繁庑。一极备,凶。一极亡,凶。曰休徵:曰肃,时雨若,曰治,时旸若;曰知,时奥若;曰谋,时寒若;曰圣,时风若。曰咎徵:曰僭,常旸若;曰舒,常奥若;曰急,常寒若;曰雾,常风若。王眚维岁,师尹维日。岁月日时毋易,百穀用成,治用明,畯民用章,家用平康。日月岁时既易,百穀用不成,治用昏不明,畯民用微,家用不宁。庶民维星,星有好风,星有好雨。日月之行,有冬有夏。月之从星,则以风雨。

“五福:一曰寿,二曰富,三曰康宁,四曰攸好德,五曰考终命。六极:一曰凶短折,二曰疾,三曰忧,四曰贫,五曰恶,六曰弱。”

于是武王乃封箕子于朝鲜而不臣也。

其后箕子朝周,过故殷虚,感宫室毁坏,生禾黍,箕子伤之,欲哭则不可,欲泣为其近妇人,乃作麦秀之诗以歌咏之。其诗曰:“麦秀渐渐兮,禾黍油油。彼狡僮兮,不与我好兮!”所谓狡童者,纣也。殷民闻之,皆为流涕。

武王崩,成王少,周公旦代行政当国。管、蔡疑之,乃与武庚作乱,欲袭成王、周公。周公既承成王命诛武庚,杀管叔,放蔡叔,乃命微子开代殷后,奉其先祀,作微子之命以申之,国于宋。微子故能仁贤,乃代武庚,故殷之馀民甚戴爱之。

微子开卒,立其弟衍,是为微仲。微仲卒,子宋公稽立。宋公稽卒,子丁公申立。丁公申卒,子湣公共立。湣公共卒,弟炀公熙立。炀公即位,湣公子鲋祀弑炀公而自立,曰“我当立”,是为厉公。厉公卒,子釐公举立。

釐公十七年,周厉王出奔彘。

二十八年,釐公卒,子惠公琤立。惠公四年,周宣王即位。三十年,惠公卒,子哀公立。哀公元年卒,子戴公立。戴公二十九年,周幽王为犬戎所杀,秦始列为诸侯。

三十四年,戴公卒,子武公司空立。武公生女为鲁惠公夫人,生鲁桓公。十八年,武公卒,子宣公力立。

宣公有太子与夷。十九年,宣公病,让其弟和,曰:“父死子继,兄死弟及,天下通义也。我其立和。”和亦三让而受之。宣公卒,弟和立,是为穆公。

穆公九年,病,召大司马孔父谓曰:“先君宣公舍太子与夷而立我,我不敢忘。我死,必立与夷也。”孔父曰:“群臣皆原立公子冯。”穆公曰:“毋立冯,吾不可以负宣公。”于是穆公使冯出居于郑。八月庚辰,穆公卒,兄宣公子与夷立,是为殇公。君子闻之,曰:“宋宣公可谓知人矣,立其弟以成义,然卒其子复享之。

殇公元年,卫公子州吁弑其君完自立,欲得诸侯,使告于宋曰:“冯在郑,必为乱,可与我伐之。”宋许之,与伐郑,至东门而还。二年,郑伐宋,以报东门之役。其后诸侯数来侵伐。

九年,大司马孔父嘉妻好,出,道遇太宰华督,督说,目而观之。督利孔父妻,乃使人宣言国中曰:“殇公即位十年耳,而十一战,民苦不堪,皆孔父为之,我且杀孔父以宁民。”是岁,鲁弑其君隐公。十年,华督攻杀孔父,取其妻。殇公怒,遂弑殇公,而迎穆公子冯于郑而立之,是为庄公。

庄公元年,华督为相。九年,执郑之祭仲,要以立突为郑君。祭仲许,竟立突。十九年,庄公卒,子湣公捷立。

湣公七年,齐桓公即位。九年,宋水,鲁使臧文仲往吊水。湣公自罪曰:“寡人以不能事鬼神,政不脩,故水。”臧文仲善此言。此言乃公子子鱼教湣公也。

十年夏,宋伐鲁,战于乘丘,鲁生虏宋南宫万。宋人请万,万归宋。十一年秋,湣公与南宫万猎,因博争行,湣公怒,辱之,曰:“始吾敬若;今若,鲁虏也。”万有力,病此言,遂以局杀湣公于蒙泽。大夫仇牧闻之,以兵造公门。万搏牧,牧齿著门阖死。因杀太宰华督,乃更立公子游为君。诸公子饹萧,公子御说饹亳。万弟南宫牛将兵围亳。冬,萧及宋之诸公子共击杀南宫牛,弑宋新君游而立湣公弟御说,是为桓公。宋万饹陈。宋人请以赂陈。陈人使妇人饮之醇酒,以革裹之,归宋。宋人醢万也。

桓公二年,诸侯伐宋,至郊而去。三年,齐桓公始霸。二十三年,迎卫公子毁于齐,立之,是为卫文公。文公女弟为桓公夫人。秦穆公即位。三十年,桓公病,太子兹甫让其庶兄目夷为嗣。桓公义太子意,竟不听。三十一年春,桓公卒,太子兹甫立,是为襄公。以其庶兄目夷为相。未葬,而齐桓公会诸侯于葵丘,襄公往会。

襄公七年,宋地霣星如雨,与雨偕下;六鶂退蜚,风疾也。

八年,齐桓公卒,宋欲为盟会。十二年春,宋襄公为鹿上之盟,以求诸侯于楚,楚人许之。公子目夷谏曰:“小国争盟,祸也。”不听。秋,诸侯会宋公盟于盂。目夷曰:“祸其在此乎?君欲已甚,何以堪之!”于是楚执宋襄公以伐宋。冬,会于亳,以释宋公。子鱼曰:“祸犹未也。”十三年夏,宋伐郑。子鱼曰:“祸在此矣。”秋,楚伐宋以救郑。襄公将战,子鱼谏曰:“天之弃商久矣,不可。”冬,十一月,襄公与楚成王战于泓。楚人未济,目夷曰:“彼众我寡,及其未济击之。”公不听。已济未陈,又曰:“可击。”公曰:“待其已陈。”陈成,宋人击之。宋师大败,襄公伤股。国人皆怨公。公曰:“君子不困人于戹,不鼓不成列。”子鱼曰:“兵以胜为功,何常言与!必如公言,即奴事之耳,又何战为?”

楚成王已救郑,郑享之;去而取郑二姬以归。叔瞻曰:“成王无礼,其不没乎?为礼卒于无别,有以知其不遂霸也。”

是年,晋公子重耳过宋,襄公以伤于楚,欲得晋援,厚礼重耳以马二十乘。

十四年夏,襄公病伤于泓而竟卒,成公元年,晋文公即位。三年,倍楚盟亲晋,以有德于文公也。四年,楚成王伐宋,宋告急于晋。五年,晋文公救宋,楚兵去。九年,晋文公卒。十一年,楚太子商臣弑其父成王代立。十六年,秦穆公卒。

十七年,成公卒。成公弟御杀太子及大司马公孙固而自立为君。宋人共杀君御而立成公少子杵臼,是为昭公。

昭公四年,宋败长翟缘斯于长丘。七年,楚庄王即位。

九年,昭公无道,国人不附。昭公弟鲍革贤而下士。先,襄公夫人欲通于公子鲍,不可,乃助之施于国,因大夫华元为右师。昭公出猎,夫人王姬使卫伯攻杀昭公杵臼。弟鲍革立,是为文公。

文公元年,晋率诸侯伐宋,责以弑君。闻文公定立,乃去。二年,昭公子因文公母弟须与武、缪、戴、庄、桓之族为乱,文公尽诛之,出武、缪之族。

四年春,楚命郑伐宋。宋使华元将,郑败宋,囚华元。华元之将战,杀羊以食士,其御羊羹不及,故怨,驰入郑军,故宋师败,得囚华元。宋以兵车百乘文马四百匹赎华元。未尽入,华元亡归宋。

十四年,楚庄王围郑。郑伯降楚,楚复释之。

十六年,楚使过宋,宋有前仇,执楚使。九月,楚庄王围宋。十七年,楚以围宋五月不解,宋城中急,无食,华元乃夜私见楚将子反。子反告庄王。王问:“城中何如?”曰:“析骨而炊,易子而食。”庄王曰:“诚哉言!我军亦有二日粮。”以信故,遂罢兵去。

二十二年,文公卒,子共公瑕立。始厚葬。君子讥华元不臣矣。

共公十年,华元善楚将子重,又善晋将栾书,两盟晋楚。十三年,共公卒。华元为右师,鱼石为左师。司马唐山攻杀太子肥,欲杀华元,华元饹晋,鱼石止之,至河乃还,诛唐山。乃立共公少子成,是为平公。

平公三年,楚共王拔宋之彭城,以封宋左师鱼石。四年,诸侯共诛鱼石,而归彭城于宋。三十五年,楚公子围弑其君自立,为灵王。四十四年,平公卒,子元公佐立。

元公三年,楚公子弃疾弑灵王,自立为平王。八年,宋火。十年,元公毋信,诈杀诸公子,大夫华、向氏作乱。楚平王太子建来饹,见诸华氏相攻乱,建去如郑。十五年,元公为鲁昭公避季氏居外,为之求入鲁,行道卒,子景公头曼立。

景公十六年,鲁阳虎来饹,已复去。二十五年,孔子过宋,宋司马桓魋恶之,欲杀孔子,孔子微服去。三十年,曹倍宋,又倍晋,宋伐曹,晋不救,遂灭曹有之。三十六年,齐田常弑简公。

三十七年,楚惠王灭陈。荧惑守心。心,宋之分野也。景公忧之。司星子韦曰:“可移于相。”景公曰:“相,吾之股肱。”曰:“可移于民。”景公曰:“君者待民。”曰:“可移于岁。”景公曰:“岁饥民困,吾谁为君!”子韦曰:“天高听卑。君有君人之言三,荧惑宜有动。”于是候之,果徙三度。

六十四年,景公卒。宋公子特攻杀太子而自立,是为昭公。昭公者,元公之曾庶孙也。昭公父公孙纠,纠父公子珰秦,珰秦即元公少子也。景公杀昭公父纠,故昭公怨杀太子而自立。

昭公四十七年卒,子悼公购由立。悼公八年卒,子休公田立。休公田二十三年卒,子辟公辟兵立。辟公三年卒,子剔成立。剔成四十一年,剔成弟偃攻袭剔成,剔成败奔齐,偃自立为宋君。

君偃十一年,自立为王。东败齐,取五城;南败楚,取地三百里;西败魏军,乃与齐、魏为敌国。盛血以韦囊,县而射之,命曰“射天”。淫于酒妇人。群臣谏者辄射之。于是诸侯皆曰“桀宋”。“宋其复为纣所为,不可不诛”。告齐伐宋。王偃立四十七年,齐湣王与魏、楚伐宋,杀王偃,遂灭宋而三分其地。

太史公曰:孔子称“微子去之,箕子为之奴,比干谏而死,殷有三仁焉”。春秋讥宋之乱自宣公废太子而立弟,国以不宁者十世。襄公之时,修行仁义,欲为盟主。其大夫正考父美之,故追道契、汤、高宗,殷所以兴,作商颂。襄公既败于泓,而君子或以为多,伤中国阙礼义,襃之也,宋襄之有礼让也。

殷有三仁,微、箕纣亲。一囚一去,不顾其身。颂美有客,书称作宾。卒传冢嗣,或叙彝伦。微仲之后,世载忠勤。穆亦能让,实为知人。伤泓之役,有君无臣。偃号“桀宋”,天之弃殷。
\end{yuanwen}

\part{卷三十九}
\chapter{晋世家第九}

\begin{yuanwen}
晋唐叔虞者,周武王子而成王弟。初,武王与叔虞母会时,梦天谓武王曰:“余命女\footnote{同“汝”。}生子,名虞,余与之唐。”及生子,文在其手曰“虞”,故遂因命之曰虞。
\end{yuanwen}

晋国的唐叔虞,是周武王的儿子,周成王的弟弟。起初,周武王与叔虞的母亲在一起时,叔虞的母亲梦见上天对周武王说:“我让你生个孩子,起名叫虞,我要把唐这个地方赐给他。”待武夫人生下孩子后,发现婴儿手心上果真写着“虞”字,因此就給这个孩子取名为虞。

\begin{yuanwen}
武王崩,成王立,唐有乱,周公诛灭唐。成王与叔虞戏,削桐叶为珪\footnote{玉制礼器名,上圆下方。}以与叔虞,曰:“以此封若\footnote{你。}。”

史佚因请择日立叔虞。成王曰:“吾与之戏耳。”

史佚曰:“天子无戏言。言则史书之,礼成之,乐歌之。”

于是遂封叔虞于唐。唐在河、汾之东,方百里,故曰唐叔虞。姓姬氏,字子于。
\end{yuanwen}

周武王驾崩后,周成王继位,唐爆发了内乱,周公派兵灭掉唐国。周成王与叔虞一起玩耍,成王将梧桐叶削成了珪璧的形状赠给了叔虞,说:“把这块唐地分封给你。”

史佚于是请求成王选择一个吉日册封叔虞为诸侯。成王说:“我只不过是与他开玩笑而已。”

史佚说:“天子的话没有戏言。天子说了话,史官都要如实记录,并举行典礼去完成它,吹奏乐曲来歌颂它。”

周成王于是就将唐地封给了叔虞。唐地位于黄河、汾水的东边,方圆百里,因此叔虞又被称为唐叔虞。他姓姬,字子于。

\begin{yuanwen}
唐叔子燮,是为晋侯。晋侯子宁族,是为武侯。武侯之子服人,是为成侯。成侯子福,是为厉侯。厉侯之子宜臼,是为靖侯。靖侯已来,年纪可推。自唐叔至靖侯五世,无其年数。
\end{yuanwen}

唐叔的儿子燮,就是晋侯。晋侯的儿子宁族,就是武侯。武侯的儿子服人,就是成侯。成侯的儿子福,就是厉侯。厉侯的儿子宜臼,就是靖侯。靖侯以后,晋国的年代都能够推算出来了。从唐叔虞到靖侯五代,都没有记录他们在位的年数。

\begin{yuanwen}
靖侯十七年,周厉王迷惑暴虐,国人作乱,厉王出奔于彘,大臣行政,故曰“共和”。
\end{yuanwen}

靖侯十七年(前842年),周厉王昏庸残暴,国人暴动,周厉王出逃前往彘地,周朝由大臣召公、周公共同治理,因此称为“共和”。

\begin{yuanwen}
十八年,靖侯卒,子釐侯司徒立。

釐侯十四年,周宣王初立。

十八年,釐侯卒,子献侯籍立。

献侯十一年卒,子穆侯费王立。
\end{yuanwen}

十八年(前841年),靖侯去世,其子釐侯司徒继位。

釐侯十四年(前827年),周宣王刚刚继位。

十八年(前823年),釐侯去世,其子献侯籍继位。

献侯在位十一年(前812年)去世,其子穆侯费王继位。

\begin{yuanwen}
穆侯四年,取齐女姜氏为夫人。七年,伐条。生太子仇。十年,伐千亩,有功。生少子,名曰成师。晋人师服曰:“异哉,君之命子也!太子曰仇,仇者雠也。少子曰成师,成师大号,成之者也。名,自命也;物,自定也。今適庶名反逆,此后晋其能毋乱乎?”
\end{yuanwen}\begin{yuanwen}

\end{yuanwen}\begin{yuanwen}

\end{yuanwen}\begin{yuanwen}

\end{yuanwen}\begin{yuanwen}

\end{yuanwen}\begin{yuanwen}

\end{yuanwen}\begin{yuanwen}

\end{yuanwen}\begin{yuanwen}

\end{yuanwen}\begin{yuanwen}

\end{yuanwen}\begin{yuanwen}

\end{yuanwen}\begin{yuanwen}

\end{yuanwen}\begin{yuanwen}

\end{yuanwen}\begin{yuanwen}

\end{yuanwen}\begin{yuanwen}

\end{yuanwen}\begin{yuanwen}

\end{yuanwen}\begin{yuanwen}

\end{yuanwen}\begin{yuanwen}

\end{yuanwen}\begin{yuanwen}

\end{yuanwen}\begin{yuanwen}

\end{yuanwen}\begin{yuanwen}

\end{yuanwen}\begin{yuanwen}

\end{yuanwen}\begin{yuanwen}

\end{yuanwen}\begin{yuanwen}

\end{yuanwen}\begin{yuanwen}

\end{yuanwen}\begin{yuanwen}

\end{yuanwen}\begin{yuanwen}

\end{yuanwen}\begin{yuanwen}

\end{yuanwen}\begin{yuanwen}

\end{yuanwen}\begin{yuanwen}

\end{yuanwen}\begin{yuanwen}

\end{yuanwen}\begin{yuanwen}

\end{yuanwen}\begin{yuanwen}

\end{yuanwen}\begin{yuanwen}

\end{yuanwen}\begin{yuanwen}

\end{yuanwen}\begin{yuanwen}

\end{yuanwen}\begin{yuanwen}

\end{yuanwen}\begin{yuanwen}

\end{yuanwen}\begin{yuanwen}

\end{yuanwen}\begin{yuanwen}

\end{yuanwen}\begin{yuanwen}

\end{yuanwen}\begin{yuanwen}

\end{yuanwen}\begin{yuanwen}

\end{yuanwen}
\begin{yuanwen}




二十七年,穆侯卒,弟殇叔自立,太子仇出奔。殇叔三年,周宣王崩。四年,穆侯太子仇率其徒袭殇叔而立,是为文侯。

文侯十年,周幽王无道,犬戎杀幽王,周东徙。而秦襄公始列为诸侯。

三十五年,文侯仇卒,子昭侯伯立。

昭侯元年,封文侯弟成师于曲沃。曲沃邑大于翼。翼,晋君都邑也。成师封曲沃,号为桓叔。靖侯庶孙栾宾相桓叔。桓叔是时年五十八矣,好德,晋国之众皆附焉。君子曰:“晋之乱其在曲沃矣。末大于本而得民心,不乱何待!”

七年,晋大臣潘父弑其君昭侯而迎曲沃桓叔。桓叔欲入晋,晋人发兵攻桓叔。桓叔败,还归曲沃。晋人共立昭侯子平为君,是为孝侯。诛潘父。

孝侯八年,曲沃桓叔卒,子鳝代桓叔,是为曲沃庄伯。孝侯十五年,曲沃庄伯弑其君晋孝侯于翼。晋人攻曲沃庄伯,庄伯复入曲沃。晋人复立孝侯子郄为君,是为鄂侯。

鄂侯二年,鲁隐公初立。

鄂侯六年卒。曲沃庄伯闻晋鄂侯卒,乃兴兵伐晋。周平王使虢公将兵伐曲沃庄伯,庄伯走保曲沃。晋人共立鄂侯子光,是为哀侯。

哀侯二年曲沃庄伯卒,子称代庄伯立,是为曲沃武公。哀侯六年,鲁弑其君隐公。哀侯八年,晋侵陉廷。陉廷与曲沃武公谋,九年,伐晋于汾旁,虏哀侯。晋人乃立哀侯子小子为君,是为小子侯。

小子元年,曲沃武公使韩万杀所虏晋哀侯。曲沃益彊,晋无如之何。

晋小子之四年,曲沃武公诱召晋小子杀之。周桓王使虢仲伐曲沃武公,武公入于曲沃,乃立晋哀侯弟缗为晋侯。

晋侯缗四年,宋执郑祭仲而立突为郑君。晋侯十九年,齐人管至父弑其君襄公。

晋侯二十八年,齐桓公始霸。曲沃武公伐晋侯缗,灭之,尽以其宝器赂献于周釐王。釐王命曲沃武公为晋君,列为诸侯,于是尽并晋地而有之。

曲沃武公已即位三十七年矣,更号曰晋武公。晋武公始都晋国,前即位曲沃,通年三十八年。

武公称者,先晋穆侯曾孙也,曲沃桓叔孙也。桓叔者,始封曲沃。武公,庄伯子也。自桓叔初封曲沃以至武公灭晋也,凡六十七岁,而卒代晋为诸侯。武公代晋二岁,卒。与曲沃通年,即位凡三十九年而卒。子献公诡诸立。

献公元年,周惠王弟穨攻惠王,惠王出奔,居郑之栎邑。

五年,伐骊戎,得骊姬、骊姬弟,俱爱幸之。

八年,士蔿说公曰:“故晋之群公子多,不诛,乱且起。”乃使尽杀诸公子,而城聚都之,命曰绛,始都绛。九年,晋群公子既亡奔虢,虢以其故再伐晋,弗克。十年,晋欲伐虢,士蔿曰:“且待其乱。”

十二年,骊姬生奚齐。献公有意废太子,乃曰:“曲沃吾先祖宗庙所在,而蒲边秦,屈边翟,不使诸子居之,我惧焉。”于是使太子申生居曲沃,公子重耳居蒲,公子夷吾居屈。献公与骊姬子奚齐居绛。晋国以此知太子不立也。太子申生,其母齐桓公女也,曰齐姜,早死。申生同母女弟为秦穆公夫人。重耳母,翟之狐氏女也。夷吾母,重耳母女弟也。献公子八人,而太子申生、重耳、夷吾皆有贤行。及得骊姬,乃远此三子。

十六年,晋献公作二军。公将上军,太子申生将下军,赵夙御戎,毕万为右,伐灭霍,灭魏,灭耿。还,为太子城曲沃,赐赵夙耿,赐毕万魏,以为大夫。士蔿曰:“太子不得立矣。分之都城,而位以卿,先为之极,又安得立!不如逃之,无使罪至。为吴太伯,不亦可乎,犹有令名。”太子不从。卜偃曰:“毕万之后必大。”万,盈数也;魏,大名也。以是始赏,天开之矣。天子曰兆民,诸侯曰万民,今命之大,以从盈数,其必有众。”初,毕万卜仕于晋国,遇屯之比。辛廖占之曰:“吉。”屯固比入,吉孰大焉。其后必蕃昌。”

十七年,晋侯使太子申生伐东山。里克谏献公曰:“太子奉冢祀社稷之粢盛,以朝夕视君膳者也,故曰冢子。君行则守,有守则从,从曰抚军,守曰监国,古之制也。夫率师,专行谋也;誓军旅,君与国政之所图也:非太子之事也。师在制命而已,禀命则不威,专命则不孝,故君之嗣適不可以帅师。君失其官,率师不威,将安用之?”公曰:“寡人有子,未知其太子谁立。”里克不对而退,见太子。太子曰:“吾其废乎?”里克曰:“太子勉之!教以军旅,”不共是惧,何故废乎?且子惧不孝,毋惧不得立。修己而不责人,则免于难。”太子帅师,公衣之偏衣,佩之金玦。里克谢病,不从太子。太子遂伐东山。

十九年,献公曰:“始吾先君庄伯、武公之诛晋乱,而虢常助晋伐我,又匿晋亡公子,果为乱。弗诛,后遗子孙忧。”乃使荀息以屈产之乘假道于虞。虞假道,遂伐虢,取其下阳以归。

献公私谓骊姬曰:“吾欲废太子,以奚齐代之。”骊姬泣曰:“太子之立,诸侯皆已知之,而数将兵,百姓附之,柰何以贱妾之故废適立庶?君必行之,妾自杀也。”骊姬详誉太子,而阴令人谮恶太子,而欲立其子。

二十一年,骊姬谓太子曰:“君梦见齐姜,太子速祭曲沃,归釐于君。”太子于是祭其母齐姜于曲沃,上其荐胙于献公。献公时出猎,置胙于宫中。骊姬使人置毒药胙中。居二日,献公从猎来还,宰人上胙献公,献公欲飨之。骊姬从旁止之,曰:“胙所从来远,宜试之。”祭地,地坟;与犬,犬死;与小臣,小臣死。骊姬泣曰:“太子何忍也!其父而欲弑代之,况他人乎?且君老矣,旦暮之人,曾不能待而欲弑之!”谓献公曰:“太子所以然者,不过以妾及奚齐之故。妾原子母辟之他国,若早自杀,毋徒使母子为太子所鱼肉也。始君欲废之,妾犹恨之;至于今,妾殊自失于此。”太子闻之,奔新城。献公怒,乃诛其傅杜原款。或谓太子曰:“为此药者乃骊姬也,太子何不自辞明之?”太子曰:“吾君老矣,非骊姬,寝不安,食不甘。即辞之,君且怒之。不可。”或谓太子曰:“可奔他国。”太子曰:“被此恶名以出,人谁内我?我自杀耳。”十二月戊申,申生自杀于新城。

此时重耳、夷吾来朝。人或告骊姬曰:“二公子怨骊姬谮杀太子。”骊姬恐,因谮二公子:“申生之药胙,二公子知之。”二子闻之,恐,重耳走蒲,夷吾走屈,保其城,自备守。初,献公使士蔿为二公子筑蒲、屈城,弗就。夷吾以告公,公怒士蔿。士蔿谢曰:“边城少寇,安用之?”退而歌曰:“狐裘蒙茸,一国三公,吾谁適从!”卒就城。及申生死,二子亦归保其城。

二十二年,献公怒二子不辞而去,果有谋矣,乃使兵伐蒲。蒲人之宦者勃鞮命重耳促自杀。重耳逾垣,宦者追斩其衣袪。重耳遂奔翟。使人伐屈,屈城守,不可下。

是岁也,晋复假道于虞以伐虢。虞之大夫宫之奇谏虞君曰:“晋不可假道也,是且灭虞。”虞君曰:“晋我同姓,不宜伐我。”宫之奇曰:“太伯、虞仲,太王之子也,太伯亡去,是以不嗣。虢仲、虢叔,王季之子也,为文王卿士,其记勋在王室,藏于盟府。将虢是灭,何爱于虞?且虞之亲能亲于桓、庄之族乎?桓、庄之族何罪,尽灭之。虞之与虢,脣之与齿,脣亡则齿寒。”虞公不听,遂许晋。宫之奇以其族去虞。其冬,晋灭虢,虢公丑奔周。还,袭灭虞,虏虞公及其大夫井伯百里奚以媵秦穆姬,而修虞祀。荀息牵曩所遗虞屈产之乘马奉之献公,献公笑曰:“马则吾马,齿亦老矣!”

二十三年,献公遂发贾华等伐屈,屈溃。夷吾将奔翟。冀芮曰:“不可,重耳已在矣,今往,晋必移兵伐翟,翟畏晋,祸且及。不如走梁,梁近于秦,秦彊,吾君百岁后可以求入焉。”遂奔梁。二十五年,晋伐翟,翟以重耳故,亦击晋于齧桑,晋兵解而去。

当此时,晋彊,西有河西,与秦接境,北边翟,东至河内。

骊姬弟生悼子。

二十六年夏,齐桓公大会诸侯于葵丘。晋献公病,行后,未至,逢周之宰孔。宰孔曰:“齐桓公益骄,不务德而务远略,诸侯弗平。君弟毋会,毋如晋何。”献公亦病,复还归。病甚,乃谓荀息曰:“吾以奚齐为后,年少,诸大臣不服,恐乱起,子能立之乎?”荀息曰:“能。”献公曰:“何以为验?”对曰:“使死者复生,生者不惭,为之验。”于是遂属奚齐于荀息。荀息为相,主国政。秋九月,献公卒。里克、邳郑欲内重耳,以三公子之徒作乱,谓荀息曰:“三怨将起,秦、晋辅之,子将何如?”荀息曰:“吾不可负先君言。”十月,里克杀奚齐于丧次,献公未葬也。荀息将死之,或曰不如立奚齐弟悼子而傅之,荀息立悼子而葬献公。十一月,里克弑悼子于朝,荀息死之。君子曰:“诗所谓‘白珪之玷,犹可磨也,斯言之玷,不可为也’,其荀息之谓乎!不负其言。”初,献公将伐骊戎,卜曰“齿牙为祸”。及破骊戎,获骊姬,爱之,竟以乱晋。

里克等已杀奚齐、悼子,使人迎公子重耳于翟,欲立之。重耳谢曰:“负父之命出奔,父死不得脩人子之礼侍丧,重耳何敢入!大夫其更立他子。”还报里克,里克使迎夷吾于梁。夷吾欲往,吕省、郤芮曰:“内犹有公子可立者而外求,难信。计非之秦,辅彊国之威以入,恐危。”乃使郤芮厚赂秦,约曰:“即得入,请以晋河西之地与秦。”及遗里克书曰:“诚得立,请遂封子于汾阳之邑。”秦缪公乃发兵送夷吾于晋。齐桓公闻晋内乱,亦率诸侯如晋。秦兵与夷吾亦至晋,齐乃使隰朋会秦俱入夷吾,立为晋君,是为惠公。齐桓公至晋之高梁而还归。

惠公夷吾元年,使邳郑谢秦曰:“始夷吾以河西地许君,今幸得入立。大臣曰:‘地者先君之地,君亡在外,何以得擅许秦者?’寡人争之弗能得,故谢秦。”亦不与里克汾阳邑,而夺之权。四月,周襄王使周公忌父会齐、秦大夫共礼晋惠公。惠公以重耳在外,畏里克为变,赐里克死。谓曰:“微里子寡人不得立。虽然,子亦杀二君一大夫,为子君者不亦难乎?”里克对曰:“不有所废,君何以兴?欲诛之,其无辞乎?乃言为此!臣闻命矣。”遂伏剑而死。于是邳郑使谢秦未还,故不及难。

晋君改葬恭太子申生。秋,狐突之下国,遇申生,申生与载而告之曰:“夷吾无礼,余得请于帝,将以晋与秦,秦将祀余。”狐突对曰:“臣闻神不食非其宗,君其祀毋乃绝乎?君其图之。”申生曰:“诺,吾将复请帝。后十日,新城西偏将有巫者见我焉。”许之,遂不见。及期而往,复见,申生告之曰:“帝许罚有罪矣,弊于韩。”兒乃谣曰:“恭太子更葬矣,后十四年,晋亦不昌,昌乃在兄。”

邳郑使秦,闻里克诛,乃说秦缪公曰:“吕省、郤称、冀芮实为不从。若重赂与谋,出晋君,入重耳,事必就。”秦缪公许之,使人与归报晋,厚赂三子。三子曰:“币厚言甘,此必邳郑卖我于秦。”遂杀邳郑及里克、邳郑之党七舆大夫。邳郑子豹奔秦,言伐晋,缪公弗听。

惠公之立,倍秦地及里克,诛七舆大夫,国人不附。二年,周使召公过礼晋惠公,惠公礼倨,召公讥之。

四年,晋饥,乞籴于秦。缪公问百里奚,”百里奚曰:“天菑流行,国家代有,救菑恤邻,国之道也。与之。”邳郑子豹曰:“伐之。”缪公曰:“其君是恶,其民何罪!”卒与粟,自雍属绛。

五年,秦饥,请籴于晋。晋君谋之,庆郑曰:“以秦得立,已而倍其地约。晋饥而秦贷我,今秦饥请籴,与之何疑?而谋之!”虢射曰:“往年天以晋赐秦,秦弗知取而贷我。今天以秦赐晋,晋其可以逆天乎?遂伐之。”惠公用虢射谋,不与秦粟,而发兵且伐秦。秦大怒,亦发兵伐晋。

六年春,秦缪公将兵伐晋。晋惠公谓庆郑曰:“秦师深矣,柰何?”郑曰:“秦内君,君倍其赂;晋饥秦输粟,秦饥而晋倍之,乃欲因其饥伐之:其深不亦宜乎!”晋卜御右,庆郑皆吉。公曰:“郑不孙。”乃更令步阳御戎,家仆徒为右,进兵。九月壬戌,秦缪公、晋惠公合战韩原。惠公马
不行,秦兵至,公窘,召庆郑为御。郑曰:“不用卜,败不亦当乎!”遂去。更令梁繇靡御,虢射为右,辂秦缪公。缪公壮士冒败晋军,晋军败,遂失秦缪公,反获晋公以归。秦将以祀上帝。晋君姊为缪公夫人,衰绖涕泣。公曰:“得晋侯将以为乐,今乃如此。且吾闻箕子见唐叔之初封,曰‘其后必当大矣’,晋庸可灭乎!”乃与晋侯盟王城而许之归。晋侯亦使吕省等报国人曰:“孤虽得归,毋面目见社稷,卜日立子圉。”晋人闻之,皆哭。秦缪公问吕省:“晋国和乎?”对曰:“不和。小人惧失君亡亲,不惮立子圉,曰‘必报雠,宁事戎、狄’。其君子则爱君而知罪,以待秦命,曰‘必报德’。有此二故,不和。”于是秦缪公更舍晋惠公,餽之七牢。十一月,归晋侯。晋侯至国,诛庆郑,修政教。谋曰:“重耳在外,诸侯多利内之。”欲使人杀重耳于狄。重耳闻之,如齐。

八年,使太子圉质秦。初,惠公亡在梁,梁伯以其女妻之,生一男一女。梁伯卜之,男为人臣,女为人妾,故名男为圉,女为妾。

十年,秦灭梁。梁伯好土功,治城沟,”民力罢怨,其众数相惊,曰“秦寇至”,民恐惑,秦竟灭之。

十三年,晋惠公病,内有数子。太子圉曰:“吾母家在梁,梁今秦灭之,我外轻于秦而内无援于国。君即不起,病大夫轻,更立他公子。”乃谋与其妻俱亡归。秦女曰:“子一国太子,辱在此。秦使婢子侍,以固子之心。子亡矣,我不从子,亦不敢言。”子圉遂亡归晋。十四年九月,惠公卒,太子圉立,是为怀公。

子圉之亡,秦怨之,乃求公子重耳,欲内之。子圉之立,畏秦之伐也。乃令国中诸从重耳亡者与期,期尽不到者尽灭其家。狐突之子毛及偃从重耳在秦,弗肯召。怀公怒,囚狐突。突曰:“臣子事重耳有年数矣,今召之,是教之反君也。何以教之?”怀公卒杀狐突。秦缪公乃发兵送内重耳,使人告栾、郤之党为内应,杀怀公于高梁,入重耳。重耳立,是为文公。

晋文公重耳,晋献公之子也。自少好士,年十七,有贤士五人:曰赵衰;狐偃咎犯,文公舅也;贾佗;先轸;魏武子。自献公为太子时,重耳固已成人矣。献公即位,重耳年二十一。献公十三年,以骊姬故,重耳备蒲城守秦。献公二十一年,献公杀太子申生,骊姬谗之,恐,不辞献公而守蒲城。献公二十二年,献公使宦者履鞮趣杀重耳。重耳逾垣,宦者逐斩其衣袪。重耳遂奔狄。狄,其母国也。是时重耳年四十三。从此五士,其馀不名者数十人,至狄。

狄伐咎如,得二女:以长女妻重耳,生伯鯈、叔刘;以少女妻赵衰,生盾。居狄五岁而晋献公卒,里克已杀奚齐、悼子,乃使人迎,欲立重耳。重耳畏杀,因固谢,不敢入。已而晋更迎其弟夷吾立之,是为惠公。惠公七年,畏重耳,乃使宦者履鞮与壮士欲杀重耳。重耳闻之,乃谋赵衰等曰:“始吾奔狄,非以为可用与,以近易通,故且休足。休足久矣,固原徙之大国。夫齐桓公好善,志在霸王,收恤诸侯。今闻管仲、隰朋死,此亦欲得贤佐,盍往乎?”于是遂行。重耳谓其妻曰:“待我二十五年不来,乃嫁。”其妻笑曰:“犁二十五年,吾冢上柏大矣。虽然,妾待子。”重耳居狄凡十二年而去。

过卫,卫文公不礼。去,过五鹿,饥而从野人乞食,野人盛土器中进之。重耳怒。赵衰曰:“土者,有土也,君其拜受之。”

至齐,齐桓公厚礼,而以宗女妻之,有马二十乘,重耳安之。重耳至齐二岁而桓公卒,会竖刀等为内乱,齐孝公之立,诸侯兵数至。留齐凡五岁。重耳爱齐女,毋去心。赵衰、咎犯乃于桑下谋行。齐女侍者在桑上闻之,以告其主。其主乃杀侍者,劝重耳趣行。重耳曰:“人生安乐,孰知其他!必死于此,不能去。”齐女曰:“子一国公子,穷而来此,数士者以子为命。子不疾反国,报劳臣,而怀女德,窃为子羞之。且不求,何时得功?”乃与赵衰等谋,醉重耳,载以行。行远而觉,重耳大怒,引戈欲杀咎犯。咎犯曰:“杀臣成子,偃之原也。”重耳曰:“事不成,我食舅氏之肉。”咎犯曰:“事不成,犯肉腥臊,何足食!”乃止,遂行。

过曹,曹共公不礼,欲观重耳骈胁。曹大夫釐负羁曰:“晋公子贤,又同姓,穷来过我,柰何不礼!”共公不从其谋。负羁乃私遗重耳食,置璧其下。重耳受其食,还其璧。

去,过宋。宋襄公新困兵于楚,伤于泓,闻重耳贤,乃以国礼礼于重耳。

过郑,郑文公弗礼。郑叔瞻谏其君曰:“晋公子贤,而其从者皆国相,且又同姓。郑之出自厉王,而晋之出自武王。”郑君曰:“诸侯亡公子过此者众,安可尽礼!”叔瞻曰:“君不礼,不如杀之,且后为国患。”郑君不听。

重耳去之楚,楚成王以適诸侯礼待之,重耳谢不敢当。赵衰曰:“子亡在外十馀年,小国轻子,况大国乎?今楚大国而固遇子,子其毋让,此天开子也。”遂以客礼见之。成王厚遇重耳,重耳甚卑。成王曰:“子即反国,何以报寡人?”重耳曰:“羽毛齿角玉帛,君王所馀,未知所以报。”王曰:“虽然,何以报不穀?”重耳曰:“即不得已,与君王以兵车会平原广泽,请辟王三舍。”楚将子玉怒曰:“王遇晋公子至厚,今重耳言不孙,请杀之。”成王曰:“晋公子贤而困于外久,从者皆国器,此天所置,庸可杀乎?且言何以易之!”居楚数月,而晋太子圉亡秦,秦怨之;闻重耳在楚,乃召之。成王曰:“楚远,更数国乃至晋。秦晋接境,秦君贤,子其勉行!”厚送重耳。

重耳至秦,缪公以宗女五人妻重耳,故子圉妻与往。重耳不欲受,司空季子曰:“其国且伐,况其故妻乎!且受以结秦亲而求入,子乃拘小礼,忘大丑乎!”遂受。缪公大欢,与重耳饮。赵衰歌黍苗诗。缪公曰:“知子欲急反国矣。”赵衰与重耳下,再拜曰:“孤臣之仰君,如百穀之望时雨。”是时晋惠公十四年秋。惠公以九月卒,子圉立。十一月,葬惠公。十二月,晋国大夫栾、郤等闻重耳在秦,皆阴来劝重耳、赵衰等反国,为内应甚众。于是秦缪公乃发兵与重耳归晋。晋闻秦兵来,亦发兵拒之。然皆阴知公子重耳入也。唯惠公之故贵臣吕、郤之属不欲立重耳。重耳出亡凡十九岁而得入,时年六十二矣,晋人多附焉。

文公元年春,秦送重耳至河。咎犯曰:“臣从君周旋天下,过亦多矣。臣犹知之,况于君乎?请从此去矣。”重耳曰:“若反国,所不与子犯共者,河伯视之!”乃投璧河中,以与子犯盟。是时介子推从,在船中,乃笑曰:“天实开公子,而子犯以为己功而要市于君,固足羞也。吾不忍与同位。”乃自隐渡河。秦兵围令狐,晋军于庐柳。二月辛丑,咎犯与秦晋大夫盟于郇。壬寅,重耳入于晋师。丙午,入于曲沃。丁未,朝于武宫,即位为晋君,是为文公。群臣皆往。怀公圉奔高梁。戊申,使人杀怀公。

怀公故大臣吕省、郤芮本不附文公,文公立,恐诛,乃欲与其徒谋烧公宫,杀文公。文公不知。始尝欲杀文公宦者履鞮知其谋,欲以告文公,解前罪,求见文公。文公不见,使人让曰:“蒲城之事,女斩予袪。其后我从狄君猎,女为惠公来求杀我。惠公与女期三日至,而女一日至,何速也?女其念之。”宦者曰:“臣刀锯之馀,不敢以二心事君倍主,故得罪于君。君已反国,其毋蒲、翟乎?且管仲射钩,桓公以霸。今刑馀之人以事告而君不见,祸又且及矣。”于是见之,遂以吕、郤等告文公。文公欲召吕、郤,吕、郤等党多,文公恐初入国,国人卖己,乃为微行,会秦缪公于王城,国人莫知。三月己丑,吕、郤等果反,焚公宫,不得文公。文公之卫徒与战,吕、郤等引兵欲奔,秦缪公诱吕、郤等,杀之河上,晋国复而文公得归。夏,迎夫人于秦,秦所与文公妻者卒为夫人。秦送三千人为卫,以备晋乱。

文公修政,施惠百姓。赏从亡者及功臣,大者封邑,小者尊爵。未尽行赏,周襄王以弟带难出居郑地,来告急晋。晋初定,欲发兵,恐他乱起,是以赏从亡未至隐者介子推。推亦不言禄,禄亦不及。推曰:“献公子九人,唯君在矣。惠、怀无亲,外内弃之;天未绝晋,必将有主,主晋祀者,非君而谁?天实开之,二三子以为己力,不亦诬乎?窃人之财,犹曰是盗,况贪天之功以为己力乎?下冒其罪,上赏其奸,上下相蒙,难与处矣!”其母曰:“盍亦求之,以死谁懟?”推曰:“尤而效之,罪有甚焉。且出怨言,不食其禄。”母曰:“亦使知之,若何?”对曰:“言,身之文也;身欲隐,安用文之?文之,是求显也。”其母曰:“能如此乎?与女偕隐。”至死不复见。

介子推从者怜之,乃悬书宫门曰:“龙欲上天,五蛇为辅。龙已升云,四蛇各入其宇,一蛇独怨,终不见处所。”文公出,见其书,曰:“此介子推也。吾方忧王室,未图其功。”使人召之,则亡。遂求所在,闻其入釂上山中,于是文公环绵上山中而封之,以为介推田,号曰介山,“以记吾过,且旌善人”。

从亡贱臣壶叔曰;“君三行赏,赏不及臣,敢请罪。”文公报曰:“夫导我以仁义,防我以德惠,此受上赏。辅我以行,卒以成立,此受次赏。矢石之难,汗马之劳,此复受次赏。若以力事我而无补吾缺者,此复受次赏。三赏之后,故且及子。”晋人闻之,皆说。

二年春,秦军河上,将入王。赵衰曰;“求霸莫如入王尊周。周晋同姓,晋不先入王,后秦入之,毋以令于天下。方今尊王,晋之资也。”三月甲辰,晋乃发兵至阳樊,围温,入襄王于周。四月,杀王弟带。周襄王赐晋河内阳樊之地。

四年,楚成王及诸侯围宋,宋公孙固如晋告急。先轸曰:“报施定霸,于今在矣。”狐偃曰:“楚新得曹而初婚于卫,若伐曹、卫,楚必救之,则宋免矣。”于是晋作三军。赵衰举郤縠将中军,郤臻佐之;使狐偃将上军,狐毛佐之,命赵衰为卿;栾枝将下军,先轸佐之;荀林父御戎,魏焠为右:往伐。冬十二月,晋兵先下山东,而以原封赵衰。

五年春,晋文公欲伐曹,假道于卫,卫人弗许。还自河南度,侵曹,伐卫。正月,取五鹿。二月,晋侯、齐侯盟于敛盂。”卫侯请盟晋,晋人不许。卫侯欲与楚,国人不欲,故出其君以说晋。卫侯居襄牛,公子买守卫。楚救卫,不卒。晋侯围曹。三月丙午,晋师入曹,数之以其不用釐负羁言,而用美女乘轩者三百人也。令军毋入僖负羁宗家以报德。楚围宋,宋复告急晋。文公欲救则攻楚,为楚尝有德,不欲伐也;欲释宋,宋又尝有德于晋:患之。先轸曰:“执曹伯,分曹、卫地以与宋,楚急曹、卫,其势宜释宋。”于是文公从之,而楚成王乃引兵归。

楚将子玉曰:“王遇晋至厚,今知楚急曹、卫而故伐之,是轻王。”王曰:“晋侯亡在外十九年,困日久矣,果得反国,险戹尽知之,能用其民,天之所开,不可当。”子玉请曰:“非敢必有功,原以间执谗慝之口也。”楚王怒,少与之兵。于是子玉使宛春告晋:“请复卫侯而封曹,臣亦释宋。”咎犯曰:“子玉无礼矣,君取一,臣取二,勿许。”先轸曰:“定人之谓礼。楚一言定三国,子一言而亡之,我则毋礼。不许楚,是弃宋也。不如私许曹、卫以诱之,执宛春以怒楚,既战而后图之。”晋侯乃囚宛春于卫,且私许复曹、卫。曹、卫告绝于楚。楚得臣怒,击晋师,晋师退。军吏曰:“为何退?”文公曰:“昔在楚,约退三舍,可倍乎!”楚师欲去,得臣不肯。四月戊辰,宋公、齐将、秦将与晋侯次城濮。己巳,与楚兵合战,楚兵败,得臣收馀兵去。甲午,晋师还至衡雍,作王宫于践土。

初,郑助楚,楚败,惧,使人请盟晋侯。晋侯与郑伯盟。

五月丁未,献楚俘于周,驷介百乘,徒兵千。天子使王子虎命晋侯为伯,赐大辂,彤弓矢百,玈弓矢千,秬鬯一卣,珪瓚,虎贲三百人。晋侯三辞,然后稽首受之。周作晋文侯命:“王若曰:父义和,丕显文、武,能慎明德,昭登于上,布闻在下,维时上帝集厥命于文、武。恤朕身、继予一人永其在位。”于是晋文公称伯。癸亥,王子虎盟诸侯于王庭。

晋焚楚军,火数日不息,文公叹。左右曰:“胜楚而君犹忧,何?”文公曰:“吾闻能战胜安者唯圣人,是以惧。且子玉犹在,庸可喜乎!”子玉之败而归,楚成王怒其不用其言,贪与晋战,让责子玉,子玉自杀。晋文公曰:“我击其外,楚诛其内,内外相应。”于是乃喜。

六月,晋人复入卫侯。壬午,晋侯度河北归国。行赏,狐偃为首。或曰:“城濮之事,先轸之谋。”文公曰:“城濮之事,偃说我毋失信。先轸曰‘军事胜为右’,吾用之以胜。然此一时之说,偃言万世之功,柰何以一时之利而加万世功乎?是以先之。”

冬,晋侯会诸侯于温,欲率之朝周。力未能,恐其有畔者,乃使人言周襄王狩于河阳。壬申,遂率诸侯朝王于践土。孔子读史记至文公,曰“诸侯无召王”、“王狩河阳”者,春秋讳之也。

丁丑,诸侯围许。曹伯臣或说晋侯曰:“齐桓公合诸侯而国异姓,今君为会而灭同姓。曹,叔振铎之后;晋,唐叔之后。合诸侯而灭兄弟,非礼。”晋侯说,复曹伯。

于是晋始作三行。荀林父将中行,先縠将右行,先蔑将左行。

七年,晋文公、秦缪公共围郑,以其无礼于文公亡过时,及城濮时郑助楚也。围郑,欲得叔瞻。叔瞻闻之,自杀。郑持叔瞻告晋。晋曰:“必得郑君而甘心焉。”郑恐,乃间令使谓秦缪公曰:“亡郑厚晋,于晋得矣,而秦未为利。君何不解郑,得为东道交?”秦伯说,罢兵。晋亦罢兵。

九年冬,晋文公卒,子襄公欢立。是岁郑伯亦卒。

郑人或卖其国于秦,秦缪公发兵往袭郑。十二月,秦兵过我郊。襄公元年春,秦师过周,无礼,王孙满讥之。兵至滑,郑贾人弦高将市于周,遇之,以十二牛劳秦师。秦师惊而还,灭滑而去。

晋先轸曰:“秦伯不用蹇叔,反其众心,此可击。”栾枝曰:“未报先君施于秦,击之,不可。”先轸曰:“秦侮吾孤,伐吾同姓,何德之报?”遂击之。襄公墨衰绖。四月,败秦师于殽,虏秦三将孟明视、西乞秫、白乙丙以归。遂墨以葬文公。文公夫人秦女,谓襄公曰:“秦欲得其三将戮之。”公许,遣之。先轸闻之,谓襄公曰:“患生矣。”轸乃追秦将。秦将渡河,已在船中,顿首谢,卒不反。

后三年,秦果使孟明伐晋,报殽之败,取晋汪以归。四年,秦缪公大兴兵伐我,度河,取王官,封殽尸而去。晋恐,不敢出,遂城守。五年,晋伐秦,取新城,报王官役也。

六年,赵衰成子、栾贞子、咎季子犯、霍伯皆卒。赵盾代赵衰执政。

七年八月,襄公卒。太子夷皋少。晋人以难故,欲立长君。赵盾曰:“立襄公弟雍。好善而长,先君爱之;且近于秦,秦故好也。立善则固,事长则顺,奉爱则孝,结旧好则安。”贾季曰:“不如其弟乐。辰嬴嬖于二君,立其子,民必安之。”赵盾曰:“辰嬴贱,班在九人下,其子何震之有!且为二君嬖,淫也。为先君子,不能求大而出在小国,僻也。母淫子僻,无威;陈小而远,无援:将何可乎!”使士会如秦迎公子雍。贾季亦使人召公子乐于陈。赵盾废贾季,以其杀阳处父。十月,葬襄公。十一月,贾季奔翟。是岁,秦缪公亦卒。

灵公元年四月,秦康公曰:“昔文公之入也无卫,故有吕、郤之患。”乃多与公子雍卫。太子母缪嬴日夜抱太子以号泣于朝,曰:“先君何罪?其嗣亦何罪?舍適而外求君,将安置此?”出朝,则抱以適赵盾所,顿首曰:“先君奉此子而属之子,曰‘此子材,吾受其赐;不材,吾怨子’。今君卒,言犹在耳,而弃之,若何?”赵盾与诸大夫皆患缪嬴,且畏诛,乃背所迎而立太子夷皋,是为灵公。发兵以距秦送公子雍者。赵盾为将,往击秦,败之令狐。先蔑、随会亡奔秦。秋,齐、宋、卫、郑、曹、许君皆会赵盾,盟于扈,以灵公初立故也。

四年,伐秦,取少梁。秦亦取晋之郩。六年,秦康公伐晋,取羁马。晋侯怒,使赵盾、赵穿、郤缺击秦,大战河曲,赵穿最有功。七年,晋六卿患随会之在秦,常为晋乱,乃详令魏寿馀反晋降秦。秦使随会之魏,因执会以归晋。

八年,周顷王崩,公卿争权,故不赴。晋使赵盾以车八百乘平周乱而立匡王。是年,楚庄王初即位。十二年,齐人弑其君懿公。

十四年,灵公壮,侈,厚敛以彫墙。”从台上弹人,观其避丸也。宰夫胹熊蹯不熟,灵公怒,杀宰夫,使妇人持其尸出弃之,过朝。赵盾、随会前数谏,不听;已又见死人手,二人前谏。随会先谏,不听。灵公患之,使鉏麑刺赵盾。盾闺门开,居处节,鉏麑退,叹曰:“杀忠臣,弃君命,罪一也。”遂触树而死。

初,盾常田首山,见桑下有饿人。饿人,示眯明也。盾与之食,食其半。问其故,曰:“宦三年,未知母之存不,原遗母。”盾义之,益与之饭肉。已而为晋宰夫,赵盾弗复知也。九月,晋灵公饮赵盾酒,伏甲将攻盾。公宰示眯明知之,恐盾醉不能起,而进曰:“君赐臣,觞三行可以罢。”欲以去赵盾,令先,毋及难。盾既去,灵公伏士未会,先纵齧狗名敖。明为盾搏杀狗。盾曰:“弃人用狗,虽猛何为。”然不知明之为阴德也。已而灵公纵伏士出逐赵盾,示眯明反击灵公之伏士,伏士不能进,而竟脱盾。盾问其故,曰:“我桑下饿人。”问其名,弗告。明亦因亡去。

盾遂奔,未出晋境。乙丑,盾昆弟将军赵穿袭杀灵公于桃园而迎赵盾。赵盾素贵,得民和;灵公少,侈,民不附,故为弑易。盾复位。晋太史董狐书曰“赵盾弑其君”,以视于朝。盾曰:“弑者赵穿,我无罪。”太史曰:“子为正卿,而亡不出境,反不诛国乱,非子而谁?”孔子闻之,曰:“董狐,古之良史也,书法不隐。宣子,良大夫也,为法受恶。惜也,出疆乃免。”

赵盾使赵穿迎襄公弟黑臀于周而立之,是为成公。

成公者,文公少子,其母周女也。壬申,朝于武宫。

成公元年,赐赵氏为公族。伐郑,郑倍晋故也。三年,郑伯初立,附晋而弃楚。楚怒,伐郑,晋往救之。

六年,伐秦,虏秦将赤。

七年,成公与楚庄王争彊,会诸侯于扈。陈畏楚,不会。晋使中行桓子伐陈,因救郑,与楚战,败楚师。是年,成公卒,子景公据立。

景公元年春,陈大夫夏徵舒弑其君灵公。二年,楚庄王伐陈,诛徵舒。

三年,楚庄王围郑,郑告急晋。晋使荀林父将中军,随会将上军,赵朔将下军,郤克、栾书、先縠、韩厥、巩朔佐之。六月,至河。闻楚已服郑,郑伯肉袒与盟而去,荀林父欲还。先縠曰:“凡来救郑,不至不可,将率离心。”卒度河。楚已服郑,欲饮马于河为名而去。楚与晋军大战。郑新附楚,畏之,反助楚攻晋。晋军败,走河,争度,船中人指甚众。楚虏我将智
。归而林父曰:“臣为督将,军败当诛,请死。”景公欲许之。随会曰:“昔文公之与楚战城濮,成王归杀子玉,而文公乃喜。今楚已败我师,又诛其将,是助楚杀仇也。”乃止。

四年,先縠以首计而败晋军河上,恐诛,乃奔翟,与翟谋伐晋。晋觉,乃族縠。縠,先轸子也。

五年,伐郑,为助楚故也。是时楚庄王彊,以挫晋兵河上也。

六年,楚伐宋,宋来告急晋,晋欲救之,伯宗谋曰:“楚,天方开之,不可当。”乃使解扬绐为救宋。郑人执与楚,楚厚赐,使反其言,令宋急下。解扬绐许之,卒致晋君言。楚欲杀之,或谏,乃归解扬。

七年,晋使随会灭赤狄。

八年,使郤克于齐。齐顷公母从楼上观而笑之。所以然者,郤克偻,而鲁使蹇,卫使眇,故齐亦令人如之以导客。郤克怒,归至河上,曰:“不报齐者,河伯视之!”至国,请君,欲伐齐。景公问知其故,曰:“子之怨,安足以烦国!”弗听。魏文子请老休,辟郤克,克执政。

九年,楚庄王卒。晋伐齐,齐使太子彊为质于晋,晋兵罢。

十一年春,齐伐鲁,取隆。鲁告急卫,卫与鲁皆因郤克告急于晋。晋乃使郤克、栾书、韩厥以兵车八百乘与鲁、卫共伐齐。夏,与顷公战于鞍,伤困顷公。顷公乃与其右易位,下取饮,以得脱去。齐师败走,晋追北至齐。顷公献宝器以求平,不听。郤克曰:“必得萧桐侄子为质。”齐使曰:“萧桐侄子,顷公母;顷公母犹晋君母,柰何必得之?不义,请复战。”晋乃许与平而去。

楚申公巫臣盗夏姬以奔晋,晋以巫臣为邢大夫。

十二年冬,齐顷公如晋,欲上尊晋景公为王,景公让不敢。晋始作六军,韩厥、巩朔、赵穿、荀骓、

十三年,鲁成公朝晋,晋弗敬,鲁怒去,倍晋。晋伐郑,取氾。

十四年,梁山崩。问伯宗,伯宗以为不足怪也。

十六年,楚将子反怨巫臣,灭其族。巫臣怒,遗子反书曰:“必令子罢于奔命!”乃请使吴,令其子为吴行人,教吴乘车用兵。吴晋始通,约伐楚。

十七年,诛赵同、赵括,族灭之。韩厥曰:“赵衰、赵盾之功岂可忘乎?柰何绝祀!”乃复令赵庶子武为赵后,复与之邑。

十九年夏,景公病,立其太子寿曼为君,是为厉公。后月馀,景公卒。

厉公元年,初立,欲和诸侯,与秦桓公夹河而盟。归而秦倍盟,与翟谋伐晋。三年,使吕相让秦,因与诸侯伐秦。至泾,败秦于麻隧,虏其将成差。

五年,三郤谗伯宗,杀之。伯宗以好直谏得此祸,国人以是不附厉公。

六年春,郑倍晋与楚盟,晋怒。栾书曰:“不可以当吾世而失诸侯。”乃发兵。厉公自将,五月度河。闻楚兵来救,范文子请公欲还。郤至曰:“发兵诛逆,见彊辟之,无以令诸侯。”遂与战。癸巳,射中楚共王目,楚兵败于鄢陵。子反收馀兵,拊循欲复战,晋患之。共王召子反,其侍者竖阳穀进酒,子反醉,不能见。王怒,让子反,子反死。王遂引兵归。晋由此威诸侯,欲以令天下求霸。

厉公多外嬖姬,归,欲尽去群大夫而立诸姬兄弟。宠姬兄曰胥童,尝与郤至有怨,及栾书又怨郤至不用其计而遂败楚,乃使人间谢楚。楚来诈厉公曰:“鄢陵之战,实至召楚,欲作乱,内子周立之。会与国不具,是以事不成。”厉公告栾书。栾书曰:“其殆有矣!原公试使人之周微考之。”果使郤至于周。栾书又使公子周见郤至,郤至不知见卖也。厉公验之,信然,遂怨郤至,欲杀之。八年,厉公猎,与姬饮,郤至杀豕奉进,宦者夺之。郤至射杀宦者。公怒,曰:“季子欺予!”将诛三郤,未发也。郤锜欲攻公,曰:“我虽死,公亦病矣。”郤至曰:“信不反君,智不害民,勇不作乱。失此三者,谁与我?我死耳!”十二月壬午,公令胥童以兵八百人袭攻杀三郤。胥童因以劫栾书、中行偃于朝,曰:“不杀二子,患必及公。”公曰:“一旦杀三卿,寡人不忍益也。”对曰:“人将忍君。”公弗听,谢栾书等以诛郤氏罪:“大夫复位。”二子顿首曰:“幸甚幸甚!”公使胥童为卿。闰月乙卯,厉公游匠骊氏,栾书、中行偃以其党袭捕厉公,囚之,杀胥童,而使人迎公子周于周而立之,是为悼公。

悼公元年正月庚申,栾书、中行偃弑厉公,葬之以一乘车。厉公囚六日死,死十日庚午,智
迎公子周来,至绛,刑鸡与大夫盟而立之,是为悼公。辛巳,朝武宫。二月乙酉,即位。

悼公周者,其大父捷,晋襄公少子也,不得立,号为桓叔,桓叔最爱。桓叔生惠伯谈,谈生悼公周。周之立,年十四矣。悼公曰:“大父、父皆不得立而辟难于周,客死焉。寡人自以疏远,毋几为君。今大夫不忘文、襄之意而惠立桓叔之后,赖宗庙大夫之灵,得奉晋祀,岂敢不战战乎?大夫其亦佐寡人!”于是逐不臣者七人,修旧功,施德惠,收文公入时功臣后。秋,伐郑。郑师败,遂至陈。

三年,晋会诸侯。悼公问群臣可用者,祁傒举解狐。解狐,傒之仇。复问,举其子祁午。君子曰:“祁傒可谓不党矣!外举不隐仇,内举不隐子。”方会诸侯,悼公弟杨干乱行,魏绛戮其仆。悼公怒,或谏公,公卒贤绛,任之政,使和戎,戎大亲附。十一年,悼公曰:“自吾用魏绛,九合诸侯,和戎、翟,魏子之力也。”赐之乐,三让乃受之。冬,秦取我栎。

十四年,晋使六卿率诸侯伐秦,度泾,大败秦军,至棫林而去。

十五年,悼公问治国于师旷。师旷曰:“惟仁义为本。”冬,悼公卒,子平公彪立。

平公元年,伐齐,齐灵公与战靡下,齐师败走。晏婴曰:“君亦毋勇,何不止战?”遂去。晋追,遂围临菑,尽烧屠其郭中。东至胶,南至沂,齐皆城守,晋乃引兵归。

六年,鲁襄公朝晋。晋栾逞有罪,奔齐。八年,齐庄公微遣栾逞于曲沃,以兵随之。齐兵上太行,栾逞从曲沃中反,袭入绛。绛不戒,平公欲自杀,范献子止公,以其徒击逞,逞败走曲沃。曲沃攻逞,逞死,遂灭栾氏宗。逞者,栾书孙也。其入绛,与魏氏谋。齐庄公闻逞败,乃还,取晋之朝歌去,以报临菑之役也。

十年,齐崔杼弑其君庄公。晋因齐乱,伐败齐于高唐去,报太行之役也。

十四年,吴延陵季子来使,与赵文子、韩宣子、魏献子语,曰:“晋国之政,卒归此三家矣。”

十九年,齐使晏婴如晋,与叔乡语。叔乡曰:“晋,季世也。公厚赋为台池而不恤政,政在私门,其可久乎!”晏子然之。

二十二年,伐燕。二十六年,平公卒,子昭公夷立。

昭公六年卒。六卿彊,公室卑。子顷公去疾立。

顷公六年,周景王崩,王子争立。晋六卿平王室乱,立敬王。

九年,鲁季氏逐其君昭公,昭公居乾侯。十一年,卫、宋使使请晋纳鲁君。季平子私赂范献子,献子受之,乃谓晋君曰:“季氏无罪。”不果入鲁君。

十二年,晋之宗家祁傒孙,叔乡子,相恶于君。六卿欲弱公室,乃遂以法尽灭其族。而分其邑为十县,各令其子为大夫。晋益弱,六卿皆大。

十四年,顷公卒,子定公午立。

定公十一年,鲁阳虎奔晋,赵鞅简子舍之。十二年,孔子相鲁。

十五年,赵鞅使邯郸大夫午,不信,欲杀午,午与中行寅、范吉射亲攻赵鞅,鞅走保晋阳。定公围晋阳。荀栎、韩不信、魏侈与范、中行为仇,乃移兵伐范、中行。范、中行反,晋君击之,败范、中行。范、中行走朝歌,保之。韩、魏为赵鞅谢晋君,乃赦赵鞅,复位。二十二年,晋败范、中行氏,二子奔齐。

三十年,定公与吴王夫差会黄池,争长,赵鞅时从,卒长吴。

三十一年,齐田常弑其君简公,而立简公弟骜为平公。三十三年,孔子卒。

三十七年,定公卒,子出公凿立。

出公十七年,”知伯与赵、韩、魏共分范、中行地以为邑。出公怒,告齐、鲁,欲以伐四卿。四卿恐,遂反攻出公。出公奔齐,道死。故知伯乃立昭公曾孙骄为晋君,是为哀公。

哀公大父雍,晋昭公少子也,号为戴子。戴子生忌。忌善知伯,蚤死,故知伯欲尽并晋,未敢,乃立忌子骄为君。当是时,晋国政皆决知伯,晋哀公不得有所制。知伯遂有范、中行地,最彊。

哀公四年,赵襄子、韩康子、魏桓子共杀知伯,尽并其地。

十八年,哀公卒,子幽公柳立。

幽公之时,晋畏,反朝韩、赵、魏之君。独有绛、曲沃,馀皆入三晋。

十五年,魏文侯初立。十八年,幽公淫妇人,夜窃出邑中,盗杀幽公。魏文侯以兵诛晋乱,立幽公子止,是为烈公。

烈公十九年,周威烈王赐赵、韩、魏皆命为诸侯。

二十七年,烈公卒,子孝公颀立。孝公九年,魏武侯初立,袭邯郸,不胜而去。十七年,孝公卒,子静公俱酒立。是岁,齐威王元年也。

静公二年,魏武侯、韩哀侯、赵敬侯灭晋后而三分其地。静公迁为家人,晋绝不祀。

太史公曰:晋文公,古所谓明君也,亡居外十九年,至困约,及即位而行赏,尚忘介子推,况骄主乎?灵公既弑,其后成、景致严,至厉大刻,大夫惧诛,祸作。悼公以后日衰,六卿专权。故君道之御其臣下。固不易哉!

天命叔虞,卒封于唐。桐珪既削,河、汾是荒。文侯虽嗣,曲沃日彊。未知本末,祚倾桓庄。献公昏惑,太子罹殃。重耳致霸,朝周河阳。灵既丧德,厉亦无防。四卿侵侮。晋祚遽亡。
\end{yuanwen}

\chapter{楚世家}

\begin{yuanwen}
楚之先祖出自帝颛顼高阳。高阳者,黄帝之孙,昌意之子也。高阳生称,称生卷章,卷章生重黎。重黎为帝喾高辛居火正,甚有功,能光融天下,帝喾命曰祝融。共工氏作乱,帝喾使重黎诛之而不尽。帝乃以庚寅日诛重黎,而以其弟吴回为重黎后,复居火正,为祝融。

吴回生陆终。陆终生子六人,坼剖而产焉。其长一曰昆吾;二曰参胡;三曰彭祖;四曰会人;五曰曹姓;六曰季连,琇姓,楚其后也。昆吾氏,夏之时尝为侯伯,桀之时汤灭之。彭祖氏,殷之时尝为侯伯,殷之末世灭彭祖氏。季连生附沮,附沮生穴熊。其后中微,或在中国,或在蛮夷,弗能纪其世。

周文王之时,季连之苗裔曰鬻熊。鬻熊子事文王,蚤卒。其子曰熊丽。熊丽生熊狂,熊狂生熊绎。

熊绎当周成王之时,举文、武勤劳之后嗣,而封熊绎于楚蛮,封以子男之田,姓琇氏,居丹阳。楚子熊绎与鲁公伯禽、卫康叔子牟、晋侯燮、齐太公子吕伋俱事成王。

熊绎生熊艾,熊艾生熊,熊生熊胜。熊胜以弟熊杨为后。熊杨生熊渠。

熊渠生子三年。当周夷王之时,王室微,诸侯或不朝,相伐。熊渠甚得江汉间民和,乃兴兵伐庸、杨粤,至于鄂。熊渠曰:“我蛮夷也,不与中国之号谥。”乃立其长子康为句亶王,中子红为鄂王,少子执疵为越章王,皆在江上楚蛮之地。及周厉王之时,暴虐,熊渠畏其伐楚,亦去其王。

后为熊毋康,毋康蚤死。熊渠卒,子熊挚红立。挚红卒,其弟弑而代立,曰熊延。熊延生熊勇。

熊勇六年,而周人作乱,攻厉王,厉王出奔彘。熊勇十年,卒,弟熊严为后。

熊严十年,卒。有子四人,长子伯霜,中子仲雪,次子叔堪,少子季徇。熊严卒,长子伯霜代立,是为熊霜。

熊霜元年,周宣王初立。熊霜六年,卒,三弟争立。仲雪死;叔堪亡,避难于濮;而少弟季徇立,是为熊徇。熊徇十六年,郑桓公初封于郑。二十二年,熊徇卒,子熊咢立。熊咢九年,卒,子熊仪立,是为若敖。

若敖二十年,周幽王为犬戎所弑,周东徙,而秦襄公始列为诸侯。

二十七年,若敖卒,子熊坎立,是为霄敖。霄敖六年,卒,子熊眴立,是为蚡冒。蚡冒十三年,晋始乱,以曲沃之故。蚡冒辏洹冒弟熊通弑蚡冒子而代立,是为楚武王。

武王十七年,晋之曲沃庄伯弑主国晋孝侯。十九年,郑伯弟段作乱。二十一年,郑侵天子之田。二十三年,卫弑其君桓公。二十九年,鲁弑其君隐公。三十一年,宋太宰华督弑其君殇公。

三十五年,楚伐随。是也。随曰:“我无罪。”楚曰:“我蛮夷也。今诸侯皆为叛相侵,或相杀。我有敝甲,欲以观中国之政,请王室尊吾号。”随人为之周,请尊楚,王室不听,还报楚。三十七年,楚熊通怒曰:“吾先鬻熊,文王之师也,蚤终。成王举我先公,乃以子男田令居楚,蛮夷皆率服,而王不加位,我自尊耳。”乃自立为武王,与随人盟而去。于是始开濮地而有之。

五十一年,周召随侯,数以立楚为王。楚怒,以随背己,伐随。武王卒师中而兵罢。子文王熊赀立,始都郢。

文王二年,伐申过邓,邓人曰“楚王易取”,邓侯不许也。六年,伐蔡,虏蔡哀侯以归,已而释之。楚彊,陵江汉间小国,小国皆畏之。十一年,齐桓公始霸,楚亦始大。

十二年,伐邓,灭之。十三年,卒,子熊畑立,是为庄敖。庄敖五年,欲杀其弟熊恽,恽奔随,与随袭弑庄敖代立,是为成王。

成王恽元年,初即位,布德施惠,结旧好于诸侯。使人献天子,天子赐胙,曰:“镇尔南方夷越之乱,无侵中国。”于是楚地千里。

十六年,齐桓公以兵侵楚,至陉山。”楚成王使将军屈完以兵御之,与桓公盟。桓公数以周之赋不入王室,楚许之,乃去。

十八年,成王以兵北伐许,许君肉袒谢,乃释之。二十二年,伐黄。二十六年,灭英。

三十三年,宋襄公欲为盟会,召楚。楚王怒曰:“召我,我将好往袭辱之。”遂行,至盂,遂执辱宋公,已而归之。三十四年,郑文公南朝楚。楚成王北伐宋,败之泓,射伤宋襄公,襄公遂病创死。

三十五年,晋公子重耳过楚,成王以诸侯客礼飨,而厚送之于秦。

三十九年,鲁僖公来请兵以伐齐,楚使申侯将兵伐齐,取穀,”置齐桓公子雍焉。齐桓公七子皆奔楚,楚尽以为上大夫。灭夔,夔不祀祝融、鬻熊故也。

夏,伐宋,宋告急于晋,晋救宋,成王罢归。将军子玉请战,成王曰:“重耳亡居外久,卒得反国,天之所开,不可当。”子玉固请,乃与之少师而去。晋果败子玉于城濮。成王怒,诛子玉。

四十六年,初,成王将以商臣为太子,语令尹子上。子上曰:“君之齿未也,而又多内宠,绌乃乱也。楚国之举常在少者。且商臣蜂目而豺声,忍人也,不可立也。”王不听,立之。后又欲立子职而绌太子商臣。商臣闻而未审也,告其傅潘崇曰:“何以得其实?”崇曰:“飨王之宠姬江羋而勿敬也。”商臣从之。江羋怒曰:“宜乎王之欲杀若而立职也。”商臣告潘崇曰:“信矣。”崇曰:“能事之乎?”曰:“不能。”“能亡去乎?”曰:“不能。”“能行大事乎?”曰:“能。”冬十月,商臣以宫
兵围成王。成王请食熊蹯而死,不听。丁未,成王自绞杀。商臣代立,是为穆王。

穆王立,以其太子宫予潘崇,使为太师,掌国事。穆王三年,灭江。四年,灭六、蓼。六、蓼,皋陶之后。八年,伐陈。十二年,卒。子庄王侣立。

庄王即位三年,不出号令,日夜为乐,令国中曰:“有敢谏者死无赦!”伍举入谏。庄王左抱郑姬,右抱越女,坐锺鼓之间。伍举曰:“原有进隐。”曰:“有鸟在于阜,三年不蜚不鸣,是何鸟也?”庄王曰:“三年不蜚,蜚将冲天;三年不鸣,鸣将惊人。举退矣,吾知之矣。”居数月,淫益甚。大夫苏从乃入谏。王曰:“若不闻令乎?”对曰:“杀身以明君,臣之原也。”于是乃罢淫乐,听政,所诛者数百人,所进者数百人,任伍举、苏从以政,国人大说。是岁灭庸。六年,伐宋,获五百乘。

八年,伐陆浑戎,遂至洛,观兵于周郊。周定王使王孙满劳楚王。楚王问鼎小大轻重,对曰:“在德不在鼎。”庄王曰:“子无阻九鼎!楚国折钩之喙,足以为九鼎。”王孙满曰:“呜呼!君王其忘之乎?昔虞夏之盛,远方皆至,贡金九牧,铸鼎象物,百物而为之备,使民知神奸。桀有乱德,鼎迁于殷,载祀六百。殷纣暴虐,鼎迁于周。德之休明,虽小必重;其奸回昏乱,虽大必轻。昔成王定鼎于郏鄏,卜世三十,卜年七百,天所命也。周德虽衰,天命未改。鼎之轻重,未可问也。”楚王乃归。

九年,相若敖氏。人或谗之王,恐诛,反攻王,王击灭若敖氏之族。十三年,灭舒。

十六年,伐陈,杀夏徵舒。徵舒弑其君,故诛之也。已破陈,即县之。群臣皆贺,申叔时使齐来,不贺。王问,对曰:“鄙语曰,牵牛径人田,田主取其牛。径者则不直矣,取之牛不亦甚乎?且王以陈之乱而率诸侯伐之,以义伐之而贪其县,亦何以复令于天下!”庄王乃复国陈后。

十七年春,楚庄王围郑,三月克之。入自皇门,郑伯肉袒牵羊以逆,曰:“孤不天,不能事君,君用怀怒,以及敝邑,孤之罪也。敢不惟命是听!宾之南海,若以臣妾赐诸侯,亦惟命是听。若君不忘厉、宣、桓、武,不绝其社稷,使改事君,孤之原也,非所敢望也。敢布腹心。”楚群臣曰:“王勿许。”庄王曰:“其君能下人,必能信用其民,庸可绝乎!”庄王自手旗,左右麾军,引兵去三十里而舍,遂许之平。潘尪入盟,子良出质。夏六月,晋救郑,与楚战,大败晋师河上,遂至衡雍而归。

二十年,围宋,以杀楚使也。围宋五月,城中食尽,易子而食,析骨而炊。宋华元出告以情。庄王曰:“君子哉!”遂罢兵去。

二十三年,庄王卒,子共王审立。

共王十六年,晋伐郑。郑告急,共王救郑。与晋兵战鄢陵,晋败楚,射中共王目。共王召将军子反。子反嗜酒,从者竖阳穀进酒醉。王怒,射杀子反,遂罢兵归。

三十一年,共王卒,子康王招立。康王立十五年卒,子员立,是为郏敖。

康王宠弟公子围、子比、子晳、弃疾。郏敖三年,以其季父康王弟公子围为令尹,主兵事。四年,围使郑,道闻王疾而还。十二月己酉,围入问王疾,绞而弑之,遂杀其子莫及平夏。使使赴于郑。伍举问曰:“谁为后?”对曰:“寡大夫围。”伍举更曰:“共王之子围为长。”子比奔晋,而围立,是为灵王。

灵王三年六月,楚使使告晋,欲会诸侯。诸侯皆会楚于申。伍举曰:“昔夏启有钧台之飨,商汤有景亳之命,周武王有盟津之誓,成王有岐阳之蒐,康王有丰宫之朝,穆王有涂山之会,齐桓有召陵之师,晋文有践土之盟,君其何用?”灵王曰:“用桓公。”时郑子产在焉。于是晋、宋、鲁、卫不往。灵王已盟,有骄色。伍举曰:“桀为有仍之会,有缗叛之。纣为黎山之会,东夷叛之。幽王为太室之盟,戎、翟叛之。君其慎终!”

七月,楚以诸候兵伐吴,围硃方。八月,克之,囚庆封,灭其族。以封徇,曰:“无效齐庆封弑其君而弱其孤,以盟诸大夫!”封反曰:“莫如楚共王庶子围弑其君兄之子员而代之立!”于是灵王使疾杀之。

七年,就章华台,下令内亡人实之。

八年,使公子弃疾将兵灭陈。十年,召蔡侯,醉而杀之。使弃疾定蔡,因为陈蔡公。

十一年,伐徐以恐吴。灵王次于乾谿以待之。王曰:“齐、晋、鲁、卫,其封皆受宝器,我独不。今吾使使周求鼎以为分,其予我乎?”析父对曰:“其予君王哉!昔我先王熊绎辟在荆山,荜露蓝蒌。以处草莽,跋涉山林以事天子,唯是桃弧棘矢以共王事。齐,王舅也;晋及鲁、卫,王母弟也:楚是以无分而彼皆有。周今与四国服事君王,将惟命是从,岂敢爱鼎?”灵王曰:“昔我皇祖伯父昆吾旧许是宅,今郑人贪其田,不我予,今我求之,其予我乎?”对曰:“周不爱鼎,郑安敢爱田?”灵王曰:“昔诸侯远我而畏晋,今吾大城陈、蔡、不羹,赋皆千乘,诸侯畏我乎?”对曰:“畏哉!”灵王喜曰:“析父善言古事焉。”

十二年春,楚灵王乐乾谿,不能去也。国人苦役。初,灵王会兵于申,僇越大夫常寿过,杀蔡大夫观起。起子从亡在吴,乃劝吴王伐楚,为间越大夫常寿过而作乱,为吴间。使矫公子弃疾命召公子比于晋,至蔡,与吴、越兵欲袭蔡。令公子比见弃疾,与盟于邓。遂入杀灵王太子禄,立子比为王,公子子晳为令尹,弃疾为司马。先除王宫,观从从师于乾谿,令楚众曰:“国有王矣。先归,复爵邑田室。后者迁之。”楚众皆溃,去灵王而归。

灵王闻太子禄之死也,自投车下,而曰:“人之爱子亦如是乎?”侍者曰:“甚是。”王曰:“余杀人之子多矣,能无及此乎?”右尹曰:“请待于郊以听国人。”王曰:“众怒不可犯。”曰:“且入大县而乞师于诸侯。”王曰:“皆叛矣。”又曰:“且奔诸侯以听大国之虑。”王曰:“大福不再,祗取辱耳。”于是王乘舟将欲入鄢。右尹度王不用其计,惧俱死,亦去王亡。

灵王于是独傍徨山中,野人莫敢入王。王行遇其故鋗人,谓曰:“为我求食,我已不食三日矣。”鋗人曰:“新王下法,有敢饟王从王者,罪及三族,且又无所得食。”王因枕其股而卧。鋗人又以土自代,逃去。王觉而弗见,遂饥弗能起。芋尹申无宇之子申亥曰:“吾父再犯王命,王弗诛,恩孰大焉!”乃求王,遇王饥于釐泽,奉之以归。夏五月癸丑,王死申亥家,申亥以二女从死,并葬之。

是时楚国虽已立比为王,畏灵王复来,又不闻灵王死,故观从谓初王比曰:“不杀弃疾,虽得国犹受祸。”王曰:“余不忍。”从曰:“人将忍王。”王不听,乃去。弃疾归。国人每夜惊,曰:“灵王入矣!”乙卯夜,弃疾使船人从江上走呼曰:“灵王至矣!”国人愈惊。又使曼成然告初王比及令尹子晳曰:“王至矣!国人将杀君,司马将至矣!君蚤自图,无取辱焉。众怒如水火,不可救也。”初王及子晳遂自杀。丙辰,弃疾即位为王,改名熊居,是为平王。

平王以诈弑两王而自立,恐国人及诸侯叛之,乃施惠百姓。复陈蔡之地而立其后如故,归郑之侵地。存恤国中,修政教。吴以楚乱故,获五率以归。平王谓观从:“恣尔所欲。”欲为卜尹,王许之。

初,共王有宠子五人,无適立,乃望祭群神,请神决之,使主社稷,而阴与巴姬埋璧于室内,召五公子斋而入。康王跨之,灵王肘加之,子比、子晳皆远之。平王幼,抱其上而拜,压纽。故康王以长立,至其子失之;围为灵王,及身而弑;子比为王十馀日,子晳不得立,又俱诛。四子皆绝无后。唯独弃疾后立,为平王,竟续楚祀,如其神符。

初,子比自晋归,韩宣子问叔向曰:“子比其济乎?”对曰:“不就。”宣子曰:“同恶相求,如市贾焉,何为不就?”对曰:“无与同好,谁与同恶?取国有五难:有宠无人,一也;有人无主,二也;有主无谋,三也;有谋而无民,四也;有民而无德,五也。”子比在晋十三年矣,晋、楚之从不闻通者,可谓无人矣;族尽亲叛,可谓无主矣;无衅而动,可谓无谋矣;为羁终世,可谓无民矣;亡无爱徵,可谓无德矣。王虐而不忌,子比涉五难以弑君,谁能济之!有楚国者,其弃疾乎?君陈、蔡,方城外属焉。苛慝不作,盗贼伏隐,私欲不违,民无怨心。先神命之,国民信之。琇姓有乱,必季实立,楚之常也。子比之官,则右尹也;数其贵宠,则庶子也;以神所命,则又远之;民无怀焉,将何以立?”宣子曰:“齐桓、晋文不亦是乎?”对曰:“齐桓,卫姬之子也,有宠于釐公。有鲍叔牙、宾须无、隰朋以为辅,有莒、卫以为外主,有高、国以为内主。从善如流,施惠不倦。有国,不亦宜乎?昔我文公,狐季姬之子也,有宠于献公。好学不倦。生十七年,有士五人,有先大夫子馀、子犯以为腹心,有魏焠、贾佗以为股肱,有齐、宋、秦、楚以为外主,有栾、郤、狐、先以为内主。亡十九年,守志弥笃。惠、怀弃民,民从而与之。故文公有国,不亦宜乎?子比无施于民,无援于外,去晋,晋不送;归楚,楚不迎。何以有国!”子比果不终焉,卒立者弃疾,如叔向言也。

平王二年,使费无忌如秦为太子建取妇。妇好,来,未至,无忌先归,说平王曰:“秦女好,可自娶,为太子更求。”平王听之,卒自娶秦女,生熊珍。更为太子娶。是时伍奢为太子太傅,无忌为少傅。无忌无宠于太子,常谗恶太子建。建时年十五矣,其母蔡女也,无宠于王,王稍益疏外建也。

六年,使太子建居城父,守边。无忌又日夜谗太子建于王曰:“自无忌入秦女,太子怨,亦不能无望于王,王少自备焉。且太子居城父,擅兵,外交诸侯,且欲入矣。”平王召其傅伍奢责之。伍奢知无忌谗,乃曰:“王柰何以小臣疏骨肉?”无忌曰:;“今不制,后悔也。”于是王遂囚伍奢。乃令司马奋扬召太子建,欲诛之。太子闻之,亡奔宋。

无忌曰:“伍奢有二子,不杀者为楚国患。盍以免其父召之,必至。”于是王使使谓奢:“能致二子则生,不能将死。”奢曰:“尚至,胥不至。”王曰:“何也?”奢曰:“尚之为人,廉,死节,慈孝而仁,闻召而免父,必至,不顾其死。胥之为人,智而好谋,勇而矜功,知来必死,必不来。然为楚国忧者必此子。”于是王使人召之,曰:“来,吾免尔父。”伍尚谓伍胥曰:“闻父免而莫奔,不孝也;父戮莫报,无谋也;度能任事,知也。子其行矣,我其归死。”伍尚遂归。伍胥弯弓属矢,出见使者,曰:“父有罪,何以召其子为?”将射,使者还走,遂出奔吴。伍奢闻之,曰:“胥亡,楚国危哉。”楚人遂杀伍奢及尚。

十年,楚太子建母在居巢,开吴。吴使公子光伐楚,遂败陈、蔡,取太子建母而去。楚恐,城郢。初,吴之边邑卑梁与楚边邑锺离小童争桑,两家交怒相攻,灭卑梁人。卑梁大夫怒,发邑兵攻锺离。楚王闻之怒,发国兵灭卑梁。吴王闻之大怒,亦发兵,使公子光因建母家攻楚,遂灭锺离、居巢。楚乃恐而城郢。

十三年,平王卒。将军子常曰:“太子珍少,且其母乃前太子建所当娶也。”欲立令尹子西。子西,平王之庶弟也,有义。子西曰:“国有常法,更立则乱,言之则致诛。”乃立太子珍,是为昭王。

昭王元年,楚众不说费无忌,以其谗亡太子建,杀伍奢子父与郤宛。宛之宗姓伯氏子嚭及子胥皆奔吴,吴兵数侵楚,楚人怨无忌甚。楚令尹子常诛无忌以说众,众乃喜。

四年,吴三公子奔楚,楚封之以扞吴。五年,吴伐取楚之六、潜。七年,楚使子常伐吴,吴大败楚于豫章。

十年冬,吴王阖闾、伍子胥、伯嚭与唐、蔡俱伐楚,楚大败,吴兵遂入郢,辱平王之墓,以伍子胥故也。吴兵之来,楚使子常以兵迎之,夹汉水阵。吴伐败子常,子常亡奔郑。楚兵走,吴乘胜逐之,五战及郢。己卯,昭王出奔。庚辰,吴人入郢。

昭王亡也至云梦。云梦不知其王也,射伤王。王走郧。郧公之弟怀曰:“平王杀吾父,今我杀其子,不亦可乎?”郧公止之,然恐其弑昭王,乃与王出奔随。吴王闻昭王往,即进击随,谓随人曰:“周之子孙封于江汉之间者,楚尽灭之。”欲杀昭王。王从臣子綦乃深匿王,自以为王,谓随人曰:“以我予吴。”随人卜予吴,不吉,乃谢吴王曰:“昭王亡,不在随。”吴请入自索之,随不听,吴亦罢去。

昭王之出郢也,使申鲍胥请救于秦。秦以车五百乘救楚,楚亦收馀散兵,与秦击吴。十一年六月,败吴于稷。会吴王弟夫概见吴王兵伤败,乃亡归,自立为王。阖闾闻之,引兵去楚,归击夫概。夫概败,奔楚,楚封之堂谿,号为堂谿氏。

楚昭王灭唐九月,归入郢。十二年,吴复伐楚,取番。楚恐,去郢,北徙都鄀。

十六年,孔子相鲁。二十年,楚灭顿,灭胡。二十一年,吴王阖闾伐越。越王句践射伤吴王,遂死。吴由此怨越而不西伐楚。

二十七年春,吴伐陈,楚昭王救之,军城父。十月,昭王病于军中,有赤云如鸟,夹日而蜚。昭王问周太史,太史曰:“是害于楚王,然可移于将相。”将相闻是言,乃请自以身祷于神。昭王曰:“将相,孤之股肱也,今移祸,庸去是身乎!”弗听。卜而河为祟,大夫请祷河。昭王曰:“自吾先王受封,望不过江、汉,而河非所获罪也。”止不许。孔子在陈,闻是言,曰:“楚昭王通大道矣。其不失国,宜哉!”

昭王病甚,乃召诸公子大夫曰:“孤不佞,再辱楚国之师,今乃得以天寿终,孤之幸也。”让其弟公子申为王,不可。又让次弟公子结,亦不可。乃又让次弟公子闾,五让,乃后许为王。将战,庚寅,昭王卒于军中。子闾曰:“王病甚,舍其子让群臣,臣所以许王,以广王意也。今君王卒,臣岂敢忘君王之意乎!”乃与子西、子綦谋,伏师闭涂,迎越女之子章立之,是为惠王。然后罢兵归,葬昭王。

惠王二年,子西召故平王太子建之子胜于吴,以为巢大夫,号曰白公。白公好兵而下士,欲报仇。六年,白公请兵令尹子西伐郑。初,白公父建亡在郑,郑杀之,白公亡走吴,子西复召之,故以此怨郑,欲伐之。子西许而未为发兵。八年,晋伐郑,郑告急楚,楚使子西救郑,受赂而去。白公胜怒,乃遂与勇力死士石乞等袭杀令尹子西、子綦于朝,因劫惠王,置之高府,欲弑之。惠王从者屈固负王亡走昭王夫人宫。白公自立为王。月馀,会叶公来救楚,楚惠王之徒与共攻白公,杀之。惠王乃复位。是岁也,灭陈而县之。

十三年,吴王夫差彊,陵齐、晋,来伐楚。十六年,越灭吴。四十二年,楚灭蔡。四十四年,楚灭杞。与秦平。是时越已灭吴而不能正江、淮北;楚东侵,广地至泗上。

五十七年,惠王卒,子简王中立。

简王元年,北伐灭莒。八年,魏文侯、韩武子、赵桓子始列为诸侯。

二十四年,简王卒,子声王当立。声王六年,,盗杀声王,子悼王熊疑立。悼王二年,三晋来伐楚,至乘丘而还。四年,楚伐周。郑杀子阳。九年,伐韩,取负黍。十一年,三晋伐楚,败我大梁、榆关。楚厚赂秦,与之平。二十一年,悼王卒,子肃王臧立。

肃王四年,蜀伐楚,取兹方。于是楚为扞关以距之。十年,魏取我鲁阳。十一年,肃王卒,无子,立其弟熊良夫,是为宣王。

宣王六年,周天子贺秦献公。秦始复彊,而三晋益大,魏惠王、齐威王尤彊。三十年,秦封卫鞅于商,南侵楚。是年,宣王卒,子威王熊商立。

威王六年,周显王致文武胙于秦惠王。

七年,齐孟尝君父田婴欺楚,楚威王伐齐,败之于徐州,而令齐必逐田婴。田婴恐,张丑伪谓楚王曰:“王所以战胜于徐州者,田盼子不用也。盼子者,有功于国,而百姓为之用。婴子弗善而用申纪。申纪者,大臣不附,百姓不为用,故王胜之也。今王逐婴子,婴子逐,盼子必用矣。复搏其士卒以与王遇,必不便于王矣。”楚王因弗逐也。

十一年,威王卒,子怀王熊槐立。魏闻楚丧,伐楚,取我陉山。

怀王元年,张仪始相秦惠王。四年,秦惠王初称王。

六年,楚使柱国昭阳将兵而攻魏,破之于襄陵,得八邑。又移兵而攻齐,齐王患之。陈轸適为秦使齐,齐王曰:“为之柰何?”陈轸曰:“王勿忧,请令罢之。”即往见昭阳军中,曰:“原闻楚国之法,破军杀将者何以贵之?”昭阳曰:“其官为上柱国,封上爵执珪。”陈轸曰:“其有贵于此者乎?”昭阳曰:“令尹。”陈轸曰:“今君已为令尹矣,此国冠之上。臣请得譬之。人有遗其舍人一卮酒者,舍人相谓曰:‘数人饮此,不足以遍,请遂画地为蛇,蛇先成者独饮之。’一人曰:‘吾蛇先成。’举酒而起,曰:‘吾能为之足。’及其为之足,而后成人夺之酒而饮之,曰:‘蛇固无足,今为之足,是非蛇也。’今君相楚而攻魏,破军杀将,功莫大焉,冠之上不可以加矣。今又移兵而攻齐,攻齐胜之,官爵不加于此;攻之不胜,身死爵夺,有毁于楚:此为蛇为足之说也。不若引兵而去以德齐,此持满之术也。”昭阳曰:“善。”引兵而去。

燕、韩君初称王。秦使张仪与楚、齐、魏相会,盟齧桑。

十一年,苏秦约从山东六国共攻秦,楚怀王为从长。至函谷关,秦出兵击六国,六国兵皆引而归,齐独后。十二年,齐湣王伐败赵、魏军,秦亦伐败韩,与齐争长。

十六年,秦欲伐齐,而楚与齐从亲,秦惠王患之,乃宣言张仪免相,使张仪南见楚王,谓楚王曰:“敝邑之王所甚说者无先大王,虽仪之所甚原为门阑之厮者亦无先大王。敝邑之王所甚憎者无先齐王,虽仪之所甚憎者亦无先齐王。而大王和之,是以敝邑之王不得事王,而令仪亦不得为门阑之厮也。王为仪闭关而绝齐,今使使者从仪西取故秦所分楚商于之地方六百里,如是则齐弱矣。是北弱齐,西德于秦,私商于以为富,此一计而三利俱至也。”怀王大悦,乃置相玺于张仪,日与置酒,宣言“吾复得吾商于之地”。群臣皆贺,而陈轸独吊。怀王曰:“何故?”陈轸对曰:“秦之所为重王者,以王之有齐也。今地未可得而齐交先绝,是楚孤也。夫秦又何重孤国哉,必轻楚矣。且先出地而后绝齐,则秦计不为。先绝齐而后责地,则必见欺于张仪。见欺于张仪,则王必怨之。怨之,是西起秦患,北绝齐交。西起秦患,北绝齐交,则两国之兵必至。臣故吊。”楚王弗听,因使一将军西受封地。

张仪至秦,详醉坠车,称病不出三月,地不可得。楚王曰:“仪以吾绝齐为尚薄邪?”乃使勇士宋遗北辱齐王。齐王大怒,折楚符而合于秦。秦齐交合,张仪乃起朝,谓楚将军曰:“子何不受地?从某至某,广袤六里。”楚将军曰:“臣之所以见命者六百里,不闻六里。”即以归报怀王。怀王大怒,兴师将伐秦。陈轸又曰:“伐秦非计也。不如因赂之一名都,与之伐齐,是我亡于秦,取偿于齐也,吾国尚可全。今王已绝于齐而责欺于秦,是吾合秦齐之交而来天下之兵也,国必大伤矣。”楚王不听,遂绝和于秦,发兵西攻秦。秦亦发兵击之。

十七年春,与秦战丹阳,秦大败我军,斩甲士八万,虏我大将军屈匄、裨将军逢侯丑等七十馀人,遂取汉中之郡。楚怀王大怒,乃悉国兵复袭秦,战于蓝田,大败楚军。韩、魏闻楚之困,乃南袭楚,至于邓。楚闻,乃引兵归。

十八年,秦使使约复与楚亲,分汉中之半以和楚。楚王曰:“原得张仪,不原得地。”张仪闻之,请之楚。秦王曰:“楚且甘心于子,柰何?”张仪曰:“臣善其左右靳尚,靳尚又能得事于楚王幸姬郑袖,袖所言无不从者。且仪以前使负楚以商于之约,今秦楚大战,有恶,臣非面自谢楚不解。且大王在,楚不宜敢取仪。诚杀仪以便国,臣之原也。”仪遂使楚。

至,怀王不见,因而囚张仪,欲杀之。仪私于靳尚,靳尚为请怀王曰:“拘张仪,秦王必怒。天下见楚无秦,必轻王矣。”又谓夫人郑袖曰:“秦王甚爱张仪,而王欲杀之,今将以上庸之地六县赂楚,以美人聘楚王,以宫中善歌者为之媵。楚王重地,秦女必贵,而夫人必斥矣。夫人不若言而出之。”郑袖卒言张仪于王而出之。仪出,怀王因善遇仪,仪因说楚王以叛从约而与秦合亲,约婚姻。张仪已去,屈原使从齐来,谏王曰:“何不诛张仪?”怀王悔,使人追仪,弗及。是岁,秦惠王卒。

二十年,齐湣王欲为从长,恶楚之与秦合,乃使使遗楚王书曰:“寡人患楚之不察于尊名也。今秦惠王死,武王立,张仪走魏,樗里疾、公孙衍用,而楚事秦。夫樗里疾善乎韩,而公孙衍善乎魏;楚必事秦,韩、魏恐,必因二人求合于秦,则燕、赵亦宜事秦。四国争事秦,则楚为郡县矣。王何不与寡人并力收韩、魏、燕、赵,与为从而尊周室,以案兵息民,令于天下?莫敢不乐听,则王名成矣。王率诸侯并伐,破秦必矣。王取武关、蜀、汉之地,私吴、越之富而擅江海之利,韩、魏割上党,西薄函谷,则楚之彊百万也。且王欺于张仪,亡地汉中,兵锉蓝田,天下莫不代王怀怒。今乃欲先事秦!原大王孰计之。”

楚王业已欲和于秦,见齐王书,犹豫不决,下其议群臣。群臣或言和秦,或曰听齐。昭雎曰:“王虽东取地于越,不足以刷耻;必且取地于秦,而后足以刷耻于诸侯。王不如深善齐、韩以重樗里疾,如是则王得韩、齐之重以求地矣。秦破韩宜阳,而韩犹复事秦者,以先王墓在平阳,而秦之武遂去之七十里,以故尤畏秦。不然,秦攻三川,赵攻上党,楚攻河外,韩必亡。楚之救韩,不能使韩不亡,然存韩者楚也。韩已得武遂于秦,以河山为塞,所报德莫如楚厚,臣以为其事王必疾。齐之所信于韩者,以韩公子眛为齐相也。韩已得武遂于秦,王甚善之,使之以齐、韩重樗里疾,疾得齐、韩之重,其主弗敢弃疾也。今又益之以楚之重,樗里子必言秦,复与楚之侵地矣。”于是怀王许之,竟不合秦,而合齐以善韩。

二十四年,倍齐而合秦。秦昭王初立,乃厚赂于楚。楚往迎妇。二十五年,怀王入与秦昭王盟,约于黄棘。秦复与楚上庸。二十六年,齐、韩、魏为楚负其从亲而合于秦,三国共伐楚。楚使太子入质于秦而请救。秦乃遣客卿通将兵救楚,三国引兵去。

二十七年,秦大夫有私与楚太子斗,楚太子杀之而亡归。二十八年,秦乃与齐、韩、魏共攻楚,杀楚将唐眛,取我重丘而去。二十九年,秦复攻楚,大破楚,楚军死者二万,杀我将军景缺。怀王恐,乃使太子为质于齐以求平。三十年,秦复伐楚,取八城。秦昭王遗楚王书曰:“始寡人与王约为弟兄,盟于黄棘,太子为质,至驩也。太子陵杀寡人之重臣,不谢而亡去,寡人诚不胜怒,使兵侵君王之边。今闻君王乃令太子质于齐以求平。寡人与楚接境壤界,故为婚姻,所从相亲久矣。而今秦楚不驩,则无以令诸侯。寡人原与君王会武关,面相约,结盟而去,寡人之原也。敢以闻下执事。”楚怀王见秦王书,患之。欲往,恐见欺;无往,恐秦怒。昭雎曰:“王毋行,而发兵自守耳。秦虎狼,不可信,有并诸侯之心。”怀王子子兰劝王行,曰:“柰何绝秦之驩心!”于是往会秦昭王。昭王诈令一将军伏兵武关,号为秦王。楚王至,则闭武关,遂与西至咸阳,朝章台,如蕃臣,不与亢礼。楚怀王大怒,悔不用昭子言。秦因留楚王,要以割巫、黔中之郡。楚王欲盟,秦欲先得地。楚王怒曰:“秦诈我而又彊要我以地!”不复许秦。秦因留之。

楚大臣患之,乃相与谋曰:“吾王在秦不得还,要以割地,而太子为质于齐,齐、秦合谋,则楚无国矣。”乃欲立怀王子在国者。昭雎曰:“王与太子俱困于诸侯,而今又倍王命而立其庶子,不宜。”乃诈赴于齐,齐湣王谓其相曰:“不若留太子以求楚之淮北。”相曰:“不可,郢中立王,是吾抱空质而行不义于天下也。”或曰:“不然。郢中立王,因与其新王市曰‘予我下东国,吾为王杀太子,不然,将与三国共立之’,然则东国必可得矣。”齐王卒用其相计而归楚太子。太子横至,立为王,是为顷襄王。乃告于秦曰:“赖社稷神灵,国有王矣。”

顷襄王横元年,秦要怀王不可得地,楚立王以应秦,秦昭王怒,发兵出武关攻楚,大败楚军,斩首五万,取析十五城而去。二年,楚怀王亡逃归,秦觉之,遮楚道,怀王恐,乃从间道走赵以求归。赵主父在代,其子惠王初立,行王事,恐,不敢入楚王。楚王欲走魏,秦追至,遂与秦使复之秦。怀王遂发病。顷襄王三年,怀王卒于秦,秦归其丧于楚。楚人皆怜之,如悲亲戚。诸侯由是不直秦。秦楚绝。

六年,秦使白起伐韩于伊阙,大胜,斩首二十四万。秦乃遗楚王书曰:“楚倍秦,秦且率诸侯伐楚,争一旦之命。原王之饬士卒,得一乐战。”楚顷襄王患之,乃谋复与秦平。七年,楚迎妇于秦,秦楚复平。

十一年,齐秦各自称为帝;月馀,复归帝为王。

十四年,楚顷襄王与秦昭王好会于宛,结和亲。十五年,楚王与秦、三晋、燕共伐齐,取淮北。十六年,与秦昭王好会于鄢。其秋,复与秦王会穰。

十八年,楚人有好以弱弓微缴加归雁之上者,顷襄王闻,召而问之。对曰:“小臣之好射鶀雁,罗鸗,小矢之发也,何足为大王道也。且称楚之大,因大王之贤,所弋非直此也。昔者三王以弋道德,五霸以弋战国。故秦、魏、燕、赵者,鶀雁也;齐、鲁、韩、卫者,青首也;驺、费、郯、邳者,罗鸗也。外其馀则不足射者。见鸟六双,以王何取?王何不以圣人为弓,以勇士为缴,时张而射之?此六双者,可得而囊载也。其乐非特朝昔之乐也,其获非特凫雁之实也。王朝张弓而射魏之大梁之南,加其右臂而径属之于韩,则中国之路绝而上蔡之郡坏矣。还射圉之东,解魏左肘而外击定陶,则魏之东外弃而大宋、方与二郡者举矣。且魏断二臂,颠越矣;膺击郯国,大梁可得而有也。王綪缴兰台,饮马西河,定魏大梁,此一发之乐也。若王之于弋诚好而不厌,则出宝弓,碆新缴,射噣鸟于东海,还盖长城以为防,朝射东莒,夕发浿丘,夜加即墨,顾据午道,则长城之东收而太山之北举矣。西结境于赵而北达于燕,三国布嬛,则从不待约而可成也。北游目于燕之辽东而南登望于越之会稽,此再发之乐也。若夫泗上十二诸侯,左萦而右拂之,可一旦而尽也。今秦破韩以为长忧,得列城而不敢守也;伐魏而无功,击赵而顾病,则秦魏之勇力屈矣,楚之故地汉中、析、郦可得而复有也。王出宝弓,碆新缴,涉鄳塞,而待秦之倦也,山东、河内可得而一也。劳民休众,南面称王矣。故曰秦为大鸟,负海内而处,东面而立,左臂据赵之西南,右臂傅楚鄢郢,膺击韩魏,垂头中国,处既形便,势有地利,奋翼鼓嬛,方三千里,则秦未可得独招而夜射也。”欲以激怒襄王,故对以此言。襄王因召与语,遂言曰:“夫先王为秦所欺而客死于外,怨莫大焉。今以匹夫有怨,尚有报万乘,白公、子胥是也。今楚之地方五千里,带甲百万,犹足以踊跃中野也,而坐受困,臣窃为大王弗取也。”于是顷襄王遣使于诸侯,复为从,欲以伐秦。秦闻之,发兵来伐楚。

楚欲与齐韩连和伐秦,因欲图周。周王赧使武公谓楚相昭子曰:“三国以兵割周郊地以便输,而南器以尊楚,臣以为不然。夫弑共主,臣世君,大国不亲;以众胁寡,小国不附。大国不亲,小国不附,不可以致名实。名实不得,不足以伤民。夫有图周之声,非所以为号也。”昭子曰:“乃图周则无之。虽然,周何故不可图也?”对曰:“军不五不攻,城不十不围。夫一周为二十晋,公之所知也。韩尝以二十万之众辱于晋之城下,锐士死,中士伤,而晋不拔。公之无百韩以图周,此天下之所知也。夫怨结两周以塞驺鲁之心,交绝于齐,声失天下,其为事危矣。夫危两周以厚三川,方城之外必为韩弱矣。何以知其然也?西周之地,绝长补短,不过百里。名为天下共主,裂其地不足以肥国,得其众不足以劲兵。虽无攻之,名为弑君。然而好事之君,喜攻之臣,发号用兵,未尝不以周为终始。是何也?见祭器在焉,欲器之至而忘弑君之乱。今韩以器之在楚,臣恐天下以器雠楚也。臣请譬之。夫虎肉臊,其兵利身,人犹攻之也。若使泽中之麋蒙虎之皮,人之攻之必万于虎矣。裂楚之地,足以肥国;诎楚之名,足以尊主。今子将以欲诛残天下之共主,居三代之传器,吞三翮六翼,以高世主,非贪而何?周书曰‘欲起无先’,故器南则兵至矣。”于是楚计辍不行。

十九年,秦伐楚,楚军败,割上庸、汉北地予秦。二十一年,秦将白起遂拔我郢,烧先王墓夷陵。楚襄王兵散,遂不复战,东北保于陈城。二十二年,秦复拔我巫、黔中郡。

二十三年,襄王乃收东地兵,得十馀万,复西取秦所拔我江旁十五邑以为郡,距秦。二十七年,使三万人助三晋伐燕。复与秦平,而入太子为质于秦。楚使左徒侍太子于秦。

三十六年,顷襄王病,太子亡归。秋,顷襄王卒,太子熊元代立,是为考烈王。考烈王以左徒为令尹,封以吴,号春申君。

考烈王元年,纳州于秦以平。是时楚益弱。

六年,秦围邯郸,赵告急楚,楚遣将军景阳救赵。七年,至新中。秦兵去。十二年,秦昭王卒,楚王使春申君吊祠于秦。十六年,秦庄襄王卒,秦王赵政立。二十二年,与诸侯共伐秦,不利而去。楚东徙都寿春,命曰郢。

二十五年,考烈王卒,子幽王悍立。李园杀春申君。幽王三年,秦、魏伐楚。秦相吕不韦卒。九年,秦灭韩。十年,幽王卒,同母弟犹代立,是为哀王。哀王立二月馀,哀王庶兄负刍之徒袭杀哀王而立负刍为王。是岁,秦虏赵王迁。

王负刍元年,燕太子丹使荆轲刺秦王。二年,秦使将军伐楚,大破楚军,亡十馀城。三年,秦灭魏。四年,秦将王翦破我军于蕲,而杀将军项燕。

五年,秦将王翦、蒙武遂破楚国,虏楚王负刍,灭楚名为郡云。

太史公曰:楚灵王方会诸侯于申,诛齐庆封,作章华台,求周九鼎之时,志小天下;及饿死于申亥之家,为天下笑。操行之不得,悲夫!势之于人也,可不慎与?弃疾以乱立,嬖淫秦女,甚乎哉,几再亡国!

鬻熊之嗣,周封于楚。僻在荆蛮,荜路蓝缕。及通而霸,僭号曰武。文既伐申,成亦赦许。子圉篡嫡,商臣杀父。天祸未悔,凭奸自怙。昭困奔亡,怀迫囚虏。顷襄、考烈,祚衰南土。
\end{yuanwen}

\chapter{越王勾践世家}

\begin{yuanwen}
越王勾践,其先禹之苗裔,而夏后帝少康之庶子也。封于会稽,以奉守禹之祀。文身断发,披草莱而邑焉。后二十馀世,至于允常。云:“于,语发声也。”允常之时,与吴王阖庐战而相怨伐。允常卒,子勾践立,是为越王。

元年,吴王阖庐闻允常死,乃兴师伐越。越王勾践使死士挑战,三行,至吴陈,呼而自刭。吴师观之,越因袭击吴师,吴师败于槜李,射伤吴王阖庐。阖庐且死,告其子夫差曰:“必毋忘越。”

三年,勾践闻吴王夫差日夜勒兵,且以报越,越欲先吴未发往伐之。范蠡谏曰:“不可。臣闻兵者凶器也,战者逆德也,争者事之末也。阴谋逆德,好用凶器,试身于所末,上帝禁之,行者不利。”越王曰:“吾已决之矣。”遂兴师。吴王闻之,悉发精兵击越,败之夫椒。越王乃以馀兵五千人保栖于会稽。吴王追而围之。

越王谓范蠡曰:“以不听子故至于此,为之柰何?”蠡对曰:“持满者与天,定倾者与人,节事者以地。卑辞厚礼以遗之,不许,而身与之市。”勾践曰:“诺。”乃令大夫种行成于吴,膝行顿首曰:“君王亡臣勾践使陪臣种敢告下执事:勾践请为臣,妻为妾。”吴王将许之。子胥言于吴王曰:“天以越赐吴,勿许也。”种还,以报勾践。勾践欲杀妻子,燔宝器,触战以死。种止勾践曰:“夫吴太宰嚭贪,可诱以利,请间行言之。”于是勾践以美女宝器令种间献吴太宰嚭。嚭受,乃见大夫种于吴王。种顿首言曰:“原大王赦勾践之罪,尽入其宝器。不幸不赦,勾践将尽杀其妻子,燔其宝器,悉五千人触战,必有当也。”嚭因说吴王曰:“越以服为臣,若将赦之,此国之利也。”吴王将许之。子胥进谏曰:“今不灭越,后必悔之。勾践贤君,种、蠡良臣,若反国,将为乱。”吴王弗听,卒赦越,罢兵而归。

勾践之困会稽也,喟然叹曰:“吾终于此乎?”种曰:“汤系夏台,文王囚羑里,晋重耳饹翟,齐小白饹莒,其卒王霸。由是观之,何遽不为福乎?”

吴既赦越,越王勾践反国,乃苦身焦思,置胆于坐,坐卧即仰胆,饮食亦尝胆也。曰:“女忘会稽之耻邪?”身自耕作,夫人自织,食不加肉,衣不重采,折节下贤人,厚遇宾客,振贫吊死,”与百姓同其劳。欲使范蠡治国政,蠡对曰:“兵甲之事,种不如蠡;填抚国家,亲附百姓,蠡不如种。”于是举国政属大夫种,而使范蠡与大夫柘稽行成,为质于吴。二岁而吴归蠡。

勾践自会稽归七年,拊循其士民,欲用以报吴。大夫逢同谏曰:“国新流亡,今乃复殷给,缮饰备利,吴必惧,惧则难必至。且鸷鸟之击也,必匿其形。今夫吴兵加齐、晋,怨深于楚、越,名高天下,实害周室,德少而功多,必淫自矜。为越计,莫若结齐,亲楚,附晋,以厚吴。吴之志广,必轻战。是我连其权,三国伐之,越承其弊,可克也。”勾践曰:“善。”

居二年,吴王将伐齐。子胥谏曰:“未可。臣闻勾践食不重味,与百姓同苦乐。此人不死,必为国患。吴有越,腹心之疾,齐与吴,疥甪也。原王释齐先越。”吴王弗听,遂伐齐,败之艾陵,虏齐高、国以归。让子胥。子胥曰:“王毋喜!”王怒,子胥欲自杀,王闻而止之。越大夫种曰:“臣观吴王政骄矣,请试尝之贷粟,以卜其事。”请贷,吴王欲与,子胥谏勿与,王遂与之,越乃私喜。子胥言曰:“王不听谏,后三年吴其墟乎!”太宰嚭闻之,乃数与子胥争越议,因谗子胥曰:“伍员貌忠而实忍人,其父兄不顾,安能顾王?王前欲伐齐,员彊谏,已而有功,用是反怨王。王不备伍员,员必为乱。”与逢同共谋,谗之王。王始不从,乃使子胥于齐,闻其讬子于鲍氏,王乃大怒,曰:“伍员果欺寡人!”役反,使人赐子胥属镂剑以自杀。子胥大笑曰:“我令而父霸,我又立若,若初欲分吴国半予我,我不受,已,今若反以谗诛我。嗟乎,嗟乎,一人固不能独立!”报使者曰:“必取吾眼置吴东门,以观越兵入也!”于是吴任嚭政。

居三年,勾践召范蠡曰:“吴已杀子胥,导谀者众,可乎?”对曰:“未可。”

至明年春,吴王北会诸侯于黄池,吴国精兵从王,惟独老弱与太子留守。勾践复问范蠡,蠡曰“可矣”。乃发习流二千人,教士四万人,君子六千人,诸御千人,伐吴。吴师败,遂杀吴太子。吴告急于王,王方会诸侯于黄池,惧天下闻之,乃祕之。吴王已盟黄池,乃使人厚礼以请成越。越自度亦未能灭吴,乃与吴平。

其后四年,越复伐吴。吴士民罢弊,轻锐尽死于齐、晋。而越大破吴,因而留围之三年,吴师败,越遂复栖吴王于姑苏之山。吴王使公孙雄肉袒膝行而前,请成越王曰:“孤臣夫差敢布腹心,异日尝得罪于会稽,夫差不敢逆命,得与君王成以归。今君王举玉趾而诛孤臣,孤臣惟命是听,意者亦欲如会稽之赦孤臣之罪乎?”勾践不忍,欲许之。范蠡曰:“会稽之事,天以越赐吴,吴不取。今天以吴赐越,越其可逆天乎?且夫君王蚤朝晏罢,非为吴邪?谋之二十二年,一旦而弃之,可乎?且夫天与弗取,反受其咎。‘伐柯者其则不远’,君忘会稽之戹乎?”勾践曰:“吾欲听子言,吾不忍其使者。”范蠡乃鼓进兵,曰:“王已属政于执事,使者去,不者且得罪。”吴使者泣而去。勾践怜之,乃使人谓吴王曰:“吾置王甬东,君百家。”吴王谢曰:“吾老矣,不能事君王!”遂自杀。乃蔽其面,曰:“吾无面以见子胥也!”越王乃葬吴王而诛太宰嚭。

勾践已平吴,乃以兵北渡淮,与齐、晋诸侯会于徐州,致贡于周。周元王使人赐勾践胙,命为伯。勾践已去,渡淮南,以淮上地与楚,归吴所侵宋地于宋,与鲁泗东方百里。当是时,越兵横行于江、淮东,诸侯毕贺,号称霸王。

范蠡遂去,自齐遗大夫种书曰:“蜚鸟尽,良弓藏;狡兔死,走狗烹。越王为人长颈鸟喙,可与共患难,不可与共乐。子何不去?”种见书,称病不朝。人或谗种且作乱,越王乃赐种剑曰:“子教寡人伐吴七术,寡人用其三而败吴,其四在子,子为我从先王试之。”种遂自杀。

勾践卒,子王鼫与立。王鼫与卒,子王不寿立。王不寿卒,子王翁立。王翁卒,子王翳立。王翳卒,子王之侯立。王之侯卒,子王无彊立。

王无彊时,越兴师北伐齐,西伐楚,与中国争彊。当楚威王之时,越北伐齐,齐威王使人说越王曰:“越不伐楚,大不王,小不伯。图越之所为不伐楚者,为不得晋也。韩、魏固不攻楚。韩之攻楚,覆其军,杀其将,则叶、阳翟危;魏亦覆其军,杀其将,则陈、上蔡不安。故二晋之事越也,不至于覆军杀将,马汗之力不效。所重于得晋者何也?”越王曰:“所求于晋者,不至顿刃接兵,而况于攻城围邑乎?原魏以聚大梁之下,原齐之试兵南阳莒地,以聚常、郯之境,则方城之外不南,淮、泗之间不东,商、于、析、郦、宗胡之地,夏路以左,不足以备秦,江南、泗上不足以待越矣。则齐、秦、韩、魏得志于楚也,是二晋不战分地,不耕而穫之。不此之为,而顿刃于河山之间以为齐秦用,所待者如此其失计,柰何其以此王也!”齐使者曰:“幸也越之不亡也!吾不贵其用智之如目,见豪毛而不见其睫也。今王知晋之失计,而不自知越之过,是目论也。王所待于晋者,非有马汗之力也,又非可与合军连和也,将待之以分楚众也。今楚众已分,何待于晋?”越王曰:“柰何?”曰:“楚三大夫张九军,北围曲沃、于中,以至无假之关者三千七百里,景翠之军北聚鲁、齐、南阳,分有大此者乎?且王之所求者,斗晋楚也;晋楚不斗,越兵不起,是知二五而不知十也。此时不攻楚,臣以是知越大不王,小不伯。复雠、庞、长沙,楚之粟也;竟泽陵,楚之材也。越窥兵通无假之关,此四邑者不上贡事于郢矣。臣闻之,图王不王,其敝可以伯。然而不伯者,王道失也。故原大王之转攻楚也。”

于是越遂释齐而伐楚。楚威王兴兵而伐之,大败越,杀王无彊,尽取故吴地至浙江,北破齐于徐州。而越以此散,诸族子争立,或为王,或为君,滨于江南海上,服朝于楚。

后七世,至闽君摇,佐诸侯平秦。汉高帝复以摇为越王,以奉越后。东越,闽君,皆其后也。

范蠡事越王勾践,既苦身戮力,与勾践深谋二十馀年,竟灭吴,报会稽之耻,北渡兵于淮以临齐、晋,号令中国,以尊周室,勾践以霸,而范蠡称上将军。还反国,范蠡以为大名之下,难以久居,且勾践为人可与同患,难与处安,为书辞勾践曰:“臣闻主忧臣劳,主辱臣死。昔者君王辱于会稽,所以不死,为此事也。今既以雪耻,臣请从会稽之诛。”勾践曰:“孤将与子分国而有之。不然,将加诛于子。”范蠡曰:“君行令,臣行意。”乃装其轻宝珠玉,自与其私徒属乘舟浮海以行,终不反。于是勾践表会稽山以为范蠡奉邑。

范蠡浮海出齐,变姓名,自谓鸱夷子皮,耕于海畔,苦身戮力,父子治产。居无几何,致产数十万。齐人闻其贤,以为相。范蠡喟然叹曰:“居家则致千金,居官则至卿相,此布衣之极也。久受尊名,不祥。”乃归相印,尽散其财,以分与知友乡党,而怀其重宝,间行以去,止于陶,以为此天下之中,交易有无之路通,为生可以致富矣。于是自谓陶硃公。复约要父子耕畜,废居,候时转物,逐什一之利。居无何,则致赀累巨万。天下称陶硃公。

硃公居陶,生少子。少子及壮,而硃公中男杀人,囚于楚。硃公曰:“杀人而死,职也。然吾闻千金之子不死于市。”告其少子往视之。乃装黄金千溢,置褐器中,载以一牛车。且遣其少子,硃公长男固请欲行,硃公不听。长男曰:“家有长子曰家督,今弟有罪,大人不遣,乃遗少弟,是吾不肖。”欲自杀。其母为言曰:“今遣少子,未必能生中子也,而先空亡长男,柰何?”硃公不得已而遣长子,为一封书遗故所善庄生。曰:“至则进千金于庄生所,听其所为,慎无与争事。”长男既行,亦自私赍数百金。

至楚,庄生家负郭,披藜藋到门,居甚贫。然长男发书进千金,如其父言。庄生曰:“可疾去矣,慎毋留!即弟出,勿问所以然。”长男既去,不过庄生而私留,以其私赍献遗楚国贵人用事者。

庄生虽居穷阎,然以廉直闻于国,自楚王以下皆师尊之。及硃公进金,非有意受也,欲以成事后复归之以为信耳。故金至,谓其妇曰:“此硃公之金。有如病不宿诫,后复归,勿动。”而硃公长男不知其意,以为殊无短长也。

庄生间时入见楚王,言“某星宿某,此则害于楚”。楚王素信庄生,曰:“今为柰何?”庄生曰:“独以德为可以除之。”楚王曰:“生休矣,寡人将行之。”王乃使使者封三钱之府。楚贵人惊告硃公长男曰:“王且赦。”曰:“何以也?”曰:“每王且赦,常封三钱之府。昨暮王使使封之。”硃公长男以为赦,弟固当出也,重千金虚弃庄生,无所为也,乃复见庄生。庄生惊曰:“若不去邪?”长男曰:“固未也。初为事弟,弟今议自赦,故辞生去。”庄生知其意欲复得其金,曰:“若自入室取金。”长男即自入室取金持去,独自欢幸。

庄生羞为兒子所卖,乃入见楚王曰:“臣前言某星事,王言欲以修德报之。今臣出,道路皆言陶之富人硃公之子杀人囚楚,其家多持金钱赂王左右,故王非能恤楚国而赦,乃以硃公子故也。”楚王大怒曰:“寡人虽不德耳,柰何以硃公之子故而施惠乎!”令论杀硃公子,明日遂下赦令。硃公长男竟持其弟丧归。

至,其母及邑人尽哀之,唯硃公独笑,曰:“吾固知必杀其弟也!彼非不爱其弟,顾有所不能忍者也。是少与我俱,见苦,为生难,故重弃财。至如少弟者,生而见我富,乘坚驱良逐狡兔,岂知财所从来,故轻弃之,非所惜吝。前日吾所为欲遣少子,固为其能弃财故也。而长者不能,故卒以杀其弟,事之理也,无足悲者。吾日夜固以望其丧之来也。”故范蠡三徙,成名于天下,非苟去而已,所止必成名。卒老死于陶,故世传曰陶硃公。

太史公曰:禹之功大矣,渐九川,定九州,至于今诸夏艾安。及苗裔勾践,苦身焦思,终灭彊吴,北观兵中国,以尊周室,号称霸王。勾践可不谓贤哉!盖有禹之遗烈焉。范蠡三迁皆有荣名,名垂后世。臣主若此,欲毋显得乎!

越祖少康,至于允常。其子始霸,与吴争彊。槜李之役,阖闾见伤。会稽之耻,勾践欲当。种诱以利,蠡悉其良。折节下士,致胆思尝。卒复雠寇,遂殄大邦。后不量力,灭于无彊。
\end{yuanwen}

\chapter{郑世家}

\begin{yuanwen}
郑桓公友者,周厉王少子而宣王庶弟也。宣王立二十二年,友初封于郑。封三十三岁,百姓皆便爱之。幽王以为司徒。和集周民,周民皆说,河雒之间,人便思之。为司徒一岁,幽王以襃后故,王室治多邪,诸侯或畔之。于是桓公问太史伯曰:“王室多故,予安逃死乎?”太史伯对曰:“独雒之东土,河济之南可居。”公曰:“何以?”对曰:“地近虢、郐,虢、郐之君贪而好利,百姓不附。今公为司徒,民皆爱公,公诚请居之,虢、郐之君见公方用事,轻分公地。公诚居之,虢、郐之民皆公之民也。”公曰:“吾欲南之江上,何如?”对曰:“昔祝融为高辛氏火正,其功大矣,而其于周未有兴者,楚其后也。周衰,楚必兴。兴,非郑之利也。”公曰:“吾欲居西方,何如?”对曰:“其民贪而好利,难久居。”公曰:“周衰,何国兴者?”对曰:“齐、秦、晋、楚乎?夫齐,姜姓,伯夷之后也,伯夷佐尧典礼。秦,嬴姓,伯翳之后也,伯翳佐舜怀柔百物。及楚之先,皆尝有功于天下。而周武王克纣后,成王封叔虞于唐,其地阻险,以此有德与周衰并,亦必兴矣。”桓公曰:“善。”于是卒言王,东徙其民雒东,而虢、郐果献十邑,竟国之。

二岁,犬戎杀幽王于骊山下,并杀桓公。郑人共立其子掘突,是为武公。

武公十年,娶申侯女为夫人,曰武姜。生太子寤生,生之难,及生,夫人弗爱。后生少子叔段,段生易,夫人爱之。二十七年,武公疾。夫人请公,欲立段为太子,公弗听。是岁,武公卒,寤生立,是为庄公。

庄公元年,封弟段于京,号太叔。祭仲曰:“京大于国,非所以封庶也。”庄公曰:“武姜欲之,我弗敢夺也。”段至京,缮治甲兵,与其母武姜谋袭郑。二十二年,段果袭郑,武姜为内应。庄公发兵伐段,段走。伐京,京人畔段,段出走鄢。鄢溃,段出奔共。于是庄公迁其母武姜于城颍,誓言曰:“不至黄泉,毋相见也。”居岁馀,已悔思母。颍谷之考叔有献于公,公赐食。考叔曰:“臣有母,请君食赐臣母。”庄公曰:“我甚思母,恶负盟,柰何?”考叔曰:“穿地至黄泉,则相见矣。”于是遂从之,见母。

二十四年,宋缪公卒,公子冯奔郑。郑侵周地,取禾。二十五年,卫州吁弑其君桓公自立,与宋伐郑,以冯故也。二十七年,始朝周桓王。桓王怒其取禾,弗礼也。二十九年,庄公怒周弗礼,与鲁易祊、许田。三十三年,宋杀孔父。三十七年,庄公不朝周,周桓王率陈、蔡、虢、卫伐郑。庄公与祭仲、高渠弥发兵自救,王师大败。祝聸射中王臂。祝聸请从之,郑伯止之,曰:“犯长且难之,况敢陵天子乎?”乃止。夜令祭仲问王疾。

三十八年,北戎伐齐,齐使求救,郑遣太子忽将兵救齐。齐釐公欲妻之,忽谢曰:“我小国,非齐敌也。”时祭仲与俱,劝使取之,曰:“君多内宠,太子无大援将不立,三公子皆君也。”所谓三公子者,太子忽,其弟突,次弟子亹也。

四十三年,郑庄公卒。初,祭仲甚有宠于庄公,庄公使为卿;公使娶邓女,生太子忽,故祭仲立之,是为昭公。

庄公又娶宋雍氏女,生厉公突。雍氏有宠于宋。宋庄公闻祭仲之立忽,乃使人诱召祭仲而执之,曰:“不立突,将死。”亦执突以求赂焉。祭仲许宋,与宋盟。以突归,立之。昭公忽闻祭仲以宋要立其弟突,九月丁亥,忽出奔卫。己亥,突至郑,立,是为厉公。

厉公四年,祭仲专国政。厉公患之,阴使其婿雍纠欲杀祭仲。纠妻,祭仲女也,知之,谓其母曰:“父与夫孰亲?”母曰:“父一而已,人尽夫也。”女乃告祭仲,祭仲反杀雍纠,戮之于市。厉公无柰祭仲何,怒纠曰:“谋及妇人,死固宜哉!”夏,厉公出居边邑栎。祭仲迎昭公忽,六月乙亥,复入郑,即位。

秋,郑厉公突因栎人杀其大夫单伯,遂居之。诸侯闻厉公出奔,伐郑,弗克而去。宋颇予厉公兵,自守于栎,郑以故亦不伐栎。

昭公二年,自昭公为太子时,父庄公欲以高渠弥为卿,太子忽恶之,庄公弗听,卒用渠弥为卿。及昭公即位,惧其杀己,冬十月辛卯,渠弥与昭公出猎,射杀昭公于野。祭仲与渠弥不敢入厉公,乃更立昭公弟子亹为君,是为子亹也,无谥号。

子亹元年七月,齐襄公会诸侯于首止,郑子亹往会,高渠弥相,从,祭仲称疾不行。所以然者,子亹自齐襄公为公子之时,尝会斗,相仇,及会诸侯,祭仲请子亹无行。子亹曰:“齐彊,而厉公居栎,即不往,是率诸侯伐我,内厉公。我不如往,往何遽必辱,且又何至是!”卒行。于是祭仲恐齐并杀之,故称疾。子亹至,不谢齐侯,齐侯怒,遂伏甲而杀子亹。高渠弥亡归,归与祭仲谋,召子亹弟公子婴于陈而立之,是为郑子。是岁,齐襄公使彭生醉拉杀鲁桓公。

郑子八年,齐人管至父等作乱,弑其君襄公。十二年,宋人长万弑其君湣公。郑祭仲死。

十四年,故郑亡厉公突在栎者使人诱劫郑大夫甫假,要以求入。假曰:“舍我,我为君杀郑子而入君。”厉公与盟,乃舍之。六月甲子,假杀郑子及其二子而迎厉公突,突自栎复入即位。初,内蛇与外蛇斗于郑南门中,内蛇死。居六年,厉公果复入。入而让其伯父原曰:“我亡国外居,伯父无意入我,亦甚矣。”原曰:“事君无二心,人臣之职也。原知罪矣。”遂自杀。厉公于是谓甫假曰:“子之事君有二心矣。”遂诛之。假曰:“重德不报,诚然哉!”

厉公突后元年,齐桓公始霸。

五年,燕、卫与周惠王弟穨伐王,王出奔温,立弟穨为王。六年,惠王告急郑,厉公发兵击周王子穨,弗胜,于是与周惠王归,王居于栎。七年春,郑厉公与虢叔袭杀王子穨而入惠王于周。

秋,厉公卒,子文公踕立。厉公初立四岁,亡居栎,居栎十七岁,复入,立七岁,与亡凡二十八年。

文公十七年,齐桓公以兵破蔡,遂伐楚,至召陵。

二十四年,文公之贱妾曰燕姞,梦天与之兰,曰:“余为伯鯈。余,尔祖也。以是为而子,兰有国香。”以梦告文公,文公幸之,而予之草兰为符。遂生子,名曰兰。

三十六年,晋公子重耳过,文公弗礼。文公弟叔詹曰:“重耳贤,且又同姓,穷而过君,不可无礼。”文公曰:“诸侯亡公子过者多矣,安能尽礼之!”詹曰:“君如弗礼,遂杀之;弗杀,使即反国,为郑忧矣。”文公弗听。

三十七年春,晋公子重耳反国,立,是为文公。秋,郑入滑,滑听命,已而反与卫,于是郑伐滑。周襄王使伯馃请滑。郑文公怨惠王之亡在栎,而文公父厉公入之,而惠王不赐厉公爵禄,又怨襄王之与卫滑,故不听襄王请而囚伯馃。王怒,与翟人伐郑,弗克。冬,翟攻伐襄王,襄王出奔郑,郑文公居王于氾。三十八年,晋文公入襄王成周。

四十一年,助楚击晋。自晋文公之过无礼,故背晋助楚。四十三年,晋文公与秦穆公共围郑,讨其助楚攻晋者,及文公过时之无礼也。初,郑文公有三夫人,宠子五人,皆以罪蚤死。公怒,溉逐群公子。子兰奔晋,从晋文公围郑。时兰事晋文公甚谨,爱幸之,乃私于晋,以求入郑为太子。晋于是欲得叔詹为僇。郑文公恐,不敢谓叔詹言。詹闻,言于郑君曰:“臣谓君,君不听臣,晋卒为患。然晋所以围郑,以詹,詹死而赦郑国,詹之原也。”乃自杀。郑人以詹尸与晋。晋文公曰:“必欲一见郑君,辱之而去。”郑人患之,乃使人私于秦曰:“破郑益晋,非秦之利也。”秦兵罢。晋文公欲入兰为太子,以告郑。郑大夫石癸曰:“吾闻姞姓乃后稷之元妃,其后当有兴者。子兰母,其后也。且夫人子尽已死,馀庶子无如兰贤。今围急,晋以为请,利孰大焉!”遂许晋,与盟,而卒立子兰为太子,晋兵乃罢去。

四十五年,文公卒,子兰立,是为缪公。

缪公元年春,秦缪公使三将将兵欲袭郑,至滑,逢郑贾人弦高诈以十二牛劳军,故秦兵不至而还,晋败之于崤。初,往年郑文公之卒也,郑司城缯贺以郑情卖之,秦兵故来。三年,郑发兵从晋伐秦,败秦兵于汪。

往年楚太子商臣弑其父成王代立。二十一年,与宋华元伐郑。华元杀羊食士,不与其御羊斟,怒以驰郑,郑囚华元。宋赎华元,元亦亡去。晋使赵穿以兵伐郑。

二十二年,郑缪公卒,子夷立,是为灵公。

灵公元年春,楚献鼋于灵公。子家、子公将朝灵公,子公之食指动,谓子家曰:“佗日指动,必食异物。”及入,见灵公进鼋羹,子公笑曰:“果然!”灵公问其笑故,具告灵公。灵公召之,独弗予羹。子公怒,染其指,尝之而出。公怒,欲杀子公。子公与子家谋先。夏,弑灵公。郑人欲立灵公弟去疾,去疾让曰:“必以贤,则去疾不肖;必以顺,则公子坚长。”坚者,灵公庶弟,去疾之兄也。于是乃立子坚,是为襄公。

襄公立,将尽去缪氏。缪氏者,杀灵公、子公之族家也。去疾曰:“必去缪氏,我将去之。”乃止。皆以为大夫。

襄公元年,楚怒郑受宋赂纵华元,伐郑。郑背楚,与晋亲。五年,楚复伐郑,晋来救之。六年,子家卒,国人复逐其族,以其弑灵公也。

七年,郑与晋盟鄢陵。八年,楚庄王以郑与晋盟,来伐,围郑三月,郑以城降楚。楚王入自皇门,郑襄公肉袒掔羊以迎,曰:“孤不能事边邑,使君王怀怒以及弊邑,孤之罪也。敢不惟命是听。君王迁之江南,及以赐诸侯,亦惟命是听。若君王不忘厉、宣王,桓、武公,哀不忍绝其社稷,锡不毛之地,使复得改事君王,孤之原也,然非所敢望也。敢布腹心,惟命是听。”庄王为卻三十里而后舍。楚群臣曰:“自郢至此,士大夫亦久劳矣。今得国舍之,何如?”庄王曰:“所为伐,伐不服也。今已服,尚何求乎?”卒去。晋闻楚之伐郑,发兵救郑。其来持两端,故迟,比至河,楚兵已去。晋将率或欲渡,或欲还,卒渡河。庄王闻,还击晋。郑反助楚,大破晋军于河上。十年,晋来伐郑,以其反晋而亲楚也。

十一年,楚庄王伐宋,宋告急于晋。晋景公欲发兵救宋,伯宗谏晋君曰:“天方开楚,未可伐也。”乃求壮士得霍人解扬,字子虎,诓楚,令宋毋降。过郑,郑与楚亲,乃执解扬而献楚。楚王厚赐与约,使反其言,令宋趣降,三要乃许。于是楚登解扬楼车,令呼宋。遂负楚约而致其晋君命曰:“晋方悉国兵以救宋,宋虽急,慎毋降楚,晋兵今至矣!”楚庄王大怒,将杀之。解扬曰:“君能制命为义,臣能承命为信。受吾君命以出,有死无陨。”庄王曰:“若之许我,已而背之,其信安在?”解扬曰:“所以许王,欲以成吾君命也。”将死,顾谓楚军曰:“为人臣无忘尽忠得死者!”楚王诸弟皆谏王赦之,于是赦解扬使归。晋爵之为上卿。

十八年,襄公卒,子悼公晞立。

悼公元年,鄦公恶郑于楚,悼公使弟睔于楚自讼。讼不直,楚囚睔。于是郑悼公来与晋平,遂亲。睔私于楚子反,子反言归睔于郑。

二年,楚伐郑,晋兵来救。是岁,悼公卒,立其弟睔,是为成公。

成公三年,楚共王曰“郑成公孤有德焉”,使人来与盟。成公私与盟。秋,成公朝晋,晋曰“郑私平于楚”,执之。使栾书伐郑。四年春,郑患晋围,公子如乃立成公庶兄繻为君。其四月,晋闻郑立君,乃归成公。郑人闻成公归,亦杀君繻,迎成公。晋兵去。

十年,背晋盟,盟于楚。晋厉公怒,发兵伐郑。楚共王救郑。晋楚战鄢陵,楚兵败,晋射伤楚共王目,俱罢而去。十三年,晋悼公伐郑,兵于洧上。郑城守,晋亦去。

十四年,成公卒,子恽立。是为釐公。

釐公五年,郑相子驷朝釐公,釐公不礼。子驷怒,使厨人药杀釐公,赴诸侯曰“釐公暴病卒”。立釐公子嘉,嘉时年五岁,是为简公。

简公元年,诸公子谋欲诛相子驷,子驷觉之,反尽诛诸公子。二年,晋伐郑,郑与盟,晋去。冬,又与楚盟。子驷畏诛,故两亲晋、楚。三年,相子驷欲自立为君,公子子孔使尉止杀相子驷而代之。子孔又欲自立。子产曰:“子驷为不可,诛之,今又效之,是乱无时息也。”于是子孔从之而相郑简公。

四年,晋怒郑与楚盟,伐郑,郑与盟。楚共王救郑,败晋兵。简公欲与晋平,楚又囚郑使者。

十二年,简公怒相子孔专国权,诛之,而以子产为卿。十九年,简公如晋请卫君还,而封子产以六邑。子产让,受其三邑。二十二年,吴使延陵季子于郑,见子产如旧交,谓子产曰:“郑之执政者侈,难将至,政将及子。子为政,必以礼;不然,郑将败。”子产厚遇季子。二十三年,诸公子争宠相杀,又欲杀子产。公子或谏曰:“子产仁人,郑所以存者子产也,勿杀!”乃止。

二十五年,郑使子产于晋,问平公疾。平公曰:“卜而曰实沈、台骀为祟,史官莫知,敢问?”对曰:“高辛氏有二子,长曰阏伯,季曰实沈,居旷林,不相能也,日操干戈以相征伐。后帝弗臧,迁阏伯于商丘,主辰,商人是因,故辰为商星。迁实沈于大夏,主参,唐人是因,服事夏、商,其季世曰唐叔虞。”当武王邑姜方娠大叔,梦帝谓己:‘余命而子曰虞,乃与之唐,属之参而蕃育其子孙。’及生有文在其掌曰‘虞’,遂以命之。及成王灭唐而国大叔焉。故参为晋星。”由是观之,则实沈,参神也。昔金天氏有裔子曰昧,为玄冥师,生允格、台骀。台骀能业其官,宣汾、洮,障大泽,以处太原。帝用嘉之,国之汾川。沈、姒、蓐、黄实守其祀。今晋主汾川而灭之。由是观之,则台骀,汾、洮神也。然是二者不害君身。山川之神,则水旱之菑禜之;日月星辰之神,则雪霜风雨不时禜之;若君疾,饮食哀乐女色所生也。”平公及叔乡曰:“善,博物君子也!”厚为之礼于子产。

二十七年夏,郑简公朝晋。冬,畏楚灵王之彊,又朝楚,子产从。二十八年,郑君病,使子产会诸侯,与楚灵王盟于申,诛齐庆封。

三十六年,简公卒,子定公宁立。秋,定公朝晋昭公。

定公元年,楚公子弃疾弑其君灵王而自立,为平王。欲行德诸侯。归灵王所侵郑地于郑。

四年,晋昭公卒,其六卿彊,公室卑。子产谓韩宣子曰:“为政必以德,毋忘所以立。”

六年,郑火,公欲禳之。子产曰:“不如修德。”

八年,楚太子建来奔。十年,太子建与晋谋袭郑。郑杀建,建子胜奔吴。

十一年,定公如晋。晋与郑谋,诛周乱臣,入敬王于周。

十三年,定公卒,子献公虿立。献公十三年卒,子声公胜立。当是时,晋六卿彊,侵夺郑,郑遂弱。

声公五年,郑相子产卒,郑人皆哭泣,悲之如亡亲戚。子产者,郑成公少子也。为人仁爱人,事君忠厚。孔子尝过郑,与子产如兄弟云。及闻子产死,孔子为泣曰:“古之遗爱也!”

八年,晋范、中行氏反晋,告急于郑,郑救之。晋伐郑,败郑军于铁。

十四年,宋景公灭曹。二十年,齐田常弑其君简公,而常相于齐。二十二年,楚惠王灭陈。孔子卒。

三十六年,晋知伯伐郑,取九邑。

三十七年,声公卒,子哀公易立。哀公八年,郑人弑哀公而立声公弟丑,是为共公。共公三年,三晋灭知伯。三十一年,共公卒,子幽公已立。幽公元年,韩武子伐郑,杀幽公。郑人立幽公弟骀,是为繻公。

繻公十五年,韩景侯伐郑,取雍丘。郑城京。

十六年,郑伐韩,败韩兵于负黍。二十年,韩、赵、魏列为诸侯。二十三年,郑围韩之阳翟。

二十五年,郑君杀其相子阳。二十七,子阳之党共弑繻公骀而立幽公弟乙为君,是为郑君。

郑君乙立二年,郑负黍反,复归韩。十一年,韩伐郑,取阳城。

二十一年,韩哀侯灭郑,并其国。

太史公曰:语有之,“以权利合者,权利尽而交疏”,甫瑕是也。甫瑕虽以劫杀郑子内厉公,厉公终背而杀之,此与晋之里克何异?守节如荀息,身死而不能存奚齐。变所从来,亦多故矣!

厉王之子,得封于郑。代职司徒,缁衣在咏。虢、郐献邑,祭祝专命。庄既犯王,厉亦奔命。居栎克入,梦兰毓庆。伯服生囚,叔瞻尸聘。釐、简之后,公室不竞。负黍虽还,韩哀日盛。
\end{yuanwen}

\part{卷四十三}
\chapter{赵世家第十三}

\begin{yuanwen}
赵氏之先,与秦共祖。至中衍,为帝大戊\footnote{即太戊,商朝第九代帝王。}御\footnote{驾车的人。}。其后世蜚廉有子二人,而命其一子曰恶来,事纣,为周所杀,其后为秦。恶来弟曰季胜,其后为赵。
\end{yuanwen}

赵国的先世,与嬴秦是同一个先祖。传承到中衍时,他担任商朝大戊帝的御者。中衍的后代蜚廉生有两个儿子,一个儿子名叫恶来,服侍殷纣王,被周人杀掉,他的后人便是嬴秦氏。恶来的弟弟名叫季胜,他的后代便是赵人。

\begin{yuanwen}
季胜生孟增。孟增幸于周成王,是为宅皋狼。皋狼生衡父,衡父生造父。造父幸于周缪王。造父取骥之乘匹,与桃林盗骊、骅骝、绿耳,献之缪王。缪王使造父御,西巡狩,见西王母,乐之忘归。而徐偃王反,缪王日驰千里马\footnote{此字疑为衍字。},攻徐偃王,大破之。乃赐造父以赵城,由此为赵氏。
\end{yuanwen}

季胜生了孟增。孟增受到周成王的宠信,又被称为宅皋狼。皋狼生衡父,衡父生造父。造父受周缪王的宠信。造父挑选千里马,和桃林的盗骊、骅骝、绿耳等好马一起进献给缪王。缪王命令造父驾车,到西方各诸侯国巡视,见到西王母,高兴得忘记返回。徐偃王趁机发动叛乱,缪王驾好马一日奔驰千里,攻打徐偃王,把他彻底击败。于是把赵城分封给造父,从此以赵为氏。

\begin{yuanwen}
自造父已下六世至奄父,曰公仲,周宣王时伐戎,为御。及千亩战,奄父脱宣王。奄父生叔带。叔带之时,周幽王无道,去周如晋,事晋文侯,始建赵氏于晋国。
\end{yuanwen}

从造父往下传承六代到了奄父,奄父名公仲,周宣王时出兵讨伐戎族,他为宣王驾车。等到在千亩战斗时,奄父帮助宣王脱离了危险。奄父生叔带。叔带时,周幽王荒淫无道,于是他就离开周国前往晋国,为晋文侯做事,开始在晋国建立起赵氏的根基。

\begin{yuanwen}
自叔带以下,赵宗益兴,五世而至赵夙。

赵夙,晋献公之十六年伐霍、魏、耿,而赵夙为将伐霍。霍公求(饹/奔)齐。晋大旱,卜之,曰“霍太山为祟\footnote{鬼神给人带来的灾祸。}”。

使赵夙召霍君于齐,复之,以奉霍太山之祀,晋复穰\footnote{丰收。}。晋献公赐赵夙耿。

夙生共孟,当鲁闵公之元年也。共孟生赵衰,字子馀。
\end{yuanwen}

从叔带开始,赵氏宗族逐渐兴盛,传承五代到了赵夙。

赵夙,在晋献公十六年(前661年)讨伐霍、魏、耿三国时,他率领军队讨伐霍国。霍公求逃奔齐国。这一年晋国大旱,献公让人占卜,说是“霍太山的神灵在作怪”。

于是献公派赵夙前往齐国迎回霍君,让他复位,主持对霍太山的祭祀,晋国也重新获得丰收。晋献公把原本属于耿国的土地赏赐给赵夙。

赵夙生下共孟,那一年是鲁闵公元年(前661年)。共孟生下赵衰,赵衰字子馀。

\begin{yuanwen}
赵衰卜事晋献公及诸公子,莫吉;卜事公子重耳,吉,即事重耳。重耳以骊姬之乱亡奔翟,赵衰从。翟伐廧咎如\footnote{当时的少数民族。},得二女,翟以其少女妻重耳,长女妻赵衰而生盾。
\end{yuanwen}

赵衰以占卜决定去侍奉晋献公还是诸位公子,结果都不吉利;占卜侍奉公子重耳的情况,吉,于是前去侍奉重耳。重耳因为骊姬陷害太子的变乱逃奔翟国,赵衰追随左右。翟国讨伐廧咎如,俘虏了两位女子,翟君把年轻的女子嫁给重耳,把年长的女子嫁给赵衰从而生下赵盾。

\begin{yuanwen}
初,重耳在晋时,赵衰妻亦生赵同、赵括、赵婴齐。赵衰从重耳出亡,凡十九年,得反国。重耳为晋文公,赵衰为原大夫,居原,任国政。文公所以反国及霸,多赵衰计策,语在晋事中。
\end{yuanwen}

起初,重耳在晋国时,赵衰的妻子已经生下了儿子赵同、赵括、赵婴齐。赵衰追随重耳流亡国外,共计十九年,才返回晋国。重耳继位成为晋文公,赵衰担任原大夫,居住在原地,掌管国家政事。晋文公能够回国并且称霸,大多是依靠赵衰的谋划,相关记载都写在《晋世家》中。

\begin{yuanwen}
赵衰既反晋,晋之妻固要迎翟妻,而以其子盾为適\footnote{通“嫡”。}嗣,晋妻三子皆下事之。晋襄公之六年,而赵衰卒,谥为成季。
\end{yuanwen}

赵衰返回晋国后,在晋国的原配夫人坚持要他接回在翟国的妻子,并且让翟妻的儿子赵盾作为继承人,自己所生的三个儿子地位均在赵盾之下并侍奉他。晋襄公六年(前622年),赵衰去世,谥号为成季。

\begin{yuanwen}
赵盾代成季任国政二年而晋襄公卒,太子夷皋年少。盾为国多难,欲立襄公弟雍。雍时在秦,使使迎之。太子母日夜啼泣,顿首谓赵盾曰:“先君何罪,释其適子而更求君?”

赵盾患之,恐其宗与大夫袭诛之,乃遂立太子,是为灵公,发兵距所迎襄公弟于秦者。灵公既立,赵盾益专国政。
\end{yuanwen}

赵盾接替成季掌控国政两年后晋襄公去世,太子夷皋年幼。赵盾由于国家处于多事之秋,想拥立襄公的弟弟雍。雍当时在秦国,就派使臣前去迎接他。太子的母亲日夜啼哭,叩头对赵盾说:“先君犯了什么罪过,为什么想抛弃他的嫡子而另外选立国君呢?”

赵盾为此事担心,害怕太子母亲的族人以及士大夫谋害他,于是就拥立太子,便是晋灵公,并派兵阻拦去秦国迎接公子雍的人。灵公即位之后,赵盾更加独揽朝政。

\begin{yuanwen}

\end{yuanwen}\begin{yuanwen}

\end{yuanwen}\begin{yuanwen}

\end{yuanwen}\begin{yuanwen}

\end{yuanwen}\begin{yuanwen}

\end{yuanwen}\begin{yuanwen}

\end{yuanwen}\begin{yuanwen}

\end{yuanwen}\begin{yuanwen}

\end{yuanwen}\begin{yuanwen}

\end{yuanwen}\begin{yuanwen}

\end{yuanwen}\begin{yuanwen}

\end{yuanwen}\begin{yuanwen}

\end{yuanwen}\begin{yuanwen}

\end{yuanwen}\begin{yuanwen}

\end{yuanwen}\begin{yuanwen}

\end{yuanwen}\begin{yuanwen}

\end{yuanwen}\begin{yuanwen}

\end{yuanwen}\begin{yuanwen}

\end{yuanwen}\begin{yuanwen}

\end{yuanwen}\begin{yuanwen}

\end{yuanwen}\begin{yuanwen}

\end{yuanwen}\begin{yuanwen}

\end{yuanwen}\begin{yuanwen}

\end{yuanwen}\begin{yuanwen}

\end{yuanwen}\begin{yuanwen}

\end{yuanwen}\begin{yuanwen}

\end{yuanwen}\begin{yuanwen}

\end{yuanwen}\begin{yuanwen}

\end{yuanwen}\begin{yuanwen}

\end{yuanwen}\begin{yuanwen}

\end{yuanwen}\begin{yuanwen}

\end{yuanwen}\begin{yuanwen}

\end{yuanwen}\begin{yuanwen}

\end{yuanwen}\begin{yuanwen}

\end{yuanwen}\begin{yuanwen}



灵公立十四年,益骄。赵盾骤谏,灵公弗听。及食熊蹯,胹不熟,杀宰人,持其尸出,赵盾见之。灵公由此惧,欲杀盾。盾素仁爱人,尝所食桑下饿人反扞救盾,盾以得亡。未出境,而赵穿弑灵公而立襄公弟黑臀,是为成公。赵盾复反,任国政。君子讥盾“为正卿,亡不出境,反不讨贼”,故太史书曰“赵盾弑其君”。晋景公时而赵盾卒,谥为宣孟,子朔嗣。

赵朔,晋景公之三年,朔为晋将下军救郑,与楚庄王战河上。朔娶晋成公姊为夫人。

晋景公之三年,大夫屠岸贾欲诛赵氏。初,赵盾在时,梦见叔带持要而哭,甚悲;已而笑,拊手且歌。盾卜之,兆绝而后好。赵史援占之,曰:“此梦甚恶,非君之身,乃君之子,然亦君之咎。至孙,赵将世益衰。”屠岸贾者,始有宠于灵公,及至于景公而贾为司寇,将作难,乃治灵公之贼以致赵盾,遍告诸将曰:“盾虽不知,犹为贼首。以臣弑君,子孙在朝,何以惩罪?请诛之。”韩厥曰:“灵公遇贼,赵盾在外,吾先君以为无罪,故不诛。今诸君将诛其后,是非先君之意而今妄诛。妄诛谓之乱。臣有大事而君不闻,是无君也。”屠岸贾不听。韩厥告赵朔趣亡。朔不肯,曰:“子必不绝赵祀,朔死不恨。”韩厥许诺,称疾不出。贾不请而擅与诸将攻赵氏于下宫,杀赵朔、赵同、赵括、赵婴齐,皆灭其族。

赵朔妻成公姊,有遗腹,走公宫匿。赵朔客曰公孙杵臼,杵臼谓朔友人程婴曰:“胡不死?”程婴曰:“朔之妇有遗腹,若幸而男,吾奉之;即女也,吾徐死耳。”居无何,而朔妇免身,生男。屠岸贾闻之,索于宫中。夫人置兒绔中,祝曰:“赵宗灭乎,若号;即不灭,若无声。”及索,兒竟无声。已脱,程婴谓公孙杵臼曰:“今一索不得,后必且复索之,柰何?”公孙杵臼曰:“立孤与死孰难?”程婴曰:“死易,立孤难耳。”公孙杵臼曰:“赵氏先君遇子厚,子彊为其难者,吾为其易者,请先死。”乃二人谋取他人婴兒负之,衣以文葆,匿山中。程婴出,谬谓诸将军曰:“婴不肖,不能立赵孤。谁能与我千金,吾告赵氏孤处。”诸将皆喜,许之,发师随程婴攻公孙杵臼。杵臼谬曰:“小人哉程婴!昔下宫之难不能死,与我谋匿赵氏孤兒,今又卖我。纵不能立,而忍卖之乎!”抱兒呼曰:“天乎天乎!赵氏孤兒何罪?请活之,独杀杵臼可也。”诸将不许,遂杀杵臼与孤兒。诸将以为赵氏孤兒良已死,皆喜。然赵氏真孤乃反在,程婴卒与俱匿山中。

居十五年,晋景公疾,卜之,大业之后不遂者为祟。景公问韩厥,厥知赵孤在,乃曰:“大业之后在晋绝祀者,其赵氏乎?夫自中衍者皆嬴姓也。中衍人面鸟噣,降佐殷帝大戊,及周天子,皆有明德。下及幽厉无道,而叔带去周適晋,事先君文侯,至于成公,世有立功,未尝绝祀。今吾君独灭赵宗,国人哀之,故见龟策。唯君图之。”景公问:“赵尚有后子孙乎?”韩厥具以实告。于是景公乃与韩厥谋立赵孤兒,召而匿之宫中。诸将入问疾,景公因韩厥之众以胁诸将而见赵孤。赵孤名曰武。诸将不得已,乃曰:“昔下宫之难,屠岸贾为之,矫以君命,并命群臣。非然,孰敢作难!微君之疾,群臣固且请立赵后。今君有命,群臣之原也。”于是召赵武、程婴遍拜诸将,遂反与程婴、赵武攻屠岸贾,灭其族。复与赵武田邑如故。

及赵武冠,为成人,程婴乃辞诸大夫,谓赵武曰:“昔下宫之难,皆能死。我非不能死,我思立赵氏之后。今赵武既立,为成人,复故位,我将下报赵宣孟与公孙杵臼。”赵武啼泣顿首固请,曰:“武原苦筋骨以报子至死,而子忍去我死乎!”程婴曰:“不可。彼以我为能成事,故先我死;今我不报,是以我事为不成。”遂自杀。赵武服齐衰三年,为之祭邑,春秋祠之,世世勿绝。

赵氏复位十一年,而晋厉公杀其大夫三郤。栾书畏及,乃遂弑其君厉公,更立襄公曾孙周,是为悼公。晋由此大夫稍彊。

赵武续赵宗二十七年,晋平公立。平公十二年,而赵武为正卿。十三年,吴延陵季子使于晋,曰:“晋国之政卒归于赵武子、韩宣子、魏献子之后矣。”赵武死,谥为文子。

文子生景叔。景叔之时,齐景公使晏婴于晋,晏婴与晋叔向语。婴曰:“齐之政后卒归田氏。”叔向亦曰:“晋国之政将归六卿。六卿侈矣,而吾君不能恤也。”

赵景叔卒,生赵鞅,是为简子。

赵简子在位,晋顷公之九年,简子将合诸侯戍于周。其明年,入周敬王于周,辟弟子朝之故也。

晋顷公之十二年,六卿以法诛公族祁氏、羊舌氏,分其邑为十县,六卿各令其族为之大夫。晋公室由此益弱。

后十三年,鲁贼臣阳虎来奔,赵简子受赂,厚遇之。

赵简子疾,五日不知人,大夫皆惧。医扁鹊视之,出,董安于问。扁鹊曰:“血脉治也,而何怪!在昔秦缪公尝如此,七日而寤。寤之日,告公孙支与子舆曰:‘我之帝所甚乐。吾所以久者,適有学也。帝告我:“晋国将大乱,五世不安;其后将霸,未老而死;霸者之子且令而国男女无别。”’公孙支书而藏之,秦谶于是出矣。献公之乱,文公之霸,而襄公败秦师于殽而归纵淫,此子之所闻。今主君之疾与之同,不出三日疾必间,间必有言也。”

居二日半,简子寤。语大夫曰:“我之帝所甚乐,与百神游于钧天,广乐九奏万舞,不类三代之乐,其声动人心。有一熊欲来援我,帝命我射之,中熊,熊死。又有一罴来,我又射之,中罴,罴死。帝甚喜,赐我二笥,皆有副。吾见兒在帝侧,帝属我一翟犬,曰:‘及而子之壮也,以赐之。’帝告我:‘晋国且世衰,七世而亡,嬴姓将大败周人于范魁之西,而亦不能有也。今余思虞舜之勋,適余将以其胄女孟姚配而七世之孙。’”董安于受言而书藏之。以扁鹊言告简子,简子赐扁鹊田四万亩。

他日,简子出,有人当道,辟之不去,从者怒,将刃之。当道者曰:“吾欲有谒于主君。”从者以闻。简子召之,曰:“譆,吾有所见子晣也。”当道者曰:“屏左右,原有谒。”简子屏人。当道者曰:“主君之疾,臣在帝侧。”简子曰:“然,有之。子之见我,我何为?”当道者曰:“帝令主君射熊与罴,皆死。”简子曰:“是,且何也?”当道者曰:“晋国且有大难,主君首之。帝令主君灭二卿,夫熊与罴皆其祖也。”简子曰:“帝赐我二笥皆有副,何也?”当道者曰:“主君之子将克二国于翟,皆子姓也。”简子曰:“吾见兒在帝侧,帝属我一翟犬,曰‘及而子之长以赐之’。夫兒何谓以赐翟犬?”当道者曰:“兒,主君之子也。翟犬者,代之先也。主君之子且必有代。及主君之后嗣,且有革政而胡服,并二国于翟。”简子问其姓而延之以官。当道者曰:“臣野人,致帝命耳。”遂不见。简子书藏之府。

异日,姑布子卿见简子,简子遍召诸子相之。子卿曰:“无为将军者。”简子曰:“赵氏其灭乎?”子卿曰:“吾尝见一子于路,殆君之子也。”简子召子毋恤。毋恤至,则子卿起曰:“此真将军矣!”简子曰:“此其母贱,翟婢也,奚道贵哉?”子卿曰:“天所授,虽贱必贵。”自是之后,简子尽召诸子与语,毋恤最贤。简子乃告诸子曰:“吾藏宝符于常山上,先得者赏。”诸子驰之常山上,求,无所得。毋恤还,曰:“已得符矣。”简子曰:“奏之。”毋恤曰:“从常山上临代,代可取也。”简子于是知毋恤果贤,乃废太子伯鲁,而以毋恤为太子。

后二年,晋定公之十四年,范、中行作乱。明年春,简子谓邯郸大夫午曰:“归我卫士五百家,吾将置之晋阳。”午许诺,归而其父兄不听,倍言。赵鞅捕午,囚之晋阳。乃告邯郸人曰:“我私有诛午也,诸君欲谁立?”遂杀午。赵稷、涉宾以邯郸反。晋君使籍秦围邯郸。荀寅、范吉射索隐范氏,晋大夫隰叔之子,士蔿之后。蔿生成伯缺,缺生武子会,会生文叔燮,燮生宣叔匄,匄生献子鞅,鞅生吉射。与午善,不肯助秦而谋作乱,董安于知之。十月,范、中行氏伐赵鞅,鞅奔晋阳,晋人围之。范吉射、荀寅仇人魏襄等谋逐荀寅,以梁婴父代之;”逐吉射,以范皋绎代之。荀栎”言于晋侯曰:“君命大臣,始乱者死。今三臣始乱而独逐鞅,用刑不均,请皆逐之。”十一月,荀栎、韩不佞、魏哆奉公命以伐范、中行氏,不克。范、中行氏反伐公,公击之,范、中行败走。丁未,二子奔朝歌。韩、魏以赵氏为请。十二月辛未,赵鞅入绛,盟于公宫。其明年,知伯文子谓赵鞅曰:“范、中行虽信为乱,安于发之,是安于与谋也。晋国有法,始乱者死。夫二子已伏罪而安于独在。”赵鞅患之。安于曰:“臣死,赵氏定,晋国宁,吾死晚矣。”遂自杀。赵氏以告知伯,然后赵氏宁。

孔子闻赵简子不请晋君而执邯郸午,保晋阳,故书春秋曰“赵鞅以晋阳畔”。

赵简子有臣曰周舍,好直谏。周舍死,简子每听朝,常不悦,大夫请罪。简子曰:“大夫无罪。吾闻千羊之皮不如一狐之腋。诸大夫朝,徒闻唯唯,不闻周舍之鄂鄂,是以忧也。”简子由此能附赵邑而怀晋人。

晋定公十八年,赵简子围范、中行于朝歌,中行文子奔邯郸。明年,卫灵公卒。简子与阳虎送卫太子蒯聩于卫,卫不内,居戚。

晋定公二十一年,简子拔邯郸,中行文子奔柏人。简子又围柏人,中行文子、范昭子遂奔齐。赵竟有邯郸、柏人。范、中行馀邑入于晋。赵名晋卿,实专晋权,奉邑侔于诸侯。

晋定公三十年,定公与吴王夫差争长于黄池,赵简子从晋定公,卒长吴。定公三十七年卒,而简子除三年之丧,期而已。是岁,越王句践灭吴。

晋出公十一年,知伯伐郑。赵简子疾,使太子毋恤将而围郑。知伯醉,以酒灌击毋恤。毋恤群臣请死之。毋恤曰:“君所以置毋恤,为能忍卼。”然亦愠知伯。知伯归,因谓简子,使废毋恤,简子不听。毋恤由此怨知伯。

晋出公十七年,简子卒,太子毋恤代立,是为襄子。

赵襄子元年,越围吴。襄子降丧食,使楚隆问吴王。

襄子姊前为代王夫人。简子既葬,未除服,北登夏屋,请代王。使厨人操铜枓以食代王及从者,行斟,阴令宰人各以枓击杀代王及从官,遂兴兵平代地。其姊闻之,泣而呼天,摩笄自杀。代人怜之,所死地名之为摩笄之山。遂以代封伯鲁子周为代成君。伯鲁者,襄子兄,故太子。太子蚤死,故封其子。

襄子立四年,知伯与赵、韩、魏尽分其范、中行故地。晋出公怒,告齐、鲁,欲以伐四卿。四卿恐,遂共攻出公。出公奔齐,道死。知伯乃立昭公曾孙骄,是为晋懿公。知伯益骄。请地韩、魏,韩、魏与之。请地赵,赵不与,以其围郑之辱。知伯怒,遂率韩、魏攻赵。赵襄子惧,乃奔保晋阳。

原过从,后,至于王泽,见三人,自带以上可见,自带以下不可见。与原过竹二节,莫通。曰:“为我以是遗赵毋恤。”原过既至,以告襄子。襄子齐三日,亲自剖竹,有硃书曰:“赵毋恤,余霍泰山山阳侯天使也。三月丙戌,余将使女反灭知氏。女亦立我百邑,余将赐女林胡之地。至于后世,且有伉王,赤黑,龙面而鸟噣,鬓麋髭皞,大膺大胸,脩下而冯,左衽界乘,奄有河宗,至于休溷诸貉,南伐晋别,北灭黑姑。”襄子再拜,受三神之令。

三国攻晋阳,岁馀,引汾水灌其城,城不浸者三版。城中悬釜而炊,易子而食。群臣皆有外心,礼益慢,唯高共不敢失礼。襄子惧,乃夜使相张孟同私于韩、魏。韩、魏与合谋,以三月丙戌,三国反灭知氏,共分其地。于是襄子行赏,高共为上。张孟同曰:“晋阳之难,唯共无功。”襄子曰:“方晋阳急,群臣皆懈,惟共不敢失人臣礼,是以先之。”于是赵北有代,南并知氏,彊于韩、魏。遂祠三神于百邑,使原过主霍泰山祠祀。

其后娶空同氏,生五子。襄子为伯鲁之不立也,不肯立子,且必欲传位与伯鲁子代成君。成君先死,乃取代成君子浣立为太子。襄子立三十三年卒,浣立,是为献侯。

献侯少即位,治中牟。

襄子弟桓子逐献侯,自立于代,一年卒。国人曰桓子立非襄子意,乃共杀其子而复迎立献侯。

十年,中山武公初立。十三年,城平邑。十五年,献侯卒,子烈侯籍立。

烈侯元年,魏文侯伐中山,使太子击守之。六年,魏、韩、赵皆相立为诸侯,追尊献子为献侯。

烈侯好音,谓相国公仲连曰:“寡人有爱,可以贵之乎?”公仲曰:“富之可,贵之则否。”烈侯曰:“然。夫郑歌者枪、石二人,吾赐之田,人万亩。”公仲曰:“诺。”不与。居一月,烈侯从代来,问歌者田。公仲曰:“求,未有可者。”有顷,烈侯复问。公仲终不与,乃称疾不朝。番吾君自代来,谓公仲曰:“君实好善,而未知所持。今公仲相赵,于今四年,亦有进士乎?”公仲曰:“未也。”番吾君曰:“牛畜、荀欣、徐越皆可。”公仲乃进三人。及朝,烈侯复问:“歌者田何如?”公仲曰:“方使择其善者。”牛畜侍烈侯以仁义,约以王道,烈侯逌然。明日,荀欣侍,以选练举贤,任官使能。明日,徐越侍,以节财俭用,察度功德。所与无不充,君说。烈侯使使谓相国曰:“歌者之田且止。”官牛畜为师,荀欣为中尉,徐越为内史,赐相国衣二袭。

九年,烈侯卒,弟武公立。武公十三年卒,赵复立烈侯太子章,是为敬侯。是岁,魏文侯卒。

敬侯元年,武公子朝作乱,不克,出奔魏。赵始都邯郸。

二年,败齐于灵丘。三年,救魏于廪丘,大败齐人。四年,魏败我兔台。筑刚平以侵卫。五年,齐、魏为卫攻赵,取我刚平。六年,借兵于楚伐魏,取棘蒲。八年,拔魏黄城。九年,伐齐。齐伐燕,赵救燕。十年,与中山战于房子。

十一年,魏、韩、赵共灭晋,分其地。伐中山,又战于中人。十二年,敬侯卒,子成侯种立。

成侯元年,公子胜与成侯争立,为乱。二年六月,雨雪。三年,太戊午为相。伐卫,取乡邑七十三。魏败我蔺。四年,与秦战高安,败之。五年,伐齐于鄄。魏败我怀。攻郑,败之,以与韩,韩与我长子。六年,中山筑长城。伐魏,败狝泽,围魏惠王。七年,侵齐,至长城。与韩攻周。八年,与韩分周以为两。九年,与齐战阿下。十年,攻卫,取甄。十一年,秦攻魏,赵救之石阿。十二年,秦攻魏少梁,赵救之。十三年,秦献公使庶长国伐魏少梁,虏其太子、痤。魏败我澮,取皮牢。成侯与韩昭侯遇上党。十四年,与韩攻秦。十五年,助魏攻齐。

十六年,与韩、魏分晋,封晋君以端氏。

十七年,成侯与魏惠王遇葛孽。十九年,与齐、宋会平陆,与燕会阿。二十年,魏献荣椽,因以为檀台。二十一年,魏围我邯郸。二十二年,魏惠王拔我邯郸,齐亦败魏于桂陵。二十四年,魏归我邯郸,与魏盟漳水上。秦攻我蔺。二十五年,成侯卒。公子緤与太子肃侯争立,緤败,亡奔韩。

肃侯元年,夺晋君端氏,徙处屯留。二年,与魏惠王遇于阴晋。三年,公子范袭邯郸,不胜而死。四年,朝天子。六年,攻齐,拔高唐。七年,公子刻攻魏首垣。十一年,秦孝公使商君伐魏,虏其将公子卬。赵伐魏。十二年,秦孝公卒,商君死。十五年,起寿陵。魏惠王卒。

十六年,肃侯游大陵,出于鹿门,大戊午扣马曰:“耕事方急,一日不作,百日不食。”肃侯下车谢。

十七年,围魏黄,不克。筑长城。

十八年,齐、魏伐我,我决河水灌之,兵去。二十二年,张仪相秦。赵疵与秦战,败,秦杀疵河西,取我蔺、离石。二十三年,韩举与齐、魏战,死于桑丘。

二十四年,肃侯卒。秦、楚、燕、齐、魏出锐师各万人来会葬。子武灵王立。

武灵王元年,阳文君赵豹相。梁襄王与太子嗣,韩宣王与太子仓来朝信宫。武灵王少,未能听政,博闻师三人,左右司过三人。及听政,先问先王贵臣肥义,加其秩;国三老年八十,月致其礼。

三年,城鄗。四年,与韩会于区鼠。五年,娶韩女为夫人。

八年,韩击秦,不胜而去。五国相王,赵独否,曰:“无其实,敢处其名乎!”令国人谓已曰“君”。

九年,与韩、魏共击秦,秦败我,斩首八万级。齐败我观泽。十年,秦取我中都及西阳。齐破燕。燕相子之为君,君反为臣。十一年,王召公子职于韩,立以为燕王,”使乐池送之。十三年,秦拔我蔺,虏将军赵庄。楚、魏王来,过邯郸。十四年,赵何攻魏。

十六年,秦惠王卒。王游大陵。他日,王梦见处女鼓琴而歌诗曰:“美人荧荧兮,颜若苕之荣。命乎命乎,曾无我嬴!”异日,王饮酒乐,数言所梦,想见其状。吴广闻之,因夫人而内其女娃嬴。孟姚也。孟姚甚有宠于王,是为惠后。

十七年,王出九门,为野台,以望齐、中山之境。

十八年,秦武王与孟说举龙文赤鼎,绝膑而死。赵王使代相赵固迎公子稷于燕,送归,立为秦王,是为昭王。

十九年春正月,大朝信宫。召肥义与议天下,五日而毕。王北略中山之地,至于房子,遂之代,北至无穷,西至河,登黄华之上。召楼缓谋曰:“我先王因世之变,以长南籓之地,属阻漳、滏之险,立长城,又取蔺、郭狼,败林人于荏,而功未遂。今中山在我腹心,北有燕,东有胡,西有林胡、楼烦、秦、韩之边,而无彊兵之救,是亡社稷,柰何?夫有高世之名,必有遗俗之累。吾欲胡服。”楼缓曰:“善。”群臣皆不欲。

于是肥义侍,王曰:“简、襄主之烈,计胡、翟之利。为人臣者,宠有孝弟长幼顺明之节,通有补民益主之业,此两者臣之分也。今吾欲继襄主之迹,开于胡、翟之乡,而卒世不见也。为敌弱,用力少而功多,可以毋尽百姓之劳,而序往古之勋。夫有高世之功者,负遗俗之累;有独智之虑者,任骜民之怨。今吾将胡服骑射以教百姓,而世必议寡人,柰何?”肥义曰:“臣闻疑事无功,疑行无名。王既定负遗俗之虑,殆无顾天下之议矣。夫论至德者不和于俗,成大功者不谋于众。昔者舜舞有苗,禹袒裸国,非以养欲而乐志也,务以论德而约功也。愚者闇成事,智者睹未形,则王何疑焉。”王曰:“吾不疑胡服也,吾恐天下笑我也。狂夫之乐,智者哀焉;愚者所笑,贤者察焉。世有顺我者,胡服之功未可知也。虽驱世以笑我,胡地中山吾必有之。”于是遂胡服矣。

使王緤告公子成曰:“寡人胡服,将以朝也,亦欲叔服之。家听于亲而国听于君,古今之公行也。子不反亲,臣不逆君,兄弟之通义也。今寡人作教易服而叔不服,吾恐天下议之也。制国有常,利民为本;从政有经,令行为上。明德先论于贱,而行政先信于贵。今胡服之意,非以养欲而乐志也;事有所止而功有所出,事成功立,然后善也。今寡人恐叔之逆从政之经,以辅叔之议。且寡人闻之,事利国者行无邪,因贵戚者名不累,故原慕公叔之义,以成胡服之功。使緤谒之叔,请服焉。”公子成再拜稽首曰:“臣固闻王之胡服也。臣不佞,寝疾,未能趋走以滋进也。王命之,臣敢对,因竭其愚忠。曰:臣闻中国者,盖聪明徇智之所居也,万物财用之所聚也,贤圣之所教也,仁义之所施也,诗书礼乐之所用也,异敏技能之所试也,远方之所观赴也,蛮夷之所义行也。今王舍此而袭远方之服,变古之教,易古人道,逆人之心,而怫学者,离中国,故臣原王图之也。”使者以报。王曰:“吾固闻叔之疾也,我将自往请之。”

王遂往之公子成家,因自请之,曰:“夫服者,所以便用也;礼者,所以便事也。圣人观乡而顺宜,因事而制礼,所以利其民而厚其国也。夫翦发文身,错臂左衽,瓯越之民也。黑齿雕题,卻冠秫绌,大吴之国也。故礼服莫同,其便一也。乡异而用变,事异而礼易。是以圣人果可以利其国,不一其用;果可以便其事,不同其礼。儒者一师而俗异,中国同礼而教离,况于山谷之便乎?故去就之变,智者不能一;远近之服,贤圣不能同。穷乡多异,曲学多辩。不知而不疑,异于己而不非者,公焉而众求尽善也。今叔之所言者俗也,吾所言者所以制俗也。吾国东有河、薄洛之水,与齐、中山同之,东有燕、东胡之境,而西有楼烦、秦、韩之边,今无骑射之备。故寡人无舟楫之用,夹水居之民,将何以守河、薄洛之水;变服骑射,以备燕、三胡、秦、韩之边。且昔者简主不塞晋阳以及上党,而襄主并戎取代以攘诸胡,此愚智所明也。先时中山负齐之彊兵,侵暴吾地,系累吾民,引水围鄗,微社稷之神灵,则鄗几于不守也。先王丑之,而怨未能报也。今骑射之备,近可以便上党之形,而远可以报中山之怨。而叔顺中国之俗以逆简、襄之意,恶变服之名以忘鄗事之丑,非寡人之所望也。”公字成再拜稽首曰:“臣愚,不达于王之义,敢道世俗之闻,臣之罪也。今王将继简、襄之意以顺先王之志,臣敢不听命乎!”再拜稽首。乃赐胡服。明日,服而朝。于是始出胡服令也。

赵文、赵造、周袑、赵俊皆谏止王毋胡服,如故法便。王曰:“先王不同俗,何古之法?帝王不相袭,何礼之循?虙戏、神农教而不诛,黄帝、尧、舜诛而不怒。及至三王,随时制法,因事制礼。法度制令各顺其宜,衣服器械各便其用。故礼也不必一道,而便国不必古。圣人之兴也不相袭而王,夏、殷之衰也不易礼而灭。然则反古未可非,而循礼未足多也。且服奇者志淫,则是邹、鲁无奇行也;俗辟者民易,则是吴、越无秀士也。且圣人利身谓之服,便事谓之礼。夫进退之节,衣服之制者,所以齐常民也,非所以论贤者也。故齐民与俗流,贤者与变俱。故谚曰‘以书御者不尽马之情,以古制今者不达事之变’。循法之功,不足以高世;法古之学,不足以制今。子不及也。”遂胡服招骑射。

二十年,王略中山地,至宁葭;西略胡地,至榆中。林胡王献马。归,使楼缓之秦,仇液之韩,王贲之楚,富丁之魏,赵爵之齐。代相赵固主胡,致其兵。

二十一年,攻中山。赵袑为右军,许钧为左军,公子章为中军,王并将之。牛翦将车骑,赵希并将胡、代。赵与之陉,合军曲阳,攻取丹丘、华阳、鸱之塞。王军取鄗、石邑、封龙、东垣。中山献四邑和,王许之,罢兵。二十三年,攻中山。二十五年,惠后卒。使周袑胡服傅王子何。二十六年,复攻中山,攘地北至燕、代,西至云中、九原。

二十七年五月戊申,大朝于东宫,传国,立王子何以为王。王庙见礼毕,出临朝。大夫悉为臣,肥义为相国,并傅王。是为惠文王。惠文王,惠后吴娃子也。武灵王自号为主父。

主父欲令子主治国,而身胡服将士大夫西北略胡地,而欲从云中、九原直南袭秦,于是诈自为使者入秦。秦昭王不知,已而怪其状甚伟,非人臣之度,使人逐之,而主父驰已脱关矣。审问之,乃主父也。秦人大惊。主父所以入秦者,欲自略地形,因观秦王之为人也。

惠文王二年,主父行新地,遂出代,西遇楼烦王于西河而致其兵。

三年,灭中山,迁其王于肤施。起灵寿,北地方从,代道大通。还归,行赏,大赦,置酒酺五日,封长子章为代安阳君。章素侈,心不服其弟所立。主父又使田不礼相章也。

李兑谓肥义曰:“公子章彊壮而志骄,党众而欲大,殆有私乎?田不礼之为人也,忍杀而骄。二人相得,必有谋阴贼起,一出身徼幸。夫小人有欲,轻虑浅谋,徒见其利而不顾其害,同类相推,俱入祸门。以吾观之,必不久矣。子任重而势大,乱之所始,祸之所集也,子必先患。仁者爱万物而智者备祸于未形,不仁不智,何以为国?子奚不称疾毋出,传政于公子成?毋为怨府,毋为祸梯。”肥义曰:“不可,昔者主父以王属义也,曰:‘毋变而度,毋异而虑,坚守一心,以殁而世。’义再拜受命而籍之。今畏不礼之难而忘吾籍,变孰大焉。进受严命,退而不全,负孰甚焉。变负之臣,不容于刑。谚曰‘死者复生,生者不愧’。吾言已在前矣,吾欲全吾言,安得全吾身!且夫贞臣也难至而节见,忠臣也累至而行明。子则有赐而忠我矣,虽然,吾有语在前者也,终不敢失。”李兑曰:“诺,子勉之矣!吾见子已今年耳。”涕泣而出。李兑数见公子成,以备田不礼之事。

异日肥义谓信期曰:“公子与田不礼甚可忧也。其于义也声善而实恶,此为人也不子不臣。吾闻之也,奸臣在朝,国之残也;谗臣在中,主之蠹也。此人贪而欲大,内得主而外为暴。矫令为慢,以擅一旦之命,不难为也,祸且逮国。今吾忧之,夜而忘寐,饥而忘食。盗贼出入不可不备。自今以来,若有召王者必见吾面,我将先以身当之,无故而王乃入。”信期曰:“善哉,吾得闻此也!”

四年,朝群臣,安阳君亦来朝。主父令王听朝,而自从旁观窥群臣宗室之礼。见其长子章劚然也,反北面为臣,诎于其弟,心怜之,于是乃欲分赵而王章于代,计未决而辍。

主父及王游沙丘,异宫,公子章即以其徒与田不礼作乱,诈以主父令召王。肥义先入,杀之。高信即与王战。公子成与李兑自国至,乃起四邑之兵入距难,杀公子章及田不礼,灭其党贼而定王室。公子成为相,号安平君,李兑为司寇。公子章之败,往走主父,主主开之,成、兑因围主父宫。公子章死,公子成、李兑谋曰:“以章故围主父,即解兵,吾属夷矣。”乃遂围主父。令宫中人“后出者夷”,宫中人悉出。主父欲出不得,又不得食,探爵鷇而食之,三月馀而饿死沙丘宫。主父定死,乃发丧赴诸侯。

是时王少,成、兑专政,畏诛,故围主父。主父初以长子章为太子,后得吴娃,爱之,为不出者数岁,生子何,乃废太子章而立何为王。吴娃死,爱弛,怜故太子,欲两王之,犹豫未决,故乱起,以至父子俱死,为天下笑,岂不痛乎!

五年,与燕鄚、易。八年,城南行唐。九年,赵梁将,与齐合军攻韩,至鲁关下。及十年,秦自置为西帝。十一年,董叔与魏氏伐宋,得河阳于魏。秦取梗阳。十二年,赵梁将攻齐。十三年,韩徐为将,攻齐。公主死。十四年,相国乐毅将赵、秦、韩、魏、燕攻齐,取灵丘。与秦会中阳。十五年,燕昭王来见。赵与韩、魏、秦共击齐,齐王败走,燕独深入,取临菑。

十六年,秦复与赵数击齐,齐人患之。苏厉为齐遗赵王书曰:

臣闻古之贤君,其德行非布于海内也,教顺非洽于民人也,祭祀时享非数常于鬼神也。甘露降,时雨至,年穀丰孰,民不疾疫,众人善之,然而贤主图之。

今足下之贤行功力,非数加于秦也;怨毒积怒,非素深于齐也。秦赵与国,以彊徵兵于韩,秦诚爱赵乎?其实憎齐乎?物之甚者,贤主察之。秦非爱赵而憎齐也,欲亡韩而吞二周,故以齐餤天下。恐事之不合,故出兵以劫魏、赵。恐天下畏己也,故出质以为信。恐天下亟反也,故徵兵于韩以威之。声以德与国,实而伐空韩,臣以秦计为必出于此。夫物固有势异而患同者,楚久伐而中山亡,今齐久伐而韩必亡。破齐,王与六国分其利也。亡韩,秦独擅之。收二周,西取祭器,秦独私之。赋田计功,王之获利孰与秦多?

说士之计曰:“韩亡三川,魏亡晋国,市朝未变而祸已及矣。”燕尽齐之北地,去沙丘、钜鹿敛三百里,韩之上党去邯郸百里,燕、秦谋王之河山,间三百里而通矣。秦之上郡近挺关,至于榆中者千五百里,秦以三郡攻王之上党,羊肠之西,句注之南,非王有已。逾句注,斩常山而守之,三百里而通于燕,代马胡犬不东下,昆山之玉不出,此三宝者亦非王有已。王久伐齐,从彊秦攻韩,其祸必至于此。原王孰虑之。

且齐之所以伐者,以事王也;天下属行,以谋王也。燕秦之约成而兵出有日矣。五国三分王之地,齐倍五国之约而殉王之患,西兵以禁彊秦,秦废帝请服,反巠分、先俞于赵。齐之事王,宜为上佼,而今乃抵罪,臣恐天下后事王者之不敢自必也。原王孰计之也。

今王毋与天下攻齐,天下必以王为义。齐抱社稷而厚事王,天下必尽重王义。王以天下善秦,秦暴,王以天下禁之,是一世之名宠制于王也。于是赵乃辍,谢秦不击齐。

王与燕王遇。廉颇将,攻齐昔阳,取之。

十七年,乐毅将赵师攻魏伯阳。而秦怨赵不与己击齐,伐赵,拔我两城。十八年,秦拔我石城。王再之卫东阳,决河水,伐魏氏。大潦,漳水出。魏厓来相赵。十九年,秦取我二城。赵与魏伯阳。赵奢将,攻齐麦丘,取之。

二十年,廉颇将,攻齐。王与秦昭王遇西河外。

二十一年,赵徙漳水武平西。二十二年,大疫。置公子丹为太子。

二十三年,楼昌将,攻魏几,不能取。十二月,廉颇将,攻几,取之。二十四年,廉颇将,攻魏房子,拔之,因城而还。又攻安阳,取之。二十五年,燕周将,攻昌城、高唐,取之。与魏共击秦。秦将白起破我华阳,得一将军。二十六年,取东胡欧代地。

二十七年,徙漳水武平南。封赵豹为平阳君。河水出,大潦。

二十八年,蔺相如伐齐,至平邑。罢城北九门大城。燕将成安君公孙操弑其王。二十九年,秦、韩相攻,而围阏与。赵使赵奢将,击秦,大破秦军阏与下,赐号为马服君。

三十三年,惠文王卒,太子丹立,是为孝成王。

孝成王元年,秦伐我,拔三城。赵王新立,太后用事,秦急攻之。赵氏求救于齐,齐曰:“必以长安君为质,兵乃出。”太后不肯,大臣彊谏。太后明谓左右曰:“复言长安君为质者,老妇必唾其面。”左师触龙言原见太后,太后盛气而胥之。入,徐趋而坐,自谢曰:“老臣病足,曾不能疾走,不得见久矣。窃自恕,而恐太后体之有所苦也,故原望见太后。”太后曰:“老妇恃辇而行耳。”曰:“食得毋衰乎?”曰:“恃粥耳。”曰:“老臣间者殊不欲食,乃彊步,日三四里,少益嗜食,和于身也。”太后曰:“老妇不能。”太后不和之色少解。左师公曰:“老臣贱息舒祺最少,不肖,而臣衰,窃怜爱之,原得补黑衣之缺以卫王宫,昧死以闻。”太后曰:“敬诺。年几何矣?”对曰:“十五岁矣。虽少,原及未填沟壑而讬之。”太后曰:“丈夫亦爱怜少子乎?”对曰:“甚于妇人。”太后笑曰:“妇人异甚。”对曰:“老臣窃以为媪之爱燕后贤于长安君。”太后曰:“君过矣,不若长安君之甚。”左师公曰:“父母爱子则为之计深远。媪之送燕后也,持其踵,为之泣,念其远也,亦哀之矣。已行,非不思也,祭祀则祝之曰‘必勿使反’,岂非计长久,为子孙相继为王也哉?”太后曰:“然。”左师公曰:“今三世以前,至于赵主之子孙为侯者,其继有在者乎?”曰:“无有。”曰:“微独赵,诸侯有在者乎?”曰:“老妇不闻也。”曰:“此其近者祸及其身,远者及其子孙。岂人主之子侯则不善哉?位尊而无功,奉厚而无劳,而挟重器多也。今媪尊长安君之位,而封之以膏腴之地,多与之重器,而不及今令有功于国,一旦山陵崩,长安君何以自讬于赵?老臣以媪为长安君之计短也,故以为爱之不若燕后。”太后曰:“诺,恣君之所使之。”于是为长安君约车百乘,质于齐,齐兵乃出。

子义闻之,曰:“人主之子,骨肉之亲也,犹不能持无功之尊,无劳之奉,而守金玉之重也,而况于予乎?”

齐安平君田单将赵师而攻燕中阳,拔之。又攻韩注人,拔之。二年,惠文后卒。田单为相。

四年,王梦衣偏裻之衣,乘飞龙上天,不至而坠,见金玉之积如山。明日,王召筮史敢占之,曰:“梦衣偏裻之衣者,残也。乘飞龙上天不至而坠者,有气而无实也。见金玉之积如山者,忧也。”

后三日,韩氏上党守冯亭使者至,曰:“韩不能守上党,入之于秦。其吏民皆安为赵,不欲为秦。有城市邑十七,原再拜入之赵,财王所以赐吏民。”王大喜,召平阳君豹告之曰:“冯亭入城市邑十七,受之何如?”对曰:“圣人甚祸无故之利。”王曰:“人怀吾德,何谓无故乎?”对曰:“夫秦蚕食韩氏地,中绝不令相通,固自以为坐而受上党之地也。韩氏所以不入于秦者,欲嫁其祸于赵也。秦服其劳而赵受其利,虽彊大不能得之于小弱,小弱顾能得之于彊大乎?岂可谓非无故之利哉!且夫秦以牛田之水通粮蚕食,上乘倍战者,裂上国之地,其政行,不可与为难,必勿受也。”王曰:“今发百万之军而攻,逾年历岁未得一城也。今以城市邑十七币吾国,

赵豹出,王召平原君与赵禹而告之。对曰:“发百万之军而攻,逾岁未得一城,今坐受城市邑十七,此大利,不可失也。”王曰:“善。”乃令赵胜受地,告冯亭曰:“敝国使者臣胜,敝国君使胜致命,以万户都三封太守,千户都三封县令,皆世世为侯,吏民皆益爵三级,吏民能相安,皆赐之六金。”冯亭垂涕不见使者,曰:“吾不处三不义也:为主守地,不能死固,不义一矣;入之秦,不听主令,不义二矣;卖主地而食之,不义三矣。”赵遂发兵取上党。廉颇将军军长平。

七月,廉颇免而赵括代将。秦人围赵括,赵括以军降,卒四十馀万皆阬之。王悔不听赵豹之计,故有长平之祸焉。

王还,不听秦,秦围邯郸。武垣令傅豹、王容、苏射率燕众反燕地。赵以灵丘封楚相春申君。

八年,平原君如楚请救。还,楚来救,及魏公子无忌亦来救,秦围邯郸乃解。

十年,燕攻昌壮,五月拔之。赵将乐乘、庆舍攻秦信梁军,破之。太子死。而秦攻西周,拔之。徒父祺出。十一年,城元氏,县上原。武阳君郑安平死,收其地。十二年,邯郸廥烧。十四年,平原君赵胜死。

十五年,以尉文封相国廉颇为信平君。燕王令丞相栗腹约驩,以五百金为赵王酒,还归,报燕王曰:“赵氏壮者皆死长平,其孤未壮,可伐也。”王召昌国君乐间而问之。对曰:“赵,四战之国也,其民习兵,伐之不可。”王曰:“吾以众伐寡,二而伐一,可乎?”对曰:“不可。”王曰:“吾即以五而伐一,可乎?”对曰:“不可。”燕王大怒。群臣皆以为可。燕卒起二军,车二千乘,栗腹将而攻鄗,卿秦将而攻代。廉颇为赵将,破杀栗腹,虏卿秦、乐间。

十六年,廉颇围燕。以乐乘为武襄君。率师从相国信平君助魏攻燕。秦拔我榆次三十七城。十九年,赵与燕易土:以龙兑、汾门、临乐与燕;燕以葛、武阳、平舒与赵。

二十年,秦王政初立。秦拔我晋阳。

二十一年,孝成王卒。廉颇将,攻繁阳,取之。使乐乘代之,廉颇攻乐乘,乐乘走,廉颇亡入魏。子偃立,是为悼襄王。

悼襄王元年,大备魏。欲通平邑、中牟之道,不成。

二年,李牧将,攻燕,拔武遂、方城。秦召春平君,因而留之。泄钧为之谓文信侯曰:“春平君者,赵王甚爱之而郎中妒之,故相与谋曰‘春平君入秦,秦必留之’,故相与谋而内之秦也。今君留之,是绝赵而郎中之计中也。君不如遣春平君而留平都。春平君者言行信于王,王必厚割赵而赎平都。”文信侯曰:“善。”因遣之。城韩皋。

三年,庞暖将,攻燕,禽其将剧辛。四年,庞暖将赵、楚、魏、燕之锐师,攻秦蕞,不拔;移攻齐,取饶安。五年,傅抵将,居平邑;庆舍将东阳河外师,守河梁。六年,封长安君以饶。魏与赵鄴。

九年,赵攻燕,取貍阳城。兵未罢,秦攻鄴,拔之。悼襄王卒,子幽缪王迁立。

幽缪王迁元年,城柏人。二年,秦攻武城,扈辄率师救之,军败,死焉。

三年,秦攻赤丽、宜安,李牧率师与战肥下,卻之。封牧为武安君。四年,秦攻番吾,李牧与之战,卻之。

五年,代地大动,自乐徐以西,”北至平阴,台屋墙垣太半坏,地坼东西百三十步。六年,大饥,民讹言曰:“赵为号,秦为笑。以为不信,视地之生毛。”

七年,秦人攻赵,赵大将李牧、将军司马尚将,击之。李牧诛,司马尚免,赵怱及齐将颜聚代之。赵怱军破,颜聚亡去。以王迁降。

八年十月,邯郸为秦。

太史公曰。吾闻冯王孙曰:“赵王迁,其母倡也,嬖于悼襄王。悼襄王废適子嘉而立迁。迁素无行,信谗,故诛其良将李牧,用郭开。”岂不缪哉!秦既虏迁,赵之亡大夫共立嘉为王,王代六岁,秦进兵破嘉,遂灭赵以为郡。

赵氏之系,与秦同祖。周穆平徐,乃封造父。带始事晋,夙初有土。岸贾矫诛,韩厥立武。宝符临代,卒居伯鲁。简梦翟犬,灵歌处女。胡服虽强,建立非所。颇、牧不用,王迁囚虏。
\end{yuanwen}

\chapter{魏世家}

\begin{yuanwen}
魏之先,毕公高之后也。毕公高与周同姓。武王之伐纣,而高封于毕,于是为毕姓。其后绝封,为庶人,或在中国,或在夷狄。其苗裔曰毕万,事晋献公。

献公之十六年,赵夙为御,毕万为右,以伐霍、耿、魏,灭之。以耿封赵夙,以魏封毕万,为大夫。卜偃曰:“毕万之后必大矣,万,满数也;魏,大名也。以是始赏,天开之矣,天子曰兆民,诸侯曰万民。今命之大,以从满数,其必有众。”初,毕万卜事晋,遇屯之比。辛廖占之,曰:“吉。屯固比入,吉孰大焉,其必蕃昌。”

毕万封十一年,晋献公卒,四子争更立,晋乱。而毕万之世弥大,从其国名为魏氏。生武子。魏武子以魏诸子事晋公子重耳。晋献公之二十一年,武子从重耳出亡。十九年反,重耳立为晋文公,而令魏武子袭魏氏之后封,列为大夫,治于魏。生悼子。

魏悼子徙治霍。生魏绛。

魏绛事晋悼公。悼公三年,会诸侯。悼公弟杨干乱行,魏绛僇辱杨干。悼公怒曰:“合诸侯以为荣,今辱吾弟!”将诛魏绛。或说悼公,悼公止。卒任魏绛政,使和戎、翟,戎、翟亲附。悼公之十一年,曰:“自吾用魏绛,八年之中,九合诸侯,戎、翟和,子之力也。”赐之乐,三让,然后受之。徙治安邑。魏绛卒,谥为昭子。生魏嬴。嬴生魏献子。

献子事晋昭公。昭公卒而六卿彊,公室卑。

晋顷公之十二年,韩宣子老,魏献子为国政。晋宗室祁氏、羊舌氏相恶,六卿诛之,尽取其邑为十县,六卿各令其子为之大夫。献子与赵简子、中行文子、范献子并为晋卿。

其后十四岁而孔子相鲁。后四岁,赵简子以晋阳之乱也,而与韩、魏共攻范、中行氏。魏献子生魏侈。魏侈与赵鞅共攻范、中行氏。

魏侈之孙曰魏桓子,与韩康子、赵襄子共伐灭知伯,分其地。

桓子之孙曰文侯都。魏文侯元年,秦灵公之元年也。与韩武子、赵桓子、周威王同时。

六年,城少梁。十三年,使子击围繁、庞,出其民。十六年,伐秦,筑临晋元里。

十七年,伐中山,使子击守之,赵仓唐傅之。子击逢文侯之师田子方于朝歌,引车避,下谒。田子方不为礼。子击因问曰:“富贵者骄人乎?且贫贱者骄人乎?”子方曰:“亦贫贱者骄人耳。夫诸侯而骄人则失其国,大夫而骄人则失其家。贫贱者,行不合,言不用,则去之楚、越,若脱鵕然,柰何其同之哉!”子击不怿而去。西攻秦,至郑而还,筑雒阴、合阳。

二十二年,魏、赵、韩列为诸侯。

二十四年,秦伐我,至阳狐。

二十五年,子击生子。

文侯受子夏经艺,客段干木,过其闾,未尝不轼也。秦尝欲伐魏,或曰:“魏君贤人是礼,国人称仁,上下和合,未可图也。”文侯由此得誉于诸侯。

任西门豹守鄴,而河内称治。

魏文侯谓李克曰:“先生尝教寡人曰‘家贫则思良妻,国乱则思良相’。今所置非成则璜,二子何如?”李克对曰:“臣闻之,卑不谋尊,疏不谋戚。臣在阙门之外,不敢当命。”文侯曰:“先生临事勿让。”李克曰:“君不察故也。居视其所亲,富视其所与,达视其所举,穷视其所不为,贫视其所不取,五者足以定之矣,何待克哉!”文侯曰:“先生就舍,寡人之相定矣。”李克趋而出,过翟璜之家。翟璜曰:“今者闻君召先生而卜相,果谁为之?”李克曰:“魏成子为相矣。”翟璜忿然作色曰:“以耳目之所睹记,臣何负于魏成子?西河之守,臣之所进也。君内以鄴为忧,臣进西门豹。君谋欲伐中山,臣进乐羊。中山以拔,无使守之,臣进先生。君之子无傅,臣进屈侯鲋。臣何以负于魏成子!”李克曰:“且子之言克于子之君者,岂将比周以求大官哉?君问而置相‘非成则璜,二子何如’?克对曰:‘君不察故也。居视其所亲,富视其所与,达视其所举,穷视其所不为,贫视其所不取,五者足以定之矣,何待克哉!’是以知魏成子之为相也。且子安得与魏成子比乎?魏成子以食禄千锺,什九在外,什一在内,是以东得卜子夏、田子方、段干木。此三人者,君皆师之。子之所进五人者,君皆臣之。子恶得与魏成子比也?”翟璜逡巡再拜曰:“璜,鄙人也,失对,原卒为弟子。”

二十六年,虢山崩,壅河。

三十二年,伐郑。城酸枣。败秦于注。三十五年,齐伐取我襄陵。三十六年,秦侵我阴晋。

三十八年,伐秦,败我武下,得其将识。是岁,文侯卒,子击立,是为武侯。

魏武侯元年,赵敬侯初立,公子朔为乱,不胜,奔魏,与魏袭邯郸,魏败而去。

二年,城安邑、王垣。

七年,伐齐,至桑丘。九年,翟败我于澮。使吴起伐齐,至灵丘。齐威王初立。

十一年,与韩、赵三分晋地,灭其后。

十三年,秦献公县栎阳。十五年,败赵北蔺。

十六年,伐楚,取鲁阳。武侯卒,子立,是为惠王。

惠王元年,初,武侯卒也,子与公中缓争为太子。公孙颀自宋入赵,自赵入韩,谓韩懿侯曰:“魏与公中缓争为太子,君亦闻之乎?今魏得王错,挟上党,固半国也。因而除之,破魏必矣,不可失也。”懿侯说,乃与赵成侯合军并兵以伐魏,战于浊泽,魏氏大败,魏君围。赵谓韩曰:“除魏君,立公中缓,割地而退,我且利。”韩曰:“不可。杀魏君,人必曰暴;割地而退,人必曰贪。不如两分之。魏分为两,不彊于宋、卫,则我终无魏之患矣。”赵不听。韩不说,以其少卒夜去。惠王之所以身不死,国不分者,二家谋不和也。若从一家之谋,则魏必分矣。故曰“君终无適子,其国可破也”。

二年,魏败韩于马陵,败赵于怀。三年,齐败我观。五年,与韩会宅阳。城武堵。为秦所败。六年,伐取宋仪台。九年,伐败韩于澮。与秦战少梁,虏我将公孙痤,取庞。秦献公卒,子孝公立。

十年,伐取赵皮牢。彗星见。十二年,星昼坠,有声。

十四年,与赵会鄗。十五年,鲁、卫、宋、郑君来朝。十六年,与秦孝公会杜平。侵宋黄池,宋复取之。

十七年,与秦战元里,秦取我少梁。围赵邯郸。十八年,拔邯郸。赵请救于齐,齐使田忌、孙膑救赵,败魏桂陵。

十九年,诸侯围我襄陵。筑长城,塞固阳。

二十年,归赵邯郸,与盟漳水上。二十一年,与秦会彤。赵成侯卒。二十八年,齐威王卒。中山君相魏。

三十年,魏伐赵,赵告急齐。齐宣王用孙子计,救赵击魏。魏遂大兴师,使庞涓将,而令太子申为上将军。过外黄,外黄徐子谓太子曰:“臣有百战百胜之术。”太子曰:“可得闻乎?”客曰:“固原效之。”曰:“太子自将攻齐,大胜并莒,则富不过有魏,贵不益为王。若战不胜齐,则万世无魏矣。此臣之百战百胜之术也。”太子曰:“诺,请必从公之言而还矣。”客曰:“太子虽欲还,不得矣。彼劝太子战攻,欲啜汁者众。太子虽欲还,恐不得矣。”太子因欲还,其御曰:“将出而还,与北同。”太子果与齐人战,败于马陵。齐虏魏太子申,杀将军涓,军遂大破。

三十一年,秦、赵、齐共伐我,秦将商君诈我将军公子卬而袭夺其军,破之。秦用商君,东地至河,而齐、赵数破我,安邑近秦,于是徙治大梁。以公子赫为太子。

三十三年,秦孝公卒,商君亡秦归魏,魏怒,不入。三十五年,与齐宣王会平阿南。

惠王数被于军旅,卑礼厚币以招贤者。邹衍、淳于髡、孟轲皆至梁。梁惠王曰:“寡人不佞,兵三折于外,太子虏,上将死,国以空虚,以羞先君宗庙社稷,寡人甚丑之,叟不远千里,辱幸至弊邑之廷,将何利吾国?”孟轲曰:“君不可以言利若是。夫君欲利则大夫欲利,大夫欲利则庶人欲利,上下争利,国则危矣。为人君,仁义而已矣,何以利为!”

三十六年,复与齐王会甄。是岁,惠王卒,子襄王立。

襄王元年,与诸侯会徐州,相王也。追尊父惠王为王。

五年,秦败我龙贾军四万五千于雕阴,围我焦、曲沃。予秦河西之地。

六年,与秦会应。秦取我汾阴、皮氏、焦。魏伐楚,败之陉山。七年,魏尽入上郡于秦。秦降我蒲阳。八年,秦归我焦、曲沃。

十二年,楚败我襄陵。诸侯执政与秦相张仪会齧桑。十三年,张仪相魏。魏有女子化为丈夫。秦取我曲沃、平周。

十六年,襄王卒,子哀王立。张仪复归秦。

哀王元年,五国共攻秦,不胜而去。

二年,齐败我观津。五年,秦使樗里子伐取我曲沃,走犀首岸门。六年,秦来立公子政为太子。与秦会临晋。七年,攻齐。与秦伐燕。

八年,伐卫,拔列城二。见卫君曰:“请罢魏兵,免成陵君可乎?”卫君曰:“先生果能,孤请世世以卫事先生。”如耳见成陵君曰:“昔者魏伐赵,断羊肠,拔阏与,约斩赵,赵分而为二,所以不亡者,魏为从主也。今卫已迫亡,将西请事于秦。与其以秦醳卫,不如以魏醳卫,卫之德魏必终无穷。”成陵君曰:“诺。”如耳见魏王曰:“臣有谒于卫。卫故周室之别也,其称小国,多宝器。今国迫于难而宝器不出者,其心以为攻卫醳卫不以王为主,故宝器虽出必不入于王也。臣窃料之,先言醳卫者必受卫者也。”如耳出,成陵君入,以其言见魏王。魏王听其说,罢其兵,免成陵君,终身不见。

九年,与秦王会临晋。张仪、魏章皆归于魏。魏相田需死,楚害张仪、犀首、薛公。楚相昭鱼谓苏代曰:“田需死,吾恐张仪、犀首、薛公有一人相魏者也。”代曰:“然相者欲谁而君便之?”昭鱼曰:“吾欲太子之自相也。”代曰:“请为君北,必相之。”昭鱼曰:“柰何?”对曰:“君其为梁王,代请说君。”昭鱼曰:“柰何?”对曰:“代也从楚来,昭鱼甚忧,曰:‘田需死,吾恐张仪、犀首、薛公有一人相魏者也。’代曰:‘梁王,长主也,必不相张仪。张仪相,必右秦而左魏。犀首相,必右韩而左魏。薛公相,必右齐而左魏。梁王,长主也,必不便也。’王曰:‘然则寡人孰相?’代曰:‘莫若太子之自相。太子之自相,是三人者皆以太子为非常相也,皆将务以其国事魏,欲得丞相玺也。以魏之彊,而三万乘之国辅之,魏必安矣。故曰莫若太子之自相也。’”遂北见梁王,以此告之。太子果相魏。

十年,张仪死。十一年,与秦武王会应。十二年,太子朝于秦。秦来伐我皮氏,未拔而解。十四年,秦来归武王后。十六年,秦拔我蒲反、阳晋、封陵。十七年,与秦会临晋。秦予我蒲反。十八年,与秦伐楚。`二十一年,与齐、韩共败秦军函谷。

二十三年,秦复予我河外及封陵为和。哀王卒,子昭王立。

昭王元年,秦拔我襄城。二年,与秦战,我不利。三年,佐韩攻秦,秦将白起败我军伊阙二十四万。六年,予秦河东地方四百里。芒卯以诈重。七年,秦拔我城大小六十一。八年,秦昭王为西帝,齐湣王为东帝,月馀,皆复称王归帝。九年,秦拔我新垣、曲阳之城。

十年,齐灭宋,宋王死我温。十二年,与秦、赵、韩、燕共伐齐,败之济西,湣王出亡。燕独入临菑。与秦王会西周。

十三年,秦拔我安城。兵到大梁,去。十八年,秦拔郢,楚王徙陈。

十九年,昭王卒,子安釐王立。

安釐王元年,秦拔我两城。二年,又拔我二城,军大梁下,韩来救,予秦温以和。三年,秦拔我四城,斩首四万。四年,秦破我及韩、赵,杀十五万人,走我将芒卯。魏将段干子请予秦南阳以和。苏代谓魏王曰:“欲玺者段干子也,欲地者秦也。今王使欲地者制玺,使欲玺者制地,魏氏地不尽则不知已。且夫以地事秦,譬犹抱薪救火,薪不尽,火不灭。”王曰:“是则然也。虽然,事始已行,不可更矣。”对曰:“王独不见夫博之所以贵枭者,便则食,不便则止矣。今王曰‘事始已行,不可更’,是何王之用智不如用枭也?”

九年,秦拔我怀。十年,秦太子外质于魏死。十一年,秦拔我郪丘。

秦昭王谓左右曰:“今时韩、魏与始孰彊?”对曰:“不如始彊。”王曰:“今时如耳、魏齐与孟尝、芒卯孰贤?”对曰:“不如。”王曰:“以孟尝、芒卯之贤,率彊韩、魏以攻秦,犹无柰寡人何也。今以无能之如耳、魏齐而率弱韩、魏以伐秦,其无柰寡人何亦明矣。”左右皆曰:“甚然。”中旗冯琴而对曰:“王之料天下过矣。当晋六卿之时,知氏最彊,灭范、中行,又率韩、魏之兵以围赵襄子于晋阳,决晋水以灌晋阳之城,不湛者三版。知伯行水,魏桓子御,韩康子为参乘。知伯曰:‘吾始不知水之可以亡人之国也,乃今知之。’汾水可以灌安邑,绛水可以灌平阳。魏桓子肘韩康子,韩康子履魏桓子,肘足接于车上,而知氏地分,身死国亡,为天下笑。今秦兵虽彊,不能过知氏;韩、魏虽弱,尚贤其在晋阳之下也。此方其用肘足之时也,原王之勿易也!”于是秦王恐。

齐、楚相约而攻魏,魏使人求救于秦,冠盖相望也,而秦救不至。魏人有唐雎者,年九十馀矣,谓魏王曰:“老臣请西说秦王,令兵先臣出。”魏王再拜,遂约车而遣之。唐雎到,入见秦王。秦王曰:“丈人芒然乃远至此,甚苦矣!夫魏之来求救数矣,寡人知魏之急已。”唐雎对曰:“大王已知魏之急而救不发者,臣窃以为用策之臣无任矣。夫魏,一万乘之国也,然所以西面而事秦,称东籓,受冠带,祠春秋者,以秦之彊足以为与也。今齐、楚之兵已合于魏郊矣,而秦救不发,亦将赖其未急也。使之大急,彼且割地而约从,王尚何救焉?必待其急而救之,是失一东籓之魏而彊二敌之齐、楚,则王何利焉?”于是秦昭王遽为发兵救魏。魏氏复定。

赵使人谓魏王曰:“为我杀范痤,吾请献七十里之地。”魏王曰:“诺。”使吏捕之,围而未杀。痤因上屋骑危,谓使者曰:“与其以死痤市,不如以生痤市。有如痤死,赵不予王地,则王将柰何?故不若与先定割地,然后杀痤。”魏王曰:“善。”痤因上书信陵君曰:“痤,故魏之免相也,赵以地杀痤而魏王听之,有如彊秦亦将袭赵之欲,则君且柰何?”信陵君言于王而出之。

魏王以秦救之故,欲亲秦而伐韩,以求故地。无忌谓魏王曰:

秦与戎翟同俗,有虎狼之心,贪戾好利无信,不识礼义德行。苟有利焉,不顾亲戚兄弟,若禽兽耳,此天下之所识也,非有所施厚积德也。故太后母也,而以忧死;穰侯舅也,功莫大焉,而竟逐之;两弟无罪,而再夺之国。此于亲戚若此,而况于仇雠之国乎?今王与秦共伐韩而益近秦患,臣甚惑之。而王不识则不明,群臣莫以闻则不忠。

今韩氏以一女子奉一弱主,内有大乱,外交彊秦魏之兵,王以为不亡乎?韩亡,秦有郑地,与大梁鄴,王以为安乎?王欲得故地,今负彊秦之亲,王以为利乎?

秦非无事之国也,韩亡之后必将更事,更事必就易与利,就易与利必不伐楚与赵矣。是何也?夫越山逾河,绝韩上党而攻彊赵,是复阏与之事,秦必不为也。若道河内,倍鄴、朝歌,绝漳滏水,与赵兵决于邯郸之郊,是知伯之祸也,秦又不敢。伐楚,道涉谷,行三千里。而攻冥戹之塞,所行甚远,所攻甚难,秦又不为也。若道河外,倍大梁,右上蔡、召陵,与楚兵决于陈郊,秦又不敢。故曰秦必不伐楚与赵矣,又不攻卫与齐矣。

夫韩亡之后,兵出之日,非魏无攻已。秦固有怀、茅、邢丘,城垝津以临河内,河内共、汲。必危;有郑地,得垣雍,决荧泽水灌大梁,大梁必亡。王之使者出过而恶安陵氏于秦,秦之欲诛之久矣。秦叶阳、昆阳与舞阳邻,听使者之恶之,随安陵氏而亡之,绕舞阳之北,以东临许,南国必危,国无害乎?

夫憎韩不爱安陵氏可也,夫不患秦之不爱南国非也。异日者,秦在河西晋,国去梁千里,有河山以阑之,有周韩以间之。从林乡军以至于今,秦七攻魏,五入囿中,边城尽拔,文台堕,垂都焚,林木伐,麋鹿尽,而国继以围。又长驱梁北,东至陶卫之郊,北至平监。所亡于秦者,山南山北,河外河内,大县数十,名都数百。秦乃在河西晋,去梁千里,而祸若是矣,又况于使秦无韩,有郑地,无河山而阑之,无周韩而间之,去大梁百里,祸必由此矣。

异日者,从之不成也,楚、魏疑而韩不可得也。今韩受兵三年,秦桡之以讲,识亡不听,投质于赵,请为天下雁行顿刃,楚、赵必集兵,皆识秦之欲无穷也,非尽亡天下之国而臣海内,必不休矣。是故臣原以从事王,王速受楚赵之约,而挟韩之质以存韩,而求故地,韩必效之。

夫存韩安魏而利天下,此亦王之天时已。通韩上党于共、甯,使道安成,出入赋之,是魏重质韩以其上党也。今有其赋,足以富国。韩必德魏爱魏重魏畏魏,韩必不敢反魏,是韩则魏之县也。魏得韩以为县,卫、大梁、河外必安矣。今不存韩,二周、安陵必危,楚、赵大破,卫、齐甚畏,天下西乡而驰秦入朝而为臣不久矣。

二十年,秦围邯郸,信陵君无忌矫夺将军晋鄙兵以救赵,赵得全。无忌因留赵。二十六年,秦昭王卒。

三十年,无忌归魏,率五国兵攻秦,败之河外,走蒙骜。魏太子增质于秦,秦怒,欲囚魏太子增。或为增谓秦王曰:“公孙喜固谓魏相曰‘请以魏疾击秦,秦王怒,必囚增。魏王又怒,击秦,秦必伤’。今王囚增,是喜之计中也。故不若贵增而合魏,以疑之于齐、韩。”秦乃止增。

三十一年,秦王政初立。

三十四年,安釐王卒,太子增立,是为景湣王。信陵君无忌卒。

景湣王元年,秦拔我二十城,以为秦东郡。二年,秦拔我朝歌。
徙野王。三年,秦拔我汲。五年,秦拔我垣、蒲阳、衍。十五年,景湣王卒,子王假立。

王假元年,燕太子丹使荆轲刺秦王,秦王觉之。

三年,秦灌大梁,虏王假,遂灭魏以为郡县。

太史公曰:吾適故大梁之墟,墟中人曰:“秦之破梁,引河沟而灌大梁,三月城坏,王请降,遂灭魏。”说者皆曰魏以不用信陵君故,国削弱至于亡,余以为不然。天方令秦平海内,其业未成,魏虽得阿衡之佐,曷益乎?

毕公之苗,因国为姓。大名始赏,盈数自正。胤裔繁昌,系载忠正。杨干就戮,智氏奔命。文始建侯,武实彊盛。大梁东徙,长安北侦。卯既无功,卬亦外聘。王假削弱,虏于秦政。
\end{yuanwen}

\chapter{韩世家}

\begin{yuanwen}
韩之先与周同姓,姓姬氏。其后苗裔事晋,得封于韩原,曰韩武子。武子后三世有韩厥,从封姓为韩氏。

韩厥,晋景公之三年,晋司寇屠岸贾将作乱,诛灵公之贼赵盾。赵盾已死矣,欲诛其子赵朔。韩厥止贾,贾不听。厥告赵朔令亡。朔曰:“子必能不绝赵祀,死不恨矣。”韩厥许之。及贾诛赵氏,厥称疾不出。程婴、公孙杵臼之藏赵孤赵武也,厥知之。

景公十一年,厥与郤克将兵八百乘伐齐,败齐顷公于鞍,获逢丑父。于是晋作六卿,而韩厥在一卿之位,号为献子。

晋景公十七年,病,卜大业之不遂者为祟。韩厥称赵成季之功,今后无祀,以感景公。景公问曰:“尚有世乎?”厥于是言赵武,而复与故赵氏田邑,续赵氏祀。

晋悼公之七年,韩献子老。献子卒,子宣子代。宣字徙居州。

晋平公十四年,吴季札使晋,曰:“晋国之政卒归于韩、魏、赵矣。”晋顷公十二年,韩宣子与赵、魏共分祁氏、羊舌氏十县。晋定公十五年,宣子与赵简子侵伐范、中行氏。宣子卒,子贞子代立。贞子徙居平阳。

贞子卒,子简子代。简子卒,子庄子代。庄子卒,子康子代。康子与赵襄子、魏桓子共败知伯,分其地,地益大,大于诸侯。

康子卒,子武子代。武子二年,伐郑,杀其君幽公。十六年,武子卒,子景侯立。

景侯虔元年,伐郑,取雍丘。二年,郑败我负黍。

六年,与赵、魏俱得列为诸侯。

九年,郑围我阳翟。景侯卒,子列侯取立。

列侯三年,聂政杀韩相侠累。九年,秦伐我宜阳,取六邑。十三年,列侯卒,子文侯立。是岁魏文侯卒。

文侯二年,伐郑,取阳城。伐宋,到彭城,执宋君。七年,伐齐,至桑丘。郑反晋。九年,伐齐,至灵丘。十年,文侯卒,子哀侯立。

哀侯元年,与赵、魏分晋国。二年,灭郑,因徙都郑。

六年,韩严弑其君哀侯。而子懿侯立。

懿侯二年,魏败我马陵。五年,与魏惠王会宅阳。九年,魏败我澮。十二年,懿侯卒,子昭侯立。

昭侯元年,秦败我西山。二年,宋取我黄池。魏取硃。六年,伐东周,取陵观、邢丘。

八年,申不害相韩,脩术行道,国内以治,诸侯不来侵伐。

十年,韩姬弑其君悼公。十一年,昭侯如秦。二十二年,申不害死。二十四年,秦来拔我宜阳。

二十五年,旱,作高门。屈宜臼曰:“昭侯不出此门。何也?不时。吾所谓时者,非时日也,人固有利不利时。昭侯尝利矣,不作高门。往年秦拔宜阳,今年旱,昭侯不以此时恤民之急,而顾益奢,此谓‘时绌举赢’。”二十六年,高门成,昭侯卒,果不出此门。子宣惠王立。

宣惠王五年,张仪相秦。八年,魏败我将韩举。十一年,君号为王。与赵会区鼠。十四,秦伐败我鄢。

十六年,秦败我脩鱼,虏得韩将宧、申差于浊泽。韩氏急,公仲谓韩王曰:“与国非可恃也。今秦之欲伐楚久矣,王不如因张仪为和于秦,赂以一名都,具甲,与之南伐楚,此以一易二之计也。”韩王曰:“善。”乃警公仲之行,将西购于秦。楚王闻之大恐,召陈轸告之。陈轸曰:“秦之欲伐楚久矣,今又得韩之名都一而具甲,秦韩并兵而伐楚,此秦所祷祀而求也。今已得之矣,楚国必伐矣。王听臣为之警四境之内,起师言救韩,命战车满道路,发信臣,多其车,重其币,使信王之救己也。纵韩不能听我,韩必德王也,必不为雁行以来,是秦韩不和也,兵虽至,楚不大病也。为能听我绝和于秦,秦必大怒,以厚怨韩。韩之南交楚,必轻秦;轻秦,其应秦必不敬:是因秦、韩之兵而免楚国之患也。”楚王曰:“善。”乃警四境之内,兴师言救韩。命战车满道路,发信臣,多其车,重其币。谓韩王曰:“不穀国虽小,已悉发之矣。原大国遂肆志于秦,不穀将以楚殉韩。”韩王闻之大说,乃止公仲之行。公仲曰:“不可。夫以实伐我者秦也,以虚名救我者楚也。王恃楚之虚名,而轻绝彊秦之敌,王必为天下大笑。且楚韩非兄弟之国也,又非素约而谋伐秦也。已有伐形,因发兵言救韩,此必陈轸之谋也。且王已使人报于秦矣,今不行,是欺秦也。夫轻欺彊秦而信楚之谋臣,恐王必悔之。”韩王不听,遂绝于秦。秦因大怒,益甲伐韩,大战,楚救不至韩。十九年,大破我岸门。太子仓质于秦以和。

二十一年,与秦共攻楚,败楚将屈丐,斩首八万于丹阳。”是岁,宣惠王卒,太子仓立,是为襄王。

襄王四年,与秦武王会临晋。其秋,秦使甘茂攻我宜阳。五年,秦拔我宜阳,斩首六万。秦武王卒。六年,秦复与我武遂。九年,秦复取我武遂。十年,太子婴朝秦而归。十一年,秦伐我,取穰。与秦伐楚,败楚将唐眛。

十二年,太子婴死。公子咎、公子虮虱争为太子。时虮虱质于楚。苏代谓韩咎曰:“虮虱亡在楚,楚王欲内之甚。今楚兵十馀万在方城之外,公何不令楚王筑万室之都雍氏之旁,韩必起兵以救之,公必将矣。公因以韩楚之兵奉虮虱而内之,其听公必矣,必以楚韩封公也。”韩咎从其计。

楚围雍氏,韩求救于秦。秦未为发,使公孙昧入韩。公仲曰:“子以秦为且救韩乎?”对曰:“秦王之言曰‘请道南郑、蓝田,出兵于楚以待公’,殆不合矣。”公仲曰:“子以为果乎?”对曰:“秦王必祖张仪之故智。”楚威王攻梁也,张仪谓秦王曰:‘与楚攻魏,魏折而入于楚,韩固其与国也,是秦孤也。不如出兵以到之,魏楚大战,秦取西河之外以归。’今其状阳言与韩,其实阴善楚。公待秦而到,必轻与楚战。楚阴得秦之不用也,必易与公相支也。公战而胜楚,遂与公乘楚,施三川而归。公战不胜楚,楚塞三川守之,公不能救也。窃为公患之。司马庚三反于郢,甘茂与昭鱼遇于商于,其言收玺,实类有约也。”公仲恐,曰:“然则柰何?”曰:“公必先韩而后秦,先身而后张仪。公不如亟以国合于齐楚,齐楚必委国于公。公之所恶者张仪也,其实犹不无秦也。”于是楚解雍氏围。

苏代又谓秦太后弟琇戎曰:“公叔伯婴恐秦楚之内虮虱也,公何不为韩求质子于楚?楚王听入质子于韩,则公叔伯婴知秦楚之不以虮虱为事,必以韩合于秦楚。秦楚挟韩以窘魏,魏氏不敢合于齐,是齐孤也。公又为秦求质子于楚,楚不听,怨结于韩。韩挟齐魏以围楚,楚必重公。公挟秦楚之重以积德于韩,公叔伯婴必以国待公。”于是虮虱竟不得归韩。韩立咎为太子。齐、魏王来。

十四年,与齐、魏王共击秦,至函谷而军焉。十六年,秦与我河外及武遂。襄王卒,太子咎立,是为釐王。

釐王三年,使公孙喜率周、魏攻秦。秦败我二十四万,虏喜伊阙。五年,秦拔我宛。六年,与秦武遂地二百里。十年,秦败我师于夏山。十二年,与秦昭王会西周而佐秦攻齐。齐败,湣王出亡。十四年,与秦会两周间。二十一年,使暴烝救魏,为秦所败,烝走开封。  二十三年,赵、魏攻我华阳。韩告急于秦,秦不救。韩相国谓陈筮曰:“事急,原公虽病,为一宿之行。”陈筮见穰侯。穰侯曰:“事急乎?故使公来。”陈筮曰:“未急也。”穰侯怒曰:“是可以为公之主使乎?夫冠盖相望,告敝邑甚急,公来言未急,何也?”陈筮曰:“彼韩急则将变而佗从,以未急,故复来耳。”穰侯曰:“公无见王,请今发兵救韩。”八日而至,败赵、魏于华阳之下。是岁,釐王卒,子桓惠王立。

桓惠王元年,伐燕。九年,秦拔我陉,城汾旁。十年,秦击我于太行,我上党郡守以上党郡降赵。十四年,秦拔赵上党,杀马服子卒四十馀万于长平。十七年,秦拔我阳城、负黍。二十二年,秦昭王卒。二十四年,秦拔我城皋、荥阳。二十六年,秦悉拔我上党。二十九年,秦拔我十三城。

三十四年,桓惠王卒,子王安立。

王安五年,秦攻韩,韩急,使韩非使秦,秦留非,因杀之。

九年,秦虏王安,尽入其地,为颍州郡。韩遂亡。

太史公曰:韩厥之感晋景公,绍赵孤之子武,以成程婴、公孙杵臼之义,此天下之阴德也。韩氏之功,于晋未睹其大者也。然与赵、魏终为诸侯十馀世,宜乎哉!

韩氏之先,实宗周武。事微国小,春秋无语。后裔事晋,韩原是处。赵孤克立,智伯可取。既徙平阳,又侵负黍。景赵俱侯,惠又僭主。秦败脩鱼,魏会区鼠。韩非虽使,不禁狼虎。
\end{yuanwen}

\chapter{田敬仲完世家}

\begin{yuanwen}
陈完者,陈厉公他之子也。完生,周太史过陈,陈厉公使卜完,卦得观之否:“是为观国之光,利用宾于王。此其代陈有国乎?不在此而在异国乎?非此其身也,在其子孙。若在异国,必姜姓。姜姓,四岳之后。物莫能两大,陈衰,此其昌乎?”

厉公者,陈文公少子也,其母蔡女。文公卒,厉公兄鲍立,是为桓公。桓公与他异母。及桓公病,蔡人为他杀桓公鲍及太子免而立他,为厉公。厉公既立,娶蔡女。蔡女淫于蔡人,数归,厉公亦数如蔡。桓公之少子林怨厉公杀其父与兄,乃令蔡人诱厉公而杀之。林自立,是为庄公。故陈完不得立,为陈大夫。厉公之杀,以淫出国,故春秋曰“蔡人杀陈他”,罪之也。

庄公卒,立弟杵臼,是为宣公。宣公二十一年,杀其太子御寇。御寇与完相爱,恐祸及己,完故奔齐。齐桓公欲使为卿,辞曰:“羁旅之臣幸得免负檐,君之惠也,不敢当高位。”桓公使为工正。齐懿仲欲妻完,卜之,占曰:“是谓凤皇于蜚,和鸣锵锵。有妫之后,将育于姜。五世其昌,并于正卿。八世之后,莫之与京。”卒妻完。完之奔齐,齐桓公立十四年矣。

完卒,谥为敬仲。仲生孟夷。敬仲之如齐,以陈字为田氏。

田孟夷生湣孟庄,田湣孟庄生文子须无。田文子事齐庄公。

晋之大夫栾逞作乱于晋,来奔齐,齐庄公厚客之。晏婴与田文子谏,庄公弗听。

文子卒,生桓子无宇。田桓子无宇有力,事齐庄公,甚有宠。

无宇卒,生武子开与釐子乞。田釐子乞事齐景公为大夫,其收赋税于民以小斗受之,其禀予民以大斗,行阴德于民,而景公弗禁。由此田氏得齐众心,宗族益彊,民思田氏。晏子数谏景公,景公弗听。已而使于晋,与叔向私语曰:“齐国之政卒归于田氏矣。”

晏婴卒后,范、中行氏反晋。晋攻之急,范、中行请粟于齐。田乞欲为乱,树党于诸侯,乃说景公曰:“范、中行数有德于齐,齐不可不救。”齐使田乞救之而输之粟。

景公太子死,后有宠姬曰芮子,生子荼。景公病,命其相国惠子与高昭子以子荼为太子。景公卒,两相高、国立荼,是为晏孺子。而田乞不说,欲立景公他子阳生。阳生素与乞欢。晏孺子之立也,阳生奔鲁。田乞伪事高昭子、国惠子者,每朝代参乘,言曰:“始诸大夫不欲立孺子。孺子既立,君相之,大夫皆自危,谋作乱。”又绐大夫曰:“高昭子可畏也,及未发先之。”诸大夫从之。田乞、鲍牧与大夫以兵入公室,攻高昭子。昭子闻之,与国惠子救公。公师败。田乞之众追国惠子,惠子奔莒,遂返杀高昭子。晏圉奔鲁。

田乞使人之鲁,迎阳生。阳生至齐,匿田乞家。请诸大夫曰:“常之母有鱼菽之祭,幸而来会饮。”会饮田氏。田乞盛阳生橐中,置坐中央。发橐,出阳生,曰:“此乃齐君矣。”大夫皆伏谒。将盟立之,田乞诬曰:“吾与鲍牧谋共立阳生也。”鲍牧怒曰:“大夫忘景公之命乎?”诸大夫欲悔,阳生乃顿首曰:“可则立之,不可则已。”鲍牧恐祸及己,乃复曰:“皆景公之子,何为不可!”遂立阳生于田乞之家,是为悼公。乃使人迁晏孺子于骀,而杀孺子荼。悼公既立,田乞为相,专齐政。

四年,田乞卒,子常代立,是为田成子。

鲍牧与齐悼公有郄,弑悼公。齐人共立其子壬,是为简公。田常成子与监止俱为左右相,相简公。田常心害监止,监止幸于简公,权弗能去。于是田常复脩釐子之政,以大斗出贷,以小斗收。齐人歌之曰:“妪乎采芑,归乎田成子!”齐大夫朝,御鞅谏简公曰:“田、监不可并也,君其择焉。”君弗听。

子我者,监止之宗人也,常与田氏有卻。田氏疏族田豹事子我有宠。子我曰:“吾欲尽灭田氏適,以豹代田氏宗。”豹曰:“臣于田氏疏矣。”不听。已而豹谓田氏曰:“子我将诛田氏,田氏弗先,祸及矣。”子我舍公宫,田常兄弟四人乘如公宫,欲杀子我。子我闭门。简公与妇人饮檀台,将欲击田常。太史子馀曰:“田常非敢为乱,将除害。”简公乃止。田常出,闻简公怒,恐诛,将出亡。田子行曰:“需,事之贼也。”田常于是击子我。子我率其徒攻田氏,不胜,出亡。田氏之徒追杀子我及监止。

简公出奔,田氏之徒追执简公于徐州。简公曰:“蚤从御鞅之言,不及此难。”田氏之徒恐简公复立而诛己,遂杀简公。简公立四年而杀。于是田常立简公弟骜,是为平公。平公即位,田常为相。

田常既杀简公,惧诸侯共诛己,乃尽归鲁、卫侵地,西约晋、韩、魏、赵氏,南通吴、越之使,脩功行赏,亲于百姓,以故齐复定。

田常言于齐平公曰:“德施人之所欲,君其行之;刑罚人之所恶,臣请行之。”行之五年,齐国之政皆归田常。田常于是尽诛鲍、晏、监止及公族之彊者,而割齐自安平以东至琅邪,自为封邑。封邑大于平公之所食。

田常乃选齐国中女子长七尺以上为后宫,后宫以百数,而使宾客舍人出入后宫者不禁。及田常卒,有七十馀男。

田常卒,子襄子盘代立,相齐。常谥为成子。

田襄子既相齐宣公,三晋杀知伯,分其地。襄子使其兄弟宗人尽为齐都邑大夫,与三晋通使,且以有齐国。

襄子卒,子庄子白立。田庄子相齐宣公。宣公四十三年,伐晋,毁黄城,围阳狐。明年,伐鲁、葛及安陵。明年,取鲁之一城。

庄子卒,子太公和立。田太公相齐宣公。宣公四十八年,取鲁之郕。明年,宣公与郑人会西城。伐卫,取毌丘。宣公五十一年卒,田会自廪丘反。

宣公卒,子康公贷立。贷立十四年,淫于酒妇人,不听政。太公乃迁康公于海上,食一城,以奉其先祀。明年,鲁败齐平陆。

三年,太公与魏文侯会浊泽,求为诸侯。魏文侯乃使使言周天子及诸侯,请立齐相田和为诸侯。周天子许之。康公之十九年,田和立为齐侯,列于周室,纪元年。

齐侯太公和立二年,和卒,子桓公午立。桓公午五年,秦、魏攻韩,韩求救于齐。齐桓公召大臣而谋曰:“蚤救之孰与晚救之?”驺忌曰:“不若勿救。”段干朋曰:“不救,则韩且折而入于魏,不若救之。”田臣思曰:“过矣君之谋也!秦、魏攻韩、楚,赵必救之,是天以燕予齐也。”桓公曰:“善”。乃阴告韩使者而遣之。韩自以为得齐之救,因与秦、魏战。楚、赵闻之,果起兵而救之。齐因起兵袭燕国,取桑丘。

六年,救卫。桓公卒,子威王因齐立。是岁,故齐康公卒,绝无后,奉邑皆入田氏。

齐威王元年,三晋因齐丧来伐我灵丘。三年,三晋灭晋后而分其地。六年,鲁伐我,入阳关。晋伐我,至博陵。七年,卫伐我,取薛陵。九年,赵伐我,取甄。

威王初即位以来,不治,委政卿大夫,九年之间,诸侯并伐,国人不治。于是威王召即墨大夫而语之曰:“自子之居即墨也,毁言日至。然吾使人视即墨,田野辟,民人给,官无留事,东方以宁。是子不事吾左右以求誉也。”封之万家。召阿大夫语曰:“自子之守阿,誉言日闻。然使使视阿,田野不辟,民贫苦。昔日赵攻甄,子弗能救。卫取薛陵,子弗知。是子以币厚吾左右以求誉也。”是日,烹阿大夫,及左右尝誉者皆并烹之。遂起兵西击赵、卫,败魏于浊泽而围惠王。惠王请献观以和解,赵人归我长城。于是齐国震惧,人人不敢饰非,务尽其诚。齐国大治。诸侯闻之,莫敢致兵于齐二十馀年。

驺忌子以鼓琴见威王,威王说而舍之右室。须臾,王鼓琴,驺忌子推户入曰:“善哉鼓琴!”王勃然不说,去琴按剑曰:“夫子见容未察,何以知其善也?”驺忌子曰:“夫大弦浊以春温者,君也;小弦廉折以清者,相也;攫之深,醳之愉者,政令也;钧谐以鸣,大小相益,回邪而不相害者,四时也:吾是以知其善也。”王曰:“善语音。”驺忌子曰:“何独语音,夫治国家而弭人民皆在其中。”王又勃然不说曰:“若夫语五音之纪,信未有如夫子者也。若夫治国家而弭人民,又何为乎丝桐之间?”驺忌子曰:“夫大弦浊以春温者,君也;小弦廉折以清者,相也;攫之深而舍之愉者,政令也;钧谐以鸣,大小相益,回邪而不相害者,四时也。夫复而不乱者,所以治昌也;连而径者,所以存亡也:故曰琴音调而天下治。夫治国家而弭人民者,无若乎五音者。”王曰:“善。”

驺忌子见三月而受相印。淳于髡见之曰:“善说哉!髡有愚志,原陈诸前。”驺忌子曰:“谨受教。”淳于髡曰:“得全全昌,失全全亡。”驺忌子曰:“谨受令,请谨毋离前。”淳于髡曰:“豨膏棘轴,所以为滑也,然而不能运方穿。”驺忌子曰:“谨受令,请谨事左右。”淳于髡曰:“弓胶昔幹,所以为合也,然而不能傅合疏罅。”驺忌子曰:“谨受令,请谨自附于万民。”淳于髡曰:“狐裘虽敝,不可补以黄狗之皮。”驺忌子曰:“谨受令,请谨择君子,毋杂小人其间。”淳于髡曰:“大车不较,不能载其常任;琴瑟不较,不能成其五音。”驺忌子曰:“谨受令,请谨脩法律而督奸吏。”淳于髡说毕,趋出,至门,而面其仆曰:“是人者,吾语之微言五,其应我若响之应声,是人必封不久矣。”居期年,封以下邳,号曰成侯。

威王二十三年,与赵王会平陆。二十四年,与魏王会田于郊。魏王问曰:“王亦有宝乎?”威王曰:“无有。”梁王曰:“若寡人国小也,尚有径寸之珠照车前后各十二乘者十枚,奈何以万乘之国而无宝乎?”威王曰:“寡人之所以为宝与王异。吾臣有檀子者,使守南城,则楚人不敢为寇东取,泗上十二诸侯皆来朝。吾臣有朌子者,使守高唐,则赵人不敢东渔于河。吾吏有黔夫者,使守徐州,则燕人祭北门,赵人祭西门,徙而从者七千馀家。吾臣有种首者,使备盗贼,则道不拾遗。将以照千里,岂特十二乘哉!”梁惠王惭,不怿而去。

二十六年,魏惠王围邯郸,赵求救于齐。齐威王召大臣而谋曰:“救赵孰与勿救?”驺忌子曰:“不如勿救。”段干朋曰:“不救则不义,且不利。”威王曰:“何也?”对曰:“夫魏氏并邯郸,其于齐何利哉?且夫救赵而军其郊,是赵不伐而魏全也。故不如南攻襄陵以弊魏,邯郸拔而乘魏之弊。”威王从其计。

其后成侯驺忌与田忌不善,公孙阅谓成侯忌曰:“公何不谋伐魏,田忌必将。战胜有功,则公之谋中也;战不胜,非前死则后北,而命在公矣。”于是成侯言威王,使田忌南攻襄陵。十月,邯郸拔,齐因起兵击魏,大败之桂陵。于是齐最彊于诸侯,自称为王,以令天下。

三十三年,杀其大夫牟辛。

三十五年,公孙阅又谓成侯忌曰:“公何不令人操十金卜于市,曰‘我田忌之人也。吾三战而三胜,声威天下。欲为大事,亦吉乎不吉乎’?”卜者出,因令人捕为之卜者,验其辞于王之所。田忌闻之,因率其徒袭攻临淄,求成侯,不胜而饹。

三十六年,威王卒,子宣王辟彊立。

宣王元年,秦用商鞅。周致伯于秦孝公。

二年,魏伐赵。赵与韩亲,共击魏。赵不利,战于南梁。宣王召田忌复故位。韩氏请救于齐。宣王召大臣而谋曰:“蚤救孰与晚救?”驺忌子曰:“不如勿救。”田忌曰:“弗救,则韩且折而入于魏,不如蚤救之。”孙子曰:“夫韩、魏之兵未弊而救之,是吾代韩受魏之兵,顾反听命于韩也。且魏有破国之志,韩见亡,必东面而愬于齐矣。吾因深结韩之亲而晚承魏之弊,则可重利而得尊名也。”宣王曰:“善。”乃阴告韩之使者而遣之。韩因恃齐,五战不胜,而东委国于齐。齐因起兵,使田忌、田婴将,孙子为师,救韩、赵以击魏,大败之马陵,杀其将庞涓,虏魏太子申。其后三晋之王皆因田婴朝齐王于博望,盟而去。

七年,与魏王会平阿南。明年,复会甄。魏惠王卒。明年,与魏襄王会徐州,诸侯相王也。十年,楚围我徐州。十一年,与魏伐赵,赵决河水灌齐、魏,兵罢。十八年,秦惠王称王。

宣王喜文学游说之士,自如驺衍、淳于髡、田骈、接予、慎到、环渊之徒七十六人,皆赐列第,为上大夫,不治而议论。是以齐稷下学士复盛,且数百千人。

十九年,宣王卒,子湣王地立。

湣王元年,秦使张仪与诸侯执政会于齧桑。三年,封田婴于薛。四年,迎妇于秦。七年,与宋攻魏,败之观泽。

十二年,攻魏。楚围雍氏,秦败屈丐。苏代谓田轸曰:“臣原有谒于公,其为事甚完,使楚利公,成为福,不成亦为福。今者臣立于门,客有言曰魏王谓韩冯、张仪曰:‘煮枣将拔,齐兵又进,子来救寡人则可矣;不救寡人,寡人弗能拔。’此特转辞也。秦、韩之兵毋东,旬馀,则魏氏转韩从秦,秦逐张仪,交臂而事齐楚,此公之事成也。”田轸曰:“柰何使无东?”对曰:“韩冯之救魏之辞,必不谓韩王曰‘冯以为魏’,必曰‘冯将以秦韩之兵东卻齐宋,冯因抟三国之兵,乘屈丐之弊,南割于楚,故地必尽得之矣’。张仪救魏之辞,必不谓秦王曰‘仪以为魏’,必曰‘仪且以秦韩之兵东距齐宋,仪将抟三国之兵,乘屈丐之弊,南割于楚,名存亡国,实伐三川而归,此王业也’。公令楚王与韩氏地,使秦制和,谓秦王曰‘请与韩地,而王以施三川,韩氏之兵不用而得地于楚’。韩冯之东兵之辞且谓秦何?曰‘秦兵不用而得三川,伐楚韩以窘魏,魏氏不敢东,是孤齐也’。张仪之东兵之辞且谓何?曰‘秦韩欲地而兵有案,声威发于魏,魏氏之欲不失齐楚者有资矣’。魏氏转秦韩争事齐楚,楚王欲而无与地,公令秦韩之兵不用而得地,有一大德也。秦韩之王劫于韩冯、张仪而东兵以徇服魏,公常执左券以责于秦韩,此其善于公而恶张子多资矣。”

十三年,秦惠王卒。二十三年,与秦击败楚于重丘。二十四年,秦使泾阳君质于齐。二十五年,归泾阳君于秦。孟尝君薛文入秦,即相秦。文亡去。二十六年,齐与韩魏共攻秦,至函谷军焉。二十八年,秦与韩河外以和,兵罢。二十九年,赵杀其主父。齐佐赵灭中山。

三十六年,王为东帝,秦昭王为西帝。苏代自燕来,入齐,见于章华东门。齐王曰:“嘻,善,子来!秦使魏厓致帝,子以为何如?”对曰:“王之问臣也卒,而患之所从来微,原王受之而勿备称也。秦称之,天下安之,王乃称之,无后也。且让争帝名,无伤也。秦称之,天下恶之,王因勿称,以收天下,此大资也。且天下立两帝,王以天下为尊齐乎?尊秦乎?”王曰:“尊秦。”曰:“释帝,天下爱齐乎?爱秦乎?”王曰:“爱齐而憎秦。”曰:“两帝立约伐赵,孰与伐桀宋之利?”王曰:“伐桀宋利。”对曰:“夫约钧,然与秦为帝而天下独尊秦而轻齐,释帝则天下爱齐而憎秦,伐赵不如伐桀宋之利,故原王明释帝以收天下,倍约宾秦,无争重,而王以其间举宋。夫有宋,卫之阳地危;有济西,赵之阿东国危;有淮北,楚之东国危;有陶、平陆,梁门不开。释帝而贷之以伐桀宋之事,国重而名尊,燕楚所以形服,天下莫敢不听,此汤武之举也。敬秦以为名,而后使天下憎之,此所谓以卑为尊者也。原王孰虑之。”于是齐去帝复为王,秦亦去帝位。

三十八年,伐宋。秦昭王怒曰:“吾爱宋与爱新城、阳晋同。韩聂与吾友也,而攻吾所爱,何也?”苏代为齐谓秦王曰:“韩聂之攻宋,所以为王也。齐彊,辅之以宋,楚魏必恐,恐必西事秦,是王不烦一兵,不伤一士,无事而割安邑也,此韩聂之所祷于王也。”秦王曰:“吾患齐之难知。一从一衡,其说何也?”对曰:“天下国令齐可知乎?齐以攻宋,其知事秦以万乘之国自辅,不西事秦则宋治不安。中国白头游敖之士皆积智欲离齐秦之交,伏式结轶西驰者,未有一人言善齐者也,伏式结轶东驰者,未有一人言善秦者也。何则?皆不欲齐秦之合也。何晋楚之智而齐秦之愚也!晋楚合必议齐秦,齐秦合必图晋楚,请以此决事。”秦王曰:“诺。”于是齐遂伐宋,宋王出亡,死于温。齐南割楚之淮北,西侵三晋,欲以并周室,为天子。泗上诸侯邹鲁之君皆称臣,诸侯恐惧。

三十九年,秦来伐,拔我列城九。

四十年,燕、秦、楚、三晋合谋,各出锐师以伐,败我济西。王解而卻。燕将乐毅遂入临淄,尽取齐之宝藏器。湣王出亡,之卫。卫君辟宫舍之,称臣而共具。湣王不逊,人侵之。湣王去,走邹、鲁,有骄色,邹、鲁君弗内,遂走莒。楚使淖齿将兵救齐,因相齐湣王。淖齿遂杀湣王而与燕共分齐之侵地卤器。

湣王之遇杀,其子法章变名姓为莒太史敫家庸。太史敫女奇法章状貌,以为非恆人,怜而常窃衣食之,而与私通焉。淖齿既以去莒,莒中人及齐亡臣相聚求湣王子,欲立之。法章惧其诛己也,久之,乃敢自言“我湣王子也”。于是莒人共立法章,是为襄王。以保莒城而布告齐国中:“王已立在莒矣。”

襄王既立,立太史氏女为王后,是为君王后,生子建。太史敫曰:“女不取媒因自嫁,非吾种也,汙吾世。”终身不睹君王后。君王后贤,不以不睹故失人子之礼。

襄王在莒五年,田单以即墨攻破燕军,迎襄王于莒,入临菑。齐故地尽复属齐。齐封田单为安平君。

十四年,秦击我刚寿。十九年,襄王卒,子建立。

王建立六年,秦攻赵,齐楚救之。秦计曰:“齐楚救赵,亲则退兵,不亲遂攻之。”赵无食,请粟于齐,齐不听。周子曰:“不如听之以退秦兵,不听则秦兵不卻,是秦之计中而齐楚之计过也。且赵之于齐楚,扞蔽也,犹齿之有脣也,脣亡则齿寒。今日亡赵,明日患及齐楚。且救赵之务,宜若奉漏甕沃焦釜也。夫救赵,高义也;卻秦兵,显名也。义救亡国,威卻彊秦之兵,不务为此而务爱粟,为国计者过矣。”齐王弗听。秦破赵于长平四十馀万,遂围邯郸。

十六年,秦灭周。君王后卒。二十三年,秦置东郡。二十八年,王入朝秦,秦王政置酒咸阳。三十五年,秦灭韩。三十七年,秦灭赵。三十八年,燕使荆轲刺秦王,秦王觉,杀轲。明年,秦破燕,燕王亡走辽东。明年,秦灭魏,秦兵次于历下。四十二年,秦灭楚。明年,虏代王嘉,灭燕王喜。

四十四年,秦兵击齐。齐王听相后胜计,不战,以兵降秦。秦虏王建,迁之共。遂灭齐为郡。天下壹并于秦,秦王政立号为皇帝。始,君王后贤,事秦谨,与诸侯信,齐亦东边海上,秦日夜攻三晋、燕、楚,五国各自救于秦,以故王建立四十馀年不受兵。君王后死,后胜相齐,多受秦间金,多使宾客入秦,秦又多予金,客皆为反间,劝王去从朝秦,不脩攻战之备,不助五国攻秦,秦以故得灭五国。五国已亡,秦兵卒入临淄,民莫敢格者。王建遂降,迁于共。故齐人怨王建不蚤与诸侯合从攻秦,听奸臣宾客以亡其国,歌之曰:“松耶柏耶?住建共者客耶?”疾建用客之不详也。

太史公曰:盖孔子晚而喜易。易之为术,幽明远矣,非通人达才孰能注意焉!故周太史之卦田敬仲完,占至十世之后;及完奔齐,懿仲卜之亦云。田乞及常所以比犯二君,专齐国之政,非必事势之渐然也,盖若遵厌兆祥云。

田完避难,奔于大姜;始辞羁旅,终然凤皇。物莫两盛,代五其昌。二君比犯,三晋争强。和始擅命,威遂称王。祭急燕、赵,弟列康、庄。秦假东帝,莒立法章。王建失国,松柏苍苍。
\end{yuanwen}

\part{卷四十七}
\chapter{孔子世家第十七}

记述了孔子一生所从事的种种活动,介绍并高度评价了他的思想学说,对其坎坷周流、困顿不遇的一生,寄寓了极大的惋惜和同情。司马迁对孔子顽强刻苦、虚心好学的精神和他那种渊博的知识学问,以及他为研究整理古代文献所付出的巨大努力与他所取得的丰富成果,表现了极大的敬仰与赞佩之情。司马迁认为孔子是我国古代足以称为“周公第二”的大圣人、大学者,并立志以孔子为楷模,要写“第二部《春秋》”,要做“孔子第二”。孔子有宏伟的政治理想,并有将这种理想付诸实践的政治才干,作品中对此有充分表现,但客观形势总是对孔子不利,以至于使他到处碰壁,司马迁对此表现了无比的愤慨与同情。《孔子世家》的悲剧气氛与整个《史记》的悲剧气氛相一致。孔子那种百折不挠、锲而不舍,宁知其不可为而为之,以及他那种不改变信念,不降低目标,绝不与恶势力同流合污的奋斗精神,使司马迁极为赞赏。在这篇作品中,司马迁塑造了一个他心目中所理想的古代士人的悲剧形象。

《孔子世家》是司马迁根据《论语》、《左传》、《孟子》、《礼记》等书中旧有的资料加以排比、谱列而成的。这项谱列工作在很大的程度上是出于司马迁的独创,因为迄今为止,还没有发现先秦的古籍中有过孔子的传记或是年谱一类的东西,因此《孔子世家》就成了远从汉代以来研究孔子思想生平的最重要的依据之一,在我国学术史上有着极其重要的地位。

\begin{yuanwen}
孔子生鲁昌平乡陬邑\footnote{zōu}。其先宋人\footnote{text}也,曰孔防叔。防叔生伯夏,伯夏生叔梁纥\footnote{hé}。纥与颜氏女野合\footnote{未经婚嫁而交合。}而生孔子\footnote{text},祷于尼丘得孔子\footnote{text}。鲁襄公二十二年而孔子生\footnote{text}。生而首上圩顶\footnote{text},故因名曰“丘”云。字仲尼,姓孔氏。
\end{yuanwen}

孔子出生在鲁国昌平乡的陬邑。他的祖先是宋国人,名叫孔防叔。防叔生了伯夏,伯夏生了叔梁纥。叔梁纥和一位姓颜的女子野合生下了孔子,他们来到尼丘山祈祷之后就生下了孔子。鲁襄公二十二年(前551年)时,孔子出生。孔子刚出生时,头顶中间较低,四周较高,所以就取名叫丘。字仲尼,姓孔。

\begin{yuanwen}
丘生而叔梁纥死\footnote{text},葬于防山\footnote{text}。防山在鲁东,由是孔子疑其父墓处,母讳\footnote{text}之也。孔子为儿嬉戏,常陈俎豆\footnote{text},设礼容\footnote{text}。孔子母死,乃殡五父之衢\footnote{text},盖其慎也。陬\footnote{zōu}人輓父之母诲孔子父墓\footnote{text},然后往合葬于防焉。
\end{yuanwen}

孔子出生之后,叔梁纥就死了,埋葬在防山。防山位于鲁国的东部,因而孔子并不知道父亲的坟墓在何处,他的母亲也不愿说。孔子小时候嬉戏游玩时,常陈列一些俎豆之类的礼器,举行仪式。孔子的母亲去世后,就停柩在五父之衢,并未埋葬,这是出于慎重的考虑。陬邑人輓父的母亲告诉孔子他父亲的墓地所在,然后孔子前去将母亲与父亲合葬在了防山。

\begin{yuanwen}
孔子要\footnote{通“腰”。}绖\footnote{dié},季氏飨士,孔子与往。阳虎绌\footnote{通“黜”,斥退,不接待。}曰:“季氏飨士,非敢飨子也。”孔子由是退。
\end{yuanwen}

孔子的腰里还系着白布孝带时,季孙氏设宴款待名士,孔子前去参加。阳虎驱赶他说:“季孙氏设宴款待名士,并没有请你来。”孔子由此就退了出来。

\begin{yuanwen}
孔子年十七\footnote{text},鲁大夫孟釐子病且死,诫其嗣懿子曰:“孔丘,圣人之后\footnote{text},灭于宋\footnote{text}。其祖弗父何始有宋而嗣让厉公\footnote{text}。及正考父佐戴、武、宣公,三命兹益恭\footnote{text},故鼎铭云:‘一命而偻\footnote{text},再命而伛\footnote{text},三命而俯,循墙而走\footnote{text},亦莫敢余侮。饘于是\footnote{text},粥于是,以糊余口。’其恭如是。吾闻圣人之后,虽不当世\footnote{text},必有达者。今孔丘年少好礼,其达者欤?吾即没,若必师之。”及釐子卒,懿子与鲁人南宫敬叔往学礼焉\footnote{text}。是岁\footnote{text},季武子卒,平子代立\footnote{text}。
\end{yuanwen}


\begin{yuanwen}
孔子贫且贱。及长,尝为季氏史,料量平;尝为司职吏而畜蕃息。由是为司空。已而去鲁,斥乎齐,逐乎宋、卫,困于陈蔡之间,于是反鲁。孔子长九尺有六寸,人皆谓之“长人”而异之。鲁复善待,由是反鲁。

鲁南宫敬叔言鲁君曰:“请与孔子適周。”鲁君与之一乘车,两马,一竖子俱,適周问礼,盖见老子云。辞去,而老子送之曰:“吾闻富贵者送人以财,仁人者送人以言。吾不能富贵,窃仁人之号,送子以言,曰:“聪明深察而近于死者,好议人者也。博辩广大危其身者,发人之恶者也。为人子者毋以有己,为人臣者毋以有己。””孔子自周反于鲁,弟子稍益进焉。

是时也,晋平公淫,六卿擅权,东伐诸侯;楚灵王兵彊,陵轹中国;齐大而近于鲁。鲁小弱,附于楚则晋怒;附于晋则楚来伐;不备于齐,齐师侵鲁。

鲁昭公之二十年,而孔子盖年三十矣。齐景公与晏婴来適鲁,景公问孔子曰:“昔秦穆公国小处辟,其霸何也?”对曰:“秦,国虽小,其志大;处虽辟,行中正。身举五羖,爵之大夫,起累绁之中,与语三日,授之以政。以此取之,虽王可也,其霸小矣。”景公说。

孔子年三十五,而季平子与郈昭伯以斗鸡故得罪鲁昭公,昭公率师击平子,平子与孟氏、叔孙氏三家共攻昭公,昭公师败,奔于齐,齐处昭公乾侯。其后顷之,鲁乱。孔子適齐,为高昭子家臣,欲以通乎景公。与齐太师语乐,闻韶音,学之,三月不知肉味,齐人称之。

景公问政孔子,孔子曰:“君君,臣臣,父父,子子。”景公曰:“善哉!信如君不君,臣不臣,父不父,子不子,虽有粟,吾岂得而食诸!”他日又复问政于孔子,孔子曰:“政在节财。”景公说,将欲以尼谿田封孔子。晏婴进曰:“夫儒者滑稽而不可轨法;倨傲自顺,不可以为下;崇丧遂哀,破产厚葬,不可以为俗;游说乞贷,不可以为国。自大贤之息,周室既衰,礼乐缺有间。今孔子盛容饰,繁登降之礼,趋详之节,累世不能殚其学,当年不能究其礼。君欲用之以移齐俗,非所以先细民也。”后景公敬见孔子,不问其礼。异日,景公止孔子曰:“奉子以季氏,吾不能。”以季孟之间待之。齐大夫欲害孔子,孔子闻之。景公曰:“吾老矣,弗能用也。”孔子遂行,反乎鲁。

孔子年四十二,鲁昭公卒于乾侯,定公立。定公立五年,夏,季平子卒,桓子嗣立。季桓子穿井得土缶,中若羊,问仲尼云“得狗”。仲尼曰:“以丘所闻,羊也。丘闻之,木石之怪夔、罔阆,水之怪龙、罔象,土之怪坟羊。”

吴伐越,堕会稽,得骨节专车。吴使使问仲尼:“骨何者最大?”仲尼曰:“禹致群神于会稽山,防风氏后至,禹杀而戮之,其节专车,此为大矣。”吴客曰:“谁为神?”仲尼曰:“山川之神足以纲纪天下,其守为神,社稷为公侯,皆属于王者。”客曰:“防风何守?”仲尼曰:“汪罔氏之君守封、禺之山,为釐姓。在虞、夏、商为汪罔,于周为长翟,今谓之大人。”客曰:“人长几何?”仲尼曰:“僬侥氏三尺,短之至也。长者不过十之,数之极也。”于是吴客曰:“善哉圣人!”

桓子嬖臣曰仲梁怀,与阳虎有隙。阳虎欲逐怀,公山不狃止之。其秋,怀益骄,阳虎执怀。桓子怒,阳虎因囚桓子,与盟而醳之。阳虎由此益轻季氏。季氏亦僭于公室,陪臣执国政,是以鲁自大夫以下皆僭离于正道。故孔子不仕,退而脩诗书礼乐,弟子弥众,至自远方,莫不受业焉。

定公八年,公山不狃不得意于季氏,因阳虎为乱,欲废三桓之適,更立其庶孽阳虎素所善者,遂执季桓子。桓子诈之,得脱。定公九年,阳虎不胜,奔于齐。是时孔子年五十。

公山不狃以费畔季氏,使人召孔子。孔子循道弥久,温温无所试,莫能己用,曰:“盖周文武起丰镐而王,今费虽小,傥庶几乎!”欲往。子路不说,止孔子。孔子曰:“夫召我者岂徒哉?如用我,其为东周乎!”然亦卒不行。
\end{yuanwen}


\begin{yuanwen}
其后定公以孔子为中都宰\footnote{text},一年,四方皆则之。由中都宰为司空,由司空为大司寇\footnote{text}。
\end{yuanwen}


\begin{yuanwen}
定公十年春\footnote{text},及齐平\footnote{text}。夏,齐大夫黎鉏言于景公曰:“鲁用孔丘,其势危齐。”乃使使告鲁为好会,会于夹谷\footnote{text}。鲁定公且以乘车好往\footnote{text}。孔子摄相事\footnote{text},曰:“臣闻有文事者必有武备,有武事者必有文备。古者诸侯出疆,必具官以从。请具左右司马\footnote{text}。”

定公曰:“诺。”具左右司马。

会齐侯夹谷,为坛位,土阶三等\footnote{text},以会遇之礼相见\footnote{text},揖让而登。献酬之礼毕\footnote{text},齐有司趋而进曰:“请奏四方之乐\footnote{text}。”

景公曰:“诺。”于是旍旄羽袚矛戟剑拨鼓噪而至\footnote{text}。

孔子趋而进,历阶而登\footnote{text},不尽一等\footnote{text},举袂而言\footnote{text}曰:“吾两君为好会,夷狄之乐何为于此!请命有司!”

有司却之\footnote{text},不去,则左右视晏子与景公\footnote{text}。景公心怍\footnote{text},麾而去之。有顷,齐有司趋而进曰:“请奏宫中之乐。”

景公曰:“诺。”优倡侏儒为戏而前\footnote{text}。

孔子趋而进,历阶而登,不尽一等,曰:“匹夫而营惑诸侯者罪当诛\footnote{text}!请命有司!”

有司加法焉,手足异处\footnote{text}。景公惧而动,知义不若,归而大恐,告其群臣曰:“鲁以君子之道辅其君,而子独以夷狄之道教寡人,使得罪于鲁君,为之奈何?”

有司进对曰:“君子有过则谢以质\footnote{text},小人有过则谢以文\footnote{text}。君若悼之\footnote{text},则谢以质。”于是齐侯乃归所侵鲁之郓、汶阳、龟阴之田以谢过\footnote{text}。
\end{yuanwen}


\begin{yuanwen}
定公十三年夏,孔子言于定公曰:“臣无藏甲,大夫毋百雉之城。”使仲由为季氏宰,将堕三都。于是叔孙氏先堕郈。季氏将堕费,公山不狃、叔孙辄率费人袭鲁。公与三子入于季氏之宫,登武子之台。费人攻之,弗克,入及公侧。孔子命申句须、乐颀下伐之,费人北。国人追之,败诸姑蔑。二子奔齐,遂堕费。将堕成,公敛处父谓孟孙曰:“堕成,齐人必至于北门。且成,孟氏之保鄣,无成是无孟氏也。我将弗堕。”十二月,公围成,弗克。
\end{yuanwen}


\begin{yuanwen}
定公十四年\footnote{text},孔子年五十六,由大司寇行摄相事\footnote{text},有喜色。门人曰:“闻‘君子祸至不惧,福至不喜。’”

孔子曰:“有是言也。不曰‘乐其以贵下人’乎?”

于是诛鲁大夫乱政者少正卯\footnote{text}。与闻国政三月,粥羔豚者弗饰贾\footnote{text},男女行者别于涂\footnote{text},涂不拾遗。四方之客至乎邑者不求有司,皆予之以归。
\end{yuanwen}


\begin{yuanwen}
齐人闻而惧,曰:“孔子为政必霸,霸则吾地近焉,我之为先并矣。盍致地焉\footnote{text}?”

黎鉏曰:“请先尝沮\footnote{text}之,沮之而不可则致地,庸迟乎!”

于是选齐国中女子好者八十人,皆衣文衣而舞《康乐\footnote{text}》,文马三十驷\footnote{text},遗鲁君。陈女乐文马于鲁城南高门外\footnote{text}。季桓子微服往观再三,将受,乃语鲁君为周道游\footnote{text},往观终日,怠于政事\footnote{text}。

子路曰:“夫子可以行矣。”

孔子曰:“鲁今且郊\footnote{text},如致膰乎大夫\footnote{text},则吾犹可以止。”

桓子卒受齐女乐,三日不听政;郊,又不致膰俎于大夫,孔子遂行,宿乎屯\footnote{text}。而师己送,曰:“夫子则非罪。”

孔子曰:“吾歌可夫?”歌曰:“彼妇之口,可以出走;彼妇之谒\footnote{text},可以死败。盖优哉游哉,维以卒岁!”

师己反,桓子曰:“孔子亦何言?”

师己以实告。桓子喟然叹曰:“夫子罪我以群婢故也夫!”
\end{yuanwen}


\begin{yuanwen}
孔子遂適卫,主于子路妻兄颜浊邹家。卫灵公问孔子:“居鲁得禄几何?”对曰:“奉粟六万。”卫人亦致粟六万。居顷之,或谮孔子于卫灵公。灵公使公孙余假一出一入。孔子恐获罪焉,居十月,去卫。
\end{yuanwen}


\begin{yuanwen}
将适陈\footnote{text},过匡\footnote{text},颜刻为仆\footnote{text},以其策指之曰:“昔吾入此,由彼缺也。”

匡人闻之,以为鲁之阳虎。阳虎尝暴匡人\footnote{text},匡人于是遂止孔子。孔子状类阳虎,拘焉五日。颜渊后\footnote{text},子曰:“吾以汝为死矣。”

颜渊曰:“子在,回何敢死!”

匡人拘孔子益急,弟子惧。

孔子曰:“文王既没,文不在兹乎?天之将丧斯文也,后死者不得与于斯文\footnote{text}也。天之未丧斯文也,匡人其如予何!”孔子使从者为宁武子臣于卫,然后得去\footnote{text}。

去即过蒲。月馀,反乎卫,主蘧伯玉家。
\end{yuanwen}


\begin{yuanwen}
灵公夫人有南子\footnote{text}者,使人谓孔子曰:“四方之君子不辱欲与寡君为兄弟\footnote{text}者,必见寡小君\footnote{text}。寡小君愿见。”

孔子辞谢,不得已而见之。夫人在絺帷\footnote{text}中。孔子入门,北面稽首\footnote{text}。夫人自帷中再拜,环珮玉声璆然\footnote{text}。

孔子曰:“吾乡为弗见\footnote{text},见之礼答焉。”

子路不说,孔子矢之曰\footnote{text}:“予所不者\footnote{text},天厌之!天厌之!”

居卫月馀,灵公与夫人同车,宦者雍渠参乘\footnote{text},出,使孔子为次乘\footnote{text},招摇巿过之\footnote{text}。

孔子曰:“吾未见好德如好色者也\footnote{text}。”于是丑之,去卫,过曹\footnote{text}。是岁,鲁定公卒\footnote{text}。
\end{yuanwen}


\begin{yuanwen}
孔子去曹适宋\footnote{text},与弟子习礼大树下。宋司马桓魋欲杀孔子\footnote{text},拔其树,孔子去\footnote{text}。弟子曰:“可以速矣。”

孔子曰:“天生德于予,桓魋其如予何!”
\end{yuanwen}


\begin{yuanwen}
孔子适郑\footnote{text},与弟子相失,孔子独立郭东门。郑人或谓子贡\footnote{text}曰:“东门有人,其颡似尧\footnote{text},其项类皋陶,其肩类子产\footnote{text},然自要以下不及禹三寸,累累若丧家之狗\footnote{text}。”

子贡以实告孔子。孔子欣然笑曰:“形状,末\footnote{text}也。而谓似丧家之狗,然\footnote{text}哉!然哉!”
\end{yuanwen}


\begin{yuanwen}
孔子遂至陈,主于司城贞子家。岁馀,吴王夫差伐陈,取三邑而去。赵鞅伐朝歌。楚围蔡,蔡迁于吴。吴败越王句践会稽。

有隼集于陈廷而死,楛矢贯之,石砮,矢长尺有咫。陈湣公使使问仲尼。仲尼曰:“隼来远矣,此肃慎之矢也。昔武王克商,通道九夷百蛮,使各以其方贿来贡,使无忘职业。于是肃慎贡楛矢石砮,长尺有咫。先王欲昭其令德,以肃慎矢分大姬,配虞胡公而封诸陈。分同姓以珍玉,展亲;分异姓以远职,使无忘服。故分陈以肃慎矢。”试求之故府,果得之。

孔子居陈三岁,会晋楚争彊,更伐陈,及吴侵陈,陈常被寇。孔子曰:“归与归与!吾党之小子狂简,进取不忘其初。”于是孔子去陈。

过蒲,会公叔氏以蒲畔,蒲人止孔子。弟子有公良孺者,以私车五乘从孔子。其为人长贤,有勇力,谓曰:“吾昔从夫子遇难于匡,今又遇难于此,命也已。吾与夫子再罹难,宁斗而死。”斗甚疾。蒲人惧,谓孔子曰:“苟毋適卫,吾出子。”与之盟,出孔子东门。孔子遂適卫。子贡曰:“盟可负邪?”孔子曰:“要盟也,神不听。”

卫灵公闻孔子来,喜,郊迎。问曰:“蒲可伐乎?”对曰:“可。”灵公曰:“吾大夫以为不可。今蒲,卫之所以待晋楚也,以卫伐之,无乃不可乎?”孔子曰:“其男子有死之志,妇人有保西河之志。吾所伐者不过四五人。”灵公曰:“善。”然不伐蒲。

灵公老,怠于政,不用孔子。孔子喟然叹曰:“苟有用我者,期月而已,三年有成。”孔子行。

佛肸为中牟宰。赵简子攻范、中行,伐中牟。佛肸畔,使人召孔子。孔子欲往。子路曰:“由闻诸夫子,“其身亲为不善者,君子不入也”。今佛肸亲以中牟畔,子欲往,如之何?”孔子曰:“有是言也。不曰坚乎,磨而不磷;不曰白乎,涅而不淄。我岂匏瓜也哉,焉能系而不食?”

孔子击磬。有荷蒉而过门者,曰:“有心哉,击磬乎!硜々乎,莫己知也夫而已矣!”

孔子学鼓琴师襄子,十日不进。师襄子曰:“可以益矣。”孔子曰:“丘已习其曲矣,未得其数也。”有间,曰:“已习其数,可以益矣。”孔子曰:“丘未得其志也。”有间,曰:“已习其志,可以益矣。”孔子曰:“丘未得其为人也。”有间,有所穆然深思焉,有所怡然高望而远志焉。曰:“丘得其为人,黯然而黑,几然而长,眼如望羊,如王四国,非文王其谁能为此也!”师襄子辟席再拜,曰:“师盖云文王操也。”

孔子既不得用于卫,将西见赵简子。至于河而闻窦鸣犊、舜华之死也,临河而叹曰:“美哉水,洋洋乎!丘之不济此,命也夫!”子贡趋而进曰:“敢问何谓也?”孔子曰:“窦鸣犊,舜华,晋国之贤大夫也。赵简子未得志之时,须此两人而后从政;及其已得志,杀之乃从政。丘闻之也,刳胎杀夭则麒麟不至郊,竭泽涸渔则蛟龙不合阴阳,覆巢毁卵则凤皇不翔。何则?君子讳伤其类也。夫鸟兽之于不义也尚知辟之,而况乎丘哉!”乃还息乎陬乡,作为陬操以哀之。而反乎卫,入主蘧伯玉家。

他日,灵公问兵陈。孔子曰:“俎豆之事则尝闻之,军旅之事未之学也。”明日,与孔子语,见蜚雁,仰视之,色不在孔子。孔子遂行,复如陈。

夏,卫灵公卒,立孙辄,是为卫出公。六月,赵鞅内太子蒯聩于戚。阳虎使太子絻,八人衰绖,伪自卫迎者,哭而入,遂居焉。冬,蔡迁于州来。是岁鲁哀公三年,而孔子年六十矣。齐助卫围戚,以卫太子蒯聩在故也。

夏,鲁桓釐庙燔,南宫敬叔救火。孔子在陈,闻之,曰:“灾必于桓釐庙乎?”已而果然。
\end{yuanwen}


\begin{yuanwen}
秋,季桓子病,辇而见鲁城\footnote{text},喟然叹曰:“昔此国几兴矣,以吾获罪于孔子\footnote{text},故不兴\footnote{text}也。”

顾谓其嗣康子曰:“我即死,若必相鲁;相鲁,必召仲尼。”

后数日,桓子卒,康子代立。已葬,欲召仲尼。

公之鱼\footnote{text}曰:“昔吾先君用之不终,终为诸侯笑。今又用之,不能终,是再为诸侯笑。”

康子曰:“则谁召而可?”

曰:“必召冉求\footnote{text}。”于是使使召冉求。

冉求将行,孔子曰:“鲁人召求,非小用之,将大用之也。”

是日,孔子曰:“归乎归乎!吾党之小子狂简,斐然成章\footnote{text},吾不知所以裁之\footnote{text}。”

子赣知孔子思归\footnote{text},送冉求,因诫曰“即用,以孔子为招”云。
\end{yuanwen}


\begin{yuanwen}
冉求既去,明年,孔子自陈迁于蔡。蔡昭公将如吴,吴召之也。前昭公欺其臣迁州来,后将往,大夫惧复迁,公孙翩射杀昭公。楚侵蔡。秋,齐景公卒。

明年,孔子自蔡如叶。叶公问政,孔子曰:“政在来远附迩。”他日,叶公问孔子于子路,子路不对。孔子闻之,曰:“由,尔何不对曰“其为人也,学道不倦,诲人不厌,发愤忘食,乐以忘忧,不知老之将至”云尔。”

去叶,反于蔡。长沮、桀溺耦而耕,孔子以为隐者,使子路问津焉。长沮曰:“彼执舆者为谁?”子路曰:“为孔丘。”曰:“是鲁孔丘与?”曰:“然。”曰:“是知津矣。”桀溺谓子路曰:“子为谁?”曰:“为仲由。”曰:“子,孔丘之徒与?”曰:“然。”桀溺曰:“悠悠者天下皆是也,而谁以易之?且与其从辟人之士,岂若从辟世之士哉!”櫌而不辍。子路以告孔子,孔子怃然曰:“鸟兽不可与同群。天下有道,丘不与易也。”

他日,子路行,遇荷丈人,曰:“子见夫子乎?”丈人曰:“四体不勤,五穀不分,孰为夫子!”植其杖而芸。子路以告,孔子曰:“隐者也。”复往,则亡。
\end{yuanwen}


\begin{yuanwen}
孔子迁于蔡三岁\footnote{text},吴伐陈。楚救陈,军于城父\footnote{text}。闻孔子在陈、蔡之间,楚使人聘孔子\footnote{text}。孔子将往拜礼\footnote{text},陈、蔡大夫谋曰:“孔子贤者,所刺讥皆中诸侯之疾。今者久留陈、蔡之间,诸大夫所设行皆非仲尼之意\footnote{text}。今楚,大国也,来聘孔子。孔子用于楚,则陈、蔡用事大夫危矣。”于是乃相与发徒役围孔子于野\footnote{text}。不得行,绝粮。从者病,莫能兴\footnote{text}。孔子讲诵弦歌不衰。子路愠见曰\footnote{text}:“君子亦有穷乎?”

孔子曰:“君子固穷,小人穷斯滥矣\footnote{text}。”
\end{yuanwen}


\begin{yuanwen}
子贡色作。孔子曰:“赐,尔以予为多学而识之者与\footnote{text}?”

曰:“然。非与?”

孔子曰:“非也。予一以贯之。”
\end{yuanwen}


\begin{yuanwen}
孔子知弟子有愠心,乃召子路而问曰:“《诗》云‘匪兕匪虎,率彼旷野’\footnote{text}。吾道非邪?吾何为于此?”

子路曰:“意者吾未仁邪\footnote{text}?人之不我信也。意者吾未知邪\footnote{text}?人之不我行也。”

孔子曰:“有是乎!由,譬使仁者而必信\footnote{text},安有伯夷、叔齐?使知者而必行,安有王子比干?”
\end{yuanwen}


\begin{yuanwen}
子路出,子贡入见。孔子曰:“赐,《诗》云‘匪兕匪虎,率彼旷野’。吾道非邪?吾何为于此?”

子贡曰:“夫子之道至大也,故天下莫能容夫子。夫子盖少贬焉\footnote{text}?”

孔子曰:“赐,良农能稼而不能为穑\footnote{text},良工能巧而不能为顺\footnote{text}。君子能修其道,纲而纪之,统而理之,而不能为容\footnote{text}。今尔不修尔道而求为容。赐,而志不远矣!”
\end{yuanwen}


\begin{yuanwen}
子贡出,颜回入见。孔子曰:“回,《诗》云‘匪兕匪虎,率彼旷野’。吾道非邪?吾何为于此?”

颜回曰:“夫子之道至大,故天下莫能容。虽然,夫子推而行之。不容何病\footnote{text},不容然后见君子!夫道之不修也,是吾丑也。夫道既已大修而不用,是有国者之丑也。不容何病,不容然后见君子!”

孔子欣然而笑曰:“有是哉颜氏之子\footnote{text}!使尔多财,吾为尔宰\footnote{text}。”
\end{yuanwen}


\begin{yuanwen}
于是使子贡至楚。楚昭王兴师迎孔子\footnote{text},然后得免。
\end{yuanwen}


\begin{yuanwen}
昭王将以书社地七百里封孔子。楚令尹子西曰:“王之使使诸侯有如子贡者乎?”曰:“无有。”“王之辅相有如颜回者乎?”曰:“无有。”“王之将率有如子路者乎?”曰:“无有。”“王之官尹有如宰予者乎?”曰:“无有。”“且楚之祖封于周,号为子男五十里。今孔丘述三五之法,明周召之业,王若用之,则楚安得世世堂堂方数千里乎?夫文王在丰,武王在镐,百里之君卒王天下。今孔丘得据土壤,贤弟子为佐,非楚之福也。”昭王乃止。其秋,楚昭王卒于城父。

楚狂接舆歌而过孔子,曰:“凤兮凤兮,何德之衰!往者不可谏兮,来者犹可追也!已而已而,今之从政者殆而!”孔子下,欲与之言。趋而去,弗得与之言。

于是孔子自楚反乎卫。是岁也,孔子年六十三,而鲁哀公六年也。

其明年,吴与鲁会缯,徵百牢。太宰嚭召季康子。康子使子贡往,然后得已。

孔子曰:“鲁卫之政,兄弟也。”是时,卫君辄父不得立,在外,诸侯数以为让。而孔子弟子多仕于卫,卫君欲得孔子为政。子路曰:“卫君待子而为政,子将奚先?”孔子曰:“必也正名乎!”子路曰:“有是哉,子之迂也!何其正也?”孔子曰:“野哉由也!夫名不正则言不顺,言不顺则事不成,事不成则礼乐不兴,礼乐不兴则刑罚不中,刑罚不中则民无所错手足矣。夫君子为之必可名,言之必可行。君子于其言,无所苟而已矣。”
\end{yuanwen}


\begin{yuanwen}
其明年\footnote{text},冉有为季氏将师,与齐战于郎,克之。季康子曰:“子之于军旅,学之乎?性之乎\footnote{text}?”

冉有曰:“学之于孔子。”

季康子曰:“孔子何如人哉\footnote{text}?”

对曰:“用之有名;播之百姓,质诸鬼神而无憾\footnote{text}。求之至于此道,虽累千社\footnote{text},夫子不利也。”

康子曰:“我欲召之,可乎?”

对曰:“欲召之,则毋以小人固之\footnote{text},则可矣。”

而卫孔文子将攻太叔,问策于仲尼。仲尼辞不知,退而命载而行\footnote{text},曰:“鸟能择木,木岂能择鸟乎!”

文子固止。会季康子逐公华、公宾、公林,以币迎孔子\footnote{text},孔子归鲁。
\end{yuanwen}


\begin{yuanwen}
孔子之去鲁凡十四岁而反乎鲁。
\end{yuanwen}

孔子离开鲁国十四年后才又回到鲁国。

\begin{yuanwen}
鲁哀公问政,对曰:“政在选臣。”季康子问政,曰:“举直错诸枉,则枉者直。”康子患盗,孔子曰:“苟子之不欲,虽赏之不窃。”然鲁终不能用孔子,孔子亦不求仕。
\end{yuanwen}


\begin{yuanwen}
孔子之时,周室微而礼乐废,《诗》、《书》缺。追迹三代之礼\footnote{text},序《书传\footnote{text}》,上纪唐虞之际\footnote{text},下至秦缪\footnote{text},编次其事。曰:“夏礼吾能言之,杞不足徵也。殷礼吾能言之,宋不足徵也。足,则吾能徵之矣。”观殷夏所损益,曰:“后虽百世可知也,以一文一质。周监二代,郁郁乎文哉。吾从周。”故《书传》、《礼记》自孔氏\footnote{text}。
\end{yuanwen}


\begin{yuanwen}
孔子语鲁大师\footnote{text}:“乐其可知也。始作翕如\footnote{text},纵之纯如\footnote{text},皦如\footnote{text},绎如也\footnote{text},以成。”“吾自卫反鲁,然后乐正,《雅》、《颂》各得其所\footnote{text}。”
\end{yuanwen}


\begin{yuanwen}
古者《诗》三千馀篇,及至孔子,去其重,取可施于礼义\footnote{text},上采契、后稷\footnote{text},中述殷、周之盛\footnote{text},至幽、厉之缺\footnote{text},始于衽席\footnote{text},故曰“《关雎》之乱以为《风》始\footnote{text},《鹿鸣》为《小雅》始\footnote{text},《文王》为《大雅》始\footnote{text},《清庙》为《颂》始\footnote{text}”。三百五篇孔子皆弦歌之\footnote{text},以求合《韶》、《武》、《雅》、《颂》之音\footnote{text}。礼乐自此可得而述,以备王道\footnote{text},成六艺\footnote{text}。
\end{yuanwen}


\begin{yuanwen}
孔子晚而喜《易\footnote{text}》,序《彖》、《系》、《象》、《说卦》、《文言\footnote{text}》。读《易》,韦编三绝\footnote{text}。曰:“假我数年,若是,我于《易》则彬彬矣\footnote{text}。”
\end{yuanwen}


\begin{yuanwen}
孔子以《诗》、《书》、《礼》、《乐》教,弟子盖三千焉\footnote{text},身通六艺者七十有二人。如颜浊邹之徒,颇受业者甚众。
\end{yuanwen}


\begin{yuanwen}
孔子以四教:文,行,忠,信。绝四:毋意,毋必,毋固,毋我。所慎:齐,战,疾。子罕言利与命与仁。不愤不启,举一隅不以三隅反,则弗复也。

其于乡党,恂恂似不能言者。其于宗庙朝廷,辩辩言,唯谨尔。朝,与上大夫言,訚訚如也;与下大夫言,侃侃如也。

入公门,鞠躬如也;趋进,翼如也。君召使儐,色勃如也。君命召,不俟驾行矣。

鱼馁,肉败,割不正,不食。席不正,不坐。食于有丧者之侧,未尝饱也。

是日哭,则不歌。见齐衰、瞽者,虽童子必变。

“三人行,必得我师。”“德之不脩,学之不讲,闻义不能徙,不善不能改,是吾忧也。”使人歌,善,则使复之,然后和之。

子不语:怪,力,乱,神。

子贡曰:“夫子之文章,可得闻也。夫子言天道与性命,弗可得闻也已。”颜渊喟然叹曰:“仰之弥高,钻之弥坚。瞻之在前,忽焉在后。夫子循循然善诱人,博我以文,约我以礼,欲罢不能。既竭我才,如有所立,卓尔。虽欲从之,蔑由也已。”达巷党人曰:“大哉孔子,博学而无所成名。”子闻之曰:“我何执?执御乎?执射乎?我执御矣。”牢曰:“子云“不试,故艺”。”
\end{yuanwen}


\begin{yuanwen}
鲁哀公十四年春\footnote{text},狩大野\footnote{text}。叔孙氏车子鉏商获兽\footnote{text},以为不祥。仲尼视之,曰:“麟也。”

取之。曰:“河不出图,雒不出书,吾已矣夫\footnote{text}!”

颜渊死,孔子曰:“天丧予\footnote{text}!”

及西狩见麟,曰:“吾道穷矣!”
\end{yuanwen}


\begin{yuanwen}
喟然叹曰:“莫知我夫!”子贡曰:“何为莫知子?”子曰:“不怨天,不尤人,下学而上达,知我者其天乎!”

“不降其志,不辱其身,伯夷、叔齐乎!”谓“柳下惠、少连降志辱身矣”。谓“虞仲、夷逸隐居放言,行中清,废中权”。“我则异于是,无可无不可。”
\end{yuanwen}


\begin{yuanwen}
子曰:“弗乎弗乎,君子病没世而名不称焉\footnote{text}。吾道不行矣,吾何以自见于后世哉\footnote{text}?”

乃因史记作春秋\footnote{text},上至隐公\footnote{text},下讫哀公十四年\footnote{text},十二公。据鲁\footnote{text},亲周\footnote{text},故殷\footnote{text},运之三代\footnote{text}。约其文辞而指博\footnote{text}。故吴、楚之君自称王,而《春秋》贬之曰“子”;践土之会实召周天子,而《春秋》讳之曰“天王狩于河阳”\footnote{text}:推此类以绳当世\footnote{text}。贬损之义,后有王者举而开之\footnote{text}。《春秋》之义行,则天下乱臣贼子惧焉\footnote{text}。
\end{yuanwen}


\begin{yuanwen}
孔子在位听讼\footnote{text},文辞有可与人共者,弗独有也。至于为《春秋》,笔则笔\footnote{text},削则削,子夏之徒不能赞一辞\footnote{text}。弟子受《春秋》,孔子曰:“后世知丘者以《春秋》,而罪丘者亦以《春秋》。”
\end{yuanwen}


\begin{yuanwen}
明岁\footnote{text},子路死于卫\footnote{text}。孔子病,子贡请见。孔子方负杖逍遥于门\footnote{text},曰:“赐,汝来何其晚也?”

孔子因叹,歌曰:“太山坏乎\footnote{text}!梁柱摧乎!哲人萎乎\footnote{text}!”

因以涕下。谓子贡曰:“天下无道久矣,莫能宗予\footnote{text}。夏人殡于东阶\footnote{text},周人于西阶,殷人两柱间。昨暮予梦坐奠两柱之间,予始殷人也。”后七日卒。
\end{yuanwen}


\begin{yuanwen}
孔子年七十三,以鲁哀公十六年四月己丑卒\footnote{text}。
\end{yuanwen}


\begin{yuanwen}
哀公诔之曰:“旻天不吊,不玦遗一老,俾屏余一人以在位,茕茕余在疚。呜呼哀哉!尼父,毋自律!”子贡曰:“君其不没于鲁乎!夫子之言曰:“礼失则昏,名失则愆。失志为昏,失所为愆。”生不能用,死而诔之,非礼也。称“余一人”,非名也。”
\end{yuanwen}


\begin{yuanwen}
孔子葬鲁城北泗上,弟子皆服三年。三年心丧毕,相诀而去,则哭,各复尽哀;或复留。唯子赣庐于冢上,凡六年,然后去。弟子及鲁人往从冢而家者百有馀室,因命曰孔里\footnote{text}。鲁世世相传以岁时奉祠孔子冢\footnote{text},而诸儒亦讲礼乡饮、大射于孔子冢\footnote{text}。孔子冢大一顷\footnote{text}。故所居堂、弟子内\footnote{text},后世因庙,藏孔子衣冠琴车书\footnote{text},至于汉二百馀年不绝。高皇帝过鲁\footnote{text},以太牢祠焉。诸侯卿相至\footnote{text},常先谒然后从政。
\end{yuanwen}


\begin{yuanwen}
孔子生鲤,字伯鱼。伯鱼年五十,先孔子死。

伯鱼生伋,字子思,年六十二。尝困于宋。子思作中庸。

子思生白,字子上,年四十七。子上生求,字子家,年四十五。子家生箕,字子京,年四十六。子京生穿,字子高,年五十一。子高生子慎,年五十七,尝为魏相。

子慎生鲋,年五十七,为陈王涉博士,死于陈下。

鲋弟子襄,年五十七。尝为孝惠皇帝博士,迁为长沙太守。长九尺六寸。

子襄生忠,年五十七。忠生武,武生延年及安国。安国为今皇帝博士,至临淮太守,蚤卒。安国生卬,卬生驩。
\end{yuanwen}


\begin{yuanwen}
太史公曰:诗有之:“高山仰止,景行行止\footnote{text}。”虽不能至,然心乡往之。余读孔氏书,想见其为人。适鲁,观仲尼庙堂车服礼器,诸生以时习礼其家,余祗回留之不能去云\footnote{text}。天下君王至于贤人众矣,当时则荣,没则已焉。孔子布衣,传十馀世,学者宗之。自天子王侯,中国言六艺者折中于夫子\footnote{text},可谓至圣矣!
\end{yuanwen}


\begin{yuanwen}
孔子之胄,出于商国。弗父能让,正考铭勒。防叔来奔,邹人掎足。尼丘诞圣,阙里生德。七十升堂,四方取则。卯诛两观,摄相夹谷。歌凤遽衰,泣麟何促!九流仰镜,万古钦躅。
\end{yuanwen}

\part{卷四十八}

\chapter{陈涉世家第十八}

司马迁为陈涉所领导的整支农民反秦起义军所立的传记,系统、全面地描写了这支起义军由发动起义、蓬勃发展、战绩辉煌到最后失败的全过程,是我国第一场伟大农民战争的忠实记录,诸如起义的原因,反秦的声势,以及早期农民战争的种种弱点,和它失败的历史教训,无不包涵其中。

杨慎:「既叙陈涉发难之颠末,又原其所以败之故,而申言之,叙事之法也。」

郝敬:「陈涉举事不效,身死族灭,亦为《世家》;项羽图王不成,亦为《本纪》;盖二人以匹夫起义,为民取残,为六王报怨,无论成败,皆足以不朽。英雄利钝有时,作史者扬励,慰人心一块耳。」朱东润:「陈涉首难,为陈王六月而死,不足以系天下之重,故不得先项羽而为本纪。然无功于汉而不入列传者,汉室之兴,由涉始也。」

\begin{yuanwen}
陈胜者,阳城人也,字涉。吴广者,阳夏人也,字叔。陈涉少时,尝与人佣耕\footnote{text},辍耕之垄上\footnote{text},怅恨久之,曰:“苟富贵,无相忘。”

庸\footnote{同“佣”,被雇用的人。}者笑而应曰:“若为庸耕,何富贵也?”

陈涉太息曰:“嗟乎,燕雀安知鸿鹄之志哉\footnote{text}!”
\end{yuanwen}

陈胜,是阳城人,字涉。吴广,是阳夏人,字叔。陈涉年少的时候,曾经和别人一起被雇佣耕田。一次陈涉耕田时停了下来,来到田埂上休息,愤恨不平了很长时间,说:“如果有一天富贵了,可不要相互忘记。”

一同受雇的耕者笑着回答他说:“你不过是个受雇耕田的,能有什么富贵呢?”

陈涉叹息着说:“唉!燕雀怎么会知晓鸿鹄的志向呢?”

\begin{yuanwen}
二世元年七月\footnote{text},发闾左(適/敵)\footnote{同“谪”,因有罪被发遣。}戍渔阳,九百人屯大泽乡\footnote{text}。陈胜、吴广皆次当行\footnote{text},为屯长\footnote{text}。会天大雨\footnote{text},道不通,度已失期\footnote{text}。失期,法皆斩。

陈胜、吴广乃谋曰:“今亡亦死,举大计亦死,等死,死国可乎\footnote{text}?”

陈胜曰:“天下苦秦久矣\footnote{text}。吾闻二世少子也\footnote{text},不当立,当立者乃公子扶苏\footnote{text}。扶苏以数谏故,上使外将兵\footnote{text}。今或闻无罪,二世杀之\footnote{text}。百姓多闻其贤,未知其死也。项燕为楚将\footnote{text},数有功,爱士卒,楚人怜之。或以为死,或以为亡。今诚以吾众诈自称公子扶苏、项燕,为天下唱\footnote{同“倡”,倡导,号召。},宜多应者。”

吴广以为然,乃行卜。卜者知其指意\footnote{text},曰:“足下事皆成,有功。然足下卜之鬼乎\footnote{text}!”

陈胜、吴广喜,念鬼\footnote{text},曰:“此教我先威众耳。”乃丹书帛曰“陈胜王”,置人所罾鱼腹中\footnote{text}。卒买鱼烹食,得鱼腹中书,固以\footnote{通“已”。}怪之矣\footnote{text}。又间令吴广之次所旁丛祠中\footnote{text},夜篝火\footnote{text},狐鸣呼曰“大楚兴,陈胜王”。卒皆夜惊恐。旦日\footnote{text},卒中往往语,皆指目陈胜\footnote{text}。
\end{yuanwen}

秦二世元年(前209年)七月,征发闾巷左侧的贫民九百人去守卫渔阳,停驻在大泽乡。陈胜、吴广都被编进队伍里,担任屯长。正遇天降大雨,道路不通,估计着已经误了到达的期限。耽误期限,根据秦朝法律就都会被斩首。

陈胜、吴广于是商议说:“如今逃走也是死,举行起义干一番大事业也是死,一样是死,为国家举大事而死怎么样?”

陈胜说:“天下百姓苦于秦朝的暴政已经很长时间了。我听说二世是始皇的小儿子,不应该即位,应该即位的是公子扶苏。扶苏由于数次劝谏始皇的缘故,始皇就派他到外地统领军队。如今有人听说他并没犯罪,二世就杀掉了他。百姓大多听说过他的贤能,并不知道他已经死了。项燕作为楚国的将军,多次立有战功,爱护手下士卒,楚国人都很爱戴他。有的人认为他死了,有的人认为他逃走了。如今我们这些人谎称是公子扶苏、项燕,成为天下反秦的倡导者,应该会有许多人响应。”

吴广认为他说得对。于是进行占卜。为他们占卜的人知道他们的想法,说:“先生要做的事情都能成功,建立大功业。但先生何不把这件事向鬼神卜问呢!”

陈胜、吴广非常高兴,思量着向鬼神卜问是什么意思,说:“这是让我先在群众中树立威信。”就用丹砂在绸上写上“陈胜王”,放入别人用网捕捉到的鱼肚子里。士兵们买回那条鱼煮着吃,得到鱼肚子里的帛书,自然就觉得奇怪了。陈胜又暗中让吴广前往戍卒驻地附近丛林中的神庙里,在夜间点起篝火,学着狐狸的叫声说“大楚兴,陈胜王”。士兵们都在夜里惊惧恐慌。第二天早上,士兵们到处议论纷纷,暗中指点、目视陈胜。

\begin{yuanwen}
吴广素爱人,士卒多为用者。将尉醉\footnote{text},广故数言欲亡\footnote{text},忿恚尉\footnote{text},令辱之,以激怒其众\footnote{text}。尉果笞广\footnote{text}。尉剑挺\footnote{text},广起,夺而杀尉。陈胜佐之,并杀两尉。召令徒属曰:“公等遇雨,皆已失期,失期当斩。藉弟令毋斩\footnote{text},而戍死者固十六七\footnote{text}。且壮士不死即已\footnote{text},死即举大名耳\footnote{text},王侯将相宁有种乎!”

徒属皆曰:“敬受命。”乃诈称公子扶苏、项燕,从民欲也\footnote{text}。袒右\footnote{text},称大楚。为坛而盟,祭以尉首\footnote{text}。陈胜自立为将军,吴广为都尉\footnote{text}。攻大泽乡,收而攻蕲\footnote{text}。蕲下,乃令符离人葛婴将兵徇蕲以东\footnote{text}。攻铚、酂、苦、柘、谯\footnote{text},皆下之。行收兵\footnote{text},比至陈\footnote{text},车六七百乘,骑千馀,卒数万人。攻陈,陈守令皆不在\footnote{text},独守丞与战谯门中\footnote{text}。弗胜,守丞死,乃入据陈。数日,号令召三老、豪杰与皆来会计事\footnote{text}。三老、豪杰皆曰:“将军身被坚执锐\footnote{text},伐无道,诛暴秦,复立楚国之社稷\footnote{text},功宜为王。”

陈涉乃立为王,号为张楚\footnote{text}。
\end{yuanwen}


\begin{yuanwen}
当此时,诸郡县苦秦吏者,皆刑其长吏,杀之以应陈涉。乃以吴叔为假王\footnote{text},监诸将以西击荥阳\footnote{text}。令陈人武臣、张耳、陈馀徇赵地,令汝阴人邓宗徇九江郡\footnote{text}。当此时,楚兵数千人为聚者,不可胜数。
\end{yuanwen}


\begin{yuanwen}
葛婴至东城,立襄彊为楚王。婴后闻陈王已立,因杀襄彊,还报。至陈,陈王诛杀葛婴。陈王令魏人周市北徇魏地。吴广围荥阳。李由为三川守,守荥阳,吴叔弗能下。陈王徵国之豪杰与计,以上蔡人房君蔡赐为上柱国。

周文,陈之贤人也,尝为项燕军视日,事春申君,自言习兵,陈王与之将军印,西击秦。行收兵至关,车千乘,卒数十万,至戏,军焉。秦令少府章邯免郦山徒、人奴产子生,悉发以击楚大军,尽败之。周文败,走出关,止次曹阳二三月。章邯追败之,复走次渑池十馀日。章邯击,大破之。周文自刭,军遂不战。

武臣到邯郸,自立为赵王,陈馀为大将军,张耳、召骚为左右丞相。陈王怒,捕系武臣等家室,欲诛之。柱国曰:“秦未亡而诛赵王将相家属,此生一秦也。不如因而立之。”陈王乃遣使者贺赵,而徙系武臣等家属宫中,而封耳子张敖为成都君,趣赵兵亟入关。赵王将相相与谋曰:“王王赵,非楚意也。楚已诛秦,必加兵于赵。计莫如毋西兵,使使北徇燕地以自广也。赵南据大河,北有燕、代,楚虽胜秦,不敢制赵。若楚不胜秦,必重赵。赵乘秦之弊,可以得志于天下。”赵王以为然,因不西兵,而遣故上谷卒史韩广将兵北徇燕地。

燕故贵人豪杰谓韩广曰:“楚已立王,赵又已立王。燕虽小,亦万乘之国也,原将军立为燕王。”韩广曰:“广母在赵,不可。”燕人曰:“赵方西忧秦,南忧楚,其力不能禁我。且以楚之彊,不敢害赵王将相之家,赵独安敢害将军之家!”韩广以为然,乃自立为燕王。居数月,赵奉燕王母及家属归之燕。

当此之时,诸将之徇地者,不可胜数。周市北徇地至狄,狄人田儋杀狄令,自立为齐王,以齐反击周市。市军散,还至魏地,欲立魏后故甯陵君咎为魏王。时咎在陈王所,不得之魏。魏地已定,欲相与立周市为魏王,周市不肯。使者五反,陈王乃立甯陵君咎为魏王,遣之国。周市卒为相。

将军田臧等相与谋曰:“周章军已破矣,秦兵旦暮至,我围荥阳城弗能下,秦军至,必大败。不如少遗兵,足以守荥阳,悉精兵迎秦军。今假王骄,不知兵权,不可与计,非诛之,事恐败。”因相与矫王令以诛吴叔,献其首于陈王。陈王使使赐田臧楚令尹印,使为上将。田臧乃使诸将李归等守荥阳城,自以精兵西迎秦军于敖仓。与战,田臧死,军破。章邯进兵击李归等荥阳下,破之,李归等死。

阳城人邓说将兵居郯,章邯别将击破之,邓说军散走陈。铚人伍徐将兵居许,章邯击破之,伍徐军皆散走陈。陈王诛邓说。

陈王初立时,陵人秦嘉、铚人董緤、符离人硃鸡石、取虑人郑布、徐人丁疾等皆特起,将兵围东海守庆于郯。陈王闻,乃使武平君畔为将军,监郯下军。秦嘉不受命,嘉自立为大司马,恶属武平君。告军吏曰:“武平君年少,不知兵事,勿听!”因矫以王命杀武平君畔。

章邯已破伍徐,击陈,柱国房君死。章邯又进兵击陈西张贺军。陈王出监战,军破,张贺死。

腊月,陈王之汝阴,还至下城父,其御庄贾杀以降秦。陈胜葬砀,谥曰隐王。

陈王故涓人将军吕臣为仓头军,起新阳,攻陈下之,杀庄贾,复以陈为楚。

初,陈王至陈,令铚人宋留将兵定南阳,入武关。留已徇南阳,闻陈王死,南阳复为秦。宋留不能入武关,乃东至新蔡,遇秦军,宋留以军降秦。秦传留至咸阳,车裂留以徇。

秦嘉等闻陈王军破出走,乃立景驹为楚王,引兵之方与,欲击秦军定陶下。使公孙庆使齐王,欲与并力俱进。齐王曰:“闻陈王战败,不知其死生,楚安得不请而立王!”公孙庆曰:“齐不请楚而立王,楚何故请齐而立王!且楚首事,当令于天下。”田儋诛杀公孙庆。

秦左右校复攻陈,下之。吕将军走,收兵复聚。鄱盗当阳君黥布之兵相收,复击秦左右校,破之青波,复以陈为楚。会项梁立怀王孙心为楚王。
\end{yuanwen}


\begin{yuanwen}
陈胜王凡六月\footnote{text},已为王,王陈\footnote{text}。其故人尝与庸耕者闻之,之陈\footnote{text},扣宫门曰:“吾欲见涉。”

宫门令欲缚之\footnote{text},自辩数\footnote{text},乃置\footnote{text},不肯为通\footnote{text}。陈王出,遮道而呼涉\footnote{text}。陈王闻之,乃召见,载与俱归。入宫,见殿屋帷帐,客曰:“夥颐\footnote{text}!涉之为王沉沉者\footnote{text}!”

楚人谓多为夥,故天下传之“夥涉为王”,由陈涉始\footnote{text}。客出入愈益发舒\footnote{text},言陈王故情。或说陈王曰:“客愚无知,颛妄言\footnote{text},轻威。”

陈王斩之。诸陈王故人皆自引去\footnote{text},由是无亲陈王者。陈王以朱房为中正\footnote{text},胡武为司过\footnote{text},主司群臣\footnote{text}。诸将徇地至\footnote{text},令之不是者\footnote{text},系而罪之,以苛察为忠。其所不善者\footnote{text},弗下吏\footnote{text},辄自治之\footnote{text},陈王信用之。诸将以其故不亲附,此其所以败也。
\end{yuanwen}


\begin{yuanwen}
陈胜虽已死,其所置遣侯王将相竟亡秦,由涉首事也。高祖时为陈涉置守冢三十家砀,至今血食\footnote{text}。

褚先生曰:地形险阻,所以为固也;兵革刑法,所以为治也。犹未足恃也。夫先王以仁义为本,而以固塞文法为枝叶,岂不然哉!吾闻贾生之称曰:

“秦孝公据殽函之固,拥雍州之地,君臣固守,以窥周室。有席卷天下,包举宇内,囊括四海之意,并吞八荒之心。当是时也,商君佐之,内立法度,务耕织,修守战之备;外连衡而斗诸侯。于是秦人拱手而取西河之外。

“孝公既没,惠文王、武王、昭王蒙故业,因遗策,南取汉中,西举巴蜀,东割膏腴之地,收要害之郡。诸侯恐惧,会盟而谋弱秦。不爱珍器重宝肥饶之地,以致天下之士。合从缔交,相与为一。当此之时,齐有孟尝,赵有平原,楚有春申,魏有信陵:此四君者,皆明知而忠信,
厚而爱人,尊贤而重士。约从连衡,兼韩、魏、燕、赵、宋、卫、中山之众。于是六国之士有甯越、徐尚、苏秦、杜赫之属为之谋,齐明、周勣、陈轸、邵滑、楼缓、翟景、苏厉、乐毅之徒通其意,吴起、孙膑、带他、兒良、王廖、田忌、廉颇、赵奢之伦制其兵。尝以什倍之地,百万之师,仰关而攻秦。秦人开关而延敌,九国之师遁逃而不敢进。秦无亡矢遗镞之费,而天下固已困矣。于是从散约败,争割地而赂秦。秦有馀力而制其弊,追亡逐北,伏尸百万,流血漂橹,因利乘便,宰割天下,分裂山河,彊国请服,弱国入朝。

“施及孝文王、庄襄王,享国之日浅,国家无事。

“及至始皇,奋六世之馀烈,振长策而御宇内,吞二周而亡诸侯,履至尊而制六合,执敲朴以鞭笞天下,威振四海。南取百越之地,以为桂林、象郡,百越之君俯首系颈,委命下吏。乃使蒙恬北筑长城而守籓篱,卻匈奴七百馀里,胡人不敢南下而牧马,士亦不敢贯弓而报怨。于是废先王之道,燔百家之言,以愚黔首。堕名城,杀豪俊,收天下之兵聚之咸阳,销锋鍉,铸以为金人十二,以弱天下之民。然后践华为城,因河为池,据亿丈之城,临不测之谿以为固。良将劲弩,守要害之处,信臣精卒,陈利兵而谁何。天下已定,始皇之心,自以为关中之固,金城千里,子孙帝王万世之业也。

“始皇既没,馀威振于殊俗。然而陈涉甕牖绳枢之子,甿隶之人,而迁徙之徒也。材能不及中人,非有仲尼、墨翟之贤,陶硃、猗顿之富也。蹑足行伍之间,俯仰仟佰之中,率罢散之卒,将数百之众,转而攻秦。斩木为兵,揭竿为旗,天下云会响应,赢粮而景从,山东豪俊遂并起而亡秦族矣。

“且天下非小弱也;雍州之地,殽函之固自若也。陈涉之位,非尊于齐、楚、燕、赵、韩、魏、宋、卫、中山之君也;鉏櫌棘矜,非銛于句戟长铩也;適戍之众,非俦于九国之师也;深谋远虑,行军用兵之道,非及乡时之士也。然而成败异变,功业相反也。尝试使山东之国与陈涉度长絜大,比权量力,则不可同年而语矣。然而秦以区区之地。致万乘之权,抑八州而朝同列,百有馀年矣。然后以六合为家,殽函为宫。一夫作难而七庙堕,身死人手,为天下笑者,何也?仁义不施,而攻守之势异也。”

天下匈匈,海内乏主,掎鹿争捷,瞻乌爰处。陈胜首事,厥号张楚。鬼怪是凭,鸿鹄自许。葛婴东下,周文西拒。始亲硃房,又任胡武。夥颐见杀,腹心不与。庄贾何人,反噬城父!
\end{yuanwen}

\chapter{外戚世家}

\begin{yuanwen}
自古受命帝王及继体守文之君,非独内德茂也,盖亦有外戚之助焉。夏之兴也以涂山,而桀之放也以末喜。殷之兴也以有娀,纣之杀也嬖妲己。周之兴也以姜原及大任,而幽王之禽也淫于襃姒。故易基乾坤,诗始关雎,书美釐降,春秋讥不亲迎。夫妇之际,人道之大伦也。礼之用,唯婚姻为兢兢。夫乐调而四时和,阴阳之变,万物之统也。可不慎与?人能弘道,无如命何。甚哉,妃匹之爱,君不能得之于臣,父不能得之于子,况卑下乎!既驩合矣,或不能成子姓;能成子姓矣,或不能要其终:岂非命也哉?孔子罕称命,盖难言之也。非通幽明之变,恶能识乎性命哉?

太史公曰:秦以前尚略矣,其详靡得而记焉。汉兴,吕娥姁为高祖正后,男为太子。及晚节色衰爱弛,而戚夫人有宠,其子如意几代太子者数矣。及高祖崩,吕后夷戚氏,诛赵王,而高祖后宫唯独无宠疏远者得无恙。

吕后长女为宣平侯张敖妻,敖女为孝惠皇后。吕太后以重亲故,欲其生子万方,终无子,诈取后宫人子为子。及孝惠帝崩,天下初定未久,继嗣不明。于是贵外家,王诸吕以为辅,而以吕禄女为少帝后,欲连固根本牢甚,然无益也。

高后崩,合葬长陵。禄、产等惧诛,谋作乱。大臣征之,天诱其统,卒灭吕氏。唯独置孝惠皇后居北宫。迎立代王,是为孝文帝,奉汉宗庙。此岂非天邪?非天命孰能当之?

薄太后,父吴人,姓薄氏,秦时与故魏王宗家女魏媪通,生薄姬,而薄父死山阴,因葬焉。

及诸侯畔秦,魏豹立为魏王,而魏媪内其女于魏宫。媪之许负所相,相薄姬,云当生天子。是时项羽方与汉王相距荥阳,天下未有所定。豹初与汉击楚,及闻许负言,心独喜,因背汉而畔,中立,更与楚连和。汉使曹参等击虏魏王豹,以其国为郡,而薄姬输织室。豹已死,汉王入织室,见薄姬有色,诏内后宫,岁馀不得幸。始姬少时,与管夫人、赵子兒相爱,约曰:“先贵无相忘。”已而管夫人、赵子兒先幸汉王。汉王坐河南宫成皋台,此两美人相与笑薄姬初时约。汉王闻之,问其故,两人具以实告汉王。汉王心惨然,怜薄姬,是日召而幸之。薄姬曰:“昨暮夜妾梦苍龙据吾腹。”高帝曰:“此贵徵也,吾为女遂成之。”一幸生男,是为代王。其后薄姬希见高祖。

高祖崩,诸御幸姬戚夫人之属,吕太后怒,皆幽之,不得出宫。而薄姬以希见故,得出,从子之代,为代王太后。太后弟薄昭从如代。

代王立十七年,高后崩。大臣议立后,疾外家吕氏彊,皆称薄氏仁善,故迎代王,立为孝文皇帝,而太后改号曰皇太后,弟薄昭封为轵侯。

薄太后母亦前死,葬栎阳北。于是乃追尊薄父为灵文侯,会稽郡置园邑三百家,长丞已下吏奉守冢,寝庙上食祠如法。而栎阳北亦置灵文侯夫人园,如灵文侯园仪。薄太后以为母家魏王后,早失父母,其奉薄太后诸魏有力者,于是召复魏氏,赏赐各以亲疏受之。薄氏侯者凡一人。

薄太后后文帝二年,以孝景帝前二年崩,葬南陵。以吕后会葬长陵,故特自起陵,近孝文皇帝霸陵。

窦太后,赵之清河观津人也。吕太后时,窦姬以良家子入宫侍太后。太后出宫人以赐诸王,各五人,窦姬与在行中。窦姬家在清河,欲如赵近家,请其主遣宦者吏:“必置我籍赵之伍中。”宦者忘之,误置其籍代伍中。籍奏,诏可,当行。窦姬涕泣,怨其宦者,不欲往,相彊,乃肯行。至代,代王独幸窦姬,生女嫖,后生两男。而代王王后生四男。先代王未入立为帝而王后卒。及代王立为帝,而王后所生四男更病死。孝文帝立数月,公卿请立太子,而窦姬长男最长,立为太子。立窦姬为皇后,女嫖为长公主。其明年,立少子武为代王,已而又徙梁,是为梁孝王。

窦皇后亲蚤卒,葬观津。于是薄太后乃诏有司,追尊窦后父为安成侯,母曰安成夫人。令清河置园邑二百家,长丞奉守,比灵文园法。

窦皇后兄窦长君,弟曰窦广国,字少君。少君年四五岁时,家贫,为人所略卖,其家不知其处。传十馀家,至宜阳,为其主入山作炭,暮卧岸下百馀人,岸崩,尽压杀卧者,少君独得脱,不死。自卜数日当为侯,从其家之长安。闻窦皇后新立,家在观津,姓窦氏。广国去时虽小,识其县名及姓,又常与其姊采桑堕,用为符信,上书自陈。窦皇后言之于文帝,召见,问之,具言其故,果是。又复问他何以为验?对曰:“姊去我西时,与我决于传舍中,丐沐沐我,请食饭我,乃去。”于是窦后持之而泣,泣涕交横下。侍御左右皆伏地泣,助皇后悲哀。乃厚赐田宅金钱,封公昆弟,家于长安。

绛侯、灌将军等曰:“吾属不死,命乃且县此两人。两人所出微,不可不为择师傅宾客,又复效吕氏大事也。”于是乃选长者士之有节行者与居。窦长君、少君由此为退让君子,不敢以尊贵骄人。

窦皇后病,失明。文帝幸邯郸慎夫人、尹姬,皆毋子。孝文帝崩,孝景帝立,乃封广国为章武侯。长君前死,封其子彭祖为南皮侯。吴楚反时,窦太后从昆弟子窦婴,任侠自喜,将兵,以军功为魏其侯。窦氏凡三人为侯。

窦太后好黄帝、老子言,帝及太子诸窦不得不读黄帝、老子,尊其术。

窦太后后孝景帝六岁崩,合葬霸陵。遗诏尽以东宫金钱财物赐长公主嫖。

王太后,槐里人,母曰臧兒。臧兒者,故燕王臧荼孙也。臧兒嫁为槐里王仲妻,生男曰信,与两女。而仲死,臧兒更嫁长陵田氏,生男蚡、胜。臧兒长女嫁为金王孙妇,生一女矣,而臧兒卜筮之,曰两女皆当贵。因欲奇两女,乃夺金氏。金氏怒,不肯予决,乃内之太子宫。太子幸爱之,生三女一男。男方在身时,王美人梦日入其怀。以告太子,太子曰:“此贵徵也。”未生而孝文帝崩,孝景帝即位,王夫人生男。

先是臧兒又入其少女兒姁,兒姁生四男。

景帝为太子时,薄太后以薄氏女为妃。及景帝立,立妃曰薄皇后。皇后毋子,毋宠。薄太后崩,废薄皇后。

景帝长男荣,其母栗姬。栗姬,齐人也。立荣为太子。长公主嫖有女,欲予为妃。栗姬妒,而景帝诸美人皆因长公主见景帝,得贵幸,皆过栗姬,栗姬日怨怒,谢长公主,不许。长公主欲予王夫人,王夫人许之。长公主怒,而日谗栗姬短于景帝曰:“栗姬与诸贵夫人幸姬会,常使侍者祝唾其背,挟邪媚道。”景帝以故望之。

景帝尝体不安,心不乐,属诸子为王者于栗姬,曰:“百岁后,善视之。”栗姬怒,不肯应,言不逊。景帝恚,心嗛之而未发也。

长公主日誉王夫人男之美,景帝亦贤之,又有曩者所梦日符,计未有所定。王夫人知帝望栗姬,因怒未解,阴使人趣大臣立栗姬为皇后。大行奏事毕,曰:“‘子以母贵,母以子贵’,今太子母无号,宜立为皇后。”景帝怒曰:“是而所宜言邪!”遂案诛大行,而废太子为临江王。栗姬愈恚恨,不得见,以忧死。卒立王夫人为皇后,其男为太子,封皇后兄信为盖侯。

景帝崩,太子袭号为皇帝。尊皇太后母臧兒为平原君。封田蚡为武安侯,胜为周阳侯。

景帝十三男,一男为帝,十二男皆为王。而兒姁早卒,其四子皆为王。王太后长女号日平阳公主,次为南宫公主,次为林虑公主。

盖侯信好酒。田蚡、胜贪,巧于文辞。王仲蚤死,葬槐里,追尊为共侯,置园邑二百家。及平原君卒,从田氏葬长陵,置园比共侯园。而王太后后孝景帝十六岁,以元朔四年崩,合葬阳陵。王太后家凡三人为侯。

卫皇后字子夫,生微矣。盖其家号曰卫氏,出平阳侯邑。子夫为平阳主讴者。武帝初即位,数岁无子。平阳主求诸良家子女十馀人,饰置家。武帝祓霸上还,因过平阳主。主见所侍美人。上弗说。既饮,讴者进,上望见,独说卫子夫。是日,武帝起更衣,子夫侍尚衣轩中,得幸。上还坐,驩甚。赐平阳主金千斤。主因奏子夫奉送入宫。子夫上车,平阳主拊其背曰:“行矣,彊饭,勉之!即贵,无相忘。”入宫岁馀,竟不复幸。武帝择宫人不中用者,斥出归之。卫子夫得见,涕泣请出。上怜之,复幸,遂有身,尊宠日隆。召其兄卫长君弟青为侍中。而子夫后大幸,有宠,凡生三女一男。男名据。

初,上为太子时,娶长公主女为妃。立为帝,妃立为皇后,姓陈氏,无子。上之得为嗣,大长公主有力焉,以故陈皇后骄贵。闻卫子夫大幸,恚,几死者数矣。上愈怒。陈皇后挟妇人媚道,其事颇觉,于是废陈皇后,而立卫子夫为皇后。

陈皇后母大长公主,景帝姊也,数让武帝姊平阳公主曰:“帝非我不得立,已而弃捐吾女,壹何不自喜而倍本乎!”平阳公主曰:“用无子故废耳。”陈皇后求子,与医钱凡九千万,然竟无子。

卫子夫已立为皇后,先是卫长君死,乃以卫青为将军,击胡有功,封为长平侯。青三子在襁褓中,皆封为列侯。及卫皇后所谓姊卫少兒,少兒生子霍去病,以军功封冠军侯,号骠骑将军。青号大将军。立卫皇后子据为太子。卫氏枝属以军功起家,五人为侯。

及卫后色衰,赵之王夫人幸,有子,为齐王。

王夫人蚤卒。而中山李夫人有宠,有男一人,为昌邑王。

李夫人蚤卒,其兄李延年以音幸,号协律。协律者,故倡也。兄弟皆坐奸,族。是时其长兄广利为贰师将军,伐大宛,不及诛,还,而上既夷李氏,后怜其家,乃封为海西侯。

他姬子二人为燕王、广陵王。其母无宠,以忧死。

及李夫人卒,则有尹婕妤之属,更有宠。然皆以倡见,非王侯有土之士女,不可以配人主也。

褚先生曰:臣为郎时,问习汉家故事者锺离生。曰:王太后在民间时所生一女者,父为金王孙。王孙已死,景帝崩后,武帝已立,王太后独在。而韩王孙名嫣素得幸武帝,承间白言太后有女在长陵也。武帝曰:“何不蚤言!”乃使使往先视之,在其家。武帝乃自往迎取之。跸道,先驱旄骑出横城门,乘舆驰至长陵。当小市西入里,里门闭,暴开门,乘舆直入此里,通至金氏门外止,使武骑围其宅,为其亡走,身自往取不得也。即使左右群臣入呼求之。家人惊恐,女亡匿内中床下。扶持出门,令拜谒。武帝下车泣曰:“嚄!大姊,何藏之深也!”诏副车载之,回车驰还,而直入长乐宫。行诏门著引籍,通到谒太后。太后曰:“帝倦矣,何从来?”帝曰:“今者至长陵得臣姊,与俱来。”顾曰:“谒太后!”太后曰:“女某邪?”曰:“是也。”太后为下泣,女亦伏地泣。武帝奉酒前为寿,奉钱千万,奴婢三百人,公田百顷,甲第,以赐姊。太后谢曰:“为帝费焉。”于是召平阳主、南宫主、林虑主三人俱来谒见姊,因号曰脩成君。有子男一人,女一人。男号为脩成子仲,女为诸侯王王后。此二子非刘氏,以故太后怜之。脩成子仲骄恣,陵折吏民,皆患苦之。

卫子夫立为皇后,后弟卫青字仲卿,以大将军封为长平侯。四子,长子伉为侯世子,侯世子常侍中,贵幸。其三弟皆封为侯,各千三百户,一曰阴安侯,一二曰发干侯,三曰宜春侯,贵震天下。天下歌之曰:“生男无喜,生女无怒,独不见卫子夫霸天下!”

是时平阳主寡居,当用列侯尚主。主与左右议长安中列侯可为夫者,皆言大将军可。主笑曰:“此出吾家,常使令骑从我出入耳,柰何用为夫乎?”左右侍御者曰:“今大将军姊为皇后,三子为侯,富贵振动天下,主何以易之乎?”于是主乃许之。言之皇后,令白之武帝,乃诏卫将军尚平阳公主焉。

褚先生曰:丈夫龙变。传曰:“蛇化为龙,不变其文;家化为国,不变其姓。”丈夫当时富贵,百恶灭除,光耀荣华,贫贱之时何足累之哉!

武帝时,幸夫人尹婕妤。邢夫人号娙娥,众人谓之“娙何”。娙何秩比中二千石,容华秩比二千石,婕妤秩比列侯。常从婕妤迁为皇后。

尹夫人与邢夫人同时并幸,有诏不得相见。尹夫人自请武帝,原望见邢夫人,帝许之。即令他夫人饰,从御者数十人,为邢夫人来前。尹夫人前见之,曰:“此非邢夫人身也。”帝曰:“何以言之?”对曰:“视其身貌形状,不足以当人主矣。”于是帝乃诏使邢夫人衣故衣,独身来前。尹夫人望见之,曰:“此真是也。”于是乃低头俯而泣,自痛其不如也。谚曰:“美女入室,恶女之仇。”

褚先生曰:浴不必江海,要之去垢;马不必骐骥,要之善走;士不必贤世,要之知道;女不必贵种,要之贞好。传曰:“女无美恶,入室见妒;士无贤不肖,入朝见嫉。”美女者,恶女之仇。岂不然哉!

钩弋夫人姓赵氏,河间人也。得幸武帝,生子一人,昭帝是也。武帝年七十,乃生昭帝。昭帝立时,年五岁耳。

卫太子废后,未复立太子。而燕王旦上书,原归国入宿卫。武帝怒,立斩其使者于北阙。

上居甘泉宫,召画工图画周公负成王也。于是左右群臣知武帝意欲立少子也。后数日,帝谴责钩弋夫人。夫人脱簪珥叩头。帝曰:“引持去,送掖庭狱!”夫人还顾,帝曰:“趣行,女不得活!”夫人死云阳宫。时暴风扬尘,百姓感伤。使者夜持棺往葬之,封识其处。

其后帝闲居,问左右曰:“人言云何?”左右对曰:“人言且立其子,何去其母乎?”帝曰:“然。是非兒曹愚人所知也。往古国家所以乱也,由主少母壮也。女主独居骄蹇,淫乱自恣,莫能禁也。女不闻吕后邪?”故诸为武帝生子者,无男女,其母无不谴死,岂可谓非贤圣哉!昭然远见,为后世计虑,固非浅闻愚儒之所及也。谥为“武”,岂虚哉!

礼贵夫妇,易叙乾坤。配阳成化,比月居尊。河洲降淑,天曜垂轩。德著任、姒,庆流娀、嫄。逮我炎历,斯道克存。吕权大宝,窦喜玄言。自兹已降,立嬖以恩。内无常主,后嗣不繁。
\end{yuanwen}

\part{卷五十}
\chapter{楚元王世家第二十}

\begin{yuanwen}
楚元王刘交者,高祖之同母少弟也,字游。
\end{yuanwen}

楚元王刘交,是高祖同母的小弟,字游。

\begin{yuanwen}
高祖兄弟四人,长兄伯,伯蚤卒。始高祖微时,尝辟事,时时与宾客过巨嫂食。嫂厌叔\footnote{指刘邦。},叔与客来,嫂详\footnote{通“佯”,假装。}为羹尽,栎釜,宾客以故去。已而视釜中尚有羹,高祖由此怨其嫂。及高祖为帝,封昆弟,而伯子独不得封。太上皇以为言,高祖曰:“某非忘封之也,为其母不长者耳。”于是乃封其子信为羹颉侯。而王次兄仲\footnote{名喜,刘濞之父。}于代。
\end{yuanwen}

高祖兄弟共四人,大哥刘伯,很早就去世了。起初高祖微贱之时,曾为了避难,经常与宾客来到大嫂家吃饭。大嫂厌恶小叔,小叔与宾客来到家里,大嫂就假装羹汤吃完了,用勺子刮着锅,宾客因此就全都走了。不久再看锅里还剩有羹汤,高祖因此就怨恨大嫂。等到高祖登基称帝,分封自己的兄弟,只有大哥的儿子没能得到封赏。太上皇就替孙子说情,高祖说:“我并不是忘了封赏他,是因为他的母亲一点也不像个长者。”于是封她的儿子刘信为羹颉侯。封二哥刘仲为代王。

\begin{yuanwen}
高祖六年,已禽楚王韩信于陈,乃以弟交为楚王,都彭城。即位二十三年卒,子夷王郢立。夷王四年卒,子王戊立。
\end{yuanwen}

高祖六年(前201年),已在陈县抓住楚王韩信,就封自己的弟弟刘交为楚王,定都在彭城。交受封楚王二十三年后去世,他的儿子夷王郢继位。夷王即位四年后去世,他的儿子戊继位。

\begin{yuanwen}
王戊立二十年,冬,坐为薄太后服私奸,削东海郡。春,戊与吴王合谋反,其相张尚、太傅赵夷吾谏,不听。戊则杀尚、夷吾,起兵与吴西攻梁,破棘壁。至昌邑南,与汉将周亚夫战。汉绝吴楚粮道,士卒饥,吴王走,楚王戊自杀,军遂降汉。
\end{yuanwen}

戊即位后第二十年,冬季,由于他在为薄太后服丧时犯下了私奸罪,被削去东海郡的封地。第二年春季,戊与吴王合谋造反,他的相国张尚、太傅赵夷吾进谏劝阻,他没有听从。戊杀掉了张尚、赵夷吾,发兵与吴王联合向西攻打梁国,攻下了棘壁。行军到昌邑南,与汉将周亚夫交战。汉军断绝了吴军和楚军的运粮通道,士卒十分饥饿,吴王败退,楚王戊自杀,于是吴、楚联军投降了汉军。

\begin{yuanwen}
汉已平吴楚,孝景帝欲以德侯子续吴,以元王子礼续楚。窦太后曰:“吴王,老人也,宜为宗室顺善。今乃首率七国,纷乱天下,奈何续其后!”不许吴,许立楚后。

是时礼为汉宗正。乃拜礼为楚王,奉元王宗庙,是为楚文王。
\end{yuanwen}

汉军已然平定了吴、楚的叛乱,孝景帝打算让德侯广的儿子接着做吴王,让元王的儿子礼接着做楚王。窦太后说:“吴王,是老一辈的人,本应为大汉宗室效忠,带头从善。现在却带头率领七国发动叛乱,扰乱天下纲常,为什么还要再立他的后代!”不同意封立吴王的后代,只允许封立楚王的后代。

那时礼是汉朝的宗正,因此封礼做了楚王,让他供奉元王的宗庙,他就是楚文王。

\begin{yuanwen}
文王立三年卒,子安王道立。安王二十二年卒,子襄王注立。襄王立十四年卒,子王纯代立。王纯立,地节二年,中人上书告楚王谋反,王自杀,国除,入汉为彭城郡。
\end{yuanwen}

文王被立为楚王三年后就去世了,他的儿子安王道继位。安王做了二十二年楚王后去世,他的儿子襄王注继位。襄王做了十四年楚王后去世,他的儿子纯继位。纯做了楚王后,地节二年(前68年),有宦官上书告发楚王谋反,楚王自杀,国号被废除,封地被收归汉朝改为彭城郡。

\begin{yuanwen}
赵王刘遂者,其父高祖中子,名友,谥曰“幽”。幽王以忧死,故为“幽”。高后王吕禄于赵,一岁而高后崩。大臣诛诸吕吕禄等,乃立幽王子遂为赵王。
\end{yuanwen}

赵王刘遂的父亲,是高祖排行居中的儿子,名友,谥号为“幽”。幽王由于忧虑而去世,因此谥号为“幽”。高后将吕禄封到赵地做王,一年后高后就去世了。大臣诛灭了吕禄等吕氏家族的人,于是立幽王的儿子遂做了赵王。

\begin{yuanwen}
孝文帝即位二年,立遂弟辟彊,取赵之河间郡为河间王,(是)为文王。立十三年卒,子哀王福立。一年卒,无子,绝后,国除,入于汉。
\end{yuanwen}

孝文帝继承皇位二年后,封立遂的弟弟辟彊,划出赵国的河间郡给他,封他为河间王,也就是文王。文王被立十三年后去世,他的儿子哀王福继位。福在位一年后去世,没生儿子,断绝了后代,国号被废除,封地被收归汉朝。

\begin{yuanwen}
遂既王赵二十六年,孝景帝时坐晁错以適削赵王常山之郡。吴楚反,赵王遂与合谋起兵。其相建德、内史王悍谏,不听。遂烧杀建德、王悍,发兵屯其西界,欲待吴与俱西。北使匈奴,与连和攻汉。汉使曲周侯郦寄击之。赵王遂还,城守邯郸,相距七月。吴楚败于梁,不能西。匈奴闻之,亦止,不肯入汉边。栾布\footnote{汉初名将。}自破齐还,乃并兵引水灌赵城。赵城坏,赵王自杀,邯郸遂降。赵幽王绝后。
\end{yuanwen}

遂做赵王已有二十六年,孝景帝时,由于犯下过失被晁错削掉了他的常山郡。吴、楚反叛,赵王就与他们合谋起兵。他的相国建德、内史王悍劝阻他,没有听从。于是就烧死了建德、王悍,出兵驻屯在赵国西部的边界,打算等吴军到后一同向西进发。向北派出使者出使匈奴,打算联合匈奴一起攻打汉朝。汉朝派曲周侯郦寄攻打赵国。于是赵王撤军返回,在邯郸城据守,相持了七个月。吴军和楚军在梁国被击败,无法西进。匈奴听说这个消息,也停止派兵,不肯进入汉朝的边界。栾布从击溃齐国的前线归来,就与郦寄会师引水灌入赵国的都城。赵国的城墙被水泡坏,赵王自杀,邯郸就投降了。赵幽王断绝了后代。

\begin{yuanwen}
太史公曰:国之将兴,必有祯祥,君子用而小人退。国之将亡,贤人隐,乱臣贵。使楚王戊毋刑申公,遵其言,赵任防与先生,岂有篡杀之谋,为天下僇\footnote{通“戮”,耻笑。}哉?贤人乎,贤人乎!非质有其内,恶能用之哉?甚矣,“安危在出令,存亡在所任”,诚哉是言也!
\end{yuanwen}

太史公说:国家即将兴起之时,必定会出现吉祥的征兆,君子得到任用而小人遭到斥退。国家即将灭亡之时,贤人纷纷归隐,乱臣日益显贵。如果楚王戊没有杀掉申公,听从他的话,赵王重用防与先生,怎么会出现篡杀的阴谋,被天下人耻笑呢?贤人啊,贤人啊!若非君王内心贤能,怎么可能会任用你们呢?这多么重要啊!“国家安危的关键在于发出的政令,国家的存亡取决于所任用的大臣”,这话多么正确啊!

\begin{yuanwen}
汉封同姓,楚有令名。既灭韩信,王于彭城。穆生置醴,韦孟作程。王戊弃德,与吴连兵。太后命礼,为楚罪轻。文襄继立,世挺才英。如何赵遂,代殒厥声!兴亡之兆,所任宜明。
\end{yuanwen}

\part{卷五十一}
\chapter{荆燕世家第二十一}

\begin{yuanwen}
荆王刘贾者,诸刘,不知其何属。初起时。汉王元年,还定三秦,刘贾为将军,定塞地,从东击项籍。
\end{yuanwen}

荆王刘贾,是刘氏宗族的人,不清楚他属于刘氏宗族的哪一支。起初,汉王元年(前206年),汉军从汉中返回关中平定三秦,刘贾担任将军,平定了塞王司马欣所占之地,刘贾跟随汉王向东进军攻击项籍。

\begin{yuanwen}
汉四年,汉王之败成皋,北渡河,得张耳、韩信军,军\footnote{驻扎。}脩武,深沟高垒,使刘贾将\footnote{率领。}二万人,骑数百,渡白马津入楚地,烧其积聚,以破其业,无以给项王军食。已而楚兵击刘贾,贾辄壁不肯与战,而与彭越相保。
\end{yuanwen}

汉王四年(前203年),汉王在成皋战败,向北渡过黄河,夺到张耳、韩信两人的军队,在修武驻扎,深挖沟,高筑垒,命令刘贾率领两万名步兵,数百名骑兵,渡过白马津深入楚军腹地,烧毁楚军屯积在当地的粮草,以此来破坏项籍的战争大业,让他们没有办法给项王的将士们提供军粮。很快楚军便进攻刘贾,刘贾坚持采取坚壁不战的策略,从而与彭越形成两军互保的局势。

\begin{yuanwen}
汉五年,汉王追项籍至固陵,使刘贾南渡淮围寿春。还至,使人间招楚大司马周殷。周殷反楚,佐刘贾举九江,迎武王黥布兵,皆会垓下,共击项籍。汉王因使刘贾将九江兵,与太尉卢绾西南击临江王共尉。共尉已死,以临江为南郡。
\end{yuanwen}

汉王五年(前202年),汉王追赶项籍直至固陵,派刘贾向南渡过淮河包围寿春。刘贾军队到达后,命人暗中招降楚军大司马周殷。周殷背叛项王,帮助刘贾攻占了九江,迎接武王黥布的军队,全都到垓下会合,一起攻打项籍。汉王因而派刘贾率领九江的军队,与太尉卢绾一同向西南进攻临江王共尉。共尉死后,在临江设置南郡。

\begin{yuanwen}
汉六年春,会诸侯于陈,废楚王信,囚之,分其地为二国。当是时也,高祖子幼,昆弟少,又不贤,欲王同姓以镇天下,乃诏曰:“将军刘贾有功,及择子弟可以为王者。”

群臣皆曰:“立刘贾为荆王,王淮东五十二城;高祖弟交为楚王,王淮西三十六城。”

因立子肥\footnote{刘肥,刘邦庶长子。}为齐王。始王昆弟刘氏也。
\end{yuanwen}

汉王六年(前201年)春季,汉王刘邦在陈县会见各诸侯王,废掉楚王韩信,并且拘禁了他,把他的封地分成两个封国。这时,高祖的儿子尚且年幼,兄弟不多,又不贤能,刘邦便想封所有同姓的宗族为王以此安抚天下,于是下诏说:“将军刘贾有战功,应挑选刘氏子弟中有战功的可以封为王的人。”

群臣都说:“立刘贾为荆王,管辖淮东五十二座城池;立高祖的弟弟刘交为楚王,管辖淮西的三十六座城池。”

因而高祖将自己的儿子刘肥封为齐王。从此开始封刘氏兄弟为王。

\begin{yuanwen}
高祖十一年秋,淮南王黥布反,东击荆。荆王贾与战,不胜,走富陵,为布军所杀。高祖自击破布。

十二年,立沛侯刘濞\footnote{bì}为吴王,王故荆地。
\end{yuanwen}

高祖十一年(前196年)秋季,淮南王黥布发动叛乱,向东攻打荆王。荆王刘贾与他交战,没有取胜,兵败逃到富陵,被黥布的军队杀死。高祖亲自领兵打败黥布。

十二年(前195年),高祖立沛侯刘濞为吴王,统辖原本属于荆王的封地。

\begin{yuanwen}
燕王刘泽者,诸刘远属也。高帝三年,泽为郎中。

高帝十一年,泽以将军击陈豨\footnote{xī},得王黄,为营陵侯。
\end{yuanwen}

燕王刘泽,是刘氏宗族的远房子孙。高帝三年(前204年),刘泽担任郎中。

高帝十一年(前196年),刘泽以将军的身份攻打陈豨,俘虏了敌将王黄,因此被封为营陵侯。

\begin{yuanwen}
高后时,齐人田生游乏资,以画干营陵侯泽。泽大说之,用金二百斤为田生寿。田生已得金,即归齐。

二年,泽使人谓田生曰:“弗与矣。”

田生如长安,不见泽,而假大宅,令其子求事吕后所幸大谒者张子卿。居数月,田生子请张卿临,亲脩具。张卿许往。田生盛帷帐共具,譬如列侯。张卿惊。酒酣,乃屏人说张卿曰:“臣观诸侯王邸弟百馀,皆高祖一切功臣。今吕氏雅故本\footnote{三字同义,本来,向来。}推毂\footnote{协助,佐助。}高帝就天下,功至大,又亲戚太后之重。太后春秋长,诸吕弱,太后欲立吕产为王,王代。太后又重发之,恐大臣不听。今卿最幸,大臣所敬,何不风大臣以闻太后,太后必喜。诸吕已王,万户侯亦卿之有。太后心欲之,而卿为内臣,不急发,恐祸及身矣。”

张卿大然之,乃风大臣语太后。太后朝,因问大臣。大臣请立吕产为吕王。太后赐张卿千斤金,张卿以其半与田生。田生弗受,因说之曰:“吕产王也,诸大臣未大服。今营陵侯泽,诸刘,为大将军,独此尚觖望\footnote{怨恨,不满。觖,jué}。今卿言太后,列十馀县王之,彼得王,喜去,诸吕王益固矣。”

张卿入言,太后然之。乃以营陵侯刘泽为琅邪王。琅邪王乃与田生之国。田生劝泽急行,毋留。出关,太后果使人追止之,已出,即还。
\end{yuanwen}

高后执政时期,齐人田子春外出游历时缺少路费,就通过为营陵侯刘泽谋划献计求取资助。刘泽十分高兴,花了黄金二百斤为田子春祝寿。田子春得到黄金后,立即返回了齐国。

第二年,刘泽派人给田子春传话说:“不要再帮助我了。”

田子春来到长安,不去见刘泽,而是租了一间大宅院,让他的儿子前去求见被吕后宠幸的大谒者子卿。住了几个月,田子春的儿子邀请张卿光临大宅,亲自为他准备宴席。张卿答应前去吃饭。田子春挂起豪华的帷帐,摆设出精美的用具,就像款待列侯一样。张卿来到以后很吃惊。酒酣之际,田子春屏退左右的人,对张卿说:“我曾经参观了一百多座诸侯王的府第,他们都是高祖时期的功臣。现在吕氏宗族向来一心,辅佐高祖夺得天下,功劳最大,诸吕又都是太后重要的亲戚。太后如今年事已高,诸吕仍然爵位很低,官职很小,太后想要封吕产为王,统辖代地。太后又不好意思提出,担心大臣们不同意。如今你是最受太后宠幸的人,就连大臣们也对你十分敬畏,为什么不示意大臣们主动向太后进言拥立诸吕为王,太后一定会很高兴。吕氏宗族如果被封王,万户侯的封赏就一定会为你所有了。太后心中想这样做,你身为内臣,不尽快去办这件事的话,恐怕就要大祸临头了。”

张卿十分赞同,于是示意大臣们向吕后进言封吕氏宗族的人为王。太后上朝,趁机向大臣们询问这件事。大臣们都请求太后封吕产为吕王。太后赏赐张卿千斤黄金,张卿将其中的一半送给了田子春。田子春不肯接受,趁机劝张卿说:“吕产已经被封为王,大臣们并未完全服气。如今的营陵侯刘泽,是刘氏宗族,而且身为大将军,不满足的地方也只是没被封王了。你现在劝说太后,让她划出十几个县封刘泽为王,刘泽得到王爵,高高兴兴地离开,诸吕的王位就更加巩固了。”

张卿立即入宫向太后进言,太后认为张卿说得很对。于是封营陵侯刘泽为琅邪王。琅邪王刘泽便和田子春一起前往封国。田子春劝刘泽尽快启程,不要停留。他们刚出函谷关,太后果然派人追来想要阻止他们,但是刘泽此时已经出关了,派去追赶的人只好回去复命。

\begin{yuanwen}
及太后崩,琅邪王泽乃曰:“帝少,诸吕用事,刘氏孤弱。”

乃引兵与齐王合谋西,欲诛诸吕。至梁,闻汉遣灌将军屯荥阳,泽还兵备西界,遂跳\footnote{脱身离去。}驱至长安。代王亦从代至。诸将相与琅邪王共立代王为天子。天子乃徙泽为燕王,乃复以琅邪予齐,复故地。
\end{yuanwen}

等到太后去世以后,琅邪王刘泽说:“皇帝年幼,诸吕当权,刘氏宗族势单力薄。”

于是率领军队与齐王一起谋划向西进兵,想要诛杀吕氏宗族。来到梁地的时候,听说汉朝派灌婴将军在荥阳屯兵,刘泽便撤兵回到自己封国的西部边界待命,然后独自快马加鞭赶到长安。代王也从代地来到长安。诸位将领、丞相和琅邪王共同拥立代王为天子。于是天子改封刘泽为燕王,重新将琅邪归还给了齐国,恢复了齐国原有的封地。

\begin{yuanwen}
泽王燕二年,薨\footnote{古代指诸侯死亡。},谥为敬王。传子嘉,为康王。
\end{yuanwen}

刘泽做燕王两年后去世,谥号敬王。他的王位传给了儿子刘嘉,刘嘉就是康王。

\begin{yuanwen}
至孙定国,与父康王姬奸,生子男一人。夺弟妻为姬。与子女三人奸。定国有所欲诛杀臣肥如令郢\footnote{yǐng}人,郢人等告定国,定国使谒者以他法劾捕格杀郢人以灭口。至元朔元年,郢人昆弟复上书具言定国阴事,以此发觉。诏下公卿,皆议曰:“定国禽兽行,乱人伦,逆天,当诛。”

上许之。定国自杀,国除为郡。
\end{yuanwen}

到刘泽的孙子刘定国的时候,他与父亲康王的姬妾通奸,生下了一个男孩。又霸占弟弟的妻子为姬妾。与三个女子通奸。刘定国想要杀死一个名叫郢人的肥如县令,郢人就将刘定国的罪状上报给朝廷,刘定国立即派人假托其他法令,说有人检举郢人,将他逮捕并杀掉灭口。到元朔元年(前128年),郢人的兄弟再次上书揭发刘定国的全部丑事,刘定国的滔天罪行才被揭露。武帝下诏让公卿们讨论,议论的人都说:“刘定国禽兽一样的行为,败坏人伦,有违天理,论罪理当处死。”

武帝批准。刘定国自杀,他的封国也被废除,成为汉的一个郡。

\begin{yuanwen}
太史公曰:荆王王也,由汉初定,天下未集,故刘贾虽属疏,然以策为王,填\footnote{通“镇”,镇抚。}江淮之间。刘泽之王,权激\footnote{用权术激发鼓动。}吕氏,然刘泽卒南面称孤者三世。事发相重,岂不为伟乎!
\end{yuanwen}

太史公说:荆王之所以被封为王,是因为汉朝当时刚刚建立,天下尚未完全安定,因而刘贾虽然只是刘氏的远房亲戚,仍然能被封为王,镇抚江淮之间。刘泽之所以被封为王,是因为通过权谋激发吕后封诸吕为王,这样刘泽才得以在南面三代为王。事情从一开始就互相牵连在一起,难道不是很奇妙伟大吗!

\begin{yuanwen}
刘贾初从,首定三秦。既渡白马,遂围寿春。始迎黥布,绝间周殷。赏功胙士,与楚为邻。营陵始爵,勋由击陈。田生游说,受赐千斤。权激诸吕,事发荣身。徙封传嗣,亡于郢人。
\end{yuanwen}

\part{卷五十二}
\chapter{齐悼惠王世家第二十二}

\begin{yuanwen}
齐悼惠王刘肥者,高祖长庶男也。其母外妇也,曰曹氏。高祖六年,立肥为齐王,食七十城,诸民能齐言者皆予齐王。
\end{yuanwen}

齐悼惠王刘肥,是高祖的庶出长子。刘肥的母亲是高祖在外面的情妇,姓曹。高祖六年(前201年),封刘肥为齐王,食邑七十座城池,百姓只要是能说齐语的都归齐王所管。

\begin{yuanwen}
齐王,孝惠帝\footnote{刘盈,刘邦与吕后之子,前194年至前188年在位。}兄也。孝惠帝二年,齐王入朝。惠帝与齐王燕饮,亢礼如家人。吕太后怒,且诛齐王。齐王惧不得脱,乃用其内史勋计,献城阳郡,以为鲁元公主\footnote{刘邦与吕后之女。}汤沐邑。吕太后喜,乃得辞就国。
\end{yuanwen}

齐王,是孝惠帝的兄长。孝惠帝二年(前193年),齐王来到都城朝拜。孝惠帝和齐王一同宴饮,以平等的礼节相处就像家人一样。吕太后很生气,要杀死齐王。齐王害怕自己无法脱身,就采用他的内史勋提供的计策,献出城阳郡作为鲁元公主的汤沐邑。吕太后很高兴,齐王才得以辞行回到齐国。

\begin{yuanwen}
悼惠王即位十三年,以惠帝六年卒。子襄立,是为哀王。

哀王元年,孝惠帝崩,吕太后称制,天下事皆决于高后。二年,高后立其兄子郦侯吕台为吕王,割齐之济南郡为吕王奉邑。

哀王三年,其弟章入宿卫于汉,吕太后封为硃虚侯,以吕禄女妻之。后四年,封章弟兴居为东牟侯,皆宿卫长安中。

哀王八年,高后割齐琅邪郡立营陵侯刘泽为琅邪王。

其明年\footnote{事在当年,“明年”误。},赵王友入朝,幽死于邸。三赵王皆废。高后立诸吕诸吕为三王,擅权用事。
\end{yuanwen}

悼惠王在位十三年后,于惠帝六年(前189年)去世。他的儿子刘襄继位,就是齐哀王。

哀王元年(前188年),孝惠帝驾崩,吕太后开始执掌朝政,天下的事务都由高后定夺。二年(前187年),高后立自己兄长的儿子郦侯吕台为吕王,割齐国的济南郡为吕王封地。

哀王三年(前186年),齐哀王的弟弟刘章来到汉朝宫廷担任宿卫,吕太后封刘章为朱虚侯,并将吕禄的女儿嫁给他。四年后,吕太后又封刘章的弟弟刘兴居为东牟侯,二人都在长安的宫廷担任宿卫。

哀王八年(前181年),高后割出齐国的琅邪郡,封营陵侯刘泽为琅邪王。

第二年,赵王刘友来到京师朝见太后,最后在府邸被幽禁而死。三个赵王先后都被废黜。高后分别封吕氏宗族的子弟为燕王、赵王、梁王,独断专行。

\begin{yuanwen}
硃虚侯年二十,有气力,忿刘氏不得职。尝入待高后燕饮,高后令硃虚侯刘章为酒吏。章自请曰:“臣,将种\footnote{将门之后。}也,请得以军法行酒。”

高后曰:“可。”

酒酣,章进饮歌舞。已而曰:“请为太后言耕田歌。”

高后儿子畜之,笑曰:“顾而父知田耳。若生而为王子,安知田乎?”

章曰:“臣知之。”

太后曰:“试为我言田。”

章曰:“深耕穊\footnote{密。jì}种,立苗欲疏,非其种者,鉏而去之。”

吕后默然。顷之,诸吕有一人醉,亡酒,章追,拔剑斩之,而还报曰:“有亡酒一人,臣谨行法斩之。”

太后左右皆大惊。业已许其军法,无以罪也。因罢。自是之后,诸吕惮\footnote{dàn}硃虚侯,虽大臣皆依硃虚侯,刘氏为益彊。
\end{yuanwen}

朱虚侯刘章二十岁的时候,为人有气概又勇武多力,他因为刘氏得不到职权而十分愤怒。朱虚侯曾经侍奉高后宴饮,高后命令朱虚侯刘章为酒吏。刘章向太后请示说:“我是将门之子,请允许我按照军法来监督酒席上的行为。”

高后说:“可以。”

酒酣耳热之际,刘章奉献上助兴的歌舞。过了一会又说:“请允许我为太后唱一首耕田歌。”

高后将他当成小孩子看待,笑着说:“你的父亲倒是懂得种田的事宜。你从出生时起就是王子,怎么会知道种田的事呢?”

刘章说:“我知晓种田的事情。”

太后说:“那你就试着给我唱种田歌吧。”

刘章唱道:“深耕密植,留苗要稀疏,不是自己种下的植物,一定要铲除不能留下。”

吕后沉默不语。不一会儿,诸吕有一人醉酒,逃离了酒席,刘章追了上去,拔剑杀了他,然后回来报告吕后说:“有一个人擅自逃离酒席,我已经按照军法将他处死了。”

太后和左右的人都非常吃惊。但是早已允许他按照军法来执行酒令,就没有怪罪他的理由。酒宴也因而结束了。从此以后,诸吕都很害怕朱虚侯,即使是大臣也都依附朱虚侯,刘氏由此日渐强盛。

\begin{yuanwen}
其明年,高后崩。赵王吕禄为上将军,吕王产为相国,皆居长安中,聚兵以威大臣,欲为乱。硃虚侯章以吕禄女为妇,知其谋,乃使人阴出告其兄齐王,欲令发兵西,硃虚侯、东牟侯为内应,以诛诸吕,因立齐王为帝。
\end{yuanwen}

第二年,高后去世。赵王吕禄担任上将军,吕王吕产担任相国,这两个人都居住在长安城中,他们聚集兵力威胁大臣,想要发动叛乱。朱虚侯刘章娶了吕禄的女儿为妻,因而了解他们的阴谋,于是派人暗中离开长安,前去通知他的哥哥齐王,想让他领兵西进,朱虚侯、东牟侯在长安城做内应,这样就能够一举诛灭诸吕,趁机拥立齐王为帝。

\begin{yuanwen}
齐王既闻此计,乃与其舅父驷钧、郎中令祝午、中尉魏勃阴谋发兵。齐相召平闻之,乃发卒卫王宫。魏勃绐召平曰:“王欲发兵,非有汉虎符验也。而相君围王,固善。勃请为君将兵卫卫王。”

召平信之,乃使魏勃将兵围王宫。勃既将兵,使围相府。召平曰:“嗟乎!道家之言‘当断不断,反受其乱’,乃是也。”遂自杀。

于是齐王以驷钧为相,魏勃为将军,祝午为内史,悉发国中兵。使祝午东诈琅邪王曰:“吕氏作乱,齐王发兵欲西诛之。齐王自以儿子,年少,不习兵革之事,原举国委大王。大王自高帝将也,习战事。齐王不敢离兵\footnote{离开军队。},使臣请大王幸之临菑见齐王计事,并将齐兵以西平关中之乱。”

琅邪王信之,以为然,(逎)驰见齐王。齐王与魏勃等因留琅邪王,而使祝午尽发琅邪国而并将其兵。
\end{yuanwen}

齐王听说这个计划以后,就和自己的舅舅驷钧、郎中令祝午以及中尉魏勃暗中谋划发兵。齐相召平听说了这件事,就先发兵包围了齐王的王宫,让齐王无法传令发兵。魏勃欺骗召平说:“君王想要发兵,却没有汉朝的兵符凭证。而相君现在包围了王宫,这本是件好事。我请求替你带领士兵包围王宫。”

召平相信了魏勃的话,于是派魏勃带领士兵包围王宫。魏勃统领军队后,却派兵包围了相府。召平说:“唉!道家的话‘应该当机立断的时候却犹豫不决,就会反过来遭受祸乱’,果然是这样啊。”于是召平就自杀了。

于是齐王任命驷钧为相国,魏勃为将军,祝午为内史,征集了国内的全部兵力出发。齐王派祝午东去欺骗琅邪王说:“吕氏贼人作乱,齐王起兵打算西进诛灭他们。齐王认为自己是晚辈,年轻,不熟悉领兵作战等军中要事,愿意将全国的军队都委托给大王。大王从高祖的时候起就担任将领,深谙战事。齐王不敢轻易离开军队,派我来请大王到临菑去与齐王商议大事,并且带领齐军向西平定关中之乱。”

琅邪王相信了,认为很对,就快马加鞭赶到临菑去见齐王。齐王与魏勃等人趁机扣留琅邪王,而派祝午调发琅邪国所有的人力物力,并且统率琅邪国的军队。

\begin{yuanwen}
琅邪王刘泽既见欺,不得反国,乃说齐王曰:“齐悼惠王高皇帝长子,推本言之,而大王高皇帝適长孙也,当立。今诸大臣狐疑未有所定,而泽于刘氏最为长年,大臣固待泽决计。今大王留臣无为也,不如使我入关计事。”齐王以为然,乃益具车送琅邪王。
\end{yuanwen}

琅邪王刘泽已经发现受骗,无法返回琅琊国,于是劝齐王说:“齐悼惠王是高皇帝长子,推本溯源来说,大王你就是高皇帝的嫡长孙,应当继承皇位。现在大臣仍然犹豫不能决断,而我刘泽在刘氏宗族中最为年长,大臣本就在等待我决定大计。现在大王将我扣留并无用处,还不如让我入关商议迎接拥立新皇帝的大事。”齐王觉得刘泽说得有道理,就准备了很多车辆将琅邪王送走。

\begin{yuanwen}
琅邪王既行,齐遂举兵西攻吕国之济南。于是齐哀王遗诸侯王书曰:“高帝平定天下,王诸子弟,悼惠王于齐。悼惠王薨,惠帝使留侯张良立臣为齐王。惠帝崩,高后用事,春秋高,听诸吕擅废高帝所立,又杀三赵王,灭梁、燕、赵以王诸吕,分齐国为四。忠臣进谏,上惑乱不听。今高后崩,皇帝春秋富,未能治天下,固恃大臣诸侯。今诸吕又擅自尊官,聚兵严威,劫列侯忠臣,矫制以令天下,宗庙所以危。今寡人率兵入诛不当为王者。”
\end{yuanwen}

琅邪王出发后,齐国便发兵向西攻打吕国的济南。于是齐哀王给各诸侯王写了一封信说:“高皇帝平定天下后,分封刘氏子弟为王,悼惠王被封在齐国。悼惠王去世后,孝惠帝派留侯张良拥立我为齐王。孝惠帝驾崩,朝中高后当权,她年纪大了,听信诸吕谗言,擅自废黜高帝所立的诸王,又先后杀死了三位赵王,消灭了梁国、燕国、赵国,并将这些封地分封给诸吕,立诸吕为王,还将齐国分割成四部分。忠臣上书劝谏,高后昏乱并不听从。如今高后已经去世,皇帝尚且年幼,不能统治天下,自然要仰仗各位大臣和诸侯王。如今诸吕又擅自提高自己的官职,聚集军队显示自己的权威,劫持列侯和忠臣,假传皇帝诏令来号令天下,刘氏的宗庙因而十分危险。如今我带领军队入关诛杀那些不应当成为王的人。”

\begin{yuanwen}
汉闻齐发兵而西,相国吕产乃遣大将军灌婴东击之。灌婴至荥阳,乃谋曰:“诸吕将兵居关中,欲危刘氏而自立。我今破齐还报,是益吕氏资也。”

乃留兵屯荥阳,使使喻齐王及诸侯,与连和,以待吕氏之变而共诛之。齐王闻之,乃西取其故济南郡,亦屯兵于齐西界以待约。
\end{yuanwen}

汉朝得知齐国发兵西进的事情后,相国吕产便派大将军灌婴领兵向东截击齐军。灌婴到达荥阳后,就谋划说:“诸吕现在率领军队盘踞在关中,想要危害刘氏而自立为帝。我如今若是打败齐军回朝报捷,就是给吕氏增加了资本。”

于是他停止进军,屯兵荥阳,派使臣通告齐王和诸侯,自己愿意与他们联合,以等待吕氏叛乱,然后共同诛灭他们。齐王听说这件事后,就立即向西进军夺回原本属于齐国的济南郡,也屯兵于齐国的西界等候,准备依约进军。

\begin{yuanwen}
吕禄、吕产欲作乱关中,硃虚侯与太尉勃、丞相平等诛之。硃虚侯首先斩吕产,于是太尉勃等乃得尽诛诸吕。而琅邪王亦从齐至长安。
\end{yuanwen}

吕禄、吕产想在关中发动叛乱,朱虚侯与太尉周勃、丞相陈平等将他们诛杀。朱虚侯首先杀死了吕产,于是太尉周勃等人才能够杀光诸吕。而琅邪王这时也从齐国赶到了长安。

\begin{yuanwen}
大臣议欲立齐王,而琅邪王及大臣曰:“齐王母家驷钧\footnote{齐王母亲的弟弟,即齐王刘襄之舅。},恶戾\footnote{凶恶乖戾。},虎而冠者也。方以吕氏故几乱天下,今又立齐王,是欲复为吕氏也。代王母家薄氏,君子长者;且代王又亲高帝子,于今见在,且最为长。以子则顺,以善人则大臣安。”

于是大臣乃谋迎立代王,而遣硃虚侯以诛吕氏事告齐王,令罢兵。
\end{yuanwen}

大臣们商议想要拥立齐王为皇帝,琅邪王却对大臣们说:“齐王的母亲驷钧,凶恶乖戾,就像戴着帽子的老虎。我们刚刚因为吕氏而差点造成天下大乱,如今又拥立齐王为皇帝的话,那就是想让他成为又一个吕氏。代王的母亲薄氏,是一个正直忠厚的人。而且代王又是高帝的亲生儿子,如今仍然健在,最为年长。儿子继位便名正言顺,拥立善良的人做皇帝那么我们这些大臣就会安心。”

于是大臣们就谋划迎立代王为皇帝,而派朱虚侯将已经把吕氏诛灭的消息通知齐王,命令他罢兵回封国。

\begin{yuanwen}
灌婴在荥阳,闻魏勃本教齐王反,既诛吕氏,罢齐兵,使使召责问魏勃。勃曰:“失火之家,岂暇先言大人而后救火乎!”

因退立,股战而栗,恐不能言者,终无他语。灌将军熟视笑曰:“人谓魏勃勇,妄庸人耳,何能为乎!”

乃罢魏勃。

魏勃父以善鼓琴见秦皇帝。及魏勃少时,欲求见齐相曹参,家贫无以自通,乃常独早夜埽齐相舍人门外。相舍人怪之,以为物,而伺之,得勃。

勃曰:“原见相君,无因,故为子埽,欲以求见。”

于是舍人见勃曹参,因以为舍人。一为参御,言事,参以为贤,言之齐悼惠王。悼惠王召见,则拜为内史。始,悼惠王得自置二千石。及悼惠王卒而哀王立,勃用事,重于齐相。
\end{yuanwen}

灌婴在荥阳,听说魏勃本来是教唆齐王叛变的,诛灭吕氏后,齐国收了兵,派人召来魏勃并责问他。魏勃说:“失火的人家,怎么会有时间先告诉家长然后再去救火呢!”

然后退立在那里,双腿颤栗,惊恐得说不能说话,一直到最后也没有说出其他的话。灌将军看了魏勃很长时间,然后笑着说:“人们都说魏勃勇敢,其实也就是个平庸的人罢了,怎么能有所作为呢!”

于是罢免了魏勃的官职。

魏勃的父亲因为善于弹琴而见过秦朝时候的皇帝。等到魏勃年少时,想要求见齐相曹参,可是家境贫穷没有财力打通关系,就时常一人半夜三更的时候在齐相舍人的门外打扫。齐相的舍人觉得很奇怪,以为是怪物出没,因而暗中观察,才发现了魏勃。

魏勃说:“我想要拜见曹相君,却没有什么门路,因此为你打扫,想要通过这种方法求见曹相君。”

于是,舍人引魏勃去见曹参,魏勃因而被曹参收为舍人。一次,他为曹参赶车的时候,和曹参谈论事情,曹参认为他很贤能,就向齐悼惠王说起他。悼惠王召见魏勃,就委任他为内史。开始时,悼惠王有权自己任命一个俸禄为二千石的官吏。等到悼惠王去世,哀王继位,魏勃当权,他的权势比齐国的相君还大。

王应麟:「《周礼》:宫伯掌王宫之士庶子。汉诸侯子入宿卫,齐王之子章是也。入京师受学,楚王之子郢客是也,其制犹古。」

吕祖谦:「齐讨吕氏,而谓之反者,高帝虽有天下共诛之之约,自汉朝言之,终为诸侯擅举兵也,况废少帝之谋,灌婴必不兴,安得不责问之乎?齐所以肯罢兵也,正以灌婴重兵在荥阳耳。」

\begin{yuanwen}
王既罢兵归,而代王来立,是为孝文帝。

孝文帝元年,尽以高后时所割齐之城阳、琅邪、济南郡复与齐,而徙琅邪王王燕,益封\footnote{加封,指增加领地。}硃虚侯、东牟侯各二千户。

是岁,齐哀王卒,太子(则)立,是为文王。
\end{yuanwen}

齐王罢兵返回齐国,代王被拥立为皇帝,就是孝文帝。

孝文帝元年(前179年),孝文帝将高后时期从齐国割出的城阳、琅邪、济南郡又重新归还给齐国,改封琅邪王为燕王,增加朱虚侯、东牟侯领地各二千户。

这一年,齐哀王去世,太子刘则继位,就是齐文王。

\begin{yuanwen}
齐文王元年,汉以齐之城阳郡立硃虚侯为城阳王,以齐济北郡立东牟侯为济北王。

二年,济北王反,汉诛杀之,地入于汉。

后二年,孝文帝尽封齐悼惠王子(罢/罷)军等七人皆为列侯。

齐文王立十四年卒,无子,国除,地入于汉。
\end{yuanwen}

齐文王元年(前178年),汉朝廷划出齐国的城阳郡给朱虚侯,立他为城阳王;划出齐国的济北郡给东牟侯,立他为济北王。

齐文王二年(前177年),济北王发动叛乱,汉朝廷将他诛杀,其封地重归汉室所有。

过了两年,孝文帝将齐悼惠王的儿子刘罷军等七人全部封为列侯。

齐文王在位十四年后去世,他没有儿子,因而国号被废除,封地收归汉室所有。

\begin{yuanwen}
后一岁,孝文帝以所封悼惠王子分齐为王,齐孝王将闾\footnote{刘肥之子,刘襄、刘章之弟。}以悼惠王子杨虚侯为齐王。故齐别郡尽以王悼惠王子:子志为济北王,子辟光为济南王,子贤为菑川王,子卬为胶西王,子雄渠为胶东王,与城阳、齐凡七王。
\end{yuanwen}

一年后,孝文帝将所封的悼惠王的几个儿子晋升为王,将齐国分割成几份分封给他们。齐孝王将闾是以悼惠王的儿子杨虚侯的身份晋升为齐王的。所以齐国其他的郡全都分给了悼惠王的儿子,让他们称王:儿子刘志被封为济北王,儿子刘辟光被封为济南王,儿子刘贤为被封为菑川王,儿子刘卬被封为胶西王,儿子刘雄渠被封为胶东王,加上城阳王、齐王,一共有七王。

\begin{yuanwen}
齐孝王十一年,吴王濞、楚王戊反,兴兵西,告诸侯曰“将诛汉贼臣晁错以安宗庙”。胶西、胶东、菑川、济南皆擅发兵应吴楚。欲与齐,齐孝王狐疑,城守不听,三国兵共围齐。齐王使路中大夫告于天子。天子复令路中大夫还告齐王:“善坚守,吾兵今破吴楚矣。”路中大夫至,三国兵围临菑数重,无从入。三国将劫与路中大夫盟,曰:“若反言汉已破矣,齐趣下三国,不且见屠。”路中大夫既许之,至城下,望见齐王,曰:“汉已发兵百万,使太尉周亚夫击破吴楚,方引兵救齐,齐必坚守无下!”三国将诛路中大夫。
\end{yuanwen}

\begin{yuanwen}\end{yuanwen}\begin{yuanwen}\end{yuanwen}\begin{yuanwen}\end{yuanwen}\begin{yuanwen}\end{yuanwen}\begin{yuanwen}\end{yuanwen}\begin{yuanwen}\end{yuanwen}\begin{yuanwen}\end{yuanwen}\begin{yuanwen}\end{yuanwen}\begin{yuanwen}\end{yuanwen}\begin{yuanwen}\end{yuanwen}\begin{yuanwen}\end{yuanwen}\begin{yuanwen}\end{yuanwen}\begin{yuanwen}\end{yuanwen}\begin{yuanwen}\end{yuanwen}\begin{yuanwen}\end{yuanwen}\begin{yuanwen}\end{yuanwen}\begin{yuanwen}\end{yuanwen}\begin{yuanwen}\end{yuanwen}\begin{yuanwen}\end{yuanwen}\begin{yuanwen}\end{yuanwen}\begin{yuanwen}\end{yuanwen}\begin{yuanwen}\end{yuanwen}\begin{yuanwen}\end{yuanwen}\begin{yuanwen}\end{yuanwen}\begin{yuanwen}\end{yuanwen}\begin{yuanwen}\end{yuanwen}\begin{yuanwen}\end{yuanwen}\begin{yuanwen}\end{yuanwen}\begin{yuanwen}\end{yuanwen}\begin{yuanwen}\end{yuanwen}\begin{yuanwen}\end{yuanwen}\begin{yuanwen}\end{yuanwen}\begin{yuanwen}\end{yuanwen}\begin{yuanwen}\end{yuanwen}\begin{yuanwen}\end{yuanwen}\begin{yuanwen}\end{yuanwen}\begin{yuanwen}\end{yuanwen}\begin{yuanwen}\end{yuanwen}\begin{yuanwen}\end{yuanwen}\begin{yuanwen}\end{yuanwen}\begin{yuanwen}\end{yuanwen}\begin{yuanwen}\end{yuanwen}\begin{yuanwen}\end{yuanwen}\begin{yuanwen}\end{yuanwen}\begin{yuanwen}\end{yuanwen}\begin{yuanwen}\end{yuanwen}\begin{yuanwen}\end{yuanwen}\begin{yuanwen}\end{yuanwen}\begin{yuanwen}\end{yuanwen}\begin{yuanwen}\end{yuanwen}\begin{yuanwen}\end{yuanwen}\begin{yuanwen}\end{yuanwen}\begin{yuanwen}\end{yuanwen}\begin{yuanwen}\end{yuanwen}\begin{yuanwen}\end{yuanwen}\begin{yuanwen}\end{yuanwen}\begin{yuanwen}\end{yuanwen}\begin{yuanwen}\end{yuanwen}\begin{yuanwen}\end{yuanwen}\begin{yuanwen}\end{yuanwen}\begin{yuanwen}\end{yuanwen}\begin{yuanwen}\end{yuanwen}\begin{yuanwen}\end{yuanwen}\begin{yuanwen}\end{yuanwen}\begin{yuanwen}\end{yuanwen}\begin{yuanwen}\end{yuanwen}\begin{yuanwen}\end{yuanwen}\begin{yuanwen}\end{yuanwen}\begin{yuanwen}\end{yuanwen}\begin{yuanwen}\end{yuanwen}\begin{yuanwen}\end{yuanwen}\begin{yuanwen}\end{yuanwen}\begin{yuanwen}\end{yuanwen}\begin{yuanwen}\end{yuanwen}\begin{yuanwen}\end{yuanwen}\begin{yuanwen}\end{yuanwen}\begin{yuanwen}\end{yuanwen}\begin{yuanwen}\end{yuanwen}
\begin{yuanwen}

齐初围急,阴与三国通谋,约未定,会闻路中大夫从汉来,喜,及其大臣乃复劝王毋下三国。居无何,汉将栾布、平阳侯等兵至齐,击破三国兵,解齐围。已而复闻齐初与三国有谋,将欲移兵伐齐。齐孝王惧,乃饮药自杀。景帝闻之,以为齐首善,以迫劫有谋,非其罪也,乃立孝王太子寿为齐王,是为懿王,续齐后。而胶西、胶东、济南、菑川王咸诛灭,地入于汉。徙济北王王菑川。齐懿王立二十二年卒,子次景立,是为厉王。

齐厉王,其母曰纪太后。太后取其弟纪氏女为厉王后。王不爱纪氏女。太后欲其家重宠,令其长女纪翁主入王宫,正其后宫,毋令得近王,欲令爱纪氏女。王因与其姊翁主奸。

齐有宦者徐甲,入事汉皇太后。皇太后有爱女曰脩成君,脩成君非刘氏,太后怜之。脩成君有女名娥,太后欲嫁之于诸侯,宦者甲乃请使齐,必令王上书请娥。皇太后喜,使甲之齐。是时齐人主父偃知甲之使齐以取后事,亦因谓甲:“即事成,幸言偃女原得充王后宫。”甲既至齐,风以此事。纪太后大怒,曰:“王有后,后宫具备。且甲,齐贫人,急乃为宦者,入事汉,无补益,乃欲乱吾王家!且主父偃何为者?乃欲以女充后宫!”徐甲大穷,还报皇太后曰:“王已原尚娥,然有一害,恐如燕王。”燕王者,与其子昆弟奸,新坐以死,亡国,故以燕感太后。太后曰:“无复言嫁女齐事。”事浸浔闻于天子。主父偃由此亦与齐有卻。

主父偃方幸于天子,用事,因言:“齐临菑十万户,市租千金,人众殷富,巨于长安,此非天子亲弟爱子不得王此。今齐王于亲属益疏。”乃从容言:“吕太后时齐欲反,吴楚时孝王几为乱。今闻齐王与其姊乱。”于是天子乃拜主父偃为齐相,且正其事。主父偃既至齐,乃急治王后宫宦者为王通于姊翁主所者,令其辞证皆引王。王年少,惧大罪为吏所执诛,乃饮药自杀。绝无后。

是时赵王惧主父偃一出废齐,恐其渐疏骨肉,乃上书言偃受金及轻重之短。天子亦既囚偃。公孙弘言:“齐王以忧死毋后,国入汉,非诛偃无以塞天下之望。”遂诛偃。

齐厉王立五年死,毋后,国入于汉。

齐悼惠王后尚有二国,城阳及菑川。菑川地比齐。天子怜齐,为悼惠王冢园在郡,割临菑东环悼惠王冢园邑尽以予菑川,以奉悼惠王祭祀。

城阳景王章,齐悼惠王子,以硃虚侯与大臣共诛诸吕,而章身首先斩相国吕王产于未央宫。孝文帝既立,益封章二千户,赐金千斤。孝文二年,以齐之城阳郡立章为城阳王。立二年卒,子喜立,是为共王。

共王八年,徙王淮南。四年,复还王城阳。凡三十三年卒,子延立,是为顷王。

顷王二十年卒,子义立,是为敬王。敬王九年卒,子武立,是为惠王。惠王十一年卒,子顺立,是为荒王。荒王四十六年卒,子恢立,是为戴王。戴王八年卒,子景立,至建始三年,十五岁,卒。

济北王兴居,齐悼惠王子,以东牟侯助大臣诛诸吕,功少。及文帝从代来,兴居曰:“请与太仆婴入清宫。”废少帝,共与大臣尊立孝文帝。

孝文帝二年,以齐之济北郡立兴居为济北王,与城阳王俱立。立二年,反。始大臣诛吕氏时,硃虚侯功尤大,许尽以赵地王硃虚侯,尽以梁地王东牟侯。及孝文帝立,闻硃虚、东牟之初欲立齐王,故绌其功。及二年,王诸子,乃割齐二郡以王章、兴居。章、兴居自以失职夺功。章死,而兴居闻匈奴大入汉,汉多发兵,使丞相灌婴击之,文帝亲幸太原,以为天子自击胡,遂发兵反于济北。天子闻之,罢丞相及行兵,皆归长安。使棘蒲侯柴将军击破虏济北王,王自杀,地入于汉,为郡。

后十年,文帝十六年,复以齐悼惠王子安都侯志为济北王。十一年,吴楚反时,志坚守,不与诸侯合谋。吴楚已平,徙志王菑川。

济南王辟光,齐悼惠王子,以勒侯孝文十六年为济南王。十一年,与吴楚反。汉击破,杀辟光,以济南为郡,地入于汉。

菑川王贤,齐悼惠王子,以武城侯文帝十六年为菑川王。十一年,与吴楚反,汉击破,杀贤。

天子因徙济北王志王菑川。志亦齐悼惠王子,以安都侯王济北。菑川王反,毋后,乃徙济北王王菑川。凡立三十五年卒,谥为懿王。子建代立,是为靖王。二十年卒,子遗代立,是为顷王。三十六年卒,子终古立,是为思王。二十八年卒,子尚立,是为孝王。五年卒,子横立,至建始三年,十一岁,卒。

胶西王卬,齐悼惠王子,以昌平侯文帝十六年为胶西王。十一年,与吴楚反。汉击破,杀卬,地入于汉,为胶西郡。

胶东王雄渠,齐悼惠王子,以白石侯文帝十六年为胶东王。十一年,与吴楚反,汉击破,杀雄渠,地入于汉,为胶东郡。

太史公曰:诸侯大国无过齐悼惠王。以海内初定,子弟少,激秦之无尺土封,故大封同姓,以填万民之心。及后分裂,固其理也。

汉矫秦制,树屏自彊。表海大国,悉封齐王。吕后肆怒,乃献城阳。哀王嗣立,其力不量。硃虚仕汉,功大策长。东牟受赏,称乱贻殃。胶东、济北,雄渠,辟光。齐虽七国,忠孝者昌。
\end{yuanwen}

\chapter{萧相国世家}

\begin{yuanwen}
萧相国何者,沛丰人也。以文无害为沛主吏掾。

高祖为布衣时,何数以吏事护高祖。高祖为亭长,常左右之。高祖以吏繇咸阳,吏皆送奉钱三,何独以五。

秦御史监郡者与从事,常辨之。何乃给泗水卒史事,第一。秦御史欲入言徵何,何固请,得毋行。

及高祖起为沛公,何常为丞督事。沛公至咸阳,诸将皆争走金帛财物之府分之,何独先入收秦丞相御史律令图书藏之。沛公为汉王,以何为丞相。项王与诸侯屠烧咸阳而去。汉王所以具知天下戹塞,户口多少,彊弱之处,民所疾苦者,以何具得秦图书也。何进言韩信,汉王以信为大将军。语在淮阴侯事中。

汉王引兵东定三秦,何以丞相留收巴蜀,填抚谕告,使给军食。汉二年,汉王与诸侯击楚,何守关中,侍太子,治栎阳。为法令约束,立宗庙社稷宫室县邑,辄奏上,可,许以从事;即不及奏上,辄以便宜施行,上来以闻。关中事计户口转漕给军,汉王数失军遁去,何常兴关中卒,辄补缺。上以此专属任何关中事。

汉三年,汉王与项羽相距京索之间,上数使使劳苦丞相。鲍生谓丞相曰:“王暴衣露盖,数使使劳苦君者,有疑君心也。为君计,莫若遣君子孙昆弟能胜兵者悉诣军所,上必益信君。”于是何从其计,汉王大说。

汉五年,既杀项羽,定天下,论功行封。群臣争功,岁馀功不决。高祖以萧何功最盛,封为酂侯,所食邑多。功臣皆曰:“臣等身被坚执锐,多者百馀战,少者数十合,攻城略地,大小各有差。今萧何未尝有汗马之劳,徒持文墨议论,不战,顾反居臣等上,何也?”高帝曰:“诸君知猎乎?”曰:“知之。”“知猎狗乎?”曰:“知之。”高帝曰:“夫猎,追杀兽兔者狗也,而发踪指示兽处者人也。今诸君徒能得走兽耳,功狗也。至如萧何,发踪指示,功人也。且诸君独以身随我,多者两三人。今萧何举宗数十人皆随我,功不可忘也。”群臣皆莫敢言。

列侯毕已受封,及奏位次,皆曰:“平阳侯曹参身被七十创,攻城略地,功最多,宜第一。”上已桡功臣,多封萧何,至位次未有以复难之,然心欲何第一。关内侯鄂君进曰:“群臣议皆误。夫曹参虽有野战略地之功,此特一时之事。夫上与楚相距五岁,常失军亡众,逃身遁者数矣。然萧何常从关中遣军补其处,非上所诏令召,而数万众会上之乏绝者数矣。夫汉与楚相守荥阳数年,军无见粮,萧何转漕关中,给食不乏。陛下虽数亡山东,萧何常全关中以待陛下,此万世之功也。今虽亡曹参等百数,何缺于汉?汉得之不必待以全。柰何欲以一旦之功而加万世之功哉!萧何第一,曹参次之。”高祖曰:“善。”于是乃令萧何,赐带剑履上殿,入朝不趋。

上曰:“吾闻进贤受上赏。萧何功虽高,得鄂君乃益明。”于是因鄂君故所食关内侯邑封为安平侯。是日,悉封何父子兄弟十馀人,皆有食邑。乃益封何二千户,以帝尝繇咸阳时何送我独赢钱二也。

汉十一年,陈豨反,高祖自将,至邯郸。未罢,淮阴侯谋反关中,吕后用萧何计,诛淮阴侯,语在淮阴事中。上已闻淮阴侯诛,使使拜丞相何为相国,益封五千户,令卒五百人一都尉为相国卫。诸君皆贺,召平独吊。召平者,故秦东陵侯。秦破,为布衣,贫,种瓜于长安城东,瓜美,故世俗谓之“东陵瓜”,从召平以为名也。召平谓相国曰:“祸自此始矣。上暴露于外而君守于中,非被矢石之事而益君封置卫者,以今者淮阴侯新反于中,疑君心矣。夫置卫卫君,非以宠君也。原君让封勿受,悉以家私财佐军,则上心说。”相国从其计,高帝乃大喜。

汉十二年秋,黥布反,上自将击之,数使使问相国何为。相国为上在军,乃拊循勉力百姓,悉以所有佐军,如陈豨时。客有说相国曰:“君灭族不久矣。夫君位为相国,功第一,可复加哉?然君初入关中,得百姓心,十馀年矣,皆附君,常复孳孳得民和。上所为数问君者,畏君倾动关中。今君胡不多买田地,贱贳贷以自汙?上心乃安。”于是相国从其计,上乃大说。

上罢布军归,民道遮行上书,言相国贱彊买民田宅数千万。上至,相国谒。上笑曰:“夫相国乃利民!”民所上书皆以与相国,曰:“君自谢民。”相国因为民请曰:“长安地狭,上林中多空地,弃,原令民得入田,毋收为禽兽食。”上大怒曰:“相国多受贾人财物,乃为请吾苑!”乃下相国廷尉,械系之。数日,王卫尉侍,前问曰:“相国何大罪,陛下系之暴也?”上曰:“吾闻李斯相秦皇帝,有善归主,有恶自与。今相国多受贾竖金而为民请吾苑,以自媚于民,故系治之。”王卫尉曰:“夫职事苟有便于民而请之,真宰相事,陛下柰何乃疑相国受贾人钱乎!且陛下距楚数岁,陈豨、黥布反,陛下自将而往,当是时,相国守关中,摇足则关以西非陛下有也。相国不以此时为利,今乃利贾人之金乎?且秦以不闻其过亡天下,李斯之分过,又何足法哉。陛下何疑宰相之浅也。”高帝不怿。是日,使使持节赦出相国。相国年老,素恭谨,入,徒跣谢。高帝曰:“相国休矣!相国为民请苑,吾不许,我不过为桀纣主,而相国为贤相。吾故系相国,欲令百姓闻吾过也。”

何素不与曹参相能,及何病,孝惠自临视相国病,因问曰:“君即百岁后,谁可代君者?”对曰:“知臣莫如主。”孝惠曰:“曹参何如?”何顿首曰:“帝得之矣!臣死不恨矣!”

何置田宅必居穷处,为家不治垣屋。曰:“后世贤,师吾俭;不贤,毋为势家所夺。”

孝惠二年,相国何卒,谥为文终侯。

后嗣以罪失侯者四世,绝,天子辄复求何后,封续酂侯,功臣莫得比焉。

太史公曰:萧相国何于秦时为刀笔吏,录录未有奇节。及汉兴,依日月之末光,何谨守管籥,因民之疾法,顺流与之更始。淮阴、黥布等皆以诛灭,而何之勋烂焉。位冠群臣,声施后世,与闳夭、散宜生等争烈矣。

萧何为吏,文而无害。及佐兴王,举宗从沛。关中既守,转输是赖。汉军屡疲,秦兵必会。约法可久,收图可大。指兽发踪,其功实最。政称画一,居乃非泰。继绝宠勤,式旌砺带。
\end{yuanwen}

\chapter{曹相国世家}

\begin{yuanwen}
平阳侯曹参者,沛人也。秦时为沛狱掾,而萧何为主吏,居县为豪吏矣。

高祖为沛公而初起也,参以中涓从。将击胡陵、方与,攻秦监公军,大破之。东下薛,击泗水守军薛郭西。复攻胡陵,取之。徙守方与。方与反为魏,击之。丰反为魏,攻之。赐爵七大夫。击秦司马枿军砀东,破之,取砀、狐父、祁善置。又攻下邑以西,至虞,击章邯车骑。攻爰戚及亢父,先登。迁为五大夫。北救阿,击章邯军,陷陈,追至濮阳。攻定陶,取临济。南救雍丘。击李由军,破之,杀李由,虏秦候一人。秦将章邯破杀项梁也,沛公与项羽引而东。楚怀王以沛公为砀郡长,将砀郡兵。于是乃封参为执帛,号曰建成君。迁为戚公,属砀郡。

其后从攻东郡尉军,破之成武南。击王离军成阳南,复攻之杠里,大破之。追北,西至开封,击赵贲军,破之,围赵贲开封城中。西击将杨熊军于曲遇,破之,虏秦司马及御史各一人。迁为执珪。从攻阳武,下轘辕、缑氏,绝河津,还击赵贲军尸北,破之。从南攻犨,与南阳守齮战阳城郭东,陷陈,取宛,虏齮,尽定南阳郡。从西攻武关、峣关,取之。前攻秦军蓝田南,又夜击其北,秦军大破,遂至咸阳,灭秦。

项羽至,以沛公为汉王。汉王封参为建成侯。从至汉中,迁为将军。从还定三秦,初攻下辩、故道、雍、斄。击章平军于好畤南,破之,围好畤,取壤乡。击三秦军壤东及高栎,破之。复围章平,章平出好畤走。因击赵贲、内史保军,破之。东取咸阳,更名曰新城。参将兵守景陵二十日,三秦使章平等攻参,参出击,大破之。赐食邑于宁秦。参以将军引兵围章邯于废丘。以中尉从汉王出临晋关。至河内,下脩武,渡围津,东击龙且、项他定陶,破之。东取砀、萧、彭城。击项籍军,汉军大败走。参以中尉围取雍丘。王武反于黄,程处反于燕,往击,尽破之。柱天侯反于衍氏,又进破取衍氏。击羽婴于昆阳,追至叶。还攻武彊,因至荥阳。参自汉中为将军中尉,从击诸侯,及项羽败,还至荥阳,凡二岁。

高祖年,拜为假左丞相,入屯兵关中。月馀,魏王豹反,以假左丞相别与韩信东攻魏将军孙军东张,大破之。因攻安邑,得魏将王襄。击魏王于曲阳,追至武垣,生得魏王豹。取平阳,得魏王母妻子,尽定魏地,凡五十二城。赐食邑平阳。因从韩信击赵相国夏说军于邬东,大破之,斩夏说。韩信与故常山王张耳引兵下井陉,击成安君,而令参还围赵别将戚将军于邬城中。戚将军出走,追斩之。乃引兵诣敖仓汉王之所。韩信已破赵,为相国,东击齐。参以右丞相属韩信,攻破齐历下军,遂取临菑。还定济北郡,攻著、漯阴、平原、鬲、卢。已而从韩信击龙且军于上假密,大破之,斩龙且,虏其将军周兰。定齐,凡得七十馀县。得故齐王田广相田光,其守相许章,及故齐胶东将军田既。韩信为齐王,引兵诣陈,与汉王共破项羽,而参留平齐未服者。

项籍已死,天下定,汉王为皇帝,韩信徙为楚王,齐为郡。参归汉相印。高帝以长子肥为齐王,而以参为齐相国。以高祖六年赐爵列侯,与诸侯剖符,世世勿绝。食邑平阳万六百三十户,号曰平阳侯,除前所食邑。

以齐相国击陈豨将张春军,破之。黥布反,参以齐相国从悼惠王将兵车骑十二万人,与高祖会击黥布军,大破之。南至蕲,还定竹邑、相、萧、留。

参功:凡下二国,县一百二十二;得王二人,相三人,将军六人,大莫敖、郡守、司马、候、御史各一人。

孝惠帝元年,除诸侯相国法,更以参为齐丞相。参之相齐,齐七十城。天下初定,悼惠王富于春秋,参尽召长老诸生,问所以安集百姓,如齐故诸儒以百数,言人人殊,参未知所定。闻胶西有盖公,善治黄老言,使人厚币请之。既见盖公,盖公为言治道贵清静而民自定,推此类具言之。参于是避正堂,舍盖公焉。其治要用黄老术,故相齐九年,齐国安集,大称贤相。

惠帝二年,萧何卒。参闻之,告舍人趣治行,“吾将入相”。居无何,使者果召参。参去,属其后相曰:“以齐狱市为寄,慎勿扰也。”后相曰:“治无大于此者乎?”参曰:“不然。夫狱市者,所以并容也,今君扰之,奸人安所容也?吾是以先之。”

参始微时,与萧何善;及为将相,有卻。至何且死,所推贤唯参。参代何为汉相国,举事无所变更,一遵萧何约束。

择郡国吏木诎于文辞,重厚长者,即召除为丞相史。吏之言文刻深,欲务声名者,辄斥去之。日夜饮醇酒。卿大夫已下吏及宾客见参不事事,来者皆欲有言。至者,参辄饮以醇酒,间之,欲有所言,复饮之,醉而后去,终莫得开说,以为常。

相舍后园近吏舍,吏舍日饮歌呼。从吏恶之,无如之何,乃请参游园中,闻吏醉歌呼,从吏幸相国召按之。乃反取酒张坐饮,亦歌呼与相应和。

参见人之有细过,专掩匿覆盖之,府中无事。

参子窋为中大夫。惠帝怪相国不治事,以为“岂少朕与”?乃谓窋曰:“若归,试私从容问而父曰:‘高帝新弃群臣,帝富于春秋,君为相,日饮,无所请事,何以忧天下乎?’然无言吾告若也。”窋既洗沐归,间侍,自从其所谏参。参怒,而笞窋二百,曰:“趣入侍,天下事非若所当言也。”至朝时,惠帝让参曰:“与窋胡治乎?乃者我使谏君也。”参免冠谢曰:“陛下自察圣武孰与高帝?”上曰:“朕乃安敢望先帝乎!”曰:“陛下观臣能孰与萧何贤?”上曰:“君似不及也。”参曰:“陛下言之是也。且高帝与萧何定天下,法令既明,今陛下垂拱,参等守职,遵而勿失,不亦可乎?”惠帝曰:“善。君休矣!”

参为汉相国,出入三年。卒,谥懿侯。子窋代侯。百姓歌之曰:“萧何为法,若画一;曹参代之,守而勿失。载其清净,民以宁一。”

平阳侯窋,高后时为御史大夫。孝文帝立,免为侯。立二十九年卒,谥为静侯。子奇代侯,立七年卒,谥为简侯。子时代侯。时尚平阳公主,生子襄。时病疠,归国。立二十三年卒,谥夷侯。子襄代侯。襄尚卫长公主,生子宗。立十六年卒,谥为共侯。子宗代侯。征和二年中,宗坐太子死,国除。

太史公曰:曹相国参攻城野战之功所以能多若此者,以与淮阴侯俱。及信已灭,而列侯成功,唯独参擅其名。参为汉相国,清静极言合道。然百姓离秦之酷后,参与休息无为,故天下俱称其美矣。

曹参初起,为沛豪吏。始从中涓,先围善置。执珪执帛,攻城略地。衍氏既诛,昆阳失位。北禽夏说,东讨田溉。剖符定封,功无与二。市狱勿扰,清净不事。尚主平阳,代享其利。
\end{yuanwen}

\chapter{留侯世家}

\begin{yuanwen}
留侯张良者,其先韩人也。大父开地,相韩昭侯、宣惠王、襄哀王。父平,相釐王、悼惠王。悼惠王二十三年,平卒。卒二十岁,秦灭韩。良年少,未宦事韩。韩破,良家僮三百人,弟死不葬,悉以家财求客刺秦王,为韩报仇,以大父、父五世相韩故。

良尝学礼淮阳。东见仓海君。得力士,为铁椎重百二十斤。秦皇帝东游,良与客狙击秦皇帝博浪沙中,误中副车。秦皇帝大怒,大索天下,求贼甚急,为张良故也。良乃更名姓,亡匿下邳。

良尝间从容步游下邳圯上,有一老父,衣褐,至良所,直堕其履圯下,顾谓良曰:“孺子,下取履!”良鄂然,欲殴之。为其老,彊忍,下取履。父曰:“履我!”良业为取履,因长跪履之。父以足受,笑而去。良殊大惊,随目之。父去里所,复还,曰:“孺子可教矣。后五日平明,与我会此。”良因怪之,跪曰:“诺。”五日平明,良往。父已先在,怒曰:“与老人期,后,何也?”去,曰:“后五日早会。”五日鸡鸣,良往。父又先在,复怒曰:“后,何也?”去,曰:“后五日复早来。”五日,良夜未半往。有顷,父亦来,喜曰:“当如是。”出一编书,曰:“读此则为王者师矣。后十年兴。十三年孺子见我济北,穀城山下黄石即我矣。”遂去,无他言,不复见。旦日视其书,乃太公兵法也。良因异之,常习诵读之。

居下邳,为任侠。项伯常杀人,从良匿。

后十年,陈涉等起兵,良亦聚少年百馀人。景驹自立为楚假王,在留。良欲往从之,道还沛公。沛公将数千人,略地下邳西,遂属焉。沛公拜良为厩将。良数以太公兵法说沛公,沛公善之,常用其策。良为他人者,皆不省。良曰:“沛公殆天授。”故遂从之,不去见景驹。

及沛公之薛,见项梁。项梁立楚怀王。良乃说项梁曰:“君已立楚后,而韩诸公子横阳君成贤,可立为王,益树党。”项梁使良求韩成,立以为韩王。以良为韩申徒,与韩王将千馀人西略韩地,得数城,秦辄复取之,往来为游兵颍川。

沛公之从雒阳南出轘辕,良引兵从沛公,下韩十馀城,击破杨熊军。沛公乃令韩王成留守阳翟,与良俱南,攻下宛,西入武关。沛公欲以兵二万人击秦峣下军,良说曰:“秦兵尚彊,未可轻。臣闻其将屠者子,贾竖易动以利。原沛公且留壁,使人先行,为五万人具食,益为张旗帜诸山上,为疑兵,令郦食其持重宝啗秦将。”秦将果畔,欲连和俱西袭咸阳,沛公欲听之。良曰:“此独其将欲叛耳,恐士卒不从。不从必危,不如因其解击之。”沛公乃引兵击秦军,大破之。北至蓝田,再战,秦兵竟败。遂至咸阳,秦王子婴降沛公。

沛公入秦宫,宫室帷帐狗马重宝妇女以千数,意欲留居之。樊哙谏沛公出舍,沛公不听。良曰:“夫秦为无道,故沛公得至此。夫为天下除残贼,宜缟素为资。今始入秦,即安其乐,此所谓‘助桀为虐’。且‘忠言逆耳利于行,毒药苦口利于病’,原沛公听樊哙言。”沛公乃还军霸上。

项羽至鸿门下,欲击沛公,项伯乃夜驰入沛公军,私见张良,欲与俱去。良曰:“臣为韩王送沛公,今事有急,亡去不义。”乃具以语沛公。沛公大惊,曰:“为将柰何?”良曰:“沛公诚欲倍项羽邪?”沛公曰:“鲰生教我距关无内诸侯,秦地可尽王,故听之。”良曰:“沛公自度能卻项羽乎?”沛公默然良久,曰:“固不能也。今为柰何?”良乃固要项伯。项伯见沛公。沛公与饮为寿,结宾婚。令项伯具言沛公不敢倍项羽,所以距关者,备他盗也。及见项羽后解,语在项羽事中。

汉元年正月,沛公为汉王,王巴蜀。汉王赐良金百溢,珠二斗,良具以献项伯。汉王亦因令良厚遗项伯,使请汉中地。项王乃许之,遂得汉中地。汉王之国,良送至襃中,遣良归韩。良因说汉王曰:“王何不烧绝所过栈道,示天下无还心,以固项王意。”乃使良还。行,烧绝栈道。

良至韩,韩王成以良从汉王故,项王不遣成之国,从与俱东。良说项王曰:“汉王烧绝栈道,无还心矣。”乃以齐王田荣反,书告项王。项王以此无西忧汉心,而发兵北击齐。

项王竟不肯遣韩王,乃以为侯,又杀之彭城。良亡,间行归汉王,汉王亦已还定三秦矣。复以良为成信侯,从东击楚。至彭城,汉败而还。至下邑,汉王下马踞鞍而问曰:“吾欲捐关以东等弃之,谁可与共功者?”良进曰:“九江王黥布,楚枭将,与项王有郄;彭越与齐王田荣反梁地:此两人可急使。而汉王之将独韩信可属大事,当一面。即欲捐之,捐之此三人,则楚可破也。”汉王乃遣随何说九江王布,而使人连彭越。及魏王豹反,使韩信将兵击之,因举燕、代、齐、赵。然卒破楚者,此三人力也。

张良多病,未尝特将也,常为画策,时时从汉王。

汉三年,项羽急围汉王荥阳,汉王恐忧,与郦食其谋桡楚权。食其曰:“昔汤伐桀,封其后于杞。武王伐纣,封其后于宋。今秦失德弃义,侵伐诸侯社稷,灭六国之后,使无立锥之地。陛下诚能复立六国后世,毕已受印,此其君臣百姓必皆戴陛下之德,莫不乡风慕义,原为臣妾。德义已行,陛下南乡称霸,楚必敛衽而朝。”汉王曰:“善。趣刻印,先生因行佩之矣。”

食其未行,张良从外来谒。汉王方食,曰:“子房前!客有为我计桡楚权者。”其以郦生语告,曰:“于子房何如?”良曰:“谁为陛下画此计者?陛下事去矣。”汉王曰:“何哉?”张良对曰:“臣请藉前箸为大王筹之。”曰:“昔者汤伐桀而封其后于杞者,度能制桀之死命也。今陛下能制项籍之死命乎?”曰:“未能也。”“其不可一也。武王伐纣封其后于宋者,度能得纣之头也。今陛下能得项籍之头乎?”曰:“未能也。”“其不可二也。武王入殷,表商容之闾,释箕子之拘,封比干之墓。今陛下能封圣人之墓,表贤者之闾,式智者之门乎?”曰:“未能也。”“其不可三也。发钜桥之粟,散鹿台之钱,以赐贫穷。今陛下能散府库以赐贫穷乎?”曰:“未能也。”“其不可四矣。殷事已毕,偃革为轩,倒置干戈,覆以虎皮,以示天下不复用兵。今陛下能偃武行文,不复用兵乎?”曰:“未能也。”“其不可五矣。休马华山之阳,示以无所为。今陛下能休马无所用乎?”曰:“未能也。”“其不可六矣。放牛桃林之阴,以示不复输积。今陛下能放牛不复输积乎?”曰:“未能也。”“其不可七矣。且天下游士离其亲戚,弃坟墓,去故旧,从陛下游者,徒欲日夜望咫尺之地。今复六国,立韩、魏、燕、赵、齐、楚之后,天下游士各归事其主,从其亲戚,反其故旧坟墓,陛下与谁取天下乎?其不可八矣。且夫楚唯无彊,六国立者复桡而从之,陛下焉得而臣之?诚用客之谋,陛下事去矣。”汉王辍食吐哺,骂曰:“竖儒,几败而公事!”令趣销印。

汉四年,韩信破齐而欲自立为齐王,汉王怒。张良说汉王,汉王使良授齐王信印,语在淮阴事中。

其秋,汉王追楚至阳夏南,战不利而壁固陵,诸侯期不至。良说汉王,汉王用其计,诸侯皆至。语在项籍事中。

汉六年正月,封功臣。良未尝有战斗功,高帝曰:“运筹策帷帐中,决胜千里外,子房功也。自择齐三万户。”良曰:“始臣起下邳,与上会留,此天以臣授陛下。陛下用臣计,幸而时中,臣原封留足矣,不敢当三万户。”乃封张良为留侯,与萧何等俱封。

上已封大功臣二十馀人,其馀日夜争功不决,未得行封。上在雒阳南宫,从复道望见诸将往往相与坐沙中语。上曰:“此何语?”留侯曰:“陛下不知乎?此谋反耳。”上曰:“天下属安定,何故反乎?”留侯曰:“陛下起布衣,以此属取天下,今陛下为天子,而所封皆萧、曹故人所亲爱,而所诛者皆生平所仇怨。今军吏计功,以天下不足遍封,此属畏陛下不能尽封,恐又见疑平生过失及诛,故即相聚谋反耳。”上乃忧曰:“为之柰何?”留侯曰:“上平生所憎,群臣所共知,谁最甚者?”上曰:“雍齿与我故,数尝窘辱我。我欲杀之,为其功多,故不忍。”留侯曰:“今急先封雍齿以示群臣,群臣见雍齿封,则人人自坚矣。”于是上乃置酒,封雍齿为什方侯,而急趣丞相、御史定功行封。群臣罢酒,皆喜曰:“雍齿尚为侯,我属无患矣。”

刘敬说高帝曰:“都关中。”上疑之。左右大臣皆山东人,多劝上都雒阳:“雒阳东有成皋,西有殽黾,倍河,向伊雒,其固亦足恃。”留侯曰:“雒阳虽有此固,其中小,不过数百里,田地薄,四面受敌,此非用武之国也。夫关中左殽函,右陇蜀,沃野千里,南有巴蜀之饶,北有胡苑之利,阻三面而守,独以一面东制诸侯。诸侯安定,河渭漕輓天下,西给京师;诸侯有变,顺流而下,足以委输。此所谓金城千里,天府之国也,刘敬说是也。”于是高帝即日驾,西都关中。

留侯从入关。留侯性多病,即道引不食穀,杜门不出岁馀。

上欲废太子,立戚夫人子赵王如意。大臣多谏争,未能得坚决者也。吕后恐,不知所为。人或谓吕后曰:“留侯善画计筴,上信用之。”吕后乃使建成侯吕泽劫留侯,曰:“君常为上谋臣,今上欲易太子,君安得高枕而卧乎?”留侯曰:“始上数在困急之中,幸用臣筴。今天下安定,以爱欲易太子,骨肉之间,虽臣等百馀人何益。”吕泽彊要曰:“为我画计。”留侯曰:“此难以口舌争也。顾上有不能致者,天下有四人。四人者年老矣,皆以为上慢侮人,故逃匿山中,义不为汉臣。然上高此四人。今公诚能无爱金玉璧帛,令太子为书,卑辞安车,因使辩士固请,宜来。来,以为客,时时从入朝,令上见之,则必异而问之。问之,上知此四人贤,则一助也。”于是吕后令吕泽使人奉太子书,卑辞厚礼,迎此四人。四人至,客建成侯所。

汉十一年,黥布反,上病,欲使太子将,往击之。四人相谓曰:“凡来者,将以存太子。太子将兵,事危矣。”乃说建成侯曰:“太子将兵,有功则位不益太子;无功还,则从此受祸矣。且太子所与俱诸将,皆尝与上定天下枭将也,今使太子将之,此无异使羊将狼也,皆不肯为尽力,其无功必矣。臣闻‘母爱者子抱’,今戚夫人日夜待御,赵王如意常抱居前,上曰‘终不使不肖子居爱子之上’,明乎其代太子位必矣。君何不急请吕后承间为上泣言:‘黥布,天下猛将也,善用兵,今诸将皆陛下故等夷,乃令太子将此属,无异使羊将狼,莫肯为用,且使布闻之,则鼓行而西耳。上虽病,彊载辎车,卧而护之,诸将不敢不尽力。上虽苦,为妻子自彊。’”于是吕泽立夜见吕后,吕后承间为上泣涕而言,如四人意。上曰:“吾惟竖子固不足遣,而公自行耳。”于是上自将兵而东,群臣居守,皆送至灞上。留侯病,自彊起,至曲邮,见上曰:“臣宜从,病甚。楚人剽疾,原上无与楚人争锋。”因说上曰:“令太子为将军,监关中兵。”上曰:“子房虽病,彊卧而傅太子。”是时叔孙通为太傅,留侯行少傅事。

汉十二年,上从击破布军归,疾益甚,愈欲易太子。留侯谏,不听,因疾不视事。叔孙太傅称说引古今,以死争太子。上详许之,犹欲易之。及燕,置酒,太子侍。四人从太子,年皆八十有馀,须眉皓白,衣冠甚伟。上怪之,问曰:“彼何为者?”四人前对,各言名姓,曰东园公,角里先生,绮里季,夏黄公。上乃大惊,曰:“吾求公数岁,公辟逃我,今公何自从吾兒游乎?”四人皆曰:“陛下轻士善骂,臣等义不受辱,故恐而亡匿。窃闻太子为人仁孝,恭敬爱士,天下莫不延颈欲为太子死者,故臣等来耳。”上曰:“烦公幸卒调护太子。”

四人为寿已毕,趋去。上目送之,召戚夫人指示四人者曰:“我欲易之,彼四人辅之,羽翼已成,难动矣。吕后真而主矣。”戚夫人泣,上曰:“为我楚舞,吾为若楚歌。”歌曰:“鸿鹄高飞,一举千里。羽翮已就,横绝四海。横绝四海,当可柰何!虽有矰缴,尚安所施!”歌数阕,戚夫人嘘唏流涕,上起去,罢酒。竟不易太子者,留侯本招此四人之力也。

留侯从上击代,出奇计马邑下,及立萧何相国,所与上从容言天下事甚众,非天下所以存亡,故不著。留侯乃称曰:“家世相韩,及韩灭,不爱万金之资,为韩报雠彊秦,天下振动。今以三寸舌为帝者师,封万户,位列侯,此布衣之极,于良足矣。原弃人间事,欲从赤松子游耳。”乃学辟穀,道引轻身。会高帝崩,吕后德留侯,乃彊食之,曰:“人生一世间,如白驹过隙,何至自苦如此乎!”留侯不得已,彊听而食。

后八年卒,谥为文成侯。子不疑代侯。

子房始所见下邳圯上老父与太公书者,后十三年从高帝过济北,果见穀城山下黄石,取而葆祠之。留侯死,并葬黄石。每上冢伏腊,祠黄石。

留侯不疑,孝文帝五年坐不敬,国除。

太史公曰:学者多言无鬼神,然言有物。至如留侯所见老父予书,亦可怪矣。高祖离困者数矣,而留侯常有功力焉,岂可谓非天乎?上曰:“夫运筹筴帷帐之中,决胜千里外,吾不如子房。”余以为其人计魁梧奇伟,至见其图,状貌如妇人好女。盖孔子曰:“以貌取人,失之子羽。”留侯亦云。

留侯倜傥,志怀愤惋。五代相韩,一朝归汉。进履宜假,运筹神算。横阳既立,申徒作扞。灞上扶危,固陵静乱。人称三杰,辩推八难。赤松原游,白驹难绊。嗟彼雄略,曾非魁岸。
\end{yuanwen}

\chapter{陈丞相世家}

\begin{yuanwen}
陈丞相平者,阳武户牖乡人也。少时家贫,好读书,有田三十亩,独与兄伯居。伯常耕田,纵平使游学。平为人长美色。人或谓陈平曰:“贫何食而肥若是?”其嫂嫉平之不视家生产,曰:“亦食糠覈耳。有叔如此,不如无有。”伯闻之,逐其妇而弃之。

及平长,可娶妻,富人莫肯与者,贫者平亦耻之。久之,户牖富人有张负,张负女孙五嫁而夫辄死,人莫敢娶。平欲得之。邑中有丧,平贫,侍丧,以先往后罢为助。张负既见之丧所,独视伟平,平亦以故后去。负随平至其家,家乃负郭穷巷,以弊席为门,然门外多有长者车辙。张负归,谓其子仲曰:“吾欲以女孙予陈平。”张仲曰:“平贫不事事,一县中尽笑其所为,独柰何予女乎?”负曰:“人固有好美如陈平而长贫贱者乎?”卒与女。为平贫,乃假贷币以聘,予酒肉之资以内妇。负诫其孙曰:“毋以贫故,事人不谨。事兄伯如事父,事嫂如母。”平既娶张氏女,赍用益饶,游道日广。

里中社,平为宰,分肉食甚均。父老曰:“善,陈孺子之为宰!”平曰:“嗟乎,使平得宰天下,亦如是肉矣!”

陈涉起而王陈,使周市略定魏地,立魏咎为魏王,与秦军相攻于临济。陈平固已前谢其兄伯,从少年往事魏王咎于临济。魏王以为太仆。说魏王不听,人或谗之,陈平亡去。

久之,项羽略地至河上,陈平往归之,从入破秦,赐平爵卿。项羽之东王彭城也,汉王还定三秦而东,殷王反楚。项羽乃以平为信武君,将魏王咎客在楚者以往,击降殷王而还。项王使项悍拜平为都尉,赐金二十溢。居无何,汉王攻下殷。项王怒,将诛定殷者将吏。陈平惧诛,乃封其金与印,使使归项王,而平身间行杖剑亡。渡河,船人见其美丈夫独行,疑其亡将,要中当有金玉宝器,目之,欲杀平。平恐,乃解衣裸而佐刺船。船人知其无有,乃止。

平遂至修武降汉,因魏无知求见汉王,汉王召入。是时万石君奋为汉王中涓,受平谒,入见平。平等七人俱进,赐食。王曰:“罢,就舍矣。”平曰:“臣为事来,所言不可以过今日。”于是汉王与语而说之,问曰:“子之居楚何官?”曰:“为都尉。”是日乃拜平为都尉,使为参乘,典护军。诸将尽讙,曰:“大王一日得楚之亡卒,未知其高下,而即与同载,反使监护军长者!”汉王闻之,愈益幸平。遂与东伐项王。至彭城,为楚所败。引而还,收散兵至荥阳,以平为亚将,属于韩王信,军广武。

绛侯、灌婴等咸谗陈平曰:“平虽美丈夫,如冠玉耳,其中未必有也。臣闻平居家时,盗其嫂;事魏不容,亡归楚;归楚不中,又亡归汉。今日大王尊官之,令护军。臣闻平受诸将金,金多者得善处,金少者得恶处。平,反覆乱臣也,原王察之。”汉王疑之,召让魏无知。无知曰:“臣所言者,能也;陛下所问者,行也。今有尾生、孝己之行而无益处于胜负之数,陛下何暇用之乎?楚汉相距,臣进奇谋之士,顾其计诚足以利国家不耳。且盗嫂受金又何足疑乎?”汉王召让平曰:“先生事魏不中,遂事楚而去,今又从吾游,信者固多心乎?”平曰:“臣事魏王,魏王不能用臣说,故去事项王。项王不能信人,其所任爱,非诸项即妻之昆弟,虽有奇士不能用,平乃去楚。闻汉王之能用人,故归大王。臣裸身来,不受金无以为资。诚臣计画有可采者,大王用之;使无可用者,金具在,请封输官,得请骸骨。”汉王乃谢,厚赐,拜为护军中尉,尽护诸将。诸将乃不敢复言。

其后,楚急攻,绝汉甬道,围汉王于荥阳城。久之,汉王患之,请割荥阳以西以和。项王不听。汉王谓陈平曰:“天下纷纷,何时定乎?”陈平曰:“项王为人,恭敬爱人,士之廉节好礼者多归之。至于行功爵邑,重之,士亦以此不附。今大王慢而少礼,士廉节者不来;然大王能饶人以爵邑,士之顽钝嗜利无耻者亦多归汉。诚各去其两短,袭其两长,天下指麾则定矣。然大王恣侮人,不能得廉节之士。顾楚有可乱者,彼项王骨鲠之臣亚父、锺离眛、龙且、周殷之属,不过数人耳。大王诚能出捐数万斤金,行反间,间其君臣,以疑其心,项王为人意忌信谗,必内相诛。汉因举兵而攻之,破楚必矣。”汉王以为然,乃出黄金四万斤,与陈平,恣所为,不问其出入。

陈平既多以金纵反间于楚军,宣言诸将锺离眛等为项王将,功多矣,然而终不得裂地而王,欲与汉为一,以灭项氏而分王其地。项羽果意不信锺离眜等。项王既疑之,使使至汉。汉王为太牢具,举进。见楚使,即详惊曰:“吾以为亚父使,乃项王使!”复持去,更以恶草具进楚使。楚使归,具以报项王。项王果大疑亚父。亚父欲急攻下荥阳城,项王不信,不肯听。亚父闻项王疑之,乃怒曰:“天下事大定矣,君王自为之!原请骸骨归!”归未至彭城,疽发背而死。陈平乃夜出女子二千人荥阳城东门,楚因击之,陈平乃与汉王从城西门夜出去。遂入关,收散兵复东。

其明年,淮阴侯破齐,自立为齐王,使使言之汉王。汉王大怒而骂,陈平蹑汉王。汉王亦悟,乃厚遇齐使,使张子房卒立信为齐王。封平以户牖乡。用其奇计策,卒灭楚。常以护军中尉从定燕王臧荼。

汉六年,人有上书告楚王韩信反。高帝问诸将,诸将曰:“亟发兵阬竖子耳。”高帝默然。问陈平,平固辞谢,曰:“诸将云何?”上具告之。陈平曰:“人之上书言信反,有知之者乎?”曰:“未有。”曰:“信知之乎?”曰:“不知。”陈平曰:“陛下精兵孰与楚?”上曰:“不能过。”平曰:“陛下将用兵有能过韩信者乎?”上曰:“莫及也。”平曰:“今兵不如楚精,而将不能及,而举兵攻之,是趣之战也,窃为陛下危之。”上曰:“为之柰何?”平曰:“古者天子巡狩,会诸侯。南方有云梦,陛下弟出伪游云梦,会诸侯于陈。陈,楚之西界,信闻天子以好出游,其势必无事而郊迎谒。谒,而陛下因禽之,此特一力士之事耳。”高帝以为然,乃发使告诸侯会陈,“吾将南游云梦”。上因随以行。行未至陈,楚王信果郊迎道中。高帝豫具武士,见信至,即执缚之,载后车。信呼曰:“天下已定,我固当烹!”高帝顾谓信曰:“若毋声!而反,明矣!”武士反接之。遂会诸侯于陈,尽定楚地。还至雒阳,赦信以为淮阴侯,而与功臣剖符定封。

于是与平剖符,世世勿绝,为户牖侯。平辞曰:“此非臣之功也。”上曰:“吾用先生谋计,战胜剋敌,非功而何?”平曰:“非魏无知臣安得进?”上曰;“若子可谓不背本矣。”乃复赏魏无知。其明年,以护军中尉从攻反者韩王信于代。卒至平城,为匈奴所围,七日不得食。高帝用陈平奇计,使单于阏氏,围以得开。高帝既出,其计礻必,世莫得闻。

高帝南过曲逆,上其城,望见其屋室甚大,曰:“壮哉县!吾行天下,独见洛阳与是耳。”顾问御史曰:“曲逆户口几何?”对曰:“始秦时三万馀户,间者兵数起,多亡匿,今见五千户。”于是乃诏御史,更以陈平为曲逆侯,尽食之,除前所食户牖。

其后常以护军中尉从攻陈豨及黥布。凡六出奇计,辄益邑,凡六益封。奇计或颇祕,世莫能闻也。

高帝从破布军还,病创,徐行至长安。燕王卢绾反,上使樊哙以相国将兵攻之。既行,人有短恶哙者。高帝怒曰:“哙见吾病,乃冀我死也。”用陈平谋而召绛侯周勃受诏床下,曰:“陈平亟驰传载勃代哙将,平至军中即斩哙头!”二人既受诏,驰传未至军,行计之曰:“樊哙,帝之故人也,功多,且又乃吕后弟吕嬃之夫,有亲且贵,帝以忿怒故,欲斩之,则恐后悔。宁囚而致上,上自诛之。”未至军,为坛,以节召樊哙。哙受诏,即反接载槛车,传诣长安,而令绛侯勃代将,将兵定燕反县。

平行闻高帝崩,平恐吕太后及吕嬃谗怒,乃驰传先去。逢使者诏平与灌婴屯于荥阳。平受诏,立复驰至宫,哭甚哀,因奏事丧前。吕太后哀之,曰:“君劳,出休矣。”平畏谗之就,因固请得宿卫中。太后乃以为郎中令,曰:“傅教孝惠。”是后吕嬃谗乃不得行。樊哙至,则赦复爵邑。

孝惠帝六年,相国曹参卒,以安国侯王陵为右丞相,陈平为左丞相。

王陵者,故沛人,始为县豪,高祖微时,兄事陵。陵少文,任气,好直言。及高祖起沛,入至咸阳,陵亦自聚党数千人,居南阳,不肯从沛公。及汉王之还攻项籍,陵乃以兵属汉。项羽取陵母置军中,陵使至,则东乡坐陵母,欲以招陵。陵母既私送使者,泣曰:“为老妾语陵,谨事汉王。汉王,长者也,无以老妾故,持二心。妾以死送使者。”遂伏剑而死。项王怒,烹陵母。陵卒从汉王定天下。以善雍齿,雍齿,高帝之仇,而陵本无意从高帝,以故晚封,为安国侯。

安国侯既为右丞相,二岁,孝惠帝崩。高后欲立诸吕为王,问王陵,王陵曰:“不可。”问陈平,陈平曰:“可。”吕太后怒,乃详迁陵为帝太傅,实不用陵。陵怒,谢疾免,杜门竟不朝请,七年而卒。

陵之免丞相,吕太后乃徙平为右丞相,以辟阳侯审食其为左丞相。左丞相不治,常给事于中。

食其亦沛人。汉王之败彭城西,楚取太上皇、吕后为质,食其以舍人侍吕后。其后从破项籍为侯,幸于吕太后。及为相,居中,百官皆因决事。

吕嬃常以前陈平为高帝谋执樊哙,数谗曰:“陈平为相非治事,日饮醇酒,戏妇女。”陈平闻,日益甚。吕太后闻之,私独喜。面质吕嬃于陈平曰:“鄙语曰‘兒妇人口不可用’,顾君与我何如耳。无畏吕嬃之谗也。”

吕太后立诸吕为王,陈平伪听之。及吕太后崩,平与太尉勃合谋,卒诛诸吕,立孝文皇帝,陈平本谋也。审食其免相。

孝文帝立,以为太尉勃亲以兵诛吕氏,功多;陈平欲让勃尊位,乃谢病。孝文帝初立,怪平病,问之。平曰:“高祖时,勃功不如臣平。及诛诸吕,臣功亦不如勃。原以右丞相让勃。”于是孝文帝乃以绛侯勃为右丞相,位次第一;平徙为左丞相,位次第二。赐平金千斤,益封三千户。

居顷之,孝文皇帝既益明习国家事,朝而问右丞相勃曰:“天下一岁决狱几何?”勃谢曰:“不知。”问:“天下一岁钱穀出入几何?”勃又谢不知,汗出沾背,愧不能对。于是上亦问左丞相平。平曰:“有主者。”上曰:“主者谓谁?”平曰:“陛下即问决狱,责廷尉;问钱穀,责治粟内史。”上曰:“苟各有主者,而君所主者何事也?”平谢曰:“主臣!陛下不知其驽下,使待罪宰相。宰相者,上佐天子理阴阳,顺四时,下育万物之宜,外镇抚四夷诸侯,内亲附百姓,使卿大夫各得任其职焉。”孝文帝乃称善。右丞相大惭,出而让陈平曰:“君独不素教我对!”陈平笑曰:“君居其位,不知其任邪?且陛下即问长安中盗贼数,君欲彊对邪?”于是绛侯自知其能不如平远矣。居顷之,绛侯谢病请免相,陈平专为一丞相。

孝文帝二年,丞相陈平卒,谥为献侯。子共侯买代侯。二年卒,子简侯恢代侯。二十三年卒,子何代侯。二十三年,何坐略人妻,弃市,国除。

始陈平曰:“我多阴谋,是道家之所禁。吾世即废,亦已矣,终不能复起,以吾多阴祸也。”然其后曾孙陈掌以卫氏亲贵戚,原得续封陈氏,然终不得。

太史公曰:陈丞相平少时,本好黄帝、老子之术。方其割肉俎上之时,其意固已远矣。倾侧扰攘楚魏之间,卒归高帝。常出奇计,救纷纠之难,振国家之患。及吕后时,事多故矣,然平竟自脱,定宗庙,以荣名终,称贤相,岂不善始善终哉!非知谋孰能当此者乎?

曲逆穷巷,门多长者。宰肉先均,佐丧后罢。魏楚更用,腹心难假。弃印封金,刺船露裸。间行归汉,委质麾下。荥阳计全,平城围解。推陵让勃,裒多益寡。应变合权,克定宗社。
\end{yuanwen}

\chapter{绛侯周勃世家}

\begin{yuanwen}
绛侯周勃者,沛人也。其先卷人,徙沛。勃以织薄曲为生,常为人吹箫给丧事,材官引彊。

高祖之为沛公初起,勃以中涓从攻胡陵,下方与。方与反,与战,卻適。攻丰。击秦军砀东。还军留及萧。复攻砀,破之。下下邑,先登。赐爵五大夫。攻蒙、虞,取之。击章邯车骑,殿。定魏地。攻爰戚、东缗,以往至栗,取之。攻齧桑,先登。击秦军阿下,破之。追至濮阳,下甄城。攻都关、定陶,袭取宛朐,得单父令。夜袭取临济,攻张,以前至卷,破之。击李由军雍丘下。攻开封,先至城下为多。后章邯破杀项梁,沛公与项羽引兵东如砀。自初起沛还至砀,一岁二月。楚怀王封沛公号安武侯,为砀郡长。沛公拜勃为虎贲令,以令从沛公定魏地。攻东郡尉于城武,破之。击王离军,破之。攻长社,先登。攻颍阳、缑氏,绝河津。击赵贲军尸北。南攻南阳守齮,破武关、峣关。破秦军于蓝田,至咸阳,灭秦。

项羽至,以沛公为汉王。汉王赐勃爵为威武侯。从入汉中,拜为将军。还定三秦,至秦,赐食邑怀德。攻槐里、好畤,最。击赵贲、内史保于咸阳,最。北攻漆。击章平、姚卬军。西定汧。还下郿、频阳。围章邯废丘。破西丞。击盗巴军,破之。攻上邽。东守峣关。转击项籍。攻曲逆,最。还守敖仓,追项籍。籍已死,因东定楚地泗、东海郡,凡得二十二县。还守雒阳、栎阳,赐与颍侯共食锺离。以将军从高帝反者燕王臧荼,破之易下。所将卒当驰道为多。赐爵列侯,剖符世世勿绝。食绛八千一百八十户,号绛侯。

以将军从高帝击反韩王信于代,降下霍人。以前至武泉,击胡骑,破之武泉北。转攻韩信军铜鞮,破之。还,降太原六城。击韩信胡骑晋阳下,破之,下晋阳。后击韩信军于硰石,破之,追北八十里。还攻楼烦三城,因击胡骑平城下,所将卒当驰道为多。勃迁为太尉。

击陈豨,屠马邑。所将卒斩豨将军乘马絺。击韩信、陈豨、赵利军于楼烦,破之。得豨将宋最、雁门守。因转攻得云中守、丞相箕肆、将勋。定雁门郡十七县,云中郡十二县。因复击豨灵丘,破之,斩豨,得豨丞相程纵、将军陈武、都尉高肆。定代郡九县。燕王卢绾反,勃以相国代樊哙将,击下蓟,得绾大将抵、丞相偃、守陉、太尉弱、御史大夫施,屠浑都。破绾军上兰,复击破绾军沮阳。追至长城,定上谷十二县,右北平十六县,辽西、辽东二十九县,渔阳二十二县。最从高帝得相国一人,丞相二人,将军、二千石各三人;别破军二,下城三,定郡五,县七十九,得丞相、大将各一人。

勃为人木彊敦厚,高帝以为可属大事。勃不好文学,每召诸生说士,东乡坐而责之:“趣为我语。”其椎少文如此。

勃既定燕而归,高祖已崩矣,以列侯事孝惠帝。孝惠帝六年,置太尉官,以勃为太尉。十岁,高后崩。吕禄以赵王为汉上将军,吕产以吕王为汉相国,秉汉权,欲危刘氏。勃为太尉,不得入军门。陈平为丞相,不得任事。于是勃与平谋,卒诛诸吕而立孝文皇帝。其语在吕后、孝文事中。

文帝既立,以勃为右丞相,赐金五千斤,食邑万户。居月馀,人或说勃曰:“君既诛诸吕,立代王,威震天下,而君受厚赏,处尊位,以宠,久之即祸及身矣。”勃惧,亦自危,乃谢请归相印。上许之。岁馀,丞相平卒,上复以勃为丞相。十馀月,上曰:“前日吾诏列侯就国,或未能行,丞相吾所重,其率先之。”乃免相就国。

岁馀,每河东守尉行县至绛,绛侯勃自畏恐诛,常被甲,令家人持兵以见之。其后人有上书告勃欲反,下廷尉。廷尉下其事长安,逮捕勃治之。勃恐,不知置辞。吏稍侵辱之。勃以千金与狱吏,狱吏乃书牍背示之,曰“以公主为证”。公主者,孝文帝女也,勃太子胜之尚之,故狱吏教引为证。勃之益封受赐,尽以予薄昭。及系急,薄昭为言薄太后,太后亦以为无反事。文帝朝,太后以冒絮提文帝,曰:“绛侯绾皇帝玺,将兵于北军,不以此时反,今居一小县,顾欲反邪!”文帝既见绛侯狱辞,乃谢曰:“吏方验而出之。”于是使使持节赦绛侯,复爵邑。绛侯既出,曰:“吾尝将百万军,然安知狱吏之贵乎!”

绛侯复就国。孝文帝十一年卒,谥为武侯。子胜之代侯。六岁,尚公主,不相中,坐杀人,国除。绝一岁,文帝乃择绛侯勃子贤者河内守亚夫,封为条侯,续绛侯后。

条侯亚夫自未侯为河内守时,许负相之,曰:“君后三岁而侯。侯八岁为将相,持国秉,贵重矣,于人臣无两。其后九岁而君饿死。”亚夫笑曰:“臣之兄已代父侯矣,有如卒,子当代,亚夫何说侯乎?然既已贵如负言,又何说饿死?指示我。”许负指其口曰:“有从理入口,此饿死法也。”居三岁,其兄绛侯胜之有罪,孝文帝择绛侯子贤者,皆推亚夫,乃封亚夫为条侯,续绛侯后。

文帝之后六年,匈奴大入边。乃以宗正刘礼为将军,军霸上;祝兹侯徐厉为将军,军棘门;以河内守亚夫为将军,军细柳:以备胡。上自劳军。至霸上及棘门军,直驰入,将以下骑送迎。已而之细柳军,军士吏被甲,锐兵刃,彀弓弩,持满。天子先驱至,不得入。先驱曰:“天子且至!”军门都尉曰:“将军令曰‘军中闻将军令,不闻天子之诏’。”居无何,上至,又不得入。于是上乃使使持节诏将军:“吾欲入劳军。”亚夫乃传言开壁门。壁门士吏谓从属车骑曰:“将军约,军中不得驱驰。”于是天子乃按辔徐行。至营,将军亚夫持兵揖曰:“介胄之士不拜,请以军礼见。”天子为动,改容式车。使人称谢:“皇帝敬劳将军。”成礼而去。既出军门,群臣皆惊。文帝曰:“嗟乎,此真将军矣!曩者霸上、棘门军,若兒戏耳,其将固可袭而虏也。至于亚夫,可得而犯邪!”称善者久之。月馀,三军皆罢。乃拜亚夫为中尉。

孝文且崩时,诫太子曰:“即有缓急,周亚夫真可任将兵。”文帝崩,拜亚夫为车骑将军。

孝景三年,吴楚反。亚夫以中尉为太尉,东击吴楚。因自请上曰:“楚兵剽轻,难与争锋。原以梁委之,绝其粮道,乃可制。”上许之。

太尉既会兵荥阳,吴方攻梁,梁急,请救。太尉引兵东北走昌邑,深壁而守。梁日使使请太尉,太尉守便宜,不肯往。梁上书言景帝,景帝使使诏救梁。太尉不奉诏,坚壁不出,而使轻骑兵弓高侯等绝吴楚兵后食道。吴兵乏粮,饥,数欲挑战,终不出。夜,军中惊,内相攻击扰乱,至于太尉帐下。太尉终卧不起。顷之,复定。后吴奔壁东南陬,太尉使备西北。已而其精兵果奔西北,不得入。吴兵既饿,乃引而去。太尉出精兵追击,大破之。吴王濞弃其军,而与壮士数千人亡走,保于江南丹徒。汉兵因乘胜,遂尽虏之,降其兵,购吴王千金。月馀,越人斩吴王头以告。凡相攻守三月,而吴楚破平。于是诸将乃以太尉计谋为是。由此梁孝王与太尉有卻。

归,复置太尉官。五岁,迁为丞相,景帝甚重之。景帝废栗太子,丞相固争之,不得。景帝由此疏之。而梁孝王每朝,常与太后言条侯之短。

窦太后曰:“皇后兄王信可侯也。”景帝让曰:“始南皮、章武侯先帝不侯,及臣即位乃侯之。信未得封也。”窦太后曰:“人主各以时行耳。自窦长君在时,竟不得侯,死后乃其子彭祖顾得侯。吾甚恨之。帝趣侯信也!”景帝曰:“请得与丞相议之。”丞相议之,亚夫曰:“高皇帝约‘非刘氏不得王,非有功不得侯。不如约,天下共击之’。今信虽皇后兄,无功,侯之,非约也。”景帝默然而止。

其后匈奴王徐卢等五人降,景帝欲侯之以劝后。丞相亚夫曰:“彼背其主降陛下,陛下侯之,则何以责人臣不守节者乎?”景帝曰:“丞相议不可用。”乃悉封徐卢等为列侯。亚夫因谢病。景帝中三年,以病免相。

顷之,景帝居禁中,召条侯,赐食。独置大胾,无切肉,又不置櫡。条侯心不平,顾谓尚席取櫡。景帝视而笑曰:“此不足君所乎?”条侯免冠谢。上起,条侯因趋出。景帝以目送之,曰:“此怏怏者非少主臣也!”

居无何,条侯子为父买工官尚方甲楯五百被可以葬者。取庸苦之,不予钱。庸知其盗买县官器,怒而上变告子,事连汙条侯。书既闻上,上下吏。吏簿责条侯,条侯不对。景帝骂之曰:“吾不用也。”召诣廷尉。廷尉责曰:“君侯欲反邪?”亚夫曰:“臣所买器,乃葬器也,何谓反邪?”吏曰:“君侯纵不反地上,即欲反地下耳。”吏侵之益急。初,吏捕条侯,条侯欲自杀,夫人止之,以故不得死,遂入廷尉。因不食五日,呕血而死。国除。

绝一岁,景帝乃更封绛侯勃他子坚为平曲侯,续绛侯后。十九年卒,谥为共侯。子建德代侯,十三年,为太子太傅。坐酎金不善,元鼎五年,有罪,国除。

条侯果饿死。死后,景帝乃封王信为盖侯。

太史公曰:绛侯周勃始为布衣时,鄙朴人也,才能不过凡庸。及从高祖定天下,在将相位,诸吕欲作乱,勃匡国家难,复之乎正。虽伊尹、周公,何以加哉!亚夫之用兵,持威重,执坚刃,穰苴曷有加焉!足己而不学,守节不逊,终以穷困。悲夫!

绛侯佐汉,质厚敦笃。始击砀东,亦围尸北。所攻必取,所讨咸克。陈豨伏诛,臧荼破国。事居送往,推功伏德。列侯还第,太尉下狱。继相条侯,绍封平曲。惜哉贤将,父子代辱!
\end{yuanwen}

\part{卷五十八}

\chapter{梁孝王世家第二十八}

\begin{yuanwen}
梁孝王武者,孝文皇帝子也,而与孝景帝同母。母,窦太后也。
\end{yuanwen}

梁孝王刘武,是孝文皇帝的儿子,与孝景帝是同母的兄弟。刘武的母亲就是窦太后。

\begin{yuanwen}
孝文帝凡四男:长子曰太子,是为孝景帝;次子武;次子参;次子胜。

孝文帝即位二年,以武为代王,以参为太原王,以胜为梁王。

二岁,徙代王为淮阳王。以代尽与太原王,号曰代王。参立十七年,孝文后二年卒,谥为孝王。子登嗣立,是为代共王。立二十九年,元光二年卒。子义立,是为代王。十九年,汉广关,以常山为限,而徙代王王清河。清河王徙以元鼎三年也。
\end{yuanwen}

孝文帝共有四个儿子:长子就是太子,即孝景帝;次子叫刘武;三子叫刘参;四子叫刘胜。

孝文帝即位后第二年,封二子刘武为代王,封三子刘参为太原王,封四子刘胜为梁王。

过了两年,文帝改封代王为淮阳王。同时把代国全部封给太原王刘参,号称代王。刘参在位共十七年,于孝文帝后元二年(前162年)去世,谥号为孝王。他的儿子刘登继承王位,这就是代共王。代共王刘登在位二十九年,于元光二年(前133年)去世。他的儿子刘义继承王位,这就是现任代王。代王刘义即位十九年以后,正值朝廷扩充关塞,以常山为界限,将代王迁为清河王。这件事发生在元鼎三年(前114年)。

\begin{yuanwen}
初,武为淮阳王十年,而梁王胜卒,谥为梁怀王。怀王最少子,爱幸异于他子。其明年,徙淮阳王武为梁王。梁王之初王梁,孝文帝之十二年也。梁王自初王通历已十一年矣。
\end{yuanwen}

当初,就在刘武被改封为淮阳王之后的第十年,梁王刘胜亡故,谥号为梁怀王。梁怀王是孝文帝的小儿子,文帝对他的宠爱胜过其他儿子。又过了一年,文帝改封淮阳王刘武为梁王。刘武被封为梁王,发生在孝文帝十二年(前168年)。梁王从最开始被封为代王到后来改封梁王,前后有十一年的时间。

\begin{yuanwen}
梁王十四年,入朝。

十七年,十八年,比年入朝,留,其明年,乃之国。

二十一年,入朝。

二十二年,孝文帝崩。

二十四年,入朝。

二十五年,复入朝。是时上未置太子也。上与梁王燕饮,尝从容言曰:“千秋万岁后传于王。”

王辞谢。虽知非至言,然心内喜。太后亦然。
\end{yuanwen}

梁王十四年(前165年),梁王入朝。

十七年、十八年,他又连续两年入朝,并留在京城,直到第二年才回到封国。

二十一年,梁王入朝。

二十二年,孝文帝去世。

二十四年,梁王再次入朝。

二十五年,又入朝。当时,景帝还没有确立太子。他与梁王在一起宴饮的时候,曾随口说道:“我死以后,就把帝位传给你。”

梁王听了以后连忙推辞。尽管梁王知道景帝所说的不是真心话,但是心里依然很高兴。窦太后也同样十分高兴。

\begin{yuanwen}
其春,吴楚齐赵七国反。吴楚先击梁棘壁,杀数万人。梁孝王城守睢阳,而使韩安国、张羽等为大将军,以距吴楚。吴楚以梁为限,不敢过而西,与太尉亚夫等相距三月。吴楚破,而梁所破杀虏略与汉中分。明年,汉立太子。其后梁最亲,有功,又为大国,居天下膏腴地。地北界泰山,西至高阳,四十馀城,皆多大县。
\end{yuanwen}

这一年的春天,吴、楚、齐、赵等七个诸侯国发动叛乱。吴、楚两国的军队首先进攻梁国的棘壁,杀死了数万人。梁孝王镇守睢阳城,任命韩安国、张羽等为大将军,率军抵挡吴、楚叛军的攻击。吴、楚联军由于受到梁王军队的阻击,一直不敢越过梁国向西进发,与太尉周亚夫等人相持长达三个月之久。后来吴、楚联军被打败,而梁王军队所斩杀、俘虏的叛军数目与汉军一样多。第二年,朝廷立了太子。此后,梁国与朝廷的关系最为亲近,立有战功,又是大国,占据着天下最为肥沃富饶的土地。梁国的地界北到泰山,西到高阳,共有四十余座城池,多数都是较大的县城。

\begin{yuanwen}
孝王,窦太后少子也,爱之,赏赐不可胜道。于是孝王筑东苑,方三百馀里。广睢阳城七十里。大治宫室,为複道\footnote{楼阁之间搭建的架空通道。},自宫连属于平台三十馀里。得赐天子旌旗,出从千乘万骑。东西驰猎,拟于天子。出言\footnote{指天子外出,沿途戒严。}??,入言警。招延四方豪桀,自山以东游说之士。莫不毕至,齐人羊胜、公孙诡、邹阳之属。公孙诡多奇邪计,初见王,赐千金,官至中尉,梁号之曰公孙将军,梁多作兵器弩弓矛数十万,而府库金钱且百巨万,珠玉宝器多于京师。
\end{yuanwen}

梁孝王是窦太后的小儿子,窦太后对他非常宠爱,赏赐给他的财宝更是不计其数。于是梁孝王修建东苑,方圆三百多里。睢阳城扩展至七十里。他大兴土木,修筑宫殿,建造架空通道,从宫殿一直连接到平台,长达三十余里。他有朝廷赐予的天子旌旗,外出的时候跟随他的车马成千上万。射猎时东奔西驰,气势犹如天子一般。他出宫时称“ ”,入宫时称“警”。他还广纳四方的豪杰,凡是崤山以东的游说之士,没有不来投奔梁王的。像齐国的羊胜、公孙诡、邹阳等人就是这一类人。公孙诡善使奇谋邪计,他初次拜见梁王时,得到一千斤黄金的赏赐,做到中尉一职,梁国尊称他为公孙将军。梁国制造了大量兵器,弩弓、长矛多达数十万件,而府库里的金钱更是多达亿万,珠玉、宝器比京师长安还要多。

\begin{yuanwen}
二十九年十月,梁孝王入朝。景帝使使持节乘舆驷马,迎梁王于关下。既朝,上疏因留,以太后亲故。王入则侍景帝同辇,出则同车游猎,射禽兽上林中。梁之侍中、郎、谒者著\footnote{登记。}籍引\footnote{门籍,在竹牒上记出入宫门人员的相关信息。当时出入皇宫要检验门籍。}出入天子殿门,与汉宦官无异。
\end{yuanwen}

二十九年(前150年)十月,梁孝王进京朝见景帝。景帝派遣使者拿着符节乘坐四匹马拉的车,赶赴关前迎接梁孝王。朝见完毕之后,梁王上疏请求留在京城,因为他是太后至亲的缘故。梁王入宫的时候就陪着景帝同乘辇车,出宫的时候就与景帝同车游猎,在上林苑射杀鸟兽。梁国的侍中、郎官、谒者只要登记便可以在天子的宫殿自由出入,与朝廷的官员没有什么区别。

\begin{yuanwen}
十一月,上废栗太子,窦太后心欲以孝王为后嗣。大臣及袁盎等有所关说于景帝,窦太后义格\footnote{止。},亦遂不复言以梁王为嗣事由此。以事秘,世莫知。乃辞归国。
\end{yuanwen}

十一月,皇上废掉栗太子刘荣,窦太后想将梁孝王立为继承人。朝中大臣以及袁盎等人用有关确立太子的法律劝说景帝,窦太后的建议因此受阻,从此以后她也就不再向景帝提起立梁王为继承人的事了。由于这件事非常隐秘,因此世人都不知道。梁孝王便告辞返回梁国。

\begin{yuanwen}
其夏四月,上立胶东王为太子。梁王怨袁盎及议臣,乃与羊胜、公孙诡之属阴\footnote{暗中。}使人刺杀袁盎及他议臣十馀人。逐其贼\footnote{刺客。},未得也。于是天子意梁王,逐贼,果梁使之。乃遣使冠盖相望\footnote{形容使者往来不断。}于道,覆按\footnote{审查,追究。}梁,捕公孙诡、羊胜。公孙诡、羊胜匿王后宫。使者责二千石急,梁相轩丘豹及内史韩安国进谏王,王乃令胜、诡皆自杀,出之。上由此怨望于梁王。梁王恐,乃使韩安国因长公主谢罪太后,然后得释。
\end{yuanwen}

这一年夏季四月,皇上立胶东王刘彻为太子。梁王对袁盎和参与讨论继承人之事的大臣十分怨恨,便与羊胜、公孙诡等人共同谋划,暗中派人刺杀袁盎以及其他参与议论此事的十多位大臣。官府缉捕凶手,但没有抓获。于是皇上便怀疑梁王,后来抓到凶手,果然是梁王所派。于是景帝派遣使臣前后相继地去梁国反复核查,并且逮捕公孙诡、羊胜。公孙诡、羊胜躲在梁王后宫里。朝廷派来的使臣十分急迫地责问梁国年俸二千石的官员,梁国相国轩丘豹和内史韩安国一同进谏梁王,梁王这才让羊胜、公孙诡自杀,然后把二人交出来。皇上从此便对梁王产生了怨恨。梁王十分害怕,就派韩安国通过长公主向窦太后谢罪,这才得到宽恕。

\begin{yuanwen}
上怒稍解,因上书请朝。既至关,茅兰说王,使乘布车\footnote{以布为帷幔的车。},从两骑入,匿于长公主园。汉使使迎王,王已入关,车骑尽居外,不知王处。太后泣曰:“帝杀吾子!”

景帝忧恐。于是梁王伏斧质\footnote{古代一种酷刑,把人置人于铁砧上,用斧头砍斫。质,铁砧。}于阙下,谢罪,然后太后、景帝大喜,相泣,复如故。悉召王从官入关。然景帝益疏王,不同车辇矣。
\end{yuanwen}

皇上的怨气渐渐地有所消解,梁孝王于是上书请求入朝晋见。梁王到达函谷关以后,茅兰劝他乘坐布车,带着两名骑马的随从入关,躲藏在长公主的园子里。朝廷派使者迎接梁王,此时梁王已经入关,而随从车骑全都停留在关外,不知道梁王在哪里。窦太后哭着说:“皇上一定杀害了我的儿子!”

景帝也为此感到忧虑恐惧。于是,梁王背负刑具伏在宫殿门前谢罪,太后和景帝见此情景非常高兴,相对而泣,感情又和以前一样了。景帝把梁王的随从官员全部召入关中。然而景帝渐渐疏远了梁王,不再与他同乘一辆辇车了。

\begin{yuanwen}
三十五年冬,复朝。上疏欲留,上弗许。归国,意忽忽不乐。北猎良山,有献牛,足出背上,孝王恶之。六月中,病热,六日卒,谥曰孝王。
\end{yuanwen}

三十五年(前144年)冬天,梁王再次进京朝见天子。上疏表明自己打算留在京师,皇上没有批准。梁王回国以后,一直闷闷不乐。后来到北边的良山游猎,有人献了一头牛,脚长在了背上,梁王很不高兴。到了六月中旬,梁王患上热病,只过了六天就去世了,死后谥号为孝王。

\begin{yuanwen}
孝王慈孝,每闻太后病,口不能食,居不安寝,常欲留长安侍太后。太后亦爱之。及闻梁王薨,窦太后哭极哀,不食,曰:“帝果杀吾子!”

景帝哀惧,不知所为。与长公主计之,乃分梁为五国,尽立孝王男五人为王,女五人皆食汤沐邑。于是奏之太后,太后乃说,为帝加壹餐。
\end{yuanwen}

梁孝王非常孝顺,他每次听说窦太后生病就寝食难安,常想留在京城长安侍奉母后。窦太后也很宠爱他。等到听说梁王去世,窦太后哭得极为悲伤,饭也不吃,说:“皇上果然杀害了我的儿子!”

景帝又哀痛又害怕,不知如何是好。他与长公主商量这件事,于是将梁国一分为五,将梁孝王刘武的五个儿子全都封为王,五个女儿则全部获赐汤沐邑。然后把这些事情上报窦太后,太后这才高兴起来,为景帝的这种做法而加了一顿饭。

\begin{yuanwen}
梁孝王长子买为梁王,是为共王;子明为济川王;子彭离为济东王;子定为山阳王;子不识为济阴王。

孝王未死时,财以巨万计,不可胜数。及死,藏府馀黄金尚四十馀万斤,他财物称是。

梁共王三年,景帝崩。共王立七年卒,子襄立,是为平王。
\end{yuanwen}

梁孝王刘武的长子刘买即位为梁王,这就是梁共王;二子刘明被封为济川王;三子刘彭离被封为济东王;四子刘定被封为山阳王;五子刘不识被封为济阴王。

梁孝王在世的时候,财产数以亿计,数也数不清。等到他死了以后,府库里剩余的黄金还有四十多万斤,其他财产也都相当于这个数目。

梁共王三年(前141年),景帝去世。梁共王在位七年以后去世,他的儿子刘襄继承王位,这就是梁平王。

\begin{yuanwen}
梁平王襄十四年,母曰陈太后。共王母曰李太后。李太后,亲平王之大母\footnote{祖母。}也。而平王之后姓任,曰任王后。任王后甚有宠于平王襄。初,孝王在时,有罍樽\footnote{饰有云雷状花纹的酒尊。罍,léi},直\footnote{通“值”。}千金。孝王诫后世,善保罍樽,无得以与人。任王后闻而欲得罍樽。平王大母李太后曰:“先王有命,无得以罍樽与人。他物虽百巨万,犹自恣也。”

任王后绝欲得之。平王襄直使人开府取罍樽,赐任王后。李太后大怒,汉使者来,欲自言,平王襄及任王后遮止,闭门,李太后与争门,措\footnote{通“笮”,挤压。}指,遂不得见汉使者。李太后亦私与食官长及郎中尹霸等士通乱,而王与任王后以此使人风止李太后,李太后内有淫行,亦已。后病薨。病时,任后未尝请病;薨,又不持丧。
\end{yuanwen}

梁平王刘襄即位之后过了十四年,他的母亲是陈太后。梁共王的母亲是李太后。李太后是梁平王的亲祖母。梁平王的王后姓任,称为任王后。任王后很受刘襄宠爱。当初梁孝王活着的时候,有一只罍樽,价值可达千金。梁孝王曾经告诫后人,要妥善保管这只罍樽,不得把它送给别人。任王后听说以后就想得到这只罍樽。梁平王的祖母李太后说:“先王有过命令,不许将罍樽送给他人。至于别的宝物,尽管价值亿万,但还可以随便挑选。”

可是任王后特别想得到罍樽。于是,梁平王直接派人打开府库取出罍樽,将它赐给了任王后。李太后知道以后大怒,朝廷的使者来到梁国,太后就打算亲自将此事告知汉使,平王和任王后极力阻拦她,把门关上,而李太后争着要开门,结果手指夹在了门缝里,就这样,她没能见到汉使。李太后也曾暗地里与食官长和郎中尹霸等人通奸,梁平王和任王后曾经派人暗示以阻止她,李太后因为自己有过淫乱行为,于是也就作罢了。后来,李太后因病去世。李太后患病期间,任王后没有请安探望;病故以后,又没有居丧守孝。

\begin{yuanwen}
元朔中,睢阳人类犴\footnote{àn}反者,人有辱其父,而与淮阳太守客出同车。太守客出下车,类犴反杀其仇于车上而去。淮阳太守怒,以让梁二千石。二千石以下求反甚急,执反亲戚。反知国阴事,乃上变事,具告知王与大母争樽状。时丞相以下见知之,欲以伤梁长吏\footnote{地位较高的官员。《汉书·百官公卿表》记载:“秩四百石至二百石,是为长吏。”。},其书闻天子。天子下吏验问,有之。公卿请废襄为庶人。天子曰:“李太后有淫行,而梁王襄无良师傅,故陷不义。”

乃削梁八城,枭任王后首于市。梁馀尚有十城。襄立三十九年卒,谥为平王。子无伤立为梁王也。
\end{yuanwen}

元朔年间,淮阳有个人名叫类犴反,有人侮辱了他的父亲,而这个人有一次与淮阳太守的客人乘坐一辆马车外出。太守的客人下车走了,类犴反就在车上将仇人杀死,然后逃走。淮阳太守大怒,因为此事责备梁国二千石官员。二千石以下的官员搜捕类犴反非常急迫,抓捕了他的亲属。类犴反知道梁国宫中的那些隐秘之事,就向朝廷上书,详细地陈述了梁平王与祖母李太后争夺罍樽的事情。当时,朝中丞相以下的官员知道这件事以后,想借此打压梁国地位较高的官员,于是将上书呈报给天子。天子将此事交给手下的官吏查问,确实有这样的事。朝中的公卿大臣都请求将梁平王刘襄贬为庶人。天子说:“李太后有过淫乱行为,梁王刘襄没有好老师,所以才陷于如此不义的境地。”

于是下令削减梁国八个城的封地,在集市上将任王后枭首示众。此后梁国还剩下十座城。刘襄在位三十九年后去世,谥号为平王。他的儿子刘无伤立为梁王。

\begin{yuanwen}
济川王明者,梁孝王子,以桓邑侯孝景中六年为济川王。

七岁,坐射杀其中尉,汉有司请诛,天子弗忍诛,废明为庶人。迁房陵,地入于汉为郡。
\end{yuanwen}

济川王刘明,是梁孝王的次子,在孝景帝中元六年(前144年),由桓邑侯升为济川王。

七年之后,他由于射杀自己的中尉而获罪,朝廷的主管官员请求将他杀掉,天子不忍,将他废黜为庶人,迁到房陵,其封地则收归朝廷,成为朝廷直接管辖的一个郡。

\begin{yuanwen}
济东王彭离者,梁孝王子,以孝景中六年为济东王。

二十九年,彭离骄悍,无人君礼,昏暮私与其奴、亡命少年数十人行剽\footnote{犹打劫。}杀人,取财物以为好。所杀发觉者百馀人,国皆知之,莫敢夜行。所杀者子上书言。汉有司请诛,上不忍,废以为庶人,迁上庸,地入于汉,为大河郡。
\end{yuanwen}

济东王刘彭离,是梁孝王的三儿子,于孝景帝中元六年(前144年)被封为济东王。

刘彭离在位二十九年,十分骄横凶悍,根本不讲究作为人君的礼仪,夜间私自带领手下奴仆、亡命少年数十人打劫杀人,将夺取他人财物作为爱好。被他所杀害的,仅仅被发现的就有一百多人,全国上下都知道此事,再没有人敢在夜间出行。被杀者的儿子上书朝廷状告刘彭离。主管官员请求杀掉他,可是皇上不忍,于是将他废为庶人,迁到上庸,其封地收归朝廷,设为大河郡。

\begin{yuanwen}
山阳哀王定者,梁孝王子,以孝景中六年为山阳王。九年卒,无子,国除,地入于汉,为山阳郡。

济阴哀王不识者,梁孝王子,以孝景中六年为济阴王。一岁卒,无子,国除,地入于汉,为济阴郡。
\end{yuanwen}

山阳哀王刘定,是梁孝王的四儿子,于孝景帝中元六年(前144年)被封为山阳王。刘定在位九年后去世,由于没有儿子,因此封国被废除,其封地收归朝廷,设为山阳郡。

济阴哀王刘不识,是梁孝王的五儿子,于孝景帝中元六年(前144年)被封为济阴王。他在位一年便去世了,也没有儿子,因此封国被废除,封地收归朝廷所有,设为济阴郡。

\begin{yuanwen}
太史公曰:梁孝王虽以亲爱之故,王膏腴之地,然会汉家隆盛,百姓殷富,故能植其财货,广宫室,车服拟于天子。然亦僭矣。
\end{yuanwen}

太史公说:梁孝王刘武虽然凭借自己是景帝的亲兄弟、太后的爱子的缘故而被封在富饶肥沃的地区为王,然而当时正值汉朝兴盛,百姓富足,他因此得以积蓄财富,扩建宫室,车马服饰比拟天子。而这也算得上是僭越了。

\begin{yuanwen}
褚先生曰:臣为郎时,闻之于宫殿中老郎吏好事者称道之也。窃以为令梁孝王怨望,欲为不善者,事从中生。今太后,女主也,以爱少子故,欲令梁王为太子。大臣不时正言其不可状,阿意治小,私说意以受赏赐,非忠臣也。齐如魏其侯窦婴之正言也,何以有后祸?景帝与王燕见\footnote{古代帝王闲居时召见或接见臣子。},侍太后饮,景帝曰:“千秋万岁之后传王。”

太后喜说。窦婴在前,据地\footnote{以手按着地,指伏地。}言曰:“汉法之约,传子適孙,今帝何以得传弟,擅乱高帝约乎!”

于是景帝默然无声。太后意不说。
\end{yuanwen}

褚先生说:我在做郎官的时候,从宫中一些喜欢说长道短的老郎吏那里听过关于梁孝王的事。我私下里认为,促使梁孝王不满、图谋皇位的祸根来自朝廷。当今的窦太后,是汉朝的女主人,她由于宠爱小儿子,所以想让梁孝王做太子。当时大臣们不但没有及时指出这样做不合适,还一味地阿谀奉承,说些微不足道的小事,私下讨好太后以求得奖赏,不是忠臣所为。如果大臣们都像魏其侯窦婴那样直言以对,又怎么会出现后来的灾祸呢?景帝与梁孝王在闲暇时会面,侍奉窦太后饮酒,景帝曾对梁孝王说:“我死以后将把皇位传给你。”

太后听了非常高兴。当时窦婴也在席前,以手伏地道:“根据汉朝的法律,皇位应该传给儿子、传给嫡孙,如今皇上怎么能把皇位传给弟弟,擅自破坏当初高帝定下的法律呢!”

于是景帝沉默不语。窦太后心中十分不悦。

\begin{yuanwen}
故成王与小弱弟立树下,取一桐叶以与之,曰:“吾用封汝。”

周公闻之,进见曰:“天王封弟,甚善。”

成王曰:“吾直与戏耳。”

周公曰:“人主无过举,不当有戏言,言之必行之。”

于是乃封小弟以应县。是后成王没齿\footnote{指一辈子。}不敢有戏言,言必行之。《孝经》曰:“非法不言,非道不行。”此圣人之法言也。今主上不宜出好言于梁王。梁王上有太后之重,骄蹇日久,数闻景帝好言,千秋万世之后传王,而实不行。
\end{yuanwen}

过去周成王曾与幼小的弟弟站在树下,将一片梧桐叶交给弟弟,说:“我就用这片叶子封你。”

周公听说以后,进见成王说:“天子分封自己的弟弟,这很好。”

成王说:“我只不过是跟他开个玩笑而已。”

周公说:“君主不应该有不得当的举动,也不应该有戏言,凡事只要说了就必须做到。”

于是成王就把应县封给了他的小弟。从那时起,成王终生都不敢有戏言,只要说了就一定做到。《孝经》中说:“凡是不合法度的话就不应该说,不合道理的事就不应该做。”这是圣人留下的明训。当今的皇上实在不应该用好听的话向梁王许愿。梁王被窦太后所看重,骄横傲慢的性格已经形成很久了,又多次听景帝许愿说好话,死后把皇位将传给自己,而实际上却并未实行。

\begin{yuanwen}
又诸侯王朝见天子,汉法凡当四见耳。始到,入小见;到正月朔旦,奉皮荐璧玉贺正月,法见;后三日,为王置酒,赐金钱财物;后二日,复入小见,辞去。凡留长安不过二十日。小见者,燕见于禁门内,饮于省中,非士人所得入也。今梁王西朝,因留,且半岁。入与人主同辇,出与同车。示风以大言而实不与,令出怨言,谋畔逆\footnote{背叛。畔,通“叛”。},乃随而忧之,不亦远乎!非大贤人,不知退让。今汉之仪法,朝见贺正月者,常一王与四侯俱朝见,十馀岁一至。今梁王常比年入朝见,久留。鄙语\footnote{俗语。}曰“骄子不孝”,非恶言也。故诸侯王当为置良师傅,相忠言之士,如汲\footnote{jí}黯、韩长孺等,敢直言极谏,安得有患害!
\end{yuanwen}

另外,各诸侯王进京朝见,按照汉朝法律的规定,总共应当朝见四次。刚到京城时,应当入宫觐见,称为“小见”;到了正月初一的早晨,应该捧着皮垫托着璧玉向皇帝祝贺正月,称为“法见”;再过三天,皇帝为诸侯王设置酒宴,赏赐金钱财物;再过两天,诸侯王再次进宫小见,然后辞别离去。凡是诸侯王朝见天子,居留京城长安总共不超过二十天。所谓小见,就是皇帝闲暇时在宫内召见,在王宫禁地宴饮,这不是普通士人能进去的。现在梁孝王向西赶赴京城朝见皇帝,趁机在宫中逗留将近半年。入宫的时候与皇帝同乘一辆辇车,外出的时候又与皇帝同乘一车。皇帝向他许诺一些大话而实际上并未兑现,致使梁孝王口出怨言,图谋造反,于是皇帝又跟着为他担忧,这岂不是太违背道理了啊!不是大贤之人,不知道谦虚退让之理。按照当今汉朝的礼仪制度,朝见天子祝贺正月,通常是一个诸侯王与四个侯同时朝见,十几年才有一次这样的机会。如今梁孝王经常连年入京朝见,而且留居很长时间。俗话说“骄纵的孩子不知孝顺”,这句话说得一点不错。所以说,应当为诸侯王安排优秀的太傅,拜忠诚而敢于直言的人为相,就像汲黯、韩长孺等人那样,敢于直言极谏,如果这样,又怎么会有祸患发生呢!

\begin{yuanwen}
盖闻梁王西入朝,谒窦太后,燕见,与景帝俱侍坐于太后前,语言私说。太后谓帝曰:“吾闻殷道亲亲\footnote{谓亲其弟而授之。},周道尊尊\footnote{谓尊祖之正体。},其义一也。安车大驾,用梁孝王为寄。”

景帝跪席举身曰:“诺。”

罢酒出,帝召袁盎诸大臣通经术者曰:“太后言如是,何谓也?”

皆对曰:“太后意欲立梁王为帝太子。”

帝问其状,袁盎等曰:“殷道亲亲者,立弟。周道尊尊者,立子。殷道质,质者法天,亲其所亲,故立弟。周道文,文者法地,尊者敬也,敬其本始,故立长子。周道,太子死,立適孙。殷道。太子死,立其弟。”

帝曰:“于公何如?”

皆对曰:“方今汉家法周,周道不得立弟,当立子。故《春秋》所以非宋宣公。宋宣公死,不立子而与弟。弟受国死,复反之与兄之子。弟之子争之,以为我当代父后,即刺杀兄子。以故国乱,祸不绝。故《春秋》曰‘君子大居正,宋之祸宣公为之’。臣请见太后白之。”

袁盎等入见太后:“太后言欲立梁王,梁王即终,欲谁立?”

太后曰:“吾复立帝子。”

袁盎等以宋宣公不立正,生祸,祸乱后五世不绝,小不忍害大义状报太后。太后乃解说,即使梁王归就国。而梁王闻其义出于袁盎诸大臣所,怨望,使人来杀袁盎。袁盎顾之曰:“我所谓袁将军者也,公得毋误乎?”

刺者曰:“是矣!”

刺之,置其剑,剑著身。视其剑,新治。问长安中削厉工,工曰:“梁郎某子来治此剑。”

以此知而发觉之,发使者捕逐之。独梁王所欲杀大臣十馀人,文吏穷本之,谋反端颇见。太后不食,日夜泣不止。景帝甚忧之,问公卿大臣,大臣以为遣经术吏往治之,乃可解。于是遣田叔、吕季主往治之。此二人皆通经术,知大礼。来还,至霸昌厩,取火悉烧梁之反辞,但空手来对景帝。景帝曰:“何如?”

对曰:“言梁王不知也。造为之者,独其幸臣羊胜、公孙诡之属为之耳。谨以伏诛死,梁王无恙也。”

景帝喜说,曰:“急趋谒太后。”

太后闻之,立起坐餐,气平复。故曰,不通经术知古今之大礼,不可以为三公及左右近臣。少见之人,如从管中(闚/窥)天也。
\end{yuanwen}

听说梁孝王西进京城朝见时,在闲暇时间拜见窦太后,与景帝一起陪坐在太后面前,高兴地谈论家中的私事。窦太后对景帝说:“我听说殷商时期的制度是亲近兄弟,周朝的制度是尊重先祖,二者的道理其实都是一样的。我离开人世以后,就把梁孝王托付给你了。”

景帝跪在席子上挺直身子答道:“是。”

酒宴结束后景帝出去,将袁盎等通晓经术的大臣召集在一起,问道:“太后说这种话,究竟是什么意思呢?”

众大臣齐声回答道:“太后想立梁王为皇上的太子。”

景帝进一步询问其中的道理,袁盎等人回答说:“殷商王朝的制度是天子亲近自己的兄弟,就是说要传位给弟弟。周朝的制度是天子要尊重先祖,就是说要传位给儿子。殷朝崇尚质朴,所以效法上天,亲近与他们亲近的人,所以立弟弟为继承人。周朝崇尚文饰,所以效法大地,‘尊’就是‘敬’的意思,敬重其本源,所以立长子为继承人。按照周朝的制度,太子死了,就立嫡长孙为继承人。而殷朝的制度则是太子死后立他的弟弟为继承人。”

景帝问道:“诸位又如何看待此事呢?”

众大臣回答说:“如今汉朝效法周朝的制度,按照周朝的制度就不应该立弟弟,而应当立儿子。所以《春秋》根据这一点对宋宣公加以指责。宋宣公去世,没有传位给儿子,而是传给弟弟。这样弟弟就成了国君,等到弟弟死后,又把国君之位还给他哥哥的儿子。弟弟的儿子争夺君位,认为自己应当接替父亲的君位,就杀了宣公的儿子。正因为这样,宋国大乱,灾祸不断。所以《春秋》说‘君子应该遵守常道,宋国的祸患是宋宣公一手造成的’。我们请求面见太后,向她阐明这个道理。”

袁盎等人进宫面见窦太后,说道:“太后说打算立梁王为太子,那么梁王去世以后,又打算立谁呢?”

窦太后说:“我会再立当今皇上的儿子。”

袁盎等人就将当年宋宣公不立嫡长子而引发祸乱,灾难绵延五代而不能断绝,小处不忍而伤及大义的道理向太后说明。窦太后这时才明白传位给儿子的道理,因而非常高兴,随后就让梁王回到封国去了。而梁孝王听说这样的议论是由袁盎等大臣提出来的,非常憎恨他们,便派人刺杀袁盎。袁盎转过头来望着刺客,说道:“我是所谓的袁将军,你不会是认错人了吧?”

刺客说:“我杀的就是你!”

于是一剑刺中袁盎,然后把剑丢在袁盎身上。官府查看刺客留下的剑,发现是新造的。向长安城里铸造刀剑的工匠询问,工匠说:“这柄剑是梁国的一个郎官定做的。”

官府由此得到线索并发现了刺客,于是派使者前去捉拿刺客。仅梁孝王所要刺杀的大臣就有十余人,负责审理此案的官吏究本溯源,发现梁孝王谋反的意图十分明显。窦太后知道此事以后不吃饭,昼夜不停地哭泣。景帝对此十分忧虑,于是向公卿大臣询问解决的办法,大臣们认为应该派通晓经术的官吏前去查办,才可以了结此案。于是景帝派田叔和吕季主前去查办这个案子。他们两个人都通晓经术,懂得大礼。办完案子回来的时候,二人到达霸昌厩,将梁孝王谋反的供词全部烧掉,只空着两手向景帝汇报办案的情况。景帝问他们:“案子办得怎么样?”

二人答道:“梁王并不知情。真正参与此事的,只不过是他的宠臣羊胜、公孙诡等人罢了。我们严格按照法律将他们诛杀,梁王平安无事。”

景帝非常高兴,说:“赶快去面见太后。”

窦太后听说此事以后,当即坐起来吃饭,心情也恢复了平和。所以说,不通晓经术,不懂古今大礼,就绝对不能担任三公及天子的左右近臣。那些孤陋寡闻之人,就好比从管中看天一样。

\begin{yuanwen}
文帝少子,徙封于梁。太后锺爱,广筑睢阳。旌旂警跸,势拟天王。功扞吴楚,计丑孙羊。窦婴正议,袁盎劫伤。汉穷梁狱,冠盖相望。祸成骄子,致此猖狂。虽分五国,卒亦不昌。
\end{yuanwen}

\chapter{五宗世家}

\begin{yuanwen}
孝景皇帝子凡十三人为王,而母五人,同母者为宗亲。栗姬子曰荣、德、阏于。程姬子曰馀、非、端。贾夫人子曰彭祖、胜。唐姬子曰发。王夫人兒姁子曰越、寄、乘、舜。

河间献王德,以孝景帝前二年用皇子为河间王。好儒学,被服造次必于儒者。山东诸儒多从之游。

二十六年卒,子共王不害立。四年卒,子刚王基代立。十二年卒,子顷王授代立。

临江哀王阏于,以孝景帝前二年用皇子为临江王。三年卒,无后,国除为郡。

临江闵王荣,以孝景前四年为皇太子,四岁废,用故太子为临江王。

四年,坐侵庙壖垣为宫,上徵荣。荣行,祖于江陵北门。既已上车,轴折车废。江陵父老流涕窃言曰:“吾王不反矣!”荣至,诣中尉府簿。中尉郅都责讯王,王恐,自杀。葬蓝田。燕数万衔土置冢上,百姓怜之。

荣最长,死无后,国除,地入于汉,为南郡。

右三国本王皆栗姬之子也。

鲁共王馀,以孝景前二年用皇子为淮阳王。二年,吴楚反破后,以孝景前三年徙为鲁王。好治宫室苑囿狗马。季年好音,不喜辞辩。为人吃。

二十六年卒,子光代为王。初好音舆马;晚节啬,惟恐不足于财。

江都易王非,以孝景前二年用皇子为汝南王。吴楚反时,非年十五,有材力,上书原击吴。景帝赐非将军印,击吴。吴已破,二岁,徙为江都王,治吴故国,以军功赐天子旌旗。元光五年,匈奴大入汉为贼,非上书原击匈奴,上不许。非好气力,治宫观,招四方豪桀,骄奢甚。

立二十六年卒,子建立为王。七年自杀。淮南、衡山谋反时,建颇闻其谋。自以为国近淮南,恐一日发,为所并,即阴作兵器,而时佩其父所赐将军印,载天子旗以出。易王死未葬,建有所说易王宠美人淖姬,夜使人迎与奸服舍中。及淮南事发,治党与颇及江都王建。建恐,因使人多持金钱,事绝其狱。而又信巫祝,使人祷祠妄言。建又尽与其姊弟奸。事既闻,汉公卿请捕治建。天子不忍,使大臣即讯王。王服所犯,遂自杀。国除,地入于汉,为广陵郡。

胶西于王端,以孝景前三年吴楚七国反破后,端用皇子为胶西王。端为人贼戾,又阴痿,一近妇人,病之数月。而有爱幸少年为郎。为郎者顷之与后宫乱,端禽灭之,及杀其子母。数犯上法,汉公卿数请诛端,天子为兄弟之故不忍,而端所为滋甚。有司再请削其国,去太半。端心愠,遂为无訾省。府库坏漏尽,腐财物以巨万计,终不得收徙。令吏毋得收租赋。端皆去卫,封其宫门,从一门出游。数变名姓,为布衣,之他郡国。

相、二千石往者,奉汉法以治,端辄求其罪告之,无罪者诈药杀之。所以设诈究变,彊足以距谏,智足以饰非。相、二千石从王治,则汉绳以法。故胶西小国,而所杀伤二千石甚众。

立四十七年,卒,竟无男代后,国除,地入于汉,为胶西郡。

右三国本王皆程姬之子也。

赵王彭祖,以孝景前二年用皇子为广川王。赵王遂反破后,彭祖王广川。四年,徙为赵王。十五年,孝景帝崩。彭祖为人巧佞卑谄,足恭而心刻深。好法律,持诡辩以中人。彭祖多内宠姬及子孙。相、二千石欲奉汉法以治,则害于王家。是以每相、二千石至,彭祖衣皁布衣,自行迎,除二千石舍,多设疑事以作动之,得二千石失言,中忌讳,辄书之。二千石欲治者,则以此迫劫;不听,乃上书告,及汙以奸利事。彭祖立五十馀年,相、二千石无能满二岁,辄以罪去,大者死,小者刑,以故二千石莫敢治。而赵王擅权,使使即县为贾人榷会,入多于国经租税。以是赵王家多金钱,然所赐姬诸子,亦尽之矣。彭祖取故江都易王宠姬王建所盗与奸淖姬者为姬,甚爱之。

彭祖不好治宫室、禨祥,好为吏事。上书原督国中盗贼。常夜从走卒行徼邯郸中。诸使过客以彭祖险陂,莫敢留邯郸。

其太子丹与其女及同产姊奸,与其客江充有卻。充告丹,丹以故废。赵更立太子。

中山靖王胜,以孝景前三年用皇子为中山王。十四年,孝景帝崩。胜为人乐酒好内,有子枝属百二十馀人。常与兄赵王相非,曰:“兄为王,专代吏治事。王者当日听音乐声色。”赵王亦非之,曰:“中山王徒日淫,不佐天子拊循百姓,何以称为籓臣!”

立四十二年卒,子哀王昌立。一年卒,子昆侈代为中山王。

右二国本王皆贾夫人之子也。

长沙定王发,发之母唐姬,故程姬侍者。景帝召程姬,程姬有所辟,不原进,而饰侍者唐兒使夜进。上醉不知,以为程姬而幸之,遂有身。已乃觉非程姬也。及生子,因命曰发。以孝景前二年用皇子为长沙王。以其母微,无宠,故王卑湿贫国。

立二十七年卒,子康王庸立。二十八年,卒,子鲋鮈立为长沙王。

右一国本王唐姬之子也。

广川惠王越,以孝景中二年用皇子为广川王。

十二年卒,子齐立为王。齐有幸臣桑距。已而有罪,欲诛距,距亡,王因禽其宗族。距怨王,乃上书告王齐与同产奸。自是之后,王齐数上书告言汉公卿及幸臣所忠等。

胶东康王寄,以孝景中二年用皇子为胶东王。二十八年卒。淮南王谋反时,寄微闻其事,私作楼车镞矢战守备,候淮南之起。及吏治淮南之事,辞出之。寄于上最亲,意伤之,发病而死,不敢置后,于是上。寄有长子者名贤,母无宠;少子名庆,母爱幸,寄常欲立之,为不次,因有过,遂无言。上怜之,乃以贤为胶东王奉康王嗣,而封庆于故衡山地,为六安王。

胶东王贤立十四年卒,谥为哀王。子庆为王。

六安王庆,以元狩二年用胶东康王子为六安王。

清河哀王乘,以孝景中三年用皇子为清河王。十二年卒,无后,国除,地入于汉,为清河郡。

常山宪王舜,以孝景中五年用皇子为常山王。舜最亲,景帝少子,骄怠多淫,数犯禁,上常宽释之。立三十二年卒,太子勃代立为王。

初,宪王舜有所不爱姬生长男棁。棁以母无宠故,亦不得幸于王。王后脩生太子勃。王内多,所幸姬生子平、子商,王后希得幸。及宪王病甚,诸幸姬常侍病,故王后亦以妒媢不常侍病,辄归舍。医进药,太子勃不自尝药,又不宿留侍病。及王薨,王后、太子乃至。宪王雅不以长子棁为人数,及薨,又不分与财物。郎或说太子、王后,令诸子与长子棁共分财物,太子、王后不听。太子代立,又不收恤棁。棁怨王后、太子。汉使者视宪王丧,棁自言宪王病时,王后、太子不侍,及薨,六日出舍,太子勃私奸,饮酒,博戏,击筑,与女子载驰,环城过市,入牢视囚。天子遣大行骞验王后及问王勃,请逮勃所与奸诸证左,王又匿之。吏求捕勃大急,使人致击笞掠,擅出汉所疑囚者。有司请诛宪王后脩及王勃。上以脩素无行,使棁陷之罪,勃无良师傅,不忍诛。有司请废王后脩,徙王勃以家属处房陵,上许之。

勃王数月,迁于房陵,国绝。月馀,天子为最亲,乃诏有司曰:“常山宪王蚤夭,后妾不和,適孽诬争,陷于不义以灭国,朕甚闵焉。其封宪王子平三万户,为真定王;封子商三万户,为泗水王。”

真定王平,元鼎四年用常山宪王子为真定王。

泗水思王商,以元鼎四年用常山宪王子为泗水王。十一年卒,子哀王安世立。十一年卒,无子。于是上怜泗水王绝,乃立安世弟贺为泗水王。

右四国本王皆王夫人兒姁子也。其后汉益封其支子为六安王、泗水王二国。凡兒姁子孙,于今为六王。

太史公曰:高祖时诸侯皆赋,得自除内史以下,汉独为置丞相,黄金印。诸侯自除御史、廷尉正、博士,拟于天子。自吴楚反后,五宗王世,汉为置二千石,去“丞相”曰“相”,银印。诸侯独得食租税,夺之权。其后诸侯贫者或乘牛车也。

景十三子,五宗亲睦。栗姬既废,临江折轴。阏于早薨,河间儒服。馀好宫苑,端事驰逐。江都有才,中山禔福。长沙地小,胶东造镞。仁贤者代,浡乱者族。兒姁四王,分封为六。
\end{yuanwen}

\chapter{三王世家}

\begin{yuanwen}
“大司马臣去病昧死再拜上疏皇帝陛下:陛下过听,使臣去病待罪行间。宜专边塞之思虑,暴骸中野无以报,乃敢惟他议以干用事者,诚见陛下忧劳天下,哀怜百姓以自忘,亏膳贬乐,损郎员。皇子赖天,能胜衣趋拜,至今无号位师傅官。陛下恭让不恤,群臣私望,不敢越职而言。臣窃不胜犬马心,昧死原陛下诏有司,因盛夏吉时定皇子位。唯陛下幸察。臣去病昧死再拜以闻皇帝陛下。”三月乙亥,御史臣光守尚书令奏未央宫。制曰:“下御史。”

六年三月戊申朔,乙亥,御史臣光守尚书令、丞非,下御史书到,言:“丞相臣青翟、御史大夫臣汤、太常臣充、大行令臣息、太子少傅臣安行宗正事昧死上言:大司马去病上疏曰:‘陛下过听,使臣去病待罪行间。宜专边塞之思虑,暴骸中野无以报,乃敢惟他议以干用事者,诚见陛下忧劳天下,哀怜百姓以自忘,亏膳贬乐,损郎员。皇子赖天,能胜衣趋拜,至今无号位师傅官。陛下恭让不恤,群臣私望,不敢越职而言。臣窃不胜犬马心,昧死原陛下诏有司,因盛夏吉时定皇子位。唯原陛下幸察。’制曰‘下御史’。臣谨与中二千石、二千石臣贺等议:古者裂地立国,并建诸侯以承天于,所以尊宗庙重社稷也。今臣去病上疏,不忘其职,因以宣恩,乃道天子卑让自贬以劳天下,虑皇子未有号位。臣青翟、臣汤等宜奉义遵职,愚憧而不逮事。方今盛夏吉时,臣青翟、臣汤等昧死请立皇子臣闳、臣旦、臣胥为诸侯王。昧死请所立国名。”

制曰:“盖闻周封八百,姬姓并列,或子、男、附庸。礼‘支子不祭’。云并建诸侯所以重社稷,朕无闻焉。且天非为君生民也。朕之不德,海内未洽,乃以未教成者彊君连城,即股肱何劝?其更议以列侯家之。”

三月丙子,奏未央宫。“丞相臣青翟、御史大夫臣汤昧死言:臣谨与列侯臣婴齐、中二千石二千石臣贺、谏大夫博士臣安等议曰:伏闻周封八百,姬姓并列,奉承天子。康叔以祖考显,而伯禽以周公立,咸为建国诸侯,以相傅为辅。百官奉宪,各遵其职,而国统备矣。窃以为并建诸侯所以重社稷者,四海诸侯各以其职奉贡祭。支子不得奉祭宗祖,礼也。封建使守籓国,帝王所以扶德施化。陛下奉承天统,明开圣绪,尊贤显功,兴灭继绝。续萧文终之后于酂,襃厉群臣平津侯等。昭六亲之序,明天施之属,使诸侯王封君得推私恩分子弟户邑,锡号尊建百有馀国。而家皇子为列侯,则尊卑相逾,列位失序,不可以垂统于万世。臣请立臣闳、臣旦、臣胥为诸侯王。”三月丙子,奏未央宫。

制曰:“康叔亲属有十而独尊者,襃有德也。周公祭天命郊,故鲁有白牡、骍刚之牲。群公不毛,贤不肖差也。‘高山仰之,景行乡之’,朕甚慕焉。所以抑未成,家以列侯可。”

四月戊寅,奏未央宫。“丞相臣青翟、御史大夫臣汤昧死言:臣青翟等与列侯、吏二千石、谏大夫、博士臣庆等议:昧死奏请立皇子为诸侯王。制曰:‘康叔亲属有十而独尊者,襃有德也。周公祭天命郊,故鲁有白牡、骍刚之牲。群公不毛,贤不肖差也。“高山仰之,景行乡之”,朕甚慕焉。所以抑未成,家以列侯可。’臣青翟、臣汤、博士臣将行等伏闻康叔亲属有十,武王继体,周公辅成王,其八人皆以祖考之尊建为大国。康叔之年幼,周公在三公之位,而伯禽据国于鲁,盖爵命之时,未至成人。康叔后扞禄父之难,伯禽殄淮夷之乱。昔五帝异制,周爵五等,春秋三等,皆因时而序尊卑。高皇帝拨乱世反诸正,昭至德,定海内,封建诸侯,爵位二等。皇子或在襁褓而立为诸侯王,奉承天子,为万世法则,不可易。陛下躬亲仁义,体行圣德,表里文武。显慈孝之行,广贤能之路。内襃有德,外讨彊暴。极临北海,西月氏,匈奴、西域,举国奉师。舆械之费,不赋于民。虚御府之藏以赏元戎,开禁仓以振贫穷,减戍卒之半。百蛮之君,靡不乡风,承流称意。远方殊俗,重译而朝,泽及方外。故珍兽至,嘉穀兴,天应甚彰。今诸侯支子封至诸侯王,而家皇子为列侯,臣青翟、臣汤等窃伏孰计之,皆以为尊卑失序,使天下失望,不可。臣请立臣闳、臣旦、臣胥为诸侯王。”四月癸未,奏未央宫,留中不下。

“丞相臣青翟、太仆臣贺、行御史大夫事太常臣充、太子少傅臣安行宗正事昧死言:臣青翟等前奏大司马臣去病上疏言,皇子未有号位,臣谨与御史大夫臣汤、中二千石、二千石、谏大夫、博士臣庆等昧死请立皇子臣闳等为诸侯王。陛下让文武,躬自切,及皇子未教。群臣之议,儒者称其术,或誖其心。陛下固辞弗许,家皇子为列侯。臣青翟等窃与列侯臣寿成等二十七人议,皆曰以为尊卑失序。高皇帝建天下,为汉太祖,王子孙,广支辅。先帝法则弗改,所以宣至尊也。臣请令史官择吉日,具礼仪上,御史奏舆地图,他皆如前故事。”制曰:“可。”

四月丙申,奏未央宫。“太仆臣贺行御史大夫事昧死言:太常臣充言卜入四月二十八日乙巳,可立诸侯王。臣昧死奏舆地图,请所立国名。礼仪别奏。臣昧死请。”

制曰:“立皇子闳为齐王,旦为燕王,胥为广陵王。”

四月丁酉,奏未央宫。六年四月戊寅朔,癸卯,御史大夫汤下丞相,丞相下中二千石,二千石下郡太守、诸侯相,丞书从事下当用者。如律令。

“维六年四月乙巳,皇帝使御史大夫汤庙立子闳为齐王。曰:于戏,小子闳,受兹青社!朕承祖考,维稽古建尔国家,封于东土,世为汉籓辅。于戏念哉!恭朕之诏,惟命不于常。人之好德,克明显光。义之不图,俾君子怠。悉尔心,允执其中,天禄永终。厥有炋臧,乃凶于而国,害于尔躬。于戏,保国艾民,可不敬与!王其戒之。”

右齐王策。

“维六年四月乙巳,皇帝使御史大夫汤庙立子旦为燕王。曰:于戏,小子旦,受兹玄社!朕承祖考,维稽古,建尔国家,封于北土,世为汉籓辅。于戏!荤粥氏虐老兽心,侵犯寇盗,加以奸巧边萌。于戏!朕命将率徂征厥罪,万夫长,千夫长,三十有二君皆来,降期奔师。荤粥徙域,北州以绥。悉尔心,毋作怨,毋俷德,毋乃废备。非教士不得从徵。于戏,保国艾民,可不敬与!王其戒之。”

右燕王策。

“维六年四月乙巳,皇帝使御史大夫汤庙立子胥为广陵王。曰:于戏,小子胥,受兹赤社!朕承祖考,维稽古建尔国家,封于南土,世为汉籓辅。古人有言曰:‘大江之南,五湖之间,其人轻心。杨州保疆,三代要服,不及以政。’于戏!悉尔心,战战兢兢,乃惠乃顺,毋侗好轶,毋迩宵人,维法维则。书云:‘臣不作威,不作福,靡有后羞。’于戏,保国艾民,可不敬与!王其戒之。”

右广陵王策。

太史公曰:古人有言曰“爱之欲其富,亲之欲其贵”。故王者壃土建国,封立子弟,所以襃亲亲,序骨肉,尊先祖,贵支体,广同姓于天下也。是以形势彊而王室安。自古至今,所由来久矣。非有异也,故弗论箸也。燕齐之事,无足采者。然封立三王,天子恭让,群臣守义,文辞烂然,甚可观也,是以附之世家。

褚先生曰:臣幸得以文学为侍郎,好览观太史公之列传。传中称三王世家文辞可观,求其世家终不能得。窃从长老好故事者取其封策书,编列其事而传之,令后世得观贤主之指意。

盖闻孝武帝之时,同日而俱拜三子为王:封一子于齐,一子于广陵,一子于燕。各因子才力智能,及土地之刚柔,人民之轻重,为作策以申戒之。谓王:“世为汉籓辅,保国治民,可不敬与!王其戒之。”夫贤主所作,固非浅闻者所能知,非博闻彊记君子者所不能究竟其意。至其次序分绝,文字之上下,简之参差长短,皆有意,人莫之能知。谨论次其真草诏书,编于左方。令览者自通其意而解说之。

王夫人者,赵人也,与卫夫人并幸武帝,而生子闳。闳且立为王时,其母病,武帝自临问之。曰:“子当为王,欲安所置之?”王夫人曰:“陛下在,妾又何等可言者。”帝曰:“虽然,意所欲,欲于何所王之?”王夫人曰:“原置之雒阳。”武帝曰:“雒阳有武库敖仓,天下旻戹,汉国之大都也。先帝以来,无子王于雒阳者。去雒阳,馀尽可。”王夫人不应。武帝曰:“关东之国无大于齐者。齐东负海而城郭大,古时独临菑中十万户,天下膏腴地莫盛于齐者矣。”王夫人以手击头,谢曰:“幸甚。”王夫人死而帝痛之,使使者拜之曰:“皇帝谨使使太中大夫明奉璧一,赐夫人为齐王太后。”子闳王齐,年少,无有子,立,不幸早死,国绝,为郡。天下称齐不宜王云。

所谓“受此土”者,诸侯王始封者必受土于天子之社,归立之以为国社,以岁时祠之。春秋大传曰:“天子之国有泰社。东方青,南方赤,西方白,北方黑,上方黄。”故将封于东方者取青土,封于南方者取赤土,封于西方者取白土,封于北方者取黑土,封于上方者取黄土。各取其色物,裹以白茅,封以为社。此始受封于天子者也。此之为主土。主土者,立社而奉之也。“朕承祖考”,祖者先也,考者父也。“维稽古”,维者度也,念也,稽者当也,当顺古之道也。

齐地多变诈,不习于礼义,故戒之曰“恭朕之诏,唯命不可为常。人之好德,能明显光。不图于义,使君子怠慢。悉若心,信执其中,天禄长终。有过不善,乃凶于而国,而害于若身”。齐王之国,左右维持以礼义,不幸中年早夭。然全身无过,如其策意。

传曰“青采出于蓝,而质青于蓝”者,教使然也。远哉贤主,昭然独见:诫齐王以慎内;诫燕王以无作怨,无俷德;诫广陵王以慎外,无作威与福。

夫广陵在吴越之地,其民精而轻,故诫之曰“江湖之间,其人轻心。杨州葆疆,三代之时,迫要使从中国俗服,不大及以政教,以意御之而已。无侗好佚,无迩宵人,维法是则。无长好佚乐驰骋弋猎淫康,而近小人。常念法度,则无羞辱矣”。三江、五湖有鱼盐之利,铜山之富,天下所仰。故诫之曰“臣不作福”者,勿使行财币,厚赏赐,以立声誉,为四方所归也。又曰“臣不作威”者,勿使因轻以倍义也。

会孝武帝崩,孝昭帝初立,先朝广陵王胥,厚赏赐金钱财币,直三千馀万,益地百里,邑万户。

会昭帝崩,宣帝初立,缘恩行义,以本始元年中,裂汉地,尽以封广陵王胥四子:一子为朝阳侯;一子为平曲侯;一子为南利侯;最爱少子弘,立以为高密王。

其后胥果作威福,通楚王使者。楚王宣言曰:“我先元王,高帝少弟也,封三十二城。今地邑益少,我欲与广陵王共发兵云。广陵王为上,我复王楚三十二城,如元王时。”事发觉,公卿有司请行罚诛。天子以骨肉之故,不忍致法于胥,下诏书无治广陵王,独诛首恶楚王。传曰“蓬生麻中,不扶自直;白沙在泥中,与之皆黑”者,土地教化使之然也。其后胥复祝诅谋反,自杀,国除。

燕土墝埆,北迫匈奴,其人民勇而少虑,故诫之曰“荤粥氏无有孝行而禽兽心,以窃盗侵犯边民。朕诏将军往征其罪,万夫长,千夫长,三十有二君皆来,降旗奔师。荤粥徙域远处,北州以安矣”。“悉若心,无作怨”者,勿使从俗以怨望也。“无俷德”者,勿使背德也。“无废备”者,无乏武备,常备匈奴也。“非教士不得从徵”者,言非习礼义不得在于侧也。

会武帝年老长,而太子不幸薨,未有所立,而旦使来上书,请身入宿卫于长安。孝武见其书,击地,怒曰:“生子当置之齐鲁礼义之乡,乃置之燕赵,果有争心,不让之端见矣。”于是使使即斩其使者于阙下。

会武帝崩,昭帝初立,旦果作怨而望大臣。自以长子当立,与齐王子刘泽等谋为叛逆,出言曰:“我安得弟在者!今立者乃大将军子也。”欲发兵。事发觉,当诛。昭帝缘恩宽忍,抑案不扬。公卿使大臣请,遣宗正与太中大夫公户满意、御史二人,偕往使燕,风喻之。到燕,各异日,更见责王。宗正者,主宗室诸刘属籍,先见王,为列陈道昭帝实武帝子状。侍御史乃复见王,责之以正法,问:“王欲发兵罪名明白,当坐之。汉家有正法,王犯纤介小罪过,即行法直断耳,安能宽王。”惊动以文法。王意益下,心恐。公户满意习于经术,最后见王,称引古今通义,国家大礼,文章尔雅。谓王曰:“古者天子必内有异姓大夫,所以正骨肉也;外有同姓大夫,所以正异族也。周公辅成王,诛其两弟,故治。武帝在时,尚能宽王。今昭帝始立,年幼,富于春秋,未临政,委任大臣。古者诛罚不阿亲戚,故天下治。方今大臣辅政,奉法直行,无敢所阿,恐不能宽王。王可自谨,无自令身死国灭,为天下笑。”于是燕王旦乃恐惧服罪,叩头谢过。大臣欲和合骨肉,难伤之以法。

其后旦复与左将军上官桀等谋反,宣言曰“我次太子,太子不在,我当立,大臣共抑我”云云。大将军光辅政,与公卿大臣议曰:“燕王旦不改过悔正,行恶不变。”于是脩法直断,行罚诛。旦自杀,国除,如其策指。有司请诛旦妻子。孝昭以骨肉之亲,不忍致法,宽赦旦妻子,免为庶人。传曰“兰根与白芷,渐之滫中,君子不近,庶人不服”者,所以渐然也。

宣帝初立,推恩宣德,以本始元年中尽复封燕王旦两子:一子为安定侯;立燕故太子建为广阳王,以奉燕王祭祀。

三王封系,旧史烂然。褚氏后补,册书存焉。去病建议,青翟上言。天子冲挹,志在急贤。太常具礼,请立齐燕,闳国负海,旦社惟玄。宵人不迩,荤粥远边。明哉监戒,式防厥愆。
\end{yuanwen}

\part{卷六十一}

\chapter{伯夷列传第一}

方孝孺:「伯夷苟知父志欲立齐,当效太伯顺父之志,隐然退避于治命之日,不当显然辞让于乱命之余也。叔齐亦不立而逃之,幸有仲子以托国,不然其如社稷何?斯皆过乎中者也!故孔子但曰『古之贤人』,而至德之称独在太伯,厥旨深矣。」

\begin{yuanwen}
夫学者载籍极博,犹考信于六艺。《诗》《书》虽缺,然虞夏之文可知也。尧将逊位,让于虞舜,舜禹之间,岳牧咸荐,乃试之于位,典职\footnote{掌管政事。}数十年,功用既兴,然后授政。示天下重器,王者大统,传天下若斯之难也。而说者曰尧让天下于许由,许由不受,耻之逃隐。及夏之时,有卞随、务光者。此何以称焉?太史公曰:余登箕山,其上盖有许由冢云。孔子序列古之仁圣贤人,如吴太伯、伯夷之伦详矣。余以所闻由、光义至高,其文辞不少概见,何哉?
\end{yuanwen}

学者们记事的典籍尽管极为广博,然而还是要在六经中寻找根据。《诗经》《尚书》虽然残缺不全,但是记载虞、夏两个朝代的文字,依然可以得见。尧帝快要退位时,将帝位让给了虞舜,舜把帝位让给大禹的时候,四方的诸侯和州牧全都赶来推荐他们,于是让他们代理职位,主持了几十年国政之后,有了很大的功绩,这才正式把政权授给他。这说明,天下是最重要的宝器,王位是最重要的统绪,天下的传承就是如此艰难。但有人说尧帝曾经把天下禅让给许由,但是许由没有接受,还把那当成一种耻辱而逃避、隐居。到夏朝时,有卞随、务光这样的人。这些人应该从哪方面来称颂呢?太史公说:我登上箕山,有人说山上有许由的坟墓。孔子依次排列古代那些仁人、圣人、贤人,比如吴太伯、伯夷等人,说得都很详细。我认为我所听说的许由、务光等人的气节很高,但是书籍却一点简略的记载都没有,什么原因呢?

\begin{yuanwen}
孔子曰:“伯夷、叔齐,不念旧恶,怨是用希\footnote{同“稀”。}。”“求仁得仁,又何怨乎?”余悲伯夷之意,睹轶诗\footnote{指《采薇》。}可异焉。
\end{yuanwen}

孔子说:“伯夷、叔齐,不会记着别人以往的恶处,对人的怨恨因此也就变得很少。”“追求仁德就能够得到仁德,又有什么可怨恨的呢?”我对伯夷的心意感到悲伤,看到他们遗留下来的逸诗后又觉得非常惊异。

\begin{yuanwen}
其传曰:伯夷、叔齐,孤竹君之二子也。父欲立叔齐,及父卒,叔齐让伯夷。伯夷曰:“父命也。”

遂逃去。叔齐亦不肯立而逃之。国人立其中子。于是伯夷、叔齐闻西伯昌善养老,盍\footnote{何不。表示反问或疑问语气。}往归焉。及至,西伯卒,武王载木主\footnote{木制的神主牌位。},号为文王,东伐纣。伯夷、叔齐叩马而谏\footnote{勒住马头进行规劝。后比喻竭力进行劝谏。}曰:“父死不葬,爰及干戈,可谓孝乎?以臣弑君,可谓仁乎?”

左右欲兵之。太公曰:“此义人也。”

扶而去之。武王已平殷乱,天下宗周,而伯夷、叔齐耻之,义不食周粟,隐于首阳山,采薇而食之。及饿且死,作歌。其辞曰:“登彼西山兮,采其薇矣。以暴易暴兮,不知其非矣。神农、虞、夏忽焉没兮,我安適归矣?于嗟徂\footnote{cú,同“殂”,死亡。}兮,命之衰矣!”

遂饿死于首阳山。
\end{yuanwen}

他们的传记是这样写的:伯夷、叔齐,他们是孤竹国国君的两个儿子。父亲想要让叔齐做国君的继承人,等到父亲死去以后,叔齐把王位让给伯夷。伯夷说:“这是父亲的命令。”

于是伯夷就逃走了。叔齐也因为不肯继承国君之位而逃走。都城中的人们把孤竹君中间的儿子立为国君。在这种情况下,伯夷、叔齐听说西伯昌能够很好地奉养老人,就想:何不去归附他呢?等到他们到达那里的时候,西伯侯已经死去了。周武王用车载着神主,尊号为周文王,向东方行进讨伐商纣王。伯夷、叔齐拉住武王的马,向他进谏:“父亲去世,还没有安葬就发动战争,这能说是孝道吗?身为臣子,却要弑杀国君,这能说是仁义吗?”

周武王左右的人想要杀死他们。姜太公说道:“这是忠义的人。”

就扶着他们离开了。等到周武王平定了殷商的暴乱,天下人都归附周朝,以周朝为宗,只有伯夷、叔齐觉得这是一种耻辱,出于对商朝的忠义而不吃周朝的粟米粮食,隐居在首阳山,采集蕨菜充饥。等到他们饿得将要死掉时,便创作了一首诗歌。那诗歌中唱道:“爬上那座西山啊,采集那儿的野菜。以暴力手段取代残暴的统治啊,却不知道这样做是错误的。神农氏、虞舜、夏禹那样清明太平的时代很快就消失了,我最后的归宿在何处呢?可叹啊,唉,还是死了吧,我的命运是那样的不济呀!”于是,两人饿死在了首阳山。

\begin{yuanwen}
由此观之,怨邪非邪?
\end{yuanwen}

从这首诗的歌词来看,他们到底有没有怨恨呢?

陈仁锡:「子长写夷齐之怨,乃所以自写其怨,寓意颇深。孔子以夷齐无怨,而太史公作传,通篇是怨。然孔子所云无怨者兄弟逊国,而太史公所云怨者以暴易暴,之间原不相乖。」「颇似论,不似传,是太史公极得意之文,亦极变体之文。」「疑有疑无,作史慎重,彷徨追赏,言外高奇,读者自知之。」凌稚隆:「尧让许由,盖庄周寓言,眇天下为不足道耳。太史公言有许由冢,乃明其实有是人,而又曰『文辞不少慨见』,则无尧让之事,已隐然言外矣。」

\begin{yuanwen}
或曰:“天道无亲,常与善人。”

若伯夷、叔齐,可谓善人者非邪?积仁絜行如此而饿死!且七十子之徒,仲尼独荐颜渊为好学。然回也屡空\footnote{经常空乏,一无所有。借指贫穷困乏。},糟糠不厌\footnote{饱。},而卒蚤夭。天之报施善人,其何如哉?盗(蹠/跖)日杀不辜,肝人之肉,暴戾恣睢,聚党数千人横行天下,竟以寿终。是遵何德哉?此其尤大彰明较著者也。若至近世,操行不轨,专犯忌讳,而终身逸乐,富厚累世不绝。或择地而蹈之\footnote{指不仕暗君,不饮盗泉,隐于世外。},时然后出言,行不由径,非公正不发愤,而遇祸灾者,不可胜数也。余甚惑焉,傥所谓天道,是邪非邪?
\end{yuanwen}

有人曾经说过:“天意是不会偏私任何人的,只是经常帮助那些好人。”

就如同伯夷、叔齐那样的人,他们可以说是善良的人吗?他们不是善良的人吗?他们积累仁德,使自己的品行变得高洁,即使这样,最终还是饿死了!况且孔子那七十个著名的弟子中,孔子唯独称赞颜回是个喜好学习的人。可是颜回却经常平穷困顿,就算是最粗糙食物都无法吃饱,最终还是早早地死去了。上天对善良的人的报答和施与,究竟是什么样子的呢?盗跖每天都会杀死无辜的人,吃掉人身上的肉,残忍暴戾放纵凶狠,纠集了同党好几千人,在天下横行,没有忌惮,竟然能够安享自然的寿命,没有早死。这又是依照什么德行来报答和施与呢?这两个例子是尤其明显重要的。说到近代,那些操守品行不够端正,专门触犯法令的人,反而可以一辈子享受安逸和快乐,他们得到的富贵非常丰厚,代代相传,从不断绝。有人居住都要挑选地方,在该说话的时候才说话,走路不走小路,不是公平正义的事情,就不发奋去做,最终反而遭遇灾祸,这样的例子数都无法数完。我对此感到十分疑惑,倘若这就是人们所说的天道,这到底是正确的还是错误的呢?

\begin{yuanwen}
子曰:“道不同不相为谋”,亦各从其志也。故曰:“富贵如可求,虽执鞭之士,吾亦为之。如不可求,从吾所好”。“岁寒,然后知松柏之后凋”。举世混浊,清士乃见。岂以其重若彼,其轻若此哉?“君子疾没世而名不称焉。”

贾子曰:“贪夫徇财,烈士徇名,夸者死权,众庶冯生\footnote{贪生。冯,同“凭”,恃。}。”“同明相照,同类相求。”“云从龙,风从虎,圣人作而万物睹。”

伯夷、叔齐虽贤,得夫子而名益彰。颜渊虽笃学,附骥尾而行益显。岩穴之士,趣\footnote{同“趋”,指出仕。}舍有时若此类,名堙\footnote{yīn}灭而不称,悲夫!闾巷之人,欲砥行立名者,非附青云之士,恶能施于后世哉?
\end{yuanwen}

孔子曾经说“走着不同道路的人,不能在一起谋划事情”,那么也只有各人遵从各人的意志了。所以说“富有和高贵的生活假如可以求得,那么即使让我做个赶车的人,我也去做。假如无法求得,就遵从我的喜好去做”。“一年中寒冷的季节来到,这之后才明白松柏是所有植物中最后凋谢的”。整个世界都变得混浊了,那些清廉高洁的人士才会出现。这难道不是由于有的人过于重视富贵,才显得轻视富贵的人如此高尚吗?“君子最大的遗憾是离开了这个世界但却没有美好的名声受人称道”。

贾子说:“贪婪的人为了财物而死,英烈的人为了名声而死,贪权势而矜夸的人为了权势而死,普通百姓只顾自己的生命。”“同为明亮的东西,就会互相照映;都是同一类的事物,就会相互间产生影响。”“龙出现时,一定有云彩跟随在后,猛虎怒吼时,也一定会刮起大风,当圣人出现在世间时,天下万物都会变得显著起来而易于理解。”

伯夷、叔齐虽然也是贤明的人,但也要有孔子的称赞才使他们的名声更加显著。颜回虽然刻苦好学,因为跟随夫子,才使他的德行更加彰明。居住在山林洞穴的隐士,有的名声显达,有的则默默无闻,这都与时机有很大的关系,这样的人如果名声被湮没而没有得到称颂,真可以说是一件悲哀的事情啊!居住在乡里的人,要想磨砺自己的德行、树立美好的名声,假如不是依附于那些青云之上的人士身后,怎么可以让自己的声名广泛地流传于后世呢?

\begin{yuanwen}
天道平分,与善徒云。贤而饿死,盗且聚群。吉凶倚伏,报施纠纷。子罕言命,得自前闻。嗟彼素士,不附青云!
\end{yuanwen}

\part{卷六十二}

\chapter{管晏列传第二}

陈仁锡:「管仲实赖鲍叔之荐,而晏子能荐贤,此连意全在空处也。先看其简处,次玩其简逸处。」

\begin{yuanwen}
管仲夷吾者,颍上人也。少时常与鲍叔牙游,鲍叔知其贤。管仲贫困,常欺鲍叔,鲍叔终善遇之,不以为言。已而鲍叔事齐公子小白,管仲事公子纠。及小白立为,桓公,公子纠死,管仲囚焉。鲍叔遂进管仲。管仲既用,任政于齐,齐桓公以霸,九合诸侯,一匡天下,管仲之谋也。
\end{yuanwen}

管仲的名字叫夷吾,是颍上人。年轻的时候,他经常与鲍叔牙交游往来,鲍叔牙知道他是个很贤能的人。管仲家里很穷,经常骗取鲍叔牙的财物,但是鲍叔牙始终都对管仲很好,从未因此而产生怨言。不久,鲍叔牙侍奉齐国的公子小白,而管仲则侍奉另一位公子纠。等到小白成为齐国国君,也就是齐桓公时,公子纠被杀死,管仲也被囚禁起来了。鲍叔牙于是向齐桓公推荐管仲。管仲被任用之后,在齐国主持国政,齐桓公因此成为霸主,九次召集诸侯,使天下复归于正,这些都是靠管仲的谋略才实现的。

\begin{yuanwen}

\end{yuanwen}
\begin{yuanwen}
\end{yuanwen}\begin{yuanwen}
\end{yuanwen}\begin{yuanwen}
\end{yuanwen}\begin{yuanwen}
\end{yuanwen}\begin{yuanwen}
\end{yuanwen}\begin{yuanwen}
\end{yuanwen}\begin{yuanwen}
\end{yuanwen}\begin{yuanwen}
\end{yuanwen}\begin{yuanwen}
\end{yuanwen}\begin{yuanwen}
\end{yuanwen}\begin{yuanwen}
\end{yuanwen}\begin{yuanwen}
\end{yuanwen}\begin{yuanwen}
\end{yuanwen}\begin{yuanwen}
\end{yuanwen}\begin{yuanwen}
\end{yuanwen}\begin{yuanwen}
\end{yuanwen}\begin{yuanwen}
\end{yuanwen}\begin{yuanwen}
\end{yuanwen}\begin{yuanwen}
\end{yuanwen}\begin{yuanwen}
\end{yuanwen}\begin{yuanwen}
\end{yuanwen}\begin{yuanwen}
\end{yuanwen}\begin{yuanwen}
\end{yuanwen}\begin{yuanwen}
\end{yuanwen}\begin{yuanwen}
\end{yuanwen}\begin{yuanwen}
\end{yuanwen}\begin{yuanwen}
\end{yuanwen}\begin{yuanwen}
\end{yuanwen}\begin{yuanwen}
\end{yuanwen}\begin{yuanwen}
\end{yuanwen}\begin{yuanwen}
\end{yuanwen}\begin{yuanwen}
\end{yuanwen}\begin{yuanwen}
\end{yuanwen}\begin{yuanwen}
\end{yuanwen}\begin{yuanwen}
\end{yuanwen}\begin{yuanwen}
\end{yuanwen}\begin{yuanwen}
\end{yuanwen}\begin{yuanwen}
\end{yuanwen}
\begin{yuanwen}\end{yuanwen}

\begin{yuanwen}\end{yuanwen}

\begin{yuanwen}\end{yuanwen}

\begin{yuanwen}\end{yuanwen}

\begin{yuanwen}\end{yuanwen}

\begin{yuanwen}\end{yuanwen}

\begin{yuanwen}\end{yuanwen}

\begin{yuanwen}\end{yuanwen}

\begin{yuanwen}\end{yuanwen}

\begin{yuanwen}\end{yuanwen}

\begin{yuanwen}\end{yuanwen}

\begin{yuanwen}\end{yuanwen}

\begin{yuanwen}\end{yuanwen}

\begin{yuanwen}\end{yuanwen}

\begin{yuanwen}\end{yuanwen}

\begin{yuanwen}\end{yuanwen}

\begin{yuanwen}\end{yuanwen}

\begin{yuanwen}
管仲曰:“吾始困时,尝与鲍叔贾,分财利多自与,鲍叔不以我为贪,知我贫也。吾尝为鲍叔谋事而更穷困,鲍叔不以我为愚,知时有利不利也。吾尝三仕三见逐于君,鲍叔不以我为不肖,知我不遭时也。吾尝三战三走,鲍叔不以我怯,知我有老母也。公子纠败,召忽死之,吾幽囚受辱,鲍叔不以我为无耻,知我不羞小节而耻功名不显于天下也。生我者父母,知我者鲍子也。”

鲍叔既进管仲,以身下之。子孙世禄于齐,有封邑者十馀世,常为名大夫。天下不多管仲之贤而多鲍叔能知人也。

管仲既任政相齐,以区区之齐在海滨,通货积财,富国彊兵,与俗同好恶。故其称曰:“仓廪实而知礼节,衣食足而知荣辱,上服度则六亲固。四维不张,国乃灭亡。下令如流水之原,令顺民心。”故论卑而易行。俗之所欲,因而予之;俗之所否,因而去之。

其为政也,善因祸而为福,转败而为功。贵轻重,慎权衡。桓公实怒少姬,南袭蔡,管仲因而伐楚,责包茅不入贡于周室。桓公实北征山戎,而管仲因而令燕修召公之政。于柯之会,桓公欲背曹沫之约,管仲因而信之,诸侯由是归齐。故曰:“知与之为取,政之宝也。”

管仲富拟于公室,有三归、反坫,齐人不以为侈。管仲卒,齐国遵其政,常彊于诸侯。后百馀年而有晏子焉。

晏平仲婴者,莱之夷维人也。事齐灵公、庄公、景公,以节俭力行重于齐。既相齐,食不重肉,妾不衣帛。其在朝,君语及之,即危言;语不及之,即危行。国有道,即顺命;无道,即衡命。以此三世显名于诸侯。

越石父贤,在縲绁中。晏子出,遭之涂,解左骖赎之,载归。弗谢,入闺。久之,越石父请绝。晏子戄然,摄衣冠谢曰:“婴虽不仁,免子于戹,何子求绝之速也?”石父曰:“不然。吾闻君子诎于不知己而信于知己者。方吾在縲绁中,彼不知我也。夫子既已感寤而赎我,是知己;知己而无礼,固不如在縲绁之中。”晏子于是延入为上客。

晏子为齐相,出,其御之妻从门间而闚其夫。其夫为相御,拥大盖,策驷马,意气扬扬甚自得也。既而归,其妻请去。夫问其故。妻曰:“晏子长不满六尺,身相齐国,名显诸侯。今者妾观其出,志念深矣,常有以自下者。今子长八尺,乃为人仆御,然子之意自以为足,妾是以求去也。”其后夫自抑损。晏子怪而问之,御以实对。晏子荐以为大夫。

太史公曰:吾读管氏牧民、山高、乘马、轻重、九府,及晏子春秋,详哉其言之也。既见其著书,欲观其行事,故次其传。至其书,世多有之,是以不论,论其轶事。

管仲世所谓贤臣,然孔子小之。岂以为周道衰微,桓公既贤,而不勉之至王,乃称霸哉?语曰“将顺其美,匡救其恶,故上下能哙亲也”。岂管仲之谓乎?

方晏子伏庄公尸哭之,成礼然后去,岂所谓“见义不为无勇”者邪?至其谏说,犯君之颜,此所谓“进思尽忠,退思补过”者哉!假令晏子而在,余虽为之执鞭,所忻慕焉。

夷吾成霸,平仲称贤。粟乃实廪,豆不掩肩。转祸为福,危言获全。孔赖左衽,史忻执鞭。成礼而去,人望存焉。
\end{yuanwen}

\part{卷六十三}

\chapter{老子韩非列传第三}

真德秀:「老子『将欲翕之必固张之;将欲取之,必固与之』,此阴谋之言也。阴谋之术,则申韩之所本也。」

郝敬:「老子世俗震耀以为神人,子长屈之管晏之下至与申不害韩非为伍,断曰『申韩原于道德之意』,其识深远,古今闻人少有及此者。世儒托迹孔门,而香火犹龙,不见笑于子长乎?」

\begin{yuanwen}
老子者,楚苦县厉乡曲仁里人也,姓李氏,名耳,字聃,周守藏室之史也。
\end{yuanwen}

老子这个人,是楚国的苦县厉乡曲仁里人,姓李,名耳,字聃,是周朝管理藏书室的官员。

\begin{yuanwen}
孔子適周,将问礼于老子。老子曰:“子所言者,其人与骨皆已朽矣,独其言在耳。且君子得其时则驾,不得其时则蓬累而行。吾闻之,良贾深藏若虚,君子盛德容貌若愚。去子之骄气与多欲,态色与淫志,是皆无益于子之身。吾所以告子,若是而已。”孔子去,谓弟子曰:“鸟,吾知其能飞;鱼,吾知其能游;兽,吾知其能走。走者可以为罔,游者可以为纶,飞者可以为矰。至于龙,吾不能知其乘风云而上天。吾今日见老子,其犹龙邪!”
\end{yuanwen}



\begin{yuanwen}
老子脩道德,其学以自隐无名为务。居周久之,见周之衰,乃遂去。至关,关令尹喜曰:“子将隐矣,彊为我著书。”于是老子乃著书上下篇,言道德之意五千馀言而去,莫知其所终。
\end{yuanwen}



\begin{yuanwen}
或曰:老莱子亦楚人也,著书十五篇,言道家之用,与孔子同时云。
\end{yuanwen}



\begin{yuanwen}
盖老子百有六十馀岁,或言二百馀岁,以其脩道而养寿也。
\end{yuanwen}



\begin{yuanwen}
自孔子死之后百二十九年,而史记周太史儋见秦献公曰:“始秦与周合,合五百岁而离,离七十岁而霸王者出焉。”或曰儋即老子,或曰非也,世莫知其然否。老子,隐君子也。
\end{yuanwen}



\begin{yuanwen}
老子之子名宗,宗为魏将,封于段干。宗子注,注子宫,宫玄孙假,假仕于汉孝文帝。而假之子解为胶西王卬太傅,因家于齐焉。

世之学老子者则绌儒学,儒学亦绌老子。“道不同不相为谋”,岂谓是邪?李耳无为自化,清静自正。

庄子者,蒙人也,名周。周尝为蒙漆园吏,与梁惠王、齐宣王同时。其学无所不闚,然其要本归于老子之言。故其著书十馀万言,大抵率寓言也。作渔父、盗跖、胠箧,以诋訿孔子之徒,以明老子之术。畏累虚、亢桑子之属,皆空语无事实。然善属书离辞,指事类情,用剽剥儒、墨,虽当世宿学不能自解免也。其言洸洋自恣以適己,故自王公大人不能器之。

楚威王闻庄周贤,使使厚币迎之,许以为相。庄周笑谓楚使者曰:“千金,重利;卿相,尊位也。子独不见郊祭之牺牛乎?养食之数岁,衣以文绣,以入大庙。当是之时,虽欲为孤豚,岂可得乎?子亟去,无污我。我宁游戏污渎之中自快,无为有国者所羁,终身不仕,以快吾志焉。”

申不害者,京人也,故郑之贱臣。学术以干韩昭侯,昭侯用为相。内脩政教,外应诸侯,十五年。终申子之身,国治兵彊,无侵韩者。

申子之学本于黄老而主刑名。著书二篇,号曰申子。

韩非者,韩之诸公子也。喜刑名法术之学,而其归本于黄老。非为人口吃,不能道说,而善著书。与李斯俱事荀卿,斯自以为不如非。

非见韩之削弱,数以书谏韩王,韩王不能用。于是韩非疾治国不务脩明其法制,执势以御其臣下,富国彊兵而以求人任贤,反举浮淫之蠹而加之于功实之上。以为儒者用文乱法,而侠者以武犯禁。宽则宠名誉之人,急则用介胄之士。今者所养非所用,所用非所养。悲廉直不容于邪枉之臣,观往者得失之变,故作孤愤、五蠹、内外储、说林、说难十馀万言。

然韩非知说之难,为说难书甚具,终死于秦,不能自脱。

说难曰:

凡说之难,非吾知之有以说之难也;又非吾辩之难能明吾意之难也;又非吾敢横失能尽之难也。凡说之难,在知所说之心,可以吾说当之。

所说出于为名高者也,而说之以厚利,则见下节而遇卑贱,必弃远矣。所说出于厚利者也。而说之以名高,则见无心而远事情,必不收矣。所说实为厚利而显为名高者也,而说之以名高,则阳收其身而实疏之;若说之以厚利,则阴用其言而显弃其身。此之不可不知也。

夫事以密成,语以泄败。未必其身泄之也,而语及其所匿之事,如是者身危。贵人有过端,而说者明言善议以推其恶者,则身危。周泽未渥也而语极知,说行而有功则德亡,说不行而有败则见疑,如是者身危。夫贵人得计而欲自以为功,说者与知焉,则身危。彼显有所出事,乃自以为也故,说者与知焉,则身危。彊之以其所必不为,止之以其所不能已者,身危。故曰:与之论大人,则以为间己;与之论细人,则以为粥权。论其所爱,则以为借资;论其所憎,则以为尝己。径省其辞,则不知而屈之;汎滥博文,则多而久之。顺事陈意,则曰怯懦而不尽;虑事广肆,则曰草野而倨侮。此说之难,不可不知也。

凡说之务,在知饰所说之所敬,而灭其所丑。彼自知其计,则毋以其失穷之;自勇其断,则毋以其敌怒之;自多其力,则毋以其难概之。规异事与同计,誉异人与同行者,则以饰之无伤也。有与同失者,则明饰其无失也。大忠无所拂悟,辞言无所击排,乃后申其辩知焉。此所以亲近不疑,知尽之难也。得旷日弥久,而周泽既渥,深计而不疑,交争而不罪,乃明计利害以致其功,直指是非以饰其身,以此相持,此说之成也。

伊尹为庖,百里奚为虏,皆所由干其上也。故此二子者,皆圣人也,犹不能无役身而涉世如此其汙也,则非能仕之所设也。

宋有富人,天雨墙坏。其子曰“不筑且有盗”,其邻人之父亦云,暮而果大亡其财,其家甚知其子而疑邻人之父。昔者郑武公欲伐胡,乃以其子妻之。因问群臣曰:“吾欲用兵,谁可伐者?”关其思曰:“胡可伐。”乃戮关其思,曰:“胡,兄弟之国也,子言伐之,何也?”胡君闻之,以郑为亲己而不备郑。郑人袭胡,取之。此二说者,其知皆当矣,然而甚者为戮,薄者见疑。非知之难也,处知则难矣。

昔者弥子瑕见爱于卫君。卫国之法,窃驾君车者罪至刖。既而弥子之母病,人闻,往夜告之,弥子矫驾君车而出。君闻之而贤之曰:“孝哉,为母之故而犯刖罪!”与君游果园,弥子食桃而甘,不尽而奉君。君曰:“爱我哉,忘其口而念我!”及弥子色衰而爱弛,得罪于君。君曰:“是尝矫驾吾车,又尝食我以其馀桃。”故弥子之行未变于初也,前见贤而后获罪者,爱憎之至变也。故有爱于主,则知当而加亲;见憎于主,则罪当而加疏。故谏说之士不可不察爱憎之主而后说之矣。

夫龙之为蟲也,可扰狎而骑也。然其喉下有逆鳞径尺,人有婴之,则必杀人。人主亦有逆鳞,说之者能无婴人主之逆鳞,则几矣。

人或传其书至秦。秦王见孤愤、五蠹之书,曰:“嗟乎,寡人得见此人与之游,死不恨矣!”李斯曰:“此韩非之所著书也。”秦因急攻韩。韩王始不用非,及急,乃遣非使秦。秦王悦之,未信用。李斯、姚贾害之,毁之曰:“韩非,韩之诸公子也。今王欲并诸侯,非终为韩不为秦,此人之情也。今王不用,久留而归之,此自遗患也,不如以过法诛之。”秦王以为然,下吏治非。李斯使人遗非药,使自杀。韩非欲自陈,不得见。秦王后悔之,使人赦之,非已死矣。

申子、韩子皆著书,传于后世,学者多有。余独悲韩子为说难而不能自脱耳。

太史公曰:老子所贵道,虚无,因应变化于无为,故著书辞称微妙难识。庄子散道德,放论,要亦归之自然。申子卑卑,施之于名实。韩子引绳墨,切事情,明是非,其极惨礉少恩。皆原于道德之意,而老子深远矣。

伯阳立教,清净无为。道尊东鲁,迹窜西垂。庄蒙栩栩,申害卑卑。刑名有术,说难极知。悲彼周防,终亡李斯。
\end{yuanwen}

\part{卷六十四}
\chapter{司马穰苴列传第四}

\begin{yuanwen}
司马\footnote{古代的官职名,但各个朝代所指官位不尽相同。春秋战国时为掌管军政、军赋的副官。}穰苴\footnote{ráng jū}者,田完\footnote{即田敬仲完,亦即陈完,是陈国陈厉公的儿子,因为国难逃奔到齐国,称为田氏,其子孙最终占有齐国。}之苗裔\footnote{后代。}也。齐景公\footnote{齐景公,原名姜杵臼,齐庄公的异母弟。《史记·齐世家》记载他“好治宫室,聚狗马,奢侈,厚赋重刑”。《论语·季氏篇》记载“齐景公有马千驷,死之日,民无德而称焉”。齐景公在位58年,在位期间,国内安定,国力得到发展,是齐国执政最长的一位国君。}时,晋伐阿、甄,而燕侵河上\footnote{黄河南岸,古代沧州、德州的北边。},齐师败绩。景公患\footnote{忧虑。}之。晏婴乃荐\footnote{推荐。}田穰苴曰:“穰苴虽田氏庶孽\footnote{妾生的孩子。},然其人文能附众,武能威敌,愿君试\footnote{尝试使用。}之。”
\end{yuanwen}

司马穰苴,是田完的后代。齐景公在位的时候,晋国攻打齐国的东阿、甄城两地,而且燕国也趁机侵犯河上地区(黄河南岸),齐国的军队打了败仗。齐景公对此感到非常担心。宰相晏婴于是向齐景公举荐了田穰苴,说:“穰苴虽然是田氏的庶子,可是他的文才能令众人归附,武艺能够威慑敌人,请国君试用这个人。”

\begin{yuanwen}
景公召穰苴,与语\footnote{讨论。}兵事,大说之,以为\footnote{任命。}将军,将兵扞\footnote{hàn,通“捍”,抵抗。}燕、晋之师。

穰苴曰:“臣素\footnote{本来。}卑贱,君擢\footnote{选拔,提拔。}之闾伍\footnote{闾与伍都是户籍的基层组织,这里指乡里、民间。}之中,加之大夫之上,士卒未附\footnote{归附,听从指挥。},百姓不信\footnote{不相信我的能力。},人微\footnote{地位卑微。}权轻,愿得君之宠臣\footnote{位高权重,深受皇帝宠信的大臣。},国之所尊\footnote{整个国家都尊敬的大臣。},以监军\footnote{官名,代表朝廷协理军务,督察将帅的人。},乃可。”

于是景公许\footnote{答应。}之,使\footnote{派遣。}庄贾往。穰苴既辞\footnote{辞别。},与庄贾约\footnote{约定。}曰:“旦日\footnote{第二天。}日中会于军门\footnote{军营大门。}。”
\end{yuanwen}

齐景公召来田穰苴,和他讨论了一些军事方面的事情之后,非常高兴,于是就让他当了将军,命他统领军队抵御燕国和晋国的军队。

田穰苴说:“我一向身份低微卑贱,国君您把我从闾巷里的平民中提拔为将军,职位比大夫还要高,士兵们不亲附我,百姓不信任我。由于我身份卑微,权力也很轻,所以我希望能够有一位受到国人尊敬的,同时也受到国君宠爱的大臣来监督我统率的军队,就可以了。”

齐景公应允了他的请求,派宠臣庄贾前往田穰苴的军营进行监督。田穰苴向齐景公辞别以后,和庄贾立下了约定,说:“第二天中午在军营的门前见面。”

\begin{yuanwen}
穰苴先驰至军,立表下漏\footnote{古代计时的工具。立表,在阳光下竖起木杆,根据阳光照射的影子的移动来计算时间。表,就指这木杆。下漏,把铜壶下穿一小孔,壶中立箭,箭杆上刻有度数,然后铜壶蓄水,使之徐徐下漏,以箭杆显露出来的刻度计算时间。},待贾。贾素\footnote{向来。}骄贵\footnote{骄傲自大。},以为将\footnote{率领。}己之军而己为监,不甚急;亲戚、左右\footnote{指亲近的人。}送之,留饮。日中而贾不至。穰苴则仆表决漏\footnote{仆表,把计时的木杆打倒。决漏,把壶里的水放出。},入\footnote{进入军营。},行军勒兵\footnote{巡行军营,指挥军队。},申明约束\footnote{章程,纪律。}。约束既定\footnote{部署完毕。},夕时,庄贾乃至。

穰苴曰:“何后期\footnote{约定的时间。}为?”

贾谢曰:“不佞\footnote{不才。自谦之词。}大夫亲戚送之,故留。”

穰苴曰:“将受命之日\footnote{接受命令之日。}则忘其家,临军约束则忘其亲,援\footnote{fú,操起,拿起。}枹鼓\footnote{战鼓。}之急则忘其身\footnote{自身的生命。}。今敌国深侵\footnote{深入国境。},邦内骚动\footnote{国内出现不稳定情况。},士卒暴露于境\footnote{指士卒在战场上厮杀。},君寝不安\footnote{安稳。}席,食不甘味,百姓之命皆悬\footnote{系,维系。}于君,何谓相送乎!”

召军正\footnote{军队中的执法官。}问曰:“军法\footnote{依照军法。}期而后至者云何?”

对曰:“当斩。”

庄贾惧,使人驰报\footnote{报告。}景公,请救。既往,未及反\footnote{同“返”。},于是遂斩庄贾以徇\footnote{示众。}三军。三军之士皆振\footnote{害怕得发抖。}栗。
\end{yuanwen}

第二天,田穰苴早就骑着马赶到了军营,并且立下了木表、漏壶来等待庄贾。庄贾一向都很骄傲,地位又很显要,他觉得既然是统率自己的军队,而且还是自己监督军队,就用不着太着急;亲戚和身边的朋友都为他饯行,挽留他喝酒。到了中午,庄贾也没有赶到军营门口。田穰苴于是推倒木表,打破漏壶,进了军营,开始巡查军队,整顿士兵,宣布各种规章号令。规章号令全都明确之后,到了傍晚时分,庄贾这才赶到军营。

田穰苴说:“为什么比我们约定的时间晚到了呢?”

庄贾向他道歉说:“不才有亲戚和朋友来送行,所以耽搁了一段时间。”

田穰苴说:“军队的统帅从接到命令的那天开始,就应该忘掉自己还有家;来到军队,接受军纪的管束,就应该忘掉自己的亲人和朋友;在擂动战鼓战况危急的关头,这时应该忘记对生命的爱惜。现在敌人已经深入我国境内,国家内部骚乱动荡,军卒们暴露在前线战场,国君睡觉都不能安稳,吃饭都觉得不香甜,所有人的性命都维系在你的身上,所谓为你送行还有什么意义呢!”

田穰苴叫来军中的执法官,问他说:“军队中的法令,对于已经约定好时间,但却延误期限的人的惩罚是怎么规定的?”

那人回答道:“应当问斩。”

庄贾害怕,赶紧派人骑着快马上报齐景公,请求救命。庄贾派出的人走了以后,还没有等到他回来,穰苴就斩杀了庄贾,并提着他的脑袋巡行三军。所有的士兵都战栗不已。

\begin{yuanwen}
久之\footnote{过了好长一段时间。},景公遣使者持节\footnote{符节。传达国君命令的信物。}赦贾,驰入军中。穰苴曰:“将在军,君令有所不受。”

问军正曰:“驰三军\footnote{驾着车马在军营里奔驰。}法何?”

正曰:“当斩。”

使者大惧。穰苴曰:“君之使不可杀之。”乃斩其仆\footnote{指使者的仆从。},车之左驸\footnote{通“辅”,夹车木。},马之左骖\footnote{cān,古代用三匹或四匹马拉车时,两边的马叫“骖”。},以徇\footnote{巡行三军。}三军。遣使者还报,然后行\footnote{出发。}。
\end{yuanwen}

很久以后,齐景公派遣使者手拿符节来到军中赦免庄贾,使者乘着马车进入军营。

田穰苴说:“将帅在军营中,对于君主下达的命令可以不必完全听从。”

田穰苴问军中的执法官:“驾着马车在军营中飞驰,军中法律是如何规定的?”

执法官说:“应当斩首。”

使者十分畏惧。田穰苴说:“国君派来的使者不可以杀掉。”于是就杀死使者的仆人,砍断了马车左边夹车的木杆,杀掉了马车左边的马,拿这些向所有的士兵巡行警示。田穰苴让使者回报齐景公之后,就率领军队出发了。

\begin{yuanwen}
士卒次舍\footnote{宿营。次,停留。舍,宿营地。},井灶\footnote{士卒挖井取水,筑灶做饭。}饮食,问\footnote{探问,慰问。}疾医药,身自拊循\footnote{慰问,安抚。}之。悉取将军之资粮享\footnote{通“飨”,以饮食款待。}士卒,身与士卒平分粮食。最比其羸弱者\footnote{最比,特别照顾到。最,特别,尤其。比,及,到。羸,léi,瘦,弱。},三日而后勒兵\footnote{治军,操练或指挥军队。}。病者皆求行\footnote{请求出战。},争奋出为之赴战。晋师闻之,为罢去\footnote{退兵。}。燕师闻之,度水而解\footnote{同“懈”,松驰,懈怠。}。于是追击之,遂取所亡\footnote{所失掉的。}封\footnote{国境线。}内故境\footnote{指封国之内曾经沦陷的土地。}而引兵归。未至国\footnote{国都。},释兵旅\footnote{解除军队的战备。},解\footnote{取消,解除。}约束,誓盟而后入邑\footnote{进入国都。}。景公与诸大夫郊迎\footnote{到城外迎接。},劳师成礼\footnote{按照一定的程式行完毕。},然后反归寝。既见穰苴,尊\footnote{尊敬,隆重地任命。}为大司马。田氏日以益\footnote{越来越。}尊于齐。
\end{yuanwen}

士兵们住宿的营帐、水井、军灶、饮水、食物、生病、就医、服药等事项,田穰苴都会亲自过问并安排。把将军应该享有的资财和粮食全部拿出来,用于款待士卒,而他自己就和士兵一样,平均分配粮食,把身体瘦弱的士兵单列出来,对他们进行特别照顾。过了三天之后便整顿、操练军队。这样一来,连那些生病的士兵都请求跟随大军一起出征,他们奋勇争先地为他出战。晋国军队听说齐国军队的境况之后,竟撤兵离开了齐国。燕国军队听说之后,也向北渡过黄河解除了对齐国的攻势。田穰苴在这种情况下率领军队追击敌军,于是收复了境内所有失陷的领土,恢复原来的边境以后才领兵回朝。齐国军队没有到达国都,田穰苴就解除战备,放松管制,士兵们宣誓立盟之后才进入都城。齐景公和诸位大夫来到郊外迎接田穰苴和他的军队,慰问犒劳军队的礼仪结束之后,齐景公才回到寝宫。齐景公召见穰苴,提拔他为大司马。田氏家族的地位也因为田穰苴一天天变得更加尊贵。

\begin{yuanwen}
已而大夫鲍氏、高、国之属\footnote{指当时齐国掌握实权的卿大夫鲍牧、高眙子、国惠子一班人。}害\footnote{嫉妒。}之,谮\footnote{zèn,说别人的坏话,诬陷,中伤。}于景公。景公退\footnote{罢官。}穰苴,苴发疾\footnote{发病。}而死。田乞、田豹之徒由此怨\footnote{怨恨。}高、国等。其后及\footnote{等到。}田常杀简公,尽\footnote{全部。}灭高子、国子之族。至常曾孙和,因自立\footnote{自立为君。}为齐威王,用兵行威,大放\footnote{通“仿”,仿效,效法。}穰苴之法,而诸侯朝\footnote{朝拜。}齐。

齐威王使大夫追论\footnote{整理的意思。}古者《司马兵法》而附穰苴于其中,因号曰《司马穰苴兵法》。
\end{yuanwen}

没过多久,大夫鲍氏、高氏、国氏等一干人设计陷害田穰苴,在齐景公跟前说田穰苴的坏话。齐景公因而罢免了田穰苴的官职,不久他就得病死了。田乞、田豹这些田氏族人因此憎恨高氏、国氏等人。后来,田常杀死齐简公,把高氏、国氏两个家族全部诛灭。到田常的曾孙田和这一辈时,便自立为齐国国君,其孙为齐威王,用兵作战、树立权威,大多效仿穰苴的做法,很多诸侯国都来齐国朝贡。

齐威王让大夫研究、讨论和整理古代就有的《司马兵法》,而且把司马穰苴治军的方法也附在里面,并借此机会把书命名为《司马穰苴兵法》。

\begin{yuanwen}
太史公曰:余读《司马兵法》,闳廓\footnote{宏大广博。}深远,虽三代\footnote{指夏商周三代。}征伐,未能竟\footnote{穷,尽。}其义,如其文也,亦少褒矣。若夫穰苴,区区为小国行师,何暇及《司马兵法》之揖让\footnote{宾主相见的礼仪,以示谦让。这里引申为相提并论。}乎?世既多\footnote{推重,赞扬。}《司马兵法》,以故不论\footnote{评论。},著穰苴之列传焉。

燕侵河上,齐师败绩。婴荐穰苴,武能威敌。斩贾以徇,三军惊惕。我卒既彊,彼寇退壁。法行司马,实赖宗戚。
\end{yuanwen}

太史公说:我读《司马兵法》,认为这本书内容广博,主旨深远,即便是夏、商、周三朝所发生的重大战争,也不能完全将它的含义囊括,像《司马穰苴兵法》那样的文章,放进书中显得稍微有些褒奖了。至于司马穰苴这样的将领,只不过是一个为小诸侯国带兵打仗的将军,哪里能与《司马兵法》处于同样地位呢?世上既然有很多《司马兵法》,因此我也就不再发表议论,只写这篇司马穰苴的列传罢了。

\part{卷六十五}

\chapter{孙子吴起列传第五}

是孙武、孙膑、吴起三个兵家人物的合传。

凌稚隆:“通篇以『兵法』二字作骨。首次武以兵法见吴王,卒斩二姬,为名将,后次膑与庞涓俱学兵法,而膑以兵法为齐威王师,及死,庞涓显,当时传后世者皆兵法也。篇终结『兵法』二字,与首句相应。”

\begin{yuanwen}
孙子武者,齐人也。以兵法见于吴王阖庐。阖庐曰:“子之十三篇,吾尽观之矣,可以小试勒兵乎?”

对曰:“可。”

阖庐曰:“可试以妇人乎?”

曰:“可。”

于是许之,出宫中美女,得百八十人。孙子分为二队,以王之宠姬二人各为队长,皆令持戟。

令之曰:“汝知而\footnote{你,你的。}心\footnote{胸口。}与左右手背乎?”

妇人曰:“知之。”

孙子曰:“前,则视心;左,视左手;右,视右手;后,即视背。”

妇人曰:“诺。”
\end{yuanwen}

孙子,名武,是齐国人。他因为擅长用兵之法得到吴王阖庐的召见。阖庐说:“您所写的十三篇兵法,我全都看过了,可以小规模地用来实际操练军队吗?”

孙武回答说:“可以。”

阖庐又说:“那么可以试着操练妇女吗?”

孙武回答说:“可以。”

阖庐于是同意用女子做实验练兵,命令宫中的宫女全都出来,共计有一百八十人。孙武将这些人分成两队,把吴王最宠爱的两位姬妾任命为队长,都让她们手持长戟。

孙武向她们发布命令说:“你们知道你们的心口、左手、右手和后背吗?”

宫女们说:“知道。”

孙武说:“我命令你们向前时,你们朝自己前胸所对的方向行进;命令你们向左转时,你们就朝自己左手的方向转;命令你们向右转时,你们就朝自己右手的方向转;命令你们向后转时,你们就朝自己的后背方向转。”

宫女们说:“是。”

\begin{yuanwen}
约束既布,乃设鈇钺\footnote{斧和钺,是腰斩、砍头的刑具。},即三令五申之。于是鼓之右,妇人大笑。

孙子曰:“约束不明,申令不熟,将之罪也。”复三令五申而鼓之左,妇人复大笑。

孙子曰:“约束不明,申令不熟,将之罪也;既已明而不如法者,吏士之罪也。”乃欲斩左右队长。

吴王从台上观,见且斩爱姬,大骇。趣使使下令曰:“寡人已知将军能用兵矣。寡人非此二姬,食不甘味,愿勿斩也。”

孙子曰:“臣既已受命为将,将在军,君命有所不受。”遂斩队长二人以徇。用其次为队长,于是复鼓之。妇人左右前后跪起皆中规矩绳墨\footnote{指木工打直线的墨线。引申为规矩和法度。},无敢出声。

于是孙子使使报王曰:“兵既整齐,王可试下观之,唯王所欲用之,虽赴水火犹可也。”

吴王曰:“将军罢休就舍,寡人不原下观。”

孙子曰:“王徒好其言,不能用其实。”

于是阖庐知孙子能用兵,卒以为将。西破强楚,入郢,北威齐晋,显名诸侯,孙子与有力焉。
\end{yuanwen}

规定全都公布完毕以后,于是陈列上用来惩罚的斧钺,随即又再三申令要听令操作。这件事做完之后,就开始敲鼓发号施令,让那些宫女向右,结果那些宫女全都大笑起来。

孙武说:“规定没有讲明白,手下的人不熟悉号令,是将领的罪过。”又多次重复号令,然后敲鼓发出号令,让那些宫女向左,宫女们又大笑起来。

孙武说:“规定没有讲明白,手下的人不熟悉号令,是将领的罪过;但明白了这些号令之后却不遵守法度,那就是军官和士卒的罪过了。”于是想斩杀左右两位队长。

吴王在高台上观看,见孙武要杀掉自己宠爱的两位姬妾,大为震骇。赶紧派使者传令给孙武:“寡人已经知道将军您是一个擅长用兵的人了。寡人如果没有这两位姬妾,吃饭都会觉得嘴里不香甜,请您不要斩杀她们。”

孙武说:“我既然已经接受大王的命令担任将领,将领在军队中,对于国君的某些命令,是可以不接受的。”于是就斩杀了当队长的两位宠妃,以此警示。然后分别任用这两位队长身后的人担任队长,又一次击鼓。宫女们向左、向右、向前、向后、跪倒、站起等各种动作都符合孙武之前所说的号令,没有人敢发出嘈杂的声音。

这时,孙武派使者报告吴王说:“士兵已经训练得整齐有序,大王可以试着从台上下来观看,是大王的吩咐她们就会听从,即使是让她们到水里、火里去,她们也能听从命令。”

吴王说道:“孙将军停止操练,回到馆舍去休息吧,寡人不想下去观看。”

孙武说:“大王只是喜欢我纸上所谈的兵法,不是真正想让它变为现实。”

吴王阖庐此时知道孙子确实擅长用兵打仗,最终任命他做了将军。后来吴国向西攻破强大的楚国,进入郢都,向北震慑住了齐国、晋国,吴国得以在诸侯国中扬名,孙武参与其中,有很大的功劳。

\begin{yuanwen}
孙武既死,后百余岁有孙膑。膑生阿、鄄之间,膑亦孙武之后世子孙也。孙膑尝与庞涓俱学兵法。庞涓既事魏,得为惠王将军,而自以为能不及孙膑,乃阴使召孙膑。膑至,庞涓恐其贤于己,疾之,则以法刑断其两足而黥之,欲隐勿见。
\end{yuanwen}

孙武死后,过了一百多年,又出了一位军事家孙膑。孙膑出生的地方位于齐国的阿邑、鄄邑之间,孙膑也是孙武的后代子孙。孙膑曾经和庞涓一起学习兵法。庞涓事奉魏国之后,成为魏惠王的将军,却自觉军事才能不如孙膑,于是暗地里派使者把孙膑召到魏国。孙膑到来以后,庞涓怕孙膑比自己贤能,对他非常嫉妒,就设法陷害孙膑,然后依法用刑砍掉他的两脚,在他的脸上刺字,想让他隐藏起来,不敢出现在人们面前。

\begin{yuanwen}
齐使者如梁,孙膑以刑徒阴见,说齐使。齐使以为奇,窃载与之齐。齐将田忌善而客待之。

忌数与齐诸公子驰逐重射\footnote{谓下重的赌注。}。孙子见其马足不甚相远,马有上、中、下辈\footnote{等。}。于是孙子谓田忌曰:“君弟\footnote{但,尽管。}重射,臣能令君胜。”

田忌信然之,与王及诸公子逐射千金。及临质,孙子曰:“今以君之下驷与彼上驷,取君上驷与彼中驷,取君中驷与彼下驷。”

既驰三辈毕,而田忌一不胜而再胜,卒得王千金。于是忌进孙子于威王。威王问兵法,遂以为师。
\end{yuanwen}

齐国的使者到了大梁,孙膑以服刑的犯人的身份偷偷会见了使者,游说使者。齐国使者认为孙膑是少有的人才,偷偷地用马车载着他回到齐国。齐国的大将田忌对他印象很好,把他当成客人一样对待。

田忌多次和齐国的贵族的公子一起赛马,下的赌注很大。孙膑看到这些人的马在脚力上的差别不是很大,可以分为上、中、下三等。于是孙膑就对田忌说:“您只管下大的赌注,我有办法让您获得胜利。”

田忌相信孙膑,于是与齐威王和诸位公子下了千金重的赌注,等到比赛即将开始的时候,孙膑对田忌说:“现在让您的下等马来跟他们的上等马比赛,拿您的上等马与他们的中等马来比赛,拿您的中等马跟他们的下等马来比赛。”

等到三场比赛全都结束之后,田忌有一场没胜,但有两场胜利了,最后得到了齐威王的千金赌注。于是田忌就向齐威王推荐孙膑。齐威王向孙膑询问用兵之法,就把孙膑当成了老师。

\begin{yuanwen}
其后魏伐赵,赵急,请救于齐。齐威王欲将孙膑,膑辞谢曰:“刑余之人不可。”

于是乃以田忌为将,而孙子为师,居辎车\footnote{有蓬盖的车。}中,坐为计谋。

田忌欲引兵之赵,孙子曰:“夫解杂乱纷纠者不控卷,救斗者不搏撠\footnote{犹言揪住。jǐ},批亢捣虚,形格势禁,则自为解耳。今梁赵相攻,轻兵锐卒必竭于外,老弱罢\footnote{同“疲”。}于内。君不若引兵疾走大梁,据其街路,旻(冲)其方虚,彼必释赵而自救。是我一举解赵之围而收弊于魏也。”

田忌从之,魏果去邯郸,与齐战于桂陵,大破梁军。
\end{yuanwen}

后来魏国攻打赵国,赵国的情况非常紧急,于是向齐国求救。齐威王想任命孙膑为主将,孙膑推辞说道:“我是一个受过刑而肢体残缺的人,不可担任主将。”

齐威王就把田忌任命为主将,而让孙膑当了军师,坐在一辆带帐篷的车中,暗中为田忌出谋划策。

田忌想要带着军队到赵国去救援,孙膑说:“要想解开乱纷纷的丝线,应小心用手去解,而不能紧紧地握着双手使劲拉扯;要想救助正在争斗的人,不能卷进去跟他们胡打乱打,要抓住争斗的关键之处,直捣其虚弱之处,由于形势的限制,争斗就自然而然地解开了。现在梁国和赵国互相争斗,魏国最精锐的士兵一定全都被派到了国外,年纪大的、身体弱的、过于疲惫的士兵肯定都留在了国内。您不如带领兵马迅速攻打魏国的都城大梁,占据魏国的交通要道,冲击魏国正处于薄弱状态的地方,对方一定会放弃攻打赵国而回兵解救自己。这样的话我们一举就解除了赵国被包围的危险,而且也可以收到让魏国自取其弊的效果。”

田忌听从了孙膑的建议,魏国的军队果然离开了邯郸,与齐军在桂陵交战,结果齐国军队打败了梁国军队。

\begin{yuanwen}
后十三岁,魏与赵攻韩,韩告急于齐。齐使田忌将而往,直走大梁。魏将庞涓闻之,去韩而归,齐军既已过而西矣。

孙子谓田忌曰:“彼三晋之兵素悍勇而轻齐,齐号为怯,善战者因其势而利导之。兵法,百里而趣利者蹶\footnote{jué,折损。}上将,五十里而趣利者军半至。使齐军入魏地为十万灶,明日为五万灶,又明日为三万灶。”

庞涓行三日,大喜,曰:“我固知齐军怯,入吾地三日,士卒亡者过半矣。”乃弃其步军,与其轻锐倍日并行逐之。

孙子度其行,暮当至马陵。马陵道陕,而旁多阻隘,可伏兵,乃斫\footnote{zhuó,砍。}大树白而书之曰“庞涓死于此树之下”。于是令齐军善射者万弩,夹道而伏,期曰“暮见火举而俱发”。

庞涓果夜至斫木下,见白书,乃钻火烛之。读其书未毕,齐军万弩俱发,魏军大乱相失。庞涓自知智穷兵败,乃自刭,曰:“遂成竖子之名!”

齐因乘胜尽破其军,虏魏太子申以归。孙膑以此名显天下,世传其兵法。
\end{yuanwen}

十三年以后,魏国和赵国一起攻打韩国,韩国请求齐国出兵相救。齐国派田忌带领兵马前往,直接向魏国都城大梁进兵。魏国主将庞涓听说这个消息之后,带领兵马离开韩国,回师魏国,这时齐国的军队已经过了魏国的边界向西继续挺进了。

孙膑对田忌说道:“他们三晋的军士一向剽悍勇猛,轻视齐国军队,齐国士兵被他们称为胆小懦弱的人,善于指挥作战的将领可以根据这样的形势而朝有利的形势进行引导。兵法上说,急行百里仓促地与敌军交战,有可能损失大将;急行五十里仓促地与敌人交战,有可能损失一半的士兵。下令让齐国军队在进入魏国土地之后,先修造十万人吃饭的炉灶,第二天修造五万人吃饭的炉灶,第三天修造三万人吃饭的炉灶。”

庞涓带兵追赶了三天,看到那些炉灶之后,非常高兴,说:“我本来就知道齐国的士兵胆小怯懦,进入我国只有三天的时间,逃跑的士兵就已经超过一半了。”于是舍弃了步兵,自己带领轻便锋锐的骑兵以每天两倍的行军速度追赶齐国军队。

孙兵估量他们行进的速度,傍晚的时候会追到马陵。马陵这个地方道路狭窄,而且两旁多是险峻阻碍的地势,可以埋伏士兵,于是用斧子砍掉一棵大树的外皮,在露出空白的地方写下“庞涓死在这棵树下面”几个字。然后命令擅长射箭的齐国士兵共一万多人,手拿弓弩,埋伏在道路两侧,与这些士兵约定“晚上看到有火光点燃就一齐放箭”。

庞涓果然在夜里赶到了那棵被砍掉树皮的大树下面,看见白色的树干上有字,就点火照看。那些字还没有看完,埋伏在那里的齐国士兵就万箭齐发,魏国军队大乱,彼此失散。庞涓知道自己的智谋已经穷尽,一定会失败,就拔剑自刎,临死前说道:“竟然成就了这个小子的名声!”

齐国军队于是乘胜完全打败了魏军,俘虏了魏国的太子申。孙膑也因为这一战而名扬天下,世上从此开始流传他的兵法。

\begin{yuanwen}
吴起者,卫人也,好用兵\footnote{text}。尝学于曾子,事鲁君。齐人攻鲁,鲁欲将吴起,吴起取齐女为妻,而鲁疑之。吴起于是欲就名,遂杀其妻,以明不与齐也\footnote{text}。鲁卒以为将。将而攻齐,大破之\footnote{text}。
\end{yuanwen}

吴起,是卫国人,喜欢研究用兵之法。曾经跟随曾子学习,后来事奉鲁国国君。齐国军队攻打鲁国,鲁国国君想要任命吴起为主将,但吴起娶了齐国女子为妻,鲁国因此国君怀疑他。吴起此时想要成就自己的名声,于是杀死自己的妻子,借此表明自己不会帮助齐国。鲁国国君最终让他做了主将。吴起带领军队攻打齐国,大败齐国军队。

\begin{yuanwen}
鲁人或恶吴起曰\footnote{text}:“起之为人,猜忍\footnote{猜忌残忍。}人也\footnote{text}。其少时,家累千金,游仕\footnote{外出寻求做官。}不遂\footnote{如愿。},遂破其家。乡党笑之,吴起杀其谤己者三十余人\footnote{text},而东出卫郭门,与其母诀,啮臂而盟\footnote{发誓。}曰\footnote{text}:‘起不为卿相,不复入卫。’遂事曾子。居顷之,其母死,起终不归。曾子薄\footnote{鄙视。}之,而与起绝。起乃之鲁,学兵法以事鲁君。鲁君疑之,起杀妻以求将。夫鲁小国,而有战胜之名,则诸侯图鲁矣。且鲁、卫,兄弟之国也\footnote{text},而君用起,则是弃卫\footnote{text}。”鲁君疑之,谢吴起。
\end{yuanwen}

鲁国有人说吴起的坏话:“吴起这个人的为人,喜欢猜忌别人,而且非常残忍。他年轻的时候,家里积攒了千金的财富,出外游历想要求官,但没有达到目的,家产也被他散尽了。同乡的人都笑话他,吴起杀死三十多个诽谤他的人,然后向东从卫国国都的城门逃走了。跟他的母亲诀别时,他狠狠地咬了自己的胳膊一口,对母亲发誓说道:‘我如果做不了上卿、相国这样的大官,就再也不回卫国。’于是侍奉曾子,跟随他学习。没过多久,他的母亲死了,他最终也没有回去为母亲奔丧。曾子对他这种行为非常鄙薄,于是与吴起断绝了师生关系。吴起于是来到鲁国,学习兵法来事奉鲁国国君。鲁国国君怀疑他,吴起就杀死妻子来求得主将之位。鲁国是个小国,而有战胜齐国的名声,因此各个诸侯国都图谋攻打鲁国。况且鲁国和卫国是兄弟国家,如果国君重用吴起,就等于是抛弃了卫国。”鲁国国君因此产生疑虑,于是罢免了吴起。

\begin{yuanwen}
吴起于是闻魏文侯贤,欲事之。文侯问李克曰\footnote{text}:“吴起何如人哉?”

李克曰:“起贪而好色\footnote{text},然用兵司马穰苴不能过也\footnote{text}。”

于是魏文候以为将,击秦,拔五城。
\end{yuanwen}

吴起在这种情况下听说魏文侯是个贤明的国君,想要事奉他。魏文侯询问李克说:“吴起这个人怎么样啊?”

李克说:“吴起这个人贪婪,又喜欢美色,但要说到用兵打仗,就连司马穰苴也不能超过他。”

魏文侯就让吴起当了大将,攻打秦国,夺得了五座城池。

\begin{yuanwen}
起之为将,与士卒最下者同衣食。卧不设席,行不骑乘,亲裹赢粮\footnote{text},与士卒分劳苦。卒有病疽\footnote{脓疮。}者,起为吮\footnote{吸。}之。卒母闻而哭之。

人曰:“子卒也,而将军自吮其疽,何哭为?”

母曰:“非然也。往年吴公吮其父,其父战不旋踵\footnote{转身。比喻畏避退缩。},遂死于敌。吴公今又吮其子,妾不知其死所矣。是以哭之。”
\end{yuanwen}

吴起当主将的时候,跟士兵中最低等的人穿一样的衣服,吃一样的食物。睡觉的时候身子底下不铺席子,平时走路的时候不骑马、不乘车,亲自背粮食,跟士兵们分担一样的劳苦。士兵们有人背上生了疮,吴起亲自为他把背上病疮中的脓给吸出来。这个士兵的母亲听说这件事后就哭了。有人对她说:“您的儿子只不过是个普通的士兵,而能够令将军亲自给他吸吮脓疮,为什么要哭呢?”母亲说:“不是这样的。前些年,吴公就曾经为儿子的父亲吸吮脓疮,后来孩子的父亲在打仗时勇猛向前,决不撤退,最终死在了敌人手里。吴公现在又为他的儿子吸吮脓疮,我不知道儿子将要死在哪里了。我因此替他哭泣。”

\begin{yuanwen}
文侯以吴起善用兵,廉平,尽能得士心,乃以为西河守,以拒秦、韩。
\end{yuanwen}

魏文侯因为吴起善于用兵打仗,廉洁公平,能够得到士兵的拥戴,于是就让他担任西河地区的守将,来抵御秦国、韩国的进攻。

\begin{yuanwen}
魏文侯既卒,起事其子武侯。武侯浮西河而下\footnote{text},中流,顾而谓吴起曰:“美哉乎山河之固,此魏国之宝也!”

起对曰:“在德不在险。昔三苗氏左洞庭,右彭蠡\footnote{text},德义不修,禹灭之。夏桀之居\footnote{text},左河、济\footnote{text},右泰华\footnote{text},伊阙在其南\footnote{text},羊肠在其北\footnote{text},修政不仁,汤放之。殷纣之国,左孟门,右太行\footnote{text},常山在其北\footnote{text},大河经其南\footnote{text},修政不德,武王杀之。由此观之,在德不在险。若君不修德,舟中之人尽为敌国也。”

武侯曰:“善。”
\end{yuanwen}

魏文侯死了以后,吴起侍奉他的儿子魏武侯。魏武侯坐着船顺着西河向下游漂流,走到中间时,回头对吴起说:“壮美啊,这险固的山河,是魏国的珍宝啊!”

吴起回答说:“国家的珍宝在于君主把德政施加给人民,而不在于山河是否险固。以前三苗族居住的地方,左边有洞庭湖,右边有彭蠡湖,却不修德政,不施行恩义,结果被大禹消灭了。夏朝的桀王所占有的领土,左边是黄河、济水,右边是泰山、华山,伊阙山在它的南面,羊肠坂在它的北面,因为桀王没有修行仁政,商汤便放逐了他。殷纣王的国家,它的左边是孟门山,右边是太行山,常山在它的北面,黄河流经它的南面,因为他不修行德政,周武王杀了他。从这些事例可以看出,一个国家的强大,在于国君施行德政,而不在于山河如何险要。假如您不修行德政,那么船里坐着的这些人都是您的敌人。”

魏武侯说:“说得太好了。”

\begin{yuanwen}
吴起为西河守,甚有声名。魏置相,相田文。吴起不悦,谓田文曰:“请与子论功,可乎?”

田文曰:“可。”

起曰:“将三军,使士卒乐死,敌国不敢谋,子孰与起?”

文曰:“不如子。”

起曰:“治百官,亲万民,实府库,子孰与起?”

文曰:“不如子。”

起曰:“守西河而秦兵不敢东乡,韩赵宾从,子孰与起?”

文曰:“不如子。”

起曰:“此三者,子皆出吾下,而位加吾上,何也?”

文曰:“主少国疑,大臣未附,百姓不信,方是之时,属之于子乎?属之于我乎?”

起默然良久,曰:“属之子矣。”

文曰:“此乃吾所以居子之上也。”

吴起乃自知弗如田文。
\end{yuanwen}

吴起做西河守将,名气很大。魏国要设置相国,以田文为相国。吴起不高兴,对田文说:“我请求跟您比比各自的功绩,可以吗?”

田文说:“可以。”

吴起说:“率领三军将士,能够让士兵乐意为主将去死,敌对的国家不敢图谋攻打,您和我谁更厉害?”

田文说:“我不如您。”

吴起说:“统率文武百官,令老百姓相亲附国家,令府库充实,您和我谁更厉害?”

田文说:“我不如您。”

吴起说:“防守西河,令秦国的军队不敢向东前进,令韩国和赵国服从魏国,您和我吴起谁更厉害?”

田文说:“我不如您。”

吴起说:“这三种情况,您都在我之下,但是权位却在我之上,为什么啊?”

田文说:“如今国君很年轻,国家都对他充满了怀疑,大臣们都不亲附国君,老百姓不信任他,在如今这种局面下,相国是让您来做呢?还是让我来做呢?”

吴起默默地想了很久,说:“国相还是应该由您来做。”

田文说:“这就是我之所以权位在您之上的原因。”

吴起于是知道自己不如田文。

\begin{yuanwen}
田文既死,公叔为相\footnote{text},尚魏公主,而害吴起\footnote{text}。

公叔之仆曰:“起易去也。”

公叔曰:“奈何?”

其仆曰:“吴起为人节廉而自喜名也。君因先与武侯言曰:‘夫吴起贤人也,而侯之国小,又与强秦壤界\footnote{text},臣窃恐起之无留心也。’武侯即曰:‘奈何?’君因谓武侯曰:‘试延以公主,起有留心则必受之;无留心则必辞矣。’以此卜之。君因召吴起而与归,即令公主怒而轻君。吴起见公主之贱君也,则必辞。”

于是吴起见公主之贱魏相,果辞魏武侯\footnote{text}。武侯疑之而弗信也。吴起惧得罪,遂去,即之楚。
\end{yuanwen}

田文死了以后,公叔做了宰相,娶了魏国的公主,但公叔从心里忌惮吴起。公叔的仆人说:“想要除去吴起,那太容易了。”

公叔说:“怎么办?”

他的仆人说:“吴起这个人做人正直清廉,又喜欢自己的美名。您可以先对武侯说:‘吴起是贤能的人,但您的国家太小了,又和强大的秦国接壤,我私下里恐怕吴起不会甘心留在魏国。’武侯听了就会说:‘怎么办啊?’您就对武侯说:‘试着把公主嫁给他,如果他有心留在魏国就会接受公主,如果他无心留在这里就一定会推辞。用这个办法来推断他的心意。’于是您就可以叫吴起跟您一起回家,让公主假装对您生气并且轻慢您。吴起看见公主轻视您,就一定会拒绝武侯。”

在吴起看到公主轻慢魏国的相国后,果然拒绝了魏武侯。魏武侯从此怀疑吴起,不再信任他。吴起害怕获罪,于是离开魏国,就到了楚国。

\begin{yuanwen}
楚悼王素闻起贤,至则相楚。明法审令\footnote{text},捐\footnote{除去,废除。}不急之官\footnote{text},废公族疏远者\footnote{text},以抚养战斗之士。要在强兵,破驰说之言从横者\footnote{text}。于是南平百越\footnote{text},北并陈、蔡\footnote{text},却三晋\footnote{text},西伐秦\footnote{text}。诸侯患楚之强,故楚之贵戚尽欲害吴起\footnote{text}。及悼王死,宗室大臣作乱而攻吴起,吴起走之王尸而伏之。击起之徒因射刺吴起,并中悼王。悼王既葬,太子立\footnote{text},乃使令尹尽诛射吴起而并中王尸者\footnote{text},坐射起而夷宗死者七十余家\footnote{text}。
\end{yuanwen}

楚悼王一向听说吴起是个贤能的人,等他到了楚国就任命他做楚国的宰相。吴起明确了楚国的法律,谨慎地发布命令,裁减了那些冗散多余的官员,废除了贵族中比较疏远的人的爵禄,节省下来的财物都用来抚恤、供养那些打仗的士兵。主要的政策都用来增强军事力量,揭穿那些前来游说的人的谎言。于是向南平定了百越族;向北吞并了陈国、蔡国的土地,打退了韩、赵、魏三国的进攻;向西攻打秦国。诸侯对楚国的强大感到担心。楚国的旧贵族都想陷害吴起。等到楚悼王死了,宗室和大臣一起发动叛乱,攻打吴起,吴起逃到停放楚悼王的尸体的地方,把尸体背在身上。攻击吴起的人趁机用箭射吴起,结果楚悼王的尸体也中箭了。楚悼王被安葬之后,太子即位为楚王,于是派令尹杀死了所有把箭射向吴起和楚悼王尸体的人。因射死吴起而导致被灭族的人有七十多家。

\begin{yuanwen}
太史公曰:世俗所称师旅,皆道孙子十三篇,吴起兵法,世多有,故弗论,论其行事所施设者。语曰:“能行之者未必能言,能言之者未必能行。”孙子筹策庞涓明矣,然不能蚤救患于被刑。吴起说武侯以形势不如德,然行之于楚,以刻暴少恩亡其躯。悲夫!

孙子兵法,一十三篇。美人既斩,良将得焉。其孙膑脚,筹策庞涓。吴起相魏,西河称贤;惨礉事楚,死后留权。
\end{yuanwen}

太史公说:世上的人一夸赞军事战法,都要说到孙武所写的《孙子兵法》十三篇,《吴起兵法》,世上流传的很多,所以不去讨论,只是讨论他们生平相关的一些事迹。古语说:“实干的人不一定擅长言辞,擅长言辞的人不一定实干。”孙膑施展计谋打败庞涓算是明智了,但却不能早些使自己免除被砍掉双脚的刑罚。吴起能够游说魏武侯形势险固不如施以德政,但到了楚国以后,却因为刻薄、急躁、少施恩泽而丧命。可悲啊!

\part{卷六十六}
\chapter{伍子胥列传第六}

\begin{yuanwen}
伍子胥者,楚人也,名员。员父曰伍奢。员兄曰伍尚。其先曰伍举,以直谏事楚庄王,有显,故其后世有名于楚。
\end{yuanwen}

\begin{yuanwen}
楚平王有太子名曰建,使伍奢为太傅,费无忌为少傅。无忌不忠于太子建。平王使无忌为太子取妇于秦,秦女好,无忌驰归报平王曰:“秦女绝美,王可自取,而更为太子取妇。”平王遂自取秦女而绝爱幸之,生子轸。更为太子取妇。

无忌既以秦女自媚于平王,因去太子而事平王。恐一旦平王卒而太子立,杀己,乃因谗太子建。建母,蔡女也,无宠于平王。平王稍益疏建,使建守城父,备边兵。

顷之,无忌又日夜言太子短于王曰:“太子以秦女之故,不能无怨望,原王少自备也。自太子居城父,将兵,外交诸侯,且欲入为乱矣。”平王乃召其太傅伍奢考问之。伍奢知无忌谗太子于平王,因曰:“王独柰何以谗贼小臣疏骨肉之亲乎?”无忌曰:“王今不制,其事成矣。王且见禽。”于是平王怒,囚伍奢,而使城父司马奋扬往杀太子。行未至,奋扬使人先告太子:“太子急去,不然将诛。”太子建亡奔宋。

无忌言于平王曰:“伍奢有二子,皆贤,不诛且为楚忧。可以其父质而召之,不然且为楚患。”王使使谓伍奢曰:“能致汝二子则生,不能则死。”伍奢曰:“尚为人仁,呼必来。员为人刚戾忍卼,能成大事,彼见来之并禽,其势必不来。”王不听,使人召二子曰:“来,吾生汝父;不来,今杀奢也。”伍尚欲往,员曰:“楚之召我兄弟,非欲以生我父也,恐有脱者后生患,故以父为质,诈召二子。二子到,则父子俱死。何益父之死?往而令雠不得报耳。不如奔他国,借力以雪父之耻,俱灭,无为也。”伍尚曰:“我知往终不能全父命。然恨父召我以求生而不往,后不能雪耻,终为天下笑耳。”谓员:“可去矣!汝能报杀父之雠,我将归死。”尚既就执,使者捕伍胥。伍胥贯弓执矢乡使者,使者不敢进,伍胥遂亡。闻太子建之在宋,往从之。奢闻子胥之亡也,曰:“楚国君臣且苦兵矣。”伍尚至楚,楚并杀奢与尚也。

伍胥既至宋,宋有华氏之乱,乃与太子建俱奔于郑。郑人甚善之。太子建又適晋,晋顷公曰:“太子既善郑,郑信太子。太子能为我内应,而我攻其外,灭郑必矣。灭郑而封太子。”太子乃还郑。事未会,会自私欲杀其从者,从者知其谋,乃告之于郑。郑定公与子产诛杀太子建。建有子名胜。伍胥惧,乃与胜俱奔吴。到昭关,昭关欲执之。伍胥遂与胜独身步走,几不得脱。追者在后。至江,江上有一渔父乘船,知伍胥之急,乃渡伍胥。伍胥既渡,解其剑曰:“此剑直百金,以与父。”父曰:“楚国之法,得伍胥者赐粟五万石,爵执珪,岂徒百金剑邪!”不受。伍胥未至吴而疾,止中道,乞食。至于吴,吴王僚方用事,公子光为将。伍胥乃因公子光以求见吴王。

久之,楚平王以其边邑锺离与吴边邑卑梁氏俱蚕,两女子争桑相攻,乃大怒,至于两国举兵相伐。吴使公子光伐楚,拔其锺离、居巢而归。伍子胥说吴王僚曰:“楚可破也。原复遣公子光。”公子光谓吴王曰:“彼伍胥父兄为戮于楚,而劝王伐楚者,欲以自报其雠耳。伐楚未可破也。”伍胥知公子光有内志,欲杀王而自立,未可说以外事,乃进专诸于公子光,退而与太子建之子胜耕于野。

五年而楚平王卒。初,平王所夺太子建秦女生子轸,及平王卒,轸竟立为后,是为昭王。吴王僚因楚丧,使二公子将兵往袭楚。楚发兵绝吴兵之后,不得归。吴国内空,而公子光乃令专诸袭刺吴王僚而自立,是为吴王阖庐。阖庐既立,得志,乃召伍员以为行人,而与谋国事。

楚诛其大臣郤宛、伯州犁,伯州犁之孙伯嚭亡奔吴,吴亦以嚭为大夫。前王僚所遣二公子将兵伐楚者,道绝不得归。后闻阖庐弑王僚自立,遂以其兵降楚,楚封之于舒。阖庐立三年,乃兴师与伍胥、伯嚭伐楚,拔舒,遂禽故吴反二将军。因欲至郢,将军孙武曰:“民劳,未可,且待之。”乃归。

四年,吴伐楚,取六与灊。五年,伐越,败之。六年,楚昭王使公子囊瓦将兵伐吴。吴使伍员迎击,大破楚军于豫章,取楚之居巢。

九年,吴王阖庐谓子胥、孙武曰:“始子言郢未可入,今果何如?”二子对曰:“楚将囊瓦贪,而唐、蔡皆怨之。王必欲大伐之,必先得唐、蔡乃可。”阖庐听之,悉兴师与唐、蔡伐楚,与楚夹汉水而陈。吴王之弟夫概将兵请从,王不听,遂以其属五千人击楚将子常。子常败走,奔郑。于是吴乘胜而前,五战,遂至郢。己卯,楚昭王出奔。庚辰,吴王入郢。

昭王出亡,入云梦;盗击王,王走郧。郧公弟怀曰:“平王杀我父,我杀其子,不亦可乎!”郧公恐其弟杀王,与王奔随。吴兵围随,谓随人曰:“周之子孙在汉川者,楚尽灭之。”随人欲杀王,王子綦匿王,己自为王以当之。随人卜与王于吴,不吉,乃谢吴不与王。

始伍员与申包胥为交,员之亡也,谓包胥曰:“我必覆楚。”包胥曰:“我必存之。”及吴兵入郢,伍子胥求昭王。既不得,乃掘楚平王墓,出其尸,鞭之三百,然后已。申包胥亡于山中,使人谓子胥曰:“子之报雠,其以甚乎!吾闻之,人众者胜天,天定亦能破人。今子故平王之臣,亲北面而事之,今至于僇死人,此岂其无天道之极乎!”伍子胥曰:“为我谢申包胥曰,吾日莫途远,吾故倒行而逆施之。”于是申包胥走秦告急,求救于秦。秦不许。包胥立于秦廷,昼夜哭,七日七夜不绝其声。秦哀公怜之,曰:“楚虽无道,有臣若是,可无存乎!”乃遣车五百乘救楚击吴。六月,败吴兵于稷。会吴王久留楚求昭王,而阖庐弟夫概乃亡归,自立为王。阖庐闻之,乃释楚而归,击其弟夫概。夫概败走,遂奔楚。楚昭王见吴有内乱,乃复入郢。封夫概于堂谿,为堂谿氏。楚复与吴战,败吴,吴王乃归。

后二岁,阖庐使太子夫差将兵伐楚,取番。楚惧吴复大来,乃去郢,徙于鄀。当是时,吴以伍子胥、孙武之谋,西破彊楚,北威齐晋,南服越人。

其后四年,孔子相鲁。


后五年,伐越。越王句践迎击,败吴于姑苏,伤阖庐指,军卻。阖庐病创将死,谓太子夫差曰:“尔忘句践杀尔父乎?”夫差对曰:“不敢忘。”是夕,阖庐死。夫差既立为王,以伯嚭为太宰,习战射。二年后伐越,败越于夫湫。越王句践乃以馀兵五千人栖于会稽之上,使大夫种厚币遗吴太宰嚭以请和,求委国为臣妾。吴王将许之。伍子胥谏曰:“越王为人能辛苦。今王不灭,后必悔之。”吴王不听,用太宰嚭计,与越平。

其后五年,而吴王闻齐景公死而大臣争宠,新君弱,乃兴师北伐齐。伍子胥谏曰:“句践食不重味,吊死问疾,且欲有所用之也。此人不死,必为吴患。今吴之有越,犹人之有腹心疾也。而王不先越而乃务齐,不亦谬乎!”吴王不听,伐齐,大败齐师于艾陵,遂威邹鲁之君以归。益疏子胥之谋。

其后四年,吴王将北伐齐,越王句践用子贡之谋,乃率其众以助吴,而重宝以献遗太宰嚭。太宰嚭既数受越赂,其爱信越殊甚,日夜为言于吴王。吴王信用嚭之计。伍子胥谏曰:“夫越,腹心之病,今信其浮辞诈伪而贪齐。破齐,譬犹石田,无所用之。且盘庚之诰曰:‘有颠越不恭,劓殄灭之,俾无遗育,无使易种于兹邑。’此商之所以兴。原王释齐而先越;若不然,后将悔之无及。”而吴王不听,使子胥于齐。子胥临行,谓其子曰:“吾数谏王,王不用,吾今见吴之亡矣。汝与吴俱亡,无益也。”乃属其子于齐鲍牧,而还报吴。

吴太宰嚭既与子胥有隙,因谗曰:“子胥为人刚暴,少恩,猜贼,其怨望恐为深祸也。前日王欲伐齐,子胥以为不可,王卒伐之而有大功。子胥耻其计谋不用,乃反怨望。而今王又复伐齐,子胥专愎彊谏,沮毁用事,徒幸吴之败以自胜其计谋耳。今王自行,悉国中武力以伐齐,而子胥谏不用,因辍谢,详病不行。王不可不备,此起祸不难。且嚭使人微伺之,其使于齐也,乃属其子于齐之鲍氏。夫为人臣,内不得意,外倚诸侯,自以为先王之谋臣,今不见用,常鞅鞅怨望。原王早图之。”吴王曰:“微子之言,吾亦疑之。”乃使使赐伍子胥属镂之剑,曰:“子以此死。”伍子胥仰天叹曰:“嗟乎!谗臣嚭为乱矣,王乃反诛我。我令若父霸。自若未立时,诸公子争立,我以死争之于先王,几不得立。若既得立,欲分吴国予我,我顾不敢望也。然今若听谀臣言以杀长者。”乃告其舍人曰:“必树吾墓上以梓,令可以为器;而抉吾眼县吴东门之上,以观越寇之入灭吴也。”乃自刭死。吴王闻之大怒,乃取子胥尸盛以鸱夷革,浮之江中。吴人怜之,为立祠于江上,因命曰胥山。

吴王既诛伍子胥,遂伐齐。齐鲍氏杀其君悼公而立阳生。吴王欲讨其贼,不胜而去。其后二年,吴王召鲁卫之君会之橐皋。其明年,因北大会诸侯于黄池,以令周室。越王句践袭杀吴太子,破吴兵。吴王闻之,乃归,使使厚币与越平。后九年,越王句践遂灭吴,杀王夫差;而诛太宰嚭,以不忠于其君,而外受重赂,与己比周也。

伍子胥初所与俱亡故楚太子建之子胜者,在于吴。吴王夫差之时,楚惠王欲召胜归楚。叶公谏曰:“胜好勇而阴求死士,殆有私乎!”惠王不听。遂召胜,使居楚之边邑鄢,号为白公。白公归楚三年而吴诛子胥。

白公胜既归楚,怨郑之杀其父,乃阴养死士求报郑。归楚五年,请伐郑,楚令尹子西许之。兵未发而晋伐郑,郑请救于楚。楚使子西往救,与盟而还。白公胜怒曰:“非郑之仇,乃子西也。”胜自砺剑,人问曰:“何以为?”胜曰:“欲以杀子西。”子西闻之,笑曰:“胜如卵耳,何能为也。”

其后四岁,白公胜与石乞袭杀楚令尹子西、司马子綦于朝。石乞曰:“不杀王,不可。”乃劫王如高府。石乞从者屈固负楚惠王亡走昭夫人之宫。叶公闻白公为乱,率其国人攻白公。白公之徒败,亡走山中,自杀。而虏石乞,而问白公尸处,不言将亨。石乞曰:“事成为卿,不成而亨,固其职也。”终不肯告其尸处。遂亨石乞,而求惠王复立之。

太史公曰:怨毒之于人甚矣哉!王者尚不能行之于臣下,况同列乎!向令伍子胥从奢俱死,何异蝼蚁。弃小义,雪大耻,名垂于后世,悲夫!方子胥窘于江上,道乞食,志岂尝须臾忘郢邪?故隐忍就功名,非烈丈夫孰能致此哉?白公如不自立为君者,其功谋亦不可胜道者哉!

谗人罔极,交乱四国。嗟彼伍氏,被兹凶慝!员独忍诟,志复冤毒。霸吴起师,伐楚逐北。鞭尸雪耻,抉眼弃德。
\end{yuanwen}

太史公说:仇恨给人带来的影响实在是太大了啊!当王的尚且不能与臣子结下仇怨,何况是地位相当的人呢!假如当初伍子胥与伍奢一道死了的话,那与蝼蛄和蚂蚁的死又有什么不同呢?但他能够放下细小的道义,洗雪了重大的耻辱,使声名流传后世,可悲啊!正当伍子胥被围困在江边的时候,在沿路乞讨糊口的时候,他的内心志向曾经有片刻忘掉对郢都、对楚平王的仇恨吗?所以他克制忍耐,成就功业声名,如果不是抱负远大、刚正有气节的男子汉,谁能够达到这种地步呢?白公假如不自立为王的话,他的功绩谋略也是很值得称道的呢!

\chapter{仲尼弟子列传}

\begin{yuanwen}
孔子曰“受业身通者七十有七人”,皆异能之士也。德行:颜渊,闵子骞,厓伯牛,仲弓。政事:厓有,季路。言语:宰我,子贡。文学:子游,子夏。师也辟,参也鲁,柴也愚,由也喭,回也屡空。赐不受命而货殖焉,亿则屡中。

孔子之所严事:于周则老子;于卫,蘧伯玉;于齐,晏平仲;于楚,老莱子;于郑,子产;于鲁,孟公绰。数称臧文仲、柳下惠、铜鞮伯华、介山子然,孔子皆后之,不并世。

颜回者,鲁人也,字子渊。少孔子三十岁。

颜渊问仁,孔子曰:“克己复礼,天下归仁焉。”

孔子曰:“贤哉回也!一箪食,一瓢饮,在陋巷,人不堪其忧,回也不改其乐。”“回也如愚;退而省其私,亦足以发,回也不愚。”“用之则行,舍之则藏,唯我与尔有是夫!”

回年二十九,发尽白,蚤死。孔子哭之恸,曰:“自吾有回,门人益亲。”鲁哀公问:“弟子孰为好学?”孔子对曰:“有颜回者好学,不迁怒,不贰过。不幸短命死矣,今也则亡。”

闵损字子骞。少孔子十五岁。

孔子曰:“孝哉闵子骞!人不间于其父母昆弟之言。”不仕大夫,不食汙君之禄。“如有复我者,必在汶上矣。”

厓耕字伯牛。孔子以为有德行。

伯牛有恶疾,孔子往问之,自牖执其手,曰:“命也夫!斯人也而有斯疾,命也夫!”

厓雍字仲弓。

仲弓问政,孔子曰:“出门如见大宾,使民如承大祭。在邦无怨,在家无怨。”

孔子以仲弓为有德行,曰:“雍也可使南面。”

仲弓父,贱人。孔子曰:“犁牛之子骍且角,虽欲勿用,山川其舍诸?”

厓求字子有,少孔子二十九岁。为季氏宰。

季康子问孔子曰:“厓求仁乎?”曰:“千室之邑,百乘之家,求也可使治其赋。仁则吾不知也。”复问:“子路仁乎?”孔子对曰:“如求。”

求问曰:“闻斯行诸?”子曰:“行之。”子路问:“闻斯行诸?”子曰:“有父兄在,如之何其闻斯行之!”子华怪之,“敢问问同而答异?”孔子曰:“求也退,故进之。由也兼人,故退之。”

仲由字子路,卞人也。少孔子九岁。

子路性鄙,好勇力,志伉直,冠雄鸡,佩豭豚,陵暴孔子。孔子设礼稍诱子路,子路后儒服委质,因门人请为弟子。

子路问政,孔子曰:“先之,劳之。”请益。曰:“无倦。”

子路问:“君子尚勇乎?”孔子曰:“义之为上。君子好勇而无义则乱,小人好勇而无义则盗。”

子路有闻,未之能行,唯恐有闻。

孔子曰:“片言可以折狱者,其由也与!”“由也好勇过我,无所取材。”“若由也,不得其死然。”“衣敝缊袍与衣狐貉者立而不耻者,其由也与!”“由也升堂矣,未入于室也。”

季康子问:“仲由仁乎?”孔子曰:“千乘之国可使治其赋,不知其仁。”

子路喜从游,遇长沮、桀溺、荷丈人。

子路为季氏宰,季孙问曰:“子路可谓大臣与?”孔子曰:“可谓具臣矣。”

子路为蒲大夫,辞孔子。孔子曰:“蒲多壮士,又难治。然吾语汝:恭以敬,可以执勇;宽以正,可以比众;恭正以静,可以报上。”

初,卫灵公有宠姬曰南子。灵公太子蒉聩得过南子,惧诛出奔。及灵公卒而夫人欲立公子郢。郢不肯,曰:“亡人太子之子辄在。”于是卫立辄为君,是为出公。出公立十二年,其父蒉聩居外,不得入。子路为卫大夫孔悝之邑宰。蒉聩乃与孔悝作乱,谋入孔悝家,遂与其徒袭攻出公。出公奔鲁,而蒉聩入立,是为庄公。方孔悝作乱,子路在外,闻之而驰往。遇子羔出卫城门,谓子路曰:“出公去矣,而门已闭,子可还矣,毋空受其祸。”子路曰:“食其食者不避其难。”子羔卒去。有使者入城,城门开,子路随而入。造蒉聩,蒉聩与孔悝登台。子路曰:“君焉用孔悝?请得而杀之。”蒉聩弗听。于是子路欲燔台,蒉聩惧,乃下石乞、壶黡攻子路,击断子路之缨。子路曰:“君子死而冠不免。”遂结缨而死。

孔子闻卫乱,曰:“嗟乎,由死矣!”已而果死。故孔子曰:“自吾得由,恶言不闻于耳。”是时子贡为鲁使于齐。

宰予字子我。利口辩辞。既受业,问:“三年之丧不已久乎?君子三年不为礼,礼必坏;三年不为乐,乐必崩。旧穀既没,新穀既升,钻燧改火,期可已矣。”子曰:“于汝安乎?”曰:“安。”“汝安则为之。君子居丧,食旨不甘,闻乐不乐,故弗为也。”宰我出,子曰:“予之不仁也!子生三年然后免于父母之怀。夫三年之丧,天下之通义也。”

宰予昼寝。子曰:“朽木不可雕也,粪土之墙不可圬也。”

宰我问五帝之德,子曰:“予非其人也。”

宰我为临菑大夫,与田常作乱,以夷其族,孔子耻之。

端沐赐,卫人,字子贡。少孔子三十一岁。

子贡利口巧辞,孔子常黜其辩。问曰:“汝与回也孰愈?”对曰:“赐也何敢望回!回也闻一以知十,赐也闻一以知二。”

子贡既已受业,问曰:“赐何人也?”孔子曰:“汝器也。”曰:“何器也?”曰:“瑚琏也。”

陈子禽问子贡曰:“仲尼焉学?”子贡曰:“文武之道未坠于地,在人,贤者识其大者,不贤者识其小者,莫不有文武之道。夫子焉不学,而亦何常师之有!”又问曰:“孔子適是国必闻其政。求之与?抑与之与?”子贡曰:“夫子温良恭俭让以得之。夫子之求之也,其诸异乎人之求之也。”

子贡问曰:“富而无骄,贫而无谄,何如?”孔子曰:“可也;不如贫而乐道,富而好礼。”
\end{yuanwen}

\begin{yuanwen}
田常欲作乱于齐,惮高、国、鲍、晏,故移其兵欲以伐鲁。孔子闻之,谓门弟子曰:“夫鲁,坟墓所处,父母之国,国危如此,二三子何为莫出?”

子路请出,孔子止之。子张、子石请行,孔子弗许。子贡请行,孔子许之。
\end{yuanwen}

田常打算在齐国叛乱,却害怕高氏、国氏、鲍氏和晏氏的势力,所以想调动他们的军队去攻打鲁国。孔子听了这个消息,对他门下的弟子说:“鲁国,是祖宗坟墓所在的地方,是养育我们的国家,国家到了如此危险的地步,你们几个人为什么不挺身而出呢?”

子路请求前去,孔子阻止了他。子张、子石请求前去,孔子也不答应。子贡请求前去,孔子答应了。

\begin{yuanwen}
遂行,至齐,说田常曰:“君之伐鲁过矣。夫鲁,难伐之国,其城薄以卑,其地狭以泄\footnote{《越绝书》、《吴越春秋》作“其池狭以浅”。},其君愚而不仁,大臣伪而无用,其士民又恶甲兵\footnote{铠甲和兵器。借指战争。}之事,此不可与战。君不如伐吴。夫吴,城高以厚,地广以深,甲坚以新,士选以饱,重器\footnote{宝器。比喻宝贵的人才。}精兵尽在其中,又使明大夫守之,此易伐也。”

田常忿然作色曰:“子之所难,人之所易;子之所易,人之所难:而以教常,何也?”

子贡曰:“臣闻之,忧在内者攻强,忧在外者攻弱。今君忧在内。吾闻君三封而三不成者,大臣有不听者也。今君破鲁以广齐,战胜以骄主,破国以尊臣,而君之功不与焉,则交日疏于主。是君上骄主心,下恣群臣,求以成大事,难矣。夫上骄则恣,臣骄则争,是君上与主有卻\footnote{同“隙”,比喻感情上的裂缝。},下与大臣交争也。如此,则君之立于齐危矣。故曰不如伐吴。伐吴不胜,民人外死,大臣内空,是君上无强臣之敌,下无民人之过,孤主制齐者唯君也。”

田常曰:“善。虽然,吾兵业已加鲁矣,去而之吴,大臣疑我,奈何?”

子贡曰:“君按兵无伐,臣请往使吴王,令之救鲁而伐齐,君因以兵迎之。”

田常许之,使子贡南见吴王。
\end{yuanwen}

子贡于是出发了,来到齐国,游说田常说:“您攻打鲁国的计划失误了。那鲁国是很难攻打的国家,它的城墙薄而矮,它的护城河窄而浅,它的国君愚昧而且不仁,他的大臣虚伪而且无能,它的士兵和百姓都厌恶打仗,这样的国家不能与它交战。您不如去攻打吴国。吴国,它的城墙高而厚,它的护城河宽而深,士兵的铠甲坚固而崭新,它的战士经过了挑选而人员充足,人才和精良的武器都在其中,又派遣贤明的大臣守卫他,这样的国家是容易攻打的。”

田常听了非常愤怒,脸色一变,说:“你认为困难的,是别人认为容易的;你认为容易的,是别人认为困难的。你用这些话来指教我,是什么用心呢?”

子贡说:“我听说过这样的话,忧虑来自于国家内部的就攻打强大的国家,忧虑来自于国家外部的就攻打弱小的国家。如今,您的忧虑是来自于国家内部的。我听说您三次求封但三次都没有成功,是因为有的大臣不听从命令。现在,您要攻占鲁国来扩大齐国疆域,如果打胜了,您的国君就会更加骄纵,打败鲁国使得齐国的大臣更加受到尊崇,然而您的功劳却不在其中,那么您和国君的关系就会一天天疏远。这样您对上使国君产生骄纵的心理,对下使群臣放纵无羁,想以此成就大事,太困难了。国君骄纵就会随心所欲,群臣骄妄就会争权夺利,这样对上与国君有矛盾,对下与大臣互相争夺。像这样的话,您在齐国的处境就很危险了。所以说攻打鲁国不如攻打吴国。攻打吴国如果不能取胜,百姓战死在国外,大臣率兵在外作战,国内空虚,这样您在上没有强臣对抗,在下没有百姓责难,能够孤立国君、控制齐国的就只有您了。”

田常说:“好。虽然如此,但是我的军队已经开赴鲁国了,现在从鲁国撤离而进兵吴国,大臣们会怀疑我,该怎么办?”

子贡说:“您按兵不动,不要发动进攻,我请求出使吴国,让它出兵援助鲁国而攻打齐国,您就趁机出兵迎击它。”

田常采纳了子贡的建议,派子贡南下拜见吴王。

\begin{yuanwen}
说曰:“臣闻之,王者不绝世,霸者无彊敌,千钧之重加铢两而移。今以万乘之齐而私千乘之鲁,与吴争彊,窃为王危之。且夫救鲁,显名也;伐齐,大利也。以抚泗上诸侯,诛暴齐以服彊晋,利莫大焉。名存亡鲁,实困彊齐。智者不疑也。”吴王曰:“善。虽然,吾尝与越战,栖之会稽。越王苦身养士,有报我心。子待我伐越而听子。”子贡曰:“越之劲不过鲁,吴之彊不过齐,王置齐而伐越,则齐已平鲁矣。且王方以存亡继绝为名,夫伐小越而畏彊齐,非勇也。夫勇者不避难,仁者不穷约,智者不失时,王者不绝世,以立其义。今存越示诸侯以仁,救鲁伐齐,威加晋国,诸侯必相率而朝吴,霸业成矣。且王必恶越,臣请东见越王,令出兵以从,此实空越,名从诸侯以伐也。”吴王大说,乃使子贡之越。
\end{yuanwen}

\begin{yuanwen}
越王除道郊迎,身御至舍而问曰:“此蛮夷之国,大夫何以俨然辱而临之?”子贡曰:“今者吾说吴王以救鲁伐齐,其志欲之而畏越,曰‘待我伐越乃可’。如此,破越必矣。且夫无报人之志而令人疑之,拙也;有报人之志,使人知之,殆也;事未发而先闻,危也。三者举事之大患。”句践顿首再拜曰:“孤尝不料力,乃与吴战,困于会稽,痛入于骨髓,日夜焦脣乾舌,徒欲与吴王接踵而死,孤之原也。”遂问子贡。子贡曰:“吴王为人猛暴,群臣不堪;国家敝以数战,士卒弗忍;百姓怨上,大臣内变;子胥以谏死,太宰嚭用事,顺君之过以安其私:是残国之治也。今王诚发士卒佐之徼其志,重宝以说其心,卑辞以尊其礼,其伐齐必也。彼战不胜,王之福矣。战胜,必以兵临晋,臣请北见晋君,令共攻之,弱吴必矣。其锐兵尽于齐,重甲困于晋,而王制其敝,此灭吴必矣。”越王大说,许诺。送子贡金百镒,剑一,良矛二。子贡不受,遂行。

报吴王曰:“臣敬以大王之言告越王,越王大恐,曰:‘孤不幸,少失先人,内不自量,抵罪于吴,军败身辱,栖于会稽,国为虚莽,赖大王之赐,使得奉俎豆而修祭祀,死不敢忘,何谋之敢虑!’”后五日,越使大夫种顿首言于吴王曰:“东海役臣孤句践使者臣种,敢修下吏问于左右。今窃闻大王将兴大义,诛彊救弱,困暴齐而抚周室,请悉起境内士卒三千人,孤请自被坚执锐,以先受矢石。因越贱臣种奉先人藏器,甲二十领,鈇屈卢之矛,步光之剑,以贺军吏。”吴王大说,以告子贡曰:“越王欲身从寡人伐齐,可乎?”子贡曰:“不可。夫空人之国,悉人之众,又从其君,不义。君受其币,许其师,而辞其君。”吴王许诺,乃谢越王。于是吴王乃遂发九郡兵伐齐。
\end{yuanwen}

\begin{yuanwen}
子贡因去之晋,谓晋君曰:“臣闻之,虑不先定不可以应卒,兵不先辨不可以胜敌。今夫齐与吴将战,彼战而不胜,越乱之必矣;与齐战而胜,必以其兵临晋。”晋君大恐,曰:“为之柰何?”子贡曰:“修兵休卒以待之。”晋君许诺。

子贡去而之鲁。吴王果与齐人战于艾陵,大破齐师,获七将军之兵而不归,果以兵临晋,与晋人相遇黄池之上。吴晋争彊。晋人击之,大败吴师。越王闻之,涉江袭吴,去城七里而军。吴王闻之,去晋而归,与越战于五湖。三战不胜,城门不守,越遂围王宫,杀夫差而戮其相。破吴三年,东向而霸。

故子贡一出,存鲁,乱齐,破吴,彊晋而霸越。子贡一使,使势相破,十年之中,五国各有变。

子贡好废举,与时转货赀。喜扬人之美,不能匿人之过。常相鲁卫,家累千金,卒终于齐。

言偃,吴人,字子游。少孔子四十五岁。

子游既已受业,为武城宰。孔子过,闻弦歌之声。孔子莞尔而笑曰:“割鸡焉用牛刀?”子游曰:“昔者偃闻诸夫子曰,君子学道则爱人,小人学道则易使。”孔子曰:“二三子,偃之言是也。前言戏之耳。”孔子以为子游习于文学。

卜商字子夏。少孔子四十四岁。

子夏问:“‘巧笑倩兮,美目盼兮,素以为绚兮’,何谓也?”子曰:“绘事后素。”曰:“礼后乎?”孔子曰:“商始可与言诗已矣。”

子贡问:“师与商孰贤?”子曰:“师也过,商也不及。”“然则师愈与?”曰:“过犹不及。”

子谓子夏曰:“汝为君子儒,无为小人儒。”

孔子既没,子夏居西河教授,为魏文侯师。其子死,哭之失明。

颛孙师,陈人,字子张。少孔子四十八岁。

子张问干禄,孔子曰:“多闻阙疑,慎言其馀,则寡尤;多见阙殆,慎行其馀,则寡悔。言寡尤,行寡悔,禄在其中矣。”

他日从在陈蔡间,困,问行。孔子曰:“言忠信,行笃敬,虽蛮貊之国行也;言不忠信,行不笃敬,虽州里行乎哉!立则见其参于前也,在舆则见其倚于衡,夫然后行。”子张书诸绅。

子张问:“士何如斯可谓之达矣?”孔子曰:“何哉,尔所谓达者?”子张对曰:“在国必闻,在家必闻。”孔子曰:“是闻也,非达也。夫达者,质直而好义,察言而观色,虑以下人,在国及家必达。夫闻也者,色取仁而行违,居之不疑,在国及家必闻。”

曾参,南武城人,字子舆。少孔子四十六岁。

孔子以为能通孝道,故授之业。作孝经。死于鲁。

澹台灭明,武城人,字子羽。少孔子三十九岁。

状貌甚恶。欲事孔子,孔子以为材薄。既已受业,退而修行,行不由径,非公事不见卿大夫。

南游至江,从弟子三百人,设取予去就,名施乎诸侯。孔子闻之,曰:“吾以言取人,失之宰予;以貌取人,失之子羽。”

宓不齐字子贱。少孔子三十岁。

孔子谓“子贱君子哉!鲁无君子,斯焉取斯?”

子贱为单父宰,反命于孔子,曰:“此国有贤不齐者五人,教不齐所以治者。”孔子曰:“惜哉不齐所治者小,所治者大则庶几矣。”

原宪字子思。

子思问耻。孔子曰:“国有道,穀\footnote{指俸禄。}。国无道,穀,耻也。”

子思曰:“克伐怨欲不行焉,可以为仁乎?”孔子曰:“可以为难矣,仁则吾弗知也。”

孔子卒,原宪遂亡在草泽中。子贡相卫,而结驷连骑,排藜藿入穷阎,过谢原宪。宪摄敝衣冠见子贡。子贡耻之,曰:“夫子岂病乎?”原宪曰:“吾闻之,无财者谓之贫,学道而不能行者谓之病。若宪,贫也,非病也。”子贡惭,不怿而去,终身耻其言之过也。

公冶长,齐人,字子长。

孔子曰:“长可妻也,虽在累绁之中,非其罪也。”以其子妻之。

南宫括字子容。

问孔子曰:“羿善射,奡荡舟,俱不得其死然;禹稷躬稼而有天下?”孔子弗答。容出,孔子曰:“君子哉若人!上德哉若人!”“国有道,不废;国无道,免于刑戮。”三复“白珪之玷”,以其兄之子妻之。

公皙哀字季次。

孔子曰:“天下无行,多为家臣,仕于都;唯季次未尝仕。”

曾字皙。

侍孔子,孔子曰:“言尔志。”曰:“春服既成,冠者五六人,童子六七人,浴乎沂,风乎舞雩,咏而归。”孔子喟尔叹曰:“吾与也!”

颜无繇字路。路者,颜回父,父子尝各异时事孔子。

颜回死,颜路贫,请孔子车以葬。孔子曰:“材不材,亦各言其子也。鲤也死,有棺而无椁,吾不徒行以为之椁,以吾从大夫之后,不可以徒行。”

商瞿,鲁人,字子木。少孔子二十九岁。

孔子传易于瞿,瞿传楚人馯臂子弘,弘传江东人矫子庸疵,疵传燕人周子家竖,竖传淳于人光子乘羽,羽传齐人田子庄何,何传东武人王子中同,同传菑川人杨何。何元朔中以治易为汉中大夫。

高柴字子羔。少孔子三十岁。

子羔长不盈五尺,受业孔子,孔子以为愚。

子路使子羔为费郈宰,孔子曰:“贼夫人之子!”子路曰:“有民人焉,有社稷焉,何必读书然后为学!”孔子曰:“是故恶夫佞者。”

漆彫开字子开。

孔子使开仕,对曰:“吾斯之未能信。”孔子说。

公伯缭字子周。

周愬子路于季孙,子服景伯以告孔子,曰:“夫子固有惑志,缭也,吾力犹能肆诸市朝。”孔子曰:“道之将行,命也;道之将废,命也。公伯缭其如命何!”

司马耕字子牛。

牛多言而躁。问仁于孔子,孔子曰:“仁者其言也讱。”曰:“其言也讱,斯可谓之仁乎?”子曰:“为之难,言之得无讱乎!”

问君子,子曰:“君子不忧不惧。”曰:“不忧不惧,斯可谓之君子乎?”子曰:“内省不疚,夫何忧何惧!”

樊须字子迟。少孔子三十六岁。

樊迟请学稼,孔子曰:“吾不如老农。”请学圃,曰:“吾不如老圃。”樊迟出,孔子曰:“小人哉樊须也!上好礼,则民莫敢不敬;上好义,则民莫敢不服;上好信,则民莫敢不用情。夫如是,则四方之民襁负其子而至矣,焉用稼!”

樊迟问仁,子曰:“爱人。”问智,曰:“知人。”

有若少孔子四十三岁。有若曰:“礼之用,和为贵,先王之道斯为美。小大由之,有所不行;知和而和,不以礼节之,亦不可行也。”“信近于义,言可复也;恭近于礼,远耻辱也;因不失其亲,亦可宗也。”

孔子既没,弟子思慕,有若状似孔子,弟子相与共立为师,师之如夫子时也。他日,弟子进问曰:“昔夫子当行,使弟子持雨具,已而果雨。弟子问曰:‘夫子何以知之?’夫子曰:‘诗不云乎?“月离于毕,俾滂沱矣。”昨暮月不宿毕乎?’他日,月宿毕,竟不雨。商瞿年长无子,其母为取室。孔子使之齐,瞿母请之。孔子曰:‘无忧,瞿年四十后当有五丈夫子。’已而果然。问夫子何以知此?”有若默然无以应。弟子起曰:“有子避之,此非子之座也!”

公西赤字子华。少孔子四十二岁。

子华使于齐,厓有为其母请粟。孔子曰:“与之釜。”请益,曰:“与之庾。”厓子与之粟五秉。孔子曰:“赤之適齐也,乘肥马,衣轻裘。吾闻君子周急不继富。”

巫马施字子旗。少孔子三十岁。

陈司败问孔子曰:“鲁昭公知礼乎?”孔子曰:“知礼。”退而揖巫马旗曰:“吾闻君子不党,君子亦党乎?鲁君娶吴女为夫人,命之为孟子。孟子姓姬,讳称同姓,故谓之孟子。鲁君而知礼,孰不知礼!”施以告孔子,孔子曰:“丘也幸,苟有过,人必知之。臣不可言君亲之恶,为讳者,礼也。”

梁鱣字叔鱼。少孔子二十九岁。

颜幸字子柳。少孔子四十六岁。

厓孺字子鲁,少孔子五十岁。

曹恤字子循。少孔子五十岁。

伯虔字子析,少孔子五十岁。

公孙龙字子石。少孔子五十三岁。

自子石已右三十五人,显有年名及受业见于书传。其四十有二人,无年及不见书传者纪于左:

厓季字子产。

公祖句兹字子之。

秦祖字子南。

漆雕哆字子敛。

颜高字子骄。

漆雕徒父。

壤驷赤字子徒。

商泽。

石作蜀字子明。

任不齐字选。

公良孺字子正。

后处字子里。

秦厓字开。

公夏首字乘。

奚容箴字子皙。

公肩定字子中。

颜祖字襄。

鄡单字子家。

句井疆。

罕父黑字子索。

秦商字子丕。

申党字周。

颜之仆字叔。

荣旂字子祈。

县成字子祺。

左人郢字行。

燕伋字思。

郑国字子徒。

秦非字子之。

施之常字子恆。

颜哙字子声。

步叔乘字子车。

原亢籍。

乐欬字子声。

廉絜,字庸。

叔仲会,字子期。

颜何,字厓。

狄黑,字皙。

邦巽字子敛。

孔忠。

公西舆如,字子上。

公西葴,字子上。

太史公曰:学者多称七十子之徒,誉者或过其实,毁者或损其真,钧之未睹厥容貌,则论言弟子籍,出孔氏古文近是。余以弟子名姓文字悉取论语弟子问并次为篇,疑者阙焉。

教兴阙里,道在郰乡。异能就列,秀士升堂。依仁游艺,合志同方。将师宫尹,俎豆琳琅。惜哉不霸,空臣素王!
\end{yuanwen}

\chapter{商君列传}

记叙了商鞅佐秦孝公实行变法,使秦国空前富强的丰功伟绩,和后来秦国发生政变,商鞅惨遭杀害的全过程。

\begin{yuanwen}
商君者,卫之诸庶孽公子也\footnote{text},名鞅,姓公孙氏,其祖本姬姓也。鞅少好刑名之学\footnote{text},事魏相公叔座为中庶子\footnote{text}。公叔座知其贤,未及进。会座病,魏惠王亲往问病,曰:“公叔病,有如不可讳\footnote{text},将奈社稷何?”

公叔曰:“座之中庶子公孙鞅,年虽少,有奇才,原王举国而听之。”

王嘿然\footnote{text}。王且去,座屏人言曰\footnote{text}:“王即不听用鞅,必杀之,无令出境。”

王许诺而去。公叔座召鞅谢曰:“今者王问可以为相者,我言若,王色不许我。我方先君后臣,因谓王‘即弗用鞅,当杀之’。王许我。汝可疾去矣,且见禽\footnote{text}。”

鞅曰:“彼王不能用君之言任臣,又安能用君之言杀臣乎?”卒不去。

惠王既去,而谓左右曰:“公叔病甚,悲乎,欲令寡人以国听公孙鞅也,岂不悖\footnote{text}哉!”
\end{yuanwen}

\begin{yuanwen}
公叔既死,公孙鞅闻秦孝公下令国中求贤者,将修缪公之业\footnote{text},东复侵地\footnote{text}。乃遂西入秦,因孝公宠臣景监以求见孝公。孝公既见卫鞅,语事良久,孝公时时睡,弗听。罢而孝公怒景监曰:“子之客妄人耳\footnote{text},安足用邪!”

景监以让卫鞅。卫鞅曰:“吾说公以帝道\footnote{text},其志不开悟矣。后五日,复求见鞅\footnote{text}。”

鞅复见孝公,益愈\footnote{text},然而未中旨。罢而孝公复让景监,景监亦让鞅。

鞅曰:“吾说公以王道而未入也\footnote{text},请复见鞅。”

鞅复见孝公,孝公善之而未用也。罢而去,孝公谓景监曰:“汝客善,可与语矣。”

鞅曰:“吾说公以霸道\footnote{text},其意欲用之矣。诚复见我,我知之矣。”

卫鞅复见孝公。公与语,不自知膝之前于席也。语数日不厌。

景监曰:“子何以中吾君?吾君之欢甚也。”

鞅曰:“吾说君以帝王之道比三代,而君曰:‘久远,吾不能待。且贤君者,各及其身显名天下,安能邑邑待数十百年以成帝王乎?’故吾以强国之术说君,君大说之耳。然亦难以比德于殷周矣。”
\end{yuanwen}

\begin{yuanwen}
孝公既用卫鞅,鞅欲变法,恐天下议己。卫鞅曰:“疑行无名,疑事无功\footnote{text}。且夫有高人之行者,固见非于世;有独知之虑\footnote{text}者,必见敖\footnote{text}于民。愚者闇于成事\footnote{text},知者见于未萌\footnote{text}。民不可与虑始,而可与乐成。论至德者不和于俗\footnote{text},成大功者不谋于众。是以圣人苟可以强国,不法其故;苟可以利民,不循其礼。”

孝公曰:“善。”

甘龙曰:“不然。圣人不易民而教\footnote{text},知者不变法而治。因民而教\footnote{text},不劳而成功;缘法而治者\footnote{text},吏习而民安之。”

卫鞅曰:“龙之所言,世俗之言也。常人安于故俗,学者溺于所闻\footnote{text}。以此两者居官守法可也,非所与论于法之外也\footnote{text}。三代不同礼而王,五伯不同法而霸。智者作法,愚者制焉;贤者更礼,不肖者拘焉。”

杜挚曰:“利不百,不变法;功不十,不易器。法古无过,循礼无邪。”

卫鞅曰:“治世不一道,便国不法古。故汤武不循古而王,夏殷不易礼而亡\footnote{text}。反古者不可非,而循礼者不足多。”

孝公曰:“善。”以卫鞅为左庶长\footnote{text},卒定变法之令\footnote{text}。
\end{yuanwen}

\begin{yuanwen}
令民为什伍,而相牧司连坐\footnote{text}。不告奸者腰斩,告奸者与斩敌首同赏,匿奸者与降敌同罚。民有二男以上不分异\footnote{text}者,倍其赋。有军功者,各以率受上爵\footnote{text};为私斗者,各以轻重被刑大小。僇力本业\footnote{text},耕织致粟帛多者复其身\footnote{text}。事末利及怠而贫者\footnote{text},举以为收孥\footnote{text}。宗室非有军功论\footnote{text},不得为属籍\footnote{text}。明尊卑爵秩等级\footnote{text},各以差次名田宅\footnote{text},臣妾衣服以家次\footnote{text}。有功者显荣,无功者虽富无所芬华\footnote{text}。
\end{yuanwen}

\begin{yuanwen}
令既具,未布,恐民之不信已,乃立三丈之木于国都市南门,募民有能徙置北门者予十金。民怪之,莫敢徙。复曰:“能徙者予五十金”。

有一人徙之,辄予五十金,以明不欺。卒下令。
\end{yuanwen}

\begin{yuanwen}
令行于民期年,秦民之国都言初令之不便者以千数\footnote{text}。于是太子犯法\footnote{text}。卫鞅曰:“法之不行,自上犯之。”

将法太子。太子,君嗣也\footnote{text},不可施刑,刑其傅公子虔,黥其师公孙贾。明日,秦人皆趋令\footnote{text}。行之十年\footnote{text},秦民大说,道不拾遗,山无盗贼,家给人足。民勇于公战,怯于私斗,乡邑大治。秦民初言令不便者有来言令便者,卫鞅曰:“此皆乱化之民也\footnote{text}。”尽迁之于边城。其后民莫敢议令。
\end{yuanwen}

\begin{yuanwen}
于是以鞅为大良造\footnote{text}。将兵围魏安邑,降之\footnote{text}。居三年,作为筑冀阙宫庭于咸阳\footnote{text},秦自雍徙都之\footnote{text}。而令民父子兄弟同室内息者为禁\footnote{text}。而集小乡邑聚为县\footnote{text},置令、丞\footnote{text},凡三十一县。为田开阡陌封疆\footnote{text},而赋税平。平斗桶权衡丈尺\footnote{text}。行之四年,公子虔复犯约,劓之\footnote{text}。居五年\footnote{text},秦人富强,天子致胙于孝公\footnote{text},诸侯毕贺。
\end{yuanwen}

\begin{yuanwen}
其明年,齐败魏兵于马陵,虏其太子申,杀将军庞涓。其明年,卫鞅说孝公曰:“秦之与魏,譬若人之有腹心疾\footnote{text},非魏并秦,秦即并魏。何者?魏居领厄之西\footnote{text},都安邑\footnote{text},与秦界河而独擅山东之利\footnote{text}。利则西侵秦\footnote{text},病则东收地\footnote{text}。今以君之贤圣,国赖以盛。而魏往年大破于齐,诸侯畔之\footnote{text},可因此时伐魏。魏不支秦,必东徙。东徙,秦据河山之固\footnote{text},东乡以制诸侯,此帝王之业也。”

孝公以为然,使卫鞅将而伐魏。魏使公子卬将而击之\footnote{text}。军既相距\footnote{text},卫鞅遗魏将公子卬书曰:“吾始与公子欢,今俱为两国将,不忍相攻。可与公子面相见,盟,乐饮而罢兵,以安秦、魏。”

魏公子卬以为然。会盟已\footnote{text},饮,而卫鞅伏甲士而袭虏魏公子卬,因攻其军,尽破之以归秦。魏惠王兵数破于齐、秦,国内空,日以削,恐,乃使使割河西之地献于秦以和\footnote{text}。而魏遂去安邑,徙都大梁\footnote{text}。

梁惠王曰:“寡人恨不用公叔座之言也。”

卫鞅既破魏还,秦封之於、商十五邑\footnote{text},号为商君\footnote{text}。
\end{yuanwen}

\begin{yuanwen}
商君相秦十年\footnote{text},宗室贵戚多怨望者\footnote{text}。赵良见商君。商君曰:“鞅之得见也,从孟兰皋,今鞅请得交,可乎?”赵良曰:“仆弗敢原也。孔丘有言曰:“推贤而戴者进,聚不肖而王者退。”仆不肖,故不敢受命。仆闻之曰:“非其位而居之曰贪位,非其名而有之曰贪名。”仆听君之义,则恐仆贪位贪名也。故不敢闻命。”商君曰:“子不说吾治秦与?”赵良曰:“反听之谓聪,内视之谓明,自胜之谓强。虞舜有言曰:“自卑也尚矣。”君不若道虞舜之道,无为问仆矣。”商君曰:“始秦戎翟之教,父子无别,同室而居。今我更制其教,而为其男女之别,大筑冀阙,营如鲁卫矣。子观我治秦也,孰与五羖大夫贤?”赵良曰:“千羊之皮,不如一狐之掖;千人之诺诺,不如一士之谔谔。武王谔谔以昌,殷纣墨墨以亡。君若不非武王乎,则仆请终日正言而无诛,可乎?”商君曰:“语有之矣,貌言华也,至言实也,苦言药也,甘言疾也。夫子果肯终日正言,鞅之药也。鞅将事子,子又何辞焉!”赵良曰:“夫五羖大夫,荆之鄙人也。闻秦缪公之贤而原望见,行而无资,自粥于秦客,被褐食牛。期年,缪公知之,举之牛口之下,而加之百姓之上,秦国莫敢望焉。相秦六七年,而东伐郑,三置晋国之君,一救荆国之祸。发教封内,而巴人致贡;施德诸侯,而八戎来服。由余闻之,款关请见。五羖大夫之相秦也,劳不坐乘,暑不张盖,行于国中,不从车乘,不操干戈,功名藏于府库,德行施于后世。五羖大夫死,秦国男女流涕,童子不歌谣,舂者不相杵。此五羖大夫之德也。今君之见秦王也,因嬖人景监以为主,非所以为名也。相秦不以百姓为事,而大筑冀阙,非所以为功也。刑黥太子之师傅,残伤民以骏刑,是积怨畜祸也。教之化民也深于命,民之效上也捷于令。今君又左建外易,非所以为教也。君又南面而称寡人,日绳秦之贵公子。诗曰:“相鼠有体,人而无礼,人而无礼,何不遄死。”以诗观之,非所以为寿也。公子虔杜门不出已八年矣,君又杀祝懽而黥公孙贾。诗曰:“得人者兴,失人者崩。”此数事者,非所以得人也。君之出也,后车十数,从车载甲,多力而骈胁者为骖乘,持矛而操闟戟者旁车而趋。此一物不具,君固不出。书曰:“恃德者昌,恃力者亡。”君之危若朝露,尚将欲延年益寿乎?则何不归十五都,灌园于鄙,劝秦王显岩穴之士,养老存孤,敬父兄,序有功,尊有德,可以少安。君尚将贪商于之富,宠秦国之教,畜百姓之怨,秦王一旦捐宾客而不立朝,秦国之所以收君者,岂其微哉?亡可翘足而待。”商君弗从。
\end{yuanwen}

\begin{yuanwen}
后五月而秦孝公卒,太子立。公子虔之徒告商君欲反,发吏捕商君\footnote{text}。商君亡至关下\footnote{text},欲舍客舍。客人不知其是商君也\footnote{text},曰:“商君之法,舍人无验者坐之\footnote{text}。”

商君喟然叹曰:“嗟乎,为法之敝一至此哉!”

去之魏。魏人怨其欺公子卬而破魏师,弗受。商君欲之他国。魏人曰:“商君,秦之贼。秦强而贼入魏,弗归\footnote{text},不可。”

遂内秦\footnote{text}。商君既复入秦,走商邑,与其徒属发邑兵北出击郑\footnote{text}。秦发兵攻商君,杀之于郑黾池\footnote{text}。秦惠王车裂商君以徇\footnote{text},曰:“莫如商鞅反者!”遂灭商君之家。
\end{yuanwen}

\begin{yuanwen}
太史公曰:商君,其天资刻薄人也\footnote{text}。迹其欲干孝公以帝王术\footnote{text},挟持浮说,非其质矣\footnote{text}。且所因由嬖臣,及得用,刑公子虔,欺魏将卬,不师赵良之言,亦足发明商君之少恩\footnote{text}矣。余尝读商君《开塞》、《耕战》书\footnote{text},与其人行事相类。卒受恶名于秦,有以也夫!

卫鞅入秦,景监是因。王道不用,霸术见亲。政必改革,礼岂因循。既欺魏将,亦怨秦人。如何作法,逆旅不宾!
\end{yuanwen}

\chapter{苏秦列传}

\begin{yuanwen}
苏秦者,东周雒阳人也。东事师于齐,而习之于鬼谷先生。

出游数岁,大困而归。兄弟嫂妹妻妾窃皆笑之,曰:“周人之俗,治产业,力工商,逐什二以为务。今子释本而事口舌,困,不亦宜乎!”苏秦闻之而惭,自伤,乃闭室不出,出其书遍观之。曰:“夫士业已屈首受书,而不能以取尊荣,虽多亦奚以为!”于是得周书阴符,伏而读之。期年,以出揣摩,曰:“此可以说当世之君矣。”求说周显王。显王左右素习知苏秦,皆少之。弗信。

乃西至秦。秦孝公卒。说惠王曰:“秦四塞之国,被山带渭,东有关河,西有汉中,南有巴蜀,北有代马,此天府也。以秦士民之众,兵法之教,可以吞天下,称帝而治。”秦王曰:“毛羽未成,不可以高蜚;文理未明,不可以并兼。”方诛商鞅,疾辩士,弗用。

乃东之赵。赵肃侯令其弟成为相,号奉阳君。奉阳君弗说之。

去游燕,岁馀而后得见。说燕文侯曰:“燕东有朝鲜、辽东,北有林胡、楼烦,西有云中、九原,南有呼沱、易水,地方二千馀里,带甲数十万,车六百乘,骑六千匹,粟支数年。南有碣石、雁门之饶,北有枣栗之利,民虽不佃作而足于枣栗矣。此所谓天府者也。

“夫安乐无事,不见覆军杀将,无过燕者。大王知其所以然乎?夫燕之所以不犯寇被甲兵者,以赵之为蔽其南也。秦赵五战,秦再胜而赵三胜。秦赵相毙,而王以全燕制其后,此燕之所以不犯寇也。且夫秦之攻燕也,逾云中、九原,过代、上谷,弥地数千里,虽得燕城,秦计固不能守也。秦之不能害燕亦明矣。今赵之攻燕也,发号出令,不至十日而数十万之军军于东垣矣。渡呼沱,涉易水,不至四五日而距国都矣。故曰秦之攻燕也,战于千里之外;赵之攻燕也,战于百里之内。夫不忧百里之患而重千里之外,计无过于此者。是故原大王与赵从亲,天下为一,则燕国必无患矣。”

文侯曰:“子言则可,然吾国小,西迫彊赵,南近齐,齐、赵彊国也。子必欲合从以安燕,寡人请以国从。”

于是资苏秦车马金帛以至赵。而奉阳君已死,即因说赵肃侯曰:“天下卿相人臣及布衣之士,皆高贤君之行义,皆原奉教陈忠于前之日久矣。虽然,奉阳君妒而君不任事,是以宾客游士莫敢自尽于前者。今奉阳君捐馆舍,君乃今复与士民相亲也,臣故敢进其愚虑。

“窃为君计者,莫若安民无事,且无庸有事于民也。安民之本,在于择交,择交而得则民安,择交而不得则民终身不安。请言外患:齐秦为两敌而民不得安,倚秦攻齐而民不得安,倚齐攻秦而民不得安。故夫谋人之主,伐人之国,常苦出辞断绝人之交也。原君慎勿出于口。请别白黑所以异,阴阳而已矣。君诚能听臣,燕必致旃裘狗马之地,齐必致鱼盐之海,楚必致橘柚之园,韩、魏、中山皆可使致汤沐之奉,而贵戚父兄皆可以受封侯。夫割地包利,五伯之所以覆军禽将而求也;封侯贵戚,汤武之所以放弑而争也。今君高拱而两有之,此臣之所以为君原也。

“今大王与秦,则秦必弱韩、魏;与齐,则齐必弱楚、魏。魏弱则割河外,韩弱则效宜阳,宜阳效则上郡绝,河外割则道不通,楚弱则无援。此三策者,不可不孰计也。

“夫秦下轵道,则南阳危;劫韩包周,则赵氏自操兵;据卫取卷,则齐必入朝秦。秦欲已得乎山东,则壁举兵而乡赵矣。秦甲渡河逾漳,据番吾,则兵必战于邯郸之下矣。此臣之所为君患也。

“当今之时,山东之建国莫彊于赵。赵地方二千馀里,带甲数十万,车千乘,骑万匹,粟支数年。西有常山,南有河漳,东有清河,北有燕国。燕固弱国,不足畏也。秦之所害于天下者莫如赵,然而秦不敢举兵伐赵者,何也?畏韩、魏之议其后也。然则韩、魏,赵之南蔽也。秦之攻韩、魏也,无有名山大川之限,稍蚕食之,傅国都而止。韩、魏不能支秦,必入臣于秦。秦无韩、魏之规,则祸必中于赵矣。此臣之所为君患也。

“臣闻尧无三夫之分,舜无咫尺之地,以有天下;禹无百人之聚,以王诸侯;汤武之士不过三千,车不过三百乘,卒不过三万,立为天子:诚得其道也。是故明主外料其敌之彊弱,内度其士卒贤不肖,不待两军相当而胜败存亡之机固已形于胸中矣,岂揜于众人之言而以冥冥决事哉!

“臣窃以天下之地图案之,诸侯之地五倍于秦,料度诸侯之卒十倍于秦,六国为一,并力西乡而攻秦,秦必破矣。今西面而事之,见臣于秦。夫破人之与破于人也,臣人之与臣于人也,岂可同日而论哉!

“夫衡人者,皆欲割诸侯之地以予秦。秦成,则高台榭,美宫室,听竽瑟之音,前有楼阙轩辕,后有长姣美人,国被秦患而不与其忧。是故夫衡人日夜务以秦权恐愒诸侯以求割地,故原大王孰计之也。

“臣闻明主绝疑去谗,屏流言之迹,塞朋党之门,故尊主广地彊兵之计臣得陈忠于前矣。故窃为大王计,莫如一韩、魏、齐、楚、燕、赵以从亲,以畔秦。令天下之将相会于洹水之上,通质,刳白马而盟。要约曰:‘秦攻楚,齐、魏各出锐师以佐之,韩绝其粮道,赵涉河漳,燕守常山之北。秦攻韩魏,则楚绝其后,齐出锐师而佐之,赵涉河漳,燕守云中。秦攻齐,则楚绝其后,韩守城皋,魏塞其道,赵涉河漳、博关,燕出锐师以佐之。秦攻燕,则赵守常山,楚军武关,齐涉勃海,韩、魏皆出锐师以佐之。秦攻赵,则韩军宜阳,楚军武关,魏军河外,齐涉清河,燕出锐师以佐之。诸侯有不如约者,以五国之兵共伐之。’六国从亲以宾秦,则秦甲必不敢出于函谷以害山东矣。如此,则霸王之业成矣。”

赵王曰:“寡人年少,立国日浅,未尝得闻社稷之长计也。今上客有意存天下,安诸侯寡人敬以国从。”乃饰车百乘,黄金千溢,白璧百双,锦绣千纯,以约诸侯。

是时周天子致文武之胙于秦惠王。惠王使犀首攻魏,禽将龙贾,取魏之雕阴,且欲东兵。苏秦恐秦兵之至赵也,乃激怒张仪,入之于秦。

于是说韩宣王曰:“韩北有巩、成皋之固,西有宜阳、商阪之塞,东有宛、穰、洧水,南有陉山,地方九百馀里,带甲数十万,天下之彊弓劲弩皆从韩出。谿子、少府时力、距来者,皆射六百步之外。韩卒超足而射,百发不暇止,远者括蔽洞胸,近者镝弇心。韩卒之剑戟皆出于冥山、棠谿、墨阳、合赙、邓师、宛冯、龙渊、太阿,皆陆断牛马,水截鹄雁,当敌则斩坚甲铁幕,革抉簠芮,无不毕具。以韩卒之勇,被坚甲,蹠劲弩,带利剑,一人当百,不足言也。夫以韩之劲与大王之贤,乃西面事秦,交臂而服,羞社稷而为天下笑,无大于此者矣。是故原大王孰计之。

“大王事秦,秦必求宜阳、成皋。今兹效之,明年又复求割地。与则无地以给之,不与则弃前功而受后祸。且大王之地有尽而秦之求无已,以有尽之地而逆无已之求,此所谓市怨结祸者也,不战而地已削矣。臣闻鄙谚曰:‘宁为鸡口,无为牛后。’今西面交臂而臣事秦,何异于牛后乎?夫以大王之贤,挟彊韩之兵,而有牛后之名,臣窃为大王羞之。”

于是韩王勃然作色,攘臂瞋目,按剑仰天太息曰;“寡人虽不肖,必不能事秦。今主君诏以赵王之教,敬奉社稷以从。”

又说魏襄王曰:“大王之地,南有鸿沟、陈、汝南、许、郾、昆阳、召陵、舞阳、新都、新郪,东有淮、颍、枣、无胥,西有长城之界,北有河外、卷、衍、酸枣,地方千里。地名虽小,然而田舍庐庑之数,曾无所刍牧。人民之众,车马之多,日夜行不绝,輷輷殷殷,若有三军之众。臣窃量大王之国不下楚。,然衡人怵王交彊虎狼之秦以侵天下,卒有秦患,不顾其祸。夫挟彊秦之势以内劫其主,罪无过此者。魏,天下之彊国也;王,天下之贤王也。今乃有意西面而事秦,称东籓,筑帝宫,受冠带,祠春秋,臣窃为大王耻之。

“臣闻越王句践战敝卒三千人,禽夫差于干遂;武王卒三千人,革车三百乘,制纣于牧野:岂其士卒众哉,诚能奋其威也。今窃闻大王之卒,武士二十万,苍头二十万,奋击二十万,厮徒十万,车六百乘,骑五千匹。此其过越王句践、武王远矣,今乃听于群臣之说而欲臣事秦。夫事秦必割地以效实,故兵未用而国已亏矣。凡群臣之言事秦者,皆奸人,非忠臣也。夫为人臣,割其主之地以求外交,偷取一时之功而不顾其后,破公家而成私门,外挟彊秦之势以内劫其主,以求割地,原大王孰察之。

“周书曰:‘釂釂不绝,蔓蔓柰何?豪氂不伐,将用斧柯。’前虑不定,后有大患,将柰之何?大王诚能听臣,六国从亲,专心并力壹意,则必无彊秦之患。故敝邑赵王使臣效愚计,奉明约,在大王之诏诏之。”

魏王曰:“寡人不肖,未尝得闻明教。今主君以赵王之诏诏之,敬以国从。”

因东说齐宣王曰:“齐南有泰山,东有琅邪,西有清河,北有勃海,北所谓四塞之国也。齐地方二千馀里,带甲数十万,粟如丘山。三军之良,五家之兵,进如锋矢,战如雷霆,解如风雨。即有军役,未尝倍泰山,绝清河,涉勃海也。临菑之中七万户,臣窃度之,不下户三男子,三七二十一万,不待发于远县,而临菑之卒固已二十一万矣。临菑甚富而实,其民无不吹竽鼓瑟,弹琴击筑,斗鸡走狗,六博蹋鞠者。临菑之涂,车毂击,人肩摩,连衽成帷,举袂成幕,挥汗成雨,家殷人足,志高气扬。夫以大王之贤与齐之彊,天下莫能当。今乃西面而事秦,臣窃为大王羞之。

“且夫韩、魏之所以重畏秦者,为与秦接境壤界也。兵出而相当,不出十日而战胜存亡之机决矣。韩、魏战而胜秦,则兵半折,四境不守;战而不胜,则国已危亡随其后。是故韩、魏之所以重与秦战,而轻为之臣也。今秦之攻齐则不然。倍韩、魏之地,过卫阳晋之道,径乎亢父之险,车不得方轨,骑不得比行,百人守险,千人不敢过也。秦虽欲深入,则狼顾,恐韩、魏之议其后也。是故恫疑虚猲,骄矜而不敢进,则秦之不能害齐亦明矣。

“夫不深料秦之无柰齐何,而欲西面而事之,是群臣之计过也。今无臣事秦之名而有彊国之实,臣是故原大王少留意计之。”

齐王曰:“寡人不敏,僻远守海,穷道东境之国也,未尝得闻馀教。今足下以赵王诏诏之,敬以国从。”

乃西南说楚威王曰:“楚,天下之彊国也;王,天下之贤王也。西有黔中、巫郡,东有夏州、海阳,南有洞庭、苍梧,北有陉塞、郇阳,地方五千馀里,带甲百万,车千乘,骑万匹,粟支十年。此霸王之资也。夫以楚之彊与王之贤,天下莫能当也。今乃欲西面而事秦,则诸侯莫不西面而朝于章台之下矣。

“秦之所害莫如楚,楚彊则秦弱,秦彊则楚弱,其势不两立。故为大王计,莫如从亲以孤秦。大王不从,秦必起两军,一军出武关,一军下黔中,则鄢郢动矣。

“臣闻治之其未乱也,为之其未有也。患至而后忧之,则无及已。故原大王蚤孰计之。

“大王诚能听臣,臣请令山东之国奉四时之献,以承大王之明诏,委社稷,奉宗庙,练士厉兵,在大王之所用之。大王诚能用臣之愚计,则韩、魏、齐、燕、赵、卫之妙音美人必充后宫,燕、代橐驼良马必实外厩。故从合则楚王,衡成则秦帝。今释霸王之业,而有事人之名,臣窃为大王不取也。

“夫秦,虎狼之国也,有吞天下之心。秦,天下之仇雠也。衡人皆欲割诸侯之地以事秦,此所谓养仇而奉雠者也。夫为人臣,割其主之地以外交彊虎狼之秦,以侵天下,卒有秦患,不顾其祸。夫外挟彊秦之威以内劫其主,以求割地,大逆不忠,无过此者。故从亲则诸侯割地以事楚,衡合则楚割地以事秦,此两策者相去远矣,二者大王何居焉?故敝邑赵王使臣效愚计,奉明约,在大王诏之。”

楚王曰:“寡人之国西与秦接境,秦有举巴蜀并汉中之心。秦,虎狼之国,不可亲也。而韩、魏迫于秦患,不可与深谋,与深谋恐反人以入于秦,故谋未发而国已危矣。寡人自料以楚当秦,不见胜也;内与群臣谋,不足恃也。寡人卧不安席,食不甘味,心摇摇然如县旌而无所终薄。今主君欲一天下,收诸侯,存危国,寡人谨奉社稷以从。”

于是六国从合而并力焉。苏秦为从约长,并相六国。

北报赵王,乃行过雒阳,车骑辎重,诸侯各发使送之甚众,疑于王者。周显王闻之恐惧,除道,使人郊劳。苏秦之昆弟妻嫂侧目不敢仰视,俯伏侍取食。苏秦笑谓其嫂曰:“何前倨而后恭也?”嫂委蒲服,以面掩地而谢曰:“见季子位高金多也。”苏秦喟然叹曰:“此一人之身,富贵则亲戚畏惧之,贫贱则轻易之,况众人乎!且使我有雒阳负郭田二顷,吾岂能佩六国相印乎!”于是散千金以赐宗族朋友。初,苏秦之燕,贷人百钱为资,乃得富贵,以百金偿之。遍报诸所尝见德者。其从者有一人独未得报,乃前自言。苏秦曰:“我非忘子。子之与我至燕,再三欲去我易水之上,方是时,我困,故望子深,是以

苏秦既约六国从亲,归赵,赵肃侯封为武安君,乃投从约书于秦。秦兵不敢闚函谷关十五年。

其后秦使犀首欺齐、魏,与共伐赵,欲败从约。齐、魏伐赵,赵王让苏秦。苏秦恐,请使燕,必报齐。苏秦去赵而从约皆解。

秦惠王以其女为燕太子妇。是岁,文侯卒,太子立,是为燕易王。易王初立,齐宣王因燕丧伐燕,取十城。易王谓苏秦曰:“往日先生至燕,而先王资先生见赵,遂约六国从。今齐先伐赵,次至燕,以先生之故为天下笑,先生能为燕得侵地乎?”苏秦大惭,曰:“请为王取之。”

苏秦见齐王,再拜,俯而庆,仰而吊。齐王曰:“是何庆吊相随之速也?”苏秦曰:“臣闻饥人所以饥而不食乌喙者,为其愈充腹而与饿死同患也。今燕虽弱小,即秦王之少婿也。大王利其十城而长与彊秦为仇。今使弱燕为雁行而彊秦敝其后,以招天下之精兵,是食乌喙之类也。”齐王愀然变色曰:“然则柰何?”苏秦曰:“臣闻古之善制事者,转祸为福,因败为功。大王诚能听臣计,即归燕之十城。燕无故而得十城,必喜;秦王知以己之故而归燕之十城,亦必喜。此所谓弃仇雠而得石交者也。夫燕、秦俱事齐,则大王号令天下,莫敢不听。是王以虚辞附秦,以十城取天下。此霸王之业也。”王曰:“善。”于是乃归燕之十城。

人有毁苏秦者曰:“左右卖国反覆之臣也,将作乱。”苏秦恐得罪归,而燕王不复官也。苏秦见燕王曰:“臣,东周之鄙人也,无有分寸之功,而王亲拜之于庙而礼之于廷。今臣为王卻齐之兵而得十城,宜以益亲。今来而王不官臣者,人必有以不信伤臣于王者。臣之不信,王之福也。臣闻忠信者,所以自为也;进取者,所以为人也。且臣之说齐王,曾非欺之也。臣弃老母于东周,固去自为而行进取也。今有孝如曾参,廉如伯夷,信如尾生。得此三人者以事大王,何若?”王曰:“足矣。”苏秦曰:“孝如曾参,义不离其亲一宿于外,王又安能使之步行千里而事弱燕之危王哉?廉如伯夷,义不为孤竹君之嗣,不肯为武王臣,不受封侯而饿死首阳山下。有廉如此,王又安能使之步行千里而行进取于齐哉?信如尾生,与女子期于梁下,女子不来,水至不去,抱柱而死。有信如此,王又安能使之步行千里卻齐之彊兵哉?臣所谓以忠信得罪于上者也。”燕王曰:“若不忠信耳,岂有以忠信而得罪者乎?”苏秦曰:“不然。臣闻客有远为吏而其妻私于人者,其夫将来,其私者忧之,妻曰‘勿忧,吾已作药酒待之矣’。居三日,其夫果至,妻使妾举药酒进之。妾欲言酒之有药,则恐其逐主母也,欲勿言乎,则恐其杀主父也。于是乎详僵而弃酒。主父大怒,笞之五十。故妾一僵而覆酒,上存主父,下存主母,然而不免于笞,恶在乎忠信之无罪也?夫臣之过,不幸而类是乎!”燕王曰:“先生复就故官。”益厚遇之。

易王母,文侯夫人也,与苏秦私通。燕王知之,而事之加厚。苏秦恐诛,乃说燕王曰:“臣居燕不能使燕重,而在齐则燕必重。”燕王曰:“唯先生之所为。”于是苏秦详为得罪于燕而亡走齐,齐宣王以为客卿。

齐宣王卒,湣王即位,说湣王厚葬以明孝,高宫室大苑囿以明得意,欲破敝齐而为燕。燕易王卒,燕哙立为王。其后齐大夫多与苏秦争宠者,而使人刺苏秦,不死,殊而走。齐王使人求贼,不得。苏秦且死,乃谓齐王曰:“臣即死,车裂臣以徇于市,曰‘苏秦为燕作乱于齐’,如此则臣之贼必得矣。”于是如其言,而杀苏秦者果自出,齐王因而诛之。燕闻之曰:“甚矣,齐之为苏生报仇也!”

苏秦既死,其事大泄。齐后闻之,乃恨怒燕。燕甚恐。苏秦之弟曰代,代弟苏厉,见兄遂,亦皆学。及苏秦死,代乃求见燕王,欲袭故事。曰:“臣,东周之鄙人也。窃闻大王义甚高,鄙人不敏,释鉏耨而干大王。至于邯郸,所见者绌于所闻于东周,臣窃负其志。及至燕廷,观王之群臣下吏,王,天下之明王也。”燕王曰:“子所谓明王者何如也?”对曰:“臣闻明王务闻其过,不欲闻其善,臣请谒王之过。夫齐、赵者,燕之仇雠也;楚、魏者,燕之援国也。今王奉仇雠以伐援国,非所以利燕也。王自虑之,此则计过,无以闻者,非忠臣也。”王曰:“夫齐者固寡人之雠,所欲伐也,直患国敝力不足也。子能以燕伐齐,则寡人举国委子。”对曰:“凡天下战国七,燕处弱焉。独战则不能,有所附则无不重。南附楚,楚重;西附秦,秦重;中附韩、魏,韩、魏重。且苟所附之国重,此必使王重矣。今夫齐,长主而自用也。南攻楚五年,畜聚竭;西困秦三年,士卒罢敝;北与燕人战,覆三军,得二将。然而以其馀兵南面举五千乘之大宋,而包十二诸侯。此其君欲得,其民力竭,恶足取乎!且臣闻之,数战则民劳,久师则兵敝矣。”燕王曰:“吾闻齐有清济、浊河可以为固,长城、钜防足以为塞,诚有之乎?”对曰:“天时不与,虽有清济、浊河,恶足以为固!民力罢敝,虽有长城、钜防,恶足以为塞!且异日济西不师,所以备赵也;河北不师,所以备燕也。今济西河北尽已役矣,封内敝矣。夫骄君必好利,而亡国之臣必贪于财。王诚能无羞从子母弟以为质,宝珠玉帛以事左右,彼将有德燕而轻亡宋,则齐可亡已。”燕王曰:“吾终以子受命于天矣。”燕乃使一子质于齐。而苏厉因燕质子而求见齐王。齐王怨苏秦,欲囚苏厉。燕质子为谢,已遂委质为齐臣。

燕相子之与苏代婚,而欲得燕权,乃使苏代侍质子于齐。齐使代报燕,燕王哙问曰:“齐王其霸乎?”曰:“不能。”曰:“何也?”曰:“不信其臣。”于是燕王专任子之,已而让位,燕大乱。齐伐燕,杀王哙、子之。燕立昭王,而苏代、苏厉遂不敢入燕,皆终归齐,齐善待之。

苏代过魏,魏为燕执代。齐使人谓魏王曰:“齐请以宋地封泾阳君,秦必不受。秦非不利有齐而得宋地也,不信齐王与苏子也。今齐魏不和如此其甚,则齐不欺秦。秦信齐,齐秦合,泾阳君有宋地,非魏之利也。故王不如东苏子,秦必疑齐而不信苏子矣。齐秦不合,天下无变,伐齐之形成矣。”于是出苏代。代之宋,宋善待之。

齐伐宋,宋急,苏代乃遗燕昭王书曰:

夫列在万乘而寄质于齐,名卑而权轻;奉万乘助齐伐宋,民劳而实费;夫破宋,残楚淮北,肥大齐,雠彊而国害:此三者皆国之大败也。然且王行之者,将以取信于齐也。齐加不信于王,而忌燕愈甚,是王之计过矣。夫以宋加之淮北,强万乘之国也,而齐并之,是益一齐也。北夷方七百里,加之以鲁、卫,彊万乘之国也,而齐并之,是益二齐也。夫一齐之彊,燕犹狼顾而不能支,今以三齐临燕,其祸必大矣。

虽然,智者举事,因祸为福,转败为功。齐紫,败素也,而贾十倍;越王句践栖于会稽,复残彊吴而霸天下:此皆因祸为福,转败为功者也。

今王若欲因祸为福,转败为功,则莫若挑霸齐而尊之,使使盟于周室,焚秦符,曰“其大上计,破秦;其次,必长宾之”。秦挟宾以待破,秦王必患之。秦五世伐诸侯,今为齐下,秦王之志苟得穷齐,不惮以国为功。然则王何不使辩士以此言说秦王曰:“燕、赵破宋肥齐,尊之为之下者,燕、赵非利之也。燕、赵不利而势为之者,以不信秦王也。然则王何不使可信者接收燕、赵,令泾阳君、高陵君先于燕、赵?秦有变,因以为质,则燕、赵信秦。秦为西帝,燕为北帝,赵为中帝,立三帝以令于天下。韩、魏不听则秦伐之,齐不听则燕、赵伐之,天下孰敢不听?天下服听,因驱韩、魏以伐齐,曰‘必反宋地,归楚淮北’。反宋地,归楚淮北,燕、赵之所利也;并立三帝,燕、赵之所原也。夫实得所利,尊得所原,燕、赵弃齐如脱鵕矣。今不收燕、赵,齐霸必成。诸侯赞齐而王不从,是国伐也;诸侯赞齐而王从之,是名卑也。今收燕、赵,国安而名尊;不收燕、赵,国危而名卑。夫去尊安而取危卑,智者不为也。”秦王闻若说,必若刺心然。则王何不使辩士以此若言说秦?秦必取,齐必伐矣。

夫取秦,厚交也;伐齐,正利也。尊厚交,务正利,圣王之事也。

燕昭王善其书,曰:“先人尝有德苏氏,子之之乱而苏氏去燕。燕欲报仇于齐,非苏氏莫可。”乃召苏代,复善待之,与谋伐齐。竟破齐,湣王出走。

久之,秦召燕王,燕王欲往,苏代约燕王曰:“楚得枳而国亡,齐得宋而国亡,齐、楚不得以有枳、宋而事秦者,何也?则有功者,秦之深雠也。秦取天下,非行义也,暴也。秦之行暴,正告天下。

“告楚曰:‘蜀地之甲,乘船浮于汶,乘夏水而下江,五日而至郢。汉中之甲,乘船出于巴,乘夏水而下汉,四日而至五渚。寡人积甲宛东下随,智者不及谋,勇土不及怒,寡人如射隼矣。王乃欲待天下之攻函谷,不亦远乎!’楚王为是故,十七年事秦。

“秦正告韩曰:‘我起乎少曲,一日而断大行。我起乎宜阳而触平阳,二日而莫不尽繇。我离两周而触郑,五日而国举。’韩氏以为然,故事秦。

“秦正告魏曰:‘我举安邑,塞女戟,韩氏太原卷。我下轵,道南阳,封冀,包两周。乘夏水,浮轻舟,彊弩在前,錟戈在后,决荥口,魏无大梁;决白马之口,魏无外黄、济阳;决宿胥之口,魏无虚、顿丘。陆攻则击河内,水攻则灭大梁。’魏氏以为然,故事秦。

“秦欲攻安邑,恐齐救之,则以宋委于齐。曰:‘宋王无道,为木人以寡人,射其面。寡人地绝兵远,不能攻也。王苟能破宋有之,寡人如自得之。’已得安邑,塞女戟,因以破宋为齐罪。

“秦欲攻韩,恐天下救之,则以齐委于天下。曰:‘齐王四与寡人约,四欺寡人,必率天下以攻寡人者三。有齐无秦,有秦无齐,必伐之,必亡之。’已得宜阳、少曲,致蔺、石,因以破齐为天下罪。

“秦欲攻魏重楚,则以南阳委于楚。曰:‘寡人固与韩且绝矣。残均陵,塞鄳戹,苟利于楚,寡人如自有之。’魏弃与国而合于秦,因以塞鄳戹为楚罪。

“兵困于林中,重燕、赵,以胶东委于燕,以济西委于赵。已得讲于魏,至公子延,因犀首属行而攻赵。

“兵伤于谯石,而遇败于阳马,而重魏,则以叶、蔡委于魏。已得讲于赵,则劫魏,不为割。困则使太后弟穰侯为和,嬴则兼欺舅与母。

“適燕者曰‘以胶东’,適赵者曰‘以济西’,適魏者曰‘以叶、蔡’,適楚者曰‘以塞鄳戹’,適齐者曰‘以宋’,此必令言如循环,用兵如刺蜚,母不能制,舅不能约。“龙贾之战,岸门之战,封陵之战,高商之战,赵庄之战,秦之所杀三晋之民数百万,今其生者皆死秦之孤也。西河之外,上雒之地,三川晋国之祸,三晋之半,秦祸如此其大也。而燕、赵之秦者,皆以争事秦说其主,此臣之所大患也。”

燕昭王不行。苏代复重于燕。

燕使约诸侯从亲如苏秦时,或从或不,而天下由此宗苏氏之从约。代、厉皆以寿死,名显诸侯。

太史公曰:苏秦兄弟三人,皆游说诸侯以显名,其术长于权变。而苏秦被反间以死,天下共笑之,讳学其术。然世言苏秦多异,异时事有类之者皆附之苏秦。夫苏秦起闾阎,连六国从亲,此其智有过人者。吾故列其行事,次其时序,毋令独蒙恶声焉。

季子周人,师事鬼谷。揣摩既就,阴符伏读。合从离衡,佩印者六。天王除道,家人扶服。贤哉代、厉,继荣党族。
\end{yuanwen}

\part{卷七十}

\chapter{张仪列传第十}

\begin{yuanwen}
张仪者,魏人也。始尝与苏秦俱事鬼谷先生,学术,苏秦自以不及张仪。
\end{yuanwen}

张仪,是魏国人。他当初曾与苏秦一同拜在鬼谷先生门下,学习游说之术,苏秦认为自己的才学比不上张仪。

\begin{yuanwen}
张仪已学游说诸侯。尝从楚相饮,已而楚相亡璧,门下意\footnote{猜疑,怀疑。}张仪,曰:“仪贫无行,必此盗相君之璧。”

共执张仪,掠笞数百,不服,醳\footnote{古“释”字。}之。其妻曰:“嘻!子毋读书游说,安得此辱乎?”

张仪谓其妻曰:“视吾舌尚在不?”

其妻笑曰:“舌在也。”

仪曰:“足矣。”
\end{yuanwen}

张仪完成学业后,就去游说诸侯。有一次,他陪同楚国的丞相喝酒,不久丞相遗失了一块玉璧,门客都怀疑是张仪偷了玉璧,说:“张仪贫穷,品行不端,一定是他偷了丞相的玉璧!”

于是大家一起捉住了张仪,拷打了他几百竹板,张仪始终没有承认,大家只好释放了他,他的妻子既悲又恨说:“唉,你要是不曾去读书游说,又怎么会遭受这样的屈辱呢?”

张仪对他的妻子说:“你看我的舌头还在不在?”

他的妻子笑着回答:“舌头还在。”

张仪说:“这样就足够了。”

\begin{yuanwen}
苏秦已说赵王而得相约从\footnote{同“纵”。}亲,然恐秦之攻诸侯,败约后负,念莫可使用于秦者,乃使人微感张仪曰:“子始与苏秦善,今秦已当路,子何不往游,以求通子之原?”

张仪于是之赵,上谒\footnote{递上名帖请求进见。}求见苏秦。苏秦乃诫门下人不为通,又使不得去者数日。已而见之,坐之堂下,赐仆妾之食。因而数让之曰:“以子之材能,乃自令困辱至此。吾宁不能言而富贵子,子不足收也。”

谢去之。张仪之来也,自以为故人,求益,反见辱,怒,念诸侯莫可事,独秦能苦赵,乃遂入秦。
\end{yuanwen}

苏秦已经说服了赵王同意与除秦国外的另外六国合纵,缔结合纵联盟,但是苏秦又担心秦国乘机攻打各诸侯国,从而招致盟约在没有缔结之前就遭到了破坏。他正忧虑没有找到一个能派往秦国的人,于是他派人暗中引导张仪说:“你当初与苏秦的交情很好,现在他已经当权,你为什么不去结交他,以实现功成名就的志向呢?”

张仪于是到了赵国,呈上名帖,请求拜见苏秦。苏秦却早已告诫守门的人不替张仪禀报,又设法让他好几天不能离开。不久,苏秦接见他,把他安排在堂下坐着,赏给他的饭菜是丫环仆人们吃的。并多次责备他说:“像你这样有才能的人,却让自己穷困潦倒到了这种地步。难道我没有能力推荐你并使你富贵吗?只是因为你不值得录用罢了。”

苏秦说完就把张仪打发走了。张仪这次本是来投奔苏秦的,自认为是老朋友了,能够得到一些好处,不料反而被羞辱,很气愤,又考虑到各国诸侯中没有可以事奉的,唯独秦国才能威胁赵国,于是就到秦国去了。

\begin{yuanwen}
苏秦已而告其舍人曰:“张仪,天下贤士,吾殆弗如也。今吾幸先用,而能用秦柄者,独张仪可耳。然贫,无因以进。吾恐其乐小利而不遂,故召辱之,以激其意。子为我阴奉之。”

乃言赵王,发金币车马,使人微随张仪,与同宿舍,稍稍近就之,奉以车马金钱,所欲用,为取给,而弗告。张仪遂得以见秦惠王。惠王以为客卿,与谋伐诸侯。
\end{yuanwen}

不久之后,苏秦告诉自己的门客说:“张仪是天下有才能的人,我大概比不上他啊。现在我侥幸比他先受到重用,但是说到能够掌握秦国大权的人,只有张仪才行。然而他目前很贫穷,没有进用的机会。我担心他为了满足小的利益,不再进取,不能成就大的功业,所以把他召来当面羞辱他,来激发他的斗志。请你替我暗中帮助他吧。”

苏秦将自己的打算奏明赵王以后,拿出金钱和车马,派人暗中跟随张仪,与张仪投宿在同一间客栈,逐渐接近他,还拿出车马钱财供他使用,凡是张仪需要的,都能供给他,却并没有告诉他实情。张仪终于得以拜见秦惠王。秦惠王任用张仪做客卿,与他商谋攻打各国诸侯。

\begin{yuanwen}

\end{yuanwen}\begin{yuanwen}

\end{yuanwen}\begin{yuanwen}

\end{yuanwen}\begin{yuanwen}

\end{yuanwen}\begin{yuanwen}

\end{yuanwen}\begin{yuanwen}

\end{yuanwen}\begin{yuanwen}

\end{yuanwen}\begin{yuanwen}

\end{yuanwen}\begin{yuanwen}

\end{yuanwen}\begin{yuanwen}

\end{yuanwen}\begin{yuanwen}

\end{yuanwen}\begin{yuanwen}

\end{yuanwen}\begin{yuanwen}

\end{yuanwen}\begin{yuanwen}

\end{yuanwen}\begin{yuanwen}

\end{yuanwen}\begin{yuanwen}

\end{yuanwen}\begin{yuanwen}

\end{yuanwen}\begin{yuanwen}

\end{yuanwen}\begin{yuanwen}

\end{yuanwen}\begin{yuanwen}

\end{yuanwen}\begin{yuanwen}

\end{yuanwen}\begin{yuanwen}

\end{yuanwen}\begin{yuanwen}

\end{yuanwen}\begin{yuanwen}

\end{yuanwen}\begin{yuanwen}

\end{yuanwen}\begin{yuanwen}

\end{yuanwen}\begin{yuanwen}

\end{yuanwen}\begin{yuanwen}

\end{yuanwen}
\begin{yuanwen}




苏秦之舍人乃辞去。张仪曰:“赖子得显,方且报德,何故去也?”舍人曰:“臣非知君,知君乃苏君。苏君忧秦伐赵败从约,以为非君莫能得秦柄,故感怒君,使臣阴奉给君资,尽苏君之计谋。今君已用,请归报。”张仪曰:“嗟乎,此在吾术中而不悟,吾不及苏君明矣!吾又新用,安能谋赵乎?为吾谢苏君,苏君之时,仪何敢言。且苏君在,仪宁渠能乎!”张仪既相秦,为文檄告楚相曰:“始吾从若饮,我不盗而璧,若笞我。若善守汝国,我顾且盗而城!”

苴蜀相攻击,各来告急于秦。秦惠王欲发兵以伐蜀,以为道险狭难至,而韩又来侵秦,秦惠王欲先伐韩,后伐蜀,恐不利,欲先伐蜀,恐韩袭秦之敝。犹豫未能决。司马错与张仪争论于惠王之前,司马错欲伐蜀,张仪曰:“不如伐韩。”王曰:“请闻其说。”

仪曰:“亲魏善楚,下兵三川,塞什谷之口,当屯留之道,魏绝南阳,楚临南郑,秦攻新城、宜阳,以临二周之郊,诛周王之罪,侵楚、魏之地。周自知不能救,九鼎宝器必出。据九鼎,案图籍,挟天子以令于天下,天下莫敢不听,此王业也。今夫蜀,西僻之国而戎翟之伦也,敝兵劳众不足以成名,得其地不足以为利。臣闻争名者于朝,争利者于市。今三川、周室,天下之朝市也,而王不争焉,顾争于戎翟,去王业远矣。”

司马错曰:“不然。臣闻之,欲富国者务广其地,欲彊兵者务富其民,欲王者务博其德,三资者备而王随之矣。今王地小民贫,故臣原先从事于易。夫蜀,西僻之国也,而戎翟之长也,有桀纣之乱。以秦攻之,譬如使豺狼逐群羊。得其地足以广国,取其财足以富民缮兵,不伤众而彼已服焉。拔一国而天下不以为暴,利尽西海而天下不以为贪,是我一举而名实附也,而又有禁暴止乱之名。今攻韩,劫天子,恶名也,而未必利也,又有不义之名,而攻天下所不欲,危矣。臣请谒其故:周,天下之宗室也;齐,韩之与国也。周自知失九鼎,韩自知亡三川,将二国并力合谋,以因乎齐、赵而求解乎楚、魏,以鼎与楚,以地与魏,王弗能止也。此臣之所谓危也。不如伐蜀完。”

惠王曰:“善,寡人请听子。”卒起兵伐蜀,十月,取之,遂定蜀,贬蜀王更号为侯,而使陈庄相蜀。蜀既属秦,秦以益彊,富厚,轻诸侯。

秦惠王十年,使公子华与张仪围蒲阳,降之。仪因言秦复与魏,而使公子繇质于魏。仪因说魏王曰:“秦王之遇魏甚厚,魏不可以无礼。”魏因入上郡、少梁,谢秦惠王。惠王乃以张仪为相,更名少梁曰夏阳。

仪相秦四岁,立惠王为王。居一岁,为秦将,取陕。筑上郡塞。

其后二年,使与齐、楚之相会齧桑。东还而免相,相魏以为秦,欲令魏先事秦而诸侯效之。魏王不肯听仪。秦王怒,伐取魏之曲沃、平周,复阴厚张仪益甚。张仪惭,无以归报。留魏四岁而魏襄王卒,哀王立。张仪复说哀王,哀王不听。于是张仪阴令秦伐魏。魏与秦战,败。

明年,齐又来败魏于观津。秦复欲攻魏,先败韩申差军,斩首八万,诸侯震恐。而张仪复说魏王曰:“魏地方不至千里,卒不过三十万。地四平,诸侯四通辐凑,无名山大川之限。从郑至梁二百馀里,车驰人走,不待力而至。梁南与楚境,西与韩境,北与赵境,东与齐境,卒戍四方,守亭鄣者不下十万。梁之地势,固战场也。梁南与楚而不与齐,则齐攻其东;东与齐而不与赵,则赵攻其北;不合于韩,则韩攻其西;不亲于楚,则楚攻其南:此所谓四分五裂之道也。

“且夫诸侯之为从者,将以安社稷尊主彊兵显名也。今从者一天下,约为昆弟,刑白马以盟洹水之上,以相坚也。而亲昆弟同父母,尚有争钱财,而欲恃诈伪反覆苏秦之馀谋,其不可成亦明矣。

“大王不事秦,秦下兵攻河外,据卷、衍、、酸枣,劫卫取阳晋,则赵不南,赵不南而梁不北,梁不北则从道绝,从道绝则大王之国欲毋危不可得也。秦折韩而攻梁,韩怯于秦,秦韩为一,梁之亡可立而须也。此臣之所为大王患也。

“为大王计,莫如事秦。事秦则楚、韩必不敢动;无楚、韩之患,则大王高枕而卧,国必无忧矣。

“且夫秦之所欲弱者莫如楚,而能弱楚者莫如梁。楚虽有富大之名而实空虚;其卒虽多,然而轻走易北,不能坚战。悉梁之兵南面而伐楚,胜之必矣。割楚而益梁,亏楚而適秦,嫁祸安国,此善事也。大王不听臣,秦下甲士而东伐,虽欲事秦,不可得矣。

“且夫从人多奋辞而少可信,说一诸侯而成封侯,是故天下之游谈士莫不日夜搤腕瞋目切齿以言从之便,以说人主。人主贤其辩而牵其说,岂得无眩哉。

“臣闻之,积羽沈舟,群轻折轴,众口铄金,积毁销骨,故原大王审定计议,且赐骸骨辟魏。”

哀王于是乃倍从约而因仪请成于秦。张仪归,复相秦。三岁而魏复背秦为从。秦攻魏,取曲沃。明年,魏复事秦。

秦欲伐齐,齐楚从亲,于是张仪往相楚。楚怀王闻张仪来,虚上舍而自馆之。曰:“此僻陋之国,子何以教之?”仪说楚王曰:“大王诚能听臣,闭关绝约于齐,臣请献商于之地六百里,使秦女得为大王箕帚之妾,秦楚娶妇嫁女,长为兄弟之国。此北弱齐而西益秦也,计无便此者。”楚王大说而许之。群臣皆贺,陈轸独吊之。楚王怒曰:“寡人不兴师发兵得六百里地,群臣皆贺,子独吊,何也?”陈轸对曰:“不然,以臣观之,商于之地不可得而齐秦合,齐秦合则患必至矣。”楚王曰:“有说乎?”陈轸对曰:“夫秦之所以重楚者,以其有齐也。今闭关绝约于齐,则楚孤。秦奚贪夫孤国,而与之商于之地六百里?张仪至秦,必负王,是北绝齐交,西生患于秦也,而两国之兵必俱至。善为王计者,不若阴合而阳绝于齐,使人随张仪。苟与吾地,绝齐未晚也;不与吾地,阴合谋计也。”楚王曰:“原陈子闭口毋复言,以待寡人得地。”乃以相印授张仪,厚赂之。于是遂闭关绝约于齐,使一将军随张仪。

张仪至秦,详失绥堕车,不朝三月。楚王闻之,曰:“仪以寡人绝齐未甚邪?”乃使勇士至宋,借宋之符,北骂齐王。齐王大怒,折节而下秦。秦齐之交合,张仪乃朝,谓楚使者曰:“臣有奉邑六里,原以献大王左右。”楚使者曰:“臣受令于王,以商于之地六百里,不闻六里。”还报楚王,楚王大怒,发兵而攻秦。陈轸曰:“轸可发口言乎?攻之不如割地反以赂秦,与之并兵而攻齐,是我出地于秦,取偿于齐也,王国尚可存。”楚王不听,卒发兵而使将军屈匄击秦。秦齐共攻楚,斩首八万,杀屈匄,遂取丹阳、汉中之地。楚又复益发兵而袭秦,至蓝田,大战,楚大败,于是楚割两城以与秦平。

秦要楚欲得黔中地,欲以武关外易之。楚王曰:“不原易地,原得张仪而献黔中地。”秦王欲遣之,口弗忍言。张仪乃请行。惠王曰:“彼楚王怒子之负以商于之地,是且甘心于子。”张仪曰:“秦彊楚弱,臣善靳尚,尚得事楚夫人郑袖,袖所言皆从。且臣奉王之节使楚,楚何敢加诛。假令诛臣而为秦得黔中之地,臣之上原。”遂使楚。楚怀王至则囚张仪,将杀之。靳尚谓郑袖曰:“子亦知子之贱于王乎?”郑袖曰:“何也?”靳尚曰:“秦王甚爱张仪而不欲出之,今将以上庸之地六县赂楚,美人聘楚,以宫中善歌讴者为媵。楚王重地尊秦,秦女必贵而夫人斥矣。不若为言而出之。”于是郑袖日夜言怀王曰:“人臣各为其主用。今地未入秦,秦使张仪来,至重王。王未有礼而杀张仪,秦必大怒攻楚。妾请子母俱迁江南,毋为秦所鱼肉也。”怀王后悔,赦张仪,厚礼之如故。

张仪既出,未去,闻苏秦死,乃说楚王曰:“秦地半天下,兵敌四国,被险带河,四塞以为固。虎贲之士百馀万,车千乘,骑万匹,积粟如丘山。法令既明,士卒安难乐死,主明以严,将智以武,虽无出甲,席卷常山之险,必折天下之脊,天下有后服者先亡。且夫为从者,无以异于驱群羊而攻猛虎,虎之与羊不格明矣。今王不与猛虎而与群羊,臣窃以为大王之计过也。

“凡天下彊国,非秦而楚,非楚而秦,两国交争,其势不两立。大王不与秦,秦下甲据宜阳,韩之上地不通。下河东,取成皋,韩必入臣,梁则从风而动。秦攻楚之西,韩、梁攻其北,社稷安得毋危?

“且夫从者聚群弱而攻至彊,不料敌而轻战,国贫而数举兵,危亡之术也。臣闻之,兵不如者勿与挑战,粟不如者勿与持久。夫从人饰辩虚辞,高主之节,言其利不言其害,卒有秦祸,无及为已。是故原大王之孰计之。

“秦西有巴蜀,大船积粟,起于汶山,浮江已下,至楚三千馀里。舫船载卒,一舫载五十人与三月之食,下水而浮,一日行三百馀里,里数虽多,然而不费牛马之力,不至十日而距扞关。扞关惊,则从境以东尽城守矣,黔中、巫郡非王之有。秦举甲出武关,南面而伐,则北地绝。秦兵之攻楚也,危难在三月之内,而楚待诸侯之救,在半岁之外,此其势不相及也。夫弱国之救,忘彊秦之祸,此臣所以为大王患也。

“大王尝与吴人战,五战而三胜,阵卒尽矣;偏守新城,存民苦矣。臣闻功大者易危,而民敝者怨上。夫守易危之功而逆彊秦之心,臣窃为大王危之。

“且夫秦之所以不出兵函谷十五年以攻齐、赵者,阴谋有合天下之心。楚尝与秦构难,战于汉中,楚人不胜,列侯执珪死者七十馀人,遂亡汉中。楚王大怒,兴兵袭秦,战于蓝田。此所谓两虎相搏者也。夫秦楚相敝而韩魏以全制其后,计无危于此者矣。原大王孰计之。

“秦下甲攻卫阳晋,必大关天下之匈。大王悉起兵以攻宋,不至数月而宋可举,举宋而东指,则泗上十二诸侯尽王之有也。

“凡天下而以信约从亲相坚者苏秦,封武安君,相燕,即阴与燕王谋伐破齐而分其地;乃详有罪出走入齐,齐王因受而相之;居二年而觉,齐王大怒,车裂苏秦于市。夫以一诈伪之苏秦,而欲经营天下,混一诸侯,其不可成亦明矣。

“今秦与楚接境壤界,固形亲之国也。大王诚能听臣,臣请使秦太子入质于楚,楚太子入质于秦,请以秦女为大王箕帚之妾,效万室之都以为汤沐之邑,长为昆弟之国,终身无相攻伐。臣以为计无便于此者。”

于是楚王已得张仪而重出黔中地与秦,欲许之。屈原曰:“前大王见欺于张仪,张仪至,臣以为大王烹之;今纵弗忍杀之,又听其邪说,不可。”怀王曰:“许仪而得黔中,美利也。后而倍之,不可。”故卒许张仪,与秦亲。

张仪去楚,因遂之韩,说韩王曰:“韩地险恶山居,五穀所生,非菽而麦,民之食大抵菽藿羹。一岁不收,收不餍糟。地不过九百里,无二岁之食。料大王之卒,悉之不过三十万,而厮徒负养在其中矣。除守徼亭鄣塞,见卒不过二十万而已矣。秦带甲百馀万,车千乘,骑万匹,虎贲之士跿簉科头贯颐奋戟者,至不可胜计。秦马之良,戎兵之众,探前趹后蹄间三寻腾者,不可胜数。山东之士被甲蒙胄以会战,秦人捐甲徒裼以趋敌,左挈人头,右挟生虏。夫秦卒与山东之卒,犹孟贲之与怯夫;以重力相压,犹乌获之与婴兒。夫战孟贲、乌获之士以攻不服之弱国,无异垂千钧之重于鸟卵之上,必无幸矣。

“夫群臣诸侯不料地之寡,而听从人之甘言好辞,比周以相饰也,皆奋曰‘听吾计可以彊霸天下’。夫不顾社稷之长利而听须臾之说,诖误人主,无过此者。

“大王不事秦,秦下甲据宜阳,断韩之上地,东取成皋、荥阳,则鸿台之宫、桑林之苑非王之有也。夫塞成皋,绝上地,则王之国分矣。先事秦则安,不事秦则危。夫造祸而求其福报,计浅而怨深,逆秦而顺楚,虽欲毋亡,不可得也。

“故为大王计,莫如为秦。秦之所欲莫如弱楚,而能弱楚者如韩。非以韩能彊于楚也,其地势然也。今王西面而事秦以攻楚,秦王必喜。夫攻楚以利其地,转祸而说秦,计无便于此者。”

韩王听仪计。张仪归报,秦惠王封仪五邑,号曰武信君。使张仪东说齐湣王曰:“天下彊国无过齐者,大臣父兄殷众富乐。然而为大王计者,皆为一时之说,不顾百世之利。从人说大王者,必曰‘齐西有彊赵,南有韩与梁。齐,负海之国也,地广民众,兵彊士勇,虽有百秦,将无柰齐何’。大王贤其说而不计其实。夫从人朋党比周,莫不以从为可。臣闻之,齐与鲁三战而鲁三胜,国以危亡随其后,虽有战胜之名,而有亡国之实。是何也?齐大而鲁小也。今秦之与齐也,犹齐之与鲁也。秦赵战于河漳之上,再战而赵再胜秦;战于番吾之下,再战又胜秦。四战之后,赵之亡卒数十万,邯郸仅存,虽有战胜之名而国已破矣。是何也?秦彊而赵弱。

“今秦楚嫁女娶妇,为昆弟之国。韩献宜阳;梁效河外;赵入朝渑池,割河间以事秦。大王不事秦,秦驱韩梁攻齐之南地,悉赵兵渡清河,指博关,临菑、即墨非王之有也。国一日见攻,虽欲事秦,不可得也。是故原大王孰计之也。”

齐王曰:“齐僻陋,隐居东海之上,未尝闻社稷之长利也。”乃许张仪。

张仪去,西说赵王曰:“敝邑秦王使使臣效愚计于大王。大王收率天下以宾秦,秦兵不敢出函谷关十五年。大王之威行于山东,敝邑恐惧慑伏,缮甲厉兵,饰车骑,习驰射,力田积粟,守四封之内,愁居慑处,不敢动摇,唯大王有意督过之也。

“今以大王之力,举巴蜀,并汉中,包两周,迁九鼎,守白马之津。秦虽僻远,然而心忿含怒之日久矣。今秦有敝甲凋兵,军于渑池,原渡河逾漳,据番吾,会邯郸之下,原以甲子合战,以正殷纣之事,敬使使臣先闻左右。

“凡大王之所信为从者恃苏秦。苏秦荧惑诸侯,以是为非,以非为是,欲反齐国,而自令车裂于市。夫天下之不可一亦明矣。今楚与秦为昆弟之国,而韩梁称为东籓之臣,齐献鱼盐之地,此断赵之右臂也。夫断右臂而与人斗,失其党而孤居,求欲毋危,岂可得乎?

“今秦发三将军:其一军塞午道,告齐使兴师渡清河,军于邯郸之东;一军军成皋,驱韩梁军于河外;一军军于渑池。约四国为一以攻赵,赵,必四分其地。是故不敢匿意隐情,先以闻于左右。臣窃为大王计,莫如与秦王遇于渑池,面相见而口相结,请案兵无攻。原大王之定计。”

赵王曰:“先王之时,奉阳君专权擅势,蔽欺先王,独擅绾事,寡人居属师傅,不与国谋计。先王弃群臣,寡人年幼,奉祀之日新,心固窃疑焉,以为一从不事秦,非国之长利也。乃且原变心易虑,割地谢前过以事秦。方将约车趋行,適闻使者之明诏。”赵王许张仪,张仪乃去。

北之燕,说燕昭王曰:“大王之所亲莫如赵。昔赵襄子尝以其姊为代王妻,欲并代,约与代王遇于句注之塞。乃令工人作为金斗,长其尾,令可以击人。与代王饮,阴告厨人曰:‘即酒酣乐,进热啜,反斗以击之。’于是酒酣乐,进热啜,厨人进斟,因反斗以击代王,杀之,王脑涂地。其姊闻之,因摩笄以自刺,故至今有摩笄之山。代王之亡,天下莫不闻。

“夫赵王之很戾无亲,大王之所明见,且以赵王为可亲乎?赵兴兵攻燕,再围燕都而劫大王,大王割十城以谢。今赵王已入朝渑池,效河间以事秦。今大王不事秦,秦下甲云中、九原,驱赵而攻燕,则易水、长城非大王之有也。

“且今时赵之于秦犹郡县也,不敢妄举师以攻伐。今王事秦,秦王必喜,赵不敢妄动,是西有彊秦之援,而南无齐赵之患,是故原大王孰计之。”

燕王曰:“寡人蛮夷僻处,虽大男子裁如婴兒,言不足以采正计。今上客幸教之,请西面而事秦,献恆山之尾五城。”燕王听仪。仪归报,未至咸阳而秦惠王卒,武王立。武王自为太子时不说张仪,及即位,群臣多谗张仪曰:“无信,左右卖国以取容。秦必复用之,恐为天下笑。”诸侯闻张仪有卻武王,皆畔衡,复合从。

秦武王元年,群臣日夜恶张仪未已,而齐让又至。张仪惧诛,乃因谓秦武王曰:“仪有愚计,原效之。”王曰:“柰何?”对曰:“为秦社稷计者,东方有大变,然后王可以多割得地也。今闻齐王甚憎仪,仪之所在,必兴师伐之。故仪原乞其不肖之身之梁,齐必兴师而伐梁。梁齐之兵连于城下而不能相去,王以其间伐韩,入三川,出兵函谷而毋伐,以临周,祭器必出。挟天子,按图籍,此王业也。”秦王以为然,乃具革车三十乘,入仪之梁。齐果兴师伐之。梁哀王恐。张仪曰:“王勿患也,请令罢齐兵。”乃使其舍人冯喜之楚,借使之齐,谓齐王曰:“王甚憎张仪;虽然,亦厚矣王之讬仪于秦也!”齐王曰:“寡人憎仪,仪之所在,必兴师伐之,何以讬仪?”对曰:“是乃王之讬仪也。夫仪之出也,固与秦王约曰:‘为王计者,东方有大变,然后王可以多割得地。今齐王甚憎仪,仪之所在,必兴师伐之。故仪原乞其不肖之身之梁,齐必兴师伐之。齐梁之兵连于城下而不能相去,王以其间伐韩,入三川,出兵函谷而无伐,以临周,祭器必出。挟天子,案图籍,此王业也。’秦王以为然,故具革车三十乘而入之梁也。今仪入梁,王果伐之,是王内罢国而外伐与国,广邻敌以内自临,而信仪于秦王也。此臣之所谓‘讬仪’也。”齐王曰:“善。”乃使解兵。

张仪相魏一岁,卒于魏也。

陈轸者,游说之士。与张仪俱事秦惠王,皆贵重,争宠。张仪恶陈轸于秦王曰:“轸重币轻使秦楚之间,将为国交也。今楚不加善于秦而善轸者,轸自为厚而为王薄也。且轸欲去秦而之楚,王胡不听乎?”王谓陈轸曰:“吾闻子欲去秦之楚,有之乎?”轸曰:“然。”王曰:“仪之言果信矣。”轸曰:“非独仪知之也,行道之士尽知之矣。昔子胥忠于其君而天下争以为臣,曾参孝于其亲而天下原以为子。故卖仆妾不出闾巷而售者,良仆妾也;出妇嫁于乡曲者,良妇也。今轸不忠其君,楚亦何以轸为忠乎?忠且见弃,轸不之楚何归乎?”王以其言为然,遂善待之。

居秦期年,秦惠王终相张仪,而陈轸奔楚。楚未之重也,而使陈轸使于秦。过梁,欲见犀首。犀首谢弗见。轸曰:“吾为事来,公不见轸,轸将行,不得待异日。”犀首见之。陈轸曰:“公何好饮也?”犀首曰:“无事也。”曰:“吾请令公厌事可乎?”曰:“柰何?”曰:“田需约诸侯从亲,楚王疑之,未信也。公谓于王曰:‘臣与燕、赵之王有故,数使人来,曰:“无事何不相见”,原谒行于王。’王虽许公,公请毋多车,以车三十乘,可陈之于庭,明言之燕、赵。”燕、赵客闻之,驰车告其王,使人迎犀首。楚王闻之大怒,曰:“田需与寡人约,而犀首之燕、赵,是欺我也。”怒而不听其事。齐闻犀首之北,使人以事委焉。犀首遂行,三国相事皆断于犀首。轸遂至秦。

韩魏相攻,期年不解。秦惠王欲救之,问于左右。左右或曰救之便,或曰勿救便,惠王未能为之决。陈轸適至秦,惠王曰:“子去寡人之楚,亦思寡人不?”陈轸对曰:“王闻夫越人庄舄乎?”王曰:“不闻。”曰:“越人庄舄仕楚执珪,有顷而病。楚王曰:‘舄故越之鄙细人也,今仕楚执珪,贵富矣,亦思越不?’中谢对曰:‘凡人之思故,在其病也。彼思越则越声,不思越则楚声。’使人往听之,犹尚越声也。今臣虽弃逐之楚,岂能无秦声哉!”惠王曰:“善。今韩魏相攻,期年不解,或谓寡人救之便,或曰勿救便,寡人不能决,原子为子主计之馀,为寡人计之。”陈轸对曰:“亦尝有以夫卞庄子刺虎闻于王者乎?庄子欲刺虎,馆竖子止之,曰:‘两虎方且食牛,食甘必争,争则必斗,斗则大者伤,小者死,从伤而刺之,一举必有双虎之名。’卞庄子以为然,立须之。有顷,两虎果斗,大者伤,小者死。庄子从伤者而刺之,一举果有双虎之功。今韩魏相攻,期年不解,是必大国伤,小国亡,从伤而伐之,一举必有两实。此犹庄子刺虎之类也。臣主与王何异也。”惠王曰:“善。”卒弗救。大国果伤,小国亡,秦兴兵而伐,大剋之。此陈轸之计也。

犀首者,魏之阴晋人也,名衍,姓公孙氏。与张仪不善。

张仪为秦之魏,魏王相张仪。犀首弗利,故令人谓韩公叔曰:“张仪已合秦魏矣,其言曰‘魏攻南阳,秦攻三川’。魏王所以贵张子者,欲得韩地也。且韩之南阳已举矣,子何不少委焉以为衍功,则秦魏之交可错矣。然则魏必图秦而弃仪,收韩而相衍。”公叔以为便,因委之犀首以为功。果相魏。张仪去。

义渠君朝于魏。犀首闻张仪复相秦,害之。犀首乃谓义渠君曰:“道远不得复过,请谒事情。”曰:“中国无事,秦得烧掇焚于君之国;有事,秦将轻使重币事君之国。”其后五国伐秦。会陈轸谓秦王曰:“义渠君者,蛮夷之贤君也,不如赂之以抚其志。”秦王曰:“善。”乃以文绣千纯,妇女百人遗义渠君。义渠君致群臣而谋曰:“此公孙衍所谓邪?”乃起兵袭秦,大败秦人李伯之下。

张仪已卒之后,犀首入相秦。尝佩五国之相印,为约长。

太史公曰:三晋多权变之士,夫言从衡彊秦者大抵皆三晋之人也。夫张仪之行事甚于苏秦,然世恶苏秦者,以其先死,而仪振暴其短以扶其说,成其衡道。要之,此两人真倾危之士哉!

仪未遭时,频被困辱。及相秦惠,先韩后蜀。连衡齐魏,倾危诳惑。陈轸挟权,犀首骋欲。如何三晋,继有斯德。
\end{yuanwen}

\part{卷七十一}
\chapter{樗里子甘茂列传第十一}

\begin{yuanwen}
樗里子者,名疾,秦惠王之弟也,与惠王异母。母,韩女也。樗里子滑稽\footnote{流酒器,可转注吐酒不已。引申为能言善辩,言辞流利。}多智,秦人号曰“智囊”。
\end{yuanwen}

樗里子,名疾,是秦惠王的弟弟,他与惠王不是同一个母亲生的。他的母亲是一个韩国女子。樗里子口齿伶俐,足智多谋,秦国人称之为“智囊”。

\begin{yuanwen}
秦惠王八年,爵樗里子右更\footnote{爵位名,二十等爵位中的十四级。},使将而伐曲沃,尽出其人,取其城,地入秦。秦惠王二十五年,使樗里子为将伐赵,虏赵将军庄豹,拔蔺。

明年,助魏章攻楚,败楚将屈丐,取汉中地。秦封樗里子,号为严君。
\end{yuanwen}

秦惠王八年(前330年),樗里子得到右更的爵位,派他领兵攻打魏国的曲沃,把当地军民全部驱逐,攻取了这座城,土地纳入秦国版图。秦惠王二十五年(前313年),秦国派樗里子担任将领攻打赵国,俘虏了赵国的将军庄豹,攻取了蔺城。

第二年,他协助魏章攻打楚国,打败了楚国的将军屈丐,攻取了汉中地区。秦国封赏樗里子,封号是严君。

\begin{yuanwen}
秦惠王卒,太子武王立,逐张仪、魏章,而以樗里子、甘茂为左右丞相。秦使甘茂攻韩,拔宜阳。使樗里子以车百乘入周。周以卒迎之,意甚敬。楚王怒,让周,以其重秦客。游腾为周说楚王曰:“知伯之伐仇犹,遗之广车\footnote{大车。},因随之以兵,仇犹遂亡。何则?无备故也。齐桓公伐蔡,号曰诛楚,其实袭蔡。今秦,虎狼之国,使樗里子以车百乘入周,周以仇犹、蔡观焉,故使长戟居前,彊弩在后,名曰卫疾,而实囚之。且夫周岂能无忧其社稷哉?恐一旦亡国以忧大王。”楚王乃悦。
\end{yuanwen}

秦惠王死后,太子武王即位,驱逐了张仪、魏章,而任用樗里子、甘茂为左右丞相。秦国派甘茂攻打韩国,攻取了宜阳。又派樗里子率领百辆战车进入周都洛阳。周天子派士兵迎接他,态度非常恭敬。楚王很生气,责备周天子,认为他太重视秦国客人。游腾替周天子游说楚王道:“知伯攻打仇犹之前,先送给他们豪华的马车,趁机派兵跟随在后面,仇犹于是灭亡了。为什么会这样呢?因为没有防备的缘故。齐桓公讨伐蔡国之时,名义上是要除掉楚国,其实是为了袭击蔡国。现在的秦国,是虎狼一样的国家,派樗里子率领战车百辆进入周都洛阳,周天子以仇犹和蔡国作为借鉴,所以将长戟布置在前面,强弩布置在后面,名义上是护卫樗里子,但实际上是将他看管起来。况且周天子怎么能够不担心自己的国家呢?只怕有一天国家灭亡了会使大王难过。”楚王于是高兴起来。

\begin{yuanwen}
秦武王卒,昭王立,樗里子又益尊重。
\end{yuanwen}

秦武王死后,秦昭王即位,樗里子的地位更加尊贵显要。

\begin{yuanwen}
昭王元年,樗里子将伐蒲。蒲守恐,请胡衍。胡衍为蒲谓樗里子曰:“公之攻蒲,为秦乎?为魏乎?为魏则善矣,为秦则不为赖\footnote{有利。}矣。夫卫之所以为卫者,以蒲也。今伐蒲入于魏,卫必折而从之。魏亡西河之外而无以取者,兵弱也。今并卫于魏,魏必彊。魏彊之日,西河之外必危矣。且秦王将观公之事,害秦而利魏,王必罪公。”

樗里子曰:“奈何?”

胡衍曰:“公释蒲勿攻,臣试为公入言之,以德卫君。”

樗里子曰:“善。”

胡衍入蒲,谓其守曰:“樗里子知蒲之病矣,其言曰必拔蒲。衍能令释蒲勿攻。”

蒲守恐,因再拜曰:“原以请。”

因效金三百斤,曰:“秦兵苟退,请必言子于卫君,使子为南面。”

故胡衍受金于蒲以自贵于卫。于是遂解蒲而去。还击皮氏,皮氏未降,又去。
\end{yuanwen}

秦昭王元年(前306年),樗里子将要攻打卫国的蒲城。蒲地的守将感到恐惧,请胡衍来。胡衍替蒲城的守将对樗里子说:“您攻打蒲城,是为了秦国呢?还是为了魏国呢?为了魏国当然好了,为了秦国就得不到什么利益了。卫国之所以成为卫国,是因为有蒲城存在。现在您要攻打蒲城使它被划入魏国的版图,卫国一定臣服并追随魏国。魏国丢失西河以外的土地而没有办法夺回来的原因,就在于兵力太弱。现在把卫国并入魏国,魏国一定会变得强大。魏国强大的时候,秦国占领的西河以外的土地一定危险了。况且秦王将会观察您的行动,如果对秦国有害而对魏国有利,秦王一定会怪罪您。”

樗里子说:“该怎么办呢?”

胡衍说:“您放过蒲城不要进攻,我试着替您进入蒲城说说这件事,让卫君记住您的恩德。”

樗里子说:“很好。”

胡衍进入蒲城,对蒲城的守将说:“樗里子知道蒲城的弱点了,他说一定要攻下蒲城。我能够让他放过蒲城不来进攻。”

蒲城的守将感到害怕,因此两次下拜说:“希望就这件事向您请教。”

顺势拿出黄金三百斤,说:“秦军如果真的撤退了,我一定会向卫君汇报您的功劳,封您为一方之主。”

所以胡衍在蒲城接受了黄金之后,使自己在卫国成为显贵。于是樗里子解除了对蒲城的包围离开了。他在回师途中攻打魏国皮氏,皮氏还没有投降,他又撤走了。

\begin{yuanwen}
昭王七年,樗里子卒,葬于渭南章台之东。曰:“后百岁,是当有天子之宫夹我墓。”

樗里子疾室在于昭王庙西渭南阴乡樗里,故俗谓之樗里子。至汉兴,长乐宫在其东,未央宫在其西,武库正直其墓。秦人谚曰:“力则任鄙,智则樗里。”
\end{yuanwen}

秦昭王七年(前300年),樗里子去世,安葬在渭水南岸章台的东边。他曾说:“百年以后,这里应该会有天子的宫殿环绕着我的坟墓。”

樗里子的住宅在秦昭王庙西边渭南阴乡的樗里,所以世俗称他为樗里子。到汉朝建立以后,长乐宫建在他坟墓的东边,未央宫建在他坟墓的西边,武库正对着他的坟墓。秦人有句俗话说:“力气大要数任鄙,智谋多要数樗里。”

\begin{yuanwen}
甘茂者,下蔡人也。事下蔡史举先生,学百家之术。因张仪、樗里子而求见秦惠王。王见而说之,使将,而佐魏章略定汉中地。
\end{yuanwen}

甘茂,楚国下蔡人,曾追随下蔡的史举先生,学习诸子百家的学术。他通过张仪、樗里子的引荐得以见到秦惠王。秦惠王见到他后非常欣赏,任命他为将,辅佐魏章攻略平定了汉中地区。

\begin{yuanwen}
惠王卒,武王立。张仪、魏章去,东之魏。蜀侯煇、相壮反,秦使甘茂定蜀。还,而以甘茂为左丞相,以樗里子为右丞相。
\end{yuanwen}

秦惠王死后,武王即位,张仪、魏章离开秦国,向东前往魏国。蜀侯公子辉、蜀相陈壮反叛,秦王派甘茂平定蜀乱。回国后,秦王任命甘茂为左丞相,任命樗里子为右丞相。

\begin{yuanwen}
秦武王三年,谓甘茂曰:“寡人欲容车通三川,以窥周室,而寡人死不朽矣。”

甘茂曰:“请之魏,约以伐韩,而令向寿辅行\footnote{副使。}。”

甘茂至,谓向寿曰:“子归,言之于王曰‘魏听臣矣,然原王勿伐’。事成,尽以为子功。”

向寿归,以告王,王迎甘茂于息壤。甘茂至,王问其故。对曰:“宜阳,大县也,上党、南阳积之久矣。名曰县,其实郡也。今王倍数险,行千里攻之,难。昔曾参之处费,鲁人有与曾参同姓名者杀人,人告其母曰‘曾参杀人’,其母织自若也。顷之,一人又告之曰‘曾参杀人’,其母尚织自若也。顷又一人告之曰‘曾参杀人’,其母投杼下机,逾墙而走。夫以曾参之贤与其母信之也,三人疑之,其母惧焉。今臣之贤不若曾参,王之信臣又不如曾参之母信(著/曾)参也,疑臣者非特三人,臣恐大王之投杼也。始张仪西并巴蜀之地,北开西河之外,南取上庸,天下不以多张子而以贤先王。魏文侯令乐羊将而攻中山,三年而拔之。乐羊返而论功,文侯示之谤书一箧。乐羊再拜稽首曰:‘此非臣之功也,主君之力也。’今臣,羁旅之臣也。樗里子、公孙奭二人者挟韩而议之,王必听之,是王欺魏王而臣受公仲侈之怨也。”

王曰:“寡人不听也,请与子盟。”

卒使丞相甘茂将兵伐宜阳。五月而不拔,樗里子、公孙奭果争之。武王召甘茂,欲罢兵。甘茂曰:“息壤在彼。”

王曰:“有之。”因大悉起兵,使甘茂击之。斩首六万,遂拔宜阳。韩襄王使公仲侈入谢,与秦平。
\end{yuanwen}

秦武王三年(前308年),对甘茂说:“我想乘坐垂着幔帐的车子经过三川郡,来看一眼周王室,这样我就死而无憾了。”

甘茂说:“我请求出使魏国,约定讨伐韩国一事,希望派向寿做副使。”

甘茂到魏国后,对向寿说:“你回去,跟大王说‘魏国听命于我了,但是希望大王暂时不要出兵攻伐’。事成之后,全算作你的功劳。”

向寿回国,把这些话报告给武王。武王在息壤迎接甘茂。甘茂来了,武王询问其中缘故。甘茂回答说:“宜阳,是韩国的大县,上党、南阳守备很久了。这里名义上是县,实际上相当于一个郡。现在大王冒着数倍的风险,远涉千里去攻打它,太难了。从前曾参居住在费地,鲁国有个与他同名的人杀人,有人告诉他的母亲说‘曾参杀人了’,他的母亲还像往常那样织布。不一会儿,又一个人告诉她‘曾参杀人了’,他的母亲仍然神态自若。一会儿又一个人告诉她‘曾参杀人了’,他的母亲扔下机杼,跳墙逃走了。按照曾参的贤德和他母亲对儿子的信任来说,有三个人怀疑,他的母亲就感到恐惧。现在我的贤德不如曾参,大王对我的信任也不如曾参的母亲信任曾参,怀疑我的又不只三人,我怕大王也像曾母那样扔下机杼。起初张仪向西兼并巴蜀地区,向北开拓西河外围的地区,向南攻取上庸,天下人不因此夸奖他,反而夸奖大王贤德。魏文侯派乐羊为将攻打中山国,三年后破城。乐羊回国后评定功劳大小,魏文侯却拿给他看一箩筐诽谤他的奏疏。乐羊两次跪拜叩头说:‘这不是我的功劳,是君上的神力。’现在我只是个寄居在秦的客卿,樗里子、公孙奭二人拿着攻打韩国的事情议论,大王一定会听信他们的话,这样大王就是在欺骗魏王而我则会受到韩相公仲侈的怨恨。”

武王说:“我不听他们的,请让我和你立约为誓。”

最后终于使丞相甘茂率兵攻打宜阳。五个月之后尚未攻下,樗里子与公孙奭果然议论这件事。武王召甘茂回国,打算停止进攻。甘茂说:“息壤之盟还在那里。”

武王说:“有这回事。”因此大规模征调军队支援,让甘茂继续进攻。秦军斩首六万,终于把宜阳攻下。韩襄王派公仲侈前来谢罪,与秦国达成和议。

\begin{yuanwen}
武王竟至周,而卒于周。其弟立,为昭王。王母宣太后,楚女也。楚怀王怨前秦败楚于丹阳而韩不救,乃以兵围韩雍氏。韩使公仲侈告急于秦。秦昭王新立,太后楚人,不肯救。公仲因甘茂,茂为韩言于秦昭王曰:“公仲方有得秦救,故敢扞楚也。今雍氏围,秦师不下殽,公仲且仰首而不朝,公叔且以国南合于楚。楚、韩为一,魏氏不敢不听,然则伐秦之形成矣。不识坐而待伐孰与伐人之利?”

秦王曰:“善。”乃下师于殽以救韩。楚兵去。
\end{yuanwen}

秦武王终究去了周都,并死在那里。他的弟弟即位,就是昭王。秦昭王的母亲宣太后,是楚国的女子。楚怀王记恨先前秦国在丹阳打败楚国而韩国不来救援,于是发兵围攻韩国雍氏。韩国派公仲侈向秦国求救。秦昭王刚即位,太后又是楚国人,所以不愿意出兵相救。公仲侈通过甘茂,甘茂替韩国向秦昭王进言说:“公仲侈刚刚得到秦国的援助,所以才敢与楚国抗衡。现在雍氏被围,秦军不出殽山,公仲氏将要昂首不来朝见,公叔氏将要把韩国南部并入楚国。楚国、韩国合为一家,魏国不敢不服从,这样攻伐秦国的形势就形成了。难道大王不知道坐着等待被人攻打和主动攻打别人哪个更有利吗?”

秦昭王说:“很好。”于是发兵出殽山来救援韩国。楚国军队撤退了。

\begin{yuanwen}

\end{yuanwen}\begin{yuanwen}

\end{yuanwen}\begin{yuanwen}

\end{yuanwen}\begin{yuanwen}

\end{yuanwen}\begin{yuanwen}

\end{yuanwen}\begin{yuanwen}

\end{yuanwen}\begin{yuanwen}

\end{yuanwen}\begin{yuanwen}

\end{yuanwen}\begin{yuanwen}

\end{yuanwen}\begin{yuanwen}

\end{yuanwen}\begin{yuanwen}

\end{yuanwen}\begin{yuanwen}

\end{yuanwen}\begin{yuanwen}

\end{yuanwen}\begin{yuanwen}

\end{yuanwen}\begin{yuanwen}

\end{yuanwen}\begin{yuanwen}

\end{yuanwen}\begin{yuanwen}

\end{yuanwen}\begin{yuanwen}

\end{yuanwen}\begin{yuanwen}

\end{yuanwen}\begin{yuanwen}

\end{yuanwen}\begin{yuanwen}

\end{yuanwen}\begin{yuanwen}

\end{yuanwen}\begin{yuanwen}




秦使向寿平宜阳,而使樗里子、甘茂伐魏皮氏。向寿者,宣太后外族也,而与昭王少相长,故任用。向寿如楚,楚闻秦之贵向寿,而厚事向寿。向寿为秦守宜阳,将以伐韩。韩公仲使苏代谓向寿曰:“禽困覆车。公破韩,辱公仲,公仲收国复事秦,自以为必可以封。今公与楚解口地,封小令尹以杜阳。秦楚合,复攻韩,韩必亡。韩亡,公仲且躬率其私徒以阏于秦。原公孰虑之也。”向寿曰:“吾合秦楚非以当韩也,子为寿谒之公仲,曰秦韩之交可合也。”苏代对曰:“原有谒于公。人曰贵其所以贵者贵。王之爱习公也,不如公孙奭;其智能公也,不如甘茂。今二人者皆不得亲于秦事,而公独与王主断于国者何?彼有以失之也。公孙奭党于韩,而甘茂党于魏,故王不信也。今秦楚争彊而公党于楚,是与公孙奭、甘茂同道也,公何以异之?人皆言楚之善变也,而公必亡之,是自为责也。公不如与王谋其变也,善韩以备楚,如此则无患矣。韩氏必先以国从公孙奭而后委国于甘茂。韩,公之雠也。今公言善韩以备楚,是外举不僻雠也。”向寿曰:“然,吾甚欲韩合。”对曰:“甘茂许公仲以武遂,反宜阳之民,今公徒收之,甚难。”向寿曰:“然则奈何?武遂终不可得也?”对曰:“公奚不以秦为韩求颍川于楚?此韩之寄地也。公求而得之,是令行于楚而以其地德韩也。公求而不得,是韩楚之怨不解而交走秦也。秦楚争彊,而公徐过楚以收韩,此利于秦。”向寿曰:“柰何?”对曰:“此善事也。甘茂欲以魏取齐,公孙奭欲以韩取齐。今公取宜阳以为功,收楚韩以安之,而诛齐魏之罪,是以公孙奭、甘茂无事也。”

甘茂竟言秦昭王,以武遂复归之韩。向寿、公孙奭争之,不能得。向寿、公孙奭由此怨,谗甘茂。茂惧,辍伐魏蒲阪,亡去。樗里子与魏讲,罢兵。

甘茂之亡秦奔齐,逢苏代。代为齐使于秦。甘茂曰:“臣得罪于秦,惧而遯逃,无所容迹。臣闻贫人女与富人女会绩,贫人女曰:‘我无以买烛,而子之烛光幸有馀,子可分我馀光,无损子明而得一斯便焉。’今臣困而君方使秦而当路矣。茂之妻子在焉,原君以馀光振之。”苏代许诺。遂致使于秦。已,因说秦王曰:“甘茂,非常士也。其居于秦,累世重矣。自殽塞及至鬼谷,其地形险易皆明知之。彼以齐约韩魏反以图秦,非秦之利也。”秦王曰:“然则柰何?”苏代曰:“王不若重其贽,厚其禄以迎之,使彼来则置之鬼谷,终身勿出。”秦王曰:“善。”即赐之上卿,以相印迎之于齐。甘茂不往。苏代谓齐湣王曰:“夫甘茂,贤人也。今秦赐之上卿,以相印迎之。甘茂德王之赐,好为王臣,故辞而不往。今王何以礼之?”齐王曰:“善。”即位之上卿而处之。秦因复甘茂之家以市于齐。

齐使甘茂于楚,楚怀王新与秦合婚而驩。而秦闻甘茂在楚,使人谓楚王曰:“原送甘茂于秦。”楚王问于范蜎曰:“寡人欲置相于秦,孰可?”对曰:“臣不足以识之。”楚王曰:“寡人欲相甘茂,可乎?”对曰:“不可。夫史举,下蔡之监门也,大不为事君,小不为家室,以苟贱不廉闻于世,甘茂事之顺焉。故惠王之明,武王之察,张仪之辩,而甘茂事之,取十官而无罪。茂诚贤者也,然不可相于秦。夫秦之有贤相,非楚国之利也。,且王前尝用召滑于越,而内行章义之难,越国乱,故楚南塞厉门而郡江东。计王之功所以能如此者,越国乱而楚治也。今王知用诸越而忘用诸秦,臣以王为钜过矣。然则王若欲置相于秦,则莫若向寿者可。夫向寿之于秦王,亲也,少与之同衣,长与之同车,以听事。王必相向寿于秦,则楚国之利也。”于是使使请秦相向寿于秦。秦卒相向寿。而甘茂竟不得复入秦,卒于魏。

甘茂有孙曰甘罗。

甘罗者,甘茂孙也。茂既死后,甘罗年十二,事秦相文信侯吕不韦。

秦始皇帝使刚成君蔡泽于燕,三年而燕王喜使太子丹入质于秦。秦使张唐往相燕,欲与燕共伐赵以广河间之地。张唐谓文信侯曰:“臣尝为秦昭王伐赵,赵怨臣,曰:‘得唐者与百里之地。’今之燕必经赵,臣不可以行。”文信侯不快,未有以彊也。甘罗曰:“君侯何不快之甚也?”文信侯曰:“吾令刚成君蔡泽事燕三年,燕太子丹已入质矣,吾自请张卿相燕而不肯行。”甘罗曰:“臣请行之。”文信侯叱曰:“去!我身自请之而不肯,女焉能行之?”甘罗曰:“大项橐生七岁为孔子师。今臣生十二岁于兹矣,君其试臣,何遽叱乎?”于是甘罗见张卿曰:“卿之功孰与武安君?”卿曰:“武安君南挫彊楚,北威燕、赵,战胜攻取,破城堕邑,不知其数,臣之功不如也。”甘罗曰:“应侯之用于秦也,孰与文信侯专?”张卿曰:“应侯不如文信侯专。”甘罗曰:“卿明知其不如文信侯专与?”曰:“知之。”甘罗曰:“应侯欲攻赵,武安君难之,去咸阳七里而立死于杜邮。今文信侯自请卿相燕而不肯行,臣不知卿所死处矣。”张唐曰:“请因孺子行。”令装治行。

行有日,甘罗谓文信侯曰:“借臣车五乘,请为张唐先报赵。”文信侯乃入言之于始皇曰:“昔甘茂之孙甘罗,年少耳,然名家之子孙,诸侯皆闻之。今者张唐欲称疾不肯行,甘罗说而行之。今原先报赵,请许遣之。”始皇召见,使甘罗于赵。赵襄王郊迎甘罗。甘罗说赵王曰:“王闻燕太子丹入质秦欤?”曰:“闻之。”曰:“闻张唐相燕欤?”曰:“闻之。”“燕太子丹入秦者,燕不欺秦也。张唐相燕者,秦不欺燕也。燕、秦不相欺者,伐赵,危矣。燕、秦不相欺无异故,欲攻赵而广河间。王不如赍臣五城以广河间,请归燕太子,与彊赵攻弱燕。”赵王立自割五城以广河间。秦归燕太子。赵攻燕,得上谷三十城,令秦有十一。

甘罗还报秦,乃封甘罗以为上卿,复以始甘茂田宅赐之。

太史公曰:樗里子以骨肉重,固其理,而秦人称其智,故颇采焉。甘茂起下蔡闾阎,显名诸侯,重彊齐楚。甘罗年少,然出一奇计,声称后世。虽非笃行之君子,然亦战国之策士也。方秦之彊时,天下尤趋谋诈哉

严君名疾,厥号“智囊”。既亲且重,称兵外攘。甘茂并相,初佐魏章。始推向寿,乃攻宜阳。甘罗妙岁,卒起张唐。
\end{yuanwen}

\chapter{穰侯列传}

\begin{yuanwen}
穰侯魏厓者,秦昭王母宣太后弟也。其先楚人,姓羋氏。

秦武王卒,无子,立其弟为昭王。昭王母故号为羋八子,及昭王即位,羋八子号为宣太后。宣太后非武王母。武王母号曰惠文后,先武王死。宣太后二弟:其异父长弟曰穰侯,姓魏氏,名厓;同父弟曰羋戎,为华阳君。而昭王同母弟曰高陵君、泾阳君。而魏厓最贤,自惠王、武王时任职用事。武王卒,诸弟争立,唯魏厓力为能立昭王。昭王即位,以厓为将军,卫咸阳。诛季君之乱,而逐武王后出之魏,昭王诸兄弟不善者皆灭之,威振秦国。昭王少,宣太后自治,任魏厓为政。

昭王七年,樗里子死,而使泾阳君质于齐。赵人楼缓来相秦,赵不利,乃使仇液之秦,请以魏厓为秦相。仇液将行,其客宋公谓液曰:“秦不听公,楼缓必怨公。公不若谓楼缓曰‘请为公毋急秦’。秦王见赵请相魏厓之不急,且不听公。公言而事不成,以德楼子;事成,魏厓故德公矣。”于是仇液从之。而秦果免楼缓而魏厓相秦。

欲诛吕礼,礼出奔齐。昭王十四年,魏厓举白起,使代向寿将而攻韩、魏,败之伊阙,斩首二十四万,虏魏将公孙喜。明年,又取楚之宛、叶。魏厓谢病免相,以客卿寿烛为相。其明年,烛免,复相厓,乃封魏厓于穰,复益封陶,号曰穰侯。

穰侯封四岁,为秦将攻魏。魏献河东方四百里。拔魏之河内,取城大小六十馀。昭王十九年,秦称西帝,齐称东帝。月馀,吕礼来,而齐、秦各复归帝为王。魏厓复相秦,六岁而免。免二岁,复相秦。四岁,而使白起拔楚之郢,秦置南郡。乃封白起为武安君。白起者,穰侯之所任举也,相善。于是穰侯之富,富于王室。

昭王三十二年,穰侯为相国,将兵攻魏,走芒卯,入北宅,遂围大梁。梁大夫须贾说穰侯曰:“臣闻魏之长吏谓魏王曰:‘昔梁惠王伐赵,战胜三梁,拔邯郸;赵氏不割,而邯郸复归。齐人攻卫,拔故国,杀子良;卫人不割,而故地复反。卫、赵之所以国全兵劲而地不并于诸侯者,以其能忍难而重出地也。宋、中山数伐割地,而国随以亡。臣以为卫、赵可法,而宋、中山可为戒也。秦,贪戾之国也,而毋亲。蚕食魏氏,又尽晋国,战胜暴子,割八县,地未毕入,兵复出矣。夫秦何厌之有哉!今又走芒卯,入北宅,此非敢攻梁也,且劫王以求多割地。王必勿听也。今王背楚、赵而讲秦,楚、赵怒而去王,与王争事秦,秦必受之。秦挟楚、赵之兵以复攻梁,则国求无亡不可得也。原王之必无讲也。王若欲讲,少割而有质;不然,必见欺。’此臣之所闻于魏也,原君之以是虑事也。周书曰‘惟命不于常’,此言幸之不可数也。夫战胜暴子,割八县,此非兵力之精也,又非计之工也,天幸为多矣。今又走芒卯,入北宅,以攻大梁,是以天幸自为常也。智者不然。臣闻魏氏悉其百县胜甲以上戍大梁,臣以为不下三十万。以三十万之众守梁七仞之城,臣以为汤、武复生,不易攻也。夫轻背楚、赵之兵,陵七仞之城,战三十万之众,而志必举之,臣以为自天地始分以至于今,未尝有者也。攻而不拔,秦兵必罢,陶邑必亡,则前功必弃矣。今魏氏方疑,可以少割收也。原君逮楚、赵之兵未至于梁,亟以少割收魏。魏方疑而得以少割为利,必欲之,则君得所欲矣。楚、赵怒于魏之先己也,必争事秦,从以此散,而君后择焉。且君之得地岂必以兵哉!割晋国,秦兵不攻,而魏必效绛安邑。又为陶开两道,几尽故宋,卫必效单父。秦兵可全,而君制之,何索而不得,何为而不成!原君熟虑之而无行危。”穰侯曰:“善。”乃罢梁围。

明年,魏背秦,与齐从亲。秦使穰侯伐魏,斩首四万,走魏将暴鸢,得魏三县。穰侯益封。

明年,穰侯与白起客卿胡阳复攻赵、韩、魏,破芒卯于华阳下,斩首十万,取魏之卷、蔡阳、长社,赵氏观津。且与赵观津,益赵以兵,伐齐。齐襄王惧,使苏代为齐阴遗穰侯书曰:“臣闻往来者言曰‘秦将益赵甲四万以伐齐’,臣窃必之敝邑之王曰‘秦王明而熟于计,穰侯智而习于事,必不益赵甲四万以伐齐’。是何也?夫三晋之相与也,秦之深雠也。百相背也,百相欺也,不为不信,不为无行。今破齐以肥赵。赵,秦之深雠,不利于秦。此一也。秦之谋者,必曰‘破齐,弊晋、楚,而后制晋、楚之胜’。夫齐,罢国也,以天下攻齐,如以千钧之弩决溃筴也,必死,安能弊晋、楚?此二也。秦少出兵,则晋、楚不信也;多出兵,则晋、楚为制于秦。齐恐,不走秦,必走晋、楚。此三也。秦割齐以啖晋、楚,晋、楚案之以兵,秦反受敌。此四也。是晋、楚以秦谋齐,以齐谋秦也,何晋、楚之智而秦、齐之愚?此五也。故得安邑以善事之,亦必无患矣。秦有安邑,韩氏必无上党矣。取天下之肠胃,与出兵而惧其不反也,孰利?臣故曰秦王明而熟于计,穰侯智而习于事,必不益赵甲四万以代齐矣。”于是穰侯不行,引兵而归。

昭王三十六年,相国穰侯言客卿灶,欲伐齐取刚、寿,以广其陶邑。于是魏人范睢自谓张禄先生,讥穰侯之伐齐,乃越三晋以攻齐也,以此时奸说秦昭王。昭王于是用范睢。范睢言宣太后专制,穰侯擅权于诸侯,泾阳君、高陵君之属太侈,富于王室。于是秦昭王悟,乃免相国,令泾阳之属皆出关,就封邑。穰侯出关,辎车千乘有馀。

穰侯卒于陶,而因葬焉。秦复收陶为郡。

太史公曰:穰侯,昭王亲舅也。而秦所以东益地,弱诸侯,尝称帝于天下,天下皆西乡稽首者,穰侯之功也。及其贵极富溢,一夫开说,身折势夺而以忧死,况于羁旅之臣乎!

穰侯智识,应变无方。内倚太后,外辅昭王。四登相位,再列封疆。摧齐挠楚,破魏围梁。一夫开说,忧愤而亡。
\end{yuanwen}

\chapter{白起王翦列传}

\begin{yuanwen}
白起者,郿人也。善用兵,事秦昭王。昭王十三年,而白起为左庶长,将而击韩之新城。是岁,穰侯相秦,举任鄙以为汉中守。其明年,白起为左更,攻韩、魏于伊阙,斩首二十四万,又虏其将公孙喜,拔五城。起迁为国尉。涉河取韩安邑以东,到乾河。明年,白起为大良造。攻魏,拔之,取城小大六十一。明年,起与客卿错攻垣城,拔之。后五年,白起攻赵,拔光狼城。后七年,白起攻楚,拔鄢、邓五城。其明年,攻楚,拔郢,烧夷陵,遂东至竟陵。楚王亡去郢,东走徙陈。秦以郢为南郡。白起迁为武安君。武安君因取楚,定巫、黔中郡。昭王三十四年,白起攻魏,拔华阳,走芒卯,而虏三晋将,斩首十三万。与赵将贾偃战,沈其卒二万人于河中。昭王四十三年,白起攻韩陉城,拔五城,斩首五万。四十四年,白起攻南阳太行道,绝之。

四十五年,伐韩之野王。野王降秦,上党道绝。其守冯亭与民谋曰:“郑道已绝,韩必不可得为民。秦兵日进,韩不能应,不如以上党归赵。赵若受我,秦怒,必攻赵。赵被兵,必亲韩。韩赵为一,则可以当秦。”因使人报赵。赵孝成王与平阳君、平原君计之。平阳君曰:“不如勿受。受之,祸大于所得。”平原君曰:“无故得一郡,受之便。”赵受之,因封冯亭为华阳君。

四十六年,秦攻韩缑氏、蔺,拔之。

四十七年,秦使左庶长王龁攻韩,取上党。上党民走赵。赵军长平,以按据上党民。四月,龁因攻赵。赵使廉颇将。赵军士卒犯秦斥兵,秦斥兵斩赵裨将茄。六月,陷赵军,取二鄣四尉。七月,赵军筑垒壁而守之。秦又攻其垒,取二尉,败其阵,夺西垒壁。廉颇坚壁以待秦,秦数挑战,赵兵不出。赵王数以为让。而秦相应侯又使人行千金于赵为反间,曰:“秦之所恶,独畏马服子赵括将耳,廉颇易与,且降矣。”赵王既怒廉颇军多失亡,军数败,又反坚壁不敢战,而又闻秦反间之言,因使赵括代廉颇将以击秦。秦闻马服子将,乃阴使武安君白起为上将军。而王龁为尉裨将,令军中有敢泄武安君将者斩。赵括至,则出兵击秦军。秦军详败而走,张二奇兵以劫之。赵军逐胜,追造秦壁。壁坚拒不得入,而秦奇兵二万五千人绝赵军后,又一军五千骑绝赵壁间,赵军分而为二,粮道绝。而秦出轻兵击之。赵战不利,因筑壁坚守,以待救至。秦王闻赵食道绝,王自之河内,赐民爵各一级,发年十五以上悉诣长平,遮绝赵救及粮食。

至九月,赵卒不得食四十六日,皆内阴相杀食。来攻秦垒,欲出。为四队,四五复之,不能出。其将军赵括出锐卒自搏战,秦军射杀赵括。括军败,卒四十万人降武安君。武安君计曰:“前秦已拔上党,上党民不乐为秦而归赵。赵卒反覆。非尽杀之,恐为乱。”乃挟诈而尽阬杀之,遗其小者二百四十人归赵。前后斩首虏四十五万人。赵人大震。

四十八年十月,秦复定上党郡。秦分军为二:王龁攻皮牢,拔之;司马梗定太原。韩、赵恐,使苏代厚币说秦相应侯曰:“武安君禽马服子乎?”曰:“然。”又曰:“即围邯郸乎?”曰:“然。”“赵亡则秦王王矣,武安君为三公。武安君所为秦战胜攻取者七十馀城,南定鄢、郢、汉中,北禽赵括之军,虽周、召、吕望之功不益于此矣。今赵亡,秦王王,则武安君必为三公,君能为之下乎?虽无欲为之下,固不得已矣。秦尝攻韩,围邢丘,困上党,上党之民皆反为赵,天下不乐为秦民之日久矣。今亡赵,北地入燕,东地入齐,南地入韩、魏,则君之所得民亡几何人。故不如因而割之,无以为武安君功也。”于是应侯言于秦王曰:“秦兵劳,请许韩、赵之割地以和,且休士卒。”王听之,割韩垣雍、赵六城以和。正月,皆罢兵。武安君闻之,由是与应侯有隙。

其九月,秦复发兵,使五大夫王陵攻赵邯郸。是时武安君病,不任行。四十九年正月,陵攻邯郸,少利,秦益发兵佐陵。陵兵亡五校。武安君病愈,秦王欲使武安君代陵将。武安君言曰:“邯郸实未易攻也。且诸侯救日至,彼诸侯怨秦之日久矣。今秦虽破长平军,而秦卒死者过半,国内空。远绝河山而争人国都,赵应其内,诸侯攻其外,破秦军必矣。不可。”秦王自命,不行;乃使应侯请之,武安君终辞不肯行,遂称病。

秦王使王龁代陵将,八九月围邯郸,不能拔。楚使春申君及魏公子将兵数十万攻秦军,秦军多失亡。武安君言曰:“秦不听臣计,今如何矣!”秦王闻之,怒,彊起武安君,武安君遂称病笃。应侯请之,不起。于是免武安君为士伍,迁之阴密。武安君病,未能行。居三月,诸侯攻秦军急,秦军数卻,使者日至。秦王乃使人遣白起,不得留咸阳中。武安君既行,出咸阳西门十里,至杜邮。秦昭王与应侯群臣议曰:“白起之迁,其意尚怏怏不服,有馀言。”秦王乃使使者赐之剑,自裁。武安君引剑将自刭,曰:“我何罪于天而至此哉?”良久,曰:“我固当死。长平之战,赵卒降者数十万人,我诈而尽阬之,是足以死。”遂自杀。武安君之死也,以秦昭王五十年十一月。死而非其罪,秦人怜之,乡邑皆祭祀焉。

王翦者,频阳东乡人也。少而好兵,事秦始皇。始皇十一年,翦将攻赵阏与,破之,拔九城,十八年,翦将攻赵。岁馀,遂拔赵,赵王降,尽定赵地为郡。明年,燕使荆轲为贼于秦,秦王使王翦攻燕。燕王喜走辽东,翦遂定燕蓟而还。秦使翦子王贲击荆,荆兵败。还击魏,魏王降,遂定魏地。

秦始皇既灭三晋,走燕王,而数破荆师。秦将李信者,年少壮勇,尝以兵数千逐燕太子丹至于衍水中,卒破得丹,始皇以为贤勇。于是始皇问李信:“吾欲攻取荆,于将军度用几何人而足?”李信曰:“不过用二十万人。”始皇问王翦,王翦曰:“非六十万人不可。”始皇曰:“王将军老矣,何怯也!李将军果势壮勇,其言是也。”遂使李信及蒙恬将二十万南伐荆。王翦言不用,因谢病,归老于频阳。李信攻平与,蒙恬攻寝,大破荆军。信又攻鄢郢,破之,于是引兵而西,与蒙恬会城父。荆人因随之,三日三夜不顿舍,大破李信军,入两壁,杀七都尉,秦军走。

始皇闻之,大怒,自驰如频阳,见谢王翦曰:“寡人以不用将军计,李信果辱秦军。今闻荆兵日进而西,将军虽病,独忍弃寡人乎!”王翦谢曰:“老臣罢病悖乱,唯大王更择贤将。”始皇谢曰:“已矣,将军勿复言!”王翦曰:“大王必不得已用臣,非六十万人不可。”始皇曰:“为听将军计耳。”于是王翦将兵六十万人,始皇自送至灞上。王翦行,请美田宅园池甚众。始皇曰:“将军行矣,何忧贫乎?”王翦曰:“为大王将,有功终不得封侯,故及大王之乡臣,臣亦及时以请园池为子孙业耳。”始皇大笑。王翦既至关,使使还请善田者五辈。或曰:“将军之乞贷,亦已甚矣。”王翦曰:“不然。夫秦王怚而不信人。今空秦国甲士而专委于我,我不多请田宅为子孙业以自坚,顾令秦王坐而疑我邪?”

王翦果代李信击荆。荆闻王翦益军而来,乃悉国中兵以拒秦。王翦至,坚壁而守之,不肯战。荆兵数出挑战,终不出。王翦日休士洗沐,而善饮食抚循之,亲与士卒同食。久之,王翦使人问军中戏乎?对曰:“方投石超距。”于是王翦曰:“士卒可用矣。”荆数挑战而秦不出,乃引而东。翦因举兵追之,令壮士击,大破荆军。至蕲南,杀其将军项燕,荆兵遂败走。秦因乘胜略定荆地城邑。岁馀,虏荆王负刍,竟平荆地为郡县。因南征百越之君。而王翦子王贲,与李信破定燕、齐地。

秦始皇二十六年,尽并天下,王氏、蒙氏功为多,名施于后世。

秦二世之时,王翦及其子贲皆已死,而又灭蒙氏。陈胜之反秦,秦使王翦之孙王离击赵,围赵王及张耳钜鹿城。或曰:“王离,秦之名将也。今将彊秦之兵,攻新造之赵,举之必矣。”客曰:“不然。夫为将三世者必败。必败者何也?必其所杀伐多矣,其后受其不祥。今王离已三世将矣。”居无何,项羽救赵,击秦军,果虏王离,王离军遂降诸侯。

太史公曰:鄙语云“尺有所短,寸有所长”。白起料敌合变,出奇无穷,声震天下,然不能救患于应侯。王翦为秦将,夷六国,当是时,翦为宿将,始皇师之,然不能辅秦建德,固其根本,偷合取容,以至筊身。及孙王离为项羽所虏,不亦宜乎!彼各有所短也。

白起、王翦,俱善用兵。递为秦将,拔齐破荆。赵任马服,长平遂阬。楚陷李信,霸上卒行。贲、离继出,三代无名。
\end{yuanwen}

\chapter{孟子荀卿列传}

\begin{yuanwen}
太史公曰:余读孟子书,至梁惠王问“何以利吾国”,未尝不废书而叹也。曰:嗟乎,利诚乱之始也!夫子罕言利者,常防其原也。故曰“放于利而行,多怨”。自天子至于庶人,好利之弊何以异哉!

孟轲,驺人也。受业子思之门人。道既通,游事齐宣王,宣王不能用。適梁,梁惠王不果所言,则见以为迂远而阔于事情。当是之时,秦用商君,富国彊兵;楚、魏用吴起,战胜弱敌;齐威王、宣王用孙子、田忌之徒,而诸侯东面朝齐。天下方务于合从连衡,以攻伐为贤,而孟轲乃述唐、虞、三代之德,是以所如者不合。退而与万章之徒序诗书,述仲尼之意,作孟子七篇。其后有驺子之属。

齐有三驺子。其前驺忌,以鼓琴干威王,因及国政,封为成侯而受相印,先孟子。

其次驺衍,后孟子。驺衍睹有国者益淫侈,不能尚德,若大雅整之于身,施及黎庶矣。乃深观阴阳消息而作怪迂之变,终始、大圣之篇十馀万言。其语闳大不经,必先验小物,推而大之,至于无垠。先序今以上至黄帝,学者所共术,大并世盛衰,因载其禨祥度制,推而远之,至天地未生,窈冥不可考而原也。先列中国名山大川,通谷禽兽,水土所殖,物类所珍,因而推之,及海外人之所不能睹。称引天地剖判以来,五德转移,治各有宜,而符应若兹。以为儒者所谓中国者,于天下乃八十一分居其一分耳。中国名曰赤县神州。赤县神州内自有九州,禹之序九州是也,不得为州数。中国外如赤县神州者九,乃所谓九州也。于是有裨海环之,人民禽兽莫能相通者,如一区中者,乃为一州。如此者九,乃有大瀛海环其外,天地之际焉。其术皆此类也。然要其归,必止乎仁义节俭,君臣上下六亲之施,始也滥耳。王公大人初见其术,惧然顾化,其后不能行之。

是以驺子重于齐。適梁,惠王郊迎,执宾主之礼。適赵,平原君侧行撇席。如燕,昭王拥彗先驱,请列弟子之座而受业,筑碣石宫,身亲往师之。作主运。其游诸侯见尊礼如此,岂与仲尼菜色陈蔡,孟轲困于齐梁同乎哉!故武王以仁义伐纣而王,伯夷饿不食周粟;卫灵公问陈,而孔子不答;梁惠王谋欲攻赵,孟轲称大王去邠。此岂有意阿世俗苟合而已哉!持方枘欲内圜凿,其能入乎?或曰,伊尹负鼎而勉汤以王,百里奚饭牛车下而缪公用霸,作先合,然后引之大道。驺衍其言虽不轨,傥亦有牛鼎之意乎?

自驺衍与齐之稷下先生,如淳于髡、慎到、环渊、接子、田骈、驺奭之徒,各著书言治乱之事,以干世主,岂可胜道哉!

淳于髡,齐人也。博闻彊记,学无所主。其谏说,慕晏婴之为人也,然而承意观色为务。客有见髡于梁惠王,惠王屏左右,独坐而再见之,终无言也。惠王怪之,以让客曰:“子之称淳于先生,管、晏不及,及见寡人,寡人未有得也。岂寡人不足为言邪?何故哉?”客以谓髡。髡曰:“固也。吾前见王,王志在驱逐;后复见王,王志在音声:吾是以默然。”客具以报王,王大骇,曰:“嗟乎,淳于先生诚圣人也!前淳于先生之来,人有献善马者,寡人未及视,会先生至。后先生之来,人有献讴者,未及试,亦会先生来。寡人虽屏人,然私心在彼,有之。”后淳于髡见,壹语连三日三夜无倦。惠王欲以卿相位待之,髡因谢去。于是送以安车驾驷,束帛加璧,黄金槽镒。终身不仕。

慎到,赵人。田骈、接子,齐人。环渊,楚人。皆学黄老道德之术,因发明序其指意。故慎到著十二论,环渊著上下篇,而田骈、接子皆有所论焉。

驺奭者,齐诸驺子,亦颇采驺衍之术以纪文。

于是齐王嘉之,自如淳于髡以下,皆命曰列大夫,为开第康庄之衢,高门大屋,尊宠之。览天下诸侯宾客,言齐能致天下贤士也。

荀卿,赵人。年五十始来游学于齐。驺衍之术迂大而闳辩;奭也文具难施;淳于髡久与处,时有得善言。故齐人颂曰:“谈天衍,雕龙奭,炙毂过髡。”田骈之属皆已死齐襄王时,而荀卿最为老师。齐尚脩列大夫之缺,而荀卿三为祭酒焉。齐人或谗荀卿,荀卿乃適楚,而春申君以为兰陵令。春申君死而荀卿废,因家兰陵。李斯尝为弟子,已而相秦。荀卿嫉浊世之政,亡国乱君相属,不遂大道而营于巫祝,信禨祥,鄙儒小拘,如庄周等又猾稽乱俗,于是推儒、墨、道德之行事兴坏,序列著数万言而卒。因葬兰陵。

而赵亦有公孙龙为坚白同异之辩,剧子之言;魏有李悝,尽地力之教;楚有尸子、长卢;阿之吁子焉。自如孟子至于吁子,世多有其书,故不论其传云。

盖墨翟,宋之大夫,善守御,为节用。或曰并孔子时,或曰在其后。

六国之末,战胜相雄。轲游齐、魏,其说不通。退而著述,称吾道穷。兰陵事楚,驺衍谈空。康庄虽列,莫见收功。
\end{yuanwen}

\part{卷七十五}

\chapter{孟尝君列传第十五}

\begin{yuanwen}
孟尝君名文,姓田氏。文之父曰靖郭君田婴。田婴者,齐威王少子而齐宣王庶弟也。田婴自威王时任职用事,与成侯邹忌及田忌将而救韩伐魏。成侯与田忌争宠,成侯卖田忌。田忌惧,袭齐之边邑,不胜,亡走。会威王卒,宣王立,知成侯卖田忌,乃复召田忌以为将。宣王二年\footnote{应为齐威王十五年。《史记》中齐国威、宣、湣三王的年代有多处错误。},田忌与孙膑、田婴俱伐魏,败之马陵,虏魏太子申而杀魏将庞涓。宣王七年,田婴使于韩、魏,韩、魏服于齐。婴与韩昭侯、魏惠王会齐宣王东阿南,盟而去。明年,复与梁惠王会甄。是岁,梁惠王卒\footnote{当年魏君称王改元,《史记》误以为魏惠王死于这一年。}。宣王九年,田婴相齐。齐宣王与魏襄王\footnote{应为齐威王与魏惠王。}会徐州而相王也。楚威王闻之,怒田婴。明年,楚伐败齐师于徐州,而使人逐田婴。田婴使张丑说楚威王,威王乃止。田婴相齐十一年,宣王卒,湣王即位。即位三年,而封田婴于薛。
\end{yuanwen}

孟尝君名文,姓田氏。田文的父亲是靖郭君田婴。田婴,是齐威王的小儿子,也是齐宣王的异母弟。田婴从齐威王在位时开始担任官职处理政事,与成侯邹忌和田忌带兵去救援韩国,攻打魏国。成侯和田忌争相邀宠,成侯出卖田忌。田忌恐惧,袭击齐国的边疆城邑,没有取胜,逃跑了。正赶上齐威王去世,宣王即位,知道成侯出卖田忌,于是重新召回田忌,任用他做将军。齐宣王二年,田忌与孙膑、田婴一起攻打魏国,在马陵打败了敌人,俘获了太子申,并杀死了魏国的将军庞涓。齐宣王七年,田婴出使韩、魏两国,使韩国和魏国归服齐国。田婴与韩昭侯、魏惠王在东阿之南会见了齐宣王,订立盟约后回国。第二年,齐宣王又跟魏惠王在甄城会见。这一年,魏惠王去世。齐宣王九年,田婴在齐国出任丞相。齐宣王与魏襄王在徐州会见,并相互承认王号。楚威王听说这件事以后,对田婴很生气。第二年,楚国在徐州打败了齐国,并派人追击田婴。田婴派张丑去游说楚威王,楚威王才收兵。田婴在齐国任丞相十一年,齐宣王死后,齐湣王即位。齐湣王继位三年,把薛邑封给田婴。

钱穆:「其时孟尝君在齐固已戴震主之威名,天下知有薛,不知有齐矣。」陈仁锡:「太史公作四君传,具见好客意,孟尝则曰『以故倾天下之士』,平原则曰『故争相倾以待士』,信陵则曰『倾平原君客』,春申则曰『招致宾客以相倾夺』。」

\begin{yuanwen}\end{yuanwen}

\begin{yuanwen}\end{yuanwen}

\begin{yuanwen}\end{yuanwen}

\begin{yuanwen}\end{yuanwen}

\begin{yuanwen}\end{yuanwen}

\begin{yuanwen}\end{yuanwen}

\begin{yuanwen}\end{yuanwen}

\begin{yuanwen}\end{yuanwen}

\begin{yuanwen}\end{yuanwen}

\begin{yuanwen}\end{yuanwen}

\begin{yuanwen}\end{yuanwen}

\begin{yuanwen}\end{yuanwen}

\begin{yuanwen}\end{yuanwen}

\begin{yuanwen}\end{yuanwen}

\begin{yuanwen}\end{yuanwen}

\begin{yuanwen}\end{yuanwen}

\begin{yuanwen}\end{yuanwen}

\begin{yuanwen}\end{yuanwen}

\begin{yuanwen}\end{yuanwen}

\begin{yuanwen}\end{yuanwen}

\begin{yuanwen}\end{yuanwen}

\begin{yuanwen}\end{yuanwen}

\begin{yuanwen}\end{yuanwen}

\begin{yuanwen}\end{yuanwen}

\begin{yuanwen}\end{yuanwen}

\begin{yuanwen}\end{yuanwen}

\begin{yuanwen}\end{yuanwen}

\begin{yuanwen}\end{yuanwen}

\begin{yuanwen}\end{yuanwen}

\begin{yuanwen}\end{yuanwen}

\begin{yuanwen}\end{yuanwen}

\begin{yuanwen}\end{yuanwen}

\begin{yuanwen}\end{yuanwen}

\begin{yuanwen}\end{yuanwen}

\begin{yuanwen}\end{yuanwen}

\begin{yuanwen}\end{yuanwen}

\begin{yuanwen}\end{yuanwen}

\begin{yuanwen}\end{yuanwen}

\begin{yuanwen}\end{yuanwen}

\begin{yuanwen}\end{yuanwen}

\begin{yuanwen}\end{yuanwen}

\begin{yuanwen}\end{yuanwen}

\begin{yuanwen}\end{yuanwen}

\begin{yuanwen}\end{yuanwen}

\begin{yuanwen}
初,田婴有子四十馀人。其贱妾有子名文,文以五月五日生。婴告其母曰:“勿举也。”其母窃举生之。及长,其母因兄弟而见其子文于田婴。田婴怒其母曰:“吾令若去此子,而敢生之,何也?”文顿首,因曰:“君所以不举五月子者,何故?”婴曰:“五月子者,长与户齐,将不利其父母。”文曰:“人生受命于天乎?将受命于户邪?”婴默然。文曰:“必受命于天,君何忧焉。必受命于户,则可高其户耳,谁能至者!”婴曰:“子休矣。”

久之,文承间问其父婴曰:“子之子为何?”曰:“为孙。”“孙之孙为何?”曰:“为玄孙。”“玄孙之孙为何?”曰:“不能知也。”文曰:“君用事相齐,至今三王矣,齐不加广而君私家富累万金,门下不见一贤者。文闻将门必有将,相门必有相。今君后宫蹈绮縠而士不得褐,仆妾馀粱肉而士不厌糟。今君又尚厚积馀藏,欲以遗所不知何人,而忘公家之事日损,文窃怪之。”于是婴乃礼文,使主家待宾客。宾客日进,名声闻于诸侯。诸侯皆使人请薛公田婴以文为太子,婴许之。婴卒,谥为靖郭君。而文果代立于薛,是为孟尝君。

孟尝君在薛,招致诸侯宾客及亡人有罪者,皆归孟尝君。孟尝君舍业厚遇之,以故倾天下之士。食客数千人,无贵贱一与文等。孟尝君待客坐语,而屏风后常有侍史,主记君所与客语,问亲戚居处。客去,孟尝君已使使存问,献遗其亲戚。孟尝君曾待客夜食,有一人蔽火光。客怒,以饭不等,辍食辞去。孟尝君起,自持其饭比之。客惭,自刭。士以此多归孟尝君。孟尝君客无所择,皆善遇之。人人各自以为孟尝君亲己。

秦昭王闻其贤,乃先使泾阳君为质于齐,以求见孟尝君。孟尝君将入秦,宾客莫欲其行,谏,不听。苏代谓曰:“今旦代从外来,见木禺人与土禺人相与语。木禺人曰:‘天雨,子将败矣。’土禺人曰:‘我生于土,败则归土。今天雨,流子而行,未知所止息也。’今秦,虎狼之国也,而君欲往,如有不得还,君得无为土禺人所笑乎?”孟尝君乃止。

齐湣王二十五年,复卒使孟尝君入秦,昭王即以孟尝君为秦相。人或说秦昭王曰:“孟尝君贤,而又齐族也,今相秦,必先齐而后秦,秦其危矣。”于是秦昭王乃止。囚孟尝君,谋欲杀之。孟尝君使人抵昭王幸姬求解。幸姬曰:“妾原得君狐白裘。”此时孟尝君有一狐白裘,直千金,天下无双,入秦献之昭王,更无他裘。孟尝君患之,遍问客,莫能对。最下坐有能为狗盗者,曰:“臣能得狐白裘。”乃夜为狗,以入秦宫臧中,取所献狐白裘至,以献秦王幸姬。幸姬为言昭王,昭王释孟尝君。孟尝君得出,即驰去,更封传,变名姓以出关。夜半至函谷关。秦昭王后悔出孟尝君,求之已去,即使人驰传逐之。孟尝君至关,关法鸡鸣而出客,孟尝君恐追至,客之居下坐者有能为鸡鸣,而鸡齐鸣,遂发传出。出如食顷,秦追果至关,已后孟尝君出,乃还。始孟尝君列此二人于宾客,宾客尽羞之,及孟尝君有秦难,卒此二人拔之。自是之后,客皆服。

孟尝君过赵,赵平原君客之。赵人闻孟尝君贤,出观之,皆笑曰:“始以薛公为魁然也,今视之,乃眇小丈夫耳。”孟尝君闻之,怒。客与俱者下,斫击杀数百人,遂灭一县以去。

齐湣王不自得,以其遣孟尝君。孟尝君至,则以为齐相,任政。

孟尝君怨秦,将以齐为韩、魏攻楚,因与韩、魏攻秦,而借兵食于西周。苏代为西周谓曰:“君以齐为韩、魏攻楚九年,取宛、叶以北以彊韩、魏,今复攻秦以益之。韩、魏南无楚忧,西无秦患,则齐危矣。韩、魏必轻齐畏秦,臣为君危之。君不如令敝邑深合于秦,而君无攻,又无借兵食。君临函谷而无攻,令敝邑以君之情谓秦昭王曰‘薛公必不破秦以彊韩、魏。其攻秦也,欲王之令楚王割东国以与齐,而秦出楚怀王以为和’。君令敝邑以此惠秦,秦得无破而以东国自免也,秦必欲之。楚王得出,必德齐。齐得东国益彊,而薛世世无患矣。秦不大弱,而处三晋之西,三晋必重齐。”薛公曰:“善。”因令韩、魏贺秦,使三国无攻,而不借兵食于西周矣。是时,楚怀王入秦,秦留之,故欲必出之。秦不果出楚怀王。

孟尝君相齐,其舍人魏子为孟尝君收邑入,三反而不致一入。孟尝君问之,对曰:“有贤者,窃假与之,以故不致入。”孟尝君怒而退魏子。居数年,人或毁孟尝君于齐湣王曰:“孟尝君将为乱。”及田甲劫湣王,湣王意疑孟尝君,孟尝君乃奔。魏子所与粟贤者闻之,乃上书言孟尝君不作乱,请以身为盟,遂自刭宫门以明孟尝君。湣王乃惊,而踪迹验问,孟尝君果无反谋,乃复召孟尝君。孟尝君因谢病,归老于薛。湣王许之。

其后,秦亡将吕礼相齐,欲困苏代。代乃谓孟尝君曰:“周最于齐,至厚也,而齐王逐之,而听亲弗相吕礼者,欲取秦也。齐、秦合,则亲弗与吕礼重矣。有用,齐、秦必轻君。君不如急北兵,趋赵以和秦、魏,收周最以厚行,且反齐王之信,又禁天下之变。齐无秦,则天下集齐,亲弗必走,则齐王孰与为其国也!”于是孟尝君从其计,而吕礼嫉害于孟尝君。

孟尝君惧,乃遗秦相穰侯魏厓书曰:“吾闻秦欲以吕礼收齐,齐,天下之彊国也,子必轻矣。齐秦相取以临三晋,吕礼必并相矣,是子通齐以重吕礼也。若齐免于天下之兵,其雠子必深矣。子不如劝秦王伐齐。齐破,吾请以所得封子。齐破,秦畏晋之彊,秦必重子以取晋。晋国敝于齐而畏秦,晋必重子以取秦。是子破齐以为功,挟晋以为重;是子破齐定封,秦、晋交重子。若齐不破,吕礼复用,子必大穷。”于是穰侯言于秦昭王伐齐,而吕礼亡。

后齐湣王灭宋,益骄,欲去孟尝君。孟尝君恐,乃如魏。魏昭王以为相,西合于秦、赵,与燕共伐破齐。齐湣王亡在莒,遂死焉。齐襄王立,而孟尝君中立于诸侯,无所属。齐襄王新立,畏孟尝君,与连和,复亲薛公。文卒,谥为孟尝君。诸子争立,而齐魏共灭薛。孟尝绝嗣无后也。

初,冯驩闻孟尝君好客,蹑蹻而见之。孟尝君曰;“先生远辱,何以教文也?”冯驩曰:“闻君好士,以贫身归于君。”孟尝君置传舍十日,孟尝君问传舍长曰:“客何所为?”答曰:“冯先生甚贫,犹有一剑耳,又蒯缑。弹其剑而歌曰‘长铗归来乎,食无鱼’。”孟尝君迁之幸舍,食有鱼矣。五日,又问传舍长。答曰:“客复弹剑而歌曰‘长铗归来乎,出无舆’。”孟尝君迁之代舍,出入乘舆车矣。五日,孟尝君复问传舍长。舍长答曰:“先生又尝弹剑而歌曰‘长铗归来乎,无以为家’。”孟尝君不悦。

居期年,冯驩无所言。孟尝君时相齐,封万户于薛。其食客三千人。邑入不足以奉客,使人出钱于薛。岁馀不入,贷钱者多不能与其息,客奉将不给。孟尝君忧之,问左右:“何人可使收债于薛者?”传舍长曰:“代舍客冯公形容状貌甚辩,长者,无他伎能,宜可令收债。”孟尝君乃进冯驩而请之曰:“宾客不知文不肖,幸临文者三千馀人,邑入不足以奉宾客,故出息钱于薛。薛岁不入,民颇不与其息。今客食恐不给,原先生责之。”冯驩曰;“诺。”辞行,至薛,召取孟尝君钱者皆会,得息钱十万。乃多酿酒,买肥牛,召诸取钱者,能与息者皆来,不能与息者亦来,皆持取钱之券书合之。齐为会,日杀牛置酒。酒酣,乃持券如前合之,能与息者,与为期;贫不能与息者,取其券而烧之。曰:“孟尝君所以贷钱者,为民之无者以为本业也;所以求息者,为无以奉客也。今富给者以要期,贫穷者燔券书以捐之。诸君彊饮食。有君如此,岂可负哉!”坐者皆起,再拜。

孟尝君闻冯驩烧券书,怒而使使召驩。驩至,孟尝君曰:“文食客三千人,故贷钱于薛。文奉邑少,而民尚多不以时与其息,客食恐不足,故请先生收责之。闻先生得钱,即以多具牛酒而烧券书,何?”冯驩曰:“然。不多具牛酒即不能毕会,无以知其有馀不足。有馀者,为要期。不足者,虽守而责之十年,息愈多,急,即以逃亡自捐之。若急,终无以偿,上则为君好利不爱士民,下则有离上抵负之名,非所以厉士民彰君声也。焚无用虚债之券,捐不可得之虚计,令薛民亲君而彰君之善声也,君有何疑焉!”孟尝君乃拊手而谢之。

齐王惑于秦、楚之毁,以为孟尝君名高其主而擅齐国之权,遂废孟尝君。诸客见孟尝君废,皆去。冯驩曰:“借臣车一乘,可以入秦者,必令君重于国而奉邑益广,可乎?”孟尝君乃约车币而遣之。冯驩乃西说秦王曰:“天下之游士冯轼结靷西入秦者,无不欲彊秦而弱齐;冯轼结靷东入齐者,无不欲彊齐而弱秦。此雄雌之国也,势不两立为雄,雄者得天下矣。”秦王跽而问之曰:“何以使秦无为雌而可?”冯驩曰:“王亦知齐之废孟尝君乎?”秦王曰:“闻之。”冯驩曰:“使齐重于天下者,孟尝君也。今齐王以毁废之,其心怨,必背齐;背齐入秦,则齐国之情,人事之诚,尽委之秦,齐地可得也,岂直为雄也!君急使使载币阴迎孟尝君,不可失时也。如有齐觉悟,复用孟尝君,则雌雄之所在未可知也。”秦王大悦,乃遣车十乘黄金百镒以迎孟尝君。冯驩辞以先行,至齐,说齐王曰:“天下之游士冯轼结靷东入齐者,无不欲彊齐而弱秦者;冯轼结靷西入秦者,无不欲彊秦而弱齐者。夫秦齐雄雌之国,秦彊则齐弱矣,此势不两雄。今臣窃闻秦遣使车十乘载黄金百镒以迎孟尝君。孟尝君不西则已,西入相秦则天下归之,秦为雄而齐为雌,雌则临淄、即墨危矣。王何不先秦使之未到,复孟尝君,而益与之邑以谢之?孟尝君必喜而受之。秦虽彊国,岂可以请人相而迎之哉!折秦之谋,而绝其霸彊之略。”齐王曰:“善。”乃使人至境候秦使。秦使车適入齐境,使还驰告之,王召孟尝君而复其相位,而与其故邑之地,又益以千户。秦之使者闻孟尝君复相齐,还车而去矣。

自齐王毁废孟尝君,诸客皆去。后召而复之,冯驩迎之。未到,孟尝君太息叹曰:“文常好客,遇客无所敢失,食客三千有馀人,先生所知也。客见文一日废,皆背文而去,莫顾文者。今赖先生得复其位,客亦有何面目复见文乎?如复见文者,必唾其面而大辱之。”冯驩结辔下拜。孟尝君下车接之,曰:“先生为客谢乎?”冯驩曰:“非为客谢也,为君之言失。夫物有必至,事有固然,君知之乎?”孟尝君曰:“愚不知所谓也。”曰:“生者必有死,物之必至也;富贵多士,贫贱寡友,事之固然也。君独不见夫趣市者乎?明旦,侧肩争门而入;日暮之后,过市朝者掉臂而不顾。非好朝而恶暮,所期物忘其中。今君失位,宾客皆去,不足以怨士而徒绝宾客之路。原君遇客如故。”孟尝君再拜曰:“敬从命矣。闻先生之言,敢不奉教焉。”

太史公曰:吾尝过薛,其俗闾里率多暴桀子弟,与邹、鲁殊。问其故,曰:“孟尝君招致天下任侠,奸人入薛中盖六万馀家矣。”世之传孟尝君好客自喜,名不虚矣。

靖郭之子,威王之孙。既彊其国,实高其门。好客喜士,见重平原。鸡鸣狗盗,魏子、冯暖。如何承睫,薛县徒存!
\end{yuanwen}

\part{卷七十六}
\chapter{平原君虞卿列传第十六}

杨慎:“言在浊世为佳公子,清世则否矣,褒贬在言外,所以称为雄深。平原之人未睹,大体可断。『虞卿非穷愁不能著书以自见于后世』,韩子《柳子厚墓志》用此意。”

\begin{yuanwen}
平原君赵胜者,赵之诸公子也。诸子中胜最贤,喜宾客,宾客盖至者数千人。平原君相赵惠文王及孝成王,三去相,三复位,封于东武城。
\end{yuanwen}

平原君赵胜这个人,是赵国王室的一位公子。在众多的公子当中,以赵胜最为贤能。他喜欢延揽宾客,前来投奔他的宾客大约有几千人。平原君辅佐了赵惠文王和赵孝成王,曾三次被免去相位,又三次恢复相位,他的封地在东武城。

\begin{yuanwen}
平原君家楼临\footnote{俯视。}民家。民家有躄\footnote{bì,跛,腿瘸。}者,槃散\footnote{通“蹒跚”,走路不稳。}行汲。平原君美人居楼上,临见,大笑之。明日,躄者至平原君门,请曰:“臣闻君之喜士,士不远千里而至者,以君能贵士而贱妾也。臣不幸有罢癃之病,而君之后宫临而笑臣,臣原得笑臣者头。”平原君笑应曰:“诺。”躄者去,平原君笑曰:“观此竖子,乃欲以一笑之故杀吾美人,不亦甚乎!”终不杀。居岁馀,宾客门下舍人稍稍引去者过半。平原君怪之,曰:“胜所以待诸君者未尝敢失礼,而去者何多也?”门下一人前对曰:“以君之不杀笑躄者,以君为爱色而贱士,士即去耳。”于是平原君乃斩笑躄者美人头,自造门进躄者,因谢焉。其后门下乃复稍稍来。是时齐有孟尝,魏有信陵,楚有春申,故争相倾以待士。

平原君美人居楼上,临见,大笑之。明日,躄者至平原君门,请曰:“臣闻君之喜士,士不远千里而至者,以君能贵士而贱妾也。臣不幸有罢癃【罢癃:指残疾。罢,通“疲”。】 之病,而君之后宫临而笑臣,臣愿得笑臣者头。”平原君笑应曰:“诺。”躄者去,平原君笑曰:“观此竖子【竖子:小子,对人的鄙称。】 ,乃欲以一笑之故杀吾美人,不亦甚乎!”终不杀。居岁余,宾客门下舍人稍稍引去者过半。平原君怪之,曰:“胜所以待诸君者未尝敢失礼,而去者何多也?”门下一人前对曰:“以君之不杀笑躄者,以君为爱色而贱士,士即去耳。”于是平原君乃斩笑躄者美人头,自造门进躄者,因谢【谢:道歉。】 焉。其后门下乃复稍稍来。是时齐有孟尝,魏有信陵,楚有春申,故争相倾【倾:倒,使之归己。】 以待士。
\end{yuanwen}

\begin{yuanwen}
秦之围邯郸,赵使平原君求救,合从于楚,约与食客门下有勇力文武备具者二十人偕。平原君曰:“使文能取胜,则善矣。文不能取胜,则歃血于华屋之下,必得定从而还。士不外索,取于食客门下足矣。”得十九人,馀无可取者,无以满二十人。门下有毛遂者,前,自赞于平原君曰:“遂闻君将合从于楚,约与食客门下二十人偕,不外索。今少一人,原君即以遂备员而行矣。”平原君曰:“先生处胜之门下几年于此矣?”毛遂曰:“三年于此矣。”平原君曰:“夫贤士之处世也,譬若锥之处囊中,其末立见。今先生处胜之门下三年于此矣,左右未有所称诵,胜未有所闻,是先生无所有也。先生不能,先生留。”毛遂曰:“臣乃今日请处囊中耳。使遂蚤得处囊中,乃颖脱而出,非特其末见而已。”平原君竟与毛遂偕。十九人相与目笑之而未废也。

毛遂比至楚,与十九人论议,十九人皆服。平原君与楚合从,言其利害,日出而言之,日中不决。十九人谓毛遂曰:“先生上。”毛遂按剑历阶而上,谓平原君曰:“从之利害,两言而决耳。今日出而言从,日中不决,何也?”楚王谓平原君曰:“客何为者也?”平原君曰:“是胜之舍人也。”楚王叱曰:“胡不下!吾乃与而君言,汝何为者也!”毛遂按剑而前曰:“王之所以叱遂者,以楚国之众也。今十步之内,王不得恃楚国之众也,王之命县于遂手。吾君在前,叱者何也?且遂闻汤以七十里之地王天下,文王以百里之壤而臣诸侯,岂其士卒众多哉,诚能据其势而奋其威。今楚地方五千里,持戟百万,此霸王之资也。以楚之彊,天下弗能当。白起,小竖子耳,率数万之众,兴师以与楚战,一战而举鄢郢,再战而烧夷陵,三战而辱王之先人。此百世之怨而赵之所羞,而王弗知恶焉。合从者为楚,非为赵也。吾君在前,叱者何也?”楚王曰:“唯唯,诚若先生之言,谨奉社稷而以从。”毛遂曰:“从定乎?”楚王曰:“定矣。”毛遂谓楚王之左右曰:“取鸡狗马之血来。”毛遂奉铜槃而跪进之楚王曰:“王当歃血而定从,次者吾君,次者遂。”遂定从于殿上。毛遂左手持槃血而右手招十九人曰:“公相与歃此血于堂下。公等录录,所谓因人成事者也。”

平原君已定从而归,归至于赵,曰:“胜不敢复相士。胜相士多者千人,寡者百数,自以为不失天下之士,今乃于毛先生而失之也。毛先生一至楚,而使赵重于九鼎大吕。毛先生以三寸之舌,彊于百万之师。胜不敢复相士。”遂以为上客。

平原君既返赵,楚使春申君将兵赴救赵,魏信陵君亦矫夺晋鄙军往救赵,皆未至。秦急围邯郸,邯郸急,且降,平原君甚患之。邯郸传舍吏子李同说平原君曰:“君不忧赵亡邪?”平原君曰:“赵亡则胜为虏,何为不忧乎?”李同曰:“邯郸之民,炊骨易子而食,可谓急矣,而君之后宫以百数,婢妾被绮縠,馀粱肉,而民褐衣不完,糟糠不厌。民困兵尽,或剡木为矛矢,而君器物锺磬自若。使秦破赵,君安得有此?使赵得全,君何患无有?今君诚能令夫人以下编于士卒之间,分功而作,家之所有尽散以飨士,士方其危苦之时,易德耳。”于是平原君从之,得敢死之士三千人。李同遂与三千人赴秦军,秦军为之卻三十里。亦会楚、魏救至,秦兵遂罢,邯郸复存。李同战死,封其父为李侯。

虞卿欲以信陵君之存邯郸为平原君请封。公孙龙闻之,夜驾见平原君曰:“龙闻虞卿欲以信陵君之存邯郸为君请封,有之乎?”平原君曰:“然。”龙曰:“此甚不可。且王举君而相赵者,非以君之智能为赵国无有也。割东武城而封君者,非以君为有功也,而以国人无勋,乃以君为亲戚故也。君受相印不辞无能,割地不言无功者,亦自以为亲戚故也。今信陵君存邯郸而请封,是亲戚受城而国人计功也。此甚不可。且虞卿操其两权,事成,操右券以责;事不成,以虚名德君。君必勿听也。”平原君遂不听虞卿。

平原君以赵孝成王十五年卒。子孙代,后竟与赵俱亡。

平原君厚待公孙龙。公孙龙善为坚白之辩,及邹衍过赵言至道,乃绌公孙龙。

虞卿者,游说之士也。蹑蹻檐簦说赵孝成王。一见,赐黄金百镒,白璧一双;再见,为赵上卿,故号为虞卿。

秦赵战于长平,赵不胜,亡一都尉。赵王召楼昌与虞卿曰:“军战不胜,尉复死,寡人使束甲而趋之,何如?”楼昌曰:“无益也,不如发重使为媾。”虞卿曰:“昌言媾者,以为不媾军必破也。而制媾者在秦。且王之论秦也,欲破赵之军乎,不邪?”王曰:“秦不遗馀力矣,必且欲破赵军。”虞卿曰:“王听臣,发使出重宝以附楚、魏,楚、魏欲得王之重宝,必内吾使。赵使入楚、魏,秦必疑天下之合从,且必恐。如此,则媾乃可为也。”赵王不听,与平阳君为媾,发郑硃入秦。秦内之。赵王召虞卿曰:“寡人使平阳君为媾于秦,秦已内郑硃矣,卿之为奚如?”虞卿对曰:“王不得媾,军必破矣。天下贺战者皆在秦矣。郑硃,贵人也,入秦,秦王与应侯必显重以示天下。楚、魏以赵为媾,必不救王。秦知天下不救王,则媾不可得成也。”应侯果显郑硃以示天下贺战胜者,终不肯媾。长平大败,遂围邯郸,为天下笑。

秦既解邯郸围,而赵王入朝,使赵郝约事于秦,割六县而媾。虞卿谓赵王曰:“秦之攻王也,倦而归乎?王以其力尚能进,爱王而弗攻乎?”王曰:“秦之攻我也,不遗馀力矣,必以倦而归也。”虞卿曰:“秦以其力攻其所不能取,倦而归,王又以其力之所不能取以送之,是助秦自攻也。来年秦复攻王,王无救矣。”王以虞卿之言赵郝。赵郝曰:“虞卿诚能尽秦力之所至乎?诚知秦力之所不能进,此弹丸之地弗予,令秦来年复攻王,王得无割其内而媾乎?”王曰:“请听子割,子能必使来年秦之不复攻我乎?”赵郝对曰:“此非臣之所敢任也。他日三晋之交于秦,相善也。今秦善韩、魏而攻王,王之所以事秦必不如韩、魏也。今臣为足下解负亲之攻,开关通币,齐交韩、魏,至来年而王独取攻于秦,此王之所以事秦必在韩、魏之后也。此非臣之所敢任也。”

王以告虞卿。虞卿对曰:“郝言‘不媾,来年秦复攻王,王得无割其内而媾乎’。今媾,郝又以不能必秦之不复攻也。今虽割六城,何益!来年复攻,又割其力之所不能取而媾,此自尽之术也,不如无媾。秦虽善攻,不能取六县;赵虽不能守,终不失六城。秦倦而归,兵必罢。我以六城收天下以攻罢秦,是我失之于天下而取偿于秦也。吾国尚利,孰与坐而割地,自弱以彊秦哉?今郝曰‘秦善韩、魏而攻赵者,必王之事秦不如韩、魏也’,是使王岁以六城事秦也,即坐而城尽。来年秦复求割地,王将与之乎?弗与,是弃前功而挑秦祸也;与之,则无地而给之。语曰‘彊者善攻,弱者不能守’。今坐而听秦,秦兵不弊而多得地,是彊秦而弱赵也。以益彊之秦而割愈弱之赵,其计故不止矣。且王之地有尽而秦之求无已,以有尽之地而给无已之求,其势必无赵矣。”

赵王计未定,楼缓从秦来,赵王与楼缓计之,曰:“予秦地如毋予,孰吉?”缓辞让曰:“此非臣之所能知也。”王曰:“虽然,试言公之私。”楼缓对曰:“王亦闻夫公甫文伯母乎?公甫文伯仕于鲁,病死,女子为自杀于房中者二人。其母闻之,弗哭也。其相室曰:‘焉有子死而弗哭者乎?’其母曰:‘孔子,贤人也,逐于鲁,而是人不随也。今死而妇人为之自杀者二人,若是者必其于长者薄而于妇人厚也。’故从母言之,是为贤母;从妻言之,是必不免为妒妻。故其言一也,言者异则人心变矣。今臣新从秦来而言勿予,则非计也;言予之,恐王以臣为为秦也:故不敢对。使臣得为大王计,不如予之。”王曰:“诺。”

虞卿闻之,入见王曰:“此饰说也,王蜰勿予!”楼缓闻之,往见王。王又以虞卿之言告楼缓。楼缓对曰:“不然。虞卿得其一,不得其二。夫秦赵构难而天下皆说,何也?曰‘吾且因彊而乘弱矣’。今赵兵困于秦,天下之贺战胜者则必尽在于秦矣。故不如亟割地为和,以疑天下而慰秦之心。不然,天下将因秦之怒,乘赵之弊,瓜分之。赵且亡,何秦之图乎?故曰虞卿得其一,不得其二。原王以此决之,勿复计也。”

虞卿闻之,往见王曰:“危哉楼子之所以为秦者,是愈疑天下,而何慰秦之心哉?独不言其示天下弱乎?且臣言勿予者,非固勿予而已也。秦索六城于王,而王以六城赂齐。齐,秦之深雠也,得王之六城,并力西击秦,齐之听王,不待辞之毕也。则是王失之于齐而取偿于秦也。而齐、赵之深雠可以报矣,而示天下有能为也。王以此发声,兵未窥于境,臣见秦之重赂至赵而反媾于王也。从秦为媾,韩、魏闻之,必尽重王;重王,必出重宝以先于王。则是王一举而结三国之亲,而与秦易道也。”赵王曰:“善。”则使虞卿东见齐王,与之谋秦。虞卿未返,秦使者已在赵矣。楼缓闻之,亡去。赵于是封虞卿以一城。

居顷之,而魏请为从。赵孝成王召虞卿谋。过平原君,平原君曰:“原卿之论从也。”虞卿入见王。王曰:“魏请为从。”对曰:“魏过。”王曰:“寡人固未之许。”对曰:“王过。”王曰:“魏请从,卿曰魏过,寡人未之许,又曰寡人过,然则从终不可乎?”对曰:“臣闻小国之与大国从事也,有利则大国受其福,有败则小国受其祸。今魏以小国请其祸,而王以大国辞其福,臣故曰王过,魏亦过。窃以为从便。”王曰:“善。”乃合魏为从。

虞卿既以魏齐之故,不重万户侯卿相之印,与魏齐间行,卒去赵,困于梁。魏齐已死,不得意,乃著书,上采春秋,下观近世,曰节义、称号、揣摩、政谋,凡八篇。以刺讥国家得失,世传之曰虞氏春秋。

太史公曰:平原君,翩翩浊世之佳公子也,然未睹大体。鄙语曰“利令智昏”,平原君贪冯亭邪说,使赵陷长平兵四十馀万众,邯郸几亡。虞卿料事揣情,为赵画策,何其工也!及不忍魏齐,卒困于大梁,庸夫且知其不可,况贤人乎?然虞卿非穷愁,亦不能著书以自见于后世云。

翩翩公子,天下奇器。笑姬从戮,义士增气。兵解李同,盟定毛遂。虞卿蹑蹻,受赏料事。及困魏齐,著书见意。
\end{yuanwen}

\chapter{魏公子列传}

以魏公子信陵君窃符救赵一事为中心,歌颂了信陵君的礼贤下士和侯嬴诸人的士为知己者死。作者对他们的活动表示了高度的钦敬,对于信陵君这样一个一切以国家利益为目的的人物最后竟在遭毁谤与受怀疑的境遇下自戕于醇酒妇人的悲惨结局,寄寓了极大的感慨和同情。在战国时代所有以养士闻名的人物里,魏公子的人品最高,在司马迁歌颂的士为知己者死的游士中,侯嬴的人品最高。他们都摆脱了个人的一般利益、一般恩怨,而是谘诹善道,以义相扶,共同保卫国家,以维护正义为终极归宿。魏公子与侯赢之间的这种关系是司马迁理想的君臣关系,是司马迁的一项重要的社会理想。

需要注意的是,侯赢为信陵君策划窃符夺晋鄙兵事,不见于《战国策》,亦不见于先秦的其他载籍,可能是大梁长老之逸闻,是司马迁首次将它写入史册。

\begin{yuanwen}
魏公子无忌者,魏昭王少子而魏安釐王异母弟也。昭王薨,安釐王即位,封公子为信陵君。是时范睢亡魏相秦\footnote{text},以怨魏齐故,秦兵围大梁,破魏华阳下军,走芒卯\footnote{text}。魏王及公子患之。
\end{yuanwen}

\begin{yuanwen}
公子为人仁而下士,士无贤不肖皆谦而礼交之,不敢以其富贵骄士。士以此方数千里争往归之,致食客三千人。当是时,诸侯以公子贤,多客,不敢加兵谋魏十馀年\footnote{text}。
\end{yuanwen}

\begin{yuanwen}
公子与魏王博\footnote{text},而北境传举烽,言:“赵寇至,且入界”。魏王释博,欲召大臣谋。公子止王曰:“赵王田猎耳,非为寇也。”复博如故。

王恐,心不在博。居顷,复从北方来传言曰:“赵王猎耳,非为寇也。”

魏王大惊,曰:“公子何以知之?”

公子曰:“臣之客有能深得赵王阴事者,赵王所为,客辄以报臣,臣以此知之。”

是后魏王畏公子之贤能,不敢任公子以国政。
\end{yuanwen}

\begin{yuanwen}
魏有隐士曰侯嬴,年七十,家贫,为大梁夷门监者\footnote{text}。公子闻之,往请,欲厚遗之。不肯受,曰:“臣修身絜行数十年\footnote{text},终不以监门困故而受公子财\footnote{text}。”

公子于是乃置酒大会宾客。坐定,公子从车骑,虚左\footnote{text},自迎夷门侯生。侯生摄敝衣冠\footnote{text},直上载公子上坐,不让,欲以观公子。公子执辔愈恭。

侯生又谓公子曰:“臣有客在市屠中,原枉车骑过之\footnote{text}。”

公子引车入巿,侯生下见其客朱亥,俾倪\footnote{text},故久立与其客语,微察公子。公子颜色愈和。当是时,魏将相宗室宾客满堂,待公子举酒。巿人皆观公子执辔。从骑皆窃骂侯生。侯生视公子色终不变,乃谢客就车\footnote{text}。至家,公子引侯生坐上坐,遍赞宾客\footnote{text},宾客皆惊。

酒酣,公子起,为寿侯生前。侯生因谓公子曰:“今日嬴之为公子亦足矣。嬴乃夷门抱关者也\footnote{text},而公子亲枉车骑,自迎嬴于众人广坐之中。不宜有所过\footnote{text},今公子故过之。然嬴欲就公子之名,故久立公子车骑巿中。过客以观公子,公子愈恭。巿人皆以嬴为小人,而以公子为长者能下士也\footnote{text}。”于是罢酒,侯生遂为上客。
\end{yuanwen}

\begin{yuanwen}
侯生谓公子曰:“臣所过屠者朱亥,此子贤者,世莫能知,故隐屠间耳。”

公子往数请之,朱亥故不复谢,公子怪之。
\end{yuanwen}

\begin{yuanwen}
魏安釐王二十年,秦昭王已破赵长平军\footnote{text},又进兵围邯郸。公子姊为赵惠文王弟平原君夫人,数遗魏王及公子书,请救于魏。魏王使将军晋鄙将十万众救赵。秦王使使者告魏王曰:“吾攻赵旦暮且下,而诸侯敢救者,已拔赵,必移兵先击之。”

魏王恐,使人止晋鄙,留军壁邺\footnote{text},名为救赵,实持两端以观望。

平原君使者冠盖相属于魏\footnote{text},让魏公子曰:“胜所以自附为婚姻者,以公子之高义,为能急人之困。今邯郸旦暮降秦而魏救不至,安在公子能急人之困也!且公子纵轻胜,弃之降秦,独不怜公子姊邪?”

公子患之,数请魏王,及宾客辩士说王万端。魏王畏秦,终不听公子。公子自度终不能得之于王,计不独生而令赵亡,乃请宾客,约车骑百馀乘\footnote{text},欲以客往赴秦军,与赵俱死。
\end{yuanwen}

\begin{yuanwen}
行过夷门,见侯生,具告所以欲死秦军状。辞决而行,侯生曰:“公子勉之矣,老臣不能从。”

公子行数里,心不快,曰:“吾所以待侯生者备矣,天下莫不闻。今吾且死,而侯生曾无一言半辞送我,我岂有所失哉?”

复引车还,问侯生。侯生笑曰:“臣固知公子之还也\footnote{text}。”

曰:“公子喜士,名闻天下。今有难,无他端而欲赴秦军,譬若以肉投馁虎,何功之有哉?尚安事客?然公子遇臣厚,公子往而臣不送,以是知公子恨之复返也。”

公子再拜,因问。侯生乃屏人间语\footnote{text},曰:“嬴闻晋鄙之兵符常在王卧内\footnote{text},而如姬最幸,出入王卧内,力能窃之。嬴闻如姬父为人所杀,如姬资之三年\footnote{text},自王以下欲求报其父仇,莫能得。如姬为公子泣,公子使客斩其仇头,敬进如姬。如姬之欲为公子死,无所辞,顾未有路耳。公子诚一开口请如姬,如姬必许诺,则得虎符夺晋鄙军,北救赵而西却秦,此五霸之伐也\footnote{text}。”

公子从其计,请如姬。如姬果盗晋鄙兵符与公子\footnote{text}。
\end{yuanwen}

\begin{yuanwen}
公子行,侯生曰:“将在外,主令有所不受,以便国家。公子即合符,而晋鄙不授公子兵而复请之,事必危矣。臣客屠者朱亥可与俱,此人力士。晋鄙听,大善;不听,可使击之。”

于是公子泣。侯生曰:“公子畏死邪?何泣也?”

公子曰:“晋鄙嚄唶宿将\footnote{text},往恐不听,必当杀之,是以泣耳,岂畏死哉?”

于是公子请朱亥。朱亥笑曰:“臣乃市井鼓刀屠者,而公子亲数存之\footnote{text},所以不报谢者,以为小礼无所用。今公子有急,此乃臣效命之秋也。”

遂与公子俱。公子过谢侯生。侯生曰:“臣宜从,老不能。请数公子行日,以至晋鄙军之日,北乡自刭,以送公子\footnote{text}。”

公子遂行。
\end{yuanwen}

\begin{yuanwen}
至邺,矫魏王令代晋鄙。晋鄙合符,疑之,举手视公子曰\footnote{text}:“今吾拥十万之众,屯于境上,国之重任,今单车来代之\footnote{text},何如哉?”

欲无听。朱亥袖四十斤铁椎,椎杀晋鄙,公子遂将晋鄙军。勒兵\footnote{text},下令军中曰:“父子俱在军中,父归;兄弟俱在军中,兄归;独子无兄弟,归养。”

得选兵八万人\footnote{text},进兵击秦军。秦军解去,遂救邯郸,存赵。

赵王及平原君自迎公子于界,平原君负䪍矢为公子先引\footnote{text}。赵王再拜曰:“自古贤人未有及公子者也。”

当此之时,平原君不敢自比于人。公子与侯生决,至军,侯生果北乡自刭。
\end{yuanwen}

\begin{yuanwen}
魏王怒公子之盗其兵符,矫杀晋鄙,公子亦自知也。已却秦存赵,使将将其军归魏,而公子独与客留赵。赵孝成王德公子之矫夺晋鄙兵而存赵\footnote{text},乃与平原君计,以五城封公子。公子闻之,意骄矜而有自功之色。

客有说公子曰:“物有不可忘,或有不可不忘。夫人有德于公子,公子不可忘也;公子有德于人,愿公子忘之也。且矫魏王令,夺晋鄙兵以救赵,于赵则有功矣,于魏则未为忠臣也。公子乃自骄而功之,窃为公子不取也。”

于是公子立自责,似若无所容者。赵王扫除自迎,执主人之礼,引公子就西阶。公子侧行辞让,从东阶上\footnote{text}。自言罪过,以负于魏,无功于赵。赵王侍酒至暮,口不忍献五城,以公子退让也。公子竟留赵。赵王以鄗为公子汤沐邑\footnote{text},魏亦复以信陵奉公子。公子留赵。
\end{yuanwen}

\begin{yuanwen}
公子闻赵有处士毛公藏于博徒\footnote{text},薛公藏于卖浆家。公子欲见两人,两人自匿,不肯见公子。公子闻所在,乃间步往从此两人游\footnote{text},甚欢。

平原君闻之,谓其夫人曰:“始吾闻夫人弟公子天下无双,今吾闻之,乃妄从博徒卖浆者游,公子妄人耳\footnote{text}。”

夫人以告公子。公子乃谢夫人去,曰:“始吾闻平原君贤,故负魏王而救赵\footnote{text},以称平原君。平原君之游,徒豪举耳\footnote{text},不求士也。无忌自在大梁时,常闻此两人贤,至赵,恐不得见。以无忌从之游,尚恐其不我欲也,今平原君乃以为羞,其不足从游。”

乃装为去。夫人具以语平原君。平原君乃免冠谢,固留公子。平原君门下闻之,半去平原君归公子,天下士复往归公子,公子倾平原君客。
\end{yuanwen}

\begin{yuanwen}
公子留赵十年不归。秦闻公子在赵,日夜出兵东伐魏。魏王患之,使使往请公子。公子恐其怒之,乃诫门下:“有敢为魏王使通者,死\footnote{text}。”

宾客皆背魏之赵,莫敢劝公子归。毛公、薛公两人往见公子曰:“公子所以重于赵,名闻诸侯者,徒以有魏也。今秦攻魏,魏急而公子不恤,使秦破大梁而夷先王之宗庙,公子当何面目立天下乎?”

语未及卒,公子立变色,告车趣驾归救魏\footnote{text}。
\end{yuanwen}

\begin{yuanwen}
魏王见公子,相与泣,而以上将军印授公子,公子遂将。

魏安釐王三十年,公子使使遍告诸侯。诸侯闻公子将,各遣将将兵救魏。公子率五国之兵破秦军于河外\footnote{text},走蒙骜\footnote{text}。遂乘胜逐秦军至函谷关\footnote{text},抑秦兵,秦兵不敢出。当是时,公子威振天下,诸侯之客进兵法,公子皆名之\footnote{text},故世俗称《魏公子兵法》。
\end{yuanwen}

\begin{yuanwen}
秦王患之,乃行金万斤于魏,求晋鄙客,令毁公子于魏王曰:“公子亡在外十年矣,今为魏将,诸侯将皆属,诸侯徒闻魏公子,不闻魏王。公子亦欲因此时定南面而王,诸侯畏公子之威,方欲共立之。”

秦数使反间,伪贺公子得立为魏王未也。魏王日闻其毁,不能不信,后果使人代公子将。公子自知再以毁废,乃谢病不朝,与宾客为长夜饮,饮醇酒,多近妇女。日夜为乐饮者四岁,竟病酒而卒\footnote{text}。其岁,魏安釐王亦薨。
\end{yuanwen}

\begin{yuanwen}
秦闻公子死,使蒙骜攻魏,拔二十城,初置东郡。其后秦稍蚕食魏,十八岁而虏魏王,屠大梁\footnote{text}。
\end{yuanwen}

\begin{yuanwen}
高祖始微少时\footnote{text},数闻公子贤。及即天子位,每过大梁,常祠公子。高祖十二年,从击黥布还\footnote{text},为公子置守冢五家,世世岁以四时奉祠公子。
\end{yuanwen}

\begin{yuanwen}
太史公曰:吾过大梁之墟,求问其所谓夷门。夷门者,城之东门也。天下诸公子亦有喜士者矣,然信陵君之接岩穴隐者,不耻下交。有以也,名冠诸侯,不虚耳。高祖每过之而令民奉祠不绝也\footnote{text}。
\end{yuanwen}

\begin{yuanwen}
信陵下士,邻国相倾。以公子故,不敢加兵。颇知硃亥,尽礼侯嬴。遂卻晋鄙,终辞赵城。毛、薛见重,万古希声。
\end{yuanwen}

\chapter{春申君列传}

\begin{yuanwen}
春申君者,楚人也,名歇,姓黄氏。游学博闻,事楚顷襄王。顷襄王以歇为辩,使于秦。秦昭王使白起攻韩、魏,败之于华阳,禽魏将芒卯,韩、魏服而事秦。秦昭王方令白起与韩、魏共伐楚,未行,而楚使黄歇適至于秦,闻秦之计。当是之时,秦已前使白起攻楚,取巫、黔中之郡,拔鄢郢,东至竟陵,楚顷襄王东徙治于陈县。黄歇见楚怀王之为秦所诱而入朝,遂见欺,留死于秦。顷襄王,其子也,秦轻之,恐壹举兵而灭楚。歇乃上书说秦昭王曰:

天下莫彊于秦、楚。今闻大王欲伐楚,此犹两虎相与斗。两虎相与斗而驽犬受其弊,不如善楚。臣请言其说:臣闻物至则反,冬夏是也;致至则危,累釭是也。今大国之地,遍天下有其二垂,此从生民已来,万乘之地未尝有也。先帝文王、庄王之身,三世不妄接地于齐,以绝从亲之要。今王使盛桥守事于韩,盛桥以其地入秦,是王不用甲,不信威,而得百里之地。王可谓能矣。王又举甲而攻魏,杜大梁之门,举河内,拔燕、酸枣、虚、桃,入邢,魏之兵云翔而不敢捄。王之功亦多矣。王休甲息众,二年而后复之;又并蒲、衍、首、垣,以临仁、平丘,黄、济阳婴城而魏氏服;王又割濮
之北,注齐秦之要,绝楚赵之脊,天下五合六聚而不敢救。王之威亦单矣。

王若能持功守威,绌攻取之心而肥仁义之地,使无后患,三王不足四,五伯不足六也。王若负人徒之众,仗兵革之彊,乘毁魏之威,而欲以力臣天下之主,臣恐其有后患也。诗曰“靡不有初,鲜克有终”。易曰“狐涉水,濡其尾”。此言始之易,终之难也。何以知其然也?昔智氏见伐赵之利而不知榆次之祸,吴见伐齐之便而不知干隧之败。此二国者,非无大功也,没利于前而易患于后也。吴之信越也,从而伐齐,既胜齐人于艾陵,还为越王禽三渚之浦。智氏之信韩、魏也,从而伐赵,攻晋阳城,胜有日矣,韩、魏叛之,杀智伯瑶于凿台之下。今王妒楚之不毁也,而忘毁楚之彊韩、魏也,臣为王虑而不取也。

诗曰“大武远宅而不涉”。从此观之,楚国,援也;邻国,敌也。诗云“趯趯
免,还犬获之。他人有心,余忖度之”。今王中道而信韩、魏之善王也,此正吴之信越也。臣闻之,敌不可假,时不可失。臣恐韩、魏卑辞除患而实欲欺大国也。何则?王无重世之德于韩、魏,而有累世之怨焉。夫韩、魏父子兄弟接踵而死于秦者将十世矣。本国残,社稷坏,宗庙毁。刳腹绝肠,折颈摺颐,首身分离,暴骸骨于草泽,头颅僵仆,相望于境,父子老弱系脰束手为群虏者相及于路。鬼神孤伤,无所血食。人民不聊生,族类离散,流亡为仆妾者,盈满海内矣。故韩、魏之不亡,秦社稷之忧也,今王资之与攻楚,不亦过乎!

且王攻楚将恶出兵?王将借路于仇雠之韩、魏乎?兵出之日而王忧其不返也,是王以兵资于仇雠之韩、魏也。王若不借路于仇雠之韩、魏,必攻随水右壤。随水右壤,此皆广川大水,山林谿谷,不食之地也,王虽有之,不为得地。是王有毁楚之名而无得地之实也。

且王攻楚之日,四国必悉起兵以应王。秦、楚之兵构而不离,魏氏将出而攻留、方与、铚、湖陵、砀、萧、相,故宋必尽。齐人南面攻楚,泗上必举。此皆平原四达,膏腴之地,而使独攻。王破楚以肥韩、魏于中国而劲齐。韩、魏之彊,足以校于秦。齐南以泗水为境,东负海,北倚河,而无后患,天下之国莫彊于齐、魏,齐、魏得地葆利而详事下吏,一年之后,为帝未能,其于禁王之为帝有馀矣。

夫以王壤土之博,人徒之众,兵革之彊,壹举事而树怨于楚,迟令韩、魏归帝重于齐,是王失计也。臣为王虑,莫若善楚。秦、楚合而为一以临韩,韩必敛手。王施以东山之险,带以曲河之利,韩必为关内之侯。若是而王以十万戍郑,梁氏寒心,许、鄢陵婴城,而上蔡、召陵不往来也,如此而魏亦关内侯矣。王壹善楚,而关内两万乘之主注地于齐,齐右壤可拱手而取也。王之地一经两海,要约天下,是燕、赵无齐、楚,齐、楚无燕、赵也。然后危动燕、赵,直摇齐、楚,此四国者不待痛而服矣。

昭王曰:“善。”于是乃止白起而谢韩、魏。发使赂楚,约为与国。

黄歇受约归楚,楚使歇与太子完入质于秦,秦留之数年。楚顷襄王病,太子不得归。而楚太子与秦相应侯善,于是黄歇乃说应侯曰:“相国诚善楚太子乎?”应侯曰:“然。”歇曰:“今楚王恐不起疾,秦不如归其太子。太子得立,其事秦必重而德相国无穷,是亲与国而得储万乘也。若不归,则咸阳一布衣耳;楚更立太子,必不事秦。夫失与国而绝万乘之和,非计也。原相国孰虑之。”应侯以闻秦王。秦王曰:“令楚太子之傅先往问楚王之疾,返而后图之。”黄歇为楚太子计曰:“秦之留太子也,欲以求利也。今太子力未能有以利秦也,歇忧之甚。而阳文君子二人在中,王若卒大命,太子不在,阳文君子必立为后,太子不得奉宗庙矣。不如亡秦,与使者俱出;臣请止,以死当之。”楚太子因变衣服为楚使者御以出关,而黄歇守舍,常为谢病。度太子已远,秦不能追,歇乃自言秦昭王曰:“楚太子已归,出远矣。歇当死,原赐死。”昭王大怒,欲听其自杀也。应侯曰:“歇为人臣,出身以徇其主,太子立,必用歇,故不如无罪而归之,以亲楚。”秦因遣黄歇。

歇至楚三月,楚顷襄王卒,太子完立,是为考烈王。考烈王元年,以黄歇为相,封为春申君,赐淮北地十二县。后十五岁,黄歇言之楚王曰:“淮北地边齐,其事急,请以为郡便。”因并献淮北十二县。请封于江东。考烈王许之。春申君因城故吴墟,以自为都邑。

春申君既相楚,是时齐有孟尝君,赵有平原君,魏有信陵君,方争下士,招致宾客,以相倾夺,辅国持权。

春申君为楚相四年,秦破赵之长平军四十馀万。五年,围邯郸。邯郸告急于楚,楚使春申君将兵往救之,秦兵亦去,春申君归。春申君相楚八年,为楚北伐灭鲁,以荀卿为兰陵令。当是时,楚复彊。

赵平原君使人于春申君,春申君舍之于上舍。赵使欲夸楚,为玳瑁簪,刀剑室以珠玉饰之,请命春申君客。春申君客三千馀人,其上客皆蹑珠履以见赵使,赵使大惭。

春申君相十四年,秦庄襄王立,以吕不韦为相,封为文信侯。取东周。

春申君相二十二年,诸侯患秦攻伐无已时,乃相与合从,西伐秦,而楚王为从长,春申君用事。至函谷关,秦出兵攻,诸侯兵皆败走。楚考烈王以咎春申君,春申君以此益疏。

客有观津人硃英,谓春申君曰:“人皆以楚为彊而君用之弱,其于英不然。先君时善秦二十年而不攻楚,何也?秦逾黾隘之塞而攻楚,不便;假道于两周,背韩、魏而攻楚,不可。今则不然,魏旦暮亡,不能爱许、鄢陵,其许魏割以与秦。秦兵去陈百六十里,臣之所观者,见秦、楚之日斗也。”楚于是去陈徙寿春;而秦徙卫野王,作置东郡。春申君由此就封于吴,行相事。

楚考烈王无子,春申君患之,求妇人宜子者进之,甚众,卒无子。赵人李园持其女弟,欲进之楚王,闻其不宜子,恐久毋宠。李园求事春申君为舍人,已而谒归,故失期。还谒,春申君问之状,对曰:“齐王使使求臣之女弟,与其使者饮,故失期。”春申君曰:“娉入乎?”对曰:“未也。”春申君曰:“可得见乎?”曰:“可。”于是李园乃进其女弟,即幸于春申君。知其有身,李园乃与其女弟谋。园女弟承间以说春申君曰:“楚王之贵幸君,虽兄弟不如也。今君相楚二十馀年,而王无子,即百岁后将更立兄弟,则楚更立君后,亦各贵其故所亲,君又安得长有宠乎?非徒然也,君贵用事久,多失礼于王兄弟,兄弟诚立,祸且及身,何以保相印江东之封乎?今妾自知有身矣,而人莫知。妾幸君未久,诚以君之重而进妾于楚王,王必幸妾;妾赖天有子男,则是君之子为王也,楚国尽可得,孰与身临不测之罪乎?”春申君大然之,乃出李园女弟,谨舍而言之楚王。楚王召入幸之,遂生子男,立为太子,以李园女弟为王后。楚王贵李园,园用事。

李园既入其女弟,立为王后,子为太子,恐春申君语泄而益骄,阴养死士,欲杀春申君以灭口,而国人颇有知之者。

春申君相二十五年,楚考烈王病。硃英谓春申君曰:“世有毋望之福,又有毋望之祸。今君处毋望之世,事毋望之主,安可以无毋望之人乎?”春申君曰:“何谓毋望之福?”曰:“君相楚二十馀年矣,虽名相国,实楚王也。今楚王病,旦暮且卒,而君相少主,因而代立当国,如伊尹、周公,王长而反政,不即遂南面称孤而有楚国?此所谓毋望之福也。”春申君曰:“何谓毋望之祸?”曰:“李园不治国而君之仇也,不为兵而养死士之日久矣,楚王卒,李园必先入据权而杀君以灭口。此所谓毋望之祸也。”春申君曰:“何谓毋望之人?”对曰:“君置臣郎中,楚王卒,李园必先入,臣为君杀李园。此所谓毋望之人也。”春申君曰:“足下置之,李园,弱人也,仆又善之,且又何至此!”硃英知言不用,恐祸及身,乃亡去。  后十七日,楚考烈王卒,李园果先入,伏死士于棘门之内。春申君入棘门,园死士侠刺春申君,斩其头,投之棘门外。于是遂使吏尽灭春申君之家。而李园女弟初幸春申君有身而入之王所生子者遂立,是为楚幽王。

是岁也,秦始皇帝立九年矣。嫪毐亦为乱于秦,觉,夷其三族,而吕不韦废。

太史公曰:吾適楚,观春申君故城,宫室盛矣哉!初,春申君之说秦昭王,及出身遣楚太子归,何其智之明也!后制于李园,旄矣。语曰:“当断不断,反受其乱。”春申君失硃英之谓邪?

黄歇辩智,权略秦、楚。太子获归,身作宰辅。珠炫赵客,邑开吴土。烈王寡胤,李园献女。无妄成灾,硃英徒语。
\end{yuanwen}

\part{卷七十九}
\chapter{范睢蔡泽列传}

\begin{yuanwen}
范睢者,魏人也,字叔。游说诸侯,欲事魏王,家贫无以自资,乃先事魏中大夫须贾。

须贾为魏昭王使于齐,范睢从。留数月,未得报。齐襄王闻睢辩口,乃使人赐睢金十斤及牛酒,睢辞谢不敢受。须贾知之,大怒,以为睢持魏国阴事告齐,故得此馈,令睢受其牛酒,还其金。既归,心怒睢,以告魏相。魏相,魏之诸公子,曰魏齐。魏齐大怒,使舍人笞击睢,折胁摺齿。睢详死,即卷以箦,置厕中。宾客饮者醉,更溺睢,故僇辱以惩后,令无妄言者。睢从箦中谓守者曰:“公能出我,我必厚谢公。”守者乃请出弃箦中死人。魏齐醉,曰:“可矣。”范睢得出。后魏齐悔,复召求之。魏人郑安平闻之,乃遂操范睢亡,伏匿,更名姓曰张禄。

当此时,秦昭王使谒者王稽于魏。郑安平诈为卒,侍王稽。王稽问:“魏有贤人可与俱西游者乎?”郑安平曰:“臣里中有张禄先生,欲见君,言天下事。其人有仇,不敢昼见。”王稽曰:“夜与俱来。”郑安平夜与张禄见王稽。语未究,王稽知范睢贤,谓曰:“先生待我于三亭之南。”与私约而去。

王稽辞魏去,过载范睢入秦。至湖,望见车骑从西来。范睢曰:“彼来者为谁?”王稽曰:“秦相穰侯东行县邑。”范睢曰:“吾闻穰侯专秦权,恶内诸侯客,此恐辱我,我宁且匿车中。”有顷,穰侯果至,劳王稽,因立车而语曰:“关东有何变?”曰:“无有。”又谓王稽曰:“谒君得无与诸侯客子俱来乎?无益,徒乱人国耳。”王稽曰:“不敢。”即别去。范睢曰:“吾闻穰侯智士也,其见事迟,乡者疑车中有人,忘索之。”于是范睢下车走,曰:“此必悔之。”行十馀里,果使骑还索车中,无客,乃已。王稽遂与范睢入咸阳。

已报使,因言曰:“魏有张禄先生,天下辩士也。曰‘秦王之国危于累卵,得臣则安。然不可以书传也’。臣故载来。”秦王弗信,使舍食草具。待命岁馀。

当是时,昭王已立三十六年。南拔楚之鄢郢,楚怀王幽死于秦。秦东破齐。湣王尝称帝,后去之。数困三晋。厌天下辩士,无所信。

穰侯,华阳君,昭王母宣太后之弟也;而泾阳君、高陵君皆昭王同母弟也。穰侯相,三人者更将,有封邑,以太后故,私家富重于王室。及穰侯为秦将,且欲越韩、魏而伐齐纲寿,欲以广其陶封。范睢乃上书曰:

臣闻明主立政,有功者不得不赏,有能者不得不官,劳大者其禄厚,功多者其爵尊,能治众者其官大。故无能者不敢当职焉,有能者亦不得蔽隐。使以臣之言为可,原行而益利其道;以臣之言为不可,久留臣无为也。语曰:“庸主赏所爱而罚所恶;明主则不然,赏必加于有功,而刑必断于有罪。”今臣之胸不足以当椹质,而要不足以待斧钺,岂敢以疑事尝试于王哉!虽以臣为贱人而轻辱,独不重任臣者之无反复于王邪?

且臣闻周有砥砨,宋有结绿,梁有县藜,楚有和朴,此四宝者,土之所生,良工之所失也,而为天下名器。然则圣王之所弃者,独不足以厚国家乎?

臣闻善厚家者取之于国,善厚国者取之于诸侯。天下有明主则诸侯不得擅厚者,何也?为其割荣也。良医知病人之死生,而圣主明于成败之事,利则行之,害则舍之,疑则少尝之,虽舜禹复生,弗能改已。语之至者,臣不敢载之于书,其浅者又不足听也。意者臣愚而不概于王心邪?亡其言臣者贱而不可用乎?自非然者,臣原得少赐游观之间,望见颜色。一语无效,请伏斧质。

于是秦昭王大说,乃谢王稽,使以传车召范睢。

于是范睢乃得见于离宫,详为不知永巷而入其中。王来而宦者怒,逐之,曰:“王至!”范睢缪为曰:“秦安得王?秦独有太后、穰侯耳。”欲以感怒昭王。昭王至,闻其与宦者争言,遂延迎,谢曰:“寡人宜以身受命久矣,会义渠之事急,寡人旦暮自请太后;今义渠之事已,寡人乃得受命。窃闵然不敏,敬执宾主之礼。”范睢辞让。是日观范睢之见者,群臣莫不洒然变色易容者。

秦王屏左右,宫中虚无人。秦王跽而请曰:“先生何以幸教寡人?”范睢曰:“唯唯。”有间,秦王复跽而请曰:“先生何以幸教寡人?”范睢曰:“唯唯。”若是者三。秦王跽曰:“先生卒不幸教寡人邪?”范睢曰:“非敢然也。臣闻昔者吕尚之遇文王也,身为渔父而钓于渭滨耳。若是者,交疏也。已说而立为太师,载与俱归者,其言深也。故文王遂收功于吕尚而卒王天下。乡使文王疏吕尚而不与深言,是周无天子之德,而文武无与成其王业也。今臣羁旅之臣也,交疏于王,而所原陈者皆匡君之事,处人骨肉之间,原效愚忠而未知王之心也。此所以王三问而不敢对者也。臣非有畏而不敢言也。臣知今日言之于前而明日伏诛于后,然臣不敢避也。大王信行臣之言,死不足以为臣患,亡不足以为臣忧,漆身为厉被发为狂不足以为臣耻。且以五帝之圣焉而死,三王之仁焉而死,五伯之贤焉而死,乌获、任鄙之力焉而死,成荆、孟贲、王庆忌、夏育之勇焉而死。死者,人之所必不免也。处必然之势,可以少有补于秦,此臣之所大原也,臣又何患哉!伍子胥橐载而出昭关,夜行昼伏,至于陵水,无以餬其口,行蒲伏,稽首肉袒,鼓腹吹篪,乞食于吴市,卒兴吴国,阖闾为伯。使臣得尽谋如伍子胥,加之以幽囚,终身不复见,是臣之说行也,臣又何忧?箕子、接舆漆身为厉,被发为狂,无益于主。假使臣得同行于箕子,可以有补于所贤之主,是臣之大荣也,臣有何耻?臣之所恐者,独恐臣死之后,天下见臣之尽忠而身死,因以是杜口裹足,莫肯乡秦耳。足下上畏太后之严,下惑于奸臣之态,居深宫之中,不离阿保之手,终身迷惑,无与昭奸。大者宗庙灭覆,小者身以孤危,此臣之所恐耳。若夫穷辱之事,死亡之患,臣不敢畏也。臣死而秦治,是臣死贤于生。”秦王跽曰:“先生是何言也!夫秦国辟远,寡人愚不肖,先生乃幸辱至于此,是天以寡人慁先生而存先王之宗庙也。寡人得受命于先生,是天所以幸先王,而不弃其孤也。先生柰何而言若是!事无小大,上及太后,下至大臣,原先生悉以教寡人,无疑寡人也。”范睢拜,秦王亦再拜

范睢曰:“大王之国,四塞以为固,北有甘泉、谷口,南带泾、渭,右陇、蜀,左关、阪,奋击百万,战车千乘,利则出攻,不利则入守,此王者之地也。民怯于私斗而勇于公战,此王者之民也。王并此二者而有之。夫以秦卒之勇,车骑之众,以治诸侯,譬若施韩卢而搏蹇兔也,霸王之业可致也,而群臣莫当其位。至今闭关十五年,不敢窥兵于山东者,是穰侯为秦谋不忠,而大王之计有所失也。”秦王跽曰:“寡人原闻失计。”

然左右多窃听者,范睢恐,未敢言内,先言外事,以观秦王之俯仰。因进曰:“夫穰侯越韩、魏而攻齐纲寿,非计也。少出师则不足以伤齐,多出师则害于秦。臣意王之计,欲少出师而悉韩、魏之兵也,则不义矣。今见与国之不亲也,越人之国而攻,可乎?其于计疏矣。且昔齐湣王南攻楚,破军杀将,再辟地千里,而齐尺寸之地无得焉者,岂不欲得地哉,形势不能有也。诸侯见齐之罢弊,君臣之不和也,兴兵而伐齐,大破之。士辱兵顿,皆咎其王,曰:‘谁为此计者乎?’王曰:‘文子为之。’大臣作乱,文子出走。攻齐所以大破者,以其伐楚而肥韩、魏也。此所谓借贼兵而赍盗粮者也。王不如远交而近攻,得寸则王之寸也,得尺亦王之尺也。今释此而远攻,不亦缪乎!且昔者中山之国地方五百里,赵独吞之,功成名立而利附焉,天下莫之能害也。今夫韩、魏,中国之处而天下之枢也,王其欲霸,必亲中国以为天下枢,以威楚、赵。楚彊则附赵,赵彊则附楚,楚、赵皆附,齐必惧矣。齐惧,必卑辞重币以事秦。齐附而韩、魏因可虏也。”昭王曰:“吾欲亲魏久矣,而魏多变之国也,寡人不能亲。请问亲魏柰何?”对曰:“王卑词重币以事之;不可,则割地而赂之;不可,因举兵而伐之。”王曰:“寡人敬闻命矣。”乃拜范睢为客卿,谋兵事。卒听范睢谋,使五大夫绾伐魏,拔怀。后二岁,拔邢丘。

客卿范睢复说昭王曰:“秦韩之地形,相错如绣。秦之有韩也,譬如木之有蠹也,人之有心腹之病也。天下无变则已,天下有变,其为秦患者孰大于韩乎?王不如收韩。”昭王曰:“吾固欲收韩,韩不听,为之柰何?”对曰:“韩安得无听乎?王下兵而攻荥阳,则巩、成皋之道不通;北断太行之道,则上党之师不下。王一兴兵而攻荥阳,则其国断而为三。夫韩见必亡,安得不听乎?若韩听,而霸事因可虑矣。”王曰:“善。”且欲发使于韩。

范睢日益亲,复说用数年矣,因请间说曰:“臣居山东时,闻齐之有田文,不闻其有王也;闻秦之有太后、穰侯、华阳、高陵、泾阳,不闻其有王也。夫擅国之谓王,能利害之谓王,制杀生之威之谓王。今太后擅行不顾,穰侯出使不报,华阳、泾阳等击断无讳,高陵进退不请。四贵备而国不危者,未之有也。为此四贵者下,乃所谓无王也。然则权安得不倾,令安得从王出乎?臣闻善治国者,乃内固其威而外重其权。穰侯使者操王之重,决制于诸侯,剖符于天下,政適伐国,莫敢不听。战胜攻取则利归于陶,国弊御于诸侯;战败则结怨于百姓,而祸归于社稷。诗曰‘木实繁者披其枝,披其枝者伤其心;大其都者危其国,尊其臣者卑其主’。崔杼、淖齿管齐,射王股,擢王筋,县之于庙梁,宿昔而死。李兑管赵,囚主父于沙丘,百日而饿死。今臣闻秦太后、穰侯用事,高陵、华阳、泾阳佐之,卒无秦王,此亦淖齿、李兑之类也。且夫三代所以亡国者,君专授政,纵酒驰骋弋猎,不听政事。其所授者,妒贤嫉能,御下蔽上,以成其私,不为主计,而主不觉悟,故失其国。今自有秩以上至诸大吏,下及王左右,无非相国之人者。见王独立于朝,臣窃为王恐,万世之后,有秦国者非王子孙也。”昭王闻之大惧,曰:“善。”于是废太后,逐穰侯、高陵、华阳、泾阳君于关外。秦王乃拜范睢为相。收穰侯之印,使归陶,因使县官给车牛以徙,千乘有馀。到关,关阅其宝器,宝器珍怪多于王室。

秦封范睢以应,号为应侯。当是时,秦昭王四十一年也。

范睢既相秦,秦号曰张禄,而魏不知,以为范睢已死久矣。魏闻秦且东伐韩、魏,魏使须贾于秦。范睢闻之,为微行,敝衣间步之邸,见须贾。须贾见之而惊曰:“范叔固无恙乎!”范睢曰:“然。”须贾笑曰:“范叔有说于秦邪?”曰:“不也。睢前日得过于魏相,故亡逃至此,安敢说乎!”须贾曰:“今叔何事?”范睢曰“臣为人庸赁。”须贾意哀之,留与坐饮食,曰:“范叔一寒如此哉!”乃取其一綈袍以赐之。须贾因问曰:“秦相张君,公知之乎?吾闻幸于王,天下之事皆决于相君。今吾事之去留在张君。孺子岂有客习于相君者哉?”范睢曰:“主人翁习知之。唯睢亦得谒,睢请为见君于张君。”须贾曰:“吾马病,车轴折,非大车驷马,吾固不出。”范睢曰:“原为君借大车驷马于主人翁。”

范睢归取大车驷马,为须贾御之,入秦相府。府中望见,有识者皆避匿。须贾怪之。至相舍门,谓须贾曰:“待我,我为君先入通于相君。”须贾待门下,持车良久,问门下曰:“范叔不出,何也?”门下曰:“无范叔。”须贾曰:“乡者与我载而入者。”门下曰:“乃吾相张君也。”须贾大惊,自知见卖,乃肉袒行,因门下人谢罪。于是范睢盛帷帐,待者甚众,见之。须贾顿首言死罪,曰:“贾不意君能自致于青云之上,贾不敢复读天下之书,不敢复与天下之事。贾有汤镬之罪,请自屏于胡貉之地,唯君死生之!”范睢曰:“汝罪有几?”曰:“擢贾之发以续贾之罪,尚未足。”范睢曰:“汝罪有三耳。昔者楚昭王时而申包胥为楚卻吴军,楚王封之以荆五千户,包胥辞不受,为丘墓之寄于荆也。今睢之先人丘墓亦在魏,公前以睢为有外心于齐而恶睢于魏齐,公之罪一也。当魏齐辱我于厕中,公不止,罪二也。更醉而溺我,公其何忍乎?罪三矣。然公之所以得无死者,以綈袍恋恋,有故人之意,故释公。”乃谢罢。入言之昭王,罢归须贾。

须贾辞于范睢,范睢大供具,尽请诸侯使,与坐堂上,食饮甚设。而坐须贾于堂下,置豆其前,令两黥徒夹而马食之。数曰:“为我告魏王,急持魏齐头来!不然者,我且屠大梁。”须贾归,以告魏齐。魏齐恐,亡走赵。匿平原君所。

范睢既相,王稽谓范睢曰:“事有不可知者三,有不柰何者亦三。宫车一日晏驾,是事之不可知者一也。君卒然捐馆舍,是事之不可知者二也。使臣卒然填沟壑,是事之不可知者三也。宫车一日晏驾,君虽恨于臣,无可柰何。君卒然捐馆舍,君虽恨于臣,亦无可柰何。使臣卒然填沟壑,君虽恨于臣,亦无可柰何。”范睢不怿,乃入言于王曰:“非王稽之忠,莫能内臣于函谷关;非大王之贤圣,莫能贵臣。今臣官至于相,爵在列侯,王稽之官尚止于谒者,非其内臣之意也。”昭王召王稽,拜为河东守,三岁不上计。又任郑安平,昭王以为将军。范睢于是散家财物,尽以报所尝困戹者。一饭之德必偿,睚眦之怨必报。

范睢相秦二年,秦昭王之四十二年,东伐韩少曲、高平,拔之。

秦昭王闻魏齐在平原君所,欲为范睢必报其仇,乃详为好书遗平原君曰;“寡人闻君之高义,原与君为布衣之友,君幸过寡人,寡人原与君为十日之饮。”平原君畏秦,且以为然,而入秦见昭王。昭王与平原君饮数日,昭王谓平原君曰:“昔周文王得吕尚以为太公,齐桓公得管夷吾以为仲父,今范君亦寡人之叔父也。范君之仇在君之家,原使人归取其头来;不然,吾不出君于关。”平原君曰:“贵而为交者,为贱也;富而为交者,为贫也。夫魏齐者,胜之友也,在,固不出也,今又不在臣所。”昭王乃遗赵王书曰:“王之弟在秦,范君之仇魏齐在平原君之家。王使人疾持其头来;不然,吾举兵而伐赵,又不出王之弟于关。”赵孝成王乃发卒围平原君家,急,魏齐夜亡出,见赵相虞卿。虞卿度赵王终不可说,乃解其相印,与魏齐亡,间行,念诸侯莫可以急抵者,乃复走大梁,欲因信陵君以走楚。信陵君闻之,畏秦,犹豫未肯见,曰:“虞卿何如人也?”时侯嬴在旁,曰:“人固未易知,知人亦未易也。夫虞卿蹑屩檐簦,一见赵王,赐白璧一双,黄金百镒;再见,拜为上卿;三见,卒受相印,封万户侯。当此之时,天下争知之。夫魏齐穷困过虞卿,虞卿不敢重爵禄之尊,解相印,捐万户侯而间行。急士之穷而归公子,公子曰‘何如人’。人固不易知,知人亦未易也!”信陵君大惭,驾如野迎之。魏齐闻信陵君之初难见之,怒而自刭。赵王闻之,卒取其头予秦。秦昭王乃出平原君归赵。

昭王四十三年,秦攻韩汾陉,拔之,因城河上广武。

后五年,昭王用应侯谋,纵反间卖赵,赵以其故,令马服子代廉颇将。秦大破赵于长平,遂围邯郸。已而与武安君白起有隙,言而杀之。任郑安平,使击赵。郑安平为赵所围,急,以兵二万人降赵。应侯席请罪。秦之法,任人而所任不善者,各以其罪罪之。于是应侯罪当收三族。秦昭王恐伤应侯之意,乃下令国中:“有敢言郑安平事者,以其罪罪之。”而加赐相国应侯食物日益厚,以顺適其意。后二岁,王稽为河东守,与诸侯通,坐法诛。而应侯日益以不怿。

昭王临朝叹息,应侯进曰:“臣闻‘主忧臣辱,主辱臣死’。今大王中朝而忧,臣敢请其罪。”昭王曰:“吾闻楚之铁剑利而倡优拙。夫铁剑利则士勇,倡优拙则思虑远。夫以远思虑而御勇士,吾恐楚之图秦也。夫物不素具,不可以应卒,今武安君既死,而郑安平等畔,内无良将而外多敌国,吾是以忧。”欲以激励应侯。应侯惧,不知所出。蔡泽闻之,往入秦也。

蔡泽者,燕人也。游学干诸侯小大甚众,不遇。而从唐举相,曰:“吾闻先生相李兑,曰‘百日之内持国秉’,有之乎?”曰:“有之。”曰:“若臣者何如?”唐举孰视而笑曰:“先生曷鼻,巨肩,魋颜,蹙齃,膝挛。吾闻圣人不相,殆先生乎?”蔡泽知唐举戏之,乃曰:“富贵吾所自有,吾所不知者寿也,原闻之。”唐举曰:“先生之寿,从今以往者四十三岁。”蔡泽笑谢而去,谓其御者曰:“吾持粱刺齿肥,跃马疾驱,怀黄金之印,结紫绶于要,揖让人主之前,食肉富贵,四十三年足矣。”去之赵,见逐。之韩、魏,遇夺釜鬲于涂。闻应侯任郑安平、王稽皆负重罪于秦,应侯内惭,蔡泽乃西入秦。

将见昭王,使人宣言以感怒应侯曰:“燕客蔡泽,天下雄俊弘辩智士也。彼一见秦王,秦王必困君而夺君之位。”应侯闻,曰:“五帝三代之事,百家之说,吾既知之,众口之辩,吾皆摧之,是恶能困我而夺我位乎?”使人召蔡泽。蔡泽入,则揖应。应侯固不快,及见之,又倨,应侯因让之曰:“子尝宣言欲代我相秦,宁有之乎?”对曰:“然。”应侯曰:“请闻其说。”蔡泽曰:“吁,君何见之晚也!夫四时之序,成功者去。夫人生百体坚彊,手足便利,耳目聪明而心圣智,岂非士之原与?”应侯曰:“然。”蔡泽曰:“质仁秉义,行道施德,得志于天下,天下怀乐敬爱而尊慕之,皆原以为君王,岂不辩智之期与?”应侯曰:“然。”蔡泽复曰:“富贵显荣,成理万物,使各得其所;性命寿长,终其天年而不夭伤;天下继其统,守其业,传之无穷;名实纯粹,泽流千里,世世称之而无绝,与天地终始:岂道德之符而圣人所谓吉祥善事者与?”应侯曰:“然。”

蔡泽曰:“若夫秦之商君,楚之吴起,越之大夫种,其卒然亦可原与?”应侯知蔡泽之欲困己以说,复谬曰:“何为不可?夫公孙鞅之事孝公也,极身无贰虑,尽公而不顾私;设刀锯以禁奸邪,信赏罚以致治;披腹心,示情素,蒙怨咎,欺旧友,夺魏公子卬,安秦社稷,利百姓,卒为秦禽将破敌,攘地千里。吴起之事悼王也,使私不得害公,谗不得蔽忠,言不取苟合,行不取苟容,不为危易行,行义不辟难,然为霸主强国,不辞祸凶。大夫种之事越王也,主虽困辱,悉忠而不解,主虽绝亡,尽能而弗离,成功而弗矜,贵富而不骄怠。若此三子者,固义之至也,忠之节也。是故君子以义死难,视死如归;生而辱不如死而荣。士固有杀身以成名,虽义之所在,虽死无所恨。何为不可哉?”

蔡泽曰:“主圣臣贤,天下之盛福也;君明臣直,国之福也;父慈子孝,夫信妻贞,家之福也。故比干忠而不能存殷,子胥智而不能完吴,申生孝而晋国乱。是皆有忠臣孝子,而国家灭乱者,何也?无明君贤父以听之,故天下以其君父为僇辱而怜其臣子。今商君、吴起、大夫种之为人臣,是也;其君,非也。故世称三子致功而不见德,岂慕不遇世死乎?夫待死而后可以立忠成名,是微子不足仁,孔子不足圣,管仲不足大也。夫人之立功,岂不期于成全邪?身与名俱全者,上也。名可法而身死者,其次也。名在僇辱而身全者,下也。”于是应侯称善。

蔡泽少得间,因曰:“夫商君、吴起、大夫种,其为人臣尽忠致功则可原矣,闳夭事文王,周公辅成王也,岂不亦忠圣乎?以君臣论之,商君、吴起、大夫种其可原孰与闳夭、周公哉?”应侯曰:“商君、吴起、大夫种弗若也。”蔡泽曰:“然则君之主慈仁任忠,惇厚旧故,其贤智与有道之士为胶漆,义不倍功臣,孰与秦孝公、楚悼王、越王乎?”应侯曰:“未知何如也。”蔡泽曰:“今主亲忠臣,不过秦孝公、楚悼王、越王,君之设智,能为主安危修政,治乱彊兵,批患折难,广地殖穀,富国足家,彊主,尊社稷,显宗庙,天下莫敢欺犯其主,主之威盖震海内,功彰万里之外,声名光辉传于千世,君孰与商君、吴起、大夫种?”应侯曰:“不若。”蔡泽曰:“今主之亲忠臣不忘旧故不若孝公、悼王、句践,而君之功绩爱信亲幸又不若商君、吴起、大夫种,然而君之禄位贵盛,私家之富过于三子,而身不退者,恐患之甚于三子,窃为君危之。语曰‘日中则移,月满则亏’。物盛则衰,天地之常数也。进退盈缩,与时变化,圣人之常道也。故‘国有道则仕,国无道则隐’。圣人曰‘飞龙在天,利见大人’。‘不义而富且贵,于我如浮云’。今君之怨已雠而德已报,意欲至矣,而无变计,窃为君不取也。且夫翠、鹄、犀、象,其处势非不远死也,而所以死者,惑于饵也。苏秦、智伯之智,非不足以辟辱远死也,而所以死者,惑于贪利不止也。是以圣人制礼节欲,取于民有度,使之以时,用之有止,故志不溢,行不骄,常与道俱而不失,故天下承而不绝。昔者齐桓公九合诸侯,一匡天下,至于葵丘之会,有骄矜之志,畔者九国。吴王夫差兵无敌于天下,勇彊以轻诸侯,陵齐晋,故遂以杀身亡国。夏育、太史噭叱呼骇三军,然而身死于庸夫。此皆乘至盛而不返道理,不居卑退处俭约之患也。夫商君为秦孝公明法令,禁奸本,尊爵必赏,有罪必罚,平权衡,正度量,调轻重,决裂阡陌,以静生民之业而一其俗,劝民耕农利土,一室无二事,力田稸积,习战陈之事,是以兵动而地广,兵休而国富,故秦无敌于天下,立威诸侯,成秦国之业。功已成矣,而遂以车裂。楚地方数千里,持戟百万,白起率数万之师以与楚战,一战举鄢郢以烧夷陵,再战南并蜀汉。又越韩、魏而攻彊赵,北阬马服,诛屠四十馀万之众,尽之于长平之下,流血成川,沸声若雷,遂入围邯郸,使秦有帝业。楚、赵天下之彊国而秦之仇敌也,自是之后,楚、赵皆慑伏不敢攻秦者,白起之势也。身所服者七十馀城,功已成矣,而遂赐剑死于杜邮。吴起为楚悼王立法,卑减大臣之威重,罢无能,废无用,损不急之官,塞私门之请,一楚国之俗,禁游客之民,精耕战之士,南收杨越,北并陈、蔡,破横散从,使驰说之士无所开其口,禁朋党以励百姓,定楚国之政,兵震天下,威服诸侯。功已成矣,而卒枝解。大夫种为越王深谋远计,免会稽之危,以亡为存,因辱为荣,垦草入邑,辟地殖穀,率四方之士,专上下之力,辅句践之贤,报夫差之雠,卒擒劲吴。令越成霸。功已彰而信矣,句践终负而杀之。此四子者,功成不去,祸至于此。此所谓信而不能诎,往而不能返者也。范蠡知之,超然辟世,长为陶硃公。君独不观夫博者乎?或欲大投,或欲分功,此皆君之所明知也。今君相秦,计不下席,谋不出廊庙,坐制诸侯,利施三川,以实宜阳,决羊肠之险,塞太行之道,又斩范、中行之涂,六国不得合从,栈道千里,通于蜀汉,使天下皆畏秦,秦之欲得矣,君之功极矣,此亦秦之分功之时也。如是而不退,则商君、白公、吴起、大夫种是也。吾闻之,‘鉴于水者见面之容,鉴于人者知吉与凶’。书曰‘成功之下,不可久处’。四子之祸,君何居焉?君何不以此时归相印,让贤者而授之,退而岩居川观,必有伯夷之廉,长为应侯。世世称孤,而有许由、延陵季子之让,乔松之寿,孰与以祸终哉?即君何居焉?忍不能自离,疑不能自决,必有四子之祸矣。易曰‘亢龙有悔’,此言上而不能下,信而不能诎,往而不能自返者也。原君孰计之!”应侯曰:“善。吾闻‘欲而不知,失其所以欲;有而不知,失其所以有’。先生幸教,睢敬受命。’于是乃延入坐,为上客。

后数日,入朝,言于秦昭王曰:“客新有从山东来者曰蔡泽,其人辩士,明于三王之事,五伯之业,世俗之变,足以寄秦国之政。臣之见人甚众,莫及,臣不如也。臣敢以闻。”秦昭王召见,与语,大说之,拜为客卿。应侯因谢病请归相印。昭王彊起应侯,应侯遂称病笃。范睢免相,昭王新说蔡泽计画,遂拜为秦相,东收周室。

蔡泽相秦数月,人或恶之,惧诛,乃谢病归相印,号为纲成君。居秦十馀年,事昭王、孝文王、庄襄王。卒事始皇帝,为秦使于燕,三年而燕使太子丹入质于秦。

太史公曰:韩子称“长袖善舞,多钱善贾”,信哉是言也!范睢、蔡泽世所谓一切辩士,然游说诸侯至白首无所遇者,非计策之拙,所为说力少也。及二人羁旅入秦,继踵取卿相,垂功于天下者,固彊弱之势异也。然士亦有偶合,贤者多如此二子,不得尽意,岂可胜道哉!然二子不困戹,恶能激乎?

应侯始困,讬载而西,说行计立,贵平宠稽。倚秦市赵,卒报魏齐。纲成辩智,范睢招携。势利倾夺,一言成蹊。
\end{yuanwen}

\part{卷八十}
\chapter{乐毅列传第二十}

黄道周:「予观古人尚哲简戆,因事蝉脱,如季札、蘧瑗、晏婴、乐毅之流,皆值祸难飘然,有以自立。」

\begin{yuanwen}
乐毅者,其先祖曰乐羊。乐羊为魏文侯将,伐取中山,魏文侯封乐羊以灵寿。乐羊死,葬于灵寿,其后子孙因\footnote{于是。}家焉。中山复国,至赵武灵王时复灭中山,而乐氏后有乐毅。
\end{yuanwen}

乐毅,他的祖先名叫乐羊。乐羊做过魏文侯的将军,曾经攻打占领了中山国,魏文侯把灵寿作为封地赐给了乐羊。乐羊死后,就安葬在灵寿,他的子孙就把家安在了灵寿。中山国后来复国,但是到赵武灵王时又一次灭了中山国,这时乐家的后代中就有了乐毅这个人。

\begin{yuanwen}
乐毅贤,好兵,赵人举之。及武灵王有沙丘之乱,乃去赵適魏。闻燕昭王以子之之乱而齐大败燕,燕昭王怨齐,未尝一日而忘报齐也。燕国小,辟\footnote{同“僻”。偏僻。}远,力不能制,于是屈身下士,先礼郭隗以招贤者。乐毅于是为魏昭王使于燕,燕王以客礼待之。乐毅辞让,遂委质\footnote{臣下向君主敬献礼物,表示献身称“委质”。质,通“贽”。}为臣,燕昭王以为亚卿,久之。
\end{yuanwen}

乐毅是一个贤能的人,喜欢军事,赵国人推荐他出来做官。等到赵武灵王在沙丘的行宫被围困饿死以后,乐毅就离开赵国到了魏国。他听说燕昭王由于子之造成的动乱,燕国被齐国打得大败,燕昭王恨透了齐国,没有一天忘记报复齐国。燕国这个国家很小,地理位置偏僻,实力根本无法打败齐国,于是燕昭王就降低身份恭敬地招纳贤士,先用周到的礼仪来对待郭隗,借此来招揽更多贤人。正在这个时候,乐毅作为魏昭王的使者来到了燕国,燕王就用招待宾客礼节来招待他。乐毅先是推辞、谦让,后来就答应委身为臣,燕昭王让他当了亚卿,乐毅担任这个职务的时间很长。

\begin{yuanwen}
当是时,齐湣王彊,南败楚相唐眛于重丘,西摧三晋于观津,遂与三晋击秦,助赵灭中山,破宋,广地千馀里。与秦昭王争重为帝,已而复归之。诸侯皆欲背秦而服于齐。湣王自矜,百姓弗堪。于是燕昭王问伐齐之事。乐毅对曰:“齐,霸国之馀业也,地大人众,未易独攻也。王必欲伐之,莫如与赵及楚、魏。”于是使乐毅约赵惠文王,别使连楚、魏,令赵嚪说秦以伐齐之利。诸侯害齐湣王之骄暴,皆争合从与燕伐齐。乐毅还报,燕昭王悉起兵,使乐毅为上将军,赵惠文王以相国印授乐毅。乐毅于是并护赵、楚、韩、魏、燕之兵以伐齐,破之济西。诸侯兵罢归,而燕军乐毅独追,至于临菑。齐湣王之败济西,亡走,保于莒。乐毅独留徇齐,齐皆城守。乐毅攻入临菑,尽取齐宝财物祭器输之燕。燕昭王大说,亲至济上劳军,行赏飨士,封乐毅于昌国,号为昌国君。于是燕昭王收齐卤获以归,而使乐毅复以兵平齐城之不下者。
\end{yuanwen}

王夫之:「有良将而不用,赵黜廉颇而亡,燕疑乐毅而偾。」

\begin{yuanwen}
乐毅留徇齐五岁,下齐七十馀城,皆为郡县以属燕,唯独莒、即墨未服。会燕昭王死,子立为燕惠王。惠王自为太子时尝不快于乐毅,及即位,齐之田单闻之,乃纵反间于燕,曰:“齐城不下者两城耳。然所以不早拔者,闻乐毅与燕新王有隙,欲连兵且留齐,南面而王齐。齐之所患,唯恐他将之来。”于是燕惠王固已疑乐毅,得齐反间,乃使骑劫代将,而召乐毅。乐毅知燕惠王之不善代之,畏诛,遂西降赵。赵封乐毅于观津,号曰望诸君。尊宠乐毅以警动于燕、齐。
\end{yuanwen}\begin{yuanwen}

\end{yuanwen}\begin{yuanwen}

\end{yuanwen}\begin{yuanwen}

\end{yuanwen}\begin{yuanwen}

\end{yuanwen}\begin{yuanwen}

\end{yuanwen}\begin{yuanwen}

\end{yuanwen}\begin{yuanwen}

\end{yuanwen}\begin{yuanwen}

\end{yuanwen}\begin{yuanwen}

\end{yuanwen}\begin{yuanwen}

\end{yuanwen}\begin{yuanwen}

\end{yuanwen}\begin{yuanwen}

\end{yuanwen}\begin{yuanwen}

\end{yuanwen}\begin{yuanwen}

\end{yuanwen}\begin{yuanwen}

\end{yuanwen}\begin{yuanwen}

\end{yuanwen}\begin{yuanwen}

\end{yuanwen}\begin{yuanwen}

\end{yuanwen}\begin{yuanwen}

\end{yuanwen}\begin{yuanwen}

\end{yuanwen}\begin{yuanwen}

\end{yuanwen}\begin{yuanwen}

\end{yuanwen}\begin{yuanwen}

\end{yuanwen}\begin{yuanwen}

\end{yuanwen}\begin{yuanwen}

\end{yuanwen}\begin{yuanwen}

\end{yuanwen}\begin{yuanwen}

\end{yuanwen}\begin{yuanwen}

\end{yuanwen}\begin{yuanwen}

\end{yuanwen}\begin{yuanwen}

\end{yuanwen}\begin{yuanwen}

\end{yuanwen}\begin{yuanwen}

\end{yuanwen}\begin{yuanwen}

\end{yuanwen}\begin{yuanwen}

\end{yuanwen}\begin{yuanwen}

\end{yuanwen}\begin{yuanwen}

\end{yuanwen}\begin{yuanwen}

\end{yuanwen}\begin{yuanwen}

\end{yuanwen}\begin{yuanwen}

\end{yuanwen}\begin{yuanwen}

\end{yuanwen}\begin{yuanwen}

\end{yuanwen}
\begin{yuanwen}


齐田单后与骑劫战,果设诈诳燕军,遂破骑劫于即墨下,而转战逐燕,北至河上,尽复得齐城,而迎襄王于莒,入于临菑。

燕惠王后悔使骑劫代乐毅,以故破军亡将失齐;又怨乐毅之降赵,恐赵用乐毅而乘燕之弊以伐燕。燕惠王乃使人让乐毅,且谢之曰:“先王举国而委将军,将军为燕破齐,报先王之雠,天下莫不震动,寡人岂敢一日而忘将军之功哉!会先王弃群臣,寡人新即位,左右误寡人。寡人之使骑劫代将军,为将军久暴露于外,故召将军且休,计事。将军过听,以与寡人有隙,遂捐燕归赵。将军自为计则可矣,而亦何以报先王之所以遇将军之意乎?”乐毅报遗燕惠王书曰:

臣不佞,不能奉承王命,以顺左右之心,恐伤先王之明,有害足下之义,故遁逃走赵。今足下使人数之以罪,臣恐侍御者不察先王之所以畜幸臣之理,又不白臣之所以事先王之心,故敢以书对。

臣闻贤圣之君不以禄私亲,其功多者赏之,其能当者处之。故察能而授官者,成功之君也;论行而结交者,立名之士也。臣窃观先王之举也,见有高世主之心,故假节于魏,以身得察于燕。先王过举,厕之宾客之中,立之群臣之上,不谋父兄,以为亚卿。臣窃不自知,自以为奉令承教,可幸无罪,故受令而不辞。

先王命之曰:“我有积怨深怒于齐,不量轻弱,而欲以齐为事。”臣曰:“夫齐,霸国之馀业而最胜之遗事也。练于兵甲,习于战攻。王若欲伐之,必与天下图之。与天下图之,莫若结于赵。且又淮北、宋地,楚魏之所欲也,赵若许而约四国攻之,齐可大破也。”先王以为然,具符节南使臣于赵。顾反命,起兵击齐。以天之道,先王之灵,河北之地随先王而举之济上。济上之军受命击齐,大败齐人。轻卒锐兵,长驱至国。齐王遁而走莒,仅以身免;珠玉财宝车甲珍器尽收入于燕。齐器设于宁台,大吕陈于元英,故鼎反乎室,蓟丘之植植于汶篁,自五伯已来,功未有及先王者也。先王以为慊于志,故裂地而封之,使得比小国诸侯。臣窃不自知,自以为奉命承教,可幸无罪,是以受命不辞。

臣闻贤圣之君,功立而不废,故著于春秋;蚤知之士,名成而不毁,故称于后世。若先王之报怨雪耻,夷万乘之彊国,收八百岁之蓄积,及至弃群臣之日,馀教未衰,执政任事之臣,脩法令,慎庶孽,施及乎萌隶,皆可以教后世。

臣闻之,善作者不必善成,善始者不必善终。昔伍子胥说听于阖闾,而吴王远迹至郢;夫差弗是也,赐之鸱夷而浮之江。吴王不寤先论之可以立功,故沈子胥而不悔;子胥不蚤见主之不同量,是以至于入江而不化。

夫免身立功,以明先王之迹,臣之上计也。离毁辱之诽谤,堕先王之名,臣之所大恐也。临不测之罪,以幸为利,义之所不敢出也。

臣闻古之君子,交绝不出恶声;忠臣去国,不絜其名。臣虽不佞,数奉教于君子矣。恐侍御者之亲左右之说,不察疏远之行,故敢献书以闻,唯君王之留意焉。

于是燕王复以乐毅子乐间为昌国君;而乐毅往来复通燕,燕、赵以为客卿。乐毅卒于赵。

乐间居燕三十馀年,燕王喜用其相栗腹之计,欲攻赵,而问昌国君乐间。乐间曰:“赵,四战之国也,其民习兵,伐之不可。”燕王不听,遂伐赵。赵使廉颇击之,大破栗腹之军于鄗,禽栗腹、乐乘。乐乘者,乐间之宗也。于是乐间奔赵,赵遂围燕。燕重割地以与赵和,赵乃解而去。

燕王恨不用乐间,乐间既在赵,乃遗乐间书曰:“纣之时,箕子不用,犯谏不怠,以冀其听;商容不达,身祇辱焉,以冀其变。及民志不入,狱囚自出,然后二子退隐。故纣负桀暴之累,二子不失忠圣之名。何者?其忧患之尽矣。今寡人虽愚,不若纣之暴也;燕民虽乱,不若殷民之甚也。室有语,不相尽,以告邻里。二者,寡人不为君取也。”

乐间、乐乘怨燕不听其计,二人卒留赵。赵封乐乘为武襄君。

其明年,乐乘、廉颇为赵围燕,燕重礼以和,乃解。后五岁,赵孝成王卒。襄王使乐乘代廉颇。廉颇攻乐乘,乐乘走,廉颇亡入魏。其后十六年而秦灭赵。

其后二十馀年,高帝过赵,问:“乐毅有后世乎?”对曰:“有乐叔。”高帝封之乐卿,号曰华成君。华成君,乐毅之孙也。而乐氏之族有乐瑕公、乐臣公,赵且为秦所灭,亡之齐高密。乐臣公善修黄帝、老子之言,显闻于齐,称贤师。

太史公曰:始齐之蒯通及主父偃读乐毅之报燕王书,未尝不废书而泣也。乐臣公学黄帝、老子,其本师号曰河上丈人,不知其所出。河上丈人教安期生,安期生教毛翕公,毛翕公教乐瑕公,乐瑕公教乐臣公,乐臣公教盖公。盖公教于齐高密、胶西,为曹相国师。

昌国忠谠,人臣所无。连兵五国,济西为墟。燕王受间,空闻报书。义士慷慨,明君轼闾。间、乘继将,芳规不渝。
\end{yuanwen}

\part{卷八十一}
\chapter{廉颇蔺相如列传第二十一}

朱熹:「和氏璧乃赵国世传之宝,若骤然被人夺,则国势亦不振矣。古人传国,皆以宝玉之属为重,若子孙不能谨守,即为不孝。当时秦王虽强,相如亦料其不敢杀己,若在他人则惧秦而不敢去矣,相如岂孟浪作事者哉!」苏辙:「相如非战国之士也,以死行义,而不屈于强秦;以礼为国,而不校于廉颇。其处刚柔进退之际,颇类学道者。使居平世,可以为天子大臣矣。」

\begin{yuanwen}
廉颇者,赵之良将也。赵惠文王十六年,廉颇为赵将伐齐,大破之,取阳晋,拜为上卿,以勇气闻于诸侯。蔺相如者,赵人也,为赵宦者令缪贤舍人。
\end{yuanwen}

廉颇,是赵国的优秀将领。赵惠文王十六年(前283年),廉颇作为赵国的主将带领军队攻打齐国,把齐国的军队打得大败,占领了阳晋城,被赵王任命为上卿,以作战勇敢闻名于诸侯。蔺相如,是赵国人,在赵国宦者令缪贤的家里做门客。

\begin{yuanwen}
赵惠文王时,得楚和氏璧。秦昭王闻之,使人遗赵王书,愿以十五城请易璧。赵王与大将军廉颇诸大臣谋:欲予秦,秦城恐不可得,徒见欺;欲勿予,即患秦兵之来。计未定,求人可使报秦者,未得。宦者令缪贤曰:“臣舍人蔺相如可使。”王问:“何以知之?”对曰:“臣尝有罪,窃计欲亡走燕,臣舍人相如止臣,曰:‘君何以知燕王?’臣语曰:‘臣尝从大王与燕王会境上,燕王私握臣手,曰“原结友”。以此知之,故欲往。’相如谓臣曰:‘夫赵彊而燕弱,而君幸于赵王,故燕王欲结于君。今君乃亡赵走燕,燕畏赵,其势必不敢留君,而束君归赵矣。君不如肉袒伏斧质请罪,则幸得脱矣。’臣从其计,大王亦幸赦臣。臣窃以为其人勇士,有智谋,宜可使。”于是王召见,问蔺相如曰:“秦王以十五城请易寡人之璧,可予不?”相如曰:“秦彊而赵弱,不可不许。”王曰:“取吾璧,不予我城,柰何?”相如曰:“秦以城求璧而赵不许,曲在赵。赵予璧而秦不予赵城,曲在秦。均之二策,宁许以负秦曲。”王曰:“谁可使者?”相如曰:“王必无人,臣原奉璧往使。城入赵而璧留秦;城不入,臣请完璧归赵。”赵王于是遂遣相如奉璧西入秦。

赵王与大将军廉颇诸大臣谋:欲予秦,秦城恐不可得,徒见欺;欲勿予,即患秦兵之来。计未定,求人可使报秦者,未得。宦者令缪贤曰:“臣舍人蔺相如可使。”王问:“何以知之?”对曰:“臣尝有罪,窃计欲亡走燕,臣舍人相如止臣,曰:‘君何以知燕王?’臣语曰:‘臣尝从大王与燕王会境上,燕王私握臣手,曰“愿结友”。以此知之,故欲往。’相如谓臣曰:‘夫赵强而燕弱,而君幸于赵王,故燕王欲结于君。今君乃亡赵走燕,燕畏赵,其势必不敢留君,而束君归赵矣。君不如肉袒【肉袒:脱去上衣,露出上身。】 伏斧质【斧质:古代杀人用的刑具。质,同“锧”,铁砧板。】 请罪,则幸得脱矣。’臣从其计,大王亦幸赦臣。臣窃以为其人勇士,有智谋,宜可使。”于是王召见,问蔺相如曰:“秦王以十五城请易寡人之璧,可予不【不:通“否”。】 ?”相如曰:“秦强而赵弱,不可不许。”王曰:“取吾璧,不予我城,奈何?”相如曰:“秦以城求璧而赵不许,曲在赵。赵予璧而秦不予赵城,曲在秦。均之二策,宁许以负秦曲。”王曰:“谁可使者?”相如曰:“王必无人,臣愿奉璧往使。城入赵而璧留秦;城不入,臣请完璧归赵。”赵王于是遂遣相如奉璧西入秦。

秦王坐章台见相如,相如奉璧奏秦王。秦王大喜,传以示美人及左右,左右皆呼万岁。相如视秦王无意偿赵城,乃前曰:“璧有瑕,请指示王。”王授璧,相如因持璧卻立,倚柱,怒发上冲冠,谓秦王曰:“大王欲得璧,使人发书至赵王,赵王悉召群臣议,皆曰‘秦贪,负其彊,以空言求璧,偿城恐不可得’。议不欲予秦璧。臣以为布衣之交尚不相欺,况大国乎!且以一璧之故逆彊秦之驩,不可。于是赵王乃斋戒五日,使臣奉璧,拜送书于庭。何者?严大国之威以修敬也。今臣至,大王见臣列观,礼节甚倨;得璧,传之美人,以戏弄臣。臣观大王无意偿赵王城邑,故臣复取璧。大王必欲急臣,臣头今与璧俱碎于柱矣!”相如持其璧睨柱,欲以击柱。秦王恐其破璧,乃辞谢固请,召有司案图,指从此以往十五都予赵。相如度秦王特以诈详为予赵城,实不可得,乃谓秦王曰:“和氏璧,天下所共传宝也,赵王恐,不敢不献。赵王送璧时,斋戒五日,今大王亦宜斋戒五日,设九宾于廷,臣乃敢上璧。”秦王度之,终不可彊夺,遂许斋五日,舍相如广成传。相如度秦王虽斋,决负约不偿城,乃使其从者衣褐,怀其璧,从径道亡,归璧于赵。

秦王斋五日后,乃设九宾礼于廷,引赵使者蔺相如。相如至,谓秦王曰:“秦自缪公以来二十馀君,未尝有坚明约束者也。臣诚恐见欺于王而负赵,故令人持璧归,间至赵矣。且秦彊而赵弱,大王遣一介之使至赵,赵立奉璧来。今以秦之彊而先割十五都予赵,赵岂敢留璧而得罪于大王乎?臣知欺大王之罪当诛,臣请就汤镬,唯大王与群臣孰计议之。”秦王与群臣相视而嘻。左右或欲引相如去,秦王因曰:“今杀相如,终不能得璧也,而绝秦赵之驩,不如因而厚遇之,使归赵,赵王岂以一璧之故欺秦邪!”卒廷见相如,毕礼而归之。

相如既归,赵王以为贤大夫使不辱于诸侯,拜相如为上大夫。秦亦不以城予赵,赵亦终不予秦璧。

其后秦伐赵,拔石城。明年,复攻赵,杀二万人。

秦王使使者告赵王,欲与王为好会于西河外渑池。赵王畏秦,欲毋行。廉颇、蔺相如计曰:“王不行,示赵弱且怯也。”赵王遂行,相如从。廉颇送至境,与王诀曰:“王行,度道里会遇之礼毕,还,不过三十日。三十日不还,则请立太子为王。以绝秦望。”王许之,遂与秦王会渑池。秦王饮酒酣,曰:“寡人窃闻赵王好音,请奏瑟。”赵王鼓瑟。秦御史前书曰“某年月日,秦王与赵王会饮,令赵王鼓瑟”。蔺相如前曰:“赵王赵王窃闻秦王善为秦声,请奏盆缻秦王,以相娱乐。”秦王怒,不许。于是相如前进缻,因跪请秦王。秦王不肯击缻。相如曰:“五步之内,相如请得以颈血溅大王矣!”左右欲刃相如,相如张目叱之,左右皆靡。于是秦王不怿,为一击鲊。相如顾召赵御史书曰“某年月日,秦王为赵王击鲊”。秦之群臣曰:“请以赵十五城为秦王寿”。蔺相如亦曰:“请以秦之咸阳为赵王寿。”秦王竟酒,终不能加胜于赵。赵亦盛设兵以待秦,秦不敢动。

既罢归国,以相如功大,拜为上卿,位在廉颇之右。廉颇曰:“我为赵将,有攻城野战之大功,而蔺相如徒以口舌为劳,而位居我上,且相如素贱人,吾羞,不忍为之下。”宣言曰:“我见相如,必辱之。”相如闻,不肯与会。相如每朝时,常称病,不欲与廉颇争列。已而相如出,望见廉颇,相如引车避匿。于是舍人相与谏曰:“臣所以去亲戚而事君者,徒慕君之高义也。今君与廉颇同列,廉君宣恶言而君畏匿之,恐惧殊甚,且庸人尚羞之,况于将相乎!臣等不肖,请辞去。”蔺相如固止之,曰:“公之视廉将军孰与秦王?”曰:“不若也。”相如曰:“夫以秦王之威,而相如廷叱之,辱其群臣,相如虽驽,独畏廉将军哉?顾吾念之,彊秦之所以不敢加兵于赵者,徒以吾两人在也。今两虎共斗,其势不俱生。吾所以为此者,以先国家之急而后私雠也。”廉颇闻之,肉袒负荆,因宾客至蔺相如门谢罪。曰:“鄙贱之人,不知将军宽之至此也。”卒相与驩,为刎颈之交。

是岁,廉颇东攻齐,破其一军。居二年,廉颇复伐齐几,拔之。后三年,廉颇攻魏之防陵、安阳,拔之。后四年,蔺相如将而攻齐,至平邑而罢。其明年,赵奢破秦军阏与下。

赵奢者,赵之田部吏也。收租税而平原君家不肯出租,奢以法治之,杀平原君用事者九人。平原君怒,将杀奢。奢因说曰:“君于赵为贵公子,今纵君家而不奉公则法削,法削则国弱,国弱则诸侯加兵,诸侯加兵是无赵也,君安得有此富乎?以君之贵,奉公如法则上下平,上下平则国彊,国彊则赵固,而君为贵戚,岂轻于天下邪?”平原君以为贤,言之于王。王用之治国赋,国赋大平,民富而府库实。

秦伐韩,军于阏与。王召廉颇而问曰:“可救不?”对曰:“道远险狭,难救。”又召乐乘而问焉,乐乘对如廉颇言。又召问赵奢,奢对曰:“其道远险狭,譬之犹两鼠斗于穴中,将勇者胜。”王乃令赵奢将,救之。

兵去邯郸三十里,而令军中曰:“有以军事谏者死。”秦军军武安西,秦军鼓譟勒兵,武安屋瓦尽振。军中候有一人言急救武安,赵奢立斩之。坚壁,留二十八日不行,复益增垒。秦间来入,赵奢善食而遣之。间以报秦将,秦将大喜曰:“夫去国三十里而军不行,乃增垒,阏与非赵地也。”赵奢既已遣秦间,卷甲而趋之,二日一夜至,今善射者去阏与五十里而军。军垒成,秦人闻之,悉甲而至。军士许历请以军事谏,赵奢曰:“内之。”许历曰:“秦人不意赵师至此,其来气盛,将军必厚集其阵以待之。不然,必败。”赵奢曰:“请受令。”许历曰:“请就鈇质之诛。”赵奢曰:“胥后令邯郸。”许历复请谏,曰:“先据北山上者胜,后至者败。”赵奢许诺,即发万人趋之。秦兵后至,争山不得上,赵奢纵兵击之,大破秦军。秦军解而走,遂解阏与之围而归。

赵惠文王赐奢号为马服君,以许历为国尉。赵奢于是与廉颇、蔺相如同位。

后四年,赵惠文王卒,子孝成王立。七年,秦与赵兵相距长平,时赵奢已死,而蔺相如病笃,赵使廉颇将攻秦,秦数败赵军,赵军固壁不战。秦数挑战,廉颇不肯。赵王信秦之间。秦之间言曰:“秦之所恶,独畏马服君赵奢之子赵括为将耳。”赵王因以括为将,代廉颇。蔺相如曰:“王以名使括,若胶柱而鼓瑟耳。括徒能读其父书传,不知合变也。”赵王不听,遂将之。

赵括自少时学兵法,言兵事,以天下莫能当。尝与其父奢言兵事,奢不能难,然不谓善。括母问奢其故,奢曰:“兵,死地也,而括易言之。使赵不将括即已,若必将之,破赵军者必括也。”及括将行,其母上书言于王曰:“括不可使将。”王曰:“何以?”对曰:“始妾事其父,时为将,身所奉饭饮而进食者以十数,所友者以百数,大王及宗室所赏赐者尽以予军吏士大夫,受命之日,不问家事。今括一旦为将,东向而朝,军吏无敢仰视之者,王所赐金帛,归藏于家,而日视便利田宅可买者买之。王以为何如其父?父子异心,原王勿遣。”王曰:“母置之,吾已决矣。”括母因曰:“王终遣之,即有如不称,妾得无随坐乎?”王许诺。

赵括既代廉颇,悉更约束,易置军吏。秦将白起闻之,纵奇兵,详败走,而绝其粮道,分断其军为二,士卒离心。四十馀日,军饿,赵括出锐卒自博战,秦军射杀赵括。括军败,数十万之众遂降秦,秦悉阬之。赵前后所亡凡四十五万。明年,秦兵遂围邯郸,岁馀,几不得脱。赖楚、魏诸侯来救,乃得解邯郸之围。赵王亦以括母先言,竟不诛也。

自邯郸围解五年,而燕用栗腹之谋,曰“赵壮者尽于长平,其孤未壮”,举兵击赵。赵使廉颇将,击,大破燕军于鄗,杀栗腹,遂围燕。燕割五城请和,乃听之。赵以尉文封廉颇为信平君,为假相国。

廉颇之免长平归也,失势之时,故客尽去。及复用为将,客又复至。廉颇曰:“客退矣!”客曰:“吁!君何见之晚也?夫天下以市道交,君有势,我则从君,君无势则去,此固其理也,有何怨乎?”居六年,赵使廉颇伐魏之繁阳,拔之。

赵孝成王卒,子悼襄王立,使乐乘代廉颇。廉颇怒,攻乐乘,乐乘走。廉颇遂奔魏之大梁。其明年,赵乃以李牧为将而攻燕,拔武遂、方城。

廉颇居梁久之,魏不能信用。赵以数困于秦兵,赵王思复得廉颇,廉颇亦思复用于赵。赵王使使者视廉颇尚可用否。廉颇之仇郭开多与使者金,令毁之。赵使者既见廉颇,廉颇为之一饭斗米,肉十斤,被甲上马,以示尚可用。赵使还报王曰:“廉将军虽老,尚善饭,然与臣坐,顷之三遗矢矣。”赵王以为老,遂不召。

楚闻廉颇在魏,阴使人迎之。廉颇一为楚将,无功,曰:“我思用赵人。”廉颇卒死于寿春。

李牧者,赵之北边良将也。常居代雁门,备匈奴。以便宜置吏,市租皆输入莫府,为士卒费。日击数牛飨士,习射骑,谨烽火,多间谍,厚遇战士。为约曰:“匈奴即入盗,急入收保,有敢捕虏者斩。”匈奴每入,烽火谨,辄入收保,不敢战。如是数岁,亦不亡失。然匈奴以李牧为怯,虽赵边兵亦以为吾将怯。赵王让李牧,李牧如故。赵王怒,召之,使他人代将。

岁馀,匈奴每来,出战。出战,数不利,失亡多,边不得田畜。复请李牧。牧杜门不出,固称疾。赵王乃复彊起使将兵。牧曰:“王必用臣,臣如前,乃敢奉令。”王许之。

李牧至,如故约。匈奴数岁无所得。终以为怯。边士日得赏赐而不用,皆原一战。于是乃具选车得千三百乘,选骑得万三千匹,百金之士五万人,彀者十万人,悉勒习战。大纵畜牧,人民满野。匈奴小入,详北不胜,以数千人委之。单于闻之,大率众来入。李牧多为奇陈,张左右翼击之,大破杀匈奴十馀万骑。灭襜褴,破东胡,降林胡,单于奔走。其后十馀岁,匈奴不敢近赵边城。

赵悼襄王元年,廉颇既亡入魏,赵使李牧攻燕,拔武遂、方城。居二年,庞暖破燕军,杀剧辛。后七年,秦破杀赵将扈辄于武遂,斩首十万。赵乃以李牧为大将军,击秦军于宜安,大破秦军,走秦将桓齮。封李牧为武安君。居三年,秦攻番吾,李牧击破秦军,南距韩、魏。

赵王迁七年,秦使王翦攻赵,赵使李牧、司马尚御之。秦多与赵王宠臣郭开金,为反间,言李牧、司马尚欲反。赵王乃使赵葱及齐将颜聚代李牧。李牧不受命,赵使人微捕得李牧,斩之。废司马尚。后三月,王翦因急击赵,大破杀赵葱,虏赵王迁及其将颜聚,遂灭赵。

太史公曰:知死必勇,非死者难也,处死者难。方蔺相如引璧睨柱,及叱秦王左右,势不过诛,然士或怯懦而不敢发。相如一奋其气,威信敌国,退而让颇,名重太山,其处智勇,可谓兼之矣!

清梠凛凛,壮气熊熊。各竭诚义,递为雌雄。和璧聘返,渑池好通。负荆知惧,屈节推工。安边定策,颇、牧之功。
\end{yuanwen}

\part{卷八十二}

\chapter{田单列传第二十二}

苏轼:“田单使人食必祭,以致飞鸟;又设为神师,皆近儿戏,无益于事。盖先以疑似置人心腹中,则夜见火牛龙文,足以骇动,取一时之胜,此其本意也。”

\begin{yuanwen}
田单者,齐诸田\footnote{指齐王田氏宗族的各个分支。}疏属\footnote{血缘比较远的宗族。}也。湣王时,单为临菑市掾\footnote{管理市场的官员。},不见知。及燕使乐毅伐破齐,齐湣王出奔,已而保莒城。燕师长驱平齐,而田单走安平,令其宗人尽断其车轴末而傅铁笼。已而燕军攻安平,城坏,齐人走,争涂\footnote{通“途”。},以轊\footnote{车轴末端。}折车败,为燕所虏,唯田单宗人以铁笼故得脱,东保即墨。燕既尽降齐城,唯独莒、即墨不下。燕军闻齐王在莒,并兵攻之。淖齿既杀湣王于莒,因坚守,距燕军,数年不下。燕引兵东围即墨,即墨大夫出与战,败死。城中相与推田单,曰:“安平之战,田单宗人以铁笼得全,习兵。”立以为将军,以即墨距燕。
\end{yuanwen}

田单,是齐国田氏王族远房本家。齐湣王在位的时候,田单是国都临菑的一名管理市政的掾吏,并不被齐国重用。等到燕昭王派乐毅带领军队攻破齐国的时候,齐湣王逃出临菑,不久又退到莒城自保。燕国军队长驱直入扫平齐国时,田单逃往安平,出逃之前,让族人们把马车车轴的两端全都锯掉,然后在外面包上了一层铁皮。不久,燕军进攻安平,城池被燕军攻破,齐国人都争着抢夺道路逃难,但因为马车互相碰撞而导致轴断车坏,很多人都被俘虏。唯独田单和他的族人,因为车轴被铁皮保护着,才得以逃脱,向东退守即墨城。燕军几乎已经让所有齐国城池投降了,单单只有莒城和即墨两座城没有攻克。燕国军队听说齐湣王躲在莒城,集中兵力攻打莒城。淖齿却在莒城中杀害了齐湣王,之后便坚定地守城,抵御燕军,燕军几年都没有攻下莒城。后来燕军的统帅带领兵马向东围攻即墨,守卫即墨的大夫带兵出城交战,结果战败身亡。城里人全都推举田单为齐军的统帅,说:“在安平的那一场战斗,田单的族人因为用铁包住车轴而得以保全性命,这说明田单对兵法很熟悉。”于是就拥立他做了齐国的将军,坚守即墨城抵御燕军。

\begin{yuanwen}
顷之,燕昭王卒,惠王立,与乐毅有隙。田单闻之,乃纵反间于燕,宣言曰:“齐王已死,城之不拔者二耳。乐毅畏诛而不敢归,以伐齐为名,实欲连兵南面而王齐。齐人未附,故且缓攻即墨以待其事。齐人所惧,唯恐他将之来,即墨残矣。”燕王以为然,使骑劫代乐毅。
\end{yuanwen}

没过多久,燕昭王死了,燕惠王继承了王位,燕惠王跟乐毅有嫌隙。田单听到这件事之后,就派人到燕国去施反间计,扬言说:“齐湣王已经死了,而且齐国没有被燕国攻占的城池只剩下两个。乐毅害怕被国君杀害,因此不敢回到国内,他以进攻齐国为名,实际上是想联合齐国的兵力,在齐国称王。由于齐国人还没有真心归顺他,所以暂且放慢攻打即墨,来等时机成熟以后在齐国称王。齐国人所害怕的事情,只是怕燕王会派遣其他将领前来,到时即墨城就会被攻破了。”燕王听信了这些话,派大将骑劫取代了乐毅。

\begin{yuanwen}
乐毅因归赵,燕人士卒忿。而田单乃令城中人食必祭其先祖于庭,飞鸟悉翔舞城中下食。燕人怪之。田单因宣言曰:“神来下教我。”乃令城中人曰:“当有神人为我师。”有一卒曰:“臣可以为师乎?”因反走。田单乃起,引还,东乡\footnote{同“向”。}坐,师事之。卒曰:“臣欺君,诚无能也。”田单曰:“子勿言也!”因师之。每出约束,必称神师。乃宣言曰:“吾唯惧燕军之劓\footnote{割掉鼻子。}所得齐卒,置之前行,与我战,即墨败矣。”燕人闻之,如其言。城中人见齐诸降者尽劓,皆怒,坚守,唯恐见得。单又纵反间曰:“吾惧燕人掘吾城外冢墓,僇先人,可为寒心。”燕军尽掘垄墓\footnote{坟墓。},烧死人。即墨人从城上望见,皆涕泣,俱欲出战,怒自十倍。
\end{yuanwen}

乐毅于是归附了赵国,燕国的百姓和官兵都因此感到悲愤不平。而田单就下令让即墨城里的老百姓每天在吃饭之前,一定要在家中的庭院祭祀自己的祖先,使得众多的飞鸟因争食祭祀的食物,在城上盘旋飞舞。燕国的官兵感到奇怪。田单趁机宣扬说:“神仙要下来传授我打败敌人的计策。”又对城里的人说:“会有位神人来做我的老师。”有一个士兵说:“像我这样的人可以当老师吗?”说完转身就要离开。田单于是赶紧起身,把他拉了回来,然后让他坐在面向东方的上位,用事奉老师的礼节事奉他。那个士兵说:“我欺骗了您,我确实没什么才能。”田单说道:“您别再说话了!”于是就把他尊为老师。每当田单要对士兵们发号施令时,就必然会说这是神师的旨意。又扬言说:“我最害怕的事情是燕国人把被俘虏的齐国士兵的鼻子割掉,然后让他们站在队伍的前面,跟我们交战,那么即墨必定会被攻克。”燕国军队的主将听说了田单的话,就按照田单话做了。即墨城里的官兵和百姓看到已经投降的齐国人被割掉了鼻子,都很生气,守卫即墨的决心更加坚定了,生怕被燕军俘虏。田单又派人在燕国军队中施展反间计,说道:“我害怕燕国人会把即墨城外的坟墓挖开,侮辱祖宗先人,会让我们感到非常寒心。”燕国人听到以后就把城外的坟墓全都挖开,并把死尸焚烧殆尽。即墨城中的百姓和官兵站在城头上看到了燕军焚烧尸骨的情景,全都放声大哭,全都想出城作战,愤怒的情绪增加了十倍。

\begin{yuanwen}
田单知士卒之可用,乃身操版插\footnote{筑土墙的工具和挖土的工具。},与士卒分功,妻妾编于行伍之间,尽散饮食飨士。令甲卒皆伏,使老弱女子乘城,遣使约降于燕,燕军皆呼万岁。田单又收民金,得千溢\footnote{同“镒”。古代重量单位,二十两为一镒。},令即墨富豪遗燕将,曰:“即墨即降,原无虏掠吾族家妻妾,令安堵\footnote{相安,安居。}。”燕将大喜,许之。燕军由此益懈。
\end{yuanwen}

田单知道此时可以令士兵出战,就亲自拿着夹板和铲子、铁锨,跟士卒们一起修建防御工事,并把他的妻妾编进队伍中,还把自己家里所有的食物都拿出来犒劳士卒。田单命令那些穿着铠甲的士兵都埋伏起来,让那些年老的、身体弱的、妇女和儿童登上城头进行防守,并派出使者到燕国的军营中去商量投降的事情,燕军都大声呼喊“万岁”。田单从百姓手中收集黄金,得到一千镒,让即墨城里有钱有势的人把这些金子送给燕国的主将,说:“即墨全城的人即将投降,希望燕国军队进入城中以后不要掳掠我们的妻子和姬妾,使我们可以平安地生活。”燕军主将非常高兴,答应了他们的请求。燕国士兵对即墨城的防备因此更加松懈了。

\begin{yuanwen}
田单乃收城中得千余牛,为绛缯衣,画以五彩龙文,束兵刃于其角,而灌脂束苇于尾,烧其端。凿城数十穴,夜纵牛,壮士五千人随其后。牛尾热,怒而奔燕军,燕军夜大惊。牛尾炬火\footnote{火把。}光明炫燿,燕军视之皆龙文,所触尽死伤。五千人因衔枚\footnote{形状如筷子,横衔口中,阻止出声,古时军中常用。}击之,而城中鼓噪从之,老弱皆击铜器为声,声动天地。燕军大骇,败走。齐人遂夷杀其将骑劫。燕军扰乱奔走,齐人追亡逐北,所过城邑皆畔燕而归田单,兵日益多,乘胜,燕日败亡,卒至河上,而齐七十余城皆复为齐。乃迎襄王于莒,入临菑而听政。

襄王封田单,号曰安平君。
\end{yuanwen}

田单就从即墨城里收集到一千多头牛,给它们披上红色绸绢做成的衣服,又在上面画上五颜六色的龙形图案,在牛角尖上绑上锋利的刀子,把渍满油脂的芦苇绑在牛尾上,点燃末端。又在城墙上凿出了几十个洞穴,在夜间把牛从洞穴里放出去,派出五千名精壮士兵跟在牛的后面。牛尾巴被火烧得发热,狂怒地朝燕军的营寨冲去,燕军在夜里大惊。牛尾上的火把将夜间照得通明如昼,燕军看到它们都是龙纹,所触及到的人非死即伤。五千名精壮士兵趁机在嘴里衔着枚,向燕军发动进攻,在城里的人们则敲鼓呐喊,跟在后面,老人、体弱者、妇女、小孩子也都击打铜器,声音震动了天地。燕军非常害怕,一路溃败奔逃。齐人于是杀死了燕军的主将骑劫。燕国的士兵混乱,溃散奔逃,齐国的军队紧紧地追赶逃亡败北的燕国军队,所经过的城邑全都背叛了燕军,转而归附田单,田单的士兵一天比一天多,借着胜利的势头继续追击,燕国的军队则一天天地溃败逃亡,最后一直退到了黄河边,而此前齐国被燕国所占领的七十多座城池全都被收复。于是田单到莒城迎请齐襄王,齐襄王进入临菑处理朝政。

齐襄王封赏田单,封号是“安平君”。

\begin{yuanwen}
太史公曰:兵以正合,以奇胜。善之者,出奇无穷。奇正还相生,如环之无端。夫始如处女,適\footnote{通“敌”。敌人。}人开户;后如脱兔,適不及距:其田单之谓邪!
\end{yuanwen}

太史公说:战争的策略在于一面与敌人正面交锋,一面用奇兵突袭制胜。擅长用兵的人,奇兵层出但又有无穷的变化。“正”和“奇”互相配合,就像一个浑然一体的圆环一样找不到开端。用兵的最初阶段,要像个柔弱、安静的处女,引诱敌人打开门户;之后就在时机到来的那一刹那,像脱逃罗网的兔子一样,使敌人来不及抵挡:这大概就是在说田单吧!

\begin{yuanwen}
初,淖齿之杀湣王也,莒人求湣王子法章,得之太史嬓之家,为人灌园。嬓女怜而善遇之。后法章私以情告女,女遂与通。及莒人共立法章为齐王,以莒距燕,而太史氏女遂为后,所谓“君王后”也。
\end{yuanwen}

当初,淖齿杀害齐湣王以后,莒城人访求齐湣王的儿子田法章,找到他的时候,他正在太史嬓家里帮着灌溉菜园。太史嬓的女儿觉得他可怜,就友好地招待他。后来法章暗地里把自己的身份告诉给她,她就跟法章私通了。等到莒城人共同拥立法章做了齐王,依靠莒城来抵御燕国军队,太史嬓的女儿就当上了王后,她就是人们所说的“君王后”。

\begin{yuanwen}
燕之初入齐,闻画邑人王蠋\footnote{zhú}贤,令军中曰“环画邑三十里无入”,以王蠋之故。已而使人谓蠋曰:“齐人多高子之义,吾以子为将,封子万家。”蠋固谢。燕人曰:“子不听,吾引三军而屠画邑。”王蠋曰:“忠臣不事二君,贞女不更二夫。齐王不听吾谏,故退而耕于野。国既破亡,吾不能存;今又劫之以兵为君将,是助桀为暴也。与其生而无义,固不如烹!”遂经\footnote{上吊,自缢。}其颈于树枝,自奋绝脰\footnote{颈,脖子。}而死。齐亡大夫闻之,曰:“王蠋,布衣也,义不北面于燕,况在位食禄者乎!”乃相聚如莒,求诸子,立为襄王。
\end{yuanwen}

燕军刚刚攻进齐国的时候,听说画邑人王蠋很贤能,主将曾经在全军发布命令说“画邑周围三十里以内不可以进入”,这都是因为王蠋是画邑人的缘故。没过多久,燕军主将派人对王蠋说道:“齐国有很多人都认为您品德高尚,我们想任命您当将军,还会赏赐您一万户的封邑。”王蠋坚决地拒绝了。燕国人说:“您若不肯接受,我们就会率领军队屠杀扫平画邑。”王蠋说:“忠心的大臣不会侍奉两个国家的君主,贞烈的女子不会再嫁第二个丈夫。齐王没有听从我的劝谏,因此我辞去官职隐居在乡野耕种。国家已经被攻破,我没有能力让它继续存在下去;现在又用武力来威胁我做你们的将领,这就像帮助夏桀王做暴虐的事情一样啊。与其活着做那些不合乎正义的事情,还不如被烹死!”于是把自己的脖子吊在树枝上,用力挣扎,扭断了脖子死去。齐国那些逃亡到外地的官员们听说了这件事情,说道:“王蠋,只是一个普通百姓,为了正义不肯向燕王称臣,更何况我们这些做官享有俸禄的人呢!”因此他们就聚集在一起,来到莒城,寻找齐湣王的儿子,拥立他做了齐王,也就是齐襄王。

\begin{yuanwen}
军法以正,实尚奇兵。断轴自免,反间先行。群鸟或众,五牛扬旌。卒破骑劫,皆复齐城。襄王嗣位,乃封安平。
\end{yuanwen}

\chapter{鲁仲连邹阳列传}

\begin{yuanwen}
鲁仲连者,齐人也。好奇伟俶傥之画策,而不肯仕宦任职,好持高节。游于赵。

赵孝成王时,而秦王使白起破赵长平之军前后四十馀万,秦兵遂东围邯郸。赵王恐,诸侯之救兵莫敢击秦军。魏安釐王使将军晋鄙救赵,畏秦,止于荡阴不进。魏王使客将军新垣衍间入邯郸,因平原君谓赵王曰:“秦所为急围赵者,前与齐湣王争彊为帝,已而复归帝;今齐已益弱,方今唯秦雄天下,此非必贪邯郸,其意欲复求为帝。赵诚发使尊秦昭王为帝,秦必喜,罢兵去。”平原君犹预未有所决。

此时鲁仲连適游赵,会秦围赵,闻魏将欲令赵尊秦为帝,乃见平原君曰:“事将柰何?”平原君曰:“胜也何敢言事!前亡四十万之众于外,今又内围邯郸而不能去。魏王使客将军新垣衍令赵帝秦,今其人在是。胜也何敢言事!”鲁仲连曰:“吾始以君为天下之贤公子也,吾乃今然后知君非天下之贤公子也。梁客新垣衍安在?吾请为君责而归之。”平原君曰:“胜请为绍介而见之于先生。”平原君遂见新垣衍曰:“东国有鲁仲连先生者,今其人在此,胜请为绍介,交之于将军。”新垣衍曰:“吾闻鲁仲连先生,齐国之高士也。衍人臣也,使事有职,吾不原见鲁仲连先生。”平原君曰:“胜既已泄之矣。”新垣衍许诺。

鲁连见新垣衍而无言。新垣衍曰:“吾视居此围城之中者,皆有求于平原君者也;今吾观先生之玉貌,非有求于平原君者也,曷为久居此围城之中而不去?”鲁仲连曰:“世以鲍焦为无从颂而死者,皆非也。众人不知,则为一身。彼秦者,弃礼义而上首功之国也,权使其士,虏使其民。彼即肆然而为帝,过而为政于天下,则连有蹈东海而死耳,吾不忍为之民也。所为见将军者,欲以助赵也。”

新垣衍曰:“先生助之将柰何?”鲁连曰:“吾将使梁及燕助之,齐、楚则固助之矣。”新垣衍曰:“燕则吾请以从矣;若乃梁者,则吾乃梁人也,先生恶能使梁助之?”鲁连曰:“梁未睹秦称帝之害故耳。使梁睹秦称帝之害,则必助赵矣。”

新垣衍曰:“秦称帝之害何如?”鲁连曰:“昔者齐威王尝为仁义矣,率天下诸侯而朝周。周贫且微,诸侯莫朝,而齐独朝之。居岁馀,周烈王崩,齐后往,周怒,赴于齐曰:‘天崩地坼,天子下席。东籓之臣因齐后至,则斮。’齐威王勃然怒曰:‘叱嗟,而母婢也!’卒为天下笑。故生则朝周,死则叱之,诚不忍其求也。彼天子固然,其无足怪。”

新垣衍曰:“先生独不见夫仆乎?十人而从一人者,宁力不胜而智不若邪?畏之也。”鲁仲连曰:“呜呼!梁之比于秦若仆邪?”新垣衍曰:“然。”鲁仲连曰:“吾将使秦王烹醢梁王。”新垣衍怏然不悦,曰:“噫嘻,亦太甚矣先生之言也!先生又恶能使秦王烹醢梁王?”鲁仲鲁曰:“固也,吾将言之。昔者九侯、鄂侯、文王,纣之三公也。九侯有子而好,献之于纣,纣以为恶,醢九侯。鄂侯争之彊,辩之疾,故脯鄂侯。文王闻之,喟然而叹,故拘之牖里之库百日,欲令之死。曷为与人俱称王,卒就脯醢之地?齐湣王之鲁,夷维子为执策而从,谓鲁人曰:‘子将何以待吾君?’鲁人曰:‘吾将以十太牢待子之君。’夷维子曰:‘子安取礼而来吾君?彼吾君者,天子也。天子巡狩,诸侯辟舍,纳筦籥,摄衽抱机,视膳于堂下,天子已食,乃退而听朝也。’鲁人投其籥,不果纳。不得入于鲁,将之薛,假途于邹。当是时,邹君死,湣王欲入吊,夷维子谓邹之孤曰:‘天子吊,主人必将倍殡棺,设北面于南方,然后天子南面吊也。’邹之群臣曰:‘必若此,吾将伏剑而死。’固不敢入于邹。邹、鲁之臣,生则不得事养,死则不得赙襚,然且欲行天子之礼于邹、鲁,邹、鲁之臣不果纳。今秦万乘之国也,梁亦万乘之国也。俱据万乘之国,各有称王之名,睹其一战而胜,欲从而帝之,是使三晋之大臣不如邹、鲁之仆妾也。且秦无已而帝,则且变易诸侯之大臣。彼将夺其所不肖而与其所贤,夺其所憎而与其所爱。彼又将使其子女谗妾为诸侯妃姬。处梁之宫。梁王安得晏然而已乎?而将军又何以得故宠乎?”

于是新垣衍起,再拜谢曰:“始以先生为庸人,吾乃今日知先生为天下之士也。吾请出,不敢复言帝秦。”秦将闻之,为卻军五十里。適会魏公子无忌夺晋鄙军以救赵,击秦军,秦军遂引而去。

于是平原君欲封鲁连,鲁连辞让者三,终不肯受。平原君乃置酒,酒酣起前,以千金为鲁连寿。鲁连笑曰:“所贵于天下之士者,为人排患释难解纷乱而无取也。即有取者,是商贾之事也,而连不忍为也。”遂辞平原君而去,终身不复见。

其后二十馀年,燕将攻下聊城,聊城人或谗之燕,燕将惧诛,因保守聊城,不敢归。齐田单攻聊城岁馀,士卒多死而聊城不下。鲁连乃为书,约之矢以射城中,遗燕将。书曰:

吾闻之,智者不倍时而弃利,勇士不却死而灭名,忠臣不先身而后君。今公行一朝之忿,不顾燕王之无臣,非忠也;杀身亡聊城,而威不信于齐,非勇也;功败名灭,后世无称焉,非智也。三者世主不臣,说士不载,故智者不再计,勇士不怯死。今死生荣辱,贵贱尊卑,此时不再至,原公详计而无与俗同。

且楚攻齐之南阳,魏攻平陆,而齐无南面之心,以为亡南阳之害小,不如得济北之利大,故定计审处之。今秦人下兵,魏不敢东面;衡秦之势成,楚国之形危;齐弃南阳,断右壤,定济北,计犹且为之也。且夫齐之必决于聊城,公勿再计。今楚魏交退于齐,而燕救不至。以全齐之兵,无天下之规,与聊城共据期年之敝,则臣见公之不能得也。且燕国大乱,君臣失计,上下迷惑,栗腹以十万之众五折于外,以万乘之国被围于赵,壤削主困,为天下僇笑。国敝而祸多,民无所归心。今公又以敝聊之民距全齐之兵,是墨翟之守也。食人炊骨,士无反外之心,是孙膑之兵也。能见于天下。虽然,为公计者,不如全车甲以报于燕。车甲全而归燕,燕王必喜;身全而归于国,士民如见父母,交游攘臂而议于世,功业可明。上辅孤主以制群臣,下养百姓以资说士,矫国更俗,功名可立也。亡意亦捐燕弃世,东游于齐乎?裂地定封,富比乎陶、卫,世世称孤,与齐久存,又一计也。此两计者,显名厚实也,原公详计而审处一焉。

且吾闻之,规小节者不能成荣名,恶小耻者不能立大功。昔者管夷吾射桓公中其钩,篡也;遗公子纠不能死,怯也;束缚桎梏,辱也。若此三行者,世主不臣而乡里不通。乡使管子幽囚而不出,身死而不反于齐,则亦名不免为辱人贱行矣。臧获且羞与之同名矣,况世俗乎!故管子不耻身在縲绁之中而耻天下之不治,不耻不死公子纠而耻威之不信于诸侯,故兼三行之过而为五霸首,名高天下而光烛邻国。曹子为鲁将,三战三北,而亡地五百里。乡使曹子计不反顾,议不还踵,刎颈而死,则亦名不免为败军禽将矣。曹子弃三北之耻,而退与鲁君计。桓公朝天下,会诸侯,曹子以一剑之任,枝桓公之心于坛坫之上,颜色不变,辞气不悖,三战之所亡一朝而复之,天下震动,诸侯惊骇,威加吴、越。若此二士者,非不能成小廉而行小节也,以为杀身亡躯,绝世灭后,功名不立,非智也。故去感忿之怨,立终身之名;弃忿悁之节,定累世之功。是以业与三王争流,而名与天壤相弊也。原公择一而行之。

燕将见鲁连书,泣三日,犹豫不能自决。欲归燕,已有隙,恐诛;欲降齐,所杀虏于齐甚众,恐已降而后见辱。喟然叹曰:“与人刃我,宁自刃。”乃自杀。聊城乱,田单遂屠聊城。归而言鲁连,欲爵之。鲁连逃隐于海上,曰:“吾与富贵而诎于人,宁贫贱而轻世肆志焉。”

邹阳者,齐人也。游于梁,与故吴人庄忌夫子、淮阴枚生之徒交。上书而介于羊胜、公孙诡之间。胜等嫉邹阳,恶之梁孝王。孝王怒,下之吏,将欲杀之。邹阳客游,以谗见禽,恐死而负累,乃从狱中上书曰:臣闻忠无不报,信不见疑,臣常以为然,徒虚语耳。昔者荆轲慕燕丹之义,白虹贯日,太子畏之;卫先生为秦画长平之事,太白蚀昴,而昭王疑之。夫精变天地而信不喻两主,岂不哀哉!今臣尽忠竭诚,毕议原知,左右不明,卒从吏讯,为世所疑,是使荆轲、卫先生复起,而燕、秦不悟也。原大王孰察之。

昔卞和献宝,楚王刖之;李斯竭忠,胡亥极刑。是以箕子详狂,接舆辟世,恐遭此患也。原大王孰察卞和、李斯之意,而后楚王、胡亥之听,无使臣为箕子、接舆所笑。臣闻比干剖心,子胥鸱夷,臣始不信,乃今知之。原大王孰察,少加怜焉。

谚曰:“有白头如新,倾盖如故。”何则?知与不知也。故昔樊于期逃秦之燕,藉荆轲首以奉丹之事;王奢去齐之魏,临城自刭以卻齐而存魏。夫王奢、樊于期非新于齐、秦而故于燕、魏也,所以去二国死两君者,行合于志而慕义无穷也。是以苏秦不信于天下,而为燕尾生;白圭战亡六城,为魏取中山。何则?诚有以相知也。苏秦相燕,燕人恶之于王,王按剑而怒,食以夬騠;白圭显于中山,中山人恶之魏文侯,文侯投之以夜光之璧。何则?两主二臣,剖心坼肝相信,岂移于浮辞哉!

故女无美恶,入宫见妒;士无贤不肖,入朝见嫉。昔者司马喜髌脚于宋,卒相中山;范睢摺胁折齿于魏,卒为应侯。此二人者,皆信必然之画,捐朋党之私,挟孤独之位,故不能自免于嫉妒之人也。是以申徒狄自沈于河,徐衍负石入海。不容于世,义不苟取,比周于朝,以移主上之心。故百里奚乞食于路,缪公委之以政;甯戚饭牛车下,而桓公任之以国。此二人者,岂借宦于朝,假誉于左右,然后二主用之哉?感于心,合于行,亲于胶漆,昆弟不能离,岂惑于众口哉?故偏听生奸,独任成乱。昔者鲁听季孙之说而逐孔子,宋信子罕之计而囚墨翟。夫以孔、墨之辩,不能自免于谗谀,而二国以危。何则?众口铄金,积毁销骨也。是以秦用戎人由余而霸中国,齐用越人蒙而彊威、宣。此二国,岂拘于俗,牵于世,系阿偏之辞哉?公听并观,垂名当世。故意合则胡越为昆弟,由余、越人蒙是矣;不合,则骨肉出逐不收,硃、象、管、蔡是矣。今人主诚能用齐、秦之义,后宋、鲁之听,则五伯不足称,三王易为也。

是以圣王觉寤,捐子之之心,而能不说于田常之贤;封比干之后,修孕妇之墓,故功业复就于天下。何则?欲善无厌也。夫晋文公亲其雠,彊霸诸侯;齐桓公用其仇,而一匡天下。何则,慈仁殷勤,诚加于心,不可以虚辞借也。

至夫秦用商鞅之法,东弱韩、魏,兵彊天下,而卒车裂之;越用大夫种之谋,禽劲吴,霸中国,而卒诛其身。是以孙叔敖三去相而不悔,于陵子仲辞三公为人灌园。今人主诚能去骄泬之心,怀可报之意,披心腹,见情素,堕肝胆,施德厚,终与之穷达,无爱于士,则桀之狗可使吠尧,而蹠之客可使刺由;况因万乘之权,假圣王之资乎?然则荆轲之湛七族,要离之烧妻子,岂足道哉!

臣闻明月之珠,夜光之璧,以闇投人于道路,人无不按剑相眄者。何则?无因而至前也。蟠木根柢,轮囷离诡,而为万乘器者。何则?以左右先为之容也。故无因至前,虽出随侯之珠,夜光之璧,犹结怨而不见德。故有人先谈,则以枯木朽株树功而不忘。今夫天下布衣穷居之士,身在贫贱,虽蒙尧、舜之术,挟伊、管之辩,怀龙逢、比干之意,欲尽忠当世之君,而素无根柢之容,虽竭精思,欲开忠信,辅人主之治,则人主必有按剑相眄之迹,是使布衣不得为枯木朽株之资也。

是以圣王制世御俗,独化于陶钧之上,而不牵于卑乱之语,不夺于众多之口。故秦皇帝任中庶子蒙嘉之言,以信荆轲之说,而匕首窃发;周文王猎泾、渭,载吕尚而归,以王天下。故秦信左右而杀,周用乌集而王。何则?以其能越挛拘之语,驰域外之议,独观于昭旷之道也。

今人主沈于谄谀之辞,牵于帷裳之制,使不羁之士与牛骥同皁,此鲍焦所以忿于世而不留富贵之乐也。

臣闻盛饰入朝者不以利汙义,砥厉名号者不以欲伤行,故县名胜母而曾子不入,邑号朝歌而墨子回车。今欲使天下寥廓之士,摄于威重之权,主于位势之贵,故回面汙行以事谄谀之人而求亲近于左右,则士伏死堀穴岩之中耳,安肯有尽忠信而趋阙下者哉!

书奏梁孝王,孝王使人出之,卒为上客。

太史公曰:鲁连其指意虽不合大义,然余多其在布衣之位,荡然肆志,不诎于诸侯,谈说于当世,折卿相之权。邹阳辞虽不逊,然其比物连类,有足悲者,亦可谓抗直不桡矣,吾是以附之列传焉。

鲁连达士,高才远致。释难解纷,辞禄肆志。齐将挫辩,燕军沮气。邹子遇谗,见诋狱吏。慷慨献说,时王所器。
\end{yuanwen}

\part{卷八十四}
\chapter{屈原贾生列传第二十四}

\begin{yuanwen}
屈原者,名平,楚之同姓也。为楚怀王左徒。博闻强志\footnote{见闻广博,记忆力强。},明于治乱,娴于辞令。入则与王图议国事,以出号令;出则接遇宾客,应对诸侯。王甚任之。

上官大夫与之同列,争宠而心害其能。怀王使屈原造为宪令,屈平属草未定。上官大夫见而欲夺之,屈平不与,因谗之曰:“王使屈平为令,众莫不知,每一令出,平伐其功,以为‘非我莫能为’也。”王怒而疏屈平。

上官大夫与之同列,争宠而心害其能。怀王使屈原造为宪令,屈平属【属:撰写。】 草稿未定。上官大夫见而欲夺之,屈平不与,因谗之曰:“王使屈平为令,众莫不知,每一令出,平伐【伐:自我夸耀。】 其功,以为‘非我莫能为’也。”王怒而疏屈平。

上官大夫跟屈原职位相同,他想跟屈原争夺楚王的宠爱,心里嫉妒屈原的才能。楚王命令屈原起草法令,屈原写好了草稿,但还没有最后确定。上官大夫看见以后就想夺取草稿,但屈原不给他,上官大夫就在楚王面前进谗言陷害屈原说:“大王让屈原起草法令,众人没有不知道的,每次一有法令颁布,屈原就夸耀自己的功劳,认为‘除了我没有人能够制定出这样的法令’。”楚王很生气,于是就疏远了屈原。

屈平疾王听之不聪也,谗谄之蔽明也,邪曲之害公也,方正之不容也,故忧愁幽思而作离骚。离骚者,犹离忧也。夫天者,人之始也;父母者,人之本也。人穷则反本,故劳苦倦极,未尝不呼天也;疾痛惨怛,未尝不呼父母也。屈平正道直行,竭忠尽智以事其君,谗人间之,可谓穷矣。信而见疑,忠而被谤,能无怨乎?屈平之作离骚,盖自怨生也。国风好色而不淫,小雅怨诽而不乱。若离骚者,可谓兼之矣。上称帝喾,下道齐桓,中述汤武,以刺世事。明道德之广崇,治乱之条贯,靡不毕见。其文约,其辞微,其志絜,其行廉,其称文小而其指极大,举类迩而见义远。其志絜,故其称物芳。其行廉,故死而不容自疏。濯淖汙泥之中,蝉蜕于浊秽,以浮游尘埃之外,不获世之滋垢,皭然泥而不滓者也。推此志也,虽与日月争光可也。



屈平既绌,其后秦欲伐齐,齐与楚从亲,惠王患之,乃令张仪详去秦,厚币委质事楚,曰:“秦甚憎齐,齐与楚从亲,楚诚能绝齐,秦原献商、于之地六百里。”楚怀王贪而信张仪,遂绝齐,使使如秦受地。张仪诈之曰:“仪与王约六里,不闻六百里。”楚使怒去,归告怀王。怀王怒,大兴师伐秦。秦发兵击之,大破楚师于丹、淅,斩首八万,虏楚将屈匄,遂取楚之汉中地。怀王乃悉发国中兵以深入击秦,战于蓝田。魏闻之,袭楚至邓。楚兵惧,自秦归。而齐竟怒不救楚,楚大困。

明年,秦割汉中地与楚以和。楚王曰:“不原得地,原得张仪而甘心焉。”张仪闻,乃曰:“以一仪而当汉中地,臣请往如楚。”如楚,又因厚币用事者臣靳尚,而设诡辩于怀王之宠姬郑袖。怀王竟听郑袖,复释去张仪。是时屈平既疏,不复在位,使于齐,顾反,谏怀王曰:“何不杀张仪?”怀王悔,追张仪不及。

明年,秦割汉中地与楚以和。楚王曰:“不愿得地,愿得张仪而甘心焉。”张仪闻,乃曰:“以一仪而当【当:抵押。】 汉中地,臣请往如楚。”如楚,又因厚币用事者臣靳尚,而设诡辩于怀王之宠姬郑袖。怀王竟听郑袖,复释去张仪。是时屈平既疏,不复在位,使于齐,顾反,谏怀王曰:“何不杀张仪?”怀王悔,追张仪不及。

其后诸侯共击楚,大破之,杀其将唐眛。

时秦昭王与楚婚,欲与怀王会。怀王欲行,屈平曰:“秦虎狼之国,不可信,不如毋行。”怀王稚子子兰劝王行:“柰何绝秦欢!”怀王卒行。入武关,秦伏兵绝其后,因留怀王,以求割地。怀王怒,不听。亡走赵,赵不内。复之秦,竟死于秦而归葬。

长子顷襄王立,以其弟子兰为令尹。楚人既咎子兰以劝怀王入秦而不反也。

屈平既嫉之,虽放流,睠顾楚国,系心怀王,不忘欲反,冀幸君之一悟,俗之一改也。其存君兴国而欲反覆之,一篇之中三致志焉。然终无可柰何,故不可以反,卒以此见怀王之终不悟也。人君无愚智贤不肖,莫不欲求忠以自为,举贤以自佐,然亡国破家相随属,而圣君治国累世而不见者,其所谓忠者不忠,而所谓贤者不贤也。怀王以不知忠臣之分,故内惑于郑袖,外欺于张仪,疏屈平而信上官大夫、令尹子兰。兵挫地削,亡其六郡,身客死于秦,为天下笑。此不知人之祸也。易曰:“井泄不食,为我心恻,可以汲。王明,并受其福。”王之不明,岂足福哉!

令尹子兰闻之大怒,卒使上官大夫短屈原于顷襄王,顷襄王怒而迁之。

屈原至于江滨,被发行吟泽畔。颜色憔悴,形容枯槁。渔父见而问之曰:“子非三闾大夫欤?何故而至此?”屈原曰:“举世混浊而我独清,众人皆醉而我独醒,是以见放。”渔父曰:“夫圣人者,不凝滞于物而能与世推移。举世混浊,何不随其流而扬其波?众人皆醉,何不餔其糟而啜其醨?何故怀瑾握瑜而自令见放为?”屈原曰:“吾闻之,新沐者必弹冠,新浴者必振衣,人又谁能以身之察察,受物之汶汶者乎!宁赴常流而葬乎江鱼腹中耳,又安能以皓皓之白而蒙世俗之温蠖乎!”

乃作怀沙之赋。其辞曰:

陶陶孟夏兮,草木莽莽。伤怀永哀兮,汩徂南土。眴兮窈窈,孔静幽墨。冤结纡轸兮,离愍之长鞠;抚情效志兮,俯诎以自抑。

刓方以为圜兮,常度未替;易初本由兮,君子所鄙。章画职墨兮,前度未改;内直质重兮,大人所盛。巧匠不斫兮,孰察其揆正?玄文幽处兮,矇谓之不章;离娄微睇兮,瞽以为无明。变白而为黑兮,倒上以为下。凤皇在笯兮,鸡雉翔舞。同糅玉石兮,一而相量。夫党人之鄙妒兮,羌不知吾所臧。任重载盛兮,陷滞而不济;怀瑾握瑜兮,穷不得余所示。邑犬群吠兮,吠所怪也;诽骏疑桀兮,固庸态也。文质疏内兮,众不知吾之异采;材朴委积兮,莫知余之所有。重仁袭义兮,谨厚以为丰;重华不可牾兮,孰知余之从容!古固有不并兮,岂知其故也?汤禹久远兮,邈不可慕也。惩违改忿兮,抑心而自彊;离湣而不迁兮,原志之有象。进路北次兮,日昧昧其将暮;含忧虞哀兮,限之以大故。

乱曰:浩浩沅、湘兮,分流汨兮。脩路幽拂兮,道远忽兮。曾唫恆悲兮,永叹慨兮。世既莫吾知兮,人心不可谓兮。怀情抱质兮,独无匹兮。伯乐既殁兮,骥将焉程兮?人生禀命兮,各有所错兮。定心广志,馀何畏惧兮?曾伤爰哀,永叹喟兮。世溷不吾知,心不可谓兮。知死不可让兮,原勿爱兮。明以告君子兮,吾将以为类兮。

于是怀石遂自(沉)汨罗以死。

屈原既死之后,楚有宋玉、唐勒、景差之徒者,皆好辞而以赋见称;然皆祖屈原之从容辞令,终莫敢直谏。其后楚日以削,数十年竟为秦所灭。

自屈原沈汨罗后百有馀年,汉有贾生,为长沙王太傅,过湘水,投书以吊屈原。

贾生名谊,雒阳人也。年十八,以能诵诗属书闻于郡中。吴廷尉为河南守,闻其秀才,召置门下,甚幸爱。孝文皇帝初立,闻河南守吴公治平为天下第一,故与李斯同邑而常学事焉,乃徵为廷尉。廷尉乃言贾生年少,颇通诸子百家之书。文帝召以为博士。

是时贾生年二十馀,最为少。每诏令议下,诸老先生不能言,贾生尽为之对,人人各如其意所欲出。诸生于是乃以为能,不及也。孝文帝说之,超迁,一岁中至太中大夫。

贾生以为汉兴至孝文二十馀年,天下和洽,而固当改正朔,易服色,法制度,定官名,兴礼乐,乃悉草具其事仪法,色尚黄,数用五,为官名,悉更秦之法。孝文帝初即位,谦让未遑也。诸律令所更定,及列侯悉就国,其说皆自贾生发之。于是天子议以为贾生任公卿之位。绛、灌、东阳侯、冯敬之属尽害之,乃短贾生曰:“雒阳之人,年少初学,专欲擅权,纷乱诸事。”于是天子后亦疏之,不用其议,乃以贾生为长沙王太傅。

贾生既辞往行,闻长沙卑湿,自以寿不得长,又以適去,意不自得。及渡湘水,为赋以吊屈原。其辞曰:共承嘉惠兮,俟罪长沙。侧闻屈原兮,自沈汨罗。造讬湘流兮,敬吊先生。遭世罔极兮,乃陨厥身。呜呼哀哉,逢时不祥!鸾凤伏窜兮,鸱枭翱翔。阘茸尊显兮,谗谀得志;贤圣逆曳兮,方正倒植。世谓伯夷贪兮,谓盗跖廉;莫邪为顿兮,铅刀为銛。于嗟嚜嚜兮,生之无故!斡弃周鼎兮宝康瓠,腾驾罢牛兮骖蹇驴,骥垂两耳兮服盐车。章甫荐屦兮,渐不可久;嗟苦先生兮,独离此咎!

讯曰:已矣,国其莫我知,独堙郁兮其谁语?凤漂漂其高遰兮,夫固自缩而远去。袭九渊之神龙兮,沕深潜以自珍。弥融爚以隐处兮,夫岂从螘与蛭螾?所贵圣人之神德兮,远浊世而自藏。使骐骥可得系羁兮,岂云异夫犬羊!般纷纷其离此尤兮,亦夫子之辜也!瞝九州而相君兮,何必怀此都也?凤皇翔于千仞之上兮,览德军而下之;见细德之险兮,摇增翮逝而去之。彼寻常之汙渎兮,岂能容吞舟之鱼!横江湖之鳣鲟兮,固将制于蚁蝼。

贾生为长沙王太傅三年,有鸮飞入贾生舍,止于坐隅。楚人命鸮曰“服”。贾生既以適居长沙,长沙卑湿,自以为寿不得长,伤悼之,乃为赋以自广。其辞曰:

单阏之岁兮,四月孟夏,庚子日施兮,服集予舍,止于坐隅,貌甚间暇。异物来集兮,私怪其故,发书占之兮,筴言其度。曰“野鸟入处兮,主人将去”。请问于服兮:“予去何之?吉乎告我,凶言其菑。淹数之度兮,语予其期。”服乃叹息,举首奋翼,口不能言,请对以意。

万物变化兮,固无休息。斡流而迁兮,或推而还。形气转续兮,变化而嬗。沕穆无穷兮,胡可胜言!祸兮福所倚,福兮祸所伏;忧喜聚门兮,吉凶同域。彼吴彊大兮,夫差以败;越栖会稽兮,句践霸世。斯游遂成兮,卒被五刑;傅说胥靡兮,乃相武丁。夫祸之与福兮,何异纠纆。命不可说兮,孰知其极?水激则旱兮,矢激则远。万物回薄兮,振荡相转。云蒸雨降兮,错缪相纷。大专槃物兮,坱轧无垠。天不可与虑兮,道不可与谋。迟数有命兮,恶识其时?

且夫天地为炉兮,造化为工;阴阳为炭兮,万物为铜。合散消息兮,安有常则;千变万化兮,未始有极。忽然为人兮,何足控抟;化为异物兮,又何足患!小知自私兮,贱彼贵我;通人大观兮,物无不可。贪夫徇财兮,烈士徇名;夸者死权兮,品庶冯生。述迫之徒兮,或趋西东;大人不曲兮,亿变齐同。拘士系俗兮,羖如囚拘;至人遗物兮,独与道俱。众人或或兮,好恶积意;真人淡漠兮,独与道息。释知遗形兮,超然自丧;寥廓忽荒兮,与道翱翔。乘流则逝兮,得坻则止;纵躯委命兮,不私与己。其生若浮兮,其死若休;澹乎若深渊之静,氾乎若不系之舟。不以生故自宝兮,养空而浮;德人无累兮,知命不忧。细故粦兮,何足以疑!

后岁馀,贾生徵见。孝文帝方受釐,坐宣室。上因感鬼神事,而问鬼神之本。贾生因具道所以然之状。至夜半,文帝前席。既罢,曰:“吾久不见贾生,自以为过之,今不及也。”居顷之,拜贾生为梁怀王太傅。梁怀王,文帝之少子,爱,而好书,故令贾生傅之。

文帝复封淮南厉王子四人皆为列侯。贾生谏,以为患之兴自此起矣。贾生数上疏,言诸侯或连数郡,非古之制,可稍削之。文帝不听。

居数年,怀王骑,堕马而死,无后。贾生自伤为傅无状,哭泣岁馀,亦死。贾生之死时年三十三矣。及孝文崩,孝武皇帝立,举贾生之孙二人至郡守,而贾嘉最好学,世其家,与余通书。至孝昭时,列为九卿。

太史公曰:余读离骚、天问、招魂、哀郢,悲其志。適长沙,观屈原所自沈渊,未尝不垂涕,想见其为人。及见贾生吊之,又怪屈原以彼其材,游诸侯,何国不容,而自令若是。读服乌赋,同死生,轻去就,又爽然自失矣。

屈平行正,以事怀王。瑾瑜比洁,日月争光。忠而见放,谗者益章。赋骚见志,怀沙自伤。百年之后,空悲吊湘。
\end{yuanwen}

\part{卷八十五}

\chapter{吕不韦列传第二十五}

记述了吕不韦由一个投机商人到涉足政治,再到执掌秦国政治的历史,作品在塑造这个以唯利是图为特征的政客上,是极有特色的。他辅助子楚,是因为“奇货可居”,想做一本万利的生意;他献出宠姬,是为借以钓得秦国的江山;他豢养门客编撰《吕氏春秋》,并以一字千金的重赏寻求能对其有所修改的人,是为了沽名钓誉。吕不韦最后因为嫪毒之乱被牵涉赐死,结束了作为商人的一生。作品的行文颇具讽刺意味,司马迁的态度与感情倾向是明显的。

吕不韦进怀孕之姬与子楚的事情,也见于《战国策》,且又与《春申君列传》所叙的事情雷同,所以梁玉绳、郭嵩焘、钱穆、马非百等都认为是出于时人的附会。且吕不韦自庄襄王元年为秦相,至始皇九年免职,前后为秦相十二年,这段时间正是秦对东方诸国大举进攻,并逐步实现吞并的时代,而本文作为一个秦国宰相的列传,竟只字未提吕不韦对于秦国的政治有何建树,这就未免过于偏狭,过于失之公正了。

牛运震:“赞语详嫪毐作乱败灭事,以补传所未及,末用一语收转吕子冷曲有神。”

\begin{yuanwen}
吕不韦者,阳翟大贾人也。往来贩贱卖贵,家累千金。

秦昭王四十年,太子死。其四十二年,以其次子安国君为太子。安国君有子二十余人。安国君有所甚爱姬\footnote{text},立以为正夫人,号曰华阳夫人。华阳夫人无子。安国君中男名子楚\footnote{text},子楚母曰夏姬,毋爱。子楚为秦质子于赵\footnote{text}。秦数攻赵,赵不甚礼子楚。
\end{yuanwen}

吕不韦,是阳翟的大商人。他往来各地,靠着低价买进货物,高价卖出去,在家里积累了千金的财富。

秦昭王四十年(前267年),太子死了。到秦昭王四十二年,便把他的第二个儿子安国君册立为太子。安国君有二十多个儿子。安国君有位非常宠爱的姬妾,册立她为正室夫人,称为“华阳夫人”。华阳夫人没有儿子。安国君有个排行居中儿子名叫子楚,子楚的生母叫夏姬,不受安国君宠爱。子楚作为秦国的人质被派到赵国。因为秦国多次攻打赵国,赵国对子楚的礼节不是很周到。

\begin{yuanwen}
子楚,秦诸庶孽孙\footnote{text},质于诸侯,车乘进用\footnote{财用。进,通“赆”,指收入的钱财。}不饶,居处困,不得意。吕不韦贾邯郸,见而怜之,曰“此奇货可居”。

乃往见子楚,说曰:“吾能大子之门。”

子楚笑曰:“且自大君之门,而乃大吾门!”

吕不韦曰:“子不知也,吾门待子门而大。”

子楚心知所谓,乃引与坐,深语。吕不韦曰:“秦王老矣,安国君得为太子。窃闻安国君爱幸华阳夫人,华阳夫人无子,能立適\footnote{通“嫡”。}嗣者独华阳夫人耳\footnote{text}。今子兄弟二十余人,子又居中,不甚见幸,久质诸侯。即\footnote{假使。}大王薨\footnote{text},安国君立为王,则子毋几\footnote{没有希望。}得与长子及诸子旦暮在前者争为太子矣\footnote{text}。”

子楚曰:“然。为之奈何?”

吕不韦曰:“子贫,客于此,非有以奉献于亲及结宾客也。不韦虽贫,请以千金为子西游,事安国君及华阳夫人,立子为適嗣。”

子楚乃顿首曰:“必如君策,请得分秦国与君共之。”
\end{yuanwen}

子楚,是秦昭王庶出的孙子,在赵国做人质,他日常所乘坐的车马和花销都不充足,生活困窘,很不得意。吕不韦到邯郸去做生意的时候,看到子楚以后非常喜欢他,说:“这个人就像一件奇货,可以囤积起来,以后卖个高价”。

于是就去求见子楚,游说他道:“我有能力光大您的门庭。”

子楚不以为然地笑着说道:“姑且先把自己家的门庭光大了,再来光大我的门庭。”

吕不韦说道:“您这就不明白了,我的门庭需要等您的门庭光大以后才能光大。”

子楚心里明白吕不韦说这番话的意思,就把他请进屋里,坐下来进行深入的交谈。吕不韦说:“秦王年纪大了,安国君得以被册立为太子。我私下里听说安国君宠爱华阳夫人,华阳夫人没有儿子,能够选立嫡子为继承人的只有华阳夫人一人。现在您的兄弟有二十多个,您又排在中间,不怎么受宠,长时间在诸侯那里做人质。即使秦王去世,安国君被立为秦王,您也不能指望去跟安国君的长子和整天在安国君面前的诸位公子争太子的位置。”

子楚说道:“说得对。那我应该怎么办呢?”

吕不韦说:“您生活贫困,在赵国做客,没有什么可以用来奉献给安国君和结交宾客的财物。我吕不韦虽然也很穷,但愿意拿出千金为您西去秦国游说,侍奉安国君和华阳夫人,让安国君册立您做嫡子继承人。”

子楚于是对吕不韦下跪磕头说道:“如果真像您所谋划的这样,那么我愿意分出秦国的土地与您共享。”

\begin{yuanwen}
吕不韦乃以五百金与子楚,为进用,结宾客;而复以五百金买奇物玩好,自奉而西游秦\footnote{text},求见华阳夫人姊,而皆以其物献华阳夫人。因言子楚贤智,结诸侯宾客遍天下,常曰“楚也以夫人为天,日夜泣思太子及夫人”。夫人大喜。

不韦因使其姊说夫人曰:“吾闻之,以色事人者,色衰而爱弛。今夫人事太子,甚爱而无子,不以此时蚤自结于诸子中贤孝者\footnote{text},举立以为適而子之,夫在则重尊\footnote{text},夫百岁之后,所子者为王,终不失势,此所谓一言而万世之利也。不以繁华\footnote{花盛,以喻人在盛年。}时树本,即色衰爱弛后,虽欲开一语,尚可得乎?今子楚贤,而自知中男也,次不得为適,其母又不得幸,自附夫人。夫人诚以此时拔以为適,夫人则竟世有宠于秦矣\footnote{text}。”

华阳夫人以为然,承太子闲,从容言子楚质于赵者绝贤,来往者皆称誉之。乃因涕泣曰:“妾幸得充后宫,不幸无子,原得子楚立以为適嗣,以托妾身\footnote{text}。”

安国君许之,乃与夫人刻玉符\footnote{古代朝廷的一种凭证。},约以为適嗣。安国君及夫人因厚馈遗子楚,而请吕不韦傅之,子楚以此名誉益盛于诸侯。
\end{yuanwen}

吕不韦于是拿出五百金交给子楚,作为他日常生活和结交朋友宾客的费用;然后又用五百金采购了稀有珍贵的玩物,亲自带着这些东西向西去游说秦国。吕不韦求见华阳夫人的姐姐,请她帮忙把所有的礼物献给华阳夫人。借这个机会说子楚既贤能又聪明,结交的诸侯和宾客遍及天下,而且子楚还经常说“我把华阳夫人当作天一样,每日每夜都流着泪想念做太子的父亲和华阳夫人”。

华阳夫人听了非常高兴。吕不韦于是借这个机会让华阳夫人的姐姐游说华阳夫人说:“我听说过这样的话,用美色来侍奉别人的,等到容貌衰老,受到的宠爱也就会慢慢地减少。现在夫人侍奉太子,很受宠爱却没有生养儿子,不如趁此时早点从太子的儿子们中选择一位贤能孝顺的来结交,在太子面前推荐、拥立为嫡子继承人,并像亲生儿子一样来对待他,那么当丈夫还在世的时候您会受到尊重,等到丈夫去世,自己立的儿子继位为王,您也不会失去依靠,这就是人们说的凭借一句话来获得万世的好处啊。如果不在容貌美丽的时候为自己树立根本,那么在自己容貌衰老、受到的宠爱越来越少的时候,就算想跟太子说句话,还能够得到机会吗?如今子楚十分贤能,他知道自己在安国君的儿子里面排行居中,按顺序是不可能被立为继承人的,他的母亲又不受宠爱,自己想要依附夫人,夫人如果真的能在此时推荐他做继承人,那么夫人终身都可以受到秦国的尊崇了。”

华阳夫人认为这话说得很有道理,就在太子有空闲的时候,委婉地说起在赵国做人质的子楚是个非常贤能的人,跟他来往的人都称赞他。华阳夫人又借机哭着对安国君说道:“我幸运地被您选进了后宫,但不幸的是没有儿子,我希望您能够册立子楚为嫡子继承您的王位,好让我在以后有个依托。”

安国君答应了华阳夫人的请求,于是跟华阳夫人一起刻下了一枚玉做的符印,约定立子楚为继承人。安国君和华阳夫人于是送给子楚丰厚的礼物,又请吕不韦来教导子楚,子楚因此在诸侯之间的名气越来越大。

\begin{yuanwen}
吕不韦取邯郸诸姬绝好\footnote{特别漂亮。}善舞者与居\footnote{text},知有身\footnote{指怀孕在身。}。子楚从不韦饮,见而说之,因起为寿,请之\footnote{讨要此人。}。吕不韦怒,念业已破家为子楚,欲以钓奇,乃遂献其姬\footnote{text}。姬自匿有身,至大期\footnote{十二个月。}时\footnote{text},生子政。子楚遂立姬为夫人。
\end{yuanwen}

吕不韦娶了邯郸城中一位非常美丽又擅长跳舞的女子做姬妾,跟自己住在一起,知道她怀了孕。有一次,子楚在跟吕不韦饮酒的时候,见到了这位姬妾,非常喜欢她,于是就站起来向吕不韦敬酒,请求把这个女子送给他。吕不韦很生气,但又一想,自己已经为子楚花去了家里的大量钱财,想要借此钓取奇货,于是就把这位姬妾献给了子楚。这位姬妾隐瞒了已经怀孕的事情,到了十二个月以后,生下了一个儿子起名叫政。子楚于是立这位姬妾为夫人。

\begin{yuanwen}
秦昭王五十年,使王齮围邯郸,急\footnote{text},赵欲杀子楚。子楚与吕不韦谋,行金六百斤予守者吏\footnote{text},得脱,亡赴秦军,遂以得归。赵欲杀子楚妻子。子楚夫人,赵豪家女也\footnote{text},得匿,以故母子竟得活。秦昭王五十六年,薨,太子安国君立为王,华阳夫人为王后,子楚为太子。赵亦奉子楚夫人及子政归秦。
\end{yuanwen}

秦昭王五十年(前257年),派王齮率领秦军包围了邯郸,情势危急,赵国想杀死子楚来泄愤。子楚跟吕不韦商量,拿出六百斤黄金贿赂看守他们的官吏,终于逃出了赵国,逃到了秦军的军营中,于是回到了秦国。赵王又想杀死子楚的妻子和儿子,但子楚的夫人是赵国一家很有势力的人家的女儿,得以被藏匿起来,所以母子两个最终得以活下来。秦昭王五十六年(前251年),秦昭王去世,太子安国君被立为秦王,华阳夫人被册立为王后,子楚被册立为太子。赵国也把子楚的夫人和儿子政送回了秦国。

\begin{yuanwen}
秦王立一年,薨,谥为孝文王。太子子楚代立,是为庄襄王。庄襄王所母华阳后为华阳太后,真母夏姬尊以为夏太后。庄襄王元年,以吕不韦为丞相,封为文信侯,食河南雒阳十万户\footnote{text}。
\end{yuanwen}

安国君被拥立为秦王只有一年的时间,就去世了,谥号是孝文王。太子子楚继承王位,他就是庄襄王。被庄襄王认为母亲的华阳夫人被尊为华阳太后,而他的生母夏姬也被尊为夏太后。庄襄王元年(前249年),任命吕不韦为相国,封号为“文信侯”,庄襄王把河南雒阳的十万户赏赐给吕不韦作为他的食邑。

\begin{yuanwen}
庄襄王即位三年,薨,太子政立为王,尊吕不韦为相国,号称“仲父\footnote{text}”。秦王年少,太后时时窃私通吕不韦。不韦家僮万人。
\end{yuanwen}

庄襄王即位三年后就死了,太子政被立为秦王,尊奉吕不韦为相国,还称他为“仲父”。由于秦王年纪很小,太后经常偷偷地与吕不韦私通。吕不韦家里的仆人有一万人。

\begin{yuanwen}
当是时,魏有信陵君,楚有春申君,赵有平原君,齐有孟尝君\footnote{text},皆下士喜宾客以相倾\footnote{text}。吕不韦以秦之强,羞不如,亦招致士,厚遇之,至食客三千人\footnote{text}。是时诸侯多辩士\footnote{text},如荀卿之徒,著书布天下。吕不韦乃使其客人人著所闻,集论以为八览、六论、十二纪\footnote{text},二十馀万言。以为备天地万物古今之事,号曰《吕氏春秋》。布咸阳市门,悬千金其上,延诸侯游士宾客有能增损一字者予千金\footnote{text}。
\end{yuanwen}

在那个时候,魏国有信陵君魏无忌,楚国有春申君黄歇,赵国有平原君赵胜,齐国有孟尝君田文,他们都礼贤下士,结交宾客,并且争得很厉害。吕不韦认为秦国如此强大,在招纳门客这方面却不如这四位公子,因此觉得耻辱,也开始招纳贤士,对待他们非常好,最后食客的数量也达到了三千人。在这个时候,诸侯国都有很多能言善辩的人,像荀卿等人,他们的著作传遍了天下。吕不韦就命他的食客各自将所见所闻记下,综合在一起成为八览、六论、十二纪,共二十多万言。自认为其中包括了天地万物古往今来的事理,于是就命名为《吕氏春秋》。把这部书放在咸阳的城门上,又在城门上悬挂了一千金,邀请从各个诸侯国来到秦国游历的士人和宾客,如果能为这部书增加或删减一个字,就奖励他一千金。

曾国藩:「吕不韦列传赞,孔子之所谓闻者,实与吕子不侔。子长读《论语》,别自有说。」

\begin{yuanwen}
始皇帝益壮,太后淫不止。吕不韦恐觉祸及己,乃私求大阴\footnote{大生殖器。}人嫪毐\footnote{text}以为舍人,时纵倡乐,使毐以其阴关\footnote{穿。}桐轮而行,令太后闻之,以啗\footnote{吃,喂,引诱。}太后。太后闻,果欲私得之。吕不韦乃进嫪毐,诈令人以腐罪\footnote{指应判处腐刑(即宫刑)的罪。}告之\footnote{text}。不韦又阴谓太后曰:“可事诈腐,则得给事中\footnote{text}。”

太后乃阴厚赐主腐者吏,诈论之,拔其须眉为宦者,遂得侍太后。太后私与通,绝爱之。有身,太后恐人知之,诈卜当避时\footnote{text},徙宫居雍。嫪毐常从,赏赐甚厚,事皆决于嫪毐\footnote{text}。嫪毐家僮数千人,诸客求宦为嫪毐舍人千馀人\footnote{text}。
\end{yuanwen}

秦始皇一天天长大,但太后一直淫乱不止。吕不韦害怕事情败露,灾祸会降临到自己的身上,就暗地里寻访到一个阴茎很大名叫嫪毐的人来当自己的门客,时常纵情地欣赏歌舞,命令嫪毐把桐木做的车轮套在自己的阴茎上,使车轮转动行走,并让太后听说这件事,来引诱太后。太后听说以后,果然想暗中得到嫪毐。吕不韦于是把嫪毐进献给太后,假装让人用应当判处腐刑的罪名来控告嫪毐。吕不韦又私下对太后说:“可以让嫪毐假装受了腐刑,就可以在供职宫中的人员中得到他。”

太后于是暗地里赏赐给施行腐刑的官员厚重的礼物,假装判处嫪毐腐刑,拔掉他的胡须和比较粗重的眉毛,假充宦官,于是得以侍奉太后。太后暗中和嫪毐私通,非常喜欢他。后来怀孕了,太后怕有人知道这件事,就谎称自己占卜不吉利,应当换个地方躲避灾难,迁居到雍地的宫殿里居住。嫪毐经常跟在太后身边,得到的赏赐非常丰厚,所有的事情都由嫪毐来决定。嫪毐家里的仆人有几千人,从各地来到秦国谋求官职来当嫪毐门客的有一千多人。

\begin{yuanwen}
始皇七年,庄襄王母夏太后薨。孝文王后曰华阳太后,与孝文王会葬寿陵。夏太后子庄襄王葬芷阳,故夏太后独别葬杜东,曰:“东望吾子,西望吾夫\footnote{text}。后百年,旁当有万家邑\footnote{text}”。
\end{yuanwen}

秦始皇七年(前240年),庄襄王的生母夏太后去世。孝文王的王后是华阳太后,已经跟孝文王合葬在寿陵。夏太后的儿子庄襄王安葬在芷阳,所以夏太后另外单独安葬在杜原之东,说是“向东可以看到我的儿子,向西可以看到我的丈夫。一百年以后,我的陵墓旁边,就会出现一座一万户规模的城邑”。

\begin{yuanwen}
始皇九年,有告嫪毐实非宦者,常与太后私乱,生子二人,皆匿之。与太后谋曰“王即薨,以子为后\footnote{text}”。

于是秦王下吏治,具得情实,事连相国吕不韦。

九月,夷嫪毐三族\footnote{text},杀太后所生两子,而遂迁太后于雍。诸嫪毐舍人皆没其家而迁之蜀。王欲诛相国,为其奉先王功大\footnote{text},及宾客辩士为游说者众,王不忍致法\footnote{予以法律制裁。}。
\end{yuanwen}

秦始皇九年(前238年),有人向秦始皇告发嫪毐实际上并不是宦官,经常和太后私通淫乱,生下了两个儿子,还把这两个孩子藏了起来。嫪毐还跟太后商量说“大王如果死了,就让我们的儿子继承王位”。

于是秦王派官员调查这件事,弄清了事情全部的真相,事情牵连到相国吕不韦。

这年九月,秦王下令杀了嫪毐的三族,又杀死了太后跟嫪毐所生的两个儿子,然后就让太后迁往雍地。所有嫪毐家的门客全都被抄没家产,然后发配到蜀地。秦王想把相国吕不韦也杀掉,因为他侍奉先王立下了很大的功劳,再加上宾客和说客为吕不韦说情的人有很多,秦王不忍心用法律制裁他。

\begin{yuanwen}
秦王十年十月\footnote{text},免相国吕不韦。及齐人茅焦说秦王\footnote{text},秦王乃迎太后于雍,归复咸阳,而出文信侯就国河南\footnote{text}。
\end{yuanwen}

秦王十年(前237年)十月,罢免了吕不韦相国的职务。等到齐国人茅焦游说秦王之后,秦王就把太后从雍地的宫殿中接回了咸阳,但是让文信侯前往位于黄河以南的封地。

\begin{yuanwen}
岁馀,诸侯宾客使者相望于道,请文信侯\footnote{text}。秦王恐其为变,乃赐文信侯书曰:“君何功于秦?秦封君河南,食十万户?君何亲于秦,号称仲父?其与家属徙处蜀!”

吕不韦自度稍侵\footnote{text},恐诛,乃饮酖\footnote{通“鸩”。毒酒。}而死。秦王所加怒吕不韦、嫪毐皆已死,乃皆复归嫪毐舍人迁蜀者。
\end{yuanwen}

过了一年多,各个诸侯国的宾客和使者在道路上络绎不绝,请文信侯到各自的国家去做官。秦王害怕吕不韦会发动叛乱,于是就写了一封信给他,说:“您对秦国有什么功劳?秦国把你封在黄河以南,食邑十万户。您和秦王有什么亲情?被秦王称为‘仲父’。您还是跟家人迁徙到蜀地去吧!”

吕不韦觉得自己正在被步步逼迫,害怕被秦王杀害,就喝毒酒自杀了。秦王怨恨的吕不韦、嫪毐都已经死了,就让迁徙到蜀地的嫪毐门客都回来了。

\begin{yuanwen}
始皇十九年,太后薨,谥为帝太后,与庄襄王会葬茝阳。
\end{yuanwen}

秦始皇十九年(前228年),太后死了,谥号为“帝太后”,跟庄襄王合葬在了茝阳。

\begin{yuanwen}
太史公曰:不韦及嫪毐贵,封号文信侯\footnote{text}。人之告嫪毐,毐闻之。秦王验左右,未发。上之雍郊\footnote{text},毐恐祸起,乃与党谋,矫太后玺发卒以反蕲年宫。发吏攻毐\footnote{text},毐败,亡走,追斩之好畤,遂灭其宗\footnote{text}。而吕不韦由此绌矣\footnote{text}。孔子之所谓“闻”者\footnote{text},其吕子乎?
\end{yuanwen}

太史公说:吕不韦和嫪毐地位尊贵,吕不韦被封为文信侯。有人告发嫪毐,嫪毐听说了。秦王向身边的人打听情况,事情并没有暴露。秦王到雍地祭天,嫪毐害怕灾祸发生,就跟同党谋划,盗用太后的玺印调拨军队,在蕲年宫发动了叛乱。秦王派出官兵攻打嫪毐,嫪毐失败逃走,秦王的军队追到好畤,砍下了嫪毐的脑袋,于是把嫪毐全族都杀死了。吕不韦也因此被罢黜。孔子所说的骗取好名声的小人,指的就是吕不韦吧?

\begin{yuanwen}
不韦钓奇,委质子楚。华阳立嗣,邯郸献女。及封河南,乃号仲父。徙蜀惩谤,悬金作语。筹策既成,富贵斯取。
\end{yuanwen}

\part{卷八十六}
\chapter{刺客列传第二十六}

\begin{yuanwen}
曹沫者,鲁人也,以勇力事鲁庄公。庄公好力。曹沫为鲁将,与齐战,三败北。鲁庄公惧,乃献遂邑之地以和。犹复以为将。
\end{yuanwen}

曹沫,是鲁国人,凭勇敢和气力侍奉鲁庄公。鲁庄公偏爱有力气的人。曹沫出任鲁国大将的时候,与齐国交战,多次被打败。鲁庄公很害怕,就献出遂邑,以此来向齐国求和。但仍继续任命曹沫担任大将。

\begin{yuanwen}
齐桓公许与鲁会于柯而盟。桓公与庄公既盟于坛上,曹沫执匕首劫齐桓公,桓公左右莫敢动,而问曰:“子将何欲?”

曹沫曰:“齐强鲁弱,而大国侵鲁亦甚矣。今鲁城坏即压齐境,君其图之。”

桓公乃许尽归鲁之侵地。既已言,曹沫投其匕首,下坛,北面就群臣之位,颜色\footnote{脸色。}不变,辞令如故。桓公怒,欲倍\footnote{同“背”,违背。}其约。管仲曰:“不可。夫贪小利以自快,弃信于诸侯,失天下之援,不如与之。”

于是桓公乃遂割鲁侵地,曹沫三战所亡地尽复予鲁。
\end{yuanwen}

齐桓公答应与鲁国在柯地见面并订下盟约。齐桓公与鲁庄公在盟坛上签订盟约以后,曹沫手拿匕首挟持了齐桓公,桓公身边的人不敢轻举妄动,桓公问:“你想要干什么?”

曹沫回答说:“齐国强大,鲁国弱小,齐国以大国的身份侵略鲁国,这样也太过分了!如今鲁国的城墙倒塌的话,就会压到齐国的边境,大王应该考虑这个问题。”

于是,齐桓公答应将侵占的鲁国土地全部归还。话说完以后,曹沫便将匕首扔掉,走下盟坛,回到面向北边属于群臣的位置上,面不改色,谈吐从容如常。齐桓公很生气,想要背弃齐鲁两国的盟约。管仲说:“不可以那样做。那些贪些小利小惠、只想让自己快乐的人,会在诸侯面前失去信用,这样一来就会失去天下人的援助,不如将土地归还给鲁国吧。”

于是,齐桓公就划出所侵占的鲁国的土地,曹沫多次战败失去的土地,全部归还给了鲁国。

\begin{yuanwen}
其后百六十有七年而吴有专诸之事。
\end{yuanwen}

又过了一百六十七年,吴国有专诸的事迹。

\begin{yuanwen}
专诸者,吴堂邑人也。伍子胥之亡楚而如吴也,知专诸之能。伍子胥既见吴王僚,说以伐楚之利。吴公子光曰:“彼伍员父兄皆死于楚而员言伐楚,欲自为报私雠也,非能为吴。”

吴王乃止。伍子胥知公子光之欲杀吴王僚,乃曰:“彼光将有内志,未可说以外事。”乃进专诸于公子光。
\end{yuanwen}

专诸,是吴国的堂邑人。伍子胥从楚国流亡到吴国时,知道专诸的本领很大。伍子胥见到吴王僚后,用讨伐楚国的利益来游说他。吴国的公子光说:“那个伍员,他的父亲兄长,全部都死在楚国,他劝大王攻打楚国,是想要为自己报私仇罢了,并非真正为吴国的利益着想。”

吴王就放弃攻伐楚国的打算。伍子胥知道公子光想要杀死吴王僚,便说:“那个公子光有在国内夺取王位的意图,目前还不能拿对外用兵的事情去说服他。”伍子胥就将专诸推荐给公子光。

\begin{yuanwen}
光之父曰吴王诸樊。诸樊弟三人:次曰馀祭,次曰夷眛,次曰季子札。诸樊知季子札贤而不立太子,以次传三弟,欲卒致国于季子札。诸樊既死,传馀祭。馀祭死,传夷眛。夷眛死,当传季子札;季子札逃不肯立,吴人乃立夷眛之子僚为王。公子光曰:“使以兄弟次邪,季子当立;必以子乎,则光真適嗣\footnote{正妻所生的长子。適,同“嫡”。},当立。”故尝阴养谋臣以求立。
\end{yuanwen}

公子光的父亲就是吴王诸樊。诸樊有三个弟弟:大弟名字叫馀祭,二弟名字叫夷眜,三弟名字叫季子札。诸樊了解三弟季子札贤德而有能力,就没有立太子,而是将王位依次传给三个弟弟,想要最后将吴国交给季子札。诸樊去世以后,将王位传给馀祭。馀祭去世后,将王位传给夷眜。夷眜去世后,本来应该将王位传给季子札;季子札却逃跑不肯接受王位,于是吴国人拥立了夷眜的儿子僚为吴王。公子光说:“假如是按照兄弟的顺序传递王位,季子应该即位;如果一定是要以儿子的顺序即位,那么我公子光才是真正的嫡子继承人,我应该继承王位。”因此公子光曾经私下供养谋臣,以便靠他们的帮助取得王位。

\begin{yuanwen}
光既得专诸,善客待之。九年而楚平王死。

春,吴王僚欲因楚丧,使其二弟公子盖馀、属庸将兵围楚之灊\footnote{qián};使延陵季子于晋,以观诸侯之变。楚发兵绝吴将盖馀、属庸路,吴兵不得还。于是公子光谓专诸曰:“此时不可失,不求何获!且光真王嗣,当立,季子虽来,不吾废也。”

专诸曰:“王僚可杀也。母老子弱,而两弟将兵伐楚,楚绝其后。方今吴外困于楚,而内空无骨鲠\footnote{}之臣\footnote{正直敢言的忠臣。鲠,gěng,通“骾”。},是无如我何。”

公子光顿首曰:“光之身,子之身也。”
\end{yuanwen}

公子光得到专诸后,像对待客人一样地款待他。吴王僚九年,楚平王去世。

这年春天,吴王僚趁楚国有丧事的机会,派遣他的两个弟弟公子盖馀和公子属庸带领军队包围楚国的灊地;又派遣延陵季子前往晋国,以便观察诸侯国的动静。楚国发兵将吴国将领盖馀、属庸的退路切断,吴国的兵马没有办法回国。这个时候,公子光对专诸说:“这是个千载难逢的好机会,千万不可以丢失,如果现在不争取的话,又怎么会有所成就呢!何况我是真正的继承人,理应即位。即使季子回来,也不会废掉我。”

专诸说:“吴王僚可以杀死。母亲年纪老迈,孩子尚在襁褓,两个弟弟现在又带兵攻伐楚国,被楚军断绝了退路。如今吴国外被楚国所困,朝廷里又没有忠诚的大臣,这样一来,就没有办法来对付我们了。”

公子光给专诸叩头说:“我公子光的命就是您的命。”

\begin{yuanwen}
四月丙子,光伏甲士于窟室中,而具酒请王僚。王僚使兵陈自宫至光之家,门户阶陛左右,皆王僚之亲戚\footnote{亲信,亲近者。}也。夹立侍,皆持长铍\footnote{长矛。一说两刃刀。}。酒既酣,公子光详为足疾,入窟室中,使专诸置匕首鱼炙之腹中而进之。既至王前,专诸擘\footnote{剖,撕开。}鱼,因以匕首刺王僚,王僚立死。左右亦杀专诸,王人扰乱。公子光出其伏甲以攻王僚之徒,尽灭之,遂自立为王,是为阖闾。阖闾乃封专诸之子以为上卿。
\end{yuanwen}

四月丙子日这天,公子光事先将全副武装的士兵埋伏在地下室,并准备好酒席邀请吴王僚前来赴宴。吴王僚派遣他的士兵排成长长的队伍,从宫廷一直排到公子光的家里,门户、台阶两旁都是吴王僚的亲信。这些人夹道站立,手里都拿着长矛。酒宴喝得正尽兴的时候,公子光借口脚痛,来到地下室,让专诸将匕首放在烤熟的鱼腹中,将烤鱼端上去。专诸端着烤鱼来到吴王僚面前,剖开鱼腹,拿出匕首立即刺向吴王僚,吴王僚当场被刺死。吴王僚身边的武士也杀死了专诸,吴王僚的人陷入纷扰混乱之中。公子光出动他事先埋伏好的士兵,攻击吴王僚的人,将他们全部消灭,公子光就自立为王,这就是吴王阖闾。阖闾将专诸的儿子封为上卿。

\begin{yuanwen}
其后七十馀年而晋有豫让之事。
\end{yuanwen}

这件事之后又过了七十多年,晋国有豫让的事情。

\begin{yuanwen}
豫让者,晋人也,故尝事范氏及中行氏,而无所知名。去而事智伯,智伯甚尊宠之。及智伯伐赵襄子,赵襄子与韩、魏合谋灭智伯,灭智伯之后而三分其地。赵襄子最怨智伯,漆其头以为饮器。豫让遁逃山中,曰:“嗟乎!士为知己者死,女为说己者容。今智伯知我,我必为报雠而死,以报智伯,则吾魂魄不愧矣。”

乃变名姓为刑人,入宫涂厕\footnote{修整厕所。涂,以泥抹墙。},中挟匕首,欲以刺襄子。襄子如厕,心动,执问涂厕之刑人,则豫让,内持刀兵,曰:“欲为智伯报仇!”

左右欲诛之。襄子曰:“彼义人也,吾谨避之耳。且智伯亡无后,而其臣欲为报仇,此天下之贤人也。”

卒醳去之。
\end{yuanwen}

豫让,是晋国人,他以前曾经侍奉过范氏和中行氏,但一直没有什么名声。离开中行氏后,前去侍奉智伯,智伯非常尊重他,宠幸他。等到智伯出兵讨伐赵襄子的时候,赵襄子和韩、魏联合,一起消灭了智伯;智伯被灭以后,他们将智伯的土地分成三份瓜分了。赵襄子十分痛恨智伯,将智伯的头颅涂上油漆,把它作为饮酒的器皿。豫让逃亡到山中,感叹说:“唉!士人甘愿为了解自己的人献出生命,女子甘愿为喜爱自己的人修饰容颜。如今智伯了解我,我一定要拼死为他报仇,以此来报答智伯,就算死了,灵魂也不会感到羞愧了。”

于是豫让改名换姓,伪装成犯罪受刑的人,潜入赵襄子的宫中修整厕所,随身带着匕首,想要刺杀襄子。赵襄子上厕所的时候,心中一惊,就让随从捉住并审问那个粉刷厕所的人,才知道是豫让,身上还藏着短剑,并说:“我要为智伯报仇!”

赵襄子的侍从都想要杀死豫让。襄子却说:“这是个有义气的人,我以后谨慎些避开他就行了。何况智伯已经死了,他没有后代,他的家臣想要替他报仇,是天下难得的贤人。”

最后释放了他,让他离开。

\begin{yuanwen}
居顷之,豫让又漆身为厉\footnote{同“疠”,恶疮。古又同“癞”,麻风病。},吞炭为哑,使形状不可知,行乞于市。其妻不识也。行见其友,其友识之,曰:“汝非豫让邪?”

曰:“我是也。”

其友为泣曰:“以子之才,委质而臣事襄子,襄子必近幸子。近幸子,乃为所欲,顾不易邪?何乃残身苦形\footnote{摧残身体,丑化形貌。},欲以求报襄子,不亦难乎!”

豫让曰:“既已委质臣事人,而求杀之,是怀二心以事其君也。且吾所为者极难耳!然所以为此者,将以愧天下后世之为人臣怀二心以事其君者也。”
\end{yuanwen}

没过多久,豫让再次全身涂满油漆,让身体溃烂,长满了恶疮,又吞下火炭让自己的声音变得沙哑,使自己的样貌不可辨认,在街上讨饭。他的妻子也不能认出他。路上见到他的朋友,他的朋友认出他,说:“你不是豫让吗?”

豫让说:“正是我。”

他的朋友流着泪说道:“凭借你的才华,如果能委身前去侍奉赵襄子的话,赵襄子一定会非常宠信您的。等到他宠信您之后,您再去干您想干的事,不就容易了吗?为什么要摧残自己的身体,丑化自己的样貌,想要用这样的办法达到向赵襄子报仇的目的,不也很困难吗!”

豫让说:“既然已经侍奉了别人,又想杀死他,这就是心怀不忠之心来服侍他的君主啊。我现在这么做非常艰难!但是我之所以坚持这样做,就是要让天下以后那些作为臣子却心怀二意去侍奉自己君主的人感到惭愧。”

\begin{yuanwen}
既去,顷之,襄子当出,豫让伏于所当过之桥下。襄子至桥,马惊,襄子曰:“此必是豫让也。”

使人问之,果豫让也。于是襄子乃数豫让曰:“子不尝事范、中行氏乎?智伯尽灭之,而子不为报雠,而反委质臣于智伯。智伯亦已死矣,而子独何以为之报雠之深也?”

豫让曰:“臣事范、中行氏,范、中行氏皆众人遇我,我故众人报之。至于智伯,国士遇我,我故国士报之。”

襄子喟然叹息而泣曰:“嗟乎豫子!子之为智伯,名既成矣,而寡人赦子,亦已足矣。子其自为计,寡人不复释子!”

使兵围之。豫让曰:“臣闻明主不掩人之美,而忠臣有死名之义。前君已宽赦臣,天下莫不称君之贤。今日之事,臣固伏诛\footnote{接受死罪。},然原请君之衣而击之,焉以致报雠之意,则虽死不恨。非所敢望也,敢布腹心\footnote{敢于说出心里话。}!”

于是襄子大义之,乃使使持衣与豫让。豫让拔剑三跃而击之,曰:“吾可以下报智伯矣!”

遂伏剑自杀。死之日,赵国志士闻之,皆为涕泣。
\end{yuanwen}

豫让走后没过多久,赵襄子正好外出,豫让便埋伏在赵襄子必定经过的一座桥的下面。赵襄子刚到桥上,马就受惊了,赵襄子说:“这一定是豫让。”

派人一查问,真的是豫让。赵襄子于是列举罪过责备豫让说:“你不也曾经服侍过范氏和中行氏吗?智伯将他们全部消灭了,但是你却没有为他们报仇,反而委身成智伯的臣子。如今智伯也已经死了,你为什么偏偏要如此卖力地为智伯报仇呢?”

豫让说:“我服侍范氏和中行氏,范氏和中行氏对待我都像对待普通人一样,因此我作为报答也像对待普通人那样对待他们。至于智伯,他对待我如同对待国士一样,因此我也应该像国士一样报答他。”

赵襄子感慨叹息,流着眼泪说:“唉!豫先生,您为智伯尽忠到这个地步,名声已经很大了,而我对您宽赦到这个程度,也已经足够了。您还是自己想个办法活命吧,我不会再放过你了!”

说完,便命令卫士将豫让围住。豫让说:“我听说圣明的君主不会掩盖别人的美德,而忠诚的臣子有为美名而死的道义。上次的事情您已赦免了我,天下没有人不称赞您的贤德。今天的事情,我本应伏法受诛,但是我恳求能够得到您的衣服来击打它,以此来表达我替智伯报仇的心意,这样一来,我就是死了也没有遗憾了。这自然不敢指望您答应,但我敢于说出我的心里话。”

当时襄子非常赞赏豫让的义气,便命令使者将衣服拿给豫让,豫让拔出剑来三次跳起来击刺它,说:“我可以到九泉之下去报答智伯了!”

于是伏剑自刎了。豫让死的那天,赵国的志士得知这个消息,都为他痛哭流泪。

\begin{yuanwen}
其后四十馀年而轵有聂政之事。
\end{yuanwen}

从这以后又过了四十多年,轵邑有聂政的事迹。

\begin{yuanwen}
聂政者,轵深井里人也。杀人避仇,与母、姊如齐,以屠为事。
\end{yuanwen}

聂政,是轵县深井里人。他因为杀人躲避仇家追杀,跟母亲、姐姐逃到齐国,以屠宰为职业。

\begin{yuanwen}
久之,濮阳严仲子事韩哀侯,与韩相侠累有卻。严仲子恐诛,亡去,游求人可以报侠累者。至齐,齐人或言聂政勇敢士也,避仇隐于屠者之间。严仲子至门请,数反\footnote{多次往返。反,同“返”。},然后具酒自(暢/畅)\footnote{敬酒。}聂政母前。酒酣,严仲子奉黄金百溢,前为聂政母寿。聂政惊怪其厚,固谢严仲子。严仲子固进,而聂政谢曰:“臣幸有老母,家贫,客游以为狗屠,可以旦夕得甘毳\footnote{甜脆的食物。毳,通“脆”。}以养亲。亲供养备,不敢当仲子之赐。”

严仲子辟人,因为聂政言曰:“臣有仇,而行游诸侯众矣;然至齐,窃闻足下义甚高,故进百金者,将用为大人\footnote{对他人父母的敬称。}粗粝之费,得以交足下之驩,岂敢以有求望邪!”

聂政曰:“臣所以降志辱身居市井屠者,徒幸以养老母;老母在,政身未敢以许人也。”

严仲子固让,聂政竟不肯受也。然严仲子卒备宾主之礼而去。
\end{yuanwen}

很久之后,濮阳人严仲子侍奉韩哀侯,因与韩国宰相侠累产生了矛盾。严仲子担心侠累杀死自己,便逃离了韩国,周游各国,寻求可以替他向侠累报仇的人。到了齐国以后,齐国有人对他说聂政是一个勇士,他为了逃避仇人的追杀,才躲藏在屠夫中间。严仲子到聂家来求见聂政,来回往返几次,之后又准备了酒食,亲自送到聂政母亲面前。酒酣耳热之际,严仲子又拿出一百镒黄金,上前为聂政的母亲祝寿。聂政对这份厚礼感到奇怪,便再三向严仲子辞谢。严仲子坚持要送,聂政辞谢说:“我很庆幸我的老母尚在,我们尽管家境贫穷,但是客居在这里,以屠狗为职业,早晚也能够得到些美食,来奉养母亲。现在我有足够的能力供养母亲,所以不敢接受仲子的赐予。”

严仲子令旁人退下,趁机对聂政说:“我有仇要报,因而遍游众多的诸侯国;然而来到齐国之后,私下听说您是个义气非常高的人,所以进献百金,以此作为您母亲买粗粮的费用,并以此来讨得朋友的欢心,怎么还敢有别的请求呢?”

聂政说:“我之所以降低自己的志向,委屈自己,在市井里做一个普通的屠夫,只是想能够通过这种方法来奉养母亲。老母尚在人世,我聂政是不敢用自己的性命来答应为他人献身的。”

严仲子再三谦让,聂政始终不肯接受。不过,最后严仲子还是尽完宾主之仪才离开。

\begin{yuanwen}
久之,聂政母死。既已葬,除服\footnote{丧服期满。},聂政曰:“嗟乎!政乃市井之人,鼓刀以屠;而严仲子乃诸侯之卿相也,不远千里,枉车骑而交臣。臣之所以待之,至浅鲜矣,未有大功可以称者,而严仲子奉百金为亲寿,我虽不受,然是者徒深知政也。夫贤者以感忿睚眦\footnote{发怒时瞪眼睛。借指很小的仇恨。yá zì}之意而亲信穷僻之人,而政独安得(嘿/默)然而已乎!且前日要政,政徒以老母;老母今以天年终,政将为知己者用。”

乃遂西至濮阳,见严仲子曰:“前日所以不许仲子者,徒以亲在;今不幸而母以天年终。仲子所欲报仇者为谁?请得从事焉!”

严仲子具告曰:“臣之仇韩相侠累,侠累又韩君之季父也,宗族盛多,居处兵卫甚设,臣欲使人刺之,终莫能就。今足下幸而不弃,请益其车骑壮士可为足下辅翼者。”

聂政曰:“韩之与卫,相去中间不甚远,今杀人之相,相又国君之亲,此其势不可以多人,多人不能无生得失,生得失则语泄,语泄是韩举国而与仲子为雠,岂不殆哉!”

遂谢车骑人徒,聂政乃辞独行。
\end{yuanwen}

很久之后,聂政的母亲去世了。聂政安葬完母亲,脱掉丧服,说道:“唉!我只是一个市井上的普通百姓,手拿着刀来屠宰牲畜罢了;而严仲子身为诸侯国的卿相,竟然不远千里,降低自己的身份屈驾前来与我结交。而我用来对待他的情义,实在是太浅薄了,我没有什么值得称赞的大功,可是严仲子却给我的母亲奉上百金作为寿礼,我尽管没有接受,但是这足以说明他十分清楚我的为人。像他这样一个贤明圣德的人,因为自己心中的仇恨,而亲近信任我这样一个家境贫寒居住在偏僻之地的人,我怎么能独自心安理得地默不作声,将这件事就这样算了呢!何况他之前邀请我,我只是因为老母健在的原因才辞谢;如今母亲已经寿终正寝了,我应当为了解自己的人去效力了。”

于是,聂政向西出发来到濮阳,见到严仲子说:“从前我没有答应仲子先生的原因,是因为母亲健在;如今老母不幸已经过世,仲子想要向谁寻仇?就请允许我替您处理这件事情吧。”

于是严仲子将事情详细地告诉聂政说:“我的仇人是韩国的宰相侠累,侠累还是韩王的叔父,他的家族势力强大,人数众多,他居住的地方守卫非常严密。我想派人前去刺杀他,始终没有成功。现在承蒙您不嫌弃,请允许我增派些车马壮士充当您的助手。”

聂政说:“韩国和卫国,两国相距并不远。现在要杀韩国的国相,而这位国相又是国君的亲戚,这种情况下,不可以派那么多人。因为人一旦多了,不可能不出现什么岔子,一旦出了岔子,就会泄露消息,一旦泄露了消息,那么韩国全国上下都会与仲子你为敌,这岂不是十分危险吗!”

于是聂政谢绝了车马人众,向严仲子辞别后独自出发了。

\begin{yuanwen}
杖剑至韩,韩相侠累方坐府上,持兵戟而卫侍者甚(卫/众)。聂政直入,上阶刺杀侠累,左右大乱。聂政大呼,所击杀者数十人,因自皮面决眼\footnote{割毁面皮,挖掉眼珠。},自屠出肠,遂以死。
\end{yuanwen}

聂政带着宝剑来到韩国,韩国的宰相侠累刚好坐在堂上,侠累身边手里拿着兵器守卫的人非常多。聂政直接冲到堂上,飞上台阶刺杀了侠累。左右的人方寸大乱,大声叫喊着,杀死了数十人,然后聂政自毁容貌,挖出双眼,又自己剖腹,腹中的肠子都流了出来,就这样死掉了。

\begin{yuanwen}
韩取聂政尸暴于市,购问\footnote{悬赏询问。}莫知谁子。于是韩县(购)之,有能言杀相侠累者予千金。久之莫知也。
\end{yuanwen}

韩国人将聂政的尸体在集市上公开,悬赏询问,没有人认识他是谁家的子弟。于是韩王悬赏征求认识刺客的人,有能够说出刺杀宰相侠累的人,赏赐千金。但过了很长时间,也没有人知道他是谁。

\begin{yuanwen}
政姊荣闻人有刺杀韩相者,贼不得,国不知其名姓,暴其尸而县之千金,乃於邑\footnote{同“呜咽”,低声哭泣。}曰:“其是吾弟与?嗟乎,严仲子知吾弟!”

立起,如韩,之市,而死者果政也,伏尸哭极哀,曰:“是轵深井里所谓聂政者也。”

市行者诸众人皆曰:“此人暴虐吾国相,王县购其名姓千金,夫人不闻与?何敢来识之也?”

荣应之曰:“闻之。然政所以蒙污辱自弃于市贩之间者,为老母幸无恙,妾未嫁也。亲既以天年下世,妾已嫁夫,严仲子乃察举吾弟困污之中而交之,泽厚矣,可奈何!士固为知己者死,今乃以妾尚在之故,重自刑以绝从\footnote{同“绝踪”,断绝跟踪追查的线索。},妾其奈何畏殁身之诛,终灭贤弟之名!”

大惊韩市人。乃大呼天者三,卒于邑悲哀而死政之旁。
\end{yuanwen}

聂政的姐姐聂荣听说有人刺杀了韩国的宰相,不知道凶手是谁,韩国人不知道凶手的姓名,所以将凶手的尸首暴露在集市上并且悬赏千金认人,就哭着说:“这难道是我的弟弟吗?唉,严仲子了解我的弟弟!”

她随即起身,来到韩国,直接来到集市上辨认尸体,死去的人竟然真的是聂政,就趴在尸体上,哭得非常悲哀,说道:“这是轵县深井里一个叫作聂政的人。”

集市上很多过路人都说:“这个人杀死了我国的宰相,韩王正在悬赏千金寻人辨认呢,夫人难道没听说这件事吗?怎么还敢来认尸呢?”

聂荣回答说:“我听说这件事了。然而聂政当初之所以选择承受屈辱,委身于市井商贩之中,是因为老母尚在人世,而且我尚未出嫁。现在母亲已经寿终正寝,而我也已经嫁了丈夫,严仲子能够在我弟弟困辱的境况中找到他跟他交往,恩泽深厚,他能怎么办呢!勇士本就应该为他的知己死,现在我弟弟因为我仍然在世的原因,又将自己的身体严重摧残,以此来断绝牵累别人的线索。我怎么能因为担心自己遭到杀身之祸,而埋没贤弟的名声呢!”

这话让韩国的百姓大受震惊。她大声连呼三声“天哪”,最后因为过度悲伤而死在聂政的旁边。

\begin{yuanwen}
晋、楚、齐、卫闻之,皆曰:“非独政能也,乃其姊亦烈女也。乡使政诚知其姊无濡忍\footnote{容忍,忍耐。}之志,不重暴骸之难,必绝险千里以列其名,姊弟俱僇于韩市者,亦未必敢以身许严仲子也。严仲子亦可谓知人能得士矣!”
\end{yuanwen}

晋、楚、齐、卫等国的人听说这件事,都说:“不只聂政是个能人,连他的姐姐也是一个性情刚烈的女子。聂政如果真的知道他的姐姐没有含忍的性格,不畏暴露尸骨的灾难,一定会穿越千里险阻来公布他的姓名,最后姐弟一起死在韩国的集市上的话,聂政也不一定就敢将自己的生命许托给严仲子。严仲子也可以说是懂得分辨人才,从而获得了聂政这样的贤才啊!”

\begin{yuanwen}
其后二百二十馀年秦有荆轲之事。
\end{yuanwen}

又过了二百二十多年,秦国有荆轲的事迹。

\begin{yuanwen}
荆轲者,卫人也。其先乃齐人,徙于卫,卫人谓之庆卿。而之燕,燕人谓之荆卿。
\end{yuanwen}

荆轲,是卫国人。他的祖先原本是齐国人,后来迁徙到卫国,卫国人称荆轲为庆卿。后来荆轲到了燕国,燕国人称他为荆卿。

\begin{yuanwen}
荆卿好读书击剑,以术说卫元君,卫元君不用。其后秦伐魏,置东郡,徙卫元君之支属于野王。
\end{yuanwen}

荆轲喜欢读书、击剑,曾以剑术游说卫元君,卫元君没有重用他。后来,秦国讨伐魏国,在魏国设置了东郡,将卫元君的旁支亲属迁徙到了野王。

\begin{yuanwen}
荆轲尝游过榆次,与盖聂论剑,盖聂怒而目\footnote{瞪眼逼视。}之。荆轲出,人或言复召荆卿。盖聂曰:“曩者\footnote{过去。这里指刚才。nǎng}吾与论剑有不称者,吾目之;试往,是宜去,不敢留。”

使使往之主人,荆卿则已驾而去榆次矣。使者还报,盖聂曰:“固去也,吾曩者目摄\footnote{通“慑”,吓唬。}之!”
\end{yuanwen}

荆轲游历期间经过榆次,跟盖聂讨论剑术,盖聂愤怒地瞪着他。荆轲出去后,有人劝盖聂再将荆轲召回来。盖聂说:“刚刚我跟荆轲讨论剑术的时候,彼此见解有不符合的地方,我瞪了他一眼;尝试去找找看吧,但是他应该离开了,不敢再留在这里了。”

盖聂派使者到荆轲寄宿的主人那里去寻找,发现荆轲早已驾车离开榆次了。使者回来报告,盖聂说:“他原本就该离开,我之前用眼睛瞪他,他畏惧了。”

\begin{yuanwen}
荆轲游于邯郸,鲁句践与荆轲博,争道,鲁句践怒而叱之,荆轲嘿而逃去,遂不复会。
\end{yuanwen}

荆轲游历到邯郸,鲁句践和荆轲下棋,因为互相争执棋路,鲁句践发怒,斥责了荆轲,荆轲默默无语,悄悄溜走了,从这以后,荆轲不再跟鲁句践见面。

\begin{yuanwen}
荆轲既至燕,爱燕之狗屠及善击筑\footnote{古代弦乐器,似琴。}者高渐离。荆轲嗜酒,日与狗屠及高渐离饮于燕市,酒酣以往,高渐离击筑,荆轲和而歌于市中,相乐也,已而相泣,旁若无人者。荆轲虽游于酒人乎,然其为人沈深好书;其所游诸侯,尽与其贤豪长者相结。其之燕,燕之处士田光先生亦善待之,知其非庸人也。
\end{yuanwen}

荆轲到达燕国后,跟燕国一个杀狗的屠夫以及擅长击筑的高渐离十分投缘。荆轲喜欢饮酒,每天都在燕国的集市上与屠夫和高渐离一起喝酒,喝到兴致高昂以后,高渐离击着筑,荆轲在集市上和着高渐离的节拍唱着歌,彼此都很高兴,很快又相对哭泣,好像旁边没有人一样。虽然荆轲与其他酒徒们素有交往,但是荆轲为人却深沉稳重,喜欢读书;他游历各诸侯国,所结交的都是当地德高望重的名士。他到达燕国以后,燕国的隐士田光先生也友好地接待他,知道他并不是一个普通人。

\begin{yuanwen}
居顷之,会燕太子丹质秦亡归燕。燕太子丹者,故尝质于赵,而秦王政生于赵,其少时与丹驩。及政立为秦王,而丹质于秦。秦王之遇燕太子丹不善,故丹怨而亡归。归而求为报秦王者,国小,力不能。其后秦日出兵山东以伐齐、楚、三晋,稍蚕食诸侯,且至于燕,燕君臣皆恐祸之至。太子丹患之,问其傅鞠\footnote{jū}武。

武对曰:“秦地遍天下,威胁韩、魏、赵氏,北有甘泉、谷口之固,南有泾、渭之沃,擅\footnote{拥有,据有。}巴、汉之饶,右陇、蜀之山,左关、殽之险,民众而士厉,兵革有馀。意有所出,则长城之南,易水以北,未有所定也。奈何以见陵\footnote{被欺凌。}之怨,欲批其逆鳞\footnote{传说中龙颈部的倒鳞,一旦触及,会使龙发怒。}哉!”

丹曰:“然则何由?”

对曰:“请入图之。”
\end{yuanwen}

没过多久,恰好遇到在秦国做人质的燕太子丹逃回燕国。燕国的太子丹,过去曾经在赵国做人质,而秦王嬴政在赵国出生,他少年时与燕国的太子丹十分要好。等到赢政即位当上了秦王后,太子丹又到秦国做人质。秦王对燕国的太子丹不友好,因此太子丹十分怨恨而逃回了燕国。回到燕国以后,太子丹四处寻找报复秦王的办法,燕国弱小,力不能及。后来,秦国经常出兵到崤山以东的地区来攻击齐国、楚国和三晋,像蚕吃桑叶一样渐渐将诸侯国的土地吞并,就快轮到燕国了。燕国的君臣都十分担心灾祸来临。太子丹担心这件事,向他的老师鞠武询问。

鞠武回答说:“秦国的领土遍布天下,对韩国、魏国、赵国都是一个威胁,北有甘泉、谷口这样坚固险要的关塞,南有泾河、渭河流域肥沃的原野,占据富饶的巴郡、汉中郡,右边有陇、蜀这样的高山险阻,左边有函谷关、崤山这样的天险,国内百姓众多而兵士勇猛,武器装备充足。假如它有向外扩张的意图,那么长城以南、易水以北的地方就都没有办法保全了。您怎么能因为自己被欺侮了就心生怨恨,想要去触碰秦王的逆鳞呢!”

太子丹说:“既然如此,那我们该怎么办呢?”

鞠武回答说:“请允许我进一步考虑这件事。”

\begin{yuanwen}
居有间,秦将樊於期得罪于秦王,亡之燕,太子受而舍之。鞠武谏曰:“不可。夫以秦王之暴而积怒于燕,足为寒心,又况闻樊将军之所在乎?是谓‘委肉当饿虎之蹊’也,祸必不振矣!虽有管、晏,不能为之谋也。原太子疾遣樊将军入匈奴以灭口。请西约三晋,南连齐、楚,北购\footnote{通“媾”,媾和,讲和。}于单于,其后乃可图也。”

太子曰:“太傅之计,旷日弥久,心惛然\footnote{烦乱。惛,糊涂。},恐不能须臾。且非独于此也,夫樊将军穷困于天下,归身于丹,丹终不以迫于彊秦而弃所哀怜之交,置之匈奴,是固丹命卒之时也。原太傅更虑之。”

鞠武曰:“夫行危欲求安,造祸而求福,计浅而怨深,连结一人之后交,不顾国家之大害,此所谓‘资怨而助祸’矣。夫以鸿毛燎于炉炭之上,必无事矣。且以雕鸷\footnote{zhì}之秦,行怨暴之怒,岂足道哉!燕有田光先生,其为人智深而勇沈,可与谋。”

太子曰:“原因太傅而得交于田先生,可乎?”

鞠武曰:“敬诺。”

出见田先生,道“太子原图国事于先生也”。

田光曰:“敬奉教。”乃造焉。
\end{yuanwen}\begin{yuanwen}

\end{yuanwen}\begin{yuanwen}

\end{yuanwen}\begin{yuanwen}

\end{yuanwen}\begin{yuanwen}

\end{yuanwen}\begin{yuanwen}

\end{yuanwen}\begin{yuanwen}

\end{yuanwen}\begin{yuanwen}

\end{yuanwen}\begin{yuanwen}

\end{yuanwen}\begin{yuanwen}

\end{yuanwen}\begin{yuanwen}

\end{yuanwen}\begin{yuanwen}

\end{yuanwen}\begin{yuanwen}

\end{yuanwen}\begin{yuanwen}

\end{yuanwen}\begin{yuanwen}

\end{yuanwen}\begin{yuanwen}



太子逢迎,卻行为导,跪而蔽席。田光坐定,左右无人,太子避席而请曰:“燕秦不两立,原先生留意也。”田光曰:“臣闻骐骥盛壮之时,一日而驰千里;至其衰老,驽马先之。今太子闻光盛壮之时,不知臣精已消亡矣。虽然,光不敢以图国事,所善荆卿可使也。”太子曰:“原因先生得结交于荆卿,可乎?”田光曰:“敬诺。”即起,趋出。太子送至门,戒曰:“丹所报,先生所言者,国之大事也,原先生勿泄也!”田光俯而笑曰:“诺。”偻行见荆卿,曰:“光与子相善,燕国莫不知。今太子闻光壮盛之时,不知吾形已不逮也,幸而教之曰‘燕秦不两立,原先生留意也’。光窃不自外,言足下于太子也,原足下过太子于宫。”荆轲曰:“谨奉教。”田光曰:“吾闻之,长者为行,不使人疑之。今太子告光曰:‘所言者,国之大事也,原先生勿泄’,是太子疑光也。夫为行而使人疑之,非节侠也。”欲自杀以激荆卿,曰:“原足下急过太子,言光已死,明不言也。”因遂自刎而死。

荆轲遂见太子,言田光已死,致光之言。太子再拜而跪,膝行流涕,有顷而后言曰:“丹所以诫田先生毋言者,欲以成大事之谋也。今田先生以死明不言,岂丹之心哉!”荆轲坐定,太子避席顿首曰:“田先生不知丹之不肖,使得至前,敢有所道,此天之所以哀燕而不弃其孤也。今秦有贪利之心,而欲不可足也。非尽天下之地,臣海内之王者,其意不厌。今秦已虏韩王,尽纳其地。又举兵南伐楚,北临赵;王翦将数十万之众距漳、鄴,而李信出太原、云中。赵不能支秦,必入臣,入臣则祸至燕。燕小弱,数困于兵,今计举国不足以当秦。诸侯服秦,莫敢合从。丹之私计愚,以为诚得天下之勇士使于秦,闚以重利;秦王贪,其势必得所原矣。诚得劫秦王,使悉反诸侯侵地,若曹沫之与齐桓公,则大善矣;则不可,因而刺杀之。彼秦大将擅兵于外而内有乱,则君臣相疑,以其间诸侯得合从,其破秦必矣。此丹之上原,而不知所委命,唯荆卿留意焉。”久之,荆轲曰:“此国之大事也,臣驽下,恐不足任使。”太子前顿首,固请毋让,然后许诺。于是尊荆卿为上卿,舍上舍。太子日造门下,供太牢具,异物间进,车骑美女恣荆轲所欲,以顺適其意。

久之,荆轲未有行意。秦将王翦破赵,虏赵王,尽收入其地,进兵北略地至燕南界。太子丹恐惧,乃请荆轲曰:“秦兵旦暮渡易水,则虽欲长侍足下,岂可得哉!”荆轲曰:“微太子言,臣原谒之。今行而毋信,则秦未可亲也。夫樊将军,秦王购之金千斤,邑万家。诚得樊将军首与燕督亢之地图,奉献秦王,秦王必说见臣,臣乃得有以报。”太子曰:“樊将军穷困来归丹,丹不忍以己之私而伤长者之意,原足下更虑之!”

荆轲知太子不忍,乃遂私见樊于期曰:“秦之遇将军可谓深矣,父母宗族皆为戮没。今闻购将军首金千斤,邑万家,将奈何?”于期仰天太息流涕曰:“于期每念之,常痛于骨髓,顾计不知所出耳!”荆轲曰:“今有一言可以解燕国之患,报将军之仇者,何如?”于期乃前曰:“为之奈何?”荆轲曰:“原得将军之首以献秦王,秦王必喜而见臣,臣左手把其袖,右手揕其匈,然则将军之仇报而燕见陵之愧除矣。将军岂有意乎?”樊于期偏袒搤捥而进曰:“此臣之日夜切齿腐心也,乃今得闻教!”遂自刭。太子闻之,驰往,伏尸而哭,极哀。既已不可奈何,乃遂盛樊于期首函封之。

于是太子豫求天下之利匕首,得赵人徐夫人匕首,取之百金,使工以药焠之,以试人,血濡缕,人无不立死者。乃装为遣荆卿。燕国有勇士秦舞阳,年十三,杀人,人不敢忤视。乃令秦舞阳为副。荆轲有所待,欲与俱;其人居远未来,而为治行。顷之,未发,太子迟之,疑其改悔,乃复请曰:“日已尽矣,荆卿岂有意哉?丹请得先遣秦舞阳。”荆轲怒,叱太子曰:“何太子之遣?往而不返者,竖子也!且提一匕首入不测之彊秦,仆所以留者,待吾客与俱。今太子迟之,请辞决矣!”遂发。

太子及宾客知其事者,皆白衣冠以送之。至易水之上,既祖,取道,高渐离击筑,荆轲和而歌,为变徵之声,士皆垂泪涕泣。又前而为歌曰:“风萧萧兮易水寒,壮士一去兮不复还!”复为羽声慷慨,士皆瞋目,发尽上指冠。于是荆轲就车而去,终已不顾。

遂至秦,持千金之资币物,厚遗秦王宠臣中庶子蒙嘉。嘉为先言于秦王曰:“燕王诚振怖大王之威,不敢举兵以逆军吏,原举国为内臣,比诸侯之列,给贡职如郡县,而得奉守先王之宗庙。恐惧不敢自陈,谨斩樊于期之头,及献燕督亢之地图,函封,燕王拜送于庭,使使以闻大王,唯大王命之。”秦王闻之,大喜,乃朝服,设九宾,见燕使者咸阳宫。荆轲奉樊于期头函,而秦舞阳奉地图柙,以次进。至陛,秦舞阳色变振恐,群臣怪之。荆轲顾笑舞阳,前谢曰:“北蕃蛮夷之鄙人,未尝见天子,故振慴。原大王少假借之,使得毕使于前。”秦王谓轲曰:“取舞阳所持地图。”轲既取图奏之,秦王发图,图穷而匕首见。因左手把秦王之袖,而右手持匕首揕之。未至身,秦王惊,自引而起,袖绝。拔剑,剑长,操其室。时惶急,剑坚,故不可立拔。荆轲逐秦王,秦王环柱而走。群臣皆愕,卒起不意,尽失其度。而秦法,群臣侍殿上者不得持尺寸之兵;诸郎中执兵皆陈殿下,非有诏召不得上。方急时,不及召下兵,以故荆轲乃逐秦王。而卒惶急,无以击轲,而以手共搏之。是时侍医夏无且以其所奉药囊提荆轲也。秦王方环柱走,卒惶急,不知所为,左右乃曰:“王负剑!”负剑,遂拔以击荆轲,断其左股。荆轲废,乃引其匕首以擿秦王,不中,中桐柱。秦王复击轲,轲被八创。轲自知事不就,倚柱而笑,箕踞以骂曰:“事所以不成者,以欲生劫之,必得约契以报太子也。”于是左右既前杀轲,秦王不怡者良久。已而论功,赏群臣及当坐者各有差,而赐夏无且黄金二百溢,曰:“无且爱我,乃以药囊提荆轲也。”

于是秦王大怒,益发兵诣赵,诏王翦军以伐燕。十月而拔蓟城。燕王喜、太子丹等尽率其精兵东保于辽东。秦将李信追击燕王急,代王嘉乃遗燕王喜书曰:“秦所以尤追燕急者,以太子丹故也。今王诚杀丹献之秦王,秦王必解,而社稷幸得血食。”其后李信追丹,丹匿衍水中,燕王乃使使斩太子丹,欲献之秦。秦复进兵攻之。后五年,秦卒灭燕,虏燕王喜。

其明年,秦并天下,立号为皇帝。于是秦逐太子丹、荆轲之客,皆亡。高渐离变名姓为人庸保,匿作于宋子。久之,作苦,闻其家堂上客击筑,傍徨不能去。每出言曰:“彼有善有不善。”从者以告其主,曰:“彼庸乃知音,窃言是非。”家丈人召使前击筑,一坐称善,赐酒。而高渐离念久隐畏约无穷时,乃退,出其装匣中筑与其善衣,更容貌而前。举坐客皆惊,下与抗礼,以为上客。使击筑而歌,客无不流涕而去者。宋子传客之,闻于秦始皇。秦始皇召见,人有识者,乃曰:“高渐离也。”秦皇帝惜其善击筑,重赦之,乃矐其目。使击筑,未尝不称善。稍益近之,高渐离乃以铅置筑中,复进得近,举筑朴秦皇帝,不中。于是遂诛高渐离,终身不复近诸侯之人。

鲁句践已闻荆轲之刺秦王,私曰:“嗟乎,惜哉其不讲于刺剑之术也!甚矣吾不知人也!曩者吾叱之,彼乃以我为非人也!”

太史公曰:世言荆轲,其称太子丹之命,“天雨粟,马生角”也,太过。又言荆轲伤秦王,皆非也。始公孙季功、董生与夏无且游,具知其事,为余道之如是。自曹沫至荆轲五人,此其义或成或不成,然其立意较然,不欺其志,名垂后世,岂妄也哉!

曹沫盟柯,返鲁侵地。专诸进炙,定吴篡位。彰弟哭市,报主涂厕。刎颈申冤,操袖行事。暴秦夺魄,懦夫增气。
\end{yuanwen}

\part{卷八十七}
\chapter{李斯列传第二十七}

陈仁锡:「先秦文章当以李斯为第一,太史公作传,载其书五篇,绝工之文也。《李斯传》是秦外纪,凡秦之兴亡与皆具矣。」

\begin{yuanwen}
李斯者,楚上蔡人也。年少时,为郡小吏,见吏舍厕中鼠食不洁,近人犬,数惊恐之。斯入仓,观仓中鼠,食积粟,居大庑之下,不见人犬之忧。于是李斯乃叹曰:“人之贤不肖譬如鼠矣,在所自处耳!”
\end{yuanwen}

李斯,是楚国上蔡人。他年少时,曾经在郡里当过小吏,看见办公处所附近厕所里的老鼠吃不干净的东西,每当有人或者狗走近的时候,老鼠多次受惊逃跑。李斯又走进粮仓,看见粮仓中的老鼠正在吃囤积在粮仓中的粟米,它们居住在大屋子里,没有受到人或狗的惊扰。看到这种情形,李斯感慨叹息说:“一个人有才还是没有才,同这些老鼠一样,是由自己所在的环境决定的啊!”

\begin{yuanwen}
乃从荀卿学帝王之术。学已成,度楚王不足事,而六国皆弱,无可为建功者,欲西入秦。辞于荀卿曰:“斯闻得时无怠,今万乘方争时,游者主事。今秦王欲吞天下,称帝而治,此布衣\footnote{指平民百姓。}驰骛\footnote{奔走。骛,wù}之时而游说者之秋也。处卑贱之位而计不为者,此禽鹿\footnote{犹“禽兽”。}视肉,人面而能强行者耳。故诟莫大于卑贱,而悲莫甚于穷困。久处卑贱之位,困苦之地,非世而恶利,自讬(托)于无为,此非士之情也。故斯将西说秦王矣。”
\end{yuanwen}

于是李斯跟随荀子学习治理天下的帝王之术。学成之后,李斯估计楚王不值得自己侍奉,而六个诸侯国的国势都已经衰弱,没有为它们建功立业的机会,就想向西进入秦国。在临行之前,李斯向荀子告辞说:“我听说如果得到了机会,就一定不要松懈怠慢,现在拥有万辆车马的诸侯国正在争取时机,游说之士掌握朝政。现在秦王想要吞并天下,称帝治理天下,这恰好是平民百姓奔走四方、游说之士施展拳脚的大好时机。地位卑贱之人没有想着去求取功名利禄,这类人就好像禽兽看到肉才想去吃,空长了一副人的面孔只是勉强可以直立行走罢了。因此最大的耻辱莫过于身份卑贱,最大的悲哀莫过于贫穷。长期处在卑贱的地位和贫苦的环境之中,还要愤世嫉俗,厌恶功名利禄,假托自己与世无争,这并不是士子原本的性情。因此我将要西行前去游说秦王。”

\begin{yuanwen}
至秦,会庄襄王卒,李斯乃求为秦相文信侯吕不韦舍人;不韦贤之,任以为郎。李斯因以得说,说秦王曰:“胥人\footnote{小人,没有才能的人。}者,去其几\footnote{同“机”。}也。成大功者,在因瑕衅而遂忍之。昔者秦穆公之霸,终不东并六国者,何也?诸侯尚众,周德未衰,故五伯迭兴,更尊周室。自秦孝公以来,周室卑微,诸侯相兼,关东为六国,秦之乘胜役诸侯,盖六世矣。今诸侯服秦,譬若郡县。夫以秦之强,大王之贤,由\footnote{通“犹”。如同,好像。}灶上骚\footnote{通“扫”。}除,足以灭诸侯,成帝业,为天下一统,此万世之一时也。今怠而不急就,诸侯复强,相聚约从,虽有黄帝之贤,不能并也。”

秦王乃拜斯为长史,听其计,阴遣谋士赍持金玉以游说诸侯。诸侯名士可下以财者,厚遗结之;不肯者,利剑刺之。离其君臣之计,秦王乃使其良将随其后。秦王拜斯为客卿。
\end{yuanwen}

李斯到达秦国以后,刚好赶上秦庄襄王去世,于是李斯请求担任秦国相国文信侯吕不韦的门客;吕不韦十分欣赏他,于是保举他为郎官。李斯因此得到游说的机会,游说秦王说:“等待机会的人,往往都丢掉了他们的机遇。成就大功业的人,能够抓住可乘之机并能下狠心。从前,秦穆公雄霸天下,但是最后都没有向东进军吞并六国,这是为什么呢?是因为诸侯国还很多,周朝的德望没有衰落殆尽,所以五霸轮流兴起,以周朝为尊。自秦孝公以来,周朝宗室日渐衰微,各诸侯国之间也是互相兼并,函谷关以东地区被六国瓜分,秦国乘胜奴役诸侯,现在算起来总共有六代了。如今诸侯服从秦国的命令,就如同郡县服从朝廷一样。凭借秦国的强大,秦王的圣明,就如同扫除灶上的灰尘一样,足以扫平诸侯,成就帝业,统一天下,这是一个千载难逢的时机。现在如果懈怠而没有抓紧做这件事情的话,等到其他诸侯国再强大起来,重新订立合纵抗秦的盟约,即使有黄帝的贤明,也没有办法吞并它们了。”

于是,秦王委任李斯为长史,听从他的计谋,暗中派遣谋士携带金玉珍宝前去各国进行游说。对各诸侯国里的名士能用礼物进行收买的,就多送礼物加以收买;如果不能收买,就用利剑将他们杀掉。这些方法都是为了离间诸侯国君臣之间的关系,然后秦王就派遣秦国的良将出兵攻打。秦王委任李斯为客卿。

\begin{yuanwen}
会韩人郑国来间\footnote{从事间谍活动。}秦,以作注溉渠,已而觉。秦宗室大臣皆言秦王曰:“诸侯人来事秦者,大抵为其主游间于秦耳,请一切逐客。”

李斯议亦在逐中。斯乃上书曰:
\end{yuanwen}

刚好在这个时候,韩国人郑国以修筑灌溉水渠的名义,来到秦国做间谍,不久被发觉。秦国的王公贵族以及大臣们都对秦王说:“从诸侯国前来侍奉秦王的人,大多数只是为了他们的国君而来游说、离间秦国罢了,请求大王将这些客卿全部驱逐出境。”

李斯也在被驱逐的客卿之中。于是李斯上书说:

\begin{yuanwen}
臣闻吏议逐客,窃以为过矣。昔缪\footnote{同“穆”。}公求士,西取由余于戎,东得百里奚(傒)于宛,迎蹇叔于宋,来丕豹、公孙支于晋。此五子者,不产于秦,而缪公用之,并国二十,遂霸西戎。孝公用商鞅之法,移风易俗,民以殷盛,国以富强,百姓乐用,诸侯亲服,获楚、魏之师,举地千里,至今治强。惠王用张仪之计,拔三川之地,西并巴、蜀,北收上郡,南取汉中,包九夷,制鄢、郢,东据成皋之险,割膏腴之壤,遂散六国之从,使之西面事秦,功施\footnote{yì,延续。}到今。昭王得范睢,废穰侯,逐华阳,强公室,杜私门,蚕食诸侯,使秦成帝业。此四君者,皆以客之功。由此观之,客何负于秦哉!向使四君卻客而不内,疏士而不用,是使国无富利之实而秦无强大之名也。
\end{yuanwen}

我听说官员们商议要驱逐客卿的事情,我私下认为这种做法是错误的。昔日秦穆公寻求贤士,在西戎寻找到由余,在东边楚国的宛地找到了百里傒,在宋地找到了蹇叔,从晋国招揽来了丕豹、公孙支。这五个人虽然都不是在秦国出生的,但是秦穆公仍然重用他们,并且在此期间秦穆公吞并了二十多个国家,这才得以在西戎称霸。秦孝公采纳施行商鞅的新法政策,改变民风民俗,所以百姓才得以殷实兴盛,国家才得以富足强大,百姓们乐意为自己的国家效力,其他国家也愿意真心归顺,这才打败了楚国、魏国的军队,占领了上千里土地,到现在仍然政治安定,国家强盛。秦惠王采纳张仪的计谋,占领了三川地区,又向西进军吞并了巴、蜀地区,向北进军攻占了上郡,向南进军占领了汉中,地域囊括九方蛮夷之地,控制了鄢、郢等地,在东面占领了地势险要的成皋地区,割取了肥沃的土壤,于是瓦解了六国的合纵联盟策略,使他们面向西方事奉秦国,这样的功绩一直延续到今天。秦昭王得到范睢这样的良臣,废黜穰侯,驱逐华阳君,让公室进一步强大,杜绝了私门权贵势力的进一步扩大,像春蚕食用桑叶一般,慢慢吞并诸侯的土地,最终为秦国统一天下大业的实现奠定了基础。这四位君主,都是依靠了其他诸侯国客卿的力量。这样看来,客卿哪里有负于秦国呢!假如这四位君主拒绝了其他诸侯国的客卿,不接纳他们,疏远他国的士人而不去重用他们,这样就会让秦国既无富足之实,又无强大之名。

\begin{yuanwen}
今陛下致昆山之玉,有随、和之宝,垂明月之珠,服太阿之剑,乘纤离之马,建翠凤之旗,树灵鼍\footnote{tuó}之鼓。此数宝者,秦不生一焉,而陛下说之,何也?必秦国之所生然后可,则是夜光之璧不饰朝廷,犀象之器不为玩好,郑、卫之女不充后宫,而骏良駃騠不实外厩,江南金锡不为用,西蜀丹青不为采\footnote{同“彩”。}。所以饰后宫、充下陈\footnote{指堂下、庭院等存放财物的地方。陈,堂下至门的过道。}、娱心意、说耳目者,必出于秦然后可,则是宛珠之簪、傅玑\footnote{镶着小珠子。傅,通“附”。玑,不圆的珠子,这里泛指珠子。}之珥,阿缟之衣、锦绣之饰不进于前,而随俗雅化佳冶窈窕赵女不立于侧也。夫击甕(瓮)叩缶弹筝搏髀\footnote{拍击大腿。髀,bì,大腿。},而歌呼呜呜快耳者,真秦之声也;《郑》、《卫》、《桑间》、《昭》、《虞》、《武》、《象》者,异国之乐也。今弃击瓮叩缶而就《郑》、《卫》,退弹筝而取《昭》、《虞》,若是者何也?快意当前,适观而已矣。今取人则不然。不问可否,不论曲直,非秦者去,为客者逐。然则是所重者在乎色乐珠玉,而所轻者在乎人民也。此非所以跨海内制诸侯之术也。
\end{yuanwen}

现在陛下得到了昆仑山的美玉,得到了随侯之珠、和氏之璧,腰上挂着明月珠,佩着太阿剑,驾着纤离马,举着翠凤旗,摆着灵鼍鼓。所有这些宝物,没有一样是秦国本土出产的,可是陛下仍然十分喜欢它们,这是什么原因呢?倘若一定要是秦国所产然后才使用,那么就不能用夜光之璧来装饰朝廷,不能赏玩犀牛角、象牙制成的器皿,后宫不能有郑国、卫国的美女,马厩中没有像駃騠那样的良马,不该使用江南的金锡,不该用西蜀的丹青当颜料。所有您用来装饰后宫、充当姬妾、令您赏心悦目、怡情悦耳的,如果一定是要出自秦国,然后才使用的话,那么,用宛地珍珠装饰的簪子、镶嵌玑珠的耳环、东阿白绢制成的衣裳、刺绣华美的装饰品,就不会呈现在您的面前,那高雅而又时髦、文静而又漂亮的赵国女子就不会站在您的身边了。而那些敲击着瓦坛瓦罐、弹着秦筝、拍打大腿、通过呜呜叫喊来满足大众欣赏要求的,才是真正的秦国音乐。像《郑》《卫》《桑间》《昭》《虞》《武》《象》这些乐曲,都是其他国家的音乐。现在大王抛弃敲打瓦坛瓦罐转而欣赏《郑》《卫》之声,不去听弹筝转而欣赏《昭》《虞》这样的曲子,这是为什么呢?只是贪图眼前的快乐,为了满足自己的视觉和听觉需求罢了。可是如今您用人却不是这样,您不管这个人能用不能用,也不问是非曲直,只要这个人不是秦国人就一律辞退,只要是客卿就一律驱逐。如此看来,陛下看重的不过只是美女、音乐、珍珠、宝玉这一类东西,而轻视的则是人才。这不是一个能够统一天下、制服诸侯的好方法。

\begin{yuanwen}
臣闻地广者粟多,国大者人众,兵强则士勇。是以太山不让土壤,故能成其大;河海不择细流,故能就其深;王者不卻众庶,故能明其德。是以地无四方,民无异国,四时充美,鬼神降福,此五帝、三王之所以无敌也。今乃弃黔首\footnote{平民百姓。}以资敌国,卻宾客以业诸侯,使天下之士退而不敢西向,裹足不入秦,此所谓“藉寇兵而赍盗粮”者也。
\end{yuanwen}

我听说土地宽广的地方所产的粮食也多,国家广大就人口众多,军队强盛士兵们就很勇敢。因此泰山不排斥任何泥土,才能够堆积成那样高大的山体;江河不挑剔任何细小的溪流,才能汇聚成如此深广的大海;而成就霸业的人不应该抛弃这么多平民百姓,这样才能显示出他伟大的德行。因此土地不分东南西北,百姓不分这国那国,一年四季的生活都充裕美好,神鬼赐予福泽,这就是五帝、三王能够天下无敌的原因所在。现在您竟然抛弃了百姓,以此来帮助敌国,排斥宾客而让他们为其他诸侯国建立功业,让天下有才能的人都向后退而不敢西行,停住脚步而不敢进入秦国,这就是人们所说的“把武器借给敌人,把粮食送给盗贼”啊!

\begin{yuanwen}
夫物不产于秦,可宝者多;士不产于秦,而原忠者众。今逐客以资敌国,损民以益雠,内自虚而外树怨于诸侯,求国无危,不可得也。
\end{yuanwen}

不是秦国出产的物品,值得视为宝物的很多;不是出生在秦国的士人,而乐意效忠秦国的也不在少数。现在您将客卿驱逐出国来资助与您敌对的国家,损害自己的百姓来让仇人更加强大,在内部削弱自己的实力同时又在外面与诸侯结下怨恨,想要让国家没有什么危险,是不可能的。

\begin{yuanwen}
秦王乃除逐客之令,复李斯官,卒用其计谋。官至廷尉。二十馀年,竟并天下,尊主为皇帝,以斯为丞相。夷郡县城,销其兵刃,示不复用。使秦无尺土之封,不立子弟为王,功臣为诸侯者,使后无战攻之患。
\end{yuanwen}

于是,秦王下令废除了驱逐客卿的诏令,重新恢复了李斯的官职,最后采用了他的计谋。李斯也升官到廷尉之职。秦国经过二十多年的战争,终于兼并了天下,尊称国王为“皇帝”,委任李斯为丞相。并将各国郡县的城墙全部夷平,销毁各郡县的武器,表示不会再使用兵器。使秦国的土地没有一尺分封给别人,就连皇帝的儿子、兄弟也没有被立为王,更不把建功立业的功臣封为诸侯,想要通过这种方法让国家永远没有战乱的祸患。

\begin{yuanwen}
始皇三十四年,置酒咸阳宫,博士仆射周青臣等颂称始皇威德。齐人淳于越进谏曰:“臣闻之,殷周之王千馀岁,封子弟功臣自为支辅\footnote{像膀臂一样辅佐。支,通“肢”。}。今陛下有海内,而子弟为匹夫,卒有田常、六卿之患,臣无辅弼,何以相救哉?事不师古而能长久者,非所闻也。今青臣等又面谀以重陛下过,非忠臣也。”

始皇下其议丞相。丞相谬其说,绌\footnote{通“黜”。贬斥。}其辞,乃上书曰:“古者天下散乱,莫能相一,是以诸侯并作。语皆道古以害今,饰虚言以乱实,人善其所私学,以非上所建立。今陛下并有天下,别白黑而定一尊;而私学乃相与非法教之制,闻令下,即各以其私学议之,入则心非,出则巷议,非主以为名,异趣\footnote{标新立异。趣,趋向。}以为高,率群下以造谤。如此不禁,则主势降乎上,党与成乎下。禁之便。臣请诸有文学《诗》、《书》百家语者,蠲\footnote{除,免。}除去之。令到满三十日弗去,黥为城旦。所不去者,医药卜筮种树之书。若有欲学者,以吏为师。”
\end{yuanwen}

秦始皇三十四年(前213年),秦始皇在咸阳宫摆设酒席招待群臣,博士仆射周青臣等人一起歌颂始皇的威名功德。齐人淳于越进言劝谏说:“我听说,殷商和周朝之所以能够统治天下一千多年,全靠分封子弟及有功之臣作为统治者的左膀右臂。现在陛下虽然得以一统天下,但是您的子弟却都还是普通的平民百姓,如果一旦出现了像田常、六卿谋朝篡位的灾祸,朝廷上又没有什么强有力的辅佐大臣,靠谁来相救呢?办事不向古代有经验的人学习而长期统治的朝代,我从来没有听说过。现在周青臣等人又在陛下面前阿谀奉承,这是在加重您的错误,并不是真正的忠臣。”

始皇将这种议论交给李斯处理。李斯认为这是一种荒谬的看法,所以决定废弃不用,于是上书说:“古时候天下四分五裂,没有谁能够统一天下,因此各诸侯纷纷起兵,互不相让。通常大家都赞颂古代而否定当代,用一些虚夸不实的文辞来扰乱当今社会的实际情况,人们都觉得只有自己一派的学问才是最好的,因此想要否定皇帝颁布的法令制度。如今陛下一统天下,分清黑白是非,海内都以皇帝为尊;而诸子百家众多学派却在一起对朝廷颁布的法令制度任意批评,听说朝廷有诏令下达,立即就用自己学派的观点来议论这个诏令,回到家中便心生不满,走出门外则一起在街头巷尾议论纷纷,想要通过批评君主的方法来为自己博得名声,认为自己的看法与朝廷的诏令不一样,这样就能显示出自己的本领高,并带领下层群众百姓一起来诽谤。放任这种情况而不加以禁止的话,上面君主的权势会下降,下面也会形成结党营私的帮派。还是禁止为好。我请求陛下恩准将人们收藏的《诗》、《书》以及诸子百家的全部著作,都一概清除干净。诏令下达三十天之后,如果还有人不服从诏令,就判处黥刑,做筑城苦役。医药、占卜、种植等书籍可以不在清除之列。如果有想学习法令的人,可以把官吏当成老师。”

\begin{yuanwen}
始皇可其议,收去《诗》《书》百家之语以愚百姓,使天下无以古非今。明法度,定律令,皆以始皇起。同文书。治离宫别馆,周遍天下。明年,又巡狩,外攘四夷,斯皆有力焉。
\end{yuanwen}

秦始皇批准了李斯的建议,没收了《诗》、《书》以及诸子百家的所有著作,通过这种方法让人民变得愚昧无知,让天下的人没有办法用古代的事情来批评现在的朝廷。秦始皇又开始修明法制,制定律令。他还统一了文字。在全国各地大兴土木,修建离宫别馆。第二年,秦始皇又四处巡视,平定了四面八方的少数民族叛乱,所有这一切,李斯都出了不少力。

\begin{yuanwen}
斯长男由为三川守,诸男皆尚秦公主,女悉嫁秦诸公子。三川守李由告归咸阳,李斯置酒于家,百官长皆前为寿,门廷车骑以千数。李斯喟然而叹曰:“嗟乎!吾闻之荀卿曰‘物禁大盛’。夫斯乃上蔡布衣,闾巷之黔首,上不知其驽下\footnote{才能低下。},遂擢至此。当今人臣之位无居臣上者,可谓富贵极矣。物极则衰,吾未知所税驾\footnote{停车。借指休息或归宿。}也!”
\end{yuanwen}

李斯的长子李由担任三川郡守,儿子们娶的是秦国公主,女儿们嫁的是秦国的皇族子弟。三川郡守李由请假回咸阳的时候,李斯在家中摆下酒宴,有身份的文武百官都前去给李斯祝寿,庭院的门前停放着数以千计的车马。李斯感慨地长叹说:“唉!我曾经听荀卿说过‘做事情一定不要做得过了头’。想我李斯原本只是一个上蔡的平民,一个街巷里的普通百姓,皇帝不嫌弃我才能低下,才将我提拔到现在这样高的职位。现在所有的臣子当中,没有谁能比我地位更高,可以说是荣华富贵达到了极致。但是事物发展到极点就开始衰落,我不知道自己的归宿在哪里啊!”

\begin{yuanwen}
始皇三十七年十月,行出游会稽,并海上,北抵琅邪。丞相斯、中车府令赵高兼行符玺令事,皆从。始皇有二十馀子,长子扶苏以数直谏上,上使监兵上郡,蒙恬为将。少子胡亥爱,请从,上许之。馀子莫从。
\end{yuanwen}

秦始皇三十七年(前210年)十月,始皇出游巡视会稽山,沿海北上,一直来到琅邪山。丞相李斯和中车府令兼符玺令赵高都跟随始皇一同前往。秦始皇一共有二十多个儿子,长子扶苏因为先后多次向皇帝直言进谏,被始皇派到上郡去监督军队,蒙恬担任将军。始皇的小儿子胡亥深受始皇宠爱,要求跟随始皇出游,始皇答应了。其他儿子都没有跟随。

\begin{yuanwen}
其年七月,始皇帝至沙丘,病甚,令赵高为书赐公子扶苏曰:“以兵属蒙恬,与丧会咸阳而葬。”

书已封,未授使者,始皇崩。书及玺皆在赵高所,独子胡亥、丞相李斯、赵高及幸宦者五六人知始皇崩,馀群臣皆莫知也。李斯以为上在外崩,无真太子\footnote{正式确立的太子。},故秘之。置始皇居辒辌车\footnote{可供人卧睡的车子。后用来载丧,成为丧车的代称。}中,百官奏事上食如故,宦者辄从辒辌车中可诸奏事。
\end{yuanwen}

这一年七月,秦始皇巡视到达了沙丘,病得十分厉害,命令赵高给公子扶苏写好诏书说:“将军队交给蒙恬管理,你赶快回到咸阳参加我的葬礼,然后将我安葬好。”

书信都已经封印好,但是诏书还没来得及交给使者,始皇就驾崩了。书信和印玺都在赵高手中,只有小儿子胡亥、丞相李斯、赵高以及始皇身边五六个亲近的宦官知道始皇驾崩,其他的臣子都不知道。李斯认为皇帝在外出巡视的时候去世,而且又没正式确立太子,因此应该保守秘密。他们将始皇的尸体安放在一辆既保温又通风干爽的车子中,文武百官仍然像往常一样向始皇奏禀国事,皇帝的饮食也与往常一样按时进献,车子里的宦官就假借皇帝之名,在车中批复百官上奏的事情。

\begin{yuanwen}
赵高因留所赐扶苏玺书,而谓公子胡亥曰:“上崩,无诏封王诸子而独赐长子书。长子至,即立为皇帝,而子无尺寸之地,为之柰何?”胡亥曰:“固也。吾闻之,明君知臣,明父知子。父捐命,不封诸子,何可言者!”赵高曰:“不然。方今天下之权,存亡在子与高及丞相耳,原子图之。且夫臣人与见臣于人,制人与见制于人,岂可同日道哉!”胡亥曰:“废兄而立弟,是不义也;不奉父诏而畏死,是不孝也;能薄而材譾,彊因人之功,是不能也:三者逆德,天下不服,身殆倾危,社稷不血食。”高曰:“臣闻汤、武杀其主,天下称义焉,不为不忠。卫君杀其父,而卫国载其德,孔子著之,不为不孝。夫大行不小谨,盛德不辞让,乡曲各有宜而百官不同功。故顾小而忘大,后必有害;狐疑犹豫,后必有悔。断而敢行,鬼神避之,后有成功。原子遂之!”胡亥喟然叹曰:“今大行未发,丧礼未终,岂宜以此事干丞相哉!”赵高曰:“时乎时乎,间不及谋!赢粮跃马,唯恐后时!”
\end{yuanwen}

\begin{yuanwen}
胡亥既然高之言,高曰:“不与丞相谋,恐事不能成,臣请为子与丞相谋之。”高乃谓丞相斯曰:“上崩,赐长子书,与丧会咸阳而立为嗣。书未行,今上崩,未有知者也。所赐长子书及符玺皆在胡亥所,定太子在君侯与高之口耳。事将何如?”斯曰:“安得亡国之言!此非人臣所当议也!”高曰:“君侯自料能孰与蒙恬?功高孰与蒙恬?谋远不失孰与蒙恬?无怨于天下孰与蒙恬?长子旧而信之孰与蒙恬?”斯曰:“此五者皆不及蒙恬,而君责之何深也?”高曰:“高固内官之厮役也,幸得以刀笔之文进入秦宫,管事二十馀年,未尝见秦免罢丞相功臣有封及二世者也,卒皆以诛亡。皇帝二十馀子,皆君之所知。长子刚毅而武勇,信人而奋士,即位必用蒙恬为丞相,君侯终不怀通侯之印归于乡里,明矣。高受诏教习胡亥,使学以法事数年矣,未尝见过失。慈仁笃厚,轻财重士,辩于心而诎于口,尽礼敬士,秦之诸子未有及此者,可以为嗣。君计而定之。”斯曰:“君其反位!斯奉主之诏,听天之命,何虑之可定也?”高曰:“安可危也,危可安也。安危不定,何以贵圣?”斯曰:“斯,上蔡闾巷布衣也,上幸擢为丞相,封为通侯,子孙皆至尊位重禄者,故将以存亡安危属臣也。岂可负哉!夫忠臣不避死而庶几,孝子不勤劳而见危,人臣各守其职而已矣。君其勿复言,将令斯得罪。”高曰:“盖闻圣人迁徙无常,就变而从时,见末而知本,观指而睹归。物固有之,安得常法哉!方今天下之权命悬于胡亥,高能得志焉。且夫从外制中谓之惑,从下制上谓之贼。故秋霜降者草花落,水摇动者万物作,此必然之效也。君何见之晚?”斯曰:“吾闻晋易太子,三世不安;齐桓兄弟争位,身死为戮;纣杀亲戚,不听谏者,国为丘墟,遂危社稷:三者逆天,宗庙不血食。斯其犹人哉,安足为谋!”高曰:“上下合同,可以长久;中外若一,事无表里。君听臣之计,即长有封侯,世世称孤,必有乔松之寿,孔、墨之智。今释此而不从,祸及子孙,足以为寒心。善者因祸为福,君何处焉?”斯乃仰天而叹,垂泪太息曰:“嗟乎!独遭乱世,既以不能死,安讬命哉!”于是斯乃听高。高乃报胡亥曰:“臣请奉太子之明命以报丞相,丞相斯敢不奉令!”
\end{yuanwen}

\begin{yuanwen}
于是乃相与谋,诈为受始皇诏丞相,立子胡亥为太子。更为书赐长子扶苏曰:“朕巡天下,祷祠名山诸神以延寿命。今扶苏与将军蒙恬将师数十万以屯边,十有馀年矣,不能进而前,士卒多秏,无尺寸之功,乃反数上书直言诽谤我所为,以不得罢归为太子,日夜怨望。扶苏为人子不孝,其赐剑以自裁!将军恬与扶苏居外,不匡正,宜知其谋。为人臣不忠,其赐死,以兵属裨将王离。”封其书以皇帝玺,遣胡亥客奉书赐扶苏于上郡。
\end{yuanwen}

\begin{yuanwen}
使者至,发书,扶苏泣,入内舍,欲自杀。蒙恬止扶苏曰:“陛下居外,未立太子,使臣将三十万众守边,公子为监,此天下重任也。今一使者来,即自杀,安知其非诈?请复请,复请而后死,未暮也。”使者数趣之。扶苏为人仁,谓蒙恬曰:“父而赐子死,尚安复请!”即自杀。蒙恬不肯死,使者即以属吏,系于阳周。
\end{yuanwen}

\begin{yuanwen}
使者还报,胡亥、斯、高大喜。至咸阳,发丧,太子立为二世皇帝。以赵高为郎中令,常侍中用事。
\end{yuanwen}

\begin{yuanwen}
二世燕居,乃召高与谋事,谓曰:“夫人生居世间也,譬犹骋六骥过决隙也。吾既已临天下矣,欲悉耳目之所好,穷心志之所乐,以安宗庙而乐万姓,长有天下,终吾年寿,其道可乎?”高曰:“此贤主之所能行也,而昬乱主之所禁也。臣请言之,不敢避斧钺之诛,原陛下少留意焉。夫沙丘之谋,诸公子及大臣皆疑焉,而诸公子尽帝兄,大臣又先帝之所置也。今陛下初立,此其属意怏怏皆不服,恐为变。且蒙恬已死,蒙毅将兵居外,臣战战栗栗,唯恐不终。且陛下安得为此乐乎?”二世曰:“为之柰何?”赵高曰:“严法而刻刑,令有罪者相坐诛,至收族,灭大臣而远骨肉;贫者富之,贱者贵之。尽除去先帝之故臣,更置陛下之所亲信者近之。此则阴德归陛下,害除而奸谋塞,群臣莫不被润泽,蒙厚德,陛下则高枕肆志宠乐矣。计莫出于此。”二世然高之言,乃更为法律。于是群臣诸公子有罪,辄下高,令鞠治之。杀大臣蒙毅等,公子十二人僇死咸阳市,十公主矺死于杜,财物入于县官,相连坐者不可胜数。
\end{yuanwen}

\begin{yuanwen}
公子高欲奔,恐收族,乃上书曰:“先帝无恙时,臣入则赐食,出则乘舆。御府之衣,臣得赐之;中厩之宝马,臣得赐之。臣当从死而不能,为人子不孝,为人臣不忠。不忠者无名以立于世,臣请从死,原葬郦山之足。唯上幸哀怜之。”书上,胡亥大说,召赵高而示之,曰:“此可谓急乎?”赵高曰:“人臣当忧死而不暇,何变之得谋!”胡亥可其书,赐钱十万以葬。
\end{yuanwen}

\begin{yuanwen}
法令诛罚日益刻深,群臣人人自危,欲畔者众。又作阿房之宫,治直、驰道,赋敛愈重,戍徭无已。于是楚戍卒陈胜、吴广等乃作乱,起于山东,杰俊相立,自置为侯王,叛秦,兵至鸿门而卻。李斯数欲请间谏,二世不许。而二世责问李斯曰:“吾有私议而有所闻于韩子也,曰‘尧之有天下也,堂高三尺,采椽不斫,茅茨不翦,虽逆旅之宿不勤于此矣。冬日鹿裘,夏日葛衣,粢粝之食,藜藿之羹,饭土匭,啜土鉶,虽监门之养不觳于此矣。禹凿龙门,通大夏,疏九河,曲九防,决渟水致之海,而股无胈,胫无毛,手足胼胝,面目黎黑,遂以死于外,葬于会稽,臣虏之劳不烈于此矣’。然则夫所贵于有天下者,岂欲苦形劳神,身处逆旅之宿,口食监门之养,手持臣虏之作哉?此不肖人之所勉也,非贤者之所务也。彼贤人之有天下也,专用天下適己而已矣,此所贵于有天下也。夫所谓贤人者,必能安天下而治万民,今身且不能利,将恶能治天下哉!故吾原赐志广欲,长享天下而无害,为之柰何?”李斯子由为三川守,群盗吴广等西略地,过去弗能禁。章邯以破逐广等兵,使者覆案三川相属,诮让斯居三公位,如何令盗如此。李斯恐惧,重爵禄,不知所出,乃阿二世意,欲求容,以书对曰:
\end{yuanwen}

\begin{yuanwen}
夫贤主者,必且能全道而行督责之术者也。督责之,则臣不敢不竭能以徇其主矣。此臣主之分定,上下之义明,则天下贤不肖莫敢不尽力竭任以徇其君矣。是故主独制于天下而无所制也。能穷乐之极矣,贤明之主也,可不察焉!
\end{yuanwen}

\begin{yuanwen}
故申子曰“有天下而不恣睢,命之曰以天下为桎梏”者,无他焉,不能督责,而顾以其身劳于天下之民,若尧、禹然,故谓之“桎梏”也。夫不能修申、韩之明术,行督责之道,专以天下自適也,而徒务苦形劳神,以身徇百姓,则是黔首之役,非畜天下者也,何足贵哉!夫以人徇己,则己贵而人贱;以己徇人,则己贱而人贵。故徇人者贱,而人所徇者贵,自古及今,未有不然者也。凡古之所为尊贤者,为其贵也;而所为恶不肖者,为其贱也。而尧、禹以身徇天下者也,因随而尊之,则亦失所为尊贤之心矣,夫可谓大缪矣。谓之为“桎梏”,不亦宜乎?不能督责之过也。
\end{yuanwen}

\begin{yuanwen}
故韩子曰:“慈母有败子而严家无格虏”者,何也?则能罚之加焉必也。故商君之法,刑弃灰于道者。夫弃灰,薄罪也,而被刑,重罚也。彼唯明主为能深督轻罪。夫罪轻且督深,而况有重罪乎?故民不敢犯也。是故韩子曰“布帛寻常,庸人不释,铄金百溢,盗跖不搏”者,非庸人之心重,寻常之利深,而盗跖之欲浅也;又不以盗跖之行,为轻百镒之重也。搏必随手刑,则盗跖不搏百镒;而罚不必行也,则庸人不释寻常。是故城高五丈,而楼季不轻犯也;泰山之高百仞,而跛牧其上。夫楼季也而难五丈之限,岂跛也而易百仞之高哉?峭堑之势异也。明主圣王之所以能久处尊位,长执重势,而独擅天下之利者,非有异道也,能独断而审督责,必深罚,故天下不敢犯也。今不务所以不犯,而事慈母之所以败子也,则亦不察于圣人之论矣。夫不能行圣人之术,则舍为天下役何事哉?可不哀邪!
\end{yuanwen}

\begin{yuanwen}
且夫俭节仁义之人立于朝,则荒肆之乐辍矣;谏说论理之臣间于侧,则流漫之志诎矣;烈士死节之行显于世,则淫康之虞废矣。故明主能外此三者,而独操主术以制听从之臣,而修其明法,故身尊而势重也。凡贤主者,必将能拂世磨俗,而废其所恶,立其所欲,故生则有尊重之势,死则有贤明之谥也。是以明君独断,故权不在臣也。然后能灭仁义之涂,掩驰说之口,困烈士之行,塞聪揜明,内独视听,故外不可倾以仁义烈士之行,而内不可夺以谏说忿争之辩。故能荦然独行恣睢之心而莫之敢逆。若此然后可谓能明申、韩之术,而脩商君之法。法脩术明而天下乱者,未之闻也。故曰“王道约而易操”也。唯明主为能行之。若此则谓督责之诚,则臣无邪,臣无邪则天下安,天下安则主严尊,主严尊则督责必,督责必则所求得,所求得则国家富,国家富则君乐丰。故督责之术设,则所欲无不得矣。群臣百姓救过不给,何变之敢图?若此则帝道备,而可谓能明君臣之术矣。虽申、韩复生,不能加也。
\end{yuanwen}

\begin{yuanwen}
书奏,二世悦。于是行督责益严,税民深者为明吏。二世曰:“若此则可谓能督责矣。”刑者相半于道,而死人日成积于市。杀人众者为忠臣。二世曰:“若此则可谓能督责矣。”
\end{yuanwen}

\begin{yuanwen}
初,赵高为郎中令,所杀及报私怨众多,恐大臣入朝奏事毁恶之,乃说二世曰:“天子所以贵者,但以闻声,群臣莫得见其面,故号曰‘朕’。且陛下富于春秋,未必尽通诸事,今坐朝廷,谴举有不当者,则见短于大臣,非所以示神明于天下也。且陛下深拱禁中,与臣及侍中习法者待事,事来有以揆之。如此则大臣不敢奏疑事,天下称圣主矣。”二世用其计,乃不坐朝廷见大臣,居禁中。赵高常侍中用事,事皆决于赵高。
\end{yuanwen}

\begin{yuanwen}
高闻李斯以为言,乃见丞相曰:“关东群盗多,今上急益发繇治阿房宫,聚狗马无用之物。臣欲谏,为位贱。此真君侯之事,君何不谏?”李斯曰:“固也,吾欲言之久矣。今时上不坐朝廷,上居深宫,吾有所言者,不可传也,欲见无间。”赵高谓曰:“君诚能谏,请为君候上间语君。”于是赵高待二世方燕乐,妇女居前,使人告丞相:“上方间,可奏事。”丞相至宫门上谒,如此者三。二世怒曰:“吾常多间日,丞相不来。吾方燕私,丞相辄来请事。丞相岂少我哉?且固我哉?”赵高因曰:“如此殆矣!夫沙丘之谋,丞相与焉。今陛下已立为帝,而丞相贵不益,此其意亦望裂地而王矣。且陛下不问臣,臣不敢言。丞相长男李由为三川守,楚盗陈胜等皆丞相傍县之子,以故楚盗公行,过三川,城守不肯击。高闻其文书相往来,未得其审,故未敢以闻。且丞相居外,权重于陛下。”二世以为然。欲案丞相,恐其不审,乃使人案验三川守与盗通状。李斯闻之。
\end{yuanwen}

\begin{yuanwen}
是时二世在甘泉,方作觳抵优俳之观。李斯不得见,因上书言赵高之短曰:“臣闻之,臣疑其君,无不危国;妾疑其夫,无不危家。今有大臣于陛下擅利擅害,与陛下无异,此甚不便。昔者司城子罕相宋,身行刑罚,以威行之,期年遂劫其君。田常为简公臣,爵列无敌于国,私家之富与公家均,布惠施德,下得百姓,上得群臣,阴取齐国,杀宰予于庭,即弑简公于朝,遂有齐国。此天下所明知也。今高有邪佚之志,危反之行,如子罕相宋也;私家之富,若田氏之于齐也。兼行田常、子罕之逆道而劫陛下之威信,其志若韩为韩安相也。陛下不图,臣恐其为变也。”二世曰:“何哉?夫高,故宦人也,然不为安肆志,不以危易心,絜行脩善,自使至此,以忠得进,以信守位,朕实贤之,而君疑之,何也?且朕少失先人,无所识知,不习治民,而君又老,恐与天下绝矣。朕非属赵君,当谁任哉?且赵君为人精廉彊力,下知人情,上能適朕,君其勿疑。”李斯曰:“不然。夫高,故贱人也,无识于理,贪欲无厌,求利不止,列势次主,求欲无穷,臣故曰殆。”二世已前信赵高,恐李斯杀之,乃私告赵高。高曰:“丞相所患者独高,高已死,丞相即欲为田常所为。”于是二世曰:“其以李斯属郎中令!”
\end{yuanwen}

\begin{yuanwen}
赵高案治李斯。李斯拘执束缚,居囹圄中,仰天而叹曰:“嗟乎,悲夫!不道之君,何可为计哉!昔者桀杀关龙逢,纣杀王子比干,吴王夫差杀伍子胥。此三臣者,岂不忠哉,然而不免于死,身死而所忠者非也。今吾智不及三子,而二世之无道过于桀、纣、夫差,吾以忠死,宜矣。且二世之治岂不乱哉!日者夷其兄弟而自立也,杀忠臣而贵贱人,作为阿房之宫,赋敛天下。吾非不谏也,而不吾听也。凡古圣王,饮食有节,车器有数,宫室有度,出令造事,加费而无益于民利者禁,故能长久治安。今行逆于昆弟,不顾其咎;侵杀忠臣,不思其殃;大为宫室,厚赋天下,不爱其费:三者已行,天下不听。今反者已有天下之半矣,而心尚未寤也,而以赵高为佐,吾必见寇至咸阳,麋鹿游于朝也。”
\end{yuanwen}

\begin{yuanwen}
于是二世乃使高案丞相狱,治罪,责斯与子由谋反状,皆收捕宗族宾客。赵高治斯,榜掠千馀,不胜痛,自诬服。斯所以不死者,自负其辩,有功,实无反心,幸得上书自陈,幸二世之寤而赦之。李斯乃从狱中上书曰:“臣为丞相治民,三十馀年矣。逮秦地之陕隘。先王之时秦地不过千里,兵数十万。臣尽薄材,谨奉法令,阴行谋臣,资之金玉,使游说诸侯,阴脩甲兵,饰政教,官斗士,尊功臣,盛其爵禄,故终以胁韩弱魏,破燕、赵,夷齐、楚,卒兼六国,虏其王,立秦为天子。罪一矣。地不广,又北逐胡、貉,南定百越,以见秦之彊。罪二矣。尊大臣,盛其爵位,以固其亲。罪三矣。立社稷,脩宗庙,以明主之贤。罪四矣。更剋画,平斗斛度量文章,布之天下,以树秦之名。罪五矣。治驰道,兴游观,以见主之得意。罪六矣。缓刑罚,薄赋敛,以遂主得众之心,万民戴主,死而不忘。罪七矣。若斯之为臣者,罪足以死固久矣。上幸尽其能力,乃得至今,原陛下察之!”书上,赵高使吏弃去不奏,曰:“囚安得上书!”
\end{yuanwen}

\begin{yuanwen}
赵高使其客十馀辈诈为御史、谒者、侍中,更往覆讯斯。斯更以其实对,辄使人复榜之。后二世使人验斯,斯以为如前,终不敢更言,辞服。奏当上,二世喜曰:“微赵君,几为丞相所卖。”及二世所使案三川之守至,则项梁已击杀之。使者来,会丞相下吏,赵高皆妄为反辞。
\end{yuanwen}

\begin{yuanwen}
二世二年七月,具斯五刑,论腰斩咸阳市。斯出狱,与其中子俱执,顾谓其中子曰:“吾欲与若复牵黄犬俱出上蔡东门逐狡兔,岂可得乎!”遂父子相哭,而夷三族。
\end{yuanwen}

\begin{yuanwen}
李斯已死,二世拜赵高为中丞相,事无大小辄决于高。高自知权重,乃献鹿,谓之马。二世问左右:“此乃鹿也?”左右皆曰“马也”。二世惊,自以为惑,乃召太卜,令卦之,太卜曰:“陛下春秋郊祀,奉宗庙鬼神,斋戒不明,故至于此。可依盛德而明斋戒。”于是乃入上林斋戒。日游弋猎,有行人入上林中,二世自射杀之。赵高教其女婿咸阳令阎乐劾不知何人贼杀人移上林。高乃谏二世曰:“天子无故贼杀不辜人,此上帝之禁也,鬼神不享,天且降殃,当远避宫以禳之。”二世乃出居望夷之宫。
\end{yuanwen}

\begin{yuanwen}
留三日,赵高诈诏卫士,令士皆素服持兵内乡,入告二世曰:“山东群盗兵大至!”二世上观而见之,恐惧,高既因劫令自杀。引玺而佩之,左右百官莫从;上殿,殿欲坏者三。高自知天弗与,群臣弗许,乃召始皇弟,授之玺。
\end{yuanwen}

\begin{yuanwen}
子婴既位,患之,乃称疾不听事,与宦者韩谈及其子谋杀高。高上谒,请病,因召入,令韩谈刺杀之,夷其三族。
\end{yuanwen}

\begin{yuanwen}
子婴立三月,沛公兵从武关入,至咸阳,群臣百官皆畔,不適。子婴与妻子自系其颈以组,降轵道旁。沛公因以属吏。项王至而斩之。遂以亡天下。
\end{yuanwen}

\begin{yuanwen}
太史公曰:李斯以闾阎历诸侯,入事秦,因以瑕衅,以辅始皇,卒成帝业,斯为三公,可谓尊用矣。斯知六之归,不务明政以补主上之缺,持爵禄之重,阿顺苟合,严威酷刑,听高邪说,废適立庶。诸侯已畔,斯乃欲谏争,不亦末乎!人皆以斯极忠而被五刑死,察其本,乃与俗议之异。不然,斯之功且与周、召列矣。
\end{yuanwen}

\begin{yuanwen}
鼠在所居,人固择地。斯效智力,功立名遂。置酒咸阳,人臣极位。一夫诳惑,变易神器。国丧身诛,本同末异。
\end{yuanwen}

\part{卷八十八}
\chapter{蒙恬列传第二十八}

苏轼:“始皇制天下轻重之势,使内外相形,以禁奸备乱者,可谓密矣。蒙恬将三十万人,威震北方,扶苏监其军;而蒙毅侍帷幄,为谋臣,虽有大奸贼,敢睥睨其间哉?不幸道病,祷祠山川。尚有人也,而遣蒙毅,故高斯得成其谋。始皇之遣毅,毅见始皇病,太子未立,而去左右,皆不可以言智。虽然,天之亡人国,其祸败必出于智之所不及。圣人为天下,不恃智以防乱,恃吾无致乱之道耳。始皇致乱之道,在用赵高。夫阉尹之祸,如毒药猛兽,未有不裂肝碎首者也。自有书契以来,惟东汉吕强,后唐张承业,此二人号称善良,岂可望一二于千万,以取必亡之祸哉?然世主皆甘心而不悔。如汉桓、灵,唐肃、代,犹不足深怪,始皇汉宣皆英主,亦沉于赵高恭显之祸,彼自以为聪明人杰也,奴仆熏腐之余何能为?及其亡国乱朝,乃与庸主不异。吾故表而出之,以戒后世人主如始皇汉宣者。”

\begin{yuanwen}
蒙恬者,其先齐人也。恬大父\footnote{祖父。}蒙骜\footnote{ào},自齐事秦昭王,官至上卿。秦庄襄王元年,蒙骜为秦将,伐韩,取成皋、荥阳,作置三川郡。二年,蒙骜攻赵,取三十七城。始皇三年,蒙骜攻韩,取十三城。五年,蒙骜攻魏,取二十城,作置东郡。始皇七年,蒙骜卒。骜子曰武,武子曰恬。恬尝书狱\footnote{在审理案件时记录案件审理情况。}典文学\footnote{管理有关文件和狱讼档案工作。文学,法律、刑狱有关的文献。}。始皇二十三年,蒙武为秦裨将军,与王翦攻楚,大破之,杀项燕。二十四年,蒙武攻楚,虏楚王。蒙恬弟毅。
\end{yuanwen}

\begin{yuanwen}
始皇二十六年,蒙恬因家世得为秦将,攻齐,大破之,拜为内史。秦已并天下,乃使蒙恬将三十万众北逐戎狄,收河南。筑长城,因地形,用制险塞,起临洮,至辽东,延袤万馀里。于是渡河,据阳山,逶蛇而北。暴师于外十馀年,居上郡。是时蒙恬威振匈奴。始皇甚尊宠蒙氏,信任贤之。而亲近蒙毅,位至上卿,出则参乘,入则御前。恬任外事而毅常为内谋,名为忠信,故虽诸将相莫敢与之争焉。

赵高者,诸赵疏远属也。赵高昆弟数人,皆生隐宫,其母被刑僇,世世卑贱。秦王闻高彊力,通于狱法,举以为中车府令。高既私事公子胡亥,喻之决狱。高有大罪,秦王令蒙毅法治之。毅不敢阿法,当高罪死,除其宦籍。帝以高之敦于事也,赦之,复其官爵。

始皇欲游天下,道九原,直抵甘泉,乃使蒙恬通道,自九原抵甘泉,巉山堙谷,千八百里。道未就。

始皇三十七年冬,行出游会稽,并海上,北走琅邪。道病,使蒙毅还祷山川,未反。

始皇至沙丘崩,祕之,群臣莫知。是时丞相李斯、公子胡亥、中车府令赵高常从。高雅得幸于胡亥,欲立之,又怨蒙毅法治之而不为己也。因有贼心,乃与丞相李斯、公子胡亥阴谋,立胡亥为太子。太子已立,遣使者以罪赐公子扶苏、蒙恬死。扶苏已死,蒙恬疑而复请之。使者以蒙恬属吏,更置。胡亥以李斯舍人为护军。使者还报,胡亥已闻扶苏死,即欲释蒙恬。赵高恐蒙氏复贵而用事,怨之。

毅还至,赵高因为胡亥忠计,欲以灭蒙氏,乃言曰:“臣闻先帝欲举贤立太子久矣,而毅谏曰‘不可’。若知贤而俞弗立,则是不忠而惑主也。以臣愚意,不若诛之。,”胡亥听而系蒙毅于代。前已囚蒙恬于阳周。丧至咸阳,已葬,太子立为二世皇帝,而赵高亲近,日夜毁恶蒙氏,求其罪过,举劾之。

子婴进谏曰:“臣闻故赵王迁杀其良臣李牧而用颜聚,燕王喜阴用荆轲之谋而倍秦之约,齐王建杀其故世忠臣而用后胜之议。此三君者,皆各以变古者失其国而殃及其身。今蒙氏,秦之大臣谋士也,而主欲一旦弃去之,臣窃以为不可。臣闻轻虑者不可以治国,独智者不可以存君。诛杀忠臣而立无节行之人,是内使群臣不相信而外使斗士之意离也,臣窃以为不可。”

胡亥不听。而遣御史曲宫乘传之代,令蒙毅曰:“先主欲立太子而卿难之。今丞相以卿为不忠,罪及其宗。朕不忍,乃赐卿死,亦甚幸矣。卿其图之!”毅对曰:“以臣不能得先主之意,则臣少宦,顺幸没世。可谓知意矣。以臣不知太子之能,则太子独从,周旋天下,去诸公子绝远,臣无所疑矣。夫先主之举用太子,数年之积也,臣乃何言之敢谏,何虑之敢谋!非敢饰辞以避死也,为羞累先主之名,原大夫为虑焉,使臣得死情实。且夫顺成全者,道之所贵也;刑杀者,道之所卒也。昔者秦穆公杀三良而死,罪百里奚而非其罪也,故立号曰‘缪’。昭襄王杀武安君白起。楚平王杀伍奢。吴王夫差杀伍子胥。此四君者,皆为大失,而天下非之,以其君为不明,以是籍于诸侯。故曰‘用道治者不杀无罪,而罚不加于无辜’。唯大夫留心!”使者知胡亥之意,不听蒙毅之言,遂杀之。

二世又遣使者之阳周,令蒙恬曰:“君之过多矣,而卿弟毅有大罪,法及内史。”恬曰:“自吾先人,及至子孙,积功信于秦三世矣。今臣将兵三十馀万,身虽囚系,其势足以倍畔,然自知必死而守义者,不敢辱先人之教,以不忘先主也。昔周成王初立,未离襁褓,周公旦负王以朝,卒定天下。及成王有病甚殆,公旦自揃其爪以沈于河,曰:‘王未有识,是旦执事。有罪殃,旦受其不祥。’乃书而藏之记府,可谓信矣。及王能治国,有贼臣言:‘周公旦欲为乱久矣,王若不备,必有大事。’王乃大怒,周公旦走而奔于楚。成王观于记府,得周公旦沈书,乃流涕曰:‘孰谓周公旦欲为乱乎!’杀言之者而反周公旦。故周书曰‘必参而伍之’。今恬之宗,世无二心,而事卒如此,是必孽臣逆乱,内陵之道也。夫成王失而复振则卒昌;桀杀关龙逢,纣杀王子比干而不悔,身死则国亡。臣故曰过可振而谏可觉也。察于参伍,上圣之法也。凡臣之言,非以求免于咎也,将以谏而死,原陛下为万民思从道也。”使者曰:“臣受诏行法于将军,不敢以将军言闻于上也。”蒙恬喟然太息曰:“我何罪于天,无过而死乎?”良久,徐曰:“恬罪固当死矣。起临洮属之辽东,城巉万馀里,此其中不能无绝地脉哉?此乃恬之罪也。”乃吞药自杀。

太史公曰:吾適北边,自直道归,行观蒙恬所为秦筑长城亭障,堑山堙谷,通直道,固轻百姓力矣。夫秦之初灭诸侯,天下之心未定,痍伤者未瘳,而恬为名将,不以此时彊谏,振百姓之急,养老存孤,务修众庶之和,而阿意兴功,此其兄弟遇诛,不亦宜乎!何乃罪地脉哉?

蒙氏秦将,内史忠贤。长城首筑,万里安边。赵高矫制,扶苏死焉。绝地何罪?劳人是。呼天欲诉,三代良然。
\end{yuanwen}

\part{卷八十九}
\chapter{张耳陈馀列传第二十九}

\begin{yuanwen}
张耳者,大梁人也。其少时,及魏公子毋忌为客。张耳尝亡命游外黄。外黄富人女甚美,嫁庸奴,亡其夫,去抵父客。父客素知张耳,乃谓女曰:“必欲求贤夫,从张耳。”女听,乃卒为请决,嫁之张耳。张耳是时脱身游,女家厚奉给张耳,张耳以故致千里客。乃宦魏为外黄令。名由此益贤。陈馀者,亦大梁人也,好儒术,数游赵苦陉。富人公乘氏以其女妻之,亦知陈馀非庸人也。馀年少,父事张耳,两人相与为刎颈交。

秦之灭大梁也,张耳家外黄。高祖为布衣时,尝数从张耳游,客数月。秦灭魏数岁,已闻此两人魏之名士也,购求有得张耳千金,陈馀五百金。张耳、陈馀乃变名姓,俱之陈,为里监门以自食。两人相对。里吏尝有过笞陈馀,陈馀欲起,张耳蹑之,使受笞。吏去,张耳乃引陈馀之桑下而数之曰:“始吾与公言何如?今见小辱而欲死一吏乎?”陈馀然之。秦诏书购求两人,两人亦反用门者以令里中。

陈涉起蕲,至入陈,兵数万。张耳、陈馀上谒陈涉。涉及左右生平数闻张耳、陈馀贤,未尝见,见即大喜。

陈中豪杰父老乃说陈涉曰:“将军身被坚执锐,率士卒以诛暴秦,复立楚社稷,存亡继绝,功德宜为王。且夫监临天下诸将,不为王不可,原将军立为楚王也。”陈涉问此两人,两人对曰:“夫秦为无道,破人国家,灭人社稷,绝人后世,罢百姓之力,尽百姓之财。将军瞋目张胆,出万死不顾一生之计,为天下除残也。今始至陈而王之,示天下私。原将军毋王,急引兵而西,遣人立六国后,自为树党,为秦益敌也。敌多则力分,与众则兵彊。如此野无交兵,县无守城,诛暴秦,据咸阳以令诸侯。诸侯亡而得立,以德服之,如此则帝业成矣。今独王陈,恐天下解也。”陈涉不听,遂立为王。

陈馀乃复说陈王曰:“大王举梁、楚而西,务在入关,未及收河北也。臣尝游赵,知其豪桀及地形,原请奇兵北略赵地。”于是陈王以故所善陈人武臣为将军,邵骚为护军,以张耳、陈馀为左右校尉,予卒三千人,北略赵地。

武臣等从白马渡河,至诸县,说其豪桀曰:“秦为乱政虐刑以残贼天下,数十年矣。北有长城之役,南有五岭之戍,外内骚动,百姓罢敝,头会箕敛,以供军费,财匮力尽,民不聊生。重之以苛法峻刑,使天下父子不相安。陈王奋臂为天下倡始,王楚之地,方二千里,莫不响应,家自为怒,人自为斗,各报其怨而攻其雠,县杀其令丞,郡杀其守尉。今已张大楚,王陈,使吴广、周文将卒百万西击秦。于此时而不成封侯之业者,非人豪也。诸君试相与计之!夫天下同心而苦秦久矣。因天下之力而攻无道之君,报父兄之怨而成割地有土之业,此士之一时也。”豪桀皆然其言。乃行收兵,得数万人,号武臣为武信君。下赵十城,馀皆城守,莫肯下。

乃引兵东北击范阳。范阳人蒯通说范阳令曰:“窃闻公之将死,故吊。虽然,贺公得通而生。”范阳令曰:“何以吊之?”对曰:“秦法重,足下为范阳令十年矣,杀人之父,孤人之子,断人之足,黥人之首,不可胜数。然而慈父孝子莫敢倳刃公之腹中者,畏秦法耳。今天下大乱,秦法不施,然则慈父孝子且倳刃公之腹中以成其名,此臣之所以吊公也。今诸侯畔秦矣,武信君兵且至,而君坚守范阳,少年皆争杀君,下武信君。君急遣臣见武信君,可转祸为福,在今矣。”

范阳令乃使蒯通见武信君曰:“足下必将战胜然后略地,攻得然后下城,臣窃以为过矣。诚听臣之计,可不攻而降城,不战而略地,传檄而千里定,可乎?”武信君曰:“何谓也?”蒯通曰:“今范阳令宜整顿其士卒以守战者也,怯而畏死,贪而重富贵,故欲先天下降,畏君以为秦所置吏,诛杀如前十城也。然今范阳少年亦方杀其令,自以城距君。君何不赍臣侯印,拜范阳令,范阳令则以城下君,少年亦不敢杀其令。令范阳令乘硃轮华毂,使驱驰燕、赵郊。燕、赵郊见之,皆曰此范阳令,先下者也,即喜矣,燕、赵城可毋战而降也。此臣之所谓传檄而千里定者也。”武信君从其计,因使蒯通赐范阳令侯印。赵地闻之,不战以城下者三十馀城。

至邯郸,张耳、陈馀闻周章军入关,至戏卻;又闻诸将为陈王徇地,多以谗毁得罪诛,怨陈王不用其筴不以为将而以为校尉。乃说武臣曰:“陈王起蕲,至陈而王,非必立六国后。将军今以三千人下赵数十城,独介居河北,不王无以填之。且陈王听谗,还报,恐不脱于祸。又不如立其兄弟;不,即立赵后。将军毋失时,时间不容息。”武臣乃听之,遂立为赵王。以陈馀为大将军,张耳为右丞相,邵骚为左丞相。

使人报陈王,陈王大怒,欲尽族武臣等家,而发兵击赵。陈王相国房君谏曰:“秦未亡而诛武臣等家,此又生一秦也。不如因而贺之,使急引兵西击秦。”陈王然之,从其计,徙系武臣等家宫中,封张耳子敖为成都君。

陈王使使者贺赵,令趣发兵西入关。张耳、陈馀说武臣曰:“王王赵,非楚意,特以计贺王。楚已灭秦,必加兵于赵。原王毋西兵,北徇燕、代,南收河内以自广。赵南据大河,北有燕、代,楚虽胜秦,必不敢制赵。”赵王以为然,因不西兵,而使韩广略燕,李良略常山,张黡略上党。

韩广至燕,燕人因立广为燕王。赵王乃与张耳、陈馀北略地燕界。赵王间出,为燕军所得。燕将囚之,欲与分赵地半,乃归王。使者往,燕辄杀之以求地。张耳、陈馀患之。有厮养卒谢其舍中曰:“吾为公说燕,与赵王载归。”舍中皆笑曰:“使者往十馀辈,辄死,若何以能得王?”乃走燕壁。燕将见之,问燕将曰:“知臣何欲?”燕将曰:“若欲得赵王耳。”曰:“君知张耳、陈馀何如人也?”燕将曰:“贤人也。”曰:“知其志何欲?”曰:“欲得其王耳。”赵养卒乃笑曰:“君未知此两人所欲也。夫武臣、张耳、陈馀杖马箠下赵数十城,此亦各欲南面而王,岂欲为卿相终己邪?夫臣与主岂可同日而道哉,顾其势初定,未敢参分而王,且以少长先立武臣为王,以持赵心。今赵地已服,此两人亦欲分赵而王,时未可耳。今君乃囚赵王。此两人名为求赵王,实欲燕杀之,此两人分赵自立。夫以一赵尚易燕,况以两贤王左提右挈,而责杀王之罪,灭燕易矣。”燕将以为然,乃归赵王,养卒为御而归。

李良已定常山,还报,赵王复使良略太原。至石邑,秦兵塞井陉,未能前。秦将诈称二世使人遗李良书,不封,曰:“良尝事我得显幸。良诚能反赵为秦,赦良罪,贵良。”良得书,疑不信。乃还之邯郸,益请兵。未至,道逢赵王姊出饮,从百馀骑。李良望见,以为王,伏谒道旁。王姊醉,不知其将,使骑谢李良。李良素贵,起,惭其从官。从官有一人曰:“天下畔秦,能者先立。且赵王素出将军下,今女兒乃不为将军下车,请追杀之。”李良已得秦书,固欲反赵,未决,因此怒,遣人追杀王姊道中,乃遂将其兵袭邯郸。邯郸不知,竟杀武臣、邵骚。赵人多为张耳、陈馀耳目者,以故得脱出。收其兵,得数万人。客有说张耳曰:“两君羁旅,而欲附赵,难;独立赵后,扶以义,可就功。”乃求得赵歇,立为赵王,居信都。李良进兵击陈馀,陈馀败李良,李良走归章邯。

章邯引兵至邯郸,皆徙其民河内,夷其城郭。张耳与赵王歇走入钜鹿城,王离围之。陈馀北收常山兵,得数万人,军钜鹿北。章邯军钜鹿南棘原,筑甬道属河,饷王离。王离兵食多,急攻钜鹿。钜鹿城中食尽兵少,张耳数使人召前陈馀,陈馀自度兵少,不敌秦,不敢前。数月,张耳大怒,怨陈馀,使张黡、陈泽往让陈馀曰:“始吾与公为刎颈交,今王与耳旦暮且死,而公拥兵数万,不肯相救,安在其相为死!苟必信,胡不赴秦军俱死?且有十一二相全。”陈馀曰:“吾度前终不能救赵,徒尽亡军。且馀所以不俱死,欲为赵王、张君报秦。今必俱死,如以肉委饿虎,何益?”张黡、陈泽曰:“事已急,要以俱死立信,安知后虑!”陈馀曰:“吾死顾以为无益。必如公言。”乃使五千人令张黡、陈泽先尝秦军,至皆没。

当是时,燕、齐、楚闻赵急,皆来救。张敖亦北收代兵,得万馀人,来,皆壁馀旁,未敢击秦。项羽兵数绝章邯甬道,王离军乏食,项羽悉引兵渡河,遂破章邯。章邯引兵解,诸侯军乃敢击围钜鹿秦军,遂虏王离。涉间自杀。卒存钜鹿者,楚力也。

于是赵王歇、张耳乃得出钜鹿,谢诸侯。张耳与陈馀相见,责让陈馀以不肯救赵,及问张黡、陈泽所在。陈馀怒曰:“张黡、陈泽以必死责臣,臣使将五千人先尝秦军,皆没不出。”张耳不信,以为杀之,数问陈馀。陈馀怒曰:“不意君之望臣深也!岂以臣为重去将哉?”乃脱解印绶,推予张耳。张耳亦愕不受。陈馀起如厕。客有说张耳曰:“臣闻‘天与不取,反受其咎’。今陈将军与君印,君不受,反天不祥。急取之!”张耳乃佩其印,收其麾下。而陈馀还,亦望张耳不让,遂趋出。张耳遂收其兵。陈馀独与麾下所善数百人之河上泽中渔猎。由此陈馀、张耳遂有卻。

赵王歇复居信都。张耳从项羽诸侯入关。汉元年二月,项羽立诸侯王,张耳雅游,人多为之言,项羽亦素数闻张耳贤,乃分赵立张耳为常山王,治信都。信都更名襄国。

陈馀客多说项羽曰:“陈馀、张耳一体有功于赵。”项羽以陈馀不从入关,闻其在南皮,即以南皮旁三县以封之,而徙赵王歇王代。

张耳之国,陈馀愈益怒,曰:“张耳与馀功等也,今张耳王,馀独侯,此项羽不平。”及齐王田荣畔楚,陈馀乃使夏说说田荣曰:“项羽为天下宰不平,尽王诸将善地,徙故王王恶地,今赵王乃居代!原王假臣兵,请以南皮为扞蔽。”田荣欲树党于赵以反楚,乃遣兵从陈馀。陈馀因悉三县兵袭常山王张耳。张耳败走,念诸侯无可归者,曰:“汉王与我有旧故,而项羽又彊,立我,我欲之楚。”甘公曰:“汉王之入关,五星聚东井。东井者,秦分也。先至必霸。楚虽彊,后必属汉。”故耳走汉。汉王亦还定三秦,方围章邯废丘。张耳谒汉王,汉王厚遇之。

陈馀已败张耳,皆复收赵地,迎赵王于代,复为赵王。赵王德陈馀,立以为代王。陈馀为赵王弱,国初定,不之国,留傅赵王,而使夏说以相国守代。

汉二年,东击楚,使使告赵,欲与俱。陈馀曰:“汉杀张耳乃从。”于是汉王求人类张耳者斩之,持其头遗陈馀。陈馀乃遣兵助汉。汉之败于彭城西,陈馀亦复觉张耳不死,即背汉。

汉三年,韩信已定魏地,遣张耳与韩信击破赵井陉,斩陈馀泜水上,追杀赵王歇襄国。汉立张耳为赵王。汉五年,张耳薨,谥为景王。子敖嗣立为赵王。高祖长女鲁元公主为赵王敖后。

汉七年,高祖从平城过赵,赵王朝夕袒韝蔽,自上食,礼甚卑,有子婿礼。高祖箕踞詈,甚慢易之。赵相贯高、赵午等年六十馀,故张耳客也。生平为气,乃怒曰:“吾王孱王也!”说王曰:“夫天下豪桀并起,能者先立。今王事高祖甚恭,而高祖无礼,请为王杀之!”张敖齧其指出血,曰:“君何言之误!且先人亡国,赖高祖得复国,德流子孙,秋豪皆高祖力也。原君无复出口。”贯高、赵午等十馀人皆相谓曰:“乃吾等非也。吾王长者,不倍德。且吾等义不辱,今怨高祖辱我王,故欲杀之,何乃汙王为乎?令事成归王,事败独身坐耳。”

汉八年,上从东垣还,过赵,贯高等乃壁人柏人,要之置厕。上过欲宿,心动问曰:“县名为何?”曰:“柏人。”“柏人者,迫于人也!”不宿而去。

汉九年,贯高怨家知其谋,乃上变告之。于是上皆并逮捕赵王、贯高等。十馀人皆争自刭,贯高独怒骂曰:“谁令公为之?今王实无谋,而并捕王;公等皆死,谁白王不反者!”乃轞车胶致,与王诣长安。治张敖之罪。上乃诏赵群臣宾客有敢从王皆族。贯高与客孟舒等十馀人,皆自髡钳,为王家奴,从来。贯高至,对狱,曰:“独吾属为之,王实不知。”吏治榜笞数千,刺剟,身无可击者,终不复言。吕后数言张王以鲁元公主故,不宜有此。上怒曰:“使张敖据天下,岂少而女乎!”不听。廷尉以贯高事辞闻,上曰:“壮士!谁知者,以私问之。”中大夫泄公曰:“臣之邑子,素知之。此固赵国立名义不侵为然诺者也。”上使泄公持节问之箯舆前。仰视曰:“泄公邪?”泄公劳苦如生平驩,与语,问张王果有计谋不。高曰:“人情宁不各爱其父母妻子乎?今吾三族皆以论死,岂以王易吾亲哉!顾为王实不反,独吾等为之。”具道本指所以为者王不知状。于是泄公入,具以报,上乃赦赵王。

上贤贯高为人能立然诺,使泄公具告之,曰:“张王已出。”因赦贯高。贯高喜曰:“吾王审出乎?”泄公曰:“然。”泄公曰:“上多足下,故赦足下。”贯高曰:“所以不死一身无馀者,白张王不反也。今王已出,吾责已塞,死不恨矣。且人臣有篡杀之名,何面目复事上哉!纵上不杀我,我不愧于心乎?”乃仰绝肮,遂死。当此之时,名闻天下。

张敖已出,以尚鲁元公主故,封为宣平侯。于是上贤张王诸客,以钳奴从张王入关,无不为诸侯相、郡守者。及孝惠、高后、文帝、孝景时,张王客子孙皆得为二千石。

张敖,高后六年薨。子偃为鲁元王。以母吕后女故,吕后封为鲁元王。元王弱,兄弟少,乃封张敖他姬子二人:寿为乐昌侯,侈为信都侯。高后崩,诸吕无道,大臣诛之,而废鲁元王及乐昌侯、信诸侯。孝文帝即位,复封故鲁元王偃为南宫侯,续张氏。

太史公曰:张耳、陈馀,世传所称贤者;其宾客厮役,莫非天下俊桀,所居国无不取卿相者。然张耳、陈馀始居约时,相然信以死,岂顾问哉。及据国争权,卒相灭亡,何乡者相慕用之诚,后相倍之戾也!岂非以势利交哉?名誉虽高,宾客虽盛,所由殆与大伯、延陵季子异矣。

张耳、陈馀,天下豪俊。忘年羁旅,刎颈相信。耳围钜鹿,馀兵不进。张既望深,陈乃去印。势利倾夺,隙末成衅。
\end{yuanwen}

\chapter{魏豹彭越列传}

\begin{yuanwen}
魏豹者,故魏诸公子也。其兄魏咎,故魏时封为宁陵君。秦灭魏,迁咎为家人。陈胜之起王也,咎往从之。陈王使魏人周市徇魏地,魏地已下,欲相与立周市为魏王。周市曰:“天下昬乱,忠臣乃见。今天下共畔秦,其义必立魏王后乃可。”齐、赵使车各五十乘,立周市为魏王。市辞不受,迎魏咎于陈。五反,陈王乃遣立咎为魏王。

章邯已破陈王,乃进兵击魏王于临济。魏王乃使周市出请救于齐、楚。齐、楚遣项它、田巴将兵随市救魏。章邯遂击破杀周市等军,围临济。咎为其民约降。约定,咎自烧杀。

魏豹亡走楚。楚怀王予魏豹数千人,复徇魏地。项羽已破秦,降章邯。豹下魏二十馀城,立豹为魏王。豹引精兵从项羽入关。汉元年,项羽封诸侯,欲有梁地,乃徙魏王豹于河东,都平阳,为西魏王。

汉王还定三秦,渡临晋,魏王豹以国属焉,遂从击楚于彭城。汉败,还至荥阳,豹请归视亲病,至国,即绝河津畔汉。汉王闻魏豹反,方东忧楚,未及击,谓郦生曰:“缓颊往说魏豹,能下之,吾以万户封若。”郦生说豹。豹谢曰:“人生一世间,如白驹过隙耳。今汉王慢而侮人,骂詈诸侯群臣如骂奴耳,非有上下礼节也,吾不忍复见也。”于是汉王遣韩信击虏豹于河东,传诣荥阳,以豹国为郡。汉王令豹守荥阳。楚围之急,周苛遂杀魏豹。

彭越者,昌邑人也,字仲。常渔钜野泽中,为群盗。陈胜、项梁之起,少年或谓越曰:“诸豪桀相立畔秦,仲可以来,亦效之。”彭越曰:“两龙方斗,且待之。”

居岁馀,泽间少年相聚百馀人,往从彭越,曰:“请仲为长。”越谢曰:“臣不原与诸君。”少年彊请,乃许。与期旦日日出会,后期者斩。旦日日出,十馀人后,后者至日中。于是越谢曰:“臣老,诸君彊以为长。今期而多后,不可尽诛,诛最后者一人。”令校长斩之。皆笑曰:“何至是?请后不敢。”于是越乃引一人斩之,设坛祭,乃令徒属。徒属皆大惊,畏越,莫敢仰视。乃行略地,收诸侯散卒,得千馀人。

沛公之从砀北击昌邑,彭越助之。昌邑未下,沛公引兵西。彭越亦将其众居钜野中,收魏散卒。项籍入关,王诸侯,还归,彭越众万馀人毋所属。汉元年秋,齐王田荣畔项王,乃使人赐彭越将军印,使下济阴以击楚。楚命萧公角将兵击越,越大破楚军。汉王二年春,与魏王豹及诸侯东击楚,彭越将其兵三万馀人归汉于外黄。汉王曰:“彭将军收魏地得十馀城,欲急立魏后。今西魏王豹亦魏王咎从弟也,真魏后。”乃拜彭越为魏相国,擅将其兵,略定梁地。

汉王之败彭城解而西也,彭越皆复亡其所下城,独将其兵北居河上。汉王三年,彭越常往来为汉游兵,击楚,绝其后粮于梁地。汉四年冬,项王与汉王相距荥阳,彭越攻下睢阳、外黄十七城。项王闻之,乃使曹咎守成皋,自东收彭越所下城邑,皆复为楚。越将其兵北走穀城。汉五年秋,项王之南走阳夏,彭越复下昌邑旁二十馀城,得穀十馀万斛,以给汉王食。

汉王败,使使召彭越并力击楚。越曰:“魏地初定,尚畏楚,未可去。”汉王追楚,为项籍所败固陵。乃谓留侯曰:“诸侯兵不从,为之柰何?”留侯曰:“齐王信之立,非君王之意,信亦不自坚。彭越本定梁地,功多,始君王以魏豹故,拜彭越为魏相国。今豹死毋后,且越亦欲王,而君王不蚤定。与此两国约:即胜楚,睢阳以北至穀城,皆以王彭相国;从陈以东傅海,与齐王信。齐王信家在楚,此其意欲复得故邑。君王能出捐此地许二人,二人今可致;即不能,事未可知也。”于是汉王乃发使使彭越,如留侯策。使者至,彭越乃悉引兵会垓下,遂破楚。项籍已死。春,立彭越为梁王,都定陶。

六年,朝陈。九年,十年,皆来朝长安。

十年秋,陈豨反代地,高帝自往击,至邯郸,徵兵梁王。梁王称病,使将将兵诣邯郸。高帝怒,使人让梁王。梁王恐,欲自往谢。其将扈辄曰:“王始不往,见让而往,往则为禽矣。不如遂发兵反。”梁王不听,称病。梁王怒其太仆,欲斩之。太仆亡走汉,告梁王与扈辄谋反。于是上使使掩梁王,梁王不觉,捕梁王,囚之雒阳。有司治反形己具,请论如法。上赦以为庶人,传处蜀青衣。西至郑,逢吕后从长安来,欲之雒阳,道见彭王。彭王为吕后泣涕,自言无罪,原处故昌邑。吕后许诺,与俱东至雒阳。吕后白上曰:“彭王壮士,今徙之蜀,此自遗患,不如遂诛之。妾谨与俱来。”于是吕后乃令其舍人彭越复谋反。廷尉王恬开奏请族之。上乃可,遂夷越宗族,国除。

太史公曰:魏豹、彭越虽故贱,然已席卷千里,南面称孤,喋血乘胜日有闻矣。怀畔逆之意,及败,不死而虏囚,身被刑戮,何哉?中材已上且羞其行,况王者乎!彼无异故,智略绝人,独患无身耳。得摄尺寸之柄,其云蒸龙变,欲有所会其度,以故幽囚而不辞云。

魏咎兄弟,因时而王。豹后属楚,其国遂亡。仲起昌邑,归汉外黄。往来声援,再续军粮。徵兵不往,菹醢何伤。
\end{yuanwen}

\chapter{黥布列传}

\begin{yuanwen}
黥布者,六人也,姓英氏。秦时为布衣。少年,有客相之曰:“当刑而王。”及壮,坐法黥。布欣然笑曰;“人相我当刑而王,几是乎?”人有闻者,共俳笑之。布已论输丽山,丽山之徒数十万人,布皆与其徒长豪桀交通,乃率其曹偶,亡之江中为群盗。

陈胜之起也,布乃见番君,与其众叛秦,聚兵数千人。番君以其女妻之。章邯之灭陈胜,破吕臣军,布乃引兵北击秦左右校,破之清波,引兵而东。闻项梁定江东会稽,涉江而西。陈婴以项氏世为楚将,乃以兵属项梁,渡淮南,英布、蒲将军亦以兵属项梁。

项梁涉淮而西,击景驹、秦嘉等,布常冠军。项梁至薛,闻陈王定死,乃立楚怀王。项梁号为武信君,英布为当阳君。项梁败死定陶,怀王徙都彭城,诸将英布亦皆保聚彭城。当是时,秦急围赵,赵数使人请救。怀王使宋义为上将,范曾为末将,项籍为次将,英布、蒲将军皆为将军,悉属宋义,北救赵。及项籍杀宋义于河上,怀王因立籍为上将军,诸将皆属项籍。项籍使布先渡河击秦,布数有利,籍乃悉引兵涉河从之,遂破秦军,降章邯等。楚兵常胜,功冠诸侯。诸侯兵皆以服属楚者,以布数以少败众也。

项籍之引兵西至新安,又使布等夜击阬章邯秦卒二十馀万人。至关,不得入,又使布等先从间道破关下军,遂得入,至咸阳。布常为军锋。项王封诸将,立布为九江王,都六。

汉元年四月,诸侯皆罢戏下,各就国。项氏立怀王为义帝,徙都长沙,乃阴令九江王布等行击之。其八月,布使将击义帝,追杀之郴县。

汉二年,齐王田荣畔楚,项王往击齐,徵兵九江,九江王布称病不往,遣将将数千人行。汉之败楚彭城,布又称病不佐楚。项王由此怨布,数使使者诮让召布,布愈恐,不敢往。项王方北忧齐、赵,西患汉,所与者独九江王,又多布材,欲亲用之,以故未击。

汉三年,汉王击楚,大战彭城,不利,出梁地,至虞,谓左右曰:“如彼等者,无足与计天下事。”谒者随何进曰:“不审陛下所谓。”汉王曰:“孰能为我使淮南,令之发兵倍楚,留项王于齐数月,我之取天下可以百全。”随何曰:“臣请使之。”乃与二十人俱,使淮南。至,因太宰主之,三日不得见。随何因说太宰曰:“王之不见何,必以楚为彊,以汉为弱,此臣之所以为使。使何得见,言之而是邪,是大王所欲闻也;言之而非邪,使何等二十人伏斧质淮南市,以明王倍汉而与楚也。”太宰乃言之王,王见之。随何曰:“汉王使臣敬进书大王御者,窃怪大王与楚何亲也。”淮南王曰:“寡人北乡而臣事之。”随何曰:“大王与项王俱列为诸侯,北乡而臣事之,必以楚为彊,可以讬国也。项王伐齐,身负板筑,以为士卒先,大王宜悉淮南之众,身自将之,为楚军前锋,今乃发四千人以助楚。夫北面而臣事人者,固若是乎?夫汉王战于彭城,项王未出齐也,大王宜骚淮南之兵渡淮,日夜会战彭城下,大王抚万人之众,无一人渡淮者,垂拱而观其孰胜。夫讬国于人者,固若是乎?大王提空名以乡楚,而欲厚自讬,臣窃为大王不取也。然而大王不背楚者,以汉为弱也。夫楚兵虽彊,天下负之以不义之名,以其背盟约而杀义帝也。然而楚王恃战胜自彊,汉王收诸侯,还守成皋、荥阳,下蜀、汉之粟,深沟壁垒,分卒守徼乘塞,楚人还兵,间以梁地,深入敌国八九百里,欲战则不得,攻城则力不能,老弱转粮千里之外;楚兵至荥阳、成皋,汉坚守而不动,进则不得攻,退则不得解。故曰楚兵不足恃也。使楚胜汉,则诸侯自危惧而相救。夫楚之彊,適足以致天下之兵耳。故楚不如汉,其势易见也。今大王不与万全之汉而自讬于危亡之楚,臣窃为大王惑之。臣非以淮南之兵足以亡楚也。夫大王发兵而倍楚,项王必留;留数月,汉之取天下可以万全。臣请与大王提剑而归汉,汉王必裂地而封大王,又况淮南,淮南必大王有也。故汉王敬使使臣进愚计,原大王之留意也。”淮南王曰:“请奉命。”阴许畔楚与汉,未敢泄也。

楚使者在,方急责英布发兵,舍传舍。随何直入,坐楚使者上坐,曰:“九江王已归汉,楚何以得发兵?”布愕然。楚使者起。何因说布曰:“事已搆,可遂杀楚使者,无使归,而疾走汉并力。”布曰:“如使者教,因起兵而击之耳。”于是杀使者,因起兵而攻楚。楚使项声、龙且攻淮南,项王留而攻下邑。数月,龙且击淮南,破布军。布欲引兵走汉,恐楚王杀之,故间行与何俱归汉。

淮南王至,上方踞床洗,召布入见,布大怒,悔来,欲自杀。出就舍,帐御饮食从官如汉王居,布又大喜过望。于是乃使人入九江。楚已使项伯收九江兵,尽杀布妻子。布使者颇得故人幸臣,将众数千人归汉。汉益分布兵而与俱北,收兵至成皋。四年七月,立布为淮南王,与击项籍。

汉五年,布使人入九江,得数县。六年,布与刘贾入九江,诱大司马周殷,周殷反楚,遂举九江兵与汉击楚,破之垓下。

项籍死,天下定,上置酒。上折随何之功,谓何为腐儒,为天下安用腐儒。随何跪曰:“夫陛下引兵攻彭城,楚王未去齐也,陛下发步卒五万人,骑五千,能以取淮南乎?”上曰:“不能。”随何曰:“陛下使何与二十人使淮南,至,如陛下之意,是何之功贤于步卒五万人骑五千也。然而陛下谓何腐儒,为天下安用腐儒,何也?”上曰:“吾方图子之功。”乃以随何为护军中尉。布遂剖符为淮南王,都六,九江、庐江、衡山、豫章郡皆属布。

七年,朝陈。八年,朝雒阳。九年,朝长安。

十一年,高后诛淮阴侯,布因心恐。夏,汉诛梁王彭越,醢之,盛其醢遍赐诸侯。至淮南,淮南王方猎,见醢,因大恐,阴令人部聚兵,候伺旁郡警急。

布所幸姬疾,请就医,医家与中大夫贲赫对门,姬数如医家,贲赫自以为侍中,乃厚餽遗,从姬饮医家。姬侍王,从容语次,誉赫长者也。王怒曰:“汝安从知之?”具说状。王疑其与乱。赫恐,称病。王愈怒,欲捕赫。赫言变事,乘传诣长安。布使人追,不及。赫至,上变,言布谋反有端,可先未发诛也。上读其书,语萧相国。相国曰:“布不宜有此,恐仇怨妄诬之。请击赫,使人微验淮南王。”淮南王布见赫以罪亡,上变,固已疑其言国阴事;汉使又来,颇有所验,遂族赫家,发兵反。反书闻,上乃赦贲赫,以为将军。

上召诸将问曰:“布反,为之柰何?”皆曰;“发兵击之,阬竖子耳。何能为乎!”汝阴侯滕公召故楚令尹问之。令尹曰:“是故当反。”滕公曰:“上裂地而王之,疏爵而贵之,南面而立万乘之主,其反何也?”令尹曰:“往年杀彭越,前年杀韩信,此三人者,同功一体之人也。自疑祸及身,故反耳。”滕公言之上曰:“臣客故楚令尹薛公者,其人有筹筴之计,可问。”上乃召见问薛公。薛公对曰:“布反不足怪也。使布出于上计,山东非汉之有也;出于中计,胜败之数未可知也;出于下计,陛下安枕而卧矣。”上曰:“何谓上计?”令尹对曰:“东取吴,西取楚,并齐取鲁,传檄燕、赵,固守其所,山东非汉之有也。”“何谓中计?”“东取吴,西取楚,并韩取魏,据敖庾之粟,塞成皋之口,胜败之数未可知也。”“何谓下计?”“东取吴,西取下蔡,归重于越,身归长沙,陛下安枕而卧,汉无事矣。”上曰:“是计将安出?”令尹对曰:“出下计。”上曰:“何谓废上中计而出下计?”令尹曰:“布故丽山之徒也,自致万乘之主,此皆为身,不顾后为百姓万世虑者也,故曰出下计。”上曰:“善。”封薛公千户。乃立皇子长为淮南王。上遂发兵自将东击布。

布之初反,谓其将曰:“上老矣,厌兵,必不能来。使诸将,诸将独患淮阴、彭越,今皆已死,馀不足畏也。”故遂反。果如薛公筹之,东击荆,荆王刘贾走死富陵。尽劫其兵,渡淮击楚。楚发兵与战徐、僮间,为三军,欲以相救为奇。或说楚将曰:“布善用兵,民素畏之。且兵法,诸侯战其地为散地。今别为三,彼败吾一军,馀皆走,安能相救!”不听。布果破其一军,其二军散走。

遂西,与上兵遇蕲西,会甀。布兵精甚,上乃壁庸城,望布军置陈如项籍军,上恶之。与布相望见,遥谓布曰:“何苦而反?”布曰:“欲为帝耳。”上怒骂之,遂大战。布军败走,渡淮,数止战,不利,与百馀人走江南。布故与番君婚,以故长沙哀王使人绐布,伪与亡,诱走越,故信而随之番阳。番阳人杀布兹乡民田舍,遂灭黥布。

立皇子长为淮南王,封贲赫为期思侯,诸将率多以功封者。

太史公曰:英布者,其先岂春秋所见楚灭英、六,皋陶之后哉?身被刑法,何其拔兴之暴也!项氏之所阬杀人以千万数,而布常为首虐。功冠诸侯,用此得王,亦不免于身为世大僇。祸之兴自爱姬殖,妒媢生患,竟以灭国!

九江初筮,当刑而王。既免徒中,聚盗江上。再雄楚卒,频破秦将。病为羽疑,归受汉杖。贲赫见毁,卒致无妄。
\end{yuanwen}

\part{卷九十二}
\chapter{淮阴侯列传第三十二}

记述了我国古代杰出的军事家韩信早年的困辱经历,与其投奔刘邦后大展奇才,佐汉破楚的历史功勋,以及最后被罗织罪名惨遭杀害的结局。司马迁同情韩信,对刘邦、吕后等人的猜忌残忍,则隐约地表现了愤慨与厌恶。韩信的杰出才干以及他的历史功勋是令人钦佩的,他因诬谋反而遭杀害也的确令人同情,但他一直想裂土称王,这无疑是刘邦建立集权国家的一大障碍,尤其使刘邦不能容忍的,是韩信为裂土分封而公然与刘邦讨价还价,甚至不惜坐视刘邦惨败于项羽。韩信又矜才自负,不仅羞与绛、灌为伍,即刘邦本人亦不在其眼目之内,这些也都是他的取死之道。故此事应从两方面分别评论。

苏轼:「抱王霸之大略,蓄英雄之壮图,志吞六合,气盖万夫。」陈亮:「汉高帝所籍以取天下者,故非一人之力,而萧何、韩信、张良盖杰然于其间。天下既定,而不免于疑。于是张良以神仙自托;萧何以谨畏自保;韩信以盖世之功,进退无以自明。萧何能知之于未用之先,而卒不能保其非叛,方且借信以为自保矣。」

\begin{yuanwen}
淮阴侯韩信者,淮阴人也。始为布衣时,贫无行\footnote{text},不得推择为吏\footnote{战国以来,有一种制度是乡官向国家举荐本乡能够做官的人。},又不能治生商贾\footnote{text},常从人寄食饮,人多厌之者。常数从其下乡南昌亭长寄食,数月,亭长妻患之,乃晨炊蓐\footnote{通“褥”。}食\footnote{text}。食时信往,不为具食。信亦知其意,怒,竟绝去。
\end{yuanwen}



\begin{yuanwen}
信钓于城下,诸母漂,有一母见信饥,饭信,竟漂数十日。信喜,谓漂母曰:“吾必有以重报母。”

母怒曰:“大丈夫不能自食\footnote{text},吾哀王孙而进食,岂望报\footnote{text}乎!”
\end{yuanwen}

\begin{yuanwen}
淮阴屠中少年有侮信者,曰:“若虽长大\footnote{text},好带刀剑,中情怯\footnote{text}耳。”

众辱之曰\footnote{text}:“信能死,刺我;不能死,出我袴下\footnote{text}。”

于是信孰视之\footnote{text},俯出袴下,蒲伏\footnote{text}。一市人皆笑信,以为怯。
\end{yuanwen}

\begin{yuanwen}
及项梁渡淮,信杖剑从之\footnote{text},居戏下\footnote{text},无所知名。项梁败,又属项羽,羽以为郎中\footnote{text}。数以策干项羽\footnote{text},羽不用。汉王之入蜀,信亡楚归汉\footnote{text},未得知名,为连敖。坐法当斩,其辈十三人皆已斩,次至信,信乃仰视,適见滕公,曰:“上不欲就天下乎?何为斩壮士!”

滕公奇其言,壮其貌,释而不斩。与语,大说之。言于上,上拜以为治粟都尉,上未之奇也。
\end{yuanwen}

\begin{yuanwen}
信数与萧何语,何奇之。至南郑,诸将行道亡者数十人\footnote{text},信度何等已数言上,上不我用,即亡。何闻信亡,不及以闻,自追之。

人有言上曰:“丞相何亡。”

上大怒,如失左右手。居一二日,何来谒上\footnote{text},上且怒且喜,骂何曰:“若亡,何也?”

何曰:“臣不敢亡也,臣追亡者。”

上曰:“若所追者谁?”

何曰:“韩信也。”

上复骂曰:“诸将亡者以十数,公无所追;追信,诈也。”

何曰:“诸将易得耳。至如信者,国士无双\footnote{text}。王必欲长王汉中,无所事信\footnote{text};必欲争天下,非信无所与计事者。顾王策安所决耳\footnote{text}。”

王曰:“吾亦欲东耳,安能郁郁久居此乎?”

何曰:“王计必欲东,能用信,信即留;不能用,信终亡耳。”

王曰:“吾为公以为将\footnote{text}。”

何曰:“虽为将,信必不留。”

王曰:“以为大将。”

何曰:“幸甚。”

于是王欲召信拜\footnote{text}之。

何曰:“王素慢无礼,今拜大将如呼小儿耳,此乃信所以去也。王必欲拜之,择良日,斋戒,设坛场\footnote{text},具礼\footnote{text},乃可耳。”

王许之。诸将皆喜,人人各自以为得大将。至拜大将,乃韩信也,一军皆惊。
\end{yuanwen}

\begin{yuanwen}
信拜礼毕,上坐\footnote{text}。

王曰:“丞相数言将军,将军何以教寡人计策?”

信谢,因问王曰:“今东乡争权天下\footnote{text},岂非项王邪?”

汉王曰:“然。”

曰:“大王自料勇悍仁强孰与项王?”

汉王默然良久,曰:“不如也。”

信再拜贺曰\footnote{text}:“惟信亦为大王不如也。然臣尝事之,请言项王之为人也。项王喑噁叱咤\footnote{text},千人皆废\footnote{text},然不能任属贤将,此特匹夫之勇耳。项王见人恭敬慈爱,言语呕呕\footnote{text},人有疾病,涕泣分食饮,至使人有功当封爵者,印刓敝\footnote{text},忍不能予\footnote{text},此所谓妇人之仁也。项王虽霸天下而臣诸侯,不居关中而都彭城。有背义帝之约\footnote{text},而以亲爱王,诸侯不平。诸侯之见项王迁逐义帝置江南\footnote{text},亦皆归逐其主而自王善地。项王所过无不残灭者,天下多怨,百姓不亲附,特劫于威强耳。名虽为霸,实失天下心。故曰其强易弱。今大王诚能反其道:任天下武勇,何所不诛!以天下城邑封功臣,何所不服!以义兵从思东归之士\footnote{text},何所不散!且三秦王为秦将\footnote{text},将秦子弟数岁矣,所杀亡不可胜计\footnote{text},又欺其众降诸侯,至新安,项王诈阬秦降卒二十馀万,唯独邯、欣、翳得脱,秦父兄怨此三人,痛入骨髓。今楚强以威王此三人,秦民莫爱也。大王之入武关,秋豪无所害,除秦苛法,与秦民约,法三章\footnote{text}耳,秦民无不欲得大王王秦者。于诸侯之约,大王当王关中,关中民咸知之。大王失职入汉中\footnote{text},秦民无不恨者\footnote{text}。今大王举而东,三秦可传檄而定\footnote{text}也。”

于是汉王大喜,自以为得信晚。遂听信计,部署诸将所击。
\end{yuanwen}

\begin{yuanwen}
八月,汉王举兵东出陈仓\footnote{text},定三秦\footnote{text}。
\end{yuanwen}

\begin{yuanwen}
汉二年,出关,收魏、河南,韩、殷王皆降。合齐、赵共击楚。四月,至彭城,汉兵败散而还。信复收兵与汉王会荥阳,复击破楚京、索之间,以故楚兵卒不能西。

汉之败卻彭城,塞王欣、翟王翳亡汉降楚,齐、赵亦反汉与楚和。六月,魏王豹谒归视亲疾,至国,即绝河关反汉,与楚约和。汉王使郦生说豹,不下。其八月,以信为左丞相,击魏。魏王盛兵蒲坂,塞临晋,信乃益为疑兵,陈船欲度临晋,而伏兵从夏阳以木罂鲊渡军,袭安邑。魏王豹惊,引兵迎信,信遂虏豹,定魏为河东郡。汉王遣张耳与信俱,引兵东,北击赵、代。后九月,破代兵,禽夏说阏与。信之下魏破代,汉辄使人收其精兵,诣荥阳以距楚。
\end{yuanwen}

\begin{yuanwen}
信与张耳以兵数万,欲东下井陉击赵\footnote{text}。赵王、成安君陈馀闻汉且袭之也,聚兵井陉口,号称二十万。广武君李左车说成安君曰:“闻汉将韩信涉西河\footnote{text},虏魏王,禽夏说,新喋血阏与\footnote{text},今乃辅以张耳,议欲下赵,此乘胜而去国远斗,其锋不可当。臣闻千里馈粮,士有饥色,樵苏后爨\footnote{text},师不宿饱。今井陉之道,车不得方轨\footnote{text},骑不得成列,行数百里,其势粮食必在其后。愿足下假臣奇兵三万人,从间道绝其辎重\footnote{text};足下深沟高垒\footnote{text},坚营勿与战。彼前不得斗,退不得还,吾奇兵绝其后,使野无所掠,不至十日,而两将之头可致于戏下。愿君留意臣之计。否,必为二子所禽矣。”

成安君,儒者也,常称义兵不用诈谋奇计,曰:“吾闻兵法十则围之,倍则战\footnote{text}。今韩信兵号数万,其实不过数千\footnote{text}。能千里而袭我,亦已罢极。今如此避而不击,后有大者,何以加之\footnote{text}!则诸侯谓吾怯,而轻来伐我\footnote{text}。”

不听广武君策,广武君策不用。
\end{yuanwen}

\begin{yuanwen}
韩信使人间视\footnote{text},知其不用,还报,则大喜,乃敢引兵遂下。未至井陉口三十里,止舍。夜半传发,选轻骑二千人,人持一赤帜,从间道萆山而望赵军\footnote{text},诫曰:“赵见我走,必空壁逐我,若疾入赵壁,拔赵帜,立汉赤帜。”

令其裨将传飧\footnote{text},曰:“今日破赵会食!”

诸将皆莫信,详应曰:“诺。”

谓军吏曰:“赵已先据便地为壁,且彼未见吾大将旗鼓,未肯击前行\footnote{text},恐吾至阻险而还。”

信乃使万人先行,出,背水陈\footnote{text}。赵军望见而大笑\footnote{text}。平旦,信建大将之旗鼓\footnote{text},鼓行出井陉口\footnote{text},赵开壁击之,大战良久。于是信、张耳详弃鼓旗,走水上军。水上军开入之\footnote{text},复疾战。赵果空壁争汉鼓旗,逐韩信、张耳。

韩信、张耳已入水上军,军皆殊死战,不可败。信所出奇兵二千骑,共候赵空壁逐利,则驰入赵壁,皆拔赵旗,立汉赤帜二千。赵军已不胜,不能得信等,欲还归壁,壁皆汉赤帜,而大惊,以为汉皆已得赵王将矣。兵遂乱,遁走,赵将虽斩之,不能禁也。于是汉兵夹击,大破虏赵军,斩成安君泜水上,禽赵王歇\footnote{text}。
\end{yuanwen}

\begin{yuanwen}
信乃令军中毋杀广武君,有能生得者购千金。于是有缚广武君而致戏下者,信乃解其缚,东乡坐,西乡对,师事之。
\end{yuanwen}

\begin{yuanwen}
诸将效首虏\footnote{text},毕贺,因问信曰:“兵法右倍山陵,前左水泽\footnote{text},今者将军令臣等反背水陈,曰破赵会食,臣等不服。然竟以胜,此何术也?”

信曰:“此在兵法,顾诸君不察耳。兵法不曰‘陷之死地而后生,置之亡地而后存’?且信非得素拊循士大夫也\footnote{text},此所谓‘驱市人而战之’\footnote{text},其势非置之死地,使人人自为战;今予之生地\footnote{text},皆走,宁尚可得而用之乎\footnote{text}!”

诸将皆服曰:“善。非臣所及也。”
\end{yuanwen}

\begin{yuanwen}
于是信问广武君曰:“仆欲北攻燕,东伐齐,何若而有功?”广武君辞谢曰:“臣闻败军之将,不可以言勇,亡国之大夫,不可以图存。今臣败亡之虏,何足以权大事乎!”信曰:“仆闻之,百里奚居虞而虞亡,在秦而秦霸,非愚于虞而智于秦也,用与不用,听与不听也。诚令成安君听足下计,若信者亦已为禽矣。以不用足下,故信得侍耳。”因固问曰:“仆委心归计,原足下勿辞。”广武君曰:“臣闻智者千虑,必有一失;愚者千虑,必有一得。故曰‘狂夫之言,圣人择焉’。顾恐臣计未必足用,原效愚忠。夫成安君有百战百胜之计,一旦而失之,军败鄗下,身死泜上。今将军涉西河,虏魏王,禽夏说阏与,一举而下井陉,不终朝破赵二十万众,诛成安君。名闻海内,威震天下,农夫莫不辍耕释耒,褕衣甘食,倾耳以待命者。若此,将军之所长也。然而众劳卒罢,其实难用。今将军欲举倦弊之兵,顿之燕坚城之下,欲战恐久力不能拔,情见势屈,旷日粮竭,而弱燕不服,齐必距境以自彊也。燕齐相持而不下,则刘项之权未有所分也。若此者,将军所短也。臣愚,窃以为亦过矣。故善用兵者不以短击长,而以长击短。”韩信曰:“然则何由?”广武君对曰:“方今为将军计,莫如案甲休兵,镇赵抚其孤,百里之内,牛酒日至,以飨士大夫醳兵,北首燕路,而后遣辩士奉咫尺之书,暴其所长于燕,燕必不敢不听从。燕已从,使諠言者东告齐,齐必从风而服,虽有智者,亦不知为齐计矣。如是,则天下事皆可图也。兵固有先声而后实者,此之谓也。”韩信曰:“善。”从其策,发使使燕,燕从风而靡。乃遣使报汉,因请立张耳为赵王,以镇抚其国。汉王许之,乃立张耳为赵王。
\end{yuanwen}

\begin{yuanwen}
楚数使奇兵渡河击赵\footnote{text},赵王耳、韩信往来救赵,因行定赵城邑,发兵诣汉\footnote{text}。

楚方急围汉王于荥阳,汉王南出,之宛、叶间,得黥布,走入成皋,楚又复急围之。

六月,汉王出成皋,东渡河\footnote{text},独与滕公俱,从张耳军脩武。至,宿传舍。晨自称汉使,驰入赵壁。张耳、韩信未起,即其卧内上夺其印符\footnote{text},以麾召诸将,易置之。信、耳起,乃知汉王来,大惊\footnote{text}。汉王夺两人军,即令张耳备守赵地。拜韩信为相国\footnote{text},收赵兵未发者击齐\footnote{text}。
\end{yuanwen}

\begin{yuanwen}
信引兵东,未渡平原\footnote{text},闻汉王使郦食其已说下齐\footnote{text},韩信欲止。范阳辩士蒯通说信\footnote{text}曰:“将军受诏击齐,而汉独发间使下齐,宁有诏止将军乎?何以得毋行也!且郦生一士,伏轼掉三寸之舌,下齐七十馀城,将军将数万众,岁馀乃下赵五十馀,为将数岁,反不如一竖儒之功乎?”

于是信然之,从其计,遂渡河。齐已听郦生,即留纵酒,罢备汉守御。信因袭齐历下军,遂至临菑\footnote{text}。齐王田广以郦生卖己\footnote{text},乃亨之\footnote{text},而走高密,使使之楚请救。韩信已定临菑,遂东追广至高密西。楚亦使龙且将,号称二十万,救齐。
\end{yuanwen}

\begin{yuanwen}
齐王广、龙且并军与信战,未合。人或说龙且曰:“汉兵远斗穷战\footnote{text},其锋不可当。齐、楚自居其地战,兵易败散。不如深壁,令齐王使其信臣招所亡城\footnote{text},亡城闻其王在,楚来救,必反汉。汉兵二千里客居,齐城皆反之,其势无所得食,可无战而降也。”

龙且曰:“吾平生知韩信为人,易与耳\footnote{text}。且夫救齐不战而降之,吾何功?今战而胜之,齐之半可得\footnote{text},何为止!”

遂战,与信夹潍水陈\footnote{text}。韩信乃夜令人为万馀囊,满盛沙,壅水上流\footnote{text},引军半渡,击龙且,详不胜,还走。

龙且果喜曰:“固知信怯也。”

遂追信渡水。信使人决壅囊,水大至。龙且军大半不得渡,即急击,杀龙且。龙且水东军散走,齐王广亡去\footnote{text}。信遂追北至城阳,皆虏楚卒。
\end{yuanwen}

\begin{yuanwen}
汉四年,遂皆降平齐\footnote{text}。使人言汉王曰:“齐伪诈多变,反覆之国也,南边楚,不为假王以镇之,其势不定。愿为假王便\footnote{text}。”

当是时,楚方急围汉王于荥阳\footnote{text},韩信使者至,发书,汉王大怒,骂曰:“吾困于此,旦暮望若来佐我,乃欲自立为王!”

张良、陈平蹑汉王足,因附耳语曰:“汉方不利,宁能禁信之王乎?不如因而立,善遇之,使自为守。不然,变生。”

汉王亦悟,因复骂曰:“大丈夫定诸侯,即为真王耳,何以假为\footnote{text}!”乃遣张良往立信为齐王\footnote{text},征其兵击楚。
\end{yuanwen}

\begin{yuanwen}
楚已亡龙且,项王恐,使盱眙人武涉往说齐王信曰:“天下共苦秦久矣,相与戮力击秦\footnote{text}。秦已破,计功割地,分土而王之,以休士卒。今汉王复兴兵而东,侵人之分,夺人之地,已破三秦,引兵出关,收诸侯之兵以东击楚,其意非尽吞天下者不休,其不知厌足如是甚也\footnote{text}。且汉王不可必\footnote{text},身居项王掌握中数矣,项王怜而活之。然得脱,辄倍约,复击项王,其不可亲信如此。今足下虽自以与汉王为厚交,为之尽力用兵,终为之所禽矣。足下所以得须臾至今者\footnote{text},以项王尚存也。当今二王之事,权在足下。足下右投则汉王胜\footnote{text},左投则项王胜。项王今日亡,则次取足下。足下与项王有故,何不反汉与楚连和,参分天下王之?今释此时,而自必于汉以击楚,且为智者固若此乎!”

韩信谢曰:“臣事项王,官不过郎中,位不过执戟,言不听,画不用,故倍楚而归汉。汉王授我上将军印,予我数万众,解衣衣我,推食食我,言听计用,故吾得以至于此。夫人深亲信我,我倍之不祥,虽死不易。幸为信谢项王!”
\end{yuanwen}

\begin{yuanwen}
武涉已去,齐人蒯通知天下权在韩信,欲为奇策而感动之,以相人说韩信曰:“仆尝受相人之术。”韩信曰:“先生相人何如?”对曰:“贵贱在于骨法,忧喜在于容色,成败在于决断,以此参之,万不失一。”韩信曰:“善。先生相寡人何如?”对曰:“原少间。”信曰:“左右去矣。”通曰:“相君之面,不过封侯,又危不安。相君之背,贵乃不可言。”韩信曰:“何谓也?”蒯通曰:“天下初发难也,俊雄豪桀建号壹呼,天下之士云合雾集,鱼鳞櫜鹓,熛至风起。当此之时,忧在亡秦而已。今楚汉分争,使天下无罪之人肝胆涂地,父子暴骸骨于中野,不可胜数。楚人起彭城,转斗逐北,至于荥阳,乘利席卷,威震天下。然兵困于京、索之间,迫西山而不能进者,三年于此矣。汉王将数十万之众,距巩、雒,阻山河之险,一日数战,无尺寸之功,折北不救,败荥阳,伤成皋,遂走宛、叶之间,此所谓智勇俱困者也。夫锐气挫于险塞,而粮食竭于内府,百姓罢极怨望,容容无所倚。以臣料之,其势非天下之贤圣固不能息天下之祸。当今两主之命县于足下。足下为汉则汉胜,与楚则楚胜。臣原披腹心,输肝胆,效愚计,恐足下不能用也。诚能听臣之计,莫若两利而俱存之,参分天下,鼎足而居,其势莫敢先动。夫以足下之贤圣,有甲兵之众,据彊齐,从燕、赵,出空虚之地而制其后,因民之欲,西乡为百姓请命,则天下风走而响应矣,孰敢不听!割大弱彊,以立诸侯,诸侯已立,天下服听而归德于齐。案齐之故,有胶、泗之地,怀诸侯以德,深拱揖让,则天下之君王相率而朝于齐矣。盖闻天与弗取,反受其咎;时至不行,反受其殃。原足下孰虑之。”

韩信曰:“汉王遇我甚厚,载我以其车,衣我以其衣,食我以其食。吾闻之,乘人之车者载人之患,衣人之衣者怀人之忧,食人之食者死人之事,吾岂可以乡利倍义乎!”蒯生曰:“足下自以为善汉王,欲建万世之业,臣窃以为误矣。始常山王、成安君为布衣时,相与为刎颈之交,后争张黡、陈泽之事,二人相怨。常山王背项王,奉项婴头而窜,逃归于汉王。汉王借兵而东下,杀成安君泜水之南,头足异处,卒为天下笑。此二人相与,天下至驩也。然而卒相禽者,何也?患生于多欲而人心难测也。今足下欲行忠信以交于汉王,必不能固于二君之相与也,而事多大于张黡、陈泽。故臣以为足下必汉王之不危己,亦误矣。大夫种、范蠡存亡越,霸句践,立功成名而身死亡。野兽已尽而猎狗烹。夫以交友言之,则不如张耳之与成安君者也;以忠信言之,则不过大夫种、范蠡之于句践也。此二人者,足以观矣。原足下深虑之。且臣闻勇略震主者身危,而功盖天下者不赏。臣请言大王功略:足下涉西河,虏魏王,禽夏说,引兵下井陉,诛成安君,徇赵,胁燕,定齐,南摧楚人之兵二十万,东杀龙且,西乡以报,此所谓功无二于天下,而略不世出者也。今足下戴震主之威,挟不赏之功,归楚,楚人不信;归汉,汉人震恐:足下欲持是安归乎?夫势在人臣之位而有震主之威,名高天下,窃为足下危之。”

韩信谢曰:“先生且休矣,吾将念之。”
\end{yuanwen}

\begin{yuanwen}
后数日,蒯通复说,曰:“夫听者事之候也,计者事之机也,听过计失而能久安者,鲜矣。听不失一二者,不可乱以言;计不失本末者,不可纷以辞。夫随厮养之役者,失万乘之权;守儋石之禄者,阙卿相之位。故知者决之断也,疑者事之害也,审豪氂之小计,遗天下之大数,智诚知之,决弗敢行者,百事之祸也。故曰‘猛虎之犹豫,不若蜂虿之致螫;骐骥之跼躅,不如驽马之安步;孟贲之狐疑,不如庸夫之必至也;虽有舜禹之智,吟而不言,不如瘖聋之指麾也’。此言贵能行之。夫功者难成而易败,时者难得而易失也。时乎时,不再来。原足下详察之。”

韩信犹豫不忍倍汉,又自以为功多,汉终不夺我齐,遂谢蒯通。蒯通说不听,已详狂为巫\footnote{text}。
\end{yuanwen}

\begin{yuanwen}
汉王之困固陵\footnote{text},用张良计\footnote{text},召齐王信,遂将兵会垓下。项羽已破\footnote{text},高祖袭夺齐王军\footnote{text}。汉五年正月,徙齐王信为楚王\footnote{text},都下邳。
\end{yuanwen}

\begin{yuanwen}
信至国,召所从食漂母,赐千金。及下乡南昌亭长,赐百钱,曰:“公,小人也,为德不卒。”

召辱己之少年令出胯下者以为楚中尉\footnote{text}。告诸将相曰:“此壮士也。方辱我时,我宁不能杀之邪?杀之无名,故忍而就于此\footnote{text}。”
\end{yuanwen}

\begin{yuanwen}
项王亡将钟离眛家在伊庐,素与信善。项王死后,亡归信。汉王怨眛\footnote{text},闻其在楚,诏楚捕眛。信初之国,行县邑\footnote{text},陈兵出入。

汉六年\footnote{text},人有上书告楚王信反。高帝以陈平计\footnote{text},天子巡狩会诸侯\footnote{text},南方有云梦\footnote{text},发使告诸侯会陈\footnote{text}:“吾将游云梦。”

实欲袭信,信弗知。高祖且至楚\footnote{text},信欲发兵反\footnote{text},自度无罪,欲谒上,恐见禽。

人或说信曰:“斩眛谒上,上必喜,无患。”

信见眜计事。

眛曰:“汉所以不击取楚,以眛在公所。若欲捕我以自媚于汉\footnote{text},吾今日死,公亦随手亡矣。”

乃骂信曰:“公非长者!”卒自刭。

信持其首,谒高祖于陈\footnote{text}。上令武士缚信,载后车。

信曰:“果若人言,‘狡兔死,良狗亨;高鸟尽,良弓藏;敌国破,谋臣亡’。天下已定,我固当亨!”

上曰:“人告公反。”

遂械系信。至洛阳\footnote{text},赦信罪,以为淮阴侯\footnote{text}。
\end{yuanwen}

\begin{yuanwen}
信知汉王畏恶其能,常称病不朝从\footnote{text}。信由此日夜怨望,居常鞅鞅\footnote{text},羞与绛、灌等列\footnote{text}。信尝过樊将军哙,哙跪拜送迎,言称臣,曰:“大王乃肯临臣!”信出门,笑曰:“生乃与哙等为伍!”

上常从容与信言诸将能不\footnote{text},各有差。

上问曰:“如我能将几何?”

信曰:“陛下不过能将十万。”

上曰:“于君何如?”

曰:“臣多多而益善耳。”

上笑曰:“多多益善,何为为我禽?”

信曰:“陛下不能将兵,而善将将\footnote{text},此信言之所以为陛下禽也。且陛下所谓天授,非人力也。”
\end{yuanwen}

\begin{yuanwen}
陈豨拜为钜鹿守\footnote{text},辞于淮阴侯。淮阴侯挈其手\footnote{text},辟左右与之步于庭,仰天叹曰:“子可与言乎?欲与子有言也。”

豨曰:“唯将军令之。”

淮阴侯曰:“公之所居,天下精兵处也\footnote{text};而公,陛下之信幸臣也\footnote{text}。人言公之畔,陛下必不信;再至,陛下乃疑矣;三至,必怒而自将。吾为公从中起,天下可图\footnote{text}也。”

陈豨素知其能也,信之,曰:“谨奉教!”

汉十年,陈豨果反\footnote{text}。上自将而往,信病不从。阴使人至豨所,曰:“弟举兵,吾从此助公。”

信乃谋与家臣夜诈诏赦诸官徒奴,欲发以袭吕后、太子。部署已定,待豨报。其舍人得罪于信,信囚,欲杀之。舍人弟上变\footnote{text},告信欲反状于吕后。吕后欲召,恐其党不就\footnote{text},乃与萧相国谋,诈令人从上所来,言豨已得死,列侯群臣皆贺。相国绐信曰:“虽疾,强入贺。”

信入,吕后使武士缚信,斩之长乐钟室\footnote{text}。信方斩,曰:“吾悔不用蒯通之计,乃为儿女子所诈\footnote{text},岂非天哉!”

遂夷信三族。
\end{yuanwen}

\begin{yuanwen}
高祖已从豨军来,至,见信死,且喜且怜之,问:“信死亦何言?”吕后曰:“信言恨不用蒯通计。”高祖曰:“是齐辩士也。”乃诏齐捕蒯通。蒯通至,上曰:“若教淮阴侯反乎?”对曰:“然,臣固教之。竖子不用臣之策,故令自夷于此。如彼竖子用臣之计,陛下安得而夷之乎!”上怒曰:“亨之。”通曰:“嗟乎,冤哉亨也!”上曰:“若教韩信反,何冤?”对曰:“秦之纲绝而维弛,山东大扰,异姓并起,英俊乌集。秦失其鹿,天下共逐之,于是高材疾足者先得焉。蹠之狗吠尧,尧非不仁,狗因吠非其主。当是时,臣唯独知韩信,非知陛下也。且天下锐精持锋欲为陛下所为者甚众,顾力不能耳。又可尽亨之邪?”高帝曰:“置之。”乃释通之罪。
\end{yuanwen}

\begin{yuanwen}
太史公曰:吾如淮阴,淮阴人为余言,韩信虽为布衣时,其志与众异。其母死,贫无以葬,然乃行营高敞地\footnote{text},令其旁可置万家。余视其母冢,良然。假令韩信学道谦让\footnote{text},不伐己功\footnote{text},不矜其能,则庶几哉\footnote{text},于汉家勋可以比周、召、太公之徒,后世血食矣\footnote{text}。不务出此,而天下已集,乃谋畔逆,夷灭宗族,不亦宜乎\footnote{text}!
\end{yuanwen}

\begin{yuanwen}
君臣一体,自古所难。相国深荐,策拜登坛。沈沙决水,拔帜传餐。与汉汉重,归楚楚安。三分不议,伪游可叹。
\end{yuanwen}

\chapter{韩信卢绾列传}

\begin{yuanwen}
韩王信者,故韩襄王孽孙也,长八尺五寸。及项梁之立楚后怀王也,燕、齐、赵、魏皆已前王,唯韩无有后,故立韩诸公子横阳君成为韩王,欲以抚定韩故地。项梁败死定陶,成饹怀王。沛公引兵击阳城,使张良以韩司徒降下韩故地,得信,以为韩将,将其兵从沛公入武关。

沛公立为汉王,韩信从入汉中,乃说汉王曰:“项王王诸将近地,而王独远居此,此左迁也。士卒皆山东人,跂而望归,及其锋东乡,可以争天下。”汉王还定三秦,乃许信为韩王,先拜信为韩太尉,将兵略韩地。

项籍之封诸王皆就国,韩王成以不从无功,不遣就国,更以为列侯。及闻汉遣韩信略韩地,乃令故项籍游吴时吴令郑昌为韩王以距汉。汉二年,韩信略定韩十馀城。汉王至河南,韩信急击韩王昌阳城。昌降,汉王乃立韩信为韩王,常将韩兵从。三年,汉王出荥阳,韩王信、周苛等守荥阳。及楚败荥阳,信降楚,已而得亡,复归汉,汉复立以为韩王,竟从击破项籍,天下定。五年春,遂与剖符为韩王,王颍川。

明年春,上以韩信材武,所王北近巩、洛,南迫宛、叶,东有淮阳,皆天下劲兵处,乃诏徙韩王信王太原以北,备御胡,都晋阳。信上书曰:“国被边,匈奴数入,晋阳去塞远,请治马邑。”上许之,信乃徙治马邑。秋,匈奴冒顿大围信,信数使使胡求和解。汉发兵救之,疑信数间使,有二心,使人责让信。信恐诛,因与匈奴约共攻汉,反,以马邑降胡,击太原。

七年冬,上自往击,破信军铜鞮,斩其将王喜。信亡走匈奴。其与白土人曼丘臣、王黄等立赵苗裔赵利为王,复收信败散兵,而与信及冒顿谋攻汉。匈奴仗左右贤王将万馀骑与王黄等屯广武以南,至晋阳,与汉兵战,汉大破之,追至于离石,破之。匈奴复聚兵楼烦西北,汉令车骑击破匈奴。匈奴常败走,汉乘胜追北,闻冒顿居代谷,高皇帝居晋阳,使人视冒顿,还报曰“可击”。上遂至平城。上出白登,匈奴骑围上,上乃使人厚遗阏氏。阏氏乃说冒顿曰:“今得汉地,犹不能居;且两主不相戹。”居七日,胡骑稍引去。时天大雾,汉使人往来,胡不觉。护军中尉陈平言上曰:“胡者全兵,请令彊弩傅两矢外乡,徐行出围。”入平城,汉救兵亦到,胡骑遂解去。汉亦罢兵归。韩信为匈奴将兵往来击边。

汉十年,信令王黄等说误陈豨。十一年春,故韩王信复与胡骑入居参合,距汉。汉使柴将军击之,遗信书曰:“陛下宽仁,诸侯虽有畔亡,而复归,辄复故位号,不诛也。大王所知。今王以败亡走胡,非有大罪,急自归!”韩王信报曰:“陛下擢仆起闾巷,南面称孤,此仆之幸也。荥阳之事,仆不能死,囚于项籍,此一罪也。及寇攻马邑,仆不能坚守,以城降之,此二罪也。今反为寇将兵,与将军争一旦之命,此三罪也。夫种、蠡无一罪,身死亡;今仆有三罪于陛下,而欲求活于世,此伍子胥所以偾于吴也。今仆亡匿山谷间,旦暮乞贷蛮夷,仆之思归,如痿人不忘起,盲者不忘视也,势不可耳。”遂战。柴将军屠参合,斩韩王信。

信之入匈奴,与太子俱;及至穨当城,生子,因名曰穨当。韩太子亦生子,命曰婴。至孝文十四年,穨当及婴率其众降汉。汉封穨当为弓高侯,婴为襄城侯。吴楚军时,弓高侯功冠诸将。传子至孙,孙无子,失侯。婴孙以不敬失侯。穨当孽孙韩嫣,贵幸,名富显于当世。其弟说,再封,数称将军,卒为案道侯。子代,岁馀坐法死。后岁馀,说孙曾拜为龙嵒侯,续说后。

卢绾者,丰人也,与高祖同里。卢绾亲与高祖太上皇相爱,及生男,高祖、卢绾同日生,里中持羊酒贺两家。及高祖、卢绾壮,俱学书,又相爱也。里中嘉两家亲相爱,生子同日,壮又相爱,复贺两家羊酒。高祖为布衣时,有吏事辟匿,卢绾常随出入上下。及高祖初起沛,卢绾以客从,入汉中为将军,常侍中。从东击项籍,以太尉常从,出入卧内,衣被饮食赏赐,群臣莫敢望,虽萧曹等,特以事见礼,至其亲幸,莫及卢绾。绾封为长安侯。长安,故咸阳也。

汉五年冬,以破项籍,乃使卢绾别将,与刘贾击临江王共尉,破之。七月还,从击燕王臧荼,臧荼降。高祖已定天下,诸侯非刘氏而王者七人。欲王卢绾,为群臣觖望。及虏臧荼,乃下诏诸将相列侯,择群臣有功者以为燕王。群臣知上欲王卢绾,皆言曰:“太尉长安侯卢绾常从平定天下,功最多,可王燕。”诏许之。汉五年八月,乃立虏绾为燕王。诸侯王得幸莫如燕王。

汉十一年秋,陈豨反代地,高祖如邯郸击豨兵,燕王绾亦击其东北。当是时,陈豨使王黄求救匈奴。燕王绾亦使其臣张胜于匈奴,言豨等军破。张胜至胡,故燕王臧茶子衍出亡在胡,见张胜曰:“公所以重于燕者,以习胡事也。燕所以久存者,以诸侯数反,兵连不决也。今公为燕欲急灭豨等,豨等已尽,次亦至燕,公等亦且为虏矣。公何不令燕且缓陈豨而与胡和?事宽,得长王燕;即有汉急,可以安国。”张胜以为然,乃私令匈奴助豨等击燕。燕王绾疑张胜与胡反,上书请族张胜。胜还,具道所以为者。燕王寤,乃诈论它人,脱胜家属,使得为匈奴间,而阴使范齐之陈豨所,欲令久亡,连兵勿决。

汉十二年,东击黥布,豨常将兵居代,汉使樊哙击斩豨。其裨将降,言燕王绾使范齐通计谋于豨所。高祖使使召卢绾,绾称病。上又使辟阳侯审食其、御史大夫赵尧往迎燕王,因验问左右。绾愈恐,闭匿,谓其幸臣曰:“非刘氏而王,独我与长沙耳。往年春,汉族淮阴,夏,诛彭越,皆吕后计。今上病,属任吕后。吕后妇人,专欲以事诛异姓王者及大功臣。”乃遂称病不行。其左右皆亡匿。语颇泄,辟阳侯闻之,归具报上,上益怒。又得匈奴降者,降者言张胜亡在匈奴,为燕使。于是上曰:“卢绾果反矣!”使樊哙击燕。燕王绾悉将其宫人家属骑数千居长城下,侯伺,幸上病愈,自入谢。四月,高祖崩,卢绾遂将其众亡入匈奴,匈奴以为东胡卢王。绾为蛮夷所侵夺,常思复归。居岁馀,死胡中。

高后时,卢绾妻子亡降汉,会高后病,不能见,舍燕邸,为欲置酒见之。高祖竟崩,不得见。卢绾妻亦病死。

孝景中六年,卢绾孙他之,以东胡王降,封为亚谷侯。

陈豨者,宛朐人也,不知始所以得从。及高祖七年冬,韩王信反,入匈奴,上至平城还,乃封豨为列侯,以赵相国将监赵、代边兵,边兵皆属焉。

豨常告归过赵,赵相周昌见豨宾客随之者千馀乘,邯郸官舍皆满。豨所以待宾客布衣交,皆出客下。豨还之代,周昌乃求入见。见上,具言豨宾客盛甚,擅兵于外数岁,恐有变。上乃令人覆案豨客居代者财物诸不法事,多连引豨。豨恐,阴令客通使王黄、曼丘臣所。及高祖十年七月,太上皇崩,使人召豨,豨称病甚。九月,遂与王黄等反,自立为代王,劫略赵、代。

上闻,乃赦赵、代吏人为豨所诖误劫略者,皆赦之。上自往,至邯郸,喜曰:“豨不南据漳水,北守邯郸,知其无能为也。”赵相奏斩常山守、尉,曰:“常山二十五城,豨反,亡其二十城。”上问曰:“守、尉反乎?”对曰:“不反。”上曰:“是力不足也。”赦之,复以为常山守、尉。上问周昌曰:“赵亦有壮士可令将者乎?”对曰:“有四人。”四人谒,上谩骂曰:“竖子能为将乎?”四人惭伏。上封之各千户,以为将。左右谏曰:“从入蜀、汉,伐楚,功未遍行,今此何功而封?”上曰:“非若所知!陈豨反,邯郸以北皆豨有,吾以羽檄徵天下兵,未有至者,今唯独邯郸中兵耳。吾胡爱四千户封四人,不以慰赵子弟!”皆曰:“善。”于是上曰:“陈豨将谁?”曰:“王黄、曼丘臣,皆故贾人。”上曰:“吾知之矣。”乃各以千金购黄、臣等。

十一年冬,汉兵击斩陈豨将侯敞、王黄于曲逆下,破豨将张春于聊城,斩首万馀。太尉勃入定太原、代地。十二月,上自击东垣,东垣不下,卒骂上;东垣降,卒骂者斩之,不骂者黥之。更命东垣为真定。王黄、曼丘臣其麾下受购赏之,皆生得,以故陈豨军遂败。

上还至洛阳。上曰:“代居常山北,赵乃从山南有之,远。”乃立子恆为代王,都中都,代、雁门皆属代。

高祖十二年冬,樊哙军卒追斩豨于灵丘。

太史公曰:韩信、卢绾非素积德累善之世,徼一时权变,以诈力成功,遭汉初定,故得列地,南面称孤。内见疑彊大,外倚蛮貊以为援,是以日疏自危,事穷智困,卒赴匈奴,岂不哀哉!陈豨,梁人,其少时数称慕魏公子;及将军守边,招致宾客而下士,名声过实。周昌疑之,疵瑕颇起,惧祸及身,邪人进说,遂陷无道。于戏悲夫!夫计之生孰成败于人也深矣!

韩襄遗孽,始从汉中。剖符南面,徙邑北通。穨当归国,龙雒有功。卢绾亲爱,群臣莫同。旧燕是王,东胡计穷。
\end{yuanwen}

\part{卷九十四}
\chapter{田儋列传第三十四}

班固:「周室既坏,至春秋末,诸侯耗尽,而炎、黄、唐、虞之苗裔尚犹颇有存者。秦灭六国,而上古遗烈扫地尽矣。楚、汉之际,豪桀相王,唯魏豹、韩信、田儋兄弟为旧国之后,然皆及身而绝。横之志节,宾客慕义,犹不能自立,岂非天虖!」

\begin{yuanwen}
田儋者,狄人也,故齐王\footnote{战国时期齐王。}田氏族也。儋从弟田荣,荣弟田横,皆豪,宗彊,能得人\footnote{人缘很好,深得人心。}。
\end{yuanwen}

田儋,是狄城人,他是战国时齐王田氏的族人。田儋的堂弟名字叫田荣,田荣的弟弟名字叫田横,他们为人豪迈,加上田氏宗族十分强大,因此深得人心。

\begin{yuanwen}
陈涉之初起王楚也,使周市略定魏地,北至狄,狄城守。田儋详\footnote{通“佯”,假装,佯装。}为缚其奴,从\footnote{带领。}少年之廷,欲谒\footnote{请见。}杀奴。见狄令,因击杀令,而召豪吏子弟曰:“诸侯皆反秦自立,齐,古之建国,儋,田氏,当王。”

遂自立为齐王,发兵以击周市。周市军还去,田儋因率兵东略定齐地。
\end{yuanwen}

陈涉最初起事称王的时候,派周巿前去攻占魏地,周巿带兵北上来到狄城,狄城守卫坚守城池,不肯投降。田儋假装捆住自己的奴仆,让一些年轻人跟着他一同来到县府,想要借杀奴仆的事情让县令来接见自己。田儋见到狄县县令的时候,立即找机会杀死县令,然后召集了当地的富豪官吏以及青年人说道:“天下诸侯都反对暴秦统治,纷纷自立为王,齐国,是古代的封建国家,我田儋是战国时期齐王田氏的族人,理应称为齐王。”

于是田儋自立为齐王,出兵攻打周巿。周巿带领军队撤走,田儋抓住机会带领军队向东进军平定了齐国的土地。

\begin{yuanwen}
秦将章邯围魏王咎于临济,急。魏王请救于齐,齐王田儋将兵救魏。章邯夜衔枚击,大破齐、魏军,杀田儋于临济下。儋弟田荣收儋馀兵东走东阿。
\end{yuanwen}

秦将章邯将魏王咎围困在临济,形势危急。魏王咎请求齐国出兵支援,齐王田儋率领兵马前去援救魏国。秦将章邯在夜晚命令部下衔枚出击,大败齐、魏两国军队,在临济城下将田儋杀死。田儋的弟弟田荣将田儋的残兵收编起来,撤退至东阿。

\begin{yuanwen}
齐人闻王田儋死,乃立故齐王建之弟田假为齐王,田角为相,田间为将,以距诸侯。
\end{yuanwen}

齐国人得知田儋战死沙场,于是拥立原来齐王田建的弟弟田假为齐王,任命田角为齐国的相国,田间为齐国的将军,让他带兵来与诸侯抗衡。

\begin{yuanwen}
田荣之走东阿,章邯追围之。项梁闻田荣之急,乃引兵击破章邯军东阿下。章邯走而西,项梁因追之。而田荣怒齐之立假,乃引兵归,击逐齐王假。假亡走楚。齐相角亡走赵;角弟田间前求救赵,因留不敢归。田荣乃立田儋子市为齐王。荣相之,田横为将,平齐地。
\end{yuanwen}

田荣兵败之后,逃到了东阿,秦将章邯紧随其后并围攻他。项梁得知田荣情况危急,于是带领兵马来到东阿城下,打败了章邯的军队。章邯向西落荒而逃,项梁乘胜追击。而田荣因为齐人将田假拥立为王而气愤不已,于是带兵返回齐国,把齐王田假驱赶出齐国。田假逃至楚国。齐国的相国田角逃亡到赵国。田角的弟弟田间在此之前就来到赵国,请求赵国发兵支援,因而趁机留在赵国不敢回到齐国。于是田荣拥立田儋的儿子田巿为齐王。田荣辅佐他,田横为大将军,平定齐地。

程昱:「昔田横,齐之世族,据千里之地,拥百万之众,与诸侯并南面称孤。」诸葛亮:「田横,齐之壮士耳,犹守义不辱。」

\begin{yuanwen}

项梁既追章邯,章邯兵益盛,项梁使使告赵、齐,发兵共击章邯。田荣曰:“使楚杀田假,赵杀田角、田间,乃肯出兵。”

楚怀王曰:“田假与国之王,穷而归我,杀之不义。”

赵亦不杀田角、田间以市\footnote{讨好。}于齐。齐曰:“蝮螫\footnote{咬伤。}手则斩手,螫足则斩足。何者?为害于身也。今田假、田角、田间于楚、赵,非直\footnote{不仅仅。}手足戚也,何故不杀?且秦复得志于天下,则齮龁\footnote{挖掘。}用事者坟墓矣。”

楚、赵不听,齐亦怒,终不肯出兵。章邯果败杀项梁,破楚兵,楚兵东走,而章邯渡河围赵于钜鹿。项羽往救赵,由此怨田荣。
\end{yuanwen}

项梁追赶章邯,章邯的士兵数量反倒日益增多。项梁派使者前去告知赵、齐两国,要求他们发兵共同攻打章邯。田荣说:“假如楚国杀死田假,赵国杀死田角、田间,我就愿意出兵相助。”

楚怀王说:“田假是我们盟国的君王,他在走投无路的时候前来投靠我们,如今我要是将他杀了是很没有道义的。”

赵国也不同意杀死田角、田间,以此去讨好齐国。齐国使者说:“蝮蛇咬伤了手,就一定要砍去手;咬伤了脚,就一定要砍去脚。这是为什么呢?因为蝮蛇有毒,会危害全身。如今田假、田角、田间对楚国、赵国来说,恐怕不仅仅是手足之忧,为什么却不肯杀死他们呢?况且假如让秦国再次称霸天下的阴谋得逞,那些为抵抗秦而奋不顾身的首领死后就都要被开棺挖坟了。”

楚国和赵国仍然不肯听从,齐国也由此而对两国心生怨恨,始终不肯出兵帮助项梁。章邯竟然真的打败并杀死了项梁,楚军大败。败退的楚军向东逃亡,而章邯渡过黄河,在钜鹿围攻赵军。项羽前去支援赵军,也因此对田荣心怀怨恨。

\begin{yuanwen}
项羽既存赵,降章邯等,西屠咸阳,灭秦而立侯王也,乃徙齐王田市更王胶东,治即墨。齐将田都从共救赵,因入关,故立都为齐王,治临淄。故齐王建孙田安,项羽方渡河救赵,田安下济北数城,引兵降项羽,项羽立田安为济北王,治博阳。田荣以负项梁不肯出兵助楚、赵攻秦,故不得王;赵将陈馀亦失职\footnote{没有得到封赏,没有着落。},不得王:二人俱怨项王。
\end{yuanwen}

项羽保全了赵国之后,降服了章邯等人,向西进军血洗咸阳,灭掉秦国并分封诸侯王,于是将齐王田巿改封为胶东王,将都城迁到即墨。齐将田都曾经跟随项羽共同援救赵国,又跟随项羽一同进入函谷关,因此项羽封田都为齐王,都城设在临淄。原齐王田建的孙子田安,在项羽刚刚渡过黄河前去援助赵国的时候,就已经在济北攻占了好几个城邑,看见项羽到来,带领士兵向项羽投降,于是项羽就封田安为济北王,都城设在博阳。田荣因为背叛了项梁,不肯出兵援助楚国、赵国联合攻打秦军,所以没有被封王。而赵国的将领陈馀也因为失职没能被封王:二人因此都对项王心存怨恨。

\begin{yuanwen}
顼王既归,诸侯各就国,田荣使人将兵助陈馀,令反赵地,而荣亦发兵以距击田都,田都亡走楚。田荣留齐王市,无令之胶东。市之左右曰:“项王彊暴,而王当之胶东,不就国,必危。”

市惧,乃亡就国\footnote{偷偷跑到封国。}。田荣怒,追击杀齐王市于即墨,还攻杀济北王安。于是田荣乃自立为齐王,尽并三齐之地。
\end{yuanwen}

项王班师回楚国后,各路诸侯王也纷纷返回自己的封地。于是田荣派人率军帮助陈馀,让他在赵地背叛项羽。而田荣也亲自出兵攻打田都,田都逃亡到楚国。田荣扣留了齐王田巿,不让他前往胶东的封国。田巿身边的随从说:“项王为人强悍暴躁,而大王原本应该是前往胶东封国的,如果不去封地的话,一定会陷入危机。”

田巿心里很害怕,于是偷偷跑到封地。田荣暴怒,追击并在即墨将田巿杀死,又回军攻打济北王,杀死济北王田安。于是田荣自立为齐王,将三齐地区合并到一起。

\begin{yuanwen}
项王闻之,大怒,乃北伐齐。齐王田荣兵败,走平原,平原人杀荣。项王遂烧夷齐城郭,所过者尽屠之。齐人相聚畔之。荣弟横,收齐散兵,得数万人,反击项羽于城阳。而汉王率诸侯败楚,入彭城。项羽闻之,乃醳\footnote{放弃。}齐而归,击汉于彭城,因连与汉战,相距荥阳。以故田横复得收齐城邑,立田荣子广为齐王,而横相之,专国政,政无巨细皆断于相。
\end{yuanwen}

项王听说此事,非常生气,于是带兵北上攻打齐国。齐王田荣的军队战败,田荣逃到平原,平原人将田荣杀死。项王将齐都的城郭夷为平地,所经之处都被血洗。齐人也聚集到一起反叛项王。田荣的弟弟田横,收编了齐国的残兵,一共获得数万名士兵,在城阳对项羽发起反抗。汉王带领诸侯大败楚军,进入彭城。项羽听说这件事,就放弃攻打齐军,收兵回楚,在彭城攻打汉军,接着持续与汉军交战,然后两军在荥阳对峙。所以田横又有机会收复了齐国的城池,立田荣的儿子田广为齐王,而田横自己则担任相国来辅助齐王,田横独揽朝政,朝中无论大事小情都由相国决断。

韩愈:「余既博观乎天下,歇有庶几乎夫子之所为……自古死者非一,夫子至今有耿光,跪陈词而荐酒,魂仿佛其来享」。司马贞:「秦项之际,天下交兵。六国树党,自置豪英。田儋殒寇,立市相荣。楚封王假,齐破郦生。兄弟更王,海岛传声。」

\begin{yuanwen}
横定齐三年,汉王使郦生往说下齐王广及其相国横。横以为然,解其历下军。汉将韩信引兵且东击齐。齐初使华无伤、田解军于历下以距汉,汉使\footnote{指郦食其。}至,乃罢守战备,纵酒,且\footnote{即将。}遣使与汉平。汉将韩信已平赵、燕,用蒯通计,度平原,袭破齐历下军,因入临淄。齐王广、相横怒,以郦生卖己,而亨郦生。齐王广东走高密,相横走博,守相田光走城阳,将军田既军于胶东。楚使龙且救齐,齐王与合军高密。汉将韩信与曹参破杀龙且,虏齐王广。汉将灌婴追得齐守相田光。至博,而横闻齐王死,自立为齐王,还击婴,婴败横之军于嬴下。田横亡走梁,归彭越。彭越是时居梁地,中立,且为汉,且为楚。韩信已杀龙且,因令曹参进兵破杀田既于胶东,使灌婴破杀齐将田吸于千乘。韩信遂平齐,乞自立为齐假王,汉因而立之。
\end{yuanwen}

田横平定齐国三年之后,汉王派郦食其前去游说齐王田广和相国田横,想让他们归顺汉王。田横认为归顺汉王是正确的选择,于是撤回他们在历下的驻军。汉王麾下将领韩信带领兵马从东面攻打齐国。起初齐国派华无伤和田解在历下驻扎军队来抵御汉兵。汉王派使者郦食其前来游说,于是齐王就解除了战备,纵情饮酒,并且派使者前去同汉军讲和。汉军将领韩信平定赵国、燕国以后,就采用蒯通进献的计策,渡过平原津,攻破齐国在历下的军队,并且乘胜追击进入临淄。齐王田广和相国田横非常生气,认为郦食其出卖了他们,于是烹杀郦食其。齐王田广向东逃到高密,相国田横逃到博阳,代理相国田光逃到城阳,将军田既驻扎在胶东。楚军派龙且前来救援齐军,齐王和龙且两军在高密会合。汉将韩信和曹参将龙且打败并且杀死了他,活捉了齐王田广。汉将灌婴前去追击并且俘获了齐国的代理相国田光。田横来到博阳的时候,听说齐王已经战死,于是自立为王,反击灌婴,灌婴在嬴城下打败了田横的军队。田横逃到梁地,依附于彭越。彭越当时正驻守在梁地,保持中立的态度,既想帮助汉王,又想帮项王。韩信杀死龙且之后,就派曹参进军胶东,杀死了齐国将领田既,又派灌婴攻打千乘,击败并除掉了齐将田吸。韩信最后得以平定齐国,上书请求汉王立自己为齐国的代理国王,汉王便顺水推舟立他为齐王。

\begin{yuanwen}
后岁馀,汉灭项籍,汉王立为皇帝,以彭越为梁王。田横惧诛,而与其徒属五百馀人入海,居岛中。高帝闻之,以为田横兄弟本定齐,齐人贤者多附焉,今在海中不收,后恐为乱,乃使使赦田横罪而召之。田横因谢曰:“臣亨陛下之使郦生,今闻其弟郦商为汉将而贤,臣恐惧,不敢奉诏,请为庶人,守海岛中。”

使还报,高皇帝乃诏卫尉郦商曰:“齐王田横即至,人马从者敢动摇者致\footnote{招致。}族夷!”

乃复使使持节\footnote{手持旌节。}具告以诏商状,曰:“田横来,大者王,小者乃侯耳;不来,且举兵加诛焉。”

田横乃与其客二人乘传诣雒阳。
\end{yuanwen}

过了一年多,汉军杀死了项籍,汉王自立为皇帝,封彭越为梁王。田横担心自己会被杀,于是与他的五百多名部属逃往东海,住在岛上。高帝得知此事,认为田横兄弟原本已经平定了齐国,齐国的大多数贤人也都归附于他们,如今让他在海岛上生存,不招抚的话,恐怕以后会发生什么变乱,于是就派使者前往东海,宽赦了田横所犯下的罪过,同时召他入宫。田横再三辞谢说:“我曾经烹杀了陛下派来的使者郦食其,如今得知他的弟弟郦商是汉朝的大将,并且十分贤能,我心中害怕,不敢奉诏前往。请允许我做一个普通的百姓,留守在这海岛之中。”

使者回来向皇帝报告,高皇帝就传诏卫尉郦商说:“齐王田横即将到来,敢伤害他们的人就会遭受灭门之祸!”

于是又派使者拿着符节将皇上传召郦商的情况一五一十地告诉田横,说:“田横如果奉诏前来,大则可以封王,小则可以为侯;如果不来,我将要发兵攻打这里。”

于是田横和跟随他的两位门客乘坐着驿车前往雒阳。

\begin{yuanwen}
未至三十里,至尸乡厩置\footnote{驿站。},横谢使者曰:“人臣见天子当洗沐。”

止留。谓其客曰:“横始与汉王俱南面称孤,今汉王为天子,而横乃为亡虏而北面事之,其耻固已甚矣。且吾亨人之兄,与其弟并肩而事其主,纵彼畏天子之诏,不敢动我,我独不愧于心乎?且陛下所以欲见我者,不过欲一见吾面貌耳。今陛下在(洛/雒)阳,今斩吾头,驰三十里间,形容尚未能败,犹可观也。”

遂自刭,令客奉其头,从使者驰奏之高帝。高帝曰:“嗟乎,有以也夫!起自布衣,兄弟三人更王,岂不贤乎哉!”

为之流涕,而拜其二客为都尉,发卒二千人,以王者礼葬田横。
\end{yuanwen}

在距离雒阳不到三十里的时候,到达尸乡驿站,田横婉言告诉使者说:“人臣朝见天子应该先进行沐浴。”

于是在尸乡驿站停了下来。田横对他的门客说:“我田横最初和汉王都是称孤道寡的人,如今那汉王成为天子,而我田横却沦落为一个亡国的俘虏,要向他俯首称臣事奉他,所受的耻辱本来就已经很严重了。何况我烹杀了人家的兄长,现在却同他的弟弟一起并肩侍侯他的主子。就算是他再敬畏天子传下的诏令,不敢伤害我,难道我就会心无惭愧吗?何况陛下想见我的原因,只不过是想要看看我的样貌罢了。如今陛下身在雒阳,要是砍下我的头,策马奔驰三十里,容貌尚且不会发生变化,还是可以看到的。”

于是自刎而死,让门客带着他的头颅,跟随使者飞车回禀高帝。高帝说:“唉,真是不简单!田横以一介平民起家,兄弟三人先后称王,难道他们不是贤能的人吗!”

高帝为他流下了眼泪,委任他的两个门客为都尉,派两千名士兵,依照侯王的礼仪安葬了田横。

\begin{yuanwen}
既葬,二客穿其冢旁孔,皆自刭,下从之。高帝闻之,乃大惊,大田横之客皆贤。“吾闻其馀尚五百人在海中。”

使使召之。至则闻田横死,亦皆自杀。于是乃知田横兄弟能得士也。
\end{yuanwen}

田横下葬之后,两个门客就在田横的墓穴旁边挖了坑,也都自刎而死,倒进坑里给田横陪葬。高帝得知此事后,十分惊讶,认为田横的门客都是十分贤能的人。“我听说田横还有五百名门客留在海岛上。”

于是派使者前去召见他们。使者到了之后,门客们听说田横死了的消息,也纷纷自杀了。如此可以看出,田横兄弟深得贤士之心。

钱穆:「田横英名乃垂百世,长为吾中华民族一人物。此亦一成功,非失败。」

\begin{yuanwen}
太史公曰:甚矣蒯通之谋,乱齐骄淮阴,其卒亡此两人\footnote{田横、韩信。}!蒯通者,善为长短说,论战国之权变,为八十一首。通善齐人安期生,安期生尝干项羽,项羽不能用其策。已而项羽欲封此两人,两人终不肯受,亡去。田横之高节,宾客慕义而从横死,岂非至贤!余因而列焉。不无善画者,莫能图,何哉?
\end{yuanwen}

太史公说:蒯通的计策实在是太过分了,它扰乱了整个齐国,又让淮阴侯变得骄傲狂妄,最终害死了田横和韩信!那蒯通擅长纵横之术,讨论战国的权宜机变,一共写了八十一篇文章。蒯通与齐国人安期生私交甚好,安期生曾经请求项羽采用他的计策,但是项羽没有采纳。后来,项羽想要赏赐这两个人,两人始终都没有接受,还逃离了楚国。田横的高风亮节,能让他的宾客们都因为仰慕他的高义而追随他一同赴死,这难道不能说是非常贤能的人吗!因此我才记述了他的事迹。天下并不缺乏善于绘画的人,但是竟然没有人将田横和他的门客慕义死节的事情描绘出来,这是为什么呢?

\begin{yuanwen}
秦项之际,天下交兵。六国树党,自置豪英。田儋殒寇,立市相荣。楚封王假,齐破郦生。兄弟更王,海岛传声。
\end{yuanwen}


\chapter{樊郦滕灌列传}

\begin{yuanwen}
舞阳侯樊哙者,沛人也。以屠狗为事,与高祖俱隐。

初从高祖起丰,攻下沛。高祖为沛公,以哙为舍人。从攻胡陵、方与,还守丰,击泗水监丰下,破之。复东定沛,破泗水守薛西。与司马枿战砀东,卻敌,斩首十五级,赐爵国大夫。常从,沛公击章邯军濮阳,攻城先登,斩首二十三级,赐爵列大夫。复常从,从攻城阳,先登。下户牖,破李由军,斩首十六级,赐上间爵。从攻围东郡守尉于成武,卻敌,斩首十四级,捕虏十一人,赐爵五大夫。从击秦军,出亳南。河间守军于杠里,破之。击破赵贲军开封北,以卻敌先登,斩候一人,首六十八级,捕虏二十七人,赐爵卿。从攻破杨熊军于曲遇。攻宛陵,先登,斩首八级,捕虏四十四人,赐爵封号贤成君。从攻长社、轘辕,绝河津,东攻秦军于尸,南攻秦军于犨。破南阳守齮于阳城。东攻宛城,先登。西至郦,以卻敌,斩首二十四级,捕虏四十人,赐重封。攻武关,至霸上,斩都尉一人,首十级,捕虏百四十六人,降卒二千九百人。

项羽在戏下,欲攻沛公。沛公从百馀骑因项伯面见项羽,谢无有闭关事。项羽既飨军士,中酒,亚父谋欲杀沛公,令项庄拔剑舞坐中,欲击沛公,项伯常蔽之。时独沛公与张良得入坐,樊哙在营外,闻事急,乃持铁盾入到营。营卫止哙,哙直撞入,立帐下。项羽目之,问为谁。张良曰:“沛公参乘樊哙。”项羽曰:“壮士。”赐之卮酒彘肩。哙既饮酒,拔剑切肉食,尽之。项羽曰:“能复饮乎?”哙曰:“臣死且不辞,岂特卮酒乎!且沛公先入定咸阳,暴师霸上,以待大王。大王今日至,听小人之言,与沛公有隙,臣恐天下解,心疑大王也。”项羽默然。沛公如厕,麾樊哙去。既出,沛公留车骑,独骑一马,与樊哙等四人步从,从间道山下归走霸上军,而使张良谢项羽。项羽亦因遂已,无诛沛公之心矣。是日微樊哙饹入营谯让项羽,沛公事几殆。

明日,项羽入屠咸阳,立沛公为汉王。汉王赐哙爵为列侯,号临武侯。迁为郎中,从入汉中。

还定三秦,别击西丞白水北,雍轻车骑于雍南,破之。从攻雍、斄城,先登击章平军好畤,攻城,先登陷阵,斩县令丞各一人,首十一级,虏二十人,迁郎中骑将。从击秦车骑壤东,卻敌,迁为将军。攻赵贲,下郿、槐里、柳中、咸阳;灌废丘,最。至栎阳,赐食邑杜之樊乡。从攻项籍,屠煮枣。击破王武、程处军于外黄。攻邹、鲁、瑕丘、薛。项羽败汉王于彭城,尽复取鲁、梁地。哙还至荥阳,益食平阴二千户,以将军守广武。一岁,项羽引而东。从高祖击项籍,下阳夏,虏楚周将军卒四千人。围项籍于陈,大破之。屠胡陵。

项籍既死,汉王为帝,以哙坚守战有功,益食八百户。从高帝攻反燕王臧荼,虏荼,定燕地。楚王韩信反,哙从至陈,取信,定楚。更赐爵列侯,与诸侯剖符,世世勿绝,食舞阳,号为舞阳侯,除前所食。以将军从高祖攻反韩王信于代。自霍人以往至云中,与绛侯等共定之,益食千五百户。因击陈豨与曼丘臣军,战襄国,破柏人,先登,降定清河、常山凡二十七县,残东垣,迁为左丞相。破得綦毋卬、尹潘军于无终、广昌。破豨别将胡人王黄军于代南,因击韩信军于参合。军所将卒斩韩信,破豨胡骑横谷,斩将军赵既,虏代丞相冯梁、守孙奋、大将王黄、将军、太仆解福等十人。与诸将共定代乡邑七十三。其后燕王卢绾反,哙以相国击卢绾,破其丞相抵蓟南,定燕地,凡县十八,乡邑五十一。益食邑千三百户,定食舞阳五千四百户。从,斩首百七十六级,虏二百八十八人。别,破军七,下城五,定郡六,县五十二,得丞相一人,将军十二人,二千石已下至三百石十一人。

哙以吕后女弟吕须为妇,生子伉,故其比诸将最亲。

先黥布反时,高祖尝病甚,恶见人,卧禁中,诏户者无得入群臣。群臣绛、灌等莫敢入。十馀日,哙乃排闼直入,大臣随之。上独枕一宦者卧。哙等见上流涕曰:“始陛下与臣等起丰沛,定天下,何其壮也!今天下已定,又何惫也!且陛下病甚,大臣震恐,不见臣等计事,顾独与一宦者绝乎?且陛下独不见赵高之事乎?”高帝笑而起。

其后卢绾反,高帝使哙以相国击燕。是时高帝病甚,人有恶哙党于吕氏,即上一日宫车晏驾,则哙欲以兵尽诛灭戚氏、赵王如意之属。高帝闻之大怒,乃使陈平载绛侯代将,而即军中斩哙。陈平畏吕后,执哙诣长安。至则高祖已崩,吕后释哙,使复爵邑。

孝惠六年,樊哙卒,谥为武侯。子伉代侯。而伉母吕须亦为临光侯,高后时用事专权,大臣尽畏之。伉代侯九岁,高后崩。大臣诛诸吕、吕须婘属,因诛伉。舞阳侯中绝数月。孝文帝既立,乃复封哙他庶子市人为舞阳侯,复故爵邑。市人立二十九岁卒,谥为荒侯。子他广代侯。六岁,侯家舍人得罪他广,怨之,乃上书曰:“荒侯市人病不能为人,令其夫人与其弟乱而生他广,他广实非荒侯子,不当代后。”诏下吏。孝景中六年,他广夺侯为庶人,国除。

曲周侯郦商者,高阳人。陈胜起时,商聚少年东西略人,得数千。沛公略地至陈留,六月馀,商以将卒四千人属沛公于岐。从攻长社,先登,赐爵封信成君。从沛公攻缑氏,绝河津,破秦军洛阳东。从攻下宛、穰,定十七县。别将攻旬关,定汉中。

项羽灭秦,立沛公为汉王。汉王赐商爵信成君,以将军为陇西都尉。别将定北地、上郡。破雍将军焉氏,周类军栒邑,苏駔军于泥阳。赐食邑武成六千户。以陇西都尉从击项籍军五月,出钜野,与锺离眛战,疾斗,受梁相国印,益食邑四千户。以梁相国将从击项羽二岁三月,攻胡陵。

项羽既已死,汉王为帝。其秋,燕王臧荼反,商以将军从击荼,战龙脱,先登陷阵,破荼军易下,卻敌,迁为右丞相,赐爵列侯,与诸侯剖符,世世勿绝,食邑涿五千户,号曰涿侯。以右丞相别定上谷,因攻代,受赵相国印。以右丞相赵相国别与绛侯等定代、雁门,得代丞相程纵、守相郭同、将军已下至六百石十九人。还,以将军为太上皇卫一岁七月。以右丞相击陈豨,残东垣。又以右丞相从高帝击黥布,攻其前拒,陷两陈,得以破布军,更食曲周五千一百户,除前所食,凡别破军三,降定郡六,县七十三,得丞相、守相、大将各一人,小将二人,二千石已下至六百石十九人。

商事孝惠、高后时,商病,不治。其子寄,字况,与吕禄善。及高后崩,大臣欲诛诸吕,吕禄为将军,军于北军,太尉勃不得入北军,于是乃使人劫郦商,令其子况绐吕禄,吕禄信之,故与出游,而太尉勃乃得入据北军,遂诛诸吕。是岁商卒,谥为景侯。子寄代侯。天下称郦况卖交也。

孝景前三年,吴、楚、齐、赵反,上以寄为将军,围赵城,十月不能下。得俞侯栾布自平齐来,乃下赵城,灭赵,王自杀,除国。孝景中二年,寄欲取平原君为夫人,景帝怒,下寄吏,有罪,夺侯。景帝乃以商他子坚封为缪侯,续郦氏后。缪靖侯卒,子康侯遂成立。遂成卒,子怀侯世宗立。世宗卒,子侯终根立,为太常,坐法,国除。

汝阴侯夏侯婴,沛人也。为沛厩司御。每送使客还,过沛泗上亭,与高祖语,未尝不移日也。婴已而试补县吏,与高祖相爱。高祖戏而伤婴,人有告高祖。高祖时为亭长,重坐伤人,告故不伤婴,婴证之。后狱覆,婴坐高祖系岁馀,掠笞数百,终以是脱高祖。

高祖之初与徒属欲攻沛也,婴时以县令史为高祖使。上降沛一日,高祖为沛公,赐婴爵七大夫,以为太仆。从攻胡陵,婴与萧何降泗水监平,平以胡陵降,赐婴爵五大夫。从击秦军砀东,攻济阳,下户牖,破李由军雍丘下,以兵车趣攻战疾,赐爵执帛。常以太仆奉车从击章邯军东阿、濮阳下,以兵车趣攻战疾,破之,赐爵执珪。复常奉车从击赵贲军开封,杨熊军曲遇。婴从捕虏六十八人,降卒八百五十人,得印一匮。因复常奉车从击秦军雒阳东,以兵车趣攻战疾,赐爵封转为滕公。因复奉车从攻南阳,战于蓝田、芷阳,以兵车趣攻战疾,至霸上。项羽至,灭秦,立沛公为汉王。汉王赐婴爵列侯,号昭平侯,复为太仆,从入蜀、汉。

还定三秦,从击项籍。至彭城,项羽大破汉军。汉王败,不利,驰去。见孝惠、鲁元,载之。汉王急,马罢,虏在后,常蹶两兒欲弃之,婴常收,竟载之,徐行面雍树乃驰。汉王怒,行欲斩婴者十馀,卒得脱,而致孝惠、鲁元于丰。

汉王既至荥阳,收散兵,复振,赐婴食祈阳。复常奉车从击项籍,追至陈,卒定楚,至鲁,益食兹氏。

汉王立为帝。其秋,燕王臧荼反,婴以太仆从击荼。明年,从至陈,取楚王信。更食汝阴,剖符世世勿绝。以太仆从击代,至武泉、云中,益食千户。因从击韩信军胡骑晋阳旁,大破之。追北至平城,为胡所围,七日不得通。高帝使使厚遗阏氏,冒顿开围一角。高帝出欲驰,婴固徐行,弩皆持满外向,卒得脱。益食婴细阳千户。复以太仆从击胡骑句注北,大破之。以太仆击胡骑平城南,三陷陈,功为多,赐所夺邑五百户。以太仆击陈豨、黥布军,陷陈卻敌,益食千户,定食汝阴六千九百户,除前所食。

婴自上初起沛,常为太仆,竟高祖崩。以太仆事孝惠。孝惠帝及高后德婴之脱孝惠、鲁元于下邑之间也,乃赐婴县北第第一,曰“近我”,以尊异之。孝惠帝崩,以太仆事高后。高后崩,代王之来,婴以太仆与东牟侯入清宫,废少帝,以天子法驾迎代王代邸,与大臣共立为孝文皇帝,复为太仆。八岁卒,谥为文侯。子夷侯灶立,七年卒。子共侯赐立,三十一年卒。子侯颇尚平阳公主。立十九岁,元鼎二年,坐与父御婢奸罪,自杀,国除。

颍阴侯灌婴者,睢阳贩缯者也。高祖之为沛公,略地至雍丘下,章邯败杀项梁,而沛公还军于砀,婴初以中涓从击破东郡尉于成武及秦军于扛里,疾斗,赐爵七大夫。从攻秦军亳南、开封、曲遇,战疾力,赐爵执帛,号宣陵君。从攻阳武以西至雒阳,破秦军尸北,北绝河津,南破南阳守齮阳城东,遂定南阳郡。西入武关,战于蓝田,疾力,至霸上,赐爵执珪,号昌文君。

沛公立为汉王,拜婴为郎中,从入汉中,十月,拜为中谒者。从还定三秦,下栎阳,降塞王。还围章邯于废丘,未拔。从东出临晋关,击降殷王,定其地。击项羽将龙且、魏相项他军定陶南,疾战,破之。赐婴爵列侯,号昌文侯,食杜平乡。

复以中谒者从降下砀,以至彭城。项羽击,大破汉王。汉王遁而西,婴从还,军于雍丘。王武、魏公申徒反,从击破之。攻下黄,西收兵,军于荥阳。楚骑来众,汉王乃择军中可为骑将者,皆推故秦骑士重泉人李必、骆甲习骑兵,今为校尉,可为骑将。汉王欲拜之,必、甲曰:“臣故秦民,恐军不信臣,臣原得大王左右善骑者傅之。”灌婴虽少,然数力战,乃拜灌婴为中大夫,令李必、骆甲为左右校尉,将郎中骑兵击楚骑于荥阳东,大破之。受诏别击楚军后,绝其饷道,起阳武至襄邑。击项羽之将项冠于鲁下,破之,所将卒斩右司马、骑将各一人。击破柘公王武,军于燕西,所将卒斩楼烦将五人,连尹一人。击王武别将桓婴白马下,破之,所将卒斩都尉一人。以骑渡河南,送汉王到雒阳,使北迎相国韩信军于邯郸。还至敖仓,婴迁为御史大夫。

三年,以列侯食邑杜平乡。以御史大夫受诏将郎中骑兵东属相国韩信,击破齐军于历下,所将卒虏车骑将军华毋伤及将吏四十六人。降下临菑,得齐守相田光。追齐相田横至嬴、博,破其骑,所将卒斩骑将一人,生得骑将四人。攻下嬴、博,破齐将军田吸于千乘,所将卒斩吸。东从韩信攻龙且、留公旋于高密,卒斩龙且,生得右司马、连尹各一人,楼烦将十人,身生得亚将周兰。

齐地已定,韩信自立为齐王,使婴别将击楚将公杲于鲁北,破之。转南,破薛郡长,身虏骑将一人。攻阳,前至下相以东南僮、取虑、徐。度淮,尽降其城邑,至广陵。项羽使项声、薛公、郯公复定淮北。婴度淮北,击破项声、郯公下邳,斩薛公,下下邳,击破楚骑于平阳,遂降彭城,虏柱国项佗,降留、薛、沛、酂、萧、相。攻苦、谯,复得亚将周兰。与汉王会颐乡。从击项籍军于陈下,破之,所将卒斩楼烦将二人,虏骑将八人。赐益食邑二千五百户。

项籍败垓下去也,婴以御史大夫受诏将车骑别追项籍至东城,之。所将卒五人共斩项籍,皆赐爵列侯。降左右司马各一人,卒万二千人,尽得其军将吏。下东城、历阳。渡江,破吴郡长吴下,得吴守,遂定吴、豫章、会稽郡。还定淮北,凡五十二县。

汉王立为皇帝,赐益婴邑三千户。其秋,以车骑将军从击破燕王臧荼。明年,从至陈,取楚王信。还,剖符,世世勿绝,食颍阴二千五百户,号曰颍阴侯。

以车骑将军从击反韩王信于代,至马邑,受诏别降楼烦以北六县,斩代左相,破胡骑于武泉北。复从击韩信胡骑晋阳下,所将卒斩胡白题将一人。受诏并将燕、赵、齐、梁、楚车骑,击破胡骑于硰石。至平城,为胡所围,从还军东垣。

从击陈豨,受诏别攻豨丞相侯敞军曲逆下,破之,卒斩敞及特将五人。降曲逆、卢奴、上曲阳、安国、安平。攻下东垣。

黥布反,以车骑将军先出,攻布别将于相,破之,斩亚将楼烦将三人。又进击破布上柱国军及大司马军。又进破布别将肥诛。婴身生得左司马一人,所将卒斩其小将十人,追北至淮上。益食二千五百户。布已破,高帝归,定令婴食颖阴五千户,除前所食邑。凡从得二千石二人,别破军十六,降城四十六,定国一,郡二,县五十二,得将军二人,柱国、相国各一人,二千石十人。

婴自破布归,高帝崩,婴以列侯事孝惠帝及吕太后。太后崩,吕禄等以赵王自置为将军,军长安,为乱。齐哀王闻之,举兵西,且入诛不当为王者。上将军吕禄等闻之,乃遣婴为大将,将军往击之。婴行至荥阳,乃与绛侯等谋,因屯兵荥阳,风齐王以诛吕氏事,齐兵止不前。绛侯等既诛诸吕,齐王罢兵归,婴亦罢兵自荥阳归,与绛侯、陈平共立代王为孝文皇帝。孝文皇帝于是益封婴三千户,赐黄金千斤,拜为太尉。

三岁,绛侯勃免相就国,婴为丞相,罢太尉官。是岁,匈奴大入北地、上郡,令丞相婴将骑八万五千往击匈奴。匈奴去,济北王反,诏乃罢婴之兵。后岁馀,婴以丞相卒,谥曰懿侯。子平侯阿代侯。二十八年卒,子彊代侯。十三年,彊有罪,绝二岁。元光三年,天子封灌婴孙贤为临汝侯,续灌氏后,八岁,坐行赇有罪,国除。

太史公曰:吾適丰沛,问其遗老,观故萧、曹、樊哙、滕公之家,及其素,异哉所闻!方其鼓刀屠狗卖缯之时,岂自知附骥之尾,垂名汉廷,德流子孙哉?余与他广通,为言高祖功臣之兴时若此云。

圣贤影响,云蒸龙变。屠狗贩缯,攻城野战。扶义西上,受封南面。郦况卖交,舞阳内援。滕灌更王,奕叶繁衍。
\end{yuanwen}

\chapter{张丞相列传}

\begin{yuanwen}
张丞相苍者,阳武人也。好书律历。秦时为御史,主柱下方书。有罪,亡归。及沛公略地过阳武,苍以客从攻南阳。苍坐法当斩,解衣伏质,身长大,肥白如瓠,时王陵见而怪其美士,乃言沛公,赦勿斩。遂从西入武关,至咸阳。沛公立为汉王,入汉中,还定三秦。陈馀击走常山王张耳,耳归汉,汉乃以张苍为常山守。从淮阴侯击赵,苍得陈馀。赵地已平,汉王以苍为代相,备边寇。已而徙为赵相,相赵王耳。耳卒,相赵王敖。复徙相代王。燕王臧荼反,高祖往击之。苍以代相从攻臧荼有功,以六年中封为北平侯,食邑千二百户。

迁为计相,一月,更以列侯为主计四岁。是时萧何为相国,而张苍乃自秦时为柱下史,明习天下图书计籍。苍又善用算律历,故令苍以列侯居相府,领主郡国上计者。黥布反亡,汉立皇子长为淮南王,而张苍相之。十四年,迁为御史大夫。

周昌者,沛人也。其从兄曰周苛,秦时皆为泗水卒史。及高祖起沛,击破泗水守监,于是周昌、周苛自卒史从沛公,沛公以周昌为职志,周苛为客。从入关,破秦。沛公立为汉王,以周苛为御史大夫,周昌为中尉。

汉王四年,楚围汉王荥阳急,汉王遁出去,而使周苛守荥阳城。楚破荥阳城,欲令周苛将。苛骂曰:“若趣降汉王!不然,今为虏矣!”项羽怒,亨周苛。于是乃拜周昌为御史大夫。常从击破项籍。以六年中与萧、曹等俱封:封周昌为汾阴侯;周苛子周成以父死事,封为高景侯。

昌为人彊力,敢直言,自萧、曹等皆卑下之。昌尝燕时入奏事,高帝方拥戚姬,昌还走,高帝逐得,骑周昌项,问曰:“我何如主也?”昌仰曰:“陛下即桀纣之主也。”于是上笑之,然尤惮周昌。及帝欲废太子,而立戚姬子如意为太子,大臣固争之,莫能得;上以留侯策即止。而周昌廷争之彊,上问其说,昌为人吃,又盛怒,曰:“臣口不能言,然臣期期知其不可。陛下虽欲废太子,臣期期不奉诏。”上欣然而笑。既罢,吕后侧耳于东箱听,见周昌,为跪谢曰:“微君,太子几废。”

是后戚姬子如意为赵王,年十岁,高祖忧即万岁之后不全也。赵尧年少,为符玺御史。赵人方与公谓御史大夫周昌曰:“君之史赵尧,年虽少,然奇才也,君必异之,是且代君之位。”周昌笑曰;“尧年少,刀笔吏耳,何能至是乎!”居顷之,赵尧侍高祖。高祖独心不乐,悲歌,群臣不知上之所以然。赵尧进请问曰:“陛下所为不乐,非为赵王年少而戚夫人与吕后有卻邪?备万岁之后而赵王不能自全乎?”高祖曰:“然。吾私忧之,不知所出。”尧曰:“陛下独宜为赵王置贵彊相,及吕后、太子、群臣素所敬惮乃可。”高祖曰:“然。吾念之欲如是,而群臣谁可者?”尧曰:“御史大夫周昌,其人坚忍质直,且自吕后、太子及大臣皆素敬惮之。独昌可。”高祖曰:“善。”于是乃召周昌,谓曰:“吾欲固烦公,公彊为我相赵王。”周昌泣曰:“臣初起从陛下,陛下独柰何中道而弃之于诸侯乎?”高祖曰:“吾极知其左迁,然吾私忧赵王,念非公无可者。公不得已彊行!”于是徙御史大夫周昌为赵相。

既行久之,高祖持御史大夫印弄之,曰:“谁可以为御史大夫者?”孰视赵尧,曰:“无以易尧。”遂拜赵尧为御史大夫。尧亦前有军功食邑,及以御史大夫从击陈豨有功,封为江邑侯。

高祖崩,吕太后使使召赵王,其相周昌令王称疾不行。使者三反,周昌固为不遣赵王。于是高后患之,乃使使召周昌。周昌至,谒高后,高后怒而骂周昌曰:“尔不知我之怨戚氏乎?而不遣赵王,何?”昌既徵,高后使使召赵王,赵王果来。至长安月馀,饮药而死。周昌因谢病不朝见,三岁而死。

后五岁,高后闻御史大夫江邑侯赵尧高祖时定赵王如意之画,乃抵尧罪,以广阿侯任敖为御史大夫。

任敖者,故沛狱吏。高祖尝辟吏,吏系吕后,遇之不谨。任敖素善高祖,怒,击伤主吕后吏。及高祖初起,敖以客从为御史,守丰二岁,高祖立为汉王,东击项籍,敖迁为上党守。陈豨反时,敖坚守,封为广阿侯,食千八百户。高后时为御史大夫。三岁免,以平阳侯曹窋为御史大夫。高后崩,与大臣共诛吕禄等。免,以淮南相张苍为御史大夫。

苍与绛侯等尊立代王为孝文皇帝。四年,丞相灌婴卒,张苍为丞相。

自汉兴至孝文二十馀年,会天下初定,将相公卿皆军吏。张苍为计相时,绪正律历。以高祖十月始至霸上,因故秦时本以十月为岁首,弗革。推五德之运,以为汉当水德之时,尚黑如故。吹律调乐,入之音声,及以比定律令。若百工,天下作程品。至于为丞相,卒就之,故汉家言律历者,本之张苍。苍本好书,无所不观,无所不通,而尤善律历。

张苍德王陵。王陵者,安国侯也。及苍贵,常父事王陵。陵死后,苍为丞相,洗沐,常先朝陵夫人上食,然后敢归家。

苍为丞相十馀年,鲁人公孙臣上书言汉土德时,其符有黄龙当见。诏下其议张苍,张苍以为非是,罢之。其后黄龙见成纪,于是文帝召公孙臣以为博士,草土德之历制度,更元年。张丞相由此自绌,谢病称老。苍任人为中候,大为奸利,上以让苍,苍遂病免。苍为丞相十五岁而免。孝景前五年,苍卒,谥为文侯。子康侯代,八年卒。子类代为侯,八年,坐临诸侯丧后就位不敬,国除。

初,张苍父长不满五尺,及生苍,苍长八尺馀,为侯、丞相。苍子复长。及孙类,长六尺馀,坐法失侯。苍之免相后,老,口中无齿,食乳,女子为乳母。妻妾以百数,尝孕者不复幸。苍年百有馀岁而卒。

申屠丞相嘉者,梁人,以材官蹶张从高帝击项籍,迁为队率。从击黥布军,为都尉。孝惠时,为淮阳守。孝文帝元年,举故吏士二千石从高皇帝者,悉以为关内侯,食邑二十四人,而申屠嘉食邑五百户。张苍已为丞相,嘉迁为御史大夫。张苍免相,孝文帝欲用皇后弟窦广国为丞相,曰:“恐天下以吾私广国。”广国贤有行,故欲相之,念久之不可,而高帝时大臣又皆多死,馀见无可者,乃以御史大夫嘉为丞相,因故邑封为故安侯。

嘉为人廉直,门不受私谒。是时太中大夫邓通方隆爱幸,赏赐累巨万。文帝尝燕饮通家,其宠如是。是时丞相入朝,而通居上傍,有怠慢之礼。丞相奏事毕,因言曰:“陛下爱幸臣,则富贵之;至于朝廷之礼,不可以不肃!”上曰:“君勿言,吾私之。”罢朝坐府中,嘉为檄召邓通诣丞相府,不来,且斩通。通恐,入言文帝。文帝曰:“汝第往,吾今使人召若。”通至丞相府,免冠,徒跣,顿首谢。嘉坐自如,故不为礼,责曰:“夫朝廷者,高皇帝之朝廷也。通小臣,戏殿上,大不敬,当斩。吏今行斩之!”通顿首,首尽出血,不解。文帝度丞相已困通,使使者持节召通,而谢丞相曰:“此吾弄臣,君释之。”邓通既至,为文帝泣曰:“丞相几杀臣。”

嘉为丞相五岁,孝文帝崩,孝景帝即位。二年,晁错为内史,贵幸用事,诸法令多所请变更,议以谪罚侵削诸侯。而丞相嘉自绌所言不用,疾错。错为内史,门东出,不便,更穿一门南出。南出者,太上皇庙堧垣。嘉闻之,欲因此以法错擅穿宗庙垣为门,奏请诛错。错客有语错,错恐,夜入宫上谒,自归景帝。至朝,丞相奏请诛内史错。景帝曰:“错所穿非真庙垣,乃外堧垣,故他官居其中,且又我使为之,错无罪。”罢朝,嘉谓长史曰:“吾悔不先斩错,乃先请之,为错所卖。”至舍,因欧血而死。谥为节侯。子共侯蔑代,三年卒。子侯去病代,三十一年卒。子侯臾代,六岁,坐为九江太守受故官送有罪,国除。

自申屠嘉死之后,景帝时开封侯陶青、桃侯刘舍为丞相。及今上时,柏至侯许昌、平棘侯薛泽、武彊侯庄青翟、高陵侯赵周等为丞相。皆以列侯继嗣,娖娖廉谨,为丞相备员而已,无所能发明功名有著于当世者。

太史公曰:“张苍文学律历,为汉名相,而绌贾生、公孙臣等言正朔服色事而不遵,明用秦之颛顼历,何哉?周昌,木彊人也。任敖以旧德用。申屠嘉可谓刚毅守节矣,然无术学,殆与萧、曹、陈平异矣。

孝武时丞相多甚,不记,莫录其行起居状略,且纪征和以来。

有车丞相,长陵人也。卒而有韦丞相代。韦丞相贤者,鲁人也。以读书术为吏,至大鸿胪。有相工相之,当至丞相。有男四人,使相工相之,至第二子,其名玄成。相工曰:“此子贵,当封。”韦丞相言曰:“我即为丞相,有长子,是安从得之?”后竟为丞相,病死,而长子有罪论,不得嗣,而立玄成。玄成时佯狂,不肯立,竟立之,有让国之名。后坐骑至庙,不敬,有诏夺爵一级,为关内侯,失列侯,得食其故国邑。韦丞相卒,有魏丞相代。

魏丞相相者,济阴人也。以文吏至丞相。其人好武,皆令诸吏带剑,带剑前奏事。或有不带剑者,当入奏事,至乃借剑而敢入奏事。其时京兆尹赵君,丞相奏以免罪,使人执魏丞相,欲求脱罪而不听。复使人胁恐魏丞相,以夫人贼杀待婢事而私独奏请验之,发吏卒至丞相舍,捕奴婢笞击问之,实不以兵刃杀也。而丞相司直繁君奏京兆尹赵君迫胁丞相,诬以夫人贼杀婢,发吏卒围捕丞相舍,不道;又得擅屏骑士事,赵京兆坐要斩。又有使掾陈平等劾中尚书,疑以独擅劫事而坐之,大不敬,长史以下皆坐死,或下蚕室。而魏丞相竟以丞相病死。子嗣。后坐骑至庙,不敬,有诏夺爵一级,为关内侯,失列侯,得食其故国邑。魏丞相卒,以御史大夫邴吉代。

邴丞相吉者,鲁国人也。以读书好法令至御史大夫。孝宣帝时,以有旧故,封为列侯,而因为丞相。明于事,有大智,后世称之。以丞相病死。子显嗣。后坐骑至庙,不敬,有诏夺爵一级,失列侯,得食故国邑。显为吏至太仆,坐官毛乱,身及子男有奸赃,免为庶人。

邴丞相卒,黄丞相代。长安中有善相工田文者,与韦丞相、魏丞相、邴丞相微贱时会于客家,田文言曰:“今此三君者,皆丞相也。”其后三人竟更相代为丞相,何见之明也。

黄丞相霸者,淮阳人也。以读书为吏,至颍川太守。治颍川,以礼义条教喻告化之。犯法者,风晓令自杀。化大行,名声闻。孝宣帝下制曰:“颍川太守霸,以宣布诏令治民,道不拾遗,男女异路,狱中无重囚。赐爵关内侯,黄金百斤。”徵为京兆尹而至丞相,复以礼义为治。以丞相病死。子嗣,后为列侯。黄丞相卒,以御史大夫于定国代。于丞相已有廷尉传,在张廷尉语中。于丞相去,御史大夫韦玄成代。

韦丞相玄成者,即前韦丞相子也。代父,后失列侯。其人少时好读书,明于诗、论语。为吏至卫尉,徙为太子太傅。御史大夫薛君免,为御史大夫。于丞相乞骸骨免,而为丞相,因封故邑为扶阳侯。数年,病死。孝元帝亲临丧,赐赏甚厚。子嗣后。其治容容随世俗浮沈,而见谓谄巧。而相工本谓之当为侯代父,而后失之;复自游宦而起,至丞相。父子俱为丞相,世间美之,岂不命哉!相工其先知之。韦丞相卒,御史大夫匡衡代。

丞相匡衡者,东海人也。好读书,从博士受诗。家贫,衡佣作以给食饮。才下,数射策不中,至九,乃中丙科。其经以不中科故明习。补平原文学卒史。数年,郡不尊敬。御史徵之,以补百石属荐为郎,而补博士,拜为太子少傅,而事孝元帝。孝元好诗,而迁为光禄勋,居殿中为师,授教左右,而县官坐其旁听,甚善之,日以尊贵。御史大夫郑弘坐事免,而匡君为御史大夫。岁馀,韦丞相死,匡君代为丞相,封乐安侯。以十年之间,不出长安城门而至丞相,岂非遇时而命也哉!

太史公曰:深惟士之游宦所以至封侯者,微甚。然多至御史大夫即去者。诸为大夫而丞相次也,其心冀幸丞相物故也。或乃阴私相毁害,欲代之。然守之日久不得,或为之日少而得之,至于封侯,真命也夫!御史大夫郑君守之数年不得,匡君居之未满岁,而韦丞相死,即代之矣,岂可以智巧得哉!多有贤圣之才,困戹不得者众甚也。

张苍主计,天下作程。孙臣始绌,秦历尚行。御史亚相,相国阿衡。申屠面折,周子廷争。其他娖々,无所发明。
\end{yuanwen}

\chapter{郦生陆贾列传}

\begin{yuanwen}
郦生食其者,陈留高阳人也。好读书,家贫落魄,无以为衣食业,为里监门吏。然县中贤豪不敢役,县中皆谓之狂生。

及陈胜、项梁等起,诸将徇地过高阳者数十人,郦生闻其将皆握齱好苛礼自用,不能听大度之言,郦生乃深自藏匿。后闻沛公将兵略地陈留郊,沛公麾下骑士適郦生里中子也,沛公时时问邑中贤士豪俊。骑士归,郦生见谓之曰:“吾闻沛公慢而易人,多大略,此真吾所原从游,莫为我先。若见沛公,谓曰‘臣里中有郦生,年六十馀,长八尺,人皆谓之狂生,生自谓我非狂生’。”骑士曰:“沛公不好儒,诸客冠儒冠来者,沛公辄解其冠,溲溺其中。与人言,常大骂。未可以儒生说也。”郦生曰:“弟言之。”骑士从容言如郦生所诫者。

沛公至高阳传舍,使人召郦生。郦生至,入谒,沛公方倨床使两女子洗足,而见郦生。郦生入,则长揖不拜,曰:“足下欲助秦攻诸侯乎?且欲率诸侯破秦也?”沛公骂曰:“竖儒!夫天下同苦秦久矣,故诸侯相率而攻秦,何谓助秦攻诸侯乎?”郦生曰:“必聚徒合义兵诛无道秦,不宜倨见长者。”于是沛公辍洗,起摄衣,延郦生上坐,谢之。郦生因言六国从横时。沛公喜,赐郦生食,问曰:“计将安出?”郦生曰:“足下起纠合之众,收散乱之兵,不满万人,欲以径入强秦,此所谓探虎口者也。夫陈留,天下之旻,四通五达之郊也,今其城又多积粟。臣善其令,请得使之,令下足下。即不听,足下举兵攻之,臣为内应。”于是遣郦生行,沛公引兵随之,遂下陈留。号郦食其为广野君。

郦生言其弟郦商,使将数千人从沛公西南略地。郦生常为说客,驰使诸侯。

汉三年秋,项羽击汉,拔荥阳,汉兵遁保巩、洛。楚人闻淮阴侯破赵,彭越数反梁地,则分兵救之。淮阴方东击齐,汉王数困荥阳、成皋,计欲捐成皋以东,屯巩、洛以拒楚。郦生因曰:“臣闻知天之天者,王事可成;不知天之天者,王事不可成。王者以民人为天,而民人以食为天。夫敖仓,天下转输久矣,臣闻其下乃有藏粟甚多,楚人拔荥阳,不坚守敖仓,乃引而东,令適卒分守成皋,此乃天所以资汉也。方今楚易取而汉反郤,自夺其便,臣窃以为过矣。且两雄不俱立,楚汉久相持不决,百姓骚动,海内摇荡,农夫释耒,工女下机,天下之心未有所定也。原足下急复进兵,收取荥阳,据敖仓之粟,塞成皋之险,杜大行之道,距蜚狐之口,守白马之津,以示诸侯效实形制之势,则天下知所归矣。方今燕、赵已定,唯齐未下。今田广据千里之齐,田间将二十万之众,军于历城,诸田宗彊,负海阻河济,南近楚,人多变诈,足下虽遣数十万师,未可以岁月破也。臣请得奉明诏说齐王,使为汉而称东籓。”上曰:“善。”

乃从其画,复守敖仓,而使郦生说齐王曰:“王知天下之所归乎?”王曰:“不知也。”曰:“王知天下之所归,则齐国可得而有也;若不知天下之所归,即齐国未可得保也。”齐王曰:“天下何所归?”曰:“归汉。”曰:“先生何以言之?”曰:“汉王与项王戮力西面击秦,约先入咸阳者王之。汉王先入咸阳,项王负约不与而王之汉中。项王迁杀义帝,汉王闻之,起蜀汉之兵击三秦,出关而责义帝之处,收天下之兵,立诸侯之后。降城即以侯其将,得赂即以分其士,与天下同其利,豪英贤才皆乐为之用。诸侯之兵四面而至,蜀汉之粟方船而下。项王有倍约之名,杀义帝之负;于人之功无所记,于人之罪无所忘;战胜而不得其赏,拔城而不得其封;非项氏莫得用事;为人刻印,刓而不能授;攻城得赂,积而不能赏:天下畔之,贤才怨之,而莫为之用。故天下之士归于汉王,可坐而策也。夫汉王发蜀汉,定三秦;涉西河之外,援上党之兵;下井陉,诛成安君;破北魏,举三十二城:此蚩尤之兵也,非人之力也,天之福也。今已据敖仓之粟,塞成皋之险,守白马之津,杜大行之阪,距蜚狐之口,天下后服者先亡矣。王疾先下汉王,齐国社稷可得而保也;不下汉王,危亡可立而待也。”田广以为然,乃听郦生,罢历下兵守战备,与郦生日纵酒。

淮阴侯闻郦生伏轼下齐七十馀城,乃夜度兵平原袭齐。齐王田广闻汉兵至,以为郦生卖己,乃曰:“汝能止汉军,我活汝;不然,我将亨汝!”郦生曰:“举大事不细谨,盛德不辞让。而公不为若更言!”齐王遂亨郦生,引兵东走。

汉十二年,曲周侯郦商以丞相将兵击黥布有功。高祖举列侯功臣,思郦食其。郦食其子疥数将兵,功未当侯,上以其父故,封疥为高梁侯。后更食武遂,嗣三世。元狩元年中,武遂侯平坐诈诏衡山王取百斤金,当弃市,病死,国除也。

陆贾者,楚人也。以客从高祖定天下,名为有口辩士,居左右,常使诸侯。

及高祖时,中国初定,尉他平南越,因王之。高祖使陆贾赐尉他印为南越王。陆生至,尉他魋结箕倨见陆生。陆生因进说他曰:“足下中国人,亲戚昆弟坟在真定。今足下反天性,弃冠带,欲以区区之越与天子抗衡为敌国,祸且及身矣。且夫秦失其政,诸侯豪桀并起,唯汉王先入关,据咸阳。项羽倍约,自立为西楚霸王,诸侯皆属,可谓至彊。然汉王起巴蜀,鞭笞天下,劫略诸侯,遂诛项羽灭之。五年之间,海内平定,此非人力,天之所建也。天子闻君王王南越,不助天下诛暴逆,将相欲移兵而诛王,天子怜百姓新劳苦,故且休之,遣臣授君王印,剖符通使。君王宜郊迎,北面称臣,乃欲以新造未集之越,屈彊于此。汉诚闻之,掘烧王先人冢,夷灭宗族,使一偏将将十万众临越,则越杀王降汉,如反覆手耳。”

于是尉他乃蹶然起坐,谢陆生曰:“居蛮夷中久,殊失礼义。”因问陆生曰:“我孰与萧何、曹参、韩信贤?”陆生曰:“王似贤。”复曰:“我孰与皇帝贤?”陆生曰:“皇帝起丰沛,讨暴秦,诛彊楚,为天下兴利除害,继五帝三王之业,统理中国。中国之人以亿计,地方万里,居天下之膏腴,人众车轝,万物殷富,政由一家,自天地剖泮未始有也。今王众不过数十万,皆蛮夷,崎岖山海间,譬若汉一郡,王何乃比于汉!”尉他大笑曰:“吾不起中国,故王此。使我居中国,何渠不若汉?”乃大说陆生,留与饮数月。曰:“越中无足与语,至生来,令我日闻所不闻。”赐陆生橐中装直千金,他送亦千金。陆生卒拜尉他为南越王,令称臣奉汉约。归报,高祖大悦,拜贾为太中大夫。

陆生时时前说称诗书。高帝骂之曰:“乃公居马上而得之,安事诗书!”陆生曰;“居马上得之,宁可以马上治之乎?且汤武逆取而以顺守之,文武并用,长久之术也。昔者吴王夫差、智伯极武而亡;秦任刑法不变,卒灭赵氏。乡使秦已并天下,行仁义,法先圣,陛下安得而有之?”高帝不怿而有惭色,乃谓陆生曰:“试为我著秦所以失天下,吾所以得之者何,及古成败之国。”陆生乃粗述存亡之徵,凡著十二篇。每奏一篇,高帝未尝不称善,左右呼万岁,号其书曰“新语”。

孝惠帝时,吕太后用事,欲王诸吕,畏大臣有口者,陆生自度不能争之,乃病免家居。以好畤田地善,可以家焉。有五男,乃出所使越得橐中装卖千金,分其子,子二百金,令为生产。陆生常安车驷马,从歌舞鼓琴瑟侍者十人,宝剑直百金,谓其子曰:“与汝约:过汝,汝给吾人马酒食,极欲,十日而更。所死家,得宝剑车骑侍从者。一岁中往来过他客,率不过再三过,数见不鲜,无久慁公为也。”

吕太后时,王诸吕,诸吕擅权,欲劫少主,危刘氏。右丞相陈平患之,力不能争,恐祸及己,常燕居深念。陆生往请,直入坐,而陈丞相方深念,不时见陆生。陆生曰:“何念之深也?”陈平曰:“生揣我何念?”陆生曰:“足下位为上相,食三万户侯,可谓极富贵无欲矣。然有忧念,不过患诸吕、少主耳。”陈平曰:“然。为之柰何?”陆生曰:“天下安,注意相;天下危,注意将。将相和调,则士务附;士务附,天下虽有变,即权不分。为社稷计,在两君掌握耳。臣常欲谓太尉绛侯,绛侯与我戏,易吾言。君何不交驩太尉,深相结?”为陈平画吕氏数事。陈平用其计,乃以五百金为绛侯寿,厚具乐饮;太尉亦报如之。此两人深相结,则吕氏谋益衰。陈平乃以奴婢百人,车马五十乘,钱五百万,遗陆生为饮食费。陆生以此游汉廷公卿间,名声藉甚。

及诛诸吕,立孝文帝,陆生颇有力焉。孝文帝即位,欲使人之南越。陈丞相等乃言陆生为太中大夫,往使尉他,令尉他去黄屋称制,令比诸侯,皆如意旨。语在南越语中。陆生竟以寿终。

平原君硃建者,楚人也。故尝为淮南王黥布相,有罪去,后复事黥布。布欲反时,问平原君,平原君非之,布不听而听梁父侯,遂反。汉已诛布,闻平原君谏不与谋,得不诛。语在黥布语中。平原君为人辩有口,刻廉刚直,家于长安。行不苟合,义不取容。辟阳侯行不正,得幸吕太后。时辟阳侯欲知平原君,平原君不肯见。及平原君母死,陆生素与平原君善,过之。平原君家贫,未有以发丧,方假贷服具,陆生令平原君发丧。陆生往见辟阳侯,贺曰:“平原君母死。”辟阳侯曰:“平原君母死,何乃贺我乎?”陆贾曰:“前日君侯欲知平原君,平原君义不知君,以其母故。今其母死,君诚厚送丧,则彼为君死矣。”辟阳侯乃奉百金往税。列侯贵人以辟阳侯故,往税凡五百金。

辟阳侯幸吕太后,人或毁辟阳侯于孝惠帝,孝惠帝大怒,下吏,欲诛之。吕太后惭,不可以言。大臣多害辟阳侯行,欲遂诛之。辟阳侯急,因使人欲见平原君。平原君辞曰:“狱急,不敢见君。”乃求见孝惠幸臣闳籍孺,说之曰:“君所以得幸帝,天下莫不闻。今辟阳侯幸太后而下吏,道路皆言君谗,欲杀之。今日辟阳侯诛,旦日太后含怒,亦诛君。何不肉袒为辟阳侯言于帝?帝听君出辟阳侯,太后大驩。两主共幸君,君贵富益倍矣。”于是闳籍孺大恐,从其计,言帝,果出辟阳侯。辟阳侯之囚,欲见平原君,平原君不见辟阳侯,辟阳侯以为倍己,大怒。及其成功出之,乃大惊。

吕太后崩,大臣诛诸吕,辟阳侯于诸吕至深,而卒不诛。计画所以全者,皆陆生、平原君之力也。

孝文帝时,淮南厉王杀辟阳侯,以诸吕故。文帝闻其客平原君为计策,使吏捕欲治。闻吏至门,平原君欲自杀。诸子及吏皆曰:“事未可知,何早自杀为?”平原君曰:“我死祸绝,不及而身矣。”遂自刭。孝文帝闻而惜之,曰:“吾无意杀之。”乃召其子,拜为中大夫。使匈奴,单于无礼,乃骂单于,遂死匈奴中。

初,沛公引兵过陈留,郦生踵军门上谒曰:“高阳贱民郦食其,窃闻沛公暴露,将兵助楚讨不义,敬劳从者,原得望见,口画天下便事。”使者入通,沛公方洗,问使者曰:“何如人也?”使者对曰:“状貌类大儒,衣儒衣,冠侧注。”沛公曰:“为我谢之,言我方以天下为事,未暇见儒人也。”使者出谢曰:“沛公敬谢先生,方以天下为事,未暇见儒人也。”郦生瞋目案剑叱使者曰:“走!复入言沛公,吾高阳酒徒也,非儒人也。”使者惧而失谒,跪拾谒,还走,复入报曰:“客,天下壮士也,叱臣,臣恐,至失谒。曰‘走!复入言,而公高阳酒徒也’。”沛公遽雪足杖矛曰:“延客入!”

郦生入,揖沛公曰:“足下甚苦,暴衣露冠,将兵助楚讨不义,足不何不自喜也?臣原以事见,而曰‘吾方以天下为事,未暇见儒人也’。夫足下欲兴天下之大事而成天下之大功,而以目皮相,恐失天下之能士。且吾度足下之智不如吾,勇又不如吾。若欲就天下而不相见,窃为足下失之。”沛公谢曰:“乡者闻先生之容,今见先生之意矣。”乃延而坐之,问所以取天下者。郦生曰:“夫足下欲成大功,不如止陈留。陈留者,天下之据旻也,兵之会地也,积粟数千万石,城守甚坚。臣素善其令,原为足下说之。不听臣,臣请为足下杀之,而下陈留。足下将陈留之众,据陈留之城,而食其积粟,招天下之从兵;从兵已成,足下横行天下,莫能有害足下者矣。”沛公曰:“敬闻命矣。”

于是郦生乃夜见陈留令,说之曰:“夫秦为无道而天下畔之,今足下与天下从则可以成大功。今独为亡秦婴城而坚守,臣窃为足下危之。”陈留令曰:“秦法至重也,不可以妄言,妄言者无类,吾不可以应。先生所以教臣者,非臣之意也,原勿复道。”郦生留宿卧,夜半时斩陈留令首,逾城而下报沛公。沛公引兵攻城,县令首于长竿以示城上人,曰:“趣下,而令头已断矣!今后下者必先斩之!”于是陈留人见令已死,遂相率而下沛公。沛公舍陈留南城门上,因其库兵,食积粟,留出入三月,从兵以万数,遂入破秦。

太史公曰:世之传郦生书,多曰汉王已拔三秦,东击项籍而引军于巩洛之间,郦生被儒衣往说汉王。乃非也。自沛公未入关,与项羽别而至高阳,得郦生兄弟。余读陆生新语书十二篇,固当世之辩士。至平原君子与余善,是以得具论之。

广野大度,始冠侧注。踵门长揖,深器重遇。说齐历下,趣鼎何惧。陆贾使越,尉佗慑怖,相说国安,书成主悟。
\end{yuanwen}

\part{卷九十八}
\chapter{傅靳蒯成列传第三十八}

\begin{yuanwen}
阳陵侯傅宽,以魏五大夫骑将从,为舍人\footnote{官僚贵族门下的客卿。},起横阳。从攻安阳、杠里,击赵贲军于开封,及击杨熊曲遇、阳武,斩首十二级,赐爵卿。从至霸上。沛公立为汉王,汉王赐宽封号共德君。从入汉中,迁为右骑将。从定三秦,赐食邑雕阴。从击项籍,待怀,赐爵通德侯。从击项冠、周兰、龙且,所将卒斩骑将一人敖下\footnote{敖仓附近。},益食邑。
\end{yuanwen}

阳陵侯傅宽,以魏五大夫的爵位加骑将的身份跟随沛公,当上了侍从官,在横阳起兵。他跟随沛公攻打安阳、杠里,在开封攻打赵贲的军队,又在曲遇、阳武一带攻打杨熊的军队,斩下十二个敌人的首级,沛公赐予他卿的爵位。后来他跟随沛公来到霸上。沛公被立为汉王之后,赐予傅宽共德君的封号。傅宽跟随汉王一同进入汉中,官职也升为右骑将。然后他跟随汉王回师平定三秦,汉王将雕阴县赐给他作为食邑。然后他又跟随汉王攻打项籍,奉命在怀县接应汉王,汉王赐予他通德侯的爵位。他又跟随汉王攻打项冠、周兰、龙且,他所率领的士卒在敖仓斩杀敌军骑将一人,于是汉王为他增加了食邑。

\begin{yuanwen}
属淮阴\footnote{跟随淮阴侯韩信。},击破齐历下军,击田解。属相国参,残\footnote{屠杀,屠戮。}博,益食邑。因定齐地,剖符世世勿绝,封为阳陵侯,二千六百户,除前所食。为齐右丞相,备齐。五岁为齐相国。
\end{yuanwen}

傅宽曾经隶属于淮阴侯韩信,在此期间曾击败齐国在历下的驻军,并追击齐国的田解。他在隶属于相国曹参的时候,一举摧毁了齐国的博县,并因此增加了食邑。由于他平定齐地有功,所以分剖符节受了封,世代相传永不断绝,他被封为阳陵侯,拥有二千六百户的食邑,过去的食邑随之撤销。此后他担任齐王韩信的右丞相,守备齐地。五年以后,他又做了齐王刘肥的相国。

\begin{yuanwen}
四月,击陈豨,属太尉勃,以相国代丞相哙击豨。一月,徙为代相国,将屯\footnote{带领屯田士兵。}。二岁,为代丞相,将屯。
\end{yuanwen}

四月,朝廷发兵平定叛臣陈豨,当时傅宽归太尉周勃管,以齐国相国的身份接替丞相樊哙征讨陈豨。次年一月,他调任代国相国,率领军队屯守边疆。两年以后,他被任命为代国丞相,依然率军屯守边疆。

\begin{yuanwen}
孝惠五年,卒,谥为景侯。子顷侯精立,二十四年卒。子共侯则立,十二年卒。子侯偃立,三十一年,坐与淮南王谋反,死,国除。
\end{yuanwen}

孝惠帝五年(前190年),傅宽去世,死后谥号为景侯。他的儿子顷侯傅精继承侯位,在位二十四年去世。傅精的儿子共侯傅则继承侯位,十二年后去世。此后,由傅则的儿子傅偃继承侯位,三十一年后,他由于参与淮南王刘安谋反之事,被处死,其封国也被废除。

\begin{yuanwen}
信武侯靳歙,以中涓从,起宛朐。攻济阳。破李由军。击秦军亳南、开封东北,斩骑千人将一人,首五十七级,捕虏七十三人,赐爵封号临平君。又战蓝田北,斩车司马二人,骑长一人,首二十八级,捕虏五十七人。至霸上。沛公立为汉王,赐歙爵建武侯,迁为骑都尉。
\end{yuanwen}\begin{yuanwen}
	
\end{yuanwen}\begin{yuanwen}
	
\end{yuanwen}\begin{yuanwen}
	
\end{yuanwen}\begin{yuanwen}
	
\end{yuanwen}\begin{yuanwen}
	
\end{yuanwen}\begin{yuanwen}
	
\end{yuanwen}\begin{yuanwen}
	
\end{yuanwen}\begin{yuanwen}
	
\end{yuanwen}\begin{yuanwen}
	
\end{yuanwen}\begin{yuanwen}
	
\end{yuanwen}\begin{yuanwen}
	
\end{yuanwen}\begin{yuanwen}
	
\end{yuanwen}\begin{yuanwen}
	
\end{yuanwen}\begin{yuanwen}
	
\end{yuanwen}\begin{yuanwen}
	
\end{yuanwen}\begin{yuanwen}
	
\end{yuanwen}\begin{yuanwen}
	
\end{yuanwen}\begin{yuanwen}
	
\end{yuanwen}\begin{yuanwen}
	
\end{yuanwen}\begin{yuanwen}
	
\end{yuanwen}\begin{yuanwen}
	
\end{yuanwen}\begin{yuanwen}
	
	
	
	
	
	
\end{yuanwen}
\begin{yuanwen}


从定三秦。别西击章平军于陇西,破之,定陇西六县,所将卒斩车司马、候各四人,骑长十二人。从东击楚,至彭城。汉军败还,保雍丘,去击反者王武等。略梁地,别将击邢说军菑南,破之,身得说都尉二人,司马、候十二人,降吏卒四千一百八十人。破楚军荥阳东。三年,赐食邑四千二百户。

别之河内,击赵将贲郝军朝歌,破之,所将卒得骑将二人,车马二百五十匹。从攻安阳以东,至棘蒲,下七县。别攻破赵军,得其将司马二人,候四人,降吏卒二千四百人。从攻下邯郸。别下平阳,身斩守相,所将卒斩兵守、郡守各一人,降鄴。从攻朝歌、邯郸,及别击破赵军,降邯郸郡六县。还军敖仓,破项籍军成皋南,击绝楚饟道,起荥阳至襄邑。破项冠军鲁下。略地东至缯、郯、下邳,南至蕲、竹邑。击项悍济阳下。还击项籍陈下,破之。别定江陵,降江陵柱国、大司马以下八人,身得江陵王,生致之雒阳,因定南郡。从至陈,取楚王信,剖符世世勿绝,定食四千六百户,号信武侯。

以骑都尉从击代,攻韩信平城下,还军东垣。有功,迁为车骑将军,并将梁、赵、齐、燕、楚车骑,别击陈豨丞相敞,破之,因降曲逆。从击黥布有功,益封定食五千三百户。凡斩首九十级,虏百三十二人;别破军十四,降城五十九,定郡、国各一,县二十三;得王、柱国各一人,二千石以下至五百石三十九人。

高后五年,歙卒,谥为肃侯。子亭代侯。二十一年,坐事国人过律,孝文后三年,夺侯,国除。

蒯成侯緤者,沛人也,姓周氏。常为高祖参乘,以舍人从起沛。至霸上,西入蜀、汉,还定三秦,食邑池阳。东绝甬道,从出度平阴,遇淮阴侯兵襄国,军乍利乍不利,终无离上心。以緤为信武侯,食邑三千三百户。高祖十二年,以緤为蒯成侯,除前所食邑。

上欲自击陈豨,蒯成侯泣曰:“始秦攻破天下,未尝自行。今上常自行,是为无人可使者乎?”上以为“爱我”,赐入殿门不趋,杀人不死。

至孝文五年,緤以寿终,谥为贞侯。子昌代侯,有罪,国除。至孝景中二年,封緤子居代侯。至元鼎三年,居为太常,有罪,国除。

太史公曰:阳陵侯傅宽、信武侯靳歙皆高爵,从高祖起山东,攻项籍,诛杀名将,破军降城以十数,未尝困辱,此亦天授也。蒯成侯周緤操心坚正,身不见疑,上欲有所之,未尝不垂涕,此有伤心者然,可谓笃厚君子矣。

阳陵、信武,结发从汉。动叶人谋,功实天赞。定齐破项,我军常冠,蒯成委质,夷险不乱。主上称忠,人臣鸧腕。
\end{yuanwen}

\chapter{刘敬叔孙通列传}

\begin{yuanwen}
刘敬者,齐人也。汉五年,戍陇西,过洛阳,高帝在焉。娄敬脱輓辂,衣其羊裘,见齐人虞将军曰:“臣原见上言便事。”虞将军欲与之鲜衣,娄敬曰:“臣衣帛,衣帛见;衣褐,衣褐见:终不敢易衣。”于是虞将军入言上。上召入见,赐食。

已而问娄敬,娄敬说曰:“陛下都洛阳,岂欲与周室比隆哉?”上曰:“然。”娄敬曰:“陛下取天下与周室异。周之先自后稷,尧封之邰,积德累善十有馀世。公刘避桀居豳。太王以狄伐故,去豳,杖马箠居岐,国人争随之。及文王为西伯,断虞芮之讼,始受命,吕望、伯夷自海滨来归之。武王伐纣,不期而会孟津之上八百诸侯,皆曰纣可伐矣,遂灭殷。成王即位,周公之属傅相焉,乃营成周洛邑,以此为天下之中也,诸侯四方纳贡职,道里均矣,有德则易以王,无德则易以亡。凡居此者,欲令周务以德致人,不欲依阻险,令后世骄奢以虐民也。及周之盛时,天下和洽,四夷乡风,慕义怀德,附离而并事天子,不屯一卒,不战一士,八夷大国之民莫不宾服,效其贡职。及周之衰也,分而为两,天下莫朝,周不能制也。非其德薄也,而形势弱也。今陛下起丰沛,收卒三千人,以之径往而卷蜀汉,定三秦,与项羽战荥阳,争成皋之口,大战七十,小战四十,使天下之民肝脑涂地,父子暴骨中野,不可胜数,哭泣之声未绝,伤痍者未起,而欲比隆于成康之时,臣窃以为不侔也。且夫秦地被山带河,四塞以为固,卒然有急,百万之众可具也。因秦之故,资甚美膏腴之地,此所谓天府者也。陛下入关而都之,山东虽乱,秦之故地可全而有也。夫与人斗,不搤其亢,拊其背,未能全其胜也。今陛下入关而都,案秦之故地,此亦搤天下之亢而拊其背也。”

高帝问群臣,群臣皆山东人,争言周王数百年,秦二世即亡,不如都周。上疑未能决。及留侯明言入关便,即日车驾西都关中。

于是上曰:“本言都秦地者娄敬,‘娄’者乃‘刘’也。”赐姓刘氏,拜为郎中,号为奉春君。

汉七年,韩王信反,高帝自往击之。至晋阳,闻信与匈奴欲共击汉,上大怒,使人使匈奴。匈奴匿其壮士肥牛马,但见老弱及羸畜。使者十辈来,皆言匈奴可击。上使刘敬复往使匈奴,还报曰:“两国相击,此宜夸矜见所长。今臣往,徒见羸瘠老弱,此必欲见短,伏奇兵以争利。愚以为匈奴不可击也。”是时汉兵已逾句注,二十馀万兵已业行。上怒,骂刘敬曰:“齐虏!以口舌得官,今乃妄言沮吾军。”械系敬广武。遂往,至平城,匈奴果出奇兵围高帝白登,七日然后得解。高帝至广武,赦敬,曰:“吾不用公言,以困平城。吾皆已斩前使十辈言可击者矣。”乃封敬二千户,为关内侯,号为建信侯。

高帝罢平城归,韩王信亡入胡。当是时,冒顿为单于,兵彊,控弦三十万,数苦北边。上患之,问刘敬。刘敬曰:“天下初定,士卒罢于兵,未可以武服也。冒顿杀父代立,妻群母,以力为威,未可以仁义说也。独可以计久远子孙为臣耳,然恐陛下不能为。”上曰:“诚可,何为不能!顾为柰何?”刘敬对曰:“陛下诚能以適长公主妻之,厚奉遗之,彼知汉適女送厚,蛮夷必慕以为阏氏,生子必为太子。代单于。何者?贪汉重币。陛下以岁时汉所馀彼所鲜数问遗,因使辩士风谕以礼节。冒顿在,固为子婿;死,则外孙为单于。岂尝闻外孙敢与大父抗礼者哉?兵可无战以渐臣也。若陛下不能遣长公主,而令宗室及后宫诈称公主,彼亦知,不肯贵近,无益也。”高帝曰:“善。”欲遣长公主。吕后日夜泣,曰:“妾唯太子、一女,柰何弃之匈奴!”上竟不能遣长公主,而取家人子名为长公主,妻单于。使刘敬往结和亲约。

刘敬从匈奴来,因言“匈奴河南白羊、楼烦王,去长安近者七百里,轻骑一日一夜可以至秦中。秦中新破,少民,地肥饶,可益实。夫诸侯初起时,非齐诸田,楚昭、屈、景莫能兴。今陛下虽都关中,实少人。北近胡寇,东有六国之族,宗彊,一日有变,陛下亦未得高枕而卧也。臣原陛下徙齐诸田,楚昭、屈、景,燕、赵、韩、魏后,及豪桀名家居关中。无事,可以备胡;诸侯有变,亦足率以东伐。此彊本弱末之术也”。上曰:“善。”乃使刘敬徙所言关中十馀万口。

叔孙通者,薛人也。秦时以文学徵,待诏博士。数岁,陈胜起山东,使者以闻,二世召博士诸儒生问曰:“楚戍卒攻蕲入陈,于公如何?”博士诸生三十馀人前曰:“人臣无将,将即反,罪死无赦。原陛下急发兵击之。”二世怒,作色。叔孙通前曰:“诸生言皆非也。夫天下合为一家,毁郡县城,铄其兵,示天下不复用。且明主在其上,法令具于下,使人人奉职,四方辐輳,安敢有反者!此特群盗鼠窃狗盗耳,何足置之齿牙间。郡守尉今捕论,何足忧。”二世喜曰:“善。”尽问诸生,诸生或言反,或言盗。于是二世令御史案诸生言反者下吏,非所宜言。诸言盗者皆罢之。乃赐叔孙通帛二十匹,衣一袭,拜为博士。叔孙通已出宫,反舍,诸生曰:“先生何言之谀也?”通曰:“公不知也,我几不脱于虎口!”乃亡去,之薛,薛已降楚矣。及项梁之薛,叔孙通从之。败于定陶,从怀王。怀王为义帝,徙长沙,叔孙通留事项王。汉二年,汉王从五诸侯入彭城,叔孙通降汉王。汉王败而西,因竟从汉。

叔孙通儒服,汉王憎之;乃变其服,服短衣,楚制,汉王喜。

叔孙通之降汉,从儒生弟子百馀人,然通无所言进,专言诸故群盗壮士进之。弟子皆窃骂曰:“事先生数岁,幸得从降汉,今不能进臣等,专言大猾,何也?”叔孙通闻之,乃谓曰:“汉王方蒙矢石争天下,诸生宁能斗乎?故先言斩将搴旗之士。诸生且待我,我不忘矣。”汉王拜叔孙通为博士,号稷嗣君。

汉五年,已并天下,诸侯共尊汉王为皇帝于定陶,叔孙通就其仪号。高帝悉去秦苛仪法,为简易。群臣饮酒争功,醉或妄呼,拔剑击柱,高帝患之。叔孙通知上益厌之也,说上曰:“夫儒者难与进取,可与守成。臣原徵鲁诸生,与臣弟子共起朝仪。”高帝曰:“得无难乎?”叔孙通曰:“五帝异乐,三王不同礼。礼者,因时世人情为之节文者也。故夏、殷、周之礼所因损益可知者,谓不相复也。臣原颇采古礼与秦仪杂就之。”上曰:“可试为之,令易知,度吾所能行为之。”

于是叔孙通使徵鲁诸生三十馀人。鲁有两生不肯行,曰:“公所事者且十主,皆面谀以得亲贵。今天下初定,死者未葬,伤者未起,又欲起礼乐。礼乐所由起,积德百年而后可兴也。吾不忍为公所为。公所为不合古,吾不行。公往矣,无汙我!”叔孙通笑曰:“若真鄙儒也,不知时变。”

遂与所徵三十人西,及上左右为学者与其弟子百馀人为绵蕞野外。习之月馀,叔孙通曰:“上可试观。”上既观,使行礼,曰:“吾能为此。”乃令群臣习肄,会十月。

汉七年,长乐宫成,诸侯群臣皆朝十月。仪:先平明,谒者治礼,引以次入殿门,廷中陈车骑步卒卫宫,设兵张旗志。传言“趋”。殿下郎中侠陛,陛数百人。功臣列侯诸将军军吏以次陈西方,东乡;文官丞相以下陈东方,西乡。大行设九宾,胪传。于是皇帝辇出房,百官执职传警,引诸侯王以下至吏六百石以次奉贺。自诸侯王以下莫不振恐肃敬。至礼毕,复置法酒。诸侍坐殿上皆伏抑首,以尊卑次起上寿。觞九行,谒者言“罢酒”。御史执法举不如仪者辄引去。竟朝置酒,无敢讙譁失礼者。于是高帝曰:“吾乃今日知为皇帝之贵也。”乃拜叔孙通为太常,赐金五百斤。

叔孙通因进曰:“诸弟子儒生随臣久矣,与臣共为仪,原陛下官之。”高帝悉以为郎。叔孙通出,皆以五百斤金赐诸生。诸生乃皆喜曰:“叔孙生诚圣人也,知当世之要务。”

汉九年,高帝徙叔孙通为太子太傅。汉十二年,高祖欲以赵王如意易太子,叔孙通谏上曰:“昔者晋献公以骊姬之故废太子,立奚齐,晋国乱者数十年,为天下笑。秦以不蚤定扶苏,令赵高得以诈立胡亥,自使灭祀,此陛下所亲见。今太子仁孝,天下皆闻之;吕后与陛下攻苦食啖,其可背哉!陛下必欲废適而立少,臣原先伏诛,以颈血汙地。”高帝曰:“公罢矣,吾直戏耳。”叔孙通曰:“太子天下本,本一摇天下振动,柰何以天下为戏!”高帝曰:“吾听公言。”及上置酒,见留侯所招客从太子入见,上乃遂无易太子志矣。

高帝崩,孝惠即位,乃谓叔孙生曰:“先帝园陵寝庙,群臣莫习。”徙为太常,定宗庙仪法。及稍定汉诸仪法,皆叔孙生为太常所论箸也。

孝惠帝为东朝长乐宫,及间往,数跸烦人,乃作衤复道,方筑武库南。叔孙生奏事,因请间曰:“陛下何自筑衤复道高寝,衣冠月出游高庙?高庙,汉太祖,柰何令后世子孙乘宗庙道上行哉?”孝惠帝大惧,曰:“急坏之。”叔孙生曰:“人主无过举。今已作,百姓皆知之,今坏此,则示有过举。原陛下原庙渭北,衣冠月出游之,益广多宗庙,大孝之本也。”上乃诏有司立原庙。原庙起,以衤复道故。

孝惠帝曾春出游离宫,叔孙生曰:“古者有春尝果,方今樱桃孰,可献,原陛下出,因取樱桃献宗庙。”上乃许之。诸果献由此兴。

太史公曰:语曰“千金之裘,非一狐之腋也;台榭之榱,非一木之枝也;三代之际,非一士之智也”。信哉!夫高祖起微细,定海内,谋计用兵,可谓尽之矣。然而刘敬脱輓辂一说,建万世之安,智岂可专邪!叔孙通希世度务,制礼进退,与时变化,卒为汉家儒宗。“大直若诎,道固委蛇”,盖谓是乎?

厦藉众幹,裘非一狐。委辂献说,釂蕝陈书。皇帝始贵,车驾西都。既安太子,又和匈奴。奉春、稷嗣,其功可图。
\end{yuanwen}

\chapter{季布栾布列传}

\begin{yuanwen}
季布者,楚人也。为气任侠,有名于楚。项籍使将兵,数窘汉王。及项羽灭,高祖购求布千金,敢有舍匿,罪及三族。季布匿濮阳周氏。周氏曰:“汉购将军急,迹且至臣家,将军能听臣,臣敢献计;即不能,原先自刭。”季布许之。乃髡钳季布,衣褐衣,置广柳车中,并与其家僮数十人,之鲁硃家所卖之。硃家心知是季布,乃买而置之田。诫其子曰:“田事听此奴,必与同食。”硃家乃乘轺车之洛阳,见汝阴侯滕公。滕公留硃家饮数日。因谓滕公曰:“季布何大罪,而上求之急也?”滕公曰:“布数为项羽窘上,上怨之,故必欲得之。”硃家曰:“君视季布何如人也?”曰:“贤者也。”硃家曰:“臣各为其主用,季布为项籍用,职耳。项氏臣可尽诛邪?今上始得天下,独以己之私怨求一人,何示天下之不广也!且以季布之贤而汉求之急如此,此不北走胡即南走越耳。夫忌壮士以资敌国,此伍子胥所以鞭荆平王之墓也。君何不从容为上言邪?”汝阴侯滕公心知硃家大侠,意季布匿其所,乃许曰:“诺。”待间,果言如硃家指。上乃赦季布。当是时,诸公皆多季布能摧刚为柔,硃家亦以此名闻当世。季布召见,谢,上拜为郎中。

孝惠时,为中郎将。单于尝为书嫚吕后,不逊,吕后大怒,召诸将议之。上将军樊哙曰:“臣原得十万众,横行匈奴中。”诸将皆阿吕后意,曰“然”。季布曰:“樊哙可斩也!夫高帝将兵四十馀万众,困于平城,今哙柰何以十万众横行匈奴中,面欺!且秦以事于胡,陈胜等起。于今创痍未瘳,哙又面谀,欲摇动天下。”是时殿上皆恐,太后罢朝,遂不复议击匈奴事。

季布为河东守,孝文时,人有言其贤者,孝文召,欲以为御史大夫。复有言其勇,使酒难近。至,留邸一月,见罢。季布因进曰:“臣无功窃宠,待罪河东。陛下无故召臣,此人必有以臣欺陛下者;今臣至,无所受事,罢去,此人必有以毁臣者。夫陛下以一人之誉而召臣,一人之毁而去臣,臣恐天下有识闻之有以闚陛下也。”上默然惭,良久曰:“河东吾股肱郡,故特召君耳。”布辞之官。

楚人曹丘生,辩士,数招权顾金钱。事贵人赵同等,与窦长君善。季布闻之,寄书谏窦长君曰:“吾闻曹丘生非长者,勿与通。”及曹丘生归,欲得书请季布。窦长君曰:“季将军不说足下,足下无往。”固请书,遂行。使人先发书,季布果大怒,待曹丘。曹丘至,即揖季布曰:“楚人谚曰‘得黄金百,不如得季布一诺’,足下何以得此声于梁楚间哉?且仆楚人,足下亦楚人也。仆游扬足下之名于天下,顾不重邪?何足下距仆之深也!”季布乃大说,引入,留数月,为上客,厚送之。季布名所以益闻者,曹丘扬之也。

季布弟季心,气盖关中,遇人恭谨,为任侠,方数千里,士皆争为之死。尝杀人,亡之吴,从袁丝匿。长事袁丝,弟畜灌夫、籍福之属。尝为中司马,中尉郅都不敢不加礼。少年多时时窃籍其名以行。当是时,季心以勇,布以诺,著闻关中。

季布母弟丁公,为楚将。丁公为项羽逐窘高祖彭城西,短兵接,高祖急,顾丁公曰:“两贤岂相戹哉!”于是丁公引兵而还,汉王遂解去。及项王灭,丁公谒见高祖。高祖以丁公徇军中,曰:“丁公为项王臣不忠,使项王失天下者,乃丁公也。”遂斩丁公,曰:“使后世为人臣者无效丁公!”

栾布者,梁人也。始梁王彭越为家人时,尝与布游。穷困,赁佣于齐,为酒人保。数岁,彭越去之巨野中为盗,而布为人所略卖,为奴于燕。为其家主报仇,燕将臧荼举以为都尉。臧荼后为燕王,以布为将。及臧荼反,汉击燕,虏布。梁王彭越闻之,乃言上,请赎布以为梁大夫。

使于齐,未还,汉召彭越,责以谋反,夷三族。已而枭彭越头于雒阳下,诏曰:“有敢收视者,辄捕之。”布从齐还,奏事彭越头下,祠而哭之。吏捕布以闻。上召布,骂曰:“若与彭越反邪?吾禁人勿收,若独祠而哭之,与越反明矣。趣亨之。”方提趣汤,布顾曰:“原一言而死。”上曰:“何言?”布曰:“方上之困于彭城,败荥阳、成皋间,项王所以不能西,徒以彭王居梁地,与汉合从苦楚也。当是之时,彭王一顾,与楚则汉破,与汉而楚破。且垓下之会,微彭王,项氏不亡。天下已定,彭王剖符受封,亦欲传之万世。今陛下一徵兵于梁,彭王病不行,而陛下疑以为反,反形未见,以苛小案诛灭之,臣恐功臣人人自危也。今彭王已死,臣生不如死,请就亨。”于是上乃释布罪,拜为都尉。

孝文时,为燕相,至将军。布乃称曰:“穷困不能辱身下志,非人也;富贵不能快意,非贤也。”于是尝有德者厚报之,有怨者必以法灭之。吴反时,以军功封俞侯,复为燕相。燕齐之间皆为栾布立社,号曰栾公社。

景帝中五年薨。子贲嗣,为太常,牺牲不如令,国除。

太史公曰:以项羽之气,而季布以勇显于楚,身屦军搴旗者数矣,可谓壮士。然至被刑戮,为人奴而不死,何其下也!彼必自负其材,故受辱而不羞,欲有所用其未足也,故终为汉名将。贤者诚重其死。夫婢妾贱人感慨而自杀者,非能勇也,其计画无复之耳。栾布哭彭越,趣汤如归者,彼诚知所处,不自重其死。虽往古烈士,何以加哉!

季布、季心,有声梁、楚。百金然诺,十万致距。出守河东,股肱是与。栾布哭越,犯禁见虏。赴鼎非冤,诚知所处。
\end{yuanwen}

\chapter{袁盎晁错列传}

\begin{yuanwen}
袁盎者,楚人也,字丝。父故为群盗,徙处安陵。高后时,盎尝为吕禄舍人。及孝文帝即位,盎兄哙任盎为中郎。

绛侯为丞相,朝罢趋出,意得甚。上礼之恭,常自送之。袁盎进曰:“陛下以丞相何如人?”上曰:“社稷臣。”盎曰:“绛侯所谓功臣,非社稷臣,社稷臣主在与在,主亡与亡。方吕后时,诸吕用事,擅相王,刘氏不绝如带。是时绛侯为太尉,主兵柄,弗能正。吕后崩,大臣相与共畔诸吕,太尉主兵,適会其成功,所谓功臣,非社稷臣。丞相如有骄主色。陛下谦让,臣主失礼,窃为陛下不取也。”后朝,上益庄,丞相益畏。已而绛侯望袁盎曰:“吾与而兄善,今兒廷毁我!”盎遂不谢。

及绛侯免相之国,国人上书告以为反,徵系清室,宗室诸公莫敢为言,唯袁盎明绛侯无罪。绛侯得释,盎颇有力。绛侯乃大与盎结交。

淮南厉王朝,杀辟阳侯,居处骄甚。袁盎谏曰:“诸侯大骄必生患,可適削地。”上弗用。淮南王益横。及棘蒲侯柴武太子谋反事觉,治,连淮南王,淮南王徵,上因迁之蜀,轞车传送。袁盎时为中郎将,乃谏曰:“陛下素骄淮南王,弗稍禁,以至此,今又暴摧折之。淮南王为人刚,如有遇雾露行道死,陛下竟为以天下之大弗能容,有杀弟之名,柰何?”上弗听,遂行之。

淮南王至雍,病死,闻,上辍食,哭甚哀。盎入,顿首请罪。上曰:“以不用公言至此。”盎曰:“上自宽,此往事,岂可悔哉!且陛下有高世之行者三,此不足以毁名。”上曰:“吾高世行三者何事?”盎曰:“陛下居代时,太后尝病,三年,陛下不交睫,不解衣,汤药非陛下口所尝弗进。夫曾参以布衣犹难之,今陛下亲以王者脩之,过曾参孝远矣。夫诸吕用事,大臣专制,然陛下从代乘六传驰不测之渊,虽贲育之勇不及陛下。陛下至代邸,西向让天子位者再,南面让天子位者三。夫许由一让,而陛下五以天下让,过许由四矣。且陛下迁淮南王,欲以苦其志,使改过,有司卫不谨,故病死。”于是上乃解,曰:“将柰何?”盎曰:“淮南王有三子,唯在陛下耳。”于是文帝立其三子皆为王。盎由此名重朝廷。

袁盎常引大体慷慨。宦者赵同以数幸,常害袁盎,袁盎患之。盎兄子种为常侍骑,持节夹乘,说盎曰:“君与斗,廷辱之,使其毁不用。”孝文帝出,赵同参乘,袁盎伏车前曰:“臣闻天子所与共六尺舆者,皆天下豪英。今汉虽乏人,陛下独奈何与刀锯馀人载!”于是上笑,下赵同。赵同泣下车。

文帝从霸陵上,欲西驰下峻阪。袁盎骑,并车揽辔。上曰:“将军怯邪?”盎曰:“臣闻千金之子坐不垂堂,百金之子不骑衡,圣主不乘危而徼幸。今陛下骋六騑,驰下峻山,如有马惊车败,陛下纵自轻,柰高庙、太后何?”上乃止。

上幸上林,皇后、慎夫人从。其在禁中,常同席坐。及坐,郎署长布席,袁盎引卻慎夫人坐。慎夫人怒,不肯坐。上亦怒,起,入禁中。盎因前说曰:“臣闻尊卑有序则上下和。今陛下既已立后,慎夫人乃妾,妾主岂可与同坐哉!適所以失尊卑矣。且陛下幸之,即厚赐之。陛下所以为慎夫人,適所以祸之。陛下独不见‘人彘’乎?”于是上乃说,召语慎夫人。慎夫人赐盎金五十斤。

然袁盎亦以数直谏,不得久居中,调为陇西都尉。仁爱士卒,士卒皆争为死。迁为齐相。徙为吴相,辞行,种谓盎曰:“吴王骄日久,国多奸。今苟欲劾治,彼不上书告君,即利剑刺君矣。南方卑湿,君能日饮,毋何,时说王曰毋反而已。如此幸得脱。”盎用种之计,吴王厚遇盎。

盎告归,道逢丞相申屠嘉,下车拜谒,丞相从车上谢袁盎。袁盎还,愧其吏,乃之丞相舍上谒,求见丞相。丞相良久而见之。盎因跪曰:“原请间。”丞相曰:“使君所言公事,之曹与长史掾议,吾且奏之;即私邪,吾不受私语。”袁盎即跪说曰:“君为丞相,自度孰与陈平、绛侯?”丞相曰:“吾不如。”袁盎曰:“善,君即自谓不如。夫陈平、绛侯辅翼高帝,定天下,为将相,而诛诸吕,存刘氏;君乃为材官蹶张,迁为队率,积功至淮阳守,非有奇计攻城野战之功。且陛下从代来,每朝,郎官上书疏,未尝不止辇受其言,言不可用置之,言可受采之,未尝不称善。何也?则欲以致天下贤士大夫。上日闻所不闻,明所不知,日益圣智;君今自闭钳天下之口而日益愚。夫以圣主责愚相,君受祸不久矣。”丞相乃再拜曰:“嘉鄙野人,乃不知,将军幸教。”引入与坐,为上客。

盎素不好晁错,晁错所居坐,盎去;盎坐,错亦去:两人未尝同堂语。及孝文帝崩,孝景帝即位,晁错为御史大夫,使吏案袁盎受吴王财物,抵罪,诏赦以为庶人。

吴楚反,闻,晁错谓丞史曰:“夫袁盎多受吴王金钱,专为蔽匿,言不反。今果反,欲请治盎宜知计谋。”丞史曰:“事未发,治之有绝。今兵西乡,治之何益!且袁盎不宜有谋。”晁错犹与未决。人有告袁盎者,袁盎恐,夜见窦婴,为言吴所以反者,原至上前口对状。窦婴入言上,上乃召袁盎入见。晁错在前,及盎请辟人赐间,错去,固恨甚。袁盎具言吴所以反状,以错故,独急斩错以谢吴,吴兵乃可罢。其语具在吴事中。使袁盎为太常,窦婴为大将军。两人素相与善。逮吴反。诸陵长者长安中贤大夫争附两人,车随者日数百乘。

及晁错已诛,袁盎以太常使吴。吴王欲使将,不肯。欲杀之,使一都尉以五百人围守盎军中。袁盎自其为吴相时,有从史尝盗爱盎侍兒,盎知之,弗泄,遇之如故。人有告从史,言“君知尔与侍者通”,乃亡归。袁盎驱自追之,遂以侍者赐之,复为从史。及袁盎使吴见守,从史適为守盎校尉司马,乃悉以其装赍置二石醇醪,会天寒,士卒饥渴,饮酒醉,西南陬卒皆卧,司马夜引袁盎起,曰:“君可以去矣,吴王期旦日斩君。”盎弗信,曰:“公何为者?”司马曰:“臣故为从史盗君侍兒者。”盎乃惊谢曰;“公幸有亲,吾不足以累公。”司马曰:“君弟去,臣亦且亡,辟吾亲,君何患!”乃以刀决张,道从醉卒隧出。司马与分背,袁盎解节毛怀之,杖,步行七八里,明,见梁骑,骑驰去,遂归报。

吴楚已破,上更以元王子平陆侯礼为楚王,袁盎为楚相。尝上书有所言,不用。袁盎病免居家,与闾里浮沈,相随行,斗鸡走狗。雒阳剧孟尝过袁盎,盎善待之。安陵富人有谓盎曰:“吾闻剧孟博徒,将军何自通之?”盎曰:“剧孟虽博徒,然母死,客送葬车千馀乘,此亦有过人者。且缓急人所有。夫一旦有急叩门,不以亲为解,不以存亡为辞,天下所望者,独季心、剧孟耳。今公常从数骑,一旦有缓急,宁足恃乎!”骂富人,弗与通。诸公闻之,皆多袁盎。

袁盎虽家居,景帝时时使人问筹策。梁王欲求为嗣,袁盎进说,其后语塞。梁王以此怨盎,曾使人刺盎。刺者至关中,问袁盎,诸君誉之皆不容口。乃见袁盎曰:“臣受梁王金来刺君,君长者,不忍刺君。然后刺君者十馀曹,备之!”袁盎心不乐,家又多怪,乃之棓生所问占。还,梁刺客后曹辈果遮刺杀盎安陵郭门外。

晁错者,颍川人也。学申商刑名于轵张恢先所,与雒阳宋孟及刘礼同师。以文学为太常掌故。

错为人穞直刻深。孝文帝时,天下无治尚书者,独闻济南伏生故秦博士,治尚书,年九十馀,老不可徵,乃诏太常使人往受之。太常遣错受尚书伏生所。还,因上便宜事,以书称说。诏以为太子舍人、门大夫、家令。以其辩得幸太子,太子家号曰“智囊”。数上书孝文时,言削诸侯事,及法令可更定者。书数十上,孝文不听,然奇其材,迁为中大夫。当是时,太子善错计策,袁盎诸大功臣多不好错。

景帝即位,以错为内史。错常数请间言事,辄听,宠幸倾九卿,法令多所更定。丞相申屠嘉心弗便,力未有以伤。内史府居太上庙壖中,门东出,不便,错乃穿两门南出,凿庙壖垣。丞相嘉闻,大怒,欲因此过为奏请诛错。错闻之,即夜请间,具为上言之。丞相奏事,因言错擅凿庙垣为门,请下廷尉诛。上曰:“此非庙垣,乃壖中垣,不致于法。”丞相谢。罢朝,怒谓长史曰:“吾当先斩以闻,乃先请,为兒所卖,固误。”丞相遂发病死。错以此愈贵。

迁为御史大夫,请诸侯之罪过,削其地,收其枝郡。奏上,上令公卿列侯宗室集议,莫敢难,独窦婴争之,由此与错有卻。错所更令三十章,诸侯皆諠譁疾晁错。错父闻之,从颍川来,谓错曰:“上初即位,公为政用事,侵削诸侯,别疏人骨肉,人口议多怨公者,何也?”晁错曰:“固也。不如此,天子不尊,宗庙不安。”错父曰:“刘氏安矣,而晁氏危矣,吾去公归矣!”遂饮药死,曰:“吾不忍见祸及吾身。”死十馀日,吴楚七国果反,以诛错为名。及窦婴、袁盎进说,上令晁错衣朝衣斩东市。

晁错已死,谒者仆射邓公为校尉,击吴楚军为将。还,上书言军事,谒见上。上问曰:“道军所来,闻晁错死,吴楚罢不?”邓公曰:“吴王为反数十年矣,发怒削地,以诛错为名,其意非在错也。且臣恐天下之士噤口,不敢复言也!”上曰:“何哉?”邓公曰:“夫晁错患诸侯彊大不可制,故请削地以尊京师,万世之利也。计画始行,卒受大戮,内杜忠臣之口,外为诸侯报仇,臣窃为陛下不取也。”于是景帝默然良久,曰:“公言善,吾亦恨之。”乃拜邓公为城阳中尉。

邓公,成固人也,多奇计。建元中,上招贤良,公卿言邓公,时邓公免,起家为九卿。一年,复谢病免归。其子章以脩黄老言显于诸公间。

太史公曰:袁盎虽不好学,亦善傅会,仁心为质,引义慷慨。遭孝文初立,资適逢世。时以变易,及吴楚一说,说虽行哉,然复不遂。好声矜贤,竟以名败。晁错为家令时,数言事不用;后擅权,多所变更。诸侯发难,不急匡救,欲报私雠,反以亡躯。语曰“变古乱常,不死则亡”,岂错等谓邪!

袁丝公直,亦多附会。揽辔见重,卻席翳赖。朝错建策,屡陈利害。尊主卑臣,家危国泰。悲彼二子,名立身败!
\end{yuanwen}

\part{卷一百二}
\chapter{张释之冯唐列传第四十二}

\begin{yuanwen}
张廷尉\footnote{九卿之一,掌管刑狱司法。}释之者,堵阳人也,字季。有兄仲同居。以訾为骑郎,事\footnote{侍奉。}孝文帝,十岁不得调\footnote{升迁。},无所知名。释之曰:“久宦减仲之产,不遂。”欲自免归。

中郎将袁盎知其贤,惜其去,乃请徙\footnote{调动。}释之补谒者。释之既朝毕,因前言便宜事。文帝曰:“卑之,毋甚高论,令今可施行也。”

于是释之言秦汉之间事,秦所以失而汉所以兴者久之。文帝称善,乃拜释之为谒者仆射。
\end{yuanwen}

廷尉张释之,是堵阳人,字季。他和哥哥张仲在一起居住。张释之因为家财颇丰而担任骑郎,侍奉孝文帝,一连十年都没有升迁,也没什么人知道他的名字。张释之说:“我长时间做官会损耗哥哥张仲的家产,心中很是不安。”他打算自己请求免职回家。

中郎将袁盎深知他的贤能,不舍得让他离去,便请求皇帝调张释之补任谒者的空缺。张释之朝见完毕之后,就顺便上前进言对国家有益的事情。文帝说:“实际一点,不要过分地高谈阔论,说国家当前应做的事情。”

于是张释之就开始谈论秦汉之间的事情,论述秦朝灭亡和汉朝兴起的原因,说了很长时间。文帝听了以后表示赞同,于是下令让张释之担任谒者仆射。

\begin{yuanwen}
释之从行,登虎圈\footnote{养虎的地方。}。上问上林尉诸禽兽簿,十馀问,尉左右视,尽不能对。虎圈啬夫从旁代尉对上所问禽兽簿甚悉,欲以观其能口对响应无穷者。文帝曰:“吏不当若是邪?尉无赖!”乃诏释之拜啬夫为上林令。

释之久之前曰:“陛下以绛侯周勃何如人也?”

上曰:“长者也。”

又复问:“东阳侯张相如何如人也?”

上复曰:“长者。”

释之曰:“夫绛侯、东阳侯称为长者,此两人言事曾不能出口,岂斅\footnote{通“效”,效法。}此啬夫谍谍\footnote{同“喋喋”,话多的样子。}利口捷给\footnote{口齿伶俐,反应快。}哉!且秦以任刀笔之吏,吏争以亟疾苛察相高,然其敝徒文具耳,无恻隐之实。以故不闻其过,陵迟而至于二世,天下土崩。今陛下以啬夫口辩而超迁之,臣恐天下随风靡靡,争为口辩而无其实。且下之化上疾于景响,举错不可不审也。”

文帝曰:“善。”乃止不拜啬夫。
\end{yuanwen}

张释之跟随皇上出行,登上虎圈。皇上向上林尉询问登记各种禽兽的情况,一连提出十几个问题,上林尉左顾右盼,一个都回答不出来。看管虎圈的啬夫在一旁代替上林尉详尽地回答了皇上所询问的关于禽兽档案的情况,想以此来显示自己犹如回声响应一般对答如流。文帝说:“作为官吏不该当像这样吗?上林尉靠不住!”于是他便诏令张释之任命啬夫为上林令。

过了很久,张释之才走上前说道:“陛下认为绛侯周勃是个什么样的人?”

皇上说:“他是一位德行端正的人。”

张释之又问:“那么东阳侯张相如又是怎样一个人呢?”

皇上又说:“他也是个德行端正的人。”

这时张释之说道:“绛侯与东阳侯都被称为德行端正的人,而这两个人在议论事情的时候都不善言辞,陛下这样做难道是让人们效法啬夫喋喋不休、伶牙俐齿吗!况且秦朝重用那些舞文弄墨的官吏,官吏们争着用办事迅速和监督苛刻来比较高低优劣,可是这样做的弊端是空具官样文书,而没有发自内心的恻隐本质。正因为这样,秦朝皇帝听不到自己的过错,所以逐渐衰败,皇位传到二世,天下便土崩瓦解了。如今陛下只因为啬夫能言善辩而对他破格提拔,我担心天下人受这风气影响而争着施展口才却不务实质。再说,身处下位的人效法上位者比影之随形、响之应声还要迅速,因此,对于一切措施,千万不能不审慎对待啊。”

文帝说:“很好。”于是放弃原来的计划,不再升迁啬夫了。

\begin{yuanwen}
上就车,召释之参乘,徐行,问释之秦之敝。具以质言。至宫,上拜释之为公车令。
\end{yuanwen}

皇上登上了车,召张释之陪乘,车子缓缓前行,皇上问张释之秦朝都有哪些弊端。张释之全都如实回答。到了宫中,皇上便下令让张释之担任公车令。

\begin{yuanwen}
顷之,太子与梁王共车入朝,不下司马门\footnote{官署的外门。},于是释之追止太子、梁王无得入殿门。遂劾不下公门不敬,奏之。薄太后闻之,文帝免冠谢曰:“教儿子不谨。”

薄太后乃使使承诏赦太子、梁王,然后得入。文帝由是奇释之,拜为中大夫。
\end{yuanwen}

过了不久,太子与梁王一起坐车进宫朝见,在经过司马门的时候没有下车,张释之便追上去拦住了太子和梁王,不让他们进入殿门。随即控告他们不在司马门下车就是犯了大不敬之罪,并奏报皇帝。薄太后听说了这件事,文帝便摘下帽子谢罪说:“我管教儿子不够严谨。”

薄太后就派遣使者带着诏书赦免了太子和梁王,然后他们才得以进入。文帝根据这件事觉得张释之与众不同,就任命他为中大夫。

\begin{yuanwen}
顷之,至中郎将。从行至霸陵,居北临厕\footnote{同“侧”,边缘险峻处。}。是时慎夫人从,上指示慎夫人新丰道,曰:“此走邯郸道也。”

使慎夫人鼓瑟,上自倚瑟而歌,意惨凄悲怀,顾谓群臣曰:“嗟乎!以北山石为椁,用纻絮斫(斮)陈,蕠\footnote{黏合。}漆其间,岂可动哉!”

左右皆曰:“善。”

释之前进曰:“使其中有可欲者,虽锢南山犹有郄\footnote{通“隙”,缝隙。};使其中无可欲者,虽无石椁,又何戚\footnote{担忧,忧虑。}焉!”

文帝称善。其后拜释之为廷尉。
\end{yuanwen}

不久,张释之升至中郎将。他跟随皇上出行来到霸陵,坐在北边面临霸陵之崖。当时慎夫人也随从出行,皇上便指着通往新丰的路给慎夫人看,说道:“这就是通往邯郸的路。”皇上命慎夫人弹瑟,他自己则和着瑟的乐曲唱歌,情意凄凉而悲哀,回过头来对群臣说:“唉!用北山的石头做外棺,将麻布、棉絮剁细充塞住石椁的缝隙,然后再用漆黏合起来,这难道还能打开吗!”

左右随从都说:“好。”

张释之走上前去进言说:“假使那里面有可以引起欲念的东西,即便是禁闭起整个南山作为棺椁,依然会有缝隙;假使那里面没有可以引起欲念的东西,那么即便没有石椁,又有什么值得忧虑的呢!”

文帝听了以后连连称赞。后来又任命他为廷尉。

\begin{yuanwen}
顷之,上行出中渭桥,有一人从穚(桥)下走出,乘舆马惊。于是使骑捕,属之廷尉。释之治问。曰:“县人来,闻跸\footnote{帝王出行时清道戒严。},匿桥下。久之,以为行已过,即出,见乘舆车骑,即走耳。”

廷尉秦当,一人犯跸,当罚金。文帝怒曰:“此人亲惊吾马,吾马赖柔和,令他马,固不败伤我乎?而廷尉乃当之罚金!”

释之曰:“法者天子所与天下公共也。今法如此而更重之,是法不信于民也。且方其时,上使立诛之则已。今既下廷尉,廷尉,天下之平也,一倾而天下用法皆为轻重,民安所措其手足?唯陛下察之。”

良久,上曰:“廷尉当是也。”
\end{yuanwen}

不久以后,有一次皇上出行途经中渭桥,突然有一个人从桥底下面跑了出来,使驾车的马匹受到惊吓。于是皇上命令骑士把那个人抓住,交给廷尉张释之处置。张释之对那个人进行审问。那人说:“我是长安县的乡下人,来到这里,听到清道戒严的命令,就在桥下藏了起来。过了很长时间,我以为皇上的队伍已经过去,就走了出来,一看到皇上的车马、仪仗队,就马上跑了。”

廷尉张释之向皇上报告此人应得的惩罚,说一个人犯了清道的禁令,应当处以罚金。文帝生气地说:“这个人使我的马受惊,幸好我的马脾性柔和,假如换做别的马,岂不必然会摔伤我吗?可是廷尉却只判他罚金!”

张释之说:“所谓法律,是天子与天下人共同遵奉的。如今依照法律就应该这样判定,而陛下要肆意更改加重处罚,这样一来法律就不会取信于民。况且在当时,皇上如果立刻杀了他也就罢了。既然现在交给廷尉处置,而廷尉又是天下公平的代表,一旦有所倾斜,天下人在使用法律时都任意地或轻或重,这样一来人民岂不是手足无措了吗?希望陛下明察。”

过了很久,皇上才说:“廷尉就应该如此。”

\begin{yuanwen}
其后有人盗高庙坐前玉环,捕得,文帝怒,下廷尉治。释之案律盗宗庙服御物者为奏,奏当弃市。上大怒曰:“人之无道,乃盗先帝庙器,吾属廷尉者,欲致之族,而君以法奏之,非吾所以共承宗庙意也。”释之免冠顿首谢曰:“法如是足也。且罪等,然以逆顺为差。今盗宗庙器而族之,有如万分之一,假令愚民取长陵一抔土,陛下何以加其法乎?”久之,文帝与太后言之,乃许廷尉当。是时,中尉条侯周亚夫与梁相山都侯王恬开见释之持议平,乃结为亲友。张廷尉由此天下称之。
\end{yuanwen}



\begin{yuanwen}
后文帝崩,景帝立,释之恐,称病。欲免去,惧大诛至;欲见谢,则未知何如。用王生计,卒见谢,景帝不过也。
\end{yuanwen}
\begin{yuanwen}
王生者,善为黄老言,处士也。尝召居廷中,三公九卿尽会立,王生老人,曰“吾穇解”,顾谓张廷尉:“为我结穇!”释之跪而结之。既已,人或谓王生曰:“独柰何廷辱张廷尉,使跪结穇?”王生曰:“吾老且贱,自度终无益于张廷尉。张廷尉方今天下名臣,吾故聊辱廷尉,使跪结穇,欲以重之。”诸公闻之,贤王生而重张廷尉。
\end{yuanwen}
\begin{yuanwen}
张廷尉事景帝岁馀,为淮南王相,犹尚以前过也。久之,释之卒。其子曰张挚,字长公,官至大夫,免。以不能取容当世,故终身不仕。
\end{yuanwen}
\begin{yuanwen}
冯唐者,其大父赵人。父徙代。汉兴徙安陵。唐以孝著,为中郎署长,事文帝。文帝辇过,问唐曰:“父老何自为郎?家安在?”唐具以实对。文帝曰:“吾居代时,吾尚食监高袪数为我言赵将李齐之贤,战于钜鹿下。今吾每饭,意未尝不在钜鹿也。父知之乎?”唐对曰:“尚不如廉颇、李牧之为将也。”上曰:“何以?”唐曰:“臣大父在赵时,为官将,善李牧。臣父故为代相,善赵将李齐,知其为人也。”上既闻廉颇、李牧为人,良说,而搏髀曰:“嗟乎!吾独不得廉颇、李牧时为吾将,吾岂忧匈奴哉!”唐曰:“主臣!陛下虽得廉颇、李牧,弗能用也。”上怒,起入禁中。良久,召唐让曰:“公柰何众辱我,独无间处乎?”唐谢曰:“鄙人不知忌讳。”
\end{yuanwen}
\begin{yuanwen}
当是之时,匈奴新大入朝,杀北地都尉卬。上以胡寇为意,乃卒复问唐曰:“公何以知吾不能用廉颇、李牧也?”唐对曰:“臣闻上古王者之遣将也,跪而推毂,曰阃以内者,寡人制之;阃以外者,将军制之。军功爵赏皆决于外,归而奏之。此非虚言也。臣大父言,李牧为赵将居边,军市之租皆自用飨士,赏赐决于外,不从中扰也。委任而责成功,故李牧乃得尽其智能,遣选车千三百乘,彀骑万三千,百金之士十万,是以北逐单于,破东胡,灭澹林,西抑彊秦,南支韩、魏。当是之时,赵几霸。其后会赵王迁立,其母倡也。王迁立,乃用郭开谗,卒诛李牧,令颜聚代之。是以兵破士北,为秦所禽灭。今臣窃闻魏尚为云中守,其军市租尽以飨士卒,私养钱,五日一椎牛,飨宾客军吏舍人,是以匈奴远避,不近云中之塞。虏曾一入,尚率车骑击之,所杀其众。夫士卒尽家人子,起田中从军,安知尺籍伍符。终日力战,斩首捕虏,上功莫府,一言不相应,文吏以法绳之。其赏不行而吏奉法必用。臣愚,以为陛下法太明,赏太轻,罚太重。且云中守魏尚坐上功首虏差六级,陛下下之吏,削其爵,罚作之。由此言之,陛下虽得廉颇、李牧,弗能用也。臣诚愚,触忌讳,死罪死罪!”文帝说。是日令冯唐持节赦魏尚,复以为云中守,而拜唐为车骑都尉,主中尉及郡国车士。
\end{yuanwen}
\begin{yuanwen}
七年,景帝立,以唐为楚相,免。武帝立,求贤良,举冯唐。唐时年九十馀,不能复为官,乃以唐子冯遂为郎。遂字王孙,亦奇士,与余善。
\end{yuanwen}
\begin{yuanwen}
太史公曰:张季之言长者,守法不阿意;冯公之论将率,有味哉!有味哉!语曰“不知其人,视其友”。二君之所称诵,可著廊庙。书曰“不偏不党,王道荡荡;不党不偏,王道便便”。张季、冯公近之矣。
\end{yuanwen}
\begin{yuanwen}
张季未偶,见识袁盎。太子惧法,啬夫无状。惊马罚金,盗环悟上。冯公白首,味哉论将。因对李齐,收功魏尚。
\end{yuanwen}

\part{卷一百三}
\chapter{万石张叔列传第四十三}

\begin{yuanwen}
万石君名奋,其父赵人也,姓石氏。赵亡,徙居温。高祖东击项籍,过河内,时奋年十五,为小吏,侍高祖。高祖与语,爱其恭敬,问曰:“若何有?”对曰:“奋独有母,不幸失明。家贫。有姊,能鼓琴。”高祖曰:“若能从我乎?”曰:“原尽力。”于是高祖召其姊为美人,以奋为中涓,受书谒,徙其家长安中戚里,以姊为美人故也。其官至孝文时,积功劳至大中大夫。无文学,恭谨无与比。

文帝时,东阳侯张相如为太子太傅,免。选可为傅者,皆推奋,奋为太子太傅。及孝景即位,以为九卿;迫近,惮之,徙奋为诸侯相。奋长子建,次子甲,次子乙,次子庆,皆以驯行孝谨,官皆至二千石。于是景帝曰:“石君及四子皆二千石,人臣尊宠乃集其门。”号奋为万石君。

孝景帝季年,万石君以上大夫禄归老于家,以岁时为朝臣。过宫门阙,万石君必下车趋,见路马必式焉。子孙为小吏,来归谒,万石君必朝服见之,不名。子孙有过失,不谯让,为便坐,对案不食。然后诸子相责,因长老肉袒固谢罪,改之,乃许。子孙胜冠者在侧,虽燕居必冠,申申如也。僮仆如也,唯谨。上时赐食于家,必稽首俯伏而食之,如在上前。其执丧,哀戚甚悼。子孙遵教,亦如之。万石君家以孝谨闻乎郡国,虽齐鲁诸儒质行,皆自以为不及也。

建元二年,郎中令王臧以文学获罪。皇太后以为儒者文多质少,今万石君家不言而躬行,乃以长子建为郎中令,少子庆为内史。

建老白首,万石君尚无恙。建为郎中令,每五日洗沐归谒亲,入子舍,窃问侍者,取亲中稖厕窬,身自浣涤,复与侍者,不敢令万石君知,以为常。建为郎中令,事有可言,屏人恣言,极切;至廷见,如不能言者。是以上乃亲尊礼之。

万石君徙居陵里。内史庆醉归,入外门不下车。万石君闻之,不食。庆恐,肉袒请罪,不许。举宗及兄建肉袒,万石君让曰:“内史贵人,入闾里,里中长老皆走匿,而内史坐车中自如,固当!”乃谢罢庆。庆及诸子弟入里门,趋至家。

万石君以元朔五年中卒。长子郎中令建哭泣哀思,扶杖乃能行。岁馀,建亦死。诸子孙咸孝,然建最甚,甚于万石君。

建为郎中令,书奏事,事下,建读之,曰:“误书!‘马’者与尾当五,今乃四,不足一。上谴死矣!”甚惶恐。其为谨慎,虽他皆如是。

万石君少子庆为太仆,御出,上问车中几马,庆以策数马毕,举手曰:“六马。”庆于诸子中最为简易矣,然犹如此。为齐相,举齐国皆慕其家行,不言而齐国大治,为立石相祠。

元狩元年,上立太子,选群臣可为傅者,庆自沛守为太子太傅,七岁迁为御史大夫。

元鼎五年秋,丞相有罪,罢。制诏御史:“万石君先帝尊之,子孙孝,其以御史大夫庆为丞相,封为牧丘侯。”是时汉方南诛两越,东击朝鲜,北逐匈奴,西伐大宛,中国多事。天子巡狩海内,修上古神祠,封禅,兴礼乐。公家用少,桑弘羊等致利,王温舒之属峻法,兒宽等推文学至九卿,更进用事,事不关决于丞相,丞相醇谨而已。在位九岁,无能有所匡言。尝欲请治上近臣所忠、九卿咸宣罪,不能服,反受其过,赎罪。

元封四年中,关东流民二百万口,无名数者四十万,公卿议欲请徙流民于边以適之。上以为丞相老谨,不能与其议,乃赐丞相告归,而案御史大夫以下议为请者。丞相惭不任职,乃上书曰:“庆幸得待罪丞相,罢驽无以辅治,城郭仓库空虚,民多流亡,罪当伏斧质,上不忍致法。原归丞相侯印,乞骸骨归,避贤者路。”天子曰:“仓廪既空,民贫流亡,而君欲请徙之,摇荡不安,动危之,而辞位,君欲安归难乎?”以书让庆,庆甚惭,遂复视事。

庆文深审谨,然无他大略,为百姓言。后三岁馀,太初二年中,丞相庆卒,谥为恬侯。庆中子德,庆爱用之,上以德为嗣,代侯。后为太常,坐法当死,赎免为庶人。庆方为丞相,诸子孙为吏更至二千石者十三人。及庆死后,稍以罪去,孝谨益衰矣。

建陵侯卫绾者,代大陵人也。绾以戏车为郎,事文帝,功次迁为中郎将,醇谨无他。孝景为太子时,召上左右饮,而绾称病不行。文帝且崩时,属孝景曰:“绾长者,善遇之。”及文帝崩,景帝立,岁馀不噍呵绾,绾日以谨力。

景帝幸上林,诏中郎将参乘,还而问曰:“君知所以得参乘乎?”绾曰:“臣从车士幸得以功次迁为中郎将,不自知也。”上问曰:“吾为太子时召君,君不肯来,何也?”对曰:“死罪,实病!”上赐之剑。绾曰:“先帝赐臣剑凡六,剑不敢奉诏。”上曰:“剑,人之所施易,独至今乎?”绾曰:“具在。”上使取六剑,剑尚盛,未尝服也。郎官有谴,常蒙其罪,不与他将争;有功,常让他将。上以为廉,忠实无他肠,乃拜绾为河间王太傅。吴楚反,诏绾为将,将河间兵击吴楚有功,拜为中尉。三岁,以军功,孝景前六年中封绾为建陵侯。

其明年,上废太子,诛栗卿之属。上以为绾长者,不忍,乃赐绾告归,而使郅都治捕栗氏。既已,上立胶东王为太子,召绾,拜为太子太傅。久之,迁为御史大夫。五岁,代桃侯舍为丞相,朝奏事如职所奏。然自初官以至丞相,终无可言。天子以为敦厚,可相少主,尊宠之,赏赐甚多。

为丞相三岁,景帝崩,武帝立。建元年中,丞相以景帝疾时诸官囚多坐不辜者,而君不任职,免之。其后绾卒,子信代。坐酎金失侯。

塞侯直不疑者,南阳人也。为郎,事文帝。其同舍有告归,误持同舍郎金去,已而金主觉,妄意不疑,不疑谢有之,买金偿。而告归者来而归金,而前郎亡金者大惭,以此称为长者。文帝称举,稍迁至太中大夫。朝廷见,人或毁曰:“不疑状貌甚美,然独无柰其善盗嫂何也!”不疑闻,曰:“我乃无兄。”然终不自明也。

吴楚反时,不疑以二千石将兵击之。景帝后元年,拜为御史大夫。天子修吴楚时功,乃封不疑为塞侯。武帝建元年中,与丞相绾俱以过免。

不疑学老子言。其所临,为官如故,唯恐人知其为吏迹也。不好立名称,称为长者。不疑卒,子相如代。孙望,坐酎金失侯。

郎中令周文者,名仁,其先故任城人也。以医见。景帝为太子时,拜为舍人,积功稍迁,孝文帝时至太中大夫。景帝初即位,拜仁为郎中令。

仁为人阴重不泄,常衣敝补衣溺袴,期为不絜清,以是得幸。景帝入卧内,于后宫祕戏,仁常在旁。至景帝崩,仁尚为郎中令,终无所言。上时问人,仁曰:“上自察之。”然亦无所毁。以此景帝再自幸其家。家徙阳陵。上所赐甚多,然常让,不敢受也。诸侯群臣赂遗,终无所受。

武帝立,以为先帝臣,重之。仁乃病免,以二千石禄归老,子孙咸至大官矣。

御史大夫张叔者,名欧,安丘侯说之庶子也。孝文时以治刑名言事太子。然欧虽治刑名家,其人长者。景帝时尊重,常为九卿。至武帝元朔四年,韩安国免,诏拜欧为御史大夫。自欧为吏,未尝言案人,专以诚长者处官。官属以为长者,亦不敢大欺。上具狱事,有可卻,卻之;不可者,不得已,为涕泣面对而封之。其爱人如此。

老病笃,请免。于是天子亦策罢,以上大夫禄归老于家。家于阳陵。子孙咸至大官矣。

太史公曰:仲尼有言曰“君子欲讷于言而敏于行”,其万石、建陵、张叔之谓邪?是以其教不肃而成,不严而治。塞侯微巧,而周文处讇,君子讥之,为其近于佞也。然斯可谓笃行君子矣!

万石孝谨,自家形国。郎中数马,内史匍匐。绾无他肠,塞有阴德。刑名张欧,垂涕恤狱。敏行讷言,俱嗣芳躅。
\end{yuanwen}

\chapter{田叔列传}

\begin{yuanwen}
田叔者,赵陉城人也。其先,齐田氏苗裔也。叔喜剑,学黄老术于乐巨公所。叔为人刻廉自喜,喜游诸公。赵人举之赵相赵午,午言之赵王张敖所,赵王以为郎中。数岁,切直廉平,赵王贤之,未及迁。

会陈豨反代,汉七年,高祖往诛之,过赵,赵王张敖自持案进食,礼恭甚,高祖箕踞骂之。是时赵相赵午等数十人皆怒,谓张王曰:“王事上礼备矣,今遇王如是,臣等请为乱。”赵王齧指出血,曰:“先人失国,微陛下,臣等当蟲出。公等柰何言若是!毋复出口矣!”于是贯高等曰:“王长者,不倍德。”卒私相与谋弑上。会事发觉,汉下诏捕赵王及群臣反者。于是赵午等皆自杀,唯贯高就系。是时汉下诏书:“赵有敢随王者罪三族。”唯孟舒、田叔等十馀人赭衣自髡钳,称王家奴,随赵王敖至长安。贯高事明白,赵王敖得出,废为宣平侯,乃进言田叔等十馀人。上尽召见,与语,汉廷臣毋能出其右者,上说,尽拜为郡守、诸侯相。叔为汉中守十馀年,会高后崩,诸吕作乱,大臣诛之,立孝文帝。

孝文帝既立,召田叔问之曰:“公知天下长者乎?”对曰:“臣何足以知之!”上曰:“公,长者也,宜知之。”叔顿首曰:“故云中守孟舒,长者也。”是时孟舒坐虏大入塞盗劫,云中尤甚,免。上曰:“先帝置孟舒云中十馀年矣,虏曾一人,孟舒不能坚守,毋故士卒战死者数百人。长者固杀人乎?公何以言孟舒为长者也?”叔叩头对曰:“是乃孟舒所以为长者也。夫贯高等谋反,上下明诏,赵有敢随张王,罪三族。然孟舒自髡钳,随张王敖之所在,欲以身死之,岂自知为云中守哉!汉与楚相距,士卒罢敝。匈奴冒顿新服北夷,来为边害,孟舒知士卒罢敝,不忍出言,士争临城死敌,如子为父,弟为兄,以故死者数百人。孟舒岂故驱战之哉!是乃孟舒所以为长者也。”于是上曰:“贤哉孟舒!”复召孟舒以为云中守。

后数岁,叔坐法失官。梁孝王使人杀故吴相袁盎,景帝召田叔案梁,具得其事,还报。景帝曰:“梁有之乎?”叔对曰:“死罪!有之。”上曰:“其事安在?”田叔曰:“上毋以梁事为也。”上曰:“何也?”曰:“今梁王不伏诛,是汉法不行也;如其伏法,而太后食不甘味,卧不安席,此忧在陛下也。”景帝大贤之,以为鲁相。

鲁相初到,民自言相,讼王取其财物百馀人。田叔取其渠率二十人,各笞五十,馀各搏二十,怒之曰:“王非若主邪?何自敢言若主!”鲁王闻之大惭,发中府钱,使相偿之。相曰:“王自夺之,使相偿之,是王为恶而相为善也。相毋与偿之。”于是王乃尽偿之。

鲁王好猎,相常从入苑中,王辄休相就馆舍,相出,常暴坐待王苑外。王数使人请相休,终不休,曰:“我王暴露苑中,我独何为就舍!”鲁王以故不大出游。

数年,叔以官卒,鲁以百金祠,少子仁不受也,曰:“不以百金伤先人名。”

仁以壮健为卫将军舍人,数从击匈奴。卫将军进言仁,仁为郎中。数岁,为二千石丞相长史,失官。其后使刺举三河。上东巡,仁奏事有辞,上说,拜为京辅都尉。月馀,上迁拜为司直。数岁,坐太子事。时左相自将兵,令司直田仁主闭守城门,坐纵太子,下吏诛死。仁发兵,长陵令车千秋上变仁,仁族死。陉城今在中山国。

太史公曰:孔子称曰“居是国必闻其政”,田叔之谓乎!义不忘贤,明主之美以救过。仁与余善,余故并论之。

褚先生曰:臣为郎时,闻之曰田仁故与任安相善。任安,荥阳人也。少孤贫困,为人将车之长安,留,求事为小吏,未有因缘也,因占著名数。武功,扶风西界小邑也,谷口蜀刬道近山。安以为武功小邑,无豪,易高也,安留,代人为求盗亭父。后为亭长。邑中人民俱出猎,任安常为人分麋鹿雉兔,部署老小当壮剧易处,众人皆喜,曰:“无怂也,任少卿分别平,有智略。”明日复合会,会者数百人。任少卿曰:“某子甲何为不来乎?”诸人皆怪其见之疾也。其后除为三老,举为亲民,出为三百石长,治民。坐上行出游共帐不办,斥免。

乃为卫将军舍人,与田仁会,俱为舍人,居门下,同心相爱。此二人家贫,无钱用以事将军家监,家监使养恶齧马。两人同床卧,仁窃言曰:“不知人哉家监也!”任安曰:“将军尚不知人,何乃家监也!”卫将军从此两人过平阳主,主家令两人与骑奴同席而食,此二子拔刀列断席别坐。主家皆怪而恶之,莫敢呵。

其后有诏募择卫将军舍人以为郎,将军取舍人中富给者,令具鞍马绛衣玉具剑,欲入奏之。会贤大夫少府赵禹来过卫将军,将军呼所举舍人以示赵禹。赵禹以次问之,十馀人无一人习事有智略者。赵禹曰:“吾闻之,将门之下必有将类。传曰‘不知其君视其所使,不知其子视其所友’。今有诏举将军舍人者,欲以观将军而能得贤者文武之士也。今徒取富人子上之,又无智略,如木偶人衣之绮绣耳,将柰之何?”于是赵禹悉召卫将军舍人百馀人,以次问之,得田仁、任安,曰:“独此两人可耳,馀无可用者。”卫将军见此两人贫,意不平。赵禹去,谓两人曰:“各自具鞍马新绛衣。”两人对曰:“家贫无用具也。”将军怒曰:“今两君家自为贫,何为出此言?鞅鞅如有移德于我者,何也?”将军不得已,上籍以闻。有诏召见卫将军舍人,此二人前见,诏问能略相推第也。田仁对曰;“提桴鼓立军门,使士大夫乐死战斗,仁不及任安。”任安对曰:“夫决嫌疑。定是非,辩治官,使百姓无怨心,安不及仁也。”武帝大笑曰:“善。”使任安护北军,使田仁护边田穀于河上。此两人立名天下。

其后用任安为益州刺史,以田仁为丞相长史。

田仁上书言:“天下郡太守多为奸利,三河尤甚,臣请先刺举三河。三河太守皆内倚中贵人,与三公有亲属,无所畏惮,宜先正三河以警天下奸吏。”是时河南、河内太守皆御史大夫杜父兄子弟也,河东太守石丞相子孙也。是时石氏九人为二千石,方盛贵。田仁数上书言之。杜大夫及石氏使人谢,谓田少卿曰:“吾非敢有语言也,原少卿无相诬汙也。”仁已刺三河,三河太守皆下吏诛死。仁还奏事,武帝说,以仁为能不畏彊御,拜仁为丞相司直,威振天下。

其后逢太子有兵事,丞相自将兵,使司直主城门。司直以为太子骨肉之亲,父子之间不甚欲近,去之诸陵过。是时武帝在甘泉,使御史大夫暴君下责丞相“何为纵太子”,丞相对言“使司直部守城门而开太子”。上书以闻,请捕系司直。司直下吏,诛死。

是时任安为北军使者护军,太子立车北军南门外,召任安,与节令发兵。安拜受节,入,闭门不出。武帝闻之,以为任安为详邪,不傅事,何也?任安笞辱北军钱官小吏,小吏上书言之,以为受太子节,言“幸与我其鲜好者”。书上闻,武帝曰:“是老吏也,见兵事起,欲坐观成败,见胜者欲合从之,有两心。安有当死之罪甚众,吾常活之,今怀诈,有不忠之心。”下安吏,诛死。

夫月满则亏,物盛则衰,天地之常也。知进而不知退,久乘富贵,祸积为祟。故范蠡之去越,辞不受官位,名传后世,万岁不忘,岂可及哉!后进者慎戒之。

田叔长者,重义轻生。张王既雪,汉中是荣。孟舒见废,抗说相明。按梁以礼,相鲁得情。子仁坐事,刺举有声。
\end{yuanwen}

\part{卷一百五}

\chapter{扁鹊仓公列传第四十五}

是战国时的秦越人与汉代的淳于意两位医生的合传,有可以说是两汉之前临床医学的总结,为我们保留了珍贵的医学史料,具有很高的研究价值。同时通过两位医生的不幸遭遇,揭示出在封建社会造福于民的高明医术也成为招来杀身之祸的缘由,实在令人扼腕,发人深省。

\begin{yuanwen}
扁鹊者\footnote{text},勃海郡郑人也,姓秦氏,名越人。少时为人舍长\footnote{招待宾客之所的管理人。}。舍客长桑君过,扁鹊独奇之,常谨遇之。长桑君亦知扁鹊非常人也。出入十余年,乃呼扁鹊私坐,间与语曰:“我有禁方\footnote{text},年老,欲传与公,公毋泄。”

扁鹊曰:“敬诺。”

乃出其怀中药予扁鹊:“饮是以上池之水\footnote{text},三十日当知物矣\footnote{text}。”

乃悉取其禁方书尽与扁鹊。忽然不见,殆非人也。扁鹊以其言饮药三十日,视见垣一方人。以此视病,尽见五藏\footnote{通“脏”。}症结,特\footnote{只,只不过。}以诊脉为名耳。为医或在齐,或在赵。在赵者名扁鹊。
\end{yuanwen}

扁鹊,是勃海郡郑地人,姓秦,名越人。他年轻时担任人家客馆的主管。一次,有个叫长桑君的客人从此地经过,住在客馆,只有扁鹊觉得他是一个非常奇特的人,所以总是恭恭敬敬地对待他。长桑君也知道扁鹊不是一般人。他在这家客馆来来去去长达十多年,后来终于把扁鹊叫进来单独和他坐在一起,悄悄地对他说:“我身上有秘藏的药方,我现在老了,想把它传给您,您千万不要泄露出去。”

扁鹊说道:“遵命。”

于是长桑君拿出怀中的药交给扁鹊,说:“用没有落地的露水送服此药,三十天以后,就可以洞彻万物了。”

于是把他的秘方拿出来全部交给扁鹊。转瞬之间,长桑君就不见了踪影,他大概不是凡人吧。扁鹊按照他的吩咐服药三十天以后,果然能够隔着墙看见另一边的人。他凭借这种本领看病,完全能够看清人体五脏的疾病所在,只不过以诊脉为名而已。他有时候在齐国行医,有时候又在赵国。在赵国时候人们称他为扁鹊。

\begin{yuanwen}
当晋昭公时\footnote{text},诸大夫强而公族弱\footnote{text},赵简子为大夫\footnote{text},专国事。简子疾,五日不知人,大夫皆惧,于是召扁鹊。扁鹊入视病,出。董安于问扁鹊。

扁鹊曰:“血脉治\footnote{正常。}也,而何怪!昔秦穆公尝如此,七日而寤。寤之日,告公孙支与子舆曰,‘我之帝所甚乐。吾所以久者,适有所学也。帝告我,‘晋国且大乱,五世不安。其后将霸,未老而死。霸者之子且令而国男女无别。’’公孙支书而藏之,秦策于是出。夫献公之乱,文公之霸,而襄公败秦师于殽而归纵淫,此子之所闻。今主君之病与之同,不出三日必间,间必有言也。”
\end{yuanwen}

晋昭公在位时,各大夫势力强盛而国君的力量却很薄弱,赵简子是晋国的大夫,独揽大权。有一次赵简子生病,一连昏迷五天不省人事,大夫们都非常害怕,于是把扁鹊召来治疗。扁鹊进来察看赵简子的病情,然后就走了出去,大夫董安于向扁鹊询问病情。

扁鹊答道:“他的血脉十分正常,你们有什么可大惊小怪的!过去秦穆公也曾出现过这种情况,昏迷了整整七天,然后才苏醒过来。等到苏醒的那天,他对公孙支和子舆说:‘我到了天帝那里非常快乐。我之所以在那儿呆了这么久,是因为恰好碰到天帝指教我。天帝告诉我说:“晋国即将大乱,延续五代都不会安宁。此后将有人成为霸主,他称霸不久就会死去。这位霸主的儿子将会使国家男女淫乱。”’公孙支将秦穆公的话记录下来收藏好,秦国的史书便据此记载了这件事。晋献公晚年国内大乱,晋文公称霸,晋襄公在殽山打败秦军以后放纵淫乱,这些您都是听说过的。现在主君的病情与秦穆公相同,不出三天就会痊愈,痊愈之后一定有话要讲。”

\begin{yuanwen}
居二日半,简子寤,语诸大夫曰:“我之帝所甚乐,与百神游于钧天\footnote{天的中央。},广乐九奏万舞,不类三代之乐,其声动心。有一熊欲援我,帝命我射之,中熊,熊死。有罴来,我又射之,中罴,罴死。帝甚喜,赐我二笥\footnote{盛物的方形竹器。},皆有副\footnote{装饰,首饰。}。吾见儿在帝侧,帝属我一翟犬,曰:“及而子之壮也以赐之。”帝告我:“晋国且世衰,七世而亡。嬴姓将大败周人于范魁之西,而亦不能有也。””

董安于受言,书而藏之。以扁鹊言告简子,简子赐扁鹊田四万亩。
\end{yuanwen}

只过了两天半,赵简子就苏醒了,他对众大夫说:“我到天帝那里非常快乐,与百神在天的中央游玩,那里有各种乐器,演奏出许多乐曲,还有各种各样的舞蹈,与上古三代时期的乐舞不同,那里的乐声动人心弦。有一只熊想要抓住我,天帝命我用弓箭射杀它,结果我射中了熊,熊就死了。又有一只罴向我走来,我又射它,射中了,结果罴也死了。天帝很高兴,就赐给我两只竹笥,里面都装着首饰。我看到我的儿子在天帝身旁,天帝交给我一只翟犬,说道:‘等你儿子长大以后,就把这个赐给他。’天帝还告诉我:‘晋国即将一代代地衰败下去,再过七代就会灭亡。嬴姓的秦国将在范魁的西边打败周朝人,然而他们也不能拥有周朝的政权。’”

董安于听了赵简子的话以后,记录并收藏起来。有人将扁鹊所说的话告诉了赵简子,赵简子于是赏赐给扁鹊田地四万亩。

\begin{yuanwen}
其后扁鹊过虢\footnote{text}。虢太子死,扁鹊至虢宫门下,问中庶子喜方者曰\footnote{text}:“太子何病,国中治穰\footnote{通“禳”。}过于众事?”

中庶子曰:“太子病血气不时,交错而不得泄,暴发于外,则为中害\footnote{text}。精神不能止邪气,邪气畜积而不得泄,是以阳缓而阴急,故暴蹶而死。”

扁鹊曰:“其死何如时?”

曰:“鸡鸣至今\footnote{text}。”

曰:“收乎?”

曰:“未也,其死未能半日也。”

“言臣齐勃海秦越人也,家在于郑,未尝得望精光侍谒于前也。闻太子不幸而死,臣能生之。”

中庶子曰:“先生得无诞之乎\footnote{text}?何以言太子可生也!臣闻上古之时,医有俞跗\footnote{text},治病不以汤液醴洒\footnote{text},镵石\footnote{石针。}挢引\footnote{导引。},案\footnote{通“按”。}扤\footnote{摇动。}毒熨\footnote{text},一拨见病之应\footnote{text},因五藏之输\footnote{通“腧”,人身上的穴位。},乃割皮解肌,诀脉结筋\footnote{text},搦\footnote{nuò}髓脑\footnote{text},揲\footnote{持。}荒\footnote{通“肓”,心脏与横膜之间。}爪\footnote{梳理。}幕\footnote{通“膜”,指横膈膜。},湔\footnote{浣洗。}浣肠胃,漱涤五藏,练精易形。先生之方能若是,则太子可生也;不能若是而欲生之,曾不可以告咳婴之儿。”

终日,扁鹊仰天叹曰:“夫子之为方也\footnote{text},若以管窥天,以(郄/卻)\footnote{同“隙”。}视文\footnote{text}。越人之为方也,不待切脉望色听声写形\footnote{text},言病之所在。闻病之阳,论得其阴;闻病之阴,论得其阳。病应见于大表,不出千里,决者至众,不可曲止也。子以吾言为不诚,试入诊太子,当闻其耳鸣而鼻张,循其两股以至于阴,当尚温也。”
\end{yuanwen}

后来扁鹊外出行医路过虢国。正赶上虢国太子病死,扁鹊就来到虢国宫门外,向喜好方术的中庶子询问道:“太子得了什么病,为什么全国都在举行祝祷活动而很多事都停止了呢?”

中庶子回答说:“太子患有气血失调的病,运行交错而无法宣泄,因此突然发作于外部,这就使内脏受到伤害。身体内部的正气无法抑制邪气,而邪气聚集起来而不能发散出来,因此太子的阳脉松弛而阴脉急迫,所以突然间昏倒而死。”

扁鹊问:“太子死了多长时间?”

中庶子答道:“从鸡鸣时一直到现在。”

扁鹊又问:“他的尸体收殓了吗?”

中庶子答道:“还没有收殓,太子死去还不到半天。”

扁鹊说:“请您转告国君,就说我是齐国渤海郡的秦越人,家住郑地,没有得到机会仰望国君的神采而拜见并侍奉在他身边。听说太子不幸亡故,我可以让他活过来。”

中庶子说:“先生不会是在胡言乱语吧?您凭什么说太子能够活过来呢!我听说上古时代,有位著名的医生叫俞跗,他治病不用汤剂、药酒、石针、导引、按摩、药熨,只要解开衣服看一下就知道疾病所在,然后沿着五脏的腧穴,剖开皮肉,疏通经脉,结扎筋腱,按动髓脑,触动膏肓,疏理隔膜,清理肠胃,洗涤五脏,修炼精气,变换形体。先生的医术如果能够达到这种程度,那么太子就能死而复生了;如果先生达不到这种程度却想让太子复活,这简直连刚刚会笑的婴儿都骗不了。”

许久,扁鹊仰天长叹道:“先生所说的那些治疗方法,就好比从竹管里窥视天空,从缝隙中观看花纹。我的治疗方法,不需要切脉、看气色、听声音、观察病人的身体形态,就能够说出病症所在。知道疾病的外在表现就可以推知内在的病因;知道疾病的内因就可以推知其外在表现。人体的疾病会通过外表反映出来,根据这一点,就可以为千里之外的病人诊断,诊病的方法很多,决不能只停留在单一的角度看问题。您觉得我的话不真实,可以试试进去诊察太子,您会听到他耳中有鸣响,看到他鼻翼微微张合,沿着他的双腿一直摸到阴部,您会觉得那里还有些温热。”

\begin{yuanwen}
中庶子闻扁鹊言,目眩然而不瞚\footnote{通“瞬”,眨眼。},舌挢然\footnote{翘起的样子。}而不下\footnote{text},乃以扁鹊言入报虢君。虢君闻之大惊,出见扁鹊于中阙,曰:“窃闻高义之日久矣,然未尝得拜谒于前也。先生过小国,幸而举之,偏国寡臣幸甚。有先生则活,无先生则弃捐填沟壑,长终而不得反。”

言未卒,因嘘唏服臆\footnote{同“愊臆”,因悲伤而气满郁塞。},魂精泄横,流涕长潸,忽忽承䀹\footnote{text},悲不能自止,容貌变更。

扁鹊曰:“若太子病,所谓‘尸蹶’者也。夫以阳入阴中,动胃繵缘\footnote{text},中经维\footnote{阻塞,壅塞。}络,别下于三焦、膀胱,是以阳脉下遂\footnote{text},阴脉上争,会气闭而不通\footnote{text},阴上而阳内行,下内鼓而不起,上外绝而不为使,上有绝阳之络,下有破阴之纽,破阴绝阳,色废脉乱,故形静如死状。太子未死也。夫以阳入阴支兰藏者生\footnote{text},以阴入阳支兰藏者死。凡此数事,皆五藏蹙中之时暴作也。良工取之\footnote{text},拙者疑殆\footnote{text}。”
\end{yuanwen}

中庶子听了扁鹊的话之后,目光呆滞,瞪大了眼睛眨也不眨,舌头翘着放不下来,于是他进去把扁鹊的话告诉了虢国国君。虢君听后十分惊讶,连忙出来在宫廷的中门召见扁鹊,对他说:“我久闻您崇高的品德,可是一直没有机会拜见您。先生路过我们这个小国,希望您能帮助我们,作为这个偏僻小国的国君,我实在太荣幸了。有先生在,太子就能活过来;如果没有先生,太子就只能抛尸荒野而填塞沟壑,永远也回不来了。”

他的话还没有说完,就悲伤抽泣,气郁胸中,精神恍惚,涕泪长流,闪闪的泪珠沾在睫毛上,悲痛得不能自已,就连容貌、神情都变了样子。

扁鹊说:“像太子现在得的这种病,就是所谓的‘尸蹶’。这种病是由于阳气进入阴脉之中,脉气缠绕对胃造成冲击,经脉受到损伤,脉络受到阻塞,分别下注入三焦、膀胱,所以阳脉下坠,阴脉上升,阴阳二气相互争扰,使得气闭不通,阴气逆而上行,阳气在内运行,下气在内鼓动而不能上升,上气在外被阻绝而不能被役使,在上有隔绝阳气的脉络,下有破坏阴气的筋纽,阴气被破坏、阳气被阻隔,就使人面色衰败、血脉混乱,所以会使身体安静得如同死去一样。其实太子并没有真的死去。阳气入袭阴气而阻绝脏气的疾病可以治愈,如果是阴气入袭阳气而阻绝脏气,那就必死无疑。这些情况,都是在五脏气机逆乱的时候突然发作的。良医能够把握变化的病理,而庸医却会因为困惑而使病人更加危险。”

\begin{yuanwen}
扁鹊乃使弟子子阳厉鍼砥石\footnote{text},以取外三阳五会\footnote{text}。有间,太子苏。乃使子豹为五分之熨,以八减之齐和煮之\footnote{text},以更熨两胁下。太子起坐。更适阴阳,但服汤二旬而复故。故天下尽以扁鹊为能生死人。

扁鹊曰:“越人非能生死人也,此自当生者,越人能使之起耳。”
\end{yuanwen}

扁鹊于是吩咐他的弟子子阳打磨针石,然后用石针来刺三阳、五会等经络腧穴。过了不大一会儿,太子就苏醒过来了。扁鹊就让弟子子豹准备能够入体五分的药熨,与八减方的药剂一起煎煮之后交替在太子的两胁下熨治。这样一来太子就能坐起来了。此后扁鹊又进一步调和太子体内的阴阳二气,只服用了二十天汤药身体就复原了。所以天下人都以为扁鹊能够令死人复活。

扁鹊说:“其实我并不能让死人复活,因为病人本来就是活的,我只不过是让他恢复健康罢了。”

\begin{yuanwen}
扁鹊过齐,齐桓侯客之\footnote{text}。入朝见,曰:“君有疾在腠理\footnote{皮肤与肌肉之间。},不治将深。”

桓侯曰:“寡人无疾。”

扁鹊出,桓侯谓左右曰:“医之好利也,欲以不疾者为功。”

后五日,扁鹊复见,曰:“君有疾在血脉,不治恐深。”

桓侯曰:“寡人无疾。”

扁鹊出,桓侯不悦。

后五日,扁鹊复见,曰:“君有疾在肠胃间,不治将深。”

桓侯不应。扁鹊出,桓侯不悦。

后五日,扁鹊复见,望见桓侯而退走。

桓侯使人问其故。扁鹊曰:“疾之居腠理也,汤熨之所及也;在血脉,鍼石之所及也;其在肠胃,酒醪之所及也;其在骨髓,虽司命无奈之何。今在骨髓,臣是以无请也。”

后五日,桓侯体病\footnote{text},使人召扁鹊,扁鹊已逃去。桓侯遂死。
\end{yuanwen}

扁鹊来到齐国,齐桓侯把他当成贵客接待。他入宫拜谒齐桓侯,说:“国君有小病在皮肤与肌肉之间,如果不及时治疗,病情将会加重。”

齐桓侯说道:“我没有病。”

扁鹊出去以后,桓侯对身边的人说:“医生总是喜好功利,想通过治疗没病的人来显示自己的业绩。”

五天以后,扁鹊又进宫拜见,说:“您的病已经到了血脉之中,如果不医治的话就会加重了。”

桓侯又说:“我没病。”

扁鹊出去以后,桓侯很不高兴。又过了五天,扁鹊再次进宫拜见,说:“您的病已到了肠胃之间,再不治疗就会加重了。”

这一次桓侯没有回答他。扁鹊出去以后,桓侯很不高兴。又过了五天,扁鹊再次进宫拜见,可是他看见桓侯以后就退出去跑掉了。

桓侯派人前去问他为什么这样做。扁鹊回答说:“疾病在皮肉之间,凭借汤剂、药熨的效力就可以达到治疗效果;病在血脉之中,依靠针刺和砭石的效力就可以达到治疗效果;病在肠胃之中,需要利用药酒达到治病的目的;如果病在骨髓,即便是掌管生命的神也没有什么办法了。如今国君的病已经到了骨髓,我因此不敢请求为他治疗了。”

五天以后,桓侯果然患了重病,于是派人去召扁鹊,可是扁鹊早就逃走了。桓侯就这样病死了。

\begin{yuanwen}
使圣人预知微,能使良医得蚤从事\footnote{治疗。},则疾可已,身可活也。人之所病,病疾多;而医之所病,病道少。故病有六不治:骄恣不论于理,一不治也;轻身重财,二不治也;衣食不能适,三不治也;阴阳并,藏气不定,四不治也;形羸不能服药,五不治也;信巫不信医,六不治也。有此一者,则重难治也。
\end{yuanwen}

如果圣明之人能够预先知道没有显露出来的疾病征兆,能够让良医对自己及时进行治疗,那么疾病就可以被治好,病人也能够存活。人们所忧虑的,是疾病太多;而医生所忧虑的,是治疗的方法太少。因此,有六种患病的情形不能医治:骄纵而不讲道理,是第一种不能医治的情形;轻视自己的身体而重视钱财,是第二种;衣着饮食调节不当,是第三种;阴阳错乱,脏腑精气失调,是第四种;身体过于羸弱,不能服药,是第五种;相信巫术而不相信医术,是第六种。如果有上述六种情形中的一种,疾病就很难治疗了。

\begin{yuanwen}
扁鹊名闻天下。过邯郸,闻贵妇人,即为带下医\footnote{text};过雒阳,闻周人爱老人,即为耳目痹医;来入咸阳,闻秦人爱小儿,即为小儿医:随俗为变。秦太医令李醯自知伎不如扁鹊\footnote{text}也,使人刺杀之。至今天下言脉者,由扁鹊也\footnote{text}。
\end{yuanwen}

扁鹊名声传遍天下。他经过赵国都城邯郸时,听说当地有尊重妇女的习俗,就做了妇科医生;经过雒阳,听说当地人敬重老人,他就做了专门治疗耳聋、眼花以及四肢痹痛等疾病的医生;到了咸阳,他听说秦国人爱护小孩,于是就做了儿科医生:总之,他随着各地的风俗习惯来改变自己的主治方向。秦国的太医令名叫李醯,他知道自己的医术比不上扁鹊,就派人刺杀了他。直到今天,天下谈论诊脉的人,都遵循扁鹊的理论。

\begin{yuanwen}
太仓公者,齐太仓长,临菑人也,姓淳于氏,名意。少而喜医方术。

高后八年,更受师同郡元里公乘阳庆。庆年七十余,无子,使意尽去其故方,更悉以禁方予之,传黄帝、扁鹊之《脉书》,五色诊病,知人死生,决嫌疑,定可治,及药论,甚精。受之三年,为人治病,决死生多验。然左右行游诸侯,不以家为家,或不为人治病,病家多怨之者。
\end{yuanwen}

太仓公,是齐国负责管理都城粮仓的长官,临菑人,复姓淳于,名意。他年轻的时候非常喜欢研究医术。

汉高后八年(前180年),他再次拜同郡元里的公乘阳庆为师学习医术。当时,阳庆已经七十多岁了,没有子嗣,他让淳于意把过去学到的医方全部抛弃掉,然后把自己的秘方全部给了他,还向他传授黄帝、扁鹊的《脉书》,以及通过观察面部不同颜色进行诊断的方法,以此来预知病人的生死,决断疑难疾病,决定是否可以医治,还传授给他有关药物的理论,非常精辟。淳于意学习三年之后,给人治病,判断生死,多能应验。然而,他在各诸侯国之间行医,不把自己的家当作家,有时候不愿给人治病,所以很多病人都对他心怀怨恨。

\begin{yuanwen}
文帝四年中,人上书言\footnote{控告,议论。}意,以刑罪当传\footnote{驿车。}西之长安。意有五女,随而泣。意怒,骂曰:“生子不生男,缓急无可使者!”

于是少女缇萦伤父之言,乃随父西。上书曰:“妾父为吏,齐中称其廉平,今坐法当刑。妾切痛死者不可复生而刑者不可复续,虽欲改过自新,其道莫由\footnote{无路可走。},终不可得。妾原入身为官婢,以赎父刑罪,使得改行自新也。”

书闻,上悲其意,此岁中亦除肉刑法。

\end{yuanwen}

文帝四年(前176年),有人上书朝廷状告淳于意,根据其罪行,应当用驿车将他向西押解到京城长安。淳于意有五个女儿,她们都跟在父亲身后哭泣。淳于意十分恼怒,骂道:“生孩子不生男孩,在危急时刻就没有能派上用场的!”

最小的女儿缇萦听了父亲的话很是伤感,于是跟随父亲西行。到了长安以后,缇萦上书朝廷说:“我的父亲是朝廷任命的官吏,齐国人都称赞他廉洁、公正,如今他因为犯法而被判刑。我很痛心死者不能复生而受刑致残者也不能康复,他们就算是想改过自新,也无路可走,终究实现不了愿望。我愿意进入官府做奴婢,以此来为父亲赎罪,使他能够获得改过自新的机会。”

皇帝看过奏书以后,十分怜悯她的心意,于是将淳于意赦免,并在这一年废除了肉刑。

\begin{yuanwen}
意家居,诏召问所为治病死生验者几何人也,主名为谁。

诏问故太仓长臣意:“方伎\footnote{即“方技”,泛指医术、医药等相关知识。}所长,及所能治病者?有其书无有?皆安受学?受学几何岁?尝有所验,何县里人也?何病?医药已,其病之状皆何如?具悉而对。”

臣意对曰:

自意少时,喜医药,医药方试之多不验者。至高后八年,得见师临菑元里公乘阳庆。庆年七十馀,意得见事之。谓意曰:“尽去而方书,非是也。庆有古先道遗传黄帝、扁鹊之《脉书》,五色诊病,知人生死,决嫌疑,定可治,及药论书,甚精。我家给富,心爱公,欲尽以我禁方书悉教公。”臣意即曰:“幸甚,非意之所敢望也。”臣意即避席再拜谒,受其脉书上下经、五色诊、奇咳术、揆度、阴阳外变、药论、石神、接阴阳禁书,受读解验之,可一年所。明岁即验之,有验,然尚未精也。要事之三年所,即尝已为人治,诊病决死生,有验,精良。今庆已死十年所,臣意年尽三年,年三十九岁也。
\end{yuanwen}

淳于意住在家里,皇上下诏书问他为人治病诊断生死应验的有多少人,这些病人都叫什么名字。

诏书问前任太仓长淳于意:“你的医术有什么专长,能治疗哪些疾病?有没有医书?在哪里学的医术?学了多少年?曾经治好的病人,都是什么地方的人?他们得了什么病?用药治疗以后,他们的病情都怎样?请全部详细地回答这些问题。”淳于意回答说:

我从年轻的时候开始,就非常喜欢医术,曾经试着用医术方剂给人治病,可是很多时候都没有效验。到了高后八年,我得以向临菑元里的公乘阳庆学习医术。那时候阳庆已经七十多岁了,我得以拜谒并侍奉他。他对我说:“把你过去所学的医书全部抛弃掉,这些都是不对的。我有古代前辈医家传下来的黄帝、扁鹊的《脉书》,以及通过观察面部不同颜色进行诊断的方法,以此来预知病人的生死,决断疑难疾病,决定是否可以医治,以及有关药物的理论,非常精辟。我家里比较富裕,心里很喜欢您,所以打算把我珍藏的秘方全部传授给您。”我当即说道:“真是荣幸,这简直不是我敢奢望的。”我马上离开座席对他拜了两次,接受了他传给我的《脉书上下经》《五色诊》《奇咳术》《揆度》《阴阳外变》《药论》《石神》《接阴阳》等医学秘书,加以学习、理解和体验,大约用了一年的时间。到了第二年,我就开始对这些医术加以应用,虽然颇有成效,但还不是十分精湛。我大约向阳庆学习了三年,曾经为病人医治,诊治病情决断生死,颇有效验,医术十分精湛。如今,阳庆已死了大约十年,我曾经向他学习过三年,现在我已经三十九岁了。

\begin{yuanwen}
齐侍御史成自言病头痛,臣意诊其脉,告曰:“君之病恶,不可言也。”

即出,独告成弟昌曰:“此病疽\footnote{毒疮。}也,内发于肠胃之间,后五日当(鍮/臃)肿,后八日呕脓死。”

成之病得之饮酒且内。成即如期死。所以知成之病者,臣意切其脉,得肝气。肝气浊而静,此内关之病也。脉法曰“脉长而弦,不得代四时者,其病主在于肝。和即经主病也,代则络脉有过”。经主病和者,其病得之筋髓里。其代绝而脉贲者,病得之酒且内。所以知其后五日而(鍮/臃)肿,八日呕脓死者,切其脉时,少阳初代。代者经病,病去过人,人则去。络脉主病,当其时,少阳初关一分,故中热而脓未发也,及五分,则至少阳之界,及八日,则呕脓死,故上二分而脓发,至界而(鍮/臃)肿,尽泄而死。热上则熏阳明,烂流络,流络动则脉结发,脉结发则烂解,故络交。热气已上行,至头而动,故头痛。
\end{yuanwen}

齐国有一位侍御史名叫成,他说自己有头疼病,我为他诊脉之后,告诉他说:“你的病非常严重,不能一下子说清楚。”

然后我便出来,单独告诉成的弟弟昌说:“他得的是疽病,这种病在肠胃之间发生,五天以后就会发肿,再过八天就会吐脓血而死。”

成的病是由于酗酒之后行房事而引起的。后来,成果然如期而死。我之所以知道成的病因,是因为我在为他切脉时,感觉到了他肝脏有病的脉气。他的肝气重浊而平静,这是内部严重而外表不太明显的疾病。脉象理论里讲“脉长而如同弓弦一般挺直,不能随着四季变化而更替,这种情况表明病在肝脏。脉虽然长而直硬却很均匀,就表示肝的经脉有病,如果出现时快时慢的代脉,就说明肝的络脉有病”。肝的经脉有病,而脉象均匀的,他的疾病来源于筋髓。脉象时快时慢,忽而停止忽而有力的,其病因是酒色过度。我之所以知道他五天以后会有毒疮肿起,再过八天会吐脓血而死,是因为在给他切脉时,少阳经络出现了代脉的脉象。代脉说明经脉生病,病情遍布全身,患者就会死去。络脉出现病症,此时,左手关部一分处就出现代脉的脉象,这是由于热气积郁体内而脓血没有排出,到了关上五分处,就是少阳经脉的边界,到了八天以后,就会吐脓血而死,这是因为到了关上二分处就会产生脓血,到了少阳经脉边界处就会肿胀,最后疮破脓泄而死。内热就会熏灼阳明经脉,同时灼伤络脉的分支,如果络脉出现病变,经脉就会郁结发肿,之后就会糜烂离解,因此络脉之间交互阻塞。热气上侵到达头部,头部被侵扰,所以常常头痛。

\begin{yuanwen}
齐王中子诸婴儿小子病,召臣意诊切其脉,告曰:“气鬲\footnote{同“膈”。}病。病使人烦懑,食不下,时呕沫。病得之(心)忧,数忔食饮。”

臣意即为之作下气汤以饮之,一日气下,二日能食,三日即病愈。所以知小子之病者,诊其脉,心气也,浊躁而经也,此络阳病也。脉法曰“脉来数疾去难而不一者,病主在心”。周身热,脉盛者,为重阳。重阳者,逿\footnote{通“荡”,侵犯。}心主。故烦懑食不下则络脉有过,络脉有过则血上出,血上出者死。此悲心所生也,病得之忧也。
\end{yuanwen}

齐王二儿子所生的最小的男孩得了病,召我前去切脉诊治,我诊完之后告诉他说:“孩子得的是气膈病。这种病会令人烦闷,吃不下东西,还经常呕出胃液。其病因是心情抑郁,经常厌食。”我当即开了下气汤让孩子服用,服药一天之后,膈气下消,两天后就可以吃东西,三天以后就痊愈了。我之所以知道这孩子的病情,是因为我在为他诊脉时,感觉到了心脏有病的脉气,他的脉象浊重而急躁,这是阳络有病。脉象理论说“脉来时壮盛迅速,去时艰涩,来去前后不一,其病根在于心脏”。浑身发热,脉象壮盛,这叫作重阳,也就是阳热过重。阳热一旦过重,就会扰乱心神。故而心中烦闷,吃不下饭食,这样一来络脉就会有病,络脉一旦有病,就会有血液向上冲出的危险,这样的话人就会死亡。这都是内心悲伤引起的,病因源于忧郁。

\begin{yuanwen}
齐郎中令循病,众医皆以为蹙\footnote{气逆。}入中,而刺之。臣意诊之,曰:“涌疝\footnote{腹痛胀满,气逆冲上。}也,令人不得前后溲\footnote{小便,大便。}。”

循曰:“不得前后溲三日矣。”

臣意饮以火齐汤,一饮得前(后)溲,再饮大溲,三饮而疾愈。病得之内。所以知循病者,切其脉时,右口气急,脉无五藏气,右口脉大而数。数者中下热而涌,左为下,右为上,皆无五藏应,故曰涌疝。中热,故溺赤也。
\end{yuanwen}

齐国有位名叫循的郎中令得了病,很多医生都认为他的病是由于逆气进入胸腹而引起的,于是用针刺法治疗。我诊断之后说:“他患的是涌疝,这种病会让人无法排泄大小便。”

循说:“我不能大小便已经整整三天了。”

我让他服用火剂汤,只喝下一剂就能够大小便了,喝了两剂大小便就十分畅通,喝下第三剂病就全好了。这种病是由于房事引起的。我之所以知道循所患的疾病,是因为我在为他诊脉时,他右手寸口的脉象十分急迫,脉象根本反映不出五脏所患的疾病,右手寸口脉象壮盛而频繁。脉象又快又频是中焦、下焦积存的热邪在涌动,他的左手脉象急迫是热邪向下流,右手脉象急迫是热邪向上涌,都反映不出五脏的脉气,所以称这种病为“涌疝”。体内积热,所以尿液呈现赤红色。

\begin{yuanwen}
齐中御府长信病,臣意入诊其脉,告曰:“热病气也。然暑汗,脉少衰,不死。”

曰:“此病得之当浴流水而寒甚,已则热。”

信曰:“唯,然!往冬时,为王使于楚,至莒县阳周水,而莒桥梁颇坏,信则(揽/擥)车辕未欲渡也,马惊,即堕,信身入水中,几死,吏即来救信,出之水中,衣尽濡\footnote{湿。},有间而身寒,已热如火,至今不可以见寒。”

臣意即为之液汤火齐逐热,一饮汗尽,再饮热去,三饮病已。即使服药,出入二十日,身无病者。所以知信之病者,切其脉时,并阴。脉法曰“热病阴阳交者死”。切之不交,并阴。并阴者,脉顺清而愈,其热虽未尽,犹活也。肾气有时间浊,在太阴脉口而希,是水气也。肾固主水,故以此知之。失治一时,即转为寒热。
\end{yuanwen}

齐国有一位名为信的中御府长患病,我进入他的房间为他诊脉,告诉他说:“你这是热病的脉气。然而由于天气炎热出汗较多,脉象略显微弱,但不会因此而死亡。”

我接着说道:“患这种病是由于在流水中洗浴时,感到十分寒冷,寒冷过后便浑身发热。”

信说:“嗯,是这样的!去年冬天,我奉齐王之命出使楚国,到达莒县阳周水边时,我看到莒桥损坏得非常严重,就揽住车辕不想过河,驾车的马受惊,就坠落河中,我也跟着落入水中,差点淹死,手下的官吏立即来救我,我才从水中出来,当时衣服完全湿透了,不一会儿就觉得身上发冷,冷过之后全身便发热如火,一直到现在也不能受寒。”

我当即给他开了液汤火剂退热,只喝下一剂之后便消汗了,喝下第二剂便退热了,喝完三剂病就好了。我让他继续服药,过了大约二十天,他的身体就像没病一样。我之所以知道信所患的疾病,是因为我在为他切脉时,发现他的脉象全都是阴脉。脉象理论说“内热、外热交杂错乱的人会死亡”。我为他切脉时,并没有发现交杂错乱的现象,都是阴脉。全都是阴脉,脉象顺的可以用清法治愈,热邪虽然没有完全清除,但仍然可以保住性命。我诊断时发现他的肾气有时重浊,在太阴寸口依稀能够感觉到这种情形,那就是水气。肾原本是主管水液运行的,由此便可知道他的病情。这种病如果一时失治,就会转为寒热病。

\begin{yuanwen}
齐王太后病,召臣意入诊脉,曰:“风瘅客脬\footnote{风热侵入膀胱。},难于大小溲,溺赤。”

臣意饮以火齐汤,一饮即前后溲,再饮病已,溺如故。病得之流汗出㵌\footnote{同“滫”,小便。}。㵌者,去衣而汗晞\footnote{干。}也。所以知齐王太后病者,臣意诊其脉,切其太阴之口,湿然风气也。脉法曰“沈之而大坚,浮之而大紧者,病主在肾”。肾切之而相反也,脉大而躁。大者,膀胱气也;躁者,中有热而溺赤。
\end{yuanwen}

齐国的太后患病,召我进宫诊脉,我诊过之后说:“太后的病是风热侵入膀胱,大小便很困难,尿液呈赤红色。”

我给她服用了火剂汤,喝下一剂大小便就通畅了,服用两剂病就好了,尿色也和原来一样了。这种病的病因是在出汗时解小便。所谓“㵌”,就是脱去衣服之后汗水被吹干而着凉。我之所以知道太后的病情,是因为当我为她切脉时,发现太阴寸口湿润,这明显是受风的脉气。脉象理论说“用力切脉时脉象又沉又大又有力,轻轻切脉时脉象大而紧张,这是肾脏有病的征兆”。但是我在切肾脉的时候,情况刚好相反,脉象粗大而躁动。粗大的脉象说明膀胱有病;躁动的脉象则说明中焦有热邪,所以尿色赤红。

\begin{yuanwen}
齐章武里曹山跗病,臣意诊其脉,曰:“肺消瘅也,加以寒热。”

即告其人曰:“死,不治。適其共养,此不当医治。”

法曰“后三日而当狂,妄起行,欲走;后五日死”。即如期死。山跗病得之盛怒而以接内。所以知山跗之病者,臣意切其脉,肺气热也。脉法曰“不平不鼓\footnote{脉搏起伏,鼓动无力。},形弊”。此五藏高之远数以经病也,故切之时不平而代。不平者,血不居其处;代者,时参击并至,乍躁乍大也。此两络脉绝,故死不治。所以加寒热者,言其人尸夺。尸夺者,形弊;形弊者,不当关灸鑱石\footnote{古时治病用的石针。}及饮毒药也。臣意未往诊时,齐太医先诊山跗病,灸其足少阳脉口,而饮之半夏丸,病者即泄注,腹中虚;又灸其少阴脉,是坏肝刚绝深,如是重损病者气,以故加寒热。所以后三日而当狂者,肝一络连属结绝乳下阳明,故络绝,开阳明脉,阳明脉伤,即当狂走。后五日死者,肝与心相去五分,故曰五日尽,尽即死矣。
\end{yuanwen}

齐国章武里的曹山跗生病,我前去为他诊脉,诊过之后说:“你得了肺消瘅,外加寒热症。”

我当即告诉他说“得了这种病必死无疑,根本没办法医治。你要适当调养,不应该继续医治了。”

医学理论说“得了这种病三天后就会发狂,妄自起来乱走;五天后就会死亡”。后来他果然如期死去。曹山跗的病源于大怒之后行房事。我之所以知道他的病情,是因为我在为他切脉的时候,发现他有肺气热。脉象理论说“脉象如果不平稳不鼓动,则病人身体羸弱”。这是五脏由上至下多次患病的结果,所以我在切脉时,脉象不平稳,并且有代脉现象。脉象不平稳,是因为血气不能归藏于肝脏;出现代脉现象,就是脉搏经常杂乱并起,时而浮躁,时而宏大。这是肺经、肝经断绝的表现,所以说这是必死无疑的不治之症。之所以说他还患有寒热症,是因为他精神涣散如同死尸一般。精神涣散如同死尸的人,身体必然羸弱;身体羸弱,就不能采用针灸方法治疗,也不能服用药性较为猛烈的药。在我前去诊治之前,齐国的太医已经先对他进行了诊治,在他的足少阳脉口用灸法熏烤,并让他服用半夏丸,结果病人立刻下泄,腹中虚弱;太医又在少阴脉用灸法熏烤,这样就重伤了他的肝的阳刚之气,像这样严重损伤病人的元气,所以他又外加寒热症。至于说他三天后会发狂,是因为肝的一条络脉横穿乳下与阳明经连结,络脉的横穿使得热邪侵入阳明经,阳明经一旦受伤,人就会狂奔。说他五天后会死,是因为肝脉和心脉相隔五分,肝脏之中的元气将在五天内耗尽,元气一旦耗尽,人也就死了。

\begin{yuanwen}
齐中尉潘满如病少腹痛,臣意诊其脉,曰:“遗积瘕也。”

臣意即谓齐太仆臣饶、内史臣繇曰:“中尉不复自止于内,则三十日死。”

后二十馀日,溲血死。病得之酒且内。所以知潘满如病者,臣意切其脉深小弱,其卒然合合也,是脾气也。右脉口气至紧小,见瘕气也。以次相乘,故三十日死。三阴俱抟者,如法;(痘/不)俱抟者,决在急期;一抟一代者,近也。故其三阴抟,溲血如前止。
\end{yuanwen}

齐国中尉潘满如得了小腹疼痛的病,我为他诊脉,说:“您的腹中遗留的气体积聚成为‘瘕症’。”

我就对齐国的太仆饶和内史繇说:“中尉大人如果再不停止房事,就会在三十天之内死去。”

结果过了二十多天,中尉就尿血而死。他的病是因为酗酒之后行房事而引起的。我之所以知道潘满如的病情,是因为我给他切脉时,发现他的脉象深沉,又小又弱,这三种脉象合在一起,是脾部有病的脉气。他右手寸口脉象弦紧而沉细,显现出瘕病的症状。根据五脏生克乘侮的次序,可断定三十天之内会死。太阴、少阴、厥阴三阴脉一同,正好符合三十天内必死的规律;三阴脉若不是一同出现,那么死亡的时间会更短;交会的阴脉与代脉交替出现,死期就会更近。所以他的三阴脉一齐出现,就像前面所说的那样尿血而亡。

\begin{yuanwen}
阳虚侯相赵章病,召臣意。众医皆以为寒中,臣意诊其脉曰:“迵风。”

迵风者,饮食下嗌而辄出不留。法曰“五日死”,而后十日乃死。病得之酒。所以知赵章之病者,臣意切其脉,脉来滑,是内风气也。饮食下嗌而辄出不留者,法五日死,皆为前分界法。后十日乃死,所以过期者,其人嗜粥,故中藏实,中藏实故过期。师言曰“安\footnote{消化容纳。}穀者过期,不安穀者不及期”。
\end{yuanwen}

阳虚侯的丞相赵章患病,召我去诊治。众医生都认为赵章的病是由于寒气进入体内所致,我给他诊过脉之后说:“他得的是‘迵风病’。”

患有迵风病的人,饮食咽下去以后,总是呕吐出来,无法留在胃里消化吸收。医学理论上说“得上此病五天就会死”,后来他过了十天才死。他的病因在于饮洒。我之所以知道赵章的病情,是因为我给他切脉时,脉象很滑,这是内风病的脉气。饮食下咽以后总是呕吐出来,医理说五天就会死,这是前面所说的“分界法”。后来他过了十天才死,之所以超过了期限,是因为病人喜欢喝粥,因此胃气充实,胃气充实才能超过期限。我的老师说“能够容纳消化水谷的人,可以超过期限而死;不能容纳消化水谷的人,期限没到就会死去”。

\begin{yuanwen}
济北王病,召臣意诊其脉,曰:“风蹶胸满。”

即为药酒,尽三石,病已。得之汗出伏地。所以知济北王病者,臣意切其脉时,风气也,心脉浊。病法“过入其阳,阳气尽而阴气入”。阴气入张,则寒气上而热气下,故胸满。汗出伏地者,切其脉,气阴。阴气者,病必入中,出及瀺水\footnote{汗水。}也。
\end{yuanwen}

济北王病了,召我去为他诊脉,我诊过以后说:“这是‘风厥’病,胸部烦闷。”

于是我就为他调配药酒,他喝了三石以后,病就好了。他的这种病是由于出汗的时候躺在地上而引起的。我之所以知道济北王的病情,是因为我给他切脉的时候,感觉到风邪的脉气,心脉十分重浊。病理说“病邪侵入体表,体表的阳气便会消耗殆尽,而阴寒之气则会乘机侵入”。阴寒之气进入人体以后就会扩张开来,寒气就会向上逆行,而阳气则向下运行,所以有胸闷的症状。我之所以知道此病是由于出汗时躺在地上而引起的,是因为我在切脉的时候,脉气阴邪。出现这种脉象,必然是疾病已经进入体内,服用药酒之后,寒湿之气就会随汗液流出。

\begin{yuanwen}
齐北宫司空命妇出於病,众医皆以为风入中,病主在肺,刺其足少阳脉。臣意诊其脉,曰:“病气疝,客于膀胱,难于前后溲,而溺赤。病见寒气则遗溺,使人腹肿。”

出於病得之欲溺不得,因以接内。所以知出於病者,切其脉大而实,其来难,是蹶阴之动也。脉来难者,疝气之客于膀胱也。腹之所以肿者,言蹶阴之络结小腹\footnote{足厥阴肝经的络脉连接小腹。}也。蹶阴有过则脉结动,动则腹肿。臣意即灸其足蹶阴之脉,左右各一所,即不遗溺而溲清,小腹痛止。即更为火齐汤以饮之,三日而疝气散,即愈。
\end{yuanwen}

齐国北宫司空的夫人出於得了病,医生们都以为是风邪进入体内引起的疾病,病根在肺部,于是针刺足少阳经脉。我为她诊脉以后说:“她得的是疝气病,疝气会影响膀胱,导致大小便困难,尿液颜色赤红。得了这种病的人一遇到寒气就会遗尿,并且使人小腹肿胀。”

这种疾病的诱因是想小便而没有去小便,接着就行房事。我之所以知道出於夫人的病情,是因为给她切脉的时候,脉象大而充实,但是脉搏来时艰难,这是因为足厥阴肝经有变动。脉搏来时艰难,是因为疝气影响膀胱。小腹之所以会肿胀,是由于足厥阴络脉连接小腹。足厥阴脉有病,与它相连的部位自然也会有所变化,这种变化就导致了小腹肿胀。于是,我在她的足厥阴肝经施以灸法,左右各一穴,她就不再小便失禁了,尿液颜色也变清了,小腹的疼痛也止住了。然后我又配制火剂汤给她服用,三天之后,疝气消散,病也就痊愈了。

\begin{yuanwen}
故济北王阿母自言足热而懑,臣意告曰:“热蹶也。”

则刺其足心各三所,案之无出血,病旋已。病得之饮酒大醉。
\end{yuanwen}

过去济北王的乳母说自己脚心发热胸中郁闷,我告诉她说:“这是热厥病。”

我随即在她的足心各刺三穴,拔针时,按住针孔,不让血液流出,她的病很快就痊愈了。她的病源于饮酒大醉。

\begin{yuanwen}
济北王召臣意诊脉诸女子侍者,至女子竖,竖无病。臣意告永巷长曰:“竖伤脾,不可劳,法当春呕血死。”

臣意言王曰:“才人女子竖何能?”

王曰:“是好为方,多伎能,为所是案法新,往年市之民所,四百七十万,曹偶四人。”

王曰:“得毋有病乎?”

臣意对曰:“竖病重,在死法中。”

王召视之,其颜色不变,以为不然,不卖诸侯所。至春,竖奉剑从王之厕,王去,竖后,王令人召之,即仆于厕,呕血死。病得之流汗。流汗者,法病内重,毛发而色泽,脉不衰,此亦内(关)之病也。
\end{yuanwen}

济北王召我去给他的那些侍女诊病,当诊到名叫竖的侍女时,她看上去根本没有病。我对永巷长说:“竖的脾脏受了伤,千万不能劳累,从病理上看,她到了春天就会吐血而死。”

我问济北王:“这位才女竖有什么才能?”

济北王回答说:“她十分喜欢医术,并有多种技能,可以在旧有医术中创出新意,她是我去年从民间买来的,共花费四百七十万钱,买了和她一样的四个人。”

济北王问道:“她莫不是有病?”

我答道:“她病得十分严重,按照病理的说法,她会死去。”

济北王把竖召来观看,看到她的脸色没有什么变化,就认为我说得不对,所以没有将她卖给别的诸侯。到了春天,竖捧剑跟随济北王去厕所,济北王离开厕所之后,她仍然留在后面,济北王就派人去叫她,结果发现她扑倒在厕所里,吐血而死。她的这种源于流汗过多。流汗过多的病人,按照病理来讲是病重在内里,毛发、脸色都很润泽,脉象也不衰减,这也属于内关一类的疾病。

\begin{yuanwen}
齐中大夫病齲齿,臣意灸其左大阳明脉,即为苦参汤,日嗽三升,出入五六日,病已。得之风,及卧开口,食而不嗽。
\end{yuanwen}

齐国中大夫得了龋齿病,我对他的左手阳明经施以灸法,并当即为他开了苦参汤,让他每天用三升漱口,前后用了五六天的时间,病就好了。其病因是外感风邪,外加睡卧时张嘴,吃完饭不漱口。

\begin{yuanwen}
菑川王美人怀子而不乳,来召臣意。臣意往,饮以莨锽\footnote{药名,也称莨岩。}药一撮,以酒饮之,旋乳。臣意复诊其脉,而脉躁。躁者有馀病,即饮以消石一齐,出血,血如豆比五六枚。
\end{yuanwen}

菑川王的嫔妃怀孕难产,就召我前去诊治。我去了以后,让她用酒送服莨 药末一小撮,她很快就将胎儿产下。我再次为她诊脉,脉象躁动。脉象躁动说明还有别的病,于是就让她服用消石一剂,结果她的阴道内出血,流出五六枚像豆子一样大小的血块。

\begin{yuanwen}
齐丞相舍人奴从朝入宫,臣意见之食闺门外,望其色有病气。臣意即告宦者平。平好为脉,学臣意所,臣意即示之舍人奴病,告之曰:“此伤脾气也,当至春鬲塞不通,不能食饮,法至夏泄血死。”

宦者平即往告相曰:“君之舍人奴有病,病重,死期有日。”

相君曰:“卿何以知之?”

曰:“君朝时入宫,君之舍人奴尽食闺门外,平与仓公立,即示平曰,病如是者死。”

相即召舍人而谓之曰:“公奴有病不?”

舍人曰:“奴无病,身无痛者。”

至春果病,至四月,泄血死。所以知奴病者,脾气周乘五藏,伤部而交,故伤脾之色也,望之杀然黄,察之如死青之兹\footnote{草席,这里指死草。}。众医不知,以为大(蟲/虫),不知伤脾。所以至春死病者,胃气黄,黄者土气也,土不胜木,故至春死。所以至夏死者,脉法曰“病重而脉顺清者曰内关”,内关之病,人不知其所痛,心急然无苦。若加以一病,死中春;一愈顺,及一时。其所以四月死者,诊其人时愈顺。愈顺者,人尚肥也。奴之病得之流汗数出,炙于火而以出见大风也。
\end{yuanwen}

齐国丞相门客的奴仆随同主人上朝进入王宫,我看见他在宫门外面吃东西,发现他脸上有病气。我立即将此事告诉了名叫平的宦官。平喜好诊脉,并向我学习,我就把那位奴仆的病指给他看,告诉他:“这是脾脏受损的面色,到了春天胸膈就会阻塞不通,无法进食,根据病理来看,他到夏天就会泄血而死。”

宦官平立即前去告诉丞相:“您门客的奴仆身上有病,而且病得很重,距离死期不远了。”

丞相问:“你是怎么知道的?”

平说:“在您上朝入宫的时候,您门客的奴仆在宫门外面一直吃东西,我与太仓公站在那儿,太仓公指着那位奴仆对我说,得了这种病是要死人的。”

丞相立即把那个门客召来问道:“您手下的奴仆有病吗?”

门客回答:“我的奴仆没有病,身上也没有感觉到疼痛。”

到了春天,那位奴仆果然病了,到四月,他便泄血而亡。我之所以知道那位奴仆的病情,是因为知道他的脾脏之气周行于五脏,脾脏一旦受到损伤,身体各部位就会交错受损,所以脸上的某些部位会显示出相应的病色,这种脾脏受损的面色,看上去发黄,再仔细看,是死草一般的青灰色。许多医生都不了解这种情况,以为病人体内生有寄生虫,而不知道是脾脏受损。之所以说这个人到春天会病重而死,是因为患有脾胃疾病的人脸色发黄,在五行之中,黄色属土,脾土不能胜肝木,因此到了肝木旺盛的春季就会死去。之所以说他到夏天会死,是因为脉象理论说“病情十分严重,而脉象却很正常,这是内关病”,患有内关病的人感觉不到疼痛,心情急躁,好像没有任何痛苦。如果再添一种病,就会在仲春二月死去;如果能够保持心情愉快、顺天养性,则可延缓一季度。他之所以死于四月,是因为我对他进行诊断时,他心情愉快、顺天养性。他做到了这一点,身体还算丰满肥腴,所以能够拖延一段时间。他的病是由于流汗过多,受火烘烤后又在外面受风邪所致。

\begin{yuanwen}
菑川王病,召臣意诊脉,曰:“蹶上为重,头痛身热,使人烦懑。”

臣意即以寒水拊其头,刺足阳明脉,左右各三所,病旋已。病得之沐发未乾而卧。诊如前,所以蹶,头热至肩。
\end{yuanwen}

菑川王患病,召我前去为他诊脉,我诊过之后说道:“这是热邪逆行侵入头部而引起的‘蹶’病,其症状是头痛身热,令人烦闷。”

我于是用冷水拍在他头上,针刺他的足阳明经脉,左右各三次,他的病很快就痊愈了。病因是洗浴之后,头发没有晾干就睡眠。诊断如前面所述,之所以称之为蹶病,是因为热邪逆行于头部,一直到肩部。

\begin{yuanwen}
齐王黄姬兄黄长卿家有酒召客,召臣意。诸客坐,未上食。臣意望见王后弟宋建,告曰:“君有病,往四五日,君要胁痛不可(俯/俛)仰\footnote{低头和仰头。},又不得小溲。不亟治,病即入濡肾。及其未舍五藏,急治之。病方今客肾濡,此所谓“肾(痺/痹)”也。”

宋建曰:“然,建故有要脊痛。往四五日,天雨,黄氏诸倩见建家京下方石,即弄之,建亦欲效之,效之不能起,即复置之。暮,要脊痛,不得溺,至今不愈。”

建病得之好持重。所以知建病者,臣意见其色,太阳色乾,肾部上及界要以下者枯四分所,故以往四五日知其发也。臣意即为柔汤使服之,十八日所而病愈。

\end{yuanwen}

齐王黄姬的哥哥黄长卿家里摆设酒席招待客人,把我也召去。各位宾客就座,这时菜还没有端上来。我看到了王后的弟弟宋建,对他说:“您有病,在四五天以前,您的腰、胁部位疼痛,不能低头和仰头,并且不能小便。如果不及时医治,疾病就会侵入肾脏。趁现在病邪还没有进入五脏,应该赶快医治。现在病邪正在侵入肾脏,这就是所谓的‘肾痹’。”

宋建说:“确实是这样。我以前有过腰脊疼痛的毛病。在四五天前,正赶上下雨,黄家的几个女婿看见我家仓库墙下的方石,就去搬弄,我也跟他们学,可是举不起来,于是便放下。到了黄昏时分,就感觉腰脊疼痛,不能小便,一直到现在也没好。”

宋建的病是由于喜欢持重物而引起的。我之所以知道他的病情,是因为我观察他的面色,发现他太阳穴这个部位色泽枯干,肾部以及腰围以下有大约四分的部位枯干,所以知道他在四五日之前曾经发病。我立刻调制了柔汤让他服下,过了大约十八天,他就痊愈了。

\begin{yuanwen}
济北王侍者韩女病要背痛,寒热,众医皆以为寒热也。臣意诊脉,曰:“内寒,月事不下也。”

即窜\footnote{熏灸。}以药,旋下,病已。病得之欲男子而不可得也。所以知韩女之病者,诊其脉时,切之,肾脉也,啬而不属。啬而不属者,其来难,坚,故曰月不下。肝脉弦,出左口,故曰欲男子不可得也。
\end{yuanwen}

济北王身边有位姓韩的侍女患上了腰背疼痛的疾病,恶寒发热,医生们都认为她患的是寒热病。我为她诊脉之后说:“你这是内寒,来不了月经。”

于是用药为她熏灸,月经很快就来了,她的病也好了。这种病的成因是想得到男子却没有得到。我之所以知道她的病情,是因为我为她诊脉时,切到了肾的病脉,脉象艰涩而不连续。艰涩而不连续,又很坚固,所以月经不通。她的肝脉如弓弦一样强直而又细长,超出左手寸口的位置,所以说她想得到男子而又得不到。

\begin{yuanwen}
临菑氾里女子薄吾病甚,众医皆以为寒热笃,当死,不治。臣意诊其脉,曰:“蛲瘕\footnote{蛲虫积聚而成瘕块。}。”

蛲瘕为病,腹大,上肤黄粗,循之戚戚然。臣意饮以芫华\footnote{芫花。}一撮,即出蛲可数升,病已,三十日如故。病蛲得之于寒湿,寒湿气宛笃不发,化为蟲。臣意所以知薄吾病者,切其脉,循其尺,其尺索刺粗,而毛美奉发,是蟲气也。其色泽者,中藏无邪气及重病。
\end{yuanwen}

临菑氾里有个名叫薄吾的女人病得十分严重,医生们都认为她得了严重的寒热病,会死去,没有办法医治。我为她诊脉之后,说:“这是‘蛲瘕病’。”

得了这种病的人肚子大,腹部皮肤颜色发黄并且粗糙,用手触摸腹部,病人会感到难受。我让病人用水送服一小撮芫花,她当即排泄出数升蛲虫,病也就痊愈了,此后过了三十天,她的身体就和过去一样了。蛲瘕病是由于寒湿气引起的,寒湿气在体内积蓄过多,不能发散,因而变化为虫。我之所以知道薄吾的病情,是因为我为她切脉时,摸她的尺部脉位,此处皮肤十分粗糙,并且毛发枯焦卷曲,这是体内有虫的症状。她面色有光泽,是由于内脏没有邪气侵入,病不太重。

\begin{yuanwen}
齐淳于司马病,臣意切其脉,告曰:“当病迵风。迵风之状,饮食下嗌辄后之。病得之饱食而疾走。”淳于司马曰:“我之王家食马肝,食饱甚,见酒来,即走去,驱疾至舍,即泄数十出。”臣意告曰:“为火齐米汁饮之,七八日而当愈。”时医秦信在旁,臣意去,信谓左右阁都尉曰:“意以淳于司马病为何?”曰:“以为迵风,可治。”信即笑曰:“是不知也。淳于司马病,法当后九日死。”即后九日不死,其家复召臣意。臣意往问之,尽如意诊。臣即为一火齐米汁,使服之,七八日病已。所以知之者,诊其脉时,切之,尽如法。其病顺,故不死。

齐中郎破石病,臣意诊其脉,告曰:“肺伤,不治,当后十日丁亥溲血死。”即后十一日,溲血而死。破石之病,得之堕马僵石上。所以知破石之病者,切其脉,得肺阴气,其来散,数道至而不一也。色又乘之。所以知其堕马者,切之得番阴脉。番阴脉入虚里,乘肺脉。肺脉散者,固色变也乘也。所以不中期死者,师言曰:“病者安穀即过期,不安穀则不及期”。其人嗜黍,黍主肺,故过期。所以溲血者,诊脉法曰“病养喜阴处者顺死,养喜阳处者逆死”。其人喜自静,不躁,又久安坐,伏几而寐,故血下泄。

齐王侍医遂病,自练五石服之。臣意往过之,遂谓意曰:“不肖有病,幸诊遂也。”臣意即诊之,告曰:“公病中热。论曰“中热不溲者,不可服五石”。石之为药精悍,公服之不得数溲,亟勿服。色将发臃。”遂曰:“扁鹊曰“阴石以治阴病,阳石以治阳病”。夫药石者有阴阳水火之齐,故中热,即为阴石柔齐治之;中寒,即为阳石刚齐治之。”臣意曰:“公所论远矣。扁鹊虽言若是,然必审诊,起度量,立规矩,称权衡,合色脉表里有馀不足顺逆之法,参其人动静与息相应,乃可以论。论曰“阳疾处内,阴形应外者,不加悍药及鑱石”。夫悍药入中,则邪气辟矣,而宛气愈深。诊法曰“二阴应外,一阳接内者,不可以刚药”。刚药入则动阳,阴病益衰,阳病益箸,邪气流行,为重困于俞,忿发为疽。”意告之后百馀日,果为疽发乳上,入缺盆,死。此谓论之大体也,必有经纪。拙工有一不习,文理阴阳失矣。

齐王故为阳虚侯时,病甚,众医皆以为蹶。臣意诊脉,以为痺,根在右胁下,大如覆杯,令人喘,逆气不能食。臣意即以火齐粥且饮,六日气下;即令更服丸药,出入六日,病已。病得之内。诊之时不能识其经解,大识其病所在。

臣意尝诊安阳武都里成开方,开方自言以为不病,臣意谓之病苦沓风,三岁四支不能自用,使人瘖,瘖即死。今闻其四支不能用,瘖而未死也。病得之数饮酒以见大风气。所以知成开方病者,诊之,其脉法奇咳言曰“藏气相反者死”。切之,得肾反肺,法曰“三岁死”也。

安陵阪里公乘项处病,臣意诊脉,曰:“牡疝。”牡疝在鬲下,上连肺。病得之内。臣意谓之:“慎毋为劳力事,为劳力事则必呕血死。”处后蹴踘,要蹶寒,汗出多,即呕血。臣意复诊之,曰:“当旦日日夕死。”即死。病得之内。所以知项处病者,切其脉得番阳。番阳入虚里,处旦日死。一番一络者,牡疝也。

臣意曰:他所诊期决死生及所治已病众多,久颇忘之,不能尽识,不敢以对。

问臣意:“所诊治病,病名多同而诊异,或死或不死,何也?”对曰:“病名多相类,不可知,故古圣人为之脉法,以起度量,立规矩,县权衡,案绳墨,调阴阳,别人之脉各名之,与天地相应,参合于人,故乃别百病以异之,有数者能异之,无数者同之。然脉法不可胜验,诊疾人以度异之,乃可别同名,命病主在所居。今臣意所诊者,皆有诊籍。所以别之者,臣意所受师方適成,师死,以故表籍所诊,期决死生,观所失所得者合脉法,以故至今知之。”

问臣意曰:“所期病决死生,或不应期,何故?”对曰:“此皆饮食喜怒不节,或不当饮药,或不当针灸,以故不中期死也。”

问臣意:“意方能知病死生,论药用所宜,诸侯王大臣有尝问意者不?及文王病时,不求意诊治,何故?”对曰:“赵王、胶西王、济南王、吴王皆使人来召臣意,臣意不敢往。文王病时,臣意家贫,欲为人治病,诚恐吏以除拘臣意也,故移名数,左右不脩家生,出行游国中,问善为方数者事之久矣,见事数师,悉受其要事,尽其方书意,及解论之。身居阳虚侯国,因事侯。侯入朝,臣意从之长安,以故得诊安陵项处等病也。”

问臣意:“知文王所以得病不起之状?”臣意对曰:“不见文王病,然窃闻文王病喘,头痛,目不明。臣意心论之,以为非病也。以为肥而蓄精,身体不得摇,骨肉不相任,故喘,不当医治。脉法曰“年二十脉气当趋,年三十当疾步,年四十当安坐,年五十当安卧,年六十已上气当大董”。文王年未满二十,方脉气之趋也而徐之,不应天道四时。后闻医灸之即笃,此论病之过也。臣意论之,以为神气争而邪气入,非年少所能复之也,以故死。所谓气者,当调饮食,择晏日,车步广志,以適筋骨肉血脉,以泻气。故年二十,是谓“易
”。法不当砭灸,砭灸至气逐。”

问臣意:“师庆安受之?闻于齐诸侯不?”对曰:“不知庆所师受。庆家富,善为医,不肯为人治病,当以此故不闻。庆又告臣意曰:“慎毋令我子孙知若学我方也。””

问臣意:“师庆何见于意而爱意,欲悉教意方?”对曰:“臣意不闻师庆为方善也。意所以知庆者,意少时好诸方事,臣意试其方,皆多验,精良。臣意闻菑川唐里公孙光善为古传方,臣意即往谒之。得见事之,受方化阴阳及传语法,臣意悉受书之。臣意欲尽受他精方,公孙光曰:“吾方尽矣,不为爱公所。吾身已衰,无所复事之。是吾年少所受妙方也,悉与公,毋以教人。”臣意曰:“得见事侍公前,悉得禁方,幸甚。意死不敢妄传人。”居有间,公孙光间处,臣意深论方,见言百世为之精也。师光喜曰:“公必为国工。吾有所善者皆疏,同产处临菑,善为方,吾不若,其方甚奇,非世之所闻也。吾年中时,尝欲受其方,杨中倩不肯,曰“若非其人也”。胥与公往见之,当知公喜方也。其人亦老矣,其家给富。”时者未往,会庆子男殷来献马,因师光奏马王所,意以故得与殷善。光又属意于殷曰:“意好数,公必谨遇之,其人圣儒。”即为书以意属阳庆,以故知庆。臣意事庆谨,以故爱意也。”

问臣意曰:“吏民尝有事学意方,及毕尽得意方不?何县里人?”对曰:“临菑人宋邑。邑学,臣意教以五诊,岁馀。济北王遣太医高期、王禹学,臣意教以经脉高下及奇络结,当论俞所居,及气当上下出入邪逆顺,以宜鑱石,定砭灸处,岁馀。菑川王时遣太仓马长冯信正方,臣意教以案法逆顺,论药法,定五味及和齐汤法。高永侯家丞杜信,喜脉,来学,臣意教以上下经脉五诊,二岁馀。临菑召里唐安来学,臣意教以五诊上下经脉,奇咳,四时应阴阳重,未成,除为齐王侍医。”

问臣意:“诊病决死生,能全无失乎?”臣意对曰:“意治病人,必先切其脉,乃治之。败逆者不可治,其顺者乃治之。心不精脉,所期死生视可治,时时失之,臣意不能全也。”

太史公曰:女无美恶,居宫见妒;士无贤不肖,入朝见疑。故扁鹊以其伎见殃,仓公乃匿迹自隐而当刑。缇萦通尺牍,父得以后宁。故老子曰“美好者不祥之器”,岂谓扁鹊等邪?若仓公者,可谓近之矣。

上池祕术,长桑所传。始候赵简,知梦钧天。言占虢嗣,尸蹶起焉。仓公赎罪,阳庆推贤。效验多状,式具于篇。正义胃大一尺五寸,径五寸,长二尺六寸,横尺,受水穀三斗五升,其中常留穀二斗,水一斗五升。小肠大二寸半,径八分分之少半,长三丈二尺,受穀二斗四升,水六升三合合之大半。回肠大四寸,径一寸半,长二丈二尺,受穀一斗,水七升半。广肠大八寸,径二寸半,长二尺八寸,受穀九升三合八分合之一。故肠胃凡长五账八尺四寸,合受水穀八斗七升六合八分合之一,此肠胃长短受水穀之数也。肝重四斤四两,左三叶,右四叶,凡七叶,主藏魂。心重十二两,中有七孔,三毛,盛精汁三合,主藏神。脾重二斤三两,扁广三寸,长五寸,有散膏半斤,主血温五藏,主藏意#肺重三斤三两,六叶两耳,凡八叶,主藏魂魄。肾有两枚,重一斤一两,主藏志。胆在肝之短叶间,重三两三铢,盛精汁三合。胃重二斤十四两,纡曲屈申,长二尺六寸,大一尺五寸,径五寸,盛穀二斗,水一斗五升。小肠重二斤十四两,长三丈二尺,广二寸半,径八分分之少半,回积十六曲,盛穀二斗四升,水六升三合合之大半。大肠重三斤十二两,长二丈一尺,广四寸,径一寸半,当齐,右回十六曲,盛穀一斗水七升半。膀胱重九两二铢,纵广九寸,盛溺九升九合。口广二寸半。脣至齿长九分。齿已后至会厌,深三寸半,大容五合也。舌重十两,长七寸,广二寸半。咽门重十两,广二寸半,至胃长一尺六寸。喉咙重十二两,广二寸,长一尺二寸九节。肛门重十二两,大八寸,径二寸太半,长二尺八寸,受穀九升三合八分合之一。

手三阳之脉,从手至头长五尺,五六合三丈。手三阴之脉,从手至胸中长三尺五寸,三六一丈八尺,五六三尺,合二丈一尺。足三阳之脉,从足至头长八尺,六八合四丈八尺。足三阴之脉,从足至胸长六尺五寸,六六三丈六尺,五六三尺,合三丈九尺。人两足蹻脉,从足至目长七尺五寸,二七一丈四尺,二五一尺合一丈五尺。督任脉各长四尺五寸,二四八尺,二五一尺,合九尺。凡脉长一十六丈二尺也,此所谓十二经脉长短之数也。寸口,脉之大会,手太阴之动也。人一呼脉行三寸,一吸脉行三寸,呼吸定息,脉行六寸。人一日一夜凡一万三千五百息。脉行五十周于身,漏水下百刻。营卫行阳二十五度,行阴二十五度。度为一周也,故五十度复会于手太阴。寸口者,五藏六府之所终始,故法于寸口也。

肺气通于鼻,鼻和则知臭香矣。肝气通于目,目和则知白黑矣。脾气通于口,口和则知穀味矣。心气通于舌,舌和则知五味矣。肾气通于耳,耳和则闻五音矣。五藏不和,则九窍不通;六府不和,则留为痈也。
\end{yuanwen}\begin{yuanwen}

\end{yuanwen}\begin{yuanwen}

\end{yuanwen}\begin{yuanwen}

\end{yuanwen}\begin{yuanwen}

\end{yuanwen}\begin{yuanwen}

\end{yuanwen}\begin{yuanwen}

\end{yuanwen}

\begin{yuanwen}
	
	
	
\end{yuanwen}

齐淳于司马病,臣意切其脉,告曰:“当病迵风。迵风之状,饮食下嗌【嗌:呕吐。】 辄后之。病得之饱食而疾走。”淳于司马曰:“我之王家食马肝,食饱甚,见酒来,即走去,驱疾至舍,即泄数十出。”臣意告曰:“为火齐米汁饮之,七八日而当愈。”时医秦信在旁,臣意去,信谓左右阁都尉曰:“意以淳于司马病为何?”曰:“以为迵风,可治。”信即笑曰:“是不知也。淳于司马病,法当后九日死。”即后九日不死,其家复召臣意。臣意往问之,尽如意诊。臣即为一火齐米汁,使服之,七八日病已。所以知之者,诊其脉时,切之,尽如法。其病顺,故不死。

齐国的淳于司马生病,我为他诊脉,诊过以后对他说:“你得的应该是‘迵风’病,这种病的症状,是饮食吃下去以后很快就吐出来。病因是饱餐之后快跑。”淳于司马说:“我到君王家里吃马肝,吃得很饱,看到有酒端上来,我就跑开了,后来骑快马回到家中,刚一到家就下泄数十次。”我告诉他:“用米汁送服火剂汤,过七八天就能痊愈。”当时,医生秦信就在旁边,我走了以后,他问身边的阁都尉:“淳于意认为司马得的是什么病?”阁都尉说:“他认为是迵风病,可以治疗。”秦信听了以后笑着说:“他这是不了解病情。淳于司马的病,根据病理来看,九天后就会死去。”过了九天,司马并没有死,他的家人又召请我。我前去向他询问最近的病情,和我当初诊断的一模一样。我就开了火剂汤和米汁让他服用,过了七八天,他的病就全好了。我之所以知道他的病情,是因为我给他诊脉时,他的脉象完全符合常规。他的病情与脉象一致,所以说他不会死。

齐中郎破石病,臣意诊其脉,告曰:“肺伤,不治,当后十日丁亥溲血死。”即后十一日,溲血而死。破石之病,得之堕马僵石上。所以知破石之病者,切其脉,得肺阴气,其来散,数道至而不一也。色又乘之。所以知其堕马者,切之得番阴脉。番阴脉入虚里,乘肺脉。肺脉散者,固色变也乘之。所以不中期死者,师言曰“病者安谷即过期,不安谷则不及期”。其人嗜黍【黍:黄米。】 ,黍主肺,故过期。所以溲血者,诊脉法曰“病养喜阴处者顺死,养喜阳处者逆死”。其人喜自静,不躁,又久安坐,伏几而寐,故血下泄。



齐国一位名叫破石的中郎得病,我为他诊脉,诊过之后对他说:“你的肺部受伤,不能医治,你会在十天以后的丁亥日尿血而死。”此后过了十一天,他果然尿血而死。破石的病,是由于从马上摔下来跌在石头上引起的。我之所以知道他的病情,是因为我给他切脉时,他的肺阴脉脉搏来得散乱,好像是从几条脉道而来,很不一致。另外,他面色赤红,这是心脉抑制肺脉的表现。之所以知道他是从马上摔下来的,是因为我切到番阴脉。番阴脉进入虚里,然后侵袭肺脉。他的肺脉出现了散脉,原来的脸色就发生了变化,这正是心脉侵袭肺脉的表现。他之所以没有在预料的死期死亡,是因为我的老师说过“病人如果能够容纳吸收水谷,就能超过期限,如果不能容纳吸收水谷,不到死期就会死去”。破石非常喜欢吃黄米,黄米补肺,所以他才能超过期限。他之所以尿血,是因为脉象理论说“病人如果喜欢安静,就会血从下出而死,病人如果喜欢活动,就会血从上出而死”。破石这个人喜欢安静,性格不急躁,又长时间坐着不动,伏在几案上睡熟,所以血从下面泄出。

齐王侍医遂病,自练五石服之。臣意往过之,遂谓意曰:“不肖有病,幸诊遂也。”臣意即诊之,告曰:“公病中热。论曰‘中热不溲者,不可服五石’。石之为药精悍,公服之不得数溲,亟勿服。色将发臃。”遂曰:“扁鹊曰‘阴石以治阴病,阳石以治阳病’。夫药石者有阴阳水火之齐,故中热,即为阴石柔齐治之;中寒,即为阳石刚齐治之。”臣意曰:“公所论远矣。扁鹊虽言若是,然必审诊,起度量,立规矩,称权衡,合色脉表里有余不足顺逆之法,参其人动静与息相应,乃可以论。论曰‘阳疾处内,阴形应外者,不加悍药及镵石’。夫悍药入中,则邪气辟矣,而宛气愈深。诊法曰‘二阴应外,一阳接内者,不可以刚药’。刚药入则动阳,阴病益衰,阳病益箸【箸:通“著”,明显。】 ,邪气流行,为重困于俞,忿发为疽。”意告之后百余日,果为疽发乳上,入缺盆,死。此谓论之大体也,必有经纪【经纪:原则,要领。】 。拙工有一不习,文理阴阳失矣。

齐王故为阳虚侯时,病甚,众医皆以为蹶。臣意诊脉,以为痹,根在右胁下,大如覆杯,令人喘,逆气不能食。臣意即以火齐粥且饮,六日气下;即令更服丸药,出入六日,病已。病得之内。诊之时不能识其经解,大识其病所在。

齐王身边有一位侍医名叫遂,他得了病,自己炼制五石药服用。我前去拜访他,他对我说:“我得了病,希望你能为我诊治。”我随即便为他诊治,诊过以后告诉他说:“您所患的是热邪侵入内脏的病。病理说‘内脏有热邪侵入,无法小便,不能服用五石药’。石药的药力极为猛烈,您服用之后小便次数会减少,赶快停止服用。从你的面色看来,你将要生痈疽。”遂说:“扁鹊曾经说过‘性寒的石药可以治疗阴虚有热的疾病,性热的石药可以治疗阳虚有寒的疾病’。药石方剂的性质有阴阳寒热的区别,因此,内脏有热病,就就用阴石柔剂来治疗;内脏有寒症,就用阳石刚剂来治疗。”我说:“您的说法错了。扁鹊虽然这样说过,但是必须要审慎地诊断,确定用药标准,确定治疗方法,反复衡量,将诊色与诊脉、表与里、有余与不足、顺与逆结合起来,同时参验病人的举止与呼吸是否和谐,之后才能下结论。医药理论说‘体内有热病,体表反映出阴冷症状的,不能用猛药及砭石的方法治疗’。猛烈的药物一旦进入体内,热邪之气就会更加恣肆,郁热也就蓄积更深。诊病理论说‘外部寒邪多于内部热邪的病,不能用性质猛烈的药来治疗’。猛烈的药物进入体内就会使阳气躁动,阴虚病症就会变得更加严重,阳气则变得更加强盛,邪气四处游走,就会层层聚结在腧穴周围,最后发展成为痈疽。”我对他说这些话之后过了一百多天,果然有痈疽生在乳上,当痈疽蔓延到锁骨上窝以后,他就死了。这就是说理论只表述大体情形,医者一定要掌握其中的要领。医术拙劣的医生如果有一处没有学到,就会使诊治失去条理,辨别阴阳发生错乱。

齐王过去担任阳虚侯的时候,有一次病得很重,医生们都认为他得了是蹶病。我为他诊脉,认为他得的是痹症,病根在右胁下面,大小如同倒扣着的杯子,使人气喘,逆气上行,所以无法进食。我就让他服用火剂粥,前后共六天,逆气便平降下来;于是我又让他改服丸药,前后又经过六天,他的病就好了。他的病是由于房事引起的。我在为他诊治时,不知道该如何用经脉理论来解释这种病,只是大体上掌握了疾病所在的部位。

臣意尝诊安阳武都里成开方,开方自言以为不病,臣意谓之病苦沓风,三岁四支不能自用,使人喑,喑即死。今闻其四支不能用,喑而未死也。病得之数饮酒以见大风气。所以知成开方病者,诊之,其脉法《奇咳》言曰“藏气相反者死”。切之,得肾反肺,法曰“三岁死”也。

安陵阪里公乘项处病,臣意诊脉,曰:“牡疝shàn 。”牡疝在鬲下,上连肺。病得之内。臣意谓之:“慎毋为劳力事,为劳力事则必呕血死。”处后蹴cù 鞠,要蹶寒【要蹶寒:腰部寒冷。蹶,通“厥”,冷。】 ,汗出多,即呕血。臣意复诊之,曰:“当旦日日夕死。”即死。病得之内。所以知项处病者,切其脉得番阳。番阳入虚里,处旦日死。一番一络者,牡疝也。

臣意曰:他所诊期决死生及所治已病众多,久颇忘之,不能尽识,不敢以对。

我曾经为安阳武都里的成开方诊断疾病,他说自己没有病,我说他将会被沓风病所折磨,三年以后四肢将不能自由活动,而且会喑哑不能说话,一旦喑哑就会死去。现在,听说他的四肢已经不能活动了,虽然喑哑却还没有死去。其病因是多次饮酒之后风邪严重侵入。我之所以知道他的病情,并为他诊断,是因为他的脉象符合《奇咳术》中的说法“脏气相反的病人会死去”。我为他切脉,脉象表明肾气反冲肺气,病理的说法是“三年会死”。

安陵阪里的公乘项处患病,我为他诊脉,诊过之后对他说:“你得了牡疝病。”牡疝发生于胸膈之下,向上与肺相连。此病源于房事不节制。我告诉他说:“你千万不要做需要用力的事,一旦做这样的事必定会吐血而死。”项处后来参加蹴鞠活动,结果腰部寒冷,出汗过多,还吐了血。我再次为他诊脉过后告诉他:“你会在第二天黄昏时分死去。”到了那个时间,他果然死了。他的疾病是由房事引起的。我之所以知道他的病情,是因为我在为他切脉时感觉到了番阳脉。番阳脉进入虚里,第二天就会死去。出现了番阳脉,又上连于肺,这就是牡疝病。

微臣淳于意说:除此之外,其他诊断出生死情况以及治愈的病例实在太多了,又由于时间太长而有所遗忘,没有完全记住,所以不敢拿这些病例来回答。

问臣意:“所诊治病,病名多同而诊异,或死或不死,何也?”对曰:“病名多相类,不可知,故古圣人为之脉法,以起度量,立规矩,县【县:通“悬”。】 权衡,案【案:通“按”,掌握。】 绳墨,调阴阳,别人之脉各名之,与天地相应,参合于人,故乃别百病以异之,有数者能异之,无数者同之。然脉法不可胜验,诊疾人以度异之,乃可别同名,命病主在所居。今臣意所诊者,皆有诊籍【诊籍:记录诊疗的籍册。】 。所以别之者,臣意所受师方适成,师死,以故表籍所诊,期决死生,观所失所得者合脉法,以故至今知之。”

又问淳于意:“在你所诊治过的疾病当中,有很多病名都是相同的而诊治结果不同,有的病人死了,有的却没死,这是为什么呢?”淳于意回答说:“疾病的名称大多是相似的,无法确切地分辨,因此古代的圣人创造了诊脉之法,以此来确定诊断的标准,设立规矩,斟酌权衡,遵循规则,协调阴阳,区分人体的脉象,并分别加以命名,与天地的变化相应,再参考人体情况,这样才能区分各种疾病,使它们有所区别,医术精湛的人能够将它们区分开,而医术拙劣的人则会将它们混淆。然而,脉法并不能全部应验,对病人进行诊断的时候要用不同的方法加以区分,这样才能将相同名称的疾病区别开来,并且说出病根在什么地方。如今凡是我诊治过的病人,都有诊治记录。我之所以这样来区分疾病,是因为我跟随老师刚刚学成医术,老师就去世了,所以我记录将诊治的情况以及决断生死的时间记录在案,据此来验证诊治的得失情况是否符合脉法,正因为这样,我现在才了解各种疾病的情况。”

问臣意曰:“所期病决死生,或不应期,何故?”对曰:“此皆饮食喜怒不节,或不当饮药,或不当针灸,以故不中期死也。”

问臣意:“意方能知病死生,论【论:分析,掌握。】 药用所宜,诸侯王大臣有尝问意者不?及文王病时,不求意诊治,何故?”对曰:“赵王、胶西王、济南王、吴王皆使人来召臣意,臣意不敢往。文王病时,臣意家贫,欲为人治病,诚恐吏以除拘臣意也,故移名数,左右不修家生,出行游国中,问善为方数者事之久矣,见事数师,悉受其要事,尽其方书意,及解论之。身居阳虚侯国,因事侯。侯入朝,臣意从之长安,以故得诊安陵项处等病也。”

又问淳于意:“你所预估的病人生死期限,有的与实际不一致,这是什么原因呢?”淳于意回答说:“出现这种情况,或是因为病人在饮食、喜怒等方面不加节制,或是因为用药不当,或是因为针灸不当,所以病人没有如期而死。”

又问淳于意:“当你能够知道病人的生死情况,并能掌握药品的适用范围时,诸侯王和大臣们有向你请教过的吗?当齐文王患病时,没有请你去诊治,这是什么原因呢?”淳于意回答说:“赵王、胶西王、济南王、吴王都曾经派人召我去,但是我不敢前往。齐文王患病的时候,我家里非常贫穷,想给人治病,但实在担心官吏委任我为侍医而将我束缚住,因此我把户籍迁到附近邻居的名下,不再管理家事,四处行医求学,长期寻访医术精湛的人并侍奉他们,我拜见并侍奉过很多老师,将他们的主要本领全部学到了,也完全领会了他们医方、医书的主旨,并且对其医术进行分析论断。由于我住在阳虚侯的封国,因此我才去侍奉他。阳虚侯入朝的时候,我跟着他来到长安,所以才能为安陵的项处等人诊治疾病。”

问臣意:“知文王所以得病不起之状?”臣意对曰:“不见文王病,然窃闻文王病喘,头痛,目不明。臣意心论之,以为非病也。以为肥而蓄精,身体不得摇,骨肉不相任,故喘,不当医治。脉法曰‘年二十脉气当趋【趋:跑动。】 ,年三十当疾步,年四十当安坐,年五十当安卧,年六十已上气当大董’。文王年未满二十,方脉气之趋也而徐之,不应天道四时。后闻医灸之即笃,此论病之过也。臣意论之,以为神气争而邪气入,非年少所能复之也,以故死。所谓气者,当调饮食,择晏日【晏日:晴朗的日子。】 ,车步广志【车步广志:驾车或者步行,以开阔心胸。】 ,以适【适:调适。】 筋骨肉血脉,以泻气。故年二十,是谓‘易 ’,法不当砭灸,砭灸至气逐。”

问臣意:“师庆安受之?闻于齐诸侯不?”对曰:“不知庆所师受。庆家富,善为医,不肯为人治病,当以此故不闻。庆又告臣意曰:‘慎毋令我子孙知若学我方也。’”

又问淳于意:“你知道齐文王卧病在床的原因是什么吗?”淳于意回答说:“我没有亲眼看到齐文王的病状,但我私下里曾听人说文王得的是气喘、头痛、视力不好的病。我心里想,这应该不是疾病。我认为文王是因为身体肥胖而积蓄了过多脂肪,身体又得不到活动,骨骼难以支撑躯体,所以才会气喘,这种情况不应当医治。脉法理论说‘人在二十岁的时候血脉正旺应该多跑动,三十岁时应该快步走,四十岁时应该安静地坐着,五十岁时应该安静地躺卧,过了六十岁,就应该使元气深藏’。文王年龄不到二十岁,正值脉气旺盛之际,可他却懒于走动,违背了自然规律。后来,我听说有医生用灸法为他治疗,结果病情立刻加重,这是论断病情上的错误。我经过分析,认为这是体内正气外争而邪气侵入的表现,不是年轻人能够康复的,所以文王最后死了。对于齐文王这种脉气旺盛的人,应当调节饮食,选择晴朗的天气外出,驾车或者步行,以开阔心胸,调适筋骨、肌肉和血脉,同时疏泻郁积体内的多余精气。因此在二十岁时,就是所谓的‘气血充实’时期,按照医理不应该用砭法和灸法治疗,一旦使用这种方法就会使人气血奔流。”

又问淳于意:“你的老师公乘阳庆是从哪儿学到的医术?齐国诸侯知不知道他的名声?”淳于意回答说:“我不知道老师阳庆是从哪儿学到的医术。阳庆家里十分富有,他精通医术,却不愿为人治病,所以才不被大家知道。阳庆还曾经告诫我:‘千万不要让我的子孙后代知道你学过我的医术。’”

问臣意:“师庆何见于意而爱意,欲悉教意方?”对曰:“臣意不闻师庆为方善也。意所以知庆者,意少时好诸方事,臣意试其方,皆多验,精良。臣意闻菑川唐里公孙光善为古传方,臣意即往谒之。得见事之,受方化阴阳及传语法,臣意悉受书之。臣意欲尽受他精方,公孙光曰:‘吾方尽矣,不为爱【爱:吝惜,吝啬。】 公所。吾身已衰,无所复事之。是吾年少所受妙方也,悉与公,毋以教人。’臣意曰:‘得见事侍公前,悉得禁方,幸甚。意死不敢妄传人。’居有间,公孙光闲处,臣意深论方,见言百世为之精也。师光喜曰:‘公必为国工【国工:这里指国医。】 。吾有所善者皆疏,同产处临菑,善为方,吾不若,其方甚奇,非世之所闻也。吾年中时,尝欲受其方,杨中倩不肯,曰“若非其人也”。胥【胥:等到。】 与公往见之,当知公喜方也。其人亦老矣,其家给富。’时者未往,会庆子男殷来献马,因师光奏马王所,意以故得与殷善。光又属意于殷曰:‘意好数,公必谨遇之,其人圣儒。’即为书以意属阳庆,以故知庆。臣意事庆谨,以故爱意也。”


曾国藩:「太史公好奇,如扁鹊、仓公、日者、龟策、货殖等事,无所不载,初无一定之例也。后世或援太史公以为例,或反引班、范以后之例而讥绳太史公,皆失之矣。」



又问淳于意:“你的老师阳庆为什么会看中并喜欢上你,以至于想把全部的医术传授给你?”淳于意回答说:“我原本没有听说过老师阳庆医术精湛。后来我之所以知道他,是因为我年轻的时候喜欢研习各家医术,我试着使用他的医方,都颇有成效,而且十分精妙。我听说菑川唐里的公孙光善于使用古代流传下来的医方,于是前去拜谒他。后来我得以拜见并侍奉他,从他那里学到了调理阴阳的医方和口传心授的医理,并全部记录下来。我想把他精妙的医术全部学来,他对我说:‘我的医方已经全部传授给你了,对你没有丝毫吝惜。现在我的身体已经衰老了,你不必再侍奉我。这些是我年轻时得到的妙方,全都给你,不要随便教给别人。’我说:‘我能够侍奉在您跟前,又得到了您的全部秘方,真是太幸运了。我就是死也不敢把医方胡乱传给他人。’过了一段日子,公孙光有了空闲,我就在他面前深入分析医方,他认为我对历代医方的论述是十分精辟的。他高兴地说:‘你将来一定会成为国医。我所擅长的医术都已经生疏了,我有位同胞兄弟住在临菑,精通医学,我比不上他,他的医方非常奇特,是世人所没有听过的。我在中年时期,曾想得到他的医方,可是我的朋友杨中倩反对,说“你不是那种可以学习医方的人”。等一会儿我和你一同前去拜见他,他就能知道你喜好医术了。如今他也老了,但家里非常富裕。’当时我们还没有去,正赶上阳庆的儿子阳殷前来献马,他是通过我的老师公孙光将马献给齐王,我也因此与阳殷熟识了。公孙光又把我托付给阳殷,告诉他说:‘淳于意非常喜欢医术,你一定要好好地对待他,他是一位道德高尚的儒士。’他就写了一封信把我推荐给阳庆,我因此认识了阳庆。我侍奉阳庆十分恭敬谨慎,他也因此而喜欢我。”

问臣意曰:“吏民尝有事学意方,及毕尽得意方不?何县里人?”对曰:“临菑人宋邑。邑学,臣意教以五诊,岁余。济北王遣太医高期、王禹学,臣意教以经脉高下及奇络结,当论俞所居,及气当上下出入邪正逆顺,以宜镵石,定砭灸处,岁余。菑川王时遣太仓马长冯信正方,臣意教以案法逆顺【案法逆顺:正反两种按摩方法。】 ,论药法,定五味及和齐汤法。高永侯家丞杜信,喜脉,来学,臣意教以上下经脉五诊,二岁余。临菑召里唐安来学,臣意教以五诊上下经脉,《奇咳》,四时应阴阳重,未成,除为齐王侍医。”

问臣意:“诊病决死生,能全无失乎?”臣意对曰:“意治病人,必先切其脉,乃治之。败逆者不可治,其顺者乃治之。心不精【精:辨别。】 脉,所期死生视可治,时时失之,臣意不能全也。”

又问淳于意:“官员和百姓曾经有人跟你学习医术吗?他们把你的医术全部学到了吗?他们都是什么地方的人?”淳于意回答说:“临菑人宋邑曾向我学习。宋邑前来求学,我教他诊断五脏疾病的脉法,他学了一年多的时间。济北王让太医高期、王禹跟我学习,我教他们经脉上下分布的情况以及奇经八脉、各种络脉的结系之处,论述腧穴所处的位置,以及脉气在人体内部运行时正邪顺逆的情况,选择合适的砭石、针灸治疗的穴位,他学了一年多时间。菑川王时常派遣在太仓署管理马匹的长官冯信向我学习医术,我教给他正反两种按摩技法,并论述了用药的方法,鉴定药物性味的原则,以及方剂配伍、调制汤药的方法。高永侯府上的管家杜信爱好诊脉,于是前来向我学习,我教给他经脉上下分布的情况以及诊断五脏疾病的脉法,他学习了两年多。临菑召里的唐安前来向我学习,我教他诊断五脏疾病的脉法、上下经脉分布的情况、《奇咳术》以及四季顺应阴阳变化的道理,他没有学成,就被任命为齐王的侍医。”

又问淳于意:“你为病人诊病,判断生死,能够做到完全没有失误吗?”淳于意回答说:“我为病人治病,一定要先为他切脉,然后才能治疗。脉象衰败或是与病情相违背的不能医治,只有脉象与病情相符才能医治。如果心中没有精确地辨别脉象,把不治之症视为可治,就会经常出现失误,我诊治疾病还做不到十全十美。”

太史公曰:女无美恶,居宫见妒;士无贤不肖,入朝见疑。故扁鹊以其伎见殃,仓公乃匿迹自隐而当【当:判罪。】 刑。缇萦通尺牍【尺牍:书信。】 ,父得以后宁。故老子曰“美好者不祥之器”,岂谓扁鹊等邪?若仓公者,可谓近之矣。

太史公说:女人无论是美还是丑,只要住在宫中就会遭人嫉妒;士人无论是贤能还是不成器,只要进入朝廷就会被人猜疑。所以扁鹊由于自己高明的医术而遇害,仓公淳于意自愿隐匿形迹却依然被判刑。缇萦上书皇上,她的父亲后来才得以平安。所以老子说的“美好的东西都是不祥之物”,这句话难道是说扁鹊这些人吗?像仓公这样的人也可以说很接近这句话的意思了。






\chapter{吴王濞列传}

\begin{yuanwen}
吴王濞者,高帝兄刘仲之子也。高帝已定天下七年,立刘仲为代王。而匈奴攻代,刘仲不能坚守,弃国亡,间行走雒阳,自归天子。天子为骨肉故,不忍致法,废以为郃阳侯。高帝十一年秋,淮南王英布反,东并荆地,劫其国兵,西度淮,击楚,高帝自将往诛之。刘仲子沛侯濞年二十,有气力,以骑将从破布军蕲西,会甀,布走。荆王刘贾为布所杀,无后。上患吴、会稽轻悍,无壮王以填之,诸子少,乃立濞于沛为吴王,王三郡五十三城。已拜受印,高帝召濞相之,谓曰:“若状有反相。”心独悔,业已拜,因拊其背,告曰:“汉后五十年东南有乱者,岂若邪?然天下同姓为一家也,慎无反!”濞顿首曰:“不敢。”

会孝惠、高后时,天下初定,郡国诸侯各务自拊循其民。吴有豫章郡铜山,濞则招致天下亡命者铸钱,煮海水为盐,以故无赋,国用富饶。

孝文时,吴太子入见,得侍皇太子饮博。吴太子师傅皆楚人,轻悍,又素骄,博,争道,不恭,皇太子引博局提吴太子,杀之。于是遣其丧归葬。至吴,吴王愠曰:“天下同宗,死长安即葬长安,何必来葬为!”复遣丧之长安葬。吴王由此稍失籓臣之礼,称病不朝。京师知其以子故称病不朝,验问实不病,诸吴使来,辄系责治之。吴王恐,为谋滋甚。及后使人为秋请,上复责问吴使者,使者对曰:“王实不病,汉系治使者数辈,以故遂称病。且夫‘察见渊中鱼,不祥’。今王始诈病,及觉,见责急,愈益闭,恐上诛之,计乃无聊。唯上弃之而与更始。”于是天子乃赦吴使者归之,而赐吴王几杖,老,不朝。吴得释其罪,谋亦益解。然其居国以铜盐故,百姓无赋。卒践更,辄与平贾。岁时存问茂材,赏赐闾里。佗郡国吏欲来捕亡人者,讼共禁弗予。如此者四十馀年,以故能使其众。

晁错为太子家令,得幸太子,数从容言吴过可削。数上书说孝文帝,文帝宽,不忍罚,以此吴日益横。及孝景帝即位,错为御史大夫,说上曰:“昔高帝初定天下,昆弟少,诸子弱,大封同姓,故王孽子悼惠王王齐七十馀城,庶弟元王王楚四十馀城,兄子濞王吴五十馀城:封三庶孽,分天下半。今吴王前有太子之郄,诈称病不朝,于古法当诛,文帝弗忍,因赐几杖。德至厚,当改过自新。乃益骄溢,即山铸钱,煮海水为盐,诱天下亡人,谋作乱。今削之亦反,不削之亦反。削之,其反亟,祸小;不削,反迟,祸大。”三年冬,楚王朝,晁错因言楚王戊往年为薄太后服,私奸服舍,请诛之。诏赦,罚削东海郡。因削吴之豫章郡、会稽郡。及前二年赵王有罪,削其河间郡。胶西王卬以卖爵有奸,削其六县。

汉廷臣方议削吴。吴王濞恐削地无已,因以此发谋,欲举事。念诸侯无足与计谋者,闻胶西王勇,好气,喜兵,诸齐皆惮畏,于是乃使中大夫应高誂胶西王。无文书,口报曰:“吴王不肖,有宿夕之忧,不敢自外,使喻其驩心。”王曰:“何以教之?”高曰:“今者主上兴于奸,饰于邪臣,好小善,听谗贼,擅变更律令,侵夺诸侯之地,徵求滋多,诛罚良善,日以益甚。里语有之,‘舐及米’。吴与胶西,知名诸侯也,一时见察,恐不得安肆矣。吴王身有内病,不能朝请二十馀年,尝患见疑,无以自白,今胁肩累足,犹惧不见释。窃闻大王以爵事有適,所闻诸侯削地,罪不至此,此恐不得削地而已。”王曰:“然,有之。子将柰何?”高曰:“同恶相助,同好相留,同情相成,同欲相趋,同利相死。今吴王自以为与大王同忧,原因时循理,弃躯以除患害于天下,亿亦可乎?”王瞿然骇曰:“寡人何敢如是?今主上虽急,固有死耳,安得不戴?”高曰:“御史大夫晁错,荧惑天子,侵夺诸侯,蔽忠塞贤,朝廷疾怨,诸侯皆有倍畔之意,人事极矣。彗星出,蝗蟲数起,此万世一时,而愁劳圣人之所以起也。故吴王欲内以晁错为讨,外随大王后车,彷徉天下,所乡者降,所指者下,天下莫敢不服。大王诚幸而许之一言,则吴王率楚王略函谷关,守荥阳敖仓之粟,距汉兵。治次舍,须大王。大王有幸而临之,则天下可并,两主分割,不亦可乎?”王曰:“善。”高归报吴王,吴王犹恐其不与,乃身自为使,使于胶西,面结之。

胶西群臣或闻王谋,谏曰:“承一帝,至乐也。今大王与吴西乡,弟令事成,两主分争,患乃始结。诸侯之地不足为汉郡什二,而为畔逆以忧太后,非长策也。”王弗听。遂发使约齐、菑川、胶东、济南、济北,皆许诺,而曰“城阳景王有义,攻诸吕,勿与,事定分之耳”。

诸侯既新削罚,振恐,多怨晁错。及削吴会稽、豫章郡书至,则吴王先起兵,胶西正月丙午诛汉吏二千石以下,胶东、菑川、济南、楚、赵亦然,遂发兵西。齐王后悔,饮药自杀,畔约。济北王城坏未完,其郎中令劫守其王,不得发兵。胶西为渠率,胶东、菑川、济南共攻围临菑。赵王遂亦反,阴使匈奴与连兵。

七国之发也,吴王悉其士卒,下令国中曰:“寡人年六十二,身自将。少子年十四,亦为士卒先。诸年上与寡人比,下与少子等者,皆发。”发二十馀万人。南使闽越、东越,东越亦发兵从。

孝景帝三年正月甲子,初起兵于广陵。西涉淮,因并楚兵。发使遗诸侯书曰:“吴王刘濞敬问胶西王、胶东王、菑川王、济南王、赵王、楚王、淮南王、衡山王、庐江王、故长沙王子:幸教寡人!以汉有贼臣,无功天下,侵夺诸侯地,使吏劾系讯治,以僇辱之为故,不以诸侯人君礼遇刘氏骨肉,绝先帝功臣,进任奸宄,诖乱天下,欲危社稷。陛下多病志失,不能省察。欲举兵诛之,谨闻教。敝国虽狭,地方三千里;人虽少,精兵可具五十万。寡人素事南越三十馀年,其王君皆不辞分其卒以随寡人,又可得三十馀万。寡人虽不肖,原以身从诸王。越直长沙者,因王子定长沙以北,西走蜀、汉中。告越、楚王、淮南三王,与寡人西面;齐诸王与赵王定河间、河内,或入临晋关,或与寡人会雒阳;燕王、赵王固与胡王有约,燕王北定代、云中,抟胡众入萧关,走长安,匡正天子,以安高庙。原王勉之。楚元王子、淮南三王或不沐洗十馀年,怨入骨髓,欲一有所出之久矣,寡人未得诸王之意,未敢听。今诸王苟能存亡继绝,振弱伐暴,以安刘氏,社稷之所原也。敝国虽贫,寡人节衣食之用,积金钱,脩兵革,聚穀食,夜以继日,三十馀年矣。凡为此,原诸王勉用之。能斩捕大将者,赐金五千斤,封万户;列将,三千斤,封五千户;裨将,二千斤,封二千户;二千石,千斤,封千户;千石,五百斤,封五百户:皆为列侯。其以军若城邑降者,卒万人,邑万户,如得大将;人户五千,如得列将;人户三千,如得裨将;人户千,如得二千石;其小吏皆以差次受爵金。佗封赐皆倍军法。其有故爵邑者,更益勿因。原诸王明以令士大夫,弗敢欺也。寡人金钱在天下者往往而有,非必取于吴,诸王日夜用之弗能尽。有当赐者告寡人,寡人且往遗之。敬以闻。”

七国反书闻天子,天子乃遣太尉条侯周亚夫将三十六将军,往击吴楚;遣曲周侯郦寄击赵;将军栾布击齐;大将军窦婴屯荥阳,监齐赵兵。

吴楚反书闻,兵未发,窦婴未行,言故吴相袁盎。盎时家居,诏召入见。上方与晁错调兵笇军食,上问袁盎曰:“君尝为吴相,知吴臣田禄伯为人乎?今吴楚反,于公何如?”对曰:“不足忧也,今破矣。”上曰:“吴王即山铸钱,煮海水为盐,诱天下豪桀,白头举事。若此,其计不百全,岂发乎?何以言其无能为也?”袁盎对曰:“吴有铜盐利则有之,安得豪桀而诱之!诚令吴得豪桀,亦且辅王为义,不反矣。吴所诱皆无赖子弟,亡命铸钱奸人,故相率以反。”晁错曰:“袁盎策之善。”上问曰:“计安出?”盎对曰:“原屏左右。”上屏人,独错在。盎曰:“臣所言,人臣不得知也。”乃屏错。错趋避东厢,恨甚。上卒问盎,盎对曰:“吴楚相遗书,曰‘高帝王子弟各有分地,今贼臣晁错擅適过诸侯,削夺之地’。故以反为名,西共诛晁错,复故地而罢。方今计独斩晁错,发使赦吴楚七国,复其故削地,则兵可无血刃而俱罢。”于是上嘿然良久,曰:“顾诚何如,吾不爱一人以谢天下。”盎曰:“臣愚计无出此,原上孰计之。”乃拜盎为太常,吴王弟子德侯为宗正。盎装治行。后十馀日,上使中尉召错,绐载行东市。错衣朝衣斩东市。则遣袁盎奉宗庙,宗正辅亲戚,使告吴如盎策。至吴,吴楚兵已攻梁壁矣。宗正以亲故,先入见,谕吴王使拜受诏。吴王闻袁盎来,亦知其欲说己,笑而应曰:“我已为东帝,尚何谁拜?”不肯见盎而留之军中,欲劫使将。盎不肯,使人围守,且杀之,盎得夜出,步亡去,走梁军,遂归报。

条侯将乘六乘传,会兵荥阳。至雒阳,见剧孟,喜曰:“七国反,吾乘传至此,不自意全。又以为诸侯已得剧孟,剧孟今无动。吾据荥阳,以东无足忧者。”至淮阳,问父绛侯故客邓都尉曰:“策安出?”客曰:“吴兵锐甚,难与争锋。楚兵轻,不能久。方今为将军计,莫若引兵东北壁昌邑,以梁委吴,吴必尽锐攻之。将军深沟高垒,使轻兵绝淮泗口,塞吴饟道。彼吴梁相敝而粮食竭,乃以全彊制其罢极,破吴必矣。”条侯曰:“善。”从其策,遂坚壁昌邑南,轻兵绝吴饟道。

吴王之初发也,吴臣田禄伯为大将军。田禄伯曰:“兵屯聚而西,无佗奇道,难以就功。臣原得五万人,别循江淮而上,收淮南、长沙,入武关,与大王会,此亦一奇也。”吴王太子谏曰:“王以反为名,此兵难以藉人,藉人亦且反王,柰何?且擅兵而别,多佗利害,未可知也,徒自损耳。”吴王即不许田禄伯。

吴少将桓将军说王曰:“吴多步兵,步兵利险;汉多车骑,车骑利平地。原大王所过城邑不下,直弃去,疾西据雒阳武库,食敖仓粟,阻山河之险以令诸侯,虽毋入关,天下固已定矣。即大王徐行,留下城邑,汉军车骑至,驰入梁楚之郊,事败矣。”吴王问诸老将,老将曰:“此少年推锋之计可耳,安知大虑乎!”于是王不用桓将军计。

吴王专并将其兵,未度淮,诸宾客皆得为将、校尉、候、司马,独周丘不得用。周丘者,下邳人,亡命吴,酤酒无行,吴王濞薄之,弗任。周丘上谒,说王曰:“臣以无能,不得待罪行间。臣非敢求有所将,原得王一汉节,必有以报王。”王乃予之。周丘得节,夜驰入下邳。下邳时闻吴反,皆城守。至传舍,召令。令入户,使从者以罪斩令。遂召昆弟所善豪吏告曰:“吴反兵且至,至,屠下邳不过食顷。今先下,家室必完,能者封侯矣。”出乃相告,下邳皆下。周丘一夜得三万人,使人报吴王,遂将其兵北略城邑。比至城阳,兵十馀万,破城阳中尉军。闻吴王败走,自度无与共成功,即引兵归下邳。未至,疽发背死。

二月中,吴王兵既破,败走,于是天子制诏将军曰:“盖闻为善者,天报之以福;为非者,天报之以殃。高皇帝亲表功德,建立诸侯,幽王、悼惠王绝无后,孝文皇帝哀怜加惠,王幽王子遂、悼惠王子卬等,令奉其先王宗庙,为汉籓国,德配天地,明并日月。吴王濞倍德反义,诱受天下亡命罪人,乱天下币,称病不朝二十馀年,有司数请濞罪,孝文皇帝宽之,欲其改行为善。今乃与楚王戊、赵王遂、胶西王卬、济南王辟光、菑川王贤、胶东王雄渠约从反,为逆无道,起兵以危宗庙,贼杀大臣及汉使者,迫劫万民,夭杀无罪,烧残民家,掘其丘冢,甚为暴虐。今卬等又重逆无道,烧宗庙,卤御物,朕甚痛之。朕素服避正殿,将军其劝士大夫击反虏。击反虏者,深入多杀为功,斩首捕虏比三百石以上者皆杀之,无有所置。敢有议诏及不如诏者,皆要斩。”

初,吴王之度淮,与楚王遂西败棘壁,乘胜前,锐甚。梁孝王恐,遣六将军击吴,又败梁两将,士卒皆还走梁。梁数使使报条侯求救,条侯不许。又使使恶条侯于上,上使人告条侯救梁,复守便宜不行。梁使韩安国及楚死事相弟张羽为将军,乃得颇败吴兵。吴兵欲西,梁城守坚,不敢西,即走条侯军,会下邑。欲战,条侯壁,不肯战。吴粮绝,卒饥,数挑战,遂夜饹条侯壁,惊东南。条侯使备西北,果从西北入。吴大败,士卒多饥死,乃畔散。于是吴王乃与其麾下壮士数千人夜亡去,度江走丹徒,保东越。东越兵可万馀人,乃使人收聚亡卒。汉使人以利啗东越,东越即绐吴王,吴王出劳军,即使人鏦杀吴王,盛其头,驰传以闻。吴王子子华、子驹亡走闽越。吴王之弃其军亡也,军遂溃,往往稍降太尉、梁军。楚王戊军败,自杀。

三王之围齐临菑也,三月不能下。汉兵至,胶西、胶东、菑川王各引兵归。胶西王乃袒跣,席,饮水,谢太后。王太子德曰:“汉兵远,臣观之已罢,可袭,原收大王馀兵击之,击之不胜,乃逃入海,未晚也。”王曰:“吾士卒皆已坏,不可发用。”弗听。汉将弓高侯穨当遗王书曰:“奉诏诛不义,降者赦其罪,复故;不降者灭之。王何处,须以从事。”王肉袒叩头汉军壁,谒曰:“臣卬奉法不谨,惊骇百姓,乃苦将军远道至于穷国,敢请菹醢之罪。”弓高侯执金鼓见之,曰:“王苦军事,原闻王发兵状。”王顿首膝行对曰:“今者,晁错天子用事臣,变更高皇帝法令,侵夺诸侯地。卬等以为不义,恐其败乱天下,七国发兵,且以诛错。今闻错已诛,卬等谨以罢兵归。”将军曰:“王苟以错不善,何不以闻?未有诏虎符,擅发兵击义国。以此观之,意非欲诛错也。”乃出诏书为王读之。读之讫,曰:“王其自图。”王曰:“如卬等死有馀罪。”遂自杀。太后、太子皆死。胶东、菑川、济南王皆死,国除,纳于汉。郦将军围赵十月而下之,赵王自杀。济北王以劫故,得不诛,徙王菑川。

初,吴王首反,并将楚兵,连齐赵。正月起兵,三月皆破,独赵后下。复置元王少子平陆侯礼为楚王,续元王后。徙汝南王非王吴故地,为江都王。

太史公曰:吴王之王,由父省也。能薄赋敛,使其众,以擅山海利。逆乱之萌,自其子兴。争技发难,卒亡其本;亲越谋宗,竟以夷陨。晁错为国远虑,祸反近身。袁盎权说,初宠后辱。故古者诸侯地不过百里,山海不以封。“毋亲夷狄,以疏其属”,盖谓吴邪?“毋为权首,反受其咎”,岂盎、错邪?

吴楚轻悍,王濞倍德。富因采山,衅成提局。憍矜贰志,连结七国。婴命始监,错诛未塞。天之悔祸,卒取奔北。
\end{yuanwen}

\chapter{魏其武安侯列传}

\begin{yuanwen}
魏其侯窦婴者,孝文后从兄子也。父世观津人。喜宾客。孝文时,婴为吴相,病免。孝景初即位,为詹事。

梁孝王者,孝景弟也,其母窦太后爱之。梁孝王朝,因昆弟燕饮。是时上未立太子,酒酣,从容言曰:“千秋之后传梁王。”太后驩。窦婴引卮酒进上,曰:“天下者,高祖天下,父子相传,此汉之约也,上何以得擅传梁王!”太后由此憎窦婴。窦婴亦薄其官,因病免。太后除窦婴门籍,不得入朝请。

孝景三年,吴楚反,上察宗室诸窦毋如窦婴贤,乃召婴。婴入见,固辞谢病不足任。太后亦惭。于是上曰:“天下方有急,王孙宁可以让邪?”乃拜婴为大将军,赐金千斤。婴乃言袁盎、栾布诸名将贤士在家者进之。所赐金,陈之廊庑下,军吏过,辄令财取为用,金无入家者。窦婴守荥阳,监齐赵兵。七国兵已尽破,封婴为魏其侯。诸游士宾客争归魏其侯。孝景时每朝议大事,条侯、魏其侯,诸列侯莫敢与亢礼。

孝景四年,立栗太子,使魏其侯为太子傅。孝景七年,栗太子废,魏其数争不能得。魏其谢病,屏居蓝田南山之下数月,诸宾客辩士说之,莫能来。梁人高遂乃说魏其曰:“能富贵将军者,上也;能亲将军者,太后也。今将军傅太子,太子废而不能争;争不能得,又弗能死。自引谢病,拥赵女,屏间处而不朝。相提而论,是自明扬主上之过。有如两宫螫将军,则妻子毋类矣。”魏其侯然之,乃遂起,朝请如故。

桃侯免相,窦太后数言魏其侯。孝景帝曰:“太后岂以为臣有爱,不相魏其?魏其者,沾沾自喜耳,多易。难以为相,持重。”遂不用,用建陵侯卫绾为丞相。

武安侯田蚡者,孝景后同母弟也,生长陵。魏其已为大将军后,方盛,蚡为诸郎,未贵,往来侍酒魏其,跪起如子姓。及孝景晚节,蚡益贵幸,为太中大夫。蚡辩有口,学槃盂诸书,王太后贤之。孝景崩,即日太子立,称制,所镇抚多有田蚡宾客计筴,蚡弟田胜,皆以太后弟,孝景后三年封蚡为武安侯,胜为周阳侯。

武安侯新欲用事为相,卑下宾客,进名士家居者贵之,欲以倾魏其诸将相。建元元年,丞相绾病免,上议置丞相、太尉。籍福说武安侯曰:“魏其贵久矣,天下士素归之。今将军初兴,未如魏其,即上以将军为丞相,必让魏其。魏其为丞相,将军必为太尉。太尉、丞相尊等耳,又有让贤名。”武安侯乃微言太后风上,于是乃以魏其侯为丞相,武安侯为太尉。籍福贺魏其侯,因吊曰:“君侯资性喜善疾恶,方今善人誉君侯,故至丞相;然君侯且疾恶,恶人众,亦且毁君侯。君侯能兼容,则幸久;不能,今以毁去矣。”魏其不听。

魏其、武安俱好儒术,推毂赵绾为御史大夫,王臧为郎中令。迎鲁申公,欲设明堂,令列侯就国,除关,以礼为服制,以兴太平。举適诸窦宗室毋节行者,除其属籍。时诸外家为列侯,列侯多尚公主,皆不欲就国,以故毁日至窦太后。太后好黄老之言,而魏其、武安、赵绾、王臧等务隆推儒术,贬道家言,是以窦太后滋不说魏其等。及建元二年,御史大夫赵绾请无奏事东宫。窦太后大怒,乃罢逐赵绾、王臧等,而免丞相、太尉,以柏至侯许昌为丞相,武彊侯庄青翟为御史大夫。魏其、武安由此以侯家居。

武安侯虽不任职,以王太后故,亲幸,数言事多效,天下吏士趋势利者,皆去魏其归武安,武安日益横。建元六年,窦太后崩,丞相昌、御史大夫青翟坐丧事不办,免。以武安侯蚡为丞相,以大司农韩安国为御史大夫。天下士郡诸侯愈益附武安。

武安者,貌侵,生贵甚。又以为诸侯王多长,上初即位,富于春秋,蚡以肺腑为京师相,非痛折节以礼诎之,天下不肃。当是时,丞相入奏事,坐语移日,所言皆听。荐人或起家至二千石,权移主上。上乃曰:“君除吏已尽未?吾亦欲除吏。”尝请考工地益宅,上怒曰:“君何不遂取武库!”是后乃退。尝召客饮,坐其兄盖侯南乡,自坐东乡,以为汉相尊,不可以兄故私桡。武安由此滋骄,治宅甲诸第。田园极膏腴,而市买郡县器物相属于道。前堂罗锺鼓,立曲旃;后房妇女以百数。诸侯奉金玉狗马玩好,不可胜数。

魏其失窦太后,益疏不用,无势,诸客稍稍自引而怠傲,唯灌将军独不失故。魏其日默默不得志,而独厚遇灌将军。

灌将军夫者,颍阴人也。夫父张孟,尝为颍阴侯婴舍人,得幸,因进之至二千石,故蒙灌氏姓为灌孟。吴楚反时,颍阴侯灌何为将军,属太尉,请灌孟为校尉。夫以千人与父俱。灌孟年老,颍阴侯彊请之,郁郁不得意,故战常陷坚,遂死吴军中。军法,父子俱从军,有死事,得与丧归。灌夫不肯随丧归,奋曰:“原取吴王若将军头,以报父之仇。”于是灌夫被甲持戟,募军中壮士所善原从者数十人。及出壁门,莫敢前。独二人及从奴十数骑驰入吴军,至吴将麾下,所杀伤数十人。不得前,复驰还,走入汉壁,皆亡其奴,独与一骑归。夫身中大创十馀,適有万金良药,故得无死。夫创少瘳,又复请将军曰:“吾益知吴壁中曲折,请复往。”将军壮义之,恐亡夫,乃言太尉,太尉乃固止之。吴已破,灌夫以此名闻天下。

颍阴侯言之上,上以夫为中郎将。数月,坐法去。后家居长安,长安中诸公莫弗称之。孝景时,至代相。孝景崩,今上初即位,以为淮阳天下交,劲兵处,故徙夫为淮阳太守。建元元年,入为太仆。二年,夫与长乐卫尉窦甫饮,轻重不得,夫醉,搏甫。甫,窦太后昆弟也。上恐太后诛夫,徙为燕相。数岁,坐法去官,家居长安。

灌夫为人刚直使酒,不好面谀。贵戚诸有势在己之右,不欲加礼,必陵之;诸士在己之左,愈贫贱,尤益敬,与钧。稠人广众,荐宠下辈。士亦以此多之。

夫不喜文学,好任侠,已然诺。诸所与交通,无非豪桀大猾。家累数千万,食客日数十百人。陂池田园,宗族宾客为权利,横于颍川。颍川兒乃歌之曰:“颍水清,灌氏宁;颍水浊,灌氏族。”

灌夫家居虽富,然失势,卿相侍中宾客益衰。及魏其侯失势,亦欲倚灌夫引绳批根生平慕之后弃之者。灌夫亦倚魏其而通列侯宗室为名高。两人相为引重,其游如父子然。相得驩甚,无厌,恨相知晚也。

灌夫有服,过丞相。丞相从容曰:“吾欲与仲孺过魏其侯,会仲孺有服。”灌夫曰:“将军乃肯幸临况魏其侯,夫安敢以服为解!请语魏其侯帐具,将军旦日蚤临。”武安许诺。灌夫具语魏其侯如所谓武安侯。魏其与其夫人益市牛酒,夜洒埽,早帐具至旦。平明,令门下候伺。至日中,丞相不来。魏其谓灌夫曰:“丞相岂忘之哉?”灌夫不怿,曰:“夫以服请,宜往。”乃驾,自往迎丞相。丞相特前戏许灌夫,殊无意往。及夫至门,丞相尚卧。于是夫入见,曰:“将军昨日幸许过魏其,魏其夫妻治具,自旦至今,未敢尝食。”武安鄂谢曰:“吾昨日醉,忽忘与仲孺言。”乃驾往,又徐行,灌夫愈益怒。及饮酒酣,夫起舞属丞相,丞相不起,夫从坐上语侵之。魏其乃扶灌夫去,谢丞相。丞相卒饮至夜,极驩而去。

丞相尝使籍福请魏其城南田。魏其大望曰:“老仆虽弃,将军虽贵,宁可以势夺乎!”不许。灌夫闻,怒,骂籍福。籍福恶两人有郄,乃谩自好谢丞相曰:“魏其老且死,易忍,且待之。”已而武安闻魏其、灌夫实怒不予田,亦怒曰:“魏其子尝杀人,蚡活之。蚡事魏其无所不可,何爱数顷田?且灌夫何与也?吾不敢复求田。”武安由此大怨灌夫、魏其。

元光四年春,丞相言灌夫家在颍川,横甚,民苦之。请案。上曰:“此丞相事,何请。”灌夫亦持丞相阴事,为奸利,受淮南王金与语言。宾客居间,遂止,俱解。

夏,丞相取燕王女为夫人,有太后诏,召列侯宗室皆往贺。魏其侯过灌夫,欲与俱。夫谢曰:“夫数以酒失得过丞相,丞相今者又与夫有郄。”魏其曰:“事已解。”彊与俱。饮酒酣,武安起为寿,坐皆避席伏。已魏其侯为寿,独故人避席耳,馀半膝席。灌夫不悦。起行酒,至武安,武安膝席曰:“不能满觞。”夫怒,因嘻笑曰:“将军贵人也,属之!”时武安不肯。行酒次至临汝侯,临汝侯方与程不识耳语,又不避席。夫无所发怒,乃骂临汝侯曰:“生平毁程不识不直一钱,今日长者为寿,乃效女兒呫嗫耳语!”武安谓灌夫曰:“程李俱东西宫卫尉,今众辱程将军,仲孺独不为李将军地乎?”灌夫曰:“今日斩头陷匈,何知程李乎!”坐乃起更衣,稍稍去。魏其侯去,麾灌夫出。武安遂怒曰:“此吾骄灌夫罪。”乃令骑留灌夫。灌夫欲出不得。籍福起为谢,案灌夫项令谢。夫愈怒,不肯谢。武安乃麾骑缚夫置传舍,召长史曰:“今日召宗室,有诏。”劾灌夫骂坐不敬,系居室。遂按其前事,遣吏分曹逐捕诸灌氏支属,皆得弃市罪。魏其侯大媿,为资使宾客请,莫能解。武安吏皆为耳目,诸灌氏皆亡匿,夫系,遂不得告言武安阴事。

魏其锐身为救灌夫。夫人谏魏其曰:“灌将军得罪丞相,与太后家忤,宁可救邪?”魏其侯曰:“侯自我得之,自我捐之,无所恨。且终不令灌仲孺独死,婴独生。”乃匿其家,窃出上书。立召入,具言灌夫醉饱事,不足诛。上然之,赐魏其食,曰:“东朝廷辩之。”

魏其之东朝,盛推灌夫之善,言其醉饱得过,乃丞相以他事诬罪之。武安又盛毁灌夫所为横恣,罪逆不道。魏其度不可柰何,因言丞相短。武安曰:“天下幸而安乐无事,蚡得为肺腑,所好音乐狗马田宅。蚡所爱倡优巧匠之属,不如魏其、灌夫日夜招聚天下豪桀壮士与论议,腹诽而心谤,不仰视天而俯画地,辟倪两宫间,幸天下有变,而欲有大功。臣乃不知魏其等所为。”于是上问朝臣:“两人孰是?”御史大夫韩安国曰:“魏其言灌夫父死事,身荷戟驰入不测之吴军,身被数十创,名冠三军,此天下壮士,非有大恶,争杯酒,不足引他过以诛也。魏其言是也。丞相亦言灌夫通奸猾,侵细民,家累巨万,横恣颍川,凌轹宗室,侵犯骨肉,此所谓‘枝大于本,胫大于股,不折必披’,丞相言亦是。唯明主裁之。”主爵都尉汲黯是魏其。内史郑当时是魏其,后不敢坚对。馀皆莫敢对。上怒内史曰:“公平生数言魏其、武安长短,今日廷论,局趣效辕下驹,吾并斩若属矣。”即罢起入,上食太后。太后亦已使人候伺,具以告太后。太后怒,不食,曰:“今我在也,而人皆藉吾弟,令我百岁后,皆鱼肉之矣。且帝宁能为石人邪!此特帝在,即录录,设百岁后,是属宁有可信者乎?”上谢曰:“俱宗室外家,故廷辩之。不然,此一狱吏所决耳。”是时郎中令石建为上别言两人事。

武安已罢朝,出止车门,召韩御史大夫载,怒曰:“与长孺共一老秃翁,何为首鼠两端?”韩御史良久谓丞相曰:“君何不自喜?夫魏其毁君,君当免冠解印绶归,曰‘臣以肺腑幸得待罪,固非其任,魏其言皆是’。如此,上必多君有让,不废君。魏其必内愧,杜门齰舌自杀。今人毁君,君亦毁人,譬如贾竖女子争言,何其无大体也!”武安谢罪曰:“争时急,不知出此。”

于是上使御史簿责魏其所言灌夫,颇不雠,欺谩。劾系都司空。孝景时,魏其常受遗诏,曰“事有不便,以便宜论上”。及系,灌夫罪至族,事日急,诸公莫敢复明言于上。魏其乃使昆弟子上书言之,幸得复召见。书奏上,而案尚书大行无遗诏。诏书独藏魏其家,家丞封。乃劾魏其矫先帝诏,罪当弃市。五年十月,悉论灌夫及家属。魏其良久乃闻,闻即恚,病痱,不食欲死。或闻上无意杀魏其,魏其复食,治病,议定不死矣。乃有蜚语为恶言闻上,故以十二月晦论弃市渭城。

其春,武安侯病,专呼服谢罪。使巫视鬼者视之,见魏其、灌夫共守,欲杀之。竟死。子恬嗣。元朔三年,武安侯坐衣襜褕入宫,不敬。

淮南王安谋反觉,治。王前朝,武安侯为太尉,时迎王至霸上,谓王曰:“上未有太子,大王最贤,高祖孙,即宫车晏驾,非大王立当谁哉!”淮南王大喜,厚遗金财物。上自魏其时不直武安,特为太后故耳。及闻淮南王金事,上曰:“使武安侯在者,族矣。”

太史公曰:魏其、武安皆以外戚重,灌夫用一时决筴而名显。魏其之举以吴楚,武安之贵在日月之际。然魏其诚不知时变,灌夫无术而不逊,两人相翼,乃成祸乱。武安负贵而好权,杯酒责望,陷彼两贤。呜呼哀哉!迁怒及人,命亦不延。众庶不载,竟被恶言。呜呼哀哉!祸所从来矣!

窦婴、田蚡,势利相雄。咸倚外戚,或恃军功。灌夫自喜,引重其中。意气杯酒,辟睨两宫。事竟不直,冤哉二公!
\end{yuanwen}

\chapter{韩长孺列传}

\begin{yuanwen}
御史大夫韩安国者,梁成安人也,后徙睢阳。尝受韩子、杂家说于驺田生所。事梁孝王为中大夫。吴楚反时,孝王使安国及张羽为将,扞吴兵于东界。张羽力战,安国持重,以故吴不能过梁。吴楚已破,安国、张羽名由此显。

梁孝王,景帝母弟,窦太后爱之,令得自请置相、二千石,出入游戏,僭于天子。天子闻之,心弗善也。太后知帝不善,乃怒梁使者,弗见,案责王所为。韩安国为梁使,见大长公主而泣曰:“何梁王为人子之孝,为人臣之忠,太后曾弗省也?夫前日吴、楚、齐、赵七国反时,自关以东皆合从西乡,惟梁最亲为艰难。梁王念太后、帝在中,而诸侯扰乱,一言泣数行下,跪送臣等六人,将兵击卻吴楚,吴楚以故兵不敢西,而卒破亡,梁王之力也。今太后以小节苛礼责望梁王。梁王父兄皆帝王,所见者大,故出称跸,入言警,车旗皆帝所赐也,即欲以侘鄙县,驱驰国中,以夸诸侯,令天下尽知太后、帝爱之也。今梁使来,辄案责之。梁王恐,日夜涕泣思慕,不知所为。何梁王之为子孝,为臣忠,而太后弗恤也?”大长公主具以告太后,太后喜曰:“为言之帝。”言之,帝心乃解,而免冠谢太后曰:“兄弟不能相教,乃为太后遗忧。”悉见梁使,厚赐之。其后梁王益亲驩。太后、长公主更赐安国可直千馀金。名由此显,结于汉。

其后安国坐法抵罪,蒙狱吏田甲辱安国。安国曰:“死灰独不复然乎?”田甲曰:“然即溺之。”居无何,梁内史缺,汉使使者拜安国为梁内史,起徒中为二千石。田甲亡走。安国曰:“甲不就官,我灭而宗。”甲因肉袒谢。安国笑曰:“可溺矣!公等足与治乎?”卒善遇之。

梁内史之缺也,孝王新得齐人公孙诡,说之,欲请以为内史。窦太后闻,乃诏王以安国为内史。

公孙诡、羊胜说孝王求为帝太子及益地事,恐汉大臣不听,乃阴使人刺汉用事谋臣。及杀故吴相袁盎,景帝遂闻诡、胜等计画,乃遣使捕诡、胜,必得。汉使十辈至梁,相以下举国大索,月馀不得。内史安国闻诡、胜匿孝王所,安国入见王而泣曰:“主辱臣死。大王无良臣,故事纷纷至此。今诡、胜不得,请辞赐死。”王曰:“何至此?”安国泣数行下,曰:“大王自度于皇帝,孰与太上皇之与高皇帝及皇帝之与临江王亲?”孝王曰:“弗如也。”安国曰:“夫太上、临江亲父子之间,然而高帝曰‘提三尺剑取天下者朕也’,故太上皇终不得制事,居于栎阳。临江王,適长太子也,以一言过,废王临江;用宫垣事,卒自杀中尉府。何者?治天下终不以私乱公。语曰:‘虽有亲父,安知其不为虎?虽有亲兄,安知其不为狼?’今大王列在诸侯,悦一邪臣浮说,犯上禁,桡明法。天子以太后故,不忍致法于王。太后日夜涕泣,幸大王自改,而大王终不觉寤。有如太后宫车即晏驾,大王尚谁攀乎?”语未卒,孝王泣数行下,谢安国曰:“吾今出诡、胜。”诡、胜自杀。汉使还报,梁事皆得释,安国之力也。于是景帝、太后益重安国。孝王卒,共王即位,安国坐法失官,居家。

建元中,武安侯田蚡为汉太尉,亲贵用事,安国以五百金物遗蚡。蚡言安国太后,天子亦素闻其贤,即召以为北地都尉,迁为大司农。闽越、东越相攻,安国及大行王恢将。未至越,越杀其王降,汉兵亦罢。建元六年,武安侯为丞相,安国为御史大夫。

匈奴来请和亲,天子下议。大行王恢,燕人也,数为边吏,习知胡事。议曰:“汉与匈奴和亲,率不过数岁即复倍约。不如勿许,兴兵击之。”安国曰:“千里而战,兵不获利。今匈奴负戎马之足,怀禽兽之心,迁徙鸟举,难得而制也。得其地不足以为广,有其众不足以为彊,自上古不属为人。汉数千里争利,则人马罢,虏以全制其敝。且彊弩之极,矢不能穿鲁缟;冲风之末,力不能漂鸿毛。非初不劲,末力衰也。击之不便,不如和亲。”群臣议者多附安国,于是上许和亲。

其明年,则元光元年,雁门马邑豪聂翁壹因大行王恢言上曰:“匈奴初和亲,亲信边,可诱以利。”阴使聂翁壹为间,亡入匈奴,谓单于曰:“吾能斩马邑令丞吏,以城降,财物可尽得。”单于爱信之,以为然,许聂翁壹。聂翁壹乃还,诈斩死罪囚,县其头马邑城,示单于使者为信。曰:“马邑长吏已死,可急来。”于是单于穿塞将十馀万骑,入武州塞。

当是时,汉伏兵车骑材官二十馀万,匿马邑旁谷中。卫尉李广为骁骑将军,太仆公孙贺为轻车将军,大行王恢为将屯将军,太中大夫李息为材官将军。御史大夫韩安国为护军将军,诸将皆属护军。约单于入马邑而汉兵纵发。王恢、李息、李广别从代主击其辎重。于是单于入汉长城武州塞。未至马邑百馀里,行掠卤,徒见畜牧于野,不见一人。单于怪之,攻烽燧,得武州尉史。欲刺问尉史。尉史曰:“汉兵数十万伏马邑下。”单于顾谓左右曰:“几为汉所卖!”乃引兵还。出塞,曰:“吾得尉史,乃天也。”命尉史为“天王”。塞下传言单于已引去。汉兵追至塞,度弗及,即罢。王恢等兵三万,闻单于不与汉合,度往击辎重,必与单于精兵战,汉兵势必败,则以便宜罢兵,皆无功。

天子怒王恢不出击单于辎重,擅引兵罢也。恢曰:“始约虏入马邑城,兵与单于接,而臣击其辎重,可得利。今单于闻,不至而还,臣以三万人众不敌,礻是取辱耳。臣固知还而斩,然得完陛下士三万人。”于是下恢廷尉。廷尉当恢逗桡,当斩。恢私行千金丞相蚡。蚡不敢言上,而言于太后曰:“王恢首造马邑事,今不成而诛恢,是为匈奴报仇也。”上朝太后,太后以丞相言告上。上曰:“首为马邑事者,恢也,故发天下兵数十万,从其言,为此。且纵单于不可得,恢所部击其辎重,犹颇可得,以慰士大夫心。今不诛恢,无以谢天下。”于是恢闻之,乃自杀。

安国为人多大略,智足以当世取合,而出于忠厚焉。贪嗜于财。所推举皆廉士,贤于己者也。于梁举壶遂、臧固、郅他,皆天下名士,士亦以此称慕之,唯天子以为国器。安国为御史大夫四岁馀,丞相田蚡死,安国行丞相事,奉引堕车蹇。天子议置相,欲用安国,使使视之,蹇甚,乃更以平棘侯薛泽为丞相。安国病免数月,蹇愈,上复以安国为中尉。岁馀,徙为卫尉。

车骑将军卫青击匈奴,出上谷,破胡茏城。将军李广为匈奴所得,复失之;公孙敖大亡卒:皆当斩,赎为庶人。明年,匈奴大入边,杀辽西太守,及入雁门,所杀略数千人。车骑将军卫青击之,出雁门。卫尉安国为材官将军,屯于渔阳。安国捕生虏,言匈奴远去。即上书言方田作时,请且罢军屯。罢军屯月馀,匈奴大入上谷、渔阳。安国壁乃有七百馀人,出与战,不胜,复入壁。匈奴虏略千馀人及畜产而去。天子闻之,怒,使使责让安国。徒安国益东,屯右北平。是时匈奴虏言当入东方。

安国始为御史大夫及护军,后稍斥疏,下迁;而新幸壮将军卫青等有功,益贵。安国既疏远,默默也;将屯又为匈奴所欺,失亡多,甚自愧。幸得罢归,乃益东徙屯,意忽忽不乐。数月,病欧血死。安国以元朔二年中卒。

太史公曰:余与壶遂定律历,观韩长孺之义,壶遂之深中隐厚。世之言梁多长者,不虚哉!壶遂官至詹事,天子方倚以为汉相,会遂卒。不然,壶遂之内廉行脩,斯鞠躬君子也。

安国忠厚,初为梁将。因事坐法,免徒起相。死灰更然,生虏失防。推贤见重,贿金贻谤。雪泣悟主,臣节可亮。
\end{yuanwen}

\chapter{李将军列传}

紧紧围绕着精于骑射,勇敢作战;热爱士卒,不贪钱财;为人简易,号令不烦三个特点,刻画了李广这样一个作者所理想的一代名将的英雄形象,而对李广的坎坷一生,尤其是对他以及他整个家族的悲惨结局,表现了无限的惋惜与同情,对汉代皇帝及其宠幸们排挤、残害李广及其家族的罪行表现了极大的愤慨,对汉代的用人制度进行了有力的批判。同时,作者在描写李广坎坷悲惨的一生际遇中,也寄寓了自己的满腔悲愤与辛酸。但司马迁由于个人的遭遇与好恶,对李广的评价有点过高,而对卫青、霍去病则过低的贬抑,是不恰当的。

\begin{yuanwen}
李将军广者,陇西成纪人也。其先曰李信,秦时为将,逐得燕太子丹者也。故槐里,徙成纪。广家世世受射\footnote{text}。

孝文帝十四年,匈奴大入萧关,而广以良家子从军击胡\footnote{text},用善骑射,杀首虏多\footnote{text},为汉中郎\footnote{text}。广从弟李蔡亦为郎,皆为武骑常侍\footnote{text},秩八百石\footnote{text}。尝从行\footnote{text},有所冲陷折关及格猛兽\footnote{text},而文帝曰:“惜乎,子不遇时!如令子当高帝时,万户侯岂足道哉\footnote{text}!”
\end{yuanwen}

\begin{yuanwen}
及孝景初立,广为陇西都尉,徙为骑郎将。吴、楚军时\footnote{text},广为骁骑都尉\footnote{text},从太尉亚夫击吴楚军\footnote{text},取旗\footnote{text},显功名昌邑下\footnote{text}。以梁王授广将军印\footnote{text},还,赏不行\footnote{text}。徙为上谷太守,匈奴日以合战。

典属国公孙昆邪为上泣曰:“李广才气,天下无双,自负其能,数与虏敌战,恐亡之。”

于是乃徙为上郡太守。后广转为边郡太守,徙上郡。尝为陇西、北地、雁门、代郡、云中太守,皆以力战为名\footnote{text}。
\end{yuanwen}

\begin{yuanwen}
匈奴大入上郡,天子使中贵人从广勒习兵击匈奴\footnote{text}。中贵人将骑数十纵\footnote{text},见匈奴三人,与战。三人还射\footnote{text},伤中贵人,杀其骑且尽。中贵人走广。广曰:“是必射雕者也。”

广乃遂从百骑往驰三人\footnote{text}。三人亡马步行\footnote{text},行数十里。广令其骑张左右翼,而广身自射彼三人者,杀其二人,生得一人,果匈奴射雕者也。已缚之上马,望匈奴有数千骑。见广,以为诱骑,皆惊,上山陈。广之百骑皆大恐,欲驰还走。

广曰:“吾去大军数十里,今如此以百骑走\footnote{text},匈奴追射我立尽。今我留,匈奴必以我为大军诱,必不敢击我。”

广令诸骑曰:“前!”

前未到匈奴陈二里所,止,令曰:“皆下马解鞍!”

其骑曰:“虏多且近,即有急,奈何?”

广曰:“彼虏以我为走,今皆解鞍以示不走,用坚其意。”

于是胡骑遂不敢击。有白马将出护其兵\footnote{text},李广上马与十馀骑奔射杀胡白马将,而复还至其骑中,解鞍,令士皆纵马卧。是时会暮,胡兵终怪之,不敢击。夜半时,胡兵亦以为汉有伏军于旁欲夜取之,胡皆引兵而去。平旦,李广乃归其大军。大军不知广所之,故弗从。
\end{yuanwen}

\begin{yuanwen}
居久之,孝景崩,武帝立,左右以为广名将也,于是广以上郡太守为未央卫尉\footnote{text},而程不识亦为长乐卫尉\footnote{text}。程不识故与李广俱以边太守将军屯\footnote{text}。及出击胡,而广行无部伍行陈\footnote{text},就善水草屯,舍止,人人自便,不击刁斗以自卫\footnote{text},莫府省约文书籍事\footnote{text},然亦远斥候\footnote{text},未尝遇害。程不识正部曲行伍营陈\footnote{text},击刁斗,士吏治军簿至明,军不得休息,然亦未尝遇害。

不识曰:“李广军极简易,然虏卒犯之\footnote{text},无以禁也;而其士卒亦佚乐,咸乐为之死。我军虽烦扰,然虏亦不得犯我。”

是时汉边郡李广、程不识皆为名将,然匈奴畏李广之略,士卒亦多乐从李广而苦程不识。程不识孝景时以数直谏为太中大夫\footnote{text}。为人廉,谨于文法。
\end{yuanwen}

\begin{yuanwen}
后汉以马邑城诱单于\footnote{text},使大军伏马邑旁谷,而广为骁骑将军,领属护军将军\footnote{text}。是时单于觉之,去,汉军皆无功。

其后四岁,广以卫尉为将军,出雁门击匈奴。匈奴兵多,破败广军,生得广。单于素闻广贤,令曰:“得李广必生致之。”

胡骑得广,广时伤病,置广两马间,络而盛卧广。行十馀里,广详死,睨其旁有一胡儿骑善马,广暂腾而上胡儿马\footnote{text},因推堕儿,取其弓,鞭马南驰数十里,复得其馀军,因引而入塞。匈奴捕者骑数百追之,广行取胡儿弓\footnote{text},射杀追骑,以故得脱。于是至汉,汉下广吏。吏当广所失亡多\footnote{text},为虏所生得,当斩,赎为庶人。
\end{yuanwen}

\begin{yuanwen}
顷之,家居数岁。广家与故颍阴侯孙屏野居蓝田南山中射猎\footnote{text}。尝夜从一骑出,从人田间饮。还至霸陵亭,霸陵尉醉,呵止广。广骑曰:“故李将军。”

尉曰:“今将军尚不得夜行\footnote{text},何乃故也!”

止广宿亭下。居无何,匈奴入杀辽西太守,败韩将军,后韩将军徙右北平,死,于是天子乃召拜广为右北平太守。广即请霸陵尉与俱,至军而斩之。
\end{yuanwen}

\begin{yuanwen}
广居右北平,匈奴闻之,号曰“汉之飞将军”,避之数岁,不敢入右北平。
\end{yuanwen}

\begin{yuanwen}
广出猎,见草中石,以为虎而射之,中石没镞\footnote{text},视之石也。因复更射之,终不能复入石矣。广所居郡闻有虎,尝自射之。及居右北平射虎,虎腾伤广,广亦竟射杀之。
\end{yuanwen}

\begin{yuanwen}
广廉,得赏赐辄分其麾下,饮食与士共之。终广之身,为二千石四十馀年\footnote{text},家无馀财,终不言家产事。广为人长,猨臂,其善射亦天性也,虽其子孙他人学者,莫能及广。广讷口少言\footnote{text},与人居则画地为军陈\footnote{text},射阔狭以饮\footnote{text}。专以射为戏,竟死。广之将兵,乏绝之处\footnote{text},见水,士卒不尽饮,广不近水;士卒不尽食,广不尝食。宽缓不苛,士以此爱乐为用。其射,见敌急\footnote{text},非在数十步之内,度不中不发,发即应弦而倒。用此,其将兵数困辱,其射猛兽亦为所伤云\footnote{text}。
\end{yuanwen}

\begin{yuanwen}
居顷之,石建卒,于是上召广代建为郎中令\footnote{text}。

元朔六年,广复为后将军,从大将军军出定襄\footnote{text},击匈奴。诸将多中首虏率\footnote{text},以功为侯者,而广军无功。

后二岁,广以郎中令将四千骑出右北平,博望侯张骞将万骑与广俱,异道\footnote{text}。行可数百里,匈奴左贤王将四万骑围广\footnote{text}。广军士皆恐,广乃使其子敢往驰之。敢独与数十骑驰,直贯胡骑\footnote{text},出其左右而还,告广曰:“胡虏易与耳\footnote{text}。”

军士乃安。广为圜陈外向\footnote{text},胡急击之,矢下如雨。汉兵死者过半,汉矢且尽。广乃令士持满毋发,而广身自以大黄射其裨将\footnote{text},杀数人,胡虏益解\footnote{text}。会日暮,吏士皆无人色,而广意气自如,益治军。军中自是服其勇也。

明日,复力战,而博望侯军亦至,匈奴军乃解去。汉军罢\footnote{text},弗能追。是时广军几没,罢归。汉法,博望侯留迟后期,当死,赎为庶人。广军功自如,无赏。
\end{yuanwen}

\begin{yuanwen}
初,广之从弟李蔡与广俱事孝文帝。景帝时,蔡积功劳至二千石\footnote{text}。孝武帝时,至代相。以元朔五年为轻车将车,从大将军击右贤王,有功中率,封为乐安侯。元狩二年中,代公孙弘为丞相。

蔡为人在下中\footnote{text},名声出广下甚远,然广不得爵邑\footnote{text},官不过九卿,而蔡为列侯\footnote{text},位至三公\footnote{text},诸广之军吏及士卒或取封侯。

广尝与望气王朔燕语\footnote{text},曰:“自汉击匈奴而广未尝不在其中,而诸部校尉以下,才能不及中人,然以击胡军功取侯者数十人,而广不为后人,然无尺寸之功以得封邑者,何也?岂吾相不当侯邪?且固命也?”

朔曰:“将军自念,岂尝有所恨乎\footnote{text}?”

广曰:“吾尝为陇西守,羌尝反,吾诱而降,降者八百馀人,吾诈而同日杀之。至今大恨独此耳。”

朔曰:“祸莫大于杀已降,此乃将军所以不得侯者也。\footnote{text}”
\end{yuanwen}

\begin{yuanwen}
后二岁,大将军、骠骑将军大出\footnote{text},击匈奴。广数自请行。天子以为老,弗许;良久乃许之,以为前将军。是岁,元狩四年也\footnote{text}。
\end{yuanwen}

\begin{yuanwen}
广既从大将军青击匈奴,既出塞,青捕虏知单于所居,乃自以精兵走之,而令广并于右将军军,出东道\footnote{text}。东道少回远\footnote{text},而大军行水草少,其势不屯行。

广自请曰:“臣部为前将军,今大将军乃徙令臣出东道,且臣结发而与匈奴战\footnote{text},今乃一得当单于,臣愿居前,先死单于。”

大将军青亦阴受上诫,以为李广老,数奇\footnote{text},毋令当单于,恐不得所欲。而是时公孙敖新失侯\footnote{text},为中将军从大将军\footnote{text},大将军亦欲使敖与俱当单于\footnote{text},故徙前将军广。

广时知之,固自辞于大将军\footnote{text}。大将军不听,令长史封书与广之莫府\footnote{text},曰:“急诣部\footnote{text},如书。”

广不谢大将军而起行\footnote{text},意甚愠怒而就部,引兵与右将军食其合军出东道。军亡导\footnote{text},或失道\footnote{text},后大将军。

大将军与单于接战,单于遁走,弗能得而还。南绝幕\footnote{text},遇前将军、右将军。广已见大将军,还入军\footnote{text}。大将军使长史持糒醪遗广\footnote{text},因问广、食其失道状,青欲上书报天子军曲折。广未对,大将军使长史急责广之幕府对簿\footnote{text}。

广曰:“诸校尉无罪,乃我自失道。吾今自上簿。”
\end{yuanwen}

\begin{yuanwen}
至莫府\footnote{text},广谓其麾下曰;“广结发与匈奴大小七十馀战,今幸从大将军出接单于兵,而大将军又徙广部行回远,而又迷失道,岂非天哉\footnote{text}!且广年六十馀矣,终不能复对刀笔之吏\footnote{text}。”

遂引刀自刭。广军士大夫一军皆哭,百姓闻之,知与不知,无老壮皆为垂涕。而右将军独下吏,当死,赎为庶人。
\end{yuanwen}

\begin{yuanwen}
广子三人,曰当户、椒、敢,为郎。天子与韩嫣戏,嫣少不逊,当户击嫣,嫣走。于是天子以为勇。当户早死,拜椒为代郡太守,皆先广死。当户有遗腹子名陵。广死军时,敢从骠骑将军。广死明年,李蔡以丞相坐侵孝景园壖地,当下吏治,蔡亦自杀,不对狱,国除。李敢以校尉从骠骑将军击胡左贤王,力战,夺左贤王鼓旗,斩首多,赐爵关内侯,食邑二百户,代广为郎中令。顷之,怨大将军青之恨其父,乃击伤大将军,大将军匿讳之。居无何,敢从上雍,至甘泉宫猎。骠骑将军去病与青有亲,射杀敢。去病时方贵幸,上讳云鹿触杀之。居岁馀,去病死。而敢有女为太子中人,爱幸,敢男禹有宠于太子,然好利,李氏陵迟衰微矣。

李陵既壮,选为建章监,监诸骑。善射,爱士卒。天子以为李氏世将,而使将八百骑。尝深入匈奴二千馀里,过居延视地形,无所见虏而还。拜为骑都尉,将丹阳楚人五千人,教射酒泉、张掖以屯卫胡。

数岁,天汉二年秋,贰师将军李广利将三万骑击匈奴右贤王于祁连天山,而使陵将其射士步兵五千人出居延北可千馀里,欲以分匈奴兵,毋令专走贰师也。陵既至期还,而单于以兵八万围击陵军。陵军五千人,兵矢既尽,士死者过半,而所杀伤匈奴亦万馀人。且引且战,连斗八日,还未到居延百馀里,匈奴遮狭绝道,陵食乏而救兵不到,虏急击招降陵。陵曰:“无面目报陛下。”遂降匈奴。其兵尽没,馀亡散得归汉者四百馀人。

单于既得陵,素闻其家声,及战又壮,乃以其女妻陵而贵之。汉闻,族陵母妻子。自是之后,李氏名败,而陇西之士居门下者皆用为耻焉。
\end{yuanwen}

\begin{yuanwen}
太史公曰:“传曰‘其身正,不令而行;其身不正,虽令不从’。其李将军之谓也?余睹李将军悛悛如鄙人\footnote{text},口不能道辞。及死之日,天下知与不知,皆为尽哀。彼其忠实心诚信于士大夫也\footnote{text}?谚曰‘桃李不言,下自成蹊\footnote{text}’。此言虽小,可以谕大也。”
\end{yuanwen}

\begin{yuanwen}
猿臂善射,实负其能。解鞍卻敌,圆阵摧锋。边郡屡守,大军再从。失道见斥,数奇不封。惜哉名将,天下无双!
\end{yuanwen}

\chapter{匈奴列传}

\begin{yuanwen}
匈奴,其先祖夏后氏之苗裔也,曰淳维。唐虞以上有山戎、獫狁、荤粥,居于北蛮,随畜牧而转移。其畜之所多则马、牛、羊,其奇畜则橐扆、驴、□、□駃騠、□騊駼、驒騱。逐水草迁徙,毋城郭常处耕田之业,然亦各有分地。毋文书,以言语为约束。兒能骑羊,引弓射鸟鼠;少长则射狐兔:用为食。士力能毌弓,尽为甲骑。其俗,宽则随畜,因射猎禽兽为生业,急则人习战攻以侵伐,其天性也。其长兵则弓矢,短兵则刀鋋。利则进,不利则退,不羞遁走。苟利所在,不知礼义。自君王以下,咸食畜肉,衣其皮革,被旃裘。壮者食肥美,老者食其馀。贵壮健,贱老弱。父死,妻其后母;兄弟死,皆取其妻妻之。其俗有名不讳,而无姓字。

夏道衰,而公刘失其稷官,变于西戎,邑于豳。其后三百有馀岁,戎狄攻大王亶父,亶父亡走岐下,而豳人悉从亶父而邑焉,作周。其后百有馀岁,周西伯昌伐畎夷氏。后十有馀年,武王伐纣而营雒邑,复居于酆鄗,放逐戎夷泾、洛之北,以时入贡,命曰“荒服”。其后二百有馀年,周道衰,而穆王伐犬戎,得四白狼四白鹿以归。自是之后,荒服不至。于是周遂作甫刑之辟。穆王之后二百有馀年,周幽王用宠姬襃姒之故,与申侯有卻。申侯怒而与犬戎共攻杀周幽王于骊山之下,遂取周之焦穫,而居于泾渭之间,侵暴中国。秦襄公救周,于是周平王去酆鄗而东徙雒邑。当是之时,秦襄公伐戎至岐,始列为诸侯。是后六十有五年,而山戎越燕而伐齐,齐釐公与战于齐郊。其后四十四年,而山戎伐燕。燕告急于齐,齐桓公北伐山戎,山戎走。其后二十有馀年,而戎狄至洛邑,伐周襄王,襄王奔于郑之氾邑。初,周襄王欲伐郑,故娶戎狄女为后,与戎狄兵共伐郑。已而黜狄后,狄后怨,而襄王后母曰惠后,有子子带,欲立之,于是惠后与狄后、子带为内应,开戎狄,戎狄以故得入,破逐周襄王,而立子带为天子。于是戎狄或居于陆浑,东至于卫,侵盗暴虐中国。中国疾之,故诗人歌之曰“戎狄是应”,“薄伐獫狁,至于大原”,“出舆彭彭,城彼朔方”。周襄王既居外四年,乃使使告急于晋。晋文公初立,欲修霸业,乃兴师伐逐戎翟,诛子带,迎内周襄王,居于雒邑。

当是之时,秦晋为彊国。晋文公攘戎翟,居于河西、洛之间,号曰赤翟、白翟。秦穆公得由余,西戎八国服于秦,故自陇以西有绵诸、绲戎、翟、镕之戎,岐、梁山、泾、漆之北有义渠、大荔、乌氏、朐衍之戎。而晋北有林胡、楼烦之戎,燕北有东胡、山戎。各分散居谿谷,自有君长,往往而聚者百有馀戎,然莫能相一。

自是之后百有馀年,晋悼公使魏绛和戎翟,戎翟朝晋。后百有馀年,赵襄子逾句注而破并代以临胡貉。其后既与韩魏共灭智伯,分晋地而有之,则赵有代、句注之北,魏有河西、上郡,以与戎界边。其后义渠之戎筑城郭以自守,而秦稍蚕食,至于惠王,遂拔义渠二十五城。惠王击魏,魏尽入西河及上郡于秦。秦昭王时,义渠戎王与宣太后乱,有二子。宣太后诈而杀义渠戎王于甘泉,遂起兵伐残义渠。于是秦有陇西、北地、上郡,筑长城以拒胡。而赵武灵王亦变俗胡服,习骑射,北破林胡、楼烦。筑长城,自代并阴山下,至高阙为塞。而置云中、雁门、代郡。其后燕有贤将秦开,为质于胡,胡甚信之。归而袭破走东胡,东胡卻千馀里。与荆轲刺秦王秦舞阳者,开之孙也。燕亦筑长城,自造阳至襄平。置上谷、渔阳、右北平、辽西、辽东郡以拒胡。当是之时,冠带战国七,而三国边于匈奴。其后赵将李牧时,匈奴不敢入赵边。后秦灭六国,而始皇帝使蒙恬将十万之众北击胡,悉收河南地。因河为塞,筑四十四县城临河,徙適戍以充之。而通直道,自九原至云阳,因边山险巉谿谷可缮者治之,起临洮至辽东万馀里。又度河据阳山北假中。

当是之时,东胡彊而月氏盛。匈奴单于曰头曼,头曼不胜秦,北徙。十馀年而蒙恬死,诸侯畔秦,中国扰乱,诸秦所徙適戍边者皆复去,于是匈奴得宽,复稍度河南与中国界于故塞。

单于有太子名冒顿。后有所爱阏氏,生少子,而单于欲废冒顿而立少子,乃使冒顿质于月氏。冒顿既质于月氏,而头曼急击月氏。月氏欲杀冒顿,冒顿盗其善马,骑之亡归。头曼以为壮,令将万骑。冒顿乃作为鸣镝,习勒其骑射,令曰:“鸣镝所射而不悉射者,斩之。”行猎鸟兽,有不射鸣镝所射者,辄斩之。已而冒顿以鸣镝自射其善马,左右或不敢射者,冒顿立斩不射善马者。居顷之,复以鸣镝自射其爱妻,左右或颇恐,不敢射,冒顿又复斩之。居顷之,冒顿出猎,以鸣镝射单于善马,左右皆射之。于是冒顿知其左右皆可用。从其父单于头曼猎,以鸣镝射头曼,其左右亦皆随鸣镝而射杀单于头曼,遂尽诛其后母与弟及大臣不听从者。冒顿自立为单于。

冒顿既立,是时东胡彊盛,闻冒顿杀父自立,乃使使谓冒顿,欲得头曼时有千里马。冒顿问群臣,群臣皆曰:“千里马,匈奴宝马也,勿与。”冒顿曰:“柰何与人邻国而爱一马乎?”遂与之千里马。居顷之,东胡以为冒顿畏之,乃使使谓冒顿,欲得单于一阏氏。冒顿复问左右,左右皆怒曰:“东胡无道,乃求阏氏!请击之。”冒顿曰:“柰何与人邻国爱一女子乎?”遂取所爱阏氏予东胡。东胡王愈益骄,西侵。与匈奴间,中有弃地,莫居,千馀里,各居其边为瓯脱。东胡使使谓冒顿曰:“匈奴所与我界瓯脱外弃地,匈奴非能至也,吾欲有之。”冒顿问群臣,群臣或曰:“此弃地,予之亦可,勿予亦可。”于是冒顿大怒曰:“地者,国之本也,柰何予之!”诸言予之者,皆斩之。冒顿上马,令国中有后者斩,遂东袭击东胡。东胡初轻冒顿,不为备。及冒顿以兵至,击,大破灭东胡王,而虏其民人及畜产。既归,西击走月氏,南并楼烦、白羊河南王。悉复收秦所使蒙恬所夺匈奴地者,与汉关故河南塞,至朝、肤施,遂侵燕、代。是时汉兵与项羽相距,中国罢于兵革,以故冒顿得自彊,控弦之士三十馀万。

自淳维以至头曼千有馀岁,时大时小,别散分离,尚矣,其世传不可得而次云。然至冒顿而匈奴最彊大,尽服从北夷,而南与中国为敌国,其世传国官号乃可得而记云。

置左右贤王,左右谷蠡王,左右大将,左右大都尉,左右大当户,左右骨都侯。匈奴谓贤曰“屠耆”,故常以太子为左屠耆王。自如左右贤王以下至当户,大者万骑,小者数千,凡二十四长,立号曰“万骑”。诸大臣皆世官。呼衍氏,兰氏,其后有须卜氏,此三姓其贵种也。诸左方王将居东方,直上谷以往者,东接秽貉、朝鲜;右方王将居西方,直上郡以西,接月氏、氐、羌;而单于之庭直代、云中:各有分地,逐水草移徙。而左右贤王、左右谷蠡王最为大,左右骨都侯辅政。诸二十四长亦各自置千长、百长、什长、裨小王、相、封都尉、当户、且渠之属。

岁正月,诸长小会单于庭,祠。五月,大会茏城,祭其先、天地、鬼神。秋,马肥,大会蹛林,课校人畜计。其法,拔刃尺者死,坐盗者没入其家;有罪小者轧,大者死。狱久者不过十日,一国之囚不过数人。而单于朝出营,拜日之始生,夕拜月。其坐,长左而北乡。日上戊己。其送死,有棺椁金银衣裘,而无封树丧服;近幸臣妾从死者,多至数千百人。举事而候星月,月盛壮则攻战,月亏则退兵。其攻战,斩首虏赐一卮酒,而所得卤获因以予之,得人以为奴婢。故其战,人人自为趣利,善为诱兵以冒敌。故其见敌则逐利,如鸟之集;其困败,则瓦解云散矣。战而扶舆死者,尽得死者家财。

后北服浑庾、屈射、丁零、鬲昆、薪犁之国。于是匈奴贵人大臣皆服,以冒顿单于为贤。

是时汉初定中国,徙韩王信于代,都马邑。匈奴大攻围马邑,韩王信降匈奴。匈奴得信,因引兵南逾句注,攻太原,至晋阳下。高帝自将兵往击之。会冬大寒雨雪,卒之堕指者十二三,于是冒顿详败走,诱汉兵。汉兵逐击冒顿,冒顿匿其精兵,见其羸弱,于是汉悉兵,多步兵,三十二万,北逐之。高帝先至平城,步兵未尽到,冒顿纵精兵四十万骑围高帝于白登,七日,汉兵中外不得相救饷。匈奴骑,其西方尽白马,东方尽青駹马,北方尽乌骊马,南方尽骍马。高帝乃使使间厚遗阏氏,阏氏乃谓冒顿曰:“两主不相困。今得汉地,而单于终非能居之也。且汉王亦有神,单于察之。”冒顿与韩王信之将王黄、赵利期,而黄、利兵又不来,疑其与汉有谋,亦取阏氏之言,乃解围之一角。于是高帝令士皆持满傅矢外乡,从解角直出,竟与大军合,而冒顿遂引兵而去。汉亦引兵而罢,使刘敬结和亲之约。

是后韩王信为匈奴将,及赵利、王黄等数倍约,侵盗代、云中。居无几何,陈豨反,又与韩信合谋击代。汉使樊哙往击之,复拔代、雁门、云中郡县,不出塞。是时匈奴以汉将众往降,故冒顿常往来侵盗代地。于是汉患之,高帝乃使刘敬奉宗室女公主为单于阏氏,岁奉匈奴絮缯酒米食物各有数,约为昆弟以和亲,冒顿乃少止。后燕王卢绾反,率其党数千人降匈奴,往来苦上谷以东。

高祖崩,孝惠、吕太后时,汉初定,故匈奴以骄。冒顿乃为书遗高后,妄言。高后欲击之,诸将曰:“以高帝贤武,然尚困于平城。”于是高后乃止,复与匈奴和亲。

至孝文帝初立,复修和亲之事。其三年五月,匈奴右贤王入居河南地,侵盗上郡葆塞蛮夷,杀略人民。于是孝文帝诏丞相灌婴发车骑八万五千,诣高奴,击右贤王。右贤王走出塞。文帝幸太原。是时济北王反,文帝归,罢丞相击胡之兵。

其明年,单于遗汉书曰:“天所立匈奴大单于敬问皇帝无恙。前时皇帝言和亲事,称书意,合欢。汉边吏侵侮右贤王,右贤王不请,听后义卢侯难氏等计,与汉吏相距,绝二主之约,离兄弟之亲。皇帝让书再至,发使以书报,不来,汉使不至,汉以其故不和,邻国不附。今以小吏之败约故,罚右贤王,使之西求月氏击之。以天之福,吏卒良,马彊力,以夷灭月氏,尽斩杀降下之。定楼兰、乌孙、呼揭及其旁二十六国,皆以为匈奴。诸引弓之民,并为一家。北州已定,原寝兵休士卒养马,除前事,复故约,以安边民,以应始古,使少者得成其长,老者安其处,世世平乐。未得皇帝之志也,故使郎中系雩浅奉书请,献橐他一匹,骑马二匹,驾二驷。皇帝即不欲匈奴近塞,则且诏吏民远舍。使者至,即遣之。”以六月中来至薪望之地。书至,汉议击与和亲孰便。公卿皆曰:“单于新破月氏,乘胜,不可击。且得匈奴地,泽卤,非可居也。和亲甚便。”汉许之。

孝文皇帝前六年,汉遗匈奴书曰:“皇帝敬问匈奴大单于无恙。使郎中系雩浅遗朕书曰:‘右贤王不请,听后义卢侯难氏等计,绝二主之约,离兄弟之亲,汉以故不和,邻国不附。今以小吏败约,故罚右贤王使西击月氏,尽定之。原寝兵休士卒养马,除前事,复故约,以安边民,使少者得成其长,老者安其处,世世平乐。’ 朕甚嘉之,此古圣主之意也。汉与匈奴约为兄弟,所以遗单于甚厚。倍约离兄弟之亲者,常在匈奴。然右贤王事已在赦前,单于勿深诛。单于若称书意,明告诸吏,使无负约,有信,敬如单于书。使者言单于自将伐国有功,甚苦兵事。服绣袷绮衣、绣袷长襦、锦袷袍各一,比余一,黄金饰具带一,黄金胥纰一,绣十匹,锦三十匹,赤綈、绿缯各四十匹,使中大夫意、谒者令肩遗单于。”

后顷之,冒顿死,子稽粥立,号曰老上单于。

老上稽粥单于初立,孝文皇帝复遣宗室女公主为单于阏氏,使宦者燕人中行说傅公主。说不欲行,汉彊使之。说曰:“必我行也,为汉患者。”中行说既至,因降单于,单于甚亲幸之。

初,匈奴好汉缯絮食物,中行说曰:“匈奴人众不能当汉之一郡,然所以彊者,以衣食异,无仰于汉也。今单于变俗好汉物,汉物不过什二,则匈奴尽归于汉矣。其得汉缯絮,以驰草棘中,衣袴皆裂敝,以示不如旃裘之完善也。得汉食物皆去之,以示不如湩酪之便美也。”于是说教单于左右疏记,以计课其人众畜物。

汉遗单于书,牍以尺一寸,辞曰“皇帝敬问匈奴大单于无恙”,所遗物及言语云云。中行说令单于遗汉书以尺二寸牍,及印封皆令广大长,倨傲其辞曰“天地所生日月所置匈奴大单于敬问汉皇帝无恙”,所以遗物言语亦云云。

汉使或言曰:“匈奴俗贱老。”中行说穷汉使曰:“而汉俗屯戍从军当发者,其老亲岂有不自脱温厚肥美以赍送饮食行戍乎?”汉使曰:“然。”中行说曰:“匈奴明以战攻为事,其老弱不能斗,故以其肥美饮食壮健者,盖以自为守卫,如此父子各得久相保,何以言匈奴轻老也?”汉使曰:“匈奴父子乃同穹庐而卧。父死,妻其后母;兄弟死,尽取其妻妻之。无冠带之饰,阙庭之礼。”中行说曰:“匈奴之俗,人食畜肉,饮其汁,衣其皮;畜食草饮水,随时转移。故其急则人习骑射,宽则人乐无事,其约束轻,易行也。君臣简易,一国之政犹一身也。父子兄弟死,取其妻妻之,恶种姓之失也。故匈奴虽乱,必立宗种。今中国虽详不取其父兄之妻,亲属益疏则相杀,至乃易姓,皆从此类。且礼义之敝,上下交怨望,而室屋之极,生力必屈。夫力耕桑以求衣食,筑城郭以自备,故其民急则不习战功,缓则罢于作业。嗟土室之人,顾无多辞,令喋喋而佔々,冠固何当?”

自是之后,汉使欲辩论者,中行说辄曰:“汉使无多言,顾汉所输匈奴缯絮米糵,令其量中,必善美而己矣,何以为言乎?且所给备善则已;不备,苦恶,则候秋孰,以骑驰蹂而稼穑耳。”日夜教单于候利害处。

汉孝文皇帝十四年,匈奴单于十四万骑入朝、萧关,杀北地都尉卬,虏人民畜产甚多,遂至彭阳。使奇兵入烧回中宫,候骑至雍甘泉。于是文帝以中尉周舍、郎中令张武为将军,发车千乘,骑十万,军长安旁以备胡寇。而拜昌侯卢卿为上郡将军,甯侯魏为北地将军,隆虑侯周灶为陇西将军,东阳侯张相如为大将军,成侯董赤为前将军,大发车骑往击胡。单于留塞内月馀乃去,汉逐出塞即还,不能有所杀。匈奴日已骄,岁入边,杀略人民畜产甚多,云中、辽东最甚,至代郡万馀人。汉患之,乃使使遗匈奴书。单于亦使当户报谢,复言和亲事。

孝文帝后二年,使使遗匈奴书曰:“皇帝敬问匈奴大单于无恙。使当户且居雕渠难、郎中韩辽遗朕马二匹,已至,敬受。先帝制:长城以北,引弓之国,受命单于;长城以内,冠带之室,朕亦制之。使万民耕织射猎衣食,父子无离,臣主相安,俱无暴逆。今闻渫恶民贪降其进取之利,倍义绝约,忘万民之命,离两主之驩,然其事已在前矣。书曰:‘二国已和亲,两主驩说,寝兵休卒养马,世世昌乐,闟然更始。’ 朕甚嘉之。圣人者日新,改作更始,使老者得息,幼者得长,各保其首领而终其天年。朕与单于俱由此道,顺天恤民,世世相传,施之无穷,天下莫不咸便。汉与匈奴邻国之敌,匈奴处北地,寒,杀气早降,故诏吏遗单于秫糵金帛丝絮佗物岁有数。今天下大安,万民熙熙,朕与单于为之父母。朕追念前事,薄物细故,谋臣计失,皆不足以离兄弟之驩。朕闻天不颇覆,地不偏载。朕与单于皆捐往细故,俱蹈大道,堕坏前恶,以图长久,使两国之民若一家子。元元万民,下及鱼鳖,上及飞鸟,跂行喙息蠕动之类,莫不就安利而辟危殆。故来者不止,天之道也。俱去前事:朕释逃虏民,单于无言章尼等。朕闻古之帝王,约分明而无食言。单于留志,天下大安,和亲之后,汉过不先。单于其察之。”

单于既约和亲,于是制诏御史曰:“匈奴大单于遗朕书,言和亲已定,亡人不足以益众广地,匈奴无入塞,汉无出塞,犯约者杀之,可以久亲,后无咎,俱便。朕已许之。其布告天下,使明知之。”

后四岁,老上稽粥单于死,子军臣立为单于。既立,孝文皇帝复与匈奴和亲。而中行说复事之。

军臣单于立四岁,匈奴复绝和亲,大入上郡、云中各三万骑,所杀略甚众而去。于是汉使三将军军屯北地,代屯句注,赵屯飞狐口,缘边亦各坚守以备胡寇。又置三将军,军长安西细柳、渭北棘门、霸上以备胡。胡骑入代句注边,烽火通于甘泉、长安。数月,汉兵至边,匈奴亦去远塞,汉兵亦罢。后岁馀,孝文帝崩,孝景帝立,而赵王遂乃阴使人于匈奴。吴楚反,欲与赵合谋入边。汉围破赵,匈奴亦止。自是之后,孝景帝复与匈奴和亲,通关市,给遗匈奴,遣公主,如故约。终孝景时,时小入盗边,无大寇。

今帝即位,明和亲约束,厚遇,通关市,饶给之。匈奴自单于以下皆亲汉,往来长城下。

汉使马邑下人聂翁壹奸兰出物与匈奴交,详为卖马邑城以诱单于。单于信之,而贪马邑财物,乃以十万骑入武州塞。汉伏兵三十馀万马邑旁,御史大夫韩安国为护军,护四将军以伏单于。单于既入汉塞,未至马邑百馀里,见畜布野而无人牧者,怪之,乃攻亭。是时雁门尉史行徼,见寇,葆此亭,知汉兵谋,单于得,欲杀之,尉史乃告单于汉兵所居。单于大惊曰:“吾固疑之。”乃引兵还。出曰:“吾得尉史,天也,天使若言。”以尉史为“天王”。汉兵约单于入马邑而纵,单于不至,以故汉兵无所得。汉将军王恢部出代击胡辎重,闻单于还,兵多,不敢出。汉以恢本造兵谋而不进,斩恢。自是之后,匈奴绝和亲,攻当路塞,往往入盗于汉边,不可胜数。然匈奴贪,尚乐关市,嗜汉财物,汉亦尚关市不绝以中之。

自马邑军后五年之秋,汉使四将军各万骑击胡关市下。将军卫青出上谷,至茏城,得胡首虏七百人。公孙贺出云中,无所得。公孙敖出代郡,为胡所败七千馀人。李广出雁门,为胡所败,而匈奴生得广,广后得亡归。汉囚敖、广,敖、广赎为庶人。其冬,匈奴数入盗边,渔阳尤甚。汉使将军韩安国屯渔阳备胡。其明年秋,匈奴二万骑入汉,杀辽西太守,略二千馀人。胡又入败渔阳太守军千馀人,围汉将军安国,安国时千馀骑亦且尽,会燕救至,匈奴乃去。匈奴又入雁门,杀略千馀人。于是汉使将军卫青将三万骑出雁门,李息出代郡,击胡。得首虏数千人。其明年,卫青复出云中以西至陇西,击胡之楼烦、白羊王于河南,得胡首虏数千,牛羊百馀万。于是汉遂取河南地,筑朔方,复缮故秦时蒙恬所为塞,因河为固。汉亦弃上谷之什辟县造阳地以予胡。是岁,汉之元朔二年也。

其后冬,匈奴军臣单于死。军臣单于弟左谷蠡王伊稚斜自立为单于,攻破军臣单于太子于单。于单亡降汉,汉封于单为涉安侯,数月而死。

伊稚斜单于既立,其夏,匈奴数万骑入杀代郡太守恭友,略千馀人。其秋,匈奴又入雁门,杀略千馀人。其明年,匈奴又复复入代郡、定襄、上郡,各三万骑,杀略数千人。匈奴右贤王怨汉夺之河南地而筑朔方,数为寇,盗边,及入河南,侵扰朔方,杀略吏民其众。

其明年春,汉以卫青为大将军,将六将军,十馀万人,出朔方、高阙击胡。右贤王以为汉兵不能至,饮酒醉,汉兵出塞六七百里,夜围右贤王。右贤王大惊,脱身逃走,诸精骑往往随后去。汉得右贤王众男女万五千人,裨小王十馀人。其秋,匈奴万骑入杀代郡都尉硃英,略千馀人。

其明年春,汉复遣大将军卫青将六将军,兵十馀万骑,乃再出定襄数百里击匈奴,得首虏前后凡万九千馀级,而汉亦亡两将军,军三千馀骑。右将军建得以身脱,而前将军翕侯赵信兵不利,降匈奴。赵信者,故胡小王,降汉,汉封为翕侯,以前将军与右将军并军分行,独遇单于兵,故尽没。单于既得翕侯,以为自次王,用其姊妻之,与谋汉。信教单于益北绝幕,以诱罢汉兵,徼极而取之,无近塞。单于从其计。其明年,胡骑万人入上谷,杀数百人。

其明年春,汉使骠骑将军去病将万骑出陇西,过焉支山千馀里,击匈奴,得胡首虏万八千馀级,破得休屠王祭天金人。其夏,骠骑将军复与合骑侯数万骑出陇西、北地二千里,击匈奴。过居延,攻祁连山,得胡首虏三万馀人,裨小王以下七十馀人。是时匈奴亦来入代郡、雁门,杀略数百人。汉使博望侯及李将军广出右北平,击匈奴左贤王。左贤王围李将军,卒可四千人,且尽,杀虏亦过当。会博望侯军救至,李将军得脱。汉失亡数千人,合骑侯后骠骑将军期,及与博望侯皆当死,赎为庶人。

其秋,单于怒浑邪王、休屠王居西方为汉所杀虏数万人,欲召诛之。浑邪王与休屠王恐,谋降汉,汉使骠骑将军往迎之。浑邪王杀休屠王,并将其众降汉。凡四万馀人,号十万。于是汉已得浑邪王,则陇西、北地、河西益少胡寇,徙关东贫民处所夺匈奴河南、新秦中以实之,而减北地以西戍卒半。其明年,匈奴入右北平、定襄各数万骑,杀略千馀人而去。

其明年春,汉谋曰“翕侯信为单于计,居幕北,以为汉兵不能至”。乃粟马发十万骑,私从马凡十四万匹,粮重不与焉。令大将军青、骠骑将军去病中分军,大将军出定襄,骠骑将军出代,咸约绝幕击匈奴。单于闻之,远其辎重,以精兵待于幕北。与汉大将军接战一日,会暮,大风起,汉兵纵左右翼围单于。单于自度战不能如汉兵,单于遂独身与壮骑数百溃汉围西北遁走。汉兵夜追不得。行斩捕匈奴首虏万九千级,北至阗颜山赵信城而还。

单于之遁走,其兵往往与汉兵相乱而随单于。单于久不与其大众相得,其右谷蠡王以为单于死,乃自立为单于。真单于复得其众,而右谷蠡王乃去其单于号,复为右谷蠡王。

汉骠骑将军之出代二千馀里,与左贤王接战,汉兵得胡首虏凡七万馀级,左贤王将皆遁走。骠骑封于狼居胥山,禅姑衍,临翰海而还。

是后匈奴远遁,而幕南无王庭。汉度河自朔方以西至令居,往往通渠置田,官吏卒五六万人,稍蚕食,地接匈奴以北。

初,汉两将军大出围单于,所杀虏八九万,而汉士卒物故亦数万,汉马死者十馀万。匈奴虽病,远去,而汉亦马少,无以复往。匈奴用赵信之计,遣使于汉,好辞请和亲。天子下其议,或言和亲,或言遂臣之。丞相长史任敞曰:“匈奴新破,困,宜可使为外臣,朝请于边。”汉使任敞于单于。单于闻敞计,大怒,留之不遣。先是汉亦有所降匈奴使者,单于亦辄留汉使相当。汉方复收士马,会骠骑将军去病死,于是汉久不北击胡。

数岁,伊稚斜单于立十三年死,子乌维立为单于。是岁,汉元鼎三年也。乌维单于立,而汉天子始出巡郡县。其后汉方南诛两越,不击匈奴,匈奴亦不侵入边。

乌维单于立三年,汉已灭南越,遣故太仆贺将万五千骑出九原二千馀里,至浮苴井而还,不见匈奴一人。汉又遣故从骠侯赵破奴万馀骑出令居数千里,至匈河水而还,亦不见匈奴一人。

是时天子巡边,至朔方,勒兵十八万骑以见武节,而使郭吉风告单于。郭吉既至匈奴,匈奴主客问所使,郭吉礼卑言好,曰:“吾见单于而口言。”单于见吉,吉曰:“南越王头已悬于汉北阙。今单于即前与汉战,天子自将兵待边;单于即不能,即南面而臣于汉。何徒远走,亡匿于幕北寒苦无水草之地,毋为也。”语卒而单于大怒,立斩主客见者,而留郭吉不归,迁之北海上。而单于终不肯为寇于汉边,休养息士马,习射猎,数使使于汉,好辞甘言求请和亲。

汉使王乌等窥匈奴。匈奴法,汉使非去节而以墨黥其面者不得入穹庐。王乌,北地人,习胡俗,去其节,黥面,得入穹庐。单于爱之,详许甘言,为遣其太子入汉为质,以求和亲。

汉使杨信于匈奴。是时汉东拔秽貉、朝鲜以为郡,而西置酒泉郡以鬲绝胡与羌通之路。汉又西通月氏、大夏,又以公主妻乌孙王,以分匈奴西方之援国。又北益广田至胘雷为塞,而匈奴终不敢以为言。是岁,翕侯信死,汉用事者以匈奴为已弱,可臣从也。杨信为人刚直屈彊,素非贵臣,单于不亲。单于欲召入,不肯去节,单于乃坐穹庐外见杨信。杨信既见单于,说曰:“即欲和亲,以单于太子为质于汉。”单于曰:“非故约。故约,汉常遣翁主,给缯絮食物有品,以和亲,而匈奴亦不扰边。今乃欲反古,令吾太子为质,无几矣。”匈奴俗,见汉使非中贵人,其儒先,以为欲说,折其辩;其少年,以为欲刺,折其气。每汉使入匈奴,匈奴辄报偿。汉留匈奴使,匈奴亦留汉使,必得当乃肯止。

杨信既归,汉使王乌,而单于复以甘言,欲多得汉财物,绐谓王乌曰:“吾欲入汉见天子,面相约为兄弟。”王乌归报汉,汉为单于筑邸于长安。匈奴曰:“非得汉贵人使,吾不与诚语。”匈奴使其贵人至汉,病,汉予药,欲愈之,不幸而死。而汉使路充国佩二千石印绶往使,因送其丧,厚葬直数千金,曰“此汉贵人也”。单于以为汉杀吾贵使者,乃留路充国不归。诸所言者,单于特空绐王乌,殊无意入汉及遣太子来质。于是匈奴数使奇兵侵犯边。汉乃拜郭昌为拔胡将军,及浞野侯屯朔方以东,备胡。路充国留匈奴三岁,单于死。

乌维单于立十岁而死,子乌师庐立为单于。年少,号为兒单于。是岁元封六年也。自此之后,单于益西北,左方兵直云中,右方直酒泉、燉煌郡。

兒单于立,汉使两使者,一吊单于,一吊右贤王,欲以乖其国。使者入匈奴,匈奴悉将致单于。单于怒而尽留汉使。汉使留匈奴者前后十馀辈,而匈奴使来,汉亦辄留相当。

是岁,汉使贰师将军广利西伐大宛,而令因杅将军敖筑受降城。其冬,匈奴大雨雪,畜多饥寒死。兒单于年少,好杀伐,国人多不安。左大都尉欲杀单于,使人间告汉曰:“我欲杀单于降汉,汉远,即兵来迎我,我即发。”初,汉闻此言,故筑受降城,犹以为远。

其明年春,汉使浞野侯破奴将二万馀骑出朔方西北二千馀里,期至浚稽山而还。浞野侯既至期而还,左大都尉欲发而觉,单于诛之,发左方兵击浞野。浞野侯行捕首虏得数千人。还,未至受降城四百里,匈奴兵八万骑围之。浞野侯夜自出求水,匈奴间捕,生得浞野侯,因急击其军。军中郭纵为护,维王为渠,相与谋曰:“及诸校尉畏亡将军而诛之,莫相劝归。”军遂没于匈奴。匈奴兒单于大喜,遂遣奇兵攻受降城。不能下,乃寇入边而去。其明年,单于欲自攻受降城,未至,病死。

兒单于立三岁而死。子年少,匈奴乃立其季父乌维单于弟右贤王呴犁湖为单于。是岁太初三年也。

呴犁湖单于立,汉使光禄徐自为出五原塞数百里,远者千馀里,筑城鄣列亭至庐朐,而使游击将军韩说、长平侯卫伉屯其旁,使彊弩都尉路博德筑居延泽上。

其秋,匈奴大入定襄、云中,杀略数千人,败数二千石而去,行破坏光禄所筑城列亭鄣。又使右贤王入酒泉、张掖,略数千人。会任文击救,尽复失所得而去。是岁,贰师将军破大宛,斩其王而还。匈奴欲遮之,不能至。其冬,欲攻受降城,会单于病死。

呴犁湖单于立一岁死。匈奴乃立其弟左大都尉且鞮侯为单于。

汉既诛大宛,威震外国。天子意欲遂困胡,乃下诏曰:“高皇帝遗朕平城之忧,高后时单于书绝悖逆。昔齐襄公复九世之雠,春秋大之。”是岁太初四年也。

且鞮侯单于既立,尽归汉使之不降者。路充国等得归。单于初立,恐汉袭之,乃自谓“我兒子,安敢望汉天子!汉天子,我丈人行也”。汉遣中郎将苏武厚币赂遗单于。单于益骄,礼甚倨,非汉所望也。其明年,浞野侯破奴得亡归汉。

其明年,汉使贰师将军广利以三万骑出酒泉,击右贤王于天山,得胡首虏万馀级而还。匈奴大围贰师将军,几不脱。汉兵物故什六七。汉复使因杅将军敖出西河,与彊弩都尉会涿涂山,毋所得。又使骑都尉李陵将步骑五千人,出居延北千馀里,与单于会,合战,陵所杀伤万馀人,兵及食尽,欲解归,匈奴围陵,陵降匈奴,其兵遂没,得还者四百人。单于乃贵陵,以其女妻之。

后二岁,复使贰师将军将六万骑,步兵十万,出朔方。彊弩都尉路博德将万馀人,与贰师会。游击将军说将步骑三万人,出五原。因杅将军敖将万骑步兵三万人,出雁门。匈奴闻,悉远其累重于余吾水北,而单于以十万骑待水南,与贰师将军接战。贰师乃解而引归,与单于连战十馀日。贰师闻其家以巫蛊族灭,因并众降匈奴,得来还千人一两人耳。游击说无所得。因杅敖与左贤王战,不利,引归。是岁汉兵之出击匈奴者不得言功多少,功不得御。有诏捕太医令随但,言贰师将军家室族灭,使广利得降匈奴。

太史公曰:孔氏著春秋,隐桓之间则章,至定哀之际则微,为其切当世之文而罔襃,忌讳之辞也。世俗之言匈奴者,患其徼一时之权,而务纳其说,以便偏指,不参彼己;将率席中国广大,气奋,人主因以决策,是以建功不深。尧虽贤,兴事业不成,得禹而九州宁。且欲兴圣统,唯在择任将相哉!唯在择任将相哉!

獫狁、薰粥,居于北边。既称夏裔,式憬周篇。颇随畜牧,屡扰尘烟。爰自冒顿,尤聚控弦。虽空帑藏,未尽中权。
\end{yuanwen}

\chapter{卫将军骠骑列传}

是《史记》中记录武帝对匈奴战争的最主要的一篇文字。卫青、霍去病是汉武帝时代杰出的将领,也是我国古代少有其比的名将,在解除汉帝国来自北部的威胁,并将匈奴势力根本削弱方面,立下了巨大的功勋。卫青比较仁厚,霍去病年轻有为,但由于司马迁对于武帝的武力征伐持否定态度,而且卫青、霍去病又都是武帝的亲戚,他们的发迹不得不说也凭借了一定裙带关系,因此对他们没有好感,这是不太公平的。但他们取得的对匈奴战争的辉煌胜利,足以让汉代人扬眉吐气,这就使司马迁也为之感到鼓舞与自豪,对他们的军事贡献也由衷佩服,尤其对卫青所指挥的那场漠北大战,司马迁可以说是进行了倾心竭力的描写。

\begin{yuanwen}
大将军卫青者\footnote{text},平阳人也。其父郑季,为吏,给事平阳侯家\footnote{text},与侯妾卫媪通\footnote{text},生青。青同母兄卫长子\footnote{text},而姊卫子夫自平阳公主家得幸天子\footnote{text},故冒姓为卫氏\footnote{text},字仲卿。
\end{yuanwen}

\begin{yuanwen}
长子更字长君。长君母号为卫媪。媪长女卫孺,次女少兒,次女卫子夫。后子夫男弟步、广皆冒卫氏。
\end{yuanwen}

\begin{yuanwen}
青为侯家人,少时归其父,其父使牧羊。先母之子皆奴畜之\footnote{text},不以为兄弟数\footnote{text}。青尝从入至甘泉居室\footnote{text},有一钳徒相青曰\footnote{text}:“贵人也,官至封侯。”

青笑曰:“人奴之生\footnote{text},得毋笞骂即足矣,安得封侯事乎!”
\end{yuanwen}

\begin{yuanwen}
青壮,为侯家骑,从平阳主。建元二年春,青姊子夫得入宫幸上。皇后\footnote{text},堂邑大长公主女也\footnote{text},无子,妒。大长公主闻卫子夫幸,有身,妒之,乃使人捕青。

青时给事建章\footnote{text},未知名。大长公主执囚青,欲杀之。其友骑郎公孙敖与壮士往篡取之\footnote{text},以故得不死。上闻,乃召青为建章监\footnote{text},侍中\footnote{text},及同母昆弟贵,赏赐数日间累千金。孺为太仆公孙贺妻。少兒故与陈掌通,上召贵掌。公孙敖由此益贵。子夫为夫人\footnote{text},青为大中大夫\footnote{text}。
\end{yuanwen}

\begin{yuanwen}
元光五年\footnote{text},青为车骑将军\footnote{text},击匈奴,出上谷\footnote{text};太仆公孙贺为轻车将军,出云中;大中大夫公孙敖为骑将军,出代郡;卫尉李广为骁骑将军,出雁门:军各万骑。青至茏城\footnote{text},斩首虏数百。骑将军敖亡七千骑;卫尉李广为虏所得,得脱归:皆当斩,赎为庶人。贺亦无功。
\end{yuanwen}

\begin{yuanwen}
元朔元年春\footnote{text},卫夫人有男,立为皇后。其秋,青为车骑将军,出雁门,三万骑击匈奴,斩首虏数千人\footnote{text}。

明年,匈奴入杀辽西太守\footnote{text},虏略渔阳二千馀人\footnote{text},败韩将军军。汉令将军李息击之,出代;令车骑将军青出云中以西至高阙\footnote{text}。遂略河南地\footnote{text},至于陇西\footnote{text},捕首虏数千,畜数十万,走白羊、楼烦王\footnote{text}。遂以河南地为朔方郡\footnote{text}。以三千八百户封青为长平侯。
\end{yuanwen}

\begin{yuanwen}
青校尉苏建有功,以千一百户封建为平陵侯。使建筑朔方城。青校尉张次公有功,封为岸头侯。天子曰:“匈奴逆天理,乱人伦,暴长虐老,以盗窃为务,行诈诸蛮夷,造谋藉兵,数为边害,故兴师遣将,以征厥罪。诗不云乎,“薄伐玁狁,至于太原”,“出车彭彭,城彼朔方”。今车骑将军青度西河至高阙,获首虏二千三百级,车辎畜产毕收为卤,已封为列侯,遂西定河南地,按榆谿旧塞,绝梓领,梁北河,讨蒲泥,破符离,斩轻锐之卒,捕伏听者三千七十一级,执讯获丑,驱马牛羊百有馀万,全甲兵而还,益封青三千户。”其明年,匈奴入杀代郡太守友,入略雁门千馀人。其明年,匈奴大入代、定襄、上郡,杀略汉数千人。
\end{yuanwen}

\begin{yuanwen}
其明年,元朔之五年春\footnote{text},汉令车骑将军青将三万骑,出高阙;卫尉苏建为游击将军,左内史李沮为强弩将军,太仆公孙贺为骑将军,代相李蔡为轻车将军\footnote{text},皆领属车骑将军,俱出朔方;大行李息、岸头侯张次公为将军\footnote{text},出右北平:咸击匈奴。

匈奴右贤王当卫青等兵\footnote{text},以为汉兵不能至此,饮醉。汉兵夜至,围右贤王,右贤王惊,夜逃,独与其爱妾一人、壮骑数百驰,溃围北去。汉轻骑校尉郭成等逐数百里,不及,得右贤裨王十馀人\footnote{text},众男女万五千馀人,畜数千百万\footnote{text},于是引兵而还\footnote{text}。

至塞,天子使使者持大将军印,即军中拜车骑将军青为大将军,诸将皆以兵属大将军,大将军立号而归\footnote{text}。
\end{yuanwen}

\begin{yuanwen}
天子曰:“大将军青躬率戎士,师大捷,获匈奴王十有馀人,益封青六千户。”而封青子伉为宜春侯,青子不疑为阴安侯,青子登为发干侯。青固谢曰:“臣幸得待罪行间,赖陛下神灵,军大捷,皆诸校尉力战之功也。陛下幸已益封臣青。臣青子在唡褓中,未有勤劳,上幸列地封为三侯,非臣待罪行间所以劝士力战之意也。伉等三人何敢受封!”天子曰:“我非忘诸校尉功也,今固且图之。”乃诏御史曰:“护军都尉公孙敖三从大将军击匈奴,常护军,傅校获王,以千五百户封敖为合骑侯。都尉韩说从大将军出窳浑,至匈奴右贤王庭,为麾下搏战获王,以千三百户封说为龙嵒侯。骑将军公孙贺从大将军获王,以千三百户封贺为南窌侯。轻车将军李蔡再从大将军获王,以千六百户封蔡为乐安侯。校尉李朔,校尉赵不虞,校尉公孙戎奴,各三从大将军获王,以千三百户封朔为涉轵侯,以千三百户封不虞为随成侯,以千三百户封戎奴为从平侯。将军李沮、李息及校尉豆如意有功,赐爵关内侯,食邑各三百户。”其秋,匈奴入代,杀都尉硃英。
\end{yuanwen}

\begin{yuanwen}
其明年春,大将军青出定襄,合骑侯敖为中将军,太仆贺为左将军,翕侯赵信为前将军,卫尉苏建为右将军,郎中令李广为后将军,右内史李沮为彊弩将军,咸属大将军,斩首数千级而还。

月馀,悉复出定襄击匈奴,斩首虏万余人。右将军建、前将军信并军三千余骑,独逢单于兵,与战一日馀,汉兵且尽。

前将军故胡人,降为翕侯,见急,匈奴诱之,遂将其馀骑可八百,奔降单于\footnote{text}。

右将军苏建尽亡其军,独以身得亡去,自归大将军。大将军问其罪正闳、长史安、议郎周霸等\footnote{text}:“建当云何?”

霸曰:“自大将军出,未尝斩裨将。今建弃军,可斩以明将军之威。”

闳、安曰:“不然。兵法‘小敌之坚,大敌之禽也’。今建以数千当单于数万,力战一日馀,士尽,不敢有二心,自归。自归而斩之,是示后无反意也。不当斩。”

大将军曰:“青幸得以肺腑待罪行间\footnote{text},不患无威,而霸说我以明威,甚失臣意。且使臣职虽当斩将,以臣之尊宠而不敢自擅专诛于境外,而具归天子,天子自裁之,于是以见为人臣不敢专权\footnote{text},不亦可乎?”

军吏皆曰“善”。

遂囚建诣行在所\footnote{text}。入塞罢兵\footnote{text}。
\end{yuanwen}

\begin{yuanwen}
是岁也,大将军姊子霍去病年十八,幸,为天子侍中。善骑射,再从大将军,受诏与壮士,为剽姚校尉,与轻勇骑八百直弃大军数百里赴利\footnote{text},斩捕首虏过当\footnote{text}。

于是天子曰:“剽姚校尉去病斩首虏二千二十八级,及相国、当户\footnote{text},斩单于大父行籍若侯产\footnote{text},生捕季父罗姑比\footnote{text},再冠军\footnote{text},以千六百户封去病为冠军侯\footnote{text}。
\end{yuanwen}

\begin{yuanwen}
上谷太守郝贤四从大将军,捕斩首虏二千馀人,以千一百户封贤为众利侯。”是岁,失两将军军,亡翕侯,军功不多,故大将军不益封。右将军建至,天子不诛,赦其罪,赎为庶人。
\end{yuanwen}

\begin{yuanwen}
大将军既还,赐千金。是时王夫人方幸于上,宁乘说大将军曰:“将军所以功未甚多,身食万户,三子皆为侯者,徒以皇后故也\footnote{text}。今王夫人幸而宗族未富贵,愿将军奉所赐千金为王夫人亲寿\footnote{text}。”

大将军乃以五百金为寿。天子闻之,问大将军,大将军以实言,上乃拜宁乘为东海都尉。
\end{yuanwen}

\begin{yuanwen}
张骞从大将军,以尝使大夏,留匈奴中久,导军,知善水草处,军得以无饥渴,因前使绝国功,封骞博望侯。
\end{yuanwen}

\begin{yuanwen}
冠军侯去病既侯三岁,元狩二年春\footnote{text},以冠军侯去病为骠骑将军\footnote{text},将万骑出陇西,有功。

天子曰:“骠骑将军率戎士逾乌盭\footnote{text},讨遬濮\footnote{text},涉狐奴\footnote{text},历五王国,辎重人众慑慴者弗取\footnote{text},冀获单于子\footnote{text}。转战六日,过焉支山千有馀里\footnote{text},合短兵,杀折兰王,斩卢胡王\footnote{text},诛全甲\footnote{text},执浑邪王子及相国、都尉,首虏八千馀级,收休屠祭天金人,益封去病二千户。”
\end{yuanwen}

\begin{yuanwen}
其夏,骠骑将军与合骑侯敖俱出北地,异道;博望侯张骞、郎中令李广俱出右北平,异道:皆击匈奴。郎中令将四千骑先至,博望侯将万骑在后至。匈奴左贤王将数万骑围郎中令,郎中令与战二日,死者过半,所杀亦过当。博望侯至,匈奴兵引去。博望侯坐行留,当斩,赎为庶人。而骠骑将军出北地,已遂深入,与合骑侯失道,不相得,骠骑将军逾居延至祁连山\footnote{text},捕首虏甚多。

天子曰:“骠骑将军逾居延,遂过小月氏\footnote{text},攻祁连山,得酋涂王\footnote{text},以众降者二千五百人,斩首虏三万二百级,获五王,五王母,单于阏氏、王子五十九人\footnote{text},相国、将军、当户、都尉六十三人,师大率减什三,益封去病五千户\footnote{text}。

赐校尉从至小月氏爵左庶长。鹰击司马破奴再从骠骑将军斩濮王,捕稽沮王,千骑将得王、王母各一人,王子以下四十一人,捕虏三千三百三十人,前行捕虏千四百人,以千五百户封破奴为从骠侯。校尉句王高不识,从骠骑将军捕呼于屠王王子以下十一人,捕虏千七百六十八人,以千一百户封不识为宜冠侯。校尉仆多有功,封为煇渠侯。”合骑侯敖坐行留不与骠骑会,当斩,赎为庶人。

诸宿将所将士马兵亦不如骠骑,骠骑所将常选\footnote{text},然亦敢深入,常与壮骑先其大军\footnote{text},军亦有天幸,未尝困绝也。然而诸宿将常坐留落不遇\footnote{text}。由此骠骑日以亲贵,比大将军。
\end{yuanwen}

\begin{yuanwen}
其秋,单于怒浑邪王居西方数为汉所破,亡数万人,以骠骑之兵也。单于怒,欲召诛浑邪王。

浑邪王与休屠王等谋欲降汉,使人先要边\footnote{text}。是时大行李息将城河上\footnote{text},得浑邪王使,即驰传以闻\footnote{text}。

天子闻之,于是恐其以诈降而袭边,乃令骠骑将军将兵往迎之。骠骑既渡河,与浑邪王众相望。浑邪王裨将见汉军而多欲不降者,颇遁去。骠骑乃驰入与浑邪王相见\footnote{text},斩其欲亡者八千人,遂独遣浑邪王乘传先诣行在所,尽将其众渡河,降者数万,号称十万。
\end{yuanwen}

\begin{yuanwen}
既至长安,天子所以赏赐者数十巨万。封浑邪王万户,为漯阴侯。封其裨王呼毒尼为下摩侯,鹰庇为煇渠侯,禽犁为河綦侯,大当户铜离为常乐侯。于是天子嘉骠骑之功曰:“骠骑将军去病率师攻匈奴西域王浑邪,王及厥众萌咸相饹,率以军粮接食,并将控弦万有馀人,诛獟駻,获首虏八千馀级,降异国之王三十二人,战士不离伤,十万之众咸怀集服,仍与之劳,爰及河塞,庶几无患,幸既永绥矣。以千七百户益封骠骑将军。”减陇西、北地、上郡戍卒之半,以宽天下之繇。
\end{yuanwen}

\begin{yuanwen}
居顷之,乃分徙降者边五郡故塞外\footnote{text},而皆在河南,因其故俗,为属国\footnote{text}。
\end{yuanwen}

\begin{yuanwen}
其明年,匈奴入右北平、定襄,杀略汉千馀人。
\end{yuanwen}

\begin{yuanwen}
其明年\footnote{text},天子与诸将议曰:“翕侯赵信为单于画计,常以为汉兵不能度幕轻留\footnote{text},今大发士卒,其势必得所欲。”

是岁元狩四年也\footnote{text}。
\end{yuanwen}

\begin{yuanwen}
元狩四年春,上令大将军青、骠骑将军去病将各五万骑,步兵转者踵军数十万\footnote{text},而敢力战深入之士皆属骠骑。骠骑始为出定襄\footnote{text},当单于。捕虏言单于东,乃更令骠骑出代郡,令大将军出定襄。

郎中令为前将军,太仆为左将军,主爵赵食其为右将军,平阳侯襄为后将军,皆属大将军。兵即度幕,人马凡五万骑,与骠骑等咸击匈奴单于。

赵信为单于谋曰:“汉兵既度幕,人马罢\footnote{text},匈奴可坐收虏耳。”

乃悉远北其辎重,皆以精兵待幕北。而适值大将军军出塞千馀里,见单于兵陈而待。于是大将军令武刚车自环为营\footnote{text},而纵五千骑往当匈奴。匈奴亦纵可万骑。会日且入,大风起,沙砾击面,两军不相见。汉益纵左右翼绕单于。单于视汉兵多,而士马尚彊,战而匈奴不利。薄莫\footnote{text},单于遂乘六骡\footnote{text},壮骑可数百,直冒汉围西北驰去\footnote{text}。时已昏,汉匈奴相纷挐\footnote{text},杀伤大当。汉军左校捕虏言单于未昏而去,汉军因发轻骑夜追之,大将军军因随其后。匈奴兵亦散走。

迟明\footnote{text},行二百馀里,不得单于。颇捕斩首虏万馀级,遂至窴颜山赵信城\footnote{text},得匈奴积粟食军。军留一日而还,悉烧其城馀粟以归\footnote{text}。
\end{yuanwen}

\begin{yuanwen}
大将军之与单于会也,而前将军广、右将军食其军别从东道,或失道,后击单于。大将军引还过幕南,乃得前将军、右将军。大将军欲使使归报,令长史簿责前将军广,广自杀。右将军至,下吏,赎为庶人。大将军军入塞,凡斩捕首虏万九千级。
\end{yuanwen}

\begin{yuanwen}
是时匈奴众失单于十馀日,右谷蠡王闻之\footnote{text},自立为单于。单于后得其众,右王乃去单于之号。
\end{yuanwen}

\begin{yuanwen}
骠骑将军亦将五万骑,车重与大将军军等,而无裨将\footnote{text}。悉以李敢等为大校,当裨将,出代、右北平千馀里,直左方兵\footnote{text},所斩捕功已多大将军。

军既还,天子曰:“骠骑将军去病率师,躬将所获荤粥之士\footnote{text},约轻赍\footnote{text},绝大幕,涉获章渠\footnote{text},以诛比车耆\footnote{text};转击左大将,斩获旗鼓;历涉离侯\footnote{text}。济弓闾\footnote{text},获屯头王、韩王等三人\footnote{text},将军、相国、当户、都尉八十三人,封狼居胥山\footnote{text},禅于姑衍\footnote{text},登临翰海\footnote{text}。执卤获丑七万有四百四十三级\footnote{text},师率减什三,取食于敌,逴行殊远而粮不绝\footnote{text},以五千八百户益封骠骑将军。”
\end{yuanwen}

\begin{yuanwen}
右北平太守路博德属骠骑将军,会与城,不失期,从至檮余山,斩首捕虏二千七百级,以千六百户封博德为符离侯。北地都尉邢山从骠骑将军获王,以千二百户封山为义阳侯。故归义因淳王复陆支、楼专王伊即靬皆从骠骑将军有功,以千三百户封复陆支为壮侯,以千八百户封伊即靬为众利侯。从骠侯破奴、昌武侯安稽从骠骑有功,益封各三百户。校尉敢得旗鼓,为关内侯,食邑二百户。校尉自为爵大庶长。军吏卒为官,赏赐甚多。而大将军不得益封,军吏卒皆无封侯者。
\end{yuanwen}

\begin{yuanwen}
两军之出塞,塞阅官及私马凡十四万匹\footnote{text},而复入塞者不满三万匹。乃益置大司马位,大将军、骠骑将军皆为大司马\footnote{text}。定令,令骠骑将军秩禄与大将军等。自是之后,大将军青日退,而骠骑日益贵。
\end{yuanwen}

\begin{yuanwen}
举大将军故人门下多去事骠骑,辄得官爵,唯任安不肯。
\end{yuanwen}

\begin{yuanwen}
骠骑将军为人少言不泄,有气敢任。天子尝欲教之孙吴兵法,对曰:“顾方略何如耳,不至学古兵法。”

天子为治第,令骠骑视之,对曰:“匈奴未灭,无以家为也\footnote{text}。”

由此上益重爱之。然少而侍中,贵,不省士\footnote{text}。其从军,天子为遣太官赍数十乘\footnote{text},既还,重车馀弃粱肉\footnote{text},而士有饥者。其在塞外,卒乏粮,或不能自振,而骠骑尚穿域蹋鞠\footnote{text}。事多此类。大将军为人仁,善退让,以和柔自媚于上,然天下未有称也\footnote{text}。
\end{yuanwen}

\begin{yuanwen}
骠骑将军自四年军后三年\footnote{text},元狩六年而卒\footnote{text}。天子悼之,发属国玄甲军\footnote{text},陈自长安至茂陵\footnote{text},为冢象祁连山\footnote{text}。
\end{yuanwen}

\begin{yuanwen}
谥之,并武与广地曰景桓侯。子嬗代侯。嬗少,字子侯,上爱之,幸其壮而将之。居六岁,元封元年,嬗卒,谥哀侯。无子,绝,国除。

自骠骑将军死后,大将军长子宜春侯伉坐法失侯。后五岁,伉弟二人,阴安侯不疑及发干侯登皆坐酎金失侯。失侯后二岁,冠军侯国除。其后四年,大将军青卒,谥为烈侯。子伉代为长平侯。

自大将军围单于之后,十四年而卒。竟不复击匈奴者,以汉马少,而方南诛两越,东伐朝鲜,击羌、西南夷,以故久不伐胡。

大将军以其得尚平阳长公主故,长平侯伉代侯。六岁,坐法失侯。

左方两大将军及诸裨将名:

最大将军青,凡七出击匈奴,斩捕首虏五万馀级。一与单于战,收河南地,遂置朔方郡,再益封,凡万一千八百户。封三子为侯,侯千三百户。并之,万五千七百户。其校尉裨将以从大将军侯者九人。其裨将及校尉已为将者十四人。为裨将者曰李广,自有传。无传者曰:

将军公孙贺。贺,义渠人,其先胡种。贺父浑邪,景帝时为平曲侯,坐法失侯。贺,武帝为太子时舍人。武帝立八岁,以太仆为轻车将军,军马邑。后四岁,以轻车将军出云中。后五岁,以骑将军从大将军有功,封为南窌侯。后一岁,以左将军再从大将军出定襄,无功。后四岁,以坐酎金失侯。后八岁,以浮沮将军出五原二千馀里,无功。后八岁,以太仆为丞相,封葛绎侯。贺七为将军,出击匈奴无大功,而再侯,为丞相。坐子敬声与阳石公主奸,为巫蛊,族灭,无后。

将军李息,郁郅人。事景帝。至武帝立八岁,为材官将军,军马邑;后六岁,为将军,出代;后三岁,为将军,从大将军出朔方:皆无功。凡三为将军,其后常为大行。

将军公孙敖,义渠人。以郎事武帝。武帝立十二岁,为骑将军,出代,亡卒七千人,当斩,赎为庶人。后五岁,以校尉从大将军有功,封为合骑侯。后一岁,以中将军从大将军,再出定襄,无功。后二岁,以将军出北地,后骠骑期,当斩,赎为庶人。后二岁,以校尉从大将军,无功。后十四岁,以因杅将军筑受降城。七岁,复以因杅将军再出击匈奴,至余吾,亡士卒多,下吏,当斩,诈死,亡居民间五六岁。后发觉,复系。坐妻为巫蛊,族。凡四为将军,出击匈奴,一侯。

将军李沮,云中人。事景帝。武帝立十七岁,以左内史为彊弩将军。后一岁,复为彊弩将军。

将军李蔡,成纪人也。事孝文帝、景帝、武帝。以轻车将军从大将军有功,封为乐安侯。已为丞相,坐法死。

将军张次公,河东人。以校尉从卫将军青有功,封为岸头侯。其后太后崩,为将军,军北军。后一岁,为将军,从大将军,再为将军,坐法失侯。次公父隆,轻车武射也。以善射,景帝幸近之也。

将军苏建,杜陵人。以校尉从卫将军青,有功,为平陵侯,以将军筑朔方。后四岁,为游击将军,从大将军出朔方。后一岁,以右将军再从大将军出定襄,亡翕侯,失军,当斩,赎为庶人。其后为代郡太守,卒,冢在大犹乡。

将军赵信,以匈奴相国降,为翕侯。武帝立十七岁,为前将军,与单于战,败,降匈奴。

将军张骞,以使通大夏,还,为校尉。从大将军有功,封为博望侯。后三岁,为将军,出右北平,失期,当斩,赎为庶人。其后使通乌孙,为大行而卒,冢在汉中。

将军赵食其,祋祤人也。武帝立二十二岁,以主爵为右将军,从大将军出定襄,迷失道,当斩,赎为庶人。

将军曹襄,以平阳侯为后将军,从大将军出定襄。襄,曹参孙也。

将军韩说,弓高侯庶孙也。以校尉从大将军有功,为龙嵒侯,坐酎金失侯。元鼎六年,以待诏为横海将军,击东越有功,为按道侯。以太初三年为游击将军,屯于五原外列城。为光禄勋,掘蛊太子宫,卫太子杀之。

将军郭昌,云中人也。以校尉从大将军。元封四年,以太中大夫为拔胡将军,屯朔方。还击昆明,毋功,夺印。

将军荀彘,太原广武人。以御见,侍中,为校尉,数从大将军。以元封三年为左将军击朝鲜,毋功。以捕楼船将军坐法死。

最骠骑将军去病,凡六出击匈奴,其四出以将军,斩捕首虏十一万馀级。及浑邪王以众降数万,遂开河西酒泉之地,西方益少胡寇。四益封,凡万五千一百户。其校吏有功为侯者凡六人,而后为将军二人。

将军路博德,平州人。以右北平太守从骠骑将军有功,为符离侯。骠骑死后,博德以卫尉为伏波将军,伐破南越,益封。其后坐法失侯。为彊弩都尉,屯居延,卒。

将军赵破奴,故九原人。尝亡入匈奴,已而归汉,为骠骑将军司马。出北地时有功,封为从骠侯。坐酎金失侯。后一岁,为匈河将军,攻胡至匈河水,无功。后二岁,击虏楼兰王,复封为浞野侯。后六岁,为浚稽将军,将二万骑击匈奴左贤王,左贤王与战,兵八万骑围破奴,破奴生为虏所得,遂没其军。居匈奴中十岁,复与其太子安国亡入汉。后坐巫蛊,族。

自卫氏兴,大将军青首封,其后枝属为五侯。凡二十四岁而五侯尽夺,卫氏无为侯者。
\end{yuanwen}

\begin{yuanwen}
太史公曰:

苏建语余曰:“吾尝责‘大将军至尊重,而天下之贤大夫毋称焉,原将军观古名将所招选择贤者\footnote{text},勉之哉!’。大将军谢曰:‘自魏其、武安之厚宾客\footnote{text},天子常切齿\footnote{text}。彼亲附士大夫,招贤绌不肖者,人主之柄也。人臣奉法遵职而已,何与招士!’”

骠骑亦放此意\footnote{text},其为将如此。
\end{yuanwen}

\begin{yuanwen}
君子豹变,贵贱何常。青本奴虏,忽升戎行。姊配皇极,身尚平阳。宠荣斯僭,取乱彝章。嫖姚继踵,再静边方。
\end{yuanwen}

\chapter{平津侯主父列传}

\begin{yuanwen}
丞相公孙弘者,齐菑川国薛县人也,字季。少时为薛狱吏,有罪,免。家贫,牧豕海上。年四十馀,乃学春秋杂说。养后母孝谨。

建元元年,天子初即位,招贤良文学之士。是时弘年六十,徵以贤良为博士。使匈奴,还报,不合上意,上怒,以为不能,弘乃病免归。

元光五年,有诏徵文学,菑川国复推上公孙弘。弘让谢国人曰:“臣已尝西应命,以不能罢归,原更推选。”国人固推弘,弘至太常。太常令所徵儒士各对策,百馀人,弘第居下。策奏,天子擢弘对为第一。召入见,状貌甚丽,拜为博士。是时通西南夷道,置郡,巴蜀民苦之,诏使弘视之。还奏事,盛毁西南夷无所用,上不听。

弘为人恢奇多闻,常称以为人主病不广大,人臣病不俭节。弘为布被,食不重肉。后母死,服丧三年。每朝会议,开陈其端,令人主自择,不肯面折庭争。于是天子察其行敦厚,辩论有馀,习文法吏事,而又缘饰以儒术,上大说之。二岁中,至左内史。弘奏事,有不可,不庭辩之。尝与主爵都尉汲黯请间,汲黯先发之,弘推其后,天子常说,所言皆听,以此日益亲贵。尝与公卿约议,至上前,皆倍其约以顺上旨。汲黯庭诘弘曰:“齐人多诈而无情实,始与臣等建此议,今皆倍之,不忠。”上问弘。弘谢曰:“夫知臣者以臣为忠,不知臣者以臣为不忠。”上然弘言。左右幸臣每毁弘,上益厚遇之。

元朔三年,张欧免,以弘为御史大夫。是时通西南夷,东置沧海,北筑朔方之郡。弘数谏,以为罢敝中国以奉无用之地,原罢之。于是天子乃使硃买臣等难弘置朔方之便。发十策,弘不得一。弘乃谢曰:“山东鄙人,不知其便若是,原罢西南夷、沧海而专奉朔方。”上乃许之。

汲黯曰:“弘位在三公,奉禄甚多。然为布被,此诈也。”上问弘。弘谢曰:“有之。夫九卿与臣善者无过黯,然今日庭诘弘,诚中弘之病。夫以三公为布被,诚饰诈欲以钓名。且臣闻管仲相齐,有三归,侈拟于君,桓公以霸,亦上僭于君。晏婴相景公,食不重肉,妾不衣丝,齐国亦治,此下比于民。今臣弘位为御史大夫,而为布被,自九卿以下至于小吏,无差,诚如汲黯言。且无汲黯忠,陛下安得闻此言。”天子以为谦让,愈益厚之。卒以弘为丞相,封平津侯。

弘为人意忌,外宽内深。诸尝与弘有卻者,虽详与善,阴报其祸。杀主父偃,徙董仲舒于胶西,皆弘之力也。食一肉脱粟之饭。故人所善宾客,仰衣食,弘奉禄皆以给之,家无所馀。士亦以此贤之。

淮南、衡山谋反,治党与方急。弘病甚,自以为无功而封,位至丞相,宜佐明主填抚国家,使人由臣子之道。今诸侯有畔逆之计,此皆宰相奉职不称,恐窃病死,无以塞责。乃上书曰:“臣闻天下之通道五,所以行之者三。曰君臣,父子,兄弟,夫妇,长幼之序,此五者天下之通道也。智,仁,勇,此三者天下之通德,所以行之者也。故曰‘力行近乎仁,好问近乎智,知耻近乎勇’ 。知此三者,则知所以自治;知所以自治,然后知所以治人。天下未有不能自治而能治人者也,此百世不易之道也。今陛下躬行大孝,鉴三王,建周道,兼文武,厉贤予禄,量能授官。今臣弘罢驽之质,无汗马之劳,陛下过意擢臣弘卒伍之中,封为列侯,致位三公。臣弘行能不足以称,素有负薪之病,恐先狗马填沟壑,终无以报德塞责。原归侯印,乞骸骨,避贤者路。”天子报曰:“古者赏有功,褎有德,守成尚文,遭遇右武,未有易此者也。朕宿昔庶几获承尊位,惧不能宁,惟所与共为治者,君宜知之。盖君子善善恶恶,君若谨行,常在朕躬。君不幸罹霜露之病,何恙不已,乃上书归侯,乞骸骨,是章朕之不德也。今事少间,君其省思虑,一精神,辅以医药。”因赐告牛酒杂帛。居数月,病有瘳,视事。

元狩二年,弘病,竟以丞相终。子度嗣为平津侯。度为山阳太守十馀岁,坐法失侯。

主父偃者,齐临菑人也。学长短纵横之术,晚乃学易、春秋、百家言。游齐诸生间,莫能厚遇也。齐诸儒生相与排摈,不容于齐。家贫,假贷无所得,乃北游燕、赵、中山,皆莫能厚遇,为客甚困。孝武元光元年中,以为诸侯莫足游者,乃西入关见卫将军。卫将军数言上,上不召。资用乏,留久,诸公宾客多厌之,乃上书阙下。朝奏,暮召入见。所言九事,其八事为律令,一事谏伐匈奴。其辞曰:

臣闻明主不恶切谏以博观,忠臣不敢避重诛以直谏,是故事无遗策而功流万世。今臣不敢隐忠避死以效愚计,原陛下幸赦而少察之。

司马法曰:“国虽大,好战必亡;天下虽平,忘战必危。”天下既平,天子大凯,春蒐秋狝,诸侯春振旅,秋治兵,所以不忘战也。且夫怒者逆德也,兵者凶器也,争者末节也。古之人君一怒必伏尸流血,故圣王重行之。夫务战胜穷武事者,未有不悔者也。昔秦皇帝任战胜之威,蚕食天下,并吞战国,海内为一,功齐三代。务胜不休,欲攻匈奴,李斯谏曰:“不可。夫匈奴无城郭之居,委积之守,迁徙鸟举,难得而制也。轻兵深入,粮食必绝;踵粮以行,重不及事。得其地不足以为利也,遇其民不可役而守也。胜必杀之,非民父母也。靡弊中国,快心匈奴,非长策也。”秦皇帝不听,遂使蒙恬将兵攻胡,辟地千里,以河为境。地固泽卤,不生五穀。然后发天下丁男以守北河。暴兵露师十有馀年,死者不可胜数,终不能逾河而北。是岂人众不足,兵革不备哉?其势不可也。又使天下蜚刍輓粟,起于黄、腄、琅邪负海之郡,转输北河,率三十锺而致一石。男子疾耕不足于粮饟,女子纺绩不足于帷幕。百姓靡敝,孤寡老弱不能相养,道路死者相望,盖天下始畔秦也。

及至高皇帝定天下,略地于边,闻匈奴聚于代谷之外而欲击之。御史成进谏曰:“不可。夫匈奴之性,兽聚而鸟散,从之如搏影。今以陛下盛德攻匈奴,臣窃危之。”高帝不听,遂北至于代谷,果有平城之围。高皇帝盖悔之甚,乃使刘敬往结和亲之约,然后天下忘干戈之事。故兵法曰“兴师十万,日费千金”。夫秦常积众暴兵数十万人,虽有覆军杀将系虏单于之功,亦適足以结怨深雠,不足以偿天下之费。夫上虚府库,下敝百姓,甘心于外国,非完事也。夫匈奴难得而制,非一世也。行盗侵驱,所以为业也,天性固然。上及虞夏殷周,固弗程督,禽兽畜之,不属为人。夫上不观虞夏殷周之统,而下近世之失,此臣之所大忧,百姓之所疾苦也。且夫兵久则变生,事苦则虑易。乃使边境之民弊靡愁苦而有离心,将吏相疑而外市,故尉佗、章邯得以成其私也。夫秦政之所以不行者,权分乎二子,此得失之效也。故周书曰“安危在出令,存亡在所用”。原陛下详察之,少加意而熟虑焉。

是时赵人徐乐、齐人严安俱上书言世务,各一事。徐乐曰:臣闻天下之患在于土崩,不在于瓦解,古今一也。何谓土崩?秦之末世是也。陈涉无千乘之尊,尺土之地,身非王公大人名族之后,无乡曲之誉,非有孔、墨、曾子之贤,陶硃、猗顿之富也,然起穷巷,奋棘矜,偏袒大呼而天下从风,此其故何也?由民困而主不恤,下怨而上不知,俗已乱而政不脩,此三者陈涉之所以为资也。是之谓土崩。故曰天下之患在于土崩。何谓瓦解?吴、楚、齐、赵之兵是也。七国谋为大逆,号皆称万乘之君,带甲数十万,威足以严其境内,财足以劝其士民,然不能西攘尺寸之地而身为禽于中原者,此其故何也?非权轻于匹夫而兵弱于陈涉也,当是之时,先帝之德泽未衰而安土乐俗之民众,故诸侯无境外之助。此之谓瓦解,故曰天下之患不在瓦解。由是观之,天下诚有土崩之势,虽布衣穷处之士或首恶而危海内,陈涉是也。况三晋之君或存乎!天下虽未有大治也,诚能无土崩之势,虽有彊国劲兵不得旋踵而身为禽矣,吴、楚、齐、赵是也。况群臣百姓能为乱乎哉!此二体者,安危之明要也,贤主所留意而深察也。

间者关东五穀不登,年岁未复,民多穷困,重之以边境之事,推数循理而观之,则民且有不安其处者矣。不安故易动。易动者,土崩之势也。故贤主独观万化之原,明于安危之机,脩之庙堂之上,而销未形之患。其要,期使天下无土崩之势而已矣。故虽有彊国劲兵,陛下逐走兽,射蜚鸟,弘游燕之囿,淫纵恣之观,极驰骋之乐,自若也。金石丝竹之声不绝于耳,帷帐之私俳优侏儒之笑不乏于前,而天下无宿忧。名何必汤武,俗何必成康!虽然,臣窃以为陛下天然之圣,宽仁之资,而诚以天下为务,则汤武之名不难侔,而成康之俗可复兴也。此二体者立,然后处尊安之实,扬名广誉于当世,亲天下而服四夷,馀恩遗德为数世隆,南面负扆摄袂而揖王公,此陛下之所服也。臣闻图王不成,其敝足以安。安则陛下何求而不得,何为而不成,何征而不服乎哉!严安上书曰:

臣闻周有天下,其治三百馀岁,成康其隆也,刑错四十馀年而不用。及其衰也,亦三百馀岁,故五伯更起。五伯者,常佐天子兴利除害,诛暴禁邪,匡正海内,以尊天子。五伯既没,贤圣莫续,天子孤弱,号令不行。诸侯恣行,彊陵弱,众暴寡,田常篡齐,六卿分晋,并为战国,此民之始苦也。于是彊国务攻,弱国备守,合从连横,驰车击毂,介胄生虮虱,民无所告愬。

及至秦王,蚕食天下,并吞战国,称号曰皇帝,主海内之政,坏诸侯之城,销其兵,铸以为锺虡,示不复用。元元黎民得免于战国,逢明天子,人人自以为更生。乡使秦缓其刑罚,薄赋敛,省繇役,贵仁义,贱权利,上笃厚,下智巧,变风易俗,化于海内,则世世必安矣。秦不行是风而其故俗,为智巧权利者进,笃厚忠信者退;法严政峻,谄谀者众,日闻其美,意广心轶。欲肆威海外,乃使蒙恬将兵以北攻胡,辟地进境,戍于北河,蜚刍輓粟以随其后。又使尉屠睢将楼船之士南攻百越,使监禄凿渠运粮,深入越,越人遁逃。旷日持久,粮食绝乏,越人击之,秦兵大败。秦乃使尉佗将卒以戍越。当是时,秦祸北构于胡,南挂于越,宿兵无用之地,进而不得退。行十馀年,丁男被甲,丁女转输,苦不聊生,自经于道树,死者相望。及秦皇帝崩,天下大叛。陈胜、吴广举陈,武臣、张耳举赵,项梁举吴,田儋举齐,景驹举郢,周市举魏,韩广举燕,穷山通谷豪士并起,不可胜载也。然皆非公侯之后,非长官之吏也。无尺寸之势,起闾巷,杖棘矜,应时而皆动,不谋而俱起,不约而同会,壤长地进,至于霸王,时教使然也。秦贵为天子,富有天下,灭世绝祀者,穷兵之祸也。故周失之弱,秦失之彊,不变之患也。

今欲招南夷,朝夜郎,降羌僰,略濊州,建城邑,深入匈奴,燔其茏城,议者美之。此人臣之利也,非天下之长策也。今中国无狗吠之惊,而外累于远方之备,靡敝国家,非所以子民也。行无穷之欲,甘心快意,结怨于匈奴,非所以安边也。祸结而不解,兵休而复起,近者愁苦,远者惊骇,非所以持久也。今天下锻甲砥剑,桥箭累弦,转输运粮,未见休时,此天下之所共忧也。夫兵久而变起,事烦而虑生。今外郡之地或几千里,列城数十,形束壤制,旁胁诸侯,非公室之利也。上观齐晋之所以亡者,公室卑削,六卿大盛也;下观秦之所以灭者,严法刻深,欲大无穷也。今郡守之权,非特六卿之重也;地几千里,非特闾巷之资也;甲兵器械,非特棘矜之用也:以遭万世之变,则不可称讳也。

书奏天子,天子召见三人,谓曰:“公等皆安在?何相见之晚也!”于是上乃拜主父偃、徐乐、严安为郎中。数见,上疏言事,诏拜偃为谒者,迁为中大夫。一岁中四迁偃。

偃说上曰:“古者诸侯不过百里,彊弱之形易制。今诸侯或连城数十,地方千里,缓则骄奢易为淫乱,急则阻其彊而合从以逆京师。今以法割削之,则逆节萌起,前日晁错是也。今诸侯子弟或十数,而適嗣代立,馀虽骨肉,无尺寸地封,则仁孝之道不宣。原陛下令诸侯得推恩分子弟,以地侯之。彼人人喜得所原,上以德施,实分其国,不削而稍弱矣。”于是上从其计。又说上曰:“茂陵初立,天下豪桀并兼之家,乱众之民,皆可徙茂陵,内实京师,外销奸猾,此所谓不诛而害除。”上又从其计。

尊立卫皇后,及发燕王定国阴事,盖偃有功焉。大臣皆畏其口,赂遗累千金。人或说偃曰:“太横矣。”主父曰:“臣结发游学四十馀年,身不得遂,亲不以为子,昆弟不收,宾客弃我,我戹日久矣。且丈夫生不五鼎食,死即五鼎烹耳。吾日暮途远,故倒行暴施之。”

偃盛言朔方地肥饶,外阻河,蒙恬城之以逐匈奴,内省转输戍漕,广中国,灭胡之本也。上览其说,下公卿议,皆言不便。公孙弘曰:“秦时常发三十万众筑北河,终不可就,已而弃之。”主父偃盛言其便,上竟用主父计,立朔方郡。

元朔二年,主父言齐王内淫佚行僻,上拜主父为齐相。至齐,遍召昆弟宾客,散五百金予之,数之曰:“始吾贫时,昆弟不我衣食,宾客不我内门;今吾相齐,诸君迎我或千里。吾与诸君绝矣,毋复入偃之门!”乃使人以王与姊奸事动王,王以为终不得脱罪,恐效燕王论死,乃自杀。有司以闻。

主父始为布衣时,尝游燕、赵,及其贵,发燕事。赵王恐其为国患,欲上书言其阴事,为偃居中,不敢发。及为齐相,出关,即使人上书,告言主父偃受诸侯金,以故诸侯子弟多以得封者。及齐王自杀,上闻大怒,以为主父劫其王令自杀,乃徵下吏治。主父服受诸侯金,实不劫王令自杀。上欲勿诛,是时公孙弘为御史大夫,乃言曰:“齐王自杀无后,国除为郡,入汉,主父偃本首恶,陛下不诛主父偃,无以谢天下。”乃遂族主父偃。

主父方贵幸时,宾客以千数,及其族死,无一人收者,唯独洨孔车收葬之。天子后闻之,以为孔车长者也。

太史公曰:公孙弘行义虽脩,然亦遇时。汉兴八十馀年矣,上方乡文学,招俊乂,以广儒墨,弘为举首。主父偃当路,诸公皆誉之,及名败身诛,士争言其恶。悲夫!

太皇太后诏大司徒大司空:“盖闻治国之道,富民为始;富民之要,在于节俭。孝经曰‘安上治民,莫善于礼’ 。‘礼,与奢也宁俭’ 。昔者管仲相齐桓,霸诸侯,有九合一匡之功,而仲尼谓之不知礼,以其奢泰侈拟于君故也。夏禹卑宫室,恶衣服,后圣不循。由此言之,治之盛也,德优矣,莫高于俭。俭化俗民,则尊卑之序得,而骨肉之恩亲,争讼之原息。斯乃家给人足,刑错之本也欤?可不务哉!夫三公者,百寮之率,万民之表也。未有树直表而得曲影者也。孔子不云乎,‘子率而正,孰敢不正’ 。‘举善而教不能则劝’ 。维汉兴以来,股肱宰臣身行俭约,轻财重义,较然著明,未有若故丞相平津侯公孙弘者也。位在丞相而为布被,脱粟之饭,不过一肉。故人所善宾客皆分奉禄以给之,无有所馀。诚内自克约而外从制。汲黯诘之,乃闻于朝,此可谓减于制度而可施行者也。德优则行,否则止,与内奢泰而外为诡服以钓虚誉者殊科。以病乞骸骨,孝武皇帝即制曰‘赏有功,襃有德,善善恶恶,君宜知之。其省思虑,存精神,辅以医药’ 。赐告治病,牛酒杂帛。居数月,有瘳,视事。至元狩二年,竟以善终于相位。夫知臣莫若君,此其效也。弘子度嗣爵,后为山阳太守,坐法失侯。夫表德章义,所以率俗厉化,圣王之制,不易之道也。其赐弘后子孙之次当为后者爵关内侯,食邑三百户,徵诣公车,上名尚书,朕亲临拜焉。”

班固称曰:公孙弘、卜式、兒宽皆以鸿渐之翼困于燕雀,远迹羊豕之间,非遇其时,焉能致此位乎?是时汉兴六十馀载,海内乂安,府库充实,而四夷未宾,制度多阙,上方欲用文武,求之如弗及。始以蒲轮迎枚生,见主父而叹息。群臣慕乡,异人并出。卜式试于刍牧,弘羊擢于贾竖,卫青奋于奴仆,日磾出于降虏,斯亦曩时版筑饭牛之朋矣。汉之得人,于兹为盛。儒雅则公孙弘、董仲舒、兒宽,笃行则石建、石庆,质直则汲黯、卜式,推贤则韩安国、郑当时,定令则赵禹、张汤,文章则司马迁、相如,滑稽则东方朔、枚皋,应对则严助、硃买臣,历数则唐都、落下闳,协律则李延年,运筹则桑弘羊,奉使则张骞、苏武,将帅则卫青、霍去病,受遗则霍光、金日磾。其馀不可胜纪。是以兴造功业,制度遗文,后世莫及。孝宣承统,纂脩洪业,亦讲论六,招选茂异,而萧望之、梁丘贺、夏侯胜、韦玄成、严彭祖、尹更始以儒术进,刘向、王襃以文章显。将相则张安世、赵充国、魏相、邴吉、于定国、杜延年,治民则黄霸、王成、龚遂、郑弘、邵信臣、韩延寿、尹翁归、赵广汉之属,皆有功迹见述于后。累其名臣,亦其次也。

平津巨儒,晚年始遇。外示宽俭,内怀嫉妒。宠备荣爵,身受肺腑。主父推恩,观时设度。生食五鼎,死非时蠹。
\end{yuanwen}

\chapter{南越列传}

\begin{yuanwen}
南越王尉佗者,真定人也,姓赵氏。秦时已并天下,略定杨越,置桂林、南海、象郡,以谪徙民,与越杂处十三岁。佗,秦时用为南海龙川令。至二世时,南海尉任嚣病且死,召龙川令赵佗语曰:“闻陈胜等作乱,秦为无道,天下苦之,项羽、刘季、陈胜、吴广等州郡各共兴军聚众,虎争天下,中国扰乱,未知所安,豪杰畔秦相立。南海僻远,吾恐盗兵侵地至此,吾欲兴兵绝新道,自备,待诸侯变,会病甚。且番禺负山险,阻南海,东西数千里,颇有中国人相辅,此亦一州之主也,可以立国。郡中长吏无足与言者,故召公告之。”即被佗书,行南海尉事。嚣死,佗即移檄告横浦、阳山、湟谿关曰:“盗兵且至,急绝道聚兵自守!”因稍以法诛秦所置长吏,以其党为假守。秦已破灭,佗即击并桂林、象郡,自立为南越武王。高帝已定天下,为中国劳苦,故释佗弗诛。汉十一年,遣陆贾因立佗为南越王,与剖符通使,和集百越,毋为南边患害,与长沙接境。

高后时,有司请禁南越关市铁器。佗曰:“高帝立我,通使物,今高后听谗臣,别异蛮夷,隔绝器物,此必长沙王计也,欲倚中国,击灭南越而并王之,自为功也。”于是佗乃自尊号为南越武帝,发兵攻长沙边邑,败数县而去焉。高后遣将军隆虑侯灶往击之。会暑湿,士卒大疫,兵不能逾岭。岁馀,高后崩,即罢兵。佗因此以兵威边,财物赂遗闽越、西瓯、骆,役属焉,东西万馀里。乃乘黄屋左纛,称制,与中国侔。

及孝文帝元年,初镇抚天下,使告诸侯四夷从代来即位意,喻盛德焉。乃为佗亲冢在真定,置守邑,岁时奉祀。召其从昆弟,尊官厚赐宠之。诏丞相陈平等举可使南越者,平言好畤陆贾,先帝时习使南越。乃召贾以为太中大夫,往使。因让佗自立为帝,曾无一介之使报者。陆贾至南越,王甚恐,为书谢,称曰:“蛮夷大长老夫臣佗,前日高后隔异南越,窃疑长沙王谗臣,又遥闻高后尽诛佗宗族,掘烧先人冢,以故自弃,犯长沙边境。且南方卑湿,蛮夷中间,其东闽越千人众号称王,其西瓯骆裸国亦称王。老臣妄窃帝号,聊以自娱,岂敢以闻天王哉!”乃顿首谢,原长为籓臣,奉贡职。于是乃下令国中曰:“吾闻两雄不俱立,两贤不并世。皇帝,贤天子也。自今以后,去帝制黄屋左纛。”陆贾还报,孝文帝大说。遂至孝景时,称臣,使人朝请。然南越其居国窃如故号名,其使天子,称王朝命如诸侯。至建元四年卒。

佗孙胡为南越王。此时闽越王郢兴兵击南越边邑,胡使人上书曰:“两越俱为籓臣,毋得擅兴兵相攻击。今闽越兴兵侵臣,臣不敢兴兵,唯天子诏之。”于是天子多南越义,守职约,为兴师,遣两将军往讨闽越。兵未逾岭,闽越王弟馀善杀郢以降,于是罢兵。

天子使庄助往谕意南越王,胡顿首曰:“天子乃为臣兴兵讨闽越,死无以报德!”遣太子婴齐入宿卫。谓助曰:“国新被寇,使者行矣。胡方日夜装入见天子。”助去后,其大臣谏胡曰:“汉兴兵诛郢,亦行以惊动南越。且先王昔言,事天子期无失礼,要之不可以说好语入见。入见则不得复归,亡国之势也。”于是胡称病,竟不入见。后十馀岁,胡实病甚,太子婴齐请归。胡薨,谥为文王。

婴齐代立,即藏其先武帝玺。婴齐其入宿卫在长安时,取邯郸樛氏女,生子兴。及即位,上书请立樛氏女为后,兴为嗣。汉数使使者风谕婴齐,婴齐尚乐擅杀生自恣,惧入见要用汉法,比内诸侯,固称病,遂不入见。遣子次公入宿卫。婴齐薨,谥为明王。

太子兴代立,其母为太后。太后自未为婴齐姬时,尝与霸陵人安国少季通。及婴齐薨后,元鼎四年,汉使安国少季往谕王、王太后以入朝,比内诸侯;令辩士谏大夫终军等宣其辞,勇士魏臣等辅其缺,卫尉路博德将兵屯桂阳,待使者。王年少,太后中国人也,尝与安国少季通,其使复私焉。国人颇知之,多不附太后。太后恐乱起,亦欲倚汉威,数劝王及群臣求内属。即因使者上书,请比内诸侯,三岁一朝,除边关。于是天子许之,赐其丞相吕嘉银印,及内史、中尉、太傅印,馀得自置。除其故黥劓刑,用汉法,比内诸侯。使者皆留填抚之。王、王太后饬治行装重赍,为入朝具。

其相吕嘉年长矣,相三王,宗族官仕为长吏者七十馀人,男尽尚王女,女尽嫁王子兄弟宗室,及苍梧秦王有连。其居国中甚重,越人信之,多为耳目者,得众心愈于王。王之上书,数谏止王,王弗听。有畔心,数称病不见汉使者。使者皆注意嘉,势未能诛。王、王太后亦恐嘉等先事发,乃置酒,介汉使者权,谋诛嘉等。使者皆东乡,太后南乡,王北乡,相嘉、大臣皆西乡,侍坐饮。嘉弟为将,将卒居宫外。酒行,太后谓嘉曰:“南越内属,国之利也,而相君苦不便者,何也?”以激怒使者。使者狐疑相杖,遂莫敢发。嘉见耳目非是,即起而出。太后怒,欲鏦嘉以矛,王止太后。嘉遂出,分其弟兵就舍,称病,不肯见王及使者。乃阴与大臣作乱。王素无意诛嘉,嘉知之,以故数月不发。太后有淫行,国人不附,欲独诛嘉等,力又不能。

天子闻嘉不听王,王、王太后弱孤不能制,使者怯无决。又以为王、王太后已附汉,独吕嘉为乱,不足以兴兵,欲使庄参以二千人往使。参曰:“以好往,数人足矣;以武往,二千人无足以为也。”辞不可,天子罢参也。郏壮士故济北相韩千秋奋曰:“以区区之越,又有王、太后应,独相吕嘉为害,原得勇士二百人,必斩嘉以报。”于是天子遣千秋与王太后弟樛乐将二千人往,入越境。吕嘉等乃遂反,下令国中曰:“王年少。太后,中国人也,又与使者乱,专欲内属,尽持先王宝器入献天子以自媚,多从人,行至长安,虏卖以为僮仆。取自脱一时之利,无顾赵氏社稷,为万世虑计之意。”乃与其弟将卒攻杀王、太后及汉使者。遣人告苍梧秦王及其诸郡县,立明王长男越妻子术阳侯建德为王。而韩千秋兵入,破数小邑。其后越直开道给食,未至番禺四十里,越以兵击千秋等,遂灭之。使人函封汉使者节置塞上,好为谩辞谢罪,发兵守要害处。于是天子曰:“韩千秋虽无成功,亦军锋之冠。”封其子延年为成安侯。樛乐,其姊为王太后,首原属汉,封其子广德为龙亢侯。乃下赦曰:“天子微,诸侯力政,讥臣不讨贼。今吕嘉、建德等反,自立晏如,令罪人及江淮以南楼船十万师往讨之。”

元鼎五年秋,卫尉路博德为伏波将军,出桂阳,下汇水;主爵都尉杨仆为楼船将军,出豫章,下横浦;故归义越侯二人为戈船、下厉将军,出零陵,或下离水,或柢苍梧;使驰义侯因巴蜀罪人,发夜郎兵,下牂柯江:咸会番禺。

元鼎六年冬,楼船将军将精卒先陷寻陕,破石门,得越船粟,因推而前,挫越锋,以数万人待伏波。伏波将军将罪人,道远,会期后,与楼船会乃有千馀人,遂俱进。楼船居前,至番禺。建德、嘉皆城守。楼船自择便处,居东南面;伏波居西北面。会暮,楼船攻败越人,纵火烧城。越素闻伏波名,日暮,不知其兵多少。伏波乃为营,遣使者招降者,赐印,复纵令相招。楼船力攻烧敌,反驱而入伏波营中。犁旦,城中皆降伏波。吕嘉、建德已夜与其属数百人亡入海,以船西去。伏波又因问所得降者贵人,以知吕嘉所之,遣人追之。以其故校尉司马苏弘得建德,封为海常侯;越郎都稽得嘉,封为临蔡侯。

苍梧王赵光者,越王同姓,闻汉兵至,及越揭阳令定自定属汉;越桂林监居翁谕瓯骆属汉:皆得为侯。戈船、下厉将军兵及驰义侯所发夜郎兵未下,南越已平矣。遂为九郡。伏波将军益封。楼船将军兵以陷坚为将梁侯。

自尉佗初王后,五世九十三岁而国亡焉。

太史公曰:尉佗之王,本由任嚣。遭汉初定,列为诸侯。隆虑离湿疫,佗得以益骄。瓯骆相攻,南越动摇。汉兵临境,婴齐入朝。其后亡国,徵自樛女;吕嘉小忠,令佗无后。楼船从欲,怠傲失惑;伏波困穷,智虑愈殖,因祸为福。成败之转,譬若纠墨。

中原鹿走,群雄莫制。汉事西驱,越权南裔。陆贾骋说,尉他去帝。嫪后内朝,吕嘉狼戾。君臣不协,卒从剿弃。
\end{yuanwen}

\chapter{东越列传}

\begin{yuanwen}
闽越王无诸及越东海王摇者,其先皆越王句践之后也,姓驺氏。秦已并天下,皆废为君长,以其地为闽中郡。及诸侯畔秦,无诸、摇率越归鄱阳令吴芮,所谓鄱君者也,从诸侯灭秦。当是之时,项籍主命,弗王,以故不附楚。汉击项籍,无诸、摇率越人佐汉。汉五年,复立无诸为闽越王,王闽中故地,都东冶。孝惠三年,举高帝时越功,曰闽君摇功多,其民便附,乃立摇为东海王,都东瓯,世俗号为东瓯王。

后数世,至孝景三年,吴王濞反,欲从闽越,闽越未肯行,独东瓯从吴。及吴破,东瓯受汉购,杀吴王丹徒,以故皆得不诛,归国。

吴王子子驹亡走闽越,怨东瓯杀其父,常劝闽越击东瓯。至建元三年,闽越发兵围东瓯。东瓯食尽,困,且降,乃使人告急天子。天子问太尉田蚡,蚡对曰:“越人相攻击,固其常,又数反覆,不足以烦中国往救也。自秦时弃弗属。”于是中大夫庄助诘蚡曰:“特患力弗能救,德弗能覆;诚能,何故弃之?且秦举咸阳而弃之,何乃越也!今小国以穷困来告急天子,天子弗振,彼当安所告愬?又何以子万国乎?”上曰:“太尉未足与计。吾初即位,不欲出虎符发兵郡国。”乃遣庄助以节发兵会稽。会稽太守欲距不为发兵,助乃斩一司马,谕意指,遂发兵浮海救东瓯。未至,闽越引兵而去。东瓯请举国徙中国,乃悉举众来,处江淮之间。

至建元六年,闽越击南越。南越守天子约,不敢擅发兵击而以闻。上遣大行王恢出豫章,大农韩安国出会稽,皆为将军。兵未逾岭,闽越王郢发兵距险。其弟馀善乃与相、宗族谋曰:“王以擅发兵击南越,不请,故天子兵来诛。今汉兵众彊,今即幸胜之,后来益多,终灭国而止。今杀王以谢天子。天子听,罢兵,固一国完;不听,乃力战;不胜,即亡入海。”皆曰“善”。即鏦杀王,使使奉其头致大行。大行曰:“所为来者诛王。今王头至,谢罪,不战而耘,利莫大焉。”乃以便宜案兵告大农军,而使使奉王头驰报天子。诏罢两将兵,曰:“郢等首恶,独无诸孙繇君丑不与谋焉。”乃使郎中将立丑为越繇王,奉闽越先祭祀。

馀善已杀郢,威行于国,国民多属,窃自立为王。繇王不能矫其众持正。天子闻之,为馀善不足复兴师,曰:“馀善数与郢谋乱,而后首诛郢,师得不劳。”因立馀善为东越王,与繇王并处。

至元鼎五年,南越反,东越王馀善上书,请以卒八千人从楼船将军击吕嘉等。兵至揭扬,以海风波为解,不行,持两端,阴使南越。及汉破番禺,不至。是时楼船将军杨仆使使上书,原便引兵击东越。上曰士卒劳倦,不许,罢兵,令诸校屯豫章梅领待命。

元鼎六年秋,馀善闻楼船请诛之,汉兵临境,且往,乃遂反,发兵距汉道。号将军驺力等为“吞汉将军”,入白沙、武林、梅岭,杀汉三校尉。是时汉使大农张成、故山州侯齿将屯,弗敢击,卻就便处,皆坐畏懦诛。

馀善刻“武帝”玺自立,诈其民,为妄言。天子遣横海将军韩说出句章,浮海从东方往;楼船将军杨仆出武林;中尉王温舒出梅岭;越侯为戈船、下濑将军,出若邪、白沙。元封元年冬,咸入东越。东越素发兵距险,使徇北将军守武林,败楼船军数校尉,杀长吏。楼船将军率钱唐辕终古斩徇北将军,为御兒侯。自兵未往。

故越衍侯吴阳前在汉,汉使归谕馀善,馀善弗听。及横海将军先至,越衍侯吴阳以其邑七百人反,攻越军于汉阳。从建成侯敖,与其率,从繇王居股谋曰:“馀善首恶,劫守吾属。今汉兵至,众彊,计杀馀善,自归诸将,傥幸得脱。”乃遂俱杀馀善,以其众降横海将军,故封繇王居股为东成侯,万户;封建成侯敖为开陵侯;封越衍侯吴阳为北石侯;封横海将军说为案道侯;封横海校尉福为缭嫈侯。福者,成阳共王子,故为海常侯,坐法失侯。旧从军无功,以宗室故侯。诸将皆无成功,莫封。东越将多军,汉兵至,弃其军降,封为无锡侯。

于是天子曰东越狭多阻,闽越悍,数反覆,诏军吏皆将其民徙处江淮间。东越地遂虚。

太史公曰:越虽蛮夷,其先岂尝有大功德于民哉,何其久也!历数代常为君王,句践一称伯。然馀善至大逆,灭国迁众,其先苗裔繇王居股等犹尚封为万户侯,由此知越世世为公侯矣。盖禹之馀烈也。

句践之裔,是曰无诸。既席汉宠,实因秦馀。驺、骆为姓,闽中是居。王摇之立,爰处东隅。后嗣不道,自相诛锄。
\end{yuanwen}

\part{卷一百一十五}

\chapter{朝鲜列传第五十五}

\begin{yuanwen}
朝鲜王满者,故燕人也。自始全燕时尝略属真番、朝鲜,为置吏,筑鄣塞。秦灭燕,属辽东外徼\footnote{边界。}。汉兴,为其远难守,复修辽东故塞,至浿水为界,属燕。燕王卢绾反,入匈奴,满亡命,聚党千馀人,魋结\footnote{把头发结成椎形。}蛮夷服而东走出塞,渡浿水,居秦故空地上下鄣,稍役属真番、朝鲜蛮夷及故燕、齐亡命者王之,都王险。
\end{yuanwen}

朝鲜王卫满,是以前的燕国人。当初在燕国全盛之时,曾经攻取真番、朝鲜,使它们归属燕国,给这些地区设置官吏,在边境上修筑城堡、关塞。秦国灭亡燕国后,朝鲜就成了辽东郡之外的边境国家。汉朝建立之后,由于朝鲜离得远,难以防守,就重新修复辽东郡过去的关塞,一直到浿水为界,属汉朝的诸侯国燕国管辖。燕王卢绾造反,逃入匈奴,卫满也逃走了,聚集一千多同党,梳着椎形的发髻,穿上了蛮夷的服装,向东逃出关塞,渡过了浿水,居住在过去秦朝称为上下鄣的空旷地方,逐渐地役使真番、朝鲜以及过去的燕国、齐国的逃亡者,在这些人中称王,在王险建都。

\begin{yuanwen}
会孝惠、高后时天下初定,辽东太守即约满为外臣,保塞外蛮夷,无使盗边;诸蛮夷君长欲入见天子,勿得禁止。以闻,上许之,以故满得兵威财物侵降\footnote{侵略、降服。}其旁小邑,真番、临屯皆来服属,方数千里。
\end{yuanwen}

赶上孝惠帝、吕后时期,天下局势刚刚稳定,辽东太守就约定让卫满做汉朝的藩属国的国君,保卫边塞之外的蛮夷,不让他们侵扰攻掠边境;各位蛮夷的首领想要来汉朝进见皇帝,不要禁止。辽东太守把这些情况报告给皇帝知道,皇帝答应了,因为这个缘故,卫满得以倚仗兵威和财物侵略、降服旁边的小国,真番、临屯都来降服归附,他统治的地区方圆几千里。

\begin{yuanwen}
	
\end{yuanwen}\begin{yuanwen}
	
\end{yuanwen}\begin{yuanwen}
	
\end{yuanwen}\begin{yuanwen}
	
\end{yuanwen}\begin{yuanwen}
	
\end{yuanwen}\begin{yuanwen}
	
\end{yuanwen}\begin{yuanwen}
	
\end{yuanwen}\begin{yuanwen}
	
\end{yuanwen}\begin{yuanwen}
	
\end{yuanwen}\begin{yuanwen}
	
\end{yuanwen}\begin{yuanwen}
	
\end{yuanwen}\begin{yuanwen}
	
\end{yuanwen}\begin{yuanwen}
	
\end{yuanwen}\begin{yuanwen}
	
\end{yuanwen}\begin{yuanwen}
	
\end{yuanwen}\begin{yuanwen}
	
\end{yuanwen}\begin{yuanwen}
	
\end{yuanwen}\begin{yuanwen}
	
\end{yuanwen}\begin{yuanwen}
	
\end{yuanwen}\begin{yuanwen}
	
\end{yuanwen}\begin{yuanwen}
	
\end{yuanwen}\begin{yuanwen}
	
\end{yuanwen}\begin{yuanwen}
	
\end{yuanwen}\begin{yuanwen}
	
\end{yuanwen}\begin{yuanwen}
	
\end{yuanwen}\begin{yuanwen}
	
\end{yuanwen}\begin{yuanwen}
	
\end{yuanwen}\begin{yuanwen}
	
\end{yuanwen}\begin{yuanwen}
	
\end{yuanwen}\begin{yuanwen}
	
\end{yuanwen}\begin{yuanwen}
	
\end{yuanwen}\begin{yuanwen}
	
\end{yuanwen}\begin{yuanwen}
	
\end{yuanwen}\begin{yuanwen}
	
\end{yuanwen}\begin{yuanwen}
	
\end{yuanwen}\begin{yuanwen}
	
\end{yuanwen}\begin{yuanwen}
	
\end{yuanwen}\begin{yuanwen}
	
\end{yuanwen}\begin{yuanwen}
	
\end{yuanwen}\begin{yuanwen}
	
\end{yuanwen}\begin{yuanwen}
	
\end{yuanwen}\begin{yuanwen}
	
\end{yuanwen}\begin{yuanwen}
	
\end{yuanwen}\begin{yuanwen}
	
\end{yuanwen}\begin{yuanwen}
	
\end{yuanwen}\begin{yuanwen}
	
\end{yuanwen}\begin{yuanwen}
	
\end{yuanwen}\begin{yuanwen}
	
\end{yuanwen}\begin{yuanwen}
	
\end{yuanwen}\begin{yuanwen}


传子至孙右渠,所诱汉亡人滋多,又未尝入见;真番旁众国欲上书见天子,又拥阏不通。元封二年,汉使涉何谯谕右渠,终不肯奉诏。何去至界上,临浿水,使御刺杀送何者朝鲜裨王长,即渡,驰入塞,遂归报天子曰“杀朝鲜将”。上为其名美,即不诘,拜何为辽东东部都尉。朝鲜怨何,发兵袭攻杀何。

天子募罪人击朝鲜。其秋,遣楼船将军杨仆从齐浮渤海;兵五万人,左将军荀彘出辽东:讨右渠。右渠发兵距险。左将军卒正多率辽东兵先纵,败散,多还走,坐法斩。楼船将军将齐兵七千人先至王险。右渠城守,窥知楼船军少,即出城击楼船,楼船军败散走。将军杨仆失其众,遁山中十馀日,稍求收散卒,复聚。左将军击朝鲜浿水西军,未能破自前。

天子为两将未有利,乃使卫山因兵威往谕右渠。右渠见使者顿首谢:“原降,恐两将诈杀臣;今见信节,请服降。”遣太子入谢,献马五千匹,及馈军粮。人众万馀,持兵,方渡浿水,使者及左将军疑其为变,谓太子已服降,宜命人毋持兵。太子亦疑使者左将军诈杀之,遂不渡浿水,复引归。山还报天子,天子诛山。

左将军破浿水上军,乃前,至城下,围其西北。楼船亦往会,居城南。右渠遂坚守城,数月未能下。

左将军素侍中,幸,将燕代卒,悍,乘胜,军多骄。楼船将齐卒,入海,固已多败亡;其先与右渠战,因辱亡卒,卒皆恐,将心惭,其围右渠,常持和节。左将军急击之,朝鲜大臣乃阴间使人私约降楼船,往来言,尚未肯决。左将军数与楼船期战,楼船欲急就其约,不会;左将军亦使人求间郤降下朝鲜,朝鲜不肯,心附楼船:以故两将不相能。左将军心意楼船前有失军罪,今与朝鲜私善而又不降,疑其有反计,未敢发。天子曰将率不能,前使卫山谕降右渠,右渠遣太子,山使不能剸决,与左将军计相误,卒沮约。今两将围城,又乖异,以故久不决。使济南太守公孙遂往之,有便宜得以从事。遂至,左将军曰:“朝鲜当下久矣,不下者有状。”言楼船数期不会,具以素所意告遂,曰:“今如此不取,恐为大害,非独楼船,又且与朝鲜共灭吾军。”遂亦以为然,而以节召楼船将军入左将军营计事,即命左将军麾下执捕楼船将军,并其军,以报天子。天子诛遂。

左将军已并两军,即急击朝鲜。朝鲜相路人、相韩阴、尼谿相参、将军王夹相与谋曰:“始欲降楼船,楼船今执,独左将军并将,战益急,恐不能与,王又不肯降。”阴、唊、路人皆亡降汉。路人道死。元封三年夏,尼谿相参乃使人杀朝鲜王右渠来降。王险城未下,故右渠之大臣成巳又反,复攻吏。左将军使右渠子长降、相路人之子最告谕其民,诛成巳,以故遂定朝鲜,为四郡。封参为澅清侯,阴为荻苴侯,唊为平州侯,长为几侯。最以父死颇有功,为温阳侯。

左将军徵至,坐争功相嫉,乖计,弃市。楼船将军亦坐兵至洌口,当待左将军,擅先纵,失亡多,当诛,赎为庶人。

太史公曰:右渠负固,国以绝祀。涉何诬功,为兵发首。楼船将狭,及难离咎。悔失番禺,乃反见疑。荀彘争劳,与遂皆诛。两军俱辱,将率莫侯矣。

卫满燕人,朝鲜是王。王险置都,路人作相。右渠首差,涉何俱上。兆祸自斯,狐疑二将。山、遂伏法,纷纭无状。
\end{yuanwen}

\chapter{西南夷列传}

\begin{yuanwen}
	

		
	\end{yuanwen}\begin{yuanwen}
		
	\end{yuanwen}\begin{yuanwen}
		
	\end{yuanwen}\begin{yuanwen}
		
	\end{yuanwen}\begin{yuanwen}
		
	\end{yuanwen}\begin{yuanwen}
		
	\end{yuanwen}\begin{yuanwen}
		
	\end{yuanwen}\begin{yuanwen}
		
	\end{yuanwen}\begin{yuanwen}
		
	\end{yuanwen}\begin{yuanwen}
		
	\end{yuanwen}\begin{yuanwen}
		
	\end{yuanwen}\begin{yuanwen}
		
	\end{yuanwen}\begin{yuanwen}
		
	\end{yuanwen}\begin{yuanwen}
		
	\end{yuanwen}\begin{yuanwen}
		
	\end{yuanwen}\begin{yuanwen}
		
	\end{yuanwen}\begin{yuanwen}
		
	\end{yuanwen}\begin{yuanwen}
		
	\end{yuanwen}\begin{yuanwen}
		
	\end{yuanwen}\begin{yuanwen}
		
	\end{yuanwen}\begin{yuanwen}
		
	\end{yuanwen}\begin{yuanwen}
		
	\end{yuanwen}\begin{yuanwen}
		
	\end{yuanwen}\begin{yuanwen}
		
	\end{yuanwen}\begin{yuanwen}
		
	\end{yuanwen}\begin{yuanwen}
		
	\end{yuanwen}\begin{yuanwen}
		
	\end{yuanwen}\begin{yuanwen}
		
	\end{yuanwen}
	\begin{yuanwen}
西南夷君长以什数,夜郎最大;其西靡莫之属以什数,滇最大;自滇以北君长以什数,邛都最大:此皆魋结,耕田,有邑聚。其外西自同师以东,北至楪榆,名为巂、昆明,皆编发,随畜迁徙,毋常处,毋君长,地方可数千里。自巂以东北,君长以什数,徙、筰都最大;自筰以东北,君长以什数,厓駹最大。其俗或士箸,或移徙,在蜀之西。自厓駹以东北,君长以什数,白马最大,皆氐类也。此皆巴蜀西南外蛮夷也。

始楚威王时,使将军庄蹻将兵循江上,略巴、黔中以西。庄蹻者,故楚庄王苗裔也。蹻至滇池,方三百里,旁平地,肥饶数千里,以兵威定属楚。欲归报,会秦击夺楚巴、黔中郡,道塞不通,因还,以其众王滇,变服,从其俗,以长之。秦时常頞略通五尺道,诸此国颇置吏焉。十馀岁,秦灭。及汉兴,皆弃此国而开蜀故徼。巴蜀民或窃出商贾,取其筰马、僰僮、髦牛,以此巴蜀殷富。

建元六年,大行王恢击东越,东越杀王郢以报。恢因兵威使番阳令唐蒙风指晓南越。南越食蒙蜀枸酱,蒙问所从来,曰“道西北牂柯,牂柯江广数里,出番禺城下”。蒙归至长安,问蜀贾人,贾人曰:“独蜀出枸酱,多持窃出市夜郎。夜郎者,临牂柯江,江广百馀步,足以行船。南越以财物役属夜郎,西至同师,然亦不能臣使也。”蒙乃上书说上曰:“南越王黄屋左纛,地东西万馀里,名为外臣,实一州主也。今以长沙、豫章往,水道多绝,难行。窃闻夜郎所有精兵,可得十馀万,浮船牂柯江,出其不意,此制越一奇也。诚以汉之彊,巴蜀之饶,通夜郎道,为置吏,易甚。”上许之。乃拜蒙为郎中将,将千人,食重万馀人,从巴蜀筰关入,遂见夜郎侯多同。蒙厚赐,喻以威德,约为置吏,使其子为令。夜郎旁小邑皆贪汉缯帛,以为汉道险,终不能有也,乃且听蒙约。还报,乃以为犍为郡。发巴蜀卒治道,自僰道指牂柯江。蜀人司马相如亦言西夷邛、筰可置郡。使相如以郎中将往喻,皆如南夷,为置一都尉,十馀县,属蜀。

当是时,巴蜀四郡通西南夷道,戍转相饟。数岁,道不通,士罢饿离湿死者甚众;西南夷又数反,发兵兴击,秏费无功。上患之,使公孙弘往视问焉。还对,言其不便。及弘为御史大夫,是时方筑朔方以据河逐胡,弘因数言西南夷害,可且罢,专力事匈奴。上罢西夷,独置南夷夜郎两县一都尉,稍令犍为自葆就。

及元狩元年,博望侯张骞使大夏来,言居大夏时见蜀布、邛竹、杖,使问所从来,曰“从东南身毒国,可数千里,得蜀贾人市”。或闻邛西可二千里有身毒国。骞因盛言大夏在汉西南,慕中国,患匈奴隔其道,诚通蜀,身毒国道便近,有利无害。于是天子乃令王然于、柏始昌、吕越人等,使间出西夷西,指求身毒国。至滇,滇王尝羌乃留,为求道西十馀辈。岁馀,皆闭昆明,莫能通身毒国。

滇王与汉使者言曰:“汉孰与我大?”及夜郎侯亦然。以道不通故,各自以为一州主,不知汉广大。使者还,因盛言滇大国,足事亲附。天子注意焉。

及至南越反,上使驰义侯因犍为发南夷兵。且兰君恐远行,旁国虏其老弱,乃与其众反,杀使者及犍为太守。汉乃发巴蜀罪人尝击南越者八校尉击破之。会越已破,汉八校尉不下,即引兵还,行诛头兰。头兰,常隔滇道者也。已平头兰,遂平南夷为牂柯郡。夜郎侯始倚南越,南越已灭,会还诛反者,夜郎遂入朝。上以为夜郎王。

南越破后,及汉诛且兰、邛君,并杀筰侯,厓駹皆振恐,请臣置吏。乃以邛都为越巂郡,筰都为沈犁郡,厓駹为汶山郡,广汉西白马为武都郡。

上使王然于以越破及诛南夷兵威风喻滇王入朝。滇王者,其众数万人,其旁东北有劳洸、靡莫,皆同姓相扶,未肯听。劳洸、靡莫数侵犯使者吏卒。元封二年,天子发巴蜀兵击灭劳洸、靡莫,以兵临滇。滇王始首善,以故弗诛。滇王离难西南夷,举国降,请置吏入朝。于是以为益州郡,赐滇王王印,复长其民。

西南夷君长以百数,独夜郎、滇受王印。滇小邑,最宠焉。

太史公曰:楚之先岂有天禄哉?在周为文王师,封楚。及周之衰,地称五千里。秦灭诸候,唯楚苗裔尚有滇王。汉诛西南夷,国多灭矣,唯滇复为宠王。然南夷之端,见枸酱番禺,大夏杖、邛竹。西夷后揃,剽分二方,卒为七郡。

西南外徼,庄蹻首通。汉因大夏,乃命唐蒙。劳洸、靡莫,异俗殊风。夜郎最大,邛、筰称雄。及置郡县,万代推功。
\end{yuanwen}

\chapter{司马相如列传}

\begin{yuanwen}
司马相如者,蜀郡成都人也,字长卿。少时好读书,学击剑,故其亲名之曰犬子。相如既学,慕蔺相如之为人,更名相如。以赀为郎,事孝景帝,为武骑常侍,非其好也。会景帝不好辞赋,是时梁孝王来朝,从游说之士齐人邹阳、淮阴枚乘、吴庄忌夫子之徒,相如见而说之,因病免,客游梁。梁孝王令与诸生同舍,相如得与诸生游士居数岁,乃著子虚之赋。

会梁孝王卒,相如归,而家贫,无以自业。素与临邛令王吉相善,吉曰:“长卿久宦游不遂,而来过我。”于是相如往,舍都亭。临邛令缪为恭敬,日往朝相如。相如初尚见之,后称病,使从者谢吉,吉愈益谨肃。临邛中多富人,而卓王孙家僮八百人,程郑亦数百人,二人乃相谓曰:“令有贵客,为具召之。”并召令。令既至,卓氏客以百数。至日中,谒司马长卿,长卿谢病不能往,临邛令不敢尝食,自往迎相如。相如不得已,彊往,一坐尽倾。酒酣,临邛令前奏琴曰:“窃闻长卿好之,原以自娱。”相如辞谢,为鼓一再行。是时卓王孙有女文君新寡,好音,故相如缪与令相重,而以琴心挑之。相如之临邛,从车骑,雍容间雅甚都;及饮卓氏,弄琴,文君窃从户窥之,心悦而好之,恐不得当也。既罢,相如乃使人重赐文君侍者通殷勤。文君夜亡奔相如,相如乃与驰归成都。家居徒四壁立。卓王孙大怒曰:“女至不材,我不忍杀,不分一钱也。”人或谓王孙,王孙终不听。文君久之不乐,曰:“长卿第俱如临邛,从昆弟假贷犹足为生,何至自苦如此!”相如与俱之临邛,尽卖其车骑,买一酒舍酤酒,而令文君当炉。相如身自著犊鼻裈,与保庸杂作,涤器于市中。卓王孙闻而耻之,为杜门不出。昆弟诸公更谓王孙曰:“有一男两女,所不足者非财也。今文君已失身于司马长卿,长卿故倦游,虽贫,其人材足依也,且又令客,独柰何相辱如此!”卓王孙不得已,分予文君僮百人,钱百万,及其嫁时衣被财物。文君乃与相如归成都,买田宅,为富人。

居久之,蜀人杨得意为狗监,侍上。上读子虚赋而善之,曰:“朕独不得与此人同时哉!”得意曰:“臣邑人司马相如自言为此赋。”上惊,乃召问相如。相如曰:“有是。然此乃诸侯之事,未足观也。请为天子游猎赋,赋成奏之。”上许,令尚书给笔札。相如以“子虚”,虚言也,为楚称;“乌有先生”者,乌有此事也,为齐难;“无是公”者,无是人也,明天子之义。故空藉此三人为辞,以推天子诸侯之苑囿。其卒章归之于节俭,因以风谏。奏之天子,天子大说。其辞曰:

楚使子虚使于齐,齐王悉发境内之士,备车骑之众,与使者出田。田罢,子虚过詑乌有先生,而无是公在焉。坐定,乌有先生问曰:“今日田乐乎?”子虚曰:“乐。”“获多乎?”曰:“少。”“然则何乐?”曰:“仆乐齐王之欲夸仆以车骑之众,而仆对以云梦之事也。”曰:“可得闻乎?”

子虚曰:“可。王驾车千乘,选徒万骑,田于海滨。列卒满泽,罘罔弥山,揜兔辚鹿,射麋脚麟。鹜于盐浦,割鲜染轮。射中获多,矜而自功。顾谓仆曰:‘楚亦有平原广泽游猎之地饶乐若此者乎?楚王之猎何与寡人?’ 仆下车对曰:‘臣,楚国之鄙人也,幸得宿卫十有馀年,时从出游,游于后园,览于有无,然犹未能遍睹也,又恶足以言其外泽者乎!’ 齐王曰:‘虽然,略以子之所闻见而言之。’

“仆对曰:‘唯唯。臣闻楚有七泽,尝见其一,未睹其馀也。臣之所见,盖特其小小者耳,名曰云梦。云梦者,方九百里,其中有山焉。其山则盘纡岪郁,隆崇嵂崒;岑岩参差,日月蔽亏;交错纠纷,上干青云;罢池陂纮,下属江河。其土则丹青赭垩,雌黄白附,锡碧金银,众色炫燿,照烂龙鳞。其石则赤玉玫瑰,琳渼琨珸,瑊玏玄厉,萩石武夫。其东则有蕙圃衡兰,芷若射干,穹穷昌蒲,江离麋芜,诸蔗猼且。其南则有平原广泽,登降纮靡,案衍坛曼,缘以大江,限以巫山。其高燥则生葴蓇苞荔,薛莎青薠。其卑湿则生藏莨蒹葭,东蔷雕胡,莲藕菰芦,菴{艹闾}轩芋,物居之,不可胜图。其西则有涌泉清池,激水推移;外发芙蓉菱华,内隐钜石白沙。其中则有神龟蛟鼍,玳瑁鳖鼋。其北则有阴林巨树,楩棻豫章,桂椒木兰,离硃杨,楂梸甹栗,橘柚芬芳。其上则有赤猿蠷蝚,鹓雏孔鸾,腾远射干。其下则有白虎玄豹,蟃蜒貙豻,兕象野犀,穷奇獌狿。

“‘于是乃使专诸之伦,手格此兽。楚王乃驾驯驳之驷,乘雕玉之舆,靡鱼须之桡旃,曳明月之珠旗,建干将之雄戟,左乌嗥之雕弓,右夏服之劲箭;阳子骖乘,纤阿为御;案节未舒,即陵狡兽,辚邛邛,槅距虚,轶野马而湜騊駼,乘遗风而射游骐;儵眒凄浰,雷动熛至,星流霆击,弓不虚发,中必决眦,洞胸达腋,绝乎心系,获若雨兽,揜草蔽地。于是楚王乃弭节裴回,翱翔容与,览乎阴林,观壮士之暴怒,与猛兽之恐惧,徼受诎,殚睹物之变态。

“‘于是郑女曼姬,被阿锡,揄纻缟,櫜纤罗,垂雾縠;襞积褰绉,纡徐委曲,郁桡谿谷;衯衯裶裶,扬袘恤削,蜚纤垂髾;扶与猗靡,吸呷萃蔡,下摩兰蕙,上拂羽盖,错翡翠之威蕤,缪绕玉绥;缥乎忽忽,若神仙之仿佛。

“‘于是乃相与獠于蕙圃,媻珊勃窣上金隄,揜翡翠,射鵕璘,微矰出,纤缴施,弋白鹄,连驾鹅,双鸧下,玄鹤加。怠而后发,游于清池;浮文鹢,扬桂枻,张翠帷,建羽盖,罔玳瑁,钓紫贝;摐金鼓,吹鸣籁,榜人歌,声流喝,水蟲骇,波鸿沸,涌泉起,奔扬会,礧石相击,硠硠潏潏,若雷霆之声,闻乎数百里之外。

“‘将息獠者,击灵鼓,起烽燧,车案行,骑就队,纚乎淫淫,班乎裔裔。于是楚王乃登阳云之台,泊乎无为,澹乎自持,勺药之和具而后御之。不若大王终日驰骋而不下舆,脟割轮淬,自以为娱。臣窃观之,齐殆不如。’ 于是王默然无以应仆也。”

乌有先生曰:“是何言之过也!足下不远千里,来况齐国,王悉发境内之士,而备车骑之众,以出田,乃欲戮力致获,以娱左右也,何名为夸哉!问楚地之有无者,原闻大国之风烈,先生之馀论也。今足下不称楚王之德厚,而盛推云梦以为高,奢言淫乐而显侈靡,窃为足下不取也。必若所言,固非楚国之美也。有而言之,是章君之恶;无而言之,是害足下之信。章君之恶而伤私义,二者无一可,而先生行之,必且轻于齐而累于楚矣。且齐东陼巨海,南有琅邪,观乎成山,射乎之罘,浮勃澥,游孟诸,邪与肃慎为邻,右以汤谷为界,秋田乎青丘,傍徨乎海外,吞若云梦者八九,其于胸中曾不蒂芥。若乃俶傥瑰伟,异方殊类,珍怪鸟兽,万端鳞萃,充仞其中者,不可胜记,禹不能名,契不能计。然在诸侯之位,不敢言游戏之乐,苑囿之大;先生又见客,是以王辞而不复,何为无用应哉!”

无是公听然而笑曰:“楚则失矣,齐亦未为得也。夫使诸侯纳贡者,非为财币,所以述职也;封疆画界者,非为守御,所以禁淫也。今齐列为东籓,而外私肃慎,捐国逾限,越海而田,其于义故未可也。且二君之论,不务明君臣之义而正诸侯之礼,徒事争游猎之乐,苑囿之大,欲以奢侈相胜,荒淫相越,此不可以扬名发誉,而適足以贬君自损也。且夫齐楚之事又焉足道邪!君未睹夫巨丽也,独不闻天子之上林乎?

“左苍梧,右西极,丹水更其南,紫渊径其北;终始霸浐,出入泾渭;酆鄗潦潏,纡馀委蛇,经营乎其内。荡荡兮八川分流,相背而异态。东西南北,驰骛往来,出乎椒丘之阙,行乎洲淤之浦,径乎桂林之中,过乎泱莽之野。汨乎浑流,顺阿而下,赴隘陕之口。触穹石,激堆埼,沸乎暴怒,汹涌滂晞,滭浡滵汩,湢测泌瀄,横流逆折,转腾潎洌,澎濞沆瀣,穹隆云挠,蜿胶戾,逾波趋浥,莅莅下濑,批壧旻壅,饹扬滞沛,临坻注壑,瀺灂霣坠,湛湛隐隐,砰磅訇潏,潏潏淈淈,湁潗鼎沸,驰波跳沫,汩槃漂疾,悠远长怀,寂漻无声,肆乎永归。然后灝溔潢漾,安翔徐徊,翯乎滈々,东注大湖,衍溢陂池。于是乎蛟龙赤螭,靧亸螹离,鰅騄鰬魠,禺禺鱋魶,揵鳍擢尾,振鳞奋翼,潜处于深岩;鱼鳖讙声,万物众夥,明月珠子,玓瓅江靡,蜀石黄鶗,水玉磊砢,磷磷烂烂,采色霅旰,丛积乎其中。鸿鹄鹔鸨,磻蟏鸀,鴂目,烦鹜鷛醁,澥昉鸕,群浮乎其上。汎淫泛滥,随风澹淡,与波摇荡,掩薄草渚,唼喋菁藻,咀嚼菱藕。

“于是乎崇山巃嵸,崔巍嵯峨,深林钜木,崭岩嵾嵯,九嵏、嶻,南山峨峨,岩纮甗锜,嶊崣崛崎,振谿通谷,蹇产沟渎,谽呀豁閜,轗陵别岛,崴磈岧瘣,丘虚崛嶮,隐辚郁鹍,登降施靡,陂池貏豸,沇溶淫鬻,散涣夷陆,亭皋千里,靡不被筑。掩以绿蕙,被以江离,糅以蘼芜,杂以流夷。尃结缕,欑戾莎,揭车衡兰,本射干,茈姜蘘荷,葴橙若荪,鲜枝黄砾,蒋芧青薠,布濩闳泽,延曼太原,丽靡广衍,应风披靡,吐芳扬烈,郁郁斐斐,众香发越,肸蚃布写,餔苾勃。“于是乎周览泛观,瞋盼轧沕,芒芒恍忽,视之无端,察之无崖。日出东沼,入于西陂。其南则隆冬生长,踊水跃波;兽则偁旄敠犛,沈牛麈麋,赤首圜题,穷奇象犀。其北则盛夏含冻裂地,涉冰揭河;兽则麒麟角湲,騊駼橐扆,蛩蛩驒騱,駃騠驴骡。

“于是乎离宫别馆,弥山跨谷,高廊四注,重坐曲阁,华榱璧珰,辇道纚属,步朓周流,长途中宿。夷颙筑堂,累台增成,岩穾洞房,俯杳眇而无见,仰攀橑而扪天,奔星更于闺闼,宛虹拖于楯轩。青虬蚴蟉于东箱,象舆婉蝉于西清,灵圉燕于间观,偓佺之伦暴于南荣,醴泉涌于清室,通川过乎中庭。槃石裖崖,嵚岩倚倾,嵯峨磼酺,刻削峥嵘,玫瑰碧琳,珊瑚丛生,渼玉旁唐,瑸斒文鳞,赤瑕驳荦,杂臿其间,垂绥琬琰,和氏出焉。

“于是乎卢橘夏孰,黄甘橙楱,枇杷橪柿,楟柰厚朴,甹枣杨梅,樱桃蒲陶,隐夫郁棣,榙濛荔枝,罗乎后宫,列乎北园。崒丘陵,下平原,扬翠叶,杌紫茎,发红华,秀硃荣,煌煌扈扈,照曜钜野。沙棠栎櫧,华氾弇栌,留落胥馀,仁频并闾,欃檀木兰,豫章女贞,长千仞,大连抱,夸条直暢,实叶葰茂,攒立丛倚,连卷累佹,崔错骫,阬衡閜砢,垂条扶于,落英幡纚,纷容萧蔘,旖旎从风,浏莅吸,盖象金石之声,管籥之音。柴池茈虒,旋环后宫,杂遝累辑,被山缘谷,循阪下隰,视之无端,究之无穷。

“于是玄猿素雌,蜼玃飞鸓,蛭蜩蠗蝚,螹胡蛫,栖息乎其间;长啸哀鸣,翩幡互经,夭蟜枝格,偃蹇杪颠。于是乎隃绝梁,腾殊榛,捷垂条,踔稀间,牢落陆离,烂曼远迁。

“若此辈者,数千百处。嬉游往来,宫宿馆舍,庖厨不徙,后宫不移,百官备具。

“于是乎背秋涉冬,天子校猎。乘镂象,六玉虬,拖蜺旌,靡云旗,前皮轩,后道游;孙叔奉辔,卫公骖乘,扈从横行,出乎四校之中。鼓严簿,纵獠者,江河为阹,泰山为橹,车骑雷起,隐天动地,先后陆离,离散别追,淫淫裔裔,缘陵流泽,云布雨施。”

“生貔豹,搏豺狼,手熊罴,足野羊,蒙鹖苏,绔白虎,被豳文,跨野马。陵三颙之危,下碛历之坻;俓鷟赴险,越壑厉水。推蜚廉,弄解豸,格瑕蛤,鋋猛氏,罥騕褭,射封豕。箭不苟害,解脰陷脑;弓不虚发,应声而倒。于是乎乘舆弥节裴回,翱翔往来,睨部曲之进退,览将率之变态。然后浸潭促节,儵夐远去,流离轻禽,槅履狡兽,轊白鹿,捷狡兔,轶赤电,遗光燿,追怪物,出宇宙,弯繁弱,满白羽,射游枭,栎蜚虡,择肉后发,先中命处,弦矢分,艺殪仆。

“然后扬节而上浮,陵惊风,历骇梠,乘虚无,与神俱,辚玄鹤,乱昆鸡。遒孔鸾,促鵕璘,拂鹥鸟,捎凤皇,捷鸳雏,掩焦明。

“道尽涂殚,回车而还。招摇乎襄羊,降集乎北纮,率乎直指,闇乎反乡。“道尽涂殚,回车而还。招摇乎襄羊,降集乎北纮,率乎直指,闇乎反乡。蹶石,历封峦,过乂鹊,望露寒,下棠梨,息宜春,西驰宣曲,濯鹢牛首,登龙台,掩细柳,观士大夫之勤略,钧獠者之所得获。徒车之所辚轹,乘骑之所蹂若,人民之所蹈騃,与其穷极倦,惊惮慴伏,不被创刃而死者,佗佗籍籍,填阬满谷,揜平弥泽。

“于是乎游戏懈怠,置酒乎昊天之台,张乐乎轇輵之宇;撞千石之钟,立万石之钜;建翠华之旗,树灵鼍之鼓。奏陶唐氏之舞,听葛天氏之歌,千人唱,万人和,山陵为之震动,川谷为之荡波。巴俞宋蔡,淮南于遮,文成颠歌,族举递奏,金鼓迭起,铿鎗铛剸,洞心骇耳。荆吴郑卫之声,韶濩武象之乐,阴淫案衍之音,鄢郢缤纷,激楚结风,俳优侏儒,狄鞮之倡,所以娱耳目而乐心意者,丽靡烂漫于前,靡曼美色于后。

“若夫青琴宓妃之徒,绝殊离俗,姣冶嫺都,靓庄刻饬,便嬛绰约,柔桡嬛嬛,妩媚佺弱;抴独茧之褕袘,眇阎易以戌削,编姺徶蘋,与世殊服;芬香沤郁,酷烈淑郁;皓齿粲烂,宜笑旳皪;长眉连娟,微睇釂藐;色授魂与,心愉于侧。

“于是酒中乐酣,天子芒然而思,似若有亡。曰:‘嗟乎,此泰奢侈!朕以览听馀“于是酒中乐酣,天子芒然而思,似若有亡。曰:‘嗟乎,此泰奢侈!朕以览听馀间,无事弃日,顺天道以杀伐,时休息于此,恐后世靡丽,遂往而不反,非所以为继嗣创业垂统也。’ 于是乃解酒罢猎,而命有司曰:‘地可以垦辟,悉为农郊,以赡萌隶;隤墙填堑,使山泽之民得至焉。实陂池而勿禁,虚宫观而勿仞。发仓廪以振贫穷,补不足,恤鳏寡,存孤独。出德号,省刑罚,改制度,易服色,更正朔,与天下为始。’

“于是历吉日以齐戒,袭朝衣,乘法驾,建华旗,鸣玉鸾,游乎六艺之囿,骛乎仁义之涂,览观春秋之林,射貍首,兼驺虞,弋玄鹤,建干戚,载云鶒,揜群雅,悲伐檀,乐乐胥,修容乎礼园,翱翔乎书圃,述易道,放怪兽,登明堂,坐清庙,恣群臣,奏得失,四海之内,靡不受获。于斯之时,天下大说,乡风而听,随流而化,喟然兴道而迁义,刑错而不用,德隆乎三皇,功羡于五帝。若此,故猎乃可喜也。

“若夫终日暴露驰骋,劳神苦形,罢车马之用,抏士卒之精,费府库之财,而无德厚之恩,务在独乐,不顾众庶,忘国家之政,而贪雉兔之获,则仁者不由也。从此观之,齐楚之事,岂不哀哉!地方不过千里,而囿居九百,是草木不得垦辟,而民无所食也。夫以诸侯之细,而乐万乘之所侈,仆恐百姓之被其尤也。”

于是二子愀然改容,超若自失,逡巡避席曰:“鄙人固陋,不知忌讳,乃今日见教,谨闻命矣。”

赋奏,天子以为郎。无是公言天子上林广大,山谷水泉万物,乃子虚言楚云梦所有甚众,侈靡过其实,且非义理所尚,故删取其要,归正道而论之。

相如为郎数岁,会唐蒙使略通夜郎西僰中,发巴蜀吏卒千人,郡又多为发转漕万馀人,用兴法诛其渠帅,巴蜀民大惊恐。上闻之,乃使相如责唐蒙,因喻告巴蜀民以非上意。檄曰:

告巴蜀太守:蛮夷自擅不讨之日久矣,时侵犯边境,劳士大夫。陛下即位,存抚天下,辑安中国。然后兴师出兵,北征匈奴,单于怖骇,交臂受事,诎膝请和。康居西域,重译请朝,稽首来享。移师东指,闽越相诛。右吊番禺,太子入朝。南夷之君,西僰之长,常效贡职,不敢怠堕,延颈举踵,喁喁然皆争归义,欲为臣妾,道里辽远,山川阻深,不能自致。夫不顺者已诛,而为善者未赏,故遣中郎将往宾之,发巴蜀士民各五百人,以奉币帛,卫使者不然,靡有兵革之事,战斗之患。今闻其乃发军兴制,

惊惧子弟,忧患长老,郡又擅为转粟运输,皆非陛下之意也。当行者或亡逃自贼杀,亦非人臣之节也。

夫边郡之士,闻烽举燧燔,皆摄弓而驰,荷兵而走,流汗相属,唯恐居后,触白刃,冒流矢,义不反顾,计不旋踵,人怀怒心,如报私雠。彼岂乐死恶生,非编列之民,而与巴蜀异主哉?计深虑远,急国家之难,而乐尽人臣之道也。故有剖符之封,析珪而爵,位为通侯,居列东第,终则遗显号于后世,传土地于子孙,行事甚忠敬,居位甚安佚,名声施于无穷,功烈著而不灭。是以贤人君子,肝脑涂中原,膏液润野草而不辞也。今奉币役至南夷,即自贼杀,或亡逃抵诛,身死无名,谥为至愚,耻及父母,为天下笑。人之度量相越,岂不远哉!然此非独行者之罪也,父兄之教不先,子弟之率不谨也;寡廉鲜耻,而俗不长厚也。其被刑戮,不亦宜乎!

陛下患使者有司之若彼,悼不肖愚民之如此,故遣信使晓喻百姓以发卒之事,因数之以不忠死亡之罪,让三老孝弟以不教诲之过。方今田时,重烦百姓,已亲见近县,恐远所谿谷山泽之民不遍闻,檄到,亟下县道,使咸知陛下之意,唯毋忽也。

相如还报。唐蒙已略通夜郎,因通西南夷道,发巴、蜀、广汉卒,作者数万人。治道二岁,道不成,士卒多物故,费以巨万计。蜀民及汉用事者多言其不便。是时邛筰之君长闻南夷与汉通,得赏赐多,多欲原为内臣妾,请吏,比南夷。天子问相如,相如曰:“邛、筰、厓、駹者近蜀,道亦易通,秦时尝通为郡县,至汉兴而罢。今诚复通,为置郡县,愈于南夷。”天子以为然,乃拜相如为中郎将,建节往使。副使王然于、壶充国、吕越人驰四乘之传,因巴蜀吏币物以赂西夷。至蜀,蜀太守以下郊迎,县令负弩矢先驱,蜀人以为宠。于是卓王孙、临邛诸公皆因门下献牛酒以交驩。卓王孙喟然而叹,自以得使女尚司马长卿晚,而厚分与其女财,与男等同。司马长卿便略定西夷,邛、筰、厓、駹、斯榆之君皆请为内臣。除边关,关益斥,西至沬、若水,南至牂柯为徼,通零关道,桥孙水以通邛都。还报天子,天子大说。

相如使时,蜀长老多言通西南夷不为用,唯大臣亦以为然。相如欲谏,业已建之,不敢,乃著书,籍以蜀父老为辞,而己诘难之,以风天子,且因宣其使指,令百姓知天子之意。其辞曰:

汉兴七十有八载,德茂存乎六世,威武纷纭,湛恩汪濊,群生澍濡,洋溢乎方外。于是乃命使西征,随流而攘,风之所被,罔不披靡。因朝厓从駹,定筰存邛,略斯榆,举苞满,结轶还辕,东乡将报,至于蜀都。

耆老大夫荐绅先生之徒二十有七人,俨然造焉。辞毕,因进曰:“盖闻天子之于夷狄也,其义羁縻勿绝而已。今罢三郡之士,通夜郎之涂,三年于兹,而功不竟,士卒劳倦,万民不赡,今又接以西夷,百姓力屈,恐不能卒业,此亦使者之累也,窃为左右患之。且夫邛、筰、西僰之与中国并也,历年兹多,不可记已。仁者不以德来,彊者不以力并,意者其殆不可乎!今割齐民以附夷狄,弊所恃以事无用,鄙人固陋,不识所谓。”

使者曰:“乌谓此邪?必若所云,则是蜀不变服而巴不化俗也。余尚恶闻若说。然斯事体大,固非观者之所觏也。余之行急,其详不可得闻已,请为大夫粗陈其略。

“盖世必有非常之人,然后有非常之事;有非常之事,然后有非常之功。非常者,固常之所异也。故曰非常之原,黎民惧焉;及臻厥成,天下晏如也。

“昔者鸿水浡出,氾滥衍溢,民人登降移徙,陭麕而不安。夏后氏戚之,乃堙鸿水,决江疏河,漉沈赡菑,东归之于海,而天下永宁。当斯之勤,岂唯民哉。心烦于虑而身亲其劳,躬胝无胈,肤不生毛。故休烈显乎无穷,声称浃乎于兹。

“且夫贤君之践位也。岂特委琐握麀,拘文牵俗,循诵习传,当世取说云尔哉!必将崇论闳议,创业垂统,为万世规。故驰骛乎兼容并包,而勤思乎参天贰地。且诗不云乎:‘普天之下,莫非王土;率土之滨,莫非王臣。’ 是以六合之内,八方之外,浸浔衍溢,怀生之物有不浸润于泽者,贤君耻之。今封疆之内,冠带之伦,咸获嘉祉,靡有阙遗矣。而夷狄殊俗之国,辽绝异党之地,舟舆不通,人迹罕至,政教未加,流风犹微。内之则犯义侵礼于边境,外之则邪行横作,放弑其上。君臣易位,尊卑失序,父兄不辜,幼孤为奴,系累号泣,内乡而怨,曰‘盖闻中国有至仁焉,德洋而恩普,物靡不得其所,今独曷为遗己’ 。举踵思慕,若枯旱之望雨。盭夫为之垂涕,况乎上圣,又恶能已?故北出师以讨彊胡,南驰使以诮劲越。四面风德,二方之君鳞集仰流,原得受号者以亿计。故乃关沬、若,徼牂柯,镂零山,梁孙原。创道德之涂,垂仁义之统。将博恩广施,远抚长驾,使疏逖不闭,阻深闇昧得耀乎光明,以偃甲兵于此,而息诛伐于彼。遐迩一体,中外提福,不亦康乎?夫拯民于沈溺,奉至尊之休德,反衰世之陵迟,继周氏之绝业,斯乃天子之急务也。百姓虽劳,又恶可以已哉?

“且夫王事固未有不始于忧勤,而终于佚乐者也。然则受命之符,合在于此矣。方将增泰山之封,加梁父之事,鸣和鸾,扬乐颂,上咸五,下登三。观者未睹指,听者未闻音,犹鹪明已翔乎寥廓,而罗者犹视乎薮泽。悲夫!”

于是诸大夫芒然丧其所怀来而失厥所以进,喟然并称曰:“允哉汉德,此鄙人之所原闻也。百姓虽怠,请以身先之。”敞罔靡徙,因迁延而辞避。

其后人有上书言相如使时受金,失官。居岁馀,复召为郎。

相如口吃而善著书。常有消渴疾。与卓氏婚,饶于财。其进仕宦,未尝肯与公卿国家之事,称病间居,不慕官爵。常从上至长杨猎,是时天子方好自击熊彘,驰逐野兽,相如上疏谏之。其辞曰:臣闻物有同类而殊能者,故力称乌获,捷言庆忌,勇期贲、育。臣之愚,窃以为人诚有之,兽亦宜然。今陛下好陵阻险,射猛兽,卒然遇轶材之兽,骇不存之地,犯属车之清尘,舆不及还辕,人不暇施巧,虽有乌获、逢蒙之伎,力不得用,枯木朽株尽为害矣。是胡越起于毂下,而羌夷接轸也,岂不殆哉!虽万全无患,然本非天子之所宜近也。

且夫清道而后行,中路而后驰,犹时有衔橛之变,而况涉乎蓬蒿,驰乎丘坟,前有利兽之乐而内无存变之意,其为祸也不亦难矣!夫轻万乘之重不以为安,而乐出于万有一危之涂以为娱,臣窃为陛下不取也。

盖明者远见于未萌而智者避危于无形,祸固多藏于隐微而发于人之所忽者也。故鄙谚曰“家累千金,坐不垂堂”。此言虽小,可以喻大。臣原陛下之留意幸察。

上善之。还过宜春宫,相如奏赋以哀二世行失也。其辞曰:

登陂阤之长阪兮,坌入曾宫之嵯峨。临曲江之隑州兮,望南山之参差。岩岩深山之谾々兮,通谷魌兮谽。汩淢噏习以永逝兮,注平皋之广衍。观众树之塕兮,览之榛榛。东驰土山兮,北揭石濑。弥节容与兮,历吊二世。持身不谨兮,亡国失埶。信谗不寤兮,宗庙灭绝。呜呼哀哉!操行之不得兮,坟墓芜秽而不脩兮,魂无归而不食。夐邈绝而不齐兮,弥久远而愈鬐。精罔阆而飞扬兮,拾九天而永逝。呜呼哀哉!

相如拜为孝文园令。天子既美子虚之事,相如见上好仙道,因曰:“上林之事未足美也,尚有靡者。臣尝为大人赋,未就,请具而奏之。”相如以为列仙之传居山泽间,形容甚癯,此非帝王之仙意也,乃遂就大人赋。其辞曰:

世有大人兮,在于中州。宅弥万里兮,曾不足以少留。悲世俗之迫隘兮,朅轻举而远游。垂绛幡之素蜺兮,载云气而上浮。建格泽之长竿兮,总光耀之采旄。垂旬始以为幓兮,抴彗星而为髾。掉指桥以偃蹇兮,又旖旎以招摇。揽欃枪以为旌兮,靡屈虹而为绸。红杳渺以眩湣兮,猋风涌而云浮。驾应龙象舆之蠖略逶丽兮,骖赤螭青虬之鞮蟉蜿蜒。低卬夭蟜据以骄骜兮,诎折隆穷蠼以连卷沛艾赳螑仡以佁儗兮,放散畔岸骧以孱颜。跮踱輵辖容以委丽兮,绸缪偃蹇怵鞨以梁倚。纠蓼叫奡蹋以艐路兮,蔑蒙踊跃腾而狂趡。莅飒卉翕熛至电过兮,焕然雾除,霍然云消。

邪绝少阳而登太阴兮,与真人乎相求。互折窈窕以右转兮,横厉飞泉以正东。悉徵灵圉而选之兮,部乘众神于瑶光。使五帝先导兮,反太一而从陵阳。左玄冥而右含雷兮,前陆离而后潏湟。厮征伯侨而役羡门兮,属岐伯使尚方。祝融惊而跸御兮,清雰气而后行。屯余车其万乘兮,綷云盖而树华旗。使句芒其将行兮,吾欲往乎南嬉。

历唐尧于崇山兮,过虞舜于九疑。纷湛湛其差错兮,杂遝胶葛以方驰。骚扰冲苁其相纷挐兮,滂濞泱轧洒以林离。钻罗列聚丛以茏茸兮,衍曼流烂坛以陆离。径入雷室之砰磷郁律兮,洞出鬼谷之嚬礨嵬靺。遍览八纮而观四荒兮,朅渡九江而越五河。经营炎火而浮弱水兮,杭绝浮渚而涉流沙。奄息总极氾滥水嬉兮,使灵娲鼓瑟而舞冯夷。时若々将混浊兮,召屏翳诛风伯而刑雨师。西望昆仑之轧沕洸忽兮,直径驰乎三危。排阊阖而入帝宫兮,载玉女而与之归。舒阆风而摇集兮,亢乌腾而一止。低回阴山翔以纡曲兮,吾乃今目睹西王母鱇然白首。载胜而穴处兮,亦幸有三足乌为之使。必长生若此而不死兮,虽济万世不足以喜。

回车朅来兮,绝道不周,会食幽都。呼吸沆瀣餐朝霞,噍咀芝英兮叽琼华。嬐侵浔而高纵兮,纷鸿涌而上厉。贯列缺之倒景兮,涉丰隆之滂沛。驰游道而脩降兮,骛遗雾而远逝。迫区中之隘陕兮,舒节出乎北垠。遗屯骑于玄阙兮,轶先驱于寒门。下峥嵘而无地兮,上寥廓而无天。视眩眠而无见兮,听惝恍而无闻。乘虚无而上假兮,超无友而独存。

相如既奏大人之颂,天子大说,飘飘有凌云之气,似游天地之间意。

相如既病免,家居茂陵。天子曰:“司马相如病甚,可往从悉取其书;若不然,后失之矣。”使所忠往,而相如已死,家无书。问其妻,对曰:“长卿固未尝有书也。时时著书,人又取去,即空居。长卿未死时,为一卷书,曰有使者来求书,奏之。无他书。”其遗札书言封禅事,奏所忠。忠奏其书,天子异之。其书曰:伊上古之初肇,自昊穹兮生民,历撰列辟,以迄于秦。率迩者踵武,逖听者风声。纷纶葳蕤,堙灭而不称者,不可胜数也。续昭夏,崇号谥,略可道者七十有二君。罔若淑而不昌,畴逆失而能存?

轩辕之前,遐哉邈乎,其详不可得闻也。五三六经载籍之传,维见可观也。书曰“元首明哉,股肱良哉”。因斯以谈,君莫盛于唐尧,臣莫贤于后稷。后稷创业于唐,公刘发迹于西戎,文王改制,爰周郅隆,大行越成,而后陵夷衰微,千载无声,岂不善始善终哉。然无异端,慎所由于前,谨遗教于后耳。故轨迹夷易,易遵也;湛恩濛涌,易丰也;宪度著明,易则也;垂统理顺,易继也。是以业隆于繦褓而崇冠于二后。揆厥所元,终都攸卒,未有殊尤绝迹可考于今者也。然犹蹑梁父,登泰山,建显号,施尊名。大汉之德,逢涌原泉,沕潏漫衍,旁魄四塞,云尃雾散,上暢九垓,下溯八埏。怀生之类霑濡浸润,协气横流,武节飘逝,迩陕游原,迥阔泳沫,首恶湮没,闇昧昭晢,昆蟲凯泽,回首面内。然后囿驺虞之珍群,徼麋鹿之怪兽,鳒一茎六穗于庖,牺双共抵之兽,获周馀珍收龟于岐,招翠黄乘龙于沼。鬼神接灵圉,宾于间馆。奇物谲诡,俶傥穷变。钦哉,符瑞臻兹,犹以为薄,不敢道封禅。盖周跃鱼陨杭,休之以燎,微夫斯之为符也,以登介丘,不亦恧乎!进让之道,其何爽与?

于是大司马进曰:“陛下仁育群生,义征不憓,诸夏乐贡,百蛮执贽,德侔往初,功无与二,休烈浃洽,符瑞众变,期应绍至,不特创见。意者泰山、梁父设坛场望幸,盖号以况荣,上帝垂恩储祉,将以荐成,陛下谦让而弗发也。挈三神之驩,缺王道之仪,群臣恧焉。或谓且天为质闇,珍符固不可辞;若然辞之,是泰山靡记而梁父靡几也。亦各并时而荣,咸济世而屈,说者尚何称于后,而云七十二君乎?夫修德以锡符,奉符以行事,不为进越。故圣王弗替,而修礼地祇,谒款天神,勒功中岳,以彰至尊,舒盛德,发号荣,受厚福,以浸黎民也。皇皇哉斯事!天下之壮观,王者之丕业,不可贬也。原陛下全之。而后因杂荐绅先生之略术,使获燿日月之末光绝炎,以展采错事,犹兼正列其义,校饬厥文,作春秋一艺,将袭旧六为七,摅之无穷,俾万世得激清流,扬微波,蜚英声,腾茂实。前圣之所以永保鸿名而常为称首者用此,宜命掌故悉奏其义而览焉。”

于是天子沛然改容,曰:“愉乎,朕其试哉!”乃迁思回虑,总公卿之议,询封禅之事,诗大泽之博,广符瑞之富。乃作颂曰:

自我天覆,云之油油。甘露时雨,厥壤可游。滋液渗漉,何生不育;嘉自我天覆,云之油油。甘露时雨,厥壤可游。滋液渗漉,何生不育;嘉穀六穗,我穑曷蓄。

非唯雨之,又润泽之;非唯濡之,氾尃濩之。万物熙熙,怀而慕思。名山非唯雨之,又润泽之;非唯濡之,氾尃濩之。万物熙熙,怀而慕思。名山显位,望君之来。君乎君乎,侯不迈哉!

般般之兽,乐我君囿;白质黑章,其仪可;旼々睦睦,君子之般般之兽,乐我君囿;白质黑章,其仪可;旼々睦睦,君子之

能。盖闻其声,今观其来。厥涂靡踪,天瑞之徵。兹亦于舜,虞氏以兴。

濯濯之麟,游彼灵畤。孟冬十月,君俎郊祀。驰我君舆,帝以享祉。三濯濯之麟,游彼灵畤。孟冬十月,君俎郊祀。驰我君舆,帝以享祉。三代之前,盖未尝有。

宛宛黄龙,兴德而升;采色炫燿,熿炳煇煌。正阳显见,于传载之,云受命所乘。

厥之有章,不必谆谆。依类讬寓,谕以封峦。厥之有章,不必谆谆。依类讬寓,谕以封峦。

披艺观之,天人之际已交,上下相发允答。圣王之德,兢兢翼翼也。故曰“兴必虑衰,安必思危”。是以汤武至尊严,不失肃祗;舜在假典,顾省厥遗:此之谓也。

司马相如既卒五岁,天子始祭后土。八年而遂先礼中岳,封于太山,至梁父禅肃然。

相如他所著,若遗平陵侯书、与五公子相难、草木书篇不采,采其尤著公卿者云。

太史公曰:春秋推见至隐,易本隐之以显,大雅言王公大人而德逮黎庶,小雅讥小己之得失,其流及上。所以言虽外殊,其合德一也。相如虽多虚辞滥说,然其要归引之节俭,此与诗之风谏何异。杨雄以为靡丽之赋,劝百风一,犹驰骋郑卫之声,曲终而奏雅,不已亏乎?余采其语可论者著于篇。

相如纵诞,窃赀卓氏。其学无方,其才足倚。子虚过吒,上林非侈。四马还邛,百金献伎。惜哉封禅,遗文卓尔。
\end{yuanwen}

\chapter{淮南衡山列传}

\begin{yuanwen}
淮南厉王长者,高祖少子也,其母故赵王张敖美人。高祖八年,从东垣过赵,赵王献之美人。厉王母得幸焉,有身。赵王敖弗敢内宫,为筑外宫而舍之。及贯高等谋反柏人事发觉,并逮治王,尽收捕王母兄弟美人,系之河内。厉王母亦系,告吏曰:“得幸上,有身。”吏以闻上,上方怒赵王,未理厉王母。厉王母弟赵兼因辟阳侯言吕后,吕后妒,弗肯白,辟阳侯不彊争。及厉王母已生厉王,恚,即自杀。吏奉厉王诣上,上悔,令吕后母之,而葬厉王母真定。真定,厉王母之家在焉,父世县也。

高祖十一年月,淮南王黥布反,立子长为淮南王,王黥布故地,凡四郡。上自将兵击灭布,厉王遂即位。厉王蚤失母,常附吕后,孝惠、吕后时以故得幸无患害,而常心怨辟阳侯,弗敢发。及孝文帝初即位,淮南王自以为最亲,骄蹇,数不奉法。上以亲故,常宽赦之。三年,入朝。甚横。从上入苑囿猎,与上同车,常谓上“大兄”。厉王有材力,力能扛鼎,乃往请辟阳侯。辟阳侯出见之,即自袖铁椎椎辟阳侯,令从者魏敬刭之。厉王乃驰走阙下,肉袒谢曰:“臣母不当坐赵事,其时辟阳侯力能得之吕后,弗争,罪一也。赵王如意子母无罪,吕后杀之,辟阳侯弗争,罪二也。吕后王诸吕,欲以危刘氏,辟阳侯弗争,罪三也。臣谨为天下诛贼臣辟阳侯,报母之仇,谨伏阙下请罪。”孝文伤其志,为亲故,弗治,赦厉王。当是时,薄太后及太子诸大臣皆惮厉王,厉王以此归国益骄恣,不用汉法,出入称警跸,称制,自为法令,拟于天子。

六年,令男子但等七十人与棘蒲侯柴武太子奇谋,以輂车四十乘反谷口,令人使闽越、匈奴。事觉,治之,使使召淮南王。淮南王至长安。

“丞相臣张仓、典客臣冯敬、行御史大夫事宗正臣逸、廷尉臣贺、备盗贼中尉臣福昧死言:淮南王长废先帝法,不听天子诏,居处无度,为黄屋盖乘舆,出入拟于天子,擅为法令,不用汉法。及所置吏,以其郎中春为丞相,聚收汉诸侯人及有罪亡者,匿与居,为治家室,赐其财物爵禄田宅,爵或至关内侯,奉以二千石,所不当得,欲以有为。大夫但、士五开章等七十人与棘蒲侯太子奇谋反,欲以危宗庙社稷。使开章阴告长,与谋使闽越及匈奴发其兵。开章之淮南见长,长数与坐语饮食,为家室娶妇,以二千石俸奉之。开章使人告但,已言之王。春使使报但等。吏觉知,使长安尉奇等往捕开章。长匿不予,与故中尉蕑忌谋,杀以闭口。为棺椁衣衾,葬之肥陵邑,谩吏曰‘不知安在’ 。又详聚土,树表其上,曰‘开章死,埋此下’ 。及长身自贼杀无罪者一人;令吏论杀无罪者六人;为亡命弃市罪诈捕命者以除罪;擅罪人,罪人无告劾,系治城旦舂以上十四人;赦免罪人,死罪十八人,城旦舂以下五十八人;赐人爵关内侯以下九十四人。前日长病,陛下忧苦之,使使者赐书、枣脯。长不欲受赐,不肯见拜使者。南海民处庐江界中者反,淮南吏卒击之。陛下以淮南民贫苦,遣使者赐长帛五千匹,以赐吏卒劳苦者。长不欲受赐,谩言曰‘无劳苦者’ 。南海民王织上书献璧皇帝,忌擅燔其书,不以闻。吏请召治忌,长不遣,谩言曰‘忌病’ 。春又请长,原入见,长怒曰‘女欲离我自附汉’ 。长当弃市,臣请论如法。”

制曰:“朕不忍致法于王,其与列侯二千石议。”

“臣仓、臣敬、臣逸、臣福、臣贺昧死言:臣谨与列侯吏二千石臣婴等四十三人议,皆曰‘长不奉法度,不听天子诏,乃阴聚徒党及谋反者,厚养亡命,欲以有为’ 。臣等议论如法。”

制曰:“朕不忍致法于王,其赦长死罪,废勿王。”

“臣仓等昧死言:长有大死罪,陛下不忍致法,幸赦,废勿王。臣请处蜀郡严道邛邮,遣其子母从居,县为筑盖家室,皆廪食给薪菜盐豉炊食器席蓐。臣等昧死请,请布告天下。”

制曰:“计食长给肉日五斤,酒二斗。令故美人才人得幸者十人从居。他可。”

尽诛所与谋者。于是乃遣淮南王,载以辎车,令县以次传。是时袁盎谏上曰:“上素骄淮南王,弗为置严傅相,以故至此。且淮南王为人刚,今暴摧折之。臣恐卒逢雾露病死。陛下为有杀弟之名,柰何!”上曰:“吾特苦之耳,今复之。”县传淮南王者皆不敢发车封。淮南王乃谓侍者曰:“谁谓乃公勇者?吾安能勇!吾以骄故不闻吾过至此。人生一世间,安能邑邑如此!”乃不食死。至雍,雍令发封,以死闻。上哭甚悲,谓袁盎曰:“吾不听公言,卒亡淮南王。”盎曰:“不可柰何,原陛下自宽。”上曰:“为之柰何?”盎曰:“独斩丞相、御史以谢天下乃可。”上即令丞相、御史逮考诸县传送淮南王不发封餽侍者,皆弃市。乃以列侯葬淮南王于雍,守冢三十户。

孝文八年,上怜淮南王,淮南王有子四人,皆七八岁,乃封子安为阜陵侯,子勃为安阳侯,子赐为阳周侯,子良为东成侯。

孝文十二年,民有作歌歌淮南厉王曰:“一尺布,尚可缝;一斗粟,尚可舂。兄弟二人不能相容。”上闻之,乃叹曰:“尧舜放逐骨肉,周公杀管蔡,天下称圣。何者?不以私害公。天下岂以我为贪淮南王地邪?”乃徙城阳王王淮南故地,而追尊谥淮南王为厉王,置园复如诸侯仪。

孝文十六年,徙淮南王喜复故城阳。上怜淮南厉王废法不轨,自使失国蚤死,乃立其三子:阜陵侯安为淮南王,安阳侯勃为衡山王,阳周侯赐为庐江王,皆复得厉王时地,参分之。东城侯良前薨,无后也。

孝景三年,吴楚七国反,吴使者至淮南,淮南王欲发兵应之。其相曰:“大王必欲发兵应吴,臣原为将。”王乃属相兵。淮南相已将兵,因城守,不听王而为汉;汉亦使曲城侯将兵救淮南:淮南以故得完。吴使者至庐江,庐江王弗应,而往来使越。吴使者至衡山,衡山王坚守无二心。孝景四年,吴楚已破,衡山王朝,上以为贞信,乃劳苦之曰:“南方卑湿。”徙衡山王王济北,所以襃之。及薨,遂赐谥为贞王。庐江王边越,数使使相交,故徙为衡山王,王江北。淮南王如故。

淮南王安为人好读书鼓琴,不喜弋猎狗马驰骋,亦欲以行阴德拊循百姓,流誉天下。时时怨望厉王死,时欲畔逆,未有因也。及建元二年,淮南王入朝。素善武安侯,武安侯时为太尉,乃逆王霸上,与王语曰:“方今上无太子,大王亲高皇帝孙,行仁义,天下莫不闻。即宫车一日晏驾,非大王当谁立者!”淮南王大喜,厚遗武安侯金财物。阴结宾客,拊循百姓,为畔逆事。建元六年,彗星见,淮南王心怪之。或说王曰:“先吴军起时,彗星出长数尺,然尚流血千里。今彗星长竟天,天下兵当大起。”王心以为上无太子,天下有变,诸侯并争,愈益治器械攻战具,积金钱赂遗郡国诸侯游士奇材。诸辨士为方略者,妄作妖言,谄谀王,王喜,多赐金钱,而谋反滋甚。

淮南王有女陵,慧,有口辩。王爱陵,常多予金钱,为中诇长安,约结上左右。元朔三年,上赐淮南王几杖,不朝。淮南王王后荼,王爱幸之。王后生太子迁,迁取王皇太后外孙修成君女为妃。王谋为反具,畏太子妃知而内泄事,乃与太子谋,令诈弗爱,三月不同席。王乃详为怒太子,闭太子使与妃同内三月,太子终不近妃。妃求去,王乃上书谢归去之。王后荼、太子迁及女陵得爱幸王,擅国权,侵夺民田宅,妄致系人。

元朔五年,太子学用剑,自以为人莫及,闻郎中雷被巧,乃召与戏。被一再辞让,误中太子。太子怒,被恐。此时有欲从军者辄诣京师,被即原奋击匈奴。太子迁数恶被于王,王使郎中令斥免,欲以禁后,被遂亡至长安,上书自明。诏下其事廷尉、河南。河南治,逮淮南太子,王、王后计欲无遣太子,遂发兵反,计犹豫,十馀日未定。会有诏,即讯太子。当是时,淮南相怒寿春丞留太子逮不遣,劾不敬。王以请相,相弗听。王使人上书告相,事下廷尉治。踪迹连王,王使人候伺汉公卿,公卿请逮捕治王。王恐事发,太子迁谋曰:“汉使即逮王,王令人衣卫士衣,持戟居庭中,王旁有非是,则刺杀之,臣亦使人刺杀淮南中尉,乃举兵,未晚。”是时上不许公卿请,而遣汉中尉宏即讯验王。王闻汉使来,即如太子谋计。汉中尉至,王视其颜色和,讯王以斥雷被事耳,王自度无何,不发。中尉还,以闻。公卿治者曰:“淮南王安拥阏奋击匈奴者雷被等,废格明诏,当弃市。”诏弗许。公卿请废勿王,诏弗许。公卿请削五县,诏削二县。使中尉宏赦淮南王罪,罚以削地。中尉入淮南界,宣言赦王。王初闻汉公卿请诛之,未知得削地,闻汉使来,恐其捕之,乃与太子谋刺之如前计。及中尉至,即贺王,王以故不发。其后自伤曰:“吾行仁义见削,甚耻之。”然淮南王削地之后,其为反谋益甚。诸使道从长安来,为妄妖言,言上无男,汉不治,即喜;即言汉廷治,有男,王怒,以为妄言,非也。

王日夜与伍被、左吴等案舆地图,部署兵所从入。王曰:“上无太子,宫车即晏驾,廷臣必徵胶东王,不即常山王,诸侯并争,吾可以无备乎!且吾高祖孙,亲行仁义,陛下遇我厚,吾能忍之;万世之后,吾宁能北面臣事竖子乎!”

王坐东宫,召伍被与谋,曰:“将军上。”被怅然曰:“上宽赦大王,王复安得此亡国之语乎!臣闻子胥谏吴王,吴王不用,乃曰‘臣今见麋鹿游姑苏之台也’ 。今臣亦见宫中生荆棘,露霑衣也。”王怒,系伍被父母,囚之三月。复召曰:“将军许寡人乎?”被曰:“不,直来为大王画耳。臣闻聪者听于无声,明者见于未形,故圣人万举万全。昔文王一动而功显于千世,列为三代,此所谓因天心以动作者也,故海内不期而随。此千岁之可见者。夫百年之秦,近世之吴楚,亦足以喻国家之存亡矣。臣不敢避子胥之诛,原大王毋为吴王之听。昔秦绝圣人之道,杀术士,燔诗书,弃礼义,尚诈力,任刑罚,转负海之粟致之西河。当是之时,男子疾耕不足于糟,女子纺绩不足于盖形。遣蒙恬筑长城,东西数千里,暴兵露师常数十万,死者不可胜数,僵尸千里,流血顷亩,百姓力竭,欲为乱者十家而五。又使徐福入海求神异物,还为伪辞曰:‘臣见海中大神,言曰:“汝西皇之使邪?”臣答曰:“然。”“汝何求?”曰:“原请延年益寿药。”神曰:“汝秦王之礼薄,得观而不得取。”即从臣东南至蓬莱山,见芝成宫阙,有使者铜色而龙形,光上照天。于是臣再拜问曰:“宜何资以献?”海神曰:“以令名男子若振女与百工之事,即得之矣。”’ 秦皇帝大说,遣振男女三千人,资之五穀种种百工而行。徐福得平原广泽,止王不来。于是百姓悲痛相思,欲为乱者十家而六。又使尉佗逾五岭攻百越。尉佗知中国劳极,止王不来,使人上书,求女无夫家者三万人,以为士卒衣补。秦皇帝可其万五千人。于是百姓离心瓦解,欲为乱者十家而七。客谓高皇帝曰:‘时可矣。’ 高皇帝曰:‘待之,圣人当起东南间。’ 不一年,陈胜吴广发矣。高皇始于丰沛,一倡天下不期而响应者不可胜数也。此所谓蹈瑕候间,因秦之亡而动者也。百姓原之,若旱之望雨,故起于行陈之中而立为天子,功高三王,德传无穷。今大王见高皇帝得天下之易也,独不观近世之吴楚乎?夫吴王赐号为刘氏祭酒,复不朝,王四郡之众,地方数千里,内铸消铜以为钱,东煮海水以为盐,上取江陵木以为船,一船之载当中国数十两车,国富民众。行珠玉金帛赂诸侯宗室大臣,独窦氏不与。计定谋成,举兵而西。破于大梁,败于狐父,奔走而东,至于丹徒,越人禽之,身死绝祀,为天下笑。夫以吴越之众不能成功者何?诚逆天道而不知时也。方今大王之兵众不能十分吴楚之一,天下安宁有万倍于秦之时,原大王从臣之计。大王不从臣之计,今见大王事必不成而语先泄也。臣闻微子过故国而悲,于是作麦秀之歌,是痛纣之不用王子比干也。故孟子曰‘纣贵为天子,死曾不若匹夫’ 。是纣先自绝于天下久矣,非死之日而天下去之。今臣亦窃悲大王弃千乘之君,必且赐绝命之书,为群臣先,死于东宫也。”于是气怨结而不扬,涕满匡而横流,即起,历阶而去。

王有孽子不害,最长,王弗爱,王、王后、太子皆不以为子兄数。不害有子建,材高有气,常怨望太子不省其父;又怨时诸侯皆得分子弟为侯,而淮南独二子,一为太子,建父独不得为侯。建阴结交,欲告败太子,以其父代之。太子知之,数捕系而榜笞建。建具知太子之谋欲杀汉中尉,即使所善寿春庄芷以元朔六年上书于天子曰:“毒药苦于口利于病,忠言逆于耳利于行。今淮南王孙建,材能高,淮南王王后荼、荼子太子迁常疾害建。建父不害无罪,擅数捕系,欲杀之。今建在,可徵问,具知淮南阴事。”书闻,上以其事下廷尉,廷尉下河南治。是时故辟阳侯孙审卿善丞相公孙弘,怨淮南厉王杀其大父,乃深购淮南事于弘,弘乃疑淮南有畔逆计谋,深穷治其狱。河南治建,辞引淮南太子及党与。淮南王患之,欲发,问伍被曰:“汉廷治乱?”伍被曰:“天下治。”王意不说,谓伍被曰:“公何以言天下治也?”被曰:“被窃观朝廷之政,君臣之义,父子之亲,夫妇之别,长幼之序,皆得其理,上之举错遵古之道,风俗纪纲未有所缺也。重装富贾,周流天下,道无不通,故交易之道行。南越宾服,羌僰入献,东瓯入降,广长榆,开朔方,匈奴折翅伤翼,失援不振。虽未及古太平之时,然犹为治也。”王怒,被谢死罪。王又谓被曰:“山东即有兵,汉必使大将军将而制山东,公以为大将军何如人也?”被曰:“被所善者黄义,从大将军击匈奴,还,告被曰:‘大将军遇士大夫有礼,于士卒有恩,众皆乐为之用。骑上下山若蜚,材幹绝人。’ 被以为材能如此,数将习兵,未易当也。及谒者曹梁使长安来,言大将军号令明,当敌勇敢,常为士卒先。休舍,穿井未通,须士卒尽得水,乃敢饮。军罢,卒尽已度河,乃度。皇太后所赐金帛,尽以赐军吏。虽古名将弗过也。”王默然。

淮南王见建已徵治,恐国阴事且觉,欲发,被又以为难,乃复问被曰:“公以为吴兴兵是邪非也?”被曰:“以为非也。吴王至富贵也,举事不当,身死丹徒,头足异处,子孙无遗类。臣闻吴王悔之甚。原王孰虑之,无为吴王之所悔。”王曰:“男子之所死者一言耳。且吴何知反,汉将一日过成皋者四十馀人。今我令楼缓先要成皋之口,周被下颍川兵塞轘辕、伊阙之道,陈定发南阳兵守武关。河南太守独有雒阳耳,何足忧。然此北尚有临晋关、河东、上党与河内、赵国。人言曰‘绝成皋之口,天下不通’ 。据三川之险,招山东之兵,举事如此,公以为何如?”被曰:“臣见其祸,未见其福也。”王曰:“左吴、赵贤、硃骄如皆以为有福,什事九成,公独以为有祸无福,何也?”被曰:“大王之群臣近幸素能使众者,皆前系诏狱,馀无可用者。”王曰:“陈胜、吴广无立锥之地,千人之聚,起于大泽,奋臂大呼而天下响应,西至于戏而兵百二十万。今吾国虽小,然而胜兵者可得十馀万,非直適戍之众,釠凿棘矜也,公何以言有祸无福?”被曰:“往者秦为无道,残贼天下。兴万乘之驾,作阿房之宫,收太半之赋,发闾左之戍,父不宁子,兄不便弟,政苛刑峻,天下熬然若焦,民皆引领而望,倾耳而听,悲号仰天,叩心而怨上,故陈胜大呼,天下响应。当今陛下临制天下,一齐海内,汎爱蒸庶,布德施惠。口虽未言,声疾雷霆,令虽未出,化驰如神,心有所怀,威动万里,下之应上,犹影响也。而大将军材能不特章邯、杨熊也。大王以陈胜、吴广谕之,被以为过矣。”王曰:“苟如公言,不可徼幸邪?”被曰:“被有愚计。”王曰:“柰何?”被曰:“当今诸侯无异心,百姓无怨气。朔方之郡田地广,水草美,民徙者不足以实其地。臣之愚计,可伪为丞相御史请书,徙郡国豪桀任侠及有耐罪以上,赦令除其罪,产五十万以上者,皆徙其家属朔方之郡,益发甲卒,急其会日。又伪为左右都司空上林中都官诏狱书,诸侯太子幸臣。如此则民怨,诸侯惧,即使辩武随而说之,傥可徼幸什得一乎?”王曰:“此可也。虽然,吾以为不至若此。”于是王乃令官奴入宫,作皇帝玺,丞相、御史、大将军、军吏、中二千石、都官令、丞印,及旁近郡太守、都尉印,汉使节法冠,欲如伍被计。使人伪得罪而西,事大将军、丞相;一日发兵,使人即刺杀大将军青,而说丞相下之,如发蒙耳。

王欲发国中兵,恐其相、二千石不听。王乃与伍被谋,先杀相、二千石;伪失火宫中,相、二千石救火,至即杀之。计未决,又欲令人衣求盗衣,持羽檄,从东方来,呼曰“南越兵入界”,欲因以发兵。乃使人至庐江、会稽为求盗,未发。王问伍被曰:“吾举兵西乡,诸侯必有应我者;即无应,柰何?”被曰:“南收衡山以击庐江,有寻阳之船,守下雉之城,结九江之浦,绝豫章之口,彊弩临江而守,以禁南郡之下,东收江都、会稽,南通劲越,屈彊江淮间,犹可得延岁月之寿。”王曰:“善,无以易此。急则走越耳。”

于是廷尉以王孙建辞连淮南王太子迁闻。上遣廷尉监因拜淮南中尉,逮捕太子。至淮南,淮南王闻,与太子谋召相、二千石,欲杀而发兵。召相,相至;内史以出为解。中尉曰:“臣受诏使,不得见王。”王念独杀相而内史中尉不来,无益也,即罢相。王犹豫,计未决。太子念所坐者谋刺汉中尉,所与谋者已死,以为口绝,乃谓王曰:“群臣可用者皆前系,今无足与举事者。王以非时发,恐无功,臣原会逮。”王亦偷欲休,即许太子。太子即自刭,不殊。伍被自诣吏,因告与淮南王谋反,反踪迹具如此。

吏因捕太子、王后,围王宫,尽求捕王所与谋反宾客在国中者,索得反具以闻。上下公卿治,所连引与淮南王谋反列侯二千石豪杰数千人,皆以罪轻重受诛。衡山王赐,淮南王弟也,当坐收,有司请逮捕衡山王。天子曰:“诸侯各以其国为本,不当相坐。与诸侯王列侯会肄丞相诸侯议。”赵王彭祖、列侯臣让等四十三人议,皆曰:“淮南王安甚大逆无道,谋反明白,当伏诛。”胶西王臣端议曰:“淮南王安废法行邪,怀诈伪心,以乱天下,荧惑百姓,倍畔宗庙,妄作妖言。春秋曰‘臣无将,将而诛’ 。安罪重于将,谋反形已定。臣端所见其书节印图及他逆无道事验明白,甚大逆无道,当伏其法。而论国吏二百石以上及比者,宗室近幸臣不在法中者,不能相教,当皆免官削爵为士伍,毋得宦为吏。其非吏,他赎死金二斤八两。以章臣安之罪,使天下明知臣子之道,毋敢复有邪僻倍畔之意。”丞相弘、廷尉汤等以闻,天子使宗正以符节治王。未至,淮南王安自刭杀。王后荼、太子迁诸所与谋反者皆族。天子以伍被雅辞多引汉之美,欲勿诛。廷尉汤曰:“被首为王画反谋,被罪无赦。”遂诛被。国除为九江郡。

衡山王赐,王后乘舒生子三人,长男爽为太子,次男孝,次女无采。又姬徐来生子男女四人,美人厥姬生子二人。衡山王、淮南王兄弟相责望礼节,间不相能。衡山王闻淮南王作为畔逆反具,亦心结宾客以应之,恐为所并。

元光六年,衡山王入朝,其谒者卫庆有方术,欲上书事天子,王怒,故劾庆死罪,彊榜服之。衡山内史以为非是,卻其狱。王使人上书告内史,内史治,言王不直。王又数侵夺人田,坏人冢以为田。有司请逮治衡山王。天子不许,为置吏二百石以上。衡山王以此恚,与奚慈、张广昌谋,求能为兵法候星气者,日夜从容王密谋反事。

王后乘舒死,立徐来为王后。厥姬俱幸。两人相妒,厥姬乃恶王后徐来于太子曰:“徐来使婢蛊道杀太子母。”太子心怨徐来。徐来兄至衡山,太子与饮,以刃刺伤王后兄。王后怨怒,数毁恶太子于王。太子女弟无采,嫁弃归,与奴奸,又与客奸。太子数让无采,无采怒,不与太子通。王后闻之,即善遇无采。无采及中兄孝少失母,附王后,王后以计爱之,与共毁太子,王以故数击笞太子。元朔四年中,人有贼伤王后假母者,王疑太子使人伤之,笞太子。后王病,太子时称病不侍。孝、王后、无采恶太子:“太子实不病,自言病,有喜色。”王大怒,欲废太子,立其弟孝。王后知王决废太子,又欲并废孝。王后有侍者,善舞,王幸之,王后欲令侍者与孝乱以汙之,欲并废兄弟而立其子广代太子。太子爽知之,念后数恶己无已时,欲与乱以止其口。王后饮,太子前为寿,因据王后股,求与王后卧。王后怒,以告王。王乃召,欲缚而笞之。太子知王常欲废己立其弟孝,乃谓王曰:“孝与王御者奸,无采与奴奸,王彊食,请上书。”即倍王去。王使人止之,莫能禁,乃自驾追捕太子。太子妄恶言,王械系太子宫中。孝日益亲幸。王奇孝材能,乃佩之王印,号曰将军,令居外宅,多给金钱,招致宾客。宾客来者,微知淮南、衡山有逆计,日夜从容劝之。王乃使孝客江都人救赫、陈喜作輣车镞矢,刻天子玺,将相军吏印。王日夜求壮士如周丘等,数称引吴楚反时计画,以约束。衡山王非敢效淮南王求即天子位,畏淮南起并其国,以为淮南已西,发兵定江淮之间而有之,望如是。

元朔五年秋,衡山王当朝,过淮南,淮南王乃昆弟语,除前卻,约束反具。衡山王即上书谢病,上赐书不朝。

元朔六年中,衡山王使人上书请废太子爽,立孝为太子。爽闻,即使所善白嬴之长安上书,言孝作輣车镞矢,与王御者奸,欲以败孝。白嬴至长安,未及上书,吏捕嬴,以淮南事系。王闻爽使白嬴上书,恐言国阴事,即上书反告太子爽所为不道弃市罪事。事下沛郡治。元年冬,有司公卿下沛郡求捕所与淮南谋反者未得,得陈喜于衡山王子孝家。吏劾孝首匿喜。孝以为陈喜雅数与王计谋反,恐其发之,闻律先自告除其罪,又疑太子使白嬴上书发其事,即先自告,告所与谋反者救赫、陈喜等。廷尉治验,公卿请逮捕衡山王治之。天子曰:“勿捕。”遣中尉安、大行息即问王,王具以情实对。吏皆围王宫而守之。中尉大行还,以闻,公卿请遣宗正、大行与沛郡杂治王。王闻,即自刭杀。孝先自告反,除其罪;坐与王御婢奸,弃市。王后徐来亦坐蛊杀前王后乘舒,及太子爽坐王告不孝,皆弃市。诸与衡山王谋反者皆族。国除为衡山郡。

太史公曰:诗之所谓“戎狄是膺,荆舒是惩”,信哉是言也。淮南、衡山亲为骨肉,疆土千里,列为诸侯,不务遵蕃臣职以承辅天子,而专挟邪僻之计,谋为畔逆,仍父子再亡国,各不终其身,为天下笑。此非独王过也,亦其俗薄,臣下渐靡使然也。夫荆楚僄勇轻悍,好作乱,乃自古记之矣。

淮南多横,举事非正。天子宽仁,其过不更。轞车致祸,斗粟成咏。王安好学,女陵作诇。兄弟不和,倾国殒命。
\end{yuanwen}

\chapter{循吏列传}

\begin{yuanwen}
太史公曰:法令所以导民也,刑罚所以禁奸也。文武不备,良民惧然身修者,官未曾乱也。奉职循理,亦可以为治,何必威严哉?

孙叔敖者,楚之处士也。虞丘相进之于楚庄王,以自代也。三月为楚相,施教导民,上下和合,世俗盛美,政缓禁止,吏无奸邪,盗贼不起。秋冬则劝民山采,春夏以水,各得其所便,民皆乐其生。

庄王以为币轻,更以小为大,百姓不便,皆去其业。市令言之相曰:“市乱,民莫安其处,次行不定。”相曰:“如此几何顷乎?”市令曰:“三月顷。”相曰:“罢,吾今令之复矣。”后五日,朝,相言之王曰:“前日更币,以为轻。今市令来言曰“市乱,民莫安其处,次行之不定”。臣请遂令复如故。”王许之,下令三日而市复如故。

楚民俗好庳车,王以为庳车不便马,欲下令使高之。相曰:“令数下,民不知所从,不可。王必欲高车,臣请教闾里使高其困。乘车者皆君子,君子不能数下车。”王许之。居半岁,民悉自高其车。

此不教而民从其化,近者视而效之,远者四面望而法之。故三得相而不喜,知其材自得之也;三去相而不悔,知非己之罪也。

子产者,郑之列大夫也。郑昭君之时,以所爱徐挚为相,国乱,上下不亲,父子不和。大宫子期言之君,以子产为相。为相一年,竖子不戏狎,斑白不提挈,僮子不犁畔。二年,市不豫贾。三年,门不夜关,道不拾遗。四年,田器不归。五年,士无尺籍,丧期不令而治。治郑二十六年而死,丁壮号哭,老人兒啼,曰:“子产去我死乎!民将安归?”

公仪休者,鲁博士也。以高弟为鲁相。奉法循理,无所变更,百官自正。使食禄者不得与下民争利,受大者不得取小。

客有遗相鱼者,相不受。客曰:“闻君嗜鱼,遗君鱼,何故不受也?”相曰:“以嗜鱼,故不受也。今为相,能自给鱼;今受鱼而免,谁复给我鱼者?吾故不受也。”

食茹而美,拔其园葵而弃之。见其家织布好,而疾出其家妇,燔其机,云“欲令农士工女安所雠其货乎”?

石奢者,楚昭王相也。坚直廉正,无所阿避。行县,道有杀人者,相追之,乃其父也。纵其父而还自系焉。使人言之王曰:“杀人者,臣之父也。夫以父立政,不孝也;废法纵罪,非忠也;臣罪当死。”王曰:“追而不及,不当伏罪,子其治事矣。”石奢曰:“不私其父,非孝子也;不奉主法,非忠臣也。王赦其罪,上惠也;伏诛而死,臣职也。”遂不受令,自刎而死。

李离者,晋文公之理也。过听杀人,自拘当死。文公曰:“官有贵贱,罚有轻重。下吏有过,非子之罪也。”李离曰:“臣居官为长,不与吏让位;受禄为多,不与下分利。今过听杀人,傅其罪下吏,非所闻也。”辞不受令。文公曰:“子则自以为有罪,寡人亦有罪邪?”李离曰:“理有法,失刑则刑,失死则死。公以臣能听微决疑,故使为理。今过听杀人,罪当死。”遂不受令,伏剑而死。

太史公曰:孙叔敖出一言,郢市复。子产病死,郑民号哭。公仪子见好布而家妇逐。石奢纵父而死,楚昭名立。李离过杀而伏剑,晋文以正国法。

奉职循理,为政之先。恤人体国,良史述焉。叔孙、郑产,自昔称贤。拔葵一利,赦父非。李离伏剑,为法而然。
\end{yuanwen}

\chapter{汲郑列传}

\begin{yuanwen}
汲黯字长孺,濮阳人也。其先有宠于古之卫君。至黯七世,世为卿大夫。黯以父任,孝景时为太子洗马,以庄见惮。孝景帝崩,太子即位,黯为谒者。东越相攻,上使黯往视之。不至,至吴而还,报曰:“越人相攻,固其俗然,不足以辱天子之使。”河内失火,延烧千馀家,上使黯往视之。还报曰:“家人失火,屋比延烧,不足忧也。臣过河南,河南贫人伤水旱万馀家,或父子相食,臣谨以便宜,持节发河南仓粟以振贫民。臣请归节,伏矫制之罪。”上贤而释之,迁为荥阳令。黯耻为令,病归田里。上闻,乃召拜为中大夫。以数切谏,不得久留内,迁为东海太守。黯学黄老之言,治官理民,好清静,择丞史而任之。其治,责大指而已,不苛小。黯多病,卧闺閤内不出。岁馀,东海大治。称之。上闻,召以为主爵都尉,列于九卿。治务在无为而已,弘大体,不拘文法。

黯为人性倨,少礼,面折,不能容人之过。合己者善待之,不合己者不能忍见,士亦以此不附焉。然好学,游侠,任气节,内行脩絜,好直谏,数犯主之颜色,常慕傅柏、袁盎之为人也。善灌夫、郑当时及宗正刘弃。亦以数直谏,不得久居位。

当是时,太后弟武安侯蚡为丞相,中二千石来拜谒,蚡不为礼。然黯见蚡未尝拜,常揖之。天子方招文学儒者,上曰吾欲云云,黯对曰:“陛下内多欲而外施仁义,柰何欲效唐虞之治乎!”上默然,怒,变色而罢朝。公卿皆为黯惧。上退,谓左右曰:“甚矣,汲黯之戆也!”群臣或数黯,黯曰:“天子置公卿辅弼之臣,宁令从谀承意,陷主于不义乎?且已在其位,纵爱身,柰辱朝廷何!”

黯多病,病且满三月,上常赐告者数,终不愈。最后病,庄助为请告。上曰:“汲黯何如人哉?”助曰:“使黯任职居官,无以逾人。然至其辅少主,守城深坚,招之不来,麾之不去,虽自谓贲育亦不能夺之矣。”上曰:“然。古有社稷之臣,至如黯,近之矣。”

大将军青侍中,上踞厕而视之。丞相弘燕见,上或时不冠。至如黯见,上不冠不见也。上尝坐武帐中,黯前奏事,上不冠,望见黯,避帐中,使人可其奏。其见敬礼如此。

张汤方以更定律令为廷尉,黯数质责汤于上前,曰:“公为正卿,上不能襃先帝之功业,下不能抑天下之邪心,安国富民,使囹圄空虚,二者无一焉。非苦就行,放析就功,何乃取高皇帝约束纷更之为?公以此无种矣。”黯时与汤论议,汤辩常在文深小苛,黯伉厉守高不能屈,忿发骂曰:“天下谓刀笔吏不可以为公卿,果然。必汤也,令天下重足而立,侧目而视矣!”

是时,汉方征匈奴,招怀四夷。黯务少事,乘上间,常言与胡和亲,无起兵。上方向儒术,尊公孙弘。及事益多,吏民巧弄。上分别文法,汤等数奏决谳以幸。而黯常毁儒,面触弘等徒怀诈饰智以阿人主取容,而刀笔吏专深文巧诋,陷人于罪,使不得反其真,以胜为功。上愈益贵弘、汤,弘、汤深心疾黯,唯天子亦不说也,欲诛之以事。弘为丞相,乃言上曰:“右内史界部中多贵人宗室,难治,非素重臣不能任,请徙黯为右内史。”为右内史数岁,官事不废。

大将军青既益尊,姊为皇后,然黯与亢礼。人或说黯曰:“自天子欲群臣下大将军,大将军尊重益贵,君不可以不拜。”黯曰:“夫以大将军有揖客,反不重邪?”大将军闻,愈贤黯,数请问国家朝廷所疑,遇黯过于平生。

淮南王谋反,惮黯,曰:“好直谏,守节死义,难惑以非。至如说丞相弘,如发蒙振落耳。”

天子既数征匈奴有功,黯之言益不用。

始黯列为九卿,而公孙弘、张汤为小吏。及弘、汤稍益贵,与黯同位,黯又非毁弘、汤等。已而弘至丞相,封为侯;汤至御史大夫;故黯时丞相史皆与黯同列,或尊用过之。黯褊心,不能无少望,见上,前言曰:“陛下用群臣如积薪耳,后来者居上。”上默然。有间黯罢,上曰:“人果不可以无学,观黯之言也日益甚。”

居无何,匈奴浑邪王率众来降,汉发车二万乘。县官无钱,从民贳马。民或匿马,马不具。上怒,欲斩长安令。黯曰:“长安令无罪,独斩黯,民乃肯出马。且匈奴畔其主而降汉,汉徐以县次传之,何至令天下骚动,罢弊中国而以事夷狄之人乎!”上默然。及浑邪至,贾人与市者,坐当死者五百馀人。黯请间,见高门,曰:“夫匈奴攻当路塞,绝和亲,中国兴兵诛之,死伤者不可胜计,而费以巨万百数。臣愚以为陛下得胡人,皆以为奴婢以赐从军死事者家;所卤获,因予之,以谢天下之苦,塞百姓之心。今纵不能,浑邪率数万之众来降,虚府库赏赐,发良民侍养,譬若奉骄子。愚民安知市买长安中物而文吏绳以为阑出财物于边关乎?陛下纵不能得匈奴之资以谢天下,又以微文杀无知者五百馀人,是所谓‘庇其叶而伤其枝’者也,臣窃为陛下不取也。”上默然,不许,曰:“吾久不闻汲黯之言,今又复妄发矣。”后数月,黯坐小法,会赦免官。于是黯隐于田园。

居数年,会更五铢钱,民多盗铸钱,楚地尤甚。上以为淮阳,楚地之郊,乃召拜黯为淮阳太守。黯伏谢不受印,诏数彊予,然后奉诏。诏召见黯,黯为上泣曰:“臣自以为填沟壑,不复见陛下,不意陛下复收用之。臣常有狗马病,力不能任郡事,臣原为中郎,出入禁闼,补过拾遗,臣之原也。”上曰:“君薄淮阳邪?吾今召君矣。顾淮阳吏民不相得,吾徒得君之重,卧而治之。”黯既辞行,过大行李息,曰:“黯弃居郡,不得与朝廷议也。然御史大夫张汤智足以拒谏,言足以饰非,务巧佞之语,辩数之辞,非肯正为天下言,专阿主意。主意所不欲,因而毁之;主意所欲,因而誉之。好兴事,舞文法,内怀诈以御主心,外挟贼吏以为威重。公列九卿,不早言之,公与之俱受其僇矣。”息畏汤,终不敢言。黯居郡如故治,淮阳政清。后张汤果败,上闻黯与息言,抵息罪。令黯以诸侯相秩居淮阳。七岁而卒。

卒后,上以黯故,官其弟汲仁至九卿,子汲偃至诸侯相。黯姑姊子司马安亦少与黯为太子洗马。安文深巧善宦,官四至九卿,以河南太守卒。昆弟以安故,同时至二千石者十人。濮阳段宏始事盖侯信,信任宏,宏亦再至九卿。然卫人仕者皆严惮汲黯,出其下。

郑当时者,字庄,陈人也。其先郑君尝为项籍将;籍死,已而属汉。高祖令诸故项籍臣名籍,郑君独不奉诏。诏尽拜名籍者为大夫,而逐郑君。郑君死孝文时。

郑庄以任侠自喜,脱张羽于戹,声闻梁楚之间。孝景时,为太子舍人。每五日洗沐,常置驿马安诸郊,存诸故人,请谢宾客,夜以继日,至其明旦,常恐不遍。庄好黄老之言,其慕长者如恐不见。年少官薄,然其游知交皆其大父行,天下有名之士也。武帝立,庄稍迁为鲁中尉、济南太守、江都相,至九卿为右内史。以武安侯魏其时议,贬秩为詹事,迁为大农令。

庄为太史,诫门下:“客至,无贵贱无留门者。”执宾主之礼,以其贵下人。庄廉,又不治其产业,仰奉赐以给诸公。然其餽遗人,不过算器食。每朝,候上之间,说未尝不言天下之长者。其推毂士及官属丞史,诚有味其言之也,常引以为贤于己。未尝名吏,与官属言,若恐伤之。闻人之善言,进之上,唯恐后。山东士诸公以此翕然称郑庄。

郑庄使视决河,自请治行五日。上曰:“吾闻‘郑庄行,千里不赍粮’,请治行者何也?”然郑庄在朝,常趋和承意,不敢甚引当否。及晚节,汉征匈奴,招四夷,天下费多,财用益匮。庄任人宾客为大农僦人,多逋负。司马安为淮阳太守,发其事,庄以此陷罪,赎为庶人。顷之,守长史。上以为老,以庄为汝南太守。数岁,以官卒。

郑庄、汲黯始列为九卿,廉,内行脩絜。此两人中废,家贫,宾客益落。及居郡,卒后家无馀赀财。庄兄弟子孙以庄故,至二千石六七人焉。

太史公曰:夫以汲、郑之贤,有势则宾客十倍,无势则否,况众人乎!下邽翟公有言,始翟公为廷尉,宾客阗门;及废,门外可设雀罗。翟公复为廷尉,宾客欲往,翟公乃人署其门曰:“一死一生,乃知交情。一贫一富,乃知交态。一贵一贱,交情乃见。”汲、郑亦云,悲夫!

河南矫制,自古称贤。淮南卧理,天子伏焉。积薪兴叹,伉直愈坚。郑庄推士,天下翕然。交道势利,翟公怆旃。
\end{yuanwen}

\chapter{儒林列传}

\begin{yuanwen}
太史公曰:余读功令,至于广厉学官之路,未尝不废书而叹也。曰:嗟乎!夫周室衰而关雎作,幽厉微而礼乐坏,诸侯恣行,政由彊国。故孔子闵王路废而邪道兴,于是论次诗书,修起礼乐。適齐闻韶,三月不知肉味。自卫返鲁,然后乐正,雅颂各得其所。世以混浊莫能用,是以仲尼干七十馀君无所遇,曰“苟有用我者,期月而已矣”。西狩获麟,曰“吾道穷矣”。故因史记作春秋,以当王法,其辞微而指博,后世学者多录焉。
\end{yuanwen}\begin{yuanwen}
	
\end{yuanwen}\begin{yuanwen}
	
\end{yuanwen}\begin{yuanwen}
	
\end{yuanwen}\begin{yuanwen}
	
\end{yuanwen}\begin{yuanwen}
	
\end{yuanwen}\begin{yuanwen}
	
\end{yuanwen}\begin{yuanwen}
	
\end{yuanwen}\begin{yuanwen}
	
\end{yuanwen}\begin{yuanwen}
	
\end{yuanwen}\begin{yuanwen}
	
\end{yuanwen}\begin{yuanwen}
	
\end{yuanwen}\begin{yuanwen}
	
\end{yuanwen}\begin{yuanwen}
	
\end{yuanwen}\begin{yuanwen}
	
\end{yuanwen}\begin{yuanwen}
	
\end{yuanwen}\begin{yuanwen}
	
\end{yuanwen}\begin{yuanwen}
	
\end{yuanwen}\begin{yuanwen}
	
\end{yuanwen}\begin{yuanwen}
	
\end{yuanwen}\begin{yuanwen}
	
\end{yuanwen}\begin{yuanwen}
	
\end{yuanwen}\begin{yuanwen}
	
\end{yuanwen}\begin{yuanwen}
	
\end{yuanwen}\begin{yuanwen}
	
\end{yuanwen}\begin{yuanwen}
	
\end{yuanwen}\begin{yuanwen}
	
\end{yuanwen}\begin{yuanwen}
	
\end{yuanwen}\begin{yuanwen}
	
\end{yuanwen}\begin{yuanwen}
	
\end{yuanwen}\begin{yuanwen}
	
\end{yuanwen}\begin{yuanwen}
	
\end{yuanwen}\begin{yuanwen}
	
\end{yuanwen}\begin{yuanwen}
	
\end{yuanwen}\begin{yuanwen}
	
\end{yuanwen}\begin{yuanwen}
	
\end{yuanwen}\begin{yuanwen}
	
\end{yuanwen}\begin{yuanwen}
	
\end{yuanwen}\begin{yuanwen}
	
\end{yuanwen}\begin{yuanwen}
	
\end{yuanwen}\begin{yuanwen}
	
\end{yuanwen}\begin{yuanwen}
	
\end{yuanwen}\begin{yuanwen}
	
\end{yuanwen}\begin{yuanwen}
	
\end{yuanwen}\begin{yuanwen}
	
\end{yuanwen}\begin{yuanwen}
	
\end{yuanwen}\begin{yuanwen}
	
\end{yuanwen}\begin{yuanwen}
	
\end{yuanwen}\begin{yuanwen}
	
\end{yuanwen}\begin{yuanwen}
	
\end{yuanwen}\begin{yuanwen}
自孔子卒后,七十子之徒散游诸侯,大者为师傅卿相,小者友教士大夫,或隐而不见。故子路居卫,子张居陈,澹台子羽居楚,子夏居西河,子贡终于齐。如田子方、段干木、吴起、禽滑釐之属,皆受业于子夏之伦,为王者师。是时独魏文侯好学。后陵迟以至于始皇,天下并争于战国,懦术既绌焉,然齐鲁之间,学者独不废也。于威、宣之际,孟子、荀卿之列,咸遵夫子之业而润色之,以学显于当世。

及至秦之季世,焚诗书,阬术士,六从此缺焉。陈涉之王也,而鲁诸儒持孔氏之礼器往归陈王。于是孔甲为陈涉博士,卒与涉俱死。陈涉起匹夫,驱瓦合適戍,旬月以王楚,不满半岁竟灭亡,其事至微浅,然而缙绅先生之徒负孔子礼器往委质为臣者,何也?以秦焚其业,积怨而发愤于陈王也。

及高皇帝诛项籍,举兵围鲁,鲁中诸儒尚讲诵习礼乐,弦歌之音不绝,岂非圣人之遗化,好礼乐之国哉?故孔子在陈,曰“归与归与!吾党之小子狂简,斐然成章,不知所以裁之”。夫齐鲁之间于文学,自古以来,其天性也。故汉兴,然后诸儒始得脩其经,讲习大射乡饮之礼。叔孙通作汉礼仪,因为太常,诸生弟子共定者,咸为选首,于是喟然叹兴于学。然尚有干戈,平定四海,亦未暇遑庠序之事也。孝惠、吕后时,公卿皆武力有功之臣。孝文时颇徵用,然孝文帝本好刑名之言。及至孝景,不任儒者,而窦太后又好黄老之术,故诸博士具官待问,未有进者。

及今上即位,赵绾、王臧之属明儒学,而上亦乡之,于是招方正贤良文学之士。自是之后,言诗于鲁则申培公,于齐则辕固生,于燕则韩太傅。言尚书自济南伏生。言礼自鲁高堂生。言易自菑川田生。言春秋于齐鲁自胡毋生,于赵自董仲舒。及窦太后崩,武安侯田蚡为丞相,绌黄老、刑名百家之言,延文学儒者数百人,而公孙弘以春秋白衣为天子三公,封以平津侯。天下之学士靡然乡风矣。

公孙弘为学官,悼道之郁滞,乃请曰:“丞相御史言:制曰‘盖闻导民以礼,风之以乐。婚姻者,居屋之大伦也。今礼废乐崩,朕甚愍焉。故详延天下方正博闻之士,咸登诸朝。其令礼官劝学,讲议洽闻兴礼,以为天下先。太常议,与博士弟子,崇乡里之化,以广贤材焉’ 。谨与太常臧、博士平等议曰:闻三代之道,乡里有教,夏曰校,殷曰序,周曰庠。其劝善也,显之朝廷;其惩恶也,加之刑罚。故教化之行也,建首善自京师始,由内及外。今陛下昭至德,开大明,配天地,本人伦,劝学脩礼,崇化厉贤,以风四方,太平之原也。古者政教未洽,不备其礼,请因旧官而兴焉。为博士官置弟子五十人,复其身。太常择民年十八已上,仪状端正者,补博士弟子。郡国县道邑有好文学,敬长上,肃政教,顺乡里,出入不悖所闻者,令相长丞上属所二千石,二千石谨察可者,当与计偕,诣太常,得受业如弟子。一岁皆辄试,能通一以上,补文学掌故缺;其高弟可以为郎中者,太常籍奏。即有秀才异等,辄以名闻。其不事学若下材及不能通一,辄罢之,而请诸不称者罚。臣谨案诏书律令下者,明天人分际,通古今之义,文章尔雅,训辞深厚,恩施甚美。小吏浅闻,不能究宣,无以明布谕下。治礼次治掌故,以文学礼义为官,迁留滞。请选择其秩比二百石以上,及吏百石通一以上,补左右内史、大行卒史;比百石已下,补郡太守卒史:皆各二人,边郡一人。先用诵多者,若不足,乃择掌故补中二千石属,文学掌故补郡属,备员。请著功令。佗如律令。”制曰:“可。”自此以来,则公卿大夫士吏斌斌多文学之士矣。

申公者,鲁人也。高祖过鲁,申公以弟子从师入见高祖于鲁南宫。吕太后时,申公游学长安,与刘郢同师。已而郢为楚王,令申公傅其太子戊。戊不好学,疾申公。及王郢卒,戊立为楚王,胥靡申公。申公耻之,归鲁,退居家教,终身不出门,复谢绝宾客,独王命召之乃往。弟子自远方至受业者百馀人。申公独以诗经为训以教,无传,疑者则阙不传。

兰陵王臧既受诗,以事孝景帝为太子少傅,免去。今上初即位,臧乃上书宿卫上,累迁,一岁中为郎中令。及代赵绾亦尝受诗申公,绾为御史大夫。绾、臧请天子,欲立明堂以朝诸侯,不能就其事,乃言师申公。于是天子使使束帛加璧安车驷马迎申公,弟子二人乘轺传从。至,见天子。天子问治乱之事,申公时已八十馀,老,对曰:“为治者不在多言,顾力行何如耳。”是时天子方好文词,见申公对,默然。然已招致,则以为太中大夫,舍鲁邸,议明堂事。太皇窦太后好老子言,不说儒术,得赵绾、王臧之过以让上,上因废明堂事,尽下赵绾、王臧吏,后皆自杀。申公亦疾免以归,数年卒。

弟子为博士者十馀人:孔安国至临淮太守,周霸至胶西内史,夏宽至城阳内史,砀鲁赐至东海太守,兰陵缪生至长沙内史,徐偃为胶西中尉,邹人阙门庆忌为胶东内史。其治官民皆有廉节,称其好学。学官弟子行虽不备,而至于大夫、郎中、掌故以百数。言诗虽殊,多本于申公。

清河王太傅辕固生者,齐人也。以治诗,孝景时为博士。与黄生争论景帝前。黄生曰:“汤武非受命,乃弑也。”辕固生曰:“不然。夫桀纣虐乱,天下之心皆归汤武,汤武与天下之心而诛桀纣,桀纣之民不为之使而归汤武,汤武不得已而立,非受命为何?”黄生曰:“冠虽敝,必加于首;履虽新,必关于足。何者,上下之分也。今桀纣虽失道,然君上也;汤武虽圣,臣下也。夫主有失行,臣下不能正言匡过以尊天子,反因过而诛之,代立践南面,非弑而何也?”辕固生曰:“必若所云,是高帝代秦即天子之位,非邪?”于是景帝曰:“食肉不食马肝,不为不知味;言学者无言汤武受命,不为愚。”遂罢。是后学者莫敢明受命放杀者。

窦太后好老子书,召辕固生问老子书。固曰:“此是家人言耳。”太后怒曰:“安得司空城旦书乎?”乃使固入圈刺豕。景帝知太后怒而固直言无罪,乃假固利兵,下圈刺豕,正中其心,一刺,豕应手而倒。太后默然,无以复罪,罢之。居顷之,景帝以固为廉直,拜为清河王太傅。久之,病免。

今上初即位,复以贤良徵固。诸谀儒多疾毁固,曰“固老”,罢归之。时固已九十馀矣。固之徵也,薛人公孙弘亦徵,侧目而视固。固曰:“公孙子,务正学以言,无曲学以阿世!”自是之后,齐言诗皆本辕固生也。诸齐人以诗显贵,皆固之弟子也。

韩生者,燕人也。孝文帝时为博士,景帝时为常山王太傅。韩生推诗之意而为内外传数万言,其语颇与齐鲁间殊,然其归一也。淮南贲生受之。自是之后,而燕赵间言诗者由韩生。韩生孙商为今上博士。

伏生者,济南人也。故为秦博士。孝文帝时,欲求能治尚书者,天下无有,乃闻伏生能治,欲召之。是时伏生年九十馀,老,不能行,于是乃诏太常使掌故朝错往受之。秦时焚书,伏生壁藏之。其后兵大起,流亡,汉定,伏生求其书,亡数十篇,独得二十九篇,即以教于齐鲁之间。学者由是颇能言尚书,诸山东大师无不涉尚书以教矣。

伏生教济南张生及欧阳生,欧阳生教千乘兒宽。兒宽既通尚书,以文学应郡举,诣博士受业,受业孔安国。兒宽贫无资用,常为弟子都养,及时时间行佣赁,以给衣食。行常带经,止息则诵习之。以试第次,补廷尉史。是时张汤方乡学,以为奏谳掾,以古法议决疑大狱,而爱幸宽。宽为人温良,有廉智,自持,而善著书、书奏,敏于文,口不能发明也。汤以为长者,数称誉之。及汤为御史大夫,以兒宽为掾,荐之天子。天子见问,说之。张汤死后六年,兒宽位至御史大夫。九年而以官卒。宽在三公位,以和良承意从容得久,然无有所匡谏;于官,官属易之,不为尽力。张生亦为博士。而伏生孙以治尚书徵,不能明也。

自此之后,鲁周霸、孔安国,雒阳贾嘉,颇能言尚书事。孔氏有古文尚书,而安国以今文读之,因以起其家。逸书得十馀篇,盖尚书滋多于是矣。

诸学者多言礼,而鲁高堂生最本。礼固自孔子时而其经不具,及至秦焚书,书散亡益多,于今独有士礼,高堂生能言之。

而鲁徐生善为容。孝文帝时,徐生以容为礼官大夫。传子至孙延、徐襄。襄,其天姿善为容,不能通礼经;延颇能,未善也。襄以容为汉礼官大夫,至广陵内史。延及徐氏弟子公户满意、桓生、单次,皆尝为汉礼官大夫。而瑕丘萧奋以礼为淮阳太守。是后能言礼为容者,由徐氏焉。

自鲁商瞿受易孔子,孔子卒,商瞿传易,六世至齐人田何,字子庄,而汉兴。田何传东武人王同子仲,子仲传菑川人杨何。何以易,元光元年徵,官至中大夫。齐人即墨成以易至城阳相。广川人孟但以易为太子门大夫。鲁人周霸,莒人衡胡,临菑人主父偃,皆以易至二千石。然要言易者本于杨何之家。

董仲舒,广川人也。以治春秋,孝景时为博士。下帷讲诵,弟子传以久次相受业,或莫见其面,盖三年董仲舒不观于舍园,其精如此。进退容止,非礼不行,学士皆师尊之。今上即位,为江都相。以春秋灾异之变推阴阳所以错行,故求雨闭诸阳,纵诸阴,其止雨反是。行之一国,未尝不得所欲。中废为中大夫,居舍,著灾异之记。是时辽东高庙灾,主父偃疾之,取其书奏之天子。天子召诸生示其书,有刺讥。董仲舒弟子吕步舒不知其师书,以为下愚。于是下董仲舒吏,当死,诏赦之。于是董仲舒竟不敢复言灾异。

董仲舒为人廉直。是时方外攘四夷,公孙弘治春秋不如董仲舒,而弘希世用事,位至公卿。董仲舒以弘为从谀。弘疾之,乃言上曰:“独董仲舒可使相缪西王。”胶西王素闻董仲舒有行,亦善待之。董仲舒恐久获罪,疾免居家。至卒,终不治产业,以脩学著书为事。故汉兴至于五世之间,唯董仲舒名为明于春秋,其传公羊氏也。

胡毋生,齐人也。孝景时为博士,以老归教授。齐之言春秋者多受胡毋生,公孙弘亦颇受焉。

瑕丘江生为穀梁春秋。自公孙弘得用,尝集比其义,卒用董仲舒。

仲舒弟子遂者:兰陵褚大,广川殷忠,温吕步舒。褚大至梁相。步舒至长史,持节使决淮南狱,于诸侯擅专断,不报,以春秋之义正之,天子皆以为是。弟子通者,至于命大夫;为郎、谒者、掌故者以百数。而董仲舒子及孙皆以学至大官。

孔氏之衰,经书绪乱。言诸六学,始自炎汉。著令立官,四方鸧腕。曲台坏壁,书礼之冠。传易言诗,云蒸雾散。兴化致理,鸿猷克赞。
\end{yuanwen}

\chapter{酷吏列传}

\begin{yuanwen}
孔子曰:“道之以政,齐之以刑,民免而无耻。道之以德,齐之以礼,有耻且格。”老氏称:“上德不德,是以有德;下德不失德,是以无德。法令滋章,盗贼多有。”太史公曰:信哉是言也!法令者治之具,而非制治清浊之源也。昔天下之网尝密矣,然奸伪萌起,其极也,上下相遁,至于不振。当是之时,吏治若救火扬沸,非武健严酷,恶能胜其任而愉快乎!言道德者,溺其职矣。故曰“听讼,吾犹人也,必也使无讼乎”。“下士闻道大笑之”。非虚言也。汉兴,破觚而为圜,斫雕而为朴,网漏于吞舟之鱼,而吏治烝烝,不至于奸,黎民艾安。由是观之,在彼不在此。
\end{yuanwen}\begin{yuanwen}
	
\end{yuanwen}\begin{yuanwen}
	
\end{yuanwen}\begin{yuanwen}
	
\end{yuanwen}\begin{yuanwen}
	
\end{yuanwen}\begin{yuanwen}
	
\end{yuanwen}\begin{yuanwen}
	
\end{yuanwen}\begin{yuanwen}
	
\end{yuanwen}\begin{yuanwen}
	
\end{yuanwen}\begin{yuanwen}
	
\end{yuanwen}\begin{yuanwen}
	
\end{yuanwen}\begin{yuanwen}
	
\end{yuanwen}\begin{yuanwen}
	
\end{yuanwen}\begin{yuanwen}
	
\end{yuanwen}\begin{yuanwen}
	
\end{yuanwen}\begin{yuanwen}
	
\end{yuanwen}\begin{yuanwen}
	
\end{yuanwen}\begin{yuanwen}
	
\end{yuanwen}\begin{yuanwen}
	
\end{yuanwen}\begin{yuanwen}
	
\end{yuanwen}\begin{yuanwen}
	
\end{yuanwen}\begin{yuanwen}
	
\end{yuanwen}\begin{yuanwen}
	
\end{yuanwen}\begin{yuanwen}
	
\end{yuanwen}\begin{yuanwen}
	
\end{yuanwen}\begin{yuanwen}
	
\end{yuanwen}\begin{yuanwen}
	
\end{yuanwen}\begin{yuanwen}
	
\end{yuanwen}\begin{yuanwen}
	
\end{yuanwen}\begin{yuanwen}
	
\end{yuanwen}\begin{yuanwen}
	
\end{yuanwen}\begin{yuanwen}
	
\end{yuanwen}\begin{yuanwen}
	
\end{yuanwen}\begin{yuanwen}
	
\end{yuanwen}\begin{yuanwen}
	
\end{yuanwen}\begin{yuanwen}
	
\end{yuanwen}\begin{yuanwen}
	
\end{yuanwen}\begin{yuanwen}
	
\end{yuanwen}\begin{yuanwen}
	
\end{yuanwen}\begin{yuanwen}
	
\end{yuanwen}\begin{yuanwen}
	
\end{yuanwen}\begin{yuanwen}
	
\end{yuanwen}\begin{yuanwen}
	
\end{yuanwen}\begin{yuanwen}
	
\end{yuanwen}\begin{yuanwen}
	
\end{yuanwen}\begin{yuanwen}
	
\end{yuanwen}\begin{yuanwen}
	
\end{yuanwen}\begin{yuanwen}
	
\end{yuanwen}\begin{yuanwen}
	
\end{yuanwen}\begin{yuanwen}
	
\end{yuanwen}\begin{yuanwen}
高后时,酷吏独有侯封,刻轹宗室,侵辱功臣。吕氏已败,遂侯封之家。孝景时,晁错以刻深颇用术辅其资,而七国之乱,发怒于错,错卒以被戮。其后有郅都、宁成之属。

郅都者,杨人也。以郎事孝文帝。孝景时,都为中郎将,敢直谏,面折大臣于朝。尝从入上林,贾姬如厕,野彘卒入厕。上目都,都不行。上欲自持兵救贾姬,都伏上前曰:“亡一姬复一姬进,天下所少宁贾姬等乎?陛下纵自轻,柰宗庙太后何!”上还,彘亦去。太后闻之,赐都金百斤,由此重郅都。

济南瞷氏宗人三百馀家,豪猾,二千石莫能制,于是景帝乃拜都为济南太守。至则族灭瞷氏首恶,馀皆股栗。居岁馀,郡中不拾遗。旁十馀郡守畏都如大府。

都为人勇,有气力,公廉,不发私书,问遗无所受,请寄无所听。常自称曰:“已倍亲而仕,身固当奉职死节官下,终不顾妻子矣。”

郅都迁为中尉。丞相条侯至贵倨也,而都揖丞相。是时民朴,畏罪自重,而都独先严酷,致行法不避贵戚,列侯宗室见都侧目而视,号曰“苍鹰”。

临江王徵诣中尉府对簿,临江王欲得刀笔为书谢上,而都禁吏不予。魏其侯使人以间与临江王。临江王既为书谢上,因自杀。窦太后闻之,怒,以危法中都,都免归家。孝景帝乃使使持节拜都为雁门太守,而便道之官,得以便宜从事。匈奴素闻郅都节,居边,为引兵去,竟郅都死不近雁门。匈奴至为偶人象郅都,令骑驰射莫能中,见惮如此。匈奴患之。窦太后乃竟中都以汉法。景帝曰:“都忠臣。”欲释之。窦太后曰:“临江王独非忠臣邪?”于是遂斩郅都。

宁成者,穰人也。以郎谒者事景帝。好气,为人小吏,必陵其长吏;为人上,操下如束湿薪。滑贼任威。稍迁至济南都尉,而郅都为守。始前数都尉皆步入府,因吏谒守如县令,其畏郅都如此。及成往,直陵都出其上。都素闻其声,于是善遇,与结驩。久之,郅都死,后长安左右宗室多暴犯法,于是上召宁成为中尉。其治效郅都,其廉弗如,然宗室豪桀皆人人惴恐。

武帝即位,徙为内史。外戚多毁成之短,抵罪髡钳。是时九卿罪死即死,少被刑,而成极刑,自以为不复收,于是解脱,诈刻传出关归家。称曰:“仕不至二千石,贾不至千万,安可比人乎!”乃贳贷买陂田千馀顷,假贫民,役使数千家。数年,会赦。致产数千金,为任侠,持吏长短,出从数十骑。其使民威重于郡守。

周阳由者,其父赵兼以淮南王舅父侯周阳,故因姓周阳氏。由以宗家任为郎,事孝文及景帝。景帝时,由为郡守。武帝即位,吏治尚循谨甚,然由居二千石中,最为暴酷骄恣。所爱者,挠法活之;所憎者,曲法诛灭之。所居郡,必夷其豪。为守,视都尉如令。为都尉,必陵太守,夺之治。与汲黯俱为忮,司马安之文恶,俱在二千石列,同车未尝敢均茵伏。

由后为河东都尉,时与其守胜屠公争权,相告言罪。胜屠公当抵罪,义不受刑,自杀,而由弃市。

自宁成、周阳由之后,事益多,民巧法,大抵吏之治类多成、由等矣。

赵禹者,斄人。以佐史补中都官,用廉为令史,事太尉亚夫。亚夫为丞相,禹为丞相史,府中皆称其廉平。然亚夫弗任,曰:“极知禹无害,然文深,不可以居大府。”今上时,禹以刀笔吏积劳,稍迁为御史。上以为能,至太中大夫。与张汤论定诸律令,作见知,吏传得相监司。用法益刻,盖自此始。

张汤者,杜人也。其父为长安丞,出,汤为兒守舍。还而鼠盗肉,其父怒,笞汤。汤掘窟得盗鼠及馀肉,劾鼠掠治,传爰书,讯鞫论报,并取鼠与肉,具狱磔堂下。其父见之,视其文辞如老狱吏,大惊,遂使书狱。父死后,汤为长安吏,久之。

周阳侯始为诸卿时,尝系长安,汤倾身为之。及出为侯,大与汤交,遍见汤贵人。汤给事内史,为宁成掾,以汤为无害,言大府,调为茂陵尉,治方中。

武安侯为丞相,徵汤为史,时荐言之天子,补御史,使案事。治陈皇后蛊狱,深竟党与。于是上以为能,稍迁至太中大夫。与赵禹共定诸律令,务在深文,拘守职之吏。已而赵禹迁为中尉,徙为少府,而张汤为廷尉,两人交驩,而兄事禹。禹为人廉倨。为吏以来,舍毋食客。公卿相造请禹,禹终不报谢,务在绝知友宾客之请,孤立行一意而已。见文法辄取,亦不覆案,求官属阴罪。汤为人多诈,舞智以御人。始为小吏,乾没,与长安富贾田甲、鱼翁叔之属交私。及列九卿,收接天下名士大夫,己心内虽不合,然阳浮慕之。

是时上方乡文学,汤决大狱,欲傅古义,乃请博士弟子治尚书、春秋补廷尉史,亭疑法。奏谳疑事,必豫先为上分别其原,上所是,受而著谳决法廷尉,絜令扬主之明。奏事即谴,汤应谢,乡上意所便,必引正、监、掾史贤者,曰:“固为臣议,如上责臣,臣弗用,愚抵于此。”罪常释。即奏事,上善之,曰:“臣非知为此奏,乃正、监、掾史某为之。”其欲荐吏,扬人之善蔽人之过如此。所治即上意所欲罪,予监史深祸者;即上意所欲释,与监史轻平者。所治即豪,必舞文巧诋;即下户羸弱,时口言,虽文致法,上财察。于是往往释汤所言。汤至于大吏,内行脩也。通宾客饮食。于故人子弟为吏及贫昆弟,调护之尤厚。其造请诸公,不避寒暑。是以汤虽文深意忌不专平,然得此声誉。而刻深吏多为爪牙用者,依于文学之士。丞相弘数称其美。及治淮南、衡山、江都反狱,皆穷根本。严助及伍被,上欲释之。汤争曰:“伍被本画反谋,而助亲幸出入禁闼爪牙臣,乃交私诸侯如此,弗诛,后不可治。”于是上可论之。其治狱所排大臣自为功,多此类。于是汤益尊任,迁为御史大夫。

会浑邪等降,汉大兴兵伐匈奴,山东水旱,贫民流徙,皆仰给县官,县官空虚。于是丞上指,请造白金及五铢钱,笼天下盐铁,排富商大贾,出告缗令,鉏豪彊并兼之家,舞文巧诋以辅法。汤每朝奏事,语国家用,日晏,天子忘食。丞相取充位,天下事皆决于汤。百姓不安其生,骚动,县官所兴,未获其利,奸吏并侵渔,于是痛绳以罪。则自公卿以下,至于庶人,咸指汤。汤尝病,天子至自视病,其隆贵如此。

匈奴来请和亲,群臣议上前。博士狄山曰:“和亲便。”上问其便,山曰:“兵者凶器,未易数动。高帝欲伐匈奴,大困平城,乃遂结和亲。孝惠、高后时,天下安乐。及孝文帝欲事匈奴,北边萧然苦兵矣。孝景时,吴楚七国反,景帝往来两宫间,寒心者数月。吴楚已破,竟景帝不言兵,天下富实。今自陛下举兵击匈奴,中国以空虚,边民大困贫。由此观之,不如和亲。”上问汤,汤曰:“此愚儒,无知。”狄山曰:“臣固愚忠,若御史大夫汤乃诈忠。若汤之治淮南、江都,以深文痛诋诸侯,别疏骨肉,使蕃臣不自安。臣固知汤之为诈忠。”于是上作色曰:“吾使生居一郡,能无使虏入盗乎?”曰:“不能。”曰:“居一县?”对曰:“不能。”复曰:“居一障间?”山自度辩穷且下吏,曰:“能。”于是上遣山乘鄣。至月馀,匈奴斩山头而去。自是以后,群臣震慴。

汤之客田甲,虽贾人,有贤操。始汤为小吏时,与钱通,及汤为大吏,甲所以责汤行义过失,亦有烈士风。

汤为御史大夫七岁,败。

河东人李文尝与汤有卻,已而为御史中丞,恚,数从中文书事有可以伤汤者,不能为地。汤有所爱史鲁谒居,知汤不平,使人上蜚变告文奸事,事下汤,汤治论杀文,而汤心知谒居为之。上问曰:“言变事纵迹安起?”汤详惊曰:“此殆文故人怨之。”谒居病卧闾里主人,汤自往视疾,为谒居摩足。赵国以冶铸为业,王数讼铁官事,汤常排赵王。赵王求汤阴事。谒居尝案赵王,赵王怨之,并上书告:“汤,大臣也,史谒居有病,汤至为摩足,疑与为大奸。”事下廷尉。谒居病死,事连其弟,弟系导官。汤亦治他囚导官,见谒居弟,欲阴为之,而详不省。谒居弟弗知,怨汤,使人上书告汤与谒居谋,共变告李文。事下减宣。宣尝与汤有卻,及得此事,穷竟其事,未奏也。会人有盗发孝文园瘗钱,丞相青翟朝,与汤约俱谢,至前,汤念独丞相以四时行园,当谢,汤无与也,不谢。丞相谢,上使御史案其事。汤欲致其文丞相见知,丞相患之。三长史皆害汤,欲陷之。

始长史硃买臣,会稽人也。读春秋。庄助使人言买臣,买臣以楚辞与助俱幸,侍中,为太中大夫,用事;而汤乃为小吏,跪伏使买臣等前。已而汤为廷尉,治淮南狱,排挤庄助,买臣固心望。及汤为御史大夫,买臣以会稽守为主爵都尉,列于九卿。数年,坐法废,守长史,见汤,汤坐床上,丞史遇买臣弗为礼。买臣楚士,深怨,常欲死之。王朝,齐人也。以术至右内史。边通,学长短,刚暴彊人也,官再至济南相。故皆居汤右,已而失官,守长史,诎体于汤。汤数行丞相事,知此三长史素贵,常凌折之。以故三长史合谋曰:“始汤约与君谢,已而卖君;今欲劾君以宗庙事,此欲代君耳。吾知汤阴事。”使吏捕案汤左田信等,曰汤且欲奏请,信辄先知之,居物致富,与汤分之,及他奸事。事辞颇闻。上问汤曰:“吾所为,贾人辄先知之,益居其物,是类有以吾谋告之者。”汤不谢。汤又详惊曰:“固宜有。”减宣亦奏谒居等事。天子果以汤怀诈面欺,使使八辈簿责汤。汤具自道无此,不服。于是上使赵禹责汤。禹至,让汤曰:“君何不知分也。君所治夷灭者几何人矣?今人言君皆有状,天子重致君狱,欲令君自为计,何多以对簿为?”汤乃为书谢曰:“汤无尺寸功,起刀笔吏,陛下幸致为三公,无以塞责。然谋陷汤罪者,三长史也。”遂自杀。

汤死,家产直不过五百金,皆所得奉赐,无他业。昆弟诸子欲厚葬汤,汤母曰:“汤为天子大臣,被汙恶言而死,何厚葬乎!”载以牛车,有棺无椁。天子闻之,曰:“非此母不能生此子。”乃尽案诛三长史。丞相青翟自杀。出田信。上惜汤。稍迁其子安世。

赵禹中废,已而为廷尉。始条侯以为禹贼深,弗任。及禹为少府,比九卿。禹酷急,至晚节,事益多,吏务为严峻,而禹治加缓,而名为平。王温舒等后起,治酷于禹。禹以老,徙为燕相。数岁,乱悖有罪,免归。后汤十馀年,以寿卒于家。

义纵者,河东人也。为少年时,尝与张次公俱攻剽为群盗。纵有姊姁,以医幸王太后。王太后问:“有子兄弟为官者乎?”姊曰:“有弟无行,不可。”太后乃告上,拜义姁弟纵为中郎,补上党郡中令。治敢行,少蕴藉,县无逋事,举为第一。迁为长陵及长安令,直法行治,不避贵戚。以捕案太后外孙脩成君子仲,上以为能,迁为河内都尉。至则族灭其豪穰氏之属,河内道不拾遗。而张次公亦为郎,以勇悍从军,敢深入,有功,为岸头侯。

宁成家居,上欲以为郡守。御史大夫弘曰:“臣居山东为小吏时,宁成为济南都尉,其治如狼牧羊。成不可使治民。”上乃拜成为关都尉。岁馀,关东吏隶郡国出入关者,号曰“宁见乳虎,无值宁成之怒”。义纵自河内迁为南阳太守,闻宁成家居南阳,及纵至关,宁成侧行送迎,然纵气盛,弗为礼。至郡,遂案宁氏,尽破碎其家。成坐有罪,及孔、暴之属皆饹亡,南阳吏民重足一迹。而平氏硃彊、杜衍、杜周为纵牙爪之吏,任用,迁为廷史。军数出定襄,定襄吏民乱败,于是徙纵为定襄太守。纵至,掩定襄狱中重罪轻系二百馀人,及宾客昆弟私入相视亦二百馀人。纵一捕鞠,曰“为死罪解脱”。是日皆报杀四百馀人。其后郡中不寒而栗,猾民佐吏为治。

是时赵禹、张汤以深刻为九卿矣,然其治尚宽,辅法而行,而纵以鹰击毛挚为治。后会五铢钱白金起,民为奸,京师尤甚,乃以纵为右内史,王温舒为中尉。温舒至恶,其所为不先言纵,纵必以气凌之,败坏其功。其治,所诛杀甚多,然取为小治,奸益不胜,直指始出矣。吏之治以斩杀缚束为务,阎奉以恶用矣。纵廉,其治放郅都。上幸鼎湖,病久,已而卒起幸甘泉,道多不治。上怒曰:“纵以我为不复行此道乎?”嗛之。至冬,杨可方受告缗,纵以为此乱民,部吏捕其为可使者。天子闻,使杜式治,以为废格沮事,弃纵市。后一岁,张汤亦死。

王温舒者,阳陵人也。少时椎埋为奸。已而试补县亭长,数废。为吏,以治狱至廷史。事张汤,迁为御史。督盗贼,杀伤甚多,稍迁至广平都尉。择郡中豪敢任吏十馀人,以为爪牙,皆把其阴重罪,而纵使督盗贼,快其意所欲得。此人虽有百罪,弗法;即有避,因其事夷之,亦灭宗。以其故齐赵之郊盗贼不敢近广平,广平声为道不拾遗。上闻,迁为河内太守。

素居广平时,皆知河内豪奸之家,及往,九月而至。令郡具私马五十匹,为驿自河内至长安,部吏如居广平时方略,捕郡中豪猾,郡中豪猾相连坐千馀家。上书请,大者至族,小者乃死,家尽没入偿臧。奏行不过二三日,得可事。论报,至流血十馀里。河内皆怪其奏,以为神速。尽十二月,郡中毋声,毋敢夜行,野无犬吠之盗。其颇不得,失之旁郡国,黎来,会春,温舒顿足叹曰:“嗟乎,令冬月益展一月,足吾事矣!”其好杀伐行威不爱人如此。天子闻之,以为能,迁为中尉。其治复放河内,徙诸名祸猾吏与从事,河内则杨皆、麻戊,关中杨赣、成信等。义纵为内史,惮未敢恣治。及纵死,张汤败后,徙为廷尉,而尹齐为中尉。

尹齐者,东郡茌平人。以刀笔稍迁至御史。事张汤,张汤数称以为廉武,使督盗贼,所斩伐不避贵戚。迁为关内都尉,声甚于宁成。上以为能,迁为中尉,吏民益凋敝。尹齐木彊少文,豪恶吏伏匿而善吏不能为治,以故事多废,抵罪。上复徙温舒为中尉,而杨仆以严酷为主爵都尉。

杨仆者,宜阳人也。以千夫为吏。河南守案举以为能,迁为御史,使督盗贼关东。治放尹齐,以为敢挚行。稍迁至主爵都尉,列九卿。天子以为能。南越反,拜为楼船将军,有功,封将梁侯。为荀彘所缚。居久之,病死。

而温舒复为中尉。为人少文,居廷惛惛不辩,至于中尉则心开。督盗贼,素习关中俗,知豪恶吏,豪恶吏尽复为用,为方略。吏苛察,盗贼恶少年投缿购告言奸,置伯格长以牧司奸盗贼。温舒为人,善事有埶者;即无埶者,视之如奴。有埶家,虽有奸如山,弗犯;无埶者,贵戚必侵辱。舞文巧诋下户之猾,以焄大豪。其治中尉如此。奸猾穷治,大抵尽靡烂狱中,行论无出者。其爪牙吏虎而冠。于是中尉部中中猾以下皆伏,有势者为游声誉,称治。治数岁,其吏多以权富。

温舒击东越还,议有不中意者,坐小法抵罪免。是时天子方欲作通天台而未有人,温舒请覆中尉脱卒,得数万人作。上说,拜为少府。徙为右内史,治如其故,奸邪少禁。坐法失官。复为右辅,行中尉事。如故操。

岁馀,会宛军发,诏徵豪吏,温舒匿其吏华成,及人有变告温舒受员骑钱,他奸利事,罪至族,自杀。其时两弟及两婚家亦各自坐他罪而族。光禄徐自为曰:“悲夫,夫古有三族,而王温舒罪至同时而五族乎!”

温舒死,家直累千金。后数岁,尹齐亦以淮阳都尉病死,家直不满五十金。所诛灭淮阳甚多,及死,仇家欲烧其尸,尸亡去归葬。

自温舒等以恶为治,而郡守、都尉、诸侯二千石欲为治者,其治大抵尽放温舒,而吏民益轻犯法,盗贼滋起。南阳有梅免、白政,楚有殷中、杜少,齐有徐勃,燕赵之间有坚卢、范生之属。大群至数千人,擅自号,攻城邑,取库兵,释死罪,缚辱郡太守、都尉,杀二千石,为檄告县趣具食;小群以百数,掠卤乡里者,不可胜数也。于是天子始使御史中丞、丞相长史督之。犹弗能禁也,乃使光禄大夫范昆、诸辅都尉及故九卿张德等衣绣衣,持节,虎符发兵以兴击,斩首大部或至万馀级,及以法诛通饮食,坐连诸郡,甚者数千人。数岁,乃颇得其渠率。散卒失亡,复聚党阻山川者,往往而群居,无可柰何。于是作“沈命法”,曰群盗起不发觉,发觉而捕弗满品者,二千石以下至小吏主者皆死。其后小吏畏诛,虽有盗不敢发,恐不能得,坐课累府,府亦使其不言。故盗贼浸多,上下相为匿,以文辞避法焉。

减宣者,杨人也。以佐史无害给事河东守府。卫将军青使买马河东,见宣无害,言上,徵为大厩丞。官事辨,稍迁至御史及中丞。使治主父偃及治淮南反狱,所以微文深诋,杀者甚众,称为敢决疑。数废数起,为御史及中丞者几二十岁。王温舒免中尉,而宣为左内史。其治米盐,事大小皆关其手,自部署县名曹实物,官吏令丞不得擅摇,痛以重法绳之。居官数年,一切郡中为小治辨,然独宣以小致大,能因力行之,难以为经。中废。为右扶风,坐怨成信,信亡藏上林中,宣使郿令格杀信,吏卒格信时,射中上林苑门,宣下吏诋罪,以为大逆,当族,自杀。而杜周任用。

杜周者,南阳杜衍人。义纵为南阳守,以为爪牙,举为廷尉史。事张汤,汤数言其无害,至御史。使案边失亡,所论杀甚众。奏事中上意,任用,与减宣相编,更为中丞十馀岁。

其治与宣相放,然重迟,外宽,内深次骨。宣为左内史,周为廷尉,其治大放张汤而善候伺。上所欲挤者,因而陷之;上所欲释者,久系待问而微见其冤状。客有让周曰:“君为天子决平,不循三尺法,专以人主意指为狱。狱者固如是乎?”周曰:“三尺安出哉?前主所是著为律,后主所是疏为令,当时为是,何古之法乎!”

至周为廷尉,诏狱亦益多矣。二千石系者新故相因,不减百馀人。郡吏大府举之廷尉,一岁至千馀章。章大者连逮证案数百,小者数十人;远者数千,近者数百里。会狱,吏因责如章告劾,不服,以笞掠定之。于是闻有逮皆亡匿。狱久者至更数赦十有馀岁而相告言,大抵尽诋以不道以上。廷尉及中都官诏狱逮至六七万人,吏所增加十万馀人。

周中废,后为执金吾,逐盗,捕治桑弘羊、卫皇后昆弟子刻深,天子以为尽力无私,迁为御史大夫。家两子,夹河为守。其治暴酷皆甚于王温舒等矣。杜周初徵为廷史,有一马,且不全;及身久任事,至三公列,子孙尊官,家訾累数巨万矣。

太史公曰:自郅都、杜周十人者,此皆以酷烈为声。然郅都伉直,引是非,争天下大体。张汤以知阴阳,人主与俱上下,时数辩当否,国家赖其便。赵禹时据法守正。杜周从谀,以少言为重。自张汤死后,网密,多诋严,官事浸以秏废。九卿碌碌奉其官,救过不赡,何暇论绳墨之外乎!然此十人中,其廉者足以为仪表,其污者足以为戒,方略教导,禁奸止邪,一切亦皆彬彬质有其文武焉。虽惨酷,斯称其位矣。至若蜀守冯当暴挫,广汉李贞擅磔人,东郡弥仆锯项,天水骆璧推咸,河东褚广妄杀,京兆无忌、冯翊殷周蝮鸷,水衡阎奉朴击卖请,何足数哉!何足数哉!

太上失德,法令滋起。破觚为圆,禁暴不止。奸伪斯炽,惨酷爰始。乳兽扬威,苍鹰侧视。舞文巧诋,怀生何恃!
\end{yuanwen}

\part{卷一百二十三}

\chapter{大宛列传第六十三}

\begin{yuanwen}
大宛之迹,见自张骞。张骞,汉中人。建元中为郎。是时天子问匈奴降者,皆言匈奴破月氏王,以其头为饮器,月氏遁逃而常怨仇匈奴,无与共击之。汉方欲事灭胡,闻此言,因欲通使。道必更匈奴中,乃募能使者。骞以郎应募,使月氏,与堂邑氏胡奴甘父俱出陇西。经匈奴,匈奴得之,传诣单于。单于留之,曰:“月氏在吾北,汉何以得往使?吾欲使越,汉肯听我乎?”留骞十馀岁,与妻,有子,然骞持汉节不失。
\end{yuanwen}\begin{yuanwen}

\end{yuanwen}\begin{yuanwen}

\end{yuanwen}\begin{yuanwen}

\end{yuanwen}\begin{yuanwen}

\end{yuanwen}\begin{yuanwen}

\end{yuanwen}\begin{yuanwen}

\end{yuanwen}\begin{yuanwen}

\end{yuanwen}\begin{yuanwen}

\end{yuanwen}\begin{yuanwen}

\end{yuanwen}\begin{yuanwen}

\end{yuanwen}\begin{yuanwen}

\end{yuanwen}\begin{yuanwen}

\end{yuanwen}\begin{yuanwen}

\end{yuanwen}\begin{yuanwen}

\end{yuanwen}\begin{yuanwen}

\end{yuanwen}\begin{yuanwen}

\end{yuanwen}\begin{yuanwen}

\end{yuanwen}\begin{yuanwen}

\end{yuanwen}\begin{yuanwen}

\end{yuanwen}\begin{yuanwen}

\end{yuanwen}\begin{yuanwen}

\end{yuanwen}\begin{yuanwen}

\end{yuanwen}\begin{yuanwen}

\end{yuanwen}\begin{yuanwen}

\end{yuanwen}\begin{yuanwen}

\end{yuanwen}\begin{yuanwen}

\end{yuanwen}\begin{yuanwen}

\end{yuanwen}\begin{yuanwen}

\end{yuanwen}\begin{yuanwen}

\end{yuanwen}\begin{yuanwen}

\end{yuanwen}\begin{yuanwen}

\end{yuanwen}\begin{yuanwen}

\end{yuanwen}\begin{yuanwen}

\end{yuanwen}\begin{yuanwen}

\end{yuanwen}\begin{yuanwen}

\end{yuanwen}\begin{yuanwen}

\end{yuanwen}\begin{yuanwen}

\end{yuanwen}\begin{yuanwen}

\end{yuanwen}\begin{yuanwen}

\end{yuanwen}\begin{yuanwen}

\end{yuanwen}\begin{yuanwen}

\end{yuanwen}\begin{yuanwen}

\end{yuanwen}\begin{yuanwen}

\end{yuanwen}\begin{yuanwen}

\end{yuanwen}\begin{yuanwen}

\end{yuanwen}\begin{yuanwen}

\end{yuanwen}\begin{yuanwen}

\end{yuanwen}\begin{yuanwen}

\end{yuanwen}\begin{yuanwen}

\end{yuanwen}\begin{yuanwen}

\end{yuanwen}\begin{yuanwen}

\end{yuanwen}\begin{yuanwen}

\end{yuanwen}\begin{yuanwen}

\end{yuanwen}\begin{yuanwen}

\end{yuanwen}\begin{yuanwen}

\end{yuanwen}\begin{yuanwen}

\end{yuanwen}\begin{yuanwen}

\end{yuanwen}\begin{yuanwen}

\end{yuanwen}\begin{yuanwen}
居匈奴中,益宽,骞因与其属亡乡月氏,西走数十日至大宛。大宛闻汉之饶财,欲通不得,见骞,喜,问曰:“若欲何之?”骞曰:“为汉使月氏,而为匈奴所闭道。今亡,唯王使人导送我。诚得至,反汉,汉之赂遗王财物不可胜言。”大宛以为然,遣骞,为发导绎,抵康居,康居传致大月氏。大月氏王已为胡所杀,立其太子为王。既臣大夏而居,地肥饶,少寇,志安乐,又自以远汉,殊无报胡之心。骞从月氏至大夏,竟不能得月氏要领。

留岁馀,还,并南山,欲从羌中归,复为匈奴所得。留岁馀,单于死,左谷蠡王攻其太子自立,国内乱,骞与胡妻及堂邑父俱亡归汉。汉拜骞为太中大夫,堂邑父为奉使君。

骞为人彊力,宽大信人,蛮夷爱之。堂邑父故胡人,善射,穷急射禽兽给食。初,骞行时百馀人,去十三岁,唯二人得还。

骞身所至者大宛、大月氏、大夏、康居,而传闻其旁大国五六,具为天子言之。曰:大宛在匈奴西南,在汉正西,去汉可万里。其俗土著,耕田,田稻麦。有蒲陶酒。多善马,马汗血,其先天马子也。有城郭屋室。其属邑大小七十馀城,众可数十万。其兵弓矛骑射。其北则康居,西则大月氏,西南则大夏,东北则乌孙,东则扜鰛、于窴。于窴之西,则水皆西流,注西海;其东水东流,注盐泽。盐泽潜行地下,其南则河源出焉。多玉石,河注中国。而楼兰、姑师邑有城郭,临盐泽。盐泽去长安可五千里。匈奴右方居盐泽以东,至陇西长城,南接羌,鬲汉道焉。

乌孙在大宛东北可二千里,行国,随畜,与匈奴同俗。控弦者数万,敢战。故服匈奴,及盛,取其羁属,不肯往朝会焉。

康居在大宛西北可二千里,行国,与月氏大同俗。控弦者八九万人。与大宛邻国。国小,南羁事月氏,东羁事匈奴。

奄蔡在康居西北可二千里,行国,与康居大同俗。控弦者十馀万。临大泽,无崖,盖乃北海云。

大月氏在大宛西可二三千里,居妫水北。其南则大夏,西则安息,北则康居。行国也,随畜移徙,与匈奴同俗。控弦者可一二十万。故时彊,轻匈奴,及冒顿立,攻破月氏,至匈奴老上单于,杀月氏王,以其头为饮器。始月氏居敦煌、祁连间,及为匈奴所败,乃远去,过宛,西击大夏而臣之,遂都妫水北,为王庭。其馀小众不能去者,保南山羌,号小月氏。

安息在大月氏西可数千里。其俗土著,耕田,田稻麦,蒲陶酒。城邑如大宛。其属小大数百城,地方数千里,最为大国。临妫水,有市,民商贾用车及船,行旁国或数千里。以银为钱,钱如其王面,王死辄更钱,效王面焉。画革旁行以为书记。其西则条枝,北有奄蔡、黎轩。

条枝在安息西数千里,临西海。暑湿。耕田,田稻。有大鸟,卵如甕。人众甚多,往往有小君长,而安息役属之,以为外国。国善眩。安息长老传闻条枝有弱水、西王母,而未尝见。

大夏在大宛西南二千馀里妫水南。其俗土著,有城屋,与大宛同俗。无大长,往往城邑置小长。其兵弱,畏战。善贾市。及大月氏西徙,攻败之,皆臣畜大夏。大夏民多,可百馀万。其都曰蓝市城,有市贩贾诸物。其东南有身毒国。

骞曰:“臣在大夏时,见邛竹杖、蜀布。问曰:‘安得此?’ 大夏国人曰:‘吾贾人往市之身毒。身毒在大夏东南可数千里。其俗土著,大与大夏同,而卑湿暑热云。其人民乘象以战。其国临大水焉。’ 以骞度之,大夏去汉万二千里,居汉西南。今身毒国又居大夏东南数千里,有蜀物,此其去蜀不远矣。今使大夏,从羌中,险,羌人恶之;少北,则为匈奴所得;从蜀宜径,又无寇。”天子既闻大宛及大夏、安息之属皆大国,多奇物,土著,颇与中国同业,而兵弱,贵汉财物;其北有大月氏、康居之属,兵彊,可以赂遗设利朝也。且诚得而以义属之,则广地万里,重九译,致殊俗,威德遍于四海。天子欣然,以骞言为然,乃令骞因蜀犍为发间使,四道并出:出駹,出厓,出徙,出邛、僰,皆各行一二千里。其北方闭氐、筰,南方闭巂、昆明。昆明之属无君长,善寇盗,辄杀略汉使,终莫得通。然闻其西可千馀里有乘象国,名曰滇越,而蜀贾奸出物者或至焉,于是汉以求大夏道始通滇国。初,汉欲通西南夷,费多,道不通,罢之。及张骞言可以通大夏,乃复事西南夷。

骞以校尉从大将军击匈奴,知水草处,军得以不乏,乃封骞为博望侯。是岁元朔六年也。其明年,骞为卫尉,与李将军俱出右北平击匈奴。匈奴围李将军,军失亡多;而骞后期当斩,赎为庶人。是岁汉遣骠骑破匈奴西数万人,至祁连山。其明年,浑邪王率其民降汉,而金城、河西西并南山至盐泽空无匈奴。匈奴时有候者到,而希矣。其后二年,汉击走单于于幕北。

是后天子数问骞大夏之属。骞既失侯,因言曰:“臣居匈奴中,闻乌孙王号昆莫,昆莫之父,匈奴西边小国也。匈奴攻杀其父,而昆莫生弃于野。乌嗛肉蜚其上,狼往乳之。单于怪以为神,而收长之。及壮,使将兵,数有功,单于复以其父之民予昆莫,令长守于西。昆莫收养其民,攻旁小邑,控弦数万,习攻战。单于死,昆莫乃率其众远徙,中立,不肯朝会匈奴。匈奴遣奇兵击,不胜,以为神而远之,因羁属之,不大攻。今单于新困于汉,而故浑邪地空无人。蛮夷俗贪汉财物,今诚以此时而厚币赂乌孙,招以益东,居故浑邪之地,与汉结昆弟,其势宜听,听则是断匈奴右臂也。既连乌孙,自其西大夏之属皆可招来而为外臣。”天子以为然,拜骞为中郎将,将三百人,马各二匹,牛羊以万数,赍金币帛直数千巨万,多持节副使,道可使,使遗之他旁国。

骞既至乌孙,乌孙王昆莫见汉使如单于礼,骞大惭,知蛮夷贪,乃曰:“天子致赐,王不拜则还赐。”昆莫起拜赐,其他如故。骞谕使指曰:“乌孙能东居浑邪地,则汉遣翁主为昆莫夫人。”乌孙国分,王老,而远汉,未知其大小,素服属匈奴日久矣,且又近之,其大臣皆畏胡,不欲移徙,王不能专制。骞不得其要领。昆莫有十馀子,其中子曰大禄,彊,善将众,将众别居万馀骑。大禄兄为太子,太子有子曰岑娶,而太子蚤死。临死谓其父昆莫曰:“必以岑娶为太子,无令他人代之。”昆莫哀而许之,卒以岑娶为太子。大禄怒其不得代太子也,乃收其诸昆弟,将其众畔,谋攻岑娶及昆莫。昆莫老,常恐大禄杀岑娶,予岑娶万馀骑别居,而昆莫有万馀骑自备,国众分为三,而其大总取羁属昆莫,昆莫亦以此不敢专约于骞。

骞因分遣副使使大宛、康居、大月氏、大夏、安息、身毒、于窴、扜鰛及诸旁国。乌孙发导译送骞还,骞与乌孙遣使数十人,马数十匹报谢,因令窥汉,知其广大。

骞还到,拜为大行,列于九卿。岁馀,卒。

乌孙使既见汉人众富厚,归报其国,其国乃益重汉。其后岁馀,骞所遣使通大夏之属者皆颇与其人俱来,于是西北国始通于汉矣。然张骞凿空,其后使往者皆称博望侯,以为质于外国,外国由此信之。

自博望侯骞死后,匈奴闻汉通乌孙,怒,欲击之。及汉使乌孙,若出其南,抵大宛、大月氏相属,乌孙乃恐,使使献马,原得尚汉女翁主为昆弟。天子问群臣议计,皆曰“必先纳聘,然后乃遣女”。初,天子发书易,云“神马当从西北来”。得乌孙马好,名曰“天马”。及得大宛汗血马,益壮,更名乌孙马曰“西极”,名大宛马曰“天马”云。而汉始筑令居以西,初置酒泉郡以通西北国。因益发使抵安息、奄蔡、黎轩、条枝、身毒国。而天子好宛马,使者相望于道。诸使外国一辈大者数百,少者百馀人,人所赍操大放博望侯时。其后益习而衰少焉。汉率一岁中使多者十馀,少者五六辈,远者八九岁,近者数岁而反。

是时汉既灭越,而蜀、西南夷皆震,请吏入朝。于是置益州、越巂、牂柯、沈黎、汶山郡,欲地接以前通大夏。乃遣使柏始昌、吕越人等岁十馀辈,出此初郡抵大夏,皆复闭昆明,为所杀,夺币财,终莫能通至大夏焉。于是汉发三辅罪人,因巴蜀士数万人,遣两将军郭昌、卫广等往击昆明之遮汉使者,斩首虏数万人而去。其后遣使,昆明复为寇,竟莫能得通。而北道酒泉抵大夏,使者既多,而外国益厌汉币,不贵其物。

自博望侯开外国道以尊贵,其后从吏卒皆争上书言外国奇怪利害,求使。天子为其绝远,非人所乐往,听其言,予节,募吏民毋问所从来,为具备人众遣之,以广其道。来还不能毋侵盗币物,及使失指,天子为其习之,辄覆案致重罪,以激怒令赎,复求使。使端无穷,而轻犯法。其吏卒亦辄复盛推外国所有,言大者予节,言小者为副,故妄言无行之徒皆争效之。其使皆贫人子,私县官赍物,欲贱市以私其利外国。外国亦厌汉使人人有言轻重,度汉兵远不能至,而禁其食物以苦汉使。汉使乏绝积怨,至相攻击。而楼兰、姑师小国耳,当空道,攻劫汉使王恢等尤甚。而匈奴奇兵时时遮击使西国者。使者争遍言外国灾害,皆有城邑,兵弱易击。于是天子以故遣从骠侯破奴将属国骑及郡兵数万,至匈河水,欲以击胡,胡皆去。其明年,击姑师,破奴与轻骑七百馀先至,虏楼兰王,遂破姑师。因举兵威以困乌孙、大宛之属。还,封破奴为浞野侯。王恢数使,为楼兰所苦,言天子,天子发兵令恢佐破奴击破之,封恢为浩侯。于是酒泉列亭鄣至玉门矣。

乌孙以千匹马聘汉女,汉遣宗室女江都翁主往妻乌孙,乌孙王昆莫以为右夫人。匈奴亦遣女妻昆莫,昆莫以为左夫人。昆莫曰“我老”,乃令其孙岑娶妻翁主。乌孙多马,其富人至有四五千匹马。

初,汉使至安息,安息王令将二万骑迎于东界。东界去王都数千里。行比至,过数十城,人民相属甚多。汉使还,而后发使随汉使来观汉广大,以大鸟卵及黎轩善眩人献于汉。及宛西小国驩潜、大益,宛东姑师、扞鰛、苏薤之属,皆随汉使献见天子。天子大悦。

而汉使穷河源,河源出于窴,其山多玉石,采来,天子案古图书,名河所出山曰昆仑云。

是时上方数巡狩海上,乃悉从外国客,大都多人则过之,散财帛以赏赐,厚具以饶给之,以览示汉富厚焉。于是大觳抵,出奇戏诸怪物,多聚观者,行赏赐,酒池肉林,令外国客遍观仓库府藏之积,见汉之广大,倾骇之。及加其眩者之工,而觳抵奇戏岁增变,甚盛益兴,自此始。

西北外国使,更来更去。宛以西,皆自以远,尚骄恣晏然,未可诎以礼羁縻而使也。自乌孙以西至安息,以近匈奴,匈奴困月氏也,匈奴使持单于一信,则国国传送食,不敢留苦;及至汉使,非出币帛不得食,不市畜不得骑用。所以然者,远汉,而汉多财物,故必市乃得所欲,然以畏匈奴于汉使焉。宛左右以蒲陶为酒,富人藏酒至万馀石,久者数十岁不败。俗嗜酒,马嗜苜蓿。汉使取其实来,于是天子始种苜蓿、蒲陶肥饶地。及天马多,外国使来众,则离宫别观旁尽种蒲萄、苜蓿极望。自大宛以西至安息,国虽颇异言,然大同俗,相知言。其人皆深眼,多须珣,善市贾,争分铢。俗贵女子,女子所言而丈夫乃决正。其地皆无丝漆,不知铸钱器。及汉使亡卒降,教铸作他兵器。得汉黄白金,辄以为器,不用为币。

而汉使者往既多,其少从率多进熟于天子,言曰:“宛有善马在贰师城,匿不肯与汉使。”天子既好宛马,闻之甘心,使壮士车令等持千金及金马以请宛王贰师城善马。宛国饶汉物,相与谋曰:“汉去我远,而盐水中数败,出其北有胡寇,出其南乏水草。又且往往而绝邑,乏食者多。汉使数百人为辈来,而常乏食,死者过半,是安能致大军乎?无柰我何。且贰师马,宛宝马也。”遂不肯予汉使。汉使怒,妄言,椎金马而去。宛贵人怒曰:“汉使至轻我!”遣汉使去,令其东边郁成遮攻杀汉使,取其财物。于是天子大怒。诸尝使宛姚定汉等言宛兵弱,诚以汉兵不过三千人,彊弩射之,即尽虏破宛矣。天子已尝使浞野侯攻楼兰,以七百骑先至,虏其王,以定汉等言为然,而欲侯宠姬李氏,拜李广利为贰师将军,发属国六千骑,及郡国恶少年数万人,以往伐宛。期至贰师城取善马,故号“贰师将军”。赵始成为军正,故浩侯王恢使导军,而李哆为校尉,制军事。是岁太初元年也。而关东蝗大起,蜚西至敦煌。

贰师将军军既西过盐水,当道小国恐,各坚城守,不肯给食。攻之不能下。下者得食,不下者数日则去。比至郁成,士至者不过数千,皆饥罢。攻郁成,郁成大破之,所杀伤甚众。贰师将军与哆、始成等计:“至郁成尚不能举,况至其王都乎?”引兵而还。往来二岁。还至敦煌,士不过什一二。使使上书言:“道远多乏食;且士卒不患战,患饥。人少,不足以拔宛。原且罢兵,益发而复往。”天子闻之,大怒,而使使遮玉门,曰军有敢入者辄斩之!贰师恐,因留敦煌。

其夏,汉亡浞野之兵二万馀于匈奴。公卿及议者皆原罢击宛军,专力攻胡。天子已业诛宛,宛小国而不能下,则大夏之属轻汉,而宛善马绝不来,乌孙、仑头易苦汉使矣,为外国笑。乃案言伐宛尤不便者邓光等,赦囚徒材官,益发恶少年及边骑,岁馀而出敦煌者六万人,负私从者不与。牛十万,马三万馀匹,驴骡橐它以万数。多赍粮,兵弩甚设,天下骚动,传相奉伐宛,凡五十馀校尉。宛王城中无井,皆汲城外流水,于是乃遣水工徙其城下水空以空其城。益发戍甲卒十八万,酒泉、张掖北,置居延、休屠以卫酒泉,而发天下七科適,及载Я给贰师。转车人徒相连属至敦煌。而拜习马者二人为执驱校尉,备破宛择取其善马云。

于是贰师后复行,兵多,而所至小国莫不迎,出食给军。至仑头,仑头不下,攻数日,屠之。自此而西,平行至宛城,汉兵到者三万人。宛兵迎击汉兵,汉兵射败之,宛走入葆乘其城。贰师兵欲行攻郁成,恐留行而令宛益生诈,乃先至宛,决其水源,移之,则宛固已忧困。围其城,攻之四十馀日,其外城坏,虏宛贵人勇将煎靡。宛大恐,走入中城。宛贵人相与谋曰:“汉所为攻宛,以王毋寡匿善马而杀汉使。今杀王毋寡而出善马,汉兵宜解;即不解,乃力战而死,未晚也。”宛贵人皆以为然,共杀其王毋寡,持其头遣贵人使贰师,约曰:“汉毋攻我。我尽出善马,恣所取,而给汉军食。即不听,我尽杀善马,而康居之救且至。至,我居内,康居居外,与汉军战。汉军熟计之,何从?”是时康居候视汉兵,汉兵尚盛,不敢进。贰师与赵始成、李哆等计:“闻宛城中新得秦人,知穿井,而其内食尚多。所为来,诛首恶者毋寡。毋寡头已至,如此而不许解兵,则坚守,而康居候汉罢而来救宛,破汉军必矣。”军吏皆以为然,许宛之约。宛乃出其善马,令汉自择之,而多出食食给汉军。汉军取其善马数十匹。中马以下牡牝三千馀匹,而立宛贵人之故待遇汉使善者名昧蔡以为宛王,与盟而罢兵。终不得入中城。乃罢而引归。

初,贰师起敦煌西,以为人多,道上国不能食,乃分为数军,从南北道。校尉王申生、故鸿胪壶充国等千馀人,别到郁成。郁成城守,不肯给食其军。王申生去大军二百里,而轻之,责郁成。郁成食不肯出,窥知申生军日少,晨用三千人攻,戮杀申生等,军破,数人脱亡,走贰师。贰师令搜粟都尉上官桀往攻破郁成。郁成王亡走康居,桀追至康居。康居闻汉已破宛,乃出郁成王予桀,桀令四骑士缚守诣大将军。四人相谓曰:“郁成王汉国所毒,今生将去,卒失大事。”欲杀,莫敢先击。上邽骑士赵弟最少,拔剑击之,斩郁成王,赍头。弟、桀等逐及大将军。

初,贰师后行,天子使使告乌孙,大发兵并力击宛。乌孙发二千骑往,持两端,不肯前。贰师将军之东,诸所过小国闻宛破,皆使其子弟从军入献,见天子,因以为质焉。贰师之伐宛也,而军正赵始成力战,功最多;及上官桀敢深入,李哆为谋计,军入玉门者万馀人,军马千馀匹。贰师后行,军非乏食,战死不能多,而将吏贪,多不爱士卒,侵牟之,以此物故众。天子为万里而伐宛,不录过,封广利为海西侯。又封身斩郁成王者骑士赵弟为新畤侯。军正赵始成为光禄大夫,上官桀为少府,李哆为上党太守。军官吏为九卿者三人,诸侯相、郡守、二千石者百馀人,千石以下千馀人。奋行者官过其望,以適过行者皆绌其劳。士卒赐直四万金。伐宛再反,凡四岁而得罢焉。

汉已伐宛,立昧蔡为橡王而去#岁馀,宛贵人以为昧蔡善谀,使我国遇侠,乃相与杀昧蔡,立毋寡昆弟曰蝉封为宛王,而遣其子入质于汉。汉因使使赂赐以镇抚之。

而汉发使十馀辈至宛西诸外国,求奇物,因风览以伐宛之威德。而敦煌置酒泉都尉;西至盐水,往往有亭。而仑头有田卒数百人,因置使者护田积粟,以给使外国者。

太史公曰:禹本纪言“河出昆仑。昆仑其高二千五百馀里,日月所相避隐为光明也。其上有醴泉、瑶池”。今自张骞使大夏之后也,穷河源,恶睹本纪所谓昆仑者乎?故言九州山川,尚书近之矣。至禹本纪、山海经所有怪物,余不敢言之也。

大宛之迹,元因博望。始究河源,旋窥海上。条枝西入,天马内向。葱岭无尘,盐池息浪。旷哉绝域,往往亭障。
\end{yuanwen}

\chapter{游侠列传}

司马迁为汉初以来社会上存在过的“布衣之侠”所立的类传。

\begin{yuanwen}
韩子曰\footnote{text}:“儒以文乱法,而侠以武犯禁。”

二者皆讥,而学士多称于世云\footnote{text}。至如以术取宰相卿大夫\footnote{text},辅翼其世主,功名俱著于春秋,固无可言者\footnote{text}。及若季次、原宪\footnote{text},闾巷人也,读书怀独行君子之德,义不苟合当世,当世亦笑之。故季次、原宪终身空室蓬户,褐衣疏食不厌。死而已四百馀年,而弟子志之不倦\footnote{text}。

今游侠,其行虽不轨于正义\footnote{text},然其言必信,其行必果\footnote{text},已诺必诚,不爱其躯,赴士之厄困,既已存亡死生矣,而不矜其能,羞伐其德\footnote{text},盖亦有足多者焉。
\end{yuanwen}

\begin{yuanwen}
且缓急\footnote{text},人之所时有也。太史公曰:昔者虞舜窘于井廪\footnote{text},伊尹负于鼎俎\footnote{text},傅说匿于傅险\footnote{text},吕尚困于棘津\footnote{text},夷吾桎梏,百里饭牛\footnote{text},仲尼畏匡\footnote{text},菜色陈、蔡。此皆学士所谓有道仁人也,犹然遭此菑\footnote{text},况以中材而涉乱世之末流乎\footnote{text}?其遇害何可胜道哉!
\end{yuanwen}

\begin{yuanwen}
鄙人有言\footnote{text}曰:“何知仁义,已飨其利者为有德。”

故伯夷丑周\footnote{text},饿死首阳山,而文武不以其故贬王;跖、蹻暴戾\footnote{text},其徒诵义无穷。由此观之,“窃钩者诛,窃国者侯,侯之门仁义存”,非虚言也。
\end{yuanwen}

\begin{yuanwen}
今拘学或抱咫尺之义\footnote{text},久孤于世\footnote{text},岂若卑论侪俗\footnote{text},与世沉浮而取荣名哉\footnote{text}!而布衣之徒,设取予然诺\footnote{text},千里诵义,为死不顾世,此亦有所长,非苟而已也\footnote{text}。故士穷窘而得委命\footnote{text},此岂非人之所谓贤豪间者邪?诚使乡曲之侠\footnote{text},予季次、原宪比权量力,效功于当世,不同日而论矣\footnote{text}。要以功见言信,侠客之义又曷可少哉\footnote{text}!
\end{yuanwen}

\begin{yuanwen}
古布衣之侠\footnote{text},靡得而闻已。近世延陵、孟尝、春申、平原、信陵之徒\footnote{text},皆因王者亲属\footnote{text},藉于有土卿相之富厚\footnote{text},招天下贤者,显名诸侯,不可谓不贤者矣。比如顺风而呼,声非加疾,其势激也。至如闾巷之侠\footnote{text},修行砥名\footnote{text},声施于天下\footnote{text},莫不称贤,是为难耳。然儒、墨皆排摈不载\footnote{text}。自秦以前,匹夫之侠\footnote{text},湮灭不见,余甚恨之。以余所闻,汉兴有朱家、田仲、王公、剧孟、郭解之徒,虽时捍当世之文罔\footnote{text},然其私义廉絜退让,有足称者。名不虚立,士不虚附。至如朋党宗强比周\footnote{text},设财役贫\footnote{text},豪暴侵凌孤弱,恣欲自快,游侠亦丑之。余悲世俗不察其意,而猥以朱家、郭解等令与暴豪之徒同类而共笑之也\footnote{text}。
\end{yuanwen}

\begin{yuanwen}
鲁朱家者,与高祖同时。鲁人皆以儒教,而朱家用侠闻。所藏活豪士以百数,其馀庸人不可胜言。然终不伐其能,歆其德,诸所尝施,唯恐见之。振人不赡,先从贫贱始。家无馀财,衣不完采,食不重味,乘不过軥牛。专趋人之急,甚己之私。既阴脱季布将军之戹,及布尊贵,终身不见也。自关以东,莫不延颈原交焉。

楚田仲以侠闻,喜剑,父事朱家,自以为行弗及。田仲已死,而雒阳有剧孟。周人以商贾为资,而剧孟以任侠显诸侯。吴楚反时,条侯为太尉,乘传车将至河南,得剧孟,喜曰:“吴楚举大事而不求孟,吾知其无能为已矣。”天下骚动,宰相得之若得一敌国云。剧孟行大类朱家,而好博,多少年之戏。然剧孟母死,自远方送丧盖千乘。及剧孟死,家无馀十金之财。而符离人王孟亦以侠称江淮之间。

是时济南瞷氏、陈周庸亦以豪闻,景帝闻之,使使尽诛此属。其后代诸白、梁韩无辟、阳翟薛兄、陕韩孺纷纷复出焉。
\end{yuanwen}

\begin{yuanwen}
郭解,轵人也,字翁伯,善相人者许负外孙也。解父以任侠,孝文时诛死。解为人短小精悍,不饮酒。少时阴贼\footnote{text},慨不快意\footnote{text},身所杀甚众。以躯借交报仇\footnote{text},藏命作奸\footnote{text},剽攻不休,及铸钱掘冢\footnote{text},固不可胜数。适有天幸,窘急常得脱,若遇赦\footnote{text}。

及解年长,更折节为俭\footnote{text},以德报怨,厚施而薄望。然其自喜为侠益甚。既已振人之命\footnote{text},不矜其功,其阴贼着于心,卒发于睚眦如故云\footnote{text}。而少年慕其行,亦辄为报仇,不使知也\footnote{text}。解姊子负解之势,与人饮,使之嚼\footnote{text}。非其任,强必灌之。人怒,拔刀刺杀解姊子,亡去。解姊怒曰:“以翁伯之义,人杀吾子,贼不得。”

弃其尸于道,弗葬,欲以辱解。解使人微知贼处。贼窘自归,具以实告解。解曰:“公杀之固当,吾儿不直。”

遂去其贼,罪其姊子,乃收而葬之。诸公闻之,皆多解之义,益附焉。
\end{yuanwen}

\begin{yuanwen}
解出入,人皆避之\footnote{text}。有一人独箕倨视之\footnote{text},解遣人问其名姓。客欲杀之。解曰:“居邑屋至不见敬,是吾德不修也,彼何罪!”

乃阴属尉史\footnote{text}曰:“是人,吾所急也\footnote{text},至践更时脱之\footnote{text}。”

每至践更,数过,吏弗求。怪之,问其故,乃解使脱之。箕踞者乃肉袒谢罪\footnote{text}。少年闻之,愈益慕解之行。
\end{yuanwen}

\begin{yuanwen}
雒阳人有相仇者,邑中贤豪居间者以十数\footnote{text},终不听。客乃见郭解。解夜见仇家,仇家曲听解。解乃谓仇家曰:“吾闻雒阳诸公在此间,多不听者。今子幸而听解,解奈何乃从他县夺人邑中贤大夫权乎!”

乃夜去,不使人知,曰:“且无用,待我去,令雒阳豪居其间,乃听之。”
\end{yuanwen}

\begin{yuanwen}
解执恭敬,不敢乘车入其县廷\footnote{text}。之旁郡国,为人请求事,事可出,出之\footnote{text};不可者,各厌其意\footnote{text},然后乃敢尝酒食。诸公以故严重之\footnote{text},争为用。邑中少年及旁近县贤豪,夜半过门常十馀车,请得解客舍养之。
\end{yuanwen}



\begin{yuanwen}
及徙豪富茂陵也\footnote{text},解家贫,不中訾\footnote{text},吏恐,不敢不徙\footnote{text}。卫将军为言\footnote{text}:“郭解家贫不中徙\footnote{text}。”

上曰:“布衣权至使将军为言,此其家不贫。”

解家遂徙。诸公送者出千馀万\footnote{text}。轵人杨季主子为县掾,举徙解\footnote{text}。解兄子断杨掾头。由此杨氏与郭氏为仇。
\end{yuanwen}



\begin{yuanwen}
解入关,关中贤豪知与不知,闻其声,争交欢解。解为人短小,不饮酒,出未尝有骑。已又杀杨季主\footnote{text}。杨季主家上书,人又杀之阙下。上闻,乃下吏捕解。解亡,置其母家室夏阳,身至临晋\footnote{text}。临晋籍少公素不知解,解冒,因求出关。籍少公已出解,解转入太原,所过辄告主人家。吏逐之,迹至籍少公\footnote{text}。少公自杀,口绝。久之,乃得解。穷治所犯,为解所杀,皆在赦前。轵有儒生侍使者坐,客誉郭解,生曰:“郭解专以奸犯公法,何谓贤!”

解客闻,杀此生,断其舌。吏以此责解,解实不知杀者。杀者亦竟绝,莫知为谁。吏奏解无罪。御史大夫公孙弘议\footnote{text}曰:“解布衣为任侠行权\footnote{text},以睚眦杀人,解虽弗知,此罪甚于解杀之\footnote{text}。当大逆无道\footnote{text}。”

遂族郭解翁伯。
\end{yuanwen}



\begin{yuanwen}
自是之后,为侠者极众,敖而无足数者。然关中长安樊仲子,槐里赵王孙,长陵高公子,西河郭公仲,太原卤公孺,临淮儿长卿,东阳田君孺,虽为侠而逡逡有退让君子之风。至若北道姚氏,西道诸杜,南道仇景,东道赵他、羽公子,南阳赵调之徒,此盗跖居民间者耳,曷足道哉!此乃乡者朱家之羞也。

太史公曰:吾视郭解,状貌不及中人,言语不足采者。然天下无贤与不肖,知与不知,皆慕其声,言侠者皆引以为名。谚曰:“人貌荣名,岂有既乎!”于戏,惜哉!

游侠豪倨,藉藉有声。权行州里,力折公卿。朱家脱季,剧孟定倾。急人之难,免雠于更。伟哉翁伯,人貌荣名。
\end{yuanwen}

\part{卷一百二十五}
\chapter{佞幸列传第六十五}

\begin{yuanwen}
谚曰“力田\footnote{努力耕种。}不如逢年\footnote{遇上风调雨顺的好年景。},善仕\footnote{善于做官。}不如遇合\footnote{遇到合适的机会。}”,固无虚言。非独女以色媚,而士宦亦有之。
\end{yuanwen}

俗话说:“努力种田,不如碰上好年景;善于做官,莫过于碰到机遇。”这真的不是空话。不仅仅是女子靠着姿色献媚得宠,士人宦官中也有这样的人。

\begin{yuanwen}
昔以色幸者多矣。至汉兴,高祖至暴抗也,然籍孺\footnote{男童。}以佞幸\footnote{巧言献媚而得到宠信。};孝惠时有闳孺。此两人非有材能,徒以婉佞贵幸,与上卧起,公卿皆因关说\footnote{都通过闳孺向皇帝报告情况、获得指示。}。故孝惠时郎侍中皆冠鵕璘\footnote{锦鸡羽毛装饰的帽子。},贝带,傅脂粉,化闳、籍之属也。两人徙家安陵。
\end{yuanwen}

过去,靠着美色而受宠的人实在很多!到了汉朝兴起,汉高祖最为暴烈刚直,但是籍孺却靠着谄媚而受宠;孝惠帝的时候有闳孺。这两个人并非靠着什么才能,靠的仅仅是柔顺、谄媚才得以显贵宠幸,他们和皇上同卧同起,公卿大臣有什么事情都得通过他们才能禀告给皇上。所以,孝惠帝时,宫中的郎官、侍中都戴着锦鸡羽毛装饰着的帽子,束着用贝壳装饰的腰带,搽着胭脂香粉,效仿闳孺、籍孺这类人。闳孺、籍孺两人将家迁到了安陵。

\begin{yuanwen}

\end{yuanwen}\begin{yuanwen}

\end{yuanwen}\begin{yuanwen}

\end{yuanwen}\begin{yuanwen}

\end{yuanwen}\begin{yuanwen}

\end{yuanwen}\begin{yuanwen}

\end{yuanwen}\begin{yuanwen}

\end{yuanwen}\begin{yuanwen}

\end{yuanwen}\begin{yuanwen}

\end{yuanwen}\begin{yuanwen}

\end{yuanwen}\begin{yuanwen}

\end{yuanwen}\begin{yuanwen}

\end{yuanwen}\begin{yuanwen}

\end{yuanwen}\begin{yuanwen}

\end{yuanwen}\begin{yuanwen}

\end{yuanwen}\begin{yuanwen}

\end{yuanwen}\begin{yuanwen}

\end{yuanwen}\begin{yuanwen}

\end{yuanwen}\begin{yuanwen}

\end{yuanwen}\begin{yuanwen}

\end{yuanwen}\begin{yuanwen}

\end{yuanwen}\begin{yuanwen}

\end{yuanwen}\begin{yuanwen}

\end{yuanwen}\begin{yuanwen}

\end{yuanwen}\begin{yuanwen}

\end{yuanwen}\begin{yuanwen}

\end{yuanwen}\begin{yuanwen}

\end{yuanwen}\begin{yuanwen}

\end{yuanwen}\begin{yuanwen}

\end{yuanwen}\begin{yuanwen}

\end{yuanwen}\begin{yuanwen}

\end{yuanwen}\begin{yuanwen}

\end{yuanwen}\begin{yuanwen}

\end{yuanwen}\begin{yuanwen}

\end{yuanwen}\begin{yuanwen}

\end{yuanwen}\begin{yuanwen}

\end{yuanwen}\begin{yuanwen}

\end{yuanwen}\begin{yuanwen}

\end{yuanwen}\begin{yuanwen}

\end{yuanwen}\begin{yuanwen}

\end{yuanwen}\begin{yuanwen}

\end{yuanwen}\begin{yuanwen}

\end{yuanwen}\begin{yuanwen}

\end{yuanwen}\begin{yuanwen}

\end{yuanwen}\begin{yuanwen}

\end{yuanwen}\begin{yuanwen}

\end{yuanwen}\begin{yuanwen}

\end{yuanwen}\begin{yuanwen}

\end{yuanwen}\begin{yuanwen}

\end{yuanwen}\begin{yuanwen}

\end{yuanwen}\begin{yuanwen}


孝文时中宠臣,士人则邓通,宦者则赵同、北宫伯子。北宫伯子以爱人长者;而赵同以星气幸,常为文帝参乘;邓通无伎能。邓通,蜀郡南安人也,以濯船为黄头郎。孝文帝梦欲上天,不能,有一黄头郎从后推之上天,顾见其衣裻带后穿。觉而之渐台,以梦中阴目求推者郎,即见邓通,其衣后穿,梦中所见也。召问其名姓,姓邓氏,名通,文帝说焉,尊幸之日异。通亦愿谨,不好外交,虽赐洗沐,不欲出。于是文帝赏赐通巨万以十数,官至上大夫。文帝时时如邓通家游戏。然邓通无他能,不能有所荐士,独自谨其身以媚上而已。上使善相者相通,曰“当贫饿死”。文帝曰:“能富通者在我也。何谓贫乎?”于是赐邓通蜀严道铜山,得自铸钱,“邓氏钱”布天下。其富如此。

文帝尝病痈,邓通常为帝唶吮之。文帝不乐,从容问通曰:“天下谁最爱我者乎?”通曰:“宜莫如太子。”太子入问病,文帝使唶痈,唶痈而色难之。已而闻邓通常为帝唶吮之,心惭,由此怨通矣。及文帝崩,景帝立,邓通免,家居。居无何,人有告邓通盗出徼外铸钱。下吏验问,颇有之,遂竟案,尽没入邓通家,尚负责数巨万。长公主赐邓通,吏辄随没入之,一簪不得著身。于是长公主乃令假衣食。竟不得名一钱,寄死人家。

孝景帝时,中无宠臣,然独郎中令周文仁,仁宠最过庸,乃不甚笃。

今天子中宠臣,士人则韩王孙嫣,宦者则李延年。嫣者,弓高侯孽孙也。今上为胶东王时,嫣与上学书相爱。及上为太子,愈益亲嫣。嫣善骑射,善佞。上即位,欲事伐匈奴,而嫣先习胡兵,以故益尊贵,官至上大夫,赏赐拟于邓通。时嫣常与上卧起。江都王入朝,有诏得从入猎上林中。天子车驾跸道未行,而先使嫣乘副车,从数十百骑,骛驰视兽。江都王望见,以为天子,辟从者,伏谒道傍。嫣驱不见。既过,江都王怒,为皇太后泣曰:“请得归国入宿卫,比韩嫣。”太后由此嗛嫣。嫣侍上,出入永巷不禁,以奸闻皇太后。皇太后怒,使使赐嫣死。上为谢,终不能得,嫣遂死。而案道侯韩说,其弟也,亦佞幸。

李延年,中山人也。父母及身兄弟及女,皆故倡也。延年坐法腐,给事狗中。而平阳公主言延年女弟善舞,上见,心说之,及入永巷,而召贵延年。延年善歌,为变新声,而上方兴天地祠,欲造乐诗歌弦之。延年善承意,弦次初诗。其女弟亦幸,有子男。延年佩二千石印,号协声律。与上卧起,甚贵幸,埒如韩嫣也。久之,浸与中人乱,出入骄恣。及其女弟李夫人卒后,爱弛,则禽诛延年昆弟也。

自是之后,内宠嬖臣大底外戚之家,然不足数也。卫青、霍去病亦以外戚贵幸,然颇用材能自进。

太史公曰:甚哉爱憎之时!弥子瑕之行,足以观后人佞幸矣。虽百世可知也。

传称令色,诗刺巧言。冠璘入侍,傅粉承恩。黄头赐蜀,宦者同轩。新声都尉,挟弹王孙。泣鱼窃驾,著自前论。
\end{yuanwen}

\chapter{滑稽列传}

\begin{yuanwen}
孔子曰:“六艺于治一也。礼以节人,乐以发和,书以道事,诗以达意,易以神化,春秋以义。”太史公曰:天道恢恢,岂不大哉!谈言微中,亦可以解纷。

淳于髡者,齐之赘婿也。长不满七尺,滑稽多辩,数使诸侯,未尝屈辱。齐威王之时喜隐,好为淫乐长夜之饮,沈湎不治,委政卿大夫。百官荒乱,诸侯并侵,国且危亡,在于旦暮,左右莫敢谏。淳于髡说之以隐曰:“国中有大鸟,止王之庭,三年不蜚又不鸣,不知此鸟何也?”王曰:“此鸟不飞则已,一飞冲天;不鸣则已,一鸣惊人。”于是乃朝诸县令长七十二人,赏一人,诛一人,奋兵而出。诸侯振惊,皆还齐侵地。威行三十六年。语在田完世家中。

威王八年,楚大发兵加齐。齐王使淳于髡之赵请救兵,赍金百斤,车马十驷。淳于髡仰天大笑,冠缨索绝。王曰:“先生少之乎?”髡曰:“何敢!”王曰:“笑岂有说乎?”髡曰:“今者臣从东方来,见道傍有禳田者,操一豚蹄,酒一盂,祝曰:‘瓯窭满篝,汙邪满车,五穀蕃熟,穰穰满家。’ 臣见其所持者狭而所欲者奢,故笑之。”于是齐威王乃益赍黄金千溢,白璧十双,车马百驷。髡辞而行,至赵。赵王与之精兵十万,革车千乘。楚闻之,夜引兵而去。

威王大说,置酒后宫,召髡赐之酒。问曰:“先生能饮几何而醉?”对曰:“臣饮一斗亦醉,一石亦醉。”威王曰:“先生饮一斗而醉,恶能饮一石哉!其说可得闻乎?”髡曰:“赐酒大王之前,执法在傍,御史在后,髡恐惧俯伏而饮,不过一斗径醉矣。若亲有严客,髡韝鞠鯱,待酒于前,时赐馀沥,奉觞上寿,数起,饮不过二斗径醉矣。若朋友交游,久不相见,卒然相睹,欢然道故,私情相语,饮可五六斗径醉矣。若乃州闾之会,男女杂坐,行酒稽留,六博投壶,相引为曹,握手无罚,目眙不禁,前有堕珥,后有遗簪,髡窃乐此,饮可八斗而醉二参。日暮酒阑,合尊促坐,男女同席,履舄交错,杯盘狼藉,堂上烛灭,主人留髡而送客,罗襦襟解,微闻芗泽,当此之时,髡心最欢,能饮一石。故曰酒极则乱,乐极则悲;万事尽然,言不可极,极之而衰。”以讽谏焉。齐王曰:“善。”乃罢长夜之饮,以髡为诸侯主客。宗室置酒,髡尝在侧。

其后百馀年,楚有优孟。

优孟,故楚之乐人也。长八尺,多辩,常以谈笑讽谏。楚庄王之时,有所爱马,衣以文绣,置之华屋之下,席以露床,啗以枣脯。马病肥死,使群臣丧之,欲以棺椁大夫礼葬之。左右争之,以为不可。王下令曰:“有敢以马谏者,罪至死。”优孟闻之,入殿门。仰天大哭。王惊而问其故。优孟曰:“马者王之所爱也,以楚国堂堂之大,何求不得,而以大夫礼葬之,薄,请以人君礼葬之。”王曰:“何如?”对曰:“臣请以彫玉为棺,文梓为椁,楩枫豫章为题凑,发甲卒为穿壙,老弱负土,齐赵陪位于前,韩魏翼卫其后,庙食太牢,奉以万户之邑。诸侯闻之,皆知大王贱人而贵马也。”王曰:“寡人之过一至此乎!为之柰何?”优孟曰:“请为大王六畜葬之。以垅灶为椁,铜历为棺,赍以姜枣,荐以木兰,祭以粮稻,衣以火光,葬之于人腹肠。”于是王乃使以马属太官,无令天下久闻也。

楚相孙叔敖知其贤人也,善待之。病且死,属其子曰:“我死,汝必贫困。若往见优孟,言我孙叔敖之子也。”居数年,其子穷困负薪,逢优孟,与言曰:“我,孙叔敖子也。父且死时,属我贫困往见优孟。”优孟曰:“若无远有所之。”即为孙叔敖衣冠,抵掌谈语。岁馀,像孙叔敖,楚王及左右不能别也。庄王置酒,优孟前为寿。庄王大惊,以为孙叔敖复生也,欲以为相。优孟曰:“请归与妇计之,三日而为相。”庄王许之。三日后,优孟复来。王曰:“妇言谓何?”孟曰:“妇言慎无为,楚相不足为也。如孙叔敖之为楚相,尽忠为廉以治楚,楚王得以霸。今死,其子无立锥之地,贫困负薪以自饮食。必如孙叔敖,不如自杀。”因歌曰:“山居耕田苦,难以得食。起而为吏,身贪鄙者馀财,不顾耻辱。身死家室富,又恐受赇枉法,为奸触大罪,身死而家灭。贪吏安可为也!念为廉吏,奉法守职,竟死不敢为非。廉吏安可为也!楚相孙叔敖持廉至死,方今妻子穷困负薪而食,不足为也!”于是庄王谢优孟,乃召孙叔敖子,封之寝丘四百户,以奉其祀。后十世不绝。此知可以言时矣。

其后二百馀年,秦有优旃。

优旃者,秦倡侏儒也。善为笑言,然合于大道,秦始皇时,置酒而天雨,陛楯者皆沾寒。优旃见而哀之,谓之曰:“汝欲休乎?”陛楯者皆曰:“幸甚。”优旃曰:“我即呼汝,汝疾应曰诺。”居有顷,殿上上寿呼万岁。优旃临槛大呼曰:“陛楯郎!”郎曰:“诺。”优旃曰:“汝虽长,何益,幸雨立。我虽短也,幸休居。”于是始皇使陛楯者得半相代。

始皇尝议欲大苑囿,东至函谷关,西至雍、陈仓。优旃曰:“善。多纵禽兽于其中,寇从东方来,令麋鹿触之足矣。”始皇以故辍止。

二世立,又欲漆其城。优旃曰:“善。主上虽无言,臣固将请之。漆城虽于百姓愁费,然佳哉!漆城荡荡,寇来不能上。即欲就之,易为漆耳,顾难为廕室。”于是二世笑之,以其故止。居无何,二世杀死,优旃归汉,数年而卒。

太史公曰:淳于髡仰天大笑,齐威王横行。优孟摇头而歌,负薪者以封。优旃临槛疾呼,陛楯得以半更。岂不亦伟哉!褚先生曰:臣幸得以经术为郎,而好读外家传语。窃不逊让,复作故事滑稽之语六章,编之于左。可以览观扬意,以示后世好事者读之,以游心骇耳,以附益上方太史公之三章。

武帝时有所幸倡郭舍人者,发言陈辞虽不合大道,然令人主和说。武帝少时,东武侯母常养帝,帝壮时,号之曰“大乳母”。率一月再朝。朝奏入,有诏使幸臣马游卿以帛五十匹赐乳母,又奉饮Я飧养乳母。乳母上书曰:“某所有公田,原得假倩之。”帝曰:“乳母欲得之乎?”以赐乳母。乳母所言,未尝不听。有诏得令乳母乘车行驰道中。当此之时,公卿大臣皆敬重乳母。乳母家子孙奴从者横暴长安中,当道掣顿人车马,夺人衣服。闻于中,不忍致之法。有司请徙乳母家室,处之于边。奏可。乳母当入至前,面见辞。乳母先见郭舍人,为下泣。舍人曰:“即入见辞去,疾步数还顾。”乳母如其言,谢去,疾步数还顾。郭舍人疾言骂之曰:“咄!老女子!何不疾行!陛下已壮矣,宁尚须汝乳而活邪?尚何还顾!”于是人主怜焉悲之,乃下诏止无徙乳母,罚谪谮之者。

武帝时,齐人有东方生名朔,以好古传书,爱经术,多所博观外家之语。朔初入长安,至公车上书,凡用三千奏牍。公车令两人共持举其书,仅然能胜之。人主从上方读之,止,辄乙其处,读之二月乃尽。诏拜以为郎,常在侧侍中。数召至前谈语,人主未尝不说也。时诏赐之食于前。饭已,尽怀其馀肉持去,衣尽汙。数赐缣帛,檐揭而去。徒用所赐钱帛,取少妇于长安中好女。率取妇一岁所者即弃去,更取妇。所赐钱财尽索之于女子。人主左右诸郎半呼之“狂人”。人主闻之,曰:“令朔在事无为是行者,若等安能及之哉!”朔任其子为郎,又为侍谒者,常持节出使。朔行殿中,郎谓之曰:“人皆以先生为狂。”朔曰:“如朔等,所谓避世于朝廷间者也。古之人,乃避世于深山中。”时坐席中,酒酣,据地歌曰:“陆沈于俗,避世金马门。宫殿中可以避世全身,何必深山之中,蒿庐之下。”金马门者,宦署门也,门傍有铜马,故谓之曰“金马门”。

时会聚宫下博士诸先生与论议,共难之曰:“苏秦、张仪一当万乘之主,而都卿相之位,泽及后世。今子大夫修先王之术,慕圣人之义,讽诵诗书百家之言,不可胜数。著于竹帛,自以为海内无双,即可谓博闻辩智矣。然悉力尽忠以事圣帝,旷日持久,积数十年,官不过侍郎,位不过执戟,意者尚有遗行邪?其故何也?”东方生曰:“是固非子所能备也。彼一时也,此一时也,岂可同哉!夫张仪、苏秦之时,周室大坏,诸侯不朝,力政争权,相禽以兵,并为十二国,未有雌雄,得士者彊,失士者亡,故说听行通,身处尊位,泽及后世,子孙长荣。今非然也。圣帝在上,德流天下,诸侯宾服,威振四夷,连四海之外以为席,安于覆盂,天下平均,合为一家,动发举事,犹如运之掌中。贤与不肖,何以异哉?方今以天下之大,士民之众,竭精驰说,并进辐凑者,不可胜数。悉力慕义,困于衣食,或失门户。使张仪、苏秦与仆并生于今之世,曾不能得掌故,安敢望常侍侍郎乎!传曰:‘天下无害菑,虽有圣人,无所施其才;上下和同,虽有贤者,无所立功。’ 故曰时异则事异。虽然,安可以不务修身乎?诗曰:‘鼓锺于宫,声闻于外。鹤鸣九皋,声闻于天。’ 。苟能修身,何患不荣!太公躬行仁义七十二年,逢文王,得行其说,封于齐,七百岁而不绝。此士之所以日夜孜孜,修学行道,不敢止也。今世之处士,时虽不用,崛然独立,塊然独处,上观许由,下察接舆,策同范蠡,忠合子胥,天下和平,与义相扶,寡偶少徒,固其常也。子何疑于余哉!”于是诸先生默然无以应也。

建章宫后閤重栎中有物出焉,其状似麋。以闻,武帝往临视之。问左右群臣习事通经术者,莫能知。诏东方朔视之。朔曰:“臣知之,原赐美酒粱饭大飧臣,臣乃言。”诏曰:“可。”已又曰:“某所有公田鱼池蒲苇数顷,陛下以赐臣,臣朔乃言。”诏曰:“可。”于是朔乃肯言,曰:“所谓驺牙者也。远方当来归义,而驺牙先见。其齿前后若一,齐等无牙,故谓之驺牙。”其后一岁所,匈奴混邪王果将十万众来降汉。乃复赐东方生钱财甚多。

至老,朔且死时,谏曰:“诗云‘营营青蝇,止于蕃。恺悌君子,无信谗言。谗言罔极,交乱四国’ 。原陛下远巧佞,退谗言。”帝曰:“今顾东方朔多善言?”怪之。居无几何,朔果病死。传曰:“鸟之将死,其鸣也哀;人之将死,其言也善。”此之谓也。

武帝时,大将军卫青者,卫后兄也,封为长平侯。从军击匈奴,至余吾水上而还,斩首捕虏,有功来归,诏赐金千斤。将军出宫门,齐人东郭先生以方士待诏公车,当道遮卫将军车,拜谒曰:“原白事。”将军止车前,东郭先生旁车言曰:“王夫人新得幸于上,家贫。今将军得金千斤,诚以其半赐王夫人之亲,人主闻之必喜。此所谓奇策便计也。”卫将军谢之曰:“先生幸告之以便计,请奉教。”于是卫将军乃以五百金为王夫人之亲寿。王夫人以闻武帝。帝曰:“大将军不知为此。”问之安所受计策,对曰:“受之待诏者东郭先生。”诏召东郭先生,拜以为郡都尉。东郭先生久待诏公车,贫困饥寒,衣敝,履不完。行雪中,履有上无下,足尽践地。道中人笑之,东郭先生应之曰:“谁能履行雪中,令人视之,其上履也,其履下处乃似人足者乎?”及其拜为二千石,佩青緺出宫门,行谢主人。故所以同官待诏者,等比祖道于都门外。荣华道路,立名当世。此所谓衣褐怀宝者也。当其贫困时,人莫省视;至其贵也,乃争附之。谚曰:“相马失之瘦,相士失之贫。”其此之谓邪?

王夫人病甚,人主至自往问之曰:“子当为王,欲安所置之?”对曰:“原居洛阳。”人主曰:“不可。洛阳有武库、敖仓,当关口,天下咽喉。自先帝以来,传不为置王。然关东国莫大于齐,可以为齐王。”王夫人以手击头,呼“幸甚”。王夫人死,号曰“齐王太后薨”。

昔者,齐王使淳于髡献鹄于楚。出邑门,道飞其鹄,徒揭空笼,造诈成辞,往见楚王曰:“齐王使臣来献鹄,过于水上,不忍鹄之渴,出而饮之,去我飞亡。吾欲刺腹绞颈而死。恐人之议吾王以鸟兽之故令士自伤杀也。鹄,毛物,多相类者,吾欲买而代之,是不信而欺吾王也。欲赴佗国奔亡,痛吾两主使不通。故来服过,叩头受罪大王。”楚王曰:“善,齐王有信士若此哉!”厚赐之,财倍鹄在也。

武帝时,徵北海太守诣行在所。有文学卒史王先生者,自请与太守俱,“吾有益于君”,君许之。诸府掾功曹白云:“王先生嗜酒,多言少实,恐不可与俱。”太守曰:“先生意欲行,不可逆。”遂与俱。行至宫下,待诏宫府门。王先生徒怀钱沽酒,与卫卒仆射饮,日醉,不视其太守。太守入跪拜。王先生谓户郎曰:“幸为我呼吾君至门内遥语。”户郎为呼太守。太守来,望见王先生。王先生曰:“天子即问君何以治北海令无盗贼,君对曰何哉?”对曰:“选择贤材,各任之以其能,赏异等,罚不肖。”王先生曰:“对如是,是自誉自伐功,不可也。原君对言,非臣之力,尽陛下神灵威武所变化也。”太守曰:“诺。”召入,至于殿下,有诏问之曰:“何于治北海,令盗贼不起?”叩头对言:“非臣之力,尽陛下神灵威武之所变化也。”武帝大笑,曰:“于呼!安得长者之语而称之!安所受之?”对曰:“受之文学卒史。”帝曰:“今安在?”对曰:“在宫府门外。”有诏召拜王先生为水衡丞,以北海太守为水衡都尉。传曰:“美言可以市,尊行可以加人。君子相送以言,小人相送以财。”

魏文侯时,西门豹为鄴令。豹往到鄴,会长老,问之民所疾苦。长老曰:“苦为河伯娶妇,以故贫。”豹问其故,对曰:“鄴三老、廷掾常岁赋敛百姓,收取其钱得数百万,用其二三十万为河伯娶妇,与祝巫共分其馀钱持归。当其时,巫行视小家女好者,云是当为河伯妇,即娉取。洗沐之,为治新缯绮縠衣,间居斋戒;为治斋宫河上,张缇绛帷,女居其中。为具牛酒饭食,十馀日。共粉饰之,如嫁女床席,令女居其上,浮之河中。始浮,行数十里乃没。其人家有好女者,恐大巫祝为河伯取之,以故多持女远逃亡。以故城中益空无人,又困贫,所从来久远矣。民人俗语曰‘即不为河伯娶妇,水来漂没,溺其人民’ 云。”西门豹曰:“至为河伯娶妇时,原三老、巫祝、父老送女河上,幸来告语之,吾亦往送女。”皆曰:“诺。”

至其时,西门豹往会之河上。三老、官属、豪长者、里父老皆会,以人民往观之者三二千人。其巫,老女子也,已年七十。从弟子女十人所,皆衣缯单衣,立大巫后。西门豹曰:“呼河伯妇来,视其好丑。”即将女出帷中,来至前。豹视之,顾谓三老、巫祝、父老曰:“是女子不好,烦大巫妪为入报河伯,得更求好女,后日送之。”即使吏卒共抱大巫妪投之河中。有顷,曰:“巫妪何久也?弟子趣之!”复以弟子一人投河中。有顷,曰:“弟子何久也?复使一人趣之!”复投一弟子河中。凡投三弟子。西门豹曰:“巫妪弟子是女子也,不能白事,烦三老为入白之。”复投三老河中。西门豹簪笔磬折,乡河立待良久。长老、吏傍观者皆惊恐。西门豹顾曰:“巫妪、三老不来还,柰之何?”欲复使廷掾与豪长者一人入趣之。皆叩头,叩头且破,额血流地,色如死灰。西门豹曰:“诺,且留待之须臾。”须臾,豹曰:“廷掾起矣。状河伯留客之久,若皆罢去归矣。”鄴吏民大惊恐,从是以后,不敢复言为河伯娶妇。

西门豹即发民凿十二渠,引河水灌民田,田皆溉。当其时,民治渠少烦苦,不欲也。豹曰:“民可以乐成,不可与虑始。今父老子弟虽患苦我,然百岁后期令父老子孙思我言。”至今皆得水利,民人以给足富。十二渠经绝驰道,到汉之立,而长吏以为十二渠桥绝驰道,相比近,不可。欲合渠水,且至驰道合三渠为一桥。鄴民人父老不肯听长吏,以为西门君所为也,贤君之法式不可更也。长吏终听置之。故西门豹为鄴令,名闻天下,泽流后世,无绝已时,几可谓非贤大夫哉!

传曰:“子产治郑,民不能欺;子贱治单父,民不忍欺;西门豹治鄴,民不敢欺。”三子之才能谁最贤哉?辨治者当能别之。

滑稽鸱夷,如脂如韦。敏捷之变,学不失词。淳于索绝,赵国兴师。楚优拒相,寝丘获祠。伟哉方朔,三章纪之。
\end{yuanwen}

\chapter{日者列传}

\begin{yuanwen}
自古受命而王,王者之兴何尝不以卜筮决于天命哉!其于周尤甚,及秦可见。代王之入,任于卜者。太卜之起,由汉兴而有。
\end{yuanwen}

\begin{yuanwen}
	
\end{yuanwen}\begin{yuanwen}
	
\end{yuanwen}\begin{yuanwen}
	
\end{yuanwen}\begin{yuanwen}
	
\end{yuanwen}\begin{yuanwen}
	
\end{yuanwen}\begin{yuanwen}
	
\end{yuanwen}\begin{yuanwen}
	
\end{yuanwen}\begin{yuanwen}
	
\end{yuanwen}\begin{yuanwen}
	
\end{yuanwen}\begin{yuanwen}
	
\end{yuanwen}\begin{yuanwen}
	
\end{yuanwen}\begin{yuanwen}
	
\end{yuanwen}\begin{yuanwen}
	
\end{yuanwen}\begin{yuanwen}
	
\end{yuanwen}\begin{yuanwen}
	
\end{yuanwen}\begin{yuanwen}
	
\end{yuanwen}\begin{yuanwen}
	
\end{yuanwen}\begin{yuanwen}
	
\end{yuanwen}\begin{yuanwen}
	
\end{yuanwen}\begin{yuanwen}
	
\end{yuanwen}\begin{yuanwen}
	
\end{yuanwen}\begin{yuanwen}
	
\end{yuanwen}\begin{yuanwen}
	
\end{yuanwen}\begin{yuanwen}
	
\end{yuanwen}\begin{yuanwen}
	
\end{yuanwen}\begin{yuanwen}
	
\end{yuanwen}\begin{yuanwen}
	
\end{yuanwen}\begin{yuanwen}
	
\end{yuanwen}\begin{yuanwen}
	
\end{yuanwen}\begin{yuanwen}
	
\end{yuanwen}\begin{yuanwen}
	
\end{yuanwen}\begin{yuanwen}
	
\end{yuanwen}\begin{yuanwen}
	
\end{yuanwen}\begin{yuanwen}
	
\end{yuanwen}\begin{yuanwen}
	
\end{yuanwen}\begin{yuanwen}
	
\end{yuanwen}\begin{yuanwen}
	
\end{yuanwen}\begin{yuanwen}
	
\end{yuanwen}\begin{yuanwen}
	
\end{yuanwen}\begin{yuanwen}
	
\end{yuanwen}\begin{yuanwen}
	
\end{yuanwen}\begin{yuanwen}
	
\end{yuanwen}\begin{yuanwen}
	
\end{yuanwen}\begin{yuanwen}
	
\end{yuanwen}\begin{yuanwen}
	
\end{yuanwen}\begin{yuanwen}
	
\end{yuanwen}\begin{yuanwen}
	
\end{yuanwen}\begin{yuanwen}
	
\end{yuanwen}\begin{yuanwen}
	
\end{yuanwen}\begin{yuanwen}
	
\end{yuanwen}\begin{yuanwen}
	
\end{yuanwen}\begin{yuanwen}
司马季主者,楚人也。卜于长安东市。

宋忠为中大夫,贾谊为博士,同日俱出洗沐,相从论议,诵易先王圣人之道术,究遍人情,相视而叹。贾谊曰:“吾闻古之圣人,不居朝廷,必在卜医之中。今吾已见三公九卿朝士大夫,皆可知矣。试之卜数中以观采。”二人即同舆而之市,游于卜肆中。天新雨,道少人,司马季主间坐,弟子三四人侍,方辩天地之道,日月之运,阴阳吉凶之本。二大夫再拜谒。司马季主视其状貌,如类有知者,即礼之,使弟子延之坐。坐定,司马季主复理前语,分别天地之终始,日月星辰之纪,差次仁义之际,列吉凶之符,语数千言,莫不顺理。

宋忠、贾谊瞿然而悟,猎缨正襟危坐,曰:“吾望先生之状,听先生之辞,小子窃观于世,未尝见也。今何居之卑,何行之汙?”

司马季主捧腹大笑曰:“观大夫类有道术者,今何言之陋也,何辞之野也!今夫子所贤者何也?所高者谁也?今何以卑汙长者?”

二君曰:“尊官厚禄,世之所高也,贤才处之。今所处非其地,故谓之卑。言不信,行不验,取不当,故谓之汙。夫卜筮者,世俗之所贱简也。世皆言曰:‘夫卜者多言夸严以得人情,虚高人禄命以说人志,擅言祸灾以伤人心,矫言鬼神以尽人财,厚求拜谢以私于己。’ 此吾之所耻,故谓之卑汙也。”

司马季主曰:“公且安坐。公见夫被发童子乎?日月照之则行,不照则止,问之日月疵瑕吉凶,则不能理。由是观之,能知别贤与不肖者寡矣。

“贤之行也,直道以正谏,三谏不听则退。其誉人也不望其报,恶人也不顾其怨,以便国家利众为务。故官非其任不处也,禄非其功不受也;见人不正,虽贵不敬也;见人有污,虽尊不下也;得不为喜,去不为恨;非其罪也,虽累辱而不愧也。

“今公所谓贤者,皆可为羞矣。卑疵而前,韱趋而言;相引以势,相导以利;比周宾正,以求尊誉,以受公奉;事私利,枉主法,猎农民;以官为威,以法为机,求利逆暴:譬无异于操白刃劫人者也。初试官时,倍力为巧诈,饰虚功执空文以主上,用居上为右;试官不让贤陈功,见伪增实,以无为有,以少为多,以求便势尊位;食饮驱驰,从姬歌兒,不顾于亲,犯法害民,虚公家:此夫为盗不操矛弧者也,攻而不用弦刃者也,欺父母未有罪而弑君未伐者也。何以为高贤才乎?

“盗贼发不能禁,夷貊不服不能摄,奸邪起不能塞,官秏乱不能治,四时不和不能调,岁穀不孰不能適。才贤不为,是不忠也;才不贤而讬官位,利上奉,妨贤者处,是窃位也;有人者进,有财者礼,是伪也。子独不见鸱枭之与凤皇翔乎?兰芷芎藭弃于广野,蒿萧成林,使君子退而不显众,公等是也。

“述而不作,君子义也。今夫卜者,必法天地,象四时,顺于仁义,分策定卦,旋式正釭,然后言天地之利害,事之成败。昔先王之定国家,必先龟策日月,而后乃敢代;正时日,乃后入家;产子必先占吉凶,后乃有之。自伏羲作八卦,周文王演三百八十四爻而天下治。越王句践放文王八卦以破敌国,霸天下。由是言之,卜筮有何负哉!

“且夫卜筮者,埽除设坐,正其冠带,然后乃言事,此有礼也。言而鬼神或以飨,忠臣以事其上,孝子以养其亲,慈父以畜其子,此有德者也。而以义置数十百钱,病者或以愈,且死或以生,患或以免,事或以成,嫁子娶妇或以养生:此之为德,岂直数十百钱哉!此夫老子所谓‘上德不德,是以有德’ 。今夫卜筮者利大而谢少,老子之云岂异于是乎?

“庄子曰:‘君子内无饥寒之患,外无劫夺之忧,居上而敬,居下不为害,君子之道也。’ 今夫卜筮者之为业也,积之无委聚,藏之不用府库,徙之不用辎车,负装之不重,止而用之无尽索之时。持不尽索之物,游于无穷之世,虽庄氏之行未能增于是也,子何故而云不可卜哉?天不足西北,星辰西北移;地不足东南,以海为池;日中必移,月满必亏;先王之道,乍存乍亡。公责卜者言必信,不亦惑乎!

“公见夫谈士辩人乎?虑事定计,必是人也,然不能以一言说人主意,故言必称先王,语必道上古;虑事定计,饰先王之成功,语其败害,以恐喜人主之志,以求其欲。多言夸严,莫大于此矣。然欲彊国成功,尽忠于上,非此不立。今夫卜者,导惑教愚也。夫愚惑之人,岂能以一言而知之哉!言不厌多。

“故骐骥不能与罢驴为驷,而凤皇不与燕雀为群,而贤者亦不与不肖者同列。故君子处卑隐以辟众,自匿以辟伦,微见德顺以除群害,以明天性,助上养下,多其功利,不求尊誉。公之等喁喁者也,何知长者之道乎!”

宋忠、贾谊忽而自失,芒乎无色,怅然噤口不能言。于是摄衣而起,再拜而辞。行洋洋也,出门仅能自上车,伏轼低头,卒不能出气。

居三日,宋忠见贾谊于殿门外,乃相引屏语相谓自叹曰:“道高益安,势高益危。居赫赫之势,失身且有日矣。夫卜而有不审,不见夺糈;为人主计而不审,身无所处。此相去远矣,犹天冠地屦也。此老子之所谓‘无名者万物之始’ 也。天地旷旷,物之熙熙,或安或危,莫知居之。我与若,何足预彼哉!彼久而愈安,虽曾氏之义未有以异也。”

久之,宋忠使匈奴,不至而还,抵罪。而贾谊为梁怀王傅,王堕马薨,谊不食,毒恨而死。此务华绝根者也。

太史公曰:古者卜人所以不载者,多不见于篇。及至司马季主,余志而著之。

褚先生曰:臣为郎时,游观长安中,见卜筮之贤大夫,观其起居行步,坐起自动,誓正其衣冠而当乡人也,有君子之风。见性好解妇来卜,对之颜色严振,未尝见齿而笑也。从古以来,贤者避世,有居止舞泽者,有居民间闭口不言,有隐居卜筮间以全身者。夫司马季主者,楚贤大夫,游学长安,通易经,术黄帝、老子,博闻远见。观其对二大夫贵人之谈言,称引古明王圣人道,固非浅闻小数之能。及卜筮立名声千里者,各往往而在。传曰:“富为上,贵次之;既贵各各学一伎能立其身。”黄直,大夫也;陈君夫,妇人也:以相马立名天下。齐张仲、曲成侯以善击刺学用剑,立名天下。留长孺以相彘立名。荥阳褚氏以相牛立名。能以伎能立名者甚多,皆有高世绝人之风,何可胜言。故曰:“非其地,树之不生;非其意,教之不成。”夫家之教子孙,当视其所以好,好含苟生活之道,因而成之。故曰:“制宅命子,足以观士;子有处所,可谓贤人。”

臣为郎时,与太卜待诏为郎者同署,言曰:“孝武帝时,聚会占家问之,某日可取妇乎?五行家曰可,堪舆家曰不可,建除家曰不吉,丛辰家曰大凶,历家曰小凶,天人家曰小吉,太一家曰大吉。辩讼不决,以状闻。制曰:‘避诸死忌,以五行为主。’ ”人取于五行者也。

日者之名,有自来矣。吉凶占候,著于墨子。齐楚异法,书亡罕纪。后人斯继,季主独美。取免暴秦,此焉终否。
\end{yuanwen}

\part{卷一百二十八}
\chapter{龟策列传第六十八}

\begin{yuanwen}
太史公曰:自古圣王将建国受命,兴动事业,何尝不宝\footnote{重视。}卜筮以助善!唐虞以上,不可记已。自三代之兴,各据祯祥\footnote{吉兆。}。涂山之兆从\footnote{吉利。}而夏启世,飞燕之卜顺故殷兴,百穀之筮吉故周王。王者决定诸疑,参以卜筮,断以蓍龟,不易之道也。
\end{yuanwen}

太史公说:从古至今,圣明的君主即将建立国家,领受天命,兴办大业,何尝不重视占卜来协助促成善举的!唐尧和虞舜之前的事,已经无法记载。自从夏、商、周这三代的逐次兴起,各自依照占卜的吉兆作为凭据。涂山氏的卜兆吉利,因此夏开启了自己的时代;飞燕的卜兆吉顺,因此殷族得以兴国;百谷的筮兆吉祥,所以周室才得以称王天下。君王在决定各种疑难的时候,就加以卜筮,用蓍草、龟甲判断吉凶,这是不曾更改的方法。

\begin{yuanwen}
蛮夷氐羌虽无君臣之序,亦有决疑之卜。或以金石,或以草木,国不同俗。然皆可以战伐攻击,推兵\footnote{进军。}求胜,各信其神,以知来事。
\end{yuanwen}

尽管蛮、夷、氐、羌各族没有君臣上下的等级之分,但是也有决断疑惑的占卜。他们有的是以金石占卜,有的是以草木占卜,各国都有不同的风俗。但是像能否征伐攻击,能否出兵取胜,他们都相信自己所推崇的神灵,以此来预知将来的事。

\begin{yuanwen}
略闻夏殷欲卜者,乃取蓍龟,已则弃去之,以为龟藏则不灵,蓍久则不神。至周室之卜官,常宝藏蓍龟;又其大小先后,各有所尚,要\footnote{关键,关窍。}其归\footnote{目的。}等耳。或以为圣王遭事无不定,决疑无不见,其设稽\footnote{磕头。}神求问之道者,以为后世衰微,愚不师智,人各自安,化分为百室,道散而无垠,故推归之至微\footnote{微妙的事理。},要絜于精神也。或以为昆蟲之所长,圣人不能与争。其处\footnote{判断。}吉凶,别然否,多中\footnote{准确。}于人。至高祖时,因\footnote{沿袭。}秦太卜官。天下始定,兵革未息。及孝惠享国日少,吕后女主,孝文、孝景因袭掌故,未遑\footnote{来不及。遑,闲暇。}讲试,虽父子畴官,世世相传,其精微深妙,多所遗失。至今上即位,博开艺能之路,悉延百端之学,通一伎之士咸得自效,绝伦超奇者为右,无所阿私,数年之间,太卜大集。会上欲击匈奴,西攘大宛,南收百越,卜筮至预见表象,先图其利。及猛将推锋执节\footnote{控制。},获胜于彼,而蓍龟时日亦有力于此。上尤加意,赏赐至或数千万。如丘子明之属,富溢贵宠,倾于朝廷。至以卜筮射\footnote{揣度。}蛊道,巫蛊时或颇\footnote{有些。}中。素有眦睚\footnote{细小的愤怒。}不快,因公行诛,恣意所伤,以破族灭门者,不可胜数。百僚荡恐\footnote{恐惧不安。},皆曰龟策能言。后事觉奸穷,亦诛三族。
\end{yuanwen}

我粗略听说过夏、殷时期打算占卜的人,便找来蓍草龟甲,占卜结束之后就丢掉它们,认为龟甲一旦收藏起来就不灵了,蓍草用得久了就不神了。等到周王室的卜官占卜后,经常将蓍草和龟甲珍藏起来备用,还有选用蓍草和龟甲的大小及先后,各自有推崇的方法,但是他们的目的却是一样的。有的人认为圣王遇到事情没有无法确定的,定断疑难没有不是明白可见的,他们设置这套求神问卜方法的原因,是担忧后代日渐衰败式微,愚蠢而不愿向聪明的人学习,人人都满足于自身的见识,教化分化出为百家,道理散乱而漫无边际,因此将事理推演归结到最为精细的地方,总括规范到精神上。还有的人认为灵龟这种动物的生长,圣人也不能与它们相比。它们判断吉凶,分辨可否,常常比人还要准确。等到了汉高祖时,沿袭秦朝制度设置了太卜官。天下刚刚统一,战争还未平息。等到了孝惠帝继位,在位的时间很短,吕后是个女君主,孝文帝、孝景帝沿袭旧的制度,没有闲暇探究卜筮的事情。虽然有的父子都担任畴官,世代传承,但是这其中的精微幽深之处,大多已经遗失。等到当今皇上继承帝位,广开艺能的门路,悉迎诸子百家的学说,通晓一门技能的人都可以发挥出自己的长处,技艺绝妙、超群出奇的人更是位列高位,没有任何偏私。几年之内,太卜人才聚集了很多。正赶上皇上打算讨伐匈奴,向西抵御大宛,向南收服百越,卜筮者很精确地预示了各种情况,事先谋划制定好对策。等到猛将在前冲锋,执节指挥战斗,在各个地区都取得大胜,而以蓍草和龟甲卜筮所推算出来的得胜时日对此也大有帮助。于是皇上十分满意,赏赐给卜筮者有的多达几千万钱。像丘子明这些人,富到了极点,尊贵而备受宠幸,其富贵荣宠超过了满朝的公卿大臣。至于利用卜筮来占卜邪术和巫术等行为,有时预测结果也颇为准确。平日里和卜官有些小恩怨的人,都被他们借着公事除掉。卜官随意中伤他人,因此而被破族灭家的人,数不胜数。百官都恐惧不安,都奉承说龟甲、蓍草能够说话。等到后来卜官诛害人的事情被发现,奸计难以奏效,他们也同样被诛灭了三族。

\begin{yuanwen}
夫摓\footnote{féng}策定数\footnote{吉凶。},灼龟观兆\footnote{观察龟甲灼烧后呈现出的裂痕。},变化无穷,是以择贤而用占焉,可谓圣人重事\footnote{慎重从事。}者乎!周公卜三龟,而武王有瘳\footnote{病愈。}。纣为暴虐,而元龟\footnote{大龟。}不占。晋文将定襄王之位,卜得黄帝之兆,卒受彤弓\footnote{朱红色的弓。}之命。献公贪骊姬之色,卜而兆有口象\footnote{口舌祸象。},其祸竟流五世。楚灵将背周室,卜而龟逆\footnote{龟甲的兆象不吉利。},终被乾谿之败。兆应信诚于内,而时人明察见之于外,可不谓两合者哉!君子谓夫轻卜筮,无神明者,悖\footnote{迷惑。};背人道,信祯祥\footnote{吉祥,指占卜。}者,鬼神不得其正。故《书》建稽疑,五谋而卜筮居其二,五占从其多,明有而不专之道也。
\end{yuanwen}

捧着蓍草来判断吉凶,灼烧龟甲来察视征兆,变化无穷,因此选择贤德的人来担任卜官,可以说是圣人也非常慎重的事情吧!周公用龟甲占卜了三次,周武王因此而病愈。商纣王以暴虐治国,用大龟占卜也得不到吉兆。晋文公即将恢复周襄王的王位,占卜得到了黄帝之兆,最终接受了朱红色的弓而成为诸侯。晋献公贪图骊姬的美色,占卜得到有口舌之祸的凶象,其灾祸竟然流传了五代。楚灵王想要背叛周天子,占卜时获得不祥之兆,最终在乾谿惨败。这样看来,征兆和应验对内诚实地预示,而当时的人能够明察事理,看到外在的结果,怎么能说这不是相互吻合的呢!君子认为那些轻视卜筮,不信任神明的人都非常荒谬;放弃人谋而迷信祥瑞的人,鬼神也难以被公正的看待。所以《尚书》九畤稽疑第七记载,所举五谋要卜和筮其中两种,五谋不一致时,要顺从占多数的意见,这表明虽有卜筮,但并非专信卜筮。

\begin{yuanwen}

\end{yuanwen}\begin{yuanwen}

\end{yuanwen}\begin{yuanwen}

\end{yuanwen}\begin{yuanwen}

\end{yuanwen}\begin{yuanwen}

\end{yuanwen}\begin{yuanwen}

\end{yuanwen}\begin{yuanwen}

\end{yuanwen}\begin{yuanwen}

\end{yuanwen}\begin{yuanwen}

\end{yuanwen}\begin{yuanwen}

\end{yuanwen}\begin{yuanwen}

\end{yuanwen}\begin{yuanwen}

\end{yuanwen}\begin{yuanwen}

\end{yuanwen}\begin{yuanwen}

\end{yuanwen}\begin{yuanwen}

\end{yuanwen}\begin{yuanwen}

\end{yuanwen}\begin{yuanwen}

\end{yuanwen}\begin{yuanwen}

\end{yuanwen}\begin{yuanwen}

\end{yuanwen}\begin{yuanwen}

\end{yuanwen}\begin{yuanwen}

\end{yuanwen}\begin{yuanwen}

\end{yuanwen}\begin{yuanwen}

\end{yuanwen}\begin{yuanwen}

\end{yuanwen}\begin{yuanwen}

\end{yuanwen}\begin{yuanwen}

\end{yuanwen}\begin{yuanwen}

\end{yuanwen}\begin{yuanwen}

\end{yuanwen}\begin{yuanwen}

\end{yuanwen}\begin{yuanwen}

\end{yuanwen}\begin{yuanwen}

\end{yuanwen}\begin{yuanwen}

\end{yuanwen}\begin{yuanwen}

\end{yuanwen}\begin{yuanwen}

\end{yuanwen}\begin{yuanwen}

\end{yuanwen}\begin{yuanwen}

\end{yuanwen}\begin{yuanwen}

\end{yuanwen}\begin{yuanwen}

\end{yuanwen}\begin{yuanwen}

\end{yuanwen}\begin{yuanwen}

\end{yuanwen}\begin{yuanwen}

\end{yuanwen}\begin{yuanwen}

\end{yuanwen}\begin{yuanwen}

\end{yuanwen}\begin{yuanwen}

\end{yuanwen}\begin{yuanwen}

\end{yuanwen}\begin{yuanwen}

\end{yuanwen}\begin{yuanwen}

\end{yuanwen}\begin{yuanwen}

\end{yuanwen}\begin{yuanwen}

\end{yuanwen}\begin{yuanwen}

\end{yuanwen}\begin{yuanwen}

\end{yuanwen}\begin{yuanwen}




余至江南,观其行事,问其长老,云龟千岁乃游莲叶之上,蓍百茎共一根。又其所生,兽无虎狼,草无毒螫。江傍家人常畜龟饮食之,以为能导引致气,有益于助衰养老,岂不信哉!

褚先生曰:臣以通经术,受业博士,治春秋,以高第为郎,幸得宿,出入宫殿中十

有馀年。窃好太史公传。太史公之传曰:“三王不同龟,四夷各异卜,然各以决吉凶,略闚其要,故作龟策列传。”臣往来长安中,求龟策列传不能得,故之大卜官,问掌故文学长老习事者,写取龟策卜事,编于下方。

闻古五帝、三王发动举事,必先决蓍龟。传曰:“下有伏灵,上有兔丝;上有捣蓍,下有神龟。”所谓伏灵者,在兔丝之下,状似飞鸟之形。新雨已,天清静无风,以夜捎兔丝去之,既以鮸烛此地烛之,火灭,即记其处,以新布四丈环置之,明即掘取之,入四尺至七尺,得矣,过七尺不可得。伏灵者,千岁松根也,食之不死。闻蓍生满百茎者,其下必有神龟守之,其上常有青云覆之。传曰:“天下和平,王道得,而蓍茎长丈,其丛生满百茎。”方今世取蓍者,不能中古法度,不能得满百茎长丈者,取八十茎已上,蓍长八尺,即难得也。人民好用卦者,取满六十茎已上,长满六尺者,既可用矣。记曰:“能得名龟者,财物归之,家必大富至千万。”一曰“北斗龟”,二曰“南辰龟”,三曰“五星龟”,四曰“八风龟”,五曰“二十八宿龟”,六曰“日月龟”,七曰“九州龟”,八曰“玉龟”:凡八名龟。龟图各有文在腹下,文云云者,此某之龟也。略记其大指,不写其图。取此龟不必满尺二寸,民人得长七八寸,可宝矣。今夫珠玉宝器,虽有所深藏,必见其光,必出其神明,其此之谓乎!故玉处于山而木润,渊生珠而岸不枯者,润泽之所加也。明月之珠出于江海,藏于蚌中,蚗龙伏之。王得之,长有天下,四夷宾服。能得百茎蓍,并得其下龟以卜者,百言百当,足以决吉凶。

神龟出于江水中,庐江郡常岁时生龟长尺二寸者二十枚输太卜官,太卜官因以吉日剔取其腹下甲。龟千岁乃满尺二寸。王者发军行将,必钻龟庙堂之上,以决吉凶。今高庙中有龟室,藏内以为神宝。

传曰:“取前足臑骨穿佩之,取龟置室西北隅悬之,以入深山大林中,不惑。”臣为郎时,见万毕石硃方,传曰:“有神龟在江南嘉林中。嘉林者,兽无虎狼,鸟无鸱枭,草无毒螫,野火不及,斧斤不至,是为嘉林。龟在其中,常巢于芳莲之上。左胁书文曰:‘甲子重光,得我者匹夫为人君,有土正,诸侯得我为帝王。’ 求之于白蛇蟠杅林中者,斋戒以待,譺然,状如有人来告之,因以醮酒佗发,求之三宿而得。”由是观之,岂不伟哉!故龟可不敬与?

南方老人用龟支床足,行二十馀岁,老人死,移床,龟尚生不死。龟能行气导引。问者曰:“龟至神若此,然太卜官得生龟,何为辄杀取其甲乎?”近世江上人有得名龟,畜置之,家因大富。与人议,欲遣去。人教杀之勿遣,遣之破人家。龟见梦曰:“送我水中,无杀吾也。”其家终杀之。杀之后,身死,家不利。人民与君王者异道。人民得名龟,其状类不宜杀也。以往古故事言之,古明王圣主皆杀而用之。

宋元王时得龟,亦杀而用之。谨连其事于左方,令好事者观择其中焉。

宋元王二年,江使神龟使于河,至于泉阳,渔者豫且举网得而囚之。置之笼中。夜半,龟来见梦于宋元王曰:“我为江使于河,而幕网当吾路。泉阳豫且得我,我不能去。身在患中,莫可告语。王有德义,故来告诉。”元王惕然而悟。乃召博士卫平而问之曰:“今寡人梦见一丈夫,延颈而长头,衣玄绣之衣而乘辎车,来见梦于寡人曰:‘我为江使于河,而幕网当吾路。泉阳豫且得我,我不能去。身在患中,莫可告语。王有德义,故来告诉。’ 是何物也?”卫平乃援式而起,仰天而视月之光,观斗所指,定日处乡。规矩为辅,副以权衡。四维已定,八卦相望。视其吉凶,介蟲先见。乃对元王曰:“今昔壬子,宿在牵牛。河水大会,鬼神相谋。汉正南北,江河固期,南风新至,江使先来。白云壅汉,万物尽留。斗柄指日,使者当囚。玄服而乘辎车,其名为龟。王急使人问而求之。”王曰:“善。”

于是王乃使人驰而往问泉阳令曰:“渔者几何家?名谁为豫且?豫且得龟,见梦于王,王故使我求之。”泉阳令乃使吏案籍视图,水上渔者五十五家,上流之庐,名为豫且。泉阳令曰:“诺。”乃与使者驰而问豫且曰:“今昔汝渔何得?”豫且曰:“夜半时举网得龟。”使者曰:“今龟安在?”曰:“在笼中。”使者曰:“王知子得龟,故使我求之。”豫且曰:“诺。”即系龟而出之笼中,献使者。

使者载行,出于泉阳之门。正昼无见,风雨晦冥。云盖其上,五采青黄;雷雨并起,风将而行。入于端门,见于东箱。身如流水,润泽有光。望见元王,延颈而前,三步而止,缩颈而卻,复其故处。元王见而怪之,问卫平曰:“龟见寡人,延颈而前,以何望也?缩颈而复,是何当也?”卫平对曰:“龟在患中,而终昔囚,王有德义,使人活之。今延颈而前,以当谢也,缩颈而卻,欲亟去也。”元王曰:“善哉!神至如此乎,不可久留;趣驾送龟,勿令失期。”

卫平对曰:“龟者是天下之宝也,先得此龟者为天子,且十言十当,十战十胜。生于深渊,长于黄土。知天之道,明于上古。游三千岁,不出其域。安平静正,动不用力。寿蔽天地,莫知其极。与物变化,四时变色。居而自匿,伏而不食。春仓夏黄,秋白冬黑。明于阴阳,审于刑德。先知利害,察于祸福,以言而当,以战而胜,王能宝之,诸侯尽服。王勿遣也,以安社稷。”

元王曰:“龟甚神灵,降于上天,陷于深渊。在患难中。以我为贤。德厚而忠信,故来告寡人。寡人若不遣也,是渔者也。渔者利其肉,寡人贪其力,下为不仁,上为无德。君臣无礼,何从有福?寡人不忍,柰何勿遣!”

卫平对曰:“不然。臣闻盛德不报,重寄不归;天与不受,天夺之宝。今龟周流天下,还复其所,上至苍天,下薄泥涂。还遍九州,未尝愧辱,无所稽留。今至泉阳,渔者辱而囚之。王虽遣之,江河必怒,务求报仇。自以为侵,因神与谋。淫雨不霁,水不可治。若为枯旱,风而扬埃,蝗蟲暴生,百姓失时。王行仁义,其罚必来。此无佗故,其祟在龟。后虽悔之,岂有及哉!王勿遣也。”

元王慨然而叹曰:“夫逆人之使,绝人之谋,是不暴乎?取人之有,以自为宝,是不彊乎?寡人闻之,暴得者必暴亡,彊取者必后无功。桀纣暴彊,身死国亡。今我听子,是无仁义之名而有暴彊之道。江河为汤武,我为桀纣。未见其利,恐离其咎。寡人狐疑,安事此宝,趣驾送龟,勿令久留。”卫平对曰:“不然,王其无患。天地之间,累石为山。高而不坏,地得为安。故云物或危而顾安,或轻而不可迁;人或忠信而不如诞谩,或丑恶而宜大官,或美好佳丽而为众人患。非神圣人,莫能尽言。春秋冬夏,或暑或寒。寒暑不和,贼气相奸。同岁异节,其时使然。故令春生夏长,秋收冬藏。或为仁义,或为暴彊。暴彊有乡,仁义有时。万物尽然,不可胜治。大王听臣,臣请悉言之。天出五色,以辨白黑。地生五穀,以知善恶。人民莫知辨也,与禽兽相若。谷居而穴处,不知田作。天下祸乱,阴阳相错。疾疾,通而不相择。妖鮟数见,传为单薄。圣人别其生,使无相获。禽兽有牝牡,置之山原;鸟有雌雄,布之林泽;有介之蟲,置之谿谷。故牧人民,为之城郭,内经闾术,外为阡陌。夫妻男女,赋之田宅,列其室屋。为之图籍,别其名族。立官置吏,劝以爵禄。衣以桑麻,养以五穀。耕之櫌之,鉏之耨之。口得所嗜,目得所美,身受其利。以是观之,非彊不至。故曰田者不彊,囷仓不盈;商贾不彊,不得其赢;妇女不彊,布帛不精;官御不彊,其势不成;大将不彊,卒不使令;侯王不彊,没世无名。故云彊者,事之始也,分之理也,物之纪也。所求于彊,无不有也。王以为不然,王独不闻玉椟只雉,出于昆山;明月之珠,出于四海;镌石拌蚌,传卖于市;圣人得之,以为大宝。大宝所在,乃为天子。今王自以为暴,不如拌蚌于海也;自以为彊,不过镌石于昆山也。取者无咎,宝者无患。今龟使来抵网,而遭渔者得之,见梦自言,是国之宝也,王何忧焉。”

元王曰:“不然。寡人闻之,谏者福也,谀者贼也。人主听谀,是愚惑也。虽然,祸不妄至,福不徒来。天地合气,以生百财。阴阳有分,不离四时,十有二月,日至为期。圣人彻焉,身乃无灾。明王用之,人莫敢欺。故云福之至也,人自生之;祸之至也,人自成之。祸与福同,刑与德双。圣人察之,以知吉凶。桀纣之时,与天争功,拥遏鬼神,使不得通。是固已无道矣,谀臣有众。桀有谀臣,名曰赵梁。教为无道,劝以贪狼。系汤夏台,杀关龙逢。左右恐死,偷谀于傍。国危于累卵,皆曰无伤。称乐万岁,或曰未央。蔽其耳目,与之诈狂。汤卒伐桀,身死国亡。听其谀臣,身独受殃。春秋著之,至今不忘。纣有谀臣,名为左彊。夸而目巧,教为象郎。将至于天,又有玉床。犀玉之器,象箸而羹。圣人剖其心,壮士斩其胻。箕子恐死,被发佯狂。杀周太子历,囚文王昌。投之石室,将以昔至明。阴兢活之,与之俱亡。入于周地,得太公望。兴卒聚兵,与纣相攻。文王病死,载尸以行。太子发代将,号为武王。战于牧野,破之华山之阳。纣不胜败而还走,围之象郎。自杀宣室,身死不葬。头悬车轸,四马曳行。寡人念其如此,肠如涫汤。是人皆富有天下而贵至天子,然而大傲。欲无厌时,举事而喜高,贪很而骄。不用忠信,听其谀臣,而为天下笑。今寡人之邦,居诸侯之间,曾不如秋毫。举事不当,又安亡逃!”卫平对曰:“不然。河虽神贤,不如昆仑之山;江之源理,不如四海,而人尚夺取其宝,诸侯争之,兵革为起。小国见亡,大国危殆,杀人父兄,虏人妻子,残国灭庙,以争此宝。战攻分争,是暴彊也。故云取之以暴彊而治以文理,无逆四时,必亲贤士;与阴阳化,鬼神为使;通于天地,与之为友。诸侯宾服,民众殷喜。邦家安宁,与世更始。汤武行之,乃取天子;春秋著之,以为经纪。王不自称汤武,而自比桀纣。桀纣为暴彊也,固以为常。桀为瓦室,纣为象郎。徵丝灼之,务以费。赋敛无度,杀戮无方。杀人六畜,以韦为囊。囊盛其血,与人县而射之,与天帝争彊。逆乱四时,先百鬼尝。谏者辄死,谀者在傍。圣人伏匿,百姓莫行。天数枯旱,国多妖祥。螟蟲岁生,五穀不成。民不安其处,鬼神不享。飘风日起,正昼晦冥。日月并蚀,灭息无光。列星奔乱,皆绝纪纲。以是观之,安得久长!虽无汤武,时固当亡。故汤伐桀,武王剋纣,其时使然。乃为天子,子孙续世;终身无咎,后世称之,至今不已。是皆当时而行,见事而彊,乃能成其帝王。今龟,大宝也,为圣人使,传之贤。不用手足,雷电将之;风雨送之,流水行之。侯王有德,乃得当之。今王有德而当此宝,恐不敢受;王若遣之,宋必有咎。后虽悔之,亦无及已。”元王大悦而喜。于是元王向日而谢,再拜而受。择日斋戒,甲乙最良。乃刑白雉,及与骊羊;以血灌龟,于坛中央。以刀剥之,身全不伤。脯酒礼之,横其腹肠。荆支卜之,必制其创。理达于理,文相错迎。使工占之,所言尽当。邦福重宝,闻于傍乡。杀牛取革,被郑之桐。草木毕分,化为甲兵。战胜攻取,莫如元王。元王之时,卫平相宋,宋国最彊,龟之力也。

故云神至能见梦于元王,而不能自出渔者之笼。身能十言尽当,不能通使于河,还报于江,贤能令人战胜攻取,不能自解于刀锋,免剥刺之患。圣能先知亟见,而不能令卫平无言。言事百全,至身而挛;当时不利,又焉事贤!贤者有恆常,士有適然。是故明有所不见,听有所不闻;人虽贤,不能左画方,右画圆;日月之明,而时蔽于浮云。羿名善射,不如雄渠、门;禹名为辩智,而不能胜鬼神。地柱折,天故毋椽,又柰何责人于全?孔子闻之曰:“神龟知吉凶,而骨直空枯。日为德而君于天下,辱于三足之乌。月为刑而相佐,见食于虾蟆。蝟辱于鹊,腾蛇之神而殆于即且。竹外有节理,中直空虚;松柏为百木长,而守门闾。日辰不全,故有孤虚。黄金有疵,白玉有瑕。事有所疾,亦有所徐。物有所拘,亦有所据。罔有所数,亦有所疏。人有所贵,亦有所不如。何可而適乎?物安可全乎?天尚不全,故世为屋,不成三瓦而陈之,以应之天。天下有阶,物不全乃生也。”

褚先生曰:渔者举网而得神龟,龟自见梦宋元王,元王召博士卫平告以梦龟状,平运式,定日月,分衡度,视吉凶,占龟与物色同,平谏王留神龟以为国重宝,美矣。古者筮必称龟者,以其令名,所从来久矣。余述而为传。

三月二月正月

十二月十一月中关内高外下四月首仰足开肣开首俛大五月横吉首俛大六月七月八月九月十月

卜禁曰:子亥戌不可以卜及杀龟。日中如食已卜。暮昏龟之徼也,不可以卜。庚辛可以杀,及以钻之。常以月旦祓龟,先以清水澡之,以卵祓之,乃持龟而遂之,若常以为祖。人若已卜不中,皆祓之以卵,东向立,灼以荆若刚木,土卵指之者三,持龟以卵周环之,祝曰:“今日吉,谨以粱卵弟黄祓去玉灵之不祥。”玉灵必信以诚,知万事之情,辩兆皆可占。不信不诚,则烧玉灵,扬其灰,以徵后龟。其卜必北向,龟甲必尺二寸。

卜先以造灼钻,钻中已,又灼龟首,各三;又复灼所钻中曰正身,灼首曰正足,各三。即以造三周龟,祝曰:“假之玉灵夫子。夫子玉灵,荆灼而心,令而先知。而上行于天,下行于渊,诸灵数k,莫如汝信。今日良日,行一良贞。某欲卜某,即得而喜,不得而悔。即得,发乡我身长大,首足收人皆上偶。不得,发乡我身挫折,中外不相应,首足灭去。”

灵龟卜祝曰:“假之灵龟,五巫五灵,不如神龟之灵,知人死,知人生。某身良贞,某欲求某物。即得也,头见足发,内外相应;即不得也,头仰足肣,内外自垂。可得占。”

卜占病者祝曰:“今某病困。死,首上开,内外交骇,身节折;不死,首仰足肣。”

卜病者祟曰:“今病有祟无呈,无祟有呈。兆有中祟有内,外祟有外。”

卜系者出不出。不出,横吉安;若出,足开首仰有外。

卜求财物,其所当得。得,首仰足开,内外相应;即不得,呈兆首仰足肣。

卜有卖若买臣妾马牛。得之,首仰足开,内外相应;不得,首仰足肣,呈兆若横吉安。

卜击盗聚若干人,在某所,今某将卒若干人,往击之。当胜,首仰足开身正,内自桥,外下;不胜,足肣首仰,身首内下外高。

卜求当行不行。行,首足开;不行,足肣首仰,若横吉安,安不行。

卜往击盗,当见不见。见,首仰足肣有外;不见,足开首仰。

卜往候盗,见不见。见,首仰足肣,肣胜有外;不见,足开首仰。

卜闻盗来不来。来,外高内下,足肣首仰;不来,足开首仰,若横吉安,期之自次。

卜迁徙去官不去。去,足开有肣外首仰;不去,自去,即足肣,呈兆若横吉安。

卜居官尚吉不。吉,呈兆身正,若横吉安;不吉,身节折,首仰足开。

卜居室家吉不吉。吉,呈兆身正,若横吉安;不吉,身节折,首仰足开。

卜岁中禾稼孰不孰。孰,首仰足开,内外自桥外自垂;不孰,足肣首仰有外。

卜岁中民疫不疫。疫,首仰足肣,身节有彊外;不疫,身正首仰足开。卜岁中有兵无兵。无兵,呈兆若横吉安;有兵,首仰足开,身作外彊情。

卜见贵人吉不吉。吉,足开首仰,身正,内自桥;不吉,首仰,身节折,足肣有外,若无渔。

卜请谒于人得不得。得,首仰足开,内自桥;不得,首仰足肣有外。

卜追亡人当得不得。得,首仰足肣,内外相应;不得,首仰足开,若横吉安。

卜渔猎得不得。得,首仰足开,内外相应;不得,足肣首仰,若横吉安。

卜行遇盗不遇。遇,首仰足开,身节折,外高内下;不遇,呈兆。

卜天雨不雨。雨,首仰有外,外高内下;不雨,首仰足开,若横吉安。

卜天雨霁不霁。霁,呈兆足开首仰;不霁,横吉。

命曰横吉安。以占病,病甚者一日不死;不甚者卜日瘳,不死。系者重罪不出,轻罪环出;过一日不出,久毋伤也。求财物买臣妾马牛,一日环得;过一日不得。行者不行。来者环至;过食时不至,不来。击盗不行,行不遇;闻盗不来。徙官不徙。居官家室皆吉。岁稼不孰。民疾疫无疾。岁中无兵。见人行,不行不喜。请谒人不行不得。追亡人渔猎不得。行不遇盗。雨不雨。霁不霁。

命曰呈兆。病者不死。系者出。行者行。来者来。市买得。追亡人得,过一日不得。问行者不到。

命曰柱彻。卜病不死。系者出。行者行。来者来。市买不得。忧者毋忧。追亡人不得。

命曰首仰足肣有内无外。占病,病甚不死。系者解。求财物买臣妾马牛不得。行者闻言不行。来者不来。闻盗不来。闻言不至。徒官闻言不徙。居官有忧。居家多灾。岁稼中孰。民疾疫多病。岁中有兵,闻言不开。见贵人吉。请谒不行,行不得善言。追亡人不得。渔猎不得。行不遇盗。雨不雨甚。霁不霁。故其莫字皆为首备。问之曰,备者仰也,故定以为仰。此私记也。

命曰首仰足肣有内无外。占病,病甚不死。系者不出。求财买臣妾不得。行者不行。来者不来。击盗不见。闻盗来,内自惊,不来。徙官不徙。居官家室吉。岁稼不孰。民疾疫有病甚。岁中无兵。见贵人吉。请谒追亡人不得。亡财物,财物不出得。渔猎不得。行不遇盗。雨不雨。霁不霁。凶。

命曰呈兆首仰足肣。以占病,不死。系者未出。求财物买臣妾马牛不得。行不行。来不来。击盗不相见。闻盗来不来。徙官不徙。居官久多忧。居家室不吉。岁稼不孰。民病疫。岁中毋兵。见贵人不吉。请谒不得。渔猎得少。行不遇盗。雨不雨。霁不霁。不吉。

命曰呈兆首仰足开。以占病,病笃死。系囚出。求财物买臣妾马牛不得。行者行。来者来。击盗不见盗。闻盗来不来。徙官徙。居官不久。居家室不吉。岁稼不孰。民疾疫有而少。岁中毋兵。见贵人不见吉。请谒追亡人渔猎不得。行遇盗。雨不雨。霁小吉。

命曰首仰足肣。以占病,不死。系者久,毋伤也。求财物买臣妾马牛不得。行者不行。击盗不行。来者来。闻盗来。徙官闻言不徙。居家室不吉。岁稼不孰。民疾疫少。岁中毋兵。见贵人得见。请谒追亡人渔猎不得。行遇盗。雨不雨。霁不霁。吉。

命曰首仰足开有内。以占病者,死。系者出。求财物买臣妾马牛不得。行者行。来者来。击盗行不见盗。闻盗来不来。徙官徙。居官不久。居家室不吉。岁孰。民疾疫有而少。岁中毋兵。见贵人不吉。请谒追亡人渔猎不得。行不遇盗。雨霁。霁小吉,不霁吉。

命曰横吉内外自桥。以占病,卜日毋瘳死。系者毋罪出。求财物买臣妾马牛得。行者行。来者来。击盗合交等。闻盗来来。徙官徙。居家室吉。岁孰。民疫无疾。岁中无兵。见贵人请谒追亡人渔猎得。行遇盗。雨霁,雨霁大吉。

命曰横吉内外自吉。以占病,病者死。系不出。求财物买臣妾马牛追亡人渔猎不得。行者不来。击盗不相见。闻盗不来。徙官徙。居官有忧。居家室见贵人请谒不吉。岁稼不孰。民疾疫。岁中无兵。行不遇盗。雨不雨。霁不霁。不吉。

命曰渔人。以占病者,病者甚,不死。系者出。求财物买臣妾马牛击盗请谒追亡人渔猎得。行者行来。闻盗来不来。徙官不徒。居家室吉。岁稼不孰。民疾疫。岁中毋兵。见贵人吉。行不遇盗。雨不雨。霁不霁。吉。

命曰首仰足肣内高外下。以占病,病者甚,不死。系者不出。求财物买臣妾马牛追亡人渔猎得。行不行。来者来。击盗胜。徙官不徙。居官有忧,无伤也。居家室多忧病。岁大孰。民疾疫。岁中有兵不至。见贵人请谒不吉。行遇盗。雨不雨。霁不霁。吉。

命曰横吉上有仰下有柱。病久不死。系者不出。求财物买臣妾马牛追亡人渔猎不得。行不行。来不来。击盗不行,行不见。闻盗来不来。徙官不徙。居家室见贵人吉。岁大孰。民疾疫。岁中毋兵。行不遇盗。雨不雨。霁不霁。大吉。

命曰横吉榆仰。以占病,不死。系者不出。求财物买臣妾马牛至不得。行不行。来不来。击盗不行,行不见。闻盗来不来。徙官不徙。居官家室见贵人吉。岁孰。岁中有疾疫,毋兵。请谒追亡人不得。渔猎至不得。行不得。行不遇盗。雨霁不霁。小吉。

命曰横吉下有柱。以占病,病甚不环有瘳无死。系者出。求财物买臣妾马牛请谒追亡人渔猎不得。行来不来。击盗不合。闻盗来来。徙官居官吉,不久。居家室不吉。岁不孰。民毋疾疫。岁中毋兵。见贵人吉。行不遇盗。雨不雨。霁。小吉。

命曰载所。以占病,环有瘳无死。系者出。求财物买臣妾马牛请谒追亡人渔猎得。行者行。来者来。击盗相见不相合。闻盗来来。徙官徙。居家室忧。见贵人吉。岁孰。民毋疾疫。岁中毋兵。行不遇盗。雨不雨。霁霁。吉。

命曰根格。以占病者,不死。系久毋伤。求财物买臣妾马牛请谒追亡人渔猎不得。行不行。来不来。击盗盗行不合。闻盗不来。徙官不徙。居家室吉。岁稼中。民疾疫无死。见贵人不得见。行不遇盗。雨不雨。不吉。

命曰首仰足肣外高内下。卜有忧,无伤也。行者不来。病久死。求财物不得。见贵人者吉。

命曰外高内下。卜病不死,有祟。市买不得。居官家室不吉。行者不行。来者不来。系者久毋伤。吉。

命曰头见足发有内外相应。以占病者,起。系者出。行者行。来者来。求财物得。吉。

命曰呈兆首仰足开。以占病,病甚死。系者出,有忧。求财物买臣妾马牛请谒追亡人渔猎不得。行不行。来不来。击盗不合。闻盗来来。徙官居官家室不吉。岁恶。民疾疫无死。岁中毋兵。见贵人不吉。行不遇盗。雨不雨。霁。不吉。

命曰呈兆首仰足开外高内下。以占病,不死,有外祟。系者出,有忧。求财物买臣妾马牛,相见不会。行行。来闻言不来。击盗胜。闻盗来不来。徙官居官家室见贵人不吉。岁中。民疾疫有兵。请谒追亡人渔猎不得。闻盗遇盗。雨不雨。霁。凶。

命曰首仰足肣身折内外相应。以占病,病甚不死。系者久不出。求财物买臣妾马牛渔猎不得。行不行。来不来。击盗有用胜。闻盗来来。徙官不徙。居官家室不吉。岁不孰。民疾疫。岁中。有兵不至。见贵人喜。请谒追亡人不得。遇盗凶。

命曰内格外垂。行者不行。来者不来。病者死。系者不出。求财物不得。见人不见。大吉。

命曰横吉内外相应自桥榆仰上柱足肣。以占病,病甚不死。系久,不抵罪。求财物买臣妾马牛请谒追亡人渔猎不得。行不行。来不来。居官家室见贵人吉。徙官不徙。岁不大孰。民疾疫有兵。有兵不会。行遇盗。闻言不见。雨不雨。霁霁。大吉。

命曰头仰足肣内外自垂。卜忧病者甚,不死。居官不得居。行者行。来者不来。求财物不得。求人不得。吉。

命曰横吉下有柱。卜来者来。卜日即不至,未来。卜病者过一日毋瘳死。行者不行。求财物不得。系者出。

命曰横吉内外自举。以占病者,久不死。系者久不出。求财物得而少。行者不行。来者不来。见贵人见。吉。

命曰内高外下疾轻足发。求财物不得。行者行。病者有瘳。系者不出。来者来。见贵人不见。吉。

命曰外格。求财物不得。行者不行。来者不来。系者不出。不吉。病者死。求财物不得。见贵人见。吉。

命曰内自举外来正足发。者行。来者来。求财物得。病者久不死。系者不出。见贵人见。吉。

此横吉上柱外内自举足肣。以卜有求得。病不死。系者毋伤,未出。行不行。来不来。见人不见。百事尽吉。

此横吉上柱外内自举柱足以作。以卜有求得。病死环起。系留毋伤,环出。行不行。来不来。见人不见。百事吉。可以举兵。

此挺诈有外。以卜有求不得。病不死,数起。系祸罪。闻言毋伤。行不行。来不来。

此挺诈有内。以卜有求不得。病不死,数起。系留祸罪无伤出。行不行。来者不来。见人不见。

此挺诈内外自举。以卜有求得。病不死。系毋罪。行行。来来。田贾市渔猎尽喜。

此狐鲗。以卜有求不得。病死,难起。系留毋罪难出。可居宅。可娶妇嫁女。行不行。来不来。见人不见。有忧不忧。

此狐彻。以卜有求不得。病者死。系留有抵罪。行不行。来不来。见人不见。言语定。百事尽不吉。

此首俯足肣身节折。以卜有求不得。病者死。系留有罪。望行者不来。行行。来不来。见人不见。

此挺内外自垂。以卜有求不晦。病不死,难起。系留毋罪,难出。行不行。来不来。见人不见。不吉。

此横吉榆仰首俯。以卜有求难得。病难起,不死。系难出,毋伤也。可居家室,以娶妇嫁女。

此横吉上柱载正身节折内外自举。以卜病者,卜日不死,其一日乃死。

此横吉上柱足肣内自举外自垂。以卜病者,卜日不死,其一日乃死。

首俯足诈有外无内。病者占龟未已,急死。卜轻失大,一日不死。

首仰足肣。以卜有求不得。以系有罪。人言语恐之毋伤。行不行。见人不见。

大论曰:外者人也,内者自我也;外者女也,内者男也。首俯者忧。大者身也,小者枝也。大法,病者,足肣者生,足开者死。行者,足开至,足肣者不至。行者,足肣不行,足开行。有求,足开得,足肣者不得。系者,足肣不出,开出。其卜病也,足开而死者,内高而外下也。

三王异龟,五帝殊卜。或长或短,若瓦若玉。其记已亡,其繇后续。江使触网,见留宋国。神能讬梦,不卫其足。
\end{yuanwen}

\part{卷一百二十九}

\chapter{货殖列传第六十九}

\begin{yuanwen}
《老子》曰:“至治之极,邻国相望,鸡狗之声相闻,民各甘其食,美其服,安其俗,乐其业,至老死不相往来。”必用此为务,輓(挽)近世涂\footnote{堵塞。}民耳目,则几无行矣。
\end{yuanwen}

《老子》一书说:“推行政治的极致,就是毗邻的两个国家能够远远地互相望见,鸡鸣狗吠彼此之间能够听到,百姓各自认为自己所吃的食物是最美味的,认为自己所穿的服装是最漂亮的,安于本地的习俗习惯,乐于经营自己所从事的事业,这样一直到年老死亡,彼此也不相往来。”如果一定要将这些作为此生追求的目标,就近代而言,除非堵塞住百姓的耳朵和眼睛,否则几乎没有办法实现。

\begin{yuanwen}
太史公曰:夫神农以前,吾不知已。至若《诗》、《书》所述虞夏以来,耳目欲极声色之好,口欲穷刍豢\footnote{牛羊犬猪。}之味,身安逸乐,而心夸矜埶能之荣。使俗之渐\footnote{渐渐形成。}民久矣,虽户说以眇\footnote{同“妙”。}论,终不能化。故善者因之,其次利道之,其次教诲之,其次整齐之,最下者与之争\footnote{争利。}。
\end{yuanwen}

太史公说:神农氏之前的事情,我已经不知道了。至于像《诗》《书》中所记载的虞、夏以来的情况,耳朵眼睛都要享受歌舞和女色的美好,嘴巴要品尝各种山珍肉类的美味,身体安于享受舒适快乐,心中想着向别人夸耀极致铺张的权势和荣华。这样的恶习慢慢侵染百姓,已经很久了。就算是用美好的理论去挨家挨户地劝说开导,最终也无法让他们感化。所以最好的办法就是顺其自然,其次是因势利导,然后才是对他们进行教诲,再其次是整顿他们的行为使之整齐划一,最下等的方法是与他们争利。

\begin{yuanwen}
夫山西\footnote{崤山以西。}饶材\footnote{木材。}、竹、(穀/榖)、纑、旄\footnote{牦牛。}、玉石;山东多鱼、盐、漆、丝、声色;江南出(棻/柟)、梓、姜、桂、金、锡、连\footnote{通“链”,铅矿石。}、丹沙、犀\footnote{犀牛角。}、玳瑁、珠玑、齿革;龙门、碣石\footnote{山名,位于今河北省昌黎附近。}北多马、牛、羊、旃裘、筋角;铜、铁则千里往往山出(釭/棋)置:此其大较\footnote{大致。}也。皆中国人民所喜好,谣俗被服饮食奉生送死之具也。故待\footnote{依靠。}农而食之,虞而出之,工而成之,商而通之。此宁有政教\footnote{政令。}发徵期会哉?人各任其能,竭其力,以得所欲。故物贱之徵贵,贵之徵贱,各劝其业,乐其事,若水之趋下,日夜无休时,不召而自来,不求而民出之。岂非道之所符,而自然之验邪?
\end{yuanwen}

崤山以西的地区盛产木材、竹子、榖木、苎麻、牦牛、玉石;崤山以东的地区盛产鱼、盐、漆、蚕丝、音乐以及美女;长江以南的地区盛产柟木、梓木、生姜、桂皮、金、锡、铅、朱砂、犀牛角、玳瑁、珠玑、兽牙和皮革;龙门、碣石以北的地区盛产马、牛、羊、毛毡、皮裘、兽筋和角;铜、铁都是在方圆千里之内,通常出产于山中,如同棋盘上的棋子一样零散分布。这就是各地物产资源的大致分布情况。所有这些东西都是中国百姓喜爱的,俗语中所说的民间用于服装、饮食、养生、送葬等方面的物品。所以人们要依赖农民出产粮食,要依赖负责山林水泽开发的虞人才能运送物品,要依赖工匠将它们制成器物,要依赖商人进行货物流通。这些难道需要依靠官府的政令教化来调发征召才能约期相会吗?人人各尽其能,各尽其力,以此来得到自己想要的东西。所以某个东西价格便宜的时候人人都会购买导致物价上涨,价格昂贵的时候人人都不会购买致使物价下跌。人们各尽其能,努力经营自己的本业,愉快地从事自己所做的事,如同水往低处流,无论白天晚上都没有停止的时候,不需要征召它就会主动前来,不需要强求百姓他们就会主动生产物品。这不正与“道”互相吻合,顺应自然的验证吗?

\begin{yuanwen}
《周书》曰:“农不出则乏其食,工不出则乏其事,商不出则三宝绝,虞不出则财匮少。”财匮少而山泽不辟矣。此四者,民所衣食之原也。原大则饶,原小则鲜\footnote{小。}。上则富国,下则富家。贫富之道,莫之夺予\footnote{改变或保持原样。},而巧者有馀,拙者不足。故太公望封于营丘,地潟卤\footnote{盐碱地。},人民寡,于是太公劝其女功\footnote{也作“女红”,指女子从事的织补、刺绣等事。},极技巧,通鱼盐,则人物归之,繦至\footnote{绳索连接不断的样子。}而辐凑\footnote{形容人群聚集。}。故齐冠带衣履天下,海岱之间敛袂\footnote{收敛衣袖,以示恭敬。}而往朝焉。其后齐中衰,管子修之,设轻重九府,则桓公以霸,九合诸侯,一匡天下;而管氏亦有三归,位在陪臣,富于列国之君。是以齐富彊至于威、宣也。
\end{yuanwen}

《周书》上说:“农民不生产粮食,粮食就会出现短缺;工匠不生产器物,器物就会短缺;商人不进行贸易往来,粮食、器物、财富等生活三宝就会彼此隔绝,无法流通,虞人如果不生产,财货就会匮乏缺少。财物匮乏缺少,那么山林水泽就没有办法做进一步的开发了。”这四个方面,是百姓衣食住行的根本所在。来源广大人民生活水平就富饶,来源狭小人民生活水平就贫困。上可以使国家富强,下可以让家庭富足。贫穷或富足的方法,没有人能够夺走或是给予,聪明的人自然会富裕有余,笨拙的人只能贫穷不足。所以,从前太公望在营丘接受封赏,那里是一块盐碱地,地广人稀,于是太公望鼓励营丘的妇女从事女红,女红手艺达到极高的境界,同时还在当地开通了渔产品和海盐的贸易,百姓和财物因此而涌向齐国,如同绳索相连,不断前来,如同车辐集聚在车毂一样,从四方八面汇聚到齐国。因此齐国人生产的帽子、束带、衣服、鞋履能够供应整个天下,东海与泰山之间的诸侯们都整理好自己的衣服,恭恭敬敬地朝拜齐国。后来齐国中道衰落,管仲重新修订了太公望所制定的政策,设置了掌管财物的九个部门,齐桓公才得以称霸诸侯,多次联合诸侯,将天下纳入正道;而管仲也因此获得三归台,他虽然只是居于陪臣的地位,但是实际上比各诸侯国的君主还要富有。正是因为这样,齐国的富裕强盛才一直延续到威王、宣王的时代。

\begin{yuanwen}
故曰:“仓廪实而知礼节,衣食足而知荣辱。”礼生于有而废于无。故君子富,好行其德;小人富,以適其力。渊深而鱼生之,山深而兽往之,人富而仁义附焉。富者得势益彰,失势则客无所之,以而不乐。夷狄益甚。谚曰:“千金之子,不死于市。”此非空言也。故曰:“天下熙熙,皆为利来;天下壤壤\footnote{通“攘攘”,纷乱的样子,与“熙熙”同义。},皆为利往。”夫千乘之王,万家之侯,百室之君,尚犹患贫,而况匹夫\footnote{平民百姓。}编户\footnote{编入户籍的平民。}之民乎!
\end{yuanwen}

因此说:“粮库府库充足百姓才能够知道礼节,衣服饮食丰足百姓才能够知道荣辱。”礼仪产生在富有的生活之中,而被废弃于贫困之时。因此君子富有的时候,就喜欢施行自己的仁德;平民富足的时候,就会安心舒适地享受生活。水足够深就会有鱼生存在其中,山足够高就会有野兽前往那里,人足够富裕自然就会有仁义附益于他。富人得势之后,身份地位会更加显赫;一旦失势以,就连门客也无处容身,所以郁郁寡欢。这样的情形在少数民族的蛮夷地区更加严重。民间有句谚语说:“财产有千斤的子弟,不会在闹市上受刑而死。”这并不是一句空话。因此说:“天下人欢欢喜喜,都是为了追求利益;天下人吵吵嚷嚷,都是为了追逐利益。”那些拥有千乘战车的君主,封有万家食邑的列侯,享有百家供奉的君子,尚且担心自己会陷于贫困,何况只是被编入户籍的普通老百姓呢!

\begin{yuanwen}
昔者越王句践困于会稽之上,乃用范蠡、计然。计然曰:“知斗则修备,时用则知物,二者形则万货之情可得而观已。故岁\footnote{岁星。}在金\footnote{西方。},穰\footnote{丰收。};水\footnote{北方。},毁\footnote{歉收。};木\footnote{东方。},饥;火\footnote{南方。},旱。旱则资\footnote{储备。}舟,水则资车,物之理也。六岁穰,六岁旱,十二岁一大饥。夫粜\footnote{出售粮食。},二十病农,九十病末\footnote{指商业。}。末病则财不出,农病则草不辟矣。上不过八十,下不减三十,则农末俱利,平粜齐物,关市不乏,治国之道也。积著\footnote{储藏货物。}之理,务完物,无息币\footnote{积压的钱。}。以物相贸易,腐败而食之货勿留,无敢居贵。论其有馀不足,则知贵贱。贵上极则反贱,贱下极则反贵。贵出如粪土,贱取如珠玉。财币欲其行如流水。”

修之十年,国富,厚赂\footnote{奖赏。}战士,士赴矢石,如渴得饮,遂报彊吴,观兵中国,称号“五霸”。
\end{yuanwen}

从前,越王句践被围困在会稽山上,于是重用了范蠡、计然。计然说:“知道要战争,就要事先做好战备;只有了解了货物的生产时节和用途,才算是真正了解货物。时节和需要二者相对照,那么各种货物的供需情况就都能够掌握得十分清楚了。因此岁星行至金位的时候,国家就会五谷丰登;行至水位的时候,粮食就会歉收;行至木位的时候,就会发生饥荒;行至火位的时候,就会发生旱灾。干旱的时候就要储备舟船,洪水的时候就要储备车辆,这就是事物发展变化的规律。农业生产通常六年丰收,六年干旱,十二年就要发生一次大范围的饥荒。出售粮食,如果粮价每斗在二十钱,这就损害农民的利益;如果粮价每斗九十钱,就会损害商人的利益。如果商人利益受损,那么钱财就流通得不顺畅;如果农民利益受损,就不会再开垦土地。粮价每斗的价钱,向上不能高于八十钱,向下不能低于三十钱,这样农民和商人才能都获得利益。官府以平价出售粮食,控制物价,使关卡的税收和市场的供应能够源源不断,这就是治理国家的正道。至于积贮货物的常理,一定是要积贮那些完好无损适合久存的货物,以免资金周转不开。用货物和货物进行贸易,容易腐败和腐蚀的货物不要保留太长时间,不要囤居这样的货物以谋求高价。能够分析研究出哪种货物供过于求,哪种货物供不应求,就能掌握物价上涨与下跌的趋势。物价上涨到了极致就会归于低,物价下跌到了极点也一定会归于高。物价非常高的时候,将手中的货物如同丢弃像粪土一样立即抛出;物价低的时候,应该将低价买进的货物视为珍珠翡翠一样囤积。钱财就会像流水那样流通自如,周转灵活。”

句践依照这个主张治理国家十年,国家富足,句践能够丰厚地犒赏战士,所以战士们个个英勇无比,迎着敌人的箭矢飞石,奋勇前进,如同口渴的人想要喝水一样,最终报仇雪恨,灭掉了强大的吴国。句践又率军北上显示军威,号称“五霸”之一。

\begin{yuanwen}

\end{yuanwen}\begin{yuanwen}

\end{yuanwen}\begin{yuanwen}

\end{yuanwen}\begin{yuanwen}

\end{yuanwen}\begin{yuanwen}

\end{yuanwen}\begin{yuanwen}

\end{yuanwen}\begin{yuanwen}

\end{yuanwen}\begin{yuanwen}

\end{yuanwen}\begin{yuanwen}

\end{yuanwen}\begin{yuanwen}

\end{yuanwen}\begin{yuanwen}

\end{yuanwen}\begin{yuanwen}

\end{yuanwen}\begin{yuanwen}

\end{yuanwen}\begin{yuanwen}

\end{yuanwen}\begin{yuanwen}

\end{yuanwen}\begin{yuanwen}

\end{yuanwen}\begin{yuanwen}

\end{yuanwen}\begin{yuanwen}

\end{yuanwen}\begin{yuanwen}

\end{yuanwen}\begin{yuanwen}

\end{yuanwen}\begin{yuanwen}

\end{yuanwen}\begin{yuanwen}

\end{yuanwen}\begin{yuanwen}

\end{yuanwen}\begin{yuanwen}

\end{yuanwen}\begin{yuanwen}

\end{yuanwen}\begin{yuanwen}

\end{yuanwen}\begin{yuanwen}

\end{yuanwen}\begin{yuanwen}

\end{yuanwen}\begin{yuanwen}

\end{yuanwen}\begin{yuanwen}

\end{yuanwen}\begin{yuanwen}

\end{yuanwen}\begin{yuanwen}

\end{yuanwen}\begin{yuanwen}

\end{yuanwen}\begin{yuanwen}

\end{yuanwen}\begin{yuanwen}

\end{yuanwen}\begin{yuanwen}

\end{yuanwen}\begin{yuanwen}

\end{yuanwen}\begin{yuanwen}

\end{yuanwen}\begin{yuanwen}

\end{yuanwen}\begin{yuanwen}

\end{yuanwen}\begin{yuanwen}

\end{yuanwen}\begin{yuanwen}

\end{yuanwen}\begin{yuanwen}

\end{yuanwen}\begin{yuanwen}

\end{yuanwen}\begin{yuanwen}

\end{yuanwen}\begin{yuanwen}

\end{yuanwen}\begin{yuanwen}

\end{yuanwen}\begin{yuanwen}

\end{yuanwen}\begin{yuanwen}

\end{yuanwen}\begin{yuanwen}

\end{yuanwen}\begin{yuanwen}

\end{yuanwen}\begin{yuanwen}

\end{yuanwen}\begin{yuanwen}

\end{yuanwen}\begin{yuanwen}

\end{yuanwen}\begin{yuanwen}

\end{yuanwen}\begin{yuanwen}




范蠡既雪会稽之耻,乃喟然而叹曰:“计然之策七,越用其五而得意。既已施于国,吾欲用之家。”乃乘扁舟浮于江湖,变名易姓,適齐为鸱夷子皮,之陶为硃公。硃公以为陶天下之中,诸侯四通,货物所交易也。乃治产积居。与时逐而不责于人。故善治生者,能择人而任时。十九年之中三致千金,再分散与贫交疏昆弟。此所谓富好行其德者也。后年衰老而听子孙,子孙脩业而息之,遂至巨万。故言富者皆称陶硃公。

子赣既学于仲尼,退而仕于卫,废著鬻财于曹、鲁之间,七十子之徒,赐最为饶益。原宪不厌糟,匿于穷巷。子贡结驷连骑,束帛之币以聘享诸侯,所至,国君无不分庭与之抗礼。夫使孔子名布扬于天下者,子贡先后之也。此所谓得埶而益彰者乎?

白圭,周人也。当魏文侯时,李克务尽地力,而白圭乐观时变,故人弃我取,人取我与。夫岁孰取穀,予之丝漆;茧出取帛絮,予之食。太阴在卯,穰;明岁衰恶。至午,旱;明岁美。至酉,穰;明岁衰恶。至子,大旱;明岁美,有水。至卯,积著率岁倍。欲长钱,取下穀;长石斗,取上种。能薄饮食,忍嗜欲,节衣服,与用事僮仆同苦乐,趋时若猛兽挚鸟之发。故曰:“吾治生产,犹伊尹、吕尚之谋,孙吴用兵,商鞅行法是也。是故其智不足与权变,勇不足以决断,仁不能以取予,彊不能有所守,虽欲学吾术,终不告之矣。”盖天下言治生祖白圭。白圭其有所试矣,能试有所长,非苟而已也。

猗顿用盬盐起。而邯郸郭纵以铁冶成业,与王者埒富。

乌氏倮畜牧,及众,斥卖,求奇缯物,间献遗戎王。戎王什倍其偿,与之畜,畜至用谷量马牛。秦始皇帝令倮比封君,以时与列臣朝请。而巴寡妇清,其先得丹穴,而擅其利数世,家亦不訾。清,寡妇也,能守其业,用财自卫,不见侵犯。秦皇帝以为贞妇而客之,为筑女怀清台。夫倮鄙人牧长,清穷乡寡妇,礼抗万乘,名显天下,岂非以富邪?

汉兴,海内为一,开关梁,弛山泽之禁,是以富商大贾周流天下,交易之物莫不通,得其所欲,而徙豪杰诸侯彊族于京师。

关中自汧、雍以东至河、华,膏壤沃野千里,自虞夏之贡以为上田,而公刘適邠,大王、王季在岐,文王作丰,武王治镐,故其民犹有先王之遗风,好稼穑,殖五穀,地重,重为邪。及秦文、、缪居雍,隙陇蜀之货物而多贾。献公徙栎邑,栎邑北卻戎翟,东通三晋,亦多大贾。昭治咸阳,因以汉都,长安诸陵,四方辐凑并至而会,地小人众,故其民益玩巧而事末也。南则巴蜀。巴蜀亦沃野,地饶卮、姜、丹沙、石、铜、铁、竹、木之器。南御滇僰,僰僮。西近邛笮,笮马、旄牛。然四塞,栈道千里,无所不通,唯襃斜绾毂其口,以所多易所鲜。天水、陇西、北地、上郡与关中同俗,然西有羌中之利,北有戎翟之畜,畜牧为天下饶。然地亦穷险,唯京师要其道。故关中之地,于天下三分之一,而人众不过什三;然量其富,什居其六。

昔唐人都河东,殷人都河内,周人都河南。夫三河在天下之中,若鼎足,王者所更居也,建国各数百千岁,土地小狭,民人众,都国诸侯所聚会,故其俗纤俭习事。杨、平阳陈西贾秦、翟,北贾种、代。种、代,石北也,地边胡,数被寇。人民矜懻忮,好气,任侠为奸,不事农商。然迫近北夷,师旅亟往,中国委输时有奇羡。其民羯羠不均,自全晋之时固已患其僄悍,而武灵王益厉之,其谣俗犹有赵之风也。故杨、平阳陈掾其间,得所欲。温、轵西贾上党,北贾赵、中山。中山地薄人众,犹有沙丘纣淫地馀民,民俗懁急,仰机利而食。丈夫相聚游戏,悲歌慷慨,起则相随椎剽,休则掘冢作巧奸冶,多美物,为倡优。女子则鼓鸣瑟,跕屣,游媚贵富,入后宫,遍诸侯。

然邯郸亦漳、河之间一都会也。北通燕、涿,南有郑、卫。郑、卫俗与赵相类,然近梁、鲁,微重而矜节。濮上之邑徙野王,野王好气任侠,卫之风也。

夫燕亦勃、碣之间一都会也。南通齐、赵,东北边胡。上谷至辽东,地踔远,人民希,数被寇,大与赵、代俗相类,而民雕捍少虑,有鱼盐枣栗之饶。北邻乌桓、夫馀,东绾秽貉、朝鲜、真番之利。

洛阳东贾齐、鲁,南贾梁、楚。故泰山之阳则鲁,其阴则齐。

齐带山海,膏壤千里,宜桑麻,人民多文采布帛鱼盐。临菑亦海岱之间一都会也。其俗宽缓阔达,而足智,好议论,地重,难动摇,怯于众斗,勇于持刺,故多劫人者,大国之风也。其中具五民。

而邹、鲁滨洙、泗,犹有周公遗风,俗好儒,备于礼,故其民龊龊。颇有桑麻之业,无林泽之饶。地小人众,俭啬,畏罪远邪。及其衰,好贾趋利,甚于周人。

夫自鸿沟以东,芒、砀以北,属巨野,此梁、宋也。陶、睢阳亦一都会也。昔尧作成阳,舜渔于雷泽,汤止于亳。其俗犹有先王遗风,重厚多君子,好稼穑,虽无山川之饶,能恶衣食,致其蓄藏。

越、楚则有三俗。夫自淮北沛、陈、汝南、南郡,此西楚也。其俗剽轻,易发怒,地薄,寡于积聚。江陵故郢都,西通巫、巴,东有云梦之饶。陈在楚夏之交,通鱼盐之货,其民多贾。徐、僮、取虑,则清刻,矜己诺。

彭城以东,东海、吴、广陵,此东楚也。其俗类徐、僮。朐、缯以北,俗则齐。浙江南则越。夫吴自阖庐、春申、王濞三人招致天下之喜游子弟,东有海盐之饶,章山之铜,三江、五湖之利,亦江东一都会也。

衡山、九江、江南、豫章、长沙,是南楚也,其俗大类西楚。郢之后徙寿春,亦一都会也。而合肥受南北潮,皮革、鲍、木输会也。与闽中、干越杂俗,故南楚好辞,巧说少信。江南卑湿,丈夫早夭。多竹木。豫章出黄金,长沙出连、锡,然堇堇物之所有,取之不足以更费。九疑、苍梧以南至儋耳者,与江南大同俗,而杨越多焉。番禺亦其一都会也,珠玑、犀、玳瑁、果、布之凑。

颍川、南阳,夏人之居也。夏人政尚忠朴,犹有先王之遗风。颍川敦愿。秦末世,迁不轨之民于南阳。南阳西通武关、郧关,东南受汉、江、淮。宛亦一都会也。俗杂好事,业多贾。其任侠,交通颍川,故至今谓之“夏人”。

夫天下物所鲜所多,人民谣俗,山东食海盐,山西食盐卤,领南、沙北固往往出盐,大体如此矣。

总之,楚越之地,地广人希,饭稻羹鱼,或火耕而水耨,果隋蠃蛤,不待贾而足,地埶饶食,无饥馑之患,以故呰窳偷生,无积聚而多贫。是故江淮以南,无冻饿之人,亦无千金之家。沂、泗水以北,宜五穀桑麻六畜,地小人众,数被水旱之害,民好畜藏,故秦、夏、梁、鲁好农而重民。三河、宛、陈亦然,加以商贾。齐、赵设智巧,仰机利。燕、代田畜而事蚕。

由此观之,贤人深谋于廊庙,论议朝廷,守信死节隐居岩穴之士设为名高者安归乎?归于富厚也。是以廉吏久,久更富,廉贾归富。富者,人之情性,所不学而俱欲者也。故壮士在军,攻城先登,陷阵卻敌,斩将搴旗,前蒙矢石,不避汤火之难者,为重赏使也。其在闾巷少年,攻剽椎埋,劫人作奸,掘冢铸币,任侠并兼,借交报仇,篡逐幽隐,不避法禁,走死地如骛者,其实皆为财用耳。今夫赵女郑姬,设形容,揳鸣琴,揄长袂,蹑利屣,目挑心招,出不远千里,不择老少者,奔富厚也。游闲公子,饰冠剑,连车骑,亦为富贵容也。弋射渔猎,犯晨夜,冒霜雪,驰阬谷,不避猛兽之害,为得味也。博戏驰逐,斗鸡走狗,作色相矜,必争胜者,重失负也。医方诸食技术之人,焦神极能,为重糈也。吏士舞文弄法,刻章伪书,不避刀锯之诛者,没于赂遗也。农工商贾畜长,固求富益货也。此有知尽能索耳,终不馀力而让财矣。

谚曰:“百里不贩樵,千里不贩籴。”居之一岁,种之以穀;十岁,树之以木;百岁,来之以德。德者,人物之谓也。今有无秩禄之奉,爵邑之入,而乐与之比者。命曰“素封”。封者食租税,岁率户二百。千户之君则二十万,朝觐聘享出其中。庶民农工商贾,率亦岁万息二千,百万之家则二十万,而更徭租赋出其中。衣食之欲,恣所好美矣。故曰陆地牧马二百蹄,牛蹄角千,千足羊,泽中千足彘,水居千石鱼陂,山居千章之材。安邑千树枣;燕、秦千树栗;蜀、汉、江陵千树橘;淮北、常山已南,河济之间千树萩;陈、夏千亩漆;齐、鲁千亩桑麻;渭川千亩竹;及名国万家之城,带郭千亩亩锺之田,若千亩卮茜,千畦姜韭:此其人皆与千户侯等。然是富给之资也,不窥市井,不行异邑,坐而待收,身有处士之义而取给焉。若至家贫亲老,妻子软弱,岁时无以祭祀进醵,饮食被服不足以自通,如此不惭耻,则无所比矣。是以无财作力,少有斗智,既饶争时,此其大经也。今治生不待危身取给,则贤人勉焉。是故本富为上,末富次之,奸富最下。无岩处奇士之行,而长贫贱,好语仁义,亦足羞也。

凡编户之民,富相什则卑下之,伯则畏惮之,千则役,万则仆,物之理也。夫用贫求富,农不如工,工不如商,刺绣文不如倚市门,此言末业,贫者之资也。通邑大都,酤一岁千酿,醯酱千瓨,浆千甔,屠牛羊彘千皮,贩穀粜千锺,薪千车,船长千丈,木千章,竹竿万个,其轺车百乘,牛车千两,木器魨者千枚,铜器千钧,素木铁器若卮茜千石,马蹄躈千,牛千足,羊彘千双,僮手指千,筋角丹沙千斤,其帛絮细布千钧,文采千匹,榻布皮革千石,漆千斗,糵麹盐豉千荅,鮐{此鱼}千斤,鲰千石,鲍千钧,枣栗千石者三之,狐龂裘千皮,羔羊裘千石,旃席千具,佗果菜千锺,子贷金钱千贯,节駔会,贪贾三之,廉贾五之,此亦比千乘之家,其大率也。佗杂业不中什二,则非吾财也。

请略道当世千里之中,贤人所以富者,令后世得以观择焉。

蜀卓氏之先,赵人也,用铁冶富。秦破赵,迁卓氏。卓氏见虏略,独夫妻推辇,行诣迁处。诸迁虏少有馀财,争与吏,求近处,处葭萌。唯卓氏曰:“此地狭薄。吾闻汶山之下,沃野,下有蹲鸱,至死不饥。民工于市,易贾。”乃求远迁。致之临邛,大喜,即铁山鼓铸,运筹策,倾滇蜀之民,富至僮千人。田池射猎之乐,拟于人君。

程郑,山东迁虏也,亦冶铸,贾椎髻之民,富埒卓氏,俱居临邛。

宛孔氏之先,梁人也,用铁冶为业。秦伐魏,迁孔氏南阳。大鼓铸,规陂池,连车骑,游诸侯,因通商贾之利,有游闲公子之赐与名。然其赢得过当,愈于纤啬,家致富数千金,故南阳行贾尽法孔氏之雍容。

鲁人俗俭啬,而曹邴氏尤甚,以铁冶起,富至巨万。然家自父兄子孙约,俯有拾,仰有取,贳贷行贾遍郡国。邹、鲁以其故多去文学而趋利者,以曹邴氏也。

齐俗贱奴虏,而刀间独爱贵之。桀黠奴,人之所患也,唯刀间收取,使之逐渔盐商贾之利,或连车骑,交守相,然愈益任之。终得其力,起富数千万。故曰“宁爵毋刀”,言其能使豪奴自饶而尽其力。

周人既纤,而师史尤甚,转毂以百数,贾郡国,无所不至。洛阳街居在齐秦楚赵之中,贫人学事富家,相矜以久贾,数过邑不入门,设任此等,故师史能致七千万。

宣曲任氏之先,为督道仓吏。秦之败也,豪杰皆争取金玉,而任氏独窖仓粟。楚汉相距荥阳也,民不得耕种,米石至万,而豪杰金玉尽归任氏,任氏以此起富。富人争奢侈,而任氏折节为俭,力田畜。田畜人争取贱贾,任氏独取贵善。富者数世。然任公家约,非田畜所出弗衣食,公事不毕则身不得饮酒食肉。以此为闾里率,故富而主上重之。

塞之斥也,唯桥姚已致马千匹,牛倍之,羊万头,粟以万锺计。吴楚七国兵起时,长安中列侯封君行从军旅,赍贷子钱,子钱家以为侯邑国在关东,关东成败未决,莫肯与。唯无盐氏出捐千金贷,其息什之。三月,吴楚平,一岁之中,则无盐氏之息什倍,用此富埒关中。

关中富商大贾,大抵尽诸田,田啬、田兰。韦家栗氏,安陵、杜杜氏,亦巨万。

此其章章尤异者也。皆非有爵邑奉禄弄法犯奸而富,尽椎埋去就,与时俯仰,获其赢利,以末致财,用本守之,以武一切,用文持之,变化有概,故足术也。若至力农畜,工虞商贾,为权利以成富,大者倾郡,中者倾县,下者倾乡里者,不可胜数。

夫纤啬筋力,治生之正道也,而富者必用奇胜。田农,掘业,而秦扬以盖一州。掘冢,奸事也,而田叔以起。博戏,恶业也,而桓发用富。行贾,丈夫贱行也,而雍乐成以饶。贩脂,辱处也,而雍伯千金。卖浆,小业也,而张氏千万。洒削,薄技也,而郅氏鼎食。胃脯,简微耳,浊氏连骑。马医,浅方,张里击锺。此皆诚壹之所致。

由是观之,富无经业,则货无常主,能者辐凑,不肖者瓦解。千金之家比一都之君,巨万者乃与王者同乐。岂所谓“素封”者邪?非也?

货殖之利,工商是营。废居善积,倚巿邪赢。白圭富国,计然强兵。倮参朝请,女筑怀清。素封千户,卓郑齐名。
\end{yuanwen}

\part{卷一百三十}

\chapter{太史公自序第七十}

\begin{yuanwen}
昔在颛顼,命南正重以司天\footnote{主管天文星象变化。},北正黎以司地\footnote{主管地面人事。}。唐虞之际,绍\footnote{继承发扬。}重黎之后,使复典之,至于夏商,故重黎氏世序天地。其在周,程伯休甫其后也。当周宣王时,失其守而为司马氏。司马氏世典周史。惠襄之间,司马氏去周適晋。晋中军随会奔秦,而司马氏入少梁。
\end{yuanwen}

过去颛顼为帝时,任命南正重主管天文,北正黎主管地事。在唐尧和虞舜之时,又任命重氏、黎氏的后人继续主管天文、地理,一直到夏朝和商朝。因此,重氏、黎氏世世代代主管天文和地理。到了周代,程伯休甫就是重氏、黎氏的后人。等到周宣王时,重氏、黎氏的后人失去了主管天地的世代官守而司马氏接替了职位。司马氏世世代代掌管周史。周惠王、周襄王在位时,司马氏从周朝离开前往晋国,晋国的中军元帅随会逃往秦国,过了没多长时间司马氏宗族迁入少梁。

\begin{yuanwen}	
自司马氏去周适晋,分散,或在卫,或在赵,或在秦。其在卫者,相中山\footnote{担任中山国的相国。}。在赵者,以传剑论显,蒯聩其后也。在秦者名错,与张仪争论,于是惠王使错将伐蜀,遂拔,因而守之。错孙靳,事武安君白起。而少梁更名曰夏阳。靳与武安君阬赵长平军,还而与之俱赐死杜邮,葬于华池。靳孙昌,昌为秦主铁官,当始皇之时。蒯聩玄孙卬为武信君将而徇\footnote{巡视。}朝歌。诸侯之相王,王卬于殷。汉之伐楚,卬归汉,以其地为河内郡。昌生无泽,无泽为汉巿长。无泽生喜,喜为五大夫,卒,皆葬高门。喜生谈,谈为太史公。
\end{yuanwen}

自从司马氏离开周朝来到晋国后,族人就分散到各地。有的留在卫国,有的留在赵国,有的留在了秦国。在卫国的叫司马喜,曾做过中山国的相国。在赵国的人,凭借传授剑术理论而声名在外,司马蒯聩就是这支人的后代。在秦国的那个人叫司马错,曾与张仪争辩过,因此秦惠王派出司马错率领军队进攻蜀国,攻克蜀地,于是就驻守在蜀地。司马错的孙子司马靳,侍奉秦国的武安君白起。此时少梁更名为夏阳。司马靳和武安君坑杀了长平之战中投降秦军的赵长平军,回到秦国后,靳和白起一同被秦王赐死在杜邮,埋葬在华池。司马靳的孙子司马昌,他担任秦朝掌管冶造铁器的官吏。秦始皇在位时,司马蒯聩的玄孙司马卬,曾做过武馆君的部将,而且带兵巡视朝歌。诸侯争着相互称王时,项羽封司马卬为殷王。汉王刘邦率领部队攻打楚国,司马卬投靠了汉王,汉将司马卬原来的封地设置为河内郡。司马昌生有儿子司马无泽,司马无泽曾做过主管集市的官吏。司马无泽生有儿子司马喜,司马喜爵位是五大夫,他们去世后,都被埋葬在高门。司马喜生有儿子司马谈,司马谈做了太史公。

\begin{yuanwen}
太史公学天官\footnote{天文学。}于唐都,受《易》于杨何,习道论\footnote{道家学说。}于黄子。太史公仕于建元、元封之间,愍\footnote{忧虑。}学者之不达其意而师悖,乃论六家之要指曰:

《易大传》:“天下一致而百虑,同归而殊涂。”夫阴阳、儒、墨、名、法、道德,此务为治者也,直所从言之异路,有省不省耳。尝窃观阴阳之术,大祥\footnote{祥瑞。}而众忌讳,使人拘而多所畏;然其序四时之大顺,不可失也。儒者博而寡要,劳而少功,是以其事难尽从;然其序\footnote{顺序,排列。}君臣父子之礼,列夫妇长幼之别,不可易也。墨者俭而难遵,是以其事不可遍循\footnote{无法一一照办。};然其彊本节用,不可废也。法家严而少恩;然其正君臣上下之分,不可改矣。名家使人俭而善失真;然其正名实,不可不察也。道家使人精神专一,动合无形,赡\footnote{赡养,供养。}足万物。其为术也,因阴阳之大顺,采儒墨之善,撮\footnote{提取,摘录。}名法之要,与时迁移,应物变化,立俗施事,无所不宜,指约而易操,事少而功多。儒者则不然。以为人主天下之仪表也,主倡而臣和,主先而臣随。如此则主劳而臣逸。至于大道之要,去健羡\footnote{刚强和欲求。},绌聪明,释此而任术。夫神大用则竭,形大劳则敝。形神骚动,欲与天地长久,非所闻也。
\end{yuanwen}

太史公跟随唐都研习天文学,跟随杨何研习《易经》,又跟随黄先生研习道家理论。太史公曾经在建元、元封年间做官。他担心当时的学者不能通晓诸家学说的精要,而所学又混乱不明,因此专们讨论阴阳、儒、墨、名、法和道德六家学说的精要:

《易大传》说:“世人的目标都是一致的,而谋虑却分为很多种;要达成的目的相同,而所经过的途径却并不一样。”阴阳家、儒家、墨家、名家、法家、道德家,这些学派的目的都是希望天下大治。只不过所遵从的理论有所不同,有考虑得周全与不周全而已。我曾经暗地里分析过阴阳家的法术,它十分看中吉凶的预兆而忌讳的东西太多,让人感到拘束而畏手畏脚;但是它主张遵照四季的序列去行事,却是不能够违背的。儒家的学说非常广博但是缺少要领,事倍而功半,所以儒家学派所倡导的事情都难以完全实施。但儒家确立了君臣父子间相处的礼仪,以及夫妇长幼之间的礼仪区别,这是不可以改变的。墨家太过节俭难以遵从,因此他们的主张不可能全部实行;但他们倡导务实节俭,却是不能抛弃的。法家主张施行严酷的立法而少给予臣民恩惠,但他们定立君臣上下的名分等级,则是无法更改的。名家让人专注于名而易于失去真实性,但他们对名、实关系进行的辩证,却无法不仔细考虑。道家令人精神专一,行动都与无形的“道”相合,使万物得以丰富和充足。道家这种学术,是顺应阴阳四时之序的学说,兼采儒家与墨家的优点,聚合名家与法家的精要,和时势随同发展,与事物的变化相随,建立风俗,应用在人事上,并无不合适的。意旨简约,且容易把握,事半而功倍。儒家则与道家不同:他们觉得君主应该是天下人的表率,君主所提倡的东西,臣下就应该附和;君主走在前面,臣下就应该在后面紧跟。如此一来,君主就会劳累不堪而臣子却十分安逸。至于大道的精要是:舍弃刚强与贪欲,去掉聪明和智慧,放弃一切而用智术治理天下。人的精神过度使用就会衰竭,身体极其劳累就要疲惫。倘若精神和身体总是难以安宁,却想要自己的生命和天地一般长久,从没听说过这样的事。

\begin{yuanwen}
夫阴阳四时、八位、十二度、二十四节各有教令,顺之者昌,逆之者不死则亡,未必然也,故曰“使人拘而多畏”。夫春生夏长,秋收冬藏,此天道之大经也,弗顺则无以为天下纲纪,故曰“四时之大顺,不可失也”。
\end{yuanwen}

在阴阳家看来,阴阳、四时、八位、十二度、二十四节气都各自的教令,顺应这些法则就会昌盛,违背这些法则,就算不死也会衰亡。这不一定是正确的。因此说阴阳家“让人感到拘束而有太多的畏惧和忌讳”。至于阴阳家说的春天万物出生、夏天成长、秋天收获、冬天贮藏,这是天地间的普遍规律,要是不遵守,就难以制定天下的纲纪。因此说:“遵照四时的顺列去行事,是不能失去的。”

\begin{yuanwen}
夫儒者以六艺为法。六艺经传以千万数,累世\footnote{连续数代。}不能通其学,当年不能究其礼,故曰“博而寡要,劳而少功”。若夫列君臣父子之礼,序夫妇长幼之别,虽百家弗能易也。
\end{yuanwen}

儒家把六经作为自身的理论依据,六经及其传文多到以千万计,连续数代都不能通晓这门学问,耗尽人的一生也无法参透这其中的礼仪。因此说“儒家学说十分广博却缺少要领,事倍而功半”。倒是确定了君臣父子间相处的礼仪,夫妇长幼尊卑的次序,纵然是百家学说,也都是不可能更改的。

\begin{yuanwen}
墨者亦尚尧舜道,言其德行曰:“堂高三尺,土阶三等,茅茨\footnote{用茅草铺在屋顶上。}不翦,采椽不刮。食土簋\footnote{用陶土制成的碗吃饭。},啜土刑\footnote{通“铏”,盛羹的器皿。},粝粱\footnote{粗疏的食物。}之食,藜霍\footnote{野草,野菜。}之羹。夏日葛衣,冬日鹿裘。”其送死,桐棺\footnote{桐木制成的棺材。}三寸,举音不尽其哀。教丧礼,必以此为万民之率。使天下法若此,则尊卑无别也。夫世异时移,事业不必同,故曰“俭而难遵”。要曰彊本节用,则人给家足之道也。此墨子之所长,虽百(长/家)弗能废也。
\end{yuanwen}

墨家也崇尚唐尧与虞舜的道术,谈论尧舜的德行说:“殿堂区区三尺高,堂下主阶只三级,茅草屋顶不修剪,原木制屋不刮削,用陶土碗盛饭,以陶土钵喝汤,用粗米做饭,以野菜做汤。夏天穿着葛布衣,冬天穿着鹿皮裘。”墨家主张为死者送葬时要用桐木制的棺材,厚度不能超过三寸,哭丧但不能过于悲伤。他们教百姓丧礼,必须要以此来当作万民的标准。要是天下人都遵照这个标准去做,那么尊卑就不再有分别了。世事不同,时势也在变化,事业也一定会不一样,因此说“太过节俭就很难遵从”。墨家学说的精要是强本而节用,这是人们兴家富足的途径。这是墨家学说的优点,纵然是百家学说,也都不可以废弃。

\begin{yuanwen}
法家不别亲疏,不殊贵贱,一断于法,则亲亲尊尊\footnote{亲近亲者,尊重尊者。}之恩绝矣。可以行一时之计,而不可长用也,故曰“严而少恩”。若尊主卑臣,明分职不得相逾越,虽百家弗能改也。
\end{yuanwen}

法家不区分亲近、疏远,不因高贵、贫贱而有所区别,一律都依照法来进行决断。这样,亲近亲属、尊敬长上的恩情伦理就失去了。可以用这种理论推行一时的计策,却不能长久地运用。因此说法家“施行严酷的立法而少给予臣民恩惠”。倒是法家倡导让君主尊贵,令臣子卑下,明确上下的名分与职守,不能有所超越,纵然是百家学说,也都无法改变。

\begin{yuanwen}
名家苛察缴绕\footnote{迂回缠绕,指繁琐复杂。},使人不得反其意,专决于名而失人情,故曰“使人俭而善失真”。若夫控\footnote{按照,依照。}名责实,参伍不失,此不可不察也。
\end{yuanwen}

名家细察琐碎,纠缠不清,让人反省却难以获得其旨意,专注于用名称对事物进行决断,而不顾及人情。因此说名家“让人专注于名而易于失去真实性”。如果定立名称,考察实际,名称与实际交错比较,在这方面,是不可能不进行认真考虑的。

\begin{yuanwen}
道家无为,又曰无不为,其实易行,其辞难知。其术以虚无为本,以因循为用。无成埶,无常形,故能究万物之情。不为物先,不为物后,故能为万物主。有法无法,因时为业;有度无度,因物与合。故曰“圣人不朽,时变是守。虚者道之常也,因者君之纲”也。群臣并至,使各自明\footnote{表现,显现。}也。其实中其声者谓之端,实不中其声者谓之窾\footnote{kuǎn,通“款”,空。}。窾言不听,奸乃不生,贤不肖自分,白黑乃形。在所欲用耳,何事不成。乃合大道,混混冥冥。光燿天下,复反无名。凡人所生者神也,所(讬/托)者形也。神大用则竭,形大劳则敝,形神离则死。死者不可复生,离者不可复反,故圣人重之。由是观之,神者生之本也,形者生之具\footnote{基础。}也。不先定其神(形),而曰“我有以治天下”,何由哉?
\end{yuanwen}

道家倡导“无为”,又认为“无所不为”,他们的主张易于实现,只是他们的文辞难于理解。道家学术将虚无作为理论基础,把顺应自然和自然相合当作实践的原则,并无长久不变的态势,也无常存不变的形状,所以可以探求万物的情理。不去超越事物,不落后于事物,所以才可以成为万物的主宰。有没有立法,依照时势而决定,制度是否有用,也要和事物相合。因此说“圣人的学说和思想难以磨灭,在于他们将顺时、通变作为原则。虚无,是道的永恒常理;顺应时势,是君主治国的纲领”。大臣们全都来到朝廷,君主应当让他们每个人都明确自己该尽的职任。他的所作所为与他的名声相合的被称作“端”,所作所为与他的名声不相符合的被称作“窾”。空话不被听取,奸邪就不能产生,贤才和庸才自然就能区分,白和黑也就显现分明,这些精要就要看是不是运用而已,什么事是办不成的呢!这才与“大道”相合,混混沌沌,光辉照耀天下,再次返回到无名。每个人生命的基础都是精神,精神寄托于形体。过度使用精神就会衰竭,身体极其劳累自会疲惫,精神与形体分开人就会死去。死去的人不可能再复生,形与神分开后也不可能重新结合,所以圣人十分注重这些问题。由此看来,精神是人生命的根本;形体是生命的依托所在。不率先安定好自己的精神与形体,却夸口说“我有能力可以治理天下”,可你有什么凭借呢!

\begin{yuanwen}
太史公既掌天官,不治民。有子曰迁。
\end{yuanwen}

太史公主掌天文后,不管理民事。太史公有儿子名叫司马迁。

\begin{yuanwen}	
迁生龙门\footnote{text},耕牧河山之阳\footnote{龙门山的南侧,黄河的西北岸。}。年十岁则诵古文\footnote{text}。二十而南游江、淮,上会稽\footnote{text},探禹穴\footnote{text},窥九疑\footnote{text},浮于沅、湘\footnote{text};北涉汶、泗,讲业齐、鲁之都\footnote{text},观孔子之遗风,乡射\footnote{儒家射箭饮酒的礼仪。}邹、峄\footnote{text};厄困鄱、薛、彭城\footnote{text},过梁、楚以归。于是迁仕为郎中\footnote{text},奉使西征巴、蜀以南\footnote{text},南略\footnote{巡视,巡查。}邛、(筰/笮)、昆明\footnote{text},还报命。
\end{yuanwen}

司马迁生于龙门,在黄河北边、龙门山的南边过着耕种牧畜的生活。十岁时就研习诵读古文经书。二十岁时,开始向南游历长江与淮河一带,登临会稽山,察探禹穴,勘察九疑山,泛舟在沅水、湘水之上。再向北渡过汶水和泗水,来到齐、鲁的都市中讲学,考察孔子所遗留的风教,在邹县、峄山进行乡射大礼。在鄱县、薛县、彭城这些地方遇到过困难,再经由梁地和楚地回到家乡。于是司马迁被任命为郎中,奉命出使,向西讨伐巴、蜀以南地区,向南巡视邛、笮、昆明这些地区,回到朝廷复命。



\begin{yuanwen}	
是岁天子始建汉家之封\footnote{text},而太史公留滞周南\footnote{指洛阳。},不得与从事\footnote{text},故发愤且卒。而子迁适使反,见父于河、洛之间\footnote{洛阳,位于洛水北侧,黄河南侧。}。太史公执迁手而泣曰:“余先周室之太史也。自上世尝显功名于虞夏,典天官事。后世中衰,绝于予乎?汝复为太史,则续吾祖矣。今天子接千岁之统\footnote{text},封泰山,而余不得从行,是命也夫,命也夫!余死,汝必为太史;为太史,无忘吾所欲论著矣\footnote{text}。且夫孝始于事亲,中于事君,终于立身。扬名于后世,以显父母,此孝之大者。夫天下称诵周公,言其能论歌文、武之德\footnote{text},宣周、邵之风,达太王王季之思虑\footnote{text},爰及公刘\footnote{text},以尊后稷也。幽、厉之后,王道缺,礼乐衰\footnote{text},孔子修旧起废,论《诗》、《书\footnote{text}》,作《春秋\footnote{text}》,则学者至今则\footnote{遵循,遵守。}之。自获麟\footnote{鲁哀公十四年狩猎捕获麒麟,孔子因伤心而不再作《春秋》。}以来四百有馀岁\footnote{text},而诸侯相兼,史记放绝\footnote{text}。今汉兴,海内一统,明主贤君忠臣死义之士,余为太史而弗论载,废天下之史文,余甚惧焉,汝其念哉!”

迁俯首流涕曰:“小子不敏,请悉论先人所次\footnote{编次。}旧闻\footnote{text},弗敢阙。”
\end{yuanwen}

这一年,皇上开始举办汉朝的封禅典礼,但是太史公有事在洛阳滞留,没能参加这场典礼,因此心中不满,病情危重就要死去。他的儿子司马迁恰好在此时出使归来,在洛阳见到了父亲。太史公抓着司马迁的手哭着说:“我们的先人,是周朝时的太史。早在上古的唐尧和虞舜之时,就曾担任南正、北正,功名显赫,主掌天文。后世渐渐衰落,祖先的积业会断绝在我的手中吗?你倘若可以再担任太史,那就能够承接我们祖先所从事的事业了。如今皇上继承了汉朝千年一统的大业,在泰山进行封禅大典,我没能随行前往,这就是命运啊!是命运啊!我死了之后,你一定会成为太史,做了太史,不要忘记我本打算完成的论著!再说孝道应该以侍奉双亲为基础,然后才能侍奉君主,最终才能为自身成就功名。在后世扬名,以此来显耀父母,这是孝道里面最为重要的。全天下人都称赞周公,说他能够赞扬歌颂文王和武王的功德,宣扬周公与召公的风尚,传达太王和王季的思想,向上到论及公刘的功业,以此来推崇始祖后稷。周幽王和周厉王之后,治理天下的王道残失,礼乐衰败不堪,孔子修编过去的典籍,重振被废弃败坏的礼乐,论述《诗》《书》,撰作《春秋》,而学者们至今仍以此为依据。自鲁哀公捕获麒麟到现在的四百多年,诸侯之间都互相征伐吞并,史书都被丢弃而断绝。如今汉朝兴起,天下统一,这期间的明主贤君忠臣为道义而死的人士,我身为太史却并未加以论述和记录,断绝了天下传承的历史文献,这让我感到十分惶恐,你一定要记住!”

司马迁低着头流着泪说:“尽管儿子不够聪敏,请允许我详细地论述记载先人所整理编辑的史料佚闻,不敢有丝毫缺略。”

\begin{yuanwen}	
卒三岁而迁为太史令\footnote{text},紬史记石室金匮之书\footnote{text}。五年而当太初元年\footnote{text},十一月甲子朔旦冬至\footnote{text},天历始改\footnote{text},建于明堂\footnote{text},诸神受纪\footnote{text}。
\end{yuanwen}

太史公司马谈去世后三年,司马迁做了太史令,整理收集历史书籍和国家收藏在石室金匮中的书籍。司马迁做了太史令五年后,恰好是太初元年(前104),十一月甲子朔日的早晨冬至,汉朝更改历法,天子在明堂中举行仪式宣布改历,祭祷上天,遍告群神,受命进行著记。

\begin{yuanwen}	
太史公\footnote{text}曰:“先人\footnote{指司马谈。}有言:‘自周公卒五百岁而有孔子。孔子卒后至于今五百岁\footnote{text},有能绍明世,正《易传\footnote{text}》,继《春秋\footnote{text}》,本《诗》、《书》、《礼》、《乐》之际?’意在斯乎!意在斯\footnote{此,指我。}乎!小子何敢让焉。”
\end{yuanwen}

太史公说:“先父曾说过:‘自周公去世以后,五百年而诞生孔子。孔子去世之后到如今已经五百年了,有能够继承清明盛世,订正《易传》,接续《春秋》,推究《诗》《书》《礼》《乐》的精义而有所著述的人吗?’他的用意就在此吧!就在此吧!小子我怎么敢草率辞让呢?”

\begin{yuanwen}
	上大夫壶遂曰:“昔孔子何为而作春秋哉?”太史公曰:“余闻董生曰:‘周道衰废,孔子为鲁司寇,诸侯害之,大夫壅之。孔子知言之不用,道之不行也,是非二百四十二年之中,以为天下仪表,贬天子,退诸侯,讨大夫,以达王事而已矣。’ 子曰:‘我欲载之空言,不如见之于行事之深切著明也。’ 夫春秋,上明三王之道,下辨人事之纪,别嫌疑,明是非,定犹豫,善善恶恶,贤贤贱不肖,存亡国,继绝世,补敝起废,王道之大者也。易著天地阴阳四时五行,故长于变;礼经纪人伦,故长于行;书记先王之事,故长于政;诗记山川谿谷禽兽草木牝牡雌雄,故长于风;乐乐所以立,故长于和;春秋辩是非,故长于治人。是故礼以节人,乐以发和,书以道事,诗以达意,易以道化,春秋以道义。拨乱世反之正,莫近于春秋。春秋文成数万,其指数千。万物之散聚皆在春秋。春秋之中,弑君三十六,亡国五十二,诸侯奔走不得保其社稷者不可胜数。察其所以,皆失其本已。故易曰‘失之豪釐,差以千里’ 。故曰‘臣弑君,子弑父,非一旦一夕之故也,其渐久矣’ 。故有国者不可以不知春秋,前有谗而弗见,后有贼而不知。为人臣者不可以不知春秋,守经事而不知其宜,遭变事而不知其权。为人君父而不通于春秋之义者,必蒙首恶之名。为人臣子而不通于春秋之义者,必陷篡弑之诛,死罪之名。其实皆以为善,为之不知其义,被之空言而不敢辞。夫不通礼义之旨,至于君不君,臣不臣,父不父,子不子。夫君不君则犯,臣不臣则诛,父不父则无道,子不子则不孝。此四行者,天下之大过也。以天下之大过予之,则受而弗敢辞。故春秋者,礼义之大宗也。夫礼禁未然之前,法施已然之后;法之所为用者易见,而礼之所为禁者难知。”

壶遂曰:“孔子之时,上无明君,下不得任用,故作春秋,垂空文以断礼义,当一王之法。今夫子上遇明天子,下得守职,万事既具,咸各序其宜,夫子所论,欲以何明?”

太史公曰:“唯唯,否否,不然。余闻之先人曰:‘伏羲至纯厚,作易八卦。尧舜之盛,尚书载之,礼乐作焉。汤武之隆,诗人歌之。春秋采善贬恶,推三代之德,襃周室,非独刺讥而已也。’ 汉兴以来,至明天子,获符瑞,封禅,改正朔,易服色,受命于穆清,泽流罔极,海外殊俗,重译款塞,请来献见者,不可胜道。臣下百官力诵圣德,犹不能宣尽其意。且士贤能而不用,有国者之耻;主上明圣而德不布闻,有司之过也。且余尝掌其官,废明圣盛德不载,灭功臣世家贤大夫之业不述,堕先人所言,罪莫大焉。余所谓述故事,整齐其世传,非所谓作也,而君比之于春秋,谬矣。”
\end{yuanwen}\begin{yuanwen}
\end{yuanwen}\begin{yuanwen}
\end{yuanwen}\begin{yuanwen}
\end{yuanwen}\begin{yuanwen}
\end{yuanwen}\begin{yuanwen}
\end{yuanwen}\begin{yuanwen}
\end{yuanwen}\begin{yuanwen}
\end{yuanwen}\begin{yuanwen}
\end{yuanwen}\begin{yuanwen}
\end{yuanwen}\begin{yuanwen}
\end{yuanwen}\begin{yuanwen}
\end{yuanwen}\begin{yuanwen}
\end{yuanwen}\begin{yuanwen}
\end{yuanwen}\begin{yuanwen}
\end{yuanwen}\begin{yuanwen}
\end{yuanwen}\begin{yuanwen}
\end{yuanwen}\begin{yuanwen}
\end{yuanwen}\begin{yuanwen}
\end{yuanwen}\begin{yuanwen}
\end{yuanwen}\begin{yuanwen}
\end{yuanwen}\begin{yuanwen}
\end{yuanwen}\begin{yuanwen}
\end{yuanwen}\begin{yuanwen}
\end{yuanwen}\begin{yuanwen}
\end{yuanwen}\begin{yuanwen}
\end{yuanwen}\begin{yuanwen}
\end{yuanwen}\begin{yuanwen}
\end{yuanwen}\begin{yuanwen}
\end{yuanwen}\begin{yuanwen}
\end{yuanwen}\begin{yuanwen}
\end{yuanwen}\begin{yuanwen}
\end{yuanwen}\begin{yuanwen}
\end{yuanwen}\begin{yuanwen}
\end{yuanwen}\begin{yuanwen}
\end{yuanwen}\begin{yuanwen}
\end{yuanwen}\begin{yuanwen}
\end{yuanwen}\begin{yuanwen}
\end{yuanwen}\begin{yuanwen}
\end{yuanwen}\begin{yuanwen}
\end{yuanwen}\begin{yuanwen}
\end{yuanwen}\begin{yuanwen}
\end{yuanwen}\begin{yuanwen}
\end{yuanwen}\begin{yuanwen}
\end{yuanwen}\begin{yuanwen}
\end{yuanwen}\begin{yuanwen}
\end{yuanwen}\begin{yuanwen}
\end{yuanwen}\begin{yuanwen}
\end{yuanwen}\begin{yuanwen}
\end{yuanwen}\begin{yuanwen}
\end{yuanwen}\begin{yuanwen}
\end{yuanwen}\begin{yuanwen}
\end{yuanwen}\begin{yuanwen}
\end{yuanwen}\begin{yuanwen}
\end{yuanwen}\begin{yuanwen}
\end{yuanwen}\begin{yuanwen}
\end{yuanwen}\begin{yuanwen}
\end{yuanwen}\begin{yuanwen}
\end{yuanwen}\begin{yuanwen}
\end{yuanwen}\begin{yuanwen}
\end{yuanwen}\begin{yuanwen}
\end{yuanwen}\begin{yuanwen}
\end{yuanwen}\begin{yuanwen}
\end{yuanwen}\begin{yuanwen}
\end{yuanwen}\begin{yuanwen}
\end{yuanwen}\begin{yuanwen}
\end{yuanwen}\begin{yuanwen}
\end{yuanwen}\begin{yuanwen}
\end{yuanwen}\begin{yuanwen}
\end{yuanwen}\begin{yuanwen}
\end{yuanwen}\begin{yuanwen}
\end{yuanwen}\begin{yuanwen}
\end{yuanwen}\begin{yuanwen}
\end{yuanwen}\begin{yuanwen}
\end{yuanwen}\begin{yuanwen}
\end{yuanwen}\begin{yuanwen}
\end{yuanwen}\begin{yuanwen}
\end{yuanwen}\begin{yuanwen}
\end{yuanwen}\begin{yuanwen}
\end{yuanwen}\begin{yuanwen}
\end{yuanwen}\begin{yuanwen}
\end{yuanwen}\begin{yuanwen}
\end{yuanwen}

\begin{yuanwen}
	
	
	
	
	
	
	
	
	
\end{yuanwen}







\begin{yuanwen}
	于是论次其文\footnote{text}。七年而太史公遭李陵之祸\footnote{text},幽于缧绁\footnote{text}。乃喟然而叹曰:“是余之罪也夫!是余之罪也夫!身毁不用矣。”退而深惟\footnote{text}曰:“夫《诗》、《书》隐约者,欲遂其志之思也。昔西伯拘羑里,演《周易\footnote{text}》;孔子厄陈、蔡,作《春秋》;屈原放逐,著《离骚》;左丘失明,厥有《国语》;孙子膑脚,而论兵法\footnote{text};不韦迁蜀,世传《吕览》;韩非囚秦,《说难》、《孤愤\footnote{text}》;《诗》三百篇,大抵贤圣发愤之所为作\footnote{text}也。此人皆意有所郁结\footnote{text},不得通其道也,故述往事,思来者。”于是卒述陶唐以来\footnote{text},至于麟\footnote{text}止,自黄帝始。
\end{yuanwen}



\begin{yuanwen}	
	维昔黄帝,法天则地,四圣遵序,各成法度;唐尧逊位,虞舜不台;厥美帝功,万世载之。作五帝本纪第一。
	
	维禹之功,九州攸同,光唐虞际,德流苗裔;夏桀淫骄,乃放鸣条。作夏本纪第二。
	
	维契作商,爰及成汤;太甲居桐,德盛阿衡;武丁得说,乃称高宗;帝辛湛湎,诸侯不享。作殷本纪第三。
	
	维弃作稷,德盛西伯;武王牧野,实抚天下;幽厉昏乱,既丧酆镐;陵迟至赧;洛邑不祀。作周本纪第四。
	
	维秦之先,伯翳佐禹;穆公思义,悼豪之旅;以人为殉,诗歌黄鸟;昭襄业帝。作秦本纪第五。
	
	始皇既立,并兼六国,销锋铸鐻,维偃干革,尊号称帝,矜武任力;二世受运,子婴降虏。作始皇本纪第六。
	
	秦失其道,豪桀并扰;项梁业之,子羽接之;杀庆救赵,诸侯立之;诛婴背怀,天下非之。作项羽本纪第七。
	
	子羽暴虐,汉行功德;愤发蜀汉,还定三秦;诛籍业帝,天下惟宁,改制易俗。作高祖本纪第八。
	
	惠之早霣,诸吕不台;崇彊禄、产,诸侯谋之;杀隐幽友,大臣洞疑,遂及宗祸。作吕太后本纪第九。
	
	汉既初兴,继嗣不明,迎王践祚,天下归心;蠲除肉刑,开通关梁,广恩博施,厥称太宗。作孝文本纪第十。
	
	诸侯骄恣,吴首为乱,京师行诛,七国伏辜,天下翕然,大安殷富。作孝景本纪第十一。
	
	汉兴五世,隆在建元,外攘夷狄,内脩法度,封禅,改正朔,易服色。作今上本纪第十二。
	
	维三代尚矣,年纪不可考,盖取之谱牒旧闻,本于兹,于是略推,作三代世表第一。
	
	幽厉之后,周室衰微,诸侯专政,春秋有所不纪;而谱牒经略,五霸更盛衰,欲睹周世相先后之意,作十二诸侯年表第二。
	
	春秋之后,陪臣秉政,彊国相王;以至于秦,卒并诸夏,灭封地,擅其号。作六国年表第三。
	
	秦既暴虐,楚人发难,项氏遂乱,汉乃扶义征伐;八年之间,天下三嬗,事繁变众,故详著秦楚之际月表第四。
	
	汉兴已来,至于太初百年,诸侯废立分削,谱纪不明,有司靡踵,彊弱之原云以世。作汉兴已来诸侯年表第五。
	
	维高祖元功,辅臣股肱,剖符而爵,泽流苗裔,忘其昭穆,或杀身陨国。作高祖功臣侯者年表第六。
	
	惠景之间,维申功臣宗属爵邑,作惠景间侯者年表第七。
	
	北讨彊胡,南诛劲越,征伐夷蛮,武功爰列。作建元以来侯者年表第八。
	
	诸侯既彊,七国为从,子弟众多,无爵封邑,推恩行义,其埶销弱,德归京师。作王子侯者年表第九。
	
	国有贤相良将,民之师表也。维见汉兴以来将相名臣年表,贤者记其治,不贤者彰其事。作汉兴以来将相名臣年表第十。
	
	维三代之礼,所损益各殊务,然要以近性情,通王道,故礼因人质为之节文,略协古今之变。作礼书第一。
	
	乐者,所以移风易俗也。自雅颂声兴,则已好郑卫之音,郑卫之音所从来久矣。人情之所感,远俗则怀。比乐书以述来古,作乐书第二。
	
	非兵不彊,非德不昌,黄帝、汤、武以兴,桀、纣、二世以崩,可不慎欤?司马法所从来尚矣,太公、孙、吴、王子能绍而明之,切近世,极人变。作律书第三。
	
	律居阴而治阳,历居阳而治阴,律历更相治,间不容翲忽。五家之文怫异,维太初之元论。作历书第四。
	
	星气之书,多杂禨祥,不经;推其文,考其应,不殊。比集论其行事,验于轨度以次,作天官书第五。
	
	受命而王,封禅之符罕用,用则万灵罔不禋祀。追本诸神名山大川礼,作封禅书第六。
	
	维禹浚川,九州攸宁;爰及宣防,决渎通沟。作河渠书第七。
	
	维币之行,以通农商;其极则玩巧,并兼兹殖,争于机利,去本趋末。作平准书以观事变,第八。
	
	太伯避历,江蛮是適;文武攸兴,古公王迹。阖庐弑僚,宾服荆楚;夫差克齐,子胥鸱夷;信嚭亲越,吴国既灭。嘉伯之让,作吴世家第一。
	
	申、吕肖矣,尚父侧微,卒归西伯,文武是师;功冠群公,缪权于幽;番番黄发,爰飨营丘。不背柯盟,桓公以昌,九合诸侯,霸功显彰。田阚争宠,姜姓解亡。嘉父之谋,作齐太公世家第二。
	
	依之违之,周公绥之;愤发文德,天下和之;辅翼成王,诸侯宗周。隐桓之际,是独何哉?三桓争彊,鲁乃不昌。嘉旦金縢,作周公世家第三。
	
	武王克纣,天下未协而崩。成王既幼,管蔡疑之,淮夷叛之,于是召公率德,安集王室,以宁东土。燕之禅,乃成祸乱。嘉甘棠之诗,作燕世家第四。
	
	管蔡相武庚,将宁旧商;及旦摄政,二叔不飨;杀鲜放度,周公为盟;大任十子,周以宗彊。嘉仲悔过,作管蔡世家第五。
	
	王后不绝,舜禹是说;维德休明,苗裔蒙烈。百世享祀,爰周陈杞,楚实灭之。齐田既起,舜何人哉?作陈杞世家第六。
	
	收殷馀民,叔封始邑,申以商乱,酒材是告,及朔之生,卫顷不宁;南子恶蒯聩,子父易名。周德卑微,战国既彊,卫以小弱,角独后亡。喜彼康诰,作卫世家第七。
	
	嗟箕子乎!嗟箕子乎!正言不用,乃反为奴。武庚既死,周封微子。襄公伤于泓,君子孰称。景公谦德,荧惑退行。剔成暴虐,宋乃灭亡。喜微子问太师,作宋世家第八。
	
	武王既崩,叔虞邑唐。君子讥名,卒灭武公。骊姬之爱,乱者五世;重耳不得意,乃能成霸。六卿专权,晋国以秏。嘉文公锡珪鬯,作晋世家第九。
	
	重黎业之,吴回接之;殷之季世,粥子牒之。周用熊绎,熊渠是续。庄王之贤,乃复国陈;既赦郑伯,班师华元。怀王客死,兰咎屈原;好谀信谗,楚并于秦。嘉庄王之义,作楚世家第十。
	
	少康之子,实宾南海,文身断发,鼋鳝与处,既守封禺,奉禹之祀。句践困彼,乃用种、蠡。嘉句践夷蛮能脩其德,灭彊吴以尊周室,作越王句践世家第十一。桓公之东,太史是庸。及侵周禾,王人是议。祭仲要盟,郑久不昌。子产之仁,绍世称贤。三晋侵伐,郑纳于韩。嘉厉公纳惠王,作郑世家第十二。
	
	维骥騄耳,乃章造父。赵夙事献,衰续厥绪。佐文尊王,卒为晋辅。襄子困辱,乃禽智伯。主父生缚,饿死探爵。王迁辟淫,良将是斥。嘉鞅讨周乱,作赵世家第十三。
	
	毕万爵魏,卜人知之。及绛戮干,戎翟和之。文侯慕义,子夏师之。惠王自矜,齐秦攻之。既疑信陵,诸侯罢之。卒亡大梁,王假厮之。嘉武佐晋文申霸道,作魏世家第十四。
	
	韩厥阴德,赵武攸兴。绍绝立废,晋人宗之。昭侯显列,申子庸之。疑非不信,秦人袭之。嘉厥辅晋匡周天子之赋,作韩世家第十五。
	
	完子避难,適齐为援,阴施五世,齐人歌之。成子得政,田和为侯。王建动心,乃迁于共。嘉威、宣能拨浊世而独宗周,作田敬仲完世家第十六。
	
	周室既衰,诸侯恣行。仲尼悼礼废乐崩,追脩经术,以达王道,匡乱世反之于正,见其文辞,为天下制仪法,垂六之统纪于后世。作孔子世家第十七。
	
	桀、纣失其道而汤、武作,周失其道而春秋作。秦失其政,而陈涉发迹,诸侯作难,风起云蒸,卒亡秦族。天下之端,自涉发难。作陈涉世家第十八。
	
	成皋之台,薄氏始基。诎意適代,厥崇诸窦。栗姬偩贵,王氏乃遂。陈后太骄,卒尊子夫。嘉夫德若斯,作外戚世家十九。
	
	汉既谲谋,禽信于陈;越荆剽轻,乃封弟交为楚王,爰都彭城,以彊淮泗,为汉宗籓。戊溺于邪,礼复绍之。嘉游辅祖,作楚元王世家二十。
	
	维祖师旅,刘贾是与;为布所袭,丧其荆、吴。营陵激吕,乃王琅邪;怵午信齐,往而不归,遂西入关,遭立孝文,获复王燕。天下未集,贾、泽以族,为汉籓辅。作荆燕世家第二十一。
	
	天下已平,亲属既寡;悼惠先壮,实镇东土。哀王擅兴,发怒诸吕,驷钧暴戾,京师弗许。厉之内淫,祸成主父。嘉肥股肱,作齐悼惠王世家第二十二。
	
	楚人围我荥阳,相守三年;萧何填抚山西,推计踵兵,给粮食不绝,使百姓爱汉,不乐为楚。作萧相国世家第二十三。
	
	与信定魏,破赵拔齐,遂弱楚人。续何相国,不变不革,黎庶攸宁。嘉参不伐功矜能,作曹相国世家第二十四。
	
	运筹帷幄之中,制胜于无形,子房计谋其事,无知名,无勇功,图难于易,为大于细。作留侯世家第二十五。
	
	六奇既用,诸侯宾从于汉;吕氏之事,平为本谋,终安宗庙,定社稷。作陈丞相世家第二十六。
	
	诸吕为从,谋弱京师,而勃反经合于权;吴楚之兵,亚夫驻于昌邑,以戹齐赵,而出委以梁。作绛侯世家第二十七。
	
	七国叛逆,蕃屏京师,唯梁为扞;偩爱矜功,几获于祸。嘉其能距吴楚,作梁孝王世家第二十八。
	
	五宗既王,亲属洽和,诸侯大小为籓,爰得其宜,僭拟之事稍衰贬矣。作五宗世家第二十九。
	
	三子之王,文辞可观。作三王世家第三十。
	
	末世争利,维彼奔义;让国饿死,天下称之。作伯夷列传第一。
	
	晏子俭矣,夷吾则奢;齐桓以霸,景公以治。作管晏列传第二。
	
	李耳无为自化,清净自正;韩非揣事情,循埶理。作老子韩非列传第三。
	
	自古王者而有司马法,穰苴能申明之。作司马穰苴列传第四。
	
	非信廉仁勇不能传兵论剑,与道同符,内可以治身,外可以应变,君子比德焉。作孙子吴起列传第五。
	
	维建遇谗,爰及子奢,尚既匡父,伍员奔吴。作伍子胥列传第六。
	
	孔氏述文,弟子兴业,咸为师傅,崇仁厉义。作仲尼弟子列传第七。
	
	鞅去卫適秦,能明其术,彊霸孝公,后世遵其法。作商君列传第八。
	
	天下患衡秦毋餍,而苏子能存诸侯,约从以抑贪彊。作苏秦列传第九。
	
	六国既从亲,而张仪能明其说,复散解诸侯。作张仪列传第十。
	
	秦所以东攘雄诸侯,樗里、甘茂之策。作樗里甘茂列传第十一。
	
	苞河山,围大梁,使诸侯敛手而事秦者,魏厓之功。作穰侯列传第十二。
	
	南拔鄢郢,北摧长平,遂围邯郸,武安为率;破荆灭赵,王翦之计。作白起王翦列传第十三。
	
	猎儒墨之遗文,明礼义之统纪,绝惠王利端,列往世兴衰。作孟子荀卿列传第十四。
	
	好客喜士,士归于薛,为齐扞楚魏。作孟尝君列传第十五。
	
	争冯亭以权,如楚以救邯郸之围,使其君复称于诸侯。作平原君虞卿列传第十六。
	
	能以富贵下贫贱,贤能诎于不肖,唯信陵君为能行之。作魏公子列传第十七。
	
	以身徇君,遂脱彊秦,使驰说之士南乡走楚者,黄歇之义。作春申君列传第十八
	
	能忍卼于魏齐,而信威于彊秦,推贤让位,二子有之。作范睢蔡泽列传第十九。
	
	率行其谋,连五国兵,为弱燕报彊齐之雠,雪其先君之耻。作乐毅列传第二十。
	
	能信意彊秦,而屈体廉子,用徇其君,俱重于诸侯。作廉颇蔺相如列传第二十一。
	
	湣王既失临淄而奔莒,唯田单用即墨破走骑劫,遂存齐社稷。作田单列传第二十二。
	
	能设诡说解患于围城,轻爵禄,乐肆志。作鲁仲连邹阳列传第二十三。
	
	作辞以讽谏,连类以争义,离骚有之。作屈原贾生列传第二十四。
	
	结子楚亲,使诸侯之士斐然争入事秦。作吕不韦列传第二十五。
	
	曹子匕首,鲁获其田,齐明其信;豫让义不为二心。作刺客列传第二十六。
	
	能明其画,因时推秦,遂得意于海内,斯为谋首。作李斯列传第二十七。
	
	为秦开地益众,北靡匈奴,据河为塞,因山为固,建榆中。作蒙恬列传第二十八。
	
	填赵塞常山以广河内,弱楚权,明汉王之信于天下。作张耳陈馀列传第二十九。
	
	收西河、上党之兵,从至彭城;越之侵掠梁地以苦项羽。作魏豹彭越列传第三十。
	
	以淮南叛楚归汉,汉用得大司马殷,卒破子羽于垓下。作黥布列传第三十一。
	
	楚人迫我京索,而信拔魏赵,定燕齐,使汉三分天下有其二,以灭项籍。作淮阴侯列传第三十二。
	
	楚汉相距巩洛,而韩信为填颍川,卢绾绝籍粮饷。作韩信卢绾列传第三十三。
	
	诸侯畔项王,唯齐连子羽城阳,汉得以间遂入彭城。作田儋列传第三十四。
	
	攻城野战,获功归报,哙、商有力焉,非独鞭策,又与之脱难。作樊郦列传第三十五。
	
	汉既初定,文理未明,苍为主计,整齐度量,序律历。作张丞相列传第三十六。
	
	结言通使,约怀诸侯;诸侯咸亲,归汉为籓辅。作郦生陆贾列传第三十七。
	
	欲详知秦楚之事,维周緤常从高祖,平定诸侯。作傅靳蒯成列传第三十八。
	
	徙彊族,都关中,和约匈奴;明朝廷礼,次宗庙仪法。作刘敬叔孙通列传第三十九。
	
	能摧刚作柔,卒为列臣;栾公不劫于埶而倍死。作季布栾布列传第四十。
	
	敢犯颜色以达主义,不顾其身,为国家树长画。作袁盎朝错列传第四十一。
	
	守法不失大理,言古贤人,增主之明。作张释之冯唐列传第四十二。
	
	敦厚慈孝,讷于言,敏于行,务在鞠躬,君子长者。作万石张叔列传第四十三。
	
	守节切直,义足以言廉,行足以厉贤,任重权不可以非理挠。作田叔列传第四十四。
	
	扁鹊言医,为方者宗,守数精明;后世序,弗能易也,而仓公可谓近之矣。作扁鹊仓公列传第四十五。
	
	维仲之省,厥濞王吴,遭汉初定,以填抚江淮之间。作吴王濞列传第四十六。
	
	吴楚为乱,宗属唯婴贤而喜士,士乡之,率师抗山东荥阳。作魏其武安列传第四十七。
	
	智足以应近世之变,宽足用得人。作韩长孺列传第四十八。
	
	勇于当敌,仁爱士卒,号令不烦,师徒乡之。作李将军列传第四十九。
	
	自三代以来,匈奴常为中国患害;欲知彊弱之时,设备征讨,作匈奴列传第五十。
	
	直曲塞,广河南,破祁连,通西国,靡北胡。作卫将军骠骑列传第五十一。
	
	大臣宗室以侈靡相高,唯弘用节衣食为百吏先。作平津侯列传第五十二。
	
	汉既平中国,而佗能集杨越以保南籓,纳贡职。作南越列传第五十三。
	
	吴之叛逆,瓯人斩濞,葆守封禺为臣。作东越列传第五十四。
	
	燕丹散乱辽间,满收其亡民,厥聚海东,以集真籓,葆塞为外臣。作朝鲜列传第五十五。
	
	唐蒙使略通夜郎,而邛笮之君请为内臣受吏。作西南夷列传第五十六。
	
	子虚之事,大人赋说,靡丽多夸,然其指风谏,归于无为。作司马相如列传第五十七。
	
	黥布叛逆,子长国之,以填江淮之南,安剽楚庶民。作淮南衡山列传第五十八。
	
	奉法循理之吏,不伐功矜能,百姓无称,亦无过行。作循吏列传第五十九。
	
	正衣冠立于朝廷,而群臣莫敢言浮说,长孺矜焉;好荐人,称长者,壮有溉。作汲郑列传第六十。
	
	自孔子卒,京师莫崇庠序,唯建元元狩之间,文辞粲如也。作儒林列传第六十一。
	
	民倍本多巧,奸轨弄法,善人不能化,唯一切严削为能齐之。作酷吏列传第六十二。
	
	汉既通使大夏,而西极远蛮,引领内乡,欲观中国。作大宛列传第六十三。
	
	救人于戹,振人不赡,仁者有乎;不既信,不倍言,义者有取焉。作游侠列传第六十四。
	
	夫事人君能说主耳目,和主颜色,而获亲近,非独色爱,能亦各有所长。作佞幸列传第六十五。
	
	不流世俗,不争埶利,上下无所凝滞,人莫之害,以道之用。作滑稽列传第六十六。
	
	齐、楚、秦、赵为日者,各有俗所用。欲循观其大旨,作日者列传第六十七。
	
	三王不同龟,四夷各异卜,然各以决吉凶。略闚其要,作龟策列传第六十八。
	
	布衣匹夫之人,不害于政,不妨百姓,取与以时而息财富,智者有采焉。作货殖列传第六十九。
\end{yuanwen}	

\begin{yuanwen}	
维我汉继五帝末流,接三代(绝)业。周道废,秦拨去古文,焚灭《诗》、《书》,故明堂石室金匮玉版图籍散乱。于是汉兴,萧何次律令,韩信申军法,张苍为章程,叔孙通定礼仪,则文学彬彬稍进,《诗》、《书》往往间出矣。自曹参荐盖公言黄老,而贾生、晁错明申、商,公孙弘以儒显,百年之间,天下遗文古事靡不毕集太史公。太史公仍父子相续纂\footnote{通“缵”,继承。}其职。曰:“於戏!余维先人尝掌斯事,显于唐虞,至于周,复典之,故司马氏世主天官。至于余乎,钦念哉!钦念哉!”

罔罗天下放失旧闻,王迹所兴,原始察终,见盛观衰,论考之行事,略推三代,录秦汉,上记轩辕,下至于兹,著十二本纪,既科条之矣。并时异世,年差不明,作十表。礼乐损益,律历改易,兵权山川鬼神,天人之际,承\footnote{通“乘”,趁着。}敝通变,作八书。二十八宿环北辰,三十辐共一毂,运行无穷,辅拂\footnote{弼。}股肱之臣配焉,忠信行道,以奉主上,作三十世家。扶\footnote{遵循,遵从。}义俶\footnote{tì}傥,不令己失时,立功名于天下,作七十列传。凡百三十篇,五十二万六千五百字,为《太史公书》。序略,以拾遗补艺,成一家之言,厥协六经异传,整齐百家杂语,藏之名山\footnote{指书府。},副在京师,俟后世圣人君子。第七十。
\end{yuanwen}	

我们汉朝继承了五帝的遗风,上接已经被中断的三代的大业。周朝末年王道颓废,秦朝又毁弃了古文,焚烧了《诗》、《书》,因此明堂、石室、金匮、玉版等处所藏图书典籍都已经散乱佚失。这个时候汉朝兴起,萧何修订了法令,韩信申明了兵法,张苍创立各种规章制度,叔孙通制定了礼仪,学术的风气渐渐兴起,《诗》、《书》等古籍也陆续在各地被发现。自从曹参推荐盖公对道家学说的讲述以后,贾生、晁错等人也申明了申不害、商鞅等法家的学问,公孙弘因为了解儒家的学术而显贵,一百年间,天下已已经发现的遗文古事没有不汇集到太史公的府第中的。太史公仍然父子相继地掌管了这一要职。太史公说:“哎呀!我的先人曾经负责管理这些事务,并且在唐、虞时代扬名,直到周朝的时候,再次掌管了这一职务,因此司马氏世代相承的掌管天文星历方面的事务。到了我这一辈,要谨慎地记在心中啊!要谨慎地记在心中啊!”

我搜罗天下散佚的文献,对帝王兴起的事迹,努力究其根源,考察其盛衰,然后依据事实进行论述并加以考订,简略地推考三代,详细地记录了秦、汉时期的事情,向上记录了轩辕,向下一直到现在,著有十二本纪,科分条例。或同时,或异世,年代相差明了,作了十表。礼乐增减,改动律历,兵法权谋,山川的形势,鬼神祭祀,天人的关系,承其敝,通其变,并作八书。二十八宿列星环绕北斗,三十根车辐环集在同一车毂周围,运行无穷,辅弼的股肱大臣,就如同星辰、辐毂相配称,他们忠诚守信、坚守臣道,恭敬地侍奉皇上。作三十世家。他们仗义而为,倜傥不羁,不会让自己失去机会,在天下立下功名。作七十列传。共计一百三十篇,五十二万六千五百字,叫作《太史公书》。大略序述,借以拾遗补充六艺,成为一家的言论,协合六经传释,整理了百家杂说,正本藏在名山中,副本留在京师,留待后世的圣人君子观览。是列传的第七十篇。

\begin{yuanwen}
太史公曰:余述历黄帝以来至太初而讫\footnote{止。},百三十篇。
	
太史良才,寔纂先德。周游历览,东西南北。事覈词简,是称实录。报任投书,申李下狱。惜哉残缺,非才妄续!
\end{yuanwen}

太史公说:我撰述从黄帝以来直到太初年间的历史,共一百三十篇。

\end{document}