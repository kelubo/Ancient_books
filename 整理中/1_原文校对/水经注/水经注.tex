% 水经注
% 水经注.tex

\documentclass[12pt,UTF8]{ctexbook}

% 设置纸张信息。
\usepackage[a4paper,twoside]{geometry}
\geometry{
	left=25mm,
	right=25mm,
	bottom=25.4mm,
	bindingoffset=10mm
}

% 设置字体,并解决显示难检字问题。
\xeCJKsetup{AutoFallBack=true}
\setCJKmainfont{SimSun}[BoldFont=SimHei, ItalicFont=KaiTi, FallBack=SimSun-ExtB]

% 目录 chapter 级别加点(.)。
\usepackage{titletoc}
\titlecontents{chapter}[0pt]{\vspace{3mm}\bf\addvspace{2pt}\filright}{\contentspush{\thecontentslabel\hspace{0.8em}}}{}{\titlerule*[8pt]{.}\contentspage}

% 设置 part 和 chapter 标题格式。
\ctexset{
	chapter/name={卷,},
	chapter/number={\chinese{chapter}}
}

% 设置古文原文格式。
\newenvironment{yuanwen}{\bfseries\zihao{4}}

% 设置署名格式。
\newenvironment{shuming}{\hfill\bfseries\zihao{4}}

% 注脚每页重新编号,避免编号过大。
\usepackage[perpage]{footmisc}

\title{\heiti\zihao{0} 水经注}
\author{郦道元}
\date{北魏}

\begin{document}

\maketitle
\tableofcontents

\frontmatter
\chapter{前言}

《水经注》是南北朝时期北魏郦道元的著作,中国古代地理名著。从书名来看,此书是为另一种称为《水经》的书作注。事情确实如此,三国时期的一位已不知姓名的作者(旧传三国时人桑钦)写了一本名叫《水经》的书,内容非常简略,全书只有八千二百多字,每一条写上此书的河流,都是公式化的:发源、简单的流程、入海,或在何处汇入另一条大河。以淮河为例,从发源、流程到结束,《水经》只写了一百九十多个字。再举条小河的例子,黄河中游古代有一条叫清水的小支流,对于此河的发源,《水经》只说:“清水出河内修武县之北黑山。”但郦道元为这十二个字写了约一千八百字的《注》文。全书《注》文超过《经》文二十多倍。《水经注》是一部三十多万字的巨构,是一部独立的古典名著。

郦道元(?--527),在北魏服官多年。当时中国南北分裂,南方是宋、齐、梁、陈四朝相继;北方经过一场混战,最后由鲜卑族的一支称为拓跋的定局,建立一个王朝,史称北魏。但时局也并不太平,郦道元奔走四方,官事匆忙,却能写出这样一部大书,确实使人称奇。据专家们研究,他写成此书时当在公元6世纪初期。但他本人在北魏孝昌三年(527)被叛将萧宝夤杀害于阴盘驿(今陕西临潼附近),此书稿本他必不随带而是留在首都洛阳,但洛阳后来在北魏的灭亡中全城焚毁。洛阳在当时是北朝最大的城市,仅寺院建筑全城就有一千三百多座,都被烧得荡然无存。北魏朝廷当然有书库,收藏朝廷的档案文卷和文献,包括《水经注》在内:无疑也都化为灰烬。郦道元写作《水经注》在当时是人们都知道的,按《魏书》和《北史》中的《郦道元传》,说他的著作除了《水经注》四十卷外,还有《本志》十三篇和《七聘》等,而且是“皆行于世”。“皆行于世”,用今天的话来说,就是“都已公开出版”。不过当时尚无雕版印刷,只靠传抄流传。像他这样品位的官员,有资格传抄的,除了朝廷以外,也只有少数亲朋好友,所以为数必然极少。《水经注》即使有几部传抄本子,但在洛阳焚毁时,必然与都城同归于尽,如同他的其他著作《本志》《七聘》等一样。但奇迹却发生了,在郦道元的几种著作中,唯独《水经注》四十卷,竟完整地收藏在隋朝的皇家书库里,这是人们后来从皇家的藏书目录《隋书·经籍志》中得知的。皇家藏书,当然只能在皇家书库中束之高阁,人们无缘阅读,不知道郦道元是怎样说“水”的?但以四十卷之数忖度,则其书必然是部大书。隋朝藏书当然由下一个王朝接收,所以唐朝的皇家藏书目录《旧唐书·经志》和《新唐书·艺文志》,也都照录不误。而且与短促的隋朝不同,唐朝重视文化,水经注》一书得以在某种机遇中受到重视。以唐玄宗署名修撰的官书《唐六典》中,记及了《水经》和《水经注》,说桑钦撰写的《水经》(这是误会,桑钦确写过《水经》,但已经亡佚),记载了全国河流一百三十七条,其中包括长江与黄河。郦善长(郦道元字善长)为《水经》作《注》,引及了支流一千二百五十二条。《唐六典》所记载的《水经》和《水经注》,都是完整的足本,但从河流的数量来说,水经注》比《水经》就几乎多了十倍。可惜当时的朝廷书库不是公共图书馆,除非朝廷自己修书,外间人是无缘窥及的。唐朝官修的如《初学记》和《元和郡县图志》等之中,才让人们看到了若干《水经注》引用的词句。

有人认为此书在唐末已流入民间,理由是当时诗人陆龟蒙在他的《和袭美寄怀南阳润卿》诗中有一句“山经水疏不离身”的话。但这个理由并不充分。第一,“山经”当然是《山海经》,但“水疏”并不一定是《水经注》。第二,陆龟蒙在当时是上层文化人,他假使通过什么关系从朝廷书库中录出一本,他也只能是“不离身”,无权让他人传抄。所以此书在唐朝末期已传入民间的话并不足信。
北宋初期,宋太宗赵光义很想发展文化事业,要朝中的文人学士用他的年号编募了几部大书如《太平御览》、《太平寰宇记》、《太平广记》,在这几部近百卷或上百卷的大书中,引及了不少《水经注》的文字。特别是《太平赛宇记》,这是一部全国性的地理书,全国境域都写到了,而几乎各地都引用了《水经注》的文句,而且引用的文句中有泾水、北)洛水、滹沱水等现在的本子上所不见的河流,
四
说明当时朝廷收藏的是从隋唐流传下来的四十卷足本,但是到了宋仁宗景祐年间(1034-1038),朝廷整理藏书,编制《崇文总目》,发现《水经注》已经缺佚了五卷,只剩了三十五卷。现存的此书四十卷,显然是后来的学者在三十五卷中分析出五卷凑数的。所以在宋初编纂的几部大书中所引的不少大河和许多小河,现在的本子中都看不到了。人们有这样的猜测:宋初编篡几部大书,引用《水经注》很多,此书从朝廷书库里取出来,几种大书的编人员都要披阅引用,人多手杂,这五卷可能就是在那时遗失的。几部大书编成以后,参引书籍就收回书库,当时未曾清点,真到编制《崇文总目》时才发现缺佚之事,到此时已经无法弥补了。清代有些研究《水经注》的学者,就以宋初的这几部大书为主,再加上唐朝和以后元明各代引及有关今本不见的文句,补出《泾水》《(北)洛水》《滹沱水》等多篇在景祐缺佚以前的河流。这当然是件好事,但所辑的无非是几条字句,写不出邮道元的文采,不免枯燥乏味,令人遗憾。
朝廷藏书的《水经注》流入民间的较为可靠的时代是北宋。当然,初期获得传抄本的仍然是少数上层文化人,唐宋八大家之一的苏轼(东坡)就是其中之一。他在《寄周安孺茶诗》中说:“嗟我乐何深,水经》亦屡读。”把诵读此书作为他的一种享受。而在他写作的文章如《石钟山记》中,也引用了《水经注》的文句。苏东坡出生于景祐巴年,当时,朝廷收藏的《水经注》也已经只存三十五卷。而他出生后不久,水经注》的第一种刊本“成都府学宫刊本”随即问世。这个刊本究竟刊于何年,因为本子早已亡佚,所以无法论定,但全书只有三十卷,无疑是个劣本,苏东坡当然不会用这样的本子吟诵取乐。在苏东坡五十岁那年(元祐二年,1087),一种到明代尚有流传的刊本即所谓“元祐刊本”刊成付印。这种刊本虽然以后也告亡佚,但明代学者有用此本作底从事校勘的,其中如吴琯刊本至今尚存,我们借此可见,“元祐刊本”已作四十卷,说明元祐年代的这位学者,已把三十五卷分析为四十卷。但苏东坡在《石钟山记》中引用的文句,是在《江水》(即长江)篇中的,并不在宋初缺佚的五卷之内,而这段三十多字的文句,从“元祐刊本"到现在流行的各种版本,都未曾收入。说明苏东坡当年所得的钞本,虽然也是景祐缺佚以后的本子,但比以后的本子更为完整。
从今天来说,《水经注》因为它的内容丰富,已经近乎一种百科全书,许多行业的学者,如历史、地理、河川、水利,甚至动植物、矿物等方面,都有参考价值。但苏东坡不是一位河川水利学者,是个文学家,他之所以说出“乐何深”的话,显是欣赏此书的生动文字。例如在《石钟山记》中所引而为现在所失的一段:“石钟山西枕彭蠡,连峰迭嶂,壁立峭削,而西、南、北皆水,四时如一,白波撼山,响如洪钟,因名。”(《赛宇记》《名胜志》等都引此一段,但以苏引最为完整)就是对此处山水的生动描写。因此说此书现在对学术界的许多行业都有价值,但此书的声名,开始无疑是从它的绝妙文章传播开来的。明末一位史学家张岱曾经说:“古人记山水,太上郦道元,其次柳子厚,在近时则袁中郎。”柳子厚即柳宗元,是唐宋八大家之一,他所写的游记文章《永州八记》擅名古今。袁中郎即明著名文学家袁宏道,也以写游记出名,有《袁中郎游记》流传,脍炙人口。但他们都在郦道元之下。著名如柳宗元,为什么在描写风景的功夫上不及郦道元,我在本书中已经举例说明。
自从明朝开始,水经注》的本子,包括刊本和钞本纷纷问世,研究此书的学者也先后相继,从各个方面从事钻研,形成一门专门的学问--郦学,而且由于研究的内容和目的不同而出现了三个学派。第一个是考据学派,因为此书从南宋以来,经过多次雕版和辗转传抄,到了明代,许多流行的本子,已经到了错误百出、不堪卒读的地步。所以有许多郦学家都在考证校勘上下工夫,最后获得了万历四十三年(1615)以朱谋伟为首校成的《水经注笺》,被清初学者称赞为“三百年来一部书”,也就是说在明朝一朝中一切著述中的唯-一部好书。但其实此书仍然存在许多缺陷。清朝初年,那学家一时涌现,大家各找不同版本,各自深校细勘,特别是乾隆年间出现的郦学三大家:全祖望、赵一清、戴震。三人中以戴年龄最幼,得以因缘进入四库馆参与《四库全书》编修,从事《水经注》的考证校勘。他以全、赵成果为主要基础,参以其他各本,特别是当时只能在四库馆内见到的《永乐大典》本,于乾隆三十九年(1774)校定了一种受到爱好山水地理的乾隆称赞的版本,随即在皇家出版机构武英殿以活字排印出版,称为“武英殿聚珍本”(简称“殿本”)。此本除了宋初缺佚的五卷无法弥补外,显然是许多版本中首屈一指之本,以后各省纷纷翻刻重版。民国以后,各大书局又铅排出版,成为此书流行最广、印数最多的版本。虽然此本还存在若干可以继续校勘之处,但总的说来,考据学派的事业已经基本完成。
第二个学派是地理学派。早在明代,已有学者认为《水经注》是河川水利之书,也就是当时所谓的经世致用之书,其重要性首在河川地理研究。清代持这种观点的郦学家也有不少。最后由清末民初的杨守敬、熊会贞师生二人,以地理为主(当然也有校勘成果),编篡成《水经注疏》一书,是所有此书版本注疏量最大之本。他们师生并同时编绘了《水经注图》,书图二者,至今都是研究历史地理特别是古代山川水利的极有价值的版本。
第三派是词章学派。此派认为《水经注》之所以不同凡响,全靠郦道元的绝妙文章,尤其是其中的山水描写。有人竟对此着迷,认为此书除了引人入胜的生动描写以外,没有别的东西。确实,在历史上大量书籍亡佚的情况下,此书能够子然独存,并且形成一门学问,其开端无疑是因为郦道元的文章出众。如大文豪苏东坡所说的“乐何深”,就是因为此书在词章上让人爱不释手的缘故。直到民国时代,中学教科书上还常常选载此书描写风景的若干片断作为教材,给学生欣赏享受,学习研究。
《水经注》是一部奇书,郦学是一门内容浩瀚的学问。现在除了国内以外,郦学研究早已流向国外,如日本和西欧,都有不少这门学问的研究者,研究的领域极广,课题很多。但是这些都是郦学家们的工作,不关一般读者的事。对于广大读者来说,还是苏东坡的那首诗:“嗟我乐何深,《水经》亦屡读。”我应中华书局之邀写作此书,目标也是针对广大一般读者,因为此书可以为我们提供文学上的欣赏和享受。清代学者称赞此书的词章:“片言只字,妙绝古今。”《水经注》不同于有些有争议的书,它可以稳稳当当地坐在历史名著的座位上,让读者在此书中获得文字咀嚼、风雅追求和情操陶冶的享受。或许也可以提高读者的写作能力,甚至吸引读者从事对此书某些专题的研究。
因为《水经注》有四十卷,其中所记的河流,有的一河分成数卷,有的一河独占一卷,但多数是几条河流合成一卷。从《河水》到《渐江水》渐江水以下的不计),书中立为标题的河流共有一百二十二条。所以本书每卷都有一个“题解”,把该卷立题的河流是今天的什么河流作点说明。因为此书写作至今已逾一千四百年,除了名称的变化以外,河流本身的变化也很不小。这一百多条河流中,有的至今仍是全国大河,有的已经移动或消失,也有个别在写书时并不存在,所以必须让读者知道。此外就是“选文”,四十卷中,每卷都有几段入选的文章。其中选入最多的当然是“片言只字,妙绝古今”的山水描写。也有一些是以史为鉴,到今天仍然铿锵有声的词句。“选文”之下,我都做了一点“注释”加以说明,王东同志协助我另外做了语词上的“注释”。最后就是每段“选文”的“译文”,这是一项非常棘手的工作,我往年曾邀集几位在文学上很有造诣的朋友做了这项难事,现在还不得不仍然依靠当年所作的充数,郦道元的神笔,显然不是我们的语体文所能表达的,何况“选文”都是《注》文中的精华。对于这方面,尽管我们曾作了较大的努力,费了不少推敲工夫,但实在是力不从心,只好请读者原谅了。


原四十卷,北宋初已亡佚五卷,后人分割三十五卷以足四十卷之数。道元(469?—527),字善长,范阳涿县(今属河北)人。历仕宣武帝、孝明帝两朝,先后出任冀州刺史于劲镇东将军府长史、鲁阳太守、东荆州刺史、河南尹,后迁御史中尉。在地方和中央,都以为政严猛著称。雍州刺史萧宝夤图谋叛乱,忌恨道元的朝贵奏请派他为关右大使,进行安抚。萧宝夤害怕道元之来不利于己,在他将要到达长安时派兵围攻,加以杀害。

道元好学博闻,广览奇书。足迹所至,大至从长城以南,到秦岭、淮河以北。他在书中征引的前代和当时地理著作,即达三百七十余种,包括一些南朝人的著述。自序说“访渎搜渠,缉而缀之”,所以很多材料是实际调查所得。《水经》只记载了水道一百三十七条,而郦注却有一千二百五十二条,增加八倍多,注文共约三十万宇,也比经文增多二十倍。
该书以水道为纲,连带叙述流经地区的山陵、湖泊、郡县、城池、关塞、名胜、亭障,以及土壤、植被、气候、水文和社会经济、民俗风习等各方面,还记载了各地有关的历史故事。书中记录作者所见的碑刻,共三百余块,利用它们作为帮助确定水道流经的依据。道元注意到水道源头的伏流,和故河道之下还有相当多的地下水等现象,对于水源的大小、湖泊的盈竭、水色的清浊、泥沙的堆积、洪水的涨落等水文变化,都很重视。他用发展变化的观点考察地理现象,对每条水道都追溯到可能追溯的最早时期。还记载了各地水利设施近三十处,称颂了许多伟大工程。对于从书本或实际调查都未能弄清的问题,道元采取谨慎态度,表示“未知所从”、“非所详也”,这样的存疑有七十余处。《水经注》对中国地理学的发展有重大贡献,在中国以至世界地理学史上,都占有重要地位,同时书中还保存了大量历史和历史地理的资料。《水经注》中山川景物的描写,作为文学作品,也得到很高评价。由于郦道元是北朝人,关于南方水系的记录不免简略,时有错误。
旧本《水经注》经文与注文混淆在一起,字句脱误甚多。清朝学者全祖望(1705-1755)、赵一清(1711一1764)、戴震(1724一1777)分别进行了区分经注的整理校订工作,各自取得良好成绩。后来有人因赵、戴两家整理校订的结果很多相同,因而认为戴震盗窃了赵一清的学术成果,是没有根据的。清末,扬守敬集以往研究之大成,撰《水经注疏》,弟子熊会贞续加补修,极为详尽,是《水经注》研究最为完备之书,已由科学出版社刊印,极便参览。


前言
《玄申记》曰:天下之多者水焉,浮天载地,高下无不至,万物无不润。盖江河海洋,占地球表面面积十分之七以上。如无载水之江河海洋,即不成其为地球;地球如果无水,一切生物和人类即无法生存;自古及今,水乃不可须臾离者。固然水之为害也,怀山襄陵,浩浩滔天,漂没生命财产,数以亿计。然若禾苗无水,必至枯槁;厂矿无水,必将停产;土地无水,沃壤亦将成为沙漠。是则对于水,务在治之得法,去害为利而已。所以古今中外,莫不以治水为经国大计。职是之故,关于水之记载,不绝于书。
我国古有桑钦之《水经》,载水流一百三十七条,注之者有郭璞、郦道元两家。郭注久佚。郦之《水经注》载水一千二百五十二条,抑且沿流之山陵川泽,城池关隘,名胜古迹,莫不备载。览郦注者不仅可以知水流之支分派别,亦且可以兼知该处之人文史地,鉴住知今,乡所裨益。因此,斯书自宋版以来,有多种版本,笺校讨论者颇不乏人。《四库全书总目》史部地理类收有郦氏《水经注》三种版本:一为《永乐大典》本。二为清朝沈炳巽所撰《水经注集释订讹》四十卷本,沈氏据明朝嘉靖时黄省曾本加以注释厘订。三为情朝赵一清所撰《水经注释》四十卷、《刊误》十二卷本,赵氏采全祖望氏之说撰成。
王国维《观堂集林》卷十二,载有郦注多种刊本跋文:一、《宋刊水经注残本跋》,此为南宋初刊本,残存十一卷有奇。二、《永乐大典本水经注跋》,王氏唯见河水至丹水二十卷,虽知有张穆所校《大典》本全书,而未及见。三、《明抄本水经注跋》,此本为海盐朱氏所藏,与宋残本及吴门顾氏所藏明影宋抄本行款并同。四、《朱谋水经注笺跋》,为明代朱谋所校笺,王氏谓朱用吴《古今逸史》本为底本而校笺者。五、《孙潜夫校水经注残本跋》。六、《聚珍本戴校水经注跋》,略言戴震校本奄有诸家之胜,而尽掠诸家厘订之功为己功,语婉而责深。王氏在诸种版本跋语中还言及柳大中本、归震川本、赵清常本,全谢山本等,就不列举。
傅增湘《藏园群书经眼录》史部地理类,载有郦氏《水经注》八种版本:一、《水经注》四十卷,宋刊本,仅存十二卷,其中还有缺页。案此即王国维所言宋刊残本十一卷有奇者。二、《水经注》四十卷,明嘉靖十三年黄省曾刊本。三、《水经注》四十卷,明崇祯二年武林严忍公刊本。四、《水经注》四十卷。清康熙时项絪刊本,有何焯校记。五、《水经注》四十卷,明写本,韩渌卿应陛手校,据朱谋本。六、《水经注笺》四十卷,明朱谋撰,万历四十三年李长度刊本。七、《水经注》四十卷,明朱谋撰,万历四十三年李长庚刊本,孙潜、袁廷梼手校并跋。八、《水经注释》四十卷,首一卷、附录二卷、《水经注笺刊误》十二卷,清赵一清撰,乾隆五十一年赵一清小山堂刊本。案此当即《四库全书总目》所载之第三种刊本。
王重民《中国善本书提要》史部地理类,载有郦注明刊本四种。一、《水经注》四十卷,明吴琯校刻本。此即《古今逸史》本。二、《水经注》四十卷,明崇祯时刻本,有评阅者姓氏页,载钟惺伯敬、朱之臣无易、谭元春友夏等人。案此即傅增湘所言崇祯二年武林严忍公刊本。三、《水经注笺》四十卷,明万历间(四十三年)刻本,明李长庚订,朱谋笺,孙汝澄、李克家同校,有李长庚、黄省曾及朱谋诸人之序。四、《水经住笺》四十卷,订、笺、校者皆同上,唯仅载李长庚序。案此三、四两种版本,与上述《藏园群书经眼录》六、七两种版本同出一源,即为朱谋笺校本。
合以上诸书所载,宋刊残本以下,郦氏《水经注》旧本有《永乐大典》本、明吴琯《古今逸史》本、明黄省曾本、明朱谋本、明抄本(抄自宋本)、明严忍公刊本、清沈炳巽本(承明黄省曾本而来)、清赵一清本、清全祖望本、清聚珍版戴震本等,此皆当时所称善本。惟宋刊残本仅十一卷有奇,王国维所见《永乐大典》本仅二十卷,孙潜夫、袁寿阶校本亦只存十五卷,皆为残缺不全之本。而聚珍版戴氏本又有掠人之美之嫌,故王国维氏以为不得不推明抄本为旧本第一。朱谋《水经注笺》有大功于郦书,亦其次也。
及至近代,有王先谦之《水经注》合校本,有杨守敬及其弟子熊会贞之《水经注疏》及《水经注图》,近又由上海人民出版社印行以明朱谋《水经注笺》为底本之王国维《水经注校》,是则致力于郦注者,盖时有人在。而王先谦之《合校水经注》,吸取诸家之长,集其大成,岂非善之善者乎!岳麓书社以王先谦合校本之善也,即据此本为底本,将郦注原文校点出版,收入《古典名著普及文库》,管巧灵编辑属书数言于卷首。我闻之而欣然命笔。盖岳麓书社出版本地学者手校之古典名著,光大学术,实为盛世盛举,孰曰不宜!不仅此也,即此新出善本,再览一遍古代水道变迁与沿流城垒盛衰之迹及其所以然,鉴往思来,岂不大有助于地平天成之道乎!故略举《四库全书总目》以来诸书所载《水经注》之若干版本,以见古籍传布之非易,亟求所以珍惜与应用之方,则《水经注》历世校订者之功为不虚费,抑亦重版此书莫大之幸焉。甲戌年孟夏之月韩国磐志于老榕书屋。(韩国磐)

\chapter{序}

《序》曰:《易》称天以一生水,故气微于北方,而为物之先也。《玄中记》曰:天下之多者水也,浮天载地,高下无所不至,万物无所不润。及其气流届石,精薄肤寸,不崇朝而泽合灵宇者,神莫与并矣。是以达者不能恻其渊冲,而尽其鸿深也。昔《大禹记》著山海,周而不备;《地理志》其所录,简而不周;《尚书》、《本纪》与《职方》俱略;都赋所述,裁不宣意;《水经》虽粗缀津绪,又阙旁通。所谓各言其志,而罕能备其宣导者矣。今寻图访赜者,极聆州域之说,而涉土游方者,寡能达其津照,纵仿佛前闻,不能不犹深屏营也。余少无寻山之趣,长违问津之性,识绝深经,道沦要博,进无访一知二之机,退无观隅三反之慧。独学无闻,古人伤其孤陋;捐丧辞书,达土嗟其面墙。默室求深,闭舟问远,故亦难矣。然毫管窥天,历筒时昭,饮河酌海,从性斯毕。窃以多暇,空倾岁月,辄述《水经》,布广前文。《大传》曰:大川相间,小川相属,东归于海。脉其枝流之吐纳,诊其沿路之所躔,访渎搜渠,缉而缀之。《经》有谬误者,考以附正文所不载,非经水常源者,不在记注之限。但绵古芒昧,华戎代袭,郭邑空倾,川流戕改,殊名异目,世乃不同。川渠隐显,书图自负,或乱流而摄诡号,或直绝而生通称,在诸交奇,洄湍决澓,躔络枝烦,条贯系夥。《十二经》通,尚或难言,轻流细漾,固难辨究,正可自献径见之心,备陈舆徒之说,其所不知,盖阙如也。所以撰证本《经》,附其枝要者,庶备忘误之私,求其寻省之易。

\mainmatter

\chapter{卷一 河水}

《水经注》以《河水》开卷,河水就是黄河。上古的地名比后代简单,黄河就称“河”,长江就称“江”。秦始皇开创郡县制以后,一个地名普遍地分成专名和通名两部分,譬如“北京市”、“丰台区”,“北京”和“丰台”都是专名,“市”和“区”都是通名。行政区域的地名分成专名和通名两部分以后,自然地名也相继出现专名和通名两部分,譬如“燕山”和“八达岭”,“燕”和“八达”都是专名,“山”和“岭”都是通名。河流的通名早期称“水”,黄河称为“河水”,长江称为“江水”,直到《水经注》时代还是这样。到后来,“河”与“江”两个专名,也被人们当作河流的通名使用,如“永定河”、“松花江”等等,现在的“黄河”,“黄”是专名,“河”是通名。“河”与“江”,原是黄河和长江的专名,后来成为一切河流的通名,为时已经很久了。

\begin{yuanwen}
仑墟在西北,三成为昆仑丘\footnote{三级的土丘称为昆仑丘。出自《尔雅·释丘》:“丘一成为敦丘,再成为陶丘......三成为昆仑丘。”晋代郭璞注:“昆仑山三重,故以名云。”意思是说:昆仑山有三级,所以叫昆仑丘。成,级,层。}。《昆仑说》\footnote{书名,不详。}曰:昆仑之山三级:下曰樊桐,一名板桐;二曰玄圃,一名阆风;上曰层城,一名天庭,是为太帝\footnote{天帝。}之居。\footnote{《水经》和《水经注》都以“昆仑”开始。卷一之中有“昆仑墟”、“昆仑虚”、“昆仑丘”、“昆仑山”等,而昆仑山至今仍是我国一条重要的山脉。“昆仑"是外来语,成书于战国前期的《山海经》中已有此词,所以这个外来语引入中国很早。因为是外来语,所以“昆仑”一词在各种古代文献中有许多不同的汉译。在《水经注》中,《河水》篇同卷又译“金陈”,卷三十六《温水》篇中译作“金潾”。上面各处的“昆仑”、“金陈”、“金潾”,都作为地名。但《温水》篇也有一个“昆仑”,却是作为族名的。由于《水经注》的《经》文和《注》文,都以“昆仑”开始,所以此词值得重视。不过因为是外来语,因而不懂它是什么意思。在《河水》篇同卷中也用梵语对“昆仑”作过解释,这是释氏《西域记》的话:“阿耨达太山,其上有大渊水,官殿楼观甚大焉。山,即昆仑山也。”说明“昆仑山”在梵语中称为“阿耨达太山”,但也查不出“阿耨达”在梵语中是什么意思。所以“昆仑”是一个已经消失的民族的语言。像“昆仑”这类已经消失的民族语言地名,在我国至今还有不少。}
\end{yuanwen}





\begin{yuanwen}
aaa去嵩高五万里,地之中也。
《禹本纪》与此同。高诱称河出昆山,伏流地中万三千里,禹导而通之;出积石山。案《山海经》:自昆仑至积石千七百四十里。自积石出陇西郡至洛,准《地志》可五千余里。又案《穆天子传》:天子自昆山入于宗周,乃里西土之数。自宗周瀍水以西,至于河宗之邦、阳纤之山,三千有四百里,自阳纤西至河首四千里,合七千四百里。《外国图》又云:从大晋国正西七万里,得昆仑之墟,诸仙居之。数说不同,道阻且长,经记绵褫,水陆路殊,径复不同,浅见末闻,非所详究,不能不聊述闻见,以志差违也。
\end{yuanwen}

\begin{yuanwen}
其高万一千里,《山海经》称方八百里\footnote{text},高万仞\footnote{text}。郭景纯以为自上二千五百余里\footnote{text}。《淮南子》称高万一千里百一十四步三尺六寸\footnote{text}。
\end{yuanwen}

\begin{yuanwen}
aaa河水《春秋说题辞》曰:河之为言荷也,荷精分布,怀阴引度也。《释名》曰:河,下也,随地下处而通流也。《考异邮》曰:河者,水之气,四渎之精也,所以流化。《元命苞》曰:五行始焉,万物之所由生,元气之腠液也。《管子》曰:水者,地之血气,如筋脉之通流者,故曰水具财也。五害之属,水最为大。水有大小,有远近,水出山而流入海者,命曰经水;引他水入于大水及海者,命曰枝水;出于地沟,流于大水,及于海者,又命曰川水也。《庄子》曰:秋水时至,百川灌河,经流之大。《孝经援神契》曰:河者,水之伯,上应天汉。《新论》曰:四读之源,河最高而长,从高注下,水流激峻,故其流急。徐干《齐都赋》曰:川读则洪河洋洋,发源昆仑,九流分逝,北朝沧渊,惊波沛厉,浮沫扬奔。《风俗通》曰:江、河、淮、济为四渎。渎,通也,所以通中国垢浊。《白虎通》曰:其德著大,故称渎。《释名》曰:渎,独也。各独出其所而入海。
出其东北陬,《山海经》曰:昆仑虚在西北,河水出其东北隅。《尔雅》曰:河出昆仑虚,色白;所渠并千七百一川,色黄。《物理论》曰:河色黄者,众川之流,盖浊之也。百里一小曲,千里一曲一直矣。汉大司马张仲议曰:河水浊,清澄一石水,六斗泥。而民竞引河溉田,令河不通利。至三且,桃花水至则河决,以其噎不泄也。禁民勿复引河,是黄河兼浊河之名矣。《述征记》曰:盟律、河津恒浊,方江为狭,比淮、济为阔。寒则冰厚数丈,冰始合,车马不敢过,要须狐行,云此物善听,冰下无水乃过。人见狐行,方渡。余案《风俗通》云:里语称狐欲渡河,无如尾何。且狐性多疑,故俗有狐疑之说,亦未必一如缘生之言也。
屈从其东南流,入渤海。
\end{yuanwen}

\begin{yuanwen}
《山海经》\footnote{text}曰:南即从极之渊也\footnote{text},一曰中极之渊,深三百仞,惟冯夷都焉\footnote{text}。《括地图》\footnote{text}曰:冯夷恒乘云车驾二龙\footnote{text}。河水又出于阳纡、陵门之山\footnote{text},而注于冯逸之山。《穆天子传》\footnote{text}曰:天子西征,至阳纡之山,河伯冯夷之所都居\footnote{text},是惟河宗氏\footnote{text}。天子乃沉珪璧礼焉\footnote{text}。河伯乃与天子披图视典\footnote{text},以观天子之宝器:玉果、璇珠、烛银、金膏等物\footnote{text},皆《河图》所载\footnote{text}。河伯以礼,穆王视图,方乃导以西迈矣\footnote{text}。===========================粤在伏羲,受龙马图于河,八卦是也。故《命历序》曰:《河图》,帝王之阶,图载江河、山川、州界之分野。后尧坛于河,受《龙图》,作《握河记》。逮虞舜、夏、商,咸亦受焉。李尤《盟津铭》:洋洋河水,朝宗于海,径自中州,《龙图》所在。《淮南子》曰:昔禹治洪水,具祷阳纡。盖于此也。高诱以为阳纡秦薮,非也。释氏《西域记》曰:阿褥达太山,其上有大渊水,宫殿楼观甚大焉。山,即昆仑山也。《穆天子传》曰:天子升于昆仑,观黄帝之宫,而封丰隆之葬。丰隆,雷公也。黄帝宫,即阿褥达宫也。
\end{yuanwen}

\begin{yuanwen}

\end{yuanwen}\begin{yuanwen}

\end{yuanwen}




其山出六大水,山西有大水,名新头河。郭义恭《广志》曰:甘水也,在西域之东,名曰新陶水,山在天竺国西,水甘,故曰甘水。有石盐,白如水精,大段则破而用之。康泰曰:安息、月氏、天竺至伽那调御,皆仰此盐。释法显曰:度葱岭,已入北天竺境。于此顺岭西南行十五日,其道艰阻,崖岸险绝,其山惟石,壁立于仞,临之目眩,欲进则投是无所。下有水,名新头河。昔人有凿石通路施倚梯者,凡度七百梯,度已,蹑悬絙过河,河两岸,相去咸八十步。九译所绝,汉之张骞、甘英皆不至也。余诊诸史传,即所谓罽宾之境,有盘石之隥,道狭尺余,行者骑步相持,絙桥相引,二十许里,方到悬度,阻险危害,不可胜言。敦义恭曰:乌秅之西,有悬度之国,山溪不通,引绳而度,故国得其名也。其人山居,佃于石壁间,累石为室。民接手而饮,所谓猿饮也。有白草、小步马,有驴无牛,是其悬度乎?释法显又言:度河便到乌长国。乌长国即是北天竺,佛所到国也。佛遗足迹于此,其迹长短在人心念,至今犹尔,及晒衣石尚在。新头河又西南流,屈而东南流,径中天竺国。两岸平地,有国名毗茶,佛法兴盛。又径蒲那般河。河边左右,有二十僧伽蓝。此水径摩头罗国,而下合新头河。自河以西,天竺诸国,自是以南,皆为中国,人民殷富。中国者,服食与中国同,故名之为中国也。泥洹已来,圣众所行,威仪法则,相承不绝。自新头河至南天竺国,迄于南海,四万里也。释氏《西域记》曰:新头河经罽宾、犍越、摩诃刺诸国,而入南海是也。阿耨达山西南有水,名遥奴;山西南小东有水,名萨罕;小东有水,名恒伽。此三水同出一山,俱入恒水。康泰《扶南传》曰:恒水之源,乃极西北,出昆仑山中,有五大源,诸水分流,皆由此五大源。枝扈黎大江出山西北流,东南注大海。枝扈黎,即恒水也,故释氏《西域记》有恒曲之目。恒北有四国,最西头恒曲中者是也。有拘夷那褐国。《法显传》曰:恒水东南流,径拘夷那褐国南,城北双树间,有希连禅河。河边,世尊于此北首般泥洹,分舍利处。支僧载《外国事》曰:佛泥洹后,天人以新白緤裹佛,以香花供养,满七日,盛以金棺,送出王宫,度一小水,水名醯兰那,去王宫可三里许,在宫北。以栴檀木为薪,天人各以火烧薪,薪了不燃,大迦叶从流沙还,不胜悲号,感动天地,从是之后,他薪不烧而自然也。王敛舍利,用金作斗,量得八斛四斗,诸国王、天龙神王各得少许。赍还本国,以造佛寺。阿育王起浮屠于佛泥洹处,双树及塔,今无复有也。此树名婆罗树,其树花名婆罗佉也。此花色白如霜雪,香无比也。竺枝《扶南记》曰:林杨国去金陈国步道二千里,车马行,无水道。举国事佛,有一道人命过烧葬,烧之数千束樵,故坐火中,乃更著石室中,从来六十余年,尸如故不朽,竺枝目见之。夫金刚常住,是明永存,舍利刹见,毕天不朽,所谓智空罔穷,大觉难测者矣。其水乱流注于恒。恒水又东径毗舍利城北。释氏《西域记》曰:毗舍利,维邪离国也。支僧载《外国事》曰:维邪离国去王舍城五十由旬,城周圆三由旬,维诘家在大城里宫之南,去宫七里许,屋宇坏尽,惟见处所尔。释怯显云:城北有大林重阁,佛住于此,本奄婆罗女家施佛起塔也。城之西北三里,塔名放弓仗。恒水上流有一国,国王小夫人生肉胎,大夫人妒之,言汝之生,不祥之征,即盛以木函,掷恒水中。下流有国王游观,见水上木函,开看,见千小儿端正殊好,王取养之,遂长大,甚勇健,所往征代,无不摧服。次欲伐父王本国,王大愁忧。小夫人问:何故愁忧?王曰:彼国王有千子,勇健无比,欲来伐吾国,是以愁尔。小夫人言:勿愁,但于城西作高楼,贼来时,上我置楼上,则我能却之。王如是言。贼到,小夫人于楼上语贼云:汝是我子,何故反作逆事?贼曰:汝是何人,云是我母?小夫人曰:汝等若不信者,尽张口仰向。小夫人即以两手捋乳,乳作五百道,俱坠千子口中。贼知是母,即放弓仗。父母作是思惟,皆得辟支佛。今其塔犹在,后世尊成道,告诸弟子,是吾昔时放弓仗处。后人得知,于此处立塔,故以名焉。千小儿者,即贤劫千佛也。释氏《西域记》曰:恒曲中次东,有僧迎扇奈揭城,佛下三道宝阶国也。《法显传》曰:恒水东南流,径僧迦施国南。佛自忉利天东下三道宝阶,为母说法处。宝阶既没,阿育王于宝阶处作塔。后作石柱,柱上作师子像,外道少信,师子为吼,怖效心诚。恒水又东径罽宾饶夷城。城南接恒水,城之西北六七里,恒水北岸,佛为诸弟子说法处。恒水又东南径沙祗国北。出沙祗城,南门道东,佛嚼杨枝刺土中,生长七尺,不增不减,今犹尚在。恒水又东南,径迦维罗卫城北。故净王宫也,城东五十里有王园,园有池水,夫人入池洗浴,出北岸二十步,东向举手,扳树生太子。太子堕地,行七步,二龙吐水浴太子,遂成井池,众僧所汲养也。太子与难陀等扑象角力,射箭入地。今有泉水,行旅所资饮也。释氏《西域记》曰:城北三里恒水上,父王迎佛处,作浮图,作父抱佛像。《外国事》曰:边维罗越国今无复王也。城池荒秽,惟有空处,有优婆塞姓释,可二十余家,是昔净王之苗裔,故为四姓,住在故城中,为优婆塞,故尚精进,犹有古风。彼日俘图坏尽,条王弥更修治一浮图,私河条王送物助成,今有十二道人住其中。太子始生对,妙后所扳树,树名须诃。阿育王以青石作后扳生太子像。昔树无复有,后诸沙门取昔树栽种之,展转相承到今,树枝如昔,尚荫石像。又太子见行七步足迹,今日文理见存。阿育王以青石挟足迹两边,复以一长青石覆上。国人今日恒以香花供养,尚见足七形,文理分明。今虽有石覆无异,或人复以数重吉贝,重覆贴着石上,逾更明也。太子生时,以龙王夹太子左右,吐水浴太子,见一龙吐水暖,一龙吐水冷,遂成二池。今尚一冷一暖矣。太子未出家前十日,出往王田阎浮树下坐,树神以七宝奉太子,太子不受,于是思惟欲出家也。王田去宫一据。据者,晋言十里也。太子以三月十五日夜出家,四天王来迎,各捧马足。尔时诸神天人侧塞,空中散天香花。此时以至河南摩强水,即于此水边作沙门。河南摩强水在迦维罗越北,相去十由旬。此水在罗阅祗瓶沙国,相去三十由旬。菩萨于是暂过,瓶沙王出见菩萨,菩萨于瓶沙随楼那果园中住一日,日暮便去半达钵愁宿。半达,晋言白也;钵愁,晋言山也。白山北去瓶沙国十里,明旦便去,暮宿昙兰山,去白山六由旬。于是径诣贝多树,贝多树在阅祗北,去昙兰山二十里。太子年二十九出家,三十五得道,此言与经异,故记所不同。竺法维曰:迎维卫国,佛所生天竺国也。三千日月、万二千天地之中央也。康泰《扶南传》曰:昔范旃时,有嘾杨国人家翔梨,尝从其本国到天竺,展转流贾至扶南,为旃说天竺土俗,道法流通,金宝委积,山川饶沃,恣所欲,左右大国,世尊重之。旃问云:今去何时可到,几年可回?梨言:天竺去此,可三万余里,往还可三年逾。及行,四年方返,以为天地之中也。恒水又东径蓝莫塔。塔边有池,池中龙守护之。阿育王欲破塔,作八万四千塔,悟龙王所供,知非世有,遂止。此中空荒无人,群象以鼻取水洒地,若苍梧、会稽,象耕、鸟耘矣。恒水又东至五河口,盖五水所会,非所详矣。阿难从摩竭国向毗舍利,欲般泥洹,诸天告阿阇世王,王追至河上,梨车闻阿难来,亦复来迎,俱到河上。阿难思惟,前则阿阇世王致恨,却则梨车复怨,即于中河,入火光三昧,烧具两般泥洹。身二分,分各在一岸,二王各持半舍利,还起二培。渡河南下一由巡,到摩竭提国巴连弗邑。邑,即是阿育王所治之城。城中宫殿皆起墙阙,雕文刻镂,累大石作山,山下作石室,长三丈,广二丈,高丈余,有大乘婆罗门子,名罗汰私婆,亦名文殊师利,住此城里、爽悟多智,事无不达,以清净自居,国王宗敬师事之。赖此一人,宏宣佛法,外不能陵。凡诸国中,惟此城为大,民人富盛,竟行仁义。阿育王坏七塔,作八万四千塔。最初作大塔,在城南二里余,此塔前有佛迹,起精舍,北户向塔,塔南有石柱,大四五围,高三丈余,上有铭,题云:阿育王以阎浮提布施四方。僧还以钱赎塔。塔北三百步,阿育王于此作泥犁城,城中有石柱,亦高三丈余,上有师子柱,有铭记,作泥犁城因缘,及年数日月。恒水又东南径小孤石山。山头有石室,石室南向,佛昔坐其中,天帝释以四十二事问佛,佛一一以指画石,画迹故在。恒水又西径王舍新城。是阿阇世王所造,出城南四里,入谷至五山里,五山周围,状若城郭,即是蓱沙王旧城也。东西五六里,南北七八里,阿阇世王始欲害佛处。其城空荒,又无人径,入谷傅山,东南上十五里,到耆阇崛山,未至顶三里,有石窟南向,佛坐禅处。西北四十步,复有一石窟,阿难坐禅处。天魔波旬化作雕鹫恐阿难,佛以神力,隔石舒手摩阿难肩,怖即得止。鸟迹、手孔悉存,故曰雕鹫窟也。其山峰秀端严,是五山之最高也。释氏《西域记》云:耆阇崛山在阿耨达王舍城东北,西望其山,有两峰双立,相去二三里,中道鹫鸟,常居其岭,土人号曰耆阇崛山。胡语耆阇,鹫也。又竺法维云:罗阅祗国有灵鹫山,胡语云耆阇崛山。山是青石,石头似鹫鸟。阿育王使人凿石,假安两翼两脚,凿治其身,今见存,远望似鹫鸟形,故曰灵鹫山也。数说不同,远迩亦异,今以法显亲宿其山,诵《首楞严》,香华烘养,闻见之宗也。又西径迦那城南。三十里,到佛苦行六年坐树处,有林木。西行三里,到佛人水洗浴、天王按树枝得扳出池处。又北行二里,得弥家女奉佛乳糜处。从此北行二里,佛于一大树下石上东向坐食糜处,树石悉在,广长六尺,高减二尺。国中寒暑均调,树木或数千岁,乃至万岁。从此东北行二十里,到一石窟,菩萨人中,西向结跏趺坐,心念若我成道,当有神验。石壁上即有佛影见,长三尺许,今犹明亮。时天地大动,诸天在空言,此非过去当来诸佛成道处,去此西南行,减半由旬,贝多树下,是过去当来诸佛成道处。诸天导引菩萨起行,离树三十步,天授吉祥草,菩萨受之,复行十五步,五百青雀飞来,绕菩萨三匝西去。菩萨前到贝多树下,敷吉祥草,东向而坐。时魔王遣三玉女从北来试菩萨。魔王自从南来,菩萨以足指按地,魔兵却散,三女变为老姥,不自服。佛于尼拘律树下方石上东向坐,梵天来诣佛处,四天王捧钵处皆立塔。《外国事》曰:毗婆梨佛在此一树下六年,长者女以金钵盛乳糜上佛,佛得乳糜,住足尼连禅河浴。浴竟,于河边啖糜竟,掷钵水中,逆流百步,钵没河中,迪梨郊龙王接取在宫供养,先三佛钵亦见。佛于河傍坐摩诃菩提树,摩诃菩提树去贝多树二里,于此树下七日,思惟道成,魔兵试佛。释氏《西域记》曰:尼连水南注恒水,水西有佛树,佛于此苦行,日食糜六年。西去城五里许,树东河上,即佛入水浴处。东上岸尼拘律树下坐修,舍女上糜于此。于是西度水,于六年树南贝多树下坐,降魔得佛也。佛图调曰:佛树中枯,其来时更生枝叶。竺法维曰:六年树去佛树五里,书其异也。法显从此东南行,还已连弗邑,顺恒水西下,得一精舍,名旷野,佛所住处。复顺恒水西下,到迦尸国波罗奈城。竺法维曰:彼罗奈国在迪维罗卫国南千二百里,中间有恒水,东南流,佛转法轮处,在国北二十里,树名春浮,维摩所处也。法显曰:城之东北十里许,即鹿野苑,本辟支佛住此,常有野鹿栖宿,故以名焉。法显从此还,居巴连弗邑。又顺恒水东行,其南岸有瞻婆大国。释氏《西域记》曰:恒曲次东有瞻婆国城,南有卜佉兰池,恒水在北,佛下说戒处也。恒水又径波丽国。即是佛外祖国也。法显曰:恒水又东到多摩梨轩国,即是海口也。释氏《西域记》曰:大秦一名梨靬。康泰《扶南传》曰:从迦那调洲西南人大湾,可七八百里,乃到枝扈黎大江口,度江径西行,极大秦也。又云:发拘利口,入大湾中,正西北入,可一年余,得天竺江口,名恒水。江口有国,号担袟,属天竺。遣黄门字兴为担袟王。释氏《西域记》曰:恒水东流入东海。盖二水所注,两海所纳,自为东西也。释氏论佛图调列《山海经》曰:西海之南,流沙之滨,赤水之后,黑水之前,有大山,名昆仑。又曰:钟山西六百里有昆仑山,所出五水,祖以《佛图调传》也。又近推得康泰《扶南传》,《传》昆仑山正与调合。如《传》,自交州至天竺最近。泰《传》亦知阿褥达山是昆仑山。释云赖得调《传》,豁然为解,乃宣为《西域图》,以语法汰。法汰以常见怪,谓汉来诸名人,不应河在敦煌南数千里,而不知昆仑所在也。释云,复书曰案《穆天子传》。穆王于昆仑侧瑶池上觞西王母,云去宗周瀍涧,万有一千一百里,何得不如调言?子今见泰《传》,非为前人不知也。而今以后,乃知昆仑山为无热丘,何云乃胡国外乎?余考释氏之言,未为佳证。《穆天子》、《竹书》及《山海经》,皆埋缊岁久,编韦稀绝,书策落次,难以缉缀。后人假合,多差远意,至欲访地脉川,不与经符,验程准途,故自无会。释氏不复根其众归之鸿致,陈其细趣,以辨其非,非所安也。今案《山海经》曰:昆仑墟在西北,帝之下都,昆仑之墟,方八百里,高万仞,上有木禾,面有九井,以玉为槛,面有九门,门有开明兽守之,百神之所在。郭璞曰:此自别有小昆仑也。又案《淮南》之书,昆仑之上,有木禾、珠树、玉树、玻树,不死树在其西,沙棠、琅玕在其东,绛树在其南,碧树、瑶树在其北。旁有四百四十门,门间四里,里间九纯,纯丈五尺。旁有九井,玉横维其西北隅,北门开,以纳不周之风,倾宫、旋室、县圃、凉风、樊桐,在昆仑阊阖之中,是其疏圃,疏圃之池,浸之黄水,黄水三周复其源,是谓丹水,饮之不死。河水出其东北陬,赤水出其东南陬,洋水出其西北陬,凡此四水,帝之神泉,以和百药,以润万物。昆仑之丘或上倍之,是谓凉风之山,登之而不死。或上倍之,是谓玄圃之山,登之乃灵,能使风雨。或上倍之,乃维上天,登之乃神,是谓太帝之居。禹乃以息土填鸿水,以为名山,掘昆仑虚以为下地。高诱曰:地或作池。则以仿佛近佛图调之说。阿耨达六水,葱岭、于阗二水之限,与经史诸书,全相乖异。又案《十洲记》,昆仑山在西海之戌地,北海之亥地。去岸十三万里,有弱水,周匝绕山,东南接积石圃,西北接北户之室,东北临大阔之井,西南近承渊之谷。此四角大山,实昆仑之支辅也。积石圃南头,昔西王母告周穆王云,去咸阳四十六万里,山高平地三万六千里,上有三角,面方,广万里,形如偃盆,下狭上广。故曰昆仑山有三角。其一角正北,干辰星之辉,名曰阆风巅;其一角正西,名曰玄圃台;其一角正东,名曰昆仑宫。其处有积金,为天塘城,面方千里,城上安金台五所,玉楼十二。其北户山、承渊山又有墉城,金台玉楼,相似如一。渊精之阙,光碧之堂,琼华之室,紫翠丹房,景烛日晖,朱霞九光,西王母之所治,真官仙灵之所宗。上通旋机,元气流布,玉衡常理,顺九天而调阴阳,品物群生,希奇特出,皆在于此。天人济济,不可具记。其北海外,又有钟山,上有金台玉阂,亦元气之所含,天帝居治处也。考东方朔之言,及《经》五万里之文,难言佛图调、康泰之《传》是矣。六合之内,水泽之藏,大非为巨,小非为细,存非为有,隐非为无,其所苞者广矣。于中同名异域,称谓相乱,亦不为寡。至如东海方丈,亦有昆仑之称,西洲铜柱,又有九府之治。东方朔《十洲记》曰:方丈在东海中央,东西南北岸,相去正等。方丈面各五千里,上专是群龙所聚,有金玉琉璃之宫,三天司命所治处,群仙不欲升天者,皆往来也。张华叙东方朔《神异经》曰:昆仑有铜柱焉,其高入天,所谓天柱也。围三千里,圆周如削,下有回屋,仙人九府治。上有大鸟,名曰希有,南向,张左翼覆东王公,右翼覆西王母,背上小处无羽,万九千里,西王母岁登翼上,之东王公也。故其柱铭曰:昆仑铜柱,其高入天,圆周如削,肤体美焉。其鸟铭曰:有鸟希有,绿赤煌煌,不鸣不食,东覆东王公,西覆西王母,王母欲东,登之自通,阴阳相须,惟会益工。《遁甲开山图》曰:五龙见教,天皇被迹,望在无外,柱州昆仑山上。荣氏注云:五龙治在五方,为五行神。五龙降天皇兄弟十二人,分五方为十二部,法五龙之迹,行无为之化,天下仙圣治。在柱州昆仑山上,无外之山在昆仑东南万二千里,五龙、天皇皆出此中,为十二时神也。《山海经》曰:昆仑之丘,实惟帝之下都,其神陆吾,是司天之九部,及帝之囿时。然六合之内,其苞远矣,幽致冲妙,难本以情,万像遐渊,思绝根寻。自不登两龙于云辙,骋八骏于龟途,等轩辕之访百灵,方大禹之集会计。儒、墨之说,孰使辨哉!又出海外,南至积石山下,有石门。《山海经》曰:河水入渤海,又出海外,西北入禹所导积石山。山在陇西郡河关县西南羌中。余考群书,咸言河出昆仑,重源潜发,沦于蒲昌,出于海水。故《洛书》曰:河自昆仑,出于重野。谓此矣。径积石而为中国河,故成公子安《大河赋》曰:览百川之宏壮,莫尚美于黄河;潜昆仑之峻极,出积石之嵯峨。释氏《西域记》曰:河自蒲昌,潜行地下,南出积石。而《经》文在此,似如不比积石,宜在蒲昌海下矣。


\chapter{卷二 河水 }
又南入葱岭山,又从葱岭出而东北流。
河水重源有三,非惟二也。一源西出捐毒之国葱岭之上。西去休循二百余里,皆故塞种也。南属葱岭,高千里。《西河旧事》曰:葱岭在敦煌西八千里,其山高大,上生葱,故曰葱岭也。河源潜发其岭,分为二水,一水西径休循国南,在葱岭西。郭义恭《广志》曰:休循国居葱岭,其山多大葱。又径难兜国北,北接休循,西南去厨宾国三百四十里。河水又西径罽宾国北。月氏之破,塞王南君罽宾,治循鲜城。土地平和,无所不有,金银珍宝,异畜奇物,逾于中夏,大国也。山险,有大头痛、小头痛之山,赤土,身热之阪,人畜同然。河水又西径月氏国南。治监氏城,其俗与安息同。匈奴冒顿单于破月氏,杀其王,以头为饮器。国遂分,远过大宛,西居大夏,为大月氏;其余小众不能去者,共保南山、羌中,号小月氏,故有大月氏、小月氏之名也。又西径安息国南,城临妫水,地方数千里,最大国也。有商贾车船行旁国,画革旁行为书记也。河水与蜺罗跂噘水同注雷翥海。释氏《西域记》曰:蜺罗跂噘出阿耨达山之北,西径于阗国。《汉书·西域传》曰:于阗之西,水皆西流,注西海。又西径四大塔北。释法显所谓糺尸罗国,汉言截头也。佛为菩萨时,以头施人,故因名国。国东有投身饲饿虎处,皆起塔。又西径揵陀卫国北。是阿育王子法益所治邑。佛为菩萨时,亦于此国以眼施人,其处亦起大塔。又有弗楼沙国,天帝释变为牧牛小儿,聚土为佛塔,法王因而成大塔,所谓四大塔也。《法显传》曰:国有佛钵,月氏王大兴兵众,来伐此国,欲持钵去,置钵象上,象不能进;更作四轮车载钵,八象共牵,复不进。王知钵缘未至,于是起塔留钵供养。钵容二斗,杂色而黑多,四际分明,厚可二分,甚光泽。贫人以少花投中便满;富人以多花供养,正复百千万斛,终亦不满。佛图调曰:佛钵,青玉也,受三斗许,彼国宝之。供养时,愿终日香花不满,则如言;愿一把满,则亦便如言。又案道人竺法维所说,佛钵在大月支国,起浮图,高三十丈,七层,钵处第二层,金络络锁县钵,钵是青石。或云悬钵虚空。须菩提置钵在金机上,佛一足迹与钵共在一处,国王、臣民,悉持梵香、七宝、壁玉供养。塔迹、佛牙、袈裟、顶相舍利,悉在弗楼沙国。释氏《西域记》曰:揵陀越王城西北有钵吐罗越城,佛袈裟王城也,东有寺。重复寻川水,西北十里有河步罗龙渊。佛到渊上浣衣处,浣石尚存。其水至安息,注雷翥海。又曰:握陀越西,西海中有安息国。竺枝《扶南记》曰:安息国去私诃条国二万里,国土临海上,即《汉书》天竺安息国也。户近百万,最大国也。《汉书·西域传》又云:梨靬、条支临西海。长老传闻,条支有弱水,西王母亦未尝见。自条支乘水西行,可百余日,近日所入也。或河水所通西海矣。故《凉土异物志》曰:葱岭之水,分流东西,西入大海,东为河源,《禹记》所云昆仑者焉。张骞使大宛而穷河源,谓极于此,而不达于昆仑也。河水自葱岭分源,东径边舍罗国。释氏《西域记》曰:有国名伽舍罗逝。此国狭小,而总万国之要道无不由。城南有水,东北流,出罗逝西山。山即葱岭也。径岐沙谷,出谷分为二水。一水东流,径无雷国北。治卢城,其俗与西夜、子合同。又东流径依耐国北。去无雷五百四十里,俗同子合。河水又东径蒲犁国北。治蒲犁谷,北去疏勒五百五十里,俗与子合同。河水又东径皮山国北。治皮山城,西北去莎车三百八十里。其一源出于阗国南山,北流与葱岭所出河合,又东注蒲昌海。河水又东与于阗河合。南源导于阗南山,俗谓之仇摩置,自置北流,径于阗国西。治西城,土多玉石,西去皮山三百八十里,东去阳关五千余里。释法显自乌帝西南行,路中无人民,沙行艰难,所径之苦,人理莫比。在道一月五日,得达于阗。其国殷庶,民笃信,多大乘学,威仪齐整,器钵无声。城南十五里有利刹寺,中有石靴,石上有足迹,彼俗言是辟支佛迹。法显所不传,疑非佛迹也。又西北流,注于河,即《经》所谓北注葱岭河也。南河又东径于阗国北,释氏《西域记》曰:河水东流三千里,至于阗,屈东北流者也。《汉书·西域传》曰:于阗已东,水皆东流。南河又东北径杆弥国北。治扞弥城,西去于阗三百九十里。南河又东径精绝国北。西去阗弥四百六十里。南河又东径且末国北,又东,右会阿耨达大水。释氏《西域记》曰:阿耨达山西北有大水,北流注牢兰海者也。其水北流径且末南山,又北径旦未城西。国治且末城,西通精绝二千里,东去鄯善七百二十里,种五谷,其俗咯与汉同。又曰:且末河东北流径且未北,又流而左会南河,会流东逝,通为注滨河。注滨河又东径鄯善国北。治伊循城,故楼兰之地也。楼兰王不恭于汉,元凤四年,霍光遣平乐监傅介子刺杀之,更立后王。汉又立其前王质子尉屠耆为王,更名其国为鄯善。百官祖道横门,王自请天子曰:身在汉久,恐为前王子所害,国有伊循城,土地肥美,愿遣将屯田积粟,令得依威重。遂置田以镇抚之。敦煌索励,字彦义,有才略。刺史毛奕表行贰师将军,将酒泉、敦煌兵千人,至楼兰屯田。起白屋,召鄯善、焉耆、龟兹三国兵各千,横断注滨河,河断之日,水奋势激,波陵冒堤。劢厉声曰:王尊建节,河堤不溢,王霸精诚,呼沱不流,水德神明,古今一也。劢躬祷祀,水犹未减,乃列阵被杖,鼓噪欢叫。且刺且射,大战三日,水乃回减,灌浸沃衍,胡人称神。大田三年,积粟百万,威服外国。其水东注泽,泽在楼兰国北扞泥城。其俗谓之东故城,去阳关于六百里,西北去乌垒千七百八十五里,至墨山国千八百六十五里,西北去车师干八百九十里。土地沙卤少田,仰谷旁国。国出玉,多蓖苇、柽柳、胡桐、白草。国在东垂,当白龙堆,乏水草,常主发导,负水担粮,迎送汉使。故彼俗谓是泽为牢兰海也。释氏《西域记》曰:南河自于阗东于北三千里,至鄯善入牢兰海者也。北河自岐沙东分南河,即释氏《西域记》所谓二支北流,径屈茨、乌夷、禅善,入牢兰海者也。北河又东北流,分为二水,枝流出焉,北河自疏勒径流南河之北。《汉书·西域传》曰:葱岭以东,南北有山,相距千余里,东西六千里,河出其中。暨于温宿之南,左合枝水,枝水上承北河于疏勒之东;西北流径疏勒国南,又东北与疏勒北山水合;水出北溪,东南流径疏勒城下。南去莎车五百六十里,有市列,西当大月氏、大宛、康居道。释氏《西域记》曰:国有佛浴床,赤真檀木作之,方四尺,王于宫中供养。汉永平十八年,耿恭以戊己校尉,为匈奴左鹿蠡王所逼,恭以此城侧涧傍水,自金蒲迁居此城,匈奴又来攻之,壅绝涧水。恭于城中穿井,深一十五丈,不得水,吏士渴乏,笮马粪汁饮之。恭乃仰天叹曰:昔贰师拔佩刀刺山,飞泉涌出,今汉德神明,岂有穷哉?整衣服,向井再拜,为吏士祷之。有顷,水泉奔出,众称万岁。乃扬水以示之,虏以为神,遂即引去。后车师叛,与匈奴攻恭,食尽穷困,乃煮铠弩,食其筋革。恭与士卒同生死,咸无二心。围恭不能下,关宠上书求救,建初元年,章帝纳司徒鲍昱之言,遣兵救之。至柳中,以校尉关宠分兵入高昌壁,攻交河城,车师降,遣恭军吏范羌将兵二千人迎恭。遇大雪丈余,仅能至,城中夜闻兵马,大恐,羌遥呼曰:我范羌也。城中皆称万岁,开门相持涕泣。尚有二十六人,衣屦穿决,形容枯槁,相依而还,枝河又东径莎车国南。治莎车城,西南去蒲犁七百四十里。汉武帝开西域,屯田于此。有铁山,出青玉。枝河又东径温宿国南。治温宿城,土地物类,与鄯善同。北至乌孙赤谷六百一十里,东通姑墨二百七十里。于此,枝河右入北河。北河又东径姑墨国南,姑墨川水注之,水导姑墨西北,历赤沙山,东南流径姑墨国西。治南城,南至于阗,马行十五日,土出铜铁及雌黄。其水又东南流,右注北河。北河又东径龟兹国南,又东,左合龟兹川水,有二源,西源出北大山南。释氏《西域记》曰:屈茨北二百里有山,夜则火光,昼日但烟,人取此山石炭,冶此山铁,恒充三十六国用。故郭义恭《广志》云:龟兹能铸冶。其水南流径赤沙山。释氏《西域记》曰:国北四十里,山上有寺,名雀高大清净。又出山东南流,枝水左派焉。又东南,水流三分,右二水俱东南流,注北河。东川水出龟兹东北,历赤沙、积梨南流,枝水右出,西南入龟兹城。音屈茨也,故延城矣,西去姑墨六百七十里。川水又东南流径于轮台之东也。昔汉武帝初通西域,置校尉,屯田于此。搜粟都尉桑弘羊奏言:故轮台以东,地广,饶水草,可溉田五千顷以上,其处温和,田美,可益通沟渠,种五谷,收获与中国同。时匈奴弱,不敢近西域,于是徙莎车,相去千余里,即是台也。其水又东南流,右会西川枝水,水有二源,俱受西川,东流径龟兹城南,合为一水。水间有故城,盖屯校所守也。其水东南注东川,东川水又东南径乌垒国南。治乌垒城,西去龟兹三百五十里,东去玉门阳关二千七百三十八里,与渠犁田官相近,土地肥饶,于西域为中故都护治焉。汉使持节郑吉,并护北道,故号都护,都护之起,自吉置也。其水又东南注大河。大河又东,右会敦薨之水,其水出焉耆之北,敦薨之山,在匈奴之西,乌孙之东。《山海经》曰:敦薨之山,敦亮之水出焉,而西流注于泑泽。出于昆仑之东北隅,实惟河源者也。二源俱道,西源东流,分为二水,左水西南流,出于焉耆之西,径流焉耆之野,屈而东南流,注于敦薨之渚。右水东南流,又分为二,左右焉耆之国。城居四水之中,在河水之洲,治员渠城,西去乌垒四百里。南会两水,同注敦薨之浦。东源东南流,分为二水,涧澜双引,洪湍濬发,俱东南流,径出焉耆之东,导于危须国西。国治危须城,西去焉耆百里。又东南流,注于敦薨之薮。川流所积,潭水斯涨,溢而为海。《史记》曰:焉耆近海多鱼鸟;东北隔大山与车师接。敦薨之水自西海径尉犁国。国治尉犁城,西去都护治所三百里,北去焉耆百里。其水又西出沙山铁关谷,又西南流,径连城别注,裂以为田。桑弘羊曰:臣愚以为连城以西,可遣屯田,以威西国,即此处也。其水又屈而南,径渠犁国西。故《史记》曰:西有大河,即斯水也。又东南流,径渠犁国。治渠犁城,西北去乌垒三百三十里。汉武帝通西域,屯渠犁,即此处也。南与精绝接,东北与尉犁接。又南流注于河。《山海经》曰:敦薨之水,西流注于泑泽。盖乱河流自西南注也。河水又东径墨山国南。治墨山城,西至尉犁二百四十里。河水又东径注宾城南。又东径楼兰城南而东注。盖坺田士所屯,故城禅国名耳。河水又东注于泑泽,即《经》所谓蒲昌海也。水积鄯善之东北,龙城之西南。龙城,故姜赖之虚,胡之大国也。蒲昌海溢,荡覆其国,城基尚存而至大。晨发西门,暮达东门。法其崖岸,余溜风吹,稍成龙形,西面向海,因名龙城。地广千里,皆为盐而刚坚也。行人所径,畜产皆布毡卧之,掘发其下,有大盐,方如巨枕,以次相累,类雾起云浮,寡见星日,少禽,多鬼怪。西接鄯善,东连三沙,为海之北隘矣。故蒲昌亦有盐泽之称也。《山海经》曰:不周之山,北望诸毗之山,临彼岳崇之山,东望泑泽,河水之所潜也,其源浑浑泡泡者也。东去玉门阳关千三百里,广轮四百里。其水澄渟,冬夏不减,其中洄湍电转,为隐沦之脉。当其澴流之上,飞禽奋翮于霄中者,无不坠于渊波矣。即河水之所潜,而出于积石也。
又东入塞,过敦煌、酒泉、张掖郡南,河自蒲昌,有隐沦之证,并间关入塞之始。自此,《经》当求实致也。
河水重源,又发于西塞之外,出于积石之山。《山海经》曰:积石之山,其下有石门,河永冒以西流,是山也,万物无不有。《禹贡》所谓导河自积石也。山在西羌之中,烧当所居也。延熹二年,西羌烧当犯塞,护羌校尉段熲讨之,追出塞,至积石山,斩首而还。司马彪曰:西羌者,自析支以西,滨于河首左右居也。河水屈而东北流,径析支之地,是为河曲矣。应劭曰:《禹贡》,析支属雍州,在河关之西,东去河关于余里,羌人所居,谓之河曲羌也。东北历敦煌、酒泉、张掖南。应劭《地理风俗记》曰:敦煌,酒泉,其水甘若酒味故也;张掖,言张国臂掖,以威羌狄。《说文》曰:郡制,天子地方千里,分为百县,县有四郡。故《春秋传》曰:上大夫县,下大夫郡。至秦,始置三十六郡,以监县矣。从邑,君声。《释名》曰:郡,群也,人所群聚也。黄义仲《十二州记》曰:郡之言君也,改公侯之封而言,君者,至尊也。郡守专权,君臣之礼弥崇。今郡字,君在其左,邑在其右,君为元首,邑以载民,故取名于君,谓之郡。《汉官》曰:秦用李斯议,分天下为三十六郡。凡郡,或以列国,陈、鲁、齐、吴是也;或以旧邑,长沙、丹阳是也;或以山陵,太山、山阳是也;或以川原,西河、河东是也;或以所出,金城城下得金,酒泉泉味如酒,豫章樟树生庭,雁门雁之所育是也;或以号令,禹合诸侯,大计东冶之山,因名会稽是也。河径其南而缠络远矣。河水自河曲,又东径西海郡南。汉平帝时,王莽秉政,欲耀威德,以服远方,讽羌献西海之地,置西海郡,而筑五县焉,周海亭隧相望。莽篡政纷乱,郡亦弃废。河水又东径允川,而历大榆、小榆谷北。羌迷唐、钟存所居也。永元五年,贯友代聂尚为护羌校尉,攻迷唐,斩获八百余级,收其熟麦数万斛,于逢留河上筑城以盛麦,且作大船,于河峡作桥渡兵,迷唐遂远依河曲。永元九年,迷唐复与钟存东寇而还。十年,谒者王信、耿谭,西击迷唐,降之,诏听还大、小榆谷。迷唐谓汉造河桥,兵来无时,故地不可居,复叛,居河曲,与羌为仇,种人与官兵击之。允川去迷唐数十里,营止,遣轻兵挑战,因引还,迷唐追之,至营因战,迷唐败走。于是西海及大、小榆谷,无复聚落。隃糜相曹凤上言:建武以来,西戎数犯法,常从烧当种起。所以然者,以其居大、小榆谷,土地肥美,又近塞内,与诸种相傍,南得钟存,以广其众,北阻大河,因以为固,又有西海鱼盐之利,缘山滨河,以广田畜,故能强大,常雄诸种。今党援沮坏,亲属离叛,其余胜兵,不过数百,宜及此时,建复西海郡县,规固二榆,广设屯田,隔塞羌胡交关之路,殖谷富边,省输转之役。上拜凤为金城西部都尉,遂开屯田二十七部,列屯夹河,与建成相首尾,后羌反,遂罢。案段国《沙州记》,吐谷浑于河上作桥,谓之河厉,长百五十步,两岸累石作基陛,节节相次,大木从横更镇压,两边俱平,相去三丈,并大材以板横次之,施钩栏甚严饰。桥在清水川东也。
又东过陇西河关县北,洮水从东南来流注之。
河水右径沙州北。段国曰:浇河西南百七十里有黄沙,沙南北百二十里,东西七十里,西极大杨川。望黄沙,犹若人委于糒于地,都不生草木,荡然黄沙,周回数百里,沙州于是取号焉。《地理志》曰:汉宣帝神爵二年,置河关县,盖取河之关塞也。《风俗通》曰:百里曰同,总名为县。县,玄也,首也,从系倒首,举首易偏矣。言当玄静,平谣役也。《释名》又曰:县,悬也,悬于郡矣。黄义仲《十三州记》曰:县,弦也,弦以贞直,言下体之居,邻民之位,不轻其誓,施绳用法,不曲如弦,弦声近县,故以取名,今系字在半也。汉高帝六年,令天下县邑城。张晏曰:令各自筑其城也。河水又东北流,入西平郡界,左合二川,南流入河。又东北,济川水注之,水西南出滥读,东北流入大谷,谓之大谷水。北径浇河城西南,北流注于河。河水又东径浇河故城北,有二城东西角倚,东北去西平二百二十里。宋少帝景平中,拜吐谷浑阿豺为安西将军、浇河公,即此城也。河水又东北径黄川城,河水又东径石城南,左台北谷水。昔段熲击羌于石城,投河坠坑而死者八百余人,即于此也。河水又东北径黄河城南,西北去西平二百一十七里。河水又东北径广违城北,右合乌头川水,水发远川,引纳支津,北径城东而北流,注于河。河水又东径邯川城南。城之左右,历谷有二水,导自北山,南径邯亭,注于河。河水又东,临津溪水注之,水自南山,北径临津城西而北流,注于河。河水又东径临津城北、白土城南。《十三州志》曰:左南津西六十里有白土城,城在大河之北,而为缘河济渡之处。魏凉州刺史郭淮破羌,遮塞于白土,即此处矣。河水又东,左会白土川水,水出白土城西北下,东南流径白土城北,又东南注于河。河水又东北会两川,右合二水,参差夹岸连壤,负险相望。河北有层山,山甚灵秀,山峰之上,立石数百丈,亭亭桀竖,竞势争高,远望,若攒图之托霄上。其下层岩峭举,壁岸无阶,悬岩之中,多石室焉。室中若有积卷矣,而世士罕有津达者,因谓之积书岩。岩堂之内,每时见神人往还矣,盖鸿衣羽裳之士,练精饵食之夫耳。俗人不悟其仙者,乃谓之神鬼。彼羌目鬼曰唐述,复因名之为唐述山。指其堂密之居,谓之唐述窟,其怀道宗玄之士,皮冠净发之徒,亦往栖托焉。故《秦川记》曰:河峡崖傍有二窟,一曰唐述窟,高四十丈;西二里有时亮窟,高百丈,广二十丈,深三十丈,藏古书五笥。亮,南安人也。下封有水,导自是山溪水,南注河,谓之唐述水。河水又东得野亭南,又东北流,历研川,谓之研川水。又东北注于河,谓之野亭口。河水又东历凤林北。凤林,山名也,五峦俱峙。耆彦云:昔有凤乌,飞游五峰,故山有斯目矣。《秦州记》曰:枹罕原北名凤林川,川中则黄河东流也。河水又东与漓水合,水导源塞外羌中,故《地理志》曰:其水出西塞外,东北流,历野虏中,径消铜城西,又东北径列城东。考《地说》无目,盖出自戎方矣。左合列水,水出西北溪,东北流径列城北,右入漓水,城居二水之会也。漓水又北径可石孤城西,西戎之名也。又东北,右合黑城溪水。水出西北山下,东南流径黑城南,又东南,枝水左出焉。又东南入漓水。漓水又东北径榆城东,榆城溪水注之。水出素和细越西北山下,东南流径细越川,夷俗乡名也。又东南出狄周峡,东南右合黑城溪之枝津,津水上承溪水,东北径黑城东,东北注之榆溪,又东南径榆城南,东北注漓水。漓水又东北径石门口,山高险峻绝,对岸若门,故峡得厥名矣。疑即皋兰山门也。汉武帝元狩三年,骠骑霍去病出陇西,至皋兰,谓是山之关塞也。应劭《汉书音义》曰:皋兰在陇西白石县塞外,河名也。孟康曰:山关名也。今是山去河不远,故论者疑目河山之间矣。漓水又东北,皋兰山水自山左右翼注漓水。漓水又东,白石川水注之,水出县西北山下,东南流,枝津东注焉。白石川水又南径白石城西而注漓水。漓水又东径白石县故城南,王莽更曰顺砾。阚駰曰:白石县在狄道西北二百八十五里,漓水径其北。今漓水径其南,而不出其北也。漓水又东径白石山北。应劭曰:白石山在东。罗溪水注之。水出西南山下。东入漓水。漓水又东,左合罕南溪水。水出罕西,东南流径罕南注之。《十二州志》曰:广大贩在枹罕西北,罕在焉。昔慕容吐谷浑自燕历阴山西驰,而创居于此。漓水又东径枹罕县故城南。应劭曰:故枹罕侯邑也。《十三州志》曰:枹罕县在郡西二百一十里。漓水在城南门前东过也。漓水又东北,故城川水注之,水有二源,南源出西南山下,东北流径金纽大岭北,又东北径一故城南,又东北与北水会。北源自西南径故城北,右入南水。乱流东北注漓水。漓水又东北,左合白石川之枝津,水上承白石川,东径白石城北,又东绝罕溪,又东径枹罕城南,又东入漓水,漓水又东北出峡,北流注于河。《地理志》曰:漓水出白石县西塞外,东至枹罕入河。河水又径左南城南。《十二州志》曰:石城西一百四十里有左南城者也,津亦取名焉。大河又东径赤岸北,即河夹岸也。《秦州记》曰:枹罕有河夹岸,岸广四十丈,义熙中,乞佛于此河上作飞桥,桥高五十丈,三年乃就。河水又东,洮水注之。《地理志》曰:水出塞外羌中。《沙州记》曰:洮水与垫江水俱出嵹台山,山南即垫江源,山东则桃水源。《山海经》曰:白水出蜀。郭景纯注云:从临桃之西倾山东南流入汉,而至垫江,故段国以为垫江水也。洮水同出一山,故知嵹台,西倾之异名也。洮水东北流,径吐谷浑中。吐谷浑者,始是东燕慕容之枝庶,因氏其字,以为首类之种号也,故谓之野虏。自洮嵹南北三百里中,地草遍是龙须,而无樵柴。洮水又东北流径洮阳曾城北。《沙州记》曰:嵹城东北三百里有曾城,城临洮水者也。建初二年,羌攻南部都尉于临洮,上遣行车骑将军马防与长水校尉耿恭救之,诸羌退聚洮阳,即此城也。洮水又东径洪和山南,城在四山中。洮水又东径迷和城北,羌名也。又东径甘枳亭,历望曲。在临洮西南,去龙桑城二百里。洮水又东径临跳县故城北。禹治洪水,西至洮水之上,见长人,受黑玉书于斯水上。洮水又东北流,屈而径索西城西。建初二年,马防、耿恭从五溪祥谷出索西,与羌战,破之,筑索西城,徙陇西南部都尉居之,俗名赤水城,亦曰临洮东城也。《沙州记》曰:从东洮至西洮百二十里者也。洮水又屈而北,径龙桑城西而西北流。马防以建初二年,从安故五溪出龙桑,开通旧路者也,俗名龙城。洮水又西北径步和亭东,步和川水注之。水出西山下,东北流出山,径步和亭北,东北注洮水。洮水又北出门峡,历求厥川,蕈川水注之。水出桑岚西溪,东流历桑岚川,又东径蕈川北,东入洮水。洮水又北历峡,径偏桥,出夷始梁,右合蕈垲川水。水东南出石底横下,北历蕈垲川,西北注洮水。洮水又东北径桑城东,又北会蓝川水。水源出求厥川西北溪,东北流径蓝川,历桑城北,东入洮水。洮水又北径外羌城西,又北径和博城东,城在山内,左合和博川水。水出城西南山下,东北径和博城南,东北注于洮水。洮水北径安故县故城西。《地理志》,陇西之属县也。《十三州志》曰:县在郡南四十七里,盖延转击狄道安故五溪反羌,大破之,即此也。洮水又北径狄道故城西。阚駰曰:今曰武始也。洮水在城西北流。又北,陇水注之,即《山海经》所谓滥水也。水出鸟鼠山西北高城岭,西径陇坻。其山岸崩落者,声闻数百里,故扬雄称响若坻颓是也。又西北历白石山下。《地理志》曰:狄道东有白石山。滥水又西北径武街城南,又西北径狄道故城东。《百官表》曰:县有蛮夷谓之道,公主所食曰邑。应劭曰:反舌左衽,不与华同,须有译言,乃通也。汉陇西郡治,秦昭王二十八年置。应劭曰:有陇堆在其东,故曰陇西也。《神仙传》曰:封君达,陇西人,服炼水银,年百岁,视之如年三十许,骑青牛,故号青牛道士。王莽更郡县之名,郡曰厌戎,县曰操虏也。昔马援为陇西太守,六年,为狄道开渠,引水种秔稻,而郡中乐业,即此水也。滥水又西北流,注于洮水。洮水右合二水,左会大夏川水。水出西山,二源合舍而乱流,径金纽城南。《十三州志》曰:大夏县西有故金纽城,去县四十里,本都尉治。又东北径大夏县故城南。《地理志》王莽之顺夏。《晋书地道记》曰:县有禹庙,禹所出也。又东北出山,注于洮水。洮水又北,翼带三水,乱流北入河。《地理志》曰:洮水北至枹罕,东入河是也。又东过金城允吾县北,金城,郡治也。汉昭帝始元六年置,王莽之西海也。莽又更允吾为修远县。河水径其南,不在其北,南有湟水出塞外,东径西王母石室、石釜、西海盐池北,故阚駰口:其西即湟水之源也。《地理志》曰:湟水所出。湟水又东南流径龙夷城,故西零之地也,《十三州志》曰:城在临羌新县西三百一十里。王莽纳西零之献,以为西海郡,治此城。湟水又东南径卑禾羌海北,有盐池。阚駰曰:县西有卑禾羌海者也,世谓之青海,东去西平二百五十里。湟水东流径湟中城北,故小月氏之地也。《十二州志》曰:西平、张掖之间,大月氏之别,小月氏之国。范晔《后汉书》曰:湟中月氏胡者,其王为匈奴所杀,余种分散,西逾葱岭,其弱者南入山,从羌居止,故受小月氏之名也。《后汉·西羌传》曰:羌无弋爱剑者,秦厉公时,以奴隶亡入三河,羌怪为神,推以为豪。河、湟之间多禽兽,以射猎为事,遂见敬信,依者甚众。其曾孙忍,因留湟中,为湟中羌也。湟水又东,右控四水,导源四溪,东北流注于湟。湟水又东径赤城北,而东入经戎峡口,右合羌水。水出西南山下,径护羌城东,故护羌校尉治。又东北径临羌城西,东北流,注于湟。湟水又东径临羌县故城北。汉武帝元封元年,以封孙都为侯国,王莽之监羌也。谓之绥戎城,非也。湟水又东,卢溪水注之。水出西南卢川,东北流,注于湟水。湟水又东径临羌新县故城南。阚駰曰:临羌新县在郡西百八十里,湟水径城南也。城有东、西门,西北隅有子城。湟水又东,右合溜溪、伏溜、石杜、蠡四川,东北流注之。左会临羌溪水。水发新县西北,东南流,历县北,东南入湟水。湟水又东,龙驹川水注之。水右出西南山下,东北流径龙驹城,北流注于湟水。湟水又东,长宁川水注之。水出松山,东南流径晋昌城,晋昌川水注之。长宁水又东南,养女川水注之。水发养女北山,有二源,皆长湍远发,南总一川,径养女山,谓之养女川。阚駰曰:长宁亭北有养女岭,即浩亹山,西平之北山也。乱流出峡,南径长宁亭东。城有东、西门,东北隅有金城,在西平西北四十里。《十三州志》曰六十里,远矣。长宁水又东南与一水合,水出西山,东南流。水南山上,有风伯祠,春秋祭之。其水东南径长宁亭南,东入长宁水。长宁水又东南流,注于湟水。湟水又东,牛心川水注之,水出西南远山,东北流,径牛心堆东,又北径西平亭西,东北入湟水。湟水又东径西平城北。东城,即故亭也。汉景帝六年,封陇西太守北地公孙浑邪为侯国。魏黄初中,立西平郡,凭倚故亭,增筑南、西、北三城以为郡治。湟水又东径土楼南。楼北依山原,峰高三百尺,有若削成。楼下有神祠,雕墙故壁存焉。阚駰曰:西平亭北有土楼神祠者也。今在亭东北五里。右则五泉注之,泉发西平亭北,雁次相缀,东北流至土楼南,北入湟水。湟水又东,右合葱谷水。水有四源,各出一溪,乱流注于湟。湟水又东径东亭北,东出漆峡,山峡也。东流,右则漆谷常溪注之,左则甘夷川水入焉。湟水又东,安夷川水注之。水发远山,西北流,控引众川,北屈径安夷城西北,东入湟水。湟水又东径安夷县故城。城有东、西门,在西平亭东七十里。阚駰曰四十里。湟水又东,左合宜春水。水出东北宜春溪,西南流至安夷城南,入湟水。湟水又东,勒且溪水注之。水出县东南勒且溪,北流径安夷城东,而北入湟水。湟水有勒且之名,疑即此号也。阚駰曰:金城河初与浩亹河合,又与勒且河合者也。湟水又东,左则承流谷水南入,右会达扶东、西二溪水,参差北注,乱流东出,期顿、鸡谷二水北流注之。又东,吐那孤、长门两川,南流入湟水。六山,名也。湟水又东径乐都城南,东流,右合来谷、乞斤二水,左会阳非、流溪、细谷三水,东径破羌县故城南。应劭曰:汉宣帝神爵二年置,城省南门。《十二州志》曰:湟水河在南门前东过。六谷水自南,破羌川自北,左右翼注。湟水又东南径小晋兴城北,故都尉治。阚駰曰:允吾县西四十里有小晋兴城。湟水又东与阁门河合,即浩亹河也。出西塞外,东入塞,径敦煌、酒泉、张掖南,东南径西平之鲜谷塞尉故城南,又东南与湛水合。水有二源,西水出白岭下,东源发于白岸谷,合为一川。东南流至雾山,注阁门河。阁门河又东径养女北山东南,左合南流川水,水出北山,南流入于阁门河。阁门河又东径浩亹县故城南。王莽改曰兴武矣。阚駰曰:浩,读阁也。故亦曰阁门水,两兼其称矣,又东流注于湟水。故《地理志》曰,浩亹水东至允吾入湟水。湟水又东径允吾县北为郑伯津,与涧水合,水出令居县西北塞外,南流径其县故城西。汉武帝元鼎二年置,王莽之罕虏也。又南径永登亭西,历黑石谷南流,注郑伯津。湟水又东径允街县故城南。汉宣帝神爵二年置,王莽之修远亭也。县有龙泉,出允街谷,泉眼之中,水文成交龙,或试挠破之,寻平成龙。畜生将饮者,皆畏避而走,谓之龙泉,下入湟水。湟水又东径枝阳县,逆水注之。水出允吾县之参街谷,东南流径街亭城南,又东南径阳非亭北,又东南径广武城西,故广武都尉治。郭淮破叛羌,治无戴,于此处也。城之西南二十许里,水西有马蹄谷。汉武帝闻大宛有天马,遣李广利伐之,始得此马,有角为奇。故汉武帝《天马之歌》曰:天马来兮历无草,径千里兮循东道。胡马感北风之思,遂顿羁绝绊,骤首而驰,晨发京城,夕至敦煌北塞外,长鸣而去,因名其处曰候马亭。今晋昌郡南及广武马蹄谷盘石上,马迹若践泥中,有自然之形,故其俗号曰天马径,夷人在边效刻,是有大小之迹,体状不同,视之便别。逆水又东径枝阳县故城南,东南入于湟水。《地理志》曰:逆水出允吾东,至枝阳入湟。湟水又东流,注于金城河,即积石之黄河也。阚駰曰:河至金城县,谓之金城河,随地为名也。释氏《西域记》曰:牢兰海东伏流龙沙堆,在屯皇东南四百里阿步于,鲜卑山。东流至金城为大河。河出昆仑,昆仑即阿耨达山也。河水又东径石城南,谓之石城津。阚駰曰:在金城西北矣。河水又东南径金城县故城北。应劭曰:初筑城得金,故曰金城也。《汉书集注》,薛瓒云:金者,取其坚固也,故墨子有金城汤池之言矣。王莽之金屏也。《世本》曰:鲧作城。《风俗通》曰:城,盛也,从土,成声。《管子》曰:内为之城,城外为之郭,郭外为之土阆。地高则沟之,下则堤之,命之曰金城。《十三州志》曰:大河在金城北门。东流,有梁泉注之,出县之南山。案耆旧言:梁晖,字始娥,汉大将军梁冀后。冀诛,入羌。后其祖父为羌所推,为渠帅而居此城。土荒民乱,晖将移居枹罕,出顿此山,为群羌围迫,无水,晖以所执榆鞭竖地,以青羊祈山,神泉涌出,榆木成林。其水自县北流注于河也。又东过榆中县北,昔蒙恬为秦北逐戎人,开榆中之地。案《地理志》,金城郡之属县也,故徐广《史记音义》曰:榆中在金城,即阮嗣宗《劝进文》所谓榆中以南者也。
又东过天水北界,苑川水出勇士县之子城南山,东北流,历此成川,世谓之子城川。又北径牧师苑,故汉牧苑之地也。羌豪迷吾等万余人,到襄武、首阳,平襄、勇士,抄此苑马,焚烧亭驿,即此处也。又曰:苑川水地,为龙马之沃土,故马援请与田户中分以自给也。有东、西二苑城,相去七十里。西城,即乞佛所都也。又北入于河也。
又东北过武威姐围县南,河水径其界东北流,县西南有泉源,东径其县南,又东北入河也。
又东北过天水勇士县北,《地理志》曰:满福也,属国都尉治,王莽更名之曰纪德。有水出县西,世谓之二十八渡水。东北流,溪涧萦曲,途出其中,径二十八渡,行者勤于溯涉,故因名焉。北径其县而下注河。又有赤晔川水,南出赤蒿谷,北流径赤晔川,又北径牛官川。又北径义城西北,北流历三城川,而北流注于河也。又东北过安定北界麦田山。
河水东北流径安定祖厉县故城西北。汉武帝元鼎三年,幸雍,遂逾陇登空同,西临祖厉河而还,即于此也。王莽更名之曰乡礼也。李斐曰:音赖。又东北,祖厉川水注之。水出祖厉南山,北流径祖厉县而西北流,注于河。河水又东北径麦田城西,又北与麦田泉水合,水出城西北,西南流注于河。河水又东北径麦田山西谷,山在安定西北六百四十里。河水又东北径于黑城北,又东北,高平川水注之,即苦水也。水出高平大陇山苦水谷。建武八年,世祖征隗嚣,吴汉从高平第一城苦水谷人,即是谷也。东北流径高平县故城东。汉武帝元鼎三年置,安定郡治也。王莽更名其县曰铺睦。西十里有独阜,阜上有故台,台侧有风伯坛,故世俗呼此阜为风堆。其水又北,龙泉水注之,水出县东北七里龙泉。东北流,注高平川。川水又北出秦长城,城在县北一十五里。又西北流径东、西二土楼故城门,北合一水。水有五源,咸出陇山西。东水发源县西南二十六里湫渊,渊在四山中。湫水北流,西北出长城北,与次水会,水出县西南四十里长城西山中,北流径魏行宫故殿东。又北,次水注之。出县西南四十里山中,北流径行宫故殿西。又北合次水,水出县西南四十八里,东北流,又与次水合。水出县西南六十里酸阳山,东北流,左会右水,总为一川。东径西楼北,东注苦水。段颎为护羌校尉,于安定高平苦水讨羌零,斩首八千级于是水之上。苦水又北与石门水合。水有五源,东水导源高平县西八十里,西北流,次水注之。水出县西百二十里如州泉,东北流,右入东水,乱流左会三川,参差相得,东北同为一川,混涛历峡,峡,即陇山之北垂也,谓之石门口,水曰石门水,在县西北八十余里。石门之水又东北注高平川。川水又北,自延水注之。水西出自延溪,东流历峡,谓之自延口,在县西北百里。又东北径延城南,东入高平川。川水又北径廉城东。按《地理志》,北地有廉县。阚駰言,在富平北。自昔匈奴侵汉,新秦之土,率为狄场,故城旧壁,尽从胡目。地理沦移,不可复识,当是世人误证也。川水又北,苦水注之。水发县东北百里山,流注高平川。川水又北,径三水县西,肥水注之。水出高平县西北二百里牵条山西,东北流,与若勃溪合。水有二源,总归一读,东北流入肥。肥水又东北流,违泉水注焉。泉流所发,导于若勃溪东,东北流入肥。肥水又东北出峡,注于高平川,水东有山,山东有三水县故城。本属国都尉治,王莽之广延亭也,西南去安定郡三百四十里。议郎张矣为安定属国都尉,治此。羌有献金马者,矣召主簿张祁入于羌前,以酒酹地曰:使马如羊,不以入厩;使金入粟,不以入怀。尽还不受,威化大行。县东有温泉,温泉东有盐池。故《地理志》曰县有盐官。今于城之东北有故城;城北有三泉,疑即县之盐官也。高平川水又北入于河。河水又东北径胸卷县故城西。《地理志》曰河水别出为河沟,东至富平,北入河。河水于此有上河之名也。


\chapter{卷三 河水 }
北过北地富平县西,河侧有两山相对,水出其间,即上河峡也,世谓之为青山峡。河水历峡北注,枝分东出。河水又北径富平县故城西。秦置北部都尉,治县城,王莽名郡为威戎,县曰持武。建武中,曹凤字仲理,为北地太守,政化尤异,黄龙应于九里谷高冈亭,角长三尺,大十围,梢至十余丈。天子嘉之,赐帛百匹,加秩中二千石。河水又北,薄骨律镇城。在河诸上,赫连果城也。桑果余林,仍列洲上。但语出戎方,不究城名。访诸耆旧,咸言故老宿彦云,赫连之世,有骏马死此,取马色以为邑号,故目城为白口骡韵之谬,遂仍今称,所未详也。河水又径典农城东,世谓之胡城。又北径上河城东,世谓之汉城。薛瓒曰:上河在西河富平县,即此也,冯参为上河典农都尉所治也。河水又北径典农城东,俗名之为吕城,皆参所屯,以事农旷。河水又东北径廉县故城东。王莽之西河亭。《地理志》曰:卑移山在西北。河水又北与枝津合。水受大河,东北径富平城,所在分裂,以溉田圃,北流入河,今无水。《尔雅》曰:灉反入。言河决复入者也。河之有灉,若汉之有潜也。河水又东北径浑怀障西。《地理志》,浑怀都尉治塞外者也。太和初,二齐平,徙历下民居此,遂有历城之名矣。南去北地三百里。河水又东北历石崖山西,去北地五百里。山石之上,自然有文,尽若虎马之状,粲然成著,类似图焉,故亦谓之画石山也。
又北过朔方临戎县西,河水东北径三封县故城东。汉武帝元狩三年置。《十三州志》曰:在临戎县西百四十里。河水又北径临戎县故城西。元朔五年立,旧朔方郡治,王莽之所谓推武也。河水又北,有枝渠东出,谓之铜口,东径沃野县故城南。汉武帝元狩三年立,王莽之绥武也,伎渠东注以溉田,所谓智通在我矣。河水又北屈而为南河出焉。河水又北迤西溢于窳浑县故城东。汉武帝元朔二年,开朔方郡县,即西部都尉治。有道,自县西北出鸡鹿塞,王莽更郡曰沟搜,县曰极武。其水积而为屠申泽,泽东西百二十里。故《地理志》曰:屠申泽在县东。即是泽也。阚駰谓之窳浑泽矣。
屈从县北东流,河水又屈而东流,为北河。汉武帝元朔二年,大将军卫青绝梓岭,梁北河是也。东径高阙南。《史记》,赵武灵王既袭胡服,自代并阴山下,至高阙为塞。山下有长城,长城之际,连山刺天,其山中断,两岸双闭,善能云举,望若阙焉。即状表目,故有高阙之名也。自闭北出荒中,阙口有城,跨山结局,谓之高阙戍。自古迄今,常置重捍,以防塞道。汉元朔四年,卫青将十万人,败右贤王于高阙。即此处也。河水又东径临河县故城北。汉武帝元朔三年,封代恭王子刘贤为侯国,王莽之监河也。
至河目县西,河水自临河县东径阳山南。《汉书注》曰:阳山在河北。指此山也。东流径石迹阜西。是阜破石之文,悉有鹿马之迹,故纳斯称焉,南屈径柯目县,在北假中。地名也。自高阙以东,夹山带河,阳山以往,皆北假也。《史记》曰秦使蒙恬将十万人,北击胡,度河取高阙据阳山北假中是也。北河又南合南河。南河上承西河,东径临戎县故城北,又东径临河县南,又东径广牧县故城北。东部都尉治,王莽之盐官也。径流二百许里,东会于河。河水又南径马阴山西。《汉书音义》曰:阳山在河北,阴山在河南。谓是山也,而即实不在河南。《史记音义》曰:五原安阳县北有马阴山。今山在县北,言阴山在河南,又传疑之,非也。余案南河、北河及安阳县以南,悉沙阜耳,无他异山。故《广志》曰:朔方郡北移沙七所,而无山以拟之,是义、志之僻也,阴山在河东南则可矣。河水又东南径朔方县故城东北,《诗》所谓城彼朔方也。汉元朔二年,大将军卫青取河南地为朔方郡,使校尉苏建筑朔方城,即此城也。王莽以为武符者也。案《地理志》云:金连盐泽、青盐泽并在县南矣。又案《魏土地记》曰:县有大盐池,其盐大而青白,名曰青盐。又名戎盐,入药分,汉置典盐官。池去平城宫千二百里,在新秦之中。服虔曰:新秦,地名,在北方千里。如淳曰:长安以北,朔方以南也。薛瓒曰:秦逐匈奴,收河南地,徙民以实之,谓之新秦也。
屈南过五原西安阳县南,河水自朔方东转,径渠搜县故城北。《地理志》,朔方有渠搜县,中部都尉治,王莽之沟搜亭也。《礼三朝记》曰:北发渠搜,南抚交趾。此举北对南。《禹贡》之所云析支、渠搜矣。河水又东,径西安阳县故城南,王莽更之曰漳安矣。河水又东,径田辟城南。《地理志》曰:故西部都尉治也。屈东过九原县南,河水又东径成宜县故城南,王莽更曰文虏也。河水又东径原亭城南。阚駰《十三州志》曰:中部都尉治。河水又东径宜梁县之故城南。阚駰曰:五原西南六十里,今世谓之石崖城。河水又东径稒阳城南,东部都尉治。又径河阴县故城北,又东径九原县故城南。秦始皇置九原郡,治此。汉武帝元朔二年,更名五原也。王莽之获降郡、成平县矣。西北接对一城,盖五原县之故城也,王莽之填河亭也。《竹书纪年》,魏襄王十七年,邯郸命吏大夫奴迁于九原,又命将军大夫通子戍吏,皆貉服矣。其城南面长河,北背连山,秦始皇逐匈奴,并河以东,属之阴山,筑亭障为河上塞。徐广《史记音义》曰:阴山在五原北。即此山也。始皇三十三年,起自临洮,东暨辽海,西并阴山,筑长城及开南越地,昼警夜作,民劳怨苦,故杨泉《物理论》曰:秦始皇使蒙恬筑长城,死者相属,民歌曰:生男慎勿举,生女哺用,不见长城下,尸骸相支拄。其冤痛如此矣。蒙恬临死曰:夫起临洮,属辽东,城堑万余里,不能不绝地脉,此固当死也。
又东过临沃县南,王莽之振武也。河水又东,枝津出焉。河水又东流,石门水南注之,水出石门山。《地理志》曰:北出石门障。即此山也。西北趣光禄城。甘露三年,呼韩邪单于还,诏遣长乐卫尉高昌侯董忠,车骑都尉韩昌等,将万六千骑,送单于居幕南,保光禄徐自为所筑城也,故城得其名矣。城东北,即怀朔镇城也。其水自障东南流,径临沃城东,东南注于河。河水又东径稠阳县故城南,王莽之固阴也。《地理志》曰:自县北出石门障,河水决其西南隅。又东南,枝津注焉,水上承大河于临沃县,东流七十里,北溉田,南北二十里,注于河。河水又东径塞泉城南而东注。又东过云中桢陵县南,又东过沙南县北,从县东屈南,过沙陵县西,大河东径咸阳县故城南。王莽之贲武也。河水屈而流,白渠水注之。水出塞外,西径定襄武进县故城北。西部都尉治,王莽更曰伐蛮,世祖建武中,封赵虑为侯国也。白渠水西北径成乐城北。《郡国志》曰:成乐,故属定襄也。《魏土地记》曰:云中城东八十里有成乐城。今云中郡治,一名石卢城也。白渠水又西径魏云中宫南。《魏土地记》曰:云中宫在云中县故城东四十里。白渠水又西南径云中故城南。故赵地。《虞氏记》云:赵武侯自五原河曲筑长城,东至阴山。又于河西造大城,一箱崩不就,乃改卜阴山河曲而祷焉。昼见群鹄游于云中,徘徊经日,见大光在其下。武侯曰:此为我乎?乃即于其处筑城,今云中城是也。秦始皇十三年,立云中郡,王莽更郡曰受降,县曰远服矣。白渠水又西北径沙陵县故城南,王莽之希恩县也。其水西注沙陵湖。又有芒干水出塞外,南径钟山。山即阴山。故郎中侯应言于汉口阴山东西千余里,单于之苑圃也,自孝武出师,攘之于漠北,匈奴失阴山,过之,未尝不哭。谓此山也。其水西南径武皋县,王莽之永武也。又南径原阳县故城西,又西南与武泉水合,其水东出武泉县之故城西南,县即王莽之所谓顺泉者也。水南流又西屈,径北舆县故城南。按《地理志》,五原有南舆县,王莽之甫利也,故此加北。旧中部都尉治。《十三州志》曰:广陵有舆,故此加北。疑太疏远也。其水又西南入芒干水。芒干水又西南径白道南谷口。有城在右,萦带长城,背山面泽,谓之白道城。自城北出有高阪,谓之白道岭。沿路惟土穴。出泉,挹之不穷。余每读《琴操》见琴慎相和,《雅歌录》云:饮马长城窟。及其跋涉斯途,远怀古事,始知信矣,非虚言也。顾瞻左右,山椒之上,有垣若颓基焉。沿溪亘岭,东西无极,疑赵武灵王之所筑也。芒干水又西南,径云中城北,白道中溪水注之,水发源武川北塞中,其水南流,径武川镇城。城以景明中筑,以御北狄矣。其水西南流,历谷,径魏帝行宫东。世谓之阿计头殿。宫城在白道岭北阜上,其城圆角而不方,四门列观,城内惟台殿而已。其水又西南历中溪,出山西南流,于云中城北,南注芒干水。芒干水又西,塞水出怀朔镇东北芒中,南流径广德殿西山下。余以太和十八年,从高祖北巡,届于阴山之讲武台,台之东,有高祖《讲武碑》,碑文是中书郎高聪之辞也。自台西出南上山,山无树木,惟童阜耳,即广德殿所在也。其殿四注两夏,堂字绮井,图画奇禽异兽之象。殿之西北,使得焜煌堂,雕楹镂桷,取状古之温室也。其时,帝幸龙荒,游鸾朔北。南秦王仇池杨难当舍著委诚,重译拜阙,陛见之所也。故殿以广德为名。魏太平真君三年,刻石树碑,勒宣时事,碑颂云:肃清帝道,振慑四荒。有蛮有戎,自彼氏羌,无思不服,重译稽颗。恂恂南秦,敛敛推亡,峨峨广德,奕奕焜煌。侍中、司徒东郡公崔浩之辞也。碑阴题宣城公李孝伯、尚书卢逻等从巨姓名,若新镂焉。其水历谷南出山,西南入芒干水。芒干水又西南注沙陵湖,湖水西南入于河。河水南入桢陵县西北,缘胡山,历沙南县东北,两山、二县之间而出。余以太和中为尚书郎,从高祖北巡,亲所径涉。县在山南,王莽之槙陆也,北去云中城一百二十里。县南六十许里。有东、西大山,山西枕河,河水南流,脉水寻《经》,殊乖川去之次,似非关究也。
又南过赤城东,又南过定襄桐过县西,定襄郡,汉高帝六年置,王莽之得降也。桐过县,王莽更名椅桐者也。河水于二县之间,济有君子之名。皇魏桓帝十一年,西幸榆中,东行代地,洛阳大贾,赍金货随帝后行,夜迷失道,往投津长曰:子封送之。渡河,贾人卒死,津长埋之。其子寻求父丧,发冢举尸,资囊一无所损。其子悉以金与之,津长不受。事闻于帝,帝曰:君子也。即名其津为君子济。济在云中城西南二百余里。河水又东南,左合一水,水出契吴东山,西径故里南,北俗谓之契吴亭。其水又西流注于河。河水又南,树颓水注之,水出东山西南流,右合中陵川水,水出中陵县西南山下,北俗谓之大浴真山,水亦取名焉。东北流,径中陵县故城东,北俗谓之北右突城。王莽之遮害也。《十三州志》曰:善无县南七十五里有中陵县,世祖建武二十五年置。其水又西北,右合一水,水出东山,北俗谓之贷敢山,水又受名焉。其水西北流,注于中陵水。中陵水又西北流,径善无县故城西。王莽之阴馆也。《十三州志》曰:旧定襄郡治。《地理志》,雁门郡治。其水又西北流,右会一水。水出东山下,北俗谓之吐文水,山又取名焉。北流径锄亭南,又西流径土壁亭南,西出峡,左入中陵水,中陵水又北分为二水,一水东北流,谓之沃水。又东径沃阳县故城南,北俗谓之可不埿城,王莽之敬阳也。又东北径沃阳城东,又东合可不埿水,水出东南六十里山下,西北流注沃水。沃水又东,径参合县南。魏因参合陉以即名也。北俗谓之仓鹤陉。道出其中,亦谓之参合口。陉在县之西北,即《燕书》所谓太子宝自河西还师参合,三军奔溃,即是处也。魏立县以隶凉城郡,西去沃阳县故城二十里。县北十里,有都尉城。《地理志》曰:沃阳县西部都尉治者也。北俗谓之阿养城。其水又东合一水,水出县东南六十里山下,北俗谓之灾豆浑水。西北流,注于沃水。沃水又东北流,注盐池。《地理志》曰:盐泽在东北者也。今盐池西南去沃阳县故城六十五里,池水徵渟,渊而不流,东西三十里,南北二十里。池北七里,即凉城郡治。池西有旧城,俗谓之凉城也,郡取名焉。《地理志》曰:泽有长丞。此城即长丞所治也。城西三里有小阜,阜下有泉,东南流注池,北俗谓之大谷北堆,水亦受目焉。中陵川水自枝津西北流,右合一水于连岭北。水出沃阳县东北山下,北俗谓之乌伏真山,水曰诰升袁河。西南流径沃阳县,左合中陵川,乱流西南与一水合,北俗谓之树颓水。水出东山下,西南流,右合诰升袁水,乱流西南注,分谓二水。左水枝分南出,北俗谓之太罗河;右水西径故城南,北俗谓之昆新城。其水自城西南流,注于河。河水又南,大罗水注之,水源上承树颓河,南流西转,径武州县故城南。《十三州志》曰:武州县在善无城西南百五十里。北俗谓之太罗城,水亦藉称焉。其水西南流,一水注之。水导故城西北五十里,南流径城西,北俗名之曰故槃回城。又南流注太罗河。太罗河又西南流,注于河。河水又左得湳水口。水出西河郡美稷县,东南流。《东观记》曰:郭伋,字细侯,为并州牧。前在州,素有恩德,老小相携道路,行部到西河美稷,数百小儿各骑竹马迎拜,伋问:儿曹何自远来?曰:闻使君到,喜,故迎。伋谢而发去,诸儿复送郭外。问:使君何日还?伋计日告之。及还,先期一日,念小儿,即止野亭,须期至乃往。其水又东南流,羌人因水以氏之。汉冲帝时,羌湳狐奴归化,盖其渠帅也。其水,俗亦谓之为遄波水,东南流入长城东。咸水出长城西咸谷,东入湳水。湳水又东南,浑波水出西北穷谷,东南流注于湳水。湳水又东径西河富昌县故城南,王莽之富成也。湳水又东流入于河。河水左合一水,出善无县故城西南八十里,其水西流,历于吕梁之山,而为吕梁洪。其山岩层岫衍,涧曲崖深,巨石崇竦,壁立千仞,河流激荡,涛涌波襄,雷渀电泄,震天动地。昔吕梁未辟,河出孟门之上,盖大禹所辟,以通河也。司马彪曰:吕梁在离石县西。今于县西历山寻河,并无过峘,至是乃为河之巨险,即吕梁矣,在离石北以东可二百有余里也。
又南过西河圁阳县东,西河郡,汉武帝元朔四年置,王莽改曰归新。圁水出上郡白土县圁谷,东径其县南。《地理志》曰:圁水出西,东入河。王莽更曰黄土也。东至长城,与神衔水合,水出县南神衔山,出峡,东至长城,入于圁。圁水又东径鸿门县。县,故鸿门亭。《地理风俗记》曰:圁阴县西五十里有鸿门亭、天封苑、火井庙,火从地中出。圁水又东,梁水注之,水出西北梁谷,东南流,注圁水。圁水又东径圁阴县北。汉惠帝五年立,王莽改曰方阴矣。又东,桑谷水注之,水出西北桑溪,东北流,入于圁。圁水又东径圁阳县南,东流注于河,河水又东,端水入焉。水西出号山。《山海经》曰:其木多漆棕,其草多穹穷,是多圁石,端水出焉,而东流注于河。河水又南,诸次之水入焉,水出上郡诸次山。《山海经》曰:诸次之山,诸次之水出焉。是山多木无草,鸟兽莫居,是多象蛇。其水东径榆林塞,世又谓之榆林山,即《汉书》所谓榆溪旧塞者也。自溪西去,悉榆柳之薮矣。缘历沙陵,届龟兹县西北。故谓广长榆也。王恢云:树榆为塞。谓此矣。苏林以为榆中在上郡,非也。案《始皇本纪》,西北逐匈奴,自榆中并河以东。属之阴山。然榆中在金城东五十许里,阴山在朔方东,以此推之,不得在上郡。《汉书音义》,苏林为失是也。其水东入长城,小榆水合焉。历涧西北,穷谷其源也。又东合首积水,水西出首积溪,东注诸次水,又东入于河。《山海经》曰:诸次之水,东流注于河,即此水也。,河水又南,汤水注之。《山海经》曰:水出上申之山,上无草木,而多硌石,下多榛楛,汤水出焉。东流注于河也。
又南离石县西,奢延水注之。水西出奢延县西南赤沙阜,东北流,《山海经》所谓生水出孟山者也。郭景纯曰:孟或作明。汉破羌将军段颎破羌于奢延泽,虏走洛川。洛川在南,俗因县土谓之奢延水,又谓之朔方水矣。东北流,径其县故城南。王莽之奢节也。赫连龙升七年,于是水之北,黑水之南,遣将作大匠梁公叱干阿利改筑大城,名曰统万城。蒸土加功,雉堞虽久,崇墉若新,并造五兵,器锐精利,乃咸百炼,为龙雀大鐶,号曰大夏龙雀。铭其背曰:古之利器,吴,楚湛卢,大夏龙雀,名冠神都,可以怀远,可以柔逋,如风靡草,威服九区。世甚珍之。又铸铜为大鼓,及飞廉、翁仲,铜驼、龙虎,皆以黄金饰之,列于宫殿之前。则今夏州治也。奢延水又东北与温泉合。源西北出沙溪,而东南流注奢延水。奢延水又东,黑水入焉。水出奢延县黑涧,东南历沙陵,注奢延水。奢延水又东台交兰水。水出龟兹县交兰谷,东南流注奢延水。奢延水又东北流与镜波水合,水源出南邪山南谷,东北流注于奢延水。奢延水又东径肤施县,帝原水西北出龟兹县,东南流。县因处龟兹降胡著称。又东南注奢延水。奢延水又东,径肤施县南。秦昭王三年置,上郡治。汉高祖并三秦,复以为郡。王莽以汉马员为增山连率,归,世祖以为上郡太守。司马彪曰:增山者,上郡之别名也。东入五龙山。《地理志》曰:县有五龙山、帝原水。自下亦为通称也。历长城东,出于白翟之中。又有平水,出西北平溪东南入奢延水。奢延水又东,走马水注之。水出西南长城北阳周县故城南桥山,昔二世赐蒙恬死于此。王莽更名上陵畤,山上有黄帝冢故也。帝崩,惟弓剑存焉,故世称黄帝仙矣。其水东流,昔段颎追羌出桥门至走马水,闻羌在奢延泽,即此处也。门,即桥山之长城门也。始皇令太子扶苏与蒙恬筑长城,起自临洮,至于碣石,即是城也。其水东北流入长城,又东北注奢延水。奢延水又东,与白羊水合,其水出于西南白羊溪,循溪东北,注于奢延水。奢延水又东入于河。《山海经》曰:生水东流注于河。河水又南,陵水注之。水出陵川北溪,南径其川,西转入河。河水又南得离石水口,水出离石北山,南流径离石县故城西。《史记》云:秦昭王伐赵取离石者也。汉武帝元朔三年,封代共王子刘绾为侯国,后汉西河郡治也。其水又南出西转径隔城县故城南。汉武帝元朔三年,封代共王子刘忠为侯国,王莽之慈平亭也。胡俗语讹,尚有千城之称。其水西流,注于河也。
又南过中阳县西,中阳县故城在东,东翼汾水,隔越重山,不滨于河也。
又南过土军县西,吐京郡治,故城,即土军县之故城也,胡、汉译言,音为讹变矣。其城圆长而不方,汉高帝十一年,以封武侯宣义为侯国。县有龙泉,出城东南,道左山下,牧马川上多产名驹骏,同滇池天马。其水西北流,至其城东南,土军水出道左高山,西南注之。龙泉水又北屈径其城东,西北入于河。河水又南合契水,傍溪东入穷谷,其源也。又南至禄谷水口,水源东穷此溪也。河水又南得大蛇水。发源溪首,西流入河。河水又南,右纳辱水。《山海经》曰:辱水出鸟山,其上多桑,其下多楮,阴多铁,阳多玉,其水东流,注于河。俗谓之秀延水。东流得浣水口,傍溪西转,穷溪便即浣水之源也。辱水又东会根水西南溪下。根水所发,而东北注辱水。辱水又东南,露跳水出西露溪东流,又东北入辱水,乱流注于河。河水又南,左合信支水,水发源东露溪,西流入千河。河水又南,左会石羊水,循溪东人,导源穷谷,西流注于河。
又南过上郡高奴县东。
域谷水东启荒原,西历长溪,西南入于河。河水又南合孔溪口。水出孔山南,历溪西流,注于河。孔山之上有穴,如车轮三所,东西相当,相去各二丈许,南北直通,故谓之孔山也。山在蒲城西南三十余里。河水又右会区水。《山海经·西次四经》之首曰阴山,西北百七十里曰申山,其上多谷、柞,其下多杻、橿,其阳多金、玉,区水出焉,而东流注于河。世谓之清水,东流入上郡长城。径老人山下,又东北流。至老人谷,傍水北出,极溪便得水源。清水又东得龙尾水口,水出北地神泉障北山龙尾溪,东北流注清水。清水又东会三湖水,水出南山三湖谷,东北流入清水。清水又东径高奴县,合丰林水,《地理志》谓之洧水也。故言高奴县有洧水,肥可,水上有肥,可接取用之。《博物志》称酒泉延寿县南山出泉水,大如筥,注地为沟,水有肥如肉汁,取著器中,始黄后黑,如凝膏,然极明,与膏无异,膏车及水碓缸甚佳,彼方人谓之石漆。水肥亦所在有之,非止高奴县洧水也。项羽以封董翳为翟王,居之三秦,此其一也。汉高祖破以县之,王莽之利平矣。民俗语讹,谓之高楼城也。丰林川长津泻注,北流会清水。清水又南,奚谷水注之,水西出奚川,东南流入清水。清水又东注于河。河水又南,蒲川水出石楼山,南径蒲城东。即重耳所奔之处也。又南历蒲子县故城西,今大魏之汾州治。徐广《晋纪》称,刘渊自离石南移蒲子者也。阚駰曰:蒲城在西北,汉武帝置。其水南出,得黄卢水口,水东出蒲子城南,东北入谷,极溪便水之源也。蒲水又南合紫川水,水东北出紫川谷,西南合江水,江水出江谷,西北入紫川水。紫川水又西北入蒲水,蒲水又西南入于河水。河水又南合黑水,水出定阳县西山,二源奇发,同泻一壑,东南流径其县北,又东南流,右合定水,俗谓之白水也。水西出其县南山定水谷,东径定阳县故城南。应劭曰:县在定水之阳也。定水又东注于黑水,乱流东南入于河。


\chapter{卷四 河水 }
又南过河东北屈县西,河水南径北屈县故城西,西四十里有风山,上有穴如轮,风气萧瑟,习常不止,当其冲飘也,略无生草,盖常不定,众风之门故也。风山西四十里,河南孟门山。《山海经》曰:孟门之山,其上多金玉,其下多黄垩、涅石。《淮南子》曰:龙门未辟,吕梁未凿,河出孟门之上,大溢逆流,无有丘陵,高阜灭之,名曰洪水。大禹疏通,谓之孟门。故《穆天子传》曰:北登孟门,九河之隥。孟门,即龙门之上口也,实为河之巨阨,兼孟门津之名矣。此石经始禹凿,河中漱广,夹岸崇深,倾崖返捍,巨石临危,若坠复倚。古之人有言,水非石凿,而能入石,信哉!其中水流交冲,素气云浮,往来遥观者,常若雾露沾人,窥深悸魄。其水尚崩浪万寻,悬流千丈,浑洪赑怒,鼓若山腾,浚彼颓叠,迄于下口。方知慎子下龙门,流浮竹,非驷马之追也。又有燕完水注之,异源合舍,西流注河。河水又南得鲤鱼,历涧东入,穷溪首便其源也。《尔雅》曰:鳝,鲔也。出巩穴,三月则上渡龙门,得渡为龙矣,否则,点额而还。非夫往还之会,何能便有兹称乎?河水又南,羊求水入焉。水东出羊求川,西径北屈县故城南。城,即夷吾所奔邑也,王莽之朕北也。《汲郡古文》曰:翟章救郑,次于南屈。应劭曰:有南,故加北。《国语》曰:二五言于献公曰:蒲与二屈,君之疆也。其水西流,注于河。河又南为采桑津。《春秋》僖公八年,晋里克败狄于采桑是也。赤水出西北罢谷川东,谓之赤石川,东入于河。河水又南合蒲水。西则两源并发,俱导一山,出西河阴山县,王莽之山宁也。阴山东麓,南水东北与长松水合,水西出丹阳山东,东北流,左入蒲水,蒲水又东北与北溪会,同为一川,东北注河。河水又南,丹水西南出丹阳山,东北径冶官东。俗谓之丹阳城,城之左右,犹有遗铜矣。其水东北会白水口,水出丹山东,而西北注之,丹水又东北入河。河水又南,黑水西出丹山东,而东北入于河。河水又南至崿谷,傍谷东北穷涧,水源所导也,西南流注于河。河水又南,洛水自猎山枝分东派,东南注于河。昔魏文侯筑馆洛阴,指谓是水也。
又南过皮氏县西,皮氏县,王莽之延平也,故城在龙门东南,不得延径皮氏,方届龙门也。又南出龙门口,汾水从东来注之。
昔者大禹导河积石,疏决梁山,谓斯处也。即《经》所谓龙门矣。《魏土地记》曰:梁山北有龙门山,大禹所凿,通孟津河口,广八十步,岩际镌迹,遗功尚存。岸上并有庙祠,祠前有石碑三所,二碑文字紊灭,不可复识,一碑是太和中立。《竹书纪年》,晋昭公元年,河赤于龙门三里。梁惠成王四年,河水赤于龙门三日。京房《易妖占》曰:河水赤,下民恨。河水又南,右合畅谷水,水自溪东南流,径夏阳县西北,东南注于河。河水又南径梁山原东。原自山东南出至河,晋之望也,在冯翊夏阳县之西北,临于河上。山崩,壅河三日不流,晋侯以问伯宗,即是处也。《春秋谷梁传》曰:成公五年,梁山崩,遏河水,三日不流。召伯尊遇辇者不避,使车右鞭之。辇者曰:所以鞭我者,其取道远矣。伯尊因问之,辇者曰:君亲缟素,率群臣哭之,斯流矣。如其言而河流。河水又南,崿谷水注之,水出县西北梁山,东南流,横溪水注之。水出三累山,其山层密三成,故俗以三累名山。案《尔雅》,山三成为昆仑丘。斯山岂亦昆仑丘乎?山下水际,有二石室,盖隐者之故居矣。细水东流,注于崿谷。侧溪山南有石室,西面有两石室,北面有二石室,皆因阿结牖,连扃接闼,所谓石室相距也。东厢石上,犹传杵臼之迹,庭中亦有旧宇处,尚仿佛前基;北坎室上,有微涓石溜,丰周瓢饮,似是栖游隐学之所。昔子夏教授西河,疑即此也,而无以辨之。溪水又东南径夏阳县故城北,故少梁也。秦惠文王十一年,更从今名矣。王莽之冀亭也。其水东南注于河。昔韩信之袭魏王豹也,以木罂自此渡。河水又南,右合陶渠水,水出西北梁山,东南流径汉阳太守殷济精庐南,俗谓之子夏庙。陶水又南径高门南,盖层阜堕缺,故流高门之称矣。又东南径华池南。池方三百六十步,在夏阳城西北四里许。故司马迁《碑文》云:高门华池,在兹夏阳。今高门东去华池三里。溪水又东南径夏阳县故城南。服虔曰:夏阳,虢邑也,在太阳东三十里。又历高阳宫北,又东南径司马子长墓北。墓前有庙,庙前有碑。永嘉四年,汉阳太守殷济瞻仰遗文,大其功德,遂建石室,立碑树桓。《太史公自叙》曰:迁生于龙门。是其坟墟所在矣。溪水东南流入河。昔魏文侯与吴起浮河而下,美河山之固,即于此也。河水又南,徐水注之。水出西北梁山,东南流径汉武帝登仙宫东,东南流,绝强梁原。右径刘仲城北,是汉祖兄刘仲之封邑也。故徐广《史记音义》曰: 阳,国名也,高祖八年,侯刘仲是也。其水东南径子夏陵北,东入河。河水又南径子夏石室东,南北有二石室,临侧河崖,即子夏庙室也。
又南过汾阴县西,河水东际汾阴脽。县故城在脽侧,汉高帝六年,封周昌为侯国。《魏土地记》曰:河东郡北八十里有汾阴城,北去汾水三里,城西北隅曰脽丘,上有后土祠。《封禅书》曰元鼎四年,始立后土祠于汾阴脽丘是也。又有万岁宫,汉宣帝神爵元年幸万岁宫,东济大河,而神鱼舞水矣。昔赵简子沉栾徼于此,曰:吾好声色,而是子致之;吾好士,六年不进一人,是长吾过而黜吾善,君子以为能谴矣。河水又径阳城东。周威烈王之十七年,魏文侯伐秦至郑,还筑汾阴脽阳,即此城也。故有莘邑矣,为太姒之国。《诗》云:在之阳,在渭之涘。又曰:缵女维莘,长子维行。谓此也。城北有瀵水,南去二水各数里,其水东径其城内,东入于河。又于城内侧中,有瀵水东南出城,注于河。城南又有瀵水,东流注于河。水南犹有文母庙,庙前有碑,去城十五里。水,即水也,县取名焉。故应劭曰:在水之阳也。河水又南,瀵水入焉。水出汾阴县南四十里,西去河三里,平地开源,濆泉上涌,大几如轮,深则不测,俗呼之为瀵魁。古人壅其流以为陂水,种稻,东西二百步。南北百余步,与阳瀵水夹河,河中渚上,又有一瀵水,皆潜相通。故吕忱曰:《尔雅》,异出同流为瀵水。其水西南流,历蒲坂西,西流注于河。河水又南径陶城西。舜陶河滨,皇甫士安以为定陶,不在此也。然陶城在蒲坂城北,城,即舜所都也。南去定山不远,或耕或陶,所在则可,何必定陶,方得为陶也?舜之陶也,斯或一焉。孟津有陶河之称,盖从此始之。南对蒲津关。汲冢《竹书纪年》,魏襄王七年,秦王来见于蒲坂关,四月,越王使公师隅来献乘舟,始罔及舟三百,箭五百万,犀角、象齿焉。
又南过蒲坂县西,《地理志》曰:县,故蒲也。王莽更名蒲城。应劭曰:秦始皇东巡,见有长坂,故加坂也。孟康曰:晋文公以赂秦。秦人还蒲于魏,魏人喜,曰:蒲反矣,故曰蒲反也。薛瓒注《汉书》曰:《秦世家》以垣为蒲反。然则本非蒲也。皇甫谧曰:舜所都也。或言蒲坂,或言平阳及潘者也。今城中有舜庙,魏秦州刺史治。太和迁都罢州,置河东郡,郡多流杂,谓之徙民。民有姓刘名堕者,宿擅工酿,采挹河流,酝成芳酎,悬食同枯枝之年,排于桑落之辰,故酒得其名矣。然香醑之色,清白若滫浆焉,别调氛氲,不与他同,兰薰麝越,自成馨逸,方土之贡,选最佳酌矣。自王公庶友,牵拂相招者,每云:索郎有顾,思同旅语。索郎反语为桑落也,更为籍征之隽句,中书之英谈。郡南有历山,谓之历观。舜所耕处也,有舜井。妫、汭二水出焉,南曰妫水,北曰妫水,西径历山下,上有舜庙。周处《风土记》曰:旧说,舜葬上虞。又《记》云:耕于历山。而始宁、剡二县界上,舜所耕田,于山下多柞树,吴、越之间,名柞为枥,故曰历山。余案周处此志为不近情,传疑则可,证实非矣。安可假木异名,附山殊称,强引大舜,即比宁壤,更为失志记之本体,差实录之常经矣。历山、妫、汭,言是则安,于彼乖矣。《尚书》所谓釐降二女于妫汭也。孔安国曰:居妫水之内。王肃曰:妫汭,虞地名。皇甫谧曰:纳二女于妫水之汭,马季长曰:水所出曰汭。然则汭似非水名,而今见有二水异源同归,浑流西注入于河。河水南径雷首山西,山临大河,北去蒲坂三十里,《尚书》所谓壶口雷首者也。俗亦谓之尧山,山上有故城,世又曰尧城。阚駰曰:蒲坂,尧都。按《地理志》曰:县有尧山、首山祠,雷首山在南。事有似而非,非而似,千载吵邈,非所详耳。又南,涑水注之,水出河北县雷首山。县北与蒲坂分,山有夷齐庙。阚駰《十三州志》曰:山,一名独头山,夷齐所隐也。山南有古冢,陵柏蔚然,攒茂丘阜,俗谓之夷齐墓也。其水西南流,亦曰雷水。《穆天子传》曰:壬戌,天子至于雷首,犬戎胡觞天子于雷首之阿,乃献良马四六,天子使孔牙受之于雷水之于是也。昔赵盾田首山,食祁弥明翳桑之下,即于此也。涑水又西南流,注于河,《春秋左传》谓之涑川者也,俗谓之阳安涧水。
又南至华阴潼关,渭水从西来注之。
汲郡《竹书纪年》曰:晋惠公十五年,秦穆公帅师送公子重耳,涉自河曲。《春秋左氏》僖公二十四年,秦伯纳之,及河,子犯以壁授公子曰:臣负羁纵,从君巡于天下,臣之罪多矣,臣犹知之,而况君乎?请由此亡。公子曰:所不与舅氏同心者,有如白水。投璧于此。子推笑曰:天开公子,子犯以为功,吾不忍与同位,遂逃焉。河水历船司空,与渭水会。《汉书·地理志》:旧京兆尹之属县也。左丘明《国语》云:华岳本一山当河,河水过而曲行,河神巨灵,手荡脚踏,开而为两,今掌足之迹,仍存华岩。《开山图》曰:有巨灵胡者,遍得坤元之道,能造山川,出江河,所谓巨灵赑屭,首冠灵山者也。常有好事之士,故升华岳而观厥迹焉。自下庙历列柏南行十一里,东回三里,至中祠,又西南出五里,至南祠,谓之北君祠,诸欲升山者,至此皆祈请焉。从此南入谷七里,又届一祠,谓之石养父母,石龛、木主存焉。又南出一里,至天井,井裁客人,穴空,迂回顿曲而上,可高六丈余。山上又有微涓细水,流入井中,亦不甚沾人。上者皆所由陟,更无别路,欲出井望空,视明如在室窥窗也。出井东南行二里,峻坂斗上斗下,降此坂二里许,又复东上百丈崖,升降皆须扳绳挽葛而行矣。南上四里,路到石壁,缘旁稍进,径百余步。自此西南出六里,又至一祠,名曰胡越寺,神像有童子之容,从祠南历夹岭,广裁三尺余,两箱悬崖数万仞,窥不见底,祀祠有感,则云与之平,然后敢度,犹须骑岭抽身,渐以就进,故世谓斯岭为搦岭矣。度此二里,便届山顶。上方七里,灵泉二所:一名蒲池,西流庄于涧;一名太上泉,东注涧下。上宫神庙近东北隅,其中塞实杂物,事难详载。自上富东北出四百五十步,有屈岭,东南望巨灵手迹,惟见洪崖、赤壁而已。都无山下上观之分均矣。河在关内南流,潼激关山,因谓之潼关。濩水注之,水出松果之山。北流径通谷,世亦谓之通谷水,东北注于河,《述征记》所谓潼谷水者也。或说因水以名地也。河水自潼关东北流,水侧有长坂,谓之黄巷坂。坂傍绝涧,陟此坂以升潼关,所谓溯黄巷以济潼矣。历北出东崤,通谓之函谷关也。邃岸天高,空谷幽深,涧道之峡,车不方轨,号曰天险。故《西京赋》曰:岩险周固,衿带易守,所谓秦得百二,并吞诸侯也。是以王元说隗嚣曰:请以一丸泥,东封函谷关,图王不成,其弊足霸矣。郭缘生《记》曰:汉末之乱,魏武征韩遂、马超,连兵此地。今际河之西,有曹公垒。道东原上,云李典营。义熙十三年,王师曾据此垒。《西征记》曰:沿路逶迆,入函道六里,有旧城,城周百余步,北临大河,南对高山。姚氏置关以守峡,宋武帝入长安,檀道济、王镇恶,或据山为营,或平地结垒,为大小七营,滨带河险,姚氏亦保据山原陵阜之上,尚传故迹矣。关之直北,隔河有层阜,巍然独秀,孤峙河阳,世谓之风陵。戴延之所谓风塠者也。南则河滨姚氏之营,与晋对岸。河水又东北,玉涧水注之,水南出玉溪,北流径皇天原西。《周固记》:开山东首上平博,方可里余,三面壁立,高千许仞,汉世祭天于其上,名之为皇天原。上有汉武帝思子台。又北径闅乡城西。《郡国志》曰:宏农湖县有闅乡。世谓之闅乡水也。魏尚书仆射闅乡侯河东卫伯儒之故邑。其水北流注于河。河水又东径闅乡城北,东与全鸠涧水合,水出南山,北径皇天原东。《述征记》曰:全节,地名也。其西名桃原,古之桃林,周武王克殷休牛之地矣。《西征赋》曰:咸征名于桃原者也。《晋太康地记》曰:桃林在闅乡南谷中。其水又北流注于河。
又东过河北县南,县与湖县分河。蓼水出襄山蓼谷,西南注于河。河水又东,永乐涧水注之,水北出于薄山,南流径河北县故城西。故魏国也。晋献公灭魏,以封毕万。卜偃曰:魏大名也,万后其昌乎?后乃县之,在河之北,故曰河北县也。今城南、西二面并去大河可二十余里,北去首山十许里,处河山之间,土地迫隘,故《魏风》著《十亩》之诗也,城内有龙泉,南流出城,又南,断而不流。永乐溪水又南入于河。余按《中山经》,即渠猪之水也。太史公《封禅书》称,华山以西名山七,薄山其一焉。薄山,即襄山也。徐广曰:薄坂县有襄山。《山海经》曰:蒲山之首,曰甘枣之山,共水出焉,而西流注于河。东则渠猪之山,渠猪之水出焉,而南流注于河。如准《封禅书》,二水无西南注河之理。今诊蓼水,川流所趣,与共水相扶,永乐溪水导源注于河,又与渠猪势合。蒲山统目总称,亦与襄山不殊。故扬雄《河东赋》曰:河灵矍踢,掌华蹈襄。注云:襄山在潼关北十余里。以是推之,知襄山在蒲坂溪水,即渠猪之水也。河水自河北城南,东径芮城。二城之中,有段干木冢。于木,晋之贤人也,魏文侯过其门,式其庐,所谓德尊万古,芳越来今矣。汲冢《竹书纪年》曰:晋武公元年,尚一军芮人乘京,荀人、董伯皆叛。匪直大荔,故芮也,此亦有焉。《纪年》又云:晋武公七年,芮伯万之母芮姜逐万,万出奔魏。八年,周师、虢师围魏,取芮伯万而东。九年,戎人逆芮伯万于郊。斯城亦或芮伯之故画也。河水右会槃涧水,水出湖县夸父山,北径汉武帝思子宫归来望恩台东,又北流入于河。河水又东径湖县故城北。昔范叔入关,遇穰侯于此矣。湖水出桃林塞之夸父山,广圆三百仞。武王伐纣,天下既定,王巡岳渎,放马华阳,散牛桃林,即此处也。其中多野马,造父于此得骅骝、绿耳、盗骊之乘,以献周穆王,使之驭以见西王母。湖水又北径湖县东,而北流入于河。《魏土地记》曰:宏农湖县有轩辕黄帝登仙处。黄帝采首山之铜,铸鼎于荆山之下,有龙垂胡于鼎,黄帝登龙,从登者七十人,遂升于天。故名其地为鼎胡。荆山在冯翊,首山在蒲坂,与湖县相连。《晋书地道记》、《太康记》并言胡县也。汉武帝改作湖。俗云黄帝自此乘龙上天也。《地理志》曰:京兆湖县有周天子祠二所,故曰胡,不言黄帝升龙也。《山海经》曰:西九十里曰夸父之山,其木多棕、柟,多竹箭,其阳多玉,其阴多铁,其北有林焉,名曰桃林,其中多马,湖水出焉,北流注于河。故《三秦记》曰:桃林塞在长安东四百里,若有军马经过,好行则牧华山,休息林下;恶行则决河漫延,人马不得过矣。河水又东合柏谷水,水出宏农县南石堤山,山下有石堤祠。铭云:魏甘露四年,散骑常侍、征南将军、豫州刺史、领宏农太守、南平公之所经建也。其水北流,径其亭下。晋公子重耳出亡及柏谷,卜适齐、楚,狐偃曰:不如之翟。汉武帝尝微行此亭,见馈亭长妻。故潘岳《西征赋》曰:长征客于柏谷,妻睹貌而献餐。谓此亭也,谷水又北流入于河。河水又东,右合门水,门水,即洛水之枝流者也。洛水自上洛县东北于拒阳城西北,分为二水。枝渠东北出,为门水也。门水又东北历阳华之山,即《山海经》所谓阳华之山,门水出焉者也。又东北历峡,谓之鸿关水。水东有城,即关亭也,水西有堡,谓之鸿关堡,世亦谓之刘、项裂地处,非也。余按上洛有鸿胪围池,是水津渠沿注,故谓斯川为鸿胪涧,鸿关之名,乃起是矣。门水又东北历邑川,二水注之。左水出于阳华之阴,东北流,径盛墙亭西,东北流,与右水合。右水出阳华之阳,东北流,径盛墙亭东,东北与左水合。即《山海经》所谓姑之水出于阳华之阴,东北流注于门水者也。又东北,烛水注之,水有二源,左水南出于衙岭,世谓之石城山,其水东北流,径石城西,东北合右水;右水出石城山,东北径石城东,东北入左水。《地理志》曰:烛水出衙岭下谷。《开山图》曰:衙山在函谷山西南。是水乱流,东注于姑之水。二水悉得通称矣。历涧东北出,谓之开方口。水侧有阜,谓之方伯堆。宋奋武将军鲁方平、建武将军薛安都等,与建威将军柳元景北入,军次方伯堆者也。堆上有城,即方平所筑也。又东北径邑川城南,即汉封窦门之故邑,川受其名,亦曰窦门,城在函谷关南七里。又东北,田渠水注之。水出衙山之白石谷,东北流径故丘亭东,是薛安都军所从城也。其水又径鹿蹄山西,山石之上有鹿蹄,自然成著,非人功所刊。历田渠川,谓之田渠水,西北流注于烛水。烛水又北入门水。水之左右,即函谷山也。门水又北径宏农县故城东。城即故函谷关校尉旧治处也,终军弃繻于此。燕丹、盂尝亦义动鸡鸣于其下,可谓深心有感,志诚难夺矣。昔老子西入关,尹喜望气于此也。故赵至《与嵇茂齐书》曰:李叟入秦,及关而叹。亦言《与嵇叔夜书》及关尹望气之所,异说纷纶,并未知所定矣。汉武帝元鼎四年,徒关于新安县,以故关为弘农县、弘农郡治。王莽更名右队。刘桓公为郡,虎相随渡河,光武问而善之。其水侧城北流而注于河。河水于此,有浢津之名。说者咸云,汉武微行柏谷,遇辱窦门,又感其妻深识之馈,既返玉阶,厚赏赍焉,赐以河津,令其鬻渡。今窦津是也。故潘岳《西征赋》云:酬匹妇其已泰,胡厥夫之谬官,袁豹之徒,并以为然。余案河之南畔,夹侧水濆有津,谓之浢津。河北县有浢水,南入于河,河水故有浢津之名,不从门始,盖事类名同,故作者疑之。《竹书》、《穆天子传》曰:天子自窴軨,乃次于浢水之阳,丁亥,入于南郑。考其沿历所踵,路直斯津,以是推之,知非因门矣,俗或谓之偃乡涧水也。河水又东,左合一水,其水二源疏引,俱导薄山,南流会成一川。其二水之内,世谓之闲原。言虞芮所争之田,所未详矣。又南注于河。河之右,曹水注之,水出南山,北径曹阳亭西。陈涉遣周章入秦,少府章邯斩之于此。魏氏以为好阳。《晋书地道记》曰:亭在弘农县东十三里。其水西北流,入于河。河水又东,菑水注之。水出常烝之山,西北径曲沃城南,又屈径其城西,西北入河。诸注述者,咸言曲沃在北,此非也。魏司徒崔浩以为曲沃地名也。余案《春秋》文公十三年,晋侯使詹嘉守桃林之塞,处此以备秦。时以曲沃之官守之故,曲沃之名,遂为积古之传矣。河水又东得七里涧,涧在陕城西七里,故因名焉。其水自南山通河,亦谓之曹阳坑。是以潘岳《西征赋》曰:行于漫渎之口,憩于曹阳之墟。袁豹、崔浩亦不非其地矣。余按《汉书》,昔献帝东迁,逼以寇难,李傕、郭汜追战于弘农涧,天子遂露次曹阳。杨奉、董承,外与傕和,内引白波、李乐等破傕,乘舆于是得进。复来战,奉等大败,兵相连缀四十余里,方得达陕。以是推之,似非曹阳。然以《山海经》求之,菑、曹字相类,是或有曹阳之名也。河水又东合潐水,水导源常烝之山。俗谓之为于山,盖先后之异名也。山在陕城南八十里。其川二源双导,同注一壑,而西北流注于河。
又东过陕县北,橐水出橐山,西北流。又有崖水出南山北谷,径崖峡,北流与干山之水会。水出于山东谷,两川合注于崖水。又东北注橐水,橐水北流出谷,谓之漫涧矣。与安阳溪水合,水出石崤南,西径安阳城南。汉昭帝封上官桀为侯国。潘岳所谓我徂安阳也。东合漫涧水。水北有逆旅亭,谓之漫口客舍也。又西径陕县故城南,又合一水,谓之渎谷水,南出近溪,北流注橐。橐水又西北径陕城西,西北入于河。河北对茅城。故茅亭,茅戎邑也。《公羊》曰晋败之大阳者也。津亦取名焉。《春秋》文公三年,秦伯伐晋,自茅津济,封崤尸而还是也。东则咸阳涧水注之,水出北虞山南,至陕津注河。河南即陕城也。昔周、召分伯,以此城为东、西之别,东城即虢邑之上阳也,虢仲之所都,为南虢,三虢,此其一焉。其大城中有小城,故焦国也,武王以封神农之后于此。王莽更名黄眉矣。戴延之云:城南倚山原,北临黄河,悬水百余仞,临之者咸悚惕焉。西北带河,水涌起方数十丈,有物居水中,父老云,铜翁仲所没处。又云,石虎载经于此沉没,二物并存,水所以涌,所未详也。或云:翁仲头髻常出,水之涨减,恒与水齐。晋军当至,髻不复出,今惟见水异耳,嗟嗟有声,声闻数里。案秦始皇二十六年,长狄十二见于临洮,长五丈余,以为善祥,铸金人十二以象之,各重二十四万斤,坐之宫门之前,谓之金狄。皆铭其胸云:皇帝二十六年,初兼天下,以为郡县,正法律,同度量,大人来见临洮,身长五丈,足六尺,李斯书也。故卫恒《叙篆》曰:秦之李斯,号为工篆,诸山碑及铜人铭,皆斯书也。汉自阿房徙之未央宫前,俗谓之翁仲矣。地皇二年,王莽梦铜人泣,恶之,念铜人铭有皇帝初兼天下文,使尚方工镌灭所梦铜人膺文。后董卓毁其九为钱。其在者三,魏明帝欲徙之洛阳,重不可胜,至霸水西停之。《汉晋春秋》曰:或言金狄泣,故留之,石虎取置邺宫。苻坚又徙之长安,毁二为钱,其一未至而苻坚乱,百姓推置陕北河中,于是金狄灭。余以为鸿河巨渎,故应不为细梗踬湍;长津硕浪,无宜以微物屯流。斯水之所以涛波者,盖《史记》所云魏文侯二十六年,虢山崩,壅河所致耳。献帝东迁,日夕潜渡,坠坑争舟,舟指可掬,亦是处矣。
又东过大阳县南,交涧水出吴山,东南流入河,河水又东。路涧水亦出吴山,东径大阳城西,西南流入于河。河水又东径大阳县故城南。《竹书纪年》曰:晋献公十有九年,献公会虞师伐虢,灭下阳;虢公丑奔卫,献公命瑕父吕甥邑于虢都。《地理志》曰:北虢也,有天子庙,王莽更名勤田。应劭《地理风俗记》曰:城在大河之阳也。河水又东,沙涧水注之,水出北虞山,东南径傅岩,历傅说隐室前。俗名之为圣人窟。孔安国《传》,傅说隐于虞、虢之间。即此处也。傅岩东北十余里,即巅軨坂也,《春秋左传》所谓入自巅軨者也。有东、西绝涧,左右幽空穷深,地壑中则筑以成道,指南北之路,谓之为軨桥也。傅说佣隐,止息于此,高宗求梦得之是矣。桥之东北有虞原,原上道东,有虞城。尧妻舜以嫔于虞者也。周武王以封太伯后虞仲于此,是为虞公。《晋太康地记》所谓北虞也。城东有山,世谓之五家冢,冢上有虞公庙。《春秋谷梁传》曰:晋献公将伐虢,荀息曰:君何不以屈产之乘,垂棘之璧,假道于虞。公曰:此晋国之宝也。曰:是取中府置外府也。公从之。及取虢灭虞,乃牵马操璧,璧则犹故,马齿长矣。即宫之奇所谓虞,虢其犹辅车相依,唇亡则齿寒,虢亡,虞亦亡矣。其城北对长坂二十许里,谓之虞坂。戴延之曰:自上及下,七山相重。《战国策》曰:昔骐骥驾盐车上于虞坂,迁延负辕而不能进。此盖其困处也。桥之东北山溪中,有小水西南庄沙涧,乱流径大阳城东。河北郡治也。沙涧水南流注于河。河水又东,左合积石、土柱二溪。并北发大阳之山,南流入于河。是山也,亦通谓之为薄山矣。故《穆天子传》曰:天子自監,己丑,南登于薄山窴軨之隥,乃宿于虞是也。
又东过砥柱间,砥柱,山名也。昔禹治洪水,山陵当水者凿之,故破山以通河。河水分流,包山而过,山见水中若柱然,故曰砥柱也。三穿既决,水流疏分,指状表目,亦谓之三门矣。山在虢城东北、大阳城东也。《搜神记》称齐景公渡于江、沈之河,鼋衔左骖,没之,众皆惕。古冶子于是拔剑从之,邪行五里,逆行三里,至于砥柱之下,乃鼋也。左手待鼋头,右手挟左骖,燕跃鹄踊而出,仰天大呼,水为逆流三百步,观者皆以为河伯也。亦或作江、沅字者也,若因地而为名,则宜在蜀及长沙。案《春秋》,此二土井景公之所不至,古冶子亦无因而骋其勇矣。刘向叙《晏子春秋》,称古冶子曰:吾尝济于河,鼋衔左骖以入砥柱之流,当是时也,从而杀之,视之乃鼋也。不言江、沅矣。又考史迁记云:景公十二年,公见晋平公;十八年,复见晋昭公。应轩所指,路直斯津。从鼋砥柱事或在兹。又云观者以为河伯,贤于江、沉之证。河伯本非江神,又河可知也。河之右侧,崤水注之。水出河南盘崤山,西北流,水上有梁,俗谓之鸭桥也。历涧东北流,与石崤水合,水出石崤山。山有二陵:南陵,夏后皋之墓也;北陵,文王所避风雨矣。言山径委深,峰阜交荫,故可以避风雨也。秦将袭郑,蹇叔致谏而公辞焉,蹇叔哭子曰:吾见其出,不见其入,晋人御师必于崤矣,余收尔骨焉。孟明果覆秦师于此。崤水又北,左合西水,乱流注于河。河水又东,干崤之水注焉。水南导于干崤之山,其水北流,缠络二道。汉建安中,曹公西讨巴、汉,恶南路之险,故更开北道,自后行旅,率多从之。今山侧附路有石铭云:晋太康三年,宏农太守梁柳修复旧道。太崤以东,西崤以西,明非一崤也。西有二石,又南五十步,临溪有恬漠先生《翼神碑》,盖隐斯山也。其水北流注于河。河水翼岸夹山,巍峰峻举,群山叠秀,重岭干霄。郑玄案《地说》,河水东流,贯砥柱,触阏流,今世所谓砥柱者,盖乃阏流也。砥柱当在西河,未详也。余案,郑玄所说非是,西河当无山以拟之。自砥柱以下,五户已上,其间百二十里,河中竦石桀出,势连襄陆,盖亦禹凿以通河,疑此阏流也。其山虽辟,尚梗湍流,激石云洄,澴波怒溢,合有十九滩,水流迅急,势同三峡,破害舟船,自古所患。汉鸿嘉四年,杨焉言从河上下,患砥柱隘,可镌广之。上乃令焉镌之,裁没水中,不能复去,而令水益湍怒,害甚平日。魏景初二年二月,帝遣都督沙丘部、监运谏议大夫寇慈,帅工五千人,岁常修治,以平河阻。晋泰始三年正月,武帝遣监运大中大夫赵国、都匠中郎将河东乐世,帅众五千余人,修治河滩,事见《五户祠铭》。虽世代加功,水流漰渀,涛波尚屯。及其商舟是次,鲜不踟蹰难济,故有众峡诸滩之言。五户,滩名也,有神祠,通谓之五户将军,亦不知所以也。
又东过平阴县北,清水从西北来注之。
清水出清廉山之西岭,世亦谓之清营山。其水东南流,出峡,峡左有城,盖古关防也。清水历其南,东流径皋落城北。眼虔曰:赤翟之都也。世谓之倚毫城,盖读声近转,因失实也。《春秋左传》所谓晋侯使太子申生伐东山皋落氏者也。与倚毫川水合,水出北山矿谷,东南流注于清。清水又东径清廉城南,又东南流,右会南溪水,水出南山而东注清水。清水又东合干枣涧水,水出石人岭下,南流,俗谓之扶苏水。又南历奸苗北马头山,亦曰白水原,西南径垣县故城北。《史记》,魏武侯二年城安邑至垣,即是县也。其水西南流,注清水。水色白浊,初会清流,乃有玄素之异也。清水又东南径阳壶城东,即垣县之壶亭丘。晋迁宋五大夫所居也。清水又东南流注于河。河水又东与教水合,水出垣县北教山,南径辅山。山高三十许里,上有泉源,不测其深,山顶周圆五六里,少草木。《山海经》曰:孟门东南有平山,水出于其上,潜于其下。又是王屋之次,疑即平山也。其水南流,历鼓钟上峡,悬洪五丈,飞流注壑,夹岸深高,壁立直上,轻崖秀举,百有余丈,峰次青松,岩悬赬石,于中历落,有翠柏生焉,丹青绮分,望若图绣矣。水广十许步,南流历鼓钟川,分为二涧:一涧西北出,百六十许里,山岫回岨,才通马步。今闻喜县东北谷口,犹有于河里,故沟存焉,今无复有水。一水历冶官西,世人谓之鼓钟城。城之左右,犹有遗铜及铜钱也。城西阜下有大泉,西流注涧,与教水合,伏入石下,南至下峡。《山海经》曰:鼓钟之山,帝台之所以觞百神。即是山也。其水重源又发,南至西马头山东截坡下,又伏流南十余里,复出,又谓之伏流水,南入于河。《山海经》曰:教山,教水出焉,而南流注于河。是水冬于夏流,实惟于河也,今世人犹谓之为干涧矣。河水又与畛水合。水出新安县青要山,今谓之疆山,其水北流入于河。《山海经》曰:青要之山,畛水出焉。即是水也。河水又东,正回之水入焉,水出騩山,疆山东阜也。东流,俗谓之疆川水,与石瓜畴川合。水出西北石涧中,东南流注于疆川水。疆川水又东径疆冶铁官东,东北流注于河。河水又东合庸庸之水,水出河东垣县宜苏山,俗谓之长泉水。《山海经》曰:水多黄贝,伊洛门也。其水北流,分为二水,一水北入河,一水又东北流注于河。河水又东径平阴县北。《地理风俗记》曰:河南平阴县,故晋阴地,阴戎之所居。又曰:在平城之南,故曰平阴也,三老董公说高祖处,陆机所谓皤皤董叟,谟我平阴者也。魏文帝改曰河阴矣。河水又会濝水,水出垣县王屋山西濝溪,夹山东南流,径故城东,即濝关也。汉光武建武二年,遣司空王梁北守濝关、天井关,击赤眉别濝,皆降之。献帝自陕北渡安邑,东出濝关,即是关也。濝水西屈,径关城南,历轵关南,径苗亭西。亭,故周之苗邑也。又东流注于河。《经》书清水,非也。是乃濝水耳。又东至邓。
洛阳西北四十二里,故邓乡矣。


\chapter{卷五 河水 }
又东过平县北,湛水从北来注之。
河水又东径河阳县故城南。《春秋经》书天王狩于河阳,王申,公朝于王所,晋侯执卫侯归于京师,《春秋左传》僖公二十八年,冬,会于温,执卫侯。是会也,晋侯召襄王以诸侯见,且使王狩。仲尼曰:以臣召君,不可以训。故书曰天王狩于河阳,言非其狩地。服虔、贾逵曰:河阳,温也。班固《汉书·地理志》,司马彪、袁山松《郡国志》,《晋太康地道记》,《十二州志》:河阳别县,非温邑也。汉高帝六年,封陈涓为侯国,王莽之河亭也。《十三州志》曰:治河上,河,孟津河也。郭缘生《述征记》曰:践土,今冶坂城。是名异《春秋》焉,非也。今河北见者,河阳城故县也,在冶坂西北,盖晋之温地,故群儒有温之论矣。《魏土地记》曰:冶坂城旧名汉祖渡,城险固,南临孟津河。河水右径临平亭北。《帝王世纪》曰:光武葬临平亭南,西望平阴者也。河水又东径洛阳县北,河之南岸有一碑,北面题云:洛阳北界,津水二渚,分属之也。上旧有河平侯祠,祠前有碑,今不知所在。郭颁《世语》曰:晋文王之世,大鱼见孟津,长数百步,高五丈,头在南岸,尾在中渚,河平侯祠即斯祠也。河水又东径平县故城北。汉武帝元朔三年,封济北贞王子刘遂为侯国,王莽之所谓治平矣,俗谓之小平也。有高祖讲武场,河北侧岸有二城相对,置北中郎府,徙诸徒隶府户,并羽林虎贲领队防之。河水南对首阳山。《春秋》所谓首戴也,《夷齐之歌》所以曰登彼西山矣。上有夷齐之庙,前有二碑。并是后汉河南尹广陵陈导、洛阳令徐循与处士平原苏腾、南阳何进等立,事见其碑。又有周公庙,魏氏起玄武观于芒垂,张景阳《玄武观赋》所谓高楼特起,竦跱岧峣,直亭亭以孤立,延千里之清飙也。朝廷又置冰室于斯阜,室内有冰井。《春秋左传》曰:日在北陆而藏冰。常以十二月采冰于河津之隘,峡石之阿,北阴之中,即《邠诗》二之日凿冰冲冲矣。而内于井室,所谓纳于凌阴者也。河南有钩陈垒,世传武王伐纣,八百诸侯所会处,《尚书》所谓不期同时也。紫微有钩陈之宿主,斗讼兵阵,故遁甲攻取之法以所攻神,与钩陈并气下制,所临之辰,则决禽敌,是以垒资其名矣。河水子斯,有盟津之目。《论衡》曰:武王伐纣,升舟,阳侯波起,疾风逆流,武王操黄钺而麾之,风波毕除,中流,白鱼入于舟,燔以告天,与八百诸侯咸同此盟。《尚书》所谓不谋同辞也。故曰孟津,亦曰盟津。《尚书》所谓东至于孟津者也。又曰富平津。《晋阳秋》曰:杜预造河桥于富平津。所谓造舟为梁也。又谓之为陶河。魏尚书仆射杜畿,以帝将幸许,试楼船,覆于陶河,谓此也。昔禹治洪水,观于河,见白面长人,鱼身,出曰:吾河精也。授禹《河图》而还于渊。及子朝篡位,与敬王战,乃取周之宝玉沉河以祈福。后二日,津人得之于河上,将卖之,则变而为石;及敬王位定,得玉者献之,复为玉也。河水又东,溴水入焉。《山海经》曰:和山,上无草木,而多瑶碧,实惟河之九都。是山也,五曲,九水出焉,合而北流,注于河。其阳多苍玉,吉神泰逢司之,是于萯山之阳,出入有光。《吕氏春秋》曰:夏后氏孔甲,田于东阳萯山,遇大风雨,迷惑,人于民室。皇甫谧《帝王世纪》以为即东首阳山也,盖是山之殊目矣。今于首阳东山,无水以应之,当是今古世悬,川域改状矣。昔帝尧修坛河、洛,择良议沈,率舜等升于首山,而遵河渚,有五老游焉,相谓:《河图》将来告帝,以期知我者,重瞳也。五老乃翻为流星而升于昴,即于此也。又东,济水注焉。又东过巩县北,河水于此有五社渡,为五社津。建武元年,朱鲔遣持节使者贾强、讨难将军苏茂,将三万人,从五社津渡,攻温。冯异遣校尉与寇恂合击之,大败,追至河上,生擒万余人,投河而死者数千人。县北有山临河,谓之崟原丘,其下有穴,谓之巩穴,言潜通淮浦,北达于河。直穴有渚,谓之鲔渚。成公子安《大河赋》曰:鱣鲤王鲔,春暮来游。《周礼》,春荐鲔。然非时及他处则无。故河自鲔穴已上,又兼鲔称。《吕氏春秋》称武王代纣至鲔水,纣使胶鬲候周师,即是处矣。
洛水从县西北流注之。
洛水于巩县东径洛汭,北对琅邪渚,入于河,谓之洛口矣。自县西来,而北流注河,清浊异流,皦焉殊别。应玚《灵河赋》曰:资灵川之遐源,出昆仑之神丘,涉津洛之阪泉,播九道于中州者也。
又东过成皋县北,济水从北来注之。
河水自洛口又东,左径平皋县南,又东径怀县南,济水故道之所入,与成皋分河,河水右径黄马坂北,谓之黄马关。孙登之去,杨骏作书与洛中敌人处也。河水又东径旋门坂北,今成皋西大坂者也。升陟此坂,而东趣成皋也。曹大家《东征赋》曰:望河洛之交流,看成皋之旋门者也。河水又东径成皋大伾山下,《尔雅》曰:山一成谓之伾。许慎、吕忱等并以为丘一成也。孔安国以为再成曰怔,亦或以为地名,非也。《尚书·禹贡》曰过洛伾,至大伾者也。郑康成曰:地喉也,沇出伾际矣。在河内修武、武德之界,济沇之水与荥播泽出入自此。然则大伾即是山矣。伾北,即《经》所谓济水从北来注之者也。今济水自温县入河,不于此也。所入者,奉沟水耳,即济沇之故渎矣。成皋县之故城在伾上,萦带伾阜,绝岸峻周,高四十许丈,城张翕险,崎而不平。《春秋传》曰:制,岩邑也,虢叔死焉。即东虢也。鲁襄公二年七月,晋成公与诸侯会于戚,遂城虎牢以逼郑求平也。盖修故耳。《穆天子传》曰:天子射鸟猎兽于郑圃,命虞人掠林。有虎在于葭中,天子将至,七萃之士高奔戎生捕虎而献之天子,命之为柙言之东虢,是曰虎牢矣。然则虎牢之名,自此始也。秦以为关,汉乃县之。城西北隅有小城,周三里,北面列观,临河岧岧孤上。景明中,言之寿春,路值兹邑,升眺清远,势尽川陆,羁途游至,有伤深情。河水南对玉门。昔汉祖与滕公潜出,济于是处也。门东对临河,侧岸有土穴,魏攻北司州刺史毛德祖于虎牢,战经二百日,不克。城惟一井,井深四十丈,山势峻峭,不容防捍,潜作地道取井。余顷因公至彼,故往寻之,其穴处犹存。河水又东合汜水,水南出浮戏山,世谓之曰方山也。北流合东关水。水出嵩渚之山,泉发于层阜之上,一源两枝,分流泻注,世谓之石泉水也。东为索水,西为东关之水。西北流,杨兰水注之,水出非山,西北流注东关水。东关水又西北,清水入焉。水自东浦西流,与东关水合,而乱流注于汜。汜水又北,右合石城水,水出石城山。其山复涧重岭,敧叠若城,山顶泉流,瀑布悬泻,下有滥泉,东流泄注。边有数十石畦,畦有数野蔬。岩侧石窟数口,隐迹存焉,而不知谁所经始也。又东北流注于汜水。汜水又北合鄤水,水西出娄山,至冬则暖,故世谓之温泉。东北流径田鄤谷,谓之田鄤溪水,东流注于汜水。汜水又北径虎牢城东。汉破司马欣、曹咎于是水之上。汜水又北流注于河。《征艰赋》所谓步汜口之芳草,吊周襄之鄙馆者也。余案昔儒之论,周襄所居在颍川襄城县,是乃城名,非为水目,原夫致谬之由,俱以汜郑为名故也,是为爽矣。又案郭缘生《述征纪》,刘澄之《永初记》,并言高祖即帝位于是水之阳,今不复知旧坛所在,卢谌、崔云,亦言是矣。余案高皇帝受天命于定陶汜水,不在此也。于是求坛,故无仿佛矣。河水又东径板城北,有律,谓之板城渚口。河水又东径五龙坞北,坞临长河。有五龙祠。应劭云:昆仑山庙在河南荥阳县。疑即此祠,所未详。又东过荥阳县北,蒗渠出焉。
大禹塞荥泽,开之以通淮、泗,即《经》所谓蒗渠也。汉平帝之世,河、汴决坏,未及得修,汴渠东侵,日月弥广,门闾故处,皆在水中。汉明帝永平十二年,议治渠,上乃引乐浪人王景问水形便。景陈利害,应对敏捷,帝甚善之,乃赐《山海经》、《河渠书》、《禹贡图》及以钱帛。后作堤,发卒数十万,诏景与将作谒者王吴治渠。筑堤防修塌,起自荥阳,东至于乘海口,千有余里。景乃商度地势,凿山开涧,防遏冲要,疏决壅积,十里一水门,更相回注,无复渗漏之患。明年渠成,帝亲巡行,诏滨河郡国置河堤员吏,如西京旧制。景由是显名,王吴及诸从事者,皆增秩一等。顺帝阳嘉中,又自汴口以东,缘河积石,为堰通渠,咸曰金堤。灵帝建宁中,又增修石门,以遏渠口。水盛则通注,津耗则辍流。河水又东北径卷之扈亭北。《春秋左传》曰:文公七年,晋赵盾与诸侯盟于扈。《竹书纪年》,晋出公十二年,河绝于扈,即于是也。河水又东径八激堤北。汉安帝永初七年,令谒者太山于岑,于石门东积石八所,皆如小山,以捍冲波,谓之八激堤。河水又东径卷县北。晋、楚之战,晋军争济,舟中之指可掬,楚庄祀河告成而还,即是处也。河水又东北径赤岸固北,而东北注。又东北过武德县东,沁水从西北来注之。
河水自武德县,汉献帝延康元年,封曹叡为侯国,即魏明帝也。东至酸枣县西,濮水东出焉。汉兴三十有九年,孝文时,河决酸枣,东溃金堤,大发卒塞之。故班固云:文堙枣野,武作《瓠歌》。谓断此口也。今无水。河水又东北,通谓之延津。石勒之袭刘曜,途出于此,以河冰泮为神灵之助,号是处为灵昌津。昔澹台子羽赍千金之壁渡河,阳侯波起,两蛟挟舟,子羽曰:吾可以义求,不可以威劫。操剑斩蛟,蛟死波休。乃投壁于河,三投而辄跃出,乃毁璧而去,示无吝意。赵建武中,造浮桥于津上,采石为中济,石无大小,下辄流去,用工百万,经年不就。石虎亲阅作工,沉璧于河,明日,壁流渚上,波荡上岸,遂斩匠而还。河水又径东燕县故城北,河水于是有棘津之名,亦谓之石济津,故南津也。《春秋》僖公二十八年,晋将伐曹,曹在卫东,假道于卫,卫人不许,还自南河济,即此也。晋伐陆浑,亦于此渡。宋元嘉中,遣辅国将军萧斌率宁朔将军王玄谟北入,宣威将军垣护之,以水军守石济,即此处也。河水又东,淇水入焉。又东径遮害亭南。《汉书·沟洫志》曰:在淇水口东十八里,有金堤,堤高一丈。自淇口东,地稍下,堤稍高,至遮害亭,高四五丈。又有宿胥口,旧河水北入处也。河水又东右径滑台城北。城有三重,中小城谓之滑台城,旧传滑台人自修筑此城,因以名焉。城即故郑廪延邑也。下有延津。《春秋传》曰孔悝为蒯聩所逐,载伯姬于平阳,行于延津是也。廪延南故城,即卫之平阳亭也。今时人谓此津为延寿津。宋元嘉中,右将军到彦之,留建成将军朱修之守此城。魏军南伐,修之执节不下,其母悲忧,一旦乳汁惊出,母乃号踊,告家人曰:我年老,非有乳时,今忽如此,吾儿必没矣,修之绝援,果以其日陷没。城,故东郡治。《续汉书》曰:延熹九年,济阴、东郡、济北、平原河水清。襄楷上疏曰:《春秋》注记未有河清,而今有之。《易乾凿度》曰:上天将降嘉应,河水先清。京房《易传》曰:河水清,天下平,天垂异,地吐妖,民厉疾,三者并作而有何清。《春秋》麟不当见而见,孔子书以为异。河者,诸侯之象,清者,阳明之征,岂独诸侯有窥京师也?明年,宫车宴驾,征解渎侯为汉嗣,是为灵帝。建宁四年二月,河水又清也。
又东北过黎阳县南,黎,侯国也。《诗·式微》,黎侯寓于卫是也。晋灼曰:黎山在其南,河水径其东。其山上碑云:县取山之名,取水之阳,以为名也。王莽之黎蒸也。今黎山之东北故城,盖黎阳县之故城也。山在城西,城凭山为基,东阻于河。故刘桢《黎阳山赋》曰:南荫黄河,左覆金城,青坛承祀,高碑颂灵。昔慕容玄明自邺率众南徙滑台,既无舟楫,将保黎阳,昏而流澌,冰合于夜中,济讫,旦而冰泮,燕民谓是处为天桥津。东岸有故城,险带长河。戴延之谓之逯明垒,周二十里,言逯明,石勒十八骑中之一,城因名焉。郭缘生曰:城,袁绍时筑。皆非也。余案《竹书纪年》,梁惠成王十一年,郑董侯使许息来致地,平丘、户牖、首垣诸邑,及郑驰道,我取枳道与郑鹿,即是城也。今城内有故台,尚谓之鹿鸣台,又谓之鹿鸣城。王玄谟自滑台走鹿鸣者也。济取名焉,故亦曰鹿鸣津,又曰白马济。津之东南有白马城,卫文公东徙,渡河都之,故济取名焉。袁绍遣颜良攻东郡太守刘延于白马,关羽为曹公斩良以报效,即此处也。白马有韦乡、韦城,故津亦有韦津之称。《史记》所谓下修武,渡韦津者也。河水旧于白马县南泆通濮、济、黄沟,故苏代说燕曰:决白马之口,魏无黄济阳。《竹书纪年》,梁惠成王十二年,楚师出河水,以水长垣之外者也。金堤既建,故渠水断,尚谓之白马渎,故渎东径鹿鸣城南,又东北径白马县之凉城北。《耆旧传》云:东郡白马县之神马亭,实中层峙,南北二百步,东西五十许步,状丘斩城也。自外耕耘垦斫,削落平尽,正南有躔陛陟上,方轨是由。西南侧城有神马寺,树木修整。西去白马津可二十许里,东南距白马县故城可五十里,疑即《开山图》之所谓白马山也。山下常有白马群行,悲鸣则河决,驰走则山崩。《注》云:山在郑北,故郑也,所未详。刘澄之云:有白马塞,孟达登之长叹。可谓于川土疏妄矣。亭上旧置凉城,县治此。白马渎又东南径濮阳县,散入濮水,所在决会,更相通注,以成往复也。河水自津东北径凉城县。河北有般祠。《孟氏记》云:祠在河中,积石为基,河水涨盛,恒与水齐。戴氏《西征记》曰:今见祠在东岸,临河累石为壁,其屋宇容身而已。殊似无灵,不如孟氏所记,将恐言之过也。河水又东北,径伍子胥庙南,祠在北岸顿丘郡界,临侧长河,庙前有碑,魏青龙三年立。河水又东北为长寿津。《述征记》曰:凉城到长寿津六十里,河之故渎出焉,《汉书·沟洫志》曰:河之为中国害尤甚,故导河自积石,历龙门二渠以引河。一则漯川,今所流也;一则北渎,王莽时空,故世俗名是渎为王莽河也。故渎东北径戚城西。《春秋》哀公二年,晋赵鞅率师,纳卫太子蒯聩于戚,宵迷,阳虎曰:右河而南必至焉。今顿丘卫国县西戚亭是也,为卫之河上邑。汉高帝十二年,封将军李必为侯国矣。故渎又径繁阳县故城东。《史记》,赵将廉颇代魏取繁阳者也。北径阴安县故城西。汉武帝元朔五年,封卫不疑为侯国。故渎又东北径乐昌县故城东。《地理志》,东郡之属县也,汉宣帝封王稚君为侯国。故渎又东北径平邑郭西。《竹书纪年》,晋烈公二年,赵城平邑;五年,田公子居恩伐邯郸,围平邑;九年,齐田肸及邯郸韩举,战于平邑,邯郸之师败。逋获韩举,取平邑新城。又东北径元城县故城西北,而至沙丘堰。《史记》曰:魏武侯公子元食邑于此,故县氏焉。郭东有五鹿墟,墟之左右多陷城。《公羊》曰:袭,邑也。说曰:袭,陷矣。《郡国志》启:五鹿,故沙鹿,有沙亭,周穆王丧盛姬,东征舍于五鹿,其女叔届此思哭,是曰女之丘,为沙鹿之异名也。《春秋左传》僖公十四年,沙鹿崩。晋史卜之曰:阴为阳雄,土火相乘,故有沙鹿崩。后六百四十五年,宜有圣女兴,其齐田乎?后王翁孺自济南徙元城,正直其地,日月当之。王氏为舜后,土也,汉火也。王禁生政君,其母梦见月入怀,年十八,诏入太子宫,生成帝,为元后。汉祚道污,四世称制,故曰火土相乘而为雄也。及崩,大夫扬雄作讳曰:太阴之精,沙鹿之灵,作合于汉,配元生成者也。献帝建安中,袁绍与曹操相御于官渡,绍逼大司农郑玄载病随军,届此而卒。郡守已下受业者,衰绖赴者千余人。玄注《五经》、谶纬、候历、天文经通于世,故范晔《赞》曰:孔书遂明,汉章中辍矣,县北有沙丘堰,堰障水也。《尚书·禹贡》曰:北过降水。不遵其道曰降,亦曰溃,至于大陆北播为九河。《风俗通》曰河播也,播为九河自此始也。《禹贡》沇州,九河既道,谓徒骇、太史、马颊、覆釜、胡苏、简、洁、句盘、鬲津也,同为逆河。郑玄曰:下尾合曰逆河。言相迎受矣。盖疏润下之势,以通河海。及齐桓霸世,塞广田居,同为一河。故自堰以北,馆陶、廮陶、贝丘、鬲般、广川、信都、东光、河间、乐成以东,城地并存,川渎多亡。汉世河决,金堤南北离其害,议者常欲求九河故迹而穿之,未知其所。是以班固云:自兹距汉北,亡八枝者也。河之故渎,自沙丘堰南分,屯氏河出焉。河水故渎东北径发于县故城西,又屈径其北。王莽之所谓戢楯矣。汉武帝以大将军卫青破右贤王功,封其子登为侯国。大河故渎又东径贝丘县故城南。应劭曰:《左氏传》,齐襄公田于贝丘是也。余案京相璠、杜预并言在博昌,即司马彪《郡国志》所谓贝中聚者也。应《注》于此事近违矣,大河故渎又东径甘陵县故城南。《地理志》之所谓盾也,王莽改曰厝治者也。汉安帝父孝德皇,以太子被废为王,薨于此,乃葬其地,尊陵曰甘陵,县亦取名焉。桓帝建和二年,改清河曰甘陵。是周之甘泉市地也。陵在渎北,丘坟高巨,虽中经发坏,犹若层陵矣,世谓之唐侯冢。城曰邑城,皆非也。昔南阳文叔良,以建安中为甘陵丞,夜宿水侧,赵人兰襄梦求改葬,叔良明循水求棺,果于水侧得棺,半许落水。叔良顾亲旧曰:若闻人传此,吾必以为不然。遂为移殡,醊而去之。大河故渎又东径艾亭城南,又东径平晋城南。今城中有浮图五层,上有金露盘,题云:赵建武八年,比释道龙和上竺浮图澄,树德劝化,兴立神庙。浮图已坏,露盘尚存,炜炜有光明。大河故渎又东北径灵县故城南。王莽之播亭也。河水于县别出为鸣犊河。河水故渎又东径鄃县故城东。吕后四年,以父婴功,封子佗袭为侯国,王莽更名之曰善陆。大河故渎又东径平原县故城西,而北绝屯氏三渎,北径绎幕县故城东北,西流径平原鬲县故城西。《地理志》曰:鬲,津也。王莽名之曰河平亭,故有穷后羿国也。应劭曰:鬲,偃姓,咎繇后。光武建武十三年,封建义将军朱祜为侯国。大河故渎又北径修县故城东,又北径安陵县西。本修之安陵乡也。《地理风俗记》曰:修县东四十里有安陵乡,故县也。又东北至东光县故城西,而北与漳水合。一水分大河故渎,北出为屯氏河,径馆陶县东,东北出。《汉书·沟洫志》曰:自塞宣防,河复北决于馆陶县,分为屯氏河,广深与大河等。成帝之世,河决馆陶及东郡金堤,上使河堤谒者王延世塞之,三十六日堤成,诏以建始五年为河平元年,以延世为光禄大夫。是水亦断。屯氏故渎水之又东北,屯氏别河出焉。屯氏别河故渎又东北径信成县,张甲河出焉。《地理志》,张甲河首受屯氏别河于信成县者也。张甲河故渎北绝清河于广宗县,分为二渎,左渎径广宗县故城西,又北径建始县故城东。田融云:赵武帝十二年,立建兴郡,治广宗,置建始,兴德五县隶焉。左渎又北径经城东、缭城西,又径南宫县西,北注绛渎。右渎东北径广宗县故城南,又东北径界城亭北,又东北径长乐郡枣强县故城东。长乐,故信都也,晋太康五年,改从今名,又东北径广川县,与绛渎水故道合。又东北径广川县故城西,又东径棘津亭南。徐广曰:棘津在广川。司马彪曰:县北有棘津城,吕尚卖食之困,疑在此也。刘澄之云:谯郡酇县东北有棘津亭,故邑也,吕尚所困处也。余案《春秋左传》,伐巢、克棘、入州来,无津字。杜预《春秋释地》又言,棘亭在酇县东北,亦不云有津字矣。而竟不知澄之于何而得是说?然天下以棘为名者多,未可咸谓之棘津也。又《春秋》昭公十七年,晋侯使荀吴帅师涉自棘津,用牲于洛,遂灭陆浑。杜预《释地》阙而不书。服虏曰:棘津,犹孟津也。徐广《晋纪》又言,石勒自葛陂寇河北,袭汲人向冰于枋头,济自棘。棘津在东郡、河内之间,田融以为即石济南津也。虽千古茫昧,理世玄远,遗文逸句,容或可寻,沿途隐显,方土可验。司马迁云:吕望,东海上人也,老而无遇,以钓干周文王。又云:吕望行年五十,卖食棘津;七十,则屠牛朝歌;行年九十,身为帝师。皇甫士安云:欲隐东海之滨,闻文王善养老,故入钓于周。今汲水城亦言有吕望隐居处。起自东海,迄于酆雍,缘其径趣,赵、魏为密,厝之谯、宋,事为疏矣。张甲故渎又东北至修县东会清河。《十三州志》曰:张甲河东北至修县入清漳者也。屯氏别河又东,枝津出焉,东径信成县故城南,又东径清阳县故城南,清河郡北,魏自清阳徙置也。又东北径陵乡南,又东北径东武城县故城南,又东北径东阳县故城南。《地理志》曰王莽更之曰胥陵矣,俗人谓之高黎郭,非也。应劭曰:东武城东北三十里有阳乡,故县也。又东散绝,无复津径。屯氏别河又东北径清河郡南,又东北径清河故城西。汉高帝六年,封王吸为侯国。《地理风俗记》曰:甘陵郡东南十七里有清河故城者,世谓之鹊城也。又东北径绎幕县南,分为二渎:屯氏别河北渎东径绎幕县故城南,东绝大河故渎,又东北径平原县,枝津北出,至安陵县遂绝。屯氏别河北渎又东北径重平县故城南。应劭曰:重合县西南八十里有重平乡,故县也。又东北径重合县故城南,又东北径定县故城南。汉武帝元朔四年,封齐孝王子刘越为侯国。《地理风俗记》曰:饶安县东南三十里有定乡城,故县也。屯氏别河北渎又东入阳信县,今无水。又东为咸河,东北流径阳信县故城北。《地理志》,渤海之属县也。东注于海。屯氏别河南渎自平原东绝大河故渎,又径平原县故城北,枝津右出,东北至安德县界,东会商河。屯氏别河南渎又东北于平原界,又有枝渠右出,至安德县遂绝。屯氏别河南读自平原城北首受大河故渎,东出,亦通谓之笃马河,即《地理志》所谓平原县有笃马河,东北入海,行五百六十里者也,东北径安德县故城西,又东北径临齐城南。始东齐未宾,大魏筑城以临之,故城得其名也。又屈径其城东,故读广四十步,又东北径重丘县故城西。《春秋》襄公二十五年,秋,同盟于重丘,伐齐故也。应劭曰,安德县北五十里有重丘乡,故县也。又东北径西平昌县故城北。北海有平昌县,故加西。汉宣帝元康元年,封王长君为侯国。故渠川派,东入般县为般河。盖亦九河之一道也。《后汉书》称公孙瓒破黄巾于般河,即此渎也。又东为白鹿渊水,南北三百步,东西千余步,深三丈余。其水冬清而夏浊,渟而不流,若夏水洪泛,水深五丈,方乃通注。般渎又径般县故城北,王莽更之曰分明也。东径乐陵县故城北。《地理志》曰:故都尉治。伏琛、晏谟言平原邑,今分为郡。又东北径阳信县故城南,东北入海。屯氏河故渎自别河东径甘陵之信乡县故城南。《地理志》曰:安帝更名安平。应劭曰:甘陵西北十七里有信乡,故县也。屯氏故渎又东径甘陵县故城北,又东径灵县北,又东北径鄃县,与鸣犊河故渎合,上承大河故渎于灵县南。《地理志》曰:河水自灵县别出为鸣犊河者也。东北径灵县东,东入鄃县,而北合屯氏渎。屯氏渎兼鸣犊之称也。又东径鄃县故城北,东北合大河故渎,谓之鸣犊口。《十三州志》曰:鸣犊河东北至修入屯氏,考渎则不至也。
又东北过卫县南,又东北过濮阳县北,瓠子河出焉。河水东径铁丘南。
《春秋左氏传》哀公二年,郑罕达帅师,邮无恤御简子,卫太子为右,登铁上,望见郑师,卫太子自投车下,即此处也。京相璠曰:铁,丘名也。杜预曰:在戚南河之北岸,有古城,戚邑也。东城有子路冢,河之西岸有竿城。《郡国志》曰:卫县有竿城者也。河南有龙渊宫,武帝元光中,河决濮阳、汜郡十六,发卒十万人塞决河,起龙渊宫。盖武帝起宫于决河之傍,龙渊之侧,故曰龙渊宫也。河水东北流而径濮阳县北,为濮阳津。故城在南,与卫县分水。城北十里有瓠河口,有金堤、宣房堰。粤在汉世,河决金堤,涿郡王尊,自徐州刺史迁东郡太守。河水盛溢,泛浸瓠子,金堤决坏,尊躬率民吏,投沉白马,祈水神河伯,亲执圭璧,请身填堤,庐居其上,民吏皆走,尊立不动而水波齐足而止,公私壮其勇节。河水又东北径卫国县南,东为郭口津。河水又东径鄄城县北。故城在河南十八里,王莽之鄄良也,沇州旧治。魏武创业始自于此。河上之邑最为峻固。《晋八王故事》曰:东海王越治鄄城,城无故自坏七十余丈,越恶之,移治濮阳。城南有魏使持节、征西将军、太尉、方城侯邓艾庙,庙南有艾碑。秦建元十二年,广武将军、沇州刺史、关内侯安定彭超立。河之南岸有新城。宋宁朔将军王玄谟前锋入河所筑也。北岸有新台。鸿基层广高数丈,卫宣公所筑新台矣,《诗》齐姜所赋也。为卢关津。台东有小城,崎岖颓侧,台址枕河,俗谓之邸阁城。疑故关津都尉治也,所未详矣。河水又东北径范县之秦亭西。《春秋经》书筑台于秦者也。河水又东北径委粟津,大河之北,即东武阳县也。左会浮水故渎,故渎上承大河于顿丘县而北出,东径繁阳县故城南。应劭曰:县在繁水之阳。张晏曰:县有繁渊,《春秋》襄公二十年,《经》书公与晋侯、齐侯盟于澶渊。杜预曰:在顿丘县南,今名繁渊。澶渊,即繁渊也,亦谓之浮水焉。昔魏徙大梁,赵以中牟易魏。故《志》曰:赵南至浮水繁阳。即是渎也。故渎东绝大河,故渎东径五鹿之野。晋文公受块于野人,即此处矣。京相璠曰:今卫县西北三十里,有五鹿城,今属顿丘县。浮水故渎又东南径卫国邑城北。故卫公国也,汉光武以封周后也。又东径卫国县故城南。古斟观。应劭曰:夏有观扈,即此城也。《竹书纪年》,梁惠成王二年,齐田寿率师伐我,围观,观降。浮水故渎又东径河牧城而东北出。《郡国志》曰:卫本观故国,姚姓,有河牧城。又东北入东武阳县,东入河,又有漯水出焉,戴延之谓之武水也。河水又东径武阳县东、范县西,而东北流也。
又东北过东阿县北,河水于范县东北流为仓亭津。《述征记》曰:仓亭津在范县界,去东阿六十里。《魏土地记》曰:津在武阳县东北七十里。津,河济名也。河水右历柯泽。《春秋左传》襄公十四年,卫孙文子败公徒于阿泽者也。又东北径东阿县故城西,而东北出流注。河水枝津东出,谓之邓里渠也。
又东北过茌平县西,河自邓里渠东北径昌乡亭北,又东北径碻磝城西。《述证记》曰:碻磝,津名也,自黄河泛舟而渡者,皆为津也。其城临水,西南崩于河。宋元嘉二十七年,以王玄谟为宁朔将军,前锋入河,平碻磝,守之。都督刘义恭以沙城不堪守,召玄谟令毁城而还,后更城之。魏立济州,治此也。河水冲其西南隅,又崩于河,即故茌平县也。应劭曰:茌,山名也。县在山之平地,故曰茌平也,王莽之功崇矣。《经》曰大河在其西,邓里渠历其东。即斯邑也。昔石勒之隶师欢,屯耕于茌平,闻鼓角碑铎之声,于是县也。西与聊城分河。河水又东北与邓里渠合,水上承大河于东阿县西,东径东阿县故城北。故卫邑也。应仲瑗曰,有西,故称东。魏封曹植为王国。大城北门内西侧,皋上有大井,其巨若轮,深六七丈,岁尝煮胶,以贡天府。《本草》所谓阿胶也。故世俗有阿井之名。县出佳缯缣,故《史记》云秦昭王服太阿之剑,阿缟之衣也。又东北径临邑县,与将渠合。又北径茌平县东,临邑县故城西,北流入于河。河水又东北流径四渎津,津西侧岸,临河有四渎祠,东对四渎口。河水东分济,亦曰济水受河也。然荥口石门水断不通,始自是出东北流,径九里与清水合。故济渎也。自河入济,自济入淮,自淮达江,水径周通,故有四渎之名也。昔赵杀鸣犊,仲尼临河而叹。自是而返曰,丘之不济,命也。夫《琴操》以为孔子临狄水而歌矣。曰:狄水衍兮风扬波,船揖颠倒更相加。余案临济,故狄也。是济所径,得其通称也。河水又径杨墟县之故城东,俗犹谓是城曰阳城矣。河水又径茌平城东,疑县徒也。城内有故台,世谓之时平城,非也。盖在、时音相近耳。
又东北过高唐县东,河水于县,漯水注之。《地理志》曰:漯水出东武阳。今漯水上承河水于武阳县东南,西北径武阳新城东。曹操为东郡所治也。引水自东门石窦北注于堂池,他南故基尚存。城内有一石甚大,城西门名冰井门,门内曲中,冰井犹存。门外有故台,号武阳台,匝台亦有隅雉遗迹。水自城东北径东武阳县故城南。应劭曰,县在武水之阳,王莽之武昌也,然则漯水亦或武水矣。臧洪为东郡太守,治此。曹操围张超于雍丘,洪以情义,请袁绍救之,不许,洪与绍绝。绍围洪,城中无食,洪呼吏士曰:洪于大义,不得不死,诸君无事,空与此祸。众泣曰:何忍舍明府也。男女八千余人,相枕而死。洪不屈,绍杀洪。邑人陈容为丞,谓曰:宁与臧洪同日死,不与将军同日生。绍又杀之,士为伤叹。今城四周,绍围郭尚存。水匝隍堑,于城东北合为一渎,东北出郭,径阳平县之冈成城西。《郡国志》曰:阳平县有冈成亭。又北径阳平县故城东。汉昭帝元平元年,封丞相蔡义为侯国。漯水又北绝莘道城之西。北有莘亭。《春秋》桓公十六年,卫宣公使伋使诸齐,令盗待于莘,伋、寿继殒于此亭。京相璠曰:今平原阳平县北十里,有故莘亭,阨限蹊要,自卫适齐之道也。望新台于河上,感二子于夙龄,诗人乘舟,诚可悲矣。今县东有二子庙,犹谓之为孝祠矣。漯水又东北径乐平县故城东。县,故清也。汉高帝八年,封窒中同于清,宣帝封许广汉少弟翁孙于乐平,并为侯国。王莽之清治矣,汉章帝建初中,更从今名也。漯水又北径聊城县故城西。城内有金城,周匝有水,南门有驰道,绝水南出,自外泛舟而行矣。东门侧有层台,秀出云表,鲁仲连所谓还高唐之兵,却聊城之众者也。漯水又东北径清河县故城北。《地理风俗记》曰:甘陵,故清河。清河在南十七里,今于甘陵县故城东南,无城以拟之。直东二十里有艾亭城,东南四十里有此城,拟即清河城也。后蛮居之,故世称蛮城也。漯水又东北径文乡城东南,又东北径博平县故城南。城内有层台秀上,王莽改之曰加睦也。右与黄沟同注川泽。黄沟承聊城郭水,水泛则津注,水耗则辍流。自城东北出,径清河城南,又东北径摄城北。《春秋》所谓聊摄以东也。俗称郭城,非也。城东西三里,南北二里,东西隅有金城,城卑下,墟郭尚存,左右多坟垅。京相璠曰:聊城县东北三十里有故摄城,今此城西去聊城二十五、六里许,即摄城者也。又东径文乡城北,又东南径王城北。魏太常七年,安平王镇平原所筑,世谓之王城。太和二十三年,罢镇立平原郡,治此城也。黄沟又东北流,左与漯水隐覆,势镇河陆,东出于高唐县,大河右边,东注漯水矣。桑钦《地理志》曰:漯水出高唐。余按《竹书》、《穆天子传》称,丁卯,天子自五鹿东征,钓于漯水,以祭淑人,是曰祭丘;己巳,天子东征,食马于漯水之上。寻其沿历径趣,不得近出高唐也。桑氏所言,盖津流所出,次于是间也。俗以是水上承于河,亦谓之源河矣。漯水又东北径援县故城西。王莽之东顺亭也。杜预《释地》曰:济南祝阿县西北有援城。漯水又东北径高唐县故城东。昔齐威王使肸子守高唐,赵人不敢渔于河,即鲁仲连子谓田巴曰:今楚军南阳,赵伐高唐者也。《春秋左传》哀公十年,赵鞅帅师伐齐,取犁及辕,毁高唐之郭。杜预曰:辕即援也。祝阿县西北有高唐城。漯水又东北径漯阴县故城北。县,故犁邑也,汉武帝元光三年封匈奴降王,王莽更名翼成。历北漯阴城南。伏琛谓之漯阳,城南有魏沇州刺史刘岱碑。《地理风俗记》曰:平原漯阴县,今巨漯亭是也。漯水又东北径著县故城南,又东北径崔氏城北。《春秋左传》襄公二十七年,崔成请老于崔者也。杜预《释地》曰:济南东朝阳县西北有崔氏城。漯水又东北径东朝阳县故城南。汉高帝七年,封都尉宰寄为侯国。《地理风俗记》曰:南阳有朝阳县,故加东。《地理志》曰:王莽之修治也。漯水又东径汉征君伏生墓南。碑碣尚存,以明经为秦博士。秦坑儒士,伏生隐焉。汉兴,教于齐、鲁之间,撰《五经》、《尚书大传》,文帝安车征之。年老不行,乃使掌故欧阳生等受《尚书》于征君,号曰伏生者也。漯水又东径邹平县故城北。古邹侯国,舜后姚姓也。又东北径东邹城北。《地理志》,千乘郡有东邹县。漯水又东北径建信县故城北。汉高帝七年,封娄敬为侯国。应劭曰:临济县西北五十里有建信城,都尉治故城者也。漯水又东北径千乘县二城间。汉高帝六年,以为千乘郡,王莽之建信也。章帝建初四年为王国,和帝永元七年,改为乐安郡。故齐地。伏琛曰:千乘城在齐城西北百五十里,隔会水,即漯水之别名也。又东北为马常坑,坑东西八十里,南北三十里,乱河枝流而入于海。河、海之饶,兹焉为最。《地理风俗记》曰:漯水东北至千乘入海,河盛则通津委海,水耗则微涓绝流。《书》浮于济、漯,亦是水者也。
又东北过杨虚县东,商河出焉。
《地理志》:杨虚,平原之隶县也。汉文帝四年,以封齐悼惠王子将闾为侯国也。城在高唐城之西南,《经》次于此,是不比也。商河首受河水,亦漯水及泽水所潭也。渊而不流,世谓之清水。自此虽沙涨填塞,厥迹尚存。历泽而北,俗谓之落里坑。径张公城西,又北,重源潜发,亦曰小漳河。商、漳声相近,故字与读移耳。商河又北径平原县东,又径安德县故城南,又东北径平昌县故城南,又东径般县故城南,又东径乐陵县故城南。汉宣帝地节四年,封侍中史子长为侯国。商河又东径朸县故城南。高后八年,封齐悼惠王子刘辟光为侯国,王莽更之曰张乡。应劭曰:般县东南六十里有朸乡城,故县也。沙沟水注之,水南出大河之阳,泉源之不合河者二百步,其水北流注商河,商河又东北流径马岭城西北,屈而东注南转,径城东。城在河曲之中,东海王越斩汲桑于是城。商河又东北径富平县故城北。《地理志》曰:侯国也。王莽曰乐安亭。应劭曰:明帝更名厌次。阚駰曰:厌次县本富平侯、车骑将军张安世之封邑。非也。案《汉书》,昭帝元凤六年,封右将军张安世为富平侯。薨,子延寿嗣国,在陈留别邑,在魏郡。《陈留风俗传》曰:陈留尉氏县安陵乡,故富平县也,是乃安世所食矣。岁入租千余万,延寿自以身无功德,何堪久居先人大国,上书请减户。天子以为有让,徙封平原,并食一邑,户口如故,而税减半。《十三州志》曰:明帝水平五年,改曰厌次矣。案《史记·高祖功臣侯者年表》,高帝六年,封元顷为侯国。徐广《音义》曰:《汉书》作爱类。是知厌次旧名,非始明帝,盖复故耳。县西有东方朔冢,冢侧有祠,祠有神验。水侧有云城,汉武帝元封四年,封齐孝王子刘信为侯国也。商河又分为二水,南水谓之长丛沟,东流倾注于海。沟南海侧,有蒲台,台高八丈,方二百步。《三齐略记》曰:鬲城东南有蒲台,秦始皇东游海上,于台上蟠蒲系马,至今每岁蒲生,萦委若有系状,似水杨,可以为箭。今东去海三十里。北水世又谓之百薄渎,东北流注于海水矣。大河又东北径高唐县故城西。《春秋左传》襄公十九年,齐灵公废太子光而立公子牙,以夙沙卫为少傅。齐侯卒,崔杼逆光,光立,杀公子牙于句渎之丘,卫奔高唐以叛。京相璠曰:本平原县也,齐之西鄙也。大河径其西而不出其东,《经》言出东,误耳。大河又北径张公城。临侧河湄,卫青州刺史张治此,故世谓之张公城。水有津焉,名之曰张公渡。河水又北径平原县故城东。《地理风俗记》曰:原,博平也,故曰平原矣。县,故平原郡治矣。汉高帝六年置,王莽改曰河平也。晋灼曰:齐西有平原。河水东北过高唐,高唐,即平原也。故《经》言,河水径高唐县东。非也。按《地理志》曰:高唐漂水所出,平原则笃马河导焉,明平原非高唐,大河不得出其东审矣。大河右溢,世谓之甘枣沟。水侧多枣,故俗取名焉。河盛则委泛,水耗则辍流。故沟又东北历长堤,径漯阴里北,东径著城北,东为陂淀渊潭相接,世谓之秽野薄。河水又东北径阿阳县故城西。汉高帝六年,封郎中万訢为侯国。应劭曰:漯阴县东南五十里有阿阳乡,故县也。
又东北过漯阳县北,河水自平原左径安德城东,而北为鹿角津。东北径般县、乐陵、朸乡至厌次县故城南,为厌次河。汉安帝永初二年,剧贼毕豪等数百乘船寇平原,县令刘雄,门下小吏所辅,浮舟追至厌次津,与贼合战,并为贼擒。求代雄,豪纵雄于此津。所辅可谓孝尽爱敬,义极君臣矣。河水右径漯阴县故城北,王莽之巨武县也。河水又东北为漯沃津,在漯沃县故城南,王莽之延亭者也。《地理风俗记》曰:千乘县西北五十里有大河,河北有漯沃城,故县也。魏改为后部亭,今俗遂名之曰右辅城。河水又东径千乘城北。伏琛之所谓千乘北城者也。
又东北过利县北,又东北过甲下邑,济水从西来注之。又东北入于海。
河水又东分为二水,枝津东径甲下城南,东南历马常坑注济。《经》言济水注河,非也。河水自枝津东北流,径甲下邑北,世谓之仓子城。又东北流,入于海。《淮南子》曰:九折注于海,而流不绝者,昆仑之输也。《尚书·禹贡》曰:夹右碣石入于河。《山海经》曰:碣石之山,绳水出焉,东流注于河。河之入海,旧在碣石,今川流所导,非禹渎也。周定王五年,河徙故渎。故班固曰:商竭,周移也。又以汉武帝元光二年,河又徙东郡,更注渤海。是以汉司空掾王璜言曰:往者,天尝连雨,东北风,海水溢,西南出侵数百里。故张折云:碣石在海中。盖沦于海水也。昔燕、齐辽旷,分置营州,今城届海滨,海水北侵,城垂沦者半。王璜之言,信而有征;碣石入海,非无证矣。


卷六 汾水、浍水、涑水、文水、原公水、洞过水、晋水、湛水 
汾水出太原汾阳县北管涔山。
《山海经》曰:《北次二经》之首,在河之东,其首枕汾,曰管涔之山,其上无木,而下多玉,汾水出焉,西流注于河。《十三州志》曰:出武州之燕京山。亦管涔之异名也。其山重阜修岩,有草无木,泉源导于南麓之下,盖稚水蒙流耳。又西南,夹岸连山,联峰接势。刘渊族子曜尝隐避于管涔之山,夜中忽有二童子入,跪曰:管涔王使小臣奉谒赵皇帝。献剑一口,置前,再拜而去。以烛视之,剑长二尺,光泽非常,背有铭曰:神剑御,除众毒。曜遂服之,剑随时变为五色也。后曜遂为胡王矣。汾水又南,与东、西温溪合。水出左右近溪,声流翼注,水上杂树交荫,云垂烟接。自是水流潭涨,波襄转泛。又南径一城东。凭墉积石,侧枕汾水,俗谓之代城。又南出二城间。其城角倚,翼枕汾流,世谓之侯莫干城。盖语出戎方,传呼失实也。汾水又南径汾阳县故城东,川土宽平,峘山夷水。《地理志》曰:汾水出汾阳县北山,西南流者也。汉高帝十一年,封靳强为侯国,后立屯农,积粟在斯,谓之羊肠仓。山有羊肠坂,在晋阳西北,石隥萦行,若羊肠焉,故仓坂取名矣。汉永平中,治呼沱、石臼河。案司马彪《后汉郡国志》,常山南行唐县有石臼谷,盖资承呼沱之水,转山东之漕,自都虑至羊肠仓,将凭汾水以漕太原。用实秦、晋,苦役连年,转运所经,凡三百八十九隘,死者无算。拜邓训为谒者,监护水功。训隐括知其难,立具言肃宗,肃宗从之,全活数千人。和熹邓后之立,叔父陔以为训积善所致也。羊肠即此仓也。又南径秀容城东。《魏土地记》曰:秀容,胡人徙居之,立秀容护军治。东去汾水六十里。南与酸水合,水源西出少阳之山,东南流注于汾水。汾水又南出山,东南流,洛阴水注之。水出新兴郡,西流径洛阴城北,又西径盂县故城南。《春秋左传》昭公二十八年,分祁氏七县为大夫之邑,以盂丙为盂大夫。洛阴水又西,径狼孟县故城南。王莽之狼调也。左右夹涧幽深,南面大壑,俗谓之狼马涧,旧断涧为城,有南、北门,门闉故壁尚在。洛阴水又西南径阳曲城北。《魏土地记》曰:阳曲,胡寄居太原界,置阳曲护军治。其水西南流,注于汾水。汾水又南径阳曲城西南注也。
东南过晋阳县东,晋水从县南东流注之。
太原郡治晋阳城。秦庄襄王三年立。《尚书》所谓既修太原者也。《春秋说题辞》曰:高平曰太原。原,端也,平而有度。《广雅》曰:大卤,太原也。《释名》曰:地不生物曰卤。卤,罏也。《谷梁传》曰:中国曰太原,夷狄曰太卤。《尚书大传》曰:东原底平,大而高平者谓之太原,郡取称焉。《魏土地记》曰:城东有汾水南流,水东有晋使持节、都督并州诸军事、镇北将军太原成王之碑。水上旧有梁,青荓殒于梁下,豫让死于津侧,亦襄子解衣之所在也。汾水西径晋阳城南。旧有介子推词,祠前有碑,庙宇倾颓,惟单碑独存矣。今文字剥落,无可寻也。
又南,洞过水从东来注之。汾水又南径梗阳县故城东。故榆次之梗阳乡也。魏献子以邑大夫魏戊也。京相璠曰:梗阳,晋邑也。今太原晋阳县南六十里榆次界有梗阳城。汾水又南,即洞过水会者也。
又南过大陵县东,昔赵武灵王游大陵,梦处女鼓琴而歌,想见其人,吴广进孟姚焉,即于此县也。王莽改曰大宁矣。汾水于县左迤为邬泽。《广雅》曰:水自汾出为汾陂。其陂东西四里,南北十余里,陂南接邬。《地理志》曰:九泽在北,并州薮也。《吕氏春秋》谓之大陆。又名之曰沤洟之泽,俗渭之邬城泊。许慎《说文》曰:漹水出西河中阳县北沙,南入河。即此水也。漹水又会婴侯之水,《山海经》称谒戾之山,婴侯之水出于其阴,北流注于祀水。水出祀山,其水殊源共舍,注于婴侯之水,乱流径中都县南,俗又谓之中都水,侯甲水注之。水发源祁县胡甲山,有长坂,谓之胡甲岭,即刘歆《遂初赋》所谓越侯甲而长驱者也。蔡邕曰:侯甲,亦邑名也,在祁县。侯甲水又西北历宜岁郊,径太谷,谓之太谷水。出谷西北流,径祁县故城南,自县连延,西接邬泽,是为祁薮也。即《尔雅》所谓昭余祁矣。贾辛邑也。辛貌丑,妻不为言,与之如皋射雉,双中之则笑也。王莽之示县也。又西径京陵县故城北。王莽更名曰致城矣。于《春秋》为九原之地也。故《国语》曰:赵文子与叔向游于九原,曰:死者若可作也,吾谁与归?叔向曰:其阳子乎?文子曰:夫阳子行并植于晋国,不免其身,智不足称。叔向曰:其舅犯乎?文子曰:夫舅犯见利不顾其君,仁不足称。吾其随会乎?纳谏不忘其师,言身不失其友,事君不援而进,不阿而退。其故京尚存。汉兴,增陵于其下,故曰京陵焉。侯甲水又西北径中都县故城南,城临际水湄。《春秋》昭公二年,晋侯执陈无宇于中都者也。汉文帝为代王,都此。武帝元封四年,上幸中都宫,殿上见光,赦中都死罪以下。侯甲水又西,合于婴侯之水,径邬县故城南,晋大夫司马弥牟之邑也。谓之邬水,俗亦曰虑水。虑、邬声相近,故因变焉。又西北入邬陂,而归于汾流矣。
又南过平陶县东,文水从西来流注之。
汾水又南与石桐水合,即绵水也。水出界休县之绵山,北流径石桐寺西。即介子推之祠也。昔子推逃晋文公之赏,而隐于绵上之山也。晋文公求之不得,乃封绵为介子推田。曰:以志吾过,且旌善人。因名斯山为介山。故袁山松《郡国志》曰:界休县有介山、绵上聚、子推庙。王肃《丧服要记》曰:昔鲁哀公祖载其父,孔子问曰:宁设桂树乎?哀公曰:不也。桂树者,起于介子推。子推,晋之人也。文公有内难,出国之狄,子推随其行,割肉以续军粮。后文公复国,忽忘子推,子推奉唱而歌,文公始悟,当受爵禄。子推奔介山,抱木而烧死,国人葬之,恐其神魂霣于地,故作桂树为。吾父生于宫殿,死于枕席,何用桂树为?余按夫子尚非璠玙送葬,安能问桂树为礼乎?王肃此证,近于诬矣。石桐水又西流注于汾水。汾水又西南径界休县故城西,王莽更名之曰界美矣。城东有征士郭林宗、宋子浚二碑。宋冲以有道司徒征,林宗县人也。辟司徒,举太尉,以疾辞。其碑文云:将蹈洪崖之遐迹,绍巢由之逸轨,翔区外以舒翼,超天衢以高峙,禀命不融,享年四十有二,建宁二年正月丁亥卒。凡我四方同好之人,永怀哀痛,乃树碑表墓,昭铭景行云。陈留蔡伯喈、范阳卢子干、扶风马日磾等,远来奔丧,持朋友服。心丧期年者如韩子助、宋子浚等二十四人,其余门人著锡衰者千数。蔡伯喈谓卢子干、马日磾曰:吾为天下碑文多矣,皆有惭容,惟郭有道,无愧于色矣。汾水之右有左部城,侧临汾水,盖刘渊为晋都尉所筑也。
又南过冠爵津,汾津名也,在界休县之西南,俗谓之雀鼠谷,数十里间道险隘,水左右悉结偏梁阁道,累石就路,萦带岩侧,或去水一丈,或高五六尺,上戴山阜,下临绝涧,俗谓之为鲁般桥。盖通古之津隘矣,亦在今之地险也。又南入河东界,又南过永安县西,故彘县也,周厉王流于彘,即此城也。王莽更名黄城,汉顺帝阳嘉三年,改曰永安。县,霍伯之都也。
历唐城东,薛瓒注《汉书》云:尧所都也,东去彘十里。汾水又南与彘水合,水出东北太岳山,《禹贡》所谓岳阳也。即霍太山矣。上有飞廉墓,飞廉以善走事纣,恶来多力见知。周武王伐纣,兼杀恶来。飞廉先为纣使北方,还无所报,乃坛于霍太山而致命焉。得石棺,铭曰:帝令处父,不与殷乱,赐汝石棺以葬。死,遂以葬焉。霍太山有岳庙,庙甚灵,鸟雀不栖其林,猛虎常守其庭,又有灵泉以供祭祀,鼓动则泉流,声绝则水竭。湘东阴山县有侯昙山,上有灵坛,坛前有石井深数尺,居常无水,及临祈祷,则甘泉涌出,周用则已,亦其比也。彘水又西流径观阜北,故百邑也。原过之从襄子也,受竹书于王泽,以告襄子。襄子斋三日,亲自剖竹,有朱书曰:余霍太山山阳侯天使也,三月丙戌,余将使汝反灭智氏,汝亦立我于百邑。襄子拜受三神之命,遂灭智氏,祠三神于百邑,使原过主之,世谓其处为观阜也。彘水又西流径永安县故城南,西南流,注于汾水。汾水又甫径霍城东,故霍国也。昔晋献公灭霍,赵夙为御,霍公求奔齐。晋国大旱,卜之曰,霍太山为祟。使赵夙召霍君奉祀。晋复穰。盖霍公求之故居也。汾水又径赵城西南。穆王以封造父,赵氏自此始也。汾水又南,霍水入焉,水出霍太山,发源成潭,涨七十步而不测其深。西南径赵城南,西流注于汾水。
又南过杨县东,涧水东出谷远县西山,西南径霍山南,又西径杨县故城北。晋大夫僚安之邑也。应劭曰:故杨侯国。王莽更名有年亭也。其水西流入于汾水。汾水径杨城西,不于东矣。《魏土地记》曰:平阳郡治杨县,郡西有汾水南流者是也。
西南过高梁邑西,黑水出黑山,西径杨城南,又西与巢山水会。《山海经》曰:牛首之山,劳水出焉,西流注干潏水。疑是水也。潏水,即巢山之水也。水源东南出巢山东谷,北径浮山东,又西北流与劳水合,乱流西北径高梁城北,西流入于汾水。汾水又南径高梁故城西,故高梁之墟也。《春秋》僖公二十四年,秦穆公纳公子重耳于晋,害怀公于此。《竹书纪年》,晋出公十三年,智伯瑶城高梁。汉高帝十二年以为侯国,封恭侯郦疥于斯邑也。
又南过平阳县东,汾水又南径白马城西。魏刑白马而筑之,故世谓之白马城。今平阳郡治。汾水又南径平阳县故城东,晋大夫赵晁之故邑也。应劭曰:县在平河之阳,尧舜并都之也。《竹书纪年》:晋烈公元年,韩武子都平阳。汉昭帝封度辽将军范明友为侯国,王莽之香平也。魏立平阳郡,治此矣。水侧有尧庙,庙前有碑。《魏土地记》曰:平阳城东十里,汾水东原上有小台,台上有尧神屋石碑。永嘉三年,刘渊徙平阳,于汾水得白玉印,方四寸,高二寸二分,龙纽。其文曰:有新宝之印,王莽所造也。渊以为天授,改永凤二年为河瑞元年。汾水南与平水合,水出平阳县西壶口山,《尚书》所谓壶口冶梁及岐也。其水东径狐谷亭北,春秋时,狄侵晋,取狐厨者也。又东径平阳城南,东入汾。俗以为晋水,非也。汾水又南历襄陵县故城西,晋大夫郤犨之邑也,故其地有犨氏乡亭矣。西北有晋襄公陵,县,盖即陵以命氏也,王莽更名曰干昌矣。
又南过临汾县东,天井水出东陉山西南,北有长岭,岭上东西有通道,即鈃隥也。《穆天子传》曰:乙酉,天子西绝鈃隥,西南至盬是也。其水三泉奇发,西北流,总成一川,西径尧城南,又西流入汾。
又屈从县南西流,汾水又径绛县故城北。《竹书纪年》:梁武王二十五年,绛中地,西绝于汾。汾水西径虒祁宫北,横水有故梁,截汾水中,凡有三十柱,柱径五尺,裁与水平,盖晋平公之故梁也。物在水,故能持久而不败也。又西径魏正平郡南,故东雍州治。太和中,皇都徙洛,罢州立郡矣。又西径王泽,浍水入焉。
又西过长修县南,汾水又西与古水合,水出临汾县故城西黄阜下,其大若轮,西南流,故沟横出焉,东注于汾,今无水。又西南径魏正平郡北,又西径荀城东。古荀国也。《汲郡古文》,晋武公灭荀以赐大夫原氏也。古水又西南入于汾。汾水又西南径长修故城南,汉高帝十一年以为侯国,封杜恬也。有修水出县甫,而西南流入汾。汾水又西径清原城北,故清阳亭也。城北有清原,晋侯蒐清原,作三军处也。汾水又径冀亭南。昔臼季使过冀野,见邵缺耨,其妻饁之,相敬如宾,言之文公,文公命之为卿,复与之冀。京相璠曰:今河东皮氏县有冀亭,古之冀国所都也。杜预《释地》曰:平阳皮氏县东北有冀亭,即此亭也。汾水又西与华水合,水出北山华谷,西南流径一故城西。俗谓之梗阳城,非也。梗阳在榆次不在此。案《故汉上谷长史侯相碑》云:侯氏出自仓颉之后,逾殷历周,各以氏分,或著楚、魏,或显齐、秦,晋卿士斯,其胄也,食采华阳,今蒲坂北亭,即是城也。其水西南流注于汾。汾水又径稷山北,在水南四十许里,山东西二十里,南北三十里,高十三里,西去介山十五里。山上有稷祠,山下稷亭。《春秋》宣公十五年,秦桓公伐晋,晋侯治兵于稷,以略狄土是也。
又西过皮氏县南,汾水西径鄈丘北,故汉氏之方泽也。贾逵云:汉法,三年祭地。汾阴方泽,泽中有方丘,故谓之方泽丘。即鄈丘也。许慎《说文》称从邑,癸声。河东临汾地名矣,在介山北,山即汾山也。其山特立,周七十里,高三十里。文颖言在皮氏县东南,则可三十里,乃非也。今准此山可高十余里,山上有神庙,庙侧有灵泉,祈祭之日,周而不耗,世亦谓之子推祠。扬雄《河东赋》曰:灵舆安步,周流容与,以览于介山,嗟文公而愍推兮,勤大禹于龙门。《晋太康记》及《地道记》与《永初记》,并言子推所逃隐于是山,即实非也。余案介推所隐者,绵山也。文公环而封之,为介推田,号其山为介山。杜预曰:在西河界休县者是也。汾水又西径耿乡城北,故殷都也。帝祖乙自相徙此,为河所毁,故《书》叙曰:祖乙圮于耿。杜预曰:平阳皮氏县东南耿乡是也。盘庚以耿在河北,迫近山川,乃自耿迁毫。晋献公灭耿,以封赵夙,后襄子与韩、魏分晋,韩康子居平阳,魏桓子都安邑,号为三晋,此其一也。汉武帝行幸河东,济汾河,作《秋风辞》于斯水之上。汾水又西径皮氏县南。《竹书纪年》:魏襄王十二年,秦公孙爱率师伐我,围皮氏,翟章率师救皮氏围,疾西风。十三年,城皮氏者也。汉河东太守潘系穿渠引汾水以溉皮氏县,故渠尚存,今无水也。又西至汾阴县,北西注于河。水南有长阜,背汾带河,阜长四五里,广二里余,高十丈,汾水历其阴,西入河。《汉书》谓之汾阴脽。应劭曰:脽,丘类也。汾阴男子公孙祥望气,宝物之精上见,祥言之于武帝,武帝于水获宝鼎焉。迁于甘泉宫,改其年曰元鼎,即此处。
浍水出河东绛县东浍交东高山,浍水东出绛高山,亦曰河南山,又曰浍山。西径翼城南。案《诗谱》言:晋穆侯迁都于绛,暨孙孝侯,改绛为翼,翼为晋之旧都也。后献公北广其城,方二里,又命之为绛。故司马迁《史记年表》称:献公九年,始城绛都。《左传》庄公二十六年,晋士城绛以深其宫是也。其水又西南合黑水,水导源东北黑水谷,西南流径翼城北,右引北川水,水出平川,南流注之,乱流西南入浍水。浍水又西南与诸水合,谓之浍交。《竹书纪年》曰:庄伯十二年,翼侯焚曲沃之禾而还,作为文公也。又有贺水,东出近川,西南至浍交入浍。又有高泉水,出东南近川,西北趣浍交注浍。又南,紫谷水东出白马山白马川。《遁甲开山图》曰:绛山东距白马山。谓是山也。西径荧庭城南,而西出紫谷,与乾河合,即教水之枝川也。《史记·白起传》称涉河取韩安邑,东至乾河是也。其水西与田川水合,水出东溪,西北至浍交入浍。又有于家水出于家谷。《竹书纪年》曰:庄伯以曲沃叛,伐翼,公子万救翼,荀叔轸追之至于家谷。有范壁水出于壁下,并西北流,至翼广城。昔晋军北入,翼广筑之,因即其姓以名之。二水合而西北流,至浍交入浍。浍水又西南与绛水合,俗谓之白水,非也。水出绛山东,寒泉奋涌,扬波北注,悬流奔壑,一十许丈。青崖若点黛,素湍如委练,望之极为奇观矣。其水西北流注于浍。应劭曰:绛水出绛县西南,盖以故绛为言也。《史记》称:智伯率韩、魏引水灌晋阳,不没者三版。智氏曰:吾始不知水可以亡人国,今乃知之。汾水可以浸安邑,绛水可以浸平阳。时韩居平阳,魏都安邑,魏桓子肘韩康子,韩康子履魏桓子,时足接于车上,而智氏以亡。鲁定公问:一言可以丧邦,有诸?孔子以为几乎,余睹智氏之谈矣,汾水灌安邑,或亦有之;绛水灌平阳,未识所由也。
西过其县南,《春秋》成公六年,晋景公谋去故绛,欲居郇瑕。韩献子曰:土薄水浅,不如新田,有汾、浍以流其恶。遂居新田,又谓之绛,即绛阳也,盖在绛、浍之阳。汉高帝六年,封越骑将军华无害为侯国。县南对绛山,面背二水。《古文琐语》曰:晋平公与齐景公乘至于浍上,见乘白骖八驷以来,有大狸身而狐尾,随平公之车,公问师旷,对首阳之神,有大狸身狐尾,其名曰者,饮酒得福,则徼之,盖于是水之上也。
又西南过虒祁宫南,宫在新田绛县故城西四十里,晋平公之所构也。时有石言于魏榆,晋侯以问师旷,旷曰:石不能言,或凭焉。臣闻之,作事不时,怨动于民,则有非言之物言也。今宫室崇侈,民力雕尽,石言不亦宜乎!叔向以为子野之言,君子矣。其宫也,背汾面侩,西则两川之交会也。《竹书纪年》曰:晋出公五年,浍绝于梁。即是水也。
又西至王泽,注于汾水。
晋智伯瑶攻赵襄子,襄子奔保晋阳,原过后至,遇三人于此泽,自带以下不见,持竹二节与原过,曰,为我遗无恤。原过受之于是泽,所谓王泽也。涑水出河东闻喜县东山黍葭谷,涑水所出,俗谓之华谷,至周阳与洮水合,水源东出清野山,世人以为清襄山也。其水东径大岭下,西流出谓之唅口。又西合涑水。郑使子产问晋平公疾,平公曰:卜云台骀为祟。史官莫知,敢问。子产曰:高辛氏有二子,长曰阏伯,季曰实沈,不能相容。帝迁阏伯于商丘,迁实沈于大夏。台骀,实沈之后,能业其官,帝用嘉之,国于汾川。由是观之,台骀,汾、洮之神也。贾逵曰:汾、洮,二水名。司马彪曰:洮水出闻喜县。故王莽以县为洮亭也。然则涑水殆亦洮水之兼称乎?
西过周阳邑南,其城南临涑水,北倚山原。《竹书纪年》:晋献公二十五年正月,翟人伐晋,周有白兔舞于市。即是邑也。汉景帝以封田胜为侯国。涑水西径董泽陂南,即古池,东西四里,南北三里。《春秋》文公六年,蒐于董,即斯泽也。涑水又与景水合,水出景山北谷。《山海经》曰:景山南望盐贩之泽,北望少泽,其草多藷、秦椒,其阴多赭,其阳多玉。郭景纯曰:盐贩之泽即解县盐池也。案《经》不言有水,今有水焉。西北流,注于涑水也。
又西南过左邑县南,涑水又西径仲邮北,又西径桐乡城北。《竹水纪年》曰:翼侯伐曲沃,大捷。武公请成于翼,至桐乃返者也。《汉书》曰:武帝元鼎六年,将幸缑氏,至左邑桐乡,闻南越破,以为闻喜县者也。涑水又西与沙渠水合,水出东南近川,西北流注于涑水。涑水又西南径左邑县故城南,故曲沃也。晋武公自晋阳徙此,秦改为左邑县,《诗》所谓从子于鹄者也。《春秋传》曰:下国有宗庙,谓之国。在绛曰下国矣,即新城也。王莽之洮亭也。涑水自城西注,水流急浚,轻津无缓,故诗人以为激扬之水,言不能流移束薪耳。水侧,即狐突遇申生处也。《春秋传》曰:秋,狐突适下国,遇太子,太子使登仆曰:夷吾无札,吾请帝以畀秦,对曰:神不歆非类,君其图之,君曰诺,请七日见我于新城西偏。及期而往,见于此处。故《传》曰:鬼神所凭,有时而信矣。涑水又西径王官城北,城在南原上。《春秋左传》成公十三年四月,晋侯使吕相绝秦曰:康犹不悛,入我河曲,伐我涑川,俘我王官。故有河曲之战是矣。今世人犹谓其城曰王城也。
又西南过安邑县西。
安邑,禹都也。禹娶涂山氏女,思恋本国,筑台以望之,今城南门,台基犹存。余案《礼》,天子诸侯,台门隅阿相降而已,未必一如《书》传也。故晋邑矣,春秋时,魏绛自魏徙此。昔文侯悬师经之琴于其门,以为言戒也。武侯二年,又城安邑,盖增广之。秦始皇使左更、白起取安邑,置河东郡。王莽更名洮队,县曰河东也。有项宁都,学道升仙,忽复还此,河东号曰斥仙。汉世又有闵仲叔,隐遁市邑,罕有知者,后以识瞻而去。涑水西南径监盐县故城,城南有盐他,上承盐水。水出东南薄山,西北流径巫咸山北。《地理志》曰:山在安邑县南。《海外西经》曰:巫咸国在女丑北,右手操青蛇,左手操赤蛇,在登葆山,群巫所从上下也。《大荒西经》云:大荒之中有灵山,巫咸、巫即、巫朌、巫彭、巫姑、巫真、巫礼、巫抵、巫谢、巫罗十巫,从此升降,百药爱在。郭景纯曰:言群巫上下灵山,采药往来也。盖神巫所游,故山得其名矣。谷口岭上,有巫咸祠。其水又径安邑故城南,又西流注于盐池。《地理志》曰:盐池在安邑西南。许慎谓之盬。长五十一里,广七里,周百一十六里,从盐省古声。吕忱曰:夙沙初作煮海盐,河东盐池谓之盬。今池水东西七十里,南北十七里,紫色澄渟,潭而不流。水出石盐,自然印成,朝取夕复,终无减损。惟山水暴至,雨澍潢潦奔泆,则盐池用耗。故公私共堨水径,防其淫滥,谓之盐水,亦谓之为堨水。《山海经》谓之盐贩之泽也。泽南面层山,天岩云秀,地谷渊深,左右壁立,间不容轨,谓之石门,路出其中,名之曰径,南通上阳,北暨盐泽。池西又有一池,谓之女盐泽,东西二十五里,南北二十里,在猗氏故城南。《春秋》成公六年,晋谋去故绛,大夫曰:郇、瑕,地沃饶近盬。服虔曰:土平有溉曰沃,盬,盐池也。土俗裂水沃麻,分灌川野,畦水耗竭,土自成盐,即所谓咸鹾也,而味苦,号曰盐田,盐盬之名,始资是矣。本司盐都尉治,领兵千余人守之。周穆王、汉章帝并幸安邑而观盐池。故杜预曰:猗氏有盐池。后罢尉司,分猗氏、安邑,置县以守之。
又南过解县东,又西南注于张阳池。
涑水又西径猗氏县故城北。《春秋》文公七年,晋败秦于令狐,至于刳首,先蔑奔秦,士会从之。阚駰曰:令狐即猗氏也。刳首在西三十里,县南对泽,即猗顿之故居也。《孔丛》曰:猗顿,鲁之穷士也,耕则常饥,桑则常寒,闻朱公富,往而问术焉。朱公告之曰:子欲速富,当畜五牸。于是乃适西河,大畜牛羊于猗氏之南,十年之间,其息不可计,赀拟王公,驰名天下,以兴富于猗氏,故曰猗顿也。涑水又西径郇城,《诗》云郇伯劳之,盖其故国也。杜元凯《春秋释地》云:今解县西北有郇城。服虔曰:郇国在解县东,郇瑕氏之墟也。余按《竹书纪年》云:晋惠公十有四年,秦穆公率师送公子重耳,围令狐,桑泉、臼衰皆降于秦师,狐毛与先轸御秦,至于庐柳,乃谓秦穆公,使公子挚来,与师言退,舍次于郇,盟于军。京相璠《春秋土地名》曰:桑泉、臼衰并在解东南,不言解,明不至解。可知《春秋》之文,与《竹书》不殊。今解故城东北二十四里有故城,在猗氏故城西北,乡俗名之为郇城,考服虔之说,又与俗符,贤于杜氏单文孤证矣。涑水又西南径解县故城南。《春秋》,晋惠公因秦返国,许秦以河外五城,内及解梁,即斯城也。涑水又西南径瑕城,晋大夫詹嘉之故邑也。《春秋》僖公三十年,秦、晋围郑,郑伯使烛之武谓秦穆公曰:晋许君焦瑕,朝济而夕设版者也。京相璠曰:今河东解县西南五里有故瑕城。涑水又西南径张阳城东。《竹书纪年》,齐师逐郑太子齿,奔张城南郑者也。《汉书》之所谓东张矣。高祖二年,曹参假左丞相,别与韩信东攻,魏将孙遬军东张,大破之。苏林曰:属河东,即斯城也。涑水又西南属于陂,陂分为二,城南面两陂,左右泽渚。东陂世谓之晋兴泽,东西二十五里,南北八里,南对盐道山。其西则石壁千寻,东则磻溪万仞,方岭云回,奇峰霞举,孤标秀出,罩络群山之表,翠柏荫峰,清泉灌顶。郭景纯云:世所谓鸯浆也。发于上而潜于下矣。厥顶方平,有良药。《神农本草》曰:地有固活、女疏、铜芸、紫菀之族也。是以缁服思元之士,鹿裘念一之夫,代往游焉。路出北,势多悬绝,来去者咸援萝腾崟,寻葛降深,于东则连木,乃陟百梯方降岩侧,縻锁之迹,仍今存焉,故亦曰百梯山也。水自山北流五里而伏,云潜通泽渚,所未详也。西陂即张泽也,西北去蒲坂十五里,东西二十里,南北四五里,冬夏积水,亦时有盈耗也。文水出大陵县西山文谷,东到其县,屈南到干陶县东北,东入于汾。
文水径大陵县故城西而南流,有泌水注之。县西南山下,武氏穿井给养,井至幽深,后一朝水溢平地,东南注文水。文水又南径平陶县之故城东,西径其城内,南流出郭,王莽更曰多穰也。文水又南径县,右会隐泉口,水出谒泉山之上顶,俗云:旸雨愆时,是谒是祷。故山得其名,非所详也。其山石崖绝险,壁立天固,崖半有一石室,去地可五十余丈,爱有层松饰岩,列柏绮望,惟西侧一处得历级升陟,顶上平地十许顷,沙门释僧光表建二刹。泉发于两寺之间,东流沥石,沿注山下,又东,津渠隐没而不恒流,故有隐泉之名矣。雨泽丰谢,则通入文水。文水又南径兹氏县故城东,为文湖,东西十五里,南北三十里,世谓之西湖,在县直东十里;湖之西侧,临湖又有一城,谓之潴城。水泽所聚谓之都,亦曰潴,盖即水以名城也。文湖又东径中阳县故城东。案《晋书地道记》、《太康地记》,西河有中阳城,旧县也。文水又东南流,与胜水合,水西出狐岐之山,东径六壁城南,魏朝旧置六壁于其下,防离石诸胡,因为大镇。太和中罢镇,仍置西河郡焉。胜水又东合阳泉水,水出西山阳溪,东径六壁城北,又东南流注于胜水。胜水又东径中阳故城南,又东合文水。文水又东南,入于汾水也。
原公水出兹氏县西羊头山,东过其县北,县,故秦置也,汉高帝更封沂阳侯婴为侯国,王莽之兹同也。魏黄初二年,分太原,复置西河郡。晋徙封陈王斌于西河,故县有西河缪王司马子政庙。碑文云:西河旧处山林,汉末扰攘,百姓失所。魏兴,更开疆宇,分割太原四县,以为邦邑,其郡带山侧塞矣。王以咸宁三年,改命爵上,明年十二月丧国。臣大农阎崇、离石令宗群等二百三十四人,刊石立碑,以述勋德。碑北庙基尚存也。又东入于汾。
水注文湖,不至汾也。
洞过水出沾县北山,其水西流,与南溪水合,水出南山,西北流注洞过水,洞过水又西北,黑水西出山,三源合舍,同归一川,东流南屈,径受阳县故城东。案《晋太康地记》,乐平郡有受阳县,卢谌《征艰赋》所谓历受阳而总辔者也。其水又西南入洞过水。洞过水又西,蒲水南出蒲谷,北流注之。洞过水又西与原过水合,近北便水源也。水西阜上有原过祠,盖怀道协灵,受书天使,忧结宿情,传芳后日,栋宇虽沦,攒木犹茂,故水取名焉。其水南流,注于洞过水也。
西过榆次县南,又西到晋阳县南,榆次县,故涂水乡,晋大夫智徐吾之邑也。《春秋》昭公八年,晋侯筑箎祁之宫,有石言晋之魏榆。服虔曰:魏,晋邑:榆,州里名也。《汉书》曰榆次,《十三州志》以为涂阳县矣。王莽之太原亭也。县南侧水有凿台,韩、魏杀智伯瑶于其下,刳腹绝肠,折颈招颐处也。其水又西南流,径武灌城西北。卢谌《征艰赋》曰:径武馆之故郛,问厥途之远近。洞过水又西南为淳湖,谓之洞过泽。泽南,途水注之,水出阳邑东北大嵰山涂谷,西南径萝蘑亭南,与蒋谷水合,水出县东南蒋溪。《魏土地记》曰:晋阳城东南百一十里至山有蒋谷大道,度轩车岭,通于武乡。水自蒋溪西北流,西径箕城北。《春秋》僖公三十三年,晋人败狄于箕。杜预《释地》曰:城在阳邑南,水北即阳邑县故城也。《竹书纪年》曰:梁惠成王九年,与邯郸、榆次、阳邑者也。王莽之繁穰矣。蒋溪又西合涂水,乱流西北入洞过泽也。西入于汾,出晋水下口者也。
刘琨之为并州也,刘曜引兵邀击之,合战于洞过,即是水也。
晋水出晋阳县西悬瓮山,县,故唐国也。《春秋左传》称唐叔未生,其母邑姜梦帝谓己曰:余名而子曰虞,将与之唐,属之参。及生,名之曰虞。《吕氏春秋》曰:叔虞与成王居,王援桐叶为圭,以授之曰:吾以此封汝。虞以告周公,周公请曰:天子封虞乎?王曰:余戏耳。公曰:天子无戏言。时唐灭,乃封之于唐。县有晋水,后改名为晋。故子夏叙《诗》称此晋也,而谓之唐,俭而用礼,有尧之遗风也。《晋书地道记》及《十三州志》并言晋水出龙山,一名结绌山,在县西北,非也。《山海经》曰:悬瓮之山,晋水出焉,今在县之西南。昔智伯之遏晋水以灌晋阳,其川上溯,后人踵其遗迹,蓄以为沼,沼西际山枕水,有唐叔虞祠。水侧有凉堂,结飞梁于水上,左右杂树交荫,希见曦景,至有淫朋密友,羁游宦子,莫不寻梁契集,用相娱慰,于晋川之中,最为胜处。又东过其县南,又东入于汾水。
沼水分为二派,北读即智氏故渠也。昔在战国,襄子保晋阳,智氏防山以水之,城不没者三版,与韩、魏望叹于此,故智氏用亡。其渎乘高,东北注入晋阳城,以周灌溉。汉未赤眉之难,郡椽刘茂负太守孙福匿于城门西下空穴中,其夜奔盂。即是处也。东南出城流,注于汾水也。其南渎于石塘之下伏流,径旧溪东南出,径晋阳城南,城在晋水之阳,故曰晋阳矣。《经》书晋苟吴帅师败狄于大卤。杜预曰:大卤,晋阳县也,为晋之旧都。《春秋》定公十三年,赵鞅以晋阳叛,后乃为赵矣。其水又东南流入于汾。
湛水出河内轵县西北山,湛水出轵县南原湛溪,俗谓之堪水也。是盖声形尽邻,故字读俱变,同于三豕之误耳。其水自溪出南流。
东过其县北,又东过波县之北,湛水南径向城东而南注。
又东过毋辟邑南,原《经》所注,斯乃湨川之所由,非湛水之间关也,是乃《经》之误证耳。湛水自向城东南径湛城东。时人谓之椹城,亦或谓之隰城矣。溪曰隰涧。隔城在东,言此非矣。《后汉郡国志》曰河阳县有湛城是也。
又东南当平县之东北,南入于河。
湛水又东南径邓,南流注于河,故河济有邓津之名矣。


卷七 济水 
济水出河东垣县东王屋山,为沇水。
《山海经》曰:王屋之山联水出焉,西北流,注于秦泽。郭景纯云:联、沇声相近,即沇水也。潜行地下,至共山南,复出于东丘。今原城东北有东丘城。孔安国曰:泉源为沇,流去为济。《春秋说题辞》曰:济,齐也;齐,度也,贞也。《风俗通》曰:济出常山房子县赞皇山,庙在东郡临邑县。济者,齐也,齐其度量也。余按二济同名,所出不同,乡原亦别,斯乃应氏之非矣。今济水重源出轵县西北平地。水有二源:东源出原城东北,昔晋文公伐原以信,而原降,即此城也。俗以济水重源所发,因复谓之济原城。其水南径其城东故县之原乡。杜预曰:沁水县西北有原城者是也。南流与西源合,西源出原城西,东流水注之。水出西南,东北流注于济。济水又东径原城南,东合北水,乱流东南注,分为二水,一水东南流,俗谓之为衍水,即沇水也。衍、沇声相近,转呼失实也。济水又东南,径絺城北而出于温矣。其一水枝津南流,注于湨。湨水出原城西北原山勋掌谷,俗谓之为白涧水,南径原城西。《春秋》,会于湨梁,谓是水之坟梁也。《尔雅》曰:梁莫大于湨梁。梁,水堤也。湨水又东南径阳城东,与南源合,水出阳城南溪。阳亦樊也。一曰阳樊。《国语》曰:王以阳樊赐晋,阳人不服,文公围之。仓葛曰:阳有夏、商之嗣典,樊仲之官守焉。君而残之,无乃不可乎?公乃出阳人。《春秋》,樊氏叛,惠王使虢公伐樊,执仲皮归于京师。即此城也。其水东北流,与漫流水合,水出织关南,东北流,又北注于湨,谓之漫流口。湨水又东合北水,乱流东南,左会济水枝渠。湨水又东径钟繇坞北,世谓之钟公垒。又东南,涂沟水注之。水出轵县西南山下,北流东转,入轵县故城中,又屈而北流出轵郭。汉文帝元年,封薄昭为侯国也。又东北流注于湨。湨水又东北径波县故城北。汉高帝封公上不害为侯国。湨水又东南流,天浆涧水注之。水出轵南皋向城北,城在皋上。俗谓之韩王城,非也。京相璠曰:或云今河内织西有城,名向,今无。杜元凯《春秋释地》亦言是矣。盖相袭之向,故不得以地名而无城也。阚駰《十三州志》曰:轵县南山西曲有故向城,即周向国也。《传》曰,向姜不安于莒而归者矣。汲郡《竹书纪年》曰:郑侯使韩辰归晋阳及向。二月,城阳、向,更名阳为河雍,向为高平。即是城也。其水有二源俱导,各出一溪,东北流,合为一川,名曰天浆溪。又东北径一故城,俗谓之冶城,水亦曰冶水。又东流注于湨。湨水又东南流,右会同水,水出南原下,东北流径白骑坞南。坞在原上,为二溪之会,北带深隍,三面阻险,惟西版筑而已。东北流径安国城西,又东北注湨水。湨水东南径安国城东,又南径毋辟邑西,世谓之无比城,亦曰马髀城,皆非也。朝廷以居废太子,谓之河阳庶人。湨水又南注于河。
又东至温县西北,为济水。又东过其县北,济水于温城西北与故渎分,南径温县故城西。周畿内国,司寇苏忿生之邑也。《春秋》僖公十年,狄灭温,温子奔卫,周襄王以赐晋文公。济水南历貌公台西。《皇览》曰:温城南有虢公台,基趾尚存。济水南流注于河。郭缘生《述征记》曰:济水河内温县注于河,盖沿历之实证,非为谬说也。济水故渎于温城西北东南出,径温城北,又东径虢公冢北。《皇览》曰:虢公冢在温县郭东,济水南大冢是也。济水当王莽之世,川渎枯竭,其后水流径通,津渠势改,寻梁脉水,不与昔同。
屈从县东南流,过隤城西,又南当巩县北,南入于河。济水故渎东南合奉沟水,水上承朱沟于野王城西,东南径阳乡城北,又东南径李城西。秦攻赵,邯郸且降,传舍吏子李同说平原君胜,分家财飨士,得敢死者三千人,李同与赴秦军,秦军退。同死,封其父为李侯。故徐广曰:河内平旱县有李城。即此城也。于城西南为陂水,淹地百许顷,兼葭萑苇生焉,号曰李陂。又径隤城西,屈而东北流,径其城北,又东径平皋城南。应劭曰:邢侯自襄国徙此。当齐桓公时,卫人伐邢,邢迁于夷仪,其地属晋,号曰邢丘。以其在河之皋,势处平夷,故曰平皋。瓒性《汉书》云:《春秋》,狄人伐邢,邢迁夷仪,不至此也。今襄国西有夷仪城,去襄国百余里,平皋是邢丘,非国也。余案《春秋》宣公六年,赤狄伐晋,围邢丘。昔晋侯送女于楚,送之邢丘,即是此处也,非无城之言。《竹书纪年》曰:梁惠成王三年,郑城邢丘。司马彪《后汉郡国志》云:县有邢丘,故邢国,周公子所封矣。汉高帝六年,封砀郡长项佗为侯国,赐姓刘氏,武帝以为县。其水又南注于河也。与河合流,又东过成皋县北,又东过荥阳县北,又东至砾溪南,东出过荥泽北。
《释名》曰:济,济也,源出河北济河而南也,《晋地道志》曰:济自大伾入河,与河水斗,南泆为荥泽。《尚书》曰:荥波既潴。孔安国曰,荥泽波水已成遏潴。阚駰曰:荣播,泽名也。故吕忱云:播水在荥阳。谓是水也。昔大禹塞其淫水而于荥阳下引河,东南以通淮、泗,济水分河东南流。汉明帝之世,司空伏恭荐乐浪人王景,字仲通,好学多艺,善能洽水。显宗诏与谒者王吴始作浚仪渠,吴用景法,水乃不害,此即景、吴所修故渎也。渠流东注,浚仪故复,谓之浚仪渠。明帝永平十五年,东巡至无盐,帝嘉景功,拜河堤谒者。灵帝建宁四年,于敖城西北垒石为门,以遏渠口,谓之石门,故世亦谓之石门水。门广十余丈,西去河三里,石铭云:建宁四年十一月,黄场石也,而主吏姓名,磨灭不可复识。魏太和中,又更修之,撤故增新,石字沦落,无复在者。水北有石门亭,戴延之所云新筑城,城周三百步,荥阳太守所镇者也。水南带三皇山,即皇室山,亦谓之为三室山也。济水又东径西广武城北。《郡国志》,荥阳县有广武城,城在山上,汉所城也。高祖与项羽临绝涧对语,责羽十罪,羽射汉祖中胸处也。山下有水,北流入济,世谓之柳泉也。济水又东径东广武城北,楚项羽城之。汉破曹咎,羽还广武,为高坛,置太公其上,曰:汉不下,吾烹之。高租不听,将窖之。项伯曰:为天下者不顾家,但益怨耳。羽从之。今名其坛曰项羽堆。夹城之间,有绝洞断山,谓之广武涧。项羽叱娄烦于其上,娄烦精魄丧归矣。济水又东径敖山北,《诗》所谓薄狩于敖者也。其山上有城,即殷帝仲丁之所迁也。皇甫谧《帝王世纪》曰仲丁自毫徙嚣于河上者也。或曰敖矣。秦置仓于其中,故亦曰敖仓城也。济水又东合荥渎,渎首受河水,有石门,谓之为荥口石门也,而地形殊卑,盖故荥播所导,自此始也。门南际河,有故碑云:惟阳嘉三年二月丁丑,使河堤谒者王海,疏达河川,遹荒庶土,往大河冲塞,侵啮金堤,以竹笼石葺土而为堨,坏隤无已,功消亿万,请以滨河郡徒,疏山采石垒以为障,功业既就,徭役用息,未详诏书,许诲立功府卿,规基经始,诏策加命,迁在沇州,乃简朱轩授使司马登,令缵茂前绪,称遂休功,登以伊、洛合注大河,南则缘山,东过大伾,回流北岸,其势郁,涛怒湍急激疾,一有决溢,弥原淹野,蚁孔之变,害起不测,盖自姬氏之所常蹙。昔崇鲧所不能治,我二宗之所劬劳于是。乃跋涉躬亲,经之营之,比率百姓,议之于臣,伐石三谷,水匠致治,立激岸侧,以捍鸿波,随时庆赐说以劝之,川无滞越,水土通演,役未逾年,而功程有毕,斯乃元勋之嘉课,上德之宏表也。昔禹修九道,《书》录其功;后稷躬稼,《诗》列于《雅》。夫不惮劳谦之勤,夙兴厥职,充国惠民,安得湮没而不章焉。故遂刊石记功,垂示于后。其辞云云:使河堤谒者山阳东缗司马登,字伯志;代东莱曲成王海,字孟坚;河内太守宋城向豹,字伯尹;丞汝南邓方,字德山;怀令刘丞,字季意;河堤椽匠等造。陈留浚仪边韶字孝先颂。石铭岁远,字多沦缺,其所灭,盖阙如也。荥渎又东南流,注于济,今无水。次东得宿须水口,水受大河,渠侧有扈亭水,自亭东南流,注于济,今无水。宿须在河之北,不在此也,盖名同耳。自西缘带山隰,秦、汉以来,亦有通否。济水与河浑涛东注,晋太和中,桓温北伐,将通之,不果而还。义熙十二年,刘公西征;又命宁朔将军刘遵考仍此渠而漕之,始有激湍东注,而终山崩壅塞,刘公于北十里更凿故渠通之。今则南渎通津,川涧是导耳。济水于此,又兼邲目。《春秋》宣公十三年,晋、楚之战,楚军于邲。即是水也。音卞。京相璠曰:在敖北。济水又东径荥阳县北。曹太祖与徐荣战,不利,曹洪授马于此处也。济水又东,砾石溪水注之。水出荥阳城西南李泽,泽中有水,即古冯池也。《地理志》曰:荥阳县,冯池在西南是也。东北流,历敖山南。《春秋》,晋、楚之战,设伏于敖前,谓是也。径虢亭北,池水又东北径荥阳县北断山,东北注于济,世谓之砾石涧,即《经》所谓砾溪矣。《经》云济出其南,非也。济水又东,索水注之,水出京县西南嵩渚山,与东关水同源分流,即古胸然水也。其水东北流,器难之水注之。《山海经》曰:少陉之山,器难之水出焉,而北流注于侵水。即此水也。其水北流径金亭,又北径京县故城西,入于旃然之水。城,故郑邑也。庄公以居弟段,号京城太叔。祭仲曰:京城过百雉,国之害也。城北有坛山冈。《赵世家》成侯二十年,魏献荥阳,因以为坛。台冈也。其水乱流,北径小索亭西。京相璠曰,京有小索亭。《世语》以为本索氏兄弟居此,故号小索者也。又为索水。索水又北径大栅城东。晋荥阳民张卓、董迈等遭荒,鸠聚流杂保固,名为大栅坞。至太平真君八年,豫州刺史崔白自虎牢移州治此,又东开广旧城,创制改筑焉。太和十七年,迁都洛邑,省州置郡。索水又屈而西流,与梧桐涧水合,水出西南梧桐谷,东北流注于索。斯水亦时有通塞,而不常流也。索水又北屈,东径大索城南。《春秋传》曰:郑子皮劳叔向于索氏。即此城也。《晋地道志》所谓京有大索、小索亭。《汉书》京、索之间也。索水又东径虢亭南。应劭曰:荥阳,故虢公之国也,今虢亭是矣。司马彪《郡国志》曰:县有虢亭,俗谓之平桃城。城内有大冢,名管叔冢。或亦谓之为号咷城,非也。盖号、虢字相类,字转失实也。《风俗通》曰:俗说高祖与项羽战于京、索,遁于薄中。羽追求之,时鸠止鸣其上,追之者以为必无人,遂得脱。及即位,异此鸠,故作鸠杖以扶老。案《广志》,楚鸠一名嗥啁,号眺之名,盖因鸠以起目焉,所未详也。索水又东北流,须水右入焉。水近出京城东北二里榆子沟,亦曰柰榆沟也,又或谓之为小索水。东北流,木蓼沟水注之,水上承京城南渊,世谓之车轮渊。渊水东北流,谓之木蓼沟。又东北入于须水。须水又东北流,于荥阳城西南北注索。索水又东径荥阳县故城南。汉王之困荥阳也,纪信曰:臣诈降楚王,宜间出。信乃乘王车出东门,称汉降楚。楚军称万岁,震动天地,王与数十骑出西门得免楚围。羽见信大怒,遂烹之。信冢在城西北三里。故蔡伯喈《述征赋》曰:过汉祖之所隘,吊纪信于荥阳。其城跨倚冈原,居山之阳,王莽立为祈队,备周六队之制。魏正始三年,岁在甲子,被癸丑诏书,割河南郡县,自巩阙以东,创建荥阳郡,并户二万五千,以南乡筑阳亭侯李胜,字公昭,为郡守。故原武典农校尉,政有遗惠,民为立祠于城北五里,号曰李君祠。庙前有石蹠,蹠上有石的,《石的铭》具存。其略曰:百族欣戴,咸推厥诚。今犹祀祷焉。索水又东径周苛冢北。汉祖之出荥阳也,令御史大夫周苛守之,项羽拔荥阳获苛,曰:吾以公为上将军,封三万户侯,能尽节乎?苛瞋目骂羽,羽怒,烹之。索水又东流,北屈西转,北径荥阳城东,而北流注济水。杜预曰:旃然水出荥阳成皋县,东入汳。《春秋》襄公十八年,楚伐郑,右师涉颖,次于旃然,即是水也。济渠水断汳沟,惟承此始,故云汳受旃然矣。亦谓之鸿沟水,盖因汉、楚分王,指水为断故也。《郡国志》曰:荥阳有鸿沟水是也。盖因城地而变名,为川流之异目。济水又东径荥泽北,故荥水所都也。京相璠曰:荥泽在荥阳县东南,与济隧合。济隧上承河水于卷县北河,南径卷县故城东,又南径衡雍城西。《春秋左传》襄公十一年,诸侯伐郑,西济于济隧。杜顶阙其地,而曰水名也。京相璠曰:郑地也。言济水荥泽中北流,至衡雍西,与出河之济会,南去新郑百里,斯盖荥播、河、济,往复径通矣。出河之济即阴沟之上源也,济隧绝焉。故世亦或谓其故道为十字沟。自于岑造八激堤于河阴,水脉径断,故渎难寻,又南会于荥泽。然水既断,民谓其处为荥泽。《春秋》,卫侯及翟人战于荥泽,而屠懿公宏演报命纳肝处也。有垂陇城,济渎出其北。《春秋》文公二年,晋士毅盟于垂陇者也。京相璠曰:垂陇,郑地。今荥阳东二十里有故垂陇城,即此是也。世谓之都尉城,盖荥阳典农都尉治,故变垂陇之名矣。渎际又有沙城,城左佩济读。《竹书纪年》,梁惠成王九年,工会郑釐侯于巫沙者也。渎际有故城,世谓之水城。《史记》:秦昭王三十二年,魏冉攻魏,走芒卯,入北宅。即故宅阳城也。《竹书纪年》曰惠成王十三年,王及郑釐侯盟于巫沙,以释宅阳之围,归釐于郑者也。《竹书纪年》:晋出公六年,齐、郑伐卫,荀瑶城宅阳。俗言水城,非矣。济水自泽东出,即是始矣。王隐曰:河决为荥,济水受焉,故有济堤矣,谓此济也。济水又东南径厘城东。《春秋经》书,公会郑伯于时来,《左传》所谓厘也。京相瑶曰:今荥阳县东四十里有故厘城也。济水右合黄水,水发源京县黄堆山,东南流,名祝龙泉,泉势沸涌,状若巨鼎扬汤。西南流,谓之龙项口,世谓之京水也。又屈而北注,鱼子沟水入焉,水出石暗涧,东北流,又北与濏濏水合。水出西溪东流,水上有连理树,其树,柞栎也,南北对生,凌空交合,溪水历二树之间,东流注于鱼水,鱼水又屈而西北注黄水。黄水又北径高阳亭东,又北至故市县,重泉水注之。水出京城西南少陉山,东北流,又北流径高阳亭西,东北流注于黄水。又东北径故市县故城南。汉高帝六年,封阎泽赤为侯国,河南郡之属县也。黄水又东北至荥泽南,分为二水:一水北入荥泽,下为船塘,俗谓之郏城陂,东西四十里,南北二十里。《竹书》、《穆天子传》曰:甲寅,天子浮于荥水,乃奏《广乐》是也。一水东北流,即黄雀沟矣。《穆天子传》曰:王寅,天子东至于雀梁者也。又东北与靖水枝津合,二水之会为黄渊,北流注于济水。又东过阳武县南,济水又东南流入阳武县,历长城东南流,蒗渠出焉。济水又东北流,南济也,径阳武县故城南。王莽更名之曰阳桓矣。又东为白马渊,渊东西二里,南北百五十步,渊流名为白马沟。又东径房城北。《穆天子传》曰天子里甫田之路,东至于房,疑即斯城也。郭《注》以为赵郡房子也。余谓穆王里郑甫而郭以赵之房邑为疆,更为非矣。济水又东径封丘县南,又东径大梁城北,又东径仓垣城,又东径小黄县之故城北。县有黄亭,说济又谓之曰黄沟。县,故阳武之东黄乡也,故水以名县。沛公起兵野战,丧皇妣于黄乡,天下平定,乃使使者以粹宫招魂幽野于是。丹蛇自水濯洗,入于梓宫,其浴处有遗发焉。故谥曰昭灵夫人,因作寝以宁神也。济水又东径东昏县故城北。阳武县之户牖乡矣,汉丞相陈平家焉。平少为社宰,以善均肉称,今民祠其社。平有功于高祖,封户牖侯,是后置东昏县也,王莽改曰东明矣。济水又东径济阳县故城南,故武父城也。城在济水之阳,故以为名,王莽改之曰济前者也。光武生济阳宫,光明照室,即其处也。《东观汉记》曰:光武以建平元年生于济阳县,是岁有嘉禾生,一茎九穗,大于凡禾,县界大熟,因名曰秀。
又东过封丘县北,北济也。自荥泽东径莱阳卷县之武修亭南。《春秋左传》成公十年,郑子然盟于修泽者也,郑地矣。杜预曰:卷东有武修亭。济水又东径原武县故城南,《春秋》之原圃也。《穆天子传》曰:祭父自圃郑来谒天子,夏,庚午,天子饮于洧上,乃遣祭父如圃郑是也。王莽之原桓矣。济渎又东径阳武县故城北,又东绝长城。案《竹书纪年》梁惠成王十二年,龙贾率师筑长城于西边,自亥谷以南,郑所城矣。《竹书纪年》云是梁惠成王十五年筑也。《郡国志》曰:长城自卷径阳武到密者是矣。济读又东径酸枣县之乌巢泽,泽北有故市亭。《晋太康地记》曰:泽在酸枣之东南,昔曹太祖纳许攸之策,破袁绍运处也。济读又东径封丘县北,南燕县之延乡也,其在《春秋》为长丘焉。应劭曰:《左传》,宋败狄于长丘,获长狄,缘斯是也。汉高帝封翟盱为侯国。濮水出焉。济渎又东径大梁城之赤亭北而东注。
又东过平丘县南,北济也。县,故卫地也。《春秋》鲁昭公十三年,诸侯盟于平丘是也。
县有临济亭,田儋死处也。又有曲济亭,皆临侧济水者。
又东过济阳县北,北济也,自武父城北。阚駰曰:在县西北,郑邑也。东径济阳县故城北。圈称《陈留风俗传》曰:县,故宋地也。《竹书纪年》:梁惠成王三十年城济阳。汉景帝中六年,封梁孝王子明为济川王。应劭曰:济川,今陈留济阳县是也。
又东过冤朐县南,又东过定陶县南,南济也。济渎自济阳县故城南,东径戎城北。《春秋》隐公二年,公会戎于潜。杜预曰:陈留济阳县东南有戎城是也。济水又东北,菏水东出焉。济水又东北径冤朐县故城南。吕后元年,封楚元王子刘执为侯国,王莽之济平亭也。济水又东径秦相魏冉家南。冉,秦宣太后弟也,代客卿寿烛为相,封于穰,益封于陶,号曰穰侯,富于王室。范雎说秦,秦王悟其擅权,免相,就封出关,辎车千乘,卒于陶,而因葬焉,世谓之安平陵,墓南崩碑尚存。济水又东北径定陶恭王陵南。汉哀帝父也,帝即位,母丁太后建平二年崩,上曰:宜起陵于恭皇之园,送葬定陶贵震山东。王莽秉政,贬号丁姬,开其椁户,火出炎四五丈,吏卒以水沃灭,乃得入,烧燔椁中器物,公卿遣子弟及诸生四夷十余万人,操持作具,助将作掘平共王母傅太后坟及丁姬家,二旬皆平。莽又周棘其处,以为世戒云。时有群燕数千,衔土投于丁姬竁中,今其坟冢,巍然尚秀,隅阿相承,列郭数周,面开重门,南门内夹道有崩碑二所,世尚谓之丁昭仪墓,又谓之长隧陵。盖所毁者,傅太后陵耳。丁姬坟墓,事与书违,不甚过毁,未必一如史说也。坟南,魏郡治也。世渭之左城,亦名之曰葬城,盖恭王之陵寝也。济水又东北径定陶县故城南,侧城东注。县,故三鬷国也,汤追桀,伐三鬷,即此。周武王封弟叔振铎之邑,故曹国也。汉宣帝甘露二年,更济阴为定陶国,王莽之济平也。战国之世,范蠡既雪会稽之耻,乃变姓名寓于陶,为朱公。以陶天下之中,诸侯四通,货物之所交易也。治产致千金,富好行德,子孙修业,遂致巨万。故言富者,皆曰陶朱公也。又屈从县东北流。
南济也。又东北右合菏水,水上承济水于济阳县东,世谓之五丈沟。又东径陶丘北。《地理志》曰:《禹贡》,陶丘在定陶西南。陶丘亭在南,墨子以为釜丘也。《竹书纪年》魏襄王十九年,薛侯来会王于釜丘者也。《尚书》所谓菏水自陶丘北,谓此也。菏水东北出于定陶县北,屈左合菏水,菏水西分济渎,东北径济阴郡南。《尔雅》曰,济别为濋。吕忱曰:水决复入为氾,广异名也。氾水又东合于菏渎。昔汉租既定天下,即帝位于定陶汇水之阳。张晏曰:氾水在济阴界,取其氾爱弘大而润下也,沮水之名,于是乎在矣。菏水又东北,径定陶县南,又东北,右合黄水枝渠,渠上承黄沟,东北合菏而北注济渎也。


卷八 济水 
又东至乘氏县西,分为二,《春秋左传》傅公三十一年,分曹地东傅,于济。济水自是东北流,出巨泽。
其一水东南流,其一水从县东北流,入巨野泽。南为菏水,北为济读,径乘氏县与济渠、濮渠合。北济自济阳县北,东北径煮枣城南。《郡国志》曰:冤朐县有煮枣城。即此也。汉高祖十二年,封革朱为侯国。北济又东北径冤朐县故城北,又东北径吕都县故城南,王莽更名之曰祁都也。又东北径定陶县故城北。汉景帝中六年,以济水出其北,东注分梁于定陶,置济阴国,指北济而定名也。又东北与濮水合。水上承济水于封丘县,即《地理志》所谓濮渠水首受济者也。阚駰曰:首受别济,即北济也。其故渎自济东北流,左迤为高梁陂,方三里。濮水又东径匡城北。孔子去卫适陈,遇难于匡者也。又东北,左会别濮,水受河于酸枣县。故杜预云:濮水出酸枣县,首受河。《竹书纪年》曰:魏襄王十年十月,大霖雨疾风,河水溢酸枣郛。汉世塞之,故班固云文堙枣野,今无水。其故读东北径南、北二棣城间。《左传》襄公五年,楚子囊伐陈,公会于城棣以救之者也。濮渠又东北径酸枣县故城南,韩国矣。圈称曰:昔天子建国名都,或以令名,或以山林,故豫章以树氏郡,酸枣以棘名邦,故曰酸枣也。《汉官仪》曰:旧河堤谒者居之城西,有韩王望气台,孙子荆《故台赋叙》曰:酸枣寺门外,夹道左右有两故台,访之故老云:韩王听讼观台,高十五仞,虽楼榭泯灭,然广基似于山岳。召公大贤,犹舍甘棠,区区小国,而台观隆崇,骄盈于世,以鉴来今,故作赋曰:蔑丘陵之逦迤,亚五岳之嵯峨。言壮观也。城北韩之市地也。聂政为濮阳严仲子刺韩相侠累,遂皮面而死,其姊哭之于此。城内有后汉酸枣令刘孟阳碑。濮水北积成陂,陂方五里,号曰同池陂。又东径胙亭东注。故胙国也。富辰所谓邢茅昨祭周公之胤也。濮渠又东北径燕城南。故南燕始姓之国也。有北燕,故以南氏县。东为阳青湖,陂南北五里,东西三十里,亦曰燕城湖,径桃城南,即《战国策》所谓酸枣、虚桃者也。汉高帝十二年,封刘襄为侯国。而东注于濮,俗谓之朝平沟。濮渠又东北,又与酸水故渎会。酸渎首受河于酸枣县,东径酸枣城北、延津南,谓之酸水。《竹书纪年》曰:秦苏胡率师代郑,韩襄败秦苏胡于酸水者也。酸渎水又东北径燕城北,又东径滑台城南,又东南径瓦亭南。《春秋》定公八年,公会晋师于瓦,鲁尚执羔,自是会始也。又东南会于濮,世谓之百尺沟。濮渠之侧有漆城。《竹书纪年》梁惠成王十六年,邯郸伐卫,取漆富丘城之者也。或亦谓之宛濮亭。《春秋》:宁武子与卫人盟于宛濮。杜预曰:长垣西南,近濮水也。京相璠曰:卫地也,似非关究,而不知其所。《竹书纪年》:梁惠成王五年,公子景贾率师伐郑,韩明战于阳,我师败遭。泽北有坛陵亭。亦或谓之大陵城,非所究也。又有桂城。《竹书纪年》:梁惠成王十七年,齐田期伐我东鄙,战于桂阳,我师败逋。亦曰桂陵。案《史记》:齐威王使田忌击魏,败之桂陵,齐于是强,自称为王,以今天下。濮渠又东径蒲城北,故卫之蒲邑。孔子将之卫,子路出于蒲者也。《韩子》曰:鲁以仲夏起长沟,子路为蒲宰,以私粟馈众。孔子使子贡毁其器焉。余按《家语》,言仲由为邱宰,修沟读,与之箪食瓢饮,夫子令赐止之,无鲁字。又人其境,三称其善,身为大夫,终死卫难。濮渠又东径韦城南,即白马县之韦乡也。史迁记曰:夏伯豕韦之故国矣。城西出而不方,城中有六大井,皆隧道下,俗谓之江井也。有驰道,自城属于长垣。濮渠东绝驰道,东径长垣县故城北。卫地也,故首垣矣。秦更从今名,王莽改为长固县。《陈留风俗传》曰:县有防垣,故县氏之。孝安帝以建光元年封元舅宋俊为侯国。县有祭城,濮渠径其北。郑大夫祭仲之邑也。杜预曰:陈留长垣县东北有祭城者也。圈称又言长垣县有罗亭,故长罗县也,汉封后将军常惠为侯国。《地理志》曰:王莽更长罗为惠泽,后汉省并。长垣有长罗泽,即吴季英牧猪处也。又有长罗冈、蘧伯玉冈。《陈留风俗传》曰:长垣县有蘧伯乡,一名新乡,有蘧亭伯玉祠、伯玉冢。曹大家《东征赋》曰:到长垣之境界兮,察农野之居民;睹蒲城之丘墟兮,生荆棘之蓁蓁。蘧氏在城之东南兮,民亦向其丘坟;惟令德之不朽兮,身既没而名存。昔吴季札聘上国,至卫,观典府宾亭父畴,以卫多君子也。濮渠又东分为二渎,北濮出焉。濮渠又东径须城北。《卫诗》云:思须与曹也。毛云:须,卫邑矣。郑云:自卫而东径邑,故思。濮渠又北径襄丘亭南。《竹书纪年》曰:襄王七年,韩明率师伐襄丘。九年,楚庶章率师来会我,次于襄丘者也。濮水又东径濮阳县故城南。昔师延为纣作靡靡之乐,武王伐纣,师延东走,自投濮水而死矣。后卫灵公将之晋,而设舍于濮水之上,夜闻新声,召师涓受之于是水也。濮水又东径济阴离狐县故城南,王莽之所谓瑞狐也。《郡国志》曰:故属东郡。濮水又东径霞密县故城北。《竹书纪年》:元公三年,鲁季孙会晋幽公于楚丘,取葭密,遂城之。濮水又东北径鹿城南。《郡国志》曰:济阴乘氏县有鹿县乡,《春秋》僖公二十一年,盟于鹿上。京、杜并谓此亭也。濮水又东与句渎合,读首受濮水枝渠于句阳县东南,径句阳县故城南。《春秋》之谷丘也。《左传》以为句渎之丘矣。县处其阳,故县氏焉。又东入乘氏县,左会濮水与济同入巨野,故《地理志》曰:濮水自濮阳南入巨野,亦《经》所谓济水自乘氏县两分,东北人于巨野也。济水故渎又北,右合洪水。水上承巨野薛训渚,历泽西北,又北径闞乡城西。《春秋》桓公十有一年,《经》书公会宋公于阚。《郡国志》曰:东平陆有阚亭。《皇览》曰:量尤冢在东郡寿张县阚乡城中,冢高七尺,常十月祠之,有赤气出如绛,民名为蚩尤旗。《十三州志》曰:寿张有蚩尤祠。又北与济渎合,自渚迄于北口百二十里,名曰洪水。桓温以太和四年率众北人,掘渠通济。至义熙十二年,刘武帝西入长安,又广其功。自洪口已上,又谓之桓公渎,济自是北注也。《春秋》庄公十八年,《经》书夏公追戎于济西。京相瑶曰:济水自巨野至济北是也。
又东北过寿张县西界,安民亭南,汶水从东北来注之。济水又北,汶水注之,戴延之所谓清口也。郭缘生《述证记》曰:清河首受洪水,北注济。或谓清即济也。《禹贡》:济东北会于汶。今枯渠注巨泽,巨泽北则清口,清水与汶会也。李钦曰:汶水出太山莱芜县,西南入济是也。济水又北径梁山东,袁宏《北征赋》曰:背梁山,截汶波。即此处也。刘澄之引是山以证梁父,为不近情矣。山之西南有吕仲悌墓。河东岸有石桥,桥本当河,河移,故厕岸也。古老言,此桥东海吕母起兵所造也。山北三里有吕母宅,宅东三里即济水。济水又北径须朐城西,城临侧济水,故须朐国也。《春秋》傅公二十一年,子鱼曰:任、宿、须朐、颛臾,风姓也。实司太皞,与有济之祀。杜预曰须朐在须昌县西北,非也。《地理志》曰:寿张西北有朐城者是也。济水西有安民亭,亭北对安民山,东临济水,水东即无盐县界也。山西有冀州刺史工纷碑,汉中平四年立。济水又北径微乡东。《春秋》庄公二十八年,《经》书冬筑郿。京相瑶曰:《公羊传》谓之微。东平寿张县西北三十里,有故微乡,鲁邑也。杜预曰:有微子冢。济水又北分为二水,其枝津西北出,谓之马颊水者也。
又北过须昌县西,京相璠曰:须朐,一国二城两名。盖迁都须昌,朐是其本。秦以为县,汉高帝十一年,封赵衍为侯国。济水于县,赵沟水注之。济水又北径鱼山东,左合马颊水。水首受济,西北流,历安民山北,又西流,赵沟出焉,东北注于济。马颊水又经桃城东。《春秋》桓公十年,《经》书公会卫侯于桃丘,卫地也。社预曰:济北东阿县东南有桃城,即桃丘矣。马颊水又东北流径鱼山南。山,即吾山也。汉武帝《瓠子歌》所谓吾山平者也。山上有柳舒城,魏东阿王曹子建每登之,有终焉之志。及其终也,葬山西,西去东阿城四十里。其水又东注于济,谓之马颊口也。济水自鱼山北径清亭东。《春秋》隐公四年,公及宋公遇于清。京相瑶曰:今济北东阿东北四十里有故清亭,即《春秋》所谓清者也。是下济水,通得清水之目焉。亦水色清深,用兼厥称矣。是故燕王曰:吾闻齐有清济,浊河以为固。即此水也。
又北过谷城县西,济水侧岸有尹卯垒,南去鱼山四十余里,是谷城县界。故《春秋》之小谷城也。齐桓公以鲁庄公二十三年城之,邑管仲焉,城内有夷吾井。《魏土地记》曰:县有谷城山,山出文石,阳各之地。《春秋》,齐侯、宋公会于阳谷者也。县有黄山台。黄石公与张子房期处也。又有狼水,出东南大槛山狼溪,西北径谷城西。又北有西流泉,出城东近山,西北径谷城北,西注狼水,以其流西,故即名焉。又西北人济水。城西北三里,有项王羽之冢,半许毁坏,石碣尚存,题云项王之墓。《皇览》云冢去县十五里,谬也。今彭城谷阳城西南,又有项羽冢,非也。余按史迁记,鲁为楚守,汉王示羽首,鲁乃降,遂以鲁公礼葬羽于谷城,宁得言彼也。济水又北径周首亭西。《春秋》文公十有一年,左丘明云:襄公二年,王子成父获长狄侨如弟荣如,埋其首于周首之北门,即是邑也。今世谓之卢子城,济北郡治也。京相瑶曰:今济北所治卢子城,故齐周首邑也。
又北过临邑县东,《地理志》曰:县有济水祠。王莽之谷城亭也。水有石门,以石为之,故济水之门也。《春秋》隐公五年,齐、郑会于石门,郑车偾济即于此也。京相璠曰:石门,齐地。今济北卢县故城西南六十里,有故石门,去水三百步,盖水渎流移,故侧岸也。济水又北径平阴城西。《春秋》襄公十八年,晋侯沉玉济河,会于鲁济,寻湨梁之盟,同伐齐,齐侯御诸平阴者也。杜预曰城在卢县故城东北,非也。京相璠曰:平阴,齐地也。在济北卢县故城西南十里。平阴城南有长城,东至海,西至济,河道所由,名防门,去平阴三里。齐侯堑防门,即此也。其水引济,故渎尚存。今防门北有光里,齐人言广,音与光同,即《春秋》所谓守之广里者也。又云,巫山在平阴东北,昔齐侯登望晋军,畏众而归。师旷、邢伯闻鸟乌之声,知齐师潜遁。人物咸沦,地理昭著,贤于杠氏东北之证矣。今巫山之上有石室,世谓之孝子堂。济水右迤,遏为湄湖,方四十余里。济水又东北径垣苗城西,故洛当城也。伏韬《北征记》曰:济水又与清河合流,至洛当者也。宋武帝西征长安,令垣苗镇此,故俗又有垣苗城之称。河水自四渎口东北流而为济。《魏土地记》曰:盟津河别流十里与清水合,乱流而东,径洛当城北,黑白异流,泾渭殊别,而东南流注也。
又东北过卢县北,济水东北与湄沟合,水上承湄湖,北流注济。《尔雅》曰:水草交曰湄,通谷者微。犍为舍人曰:水中有草木文合也。郭景纯曰:微,水边通谷也。《释名》曰:湄,眉也,临水如眉临目也。济水又径卢县故城北,济北郡治也。汉和帝永元二年,分泰山置,盖以济水在北故也。济水又径什城北。城际水湄,故邸阁也。祝阿人孙什,将家居之,以避时难,因谓之什城焉。济水又东北与中川水合,水东南出山在县之分水岭,溪一源两分,泉流半解,亦谓之分流交。半水南出大山人位,半水出山在县,西北流径东太原郡南,郡治山炉固,北与宾溪水合。水出南格马山宾溪谷,北径卢县故城北、陈敦戍南,西北流与中川水合,谓之格马口。其水又北径卢县故城东,而北流入济,俗谓之为沙沟水。济水又东北,右会玉水。水导源太山朗公谷,旧名琨瑞溪。有沙门竺僧朗,少事佛图澄,硕学渊通,尤明气纬,隐于此谷,因谓之朗公谷。故车频《秦书》云:符坚时,沙门竺僧朗,尝从隐士张巨和游,巨和常穴居,而朗居琨瑞山,大起殿舍,连楼累阁,虽素饰不同,并以静外致称。即此谷也。水亦谓之琨瑞水也。其水西北流径玉符山,又曰玉水。又西北径猎山东,又西北枕祝阿县故城东,野井亭西。《春秋》昭公二十五年,《经》书齐侯唁公于野井是也。《春秋》襄公十九年,诸侯盟于祝柯,《左传》所谓督阳者也。汉兴,改之曰阿矣。汉高帝十一年,封高邑为侯国,王莽之安成者也。故俗谓是水为祝阿涧水,北流注于济。建武五年,耿弇东击张步,从朝阳桥济渡兵,即是处也。济水又东北,泺水入焉。水出历城县故城西南,泉源上奋,水涌若轮。《春秋》桓公十八年,公会齐侯于泺是也。俗谓之为娥姜水,以泉源有舜妃娥英庙故也。城南对山,山上有舜祠,山下有大穴,谓之舜井,抑亦茅山禹井之比矣。《书》,舜耕历山,亦云在此,所未详也。其水北为大明湖,西即大明寺,寺东北两面侧湖,此水便成净他也。池上有客亭,左右楸桐,负日俯仰,目对鱼鸟,水木明瑟,可谓濠梁之性,物我无违矣。湖水引渎,东入西郭,东至历城西而侧城北注,陂水上承东城,历把下泉,泉源竞发。其水北流径历城东,又北,引水为流杯池,州僚宾燕,公私多萃其上。分为二水,右水北出,左水西径历城北,西北为陂,谓之历水,与泺水会。又北,历水枝津。首受历水于历城东,东北径东城西而北出郭,又北注泺水。又北,听水出焉。泺水又北流注于济,谓之泺口也。济水又东北,华不注山单椒秀泽,不连丘陵以自高;虎牙桀立,孤峰特拔以刺天。青崖翠发,望同点黛。山下有华泉,故京相璠《春秋土地名》曰:华泉,华不注山下泉水也。《春秋左传》成公二年,齐顷公与晋郤克战于鞍,齐师败绩,逐之,三周华不注,逢丑父与公易位,将及华泉,骖絓于木而止。丑父使公下,如华泉取饮,齐侯以免。韩厥献丑父,郤子将戮之。呼曰:自今无有代其君任患者,有一于此,将为戮矣。郤子曰:人不难以死免其君,我戮之不祥,赦之以劝事君者。乃免之。即华水也。北绝听渎二十里,注于济。
又东北过台县北,巨合水南出鸡山西北,北径巨合故城西。耿弇之讨张步也,守巨里。即此城也。三面有城,西有深坑,坑西即弇所营也,与费邑战,斩邑于此。巨合水又北合关卢水,水导源马耳山,北径博亭城西,西北流至平陵城,与武原水合。水出谭城南平泽中,世谓之武原渊。北径谭城东,俗谓之布城也。又北径东平陵县故城西。故陵城也,后乃加平,谭国也。齐桓之出过谭,谭不礼焉,鲁庄公九年即位,又不朝,十年灭之。城东门外有乐安任照先碑,济南郡治也,汉文帝十六年,置为王国,景帝二年为郡,王莽更名乐安。其水又北径巨合城东,汉武帝以封城阳顷王子刘发为侯国。其水合关卢水,西出注巨合水。巨合水西北径台县故城南。汉高帝六年,封东郡尉戴野为侯国,王莽之台治也。其水西北流,白野泉水注之,水出台城西南白野泉北,径留山西北流,而右注巨合水。巨合水又北,听水注之;水上承泺水,东流北屈,又东北流,注于巨合水,乱流又北入于济。济水又东北,合芹沟水,水出台县故城东南,西北流,径台城东,又西北入于济水。
又东北过管县南。
济水东径县故城南。汉文帝四年,封齐悼惠王子罢军为侯国。右纳百脉水,水出土鼓县故城西,水源方百步,百泉俱出,故谓之百脉水。其水西北流,径阳丘县故城中。汉孝文帝四年,以封齐悼惠王子刘安为阳丘侯,世谓之章丘城,非也。城南有女郎山,山上有神祠,俗谓之女郎祠,左右民祀焉。其水西北出城,北径黄巾固。盖贼所屯,故固得名焉。百脉水又东北流注于济。济水又东,有杨渚沟水,出逢陵故城西南二十里,西北径土鼓城东,又西北径章丘城东,又北径宁戚城西,而北流注于济水也。
又东过梁邹县北,陇水南出长城中,北流至般阳县故城西,南与般水会,水出县东南龙山,俗亦谓之为左阜水,西北径其城南。王莽之济南亭也。应劭曰:县在般水之阳,故资名焉。其水又南屈,西人陇水。陇水北径其县,西北流至萌水口,水出西南甲山,东北径萌山西,东北入于陇水。陇水又西北至梁邹东南,与鱼子沟水合,水南出长白山东柳泉口。山,即陈仲子夫妻之所隐也。《盂子》曰:仲子,齐国之世家,兄戴禄万钟,仲子非而不食,避兄离母,家于於陵,即此处也。其水又径於陵县故城西,王莽之於陆也。世祖建武十五年,更封则乡侯侯霸之子昱为侯国。其水北流注于陇水,陇水,即古袁水也。故京相瑶曰:济南梁邹县有袁水者也。陇水又西北径梁邹县故城南,又北屈径其城西。汉高祖六年,封武虎为侯国,其水北注济。城之东北,又有时水西北注焉。
又东北过临济县南,县,故狄邑也,王莽更名利居。《汉记》,安帝永初二年,改从今名,以临济故。《地理风俗记》云:乐安太守治。晏谟《齐记》曰:有南北二城隔济水,南城即被阳县之故城也,北枕济水。《地理志》曰:侯国也。如淳曰:一作疲,音罢,军之罢也。《史记·建元以来王子侯者年表》曰:汉武帝元朔四年,封齐孝上子敬侯刘燕之国也。今渤海侨郡治。济水又东北,迆为渊诸,谓之平州。漯沃县侧有平安故城,俗谓之会城,非也。按《地理志》,千乘郡有平安县,侯国也,王莽曰鸿睦也。应劭曰:博昌县西南三十里有平安亭,故县也。世尚存平州之名矣。济水又东北径高昌县故城西。案《地理志》,千乘郡有高昌县,汉宣帝地节四年,封董忠为侯国。世谓之马昌城,非也。济水又东北径乐安县故城南。伏琛《齐记》曰:博昌城西北五十里有南、北二城,相去三十里,隔时、济二水。指此为博昌北城,非也。乐安与博昌、薄姑分水,俱同西北,薄姑去齐城六十里,乐安越水差远,验非尤明。班固曰:千乘郡有乐安县。应劭曰:取休令之名矣。汉武帝元朔五年,封李蔡为侯国。城西三里有任光等冢,光是宛县人,不得为博昌明矣。济水又径薄姑城北。《后汉郡国志》曰:博昌县有薄姑城。《地理书》曰:吕尚封于齐郡薄姑。薄姑故城在临淄县西北五十里,近济水。史迁曰:献公徙薄姑。城内有高台。《春秋》昭公二十年,齐景公饮于台上,曰:古而不死,何乐如之。晏平仲对曰:昔爽鸠氏始屠之,季萴因之,有逢伯陵又因之,薄姑氏又因之,而后太公因之。臣以为古若不死,爽鸠氏之乐,非君之乐。即于是台也。济水又东北径狼牙固西而东北流也。
又东北过利县西,《地理志》,齐郡有利县,王莽之利治也。晏谟曰:县在齐城北五十里也。
又东北过甲下邑,入于河。
济水东北至甲下邑南,东历琅槐县故城北。《地理风俗记》曰:博昌东北八十里有琅槐乡,故县也。《山海经》曰:济水绝巨野注渤海,入齐琅槐东北者也。又东北,河水枝津注之,《水经》以为入河,非也。斯乃河水注济,非济入河。又东北入海。郭景纯曰:济自荣阳至乐安博昌入海。今河竭,济水仍流不绝,《经》言入河,二说并失。然河水于济、漯之北,别流注海。今所辍流者,惟漯水耳,郭或以为济注之,即实非也。寻经脉水,不如《山经》之为密矣。
其一水东南流者,过乘氏县南,菏水分济于定陶东北,东南右合黄沟枝流,俗谓之界沟也。北径已氏县故城西,又北径景山东。《卫诗》所谓景山与京者也。毛公曰:景山,大山也。又北径楚丘城西。《郡国志》曰:成武县有楚丘亭。杜顶云,楚丘在成武县西南,卫懿公为狄所灭,卫文公东徙渡河,野处曹邑,齐桓公城楚丘以迁之。故《春秋》称邢迁如归,卫国忘亡。即《诗》所谓升彼虚矣,以望楚矣,望楚与堂,景山与京。故郑玄言,观其旁邑及山川也。又东北径成武城西,又东北径郈城东。疑郈徙也,所未详矣。又东北径梁丘城西。《地理志》曰:昌邑县有梁丘乡。《春秋》庄公三十二年,宋人、齐人会于梁丘者也。杜预曰:高平昌邑县西南有梁丘乡。又东北于乘氏县西而北注菏水,菏水又东南径乘氏县故城南。县,即《春秋》之乘丘也。故《地理风俗记》曰:济阴乘氏县,故宋乘丘邑也。汉孝景中五年,封梁孝王子买为侯国也。《地理志》曰:乘氏县,泗水东南至唯陵入淮。《郡国志》曰:乘氏有泗水。此乃菏泽也。《尚书》有导菏泽之说,自陶丘北,东至于菏,无泗水之文。又曰:导菏泽,被孟猪。孟猪在睢阳县之东北.阚駰《十三州记》曰:不言入而言被者,明不常入也。水盛,方乃覆被矣。泽水淼漫,俱钟淮泗,故志有睢陵入淮之言,以通苞泗名矣。然诸水注泗者多不止此,可以终归泗水,便得擅通称也。或更有泗水亦可是水之兼其目,所未详也。
又东过昌邑县北,菏水又东径昌邑县故城北。《地理志》曰:县,故梁也。汉景帝中六年,分梁为山阳国,武帝天汉四年,更为昌邑国,以封昌邑王髆,贺废,国除,以为山阳郡,王莽之巨野郡也。后更为高平郡,后汉沇州治。县令王密,怀金谒东莱太守杨震,震不受,是其慎四知处也。大城东北有金城,城内有杭州刺史河东薛季像碑,以郎中拜判令,甘露降园。熹平四年迁州,明年甘露复降殿前树,从事冯巡、主簿华操等相与褒树,表勒棠政。次西有沇州刺史茂陵杨叔恭碑,从事孙光等以建宁四年立,西北有东太山成人班孟坚碑,建和十年,尚书右丞拜沇州刺史从事秦闰等,刊石颂德政,碑咸列焉。
又东过金乡县南,《郡国志》曰:山阳有金乡县。菏水径其故城南,世谓之故县城北有金乡山也。
又东过东绢县北,菏水又东径汉平狄将军扶沟侯淮阳朱鲔冢。墓北有石庙。菏水又东径东缗具故城北,故宋地。《春秋》僖公二十三年,齐侯伐宋围缗。《十三州记》曰:山阳有东缗县,邹衍曰:余登缗城以望宋都者也。后汉世祖建武十一年,封冯异长子璋为侯国。
又东过方与县北,为菏水。
菏水东径重乡城南,《左传》所谓臧文仲宿于重馆者也。菏水又东径武棠亭北,《公丰》以为济上邑也。城有台,高二丈许,其下临水,昔鲁侯观鱼于棠,谓此也。在方与县故城北十里。《经》所谓菏水也。菏水又东径泥母亭北。《春秋左传》僖公七年,秋,盟于宁母,谋伐郑也。菏水又东与巨野黄水合,菏泽别名也。黄水上承巨泽诸陂,泽有蒙淀、盲陂。黄湖水东流,谓之黄水,又有薛训诸水,自渚历薛村前,分为二流,一水东注黄水,一水西北入泽,即洪水也。黄水东南流,水南有汉荆州刺史李刚墓。刚字叔毅,山阳高平人,裹平元年卒。见其碑。有石阙、祠堂、石室三间,椽架高丈余,镂石作椽瓦屋,施平天造,方井侧荷梁柱,四壁隐起,雕刻为君臣、官属,龟龙、鳞凤之文,飞禽、走兽之像,作制工丽,不甚伤毁。黄水又东径巨野县北。何承天曰,巨野湖泽广大,南通洙、泗,北连清、济,旧县故城正在泽中,故欲置戍于此城,城之所在,则巨野泽也。衍东北出为大野矣,昔西狩获鳞于是处也。《皇览》曰:山阳巨野县有肩髀冢,重聚大小,与髀冢等。传言蚩尤与黄帝战,克之于涿鹿之野,身体异处,故别葬焉。黄水又东径咸亭北。《春秋》桓公七年,《经》书焚咸丘者也。水南有金乡山,县之东界也。金乡数山,皆空中穴口,谓之隧也。戴延之《西征记》曰:焦氏山北数里,汉司隶校尉鲁峻穿山得白蛇、白兔,不葬,更葬山南,凿而得金,故曰金乡山。山形峻峭,篆前有石祠、石庙,四壁皆青石隐起,自书契以来,忠臣、孝子、贞妇、孔子及弟子七十二人形像,像边皆刻石记之,文字分明,又有石床,长八尺,磨莹鲜明,叩之声闻远近。时太尉、从事中郎傅珍之咨议参军周安穆拆败石床,各取去,为鲁氏之后所讼,二人并免官。焦氏山东即金乡山也,有冢,谓之秦王陵,山上二百步得冢口,堑深十丈,两壁峻峭,广二丈,入行七十步,得埏门,门外左右皆有空,可容五六十人,谓之白马空,埏门内二丈,得外堂,外堂之后,又得内堂,观者皆执烛而行,虽无他雕镂,然治石甚精,或云是汉昌邑哀王冢,所未详也。东南有范巨卿冢,名件犹存。巨卿名式,山阳之金乡人,汉荆州刺史,与汝南张劭、长沙陈平子石交,号为死友矣。黄水又东南径任城郡之亢父县故城西,夏后氏之任国也。汉章帝元和元年,别为任城在北,王莽之延就亭也。县有诗亭,《春秋》之诗国也,王莽更之曰顺父矣。《地理志》,东平属县也。世祖建武二年,封刘隆为侯国。其水谓之桓公沟,南至方与县,入于菏水。菏水又东径秦梁,夹岸积石一里,高二丈,言秦始皇东巡所造,因以名焉。菏水又东过湖陆县南,东入干泗水。
泽水所钟也。《尚书》曰:浮于淮、泗,达于菏是也。《东观汉记》曰:苏茂杀淮阳太守,得其郡,营广乐,大司马吴汉围茂,茂将其精兵突至湖陵,与刘永相会,济阴山阳济兵于此处也。
又东南过沛县东北,济与泗乱,故济纳互称矣。《东观汉记·安平侯盖延传》曰:延为虎牙大将军,与永等战,永军反走,溺水者半,复与战,连破之,遂平沛、楚,临淮悉降。延令沛修高祖庙,置啬夫、祝宰、乐人,因斋戒祠高庙也。
又东南过留县北,留县故城,翼佩泗、济,宋邑也,《春秋左传》所谓侵宋吕留也。故繁休伯《避地赋》曰:朝余发乎泗洲,夕余宿于留乡者也。张良委身汉祖,始自此矣,终亦取封焉。城内有张良庙也。
又东过彭城县北,获水从西来注之。
济水又南径彭城县故城东北隅,不东过也。获水自西注之,城北枕水湄。济水又南径彭城县故城东,不径其北也,盖《经》误证。
又东南过徐县北,《地理志》曰:临淮郡,汉武帝元狩五年置,治徐县,王莽更之曰淮平,县曰徐调,故徐国也。《春秋》昭公三十年,吴子执钟吾子,遂伐徐,防山以水之,遂灭徐。徐子奔楚,楚救徐弗及,遂城夷以处之。张华《博物志》录著作令史茅温所为送。刘成国《徐州地理志》云:徐偃王之异言,徐君宫人娠而生卵,以为不祥,弃之于水滨。孤独母有犬,名曰鹄仓,猎于水侧,得弃卵,衔以来归,孤独母以为异,覆暖之,遂成儿,生时偃,故以为名。徐君宫中闻之,乃更录取,长而仁智,袭君徐国。后鹄仓临死,生角而九尾,实黄龙也。僵王葬之徐中,今见有狗垄焉。偃王治国,仁义著闻,欲舟行上国,乃通沟陈、蔡之间。得朱弓矢,以得天瑞,遂因名为号,自称徐偃王,江、淮诸侯服从者三十六国。周王闻之,遣使至楚,令伐之,偃王爱民不斗,遂为楚败,北走彭城武原县东山下,百姓随者万数。因名其山为徐山,山上立石室庙,有神灵,民人请祷焉。依文即事,似有符验,但世代绵远,难以详矣。今徐城外有徐君墓,昔延陵季子解剑于此,所谓不违心许也。
又东至下邳睢陵县南,入于淮。
济水与泗水,浑涛东南流,至角城,同入淮。《经》书睢陵,误耳。


卷九 清水、沁水、淇水、荡水、洹水 
清水出河内修武县之北黑山,黑山在县北白鹿山东,清水所出也,上承诸陂散泉,积以成川。南流西南屈,瀑布乘岩,悬河注壑二十余丈,雷赴之声,震动山谷。左右石壁层深,兽迹不交,隍中散水雾合,视不见底。南峰北岭,多结禅栖之士,东岩西谷,又是刹灵之图,竹柏之怀,与神心妙远,仁智之性,共山水效深,更为胜处也。其水历涧飞流,清冷洞观,谓之清水矣。溪曰瑶溪,又曰瑶涧。清水又南,与小瑶水合,水近出西北穷溪,东南流注清水。清水又东南流,吴泽陂水注之,水上承吴陂于修武县故城西北。修武,故宁也,亦曰南阳矣。马季长曰:晋地自朝歌以北至中山为东阳,朝歌以南至轵为南阳。故应劭《地理风俗记》云:河内殷国也,周名之为南阳。又曰:晋始启南阳。今南阳城是也。秦始皇改曰修武,徐广、王隐并言始皇改。瓒注《汉书》云:案《韩非书》,秦昭王越赵长平,西代修武,时秦未兼天下,修武之名久矣。余案《韩诗外传》言,武王伐纣,勒兵于宁,更名宁曰修武矣,魏献子田大陆还,卒于宁是也。汉高帝八年,封都尉魏遬为侯国,亦曰大修武。有小,故称大。小修武在东,汉祖与滕公济自玉门津,而宿小修武者也。大陆即吴泽矣。《魏土地记》曰:修武城西北二十里有吴泽水。陂南北二十许里,东西三十里,西则长明沟入焉。水有二源,北水上承河内野王具东北界沟,分枝津为长明沟。东径雍城南,寒泉水注之,水出雍城西北,泉流南注,径雍城西。《春秋》僖公二十四年,王将以狄伐郑,富辰谏曰:雍,文之昭也。京相璠曰:今河内山阳西有故雍城。又东南注长明沟,沟水又东径射犬城北。汉大司马张扬为将杨丑所害,眭固杀丑屯此,欲北合袁绍。《典略》曰:眭固字白菟。或戒固曰:将军字菟,而此邑名犬,菟见大,其势必惊,宜急去,固不从。汉建安四年,魏太祖斩之干此。以魏种为河内太守,守之。沇州叛,太祖曰:惟种不弃孤。及走,太祖怒曰:种不南走越,北走胡,不汝置也。射犬平,禽之。公曰:惟其才也。释而用之。长明沟水东入石涧,东流,蔡沟水入焉,水上承州县北,白马沟东分,谓之蔡沟。东会长明沟水,又东径修武县之吴亭北,东入吴陂。次北有苟泉水入焉,水出山阳县故修武城西南,同源分派,裂为二水。南为苟泉,北则吴渎,二渎双导,俱东入陂。山阳县东北二十五里有陆真阜,南有皇母、马鸣二泉,东南合注于吴陂也。次陆真阜之东北,复覆釜堆,堆南有三泉,相去四五里,参差次合,南注于陂。泉在浊鹿城西,建安二十五年,魏封汉献帝为山阳公。浊鹿城,即是公所居也。陂水之北际泽,侧有隤城。《春秋》隐公十一年,王以司寇苏忿生之田,攒茅、隤十二邑与郑者也。京相瑶曰:河内修武县北有故隤城,实中。今世俗谓之皮垣,方四百步,实中,高八丈。际陂,北隔水十五里,俗所谓兰丘也,方二百步。西十里又有一丘际山,世谓之敕丘,方五百步,形状相类,疑即古攒茅也。杜预曰:二邑在修武县北,所未详也。又东,长泉水注之,源出白鹿山东南,伏流径十三里,重源浚发于邓城西北,世亦谓之重泉水也。又径七贤祠东。左右筠篁列植,冬夏不变贞萎,魏步兵校尉陈留阮籍,中散大夫谯国嵇康,晋司徒河内山涛,司徒琅邪王戎,黄门郎河内向秀,建威参军沛国刘伶,始平太守阮咸等,同居山阳,结自得之游,时人号之为竹林七贤。向子期所谓山阳旧居也,后人立庙于其处。庙南又有一泉,东南流注于长泉水。郭缘生《述征记》所云,白鹿山东南二十五里有嵇公故居,以居时有遗竹焉,盖谓此也。其水又南径邓城东,名之为邓渎,又谓之为白屋水也。昔司马懿征公孙渊还达白屋,即于此也。其水又东南流径隤城北,又东南历泽注于陂。陂水东流,谓之八光沟,而东流注于清水,谓之长清河,而东周永丰坞,有丁公泉发于焦泉之右。次东得焦泉,泉发于天门之左、天井固右。天门山石自空,状若门焉,广三丈,高两匹,深丈余,更无所出,世谓之天门也。东五百余步,中有石穴西向,裁得客人,东南入,径至天井,直上三匹有余,扳蹑而升,至上平,东西二百步,南北七百步,四面险绝,无由升陟矣。上有比丘释僧训精舍,寺有十余僧,给养难周,多出下平,有志者居之。寺左右杂树疏颁,有一石泉,方丈余,清水湛然,常无增减,山居者资以给饮。北有石室二口,旧是隐者念一之所,今无人矣。泉发于北阜,南流成溪,世谓之焦泉也。次东得鱼鲍泉,次东得张波泉,次东得三渊泉,梗河参连,女宿相属,是四川在重门城西并单川南注也。重门城,昔齐王芳为司马师废之,宫于此,即《魏志》所谓送齐王于河内重门者也。城在共县故城西北二十里。城南有安阳陂,次东又得卓水陂,次东有百门陂,陂方五百步,在共县故城西。汉高帝八年,封卢罢师为共侯,即共和之故国也。共伯既归帝政,逍遥于共山之上。山在国北,所谓共北山也,仙者孙登之所处,袁彦伯《竹林七贤传》,嵇叔夜尝采药山泽,遇之于山,冬以被发自覆,夏则编草为裳,弹一弦琴,而五声和。其水三川南合,谓之清川。又南径凡城东。司马彪、袁山松《郡国志》曰:共县有凡亭,周凡伯国。《春秋》隐公七年,《经》书,王使凡伯来聘是也。杜预曰:汲郡共县东南有凡城。今在西南。其水又西南与前四水总为一渎,又谓之陶水,南流注于清水。清水又东周新丰坞,又东注也。
东北过获嘉县北,《汉书》称越相吕嘉反,武帝元鼎六年,巡行于汲郡中乡,得吕嘉首,因以为获嘉县。后汉封侍中冯石为侯国。县故城西有汉桂阳太守赵越墓,冢北有碑。越字彦善,县人也。累迁桂阳郡、五官将、尚书仆射,遭忧服阕,守河南尹,建宁中卒。碑东又有一碑,碑北有石柱、石牛、羊、虎俱碎,沦毁莫记。清水又东周新乐城,城在获嘉县故城东北,即汲之新中乡也。
又东过汲县北,县,故汲郡治,晋太康中立。城西北有石夹水,飞湍浚急,人亦谓之磻溪,言太公尝钓于此也。城东门北侧有太公庙,庙前有碑,碑云:太公望者,河内汲人也。县民故会稽太守杜宣白令崔瑗曰:太公本生于汲,旧居犹存。君与高、国同宗,太公载在《经》《传》,今临此国,宜正其位,以明尊祖之义。于是国老王喜,廷掾郑笃,功曹邠勤等,咸曰:宜之。遂立坛祀,为之位主。城北三十里,有太公泉。泉上又有太公庙,庙侧高林秀木,翘楚竞茂,相传云:太公之故居也。晋太康中,范阳卢无忌为汲令,立碑于其上。太公避纣之乱,屠隐市朝,遁钓鱼水,何必渭滨,然后磻溪?苟惬神心,曲渚则可,磻溪之名,斯无嫌矣。清水又东径故石梁下,梁跨水上,桥石崩褫,余基尚存。清水又东与仓水合。水出西北方山,山西有仓谷,谷有仓玉、珉石,故名焉。其水东南流,潜行地下,又东南复出,俗谓之雹水。东南历坶野,自朝歌以南,南暨清水,土地平衍,据皋跨泽,悉坶野矣。《郡国志》曰:朝歌县南有牧野。《竹书纪年》曰:周武王率西夷诸侯伐殷,败之于坶野。《诗》所谓坶野洋洋,檀车煌煌者也。有殷大夫比于冢,前有石铭,题隶云:殷大夫比于之墓。所记惟此。今已中折,不知谁所志也。太和中,高祖孝文皇帝南巡,亲幸其坟,而加吊焉,刊石树碑,列于墓隧矣。雹水又东南入于清水。清水又东南径合城南,故三会亭也,似淇、清合河,放受名焉。清水又屈而南径凤皇台东北南注也。
又东入于河。
谓之清口,即淇河口也,盖互受其名耳。《地理志》曰:清河水出内黄县南,无清水可来,所有者惟钟是水耳。盖河徒南注,清水渎移,汇流径绝,余目尚存。故东川有清河之称,相嗣不断,曹公开白沟,遏水北注,方复故渎矣。
沁水出上党涅县谒戾山,沁水即涅水也。或言出谷远县羊头山靡谷,三源奇注,径泻一隍。又南会三水,历落出左右近溪,参差翼注之也。
南过谷远县东,又南过陭氏县东,谷远县,王莽之谷近也。沁水又南径陭氏县故城东,刘聪以詹事鲁繇为冀州,治此也。沁水又南历陭氏关,又南与水合,水出东北巨骏山,乘高泻浪,触石流响,世人因声以纳称。西南流注于沁。沁水又南与秦川水合,水出巨骏山东,带引众溪,积以成川,又西南径端氏县故城东。昔韩、赵、魏分晋,迁晋君于端氏县,即此是也。其水南流,入于沁水。
又南过阳阿县东,沁水南径阳阿县故城西。《魏土地记》曰:建兴郡治阳阿县。郡西四十里有沁水南流。沁水又南与濩泽水合,水出濩泽城西白涧岭下,东径濩泽。《墨子》曰:舜渔濩泽。应劭曰:泽在县西北,又东径濩泽县故城南,盖以泽氏县也。《竹书纪年》,梁惠成王十九年,晋取玄武、濩泽者也。其水际城东注,又东合清渊水,水出其县北,东南径濩泽城东,又南入于泽水。泽水又东得阳泉口,水出鹿台山。山上有水,渊而不流,其水东径阳陵城南,即阳阿县之故城也。汉高帝六年,封卞訢为侯国。水历嶕峣山东,下与黑岭水合,水出西北黑岭下,即开瞪也。其水东南流径北乡亭下,又东南径阳陵城东,南注阳泉水。阳泉水又南注濩泽水。泽水又东南,有上涧水注之,水导源西北辅山,东径铜于崖南,历析城山北,山在濩泽南,《禹贡》所谓砥柱、析城,至于王屋也。山甚高峻,上平坦。下有二泉,东浊西清,左右不生草木,数十步外多细竹。其水自山阴东入濩泽水。濩泽水又东南注于沁水。沁水又东南,阳阿水左入焉,水北出阳阿川,南流径建兴郡西,又东南流径午壁亭东,而南入山。其水沿波漱石,漰涧八丈,环涛毂转,西南流入于沁水。沁水又南五十余里,沿流上下,步径裁通,小竹细笋,被于山渚,蒙茏茂密,奇为翳荟也。
又南出山,过沁水县北,沁水甫径石门,谓之沁口。《魏土地记》曰:河内郡野王县西七十里有沁水,左径沁水城西,附城东南流也。石门是晋安平献王司马孚之为魏野王典农中郎将之所造也。按其表云:臣孚言,臣被明诏,兴河内水利。臣既到,检行沁水,源出铜鞮山,屈曲周回,水道九百里,自太行以西,王屋以东,层岩高峻,天时霖雨,众谷走水,小石漂迸,木门朽败,稻田泛滥,岁功不成。臣辄按行,去堰五里以外,方石可得数万余枚。臣以为累方石为门,若天旸旱,增堰进水,若天霖雨。陂泽充溢,则闭防断水,空渠衍涝,足以成河。云雨由人,经国之谋,暂劳永逸,圣王所许,愿陛下特出臣表,敕大司农府给人工,勿使稽延,以赞时要。臣孚言。诏书听许。于是夹岸累石,结以为门,用代木门枋,故石门旧有枋口之称矣。溉田顷亩之数,间二岁月之功,事见门侧石铭矣。水西有孔山,山上石穴洞开,穴内石上,有车辙、牛迹。《耆旧传》云:自然成著,非人功所就也。其水南分为二水,一水南出,为朱沟水。沁水又径沁本县故城北,盖藉水以名县矣。《春秋》之少水也。京相璠曰:晋地矣。又云:少水,今沁水也。沁水又东径沁水亭北,世谓之小沁城。沁水又东,右合小沁水,水出北山台淳渊,南流为台淳水,东南入沁水。沁水又东,倍涧水注之,水北出五行之山,南流注于沁水。
又东过野王县北,沁水又东,邗水注之,水出太行之阜山,即五行之异名也。《淮南子》曰:武王欲筑宫于五行之山,周公曰:五行险固,德能覆也,内贡回矣,使吾暴乱,则伐我难矣。君子以为能持满。高诱云:今太行山也,在河内野王县西北上党关。《诗》所谓徒殆野王道,倾盖上党关。即此由矣。其水南流径邗城西,故邗国也。城南有邗台,《春秋》值公二十四年,王将伐郑,富辰谏曰:邗,武之穆也,京相璠曰:今野王西北三十里有故邗城,邗台是也。今故城当太行南路,道出其中,汉武帝封李寿为侯国。邗水又东南径孔子庙东。庙庭有碑,魏太和元年,孔灵度等以旧字毁落,上求修复。野王令范众爱、河内太守元真、刺史咸阳公高允表闻,立碑于庙。抬中刘明、别驾吕次文、主簿向班虎、荀灵龟,以宣尼大圣,非碑颂所称,宜立记焉。云仲尼伤道不行,欲北从赵鞅,闻杀鸣铎,遂旋车而反,及其后也,晋人思之,于太行岭南为之立庙,盖往时回辕处也。余按诸子书及史籍之文,并言仲尼临河而叹曰:丘之不济,命也。夫是非太行回辕之言也。碑云:鲁国孔氏,官于洛阳,因居庙下,以奉蒸尝。斯言是矣。盖孔氏迁山下,追思圣祖,故立庙存飨耳。其犹刘累迁鲁,立尧祠于山矣,非谓回辕于此也。邗水东南径邗亭西。京相璠曰:又有亭在台西南三十里,今是亭在邗城东南七、八里,盖京氏之谬耳。或更有之,余所不详。其水又南流注于沁,沁水东径野工县故城北。秦昭王四十四年,白起攻太行,道绝而韩之野王降。始皇拔魏东地,置东郡,卫元君自濮阳徙野王,即此县也。汉高帝无年,为殷国,二年为河内郡,王莽之后队,县曰平野矣。魏怀州刺史治,皇都迁洛,省州复郡。水北有华岳庙,庙侧有攒柏数百根,对郭临川,负冈荫诸,青青弥望,奇可玩也。怀州刺史顿丘李洪之之所经构也。庙有碑焉,是河内郡功曹山阳荀灵龟以和平四年造,天安元年立。沁水又东,朱沟枝津入焉。又东与丹水合,水出上党高都具故城东北阜下,俗谓之源源水。《山海经》曰:沁水之东有林焉,名曰丹林,丹水出焉。即斯水矣。丹水自源东北流,又屈而东注,左会绝水。《地理志》曰:高都县有莞谷,丹水所出,东南入绝水是也。绝水出泫氏县西北杨谷,故《地理志》曰:杨谷,绝水所出。东南流,左会长平水,水出长平县西北小山,东南流径其县故城。泫氏之长平亭也。《史记》曰:秦使左庶长王齕攻韩,取上党,上党民走赵,赵军长平,使廉颇为将,后遣马服君之子赵括代之。秦密使武安君白起攻之,括四十万众降起,起坑之于此。《上党记》曰:长平城在郡之南,秦垒在城西,二军共食流水,涧相去五里。秦坑赵众,收头颅筑台于垒中,因山为台,崔嵬桀起,今仍号之曰白起台。城之左右沿山亘隰,南北五十许里,东西二十余里,悉秦、赵故垒,遗壁旧存焉。汉武帝元朔二年,以封将军卫青为侯国。其水东南流,注绝水。绝水又东南流径泫氏县故城北。《竹书纪年》曰:晋烈公元年,赵献子城泫氏。绝水东南与泫水会,水导源县西北该谷,东流径一故城南,俗谓之都乡城。又东南径泫氏县故城南,世祖建武六年,封万普为侯国。而东会绝水,乱流东南入高都县,右入丹水。《上党记》曰:长平城在郡南山中,丹水出长平北山,南流。秦坑赵众,流血丹川,由是俗名为丹水,斯为不经矣。丹水又东南流注于丹谷,即刘越石《扶风歌》所谓丹水者也。《晋书地道记》曰:县有太行关,丹溪为关之东谷,途自此去,不复由关矣。丹水又径二石人北,而各在一山,角倚相望,南为河内,北曰上党。二郡以之分境。丹水又东南历西岩下,岩下有大泉涌发,洪流巨输,渊深不测,濒藻茭芹,竟川含绿,虽严辰肃月,无变暄萋。丹水又南,白水注之,水出高都县故城西,所谓长平白水也,东南流历天井关。《地理志》曰:高都县有天井关。蔡邕曰:太行山上有天井关,在井北,遂因名焉。故刘歆《遂初赋》曰:驰太行之险峻,入天井之高关,太元十五年,晋征虏将军朱序破慕容永于太行,遣军至白水,去长子百六十里。白水又东,天井溪水会焉。水出天井关,北流注白水,世谓之北流泉。白水又东南流入丹水,谓之白水交。丹水又东南出山,径鄈城西。城在山际,俗谓之期城,非也。司马彪《郡国志》曰:山阳有鄈城。京相璠曰:河内山阳西北六十里有鄈城。《竹书纪年》曰:梁惠成王元年,赵成侯偃、韩懿侯若伐我葵。即此城也。丹水又南屈而西转,光沟水出焉。丹水又西径苑乡城北,南屈东转,径其城南,东南流注子沁,谓之丹口。《竹书纪年》曰:晋出公五年,丹水三日绝不流;幽公九年,丹水出相反击。即此水也。沁水又东,光沟水注之,水首受丹水,东南流,界沟水出焉,又南人沁水。沁水又东南流径成乡城北,又东径中都亭南,左合界沟水,水上承光沟,东南流,长明沟水出焉。又南径中都亭西,而南流注于沁水也。
又东过州县北,县,故州也。《春秋左传》隐公十有一年,周以赐郑公孙段。韩宣子徙居之。有白马沟水注之,水首受白马湖,湖一名朱管陂,陂上承长明沟。湖水东南流,径金亭西,分为二水,一水东出为蔡沟,一水南注于沁也。
又东过怀县之北,《韩诗外传》曰:武王伐纣到邢丘,更名邢丘曰怀。春秋时,赤翟伐晋围怀是也。王莽以为河内,故河内郡治也。旧三河之地矣。韦昭曰:河南、河东、河内为三河也。县北有沁阳城,沁水径其甫而东注也。
又东过武德县甫,又东南至荣阳县北,东入于河。
沁水于县南,水积为陂,通结数湖,有朱沟水注之,其水上承沁水于沁水县西北,自枋口东南流,奉沟水右出焉。又东南流,右泄为沙沟水也。其水又东南,于野工城西,枝渠左出焉,以周城溉,东径野工城南,又屈径其城东而北注沁水。朱沟自枝渠东南,径州城南,又东径怀城南,又东径殷城北.郭缘生《述征记》曰:河之北岸,河内怀县有殷城。或谓楚、汉之际,殷王印治之,非也。余案《竹书纪年》云:秦师伐郑,次于怀,城殷。即是城也,然则殷之为名久矣,知非从印始。昔刘曜以郭默为殷州刺史,督缘河诸军事,治此。朱沟水又东南注于湖。湖水右纳沙沟水,水分朱沟南派,东南径安昌城西。汉成帝河平四年,封丞相张禹为侯国。今城之东南有古冢,时人谓之张禹墓。余按《汉书》,禹,河内轵人,徒家莲勺,鸿嘉元年,禹以老乞骸骨,自治冢奎,起祠堂于平陵之肥牛亭,近延陵,奏请之,诏为徒亭,哀帝建平二年薨,遂葬于彼,此则非也。沙沟水又东径隰城北.《春秋》值公二十五年,取太叔于温,杀之于隰城是也。京相璠曰:在怀县西南。又径殷城西,东南流入于陂,陂水又值武德县,南至荥阳县北,东南流入于河。先儒亦咸谓是沟为济渠。故班固及阚駰并言济水至武德入河。盖济水枝渎条分,所在布称,亦兼丹水之目矣。
淇水出河内隆虑县西大号山,《山海经》曰:淇水出沮枷山。水出山侧,颓波漰注,冲激横山。山上合下开,可减六七十步,巨石磥砢,交积隍涧,倾澜涝荡,势同雷转,激水散氛,暖若雾合。又东北,沾水注之。水出壶关县东沾台下,石壁崇高,昂藏隐天,泉流发于西北隅,与金谷水合,金谷即沾台之西溪也。东北会沾水,又东流注淇水。淇水又径南罗川,又历三罗城北,东北与女台水合,水发西北三女台下,东北流注于淇。淇水又东北历淇阳川,径石城西北,城在原上,带涧枕淇。淇水又东北,西流水注之,水出东大岭下,西流径石楼南,在北陵石上,练垂禁立,亭亭极峻。其水,西流水也。又东径冯都垒南,世谓之淇阳城,在西北三十里。淇水又东出山,分为二水,水会立石堰,遏水以沃白沟,左为菀水,右则淇水,自元甫城东南径朝歌县北。《竹书纪年》,晋定公十八年,淇绝于旧卫,即此也。淇水又东,右合泉源水,水有二源,一水出朝歌城西北,东南流。老人晨将渡水而沉吟难济,纣问其故,左右曰:老者髓不实,故晨寒也。纣乃于此斫胫而视髓也。其水南流东屈,径朝歌城南。《晋书地道记》曰:本沫邑也。《诗》云:爱采唐矣,沫之乡矣。殷王武丁始迁居之,为殷都也。纣都在《禹贡》冀州大陆之野。即此矣。有糟丘、酒池之事焉,有新声靡乐,号邑朝歌,晋的曰:《史记·乐书》,纣作朝歌之音,朝歌者,歌不时也。故墨子闻之,恶而回车,不径其邑。《论语比考谶》曰:邑名朗歌,颜渊不舍,七十弟子掩目,宰予独顾,由蹙堕车。来均曰:子路患宰予顾视凶地,故以足蹙之使堕车也。今城内有殷鹿台,纣昔自投于火处也。《竹书纪年》曰:武王亲禽帝受辛于南单之台,遂分天之明。南单之台,盖鹿台之异名也。武王以殷之遗民封纣子武庚于兹邑,分其地为三,曰邶、鄘、卫,使管叔、蔡叔、霍叔辅之,为三监。叛,周讨平以封康叔为卫。箕子佯狂自悲,故《琴操》有《箕子操》,径其墟,父母之邦也,不胜悲,作《麦秀歌》。后乃属晋。地居河、淇之间,战国时皆属于赵,男女淫纵,有纣之余风。汉以虞诩为令,朋友以难治致吊,诩曰:不遇盘根错节、何以别利器乎?又东与左水合,谓之马沟水,水出朝歌城北,东流南屈,径其城东。又东流与美沟合,水出朝歌西北大岭下,东流径骆驼谷,于中透迆九十曲,故俗有美沟之目矣。历十二崿崿流相承,泉响不断,返水捍注,卷复深隍,隍间积石千通,水穴万变,观者若思不周,赏情乏图状矣。其水东径朝歌城北,又东南流注马沟水,又东南注淇水,为肥泉也。故《卫诗》曰:我思肥泉,兹之永叹。《毛注》云:同出异归为肥泉。《尔雅》曰;归异出同曰肥。《释名》曰:本同出时,所浸润水少,所归枝散而多,似肥者也。犍为舍人曰:水异出流行,合同曰肥。今是水异出同归矣。《博物志》谓之澳水。《诗》云:瞻彼淇澳, 竹猗猗。毛云: ,王刍也;竹,编竹也。汉武帝塞决河,斩淇园之竹木以为用。寇询为河内,伐竹淇川,治矢百余万,以输军资。今通望淇川,无复此物。惟王刍编草不异毛兴。又言,澳,限也。郑亦不以为津源,而张司空专以为水流入于淇,非所究也。然斯水即《诗》所谓泉源之水也。故《卫诗》云:泉源在左,淇水在右,卫女思归。指以为喻淇水左右,盖举水所人为左右也。淇水又南历枋堰,旧淇水口,东流径黎阳县界,南入柯。《地理志》曰:淇水出共,东至黎阳入河。《沟洫志》曰:遮害亭西十八里至淇水口是也。汉建安九年,魏武王千水口下大坊木以成堰,遏淇水东人白沟以通漕运,故时人号其处为仿头。是以卢湛《征艰赋》曰:后背洪枋巨堰,深渠高堤者也。自后遂废。魏熙平中复通之,故渠历仿城北,东出今读,破故竭。其堰,悉铁柱木石参用,其故渎南径仿城西,又南分为二水,一水南注清水,水流上下更相通注,河清水盛,北入故渠自此始矣。一水东流,径妨城南,东与菀口合。菀水上承淇水于元甫城西北,自石堰东、菀城西,屈径其城南,又东南流历土军东北,得旧石逗。故五水分流,世号五穴口,今惟通并为二水,一水西注淇水,谓之夭井沟,一水径土军东分为蓼沟,东入白祀陂。又南分东入同山陂,溉田七十余顷。二陂所结,即台阴野矣。菀水东南入淇水。淇水右合宿胥故渎,渎受河于顿丘县遮害亭东、黎山西,北会淇水处立石堰,遏水令更东北注。魏武开白沟,因宿胥故渎而加其功也。故苏代曰:决宿胥之口,魏无虚顿丘。即指是渎也。淇水又东北流,谓之白沟,径雍榆城南。《春秋》襄公二十三年,叔孙豹救晋,次于雍榆者也。淇水又北径其城东,东北径同山东,又东北径帝喾冢西。世谓之顿丘台,非也。《皇览》曰:帝喾冢在东郡濮阳顿丘城南,台阴野中者也。又北径白祀山东,历广阳里,径颛顼冢西。俗谓之殷王陵,非也。《帝王世纪》曰:颛顼葬东郡顿丘城南,广阳里大冢者是也。淇水又北屈而西转,径顿丘北,故阚駰云:顿丘在淇水南。《尔雅》曰:山一成谓之顿丘,《释名》谓一顿而成丘,无高下小大之杀也。《诗》所谓送子涉淇,至于顿丘者也。魏徙九原、西河、土军诸胡,置土军于丘侧,故其名亦曰土军也。又屈径顿丘县故城西。《古文尚书》以为观地矣。盖太康弟五君之号曰五观者也。《竹书纪年》,晋走公三十一年城顿丘。《皇览》曰:顿丘者,城门名,顿丘道,世谓之殷。皆非也。盖因丘而为名,故曰顿丘矣。淇水东北径在人山东、牵城西。《春秋左传》定公十四年,公会齐侯、卫侯于牵者也。杜预曰:黎阳东北有牵城。即此城矣。淇水又东北径石柱冈,东北注矣。
东过内黄县南,为白沟。
淇水又东北径并阳城西。世谓之辟阳城,非也,即《郡国志》所谓内黄县有并阳聚者也。白沟又北,左合荡水。又东北流径内黄县故城南,县右对黄泽。《郡国志》曰:县有黄泽者也。《地理风俗记》曰:陈留有外黄,故加内,《史记》曰赵廉颇伐魏取黄,即此县。
屈从县东北,与洹水合。
白沟自县北径戏阳城东,世谓之羛阳聚。《春秋》昭公十年,晋荀盈如齐逆女,还,卒戏阳是也。白沟又北径高城亭东,洹水从西南来注之。又北径问亭东,即魏界也,魏县故城。应劭曰:魏武侯之别都也。城内有武侯台,王莽之魏城亭也。左与新河合,洹水枝流也。白沟又东北径铜马城西,盖光武征铜马所筑也,故城得其名矣。白沟又东北径罗勒城东,又东北,漳水注之,谓之利漕口。自下清漳、白沟、淇河,咸得通称也。
又东北过馆陶县北,又东北过清渊县西,白沟水又东北径赵城西,又北,阿难河出焉。盖魏将阿难所导,以利衡渎,遂有阿难之称矣。白沟又东北径空陵城西,又北径乔亭城西,东去馆陶县故城十五里。县,即《春秋》所谓冠氏也,魏阳平郡治也。其水又屈径其县北,又东北径平恩县故城东。《地理风俗记》曰:县,故馆陶之别乡也。汉宣帝地节三年置,以封后父许伯为侯国。《地理志》,王莽之延平县矣。其水又东过清渊县故城西,又历县之西北为清渊,故县有清渊之名矣。世谓之鱼池城,非也。其水又东北径榆阳城北,汉武帝封太常江德为侯国。文颖曰:邑在魏郡清渊。世谓之清渊城,非也。
又东北过广宗县东,为清河。
清河东北径广宗县故城南。和帝永元五年,封皇太子万年为王国。田融言,赵立建兴郡于城内,置临清县于水东,自赵石始也。猜河之右有李云墓,云字行祖,甘陵人,好学,善阴阳,举孝廉,迁白马令。中常侍单超等,立掖庭民女毫氏为后,后家封者四人,赏赐巨万。云上书移副三府曰:孔子云,帝者,谛也,今尺一拜用,不经御省,是帝欲不谛乎?帝怒,下狱杀之。后冀州刺史贾琼使行部过柯云墓,刻石表之,今石柱尚存,俗犹谓之李氏石柱。清河又东北径界城亭东。水上有大梁,谓之界城桥。《英雄记》曰:公孙瓒击青州黄中贼,大破之,还屯广宗。袁本初自往征瓒,合战于界桥南二十里,绍将麴义破瓒于界城桥,斩瓒。冀州刺史严纲又破瓒殿兵于桥上,即此梁也。世谓之鬲城桥,盖传呼失实矣。清河又东北径信乡西。《地理风俗记》曰:甘陵西北十七里有信乡,故县也。清河又北径信成县故城西。应劭曰:甘陵西北五十里有信成亭,故县也。赵置水东县于此城,故亦曰水东城。清河又东北径清阳县故城西。汉高祖置清河郡,治此。景帝中三年,封皇子乘为王国,王莽之平河也。汉光武建武二年,西河鲜于冀为清河太守,作公廨未就而亡,后守赵高计功用二百万,五官黄秉、功曹刘适言:四百万钱。于是冀乃鬼见白日,道从入府,与高及秉等对,共计校定,为适、秉所割匿,冀乃书表自理其略,言高贵不尚节,亩垄之夫,而箕踞遗类,研密失机,婢妾其性,媚世求显,偷窃很鄙,有辱天官,易讥负乘,诚高之谓,臣不胜鬼言。谨因千里驿闻,付高上之。便西北去三十里,车马皆灭不复见,秉等皆伏地物故。高以状闻,诏下,还冀西河田宅妻子焉。兼为差代,以弭幽中之讼。汉桓帝建和三年,改清河为甘陵王国,以王妖言,徙,其年立甘陵郡,治此焉。
又东北过东武城县西,清河又东北径陵乡西。应劭曰:东武城西南七十里有陵乡,故县也。后汉封太仆梁松为侯国,故世谓之梁侯城,遂立侯城县治也。清河又东北径东武城县故城西。《史记》,赵公子胜,号平原君,以解邯郸之功,受封于此。定襄有武城,故加东矣。清河又东北径复阳县故城西。汉高祖七年,封右司马陈胥为侯国,王莽更名之曰乐岁。《地理风俗记》曰:东武城西北三十里有复阳亭,故县也。世名之曰槛城,非也。清河又东北流,径枣强县故城西。《史记·建元以来王子侯者年表》云:汉武帝元朔二年,封广川惠王子晏为侯国也。应劭《地理风俗记》曰:东武城县西北五十里,有枣强城,故县也。又北过广川县东,清河北径广川县故城南。阚駰曰:县中有长河为流,故曰广川也。水侧有羌垒,姚氏之故居也。今广川县治。清河又东北径历县故城南。《地理志》,信都之属县也,王莽更名曰历宁也。应劭曰:广川县西北三十里有历城亭,故县也。今亭在县东如北。水济尚谓之为历口渡也。
又东过脩县南,又东北过东光县西,清河又东北,左与张甲屯绛故渎合,阻深堤高鄣,无复有水矣。又径脩县故城南,屈径其城东。脩音条,王莽更名之曰脩治,《郡国志》曰:故属信都。清河又东北,左与横漳枝津故渎合,又东北径脩国故城东。汉文帝封周亚夫为侯国,故世谓之北脩城也。清河又东北径邸阁城东,城临侧清河。晋脩县治,城内有县长鲁国孔明碑。清河又东至东光县西,南径胡苏亭。《地理志》,东光有胡苏亭者也。世谓之羌城,非也。又东北,右会大河故渎,又径东光县故城西。后汉封耿纯为侯国。初平二年,黄巾三十万人入渤海,公孙瓒破之于东光界,追奔是水,斩首三万,流血丹水。即是水也。
又东北过南皮县西,清河又东北,无棣沟出焉。东径南皮县故城南,又东径乐亭北。《地理志》之临乐县故城也,王莽更名乐亭。《晋书地道志》、《太康地记》:乐陵国有新乐县。即此城矣。又东径新乡城北。即《地理志》高乐故城也,王莽更之曰为乡矣。无棣沟又东分为二渎,无棣沟又东径乐陵郡北,又东屈而北出,又东转径苑乡县故城南,又东南径高成县故城南,与枝渎合。枝渎上承无棣沟,南径乐陵郡西,又东南径千童县故城东。《史记·建元以来王子侯者年表》曰:故重也,一作千钟。汉武帝元朔四年,封河间献王子刘阴力侯国。应劭曰:汉灵帝改曰饶安也,沧州治。枝渎又南东屈,东北注无棣沟。无棣沟又东北径一故城北,世谓之功城也。又东北径盐山东北入海。《春秋》僖公四年,齐、楚之盟于召陵也,管仲曰:昔召康公赐命先君太公履,北至于无棣,益四履之所也。京相璠曰:旧说无棣在辽西孤竹县。二说参差,未知所定,然管仲以责楚无棣在此,方之为近,既世传已久,且以闻见书之。清河又东北径南皮县故城西。《十三州志》曰:章武有北皮亭,故此曰南皮也,王莽之迎河亭。《史记·惠景侯者年表》云:汉景帝后七年,封孝文后兄子彭祖为侯国,建安中,魏武擒袁谭于此城也。清河又北径北皮城东,左会滹沱别河故渎,谓之合口。城谓之合城也。《地理风俗记》曰:南皮城北五十里有北皮城,即是城矣。
又东北过浮阳县西,清河东北流,浮水故渎出焉。按《史记》,赵之南界有浮水焉,浮水在南,而此有浮阳之称者。盖浮水出入,津流同逆,混并清漳二读,河之旧道,浮水故迹,又自斯别,是县有浮阳之名也。首受清河于县界,东北径高成县之苑乡城北,又东径章武县之故城北。汉景帝后七年,封孝文后弟窦广国为侯国,王莽更名桓章。晋太始中立章武郡,治此。浮水故渎又东径箧山北。《魏土地记》曰:高成东北五十里有筐山,长七里。浮渎又东北径柳县故城南。汉武帝元朔四年,封齐孝王子刘阳为侯国。《地理风俗记》曰:高成县东北五十里有柳亭,故县也。世谓之辟亭,非也。浮渎又东北径汉武帝望海台,又东注于海。应劭曰:浮阳县,浮水所出入海,朝夕往来,日再。今沟无复有水也。清河又北分为二渎,枝分东出,又谓之浮渎。清河又北径浮阳县故城西,王莽之浮城也。建武十五年,更封骁骑将军平乡侯刘歆为侯国,浮阳郡治。又东北,滹沱别渎注焉,谓之合口也。
又东北过邑北,水出焉。
又东北过乡邑南,清河又东,分为二水,枝津右出焉。东径汉武帝故台北。《魏土地记》曰:章武县东百里有武帝台,南北有二台,相去六十里,基高六十丈,俗云,汉武帝东巡海上所筑。又东注于海。清河又东北径纻姑邑南。俗谓之新城,非也。又东北过穷河邑南,清河又东北径穷河邑南。俗谓之三女城,非也。东北至泉州县,北入滹沱。《水经》曰:笥沟东南至泉州县与清河合,自下为派河尾也,又东,泉州渠出焉。
又东北过漂榆邑,入于海。
清河又东径漂榆邑故城南,俗谓之角飞城。《赵记》云:石勒使王述煮盐于角飞,即城异名矣。《魏土地记》曰:高城县东北百里,北尽漂榆,东临巨海,民咸煮海水,藉盐为业。即此城也。清河自是入于海。
荡水出河内荡阴县西山东,荡水出县西石尚山,泉流径其县故城南,县因水以取名也。晋伐成都王颖,败帝于是水之南。卢谌《四王起事》曰:惠帝征成都王颖,战败时,举辇司马八人,辇犹在肩上,军人竟就杀举辇者,乘舆顿地,帝伤三矢,百僚奔散,唯侍中秘绍扶帝,士将兵之,帝曰:吾吏也,勿害之。众曰:受太弟命,惟不犯陛下一人耳。遂斩之,血污帝袂。将洗之,帝曰:嵇侍中血,勿洗也。此则嵇延祖殒命之所。
又东北至内黄县,入于黄泽。
羑水出荡阴西北韩大牛泉。《地理志》曰:县之西山,羑水所出也。羑水又东径韩附壁北,又东流径羑城北,故羑里也。《史记音义》曰:牖里在荡阴县。《广雅》:牖,狱犴也。夏曰夏台,殷曰羑里,周曰囹圄,皆圜土。昔殷纣纳崇侯虎之言,囚西伯于此。散宜生、南宫括见文工,乃演《易》用明否泰始终之义焉。羑城北,水积成渊,方十余步,深一丈余。东至内黄与防水会,水出西山马头涧,东径防城北,卢湛《征艰赋》所谓越防者也。其水东南流注于羑水,又东历黄泽入荡水。《地理志》曰:羑水至内黄入荡者也。荡水又东与长沙沟水合,其水导源黑山北谷,东流径晋鄙故垒北。谓之晋鄙城,名之为魏将城,昔魏公子无忌矫夺晋鄙军于是处。故班叔皮《游居赋》曰:过荡阴而吊晋鄙,责公子之不臣者也。其水又东,谓之宜师沟,又东径荡阴县南,又东径枉人山,东北至内黄县,右入荡水,亦谓之黄雀沟。是水,秋夏则泛,春冬则耗。荡水又径内黄城南。陈留有外黄,故称内也。东注白沟。
洹水出上党泫氏县,水出洹山,山在长子县也。
东过隆虑县北,县北有隆虑山,昔帛仲理之所游神也。县因山以取名。汉高帝六年,封周灶为侯国。应助曰:殇帝曰隆,故改从林也。县有黄华水,出于神园之山黄华谷北崖上。山高十七里,水出木门带,带即山之第三级也,去地七里,悬水东南注壑,直泻岩下,状若鸡翘,故谓之鸡翘洪。盖亦天台,赤城之流也。其水东流至谷口,潜入地下,东北十里复出,名柳渚,渚周四五里,是黄华水重源再发也。东流,苇泉水注之,水出林虑山北泽中,东南流,与双泉合,水出鲁般门,东下流入苇泉水。苇泉水又东南,流注黄华水,谓之陵阳水。又东,入于洹水也。
又东北出山,过邺县南,洹水出山,东径殷墟北。《竹书纪年》曰:盘庚即位,自奄迁于北蒙,曰殷。昔者项羽与章邯盟于此地矣。洹水又东,枝津出焉,东北流径邺城南,谓之新河。又东,分为二水,一水北径东明观下。昔慕容隽梦石虎啮其臂,寤而恶之,购求其尸,而莫之知。后宫劈妾言,虎葬东明观下,于是掘焉,下度三泉,得其棺,剖棺出尸,尸僵不腐,隽骂之曰:死胡,安敢梦生天子也!使御史中尉阳约数其罪而鞭之。此盖虎始葬处也。又北径建春门,石梁不高大,治石工密,旧桥首夹建两石柱,螭矩趺勒甚佳。乘舆南幸,以其作制华妙,致之平城东侧西阙,北对射堂,绿水平潭,碧林侧浦,可游憩矣。其水西径魏武玄武故苑。苑旧有玄武池,以肄舟楫,有鱼梁、钓台、竹木、灌丛,今池林绝灭,略无遗迹矣。其水西流注于漳。南水东北径女亭城北,又东北径高陵城南,东合坰沟,又东径鸬鹚陂,北与台陂水合。陂东西三十里,南北注白沟河,沟上承洹水,北绝新河,北径高陵城东,又北径斥丘县故城西。县南角有斥丘,盖因丘以氏县,故乾侯矣。《春秋经》书,昭公二十八年,公如晋,次于乾侯也。汉高帝六年,封唐厉为侯国,王莽之利丘矣。又屈径其城北,东北流注于白沟。洹水自邺东径安阳县故城北。徐广《晋纪》曰:石遵自李城北入,斩张豺于安阳是也。《魏土地记》曰:邺城南四十里有安阳城,城北有洹水东流者也。洹水又东至长乐县,左则枝沟出焉。洹水又东径长乐县故城南。按《晋书·地理志》曰:魏郡有长乐县也。
又东过内黄县北,东入于白沟。
洹水径内黄县北东流,注于白沟,世谓之洹口也。许慎《说文》、吕忱《字林》,并云洹水出晋、鲁之间。昔声伯梦涉洹水,或与己琼瑰而食之,泣而又为琼瑰,盈其怀矣。从而歌曰:济洹之水,赠我以琼瑰,归乎,归乎,琼瑰盈吾怀乎!后言之,之暮而卒。即是水也。 


卷十 浊漳水、清漳水 
浊漳水出上党长子县西发鸠山,漳水出鹿谷山,与发鸠连麓而在南。《淮南子》谓之发苞山,故异名互见也。左则阳泉水注之,右则伞盖水入焉。三源同出一山,但以南北为别耳。东过其县南,又东,尧水自西山东北流,径尧庙北,又东径长子县故城南。周史辛甲所封邑也。《春秋》襄公十八年,晋人执卫行人石买于长子,即是县也。秦置上党郡,治此。其水东北流入漳水。漳水东会于梁水,梁水出南梁山,北流径长子县故城南。《竹书纪年》曰:梁惠成王十二年,郑取屯留、尚子、涅。尚子,即长子之异名也。梁水又北入漳水。
屈从县东北流,陶水南出陶乡,北流径长子城东,西转径其城北,东注于漳水。
又东过壶关县北,又东北过屯留县南,漳水东径屯留县南,又屈径其城东,东北流,有绛水注之。水西出谷远县东发鸠之谷,谓之为滥水也。东径屯留县故城南。故留吁国也,潞氏之属。《春秋》襄公十八年,晋人执孙蒯于纯留是也。其水东北流入于漳。故桑钦云:绛水出屯留西南,东入漳,漳水又东,冻水注之,水西出发鸠山,东径余吾县故城南。汉光武建武六年,封景丹子尚为侯国。冻水又东径屯留县故城北。《竹书纪年》,梁惠成王元年,韩共侯、赵成侯迁晋桓公于屯留。《史记》,赵肃侯夺晋君端氏而徙居之此矣。其水又东流注于漳。故许慎曰:水出发鸠山入漳,从水,东声也。漳水又东北径壶关县故城西,又屈径其城北。故黎国也,有黎亭,县有壶口关,故曰壶关矣。吕后元年,立孝惠后宫子武为侯国。汉有壶关三老公乘兴上书讼卫太子,即色人也。县在屯留东,不得先壶关而后屯留也。漳水历鹿台山与铜鞮水合,水出铜鞮县西北石瞪山,东流与专池水合,水出八特山,东北流入铜鞮水。铜鞮水又东南合女谏水,水西北出好松山,东南流,北则苇池水与公主水合而右往之;南则榆交水与皇后水合而左入焉。乱流东南,注于铜鞮水。铜鞮水又东径李憙墓,墓前有碑,碑石破碎,故李氏以太和元年立之。其水又东径故城北,城在山阜之上,下临岫壑,东、西、北三面,阻袤二里,世谓之断梁城,即故县之上虒亭也,铜鞮水又东径铜鞮县故城北,城在水南山中,晋大夫羊舌赤铜鞮伯华之邑也。汉高祖破韩王信于此县。铜鞮水又东南流径顷城西,即县之下虒聚也。《地理志》曰:县有土虒亭,下虒聚者也。铜鞮水又南径胡邑西,又东屈径其城南,又东径襄垣县,入于漳。漳水又东北流径襄垣县故城南,王莽之上党亭。潞县北,县,故赤翟潞子国也。其相丰舒有俊才,而不以茂德,晋伯宗数其五罪,使苟林父灭之。阚駰曰:有潞水,为冀州浸,即漳水也。余案《燕书》,王猛与慕容评相遇于潞川也。评障锢山泉,鬻水与军,入绢匹,水二石,无他大川,可以为浸,所有巨浪长湍,惟漳水耳。故世人亦谓浊漳为潞水矣。县北对故台壁,漳水径其南。本潞子所立也,世名之为台壁。慕容垂伐慕容永于长子,军次潞川,永率精兵拒战,阻河自固,垂阵台壁,一战破之,即是处也。漳水于是左合黄须水口,水出台壁西张讳岩下。世传岩赤则土罹兵害,故恶其变化无常,恒以石粉污之令白,是以俗目之为张讳岩。其水南流,径台壁西,又南入于漳。漳水又东北历望夫山,山之南有石人仁于山上,状有怀于云表,因以名焉。有涅水西出覆甑山,而东流与西汤溪水合,水出涅县西山汤谷,五泉俱会。谓之五会之泉,交东南流,谓之西汤水,又东南流注涅水。涅水又东径涅县故城南,县氏涅水也。东与白鸡水合,水出县之西山,东径其县北,东南流入涅水。涅水又东南,武乡水会焉,水源出武山西南,径武乡县故城西,而南得清谷口。水源出东北长山清谷,西南与鞸鞛、白壁二水合,南入武乡水,又南得黄水口,黄水三源,同注一壑,东南流与隐室水合,水源西北出隐室山,东南注黄水。又东入武乡水。武乡水又东南注于涅水。涅水又东南流,注于樟水。漳水又东径磻阳城北,仓谷水入焉。水出林虑县之仓谷溪,东北径鲁班门西。双阙昂藏,石壁霞举,左右结石修防,崇基仍存。北径偏桥东,即林虑之娇岭抱犊固也。石隥西陛,陟踵修上五里余,隥崿路中断四五丈,中以木为偏桥,劣得通行,亦言故有偏桥之名矣。自上犹须攀萝们葛,方乃自津,山顶,即庾衮眩坠处也。仓谷溪水又北合白木溪。溪水出壶关县东白木川,东径百晦城北,盖同仇池百顷之称矣。又东径林虑县之石门谷,又注于仓溪水。仓溪水又北径磻阳城东而北流,注于漳水。漳水又东径葛公亭北而东注矣。
又东过武安县,漳水于县东,清漳水自涉县东南来注之,世谓决入之所为交漳口也。
又东出山,过邺县西,漳水又东径三户峡为三户津。张晏曰:三户,地名也,在梁期西南。孟康曰:津,峡名也,在邺西四十里。又东,汗水注之,水出武安县山,东南流径汗城北。昔项羽与蒲将军英布济自三户,破章邯于是水。汗水东注于漳水。漳水又东径武城南,世谓之梁期城。梁期在邺北,俗亦渭之两期城,皆为非也。司马彪《郡国志》曰:邺县有武城,武城即期城矣。漳水又东北径西门豹祠前。祠东侧有碑,隐起为字,祠堂东头石柱勒铭曰:赵建武中所修也。魏文帝《述征赋》曰:羡西门之嘉迹,忽遥睇其灵字。漳水右与枝水合。其水上承漳水于邯会西,而东别与邯水合,水发源邯山东北,径邯会县故城西,北注枝水,故曰邯会也。张晏曰:漳水之别,自城西南与邯山之水会,今城旁犹有沟渠存焉。汉武帝元朔二年,封赵敬肃王子刘仁为侯国。其水又东北入于漳。昔魏文侯以西门豹为邺令也,引漳以溉邺,民赖其用。其后至魏襄王,以史起为邺令,又堰漳水以灌邺田,咸成沃壤,百姓歌之。魏武王又堨漳水,回流东注,号天井堰。二十里中,作十二墱,墱相去三百步,令互相灌注,一源分为十二流,皆悬水门。陆氏《邺中记》云:水所溉之处,名曰堰陵泽。故左思之赋魏都,谓墱流十二,同源异口者也。魂武之攻邺也,引漳水以围之。《献帝春秋》曰:司空邺城围周四十里,初浅而狭,如或可越,审配不出争利,望而笑之,司空一夜增修,广深二丈,引漳水以注之,遂拔邺。本齐桓公所置也,故《管子》曰:筑五鹿、中牟、邺,以卫诸夏也。后属晋,魏文侯七年,始封此地,故曰魏也。汉高帝十二年,置魏郡,治邺县,王莽更名魏城。后分魏郡,置东、西部都尉,故曰三魏。魏武又以郡国之旧,引漳流自城西东入,径铜雀台下,伏流入城东注,谓之长明沟也。渠水又南径止车门下。魏武封于邺为北宫,富有文昌殿。沟水南北夹道,枝流引灌,所在通溉,东出石窦堰下,注之隍水。故魏武《登台赋》曰:引长明,灌街里。谓此渠也。石氏于文昌故殿处,造东、西太武二殿于济北谷城之山,采文石为基,一基下五百武直宿卫。屈柱跌瓦,悉铸铜为之,金漆图饰焉。又徙长安、洛阳铜人,置诸宫前,以华国也。城之西北有三台,皆因城为之基,巍然崇举,其高若山,建安十五年魏武所起,平坦略尽。《春秋古地》云:葵丘,地名,今邺西三台是也。谓台已平,或更有见,意所未详、中曰铜雀台,高十丈,有屋百一间,台成,命诸子登之,并使为赋。陈思王下笔成章,美捷当时。亦魏武望奉常王叔治之处也,昔严才与其属攻掖门,修闻变,车马未至,便将宫属步至宫门,太祖在铜雀台望见之曰:彼来者必王叔治也。相国钟繇曰:旧京城有变,九卿各居其府,卿何来也?修曰:食其禄,焉避其难,居府虽旧,非赴难之义。时人以为美谈矣。石虎更增二丈,立一屋,连栋接榱,弥覆其上,盘回隔之,名曰命子窟。又于屋上起五层搂,高十五丈,去地二十七丈,又作铜雀于楼巅,舒翼若飞。南则金虎台,高八丈,有屋百九间。北曰冰井台,亦高八丈,有屋百四十五间,上有冰室,室有数井,井深十五丈,藏冰及石墨焉。石墨可书,又然之难尽,亦谓之石炭。又有粟窖及盐窖,以备不虞。今窖上犹有石铭存焉。左思《魏都赋》曰三台列峙而峥嵘者也。城有七门,南曰凤阳门,中曰中阳门,次曰广阳门,东曰建春门,北曰广德门,次曰厩门,西曰金明门,一曰白门。凤阳门三台洞开,高三十五丈,石氏作层观架其上,置铜凤,头高一丈六尺。东城上,石氏立东明观,观上加金博山,谓之骼天。北城上有齐斗楼,超出群榭,孤高特立。其城东西七里,南北五里,饰表以砖,百步一楼,凡诸宫殿,门台、隅雉,皆加观榭。层甍反宇,飞檐拂云,图以丹青,色以轻素。当其全盛之时,去邺六七十里,远望苕亭,巍若仙居。魏因汉祚,复都洛阳,以谯为先人本国,许昌为汉之所居,长安为西京之遗迹,邺为王业之本基,故号五都也。今相州刺史及魏郡治。漳水自西门豹祠北径赵阅马台西,基高五丈,列观其上,石虎每讲武于其下,升观以望之,虎自台上放鸣镝之矢,以为车骑出入之节矣。漳水又北径祭陌西。战国之世,俗巫为河伯取妇,祭于此陌。魏文侯时,西门豹为邺令,约诸三老曰:为河伯娶妇,幸来告知,吾欲送女。皆曰:诺。至时,三老、廷掾,赋敛百姓,取钱百万,巫觋行里中,有好女者,祝当为河伯妇,以钱三万聘女,沐浴脂粉如嫁状。豹往会之,三老、巫、掾与民咸集赴观。巫妪年七十,从十女弟子。豹呼妇视之,以为非妙,令巫妪入报河伯,投巫于河中。有顷曰:何久也,又令三弟子及三老入白,并投于河。豹磬折曰:三老不来,奈何?复欲使廷掾、豪长趣之,皆叩头流血,乞不为河伯取妇。淫祀虽断,地留祭陌之称焉。又慕容俊投石虎尸处也。田融以为紫陌也。赵建武十一年,造紫陌浮桥于水上,为佛图澄先造生墓于紫陌,建武十五年卒,十二月葬焉,即此处也。漳水又对赵氏临漳宫。宫在桑梓苑,多桑木,故苑有其名。三月三日及始蚕之月,虎帅皇后及夫人采桑于此。今地有遗桑,塘无尺雉矣。漳水又北,滏水入焉。漳水又东径梁期城南。《地理风俗记》曰:邺北五十里有梁期城,故县也。汉武帝元鼎五年,封任破胡为侯国。晋惠帝永兴元年,骠骑王浚遣乌丸渴末径至梁期,候骑到邺,成都王颖遣将军石超讨末,为末所败于此也。又径平阳城北。《竹书纪年》曰:梁惠成王元年,邺师败邯郸师于平阳者也。司马彪《郡国志》曰:邺有平阳城。即此地也。
又东过列入县南,漳水又东,右径斥丘县北,即裴县故城南。王莽更名之曰即是也。《地理风俗记》曰:列入县西南六十里有即裴城,故县也。漳水又东北径列入县故城南,王莽更名之为列治也。《竹书纪年》曰;梁惠成王八年,惠成王伐邯郸取列入者也。于县右合白渠故渎,白渠水出魏郡武安县钦口山,东南流径邯郸县南,又东与拘涧水合。水导源武始东山白渠,北俗犹谓是水为拘河也。拘涧水又东,又有牛首水入焉,水出邯郸县西堵山,东流分为二水,洪湍双逝,澄映两川。汉景帝时,七国悖逆,命曲周侯郦寄攻赵,围邯郸,相捍七月,引牛首拘水灌城,城坏,王自杀,其水东入邯郸城,径温明殿南。汉世祖擒王郎、幸邯郸昼卧处也。其水又东径丛台南。六国时,赵王之台也。《郡国志》曰:邯郸有丛台。故刘劭《赵都赋》曰:结云阁于南宇,立丛台于少阳者也。今遗基旧墉尚在。其水又东历邯郸阜,张晏所谓邯山在东城下者也。曰单,尽也,城郭从邑,故加邑,邯郸之名,盖指此以立称矣。故赵郡治也,《长沙耆旧传》称,桓楷为赵郡太守,尝有遗囊粟于路者,行人挂囊粟于树,莫敢取之,即于是处也。其水又东流出城,又合成一川也。又东,澄而为渚,渚水东南流,注拘涧水,又东入白渠,又东,故渎出焉。一水东为泽渚,曲梁县之鸡泽也。《国语》所谓鸡丘矣。东北通澄湖,白渠故渎南出所在,枝分右出,即邯沟也。历邯沟县故城东,盖因沟以氏县也。《地理风俗记》曰:即裴城,西北二十里有邯沟城,故县也。又东径肥乡县故城北。《竹书纪年》曰:梁惠成王八年,伐邯郸取肥者也。《晋书地道记》曰:太康中,立以隶广平也。渠道交径,互相缠魔,与白渠同归,径列人右会漳津,今无水。《地理志》曰:白渠东至列人入漳是也。
又东北过斥漳县南,应劭曰:其国斥卤,故曰斥漳。汉献帝建安十八年,魏太祖凿渠,引漳水东入清洹以通河漕,名曰利漕渠。漳津故渎水断,旧溪东北出,涓流濗注而已。《尚书》所谓覃怀底绩,至于衡漳者也。孔安国曰:衡,横也,言漳水横流也。又东北径平恩县故城西。应劭曰:县,故馆陶之别乡,汉宣帝地节三年置,以封后父许伯为侯国,王莽更曰延平也。
又东北过曲周县东,又东北过巨鹿县东,衡漳故渎东北径南曲县故城西。《地理志》,广平有南曲县。应劭曰:平恩县北四十里有南曲亭,故县也。又径曲周县故城东。《地理志》曰:汉武帝建元四年置,王莽更名直周。余按《史记》,大将军郦商以高祖六年封曲周县为侯国,又考《汉书》同。是知曲周旧县,非始孝武。啸父冀州人,在县市补履数十年,人奇其不老,求其术而不能得也。衡漳又北径巨桥邸阁西,旧有大梁横水,故有巨桥之称。昔武王伐纣,发巨桥之粟,以赈殷之饥民。服虔曰:巨桥,仓名。许慎曰:巨鹿水之大桥也。今临侧水湄,左右方一二里中,状若丘墟,盖遗囤故窖处也。衡水又北径巨鹿县故城东。应劭曰:鹿者,林之大者也。《尚书》曰:尧将禅舜,纳之大麓之野,烈风雷雨不迷,致之以昭华之玉。而县取目焉,路温舒,县之东里人,父为里监门,使温舒牧羊泽中,取蒲牒用写书,即此泽也。巨鹿郡治。秦始皇二十五年灭赵以为巨鹿郡。汉景帝中元年,为广平郡,武帝征和二年,以封赵敬肃王子为平于国。世祖中兴,更为巨鹿也。郑玄注《尚书》引《地说》云:大河东北流,过绛水千里,至大陆为地腹,如《志》之言大陆在巨鹿。《地理志》曰:水在安平信都巨鹿。与信都相去不容此数也。水土之名变易,世失其处,见降水则以为绛水,故依而废渎,或作绛字,非也。今河内共北山,淇水出焉,东至魏郡黎阳入河,近所谓降水也。降读当如郕,降于齐师之降,盖周时国于此地者,恶言降,故改云共耳。又今河所从去大陆远矣,馆陶北屯氏河,其故道与?余按郑玄据《尚书》,有东过洛油,至于大伾,北过降水至于大陆。推次言之,故以淇水为降水,共城为降城,所未详也。稽之群书,共县本共和之故国,是有共名,不因恶降而更称。禹著《山经》,淇出沮洳。淇澳《卫诗》,列目又远,当非改绛,革为今号。但是水导源共北山,玄欲成降义,故以淇水为降水耳。即如玄引《地说》,黎阳巨鹿,非千里之径,直信都于大陆者也。惟屯氏北出馆陶,事近之矣。按《地理志》云:绛水发源屯留,下乱漳津。是乃与漳俱得通称,故水流间关,所在著目,信都复见绛名,而东入于海。寻其川脉,无他殊渎,而衡漳旧道,与屯氏相乱,乃《书》有过降之文,与《地说》千里之志,即之途致与《书》相邻,河之过降,当应此矣,下至大陆,不异《经》说,自宁迄于巨鹿,出于东北,皆为大陆。语之缠络,厥势眇矣。九河既播,八枝代绝。遗迹故称,往往时存,故鬲、般列于东北,徒骇渎联漳、绛,同逆之状粗分,陂障之会犹在。按《经》考渎,自安故目矣。漳水又历经县故城西,水有故津,谓之薄落津。昔袁本初还自易京,上已届此,率其宾从,禊饮于斯津矣。衡漳又径沙丘台东。纣所成也,在巨鹿故城东北七十里,赵武灵王与秦始皇并死于此矣。又径铜马祠东,汉光武庙也。更始三年秋,光武追铜马于馆陶,大破之,遂降之。贼不自安,世祖令其归营,乃轻骑行其垒,贼乃相谓曰:萧王推赤心置人腹中,安得不投死乎?遂将降人分配诸将,众数十万人,故关西号世祖曰铜马帝也,祠取名焉。庙侧有碑,述河内修武县张导,字景明,以建和三年为巨鹿太守,漳津泛滥,土不稼穑,导披按地图,与丞彭参、掾马道嵩等,原其逆顺,揆其表里,修防排通,以正水路,功绩有成,民用嘉赖。题云:《漳河神坛碑》。而俗老耆儒,犹揭斯庙为铜马刘神寺,是碑顷因震裂,余半不可复识矣。又径南宫县故城西。汉惠帝元年,以封张越人子买为侯国,王莽之序中也。其水与隅醴通为衡津。又有长芦淫水之名,绛水之称矣。今漳水既断,绛水非复缠络矣。又北,绛渎出焉,今无水,故渎东南径九门城南,又东南径南宫城北,又东南径缭城县故城北。《十三州志》召:经县东五十里有缭城,故县也。左径安城南,故信都之安城乡也。更始二年,和戎卒正邳彤与上会信都南安城乡,上大悦,即此处也。故渎又东北径辟阳亭。汉高帝六年,封审食其为侯国,王莽之乐信也。《地理风俗记》曰:广川西南六十里有辟阳亭,故县也。绛渎又北径信都城东,散入泽渚,西至于信都城,东连于广川县之张甲故渎,同归于海。故《地理志》曰:《禹贡》,绛水在信都东入于海也。又北过堂阳县西,衡水自县,分为二水,其一水北出,径县故城西。世祖自信都以四千人先攻堂阳降水者也。水上有梁,谓之旅津渡,商旅所济故也。其右水东北注,出石门。门石崩褫,余基殆在。谓之长芦水,盖变引葭之名也。长芦水东径堂阳县故城南。应劭曰:县在堂水之阳。《谷梁传》曰:水北为阳也。今于县故城南,更无别水,惟是水东出,可以当之,斯水盖包堂水之兼称矣。长芦水又东径九门城北,故县也。又东径抉柳县故城南。世祖建武三十年,封寇恂子损为侯国。又东屈北径信都县故城西,信都郡治也,汉高帝六年置。景帝中二年,为广川惠王越国,王莽更为新博,县曰新博亭,光武自蓟至信都是也,明帝水平十五年,更名乐成,安帝延光中,改曰安平。城内有汉冀州从事安平赵征碑,又有魏冀州刺史陈留丁绍碑,青龙三年立。城南有献文帝《南巡碑》。其水侧城北注,又北径安阳城东,又北径武阳城东。《十三州志》曰:抉柳县东北有武阳城,故县也。又北为博广池,池多名蟹佳虾,岁贡王朝,以充膳府。又北径下博县故城东,而北流注于衡水也。又东北过扶柳县北,又东北过信都县西。
扶柳县故城在信都城西,衡水径其西。县有扶泽,泽中多柳,故曰扶柳也。衡水又北径昌城县故城西。《地理志》,信都有昌城县。汉武帝以封城阳顷王子刘差为侯国。阚駰曰:昌城本名阜城矣。应劭曰:堂阳县北三十里有昌城,故县也。世祖之下堂阳,昌城人刘植率宗亲子弟据邑以奉世祖是也。又径西粱县故城东。《地理风俗记》曰:扶柳县西北五十里有西梁城,故县也。世以为五梁城,盖字状致谬耳。衡漳又东北径桃县故城北。汉高祖十二年,封刘襄为侯国,王莽改之曰桓分也。合斯洨故渎,斯洨水首受大白渠,大白渠首受绵蔓水,绵蔓水上承桃水,水出乐平郡之上艾县,东流,世谓之曰桃水,东径靖阳亭南,故关城也。又北流,径井陉关下,注泽发水,乱流东北径常山蒲吾县西,而桃水出焉,南径蒲吾县故城西,又东南流径桑中县故城北。世谓之石勒城,盖赵氏增城之,故擅其目,俗又谓之高功城。《地理志》曰:侯国也。桃水又东南流,径绵蔓县故城北,王莽之绵延也。世祖建武二年,封郭况为侯国。自下通谓之绵蔓水。绵蔓水又东流,径乐阳县故城西,右合并陉山水,水出井陉山,世谓之鹿泉水,东北流,屈径陈余垒西,俗谓之故壁城。昔在楚、汉,韩信东入,余拒之于此,不纳左车之计,悉众西战,信遣奇兵自间道出,立帜于其垒,师奔失据,遂死汦上。其水又屈径其垒南,又南径城西,东注绵蔓水。绵蔓水又屈从城南,俗名曰临清城,非也。《地理志》曰:侯国矣。王莽更之曰畅苗者也。《东观汉记》曰:光武使邓禹发房子兵二千人,以铫期为偏将军,别攻真定宋子,余贼拔乐阳禀肥垒者也。绵蔓水又东径乌子堰,枝津出焉。又东,谓之大白渠,《地理志》所谓首受绵蔓水者也。白渠水又东南径关县故城北。《地理志》,常山之属县也。又东为成郎河,水上有大梁,谓之成郎桥。又东径耿乡南,世祖封前将军耿纯为侯国,世谓之宜安城。又东径宋子县故城北,又谓之宋子河。汉高帝八年,封许瘛为侯国,王莽更名宜子。昔高渐离击筑佣工,自此入秦。又东径敬武县故城北。按《地理志》,巨鹿之属县也。汉元帝封女敬武公主为汤沐邑。阚駰《十三州记》曰:杨氏县北四十里有敬武亭,故县也。今其城实中,小邑耳,故俗名之曰敬武垒,即古邑也。白渠水又东,谓之斯洨水。《地理志》曰:大白渠东南至下曲阳入斯洨者也。东分为二水,枝津右出焉,东南流,谓之百尺沟,又东南径和城北。世谓之初丘城,非也。汉高帝十一年,封郎中公孙昔为侯国。又东南径贳城西,汉高帝六年,封吕博为侯国。百尺沟东南散流,径历乡东而南入泜湖,东注衡水也。斯洨水自枝津东径贳城北,又东积而为陂,谓之阳縻渊。渊水左纳白渠枝水,俗谓之洨水,水承白渠于藁城县之乌子堰。又东径肥累县之故城南,又东径陈台南,台甚宽广,今上阳台屯居之。又东径新丰城北。按《地理志》云:巨鹿有新市县,侯国也。王莽更之曰乐市,而无新丰之目,所未详矣。其水又东径昔阳城南。世谓之曰直阳城,非也,本鼓聚矣。《春秋左传》昭公十五年,晋荀吴帅师伐鲜虞,围鼓三月,鼓人请降,穆子曰:犹有食色,不许。军吏曰:获城而弗取,勤民而顿兵,何以事君?穆子曰:获一邑而教民怠,将焉用邑也。贾怠无卒,弃旧不祥,鼓人能事其君,我亦能事吾君,率义不爽,好恶不愆,城可获也。有死义而无二心,不亦可乎?鼓人告食竭力尽,而后取之。克鼓而返,不戮一人,以鼓子鸢鞮归,既献而返之。鼓子又叛,荀吴略东阳,使师伪,负甲息于门外,袭而灭之。以鼓子鸢鞮归,使涉佗守之者也。《十三州志》曰:今其城,昔阳亭是矣。京相璠曰:白狄之别也。下曲阳有鼓聚,故鼓子国也。白渠枝水又东径下曲阳城北,又径安乡县故城南。《地理志》曰:侯国也。又东径贳县,入斯洨水。斯洨水又东径西梁城南,又东北径乐信县故城南。《地理志》,巨鹿属县,侯国也。又东入衡水。衡水又北为袁谭渡,盖谭自邺往还所由,故济得厥名。
又东北过下博县之西,衡水又北径邬县故城东。《竹书纪年》,梁惠成王三十年,秦封卫鞅于邬,改名曰商。即此是也。故王莽改曰秦聚也。《地理风俗记》曰:县北有邬阜,盖县氏之。又右径下博县故城西,王莽改曰闰博。应劭曰:太山有博,故此加下。汉光武自滹沱南出,至此失道,不知所以。遇白衣老父曰:信都为长安守,去此八十里。世祖赴之,任光开门纳焉,汉氏中兴始基之矣。寻求老父不得,议者以为神。衡漳又东北历下博城西,逶迆东北注,谓之九。西径乐乡县故城南,王莽更之曰乐丘也。又东,引葭水注之。
又东北过阜城县北,又东北至昌亭,与滹沱河会。《经》叙阜城于下博之下,昌亭之上。考地非比,于事为同。勃海阜城又在东昌之东,故知非也。漳水又东北径武邑郡南,魏所置也。又东径武强县北,又东北径武隧县故城南。按《史记》秦破赵将扈辄于武隧,斩首十万,即于此处也。王莽更名桓隧矣。白马河注之,水上承滹沱,东径乐乡县北,饶阳县南,又东南径武邑郡北,而东入衡水,谓之交津口。衡漳又东径武邑县故城北,王莽之顺桓也。晋武帝封子于县以为王国,后分武邑、武隧、观津为武邑郡,治此。衡漳又东北,右合张平口,故沟上承武强渊,渊之西南,侧永有武强县故治,故渊得其名焉。《东观汉记》曰:光武拜王梁为大司空,以为侯国。耆宿云:邑人有行于途者,见一小蛇,疑其有灵,持而养之,名曰担生。长而吞噬人,里中患之,遂捕系狱,担生负而奔,邑沦为湖,县长及吏咸为鱼矣。今县治东北半里许落水。渊水又东南结而为湖,又谓之郎君渊。耆宿又言:县沦之日,其子东奔,又陷于此,故渊得郎君之目矣。渊水北通,谓之石虎口,又东北为张平泽。泽水所泛,北决堤口,谓之张刀沟,北注衡漳,谓之张平口,亦曰张平沟。水溢则南注,水耗则辍流。衡漳又径东昌县故城北,《经》所谓昌亭也。王莽之田昌也,俗名之曰东相,盖相、昌声韵合,故致兹误矣。西有昌城,故目是城为东昌矣。衡漳又东北,左会滹沱故渎,谓之合口。衡漳又东北,分为二川,当其水泆处,名之曰李聪涣。
又东北至乐成陵县北别出,衡漳于县无别出之渎,出县北者,乃滹沱别水,分滹沱故渎之所缠络也。衡漳又东,分为二水,左出为向氏口,渎水自此决入也。衡漳又东,径弓高县故城北。汉文帝封韩王信之子韩隤当为侯国,王莽之乐成亭也。衡漳又东北,右合柏梁溠,水上承李聪涣,东北为柏梁溠,东径蒲领县故城南。汉武帝元朔三年,封广川惠王子刘嘉为侯国。《地理风俗记》云:修县西北八十里有蒲领乡,故县也。又东北会桑社枝津,又东北径弓高城北,又东注衡漳,谓之柏梁口。衡漳又东北,右会桑社沟,沟上承从陂,世称卢达从薄,亦谓之摩诃河。东南通清河,西北达衡水。春秋雨泛,观津城北方二十里,尽为泽薮,盖水所钟也。其渎径观津县故城北,乐毅自燕降赵,封之于此,邑号望诸君,王莽之朔定亭也。又南屈东径窦氏青山南,侧堤东出。青山,即汉文帝窦后父少翁冢也,少翁是县人,遭秦之乱,渔钓隐身,坠渊而死。景帝立,后遣使者填以葬父,起大坟于观津城东南,故民号曰青山也。又东径董仲舒庙南。仲舒,广川人也,世犹谓之董府君祠,春秋祷祭不辍。旧沟又东径修市县故城北。汉宣帝本始四年,封清河纲王子刘寅为侯国,王莽更之曰居宁也。俗谓之温城,非也。《地理风俗记》曰:修县西北二十里有修市城,故县也。又东会从陂,陂水南北十里,东西六十步,子午潭涨,渊而不流,亦谓之桑社渊。从陂南出,夹堤东派,径修县故城北,东合清漳。漳泛则北注,泽盛则南播,津流上下,互相径通。从陂北出,东北分为二川,一川北径弓高城西面北注柏梁溠,一川东径弓高城南。又东北,杨津沟水出焉。衡水东径阜城县故城北、乐成县故城南,河间郡治。《地理志》曰:故赵也。汉文帝二年,别为国。应劭曰:在两河之间也。景帝九年,封子德为河间王,是为献王。王莽更名,郡曰朔定,县曰陆信。褚先生曰:汉宣帝地节三年,封大将军霍光兄子山为侯国也。封子开于此,桓帝追尊祖父孝王开为孝穆王,以其邑奉山陵,故加陵曰乐成陵也。今城中有故池,方八十步,旧引衡水北入城注池。池北对层台,基隍荒芜,示存古意也。
又东北过成平县南。
衡漳又东径建成县故城南。按《地理志》,故属勃海郡。褚先生曰,汉昭帝元凤三年,封丞相黄霸为侯国也。成平县故城在北,汉武帝元朔三年,封河间献王子刘礼为侯国,王莽之泽亭也。城南北相直。衡漳又东右会杨津沟水,水自陂东径阜城南。《地理志》,勃海有阜城县,王莽更名吾城者,非《经》所谓阜城也。建武十五年,世祖更封大司马王梁为侯国。杨津沟水又东北径建成县,左入衡水,谓之杨津口。衡漳又东,左会滹沱别河故渎,又东北入清河,谓之合口,又径南皮县之北皮亭,而东北径浮阳县西,东北注也。
又东北过章武县西,又东北过平舒县南,东入海。清漳径章武县故城西,故邑也。枝渎出焉,谓之水。东北径参户亭,分为二渎。应劭曰:平舒县西南五十里有参户亭,故县也。世谓之平虏城。枝水又东注,谓之蔡伏沟。又东积而为淀。一水径亭北,又径东平舒县故城南。代郡有平舒城,故加东。《地理志》,勃海之属县也。《魏土地记》曰:章武郡治,故世以为章武故城,非也。又东北分为二水,一右出为淀,一水北注滹沱,谓之口。清漳乱流而东注于海。
清漳水出上党沾县西北少山大要谷,南过县西,又从县南屈,《淮南子》曰:清漳出谒戾山。高诱云:山在沾县。今清漳出沾县故城东北,俗谓之沾山。后汉分沾县为乐平郡,治沾县。水出乐平郡沾县界。故《晋太康地记》曰:乐平县旧名沾县。汉之故县矣。其山亦曰鹿谷山,水出大要谷,南流径沾县故城东,不历其西也。又南径昔阳城。《左传》昭公十二年,晋荀吴伪会齐师者,假道于鲜虞,遂入昔阳。杜预曰:乐平沾县东有昔阳城者是也。其水又南得梁榆水口,水出梁榆城西大嵰山,水有二源,北水东南流,径其城东南,注于南水。南水亦出西山,东径文当城北,又东北径梁榆城南,即阏与故城也。秦伐赵阏与,惠文王使赵奢救之,奢纳许历之说,破秦于阏与,谓此也。司马彪、袁山松《郡国志》并言涅县有阏与聚。卢谌《征艰赋》曰:访梁榆之墟郭,吊阏与之旧都。阚駰亦云:阏与,今梁榆城是也。汉高帝八年,封冯解散为侯国。其水左台北水,北水又东南入于清漳。清漳又东南与轑水相得。轑水出轑阳县西北轑山,南流径轑阳县故城西南,东流至粟城,注于清漳也。
东过涉县西,屈从县南,按《地理志》,魏郡之属县也。漳水于此有涉河之称,盖名因地变也。东至武安县南黍窖邑,入于浊漳。


卷十一  易水、滱水
易水出涿郡故安县阎乡西山,易水出西山宽中谷,东径五大夫城南。昔北平侯王谭,不从王莽之政,子兴生五子。并避时乱,隐居此山,故其旧居,世以为五大夫城,即此。《岳赞》云:五王在中,庞葛连续者也。易水又东,左与子庄溪水合,水北出子庄关,南流径五公城西,屈径其城南。五公,即王兴之五子也。光武即帝位,封为五侯:元才北平侯,益才安憙侯,显才蒲阴侯,仲才新市侯,秀才为唐侯。所谓中山五王也,俗又以五公名居矣。二城并广一里许,俱在冈阜之上,上斜而下方。其水东南入于易水。易水又东,右会女思谷水,水出西南女思涧,东北流注于易,谓之三会口。易水又东届关门城西南,即燕之长城门也。与樊石山水合,水源西出广昌县之樊石山,东流径覆釜山下,东流注于易水。易水又东历燕之长城,又东径渐离城南,盖太子丹馆高渐离处也。易水又东径武阳城南,盖易自宽中历武夫关东出,是兼武水之称。故燕之下都,擅武阳之名。左得濡水枝津故渎。武阳大城东南小城,即故安县之故城也,汉文帝封丞相申屠嘉为侯国。城东西二里,南北一里半。高诱云:易水径故安城南城外东流。即斯水也。诱是涿人,事经明证。今水被城东南隅,世又谓易水为故安河。武阳,盖燕昭王之所城也,东西二十里,南北十七里。故傅逮《述游赋》曰:出北蓟,历良乡,登金台,观武阳,两城辽廓,旧迹冥芒。盖谓是处也。易水东流而出于范阳。
东过范阳县南,又东过容城县南,易水径范阳县故城南。秦末,张耳、陈余为陈胜略地,燕、赵命蒯通说之,范阳先下是也。汉景帝中二年,封匈奴降王代为侯国,王莽之顺阴也。昔慕容垂之为范阳也,戍之即斯。意欲图还上京,阻于行旅,造次不获,遂中。易水又东与濡水合,水出故安县西北穷独山南谷,东流与源泉水合,水发北溪,东南流注濡水。濡水又东南径樊于期馆西,是其授首于荆轲处也。濡水又东南流径荆轲馆北,昔燕丹纳田生之言,尊轲上卿,馆之于此。二馆之城,涧曲泉清,山高林茂,风烟披薄,触可栖情,方外之士,尚凭依旧居,取畅林木。濡水又东径武阳城西北,旧堨濡水,枝流南入城径柏冢西,冢垣城侧,即水塘也。四周茔域深广,有若城焉。其水侧有数陵,坟高壮,望若青丘,询之古老,访之史籍,并无文证,以私情求之,当是燕都之前故坟也,或言燕之坟茔,斯不然矣。其水之故渎南出,屈而东转,又分为二渎。一水径故安城西,侧城南注易水,夹塘崇峻,邃岸高深。左右百步,有二钓台,参差交峙,迢递相望,更为佳观矣。其一水东出注金台陂,陂东西六七里,南北五里。侧陂西北有钓台高丈余,方可四十步,陂北十余步有金台,台上东西八十许步,南北如减。北有小金台,台北有兰马台,并悉高数丈,秀峙相对,翼台左右,水流径通,长庑广字,周旋被浦,栋堵咸沦,柱础尚存,是其基构可得而寻访。诸耆旧咸言,昭王札宾,广延方士,至于郭隗、乐毅之徒,邹衍、剧辛之俦,宦游历说之民,自远而届者多矣。不欲令诸侯之客,伺隙燕邦,故修连下都馆之南垂,言燕昭创之于前,子丹踵之于后,故雕墙败馆,尚传镌刻之石,虽无经记可凭,察其古迹,似符宿传矣。濡水自堰又东径紫池堡西,屈而北流,又有浑塘沟水注之,水出遒县西白马山南溪中,东南流入濡水。濡水又东至塞口,古累石堰水处也。濡水旧枝分南入城东大陂,陂方四里,今无水。陂内有泉,渊而不流,际池北侧,俗谓圣女泉。濡水又东得白杨水口,水出道县西山白杨岭下,东南流入濡水,时人谓之虎眼泉也。濡水东合檀水,水出遒县西北檀山西南,南流与石泉水会,水出石泉固东南隅,水广二十许步,深三丈。固在众山之内,平川之中,四周绝涧阻水,八丈有余。石高五丈,石上赤土,又高一匹,壁立直上,广四十五步,水之不周者,路不容轨,仅通人马,谓之石泉固。固上宿有白杨寺,是白杨山神也。寺侧林木交荫,丛柯隐景,沙门释法澄建刹于其上,更为思玄之胜处也。其水南流注于檀水,故俗有并沟之称焉。其水又东南流,历故安县北而南注濡水。濡水又东南流,于容城县西北大利亭东南合易水而注巨马水也。故《地理志》曰:故安县阎乡,易水所出,至范阳入濡水。阚駰亦言是矣,又曰濡水合渠。许慎曰:濡水入涞。涞、渠二号,即巨马之异名。然二易俱出一乡,同入濡水,南濡。北易至涿郡范阳县会北濡,又并乱流入涞,是则易水与诸水互摄通称,东径容城县故城北,浑涛东注,至勃海平舒县与易水合。阚駰曰:涿郡西界代之易水。而是水出代郡广昌县东南郎山东北燕王仙台东。台有三峰,甚为崇峻,腾云冠峰,高霞翼岭,岫壑冲深,含烟罩雾,耆旧言:燕昭王求仙处。其东谓之石虎冈,范晔《汉书》云:中山简王焉之窆也。厚其葬,采涿郡山石,以树坟茔,陵隧碑兽,并出此山,有所遗二石虎,后人因以名冈。山之东麓,即泉源所导也,《经》所谓阎乡西山。其水东流,有毖水南会,浑波同注,俗谓之为雹河。司马彪《郡国志》曰,雹水出故安县。世祖令耿况击故安西山贼吴耐蠡符雹上十余营,皆破之。即是水者也。易水又东径孔山北,山下有钟乳穴,穴出佳乳,采者篝火寻沙,入穴里许,渡一水,潜流通注,其深可涉,于中众穴奇分,令出入者疑迷不知所趣,每于疑路,必有历记,返者乃寻孔以自达矣。上又有大孔,豁达洞开,故以孔山为名也。其水又东径西故安城南,即阎乡城也。历送荆陉北,耆旧云:燕丹饯荆轲于此,因而名焉,世代已远,非所详也。遗名旧传,不容不诠,庶广后人传闻之听。易水又东流屈径长城西,又东流南径武隧县南,新城县北。《史记》曰:赵将李牧伐燕,取武隧方城是也。俗又谓是水为武隧津,津北对长城门,谓之汾门。《史记·赵世家》云:孝成王十九年,赵与燕易土,以龙兑,汾门与燕,燕以葛城,武阳与赵。即此也。亦曰汾水门,又谓之梁门矣。易水东分为梁门陂,易水又东,梁门陂水注之,水上承易水于梁门,东入长城,东北入陂。陂水北接范阳陂,陂在范阳城西十里,方十五里,俗亦谓之为盐台陂。陂水南通梁门淀,方三里,淀水东南流,出长城注易,谓之范水,易水自下,有范水通目。又东径范阳县故城南,即应劭所谓范水之阳也。易水又东径樊舆县故城北。汉武帝元朔五年,封中山靖王子刘条为侯国,王莽更名握符矣。《地理风俗记》曰:北新城县东二十里有樊舆亭,故县也。易水又东径容城县故城南,汉高帝六年,封赵将夜于深泽,景帝中三年,以封匈奴降王唯徐卢于容城,皆为侯国,王莽更名深泽也。易水又东,埿水注之,水上承二陂于容城县东南,谓之大埿淀、小埿淀。其水南流注易水,谓之埿洞口,水侧有浑埿城,易水径其南,东合滱水。故桑钦曰:易水出北新城西北,东入滱,自下滱,易互受通称矣。易水又东径易京南。汉末,公孙瓒害刘虞于蓟下,时童谣云:燕南垂,赵北际,惟有此中可避世。瓒以易地当之,故自蓟徙临易水,谓之易京城,在易城西四五里。赵建武四年,石虎自辽西南达易京,以京障至固,令二万人废坏之。今者,城壁夷平,其楼基尚存,犹高一匹,余基上有井,世名易京楼,即瓒所保也。故瓒《与子书》云:袁氏之攻,状若鬼神,冲梯舞于楼上,鼓角鸣于地中。即此楼也。易水又东径易县故城南,昔燕文公徙易,即此城也。阚駰称太子丹遣荆轲刺秦王,与宾客知谋者,祖道于易水上。《燕丹子》称,荆轲入秦,太子与知谋者,皆素衣冠送之于易水之上,荆轲起为寿,歌曰:风萧萧兮易水寒,壮士一去兮不复还。高渐离击筑,宋如意和之,为壮声,士发皆冲冠;为哀声,士皆流涕。疑于此也。余按遗传旧迹,多在武阳,似不饯此也。汉景帝中三年,封匈奴降王仆为侯国也。
又东过安次县南,易水径县南,鄚县故城北,东至文安县与滹沱合。《史记》,苏秦曰:燕长城以北,易水以南。正谓此水也。是以班固、阚駰之徒,咸以斯水谓之南易。
又东过泉州县南,东入于海。
《经》书水之所历,沿次注海也。
滱水出代郡灵丘县高氏山,即沤夷之水也,出县西北高氏山。《山海经》曰:高氏之山,滱水出焉,东流注于河者也。其水东南流,山上有石铭,题言冀州北界,故世谓之石铭陉也。其水又南径候塘,川名也。又东合温泉水,水出西北暄谷,其水温热若汤,能愈百疾,故世谓之温泉焉。东南流径兴豆亭北,亭在南原上,敧倾而不正,故世以敧城目之。水自原东南注于滱。滱水又东,莎泉水注之,水导源莎泉南流,水侧有莎泉亭,东南入于滱水。滱水又东径灵丘县故城南。应劭曰:赵武灵王葬其东南二十里,故县氏之。县,古属代,汉灵帝光和元年,中山相臧昊上请别属也。瓒注《地理志》曰:灵丘之号,在武灵王之前矣。又按司马迁《史记》,赵敬侯九年,败齐于灵丘,则名不因武灵王事,如瓒《注》。流水自县南流入峡,谓之隘门,设隘于峡,以讥禁行旅。历南山,高峰隐天,深溪埒谷,其水沿涧西转,径御射台南,台在北阜上,台南有御射石碑,南则秀嶂分霄,层崖刺天,积石之峻,壁立直上,“车驾沿溯,每出是所游艺焉。滱水西流,又南转东屈径北海王详之石碣南,御射碑石柱北而南流也。
东南过广昌县南,滱水东径嘉牙川,有一水南来注之,水出恒山北麓,稚川三合,径嘉牙亭东而北流,注于滱水。水之北山,行即广昌县界。滱水又东径倒马关,关山险隘,最为深峭,势均诗人高冈之病良马,傅险之困行轩,故关受其名焉。关水出西南长溪下,东北历关注滱。滱水南,山上起御坐于松园,建祗洹于东圃。东北二面,岫嶂高深,霞峰隐日,水望澄明。渊无潜甲,行李所径,鲜不徘徊忘返矣。又东南过中山上曲阳县北,恒水从西来注之。
滱水自倒马关南流与大岭水合,水出山西南大岭下,东北流出峡,峡右山侧,有祗洹精庐,飞陆陵山,丹盘虹梁,长津泛澜,萦带其下。东北流注于滱。滱水又屈而东合两岭溪水,水出恒山北阜,东北流历两岭间,北岭虽层陵云举,犹不若南峦峭秀,自水南步远峰,石隥逶迤,沿途九曲,历睇诸山,咸为劣矣,抑亦羊肠、邛崃之类者也。齐宋通和,路出其间。其水东北流,注于滱水。又东,左合悬水,水出山原岫盘谷,轻湍浚下,分石飞悬,一匹有余,直灌山际,白波奋流,自成潭渚。其水东南流,扬湍注于滱。滱水又东流历鸿山,世谓是处为鸿头,疑即《晋书地道记》所谓鸿上关者也。关尉治北平而画塞于望都东北,去北平不远,兼县土所极也。滱水于是,左纳鸿上水,水出西北近溪,东南流注于滱水也。
又东过唐县南。
滱水又东径左人城南。应劭曰:左人城在唐县西北四十里。县有雹水,亦或谓之为唐水也,水出中山城之西如北。城内有小山,在城西侧而锐上,若委粟焉,疑即《地道记》所云望都县有委粟关也。俗以山在邑中,故亦谓之中山城;以城中有唐水,因复谓之为广唐城也。《中山记》以为中人城,又以为鼓聚,殊为乖谬矣。言城中有山,故曰中山也,中山郡治。京相璠曰:今中山望都东二十里有故中人城。望都城东有一城名尧姑城,本无中人之传,璠或以为中人,所未详也。《中山记》所言中人者,城东去望都故城十余里,二十里则减,但苦其不东,观夫异说,咸为爽矣。今此城于卢奴城北如西六十里。城之西北,泉源所导,西径郎山北,郎、唐音读近,实兼唐水之传。西流历左人亭注滱水。滱水又东,左会一水,水出中山城北郎阜下,亦谓之唐水也。然于城非在西,俗又名之为雹水,又兼二名焉。西南流入滱,并所未详,盖传疑耳。滱水又东,恒水从西来注之。自下滱水兼纳恒川之通称焉。即《禹贡》所谓恒、卫既从也。滱水又东,右苞马溺水,水出上曲阳城东北马溺山,东北流径伏亭。《晋书地道记》曰:望都县有马溺关。《中山记》曰:八渡马溺。是山曲要害之地,二关势接,疑斯城即是关尉宿治,异目之来,非所详矣。马溺水又东流注于滱。滱水又东径中人亭南。《春秋左传》昭公十三年,晋荀吴率师侵鲜虞及中人,大获而归者也。滱水又东径京丘北,世谓之京陵,南对汉中山顷王陵,滱水北对君子岸,岸上有哀王子宪王陵,坎下有泉源,积水亦曰泉上岸。滱水又东径白土北,南即靖王子康王陵,三坟并列者是。滱水又东径乐羊城北,《史记》称魏文侯使乐羊灭中山,盖其故城中山所造也,故城得其名。滱水又东径唐县故城南,此二城俱在滱水之阳,故曰滱水径其南。城西又有一水,导源县之西北平地,泉涌而出,俗亦谓之为唐水也。东流至唐城西北隅,堨而为湖,俗谓之唐池。莲荷被水,嬉游多萃其上,信为胜处也。其水南入小沟,下注滱水,自上历下,通禅唐川之兼称焉。应劭《地理风俗记》曰:唐县西四十里得中人亭。今于此城取中人乡,则四十也。唐水在西北入滱,与应符合,又言尧山者,在南则无山以拟之为非也。阚駰《十三州志》曰:中山治卢奴,唐县故城在国北七十五里。駰所说北则非也。《史记》曰:帝喾氏没,帝尧氏作,始封于唐。望都县在南,今此城南对卢奴故城,自外无城以应之。考古知今,事义全违,俗名望都故城则八十许里,距中山城则七十里,验途推邑,宜为唐城。城北去尧山五里,与七十五里之说相符。然则俗谓之都山,即是尧山,在唐东北望都界。皇甫谧曰:尧山亦名豆山。今山于城北如东,崭绝孤峙,虎牙桀立,山南有尧庙,是即尧所登之山者也。《地理志》曰:尧山在南。今考此城之南,又无山以应之,是故先后论者,咸以《地理记》之说为失。又即俗说以唐城为望都城者,自北无城以拟之,假复有之,途程纡远,山河之状全乖,古证传为疏罔。是城西北豆山西足,有一泉源,东北流径豆山下合苏水,乱流转注东入滱,是岂唐水平?所未详也。又于是城之南如东十余里,有一城,俗谓之高昌县城,或望都之故城也。县在唐南,皇甫谧曰:相去五十里。稽诸城地,犹十五里,盖书误耳。此城之东,有山孤峙,世以山不连陵,名之曰孤山,孤、都声相近,疑即所谓都山也。《帝王世纪》曰:尧母庆都所居,故县目曰望都。张晏曰:尧山在北,尧母庆都山在南,登尧山见都山,故望都县以为名也。唐亦中山城也,为武公之国,周同姓。周之衰也,国有赤狄之难,齐桓霸诸侯,疆理邑土,遣管仲攘戎狄,筑城以固之。其后,桓公不恤国政,周王问太史余曰:今之诸侯,孰先亡乎?对曰:天生民而令有别,所以异禽兽也。今中山淫昏康乐,逞欲无度,其先亡矣。后二年果灭。魏文侯以封太子击也,汉高祖立中山郡,景帝三年为王国,王莽之常山也。魏皇始二年,破中山,立安州,天兴三年,改曰定州,治水南卢奴县之故城。昔耿伯昭归世祖于此处也。滱水之右,卢水注之,水上承城内黑水池。《地理志》曰卢水出北平,疑为疏阔;阚駰、应劭之徒,咸亦言是矣。余按卢奴城内西北隅有水,渊而不流,南北百步,东西百余步水,色正黑,俗名曰黑水池。或云水黑曰卢,不流曰奴,故此城藉水以取名矣。池水东北际水,有汉中山王故宫处,台殿观榭,皆上国之制,简王尊贵,壮丽有加,始筑两宫,开四门,穿北城,累石为窦,通池流于城中,造鱼池、钓台、戏马之观,岁久颓毁,遗基尚存,今悉加土,为利刹灵图。池之四周。居民骈比。填褊秽陋,而泉源不绝。暨赵石建武七年,遣北中郎将始筑小城,兴起北榭,立宫造殿,后燕因其故宫,建都中山小城之南,更筑隔城,兴复宫观,今府榭犹传故制,自汉及燕。池水径石窦,石窦既毁,池道亦绝,水潜流出城,潭积微涨,涓水东北注于滱。滱水又东径汉哀王陵北。冢有二坟,故世谓之两女陵,非也。哀王是靖王之孙,康王之子也。滱水又东,右会长星沟,沟出上曲阳县西北长星渚,渚水东流又合洛光水,水出洛光沟,东入长星水,乱流东径恒山下庙北。汉末丧乱,山道不通,此旧有下阶神殿,中世以来,岁书法族焉。晋、魏改有东西二庙,庙前有碑阙,坛场列柏焉。其水又东径上曲阳县故城北,本岳牧朝宿之邑也。古者,天子巡狩,常以岁十一月至于北岳,侯伯皆有汤沐邑,以自斋洁。周昭王南征不还,巡狩礼废,邑郭仍存。秦罢井田,因以立县。城在山曲之阳,是曰曲阳,有下,故此为上矣。王莽之常山亭也。又东南流,胡泉水注之,水首受胡泉,径上曲阳县南,又东径平乐亭北,左会长星川,东南径卢奴城南,又东北,川渠之左有张氏墓,冢有汉土谷太守议郎张平仲碑,光和中立。川渠又东北合滱水,水有穷通,不常津注。
又东过安憙县南,县,故安险也。其地临险,有井涂之难,汉武帝元朔五年,封中山靖王子刘应为侯国,王莽更名宁险,汉章帝改曰安憙,《中山记》曰:县在唐水之曲,山高岸险,故曰安险;邑丰民安,改曰安憙。秦氏建元中,唐水泛涨,高岸崩颓,城角之下有大积木,交横如梁住焉。后燕之初,此木尚在,未知所从。余考记稽疑,盖城地当初,山水渀荡,漂沦巨栰,阜积于斯,沙息壤加,渐以成地,板筑既兴,物固能久耳。滱水又东径乡城北,旧卢奴之乡也。《中山记》曰:卢奴有三乡,斯其一焉,后隶安吝。城郭南有汉明帝时孝子王立碑。
又东过安国县北,滱水历县东分为二水,一水枝分,东南流径解渎亭南。汉顺帝阳嘉元年,封河间孝王子淑于解渎亭为侯国,孙宏,即灵帝也。又东南径任丘城南,又东南径安郭亭南。汉武帝元朔五年,封中山靖王子刘传富为侯国。其水又东南流,入于滹沱。滱水又东北流径解渎亭北而东北注。
又东过博陵县南,滱水东北径蠡吾县故城南。《地理风俗记》曰:县,故饶阳之下乡者也,自河问分属博陵。汉安帝元初七年,封河间王开子翼为都乡侯,顺帝永建五年,更为侯国也。又东北径博陵县故城南,即古陆成。汉武帝元朔二年,封中山靖王子刘贞为侯国者也。《地理风俗记》曰:博陵县,《史记》蠡吾故县矣。汉质帝本初元年,继孝冲为帝,追尊父翼陵曰博陵,因以为县,又置郡焉。汉末,罢还安平,晋太始年复为郡,今谓是城为野城。滱水又东北径侯世县故城南,又东北径陵阳亭东,又北,左会博水,水出望都县,东南流径其县故城南,王莽更名曰顺调矣。又东南,潜入地下。博水又东南循渎,重源涌发。东南径三梁亭南,疑即古勺梁也。《竹书纪年》曰:燕人伐赵,围浊鹿,赵武灵王及代人救浊鹿,败燕师于勺梁者也。今广昌东岭之东有山,俗名之曰浊鹿逻。城地不远,土势相邻,以此推之,或近是矣,所未详也。博水又东南径谷粱亭南,又东径阳城县,散为泽渚。诸水潴涨,方广数里,匪直蒲笋是丰,实亦偏饶菱藕,至若娈婉丱童,及弱年崽子,或单舟采菱,或叠舸折芰,长歌阳春,爱深绿水,掇抬者不言疲,谣咏者自流响,于时行旅过瞩,亦有慰于羁望矣。世谓之为阳城淀也。阳城县故城近在西北,故陂得其名焉。《郡国志》曰:蒲阴县有阳城者也。今城在县东南三十里。其水又伏流循渎,届清梁亭西北,重源又发。博水又东径白堤亭南,又东径广望县故城北。汉武帝元朔二年,封中山靖王子刘忠为侯国。又东合堀沟,沟上承清梁陂。又北径清凉城东,即将梁也。汉武帝元朔二年,封中山靖王子刘朝平为侯国。其水东北入博水。博水又东北,左则濡水注之,水出蒲阴县西昌安郭南。《中山记》曰:郭东有舜氏甘泉,有舜及二妃祠。稽诸传记,无闻此处,世代云远,异说之来,于是乎在矣。其水自源东径其县故城南,在渚回湍,率多曲复,亦谓之为曲逆水也。张晏曰:濡水于城北曲而西流,是受此名,故县亦因水名而氏曲逆矣。《春秋左传》哀公四年,齐国夏伐晋,取曲逆是也。汉高帝击韩王信,自代过曲逆,上其城,望室字甚多,曰:壮哉!吾行天下,惟洛阳与是耳。诏以封陈平为曲逆侯。王莽更名顺平。濡水又东与苏水合,水出县西南近山,东北流径尧姑亭南,又东径其县入濡。濡水又东,得蒲水口,水出西北蒲阳山,西南流,积水成渊,东西百步,南北百余步,深而不测。蒲水又东南流,水侧有古神祠,世谓之为百祠。亦曰蒲上祠,所未详也。又南径阳安亭东。《晋书地道记》曰:蒲阴县有阳安关,盖阳安关都尉治。世俗名斯川为阳安扩。蒲水又东南历扩,径阳安关下,名关皋为唐头坂。出关北流,又东流径夏屋故城,实中险绝。《竹书纪年》曰:魏殷臣、赵公孙裒伐燕,还取夏屋,城曲逆者也。其城东侧,因阿仍墉筑一城,世谓之寡妇城,贾复从光武追铜马、五幡于北平所作也。世俗音转,故有是名矣。其水又东南流径蒲阴县故城北,《地理志》曰:城在蒲水之阴。汉章帝章和二年,行巡北岳,以曲逆名不善,因山水之名,改曰蒲阴焉。水右合鱼水,水出北平县西南鱼山,山石若巨鱼,水发其下,故世俗以物色名川。又东流注于蒲水,又东入濡。故《地理志》曰:蒲水、苏水,并从县东入濡水。又东北径乐城南,又东入博水,自下博水亦兼濡水通称矣。《春秋》昭公七年,齐与燕盟于濡上。杜预曰:濡水出高阳县东北,至河间鄚县入易水,是濡水与滹沱、滱、易互举通称矣。博水又东北,徐水注之,水西出广昌县东南大岭下。世谓之广昌岭,岭高四十余里,二十里中委折五回,方得达其上岭,故岭有五回之名,下望层山,盛若蚁蛭,实兼孤山之称,亦峻竦也。徐水三源奇发,齐泻一涧,东流北转径东山下,水西有御射碑。徐水又北流西屈径南崖下,水阴又有一碑。徐水又随山南转径东崖下,水际又有一碑。凡此三铭,皆翼对层峦,岩障深高,壁立霞峙。石文云:皇帝以太延元年十二月,车驾东巡,径五回之险邃,览崇岸之竦峙,乃停驾路侧,援弓而射之,飞矢逾于岩山,刊石用赞元功。夹碑并有层台二所,即御射处也。碑阴皆列树碑官名。徐水东北屈径郎山,又屈径其山南,众岑竞举,若竖鸟翅,立石崭岩,亦如剑秒,极地险之崇峭。汉武之世,戾太子以巫蛊出奔,其子远遁斯山,故世有郎山之名。山南有《郎山君碑》,事具其文。徐水又径郎山君中子触锋将军庙南,庙前有碑,晋惠帝永康元年八月十四日壬寅;发诏锡君父子,法祠其碑。刘曜光初七年,前顿丘太守郎宣,北平太守阳平邑振等,共修旧碑,刻石树颂焉。徐水又径北平县,县界有汉熹平四年幽、冀二州以戊子诏书,遣冀州从事王球,幽州从事张昭,郡县分境,立石标界,具揭石文矣。徐水又东南流历石门中,世俗谓之龙门也。其山上合下开,开处高六丈,飞水历其间,南出乘崖,倾涧泄注,七丈有余,渀荡之音,奇为壮猛,触石成井,水深不测,素波自激,涛襄四陆,瞰之者惊神,临之者骇魄矣。东南出山径其城中,有故碑,是太白君碑,郎山君之元子也。其水又东流,汉光武追铜马、五幡于北平破之于顺水北,乘胜追北,为其所败,短兵相接,光武自投崖下,遇突骑王丰,于是授马退保范阳。顺水,盖徐州之别名也。徐水又东径蒲城北,又东径清苑城。又东南与卢水合,水出蒲城西,俗谓之泉头水也。《地理志》曰北平县有卢水,即是水也。东径其城,又东南,左入徐水。《地理志》曰:东至高阳入博,今不能也。徐水又东,左合曹水,水出西北朔宁县曹河泽,东南流,左合岐山之水,水出岐山,东径邢安城北,又东南入曹河。曹水又东南径北新城县故城南,王莽之朔平县也。曹水又东人于徐水。徐水又东南径故城北,俗谓之祭隅城,所未详也。徐水又东注博水。《地理志》曰:徐水出北平,东至高阳入于博,又东入滱、《地理志》曰:博水自望都,东至高阳入于滱是也。
又东北入于易。
滱水又东北径依城北,世谓之依城河。《地说》无依城之名,即古葛城也。《郡国志》曰:高阳有葛城,燕以与赵者也。滱水又东北径阿陵县故城东,王莽之阿陆也。建武二年,更封左将军任光为侯国。滱水东北至长城注于易水者也。


卷十二  圣水、巨马水 
圣水出上谷,故燕地,秦始皇二十三年,置上谷郡。王隐《晋书地道志》曰:郡在谷之头,故因以上谷名焉。王莽更名朔调也。水出郡之西南圣水谷,东南流径大防岭之东首,山下有石穴,东北洞开,高广四五丈,入穴转更崇深,穴中有水。耆旧传言,昔有沙门释惠弥者,好精物隐,尝篝火寻之,傍水入穴三里,有余穴分为二:一穴殊小,西北出,不知趣诣;一穴西南出,入穴经五六日方还,又不测穷深。其水夏冷冬温,春秋有白鱼出穴,数日而返,人有采捕食者,美珍常味,盖亦丙穴嘉鱼之类也。是水东北流入圣水。圣水又东径玉石山,谓之玉石口,山多珉玉、燕石,故以玉石名之。其水伏流里余,潜源东出,又东。颓波泻涧,一丈有余,屈而南流也。
东过良乡县南,圣水南流,历县西转,又南径良乡县故城西,王莽之广阳也,有防水注之,水出县西北大防山南,而东南流径羊头阜下,俗谓之羊头溪。其水又东南流,至县东入圣水。圣水又南与乐水合,水出县西北大防山南,东南流,历县西而东南流注圣水。圣水又东径其县故城南,又东径圣聚南,盖藉水而怀称也。又东与侠河合,水出良乡县西甘泉原东谷,东径西乡县故城北,王莽之移风也,世谓之都乡城。案《地理志》,涿郡有西乡县而无都乡城,盖世传之非也。又东径良乡城南,又东北注圣水,世谓之侠活河,又名之曰非理之沟也。
又东过阳乡县北,圣水自涿县东与桃水合,水首受涞水于徐城东南良乡,西分垣水,世谓之南沙沟,即桃水也。东径逎县北,又东径涿县故城下与涿水合,世以为涿水,又亦谓之桃水,出涿县故城西南奇沟东八里大坎下,数泉同发,东径桃仁墟北,或曰因水以名墟,则是桃水也,或曰终仁之故居,非桃仁也。余案《地理志》,桃水上承涞水,此水所发,不与《志》同,谓终为是。又东北与乐堆泉合,水出堆东,东南流注于涿水。涿水又东北径涿县故城西注于桃。应劭曰:涿郡,故燕,汉高帝六年置,其南有涿水郡,盖氏焉。阚駰亦言是矣。今于涿城南无水以应之,所有惟西南有是水矣。应劭又云:涿水出上谷涿鹿县,余案涿水自涿鹿东注漯水。漯水东南径广阳郡与涿郡分水,汉高祖六年,分燕置涿郡,涿之为名;当受涿水通称矣,故郡、县氏之。但物理潜通,所在分发,故在匈奴为涿耶水,山川阻阔,并无沿注之理,所在受名者,皆是经隐显相关,遥情受用,以此推之,事或近矣,而非所安也。桃水又东径涿县故城北,王莽更名垣翰,晋大始元年,改曰范阳郡。今郡理涿县故城,城内东北角有晋康王碑,城东有范阳王司马虓庙碑。桃水又东北与垣水会,水上承涞水,于良乡县分桃水,世谓之北沙沟。故应劭曰:垣水出良乡,东径垣县故城北。《史记音义》曰:河间有武垣县,涿有垣县。汉景帝中三年,封匈奴降王赐为侯国,王莽之垣翰亭矣,世渭之顷城,非也。又东径顷,亦地名也,故有顷上言,世名之顷前河。又东,洛水注之,水上承鸣泽渚,渚方十五里。汉武帝元封四年,行幸鸣泽者也。服虔曰:泽名,在逎县北界。即此泽矣。西则独树水注之,水出逎县北山,东入渚。北有甘泉水注之,水出良乡西山,东南径西乡城西,而南注鸣泽渚。渚水东出为洛水,又东径西乡城南,又东径垣县而南入垣水。垣水又东径涿县北,东流注于桃。故应劭曰:垣水东入桃。阚駰曰:至阳乡注之。今案经脉而不能届也。桃水东径阳乡,东注圣水。圣水又东,广阳水注之,水出小广阳西山,东径广阳县故城北;又东,福禄水注焉。水出西山,东南径广阳县故城南,东入广阳水,乱流东南至阳乡县,右注圣水。圣水又东南径阳乡城西,不径其北矣。县,故涿之阳亭也。《地理风俗记》曰:涿县东五十里有阳乡亭,后分为县。王莽时,更名章武,即长乡县也。案《太康地记》,涿有长乡而无阳乡矣。圣水又东径长兴城南,又东径方城县故城北.李牧伐燕取方城是也,魏封刘放为侯国。圣水又东,左会白祀沟,沟水出广阳县之娄城东,东南流,左合娄城水,水出平地。导源东南流,右注白祀水,乱流东南径常道城西。故乡亭也,西去长乡城四十里,魏少帝璜甘露三年所封也。又东南入圣水。圣水又东南径韩城东。《诗·韩奕章》曰:溥彼韩城,燕师所完,王锡韩侯,其追其貊,奄受北国。郑玄曰:周封韩侯,居韩城为侯伯,言为猃夷所逼,稍稍东迁也。王肃曰:今涿郡方城县有韩侯城,世谓之寒号城,非也。圣水又东南流,右会清淀水,水发西淀,东流注圣水,谓之刘公口也。
又东过安次县南,东入于海。
圣水又东径勃海安次县故城南。汉灵帝中平三年,封荆州刺史王敏为侯国。又东南流注于巨马河而不达于海也。
巨马河出代郡广昌县涞山,即涞水也,有二源,俱发涞山,东径广昌县故城南,王莽之广屏矣,魏封乐进为侯国。滦水又东北径西射鱼城东南而东北流;又径东射鱼城南,又屈径其城东。《竹书纪年》曰:荀瑶伐中山取穷鱼之丘。穷、射字相类,疑即此城也,所未详矣。涞水又径三女亭西,又径楼亭北,左属白涧溪,水有二源,合注一川,川石皓然,望同积雪,故以物色受名。其水又东北流,谓之石槽水,伏流地下,溢则通津委注,谓之白涧口。涞水又东北,桑谷水注之,水南发桑溪,北注涞水。涞水又北径小黉东,又东径大黉南,盖霍原隐居教授处也。徐广云:原隐居广阳山,教授数千人,为王浚所害,虽千古世悬,犹表二黉之称,既无碑颂,竟不知定谁居也。涞水又东北历紫石溪口与紫水合,水北出圣人城北大亘下,东南流,左会磊砢溪水,盖山崩委涧,积石沦隍,故溪涧受其名矣。水出东北,西南流注紫石溪水。紫石溪水又径圣人城东,又东南,右会檐车水,水出檐车硎,东南流径圣人城南,南流注紫石水,又南注于涞水。涞水又东南径榆城南,又屈径其城东,谓之榆城河。涞水又南径藏刀山下,层岩壁立,直上干霄,远望崖侧,有若积刀,镮镮相比。咸悉西首。涞水东径徐城北,故渎出焉,世谓之沙沟水。又东,督亢沟出焉。一水东南流,即督亢沟也;一水西南出,即涞水之故渎矣。水盛则长津宏注,水耗则通波潜伏,重源显于逎县,则旧川矣。
东过逎县北,涞水上承故渎于县北垂,重源再发,结为长潭,潭广百许步,长数百步,左右翼带涓流,控引众水,自成渊渚。长川漫下十许里,东南流径逎县故城东。汉景帝中三年,以封匈奴降王隆疆为侯国,王莽更名逎屏也。谓之巨马河,亦曰渠水也。又东南流,袁本初遣别将崔巨业攻固安不下,退还,公孙瓒追击之于巨马水,死者六七千人,即此水也。又东南径范阳县故城北,易水注之。
又东南过容城县北,巨马水又东,郦亭沟水注之,水上承督亢沟水于逎县东,东南流,历紫渊东。余六世祖乐浪府君,自涿之先贤乡爰宅其阴,西带巨川,东翼兹水,枝流津通,缠络墟圃,匪直田渔之赡可怀,信为游神之胜处也。其水东南流,又名之为郦亭沟。其水又西南转,历大利亭南入巨马水。又东径容城县故城北。又东,督亢沟水注之,水上承涞水于涞谷,引之则长津委注,遏之则微川辍流,水德含和,变通在我。东南流径逎县北,又东径涿县郦亭楼桑里南,即刘备之旧里也。又东径督亢泽,泽苞方城县,县故属广阳,后隶于涿。《郡国志》曰:县有督亢亭。孙畅之《述画》有《督亢地图》,言燕太子丹使荆轲赍入秦,秦王杀轲,图亦绝灭。《地理书上古圣贤冢地记》曰,督亢地在涿郡。今故安县南有督亢陌,幽州南界也。《风俗通》曰:沆,漭也。言乎淫淫漭漭,无崖际也。沆泽之无水,斥卤之谓也。其水自泽枝分,东径涿县故城南,又东径汉侍中卢植墓南,又东,散为泽渚,督亢泽也。北屈注于桃水。督亢水又南,谓之白沟水,南径广阳亭西,而南合枝沟,沟水西受巨马河,东出为枝沟,又东注白沟,白沟又南,入于巨马河。巨马河又东南径益昌县,护淀水右注之,水上承护陂于临乡县故城西,东南径临乡城南,汉封广阳顷王千云为侯国。《地理风俗记》曰:方城南十里有临乡城,故县也。淀水又东南径益昌县故城西,南入巨马水。巨马水东径益昌县故城南,汉封广阳顷王子婴为侯国,王莽之有秩也。《地理风俗记》曰:方城县东八十里有益昌城,故县也。又东,八丈沟水注之,水出安次县东北平地,东南径安次城东,东南径泉州县故城西,又南,右合滹沱河枯沟,沟自安次西北,东径常道城东、安次县故城西,晋司空刘琨所守以拒石勒也。又东南至泉州县西南,东入八丈沟,又南入巨马河,乱流东注也。
又东过勃海东平舒县北,东入于海。
《地理志》曰:涞水东南至容城入于河。河,即濡水也,盖互以明会矣。巨马水于平舒城北,南入于滹沱,而同归于海也。


卷十三  漯水 
漯水出雁门阴馆县,东北过代郡桑乾县南,漯水出于累头山,一曰治水。泉发于山侧,沿波历涧,东北流出山,径阴馆县故城西。县,故楼烦乡也。汉景帝后三年置,王莽更名富臧矣。魏皇兴三年,齐平,徙其民于县,立平齐郡。漯水又东北流,左会桑乾水,县西北上平,洪源七轮,谓之桑乾泉,即溹涫水者也。耆老云:其水潜通,承太原汾阳县北燕京山之大池,池在山原之上,世谓之天池,方里余,澄渟镜净,潭而不流,若安定朝那之湫渊也。清水流潭,皎焉冲照,池中尝无斥草,及其风箨有沦,辄有小鸟翠色,投渊衔出,若会稽之耘鸟也。其水阳熯不耗,阴霖不滥,无能测其渊深也。古老相传,言尝有人乘车于池侧,忽过大风,飘之于水,有人获其轮于桑乾泉,故知二水潜流通注矣。池东隔阜又有一石池,方可五六十步,清深镜洁,不异大池。桑乾水自源东南流,右会马邑川水,水出马邑西川,俗谓之磨川矣。盖狄语音讹,马、磨声相近故尔。其水东径马邑县故城南。于宝《搜神记》曰:昔秦人筑城于武州塞内以备胡,城将成而崩者数矣。有马驰走一地,周旋反覆,父老异之,因依以筑城,城乃不崩,遂名之为马邑。或以为代之马城也,诸记纷竞,未识所是。汉以斯邑封韩王信,后为匈奴所围,信遂降之。王莽更名之曰章昭。其水东注桑乾水,桑乾水又东南流,水南有故城,东北临河,又东南,右合漯水,乱流枝水南分。桑乾水又东,左合武州塞水,水出故城,东南流出山,径日没城南,盖夕阳西颓,戎车所薄之城故也。东有日中城,城东又有早起城,亦曰食时城,在黄爪阜北曲中。其水又东流,右注桑乾水。桑乾水又东南径黄瓜阜曲西,又屈径其堆南。徐广曰:猗卢废嫡子曰利孙于黄瓜堆者也。又东,右合枝津,枝津上承桑乾河,东南流径桑乾郡北,大魏因水以立郡,受厥称焉。又东北,左合夏屋山水,水南出夏屋山之东溪,西北流径故城北,所未详也。又西北入桑乾枝水,桑乾枝水又东流,长津委浪通结两湖,东湖西浦,渊潭相接,水至清深,晨凫夕雁,泛滥其上,黛甲素鳞,潜跃其下,俯仰池潭。意深鱼鸟,所寡惟良木耳。俗谓之南池,池北对陶县之故城,故曰南池也。南池水又东北注桑乾水,为漯水,自下并受通称矣。漯水又东北径石亭西,盖皇魏大赐三年之所经建也。漯水又东北径白狼堆南,魏烈祖道武皇帝于是遇白狼之瑞。故斯阜纳称焉。阜上有故宫,庙楼榭基雉尚崇,每至鹰隼之秋,羽猎之日,肆阅清野,为升眺之逸地矣。漯水又东流四十九里,东径巨魏亭北,又东,崞川水注之,水南出崞县故城南,王莽之崞张也。县南面玄岳,右背崞山,处二山之中,故以崞张为名矣。其水又西出山。谓之崞口,北流径繁畤县故城东,王莽之当要也。又北径巨魏亭东,又北径剧阳县故城西,王莽之善阳也。按《十三州志》曰:在阴馆县东北一百三里。其水又东注于漯水,漯水又东径班氏县南,如浑水注之,水出凉城旋鸿县西南五十余里,东流径故城南,北俗谓之独谷孤城,水亦即名焉。东合旋鸿池水,水出旋鸿县东山下,水积成池,北引鱼水,水出鱼溪,南流注池。池水吐纳川流,以成巨沼,东西二里,南北四里,北对凉川城之南池,池方五十里,俗名乞伏袁池。虽隔越山阜,鸟道不远,云霞之间常有。西南流径旋鸿县南,右合如浑水。是总二水之名矣。如浑水又东南流径永固县,县以太和中,因山堂之目以氏县也。右会羊水,水出平城县之西苑外武州塞,北出东转,径燕昌城南。按《燕书》,建兴十年,慕容垂自河西还,军败于参合,死者六万人。十一年,垂众北至参合,见积骸如山,设祭吊之礼,死者父兄皆号泣,六军哀恸,垂惭愤呕血,因而寝疾焉。舆过平城北四十里,疾笃,筑燕昌城而还,即此城也。北俗谓之老公城。羊水又东注于如浑水,乱流径方山南,岭上有文明太皇太后陵,陵之东北有高祖陵。二陵之南有永固堂,堂之四周隅雉,列榭阶栏及扉户、梁壁、椽瓦,悉文石也。檐前四柱,采洛阳之八风谷黑石为之,雕镂隐起,以金银间云矩,有若锦焉。堂之内外.四侧结两石跌,张青石屏风,以文石为缘,并隐起忠孝之容,题刻贞顺之名。庙前镌石为碑兽,碑石至佳,左右列柏,四周迷禽暗日。院外西侧,有思远灵图,图之西有斋堂,南门表二石阙,阙下斩山,累结御路,下望灵泉宫池,皎若圆镜矣。如浑水又南至灵泉池,枝津东南注池,池东西百步,南北二百步。池渚旧名白杨泉,泉上有白杨树,因以名焉,其犹长杨、五柞之流称矣。南面旧京,北背方岭,左右山原,亭观绣峙,方湖反景,若三山之倒水下。如浑水又南径北宫下,旧宫人作薄所在。如浑水又南,分为二水,一水西出南屈,入北苑中。历诸池沼,又南径虎圈东,魏太平真君五年,成之以牢虎也。季秋之月,圣上亲御圈上。敕虎士效力于其下,事同奔戎,生制猛兽,即《诗》所谓袒裼暴虎,献于公所也。故魏有《捍虎图》也。又径平城西郭内,魏太常七年所城也。城周西郭外有郊天坛,坛之东侧有《郊天碑》,建兴四年立。其水又南屈,径平城县故城南。《史记》曰:高帝先至平城。《史记音义》曰在雁门,即此县矣。王莽之平顺也。魏天兴二年,迁都于此,太和十六年,破安昌诸殿,造太极殿东、西堂及朝堂,夹建象魏、乾元、中阳、端门、东、西二掖门、云龙、神虎、中华诸门,皆饰以观阁。东堂东接太和殿,殿之东阶下有一碑,太和中立,石是洛阳八风谷之缁石也。太和殿之东北,接紫宫寺,南对承贤门,门南即皇信堂,堂之四周,图古圣、忠臣、烈士之容,刊题其侧,是辩章郎彭城张僧达、乐安蒋少游笔。堂南对白台,台甚高广,台基四周列壁,阁道自内而升,国之图箓秘籍,悉积其下。台西即朱明阁,直侍之官,出入所由也。其水夹御路,南流径蓬台西。魏神瑞三年,又建白楼,楼甚高谏,加观榭于其上,表里饰以石粉,皜曜建素,赭白绮分,故世谓之白楼也。后置大鼓于其上,晨昏伐以千椎,为城里诸门启闭之候,谓之戒晨鼓也。又南径皇舅寺西,是太师昌黎王冯晋国所造,有五层浮图,其神图像皆合青石为之,加以金、银、火齐,众彩之上,炜炜有精光。又南径永宁七级浮图西,其制甚妙,工在寡双。又南,远出郊郭,弱柳荫街,丝杨被浦,公私引裂,用周园溉,长塘曲池,所在布濩,故不可得而论也。一水南径白登山西,服虔曰:白登,台名也,去平城七里。如淳曰:平城旁之高城若丘陵矣。今平城东十七里有台,即白登台也,台南对冈阜,即白登山也。故《汉书》称上遂至平城,上白登者也。为匈奴所围处,孙畅之《述画》曰:汉高祖被围七日,陈平使能画作美女,送与冒顿,阏氏恐冒顿胜汉,其宠必衰,说冒顿解围于此矣。其水又径宁先宫东,献文帝之为太上皇,所居故宫矣。宫之东次,下有两石柱,是石虎邺城东门石桥柱也。按柱勒赵建武中造,以其石作工妙,徙之于此。余为尚书祠部,与宜都王穆罴同拜北郊,亲所经见,柱侧悉镂云矩,上作蟠螭,甚有形势,信为工巧,去《子丹碑》则远矣。其水又南径平城县故城东,司州代尹治皇都洛阳,以为恒州。水左有大道坛庙,始光二年,少室道士寇谦之所议建也。兼诸岳庙碑,亦多所署立,其庙阶三成,四周栏槛上阶之上,以木为圆基,令互相枝梧,以版砌其上,栏陛承阿上圆,制如明堂,而专室四户,室内有神坐,坐右列玉磬,皇舆亲降,受箓灵坛,号曰天师,宣扬道式,暂重当时。坛之东北,旧有静轮宫,魏神四年造,抑亦柏梁之流也。台榭高广,超出云间,欲令上延霄客,下绝嚣浮。太平真君十一年,又毁之。物不停固,白登亦继褫矣。水右有三层浮图,真容鹫架悉结石也。装制丽质,亦尽美善也。东郭外,太和中阉人宕昌公钳耳庆时,立祗洹舍于东皋,椽瓦梁栋,台壁櫺陛,尊容圣像,及床坐轩帐,悉青石也。图制可观,所恨惟列壁合石,疏而不密。庭中有《祗洹碑》,碑题大篆,非佳耳。然京邑帝里,佛法丰盛,神图妙塔,桀跱相望,法轮东转,兹为上矣。其水自北苑南出,历京城内。河干两湄,太和十年累石结岸,夹塘之上,杂树交荫,郭南结两石桥,横水为梁。又南径藉田及药圃西、明堂东,明堂上圆下方,四周十二堂九室,而不为重隅也。室外柱内,绮井之下,施机轮,饰缥碧,仰象天状,画北道之宿焉,盖天也。每月随斗所建之辰,转应天道,此之异古也。加灵台于其上,下则引水为辟雍,水侧结石为塘,事准古制,是太和中之所经建也。如浑水又南与武州川水会,水出县西南山下,二源翼导,俱发一山,东北流,合成一川,北流径武州县故城西,王莽之桓州也。又东北,右合黄水,水西出黄阜下,东北流,圣山之水注焉,水出西山,东流注于黄水。黄水又东注武州川,又东历故亭北,右合火山西溪水,水导源火山,西北流,山上有火井,南北六七十步,广减尺许,源深不见底,炎势上升,常若微雷发响,以草爨之,则烟腾火发。东方朔《神异传》云:南方有火山焉,长四十里,广四、五里,其中皆生不烬之木,昼夜火然,得雨猛风不灭。火中有鼠,重百斤,毛长二尺余,细如丝,色白,时时出外,以水逐而沃之则死,取其毛绩以为布,谓之火浣布。是山亦其类也,但卉物则不能。然其山以火从地中出,故亦名荧台矣。火井东五六尺,又东有汤井,广轮与火井相状,热势又同,以草内之,则不然,皆沾濡露结,故俗以汤井为目。井东有火井祠,以时祀祭焉。井北百余步有东、西谷,广十许步,南崖下有风穴,厥大容人,其深不测,而穴中肃肃,常有微风,虽三伏盛暑,犹须袭裘,寒吹陵人,不可暂停。而其山出雏乌,形类雅乌,纯黑而姣好,音与之同,缋采绀发,觜若丹砂,性驯良而易附,丱童幼子,捕而执之,赤觜乌亦曰阿雏乌,按《小尔雅》,纯黑反哺,谓之慈乌;小而腹下白不反哺者,谓之雅乌;白脰而群飞者,谓之燕乌;大而白脰者,谓之苍乌。《尔雅》曰:斯,卑居也。孙炎曰:卑居,楚乌。犍为舍人以为壁居。《说文》谓之雅。雅,楚乌。《庄子》曰:雅,贾矣。马融亦曰:贾,乌也。又案《瑞应图》,有三足乌、赤乌、白乌之名,而无记于此乌,故书其异耳。自恒山已北,并有此矣。其水又东北流注武州川水,武州川水又东南流,水侧有石祗洹舍并诸窟室,比丘尼所居也。其水又东转径灵岩南,凿石开山,因岩结构,真容巨壮,世法所希,山堂水殿,烟寺相望,林渊锦镜,缀目新眺。川水又东南流出山。《魏土地记》曰:平城西三十里武州塞口者也。自山口枝渠东出入苑,溉诸园池。苑有洛阳殿,殿北有宫馆。一水自枝渠南流东南出,火山水注之,水发火山东溪,东北流出山,山有石炭火之,热同樵炭也。又东注武州川,径平城县南,东流注如浑水。又南流径班氏县故城东,王莽之班副也。阚駰《十二州志》曰:班氏县在郡西南百里,北俗谓之去留城也。如浑水又东南流注于漯水。漯水又东径平邑县故城南,赵献侯十三年,城平邑。《地理志》属代,王莽所谓平胡也。《十三州志》曰:城在高柳南百八十里。北俗谓之丑寅城。漯水又东径沙陵南,魏金田之地也,事同曹武邺中走矣。漯水又东径狋氏县故城北,王莽更名之曰狋聚也。《十三州志》曰:县在高柳南百三十里,俗谓之苦力干城矣。漯水又东径道人县故城南,《地理志》,王莽之道仁也。《地理风俗记》曰:初筑此城,有仙人游其地,故因以为城名矣。今城北有渊,潭而不流,故俗谓之为平湖也。《十三州志》曰:道人城在高柳东北八十里,所未详也。漯水又东径阳原县故城南,《地理志》,代郡之属县也,北俗谓之比狋州城。漯水又东,安阳水注之,水出县东北潭中,北俗谓之太拔回水,自潭东南流注于漯水。又东径东安阳县故城北。赵惠文王三年,主父封长子章为代安阳君,此即章封邑,王莽之竟安也。《地理风俗记》曰:五原有西安阳,故此加东也。漯水又东径昌平县,温水注之,水出南坟下,三源俱导,合而南流,东北注漯水。漯水又东径昌平县故城北,王莽之长昌也。昔牵招为魏鲜卑校尉,屯此。漯水又东北径桑乾县故城西,又屈径其城北,王莽更名之曰安德也。《魏土地记》曰:代城北九十里有桑乾城。城西渡桑乾水,去城十里,有温汤,疗疾有验,《经》言出南,非也,盖误证矣。魏任城王彰以建安二十三年代乌丸,入涿郡,逐北遂至桑乾,正于此也。漯水又东流,祁夷水注之,水出平舒县,东径平舒县之故城南泽中。《史记》,赵孝成王十九年,以汾门予燕易平舒。徐广曰:平舒在代,王莽更名之曰平葆,后汉世祖建武七年,封扬武将军马成为侯国。其水控引众泉,以成一川。《魏土地记》曰:代城西九十里有平舒城,西南五里,代水所出,东北流,言代水非也。祁夷水又东北径兰亭南,又东北径石门关北,旧道出中山故关也。又东北流,水侧有故池。按《魏土地记》曰:代城西南三十里有代王鱼池,池西北有代王台,东去代城四十里。祁夷水又东北得飞狐谷,即广野君所谓杜飞狐之口也。苏林据郦公之说,言在上党,即实非也。如淳言在代,是矣。晋建兴中,刘琨自代出飞狐口,奔于安次,即于此道也。《魏土地记》曰:代城南四十里有飞狐关,关水西北流径南舍亭西,又径句琐亭西,西北注祁夷水。祁夷水又东北流径代城西,卢植言:初筑此城,板干一夜自移于此,故代西南五十里大泽中营城自护,结苇为九门。于是就以为治城。圆匝而不方,周四十六里,开九门,更名其故城曰东城。赵灭代,汉封孝文为代王。梅福上事曰:代谷者,恒山在其南,北塞在其北,谷中之地。上谷在东,代郡在西,是其地也。王莽更之曰厌狄亭。《魏土地记》曰:城内有二泉,一泉流出城西门,一泉流出城北门,二泉皆北注代水。祁夷水又东北,热水注之,水出绫罗泽,泽际有热水亭。其水东北流,注祁夷水。祁夷水又东北,谷水注之,水出昌平县故城南,又东北入祁夷水。祁夷水右会逆水,水导源将城东,西北流径将城北。在代城东北十五里,疑即东代矣,而尚传将城之名。卢植曰:此城方就而板于自移。应劭曰:城徙西南,去故代五十里,故名代曰东城。或传书倒错,情用疑焉,而无以辨之。逆水又西,注于祁夷之水,逆之为名,以西流故也。祁夷水东北径青牛渊,水自渊东注之。耆彦云:有潜龙出于兹浦,形类青牛焉,故渊潭受名矣。潭深不恻,而水周多莲藕生焉。祁夷水又北径一故城西,西去代城五十里,又疑是代之东城,而非所详也。又径昌平郡东,魏太和中置,西南去故城六十里。又北,连水入焉,水出瞀瞀县东,西北流,径瞀瞀县故城南,又西径广昌城南。《魏土地记》曰:代南二百里有广昌城,南通大岭。即实非也。《十三州记》曰:平舒城东九十里有广平城,疑是城也。寻其名状,忖理为非。又西径王莽城南,又西,到刺山水注之,水出到刺山西。山甚层峻,未有升其巅者。《魏土地记》曰:代城东五十里有到刺山,山上有佳大黄也。其水北流径一故亭东,城北有石人,故世谓之石人城,西北注连水。连水又北径当城县故城西,高祖十二年,周勃定代,斩陈稀于当城,即此处也。应劭曰:当桓都山作城,故曰当城也。又径故代东而西北流注祁夷水。祁夷水西有随山,山上有神庙,谓之女郎祠,方俗所祠也。祁夷水又北径桑乾故城东。而北流注于漯水。《地理志》曰:祁夷水出平舒县,北至桑乾入漯是也。漯水又东北径石山水口,水出南山,北流径空侯城东,《魏土地记》曰:代城东北九十里有空侯城者也。其水又东北流注漯水。漯水又东径潘县故城北,东合协阳关水,水出协溪。《魏土地记》曰:下洛城西南九十里有协阳关,关道西通代郡。其水东北流,历笄头山,阚駰曰:笄头山在潘城南。即是山也。又北径潘县故城,左会潘泉故渎,渎旧上承潘泉于潘城中,或云,舜所都也。《魏土地记》曰:下洛城西南四十里有潘城,城西北三里,有历山,山上有虞舜庙。《十三州记》曰:广平城东北百一十里有潘县。《地理志》曰:王莽更名树武。其泉从广十数步,东出城,注协阳关水,雨盛则通注,阳旱则不流,惟洴泉而已。关水又东北流,注于漯水。漯水又东径雍洛城南,《魏土地记》曰:下洛城西南二十里有雍洛城,桑乾水在城南东流者也。漯水又东径下洛县故城南,王莽之下忠也,魏燕州广宁县广宁郡治。《魏土地记》曰:去平城五十里,城南二百步有尧庙。漯水又东径高邑亭北,又东径三台北,漯水又东径无乡城北。《地理风俗记》曰:燕语呼毛为无,今改宜乡也。漯水又东,温泉水注之,水上承温泉于桥山下。《魏土地记》曰:下洛城东南四十里有桥山,山下有温泉,泉上有祭堂,雕檐华字,被于浦上,石池吐泉,汤汤其下,炎凉代序,是水灼焉,无改能治百疾,是使赴者若流。池水北流,人于漯水。漯水又东,左得于延水口,水出塞外柔玄镇西长川城南小山。《山海经》曰:梁渠之山,无草木,多金玉,修水出焉。东南流径且如县故城南,应劭曰:当城西北四十里有且如城,故县也。代称不拘名号变改。校其城郭,相去远矣。《地理志》曰:中部都尉治。于延水出县北塞外,即修水也。修水又东南径马城县故城北,《地理志》曰:东部都尉治。《十三州志》曰:马城在高柳东二百四十里,俗谓是水为河头,河头出戎方。土俗变名耳。又东径零丁城南,右合延乡水,水出县西山,东径延陵县故城北。《地理风俗记》曰:当城西北有延陵乡,故县也。俗指为琦城。又东径罗亭,又东径马城南,又东注修水,又东南于大宁郡北,右注雁门水。《山海经》曰:雁门之水,出于雁门之山。雁出其门,在高柳北,高柳在代中,其山重峦叠,霞举云高,连山隐隐,东出辽塞。其水东南流径高柳县故城北,旧代郡治。秦始皇二十三年虏赵王,迁以国为郡,王莽之所谓厌狄也。建武十九年,世祖封代相堪为侯国,昔牵招斩韩忠于此处。城在平城东南六七十里,于代为西北也。雁门水又东南流,屈径一故城,背山面泽,北俗谓之叱险城。雁门水又东南流,屈而东北,积而为潭,其陂斜长而不方,东北可二十余里,广十五里,蒹葭藂生焉。敦水注之,其水导源西北少咸山之南麓,东流径参合县故城南。《地理风俗记》曰:道人城北五十里有参合乡,故县也。敦水又东, 水注之,水出东阜下,西北流径故城北,俗谓之和堆城,又北合敦水,乱流东北注雁门水。故《山海经》曰:少咸之山,敦水出焉,东流注于雁门之水。郭景纯曰:水出雁门山。谓斯水也。雁门水又东北入阳门山,谓之阳门水,与神泉水合,水出苇壁北,水有灵焉,及其密云不雨,阳旱愆期,多祷请焉。水有二流,世谓之比连泉,一水东北径一故城东,世谓之石虎城,而东北流注阳门水,又东径三会亭北,又东径西伺道城北,又东,托台谷水注之。水上承神泉于苇壁北,东径阳门山甫托台谷,谓之托台水,汲引泉溪,浑涛东注,行者间十余渡,东径三会城南,又东径托台亭北,又东北径马头亭北,东北注雁门水。雁门水又东径大宁郡北,魏太和中置,有修水注之,即《山海经》所谓修水东流注于雁门水也。《地理志》有于延水而无雁门、修水之名,《山海经》有雁门之目,而无说于延河,自下亦通谓之于延水矣。水侧有桑林,故时人亦谓是水为藂桑河也。斯乃北土寡桑,至此见之,因以名焉。于延水又东径冈城南,按《史记》,蔡泽,燕人也,谢病归相,秦号冈成君。疑即泽所邑也,世名武冈城。于延水又东,左与宁川水合,水出西北,东南流径小宁县故城西,东南流注于延水。于延水又东,径小宁县故城南,《地理志》宁县也。西部都尉治,王莽之博康也。《魏土地记》曰:大宁城西二十里有小宁城,昔邑人班丘仲居水侧,卖药于宁百余年,人以为寿。后地动宅坏,仲与里中数十家皆死,民人取仲尸弃于延水中,收其药卖之,仲被裘从而诘之,此人失怖,叩头求哀。仲曰:不恨汝,故使人知我耳,去矣!后为夫余王驿使来宁,此方人谓之谪仙也。于延水又东,黑城川水注之,水有三源,出黑土城西北.奇源合注,总为一川,东南径黑土城西,又东南流径大宁县西而南入延河。延河又东径大宁县故城南。《地理志》云:广宁也,王莽曰广康矣。《魏土地记》曰:下洛城西北百三十里有大宁城。于延水又东南径茹县故城北,王莽之谷武也,世谓之如口城。《魏土地记》曰:城在鸣鸡山西十里,南通大道,西达宁川。于延水又东南径鸣鸡山西。《魏土地记》曰:下洛城东北三十里有延河东流,北有鸣鸡山。《史记》曰:赵襄子杀代王于夏屋而并其土,襄子迎其姊于代。其姊,代之夫人也,至此曰:代已亡矣,吾将何归乎?遂磨等于山而自杀。代人怜之,为立祠焉,因名其山为磨笄山。每夜有野鸡,群鸣于祠屋上,故亦谓之为鸣鸡山。《魏土地记》云:代城东南二十五里有马头山,其侧有钟乳穴,赵襄子既害代王,迎姊,姊代夫人,夫人曰:以弟慢夫,非仁也;以夫怨弟,非义也。磨笄自刺而死。使者自杀,民怜之,为立神屋于山侧,因名之为磨笄之山。未详孰是。于延水又南径且居县故城南,王莽之所谓久居也。其水东南流,注于漯水。《地理志》曰:于延水东至广宁入沽。
又东过涿鹿县北,涿水出涿鹿山,世谓之张公泉,东北流径涿鹿县故城南,王莽所谓抪陆也。黄帝与蚩尤战于涿鹿之野,留其民于涿鹿之阿,即于是也。其水又东北与阪泉合,水导源县之东泉。《魏土地记》曰:下洛城东南六十里有涿鹿城,城东一里有阪泉,泉上有黄帝祠。《晋太康地理记》曰:阪泉亦地名也。泉水东北流与蚩尤泉会,水出蚩尤城,城无东面。《魏土地记》称,涿鹿城东南六里有蚩尤城。泉水渊而不流,霖雨并则流注,阪泉乱流,东北入涿水。涿水又东径平原郡南,魏徙平原之民置此,故立侨郡,以统流杂。涿水又东北径祚亭北,而东北入漯水。亦云涿水枝分入匈奴者,谓之涿邪水。地理潜显,难以究昭,非所知也。漯水又东南,左会清夷水,亦谓之沧河也。水出长亭南,西径北城村故城北,又西北,平乡川水注之,水出平乡亭西,西北流注清夷水。清夷水又西北径阴莫亭,在居庸县南十里。清夷水又西会牧牛山水。《魏土地记》曰:沮阳城东八十里有牧牛山,下有九十九泉,即沧河之上源也。山在县东北三十里,山上有道武皇帝庙。耆旧云,山下亦有百泉竞发,有一神牛驳身,自山而降,下饮泉竭,故山得其名。今山下导九十九泉,积以成川。西南流,谷水与浮图沟水注之,水出夷舆县故城西南,王莽以为朔调亭也。其水俱西南流,注于沧水。沧水又西南,右合地裂沟,古老云,晋世地裂,分此界间成沟壑。有小水,俗谓之分界水,南流入沧河。沧河又西径居庸县故城南,魏上谷郡治。昔刘虞攻公孙瓒不克,北保此城,为瓒所擒。有粟水入焉。水出县下城西,枕水又屈径其县南,南注沧河。沧河又西,右与阳沟水合,水出县东北,西南流径居庸县故城北,西径大翮、小翮山南,高峦截云,层陵断雾,双阜共秀,竟举群峰之上,郡人王次仲,少有异志,年及弱冠,变苍颔旧文为今隶书。秦始皇时官务烦多,以次仲所易文简便,于事要奇而召之,三征而辄不至。次仲履真怀道,穷数术之美。始皇怒其不恭,令槛车送之。次仲首发于道,化为大鸟,出在车外,翻飞而去,落二翮于斯山,故其峰峦有大翮、小翮之名矣。《魏土地记》曰:沮阳城东北六十里有大翮、小翮山,山上神名大翮神,山屋东有温汤水口。其山在县西北二十里,峰举四十里,上庙则次仲庙也。右出温汤,疗治万病,泉所发之麓,俗谓之土亭山。此水炎热倍甚诸汤,下足便烂人体。疗疾者要须别引消息用之耳,不得言。大翮山东,其水东南流,左会阳沟水,乱流南注沧河。沧河又左得清夷水口。《魏土地记》曰:牧牛泉西流,与清夷水合者也。自下二水互受通称矣。清夷水又西,灵亭水注之,水出马兰西泽中,众泉泻溜归于泽,泽水所钟,以成沟读。渎水又左与马兰溪水会,水导源马兰城,城北负山势,因阿仍溪,民居所给,惟仗此水,南流出城,东南入泽水。泽水又南径灵亭北,又屈径灵亭东,次仲落鸟翮于此,故是亭有灵亭之称矣。其水又南流,注于清夷水。清夷水又西与泉沟水会,水导源川南平地,北注清夷水。清夷水又西南得桓公泉,盖齐桓公霸世,北伐山戎,过孤竹西征,束马悬车,上卑耳之西极,故水受斯名也。水源出沮阳县东,而西北流入清夷水。清夷水又西径沮阳县故城北,秦上谷郡治此,王莽改郡曰朔调,县曰沮阴。阚駰曰:涿鹿东北至上谷城六十里。《魏土地记》曰:城北有清夷水西流也。其水又屈径其城西,南流注于源水。漯水南至马陉山,谓之落马洪。又东南出山,漯水又南出山,瀑布飞梁,悬河注壑,漰湍十许丈,谓之落马洪,抑亦孟门之流也。漯水自南出山,谓之清泉河,俗亦谓之曰干水,非也。漯水又东南径良乡县之北界,历梁山南,高梁水出焉。
过广阳蓟县北,漯水又东径广阳县故城北。谢承《后汉书》曰:世祖与铫期出蓟至广阳,欲南行,即此城也,谓之小广阳。漯水又东北径蓟县故城南,《魏土地记》曰蓟城南七里有清泉河,而不径其北,盖《经》误证矣。昔周武王封尧后于蓟,今城内西北隅有蓟丘,因丘以名邑也。犹鲁之曲阜、齐之营丘矣。武王封召公之故国也,秦始皇二十三年灭燕,以为广阳郡,汉高帝以封卢绾为燕王,更名燕国,王莽改曰广有,县曰代戎。城有万载宫、光明殿,东掖门下,旧慕容俊立铜马像处。昔慕容廆有骏马,赭白有奇相,逸力至俊,光寿元年,齿四十九矣,而骏逸不亏。俊奇之,比鲍氏骢,命铸铜以图其像,亲为铭赞,镌颂其旁,像成而马死矣。大城东门内道左,有魏征北将军建成乡景侯刘靖碑。晋司隶校尉王密表靖,功加于民,宜在祀典,以元康四年九月二十日刻石建碑,扬于后叶矣。漯水又东与洗马沟水合,水上承蓟水,西注大湖。湖有二源,水俱出县西北平地,导源流结西湖,湖东西二里,南北三里,盖燕之旧池也。绿水澄澹,川亭望远,亦为游瞩之胜所也。湖水东流为洗马沟,侧城南门东注,昔铫期奋戴处也。其水又东入漯水,漯水又东径燕王陵南,陵有伏道,西北出蓟城中。景明中造浮图建刹,穷泉掘得此道,王府所禁,莫有寻者。通城西北大陵,而是二坟,基趾磐固,犹自高壮,竟不知何王陵也。漂水又东南,高梁之水注焉,水出蓟城西北平地,泉流东注,径燕王陵北,又东径蓟城北,又东南流。《魏土地记》曰:蓟东十里有高梁之水者也。其水又东南入漯水。又东至渔阳雍奴县西,入笥沟。
汉光武建武二年,封颖川太守寇恂为雍奴侯。魏遣张、乐进围雍奴,即此城矣。笥沟,潞水之别名也。《魏土地记》曰:清泉河上承桑乾河,东流与潞河合。漯水东入渔阳,所在枝分,故俗谚云:高梁无上源,清泉无下尾。盖以高梁微涓浅薄,裁足津通,凭藉涓流,方成川甽。清泉至潞,所在枝分,更为微津,散漫难寻故也。


卷十四  湿余水、沽河、鲍丘水、濡水、大辽水、小辽水、浿水 
湿余水出上谷居庸关东,关在沮阳城东南六十里居庸界,故关名矣。更始使者入上谷,耿况迎之于居庸关,即是关也。其水导源关山,南流历故关下。溪之东岸有石室三层,其户牖扇扉,悉石也,盖故关之候台矣。南则绝谷,累石为关垣,崇墉峻壁,非轻功可举,山岫层深,侧道褊狭,林鄣邃险,路才容轨,晓禽暮兽,寒鸣相和,羁官游子,聆之者莫不伤思矣。其水历山南径军都县界,又谓之军都关。《续汉书》曰:尚书卢植隐上谷军都山是也。其水南流出关,谓之下口,水流潜伏十许里也。
东流过军都县南,又东流过蓟县北,湿余水故渎东径军都县故城南,又东,重源潜发,积而为潭,谓之湿余潭。又东流,易荆水注之,其水导源西北千蓼泉,亦曰丁蓼水,东南流径郁山西,谓之易荆水。公孙瓒之败于鲍丘也,走保易荆,疑阻此水也。易荆水又东,左合虎眼泉水,出平川,东南流入易荆水。又东南与孤山之水合,水发川左,导源孤山,东南流入易荆水,谓之塔界水。又东径蓟城,又东径昌平县故城南,又谓之昌乎水。《魏土地记》曰:蓟城东北百四十里有昌平城,城西有昌平河,又东流注湿余水。湿余水又东南流,左合芹城水,水出北山,南径芹城,东南流注湿余水。湿余水又东南流径安乐故城西,更始使谒者韩鸿北徇承制,拜吴汉为安乐令,即此城也。
又北屈东南至狐奴县西,入于沽河。
昔彭宠使狐奴令王梁南助光武,起兵自是县矣。湿余水于县西南东入沽河。故《地理志》曰:湿余水自军都县东至潞南入沽是也。
沽河从塞外来,沽河出御夷镇西北九十里丹花岭下,东南流,大谷水注之,水发镇北大谷溪,西南流,径独石北界,石孤生,不因阿而自峙。又南,九源水注之,水导北川,左右翼注。八川共成一水,故有九源之称,其水南流,至独石注大谷水。大谷水又南径独石西,又南径御夷镇城西。魏太和中,置以捍北狄也。又东南,尖谷水注之,水源出镇城东北尖溪,西南流径镇城东,西南流注大谷水,乱流南注沽水。又南出峡,夹岸有二城,世谓之独固门,以其藉险凭固,易为依据,岩壁升耸,疏通若门,故得是名也。沽水又南,左合乾溪水,引北川西南径一故亭东,又西南注沽水。沽水又酒南径赤城东,赵建武年,并州刺史王霸为燕所败,退保此城。城在山阜之上,下枕深隍,溪水之名,藉以变称,故河有赤城之号矣。沽水又东南与鹊谷水合,水有二源,南即阳乐水也,出且居县。《地理志》曰:水出县东南流径大翮山、小翮山北,历女祁县故城南。《地理志》曰:东部都尉治。王莽之祁县也。世谓之横水,又谓之阳田河。又东南径一故亭,又东,左与候卤水合,水出西北山,东南流径候卤城北,城在居庸县西北二百里,故名云候卤,太和中,更名御夷镇。又东南流注阳乐水。相乐水又东南傍狼山南,山石白色特上,亭亭孤立,超出群山之表。又东南径温泉东,泉在山曲之中。又径赤城西,屈径其城南,东南入赤城河。河水又东南,右合高峰水,水出高峰戍东南,城在山上,其水西南流,又屈而东南,入沽水。沽水又西南流出山,径渔阳县故城西,而南合七度水,水出北山黄颁谷,故亦谓之黄颁水,东南流注于沽水。沽水又南,渔水注之,水出县东南平地泉流,西径渔阳县故城南。应劭曰:在渔水之阳也。考诸他说,则无闻,脉水寻川,则有自。今城在斯水之阳,有符应说,渔阳之名当属此,秦发闾左戍渔阳。即是城也。渔水又西南入沽水。沽水又南与螺山之水合,水出渔阳城南小山。《魏土地记》曰:城南五里有螺山,其水西南入沽水。沽水又南径安乐县故城东。《晋书地道记》曰:晋封刘禅为公国。俗谓之西潞水也。南过渔阳狐奴县北,西南与湿余水合,为潞河;沽水西南流径狐奴山西,又南径狐奴县故城西。渔阳太守张堪,于县开稻田,教民种殖,百姓得以殷富。童谣歌曰:桑无附枝,麦秀两歧,张君为政,乐不可支。视事八年,匈奴不敢犯塞。沽水又南,阳重沟水注之,水出狐奴山,南转径狐奴城西,王莽之所谓举符也。侧城南注,右会沽水。沽水又南,湿余水注之。沽水又南,左会鲍丘水,世所谓东潞也。沽水又南径潞县为潞河。《魏土地记》曰:城西三十里有潞河是也。
又东南至雍奴县西,为笥沟;漯水入焉,俗谓之合口也。又东,鲍丘水于县西北而东出。又东南至泉州县,与清河合,东入于海。清河者,派河尾也。沽河又东南径泉州县故城东,王莽之泉调也。沽水又东南合清河,今无水。清、淇、漳、洹、滱、易、涞、濡、沽、滹沦,同归于海。故《经》曰派河尾也。
鲍丘水从塞外来,南过渔阳县东,鲍丘水出御夷北塞中,南流径九庄岭东,俗谓之大榆河。又南径镇东南九十里西密云戍西,又南,左合道人溪水,水出北川,南流径孔山西,又历密云戍东,左合孟广水,水出下, 甚层峻,峨峨冠众山之表。其水西径孔山南,上有洞穴开明,故土俗以孔山流称。水又西南至密云戍东,西注道人水,乱流西南径密云戍城南,右会大榆河。有东密云,故是城言西矣。大榆河又东南流,白杨泉水注之,北发白杨溪望离,右注大榆河。又东南,龙刍溪水自坎注之,大榆河又东南出峡,径安州旧渔阳郡之滑盐县南,左合县之北溪水,水出县北广长堑南。太和中,掘此以防北狄。其水南流径滑盐县故城东,王莽更名匡德也,汉明帝改曰盐田,右承治,世谓之斛盐城,西北去御夷镇二百里。南注鲍丘水,又南径奚县故城东,王莽更之曰敦德也。鲍丘水又西南径犷平县故城东,王莽之所谓平犷也。又南合三城水,水出臼里山,西径三城,谓之三城水。又径香陉山,山上悉生槁本香,世故名焉。又西径石窟南,窟内宽广,行者依焉;窟内有水,渊而不流,栖薄者取给焉。又西北径伏凌山南与石门水合,水出伏凌山,山高峻,岩鄣寒深,阴崖积雪,凝冰夏结,事同《离骚》峨峨之咏,故世人因以名山也。一水西南流注之,是水有桑谷之名,盖沿出桑溪故也。又西南径犷平城东南,而右注鲍丘水。鲍丘水又东南径渔阳县故城南,渔阳郡治也。秦始皇二十二年置,王莽更名通潞,县曰得渔。鲍丘水又西南流,公孙瓒既害刘虞,乌丸思刘氏之德,迎其子和,合众十万,破瓒于是水之上,斩首一万。鲍丘水又西南历狐奴城东,又西南流注于沽河,乱流而南。
又南过潞县西,鲍丘水入潞,通得潞河之称矣。高梁水注之,水首受漯水于戾陵堰,水北有梁山,山有燕刺王旦之陵,故以戾陵名堰。水自堰枝分,东径梁山南,又东北径刘靖碑北。其词云:魏使持节、都督河北道诸军事、征北将军、建城乡侯沛国刘靖,字文恭,登梁山以观源流,相漯水以度形势,嘉武安之通渠,羡秦民之殷富。乃使帐下丁鸿,督军士千人,以嘉平二年,立遏于水,导高梁河,造戾陵遏,开车箱渠,其遏表云:高梁河水者,出自并州,潞河之别源也。长岸峻固,直截中流,积石笼以为主遏,高一丈,东西长三十丈,南北广七十余步。依北岸立水门,门广四丈,立水十丈。山水暴发,则乘遏东下,平流守常,则自门北人,灌田岁二千顷。凡所封地,百余万晦。至景元三年辛酉,诏书以民食转广,陆费不赡,遣谒者樊晨更制水门,限田千顷,刻地四千三百一十六顷,出给郡县,改定田五千九百三十顷,水流乘车箱渠,自蓟西北径昌平,东尽渔阳潞县,凡所润含,四五百里,所灌田万有余顷。高下孔齐,原隰底平,疏之斯溉,决之斯散,导渠口以为涛门,洒滮池以为甘泽,施加于当时,敷被于后世。晋元康四年,君少子骁骑将军、平乡侯宏,受命使持节监幽州诸军事,领护乌丸校尉宁朔将军,遏立积三十六载,至五年夏六月,洪水暴出,毁损四分之三,剩北岸七十余丈,上渠车箱,所在漫溢,追惟前立遏之勋,亲临山川,指授规略,命司马关内侯逢恽,内外将士二千人,起长岸,立石渠,修主遏,治水门,门广四丈,立水五尺,兴复载利通塞之宜,准遵旧制,凡用功四万有余焉。诸部王侯,不召而自至,繈负而事者,盖数千人。《诗》载经始勿亟,《易》称民忘其劳,斯之谓乎。于是二府文武之士,感秦国思郑渠之绩,魏人置豹祀之义,乃遇慕仁政,追述成功。元康五年十月十一日,刊石立表,以纪勋烈,并记遏制度,永为后式焉。事见其碑辞。又东南流,径蓟县北,又东至潞县,注于鲍丘水。又南径潞县故城西,王莽之通潞亭也。汉光武遣吴汉、耿弇等破铜马五幡于潞东,谓是县也。屈而东南流,径潞城南。世祖拜彭宠为渔阳太守,治此。宠叛,光武遣游击将军邓隆伐之,军于是水之南,光武策其必败,果为宠所破,遗壁故垒存焉。鲍丘水又东南入夏泽,泽南纤曲淆十余里,北佩谦泽,眇望无垠也。
又南至雍奴县北,屈东入于海。
鲍丘水自雍奴县故城西北,旧分笥沟水东出,今笥沟水断,众川东注,混同一渎,东径其县北,又东与泃河合,水出右北平无终县西山白杨谷,西北流径平谷县,屈西南流,独乐水入焉。水出北抱犊固南,径平谷县故城东。后汉建武元年,光武遣十二将追大枪五幡及平谷,大破之于是县也。其水南流入于泃。泃水又左合盘山水,水出山上,其山峻险,人迹罕交,去山三十许里,望山上水,可高二十余里,素湍皓然,颓波历溪,沿流而下,自西北转注于泃水。泃水又东南径平谷县故城,东南与洳河会,水出北山,山在傂奚县故城东南,东南流径博陆故城北,又屈径其城东。世谓之平陆城,非也。汉武帝玺书,封大司马霍光为侯国。文颖曰:博大陆平,取其嘉名而无其县,食邑北海、河东。薛瓒曰:按渔阳有博陆城,谓此也。今城在且居山之阳,处平陆之上,匝带川流,面据四水,文氏所谓无县目,嘉美名也。洳水又东南流径平谷县故城西,而东南流注于泃河。沟河又南径紻城东,而南合五百沟水,水出七山北,东径平谷县之紻城南,东入于泃河。泃河又东南径临泃城北,屈而历其城东,侧城南出。《竹书纪年》,梁惠成王十六年,齐师及燕战于泃水,齐师遁,即是水也。泃水又南入鲍丘水。鲍丘水又东合泉州渠口,故渎上承滹沱水于泉州县,故以泉州为名。北径泉州县东,又北径雍奴县东,西去雍奴故城百二十里,自滹沱北入其下,历水泽百八十里,入鲍丘河,谓之泉州口。陈寿《魏志》曰:曹太祖以蹋顿扰边,将征之,从泃口凿渠径雍奴、泉州以通河海者也。今无水。鲍丘水又东,庚水注之,水出右北平徐无县北塞中,而南流历徐无山得黑牛谷水,又得沙谷水,并西出山,东流注庚水。昔田子泰避难居之,众至五千家。《开山图》曰:山出不灰之木,生火之石。按注云:其木色黑似炭而无叶,有石赤色如丹,以二石相磨,则火发,以然无灰之木,可以终身,今则无之。其水又径徐无县故城东,王莽之北顺亭也。《魏土地记》曰:右北平城东北百一十里有徐无城。其水又西南与周卢溪水合,水出徐无山,东南流注庚水。庚水又西南流,泃水注之,水出右北平俊靡县,王莽之俊麻也。东南流,世谓之车泃水。又东南流与温泉水合,水出北山温溪,即温源也。养疾者不能澡其炎漂,以其过的故也。《魏土地记》曰:徐无城东有温汤。即此也。其水南流百步,便伏流入于地下,水盛则通注。泃水又东南径石门峡,山高崭绝,壁立洞开,俗谓之石门口。汉中平四年,渔阳张纯反,杀右北平太守刘政、辽东大守阳纮。中平五年,诏中郎将孟益率公孙瓒讨纯,战于石门,大破之。泃水又东南流,谓之北黄水,又屈而为南黄水,又西南径无终山,即帛仲理所合神丹处也。又于是山作金五千斤以救百姓,山有阳翁伯玉田,在县西北有阳公坛社,即阳公之故居也。《搜神记》曰:雍伯,洛阳人,至性笃孝,父母终殁,葬之于无终山,山高八十里,而上无水,雍伯置饮焉,有人就饮,与石一斗,令种之,玉生其田。北平徐氏有女,雍伯求之,要以白壁一双,媒者致命,伯至玉田求得五双,徐氏妻之,遂即家焉。《阳氏谱叙》言,翁伯是周景王之孙,食采阳樊,春秋之末,爱宅无终,因阳樊而易氏焉。爱人博施,天祚玉田,其碑文云,居于县北六十里翁同之山,后潞徙于西山之下,阳公又迁居焉,而受玉田之赐,情不好宝,玉田自去。今犹谓之为玉田阳。于宝曰:于种石处,四角作大石柱,各一丈,中央一顷之地,名曰玉田,至今相传云。玉田之揭,起于此矣,而今不知所在,同子《谱叙》自去文矣。蓝水注之,水出北山,东流屈而南,径无终县故城东。故城,无终子国也。《春秋》襄公四年,无终子嘉父使孟乐如晋,因魏绛纳虎豹之皮,请和诸戎是也,故燕地矣。秦始皇二十二年灭燕,置右北平郡,治此,王莽之所谓北顺也。汉世李广为郡,出遇伏石,谓虎也,射之饮羽,即此处矣。《魏土地记》曰:右北平城西北百三十里有无终城。其水又南入水, 水又西南入于庚水。《地理志》曰: 水出俊靡县南,至无终东入庚水。庚水,世亦谓之为水也。南径燕山下,悬岩之侧有石鼓,去地百余丈,望若数百石国,有石梁贯之,鼓之东南,有石援桴,状同击势。耆旧言,燕山石鼓,鸣则土有兵。庚水又南径北平城西,而南入鲍丘水,谓之口。鲍丘水又东径右北平郡故城南。《魏土地记》曰:蓟城东北三百里有右北平城。鲍丘水又东,巨梁水注之,水出土垠县北陈宫山,西南流径观鸡山,谓之观鸡水。水东有观鸡寺,寺内起大堂,甚高广,可容千僧,下悉结石为之,上加涂塈,基内疏通,枝经脉散,基侧室外,四出爨火,炎势内流,一堂尽温,盖以此土寒严,霜气肃猛,出家沙门,率皆贫薄,施主虑阙道业,故崇斯构,是以志道者多栖托焉。其水又西南流,右合区落水,水出县北山,东南流入巨梁水。巨梁水又南径土垠县故城西,左会寒渡水,水出县东北,西南流至县,右注梁河。梁河又南,涧于水注之,水出东北山,西南流径土垠县故城东,西南流入巨梁水。巨梁水又东南,右合五里水,水发北平城东北五里山,故世以五里名沟,一名田继泉,西流南屈,径北平城东,东南流注巨梁河,乱流入于鲍丘水。自是水之南,南极滹沱,西至泉州雍奴,东极于海,谓之雍奴薮。其泽野有九十九淀,枝流条分,往往径通,非惟梁河、鲍丘归海者也。
濡水从塞外来,东南过辽西令支县北,濡水出御夷镇东南,其水二源双引,夹山西北流,出山,合成一川。又西北径御夷故城东,镇北百四十里,北流,左则连渊水注之,水出故城东,西北流径故城甫,又西北径绿水池南,池水渊而不流。其水又西屈而北流,又东径故城北,连结两沼,谓之连渊浦。又东北注难河,难河右则汗水人焉,水出东坞南,西北流径沙野南,北人名之曰沙野,镇东北二百三十里,西北入难河,濡、难声相近,狄俗语讹耳。濡水又北径沙野西,又北径箕安山东,屈而东北流,径沙野北,东北流径林山北,水北有池,潭而不流。濡水又东北流径孤山南,东北流,吕泉水注之,水出吕泉坞西,东南流,屈而东,径坞南东北流,三泉水注之,其源三泉雁次,合为一水,镇东北四百里,东南注吕泉水。吕泉水又东径孤山北,又东北,逆流水注之,水出东南,导泉西流,右屈而东北注,木林山水会之,水出山南,东注逆水,乱流东北注濡河。濡河又东,盘泉入焉,水自西北,东南流,注濡河。濡河又东南,水流回曲,谓之曲河。镇东北三百里。又东出峡入安州界,东南流径渔阳白檀县故城。《地理志》曰:濡水出县北蛮中,汉景帝诏李广曰:将军其帅师东辕,弧节白檀者也。又东南流,右与要水合,水出塞外,三川并导,谓之大要水也。东南流径要阳县故城东,本都尉治,王莽更之曰要术矣。要水又东南流,径白檀县而东南流,人于濡。濡水又东南,索头水注之,水北出索头川,南流径广阳侨郡西,魏分右北平,置今安州治。又南流,注于濡。濡水又东南流,武列水入焉,其水三川派合,西源右力溪水,亦曰西藏水,东南流出溪,与幡泉水合。泉发州东十五里,东流九十里,东注西藏水。西藏水又西南流,东藏水注之,水出东溪,一口东藏水,西南流出谷,与中藏水合,水导中溪,南流出谷,南注东藏水。故目其川曰三藏川,水曰三藏水。东藏水又南,右入西藏水,乱流右会龙泉水,水出东山下,渊深不测,其水西南流,注于三藏水。三藏水又东南流,与龙刍水合,西出于龙刍之溪,东流入三藏水。又东南流径武列溪,谓之武列水。东南历石挺下,挺在层峦之上,孤石云举,临崖危峻,可高百余仞,牧守所经,命选练之士,弯张弧矢,无能届其崇标者。其水东合流入濡。濡水又东南,五渡水注之,水北出安乐县丁原山,南流径其县故城西,本三会城也。其水南入五渡塘,于其川也,流纤曲,溯涉者频济,故川塘取名矣。又南流注于濡。濡水又与高石水合,水东出安乐县东山,西流历三会城南,西入五渡卅,下注濡水。濡水又东南径卢龙塞。塞道,自无终县东出渡濡水,向林兰陉,东至清陉。卢龙之险,峻坂索折,故有九缘之名矣。燕景昭元玺二年,遣将军步浑治卢龙塞道,焚山刊石,令通方轨,刻石岭上,以记事功,其铭尚存。而庾杲之注《扬都赋》言,卢龙山在平冈城北,殊为盂浪,远失事实。余按卢龙东越清陉,至凡城二百许里,自凡城东北出,趣平冈故城可百八十里,向黄龙则五百里,故陈寿《魏志》,田畴引军出卢龙塞,堑山堙谷,五百余里径白檀,历平冈,登白狼,望柳城。平冈在卢龙东北远矣。而仲初言在南,非也。濡水又东南径卢龙故城东,汉建安十二年,魏武征蹋顿所筑也。濡水又南,黄洛水注之,水北出卢龙山,南流入于濡。濡水又东南,洛水合焉,水出卢龙塞西,南流注濡水。濡水又屈而流,左得去润水,又合敖水,二水并自卢龙西注濡水。濡水又东南流径令支县故城东,王莽之令氏亭也。秦始皇二十二年分燕置,辽西郡令支隶焉。《魏土地记》曰:肥如城西十里有濡水,南流径孤竹城西,右合玄水。世谓之小濡水,非也。水出肥如县东北玄溪,西南流径其县东,东屈南转,西回径肥如县故城甫,俗又谓之肥如水。故城,肥子国。应劭曰:晋灭肥,肥子奔燕,燕封于此,故曰肥如也。汉高帝六年,封蔡寅为侯国。西南流,右会卢水,水出县东北沮溪,南流,谓之大沮水。又南,左合阳乐水,水出东北阳乐县溪。《地理风俗记》曰:阳乐,故燕地,辽西郡治,秦始皇二十二年置。《魏土地记》曰:海阳城西南有阳乐城。其水又西南入于沮水,谓之阳口。沮水又西南,小沮水注之,水发冷溪,世谓之冷池。又南得温泉水口,水出东北温溪,自溪西南流,入于小沮水。小沮水又南流与大沮水合,而为卢水也。桑钦《说卢子之书》言:晋既灭肥,迁其族于卢水。卢水有二渠,号小沮、大沮,合而入于玄水。又南与温水合,水出肥如城北,西流注于玄水。《地理志》曰:卢水南入玄。玄水又西南径孤竹城北,西入濡水。故《地理志》曰:玄水东入濡,盖自东面注也。《地理志》曰:令支有孤竹城,故孤竹国也。《史记》曰:孤竹君之二于伯夷、叔齐,让国于此,而饿死于首阳。汉灵帝时,辽西太守廉翻梦人谓己曰:余,孤竹君之子,伯夷之弟,辽海漂吾棺椁,闻君仁善,愿见藏覆。明日视之,水上有浮棺,吏嗤笑者皆无疾而死,于是改葬之。《晋书地道志》曰:辽西人见辽水有浮棺,欲破之,语曰:我孤竹君也,汝破我何为?因为立祠焉。祠在山上,城在山侧。肥如县南十二里,水之会也。
又东南过海阳县西,南入于海。
濡水自孤竹城东南径西乡北,瓠沟水注之,水出城东南,东流注濡水。
濡水又径故城南,分为二水,北水枝出,世谓之小濡水也。东径乐安亭北,东南入海。濡水东南流,径乐安亭南,东与新河故渎合,渎自雍奴县承鲍丘水东出,谓之盐关口。魏太祖征蹋顿,与泃口俱导也,世谓之新河矣。陈寿《魏志》云:以通海也。新河又东北绝庚水,又东北出,径右北平,绝泃渠之水,又东北径昌城县故城北,王莽之淑武也。新河又东分为二水,枝渎东南入海。新河自枝渠东出合封大水,谓之交流口。水出新安平县,西南流径新安平县故城西,《地理志》,辽西之属县也。又东南流,龙鲜水注之,水出县西北,世谓之马头水,二源俱导,南合一川,东流注封大水。《地理志》曰:龙鲜水东入封大水者也。乱流南会新河,南注于海。《地理志》曰:封大水于海阳县南入海。新河又东出海阳县与缓虚水会,水出新安平县东北,世谓之大笼川,东南流径令支城西,西南流与新河合,南流注于海。《地理志》曰:缓虚水与封大水,皆南入海。新河又东与素河会,谓之白水口,水出令支县之蓝山,南合新河,又东南入海。新河又东至九口,枝分南注海。新河又东径海阳县故城南,汉高祖六年,封摇毋余为侯国。《魏土地记》曰:令支城南六十里有海阳城者也。新河又东与清水会,水出海阳县,东南流径海阳城东,又南合新河,又南流十许里,西人九挝注海。新河东绝清水,又东,木究水出焉,南入海。新河又东,左迤为北阳孤淀,淀水右绝新河,南注海。新河又东会于濡。濡水又东南至絫县碣石山,文颖曰:碣石在辽西絫县,王莽之选武也。絫县并属临渝,王莽更临渝为冯德。《地理志》曰:大碣石山在右北平骊成县西南,王莽改曰揭石也。汉武帝亦尝登之以望巨海,而勒其石于此。今枕海有石如甬道数十里,当山顶有大石如柱形,往往而见,立于巨海之中,潮水大至则隐,及潮波退,不动不没,不知深浅,世名之天桥柱也。状若人造,要亦非人力所就,韦昭亦指此以为碣石也。《三齐略记》曰:始皇于海中作石桥,海神为之竖柱。始皇求与相见,神曰:我形丑,莫图我形,当与帝相见。乃入海四十里,见海神,左右莫动手,工人潜以脚画其状。神怒曰:帝负约,速去。始皇转马还,前脚犹立,后脚随崩,仅得登岸,画者溺死于海,众山之石皆倾注,今犹发发东趣,疑即是也。濡水于此南入海,而不径海阳县西也。盖《经》误证耳。又按《管子》,齐桓公二十年,征孤竹,未至卑耳之溪十里,闟然止,瞠然视,援弓将射,引而未发,谓左右曰:见前乎?左右对曰:不见。公曰:寡人见长尺而人物具焉,冠,右袪衣,走马前,岂有人若此乎?管仲对曰:臣闻岂山之神有偷儿,长尺人物具,霸王之君兴,则岂山之神见。且走马前,走,导也;袪衣,示前有水;右袪衣,示从右方涉也。至卑耳之溪,有赞水者,从左方涉,其深及冠,右方涉,其深至膝,已涉大济,桓公拜曰:仲父之圣至此,寡人之抵罪也久矣。今自孤竹南出,则巨海矣,而沧海之中,山望多矣,然卑耳之川若赞溪者,亦不知所在也。昔在汉世,海水波襄,吞食地广,当同碣石,苞沦洪波也。大辽水出塞外卫白平山,东南入塞,过辽东襄平县西,辽水亦言出砥石山,自塞外东流,直辽东之望平县西,王莽之长说也。屈而西南流径襄平县故城西。秦始皇二十二年灭燕,置辽东郡,治此。汉高帝八年,封纪通为侯国,土莽之昌平也,故平州治。又南径辽队县故城西,王莽更名之曰顺睦也。公孙渊遣将军毕衍拒司马懿于辽队,即是处也。
又东南过房县西,《地理志》:房,故辽东之属县也。辽水右会白狼水,水出右北平白狼县东南,北流西北屈,径广成县故城南,王莽之平虏也,俗谓之广都城。又西北,石城川水注之,水出西南石城山,东流径石城县故城南。《地理志》,右北平有石城县。北屈径白鹿山西,即白狼山也。《魏书国志》曰:辽西单于蹋顿尤强,为袁氏所厚,故袁尚归之。数入为害,公出卢龙,堑山堙谷五百余里,未至柳城二百里,尚与蹋顿将数万骑逆战,公登白狼山望柳城,卒与虏遇,乘其不整,纵兵击之,虏众大崩,斩蹋顿,胡、汉降者二十万口。《英雄记》曰:曹澡于是击马鞍干马上作十片。即于此也。《博物志》曰:魏武于马上逢狮子,使格之,杀伤甚众,王乃自率常从健儿数百人击之,狮子吼呼奋越,左右咸惊,王忽见一物从林中出如狸,超上王车轭上,狮子将至,此兽便跳上狮子头上,狮子即伏不敢起。于是遂杀之,得狮子而还。未至洛阳四十里,洛中鸡狗皆无鸣吠者也。其水又东北入广成县,东注白狼水。白狼水北径白狼县故城东,王莽更名伏狄。白狼水又东,方城川水注之,水发源西南山下,东流北屈,径一故城西,世谓之雀目城,东屈径方城北,东入白狼水。白狼水又东北径昌黎县故城西,《地理志》曰:交黎也,东部都尉治,王莽之禽虏也。应劭曰:今昌黎也。高平川水注之,水出西北平川,东流径倭城北,盖倭地人徒之。又东南径乳楼城北,盖径戎乡,邑兼夷称也。又东南注白狼水。白狼水又东北,自鲁水注之,水导西北远山,东南注白狼水。白狼水又东北径龙山西,燕慕容皝以柳城之北、龙山之南,福地也,使阳裕筑龙城,改柳城为龙城县十二年,黑龙、白龙见干龙山,皝亲观,龙去二百步,祭以大牢二,龙交首嬉翔,解角而去。皝悦,大赦,号新宫曰和龙宫,立龙翔祠于山上。白狼水又北径黄龙城东,《十二州志》曰:辽东属国都尉治昌辽道有黄龙亭者也,魏营州刺史治。《魏土地记》曰:黄龙城西南有白狼河,东北流附城东北下,即是也。又东北,滥真水出西北塞外,东南历重山,东南入白狼水。白狼水又东北出,东流分为二水,右水,疑即渝水也。《地理志》曰:渝水首受白狼水,西南循山,径一故城西,世以为河连城,疑是临渝县之故城,王莽曰冯德者矣。渝水南流东屈,与一水会,世名之曰伦水,盖戎方之变名耳。疑即《地理志》所谓侯水北入渝者也。《十二州志》曰:侯水南入渝。《地理志》盖言自北而南也。又西南流注于渝。渝水又东南径一故城东,俗曰女罗城,又南径营丘城西。营丘在齐而名之于辽、燕之间者,盖燕、齐辽迥,侨分所在。其水东南入海。《地理志》曰:渝水自塞外南入海。一水东北出塞为白狼水,又东南流至房县注于辽。《魏土地记》曰:白狼水下人辽也。又东过安市县西,南入于海。
《十三州志》曰:大辽水自塞外西南至安市入于海。又玄菟高句丽县有辽山,小辽水所出,县,故高句丽,胡之国也。汉武帝元封二年,平右渠,置玄菟郡于此,王莽之下句丽。水出辽山,西南流径辽阳县与大梁水会,水出北塞外,西南流至辽阳人小辽水。故《地理志》曰:大梁水西南至辽阳入辽。《郡国志》曰:县,故属辽东,后入玄菟。其水西南流,故谓之为梁水也。小辽水又西南径襄平县为淡渊,晋永嘉三年涸。小辽水又径辽队县人大辽水。司马宣王之平辽东也,斩公孙渊于斯水之上者也。
西南至辽队县,入于大辽水也。
浿水出乐浪镂方县,东南过临浿县,东入于海。
许慎云:浿水出镂方,东入海。一曰出浿水县。《十三州志》曰:浿水县在乐浪东北,镂方县在郡东。盖出其县南径镂方也。昔燕人卫满自浿水西至朝鲜。朝鲜,故箕子国也。箕子教民以义,田织信厚,约以八法,而下知禁,遂成礼俗。战国时,满乃王之,都王险城,地方数千里,至其孙右渠。汉武帝元封二年,遣楼船将军杨朴、左将军荀彘讨右渠,破渠于浿水,遂灭之。若浿水东流,无渡浿之理,其地今高句丽之国治,余访蕃使,言城在浿水之阳。其水西流径故乐浪朝鲜县,即乐浪郡治。汉武帝置。而西北流。故《地理志》曰:浿水西至增地县入海。又汉兴,以朝鲜为远,循辽东故塞至浿水为界。考之今古,于事差谬,盖《经》误证也。


卷十五  洛水、伊水、瀍水、涧水 
洛水出京兆上洛县遭举山,《地理志》曰:洛出冢岭山。《山海经》曰:出上洛西山。又曰:讙举之山,洛水出焉。东与丹水合,水出西北竹山东,南流注于洛。洛水又东,尸水注之,水北发尸山,南流入洛。洛水又东得乳水,水北出良余山,南流注于洛。洛水又东会于龙余之水,水出蛊尾之山,东流入洛。洛水又东至阳虚山,合玄扈之水。《山海经》曰:洛水东北流,注于玄扈之水是也。又曰自鹿蹄之山以至玄扈之山,凡九山,玄扈亦山名也,而通与讙举,为九山之次焉。故《山海经》曰:此二山者洛间也。是知玄扈之水,出于玄扈之山,盖山水兼受其目矣。其水径于阳虚之下。《山海经》又曰:阳虚之山,临于玄扈之水,是为洛汭也。《河图玉版》曰:仓颌为帝,南巡,登阳虚之山,临于玄扈洛汭之水。灵龟负书,丹甲青文以授之,即于此水也。洛水又东历清他山,东合武里水,水南出武里山,东北流注于洛。洛水又东,门水出焉。《尔雅》所谓洛别为波也。洛水又东,要水入焉,水南出三要山,东北径拒阳城西,而东北流入于洛。洛水又东与获水合,水南出获舆山,俗谓之备水也。东北径获舆川,世名之为舆川,东北流,注于洛。洛水又东径熊耳山北,《禹贡》所谓导洛自熊耳。《博物志》曰:洛出熊耳,盖开其源者是也。东北过卢氏县南,洛水径隖渠关北,隖渠水南出隖渠山,即荀渠山也。其水一源两分,川流半解,一水西北流,屈而东北,入于洛。《山海经》曰:熊耳之山,浮豪之水出焉,西北流注于洛。疑即是水也。荀渠,盖熊耳之殊称,若太行之归山也。故《他说》曰:熊耳之山,地门也,洛水出其间。是亦总名矣。其一水东北径隖渠城西,故关城也。其水东北流,注于洛。洛水又东径卢氏县故城南。《竹书纪年》,晋出公十九年,晋韩龙取卢氏城。王莽之昌富也。有卢氏川水注之,水北出卢氏山,东南流径卢氏城东,东南流注于洛。洛水又东,翼合三川,并出县之南山,东北注洛。《开山图》曰:卢氏山宜五谷,可避水灾,亦通谓之石城山。山在宜阳山西南,千名之山,咸处其内,陵阜原隰,易以度身者也。又有葛蔓谷水,自南山流注洛水。洛水又东径高门城南,即《宋书》所谓后军外兵庞季明入卢氏进达高门木城者也。洛水东与高门水合,水出北山,东南流合洛水枝津,水上承洛水,东北流径石勒城北,又东径高门城北,东入高门水,乱流南注洛。洛水又东,松阳溪水注之,水出松阳山,北流注于洛。洛水又东径黄亭南,又东合黄亭溪水,水出鹈鹕山。山有二峰,峻极于天,高崖云举,亢石无阶,猿徒丧其捷巧,鼯族谢其轻工,及其长霄昌岭,层霞冠峰,方乃就辨优劣耳,故有大,小鹈鹕之名矣。溪水东南流历亭下,谓之黄亭溪水,又东南入于洛水。洛水又东得荀公溪口,水出南山荀公涧,即庞季明所入荀公谷者也。其水历谷东北流,注于洛水。洛水又东径檀山南,其山四绝孤峙,山上有坞聚,俗谓之檀山坞。义熙中,刘公西入长安,舟师所届,次于洛阳,命参军戴延之与府舍人虞道元即舟溯流,穷览洛川,欲知水军可至之处。延之届此而返,竟不达其源也。洛水又东,库谷水注之,水自宜阳山南。三川并发,合为一溪,东北流注于洛。洛水又东得鹈鹕水口,水北发鹈鹕涧,东南流入千洛。洛水又径仆谷亭北,左合北水,水出北山,东南流注于洛。洛水又东,侯谷水出南山,北流入于洛。洛水又东径龙骧城北,龙骧将军王镇恶,从刘公西入长安,陆行所由,故城得其名。洛水又东,左合宜阳北山水,水自北溪南流注洛。洛水又东,广由涧水注之,水出南山由溪,北流径龙骧城东,而北流入于洛。洛水又东,右得直谷水,水出南山,北径屯城。西北流注于洛水也。
又东北过蠡城邑之南,城西有坞水,出北四里山上,原高二十五丈,故邑池县治,南对金门坞,水南五里,旧宜阳县治也。洛水右会金门溪水,水南出金门山,北径金门坞,西北流入于洛。洛水又东合款水,其水二源并发,两川径引,谓之大款水也,合而东南入于洛。洛水又东,黍良谷水入焉,水南出金门山。《开山图》曰:山多重固在韩。建武二年,强弩大将军陈俊转击金门、白马,皆破之,即此也,而东北流注于洛。洛水又东,左合北溪,南流入于洛也。
又东过阳市邑南,又东北过于父邑之南,大阴谷水南出太阴溪,北流注于洛。洛水又东合白马溪水,水出宜阳山,涧有大石,厥状似马,故溪涧以物色受名也。溪水东北流注于洛。洛水又东,有昌涧水注之,水出西北宜阳山,而东南流,径宜阳故郡南,旧阳市邑也,故洛阳都典农治,此后改为郡。其水又南注于洛。洛水又东径一合坞南,城在川北原上,高二十丈,南、北、东三箱,天险峭绝,惟筑西面即为固,一合之名,起于是矣。刘曜之将攻河南也,晋将军魏该奔于此,故于父邑也。洛水又东合杜阳涧水,水出西北杜阳溪,东南径一合坞,东与槃谷水合,乱流东南入洛。洛水又东,渠谷水出宜阳县南女几山,东北流径云中坞,左上迢遰层峻,流烟半垂,缨带山阜,故坞受其名。渠谷水又东北入洛水。臧荣绪《晋书》称,孙登尝经宜阳山,作炭人见之与语,登不应,作炭者觉其情神非常,咸共传说。太祖闻之,使阮籍往观,与语,亦不应,籍因大啸,登笑曰:复作向声,又为啸。求与俱出,登不肯,籍因别去。登上峰行且啸,如萧韶笙簧之音,声振山谷。籍怪而问作炭人,作炭人曰:故是向人声。籍更求之,不知所止,推问久之,乃知姓名。余按孙绰之叙《高士传》,言在苏门山,又别作《登传》。孙盛《魏春秋》亦言在苏门山,又不列姓名。阮嗣宗感之,著《大人先生论》,言吾不知其人,既神游自得,不与物交。阮氏尚不能动其英操,复不识何人而能得其姓名。
又东北过宜阳县南,洛水之北有熊耳山,双峦竞举,状同熊耳,此自别山,不与《禹贡》导洛自熊耳同也。昔汉光武破赤眉樊崇,积甲仗与熊耳平,即是山也。山际有池,池水东南流,水侧有一池,世谓之渑池矣。又东南径宜阳县故城西,谓之西度水,又东南流入于洛。洛水又东径宜阳县故城南。秦武王以甘茂为左丞相,曰:寡人欲通三川、窥周室,死不朽矣。茂请约魏以攻韩,斩首六万,遂拔宜阳城,故韩地也,后乃县之。汉哀帝封息夫躬为侯国,城之西门,赤眉樊崇与盆子及大将等,奉玺绶剑壁处。世祖不即见,明日,陈兵于洛水见盆子等,谓盆子丞相徐宣曰:不悔乎?宣曰:不悔。上叹曰:卿庸中皦皦,铁中铮铮也。洛水又东与厌染之水合,水出县北傅山大陂,山无草木,其水自陂北流,屈而东南注,世谓之五延水。又东南流径宜阳县故城东,东南流注于洛。洛水又东南,黄中涧水出北阜,二源奇发,总成一川,东流注于洛。洛水又东,禄泉水注之,其水北出近溪。洛水又东,共水入焉,水北出长石之山,山无草木,其西有谷焉,厥名井谷,共水出焉。南流得尹溪口,水出西北尹谷,东南往之,共水又西南与左涧水会,水东出近川,西流注于共水。共水又南与李谷水合,水出西北李溪,东南注蓁水。蓁水发源蓁谷,西南流与李谷水合,而西南流入共水。共水,世谓之石头泉,而南流注于洛。洛水又东,黑涧水南出陆浑西山,历于黑涧,西北入洛。洛水又东,临亭川水注之,水出西北近溪,东南与长涧水会,水出北山南人临亭水,又东南历九曲西,而南入洛水也。
又东北出散关南,洛水东径九曲南,其地十里,有坂九曲。《穆天子传》所谓天子西征,升于九阿,此是也。洛水又东与豪水会,水出新安县密山,南流历九曲东,而南流入于洛。洛水之侧有石墨山,山石尽黑,可以书疏,故以石墨名山矣。洛水又东,枝渎左出焉。东出关,绝惠水,又径清女冢南,冢在北山上。《耆旧传》云:斯女清贞秀古,迹表来今矣。枝渎又东径周山,上有周灵王冢。《皇览》曰:周灵王葬于河南城西南周山上,盖以王生而神,效谥曰灵。其冢,人祠之不绝。又东北径柏亭南,《皇览》曰周山在柏亭西北,谓斯亭也。又东北径三王陵东北出,三王,或言周景玉、悼王、定王也。魏司徒公崔浩注《西征赋》云:定当为敬,子朝作难,西周政弱人荒,悼、敬二王,与景王俱葬于此,故世以三王名陵。《帝王世纪》曰:景王葬于翟泉,今洛阳太仓中大冢是也。而复传言在此,所未详矣。又悼、敬二王,稽诸史传,复无葬处。今陵东有石碑,录赧王以上世王名号,考之碑记,周墓明矣。枝渎东北历制乡,径河南县王城西,历郊鄏陌。杜预《释地》曰:县西有郏鄏陌,谓此也。枝渎又北入谷,盖经始周启,渎久废不修矣。洛水自枝渎又东出关,惠水右注之,世谓之八关水。戴延之《西征记》谓之八关泽,即《经》所谓散关,鄣自南山,横洛水,北属于河,皆关塞也,即杨仆家僮所筑矣。惠水出白石山之阳。东南流与瞻水合,水东出娄涿之山,而南流入惠水。惠水又东南,谢水北出瞻诸之山;东南流,又有交触之水,北出廆山,南流,俱合惠水。惠水又南流径关城北,二十里者也。其城西阻塞垣,东枕惠水。灵帝中平元年,以河南尹何进为大将军,率五营士屯都亭,置函谷、广城、伊阙、大谷、轘辕、旋门、小平津,盂津等八关,都尉官治此,函谷为之首,在八关之限,故世人总其统目,有八关之名矣。其水又南流入于洛水。《山海经》曰:白石之山惠水出其阳,而南流注于洛。谓是水也。洛水又与虢水会,水出扶猪之山,北流注于洛水。之南,则鹿蹄之山也,世谓之非山。其山阴则峻绝百仞,阳则原阜隆平,甘水发于东麓,北流注于洛水也。
又东北过河南县南,《周书》称周公将致政,乃作大邑成周于中土,南系于洛水,北因于郏山,以为天下之大凑。《孝经援神契》曰:八方之广,周洛为中,谓之洛邑。《竹书纪年》,晋定公二十年,洛绝于周。魏襄王九年,洛入成周,山水大出。南有甘洛城,《郡国志》所谓甘城也。《地记》曰:洛水东北过五零陪尾,北与涧、瀍合,是二水,东入千金渠,故渎存焉。又东过洛阳县南,伊水从西来往之。
洛阳,周公所营洛邑也,故《洛浩》曰:我卜瀍水东,亦惟洛食。其城方七百二十丈,南系于洛水,北因于郏山,以为天下之凑。方六百里,因西八百里,为千里。《春秋》昭公三十二年,晋合诸侯大夫戍成周之城,故亦曰成周也。司马迁《自序》云:太史公留滞周南。挚仲治曰:古之周南,今之洛阳,汉高祖始欲都之,感娄敬之言,不日而驾行矣。属光武中兴,宸居洛邑,逮于魏、晋,咸两宅焉。故《魏略》曰:汉火行忌水,故去其水而加佳,魏为土德,土水之牡也,水得上而流,土得水而柔,除佳加水。《长沙耆旧传》云:祝良,字召卿,为洛阳令。岁时亢旱,夭子祈雨不得,良乃曝身阶庭,告诚引罪,自晨至中,紫云水起,甘雨登降。人为歌曰:天久不雨,烝人失所,天王自出,祝令特苦,精符感应,滂沱下雨。则县司及河南尹治,司隶,周官也,汉武帝使领徒隶,董督京畿后,因名司州焉。《地记》曰:洛水东入于中提山间,东流会于伊是也。昔黄帝之时,天大雾三日,帝游洛水之上,见大鱼,杀五牲以蘸之,天乃甚雨,七日七夜鱼流,始得图书,今《河图视萌篇》是也。昔王子晋好吹凤笙,招延道士,与浮丘同游伊洛之浦,含始又受玉鸡之瑞于此水,亦洛神宓妃之所在也。洛水又东,合水南出半石之山,北径合水坞,而东北流注于公路涧。但世俗音讹,号之曰光禄涧,非也。上有袁木固,四周绝涧,迢递百仞,广四五里,有一水,渊而不流,故溪涧即其名也。合水北与刘水合,水出半石东山,西北流径刘聚,三面临涧,在猴氏西南,周畿内刘子国,故谓之刘涧。其水西北流注于合水,合水又北流注于洛水也。
又东过偃师县南,洛水东径计素渚,中朝时,百国贡计所顿,故渚得其名。又直偃师故县南,与猴氏分水。又东,休水自南注之,其水导源少室山,西流径穴山南,而北与少室山水合,水出少室北溪,西南流注休水。休水又左会南溪水,水发大穴南山,北流入休水。休水又西南北屈,潜流地下,其故渎北屈出峡,谓之大穴口,北历覆釜堆东,盖以物象受名矣。又东届零星坞,水流潜通,重源又发,侧缑氏原,《开山图》谓之缑氏山也。亦云仙者升焉,言王子晋控鹄斯阜,灵王望而不得近,举手谢而去,其家得遗展,俗亦谓之为抚父堆,堆上有子晋祠。或言在九山非此,世代已远,莫能辨之。刘向《列仙传》云:世有萧管之声焉。休水又径延寿城南,缑氏县治,故滑费,《春秋》滑国所都也。王莽更名中亭,即缑氏城也。城有仙人祠,谓之仙人观。休水又西转北屈,径其城西。水之西南有司空密陵元侯郑袤庙碑,文缺不可复识。又有晋城门校尉昌原恭侯郑仲林碑,晋泰始六年立。休水又北流注于洛水。洛水又东径百谷坞北。戴延之《西征记》曰:坞在川南,因高为坞,高十余丈,刘武王西入长安,舟师所保也。洛水又北,阳渠水注之。《竹书纪年》,晋襄公六年,洛绝于涧。即此处也。洛水又北径偃师城东,东北历中,水南谓之南,亦曰上也。径訾城西,司马彪所谓皆聚也,而水注之,水出北山溪,其水南流,世谓之温泉水。水侧有僵人穴,穴中有僵尸,戴延之《从刘武王西征记》曰:有此尸,尸今犹在。夫物无不化之理,魄无不迁之道,而此尸无神识,事同木偶之状,喻其推移,未若正形之速迁矣。水又东南,于皆城西北东入洛水。故京相璠曰:今巩洛渡北,有谷水东入洛,谓之下。故有上、下之名,亦谓之北,于是有南、北之称矣。
又有城,盖周大夫肸之旧邑。洛水又东径訾城北,又东,罗水注之,水出方山罗川,西北流,蒲他水注之,水南出蒲陂,西北流合罗水,谓之长罗川。亦曰罗中也,盖肸子罗之宿居,故川得其名耳。罗水又西北,白马溪水注之,水出嵩山北麓,径白马坞东,而北入罗水。西北流,白桐涧水注之,水出嵩麓桐溪,北流径九山东,又北,九山溪水入焉。水出百称山东谷,其山孤峰秀出,嶕峣分立。仲长统曰:昔密有卜成者,身游九山之上,放心不拘之境,谓是山也。山际有九山庙,庙前有碑云:九显灵府君者,太华之元子,阳九列名,号曰九山府君也。南据嵩岳,北带洛澨,晋元康二年九月,太岁在戌,帝遣殿中中郎将、关内侯樊广,缑氏令王与,主簿傅演,奉宣诏命,兴立庙殿焉。又有《百虫将军显灵碑》,碑云:将军姓伊氏,讳益,字隤敳,帝高阳之第二子伯益者也。晋元康五年七月七日,顺人吴义等建立堂庙,水平元年二月二十日刻石立颂,赞示后贤矣。其水东北流入白桐涧,又北径袁公坞东,盖公路始固有此也,故有袁公之名矣。北流注于罗水。罗水又西北径袁公坞北,又西北径潘岳父子墓前。有碑,岳父茈,琅琊太守,碑石破落,文字缺败。岳碑题云:给事黄门侍郎潘君之碑。碑云:君遇孙秀之难,阖门受祸,故门主感覆醢以增恸,乃树碑以记事。太常潘尼之辞也。罗水又于訾城东北入于洛水也。
又东北过巩县东,又北入于河。
洛水又东,明乐泉水注之,水出南原下,三泉并导,故世谓之五道泉,即古明溪泉也。《春秋》昭公二十二年,师次于明溪者也。洛水又东径巩县故城南,东南所居也,本周之畿内巩伯国也。《春秋左传》所谓尹文父涉于巩,即于此也。洛水又东,浊水注之。即古黄水也,水出南原。京相璠曰:訾城北三里有黄亭,即此亭也。《春秋》所谓次于黄者也。洛水又东北,涧水发南溪石泉,世亦名之为石泉水也。京相璠曰:巩东地名坎欲,在泂水东。疑即此水也。又径盘谷坞东,世又名之曰盘谷水。司马彪《郡国志》,巩有坎欿聚。《春秋》僖公二十四年,王出及坎欿。服虔亦以为巩东邑名也。今考厥文若状焉,而不能精辨耳。《晋太康地记》、《晋书地道记》,并言在巩西,非也。其水又北入洛,洛水又东北流,人于河。《山海经》曰:洛水,成皋西人河是也。谓之洛欿,即什谷也。故张仪说秦曰:下兵三川,塞什谷之口。谓此川也。《吏记音义》曰:巩县有谷水者也。黄帝东巡河,过洛,修坛沉壁,受《龙图》于河,《龟书》于洛,赤文绿字。尧帝又修坛河、洛,择良即沉,荣光出河,休气四塞,白云起,回风逝,赤文绿色,广袤九尺,负理平上,有列星之分,七政之度。《帝王录》记兴亡之数,以授之尧,又东沉书于日稷,赤光起,玄龟负书,背甲赤文成字,遂禅于舜。舜又习尧礼,沉书于日稷,赤光起,玄龟负书至于稷下,荣光休至,黄龙卷甲,舒图坛畔,赤文绿错以授舜。舜以禅禹,殷汤东观于洛,习礼尧坛,降璧三沉,荣光不起,黄鱼双跃,出济于坛。黑乌以浴,随鱼亦上,化为黑玉赤勒之书,黑龟赤文之题也,汤以伐桀,故《春秋说题辞》曰:河以道坤出天苞,洛以流川吐地符,王者沉礼焉,《竹书纪年》曰,洛伯用与河伯冯夷斗,盖洛水之神也。昔夏太康失政,为羿所逐,其昆弟五人,须于洛汭,作《五子之歌》,于是地矣。
伊水出南阳鲁阳县西蔓渠山,《山海经》曰:蔓渠之山,伊水出焉。《淮南子》曰:伊水出上魏山。《地理志》曰:出熊耳山即麓大同,陵峦互别耳。伊水自熊耳东北径鸾川亭北,姦水出姦山,北流际其城东而北入伊水。世人谓伊水为姦水,姦水为交水,故名斯川为鸾川也。又东为渊潭,潭浑若沸,亦不测其深浅也。伊水又东北径东亭城南,又屈径其亭东,东北流者也。东北过郭落山,阳水出阳山阳溪,世人谓之太阳谷,水亦取名焉。东流入伊水,伊水又东,北鲜水入焉,水出鲜山,北流注于伊。伊水又与蛮水合,水出卢氏县之蛮谷,东流入于伊。
又东北过陆浑县南,《山海经》曰:滽滽之水,出于厘山,南流注于伊水。今水出陆浑县之西南王母涧,涧北山上有王母祠,故世因以名溪,东流注于伊水,即滽滽之水也。伊水历崖口,山峡也。翼崖深高,壁立若阙,崖上有坞,伊水径其下,历峡北流,即古三涂山也。杜预《释地》曰:山在县南。阚駰《十三州志》云:山在东南。今是山在陆浑故城东南八十许里。《周书》,武王问太公曰:吾将因有夏之居,南望过于三涂,北瞻望于有河。《春秋》昭公四年,司马侯曰:四岳、三涂、阳城、太室、荆山、中南、九州之险也,服虔曰:三涂、大行、轘辕、崤、渑,非南望也。京相璠著《春秋土地名》,亦云:山名也。以服氏之说,涂,道也。准《周书》南望之文,或言宜为轘辕、大谷、伊阙,皆为非也。《春秋》,晋伐陆浑,请有事于三涂。知是山明矣。有七谷水注之,水西出女儿山之南七溪山,上有西王母祠,东南流注于伊水。又北,蚤谷水注之,水出女儿山之东谷,东径故亭南,东流入千伊水。伊水又东北径伏流岭东,岭上有昆仑祠,民犹祈焉。刘澄之《永初记》称,陆浑县西有伏流坂者也。今山在县南崖口北二十里许,西则非也。北与温泉水合,水出新城县之狼皋山西南阜下,西南流会于伊水。伊水又东北径伏睹岭,左纳焦涧水,水西出鹿髆山,东流径孤山南。其山介立丰上,单秀孤峙,故世谓之方山,即刘中书澄之所谓县有孤山者也。东历伏睹岭南,东流注于伊。伊水又东北,涓水注之,水出陆浑西山,即陆浑都也。寻郭文之故居,访胡昭之遗像,世去不停,莫识所在。其水有二源,俱导而东注虢略。在陆浑县西九十里也,司马彪《郡国志》曰县西虢略地,《春秋》所谓东尽虢略者也。北水东流合侯涧水,水出西北侯溪,东南流注于涓水。涓水又东径陆浑县故城北。平王东迁,辛有适伊川,见有被发而祭于野者曰:不及百年,此其戎乎。鲁僖公二十二年,秦、晋迁阶浑之戎于伊川,故县氏之也。涓水东南流,左合南水,水出西山七谷,亦谓之七谷水。阻涧东逝,历其县南,又东南左会北水,乱流,左合禅渚水,水上承陆浑县东禅渚,渚在原上,陂方十里,佳饶鱼苇,即《山海经》所谓南望禅渚,禹父之所化。郭景纯注云:禅,一音暖,鲧化羽渊而复在此然已变怪,亦无往而不化矣。世谓此泽为慎望陂,陂水南流注于涓水。涓水又东南注于伊水。昔有莘氏女采桑于伊川,得婴儿于空桑中,言其母孕于伊水之滨,梦神告之曰:臼水出而东走。母明视而见臼水出焉。告其邻居而走,顾望其邑,威为水矣。其母化为空桑,子在其中矣。莘女取而献之,命养于庖,长而有贤德,殷以为尹,曰伊尹也。
又东北过新城县南,马怀桥长水出新城西山,东径晋使持节、征南将军宗均碑南。均字文平,县人也。其碑,太始三年十二月立。其水又东流入于伊。又有明水出梁县西狼皋山,俗谓之石涧水也。西北流径杨亮垒南,西北合康水,水亦出狼皋山,东北流径范坞北与明水合,又西南流入于伊。《山海经》曰:放皋之山,明水出焉,南流注于伊水是也。伊水又与大戟水会,水出梁县西,有二源,北水出广成泽,西南径杨志坞北与南水合,水源南出广成泽,西流径陆浑县南。《河南十二县境簿》曰:广成泽在新城县界黄阜。西北流,屈而东,径杨志坞南;又北屈径其坞东,又径坞北。同注老倒涧,俗谓之老倒涧水,西流入于伊。伊水又北径新城东与吴涧水会,水出县之西山,东流南屈,径其县故城西,又东转径其县南,故蛮子国也。县有鄤聚,今名蛮中是也,汉惠帝四年置县。其水又东北流,庄于伊水。伊水又北径当阶城西,大狂水入焉,水东出阳城县之大山。《山海经》曰:大之山多琈之玉。其阳,狂水出焉。西南流,其中多三足龟,人食之者无大疾,可以已肿。狂水又西径纶氏县故城南。《竹书纪年》曰:楚吾得帅师及秦伐郑围纶氏者也。左与倚薄山水合,水北出倚薄之山,南径黄城西,又南径纶氏县故城东,而南流注于狂水。狂水又西,八风溪水注之,水北出八风山,南流径纶氏县故城西,西南流入于狂水。狂水又西得三交水口,水有三源,各导一溪,并出山南流合舍,故世有三交之名也。石上菖蒲,一寸九节,为药最妙,服久化仙。其水西南流注于狂水。狂水又西径缶高山北,西南与湮水合,水出东北湮谷,西南流径武林亭东北,又屈径其亭南,其水又西南径湮阳亭东,盖藉水以名亭也,又东南流入于狂。狂水又西径湮阳城南。又西径当阶城南,而西流注于伊。伊水又北,土沟水出玄望山西,东径玄望山南,又东径新城县故城北,东流注于伊水。伊水又北,板桥水入焉,水出西山,东流入于伊水。伊水又北会厌涧水,水出西山,东流径邥垂亭南。《春秋左传》文公十六年,秋,周甘歜败戎于邥垂者也。服虔曰:邥垂在高都南。杜预《释地》曰:河南新城县北有邥垂亭。司马彪《郡国志》曰:新城有高都城。今亭在城南七里,遗基存焉。京相璠曰:旧说言邥垂在高都南,今上党有高都县。余谓京论疏远,未足以证,无如虔说之指密矣。其水又东注于伊水。伊水又北径高都城东。徐广《史记音义》曰:今河南新城县有高都城。《竹书纪年》,梁惠成王十六年,东周与郑高都利者也。又来儒之水出于半石之山,西南流径斌轮城北,西历艾涧,以其水西流,又谓之小狂水也。其水又西南径大石岭南,《开山图》所谓大石山也。山下有《大石岭碑》,河南隐士通明,以汉灵帝中平六年八月戊辰。于山堂立碑,文字浅鄙,殆不可寻。魏文帝猎于此山,虎超乘舆,孙礼拔剑投虎于是山。山在洛阳南,而刘澄之言在洛东北,非也。山阿有魏明帝高平陵,王隐《晋书》曰:惠帝使校尉陈总仲元诣洛阳山请雨,总尽除小祀,惟存大石而祈之,七日大雨。即是山也。来儒之水又西南径赤眉城南,又西至高都城东西入伊水,谓之曲水也。
又东北过伊阙中,伊水径前亭西。《左传》昭公二十二年,晋箕遗、乐征、右行诡济师,取前城者也。京相璠曰:今洛阳西南五十里伊阙外前亭矣。服虔曰:前读为泉,周地也。伊水又北入伊阙,昔大禹疏以通水,两山相对,望之若阙,伊水历其间北流,故谓之伊阙矣,《春秋》之阙塞也。昭公二十六年,赵鞅使女宽守阙塞是也。陆机云:洛有四阙,斯其一焉。东岩西岭,并镌石开轩,高甍架峰,西侧灵岩下,泉流东注,入于伊水。傅毅《反都赋》曰:因龙门以畅化,开伊阙以达聪也。阙左壁有石铭云:黄初四年六月二十四日辛巳,大出水,举高四丈五尺,齐此已下。盖记水之涨减也。右壁又有石铭云:元康五年,河南府君循大禹之轨,部督邮辛曜、新城令王琨,部监作掾董猗、李褒,斩岸开石,平通伊阙,石文尚存也。
又东北至洛阳县南,北入于洛。
伊水自阙东北流,枝津右出焉。东北引溉,东会合水,同注公路涧,入于洛,今无水。《战国策》曰:东周欲为田,西周不下水,苏子见西周君曰,今不下水,所以富东周也,民皆种他种,欲贫之,不如下水以病之,东周必复种稻,种稻而复夺之,是东周受命于君矣。西周遂下水,即是水之故渠也。伊水又东北,枝渠左出焉,水积成湖,北流注于洛,今无水。伊水又东北至洛阳县南,径圜丘东,大魏郊天之所准汉故事建之。《后汉书·郊祀志》曰:建武二年,初制郊兆于洛阳城南七里,为圜坛八陛,中又为重坛,天地位其上,皆南向,其外坛,上为五帝位,其外为谴,重营皆紫,以像紫宫。按《礼》,天子大裘而冕,祭皞天上帝于此,今兖冕也。坛壝无复紫矣。伊水又东北流,注于洛水。《广志》曰:鲵鱼声如小儿啼,有四足,形如鲮鳢,可以治牛,出伊水也。司马迁谓之人鱼,故其著《史记》曰:始皇帝之葬也,以人鱼膏为烛。徐广曰:人鱼似鲇而四足,即鲵鱼也。
瀍水出河南谷城县北山,县北有朁亭,瀍水出其北梓泽中,梓泽,地名也。泽北对原阜,即裴氏墓茔所在,碑阙存焉。其水历泽东南流,水西有一原,其上平敞,古朁亭之处也。即潘安仁《西征赋》所谓越街邮者也。
东与千金渠合,《周书》曰:我卜瀍水西。谓斯水也。东南流,水西南有帛仲理墓,墓前有碑,题云:真人帛君之表。仲理名护,益州巴郡人,晋永宁二年十一月立。瀍水又东南流,注于谷。谷水自千金堨东注,谓之千金渠也。
又东过洛阳县南,又东过偃师县,又东入于洛。
涧水出新安县南白石山,《山海经》曰:白石之山,惠水出于其阳,东南注于洛,涧水出于其阴,北流注于谷。世谓是山曰广阳山,水曰赤岸水,亦曰石子涧。《地理志》曰:涧水在新安县,东南入洛。是为密矣。东北流历函谷东坂东,谓之八特板。东南入于洛。
孔安国曰:涧水出渑池山,今新安县西北有一水,北出渑池界,东南流径新安县,而东南流入于谷水。安国所言当斯水也。然谷水出渑池,下合涧水,得其通称,或亦指之为涧水也。并未之祥耳。今孝水东十里有水,世谓之慈涧,又谓之涧水。按《山海经》则少水也,而非涧水,盖习俗之误耳。又按河南有离山水,谓之为涧水,水西北出离山,东南流历郏山,于谷城东而南流注于谷,旧与谷水乱流,南人于洛。今谷水东入千金渠,涧水与之俱,东入洛矣。或以是水并为周公之所相卜也。吕忱曰:今河南死水。疑其是此水也。然意所未详,故并书存之耳。


卷十六  谷水、甘水、漆水、浐水、沮水 
谷水出宏农黾池县南墦冢林谷阳谷,《山海经》曰:傅山之西有林焉,曰墦冢,谷水出焉,东流注于洛,其中多玉。今谷水出千崤东马头山谷阳谷,东北流历黾池川,本中乡地也。汉景帝中二年,初城,徙万户为县,因崤黾之池以目县焉。亦或谓之彭池,故徐广《史记音义》曰:黾,或作彭。谷水出处也。谷水又东径秦、赵二城南。司马彪《续汉书》曰:赤眉从黾池自利阳南,欲赴宜阳者也。世谓之俱利城。耆彦曰:昔秦、赵之会,各据一城,秦王使赵王鼓瑟。蔺相如令秦王击缶处也。冯异又破赤眉于是川矣,故光武《玺书》曰:始虽垂翅回溪,终能奋翼黾池,可谓失之东嵎,收之桑榆矣。谷水又东径土崤北,所谓三崤也。谷水又东,左会北溪,溪水北出黾池山,东南流注于谷。疑即孔安国所谓涧水也。谷水又东径新安县故城南,北夹流而西接崤黾。昔项羽西入秦,坑降卒二十万于此,国灭身亡,宜矣。谷水又东径千秋亭南,其亭累石为垣,世谓之千秋城也。潘岳《西征赋》曰:亭有千秋之号,子无七旬之期。谓是亭也。又东径雍谷溪,回岫萦纡,石路阻峡,故亦有峡石之称矣。谷水历侧,左与北川水合,水有二源,并导北山,东南流合成一水,自乾注巽,入于谷。谷水又东径缺门山,山阜之不接者里余,故得是名矣。二壁争高,斗耸相乱,西瞻双阜,右望如砥。谷水自门而东,广阳川水注之,水出广阳北山,东南流注于谷。南望微山,云峰相乱。谷水又径白超垒南,戴延之《西征记》云:次至白超垒,去函谷十五里,筑垒当大道,左右有山夹立,相去百余步,从中出北,乃故关城,非所谓白超垒也。是垒在缺门东十五里,垒侧旧有坞,故冶官所在。魏、晋之日,引谷水为水冶,以经国用,遗迹尚存。谷水又东,石默溪水出微山东麓石默溪,东北流入于谷。谷水又东,宋水北流注于谷。谷水又东径魏将作大匠田丘兴墓南,二碑存焉。俭父也。《管辂别传》曰:辂尝随军西征,过其墓而叹,谓士友曰:玄武藏头,青龙无足,白虎衔尸,朱雀悲哭,四危已备,法应灭族。果如其言。谷水又东径函谷关南,东北流,皂涧水注之,水出新安县,东南流径田丘兴墓东,又南径函谷关西,关高险狭,路出廛郭。汉元鼎三年,楼船将军杨仆数有大功,耻居关外,请以家僮七百人,筑塞徒关于新安,即此处也。昔郭丹西入关,感慨于其下曰:不乘驷马高车,终不出此关也。去家十二年,果如志焉。皂涧水又东流入于谷。谷水又东北径函谷关城东,右合爽水。《山海经》曰:白石山西五十里曰谷山,其上多谷,其下多桑,爽水出焉。世谓之纻麻涧,北流注于谷,其中多碧绿。谷水又东,涧水注之。《山海经》曰:娄涿山西四十里曰白石之山,涧水出焉,北流注于谷。挚仲治《三辅决录注》云:马氏兄弟五人,共居涧、谷二水之交,作五门客,因舍以为名。今在河南西四十里。以《山海经》推校,里数不殊仲治所记,水会尚有故居处。斯则涧水也,即《周书》所谓我卜涧水东。言是水也。自下通谓涧水为谷水之兼称焉。故《尚书》曰:伊、洛、瀍、涧,既入于河。而无谷水之目,是名亦通称矣。刘澄之云:新安有涧水,源出县北,又有渊水,未知其源。余考诸地记,并无渊水,但渊、涧字相似,时有字错为渊也。故阚駰《地理志》曰:《禹贡》之渊水,是以知传写书误,字谬舛真,澄之不思所致耳。既无斯水,何源之可求乎?谷水又东,波水注之。《山海经》曰:瞻诸山西三十里娄涿之山,无草木,多金玉,波水出于其阴,世谓之百答水,北流注于谷,其中多茈石、文石。谷水又东,少水注之。《山海经》曰:瞻诸山西三十里曰瞻诸之山,其阳多金,其阴多文石,少水出于其阴。控引众溪,积以成川,东流注于谷,世谓之慈涧也。谷水又东,俞随之水注之。《山海经》曰,平蓬山西十里廆山,其阳多琈之玉,俞随之水出于其阴,北流注于谷,世谓之孝水也。潘岳《西征赋》曰:澡孝水以濯缨,嘉美名之在兹。是水在河南城西十余里,故吕忱曰:孝水在河南。而戴延之言在函谷关西。刘澄之又云出檀山。檀山在宜阳县西,在谷水南,无南入之理。考寻兹说,当承缘生《述征》谬志耳。缘生从戍行旅,征途讯访,既非旧土,故无所究,今川澜北注,澄映泥泞,何得言枯涸也。皆为疏僻矣。
东北过谷城县北,城西临谷水,故县取名焉。谷水又东径谷城南,不历其北,又东,洛水枝流入焉,今无水也。
又东过河南县北,东南入于洛。
河南王城西北,谷水之右有石碛,碛南出为死谷,北出为湖沟。魏太和四年,暴水流高三丈,此地下停流以成湖渚,造沟以通水,东西十里,决湖以注瀍水。谷水又径河南王城西北,所谓成周矣。《公羊》曰:成周者何?东周也。何休曰:名为成周者,周道始成,王所都也。《地理志》曰:河南河南县,故郏、鄏地也。京相璠曰:郏山名;鄏地邑也。卜年定鼎,为王之东都,谓之新邑,是为王城。其城东南名曰鼎门,盖九鼎所从入也,故谓是地为鼎中。楚子伐陆浑之戎,问鼎于此;《述征记》曰:谷、洛二水,本于王城东北合流,所谓谷、洛斗也。今城之东南缺千步,世又谓之谷、洛斗处,俱为非也。余按史传,周灵王之时,谷、洛二水斗,毁王宫,王将堨之,太子晋谏王,不听,遗堰三堤尚存。《左传》襄公二十五年,齐人城郏,穆叔如周贺。韦昭曰:洛水在王城南,谷水在王城北。东入于瀍。至灵王时,谷水盛出于王城西,而南流合子洛;两水相格,有似于斗,而毁王城西南也。颍容著《春秋条例》言,西城梁门枯水处,世谓之死谷是也。始知缘生行中造次,入关经究故事,与实违矣。考王封周桓公于是为西周,及其孙惠公,封少子于巩为东周,故有东西之名矣。秦灭周,以为三川郡,项羽封申阳为河南王,汉以为河南郡,王莽又名之曰保忠信卿:光武都各阳,以为尹。尹,正也,所以董正京畿,率先百郡也。谷水又东流径乾祭门北,子朝之乱,晋所开也,东至千金堨。《河南十二县境簿》曰:河南县城东十五里有千金堨。《洛阳记》曰:千金堨旧堰谷水,魏时更修此堰,谓之千金堨竭。积石为堨而开沟渠五所,谓之五龙渠。渠上立堨,堨之东首,立一石人,石人腹上刻勒云:太和五年二月八白庚戌造筑此堨,更开沟渠此水衡渠上,其水助其坚也,必经年历世,是故部立石人以记之云尔。盖魏明帝修王、张故绩也。堨是都水使者陈协所造。《语林》曰:陈协数进阮步兵酒,后晋文王欲修九龙堰,阮举协,文王用之。掘地得古承水铜龙六枚,堰遂成。水历堨东注,谓之千金渠。逮于晋世,大水暴注,沟渎泄坏,又广功焉。石人东胁下文云:太始七年六月二十三日,大水迸瀑,出常流上三丈,荡坏二堨,五龙泄水,南注泻下,加岁久漱啮,每涝即坏,历载消弃大功,今故无令遏,更于西开泄,名曰代龙渠,地形正平,诚得为泄至理。千金不与水势激争,无缘当坏,由其卑下,水得逾上漱啮故也。今增高千金于旧一丈四尺,五龙自然必历世无患。若五龙岁久复坏,可转于西更开二堨、二渠。合用二十三万五千六百九十八功,以其年十月二十三日起作,功重人少,到八年四月二十日毕。代龙渠即九龙渠也。后张方入洛,破千金堨。永嘉初,汝阴太守李矩、汝南太守袁孚修之,以利漕运,公私赖之。水积年渠堨颓毁,石砌殆尽,遗基见存,朝廷太和中修复故堨。按千金堨石人西胁下文云:若沟渠久疏,深引水者当于河南城北、石碛西,更开渠北出,使首狐丘。故沟东下,因故易就碛坚,便时事业已讫,然后见之。加边方多事,人力苦少,又渠堨新成,未患于水,是以不敢预修通之。若于后当复兴功者,宜就西碛,故书之于石,以遗后贤矣。虽石碛沦败,故迹可凭,准之于文,北引渠东合旧渎。旧渎又东,晋惠帝造石梁于水上,按桥西门之南颊文,称晋元康二年十一月二十日,改治石巷、水门,除竖枋,更为函枋,立作覆枋,屋前后辟级续石障,使南北入岸,筑治嫩处,破石以为杀矣。到三年三月十五日毕讫。并纪列门广长深浅于左右巷,东西长七尺,南北龙尾广十二丈,巷渎口高三丈,谓之皋门桥。潘岳《西征赋》曰:驻马皋门。即此处也。谷水又东,又结石梁,跨水制城,西梁也。谷水又东,左会金谷水,水出太白原,东南流历金谷,谓之金谷水,东南流径晋卫尉卿石崇之故居。石季伦《金谷诗集叙》曰:余以元康七年,从太仆出为证虏将军,有别庐在河南界金谷涧中,有清泉茂树,众果、竹、柏、药草备具。金谷水又东南流入于谷。谷水又东径金墉城北,魏明帝于洛阳城西北角筑之,谓之金墉城。起层楼于东北隅,《晋宫阁名》曰:金墉有崇天堂,即此。地上架木为榭,故白楼矣。皇居创徙,宫极未就。止跸于此。构宵榭于故台,所谓台以停停也。南曰乾光门,夹建两观,观下列朱桁于堑,以为御路。东曰含春门,北有门,城上西面列观,五十步一睥睨,屋台置一钟以和漏鼓,西北连庑函荫,墉比广榭。炎夏之日,高视常以避暑。为绿水池一所,在金墉者也。谷水径洛阳小城北,因阿旧城,凭结金墉,故向城也。永嘉之乱,结以为垒,号洛阳垒,故《洛阳记》曰:陵云台西有金市,金市北对洛阳垒者也。又东历大夏门下,故夏门也。陆机《与弟书》云:门有三层,高百尺,魏明帝造,门内东侧,际城有魏明帝所起景阳山,余基尚存。孙盛《魏春秋》曰:景初元年,明帝愈崇宫殿雕饰观阁,取白石英及紫石英及五色大石于太行谷城之山,起景阳山于芳林园,树松竹草木,捕禽兽以充其中。于时百役繁兴,帝躬自掘土,率群臣三公已下,莫不展力。山之东,旧有九江,陆机《洛阳记》曰:九江直作圆水。水中作圆坛三破之,夹水得相径通。《东京赋》曰:濯龙芳林,九谷八溪,芙蓉覆水,秋兰被涯。今也山则块阜独立,江无复仿佛矣。谷水又东,枝分南入华林园,历疏圃南,圃中有古玉井,井悉以珉玉为之,以缁石为口,工作精密,犹不变古,璨焉如新。又径瑶华宫南,历景阳山北,山有都亭堂上结方湖,湖中起御坐石也。御坐前建蓬莱山,曲池接筵,飞沼拂席,南面射侯,夹席武峙,背山堂上,则石路崎岖,岩嶂峻险,云台风观,缨峦带阜,游观者升降阿阁,出入虹陛,望之状凫没鸾举矣。其中引水飞皋,倾澜瀑布,或枉渚声溜,潺潺不断,竹柏荫于层石,绣薄丛于泉侧,微飙暂拂,则芳溢于六空,实为神居矣。其水东注天渊池,池中有魏文帝九华台,殿基悉是洛中故碑累之,今造钓台于其上。池南直魏文帝茅茨堂,前有《茅茨碑》,是黄初中所立也。其水自天渊池东出华林园,径听讼观南,故平望观也。魏明帝常言,狱,天下之命也,每断大狱,恒幸观听之。以太和三年,更从今名。观西北接华林,隶簿昔刘桢磨石处也。《文士传》曰:文帝之在东宫也,宴诸文学,酒酣,命甄后出拜,坐者咸伏,惟刘桢平视之。太祖以为不敬,送徒隶簿。后太祖乘步牵车乘城,降阅簿作,诸徒咸敬,而桢拒坐,磨石不动。太祖曰,此非刘桢也,石如何性。桢曰:石出荆山玄岩之下,外炳五色之章,内秉坚贞之志,雕之不增文,磨之不加莹,禀气贞正,禀性自然。太祖曰:名岂虚哉?复为文学。池水又东流入洛阳县之南池,池,即故翟泉也,南北百一十步,东西七十步。皇甫谧曰:悼王葬景王于翟泉,今洛阳太仓中大冢是也,《春秋》定公元年,晋魏献子合诸侯之大夫于翟泉,始盟城周。班固、服虔、皇甫谧咸言翟泉在洛阳东北,周之墓地。今案周威烈王葬洛阳城内东北隅,景王冢在洛阳太仓中,翟泉在两冢之间,侧广莫门道东,建春门路北.路,即东宫街也,于洛阳为东北。后秦封吕不韦为洛阳十万户侯,大其城,并得景王冢矣,是其墓地也。及晋永嘉元年,洛阳东北步广里地陷,有二鹅出,苍色者飞翔冲天,白色者止焉。陈留孝廉董养曰:步广,周之翟泉,盟会之地,今色苍胡象矣,其可尽言乎?后五年,刘曜、王弥入洛,帝居平阳。陆机《洛阳记》曰:步广里在洛阳城内,宫东是翟泉所在,不得于太仓西南也。京相璠与裴司空彦季修《晋舆地图》,作《春秋地名》,亦言今太仓西南池水名翟泉。又曰:旧说言翟泉本自在洛阳北苌宏城,成周乃绕之。杜预因其一怔,谓必是翟泉,而即实非也。后遂为东宫池。晋《中州记》曰:惠帝为太子,出闻虾蟆声,问人为是官虾蟆、私虾蟆,侍臣贾允对曰:在官地为官虾蟆,在私地为私虾蟆。令曰:若官虾蟆,可给廪。先是有谶云:虾蟆当贵。昔晋朝收愍怀太子于后池,即是池也。其一水自大夏门东径宣武观,凭城结构,不更增墉,左右夹列步廊,参差翼跂,南望天渊池,北瞩宣武场。《竹林七贤论》曰:王戎幼而清秀,魏明帝于宣武场上为栏苞虎牙,使力士袒裼,迭与之搏,纵百姓观之。戎年七岁,亦往观焉,虎乘间薄栏而吼,其声震地,观者无不辟易颠仆,戎亭然不动。帝于门上见之,使问姓名而异之。场西故贾充宅地。谷水又东径广莫门北,汉之谷门也。北对芒阜,连岭修亘,苞总众山,始自洛口,西逾平阴悉芒垅也。《魏志》曰:明帝欲平北芒,令登合见孟津,侍中辛毗谏曰:若九河溢涌,洪水为害,丘陵皆夷,何以御之?帝乃止。谷水又东屈南,径建春门石桥下,即上东门也。阮嗣宗《咏怀诗》曰:步出上东门者也。一曰上升门,晋曰建阳门。《百官志》曰:洛阳十二门,每门候一人,六百石。《东观汉记》曰:郅恽为上东门候,光武尝出,夜还,诏开门欲入,恽不内,上令从门间识面,恽曰:火明辽远。遂拒不开,由是上益重之。亦袁本初挂节处也。桥首建两石柱,桥之《右柱铭》云,阳嘉四年乙酉壬申,诏书以城下漕渠,东通河、济,南引江、淮,方贡委输,所由而至,使中谒者魏郡清渊马宪监作石桥梁柱,敦敕工匠尽要妙之巧,攒立重石,累高周距,桥工路博,流通万里云云。河南尹邳崇、丞渤海重合双福、水曹掾中牟任防、史王荫、史赵兴、将作吏睢阳申翔,道桥掾成皋卑国,洛阳令江双,丞平阳降监掾王腾之,主石作右北平山仲,三月起作,八月毕成。其水依柱,又自乐里道屈而东出阳渠。昔陆机为成都王颖入洛败北而返,水南即马市,旧洛阳有三市,斯其一也。亦嵇叔夜为司马昭所害处也。北则白社故里,昔孙子荆会董威辇于白社,谓此矣。以同载为荣,故有《威辇图》。又东径马市石桥,桥南有二石柱,并无文刻也。汉司空渔阳王梁之为河南也,将引谷水以溉京都,渠成而水不流,故以坐免。后张纯堰洛以通漕,洛中公私穰赡。是渠今引谷水,盖纯之创也。按陆机《洛阳记》、刘澄之《永初记》言,城之西面有阳渠,周公制之也。昔周迁殷民于洛邑,城隍逼狭,卑陋之所耳。晋故城成周以居敬王,秦又广之,以封不韦,以是推之,非专周公可知矣。亦谓之九曲渎,《河南十二县境簿》云:九曲渎在河南巩县西,西至洛阳。又按傅畅《晋书》云:都水使者陈狼凿运渠,从洛口入注九曲至东阳门。是以阮嗣宗《咏怀诗》所谓朝出上东门,遥望首阳岑:又言遥遥九曲间,裴徊欲何之者也。阳渠水南暨阊阖门,汉之上西门者也。《汉宫记》曰:上西门所以不纯白者,汉家厄于戍,故以丹镂之。太和迁都,徙门南侧。其水北乘高渠,枝分上下,历故石桥东入城,径望先寺,中有碑,碑侧法《子丹碑》作龙矩势,于今作则佳,方古犹劣。渠水又东历故金市南,直千秋门右,宫门也。又枝流入石逗伏流,注灵芝九龙池。魏太和中,皇都迁洛阳,经构宫极,修理街渠,务穷隐,发石视之,曾无毁坏。又石工细密,非今之所拟,亦奇为精至也,遂因用之。其一水自千秋门南流径神虎门下,东对云龙门,二门衡栿之上,皆刻云龙风虎之状,以火齐薄之,及其晨光初起,夕景斜辉,霜文翠照,陆离眩目。又南径通门、掖门西,又南流东转,径阊阖门南。案《礼》,王有五门,谓皋门、库门、雉门、应门、路门,路门一曰毕门,亦曰虎门也。魏明帝上法太极于格阳,南宫起太极殿于汉崇德殿之故处,改雉门为阊阖门。昔在汉世,洛阳宫殿门题,多是大篆,言是蔡邕诸子。自董卓焚宫殿,魏太祖平荆州,汉吏部尚书安定梁孟皇善师宜官八分体,求以赎死。太祖善其法,常仰系帐中爱玩之,以为胜宜官。北宫榜题,咸是鹄笔,南宫既建,明帝令侍中京兆韦诞以古篆书之。皇都迁洛,始令中书舍人沈含馨以隶书书之;景明、正始之年,又敕符节令江式以大篆易之。今诸桁榜题,皆是式书。《周官》,太宰以正月悬治法于象魏。《广雅》曰:阙,谓之象魏。《风俗通》曰:鲁昭公设两观于门,是谓之阙,从门,欮声,《尔雅》曰:观谓之阙。《说文》曰:阙,门观也。《汉官典职》曰:偃师去洛四十五里,望朱雀阙,其上郁然与天连,是明峻极矣。《洛阳故宫名》有朱雀阙、白虎阙、苍龙阙、北阙、南宫阙也。《东观汉记》曰:更始发洛阳,李松奉引车马奔触北阙铁柱门,三马皆死。即斯阙也。《白虎通》曰:门必有阙者何?阙者,所以饰门,别尊卑也。今阊阖门外夹建巨阙,以应天宿,虽不如礼,犹象而魏之,上加复思,以易观矣。《广雅》曰:复思谓之屏。《释名》曰:屏,自障屏也;罦思在门外,罦复也。臣将入请事于此,复重思之也。汉末兵起,坏园陵罦思,曰无使民复思汉也。故《盐铁论》曰:垣阙罦思。言树屏隅角所架也。颖容又曰:阙者,上有所失,下得书之于阙,所以求论誉于人,故谓之阙矣。今阙前水南道右,置登闻鼓以纳谏。昔黄帝立明堂之议,尧有衢室之间,舜有告善之旌,禹有立鼓之讯,汤有总街之诽,武王有灵台之复,皆所以广设过误之备也。渠水又枝分,夹路南出,径太尉、司徒两坊间,谓之铜驼街。旧魏明帝置铜驼诸兽于阊阖南街,陆机云:驼高九尺,脊出太尉坊者也。水西有永宁寺,熙平中始创也,作九层浮图,浮图下基方十四丈,自金露槃下至地四十九丈,取法代都七级,而又高广之,虽二京之盛,五都之富,利刹灵图,未有若斯之构。按《释法显行传》,西国有爵离浮图,其高与此相状,东都西域,俱为庄妙矣。其地是曹爽故宅,经始之日,于寺院西南隅得爽窟室,下入土可丈许,地壁悉累方石砌之,石作细密,都无所毁,其石悉入法用,自非曹爽,庸匠亦难复制此。桓氏有言,曹子丹生此豚犊,信矣。渠左是魏、晋故庙地,今悉民居,无复遗墉也。渠水又西历庙社之间,南注南渠。庙社各以物色辨。方《周礼》,庙及路寝,皆如明堂,而有燕寝焉。惟祧庙则无,后代通为一庙,列正室于下,无复燕寝之制。《礼》,天子建国,左庙右社,以石为主,祭则希冕。今多王公摄事,王·者不亲拜焉。咸宁元年,洛阳大风,帝庙树折,青气属天,元王东渡,魏社代昌矣。渠水自铜驼街东径司马门南,魏明帝始筑,阙崩,压杀数百人,遂不复筑,故无阙门。南屏中旧有置铜翁仲处,金狄既沦,故处亦褫,惟坏石存焉。自此南直宣阳门,经纬通达,皆列驰道,往来之禁,一同两汉。曹子建尝行御街,犯门禁,以此见薄。渠水又东径杜元凯所谓翟泉北,今无水。坎方九丈六尺,深二丈余,似是人功而不类于泉陂,是验非之一证也。又皇甫谧《帝王世纪》云:王室定遂徒居,成周小,不受王都,故坏翟泉而广之,泉源既塞,明无故处,是验非之二证也。杜预言:翟泉在太仓西南,既言西南,于洛阳不得为东北,是验非之三证也。稽之地说,事几明矣,不得为翟泉也。渠水历司空府前,径太仓南,出东阳门石桥下,注阳渠。谷水自阊阖门而南径土山东,水西三里有坂,坂上有土山,汉大将军梁冀所成,筑土为山,植木成苑,张璠《汉记》曰:山多峭坂,以象二崤,积金玉,采捕禽兽,以充其中,有人杀苑兔者,迭相寻逐,死者十三人。南出径西阳门,旧汉氏之西明门也,亦曰雍门矣。旧门在南,太和中以故门邪出,故徙是门。东对东阳门。谷水又南径白马寺东。昔汉明帝梦见大人,金色,项佩白光。以问群臣,或对曰:西方有神名曰佛,形如陛下所梦,得无是乎?于是发使天竺,写致经像,始以榆盛经,白马负图,表之中夏。故以白马为寺名。此榆后移在城内愍怀太子浮图中,近世复迁此寺,然金光流照,法轮东转,创自此矣。谷水又南径平乐观东,李尤《平乐观赋》曰:乃设平乐之显观,章秘伟之奇珍。华峤《后汉书》曰:灵帝于平乐观下起大坛,上建十二重,五采华盖高十丈,坛东北为小坛,复建九重,华盖高九丈,列奇兵骑士数万人,天子住大盖下。礼毕,天子躬擐甲,称无上将军,行阵三匝而还,设秘戏以示远人。故《东京赋》曰:其西则有平乐都场,示远之观,龙雀蟠蜿,天马半汉。应劭曰:飞廉神禽,能致风气,古人以良金铸其象。明帝永平五年,长安迎取飞廉并铜马,置上西门外平乐观。今于上西门外无他基观,惟西明门外独有此台,巍然广秀,疑即平乐观也。又言皇女稚殇,埋于台侧,故复名之曰皇女台。晋灼曰:飞廉,鹿身头如雀,有角而蛇尾豹文。董卓销为金,用铜马徙于建始殿东阶下,胡军丧乱,此象遂沦。谷水又南径西明门,故广阳门也。门左枝渠东派入城,径太社前,又东径太庙南,又东于青阳门右下注阳渠。谷水又南,东屈径津阳门南,故津门也。昔洛水泛泆漂害者众,津阳城门校尉将筑以遏水,谏议大夫陈宣止之曰:王尊臣也,水绝其足,朝廷中兴,必不入矣。水乃造门而退。谷水又东径宣阳门南,故苑门也。皇都迁洛,移置于此,对阊阖门南,直洛水浮桁。故《东京赋》曰:溯洛背河,左伊右瀍者也。夫洛阳考之中土,卜惟洛食,实为神也。门左即洛阳池处也。池东旧平城门所在矣,今塞。北对洛阳南宫,故蔡邕曰:平城门,正阳之门,与宫连属,郊祀法驾所由从出门之最尊者。《洛阳诸宫名》曰,南宫有謻台临照台。《东京赋》曰:其南则有謻门,曲榭邪阻城洫。《注》云:謻门,冰室门也;阻,依也;洫,城下池也。皆屈曲邪行依城池为道。故《说文》曰:隍,城池也。有水曰池,无水曰隍矣。謻门即宣阳门也,门内有宣阳冰室,《周礼》有冰人,日在北陆而藏之西陆,朝觌而出之。冰室旧在宣阳门内,故得是名。门既拥塞,冰室又罢。谷水又径灵台北,望云物也。汉光武所筑,高六丈,方二十步。世祖尝宴于此台,得鼮鼠于台上,亦谏议大夫第五子陵之所居,伦少子也,以清正,洛阳无主人,乡里无田宅,寄止灵台,或十日不炊,司隶校尉南阳左雄,尚书庐江朱孟兴等,皆伦故孝廉功曹,各致礼饷,并辞不受,永建中卒。谷水又东径平昌门南,故平门也。又径明堂北,汉光武中元元年立。寻其基构,上圆下方,九室重隅十二堂。蔡邕《月令章句》同之,故引水于其下为辟雍也。谷水又东径开阳门南,《晋宫阁名》曰:故建阳门也,《汉官》曰:开阳门始成,未有名宿,昔有一柱来,在楼上。琅琊开阳县上言:县南城门,一柱飞去。光武皇帝使来,识视良是,遂坚缚之,因刻记年月日以名焉。何汤字仲弓,尝为门候,上微行夜还,汤闭门不内,朝廷嘉之。又东径国子太学石经北,《周礼》有国学,教成均之法。《学记》曰:古者,家有塾,党有痒,遂有序,国有学。亦有虞氏之上庠、下庠,夏后氏之东序、西序,殷人之左学、右学,周人之东胶、虞庠。《王制》云:养国老于上庠,养庶老于下庠,故有太学、小学,教国之子弟焉,谓之国子。汉魏以来,置太学于国子堂。东汉灵帝光和六年,刻石镂碑载五经,立于太学讲堂前,悉在东侧。蔡邕以熹平四年,与五官中郎将堂溪典,光禄大夫杨赐,谏议大夫马日磾,议郎张驯、韩说,太史令单扬等,奏求正定《六经》文字。灵帝许之,邕乃自书丹于碑,使工镌刻,立于太学门外。于是后儒晚学,咸取正焉。及碑始立,其观视及笔写者,车乘日千余辆,填塞街陌矣。今碑上悉铭刻蔡邕等名。魏正始中,又立古、篆、隶《三字石经》,古文出于黄帝之世,仓颉本鸟迹为字,取其孳乳相生,故文字有六义焉。自秦用篆书,焚烧先典,古文绝矣。鲁恭王得孔子宅书,不知有古文,谓之科斗书,盖因科斗之名,遂效其形耳。言大篆出于周宣之时,史籀创著。平王东迁,文字乖错,秦之李斯及胡母敬,又改籀书谓之小篆,故有大篆、小篆焉。然许氏《字说》专释于篆,而不本古文,言古隶之书起于秦代,而篆字文繁,无会剧务,故用隶人之省,谓之隶书,或云即程邈于云阳增损者,是言隶者,篆捷也。孙畅之尝见青州刺史傅宏仁说临淄人发古冢,得桐棺前和外隐为隶字,言齐太公六世孙,胡公之棺也。惟三字是古,余同今书,证知隶自出古,非始于秦。魏初,传古文出邯郸淳,《石经》古文,转失淳法,树之于堂西,石长八尺,广四尺,列石于其下,碑石四十八枚,广三十丈。魏明帝又刊《典论》六碑,附于其次。陆机言,《太学赞》别一碑,在讲堂西,下列《石龟碑》,载蔡邕、韩说、堂溪典等名。《太学弟子赞》复一碑,在外门中。今二碑并无。《石经》东有一碑,是汉顺帝阳嘉元年立,碑文云:建武二十七年造太学,年积毁坏。永建六年九月,诏书修太学,刻石记年,用作工徒十一万二千人,阳嘉元年八月作毕。碑南面刻颂,表里镂字,犹存不破。《汉石经》北有晋《辟雝行礼碑》,是太始二年立,其碑中折,但世代不同,物不停故,《石经》沦缺,存半毁几,驾言永久,谅用怃焉。考古有三雝之文,今灵台太学,并无辟雝处。晋永嘉中,王弥、刘曜入洛,焚毁二学,尚仿佛前基矣。谷水于城东南隅枝分北注,径青阳门东,故清明门也,亦曰税门,亦曰芒门。又北径东阳门东,故中东门也。又北径故太仓西,《洛阳地记》曰:大城东有太仓,仓下运船常有千计。即是处也。又北入洛阳沟。谷水又东左边为池,又东右出为方湖,东西百九十步,南北七十步,故水衡署之所在也。谷水又东南转屈而东注,谓之阮曲,云阮嗣宗之故居也。谷水又东注鸿池陂,《百官志》曰:鸿池,池名也,在洛阳东二十里,丞一人,二百石。池东西千步,南北千一百步,四周有塘池,中又有东西横塘,水溜径通,故李尤《鸿池陂铭》曰:鸿泽之陂,圣王所规,开源东注,出自城池也。其水又东,左合七里涧。晋《后略》曰:成都王颖使吴人陆机为前锋都督,伐京师,轻进,为洛军所乘,大败于鹿苑,人相登蹑,死于堑中及七里涧,涧为之满。即是涧也。涧有石梁,即旅人桥也。昔孙登不欲久居洛阳,知杨氏荣不保终,思欲遁迹林乡,隐沦妄死,杨骏埋之于此桥之东,骏后寻亡矣。《搜神记》曰:太康末,京、洛始为折杨之歌,有兵革辛苦之辞。骏后被诛,太后幽死,折杨之应也。凡是数桥,皆累石为之,亦高壮矣,制作甚佳,虽以时往损功,而不废行旅。朱超石《与兄书》云:桥去洛阳宫六七里,悉用大石,下圆以通水,可受大舫过也。题其上云:太康三年十一月初就功,日用七万五千人,至四月末止。此桥经破落,复更修补,今无复文字。阳渠水又东流径汉广野君郦食其庙南,庙在北山上,成公绥所谓偃师西山也。山上旧基尚存,庙宇东向,门有两石人对倚,北石人胸前铭云:门亭长石人。西有二石阙,虽经颓毁,犹高丈余。阙西,即庙故基也。基前有碑,文字剥缺,不复可识,子安仰澄芬于万古,赞清徽于庙像,文字厥集矣。阳渠水又东径毫殷南,昔盘庚所迁,改商曰殷,此始也。班固曰:尸乡,故殷汤所都者也。故亦曰汤亭。薛瓒《汉书注》、皇甫谧《帝王世纪》,并以为非,以为帝喾都矣。《晋太康记》、《地道记》,并言田横死于是亭,故改曰尸乡,非也。余按司马彪《郡国志》,以为春秋之尸氏也。其泽,野负原夹,郭多坟陇焉。即陆士衡会王辅嗣处也。袁氏《王陆诗叙》:机初入洛,次河南之偃师,时忽结阴,望道左若民居者,因往逗宿,见一少年,姿神端远,与机言玄,机服其能而无以酬折,前致一辩,机题纬古今,综检名实,此少年不甚欣解。将晓,去,税驾逆旅,妪曰:君何宿而来?自东数十里无村落,止有山阳王家墓。机乃怪怅,还睇昨路,空野霾云,攒木蔽日,知所遇者,审王弼也。此山即祝鸡翁之故居也。《搜神记》曰:祝鸡翁者,洛阳人也,居尸乡北山下,养鸡百年余,鸡至于余头,皆有名字。欲取,呼之名,则种别而至。后之吴山,莫知所去矣。谷水又东径偃师城南。皇甫谧曰:帝喾作都于亳,偃师是也。王莽之所谓师氏者也。谷水又东流注于洛水矣。
甘水出宏农宜阳县鹿蹄山,山在河南陆浑县故城西北,俗谓之纵山。水之所导,发于山曲之中,故世人目其所为甘掌焉。
东北至河南县南,北入洛。
甘水发源东北流,北屈径一故城东,在非山上,世谓之石城也。京相璠曰:或云甘水西山上,夷污而平有故甘城,在河南城西二十五里。指谓是城也。余按甘水东十许里洛城南,有故甘城焉,北对河南故城,世谓之鉴洛城,鉴、甘声相近,既故甘城也,为王子带之故邑矣。是以昭叔有甘公之称焉。甘水又与非山水会,水出非山东谷,东流入于甘水。甘水又于河南城西北入洛。《经》言县南,非也。京相璠曰:今河南县西南有甘水,北入洛。斯得之矣。漆水出扶风杜阳县俞山东,北入于渭。
《山海经》曰:羭次之山,漆水出焉,北流注于渭。盖自北而南矣。《尚书·禹贡》、太史公《禹本纪》云:导渭水东北至径,又东过漆、沮入于河。孔安国曰:漆、沮,一水名矣,亦曰洛水也,出冯翊北。周太王去邠,度漆逾梁山,止岐下,故《诗)云:民之初生,自土沮漆。又曰:率西水浒,至于岐下。是符《禹贡》、《本纪》之说。许慎《说文》称,漆水出右扶风杜阳县岐山,东入渭,从水,黍声。又云:一曰漆城池也。潘岳《关中记》曰:关中有泾、渭、灞、浐、酆、鄗、漆、沮之水,酆、鄗、漆、沮四水,在长安西南鄠县,漆、沮皆南注,酆、鄗水北注。《开山图》曰:丽山西北有温池。温池西南八十里岐山,在杜阳北。长安西有渠,谓之漆渠。班固《地理志》云:漆水在漆县西。阚駰《十三州志》又云:漆水出漆县西,北至岐山,东入渭。今有水出杜阳县岐山北漆溪,谓之漆渠,西南流注岐水。但川土奇异,今说互出,考之经史,各有所据,识浅见浮,无以辨之矣。
庐水出京兆蓝田谷,北入于灞。
《地理志》曰:浐水出南陵县之蓝田谷,西北流,与一水合,水出西南莽谷,东北流注浐水。浐水又北历蓝田川,北流注于灞水。《地理志》曰:浐水北至霸陵入霸水。
俎水出北地直路县,东过冯翊祋祤县北,东人于洛。
《地理志》曰:沮出直路县西,东入洛。今水自直路县东南,径谯石山东南流,历檀台川,俗谓之檀台水,屈而夹山西流,又西南径宜君川,世又谓之宜君水。又得黄嵚水口,水西北出云阳县石门山黄嵚谷,东南流注宜君水。又东南流径祋祤县故城西,县以汉景帝二年置,其水南合铜官水,水出县东北,西南径铜官川,谓之铜官水。又西南流径祋祤县东,西南流径其城南原下,而西南注宜君水。宜君水又南出土门山西,又谓之沮水。又东南历土门南原下,东径怀德城南,城在北原上。又东径汉太上皇陵北,陵在南原上,沮水东注郑渠。昔韩欲令秦无东伐,使水工郑国间秦凿泾引水,谓之郑渠,渠首上承泾水于中山西邸瓠口,所谓瓠中也。《尔雅》以为周焦获矣。为渠并北山,东注洛三百余里,欲以溉田。中作而觉,秦欲杀郑国。郑国曰:始臣为间,然渠亦秦之利。卒使就渠,渠成而用注填阏之水,溉泽卤之地四万余顷,皆亩一钟,关中沃野,无复凶年,秦以富强,卒并诸侯,命曰郑渠。渠渎东径宜秋城北,又东径中山南。《河渠书》曰:凿泾水自中山西。《封禅书》:汉武帝获宝鼎于汾阴,将荐之甘泉,鼎至中山,氤氲有黄云盖焉。徐广《史记音义》曰:关中有中山,非冀州者也。指证此山,俗谓之仲山,非也。郑渠又东径舍车宫南绝冶谷水。郑渠故渎又东径嶻薛山南,池阳县故城北,又东绝清水。又东径北原下,浊水注焉,自浊水以上,今无水。浊水上承云阳县东大黑泉,东南流,谓之浊谷水。又东南出原,注郑渠。又东历原径曲梁城北,又东径太上陵南原下,北屈径原东与沮水合,分为二水,一水东南出,即浊水也,至白渠与泽泉合,俗谓之漆水,又谓之为漆沮水。绝白渠,东径万年县故城北为栎阳渠,城,即栎阳宫也。汉高帝葬皇考于是县,起坟陵,署邑号,改曰万年也。《地理志》曰:冯翊万年县,高帝置,王莽曰异赤也。故徐广《史记音义》曰:栎阳,今万年矣。阚駰曰:县西有泾、渭,北有小河。谓此水也。其水又南屈,更名石川水,又西南径郭城西与白渠枝渠合,又南入于渭水也。其一水东出,即沮水也,东与泽泉合,水出沮东泽中,与沮水隔原,相去十五里,俗谓是水为漆水也。东流径薄昭墓南,冢在北原上。又径怀德城北,东南注郑渠,合沮水。又自沮直绝注浊水,泵白渠合焉,故浊水得漆沮之名也。沮循郑渠,东径当道城南。城在频阳县故城南,频阳宫也,秦厉公置。城北有频山,山有汉武帝殿,以石架之。县在山南,故曰频阳也。应劭曰:县在频水之阳。今县之左右,无水以应之,所可当者,惟郑渠与沮水。又东径莲芍县故城北,《十二州志》曰:县以草受名也,沮水又东径汉光武故城北,又东径粟邑县故城北,王莽更名粟城也。后汉封骑都尉耿夔为侯国。其水又东北流,注于洛水也。


卷十七  渭水 
渭水出陇西首阳县渭谷亭南鸟鼠山,渭水出首阳县首阳山渭首亭南谷,山在鸟鼠山西北。此县有高城岭,岭上有城,号渭源城,渭水出焉。三源合注,东北流径首阳县西,与别源合,水南出鸟鼠山渭水谷,《尚书·禹贡》所谓渭出鸟鼠者也。《地说》曰:鸟鼠山,同穴之枝干也。渭水出其中,东北过同穴枝间,既言其过,明非一山也。又东北流而会于殊源也。渭水东南流径首阳县南,右得封溪水,次南得广相溪水,次东得共谷水,左则天马溪水,次南则伯阳谷水,并参差翼注,乱流东南出矣。
东北过襄武县北,广阳水出西山,二源合注,共成一川,东北流注于渭。渭水又东南径襄武县东北,荆头川水入焉,水出襄武西南鸟鼠山荆谷,东北径襄武县故城北。王莽更名相桓,汉护羌校尉温序行部,为隗嚣部将苟宇所拘,衔须自刎处也。其水东北流注于渭,渭水常若东南,不东北也。又东,枲水注之,水出西南雀富谷,东北径襄武县南,东北流入于渭。《魏志》称:咸熙二年,襄武上言,大人见,身长三丈余,迹长三尺二寸,白发,著黄单衣巾,拄杖,呼民王始语云,今当太平。十二月,天禄永终,历数在晋。遂迁魏而事晋。又东过镡道县南,右则岑溪水,次则同水,俱左注之。次则过水右注之,渭水又东南,径源道县故城西。昔秦孝公西斩戎之源王。应劭曰:源,戎邑也。汉灵帝中平五年,别为南安郡。赤亭水出郡之东山赤谷,西流径城北,南人渭水。渭水又径城南,得粟水,水出西南安都谷,东北流注于渭。渭水又东,新兴川水出西南乌鼠山,二源合舍、东北流与彰川合。水出西南溪下,东北至彰县南。本属故道候尉治,后汉县之,永元元年,和帝封耿秉为侯国也。万年川水出南山,东北流注之。又东北注新兴川。又东北径新兴县北,《晋书地道记》,南安之属县也。其水又东北,与南川水合,水出西南山下,东北台北水,又东北注于渭水。渭水又东径武城县西,武城川水入焉。津源所导,出鹿部西山,两源合注,东北流径鹿部南,亦谓之鹿部水。又东北,昌丘水出西南丘下,东北注武城水,乱流东北注渭水。渭水又东入武阳川。又有关城川水出南,安城谷水出北,两川参差注渭水。渭水又东,有落门西山东流,三谷水注之,三川统一,东北流,注于渭水。有落门聚,昔冯异攻落门,未拔而薨。建武十年,来歙又攻之,擒魄嚣子纯,陇右平。渭水自落门东至黑水峡,左右六水夹往。左则武阳溪水,次东得土门谷水,俱出北山,南流入渭。右则温谷水,次东有故城溪水,次东有间里溪水,亦名习溪水,次东有黑水,井出南山。北流入渭,渭水又东出黑水峡,历冀川。
又东过冀县北,渭水自黑水峡至岑峡,南北十一水注之。北则温谷水,导平襄县南山温溪,东北流,径平襄县故城南,故襄戎邑也。王莽之所谓平相矣。其水东南流,历三堆南,又东流南屈,历黄槐川,梗津渠,冬则辍流,春夏水盛则通川注渭。次则牛谷水,南入渭水。南有长堑谷水,次东有安蒲溪水,次东有衣谷水,并南出朱国山。山在梧中聚,有石鼓不击自鸣,鸣则兵起。汉成帝鸿嘉三年,天水冀南山有大石自鸣,声隐隐如雷,有顷止,闻于平襄二百四十里,野鸡皆鸣,石长丈三尺,广厚略等,著崖胁,去地百余丈,民俗名曰石鼓。石鼓鸣则有兵,是岁,广汉钳子攻死囚,盗库兵,略吏民,衣绣衣,自号为仙君,党与漫广,明年冬伏诛,自归者三千余人。信而有征矣。其水北径冀县城北。秦武公十年,伐冀戎,县之。故天水郡治,王莽更名镇戎县曰冀治,汉明帝永平十七年改曰汉阳郡,城即隗嚣称西伯所居也。后汉马超之围冀也,凉州别驾阎伯俭潜出水中,将告急夏侯渊,为超所擒,令告城无救。伯俭曰大军方至,咸称万岁。超怒,数之。伯俭曰:卿欲令长者出不义之言乎?遂杀之。渭水又东合冀水,水出冀谷,次东有浊谷水,次东有当里溪水,次东有托里水,次东有渠谷水,次东有黄土川水,俱出南山,北径冀城东,而北流注于渭。渭水又东出岑峡,入新阳川,径新阳下城南,溪谷,赤蒿二水,并出南山。东北入渭水。渭水又东与新阳崖水合,即陇水也。东北出陇山,其水西流,右径瓦亭南。隗嚣闻略阳陷,使牛邯守瓦亭,即此亭也。一水亦出陇山,东南流,历瓦亭北,又西南合为一水,谓之瓦亭川。西南流,径清宾溪北,又西南与黑水合,水出黑城北。西南径黑城西,西南流,莫吾南川水注之。水东北出陇垂,西南流,历黑城南,注黑水。黑水西南出悬镜峡,又西南入瓦亭水,又有水自西来会,世谓之鹿角口。又南径阿阳县故城东。中平元年,北地羌胡与边章侵陇右,汉阳长史盖勋屯阿阳以拒贼,即此城也。其水又南与燕无水合,水源延发东山,西注瓦亭水。瓦亭水又南,左会方城川,西注瓦亭水。瓦亭水又南,径成纪县东,历长离川,谓之长离水。右与成纪水合,水导源西北当亭川,东流出破石峡,津流遂断。故渎东径成纪县。故帝大皞庖牺所生之处也。汉以为夭水郡县,王莽之阿阳郡治也。又东,潜源隐发,通入成纪水,东南入瓦亭水。瓦亭水又东南,与受渠水相会,水东出大陇山,西径受渠亭北,又西南入瓦亭水。瓦亭水又西南流,历僵人峡。路侧岩上有死人僵尸峦穴,故岫壑取名焉。释鞍就穴直上,可百余仞,石路逶迤,劣通单步,僵尸倚窟,枯骨尚全,唯无肤发而已。访其川居之士,云其乡中父老作童儿时,已闻其长旧传此,当是数百年骸矣。其水又西南与略阳川水合,水出陇山香谷西,西流,右则单溪西注,左则阁川水入焉。其水又西历蒲池郊,石鲁水出东南石鲁溪,西北注之。其水又西历略阳川,西得破社谷水,次西得平相谷水,又西得金里谷水,又西得南室水,又西得蹄谷水,并出南山,北流于略阳城东,扬波北注。川水又西径略阳道故城北,埿渠水出南山,北径埿峡北,入城。建武八年,中郎将来歙,与祭遵所部护军王忠、右辅将军朱宠将二千人,皆持卤刀斧。自安民县之杨城。元始二年,成帝罢安定滹沱苑以为安民县,起官寺市里。从番须回中,伐树木,开山道,至略阳,夜袭击嚣,拒守将金梁等,皆杀之,因保其城,隗嚣闻略阳陷,悉众以攻歙,激水灌城。光武亲将救之,嚣走西城,世祖与来歙会于此。其水自城北注川,一水二川,盖嚣所堨以灌略阳也。川水西得白杨泉,又西得蒲谷水,又西得蒲谷西川,又西得龙尾溪水,与蒲谷水合,俱出南山飞清,北入川水。川水又西南得水洛口,水源东导陇山,西径水洛亭,西南流,又得犊奴水口,水出陇山,西径犊奴川,又西径水洛亭南,西北注之,乱流西南,径石门峡,谓之石门水,西南注略阳川。略阳川水又西北流入瓦亭水。瓦亭水又西南出显亲峡,石宕水注之。水出北山,山上有女娲祠。庖羲之后,有帝女蜗焉,与神农为三皇矣。其水南流,注瓦亭水,瓦亭水又西南径显亲县故城东南,汉封大鸿胪窦固为侯国。自石宕次得虾蟆溪水,次得金黑水,又得宜都溪水,咸出左右,参差相入。瓦亭水又东南合安夷川口,水源东出胡谷,西北流历夷水川,与东阳川水会,谓之取阳交。又西得何宕川水,又西得罗汉水,并自东北西南注夷水。夷水又西径显亲县南,西注瓦亭水。瓦亭水又东南,得大华谷水,又东南,得折里溪水,又东,得六谷水,皆出近溪湍峡,注瓦亭水。又东南出新阳峡,崖岫壁立,水出其间,谓之新阳崖水,又东南注于渭也。
又东过上邦县,渭水东历县北邽山之阴,流径固岭东北,东南流,兰渠川水出自北山,带佩众溪,南流注于渭。渭水东南与神涧水合。《开山图》所谓灵泉池也,俗名之为万石湾。渊深不测,实为灵异,先后漫游者,多罹其毙。渭水又东南,得历泉水,水北出历泉溪,东南流注于渭。渭水又东南,出桥亭西,又南得藉水口,水出西山,百涧声流,总成一川,东历当亭川,即当亭县洽也。左则当亭水,右则曾席水注之。又东与大弁川水合,水出西山,二源合注,东历大弁川,东南流注于藉水。藉水又东南流,与竹岭水合,水出南山竹岭,二源同泻,东北入籍水。藉水又东北径上邦县,左佩四水:东会占溪水,次东有大鲁谷水,次东得小鲁谷水,次东有杨反谷水,咸自北山流注藉水。藉水右带四水,竹岭东得乱石溪水,次东得木门谷水,次东得罗城溪水,次东得山谷水,皆导源南山,北流入籍水。藉水又东,黄瓜水注之,其水发源黄爪西谷,东流径黄爪县北,又东,清溪白水左右夹注。又东北,大旱谷水南出旱溪,历涧北流,泉溪委漾,同注黄瓜水。黄爪水又东北历赤谷,咸归于藉。藉水又东,得毛泉谷水,又东径上邦城南,得核泉水,并出南山,北流注于藉。藉水,即洋水也。北有濛水注焉,水出县西北邽山,翼带众流,积以成溪,东流南屈,径上邦县故城西,侧城南出。上邦,故邦戎国也。秦武公十年,伐邽县之,旧天水郡治。五城相接,北城中有湖水,有白龙出是湖,风雨随之,故汉武帝元鼎三年,改为天水郡。其乡居悉以板盖屋,《诗》所谓西戎板屋也,濛水又南注藉水。《山海经》曰:邽山,邽水出焉,而南流注于洋,谓是水也。藉水又东得阳谷水,又得宕谷水,并自南山北入于藉。藉水又东合段溪水,水出西南马门溪,东北流合藉水。藉水又东入于渭。渭水又历桥亭南,而径绵诸县东,与东亭水合,亦谓之为桥水也,清水又或为通称矣。水源东发小陇山,众川泻注,统成一水,西入东亭川.为东亭水,与小祗、大祗二水合。又西北得南神谷水,三川并出,东南差池泻注。又有埋蒲水,翼带二川,与延水并西南注东亭水。东亭水又西,右则叹沟水,次西得宕谷水,水出东南,二溪西北流,注东亭川。东亭川水,右则温谷水,出小陇山,又西,莎谷水出南山莎溪,西南注东亭川水。东亭川水又西得清水口,水导源东北陇山,二源俱发,西南出陇口,合成一水,西南流,历细野峡,径清池谷,又径清水县故城东。王莽之识睦县矣。其水西南合东亭川,自下亦通谓之清水矣。又径清水城南,又西与秦水合,水出东北大陇山秦谷,二源双导,历三泉合成一水,而历秦川。川有故秦亭,秦仲所封也。秦之为号,始自是矣。秦水西径降陇县故城南,又西南,自亥、松多二水出陇山,合而西南流,径降陇城北,又西南注秦水。秦水又西南,历陇川,径六盘口,过清水城西,南注清水。清水上下,咸谓之秦川。又西,羌水注焉。水北出羌谷,引纳众流,合以成溪。水星会,谓之小羌水。西南流,左则长谷水西南注之,右则东部水东南入焉。羌水又南入清水。清水又西南得绵诸水口,其水导源西北绵诸溪,东南有长思水,北出长思溪,南入绵诸水。又东南,历绵诸道故城北,东南入清水。清水东南注渭。渭水又东南合泾谷水,水出西南径谷之山,东北流,与横水合,水出东南横谷。西北径横水圹,又西北入泾谷水。乱流西北,出泾谷峡,又西北,轩辕谷水注之,水出南山轩辕溪。南安姚瞻以为黄帝生于天水,在上邦城东七十里轩辕谷。皇甫谧云生寿丘,丘在鲁东门北,未知孰是也。其水北流注泾谷水。泾谷水又西北,白城溪东北流,白娥泉水出其西,东注白城水。启城水又东北入泾谷水。泾谷水又东北,历董亭下。杨难当使兄子保宗镇董亭,即是亭也。其水东北流注于渭。《山海经》曰泾谷之山,泾水出焉,东南流注于渭,是也。渭水又东,伯阳谷水入焉。水出刑马之山伯阳谷。北流,白水出东南白水溪,西北注伯阳水。伯阳水又西北历谷,引控群流,北注渭水。渭水又东历大利,又东南流,苗谷水注之。水南出刑马山,北历平作,西北径苗谷,屈而东径伯阳城南,谓之伯阳川。盖李耳西入,往径所由,故山原畎谷,往往播其名焉。渭水东南流,众川泻浪,雁次鸣注,左则伯阳东溪水注之,次东得望松水,次东得毛六溪水,次东得皮周谷水,次东得黄杜东溪水,出北山,南入渭水。其右则明谷水,次东得丘谷水,次东得丘谷东溪水,次东有钳岩谷水,并出南山,东北注渭。渭水又东南,出石门,度小陇山,径南由县南,东与楚水合,世所谓长蛇水。水出汧县之数历山也。南流,径长蛇戍东。魏和平三年筑,徙诸流民以遏陇寇。楚水又南流,注于渭。阚駰以是水为汧水焉。渭水又东,汧汗二水入焉。余按诸地志,汧水出济县西北,阚駰《十三州志》与此同,复以汧水为龙鱼水,盖以其津流径通,而更摄其通称矣。渭水东入散关。《抱朴子·神仙传》曰:老子西出关,关令尹喜候气,知真人将有西游者,遇老子,强令之著书,耳不得已,为著《道》、《德》二经,谓之《老子》书也。有老子庙。干宝《搜神记》云:老子将西入关,关令尹喜好道之士,睹真人当西,乃要之途也。皇甫士安《高士传》云:老子为周柱下史,及周衰,乃以官隐,为周守藏室史,积八十余年,好无名接,而世莫知其真人也。至周景王十年,孔子年十七,遂适周见老聃。然幽王失道,平王东迁,关以捍移,人以职徒,尹喜候气,非此明矣。往径所由,兹焉或可。渭水又东径西武功北,俗以为散关城,非也。褚先生乃曰:武功,扶风西界小邑也。蜀口栈道近山,无他豪,易高者是也。渭水又与扞水合,水出周道谷北,径武都故道县之故城西。王莽更名曰善治也。故道县有怒特祠,《列异传》曰:武都故道县有怒特祠,云神本南山大梓也。昔秦文公二十七年代之,树疮随合,秦文公乃遣四十人持斧斫之,犹不断,疲士一人,伤足不能去,卧树下,闻鬼相与言曰:劳攻战乎?其一曰足为劳矣。又曰:秦公必持不休。答曰:其如我何?又曰:赤灰跋于子何如?乃默无言,卧者以告。令士皆赤衣,随所所以灰跋,树断,化为牛入水,故秦为立祠。其水又东北历大散关而入渭水也。渭水又东南,右合南山五溪水,夹涧流注之。又东过陈仓县西。
县有陈仓山,山上有陈宝鸡鸣祠。昔秦文公感伯阳之言,游猎于陈仓,遇之于此坂,得若石焉,其色如肝,归而宝祠之,故曰陈宝。其来也自东南,晖晖声若雷,野鸡皆鸣,故曰鸡鸣神也。《地理志》曰:有上公、明星、黄帝孙、舜妻盲冢祠。有羽阳宫,秦武王起。应劭曰:县氏陈山。姚睦曰:黄帝都陈言在此。荣氏《开山图注》曰:伏牺生成纪,徙治陈仓,非陈国所建也。魏明帝遣将军太原郝昭筑陈仓城成,诸葛亮围之。亮使昭乡人靳祥说之,不下。亮以数万攻昭千余人,以云梯、冲车、地道逼射昭,昭以火射连石拒之。亮不利而还。今研水对亮城,是与昭相御处也。陈仓水出于陈仓山下,东南流注于渭水。渭水又东与绥阳溪水合,其水上承斜水,水自斜谷分注绥阳溪,北届陈仓,入渭。故诸葛亮《与兄瑾书》曰:有绥阳小谷,虽山崖绝险,溪水纵横,难用行军。昔逻候往来,要道通入。今使前军斫治此道,以向陈仓,足以扳连贼势,使不得分兵东行者也。渭水又东径郁夷县故城南。《地理志》曰,有汧水祠。王莽更之曰郁平也。《东观汉记》曰:隗嚣围来歙于略阳。世祖诏曰:桃花水出,船槃皆至郁夷,陈仓分部而进者也。汧水入焉。水出汧县之蒲谷乡弦中谷,决为弦蒲薮。《尔雅》曰:水决之泽为汧。汧之为名,实兼斯举。水有二源,一水出县西山,世谓之小陇山。岩嶂高险,不通轨辙。故张衡《四愁诗》曰:我所恩兮在汉阳,欲往从之陇坂长。其水东北流,历涧,注以成渊,潭涨不测。出五色鱼,俗以为灵,而莫敢采捕。因谓是水为龙鱼水,自下亦通谓之龙鱼川。川水东径汧县故城北。《史记》,秦文公东猎汧田,因遂都其地,是也。又东历泽,乱流为一。右得白龙泉,泉径五尺,源穴奋通。沦漪四泄,东北流,注于汧。汧水又东,会一水,水发南山西侧。俗以此山为吴山,三峰霞举,叠秀云天,崩峦倾返,山顶相捍,望之恒有落势。《地理志》曰:吴山在县西,古文以为汧山也。《国语》所谓西虞矣。山下石穴,广四尺,高七尺,水溢石空,悬波侧注,漰渀震荡,发源成川,北流注于汧。自水会上下,咸谓之为龙鱼川。汧水又东南,径隃麋县故城南,王莽之扶亭也。昔郭歙耻王莽之征,而遁迹于斯。建武四年,光武封耿况为侯国矣。汧水东南历慈山,东南径郁夷县[北],平阳故城南。《史记》秦宁公二年,徙平阳。徐广曰:故郿之平阳亭也。城北有《汉邠州刺史赵融碑》,灵帝建安元年立。汧水又东流,注于渭。渭水之右,磻溪水注之。水出南山兹谷,乘高激流,注于溪中。溪中有泉,谓之兹泉,泉水潭积,自成渊渚,即《吕氏春秋》所谓太公钓兹泉也。今人谓之丸谷,石壁深高,幽隍邃密,林障秀阻,人迹罕交。东南隅有一石室,盖太公所居也。水次平石钓处,即太公垂钓之所也。其投竿跽饵,两膝遗迹犹存,是有磻溪之称也。其水清泠神异,北流十二里,注于渭,北去维堆城七十里。渭水又东径积石原,即北原也。青龙二年,诸葛亮出斜谷,司马懿屯渭南。雍州刺史郭淮策亮必争北原而屯,遂先据之。亮至,果不得上。渭水又东径五丈原北。《魏氏春秋》曰:诸葛亮据渭水南原,司马懿谓诸将曰:亮若出武功,依山东转者,是其勇也。若西上五丈原,诸君无事矣。亮果屯此原,与懿相御。渭水又东径郿县故城南。《地理志》曰:右辅都尉治。《魏氏春秋》,诸葛亮寇郿,司马懿据郿拒亮,即此县也。渭水又东径郿坞南。《汉献帝传》曰:董卓发卒筑郿坞,高与长安城等,积谷为三十年储,自云事成,雄据天下,不成,守此足以毕老,其愚如此。


卷十八  渭水 
又东过武功县北,渭水于县,斜水自南来注之。水出县西南衙岭山,北历斜谷,径五丈原东。诸葛亮《与步骘书》曰:仆前军在五丈原,原在武功西十里余。水出武功县,故亦谓之武功水也。是以诸葛亮《表》云:臣遣虎步监孟琰,据武功水东。司马懿因水长,攻琰营,臣作竹桥,越水射之。桥成驰去。其水北流注于渭。《地理志》曰:斜水出衙岭北,至郿注渭。渭水又东,径马冢北。诸葛亮《与步骘书》曰:马冢在武功东十余里,有高势,攻之不便,是以留耳。渭水又径武功县故城北,王莽之新光也。《地理志》曰:县有太一山,古文以为终南。杜预以为中南也。亦曰:太白山在武功县南,去长安二百里,不知其高几何。俗云:武功太白,去天三百。山下军行,不得鼓角,鼓角则疾风雨至。杜彦达曰:太白山,南连武功山,于诸山最为秀杰,冬夏积雪,望之皓然。山上有谷春祠。春,栎阳人,成帝时病死而尸不寒,后忽出栎南门及光门上,而入太白山。民为立祠于山岭,春秋来祠,中上宿焉。山下有太白祠,民所祀也。刘曜之世,是山崩,长安人刘终于崩所得白玉,方一尺,有文字,曰:皇亡皇亡败赵昌,井水竭,构五梁,咢西小衰困嚣丧。呜呼!呜呼!赤牛奋靷其尽乎!时群官毕贺。中书监刘均进曰:此国灭之象,其可贺乎?终如言矣。渭水又东,温泉水注之。水出太一山,其水沸涌如汤。杜彦达曰:可治百病,世清则疾愈,世浊则无验。其水下合溪流,北注十三里,入渭。渭水又东,径漦县故城南。旧邰城也。后稷之封邑矣,《诗》所谓即有邰家室也。城东北有姜漦祠,城西南百步有稷祠,郿之漦亭也。王少林之为漦县也,路径此亭。亭长曰:亭凶杀人。少林曰:仁胜凶邪,何鬼敢忤?遂宿。夜中,闻女子称冤之声。少林曰:可前来理。女子曰:无衣,不敢进。少林投衣与之。女子前诉曰:妾夫为涪令,之官过宿此亭,为亭长所杀。少林曰:当为理寝冤,勿复害良善也。因解衣于地,忽然不见。明告亭长,遂服其事,亭遂清安。渭水又东径雍县南,雍水注之。水出雍山,东南流,历中牢溪,世谓之中牢水,亦曰冰井水,南流径胡城东。俗名也。盖秦惠公之故居,所谓祈年宫也。孝公又谓之为橐泉宫,按《地理志》曰在雍。崔駰曰:穆公冢在橐泉宫祈年观下,《皇览》疥言是矣。刘向曰:穆公葬无丘垄处也。《史记》曰:穆公之卒,从死者百七十七人,良臣子车氏奄息,仲行、鍼虎,亦在从死之中。秦人哀之,为赋《黄鸟》焉。余谓崔駰及《皇览》谬志也。惠公、孝公并是穆公之后,继世之君矣,子孙无由起宫于祖宗之坟陵矣。以是推之,知二证之非实也。雍水又东,左会左阳水,世名之西水。水北出左阳溪,南流径岐州城西。魏置岐州刺史治。左阳水又南流,注于雍水,雍水又与东水合,俗名也。北出河桃谷,南流,右会南源,世谓之返眼泉。乱流南,径岐州城东,而南合雍水,州居二水之中,南则两川之交会也。世亦名之为淬空水。东流,邓公泉注之,水出邓艾祠北,故名曰邓公泉。数源俱发于雍县故城南。县故秦德公所居也。《晋书地道记》以为西虢地也。《汉书·地理志》以为西虢县。《太康地记》曰:虢叔之国矣,有虢宫,平王东迁,叔自此之上阳为南虢矣。雍有五峙祠,以上祠祀五帝。昔秦文公田于汧、渭之间,梦黄蛇自天下属地,其口止于鄜衍,以为上帝之神,于是作鄜畤,祀白帝焉。秦宣公作密畤于渭南,祀青帝焉。灵公又于吴阳作上畤,祀黄帝,作下畤,祀炎帝焉。献公作畦畤于栎阳而祀白帝。汉高帝问曰:天有五帝,今四何也?博士莫知其故。帝曰:我知之矣,待我而五。遂立北畤,祀黑帝焉。应劭曰:四面积高曰雍。阚駰曰:宜为神明之隩,故立群祠焉。又有凤台、凤女祠。秦穆公时,有箫史者,善吹箫,能致白鹄、孔雀。穆公玄弄玉好之,公为作凤台以居之。积数十年,一旦随凤去,云雍宫世有箫管之声焉。今台倾祠毁,不复然矣。邓泉东流注于雍,自下虽会他津,犹得通称。故《禹贡》有雍、沮会同之文矣。雍水又东径召亭南,世谓之树亭川,盖召、树声相近,误耳。亭故召公之采邑也。京相璠曰:亭在周城南五十里。《后汉·郡国志》曰:郿县有召亭。谓此也。雍水又东南流,与横水合。水出杜阳山。其水南流,谓之杜阳川。东南流。左会漆水,水出杜阳县之漆溪,谓之漆渠。故徐广曰:漆水出杜阳之岐山者,是也。漆渠水南流,大峦水注之。水出西北大道川,东南流入漆,即故岐水也。《淮南子》曰,岐水出石桥山,东南流。相如《封禅书》曰:收龟于岐。《汉书音义》曰:岐,水名也,谓斯水矣。二川并逝,俱为一水,南与横水合,自下通得岐水之目,俗谓之小横水,亦或名之米流川。径岐山西,又屈径周城南。城在岐山之阳而近西,所谓居岐之阳也,非直因山致名,亦指水取称矣。又历周原下,北则中水乡成周聚,故曰有周也。水北即岐山矣。昔秦盗食穆公马处也。岐水又东,径姜氏城南为姜水。按《世本》,炎帝姜姓。《帝王世纪》曰:炎帝,神农氏,姜姓。母女登,游华阳,感神而生炎帝,长于姜水,是其地也。东注雍水。雍水又南,径美阳县之中亭川,合武水。水发杜阳县大岭侧,东西三百步,南北二百步,世谓之赤泥岘,沿波历涧,俗名大横水也。疑即杜水矣。其水东南流,东径杜阳县故城,世谓之故县川。又故虢县有杜阳山,山北有杜阳谷,有地穴,北入,亦不知所极,在天柱山南。故县取名焉;亦指是水而摄目矣,即王莽之通杜也。故《地理志》曰:县有杜水。杜水又东,二坑水注之;水有二源,一水出西北,与渎水合,而东历五将山,又合乡谷水。水出乡溪,东南流入杜水,谓之乡谷川。又南,莫水注之。水出好畤县梁山大岭东,南径梁山宫西,故《地理志》曰:好畤有梁山宫,秦始皇起。水东有好畤县故城,王莽之好邑也。世租建武二年,封建成大将军耿弇为侯国。又南径美阳县之中亭川,注雍水,谓之中亭水。雍水又南径美阳县西。章和二年,更封彰侯耿秉为侯国。其水又南流注于渭。渭水又东,洛谷之水,出其南山洛谷,北流径长城西。魏甘露二年,蜀遣姜维出洛谷,围长城,即斯地也。
又东,芒水从南来流注之。
芒水出南山芒谷,北流径玉女房。水侧山际有石室,世谓之玉女房。芒水又北径盩厔县之竹圃中,分为二水。汉冲帝诏曰:翟义作乱于东,霍鸿负倚盩厔芒竹,即此也。其水分为二流,一水东北为枝流,一水北流注于渭也。


卷十九  渭水 
又东过槐里县南,又东,涝水从南来注之。
渭水径县之故城南。《汉书集注》,李奇谓之小槐里。县之西城也。又东与芒水枝流合,水受芒水于竹圃。东北流,又屈而北入于渭。渭水又东北径黄山宫南,即《地理志》所谓县有黄山宫,惠帝二年起者也。《东方朔传》曰:武帝微行,西至黄山宫,故世谓之游城也。就水注之。水出南山就谷,北径大陵西。世谓之老子陵。昔李耳为周柱史,以世衰入戎,于此有冢,事非经证。然庄周著书云:老聃死,秦失吊之,三号而出。是非不死之言。人禀五行之精气,阴阳有终变,亦无不化之理。以是推之,或复如传。古人许以传疑。故两存耳。就水历竹圃,北与黑水合。水上承三泉,就水之右,三泉奇发,言归一渎,北流左注就水,就水又北流注于渭。渭水又东合田溪水,水出南山田谷,北流径长杨宫西,又北径盩厔县故城西。又东北与一水合,水上承盩厔县南源,北径其县东。又北径思乡城西,又北注田溪。田溪水又北流注于渭水也。县北有蒙笼渠,上承渭水于郿县东,径武功县为成林渠。东径县北,亦曰灵轵渠,《河渠书》以为引堵水。徐广曰:一作诸川,是也。渭水又东径槐里县故城南。县,古犬丘邑也,周懿王都之。秦以为废丘,亦曰舒丘。中平元年,灵帝封左中郎将皇甫嵩为侯国。县南对渭水,北背通渠。《史记·秦本纪》云:秦武王三年,渭水赤三日。秦昭王三十四年,渭水又大赤三日。《洪范五行传》云:赤者,火色也,水尽赤,以火沴水也。渭水,秦大川也,阴阳乱,秦用严刑败乱之象。后项羽入秦,封司马欣为塞王,都栎阳;董翳为翟王,都高奴;章邯为雍王,都废丘,为三秦。汉祖北定三秦,引水灌城,遂灭章邯。三年,改曰槐里。王莽更名槐治也,世谓之为大槐里。晋太康中,始平郡治也。其城递带防陆,旧渠尚存,即《汉书》所谓槐里环堤者也。东有漏水,出南山赤谷。东北流径长杨宫东,宫有长杨树,因以为名,漏水又北历苇圃西,亦谓之仙泽。又北径望仙宫。又东北,耿谷水注之,水发南山耿谷,北流与柳泉合。东北径五柞宫西。长杨、五柞二宫,相去八里,并以树名宫,亦犹陶氏以五柳立称。故张晏曰:宫有五柞树。在盩厔县西。其水北径仙泽东,又北径望仙宫东,又北与赤水会,又北径思乡城东,又北注渭水。渭水又东合甘水,水出南山甘谷,北径秦文王萯阳宫西,又北径五柞宫东,又北径甘亭西。在水东鄠县。昔夏启伐有扈,作誓于是亭。故马融曰:甘有扈南郊地名也。甘水又东得涝水口。水出南山涝谷,北径汉宜春观东,又北径鄠县故城西。涝水际城北出,合美陂水。水出宜春观北。东北流注涝水。涝水北注甘水而乱流入于渭。即上林故地也。《东方朔传》称:武帝建元中微行,北至池阳,西至黄山,南猎长杨,东游宜春,夜漏十刻乃出,与侍中、常侍武骑、待诏及陇西、北地良家子能骑射者,期诸殿下,故有期门之号。且明入山下,驰射鹿豕狐兔,手格熊罴。上大欢乐之。上乃使大中大夫虞邱寿王与待诏能用算者,举籍;阿城以南,盩厔以东,宜春以西,提封顷亩及其贾直,属之南山以为上林苑。东方朔谏,秦起阿房而天下乱,因陈泰阶六符之事。上乃拜大中大夫、给事中,赐黄金百斤。卒起上林苑。故相如请为天子游猎之赋,称乌有先生、亡是公而奏《上林》也。
又东,丰水从南来注之。
丰水出丰溪西,北流分为二水,一水东北流为枝津,一水西北流又北交,水自东入焉。又北,昆明池水注之,又北径灵台西,又北至石墩,注于渭。《地说》云:渭水又东,与丰水会于短阴山内。水会,无他高山异峦,所有惟原阜石激而已。水上旧有便门桥,与便门对直,武帝建元三年造。张昌曰:桥在长安西北,茂陵东。如淳曰,去长安四十里。渭水又径太公庙北。庙前有太公碑,文字虢缺,今无可寻。渭水又东北与鄗水合,水上承鄗池于昆明池北。周武王之所都也。故《诗》云:考卜维王,宅是鄗京,维龟正之,武王成之。自汉武帝穿昆明池于是地,基构沦虢,今无可究。《春秋后传》曰:使者郑容入柏谷关,至平舒置,见华山有素车白马,问郑容安之?答曰:之咸阳。车上人曰:吾华山君使,愿托书致鄗池君。子之咸阳,过鄗池,见大梓下有文石,取以款列梓,当有应者,以书与之。勿妄发,致之得所欲。郑容行至鄗池,见一梓下,果有文石,取以款梓。应曰:诺。郑容如睡,觉而见宫阙,若王者之居焉。谒者出,受书,入,有顷,闻语声言:祖为死。神道茫昧,理难辨测,故无以精其幽致矣。鄗水又北流,西北注,与盩厔池合。水出鄗池西,而北流入于鄗。《毛诗》云:鄗,流貌也。而世传以为水名矣。郑玄曰:丰鄗之间,水北流也。鄗水北径清泠台西,又径磁石门西。门在阿房前,悉以磁石为之,故专其目。令四夷朝者,有隐甲怀刃入门而胁之以示神,故亦曰胡门也。鄗水又北,注于渭。渭水北有杜邮亭,去咸阳十七里、今名孝里亭,中有白起祠。嗟乎!有制胜之功,惭尹、商之仁,是地即其伏剑处也。渭水又东北径渭城南,文颖以为故咸阳矣。秦孝公之所居高宫也。献公都栎阳,天雨金。周太史儋见献公曰:周故与秦国合而别,别五百岁复合,合七十岁而霸王出。至孝公作咸阳,筑冀阙,而徙都之。故《西京赋》曰:秦里其朔,实为咸阳。太史公曰:长安,故咸阳也。汉高帝更名新城。武帝元鼎三年别为渭城,在长安西北,渭水之阳。王莽之京城也。始隶扶风,后并长安。南有鄗水注之,水上承皇子陂于樊川,其地即杜之樊乡也。汉祖至栎阳,以将军樊哙灌废丘,最赐邑于此乡也。其水西北流径杜县之杜京西,西北流径杜伯冢南。杜伯与其友左儒仕宣王,儒无罪见害,杜伯死之,终能报恨于宣王。故成公子安五言诗曰:谁谓鬼无知,杜伯射宣王。鄗水又西北径下杜城,即杜伯国也。鄗水又西北,枝合故渠,渠有二流,上承交水。合于高阳原,而北径河池陂东,而北注鄗水。鄗水又北与昆明故池会,又北径秦通六基东。又北径鄗水陂东,又北得陂水。水上承其陂,东北流入于鄗水。鄗水又北径长安城西,与昆明池水合。水上承池于昆明台,故王仲都所屠也。桓谭《新论》称,元帝被病,广求方士。汉中送道士王仲都,诏问所能。对曰:能忍寒暑。乃以隆冬盛寒日,令袒,载驷马,于上林昆明池上环冰而驰。御者厚衣狐裘寒战,而仲都独无变色,卧于池台上,曛然自若。夏大暑日,使曝坐,环以十炉火。不言热,又身不汗。池水北径鄗京东,秦阿房宫西。《史记》曰:秦始皇三十五年,以咸阳人多,先王之宫小,乃作朝宫于渭南,亦曰阿城也。始皇先作前殿阿房,可坐万人,下可建五丈旗,周驰为阁道,自殿直抵南山。表山巅为阙。为复道自阿房度渭,属之咸阳,象天极阁道,绝汉抵营室也。《关中记》曰:阿房殿在长安西南二十里。殿东西千步,南北三百步,庭中受十万人。其水又屈而径其北.东北流注堨水陂。陂水北出,径汉武帝建章宫东,于凤阙南,东注泬水。泬水又北径凤阙东。《三辅黄图》曰:建章宫,汉武帝造,周二十余里,千门万户。其东凤阙,高七丈五尺,俗言贞女楼,非也。《汉武帝故事》云:阙高二十丈。《关中记》曰:建章宫圆阙,临北道,有金凤在阙上,高丈余,故号凤阙也。故繁钦《建章凤阙赋叙》曰:秦汉规模,廓然毁泯,惟建章凤阙,岿然独存。虽非象魏之制,亦一代之巨观也。泬水又北,分为二水,一水东北流,一水北径神明台东。《傅子·宫室》曰:上于建章中作神明台、并于楼,咸高五十余丈,皆作悬阁,辇道相属焉。《三辅黄图》曰:神明台在建章宫中,上有九室,今人谓之九子台,即实非也。泬水又径渐台东。《汉武帝故事》曰:建章宫北有太液池,池中有渐台,高三十丈。渐,浸也,为池水所渐。一说,星名也。南有璧门三层,高三十余丈、中殿十二间,阶陛咸以玉为之。铸铜凤,高五丈,饰以黄金,楼屋上。椽首,薄以玉璧,因曰璧玉门也。泬水又北流注渭,亦谓是水为泬水也。故吕忱曰:泬水出杜陵县。《汉书音义》曰:泬,水声,而非水也。亦曰高都水。前汉之末,王氏五侯大治池宅,引泬水入长安城,故百姓歌之曰:五侯初起,曲阳最怒。坏决高都,竟连五杜。土山渐台,像西白虎。即是水也。
又东过长安县北,渭水东分为二水。《广雅》曰:水自渭出为泶,其犹河之有雍也。此渎东北流,径《魏雍州刺史郭淮碑》南。又东南合一水,径两石人北。秦始皇造桥,铁镦重不能胜,故刻石作力士孟贲等像以祭之,镦乃可移动也。又东径阳侯祠北,涨辄祠之。此神能为大波,故配食河伯也。后人以为邓艾祠。悲哉!谗胜道消,专忠受害矣。此水又东注渭水。水上有梁,谓之渭桥,秦制也,亦曰便门桥。秦始皇作离宫于渭水南北,以象天宫。故《三辅黄图》曰:渭水贯都以象天汉,横桥南度以法牵牛。南有长乐宫,北有咸阳宫,欲通二宫之间,故造此桥,广六丈,南北三百八十步,六十八间,七百五十柱,百二十二梁,桥之南北有堤激,立石柱,柱南,京兆主之,柱北,冯翊主之,有令丞,各领徒千五百人,桥之北首,垒石水中,故谓之石柱桥也。旧有忖留神像。此神尝与鲁班语,班令其人出。忖留曰:我貌很丑,卿善图物容,我不能出。班于是拱手与言曰:出头见我。忖留乃出首,班于是以脚画地,忖留觉之,便还没水,故置其像于水,惟背以上立水上。后董卓入关,遂焚此桥。魏武帝更修之,桥广三丈六尺。忖留之像,曹公乘马见之,惊,又命下之。《燕丹子》曰:燕太子丹质于秦,秦王遇之无礼,乃求归。秦王为机发之桥,欲以陷丹,丹过之,桥不为发。又一说,交龙扶舆而机不发。但言今不知其故处也。渭水又东与泬水枝津合,水上承泬水,东北流径邓艾祠南,又东分为二水,一水东入逍遥园,注藕池。池中有台观,莲荷被浦,秀实可玩。其一水,北流注于渭。渭水又东径长安城北。汉惠帝元年筑,六年成,即咸阳也。秦离宫无城,故城之。王莽更名常安。十二门,东出北头第一门,本名宣平门,王莽更名春王门正月亭,一曰东都门,其郭门亦曰东都门,即逢萌挂冠处也。第二门本名清明门,一曰凯门,王莽更名宣德门布恩亭。内有藉田仓,亦曰藉田门。第三门本名霸城门,王莽更名仁寿门无疆亭。民见门色青,又名青城门,或曰青绮门,亦曰青门。门外旧出好瓜。昔广陵人邵平为秦东陵侯,秦破,为布衣,种瓜此门,瓜美,故世谓之东陵瓜。是以阮籍《咏怀诗》云:昔闻东陵瓜,近在青门外,连畛拒阡陌,子母相钩带。指谓此门也。南出东头第一门,本名覆盎门,王莽更名永清门长茂亭。其南有下杜城。应劭曰:故杜陵之下聚落也,故曰下杜门。又曰端门,北对长乐宫。第二门本名安门,亦曰鼎路门,王莽更名光礼门显乐亭。北对武库。第三门本名平门,又曰便门,王莽更名信平门诚正亭。一曰西安门,北对未央宫。西出南头第一门,本名章门,王莽更名万秋门亿年亭,亦曰光华门也。第二门本名直门,王莽更名直道门端路亭,故龙楼门也。张晏曰:门楼有铜龙。《三辅黄图》曰:长安西出第二门即此门也。第三门本名西城门,亦曰雍门,王莽更名章义门著义亭。其水北入,有函里,民名曰函里门,亦曰突门。北出西头第一门,本名横门,王莽更名霸都门左幽亭。如淳曰:横音光,故曰光门。其外郭有都门,有棘门。徐广曰:棘门在渭北。孟康曰:在长安北,秦时宫门也。如淳曰:《三辅黄图》曰棘门,在横门外。按《汉书》,徐厉军于此,备匈奴。又有通门、亥门也。第二门,本名厨门,又曰朝门,王莽更名建子门广世亭,一曰高门。苏林曰:高门,长安城北门也。其内有长安厨官在东,故名曰厨门也。如淳曰:今名广门也。第三门本名杜门,亦曰利城门,王莽更名进和门临水亭。其外有客舍,故民曰客舍门,又曰洛门也。凡此诸门,皆通逵九达,三途洞开,隐以金椎,周以林木,左出右入,为往来之径。行280 者升降,有上下之别。汉成帝之为太子,元帝尝急召之。太子出龙楼门,不敢绝驰道,西至直城门,方乃得度。上怪迟,问其故,以状对。上悦,乃著令,令太子得绝驰道也。渭水东合昆明故渠,渠上承昆明池东口,东径河池陂北,亦曰女观陂。又东合泬水,亦曰漕渠。又东径长安县南,东径明堂南。旧引水为辟雍处,在鼎路门东南七里。其制上圆下方,九宫十二堂,四向五室。堂北三百步,有灵台,是汉平帝元始四年立。渠南有汉故圜丘,成帝建始二年罢雍五峙,始祀皇天上帝于长安南郊。应劭曰:天郊在长安南。即此也。故渠之北,有白亭、博望苑,汉武帝为太子立,使通宾客,从所好也。太子巫蛊事发,斫杜门东出。史良娣死,葬于苑北,宣帝以为戾园,以倡优千人乐思后园庙,故亦曰千乡。故渠又东而北屈,径青门外,与泬水枝渠会,渠上承泬水于章门西。飞渠引水入城。东为仓池,池在未央宫西。池中有渐台,汉兵起,王莽死于此台。又东径未央宫北。高祖在关东,令萧何成未央宫。何斩龙首山而营之。山长六十余里,头临渭水,尾达樊川。头高二十丈,尾渐下高五六丈,土色赤而坚,云昔有黑龙从南山出,饮渭水,其行道因山成迹,山即基,阙不假筑,高出长安城。北有玄武阙,即北阙也。东有苍龙阙,阙内有阊阖、止车诸门。未央殿东有宣室、玉堂、麒麟、含章、白虎、凤皇、朱雀、鹓鸾、昭阳诸殿,天禄、石渠、麒麟三阁。未央宫北即桂宫也,周十余里,内有明光殿、走狗台、柏梁台,旧乘复道,用相径通。故张衡《西京赋》曰:钩陈之外,阁道穹隆属长乐与明光,径北通于桂宫。故渠出二宫之间,谓之明渠也。又东历武库北。旧樗里子葬于此,樗里子名疾,秦惠王异母弟也,滑稽多智,秦人号曰智囊,葬于昭王庙西,渭南阴乡樗里,故俗谓之樗里子。云:我百岁后,是有天于之宫夹我墓。疾以昭王七年卒,葬于渭南章台东。至汉,长乐宫在其东,未央宫在其西,武库直其墓。秦人喭曰力则任鄙,智则樗里是也。明渠又东径汉高祖长乐宫北,本秦之长乐宫也,周二十里,殿前列铜人,殿西有长信、长秋、永寿、永昌诸殿。殿之东北有池,池北有层台,俗谓是池为酒池,非也。故渠北有楼,竖汉京兆尹司马文预碑。故渠又东出城,分为二渠,即《汉书》所谓王渠者也。苏林曰:王渠,官渠也,犹今御沟矣。晋灼曰:渠名也,在城东覆盎门外。一水径杨桥下,即青门桥也。侧城北,径邓艾祠西,而北注渭,今无水。其一水,右入昆明故渠,东径奉明县广城乡之廉明苑南。史皇孙及工夫人葬于郭北,宣帝迁苑南,卜以为悼园,益园民千六百家,立奉明县以奉二园。园在东部门。昌邑王贺自霸御法驾;郎中令龚遂骏乘,至广明东都门是也。故渠东北径汉太尉夏侯婴冢西。葬日,柩马悲鸣,轻车罔进,下得石椁,铭云:于嗟滕公居此室!故遂葬焉。冢在城东八里,饮马桥南四里,故时人谓之马冢。故渠又北,分为二渠,一水东径虎圈南,而东入霸,一水北合渭,今无水。
又东过霸陵县北,霸水从县西北流注之。
霸者,水上地名也。古曰滋水矣,秦穆公霸世,更名滋水为霸水,以显霸功。水出蓝田县蓝田谷,所谓多玉者也。西北有铜谷水,次东有辋谷水,二水合而西注,又西流入埿水。埿水又西径峣关北,历峣柳城。东西有二城,魏置青埿军于城内,世亦谓之青埿城也。秦二世三年,汉祖入自武关,攻秦,赵高遣将距于峣关者也。《土地记》曰:蓝田县南有峣关,地名晓柳,道通荆州。《晋地道记》曰:关当282 上洛县西北。埿水又西北流入霸。霸水又北历蓝田川,径蓝田县东。《竹书纪年》,梁惠成王三年,秦子向命力蓝君,盖子向之故邑也。川有汉临江王荣冢。景帝以罪征之,将行,祖于江陵北门,车轴折,父老泣曰:吾王不反矣!荣至,中尉郅都急切责王,王年少,恐而自杀,葬于是川。有燕数万,衔土置冢上,百姓矜之。霸水又左合浐水,历白鹿原东,即霸川之西故芷阳矣。《史记》,秦襄王葬芷阳者是也,谓之霸上。汉文帝葬其上,谓之霸陵。上有四出道以泻水,在长安东南三十里。故王仲宣赋诗云:南登霸陵岸,回首望长安。汉文帝尝欲从霸陵上,西驰下峻坂。袁盎揽辔于此处。上曰:将军怯也?盎曰:臣闻千金之子,坐不垂堂,百金之子,立不倚衡。圣人不乘危。今驰不测,如马惊车败,奈高庙何?上乃止。霸水又北,长水注之。水出杜县白鹿原,其水西北流,渭之荆溪。又西北,左合狗枷川水,水有二源。西川上承磈山之斫槃谷,次东有苦谷,二水合而东北流,径风凉原西。《关中图》曰:丽山之西,川中有阜,名曰风凉原,在磈山之阴,雍州之福地。即是原也。其水傍溪北注,原上有汉武帝祠。其水右合东川,水出南山之石门谷,次东有孟谷,次东有大谷,次东有雀谷,次东有土门谷,五水北出谷,西北历风凉原东,又北与西川会,原为二水之会,乱流北径宣帝许后陵东北,去杜陵十里。斯川于是有狗枷之名。川东亦曰白鹿原也。上有狗枷堡,《三秦记》曰:丽山西有白鹿原,原上有狗枷堡,秦襄公时有大狗来下,有贼则狗吠之,一堡无患,故川得厥目焉。川水又北径杜陵东。元帝初元元年,葬宣帝社陵,北去长安五十里。陵之西北,有杜县故城。秦武公十一年县之。汉宣帝元康元年,以社东原上为初陵,更名杜县为杜陵。王莽之饶安也。其水又北注荆溪,荆溪水又北径霸县,又有温泉入焉。水发自原下,入荆溪水,乱流注于霸,俗谓之浐水,非也。《史记·封禅书》,文帝出长门,《注》云:在霸陵县。有故亭,即《郡国志》所谓长门亭也。《史记》曰:霸、浐、长水也,虽不在祠典,以近咸阳秦、汉都,泾、渭、长水,尽得比大川之礼。昔文帝居霸陵,北临厕,指新丰路示慎夫人曰:此走邯郸道也。因使慎夫人鼓瑟,上自倚瑟而歌,凄怆悲怀,顾谓群巨曰:以北山石为椁,用纻絮斫陈漆其间,岂可动哉?释之曰:使其中有可欲,虽锢南山,犹有隙;使无可欲,虽无石椁,又何戚焉?文帝曰:善!拜廷尉。韦昭曰:高岸夹水为厕。今斯原夹二水也。霸水又北会两川,又北,故渠右出焉。霸水又北径王莽九庙南。王莽地皇元年,博征天下工匠,坏撤西苑、建章诸宫馆十余所,取材瓦以起九庙,算及吏民,以义入钱谷,助成九庙。庙殿皆重屋。太初祖庙,东西南北各四十丈,高十七丈,余庙半之。为铜薄栌,饰以金银雕文,穷极百工之巧,褫高增下,功费数百巨万,卒死者万数。霸水又北径帜道,在长安县东十三里。王莽九庙在其南。汉世,有白蛾群飞,自东都门过枳道,吕后拔除于霸上,还见仓狗戟胁于斯道也。水上有桥,谓之霸桥。地皇三年,霸桥木灾,自东起,卒数千以水泛沃救不灭,晨焚夕尽。王莽恶之,下书曰:甲午火桥,乙未,立春之日也。予以神明圣祖黄虞遗统受命,至于地皇四年为十五年,正以三年终冬,绝灭霸驳之桥,欲以兴成新室,统一长存之道。其名霸桥,为长存桥。霸水又北,左纳漕渠,绝霸右出焉。东径霸城北,又东径子楚陵北。皇甫谧曰:秦庄王葬于芷阳之丽山。京兆东南霸陵山,刘向曰:庄王大其名,立坟者也。《战国策》曰:庄王字异人,更名子楚,故世人犹以子楚名陵。又东径新丰县,右会故渠。渠上承霸水,东北径霸城县故城南。汉文帝之霸陵县也,王莽更之曰水章。魏明帝景初元年,徙长安金狄,重不可致,因留霸城南。人有见蓟子训与父老共摩铜人曰:正见铸此时,计尔日已近五百年矣。故渠又东北径刘更始冢西。更始三年,为赤眉所杀,故侍中刘恭,夜往,取而埋之。光武使司徒邓禹收葬于霸陵县。更始尚书仆射、行大将军事鲍永,持节安集河东,闻更始死,归世祖,累迁司隶校尉。行县,经更始墓,遂下拜,哭尽哀而去。帝问公卿,大中大夫张湛曰:仁不遗旧,忠不忘君,行之高者。帝乃释。又东北径新丰县,右合漕渠,汉大司农郑当时所开也。以渭水难漕,命齐水工徐伯发卒穿渠引渭。其渠自昆明池南傍山原,东至于河,且田且漕,大以为便。今无水。霸水又北径秦虎圈东。《列士传》曰:秦昭王会魏王,魏王不行,使朱亥奉壁一双。秦王大怒,置朱亥虎圈中。亥瞋目视虎,毗裂,血出溅虎,虎不敢动,即是处也。霸水又北,入于渭水。渭水又东,会成国故渠。渠,魏尚书左仆射卫臻征蜀所开也。号成国渠,引以浇田。其渎上承汧水于陈仓东,东径郿及武功、槐里县北。渠左有安定梁严冢,碑碣尚存。又东径汉武帝茂陵南,故槐里之茂乡也。应劭曰:帝自为陵,在长安西北八十余里。《汉武帝故事》曰:帝崩后,见形谓陵令薛平曰:吾虽失势,犹为汝君,奈何令吏卒上吾陵磨刀剑乎?自今以后可禁之。平顿首谢,因不见。推问陵旁,果有方石,可以为砺,吏卒常盗磨刀剑,霍光欲斩之。张安世曰:神道茫昧,不宜为法。乃止。故阮公《咏怀诗》曰:失势在须臾,带剑上吾丘。陵之西而北一里,即李夫人冢。冢形三成,世谓之英陵。夫人兄延年知音,尤善歌舞,帝爱之。每为新声变曲,闻者莫不感动。常侍上,起舞,歌曰:北方有佳人,绝世而独立。一顾倾人城,再顾倾人国。宁不知倾城复倾国,佳人难再得!上曰:世岂有此人乎?平阳主曰:延年女弟。上召见之,妖丽,善歌舞,得幸,早卒。上悯念之,以后礼葬,悲思不已,赋诗悼伤。故渠又东径茂陵县故城南。武帝建元二年置。《地理志》曰:宣帝县焉,王莽之宣成也。故渠又东径龙泉北,今人谓之温泉,非也。渠北故坂北即龙渊庙。如淳曰:《三辅黄图》有龙渊宫,今长安城西有其庙处,盖宫之遗也。故渠又东径姜原北。渠北有汉昭帝陵,东南去长安七十里。又东径平陵县故城南。《地理志》曰:昭帝置。王莽之广利也。故渠之南有窦氏泉,北有徘徊庙。又东径汉大将军魏其侯窦婴冢南,又东径成帝延陵南,陵之东北五里,即平帝康陵坂也。故渠又东,径渭陵南。元帝永光四年,以渭城寿陵亭原上为初陵,诏不立县邑。又东径哀帝义陵南。又东径惠帝安陵南,陵北有安陵县故城。《地理志》曰:惠帝置,王莽之嘉平也。渠侧有杜邮亭。又东,径渭城北。《地理志》曰:县有兰池宫。秦始皇微行,逢盗于兰池,今不知所在。又东径长陵南,亦曰长山也。秦名天子冢曰山,汉曰陵,故通曰山陵矣。《风俗通》曰:陵者,天生自然者也,今王公坟垅称陵。《春秋左传》曰:南陵,夏后皋之墓也。《春秋说题辞》曰:丘者,墓也,冢者,种也,种墓也。罗倚于山,分卑尊之名者也。故渠又东径汉丞相周勃冢南,冢北有亚夫冢。故渠东南谓之周氏曲,又东南径汉景帝阳陵南,又东南注于渭,今无水。渭水又东,径霸城县北,与高陵分水。水南有定陶恭王庙、傅太后陵。元帝崩,傅昭仪随王归国,称定陶太后。286 后十年,恭王薨,子代为王。征为太子,太子即帝位,立恭王寝庙于京师,比宣帝父悼皇故事。元寿元年,傅后崩,合葬渭陵。潘岳《关中记》,汉帝后同茔则为合葬,不共陵也,诸侯皆如之。恭王庙在霸城西北,庙西北即傅太后陵。不与元帝同茔,渭陵非谓元帝陵也,盖在渭水之南,故曰渭陵也。陵与元帝齐者,谓同十二丈也。王莽奏毁傅太后冢,冢崩,压杀数百人。开棺,臭闻数里。公卿在位,皆阿莽旨,入钱帛,遣子弟,及诸生四夷凡十余万人,操持作具,助将作掘傅后冢,二旬皆平,周棘其处,以为世戒。今其处积土犹高,世谓之增墀,又亦谓之增阜,俗亦谓之成帝初陵处,所未详也。渭水又径平阿侯王谭墓北,冢次有碑。左则泾水注之。渭水又东,径鄣县西,盖陇西郡之鄣徙也。渭水又东,得白渠技口,又东与五丈渠合。水出云阳县石门山,谓之清水。东南流,径黄嵌山西,又南入祋祤县,历原南出,谓之清水口。东南流,绝郑渠,又东南,入高陵县,径黄白城西,本曲梁宫也。南绝白渠,屈而东流,谓之曲梁水。又东南,径高陵县故城北,东南绝白渠枝渎,又东南,入万年县,谓之五丈渠。又径藕原东,东南流,注于渭。渭水右径新丰县故城北,东与鱼池水会。水出丽山东北,本导源北流,后秦始皇葬于山北,水过而曲行,东注北转。始皇造陵取土,其地污深,水积成池,谓之鱼池也。在秦皇陵东北五里,周围四里。池水西北流,径始皇冢北。秦始皇大兴厚葬,营建冢扩于丽戎之山,一名蓝田,其阴多金,其阳多玉。始皇贪其美名,因而葬焉。斩山凿石,下锢三泉。以铜为椁,旁行周回三十余里。上画天文星宿之象,下以水银为四渎百川,五岳九州,具地理之势。宫观百官,奇器珍宝,充满其中。令匠作机弩,有所穿近,辄射之。以人鱼膏为灯烛,取其不灭者,久之,后宫无子者,皆使殉葬,甚众。坟高五丈,周回五里余。作者七十万人,积年方成。而周章百万之师已至其下,乃使章邯领作者以御难,弗能禁。项羽入关,发之以三十万人,三十日,运物不能穷。关东盗贼,销椁取铜。牧人寻羊,烧之,火延九十日,不能灭。北对鸿门十里。池水又西北流,水之西南有温泉,世以疗疾。《三秦记》曰:丽山西北有温水,祭则得入,不祭则烂人肉。俗云:始皇与神女游而忤其旨,神女唾之生疮,始皇谢之,神女为出温水,后人因以浇洗疮。张衡《温泉赋序》曰:余出丽山,观温泉,浴神井,嘉洪泽之普施,乃为之赋云。此汤也,不使的人形体矣。池水又径鸿门西,又径新丰县故城东,故丽戎地也。高祖王关中,太上皇思东归,故象旧里,制兹新邑,立城社,树枌榆,令街庭若一,分置丰民,以实兹邑,故名之为新丰也。汉灵帝建宁三年,改为都乡,封段颎为侯国。后立阴槃城。其水际城北出,世谓是水为阴槃水,又北绝漕渠,北注于渭。渭水又东,径鸿门北,旧大道北下坂口名也。右有鸿亭。《汉书》:高祖将见项羽。《楚汉春秋》曰:项王在鸿门,亚父曰:吾使人望沛公,其气冲天,五色采相缪,或似龙,或似云,非人臣之气,可诛之。高祖会项羽,范增目羽,羽不应。樊哙杖盾撞人入,食豕肩于此,羽壮之。《郡国志》曰:新丰县东有鸿门亭者也。郭缘生《述征记》,或云,霸城南门曰鸿门也。项羽将因会危高祖,羽仁而弗断。范增谋而不纳,项伯终护高祖以获免。既抵霸上,遂封汉王。按《汉书注》,鸿门在新丰东十七里,则霸上应百里。按《史记》,项伯夜驰告张良,良与俱见高祖,仍使夜返。考其道里,不容得尔。今父老传在霸城南门数288 十里,于理为得。按缘生此记,虽历览《史》、《汉》,述行涂经见,可谓学而不思矣。今新丰县故城东三里有坂,长二里余,堑原通道,南北洞开,有同门状,谓之鸿门。孟康言在新丰东十七里,无之。盖指县治而言,非谓城也。自新丰故城西,至霸城五十里,霸城西十里,则霸水,西二十里则长安城。应劭曰:霸水上地名,在长安东三十里,即霸城是也。高租旧停军处,东去新丰既远,何由项伯夜与张良共见高祖乎?推此言之,知缘生此记乖矣!渭水又东,石川水南注焉。渭水又东,戏水注之。水出丽山冯公谷,东北流,又北径丽戎城东。《春秋》晋献公五年,伐之,获丽姬于是邑。丽戎,男国也,姬姓。秦之丽邑矣。又北,右总三川,径鸿门东,又北径戏亭东。应劭曰:戏,宏农湖县西界也。地隔诸县,不得为湖县西。苏林曰:戏,邑名,在新丰东南四十里。盂康曰:乃水名也,今戏亭是也。昔周幽王悦褒拟,姒不笑,王乃击鼓举烽,以征诸侯。诸侯至,无寇,褒姒乃笑,王甚悦之。及犬戎至,王又举烽以征诸侯,诸侯不至,遂败幽王于戏水之上,身死于丽山之北。故《国语》曰:幽灭者也。汉成帝建始二年,造延陵为初陵,以为非吉,于霸曲亭南更营之。鸿嘉元年,于新丰戏乡为昌陵县,以奉初陵。永始元年,诏以昌陵卑下,客土疏恶,不可为万岁居,其罢陵作,令吏民反故,徙将作大匠解万年燉煌。《关中记》曰:昌陵在霸城东二十里,取土东山,与粟同价,所费巨万,积年无成,即此处也。戏水又北分为二水,并注渭水。渭水又东,泠水入焉。水南出肺浮山,盖丽山连麓而异名也。北会三川,统归一壑,历阴槃、新丰两原之间,北流注于渭。渭水又东,酋水南出倒虎山,西总五水,单流径秦步高宫东,世名市丘城。历新丰原东,而北径步寿宫西,又北入渭。渭水又东得西阳水,又东得东阳水,并南出广乡原北垂,俱北入渭。渭水又东径下邦县故城南。秦伐邦,置邦戎于此,有上邦,故加下也。渭水又东与竹水合。水南出竹山,北径媚加谷,历广乡原东,俗谓之大赤水,北流注于渭。渭水又东得白渠口。大始二年,赵国中大夫白公,奏穿渠。引泾水,首起谷口,出于郑渠南,名曰白渠。民歌之曰:田于何所,池阳谷口。郑国在前,白渠起后。即水所始也。东径宜春城南,又东南径池阳城北,枝渎出焉,东南历藕原下,又东径鄣县故城北,东南入渭。今无水。白渠又东,枝渠出焉。东南径高陵县故城北。《地理志》曰:左辅都尉治,王莽之千春也。《太康地记》谓之曰高陆也。车频《秦书》曰:苻坚建元十四年,高陆县民穿井,得龟,大二尺六寸,背文负八卦古字,坚以石为池,养之,十六年而死,取其骨以问吉凶,名为客龟。大卜佐高鲁梦客龟言,我将归江南,不遇,死于秦。鲁于梦中自解曰:龟三万六千岁而终,终必亡国之征也。为谢玄破于淮肥,自缢新城浮图中,秦祚因即沦矣。又东径栎阳城北。《史记》,秦献公二年,城栎阳,自雍徙居之。十八年雨金于是处也。项羽以封司马欣为塞王。按《汉书》,高帝克关中,始都之,王莽之师亭也。后汉建武二年,封骠骑大将军景丹为侯国。丹让,世祖曰:富贵不还故乡,如衣锦夜行,故以封卿。白渠又东,径秦孝公陵北,又东南径居陵城北,莲芍城南,又东注金氏陂,又东南注于渭。故《汉书·沟洫志》曰白渠首起谷口,尾入栎阳是也。今无水。
又东过郑县北,渭水又东径峦都城北,故蕃邑,殷契之所居。《世本》曰:290 契居著。阚駰曰:蕃在郑西。然则今峦城是矣,俗名之赤城,水曰赤水,非也。苻健入秦,据此城以抗杜洪。小赤水即《山海经》之灌水也,水出石脆之山,北径萧加谷于孤柏原西,东北流与禺水合,水出英山,北流与招水相得,乱流西北注于灌。灌水又北注于渭,渭水又东,西石桥水南出马岭山,积石据其东,丽山距其西,源泉上通,悬流数十,与华岳同体。其水北径郑城西,水上有桥,桥虽崩褫,旧迹犹存,东去郑城十里,故世以桥名水也。而北流注于渭,阚駰谓之新郑水。渭水又东径郑县故城北。《史记》,秦武公十一年,县之。郑桓公友之故邑也。《汉书》薛瓒《注》言,周自穆王已下,都于西郑,不得以封桓公也。幽王既败,虢、侩又灭,迁居其地,国于郑父之丘,是为郑桓公。无封京兆之文。余按迁《史记》,考《春秋》、《国语》、《世本》言,周宣王二十二年,封庶弟友于郑。又《春秋》、《国语》并言桓公为周司徒,以王室将乱,谋于史伯而寄帑与贿于虢侩之间。幽王霣于戏,郑桓公死之。平王东迁,郑武公辅王室,灭虢、侩而兼其土。故周桓公言于王曰:我周之东迁,晋、郑是依。乃迁封于彼。《左传》隐公十一年,郑伯谓公孙获曰:吾先君新邑于此,其能与许争乎?是指新郑为言矣。然班固、应劭、郑玄、皇甫谧、裴頠、王隐、阚駰及诸述作者,咸以西郑为友之始封,贤于薛瓒之单说也,无宜违正经而从逸录矣。赤盾樊崇于郭北设坛,把城阳景王,而尊右挍卒史刘侠卿牧牛儿盆子为帝。年十五,被发徒跣,为具蜂单衣,半头赤帻,直綦履。顾见众人拜,恐畏欲啼,号年建世。后月余,乘白盖小车,与崇及尚书一人相随,向郑,北渡渭水,即此处也。城南山北有五部神庙,东南向华岳。庙前有碑,后汉光和四年郑县令河东裴毕字君先立。渭水又东与东石桥水会,故沈水也。水南出马岭山,北流径武平城东。按《地理志》,左冯翊有武城县,王莽之桓城也。石桥水又径郑城东,水有故石梁。《述征记》曰郑城东西十四里,各有石梁者也。又北径沈阳城北,注于渭。《汉书·地理志》,左冯翊有沈阳县,王莽更之曰制昌也。盖藉水以取称矣。渭水又东,敷水注之。水南出石山之敷谷,北径告平城东。耆旧所传,言武王代纣,告太平于此,故城得厥名,非所详也。敷水又北径集灵宫西。《地理志》曰:华阴县有集灵宫,武帝起。故张昶《华岳碑》称汉武慕其灵,筑宫在其后。而北流注于渭。渭水又东,粮余水注之。水南出粮余山之阴,北流入于渭,俗谓之宣水也。渭水又东,合黄酸之水,世名之为千渠水。水南出升山,北流注于渭。渭水又东径平舒城北。城侧枕渭滨,半破沦水,南面通衢。昔秦始皇之将亡也,江神素车白马,道华山下,返壁于华阴平舒道,曰:为遗镐池君。使者致之,乃二十八年渡江所沉壁也。即江神返壁处也。渭水之阳,即怀德县界也。城在渭水之北,沙苑之南。即怀德县故城也,世谓之高阳城,非矣。《地理志》曰:《禹贡》北条荆山,在南,山下有荆渠。即夏后铸九鼎处也。王莽更县曰德驩。渭水又东,径长城北,长涧水注之。水南出太华之山,侧长城东而北流,注于渭水。《史记》,秦孝公元年,楚、魏与秦接界。魏筑长城,自郑滨洛者也。
又东过华阴县北,洛水人焉,阚駰以为漆沮之水也。《曹瞒传》曰:操与马超隔渭水,每渡渭,辄为超骑所冲突。地多沙,不可筑城。娄子怕说:今寒,可起沙为城,以水灌之,一宿而成。操乃多作缣囊以堙水,夜汲作城,比明,城立于是水之次也。渭水径具故城北,《春秋》之阴晋也。秦惠文王五年,改曰宁秦。汉高帝八年,更名华阴。王莽之华坛也。县有华山。《山海经》曰:其高五千仞,削成而四方,远而望之,又若华状,西南有小华山也。韩子曰:秦昭王令工施钩梯,上华山,以节柏之心为博箭,长八尺,棋长八寸,而勒之曰:昭王尝与天神博于是。《神仙传》曰:中山卫叔卿尝乘云车,驾白鹿,见汉武帝,帝将臣之,叔卿不言而去。武帝悔,求得其子度世,令追其父。度世登华山,见父与数人博于石上,勅度世令还。山层云秀,故能怀灵抱异耳。山上有二泉,东西分流,至若山雨滂湃,洪津泛洒,挂溜腾虚,直泻山下。有汉文帝庙,庙有石阙数碑。一碑是建安中立,汉镇远将军段煨更修祠堂。碑文,汉给事黄门侍郎张昶造,昶自书之,文帝又刊其二十余字,二书存垂名海内。又刊侍中、司隶校尉钟繇,宏农太守田丘俭姓名,广六行,郁然修平。是太康八年,宏农太守河东卫叔始为华阴令,河东裴仲恂,役其逸力,修立坛庙,夹道树柏,迄于山阴,事见永兴元年华百石所造碑。渭水又东,沙渠水注之。水出南山,北流,西北入长城。城自华山,北达于河。《华岳铭》曰:秦、晋争其祠,立城建其左者也。郭著《述征记》,指证魏之立长城,长城在后,不得在斯,斯为非矣。渠水又北注于渭。《三秦记》曰:长城北有平原,广数百里,民井汲巢居,井深五十尺。渭水又东,径定城北。《西征记》曰:城因原立。《述征记》曰:定城去潼关三十里,夹道各一城。渭水又东,泥泉水注之。水出南山灵谷,而北流注于渭水也。渭水又东合沙渠水,水即符禺之水也。南出符石,又径符禺之山,北流入于渭。东入于河。《春秋》之渭油也。《左传》闵公二年,虢公败犬戎于渭队。服虔曰:队谓汭也。杜预曰:水之限曲曰汭。王肃云:汭,入也。吕忱云:汭者,水相入也。水会即船司空所在矣。《地理志》曰:渭水东至船司空入河。服虔曰:县名,都官。《三辅黄图》有船库官,后改为县。王莽之船利者也。


卷二十  漾水、丹水 
漾水出陇西氏道县.冢山,东至武都沮县为汉水。常璩《华阳国志》曰:汉水有二源,东源出武都氐道县漾山,为漾水。《禹贡》导漾东流为汉是也。西源出陇西西县.冢山,会白水,径葭萌入汉。始源曰沔。按沔水出东狼谷,径沮县入汉。《汉中记》曰:.冢以东,水皆东流,.冢以西,水皆西流。即其地势源流所归,故俗以.冢为分水岭。即此推沔水无西入之理。刘澄之云:有水从阿阳县,南至梓潼、汉寿,入大穴,暗通冈山。郭景纯亦言是矣。冈山穴小,本不容水,水成大泽而流,与汉合。庾仲雍又言,汉水自武遂川,南入蔓葛谷,越野牛,径至关城合西汉水。故诸言汉者,多言西汉水至葭萌人汉。又曰:始源曰沔,是以《经》云漾水出氐道县东至沮县为汉水,东南至广魏白水。诊其沿注,似与三说相符,而未极西汉之源矣。然东西两川,俱受沔、汉之名者,义或在兹矣。班固《地理志》、司马彪、袁山松《郡国志》,并言汉有二源,东出氐道,西出西县之.冢山。阚駰云:汉或为漾。漾水出昆仑西北隅,至氐道,重源显发,而为漾水。又言,陇西西县.冢山,在西,西汉水所出,南入广魏白水。又云:漾水出豲道,东至武都入汉。许慎、吕忱并言,漾水出陇西豲道,东至武都为汉水,不言氐道。然豲道在冀之西北,又隔诸川,无水南入,疑出豲道之为谬矣。又云:汉,漾也,东为沧浪水。《山海经》曰:.冢之山,汉水出焉,而东南流注于江。然东西两川,俱出.冢而同为汉水者也。孔安国曰:泉始出为漾,其犹蒙耳。而常璩专为漾山漾水,当是作者附而为山水之殊目矣。余按《山海经》,漾水出昆仑西北隅,而南流注于丑涂之水。《穆天子传》曰:天子自春山西征,至于赤乌氏。己卯,北征,庚辰,济于洋水,辛巳,入于曹奴。曹奴人戏,觞天子于洋水之上,乃献良马九百,牛羊七千,天子使逢固受之。天子乃赐之黄金之鹿,戏乃膜拜而受。余以太和中,从高祖北巡,狄人犹有此献。虽古今世殊,而所贡不异。然川流隐伏,卒难详照,地理潜闷,变通无方,复不可全言阚氏之非也。虽津流派别,枝渠势悬,原始要终,潜流或一,故俱受汉、漾之名,纳方土之称,是其有汉川、汉阳、广汉、汉寿之号,或因其始,或据其终,纵异名互见,犹为汉漾矣。川共目殊,或亦在斯。今西县.家山,西汉水所导也,然微涓细注,若通幂历,津注而已。西流与马池水合,水出上邦西南六十余里,谓之龙渊水,言神马出水,事同余吾来渊之异,故因名焉。《开山图》曰:陇西神马山有渊池,龙马所生。即是水也。其水西流,谓之马池川。又西流入西汉水。西汉水又西南流,左得兰渠溪水,次西有山黎谷水,次西有铁谷水,次西有石耽谷水,次西有南谷水,并出南山,扬湍北注;右得高望谷水,次西得西溪水,次西得黄花谷水,咸出北山,飞波南入西汉水,又西南,资水注之。水北出资川,导源四壑,南至资峡,总为一水,出峡西南流,注西汉水,西汉水又西南得峡石水口,水出苑亭西草黑谷。三溪西南至峡石口,合为一渎,东南流,屈而南注西汉水。西汉水又西南,合杨廉川水,水出西谷,众川泻流,合成一川。东南流,径西县故城北。秦庄公伐西戎,破之。周宣王与其先大骆犬丘之地,为西垂大夫,亦西垂宫也。王莽之西治矣。建武八年,世祖至阿阳,窦融等悉会。天水震动,隗嚣将妻子奔西城,从杨广。广死,嚣愁穷城守,时颍川贼起,车驾东归,留吴汉、岑彭围嚣。岑等壅西谷水,以缣幔盛土为堤,灌城,城未没丈余。水穿壅不行,地中数丈涌出,故城不坏。王元请蜀救至,汉等退还上邦。但广、廉字相状,后人因以人名名之,故习讹为杨廉也,置杨廉县焉。又东南流,右会茅川水,水出西南戎溪,东北流,径戎丘城甫。吴汉之围西城,王捷登城,向汉军曰:为隗王城守者,皆必死,无二心,愿诸将亟罢,请自杀以明之。遂刎颈而死。又东北流,注西谷水,乱流东南,入于西汉水。西汉水又西南,径始昌峡,《晋书地道记》曰:天水始昌县,故西城也。亦曰清崖峡。西汉水又西南,径宕备戍南,左则宕备水自东南,西北注之。右则盐官水南入焉。水北有盐官,在.冢西五十许里,相承营煮不辍,味与海盐同。故《地理志》云西县有盐官是也。其水东南径宕备戍西,东南入汉水。汉水又西南,合左谷水,水出南山穷溪,北注汉水。又西南,兰皋水出西北五交谷,东南历祁山军,东南入汉水。汉水又西南,径祁山军南,鸡水南出鸡谷,北径水南县西,北流注于汉。汉水又西,建安川水入焉。其水导源建威西北山,白石戍东南,二源合注。东径建威城南,又东与兰坑水会,水出西南近溪,东北径兰坑城西,东北流注建安水。建安水又东径兰坑城北,建安城甫,其地故西县之历城也。杨定自陇右徙治历城,即此处也,去仇池百二十里,后改为建安城。其水又东合错水,水出错水戍东南,而东北入建安水。建安水又东北,有雉尾谷水,又东北,有大谷水,又北,有小祁山水,并出东溪,扬波西注。又北,左会胡谷水,水西出胡谷,东径金盘、历城二军北,军在水南层山上。其水又东注建安水。建安水又东北,径塞峡。元嘉十九年,宋太祖遣龙骧将军裴方明伐杨难当,难当将妻子北奔,安西参军鲁尚期追出塞峡,即是峡矣。左山侧有石穴洞,人言潜通下辨,所未详也。其水出峡西北流,注汉水。汉水北连山秀举,罗峰竞峙。祁山在.冢之西七十许里,山上有城,极为严固。昔诸葛亮攻祁山,即斯城也。汉水径其南。城南三里,有亮故垒,垒之左右,犹丰茂宿草,盖亮所植也,在上邦西南二百四十里。《开山图》曰:汉阳西南有祁山,溪径逶迤,山高岩险,九州之名阻,天下之奇峻。今此山于众阜之中,亦非为杰矣。汉水又西南,与甲谷水合,水出西南甲谷,东北流注汉水。汉水又西径南蚜北蚜中。上下有二城相对,左右坟垅低昂,亘山被阜。古谚云:南蚜北蚜,万有余家。诸葛亮《表》言:祁山去沮县五百里,有民万户,瞩其丘墟,信为殷矣。汉水西南径武植戍南。武植戌水发北山,二源奇发,合于安民戍南,又南径武植戍西,而西南流,注于汉水。汉水又西南,径平夷戍南,又西南,夷水注之。水出北山,南径其戍,西南入汉水。汉水又西径兰仓城南,又南,右会两溪,俱出西山,东流注于汉水。张华《博物志》云:温水出鸟鼠山,下注汉水。疑是此水,而非所详也。汉水又南入嘉陵道,而为嘉陵水。世俗名之为阶陵水,非也。汉水又东南,得北谷水,又东南得武街水,又东南得仓谷水,右三水并出西溪,东流注汉水。汉水又东南径瞿堆西,又屈径瞿堆南。绝壁峭峙,孤险云高,望之形若298 覆唾壶,高二十余里,羊肠蟠道三十六回,《开山图》谓之仇夷,所谓积石嵯峨,嵚岑隐阿者也。上有平田百顷,煮土成盐,因以百顷为号。山上丰水泉,所谓清泉涌沸,润气上流者也。汉武帝元鼎六年开,以为武都郡。天池大泽在西,故以都为目矣。王莽更名乐平郡,县曰循虏。常璩、范晔云:郡居河池,一名仇池,地方百顷,即指此也。左右悉白马氐矣。汉献帝建安中,有天水氐杨腾者,世居陇右,为氐大帅。子驹,勇健多计,徙居仇池。魏拜为百顷氐王。汉水又东合洛谷水,水有二源,同注一壑,径神蛇戍西。左右山溪多五色蛇,性驯良,不为物毒。洛谷水又南径虎馗戍东,又南径仇池郡西,瞿堆东,西南入汉水。汉水又东合洛溪水,水北发洛谷,南径威武戍南,又西南与龙门水合,水出西北龙门谷,东流与横水会,东北穷溪,即水源也。又南径龙门戍东,又东南入洛溪水,又东南径上禄县故城西,修源浚导,径引北溪,南总两川,单流纳汉。汉水又东南径浊水城南,又东南会平乐水,水出武街东北四十五里,东驰。南溪导源东北流,山侧有甘泉,涌波飞清,下注平乐水。又径甘泉戍甫,又东径平乐戍南,又东入汉,谓之会口。汉水东南径修城道南,与修水合。水总二源,东北合汉。汉水又东南于槃头郡南,与浊水合。水出浊城北,东流与丁令溪水会。其水北出丁令谷,南径武街城西,东南入浊水。浊水又东径武街城南,故下辨县治也。李琀、李稚以氐王杨难敌妻死,葬阴平。袭武街,为氐所杀于此矣,今广业郡治。浊水又东,宏休水注之。水出北溪,南径武街城东,而南流注于浊水。浊水又东径白石县南。《续汉书》曰:虞诩为武都太守,下辨东三十余里有峡,峡中白水生大石,障塞水流,春夏辄濆溢,败坏城郭。诩使人烧石,以醢灌之。石皆碎裂,因镌去焉。遂无泛溢之害。浊水即白水之异名也。浊水又东南,埿阳水北出埿谷,南径白石县东,而南入浊水。浊水又东南与仇鸠水合,水发鸠溪,南径河池县故城西,王莽之乐平亭也。其水西南流注浊水。浊水又东南与河池水合,水出河池北谷,南径河池戍东,西南入浊水。浊水又东南,两当水注之。水出陈仓县之大散岭,西南流入故道川,谓之故道水。西南径故道城东,魏征仇池,筑以置戍。与马鞍山水合。水东出马鞍山,历谷西流,至故道城东,西入故道水。西南流,北川水注之,水出北洛埿山南。南流径唐仓城下,南至困冢川,入故道水。故道水又西南历广香交,合广香川水,水出南田县利乔山,南流至广香川,谓之广香川水。又南注故道水,谓之广香交。故道水又西南,入秦冈山,尚婆水注之。山高入云,远望增状,若岭纤曦轩,峰枉月驾矣。悬崖之侧,列壁之上,有神象若图,指状妇人之容。其形上赤下白,世名之曰圣女神,至于福应愆违,方俗是祈。水源北出利乔山,南径尚婆川,谓之尚婆水。历两当县之尚婆城南,魏故道郡治也。西南至秦冈山,入故道水。故道水又右会黄卢山水,水出西北天水郡黄卢山腹,历谷南流,交注故道水。故道水南入东益州之广业郡界,与沮水枝津合,谓之两当溪,水上承武都沮县之沮水渎,西南流,注于两当溪。虞诩为郡漕谷市在沮,从沮县至下辨,山道险绝,水中多石,舟车不通,驴马负运,僦五致一。诩乃于沮受僦直,约自致之。即将吏民按行,皆烧石木,开漕船道。水运通利,岁省万计,以其僦廪与吏士,年四十余万也。又西南,注于浊水,浊水南径槃头郡东,而南合凤溪水,水上承浊水于广业郡,南径凤溪,中有二石双高,其形若阙,汉世有凤凰止焉,故谓之风凰台,北去郡三里。水出台下东南流,左注浊水。浊水又南注汉水。汉水又东南历汉曲,径挟崖,与挟崖水合。水西出担潭交,东流入汉水。汉水又东,径武兴城南,又东南与北谷水合,水出武兴东北,而西南径武兴城北,谓之北谷水。南转径其城东,而南与一水合,水出东溪,西流注北谷水。又南流,注汉水。汉水又西南,径关城北,除水出西北除溪,东南流入于汉。汉水又西南,径通谷,通谷水出东北通溪,上承漾水,西南流,为西汉水。汉水又西南,寒水注之。水东出寒川,西流入汉。汉水又西,径石亭戍。广平水西出百顷川,东南流注汉。又有平阿水,出东山,西流注汉水。汉水又径晋寿城西,而南合汉寿水。水源出东山,西径东晋寿故城南,而西南人于汉水也。
又东南至广魏白水县西。又东南至葭萌县东北,与羌水合。白水西北出于临洮县西南西倾山,水色白浊,东南流与黑水合,水出羌中,西南径黑水城西,又西南入白水。白水又东径洛和城南,洛和水西南出和溪,东北流,径南黑水城西,而北注白水。白水又东南径邓至城南。又东南与大夷祝水合,水出夷祝城西南,穷溪,北注夷水。又东北合羊洪水,水出东南羊溪,西北径夷祝城东,又西北流,屈而东北,注于夷水。夷水又东北入白水,白水又东,与安昌水会,水源发卫大西溪,东南径邓至安昌郡甫,又东南,合无累水,无累水出东北近溪,西南入安昌水。安昌水又东南人白水,白水又东南,入阴平,得东维水,水出西北维谷,东南径维城西,东南入白水。白水又东南,径阴平道故城南。王莽更名摧虏矣,即广汉之北部也。广汉属国都尉治,汉安帝永初三年分广汉蛮夷置。又有白马水,出长松县西南白马溪,东北径长松县北,而东北注白水。白水又东,径阴平大城北,盖其渠帅自故城徙居也。白水又东,偃溪水出西南偃溪,东北流径偃城西,而东北流入白水。白水又东,径偃城北,又东北,径桥头。昔姜维之将还蜀也,雍州刺史诸葛绪邀之于此,后期不及,故维得保剑阁,而钟会不能入也。白水又与羌水合,自下羌水又得其通称矣。白水又东,径郭公城南。昔郭淮之攻廖化于阴平也,筑之,故因名焉。白水又东,雍川水出西南雍溪,东北注白水。白水又东,合空冷水,傍溪西南,穷谷,即川源也。白水又东南与南五部水会。水有二源,西源出五部溪,东南流,东源出郎谷,西南合注白水。白水又东南,径建昌郡东,而北与一水合,二源同注,共成一溪,西南流入于白水。白水又东南,径白水县故城东,即白水郡治也。《经》云汉水出其西,非也。白水又东南,与西谷水相得,水出西溪,东流径白水城南,东南入白水。白水又南,左会东流水,东入极溪,便即水源也。白水又南径武兴城东,又东南,左得刺稽水口,溪东北出,便水源矣。白水又东南,清水左注之。庾仲雍曰:清水自祁山来,合白水,斯为盂浪也。水出于平武郡东北,瞩累亘下,甫径平武城东,屈径其城南,又西历平洛郡东南,屈而南径南阳侨郡东北,又东南,径新巴县东北,又东南径始平侨郡南,又东南径小剑戍北。西去大剑三十里,连山绝险,飞阁通衢,故谓之剑阁也。张载铭曰:一人守险,万夫趦趄。信然。故李特至剑阁而叹曰:刘氏有如此地,而面缚于人,岂不奴才也?小剑水西南出剑谷,东北流径其戍下,入清水。清水又东南,注白水。白水又东南,于吐费城南,即西晋寿之东北也。东南流,注汉水。西晋寿,即蜀王弟葭萌所封,为苴侯邑,故遂名城为葭萌矣。刘备改曰汉寿,太康中又曰晋寿。水有津关。段元章善风角,弟子归,元章封笥药授之,曰:路有急难,开之。生到葭萌,从者与津吏诤,打伤,开笥得书,言其破头者,可以此药裹之。生乃叹服,还卒业焉。亦廉叔度抱父柩自沉处也。
又东南过巴郡阆中县。
巴西郡治也,刘璋之分三巴,此其一焉。阚駰曰:强水出阴平西北强山,一曰强川。姜维之还也,邓艾遣天水太守王颀败之于强川,即是水也。其水东北,径武都、阴平、梓潼、南安入汉水。汉水又东南,径津渠戍东,又南径阆中县东。阆水出阆阳县,而东径其县南,又东注汉水。昔刘璋之攻霍峻于葭萌也,自此水上。张达、范强害张飞于此县。汉水又东南,得东水口,水出巴岭,南历獠中,谓之东游水。李寿之时,獠自牂柯北入,所在诸郡,布满山谷。其水西南,径宋熙郡东,又东南径始平城东,又东南,径巴西郡东,又东入汉水。汉水又东,与濩溪水合,水出獠中,世亦谓之为清水也。东南流,注汉水。汉水又东南,径宕渠县东,又东南,合宕渠水,水西北出南郑县巴岭,与槃余水同源派注,南流,谓之北水,东南流,与难江水合,水出东北小巴山,西南注之。又东南流,径宕渠县,谓之宕渠水,又东南,入于汉。
又东南过江州县东,东南入于江。
涪水注之。庚仲雍所谓涪内水者也。
丹水出京兆上洛县西北冢岭山,一名高猪岭也。丹水东南流,与清池水合,水源东北出清池山,西南流,入于丹水。
东南过其县南。县故属京兆,晋分为郡。《地道记》曰:郡在洛上,故以为名。《竹书纪年》,晋烈公三年,楚人伐我南鄙,至于上洛。楚水注之,水源出上洛县西南楚山。昔四皓隐于楚山,即此山也。其水两源,合舍于四皓庙东,又东径高车岭南,翼带众流,北转入丹水。岭上有四皓庙。丹水自仓野,又东历兔和山,即春秋所谓左师军于兔和,右师军于仓野者也。
又东南过商县南,又东南至于丹水县,入于均。
契始封商。《鲁连子》曰:在太华之阳。皇甫谧、阚駰并以为上洛商县也。殷商之名,起于此矣。丹水自商县东南流注,历少习,出武关。应劭曰:秦之南关也,通南阳郡。《春秋左传》哀公四年,楚左司马使谓阴地之命大夫士蔑曰:晋、楚有盟,好恶同之,不然,将通于少习以听命者也。京相璠曰:楚通上洛阨道也。汉祖下析、郦,攻武关。文颖曰:武关在析县西百七十里,宏农界也。丹水又东南流入臼口,历其戍下。又东南,析水出析县西北,宏农卢氏县大蒿山。南流径修阳县故城北,县即析之北乡也。又东入析县,流结成潭,谓之龙渊,清深神异。耆旧传云:汉祖入关,径观是潭,其下若有府舍焉。事既非恒,难以详矣。其水又东径其县故城北,盖《春秋》之白羽也。《左传》昭公十八年,楚使王子胜迁许于析是也。郭仲产云:相承言此城汉高所筑,非也。余按《史记》楚襄王元年,秦出武关,斩众五万,取析十五城。汉祖入关,亦言下析、郦,非无城之言,修之则可矣。析水又历其县东,王莽更名县为君亭也。而南流入丹水县,注于丹水,故丹水会均,有析口之称。丹水又东南,径一故城南,名曰三户城。昔汉祖入关,王陵起兵丹水,以归汉祖,此城疑陵所筑也。丹水又径丹水县故城西南。县有密阳乡,古商密之地,昔楚申息之师所戍也,《春秋》之三户矣。杜预曰:县北有三户亭。《竹书纪年》曰:壬寅,孙何侵楚,入三户郛者是也。水出丹鱼,先夏至十日夜,伺之,鱼浮水侧,赤光上照如火,网而取之,割其血以涂足,可以步行水上,长居渊中。丹水东南流,至其县南。黄水北出芬山黄谷,南径丹水县,南注丹水。黄水北有墨山,山石悉黑,绩彩奋发,黝焉若墨,故谓之墨山。今河南新安县有石墨山,斯其类也。丹水南有丹崖山,山悉赬壁霞举,若红云秀天,二岫更为殊观矣。丹水又南,径南乡县故城东北。汉建安中,割南阳右壤为南乡郡。逮晋封宣帝孙畅为顺阳王,因立为顺阳郡。而南乡为县,旧治酇城。永嘉中,丹水浸没,至永和中,徙治南乡故城。城南门外,旧有郡社柏树,大三十围。萧欣为郡,伐之,言有大蛇从树腹中坠下,大数围,长三丈,群小蛇数十,随入南山,声如风雨。伐树之前,见梦于欣,欣不以厝意,及伐之,更少日,果死。丹水又东,径南乡县北。兴宁未,太守王靡之改筑今城。城北半据在水中,左右夹涧深长。及春夏水涨,望若孤洲矣。城前有晋顺阳太守丁穆碑,郡民范宁立之。丹水径流两县之间,历于中之北,所谓商于者也。故张仪说楚绝齐,许以商于之地六百里,谓以此矣。《吕氏春秋》曰:尧有丹水之战,以服南蛮。即此水也。又南合均水,谓之析口。


卷二十一  汝水 
汝水出河南梁县勉乡西天息山,《地理志》曰:出高陵山,即猛山也。亦言出南阳鲁阳县之大盂山。又言:出弘农卢氏县还归山。《博物志》曰:汝出燕泉山,并异名也。余以水平中,蒙除鲁阳太守,会上台下,列山川图,以方志参差,遂令寻其源流。此等既非学徒,难以取悉,既在径见,不容不述。今汝水西出鲁阳县之大盂山蒙柏谷,岩鄣深高,山岫邃密,石径崎岖,人迹裁交,西即卢氏界也。其水东北流,径太和城西,又东流径其城北。左右深松列植,筠柏交荫,尹公度之所栖神处也。又东届尧山西岭下,水流两分,一水东径尧山南,为滍水也,即《经》所言滍水出尧山矣。一水东北出为汝水,历蒙柏谷,左右岫壑争深,山阜竞高,夹水层松茂柏,倾山荫渚,故世人以名也。津流不已,北历长白沙口,狐白溪水注之。夹岸沙涨若雪,因以取名。其水南出狐白川,北流注汝水,汝水又东北趣狼皋山者也。
东南过其县北,汝水自狼皋山东出峡,谓之汝阨也。东历麻解城北,故鄤乡城也,谓之蛮中。《左传》所谓单浮余围蛮氏,蛮氏溃者也。杜预曰:城在河南新城县之东南,伊洛之戎,陆浑蛮氏城也。俗以为麻解城,盖蛮、麻读声近故也。汝水又径周平城南。京相璠曰:霍阳山在周平城东南者也。汝水又东,与三屯谷水合,水出南山,北流,径石碣东。柱侧刊云河南界。又有一碣,题言洛阳南界。碑柱相对,既无年月,竟不知何代所表也。其水又北流,注于汝水。汝水又东与广成泽水合,水出狼皋山北泽中。安帝永初元年,以广成游猎地假与贫民。元初二年,邓大后临朝,邓骘兄弟辅政。世士以为文德可兴,武功宜废,寝蒐狩之礼,息战阵之法。于时,马融以文武之道,圣贤不坠,五材之用,无或可废,作《广成颂》云:大汉之初基也,揆厥灵囿,营于南郊。右三涂,左枕嵩岳,面据衡阴,背箕王屋,浸以波、溠,演以荣、洛。金山、石林,殷起乎其中。神泉侧出,丹水、涅池。怪石浮磐,耀焜于其陂。桓帝延熹元年,校猎广成,遂幸函谷关。其水自泽东南流,径温泉南,与温泉水合。温水数源,扬波于川左,泉上,华宇连荫,茨甍交拒,方塘石沼,错落其间。颐道者多归之。其水东南流,注广成泽水。泽水又东南入于汝水。汝水又东,得鲁公水口。水上承阳人城东鲁公陂。城,古梁之阳人聚也,秦灭东周,徙其君于此。陂水东南流,合于涧水,水出北山,南流注之,又乱流,注于汝水。汝水之右,有霍阳聚。汝水径其北,东合霍阳山水,水出南山。杜顶曰:河南粱县有霍山者也。其水东北流,径霍阳聚东。世谓之华浮城,非也。《春秋左传》哀公四年,楚侵梁及霍。服虔曰,梁、霍,周南鄙也。建武二年,世祖遣征虏将军祭遵攻蛮中山贼张满,时厌新、柏华余贼合,攻得霍阳聚,即此。霍阳山水又径梁城西。按《春秋》,周小邑也,于战国为南梁矣。故《经》云,汝水径其县北,俗谓之治城,非也,以北有注城故也。今置治城县,治霍阳山。水又东北流,注于汝水。汝水又左合三里水,水北出梁县西北,而东南流,径其县故城西,故惮狐聚也。《地理志》云:秦灭西周,徙其君于此,因乃县之。杜预曰:河南县西南有梁城,即是县也。水又东南,径注城南。司马彪曰:河南梁县有注城。《史记》魏文侯三十二年,败秦于注者也。又与一水合,水发注城东坂下,东南流注三里水。三里水又乱流入于汝。汝水又东径成安县故城北。按《地理志》,颍川郡有成安县,侯国也。《史记·建元以来功臣侯者年表》曰:汉武帝元鼎五年,校尉韩千秋击南越,死,封其子韩延年为成安侯。即此邑矣。世谓之白泉城,非也,俗谬耳。汝水又东,为周公渡,藉承休之徽号,而有周公之嘉称也。汝水又东,黄水注之。水出梁山,东南径周承休县故城东,为承休水。县,故子南国也。汉武帝元鼎四年,幸洛阳,巡省豫州,观于周室,邈而无祀。询问耆老,乃得孽子嘉,封为周子南君,以奉周把。按《汲冢古文》,谓卫将军文子为子南弥牟,其后有子南劲。《纪年》,劲朝于魏。后惠成王如卫,命子南为侯。秦并六国,卫最后灭。疑嘉是卫后,故氏子南而称君也。初元五年,为周承休邑。《地理志》曰:侯国也,元帝置。元始二年,更曰郑公。王莽之嘉美也。故汝渡有周公之名,盖藉邑以纳称,世谓之黄城,水曰黄水,皆非也。其水又东南,径白茅台东,又南径梁瞿乡西,世谓之期城,非也。按《后汉书》,世祖自颍川往梁瞿乡,冯鲂先诣行所,即是邑也。水积为陂,世谓之黄陂。东转,径其城南,东流,右合汝水。
又东南过颍川郏县南。
汝水又东与张磨泉合,水发北阜,春夏水盛,则甫注汝水。汝水又东,分为西长湖,湖水南北五十余步,东西三百步。汝水又东,扈涧水北出大刘山,南径木蓼堆东、郊城西,南流入于汝。汝水又右,迤为湖。湖水南北八九十步,东西四五百步,俗谓之东长湖,湖水下入汝,古养水也。水出鲁阳县北将孤山北长冈下。数泉俱发,东历永仁三堆南。又东径沙川,世谓之沙水,历山符垒北,又东径沙亭南,故养阴里也。司马彪《郡国志》曰:襄城有养阴里。京相璠曰:在襄城郏县西南。养,水名也,俗以是水为沙水,故亦名之为沙城,非也。又城处水之阳,而以阴为称,更用惑焉。但流杂间居,裂溉互移,致令川渠异容,津途改状,故物望疑焉。又右会堇沟水,水出沛公垒西六十许步。盖汉祖入关,往征是由,故地擅斯目矣。其水东北注养水,养水又东北入东长湖,乱流注汝水也。汝水又径郊县故城南,《春秋》昭公十九年,楚令尹子瑕之所城也。滶水注之。水出鲁阳县之将孤山,东南流。许慎云:水出南阳鲁阳,入父城,从水,敖声。吕忱《字林》亦言在鲁阳。滶水东入父城县,与桓水会。水出鲁阳北山,水有二源,奇导于贾复城,合为一渎。径贾复城北,复南,击郾所筑也。俗语讹谬,谓之寡妇城,水曰寡妇水。此渎水有穷通,故有枯渠之称焉。其水东北流至父城县北,右注滶水,乱流又东北至郏,入汝。汝水又东南,左合蓝水。水出阳翟县重岭山,东南流,径纪氏城西,有层台,谓之纪氏台。《续汉书》曰:世租车驾西征,盗贼群起。郊令冯鲂为贼延哀所攻,力屈。上诣纪氏,群贼自降,即是处,在郏城东北十余里。其水又东南流,径黄阜东,而南入汝水。汝水又东南流,与白沟水合,水出夏亭城西,又南径龙城西。城西北即摩陂也,纵广可十五里。魏青龙元年,有龙见于郊之摩陂,明帝幸陂观龙,于是改摩陂曰龙陂,其城曰龙城。其水又南入于汝水。汝水又东南与龙山水会,水出龙山龙溪,北流,际父城县故城东。昔楚平王大城城父以居太子建,故杜预曰:即襄城之父城县也。冯异据之,以降世祖,用报巾车之恩也。其水又东北流,与二水合,俱出龙山,北流注之,又东北入于汝水。汝水又东南,径襄城县故城南。王隐《晋书地道记》曰:楚灵王筑。刘向《说苑》曰:襄城君始封之日,服翠衣,带玉佩,徙倚于流水之上,即是水也。楚大夫庄辛所说处。后乃县之。吕后元年,立孝惠后宫子义为侯国,王莽更名相成也。黄帝尝遇牧童于其野,故嵇叔夜赞曰:奇矣难测,襄城小童,倦游六合,来憩兹邦也。其城南对汜城,周襄王出郑居汜,即是此城也。《春秋》襄公二十六年,楚代郑,涉汜而归。杜预曰:涉汝水于汜城下也。晋襄城郡治。京相璠曰:周襄王居之,故曰襄城也。今置关于其下。汝水又东南流,径西不羹城南。《春秋左传》昭公十二年,楚灵王曰:昔诸侯远我而畏晋,今我大城陈、蔡、不羹,赋皆千乘,诸侯其畏我乎?《东观汉记》曰:车骑马防以前参药,勤劳省闼,增封侯国襄城羹亭千二百五十户,即此亭也。汝水又东南,径繁丘城南,而东南出也。
又东南,过定陵县北。
湛水出犨县北鱼齿山西北,东南流,历鱼齿山下,为湛浦,方五十余步,《春秋》襄公十六年,晋伐楚,报杨梁之役。楚公子格及晋师战于湛阪,楚师败绩,遂侵方城之外。今水北悉枕翼山阜,于父城东南,湛水之北,山有长阪,盖即湛水以名阪,故有湛阪之名也。湛水又东南径蒲城北。京相璠曰:昆阳县北有蒲城,蒲城北有湛水者是也。湛水又东,于汝水九曲北,东入汝。杜预亦以是水为湛水矣。《周310 礼》:荆州,其浸颖、湛。郑玄云未闻,盖偶有不照也。今考地则不乖其土,言水则有符经文矣。汝水又东南,径定陵县故城北。汉成帝元延三年,封侍中、卫尉淳于长为侯国,王莽更之曰定城矣。《东观汉记》曰:光武击王莽二公,还到汝水上,于涯,以手饮水,澡颊尘垢,谓傅俊曰:今日疲倦,诸君宁备也?即是水也。水右则滍水左入焉,左则百尺沟出矣。沟水夹岸层崇,亦谓之为百尺堤也。自定陵城北,通颍水于襄城县,颖盛则南播,汝泆则北注,沟之东有澄潭,号曰龙渊,在汝北四里许,南北百步,东西二百步,水至清深,常不耗竭,佳饶鱼笋,湖溢,则东注漷水矣。汝水又东南,昆水注之,水出鲁阳县唐山,东南流,径昆阳县故城西。更始元年,王莽征天下能为兵法者,选练武卫,招募猛士,旌旗辎重,千里不绝。又驱诸犷兽,虎豹犀象之属,以助威武。自秦、汉出师之盛,未尝有也。世祖以数千兵徼之阳关,诸将见寻、邑兵盛,反走入昆阳。世祖乃使成国上公王凤、廷尉大将军王常留守,夜与十三骑出城南门,收兵于郾。寻、邑围城数十重,云车十余丈,瞰临城中,积弩乱发,矢下如雨。城中人负户而汲。王凤请降,不许。世祖帅营部俱进,频破之。乘胜,以敢死三千人,径冲寻、邑兵,败其中坚于是水之上,遂杀王寻。城中亦鼓噪而出,中外合势,震呼动天地。会大雷风,屋瓦皆飞,莽兵大溃。昆水又屈径其城南。世祖建武中,封侍中傅俊为侯国。故《后汉郡国志》有昆阳县,盖藉水以氏县也。昆水又东,径定陵城南,又东,注汝水。汝水又东南,径奇頟城西北。今甫颍川郡治也。水出焉,世亦谓之大水。《尔雅》曰:河有雍,汝有濆。然则濆者,汝别也。故其下夹水之色,犹流汝阳之名,是或濆、之声相近矣,亦或下合、颍,兼统厥称耳。
又东南,过郾县北。
汝水径奇頟城西,东南流。其城衿带两水,侧背双流。汝水又东南流,径郾县故城北,故魏下邑也。《史记》,楚昭阳伐魏,取郾是也。汝水又东,得醴水口,水出南阳雉县,亦云导源雉衡山,即《山海经》云衡山也。郭景纯以为南岳,非也。马融《广成颂》曰面据衡阴,指谓是山,在雉县界,故世谓之雉衡山。依《山海经》,不言有水。然醴水东流,历唐山下,即高凤所隐之山也。醴水又东南,与皋水合,水发皋山。郭景纯言,或作章山。东流注于醴水。醴水又东南,径唐城北,南入城,而西流出城。城盖因山以即称矣。醴水又屈而东南流,径叶县故城北。《春秋》成公十五年,许迁于叶者也。楚盛,周衰,控霸南土,欲争强中国,多筑列城于北方,以逼华夏,故号此城为万城,或作方字。唐勒《奏土论》曰:我是楚也,世霸南土,自越以至叶垂,弘境万里,故号曰万城也。余按《春秋》,屈完之在召陵,对齐侯曰:楚国方城以为城。杜预曰:方城,山名也,在叶南,未详孰是。楚惠王以封诸梁子高,号曰叶公,城即子高之故邑也。叶公好龙,神龙下之。河东王乔之为叶令也,每月望,常自县诣台朝,帝怪其来数而不见车骑,显宗密令太史伺望之,言其临至,辄有双凫从东南飞来。于是候凫至,举罗张之,但得一只舃。乃诏尚方诊视,则四年中所赐尚书官属履也。每当朝时,叶门下鼓不击自鸣,闻于京师。后天下玉棺于堂前,吏民推排,终不摇动。乔曰:天帝独欲召我耶?乃沐浴服饰寝其中,盖便立覆。宿昔,葬于城东,土自成坟。其夕,县中牛皆流汗喘乏,而人无知者。百姓为立庙,号叶君祠。牧守每班录,312 皆先谒拜之。吏民祈祷,无不如应,若有违犯,亦立能为祟,帝乃迎取其鼓,置都亭下,略无复声焉。或云:即古仙人王乔也,是以干氏书之于神化。醴水又径其城东,与烧车水合,水西出苦菜山,东流侧叶城南,而下注醴水。醴水又东,径叶公庙北。庙前有《沈子高诸梁碑》。旧秦、汉之世,庙道有双阙、几筵。黄巾之乱,残毁颓阙。魏太和、景初中,令长修饰旧宇。后长陈晞以正始元年立碑,碑字破落,遗文殆存,事见其碑。醴水又东,与叶西陂水会。县南有方城山,屈完所谓楚国方城以为城者也。山有涌泉,北流,畜之以为陂,陂塘方二里。陂水散流,又东,径叶城南,而东北注醴水。醴水又东,注叶陂。陂,东西十里,南北七里,二陂,并诸梁之所堨也。陂水又东,径阳县故城北。又东径定陵城东南,与芹沟水合。其水导源叶县,东径阳城北,又东径定陵县南,又东南流注醴。其水径流昆、醴之间,缠络四县之中,疑即吕比所谓岘水也。今于定陵更无别水,惟是水可当之。醴水东径郾县故城南,左入汝。《山海经》曰:醴水东流注于水也。汝水又东南流,径邓城西。《春秋左传》桓公二年,蔡侯、郑伯会于邓者也。汝水又东南流, 水注之。
又东南,过汝甫上蔡县西。
汝南郡,楚之别也。汉高祖四年置,王莽改郡曰汝汾。县故蔡国,周武王克殷,封其弟叔度于蔡。《世本》曰:上蔡也,九江有下蔡,故称上。《竹书纪年》曰:魏章率师及郑师伐楚,取上蔡者也。永初元年,安帝封邓骘为侯国。汝水又东,径悬瓠城北。王智深云:汝南太守周矜起义于悬瓠者,是矣。今豫州刺史汝南郡治。城之西北,汝水枝别左出,西北流,又屈西东转,又西南会汝,形若垂瓠。耆彦云:城北名马湾,中有地数顷,上有栗园,栗小,殊不并固安之实也。然岁贡三百石,以充天府。水渚,即栗州也。树木高茂,望若屯云积气矣。林中有栗堂、射埻,甚间敞,牧宰及英彦,多所游薄。其城上西北隅,高祖以太和中幸悬瓠,平南王肃起高台于小城,建层楼于隅阿,下际水湄,降眺栗渚,左右列谢,四周参差竞峙,奇为佳观也。又东南,过平舆县南。
溱水出浮石岭北青衣山,亦谓之青衣水也。东南径朗陵县故城西。应劭曰:西南有朗陵山,县以氏焉。世祖建武中,封城门校尉臧宫为侯国也。凑水又南屈径其县南,又东北。径北宜春县故城北。王莽更名之为宣孱也。豫章有宜春,故加北矣,元初三年,安帝封后父侍中阎畅为侯国。溱水又东北,径马香城北,又东北,入汝。汝水又东南,径平舆县南,安成县故城北。王莽更名至成也。汉武帝元光六年,封长沙定王子刘苍为侯国矣。汝水又东南,陂水注之,水首受慎水于慎阳县故城南陂。陂水两分,一水自陂北,绕慎阳城四周城堑。颍川荀淑遇县人黄叔度于逆旅,与语移日,曰:子,吾师表也。范奕论曰:黄宪言论风旨,无所传闻。然士君子见之者,靡不服深远,去疵吝,将以道周性全,无得而称乎?堑水又自渎东北流,注北陂。一水自陂东北流,积为鲷陂。陂水又东北,又结而为陂,世谓之窖陂。陂水上承慎阳县北陂,东北流,积而为土陂。陂水又东为窖陂。陂水又东南流,注壁陂。陂水又东北为太陂。陂水又东,入汝。汝水又东南,径平陵亭北,又东南,径阳遂乡北。汝水又东,径栎亭北。《春秋》之棘栎也。杜预曰:汝阴新蔡县东北有栎亭。今城在新蔡故城西北,城北314 半沦水。汝水又东南,径新蔡县故城南。昔管、蔡间王室,放蔡叔而迁之。其子胡,能率德易行,周公举之为卿士,以见于王。王命之以蔡,申吕地也。以奉叔度祀,是为蔡仲矣。宋忠曰:故名其地为新蔡。王莽所谓新迁者也。世祖建武二十八年,封吴国为侯国。《汝南先贤传》曰:新蔡郑敬字次都,为郡功曹。都尉高懿厅事前,有槐树,白露类甘露者。懿问掾属,皆言是甘露。敬独曰:明府政未能致甘露,但树汁耳。懿不悦,托疾而去。汝水又东南,左会澺水。水上承汝水别流于奇颔城东,东南流为练沟,径召陵具西,东南流注,至上蔡西冈北,为黄陵陂。陂水东流,于上蔡冈东为蔡塘。又东径平舆县故城南,为澺水。县,旧沈国也,有沈亭。《春秋》定公四年,蔡灭沈,以沈子嘉归。后楚以为县。《史记》曰:秦将李信攻平舆,败之者也。建武三十年,世祖封铫统为侯国。本妆南郡治,昔费长房为市吏,见王壶公,悬壶郡市,长房从之,因而自远,同入此壶,隐沦仙路,骨谢怀灵,无会而返。虽能役使鬼神,而终同物化。城南里余有神庙,世谓之张明府祠,水旱之不节,则祷之。庙前有《圭碑》,文字紊碎,不可复寻。碑侧有小石函。按《桂阳失贤画赞》,临武张熹字季智,为平舆令。时天大旱,熹躬祷雩,未获嘉应,乃积薪自焚。主薄侯崇,小吏张化,从熹焚焉。火既燎,天灵感应,即澍雨。此熹自焚处也。澺水又东南,左迤为葛陂,陂方数十里。水物含灵,多所苞育。昔费长房投杖于陂而龙变所在也,又劾东海君于是陂矣。陂水东出为鲷水,俗谓之三丈陂,亦曰三严水。水径鲷阳县故城南。应劭曰:县在鲷水之阳。汉明帝水平中,封卫尉阴兴子庆为侯国也。县有葛陵城,建武十五年,更封安城侯铫丹为侯国。城之东北,有楚武王冢,民谓之楚王琴,城北祝社里下,土中得铜鼎,铭曰楚武王,是知武王隧也。鲷陂东注为富水,水积之处,谓之陂塘,津渠交络,枝布川隰矣。澺水自葛陂东南,径新蔡县故城东,而东南流注于汝,汝水又东南径下桑里,左迤为横塘陂,又东北为青陂者也。汝水又东南,径壶丘城北,故陈地。《春秋左传》文公九年,楚侵陈,克壶丘,以其服于晋,是也。汝水又东,与青陂合,水上承慎水于慎阳县之上慎陂,左沟北注马城陂,陂西有黄丘亭。陂水又东,径新息亭北,又东为绸陂。陂水又东,径新息县,结为墙陂。陂水又东,径遂乡东南,而为壁陂。又东为青陂,陂东对大吕亭。《春秋外传》曰:当成周时,南有荆蛮、申、吕,姜姓矣,蔡平侯始封也。西南有小吕亭,故此称大也。侧陂南有青陂庙,庙前有陂,汉灵帝建宁三年,新蔡长河南缑氏李言上请修复青陂。司徒臣训、尚书臣袭,奏可洛阳宫,于青陂东塘南树碑。碑称青陂在县坤地。源起桐柏淮川,别流入于潺湲。径新息墙陂,衍入褒信界,灌溉五百余顷。陂水又东,分为二水,一水南入淮,一水东南径白亭北,又东径吴城南。《史记》,楚惠王二年,子西召太子建之子胜于吴。胜入居之,故曰吴城也。又东北屈径壶丘东,而北流注于汝水,世谓之薄溪水。汝水又东,径褒信县故城北,而东注矣。
又东至原鹿县,汝水又东南径县故城西。杜预《释地》曰:汝阴有原鹿县也。
南入于淮。
所谓汝口,侧水有汝口戍,淮、汝之交会也。


卷二十二  颍水、洧水、潩水、潧水、渠沙水 
颍水出颍川阳城县西北少室山,秦始皇十七年,灭韩,以其地为颍川郡,盖因水以著称者也。汉高帝二年,以为韩国。王莽之左队也。《山海经》曰:颍水出少室山。《地理志》曰:出阳城县阳乾山,今颍水有三源奇发,右水出阳乾山之颍谷。《春秋》颍考叔为其封人。其水东北流。中水导源少室通阜,东南流,径负黍亭东。《春秋》定公六年,郑伐冯、滑、负黍者也。冯敬通《显志赋》曰:求善卷之所在,遇许由于负黍。京相璠曰:负黍在颍川阳城县西南二十七里。世谓之黄城也。亦或谓是水为本,东与右水合。左水出少室南溪,东合颍水,故作者互举二山,言水所发也。《吕氏春秋》曰:卞随耻受汤让,自投此水而死。张显《逸民传》、嵇叔夜《高士传》并言投泂水而死,未知其孰是也。东南过其县南,颍水又东,五渡水注之。其水导源崈高县东北太室东溪。县,汉武帝置,以奉太室山,俗谓之崧阳城。及春夏雨泛,水自山顶而迭相灌澍,崿流相承,为二十八浦也。旸旱辍津,而石潭不耗,道路游憩者,惟得餐饮而已,无敢澡盥其中,苟不如法,必数日不豫,是以行者惮之。山下大潭周数里,而清深肃洁。水中有立石,高十余丈,广二十许步,上甚平整。缁素之士,多泛舟升陟,取畅幽情。其水东南径阳城西,石溜萦委,溯者五涉,故亦谓之五渡水。东南流入颍水。颀水径其县故城南。昔舜禅禹,禹避商均,伯益避启,并于此也。亦周公以上圭测日景处。汉成帝永始元年,封赵临为侯国也。县南对箕山,山上有许由冢,尧所封也。故太史公曰:余登箕山,其上有许由墓焉。山下有牵牛墟。侧颍水有犊泉,是巢父还牛处也。石上犊迹存焉。又有许由庙,碑阙尚存,是汉颍川大守朱宠所立。颍水径其北,东与龙渊水合,其水导源龙渊,东南流,径阳城北,又东南入于颍。颍水又东,平洛溪水注之。水发玉女台下平洛涧,世谓之平洛水。吕忧所谓勺水出阳城山,盖斯水也。又东甫流,注于颍。颍水又东出阳关,历康城南。魏明帝封尚书右仆射卫臻为康乡侯。此即臻封邑也。
又东南过阳翟县北。
颍水东南流,径阳关聚,聚夹水相对,俗谓之东西二土城也。颍水又径上棘城西,又屈径其城南。《春秋左传》襄公十八年,楚师代郑,城上棘以涉颍者也。县西有故堰,堰石崩褫,颓基尚存,旧遏颍水枝流所出也,其故渎东南径三封山北,今无水。渠中又有泉流出焉。时人谓之嵎水,东径三封山东,东南历大陵西连山,、亦曰启筮亭,启享神于大陵之上,即钩台也。《春秋左传》曰夏启有钩台之飨是也。杜预曰;河南阳翟县南有钩台,其水又东南流,水积为陂,陂方十里,俗谓之钩台陂,盖陂指台取名也。又西南流,径夏亭城西,又屈而东南,为郏之靡陂,颍水自堨东径阳翟县故城北。夏禹始封子此,为夏国。故武王至周曰:吾其有夏之居乎?遂营洛邑。徐广曰:河南阳城,阳翟则夏地也。《春秋经》书,秋,郑伯突入于栎。《左传》桓公十五318 年,突杀檀怕而居之。服虔曰:檀伯,郑守栎大夫,栎,郑之大都。宋忠曰:今阳翟也。周未,韩景侯自新郑徙都之。王隐曰:阳翟,本栎也。故颍川郡治也。城西有《郭奉孝碑》,侧水有《九山祠碑》。丛柏犹茂,北枕川流也。又东南过颍阳县西,又东南过颍阴县西南。应劭曰:县在颍水之阳,故邑氏之。按《东观汉记》,汉封车骑将军马防为侯国。防,城门校尉,位在九卿上,绝席。颍水又南径颍乡城西。颍朗县故城在东北,旧许昌典农都尉治也。后改为县,魏明帝封侍中辛毗为侯国也。颍水又东南径柏祠曲东,历冈丘城南,故汾丘城也。《春秋左传》襄公十八年,楚子庚治兵于汾。司马彪曰:襄城县有汾丘。杜预曰:在襄城县之东北也。径繁昌故县北,曲蠡之繁阳亭也。《魏书·国志》曰:文帝以汉献帝延康元年,行至曲蠡,登坛受禅于是地,改元黄初,其年以颍阴之繁阳亭为繁昌县。城内有三台,时人谓之繁昌台。坛前有二碑。昔魏文帝受禅于此,自坛而降。曰:舜禹之事,吾知之矣!故其石铭曰遂于繁昌筑灵坛也。于后其碑六字生金;论者以为司马金行,故曹氏六世,迁魏而事晋也。颍水又东南流,径青陵亭城北。北对青陵陂,陂纵广二十里,颍水径其北,枝入为陂,陂西则漷水注之,水出囊城县之邑城下,东流注于陂,陂水又东人临颍县之狼陂。颍水又东南流,而历临颍县也。
又东南过临颍县南,又东南过汝甫强县北,洧水从河南密县东流注之。
临颍,旧县也。颍水自县西注,小水出焉。《尔雅》曰:颍别为沙。
郭景纯曰:皆大水溢出,别为小水之名也。亦犹江别为沱也。颍水又东南,径皋城北。郎古皋城亭矣。《春秋经》书,公及诸侯盟于皋鼬者也。皋、泽字相似,名与字乖耳。颖水又东径阳城南。《竹书纪年》曰:孙何取阳。强城在东北,颍水不得径其北也。颍水又东南, 水入焉,非洧水也。
又东过西华县北,王莽更名之曰华望也,有东故言西矣。世祖光武皇帝建武中,封邓晨为侯国。汉济北戴封,字平仲,为西华令,遇天旱,慨治功无感,乃积柴坐其上以自焚,火起而大雨暴至,远近叹服。永元十二年,征太常焉。县北有习阳城,颍水径其南,《经》所谓洧水流注之也。
又南过女阳县北。
县故城南有汝水枝流,故县得厥称矣。阚駰曰:本汝水别流,其后枯竭,号曰死汝水,故其字无水。余按汝、女乃方俗之音,故字随读改,未必一如阚氏之说,以穷通损字也。颍水又东,大水注之。又东南径博阳县故城东。城在南顿县北四十里,汉宣帝封邴吉为侯国,王莽更名乐嘉。
又东南过南顿县北, 水从西来流注之。
水于乐嘉县入颍,不至于顿。顿,故顿子国也,周之同姓。、《春秋》僖公二十五年,楚伐陈,纳顿子于顿是也。俗谓之颍阴城,非也。颍水又东南径陈县南,又东南左会交口者也。
又东南至新阳县北,范渠水从西北来注之。
《经》云蒗渠者,百尺沟之名别也。颍水南合交口,新沟自是东出。
颍上有堰,谓之新阳堰,俗谓之山阳堨,非也。新沟自颍北东出,县在水北,故应劭曰:县在新水之阳。今县故城在东,明颍水不出其北,盖:《经》误耳。颍水自堰东南派,径项县故城北。《春秋》僖公十七年,鲁灭项是矣。颍水又东,右合谷水,水上承平乡诸陂,东北径南顿县故城南,侧城东注。《春秋左传》所谓顿迫于陈而奔楚,自顿徒南,故曰南顿也。今其城在顿南三十余里。又东径项城中,楚襄王所郭,以为别都。都内西南小城,项县故城也。旧颍州治。谷水径小城北,又东径魏豫州刺史贾逵祠北。王隐言祠在城北,非也,庙在小城东。昔王凌为宣王司马懿所执,届庙而叹曰:贾梁道王凌,魏之忠臣,惟汝有灵知之。遂仰鸩而死。庙前有碑,碑石金生。干宝曰:黄金可采,为晋中兴之瑞。谷水又东流,出城东注颍。颍水又东,侧颍有公路城。袁术所筑也,故世因以术字名城矣。颍水又东,径临颍城北。城临水,阙南面。又东径云阳二城间,南北翼水,并非所具。又东径丘头。丘头南枕水,《魏书·郡国志》曰:宣王军次丘头,王凌面缚水次,故号武丘矣。颍水又东南流,于故城北,细水注之。水上承阳都陂,陂水枝分,东南出为细水,东径新阳县故城北,又东南径宋县故城北。县即所谓郪丘者也,秦伐魏娶郪丘,谓是邑矣。汉成帝绥和元年,诏封殷后于沛,以存三统。平帝元始四年,改曰来公。章帝建初四年。徙邑于此,故号新郪,为宋公国也,王莽之新延矣。细水又甫径细阳县,新沟水注之。沟首受交口,东北径新阳县故城南。汉高帝六年,封吕青为侯国、王莽更名曰新明也,故应劭曰:县在新水之阳。今无水,故渠旧道而已。东入泽渚,而散流入细。细水又东南径细阳县放城南。王莽更之曰乐庆也。世祖建武中,封岑彭子遵为侯国。细水又东南,积而为陂。谓之次塘,公私引裂,以供田溉。又东南流,屈而西南人颖。《地理志》曰:细水出细阳县东南入颍。颍水又东南流,径胡城东,故胡子国也。《春秋》定公十五年,楚灭胡,以胡子豹归是也。杜预《释地》曰:汝阴县西北有胡城也。颍水又东南,汝水枝津注之。水上承汝水别渎于奇洛,城东三十里,世谓之大水也。东南径召陵县故城南。《春秋左传》僖公四年,齐桓公师于召陵,责楚贡不入,即此处也。城内有大井,径数丈,水至清深。阚駰曰:召者,高也。其地丘墟,井深数丈,故以名焉。又东南径征羌县,故召陵县之安陵乡,安陵亭也。世祖建武十一年,以封中郎将来歙。歙以征定西羌功,故更名征羌也。阙駰引《战国策》,以为秦昭王欲易地,谓此非也。汝水别渎又东径公路台北,台临水方百步,袁术所筑也。汝水别沟又东径西门城,即南利也。汉宣帝封广陵厉王子刘昌为侯国。县北三十里有孰城,号曰北利。故渎出于二利之间,间关女阳之县,世名之死女。县取水名,故曰女阳也,又东径南顿县故城北,又东南径鲖阳城北,又东径邪乡城北,又东径固始县故城北。《地理志》:县,故寝也。寝丘在南,故藉丘名县矣。王莽更名之曰闰治。孙叔敖以土浸薄,取而为封,故能绵嗣。城北犹有《叔敖碑》。建武二年,司空李通,又慕叔敖受邑,故光武以嘉之,更名固始。别汝又东径蔡冈北。冈上有平阳侯相蔡昭冢。昭字叔明,周后稷之胄。冢有石阙,阙前有二碑,碑字论碎,不可复识。羊虎倾低,殆存而已。枝汝又东北流径胡城南”而东历女阴县故城西北。东人颍水。颍水又东径女阴县故城北。《史记·高祖功臣侯者年表》曰:高祖六年,封夏侯婴为侯国。王莽更名之曰汝坟也。县在汝水之阴,故以汝水纳称。城西有一城,故陶丘乡也,汝阴郡治。城外东北隅,有旧台,翼城若丘,俗谓之女郎台,虽经颓毁,犹自广崇,上有一井。疑故陶丘乡,所未详。又东南至慎县,东南入于淮。颍水东南流,左合上吴、百尺二水,俱承次塘细陂,南流注于颖。颖水又东南,江肢水注之。水受大漴陂,陂水南流,积为江陂,南径慎城西,侧城南流入于颍。颍水又径慎县故城南,县故楚邑,白公所居以拒吴。《春秋左传》哀公十六年,吴人伐慎,白公败之。王莽之慎治也。世祖建武中,封刘赐为侯国。颍水又东南径蜩郭东,俗谓之郑城矣。又东南入于淮。《春秋》昭公十二年,楚子狩于州来,次于颍尾。盖颍水之会淮也。
洧水出河南密县西南马领山,水出山下。亦言出颍川阳城山,山在阳城县之东北,盖马领之统目焉。洧水东南流,径一故台南,俗谓之阳子台。又东径马岭坞北,坞在山上,坞下泉流北注,亦谓洧别源也,而入于洧水。洧水东流,绥水会焉,水出方山绥溪,即《山海经》所谓浮戏之山也。东南流,径汉宏农太守张伯雅墓。茔域四周,垒石为垣,隅阿相降,列于绥水之阴庚门,表二石阙,夹对石兽于阙下,累前有石庙,列植三碑。碑云;德字伯雅,河南密人也。碑侧树两石人,有数石柱及诸石兽矣。旧引绥水南人茔域。而为池沼。沼在丑地,皆蟾蠩吐水,石隍承溜。池之南,又建石楼。石庙前,又翼列诸兽。但物谢时沦,调毁殆尽。夫富而非义,比之浮云,况复此乎?王孙、士安斯为达矣。绥水又东南流,径上郭亭南,东南注洧。洧水又东,囊荷水注之。水出北山子节溪,亦谓之子节水,东南流注于洧。洧水又东会沥滴泉,水出深溪之侧,泉流丈余,悬水散注。故世士以沥滴称,南流入洧水也。
东南过其县南。
流水又东南流,与承云二水合,俱出承云山,二源双导,东南流注于洧。世谓之东、西承云水。情水又东,微水注之。水出微山,东北流入于洧。洧水又东径密县故城南。《春秋》谓之新城。《左传》僖公六年,会诸侯伐郑,围新密,郑所以不时城也。今县城东门南侧,有汉密令卓茂祠。茂字子康,南阳宛人。温仁宽雅,恭而有札,人有认其马者,茂与之,曰:若非公马,幸至丞相府归我。遂挽车而去。后马主得马,谢而还之。任汉黄门郎,迁密令,举善而教,口无恶言,教化大行,道不拾遗,蝗不入境,百姓为之立祠,享祀不辍矣。洧水又左会璅泉水,水出玉亭西,北流注于洧水。洧水又东南与马关水合,水出玉亭下,东北流历马关,谓之马关水。又东北注于洧。洧水又东合武定水,水北出武定冈,西南流,又屈而东南流,径零鸟坞西,侧坞东南流。坞侧有水,悬流赴壑,一匹有余,直注涧下,沦积成渊。嬉游者瞩望,奇为佳观。俗人睹此水挂于坞侧,遂目之为零鸟水。东南流入于洧。洧水又东与虎牍山水合,水发南山虎牍溪,东北流入洧。洧水又东南,赤涧水注之。水出武定冈,东南流径皇台冈下。又历冈东,东南流注于洧。南水又东南流,潧水注之。洧水又东南径郐城南。《世本》曰:陆终娶于鬼方氏之妹,谓之女聩,是生六子,孕三年,启其左胁,三人出焉。破其右胁,三人出焉。其四曰莱言,是为郐人,郐人者,郑是也。郑桓公问于史伯曰:王室多难,予安逃死乎?史伯曰:虢、郐,公之民,迁之可也。郑氏东迁,虢、郐献十邑焉。刘桢云:郐在豫州外方之北,北邻于虢,都荣之南,左济右洛,居两水之间,食溱、洧焉。徐广曰:郐在密县,妘姓矣,不得在外方之北也。洧水又东径阴权北,水有梁焉,俗谓是济为参辰口。《左传》襄公九年,晋代郑,济于阴坂,次于阴口而还是也。杜预曰:阴坂,洧津也。服虔曰:水南曰阴。口者,水口也。参、阴声相近,盖传呼之谬耳。又晋居参之分,实沈之土。郑处大辰之野,阙伯之地,军师所次,故济得其名也。又东过郑县南,潧水从西北来注之。
洧水又东径新郑县故城中。《左传》襄公元年,晋韩厥、荀偃帅诸侯伐郑,入其郛,败其徒兵于洧上是也。《竹书纪年》:晋文侯二年,周惠王子多父伐郐,克之,乃居郑父之丘,名之曰郑,是曰桓公。皇甫士安《帝王世纪》云:或言县故有熊氏之墟,黄帝之所都也。郑氏徙居之,故曰新郑矣。城内有遗祠,名曰章乘是也,洧水又东,为洧渊水。《春秋传》曰:龙斗于时门之外洧渊,即此潭也。今洧水自郑城西北入,而东南流,径郑城南。城之南门内,旧外蛇与内蛇斗,内蛇死。六年,大夫傅瑕杀郑子纳厉公,是其征也。水南有郑庄公望母台。庄姜恶公寤生,与段京居。段不弟,姜氏无训。庄公居夫人于城颍。誓曰:不及黄泉,无相见也,故成台以望母,用伸在心之思,感考叔之言,忻大隧之赋:泄泄之慈有嘉,融融之孝得常矣。洧水又东与黄水合,《经》所谓潧水,非也。黄水出太山南黄泉,东南流径华城西。史伯谓郑桓公曰:华,君之土也。韦昭曰:华,国名矣。《史记》秦昭王三十三年,白起攻魏,拔华阳,走芒卯,斩首十五万。可马彪曰:华阳,亭名,在密县。嵇叔夜常采药于山泽。学琴于古人,即此亭也。黄水东南流,又与一水合。水出华城南冈,一源两分,泉流派别,东为七虎涧水,西流即是水也。其水西南流,注于黄水,黄即《春秋》之所谓黄崖也。故杜预云:苑陵县西有黄水者也。又东南流,水侧有二台,谓之积粟台,台东即二水之会也。捕獐山水注之,水东出捕獐山,西流注于黄水。黄水又南至郑城北,东转于城之东北,与黄沟合。水出捕獐山,东南流至郑城东,北入黄水。黄水又东南,径龙渊东南,七里沟水注之,水出隙候亭东南平地,东注,又屈而南流,径升城东,又南历烛城西,即郑大夫烛之武邑也。又南流注于洧水也。
又东南过长社县北,洧水东南流,南濮、北濮二水人焉,濮音仆。洧水又东南与龙渊水合,水出长社县西北,有故沟,上承洧水,水盛则通注龙渊,水减则律渠辍流。其渎中泉,南注东转为渊,绿水平潭,清洁澄深,俯视游鱼,类若乘空矣,所谓渊无潜鳞也,又东径长社县故城北,郑之长葛邑也。《春秋》隐公五年,宋人伐郑,围长葛是也。后社树暴长,故曰长社,魏颖川郡治也。余以景明中出宰兹郡,于南城西侧,修立客馆。版筑既兴,于土下得一树根,甚壮大,疑是故社怪长暴茂者也。稽之故说,县无龙渊水名,盖出近世矣。京相璠《春秋土地名》曰:长社北界有禀水。但是水导于隍堑之中,非北界之所谓。又按京、社地名,并云:长社县北有长葛乡。斯乃县徙于南矣。然则是水即禀水也。其水又东南径棘城北。《左传》所谓楚子伐郑,救齐,次于棘泽者也。禀水又东,左注洧水。洧水又东南,分为二水,其枝水东北流庄沙,一水东径许昌县。故许男国也,姜姓。四岳之后矣。《穆天子传》所谓天子见许男于洧上者也。汉章帝建初四年,封马光为侯国。《春秋佐助期》曰:汉以许失天下。及魏承汉历,遂改名许昌也。城内有景福殿基,魏朗帝太和中造,准价八百余万。洧水又东,人汶仓城内,俗以是水为汶水,故有汶仓之名,非也,盖洧水之邸阁耳。洧水又东径鄢陵县故城南。李奇曰:六国为安陵也。昔秦求易地,唐且受使于此。汉高帝十二年,封都尉朱濞为侯国。王莽更名左亭。洧水又东,鄢陵陂水注之,水出鄢陵南陂东,西南流,注于洧水也。又东南过新汲县东北,洧水自鄢陵东径桐丘南,俗谓之天井陵,又曰冈,非也。洧水又屈而南流,水上有梁,谓之桐门桥,藉桐丘以取称,亦言取桐门亭而著目焉。然不知亭之所在,未之详也。洧水又东南,径桐丘城。《春秋左传》庄公二十八年,楚代郑。郑人将奔桐丘,即此城也。杜预《春秋释地》曰:颖川,许昌城东北,京相璠曰:郑地也。今图无,而城见存,西南去许昌故城可三十五里。俗名之曰堤,其城南即长堤,固洧水之北防也。西面桐丘,其城邪长而不方,盖凭丘之称,即城之名矣。洧水又东径新汲县故城北。汉宣帝神雀二年。置于许之汲乡曲洧城,以河内有汲县,故加新也。城在洧水南堤上。又东,洧水右迤为濩陂。洧水又径匡城南,扶沟之匡亭也。又东,洧水左迤为鸭子陂,谓之大穴口也。又东南过茅城邑之东北。洧水自大穴口,东南径洧阳城,西南径茅城东北又南,左合甲庚沟。沟水上承洧水于大穴口,东北枝分,东径洧阳故城南,俗谓之复阳城,非也。盖洧、复字类音读变。汉建安中,封司空祭酒郭奉孝为侯国,其水又东南,为鸭子陂。陂广十五里,余波南入甲庚沟,西注洧,东北泻沙。洧水又南径一故城西,世谓之思乡城,西去洧水十五里,洧水又右合濩陂水,水上承洧水于新汲县,南径新汲县故城东,又南积而为陂。陂之西北,即长社城。陂水东翼洧堤。西面茅邑,自城北门列筑堤道,迄于此冈,世尚谓之茅冈。即《经》所谓茅城邑也。陂水北出,东入洧津,西北纳异流。
又东过习阳城西,折入于颍。
洧水又东南径辰亭东,俗谓之田城,非也。盖田、辰声相近,城亭音韵联故也。《经》书:鲁宣公十一年,楚子、陈侯、郑伯盟于辰陵也。京相璠曰:颍川长平有故辰亭。杜预曰:长平县东南有辰亭。今此城在长平城西北,长平城在东南,或杜氏之谬,《传》书之误耳。长平东南涝陂北畔,有一阜,东西减里,南北五十许步,俗谓之新亭台。又疑是杜氏所谓辰亭,而未之详也。洧水又南径长平县故城西,王莽之长正也。洧水又南,分为二水,枝分东出,谓之五梁沟,径习阳城北,又东径赭丘南,丘上有故城。《郡国志》曰:长平故属汝南县,有赭丘城,即此城也。又东径长平城南,东注涝陂。洧水南出,谓之鸡笼水,故水会有笼口之名矣。洧水又东径习阳城西,西南折入颍,《地理志》曰:洧水东南至长平县入颍者也。
潩水出河南密县大騩山,大騩即具茨山也。黄帝登具茨之山,升于洪堤上,受《神芝图》于华盖童子,即是山也。潩水出其阿流而为陂,俗谓之玉女池。东径怪山北,《史记》魏襄王六年,败楚于陉山者也。山上有郑祭仲家。冢西有子产墓,累石为方坟,坟东有庙,并东北向郑城。杜元凯言不忘本。际庙旧有一枯柏树,其尘根故株之上,多生稚柏成林,列秀青青,望之,奇可嘉矣。潩水又东南径长社城西北,甫濮、北濮二水出焉。刘澄之著《永初记》云:《水经》,濮水源出大騩山,东北流注泗,卫灵闻音于水上。殊为乖矣。余按《水经》为潩水,不为濮也。是水首受潩水,川渠双引,俱东注洧。洧与之过沙,枝流派乱,互得通称。是以《春秋》昭公九年,迁城父人于陈,以夷濮西田益之。京相璠曰:以夷之濮西田益也。杜预亦言,以夷田在濮水西者与城父人。服虔曰。濮,水名也。且字类音同,津澜邈别,不得为北濮上源。师氏传音于其上矣。潩水又南径钟亭西,又东南径皇台西,又东南径关亭西,又东南径宛亭西,郑大夫宛射犬之故邑也。潩水又南,分为二水,一水南出径胡城东,故颍阴县之狐人亭也。其水南结为陂,谓之胡城陂。潩水自枝渠东径曲强城东,皇陂水注之。水出西北皇台七女冈北,皇陂即古长社县之浊泽也。《史记》魏惠王元年,韩懿侯与赵成侯合军伐魏,战于浊泽是也。其肢北对鸡鸣城,即长社县之浊城也。陂水东南流、径胡泉城北,故颍阴县之狐宗乡也。又东合胡城陂水,水上承皇陂,而东南流注于黄水,谓之合作口。而东径曲强城北,东流入潩水。时人谓之敕水,非也。敕、潩音相类,故字从声变耳。潩水又径东西武亭间,两城相对,疑是古之岸门,史迁所谓走犀首于岸门者也。徐广曰颍阴有岸亭,未知是否。潩水又南径射大城东,即郑公孙射犬城也,盖俗谬耳。潩水又南,径颍阴县故城西。魏明帝封司空陈群为侯国。其水又东南径许昌城南,又东南,与宣梁陂水合,陂上承狼陂。于颍阴城西南,陂南北二十里,东西十里。《春秋左传》曰楚子伐郑,师于狼渊是也。其水东南入许昌县,径巨陵城北,郑地也。《春秋左氏传》庄公十四年,郑厉公获傅瑕于大陵。京相璠曰:颍川临颖县东北二十五里,有故巨陵亭,古大陵也。其水又东积而为陂,谓之宣梁陂也。陂水又东南人潩水。潩水又西南流径陶城西,又东南径陶陂东。
东南入于颍。
潧水出郑县西北平地,潧水出郐城西北鸡络坞下,东南流,径贾复城西。东南流,左合水,水出贾复城东,南流注于潧。潧水又南,左会承云山水,水出西北承云山,东南历浑子冈东注,世谓冈峡为五鸣口,东南流,注于潧。潧水又东南流,历下田川,径郐城西,渭之为柳泉水也。故史伯答桓公曰:君以成周之众,奉辞伐罪,若克虢、郐,君之土也。如前华后河,右洛左济,主芣騩而食潧洧,修典刑以守之,可以少固,即谓此矣。潧水又南,悬流奔壑,崩注丈余,其下积水成潭,广口十许步,渊深难测,又南注于洧,《诗》所谓溱与洧者也,世亦谓之为郐水也。
东过其县北,又东南过其县东,又南入于洧水。自郐、潧东南,更无别渎,不得径新郑而会洧也。郑城东人洧者,黄崖水也。盖《经》误证耳。渠出荥阳北河,东南过中牟县之北。
《风俗通》曰:渠者,水所居也,渠水自河与济乱流,东径荥泽北,东南分济,历中牟县之圃田泽北,与阳武分水。泽多麻黄草,故《述征记》曰:践县境便睹斯卉,穷则知逾界。今虽不能,然谅亦非谬。《诗》所谓东有圃草也。皇武子曰:郑之有原圃,犹秦之有具圃。泽在中牟县西,西限长城,东极官渡,北佩渠水,东西四十许里,南北二十许里。中有沙冈,上下二十四浦,津流径通,渊潭相接,各有名焉。有大渐、小渐、大灰、小灰、义鲁、练秋、大白杨、小白杨、散吓、禹中、羊圈、大鹄。小鹄、龙泽、蜜罗、大哀、小哀、大长、小长、大缩、小缩、伯丘、大盖、牛眼等浦,水盛则北注,渠溢则南播,故《竹书纪年》梁惠成王十年,入河水干甫田,又为大沟而引甫水者也。又有一读,自酸枣受河,导自濮读,历酸枣,径阳武县南出,世谓之十字沟,而属于渠。或谓是读为梁惠之年所开,而不能详也。斯浦乃水泽之所钟,为郑隰之渊薮矣。渠水右合五池沟。沟上承泽水,下流注渠,谓之五池口。魏嘉平三年,司马懿帅中军讨太尉王凌于寿春,自彼而还,帝使侍中韦诞劳军于五池者也。今其地为五池乡矣。渠水又东,不家沟水注之,水出京县东南梅山北溪。《春秋》襄公十八年,楚子冯、公子格率锐师侵费,右回梅山。杜预曰:在密东北。即是山也。其水自溪东北流,径管城西。故管国也,周武王以封管叔矣。成王幼弱,周公摄政,管叔流言曰:公将不利于孺子。公赋《鸱鴞》以伐之,即东山之师是也。《左传》宣公十二年,晋师救郑,楚次管以待之。杜预曰京县东北有管城者是也。俗谓之为管水。又东北分为二水,一水东北流,注黄雀沟,谓之黄渊,渊周百步。其一水东越长城,东北流,水积为渊,南北二里,东西百步,谓之百尺水。北入圃田泽,分为二水。一水东北径东武强城北。《汉书·曹参传》:击羽婴于昆阳,追至叶,还攻武强,因至荥阳:薛瓒云,按武强城在阳武县。即斯城也。汉高帝六年,封骑将庄不识为侯国。又东北流,左注于渠,为不家水口也。一水东流,又屈而甫转,东南注白沟也。渠水又东,清池水注之。水出清阳亭西南平地,东北流,径清阳亭南,东流,即故清人城也。《诗》所谓清人在彭,彭为高克邑也。故杜预《春秋释地》云中牟县西有清阳亭是也。清水又屈而北流,至清口泽,七虎涧水注之。水出华城南冈,一源两派,律川趣别,西人黄雀沟,东为七虎溪,亦谓之为华水也。又东北流,紫光沟水注之,水出华阳城东北,而东流,俗名曰紫光涧。又东北注华水。华水又东径棐城北,即北林亭也。《春秋》丈公与郑伯宴于棐林,子家赋《鸿雁》者也。《春秋》宣公元年,诸侯会于棐林以伐郑,楚救郑,遇于北林。服虔曰:北林,郑南地也。京相璠曰:今荣阳苑陵县有故林乡,在新郑北,故曰北林也。余按林乡故城,在新郑东如北七十许里,苑陵故城在东南五十许里,不得在新郑北也。考京、服之说,并为疏矣。杜预云:荥阳中牟县西南,有林亭,在郑北。今是亭南去新郑县故城四十许里。盖以南有林乡亭,故杜预据是为北林,最为密矣。又以林乡为棐,亦或疑焉。诸侯会棐楚遇于此,宁得知不在是而更指他处也?积古之传,事或不谬矣。又东北径鹿台南冈,北出为七虎涧,东流,期水注之。水出期城西北平地,世号龙渊水。东北流,又北径期城西,又北与七虎涧合,谓之虎溪水,乱流东注,径期城北,东会清口水。司马彪《郡国志》曰:中牟有清口水。即是水也。清水又东北,白沟水注之。水有二源,北水出密之梅山东南,而东径靖城南,与南水合。南水出大山,西北流至靖城南,左注北水,即承水也。《山海经》曰:承水出太山之阴,东北流,注于役水者也。世亦谓之靖涧水。又东北流,大水注之。水出大山东平地。《山海经》曰:太水出于大山之阳,而东南流注于役水,世谓之礼水也。东北径武陵城西,东北流,注于承水。承水又东北人黄瓮涧,北径中阳城西。城内有旧台甚秀,台侧有陂池,池水清深。涧水又东,屈径其城北。《竹书纪年》,梁惠成王十七年,郑釐侯来朝中阳者也。其水东北流,为白沟,又东北径伯禽城北,盖伯禽之鲁往径所由也。屈而南流,东注于清水,即潘岳《都乡碑》所谓自中牟故县以西,西至于清沟,指是水也。乱流东径中牟宰鲁恭祠南,汉和帝时,右扶风鲁恭,字仲康,以大尉掾迁中牟令。政专德化,不任刑罚,吏民敬信,蝗不入境。河南尹袁安疑不实,使部掾肥亲按行之,恭随亲行阡陌,坐桑树下,雉止其旁。有小儿,亲曰:儿何不击雉?曰:将雏。亲起曰:虫不入境,一异;化及鸟鲁,二异;竖子怀仁,三异。久留非优贤,请还。是年,嘉禾生县庭。安美其治,以状上之。征博士、恃中,车驾每出,恭常陪乘。上顾问民政,无所隐讳。故能遗爱自古,祠享来今矣。清沟水又东北径沈清亭,疑即博浪亭也。服虔曰:博浪,阳武南地名也。今有亭,所未详也。历博浪泽,昔张良为韩报仇于秦,以金椎击秦始皇,不中,中其副车于此。又北分为二水,枝津东注清水。清水自枝流北注渠,谓之清沟口。渠水又左径阳武县故城南,东为官渡水,又径曹大祖垒北。有高台,谓之官渡台,渡在中牟,故世又谓之中牟台。建安五年,太祖营官渡,袁绍保阳武。绍连营稍前,依沙堆为屯,东西数十里。公亦分营相御,合战不利。绍进临官渡,起土山地道以逼垒,公亦起高台以捍之,即中牟台也。今台北土山犹在,山之东悉绍旧营,遗基并存。渠水又东径田丰祠北,袁本初惭不纳其言,害之。时人嘉其诚谋,无辜见戮,故立祠于是,用表袁氏覆灭之宜矣。又东,役水注之。水出苑陵县西,隙候亭东。世谓此亭为郤城,非也,盖隙、隙声相近耳。中平陂,世名之埿泉也,即古役水矣。《山海经》曰:役山,役水所出,北流注于河。疑是水也。东北流径苑陵县故城北、东北流径焦城东,阳丘亭西,世谓之焦沟水。《竹书纪年》,梁惠成王十六年,秦公孙壮率师伐郑。围焦城,不克,即此城也。俗谓之驿城,非也。役水自阳丘亭东流,径山民城北,为高榆渊。《竹书纪年》,梁惠成壬十六年,秦公孙壮率师城上积、安陵山民者也。又东北为酢沟,又东北,鲁沟水出焉。役水又东北,埿沟水出焉。又东北为八丈沟,又东,清水枝律注之,水自沈城东派,注于役水。役水又东径曹公垒南,东与沫水合。《山海经》云:沫山,沫水所出,北流注于役。今是水出中牟城西南,疑即沫水也。东北流,径中牟县故城西。昔赵献侯自耿都此。班固云:赵自邯郸徙焉。赵襄子时,佛胖以中牟叛,置鼎于庭,不与己者烹之,田英将寨裳赴鼎处也。薛瓒注《汉书》云:中牟在春秋之时,为郑之堰也。及三卿分晋,则在魏之邦土,赵自漳北,不及此也。《春秋传》曰:卫侯如晋,过中牟,非卫适晋之次也。《汲郡古文》曰:齐师伐赵东鄙,围中牟。此中牟不在赵之东也。按中牟当在漯水之上矣。按《春秋》齐伐晋夷仪,晋车千乘在中牟,卫侯过中牟,中牟人欲伐之。卫褚师圃亡在中牟,曰:卫虽小,其君在,未可胜也。齐师克城而骄,遇之必败,乃败齐师。服虔不列中牟所在。杜预曰今荥阳有中牟,回远,疑为非也。然地理参差,土无常域,随其强弱,自相吞并,疆里流移,宁可一也?兵车所指,径纡难知。自魏徙大梁,赵以中牟易魏。故赵之南界,极于浮水,匪直专漳也。赵自西取后止中牟。齐师伐其东鄙,于宜无嫌,而瓒径指漯水,空言中牟所在,非论证也。汉高帝十一年,封单父圣为侯国。沫水又东北,注于役水。昔魏太祖之背董卓也,间行出中牟,为亭长所录。郭长公《世语》云:为县所拘,功曹请释焉。役水又东北径中牟泽,即郑太叔攻萑蒲之盗于是泽也。其水东流,北屈注渠。《续述征记》所谓自酱魁城到酢沟十里者也,渠水又东流而左会渊水,其水上承圣女陂,陂周二百余步,水无耗竭,湛然清满,而南流注于渠。
渠水又东南而注大梁也:又东至浚仪县,渠水东南径赤城北,戴延之所谓西北有大梁亭,非也。《竹书纪年》,梁惠成王二十八年,穰疵率师及郑孔夜战于梁赫,郑师败逋,即此城也。左则故渎出焉。秦始皇二十二年,王责断故渠,引水东南出以灌大梁,谓之梁沟。又东径大梁城南,本春秋之阳武高阳乡也,于战国为大梁,周梁伯之故居矣。梁伯好土功,大其城,号曰新里。民疲而溃,秦遂取焉。后魏惠王自安邑徙都之,故曰梁耳。《竹书纪年》,梁惠成王六年四月甲寅,徙都于大梁是也。秦灭魏以为县。汉文帝封孝王于梁,孝王以土地下湿,东都睢阳,又改曰梁。自是置县,似大梁城广,居其东城夷门之东。夷门,即侯赢抱关处也。《续述征记》以此城为师旷城,言:郭缘生曾游此邑,践夷门,升吹台,终古之迹,缅焉尽在。余谓此乃梁氏之台门,魏惠之都居,非吹台也,当是误证耳。《西征记》论仪封人即此县,又非也。《竹书纪年》,梁惠成王三十一年三月,为大沟于北郛,以行圃田之水。《陈留风俗传》曰:县北有浚水,像而仪之,故曰浚仪。余谓故汳沙为阴沟矣。浚之故曰浚,其犹《春秋》之浚诛乎?汉氏之浚仪水,无他也,皆变名矣。其国多池沼,时池中出神剑,到今其民像而作之,号大梁氏之剑也。渠水又北屈,分为二水。《续述征记》曰:汳沙到浚仪而分也。汳东注,沙南流。其水更南流,径梁王吹台东。《陈留风俗传》曰:县有仓颉、师旷城,上有列仙之吹台、北有牧泽,泽中出兰蒲,上多俊髦,衿带牧泽,方十五里,俗谓之蒲关泽,即谓此矣。梁王增筑以为吹台,城隍夷灭,略存故迹。今层台孤立于牧泽之右矣,其台方百许步,即阮嗣宗《咏怀诗》所谓驾言发魏都,南向望吹台,萧管有遗音;梁王安在哉?晋世丧乱,乞活凭居,削堕,故基,遂成二层。上基犹方四五十步,高一丈余,世谓之乞活台,又谓之繁台城。渠水于此,有阴沟、鸿沟之称焉。项羽与汉高分王,指是水以为东西之别。苏秦说魏襄王曰:大王之地,南有鸿沟是也。故尉氏县有波乡波亭,鸿沟乡鸿沟亭,皆藉水以立称也。今萧县西亦有鸿沟亭,梁国难阳县东,有鸿口亭,先后谈者,亦指此以为楚,汉之分王,非也。盖《春秋》之所谓红泽者矣。渠水右与氾水合,水上承役水于苑陵县。县。故郑都也。王莽之左亭县也。役水枝津,东派为埿水者也,而世俗谓之氾沟水也。《春秋左传》僖公三十年,晋侯、秦伯围郑。晋军函陵,秦军氾南,所谓东氾者也。其水又东北径中牟县南,又东北径中牟泽,与渊水合,水出中牟县故城北,城有层台。按郭长公《世语》及干宝《晋纪》,并言:中牟县故魏任城玉台下池中,有汉时铁锥,长六尺,入地三尺,头西南指不可动,正月朔自正,以为晋氏中兴之瑞,而今不知所在。或言在中阳城池台,未知焉是。渊水自池西出,屈径其城西,而东南流注于汜。氾水又东径大梁亭南,又东径梁台南,东注渠。渠水又东南流,径开封、县,睢、涣二水出焉。右则新沟注之。其水出逢池,池上承役水于苑陵县,别为鲁沟水,东南流,径开封县故城北。汉高帝十一年,封陶舍为侯国也。《陈留志》称:阮简,字茂宏,为开封令。县侧有劫贼,外白甚急数,简方围棋长啸。吏云:劫急。简曰:局上有劫亦甚急。其耽乐如是。故《语林》曰:王中郎以围棋为坐隐,或亦谓之为手谈,又谓之为棋圣。鲁沟南际富城,东南入百尺陂,即古之逢泽也。徐广《史记音义》曰:秦使公于少官率师会诸侯逢泽,汲郡墓《竹书纪年》作秦孝公会诸侯于逢泽,斯其处也。故应德琏《西征赋》曰:驾衡东指,弭节逢泽。其水东北流为新沟。新沟又东北流,径牛首乡北,谓之牛建城。又东北注渠,即沙水也。音蔡,许慎正作沙音,言水散石也。从水少,水少沙见矣。楚东有沙水,谓此水也。
又屈南至扶沟县北,沙水又东南,径牛首乡东南,鲁沟水出焉,亦谓之宋沟也。又径陈留县故城南。孟康曰:留,郑邑也。后为陈所并,故曰陈留矣。鲁沟水又东南,径圉县故城北。县苦楚难,修其干戈,以圉其患,故曰圉也。或曰边陲之号矣。历万人散。王莽之篡也,东郡太守翟义兴兵讨莽,莽遣奋威将军孙建,击之于圉北,义师大败,尸积万数,血流溢道,号其处为万人散,百姓哀而祠之。又历鲁沟亭,又东南至阳夏县故城西。汉高祖六年,封陈豨为侯国。鲁沟又南人涡,今无水也。沙水又东南径斗城西。《左传》襄公三十年,于产殡伯有尸,其臣葬之于是也。沙水又东南径牛首亭东。《左传》桓公十四年,来人与诸侯伐郑东郊,取牛首者也,俗谓之车牛城矣。沙水又东南,八里沟水出焉。又东南径陈留县裘氏乡裘氏亭西,又径澹台子羽冢东,与八里沟合。按《陈留风俗传》曰:陈留县襄氏乡,有澹台子羽冢,又有子羽祠,民祈祷焉。京相璠曰:今泰山南武城县,有澹台子羽冢,县人也。未知孰是。因其方志所叙,就记缠络焉。沟水上承沙河,而西南流,径牛首亭南,与百尺陂水合。其水自陂,南径开封城东三里冈,左屈而西流南转,注八里沟。又南得野兔水口。水上承西南兔氏亭北野兔陂。郑地也。《春秋传》云:郑伯劳屈生于兔氏者也。陂水东北入八里沟,八里沟水又南径石仓城西,又南径兔氏亭东,又南径召陵亭西,东入沙水。沙水南径扶沟县故城东。县,即颍川之谷平乡也。有扶亭,又有洧水沟,故县有扶沟之名焉。建武元年,汉光武封平狄将军朱鲔为侯国。沙水又东与康沟水合,水首受洧水于长社县东,东北径向冈西,即郑之向乡也。后人遏其上口,今水盛则北注,水耗则辍流。又有长明沟水注之,水出苑陵县故城西北,县有二城,此则西城也。二城以东,悉多陂泽,即古制泽也。京相璠曰:郑地。杜预曰:泽在荥阳苑陵县东,即《春秋》之制田也。故城西北平地出泉,谓之龙渊泉。泉水流径陵丘亭西,又西,重泉水注之,水出城西北平地。泉涌南流,径陵丘亭西,西南注龙渊水。龙渊水又东南,径凡阳亭西,而南入白雁陂。陂在长社县东北,东西七里,南北十里,在林乡之西南。司马彪《郡国志》曰:苑陵有林乡亭。白雁陂又引渎南流,谓之长明沟,东转北屈,又东径向城北,城侧有向冈,《左传》襄公十一年,诸侯伐郑师于向者也,又东,右迤为染泽陂,而东注于蔡泽陂。长明沟水又东径尉氏县故城南,圈称云:尉氏,郑国之东鄙。弊狱官名也。郑大夫尉氏之邑。故栾盈曰:盈将归死于尉氏也。沟渎自是三分,北分为康沟,东径平陆县故城北。高后元年,封楚元王子礼为侯国。建武元年,以户不满三千,罢为尉氏县之陵树乡。又有陵树亭,汉建安中,封尚书苟攸为陵树乡侯。故《陈留风格传》曰:陵树乡,故平陆县也。北有大泽,名曰长乐厩。康沟又东径扶沟县之白亭北。《陈留风俗传》曰:扶沟县有帛乡帛亭,名在七乡十二亭中。康沟又东径少曲亭。《陈留风俗传》曰:尉氏县有少曲亭,俗谓之小城也。又东南径扶沟县故城东。而东南注沙水。沙水又南会南水。其水南流,又分为二永。一水南径关亭东,又东南流,与左水合。其水自枝渎南径召陵亭西,疑即扶沟之亭也。而东南合右水,世以是水与鄢陵陂水双导,亦谓之双沟。又东南人沙水,沙水南与蔡泽陂水合。水出鄢陵城西北。《春秋》成公十六年,晋、楚相遇于鄢陵,吕錡射中共王目,王召养由基,使射杀之。亦子反醉酒自毙处也。陂东西五里,南北十里。陂水东径匡城北,城在新汲县之东北,即扶沟之匡亭也,亭在匡城乡。《春秋》文公元年,诸侯朝晋,卫成公不朝,使孔达侵郑,伐绵訾及匡,即此邑也。今陈留、长垣县南有匡城,即平丘之匡亭也。襄邑又有承匡城,然匡居陈、卫之间,亦往往有异邑矣。陂水又东南至扶沟城北,又东南入沙水。沙水又南,径小扶城西,而东南流也。城即扶沟县之平周亭,东汉和帝永元中,封陈敬王子参为侯国。沙水又东南径大扶城西,城即扶乐故城也。城北二里有《袁良碑》,云:良,陈国扶乐人。后汉世祖建武十六年,更封刘隆为扶乐侯,即此城也。涡水于是分焉,不得在扶沟北,便分为二水也。
其一者,东南过陈县北。
沙水又东南径东华城西,又东南,沙水枝渎,西南达洧,谓之甲庚沟,今无水。沙水又南与广漕渠合,上承庞官陂,云邓艾所开也。虽水流废兴,沟渎尚伙。昔贾逵为魏豫州刺史,通运渠二百里余,亦所谓贾侯渠也。而川渠径复,交锗畛陌,无以辨之。沙水又东径长平县故城北,又东南径陈城北,故陈国也,伏羲、神农并都之。城东北三十许里,犹有羲城实中,舜后妫满,为周陶正。武王赖其器用,妻以元女大姬,而封诸陈,以备三格。太姬好祭祀,故《诗》所谓坎其击鼓,宛丘之下。宛丘在陈城南道东。王隐云:渐欲平,今不知所在矣。楚讨陈,杀夏征舒于栗门,以为夏州后。城之东门内有池,池水东西七十步,南北八十许步,水至清洁而不耗竭,不生鱼草。水中有故台处,《诗》所谓东门之池也。城内有《汉相王君造四县邸碑》,文字剥缺,不可悉识。其略曰:惟兹陈国,故曰淮阳郡云云,清惠著闻,为百姓畏爱,求贤养士,千有余人,赐与田宅。吏舍,自损俸钱,助之成邸。五官掾西华陈骐等二百五人,以延熹二年云云,故其颂曰:修德立功,四县回附。今碑之左右,遗塘尚存,基础犹在。时人不复寻其碑证,云孔子庙学,非也。后楚襄王为秦所灭,徙都于此。丈颖曰西楚矣,三楚斯其一焉。城南郭里,又有一城,名曰淮阳城,子产所置也。汉高祖十一年,以为淮阳国。王莽更名郡为新平,县曰陈陵,故豫州治。王隐《晋书地道记》云,城北有故沙,名之为死沙。而今水流津通,漕运所由矣。沙水又东而南屈,径陈城东,谓之百尺沟。又南分为二水,新沟水出焉。沟水东南流,书水注之,水源上承涝陂。陂在陈城西北,南暨荦城,皆为陂矣。陂水东流,谓之谷水,东径涝城北,王隐曰荦北有谷水是也。牵即怪矣。《经》书公会齐、宋于桂者也。社预曰:柽即牵也,在陈县西北,为非。柽,小城也,在陈郡西南。谷水又东径陈城南,又东流入于新沟水,又东南注于颍,谓之交口,水次有大堰,即古百尺堰也。《魏书》:《国志》曰,司马宣王讨太尉王凌,大军掩至百尺堨,即此堨也。今俗呼之为山阳堰,非也。盖新水首受颖于百尺沟,故堰兼有新阳之名也,以是推之,悟故俗谓之非矣。
又东南至汝南新阳县北,沙水自百尺沟,东径宁平县之故城南。《晋阳秋》称:晋太傅东海王越之东奔也,石勒追之,焚尸于此。数十万众,敛手受害,勒纵骑围射,尸积如山。王夷甫死焉。余谓俊者所以智胜群情,辨者所以丈身袪惑,夷甫虽体荷俊令,口擅雌黄,污辱君亲,获罪羯勒,史官方之华、王,谅为褒矣。沙水又东,积而为陂,谓之阳都陂。明水注之,水上承沙水枝津,东出径汝南郡之宜禄县故城北,王莽之赏都亭也。明水又东北流注于陂,陂水东南流,谓之细水。又东径新阳县北,又东,高陂水东出焉。沙水又东,分为二水,即《春秋》所谓夷濮之水也。枝津北径谯县故城西,侧城人涡。沙水东南径城父县西南,枝津出焉,俗谓之章水。一水东注,即濮水也。俗谓之艾水,东径城父县之故城南,东流注也。
又东南过山桑县北,山桑故城在涡水北,沙水不得径其北明矣,《经》言过北,误也。又东南过龙亢县南,沙水径故城北,又东南径白鹿城北,而东注也。又东南过义成县西,南纻入于淮。
义成县故属沛,后隶九江。沙水东流,注于淮,谓之沙汭。京相璠曰:楚东地也。《春秋左传》昭公二十六年,楚令尹子常以舟师及沙汭而还。社预曰:沙,水名也。


卷二十三  阴沟水、汳水、获水 
阴沟水出河南阳武县蒗渠,阴沟首受大河于卷县,故渎东南径卷县故城南,又东径蒙城北。《史记》秦庄襄王元年,蒙骛击取成皋、荥阳,初置三川郡,疑即骛所筑也,干事未洋。故渎东分为二,世谓之阴沟水。京相璠以为出河之济,又非所究。俱东绝济隧,右渎东南径阳武城北,东南绝长城,径安亭北,又东北会左渎。左渎又东绝长城,径垣雍城南。昔晋文公战胜于楚,周襄王劳之于此。故《春秋》书甲午,至于衡雍,作王宫于践土。《吕氏春秋》曰,尊天子于衡雍者也。《郡国志》曰:卷县有垣雍城,即《史记》所记韩献秦垣雍是也。又东径开光亭南,又东径清阳亭南,又东合右渎,又东南径封丘县,绝济渎,东南至大梁,合蒗渠。梁沟既开,蒗渠故读实兼阴沟,浚仪之称,故云出阳武矣。东南径大梁城北,左屈与梁沟合,俱东南流,同受鸿沟沙水之目。其川流之会,左渎东导者,即水也。盖津源之变名矣。故《经》云:阴沟出蒗渠也。
东南至沛为水。
阴沟始乱蒗,终别于沙,而水出焉。水受沙水于扶沟县。许慎又曰: 水首受淮阳扶沟县蒗渠,不得至沛,方为水也。《尔雅》曰:为洵。郭景纯曰:大水泆为小水也。吕忱曰:洵, 水也。水径大扶城西。城之东北,悉诸袁旧墓,碑字倾低,羊、虎碎折,惟司徒滂、蜀郡大守腾、博平令光碑字所存惟此,自余殆不可寻。水又东南径阳夏县西,又东径邈城北。城实中而西有隙郭。水又东径大棘城南,故鄢之大棘乡也。《春秋》宣公二年,宋华元与郑公子归生战于大棘,获华元。《左传》曰:华元杀羊食士,不及其御,将战,羊斟曰:畴昔之羊,子为政。今日之事,我为政。遂御人郑,故见获焉。后其地为楚庄所并。故圈称曰:大棘,楚地,有楚太子建之坟,及伍员钓台,池沼具存。水又东径安平县故城北。《陈留风俗传》曰:大棘乡,故安平县也。士人敦惷,易以统御。水又东径鹿邑城北,世谓之虎乡城,非也。《春秋》之鸣鹿矣。杜预曰:陈国武平西南,有鹿邑亭是也。城南十里,有《晋中散大夫胡均碑》,元康八年立。水之北,有《汉温令许续碑》。续字嗣公,陈国人也,举贤良,拜议郎,迁温令。延熹中立。水又东径武平县故城北。城之西南七里许,有《汉尚书令虞诩碑》。碑题云:《虞君之碑》,讳诩,字定安,虞仲之后。为朝歌令,武都大守。文字多缺,不复可寻。按范晔《汉书》:诩字升卿,陈国武平人。祖为县狱吏,治存宽恕,尝曰子公为里门,子为丞相,吾虽不及于公,子孙不必不为九卿,故字诩曰升卿,定安盖其幼字也。魏武王初封于此,终以武平华夏矣。水又东径广乡城北。圈称曰:襄邑有蛇丘亭,故广乡矣,改曰广世。后汉顺帝阳嘉四年,封侍中挚填为侯国、即广乡也。水又东径苦县西南,分为二水。枝流东北注于赖城人谷,谓死也。水又东南屈,径苦县故城南。《郡国志》曰:《春秋》之相也。王莽更名之曰赖陵矣。城之四门,列筑驰道,东起赖乡,南自南门,越水直指故台,西面南门,列道径趣广乡道西门驰道。西届武平北门驰道,暨于北台。水又东北屈,至赖乡西,谷水注之。谷水首受涣水于襄邑县东,东径承匡城东。《春秋经》书:夏,叔仲彭生会晋郤缺于承匡。《左传》曰:谋诸侯之从楚者。京相璠曰:今陈留襄邑西三十里,有故承匡城。谷水又东南,径已吾县故城西。《陈留风俗传》曰:县故宋也,杂以陈、楚之地,故梁国宁陵县之徙种龙乡也。以成哀之世,户至八九千,冠带之徒,求置县矣。永元十一年,陈王削地,以大棘乡、直阳乡,十二年,自鄢隶之,命以嘉名曰已吾,犹有陈、楚之俗焉。谷水又东径柘县故城东。《地理志》淮阳之属县也。城内有柘令许君《清德颂》,石碎字紊,惟此文见碑。城西南里许,有《汉阳台令许叔种碑》,光和中立;又有《汉故乐成陵令太尉掾许婴碑》,婴字虞卿,司隶校尉之子,建宁元年立;余碑文字,碎灭不复可观,当似司隶诸碑也。谷水又东径苦县故城中,水泛则四局隍堑,耗则孤律独逝。谷水又东径赖乡城南。其城实中,东北隅有台偏高,俗以是台在谷水北,其城又谓之谷阳台,非也。谷水自此东人水。水又北径老子庙东。庙前有二碑,在南门外。汉桓帝遣中官管霸祠老子,命陈相边韶撰文碑。北有双石阙,甚整顿。石阙南侧,魏文帝黄初三年,经谯所勒;阙北东侧,有孔于庙,庙前有一碑,西面,是陈相鲁国孔畴建和三年立;北则老君庙,庙东院中,有九井焉。又北, 水之侧,又有李母庙。庙在老子庙北,庙前有李母冢。冢东有碑,是永兴元年谯令长沙王阜所立。碑云:老子生于曲、间。水又屈东,径相县故城南。其城卑小实中。边韶《老子碑》文云:老子,楚相县人也。相县虚荒,今属苦,故城犹存,在赖乡之东。水处其阳,疑即此城也。自是无郭以应之。水又东,径谯县故城北。《春秋左传》僖公二十二年,楚成得臣帅师伐陈,遂取谯,城顿而还是也。王莽之延成亭也。魏立谯郡,沇州治。沙水自南枝分,北径谯城西,而北注。水四周城侧,械南有曹嵩冢,冢北有碑,碑北有庙堂,余基尚存,柱础仍在。庙北有二石阙双峙,高一丈六尺,榱栌及柱,皆雕镂云矩,上罦罳已碎。阙北有圭碑,题云,《汉故中常侍长乐大仆特进费亭侯曹君之碑》,延熹三年立。碑阴又刊诏策,二碑文同。夹碑东西,列对两石马,高八尺五寸,石作粗拙,不匹光武隧道所表象马也。有腾兄冢。冢东有碑,题云:《汉故颖川太守曹君之碑》,延熹九年卒,而不刊树碑岁月;坟北有其元子炽冢,冢东有碑,题云,《汉故长水校尉曹君之碑》。历太中大夫、司马、长史、侍中,迁长水,年三十九卒,熹平六年造。炽弟胤家,家东有碑,题云:《汉谒者曹君之碑》,熹平六年立。城东有曹太祖旧宅所在,负郭对廛,侧隍临水。《魏书》曰:太祖作议郎,告疾归乡里,筑室城外,春、夏习读书传,秋、冬射猎,以自娱乐。文帝以汉中平四年生于此,上有青云如车盖,终日乃解,即是处也。后文帝以延康元年幸谯,大飨父老,立坛于故宅。坛前树碑,碑题云:《大飨之碑》。碑之东北, 水南,有谯定王司马士会冢。冢前有碑,晋永嘉三年立。碑南二百许步,有两石柱。高丈余,半下为柬竹交文,作制极工。石榜云:晋故使持节、散骑常侍、都督扬州、江州诸军事、安东大将军、谯定王河内温司马公墓之神道。水又东径朱龟墓北,东南流。冢甫枕道有碑,碑题云,《汉故幽州刺史朱君之碑》。龟字伯灵,光和六年卒官,故吏别驾从事史右北平,无终年化中平二年造。碑阴刊故吏姓名,悉蓟、涿及上谷、北平等人。水东南径层丘北,丘阜独秀,巍然介立,故壁垒所在也。水又东南,径城父县故城北,沙水枝分注之。水上承沙水于思善县,世谓之章水,故有章头之名也。东北流径城父县故城西,侧城东北流,人于。水又东径下城父北。《郡国志》曰:山桑县有下城父聚者也。水又屈径其聚东郎山西,又东南屈,径郎山南。山东有垂惠聚,世谓之礼城。袁山松《郡国志》曰:山桑县有垂惠聚,即此城也。水又东南径阳城北。临侧水,魏大和中,为州治,以盖表为刺史,后罢州立郡,衿带遏戍。水又东南径龙亢县故城南,汉建武十三年,世祖封傅昌为侯国。故语曰:沛国龙亢至山桑者也。水又屈而南流,出石梁。梁石崩褫,夹岸积石,高二丈,水历其间。又东南流,径荆山北,而东流注也。
又东南至下邳淮陵县,入于淮。
水又东,左合北肥水。北肥水出山柔县西北泽薮,东南流,左右翼佩,数源异出同归,盖微脉涓注耳。东南流,径山桑邑南,俗谓之北平城。昔文钦之封山桑侯,疑食邑于此。城东南有一碑,碑文悉破无验,惟碑背放吏姓名尚存;熹平元年义士门生沛国萧刘定兴立。北肥水又东径山桑县故城南,俗谓之都亭,非也。今城内东侧,犹有山亭桀立,陵阜高峻,非洪台所拟。《十三州志》所谓山生于邑,其亭有桑,因以氏县者也。郭城东有《文穆冢碑》,三世二千石,穆郡户曹史,征试博士、太常丞,以明气候,擢拜侍中、右中郎将,迁九江、彭城、陈留三郡,光和中卒。故吏涿郡太守彭城吕虔等立。北肥水又东,积而为陂,谓之瑕陂。陂水又东南径瑕城南。《春秋左传》成公十六年,楚师还及瑕,即此城也。故京相璠曰:瑕,楚地。北肥水又东南径向县故城南。《地理志》曰:故向国也。《世本》曰:许、州、向、申,姜姓也,炎帝后。京相璠曰:向,沛国县,今并属谯国龙亢也。杜预曰:龙亢县东有向城,汉世祖建武十三年,更封富波侯王霸为侯国,即此城也。俗谓之圆城,非。又东南径义成南,世谓之褚城,非。又东入于, 水又东注淮,《经》言下邱淮陵人淮,误矣。
水出阴沟于浚仪县北,阴沟即蒗渠也。亦言汳受旃然水,又云:丹、沁乱流,于武德绝河,南入荥阳合汳,故汳兼丹水之称,河济水断,汳承旃然而东,自王贲灌大梁,水出县南,而不径其北,夏水洪泛,则是渎津通,故渠即阴沟也。于大梁北又曰浚水矣。故圈称著《陈留风俗传》,曰浚水径其北者也。又东,汳水出焉。故《经》云:汳出阴沟于浚仪县北也。汳水东径仓垣城南,即浚仪县之仓垣亭也。城临汳水,陈留相毕邈治此。征东将军荀晞之西也,逸走归京,晞使司马东莱王赞代据仓垣,断留运漕。汳水又东径陈留县之鉼乡亭北。《陈留风俗传》所谓县有鉼乡亭,即斯亭也。汳水又径小黄县故城南。《神仙传》称:灵寿光,扶风人,死于江陵胡罔家,罔殡埋之。后百余日。人有见光于此县,寄书与罔。罔发视之,惟有履存。汳水又东径鸣雁亭南。《春秋左传》成公十六年,卫侯伐郑,至于呜雁者也。杜预《释地》云:在雍丘县西北。今俗人尚谓之为白雁亭。汳水又东径雍丘县故城北,径阳乐城南。《西征记》曰:城在汳北一里,周五里,雍丘县界。汳水又东,有故渠出焉,南通睢水,谓之董生决。或言,董氏作乱,引水南通睢水,故斯水受名焉。今无水。汳水又东,枝津出焉,俗名之为落架口。《西征记》曰:落架,水名也。《续述征记》曰:在董生决下二里。汳水又径外黄县南,又东径莠仓城北。《续述征记》曰:莠仓城去大游墓二十里。又东径大齐城南。《陈留风俗传》曰:外黄县有大齐亭。又东径科城北。《陈留风俗传》曰:县有科禀亭,是则科禀亭也。汳水又东径小齐城南。汳水又南径利望亭南。《风俗传》曰:故成安也。《地理志》:陈留,县名。汉武帝以封韩延年为侯国。汳水又东,龙门故渎出焉。渎旧通睢水,故《西征记》曰:龙门,水名也。门北有土台,高三丈余,上方数十步。汳水又东径济阳考城县故城南,为菑获渠。考城县,周之采邑也,于春秋为戴国矣。《左传》隐公十年秋,宋、卫、蔡伐戴是也。汉高帝十一年秋,封彭祖为侯国。《陈留风俗传》曰:秦之谷县也。后遭汉兵起,邑多灾年,故改曰菑县。王莽更名嘉谷。章帝东巡过县,诏曰:陈留菑县,其名不善。高祖鄙柏人之邑,世宗休闻喜而显获嘉应亨吉元符,嘉皇灵之顾,赐越有光列考武皇,其改菑县曰考城。是渎盖因县以获名矣,汳水又东径宁陵县之沙阳亭北,故沙随国矣。《春秋左传》成公十六年秋,会于沙随,谋伐郑也。杜预《释地》曰:在梁国宁陵县北沙阳亭。是也,世以为堂城,非也。汳水又东径黄蒿坞北。《续途征记》曰:堂城至黄蒿二十里。汳水又东径斜城下。《续述征记》曰:黄蒿到斜城五里。《陈留风俗传》曰:考城县有斜亭。汳水又东径周坞侧。《续述征记》曰:斜城东三里。晋义熙中,刘公遣周超之自彭城缘汳故沟。斩树穿道七百余里,以开水路,停泊于此,故兹坞流称矣。汳水又东径葛城北,故葛伯之国也。孟子曰:葛伯不祀。汤问曰:何为不祀?称:无以供祠祭。遗葛伯,葛伯又不祀,汤又问之,曰:无以供牺牲。汤又遗之,又不祀,汤又问之,曰:无以供粢盛。汤使毫众往为之耕,老弱馈食。葛伯又率民夺之,不授者则杀之。汤乃伐葛。葛于六国属魏。魏安釐王以封公于无忌,号信陵君,其地葛乡,即是城也,在宁陵县西十里。汳水又东径神坑坞,又东径夏侯长坞。《续述征记》曰:夏侯坞至周坞,各相距五里。汳水又东径梁国睢阳县故城北,而东历襄乡坞南。《续述征记》曰:西去夏侯坞二十里。东一里即襄乡浮图也,汳水径其南,汉熹平中某君所立,死,因葬之,其弟刻石树碑,以旌厥德。隧前有狮子、天鹿,累砖作百达柱八所。荒芜颓毁,雕落略尽矣。
又东至梁郡蒙县,为获水,余波南入睢阳城中。
汳水又东径贳城南。俗谓之薄城,非也。阚駰《十三州志》以为贯城也,在蒙县西北。《春秋》僖公二年,齐侯、朱公。江、黄盟于贯,杜预以为贯也。云:贳、贯字相似。贯在齐,谓贯泽也,是矣,非此也。今于此地,更无他城在蒙西北,惟是邑耳。考文准地,贳邑明矣,非毫可知。汳水又东径蒙县故城北,俗谓之小蒙城也。《西征记》:城在汳水南十五六里,即庄周之本色也,为蒙之漆园吏,郭景纯所谓漆园有傲吏者也。悼惠施之没杜门干此邑矣。汳水自县南出,今无复有水,惟睢阳城南侧,有小水南流,人于睢。城南二里,有《汉太傅掾桥载墓碑》。载字元宾,梁国睢阳人也。睢阳公子,熹平五年立。城东百步,有石室,刊云:汉鸿胪桥仁饲。城北五里,有石虎、石柱而无碑志,不知何时建也。汳水又东径大蒙城北。自古不闻有二蒙,疑即蒙毫也,所谓景薄为北毫矣。椒举云:商汤有景毫之命者也。阚駰曰:汤都也。毫本帝喾之墟,在《禹贡》豫州河、洛之间,今河南偃师城西二十里尸乡亭是也。皇甫谧以为考之事实,学者失之。如孟子之言,汤居毫,与葛为邻,是即毫与葛比也。汤地七十里,葛又伯耳,封域有限,而宁陵去偃师八百里,不得童子馈饷而为之耕。今梁国自有二毫,南毫在谷熟,北毫在蒙,非偃师也。古文《仲虺之浩》曰:葛伯仇饷,征自葛始,即孟子之言是也。崔駰曰:汤冢在济阴薄县北。《皇览》曰:薄城北郭东三里,平地有汤冢。冢四方,方各十步,高七尺,上平也。汉哀帝建平元年,大司空史部长卿按行水灾,因行汤冢,在汉属扶风,今征之回渠亭,有汤池征陌是也。然不经见,难得而详。按《秦宁公本纪》云:二年伐汤,三年与毫战,毫王奔戎,遂灭汤。然则周桓王时,自有毫王号汤,为秦所灭,乃西戎之国,葬于征者也,非殷汤矣。刘向言殷汤无葬处为疑。杜预曰:梁国蒙县北,有薄伐城,城中有成汤冢,其西有箕子冢,今城内有故冢方坟,疑即社元凯之所谓汤冢者也。而世谓之王子乔冢。冢侧有碑,题云:《仙人王子乔碑》。曰:王子乔者,盖上世之真人,闻其仙不知兴何代也。博问道家,或言颖川,或言产蒙,初建此城,则有斯丘,传承先民,曰王氏墓。暨于永和之元年冬十二月,当腊之时夜,上有哭声,其音甚哀。附居者王伯怪之,明则祭而察焉。时天鸿雪,下无人径,有大鸟迹在祭祀处,左右咸以为神。其后有人著大冠,绛单衣,杖竹,立冢前,呼采薪孺子伊永昌,曰:我,王予乔也,勿得取吾坟上树也。忽然不见。时令泰山万熹稽故老之言,感精瑞之应,乃造灵庙,以休厥神。于是好道之俦,自远方集,或弦琴以歌太一,以罩思以历丹丘,知至德之宅兆,实真人之祖先。延熹八年秋八月,皇帝遣使者奉牺牲致礼,祠濯之敬,肃如也。国相东莱王璋,字伯仪,以为神圣所兴,必有铭表,乃与长史边乾遂树之玄石,纪颂遗烈。观其碑文,意似非远,既在径见,不能不书存耳。
获水出汳水于梁郡蒙县北,《汉书·地理志》曰:获水首受甾获渠,亦兼丹水之称也。《竹书纪年》曰:未杀其大夫皇瑗于丹水之上。又曰:宋大水,丹水壅不流。盖汳水之变名也。获水自蒙东出,水南有《汉故绎幕令匡碑》。匡字公辅,鲁府君之少子也。碑字碎落,不可寻识,竟不知所立岁月也。获水又东径长乐固北,己氏县南,东南流径于蒙泽。《十三州志》曰:蒙泽在县东。《春秋》庆公十二年,宋万与公争博,杀闵公于斯泽矣。获水又东,径虞县故城北,古虞国也。昔夏少康逃奔有虞,为之庖正。虞思于是妻之以二姚者也,王莽之陈定亭也。城东有《汉司徒盛允墓碑》。允字伯世,梁国虞人也。其先奭氏,至汉中叶,避孝元皇帝讳,改姓曰盛。世济其美,以迄于公。察孝廉,除郎,累迁司空、司徒,延熹中立。墓中有石庙,庙字倾颓,基构可寻。获水又东南径空桐泽北。泽在虞城东南。《春秋》哀公二十六年冬,宋景公游于空泽,辛已,卒于连中。大尹左师兴空泽之士,千甲,奉公自空桐入,如沃宫者矣。获水又东径龙谯固,又东合黄水口,水上承黄陂,下注获水。获水又东入栎林,世渭之九里柞。获水又东南径下邑县故城北。楚考烈王灭鲁,顷公亡,迁下邑。又楚、汉彭城之战,吕后兄泽,军于下邑。高祖败,还从泽军。子房肇捐地之策,收垓下之师,陆机所谓即下邑者也。王莽更名下治矣。获水又东径砀县故城北。应劭曰:县有砀山,山在东,出文石,秦立砀郡,盖取山之名也,玉莽之节砀县也。山有梁孝王墓,其冢斩山作郭,穿石为藏,行一里,到藏中,有数尺水,水有大鲤鱼,黎民谓藏有神,不敢犯神。凡到藏皆洁斋而进,不斋者至藏,辄有兽噬其足,兽难得见,见者云似狗,所未详也。山上有梁孝王祠。获水又东,谷水注之,上承砀陂,陂中有香城,城在四水之中。承诸陂散流,为零水、滚水、清水也,积而成潭,谓之砀水。赵人有琴高者,以善鼓琴为康王舍人,行彭涓之术,浮游砀郡间二百余年,后入砀水中取龙子,与弟子期曰:皆洁斋待于水旁,设屋祠,果乘赤鲤鱼出,人坐祠中,砀中有可万人观之,留月余,复人水也。陂水东注,谓之谷水,东径安山北,即砀北山也。山有陈胜墓,秦乱,首兵伐秦,弗终厥谋死,葬于砀,谥曰隐王也。谷水又东北注于获水。获水又东历蓝田乡郭,又东径梁国抒秋县故城南,王莽之予秋也。获水又东历洪沟,东注,南北各一沟,沟首对获,世谓之鸿沟,非也。《春秋》昭公八年,秋,蒐于红。杜预曰:沛国萧县西有红亭,即《地理志》之县也。景帝三年,封楚元王子富为侯国,王莽之所谓贡矣。盖沟名音同,非楚汉所分也。又东过萧县南,睢水北流注之。
萧县南对山,世谓之萧城南山也。戴延之谓之同孝山,云:取汉阳城侯刘德所居里名目山也。刘澄之云:县南有冒山。未详孰是也。山有箕谷,谷水北流注获,世谓之西流水,言水上承梧桐陂,陂水西流,因以为名也。余尝径萧邑,城右惟是水北注获水,更无别水,疑即《经》所谓睢水也。城东西及南三面,临侧获水,故沛郡治,县亦同居矣。城甫旧有石桥耗处,积石为梁,高二丈,今荒毁殆尽,亦不具谁所造也。县本萧叔国,宋附庸,楚灭之。《春秋》宣公十二年,楚伐萧,萧溃,申公巫臣曰:师人多寒,王巡三军抚之,士同挟纩,盖恩使之然矣。萧女聘齐为顷公之母,郤克所谓萧同叔子也。获水又东历龙城,不知谁所创筑也。获水又东径同孝山北。山阴有楚元王冢,上圆下方,累石为之,高十余丈,广百许步,经十余坟,悉结石也。获水又东,净净沟水注之。水上承梧桐陂。西北流,即刘中书澄之所谓白沟水也。又北入于获,俗名之曰净净沟也。又东至彭城县北,东入干泗。
获水自净净沟东径阿育王寺北,或言楚王英所造,未所详也。盖遵育王之遗法,因以名焉。与安陂水合,水上承安陂余波,北径阿育王寺侧,水上有梁,谓之玄注桥,水旁有石墓,宿经开发,石作工奇,殊为壮构,而不知谁冢,疑即澄之所谓凌冢也。水北流,注于获。获水又东径弥黎城北。刘澄之《永初记》所谓城之西南,有弥黎城者也。获水于彭城西南,回而北流,径彭城。城西北旧有楚大夫龚胜宅,即楚老哭胜处也。获水又东,转径城北而东注泗。水北三里有石冢被开,传言楚元王之孙刘向冢,未详是否。城,即殷大夫老彭之国也。于春秋为宋地,楚伐宋并之,以封鱼石崔于。季珪《述初赋》曰:想黄公于邳圯,勤鱼石于彭城。即是县也。孟康曰:旧名江陵,为南楚,陈为东楚,彭城为西楚。文颖曰:彭城,故东楚也。项羽都焉,谓之西楚,汉祖定天下,以为楚郡,封弟交为楚王都之。宣帝地节元年,更为彭城郡。王莽更之曰和乐郡也,徐州治。城内有汉司徒袁安、魏中郎将徐庶等数碑,并列植于街右,咸曾为楚相也。大城之内有金城。东北小城,刘公更开广之,皆垒石高四丈,列堑环之。小城西又有一城,是大司马琅邪王所修,因项羽故台经始,即构宫观门阁,惟新厥制。义熙十二年,霖雨骤澍,汳水暴长,城遂崩坏。冠军将军,彭城刘公之子也,登更筑之,悉以砖垒,宏壮坚峻,楼橹赫奕,南北所无。宋平北将军、徐州刺史河东薛安都举城归魏,魏遣博陵公尉苟仁、城阳公孔怕恭援之,邑阁如初,观不异昔。自后毁撤,一时俱尽。间遗工雕镂,尚存龙云逞势,奇为精妙矣。城之东北角,起层楼于其上,号曰彭祖楼。《地理志》曰:彭城县,古彭祖国也。《世本》曰:陆终之子,其三曰籛,是为彭祖,彭祖城是也。下曰彭租冢。彭祖长年八百,绵寿永世,于此有冢,盖亦元极之化矣。其楼之侧,襟汳带泗,东北为二水之会也。耸望川原,极目清野,斯为佳处矣。


卷二十四  睢水、瓠子河、汶水 
睢水出梁郡鄢县,睢水出陈留县西蒗渠,东北流。《地理志》曰:睢水首受陈留浚仪狼汤水也,《经》言出鄢,非矣。又东径高阳故亭北。俗谓之陈留北城,非也。苏林曰:高阳者,陈留北县也。按在留故乡聚名也。有《汉广野君庙碑》。延熹六年十二月,雍丘令董生,仰余徽于千载,遵茂美于绝代,命县人长照为文,用章不朽之德,其略云:辍洗分餐,咨谋帝猷,陈郑有啄鹿之功,海岱无牧野之战,大康华夏,绥静黎物,生民以来,功盛莫崇,今故字无闻,而单碑介立矣。《陈留风俗传》曰:郦氏居于高阳,沛公攻陈留县,郦食其有功,封高阳侯。有郦峻,字文山,官至公府掾。大将军商有功,食邑于涿,故自陈留徙涿。县有鉼亭鉼乡。建武二年,世祖封王常为侯国也。睢水又东径雍丘县故城北。县,旧杞国也。殷汤周武以封夏后,继禹之嗣。楚灭杞,秦以为县。圈称曰:县有五陵之名、故以氏县矣。城内有夏后祠。昔在二代,享祀不辍。秦始皇因筑其表为大城,而以县焉。睢水又东,水积成湖,俗谓之白羊陂。陂方四十里,右则奸梁陂水注之。其水上承肢水,东北径雍丘城北,又东分为两渎、谓之双沟,俱入白羊陂。陂水东合洛架口,水上承汳水,谓之洛架水,东南流入于睢水。睢水又东径襄邑县故城北,又东径雍丘城北。睢水又东径宁陵县故城南。故葛伯国也,王莽改曰康善矣。历鄢县北,二城南北相去五十里,故《经》有出鄢之文。城东七里,水次有单父令杨彦,尚书郎杨禅字文节,兄弟二碑,汉光和中立也。东过睢阳县南,睢水又东径横城北。《春秋左传》昭公二十一年,乐大心御华向于横。
杜预曰:梁国睢阳县南,有横亭,今在睢阳县西南,世谓之光城,盖光、横声相近,习传之非也。睢水又径新城北,即宋之新城亭也。《春秋左传》文公十四年,公会宋公、陈侯、卫侯、郑伯、许男、曹伯、晋赵盾,盟于新城者也。睢水又东径高乡亭北,又东径毫城北。南毫也,即汤所都矣。睢水又东径睢阳县故城南。周成王封微子启于宋以嗣殷后,为宋都也。昔宋元君梦江使乘辎车,被绣衣,而谒于元君,元君感卫平之言,而求之于泉阳,男子余且,献神龟于此矣。秦始皇二十二年,以为砀郡。汉高祖尝以沛公为砀郡长。天下既定,五年为梁国。文帝十二年封少子武为梁王,太后之爱子,景帝宠弟也。是以警卫貂侍,饰同天子,藏珍积宝,多拟京师,招延豪杰,士咸归之,长卿之徒,免官来游。广睢阳城七十里,大治宫观,台苑屏榭,势并皇居,其所经构也。役夫流唱.必曰《睢阳曲》,创传由此始也。城西门,即寇先鼓琴处也。先好钓,居睢水旁,宋景公问道不告,杀之。后十年,止此门,鼓琴而去。宋人家家奉事之。南门曰卢门也。春秋华氏居卢门,里叛。杜预曰:卢门,宋城南门也。司马彪《郡国志》曰:睢阳县有卢门亭,城内有高台,甚秀广,巍然介立,超焉独上,谓之蠡台、亦曰升台焉。当昔全盛之时,故与云霞竟远矣。《续述征记》曰:回道似蠡,故谓之蠡台。非也。余按《阙子》,称宋景公使工人为弓,九年乃成。公曰:何其迟也?对曰,臣不复见君矣,臣之精尽于弓矣。献弓而归,三日而死。景公登虎圈之台,援弓东面而射之,矢逾于盂霜之山,集于彭城之东,余势逸劲,犹饮羽于石梁。然则蠡台即是虎圈台也,盖宋世牢虎所在矣。晋太和中,大司马桓温入河,命豫州刺史袁真开石门。鲜卑坚戍此台,真顿甲坚城之下,不果而还。蠡台如西,又有一台,俗谓之女郎台。台之西北城中,有凉马台。台东有曲池,池北列两钓台,水周六七百步。蠡台直东,又有一台,世谓之雀台也。城内东西道北,有晋梁王妃王氏陵表,并列二碎,碑云:妃讳粲,字女仪,东莱曲城人也,齐北海府君之孙司空东武景侯之季女。咸熙元年,嫔于司马氏,泰始二年,妃于国。太康五年薨。营陵于新蒙之,大康九年立碑。东即梁王之吹台也。基陛阶础尚在,今建追明寺故宫东,即安梁之旧地也。齐周五六百步,水列钓台。池东又有一台,世谓之清泠台。北城凭隅,又结一池台,晋灼曰:或说平台在城中东北角,亦或言兔园在平台侧。如淳曰:平台,离宫所在。今城东二十里有台,宽广而不甚极高,俗谓之平台。余按《汉书·梁孝王传》称王以功亲为大国,筑东苑,方三百里,广睢阳城七十里,大治宫室,为复道,自宫连属于平台三十余里。复道自宫东出杨之门。左阳门,即睢阳东门也,连属于平台则近矣,属之城隅则不能,是知平台不在城中也。梁王与邹枚、司马相如之徒,极游于其上。故齐随郡王《山居序》所谓西园多士,平台盛宾,邹马之客咸在,《伐木》之歌屡陈;是用追芳昔娱,神游千古,故亦一时之盛事。谢氏赋雪,亦曰梁王不悦,游于兔园。今也歇堂沦字,律管埋音,孤基块立,无复曩日之望矣。城北五六里,便得汉大尉桥玄墓,冢东有庙,即曹氏盂德亲酹处。操本微素,尝候于玄,玄曰:天下将乱,能安之者,其在君乎?操感知己,后经玄墓,祭云:操以顽质,见纳君子,士死知己,怀此无忘。又承约言,徂没之后,路有经由,不以斗酒只鸡,过相沃酹,车过三步,腹痛勿怨。虽临时戏言,非至亲笃好,胡肯为此辞哉!凄怆致祭,以申宿怀。冢列数碑,一是汉朝群儒,英才哲士,感桥氏德行之美。乃共刊石立碑,以示后世。一碑是故吏司徒博陵崔列、廷尉河南吴整等,以为至德在己,扬之由人,苟不皦述,夫何考焉?乃共勒嘉石,昭明芳烈。一碑是陇西枹罕北次陌砀守长骘,为左尉汉阳豲道赵冯孝高,以桥公尝牧凉州,感三纲之义,慕将顺之节,以为公之勋美,宜宣旧邦、乃树碑颂,以昭令德。光和七年,主记掾李友字仲僚作碑文。碑阴有《右鼎文》,建宁三年拜司空。又有《中鼎文》。建宁四年拜司徒。又有《左鼎文》,光和元年拜大尉。《鼎铭》文曰:故臣门人,相与述公之行,咨度体则,文德铭于三鼎,武功勒于征钺,书于碑阴,以昭光懿。又有《钺文》称,是用镂石假象,作兹征钺军鼓,陈之于东阶,亦以昭公之文武之勋焉。庙南列二往,柱东有二石羊,羊北有二石虎。庙前东北,有石驼,驼西北有二石马,皆高大,亦不甚雕毁。惟庙颓构,粗传遗墉,石鼓仍存,钺今不知所在。睢水于城之阳,积而为逢洪陂。陂之西南有陂,又东合明水。水上承城南大池,池周千步,南流会睢,谓之明水,绝睢注涣。睢水又东南流,历于竹圃。水次绿竹萌渚,菁菁实望,世人言粱王竹园也。睢水又东,径谷熟县故城北。睢水又东,蕲水出焉。睢水又东,径粟县故城北。《地理志》曰:侯国也。王莽曰成富。睢水又东,径太丘县故城北。《地理志》曰:故敬丘也。汉武帝元朔三年,封鲁恭王子节侯刘政为侯国。汉明帝更从今名。《列仙传》曰:仙人文宾,邑人,卖靴履为业,以正月朔日,会故妪于乡亭西社,教令服食不老。即此处矣。睢水又东,径芒县故城北。汉高帝六年,封耏跖为侯国。王莽之传治。世祖改曰临睢。城西二里,水南有《豫州从事皇毓碑》,殒身州牧,阴君之罪,时年二十五。临睢长平舆李君,二千石丞,纶氏夏文则,高其行而悼其殒,州国咨嗟,旌闾表墓,昭叙令德,式示后人。城内有《临睢长左冯翊王君碑》,善有治功,累迁广汉属国都尉,吏民恩德。县人公府掾陈盛孙,郎中兑定兴、刘伯鄜等,共立石表政,以刊远绩。县北与砀县分水。有砀山。芒、砀二县之间,山泽深固,多怀神智。有仙者涓子、主柱,并隐砀山得道,汉高祖隐之。吕后望气知之,即于是处也。京房《易候》曰:何以知贤人隐?师曰:视囚方常有大云,五色具而不雨,其下贤人隐矣。
又东过相县南,屈从城北,东流,当萧县南,入于陂。相县,故宋地也。秦始皇二十三年,以为泗水郡。汉高帝四年,改曰沛郡,治此。双武帝元狩六年,封南越桂林监居翁为侯国,曰湘成也。工莽更名,郡曰吾符,县曰吾符亭。睢水东径石马亭。亭西有汉故伙波将军马援墓。睢水又东径相县故城南。宋共公之所都也。国府园中,犹有伯姬黄堂基。堂夜被火,左右曰:夫人少避。伯姬曰:妇人之义,保傅不具,夜不下堂。遂遇火而死,斯堂即伯姬焚死处也。城西有伯姬冢。昔郑浑为沛郡太守,于萧、相二县,兴陂堰,民赖其利,刻石颂之,号曰郑陂。睢水又左合白沟水,水上承梧桐陂,陂侧有梧桐山,陂水西南流,径相城东,而南流注于睢。睢盛则北流入于陂,陂溢则西北注于睢,出入回环,更相通注,故《经》有入陂之文。睢水又东径彭城郡之灵壁东,东南流。《汉书》,项羽败汉王于灵壁东,即此处也。又云东通谷泗。服虔曰:水名也,在沛国相界,未详。睢水径谷熟,两分睢水而为蕲水,故二水所在枝分,通谓兼称。谷水之名,盖因地变,然则谷水即睢水也。又云汉军之败也,睢水为之不流。睢水又东南径竹县故城南。《地理志》曰:王莽之笃亭也。李奇曰:今竹邑县也。睢水又东与澕湖水合,水上承甾丘县之渒陂,南北百余里,东西四十里,东至朝解亭,西届彭城甾丘县之故城东。王莽更名之曰善丘矣。其水自陂南系于睢水,又东睢水南,八丈故沟水注之。水上承蕲水而北会睢水,又东径符离县故城北。汉武帝元狩四年,封路博德为侯国,王莽之符合也。睢水又东径临淮郡之取虑县故城北。昔汝南步游张少失其母,及为县令,遇母于此,乃使良马踟蹰,轻轩罔进,顾访病姬,乃其母也。诚愿宿凭,而冥感昭征矣。睢水又东合乌慈水,水出县西南乌慈渚,潭涨东北流,与长直故渎合。读旧上承蕲水,北流八十五里,注乌慈水。乌慈水又东径取虑县南,又东屈径其城东,而北流注于睢。睢水又东径睢陵县故城北。汉武帝元朔元年,封江都易王子刘楚为侯国,王莽之睢陆也。睢水又东,与潼水故渎会。旧上承潼县西南潼陂,东北流,径潼县故城北,又东北径睢陵县下会睢水。睢水又东南流,径下相县故城南。高祖十二年,封庄侯泠耳为侯国。应劭曰:相水出沛国相县,故此加下也。然则相又是睢水之别名也,东南流入于泗,谓之睢口,《经》止萧县,非也。所谓得其一而亡其二矣。瓠子河出东郡濮阳县北河。县北十里,即瓠河口也。《尚书·禹贡》:雷夏既泽,雍沮会同。《尔雅》曰:水自河出为雍。许慎曰:雍者,河雍水也。暨汉武帝元光三年,河水南泆,漂害民居。元封二年,上使汲仁、郭昌发卒数万人,塞瓠子决河,于是上自万里沙还,临决河,沉白马、玉壁,令群臣将军以下,皆负薪填决河。上悼功之不成,乃作歌曰:瓠子决兮将奈何?浩浩洋洋,虑殚为河。弹为河兮地不宁,功无已时兮吾山平。吾山平兮巨野溢,鱼沸郁兮柏冬日。正道弛兮离常流,蚊龙骋兮放远游。归旧川兮神哉沛,不封禅兮安知外。皇谓河公兮何不仁,泛滥不止兮愁吾人。啮桑浮兮淮泗满,久不返兮水维缓。一曰河汤汤兮激潺潺浸,北渡回号汛流难。搴长茭兮湛美玉,河公许兮薪不属。薪不属兮卫人罪。烧萧条兮噫乎何以御水!隤竹林兮楗石菑,宣防塞兮万福来。于是卒塞瓠子口,筑宫于其上,名曰宣房宫。故亦谓瓠子堰为宣房堰,而水亦以瓠子受名焉。平帝已后,未及修理,河水东浸,日月弥广。永平十二年,显宗诏乐浪人王景治渠筑堤,起自荥阳,东至千乘,一千余里。景乃防遏冲要,疏决壅积,瓠子之水,绝而不通,惟沟渎存焉。河水旧东决,径濮阳城东北。故卫也,帝颛顼之墟。昔颛顼自穷桑徙此,号曰商丘,或谓之帝丘,本陶唐氏火正阏伯之所居,亦夏伯昆吾之都,殷相土又都之。故《春秋传》曰:阏伯居商丘,相土因之,是也。卫成公自楚丘迁此。秦始皇徙卫君角于野王,置东郡,治濮阳县,濮水径其南,故曰濮阳也。章邯守濮阳,环之以水。张晏曰:依河水自固。又东径咸城南。《春秋》僖公十三年,夏,会于咸。杜预曰:东郡濮阳县东南,有咸城者是也。瓠子故渎,又东径桃城南。《春秋传》曰:分曹地,自洮以南,东傅于济,尽曹地也。今鄄城西南五十里有姚城,或谓之洮也。瓠渎又东南径清丘北。《春秋》宣公十二年,《经》书楚灭萧,晋人、宋、卫、曹同盟于清丘。京相璠曰:在今东郡濮阳县东南三十里,魏东都尉治。
东至济阴句阳县,为新沟。
瓠河故渎,又东径句阳县之小成阳,城北侧渎。《帝王世纪》曰:尧葬济阴成阳西北四十里,是为谷林,墨子以为尧堂高三尺,土阶三等,北教八狄,道死,葬蛩山之阴。《山海经》曰:尧葬狄山之阳,一名崇山。二说各殊,以为成阳,近是尧冢也。余按小成阳在成阳西北半里许实中,俗喭以为囚尧城,士安盖以是为尧冢也。瓠子北有部关县故城:县有羊里亭,瓠河径其南,为羊里水,盖资城地而变名,犹《经》有新沟之异称矣。黄初中,贾逵为豫州刺史,与诸将征吴于洞浦,有功,魏封逵为羊里亭侯,邑四百户,即斯亭也。俗名之羊子城,非也,盖韵近字转耳。又东,右会濮水枝津,水上承濮渠,东径沮丘城南。京相璠曰:今濮阳城西南十五里,有沮丘城,六国时沮楚同音,以为楚丘,非也。又东径浚城南,西北去濮阳三十五里。城侧有寒泉冈,即《诗》所谓爱有寒泉,在浚之下。世谓之高平渠,非也。京相璠曰:濮水故道,在濮阳南者也。又东径句阳县西,句渎出焉。濮水枝渠又东北径句阳县之小成阳东垂亭西,而北入瓠河。《地理志》曰:濮水首受泲沛于封丘县东北,至都关,入羊里水者也。又按《地理志》:山阳郡有都关县,今其城在廪丘城西。考地志,句阳、廪丘,俱属济阴,则都关无隶山阳理。又按《地理志》,成都亦是山阳之属县矣。而京、杜考地验城,又并言在廪丘城南,推此而论,似《地理志》之误矣,或亦疆理参差所未详。瓠渎又东径垂亭北。《春秋》隐公八年,宋公、卫侯遇于犬丘,《经》书垂也。京相璠曰:今济阴句阳县小成阳东五里,有故垂亭者也。
又东北过廪丘县,为濮水。
瓠河又左径雷泽北,其泽薮在大成阳县故城西北十余里。昔华胥履大迹处也。其陂东西二十余里,南北十五里,即舜所渔也。泽之东南即成阳县,故《史记》曰:武王封弟叔武于成。应劭曰:其后乃迁于成之阳,故曰成阳也。《地理志》曰;成阳有尧冢灵台,今成阳城西二里,有尧陵,陵南一里,有尧母庆都陵。于城为西南,称曰灵台,乡曰崇仁,邑号修义,皆立庙。四周列水,潭而不流,水泽通泉,泉不耗竭,至丰鱼笋,不敢采捕。前并列数碑,栝柏数株,檀马成林,二陵南北,列驰道径通,皆以砖砌之,尚修整。尧陵东城西五十余步,中山夫人祠,尧妃也。石壁阶墀仍旧,南、西、北三面,长栎联荫,扶疏里余。中山夫人祠南,有仲山甫冢,冢西有石庙,羊虎倾低,破碎略尽,于城为西南,在灵台之东北。按郭缘生《述征记》,自汉迄晋二千石及丞尉,多刊石,述叙尧即位至永嘉三年二千七百二十有一载,记于尧妃祠。见汉建宁五年五月,成阳令管遵所立碑,文云:尧陵北,仲山甫墓南,二冢间有伍员祠。晋大安中立一碑,是永兴中建,今碑祠并无处所。又言尧陵在城南九里,中山夫人祠在城南二里,东南六里尧母庆都冢,尧陵北二里,有仲山甫墓。考地验状,咸为疏僻,盖闻疑书疑耳。雷泽西南十许里有小山,孤立峻上,亭亭杰峙,谓之历山。山北有小阜,南属迤泽之东北,有陶墟,缘生言舜耕陶所在,墟阜联属,滨带瓠河也。郑玄曰:历山在河东,今有舜井。皇甫谧或言,今济阴历山是也。与雷泽相比,余谓郑玄之言为然。故扬雄《河水赋》曰:登历观而遥望兮,聊浮游于河之岩。今雷首山西枕大河,校之图纬,于事为允。士安又云:定陶西南陶丘,舜所陶处也。不言在此,缘生为失。瓠河之北,即廪丘县也。王隐《晋书地道记》曰,廪丘者,春秋之所谓齐邑矣,实表东海者也。《竹书纪年》,晋烈公十一年,田悼子卒,田布杀其大夫公孙孙,公孙会以廪丘叛于赵。田布围廪丘,翟角、赵孔屑韩师救廪丘,及田布战干龙泽,田师败通是也。瓠河与濮水俱东流,《经》所谓过廪丘,为濮水者也。县南瓠北,有羊角城,《春秋传》曰:乌余取卫羊角,遂袭我高鱼,有大雨,自窦入,介于其库。登其城,克而取之者也。京相璠曰:卫邑也。今东郡廪丘县南,有羊角城。高鱼鲁邑也,今廪丘东北,有故高鱼城。俗谓之交鱼城,谓羊角为角逐城,皆非也。瓠河又径阳晋城南。《史记》,苏秦说齐曰:过卫阳晋之道,径于亢父之险者也。今阳晋城在廪丘城东南十余里,与都关为左右也。张仪曰:秦下甲攻卫阳晋,大关天下之匈。徐广《史记音义》云:关一作开,东之亢父,则其道矣。瓠河之北,又有郕都城。《春秋》隐公五年,郕侵卫。京相璠曰:东郡廪丘县南三十里,故郕都故城。褚先生曰:汉封金安上为侯国,王莽更名之曰城谷者也。瓠河又东,径黎县故城南,王莽改曰黎治矣。盂康曰:今黎阳也。薛瓒言:按黎阳在魏郡,非黎县也。世谓黎侯城,昔黎侯寓于卫,《诗》所谓胡为乎泥中。毛云:泥中,邑名,疑此城也。土地污下,城居小阜,魏濮阳郡治也。瓠河又东,径底县故城南。《地理志》:济阴之属县也。褚先生曰:汉武帝封金日待为侯国,王莽之万岁矣,世犹谓之为万岁亭也。瓠河又东径郓城南。《春秋左传》成公十六年,公自沙随还,待于郓。京相璠曰:《公羊》作运字,今东郡廪丘县东八十里,有故运城,即此城也。
又北过东郡范县东北,为济渠,与将渠合。
瓠河自运城东北径范县,与济濮枝渠合,故渠上承济渎于乘氏县,北径范县,左纳瓠渎,故《经》有济渠之称。又北,与将渠合,渠受河于范县西北,东南径秦亭南。杜预《释地》曰:东平范县西北,有秦亭者也。又东南径范县故城南,王莽更名建睦也。汉兴平中,靳允为范令,曹太祖东征陶谦于徐州,张邈迎吕布,郡县响应。程昱说允曰:君必固范,我守东阿,田单之功可立。即斯邑也。将渠又东会济渠,自下邀谓之将渠,北径范城东,俗又谓之赵沟,非也。
又东北过东阿县东,瓠河故渎,又东北,左合将渠枝渎。枝渎上承将渠于范县,东北径范县北,又东北径东阿城南,而东入瓠河故渎,又北径东阿县故城东。《春秋经》书:冬,及齐侯盟于柯。
《左传》曰:冬盟于柯,始及齐平。杜预曰:东阿即柯邑也。按《国语》,曹沫挟匕首劫齐桓公返,遂邑于此矣。
又东北过临邑县西,又东北过在平县东,为邓里渠。
自宣防已下,将渠已上,无复有水,将渠下水首受河,自北为邓里渠。
又东北过祝阿县,为济渠。
河水自四渎口出为济水,济水二渎合而东注于祝阿也。又东北至梁邹县西,分为二。
脉水寻梁邹,济无二流,盖《经》之误。
其东北者为济河,其东者为时水,又东北至济西,济河东北入于海,时水东至临淄县西,屈南过太山华县东,又南至费县,东人于沂。
时,即耏水也,音而。《春秋》襄公三年,齐晋盟于耏者也。京相璠曰:今临淄,惟有水西北入济,即《地理志》之如水矣。耏、如声相似,然则水即耏水也。盖以与时合,得通称矣。时水自西安城西南,分为二水,枝津别出,西流,德会水注之。水出昌国县黄山西,北流径昌国县故城南。昔乐毅攻齐有功,燕昭王以是县封之,为昌国君。德会水又西北五里,泉水注之。水出县南黄阜,北流径城西北入德会,又西北,世谓之沧浪沟,又北流注时水。《地理志》曰:德会水出昌国西北,至西安人如是也。时水又西,径东高苑城中而西注也。俗人遏令侧城南注,又屈径其城南。《史记》,汉文帝十五年,分齐为胶西王国,都高苑。徐广《音义》曰:乐安有高苑城,故俗谓之东高苑也。其水又北注故渎,又西,盖野沟水注之,源导延乡城东北,平地出泉。西北径延乡城北。《地理志》:千乘有延乡县,世人谓故城为从城,延、从字相似,读随字改,所未详也。西北流,世谓之盖野沟,又西北流,径高苑县北,注时水。时水又西径西高苑县故城南。汉高帝六年,封丙倩为侯国,王莽之常乡也。其水侧城西注。京相璠曰:今乐安博昌县南界,有时水西通济。其源上出盘阳,北至高苑,下有死时,中无水。杜预亦云:时水于乐安枝流,旱则竭涸,为春秋之干时也。《左传》庄公九年,齐鲁战地,鲁师败处也。时水西北至梁邹城,入于济。非济入时,盖时来注济,若济分东流,明不得以时为名。寻时,济更无别流,南延华费之所,斯为谬矣。
汶水出泰山莱芜县原山,西南过其县南。
莱芜县在齐城西南原山,又在县西南六十许里。《地理志》,汶水与淄水俱出原山西南入济,故不得过其县南也。《从征记》曰:汶水出县西南流。又言:自入莱芜谷,夹路连山百数里,水隍多行石涧中,出药草,饶松柏,林藿绵蒙,崖壁相望,或倾岑阻径,或回岩绝谷,清风鸤条,山壑俱响,凌高降深,兼惴栗之惧,危蹊断径,过悬度之艰。未出谷十余里,有别谷在孤山。谷有清泉,泉上数丈,有石穴二口,容人行。入穴丈余,高九尺许,广四五丈。言是昔人居山之处,薪爨烟墨,犹存谷中,林木致密,行人鲜有能至矣。又有少许山田,引灌之踪尚存。出谷有平丘,面山傍水,土人悉以种麦,云此丘不宜殖稷黍而宜麦,齐人相承以殖之,意谓麦丘所栖愚公谷也。何其深沉幽翳,可以托业抬生如此也。余时径此,为之踌蹰,为之屡眷矣。余按麦丘愚公在齐川谷犹传其名,不在鲁,盖志者之谬耳。汶水又西南径赢县故城南。《春秋左传》桓公三年,公会齐侯于赢,成婚于齐也。
又西南过奉高县北。
奉高县,汉武帝元封元年立,以奉泰山之祀,泰山郡治也。县北有吴季札子墓,在汶水南曲中。季札之聘上国也,丧子于赢博之间,即此处也。《从征记》曰:赢县西六十里,有季札儿冢,冢圆,其高可隐也。前有石铭一所,汉末奉高令所立,无所述叙,标志而已。自昔恒蠲民户洒扫之,今不能。然碑石糜碎,靡有遗矣,惟故跌存焉。
屈从县西南流,汶出牟县故城西南阜下,俗谓之胡卢堆。《淮南子》曰:汶出弗其。高诱曰:山名也,或斯阜矣。牟县故城在东北,古牟国也,春秋时,牟人朝鲁,故应劭曰:鲁附庸也。俗谓是水为牟汶也。又西南径奉高县故城西,西南流注于坟。坟水又南,右合北汶水。水出分水溪,源与中川分水,东南流,径泰山东,右合天门下溪水:”水出泰山天门下谷,东流。古者帝王升封,咸憩此水。水上往往有石窍存焉,盖古设舍所跨处也。马第伯书云:光武封泰山,第伯从登,山去平地二十里,南向极望无不睹。其为高也,如视浮云,其峻也,石壁窅,如无道径,遥望其人,或为白石,或雪,久之白者移过,乃知是人,仰视岩石松树,郁郁苍苍,如在云中,俯视溪谷,碌碌不可见丈尺。直上七里天门,仰视天门,如从穴中视天矣。应劭《汉官仪》云:泰山东南山顶,名曰日观。日观者,鸡一鸣时,见日始欲出,长三丈许,故以名焉。其水自溪而东,浚波庄壑,东南流,径龟阴之田。龟山在博县北十五里,昔夫子伤政道之陵迟,望山而怀操,故《琴操》有《龟山操》焉。山北即龟阴之田也,《春秋》定公十年,齐人来归龟阴之田是也。又合环水,水出泰山南溪,南流,历中下两庙间。《从征记》曰:泰山有下中上三庙,墙闭严整,庙中柏树夹两阶,大二十余围,盖汉武所植也。赤眉尝斫一树,见血而止,今斧创犹存。门阁三重,楼榭四所,三层坛一所,高丈余,广八尺。树前在大井,极香冷,异于凡水,不知何代所掘,不常浚谍,而水旱不减。库中有汉时故乐器及神车木偶,皆靡密巧丽。又有石虎。建武十三年,永贵侯张余上金马一匹,高二尺余,形制甚精。中庙去下庙五里,屋宇又崇丽于下庙,庙东西夹涧。上庙在山顶,即封禅处也。其水又屈而东流,又东南径明堂下。汉武帝元封元年,封泰山,降坐明堂于山之东北阯。武帝以古处险狭而不显也,欲治明堂于奉高傍,而未晓其制。济南人公玉带上黄帝时明堂图,图中有一殿,四面无壁,以茅盖之,通水,圜宫垣为复道,上有楼,从西南入,名曰昆仑。天子从之入,以拜把上帝焉。于是上令奉高作明堂于汶上,如带图也。古引水为辟雍处,基渎存焉。世谓此水为石汶。《山海经》曰:环水出泰山,东流庄于汶。即此水也。环水又左入于汶水。汶水数川合注,又西南流径徂徕山西。山多松柏,《诗》所谓徂徕之松也。《广雅》曰:道梓松也。《抱朴子》称《玉策记》曰:千岁之松中有物,或如青牛,或如青犬,或如人,皆寿万岁。又称天陵有偃盖之松也,所谓楼松也,《鲁连子》曰:松枞高十仞而无枝,非忧正室之无柱也。《尔雅》曰:松叶柏身曰枞。《邹山记》曰:徂徕山在梁甫、奉高、博三县界,犹有美松,亦曰尤徕之山也。赤眉渠师樊崇所保也,故崇自号尤徕三老矣。山东有巢父庙,山高十里。山下有陂水,方百许步,三道流注,一水东北沿溪而下,屈径县南,西北流,人于汶。一水北流,历涧,西流入于汶。一水南流,径阳关亭南。《春秋》襄公十七年,逆臧纥自阳关者也。又西流入于汶水也。
过博县西北。
汶水南径博县故城东。《春秋》哀公十一年,会吴伐齐取博者也。灌婴破田横于城下。屈从其城南西流,不在西北也。汶水又西南径龙乡故城南。《春秋》成公二年,齐侯围龙,龙囚顷公嬖人卢蒲就魁,杀而膊诸城上,齐侯亲鼓取龙者也。汉高帝八年,封谒者陈署为侯国。汶水又西南径亭亭山东,黄帝所禅也,山有神庙。水上有石门,旧分水下溉处也。汶水又西南径阳关故城西。本钜平县之阳关亭矣。阳虎据之以叛,伐之,虎焚莱门而奔齐者也。汶水又南,左会淄水,水出泰山梁父县东,西南流,径菟裘城北。《春秋》隐公十一年营之, 公谓羽父曰: 吾将归老焉。故《郡369国志》曰:梁父有菟裘聚。淄水又径梁父县故城南,县北有梁父山。《开山图》曰:泰山在左,亢父在右,亢父知生,梁父主死。王者封泰山,禅梁父,故县取名焉。淄水又西南径柴县故城北。《地理志》:泰山之属县也。世谓之柴汶矣。淄水又径郕县北。汉高帝六年,封董渫为侯国。《春秋》,齐师围郕,郕人伐齐,饮马于斯水也。昔孔子行于郕之野,遇荣启期于是,衣鹿裘,被发琴歌三乐之欢,夫子善其能宽矣。淄水又西径阳关城南,西流注于汶水。汶水又南径矩平县故城东,而西南流。城东有鲁道,《诗》所谓鲁道有荡,齐子由归者也。今汶上夹水有文姜台。汶水又西南流,《诗》云汶水滔滔矣。《淮南子》曰:狢渡汶则死,天地之性,倚伏难寻,固不可以情理穷也。汶水又西南径鲁国汶阳县北,王莽之汶亭也。县北有曲水亭,《春秋》桓公十二年,《经》书:公会杞侯、莒子盟于曲池。《左传》曰:平杞,莒也。故杜预曰:鲁国汶阳县北有曲水亭。汉章帝元和二年,东巡泰山,立行宫于汶阳,执金吾耿恭屯于汶上,城门基堑存焉,世谓之阙陵城也。汶水又西径汶阳县故城北而西注。
又西南过蛇丘县南,汶水又西,洸水注焉。又西径蛇丘县南,县有铸乡城。《春秋左传》宣叔娶于铸。杜预曰:济北,蛇丘县所治铸乡城者也。
又西南过刚县北,《地理志》:刚,故阐也,王莽更之曰柔。应劭曰:《春秋经》书:齐人取讙及阐。今阐亭是也。杜预《春秋释地》曰:阐在刚县北,刚城东有一小亭,今刚县治,俗人又谓之阐亭。京相璠曰:刚县西四十里有阐亭。未知孰是。汶水又西,蛇水注之。水出县东北泰山,西南流,径汶阳之田。齐所侵也。自汶之北,平畅极目,僖公以赐季友。蛇水又西南径铸城西,《左传》所谓蛇渊囿也。故京相璠曰:今济北有蛇丘城,城下有水,鲁囿也。俗谓之浊须水,非矣。蛇水又西南径夏晖城南。《经》书:公会齐侯于下讙是也。今俗谓之夏晖城。盖《春秋左传》桓公三年,公子翚如齐,齐侯送姜氏于下讙,非礼也。世有夏晖之名矣。蛇水又西南入汶。汶水又西,沟水注之,水出东北马山,西南流,径棘亭南。《春秋》成公三年《经》书:秋,叔孙侨如帅师围棘。《左传》曰:取汶阳之田,棘不服,围之。南去汶水八十里。又西南径遂城东。《地理志》曰:蛇丘,遂乡,故遂国也。《春秋》庄公十三年,齐灭遂而戍之者也。京相璠曰:遂在蛇丘东北十里,杜预亦以为然。然县东北无城以拟之,今城在蛇丘西北,盖杜预传疑之非也。又西径下讙城西而入汶水。汶水又西径春亭北。考古无春名,惟平陆县有崇阳亭,然是亭东去刚城四十里,推璠所《注》则符,并所未详也。
又西南过东平章县南,《地理志》曰:东平国,故梁也。景帝中六年,别为济东国。武帝元鼎元年为大河郡,宣帝甘露二年为东平国,王莽之有盐也。章县按《世本》,任姓之国也,齐人降章者也,故城在无盐县东北五十里。汶水又西南,有泌水注之,水出肥成县东北原,西南流,径肥成县故城南。乐正子春谓其弟子曰:子适齐过肥,肥有君子焉。左径句窳亭北。章帝元和二年,凤凰集肥成句窳亭,复其租而巡泰山,即是亭也。泌水又西南径富成县故城西,王莽之成富也。其水又西南流注于汶,汶水又西南径桃乡县故城西。王莽之鄣亭也。世以此为鄣城,非,盖因巨新之故目耳。
又西南过无盐县南,又西南过寿张县北,又西南至安民亭,入于济。
汶水自桃乡四分,当其派别之处,谓之四汶口。其左二水双流,西南至无盐县之郈乡城南。郈,昭伯之故邑也,祸起斗鸡矣。《春秋左传》定公十二年,叔孙氏堕郈,今其城无南面。汶水又西南径东平陆县故城北。应劭曰:古厥国也,今有厥亭。汶水又西径危山南,世谓之龙山也。《汉书·宣元六王传》曰:哀帝时,无盐危山土自起,覆草,如驰道状,又瓠山石转立。晋灼曰:《汉注》作报山。山胁石一枚,转侧起立,高九尺六寸,旁行一丈,广四尺,东平王云及后谒曰:汉世石立,宣帝起之表也。自之石所祭,治石象报山立石,束倍草,并祠之。建平三年,息夫躬告之,王自杀,后谒弃市,国除。汶水又西合为一水,西南入茂都淀。淀,陂水之异名也。淀水西南出,谓之巨野沟。又西南径致密城南。《郡国志》曰:须昌县有致密城,古中都也,即夫子所宰之邑矣,制养生送死之节,长幼男女之礼,路不拾遗,器不雕伪矣。巨野沟又西南入桓公河,北水西出淀,谓之巨良水,西南径致密城北,西南流注洪渎。次一汶,西径郈亭北.又西至寿张故城东,潴为泽渚。初平三年,曹公击黄巾于寿张东,鲍信战死于此。其右一汶,西流径无盐县之故城南,旧宿国也。齐宣后之故邑,所谓无盐丑女也。汉武帝元朔四年,封城阳共王子刘庆为东平侯,即此邑也。王莽更名之曰有盐亭。汶水又西径郈乡城南。《地理志》所谓无盐有郈乡者也。汶水西南流,径寿张县故城北。《春秋》之良县也。县有寿聚,汉曰寿良。应劭曰:世祖叔父名良,故光武改曰寿张也。建武十二年,世祖封樊宏为侯国。汶水又西南,长直沟水注之。水出须昌城东北谷阳山,南径须昌城东,又南,漆沟水注焉。水出无盐城东北五里阜山下,西径无盐县故城北。水侧有东平宪王仓冢,碑阙存焉。元和二年,章帝幸东平,把以太牢,亲拜祠坐,赐御剑于陵前。其水又西流注长直沟,沟水奇分为二:一水西径须昌城南入济,一水南流注于汶。汶水又西流入济,故《淮南子》曰:汶出弗其,西流合济。高诱云:弗其,山名,在朱虚县东。余按诱说是,乃东汶,非《经》所谓入济者也,盖其误证耳。


卷二十五  泗水、沂水、洙水 
泗水出鲁卞县北山,《地理志》曰,出济阴乘氏县。又云:出卞县北。《经》言北山,皆为非矣。《山海经》曰:泗水出鲁东北。余昔因公事,沿历徐沇,路径洙、泗,因令寻其源流。水出卞县故城东南,桃墟西北。《春秋》昭公七年,谢息纳季孙之言,以孟氏成邑与晋而迁于桃。杜预曰:鲁国卞县东南有桃墟。世谓之曰陶墟,舜所陶处也,井曰舜井,皆为非也。墟有漏泽,方十五里,渌水澂渟,三丈如减。泽西际阜,俗谓之妫亭山,盖有陶墟、舜井之言,因复有妫亭之名矣。阜侧有三石穴,广圆三四尺。穴有通否,水有盈漏,漏则数夕之中,倾陂竭泽矣。左右民居,识其将漏,预以木为曲洑,约障穴口,鱼鳖暴鳞,不可胜载矣。自此连冈通阜,西北四十许里,冈之西际,使得泗水之源也。《博物志》曰:泗出陪尾。盖斯阜者矣。石穴吐水,五泉俱导,泉穴各径尺余。水源南侧有一庙,栝柏成林,时人谓之原泉祠,非所究也。泗水西径其县故城南。《春秋》襄公二十九年,季武子取卞曰:闻守卞者将叛,臣率徒以讨之是也。南有姑蔑城。《春秋》隐公元年,公及邾仪父盟于蔑者也。水出二邑之间,西径郚城北。《春秋》文公七年,《经》书:公伐邾,三月甲戌,取须句,遂城郚。杜预曰:鲁邑也,卞县南有郚城备邾难也。泗水自卞而会于洙水也。
西南过鲁县北,泗水又西南流,径鲁县,分为二流,水侧有一城,为二水之分会也。北为洙渎。《春秋》庄公九年,《经》书:冬,浚洙。京相璠、服虔、杜预并言:洙水在鲁城北,浚深之,为齐备也。南则泗水。夫子教于洙、泗之间,今于城北二水之中,即夫子领徒之所也。《从征记》曰:洙、泗二水,交于鲁城东北十七里。阙里背洙面泗,南北百二十步,东西六十步,四门各有石阃。北门去洙水百步余。后汉初,阙里荆棘,自辟,从讲堂至九里。鲍永为相,因修飨祠,以诛鲁贼彭丰等。郭缘生育泗水在城南。非也。余按《国语》:宣公夏滥于泗渊,里革断罟弃之。韦昭云:泗在鲁城北。《史记》、《冢记》、王隐《地道记》咸言,葬孔子于鲁城北泗水上。今泗水南有夫子冢。《春秋孔演图》曰:鸟化为书,孔子奉以告天,赤爵衔书上,化为黄玉,刻曰:孔提命,作应法,为赤制。《说题辞》曰:孔子卒,以所受黄玉葬鲁城北,即子贡庐墓处也。谯周云:孔子死后,鲁人就冢次而居者百有余家,命曰孔里。《孔丛》曰:夫子墓茔方一里,在鲁城北六里泗水上。诸孔氏封五十余所,人名昭穆,不可复识。有铭碑三所,兽碣具存。《皇览》曰:弟子各以四方奇木来植,故多诸异树,不生棘木刺草,今则无复遗条矣。泗水自城北,南径鲁城西南,合沂水。沂水出鲁城东南尼丘山西北,山即颜母所祈而生孔子也。山东十里有颜母庙。山南数里,孔子父葬处,《礼》所谓防墓崩者也。平地发泉,流径鲁县故城南。水北东门外,即爱居所止处也。《国语》曰:海鸟曰爱居,止于鲁城东门之外三日,臧文仲祭之,展禽讥焉。故《庄子》曰:海鸟止郊,鲁侯觞之,奏以广乐,具以太牢,三日而死,此养非所养矣。门郭之外,亦戎夷死处。《吕氏春秋》曰:昔戎夷违齐如鲁,天大寒而后门,与弟子宿于郭门外,寒愈甚,谓弟子曰:子与我衣,我活,我与子衣,子活。我国士也,为天下惜。子不肖人,不足爱。弟子曰:不肖人,恶能与国士并衣哉?戎叹曰:不济夫!解衣与弟子,半夜而死。沂水北对稷门。昔圉人荦有力,能投盖于此门,服虔曰:能投千钩之重过门之上也。杜预谓走接屋之桷,反覆门上也。《春秋》僖公二十年,《经》书:春,新作南门。《左传》曰:书不时也。杜预曰:本名稷门,僖公更高大之,今犹不与诸门同,改名高门也。其遗基犹在,地八丈余矣。亦曰零门。《春秋左传》庄公十年,公子偃请击宋师,窃从雩门蒙皋比而出者也。门南隔水有雩坛,坛高三丈,曾点所欲风舞处也。高门一里余道西,有《道儿君碑》,是鲁相陈君立。昔曾参居此,枭不入郭。县即曲阜之地,少昊之墟。有大庭氏之库,《春秋》竖牛之所攻也。故刘公于《鲁都赋》曰:戢武器于有炎之库,放戎马于巨野之坰。周成王封姬旦于曲阜,曰鲁。秦始皇二十三年以为薛郡,汉高后元年为鲁国。阜上有季氏宅,宅有武子台,今虽崩夷,犹高数丈。台西百步,有大井,广三丈,深十余丈,以石垒之,石似磬制。《春秋》定公十二年,公山不狃帅费入攻鲁,公入季氏之宫,登武子之台也,台之西北二里,有周公台,高五丈,周五十步。台南四里许,则孔庙,即夫子之故宅也。宅大一顷,所居之堂,后世以为庙。汉高祖十三年,过鲁,以太牢祀孔子。自秦烧《诗》、《书》,经典沦缺。汉武帝时,鲁恭王坏孔子旧宅,得《尚书》、《春秋》、《论语》、《孝经》,时人已不复知有古文,谓之科斗书,汉世秘之,希有见者。于时闻堂上有金石丝竹之音,乃不坏。庙屋三间,夫子在西间东向,颜母在中间南向,夫人隔东一间东向。夫子床前,有石砚一枚,作甚朴,云平生时物也。鲁人藏孔子所乘车于庙中,是颜路所请者也。献帝时,庙遇火,烧之。水平中,钟高意为鲁相,到官,出私钱万三千文,付户曹孔治夫子车,身入庙,拭几席剑履。男子张伯除堂下草,土中得玉壁七枚。伯怀其一,以六枚白意。意令主簿安置几前。孔子寝堂床首,有悬瓮。意召孔,问:何等瓮也?对曰:夫子瓮也,背有丹书,人勿敢发也。意曰:夫子圣人,所以遗瓮,欲以悬示后贤耳。发之,中得素书,文曰:后世修吾书,董仲舒。护吾车,拭吾履,发吾笥,会稽钟离意。璧有七,张伯藏其一。意即召问伯,果服焉。魏黄初元年,文帝令郡国修起孔子旧庙,置百石吏卒。庙有夫子像,列二弟子执卷立侍,穆穆有询仰之容。汉、魏以来,庙列七碑,二碑无字。栝柏犹茂。庙之西北二里,有颜母庙,庙像犹严,有修栝五株。孔庙东南五百步,有双石阙,即灵光之南闭。北百余步,即灵光殿基,东西二十四丈,南北十二丈,高丈余。东西廊庑别舍,中间方七百余步。阙之东北有浴池,方四十许步。池中有钓台,方十步,台之基岸悉石也。遗基尚整,故王延寿赋曰:周行数里,仰不见日者也。是汉景帝程姬子鲁恭王之所造也。殿之东南,即泮宫也,在高门直北道西。宫中有台,高八十尺,台南水东西百步,南北六十步,台西水南北四百步,东西六十步,台池咸结石为之,《诗》所谓思乐泮水也。沂水又西径圜丘北,丘高四丈余。沂水又西流,昔韩雉射龙于斯水之上。《尸子》曰;韩雉见申羊于鲁,有龙饮于沂。韩雉曰:吾闻之,出见虎,搏之,见龙,射之,今弗射,是不得行吾闻也。遂射之。沂水又西,右注泗水也。
又西过瑕丘县东,屈从县东南流,漷水从东来注之。
瑕丘,鲁邑,《春秋》之负瑕矣。哀公七年,季康子伐邾,囚诸负瑕是也。应劭曰:瑕丘在县西南。昔卫大夫公叔文子升于瑕丘,遽伯玉从。文子曰:乐哉斯丘!死则我欲葬焉。伯玉曰:吾子乐之,则瑷请前。刺其欲害民良田也。瑕丘之名,盖因斯以表称矣。曾子吊诸负夏,郑玄、皇甫谧并言卫地,鲁、卫虽殊,土则一也。漷水出东海合乡县。汉安帝永初七年,封马光子朗为侯国。其水西南流入邾。《春秋》哀公二年,季孙斯伐邾,取漷东田及沂西田是也。漷水又径鲁国邹山东南,而西南流,《春秋左传》所谓峄山也,邾文公之所迁。今城在邹山之阳,依岩阻以墉固,故邾娄之国,曹姓也。叔梁纥之邑也,孔子生于此。后乃县之,因邹山之名以氏县也。王莽之邹亭矣。京相璠曰:《地理志》,峄山在邹县北,绎邑之所依以为名也。山东西二十里,高秀独出,积石相临,殆无土壤,石间多孔穴,洞达相通,往往有如数间屋处,其俗谓之峄孔。遭乱,辄将家入峄,外寇虽众,无所施害。晋永嘉中,太尉郗鉴将乡曲保此山,胡贼攻守不能得。今山南有大峄,名曰郗公峄。山北有绝岩,秦始皇观礼于鲁,登于峄山之上,命丞相李斯,以大篆勒铭山岭,名曰昼门,《诗》所谓保有凫峄者也。漷水又西南径蕃县故城南。又西径薛县故城北,《地理志》曰:夏车正奚仲之国也。《竹书纪年》梁惠成王三十一年,邳迁于薛,改名徐州。城南山上有奚仲冢。《晋太康地记》曰:奚仲冢在城南二十五里山上,百姓谓之神灵也。齐封田文于此,号孟尝君,有惠喻。今郭侧犹有文冢,结石为郭,作制严固,莹丽可寻,行人往还,莫不径观,以为异见矣。漷水又西,径仲虺城北。《晋太康地记》曰:奚仲迁于邳,仲虺居之,以为汤左相。其后当周,爵称侯,后见侵削,霸者所继为伯,任姓也。应欲曰:邳在薛。徐广《史记音义》曰:楚元王子郢客,以吕后二年,封上邳侯也。有下故此为上矣。《晋书地道记》曰:仲虺城在薛城西三十里。漷水又西至湖陆县,入于泗,故京相璠曰:薛县漷水,首受蕃县,西注山阳湖陆是也。《经》言瑕丘东,误耳。
又南过平阳县西,县,即山阳郡之南平阳县也。《竹书纪年》曰:梁惠成王二十九年,齐田肸及宋人伐我东鄙,围平阳者也。王莽改之曰黾平矣。泗水又南径故城西,世谓之漆乡;应劭《十三州记》曰:漆乡,邾邑也。杜预曰:平阳东北有漆乡。今见有故城,西南方二里,所未详也。
又南过高平县西,洸水从西北来流注之。
泗水南径高平山,山东西十里,南北五里,高四里,与众山相连,其山最高,顶上方平,故谓之高平山,县亦取名焉。泗水又南径高平县故城西。汉宣帝地节三年,封丞相魏相为侯国。高帝七年,封将军陈锴为橐侯。《地理志》:山阳之属县也。王莽改曰高平。应劭曰:章帝改。按本《志》曰王莽改名,章帝因之矣。所谓洸水者,洙水也,盖洸、洙相入,互受通称矣。又南过方与县东,汉哀帝建平四年,县女子田无啬生子。先未生二月,儿啼腹中,及生不举,葬之陌上。三日,人过闻啼声,母掘养之。
菏水从西来往之。
菏水,即济水之所苞注以成湖泽也。而东与泗水合于湖陵县西六十里谷庭城下,俗谓之黄水口。黄水西北通巨野泽,盖以黄水沿注于菏,故因以名焉。
又屈东南过湖陆县南,涓涓水从东北来流注之。
《地理志》:故湖陵县也,菏水在南,王莽改曰湖陆。应劭曰:一名湖陵,章帝封东平王苍子为湖陆侯,更名湖陆也。泗水又东,径郗鉴所筑城北,又东,径湖陵城东南。昔桓温之北入也,范懽擒慕容忠于此,城东有《度尚碑》。泗水又左会南梁水。《地理志》曰:水出著县。今县之东北,平泽出泉若轮焉,发源成川,西南流,分为二水。北水枝出西径蕃县北,西径膝城北。《春秋左传》隐公十一年,滕侯、薛侯来朝,争长。薛侯曰:我先封。滕侯曰:我周之卜正也。薛,庶姓也,我不可以后之。公使羽父请薛侯曰:君辱在寡人,周谚有之曰:山有木,工则度之:宾有礼,主则择之。周之宗盟,异姓为后。寡人若朝于薛,不敢与诸任齿。君若辱贶寡人,则愿以滕君为请。薛侯许之,乃长滕侯者也。汉高祖封夏侯婴为侯国,号曰滕公。邓展曰:今沛郡公丘也。其水又溉于丘焉。县故城在滕西北,城周二十里,内有子城。按《地理志》即滕也。周懿王子错叔绣文公所封也。齐灭之,秦以为县。汉武帝元朔三年,封鲁恭王子刘顺为侯国。世以此水溉我良田,遂及百秭,故有两沟之名焉。南梁水自枝渠西南,径鲁国蕃县故城东,俗以南邻于漷,亦谓之西漷水。南梁水又屈径城南,应劭曰:县,古小邾邑也。《地理志》曰,其水西流,注于济渠,济在湖陆西,而左注泗,泗、济合流,故《地记》或言济入泗,泗亦言入济,互受通称,故有人济之文。阚駰《十三州志》曰:西至湖陆入泗是也。《经》无南梁之名,而有涓涓之称,疑即是水也。戴延之《西征记》亦言湖陆县之东南,有涓涓水,亦无记子南梁,谓是吴王所道之渎也。余按湖陆西南,止有是水。延之盖以《国语》云,吴王夫差起师,将北会黄池,掘沟于商、鲁之间,北属之沂,西属于济。以是言之,故谓是水为吴王所掘,非也。余以水路求之。止有泗川耳。盖北达沂西,北径于商鲁而接于济矣,吴所浚广耳。非谓起自东北受沂西南注济也。假之有通,非吴所趣,年载诚眇,人情则近,以今忖古,益知延之之不通情理矣。泗水又南,漷水注之,又径薛之上邳城西,而南注者也。
又东过沛县东,昔许由隐于沛泽,即是县也。县盖取泽为名,宋灭属楚,在泗水之滨,于秦为泗水郡治。黄水注之。黄水出小黄县黄乡黄沟。《国语》曰:吴子会诸侯于黄池者也。黄水东流,径外黄县故城南。张晏曰:魏郡有内黄县,故加外也。薛瓒曰:县有黄沟,故县氏焉。圈称《陈留风俗传》曰:县南有渠水,于《春秋》为宋之曲棘里。故宋之别都矣。《春秋》昭公二十五年,宋元公卒于曲棘是也。宋华元居于稷里。宣公十五年,楚郑围宋,晋解扬违楚,致命于此。宋人惧,使华元乘闉夜入楚师,登子反之床,曰:寡君使元以病合,弊邑易子而食,析骸以爨,城下之盟,所不能也。子反退一舍,宋楚乃平。今城东闉上犹有华元祠,祠之不辍。城北有华元冢。黄沟自城南,东径葵丘下。《春秋》僖公九年,齐桓公会诸侯于葵丘,宰孔曰:齐侯不务德而勤远略,北伐山戎,南伐楚,西为此会,东略之不知,西则否矣,其在乱乎?君务靖乱,无勤于行,晋侯乃还,即此地也。黄沟又东注大泽,蒹葭萑苇生焉,即世所谓大荠陂也。陂水东北流,径定陶县南,又东径山阳郡成武县之楚丘亭北,黄沟又东径成武县故城南,王莽更之曰成安也。黄沟又东北径郜城北。《春秋》桓公二年,《经》书取郜大鼎于宋,戊申纳于太庙。《左传》曰:宋督攻孔父而取其妻,杀殇公而立公子冯,以郜大鼎赂公,臧哀伯谏为非礼。《十三州志》曰:今成武县东南有郜城,俗谓之北郜者也。黄沟又东径平乐县故城南,又东右合泡水,即丰水之上源也。水上承大莽陂,东径贳城北,又东径已氏县故城北,王莽之已善也。县有伊尹冢。崔駰曰,殷帝沃丁之时,伊尹卒,葬于薄。《皇览》曰:伊尹冢在济阴已氏平利乡。皇甫谧曰:伊尹年百余岁而卒,大雾三日,沃丁葬以天子之礼,亲自临丧,以报大德焉。又东径孟诸泽。杜预曰:泽在梁国睢阳县东北。又东径郜城县故城南。《地理志》:山阳县也,王莽更名之曰告成矣,故世有南郜、北郜之论也。又东径单父县故城南。昔宓子贱之治也,孔子使巫马期观政,入其境,见夜渔者,问曰,子得鱼辄放,何也?曰:小者,吾大夫欲长育之故也。子闻之曰:诚彼形此,子贱得之善矣。惜哉!不齐所治者小也。王莽更名斯县为利父矣。世祖建武十三年,封刘茂为侯国。又东径平乐县,右合泡水。水上承睢水于下邑县界,东北注一水。上承睢水于杼秋县界,北流,世又谓之瓠卢沟,水积为渚。渚水东北流,二渠双引,左合沣水,俗谓之二泡也。自下沣泡,并得通称矣。故《地理志》曰:平乐,侯国也。泡水所出,又径丰西泽,谓之丰水。《汉书》称高祖送徒丽山,徒多亡。到丰西泽,有大蛇当径,放剑斩之,此即汉高祖斩蛇处也。又东径大堰,水分为二。又东径丰县故城南,王莽之吾丰也。水侧城东北流,右合枝水,上承丰西大堰,派流东北,径丰城北,东注沣水。沣水又东合黄水,时人谓之狂水,盖狂、黄声相近,俗传失实也。自下黄水又兼通称矣。水上旧有梁,谓之泡桥。王智深《宋史》云:宋太尉刘义恭于彭城,遣军主稽玄敬北至城觇候魏军,魏军于清西望见玄敬士众,魏南康侯社道俊引趣泡桥,沛县民逆烧泡桥,又于林中打鼓,俊谓宋军大至,争渡泡水,水深酷寒,冻溺死者殆半。清水,即泡水之别名也。沈约《宋书》称魏军欲渡清西,非也。泡水又东径沛县故城南。秦末兵起,萧何、曹参迎汉祖于此城。高帝十一年,封合阳侯刘仲子为侯国。城内有汉高祖庙,庙前有三碑,后汉立庙基,以青石为之,阶陛尚存。刘备之为徐州也,治此,袁术遣纪灵攻备,备求救吕布,布救之。屯小沛,招灵请备共饮,布谓灵曰:玄德,布弟也,布性不喜合斗,但喜解斗。乃植戟于门,布弯弓曰:观布射戟,小枝中者,当各解兵,不中可留决斗。一发中之,遂解。此即布射朝枝处也。《述征记》曰:城极大,四周堑通丰水。丰水于城南东注泗,即泡水也。《地理志》曰:泡水自平乐县东北至沛入泗者也。泗水南径小沛县东。县治故城南垞上,东岸有泗水亭,汉祖为泗水亭长,即此亭也。故亭今有高祖庙,庙前有碑,延熹十年立。庙阙崩褫,略无全者。水中有故石梁处,遗石尚存。高租之破黥布也,过之,置酒沛宫,酒酣歌舞,慷慨伤怀曰:游子思故乡也。泗水又东南流,径广戚县故城南。汉武帝元朔元年,封刘择为侯国,王莽更之曰力聚也。泗水又径留县,而南径垞城东。城西南有崇侯虎庙,道沦遗爱,不知何因而远有此图。泗水又南径床大夫桓冢西。山枕泗水、西上尽石,凿而为冢,今人谓之石郭者也。郭有二重,石作工巧。夫子以为不如死之速朽也。
又东南过彭城县东北,泗水西有龙华寺,是沙门释法显远出西域,浮海东还,持龙华图,首创此制。法流中夏,自法显始也。其所持天竺二石,仍在南陆东基堪中,其石尚光洁可爱。泗水又南,获水入焉,而南径彭城县故城东。周显王四十二年,九鼎沦没泗渊。秦始皇时,而鼎见于斯水。始皇自以德合三代,大喜,使数千人没水求之,不得,所谓鼎伏也。亦云系而行之未出,龙齿啮断其系,故语曰:称乐大早绝鼎系。当是孟浪之传耳。泗水又径龚胜墓南,墓碣尚存。又经亚父冢东。《皇览》曰:亚父家在庐江县郭东居巢亭中。有亚父井,吏民亲事,皆祭亚父于居巢厅上。后更造祠于郭东,至今祠之。按《汉书·项羽传》,历阳人范增,未至彭城而发疽死,不言之居巢。今彭城南,有项羽凉马台,台之西南山麓上,即其冢也。增尔慕范蠡之举,而自绝于斯,可谓褊矣。推考书事,墓近于此也。
又东南过吕县南。
吕,宋邑也。《春秋》襄公元年,晋师伐郑及陈,楚子辛救郑,侵宋吕留是也。县对泗水,汉景帝三年,有白颈乌与黑乌群斗于县,白颈乌不胜,堕泗水中,死者数千。京房《易传》曰:逆亲亲厥妖,白黑鸟斗时,有吴楚之反。泗水之上,有石梁焉,故曰吕梁也。昔宋景公以弓工之弓,弯弧东射,矢集彭城之东,饮羽于石梁,即斯梁也。悬涛崩渀,实为泗险,孔子所谓鱼鳖不能游。又云悬水三十仞,流沫九十里、今则不能也。盖惟岳之喻,未便极天明矣。《晋太康地记》曰:水出磬石。《书》所谓泗滨浮磬者也。泗水又东南流,丁溪水注之。溪水上承泗水于吕县,东南流,北带广隰,山高而注于泗川。泗水冬春浅涩,常排沙通道,是以行者多从此溪,即陆机《行思赋》所云:乘丁水之捷岸,排泗川之积沙者也。晋太元九年,左将军谢玄,于吕梁遣督护闻人奭,用工丸万,拥水立七拖,以利运漕者。
又东南过下邳县西,泗水历县,径葛峄山东,即奚仲所迁邳峄者也。泗水又东南径下邳县故城西。东南流,沂水流注焉。故东海属县也。应劭曰:奚仲自薛徙居之,故曰下邳也。汉徙齐王韩信为楚王,都之,后乃县焉,王莽之闰俭矣。东阳郡治。文颖曰:秦嘉,东阳郡人。今下邳是也。晋灼曰:东阳县本属临淮郡。明帝分属下邳,后分属广陵。故张晏曰:东阳郡,今广陵郡也,汉明帝置下邳郡矣。城有三重,其大城中,有大司马石苞、镇东将军胡质、司徒王浑、监军石崇四碑。南门谓之白门,魏武擒陈宫于此处矣。中城,吕布所守也。小城,晋中兴北中郎将荀羡郗昙所治也。昔泰山吴伯武少孤,与弟文章相失二十余年,遇于县市,文章欲欧伯武,心神悲恸,因相寻问,乃兄弟也。县为沂、泗之会也。又有武原水注之,水出彭城武原县西北.会注陂南,径其城西,王莽之和乐亭也,县东有徐庙山,山因徐徙,即以名之也。山上有石室,徐庙也。武原水又南合武水,谓之泇水。南径刚亭城,又南至下邳入泗,谓之武原水口也。又有桐水,出西北东海容丘县东南,至下邳入泗。泗水东南径下相县故城东,王莽之从德也。城之西北,有汉太尉陈球墓,墓前有三碑,是弟子管宁、华歆等所造。初平四年,曹操攻徐州,破之,拔取虑、睢陵、夏丘等县,以其父避难,被害于此,屠其男女十万,泗水为之不流,自是数县人无行迹,亦为暴矣。泗水又东南,得睢水口。泗水又径宿预城之西,又径其城南。故下邳之宿留县也,王莽更名之曰康义矣。晋元皇之为安东也,督运军储而为邸阁也。魏太和中,南徐州治,后省为戍。梁将张惠绍北入,水军所次,凭固斯城,更增修郭堑,其四面引水环之,今城在泗水之中也。又东南入于淮。
泗水又东径陵栅南。《西征记》曰:旧陵县之治也。泗水又东南径淮阳城北。城临泗水,昔訢丘斩饮马斩蛟,眇目于此处也。泗水又东南径魏阳城北。城枕泗川,陆机《行思赋》曰:行魏阳之在渚。故无魏阳,疑即泗阳县故城也,王莽之所谓淮平亭矣。盖魏文帝幸广陵,所由或因变之,未详也。泗水又东径角城北,而东南流注于淮。考诸《他说》,或言泗水于睢陵入淮,亦云于下相入淮,皆非实录也。
沂水出泰山盖县艾山,郑玄云:出沂山,亦或云临乐山。水有二源,南源所导,世谓之柞泉;北水所发,俗谓之鱼穷泉。俱东南流,合成一川。右会洛预水,水出洛预山,东北流注之。沂水东南流,左合桑预水,水北出桑预山,东注于沂水。沂水又东南,螳蜋水入焉。水出鲁山,东南流,右注沂水。沂水又东径盖县故城南,东会连绵之水。水发连绵山,南流,径盖城东而南入沂。沂水又东径浮来之山,《春秋经》书,公及莒人盟于浮来者也,即公来山也。在邳乡西。故号曰邳来之间也。浮来之水注之。其水左控三川,右会甘水而注于沂。沂水又南径爆山西,山有二峰,相去一里,双峦齐秀,圆峙若一。沂水又东南径东莞县故城西,与小沂水合。孟康曰:县,故郓邑,今郓亭是也。汉武帝元朔二年,封城阳共王子吉为东莞侯。魏文帝黄初中,立为东莞郡,《东燕录》谓之团城,刘武帝北伐广固,登之以望王难。魏南青州治。《左氏怜》曰:莒鲁争郓,为日久矣。今城北郓亭是也。京相璠曰:琅邪姑幕县南四十里员亭,故鲁郓邑,世变其字,非也。《郡国志》:东莞有郓亭,今在团城东北四十里,犹谓之故东莞城矣。小沂水出黄孤山西,南流径其城北,西南注于沂。沂水又南与间山水合,水出闾山,东南流,右佩二水,总归于沂。沂水南径东安县故城东,而南合时密水,水出时密山。春秋时莒地。《左传》,莒人归井仲于鲁,及密而死是也。时密水东流,径东安城南。汉封鲁孝王子强为东安侯。时密水又东南流入沂。沂水又南,桑泉水北出五女山,东南流,巨围水注之。水出巨围之山,东南注于桑泉水。桑泉水又东南,堂阜水入焉,其水导源堂阜。《春秋》庄公九年,管仲请囚,鲍叔受之,及堂阜而税之。杜顶曰:东莞蒙阴县西北,有夷吾亭者是也。堂阜水又东南注桑泉水。桑泉水又东南径蒙阴县故城北,王莽之蒙恩也。又东南与崮水合,水有二源双会,东导一川,俗谓之汶水也。东径蒙阴县,注桑泉水。又东南,卢川水注之。水出鹿岭山,东南流,左则二川臻凑,右则诸葛泉源,斯奔乱流径城阳之卢县,故盖县之卢上里也。汉武帝元朔二年,封城阳共王子刘豨为侯国,王莽更名之曰著善矣。又东南注于桑泉水。桑泉水又东南,右合蒙阴水。水出蒙山之阴,东北流。昔琅邪承宫避乱此山,立性好仁,不与物竟,人有认其黍者,舍之而去。其水东北流,人于沂。沂水又南径阳都县故城东。县,故阳国也。齐同盟,齐利其地而迁之者也。汉高帝六年,封将军丁复为侯国。沂水又南与蒙山水合,水出蒙山之阴,东流,径阳都县南,东注沂水。沂水又左合温水,水上承温泉陂,而西南入于沂水者也。
南过琅邪临沂县东,又南过开阳县东。
沂水南径中丘城西。《春秋》隐公七年。夏,城中丘。《左传》曰:书不时也。沂水又南径临沂县故城东。《郡国志》曰:琅邪有临沂县,故属东海郡。有治水注之。水出泰山南武阳县之冠石山。《地理志》曰:冠石山,治水所出。应劭《地理风俗记》曰,武水出焉,盖水异名也。东流,径蒙山下,有祠。治水又东南径颛臾城北。《郡国志》曰:县有颛臾城,季氏将伐之,孔子曰:昔者先王以为东蒙主,社稷之臣,何以伐之为?冉有曰:今夫颛臾,固而便近于费者也。治水又东南流,径费县故城南。《地理志》:东海之属县也。为鲁季孙之邑,子路将堕之,公山弗扰师袭鲁,弗克。后季氏为阳虎所执,弗扰以费畔,即是邑也。汉高帝六年,封陈贺为侯国,王莽更名之曰顺从也。许慎《说文》云,沂水出东海费县东,西入泗,从水,斤声。吕忱《字林》,亦言是矣。斯水东南所注者,沂水在西,不得言东南趣也,皆为谬矣。故世俗谓此水为小沂水。治水又东南径访城南。《春秋》隐公八年:郑伯请释泰山之把而祀周公,使宛归泰山之祊,而易许田。杜预《释地》曰:祊,郑祀泰山之邑也。在琅邪费县东南。治水又东南流,注于沂。沂水又南径开阳县故城东。县,故鄅国也。《春秋左传》昭公十八年,邾人袭鄅,尽俘以归,各鄅子曰,余无归矣,从帑于邾是也。后更名开阳矣。《春秋》哀公三年,《经》书季孙斯、叔孙州仇帅师城启阳者是矣,县故琅邪郡治也。又东过襄贲县东,屈从县南,西流,又屈南过郯县西。
《鲁连子》称陆子谓齐滑王曰:鲁费之众臣,甲舍于囊贲者也。王莽更名章信也。郯,故国也,少昊之后。《春秋》昭公十七年,郯子朝鲁,公与之宴,昭子叔孙湣问曰:少吴鸟名官,何也?郯子曰:吾祖也,我知之矣。黄帝、炎帝以云火纪官,太皞以龙纪,少皞瑞凤鸟,统历鸟官之司,议政斯在,孔子从而学焉。既而告人曰,天子失官,学在四夷者也。《竹书纪年》,晋烈公四年,越子末句灭郯,以郯子鸪归。县,故旧鲁也。东海郡治。秦始皇以为郯郡。汉高帝二年,更从今名,即王莽之沂平者也。
又南过良城县西,又南过下邳县西,南入于泗。《春秋左传》曰:昭公十三年秋,晋侯会吴子于良,吴子辞水道不可以行,晋乃还。是也。《地理志》曰:良城,王莽更名承翰矣。沂水于下邳县北,西流分为二水,一水于城北西南入泗,一水径城东屈从县南,亦注泗,谓之小沂水。水上有桥,徐、泗间以为妃。昔张子房遇黄石公于圮上,即此处也。建元二年,曹操围吕布于此,引沂、泗灌城而擒之。
洙水出泰山盖县临乐山,《地理志》曰:临乐山、洙水所出,西北至盖,入泗水。或作池字,盖字误也。洙水自山西北径盖县。汉景帝中五年,封后兄王信为侯国。又西径泰山东平阳县。《春秋》宣公八年冬,城平阳。杜预曰:今泰山平阳县是也。河东有平阳,故此加东矣。晋武帝元康九年,改为新泰县也。西南至卞县入于泗。
洙水西南流,盗泉水注之。泉出卞城东北卞山之阴。《尸子》曰:孔子至于暮矣而不宿,于盗泉渴矣而不饮,恶其名也。故《论语比考谶》曰:水名盗泉,仲尼不漱。即斯泉矣。西北流,注于洙水。洙水又西南流于卞城西,西南入泗水。乱流西南,至鲁县东北,又分为二水。水侧有故城,两水之分会也。洙水西北流,径孔里北,是谓洙泗之间矣。春秋之浚洙,非谓始导矣,盖深广之耳。洙水又西南,枝津出焉。又南径瑕丘城东,而南入石门,古结石为389水门,跨于水上也。西南流,世谓之杜武沟。洙水又西南径南平阳县之显间亭西,邾邑也。《春秋》襄公二十一年,《经》书邾庶其以漆、闾丘来奔者也。杜预曰:平阳北有显闾亭。《十三州记》曰:山阳南平阳县,又有闾丘乡。《从征记》曰:杜谓显闾,闾丘也。今按漆乡在县东北,漆乡东北十里,见有闾丘乡,显间非也。然则显间自是别亭,未知孰是。又南,洸水注之。吕忱曰:洗水出东平阳,上承汶水于刚县西、阐亭东。《尔雅》曰:汶别为阐,其犹洛之有波矣。洸水西南流,径盛乡城西。京相璠曰:刚县西南,有盛乡城者也。又南径泰山宁阳县故城西。汉武帝元朔三年,封鲁共王子刘恬为侯国,王莽改之曰宁顺也。又南,洙水枝津注之。水首受洙,西南流,径瑕丘城北,又西径宁阳城南,又西南入于洸水。洸水又西南,径泰山郡乘丘县故城东。赵肃侯二十年,韩将举与齐魏战于乘丘,即此县也。汉武帝元朔五年,封中山靖王子刘将夜为侯国也。洸水又东南流,注于洙。洙水又南至高平县,南入于泗水。西有茅乡城,东去高平三十里。京相璠曰:今高平县西三十里,有故茅乡城者也。


\chapter{卷二十六  沭水、巨洋水、淄水、汶水、潍水、胶水 }

\begin{yuanwen}
沭水出琅邪东莞县西北山,大弁山与小泰山连麓而异名也。引控众流,积以成川。东南流,径邳乡南,南去县八十许里。城有三面而不周于南,故俗谓之半城。沭水又东南流:左合岘水。水北出大岘山,东南流,径邳乡东,东南流注于沭水也。
东南过其县东,沭水左与箕山之水合。水东出诸县西箕山,刘澄之以为许由之所隐也,更为巨谬矣。其水西南流,注于沭水也。
又东南过莒县东,《地理志》曰:莒子之国,盈姓也,少昊后。
\end{yuanwen}

\begin{yuanwen}
	
\end{yuanwen}

\begin{yuanwen}
	
\end{yuanwen}\begin{yuanwen}

\end{yuanwen}\begin{yuanwen}

\end{yuanwen}\begin{yuanwen}

\end{yuanwen}\begin{yuanwen}

\end{yuanwen}\begin{yuanwen}

\end{yuanwen}\begin{yuanwen}

\end{yuanwen}\begin{yuanwen}

\end{yuanwen}\begin{yuanwen}

\end{yuanwen}\begin{yuanwen}

\end{yuanwen}\begin{yuanwen}

\end{yuanwen}\begin{yuanwen}

\end{yuanwen}\begin{yuanwen}

\end{yuanwen}\begin{yuanwen}

\end{yuanwen}\begin{yuanwen}

\end{yuanwen}
《列女传》曰:齐人杞梁殖袭莒,战死,二其妻将赴之,道逢齐庄公,公将吊之。杞梁妻曰:如殖死有罪,君何辱命焉?如殖无罪,有先人之敝庐在下,妾不敢与郊吊。公旋车吊诸室,妻乃哭于城下,七日而城崩。故《琴操》云:殖死,妻援琴作歌曰:乐莫乐兮新相知,悲莫悲兮生别离。哀感皇天,城为之堕。即是城也。其城三重,并悉祟峻,惟南开一门。内城方十二里,郭周四十许里。《尸子》曰:莒君好鬼巫而国亡。无知之难,小白奔焉。乐毅攻齐,守险全国。秦始皇县之。汉兴,以为城阳国,封朱虚侯章治莒。王莽之莒陵也。光武合城阳因为琅邪国,以封皇子京,雅好宫室,穷极伎巧,壁带饰以金银。明帝时,京不安莒,移治开阳矣。沭水又南,袁公水东出清山,遵坤维而注沭。沭水又南,浔水注之。水出于巨公之山,西南流,旧堨以溉田,东西二十里,南北十五里。浔水又西南流入沭,沭水又南与葛陂水会。水发三柱山,西南流,径辟土城南,世谓之辟阳城。《史记·建元以来王子侯者年表》曰:汉武帝元朔二年,封城阳共王子节侯刘壮为侯国也。其水于邑,积以为陂,谓之辟阳湖。西南流,注于沭水也。
又南过阳都县东,入于沂。
沭水自阳都县又南,会武阳沟水。水东出仓山,山上有故城,世谓之监宫城,非也。即古有利城矣。汉武帝元朔四年,封城阳共王子刘钉为侯国也。其城因山为基,水导山下西北流,谓之武阳沟。又西至即丘县,注于沭。沭水又南径东海郡即丘县,故《春秋》之祝丘也。桓公五年,《经》书:齐侯、郑伯如纪,城祝丘。《左传》曰:齐郑朝纪,欲袭之。汉立为县,王莽更之曰就信也。《郡国志》曰:自东海分属琅邪。阚駰曰:即祝鲁之音,盖字承读变矣。沭水又南径东海厚丘县,王莽更之曰祝其亭也。分为二渎,一渎西南出,今无水,世谓之枯沭。一渎南径建陵县故城东,汉景帝六年,封卫绾为侯国,王莽更之曰付亭也。沭水又南径建陵山西。魏正光中,齐王之镇徐州也,立大堨,遏水西流,两渎之会,置城防之,曰曲沭戌。自堨流三十里,西注沭水旧渎,谓之新渠。旧渎自厚丘西南出,左会新渠,南入淮阳宿预县,注泗水。《地理志》所谓至下邳注泗者也。《经》言于阳都入沂,非矣。沭水左渎自大堰水断,故渎东南出,桑堰水注之。水出襄贲县,泉流东注沭渎,又南,左合横沟水,水发渎右,东入沭之故渎。又南暨于遏,其水西南流,径司吾山东。又径司吾县故城西。《春秋左传》,楚执钟吾子以为司吾县,王莽更之曰息吾也。又西南至宿预注泗水也。沭水故渎,自下堰,东南径司吾城东,又东南历柤口城中。柤水出于楚之祖地。《春秋》襄公十年,《经》书公与晋及诸侯会吴于柤。京相璠曰:宋地。今彭城逼阳县西北有租水沟,去逼阳八十里。东南流,径傅阳县故城东北。《地理志》曰:故逼阳国也。《春秋左传》襄公十年夏,四月戊午,会于柤。晋荀偃、士匄请伐逼阳而封宋向戌焉。荀营曰:城小而固,胜之不武,弗胜为笑。固请,丙寅,围之,弗克。孟氏之臣秦堇父,辇重如役。逼阳人启门,诸侯之士门焉。县门发,鄹人纥抉之以出门者。狄虒弥建大车之轮,而蒙之以甲,以为橹,左执之,右拔戟,以成一队。盂献子曰:《诗》所谓有力如虎者也。主人县布,堇父登之,及堞而绝之,坠,则又县之。苏而复上者三。主人辞焉,乃退,带其断以徇于军三日。诸侯之师久于逼阳,请归。智伯怒曰:七日不克,尔乎取之,以谢罪也。荀偃、士匄攻之,亲受矢石,遂灭之,以逼阳子归,献于武官,谓之夷俘。逼阳,妘姓也。汉以为县,汉武帝元朔三年,封齐孝王子刘就为侯国,王莽更之曰辅阳也。《郡国志》曰:逼阳有柤水。柤水又东南,乱于沂而注于沭,谓之柤口,城得其名矣。东南至朐县,入游注海也。
巨洋水出朱虚县泰山北,过其县西,泰山,即东小泰山也。巨洋水,即《国语》所谓具水矣。袁宏谓之巨昧,王韶之以为巨蔑,亦或曰朐瀰,皆一水也,而广其目焉。其水北流,径朱虚县故城西。汉惠帝二年,封齐悼惠王子刘章为侯国。《地理风俗记》曰:丹山在西南,丹水所出,东入海。丹水由朱虚丘阜矣,故言朱虚。城西有长坂远峻,名为破车。岘城东北二十里有丹山,世谓之凡山,县在西南,非山也。丹、凡字相类,音从字变也。丹水有二源,各导一山,世谓之东丹、西丹水也。西丹水自凡山北流,径剧县故城东,东丹水注之。水出方山,山有二水,一水即东丹水也。北径县,合西丹水而乱流,又东北出径渏薄涧北。渏水亦出方山,流入平寿县,积而为清,水盛则北注,东南流,屈而东北流,径平寿县故城西,而北入丹水,谓之鱼合口。丹水又东北径望海台东,东北注海,盖亦县所氏者也。
又北过临胸县东,巨洋水自朱虚北入临朐县,熏冶泉水注之,水出西溪,飞泉侧濑,于穷坎之下,泉溪之上。源麓之侧,有一祠,目之为冶泉祠。按《广雅》,金神谓之清明,斯地盖古冶官所在,故水取称焉。水色澄明,而清泠特异,渊无潜石,浅镂沙文,中有古坛,参差相对,后人微加功饰,以为嬉游之处。南北邃岸凌空,疏木交合。先公以太和中,作镇海岱,余总角之年,侍节东州。至若炎夏火流,闲居倦想,提琴命友,嬉娱永日,桂笋寻波,轻林委浪,琴歌既洽,欢情亦畅,是焉栖寄,实可凭衿。小东有一湖,佳饶鲜笋,匪直芳齐芍药,实亦洁并飞鳞。其水东北流入巨洋,谓之熏冶泉。又径临朐县故城东。城,古伯氏骈邑也。汉武帝元朔元年,封菑川懿王子刘奴为侯国。应劭曰:临朐,山名也。故县氏之。朐亦水名,其城侧临朐川,是以王莽用表厥称焉。城上下沿水,悉是刘武皇北伐广固,营垒所在矣。巨洋又东北径委粟山东,孤阜秀立,形若委粟。又东北,洋水注之。水西出石膏山西北石涧口,东南径逢山祠西。洋水又东南,历逢山下,即石膏山也。山麓三成,壁立直上,山上有石鼓,鸣则年凶。郭缘生《续述征记》曰:逢山在广固南三十里,有祠并石鼓,齐地将乱,石人辄打石鼓,声闻数十里。洋水历其阴而东北流,世谓之石沟水。东北流,出于委粟山北,而东注于巨洋,谓之石沟口。然是水下流,亦有时通塞,及其春夏水泛,川澜无辍,亦或谓之为龙泉水。《地理志》:石膏山,洋水所出是也。今于此县,惟是渎当之,似符群证矣。巨洋水又东北,得邳泉口,泉源西出平地,东流注于巨洋水。巨洋水又北会建德水,水西发逢山阜,而东流入巨洋水也。
又北过剧县西,巨洋水又东北合康浪水,水发县西南山,无事树木,而圆峭孤峙,巑岏分立。左思《齐都赋》曰岭镇其左是也。康浪水北流,注于巨洋。巨洋又东北径剧县故城西,古纪国也。《春秋》庄公四年,纪侯不能下齐,以与弟季大去其国,违齐难也。后改曰剧,故《鲁连子》曰:朐剧之人,辩者也。汉文帝十八年,别为菑川国,后并北海。汉武帝元朔二年,封菑川懿王子刘错为侯国,王莽更之曰俞县也。城之北侧有故台,台西有方池。晏谟曰:西去齐城九十七里,耿弇破张步于临淄,追至巨洋水上,僵尸相属。即是水也。巨洋又东北径晋龙骧将军、幽州刺史辟闾浑墓东,而东北流。墓侧有一坟,甚高大,时人咸谓之为马陵,而不知谁之丘垄也。巨洋水又东北径益县故城东,王莽更之曰涤荡也。晏谟曰:南去齐城五十里,司马宣王代公孙渊,北徙丰人住于此城,遂改名为南丰城也。又东北积而为潭,枝津出焉、谓之百尺沟,西北流,径北益都城。汉武帝元朔二年封菑川懿王子刘胡为侯国。又西北流而注于巨淀矣。又东北过寿光县西,巨洋水自巨淀湖东北流,径县故城西,王莽之翼平亭也。汉光武建武二年,封更始子鲤为侯国。城之西南,水东有孔子石室,故庙堂也。中有孔子像,弟子问经,既无碑志,未详所立。巨洋又东北流,尧水注之。水出剧县南角崩山,即故义山也,俗人以其山角若崩,因名为角扇山,亦名为角林山,皆世俗音讹也。水,即蕤水矣。《地理志》曰:剧县有义山,蕤水所出也。北径山东,俗亦名之为青山矣。尧水又东北径东西寿光二城间。应劭曰:寿光县有灌亭。杜预曰:在县东南,斟灌国也。又言斟亭在平寿县东南,平寿故城在白狼水西,今北海郡治。水上承营陵县之下流,东北径城东,西入别画湖,亦曰朕怀湖。湖东西二十里,南北三十里,东北入海。斟亭在溉水东,水出桑犊亭东覆甑山。亭,故高密郡治,世谓之故郡城。山谓之塔山,水曰鹿孟水,亦曰戾孟水,皆非也。《地理志》:桑犊,北海之属县矣。有覆甑山,溉水所出,北径斟亭西北,合白狼水。按《地理志》:北海有斟县。京相璠曰:故斟寻国,禹后。西北去灌亭九十里。溉水又北径寒亭西,而入别画湖。《郡国志》曰:平寿有斟城,有寒亭。薛瓒《汉书集注》云:按《汲郡古文》,相居斟灌,东郡灌是也。明帝以封周后,改曰卫。斟寻在河南,非平寿也。又云:太康居斟寻,羿亦居之,桀又居之。《尚书序》曰:太康失国,兄弟五人,徯于河汭,此即太康之居,为近洛也。余考瓒所据,今河南有寻地,卫国有观土,《国语》曰:启有五观,谓之奸子。五观,盖其名也,所处之邑,其名曰观。皇甫谧曰:卫地。又云:夏相徒帝丘,依同姓之诸侯于斟寻氏。即《汲冢书》云相居斟灌也。既依斟寻,明斟寻非一居矣。穷后既仗善射篡相,寒促亦因逢蒙弑羿,即其居以生浇,因其室而有豷。故《春秋》襄公四年,魏绛曰:浇用师灭斟灌及斟寻氏,处浇于过,处豷干戈。是以伍员言于吴子曰:过浇杀斟灌以伐斟寻是也。有夏之遗臣曰靡,事羿,羿之死也,逃于鬲氏。今鬲县也。收斟灌、斟寻二国之余烬,杀寒浞而立少康,灭之,有穷遂亡也。是盖寓其居而生其称,宅其业而表其邑,纵遗文沿褫,亭郭有传,未可以彼有灌目,谓专此为非,舍此寻名,而专彼为是。以土推传,应氏之据亦可按矣。尧水又东北注巨洋。伏琛、晏谟并言,尧尝顿驾于此,放受名焉,非也。《地理志》曰:蕤水自剧东北,至寿光入海。沿其径趣,即是水也。又东北入于海。
巨洋水东北径望海台西,东北流。伏琛、晏谟并以为平望亭在平寿县故城西北八十里,古县;又或言秦始皇升以望海,因曰望海台,未详也。按《史记》,汉武帝元朔二年,封菑川懿王子刘赏为侯国。又东北注于海也。
淄水出泰山莱芜县原山,淄水出县西南山下,世谓之原泉。《地理志》曰:原山,淄水所出。故《经》有原山之论矣。《淮南子》曰:永出自饴山,盖山别名也。东北流,径莱芜谷,屈而西北流,径其县故城南。《从征记》曰:城在莱芜谷,当路阻绝,两山间道,由南北门。汉末,有范史云为莱羌令,言莱芜在齐,非鲁所得。引旧说云:齐灵公灭莱。莱民播流此谷,邑落荒芜,故曰莱芜。《禹贡》所谓莱夷也。夹谷之会,齐侯使莱人以兵劫鲁侯,宣尼称夷不乱华是也。余按泰无、莱柞,并山名也,郡县取目焉。汉高祖置。《左传》曰:与之无山及莱柞是也。应劭《十三州记》曰:太山莱芜县,鲁之莱柞邑。淄水又西北,转径城西,又东北流,与一水合。水出县东南,俗谓之家桑谷水。《从征记》名曰圣水。《列仙传》曰:鹿皮公者,淄川人也,少为府小史,才巧,举手成器。山岑上有神泉,人不能到,小史白府君,请木工斤斧三十人,作转轮,造县阁,意思横生,数十日,梯道成,上其巅,作祠屋留止其旁。其二问以自固,食芝草,饮神泉,七十余年。淄水来山下,呼宗族得六十余人,命上山半,水出尽漂一郡,没者万计。小史辞迫家室,令下山著鹿皮衣,升阁而去。后百余年下,卖药齐市也。其水西北流,注淄水,淄水又北出山,谓之莱芜口,东北流者也。
东北过临淄县东,淄水自山东北流,径牛山西,又东径临淄县故城南,东得天齐水口,水出南郊山下,谓之天齐渊。五泉并出,南北三百步,广十步。山即牛山也。左思《齐都赋》曰:牛岭镇其南者也。水在齐八祠中,齐之为名起于此矣。《地理风俗记》曰:齐所以为齐者,即天齐渊名也。其水北流,注于淄水,淄水又东径四豪冢北。水南山下,有四冢,方基圆坟,咸高七尺,东西直列。是田氏四王冢也。淄水又东北径荡阴里西。水东有冢,一基三坟,东西八十步,是列士公孙接、田开疆、古冶子之坟也。晏子恶其勇而无礼,投桃以毙之。死,葬阳里,即此也。淄水又北径其城东。城临淄水,故曰临淄,王莽之齐陵县也。《尔雅》曰:水出其前,左为营丘,武王以其地封大公望,赐之以四履,都营丘为齐。或以为都营陵。《史记》周成王封师尚父于营丘,东就国,道宿,行迟,莱侯与之争营丘。逆旅之人曰:吾同时难得而易失,客寝安,殆非就封者也。太公闻之,夜衣而行至营丘。陵亦丘也。献公自营丘徙临淄。余按营陵城南无水,惟城北有一水,世谓之白狼水,西出丹山,俗谓凡山也,东北流,由《尔雅》出前左之文,不得以为营丘矣。营丘者,山名也。《诗》所谓子之营兮,遭我乎峱之间兮。作者多以丘陵号同,缘陵又去莱差近,咸言太公所封。考之《春秋经》书,诸侯城缘陵。《左传》曰迁杞也。《毛诗》郑注,并无营字,瓒以为非近之。今临淄城中有丘,在小城内,周回三百步,高九丈,北降丈五,淄水出其前,故有营丘之名,与《尔雅》相符。城对天齐渊,故城有齐城之称。是以晏子言始爽鸠氏居之,逢伯陵居之,太公居之。又曰:先君太公筑营之丘,季札观风,闻齐音曰:泱泱乎,大风也哉!表东海者,其太公乎?田巴入齐,过淄自镜。郭景纯言齐之营丘,淄水径其南及东也,非营陵明矣。献公之徙,其犹晋氏深翼名绛,非谓自营陵而之也。其外郭,即献公所徙临淄城也,世谓之虏城。言齐湣王伐燕,燕王哙死,虏其民实诸郭,因以名之。秦始皇三十四年,灭齐为郡,治临淄。汉高帝六年,封子肥于齐为王国,王莽更名济南也,《战国策》曰:田单为齐相,过淄水,有老人涉淄而出,不能行,坐沙中,单乃解裘于斯水之上也。
又东过利县东,淄水自县东北流,径东安平城北。又东径巨淀县故城南。征和四年,汉武帝幸东莱,临大海,三月耕巨淀,即此也。县东南则巨淀湖,盖以水受名也。淄水又东北径广饶县故城南。汉武帝元鼎中,封菑川靖王子刘国为侯国。淄水又东北,马车渎水注之。受巨淀,淀即浊水所注也。吕忱曰:浊水一名溷水,出广县为山,世谓之冶岭山。东北流,径广固城西。城在广县西北四里,四周绝涧,阻水深隍,晋永嘉中,东莱人曹嶷所造也。水侧山际,有五龙口。义熙五年,刘武帝伐慕容超于广固也,以藉险难攻,兵力劳弊,河间人玄丈说裕云:昔赵攻曹嶷,望气者以为漫水带城,非可攻拔,若塞五龙口,城当必陷。石虎从之,嶷请降。降后五日,大雨,雷电震开。后慕容格之攻段龛,十旬不拔,塞口而龛降,降后无儿,又震开之。今旧基犹存,宜试修筑。裕塞之,超及城内男女皆悉脚弱,病者大半,超遂出奔,为晋所擒也。然城之所跨,实凭地险,其不可固城者在此。浊水东北流,径尧山东。《从征记》曰:广固城北三里,有尧山祠。尧因巡狩登此山,后人遂以名山,庙在山之左麓,庙像东面,华宇修整,帝图严饰,轩冕之容穆然,山之上顶,旧有上祠,今也毁废,无复遗式。盘石上尚有人马之迹,徒黄石而已,惟刀剑之踪逼真矣。至于燕锋代锷,魏铗齐鋩,与今剑莫殊,以密模写,知人功所制矣。西望胡公陵,孙畅之所云:青州刺史傅弘仁,言得铜棺隶书处。浊水又东北流,径东阳城北,东北流,合长沙水。水出逢山北阜,世谓之阳水也。东北流。径广县故城西。旧青州刺史治,亦曰青州城。阳水又东北流,石井水注之。水出南山,山顶洞开,望若门焉,俗谓是山为譬头山。其水北流注井,井际广城东侧,三面积石,高深一匹有余。长津激浪,瀑布而下,澎赑之音,惊川聒谷,漰渀之势,状同洪何。北流入阳水。余生长东齐,极游其下,于中阔绝,乃积绵载。后因王事,复出海岱。郭金紫惠同石井,赋诗言意,弥日嬉娱,尤慰羁心,但恨此本时有通塞耳。阳水东径故七级寺禅房南。水北则长庑遍驾,迥阁承阿,林之际则绳坐疏班,锡钵间设,所谓修修释子,眇眇禅栖者也。阳水又东径东阳城东南。义熙中,晋青州刺史羊穆之筑此,以在阳水之阳,即谓之东阳城。世以浊水为西阳水故也。水流亦有时穷通,信为灵矣。昔在宋世,是水绝而复流,刘晃赋《通津》焉,魏太和中,此水复竭,辍流积年,先公除州,即任未期,是水复通,澄映盈川,所谓幽谷枯而更溢,穷泉辍而复流矣。海岱之士,又颂通津焉。平昌尨民孙道相颂曰:惟彼渑泉,竭逾三龄,祈尽圭壁,谒穷斯牲,道从隆替,降由圣明。耋民河间赵嶷颂云:敷化未期,元泽潜施,枯源扬澜,涸川涤陂。北海郭钦曰:先政辍津,我后通洋。但颂广文烦,难以具载。阳水又北屈径汉城阳景王刘章庙东,东注于巨洋。后人堨断,令北注浊水。时人通谓浊水为阳水,故有南阳、北阳水之论。二水浑流,世谓之为长沙水也,亦或通名之为泥水。故晏谟、伏琛为《齐记》,并云东阳城既在渑水之阳,宜为渑阳城,非也。世又谓阳水为洋水,余按群书,盛言洋水出临朐县,而阳水导源广县,两县虽邻,川土不同,干事疑焉。浊水又北径臧氏台西,又北径益城西,又北流注巨淀。《地理志》曰:广县为山,浊水所出,东北至广饶入巨淀,巨淀之右,又有女水注之。水出东安平县之蛇头山,《从征记》曰:水西有桓公冢,甚高大,墓方七十余丈;高四丈,圆坟围二十余丈,高七丈余,一墓方七丈;二坟;晏谟曰:依《陵记》非葬礼,如承世故,与其母同墓而异坟,伏琛所未详也,冢东山下女水,原有桓公祠,侍其衡奏魏武王所立,曰:近日路次齐郊,瞻望桓公坟垄,在南山之阿,请为立祀,为块然之主。郭缘生《述征记》曰:齐桓公冢,在齐城南二十里,因山为坟,大冢东有女水,或云齐桓公女冢在其上,故以名水也。女水导川东北流,甚有神焉、化隆则水生,政薄则津竭。燕建平六年,水忽暴竭,玄明恶之,寝病而亡。燕太上四年,女水又竭,慕容超恶之,燕祚遂沦。女水东北流,径东安平县故城南。《续述征记》曰:女水至安平城南,伏流十五里,然后更流,北注阳水。城,故酅亭也。《春秋》鲁庄公三年,纪季以酅入齐。《公羊传》曰:季者何?纪侯弟也。贤其服罪,请酅以奉五祀,田成子单之故邑也。后以为县,博陵有安平,故此加东也。世祖建武七年,封菑川王子刘茂为侯国。又径东安平城东,东北径垄丘东,东北入巨淀。《地理志》曰:菟头山,女水所出,东北至临淄,人巨淀,又北为马车渎,北合淄水。又北,时渑之水注之。时水出齐城西北二十五里,平地出泉,即如水也。亦谓之源水。因水色黑,俗又目之为黑水。西北径黄山东,又北历愚山东,有愚公冢。时水又屈而径杜山北。有愚公谷,齐桓公时,公隐于谷,邻有认其驹者,公以与之。山即杜山之通阜,以其人状愚,故谓之愚公。水有石梁,亦谓之为石梁水。又有水注之。水出时水东,去临淄城十八里,所谓中也。俗以水为宿留水,西北入于时水。盂子去齐,三宿而后出,故世以此而变水名也。水南山西,有王歜墓,昔乐毅伐齐,贤而封之,歜不受,自缢而死。水侧有田引水,溉迹尚存。时水又西北径西安县故城南。本渠丘也,齐大夫雍廪之邑矣。王莽更之曰东宁。时水又西至石洋堰,分为二水,谓之石洋口。枝津西北至梁邹入济。时水又北径西安城西,又北,京水、系水注之。水出齐城西南,世谓之寒泉也。东北流,直申门西,京相璠、杜预并言申门,即齐城南面西第一门矣,为申池。昔齐懿公游申池,邴歜、阎职二人,害公于竹中。今池无复仿佛,然水侧尚有小小竹木,以时遗生也。左思《齐都赋注》:申池,在海滨齐薮也。余按《春秋》襄公十八年,晋伐齐,戊戌,伐雍门之萩。己亥,焚雍门,壬寅,焚东北二郭,甲辰,东侵及潍,南及沂,而不言北掠于海。且晋献子尚不辞死以逞志,何容对仇敌而不惩,暴草木于海嵎乎?又炎夏火流,非远游之辰,懿公见弑,盖是白龙鱼服,见困近郊矣。左氏舍近举远,考古非矣。杜预之言,有推据耳。系水傍城北流,径阳门西。水次有故封处,所谓齐之稷下也。当战国之时,以齐宣王喜文学,游说之士,邹衍、淳于髠、田骈、接子、慎到之徒七十六人,皆赐列第为上大夫,不治而论议,是以齐稷下学士复盛,且数百十人。刘向《别录》以稷为齐城门名也。谈说之士,期会于稷门下,故曰稷下也。《郑志》,张逸问《书赞》云:我先师棘下生,何时人?郑玄答云:齐田氏时,善学者所会处也,齐人号之棘下生,无常人也。余按《左传》昭公二十二年,莒子如齐,盟于稷门之外。汉以叔孙通为博士,号稷嗣君。《史记音义》曰:欲以继踪齐稷下之风矣。然棘下又是鲁城内地名。《左传》定公八年,阳虎劫公,伐孟氏,入自上东门,战于南门之内,又战于棘下者也。盖亦儒者之所萃焉。故张逸疑而发问,郑玄释而辩之,虽异名互见,大归一也。城内有故台,有营丘,有故景王祠,即朱虚侯章庙矣。《晋起居注》云:齐有大蛇,长三百步,负小蛇,长百余步,径于市中,市人悉观,自北门所入处也。北门外东北二百步,有齐相晏婴冢宅。《左传》:晏子之宅近市,景公欲易之,而婴弗更,为诫曰:吾生则近市,死岂易志。乃葬故宅,后人名之曰清节里。系水又北径临淄城西门北,而西流径梧宫南。昔楚使聘齐,齐王飨之梧宫,即是宫矣。其地犹名梧台里。台甚层秀,东西百余步,南北如减,即古梧宫之台。台东即阙子,所谓宋愚人得燕石处。台西有《石社碑》,犹存。汉灵帝熹平五年立,其题云梧台里。系水又西径葵丘北。《春秋》庄公八年,襄公使连称、管至父戍葵丘。京相璠曰:齐西五十里,有葵丘地,若是,无庸戍之。僖公九年,齐桓会诸侯于葵丘,宰孔曰:齐侯不务修德而勤远略。明葵丘不在齐也。引河东汾阴、葵丘,山阳西北葵城,宜在此,非也。余原《左传》连称、管至父之戍葵丘,以瓜时为往还之期。请代弗许,将为齐乱,故令无宠之妹,候公于宫。因无知之绌,遂害襄公。若出远无代,宁得谋及妇人,而为公室之乱乎?是以杜预稽《春秋》之旨,即传安之注于临淄西,不得舍近托远,苟成己异,于异可殊,即义为负。然则葵丘之戍,即此地也。系水西左迤为潭,又西径高阳侨郡南,魏所立也。又西北流注于时。时水又东北流,渑水庄之。水出营城东,世谓之汉溱水也。西北流,径营城北。汉景帝四年,封齐悼惠王子刘信都为侯国。渑水又西径乐安博昌县故城南。应劭曰:昌水出东莱昌阳县,道远不至,取其嘉名。阚駰曰:县处势平,故曰博昌。渑水西历贝丘。京相璠曰:博昌县南近渑水,有地名贝丘,在齐城西北四十里。《春秋》庄公八年,齐侯田于贝丘,见公子彭生,豕立而位。齐侯坠车伤足于是处也。渑水又西北人时水。《从征记》又曰:水出临淄县北,径乐安博昌南界,西入时水者也,自下通谓之为渑也。昔晋侯与齐侯宴,齐侯曰有酒如渑,指喻此水也。时水又屈而东北,径博昌城北。时水又东北,径齐利县故城北,又东北径巨淀县故城北,又东北径广饶县故城北,东北入淄水。《地理风俗记》曰:淄入濡。《淮南子》曰:白公问微言曰:若以水投水,如何?孔子曰:淄、渑之水合,易牙尝而知之。谓斯水矣。
又东北入于海。
淄水入马车渎,乱流东北,径琅槐故城南,又东北径马井城北,与时渑之水,互受通称,故邑流其号。又东北至皮丘坑入于海,故晏谟、伏琛并言淄渑之水,合于皮丘坑西。《地理志》曰:马车渎至琅槐入于海。盖举县言也。
汶水出朱虚县泰山,山上有长城,西接岱山,东连琅邪巨海,千有余里,盖田氏之所造也。
《竹书纪年》,梁惠成王二十年,齐筑防以为长城。《竹书》又云:晋烈公十二年,王命韩景子、赵烈子、翟员伐齐,入长城。《史记》所谓齐威王越赵侵我伐长城者也。伏琛、晏谟并言水出县东南峿山,山在小泰山东者也。北过其县东,汶水自县东北,径郚城北。《地理风俗记》曰:朱虚县东四十里,有郚城亭,故县也。又东北径管宁冢东。故晏谟言:柴阜西南,有魏独行君子管宁墓,墓前有碑。又东北径柴阜山北。山之东有征士邴原冢,碑志存焉。汶水又东北,径汉青州刺史孙嵩墓西,有碑碣。汶水又东径安丘县故城北。汉高帝八年,封将军张说为侯国。《地理志》曰:王莽之诛郅也。孟康曰:今渠丘亭是也。伏琛、晏谟《齐记》并言莒渠丘亭在安丘城东北十里,非矣。城对牟山,山之西南有孙宾硕兄弟墓,碑志并在也。
又北过淳于县西,又东北入于潍。
故夏后氏之斟灌国也。周武王以封淳于公,号曰淳于国。《春秋》桓公六年:冬,州公如曹。《传》曰:淳于公如曹,度其国危,遂不复也。其城东北,则两川交会也。
潍水出琅邪箕县潍山,琅邪,山名也,越王句践之故国也。句践并吴,欲霸中国,徙都琅邪。
秦始皇二十六年,灭齐,以为郡。城即秦皇之所筑也。遂登琅邪大乐之山,作层台于其上,谓之琅邪台。台在城东南十里,孤立,特显出于众山,上下周二十里余,傍滨巨海。秦王乐之,因留三月。乃徙黔首三万户于琅邪山下,复十二年。所作台基三层,层高三丈,上级平敞,方二百余步,广五里,刊石立碑,纪秦功德。台上有神渊,渊至灵焉。人污之则竭,斋洁则通。神庙在齐八祠中。汉武帝亦尝登之。汉高帝吕后七年,以为王国。文帝三年,更名为郡,王莽改曰填夷矣。潍水导源潍山。许慎、吕忱云:潍水出箕屋山。《淮南子》曰:潍水出覆舟山。盖广异名也。东北径箕县故城西,又西,析泉水注之。水出析泉县北松山东,南流径析泉县东,又东南径仲固山东,北流入于潍。《地理志》曰:至箕县北入潍者也。潍水又东北径诸县故城西。《春秋》文公十二年,季孙行父城诸及郓。《传》曰:城其下邑也。王莽更名诸并矣。潍水又东北,涓水注之。水出马耳山,山高百丈,上有二石并举,望齐马耳,故世取名焉。东去常山三十里,涓水发于其阴,北径娄乡城东。《春秋》昭公五年,《经》书夏莒牟夷,以牟娄防兹来奔者也。又分诸县之东,为海曲县,故俗人谓此城为东诸城。涓水又北注于潍水。
东北过东武县西,县因冈为城,城周三十里。汉高帝六年,封郭蒙为侯国,王莽更名之曰祥善矣。又北,左合扶淇之水,水出西南常山,东北流注潍。晏、伏并以潍水为扶淇之水,以扶淇之水为潍水,非也。按经脉志,潍自箕县北,径东武县西,北流合扶淇之水。晏谟、伏琛云:东武城西北二里潍水者,即扶淇之水也。潍水又北,右合卢水,即久台水也。《地理志》曰:水出琅邪横县故山,王莽之令丘也。山在东武县故城东南,世谓之卢山也。西北流,径昌县故城西,东北流。《齐地记》曰:东武城东南有卢水,水侧有胜火木,方俗音曰柽子,其木经野火烧死,炭不灭。故东方朔云:不灰之木者也。其水又东北流,径东武县故城东,而西北入潍。《地理志》曰:久台水东南至东武入潍者也。《尚书》所谓潍、淄其道矣。
又北过平昌县东,潍水又北径石泉县故城西,王莽之养信也。《地理风俗记》曰:平昌县东南四十里,有石泉亭,故县也。潍水又北径平昌县故城东,荆水注之。水出县南荆山阜,东北流径平昌县故城东。汉文帝封齐悼惠王肥子印为侯国。城之东南角有台,台下有井,与荆水通。物坠于井,则取之荆水,昔常有龙出入于其中,故世亦谓之龙台城也。荆水又东北流,注于潍。潍水又北,浯水注之,水出浯山,世谓之巨平山也。《地理志》曰:灵门县有高山、壶山,浯水所出,东北入潍,今是山西接浯山。许慎《说文》言水出灵门山,世谓之浯汶矣。其水东北径姑幕县故城东。县有五色土,王者封建诸侯,随方受之。故薄姑氏之国也。阚駰曰:周成王时,薄姑与四国作乱,周公灭之,以封太公。是以《地理志》曰:或言薄姑也。王莽曰季睦矣。应劭曰:《左传》曰薄姑氏国,太公封焉。薛瓒《汉书注》云:博昌有薄姑城。未知孰是。浯水又东北径平昌县故城北,古堨此水,以溢溉田,南注荆水。浯水又东北流,而注于潍水也。
又北过高密县西,应劭曰:县有密水,故有高密之名也。然今世所谓百尺水者,盖密水也。水有二源,西源出奕山,亦曰鄣日山。山势高峻,隔绝阳曦。晏谟曰:山状鄣日,是有此名。伏琛曰:山上鄣日,故名鄣日山也。其水东北流,东源出五弩山,西北流,同泻一壑,俗谓之百尺水,古人堨以溉田数十顷。北流径高密县西,下注潍水,自下亦兼通称焉。乱流历县西碑产山西。又东北,水有故堰,旧凿石竖柱,断潍水,广六十许步,掘东岸,激通长渠,东北径高密县故城南。明帝永平中,封邓震为侯国。县南十里,蓄以为塘,方二十余里,古所谓高密之南都也,溉田一顷许。陂水散流,下注夷安泽。潍水自堰北,径高密县故城西。汉文帝十六年,别为胶西国。宣帝本始元年,更为高密国。王莽之章牟也。潍水又北。昔韩信与楚将龙且,夹潍水而阵于此。信夜令为万余囊,盛沙以遏潍水,引军击且,伪退,且追北,信决水,水大至,且军半不得渡,遂斩龙且于是水。水西有厉阜,阜上有汉司农卿郑康成冢,石碑犹存。又北径昌妥县故城东。汉明帝永平中,封邓袭为侯国也。《郡国志》曰:双安帝延光元年复也。
又北过淳于县东,潍水又北,左会汶水,北径平城亭西,又东北径密乡亭西。《郡国志》曰:淳于县有密乡。《地理志》,皆北海之属县也。应劭曰:淳于县东北六十里,有平城亭,又四十里,有密乡亭,故县也。潍水又东北径下密县故城西。城东有密阜。《地理志》曰:有三户山祠。余按应劭曰:密者水名,是有下密之称。俗以之名阜,非也。
又东北过都昌县东,潍水东北径逢萌墓。萌,县人也,少有大节,耻给事县亭,遂浮海至辽东,复还,在不其山隐学。明帝安车征萌,以佯狂免。又北径都昌县故城东。汉高帝六年,封朱轸为侯国。北海相孔融,为黄巾贼管亥所围于都昌也,太史慈为融求救,刘备持的突围其处也。又东北入于海。
胶水出黔陬县胶山北,过其县西,《齐记》曰:胶水出五弩山,盖胶山之殊名也。北径祝兹县故城东。汉武帝元鼎中,封胶东康王子延为侯国。又径扶县故城西,《地理志》,琅邪之属县也。汉文帝元年,封吕平为侯国。胶水又北径黔陬故城西。袁山松《郡国志》曰:县有介亭。《地理志》曰:故介国也。《春秋》僖公九年,介葛卢来朝,闻牛鸣曰:是生三牺,皆用之。问之,果然。晏谟、伏琛并云县有东西二城,相去四十里有胶水,非也。斯乃拒艾水也。水出县西南拒艾山,即《齐记》所谓黔艾山也。东北流,径柜县故城西,王莽之祓同也,世谓之王城。又谓是水为洋水矣。又东北流,晏,伏所谓黔陬城西四十里有胶水者也。又东入海。《地理志》:琅邪有柜县,根艾水出焉。东入海,即斯水也。今胶水北流径西黔陬城东,晏、伏所谓高密郡侧有黔陬县。《地理志》曰:胶水出邞县,王莽更之纯德矣,疑即是县,所未详也。
又北过夷安县东,县,故王莽更名之原亭也。应劭曰:故莱夷维邑也。太史公曰:晏平仲莱之夷维之人也。汉明帝永平中,封邓珍为侯国。西去潍水四十里。胶水又北径胶阳县东,晏、伏并谓之东亭,自亭结路,南通夷安。《地理风俗记》曰:淳于县东南五十里,有胶阳亭,故县也。又东北流,左会一水,世谓之张奴水。水发夷安县东南阜下,西北流,历胶阳县,注于胶。胶水之左为泽渚,东北百许里,谓之夷安潭。潭周四十里,亦潍水枝津之所注也。胶水又东北,径下密县故城东,又东北,径胶东县故城西。汉高帝九年,别为国。景帝封子寄为王国,王莽更之郁袟也,今长广郡治。伏琛、晏谟言胶水东北回,达于胶东城北百里,流注于海。又北过当利县西北,入于海。
县,故王莽更名之为东莱亭也。又北径平度县。汉武帝元朔二年,封菑川懿王子刘衍为侯国,王莽更名之曰利卢也。县有土山,胶水北历土山,注于海。海南,土山以北,悉盐坑相承,修煮不辍,北眺巨海,杏冥无极,天际两分,白黑方别,所谓溟海者也。故《地理志》曰:胶水北至平度入海也。


卷二十七  沔水 
沔水出武都沮县东狼谷口,沔水一名沮水。阚駰曰:以其初出沮洳然,故曰沮水也。县亦受名焉。
导源南流,泉街水注之。水出河池县,东南流,入沮县;会于沔。沔水又东南,径沮水戍而东南流,注汉,曰沮口。所谓沔汉者也。《尚书》曰:.冢导漾,东流为汉。《山海经》所谓汉出鲋嵎山也。东北流,得献水口。庾仲雍云:是水南至关城,合西汉水。汉水又东北,合沮口,同为汉水之源也。故如淳曰:此方人谓汉水为沔水。孔安国曰:漾水东流为沔,盖与沔合也。至汉中为汉水,是互相通称矣。沔水又东径白马戍南、浕水入焉。水北发武都氏中,南径张鲁城东。鲁,沛国张陵孙。陵学道于蜀鹤鸣山,传业衡,衡传于鲁。鲁至,行宽惠,百姓亲附,烘道之费,米限五斗,故世号五斗米道。初平中,刘焉以鲁为督义司马,住汉中,断绝谷道,用远城治,因即崤岭,周回五里,东临濬谷,杳然百寻。西北二面,连峰接崖,莫究其极。从南为盘道,登陟二里有余。浕水又南径张鲁治东,水西山上,有张天师堂,于今民事之。庾仲雍谓山为白马塞,堂为张鲁治。东对白马城,一名阳平关。浕水南流入沔,谓之浕口。其城西带浕水,南面沔川,城侧二水之交,故亦曰浕口城矣。沔水又东径武侯垒南,诸葛武侯所居也,南枕沔水。水南有亮垒,背山向水,中有小城,回隔难解。沔水又东,径沔阳县故城南。城,旧言汉祖在汉中,萧何所筑也。汉建安二十四年,刘备并刘璋,北定汉中,始立坛,即汉中王位于此。其城南临汉水,北带通逵,南面崩水三分之一,观其遗略,厥状时传。南对定军山,曹公南征汉中,张鲁降,乃命夏侯渊等守之。刘备自阳平关南渡沔水,遂斩渊首,保有汉中,诸葛亮之死也,遗令葬于其山,因即地势,不起坟垄,惟深松茂柏,攒蔚川阜,莫知墓茔所在。山东名高平,是亮宿营处,有亮庙。亮薨,百姓野祭。步兵校尉习隆、中书郎向充共表云:臣闻周人思召伯之德,甘棠为之不伐。越王怀范蠡之功,铸金以存其像。亮德轨遐迩,勋盖来世,王室之不坏,实赖斯人。而使百姓巷祭,戎夷野祀,非所以存德念功,追述在昔者也。今若尽顺民心,则黩而无典,建之京师,又逼宗庙,此圣怀所以惟疑也。臣谓宜近其墓,立之沔阳,断其私祀,以崇正礼。始听立祀,斯庙盖所启置也。钟士季征蜀,枉驾设祠茔东,即八阵图也。遗基略在,崩褫难识。沔水又东径西乐城北。城在山上,周三十里,甚险固。城侧有谷,谓之容裘谷。道通益州,山多群獠,诸葛亮筑以防遏。梁州刺史杨亮以即险之固,保而居之,力苻坚所败。后刺史姜守、潘猛,亦相仍守此城。城东容裘,溪水注之,俗谓之洛水也。水南导巴岭山,东北流。水左有故城,凭山即险,四面阻绝。昔先主遣黄忠据之,以拒曹公。溪水又北径西乐城东,而北流注于汉。汉水又左得度口水,出阳平北山。水有二源,一曰清检,出佳鳠;一曰浊检,出好鲋,常以二月八月取之,美珍常味。度水南径阳平县故城东,又南径沔阳县故城东,西南流,注于汉水。汉水又东,右会温泉水口。水发山北平地,方数十步,泉源沸涌,冬夏汤汤,望之则白气浩然,言能瘥百病云。洗浴者,皆有硫黄气,赴集者常有百数。池水通注汉水。汉水又东,黄沙水左注之。水北出远山,山谷邃险,人迹罕交,溪曰五丈溪。水侧有黄沙屯,诸葛亮所开也。其水南注汉水。南有女郎山,山上有女郎冢。远望山坟,嵬嵬状高,及即其所,裁有坟形。山上直路下出,不生草木,世人谓之女郎道。下有女郎庙及捣衣石,言张鲁女也。有小水北流入汉,谓之女郎水。汉水又东合褒水,水西北出衙岭山,东南径大石门,历故栈道下谷,俗谓千梁无柱也。诸葛亮《与兄瑾书》云:前赵子尤退军,烧坏赤崖以北阁道,缘谷百余里,其阁梁一头入山腹,其一头立柱于水中,今水大而急,不得安柱,此其穷极不可强也。又云:顷大水暴出,赤崖以南桥阁悉坏,时赵子龙与邓伯苗,一戍赤崖屯田,一戍赤崖口,但得缘崖,与伯苗相闻而已。后诸葛亮死于五丈原,魏延先退而焚之,谓是道也。自后按旧修路者,悉无复水中柱,径涉者,浮梁振动,无不遥心眩目也。褒水又东南径三交城,城在三水之会故也。一水北出长安,一水西北出仇池,一水东北出太白山,是城之所以取名矣。褒水又东南,得丙水口,水上承丙穴,穴出嘉鱼,常以三月出,十月入地,穴口广五六尺,去平地七八尺,有泉悬注,鱼自穴下透入水,穴口向丙,故曰丙穴。下注褒水,故左思称嘉鱼出于丙穴,良木攒于褒谷矣。褒水又东南历小石门。门穿山通道,六丈有余。刻石言:汉明帝永平中,司隶校尉犍为杨厥之所开,逮桓帝建和二年,汉中太守同郡王升,嘉厥开凿之功,琢石颂德。以为石牛道。来敏《本蜀论》云:秦惠王欲伐蜀而不知道,作五石牛,以金置尾下,言能屎金,蜀王负力,令五丁引之成道。秦使张仪、司马错寻路灭蜀,因曰石牛道,厥盖因而广之矣。《蜀都赋》曰:阻以石门,其斯之谓也。门在汉中之西,褒中之北。褒水又东南历褒口,即褒谷之南口也。北口曰斜,所谓北出褒斜。褒水又南径褒县故城东,褒中县也。本褒国矣,汉昭帝元凤六年置。褒水又南流,入于汉。汉水又东径万石城下。城在高原上,原高十余丈,四面临平,形若覆瓮。水南遏水为阻,西北并带汉水,其城宿是流杂聚居,故世亦谓之流杂城。汉水又东径汉庙堆下。昔汉女所游,侧水为钓台,后人立庙于台上。世人睹其颓基崇广,因谓之汉庙堆。传呼乖实,又名之为汉武堆,非也。
东过南郑县南,县,故褒之附庸也。周显王之世,蜀有褒汉之地。至六国,楚人兼之。
怀王衰弱,秦略取焉。周赧王二年,秦惠王置汉中郡,因水名也。《耆旧传》云:南郑之号,始于郑桓公。桓公死于大戎,其民南奔,故以南郑为称,即汉中郡治也。汉高祖入秦,项羽封为汉王。萧何曰:天汉美名也。遂都南郑。大城周四十二里,城内有小城,南凭津流,北结环雉,金墉漆井,皆汉所修筑。地沃川险,魏武方之鸡肋,曰:释骐骥而不乘焉,皇皇而更求。遂留杜子绪镇南郑而还。晋咸康中,梁州刺史司马勋,断小城东面三分之一以为梁州,汉中郡南郑县治也。自齐、宋、魏咸相仍焉。水南,即汉阴城也,相承言吕后所居也。有廉水出巴岭山,北流径廉川,故水得其名矣。廉水又北住汉水。汉水右合池水,水出旱山。山下有祠,列石十二,不辨其由,盖社主之流,百姓四时祈祷焉。俗谓之獠子水,夹溉诸田,散流左注汉水。汉水又东,得长柳渡。长柳,村名也。汉太尉李固墓,碑铭尚存,文字剥落,不可复识。汉水又东径胡城南。义熙十五年,城上有密云细雨,五色昭彰,人相与谓之庆云休符。当出晓而云霁,乃觉城崩,半许沦水,出铜钟十二枚。刺史索邈奉送洛阳,归之宋公府。南对扁鹊城,当是越人旧所径涉,故邑流其名耳。汉水出于二城之间,右会磐余水。水出南山巴岭上,泉流两分,飞清派注,南入蜀水,北注汉津,谓之磐余口。庾仲雍曰:磐余去胡城二十里。汉水又左会文水,水即门水也,出胡城北山石穴中。长老云:杜阳有仙人宫,石穴宫之前门,故号其川为门川,水为门水。东南流,径胡城北。三城奇对,隔谷罗布,深沟固垒,高台相距。门水右注汉水,谓之高桥溪口。汉水又东,黑水注之。水出北山,南流入汉。庾仲雍曰:黑水去高桥三十里。诸葛亮笺云:朝发南郑,暮宿黑水,四五十里。指谓是水也,道则百里也。
又东过成固县南,又东过魏兴安阳县南,涔水出自旱山,北注之。
常璩《华阳国志》曰:蜀以成因为乐城县也。安阳县故隶汉中,魏分汉中,立魏兴郡,安阳隶焉。涔水出西南而东北人汉。左谷水出西北,即婿水也。北发听山,山下有穴水,穴水东南流,历平川中,谓之婿乡,水口婿水。川有唐公祠。唐君字公房,成固人也。学道得仙,入云台山,合丹服之,白日升天、鸡鸣天上,狗吠云中,惟以鼠恶,留之。鼠乃感激,以月晦日,吐肠胃更生,故时人谓之唐鼠也。公房升仙之日,婿行未还,不获同阶云路,约以此川为居,言无繁霜蛟虎之患,其俗以为信然。因号为婿乡,故水亦即名焉。百姓为之立庙于其处也,刊石立碑,表述灵异。婿水南历婿乡溪,出山东南流,径通关势南。山高百余丈,上有匈奴城,方五里,濬堑三重,高祖北定三秦,萧何守汉中,欲修北道通关中,故名为通关势。婿水又东径七女冢。冢夹水罗布,如七星,高十余丈,周回数亩。元嘉六年,大水破坟,坟崩,出铜不可称计。得一砖,刻云:项氏伯无子,七女造墩。世人疑是项伯冢。水北有七女池,池东有明月池,状如偃月,皆相通注,谓之张良渠,盖良所开也。婿水径樊哙台南,台高五六丈,上容百许人。又东南径大成固北。城乘高势,北临婿水,水北有韩信台,高十余丈,上容百许人。相传高祖斋七日,置台设九宾礼,以礼拜信也。婿水东回南转,又径其城东,而南入汉水,谓之三水口也。汉水又东会益口,水出北山益谷,东南流,注于汉水。汉水又东至城南,与洛谷水合。水北出洛谷,谷北通长安,其水南流,右则水注之。水发西溪,东南流,合为一水,乱流南出,际其城西,南注汉水。汉水又东径小成固南。州治大成固,移县北,故曰小成固。城北百二十里,有兴势坂,诸葛亮出洛谷,戍兴势,置烽火楼处。通照汉水,东历上涛,而径于龙下,盖伏石惊湍,流屯激怒,故有上下二涛之名。龙下,地名也。有丘郭坟墟,旧谓此馆为龙下亭。自白马迄此,则平川夹势,水丰壤沃,利方三蜀矣。度此溯洄从汉,为山行之始。汉水又东径石门滩,山峡也,东会西水,水北出秦岭西谷,南历重山。与寒泉合。水东出寒泉岭,泉涌山顶,望之交横似若瀑布,颓波激石,散若雨洒,势同厌原风雨之池。其水西流入于西水。西水又南注汉,谓之西口。汉水又东径妫虚滩。《世本》曰:舜居妫汭,在汉中西城县,或言妫墟在西北,舜所居也。或作姚虚,故后或姓姚,或姓妫,妫姚之异,是妄未知所从。余按应劭之言,是地于西城为西北也。汉水又东径猴径滩,山多猴猿,好乘危缀饮,故滩受斯名焉。汉水又东径小大黄金南。山有黄金峭,水北对黄金谷,有黄金戍,傍山依峭,险折七里。氏掠汉中,阻此为戍,与铁城相对,一城在山上,容百余人,一城在山下,可置百许人,言其险峻,故以金铁制名矣。昔杨难当令魏兴太守薛健据黄金,姜宝据铁城。宋遣秦州刺史萧思话西讨,话令阴平太守萧垣攻拔之,贼退酉水矣。汉水又东,合蘧蒢溪口。水北出就谷,在长安西南,其水南流,径巴溪戍西,又南径阳都坂东。坂自上及下盘折十九曲,西连寒泉岭。《汉中记》曰:自西城涉黄金峭、寒泉岭、阳都坂,峻崿百重,绝壁万寻,既造其峰,谓已逾崧岱,复瞻前岭,又倍过之。言陟羊肠,超烟云之际,顾看向途,杳然有不测之险。山丰野牛野羊,腾岩越岭,驰走若飞,触突树木,十围皆倒,山殚民阻,地穷坎势矣。其水南历蘧蒢溪,谓之蘧蒢水,而南流注于汉,谓之蒢口。汉水又东,右会洋水,川流漫阔,广几里许。洋水导源巴山,东北流,径平阳城。《汉中记》曰:本西乡县治也。自成固南入三百八十里,距南郑四百八十里。洋川者,汉戚夫人之所生处也。高祖得而宠之,夫人思慕本乡,追求洋川米,帝为驿致长安。蠲复其乡,更名曰县,故又目其地为祥川,用表夫人载诞之休祥也。城即定远矣。汉顺帝永光七年,封班超以汉中郡南郑县之西乡为定远侯,即此也。洋水又东,北流入汉。谓之城阳水口也。汉水又东历敖头。旧立仓储之所,傍山通道,水陆险凑。魏兴安康县治,有戍统领流杂。汉水又东合直水,水北出子午谷岩岭下。又南枝分,东注旬水,又南径蓰阁下,山上有戍,置于崇阜之上,下临深渊。张子房烧绝栈阁,示无还也。又东南历直谷,径直城西,而南流注汉。汉水又东径直城南,又东径千渡,而至虾蟆頧,历汉阳口,而届于彭溪龙灶矣,并溪涧滩碛之名也。汉水又东,径晋昌郡之宁都县南,县治松溪口。又东径魏兴郡广城县,县治王谷。谷道南出巴獠,有盐井,食之令人瘿疾。汉水又东径鱼脯谷口,旧西城、广城二县。指此谷而分界也。
又东过西城县南。
汉水又东径鳖池南鲸滩,鲸,大也。《蜀都赋》曰:流汉汤汤,惊浪雷奔,望之天回,即之云昏者也。汉水又东径岚谷北口,嶂远溪深,涧峡险邃,气萧萧以瑟瑟,风而飕飕,故川谷擅其目矣。汉水又东,右得大势,势阻急溪,故亦曰急势也。依山为城,城周二里,在峻山上,梁州督护吉挹所治,苻坚遣偏军韦钟伐挹,挹固守二年不能下,无援遂陷。汉水右对月谷口,山有坂月川于中,黄壤沃衍,而桑麻列植,佳饶水田,故孟达与诸葛亮书,善其川土沃美也。汉水又东径西城县故城南。《地理志》:汉中郡之属县也。汉未,为西城郡。建安二十四年,刘备以申仪为西城太守,仪据郡降魏,魏文帝改为魏兴郡,治故西城县之故城也。氏略汉川梁州,移治于此城。内有舜祠、汉高帝庙,置民九户,岁时奉祠焉。汉水又东为鱣湍,洪波渀荡,漰浪云颓。古耆旧言,有鱣鱼奋鳍溯流,望涛直上,至此则暴鳃失济,故因名湍矣。汉水又东合旬水,水北出旬山,东南流,径平阳戍下,与直水枝分东注,径平阳戍,入旬水,旬水又东南径旬阳县,与柞水合。水西出柞溪,南流径重岩堡西,屈而东流,径其堡南,东南注于旬水。旬水又东南径旬阳县南。县北山有悬书崖,高五十丈,刻石作字,人不能上,不知所道。山下有石坛,上有马迹五所,名曰马迹山。旬水东南注汉,谓之旬口。汉水又东径木兰寨南,右岸有城,名伎陵城,周回数里,左岸垒石数十行,重垒数十里,中谓是处为木兰寨,云:吴朝遣军救盂达于此矣。汉水又东,左得育溪,兴晋、旬阳二县分界于是谷。汉水又东合甲水口,水出秦岭山,东南流,径金井城南,又东径上庸郡北,与关祔水合。水出上洛阳亭县北青泥西山,南径阳亭聚西,俗谓之平阳水。南合丰乡川水,水出弘农丰乡东山,西南流径丰乡故城南。京相璠曰:南乡淅县,有故酆乡,《春秋》所谓丰浙也。于《地理志》属弘农,今属南乡。又西南合关淅水。关祔水又南入上津,注甲水。甲水又东南径魏兴郡之兴晋县南,晋武帝太康中立。甲水又东,右入汉水。汉水又东为龙渊,渊上有胡鼻山,石类胡人鼻故也,下临龙井渚,渊深数丈。汉水又东径魏兴郡之锡县故城北,为白石滩。县故《春秋》之锡穴地也,故属汉中,王莽之锡治也。具有锡义山,方圆百里,形如城,四面有门,上有石坛,长数十丈,世传列仙所居,今有道士被发饵术,恒数十人。山高谷深,多生薇蘅草,其草有风不偃,无风独摇。汉水又东径长利谷南,入谷有长利故城,旧县也。汉水又东历姚方,盖舜后枝居是处,故地留姚称也。


卷二十八  沔水 
沔水又东过堵阳县,堵水出自上粉县,北流注之。
堵水出建平郡界故亭谷,东历新城郡。郡,故汉中之房陵县也。世祖建武元年,封邓晨为侯国。汉末以为房陵郡。魏文帝合房陵、上庸、西城,立以为新城郡,以盂达为太守,治房陵故县。有粉水,县居其上,故曰上粉县也。堵水之旁有别溪,岸侧土色鲜黄,乃云可啖。有言饮此水者,令人无病而寿,岂有信乎?又有白马山,山石似马,望之逼真。侧水谓之白马塞。盂达为守,登之而叹曰:刘封、申耽据金城千里,而更失之乎!为《上堵吟》,音韵哀切,有恻人心,今水次尚歌之。堵水又东北径上庸郡,故庸国也。《春秋》文公十六年,楚人、秦人、巴人灭庸。庸,小国,附楚,楚有灾不救,举群蛮以叛,故灭之以为县,属汉中郡。汉末又分为上庸郡,城三面际水。堵水又东径方城亭南,东北历山下,而北径堵阳县南,北流注于汉,谓之堵口。汉水又东,谓之涝滩,冬则水浅,而下多大石。又东为净滩,夏水急盛,川多湍洑,行旅苦之。故谚曰:冬涝夏净,断官使命。言二滩阻碍。又东过郧乡南,汉水又东径郧乡县南之西山,上有石虾蟆,仓卒看之,与真不别。汉水又东径郧乡县故城南,谓之郧乡滩。县故黎也,即长利之郧乡矣。《地理志》曰:有郧关,李奇以为郧子国。晋太康五年,立以为县。汉水又东径琵琶谷口,梁、益二州,分境于此,故谓之琵琶界也。
又东北流,又屈东南,过武当县东北,县西北四十里,汉水中有洲,名沧浪洲。庾仲雍《汉水记》谓之千龄洲,非也。是世俗语讹,音与字变矣。《地说》曰:水出荆山,东南流,为沧浪之水,是近楚都。故渔父歌曰:沧浪之水清兮,可以濯我缨;沧浪之水浊兮,可以濯我足。余按《尚书·禹贡》言,导漾水东流为汉,又东为沧浪之水,不言过而言为者,明非他水决入也,盖汉沔水自下有沧浪通称耳。缠络鄢郢,地连纪鄀。咸楚都矣。渔父歌之,不违水地,考按经传,宜以《尚书》为正耳。汉水又东为佷子潭,潭中有石碛洲,长六十丈,广十八丈,世亦以此洲为佷子葬父于斯,故潭得厥目焉,所未详也。汉水又东南径武当县故城北,世祖封邓晨子棠为侯国。内有一碑,文字磨灭,不可复识,俗相传言是《华君铭》,亦不详华君何代之士。汉水又东,平阳川水注之。水出县北伏亲山,南历平阳川,径平阳故城下,又南流注于沔。沔水又东南径武当县故城东,又东,曾水注之。水导源县南武当山,一曰太和山,亦曰上山,山形特秀,又曰仙室。《荆州图副记》曰:山形特秀,异于众岳,峰首状博山香炉,亭亭远出,药食延年者萃焉。晋咸和中,历阳谢允,舍罗邑宰,隐遁斯山,故亦曰谢罗山焉。曾水发源山麓,径越山阴,东北流注于沔,谓之曾口。沔水又东径龙巢山下。山在沔水中,高十五丈,广员一里二百三十步,山形峻峭,其上秀林茂木,隆冬不凋。
又东南过涉都城东北, 故乡名也。按《郡国志》,筑阳县有涉都乡者也。汉武帝元封元年,封南海守降侯子嘉为侯国。均水于县人酒,谓之均口也。
又东南过酇县之西南,县治故城南临沔水,谓之酇头。汉高帝五年,封萧何为侯国也。薛瓒曰:今南乡酇头是也。《茂陵书》曰在南阳,王莽更名南庚者也。
又南过谷城东,又南过阴县之西,沔水东径谷城南,而不径其东矣。城在谷城山上,《春秋》谷伯绥之邑也。墉闽颓毁,基堑亦存。沔水又东南径阴县故城西,故下阴也。《春秋》昭公十九年,楚工尹赤迁阴于下阴是也。县东有冢。县令济南刘熹,字德怡。魏时宰县,雅好博古,教学立碑,载生徒百有余人,不终业而天者,因葬其地,号曰生坟。沔水又东南得洛溪口,水出县西北集池陂,东南流,径洛阳城,北枕洛溪,溪水东南注沔水也。
又南过筑阳县东,筑水出自房陵县,东过其县,南流注之。
沔水又南,泛水注之。水出粱州阆阳县。魏遣夏侯渊与张沔下巴西,进军宕渠。刘备军泛口,即是水所出也。张飞自别道袭张于此水, 败,弃马升山,走还汉中。泛水又东径巴西,历巴渠北新城、上庸,东径泛阳县故城南,晋分筑阳立。自县以上,山深水急,在渚崩湍,水陆径绝。又东径学城南,梁州大路所由也。旧说,昔者有人立学都于此,值世荒乱,生徒罔依,遂共立城以御难,故城得厥名矣。泛水又东流注于沔,谓之泛口也。沔水又南径阙林山东,本郡陆道之所由。山东有二碑,其一即记阙林山,文曰:君国者不跻高堙下,先时或断山冈,以通平道,民多病,守长冠军张仲瑜乃与邦人筑断故山道,作此铭。其一《郭先生碑》,先生名辅,字甫成,有孝友悦学之美,其女为立碑于此,无年号,皆不知何代人也。沔水又南径筑阳县东,又南,筑水注之。杜预以为彭水也。水出梁州新城郡魏昌县界,县以黄初中分房陵立。筑水东南流径筑阳县,水中有孤石挺出,其下澄潭,时有见此石根,如竹根而黄色,见者多凶,相与号为承受石,所未详也。筑水又东径筑阳县故城南,县故楚附庸也。秦平鄢郢,立以为县。王莽更名之曰宜禾也。建武二十八年,世祖封吴盱为侯国。筑水又东流注于沔,谓之筑口。沔水又南径高亭山东,山有灵焉,士民奉之,所请有验。沔水又东为漆滩,新野郡山都县与顺阳,筑阳分界于斯滩矣。
又东过山都县东北,河南有固城,城侧沔川,即新野山都县治也。旧南阳之赤乡矣,秦以为县。汉高后四年,封卫将军王恬启为侯国。沔北有和城,即《郡国志》所谓武当县之和城聚,山都县旧尝治此,故亦谓是处为故县滩。沔水北岸数里,有大石激,名曰五女激。或言:女父为人所害,居固城,五女思复父怨,故立激以攻城。城北今沦于水。亦云:有人葬沔北,墓宅将为水毁,其人五女无男,皆悉巨富,共修此激以全坟宅。然激作甚工。又云:女嫁为阴县佷子妇,家赀万金,而自少小不从父语。父临亡,意欲葬山上,恐儿不从,故倒言葬我著渚下石碛上。佷子曰:我由来不奉教,今从语。遂尽散家财,作石冢,积土绕之,成一洲,长数百步。元康中,始为水所坏,今石皆如半榻许,数百枚聚在水中。佷子是前汉人。襄阳太守胡烈有惠化,补塞堤决,民赖其利。景元四年九月,百姓刊石铭之,树碑于此。沔水又东,偏浅,冬月可涉渡,谓之交湖。兵戎之交,多自此济。晋太康中得鸣石于此水,撞之,声闻数里。沔水又东径乐山北,昔诸葛亮好为《梁甫吟》,每所登游,故俗以乐山为名。酒水又东径隆中,历孔明旧宅北。亮语刘禅云:先帝三顾臣于草庐之中,咨臣以当世之事。即此宅也。车骑沛国刘季和之镇襄阳也,与犍为人李安,共观此宅,命安作《宅铭》云:天子命我于沔之阳,听鼓鞞而永思,庶先哲之遗光。后六十余年,永平之五年,习凿齿又为其宅铭焉。又东过襄阳县北,沔水又东径万山北,山上有《邹恢碑》,鲁宗之所立也。山下潭中,有《杜元凯碑》,元凯好尚后名,作两碑,并述己功,一碑沉之岘山水中,一碑下之于此潭,曰百年之后,何知不深谷为陵也。山下水曲之隈,云汉女昔游处也。故张衡《南部赋》曰:游女弄珠于汉皋之曲。汉皋即万山之异名也。沔水又东合檀溪水。水出县西柳子山下,东为鸭湖,湖在马鞍山东北,武陵王爱其峰秀,改曰望楚山,溪水自湖两分,北渠即溪水所导也。北径汉阴台西,临流望远,按眺农圃,情邈灌蔬,意寄汉阴,故因名台矣。又北径檀溪,谓之檀溪水。水侧有沙门释道安寺,即溪之名,以表寺目也。溪之阳有徐元直、崔州平故宅,悉人居。故习凿齿《与谢安书》云:每省家舅,纵目檀溪,念崔、徐之交,未尝不抚膺踌躇,惆怅终日矣。溪水傍城北注,昔刘备为景升所谋,乘的颅马西走,坠于斯溪。西去城里余,北流注于沔。一水东南出。应劭曰:城在襄水之阳,故曰襄阳。是水当即襄水也。城北枕沔水,即襄阳县之故城也。王莽之相阳矣。楚之北津戍也,今大城西垒是也。其土,古鄢、鄀、卢、罗之地,秦灭楚,置南郡,号此为北部。建安十三年,魏武平荆州,分南郡,立为襄阳郡。荆州刺史治,邑居殷赈,冠盖相望,一都之会也。城南门道东有三碑,一碑是《晋太傅羊祜碑》,一碑是《镇南将军杜预碑》,一碑是《安南将军刘俨碑》,并是学生所立。城东门外二百步刘表墓,太康中为人所发,见表夫妻,其尸严然,颜色不异,犹如平生。墓中香气远闻,三四里中,经月不歇。今坟冢及祠堂犹高显整顿。城北枕沔水,水中常苦蛟害。襄阳太守邓遐,负其气果,拔剑入水,蛟绕其足,遐挥剑斩蛟,流血丹水,自后患除,无复蛟难矣。昔张公遇害,亦亡剑于是水。后雷氏为建安从事;径践濑溪,所留之剑,忽于其怀跃出落水,初犹是剑,后变为龙,做吴均《剑骑诗》云:剑是两蛟龙。张华之言,不孤为验矣。沔水又径平鲁城南。城,鲁宗之所筑也,故城得厥名矣。东对樊城。樊,仲山甫所封也。《汉晋春秋》称:桓帝幸樊城,百姓莫不观,有一老父,独耕不辍。议郎张温使问焉,父笑而不答,温因与之言,问其姓名,不告而去。城周四里,南半沦水,建安中关羽围于禁于此城,会沔水泛溢三丈有余,城陷,禁降。庞德奋剑乘舟,投命于东冈。魏武曰:吾知于禁三十余载,至临危授命,更不如庞德矣。城西南有曹仁记水碑,杜元凯重刊其后,书伐吴之事也。
又从县东屈西南,淯水从北来注之。
襄阳城东有东白沙,白沙北有三洲,东北有宛口,即淯水所入也。沔水中有鱼梁洲,庞德公所居。士元居汉之阴,在南白沙,世故谓是地为白沙曲矣。司马德操宅洲之阳,望衡对字,欢情自接,泛舟褰裳,率尔休畅,岂待还桂柁于千里,贡深心于永思哉。水南有层台,号曰景升台,盖刘表治襄阳之所筑也。言表盛游于此,常所止憩。表性好鹰,尝登此台,歌《野鹰来曲》,其声韵似孟达《上堵吟》矣。沔水又径桃林亭东,又径岘山东,山上有桓宣所筑城,孙坚死于此。又有《桓宣碑》。羊祜之镇襄阳也,与邹润甫尝登之,及祜薨,后人立碑于故处,望者悲感,杜元凯谓之堕泪碑。山上又有《征南将军胡罴碑》,又有《征西将军周访碑》,山下水中,杜元凯沉碑处。沔水又东南径蔡洲,汉长水校尉蔡瑁居之,故名蔡洲。洲东岸西有洄湖,停水数十亩,长数里,广减百步,水色常绿。杨仪居上洄,杨顒居下洄,与蔡洲相对,在岘山南广昌里。又与襄阳湖水合,水上承鸭湖,东南流径岘山西,又东南流注白马陂水。又东入侍中襄阳侯习郁鱼池。郁依范蠡养鱼法,作大陂,陂长六十步,广四十步,池中起钓台。池北亭,郁墓所在也。列植松篁于池侧,沔水上郁所居也。又作石洑逗,引大池水于宅北,作小鱼池,池长七十步,广二十步。西枕大道,东北二边,限以高堤,楸竹夹植,莲芡覆水,是游宴之名处也。山季伦之镇襄阳,每临此池,未尝不大醉而还,恒言此是我高阳池。故时人为之歌曰:山公出何去,往至高阳池,日暮倒载归,酩酊无所知。其水下入沔。沔水西又有孝子墓。河南秦氏,性至孝,事亲无倦,亲没之后,负土成坟,常泣血墓侧。人有咏《蓼莪》,氏为泣涕,悲不自胜。于墓所得病,不能食,虎常乳之,百余日卒。今林木幽茂,号曰孝子墓也。其南有蔡瑁冢,冢前刻石为大鹿,状甚大,头高九尺,制作甚工。沔水又东南径邑城北,习郁襄阳侯之封邑也,故曰邑城矣。沔水又东合洞口,水出安昌县故城东北大父山,西南流,谓之白水。又南径安昌故城东,屈径其县南,县,故蔡阳之白水乡也。汉元帝以长沙卑湿,分白水、上唐二乡为春陵县。光武即帝位,改为章陵县,置园庙焉。魏黄初二年,更从今名,故义阳郡治也。白水又西南流,而左会昆水。水导源城东南小山,西流径金山北,又西南流径县南,西流注于白水。水北有白水陂,其阳有汉光武故宅,基址存焉,所谓白水乡也。苏伯阿望气处也。光武之征秦丰,幸旧邑,置酒极懽。张平子以为真人南巡,观旧里焉。《东观汉记》曰:明帝幸南阳,祀旧宅,召校官子弟作雅乐,奏《鹿鸣》,上自御埙篪和之,以娱宾客,又于此宅矣。白水又西合浕水,水出于襄乡县东北阳中山,西径襄乡县之故城北。按《郡国志》,是南阳之属县也。浕水又西径蔡阳县故城东,西南流注于白水,又西径其城南。建武十三年,世祖封城阳王祉世子本为侯国。应劭曰:蔡水出蔡阳,东入淮。今于此城南,更无别水,惟是水可以当之。川流西注,苦其不东,且淮源阻碍,山河无相入之理,盖应氏之误耳。洞水又西南流注于沔水。又东过中庐县东,维水自房陵县维山东流注之。
县,即春秋庐戎之国也。县故城南,有水出西山。山有石穴出马,谓之马穴山。汉时,有数百匹马出其中,马形小,似巴滇马。三国时,陆逊攻襄阳于此穴,又得马数十匹,送建业。蜀使至,有家在滇池者,识其马毛色,云其父所乘,马对之流涕。其水东流百四十里,径城南,名曰浴马港。言初得此马,洗之于此,因以名之。亦云乘出沔次浴之,又曰洗马厩。渡沔宿处,名之曰骑亭。然候水诸蛮,北遏是水,南壅维川,以周田溉,下流入沔。沔水东南流,径犁丘故城西,其城下对缮州,秦丰居之,故更名秦洲。王莽之败也,秦丰阻兵于犁丘。犁丘城在观城西二里。建武三年,光武遣征南岑彭击丰,四年朱祐自观城擒丰子犁丘是也。沔水又南与疏水合,水出中庐县西南,东流至具北界,东入沔水,谓之疏口也。水中有物,如三四岁小儿,鳞甲如鲮鲤,射之不可入。七八月中好在碛上自曝,膝头似虎,掌爪常没水中,出膝头,小儿不知,欲取弄戏,便杀人。或曰人有生得者,摘其皋厌可小小使,名为水虎者也。
又南过县东北,沔水之左,有骑城,周回二里余,高一丈六尺,即骑亭也。县,故楚邑也。秦以为县,汉高帝十一年,封黄极忠为侯国。县南有黄家墓,墓前有双石阙,雕制甚工,俗谓之黄公阙。黄公名尚,为汉司徒。沔水又东径猪兰桥。桥本名木兰桥,桥之左右,丰蒿荻,于桥东刘季和大养猪,襄阳大守曰:此中作猪屎臭,可易名猪兰桥。百姓遂以为名矣。桥北有习郁宅,宅侧有鱼池,池不假功,自然通洫,长六七十步,广十丈,常出名鱼。沔水又南,得木里水会。楚时,于宜城东,穿渠上口,去城三里。汉南郡太守王宠又凿之,引蛮水灌田,谓之木里沟,径宜城东而东北入于沔,谓之木里水口也。
又南过宜城县东,夷水出自房陵,东流注之。
夷水,蛮水也。桓温父名夷,改曰蛮水。夷水导源中庐县界康狼山,山与荆山相邻。其水东南流,历宜城西山,谓之夷溪。又东南径罗卅城,故罗国也。又谓之鄢水,《春秋》所谓楚人伐罗渡鄢者也。夷水又东南流,与零水合,零水即沶水也。上通梁州没阳县之默城山,司马懿出沮之所由。其水东径新城郡之沶乡县,县,分房陵立,谓之沶水。又东历軨乡,谓之軨水。晋武帝平吴,割临沮之北乡,中庐之南乡,立上黄县,治軨乡。沶水又东历宜城西山,谓之沶溪。东流合于夷水,谓之沶口也。与夷水乱流东出,谓之淇水。径蛮城南,城在宜城南三十里,《春秋》莫敖自罗败退,及鄢,乱次以济淇水是也。夷水又东注于沔。昔白起攻楚,引西山长谷水,即是水也。旧堨去城百许里,水从城西,灌城东入,注为渊,今熨斗陂是也。水溃城东北角,百姓随水流死于城东者,数十万,城东皆臭,因名其陂为臭池。后人因其渠流,以结陂田。城西陂谓之新陂,覆地数十顷。西北又为土门陂,从平路渠以北,木兰桥以南,西极土门山,东跨大道,水流周通。其水自新陂东入城。城,故鄢郢之旧都,秦以为县,汉惠帝三年,改曰宜城。其水历大城中,径汉南阳太守秦颉墓北,墓前有二碑。颉,鄀人也,以江夏都尉出为南阳太守。径宜城中,见一家东向,颉住车视之,曰:此居处可作冢。后卒于南阳,丧还至昔住车处,车不肯进,故吏为市此宅,葬之,孤坟尚整。城南有宋玉宅。玉,邑人,隽才辩给,善属文而识音也。其水又径金城前,县南门有古碑,犹存。其水又东出城,东注臭池。臭池溉田,陂水散流,又人朱湖陂,朱湖陂亦下灌诸田,余水又下入木里沟。木里沟是汉南郡太守王宠所凿,故渠引鄢水也,灌田七百顷,白起渠溉三千顷,膏良肥美,更为沃壤也。县有太山,山下有庙,汉末名士居其中,刺史二千石卿长数十人,朱轩华盖,同会于庙下。荆州刺史行部见之,雅叹其盛,号为冠盖里,而刻石铭之。此碑于永嘉中始为人所毁,其余文尚有可传者。其辞曰:峨峨南岳,烈烈离明。实敷俊义,君子以生。惟此君子,作汉之英。德为龙光,声化鹤鸣。此山以建安三年崩,声闻五六十里,雉皆屋雊。县人恶之,以问侍中庞季,季云山崩川竭,国土将亡之占也。十三年,魏武平荆州,沔南雕散。沔水又径鄀县故城南,古鄀子之国也。秦、楚之间,自商、密迁此,为楚附庸,楚灭之以为邑。县南临沔律,津南有石山,上有古烽火台。县北有大城,楚昭王为吴所迫,自纪郢徙都之,即所谓鄢、鄀、卢、罗之地也,秦以为县。沔水又东,敖水往之。水出新市县东北,又西南径太阳山,西南流径新市县北,又西南而右合枝水,水出大洪山而西南流,径襄阳鄀县界西南,径狄城东南,左注敖水。敖水又西南流注于沔,寔曰敖口。沔水又南径石城西,城因山为固,晋太傅羊祜镇荆州立。晋惠帝元康九年,分江夏西部,置竟陵郡。治此。沔水又东南与臼水合,水出竟陵县东北聊屈山,一名卢屈山,西流注于沔。鲁定公四年,吴师入郢,昭王奔随,济于成臼。谓是水者也。
又东过荆城东,沔水自荆城东南流,径当阳县之章山东。山上有故城,太尉陶侃伐杜曾所筑也。《禹贡》所谓内方至于大别者也。既携带沔流,寔会《尚韦》之文矣。沔水又东,右会权口。水出章山,东南流径权城北,古之权国也。《春秋》鲁庄公十八年,楚武王克权,极叛,围而杀之、迁权于那处是也。东南有那口城。权水又东入于沔。沔水又东南与扬口合,水上承江陵县赤湖。江陵西北有纪南城,楚文王自丹阳徙此,平王城之。班固言:楚之郢都也。城西南有赤坂冈,冈下有渎水,东北流入城,名曰子胥渎,盖吴师入郢所开也,谓之西京湖。又东北出城西南,注于龙陂,陂,古天井水也,广圆二百余步,在灵溪东,江堤内。水至渊深,有龙见于其中,故曰龙陂。陂北有楚庄王钓台,高三丈四尺,南北六丈,东西九丈。陂水又径郢城南,冻北流,谓之扬水。又东北,路白湖水注之。湖在大港北,港南曰中湖,南堤下曰昏官湖,三湖合为一水,东通荒谷,荒谷东岸有冶父城,《春秋传》曰:莫敖缢于荒谷,群帅囚于冶父。谓此处也。春夏水盛,则南通大江,否则南迄江堤,北径方城西。方城即南蛮府也。又北与三湖会,故盛弘之曰:南蛮府东有三湖,源同一水,盖徙治西府也。宋元嘉中,通路白湖下注扬水,以广运漕。扬水又东历天井北,井在方城北里余,广圆二里,其深不测。井有潜室,见辄兵。西岸有天井台,因基旧堤,临际水湄,游憩之佳处也。扬水又东北流,东得赤湖水口,湖周五十里,城下陂池,皆来会同。湖东北有大暑台,高六丈余,纵广八尺,一名清暑台,秀宇层明,通望周博,游者登之,以畅远情。扬水又东入华容县,有灵溪水,西通赤湖水口,已下多湖,周五十里,城下陂他,皆来会同。又有子胥渎,盖人郢所开也。水东人离湖,湖在县东七十五里,《国语》所谓楚灵王阙为石郭陂,汉以象帝舜者也。湖侧有章华台,台高十丈,基广十五丈。左丘明曰:楚筑台于章华之上。韦昭以为,章华亦地名也。王与伍举登之。举曰:台高不过望国之氛祥,大不过容宴之俎豆。盖讥其奢而谏其失也。言此渎,灵王立台之曰漕运所由也。其水北流注于扬水,扬水又东北与柞溪水合,水出江陵县北,盖诸池散流、咸所会合,积以成川。东流径启宗之垒南,当驿路,水上有大桥,隆安三年,桓玄袭殷仲堪于江陵,仲堪北奔,缢于此桥。柞溪又东注船官湖,湖水又东北入女观湖,湖水又东入于扬水。扬水又北径竟陵县西,又北纳巾、吐柘,柘水即下扬水也。巾水出县东百九十里,西径巾城。城下置巾水戍,晋元熙二年,竟陵郡巾水戍得铜钟七口,言之上府。巾水又西径竟陵县北,西注扬水,谓之巾口。水西有古竟陵大城,古郧国也。郧公辛所治,所谓郧乡矣。昔白起拔郢,东至竟陵,即此也。秦以为县,王莽之寄平矣。世祖建武十三年,更封刘隆为侯国。城旁有甘鱼陂,《左传》昭公十三年,公子黑肱为令尹,次于鱼陂者也。扬水又北注于沔,谓之扬口,中夏口也。曹太祖之追刘备于当阳也,张飞按矛于长坂,备得与数骑斜趋汉津,遂济夏口是也。沔水又东得浐口,其水承大浐、马骨诸猢水,周三四百里,及其夏水来同,渺若沧海,洪潭巨浪,萦连江沔。故郭景纯《江赋》云:其旁则有朱浐、丹漅是也。
又东南过江夏云杜县东,夏水从西来注之。
即堵口也,为中夏水。县,故亭,《左传》若敖娶于是也。《禹贡》所谓云土梦作,故县取名焉。县有云梦城,城在东北。沔水又东径左桑,昔周昭王南征,船人胶舟以进之。昭王渡沔,中流而没,死于是水。齐、楚之会,齐侯曰:昭王南征而不复,寡人是问。屈完曰:君其问诸水滨。庾仲雍言:村老云:百姓佐昭王丧事于此,成礼而行,故曰佐丧。左桑,字失体耳。沔水又东合巨亮水口。水北承巨亮湖,南达于沔。沔水又东得合驿口,庾仲雍言:须导村耆旧云:朝廷驿使,合王丧于是,因以名焉。今须导村正有大敛口,言昭王于此殡敛矣。沔水又东,谓之横桑,言得昭王丧处也。沔水又东,谓之郑公潭,言郑武公与王同溺水于是。余谓世数既悬,为不近情矣。斯乃楚之郑乡,守邑大夫僭言公,故世以为郑公潭耳。沔水又东得死沔,言昭王济沔自是死,故有死沔之称,王尸岂逆流乎?但千古芒昧,难以昭知,推其事类,似是而非矣,沔水又东与力口合,有溾水,出竟陵郡新阳县西南池河山,东流径新阳县南,县治云杜故城,分云杜立。溾水又东南,流注宵城县南大湖,又南入于沔水,是曰力口。沔水又东南,溾水入焉。沔水又东径沌水口,水南通县之太白湖,湖水东南通江,又谓之沌口。沔水又东径沌阳县北,处沌水之阳也。沔水又东径临嶂故城北,晋建兴二年,太尉陶侃为荆州,镇此也。
又南至江夏沙羡县北,南入于江。
庾仲雍曰:夏口亦曰沔口矣。《尚书·禹贡》云:汉水南至大别入江。
《春秋左传》定公四年,吴师伐郢,楚子常济汉而陈,自小别至于大别。京相璠《春秋土地名》曰:大别,汉东山名也,在安丰县南。杜预《释地》曰:二别近汉之名,无缘乃在安丰也。按《地说》言:汉水东行,触大别之阪。南与江合,则与《尚书》、杜预相符,但今不知所在矣。


卷二十九  沔水、潜水、湍水、均水、粉水、白水、比水 
沔水与江合流,又东过彭蠡泽,《尚书·禹贡》汇泽也。郑玄曰:汇,回也。汉与江斗,转东成其泽矣。又东北出居巢县南,古巢国也。汤伐桀,桀奔南巢,即巢泽也。《尚书》周有巢伯来朝。《春秋》文公十二年,夏,楚人围巢。巢,群舒国也。舒叛,故围之。永平元年,汉明帝更封菑丘侯刘般为侯国也。江水自濡须口又东,左会栅口,水导巢湖,东径乌上城北,又东径南谯侨郡城南,又东绝塘径附农山北,又东,左会清溪水,水出东北马子砚之清溪也。东径清溪城南,屈而西南,历山西南流,注栅水,谓之清溪口。栅水又东,左会白石山水,水发白石山西,径李鹊城南,西南注栅水。栅水又东南,积而为窦湖,中有洲,湖东有韩综山,山上有城。山北湖水东出,为后塘北湖,湖南即塘也。塘上有颖川侨郡故城也。窦湖水东出,谓之窦湖口。东径刺史山北,历韩综山南,径流二山之间,出王武子城北,城在刺史山上。湖水又东径右塘穴北,为中塘,塘在四水中。水出格虎山北,山上有虎山城,有郭僧坎城,水北有赵祖悦城,并故东关城也。昔诸葛恪帅师作东兴堤,以遏巢湖,傍山筑城,使将军全端、留略等,各以千人守之。魏遣司马昭督镇东诸葛诞,率众攻东关三城,将毁堤遏,诸军作浮梁,陈堤上,分兵攻城。恪遣冠军丁奉等,登塘鼓噪奋击,朱异等以水军攻浮梁。魏征东胡遵军士争渡,梁坏,投水而死者数千。塘即东兴堤,城亦关城也。栅水又东南径高江产城南,胡景略城北,又东南径张祖禧城南,东南流,屈而北径郑卫尉城西。魏事已久,难用取悉,推旧访新,略究如此。又北委折蒲,浦出焉。栅水又东南流注于大江,谓之栅口。
又东过牛渚县南,又东至石城县,《经》所谓石城县者,即宣城郡之石城县也。牛渚在姑熟、乌江两县界中,于石城东北减五百许里,安得径牛渚而方届石城也?盖《经》之谬误也。分为二:其一东北流,其一又过毗陵县北,为北江。《地理志》,毗陵县,会稽之属县也。丹徒县北二百步有故城,本毗陵郡治也。旧去江三里,岸稍毁,遂至城下。城北有扬州刺史刘繇墓,沦于江。江即北江也,《经》书为北江则可,又言东至余姚则非,考其径流,知《经》之误矣。《地理志》曰:江水自石城东出,径吴国南,为南江。江水自石城东入为贵口,东径石城县北。晋太康元年立,隶宣城郡。东合大溪。溪水首受江北,径其县故城东,又北入南江。南江又东,与贵长池水合。水出县南郎山,北流为贵长池。池水又北注于南江。南江又东,径宣城之临城县南,又东合泾水,南江又东,与桐水合。又东径安吴县,号曰安吴溪。又东,旋溪水注之。水出陵阳山下,径陵阳县西,为旋溪水。昔县人阳子明钓得白龙处。后三年,龙迎子明上陵阳山,山去地千余丈。后百余年,呼山下人,令上山半,与语溪中。子安问子明钓车所在。后二十年,子安死,山下有黄鹤栖其冢树,鸣常呼子安,故县取名焉。晋咸康四年,改曰广阳县。溪水又北,合东溪水,水出南里山,北径其县东。桑钦曰:淮水出县之东南,北入大江。其水又北历蜀由山,又北,左合旋溪,北径安吴县东。晋太康元年分宛陵立。县南有落星山,山有悬水,五十余丈,下为深潭。潭水东北流,左入旋溪,而同注南江。南江之北,即宛陵县界也。南江又东径宁国县南。晋太康元年分宛陵置。南江又东径故鄣县南,安吉县北。光和之末,天下大乱,此乡保险守节,汉朝嘉之。中平二年,分故鄣之南乡以为安吉县。县南有钓头泉,悬涌一仞,乃流于川。川水下合南江,南江又东北为长渎历湖口。南江东注于具区,谓之五湖口。五湖谓长荡湖、太湖、射湖、贵湖、滆湖也。郭景纯《江赋》曰:注五湖以漫漭。盖言江水经纬五湖而苞注太湖也。是以左丘明述《国语》曰:越伐吴,战于五湖是也。又云范蠡灭吴返,至五湖而辞越,斯乃太湖之兼摄通称也。虞翻曰:是湖有五道,放曰五湖。韦昭曰:五湖,今太湖也,《尚书》谓之震泽;《尔雅》以为具区,方圆五百里,湖有苞山,《春秋》谓之夫椒山,有洞室,入地潜行,北通琅邪冢武县,俗谓之洞庭。旁有青山,一各夏架山,山有洞穴,潜通洞庭。山土有石鼓,长丈余,鸣则有兵。故《吴记》曰:太湖有苞山,在国西百余里,居者数百家,出弓弩材。旁有小山,山有石穴,南通洞庭,深远莫知所极。三苗之国,左洞庭,右彭蠡,今宫亭湖也。以太湖之洞庭对彭蠡则左右可知也。余接二湖俱以洞庭为目者,亦分为左右也,但以趣瞩为方耳。既据三苗,宜以湘江为正,是以郭景纯之《江赋》云:爱有包山洞庭,巴陵地道,潜达旁通,幽岫窈窕。《山海经》曰:浮玉之山,北望具区,苕水出于其阴,北流注于具区。谢康乐云:《山海经》浮玉之山,在句余东五百里,便是句余县之东山,乃应入海。句余,今在余姚鸟道山西北,何由北望具区也?以为郭于地理甚昧矣。言洞庭南口有罗浮山,高三千六百丈。浮山东石楼下,有两石鼓,叩之清越,所谓神钲者也。事备《罗浮山记》。会稽山宜直湖南,又有山阴溪水入焉。山阴西四十里,有二溪:东溪广一丈九尺,冬暖夏冷;西溪广三丈五尺,冬冷夏暖。二溪北出,行三里,至徐村,合成一溪,广五丈余,而温凉又杂,盖《山海经》所谓苕水也。北径罗浮山,而下注于太湖,故言出其阴,入于具区也。湖中有大雷、小雷三山,亦谓之三山湖,又谓之洞庭湖。杨泉《五湖赋》曰:头首无锡,足蹄松江,负乌程于背上,怀太吴以当胸,昨岭崔嵬,穹隆纡曲。大雷、小雷湍波相逐,用言湖之苞极也。太湖之东,吴国西十八里,有岞岭山。俗说此山本在太湖中,禹治水,移进近吴。又东及西南有两小山,皆有石如卷笮,俗云禹所用牵山也。太湖中有浅地,长老云是笮岭山蹠。自此以东差深,言是牵山之沟。此山去太湖三十余里,东则松江出焉,上承太湖,更径笠泽,在吴南松江左右也。《国语》曰:越伐吴,吴御之笠泽,越军江南,吴军江北者也。虞氏曰:松江北去吴国五十里,江侧有丞、胥二山,山各有庙。鲁哀公十三年,越使二大夫畴无余、讴阳等伐吴,吴人败之,获二大夫,大夫死,故立庙于山上,号曰丞、胥二王也。胥山上今有坛石,长老云,胥神所治也。下有九折路,南出太湖,阖闾造,以游姑胥之台以望太湖也。松江自湖东北流,径七十里,江水歧分,谓之三江口。《吴越春秋》称:范蠡去越,乘舟出三江之口,入五湖之中者也。此亦别为三江五湖,虽名称相乱,不与《职方》同。庾仲初《扬都赋注》曰:今太湖东注为松江,下七十里有水口。分流:东北入海为娄江,东南入海为东江,与松江而三也。《吴记》曰:一江东南行七十里,入小湖,为次溪,自湖东南出,谓之谷水。谷水出吴小湖,径由卷县故城下。《神异传》曰:由卷县,秦时长水县也。始皇时,县有童谣曰:城门当有血,城陷没为湖。有老妪闻之,忧惧,旦往窥城门,门侍欲缚之,妪言其故。妪去后,门侍杀犬,以血涂门。妪又在见血,走去不敢顾。忽有大水长,欲没县。主簿令干入白令,令见干,曰:何忽作鱼?于又曰:明府亦作鱼。遂乃沦陷为谷矣。因目长长城水曰谷水也。《吴记》曰:谷中有城,故由卷县治也,即吴之柴辟亭,故就李乡槜李之地,秦始皇恶其势王,令囚徒十余万人污其土,表以污恶名,改曰囚卷,亦曰由卷也。吴黄龙三年,有嘉禾生卷县,改曰禾兴。后太子讳和,改为嘉兴,《春秋》之槜李城也。谷水又东南径嘉兴县城西。谷水又东南径盐官县故城南,旧吴海昌都尉治。晋太康中分嘉兴立。《太康地道记》吴有盐官县。乐资《九州志》曰:县有秦延山,秦始皇径此,美人死,葬于山上,山下有美人庙。谷水之右有马皋城,故司盐都尉城,吴王濞煮海为盐,于此县也。是以《汉书·地理志》曰:县有盐官。东出五十里有武原乡,放越地也,秦于其地置海盐县。《地理志》曰:县故武原乡也。后县沦为柘湖,又徙治武原乡,改曰武原县。王莽名之展武。汉安帝时,武原之地又沦为湖,今之当湖也,后乃移此。县南有秦望山,秦始皇所登以望东海,故山得其名焉。谷水于县出为澉浦,以通巨海。光熙元年,有毛民三人,集于县,盖泛于风也。
又东至会稽余姚县,东入于海。谢灵运云:具区在余暨,然则余暨是余姚之别名也。今余暨之南,余姚西北,浙江与浦阳江同会归海。但水名已殊,非班固所谓南江也。郭景纯曰:三江者,岷江、松江、浙江也。然浙江出南蛮中,不与岷江同。作者述志,多言江水至山阴为浙江。今江南枝分历乌程县,南通余杭县,则与浙江合。故阚駰《十三州志》曰:江水至会稽与浙江合。浙江自临平湖,南通浦阳江,又于余暨东,合浦阳江,自秦望分派,东至余姚县又为江也。东与车箱水合,水出车箱山,乘高瀑布,四十余丈,虽有水旱,而澍无增减。江水又东径黄桥下。临江有汉蜀郡太守黄昌宅,桥本昌创建也,昌为州书佐,妻遇贼相失,后会于蜀,复修旧好。江水又东径赭山南。虞翻尝登此山四望,诫子孙可居江北,世有禄位,居江南则不昌也。然住江北者,相继代兴,时在江南者辄多沦替。仲翔之言为有征矣。江水又经官仓,仓即日南太守虞国旧宅,号曰西虞,以其兄光居县东故也。是地即其双雁送故处,江水又东径余姚县故城南,县城是吴将朱然所筑,南临江津,北背巨海,夫子所谓沧海浩浩,万里之渊也。县西去会稽百四十里,因句余山以名县。山在余姚之南,句章之北也。江水又东径穴湖塘,湖水沃其一县,并为良畴矣。江水又东注于海,是所谓三江者也。故子胥曰:吴越之国,三江环之,民无所移矣。但东南地卑,万流所凑,涛湖泛决,触地成川,枝津交渠,世家分伙,故川旧渎,难以取悉,虽粗依县地,缉综所缠,亦未必一得其实也。
潜水出巴郡宕渠县,潜水盖汉水棱分潜出,故受其称耳。今爰有大穴,潜水入焉。通冈山下,西南潜出,谓之伏水,或以为古之潜水。郑玄曰:汉别为潜,其穴本小,水积成泽,流与汉合,大禹自导汉疏通,即为西汉水也。故《书》曰:沱潜既道。刘澄之称白水入潜,然白水与羌水合入汉,是犹汉水也。县以延熙中分巴立宕渠郡,盖古责国也,今有责城。县有渝水,夹水上下;皆责民所居。汉祖入关,从定三奏,其人勇健,好歌舞,高祖爱习之,今《巴渝舞》是也。县西北有余曹水,南径其县,下注潜水。县有车骑将军冯绲、桂阳大守李温冢。二子之灵,常以三月还乡,汉水暴长,郡县吏民,莫不于水上祭之,今所谓冯李也。
又南入于江。
瘦仲雍云:垫江有别江,出晋寿县,即潜水也。其南源取道巴西,是西汉水也。湍水出郦县北芬山,南流过其县东,又甫过冠军县东,湍水出弘农界翼望山,水甚清彻,东南流径南阳郦县故城东,《史记》所谓下郦析也。仅武帝元朔元年,封左将黄同为侯国。湍水又南,菊水注之。水出西北石涧山芳菊溪,亦言出析谷,盖溪涧之异名也。源旁悉生菊草,潭涧滋液,极成甘美,云此谷之水土,餐挹长年。司空王畅、太傅袁隗、太尉胡广,并汲饮此水,以自绥养。是以君子留心,甘其臭尚矣。菊水东南流入于湍。湍水又径其县东南,历冠军县西北。有楚堨,高下相承八重,周十里,方塘蓄水,泽润不穷。湍水又径冠军县故城东,县本穰县之一阳乡、宛之临駣聚,汉武帝以霍去病功冠诸军,故立冠军县以封之。水西有《汉太尉长史邑人张敏碑》,碑之西有魏征南军司张詹墓。墓有碑,碑背刊云:白楸之棺,易朽之裳,钢铁不入,丹器不藏,嗟矣后人,幸勿我伤!自后古坟旧冢,莫不夷毁,而是墓至元嘉初,尚不见发。六年,大水,蛮饥,始被发掘。说者言:初开,金、银、铜、锡之器,朱漆雕刻之饰,烂然。有二朱漆棺,棺前垂竹帘,隐以金钉。墓不甚高,而内极宽大,虚设白揪之言,空负黄金之实。虽意铜南山,宁同寿乎?湍水又径穰县为六门肢。汉孝元之世,南阳太守邵信臣,似建昭五年断湍水,立穰西石塌。至元始五年,更开三门为六石门,故号六门竭也,溉穰、新野、昆阳三县五千余顷。汉未毁废,遂不修理。晋太康三年,镇南将军杜预复更开广,利加于民,今废不修矣。六门侧又有六门碑,是部曲主安阳亭侯邓达等,以太康五年立。湍水又径穰县故城北,又东南径魏武故城之西南,是建安三年,曹公攻张绣之所筑也。
又东过白牛邑南,湍水自白牛邑南,建武中,世祖封刘嵩为侯国。东南径安众县故城南,县本宛之西乡,汉长沙定王子康侯丹之邑也。湍水东南流,涅水注之。水出涅阳县西北歧棘山,东南径涅阳县故城西。汉武帝元朔四年,封路最为侯国。王莽之所谓前亭也。应劭曰:在涅水之阳矣。县南有二碑,碑字紊灭,不可复识,云是《左伯豪碑》。涅水又东南径安众县,竭而为陂,谓之安众港。魏太祖破张绣于是处,与荀或书曰:绣遏吾归师,迫我死地。盖于二水之间以为沿涉之艰阻也。涅水又东南流注于湍水。
又东南至新野县,湍水至县西北,东分为邓氏陂。汉太傅邓禹故宅,与奉朝请西华侯邓晨故宅隔陂,邓飏谓晨宅略存焉。
东入于淯。均水出析县北山,南流过其县之东,均水发源弘农郡之声氏县熊耳山,山南即修阳、葛阳二县界也。双峰齐秀,望若熊耳,因以为名。齐桓公召陵之会,西望熊耳,即此山也。太史公司马迁皆尝登之。县即析具之北乡,故言出析县北山也。均水又东南流径其县下,南越南乡县,又南流与丹水合。
又南当涉都邑北,南入于沔。
均水南径顺阳县西,汉哀帝更为博山县,明帝复曰顺阳。应劭曰:县在顺水之阳,今于是县则无闻于顺水矣。章帝建初四年,封卫尉马廖为侯国。晋太康中,立为顺阳郡县。西有石山,南临均水。均水又南流注于沔水,谓之均口者也。
故《地理志》谓之淯水,言熊耳之山,淯水出焉。又东南至顺阳,入于沔。粉水出房陵县,东流过郢邑南,粉水导源东流,径上粉县,取此水以渍粉,则皓耀鲜洁,有异众流,故县水皆取名焉。
又东过谷邑南,东入于沔。
粉水至筑阳县西而下注于沔水,谓之粉口。粉水旁有文将军冢,墓隧前有石虎、石柱,甚修丽。闾丘羡之为南阳,葬妇,墓侧,将平其域,夕忽梦文谏止,羡之不从。后羡之为杨佺期所害,论者以为文将军之祟也。
白水出朝阳县西,东流过其县南,王莽更名朝阳为厉信县。应勘曰:县在朝水之阳。今朝水径其北而不出其南也。盖邑郭沦移,川渠状改,故名旧传,遗称在今也。
又东至新野县南,东入于淯。
比水出比阳东北太胡山,东南流过其县南,泄水从南来注之。
太胡山在比阳北,如东,三十余里,广圆五六十里,张衡赋南都,所谓天封太狐者也。应劭曰:比水出比阳县,东入蔡。《经》云:泄水从南来注之。然比阳无泄水,盖误引寿春之沘泄耳。余以延昌四年,蒙除东荆州刺史,州治比阳县故城,城南有蔡水,出南磐石山,故亦曰磐石川,西北流注于比,非泄水也。《吕氏春秋》曰:齐令章子与韩、魏攻荆,荆使唐蔑应之,夹比而军,欲视水之深浅,荆人射之而莫知也。有刍者曰:兵盛则水浅矣。章子夜袭之,斩蔑于是水之上也。比水又西,澳水注之。水北出茈丘山,东流,屈而南转,又南入于比水。按《山海经》云:澳水又北入视,不注比水。余按吕忱《字林》及《难字》、《尔雅》,并言水在比阳,脉其川流所会,诊其水土津注,宜是水,音药也。比水又西南,历长冈旧月城北,比水右会马仁陂水,水出阴北山,泉流竞凑,水积成湖,盖地百顷,谓之马仁陂。陂水历其县下西南,堨之以溉田畴。公私引裂,水流遂断,故读尚存。比水又南径会口,与堵水枝津合。比水又南与渲水会。澧水源出于桐柏山,与淮同源,而别流西注,故亦谓水为派水。澧水西北流,径平氏县故城东北,王莽更名其县曰平善。城内有《南阳都乡正卫弹劝碑》。澧水又西北合浚水,水出湖阳北山。西流北屈,径平氏城西,而北入澧水。澧水又西注比水。比水自下,亦通谓之为派水。昔汉光武破甄阜、梁丘赐于比水西,斩之于斯水也。比水又甫,赵、醴二渠出焉:比水又西南流,谢水注之。水出谢城北,其源微小,至城渐大。城周回侧水,申伯之都邑,《诗》所谓申伯番番,既入于谢者也。世祖建武十三年,封樊重少子丹为谢阳侯,即其国也。然则是水即谢水也。高岸下深,浚流徐平,时人目之为淳水。城戍又以渟为目,非也。其城之西,旧棘阳县治,故亦谓之棘阳城也。谢水又东南径新都县,左注比水。比水又西南流,径新都县故城西,王莽更之曰新林。《郡国志》以为新野之东乡,故新都者也。
又西至新野县,南入于淯。
比水于冈南,西南流,戍在冈上,比水又西南,与南长、坂门二水合。
其水东北出湖阳东隆山。山之西侧有《汉日南大守胡著碑》。子珍,骑都尉,尚湖阳长公主,即光武之伯姊也。庙堂皆以青石为阶陛。庙北有石堂,珍之玄孙桂阳太守场,以延熹四年遭母忧,于墓次立石祠,勒铭于梁,石宇倾颓,而梁字无毁。盛弘之以为樊重之母畏雷室,盖传疑之谬也。隆山南有一小山,山权有两石虎相对夹隧道,虽处蛮荒,全无破毁,作制甚工,信为妙矣。世人因谓之为石虎山。其水西南流,径湖阳县故城南。《地理志》曰:故廖国也。《竹书纪年》曰:楚共王会宋平公于湖阳者矣。东城中有二碑,似是《樊重碑》,悉载故吏人名。司马彪曰:仲山甫封于樊,因氏国焉,爱自宅阳,徒居湖阳,能治田,殖至三百顷,广起庐舍,高楼连阁,波陂灌注,竹木成林,六畜放牧,鱼蠃梨果,檀棘桑麻,闭门成市。兵弩器械,赀至百万。其兴工造作,为无穷之功,巧不可言,富拟封君。世祖之少,数归外氏,及之长安受业,赍送甚至。世祖即位,追爵敬侯。诏湖阳为重立庙,置吏奉祠。巡词章陵,常幸重墓。其水四周城溉,城之东南,有若令樊萌、中常侍樊安碑。城南有数碑无字。又有石庙数间,依于墓侧,栋字崩毁,惟石壁而已,亦不知谁之胄族矣。其水南入大湖,湖阳之名县,藉兹而纳称也。湖水西南流,又与湖阳诸陂散水合,谓之板桥水。又西南与醴渠合,又有赵渠注之。二水上承派水。南径新都县故城东,两读双引,南合板桥水。板桥水又西南与南长水会。水上承唐子襄乡诸陂散流也。唐子陂在唐子山西南,有唐子亭。汉光武自新野屠唐子乡,杀湖阳尉于是地。陂水清深,光武后以为神渊。西南流于新野县,与板桥水合,西南注于比水。比水又西南流,注于淯水也。


卷三十  淮水 
淮水出南阳平氏县胎替山,东北过桐柏山,《山海经》曰:淮出余山,在朝阳东,义乡西。《尚书》:导淮自桐柏。《地理志》曰,南阳平氏县,王莽之平善也。《风俗通》曰:南阳平氏县桐柏大复山在东南,淮水所出也。淮,均也。《春秋说题辞》曰:淮者,均其势也。《释名》曰:淮,韦也。韦绕扬州北界,东至于海也。《尔雅》曰:淮为浒。然淮水与醴水同源俱导,西流为醴,东流为淮。潜流地下,三十许里,东出桐柏之大复山南,谓之阳口。水南即复阳县也。阚駰言复阳县,胡阳之乐乡也。元帝元延二年置,在桐柏大复山之阳,故曰复阳也。《东观汉记》曰:朱祐少孤,归外家复阳刘氏。山南有淮源庙,庙前有碑,是南阳郭苞立。又二碑,并是汉延熹中守、令所造,文辞鄙拙,殆不可观。故《经》云东北过桐柏也。淮水又东径义阳县,县南对固成山。山有水,注流数丈,洪涛灌山,遂成巨井,谓之石泉水,北流注于淮。淮水又径义阳县故城南,义阳郡治也,世谓之白茅城,其城圆而不方。阚駰言晋太始中,割南阳东鄙之安昌、平林、平氏、义阳四县,置义阳郡于安昌城。又《太康记》、《晋书地道记》,并有义阳郡,以南阳属县为名。汉武帝无狩四年,封北地都尉卫山为侯国也。有九渡水注之,水出鸡翅山,溪涧潆委,沿溯九渡矣。其犹零阳之为九渡水,故亦谓之为九渡焉。于溪之东山有一水,发自山椒下数丈,素湍直注,颓波委壑,可数百丈,望之若霏幅练矣,下注九渡水,九渡水又北流注于淮。
东过江夏平春县北,淮水又东,油水注之。水出县西南油溪,东北流径平春县故城南。汉章帝建初四年,封子全力王国。油水又东曲,岸北有一土穴,径尺,泉流下注,沿波三丈,入于油水,乱流南屈,又东北注于淮。淮水又东北径城阳县故城南,汉高帝十二年,封定侯奚意为侯国。王莽之新利也。魏城阳郡治。淮水又东北与大木水合,水西出大木山,山即晋车骑将军祖逖自陈留将家避难所居也。其水东径城阳县北,而东入于淮。淮水又东北流,左会湖水,傍川西南出,穷溪得其源也,淮水又东径安阳县故城南,江国也,赢姓矣。今其地有江亭。《春秋》文公四年,楚人灭江,秦伯降服出次,曰:同盟灭,虽不能救,敢不矜乎?汉乃县之。文帝八年,封淮南厉王子刘勃为侯国。王莽之均夏也。淮水又东,得溮口水源,南出大溃山,东北流,翼带三川,乱流北注溮水。又北径贤首山西,又北出,东南屈,径仁顺城南,故义阳郡治,分南阳置也。晋太始初,以封安平献王罕长子望,本治在石城山上,因梁希侵逼,徙洽此城。梁司州刺史马仙琕不守,魏置郢州也。昔常珍奇自悬瓠遣三千骑援义阳行事庞定光,屯于溮水者也。溮水东南流,历金山北,山无树木,峻峭层峙。溮水又东径义阳故城北,城在山上,因倚陵岭,周回三里,是郡昔所旧治城。城南十五步,对门有天井,周百余步,深一丈。东径钟武县故城南,本江夏之属县也。主莽之当利县矣。又东径石城山北,山甚高峻。《史记》曰:魏攻冥阨。《音义》曰:冥阨或育在县葙山也。案《吕氏春秋》九塞,其一也。溮水径县故城南,建武中,世祖封邓邯为侯。案苏林曰:音盲。狮水又东径七井冈南,又东北注于淮。淮水又东至谷口。谷水甫出鲜金山,北流,瑟水注之。水出西南具山,东北径光淹城东,而北径青山东,罗山西,俗谓之仙居水,东北流注于谷水。谷水东北人于淮。
又东过新息县南,淮水东径故息城南。《春秋左传》隐公十一年,郑、息有违言,息侯伐郑,郑伯败之者也。淮水又东径浮光山北,亦曰扶光山,即弋阳山也。出名玉及黑石,堪为棋。其山俯映长淮,每有光辉。淮水又东,径新息县故城南。应劭曰:息后徙东,故加新也。王莽之新德也。光武十九年,封马援为侯国。外城北门内有新息长贾彪庙,庙前有碑。面南又有《魏妆南太守程晓碑》。魏太和中,蛮田益宗效诚,立东豫州,以益宗为刺史。淮水又东合慎水。水出慎阳县西,而东径慎阳县故城南,县取名焉。汉高帝十一年,封栾说为侯国。颍阴刘陶为县长,政化大行,道不拾遗,以病去官。童谣歌曰:悒然不乐,思我刘君,何时复来,安此下民?见思如此。应劭曰:慎水所出,东北入淮。慎水又东流,积为慎陂。陂水又东南流为上慎陂,又东为中慎陂,又东南为下慎陂,皆与鸿部陂水散流。其破首受淮川,左结鸿陂。汉成帝时,翟方进奏毁之。建武中,汝南太守邓晨欲修复之,知许伟君晓知水脉,召与议之。伟君言:成帝用方进言毁之,寻而梦上天,天帝怒曰:何敢败我濯龙渊?是后民失其利,时有童谣曰:败我陂,翟子威,反乎覆,陂当复。明府兴复废业,童谣之言,将有征矣。遂署都水掾,起塘四百余里,百姓得其利。肢水散流,下合慎水,而东南径息城北,又东南入淮。谓之慎口。淮水又东,与申陂水合。水上承申陂于新息县北,东南流,分为二水。一水径深丘西,又屈径其南,南派为莲湖,水南流注于淮。淮水又左迤流结两湖,谓之东、西莲湖矣。淮水又东,右合壑水。水出白沙山,东北径柴亭西,俗谓之柴水。又东北流,与潭溪水合。水发潭谷,东北流,右会柴水。柴水又东径黄城西,故弋阳县也。城内有二城,西即黄城也。柴水又东北入于淮。谓之柴口也。淮水又东北,申陂枝水注之。水首受陂水于深丘北,东径钓台南。台在水曲之中,台北有琴台。又东径阳亭南,东南合淮。淮水又东径淮阴亭北,又东径白城南,楚白公胜之邑也。东北去白亭十里。淮水又东径长陵戍南,又东,青陂水注之。分青陂东渎,东南径白亭西,又南于长陵戍东,东南入于淮,淮水又东北合黄水,水出黄武山,东北流,本陵关水注之。水导源木陵山,西北流注于黄水。黄水又东径晋西阳城南,又东径光城南,光城左郡治。又东北径高城南,故弦国也。又东北径弋阳郡,东有虞丘郭,南有子肯庙。黄水又东北入于淮,谓之黄口。淮水又东北径褒信县故城南,而东流注也。
又东过期思县北,县故蒋国周公之后也。《春秋》文公十年,楚王田于孟诸,期思公复遂为右司马,楚灭之以为县。汉高帝十二年,以封贲赫为侯国。城之西北隅,有楚相孙叔敖庙,庙前有碑。淮水又东北,淠水注之。水出戈阳县南垂山,西北流历阴山关,径二城间。旧有贼难,军所顿防。西北出山,又东北流,径新城戍东。又东北得诏虞水口,西北去弋阳虞丘郭二十五里。水出南山,东北流。径诏虞亭东,而北入淠水。又东北注淮,俗曰白鹭水。
又东过原鹿县南,汝水从西北来注之。
县即《春秋》之鹿上也。《左传》情公二十一年,宋人为鹿上之盟,以求诸侯于楚。建武十五年,世祖更封侍中、执金吾、阴乡侯阴识为侯国者也。又东过庐江安丰县东北,决水从北来注之。
庐江,故淮南也。汉文帝十六年,别以为国。应劭曰:故庐子国也。决水自舒寥北注,不于北来也。安丰东北注淮者,穷水矣,又非决水,皆误耳。淮水又东,谷水入焉。水上承富水,东南流,世谓之谷水也。东径原鹿县故城北,城侧水南。谷水又东径富肢县故城北,俗谓之成闾亭,非也。《地理志》:汝南郡有富肢县。建武二年,世祖改封平乡侯,王霸为富陂侯。《十三州志》曰:汉和帝永元九年,分汝阴置,多陂塘以溉稻,故曰富陂县也。谷水又东,于汝阴城东南注淮。淮水又东北,左会润水,水首受富陂,东南流为高塘陂,又东,积而为陂水,东注焦陵陂。陂水北出为慎陂,肢水潭涨,引读北注汝阴,四周隍堑,下注颍水。焦湖东注,谓之润水。径汝阴县东,径荆亭北,而东入淮。淮水又东北,穷水入焉。水出六安国安风县穷谷。《春秋左传》楚救灊,司马沈尹戍与吴师遇于穷者也。川流泄注于决水之右,北灌安风之左,世谓之安风水,亦曰穷水。音戎,并声相近,字随读转。流结为陂,谓之穷陂。塘堰虽沦,犹用不辍,陂水四分,农事用康。北流注于淮。京相璠曰:今安风有穷水,北入淮。淮水又东为安风津。水南有城,故安风都尉治。后立霍丘戍,淮中有洲,俗号关洲,盖津关所在,故斯洲纳称焉。《魏书》、《国志》有曰:司马景王征田丘俭,使镇东将军、豫州刺史诸葛诞从安风津先至寿春。俭败,与小弟秀藏水草中。安风津都尉部民张属斩之,传首京都。即斯津也。
又东北至九江寿春县西,沘水、泄水合北注之,又东,颍水从西北来流注之。
淮水又东,左合批口。又东径中阳亭北,为中阳渡,水流浅碛,可以厉也。淮水又东流与颍口会。东南径苍陵城北,又东北流径寿春县故城西,县即楚考烈王自陈徙此。秦始皇立九江郡,治此,兼得庐江、豫章之地,故以九江名郡。汉高帝四年为淮南国,孝武元狩六年复为九江焉。文颖曰:《史记·货殖传》曰:淮以北,沛、陈、汝南、南郡为西楚,彭城以东,东海、吴、广陵为东楚。衡山、九江、江南、豫章、长沙为南楚。是为三楚者也。淮水又北,左合椒水。水上承淮水,东北流径她城南,又历其城东,亦谓之清水,东北流注于淮水,谓之清水口者,是此水焉。
又东过寿春县北,肥水从县东北流注之。
淮水于寿阳县西北,肥水从城西而北入于淮,谓之肥口。淮水又北,夏肥水注之。水上承沙水于城父县、右出东南流径城父县故城南,王莽之思善也。县故焦夷之地,《春秋左传》昭公九年,楚公子弃疾迁许于夷,寔城父矣。取州来淮北之田以益之,伍举授许男田。杜预曰:此时改城父为夷,故《传》寔之者也。然丹迁城父人于陈,以夷濮西田益之,言夷田在濮水西者也。然则濮水即沙水之兼称,得夏肥之通目矣。汉桓帝永寿元年,封大将军梁冀孙桃为侯国也。夏肥水自县,又东径思善县之故城南,汉章帝章和三年分城父立。夏肥水又东为高陂,又东为大慎陂。水出分为二流,南为夏肥水,北为鸡陂。夏肥水东流,左合鸡水,水出鸡陂,东流为黄陂,又东南流,积为茅陂,又东为鸡水。《吕氏春秋》曰:宋人有取道者,其马不进,投之鸡水是也。鸡水右会夏肥水而乱流东注,俱入于淮。淮水又北径山硖中,谓之硖石。对岸山上,结二城,以防津要。西岸山上有马迹,世传淮南王乘马升仙所在也。今山之东南石上,有大小马迹十余所,仍今存焉。淮水又北径下蔡县故城东,本州来之城也。吴季札始封延陵,后邑州来,故曰延州来矣。《春秋》哀公二年,蔡昭侯自新蔡迁于州来,谓之下蔡也。淮之东岸,又有一城,即下蔡新城也。二城对据。翼带淮濆。淮水东径八公山北,山上有老子庙。淮水历潘城南。置潘溪戍。戍东侧潘溪,吐川纳淮,更相引注。又东径梁城,临侧淮川,川左有湄城。淮水左迤为湄湖。淮水又右纳洛川于西曲阳县北,水分阎溪,北绝横塘。又北径萧亭东,又北,鹊甫溪水人焉。水山东鹊甫谷,西北流径鹊甫亭南,西北流注于洛水。北径西曲阳县故城东,王莽之延平亭也。应劭曰:具在淮曲之阳,下邳有曲阳,故是加西也。洛涧北历秦墟,下注淮,谓之洛口。《经》所谓淮水径寿春县北,肥水从县东北注者也,盖《经》之谬矣。考川定土,即实为非,是曰洛涧,非肥水也。淮水又北径莫邪山西,山南有阴陵县故城。汉高祖五年,项羽自该下,从数百骑,夜驰渡淮,至阴陵,迷失道左,陷大泽,汉令骑将灌婴以五千骑追及之于斯县者也。案《地理志》,王莽之阴陆也。后汉九江郡治。时多虎灾,百姓苦之,南阳宗均为守,退贪残,进忠良,虎悉东渡江。
又东过当涂县北, 水从西北来注之。
淮水自莫邪山,东北径马头城北,魏马头郡治也,故当涂县之故城也。
《吕氏春秋》曰:禹娶涂山氏女,不以私害公,自辛至甲四日,复往治水。故江淮之俗,以辛壬癸甲为嫁娶日也。禹墟在山西南,县即其地也。《地理志》曰:当涂,侯国也。魏不害以图守尉,捕淮阳反者公孙勇等,汉以封之。王莽更名山聚也。淮水又东北,濠水注之,水出莫邪山东北溪。溪水西北引渎,径禹墟北,又西流注于淮。淮水又北,沙水注之,《经》所谓蒗渠也。淮之西有平阿县故城,王莽之平宁也。建武十三年,世祖更封耿阜为侯国。《郡国志》曰:平阿县有涂山。淮出于荆山之左,当涂之右,奔流二山之间而扬涛北注也。《春秋左传》哀公十年,大夫对孟孙曰:禹会诸侯于涂山,执王帛者万国。杜预曰:涂山在寿春东北。非也。余按《国语》曰:吴伐楚,堕会稽,获骨焉,节专车。吴子使来聘,且问之。客执骨而问曰:敢问骨何为大?仲尼曰:丘闻之:昔禹致群神于会稽之山,防风氏后至,禹杀之,其骨专车,此为大也。盖丘明亲承圣旨,录为实证矣。又案刘向《说苑·辨物》,王肃之叙孔子什二世孙孔猛所出先人书《家语》,并出此事,故涂山有会稽之名。考校群韦及方土之目,疑非此矣,盖周穆之所会矣,淮水于荆山北,水东南注之,又东北径沛郡义城县东。司马彪曰:后隶九江也。
又东过钟离县北,《世本》曰:钟离,赢姓也。应劭曰:县故钟离子国也,楚灭之以为县。《春秋左传》所谓吴公子光伐楚,拔钟离者也。王莽之蚕富也。豪水出阴陵县之阳亭北,小屈,有石穴,不测所穷。言穴出钟乳,所未详也。豪水东北流,径其县西,又屈而南,转东,径其城南,又北历其城东,径小城而北流,注于淮。淮水又东,径夏丘县南,又东,涣水入焉。水首受蒗渠于开封县。《史记》韩釐王二十一年,使暴救魏,为秦所败, 走开封者也。东南流径陈留北,又东南,西入九里注之。涣水又东南流,径雍丘县故城南,又东径承匡城,又东径襄邑县故城南,故宋之承匡襄牛之地,宋襄公之所葬,故号襄陵矣。《竹书纪年》:梁惠成王十七年,宋景鄯卫公孙仓会齐师,围我襄陵。十八年,惠成王以韩师败诸侯师于襄陵。齐侯使楚景舍来求成,即于此也。西有承匡城,《春秋》会于承匡者也。秦始皇以承匡卑湿,徙县于襄陵,更为襄邑。王莽以为襄平也。汉桓帝建和元年,封梁冀子胡狗为侯国。《陈留风俗传》曰:县南有涣水,故《传》曰:睢、涣之间出文章,天子郊庙御服出焉,《尚书》所谓厥篚织文者也。涣水又东南径已吾县故城南,又东径鄫城北。《春秋》襄公元年,《经》书,晋韩厥帅师伐郑,鲁仲孙蔑会齐、曹、邾、杞,次于鄫。杜预曰:陈留囊邑县东南有鄫城。涣水又东南径邵城北,新城南,又东南,左合明沟,沟水自蓬洪陂东南流,谓之明沟,下入涣水。又径毫城北。《帝王世记》曰:谷熟为南毫,即汤都也。《十三州志》曰:汉武帝分谷熟置。《春秋》庄公十二年,宋公子御说奔毫者也。涣水东径谷熟城南。汉光武建武二年,封更始子歆为侯国。又东径杨亭北。《春秋左氏传》襄公十二年,楚子囊、秦庶长无地,伐宋师于杨梁,以报晋之取郑也。京相璠曰:宋地矣。今睢阳东南三十里,有故杨梁城,今曰阳亭也,俗名之曰缘城,非矣。西北去梁国八十里。涣水又东径沛郡之建平县故城南,汉武帝元凤元年,封杜延年为侯国,王莽之田平也。又东径酂县故城南。《春秋》襄公十年,公会诸侯及齐世子光于。今其地聚是也,玉莽之酇治矣。涣水又东南径费亭南。汉建和元年,封中常侍沛国曹腾为侯国。腾,字季兴,谯人也。永初中,定桓帝策,封亭侯,此城即其所食之邑也。涣水又东径銍县故城南。昔吴广之起兵也,使葛婴下之。涣水又东,苞水注之。水出谯城北白汀陂。陂水东流径酇县南,又东,径郸县故城南。汉景帝中元年,封周应为侯国。王莽更之曰单城也。音多。又东径嵇山北,秘氏故居。嵇康本姓奚,会稽人也。先人自会稽迁于谯之酇县,改为嵇氏。取嵇字之上以为姓,盖志本也。《嵇氏谱》曰:谯有嵇山,家于其侧,遂以为氏。县,魏黄初中,文帝以酇城父、山桑、銍置谯郡,故隶谯焉。苞水东流入涣,涣水又东南径蕲县故城南。《地理志》曰:故甀乡也。汉高帝破黥布于此县,旧都尉治,王莽之蕲城也。水上有古石梁处,遗基尚存。涣水又东径谷阳县,左会八丈故渎。渎上承洨水,南流注于涣。涣水又东径谷阳戍南,又东南径谷阳故城东北,右与解水会。水上承县西南解塘,东北流径谷阳城南,即谷水也。应肋曰:城在谷水之阳,又东北流注于涣。涣水又东南径白石戍南,又径虹城南,洨水注之。水首受蕲水于蕲县,东南流径谷阳县,八丈故渎出焉。又东合长直故沟。沟上承蕲水,南会于洨。位水又东南流,径洨县故城北。县有垓下聚,汉高祖破项羽所在也。王莽更名其县曰肴城。应劭曰:洨水所出,音绞,《经》之绞也。洨水又东南,与涣水乱流而入于淮,故应劭曰:洨水南入淮。淮水又东至巉石山,潼水注之。水首受潼县西南潼陂。县故临淮郡之属县,王莽改曰成信矣。南径沛国夏丘县,绝蕲水,又南径夏丘县故城西,王莽改曰归思也。又东南流,径临潼戍西,又东南至巉石,西南入淮。淮水又东径浮山。山北对巉石山,梁氏天监中,立堰于二山之间,逆天地之心,乖民神之望,自然水溃坏矣。淮水又东径徐县南,历涧水注之。导徐城西北徐陂,陂水南流,绝蕲水。径历涧戍西,东南流注于淮。淮水又东,池水注之。水出东城县,东北流,径东城县故城南。汉以数干骑追羽,羽帅二十八骑引东城因四聩山,斩将而去,即此处也。《史记》:孝文帝八年,封淮南厉王子刘良为侯国。《地理志》:王莽更名之曰武城也。池水又东北,流历二山间,东北入于淮,谓之池河口也。淮水又东,薪水注之。水首受睢水于谷熟城东北,东径建城县故城北。汉武帝元朔四年,封长沙定王子刘拾为侯国。王莽之多聚也。蕲水又东南,径蕲县。县有大泽乡,陈涉起兵于此,篝火为狐鸣处也。南则江水出焉。薪水又东南,北八丈故读出焉。又东流,长直故沟出焉。又东入夏丘县,东绝潼水,径夏丘县故城北,又东南径潼县南,又东流入徐县,东绝历涧,又东径大徐县故城南,又东注于淮。淮水又东历客山,径盱胎县故城南。《地理志》曰:都尉治。汉武帝元朔元年,封江都易王子刘蒙之为侯国。王莽更名之曰匡武。淮水又东径广陵淮阳城南,城北临泗水,阻于二水之间。《述征记》淮阳太守治,自后置戍,县亦有时废兴也。又东北至下邳淮阴县西,泗水从西北来流往之。淮、泗之会,即角城也。左右两川,翼夹二水,决入之所,所谓泗口也。
又东过淮阴县北,中渎水出白马湖,东北注之。淮水右岸,即淮阴也。
城西二里有公路浦。昔袁术向九江,将东奔袁谭。路出斯浦,因以为名焉。又东径淮阴县故城北。北临淮水,汉高帝六年,封韩信为侯国。王莽之嘉信也。昔韩信去下乡面钓于此处也。城东有两冢,西者,即漂母冢也。周回数百步,高十余丈。昔漂母食信于淮阴,信王下邳,盖投金增陵以报母矣。东一陵即信母冢也。县有中渎水,首受江于广陵郡之江都县。县城临江,应劭《地理风俗记》曰:县为一都之会,故曰江都也。县有江水祠,俗谓之伍相庙也。子胥但配食耳。岁三祭,与五岳同。旧江水道也。昔吴将伐齐,北霸中国,自广陵城东南筑邗城,城下掘深沟,谓之韩江,亦曰邗溟沟,自江东北通射阳湖。《地理志》所谓渠水也,西北至未口入淮。自永和中,江都水断:其水上承欧阳埭,引江入埭,六十里至广陵城。楚、汉之间为东阳郡,高祖六年力荆国,十一年为吴城,即吴王濞所筑也。景帝四年,更名江都。武帝元狩三年,更曰广陵。王莽更名郡曰江平,县曰定安。城东水上有梁,谓之洛桥。中读水自广陵北出武广湖东、陆阳湖西。二湖东西相直五里;水出其间,不注樊梁湖。旧道东北出,至博芝、射阳二湖。西北出夹邪,乃至山阳矣。至永和中,患湖道多风,陈敏因穿樊梁湖北口,下注津湖径渡,渡十二哩,方达北口,直至夹邪。兴宁中,复以津湖多风,又自湖之南口,沿东岸二十里,穿渠入北口,自后行者不复由湖。故蒋济《三州论》曰,淮湖纡远,水陆异路,山阳不通,陈敏穿沟,更凿马濑,百里渡湖者也。自广陵出山阳白马湖,径山阳城西,即射阳县之故城也。应劭曰:在射水之阳。汉高祖六年,封楚左令尹项缠为侯国也。王莽更之曰监淮亭。世租建武十五年,封子荆为山阳公,治此,十七年为王国。城本北中郎将庾希所镇。中渎水又东,谓之山阳浦,又东入淮,谓之山阳口者也。
又东,两小水流注之。
淮水左径泗水国南,故东海郡也。徐广《史记音义》曰:泗水,国名。
汉武帝元鼎四年初置,都淩。封常山宪王子思王商为国。《地理志》曰:王莽更泗水郡为水顺,淩县为生淩,淩水注之,水出淩县,东流径其县故城东,而东南流注于淮,实曰淩口也。应劭曰淩水出县西南入淮,即《经》之所谓小水者也。
又东至广陵淮浦县,人于海。
应劭曰:淮崖也,盖临侧淮渎,故受此名。淮水径县故城东,王莽更名之曰淮敬。淮水于县枝分,北为游水,历朐县与沐合。又径朐山西,山侧有朐县故城。秦始皇三十五年,于胸县立石海上,以为秦之东门。崔琰《述初赋》曰:倚高舻以周眄兮,观秦门之将将者也。东北海中有大洲,谓之郁洲,《山海经》所谓郁山在海中者也。言是山自苍梧徙此,云山上犹有南方草木。今郁州治。故崔珪之叙《述初赋》,言郁州者,故苍梧之山也。心悦而怪之,闻其上有仙士石室也,乃往观焉。见一道人独处,休休然不谈不对,顾非已及也。邱其《赋》所云吾夕济于郁洲者也。游水又北径东海利成县故城东,故利乡也,汉武帝元朔四年,封城阳共王子婴为侯国。王莽更之曰流泉。游水又北,历羽山西,《地理志》曰:羽山在祝其县东南。《尚书》曰:尧畴咨四岳得舜,进十六族,殛鲧于羽山,是为梼机,与驩兜、三苗、共工同其罪,故世谓之四凶。鲧既死,其神化为黄熊,入于羽渊,是为夏郊,三代祀之。故《连山易》曰:有崇伯鲧,伏于羽山之野者是也。游水又北径祝其县故城西。《春秋经》书:夏,公会齐侯于夹多。《左传》定公十年,公及齐平,会于祝其,实夹谷也。服虔曰:地二名。王莽更之曰犹亭。县之东有夹口浦,游水左径琅邪计斤县故城之西。《地理志》曰:莒子始起于此,后徙莒,有盐官,故世谓之南莒也。游水又东北径赣榆县北,东侧巨海,有秦始皇碑在山上,去海百五十步,潮水至,加其上三丈,去则三尺,所见东北倾石,长一丈八尺,广五尺,厚三尺八寸,一行十二字。游水又东北径纪鄣故城南。《春秋》昭公十九年,齐伐莒,莒子奔纪鄣。莒之妇人,怒莒子之害其夫,老而托纺焉。取其绅而夜缒,缒绝,鼓噪,城上人亦噪。莒共公惧,启西门而出,齐遂入纪。故纪子帛之国。《谷梁传》曰:吾伯姬归于纪者也。杜预曰:纪鄣,地二名。东海赣榆县东北,有故纪城,即此城也。游水东北入海,旧吴之燕岱,常泛巨海,惮其涛险,更沿溯是渎,由是出。《地理志》曰:游水自淮浦北入海。《尔雅》曰:淮别为符。游水亦枝称者也。


卷三十一  滍水、淯水、隐水、灈水、亲水、无水、溳水 
滍水出南阳鲁阳县西之尧山,尧之未孙刘累,以龙食帝孔甲,孔甲又求之,不得。累惧而迁于鲁县,立尧祠于西山,谓之尧山。故张衡《南部赋》曰:奉先帝而追孝,立唐祠于尧山。尧山在太和川大和城东北,溃水出焉。张衡《南都赋》曰:其川读则滍、澧、、浕,发源岩穴,布濩漫汗,漭沆洋溢,总括急趣,箭驰风疾者也。滍水又历太和川,东,径小和川,又东,温泉水注之。水出北山阜,七源奇发,炎热特甚。阚駰曰:县有汤水,可以疗疾。汤侧又有寒泉焉,地势不殊,而炎凉异致,虽隆火盛日,肃若冰谷矣,浑流同溪,南注滍水。滍水又东径胡木山,东流又会温泉口,水出北山阜,炎势奇毒。腐疾之徒,无能澡其冲漂。救痒者咸去汤十许步别池,然后可入。汤侧有石铭云:皇女汤,可以疗万疾者也。故杜彦达云:然如沸汤,可以熟米,饮之,愈百病。道士清身沐浴,一日三饮,多少自在。四十日后,身中万病愈,三虫死。学道遭难逢危,终无悔心,可以牢神存志。即《南都赋》所谓汤谷涌其后者也。然宛县有紫山,山东有一水,东西十五里,南北二百步,湛然冲满,无所通会,冬夏常温,世亦谓之汤谷也。非鲁阳及南阳之县故也。张平子广言土地所苞,明非此矣。滍水又东,房阳川水注之。水出南阳雉县西房阳川,北流注于滍。滍水之北有积石焉,世谓女灵山。其山平地介立,不连冈以成高,峻石孤峙,不托势以自远,四面壁绝,极能灵举,远望亭亭,状若单楹插霄矣。北面有如颓落,劣得通步,好事者时有扳涉耳。滍水又与波水合,水出霍阳西川,大岭东谷,俗谓之歇马岭,川曰广阳川,非也。即应劭所谓孤山,波水所出也。马融《广成颂》曰:浸以波、溠。其水又南径蛮城下,盖蛮别邑也,俗谓之麻城,非也。波水又南,分三川于白亭东,而俱南入滍水。溃水自下兼波水之通称也。是故阚滍有东北至定陵入汝之文。滍水又东径鲁阳县故城南,城即刘累之故邑也,有鲁山,县居其阳,故因名焉。王莽之鲁山也。昔在于楚,文子守之,与韩遘战,有返景之诚。内有《南阳都乡正卫为碑》。滍水右合鲁阳关水,水出鲁阳关外分头山横岭下夹谷,东北出入滍。滍水又东北合牛兰水。水发县北牛兰山,东南径鲁阳城东,水侧有《汉阳侯焦立碑》。牛兰水又东南,与柏树溪水合。水出鲁山北峡谷中,东南流径鲁山西,而南合牛兰水。又东南径鲁山南。阚駰曰:鲁阳县,今其地鲁山是也。水南注于滍。滍水东径应城南,故应乡也,应侯之国。《诗》所谓应侯顺德者也。彭水注之,俗谓之小滍水。水出鲁阳县南彭山蚁坞东麓,北流径彭山西,下有彭山庙,庙前有《彭山碑》,汉桓帝元嘉三年杜仲长立。彭山径其西北,汉安邑长尹俭墓东。冢西有石庙,庙前有两石阙,阙东有碑,阙南有二狮子相对,南有石碣二枚,石柱西南有两石羊,中平四年立。彭水又东北流,直应城南而入滍。滍水又左合桥水,水出鲁阳县北恃山,东南径应山北,又南径应城西。《地理志》曰:故父城县之应乡也。周武王封其弟为侯国。应肋曰:《韩诗外传》称:周成王与弟戏,以桐叶为圭,曰:吾以封汝。周公曰:天子无戏言。王乃应时而封,故曰应侯,乡亦曰应乡。按《吕氏春秋》云。成王以桐叶为圭,封叔虞,非应侯也。《汲郡古文》殷时已有应国,非成王矣。战国范雄所封邑也。谓之应水。滍水又东径犨县故城北。《左传》昭公元年,冬,楚公子围使伯州犁城犨是也。出于鱼齿山下。《春秋》襄公十八年,楚伐郑,次于鱼陵,涉于鱼齿之下,甚雨,楚师多冻,役徒几尽。晋人闻有楚师。师旷曰:不害。吾骤歌北风,又歌南风,南风不竞,多死声,楚必无功矣。所涉即滍水也。水南有汉中常侍、长乐太仆吉成侯州苞冢。冢前有碑,基西枕冈城,开四门,门有两石兽,坟倾墓毁,碑兽沦移。人有掘出一兽,犹全不破,甚高壮,头去地减一丈许,作制甚工,左膊上刻作辟邪字。门表堑上起石桥,历时不毁。其碑云:六帝四后,是咨是诹。盖仕自安帝,没于桓后。于时阍阉擅权,五侯暴世,割剥公私,以事生死。夫封者表有德,碑者颂有功,自非此徒,何用许为?石至十春,不若速朽,苞墓万古,只彰俏辱,呜呼!愚亦甚矣。滍水又东,滍水注之,俗谓之秋水,非也。水有二源,东源出其县西南践犊山东崖下,水方五十许步、不恻其深,东北流径犨县南,又东北屈径其县东,而北合西源水。西源出县西南颇山北阜下,东北径犨城西,又屈径其县北,东合右水,乱流北注于滍。汉高祖入关,破南阳太守吕滍于犨东,即于是地,滍水之阴也。滍水又东南径昆阳县故城北。昔汉光武与王寻、王邑战于昆阳,败之。走者相腾践,奔殪百余里间。会大雨如注,滍川盛溢,虎豹皆股战。士卒争赴,溺死者以万数,水为不流。王邑、严尤、陈茂轻骑,皆乘尸而度矣。东北过颍川定陵县西北,又东过郾县南,东入于汝。
滍水东径西不羹亭南,亭北背汝水,于定陵城北,东入汝。
郾县在南,不得过。
淯水出弘农卢氏县支离山,东南过南阳西鄂县西北,又东过宛县南,淯水导源,东流径郦县故城北。郭仲产曰:郦县故城在支离山东南,郦旧县也。三仓曰樊、邓、郦。郦有二城,北郦也。汉祖入关,下淅郦,即此县也。淯水又东南流历雉县之衡山,东径百章郭北,又东,鲁阳关水注之。水出鲁阳县南分水岭,南水自岭南流,北水从岭北注,故世俗谓此岭为分头也。其水南流径鲁阳关,左右连山插汉,秀木于云,是以张景阳诗云:朝登鲁阳关,峡路峭且深。亦司马芝与母遇贼处也。关水历雉衡山西南径皇后城西。建武元年,世祖遣侍中傅俊,持节迎光烈皇后于清阳。俊发兵三百余人,宿卫皇后道路,归京师。盖税舍所在,故城得其名矣。山有石室,甚饰洁,相传名皇后浴室,又所幸也。关水又西南径雉县故城南,昔秦文公之世有伯阳者,逢二童曰舀,曰被。二童,二雉也。得雌者霸,雄者王。二童翻飞,化为双雉。光武获雉于此山,以为中兴之祥,故置县以名焉。关水又屈而东南流,注于淯。淯水又东南流,径博望县故城东。郭仲产曰,在郡东北百二十里,汉武帝置。校尉张骞,随大将军卫青西征,为军前导,相望水草,得以不乏。元光六年,封春为侯国。《地理志》,南阳有博望县,王莽改之曰宜乐也。清水又东南径西鄂故城东。应助曰:江夏有鄂,故加西也。昔刘表之攻杜子绪于西鄂也,功曹柏孝长闻战鼓之音,惧而闭户,蒙被自覆,渐登城而观,言勇可习也。清水又南,洱水注之,水出弘农郡卢氏县之熊耳山。东南径郦县北,东南径房阳城北。汉哀帝四年,封南阳太守孙宠为侯国。俗谓之房阳川。又径西鄂具南,水北有张平于墓。墓之东,侧坟有《平子碑》,文字悉是古文篆额,是崔瑗之辞。盛弘之、郭仲产并云:夏侯孝若为郡,薄其文,复刊碑阴为铭。然碑阴二铭,乃是崔子玉及陈翕耳,而非孝若,悉是隶字,二首并存,尝无毁坏。又言墓次有二碑,今惟见一碑,或是余夏景驿途疲而莫究矣。水南道侧,有二石楼,相去六七丈,双峙齐竦,高可丈七八,柱圆围二丈有余,石质青绿,光可以鉴。其上栾护承拱,雕檐四注,穷巧绔刻,妙绝人工。题言:蜀郡太守,姓王,字子雅,南阳西鄂人,有三女无男,而家累千金。父没当葬,女自相谓曰:先君生我姊妹,无男兄弟,今当安神玄宅,翳灵后土,冥冥绝后,何以彰吾君之德?各出钱五百万,一女筑墓,二女建楼,以表孝思。《铭》云:墓楼东,平林下,近坟墓,而不能测其处所矣。洱水又东南流,注于淯水,世谓之肄水,肄、洱声相近,非也。《地理志》曰:熊耳之山,出三水,洱水其一焉,东南至鲁阳入沔是也。淯水又南径预山东,山上有神庙,俗名之为独山也。山南有魏车骑将军黄权夫妻二冢,地道潜通。其冢前有四碑,其二,魏明帝立,二是其子及臣吏所树者也。清水又西南径史定伯碑南,又西为瓜里津,水上有三梁,谓之瓜里渡。自宛道途,东出堵阳,西道方城。建武三年,世祖自堵阳西入,破虏将军邓奉怨汉掠新野,拒瓜里,上亲搏战,降之夕阳下,遂斩奉。《郡国志》所谓宛有爪里津、夕阳聚者也。阻桥即桓温故垒处。温以升平五年,与范汪众军北讨所营。淯水又西南径晋蜀郡太守邓义山墓南,又南径宛城东。其城故申伯之都,楚文王灭申以为县也。秦昭襄王使白起为将,代楚取郢,即以此地为南阳郡,改县曰宛。王莽更名郡曰前队,县曰南阳。刘善曰:在中国之南而居阳地,故以为名。大城西南隅,即古宛城也,荆州刺史治,故亦谓之荆州城。今南阳郡治大城,其东城内有旧殿基,周二百步,高八尺,陛阶皆砌以青石。大城西北隅有殿基,周百步,高五尺,盖更始所起也。城西三里,有古台,高三丈余,文帝黄初中南巡行所筑也。淯水又屈而径其县南,故《南都赋》所言淯水荡其胸者也。王莽地皇二年,朱鲔等共于城南会诸将,设坛,燔燎,立圣公为天子于斯水上。《世语》曰:张绣反,公与战败,子昂不能骑,迸马于公,而昂遇害。《魏书》曰:公南征至宛,临淯水,祠阵亡将士,歔欷流涕,众皆哀恸。淯水又南,梅溪水注之。水出县北紫山,南径百里奚故宅。奚,宛人也。于秦为贤大夫,所谓迷虞智秦者也。梅溪又径宛西吕城东。《史记》曰:吕尚先祖为四岳,佐禹治水,有功。虞、夏之际,受封于吕,故因氏为吕尚也。徐广《史记音义》曰:吕在宛县。高后四年,封昆弟子吕忿为吕城侯,疑即此也。又案新蔡县有大吕、小吕亭,而未知所是也。梅溪又南径杜衍县东,故城在西。汉高帝七年,封郎中王翳为侯国,工莽更之曰闰衍矣。土地垫下,湍溪是注,古人于安众竭之,令游水是储,谓之安众港。世祖建武三年,上自宛遣颍阳侯祭遵西击邓奉弟终,破之于杜衍,进兵涅阳者也。梅溪又南,谓之石桥水,又谓之女溪,南流而左注淯水。淯水之南,又有南就聚,《郡国志》所谓南阳宛县有南就聚者也。郭仲产言:宛城南三十里,有一城,甚卑小,相承名三公城,汉时邓禹等归乡饯离处也。盛弘之著《荆州记》以为三公置。余案淯水左右,旧有二澨,所谓南澨、北澨者,水侧之濆。聚在淯阳之东北,考古推地则近矣。城侧有范蠡祠,蠡,宛人,祠即故宅也。后汉未,有范曾,字子闵,为大将军司马,讨黄中贼,至此祠,为蠡立碑,文勒可寻。夏侯湛之为南阳,又为立庙焉。城东有大将军何进故宅,城西有孔嵩旧居。嵩字仲山,宛人,与山阳范式有断金契。贫无养亲,赁为阿街卒,遣迎式。式下车把臂曰:子怀道卒伍,不亦痛乎!嵩曰:侯赢贱役,晨门,卑下之位,古人所不耻,何痛之有?故其《赞》曰:仲山通达,卷舒无方,屈身厮役,挺秀含芳。
又屈南过淯阳县东,淯水又南入县,径小长安。司马彪《郡国志》曰:县有小长安聚。谢沈《汉书》称:光武攻清阳不下,引兵欲攻宛,至小长安,与甄阜战,败于此。淯水又西南径其县故城南。桓帝延熹七年,封邓秉为侯国。县故南阳典农治,后以为淯阳郡,省郡复县,避晋简文讳,更名云阳焉。淯水又径安乐郡北。汉桓帝建和元年,封司徒胡广为淯阳县安乐乡侯,今于其国立乐宅戍。郭仲产《襄阳记》曰:南阳城南九十里,有晋尚书令乐广故宅。广字彦辅,善清言,见重当时。成都王,广女婿,长沙王猜之。广曰:宁以一女而易五男。犹疑之,终以忧殒。其故居今置戍,因以为名。
又南过新野县西,淯水又南入新野县,枝津分派,东南出,隰衍苞注,左积为陂,东西九里,南北十五里,陂水所溉,咸为良沃。淯水又南与湍水会,又南径新野县故城西。世祖之败小长安也,姊元遇害。上即位,感悼姊没,追谥元为新野节义长公主,即此邑也。晋咸宁二年,封大司马扶风武王少子歆为新野郡公,割南阳五属,棘阳、蔡阳、穰、邓、山都封焉。王文舒更立中隔,西即郡治,东则民居,城西傍淯水。又东与朝水合,水出西北赤石山,而东南径冠军县界,地名沙渠。又东南径穰县故城南,楚别邑也。秦拔鄢郢,即以为县,秦昭王封相魏冉为侯邑。王莽更名曰农穰也。魏荆州刺史治。朝水又东南,分为二水,一水枝分东北为樊氏陂,陂东西十里,南北五里,俗谓之凡亭陂。陂东有樊氏故宅,樊氏既灭,庾氏取其陂,故谚曰:陂汪汪,下田良,樊子失业痰公昌。昔在晋世,杜预继信臣之业,复六门陂,遏六门之水,下结二十九陂。诸陂散流,咸入朝水,事见《六门碑》。六门既陂,诸陂遂断。朝水又东径朝阳县故城北,而东南注于淯水。又东南与棘水合,水上承堵水。堵水出棘阳县北山,数源并发,南流径小堵乡,谓之小堵水。世祖建武二年,成安侯臧宫从上击堵乡。东源方七八步,腾涌若沸,故世名之腾沸水。南流径于堵乡,谓之堵水。建武三年,祭遵引兵南击董訢于堵乡。以水氏县,故有堵阳之名也。《地理志》曰:县有堵水,王莽曰阳城也。汉哀帝改为顺阳。建武二年,更封安阳侯朱佑为堵阳侯。堵水于县,竭以为陂,东西夹冈,水相去五六里,古今断冈两舌,都水潭涨,南北十余里,水决南溃,下注为湾。湾分为二,西为堵水,东为荣源。堵水参差,流结两湖,故有东陂、西陂之名。二陂所导,其水枝分,东南至会口人比。是以《地理志》比水、堵水,皆言入蔡,互受通称故也。二湖流注,合为黄水、惟所受焉。径棘阳县之黄淳聚,又谓之为黄淳水者也。谢沈《后汉书》,甄阜等败光武于小长安东、乘胜南渡黄淳水前营,背阻两川,谓临比水,绝后桥,示无还心。汉兵击之,三军溃,溺死黄淳水者二万人。又南径棘阳县故城西。应劭曰:县在棘水之阳,是知斯水为棘水也。汉高帝七年,封杜得臣为侯国。后汉兵起,击唐子乡,杀湖阳尉,进拔棘阳,邓晨将宾客会光武于此县也。棘水又南径新野县,历黄邮聚。世祖建武三年,傅俊、岑彭进击秦丰,先拨黄邮者也,谓之黄邮水。大司马吴汉破秦丰于斯水之上。其聚落悉为蛮居,犹名之为黄邮蛮。棘水自新野县东,而南流入于涓水,谓之为力口也。棘、力声相近,当为棘口也。又是方俗之音,故字从读变,若世以棘子木为力子木是也。涓水又东南径士林东,戍名也,戍有邪阁。水左有豫章大肢,下灌良畴三千许顷也。南过邓县东,县故邓侯吾离之国也,楚文王灭之,秦以为县。淯水右合浊水,俗谓之弱沟水。上承白水于朝阳县,东南流径邓县故城南。习凿齿《襄阳记》曰:楚王至邓之浊水,去襄阳二十里。即此水也。浊水又东径邓塞北,即邓城东南小山也,方俗名之为邓塞,昔孙文台破黄祖于其下。浊水东流注于淯,淯水又南径邓塞东,又径城东,古子国也。盖邓之南鄙也。昔巴子请楚与邓为好, 人夺其币,即是邑也。司马彪以为邓之聚矣。
南入于沔。
水出强县南泽中,东入颍。
水出颍川阳城县少室山,东流注于颖水,而乱流东南径临颍县西北,小水出焉。东径临颍县故城北。水又东径阳城北.又东径强县故城甫,建武二年,世祖封扬化将军坚镡为侯国。水东为陶枢陂,余按阳城在水南,然则此城正应为阴城,而有阳之名者,明在南犹有水,故此城以阳为名矣,颍水之南有二渎。其南渎东南流,历临颍亭西,东南入汝,今无水也,疑即水之故渎矣。汝水于奇雒城西,别东派,时人谓之大水,东北流,枝渎右出,世谓之死汝也。别汝又东北,径召陵城北,练沟出焉。别汝又东,汾沟出焉。别汝又东,径征羌城北。水南有汾陂,俗音粪。汾水自别汝东注,而为此陂。水积征羌城北四五里,方三十里许。读左合小水,水上承狼陂南流,名曰巩水。青陵陂水自陂东注之。东回又谓之小水,而南流注于大水。大水取称,盖藉沿注,而总受其目矣。又东径西华县故城南,又东径汝阳县故城北,东注于颖。
水出汝南吴房县西北奥山,东过其县,北入于汝。县西北有棠溪城。
故房子国。《春秋》定公五年,吴王阖闾弟夫概奔楚,封之于棠溪,故曰吴房也。汉高帝八年,封庄侯杨武为侯国。建武中,世祖封泗水王歙子为棠溪侯。山溪有白羊渊,渊水旧出山羊,汉武帝元封二年,白羊出此渊,畜牧者祷祀之。俗禁拍手,尝有羊出水,野母惊拍,自此绝焉。渊水下合灈水,灈水东径灈阳县故城西,东流入灈水,乱流径其县南,世祖建武二十八年,封吴汉孙旦为侯国。其水又东入于汝水。
水出阴县东上界山,《山海经》谓之视水也。郭景纯《注》:或曰,视宜为,出葴山。许慎云:出中阳山。皆山之殊目也。而东与泌水合,水出阴县旱山,东北流注。水又东北,杀水出西南大熟之山,东北流入于。水又东,沦水注之,水出宣山,东南流注水。水又东得奥水口,水西出奥山,东入于水也。
东过吴房县南,又东过灈阳县南,应劭曰:灈水出吴房县,东入,县之西北,即两川之交会也。又东过上蔡县南,东入汝。
水出阴县西北扶予山,东过其县南,《山海经》曰:朝歌之山, 水出焉,东南流注于荣。《经》书扶予者,其山之异名乎?荥水上承堵水,东流,左与西辽水合,又东,东辽水注之,俱导北山,而南流注于荣。荥水又东北,于阴县北左会水,其道稍西,不出其县南,其故城在山之阳,汉光武建武中,封岑彭为侯国,汉以为阳山县。魏武与张绣战于宛,马名绝景,为流矢所中,公伤右臂,引还阴,即是地也。城之东有马仁陂。郭仲产曰:肢在比阳县西五十里,盖地百顷,其所周溉田万顷,随年变种,境无俭岁。陂水三周其隍,故渎自隍西南而会于比, 水不得复径其南也。且邑号阴,故无出南之理,出南则为阳也。非直不究,又不恩矣。水又东北,澧水注之。水出雉衡山,东南径建城东,建当为卷,字读误耳。《郡国志》云:叶县有卷城。其水又东流入干扰。水东北径于东山西,西流入。水之左,即黄城山也。有溪水出黄城山,东北径方城。《郡国志》曰:叶县有方城。郭仲产曰:苦菜、于东之间,有小城,名方城,东临溪水。寻此城致号之由,当因山以表名也。苦菜即黄城也,及于东,通为方城矣。世谓之方城山水,东流注水。故《圣贤冢墓记》曰:南阳叶邑方城西有黄城山,是长沮、桀溺耦耕之所,有东流水,则子路问津处。《尸子》曰:楚狂接舆耕千方城。盖于此也。盛弘之云:叶东界有故城,始犨县,东至水,达比阳界,南北联联数百里,号为方城,一谓之长城,云郦县有故城一面,未详里数,号为长城,即此城之西隅。其间相去六百里,北面虽无基筑,皆连山相接,而汉水流其南,故屈完答齐桓公云楚国方城以为城,汉水以为池。《郡国志》曰:叶县有长山,曰方城。指此城也。水又东北,历舞阳县故城南,汉高祖六年,封樊哙为侯国也。
又东过西平县北,县故柏国也。《春秋左传》所谓江、黄、道、柏,方睦于齐也。汉曰西平,其西吕墟,即西陵亭也。西陵平夷,故曰西平。汉宣帝甘露三年,封丞相于定国为侯国。王莽更之曰新亭。《晋大康地记》曰:县有龙泉水,可以砥砺刀剑,特坚利,故有坚白之论矣。是以龙泉之剑为楚宝也。县出名金,古有铁官。
又东过郾县南,郾县故城去此远矣,不得过。
又东过定颍县北,东入于汝。
汉安帝永初二年,分汝南郡之上蔡县,置定颖县。顺帝永建元年,以阳翟郭镇为尚书令,封定颍侯,即此邑也。
溳水出蔡阳县,溳水出县东南大洪山,山在随郡之西南,竟陵之东北,槃基所跨,广圆百余里。峰曰悬钩,处平原众阜之中,为诸岭之秀,山下有石门,夹鄣层峻,岩高皆百许仞。入石门,又得钟乳穴。穴上素崖壁立,非人迹所及。穴中多钟乳,凝膏下垂,望齐冰雪。微律细液,滴沥不断。幽穴潜远,行者不极穷深,以穴内常有风热,无能经久故也。溳水出于其阴。初流浅狭,远乃广厚,可以浮舟袱巨川矣。时人以损水所导,故亦谓之为溳山矣。溳水东北流合石水,石水出大洪山,东北流注于涢,谓之小涢水。而乱流东北,径上唐县故城南。本蔡阳之上唐乡,旧唐侯国。《春秋》定公三年,唐成公如楚,有两肃霜马,子常欲之,弗与,止之三年,唐人窃马而献之,子常归唐侯是也。溳水又东,均水注之,水出大洪山,东北流径土山北,又东北流入于溃水。溳水又屈而东南流。
东南过随县西,县故随国矣。《春秋左传》所谓汉东之国,随为大者也。楚灭之以为县。晋武帝太康中立为郡。有搓水出县西北黄山,南径西县西,又东南, 水入焉。水出桐柏山之阳。吕忱曰:水在义阳。水东南径西县西,又东南注于, 水又东南径随县故城西。《春秋》鲁庄公四年,楚武王伐随。令尹斗祁、莫敖屈重,除道梁搓,军临于随,谓此水也。水侧有断蛇丘,随侯出而见大蛇中断,因举而药之,故谓之断蛇丘。后蛇衔明珠报德,世谓之随侯珠,亦曰灵蛇珠。丘南有随季梁大夫池,其水又南与义井水合,水出随城东南,井泉尝涌溢而津注,冬夏不异,相承谓之义井,下流合溳。溳水又南流注于溳。溳水又会于支水,水溳亦出大洪山,而东流注于溳。溳水又径随县南,随城山北,而东南注。又南过江夏安陆县西,随水出随郡永阳县东石龙山,西北流,南回,径永阳县西,历横尾山,即《禹贡》之陪尾山也。随水又西南,至安陆县故城西,入于溳,故郧城也。因冈为墉,峻不假筑。溳水又南径石岩山北。昔张昌作乱,于其下笼彩凤以惑众。晋太安二年,镇南将军刘弘遣牙门皮初与张昌战于清水,昌败,追斩于江涘。即《春秋左传》定公四年,吴败楚于柏举,从之,及于清发,盖溳水兼清水之目矣。又东南流而右会富水,水出竟陵郡新市县东北大阳山。水有二源:大富水出山之阳,南流而左合小富水。水出山之东,而南径三王城东。前汉未,王匡、王凤、王常所屯,故谓之三王城。城中有故碑,文字阙落,不可复识。其水屈而西南流,右合大富水,俗渭之大泌水也。又西南流径杜城西,新市县治也,《郡国志》以为南新市也。中山有新市,故此加南,分安陆县立。又王匡中兴初,举兵于县,号曰新市兵者也。富水又东南流,于安陆界左合土山水,世谓之章水。水出土山,南径随郡平林县故城西,俗谓之将陂城,与新市接界,故中兴之始,兵有新市、平林之号。又南流,右入富水,富水又东入于溳,溳水又径新城南。永和五年,晋大司马桓温筑。溳水又会温水,温水出竟陵之新阳县东泽中。口径二丈五尺,垠岸重沙,端净可爱,靖以察之,则渊泉如镜,闻入声,则扬汤奋发,无所复见矣。其热可以鸡,洪测百余步,冷若寒泉。东南流注于溳水。又右得潼水,水出江夏郡之曲陵县西北潼山,东南流径其县南,县治石潼故城,城圆而不方。东入安陆,注于溳水。
又东南入于夏。
溳水又南,分为二水,东通滠水,西入于沔,谓之溳口也。


卷三十二  漻水、蕲水、决水、沘水、泄水、肥水、施水、沮水、漳水、夏水、羌水、涪水、梓潼水、涔水 
漻水出江夏平春县西,漻水北出大义山,南至厉乡西,赐水入焉。水源东出大紫山,分为二水,一水西径厉乡南,水南有重山,即烈山也。山下有一穴,父老相传,云是神农所生处也,故《礼》谓之烈山氏。水北有九井,子书所谓神农既诞,九井自穿,谓斯水也。又言汲一井则众水动,井今堙塞,遗迹仿佛存焉。亦云赖乡,故赖国也,有神农社。赐水西南流入于漻,即厉水也,赐、厉声相近,宜为厉水矣。一水出义乡西南入随,又注漻。漻水又南径随县注安陆也。南过安陆入于溳。
薪水出江夏蕲春县北山,山即蕲柳也。水首受希水,枝津西南流,历蕲山,出蛮中,故以此蛮为五水蛮。五水谓已水、希水、赤亭水、西归水、蕲水其一焉。蛮左凭居,阻藉山川,世为抄暴。宋世沈庆之于西阳上下,诛代蛮夷,即五水蛮也。
南过其县西,晋改为蕲阳县,县徙江洲,置大阳戍,后齐齐昌郡移治于此也。又南至蕲口南,入于江。
蕲水南对蕲阳洲,入于大江,谓之蕲口。洲上有蕲阳县徙。决水出庐江雩娄县南大别山,俗谓之为檀公岘,盖大别之异名也。其水历山委注而络其县矣。
北过其县东,县故吴也。《春秋左传》襄公二十六年,楚子秦人侵吴,及雩娄,闻吴有备而还是也。《晋书地道记》云:在安丰县之西南,即其界也。故《地理志》曰,决水出雩娄。
又北过安丰县东,决水自雩娄县北,径鸡各亭。《春秋》昭公二十三年,吴败诸侯之师于鸡父者也。安丰县故城,今边城郡治也。王莽之美丰也。世祖建武八年,封大将军、凉州牧窦融为侯国,晋立安丰郡。决水自县西北流,径寥县故城东,又径其北,汉高帝六年,封孔藂为侯国,世谓之史水。决水又西北,灌水注之,其水导源庐江金兰县西北东陵乡大苏山,即淮水也。许慎曰:出雩娄县。俗谓之浍水。褚先生所谓神龟出于江、灌之间,嘉林之中,盖谓此水也。灌水东北径蓼县故城西,而北注决水,故《地理志》曰:决水北至蓼入淮,灌水亦至蓼入决。《春秋》宣公八年,冬,楚公子灭舒蓼。臧文仲闻之曰:皋陶庭坚,不祀忽诸,德之不逮,民之无援,哀哉!决水又北,右会阳泉水,水受决水,东北流径阳泉县故城东,故阳泉乡也。汉献帝中,封太尉黄琬为侯国。又西北流,左入决水,谓之阳泉口也。
又北入于淮。
俗谓之法口,非也,斯决、灌之口矣。余往因公至于淮津,舟车所届,次于决水,访其民宰,与古名全违。脉水寻经,方知决口。盖灌、浍声相伦,习俗害真耳。
砒水出庐江灊县西南,霍山东北,灊者,山、水名也。《开山图》灊山围绕大山为霍山。郭景纯曰:灊水出焉。县即其称矣。《春秋》昭公二十七年,吴因楚丧,围灊是也。《地理志》曰:批水出灊山,不言霍山,灊,字或作淠。淠水又东北径博安县,泄水出焉。东北过六县东,淠水东北,右会灊鼓川水,水出东南灊鼓川,西北流,左注淠水。淠水又西北径马亨城西,又西北径六安县故城西,县故皋陶国也。夏禹封其少子,奉其祀。今县都陂中有大冢,民传曰公琴者,即皋陶累也。楚人谓冢为琴矣。汉高帝元年,别为衡山国,五年属淮南,文帝十六年,复为衡山国。武帝元狩二年,别为六安国,王莽之安风也,《汉书》所谓以舒屠六。晋太康三年庐江郡治。淠水又西北,分为二水,芍陂出焉。又北径五门亭西,西北流径安丰县故城西。《晋书地道记》:安丰郡之属县也,俗名之曰安城矣。
又北会濡水,乱流西北注也。
北入于淮。
水之决会,谓之沘口也。
泄水出博安县,博安县,《地理志》之博乡县也,王莽以为扬陆矣。泄水自县,上承沘水于麻步川,西北出,历濡溪,谓之濡水也。北过芍陂,西与沘水合,泄水自濡溪径安丰县,北流注于淠,亦谓之濡须口。西北入于淮。
乱流同归也。
肥水出九江成德县广阳乡西,吕忱《字林》曰:肥水出良余山,俗谓之连枷山,亦或以为独山也。北流分为二水,施水出焉。肥水又北径获城东,又北径获丘东,右会施水枝津,水首受施水于合肥县城东,西流径成德县,注于肥水也。
北过其县西,北入芍陂。
肥水自荻丘,北径成德县故城西,王莽更之曰平阿也。又北径芍陂东,又北径死虎塘东,芍陂渎上承井门,与芍陂更相通注,故《经》言入芍陂矣。肥水又北,右合阎涧,水上承施水于合肥县,北流径浚遒县西,水积为阳湖,阳湖水自塘西北径死虎亭南,夹横塘西注。宋泰始初,豫州司马刘顺帅众八千,据其城地。以拒刘勔,赵叔宝以精兵五千,送粮死虎,刘勔破之此塘。水分为二,洛涧出焉,阎浆水注之。水受芍陂,陂水上承涧水于五门亭南,别为断神水,又东北径五门亭东,亭为二水之会也。断神水又东北径神迹亭东,又北,谓之豪水,虽广异名,事实一水,又东北径白芍亭东,积而为湖,谓之芍陂。陂周百二十许里,在寿春县南八十里,言楚相孙叔敖所造,魏太尉玉淩与吴将张休战于芍陂,即此处也。陂有五门,吐纳川流,西北为香门陂,陂水北径孙叔敖祠下,谓之芍陂读。又北分为二水:一水东注黎浆水,黎浆水东径黎浆亭南。文钦之叛,吴军北入,诸葛绪拒之于黎浆。即此水也。东注肥水,谓之黎浆水口。
又北过寿春县东,肥水自黎浆北径寿春县故城东为长濑津,津侧有谢堂北亭,迎送所薄,水陆舟车,是焉萃止。又西北,右合东溪,溪水引渎北出,西南流径导公寺西。寺侧因溪,建刹五层,屋字间敞,崇虚掏觉也。又西南流注于肥,肥水又西径东台下,台即寿春外郭,东北隅阿之榭也。东侧有一湖,三春九夏,红荷覆水,引读城隍,水积成潭,谓之东台湖,亦肥南播也。肥水西径寿春县故城北,右合北溪,水导北山泉源下注,漱石颓隍,水上长林插天,高柯负日,出于山林精舍右,山渊寺左。道俗嬉游,多萃其下。内外引汲,泉同七净。溪水沿注西南,径陆道士解南。精庐临侧川溪,大不为广,小足闲居,亦胜境也。溪水西南注于肥水。北入于淮。
肥水又西分为二水,右即肥之故读,遏为船官湖,以置舟舰也。肥水左渎,又西径石桥门北,亦曰草市门,外有石梁。渡北洲,洲上有西昌寺。寺三面阻水,佛堂设三像,真容妙相,相服精炜。是萧武帝所立也。寺西即船官坊,苍兕、都水,是营是作。湖北对八公山,山无树木,惟童阜耳。山上有淮南王刘安庙。刘安是汉高帝之孙,厉王长子也。折节下士,笃好儒学,养方术之徒数十人,皆为俊异焉。多神仙秘法鸿宝之道。忽有八公,皆须眉皓素,诣门希见。门者曰:吾王好长生,今先生无住衰之术,未敢相闻。八公咸变成童,王甚敬之。八士并能炼金化丹,出入无间,乃与安登山,薶金于地,白日升天。余药在器,鸡犬舐之者,俱得上升。其所升之处。践石皆陷,人马迹存焉。故山即以八公为目,余登其上,人马之迹无闻矣,惟庙像存焉,庙中图安及八士像,皆坐床帐如平生,被服纤丽,咸羽扇裙彼,中壶枕物,一如常居。庙前有碑,齐永明十年所建也。山有隐室石井,皆崔琰所谓余下寿春,登北岭淮甫之道室,八公石井在焉。亦云:左吴与王春、傅生等寻安,伺诣玄洲,还为著记,号曰《八公记》,都不列其鸡犬升空之事矣。按《汉书》,安反,伏诛,葛洪明其得道,事备《抱朴子》及《神仙传》。肥水又左纳芍陂渎。渎水自黎浆分水,引渎寿春城北,径芍陂门右,北入城。昔巨鹿时苗为县长,是其留犊处也。渎东有东都街,街之左道北,有宋司空刘勔庙。宋元徽二年,建于东乡孝义里。庙前有碑,时年碑功方创,齐永明元年方立。沈约《宋书》言泰始元年,豫州刺史殷琰反,明帝假勔辅国将军,讨之,琰降。不犯秋毫,百姓来苏,生为立碑,文过其实。建元四年,故吏颜幼明为其庙铭,故佐庞珽为庙赞,夏侯敬友为庙颂,并附刊于碑侧。渎水又北径相国城东,刘武帝伐长安所筑也,堂字厅馆仍故,以相国为名。又北出城,注肥水。又西径金城北,又西,左合羊头溪水。水受芍陂,西北历羊头溪,谓之羊头涧水。北径熨湖,左会烽水渎,渎受淮于烽村南,下注羊头溪,侧径寿春城西,又北历象门,自沙门北出金城西门逍遥楼下,北注肥渎。肥水北注旧渎之横塘,为玄康南路驰道,左通船官坊也。肥水径玄康城西北流,北出,水际有曲水堂,亦嬉游所集也。又西北流,昔在晋世,谢玄北御苻坚,祈八公山,及置阵于肥水之滨,坚望山上草木,威为人状,此即坚战败处。非八公之灵有助,盖荷氏将亡之惑也。肥水又西北注于淮,是曰肥口也。施水亦从广阳乡肥水别,东南入于湖。
施水受肥于广阳乡,东南流径合肥县。应劭曰:夏水出城父东南,至此与肥合,故曰合肥。阚駰亦言出沛国城父东,至此合为肥。余按川殊派别。无沿注之理,方知应、阚二说,非实证也。盖夏水暴长,施合于肥,故曰合肥也。非谓夏水、施水自成德东径合肥县城南。城居四水中,又东有逍遥津,水上旧有梁。孙权之攻合肥也,张辽败之于津北,桥不撤者两版。权与甘宁蹴马趋津,谷利自后著鞭助势,遂得渡梁。凌统被铠落水,后到追亡,流涕津渚。施水又东,分为二水,枝水北出焉,下注阳渊。施水又东径湖口戍,东注巢湖,谓之施口也。
沮水出汉中房陵县淮水,东南过临沮县界,沮水出东汶阳郡沮阳县西北景山,即荆山首也。高峰霞举,峻棘层云。
《山海经》云:金玉是出,亦沮水之所导。故《淮南子》曰:沮出荆山。高诱云:荆山在左冯翊怀德县,盖以洛水有漆沮之名故也。斯谬证耳。杜预云:水出新城郡之西南发阿山,盖山异名也。沮水东南流,径沮阳县东南,县有潼水,东径其县南,下入沮水。沮水又东南径汶阳郡北,即高安县界。郡治锡城,县居郡下,城故新城之下邑。义熙初分新城立。西表悉重山也。沮水南径临沮县西,青溪水注之。水出县西青山,山之东有滥泉,即青溪之源也。口径数丈,其深不测,其泉甚灵洁。至于炎阳有亢,阴雨无时,以秽物投之,辄能暴雨。其水导源东流,以源出青山,故以青溪为名。寻源浮溪,奇为深峭。盛弘之云:稠木傍生,凌空交合,危楼倾崖,恒有落势,风泉传响于青林之下,岩猿流声于白云之上,游者常若目不周玩,情不给赏。是以林徒栖托,云客宅心,泉侧多结道士精庐焉。青溪又东流入于沮水。沮水又屈径其县南。晋咸和中,为沮阳郡治也。沮水又东南,径当阳县故城北。城因冈为阻,北枕沮川,其故城在东百四十里,谓之东城,在绿林长坂南,长坂即张翼德横矛处也。沮水又东南径驴城西,磨城东,又南径麦城西,昔关云长诈降处,自此遂叛。《传》云:子胥造驴、磨二城以攻麦邑,即谚所云东驴西磨,麦城自破者也。沮水又南径楚昭王墓。东对麦城,故王仲宣之赋《登楼》云西接昭丘是也。沮水又南与漳水合焉。又东南过枝江县,东南人于江。沮水又东南径长城东,又东南流注于江,谓之诅口也。漳水出临沮县东荆山,东南过蓼亭,又东过章乡南,荆山在景山东百余里,新城沶乡县界。虽群峰竞举,而荆山独秀。漳水东南流,又屈西南,径编县南,县旧城之东北百四十里也。西南高阳城,移治许茂故城,城南临漳水。又南历临沮县之章乡南。昔关羽保麦城,诈降而遁,潘璋斩之于此。漳水又南径当阳县,又南径麦城东,王仲宣登其东南隅,临漳水而赋之曰:夹清漳之通浦,倚曲沮之长洲是也。漳水又南,沶水注之。《山海经》曰:沶水出东北宜诸之山,南流注于漳水。
又南至枝江县北乌扶邑,入于沮。
《地理志》曰:《禹贡》南条荆山,在临沮县之东北,漳水所出,东至江陵入阳水,注于沔。非也。今漳水于当阳县之东南百余里而右会沮水也。夏水出江津,于江陵县东南,江津豫章口东有中夏口,是夏水之首,江之汜也。屈原所谓过夏首而西浮,顾龙门而不见也。龙门即郢城之东门也。
又东过华容县南,县故容城矣。《春秋》鲁定公四年,许迁于容城是也。北临中夏水,自县东北,径成都郡故城南,晋永嘉中,西蜀阻乱,割华容诸城为成都王颖国。夏水又径交趾太守胡宠墓北。汉太傅广身陪陵,而此墓侧有广碑,故世谓广冢,非也。其文言是蔡伯喈之辞。历范西戎墓南。王隐《晋书地道记》曰:陶朱冢在华容县,树碑云是越之范蠡。《晋太康地记》、盛弘之《荆州记》、刘澄之《记》,并言在县之西南。
郭仲产言在县东十里,检其碑题云:故西戎令范君之墓。碑文缺落,不详其人,称蠡是其先也。碑是永嘉二年立。观其所述,最为究悉,以亲径其地,故违众说,从而正之。夏水又东,径监利县南。晋武帝太康五年立。县土卑下,泽多陂池,西南自州陵东界,径于云杜沌阳,为云梦之薮矣。韦昭曰:云梦在华容县。按《春秋》鲁昭公三年,郑伯如楚,子产备田具,以田江南之梦。郭景纯言华容县东南巴丘湖,是也。杜预云:枝江县、安陆县有云梦。盖跨川亘隰,兼苞势广矣。夏水又东,夏杨水注之。水上承杨水于竟陵县之柘口,东南流与中夏水合,谓之夏杨水。又东北径江夏惠怀县北,而东北注。
又东至江夏云杜县,入干沔。
应劭《十三州记》曰:江别入沔为夏水,源夫夏之为名,始于分江,冬竭夏流,故纳厥称,既有中夏之目,亦苞大夏之名矣。当其决入之所,谓之堵口焉。郑玄注《尚书》,沧浪之水,言今谓之夏水,来同,故世变名焉。刘澄之著《永初山川记》云:夏水,古文以为沧浪,渔父所歌也。因此言之,水应由沔。今按夏水是江流沔,非沔入夏。假使沔注夏,其势西南,非《尚书》又东之文,余亦以为非也。自堵口下沔水,通兼夏目,而会于江,谓之夏汭也。故《春秋左传》称吴伐楚,沈尹射奔命夏汭也。杜顶曰:汉水曲入江,即夏口矣。
羌水出羌中参狼谷,彼俗谓之天池白水矣。《地理志》曰:出陇西羌道。东南流径宕昌城东,西北去天池五百余里。羌水又东南,径宕婆川城东而东南注。昔姜维之寇陇右也,闻钟会入汉中,引还,知雍州刺史诸葛绪屯桥头,从孔函谷将出北道。绪邀之此路,维更从北道渡桥头,入剑阁,绪追之不及。羌水又东南,阳部水注之。水发东北阳部溪,西南径安民戍,又西南注羌水,又东南径武街城西南,又东南径葭芦城西,羊汤水入焉。水出西北阴平北界汤溪,东南径北部城北,又东南径五部城南,东南右合妾水傍西南出,即水源所发也。羌水又径葭芦城南,径余城南,又东南,左会五部水。水有二源,出南北五部溪,西南流合为一水,屈而东南注羌水。羌水又东南流至桥头,合白水,东南去白水县故城九十里。
又东南至广魏白水县,与汉水合,又东南过巴郡阆中县,又南至垫江县,东南入于江。
涪水出广魏涪县西北,涪水出广汉属国刚氐道徼外,东南流径涪县西,王莽之统睦矣。臧宫进破涪城,斩公孙恢于涪,自此水上。县有潺水,出潺山。水源有金银矿,洗取火合之,以成金银。潺水历潺亭而下注涪水。涪水又东南径绵竹县北。臧宫溯涪至于阳,公孙述将王元降,遂拔绵竹。涪水又东南,与建始水合,水发平洛郡西溪,西南流,屈而东南流,人于涪。涪水又东南径江油戍北。邓艾自阴平景谷步道,悬兵束马人蜀,径江油广汉者也。涪水又东南,径南安郡南,又南与金堂水会,水出广汉新都县,东南流入涪。涪水又南,枝津出焉,西径广汉五城县,为五城水,又西至成都,入于江。
南至小广魏,与梓潼水合。
小广魏即广汉县地,王莽更名曰广信也。
梓潼水出其县北界,西南人于涪。
故广汉郡,公孙述改为梓潼郡。刘备嘉霍峻守葭萌之功,又分广汉以北,别为梓潼郡,以峻为守。县有五女,蜀王遣五丁迎之,至此,见大蛇入山穴,五丁引之,山崩,压五丁及五女,因氏山为五妇山,又曰五妇候,驰水所出。一曰五妇水,亦曰潼水也。其水导源山中,南径梓潼县。王莽改曰子同矣。自县南径涪城东,又南入于涪水,谓之五妇水口也。
又西南至小广魏南,入于垫江。
亦言涪水至此入汉水,亦谓之为内水也。北径垫江。昔岑彭与臧宫自江州从涪水上。公孙述令延岑盛兵于沈水。宫左步右骑,夹船而进,势动山谷,大破岑军,斩首溺水者万余人,水为浊流。沈水出广汉县,下入涪水也。涔水出汉中南郑县东南旱山,北至安阳县,南入于沔。涔水即黄水也,东北流,径成固南城北。城在山上,或言韩信始立,或言张良创筑,未知定所制矣。义熙九年,索遐为果州刺史,自成固治此,故谓之南城。城周七里,衿涧带谷,绝壁百寻,北谷口造城东门,傍山寻涧,五里有余,盘道登陟,方得城治。城北水旧有桁,北渡涔水。水北有赵军城,城北又有桁渡沔,取北城,城即大成固县治也。黄水右岸有悦归馆,涔水历其北,北至安阳左入沔,为涔水口也。


卷三十三  江水 
岷山在蜀郡氐道县,大江所出,东南过其县北。岷山即渎山也,水曰渎水矣。又谓之汶阜山,在徼外,江水所导也。《益州记》曰:大江泉源,即今所闻,始发羊膊岭下,缘崖散漫,小水百数,殆未滥觞矣。东南下百余里,至白马岭,而历天彭阙,亦谓之为天彭谷也。秦昭王以李冰为蜀守,冰见氐道县有天彭山,两山相对,其形如阙,谓之天彭门,亦曰天彭阙。江水自此已上,至微弱,所谓发源滥觞者也。汉元延中,岷山崩,壅江水,三日不流。扬雄《反离骚》云:自岷山投诸江流,以吊屈原,名曰《反骚》也。江永自天彭阙东径汶关,而历氐道县北。汉武帝元鼎六年,分蜀郡北部;置汶山郡以统之。县本秦始皇置,后为升迁县也。《益州记》曰:自白马岭回行,二十余里至龙涸,又八十里至蚕陵县。又南下六十里至石镜。又六十余里而至北部,始百许步。又西百二十余里,至汶山故郡,乃广二百余步。又西南百八十里至湿坂,江稍大矣。故其精则井络缠曜,江汉昞灵。《河图括地象》曰:岷山之精,上为井络。帝以会昌,神以建福;故《书》曰:岷山导江,泉流深远,盛为四渎之首。《广雅》曰:江,贡也。《风俗通》曰:出珍物,可贡献。《释名》曰:江,共也。小水流入其中,所公共也。东北百四十里曰崃山,中江所出,东注于大江。崃山,邛崃山也,在汉嘉严道县。一曰新道,南山有九折坂,夏则凝冰,冬则毒寒,王阳按辔处也。平恒言是中江所出矣。郭景纯《江赋》曰:流二江于崌、崃。又东百五十里曰崌山,北江所出,东注于大江,《山海经》曰:崌山,江水出焉,东注大江。其中多怪蛇。江水又径汶江道。汶出徼外岷山西,玉轮坂下而南行,又东径其县而东注于大江。故苏代告楚曰:蜀地之甲,浮船于汶,乘夏水而下江,五日而至郢,谓是水也。又有湔水入焉。水出绵虒道,亦曰绵虒县之玉垒山。吕忱云:一曰半浣水也。下注江。江水又东别为沱,开明之所凿也。郭景纯所谓玉垒作东别之标者也。县即汶山郡治,刘备之所置也。渡江有笮桥。江水又历都安县。县有桃关、汉武帝祠。李冰作大堰于此,壅江作堋。堋有左、右口,谓之湔堋,江人郫江、捡江以行舟。《益州记》曰:江至都安堰其右,捡其左,其正流遂东,郫江之右也。眉山颓水,坐致竹木,以溉诸郡。又穿羊摩江、灌江,西于玉女房下白沙邮,作三石人,立水中。刻要江神,水竭不至足,盛不没肩。是以蜀人旱则藉以为溉,雨则不遏其流。故《记》曰:水旱从人,不知饥馑,沃野千里,世号陆海,谓之天府也。邮在堰上,俗谓之都安大堰,亦曰湔堰,又谓之金堤。左思《蜀都赋》云西逾金堤者也。诸葛亮北征,以此堰农本,国之所资,以征丁千二百人主护之,有堰官。益州刺史皇甫晏至都安,屯观坂。从事何旅曰:今所安营,地名观坂,上观下反,其征不祥。不从,果为牙门张和所杀。江水又径临邛县,王莽之监邛也。县有火井、盐水,昏夜之时,光兴上照。江水又径江原县,王莽更名邛原也,都江水出焉。江水又东北径郫县下,县民有姚精者,为叛夷所杀,掠其二女。二女见梦其兄,当以明日自沉江中,丧后日当至,可伺候之,果如所梦,得二女之尸于水,郡、县表异焉。江水又东径成都县,县以汉武帝元鼎二年立。县有二江,双流郡下。故扬子云《蜀都赋》曰:两江珥其前者也。《风俗通》曰:秦昭王使李冰为蜀守,开成都两江,溉田万顷。江神岁取童女二人为妇。冰以其女与神为婚,径至神祠,劝神酒,酒杯恒澹澹,冰厉声以责之,因忽不见。良久,有两牛斗于江岸旁,有间,冰还,流汗,谓官属曰:吾斗大亟,当相助也。南向腰中正白者,我绶也。主簿刺杀北面者,江神遂死。蜀人慕其气决,凡壮健者,因名冰儿也。秦惠王二十六年,遣张仪与司马错等灭蜀,遂置蜀郡焉。王莽改之曰导江也。仪筑成都以象咸阳。晋太康中,蜀郡为王国,更为成都内史,益州刺史治。《地理风俗记》曰:华阳黑水惟梁州。汉武帝元朔二年,改梁曰益州,以新启犍为、牂柯、越巂,州之疆壤益广,故称益云,初治广汉之雒县,后乃徙此。故李固《与弟圄书》曰,固今年五十七,鬓发已白,所谓容身而游,满腹而去,周观天下,独未见益州耳。昔严夫子常言:经有五,涉其四,州有九,游其八。欲类此子矣。初,张仪筑城,取土处去城十里,因以养鱼,今万顷池是也。城北又有龙堤池,城东有千秋池,西有柳池,西北有天井池。津流径通,冬夏不竭。西南两江有七桥,直西门郫江上,曰冲治桥,西南石牛门曰市桥,吴汉入蜀,自广都令轻骑先往焚之。桥下谓之石犀渊,李冰昔作石犀五头,以厌水精,穿石犀渠于南江,命之曰犀牛里,后转犀牛二头,一头在府市市桥门,一头沉之于渊也,大城南门曰江桥,桥南曰万里桥,西上曰夷星桥,下曰笮桥。南岸道东有文学,始文翁为蜀守,立讲堂,作石室于南城。永初后,学堂遇火,后守更增二石室。后州夺郡学,移夷星桥南岸道东。道西城,故锦官也。言锦工织锦,则濯之江流,而锦至鲜明,濯以他江,则锦色弱矣,遂命之为锦里也。蜀有回复水,江神尝溺杀人,文翁为守,祠之,劝酒不尽,拔剑击之,遂不为害。江水东径广都县,汉武帝元朔二年置,王莽之就都亭也。李冰识察水脉,穿县盐井,江西有望川原,凿山崖度水,结诸陂池,故盛养生之饶,即南江也。又从冲治桥北折曰长升桥。城北十里曰升仙桥,有送客观,司马相如将人长安,题其门曰:不乘高车驷马,不过汝下也。后人邛蜀,果如志焉。李冰沿水造桥,上应七宿,故世租谓吴汉曰:安军宜在七桥连星间。汉自广都乘胜进逼成都,与其副刘尚南北相望,夹江为营,浮桥相对。公孙述使谢丰扬军市桥出汉后,袭破汉,坠马落水,缘马尾得出,入壁。命将夜潜渡江就尚,击丰,斩之于是水之阴。江北则左对繁田,文翁又穿湔以溉灌繁田千七百顷。湔水又东绝绵洛,径五城界,至广都北岸,南入于江,谓之五城水口,斯为北江。江水又东至南安为壁玉律,故左思云东越玉津也。
又东南过犍为武阳县,青衣水、沫水从西南来,合而注之。
县故大夜郎国,汉武帝建元六年,开置郡县。太初四年,益州刺史任安城武阳。王莽更名郡曰西顺,县曰戢成。光武谓之士大夫郡。有江人焉。出江原县,首受大江,东南流至武阳县,注于江。县下江上,旧有大桥,广一里半,谓之安汉桥。水盛岁坏,民苦治功。后太守李严凿天社山,寻江通道。此桥遂废。县有赤水,下注江。建安二十九年,有黄龙见此水,九日方去。此县藉江为大堰,开六水门,用灌郡下。北山,昔者王乔所升之山也。江水又与文井江会,李冰所导也。自莋道与蒙溪分水,至蜀郡临邛县,与布仆水合。水出徼外成都西沈黎郡。汉武元封四年,以蜀都西部邛莋置,理旄牛道,天汉四年置都尉,主外羌,在邛峡山表。自蜀西度邛莋,其道至险,有弄栋人渡之难,扬母阁路之阻。水从县西布仆来,分为二流。一水径其道,又东径临邛县入文井水。文井水又东径江原县,县滨文井江。江上有常氏堤,跨四十里。有朱亭,亭南有青城山,山上有嘉谷山,下有蹲鸱,即芋也。所谓下有蹲鸱,至老不饥,卓氏之所以乐远徙也。文井江文东至武阳县天社山下,入江。其一水南径越巂邛都县西,东南至云南郡之青岭县,入于仆。郡本云川地也,蜀建兴三年置。。仆水又南径永昌郡邪龙县而与贪水合。水出青蛉县,上承青蛉水,径叶榆县,又东南至邪龙入于仆。仆水又径宁州建宁郡,州故康降都督屯,故南人谓之屯下。刘禅建兴三年分益州郡置。历双柏县,即水入焉,水出秦臧县牛兰山,南流至双柏县,东注仆水。又东至来唯县入劳水。水出徼外,东径其县,与仆水合。仆水东至交州交趾郡泠县,南流入于海。江水自武阳东至彭亡聚。昔岑彭与吴仅溯江水人蜀,军次是地,知而恶之。会日暮不移,遂为刺客所害。谓之平模水,亦曰外水。此地有彭冢,言彭祖冢焉。江水又东南径南安县,西有熊耳峡,连山竞险,接岭争高。汉河平中,山崩地震,江水逆流。悬溉有滩,名垒坻,亦曰盐溉、李冰所平也。县治青衣江会,衿带二水矣,即蜀王开明故治也。来敏《本蜀论》曰:荆人鄨令死,其尸随水上,荆人求之不得,令至汶山下,复生,起见望帝。望帝者,杜宇也,从天下。女子朱利,自江源出,为宇妻,遂王于蜀,号曰望帝;望帝立以为相。时巫山峡而蜀水不流,帝使令凿巫峡通水,蜀得陆处。望帝自以德不若,遂以国禅,号曰开明。县南有峨盾山,有蒙水,即大渡水也。水发蒙溪,东南流与俄水合。水出徼外,径汶江道。吕忱曰:渽水出蜀。许慎以为俄水也。出蜀汶江徼外。从水,我声。南至南安,入大渡水。大渡水又东人江,故《山海经》曰蒙水出汉阳西,入江滠阳西。
又东南过道县北,若水、淹水合从西来注之。又东,渚水北流注之。
县本人居之。《地理风俗记》曰:夷中最仁,有仁道,故字从人。《秦纪》所谓僮之富者也。其邑,高后六年城之。汉武帝感相如之言,使县令南通道,费功无成,唐蒙南入,斩之,乃凿石开阁,以通南中。迄于建宁,二千余里,山道广丈余,深三四丈,其錾凿之迹犹存。王莽更曰治也,山多犹猢,似猴而短足,好游岩树,一腾百步,或三百丈,顺往倒返,乘空若飞。县有蜀王兵兰,其神作大难江中,崖峻阻险,不可芽凿,李冰乃积薪烧之,故其处悬岩,犹有五色焉。赤白照水玄黄,鱼从来,至此而止,言畏崖屿不更上也。《益部耆旧传》曰:张真妻,黄氏女也,名帛。真乘船覆没,求尸不得。帛至没处滩头,仰天而叹,遂自沉渊。积十四日,帛持真手于滩下出。时人为说曰:符有先络, 道有张帛者也。江水又与符黑水合,水出宁州南广郡南广县。县故犍为之属县也,汉武帝太初元年置,刘禅延熙中分以为郡。导源汾关山,北流,有大涉水注之。水出南广县,北流注符黑水,又北径道入江,谓之南广口。渚水则未闻也。
又东过江阳县南,洛水从三危山,东过广魏洛县南,东南注之。洛水出洛县漳山,亦言出梓潼县柏山。《山海经》曰:三危在燉煌南,与岷山相接,山南带黑水。又《山海经》不言洛水所导。《经》曰出三危山,所未详。常璩云:李冰导洛通山水,流发瀑口,径什邡县。汉高帝六年,封雍齿为侯国,王莽更名曰美信也。洛水又南径洛县故城南,广汉郡治也。汉高祖之为汉王也,发巴渝之士,北定三秦。六年,乃分巴蜀,置广汉郡于乘乡。王莽之就都,县曰吾雒也。汉安帝永初二年,移治涪城,后治洛县。先是洛县城南,每阴雨常有哭声,闻于府中。积数十年,沛国陈宠为守,以乱世多死亡,暴骸不葬故也。乃悉收葬之,哭声遂绝。刘备自将攻洛,庞士元中流矢死于此。益州旧以蜀郡、广汉、健为为三蜀。土地沃美、人士隽乂,一州称望。县有沈乡,去江七里,姜士游之所居。诗至孝,母好饮江水,嗜鱼脍,常以鸡鸣溯流汲江。子坐取水溺死,妇恐姑知,称托游学,冬夏衣服,实投江流。于是至孝上通,涌泉出其舍侧,而有江之甘焉。诗有田滨江泽卤,泉流所溉,尽为沃野。又涌泉之中,旦旦常出鲤鱼一双以膳焉,可谓孝悌发于方寸,徽美著于无穷者也。洛水又南径新都县,蜀有三都,谓成都、广都,此其一焉。与绵水合。水西出绵竹县,又与湔水合,亦谓之郫江也,又言是涪水。吕忱曰:一曰湔。然此二水俱与洛会矣。又径犍为牛鞞县为牛鞞水。昔罗尚乘牛鞞水,东征李雄,谓此水也。县以汉武帝元封二年置。又东径资中县,又径汉安县,谓之绵水也。自上诸县,咸以溉灌,故语曰:绵洛为没沃也。绵水至江阳县方山下入江,谓之绵水口,亦曰中水。江阳县枕带双流,据江、洛会也。汉景帝六年封赵相苏嘉为侯国,江阳郡治也。故犍为枝江都尉,建安十八年刘璋立。江中有大阙、小阙焉。季春之月,则黄龙堆没,阙乃平也。昔世祖微时,过江阳县,有一子。望气者言江阳有贵儿象,王莽求之,而獠杀之。后世祖怨,为子立祠于县,谪其民罚布数世。扬雄《琴清英》曰:尹吉甫子伯奇至孝,后母谮之,自投江中。衣苔带藻,忽梦见水仙赐其美药,思惟养亲,扬声悲歌。船人闻之而学之。吉甫闻船人之声,疑似伯奇,援琴作《子安之操》。江水径汉安县北。县虽迫山川,土地特美,蚕桑鱼盐家有焉。江水东径樊石滩,又径大附滩,频历二险也。
又东过符县北邪,东南鰼部水从符关东北注之。
县故巴夷之地也。汉武帝建元六年,以唐蒙为中郎将,从万人出巴符关者也,元鼎二年立,王莽之符信矣。县治安乐水会,水源南通宁州平夷郡鄨县,北径安乐县界之东,又径符县下,北入江。县长赵祉遣吏先尼和,以永建元年十二月诣巴郡,没死成湍滩,予贤求丧不得。女络,年二十五岁,有二子,五岁以还。至二年二月十五日,尚不得丧、络乃乘小船,至父没处,哀哭自沉。见梦告贤曰:至二十一日,与父俱出。至日,父子果浮出江上。郡、县上言,为之立碑,以旌孝诚也。其鰼部之水,所未闻矣,或是水之殊目,非所究也。
又东北至巴郡江州县东,强水、涪水、汉水、白水、宕渠水五水合,南流注之。
强水即羌水也。宕渠水即潜水、渝水矣。巴水出晋昌郡宣汉县已岭山,郡隶梁州,晋太康中立,治汉中。县南去郡八百余里,故蜀巴渠。西南流历巴中,径巴郡故城南,李严所筑大城北,西南入江。庾仲雍所谓江州县对二水口,右则涪内水,左则蜀外水、即是水也。江州县,故巴子之都也。《春秋》桓公九年,巴子使韩服告楚,请与邓好是也。及七国称王,巴亦王焉。秦惠王遣张仪等救苴侯于巴,仪贪巴、苴之富,因执其王以归,而置巴郡焉,治江州。汉献帝初平元年,分巴为三郡,于江州则永宁郡治也。至建安六年,刘璋纳蹇胤之讼,复为巴郡,以严颜为守。颜见先主入蜀,叹曰:独坐穷山,放虎自卫。此即拊心处也。汉世郡治江州巴水北,北府城是也。后乃徙南城。刘备初以江夏费观为太守,领江州都督。后都护李严,更城周十六里,造苍龙、白虎门,求以五郡为巴州治,丞相诸葛亮不许,竟不果。地势侧险,皆重屋累居,数有火害,又不相容、结舫水居者五百余家,承二江之会,夏水增盛,坏散颠没,死者无数。县有官橘、官荔枝园,夏至则熟。二千石常设厨膳,命士大夫共会树下食之。县北有稻田,出御米也。县下又有清水穴,巴人以此水为粉,则皜曜鲜芳,贡粉京师,因名粉水,故世谓之为江州堕林粉。粉水亦谓之为粒水矣。江之北岸,有涂山,南有夏禹庙、涂君祠,庙铭存焉。常璩、庾仲雍并言禹娶于此。余案群书,咸言禹娶在寿春当涂,不于此也。又东至枳县西,延江水从牂柯郡北流西屈注之。
江水东径阳关巴子梁,江之两岸,犹有梁处,巴之三关,斯为一也。延熙中,蜀车骑将军邓芝为江州都督,治此。江水又东,右径黄葛峡,山高险,全无人居。江水又左径明月峡,东至梨乡,历鸡鸣峡。江之南岸有枳县治。《华阳记》曰:枳县在江州巴郡东四百里,治涪陵水会。庾仲雍所谓有别江出武陵者也。水乃延江之枝津,分水北注,径涪陵入江,故亦云涪陵水也。其水南导武陵郡,昔司马错溯舟此水,取楚黔中地。延熙中,邓芝伐徐巨射玄猿于是县。猿自拔矢,卷木叶塞射创。芝叹曰:伤物之生,吾其死矣。江水又东径涪陵故郡北,后乃并巴郡,遂罢省。江水又东径文阳滩,滩险难上。江水又东径汉平县二百余里,左自涪陵东出百余里,而届于黄石,东为桐柱滩。又径东望峡,东历平都,峡对丰民洲,旧巴子别都也。《华阳记》曰:巴子虽都江州,又治平都。即此处也。有平都县,为巴郡之隶邑矣。县有天师治,兼建佛寺,甚清灵。县有市肆,四日一会。江水右径虎须滩,滩水广大,夏断行旅。江水又东径临江县南,王莽之监江县也。《华阳记》曰:县在枳东四百里,东接胸忍。县有盐官。自县北入盐井溪,有盐井营户。溪水沿汪江。江水又东得黄华水口,江浦也,左径石城南。庾仲雍曰:临江至石城黄华口一百里。又东至平洲,洲上多居民。又东径壤涂而历和滩。又东径界坛,是地巴东之西界,益州之东境,故得是名也。
又东过鱼复县南,夷水出焉。
江水又东,右得将龟溪口。《华阳记》曰:朐忍县出灵龟,咸熙元年,献龟于相府,言出自此溪也。江水又东,会南、北集渠,南水出涪陵县界,谓之阳溪。北流径巴东郡之南浦侨县西。溪硖侧,盐井三口,相去各数十步,以木为桶,径五尺,修煮不绝。溪水北流注于江,谓之南集渠口,亦曰于阳溪口。北水出新浦县北高梁山分溪。南流径其县西,又南百里,至朐忍县,南入于江,谓之北集渠口,别名班口,又曰分水口,朐忍尉治此。江水又东,右径汜溪口,盖江汜决入也。江水又东,径石龙而至于博阳二村之间,有盘石,广四百丈,长六里,阻塞江川,夏没冬出,基亘通诸。又东径羊肠虎臂滩。杨亮为益州,至此舟覆,惩其波澜,蜀人至今犹名之为使君滩。江水又东,彭水注之。水出已渠郡獠中,东南流径汉丰县东,清水注之。水源出西北巴渠县东北巴岭南獠中,即巴渠水也。西南流至其县,又西入峡,檀井溪水出焉。又西出峡,至汉丰县东而西注彭溪,谓之清水口。彭溪水又南,径朐忍县西六十里,南流注于江,谓之彭溪口。江水又东,右径朐忍县故城南。常璩曰:县在巴东郡西二百九十里,县治故城,跨其山阪,南临大江。江之南岸有方山,山形方峭,枕侧江濆。江水又东径瞿巫滩,即下瞿滩也,又谓之博望摊。左则汤溪水注之,水源出县北六百余里上庸界,南流历县,翼带盐井一百所,巴川资以自给。粒大者,方寸,中央隆起,形如张伞,故因名之曰伞子盐。有不成者,形亦必方,异于常盐矣。王隐《晋书地道记》曰:人汤口四十三里,有石,煮以为盐。石大者如升,小者如拳,煮之,水竭盐成,盖蜀火井之伦,水火相得乃佳矣。汤水下与檀溪水合,水上承巴渠水,南历檀井溪,谓之檀井水。下入汤水。汤水又南人于江,名曰汤口。江水又径东阳滩。江上有破石,故亦通谓之破石滩,苟延光没处也。常璩曰:水道有东阳、下瞿数滩,山有大小石城势,灵寿木及橘圃也。故《地理志》曰:县有橘官,有民市。江水又径鱼复县之故陵,旧郡治故陵溪西二里故陵村,溪即永谷也。地多木瓜树,有子大如,白黄,实甚芬香,《尔雅》之所谓楙也。江水又东为落牛滩,径故陵北。江侧有六大坟。庾仲雍曰:楚都丹阳所葬,亦犹枳之已陵矣,故以故陵为名也,有鱼复尉,戍此,江之左岸有巴乡村,村人善酿,故俗称巴乡清,郡出名酒。村侧有溪,溪中多灵寿木。中有鱼,其头似羊,丰肉少骨,美于余鱼。溪水伏流径平头山,内通南浦故县陂湖。其地平旷,有湖泽,中有菱芡鲫雁,不异外江,凡此等物,皆人峡所无,地密恶蛮,不可轻至。江水又东,右径夜清而东历朝阳道口,有县治,治下有市,十日一会。江水又东,左径新市里南,常璩曰:巴旧立市于江上,今新市里是也。江水又东,右合阳元水,水出阳口县西南,高阳山东,东北流径其县南,东北流,丙水注之。水发县东南柏枝山,山下有丙穴,穴方数丈,中有嘉鱼,常以春末游诸,冬初入穴,抑亦褒汉丙穴之类也。其水北流入高阳溪。溪水又东北流,注于江,谓之阳元口。江水又东径南乡峡,东径永安宫南,刘备终于此,诸葛亮受遗处也。其间平地可二十许里,江山迥阔,入峡所无。城周十余里,背山面江,颓塘四毁,荆棘成林,左右民居多星其中。江水又东径诸葛亮图垒南,石债平旷,望兼川陆,有亮所造八阵图,东跨故垒,皆累细石为之。自垒西去,聚石八行,行间相去二丈,因曰:八阵既成,自今行师庶不覆败。皆图兵势行藏之权,自后深识者所不能了。今夏水漂荡,岁月消损,高处可二三尺,下处磨灭殆尽。江水又东径赤岬城西,是公孙述所造,因山据势,周回七里一百四十步,东高二百丈,西北高千丈,南连基白帝山,甚高大,不生树木。其石悉赤。土人云,如人袒胛,故谓之赤岬山。《淮南子》曰:徬徨于山岬之旁。《注》曰:岬,山胁也。郭仲产曰:斯名将因此而兴矣。江水又东径鱼复县故城南,故鱼国也。《春秋左传》文公十六年。庸与群蛮叛,楚庄王代之,七遇皆北,惟裨、鯈、鱼人逐之是也。《地理志》江关都尉治。公孙述名之为白帝,取其王色。蜀章武二年,刘备为吴所破,改白帝为永安,巴东郡治也。汉献帝兴平元年,分巴为二郡,以鱼复为故陵郡。蹇胤诉刘璋,改为巴东郡,治白帝山,城周回二百八十步,北缘马岭,接赤岬山,其间平处,南北相去八十五丈,东西七十丈,又东傍东瀼溪,即以为隍。西南临大江,窥之眩目。惟马岭小差委迤。犹斩山为路,羊肠数四,然后得上。益州刺史鲍陋镇此,为谯道福所围,城里无泉,乃南开水门,凿石为函道,上施木天公,直下至江中,有似猿臂相牵,引汲然后得水。水门之西,江中有孤石,为淫顶石,冬出水二十余丈,夏则没。亦有裁出处矣。县有夷溪,即佷山清江也。《经》所谓夷水出焉。江水又东径广溪峡,斯乃三峡之首也。其间三十里,颓岩倚木,厥势殆交。北岸山上有神渊,渊北有白盐崖,高可千余丈,俯临神渊。土人见其高白,故因名之。天旱,燃木岸上,推其灰烬,下秽渊中,寻即降雨。常璩曰:县有山泽水神,旱时鸣鼓请雨,则必应嘉泽。《蜀都赋》所谓应鸣鼓而兴雨也。峡中有瞿塘、黄龛二滩,夏水回复,沿溯所忌。瞿塘滩上有神庙,尤至灵验。刺史二千石径过,皆不得鸣角伐鼓。商旅上水,恐触石有声,乃以布裹篙足。今则不能尔,犹飨荐不辍。此峡多猿,猿不生北岸,非惟一处,或有取之,放著北山中,初不闻声,将同狢兽渡汶而不生矣。其峡,盖自昔禹凿以通江,郭景纯所谓巴东之峡,夏后疏凿者。


卷三十四  江水 
又东出江关,入南郡界。
江水自关,东径弱关、捍关。捍关,廪君浮夷水所置也。弱关在建平、秭归界。昔巴、楚数相攻伐,藉险置关,以相防捍,秦兼天下,置立南郡,自巫东上皆其域也。
又东过巫县南,盐水从县东南流注之。
江水又东,乌飞水注之。水出天门郡溇中县界,北流径建平郡沙渠县南,又北流径巫县南,西北历山道三百六十里,注于江,谓之乌飞口。江水又东径巫县故城南,县故楚之巫郡也。秦省郡立县,以隶南郡。吴孙休分为建平郡,治巫城。城缘山为塘,周十二里一百一十步,东西北三面皆带傍深谷,南临大江,故夔国也。江水又东,巫溪水注之。溪水导源梁州晋兴郡之宣汉县东,又南径建平郡泰昌县南,又径北井县西,东转历其县北。水南有盐井,井在县北,故县名北井,建平一郡之所资也。盐水下通巫溪,溪水是兼盐水之称矣。溪水又南,屈径巫县东。县之东北三百步,有圣泉,谓之孔子泉。其水飞清石穴,洁并高泉,下注溪水。溪水又南入于大江。江水又东径巫峡,杜宇所凿以通江水也。郭仲产云:按《地理志》,巫山在县西南,而今县东有巫山,将郡、县居治无恒故也。江水历峡东,径新崩滩,此山汉和帝永元十二年崩,晋太元二年又崩。当崩之日,水逆流百余里,涌起数十丈。今滩上有石,或圆如箪,或方似屋,若此者甚众,皆崩崖所陨,致怒湍流,故谓之新崩滩。其颓岩所余,比之诸岭,尚为竦桀,其下十余里,有大巫山,非惟三峡所无,乃当抗峰岷、峨,偕蛉衡、疑。其翼附群山,井概青云,更就霄汉,辨其优劣耳。神孟涂所处,《山海经》曰:夏后启之臣孟涂,是司神于巴。巴人讼于孟涂之所,其衣有血者执之。是请生居山上,在丹山西。郭景纯云:丹山在丹阳,属巴,丹山西即巫山者也。又帝女居焉。宋玉所谓天帝之季女,名曰瑶姬,未行而亡,封于巫山之阳。精魂为草,实为灵芝,所谓巫山之女,高唐之阻,旦为行云,暮力行雨,朝朝暮暮,阳台之下。旦早视之,果如其言,故为立庙,号朝云焉。其间首尾百六十里,谓之巫峡,盖因山为名也。自三峡七百里中,两岸连山,略无阙处,重岩叠嶂,隐大蔽日,自非停午夜分,不见曦月。至于夏水襄陵,沿溯阻绝,或王命急宣,有时朝发白帝,暮到江陵,其间千二百里,虽乘奔御风,不以疾也。春冬之时,则素湍绿潭,回清倒影,绝多生怪柏,悬泉瀑布,飞漱其间,清荣峻茂,良多趣味。每至晴初霜旦,林寒涧肃,常有高猿长啸,属引凄异,空谷传响,哀转久绝。故渔者歌曰:巴东三峡巫峡长,猿鸣三声泪沾裳。江水又东径石门滩。滩北岸有山,山上合下开,洞达东西,缘江步路所由,刘备为陆逊所破,走径此门,追者甚急,备乃烧铠断道。孙桓为逊前驱,奋不顾命,斩上夔道,截其要径,备逾山越险,仅乃得免,忿恚而叹曰:吾昔至京,桓尚小儿.而今迫孤,乃至于此。遂发愤而薨矣。
又东过秭归县之南,县故归乡,《地理志》曰:归子国也。《乐纬》曰:昔归典叶声律。宋忠曰:归即夔。归乡盖夔乡矣。古楚之嫡嗣有熊挚者,以废疾不立,而居于夔,为楚附庸。后王命为夔子。《春秋》值公二十六年,楚以其不祀灭之者也。袁山松曰:屈原有贤姊,闻原放逐,亦来归,瑜令自宽全。乡人冀其见从,因名曰秭归。即《离骚》所谓女嬃蝉媛以署余也。县城东北,依山即坂,周回二里,高一丈五尺,南临大江。古老相传,谓之刘备城,盖备征吴所筑也。县东北数十里,有屈原旧田宅。虽畦堰縻漫,犹保屈田之称也。县北一百六十里,有屈原故宅,累石为室基,名其地曰乐平里。宅之东北六十里,有女嬃庙,捣衣石犹存。故《宜都记》曰:种归盖楚子熊绎之始国,而屈原之乡里也。原田宅于今具存,指谓此也。江水又东径一城北,其城凭岭作固,二百一十步,夹溪临谷,据山枕江,北对丹阳城,城据山跨阜,周八里二百八十步,东北两面,悉临绝涧,西带亭下溪,南枕大江,险峭壁立,信天固也。楚子熊绎始封丹阳之所都也。《地理志》,以为吴之丹阳。论者云:寻吴楚悠隔, 缕荆山,无容远在吴境,是为非也。又楚之先王陵墓在其间,盖为征矣。江水又东南径夔城南,跨据川阜,周回一里百一十八步,西北背枕深谷,东带乡口溪,南侧大江。城内西北角有金城,东北角有圆土狱,西南角有石井口,径五尺。熊挚始治巫城,后疾移此,盖夔徙也。《春秋左传》值公二十六年,楚令尹子玉城夔者也。服虔曰:在巫之阳,秭归归乡矣。江水又东径归乡县故城北。袁山松曰:父老传言,原既流放,忽然蹔归,乡人喜悦,因名曰归乡。抑其山秀水清,故出俊异,地险流疾,故其性亦隘。《诗》云:惟岳降神,生甫及申。信与!余谓山松此言,可谓因事而立证,恐非名县之本旨矣。县城南面重岭,北背大江。东带乡口溪,溪源出县东南数百里,西北入县。径狗峡西,峡崖龛中石,隐起有狗形,狗状具足,故以狗名峡。乡口溪又西北径县下入江,谓之乡口也。江水又东径信陵县,南临大江,东傍深溪,溪源北发梁州上庸县界,南流径县下,而注于大江也。
又东过夷陵县南,江水自建平至东界峡,盛弘之谓之空泠峡。峡甚高峻,即宜都、建平二郡界也。其间远望,势交岭表,有五六峰,参差互出。上有奇石,如二人像,攘袂相对,俗传两郡督邮争界于此,宜都督邮,厥势小东倾,议者以为不如也。江水历峡东,径宜昌县之插灶下,江之左岸,绝岸壁立数百丈,飞鸟所不能栖。有一火烬,插在崖间,望见可长数尺。父老传言,昔洪水之时,人薄舟崖侧,以余烬插之岩侧,至今犹存,故先后相承谓之插灶也。江水又东径流头滩,其水并峻激奔暴,鱼鳖所不能游。行者常苦之,其歌曰:滩头白勃坚相持,倏忽沦没别无期。袁山松曰:自蜀至此,五千余里,下水五日,上水百日也。江水又东径宜昌县北,分夷道很山所立也。县治江之南岸,北枕大江,与夷陵对界。《宜都记》曰:渡流头滩十里,便得宜昌县。江水又东径狼尾滩而历人滩。袁山松曰:二滩相去二里。人滩水至峻峭,南岸有青石,夏没冬出,其石.崟,数十步中,悉作人面形,或大或小。其分明者,须发皆具,因名曰人滩也。江水又东径黄牛山,下有滩,名曰黄牛滩。南岸重岭叠起,最外高崖间有石色如人负刀牵牛,人黑牛黄,成就分明,既人迹所绝,莫得究焉,此岩既高,加以江湍纡回,虽途径信宿,犹望见此物,故行者谣曰:朝发黄牛,暮宿黄牛,三朝三暮,黄牛如故。言水路纡深,回望如一矣。江水又东径西陵峡,《宜都记》曰:自黄牛滩东入西陵界,至峡口百许里,山水纡曲,而两岸高山重障,非日中夜半,不见日月。绝壁或千许丈,其石彩色,形容多所像类。林木高茂,略尽冬春。犹鸣至清,山谷传响,泠泠不绝。所谓三峡,此其一也。山松言:常闻峡中水疾,书记及口传,悉以临惧相戒,曾无称有山水之美也。及余来践脐此境,既至欣然,始信耳闻之不如亲见矣。其叠崿秀峰,奇构异形,固难以辞叙。林木萧森,离离蔚蔚,乃在霞气之表。仰瞩俯映,弥习弥佳。流连信宿,不觉忘返,目所履历,未尝有也。既自欣得此奇观,山水有灵,亦当惊知己于千古矣。江水历禹断江南。峡北有七谷村,两山间有水清深,潭而不流。又耆旧传言,昔是大江,及禹治水,此江小不足泻水,禹更开今峡口,水势并冲,此江遂绝,于今谓之断江也。江水出峡,东南流,径故城洲。洲附北岸,洲头曰郭洲,长二里,广一里,上有步阐故城,方圆称洲,周回略满,故城洲上,城周五里,吴西陵督步骘所筑也。孙皓凤凰元年,骘息阐复为西陵督,据此城降晋,晋遣太傅羊祜接援,未至,为陆抗所陷也。江水又东径故城北,所谓陆抗城也。城即山为墉,四面天险。江南岸有山孤秀,从江中仰望,壁立峻绝。袁山松为郡,尝登之瞩望焉。故其《记》云:今自山南上至其岭,岭容十许人,四面望诸山,略尽其势。俯临大江,如萦带焉,视舟如凫雁矣。北对夷陵县之故城。城南临大江。秦令白起伐楚,三战而烧夷陵者也。应劭曰:夷山在西北,盖因山以名县也。王莽改曰居利。吴黄武元年,更名西陵也。后复曰夷陵。县北三十里,有石穴,名曰马穿。尝有白马出穴,人逐之入穴,潜行出汉中。汉中人失马,亦尝出此穴,相去数千里。袁山松言江北多连山,登之望江南诸山,数十百重,莫识其名,高者千仞,多奇形异势,自非烟寨雨霁,不辨见此远山矣。余尝往返十许过,正可再见远峰耳。江水又东径白鹿岩。沿江有峻壁百余丈,猿所不能游。有一白鹿,陵峭登崖,乘岩而上,故世名此岩为白鹿岩。江水又东历荆门、虎牙之间。荆门在南,上合下开,暗彻山南,有门像,虎牙在北,石壁色红,间有白文类牙形,并以物像受名。此二山,楚之西塞也。水势急峻,故郭景纯《江赋》曰:虎牙桀竖以屹崒,荆门阙竦而盘薄,圆渊九回以悬腾,湓流雷呴而电激者也。汉建武十一年,公孙述遣其大司徒任满、翼江王田戎,将兵数万,据险为浮桥,横江以绝水路,营垒跨山,以塞陆道。光武遣吴汉、岑彭将六万人击荆门,汉等率舟师攻之,直冲浮桥,因风纵火,遂斩满等矣。
又东南过夷道县北,夷水从佷山县南,东北注之。
夷道县,汉武帝伐西南夷,路由此出,故曰夷道矣。王莽更名江南。桓温父名彝,改曰西道。魏武分南郡置临江郡。刘备改曰宜都。郡治在县东四百步。故城,吴丞相陆逊所筑也。为二江之会也。北有湖里渊,渊上橘袖蔽野,桑麻暗日,西望佷山诸岭,重峰叠秀,青翠相临,时有丹霞白云,游曳其上。城东北有望堂,地特峻,下临清江,游瞩之名处也。县北有女观山,厥处高显,回眺极目。古老传言,昔有思妇,夫官于蜀,屡愆秋期。登此山绝望,忧感而死,山木枯悴,鞠为童枯。乡人哀之,因名此山为女观焉。葬之山顶,今孤坟尚存矣。
又东过枝江县南,沮水从北来注之。
江水又东,径上明城北。晋太元中,荷坚之寇荆州也,刺史桓冲徙渡江南,使刘波筑之,移州治此城。其地夷敞,北据大江。江汜枝分,东入大江,吴治洲上,故以枝江为称。《地理志》曰:江沱出西,东入江是也。其地故罗国,盖罗徙也。罗故居宜城西山,楚文王又徒之于长沙,今罗县是矣。县西三里有津乡,津乡,里名也。《春秋》庄公十九年,巴人伐楚,楚子御之,大败于津。应劭曰:南郡江陵有津乡。今则无闻矣。郭仲产云,寻楚御巴人,枝江是其涂。便此津乡殆即其地也。盛弘之曰:县旧治沮中,后移出百里洲,西去郡百六十里。县左右有数十洲,槃布江中,其百里洲最为大也。中有桑田甘果,映江依洲。自县西至上明,东及江津,其中有九十九洲。楚谚云:洲不百,故不出王者。桓玄有问鼎之志,乃增一洲,以充百数。僭号数旬,宗灭身屠。及其倾败,洲亦消毁。今上在西,忽有一洲自生,沙流回薄,成不淹时,其后未几,龙飞江汉矣。县东二里,有县人刘凝之故宅。凝之字志安,兄盛公,高尚不仕。凝之慕老莱、严子陵之为人,立屋江湖,非力不食。妻梁州刺史郭诠女,亦能安贫。宋元嘉中,夫妻隐于衡山,终焉不返矣。县东北十里,土台北岸有迤洲,长十余里,义熙初,烈武王斩桓谦处。县东南二十里,富城洲上,有道士范侪精庐。自言巴东人,少游荆土。而多盘桓县界。恶衣粗食,萧散自得,言来事多验,而辞不可详。人心欲见,歘然而对,貌言寻求,终弗遇也。虽径跨诸洲,而舟人未尝见其济涉也。后东游广陵,卒于彼土。侪本无定止处,宿憩一小庵而已。弟子慕之,于其昔游,共立精舍,以存其人。县有陈留王子香庙,颂称子香于汉和帝之时,出为荆州刺史,有惠政,天子征之,道卒枝江亭中。常有三白虎,出入人间,送丧逾境。百姓追美甘棠,以永元十八年,立庙设祠,刻石铭德,号曰枝江白虎王君。其子孙至今犹谓之为白虎王。江水又东会沮口,楚昭王所谓江、汉,沮、漳,楚之望也。
又南过江陵县南。
县北有洲,号曰枚回洲,江水自此两分而为南、北江也。北江有故乡洲。元兴之末,桓玄西奔,毛祐之与参军费恬射玄于此洲。玄子升,年六岁,辄拔去之。王韶之云:玄之初奔也,经日不得食,左右进粗粥,咽不能下。升抱玄胸抚之,玄悲不自胜。至此,益州都护冯迁斩玄于此洲,斩升于江陵矣。下有龙洲,洲东有宠洲,二洲之间,世擅多鱼矣。渔者投罟历网,往往挂绝,有潜客泳而视之,见水下有两石牛,尝为罾害矣。敌渔者莫不击浪浮舟;鼓枻而去矣。其下谓之邴里洲,洲有高沙湖,湖东北有小水通江,名曰曾口。江水又东径燕尾洲北,合灵溪水,水无泉源,上承散水,合承大溪,南流注江。江、溪之会有灵溪戍,背阿面江,西带灵溪,故戍得其名矣。江水东得马牧口,江水断洲通会。江水又东径江陵县故城南,《禹贡》荆及衡阳惟荆州,盖即荆山之称而制州名矣。故楚也。子革曰:我先君僻处荆山以供王事,遂迁纪郢。今城,楚船官地也。《春秋》之渚宫矣。秦昭襄王二十九年,使白起拔鄢鄂,以汉南地而置南郡焉。《周书》曰:南,国名也。南氏有二臣,力钩势敌,竞进争权,君弗能制。南氏用分为二南国也。按韩婴叙《诗》云:其地在南郡、南阳之间。《吕氏春秋》所谓禹自涂山巡省南土者也。是郡取名焉。后汉景帝以为临江王荣国。王坐侵庙壖地为宫,被征,升车,出北门面轴折。父老窃流涕曰:吾王不还矣!自后北门不开,盖由荣非理终也。汉景帝二年,改为江陵县。王莽更名郡曰南顺,县曰江陆。旧城,关羽所筑。羽北围曹仁,吕蒙袭而据之。羽曰:此城吾所筑,不可攻也。乃引而退,杜元凯之攻江陵也,城上人以瓠系狗颈示之,元凯病瘿故也。及城陷,杀城中老小,血流沾足。论者以此薄之。江陵城地东南倾,故缘以金堤,自灵溪始。桓温令陈遵造。遵善于方功,使人打鼓,远听之,知地势高下,依傍创筑,略无差矣。城西有栖霞楼,俯临通隍,吐纳江流,城南有马牧城,西侧马径,此洲始自枚回,下迄于此,长七十余里,洲上有奉城,故江津长所治,旧主度州郡贡于洛阳,因谓之奉城。亦曰江津戍也。戍南对马头岸。昔陆抗屯此,与羊祜相对,大宏信义,谈者以为华元、子反复见于今矣。北对大岸,谓之江津口,故洲亦取名焉。江大自此始也。《家语》曰:江水至江津,非方舟避风,不可涉也,故郭景纯云:济江津以起涨,言其深广也。江水又东径郢城南,子囊遗言所筑城也。《地理志》曰:楚别邑,故郢矣。王莽以为郢亭。城中有赵台卿冢,歧平生自所营也。冢图宾主之容,用存情好,叙其宿尚矣。江水又东得豫章口,夏水所通也,西北有豫章冈,盖因冈而得名矣。或言因楚王豫章台名,所未详也。


卷三十五  江水 
又东至华容县西,夏水出焉。
江水左迤为中夏水,右则中郎浦出焉。江浦右迤,南派屈西,极水曲之势,世谓之江曲者也。
又东南当华容县南,涌水入焉。
江水又东,涌水注之。水自夏水南通于江,谓之涌口。二水之间,《春秋》所谓阎敖游涌而逸者也。江水又径南平郡孱陵县之乐乡城北,吴陆抗所筑,后王濬攻之,获吴水军督陆景于此渚也。
又东南,油水从东南来注之。
又东,右合油口,又东径公安县北。刘备之奔江陵,使筑而镇之。曹公闻孙权以荆州借备,临书落笔。杜预克定江南,罢华容置之,谓之江安县,南郡治。吴以华容之南乡为南郡,晋太康元年改曰南平也。县有油水,水东有景口,口即武陵郡界。景口东有沦口,沦水南与景水合。又东通澧水及诸陂湖,自此渊潭相接,悉是南蛮府屯也。故侧江有大城,相承云仓储城,即邸阁也。江水左会高口,江浦也,右对黄州。江水又东得故市口,水与高水通也。江水又右径阳岐山北,山枕大江,山东有城,故华容县尉旧治也。大江又东,左合子夏口。江水左迤北出,通于夏水,故曰子夏也。大江又东,左得侯台水口,江浦也。大江右得龙穴水口,江浦右迤也。北对虎洲。又洲北有龙巢,地名也,昔禹南济江,黄龙夹舟,舟人五色无主。禹笑曰:吾受命于天,竭力养民,生,性也,死,命也,何忧龙哉?于是二龙弭鳞掉尾而去焉,故水地取名矣。江水自龙巢而东,得俞口,夏水泛盛则有,冬无之。江之北岸,上有小城,故监利县尉治也。又东得清阳土坞二口,江浦也。大江右径石首山北,又东径赭要。赭要,洲名,在大江中,次北湖洲下。江水左得饭筐上口,秋夏水通下口,上下口间,相距三十余里。赭要下即杨子洲,在大江中,二洲之间,常苦蚊害。昔荆佽飞济此,遇两蚊,斩之,自后罕有所患矣。江之右岸则清水口,口上即钱官也。水自牛皮山东北通江,北对清水洲,洲下接生江洲,南即生江口,水南通澧浦。江水左会饭筐下口,江浦所入也。江水又右得上檀浦,江溠也。江水又东径竹町南,江中有观洋溠,溠东有大洲,洲东分为爵洲,洲南对湘江口也。
又东至长沙下隽县北,澧水、沅水、资水合东流注之。凡此诸水,皆注于洞庭之陂,是乃湘水,非江川。湘水从南来注之。
江水右会湘水,所谓江水会者也。江水又东,左得二夏浦,俗谓之西江口。又东径忌置山南,山东即隐口浦矣。江之右岸有城陵山,山有故城,东接微落山,亦曰晖落矶。江之南畔名黄金濑,濑东有黄金浦,良父口,夏浦也。又东径彭城口,水东有彭城矶,故水受其名,即玉涧,水出巴丘县东玉山玉溪,北流注于江。江水自彭城矶东径如山北,北对隐矶,二矶之间,有独石孤立大江中,山东江浦,世谓之白马口。江水又左径白螺山南,右历鸭兰矶北,江中山也。东得鸭兰、治浦二口,夏浦也。江水左径上乌林南,村居地名也。又东径乌黎口,江浦也,即中乌林矣。又东径下乌林南,吴黄盖败魏武于乌林,即是处也。江水又东,左得子练口,北通练浦,又东合练口,江浦也。南直练洲,练名所以生也。江之右岸得蒲矶口,即陆口也。水出下隽县西三山溪,其水东径陆城北,又东径下隽县南,故长沙旧县,王莽之闰隽也。宋元嘉十六年,割隶巴陵郡。陆水又屈而西北流,径其县北,北对金城,吴将陆涣所屯也。陆水又人蒲圻县北,径吕蒙城西。昔孙权征长沙零、桂所镇也。陆水又径蒲矶山,北入大江,谓之刀环口。又东径蒲矶山北,北对蒲圻洲,亦曰擎洲,又曰南洲,洲头即蒲圻县治也,晋太康元年置。洲上有白面洲,洲南又有澋口,水出豫章艾县,东入蒲圻县,至沙阳西北鱼岳山入江,山在大江中,扬子洲南,孤峙中洲。江水左得中阳水口,又东得白沙口,一名沙屯,即麻屯口也。本名蔑默口,江浦矣。南直蒲圻洲,水北入百余里,吴所屯也。又径鱼岳山北,下得金梁洲。洲东北对渊洲,一名渊步洲。江濆从洲头以上,悉壁立无岸,历蒲圻至白沙,方有浦,上甚难。江中有沙阳洲,沙阳县治也。县本江夏之沙羡矣。晋太康中改曰沙阳县。宋元嘉十六年,割隶巴陵郡。江之右岸有雍口,亦谓之港口。东北流为长洋港。又东北径石子冈,冈上有故城,即州陵县之故城也。庄辛所言左州侯国矣。又东径州陵新治南,王莽之江夏也。港水东南流注于江,谓之洋口。南对龙穴洲,沙阳洲之下尾也。洲里有驾部口,宋景平二年,迎文帝于江陵,法驾顿此,因以为名。文帝车驾发江陵。至此黑龙跃出,负帝所乘舟,左右失色。上谓长史王昙首曰:乃夏禹所以受天命矣,我何德以堪之?故有龙穴之名焉。江水又东,右得聂口,江浦也,左对聂洲。江水左径百人山南,右径赤壁山北,昔周瑜与黄盖诈魏武大军处所也。江水东径大军山南。山东有山屯,夏浦江水左迤也。江中有石浮出,谓之节度石。右则涂水注之,水出江州武昌郡武昌县金山,西北流径汝南侨郡故城南,咸和中,寇难南逼,户口南渡,因置斯郡,治于涂口。涂水历县西,又西北流注于江。江水又东径小军山南,临侧江津,东有小军浦。江水又东径鸡翅山北,山东即土城浦也。又东北至江夏沙羡县西北,沔水从北来注之。
沌水上承沌阳县之太白湖,东南流为沌水,径沌阳县南,注于江,谓之沌口。有沌阳都尉治。晋永嘉六年,王敦以陶侃为荆州,镇此。明年徙林鄣。江水又东径叹父山,南对叹州,亦曰叹步矣。江之右岸当鹦鹉洲,南有江水右迤,谓之驿渚。三月之末,水下通樊口水。江水又东径鲁山南,古翼际山也。《地说》曰:汉与江合于衡北翼际山旁者也。山上有吴江夏太守陆涣所治城,盖取二水之名。《地理志》曰:夏水过郡入江,故曰江夏也。旧治安陆;权高帝六年置,吴乃徙此。城中有《晋征南将军荆州刺史胡奋碑》。又有平南将军王世将刻石记征杜曾事。有刘琦墓及庙也。山左即沔水口矣。沔左有郤月城,亦曰偃月垒,戴监军筑,故曲陵县也。后乃沙羡县治。昔魏将黄祖所守,遣董袭、凌统攻而擒之。祢衡亦遇害于此。衡恃才倜傥;肆狂猖于无妄之世,保身不足,遇非其死,可谓咎悔之深矣。江之右岸有船官浦,历黄鹄矶西而南矣,直鹦鹉洲之下尾。江水溠曰洑浦,是曰黄军浦,昔吴将黄盖军师所屯,故浦得其名,亦商舟之所会矣。船官浦东即黄鹄山。林涧甚美,谯郡戴仲若野服居之。山下谓之黄鹄岸,岸下有湾,目之为黄鹄湾。黄鹄山东北对夏口城,魏黄初二年孙权所筑也。依山傍江,开势明远,凭墉藉阻,高观枕流,上则游目流川,下则激浪崎岖,实舟人之所艰也。对岸则人沔津,故城以夏口为名,亦沙羡县治也。江水左得湖口,水通太白湖,又东合滠口,水上承溳水于安陆县而东径滠阳县北,东流注于江。江水又东,湖水自北南注,谓之嘉吴。江右岸频得二夏浦,北对东城洲西,浦侧有雍伏戍。江之右岸,东会龙骧水口,水出北山蛮中。江之左有武口,水上通安陆之延头,宋元嘉二年,卫将军、荆州刺史谢晦,阻兵上流,为征北檀道济所败,走奔于此,为戍主光顺之所执处也。南至武城,俱入大江。南直武洲,洲南对杨桂水口,江水南出也,通金女、大文、桃班三治。吴旧屯所在,荆州界尽此。江水东径若城南。庾仲雍《江水记》曰:若城至武城口三十里者也。南对郭口夏浦,而不常泛矣。东得苦菜夏浦,浦东有苦菜山,江径其北,故浦有苦菜之名焉,山上有菜苦可食。江水左得广武口江浦也。江之右岸有李姥浦,浦中偏无蚊蚋之患矣。北对峥嵘洲,冠军将军刘毅破桓玄于此洲,玄乃挟天子西走江陵矣。
又东过邾县南,江水东径白虎矶北,山临侧江濆,又东会赤溪夏浦浦口,江水右迤也。
又东径贝矶北,庾仲雍谓之沛岸矣。江右岸有秋口,江浦也,又东得鸟石水,出乌石山,南流注于江。江水右得黎矶,矶北亦曰黎岸也。山东有夏浦,又东径上碛北,山名也,仲雍谓之大小竹碛也。北岸峰火洲,即举洲也。北对举口,仲雍作莒字,得其音而忘其字,非也。举水出龟头山,西北流径蒙茏戍南,梁定州治,蛮田秀超为刺史。举水又西流,左合垂山之水,水北出垂山之阳,与弋阳淠水同发一山,故是水合之。水之东有南口戍,又南径方山戍西,西流注于举水。又西南径梁司、豫二州东,蛮田鲁生为刺史,治湖陂城,亦谓之水城也。举水又西南径颜城南,又西南径齐安郡西,倒水注之。水出黄武山,南流径白沙戍西,又东南径梁达城戍西,东南合举水,举水又东南历赤亭下,谓之赤亭水,又分为二水,南流注于江,谓之举口。南对举洲,《春秋左传》定公四年,吴、楚陈于柏举。京相璠曰:汉东地矣。江夏有溳水,或作举,疑即此也。左水东南流入于江。江浒曰文方口。江之右岸有凤鸣口,江浦也。浦侧有凤鸣戍。江水又东径邾县故城南,楚宣王灭邾,徒居于此,故曰邾也。汉高帝元年,项羽封吴芮为衡山王,都此。晋咸和中,庾翼为西阳太守,分江夏立。咸康四年,豫州刺史毛宝,西阳太守樊俊共镇之,为石虎将张格度所陷,自尔丘墟焉。城南对芦洲,旧吴时筑客舍于洲上,方便惟所止焉,亦谓之罗洲矣。
鄂县北,江水右得樊口,庾仲雍《江水记》云:谷里袁口,江津南入,历樊山上下三百里,通新兴、马头二治。樊口之北有湾。昔孙权装大船,名之曰长安,亦曰大舶,载坐直之士三千人,与群臣泛舟江津,属值风起,权欲西取芦洲。谷利不从,乃拔刀急上,令取樊口薄舶船,至岸而败,故名其处为败舶湾。因凿樊山为路以上,人即名其处为吴造岘,在樊口上一里,今厥处尚存。江水又左径赤鼻山南,山临侧江川。又东径西阳郡南,郡治即西阳县也。《晋书地道记》以为弦子国也。江之右岸有鄂县故城,旧樊楚地,《世本》称熊渠封其中子红为鄂王。《晋太康地记》以为东鄂矣。《九州记》曰:鄂,今武昌也。孙权以魏黄初元年,自公安徒此,改曰武昌县。鄂县徙治于袁山东,又以其年立为江夏郡,分建业之民千家以益之。至黄龙元年,权迁都建业,以陆逊辅太子镇武昌。孙皓亦都之,皓还东,令滕牧守之。晋惠帝永平中,始置江州,傅综为刺史,治此城,后太尉庾亮之所镇也。今武昌郡治。城南有袁山,即樊山也。《武昌记》曰:樊口南有大姥庙,孙权常猎于山下,依夕,见一姥,问权猎何所得。曰:正得一豹。母曰:何不竖豹尾?忽然不见。应助《汉官序》曰:豹尾过后,执金吾罢屯,解围。天于卤簿中,后属车施豹尾。于道路,豹尾之内为省中。盖权事应在此,故为立庙也。又孙皓亦尝登之,使将害常侍王蕃,而以其首虎争之。北背大江,江上有钓台,权常极饮其上,曰:堕台醉乃已。张昭尽言处。城西有郊坛,权告天即位于此,顾谓公卿曰:鲁子敬尝言此,可谓明于事势矣。城东故城,言汉将灌婴所筑也。江中有节度石三段,广百步,高五六丈,是西阳、武昌界,分江于斯石也。又东得次浦,江浦也。东径五矶北,有五山,沿次江阴,故得是名矣,仲雍谓之五圻。江水左则巴水注之,水出雩娄县之下灵山,即大别山也。与决水同出一山,故世谓之分水山,亦或曰巴山。南历蛮中,吴时,旧立屯于水侧,引巴水以溉野。又南径巴水戍,南流注于江,谓之巴口。又东径噡县故城南,故弦国也。《春秋》僖公五年秋,楚灭弦,弦子奔黄者也。汉惠帝元年,封长沙相利仓为侯国,城在山之阳,南对五洲也。江中有五洲相接,故以五洲为名。宋孝武帝举兵江州,建牙洲上,有紫云荫之,即是洲也。东会希水口,水出灊县霍山西麓。山北有灊县故城,《地理志》曰:县南有天柱山。即霍山也。有祠南岳庙。音潜。齐立霍州治此。西南流分为二水,枝津出焉。希水又南,积而为湖,谓之希湖。湖水又南流,径噡县东而南流注于江,是曰希水口者也。然水流急濬,霖雨暴涨,漂滥无常,行者难之。大江右岸有厌里口,安乐浦。从此至武昌,尚方作部诸屯相接,枕带长江。又东得桑步,步下有章浦,本西阳郡治,今悉荒芜。江水左得赤水浦,夏浦也。江水又东径南阳山南,又曰芍矶,亦曰南阳矶,仲雍谓之南阳圻,一名洛至圻,一名石姥,水势迅急。江水又东径西陵县故城南,《史记》秦昭王遣白起伐楚,取西陵者也。汉章帝建初二年,封阴堂为侯国。江水东历孟家溠。江之右岸有黄石山,水径其北,即黄石矶也。一名石茨圻,有西陵县。县北则三洲也。山连延江侧,东山偏高,谓之西塞,东对黄公九矶,所谓九圻者也。于行小难,两山之间为阙塞。从此济于土复,土复者,北岸地名也。
又东过蕲春县南,蕲水从北东注之。
江水又得苇口,江浦也。浦东有苇山,江水东径山北。北崖有东湖口,江波左边,流结成湖,故谓之湖口矣。江水又东得空石口,江浦在右,临江有空石山,南对石穴洲,洲上有蕲阳县治。又东,蕲水注之。江水又东,径蕲春县故城南。世祖建武三十年,封陈俊子浮为侯国。江水又东,得铜零口,江浦也。大江右径虾蟆山北,而东会海口。水南通大湖,北达于江。左右翼山,江水径其北,东合臧口,江浦也。江水又左径长风山南,得长风口,江浦也。江水又东径积布山南,俗谓之积布矶,又曰积布圻,庾仲雍所谓高山也。此即西阳、寻阳二郡界也。右岸有土复口,江浦也。夹浦有江山,山东有护口,江浦也。庾仲雍谓之朝二浦也。
又东过下雉县北,利水从东陵西南往之。
江水东径琵琶山南,山下有琵琶湾。又东径望夫山南,又东得苦菜水口,夏浦也。江之有岸,富水注之,水出阳新县之青湓山,西北流径阳新县,故豫章之属县矣。地多女鸟。《玄中记》曰:阳新男子,于水次得之,遂与共居,生二女,悉农羽而去。豫章间养儿,不露其衣,言是鸟落尘于儿衣中,则令儿病,故亦谓之夜飞游女矣。又西北径下雉县,王莽更名之润光矣,后并阳新。水之左右,公私裂溉,咸成沃壤。旧吴屯所在也。江水又东,右得兰溪水口。并江浦也。又东,左得青林口,水出庐江郡之东陵乡,江夏有西陵县,故是言东矣。《尚书》云江水过九江至于东陵者也。西南流,水积为湖,湖西有青林山。宋太始元年,明帝遣沈攸之西代子勋,伐栅青山,睹一童子甚丽,问伐者曰:取此何为?答欲讨贼。童子曰:下旬当平,何劳伐此?在众人之中,忽不复见。故谓之青林湖。湖有鲫鱼,食之肥美,辟寒暑,湖水西流,谓之青林水。又西南,历寻阳,分为二水:一水东流,通大雷。一水西南流注于江,《经》所谓利水也。右对马头岸,自富口迄此五十余里,岸阻江山。


卷三十六  青衣水、桓水、若水、沫水、延江水、存水、温水 
青衣水出青衣县西蒙山,东与沫水合也。
县故青衣羌国也。《竹书纪年》梁惠成王十年,瑕阳人自秦道岷山、青衣水来归。汉武帝天汉四年,罢沈黎郡,分两部都尉,一治青衣,主汉民。公孙述之有蜀也,青衣不服,世祖嘉之,建武十九年以为郡。安帝延光元年,置蜀郡属国都尉,青衣王子心慕汉制,上求内附。顺帝阳嘉二年,改曰汉嘉。嘉得此良臣也。县有蒙山,青衣水所发,东径其县,与沫水会于越巂郡之灵关道。青衣水又东,邛水注之,水出汉嘉严道邛来山,东至蜀郡临邛县,东入青衣水。至犍为南安县,入于江。
青衣水径平乡,谓之平乡江。《益州记》曰:平乡江东径峨眉山,在南安县界,去成都南千里。然秋日清澄,望见两山相峙,如蛾眉焉。青衣水又东流注于大江。
桓水出蜀郡岷山,西南行羌中,入于南海。
《尚书·禹贡》:岷、.既艺,沱、潜既道,蔡、蒙旅平,和夷底绩。
郑玄曰:和上,夷所居之地也,和读曰桓。《地理志》曰:桓水出蜀郡蜀山西南行羌中者也。《尚书》又曰:西倾因桓是来。马融、王肃云:西治倾山,惟因桓水是来,言无他道也。余按《经》据《书》岷山、西倾,俱有桓水。桓水出西倾山,更无别流,所导者惟斯水耳。浮于潜、汉而达江、沔。故《晋地道记》曰:梁州南至桓水,西抵黑水,东限扞关。今汉中、巴郡、汶山、蜀郡、汉嘉、江阳、朱提、涪陵、阴平、广汉、新都、梓潼、犍为、武都、上庸、魏兴、新城,皆古梁州之地。自桓水以南为夷,《书》所谓和夷底绩也。然所可当者,惟斯水与江耳。桓水盖二水之别名,为两川之通称矣。郑玄注《尚书》,言织皮谓西戎之国也。西倾,雍州之山也,雍、戎二野之间,人有事于京师者,道当由此州而来。桓是陇坂名,其道盘桓旋曲而上,故名曰桓,是今其下民谓是坂曲为盘也,斯乃玄之别致,恐乖《尚书》因桓之义,非浮潜入渭之文。余考校诸书,以具闻见。今略缉综川流沿注之绪,虽今古异容,本其流俗,粗陈所由。然自西倾至葭萌,人于西汉,即郑玄之所谓潜水者也。自西汉溯流而届于晋寿界。沮漾枝津南,历冈穴,迤逦而接汉,沿此入漾,《书》所谓浮潜而逾沔矣。历汉川至南郑县,属于褒水。溯褒暨于衙岭之南溪水,枝灌于斜川届于武功,而北达于渭水。此乃水陆之相关,川流之所经,复不乖《禹贡》入渭之宗,实符《尚书》乱河之义也。
若水出蜀郡旄牛徼外,东南至故关为若水也。
《山海经》曰:南海之内,黑水之间,有木名曰若木,若水出焉。又云:灰野之山有树焉,青叶赤华,厥名若木。生昆仑山西,附西极也。《淮南子》曰:若木在建木西,木有十华,其光照下地。故屈原《离骚·天问》曰:羲和未阳,若华何光?是也。然若木之生,非一所也。黑水之间,厥木所植,水出其下,故水受其称焉。若水沿流,间关蜀土,黄帝长子昌意,德劣不足绍承大位,降居斯水,为诸侯焉。娶蜀山氏女,生颛顼于若水之野。有圣德,二十登帝位,承少皞金官之政,以水德宝历矣。若水东南流,鲜水注之,一名州江。大度水出徼外,至旄牛道,南流入于若水。又径越巂大莋县入绳。绳水出徼外。《山海经》曰:巴遂之山,绳水出焉,东南流分为二水:其一水,枝流东出,径广柔县,东流注于江。其一水,南径旄牛道,至大莋与若水合,自下亦通谓之为绳水矣。作,夷也,汶山曰夷,南中曰昆弥,蜀曰邛,汉嘉、越巂曰莋,皆夷种也。
南过越巂邛都县西,直南至会无县,淹水东南流注之。邛都县,汉武帝开邛巂置之。县陷为池,今因名为邛池,南人谓之邛河。河中有蚌巂山,应劭曰:有巂水言越此水以章休盛也。后复后叛,元鼎六年,汉兵自越巂水伐之,以为越巂郡,治邛都县。王莽遣任贵为领戎大尹,守之,更名为集巂也。县故邛都国也。越巂水即绳、若矣,似随水地而更名矣。又有温水,冬夏常热,其源可燖鸡豚。下汤沐洗,能治宿疾。昔李骧败李流于温水是也。若水又径会无县,县有骏马河,水出县东高山。山有天马径,厥迹存焉。马日行千里,民家马牧之山下,或产骏驹,言是天马子。河中有贝子胎铜,以羊祠之,则可取也。又有孙水焉。水出台高县,即台登县也。孙水一名白沙江,南流径邛都县,司马相如定西南夷,桥孙水,即是水也。又南至会无,入若水。若水又南径云南郡之遂久县,青岭水入焉。水出青蛉县西,东径其县下,县以氏焉,有石猪圻,长谷中有石猪,子母数千头。长老传言,夷昔牧此,一朝化为石,迄今夷人不敢往牧。贪水出焉。青蛉水又东,注于绳水。绳水又径三绛县西,又径姑复县北,对三蜂县,淹水注之。三绛一曰小会无,故《经》曰:淹至会无注若水。若水又与母血水合。水出益州郡弄栋县东农山母血谷,北流径三绛县南,北入绳。绳水又东,涂水注之。水出建宁郡之牧靡南山。县、山并即草以立名。山在县东北乌句山南五百里,山生牧靡,可以解毒。百卉方盛,鸟多误食,乌喙口中毒,必急飞往牧靡山,啄牧靡以解毒也。涂水导源腊谷,西北流至越巂入绳,绳水又径越巂郡之马湖县,谓之马湖江。又左合卑水,水出卑水县,而东流注马湖江也。
又东北至犍为朱提县西,为泸江水。
朱提山名也。应劭曰:在县西南,县以氏焉。犍为属国也,在郡南千八百许里。建安二十年,立朱提郡,郡治县故城。郡西南二百里,得所绾堂琅县,西北行上高山,羊肠绳屈八十余里,或攀木而升,或绳索相牵而上,缘涉者若将阶天。故袁休明《巴蜀志》云:高山嵯峨,岩石磊落,倾侧蒙回,下临峭壑,行者扳缘,牵援绳索。三蜀之人,及南中诸郡,以为至险。有泸津,东去县八十里,水广六七百步,深十数丈,多瘴气,鲜有行者。晋明帝太宁二年,李骧等侵越巂,攻台登县,宁州刺史王逊遣将军姚岳击之,战于堂琅,骧军大败,岳追之,至泸水,赴水死者千余人。逊以岳等不穷追,怒甚,发上冲冠,帢裂而卒。按永昌郡有兰仓水,出西南博南县,汉明帝永平二年置。博南,山名也,县以氏之。其水东北流径博南山,汉武帝时,通博南山道,渡兰仓津,土地绝远,行者昔之。歌曰:汉德广,开不宾,渡博南,越仓津,渡兰仓,为作人!山高四十里。兰仓水出金沙,越人收以为黄金。又有珠光穴,穴出光珠,又有琥珀、珊瑚、黄、白、青珠也。兰仓水又东北径不韦县,与类水合,水出巂唐县,汉武帝置。类水西南流。曲折又北流,东至不韦县,注兰仓水。又东与禁水合。水自永昌县而北径其郡西,水左右甚饶犀象,山有钩蛇,长七八丈,尾末有岐,蛇在山涧水中,以尾钩岸上人牛食之。此水傍瘴气特恶。气中有物,不见其形,其作有声,中木则折,中人则害,名曰鬼弹。惟十一月、十二月差可渡,正月至十月,径之无不害人。故郡有罪人,徙之禁旁,不过十日皆死也。禁水又北注泸津水,又东径不韦县北而东北流,两岸皆高山,数百丈,泸峰最为杰秀,孤高三千余丈。是山于晋太康中崩,震动郡邑。水之左右,马步之径裁通,而时有瘴气,三月、四月,径之必死,非此时犹令人闷吐。五月以后,行者差得无害。故诸葛亮表言:五月渡泸,并日而食,臣非不自惜也,顾王业不可偏安于蜀故也。《益州记》曰:泸水源出曲罗巂下三百里,曰泸水。两峰有杀气,暑月旧不行,故武侯以夏渡为艰。泸水又下合诸水而总其目焉,故有庐江之名矣。自朱提至道有水步道,水道有黑水、羊官水,至险难。三津之阻,行者苦之,故俗为之语曰:栖溪赤水,盘蛇七曲,盘羊乌栊,气与天通。看都濩泚,住柱呼伊,庲降贾子,左担六里。又有牛叩头、马搏颊坂,其艰险如此也。
又东北至道县,入于江。
若水至道,又谓之马湖江,绳水、泸水、孙水、淹水、大渡水随决入而纳通称。是以诸书录记群水,或言入若,又言注绳,亦咸言至道入江,正是异水沿注,通为一律,更无别川可以当之。水有孝子石,昔县人有隗叔通者,性至孝,为母给江膂水,天为出平石,至江膂中。今犹谓之孝子石,可谓至诚发中而休应自天矣。
沫水出广柔微外,县有石纽乡,禹所生也。今夷人共营之,地方百里,不敢居牧。有罪逃野,捕之者不逼,能藏三年,不为人得,则共原之,言大禹之神所祐之也。东南过旄牛县北,又东至越巂灵道县,出蒙山南,灵道县一名灵关道,汉制:夷狄曰道。县有铜山,又有利慈渚。晋太始九年,黄龙二见于利慈池。县令董玄之率吏民观之,以白刺史王濬,濬表上之晋朝,改护龙县也。沫水出岷山西,东流过汉嘉郡,南流冲一高山,山上合下开,水径其间,山即蒙山也。
东北与青衣水合,《华阳国志》曰:二水于汉嘉青衣县东,合为一川,自下亦谓之为青衣水。沫水又东,径开刊县,故平乡也,晋初置。沫水又东径临邛南,而东出于江原县也。
东入于江。
昔沫水自蒙山至南安西溷崖,水脉漂疾,破害舟船,历代为患。蜀郡太守李冰发卒,凿平溷崖。河神赑怒,冰乃操刀入水与神斗,遂平溷崖,通正水路,开处即冰所穿也。延江水出健为南广县,东至牂柯鄨县,又东屈北流,鄨县故健为郡治也,县有犍山,晋建兴元年,置平夷郡。县有鄨水,出鄨邑西不狼山,东与温水合。温水一曰暖水,出犍为符县,而南入鄨水,鄨水亦出符县,南与温水会,阚駰谓之阚水,俱南入鄨水。鄨水于其县而东注延江水。延江水又与汉水合,水出犍为、汉阳道山闟谷,王莽之新通也。东至鄨邑入延江水也。
至巴郡涪陵县注更始水。
更始水,即延江枝分之始也。延江水北入涪陵水,涪陵水出县东,故巴郡之南鄙,王莽更名巴亭,魏武分邑立为涪陵郡。张堪为县,会公孙述击堪,同心义士选习水者,筏渡堪于小别江,即此水也。其水北至枳县入江,更始水东入巴东之南浦县,其水注引渎口石门,空岫阴深,邃涧暗密,倾崖上合,恒有落势,行旅避瘴,时有经之,无不危心于其下。又谓之西乡水,亦谓之西乡溪,溪水间关二百许里,方得出山。又通波注远,复二百余里,东南入迁陵县也。
又东南至武陵西阳县,入于西水。
《武陵先贤传》曰:潘京世长为郡主簿,太守赵伟甚器之,问京:贵郡何以名武陵?京答曰:鄙郡本名义陵,在辰阳县界,与夷相接,数为所破。光武时移抬东山之上,遂尔易号。《传》曰:止戈为武。《诗》云高平曰陵,于是名焉。西水北岸有黚阳县,许慎曰:温水南入黚。盖鄨水以下津流沿注之通称也。故县受名焉。西乡溪口在迁陵县故城上五十里,左合酉水,西水又东际其故城北,又东径西阳故县南而东出也。两县相去,水道可四百许里,于酉阳合也。西水东南至沅陵县,入于沅。
存水出犍为县,王莽之孱也。益州大姓雍闿反,结垒于山,系马柳柱,柱生成林,今夷人名曰雍无粱林,梁,夷言马也。存水自县东南流,径牧靡县北,又东径且兰县北而东南出也。
东南至郁林定周县,为周水。
存水又东,径牂柯郡之毋敛县北,而东南与毋敛水合,水首受牂柯水,东径毋敛县为毋敛水,又东注于存水。存水又东径郁林定周县,为周水,盖水变名也。又东北至潭中县,注于潭。
温水出牂柯夜郎县,县故夜郎侯国也,唐蒙开以为县,王莽名曰同亭矣。温水自县西北流,径谈藁,与迷水合。水西出益州郡之铜濑县谈虏山,东径谈藁县,右注温水。温水又西径昆泽县南,又径味县,县故滇国都也。诸葛亮讨平南中,刘禅建兴三年,分益州郡,置建宁郡于此。水侧皆是高山,山水之间,悉是木耳夷居,语言不同,嗜欲亦异,虽曰山居,土差平和而无瘴毒。温水又西南径滇池城,池在县西,周三百许里,上源深广,下流浅狭,似如倒流,故曰滇池也。长老传言:池中有神马,家马交之,则生骏驹,日行五百里。晋太元十四年,宁州刺史费统言:晋宁郡滇池县两神马,一白一黑,盘戏河水之上。有滇州,元封二年立益州郡,治滇池城。刘禅建宁郡也。温水又西会大泽,与叶榆仆水合。温水又东南,径牂柯之毋单县。建兴中,刘禅割属建宁郡。桥水注之。水上承俞元之南池。县治龙池洲,周四十七里。一名河水,与邪龙分浦,后立河阳郡,治河阳县,县在河源洲上,又有云平县,并在洲中。桥水东流至毋单县,注于温。温水又东南,径兴古郡之毋棳县东。王莽更名有棳也。与南桥水合。水出县之桥山,东流,梁水注之。梁水上承河水于俞元县而东南径兴古之胜休县,王莽更名胜棳县。梁水又东径毋棳县,左注桥水。桥水又东,注于温。温水又东南,径律高县南,刘禅建兴三年,分牂柯置兴古郡,治温县。《晋书地道记》治此。温水又东南,径梁水郡南,温水上合梁水,故自下通得梁水之称,是以刘禅分兴古之盢南,置郡于染本县也。温水东南,径镡封县北,又径来惟县东,而仆水右出焉。
又东至郁林广郁县,为郁水。
秦桂林郡也。汉武帝元鼎六年,更名郁林郡。王莽以为郁平郡矣。应劭《地理风俗记》曰:《周礼》郁人掌器,凡祭醊宾客之事,和郁鬯以实樽彝。郁,芳草也,百草之华,煮以合酿黑黍,以降神者也。或说,今郁金香是也。一曰郁人所贡,因氏郡矣。温水又东径增食县,有文象水注之,其水导源牂柯句町县。应劭曰:故句町国也。王莽以为从化。文象水、蒙水与卢惟水、来细水、伐水,并自县东,历广郁至增食县,注于郁水也。
又东至领方县,东与斤南水合。
县有朱涯水,出临尘县,东北流,泿水注之。水源上承牂柯水,东径增食县而下注朱涯水。朱涯水又东北径临尘县,王莽之监尘也。县有斤南水、侵离水,并径临尘,东入领方县,流注郁水。
东北入于郁。
郁水即夜郎豚水也。汉武帝时,有竹王兴于豚水,有一女子,浣于水滨,有三节大竹,流入女子足间,推之不去。闻有声,持归破之,得一男儿。遂雄夷濮,氏竹为姓。所捐破竹,于野成林,今竹王祠竹林是也。王尝从人止大石上,命作羹。从者白无水。王以剑击石出水,今竹王水是也。后唐蒙开牂柯,斩竹王首,夷獠咸怨,以竹王非血气所生,求为立祠。帝封三子为侯,及死,配父庙,今竹王三郎祠,其神也。豚水东北流,径谈藁县,东径牂柯郡且兰县,谓之牂柯水。水广数里,县临江上,故且兰侯国也。一名头兰,牂柯郡治也。楚将庄蹻溯沅伐夜郎,椓牂柯系船,因名且兰为牂柯矣。汉武帝元鼎六年开。王莽更名同亭。有柱浦关。牂柯亦江中两山名也。左思《吴都赋》云吐浪牂柯者也。元鼎五年,武帝伐南越,发夜郎精兵,下牂柯江,同会番禺是也。牂柯水又东南径毋敛县西,毋敛水出焉。又东,驩水出焉。又径郁林广郁县为郁水。又东北径领方县北,又东径布山县北,郁林郡治也。吴陆绩曰:从今以去六十年,车同轨,书同文。至大康元年,晋果平吴。又径中留县南,与温水合,又东入阿林县,潭水性之。水出武陵郡镡成县玉山,东流径郁林郡潭中县,周水自西南来注之。潭水又东南流,与刚水合。水西出样柯毋敛县,王莽之有敛也,东至潭中入潭。潭水又径中留县东,阿林县西,右入郁水。《地理志》曰:桥水东至中留入潭。又云:领方县又有桥水。余诊其川流,更无殊津,正是桥、温乱流,故兼通称。作者咸言至中留入潭,潭水又得郁之兼称,而字当为温,非桥水也,盖书字误矣。郁水右则留水注之,水南出布山县下,径中留入郁,郁水东径阿林县,又东径猛陵县,泿水注之。又东径苍梧广信县,漓水注之。郁水又东,封水注之。水出临贺郡冯乘县西,谢沭县东界牛屯山。亦谓之临水,东南流径萌渚峤西,又东南左合娇水。庾仲初云:水出萌渚峤南流入于临。临水又径临贺县东,又南至郡,左会贺水,水出东北兴安县西北罗山,东南流径兴安县西。盛弘之《荆州记》云:兴安县水边有平石,上有石履,言越王渡溪,脱履于此。贺水又西南流,至临贺郡东,右注临水。郡对二水之交会,故郡、县取名焉。临水又西南流,径郡南,又西南径封阳县东,为封溪水。故《地理志》曰:县有封水。又西南流入广信县,南流注于郁水,谓之封溪水口者也。郁水又东径高要县,牢水注之。水南出交州合浦郡,治合浦县,汉武帝元鼎六年平越所置也。王莽更名曰桓合。县曰桓亭,孙权黄武七年,改曰珠官郡。郡不产谷,多采珠宝,前政烦苛,珠徙交趾。会稽孟伯周为守,有惠化。去珠复还。郡统临允县,王莽之大允也。牢水自县北流径高要县,入于郁水。郁水南径广州南海郡西,泿水出焉。又南,右纳西随三水,又南径四会浦水,上承日南郡卢容县西古郎究浦,内漕口,马援所漕。水东南曲屈通郎湖,湖水承金山郎究,究水北流,左会卢容、寿泠二水。卢容水出西南区粟城南高山,山南长岭连接天障岭西。卢容水凑隐山绕西卫北,而东径区粟城北,又东,右与寿泠水合,水出寿泠县界。魏正始九年,林邑进侵至寿泠县,以为疆界,即此县也。寿冷县以水凑,故水得其名。东径区粟故城南。考古《志》并无区粟之名。应劭《地理风俗记》曰:日南故秦象郡,汉武帝元鼎六年开日南郡,治西卷县。《林邑记》曰:城去林邑步道四百余里。《交州外域记》曰:从日南郡南去,到林邑国,四百余里。准径相符,然则城故西卷县也。《地理志》曰:水入海,有竹可为杖。王莽更之曰日南亭。《林邑记》曰:其城治二水之间,三方际山,南北瞰水。东西涧浦,流凑城下。城西折十角,周围六里,一百七十步,东西度六百五十步,砖城二丈,上起砖墙一丈,开方隙孔。砖上倚板,板上五重层阁,阁上架屋,屋上架楼,楼高者七八丈,下者五六丈。城开十三门,凡宫殿南向,屋宇二千一百余间。市居周绕,阻峭地险。故林邑兵器战具,悉在区粟。多城垒,自林邑王范胡达始。秦余徙民,染同夷化,日南旧风,变易俱尽,巢栖树宿,负郭接山,榛棘蒲薄,腾林拂云,幽烟冥缅,非生人所安。区粟建八尺表,日影度南八寸,自此影以南,在日之南,故以名郡。望北辰星,落在天际,日在北,故开北户以向日,此其大较也。范泰《古今善言》曰:日南张重,举计入洛。正旦大会,明帝问:日南郡北向视日邪?重曰:今郡有云中、金城者,不必皆有其实。日亦俱出于东耳。至于风气暄暖,日影仰当,官民居止随情,面向东西南北,回背无定。人性凶悍,果于战斗,便山习水,不闲平地。古人云:五岭者,天地以隔内外,况绵途于海表,顾九岭而弥邈,非复行路之径阻,信幽荒之冥域者矣。寿泠水自城南,东与卢容水合,东注郎究,究水所积,下潭为湖,谓之郎湖浦口。有秦时象郡,墟域犹存。自湖南望,外通寿泠,从郎湖入四会浦。元嘉二十年,以林邑顽凶,历代难化,恃远负众,慢威背德。北宝既臻,南金阙贡,乃命偏将,与龙骧将军、交州刺史檀和之,陈兵日南,修文服远。二十三年,扬旌从四会浦口,入郎湖。军次区粟,进逼围城,以飞梯云桥,悬楼登垒。钲鼓大作,虎士电怒,风烈火扬,城摧众陷,斩区粟王范扶龙首,十五以上,坑截无赦,楼阁雨血,填尸成观。自四会南入,得卢容浦口。晋太康三年,省日南郡属国都尉,以其所统卢容县,置日南郡,及象林县之故治。《晋书地道记》曰:郡去卢容浦口二百里,故秦象郡,象林县治也。永和五年,征西桓温遣督护滕畯、率交、广兵伐范文于旧日南之卢容县,为文所败,即是处也。退次九真,更治兵,文被创死,子佛代立。七年,畯与交州刺史杨平,复进军寿泠浦,入顿郎湖,讨佛于日南故治。佛蚁聚连垒五十余里,畯、平破之。佛逃窜川薮,遣大帅面缚请罪军门。遣武士陈延劳佛,与盟而还。康泰《扶南记》曰:从林邑至日南卢容浦口,可二百余里。从口南发,往扶南诸国,常从此口出也。故《林邑记》曰:尽纮沧之徼远,极流服之先外,地滨沧海,众国津径。郁水南通寿泠,即一浦也。浦上承交趾郡南都官塞浦。《林邑记》曰:浦通铜鼓、外越、安定、黄冈心口,盖藉度铜鼓,即骆越也。有铜鼓,因得其名。马援取其鼓以铸铜马。至凿口,马援所凿,内通九真、浦阳。《晋书地道记》九德郡有浦阳县。《交州记》曰:凿南塘者,九真路之所经也,去州五百里。建武十九年,马援所开。《林邑记》曰:外越纪粟。望都纪粟出浦阳,渡便州至典由,渡故县至咸驩。咸驩属九真。咸驩已南,獐麂满冈,鸣咆命畴,警啸聒野。孔雀飞翔,蔽日笼山。渡治口,至九德。按《晋书地道记》有九德县,《交州外域记》曰:九德县属九真郡,在郡之南,与日南接。蛮卢舆居其地,死,子宝纲代,孙党服从吴化,定为九德郡,又为隶之。《林邑记》曰:九德,九夷所极,故以名郡。郡名所置,周越裳氏之夷国。《周礼》九夷,远极越裳。白雉象牙,重九译而来。自九德通类口,水源从西北远荒,径宁州界来也。九德浦内径越裳究、九德究、南陵究。按《晋书地道记》,九德郡有南陵县,晋置也。竺枝《扶南记》,山溪濑中谓之究。《地理志》曰:郡有小水五十二,并行大川,皆究之谓也。《林邑记》曰:义熙九年,交趾太守杜慧度造九真水口,与林邑王范胡达战,擒斩胡达二子,虏获百余人,胡达遁。五月,慧度自九真水历都粟浦,复袭九真。长围跨山,重栅断浦,驱象前锋,接刃城下,连日交战,杀伤乃退。《地理志》曰:九真郡,汉武帝元鼎六年开,治胥浦县。王莽更之曰驩成也。《晋书地道记》曰:九真郡有松原县。《林邑记》曰:松原以西,鸟鲁驯良,不知畏弓。寡妇孤居,散发至老。南移之岭,崪不逾仞。仓庚怀春于其北,翡翠熙景乎其南,虽嘤欢接响,城隔殊非,独步难游,俗姓涂分故也。自南陵究出于南界蛮,进得横山。太和三年,范文侵交州,于横山分界。度比景庙,由门浦至古战湾,吴赤乌十一年,魏正始九年,交州与林邑于湾大战,初失区粟也。渡卢容县,日南郡之属县也。自卢容县至无变越烽火,至比景县,日中,头上景当身下,与景为比。如淳曰:故以比景名县。阚駰曰:比读荫庇之庇,景在己下,言为身所庇也。《林邑记》曰:渡比景至朱吾,朱吾县浦,今之封界。朱吾以南,有文狼人,野居无室宅,依树止宿,食生鱼肉,采香为业,与人交市,若上皇之民矣。具南有文狼究,下流径通。《晋书地道记》曰:朱吾县属日南郡,去郡二百里。此县民,汉时不堪二千石长吏调求,引屈都乾为国。《林邑记》曰:屈都,夷也。朱吾浦内通无劳湖,无劳究水通寿泠浦。元嘉元年,交州刺史阮弥之征林邑,阳迈出婚,不在。奋威将军阮谦之领七千人,先袭区粟,已过四会,未入寿泠,三日三夜,无顿止处。凝海直岸,遇风大败。阳迈携婚都部伍三百许船,来相救授。谦之遭风,余数船舰,夜于寿泠浦里相遇,暗中大战,谦之手射阳迈柁工,船败纵横。昆仑单舸,接得阳迈。谦之以风溺之余,制胜理难。自此还渡寿泠,至温公浦。升平三年,温放之征范佛于湾分界阴阳圻,入新罗湾,至焉下,一名阿贲浦,入彭龙湾,隐避风波,即林邑之海渚。元嘉二十三年,交州刺史檀和之破区粟,已,飞旍盖海,将指典冲,于彭龙湾上鬼塔,与林邑大战,还渡典冲。林邑入浦,令军大进,持重故也。浦西即林邑都也,治典冲,去海岸四十里。处荒流之檄表,国越裳之疆南,秦、汉象郡之象林县也,东滨沧海,西际徐狼,南接扶南,北连九德,后去象林、林邑之号。建国起自汉末,初平之乱,人怀异心,象林功曹姓区,有子名逵,攻其县,杀令,自号为王。值世乱离,林邑遂立,后乃袭代,传位子孙。三国鼎争,未有所附。吴有交土,与之邻接,进侵寿泠,以为疆界。自区逵以后,国无文史,失其纂代,世数难详,宗胤灭绝,无复种裔。外孙范熊代立,人情乐推。后熊死,子逸立,有范文,日南西卷县夷帅范椎奴也。文为奴时,山涧牧羊,于涧水中,得两鲤鱼,隐藏挟归,规欲私食。郎知检求,文大惭惧,起托云:将而石还,非为鱼也。郎至鱼所,见是两石,信之而去。文始异之。石有铁,文入山中,就石冶铁,锻作两刀。举刀向鄣,因祝曰:鲤鱼变化,冶石成刀,斫石鄣破者,是有神灵,文当得此,为国君王。斫不入者,是刀无神灵。进斫石鄣,如龙渊、干将之斩芦藁,由是人情渐附。今斫石尚在,鱼刀犹存,传国子孙,如斩蛇之剑也。椎尝使文远行商贾,北到上国,多所闻见,以晋愍帝建兴中,南至林邑,教王范逸,制造城他,缮治戎甲,经始廓略。王爱信之,使为将帅,能得众心。文谗王诸子,或徙或奔,王乃独立。成帝咸和六年死,无胤嗣。文迎王子于外国,海行取水,置毒子中。饮而杀之,遂胁国入,自立为王。取前王妻妾,置高楼土,有从己者,取而纳之,不从己者,绝其饮食而死。《江东旧事》云:范文,本扬州人,少被掠为奴,卖堕交州。年十五六,遇罪当得杖,畏怖因逃,随林邑贾人渡海远去,没入于王,大被幸爱。经十余年,王死,文害王二子,诈杀侯将,自立为王,威加诸国。或夷椎蛮语,口食鼻饮,或雕面镂身,狼裸种,汉魏流赭,威为其用。建元二年,攻日南、九德、九真,百姓奔迸,千里无烟,乃还林邑。林邑西去广州二千五百里,城西南角高山长岭,连接天鄣岭,北接涧。大源淮水出郍郍远界,三重长洲,隐山绕西,卫北回东,其岭南开涧,小源淮水出松根界,上山壑流,隐山绕南曲街回,东合淮流,以注典冲。其城西南际山,东北瞰水,重堑流浦,周绕城下。东南堑外,因傍薄城,东西横长,南北纵狭,北边西端,回折曲入。城周围八里一百步,砖城二丈,上起砖墙一丈,开方隙孔,砖上倚板,板上层阁,阁上架屋,屋上构楼。高者六七丈,下者四五丈。飞观鸱尾,迎风拂云,缘山瞰水,骞翥嵬崿。但制造壮拙。稽古夷俗,城开四门,东为前门,当两淮渚滨,于曲路有古碑,夷书铭赞前王胡达之德。西门当两重堑,北回上山,山西即淮流也。南门度两重堑,对温公垒。升平二年,交州刺史温放之,杀交趾太守杜宝别驾阮朗,遂征林邑,水陆累战,佛保城自守,重求请服,听之。今林邑东城南五里,有温公二垒是也。北门滨淮,路断不通。城内小城,周围三百二十步,合堂瓦殿,南壁不开,两头长屋,脊出南北,南拟背日。西区城内,石山顺淮面阳,开东向殿,飞檐鸱尾,青琐丹墀,榱题桷椽,多诸古法。阁殿上柱,高城丈余五,牛屎为埿。墙壁青光回度,曲掖绮牖,紫窗椒房,嫔媵无别,宫观,路寝,永巷,共在殿上,临踞东轩,径与下语。子弟臣侍,皆不得上。屋有五十余区,连甍接栋,檐宇相承。神祠鬼塔,小大八庙,层台重树,状似佛刹。郭无市里,邑寡人居,海岸萧条,非生民所处,而首渠以永安,养国十世,岂久存哉?元嘉中,檀和之征林邑,其王阳迈,举国夜奔窜山薮。据其城邑,收宝巨亿。军还之后,阳迈归国,家国荒殄,时人靡存,踌蹰崩擗,愤绝复苏,即以元嘉二十三年死。初,阳迈母怀身,梦人铺阳迈金席,与其儿落席上,金光色起、昭晣艳曜。华俗谓上金为紫磨金,夷俗谓上金为阳迈金,父胡达死,袭王位,能得人情,自以灵梦,为国祥庆。其太子初名咄,后阳迈死,咄年十九,代立,慕先君之德,复改名阳迈。昭穆二世,父子共名,知林邑之将亡矣。其城隍堑之外,林棘荒蔓,榛梗冥郁,藤盘筀秀,参错际天。其中香桂成林,气清烟澄。桂父,县人也,栖居此林,服桂得道。时禽异羽,翔集间关,兼比翼鸟,不比不飞,鸟名归飞,鸣声自呼,此恋乡之思孔悲,桑梓之敬成俗也,豫章俞益期,性气刚直,不下曲俗,容身无所,远适在南,《与韩康伯书》曰:惟槟榔树,最南游之可观,但性不耐霜,不得北植,不遇长者之目,令人恨深。尝对飞鸟恋土,增思寄意。谓此鸟其背青,其腹赤,丹心外露,鸣情未达,终日归飞,飞不十千,路余万里,何由归哉!九真太守任延,始教耕犁,俗化交土,风行象林,知耕以来,六百余年,火耨耕艺,法与华同,名白田,种白谷,七月火作,十月登熟;名赤田,种赤谷,十二月作,四月登熟,所谓两熟之稻也。至于草甲萌芽,谷月代种,穜早晚,无月不秀,耕耘功重,收获利轻,熟速故也。米不外散,恒为丰国。桑蚕年八熟茧。《三都赋》所谓八蚕之绵者矣。其崖小水,常吐飞溜,或雪霏沙涨,清寒无底,分溪别壑,津济相通。其水自城东北角流,水上悬起高桥,渡淮北岸,即彭龙、区粟之通逵也。檀和之东桥大战,阳迈被创落象,即是处也。其水又东南流径船官口,船官川源徐狼外,夷皆裸身,男以竹筒掩体,女以树叶蔽形,外名狼,所谓裸国者也。虽习俗裸袒,犹耻无蔽,惟依瞑夜,与人交市睹中,奥金便知好恶,明朝晓看,皆如其言。自此外行,得至扶南。按竺枝《扶南记》曰:抉南去林邑四千里,水步道通。檀和之令军入邑浦,据船官口城六里者也。自船官下注大浦之东湖,大水连行,潮上西流。潮水日夜长七八尺,从此以西,朔望并潮,一上七日,水长丈六七。六日之后,日夜分为再潮,水长一二尺。春夏秋冬,厉然一限,高下定度,水无盈缩,是为海运,亦曰象水也,又兼象浦之名。《晋功臣表》所谓金潾清径,象渚澄源者也。其川浦渚,有水虫弥微,攒木食船,数十日坏。源潭湛濑,有鲜鱼,色黑,身五丈,头如马首,伺人入水,便来为害。《山海经》曰:离耳国、雕题国皆在郁水南。《林邑记》曰:汉置九郡,儋耳与焉。民好徒跣,耳广垂以为饰,虽男女亵露,不以为羞。暑亵薄日,自使人黑,积习成常,以黑为美。《离骚》所谓玄国矣。然则信耳,即离耳也。王氏《交广春秋》曰:朱崖、儋耳二郡,与交州俱开,皆汉武帝所置,大海中,南极之外,对合浦徐闻县,清朗无风之日,径望朱崖州,如囷廪大。从徐闻对渡,北风举帆,一日一夜而至。周回二千余里,径度八百里。人民可十万余家,皆殊种异类。被发雕身,而女多姣好,白皙,长发美鬓。犬羊相聚,不服德教。儋耳先废,朱崖数叛,元帝以贾捐之议罢郡。杨氏《南裔异物志》曰:儋耳、朱崖俱在海中,分为东蕃。故《山海经》曰:在郁水南也。郁水又南,自寿泠县注于海。昔马文渊积石为塘,达于象浦,建金标为南极之界。俞益期《笺》曰:马丈渊立两铜柱于林邑岸北,有遗兵十余家不反,居寿泠岸南,而对铜柱。悉姓马,自婚姻,今有二百户。交州以其流寓,号曰马流。言语饮食,尚与华同。山川移易,铜柱今复在海中,正赖此民以识故处也。《林邑记》曰:建武十九年,马援树两铜柱于象林南界,与西屠国分,汉之南疆也。土人以之流寓,号曰马流,世称汉子孙也。《山海经》曰:郁水出象郡而西南注南海,入须陵东南者也。应劭曰:郁水出广信东入海。言始或可,终则非矣。


卷三十七  淹水、叶榆水、夷水、油水、澧水、沅水、泿水 
淹水出越巂遂久县徼外,吕忱曰:淹水,一曰复水也。
东南至青蛉县。
县有禺同山,其山神有金马、碧鸡,光景倏忽,民多见之。汉宣帝遣谏大夫王褒祭之,欲至其鸡、马,褒道病而卒,是不果焉。王褒《碧鸡颂》曰:敬移金精神马,缥缥碧鸡。故左太冲《蜀都赋》曰:金马骋光而绝影,碧鸡倏忽而耀仪。又东过姑复县南,东入于若水。
淹水径县之临池泽,而东北径云南县西,东北注若水也。益州叶榆河,出其县北界,屈从县东北流,县故滇池叱榆之国也,汉武帝元封二年,使唐蒙开之,以为益州郡。郡有叶榆县,县西北八十里,有吊鸟山,众鸟千百为群,其会呜呼啁哳,每岁七八月至,十六七日则止。
一岁六至,雉雀来吊,夜燃火伺取之。其无嗉不食,似特悲者,以为义鸟,则不取也。俗言凤凰死于此山,故众鸟来吊,因名吊鸟。县之东有叶榆泽,叶榆水所钟而为此川薮也。
过不韦县,县故九隆哀牢之国也。有牢山,其先有妇人,名沙壹,居于牢山。捕鱼水中,触沉木,若有感,因怀孕,产十子。后沉木化为龙出水,九子惊走。小子不能去,背龙而坐,龙因舐之。其母鸟语,谓背为九,谓坐为隆,因名为九隆。及长,诸兄遂相共推九隆为王。后牢山下,有一夫一妇,生十女,九隆皆以为妻,遂因孳育,皆画身像龙文,衣皆著尾。九隆死,世世不与中国通。汉建武二十三年,王遣兵来,乘革船南下,攻汉鹿茤民。鹿茤民弱小,将为所擒,于是天大震雷疾雨,南风漂起,水为逆流,波涌二百余里,革船沉没,溺死数千人。后数年,复遣六王,将万许人攻鹿茤。鹿茤王与战,杀六王,哀牢耆老共埋之。其夜,虎掘而食之。明旦,但见骸骨,惊怖引去。乃惧,谓其耆老小王曰:哀牢犯徼,自古有之。今此攻鹿爹,辄被天诛。中国有受命之王乎?何天祐之明也。即遣使诣越巂奉献。求乞内附,长保塞徼。汉明帝水平十二年,置为永昌郡。郡治不韦县,盖秦始皇徙吕不韦子孙于此,故以不韦名县。北去叶榆六百余里,叶榆水不径其县,自不韦北注者,卢仓禁水耳。叶榆水自县南,径遂久县东,又径姑复县西,与淹水合。又东南径永昌邪龙县,县以建兴三年,刘禅分隶云南,于不韦县为东北。
东南出益州界,叶榆水自邪尤县东南,径秦臧县,甫与濮水同注滇池泽于连然、双柏县也。叶榆水自泽,又东北径滇池县南,又东径同并县南,又东径漏江县,伏流山下,复出蝮口,谓之漏江。左恩《蜀都赋》曰:漏江袱流溃其阿,汩若汤谷之扬涛,沛若蒙汜之涌波。诸葛亮之平南中也,战于是水之南。叶榆水又径贲古县北,东与盘江合。盘水出律高县东南盢町山,东径梁水郡北,责古县南。水广百余步,深处十丈,甚有瘴气。朱褒之反,李恢追至盘江者也。建武十九年,伏彼将军马援上言,从贲泠出贲古,击益州。臣所将骆越万余人,便习战斗者二千兵以上,弦毒矢利,以数发,矢注如雨,所中辄死。愚以行兵,此道最便。盖承藉水利,用为神捷也。盘水又东径汉兴县。山溪之中,多生邛竹,桄榔树,树出面,而夷人资以自给。故《蜀都赋》曰:邛竹缘岭。又曰,面有桄榔,盘水北入叶榆水,诸葛亮入南,战于盘东是也。入牂柯郡西随县北为西随水,又东出进桑关,进桑县,牂柯之南部都尉治也。水上有关,故曰进桑关也。故马援言从泠水道出,进桑王国至益州贲古县,转输通利,盖兵车资运所由矣。自西随至交趾,崇山接险,水路三千里。叶榆水又东南,绝温水,而东南注于交趾。过交趾泠县北,分为五水,络交趾郡中,至南界复合为三水,东入海。《尚书大传》曰,尧南抚交趾,于《禹贡》荆州之南垂,幽荒之外,故越也。《周礼》南八蛮,雕题、交趾,有不粒食者焉。《春秋》不见于传,不通于华夏,在海岛,人民鸟语。秦始皇开越岭南,立苍梧、南海、交趾、象郡。汉武帝元鼎二年,始并百越,启七郡,于是乃置交趾刺史以督领之,初治广信,所以独不称州。时又建朔方,明已始开北垂,遂辟交趾于南,为子孙基址也。泠县,汉武帝元鼎六年开,都尉治。《交州外域记》曰:越王令二使者典主交趾、九真二郡民,后汉遣伏波将军路博德讨越王。路将军到合浦,越王令二使者赍牛百头,酒千钟,及二郡民户口簿,诣路将军。乃拜二使者为交趾、九真太守。诸雒将主民如故。交趾郡及州本治于此也,州名为交州。后朱雒将子名诗,索泠雒将女名征侧为妻,侧为人有胆勇,将诗起贼,攻破州郡,服诸雒将,皆属征侧,为王,治泠县,复交趾、九真二郡民二岁调赋。后汉遣伏波将军马援将兵讨侧,诗走入金溪究,三岁乃得。尔时西蜀并遣兵共讨侧等,悉定郡县为令长也。山多大蛇,名曰髯蛇,长十丈,围七八尺,常在树上伺鹿兽。鹿兽过,便低头绕之,有顷,鹿死,先濡令湿讫,便吞,头角骨皆钻皮出。山夷始见蛇不动时,便以大竹签签蛇头至尾,杀而食之,以为珍异。故杨氏《南裔异物志》曰:髯惟大蛇,既洪且长。采色驳荦,其文锦章。食豕吞鹿,腴成养创。宾享嘉宴,是豆是觞。言其养创之时,肪腴甚肥。搏之,以妇人衣投之,则蟠而不起,走便可得也。北二水:左水东北径望海县南,建武十九年,马援征征侧置,又东径龙渊县北,又东合南水。水自泠县东,径封溪县北。《交州外域记》曰:交趾昔未有郡县之时,土地有滩田。其田从潮水上下,民垦食其田,因名为雒民。设雒王、雒侯,主诸郡县。县多为雒将,雒将铜印青绶。后蜀王子将兵三万,来讨雒王、雒侯,服诸雒将,蜀王子因称为安阳王。后南越王尉佗举众攻安阳王。安阳王有神人,名皋通,下辅佐,为安阳王治神弩一张,一发杀三百人。南越王知不可战。却军住武宁县。按《晋太康记》县属交趾。越遣太子名始,降服安阳王,称臣事之。安阳王不知通神人,遇之无道,通便去,语王曰:能持此弩王天下,不能持此弩者亡天下。通去,安阳王有女名曰媚珠,见始端正,珠与始交通。始问珠,令取父弩视之,始见督,便盗以锯截弩讫,便逃归报南越王。南越进兵攻之,安阳王发弩,弩折,遂败。安阳王下船,径出于海。今平道县后王宫城见有故处。《晋太康地记》县属交趾。越遂服诸雒将。马援以西南治远,路径千里,分置斯县。治城郭,穿渠,通导溉灌,以利其民。县有猩猩兽,形若黄狗,又状貆纯。人面,头颜端正,善与人言,音声丽妙,如妇人好女。对语交言,闻之无不酸楚。其肉甘美,可以断谷,穷年不厌。又东径浪泊,马援以其地高,自西里进屯此。又东径龙渊县故城南,又东,左台北水,建安二十三年,立州之始,蛟龙蟠编于南、北二津,故改龙渊,以尤编为名也。卢循之寇交州也,交州刺史杜慧度,率水步军晨出南津,以火箭攻之,烧其船舰,一时溃散,循亦中矢赴水而死。于是斩之,传首京师。慧度以斩循勋,封龙编侯。刘欣期《交州记》曰:龙编县功曹左飞,曾化为虎,数月,还作吏。既言其化,亦化无不在,牛哀易虎,不识厥兄,当其革状,安知其讹变哉?其水又东径曲易县,东流注于泿郁。《经》言于郡东界,复合为三水,此其二也。其次一水东径封溪县南,又西南径西于县南,又东径赢陛县北,又东径北带县南,又东径稽徐县,径水注之。水出龙编县高山,东南流入稽徐县,注于中水。中水又东径羸陵县南,《交州外域记》曰:县本交趾郡治也。《林邑记》曰:自交趾南行,都官塞浦出焉。其水自县东径安定县,北带长江,江中有越王所铸铜船,潮水退时,人有见之者。其水又东流,隔水有泥黎城,言阿育王所筑也。又东南合南水。南水又东南,径九德郡北。《交州外域记》曰:交趾郡界有扶严究,在郡之北,隔渡一江。即是水也。江水对交趾朱县,又东径浦阳县北,又东径无切县北。建武十九年九月,马援上言:臣谨与交趾精兵万二千人,与大兵合二万人,船车大小二千艘,自入交趾,于今为盛。十月,援南入九真,至无切县,贼渠降。进入余发,渠帅朱伯弃郡,亡入深林巨薮。犀象所聚,羊牛数千头,时见象数十百为群。援又分兵入无编县,王莽之九真亭。至居风县,帅不降,并斩级数十百,九真乃靖。其水又东径句漏县,县带江水,江水对安定县,《林邑记》所谓外越、安定、纪粟者也。县江中有潜牛,形似水牛,上岸斗,角软还入江水,角坚复出。又东与北水合,又东注郁,乱流而逝矣。此其三也。平撮通称,同归郁海,故《经》有入海之文矣。
夷水出巴郡鱼复县江,夷水即佷山清江也,水色清照,十丈分沙石,蜀人见其澄清,因名清江也。昔凛君浮土舟于夷水,据捍关而王巴。是以法孝直有言:鱼复捍关,临江据水,实益州祸福之门。夷水又东径建平沙渠县,县有巫城水,南岸山道五百里,其水历县东出焉。
东南过佷山县南,夷水自沙渠县入,水流浅狭,裁得通船。东径难留城南,城即山也。独立峻绝,西面上里余,得石穴。把火行百许步,得二大石碛,并立穴中,相去一丈,俗名阴阳石。阴石常湿,阳石常燥。每水旱不调,居民作威仪服饰,往入穴中,旱则鞭阴石,应时雨多,雨则鞭阳石,俄而天晴。相承所说,往往有效,但捉鞭者不寿,人颇恶之,故不为也。东北面又有石室,可容数百人。每乱,民人室避贼,无可攻理,因名难留城也,昔巴蛮有五姓,未有君长,俱事鬼神,乃共掷剑于石穴,约能中者奉以为君。巴氏子务相乃中之。又令各乘土舟,约浮者当以为君,惟务相独浮,因共立之,是为廪君。乃乘土舟,从夷水下,至盐阳。盐水有神女,谓廪君曰:此地广大,鱼盐所出,愿留共居。廪君不许,盐神暮辄来宿,旦化为虫,群飞蔽日,天地晦瞑,积十余日。廪君因伺便射杀之,天乃开明。廪君乘土舟,下及夷城。夷城石岸险曲,其水亦曲,廪君望之而叹,出崖为崩。廪君登之。上有平石,方二丈五尺,因立城其傍而居之,四姓臣之。死,精魂化而为白虎,故巴氏以虎饮人血,遂以人祀。盐水即夷水也。又有盐石,即阳石也。盛弘之以是推之,疑即廪君所射盐神处也。将知阴石,是对阳石立名矣。事既鸿古,难为明征。夷水又东径石室,在层岩之上。石室南向,水出其下,悬崖千仞,自水上径望见。每有陟山岭者,扳木侧足而行,莫知其谁。村人骆都,小时到此室边采蜜,见一仙人,坐石床上;见都,凝瞩不转。都还招村人重往,则不复见。乡人今名为仙人室。袁山松云:都孙息尚存。夷水又东与温泉三水合。大溪南北夹岸,有温泉对注,夏暖冬热,上常有雾气,疡痍百病,浴者多愈。父老传此泉先出盐,于今水有盐气。夷水有盐水之名,此亦其一也。夷水又东径佷山县故城南,县即山名也。孟康曰:音恒,出药草。恒山今世以银为音也,旧武陵之属县。南一里即清江东注矣。南对长杨溪。溪水西南潜穴,穴在射堂村东六七里,谷中有石穴,清泉溃流三十许步,复入穴,即长杨之源也。水中有神鱼,大者二尺,小者一尺。居民钓鱼,失陈所须多少,拜而请之,拜讫,投钩饵。得鱼过数者,水辄波涌,暴风卒起,树木摧折。水侧生异花,路人欲摘者,皆当先请,不得辄取。水源东北之风井山,回曲有异势,穴口大如盆。袁山松云:夏则风出,冬则风入,春秋分则静。余往观之,其时四月中,去穴数丈,须臾寒飘卒至,六月中,尤不可当。往人有冬过者,置笠穴中,风吸之。经月还步杨溪,得其笠,则知潜通矣。其水重源显发,北流注于夷水。此水清泠,甚于大溪,纵暑伏之辰,尚无能澡其津流也。县北十余里,有神穴,平居无水,时有渴者,诚启请乞,辄得水。或戏求者,水终不出。县东十许里至平乐村,又有石穴,出清泉,中有潜龙,每至大旱,平乐左近村居,辇草秽著穴中。龙怒,须臾水出,荡其草秽,傍侧之田,皆得浇灌。从平乐顺流五六里,东亭村北山甚高峻,上合下空,空窍东西广二丈许,起高如屋,中有石床,甚整顿,傍生野韭。人往乞者,神许,则风吹别分,随偃而输,不得过越,不偃而输,辄凶。往观者去时特平,暨处自然恭肃矣。
又东过夷道县北,夷水又东径虎滩,岸石有虎像,故因以名滩也。夷水又东径釜濑,其石大者如釜,小者如刁斗,形色乱真,惟实中耳。夷水又东北,丹水注之。其源百里,出西南望州山,山形竦峻,峰秀甚高。东北白岩壁立,西南小演通行。登其顶平,可有三亩许,上有故城,城中有水,登城望见一州之境,故名望州山,俗语讹,今名武钟山。山根东有涌泉成溪,即丹水所发也。下注丹水,夭阴欲雨,辄有赤气,故名曰丹水矣。丹水又径亭下,有石穴甚深,未尝测其远近。穴中蝙蝠,大如乌,多倒悬。《玄中记》曰:蝙蝠百岁者倒悬,得而服之,使人神仙。穴口有泉,冬温夏冷,秋则入藏,春则出游。民至秋,阑断水口,得鱼,大者长四五尺,骨软肉美,异于余鱼。丹水又径其下,积而为渊。渊有神龙,每旱,村人以芮草投渊上流,鱼则多死。龙怒,当时大雨。丹水又东北流,两岸石上有虎迹甚多,或深或浅,皆悉成就自然,咸非人工。丹水又北注于夷水,水色清澈,与大溪同。夷水又东北径夷道县北而东注。
东入于江。夷水又径宜都北,东入大江,有泾渭之比,亦谓之佷山北溪。水所经皆石山,略无土岸。其水虚映,俯视游鱼,如乘空也。浅处多五色石,冬夏激素飞清,傍多茂木空岫,静夜听之,恒有清响。百鸟翔禽,哀鸣相和,巡颓浪者,不觉疲而忘归矣。
油水出武陵孱陵县西界,县有白石山,油水所出,东径其县西,与洈水合。水出高城县洈山,东径其县下,东至孱陵县,入油水也。东过其县北,县治故城,王莽更名孱陆也。刘备孙夫人,权妹也。又更修之,其城背油向泽。
又东北入于江。
油水自孱陵县之东北,径公安县西,又北流注于大江。澧水出武陵充县西历山,东过其县南,澧水自县东径临澧、零阳二县故界。水之南岸,白石双立,厥状类人,高各三十丈,周四十丈。古老传言,昔充县尉与零阳尉并论封境,因相伤害,化而为石,东标零阳,西揭充县。充县废省,临澧即其地,县即充县之故治,临侧澧水,故为县名,晋大康四年置。澧水又东,茹水注之。水出龙茹山,水色清澈,漏石分沙。庄辛说楚襄王所谓饮茹溪之流者也。茹水东注澧水。
又东过零阳县之北,澧水东与温泉水会。水发北山石穴中,长三十丈,冬夏沸涌,常若汤焉。温水南流注于澧水。澧水又东合零溪水,源南出零阳之山,历溪北注澧水。澧水又东,九渡水注之。水南出九渡山,山下有溪,又以九渡为名。山兽咸饮此水,而径越他津,皆不饮之。九渡水北径仙人楼下,傍有石形极方峭,世名之为仙楼。水自下历溪曲折,透迤倾注。行者间关,每所寨溯,山、水之号,盖亦因事生焉。九渡水又北流,注于澧水。澧水又东娄水入焉。水源出巴东界,东径天门郡娄中县北,又东径零阳县,注于澧水。澧水又东,径零阳县南,县即零溪以著称矣。澧水又径渫阳县,右会渫水。水出建平郡,东径渫阳县南,晋太康中置。渫水又左合黄水,黄水出零阳县西,北连巫山溪,出雄黄,颇有神异。采常以冬月祭祀,凿石深数丈,方得佳黄,故溪水取名焉。黄水北流注于渫水。渫水又东注澧水,谓之渫口。澧水又东径澧阳县南,南临澧水,晋太康四年立天门郡治也。吴永安六年,武陵郡嵩梁山,高峰孤竦,素壁千寻,望之苕亭,有似香炉。其山洞开,玄朗如门,高三百丈,广二百丈,门角上各生一竹,倒垂下拂,谓之天帚。孙休以为嘉样,分武陵,置天门郡。澧水又东历层步山,高秀特出。山下有峭涧,泉流所发,南流注于澧水。
又东过作唐县北。
作唐县,后汉分孱陵县置。澧水入县,左合涔水。水出西北天门郡界,南流径涔坪屯,屯堨涔水,溉田数千顷。又东南流注于澧水。澧水又东,澹水出焉。澧水又南径故郡城东,东转径作唐县南。澧水又东径南安县南,晋太康元年分孱陵立。澹水注之。水上承澧水于作唐县,东径其县北,又东注于澧,谓之澹口。王仲宣《赠士孙丈始诗》曰:悠悠澹澧者也。澧水又东,与赤沙湖水会,湖水北通江而南注澧,谓之沙口。澧水又东南注于沉水,曰澧口,盖其枝渎耳。《离骚》曰:沅有芷兮澧有兰。
又东至长沙下隽县西北,东入于江。
澧水流注于洞庭湖,俗谓之曰澧江口也。沅水出牂柯且兰县,为旁沟水,又东至镡成县,为沅水。
东过无阳县,无水出故且兰,南流至无阳故县。县对无水,因以氏县。无水又东南入沅,谓之无口。沅水东径无阳县,南临运水。水源出东南岸许山,西北径其县南,流注于熊溪。熊溪南带移山,山本在水北,夕中风雨,旦而山移水南,故山以移为名,盖亦苍梧郁州,东武怪山之类也。熊溪下注沅水。沅水又东径辰阳县,县有龙溪,水南出于龙峤之山,北流入于沅。沅水又东,滏水注之。水南出扶阳之山,北流会于沅。沅水又东,与序溪合,水出武陵郡义陵县鄜梁山,西北流径义陵县,王莽之建平县也,治序溪。其城,刘备之种归,马良出五溪,绥抚蛮夷,良率诸蛮所筑也。所治序溪,最为沃壤,良田数百顷,特宜稻,修作无废。又西北入于沅。沅水又东,合淑水,水导源椒溪,北流注沅。沅水又东径辰阳县南,东合辰水。水出县三山谷,东南流,独母水注之。水源南出龙门山,历独母溪,北入辰水。辰水又径其县北,旧治在辰水之阳,故即名焉。《楚辞》所谓夕宿辰阳者也。王莽更名会亭矣。辰水又右会沅水,名之为辰溪口。武陵有五溪,谓雄溪、溪、无溪、酉溪,辰溪其一焉。夹溪悉是蛮左所居,故谓此蛮五溪蛮也。水又径沅陵县西,有武溪,源出武山,与西阳分山。水源石上有盘瓠迹犹存矣。盘瓠者,高辛氏之畜狗也,其毛五色,高辛氏患犬戎之暴,乃募天下有能得犬戎之将军吴将军头者,妻以少女。下令之后,盘瓠遂行吴将军之首于阙下,帝大喜,未知所报。女闻之,以为信不可违,请行,乃以配之,盘瓠负女入南山,上石室中。所处险绝,人迹不至。帝悲思之,遣使不得进,经二年,生六男六女。盘瓠死,因自相夫妻。织绩木皮,染以草实,好五色衣,裁制皆有尾。其母白帝,赐以名山。其后滋蔓,号曰蛮夷。今武陵郡夷,即盘瓠之种落也。其狗皮毛,嫡孙世宝录之。武水南流注于沅。沅水又东,施水注之。水南出施山,溪源有阳欺崖,崖色纯素,望同积雪。下有二石室,先有人居处其间。细泉轻流,望川竞注,故不可得以言也。施水北流会于沅。沅水又东径沅陵县北,汉故顷侯吴阳之邑也。王莽改曰沅陆。县北枕沅水。沅水又东径县故治北,移县治。县之旧城置都尉府。因冈傍阿,势尽川陆,临沅对酉,二川之交会也。酉水导源益州巴郡临江县,故武陵之充县西源山,东南流,在无阳故具甫,又东径迁陵故县界,与西乡溪合,即延江之枝津,更始之下流,谓之西乡溪口。酉水又东径迁陵县故城北,王莽更名曰迁陆也。酉水东径酉阳故县南,县故西陵也。酉水又东径沅陵县北,又东南径潘承明垒西,承明讨五溪蛮,营军所筑也。其城跨山枕谷。西水又南注沅水,阚駰谓之受水,其水所决入,名曰酉口。沅水又径窦应明城侧,应明以元嘉初代蛮所筑也。沅水又东,溪水南出茗山,山深回险,人兽阻绝,溪水北泻沅川。沅水又东,与诸鱼溪水合,水北出诸鱼山,山与天门郡之澧阳县分岭,溪水南流会于沅。沅水又东,夷水入焉。水南出夷山,北流注沅。夷山东接壶头山,山高一百里,广圆三百里。山下水际,有新息侯马援征武溪蛮停军处,壶头径曲多险,其中纡折千滩。援就壶头,希效早成,道遇瘴毒,终没于此。忠公获谤,信可悲矣!刘澄之曰:沅水自壶头枝分,跨三十三渡,径交趾龙编县东北,入于海。脉水寻梁,乃非关究,但古人许以传疑,聊书所闻耳。
又东北过临沅县南,临沅县与沅南县分水。沅南县西有夷望山,孤竦中流,浮险四绝,昔有蛮民避寇居之,故谓之夷望也。南有夷望溪水,南出重山,远注沅。沅水又东得关下山,东带关溪,泻注沅渎。沅水又东历临沅县西,为明月池白壁湾。湾状半月,清潭镜澈,上则风籁空传,下则泉响不断。行者莫不拥楫嬉游,徘徊爱玩。沅水又东,历三石涧,鼎足均峙,秀若削成,其侧茂竹便娟,致可玩也。又东带绿萝山,绿萝蒙,颓岩临水,实钓渚渔咏之胜地,其迭响若钟音,信为神仙之所居。沅水又东径平山西,南临沅水,寒松上荫,清泉下注,栖托者不能自绝于其侧。沅水又东径临沅县南,县南临沅水,因以为名,王莽更之曰监沅也。县南有晋征士汉寿人袭玄之墓。铭,太元中车武子立。县治武陵郡下,本楚之黔中郡矣。秦昭襄王二十六年,使司马错以陇蜀军攻楚,楚割汉北与秦,至三十年,秦又取楚巫黔及江南地,以为黔中郡。汉高祖二年,割黔中故治为武陵郡,王莽更之曰建平也。南对沅南县,后汉建武中所置也。县在沅水之阴,因以沅南为名。县治故城,昔马援讨临乡所筑也。沅水又东历小湾,谓之枉渚。渚东里许,便得枉人山。山西带修溪一百余里,茂竹便娟,披溪荫渚,长川径引,远注于沅。沅水又东入龙阳县,有澹水,出汉寿县西杨山。南流东折,径其县南。县治索城,即索县之故城也。汉顺帝阳嘉中改从今名。阚駰以为兴水所出,东入沅。而是水又东历诸湖,方南注沅,亦曰渐水也。水所入之处,谓之鼎口。沅水又东历龙阳县之汜洲,洲长二十里,吴丹杨太守李衡植柑于其上。临死,敕其子曰:吾州里有木奴千头,不责衣食,岁绢千匹。太史公曰:江陵千树橘,可当封君。此之谓矣。吴末,衡柑成,岁绢千匹。今洲上犹有陈根余枿,盖其遗也。沅水又东径龙阳县北,城侧沅水,沅水又东合寿溪,内通大溪口,有木连理,根各一岸,而凌空交合。其上承诸湖,下注沅水。
又东至长沙下隽县西,北入于江。
沅水下注洞庭湖,方会于江。
沅水出武陵镡成县北界沅水谷,《山海经》曰:祷过之山,沅水出焉,而南流注于海是也。南至郁林潭中县,与邻水合。
水出无阳县,县故镡成也。晋义熙中,改从今名。俗谓之移溪,溪水南历潭中,注于泿水。
又东至苍梧猛陵县为郁溪,又东至高要县为大水。郁水出郁林之阿林县,东径猛陵县。猛陵县在广信之西南,王莽之猛陆也。泿水于县左合郁溪,乱流径广信县,《地理志》苍梧郡治,武帝元鼎六年开。王莽之新广郡,县曰广信亭。王氏《交广春秋》曰:元封五年,交州自赢陛县移治于此。建安十六年,吴遣临淮步骘为交州刺史,将武吏四百人之交州,道路不通。苍梧太守长沙吴巨拥众五千,骘有疑于巨,先使渝巨,巨迎之于零陵,遂得进州。巨既纳骘,而后有悔,骘以兵少,恐不存立。巨有都督区景,勇略与巨同,士为用。骘恶之,阴使人请巨,巨往告景,勿诣骘。骘请不已,景又往,乃于厅事前中庭俱斩,以首徇众,即此也。郁水又径高要县。《晋书·地理志》曰:县东去郡五百里,刺史夏避毒,徙县水居也。县有鹄奔亭,广信苏施妻始珠,鬼讼于交州刺史何敞处,事与斄亭女鬼同。王氏《交广春秋》曰:步骘杀吴巨、区景,使严舟船,合兵二万,下取南海。苍梧人衡毅、钱博,宿巨部伍,兴军逆骘于苍梧高要峡口,两军相逢于是,遂交战,毅与众投水死者,千有余人。
又东至南海番禺县西,分为二:其一,南入于海。
郁水分泿南注。
其一,又东过县东,南入于海。
泿水东别径番禺,《山海经》谓之贲禺者也。交州治中合浦姚文式问云:何以名为番禺?答曰:南海郡昔治在今州城中,与番禺县连接,今入城东南偏,有水坑陵,城倚其上,闻此县人名之为番山。县名番禹,傥谓番山之禹也。《汉书》所谓浮牂柯,下离津,同会番禺,盖乘斯水而入越也。秦并天下,略定扬越,置东南一尉,西北一候,开南海以谪徙民。至二世时,南海尉任嚣召龙川令赵佗曰:闻陈胜作乱,豪桀叛秦,吾欲起兵,阻绝新道,番禺负险,可以为国。会病绵笃,无人与言,故召公来,告以大谋。嚣卒,佗行南海尉事,则拒关门设守,以法诛秦所置吏,以其党为守,自立为王。高帝定天下,使陆贾就立佗为南越王,剖符通使。至武帝元鼎五年,遣伏波将军路博德等攻南越王,五世九十二岁而亡。以其地为南海、苍梧、郁林、合浦、交趾、九真、日南也。建安中,吴遣步骘为交州,骘到南海见土地形势,观尉佗旧治处,负山带海,博敞渺目,高则桑土,下则沃衍,林麓鸟兽,于何不有?海怪鱼鳖,鼋鼍鲜鳄,珍怪异物,千种万类,不可胜记。佗因冈作台,北面朝汉,圆基千步,直峭百丈,顶上三亩,复道回环,逶迤曲折,朔望升拜,名曰朝台。前后刺史、郡守,迁除新至,未尝不乘车升履,于焉逍遥。骘登高远望,睹巨海之浩茫,观原薮之殷阜,乃曰:斯诚海岛膏腴之地,宜为都邑。建安二十二年,迁州番禺,筑立城郭,绥和百越,遂用宁集。交州治中姚文式《问答》云:朝台在州城东北三十里。裴渊《广州记》曰:城北有尉伦墓,墓后有大冈,谓之马鞍冈。秦时占气者言,南方有天子气。始皇发民凿破此冈,地中出血,今凿处犹存。以状取目,故冈受厥称焉。王氏《交广春秋》曰:越王赵佗,生有奉制称藩之节,死有秘奥神密之墓。佗之葬也,因山为坟,其垅茔可谓奢大,葬积珍玩。吴时,遣使发掘其墓,求索棺柩,凿山破石,费日损力,卒无所获。佗虽奢潜,慎终其身,乃今后人不知其处,有似松、乔迁景,牧竖固无所残矣。邓德明《南康记》曰:昔有卢耽,仕州为治中,少栖仙术,善解云飞。每夕,辄凌虚归家,晓则还州。尝于元会至朝,不及朝列,化为白鹄,至阙前,回翔欲下,威仪以石掷之,得一只履,耽惊还就列,内外左右,莫不骇异。时步骘为广州,意甚恶之,便以状列闻,遂至诛灭。《广州记》称吴平,晋滕修为刺史,修乡人语修,虾须长一赤,修责以为虚。其人乃至东海,取虾须,长四赤,速送示修,修始服谢,厚为遣。其一水南入者,郁川分派,径四会入海也。其一即川东别径番禺城下,《汉书》所谓浮牂柯、下离津,同会番禺,盖乘斯水而入于越也。泿水又东径怀化县,入于海。水有鱼。裴渊《广州记》曰: 鱼长二丈,大数围,皮皆鑢物。生子,子小随母食,惊则还入母腹。《吴录·地理志》曰: 鱼子朝索食,暮入母腹。《南越志》曰:暮从脐入,旦从口出。腹里两洞,肠贮水以养子。肠容二子,两则四焉。
其余水又东至龙川为涅水,屈北入员水。
泿水枝津衍注,自番禺东历增城县。《南越志》曰:县多。,山鸡也。光采鲜明,五色炫耀,利距善斗,世以家鸡斗之,则可擒也。又径博罗县西界龙川,左思所谓目龙川而带坰者也。赵佗乘此县而跨据南越矣。员水又东南一千五百里,入南海。
东历揭阳县,王莽之南海亭,而注于海也。


卷三十八  资水、涟水、湘水、漓水、溱水 
资水出零陵都梁县路山,资水出武陵郡无阳县界唐糺山,盖路山之别名也,谓之大溪水。东北径邵陵郡武冈县南,县分都梁之所置也。县左右二冈对峙,重阻齐秀,间可二里,旧传后汉代五溪蛮,蛮保此冈,故曰武冈,县即其称焉。大溪径建兴县南,又径都梁县南,汉武帝元朔五年,以封长沙定王子敬侯遂之邑也。县西有小山,山上有渟水,既清且浅,其中悉生兰草,绿叶紫茎,芳风藻川,兰馨远馥。俗谓兰为都梁,山因以号,县受名焉。
东北过夫夷县,夫水出县西南零陵县界少延山,东北流径扶县南,本零陵之夫夷县也。
汉武帝元朔五年,以封长沙定王子敬侯义之邑也。夫水又东往邵陵水,谓之邵陵浦,水口也。
东北过邵陵县之北,县治郡下,南临大溪,水径其北,谓之邵陵水。魏咸熙二年,吴宝鼎元年,孙皓分零陵北部,立邵陵郡于邵陵县,县故昭陵也。溪水东得高平水口,水出武陵郡沉陵县首望山,西南径高平县南,又东入邵陵县界,南入于邵水。邵水又东会云泉水,水出零陵永昌县云泉山,西北流径邵阳南,县故昭阳也。云泉水又北注邵陵水,谓之邵阳水口。自下东北出益阳县,其间径流山峡,名之为茱萸江,盖水变名也。
又东北过益阳县北,县有关羽濑,所谓关侯滩也。南对甘宁故垒,昔关羽屯军水北,孙权令鲁肃、甘宁拒之于是水。宁谓肃曰:羽闻吾咳唾之声,不敢渡也,渡则成擒矣。羽夜闻宁处分,曰兴霸声也,遂不渡。茱萸江又东径益阳县北,又谓之资水。应劭曰:县在益水之阳。今无益水,亦或资水之殊目矣。然此县之左右,处处有深潭,渔者咸轻舟委浪,谣咏相和。罗君章所谓其声绵邈者也。水南十里,有井数百口,浅者四五尺,或三五丈,深者亦不测其深。古老相传,昔人以杖撞地,辄便成井。或云古人采金沙处,莫详其实也。又东与沅水合于湖中,东北入于江也。
湖即洞庭湖也。所入之处,谓之益阳江口。
涟水出连道县西,资水之别。
水出邵陵县界,南径连道县,县故城在湘乡县西百六十里。控引众流,合成一溪。东入衡阳、湘乡县,历石鱼山,下多玄石,山高八十余丈,广十里。石色黑而理若云母,开发一重,辄有鱼形,鳞首尾,宛若刻画,长数寸,鱼形备足,烧之作鱼膏腥,因以名之。涟水又径湘乡县,南临涟水,本属零陵,长沙定王子昌邑。涟水又屈径其县东,而入湘南县也。
东北过湘南县南,又东北至临湘县西南,东入于湘。涟水自湘甫县东流,至衡阳湘西县界,入于湘水也。于临湘县为西南者矣。
湘水出零陵始安县阳海山,即阳朔山也。应劭曰:湘出零山。盖山之殊名也。山在始安县北,县故零陵之南部也。魏咸熙二年,孙皓之甘露元年,立始安郡。湘、漓同源,分为二水,南为漓水,北则湘川,东北流。罗君章《湘中记》曰:湘水之出于阳朔则觞为之舟,至洞庭,日月若出入于其中也。东北过零陵县东,越城峤水,南出越城之峤,峤即五岭之西岭也。秦置五岭之戍,是其一焉。北至零陵县,下注湘水。湘水又径零陵县南,又东北径观阳县,与观水合。水出邻贺郡之谢沭县界,西北径观阳县西,县盖即水为名也。又西北流注于湘川,谓之观口也。
又东北过洮阳县东。
洮水出县西南大山,东北径其县南,即洮水以立称矣。汉武帝元朔五年,封长沙定王子节侯拘为侯国。王莽更名之曰洮治也。其水东流注于湘水。又东北过泉陵县西,营水出营阳泠道县南山,西流径九疑山下,蟠基苍梧之野,峰秀数郡之间。罗岩九举,各导一溪,岫壑负阻,异岭同势,游者疑焉,故曰九疑山。大舜窆其阳,商均葬其阴。山南有舜庙,前有石碑,文字缺落,不可复识。自庙仰山极高,直上可百余里。古老相传,言未有登其峰者。山之东北,泠道县界,又有舜庙。县甫有舜碑,碑是零陵太守徐俭立。营水又西径营道县,冯水注之。水出临贺郡冯乘县东北冯冈。其水导源冯溪,西北流,县以托名焉。冯水带约众流,浑成一川,谓之北渚。历县北,西至关下,关下,地名也,是商舟改装之始。冯水又左,合萌渚之水。水南出于萌渚之峤,五岭之第四岭也,其山多锡,亦谓之锡方矣。渚水北径冯乘县西,而北注冯水。冯水又径营道县而右会营水。营水又西北屈而径营道县西,王莽之九疑亭也。营水又东北径营浦县南。营阳郡治也。魏咸熙二年,吴孙皓分零陵置,在营水之阳,故以名郡矣。营水又北,都溪水注之。水出春陵县北二十里仰山,南径其县西。县本泠道县之眷陵乡,盖因言溪为名矣。汉长沙定王分以为县,武帝元朔五年,封王中子买为舂陵侯。县故城东,又有一城,东西相对,各方百步。古老相传,言汉家旧城,汉称犹存,知是节侯故邑也。城东角有一碑,文字缺落,不可复识。东南三十里尚有节侯庙。都溪水又南径新宁县东,县东傍都溪,溪水又西径县南,左与五溪俱会。县有五山,山有一溪,五水会于县门,故曰都溪也。都溪水自县又西北流,径泠道县北,与泠水合。水南出九疑山,北流径其县西南,县指泠溪以即名,王莽之泠陵县也。泠水又北流注于都溪水,又西北入于营水。营水又北流,入营阳峡,又北至观阳县而出于峡,大小二峡之间,为沿溯之极艰矣。营水又西北,径泉陵县西,汉武帝元朔五年,以封长沙定王子节侯贤之邑也。王莽名之曰溥润,零陵郡治,故楚矣。汉武帝元鼎六年分桂阳置。太史公曰:舜葬九疑,实惟零陵。郡取名焉,王莽之九疑郡也。下邳陈球为零陵太守,桂阳贼胡兰攻零陵,激流灌城,球辄于内因地势反决水淹贼,相拒不能下。县有自上乡。《零陵先贤传》曰:郑产,字景载,泉陵人也,为白土啬夫。汉末多事,国用不足,产子一岁,辄出口钱,民多不举子。产乃敕民勿得杀子,口钱当自代出。产言其郡、县,为表上言,钱得除,更名白土为更生乡也。《晋书地道记》曰:县有香茅,气甚芬香,言贡之以缩酒也。营水又北流注于湘水。湘水又东北与应水合。水出邵陵县历山,崖瞪险阻,峻崿万寻,澄源湛于下,应水涌于上。东南流径应阳县南,晋分观阳县立,盖即应水为名也。应水又东南流,径有鼻墟南。王隐曰:应阳县本泉陵之北部,东五里有鼻墟,言象所封也。山下有象庙,言甚有灵,能兴云雨。余所闻也,圣人之神曰灵,贤人之精气为鬼,象生不慧,死灵何寄乎?应水又东南流而注于湘水。湘水又东北得口,水出永昌县北罗山。东南流径石燕山东,其山有石,绀而状燕,因以名山。其石或大或小,若母子焉,及其雷风相薄,则石燕群飞,颉颃如真燕矣。罗君章云:今燕不必复飞也。其水又东南径永昌县南,又东流注于湘水。又东北径祁阳县南,又有余溪水注之。水出西北邵陵郡邵陵县,东南流注于湘。其水扬清泛浊,水色两分。湘水又北与宜溪水合,水出湘东郡之新宁县西南新平故县东,新宁,故新平也。众川泻浪,共成一津。西北流,东岸山下有龙穴,宜水径其下,天旱则拥水注之,便有雨降。宜水又西北注于湘。湘水又西北,得舂水口,水上承营阳春陵县西北潭山,又北径新宁县东,又西北流注于湘水也。又东北过重安县东,又东北过县西,承水从东南来注之。承水出衡阳重安县西,邵陵县界邪姜山,东北流至重安县,径舜庙下,庙在承水之阴。又东合略塘,相传云:此塘中有铜神,今犹时闻铜声于水,水辄变绿,作铜腥,鱼为之死。承水又东北径重安县南,汉长沙顷王子度邑也,故零陵之钟武县。王莽更名曰钟桓也。武水入焉。水出钟武县西南表山,东流至钟武县故城南。而东北流至重安县,注于承水,至湘东临承县北、东注于湘,谓之承口。临承即故酃县也,县即湘东郡治也。郡旧治在湘水东,故以名郡。魏正元二年,吴主孙亮分长沙东部立。县有石鼓,高六尺,湘水所径,鼓鸣则土有兵革之事。罗君章云:扣之,声闻数十里,此鼓今无复声。观阳县东有裴岩,其下有石鼓,形如覆船,扣之清响远彻,其类也。湘水又北历印石,石在衡山县南,湘水右侧。盘石或大或小,临水,石悉有迹,其方如印。累然行列,无文字,如此可二里许,因名为印石也。湘水又北径衡山县东,山在西南,有三峰,一名紫盖,一名石囷,一名芙蓉,芙蓉峰最为竦杰,自远望之,苍苍隐天。故罗含云:望若阵云,非清霁素朝,不见其峰。丹水涌其左,澧泉流其右。《山经》谓之峋嵝,为南岳也。山下有舜庙,南有祝融冢。楚灵王之世,山崩,毁其坟,得《营丘九头图》。禹治洪水,血马祭山,得金简玉字之书。芙蓉峰之东,有仙人石室,学者经过,往往闻讽诵之音矣。衡山东南二面,临映湘川,自长沙至此江湘六百里中,有九向九背,故渔者歌曰:帆随湘转,望衡九面。山上有飞泉下注,下映青林,直注山下,望之若幅练在山矣。湘水又东北径湘南县东,又历湘西县南,分湘南置也,衡阳郡治。魏甘露二年,吴孙亮分长沙西部立治,晋湘南太守何承天徙治湘西矣。《十三州志》曰:日华水出桂阳郴县日华山西,至湘南县入湘。《地理志》曰:郴县有耒水,出耒山西,至湘南西入湘。湘水又北径麓山东,其山东临湘川,西傍原隰,息心之士,多所萃焉。
又东北过阴山县西,洣水从东南来注之。又北过醴陵县西,漉水从东南来注之。
《续汉书·五行志》曰:建安八年,长沙醴陵县有大山,常鸣如牛呴声,积数年。后豫章贼攻没县亭,杀掠吏民,因以为候。湘水又北径建宁县,有空泠峡,惊浪雷奔,濬同三峡。湘水又北径建宁县故城下,晋太始中立。又北过临湘县西,浏水从县西北流注。
县南有石潭山,湘水径其西。山有石室、石床,临对清流。湘水又北径昭山西,山下有旋泉,深不可测,故言昭潭无底也。亦谓之曰湘州潭。湘水又北径南津城西,西对橘洲,或作吉学字,为南津洲尾。水西有橘洲子戍,故郭尚存。湘水又北,左会瓦官水口,湘浦也。又径船官西,湘洲商舟之所次也。北对长沙郡,郡在水东州城南,旧治在城中,后乃移此。湘水左径麓山东,上有故城,山北有白露水口,湘浦也。又右径临湘县故城西县治,湘水滨临川侧,故即名焉。王莽改号抚陆,故楚南境之地也。秦灭楚,立长沙郡,即青阳之地也。秦始皇二十六年,令曰:荆王献青阳以西。《汉书·邹阳传》曰:越水长沙,还舟青阳。《注》:张晏曰:青阳,地名也。苏林曰:青阳,长沙县也。汉高祖五年,以封吴芮为长沙王,是城即芮筑也。汉景帝二年,封唐姬子发为王,都此,王莽之镇蛮郡也。于《禹贡》则荆州之域。晋怀帝以永嘉元年,分荆州、湘中请郡,立湘州,治此。城之内,郡廨西有陶佩庙,云旧是贾谊宅地,中有一井,是谊所凿,极小而深,上敛下大,其状似壶。傍有一脚石床,才容一人坐,形制甚古。流俗相承,云谊宿所坐床。又有大柑树,亦云谊所植也。城之西北有故市,北对临湘县之新治。县治西北有北津城,县北有吴芮家,广逾六十八丈,登临写目,为廛郭之佳憩也。郭颁《世语》云:魏黄初末,吴人发芮冢,取木,于县立孙坚庙,见芮尸,容貌衣服并如故。吴平后,与发冢人于寿春见南蛮校尉吴纲,曰:君形貌何类长沙王吴芮乎?但君微短耳。纲瞿然曰:是先祖也。自芮卒至家发四百年,至见纲又四十余年矣。湘水左合誊口,又北得石椁口,并湘浦也。右合麻溪水口,湘浦也。湘水又北径三石山东,山枕侧湘川,北即三石水口也,湘浦矣。水北有三石戍,戍城为二水之会也。湘水又径浏口戍西,北对浏水。又北,沩水从西南来注之。
沩水出益阳县马头山,东径新阳县南,晋太康元年改曰新康矣。沩水又东入临湘县,历沩口戍东,南注湘水。湘水又北合断口,又北则下营口,湘浦也。湘水之左岸有高口水,出益阳县西,北径高口戍南,又西北,上鼻水自鼻洲上口,受湘西入焉,谓之上鼻浦。高水西北与下鼻浦合,水自鼻洲下口,首受湘川,西通高水,谓之下鼻口。高水又西北,右屈为陵子潭,东北流注湘为陵子口。湘水自高口戍东,又北,右会鼻洲,左合上鼻口,又北,右对下鼻口,又北,得陵子口,湘水右岸,铜官浦出焉。湘水又北径铜官山,西临湘水,山土紫色,内含云母,故亦谓之云母山也。
又北过罗县西, 水从东来流性。
湘水又北径锡口戍东,又北左派,谓之锡水。西北流径锡口戍北,又西北流,屈而东北,注玉水焉。水出西北玉池,东南流注于锡浦,谓之玉池口。锡水又东北,东湖水注之。水上承玉池之东湖也,南注于锡,谓之三阳径,水南有三戍,又东北注于湘。湘水自锡口北出,又得望屯浦,湘浦也。湘水又北,枝津北出,谓之门径也。湘水纡流西北,东北合门水,谓之门径口。又北得三溪水口,水东承大湖,西通湘浦,三水之会,故得三溪之目耳。又北,东会大对水口,西接三津径。湘水又北径黄陵亭西,右合黄陵水口,其水上承大湖,湖水西流,径二妃庙南,世谓之黄陵庙也。言大舜之陟方也,二妃从征,溺于湘江。神游洞庭之渊,出入滞湘之浦。潇者,水清深也。《湘中记》曰,湘川清照五六丈,下见底石,如樗蒱矢,五色鲜明,白沙如霜雪,赤崖若朝霞,是纳潇湘之名矣,故民为立祠于水侧焉。荆州牧刘表刊石立碑,树之于庙,以硅不朽之传矣。黄水又西流入于湘,谓之黄陵口。昔王子中有异才,年二十而得恶梦,作《梦赋》。二十一,溺死于湘浦,即斯川矣。湘水又北径白沙戍西,又北,右会东町口水也。湘水又左合决湖口,水出西肢,东通湘渚。湘水又北,汨水注之。水东出豫章艾县桓山,西南径吴昌县北,与纯水合。水源出其县东南纯山,西北流,又东径其县南,又北径其县故城下。县是吴主孙权立。纯水又右会汨水。汨水又西径罗县北,本罗子国也。故在襄阳宜城县西,楚文王移之于此。秦立长沙郡,因以为县,水亦谓之罗水。汨水又西,径玉笥山。罗含《湘中记》云:屈潭之左,有玉笛山,道士遗言,此福地也,一曰地脚山。汨水又西为屈潭,即汨罗渊也。屈原怀沙自沉于此,故渊潭以屈为名。昔贾谊、史迁皆尝径此,弭檝江波,投吊于渊。渊北有屈原庙,庙前有碑。又有《汉南太守程坚碑》,寄在原庙。汨水又西径汨罗戍南,西流注于湘,《春秋》之罗油矣,世谓之汨罗口。湘水又北,枝分北出径泪罗戍西,又北径磊石山东,又北径磊石戍西,谓之苟导径矣,而北合湘水。湘水自阳罗口,西北径磊石山西,而北对青草湖,亦或谓之为青草山也。西对悬城口,湘水又北得九口,并湘浦也。湘水又东北,为青草湖口,右会苟导泾北口,与劳口合,又北得同拌口,皆湘浦右迤者也。又北过下隽县西,微水从东来流注。
湘水左会清水口,资水也。世谓之益阳江。湘水之左径鹿角山东,右径谨亭戍西,又北合查浦,又北得万石浦,咸湘浦也。侧湘浦北有万石戍。湘水左则沅水注之,谓之横房口,东对微湖,世或谓之麋湖也。右属微水,即《经》所谓微水经下隽者也。西流注于江,谓之麋湖口。湘水又北径金浦戍,北带金浦水,湖溠也。湘水左则澧水注之,世谓之武陵江。凡此四水,同注洞庭,北会大江,名之五渚。《战国策》曰:秦与荆战,大破之,取洞庭五渚者也。湖水广圆五百余里,日月若出没于其中。《山海经》云:洞庭之山,帝之二女居焉。沅、澧之风,交潇、湘之浦,出入多飘风暴雨。湖中有君山、编山。君山有石穴,潜通吴之包山,郭景纯所谓巴陵地道者也。是山,湘君之所游处,故曰君山矣。昔秦始皇遭风于此,而问其故。博士曰:湘君出入则多风。秦王乃赭其山。汉武帝亦登之,射蛟于是山,东北对编山,山多箎竹。两山相次去数十里,回峙相望,孤影若浮。湖之石岸有山,世谓之笛乌头石。石北右会翁湖口。水上承翁湖,左合洞浦,所谓三苗之国,左洞庭者也。又北至巴丘山,入于江。
山在湘水右岸。山有巴陵故城,本吴之巴丘邪阁城也。晋太康元年,立巴陵县于此,后置建昌郡。宋元嘉十六年,立巴陵郡,城跨冈岭,滨阻三江。巴陵西对长洲,其洲南分湘浦,北届大江,故曰三江也。三水所会,亦或谓之三江口矣。夹山列关,谓之射猎,又北对养口,咸湘浦也。水色青异,东北入于大江,有清浊之别,谓之江会也。
漓水亦出阳海山,漓水与湘水出一山而分源也。湘、漓之间,陆地广百余步,谓之始安峤,峤即越城峤也。峤水自峤之阳,南流注漓,名曰始安水。故庾仲初之赋《扬都》云:判五岭而分流者也。漓水又南与沩水合,水出西北邵陵县界,而东南流至零陵县西,南径越城西。建安十六年,交州刺史赖恭自广信合兵小零陵越城迎步骘,即是地也。沩水又东南流,注于漓水,《汉书》所谓出零陵下漓水者也。漓水又南合弹丸溪,水出于弹丸山。山有涌泉,奔流冲激。山嵁及溪中,有石若丸,自然珠圆,状弹丸矣,故山水即名焉。验其山有石窦,下深数丈,洞穴深远,莫究其极。溪水东流注于漓水。漓水又南径始兴县东,魏元帝咸熙二年,吴孙皓分零陵南部,立始兴县。漓水又南,右会洛溪,溪水出永丰县西北洛溪山,东流径其县北,县本苍梧之北乡,孙皓割以为县。洛溪水又东南径始安县,而东注漓水。漓水又东南流,入熙平县,径羊濑山,山临漓水,石间有色类羊。又东南径鸡濑山,山带漓水,石色状鸡,故二山以物象受名矣。漓水又南,得熙平水口,水源出县东龙山,西南流径其县南,又西与北乡溪水合,水出县东北北乡山,西流径其县北,又西流南转,径其县西,县本始安之扶乡也,孙皓割以为县。溪水又南注熙平水,熙平水又西注于漓水。县南有朝夕塘,水出东山西南,有水从山下注塘,一日再增再减,盈缩以时,未尝愆期,同于潮水,因名此塘为朝夕塘矣。漓水又西径平乐县界,左合平乐溪口,水出临贺郡之谢沭县南历山,西北流径谢沭县西南,西南流至平乐县东南,左会谢沭众溪,派流凑合,西径平乐南。孙皓割苍梧之境,立以为县,北隶始安。溪水又西南流,注于漓水,谓之平乐水。
南过苍梧荔浦县,濑水出县西北鲁山之东,径其县西,与濡水合。水出永丰县西北濡山,东南径其县西,又东南流入荔浦县,注于濑溪,又注于漓水,漓水之上有关。漓水又南,左合灵溪水口,水出临贺富川县北符灵冈,南流径其县东,又南注于漓水也。又南至广信县,入于郁水。
溱水出桂阳临武县南,绕城西北屈东流。
溱水导源县西南,北流径县西,而北与武溪合。《山海经》曰:肄水出临武西南,而东南注于海。入番禺西。肄水盖溱水之别名也。武溪水出临武县西北桐柏山,东南流,右合溱水,乱流东南径临武县西,谓之武溪。县侧临溪东,因曰临武县,王莽更名大武也。溪又东南流,左会黄岑溪水,水出郴县黄岑山,西南流,右合武溪水。武溪水又南入重山,山名蓝豪,广圆五百里,悉曲江县界。崖峻险阻,岩岭干天,交柯云蔚,霆天晦景,谓之泷中。悬湍回注,崩浪震山,名之泷水。
东至曲江县安聂邑东,屈西南流。
泷水又南出峡,谓之泷口。西岸有任将军城,南海都尉任嚣所筑也。嚣死,尉佗自龙川始居之。东岸有任将军庙。泷水又南合泠水,泠水东出泠君山,山,群峰之孤秀也。晋太元十八年,崩十余丈。于是悬涧瀑挂,倾流注壑,颓波所入,灌于泷水。泷水又右合林水,林水出县东北洹山。王歆之《始兴记》曰:林水源里有石室,室前磐石上,行罗十瓮,中悉是饼银。采伐遇之,不得取,取必迷闷。晋太元初,民封驱之家仆,密窃三饼归,发看,有大蛇螫之而死。《湘州记》曰:其夜,驱之梦神语曰:君奴不谨,盗银三饼,即日显戮,以银相偿。觉视,则奴死银在矣。林水自源西注于泷水。又与云水合,水出县北汤泉。泉源沸涌,浩气云浮,以腥物投之,俄顷即热。其中时有细赤鱼游之,不为灼也。西北合泷水。又有藉水,上承沧海水,有岛屿焉。其水吐纳众流,西北注于泷水。泷水又南历灵鹫山,山本名虎郡山,亦曰虎市山,以虎多暴故也。晋义熙中,沙门释僧律,葺宇岩阿,猛虎远迹,盖律仁感所致,因改曰灵鹫山。泷水又南径曲江县东,云县昔号曲红,曲红,山名也,东连冈是矣。泷中有碑,文曰:自瀑亭至乎曲红。按《地理志》,曲江,旧县也。王莽以为除虏。始兴郡治。魏文帝咸熙二年,孙皓分桂阳南部立。县东傍泷溪,号曰北泷水,水左即东溪口也。水出始兴东江州南康县界石阎山,西流而与连水合,水出南康县凉热山连溪,山即大庾岭也。五岭之最东矣,故曰东峤山。斯则改装之次,其下船路,名涟溪。涟水南流,注于东溪,谓之涟口,庾仲初谓之大庾峤水也。东溪亦名东江,又曰始兴水。又西,邪阶水注之。水出县东南邪阶山。水有别源,曰巢头,重岭衿泷,湍奔相属,祖源双注,合为一川。水侧有鼻天子城,鼻天子,所未闻也。邪阶水又西北注于东江。江水又西径始兴县南,又西入曲江县,邸水注之。水出浮岳山,山蹑一处,则百余步动,若在水也,因名浮岳山。南流注于东江,东江又西,与利水合。水出县之韶石北山,南流径韶石下,其高百仞,广圆五里,两石对峙,相去一里,小大略均,似双阙,名曰韶石。古老言,昔有二仙,分而憩之,自尔年丰,弥历一纪。利水又南径灵石下,灵石一名逃石,高三十丈,广圆五百丈。耆旧传言,石本桂林武城县,因夜迅雷之变,忽然迁此,彼人来见,叹曰:石乃逃来。因名逃石,以其有灵运徙,又曰灵石,其杰处,临江壁立,霞驳有若缋焉。水石惊濑,传响不绝,商舟淹留,聆玩不已。利水南注东江,东江又西注于北江,谓之东江口。溱水自此有始兴大庾之名,而南入浈阳县也。
过浈阳县,出洭浦关,与桂水合。
溱水南径浈阳县西,旧汉县也,王莽之綦武矣。县东有浈石山,广圆三十里,挺崿大江之北,盘址长川之际。其阳有石室,渔叟所憩。昔欲于山北开达郡之路,辄有大蛇断道,不果。是以今行者,必于石室前泛舟而济也。溱水又西南,历皋口、太尉二山之间,是曰浈阳峡。两岸杰秀,壁立亏天。昔尝凿石架阁,令两岸相接,以拒徐道覆。溱水出峡,左则浈水注之,水出南海龙川县西,径浈阳县南,右注溱水。故应劭曰:浈水西入溱是也。溱水又西南,洭水入焉。《山海经》所谓湟水,出桂阳西北山,东南注肄入敦浦西者也。溱水又西南径中宿县会一里水,其处隘,名之为观岐。连山交枕,绝崖壁竦,下有神庙,背阿面流,坛宇虚肃,庙渚攒石巉岩,乱峙中川。时水洊至,鼓怒沸腾,流木沦没,必无出者。世人以为河怕下材。晋中朝时,县人有使者至洛,事讫,将还。忽有一人寄其书云:吾家在观岐前,石间悬藤,即其处也。但叩藤,自当有人取之。使者谨依其言,果有二人出外取书,并延入水府,衣不沾濡。言此似不近情,然造化之中,无所不有,穆满西游,与河宗论宝。以此推之,亦为类矣。溱水又西南径中宿县南,吴孙皓分四会之北乡立焉。
南入于海。
溱水又南注于郁,而入于海。


卷三十九  洭水、深水、钟水、耒水、洣水、漉水、浏水、水、赣水、庐江水 
洭水出桂阳县卢聚,水出桂阳县西北上驿山卢溪,为卢溪水,东南流径桂阳县故城,谓之洭水。《地理志》曰:洭水出桂阳,南至四会是也。洭水又东南流,峤水注之,水出都峤之溪,溪水下流,历峡南出,是峡谓之贞女峡。峡西岸高岩,名贞女山。山下际有石,如人形,高七尺,状如女子,故名贞女峡。古来相传,有数女取螺于此,遇风雨昼晦,忽化为石。斯诚巨异,难以闻信。但启生石中,挚呱空桑,抑斯类矣。物之变化,宁以理求乎?溪水又合洭水。洭水又东南入阳山县,右合涟口,水源出县西北百一十里石塘村,东南流,水侧有豫章木,本径可二丈,其株根犹存,伐之积载,而斧迹若新。羽族飞翔不息,其旁众枝,飞散远集,乡亦不测所如,惟见一枝,独在含洭水矣。涟水东南流注于洭。洭水又东南流,而右与斟水合。水导源近出东岩下,穴口若井,一日之中,十溢十竭,信若潮流,而注洭水。洭水又南径阳山县故城西,耆旧传曰:往昔县长临县,辄迁擢超级,太史径观言势使然。掘断连冈,流血成川,城因倾阤,遂即倾败。阁下大鼓,飞上临武,乃之桂阳,追号圣鼓,自阳山达乎桂阳之武步驿,所至循圣鼓道也。其道如堑,迄于鼓城矣。洭水又径阳山县南,县故含洭县之桃乡,孙皓分立为县也。洭水又东南流也。东南过含洭县,应劭曰:洭水东北入沅。瓒注《汉书》,沅在武陵,去洭远,又隔湘水,不得入沅。洭水东南,左合翁水。水出东北利山湖,湖水广圆五里,洁逾凡水,西南流注于洭,谓之翁水口。口已下,东岸有圣鼓杖,即阳山之鼓杖也。横在川侧,虽冲波所激,未尝移动。百鸟翔鸣,莫有萃者。船人上下,以篙撞者,辄有疟疾。洭水又东南,左合陶水,水东出尧山。山盘纡数百里,有赭岩迭起,冠以青林,与云霞乱采。山上有白石英,山下有平陵,有大堂基,耆旧云,尧行宫所。陶水西径县北,右注洭水。洭水又径含洭县西。王歆《始兴记》曰:县有白鹿城,城南有白鹿冈。咸康中,郡民张鲂为县,有善政,白鹿来游,故城及冈并即名焉。
南出洭浦关,为桂水。
关在中宿县。洭水出关,右合溱水,谓之洭口。《山海经》谓之湟水。
徐广曰:湟水一名洭水,出桂阳,通四会,亦曰洭水也。汉武帝元鼎元年,路博德为伏波将军,征南越,出桂阳,下湟水,即此水矣。桂水其别名也。深水出桂阳卢聚,吕忱曰:深水,一名邃水,导源卢溪,西入营水,乱流营波,同注湘津。许慎云:深水出桂阳南平县也。《经》书桂阳者,县本隶桂阳郡,后割瞩始兴。县有卢溪、卢聚山,在南平县之南,九疑山东也。
西北过零陵营道县南,又西北过营浦县南,又西北过泉陵县,西北七里至燕室,邪入于湘。
水上有燕室丘,亦因为聚名也。其下水深不测,号曰龙渊。钟水出桂阳南平县都山,北过其县东,又东北过宋渚亭,又北过钟亭,与漼水合。
都山即都庞之峤,五岭之第三岭也。钟水即峤水也。庾仲初曰:峤水南入始兴溱水,注于海。北入桂阳,湘水注于江是也。漼水,即桂水也。漼、桂声相近,故字随读变,《经》仍其非矣。桂水出桂阳县北界山,山壁高耸,三面特峻,石泉悬注,瀑布而下。北径南平县,而东北流届钟亭,右会钟水,通为桂水也。故应劭曰:桂水出桂阳,东北入湘。
又北过魏宁县之东,魏宁,故阳安也。晋太康元年改曰晋宁。县在桂阳郡东,百二十里。县南、西二面,阻带清溪,桂水无出县东理,盖县邑流移,今古不同故也。又北入于湘。
耒水出桂阳郴县南山。
耒水发源出汝城县东乌龙白骑山,西北流径其县北,西流三十里,中有十四濑,各数百步,濬流奔急,竹节相次,亦为行旅溯涉之艰难也。又西北径晋宁县北,又西,左合清溪水口,水出县东黄皮山,西南流历县南,又西北注于耒水。汝城县在郡东三百余里,山又在县东,耒水无出南山理也。又北过其县之西,县有渌水,出县东侠公山。”西北流,而南屈注于耒,谓之程乡溪。郡置酒官,酝于山下,名曰程酒,献同酃也。耒水又西,黄水注之。水出县西黄岑山,山则骑田之峤,五岭之第二岭也。黄水东北流。按盛弘之云:众山水出注于大溪,号曰横流溪。溪水甚小,冬夏不干,俗亦谓之为贪泉,饮者辄冒于财贿,同于广州石门贪流矣。廉介为二千石,则不饮之。昔吴隐之挹而不乱,贪岂谓能渝其贞乎?盖亦恶其名也。刘澄之谓为一涯溪,通四会,殊为孟浪而不悉也。庾仲初云:峤水南入始兴溱水,注海。即黄岑水入武溪者也。北水入桂阳湘水,注于大江,即是水也。右则千秋水注之。水出西南万岁山,山有石室,室中有钟乳。山上悉生灵寿木,溪下即千秋水也。水侧民居,号万岁村。其水下合黄水,黄水又东北径其县东,右合除泉水。水出县南湘陂村,村有圆水,广圆可二百步,一边暖,一边冷。冷处极清绿,浅则见石,深则见底。暖处水白且浊,玄素既殊,凉暖亦异,厥名除泉,其犹江乘之半汤泉也。水盛则泻黄溪,水耗则津径辍流。郴,旧县也,桂阳郡治也,汉高帝二年分长沙置。《地理志》曰:桂水所出,因以名也。王莽更名南平,县曰宣风。项羽迁义帝所筑也。县南有义帝冢,内有石虎,因呼为白虎郡。《东观汉记》曰:茨充,字子河,为桂阳太守,民惰懒,少粗履,足多剖裂,茨教作履。今江南知织履,皆充之教也。黄溪东有马岭山,高六百余丈,广圆四十许里。汉末,有郡民苏眈,栖游此山。《桂阳列仙传》云:眈,郴县人。少孤,养母至孝。言语虚无,时人谓之痴。常与众儿共牧牛,更直为帅,录牛无散。每至眈为帅,牛辄徘徊左右,不逐自还。众儿曰:汝直,牛何道不走耶?眈曰:非汝曹所知。即面辞母云:受性应仙,当违供养。涕泗。又说:年将大疫,死者略半。穿一井饮水,可得无恙。如是有哭声甚哀。后见眈乘白马,还此山中,百姓为立坛祠,民安岁登,民因名为马岭山。黄水又北流注于耒水,谓之郴口。耒水又西径华山之阴,亦曰华石山,孤峰特耸,枕带双流。东则黄溪、耒水之交会也。耒水东流沿注,不得北过其县西也。两岸连山,石泉悬溜,行者辄徘徊留念,情不极已也。
又北过便县之西,县故惠帝封长沙王子吴浅为侯国,王莽之便屏也。县界有温泉水,在郴县之西北,左右有田数千亩,资之以溉。常以十二月下种,明年三月谷熟,度此水冷,不能生苗。温水所溉,年可三登。其余彼散流,入于耒水也。又西北过耒阳县之东,耒阳,旧县也,盖因水以制名。王莽更名南平亭。东傍耒水,水东肥南,有郡故城。县有溪水,东出侯计山,其水清辙,冬温夏冷。西流谓之肥川。川之北有卢塘。塘他八顷,其深不测,有大鱼常至,五月辄一奋跃,水涌数丈,波襄四陆,细鱼奔迸,随水登岸,不可胜计。又云:大鱼将欲鼓作,诸鱼皆浮聚。水侧注。西北径蔡洲,洲西即蔡伦故宅,傍有蔡子池。伦,汉黄门,顺帝之世,捣故鱼网为纸,用代简素,自其始也。
又北过酃县东,县有酃湖,湖中有洲,洲上民居,彼人资以给酿,酒甚醇美,谓之酃酒,岁常贡之。湖边尚有酃县故治,西北去临承县十五里。从省隶。《十二州志》曰:大别水南出耒阳县太山,北至酃县入湖也。
北入于湘。
耒水西北至临承县,而右注湘水,谓之耒口也。
洣水出荼陵县上乡,西北过其县西,水出江州安成郡广兴县太平山,西北流,径荼陵县之南。汉武帝元朔四年,封长沙定王子节侯訢之邑也。王莽更名声乡矣。洣水又屈而过其县,西北流注也。《地理志》谓之泥水者也。
又西北过攸县南,攸水出东南安成郡安复县封侯山,西北流径其县北,县北带攸溪,盖即溪以名县也。汉武帝元朔四年,封长沙定王子则为攸舆侯,即《地理志》所谓攸县者也。攸水又西南流入荼陵县,入于洣水也。
又西北过阴山县南,县本阳山县也,县东北犹有阳山故城,即长沙孝王子宗之邑也。言其势王,故堑山堙谷,改曰阴山县。县上有容水,自侯昙山下注洣水,谓之容口。水有大穴,容一百石,水出于此,因以名焉。洣水又西北径其县东,又西径历口。县有历水,下注洣,谓之历口。洣水又西北,与洋湖水会。水出县西北乐薮冈下洋湖,湖去冈七里,湖水下注洣,谓之洋湖口。洣水东北有峨山,县东北又有武阳龙尾山,并仙者羽化之处。上有仙人及龙马迹,于其处得遗咏,虽神栖白云,属想芳流,藉念泉乡,遗咏在兹。览其余诵,依然息远,匪直邈想霞踪,爱其文咏可念,故端犊抽札,以诠其咏。其略曰:登武阳,观乐薮,峨岭千蕤洋湖口,命蜚螭,驾白驹,临天水,心踟蹰,千载后,不知如。盖胜赏神乡,秀情超拔矣。又西北入于湘。
流水出醴陵县东漉山,西过其县南,醴陵县,高后四年,封长沙相侯越为国。县南临渌水,水东出安城乡翁陵山。余谓漉、渌声相近,后人藉便,以渌为称,虽翁陵名异,而即麓是同。屈从县西,西北流至流浦,注入于湘。浏水出临湘县东南浏阳县,西北过其县,东北与涝水合。浏水出县东江州豫章县首裨山,导源西北流,径其县南,县凭溪以即名也。又西北注于临湘县也。
西入于湘。
水出豫章艾县,《春秋左氏传》曰:吴公子庆忌谏夫差,不纳,居于艾是也。王莽更名治翰。
西过长沙罗县西,罗子自枝江徙此,世犹谓之为罗侯城也。水又西流,积而为陂,谓之町湖也。又西至累石山,入于湘水。
累石山在北,亦谓之五木山,山方尖如五木状,故俗人藉以名之。山在罗口北。水又在罗水南,流注于湘,谓之东町口者也。
赣水出豫章南野县西,北过赣县东,《山海经》曰:赣水出聂都山,东北流注于江,入彭泽西也。班固称南野县,彭水所发,东入湖汉水。庾仲初谓大庾峤水,北入豫章,注于江者也。《地理志》曰:豫章水出赣县西南,而北入江。盖控引众流,总成一川,虽称谓有殊,言归一水矣。故《后汉郡国志》曰:赣有豫章水。雷次宗云:似因此水为其地名。虽十川均流,而此源最远,故独受名焉。刘澄之曰:县东南有章水,西有贡水,县治二水之间。二水合赣字,因以名县焉。是为谬也,刘氏专以字说水,而不知远失其实矣。豫章水导源东北流,径南野县北。赣川石阻,水急行难。倾波委注,六十余里,又北径赣县东,县即南康郡治,晋太康五年分庐江立。豫章水右会湖汉水,水出雩都县,导源西北流,径金鸡石,其石孤竦临川,耆老云:时见金鸡出于石上,故石取名焉。湖汉水又西北径赣县东,西入豫章水也。
又西北过庐陵县西,庐陵县,即王莽之桓亭也。《十三州志》称:庐水西出长沙安成县。武帝元光六年,封长沙定王子刘苍为侯国,即王莽之用成也。吴宝鼎中立,以为安成郡,东至庐陵入湖汉水也。
又东北过石阳县西,汉和帝水平九年,分庐陵立。汉献帝初平二年,吴长沙桓王立庐陵郡,治此。豫章水又径其郡南,城中有井,其水色半清半黄,黄者如灰汁,取作饮粥,悉皆金色,而甚芬香。
又东北过汉平县南,又东北过新淦县西,牵水西出宜春县,汉武帝元光六年,封长沙定王子刘成为侯国,王莽之修晓也。牵水又东径吴平县,旧汉平也,晋太康元年改为吴平矣。牵水又东径新淦县,即王莽之偶亭,而注于豫章水。湖汉及赣,并通称也。又淦水出其县下,注于赣水。
又北过南昌县西,旺水出南城县,西北流径南昌县南,西注赣水。又有浊水注之,水出康乐县,故阳乐也。浊水又东径望蔡县,县因汝南上蔡民萍居此土,晋太康元年,改为望蔡县。浊水又东径建成县,汉武帝元光四年,封长沙定王子刘拾为侯国。王莽更名之曰多聚也。县出然石,《异物志》曰:石色黄白而理疏,以水灌之便热,以鼎著其上,炊足以熟。置之则冷,灌之则热,如此无穷。元康中,雷孔章入洛,赍石以示张公。张公曰:此谓然石。于是乃知其名。浊水又东至南昌县,东流入于赣水。赣水又历白社西,有徐孺子墓。吴嘉禾中,太守长沙徐熙于墓隧种松,太守南阳谢景于墓侧立碑。永安中,太守梁郡夏侯嵩于碑傍立恩贤亭。松大合抱,亭世修治,至今谓之聘君亭也。赣水又北历南塘,塘之东有孺子宅,际湖南小洲上。孺子名稚,南昌人,高尚不仕。太尉黄琼辟,不就。桓帝问尚书令陈蕃:徐稚,袁闳谁为先后?蕃答称:袁生公族,不镂自雕。至于徐稚,杰出薄域,故宜为先。桓帝备礼征之,不至。太原郭林宗有母忧,稚往吊之,置生刍于庐前而去。众不知其故,林宗曰:必孺子也。《诗》云:生刍一束,其人如玉。吾无德以堪之。年七十二,卒。赣水又径谷鹿洲,即蓼子洲也,旧作大艑处。赣水又北径南昌县故城西,于春秋属楚,即令尹子荡师于豫章者也,秦以为庐江南部。汉高祖六年,始命陈婴定豫章置南昌县,以为豫章郡治,此即陈婴所筑也。王莽更名县曰宜善,郡曰九江焉:刘歆云:湖汉等九水入彭蠡,故言九江矣,陈蕃为太守,署徐稚为功曹。蕃在郡,不接宾客,惟稚来,特设一榻,去则悬之,此即悬榻处也。建安中,更名西安,晋又名为豫章。城之南门曰松阳门,门内有樟树,高七丈五尺,大二十五围,枝叶扶疏,垂荫数亩。应劭《汉官仪》曰:豫章樟树生庭中,故以名郡矣。此树尝中枯,逮晋永嘉中,一旦更茂,丰蔚如初,咸以为中宗之祥也。《礼斗威仪》曰:君政讼平,豫樟常为生。太兴中,元皇果兴大业于南。故郭景纯《南部赋》云:弊樟擢秀于祖邑是也。以宣王祖为豫章故也。赣水北出,际西北历度支步,是晋度支校尉立府处,步即水渚也。赣水又径郡北,为津步,步有故守贾萌庙,萌与安侯张普争地,为普所害,即日灵见津渚,故民为立庙焉。水之西岸有盘石,谓之石头,津步之处也。西行二十里曰散原山,叠嶂四周,杳邃有趣。晋隆安末,沙门竺昙显建精舍于山南,僧徒自远而至者相继焉。西北五六里,有洪井,飞流悬注,其深无底,旧说洪崖先生之井也。北五六里有风雨池,言山高濑激,激著树木,罪散远洒若雨。西有鸾冈,洪崖先生乘鸾所憩泊也。冈西有鹄岭,云王子乔控鹄所径过也。有二崖,号曰大萧、小萧,言萧史所游萃处也,雷次宗云:此乃系风捕影之论。据实本所未辨,聊记奇闻,以广井鱼之听矣。又按谢庄诗,庄常游豫章,观井赋诗。言鸾冈四周有水,谓之鸾陂。似非虚论矣。东大湖十里二百二十六步,北与城齐,南缘回折至甫塘,本通章江,增减与江水同。汉永元中,太守张躬筑塘以通南路,兼遏此水。冬夏不增减,水至清深,鱼甚肥美。每于夏月,江水溢塘而过,民居多被水害。至宋景平元年,太守蔡君西起堤,开塘为水门,水盛旱则闭之,内多则泄之。自是居民少患矣。赣水又东北径王步,步侧有城,云是孙奋为齐王镇此,城之。今谓之王步,盖齐王之渚步也。郡东南二十余里,又有一城,号曰齐王城。筑道相通,盖其离宫也。赣水又北径南昌左尉廨西,汉成帝时,九江梅福为南昌尉,居此。后福一旦舍妻子,去九江,传云得仙。赣水又北径龙沙西,沙甚洁白,高峻而阤,有龙形,连亘五里中,旧俗九月九日升高处也。昔有人于此沙得故冢刻砖,题云:西去江七里半,筮言其吉,卜言其凶。而今此冢垂没于水,所谓筮短龟长也。赣水又径椒丘城下,建安四年,孙策所筑也。赣水又历钓圻邸阁下。度支校尉治,太尉陶侃移置此也。旧夏月,邸阁前洲没,去浦远,景平元年,校尉豫章因运出之力,于渚次聚石为洲,长六十余丈。洲里可容数十舫。赣水又北径阳县,王莽之豫章县也。余水注之。永东出余汗县,王莽名之曰治干也。余水北至阳县注赣水。赣水又与鄱水合,水出鄱阳县东,西径其县南武阳乡也。地有黄金采,王莽改曰乡亭。孙权以建安十五年,分为鄱阳郡。鄱水又西流,注于赣。又有缭水入焉。其水导源建昌县,汉元帝永光二年分海昏立。缭水东径新吴县,汉中平中立。缭水又径海昏县。王莽更名宜生。谓之上缭水,又谓之海昏江,分为二水。县东津上有亭,为济渡之要。其水东北径昌邑城,而东出豫章大江,谓之慨口。昔汉昌邑王之封海昏也,每乘流东望,辄愤慨而还,世因名焉。其一水枝分别注,入于循水也。
又北过彭泽县西,循水出艾县西,东北径豫宁县,故西安也,晋太康元年更从今名。循水又东北径永循县,汉灵帝中平二年立。循水又东北注赣水,其水总纳十川,同臻一渎,俱注于彭蠡也。北入于江。
大江南,赣水总纳洪流,东西四十里,清潭远涨,绿波凝净,而会注于江川。
庐江水出三天子都,北过彭泽县西,北入于江。
《山海经》三天子都,一曰天子鄣。王彪之《庐山赋叙》曰:庐山,彭泽之山也。虽非五岳之数,穹隆嵯峨,实峻极之名山也。孙放《庐山赋》曰:寻阳郡南有庐山,九江之镇也。临彭蠡之泽,接平敞之原。《开山图》曰:山四方,周四百余里,叠鄣之岩万仞,怀灵抱异,苞诸仙迹。《豫章旧志》曰:庐俗,字君孝,本姓匡,父东野王,共鄱阳令吴芮佐汉定天下而亡。汉封俗于阳,曰越庐君。俗兄弟七人皆好道术,遂寓精于宫庭之山。故世谓之庐山。汉武帝南巡,睹山以为神灵,封俗大明公远法师。《庐山记》曰:殷、周之际,匡俗先生,受道仙人,共游此山,时人谓其所止为神仙之庐,因以名山矣。又按周景式曰:庐山匡俗,字子孝,本东里子,出周武王时,生而神灵,屡逃征聘,庐于此山,时人敬事之。俗后仙化,空庐犹存,弟子睹室悲哀,哭之旦暮,事同乌号。世称庐君,故山取号焉。斯耳传之谈,非实证也。故《豫章记》以庐为姓,因庐以氏,周氏远师,或托庐慕为辞,假凭庐以托称。二证既违,二情互爽。按《山海经》创之大禹,记录远矣。故《海内东经》曰:庐江出三天子都,入江彭泽西,是曰庐江之名,山水相依,互举殊称,明不因匡俗始,正是好事君子,强引此类,用成章句耳。又按张华《博物志·曹著传》,其神自云姓徐,受封庐山。后吴猛经过,山神迎猛,猛语曰:君王此山,近六百年,符命已尽,不宜久居,非据。猛又赠诗云:仰瞩列仙馆;俯察王神宅,旷载畅幽怀,倾盖付三益。此乃神道之事,亦有换转,理难详矣。吴猛,隐山得道者也。《寻阳记》曰:庐山上有三石梁,长数十丈,广不盈尺,杳然无底。吴猛将弟子登山,过此梁,见一翁坐桂树下,以玉杯承甘露浆与猛。又至一处,见数人,为猛设玉膏。猛弟子窈一宝,欲以来示世人,梁即化如指,猛使送宝还,手牵弟子,令闭眼相引而过。其山川明净,风泽清旷,气爽节和,土沃民逸。嘉遁之士,继响窟岩。龙潜风采之贤,往者忘归矣。秦始皇、汉武帝及太史公司马迁咸升其岩,望九江而眺钟、彭焉。庐山之北有石门水,水出岭端,有双石高竦,其状若门,因有石门之目焉。水导双石之中,悬流飞瀑,近三百许步,下散漫十许步,上望之连天,若曳飞练于霄中矣。下有磐石,可坐数十人。冠军将军刘敬宣,每登陟焉。其水历涧,径龙泉精舍南。太元中,沙门释慧远所建也。其水下入江。南岭,即彭蠡泽西天子鄣也。峰隥险峻,人迹罕及。岭南有大道,顺山而下,有若画焉。传云匡先生所通至江道。岩上有宫殿故基者三,以次而上,最上者极于山峰。山下又有神庙,号曰宫亭庙,故彭湖亦有宫亭之称焉。余按《尔雅》云:大山曰宫。宫之为名,盖起于此,不必一由三宫也。山庙甚神,能分风擘流,住舟遣使,行旅之人,过必敬祀而后得去。故曹毗咏云:分风为贰,擘流为两。昔吴郡太守张公直自守征还,道由庐山。子女观祠,婢指女戏妃像人。其妻夜梦致聘,怖而遽发,明引中流,而船不行。合船惊惧,曰:爱一女而合门受祸也。公直不忍,遂令妻下女于江。其妻布席水上,以其亡兄女代之,而船得进。公直方知兄女,怒妻曰:吾何面目于当世也。复下己女于水中。将渡,遥见二女于岸侧,傍有一吏立,曰:吾庐君主簿,敬君之义,悉还二女。故于宝书之于感应焉,山东有石镜,照水之所出。有一圆石,悬崖明净,照见人形。晨光初散,则延曜入石,豪细必察,故名石镜焉。又有二泉,常悬注,若白云带山。《庐山记》曰:白水在黄龙南,即瀑布也。水出山腹,挂流三四百丈,飞湍林表,望若悬素。注处悉成巨并,其深不测。其永下入江渊。庐山之南,有上霄石,高壁缅然,与霄汉连接。秦始皇三十六年,叹斯岳远,遂记为上霄焉。上霄之南,大禹刻石志其丈尺里数,今犹得刻石之号焉。湖中有落垦石,周回百分步,高五丈,上生竹木。传曰:有星坠此,因以名焉。又有孤石,介立大湖中,周回一里,竦立百丈,矗然高峻,特为瑰异。上生林木,而飞禽罕集,言其上有玉膏可采,所未详也。春旧云:昔禹治洪水至此,刻石纪功,或言秦始皇所勒,然岁月已久,莫能合辨之也。


卷四十  渐江水、斤江水 
江以南至日南郡二十水,禹贡山水泽地所在,江水出三天子都,《山海经》谓之浙江也。《地理志》云:水出丹阳黟县南蛮中,北径其县,南有博山,山上有石,特起十丈,上峰若剑杪。时有灵鼓潜发,正长临县,以山鼓为候,一鸣,官长一年,若长雷发声,则官长不吉。浙江又北历黟山,县居山之阳,故县氏之。汉成帝鸿嘉二年,以为广德国,封中山宪王孙云客王于此。晋太康中,以为广德县,分隶宣城郡。会稽陈业,洁身清行,遁迹此山。浙江又北径歙县东,与一小溪合。水出县东北翁山,西径故城南,又西南入浙江。又东径遂安县南。溪广二百步,上立杭以相通,水甚清深,潭不掩鳞,故名新定。分歙县立之。晋太康中,又改从今名。浙江又左合绝溪,溪水出始新县西,东径县故城南,为东西长溪。溪有四十七濑,濬流惊急,奔波聒天。孙权使贺齐讨黟、歙山贼,贼固黟之林历山,山甚峻绝,又工禁五兵。齐以铁杙椓山,升出不意,又以白棓击之,气禁不行,遂用奇功平贼。于是立始新之府于歙之华乡,令齐守之,后移出新亭。晋太康元年,改曰新安郡。溪水东注浙江。浙江又东北径建德县南。县北有乌山,山下有庙,庙在县东七里。庙渚有大石,高十丈,围五尺,水濑濬激而能致云雨。浙江又东径寿昌县南,自建德至此,八十里中有十二濑,濑皆峻险,行旅所难。县南有孝子夏先墓,先少丧二亲,负土成墓。数年不胜哀,卒。浙江又北径新城县,桐溪水注之。水出吴兴郡于潜县北天目山。山极高峻,崖岭竦叠,西临峻涧。山上有霜木,皆是数百年树,谓之翔凤林。东面有瀑布,下注数亩深沼,名曰浣龙池。他水南流径县西,为县之西溪。溪水又东南与紫溪合。水出县西百丈山,即潜山也。山水东南流,名为紫溪,中道夹水,有紫色磐石,石长百余丈,望之如朝霞。又名此水为赤濑,盖以倒影在水故也。紫溪又东南流,径白石山之阴。山甚峻极,北临紫溪。又东南连山夹水,两峰交峙,反项对石,往往相捍。十余里中,积石磊砢,相挟而上,涧下白沙细石,状若霜雪。水木相映,泉石争晖,名曰楼林。紫溪东南流,径桐庐县东为桐溪。孙权藉溪之名以为县目,割富春之地,立桐庐县。自县至于潜,凡十有六濑,第二是严陵濑。濑带山,山下有一石室。汉光武帝时,严子陵之所居也。故山及濑,皆即人姓名之。山下有磐石,周回十数丈,交枕潭际,盖陵所游也。桐溪又东北:径新城县入浙江。县故富春地,孙权置,后省并桐庐,咸和九年,复立为县。浙江又东北入富阳县,故富春也,晋后名春,改曰富阳也。东分为湖浦。浙江又东北径富春县南,县故王莽之诛岁也。江南有山,孙武皇之先所葬也。汉末,墓上有光,如云气属天。黄武五年,孙权以富春为东安郡,分置诸郡,以讨士宗。浙江又东北径亭山西,山上有孙权父冢。
北过余杭,东入于海。
浙江径县左,合余干大溪。江北即临安县界,水北对郭文宅,宅傍山面溪,宅东有郭文墓。晋建武元年,骠骑王导迎文,置之西园。文逃此而终,临安令改葬之。建武十六年,县民郎稚作乱,贺齐讨之。孙权分余杭,立临水县,晋改曰临安县,因冈为城,南门尤高。谢安莅郡游县,径此门,以为难为亭长。浙江又东径余杭故县南,新县北。秦始皇南游会稽,途出是地,因立为县。王莽之淮睦也。汉末陈浑移筑南城,县后溪南大塘,即浑立以防水也。县南有三碑,是顾扬、范宁等碑。县南有大壁山,郭文自陆浑迁居也。浙江又东径乌伤县北,王莽改曰乌孝,《郡国志》谓之乌伤。《异苑》曰:东阳颜乌以淳孝著闻,后有群乌助衔土块为坟,乌口皆伤,一境以为颜乌至孝,故致慈乌,欲令孝声远闻,又名其县曰乌伤矣。浙江又东北流至钱塘县,谷水入焉。水源西出太末县,县是越之西鄙,姑蔑之地也。秦以为县。王莽之末理也。吴宝鼎中,分会稽立,隶东阳郡。谷水东径独松故冢下,冢为水毁,其砖文:筮言吉,龟言凶,百年堕水中。今则同龟繇矣。谷水又东径长山县南,与永康溪水合,县即东阳郡治也。县,汉献帝分乌伤立;郡,吴宝鼎中分会稽置。城居山之阳,或谓之长仙县也,言赤松采药此山,因而居之,故以为名。后传呼乖谬,字亦因改。溪水南出永康县。县,赤乌中分乌伤上浦立。刘敬叔《异苑》曰:孙权时,永康县有人入山,遇一大龟,即束之以归。龟便言曰:游不量时,为君所得。担者怪之,载出,欲上吴王。夜宿越里,缆船于大桑树。宵中,树忽呼龟曰:元绪,奚事尔也?龟曰:行不择日,今方见烹,虽尽南山之樵,不能溃我。树曰:诸葛元逊识性渊长,必致相困。令求如我之徒,计将安治?龟曰:子明,无多辞。既至建业,权将煮之,烧柴万车,龟犹如故。诸葛格曰:燃以老桑乃熟。献人仍说龟言,权使伐桑,取煮之即烂。故野人呼龟曰元绪。其水飞湍北注,至县南门,入谷水。谷水又东,定阳溪水注之。水上承信安县之苏姥布。县本新安县,晋武帝太康三年改曰信安。水悬百余丈,濑势飞注,状如瀑布。濑边有石如床,床上有石牒,长三尺许,有似杂采帖也。《东阳记》云:信安县有悬室坂。晋中朝时,有民王质,伐木至石室中,见童子四人,弹琴而歌。质因留,倚柯听之。童子以一物如枣核与质,质含之,便不复饥。俄顷,童子曰:其归。承声而去,斧柯漼然烂尽。既归,质去家已数十年,亲情凋落,无复向时比矣。其水分纳众流,混彼东逝,径定阳县。夹岸缘溪,悉生支竹,及芳枳木连,杂以霜菊金橙。白沙细石,状如凝雪。石溜湍波,浮响无辍。山水之趣,尤深人情。县,汉献帝分信安立,溪亦取名焉。溪永又东径长山县北,北对高山。山下水际,是赤松羽化之处也。炎帝少女追之,亦俱仙矣。后人立庙于山下。溪水又东入于谷水,谷水又东径乌伤县之云黄山,山下临溪水,水际石壁杰立,高百许丈。又与吴宁溪水合。水出吴宁县下,径乌伤县入谷,谓之乌伤溪水。闽中有徐登者,女子化为丈夫,与东阳赵昞,并善越方,时遭兵乱,相遇于溪,各示所能。登先禁溪,水为不流。昞次禁枯柳,柳为生荑。二人相示而笑。登年长,昞师事之。后登身故,昞东入章安,百姓未知;昞乃升茅屋,梧鼎而爨、主人惊怪,昞笑而不应,屋亦不损。又尝临水求渡,船人不许。昞乃张盖坐中,长啸呼风,乱流而济。于是百姓神服,从者如归,章安令恶而杀之,民立祠于永宁,而蚊蚋不能入。昞秉道怀术,而不能全身避害,事伺苌弘,宋元之龟,厄运之来,故难救矣。谷水又东入钱唐县,而左入浙江。故《地理志》曰:谷水自太末东北至钱唐入浙江是也。浙江又东径灵隐山,山在四山之中,有高崖洞穴,左右有石室三所,又有孤石壁立,大三十围,其上开散,状如莲花。昔有道士,长往不归,或因以稽留为山号。山下有钱唐故县。浙江径其南,王莽更名之曰泉亭。《地理志》曰:会稽西部都尉治。《钱唐记》曰:防海大塘在县东一里许,郡议曹华信家议立此塘,以防海水。始开募,有能致一斜土者,即与钱一千。旬月之间,来者云集,塘未成而不复取。于是载土石者皆弃而去,塘以之成,故改名钱塘焉。县南江侧有明圣湖。父老传言,湖有金牛,古见之,神化不测,湖取名焉。县有武林山,武林水所出也。阚駰云:山出钱水,东入海。《吴地记》言,县惟浙江,今无此水。县东有定、包诸山,皆西临浙江。水流于两山之间,江川急浚,兼涛水昼夜再来,来应时刻,常以月晦及望尤大,至二月、八月最高,峨峨二丈有余。《吴越春秋》以为子胥、文种之神也。昔子胥亮于吴,而浮尸于江。吴人怜之,立祠于江上,名曰胥山。《吴录》云:胥山在太湖边,去江不百里,故曰江上。文种诚于越,而伏剑于山阴,越人哀之,葬于重山。文种既葬一年,子胥从海上负种俱去,游夫江海。故潮水之前扬波者,伍子胥,后重水者,大夫种。是以枚乘曰:涛无记焉。然海水上潮,江水逆流,似神而非,于是处焉。秦始皇三十七年,将游会稽,至钱唐,临浙江,所不能渡,故道余杭之西津也。浙江北合诏息湖,湖本名阼湖,因秦始皇帝巡狩所憩,故有诏息之名也。浙江又东合临平湖。《异苑》曰:晋武时,吴郡临平岸崩,出一百鼓,打之无声,以问张华。华云:可取蜀中桐材,刻作鱼形,扣之则鸣矣。于是如言,声闻数十里。刘道民诗曰:事有远而合,蜀桐鸣吴石。传言此湖草壅塞,天下乱,是湖开,天下平。孙皓天玺元年,吴郡上言,临平湖自汉末秽塞,今更开通。又于湖边得石函,函中有小石,青白色,长四寸,广二寸余,刻作皇帝字。于是改天册为天玺元年。孙盛以为元皇中兴之符征,五湖之石瑞也。《钱唐记》曰:桓玄之难,湖水色赤,荧荧如丹。湖水上通浦阳江,下注浙江,名曰东江,行旅所从以出浙江也。浙江又径固陵城北,昔范蠡筑城于浙江之滨,言可以固守,谓之固陵,今之西陵也。浙江又东径柤塘,谓之柤渎。昔太守王朗拒孙策,数战不利。孙静果说策曰:朗负阻城守,难可卒拔。柤渎去此数十里,是要道也。若从此出,攻其无备。破之必矣。策从之,破朗于涸陵。有西陵湖,亦谓之西城湖。湖西有湖城山,东有夏架山。湖水上承妖皋溪而下注浙江。又径会稽山阴县,有苦竹里,里有旧城,言句践封范蠡子之邑也。浙江又东与兰溪合,湖南有天柱山,湖口有亭,号曰兰亭,亦曰兰上里。太守王羲之、谢安兄弟,数往造焉。吴郡太守谢勖封兰亭侯,盖取此亭以为封号也。太守王羲之,移亭在水中。晋司空何无忌之临郡也,起亭于山椒,极高尽眺矣。亭宇虽坏,基陛尚存。浙江又径越王允常冢北,冢在木客村。耆彦云:句践使工人伐荣楯。欲以献吴,久不得归,工人忧思,作《木客吟》。后人因以名地。句践都琅邪,欲移允常冢,冢中生分风,飞沙射人,人不得近。句践谓不欲,遂止。浙江又东北得长湖口,湖广五里,东西百三十里,沿湖开水门六十九所,下溉田万顷,北泻长江。湖南有覆斗山,周五百里,北连鼓吹山,山西枕长溪,溪水下注长湖。山之西岭有贺台,越入吴,还而成之,故号曰贺台矣。又有秦望山,在州城正南,为众峰之杰,陟境便见。《史记》云:秦始皇登之以望南海。自平地以取山顶七里,悬隥孤危,径路险绝。《记》云:扳萝扪葛,然后能升。山上无甚高木,当由地迥多风所致。山南有嶕岘,岘里有大城,越王无余之旧都也。故《吴越春秋》云:句践语范蠡曰:先君无余,国在南山之阳,社稷宗庙在湖之南。又有会稽之山,古防山也,亦谓之为茅山,又曰栋山。《越绝》云:栋犹镇也。盖《周礼》所谓扬州之镇矣。山形四方,上多金玉,下多玦石。《山海经》曰:夕水出焉,南流注于湖。《吴越春秋》称覆釜山之中,有金简玉字之书,黄帝之遗谶也。山下有禹庙,庙有圣姑像。《礼乐纬》云:禹治水毕,天赐神女圣姑,即其像也。山上有禹冢。昔大禹即位十年,东巡狩,崩于会稽,因而葬之。有鸟来为之耘,春拔草根,秋啄其秽,是以县官禁民不得妄害此鸟,犯则刑无赦。山东有湮井,去庙七里,深不见底,谓之禹井,云东游者多探其穴也。秦始皇登会稽山刻石纪功,尚存山侧,孙畅之《述书》云:丞相李斯所篆也。又有石匮山,石形似匮,上有金简玉字之书,言夏禹发之,得百川之理也。又有射的山,远望山的状若射侯,故谓射的。射的之西,有石室,名之为射堂。年登否常占射的。以为贵贱之准。的明则米贱,的暗则米贵,故谚云:射的白,斛米百,射的玄,斛米千。北则石帆山,山东北有孤石,高二十余丈,广八丈,望之如帆,因以为名。北临大湖,水深不测,传与海通。何次道作郡,常于此水中得乌贼鱼。南对精庐,上荫修木,下瞰寒泉。西连会稽山,皆一山也。东带若邪溪,《吴越春秋》所谓欧冶涸而出铜,以成五剑。溪水上承嶕岘麻溪,溪之下,孤潭,周数亩,甚清深,有孤石临潭。乘崖俯视,猿狖惊心,寒木被潭,森沉骇观。上有一栎树,谢灵运与从弟惠连常游之,作连句,题刻树侧。麻潭下注若邪溪,水至清照,众山倒影,窥之如画。汉世刘宠作郡,有政绩,将解任去治,此溪父老,人持百钱出送,宠各受一文。然山栖遁逸之士,谷隐不羁之民,有道则见,物以感远为贵,荷钱致意,故受者以一钱为荣,岂藉费也,义重故耳。溪水下注大湖。邪溪之东,又有寒溪,溪之北有郑公泉,泉方数丈,冬温夏凉。汉太尉郑弘宿居潭侧,因以名泉。弘少以苦节自居,恒躬采伐,用贸粮膳。每出入溪津,常感神风送之,虽凭舟自运,无杖楫之劳。村人贪藉风势,常依随往还,有淹留者,徒辈相谓,汝不欲及郑风邪?其感致如此。湖水自东亦往江通海,水侧有白鹿山。山北湖塘上旧有亭,吴黄门郎杨哀明居于弘训里,太守张景数往造焉,使开渎作埭,埭之西作亭,亭、埭皆以杨为名。孙恩作贼,从海来,杨亭被烧,后复修立,厥名犹在。东有铜牛山,山有铜穴,三十许丈,穴中有大树神庙。山上有冶宫,山北湖下有练塘里。《吴越春秋》云:旬践练冶铜锡之处。采炭于南山,故其间有炭渎。句践臣吴,吴王封句践于越百里之地,东至炭渎是也。县南九里有侯山,山孤立长湖中。晋车骑将军孔敬康少时,遁世栖迹此山。湖北有三小山,谓之鹿野山,在县南六里,按《吴越春秋》,越之麋苑也。山有石室,言越王所游息处矣。县南湖北有陈音山。楚之善射者曰陈音,越王问以射道,又善其说,乃使简士习射北郊之外。按《吴越春秋》,音死,葬于国西山上。今陈音山乃在国南五里。湖北有射堂及诸邸舍,连衍相属。又于湖中筑塘,直指南山,北即大越之国。秦改为山阴县,会稽郡治也。太史公曰:禹会诸侯计于此,命曰会槽。会稽者,会计也。始以山名,因为地号。夏后少康封少子杼以奉禹祠为越。世历殷、周,至于允常,列于《春秋》。允常卒,句践称王,都于会稽。《吴越春秋》所谓越王都埤中,在诸暨北界。山阴康乐里有地名邑中者,是越事吴处。故北其门,以东为右,西为左,故双阙在北门外。阙北百步有雷门,门楼两层,句践所造,时有越之旧木矣。州郡馆宇,屋之大瓦,亦多是越时故物。句践霸世,徙都琅邪,后为楚伐,始还浙东。城东郭外有灵汜,下水甚深,旧传下有地道,通于震泽。又有句践所立宗庙,在城东明里中甘滂南。又有王笥、竹林、云门、天柱精舍,并疏山创基,架林裁宇,割涧延流,尽泉石之好,水流径通。浙江又北径山阴县西。西门外百余步有怪山,本琅邪郡之东武县山也,飞来徙此,压杀数百家。《吴越春秋》称怪山者,东武海中山也,一名自来山,百姓怪之,号曰怪山。亦云:越王无疆为楚所伐,去琅邪,止东武,人随居山下。远望此山,其形似龟,故亦有龟山之称也。越起灵台于山上,又作三层楼以望云物。川土明秀,亦为胜地。故王逸少云:从山阴道上,犹如镜中行也。浙江之上,又有大吴王、小吴王村,并是阖闾、夫差伐越所舍处也。今悉民居,然犹存故目。昔越王为吴所败,以五千余众,栖于稽山,卑身待士,施必及下。《吕氏春秋》曰:越王之栖于会稽也,有酒投江,民饮其流,而战气自倍。所投即浙江也。许慎、晋灼并言江水至山阴为浙江。江之西岸有朱室坞,句践百里之封,西至朱室,谓此也。浙江又东北径重山西,大夫文种之所葬也。山上有白楼亭,亭本在山下,县令殷朗移置今处。沛国桓俨,避地会稽,闻陈业履行高洁,往候不见,俨后浮海,南入交州,临去,遗书与业,不因行李系白楼亭柱而去。升陟远望,山湖满目也。永建中,阳羡周嘉上书,以县远赴会至难,求得分置,遂以浙江西为吴,以东为会稽。汉高帝十二年,一吴也,后分为二,世号三吴,吴兴、吴郡,会稽其一焉。浙江又东径御儿乡,《万善历》曰:吴黄武六年正月,获彭绮。是岁,由拳西乡有产儿,堕地便能语,云:天方明,河欲清。鼎脚折,金乃生。因是诏为语儿乡。非也。御儿之名远矣,盖无智之徒,因藉地名,生情穿凿耳。《国语》曰:句践之地,北至御儿是也。安得引黄武证地哉?韦昭曰:越北鄙在嘉兴。浙江又东径柴辟南,旧吴楚之战地矣。备候于此,故谓之辟塞,是以《越绝》称吴故从由拳、辟塞渡会稽,凑山阴是也。又径永兴县北,县在会稽东北百二十里,故余暨县也。应劭曰:阖闾弟夫概之所邑。王莽之余衍也。汉末,童谣云:天子当兴东南三余之间。故孙权改曰永兴。县滨浙江,又东合浦阳江,江水导源乌伤县,东径诸暨县,与泄溪合。溪广数丈,中道有两高山夹溪,造云壁立,凡有五泄。下泄悬三十余丈,广十丈,中三泄不可得至,登山远望,乃得见之。悬百余丈,水势高急,声震水外。上泄悬二百余丈,望若云垂,此是瀑布,土人号为泄也。江水又东径诸暨县南,县临对江流,江南有射堂。县北带乌山,故越地也。先名上诸暨,亦曰何无矣。故《国语》曰:句践之地,南至句无。王莽之疏虏也。夹水多浦,浦中有大湖,春夏多永,秋冬涸浅。江水又东南径剡县,与白石山水会。山上有瀑布,悬水三十丈,下注浦阳江。浦阳江水又东流南屈,又东回北转,径剡县东,王莽之尽忠也。县开东门向江,江广二百余步,自昔耆旧传,县不得开南门,开南门则有贼盗。江水翼县转注,故有东渡、西汲焉。东南二渡,通临海,并泛单船为浮航。西渡通东阳,并二十五船为桥航。江边有查浦,浦东行二百余里,与句章接界。浦里有六里,有五百家,并夹浦居,列门向水,甚有良田。有青溪、余洪溪、大发溪、小发溪,江上有溪,六溪列溉,散入江。夹溪上下,崩崖若倾,东有簟山、南有黄山,与白石三山,为县之秀峰。山下众流泉导,湍石激波,浮险四注。浦阳江又东径石桥,广八丈,高四丈。下有石井,口径七尺,桥上有方石,长七尺,广一丈二尺。桥头有磐石,可容二十人坐。溪水两旁悉高山,山有石壁二十许丈,溪中相攻,赑响外发,未至桥数里,便闻其声。江水北径嵊山,山下有亭,亭带山临江,松岭森蔚,沙渚平静。浦阳江又东北径始宁县山之成功峤。峤壁立临江,欹路峻狭,不得并行。行者牵木稍进,不敢俯视。峤酉有山,孤峰特上,飞禽罕至。尝有采药者,沿山见通溪,寻上,于山顶树下,有十二方石,地甚光洁。还复更寻,遂迷前路。言诸仙之所憩宴,故以坛宴名山。峤北有浦,浦口有庙,庙甚灵验,行人及樵伐者皆先敬焉。若相侵窃,必为蛇虎所伤。北则山与山接,二山虽曰异县,而峰岭相连。其间倾涧怀烟,泉溪引雾,吹畦风馨,触岫延赏。是以王元琳谓之神明境,事备谢康乐《山居记》。浦阳江自山东北,径太康湖,车骑将军谢玄田居所在。右滨长江,左傍连山,平陵修通,澄湖远镜。于江曲起楼,楼侧悉是桐梓,森耸可爱,居民号为桐亭楼,楼两面临江,尽升眺之趣。芦人渔子,泛滥满焉。湖中筑路,东出趋山,路甚平直。山中有三精舍,高甍凌虚、垂檐带空,俯眺平林,烟沓在下,水陆宁晏,足为避地之乡矣。江有琵琶圻,圻有古冢堕水,甓有隐起字云:筮吉龟凶,八百年,落江中。谢灵运取甓诣京,咸传观焉。乃如龟繇,故知冢已八百年矣。浦阳江又东北径始宁县西,本上虞之南乡也。汉顺帝永建四年,阳羡周嘉上书,始分之。旧治水西,常有波潮之患。晋中兴之初,治今处。县下有小江,源出山,谓之浦,径县下西流注于浦阳山,下注此浦。浦西通山阴浦而达于江。江广百丈,狭处二百步,高山带江,重荫被水,江阅渔商,川交樵隐,故桂掉兰枻,望景争途。江南有故城,大尉刘牢之讨孙恩所筑也。江水东径上虞县南,王莽之会稽也。本司盐都尉治,地名虞宾。《晋太康地记》曰:舜避丹朱于此,故以名县,百官从之,故县北有百官桥。亦云:禹与诸侯会事讫,因相虞乐,故曰上虞。二说不同,未详孰是?县南有兰风山,山少木多石,驿路带山;傍江路边皆作栏干。山有三岭,枕带长江,苕苕孤危,望之若倾。缘山之路,下临大川,皆作飞阁栏干,乘之而渡,谓此三岭为三石头。丹阳葛洪遁世居之,基井存焉。琅邪王方平性好山水,又爱宅兰风,垂钓于此,以永终朝。行者过之,不识,问曰:卖鱼师,得鱼卖否?方平答曰:钓亦不得,得复不卖。亦谓是水为上虞江。县之东郭外有渔浦湖,中有大独、小独二山。又有覆舟山。覆舟山下有渔浦王庙,庙今移入里山。此三山孤立水中。猢外有青山、黄山、泽兰山,重岫叠岭,参差入云。泽兰山头有深潭,山影临水,水色青绿。山中有诸坞,有石健一所,右临白马潭。潭之深无底,传云创湖之始,边塘屡崩,百姓以白马祭之,因以名水。湖之南,即江津也。江南有上塘、阳中二里。隔在湖南,常有水患。太守孔灵符遏蜂山前湖以为埭。埭下开渎,直指南津。又作水楗二所以舍,此江得无淹溃之害。县东有龙头山,山崖之间有石井,冬夏常冽清泉,南带长江,东连上陂。江之道南有《曹娥碑》。娥父旴,迎涛溺死,娥时年十四,哀父尸不得,乃号踊江介,因解衣投水,祝曰:若值父尸,衣当沉。若不值,衣当浮。裁落便沉,娥遂于沉处赴水而死。县令度尚,使外甥邯郸子礼为碑文,以彰孝烈。江滨有马目山,洪涛一上,波隐是山,势沦嵊亭,间历数县,行者难之。县东北上,亦有孝子杨威母墓。威少失父,事母至孝,常与母入山采薪,为虎所逼,自计不能御,于是抱母,且号且行,虎见其情,遂弭耳而去。自非诚贯精微,孰能理感于英兽矣。又有吴渎,破山导源,注于肯江。上虞江东径周市而注永兴。《地理志》云:县有仇亭,柯水东入海。仇亭在县之东北十里江北,柯水疑即江也。又东北径永兴县东与浙江合,谓之浦阳江。《地理志》又云:县有萧山,潘水所出,东入海。又疑是浦阳江之别名也,自外无水以应之。浙江又东注于海。故《山海经》曰:浙江在闽西北入海。韦昭以松江、浙江、浦阳江为三江。
斤江水出交阯龙编县,东北至郁林领方县,东注于郁。
《地理志》云:径临尘县至领方县,注于郁。
容容、夜、湛、乘、牛渚、须无、无濡、营进、皇无、地零、侵离、侵离水出广州晋兴郡,郡以太康中分郁林置。东至临尘,入郁。
无会、重濑、夫省、无变、由蒲、王都、融、勇外,此皆出日南郡西,东入于海。容容水在南垂,名之,以次转北也。
右二十水,从江已南至日南郡也。
嵩高为中岳,在颖川阳城县西北。
《春秋说题辞》曰:阴含阳,故石凝为山。《国语》曰:禹封九山,山,土之聚也。《尔雅》曰:山大而高曰崧。合而言之为崧高,分而名之为二室,西南有少室,东北有太室。《嵩高山记》曰:山下岩中有一石室,云有自然经书,自然饮食。又云:山有玉女台,言汉武帝尝见之,因以名台。
泰山为东岳,在泰山博县西北。
岱宗也。王者封禅于其山,示增高也。有金策玉检之事焉。霍山为南岳,在庐江灊县西南。
天柱山也。《尔雅》云:大山宫,小山为霍,《开山图》曰:其山上侵神气,下固穷泉。
华山为西岳,在弘农华阴县西南。
《古文》之惇物山也。
雷首山在河东蒲坂县东南。
砥柱山在河东大阳县东河中。
王屋山在河东垣县东北也。
昔黄帝受丹诀于是山也。
大行山在河内野王县西北。
王烈得石髓处也。
恒山为北岳,在中山上曲阳县西北。
碣石山在辽西临渝县南水中也。
大禹凿其石,夹右而纳河,秦始皇、汉武帝皆尝登之。海水西侵,岁月逾甚,而苞其山,故言水中矣。
析城山在河东濩泽县西南。
大岳山在河东永安县。
壶口山在河东北屈县东南。
龙门山在河东皮氏县西。
梁山在冯翊夏阳县西北河上。
荆山在冯翊怀德县南。
岐山在扶风美阳县西北。
汧山在扶风汧县之西也。
陇山、终南山、惇物山在扶风武功县西南也。
西倾山在陇西临洮县西南。
《禹贡》中条山也。
.冢山在陇西氐道县之南。
南条山也。
鸟鼠同穴山在陇西首阳县西南。
郑玄曰:鸟鼠之山有鸟焉,与鸟飞行而处之。又有止而同穴之山焉,是二山也。鸟名为,似鸡而黄黑色,鼠如家鼠而短尾,穿地而共处,鼠内而鸟外。孔安国曰:共为雌雄。杜彦达曰:同穴止宿,养子互相哺食,长大乃止。张晏言不相为牝牡,故因以名山。
积石在陇西河关县西南。
《山海经》云:山在邓林东,河所入也。
都野泽在武威县东北。
县在姑臧城北三百里,东北即休屠泽也,《古文》以为猪野也。其水上承姑臧武始泽,泽水二源,东北流为一水,径姑臧县故城西,东北流,水侧有灵渊池。王隐《晋书》曰:汉末,博士燉煌侯瑾,善内学,语弟子曰:凉州城西,泉水当竭,有双阙起其上。至魏嘉平中,武威太守条茂,起学舍,筑阙于此泉。太守填水,造起门楼,与学阙相望。泉源徙发,重导于斯,故有灵渊之名也。泽水又东北流径马城东,城即休屠县之故城也。本匈奴休屠王都。谓之马城河,又东北与横水合,水出姑臧城下,武威郡,凉州治。《地理风俗记》曰:汉武帝元朔三年,改雍曰凉州,以其金行,土地寒凉故也。迁于冀,晋徙治此。王隐《晋书》曰:凉州有龙形,故曰卧龙城。南北七里,东西三里,本匈奴所筑也。及张氏之世居也,又增筑四城,箱各千步,东城殖园果,命曰讲武场,北城殖园果,命曰玄武圃,皆有宫殿。中城内作四时宫,随节游幸,并旧城为五。街衢相通,二十二门。大缮官殿观阁,采绮妆饰,拟中夏也。其水侧城北流,注马城河。河水又东北,清涧水入焉,俗亦谓之为五涧水也。水出姑臧城东,而西北流注马城河。河水又与长泉水合,水出姑臧东揟次县,王莽之播德也。西北历黄沙阜,而东北流注马城河,又东北径宣成县故城南,又东北径平泽、晏然二亭东,又东北径武威县故城东。汉武帝太初四年,匈奴浑邪玉杀休屠王,以其众置武威县,武威郡治。王莽更名张掖。《地理志》曰:谷水出姑臧南山,北至武威入海。届此水流两分,一水北入休屠泽,俗谓之为西海;一水又东径百五十里入猪野,世谓之东海,通谓之都野矣。
合离山在酒泉会水县东北。
合黎山也。
流沙地在张掖居延县东北。
居延泽在其县故城东北,《尚书》所谓流沙者也,形如月生五日也。弱水入流沙,流沙,沙与水流行也。亦言出钟山,西行极崦嵫之山,在西海郡北。山有石赤白色,以两石相打,则水润。打之不已,润尽则火出,山石皆然,炎起数丈,径日不灭。有大黑风,自流沙出奄之,乃灭,其石如初。言动火之事,发疾经年,放不敢轻近耳。流沙又径浮渚,历壑市之国,又径于鸟山之东,朝云国西,历昆山西南,出于过瀛之山。《大荒西经》云:西南海之外,流沙出焉,径夏后开之东,开上三嫔于天,得《九辩》与《九歌》焉。又历员丘不死山之西,入于南海。
三危山在燉煌县南。
《山海经》曰:三危之山,三青鸟居之。是山也,广圆百里,在鸟鼠山西,即《尚书》所谓窜三苗于三危也。《春秋传》曰:允姓之奸,居于瓜州。瓜州,地名也。杜林曰:燉煌,古瓜州也。州之贡物,地出好瓜,民因氏之。瓜州之戎并于月氏者也。汉武帝元鼎六年,分酒泉置,南七里有鸣沙山,故亦曰沙州也。
朱圉山在天水北,冀城南。
即冀县山,有石鼓,《开山图》谓之天鼓山。九州害起则鸣,有常应。
又云:石鼓山有石鼓,于星为河鼓,星动则石鼓鸣,石鼓鸣则秦土有殃。鸣浅殃万物,鸣深则殃君王矣。岷山在蜀郡湔氐道西。
《汉书》以为渎山者也。
熊耳山在弘农卢氏县东。
是山也,谷水出其北林也。
荆山在南郡临沮县东北。
东条山也。卞和得玉璞于是山,楚王不理,怀璧哭于其下,王后使玉人理之,所谓和氏之玉焉。
内方山在江夏竟陵县东北。
《禹贡注》:章山也。
大别山在庐江安丰县西南。
外方山,崧高是也。
桐柏山在南阳平氏县东南。
陪尾山在江夏安陆县东北。
衡山在长沙湘南县南。
禹治洪水,血马祭衡山,于是得金简玉字之书。按省玉字,得通水理也。九江地在长沙下巂县西北。
云梦泽在南郡华容县之东。
东陵地在庐江金兰县西北。
敷浅原地在豫章历陵县西。
彭蠡泽在豫章彭泽县西北。
《尚书》所谓彭蠡既猪,阳鸟攸居也。
中江在丹阳芜湖县西南,东至会稽阳羡县,入于海。震泽在吴县南五十里。
北江在毗陵北界,东入于海。
峄阳山在下邳县之西。羽山在东海祝其县南也。
县即王莽之犹亭也。《尚书》殛鲧于羽山,谓是山也。山西有羽渊,禹父之所化,其神为黄熊以入渊矣。故《山海经》曰:洪水滔天,鲧窃帝之息壤以堙水,不待帝命。帝令祝融杀鲧羽郊者也。
陶丘在济阴定陶县之西南。
陶丘,丘再成也。
菏泽在定陶县东。
雷泽在济阴成阳县西北。
菏水在山阳湖陆县南。
蒙山在太山蒙阴县西南。
大野泽在山阳巨野县东北。
大邳地在河南成皋县北。
《尔雅》曰:山一成谓之邳,然则大邳山名,非地之名也。明都泽在梁郡睢阳县东北。
益州沱水在蜀郡汶江县西南。其一在郫县西南,皆还入江。荆州沱水在南郡枝江县。
三澨地在南郡邔县北沱。
《尚书》曰:导汉水,过三澨。《地说》曰:沔水东行,过三澨合流,触大别山阪。故马融、郑玄、王肃。孔安国等,咸以为三澨,水名也。许慎言:澨者,埤增水边土,人所止也。按《春秋左传》:文公十有六年,楚军次于句澨,以伐诸庸。宣公四年,楚令尹子越师于漳澨。定公四年,左司马戍败吴师于雍澨。昭公二十三年,司马薳越缢于薳澨。
服虔或谓之邑,又谓之地。京相璠、杜预亦云:本际及边地名也。今南阳、淯阳二县之间,淯水之滨,有南澨、北澨矣。而诸儒之论,水陆相半,又无山源出处之所,津途关路,惟郑玄及刘澄之言在竟陵县界。《经》云邔县北沱,然沱流多矣,论者疑焉,而不能辨其所在。
右《禹贡》山水泽地所在,凡六十。

(全文完)
\chapter{1}
\section{1}
\section{2}

\backmatter

\end{document}