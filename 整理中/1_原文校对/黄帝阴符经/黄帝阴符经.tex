% 黄帝阴符经
% 黄帝阴符经.tex

\documentclass[12pt,UTF8]{ctexbook}

% 设置纸张信息。
\usepackage[a4paper,twoside]{geometry}
\geometry{
	left=25mm,
	right=25mm,
	bottom=25.4mm,
	bindingoffset=10mm
}

% 设置字体,并解决显示难检字问题。
\xeCJKsetup{AutoFallBack=true}
\setCJKmainfont{SimSun}[BoldFont=SimHei, ItalicFont=KaiTi, FallBack=SimSun-ExtB]

% 目录 chapter 级别加点(.)。
\usepackage{titletoc}
\titlecontents{chapter}[0pt]{\vspace{3mm}\bf\addvspace{2pt}\filright}{\contentspush{\thecontentslabel\hspace{0.8em}}}{}{\titlerule*[8pt]{.}\contentspage}

% 设置 part 和 chapter 标题格式。
\ctexset{
	chapter/name={},
	chapter/number={},
	section/name={},
	section/number={}
}

% 设置古文原文格式。
\newenvironment{yuanwen}{\bfseries\zihao{4}}

% 设置署名格式。
\newenvironment{shuming}{\hfill\bfseries\zihao{4}}

% 注脚每页重新编号,避免编号过大。
\usepackage[perpage]{footmisc}

\title{\heiti\zihao{0} 黄帝阴符经}
\author{}
\date{}

\begin{document}

\maketitle
\tableofcontents

\frontmatter
\chapter{前言}

十二家注《黄帝阴符经》早称太公等《阴符经注》,与道藏本《黄帝阴符经集注》均相同,初称《阴符》、《黄帝天机之书》,旧题黄帝撰;《百子全书》8 中所辑《阴符经》,未题著作人,仅题汉张良注,内容完全相同。

《黄帝阴符经》全书一卷三篇,共有四百六十一个字,注解者有伊尹、太公、范蠡、鬼谷子、张良、诸葛亮、李筌,所以有“七家注《阴符经》”之说,实际上不止七家,还有广成子、老子、孙子、吴子、鹖冠子等五家,这就不是七家注而是十二家注了。

四库全书本收辑的太公、范蠡、鬼谷子、张良、诸葛亮、李筌等六家注《阴符经》,是以道藏本《黄帝阴符经集注》为原本的,但书名则删去“黄帝”二字,注者删去“伊尹”,对广成子、老子、孙子、吴子、鹖冠子这五家只承认其注与姓名,不承认其是注的一家,因之又有六家注《阴符经》之说。其实是十二家作注的。

四库本删去“黄帝”二字是不妥当的。书名有黄帝二字,并不意味着《阴符经》就是黄帝修撰的。中华民族尊重黄帝为白己的始祖,主要是尊重他的政道、治道、兵道的智谋韬略思想,后代人许多著作都托名于黄帝,其根本原因就在这里。《阴符经》是黄帝奠定雏形的,成书以后,托名于他,有何不可?如果删去黄帝二字,那么《阴符经》的雏形究竟是谁奠定的,以后又根据什么雏形来成书的问题就尖锐起来了,四库本何以解释?我认为不要删去黄帝二字为好,否则连一个历代都公认的黄帝为书名也保不住,而且还给研究者增加了困难。

四库本删去伊尹为注家之一也是不妥当的。伊尹是辅商灭夏的政治家、军事家,他的政道、治道、兵道均较突出,并且已有甲骨文字记事记言,他为《阴符经》用甲骨文字作注是最早的一家。保留伊尹这一家,总是比删去的好!

四库本称《阴符经》为六家注,道藏本称为七家注,就是不把广成子、老子、孙子、吴子、鹖冠子这五家注放在眼里,这就更不妥当了。广成子是上古时人,空同氏部落领袖,黄帝曾向他问过至道,他授给黄帝“自成之经”(这不是指书,而是指经典式的语言);老子、孙子都是春秋时人,老子著有《老子》,孙子著有《孙子》,老子是大思想家、哲学家,孙子是大军事家、兵学家;吴子、鹖冠子都是战国
时人,吴子著有《吴子》,鹖冠子著有《鹖冠子》。吴子是大军事家、政治改革家,鹖冠子既是思想家,又是哲学家。这五大家的思想言论为《阴符经》作注,岂不更能挖掘《阴符经》的本义,有何不可!书中有他们的注,却不承认他们是注的一家,真是岂有此理!可能有人会说:如果把广成子列为注解的一家,那么这不就是说,在他之前就已有《阴符经》吗?不错,在他之前是没有《阴符经》的,根本就没有这个条件;但是作为谋略思想的言论还是有的。《阴符经》后来成书,用他的思想言论补上去作注,并承认他是作注的一家,我看没有什么不对!其他各家的注和承认他们是作注的一家,不都是这样的吗?请问四库本、道藏本的先生们,你们有谁能同这五家相提并论?
我认为要承认这五大家都是作注的最好。

道藏本的前面有《序》,并标明《序》的作者是“蜀相诸葛亮撰”,四库本收辑时干脆删掉,这也是不妥当的。诸葛亮为《阴符经》作注,又写这篇序,是有感而发,并有条件办得的。例如:序中有“……观乎《阴符》造化在乎手,生死在乎人。故圣人藏之于心,所以陶甄天地,聚散天下,而不见其迹者,天机也。”这恰是对《阴符》的解释!他又说:“故黄帝得之以登云天,汤武得之以王天下,五霸得之以统诸侯。夫臣易而主难,不可以轻用。太公九十非不遇,盖审其主焉。若使哲士执而用之,立石为主,刻木为君,亦可以享天下。
夫臣尽其心,而主反怖有之·不亦难乎?”诸葛亮辅佐刘备,创立蜀
国,使天下三分;辅佐刘禅,继续贯彻了:东和孙吴,北拒曹魏、南
抚夷越,西和羌戎。刘禅做了什么事呢?岂不是“立石为主,刻木为
君,亦可以享天下”!说他有条件办得到的,主要他的政道、治道、兵
道等主要活动和当时的文字与书写工具。所以序中又说了:“……范
蠡重而长,文种轻而亡,岂不为泄天机?天机泄者沉三劫,宜然。故
圣人藏诸名山,传之同好,隐之金匮,恐小人窃而弄之”。有人抓住
“天机泄者沉三劫”,“藏诸名山,传之同好”这两句,硬说是李筌作
伪的。认为前一句是李筌写《黄帝阴符经疏序》中曾表达过的思想,
即与“小人学之受其殃”、“后来同好,敬尔天机,无妄传也”同一个
意思。如果这样,那么宋代张商英为《素书序》中说的:“……上有
秘戒,不许传于不道不神、不圣不贤之人,若非其人,必受其殃;得
人不传·亦受其殃”,有何区别?这种带有告戒性质的语言,谁都可
以写,不是李筌的专利吧!至于后一句,说是李筌记叙获得《阴符
经》残卷时提到的一句话,究竟有没有这回事还是问题呢!试想,北
魏道士寇谦之身在北魏政权中,声势显赫。当时正在忙他的“诵诫”
与“建功斋请”,企图把服饵修炼之术同符水禁咒之术合而为一,他
怎么可能把《阴符经》送往黄河以南的嵩山石崖中藏起来呢?所谓
“藏诸名山,传之同好”,不过是李筌编造的神话而已。诸葛亮与此是
不同的。诸葛亮曾在灵山向酆公玖学习过兵法阵图等,又经酆公玖引
至武当山北极教主处学习过《琅书》、《灵符》等,什么是名山·他心
里明白;什么是“同好”,他心更明白。姜维只是他的军事事业方面
的接班人,不是他的智谋韬略思想方面的同好;并且像秘传于帝王将
相中的《阴符经》,通常情况是藏于金匮,隐于石室,那么,诸葛亮
写了“藏之名山,传之同好,隐之金匮”,错在何处?认定李筌作伪
是可以的,删掉诸葛亮写的这篇序是不可以的,因为没有什么根据。
“尽信书”不对,“不如无书”也不对,总得要有根据来删掉才对。没
有根据就不能武断地删掉。

因此,我们认为,要恢复“黄帝”二字,要保留“伊尹”,要保
留诸葛亮的《序》文,要承认广成子等五家是为《阴符经》作注的五
家。这样,既尊重了他们的智慧结晶,又促进读者的深入把握,问时
还可作到真正地接受这文化遗产。从实际出发,事实求是,所以称
《阴符经》为:十二家注《黄帝阴符经》。这个译注本,就称之为:十
二家注《黄帝阴符经》译注。
在没有文字、书写工具的时代,仅靠心领神会、口授秘传于少数
统治者中,以《阴符经》的智谋韬略武装头脑,这是困难的;在有了
初创的雏形文字、甲骨文字、籀文文字、大篆、小篆、楷书、草书等
的传播,又是不容易的。上下五千年中,《阴符经》把政道、治道、兵
道的需要,给帝王将相头脑中注入了不可胜数的智谋韬略思想,这是
极其宝贵的。因之称《阴符经》为黄老之学的经典著作,所提示的都
是道家可遵从的政道、治道、兵道的谋略思想。不料渊源于古代巫术
的道教兴起之后,道教的徒子徒孙们一哄而起,争相为《阴符经》本作注,人云亦云,因而出现了大量的注本。例如:宋代郑樵《通志
略》客录有38个注本;明代的《正统道藏》收集有22种,还有少数
儒家的注本没有列入。明代吕坤说:“不啻一百家”。清代《清史稿·
艺文志》著录的乂新增加了六种,还有几个集注本。这些注本的注者
大都是道教徒,在注解中牵强附会,而且故弄玄虚、企图把研究大自
然变化规律、人类社会盛衰兴亡的黄老之学道家思想,变为道教巫
术、诵诫、斋戒、建功斋请的护身符,还有些儒者妄图把儒、道、佛
融于一炉,因而给《阴符经》蒙上层层黑幕,使读者产生种种误解,
甚至造成理解上的困难。中华人民共和国创立之后,我国学者王明,
在哲学研究第五期《试论〈阴符经〉及其唯物主义思想》中说:“……
《阴符经》这份宝贵的先民文化遗产,我们应该珍视它,研究它,批
判地接受它,特别是它的唯物主义学说”。这就是说《阴符经》的谋
略思想是唯物主义的,它的生命力是很强的,不能让它再背上层层黑
幕而含恨于人间了,应该恢复它的本来面目。
更为可贵的,历史上有些大书法家书写《阴符经》,作为碑帖,以
广流传,辩别真伪。

东晋王羲之曾书写《阴符经》,并为之石刻。后元成宗大德丁未
(公元1307年)十月,黄仲奎写跋中说:“《阴符经》--名《黄帝天
机之书》。曩见王右军(即王羲之)石刻《阴符》,文与今文小异”。
(见明代郁逢庆《书画题跋记》。
唐初欧阳询曾书写《阴符经》与其子善奴。南宋岳珂在其《宝真
斋书法赞》写跋云:“右太子率更欧阳询字、信本《阳符经》真迹,一
卷”。
唐初诸遂良曾三次书写《阴符经》。贞观六年(公元632年),奉
敕书写《阴符经》五十卷,草书·白题为“起居郎臣褚遂良奉敕书”。

贞观十四年(公元640年),又书写一次。南朱楼钥《攻魏集》中说:
在“都下三茅宁寿观见褚河南真迹注本”。永徽五年(公元654年),
奉旨书一百二十本,小楷。文征明《停云馆帖》所刻诸遂良《小字阴
符经》,卷末云:“唐永徽初,褚遂良尝书一百本”。(见余嘉锡《四库
提要辩证》引南宋楼钥《攻魏集》卷七十二《褚河南〈阴符经〉跋》。)
元代赵孟颓(公元1254-公元1322),字子昂,曾“小楷《阴符
经》”。(见明代郁逢庆《书画题跋记》。)
从以上所述,可以看到《阴符经》的历程是光明磊落的,虽然旅
途坎坷,但瑕不掩瑜。《阴符经》的基本情况大致如此。

(二)《黄帝阴符经》为何冠“黄帝”之名,定“阴符”之实,为何有这么多的名家作注解

据初步统计:书名冠以“黄帝”二字者,在《汉书·艺文志》中
著录 15 种,《隋书·经籍志》著录26 种,《旧唐书·艺文志》著录18
种,《新唐书·艺文志》著录30种,共计89种。这些书籍虽然是后
人编纂的托名之作,但都以黄帝时代的政道、治道、兵道等思想资料
及文字材料为原始依据,掺杂作者所见所闻而著作成书的。例如《汉
书·艺文志》中《黄帝》十六篇;《封胡》五篇;《风后兵法》十三篇
(按:现在还有《风后握奇机》)、图二卷;《孤虚》二十卷;《力牧兵
法》十五篇;《鬼容区兵法》三篇;《黄帝四经》四篇;《黄帝君臣》十
篇;《黄帝铭》六篇……等。又如《新唐书·艺文志》著录有:《黄帝
问玄女经》三卷;《黄帝用兵法诀》;《黄帝兵法孤虚推记》一卷;《黄
帝太一兵历》等。以上这些书籍大部分谋略思想内容,在《阴符经》

各篇中都有或抽象或具体的表达。所以《阴符经》成为研究政道、治
道、兵道的一种谋略思想宝库,也可以认为它是道家研究兵学的-部
奇异兵书。也可认为它是黄帝君臣综合谋略思想的成果,由黄帝奠定
了《阴符经》的雏形。对于上述的社会文化现象,过去人们都习以为
常,明知此书并非黄帝所撰,但书名却仍然沿用不改。所以《阴符
经》冠以“黄帝”二字为全名就称为《黄帝阴符经》。
为什么又定以“阴符”之实呢?
据《阴符经三皇玉诀》中说,黄帝问曰:“阴符者,何也?”广成
子曰:“此‘阴符’二字上可通天,下可察地,中可化生万物,为人
最尊。阴者,暗也;符者,合也。古之圣人内动之机,可以明天地造
化之根,至道推移之源,性命之本,生死之机;知者可究合天人之机,
操运长生之体,故曰‘阴符’也。”这里的核心意思是:最高统治者
掌握统治天下的生杀大权与智谋韬略,藏之于头脑中,然后再运转于
自然界和人类社会中去,这是对的;但也暴露了天道和人道,自然和
人为的关系,即孔子、墨子、子思、孟子、庄子、荀子等所争论的天
人之辩,以此来解释阴符的。这又不对了。所谓“究合天人之机,操
运长生(指国运长久)之体(指政治制度)”(道教徒则利用这两句作
为修身炼性,以谋长生之道),就叫做“阴符”。如果这样,那么大自
然的运行规律,人类社会的盛衰兴亡·又该怎么办呢?这种解释不能
完全说明“阴符”的含义。
据《鬼谷子·揣篇》中说:“古之善用天下者,必量天下之权,而
揣诸侯之情。量权不审,不知强弱轻重之称;揣权不审,不知隐匿变
化之动静。”这是“阴”;《鬼谷子·摩篇》中说:“摩之·符也。内符
者,揣之主也。用之有道,其道必隐微,摩之以其所欲,测而探之,
内符必应,其应也,必有为之。”这就是“符”。这里的核心思想是:主观的智谋韬略要“揣”,同客观的书物是否“内符”,则要去“摩”,
都符合了,才能有所作为。这样“揣”“摩”的原理恰恰就是“阴
符”原理,所以梁陶宏景给《鬼谷子·本经阴符七术》作注说:“阴
符者,私志于内,物应于外,若合符契,故口阴符。”陶宏景的核心
思想是:“私志于内”就是要求主观上的智谋韬略,必须要“物应于
外”,即必须符合客观的真实情形,而且要求“若合符契”,就象君主
与将帅合质刀刻木雕的兵符那样,准确无误。这样来解释阴符作为一
种谋略观和认识论·同《黄帝阴符》的实际思想内容就完全吻合了。
这是可以取法的。
据赤松子等《黄帝阴符经集解》说:“阴者性之宗,符者命之本,
此阴符之者。内以修身,外治家国,包罗天地,总御群方,古今得道
仙真,皆因此义,以至于‘无为’矣”。这里的核心思想是:道家、神
仙家内则修身养性(无为也),外则治国治军(无所不为也),还要
“包罗天地,总御群方”。这不就是战国时代子思、孟子提出的,西汉
董仲舒强调的所谓“天人合一”,而为宋代张载、程颐、朱熹等所反
对的吗!问题在于他们多从“性”“理”“命”方面来解释阴符,借此
来论证“天人合一”的关系,因而又暴露了天人关系的一种神秘学说、
象这样解释阴符,就只能陷入唯心论的圈子里去了,因之是不可取
的。
再看诸葛亮为《阴符经》写《序》其中怎么说的:“……观乎
《阴符》,造化在乎手,生死在乎人。故圣人藏之于心,所以陶甄天地,
聚散天下,而不见其迹者,天机也。”这里的核心思想是:“藏之于
心”是“阴”,“陶甄天地,聚散天下”是“符”,“不见其迹”就是
“天机”。质而言之·阴符就是天机。天机就是“造化在乎手,生死在
乎人”这个统治者的生杀大权了,当然更是政道、治道、兵道的智谋韬略认识过程和智慧实体了。这是以唯物论的观点来解释阴符的。这
就是为什么定以“阴符”之实的原因。
研究《阴符》没有别的要求,只求把主观认识同客观实际统一起
来,简而言之,就是主客观致,才能真正达到“宇宙在乎手,万化
生乎身”的目的。
那么,为什么有如此众多的政治家、军事家、谋略家、思想家、
哲学家为《阴符经》作注或专释呢!
现在对《阴符经》中作注的十二家的活动以及能够作注的基本条
件,简介如下:
广成子。上古时仙人。其实他是空同氏部落首领,居于崆峒山
(今甘肃平凉西六盘山)石室中。据《庄子·在宥》记载:黄帝立为
天了十九年,令行天下,闻广成子在于空同之上,故往见之曰:“我
闻孟子达于‘至道’,敢问至道之精!吾欲取天地之精,以佐五谷,以
养民人(意指政道、治道);吾又欲官阴阳,以遂群生(意指治道、兵
道)。为之奈何?”广成子曰:“而所欲问者,物之质也;而所欲官者,
物之残也;自而治天下,云气不待族而雨,草木不待黄而落,日月之
光,益以荒矣;而佞人之心翦翦者(意为:佞人的思想总是要想方设
法来消灭不同道的人以发展自己的).又奚足以语至道!(那又何必同
他谈论至道!)”三个月之后,黄帝又去问:“吾子达于至道,敢问治
身奈何?”广成子日“至道。至道之精,窈窈冥冥;至道之极,昏昏
默默。……人其尽死,而我独存乎?”对此,宋代苏轼著有《广成子
解》。广成子讲的同《阴符经》中的“自然之道静,故天地万物生。天
地之道浸,故阴阳胜,…至静之道,律历所不能契。”在含义上是相
同的或相近的。他认为《阴符经》的言论,后人取用两条:一是“积
火焚五毒,五毒即五味,五味尽,可以长生也。”二是“甲子合阳九之数也。卦象出师众之法,出师以律,动合鬼神,顺天应时,而用鬼
神之道也。”以当时条件来看,一是没有文字记载,二是“结绳而用
之”(记事),广成子还给了黄帝“自成之经”(这不是指书,而是指
结绳所记的经典式语言),所以广成子的注,理应理解是后人补上去
的。
伊尹。夏末商初有莘氏人,汤娶有莘氏女为妇,伊尹负鼎镬陪嫁
于汤。他以五味之理说汤,致于王道;并说素王及九主之事,得任国
政,佐汤灭夏,为商朝相国。他又辅佐商的第二代、第三代,后死于
商朝第五代时。伊尹是大政治家、军事家。《汉书·艺文志》把他的
著作《伊尹》列人道家类。当然此书不是手稿,而是后人依据他的思
想言论材料编纂起来,托名于他的,当时除符号文字外,已有甲骨文
字流传了。因为道家源出于史官,历论成败和存亡,盛衰与祸福,以
及古今之至道,然后知秉要执本,清虚以自守,卑弱以自持,此君人
南面之术。……并日独位清虚,可与为治。这些思想观念,在《伊
尹》都有反映,所以他给《阴符经》作注,是完全具有条件的,是在
研究《阴符经》之后的情感表达。其注虽然只有五条,但均为谋略思
想的针砭良言,含意精深。
太公吕尚,本姓姜,名尚,字子牙,因其祖先曾封于吕,以封国
为姓,所以又称吕尚。姬昌(周文王)得他,号为太公望,周武王姬
发尊之为“师尚父”。他辅周灭殷商,是一位大军事家、大政治家,以
功封于齐。唐玄宗为尊重军事,给他建立太公庙;唐肃宗上元元年
(公元760年),追谥他为“武成王”兵圣;唐德宗、宋太祖均是非常
推崇他。《六韬》是战国时人根据他的思想言论材料编纂成书的。宋
神宗年间把《六韬》列人《武经七书》中。《汉书·艺文志》把《太
公兵法》列入道家类,还有《太公阴符》、《周书阴符》。吕尚为《阴符经》作注是最具有条件的,上篇中有四条,下篇中有四条,俱是解
释原文谋略思想的本意的,是最具有权威性的注释。
老子,即李耳字伯阳。春秋时的思想家、道家的创始人,做过周
朝“守藏室之史”(即管理藏书的史官),孔子曾向他问礼。他是楚国
苦县(今河南鹿邑东)历乡曲仁里人,后退隐。著有《老子》亦称
《道德经》,当时是以籀文文字著作和流传的。《汉书·艺文志》中说:
“能讽(读)书九千字以上乃得为史。”许慎《说文解字叙》说:“尉
律、学僮(童)十七已(以)上始试,讽籀文九千字乃得为史。”他
为《阴符经》作注是完全具备条件的,仅注一条,如树大纛,一针见
血。后来的道教兴起之后尊他为始祖。
孙子。姓孙名武,本为齐人,因齐内乱,客居吴国。他是春秋时
期的大军事家、大兵法家,遇吴王招贤,经伍员推荐给吴王夫差,任
为将军,他辅吴西乱强楚,北威齐晋。在唐宋两代所建的武成王庙中,
他是列为兵哲十人之一的。他著作的《孙子》十三篇,当时用籀文后
用篆、楷流传至今,宋神宗年间把《孙子》列为《武经七书》之首,
称为兵学鼻祖。他用言为《阴符经》作注是完全具有条件的,作注二
条,主要精神是用治军之道来解释原文含义。
范蠡。楚国人,仕越,为越相国,是大军事家、谋略家,辅越灭
吴。宋太祖选他为武成王庙中的兵哲十人之一。他初劝越王勾践不必
现在就急于攻吴;后劝勾践立即攻吴,吴来求和,他即劝阻;最后劝
勾践干脆灭吴,以便称霸东南。《汉书·艺文志》著录有《范蠡》二
篇,当时用籀文、大篆流传,至汉书时已用小篆、楷书流传了。他为
《阴符经》作注是完全具有条件的,作注一条,他的思想与《阴符
经》是一致的,可以说是大量的·-致。
吴子,即吴起,战国时卫人,先事鲁,因杀妻求将,鲁人恶之、遂去鲁仕魏文侯、魏武王,为西河守将,抗拒秦韩,大小战七十余次
之多,后遭诬陷。吴起逃楚,被任为相,力主改革,得罪旧贵族,楚
悼王死后,被射死。《汉书·艺文志》著录有《吴起》四十八篇,今
本《吴子》六篇。所以吴子是一位军事家、政治家,也是政治的改革
者。他用言为《阴符经》作注是完全有条件的,作注一条,主要精神
以治军之道来解释原文的含义。
鹖冠子。无姓名,战国时楚人·以鹖羽插冠为号,故称鹖冠子。
据传他“初本黄(帝)老(子)而未流于刑名(刑名家的刑名法术之
学),是一位谋略家、思想哲学家。著作有《鹖冠子》一书,当时以
籀文、大篆·后为小篆楷书流传至今。他用言为《阴符经》作注一条,
词语铿锵,发人深思猛省。
鬼谷子。据传姓王名诩,战国时楚人,隐居于颍上阳城鬼谷,因
此为号。苏秦、张仪均师之,孙膑庞涓曾向他学兵法。他是一位大谋
略家,是纵横家的鼻祖,不是政治家的政治家,不是军事家的军事家。
著有《鬼谷子》一卷,在其《本经》中有《阴符七术》,当时以籀文
大篆、小篆流传。他为《阴符经》作注五条,见解独树一帜,非常有
特色。
张良。字子房,其家五世相韩,韩亡后,他约一力士击秦始皇,
不中·逃匿下邳,在圮桥接受黄石公授于《太公兵法》,为帝王师。他
辅佐刘邦,击灭项羽,一统天下,创建汉王朝。他是一一位大军事家、
大谋略家,刘邦在论功授封时说:“运筹策于帷幄之中,决胜于千里
之外,吾不如子房。”晚年为避祸,脱然高引,不知所之。唐宋两代
封建武成王庙时,他均被追谥为“兵亚圣”。他曾同韩信序次兵法。他
曾传黄石公《素书》。《新唐书·艺文志》著录有:《张良经》一卷,
《张氏七篇》七卷(张氏即张良)。张良为《阴符经》作注十四条,言简意赅,针针见血。有的《阴符经》本称“汉张良注”,这是他具备
的条件太优越了。
诸葛亮。祖籍为山东瑯玡人。后随叔移居南阳卧龙岗,叔死后移
居襄阳西隆中。他外游历学,经司马徽向汝南灵山酆公玖学习兵法、
阵图《孤虚旺象》《三才秘录》,酆公玖又引他去南郡武当山北极教主
处,学习了《琅书》、《金简》、《玉册》、《灵符》等六甲秘文,五行道
法。后司马徽见之,改容曰:“真第一流也。”他在刘备三顾茅庐谈论
天下大事时,提出“隆中对”的天下三分决策,后佐刘备,建立蜀国,
任丞相。他是一位大政治家、大军事家、大谋略家,这是他本身具备
的条件决定的。唐宋两代均追谥他为武成王庙中的兵哲十人之一。
《宋史·艺文志》著录有《诸葛亮行兵法》五卷,《诸葛亮用兵法》一
卷。这是后人托名的。现代辑有《诸葛亮全集》,其中有《将苑》、
《兵法》、《军令》等。他为《阴符经》作注五条,独有创见,发人深
思。他为《阴符经》作注的条件太优越了,他作的注,正如他说的:
……“鬼神之情,阴阳之理,昭著乎象,无不尽矣。”
李筌。里籍不详。唐玄宗时做过节度副史及剌史。他曾给《孙
子》作注(见《十一家注孙子》)。后为道教徒,号少山石达观子。他
著有《太白阴经》,其中错误很多。他编造神话,故弄玄虚,写了一
本《黄帝阴符经疏》和《序》,与《黄帝阴符经》唱反调,作注十九
条,大谈神仙修炼之术,甚至把《阴符经》的上篇改为“卷上神仙抱
一演道章”,中篇改为“卷中富国安人演法章”,下篇改为“强兵战胜
演术章”,以此修改来贯彻他的“百言演道、百言演法、百言演术”谬
论,企图把《黄帝阴符经》的智谋韬略思想观念篡改为道教的巫术、
诵诫清斋术。在他作的注中,旁征博引的典故不少,但搞错的却很多。
占的篇幅虽多,但在研究《阴符经》的政道、治道、兵道等的谋略思想观念方面的参考价值确是不大。正如吕坤说的:“经深矣,解之者
又深则道愈晦”了。并且他的《黄帝阴符经疏》与序,均遭到同时代
张果的《黄帝阴符经注》的指责或批判。李筌这一家的注,真是败笔。
但也作为十二家注之一。
从以上所述可以看到:只有四百六十一个字的《黄帝阴符经》,竟
然有如此众多的大政治家、军事家、谋略家、思想家、哲学家都在研
究它,为它作注(还有百家的注本不计),究竟是什么原因呢?
首先:《黄帝阴符经》是统治者提倡和运转政道、治道、兵道的
智谋韬略思想宝库,被称为《黄帝天机之书》,是道家研究兵学的--
部奇异兵书。仅此就可见它的份量之大,毋庸赘言是很诱惑人的心智
的。
其次是当代人王明先生在《哲学研究》第五期发表的《试论〈阴
符经〉及其唯物主义思想》中说的那祥:“这部书中有朴素的唯物思
想,也有自发的辩证思想因素。如果认为《老子》和《阴符经》两部
书都有唯物主义思想和辩证观念的话,我觉得《老子》书里鲜明的部
分是朴素的辩证法,《阴符经》里突出的部分是朴素的唯物论。《老
子》的思想一向被哲学史工作者所重视,这是理所当然的,而《阴符
经》这份宝贵的先民文化遗产,我们似乎应该珍视它研究它,批判地
接受它,特别是它的唯物主义光辉的学说”。经王明先生如此透彻地
分析和评价《阴符经》,这就是促使我们理解了:为什么有那么多的
大政治家、大军事家、大谋略家、大思想家、大哲学家给《阴符经》
作注了。说句老实话,在经史子集浩如瀚海的典籍中,象《阴符经》
这样,有如此众多的大政治家、大军事家、大谋略家、大思想家、大
哲学家作注,还没有见到过,至少在目前是如此。
再其次:《黄帝阴符经》由来久远,内容纯粹,反映的都是有关研究政道、治道、兵道的谋略思想,也就是说它是吸收了匕古时代
“至德之世”结绳记事记言的精华。据《庄子·肤箧》中说:“子独不
知至德之世乎。昔者容成氏、大庭氏、伯皇氏、中央氏、栗陆氏、骊
畜氏、轩辕氏、赫胥氏、尊户氏、祝融氏、伏羲氏、神农氏,当是时
也,民结绳而用之……”有了这些精华,后代人当然要研究它。继空
同氏部落首领广成子发端之言作注之后,就有十一家来研究它并为
它作注。宋代张伯端、明代吕坤,当代王明都有敏锐专释,可见它的
典籍地位之高,影响之大,也就成为必然了。有如此众多的人作注就
不奇怪啦!
也许有人说:这十二位大政治家、军事家、谋略家、思想家、哲
学家的语言给《黄帝阳符经》作注,是某一位学者引用他们的语言来
作注的。这有没有可能呢?我们认为:不大可能。因为大部分注语,
在经史子集中,从未出现过,并且一个人的能力有限,不可能掌握这
么多的语言思想材料和积累自结绳时代以来的雏形文字、甲骨文字、
籀文、大篆、小篆、正楷、草书等文字资料。质而言之,研究《阴符
经》的人太多了,还没有发现有哪一位学者独具如此的天才,能够匹
敌于那些大政治家、大军事家、大谋略家、大思想家、大哲学家的慧
眼和思想观念。所以这一说法是不可能的。
还有人说:注是李筌最后完成的。如果这祥,那么李筌为什么唱
反调,他注的多,但价值不大。所以此说不能成立。
(三)《黄帝阴符经》
是何人所撰、何时成书的
《黄帝阴符经》由来久远,辞语奇特,气魄很大,胆略宏伟,言必有据、理必辩证,-般作者是难以望其项背的。例如“观天之道,
执天之行,尽矣。故天有五贼,见之者昌。五贼在心,施行于天,宇
宙在乎手,万化生乎身”。又如“天生天杀,道之理也"。又如“绝利
一源,用师十倍;三及昼夜,用师万倍”,等等。在无数的经史子集
中是见不到这种命题和论断的。因此它的作者是谁,它的成书是何年
代,这是人们十分关心的问题。要理解这一问题,首先要看其是否具
备条件,再看那个时代的文字、语言和书写工具的情况,只有抓着这
些根子,才能顺藤摸瓜,推论与判断。研究者对此有多种说法,莫衷
一是。下面简介几种·仅供参考。
1、广成子修撰说
据《武经七书掌解》中,关于黄石公《三略》解有:“……三略
之玄微简要,不类广成子之《阴符经》,而雍容典则与《素书》相表
里”。这里含有《阴符经》是广成子修撰的意思是很明白的。那么广
成子是不是《阴符经》的修撰人呢?
广成子是空同氏部族的首脑,古称仙人,黄帝向他问至道时,他
说他已活了一千二百岁呢!他的思想及其主要活动方面,与《阴符
经》有相近之处,后人引用他的言论为《阴符经》作注的有两条。但
这样说是很不够的。以当时条件来看,只有结绳记事记言,没有文字
和书写工具,无法记载;广成子虽然是一个部族的首脑,但他远在西
陲,即使可能了解伏羲氏、神农氏(按:神农氏不是指一个人,而是
指一个历史时期中沿袭的多少个人,都称作神农氏。)、炎帝、黄帝、
蚩尤等的政治设施与战争情况,他也没有能力来总结政道、治道、兵
道的谋略思想,更不必说当时的西方落后于东方了。当时所处年代相
当于父系氏族社会时期,有了奴隶制因素,也就产生了以掠夺财物和
奴隶为日的的战争,也有“日中为市”的某些政治措施。这些根本问题在他思想言论中,如说有某些见解是可能的;如说他是《阴符经》
的修撰,那就是不可叮能的。关键在于他没有修撰的条件。
2、九天玄女传授说
九天玄女,据传是上古时的一位女神。人头鸟身。黄帝与蚩尤战
于涿鹿之野,玄女下降,以六壬、遁甲、兵符、图策、印剑等物授予
黄帝,并为制夔牛(即大牛)鼓八十面,遂破蚩尤。(见《云笈七箴
·九天玄女传》《黄帝内传》)。实际上,九天玄女娘娘是母系氏族末,
向父系氏族社会过渡时的一个部族联盟的图腾形象。他们部族来支
援黄帝作战,战胜蚩尤,这是可能的;如说“以六壬、遁甲、兵符、
图策、印剑等物授予黄帝”,这是道教徒的乱编胡诌了。还有个道教
徒唐代蹇昌仁在《黄帝阴符经解序》中说:黄帝得玄女授《阴符》精
义,能内合天机,外契人事,则三百言实玄女之所授,而百言乃黄帝
之演绎者也,故辞要而旨远,义深而理渊。这个蹇昌仁还根据《黄帝
问玄女兵法》、《玄女法》编了一篇《阴符经事迹》,说“……西王母
再遣九天玄女授帝秘诀一十九条,《阴符经》三百余言。至于金丹玉
篆之文,宝符飞崖之术,入火履水之法无不备焉。”作为道教徒的故
弄玄虚、胡编乱诌,我们就不管其真伪了。从研究《阴符经》是谁修
撰来讲,道教徒特别是那个蹇昌仁的胡说八道,就是全错的。错在哪
里?①所谓“三百言”道藏本都有不少注本是这样说的,唐李筌编造
的骊山老母《黄帝阴符》也这样说。其实唐初欧阳询、诸遂良的手写
真本都包括了后面的一百一十四字。②八卦甲于问题。史书都说明八
卦是伏羲氏创造的,甲子是黄帝命大挠创造的。蹇昌仁可能并不了
解。③蹇昌仁编的《黄帝阴符经解序》及《阴符经事迹》企图说明
《阴符经》是九天玄女传授的,这怎么可能呢?关键在于当时没有这
个条件,九天玄女部族也没有这么高深水平,更不用说他们不可能会有政道、治道、兵道的智谋韬略思想观念,说是九天玄女传授的,只
是神话!
3、黄帝修撰说
因为《阴符经》与黄帝的关系,总是密不可分的,所以在黄帝之
前有广成子、九天玄女等的传授;在黄帝之后有太公等的集注本,东
晋王羲之、唐初欧阳询、诸遂良等的手写本及碑帖本、道藏本中绝大
多数的注解本,还有不少的集注本等等,大都题作《黄帝阴符经》,并
且多数都署名为“黄帝撰”,四库全书本则说“传为黄帝撰”,还有写
作“旧题黄帝撰”的。我认为;如果署名,那就以“旧题黄帝撰”更
好些。
那么《黄帝阴符经》究竟是不是黄帝修撰的?探讨的关键还是在
于修撰的条件。例如:黄帝本人的政道、治道、兵道等智谋韬略思想
与客观现实是否符应,他的社会实践活动,有无文字及书写工具等基
本手段。
据《史记·黄帝本纪》说:神农氏衰,诸侯相侵伐,暴虐百姓,
而神农氏不能征讨。轩辕乃习用干戈,以征不来朝享诸侯,诸侯咸来
宾从,及有讨平蚩尤。炎帝欲侵陵诸侯,诸侯咸归轩辕,轩辕乃修德
振兵,治五行,种百谷,抚万民,度四方,组训军队,冠以熊罴貔貅
䝙虎等图腾名号,三战胜炎帝,九战之后杀蚩尤,被拥立为天子,代
神农氏,是为黄帝。
天下有不顺者,黄帝从而征之。东至于海,南至于江,西至于崆
峒,北逐薰粥,合符于釜山,建邑于涿鹿山下,迁徒往来无常处所。
以云记事(比结绳记事有改进了)黄帝受神策,命大挠造甲子,容成
造历,命史官苍颉造文字(即今汉字的雏形,在此之前可能仍用结绳
记事,也可能已有某种符号字型了。)命风后为相,力牧为将,常先、大鸿(即鬼容区)等治民(按:此民指百姓、人民、军队),顺天地
之纪,阴阳之占,死生之说,存亡之难。时播百谷草木·淳化鸟兽虫
豸。日月扬光,海水不啸,天不异灾,地无别害,山出珍宝,水少洪
波,劳动心力耳目,节用水火材物。由上述司马迁所记载的。虽然有
些夸张,但也可见其盛世,这都是黄帝的政道、治道、兵道的智谋韬
略所显露的无穷威力,也就是“天机”说法的由来,有人称《黄帝阴
符经》为《黄帝天机之书》,即由此出。所以《阴符经》各种本子都
说是黄帝修撰的,这可算是基本条件理由了。
《汉书·艺文志》著录,在黄帝时有《神农兵法》一篇,《黄帝》
十六篇,《封胡》五篇,《风后》十三篇,《力牧》十五篇,《鬼容区》
三篇。这些书都是后人的托名本,但可说明黄帝时代的政道、治道、
兵道的谋略思想,确属惊人,非常宏伟壮观,这也可算作一个基本条
件的产生来源。既然如此,黄帝就不能不考虑把这些谋略思想观念传
之于后代了,所以帝颛顼、帝喾、帝尧、帝舜、大禹王、成汤王、周
文王、周武王等都是身体力行这些谋略思想观念的。由于当时史官苍
颉初创文字,还没有条件来记载,所以这些智谋韬略思想观念,仅在
统治者当中秘传,而被统治者就无法知道其奥妙了。黄帝及其时代的
政治兵道的措施与智谋韬略思想观念,是否可作为《阴符经》的雏形
或与之有相当的血缘关系?这不能说是妄议吧!
再看黄帝的另一些社会活动及其主客观思想观念。据《庄子·在
宥》中说;“黄帝立为天子十九年,令行于天下”前往问广成子,说:
“吾欲取天地之精以佐五谷,以养民人;吾又欲官(管)阴阳,从遂
群生,为之奈何?”广成子告诉他说:“至道之精,窈窈冥冥;至道之
极,昏昏默默……人其尽死,而我独存乎?”这一问一答有两点值得
研究。①黄帝去问“至道”,说明他是为了政道、治道、兵道的难题而去的;②广成子说的窈冥昏默,就是无视无听,抱神以静,则“无
为”也。可能广成子发见黄帝过于繁忙,劝他保持国运长久,人民安
康,自己也可长生,只有“无为”,才能“无不为”;在生死观念上,
广成子告知黄帝,意思是你无法独存,人民也不可尽死,就是说:该
生则生,该死则死,幸生不生,不怕死反而不死了。上述两点的思想
观念在《阴符经》中是有多方面的表达。由此可见,黄帝与广成子在
对政道、治道、兵道的谋略思想观念方面是相同的,这对《阴符经》
的修撰是有重大影响的,不能忽视啊!
据《抱朴子·地真》中说:黄帝“西见中黄子,受‘九加(品)
之方’,过崆峒从广成子受‘自成之经’。”这个“九加(品)之方”可
能是《阴符经》中的“九窍之邪,在乎三要,可以动静。”九加即九
窍,三要即耳目口,就是太公吕尚说的:“耳可凿而塞,目可穿面眩,
口可利而讷,兴师动众,万夫莫议其奇,在三者,或可动,或可静之。”
这就是说严守机密;其次,《阴符经》下篇中说:“瞽者善听,聋者善
视,绝利一源,用师十倍,三反昼夜,用师万倍。”这就是说,要把
敌方变成聋子、瞎子,然后任我而宰割之,所以太公吕尚说:“目动
而心应之,见可则行,见否则止。”何等主动自由I这就是“九加
(品)之方”的奥秘。所谓“自成之经”是广成子对黄帝讲的某些经
典语言,例如:“以为积火焚五毒,五毒即五味,五味尽,可以长生
也”。又如:“甲子合阳九之数也,卦象出师众之法,出师以律,动合
鬼神,顺天应时,而用鬼神之道也。”这个“自成之经”,也许是用某
种符号的原始结绳记事的概括,这对黄帝的思想观念是有不小影响
的,同样,我们不能忽视啊!
由于黄帝本人的亲身经验教训,九加之方与自成之经的影响,这
就使他有可能总结…种发人深思猛省的君人南面之道,即主客观一致的东西,名之为“阴符”,这是无可非议的。但是限于当时文字还
在初创,又没有书写工具可记载,因而不能修撰成一卷书,只能以主
观上的谋略思想观念内容,要求符合于客观的实际,一代一代的把这
一“阴符”秘传下去,作为君人南面的先机之术。大约从公元前26世
纪起到公元前16世纪·有甲骨文字的商初这一千多年的过程中,“阴
符”经历了帝颛顼、帝喾、帝挚、帝尧、帝舜、大禹及其后十六代君
主的继承,特别是帝颛顼、帝喾都是精明的。而且还有八恺八元的辅
助·帝舜更重他用,所以“阴符”的谋略思想观念,是尽情得到发挥
的,在这千余年的秘传中,“阴符”可能已经不是书而又是书了,因
为文宇从初创到商初进化很快、很大,书写工具可能也有改进,因面
伊尹就有可能读到“阴符”,于是就为“阴符”作注五条,并且都是
“阴符”下篇及其最后,这个贡献是很大的。在此之前,黄帝“阴
符”的雏形,在千余年秘传中,经过帝王将相的运用和推导.由雏形
文字组成“阴符”雏形,就可能成为甲骨文字的书型,这种书型的
《阴符》才改变其本来面目“阴符”为书型《阴符》,在商代之前就已
完成《阴符》书型的书,但还不是象我们今天所理想见到的书型的书,
要知道它是用甲骨文字组成的书啊!
由上述我们可以明白:这个不是书的“阴符”,是由黄帝定下来
的雏形,经过千余年的秘传,才由“阴符”雏形成为甲骨文字的《阴
符》书型,称之为《黄帝阴符》。因为“阴符”不是用文字记载,而
是以心领神会,口头秘传,所以称为“阴符”雏形,但在有甲骨文字
积累成书之后,即可阅读·故称为《阴符》书型。由于文字和书写工
具的限制,“阴符”雏形不可能有很多字数。看来伊尹作注时,《阴
符》是有45!个字的,从那时起,流传至今,还是451个字。那些大
政治家、大军事家、大谋略家、大思想家、大哲学家们只为《阴符》作注,让人理解,并未改动、增添或删减一个字,这真是黄帝时代的
政道、治道、兵道的智谋韬略思想观念的伟大贡献!
可能有人会说:从黄帝到商初,“阴符”雏形在帝王将相靠心领
神会,口头秘传,竟达千余年,为何没有一点流失或遗误呢?
这里要明白以下情形。
黄帝把雏型的“阴符”传授时,究竟有多少字、哪些内容,很难
知道,因为是靠心领神会,口头秘传,究竟有多少内容与多少字,恐
怕连他自己也未确定,当时并未定型,仅是秘传的雏型。但伊尹作注
时是有甲骨文字的书型的,而且伊尹的注都在下篇和最后,可见“阴
符”秘传至商初之前,就已经是451个甲骨文字书型的《阴符》了。
其次:有历代史官的功劳。黄帝命史官苍颉造文字,这与秘传
“阴符”雏型是有重大关系的,不管用什么文字或符号,史官们总是
有记载的,史官们除了在帝王将相聚会时记言记事外·他们有条件理
解帝王将相的思想、言论各种资料,并且又能运用藏于金柜石室的各
项材料,这样,在秘传千余年中的各代史官们,对雏型“阴符”的秘
传情况是清楚的,通过他们的具体努力,使“阴符”雏形在秘传千余
年到商代之前完成似书的书型《阴符》,或者可能在夏代前中就已完
成了,这是根据文字的发展完全可能办得到的。
自从伊尹为《黄帝阴符》作注后,又秘传至商末周初,这时有些
典笈的文字已逐步由甲骨文向籀文方向发展,并有用布帛、竹简作为
书写工具的趋势。殷商灭亡,周代兴盛,人才荟萃,天下一统,史官
们就更忙了,帝王将相也忙于整理自己的经验教训史料以育后人,因
此出现了《周书阴符》、《太公阴符》。为什么这两本书不叫别的什么
书名,而偏叫“阴符”呢?这与在当时的秘传的《黄帝阴符》是有某
种渊源的。因为《黄帝阴符》是在帝王将相中的秘传品,是不公开的流传的,而《周书阴符》、《太公阴符》则不是秘传品,是公开流传的。
所以《战国策·秦策一》、《史记·苏秦列传》中都说苏秦拥有并攻读
过这两本书,而“揣摩成矣”,但苏秦就没有读到过《黄帝阴符》,这
是可以理解的。今天我们见不到《周书阴符》、《太公阴符》了,但是
在《黄帝阴符》中却可看到太公吕尚为《黄帝阴符》作注的九条,数
量是多的;并且还可从《六韬·龙韬》中看到《阴符第二十四、阴书
第二十五》,这都是与《黄帝阴符》的渊源很深哩!因为《黄帝阴
符》不是周代的作品,而是从以前的朝代秘传下来的。太公吕尚为之
作注的目的,是为了帮助帝王将相读懂而已,当时的《黄帝阴符》不
是成卷的书,而是甲骨文字或籀文组成的书型。所以太公吕尚促成
《黄帝阴符》早日能成为成卷成本的书,贡献是很大的。
在周代盛世之际,甲骨文已让位于籀文、大篆书,书写工具也在
向有利集中成卷方向发展。许慎《说文解字叙》说:“尉律、学僮
(童)十七已(以)上始试,讽(读也)籀书九千字乃得为史。”可见
当时文字的演化书写工具的发展,为史者需要能读九千个字的籀文
或大篆书,这是十分高超的艺文学术水平啊!发展到春秋战国时,人
才辈出,百家争鸣,《黄帝阴符》也就更有相应发展的余地了。在春
秋时期的老子李耳、孙子、范蠡,战国时期的吴子、鬼谷子、鹃冠子
等这些大思想家哲学家、大军事家兵法家、大政治家都为《黄帝阴
符》作注,阐述黄帝以来的政道、治道、兵道的原理与智谋韬略思想
观念,再加上《庄子》、《列子》的呐喊和补证,孕育了两千多年,由
雏形一→书型一→而写帛刻简成书,诞生的条件完全具备了。
宋代朱熹在《阴符经考异序》中说:“伊川程子(即程颐)曰:
‘《阴符经》何时书?非商末即周末。’邵子(即邵雍)日:‘《阴符
经》,七国时(即战国时代)书也’。”近代学者梁启超说:《阴符经》“置之战国之末,与《易·系辞》、《老子》同时可耳,盖其思想与二
书相近也。”(见《古书真伪及其年代》。)
因此,我们认为:《阴符经》不是黄帝修撰的,而是黄帝奠定的
雏形,称“阴符”;夏末商初已有甲骨文字组合的书型,可以称《阴
符》;在战国末期,刻简写帛成书,全称之为《黄帝阴符经》。简称
《阴符经》、《阴符》。《黄帝阴符经》不是某一个人修撰的作品,而是
经过了两千余年的漫长岁月,在五帝、三王、春秋、战国秘传中,历
代史官们的具体工作,组织整理而成的集体创作作品。依照黄老之学
的性质和范围,把《黄帝阴符经》列入道家类的经典著作之一。
4、其他修撰者说
有人依据《宋史·艺文志》著录的张鲁《阴符经元义》,便推断
《阴符经》产生的时代“决不会晚于东汉末年”;据此,又推断《阴
符》是东汉末张鲁修撰的。既然如此,诸葛亮为《阴符经》作注就没
有根据了。这不可信。张鲁既然是《阴符经元义》的注者,怎么又变
成是《阴符经》的修撰者呢?这里的关键是:道家和道教二者的思想
观念根本不同。由于有人分不清道家(源出于史官,老子、庄子是其
代表人物)、道教(源出于古代巫医,东汉末道教是张道陵兴起的、张
鲁就是张道陵的孙子。)的历史渊源,不从道家经典《阴符经》成书
的历史条件、哲学思想条件、文字演变与书写工具发展的条件等方
面,来研究《阴符经》的作者、成书年代的可能性,硬是从道教的徒
子徒孙们的诸多注本中去考证《阴符》的作者及成书年代,岂不是徒
劳吗?即使抬出道教奉为教主的老子李耳,李耳也会接受的,因为他
就为《阴符》作过注,读过《阴符经》。当然更不能否定五帝三王、春
秋战国时代多少的政治家、军事家、谋略家、思想家、哲学家对《阴
符经》诞生的功绩。所以不能说《阴符经》是东汉末年张鲁修撰的。还有人说:《阴符经》是东晋时杨羲、许谧修撰的;也有人说:
《阴符经》是北魏道士“寇谦之作伪”而修撰的;也还有人说:《阴符
经》是唐代李筌杜撰的。这些考证与说法、是人云亦云的,作研究参
考,价值不大。
还有人认为《阴符经》与《周易参同契》的思想内容、文字风格
“有若下相似之处”,便又断定《阴符经》是东汉末年的作品。并且还
借用当代王明提到过的:“清朝顾榱三《补后汉书·艺文志》卷八兵
家类、著录程遐《阴符经注》-一卷”为“颇有参考价值”,作为旁证,
这真有点莫名其妙了。所谓“思想内容”,《周易参同契》能与《阴符
经》相比吗?再说所谓“文字风格”,这是糊弄人的吹牛脱身法,说
实在的.《阴符经》的文字风格,《周易参同契》的文字风格能比得上
吗!由于道家与道教的思想体系不同,所以道教徒不可能理解道家的
思想观念,没有实际意义,更无参号价值。

(四)《黄帝阴符经》成书以后情况怎样呢
1、注解继续
《阴符》在战国时代成书以前两千余年的情况如前所述,成书以
后又经历两千余年的情况怎样?由于文字由秦始皇灭六国一统天下
之后也得到统一,籀文、大小篆书已被楷书、隶书逐步所代替,书写
工具也逐步由竹简、布帛向纸张过渡,所以《阴符》走向繁荣流传是
必然的。经过许多坎坷路程,才到达当代的社会主义中国。现在我们
的条件是好的,有书可读,有资料可参考,这是总的情况。具体的情
况则如下概述。
张良是接受黄石公传授《太公兵法》的唯一的“王者师”,他的智谋韬略是惊人的。他为《阴符》作注十四条,在各名家中是最多的
一个,这是他的巨大贡献。
诸葛亮在人们思想中是智慧的代表,曾为《阴符》作注五条.写
《序》一篇·对《阴符经》的传播起了椎动作用。
李筌为《阴符》作注十九条,条数虽多、但理解不足,又因思想
观念不同,所以同《阴符》的原文思想观念大唱反调,大谈神仙术,
引用历史典故,故弄玄虚,搞乱情节。甚至编造神话,给人一个印
象,《阴符》是李筌杜撰的。他写了一本《黄帝阴符经疏》和《序》.
哗众取宠,例如:他西行至骊山,见一老母,口说:“火生于木,祸
发必克”。他惊问:“此是《黄帝阴符经》中之文,母何得而言?”老
母说:“吾受此经三元六周甲子矣!”按古代术数家的说法:以六十年
为一甲子,第一甲子为上元、第二甲子为中元,第三甲子为下元,合
称“三元”,就是180年了。《晋书·苻坚载记下》:“从上元人皇起.
至中元,穷于下元,天地一变,尽三元而止。”这也是180年。三元
六周是1080年,三元六周甲子则是1140年。就是说,这位骊山老母
接受《阴符经》已有1140年了。如果从战国中期《阴符经》成书算
起,至唐玄宗天宝年间约为1100年左右,这也说明《阴符经》成书
年代是在战国中后期。但是骊山老母怎么会在1100多年前的战国中
后期受书而且活了一千一百多岁,以前活了多少岁还没有计算在内
呢!这当然是编造的神话。更奇怪的是:古代术数家还有一种说法:
4,617 年为一元,三元是13,851年,再加六周甲子360年,那就是14,
211年了。就是说:骊山老母在14,211年前就受领《阴符经》了,而
她活的岁数就不知道有多大了。李筌编造这个荒诞无稽的神话也太
离奇了,令人难以相信。他还编造骊山老母给他讲授《阴符经》,实
际上是在出售其私纂的《黄帝阴符经疏》,当时就遭到张果《黄帝阴符经注》的分析批判。因此,李筌的注解参考价值不大,原因是在于
他并没有理解《阴符经》的思想观念的本领,可是他却给《阴符经》
蒙上一层黑幕了。
2、注本繁多
道家的注本也有,如赤松子等《黄帝阴符经集解》。
道教注本,从东汉末张道陵奉老子李耳为道教教主之后,张鲁搞
了本《黄帝阴符经元义》之后,不少道教徒既不明白《阴符经》的思
想观念,又不懂《阴符经》是唯物主义学说,因而以唯心主义的思想
观念,一哄而起,争先为《阴符注》写注本,借以欺骗其教徒,蒙蔽
世人,《清史稿》中收集的注本,竟可达百家之多,除了人云亦云、换
汤不换药之外,还给《阴符经》盖上无数的黑幕。从公元200年起,
到公元1911年止的近两千年中、在道教徒子徒孙们的搬弄歪曲下,
《阴符经》走的都是坎坷旅途,世人一听到说,有本《黄帝阴符经》
便不约而同地说:“那本书是搞封建迷信的”。真叫有识之士哭笑不
得,甚至有某些当代人也这样看,并用高傲睥睨态度来对《黄帝阴符
经》。
3、还有企图熔偶道佛于一炉的儒家注本
首先必须明白,什么是儒家,什么是儒教;其次他们给《阴符
经》写注本的意图。
儒家是崇奉孔子学说的重要学派,《汉书·艺文志》列为“九
流”之一。学说内容主要是“祖述尧舜,宪章文武”,崇尚礼乐和仁
义,提倡忠恕和中庸,政治主张“德治”和“仁政”。战国时期,儒
家有八派;汉武帝罢黜百家,独尊儒家学说,因而有董仲舒、刘歆为
代表的今古文经学和谶纬之学;魏晋时期有王弼、何晏等以老庄思想
解释儒经的玄学;唐代韩愈为排佛而倡导“道统”说;宋代明朝有兼取佛道思想的程颐、朱熹派和陆九渊、王守仁派的理学;清代前期汉
学、宋学之争、中叶以后有今文经学、古文经学之争;五四运动前后
随着封建社会没落而日渐丧失其正统思想地位。由此可见儒家与道
家、佛家的思想到后来是纠缠不已,彼此难分的。理学派最典型了。
所谓儒教,就是把孔子学说当成宗教·并和道教、佛教并列。历
来封建统治者都企图把孔子神圣化。儒家中的今文经学派从董仲舒
到康有为都曾看待孔子如同宗教之教主。但“孔子创教”的说法.则
始于康有为的《孔子改制号》。
不管是儒家或儒教徒,他们写《阴符经》的注本,其目的就是企
图熔儒道佛三家思想于一炉。最典型的就是宋代的大儒朱熹。他干了
些什么呢?
朱熹化名为“崆峒道七邹诉”,搞了一本《阴符经考异》,这是很
有趣的。他想给道教注本涂上儒家思想脂粉,以求儒道熔于-炉。当
然这是可悲的。
更可悲的是:这本《阴符经考异》是不是他写的,都还值得论证。
据查:《阴符经考异》与道藏本的蔡氏《黄帝阴符经注》(蔡氏本的原
文没有最后114字)文字完全相同。《宋史·艺文志》著录蔡珪《阴
符经注》一卷,《阴符经要义》一卷,《阴符经小解》一卷。蔡珪是北
宋灭亡后随父蔡松年入金的。他是金朝海陵王完颜亮的天德年间
(1149年-1152年)进士,以文学显名,有多种著作,其生年不详,
死于1174年。朱熹生于1130年,死于1200年,比蔡珪晚死26年。
那么,朱熹搞的《阴符经考异》是否在蔡珪死后的26年内?蔡氏本
在前,不可能抄袭朱氏本。而朱氏本在后,难道朱熹抄袭了蔡氏本?
抑或是后人著录上发生了错误?这底确是值得探讨的事。
还有一些所谓“儒者”的注本,那就不去研究了。他们大都在做熔儒道佛于一炉的工作,很难有可称道处。
有些儒者并不完全理解,道教是古代遗传下来的巫术。这些道教
徒只知道天师张道陵是他们的创教人,硬抬老子李伯阳为教主。自从
张鲁的五斗米道、寇谦之的乐章诵诫新法·陆修静的斋戒仪范出来之
后,哪里能够去理解《黄帝阴符经》的唯物主义学说和政道、治道、
兵道的智谋韬略思想观念呢?还有些怀疑派在那儿指手划脚。有些儒
者装得比黄帝和帝王将相,以及那些大政治家、大军事家、大谋略家、
大思想家、大哲学家还高明呢。四库本的儒者先生们,对“黄帝”二
字敢于删掉。对广成子等五大家不予承认,对伊尹给予除名.对诸葛
亮的《序》予以否定,真是太武断了。有失学术研究的风度。
(五)《黄帝阴符经》智谋韬略的思想性
《黄帝阴符经》全书仅有四百五十一字,但其内容都是研究政道、
治道、兵道的智谋韬略思想观念,又经十二家作注,深人肯定与发掘,
它就成为道家经典智谋韬略学!如果从思想性方面来讲:它具有朴素
的唯物主义思想,自发的辩证法思想因素,而突出的部分是朴素的唯
物论。因而它又是道家研究兵道的一部奇异兵书。下面作一浅析。
1、唯物主义思想的思想宝库
《黄帝阴符经》开始就说:“观天之道,执天之行,尽矣。”这就
告诉我们:当你站在大自然与人类社会舞台上,以主观世界观察客观
世界的同时,不仅按照客观规律去认识人、事、物的变化与转化,而
且要有胆识、有魄力、有勇有谋的有所作为,求得更能适应人的生存
和发展。否则就会被消灭在懵懵懂懂之中了。秦始皇接受尉缭“灭六
国统一天下”的决策,终于变成现实;韩信“登坛对”,把西楚霸王置于砧板上,让刘邦去宰割他,以成立西汉政权,变成现实;张良反
对立六国之后,目的是在团结争夺天下的力量,后来建议召集商山四
皓以羽翼太子,目的在于使汉王朝免于陷入分裂,也变成了现实;诸
葛亮“隆中对”,把北拒曹操,东和孙权,取益州为基业,造成天下
三分,也变成了现实。这几位帝王将相仰观天上,俯视入间的胆略气
魄不是空荡无物的,而是实体存在的,所以“观天之道,执天之行,
尽矣”,这是真正地审时度势的唯物主义思想。
书中又说:“阴阳相推,而变化顺矣。”我们知道:阴阳是-一种矛
盾势力运动。这种矛盾势力运动,是相推相胜的,因而就有变化顺应
和变化不顺应的情形出现。例如:刘邦在取得天下统治权之后,为了
巩固刘家政权,不惜大锄功臣,杀韩信,醯彭越,斩英布,萧何死于
狱中,张良从赤松子游,刘家政权虽然得到稳定,但是政权威力从此
衰退矣,这是权力之争不顺应变化的情形;刘秀云台羽将图像,保持
相安无事,这是变化顺应的情形;赵匡胤杯酒释兵权,夺回了军权,
杜绝谋反,从而巩固了赵家政权,石守信等将帅自动交出兵权,避免
了杀身之祸,这是变化的互相顺应的情形。因为在私有制社会里的权
力斗争,就意味着政权的变化,这是无法避免和解决的,所以叫做阴
阳相推,力求变化顺应;而在公有制社会里只有权力分配和运用,不
存在权力斗争,所以就运用“自然之道浸,故阴阳胜”了。张良对此
就注解说:“天地之道,浸微而推胜之”。这里有一个“相推”和“变
化顺”与“变化不顺”的阴阳相胜的原理,是以唯物主义为基础的运
动过程。
《阴符经》中很多论点都包含唯物主义思想。被广泛引用的是
“火生于术,祸发必克;奸生于国,时动必溃。”例如:唐玄宗是以平
定两次宫庭政变而登基的,当时不杀掉韦后、安乐公主及消灭太平公主,铲掉她们的恶势力是不行的;唐玄宗是全唐盛世主宰者,全靠姚
崇、宋璟、张九龄等相的治国之道,边疆将帅的镇抚和兵道形成的;
他的衰败,全是李林甫、杨国忠二相造成的;军事上的垮台在于滥施
恩宠于安禄山,企图换取东北方面最不可靠的安宁,在几十年的文偃
武嬉局面下,导致全唐盛世崩溃,自已却颠沛流离转到成都附近的宝
光寺与和尚为优了。其子李亨依靠谋略家、将帅郭子仪等消灭安史之
乱才复活李家政权。与此同时的少室山达观子李筌就借机发挥了。他
注解说:“火生于木,火发而木焚;奸生于国,奸成而国灭。木中藏
火,火始于无形,国中藏奸,奸始于无象。”那个口蜜腹剑的李林甫,
办事很能的杨国忠·跪称父亲母亲以谋作乱的安禄山等等·都是“奸
生于无形,奸成而国灭的。”象以上的论点和历史事实太多了。《阴符
经》不愧是具有唯物主义的思想宝库!
2、相反相成,相生相克,道不同,不相为谋的辩证法思想因素
《阴符经》中说:“天有五贼”。太公吕尚注解说:“圣人谓之五贼,
天下谓之五德(得)”。其实,这是相反相成的原理。贼是一种败坏力
量,它的反面就是一种建设力量。五贼的反面就是五德(得)。贼必
有所失,德必有所得,有得必有失,有失必有得,说实在的,没有得
失相当的原理。事物是转化的,那么得可转化为失,失可转化为得,
利可转化为害,害可以转化为利,这是在一定条件下的活动。毛泽东
告诉我们:“东方不亮西方亮,黑了南方有北方,不愁没有回旋余地。”
这是相反相成有得有失的辩证思想的运用。例如刘备取得荆州,却遭
到了孙权、曹操两面的虎视耽耽的生死威胁,失去荆州,却巩固了益
州,造成三国鼎立,三分天下有四十五年。因为五贼是客观存在的,
所以五得就是主观努力的结果。假如能够利用人、事、物中的利害变
化与相互转化的契机,那么他要作的事情,就会顺利成功。例如:《太平御览》卷七九引《蒋子万机论》说:“黄帝之初,养性爱民·不
好战伐,而四帝各以方色称号,交共谋之,边城日惊、介胄不释,黄
帝……于是遂好营垒·以灭四帝。”“既经战胜了四帝,统了天下,
四面八方的诸侯都归向他。”这不就是以丘得消灭五贼的典型例证
吗!不仅黄帝如此·而且商汤伐桀,周武王伐纣,也是如此。这就是
一帝二王都顺应了天道、地理、人情而统一了天下。实际上仍然是
“五得”消灭“五贼”。
所谓“五贼”就是“贼命、贼物、贼时、贼功、贼神”,面“五
得”就是“得命、得物、得时、得功、得神”。其中得神是最重要的,
就是在精神世界里运用神机妙算,以达到“以小取大,因小灭大”的
目的。例如:范蠡助越灭吴、是经过二十一年的艰苦卓绝的斗争;张
良、韩信助刘邦兴汉灭楚,是经过千辛万苦的四年战争的斗争,这么
长时间的斗争最后能够取得胜利,是“五得”消灭“五贼”的结果,
而神机妙算的作用是绝不可忽视的。所以《阴符经》中说:“五贼在
心,施行于天,宇宙在乎手,万化生乎身。”这主要说相反相成的自
发的辩证法思想运用的概貌,大致如是。
《阴符经》中说:“大地,万物之资;万物,人之资;人,万物之
盗。三盗既宜,三才既安。”这是什么意思呢?
首先,盗,是撷取、残害。其次,三盗即天地、万物、人。冉其
次,三才即天地人。就是说:天地生万物,但也摧残万物;万物养育
人类,却又杀害人类;人类繁殖并培育万物,却又摧残万物。这些都
是生之者变为杀之者。天地、万物、人互相撷取和残害,只要取之以
时,措施得当,不违背常理,就能各得其宜,三才这天,地、人也就
相安无事了。否则,就是“天发杀机,移星易宿;地发杀机,龙蛇起
陆;人发杀机,天地反覆”了。在这种相生相克原理指引下.相生之中包含相克之理,在相克之中,包含相生之理,相生相克,各得其宜,
各安其位。上下五千年,从黄帝到现代.各开国时期真实情况的反映,
不都是如此吗?这种“三盗”、“二才”相生相克原理。不正是白发的
辩证法思想因素吗?所以《阴符经》中说:“天生天杀·道之理也。”
最奥妙的是:“其盗机也,天下莫能见,莫能知。”就是说:“三盗”之
机·要紧紧抓住不放·立即执行。所谓“机”,即五贼与五得的时机,
易见而难知,见近而知远。因之,范蠡劝勾践立即伐吴时说:“从时
(机)者,犹救火、追亡人也。蹶而趋之,帷恐弗及。”关键是机不可
失,时不再来。否则“沉水入火,自取灭亡”。象这样的辩证法思想
因素,在《阴符经》中真的很多,对我们研究人、事、物,研究政道、
治道、兵道等方面都有补益和参考价值的。如果不懂得“相生相克”
原理,一定会吃亏,
对《阴符经》的辩证法思想因素,李筌不大懂,这不足为奇.因
为他是做过刺史和节度使副使的,又是道教中的文人才子,其思想观
念与《阴符经》总是唱反调的,即使唱了一个好听的调子,也是“先
粉后刮”的。他曾认为:对《阴符经》:“观其精微,《黄庭》八景不
足以为学;察其志要,经传子史不足以为文;任其巧智,孙吴韩白不
足以为奇。”表面看来好象是崇高评论,实际上却是不惜血本的降价
处理。我就不相信,有了《阴符经》就使“经传子史不足以为文,孙
吴韩白不足以为奇”了。这叫做“捧得越高,跌得越重”。那么是否
让写经传子史的都应该“沉水入火?”叫孙武、吴起、韩信、白起这
几位大军事家都应该去“自取灭亡”!先粉后刮是道教徒惯用手法,看
来李筌也不例外。
不仅此也,还有大儒朱熹,他借闾丘次孟的话说:“《阴符经》所
谓‘自然之道静,故天地万物生。天地之道浸,故阴阳胜。阴阳相推,而变化顺矣。’此数语,虽《六经》之言无以加”。这是他要熔儒道于
一炉的真心实意,但是未免太骄奢了吧!难怪他还有佛家的思想哩!
这真是无独有偶。一道、一儒道佛的李筌、朱熹怎么能分辨《阴符
经》的辩证法思想因素呢!本来就不是一家人啊!
我们不妨再来看看诸葛亮为《阴符经》写的《序》中有几句话:
“观乎《阴符》,造化在乎手,生死在乎人。故圣人藏之于心,所以陶
甄天地,聚散天下,而不见其迹者,天机也”。所谓“天机”,就是帝
王将相头脑中的智谋韬略,对待人、事、物的思想观念和推行的政策
与措施。所谓“造化在乎手,生死在乎人”,就是帝王将相掌握治国
治军的统治权势和生杀大权,可以依据形势以宰割天下·凭一时的喜
怒,决定人之生死。这不明白地告诉人“天性,人也;人心,机也;
立天之道,以定人也”吗?看来这是他对《阴符经》透视的精髓所在。
因为诸葛亮是一位大政治家、军事家、谋略家、智谋韬略、思想观念
同《阴符经》是一致的,所以他写的《序》对《阴符经》的观察与评
论,是从辩证法思想因素出发的。当然李筌、朱熹对《阴符经》的观
察与评论,同诸葛亮相比较,那就差得太远了。这叫做“道不同,不
相为谋”。
《阴符经》是研究政道、治道、兵道的智谋韬略思想宝库,历代
统治者除了学习不少东西外,对“天人合发,万变定基”是提心吊胆
的。所以他们就“五贼在心,施行于天。宇宙在乎手,万化生乎身”
了,因而借口“天生天杀,道之理也。”正是如此,两汉统治中国达
425年,两宋达319年,唐代达289年,明代达276年,清代达267
年,两晋达139年,元代达89年。这就是中国封建王朝为什么能各
自统治如此之久的奥秘所在了。明代吕坤曾经说:“《阴符经》…其
言洞察精微,极天人之蕴奥,黄帝得之以御世,老氏得之以养生.兵家释之以制理,术家得之以成变化而行鬼神,纵横家得之以股掌人
群,低昂时变。”他又说:“自有《阴符经》以来,注者不啻百家,要
不出三见;曰儒、曰道、曰禅(佛也),倚其一,则三见皆边也。夫
玄金在熔,万物可铸;谓称锤是铁则可,谓铁是称锤则不可。是书也,
譬之江河之水,惟人所挹;其挹也,惟人所用。”这个说法,从唯物
出发,发挥辩证法思想。历代统治者把《阴符经》作为秘传品,不是
偶然的,就怕“天人合发”,他的基业要万变,让人家在“天人合
发”的“万变”来定新的基业了。所以统治者的手中,拿的不是灯草,
而是铁锤!政治家、军事家、谋略家、思想家、哲学家的“无动无所
不动”,道家的“无为无所不为”的核心思想全在于此。


《黄帝阴符经》又称《阴符经》、《黄帝天机经》。李筌分为神仙抱一之道、富国安人之法、强兵战胜之术。全书以隐喻论述养生,愚者不察,谓兵法权谋等说或谓苏秦之“太公阴符之谋”皆离旨甚远。

作者,旧题黄帝撰。疑似作者,黄帝、苏秦、寇谦之、李筌。

创作年代,商朝/战国/北魏/唐朝

《黄帝阴符经》与《混元阳符经》相配,论涉养生要旨、气功、八卦、天文历法等方面。关于成书有人说黄帝,有人说是战国时的苏秦,近代学者多认为其成书于南北朝。

作为一部高度精炼的道教经书,《黄帝阴符经》正如其他许多具有理性精神之道教学者所撰之作品一样,不是简单因袭易学义理派的言辞,而是运用其义理思维,以《易》通《老》,演述神仙抱一之道、富国安人之法、强兵战胜之术,全书以隐喻论述养生,愚者不查谓兵法权谋等说或谓苏秦之“太公阴符之谋”皆离旨甚远。如道教《纯阳演正孚佑帝君既济真经》,通篇全部以军事术语写成,不知者初见会认定是一篇兵书。因此李筌、张果老、朱熹等人曾先后为《阴符经》作注。朱熹虽然认为其伪但认为“非深于道者不能作”。 [1]

《阴符经》旧题黄帝撰,所以也叫做《黄帝阴符经》。因而有题称伊尹、太公、范蠡、鬼谷子、张良、诸葛亮等注解。这一说,最不合理。宋黄庭坚说:“《阴符经》出于唐李筌。熟读其文,知非黄帝书也”,“又妄托子房、孔明诸贤训注,尤可笑。惜不经柳子厚一掊击也”①。好事者说黄帝撰经,并且假托太公、张良等作注,这些都是显明的依托古人说法,不可置信。自唐李筌为《阴符经》作注,以后累朝均不乏好事者步其后尘,迨至晚清,《阴符经》注解本已不下百余种,今仅存于明《正统道藏》的便有二十四种。注解虽多,但众说纷纭,见解芜杂。

《阴符经》的作者,历来说法不一,共有四种说法:

第一种观点认为是黄帝所撰,伊尹、太公、范蠡、鬼谷子等注。

第二种观点认为是北魏寇谦之所作,其根据是因为杜光庭《神仙感遇传》谓此书是“上清道士寇谦之藏诸名山”。

第三种观点认为是唐代中期的李荃所作,持这种观点的有宋代的黄庭坚、朱熹等。

第四种观点认为是南北朝时一位“深于道者”所作。

朱熹在《阴符经考异序》引:“

邵子曰:《阴符经》七国时书也。

伊川程子曰:《阴符经》何时书 ?非商末则周末。“

但是朱熹并不同意他们的说法,宋朱熹《阴符经考异》中说:"《阴符经》三百言,李筌得于石室中,云寇谦之所藏,出于黄帝。河南邵氏以为战国时书,程子以为非商末即周末。世数久远,不得而详言。以文字气象言之,必非古书,然非深于道者不能作也。 ……或曰此书即筌之所为,得于石室者伪也。其词支而晦,故人各得以其所见为说耳。筌本非深于道者也。是果然欲?吾不得而知也。"

《道藏》中收录的《黄帝阴符经疏序》词上有差异,即言“魏真君”而不言的记载与此基本相同,只是在个别字“大魏真君”。《阴符经》全称《黄帝阴符经》,古以为出自黄帝之手,此当属托名。不过,其说却自有来历。宋代以来学者,始对此书之作者及产生年代提出质疑,但意见颇不一致。宋人黄庭坚《山谷题跋》及朱熹《阴符经考异》以为此书是李筌假托黄帝名自造;清人姚际恒、全祖望等学者认为此书乃魏寇谦之伪托;今人余嘉锡及王明先生均对此有考辨。

余氏《四库提要辨证》指出:“昔晋哀帝兴宁二年紫虚元君上真司命南岳魏夫人下降,授弟子杨羲以《上清真经》,使作隶字写出,以传句容许谧并第(弟)三息许翔,事见《真诰·运题象》。于时所出道经甚多,《黄庭经》即出于是时,……其后杜京产将诸经书往剡南,吾疑《阴符经》即为此辈所作。以其有强兵战胜之术,故京产弟子孙恩遂因之以作乱。”杜京产为魏晋时人,余氏以为此书为魏晋人杜京产所作。王明先生认为:《阴符经》的一个重要思想“天地,万物之盗;万物,人之盗”,不见于古籍,最早出自《列子·天瑞篇》。

王氏引述了《列子》书多条行文来证实《阴符经》之思想来源出自《列子》,又据一些学者关于《列子》属“伪书”的观点,认定《阴符经》当出南北朝时道门中人或当时隐者之手。虽然《阴符经》之思想来源出自《列子》是无疑的,但近年来许多学者对《列子》一书之年代重新进行考证,如许抗生先生所作《列子考辨》,根据先秦与两汉不少典籍引用《列子》文句的事实,认为《列子》当属战国时代之作品,但在许多地方经过后人增改。⑧《列子》为早期黄老道家典籍。如此一来,则《阴符经》之出世年代是否属于南北朝则尚待进一步研究。

注解及杂著

《黄帝阴符经》在宋郑樵《通志》上所载书目共有39种之多,明《正统道藏》所收的成书也不下20种,后之学者纵然把这些注解都阅遍了,恐怕仍旧不能明白《黄帝阴符经》是怎么一回事。《战国策》言:苏秦得“太公阴符之谋,伏而读之”;而《史记·苏秦传》则言: “得周书阴符,伏而读之”。西汉国家藏书目录《汉书·艺文志》道家类曾有著录曰:“《太公》237篇,《谋》81篇,《言》71篇,《兵》85篇。”班固注“吕望为周师尚父,本有道者。”清沈钦韩说:《谋》者即太公之《阴谋》,《言》者即太公之《金匮》,《兵》者即《太公兵法》,苏秦曾得《太公谋》八十一篇,从中悟出纵横术 [2]。至于《道藏》中现存之《黄帝阴符经》,很容易看出它是专门修炼家言,与兵家无涉,凡以兵家的权谋术数作注者皆文不对题,其间亦有不谈兵而泛论国家政治及人事得失者,都与《黄帝阴符经》的宗旨相去甚远。宋儒朱熹虽不识《黄帝阴符经》作用,但也有几句好评的。

《黄帝阴符经》原文有300余字的,也有400余字的,何种版本为可靠?已无从断定。所幸其中要紧的话在各种版本上都一致保存,大体尚无妨碍。惟注解总嫌芜杂,阅之徒乱人意。有些地方,经文并不难懂,如果看了注解以后,再和经文两相对照,就觉得满纸都是荆棘。不知它在那里说什么话?即如经文“君子得之固躬,小人得之轻命”,本意是说正派人得到这个法子,能够使自己身体坚固;邪派人得到这个法子,反而轻易促短自己的寿命。有些版本把“固躬”改作“固穷”,或许因为《论语》有“君子固穷”之说,遂妄改之,但不思与上文“其盗机也,天下莫能见,莫能知“三句怎样可以连在一起?“盗机”的作用和“固穷”的品格究竟有什么相干?又如经文“天人合发。’’一句,本是修炼家的专门术语,注家不得其解,把它改作“天人合德”,一字之差,竟至点金成铁。而且“天人合发”的“发”字是根据上文“天发杀机、人发杀机”两句而来,若把“发”字改为“德”字,试问有何根据?又如“天发杀机,龙蛇起陆”,原文只有两句,后来各种版本把两句改成四句,而改法又不相同:(1)“天发杀机,移星易宿;地发杀机,龙蛇起陆”。(2)“天发杀机,龙蛇起陆,地发杀机,星辰陨伏”。(3)“天发杀机,星辰陨伏;地发杀机,龙蛇起陆”。他们所添改的四句都不及原来的两句好,反而觉得累赘。原文是“天”与“人”相对待,不需要把“地”排列进去;他们把“天、地、人”三才并列,遂失却原文的意旨。原文“龙蛇起陆”是隐语,不是真有这件事,而他们当真地认为龙蛇在地下潜藏不住,都跑到地面上来了,因此就凭自己的理想,加入“地发杀机”一句;又因‘‘天发杀机’’没有下文,变成孤立的句子,于是再用“星辰陨伏”或“移星易宿,,以补足原文语气,读者更莫名其妙。又如经文‘‘其盗机也,天下莫能见,莫能知”,而李筌的注本上则多了两个‘‘不’’字,作“天下莫不能见,莫不能知”,这是显然的错误,但李筌并未加以校正,而且将错就错的曲为之说,原来是很容易懂的话,竞弄得非常难懂,所以后人读《黄帝阴符经》,最好不要看各家注解。 [3]

关于《阴符经》又称《黄帝阴符经》。经文很短,共有400余字;但据一般说,从“观天之道”起,至“我以时物文理哲”为止,是它的原文,仅300余字,所以《悟真篇》云:“阴符宝字逾三百。”自“我以时物文理哲”以下100余字,说是后人增补,但这一段文字,是宋代以来即已经有了的,如朱熹在注《阴符经》时,即非常赞赏其中的“自然之道静,故天地万物生;天地之道浸,故阴阳胜”几句话,他说:“四句说得极妙”。又说:“浸字下得最好”。也有人说:这一段最早见于柳公权书《阴符》(《宣和书谱》有唐柳公权书《阴符经》),如《黄帝阴符经注解》引高氏《纬略》说:“蔡端明云:柳书《阴符经》之最精者,善藏笔锋”。那么,应当更早了。至于它究竟是多少字?因为各家传本不同,我们也不能肯定。

它的内容,各家看法并不一致,悬殊很大。有的认为它是谈道家修养方法的书,但其中又有谈“道”和谈“丹”之分;有的认为它是纵横家的书,所谈都是权谋术数;也有人认为它是兵家的书。比较来说,以第一种看法为多,因为在《阴符经》上篇中是很清楚地说出“知之修炼是谓圣人”。可见它的宗旨所在,是说道家的修养方法,主要是“观天之道,执天之行”,并认为能够做到这一点就可以“宇宙在乎手,万化生乎身”,也就是掌握了长生久视的自主之权。宋代的学者,像周敦颐、程颐、程颢、朱熹他们都很喜欢《阴符经》,对这一部书十分推重。但当时也有一些学者则不同意他们的看法,如黄震说:“经以符言.既异矣;符以阴言,尤异矣”。又说它“言用兵而不能明其所以用兵,言修炼而不能明其所以修炼,言鬼神而不能明其所以鬼神,盖异端之士掇拾异说而本无所定见者,此其所以为阴符欤!” [4]

成书背景

《阴符经》传说是轩辕黄帝所写,但实质上不可能,有人说出于先秦,最早给它写注的李筌说是寇谦之所传并藏之于名山,这些都是传说,现代学者认为是北朝的人所写,而且最初与道教无关。事实在唐代,《阴符经》没有受到主流道教的关注,虽然李筌之后,张果也曾经作注,柳公权有《阴符经》的书法作品,但直到唐末五代杜光庭注《阴符经》,这部经才算正式被道教吸纳,因为它不是由道教内的人写的,那么被道教接受就需要一个过程。但是之后,内丹学和宋明理学都比较看重这部经,甚至认为这部经可以跟《老子》相比,所以后来《阴符经》地位比较高。 [5]

\chapter{版本}

《道藏》丛本
《四库全书》丛本
《广汉魏丛书》本
《墨海金壶》丛本


历代评价

历代经注者有:太公、范蠡、鬼谷子、张良、诸葛亮、李筌及朱熹。
唐 李筌《黄帝阴符经疏序》:“少室山达观子李筌,好神仙之道,常历名山,博采方术。至嵩山虎口岩石壁中,得阴符本,绢素书,朱漆轴,以绛缯缄之,封云:“魏真君二年七月七日,上清道士寇谦之藏诸名山,用传同好。”其本糜烂,应手灰灭。筌略抄记,虽诵在口,竟不能晓其义理。因入秦,至骊山下,逢一老母,髽髻当顶,余发倒垂,敝衣扶杖路旁。见遗火烧树,自语曰:“火生于木,祸发必克。” 筌惊而问之曰:“此是《黄帝阴符》上文,母何得而言?” 母曰:“吾受此符三元六甲周甲子矣。谨按《太一遁甲经》云: ‘一元六十岁行一甲子;三元行一百八十岁,三甲子为一周;六周积算,一千八十岁。’年少从何而知?” 筌稽首再拜,具告得处。母笑曰:“年少颧颊贯于生门,命轮齐于月角,血脑未减,心影不偏,性贤而好法,神勇而乐智,是吾弟子也。然五十六年当有大厄。”因出丹书符,冠杖端,刺筌口,令跪而吞之,曰:“天地相保。” 乃坐树下,说《阴符》玄义。言竟,诫筌曰:“《黄帝阴符》三百言,百言演道,百言演法,百言演术。参演其三,混而为一,圣贤智愚,各量其分,得而学之矣。上有神仙抱一之道,中有富国安民之法,下有强兵战胜之术。圣人学之得其道,贤人学之得其法,智人学之得其术,小人学之受其殃。识分不同也。皆内出于天机,外合于人事,若巨海之朝百谷,止水之含万象。其机张,包宇宙,括九夷,不足以为大;其机弥,隐微尘,纳芥子,不足以为小。观其精微,《黄庭》八景不足以为学;察其至要,经传子史不足以为文;任其巧智,孙吴韩白不足以为奇。是以动植之性,成败之数,死生之理,无非机者,一名《黄帝天机之书》。九窍四肢不具,悭贪、愚痴、风痫、狂诳者,并不得闻。如传同好,必清斋三日,不择卑幼,但有本者为师,不得以富贵为重、贫贱为轻,违者夺二十纪。
《河图》、《洛书》云:‘黄帝曰:圣人生,天帝赐算三万六千七百二十纪,主一岁。若有过,司命辄夺算,算尽夺纪,纪尽则身死;有功德,司命辄与算,算得与纪,纪得则身不死,长生矣。’每年七月七日写一卷,藏诸名山岩石间,得算一千二百。本命日诵七遍,令人多智慧,益心机,去邪魅,销灾害,出三尸,下九虫。所以圣人藏之金匮,不妄传也。” 母语毕,日已晡矣。曰:“吾有麦饭,相与为食。”因袖中出一瓠,令筌取水。筌往谷中盛水,其瓠忽重,可百余斤,力不能制,便沉于泉,随觅不得,久而却来,已失母所在,唯留麦饭一升。筌悲泣号诉,至夕不复见。筌乃食麦饭而归,渐觉不饥,至令能数日不食,亦能一日数食,气力自倍。筌所注《阴符》,并依骊山母所说,非筌自能。后来同好,敬尔天机,无妄传也。”
宋蹇昌辰《阴符经解》序:“……黄帝始祖,道家者流。欲广真风,得玄女三百余言,复系以一百余字,综合万化之机,权统群灵之妙,藏微隐妙,赅天括地,其经简,其意深,理归于自得者也。”
宋任照《黄帝阴符经注解》序:“阴者暗也,符者合也。故天道显而彰乎大理,人道通乎妙而不知,是以黄帝修《阴符经》以明道,与人道有暗合大理之妙,故谓之阴符焉。”
宋袁淑真《黄帝阴符经集解》序:“黄帝智穷恍惚,思极杳冥,辨天人合变之机,演阴阳动静之妙。经云:‘知之修炼,谓之圣人。’所以黄帝得之以登云天,信其明矣。黄帝阐弘道义,务救世人,诚恐后来昧于修习,乃集其要三百余言,洞启真源,传示于世。”
宋张伯端《悟真篇》:“阴符宝字逾三百,道德灵文止五千。”
宋黄庭坚:“《阴符经》出于唐李筌。熟读其文,知非黄帝书也”,“又妄托子房、孔明诸贤训注,尤可笑。惜不经柳子厚一掊击也”①。
宋伊川程子《阴符经考异》:“《阴符经》何时书 ?非商末则周末。”
宋邵子(邵雍)《阴符经考异》:“《阴符经》七国时书也。”
宋朱熹《阴符经考异》:"《阴符经》三百言,李筌得于石室中,云寇谦之所藏,出于黄帝。河南邵氏以为战国时书,程子以为非商末即周末。世数久远,不得而详言。以文字气象言之,必非古书,然非深于道者不能作也。 ……或曰此书即筌之所为,得于石室者伪也。其词支而晦,故人各得以其所见为说耳。筌本非深于道者也。是果然欲?吾不得而知也。"
宋黄震《黄氏日钞》:“经以符言.既异矣;符以阴言,尤异矣”,“言用兵而不能明其所以用兵,言修炼而不能明其所以修炼,言鬼神而不能明其所以鬼神,盖异端之士掇拾异说而本无所定见者,此其所以为阴符欤!”
宋晁公武《郡斋读书志 ·阴符经一卷》:右唐少室山布衣李筌云:《阴符经》者,黄帝之书。或曰受之于广成子或曰受之玄女。或曰黄帝与风后玉女论阴阳六甲,退而自著其书。
孟绰然《黄帝阴符经注》序:“《阴符经》三百字,言简而意详,文深而事备,天地生杀之机,阴阳造化之理,妙用真功,包涵总括尽在其中也。昔轩辕黄帝,万机之暇,渊默冲虚,获遇真经,就崆峒山而问天真皇人广成子先生,得其真趣,勤而行之。一旦鼎湖,乘火龙而登天。斯文遂传于后世也。”
王道渊《黄帝阴符经夹颂解注》序: “阴符之所以作也,昔黄帝慕道心切,故往崆峒山拜广成子而问至道,授以返还长生之诀,复于峨眉山又拜天真皇人。”
明胡应麟《笔丛·四部正伪》:……杨用修直云:筌作非也。或以唐永徽初褚遂良尝写一百本,今墨迹尚存。夫曰:遂良书则既盛行当世,筌何得托于轩辕?意世无传本,遂良奉敕录于秘书,人不恒靓也。余按《国策》,苏秦于诸侯不遂,因读阴符至刺股,则此书自战国以前有之,而《汉书艺文志》不载,盖毁于兵火。故《隋志》有《太公阴符钤录》一卷,又《周书阴符》九卷,未知孰是,当居一于斯。或疑季子所攻必权术,而《阴符》兼养生。夫《阴符》实兵家之祖,非养生可概也。此书固匪黄帝,亦匪太公,其为苏子所读则了然。
清姚际恒《古今伪书考 ·阴符经》:此书言虚无之道,言修炼之术,以 “气”作“炁”,乃道家书,必寇谦之所作而筌得之耳。其云得于石壁中,则妄也。……或谓即筌所为,亦非也,褚遂良书之以传于世。
清黄云眉《古今伪书考补证 ·阴符经》:……此乱世之书也,奈何欲上污古圣也哉! 《史记》:“苏秦得《周书阴符》而读之。” 《索隐》引《战国策》谓:“《太公阴符》之谋。”则《阴符》或即《太公兵法》?然《风后握奇经》传有吕尚增字本,此《阴符经》义殊不类,而以为出于黄帝,殆所谓无稽之言也。(《此君园文集》卷二十五)眉按: ……杨慎谓:“《阴符经》盖出后汉末。唐人文章引用者,惟吴武陵《上韩舍人行军书》有“禽之制在气”一语;梁肃《受命宝赋》有“天人合发,区宇乐推”一语;冯用之《权论》、《机论》两引用之。此外绝无及之者。”(《升庵全集》卷四十六)可知唐人见此书者极少,而慎犹疑为汉末人作,何也。
清余嘉锡《四库提要辩证·道家类·阴符经解一卷》: “案《隋书 ·经籍志》,有《太公阴符钤录》,又《周书阴符》九卷,皆不云黄帝。《集仙传》始称唐李筌于虎口岩石室得此书。题曰:“大魏真君二年七月七日,道士寇谦之藏之名山,用传同好。”已糜烂。筌抄读数千篇,竟不露其意,后于骊山遇老母,乃传授微旨,为之作注。其说怪诞不足信。胡应麟《笔丛》,乃谓苏秦所读即此书,故其书非伪,而托于黄帝,则李筌之伪。考《战国策》载,苏秦发箧得《太公阴符》具有明文。又历代史志,皆以《周书阴符》著录兵家,而《黄帝阴符》入道家,亦足为判然两书之证。应麟假借牵合,殊为未确。嘉锡案: ……昔晋哀帝兴宁二年,紫虚元君上真司命南岳魏夫人下降,授弟子杨羲以上清真经,使作隶写出,以传句容许谧并第三息许拥,事见《真诰·运题象》,于时所出道经甚多,《黄庭经》即出于是时。……其后杜京产将诸经书往剡南,吾疑《阴符经》即为此辈所作。以其有强兵战胜之术,故京产弟子孙恩,遂因之以作乱。”
清梁启超《古书真伪及其年代 ·阴符经》:“……清眺际恒曰:“必寇谦之所作,而筌得之耳。”……王谟“《阴符》是太公书兵法,以为黄帝书固谬。余则谓其文简洁,不似唐人文字,姚、王所言甚是。特亦未必太公或寇谦之所作,置之战国末,与《系辞》、《老子》同时可耳。盖其思想与二书相近也。”
清徐大椿曰:“阴符赞易之书也。”
清杨文会《阴符经发隐》曰:“隐微难见,故名为阴;妙合大道,名之为符。经者,万古之常法也,后人撰述如纬。”略补注:黄者中央之色,帝者晦明之先,中以统五行,帝以先万物,调合万有,诚乎中庸也。
历代名流学者,根据著作与历史条件,内容与风格,站在学术研究立场上分析此书,认为有成于周初、春秋战国或汉晋等朝代的黄老学派之手,判断各异。 [7]

据说《阴符经》是唐朝著名道士李筌在河南省境内的登封嵩山少室虎口岩石壁中发现的,此后才传抄流行于世。根据李筌对本经典的解释著作《黄帝阴符经疏》,可以把它的内容概括为两个部分:首先讲述观察自然界及其发展变化的客观规律,所以,天性运行为自然规律,人心则顺应自然规律;其次阐明了天、地、人生杀的变化情况,人的生杀之气的放和收,应与自然同步,才能把握好事物成功的机遇。然后,阐明人后天禀性巧拙的生成和耳目口鼻的正确运用,主要效法自然五行相生原则,修炼自身。


\mainmatter

\chapter{黄帝阴符经}

\section{上篇}
\begin{yuanwen}
观\footnote{观察。如仰观、俯视。}天\footnote{指大自然和人类社会。如指大自然时则是指星球运行状态、位移度数,把它同气象、季节联系在一起,推算律历,为利于生产、战争活动而提供条件作参考;把天象变异(如日蚀、月蚀、星殒、迅雷、烈风、暴雨)发生,用来推测吉凶祸福。《荀子·天论》主张:控制自然,为人所用。如指人类社会时则指哲学、政治、军事、经济、文化等的世界观和方法论,探测智谋韬略,以利于决策取胜,诡称为“天机”。}之道\footnote{指大自然的运转规律和人类社会的盛衰兴亡、生死得失的经验教训。},执\footnote{掌握。}天之行\footnote{这里指大自然与人类社会的运动变化的活动准则。},尽\footnote{完。如无穷无尽。}矣。
\end{yuanwen}

纵情观察大自然与人类社会的变化运转规律。掌握大自然与人类社会的变化运转规律的活动准则就是执行天机。如果能做到“观天之道,执天之行”,那么这些规律就全被理解和掌握以至于无穷无尽了。

(假如我们站在大自然和人类社会的舞台上),仰观天上的日月星辰运行状态,天体位移度数,气象条件演化,俯视人间的盛衰兴亡和生死得失的经验教训,又能执行这些共识的规律.那么天上人间活动准则的天机,就被我们完全理解与掌握以至于无穷无尽了。

观察天道运行之规律,并按照天之运行修炼自身,一切修道的内容都可以包括在其中了。

观察天体的运行规律,掌握其规律并按其规律去做,则天地阴阳动静之道就全包括在内了。

修道所追求的就是"天人合一"的境界,天人合一的程度越深,修炼的层次也就越高。故此明白了天人合一,也就等于明白了修道。这句话点明了全经主题,是全篇总纲。

\begin{yuanwen}
故天有五贼\footnote{贼害,戕害。},见\footnote{识别,发现。}之者昌\footnote{精进,成功。}。
\end{yuanwen}

天上有金木水火土五大行星,喻指五行。五行生克制化,莫不戕害我身,使我堕入其中,尝受生老病死之苦,不能做自己命运的主人。而在修练之人,则可识其贼性,探得造化之根源,使五行颠倒,造化逆行,自能反夺五行之造化,使"贼"化为"昌",反而促使我之道成。

天体有五行之气,五行相生顺时而行,则万物昌盛。

\begin{yuanwen}
五贼在心\footnote{指修练人之心。},施行于天\footnote{指身外之宇宙。}。
\end{yuanwen}

反夺五行造化,在于一心之运用。此乃空空洞洞,不执不失之道心,非世俗顽恶之人心。其所反夺之源,在于体外之宇宙。由于色身有限,宇宙无限,要从宇宙之中施行反夺,才能获取无穷之造化。修成恒古不灭之先天元神,长生久视,"天地有坏,这个不坏"。大道之奥妙,早已揭示无遗矣。

了解和掌握了五行生克制化之理,合天而行。

\begin{yuanwen}
宇宙在乎手\footnote{手掌,手通心,亦指自心。},万化生乎身。
\end{yuanwen}

人能认清五贼,追根溯源,还归本来,求得宇宙总持之门,自然成为造化主人。此时无穷宇宙,如同在我掌中;万物变化,亦好似生于自身。又手通心,亦指宇宙变化,自心了然可知。这等气魄。若非修道之士其谁人能之。

则宇宙就掌握在手中,万物就生乎身上。

\begin{yuanwen}
天性,人也。人心,机\footnote{时机(此经重在"机"字,包括所有的道功、法、道妙。吾人修道,采药得丹,全在火候,全在掌握时机)。}也。立\footnote{遵循。}天之道,以定人也。
\end{yuanwen}

吾人未生之前,不过元神混沌之体,谓之天性;既生我后,化为后天气质之性,谓之人心。天性既可化为人心,吾人自可明通此机,遵循天道,去掉人心,返归天性。老子谓之"归根复命",大道之根源在此。

天性即是人性,人性即是天机,天人合一。天道定了,人道也就定了。

\begin{yuanwen}
天发杀机,移星易\footnote{变易,变化。}宿。地发杀机,龙蛇\footnote{指水患地震。}起陆。人发杀机,天地反覆。天人合发,万(变/化)定基。
\end{yuanwen}

天发杀机,日月相蚀,陨星坠落。地发杀机,洪水地震,起于四野。人发杀机,天翻地覆,灾异横起。要在人能合乎天道,天人齐发,则万种变化,可以定其基矣,以上虽言杀机,但生杀互根,杀机即是生机。人能发杀机于天地,即是反夺生机于自身。丹道谓之"大死再活"置之死地而后生,是也。

以上虽言杀机,但是生杀互根,修练人须由此悟去,杀机即是生机。人能发杀机于天地,即是反夺生机于自身。丹经谓之"大死再活","置之死地而后生"。

五行逆行则天发杀机,星体移位,黑白颠倒,灾难将至;地发杀机,则山崩水溢,龙蛇不安其位;人发杀机,则翻天覆地,山河动摇。若是人合天机同发,则万物将在一个新的基础上定下来。

\begin{yuanwen}
性\footnote{指人心。}有巧拙,可以伏藏。九窍之邪\footnote{邪妄。},在乎三要\footnote{指耳、目、口三宝。},可以动静。
\end{yuanwen}

修练之人,要在杀机中反夺生机,必须人天合发,即人性合乎天性。但人性有巧有拙,务使伏巧为拙,使外拙而内巧,拙中藏巧,才合乎天性。但是人心有巧有拙,务使巧伏为拙,使外拙而内巧,拙中藏巧,方才合乎天性。

伏藏之道,在于九窍(即耳、目、口、鼻、脐、外肾、谷道。)九窍皆邪妄出入之门户,而关键更在于耳、目、口三者。精通于耳,气通于口,神通于目,动则外漏,静则内藏,使动化为静,则三要皆成为三宝矣。

人有聪明智慧的一面,也有愚蠢笨拙的一面,但都不要显示出来,要善于隐藏。人有九窍,能招惹邪恶是非,其中唯有耳、目、口这三者是最重要的。耳能听,目能视,口能说,它们可以动也可以静。

\begin{yuanwen}
火\footnote{指人之心火。}生于木\footnote{木能生火,喻为元神。},祸发必克。奸生于国,时动必溃。知之修炼,谓之圣人。
\end{yuanwen}

钻木取火,古人经验。但火性太炽,则木反为火伤。比喻人之心火过旺,必伤无神。推之治国,其理亦同,国家出了奸臣,祸国殃民,动荡之时,必然崩溃。犹人炼意不净,滋生妄念,定有伤丹之度。可见祸福生杀,太过不及,差之毫厘,谬之千里。识得其机,修之炼之,才是圣人。

火生于木,火燃烧起来木就变成灰烬了。奸贼生于国内,若奸贼得逞则国家就要灭亡。唯有圣人能修身炼性,防微杜渐。

\section{中篇}

\begin{yuanwen}
天生天杀,道之理也。
\end{yuanwen}

天生天杀,阴阳消长,乃顺行之自然。

天生万物(春生夏长),天亦杀万物(秋敛冬藏),这是自然规律。

\begin{yuanwen}
天地,万物之盗\footnote{逆取,反夺。}。万物,人之盗。人,万物之盗。三盗既宜\footnote{平衡,协调。},三才既安。故曰:“食其时,百骸理。动其机,万化安。”人知其神之神,不知不神之所以神也。日月有数,大小有定。圣功生焉,神明出焉。其盗机也,天下莫能见,莫能知。君子得之固躬,小人得之轻命。
\end{yuanwen}

但天杀之机,即是反夺生气之机也,又为逆行修道之枢要。天地从万物中反夺,万物从人中反夺,人从万物中反夺。三者互相反夺,合平衡,才合乎生杀之道,成为自然。

天地生万物亦杀万物,万物生人亦杀人,人生万物亦杀万物。三者相互为盗又寓相生之理,使天、地、人各得其位,各司其职,万物育生。





译文:“看上天运行的轨迹,做上天赋予的使命,(万事万物的奥妙)就尽了。天有金木水火土(五行相克),看见的人会昌盛。五行在心中体会,施行合天的行动。这样,宇宙虽大,仍在一掌之中(天地都来一掌中),千变万化,不出一身之外(人身为一小天地)。”

译文:“上天之性是人的根本,人心却是诈伪。所以要以上天之道来定人心。”

译文:“上天若出现五行相克,就会使星宿移位;大地若出现五行相克,就会使龙蛇飞腾;人体内若出现五行相克,就能使小天地颠倒。倘若人能顺应自然而同时发生五行相克,就能使各种变化稳定下来。”

译文:“人性虽有巧有拙,却可以隐藏起来。九窍是否沾惹外邪,关键在于耳、目、口三窍之动静。三窍动则犹如木头着火,灾祸发生必被攻克;如x有奸邪,时间一到必致溃亡。懂得如此修炼,称为圣人。”



译文:“万物顺应天地之规律而自然生长;人利用万物而富足;万物依靠人而昌盛。只要天地、万物与人之间各得其宜,那么它们就会安定下来。所以说:休养要遵循时令,身体才会得到调理;行动要把握时机,万物才会变得安定。人们只懂得“盗”的神妙莫测而以为神(世人只知偷盗不被查觉,谓之‘神’),却不知“盗”不神妙莫测才是最神妙莫测的(却不知顺天地、万物之规律而公开盗之,方为‘神’)。要知道,太阳与月亮各有规律,大与小都有定规,只有懂得这些道理,才会有大功产生,才会有神明护佑。这些“盗”的机巧是天下之人所不能见、所不能知的。有悟性的人得到它,就会躬行(能顺应自然);无悟性的人得到它,却会丧命(因违法偷盗)。”

\section{下篇}

\begin{yuanwen}
瞽者善听,聋者善视。绝利一源,用师十倍。三(反/返)昼夜,用师万倍。心生于物,死于物,机在于目。天之无恩,而大恩生。迅雷烈风,莫不蠢然。
\end{yuanwen}



故曰食其时,百骸理。动其机,万化安。人知其神之神,不知不神所以神也。


【字解】食,掌握,采取。动,发动。


【释义】欲求修炼,贵在能知生杀予夺之时机。按时采取,从天地万物中反夺生机,陶铸自身筋骨,才能成为乾健之躯。乘机发动,借生杀变化之机,反夺造化,安定自身。丹功每次提高阶段,都在掌握时机。


平常人只知后天思虑之神为神,不知先天不神之神,才是真神。大要修道,先使后天识神归于先天不神,空空洞洞,虚灵不昧,才能时至神知,机动觉随,反夺造化,调理百骸,得成修炼之动。


【白话文】


所以古语说:“饮食得其时,则人体得到调养生息;行动符合天机,则万物安泰。”人只知道万物从阴阳而生,将这看不见、摸不着的东西称之为“神”;却不知道这个“神”是从至道虚无的“不神”而来的。


图片​


日月有数,小大有定。圣功生焉,神明出焉。


【字解】数,定数。定,周期。圣功,即修练之功。


【释义】太阳东升西降,月亮晦朔弦望,皆有定数。小往则大来,大往则小来,阳大阴小,与日月之出没相同。我能知往推来,食其时而动其机,采日精月华,夺天地正气,自可完成修真成圣之动,神明由此而出。


【白话文】


日月之行必有常数,日月之大小也有定数。圣人掌握其规律推而测之,就能昭示神明之道。



图片​


其盗机也,天下莫能见,莫能知。君子得之固躬,小人得之轻命。


【字解】固,固然。躬,躬敬,谨慎。轻,丧失,夭折。命,生命。


【释义】盗机即反夺之机也,反夺造化之功,皆无形象可言,若有形象,便落后天,故天下无见之知之者。先有见知,便失真机。采炼之时,若为"寂然不动,感而遂通"之先天元肖,无形无象,不可得见,则可成丹。此时若动情识,迅即化为后天浊质,可以见如,必有走漏之危,即使追回采炼,亦不能成丹。此反夺之机,君子得之固然谨慎,倍受奉行,可以长生久视。小人得之轻视造化,修功差驰,反促其寿也。


【白话文】


三盗的实现都是在寂静中完成的,形迹未露,人们看不见、摸不着,求知者对它认识不一。君子得之能顺时而行,用以健壮身体、修身养性;小人得之则违时而行,恃才妄为,反而害身。


图片​


瞽者善听,聋者善视。绝利一源,用师十倍:三反昼夜,用师万倍。


【字解】师,兵事。修道与用兵理同。


【释义】双目失明的人,视不外漏,专一于耳,所以听觉灵敏。两耳失聪的人,听不外漏,专一于目,因而视觉灵敏。专心用于一处,便可得到用兵十倍的效力。反复昼夜地不断用心,则可得到用兵万倍的效力。丹法与用兵相同,二者一理,运用之妙,都在专一。


【白话文】


瞎子有目不能视,却擅长于听;聋子有耳不能听,却擅长于视。不能听或不能视就杜绝了外界的种种诱惑,胜于众人十倍;如若再能昼夜反省自己,则就胜于众人万倍了。

图片​


心生于物,死于物,机在目。


【字解】心,指人心。目,指眼睛。


【释义】人生之初,心本虚空,渐为外物所扰,因而产生各种念想,损人心性,损尽则死。修道下手,还虚第一,盖"魔由心生,境由心造",心若不虚,反而自惹魔障,坏我功修。故须收心离境,聚性止念。其机在目,神生于心,发于二目,乃丹动之枢机。内视、采药、烹炼、养胎及至出神等等,均以目力机。


【白话文】


心(人的思想)来源于客观事物,并随着客观事物的变化而变化、发展而发展、灭亡而灭亡,其机关就在于目(目是“三要”之首)。


图片​


天之无恩而大恩生,迅雷烈风莫不蠢然。


【字解】天,指天空,于天道。


【释义】天本空空洞洞,无识无知,毫无施恩之意,而其行四时,育万物,大恩遂生焉。迅雷烈风受其驱使,而蠢蠢然不能自主。此乃大道隐含之力量,不可思议,修真悟道之士,当由此参证之。


【白话文】


宇宙天体按着自己的规律运行着,无意施恩于万物,而万物却得其恩泽雨露而生长。阴阳相交,产生了雷电风雨,使万物自由自在地发育生长。


\begin{yuanwen}
至乐性(馀/愚),至静性廉。天之至私,用之至公。
\end{yuanwen}

【字解】至,到,真正。余,余闲。廉,清廉。


【释义】至乐的人,心胸坦荡,性有余闲。至静的人,心性收敛,廉而不失。修炼的人悟到虚静之时,心忽开朗,舒适畅快,妙不可言,就是达到至乐至静的境界了。


天道驱风使雷,运行四时,看似至私,而作用于万物生化,却无偏无倚,一视同仁,实为至公。犹天性降之于人,虽为个人所私,实际贤愚皆同,人人均有。天性与太虚等量,大公无私,若至私而实至公也。


【白话文】


人的性格有“至乐”“至静”之分,至乐者性格开朗,宽裕优容;至静者思维缜密,廉洁无染。天也有“至公”和“至私”两个方面,它将天地万物都包容于一身,似其自私;然而万物又无偿地使用它,则又大公无私。


\begin{yuanwen}
禽之制在(气/炁)。生者死之根,死者生之根。恩生于害,害生于恩。
\end{yuanwen}


【字解】禽,通"擒",制服之意。


【释义】制服的诀窍在于肖,肖聚则生,肖散则死。生与死互为本根,生于何处,死于何处,人由男女而生,亦因男女而死。恩害相生,亦同于生死。由人心返还天性,为死处求生,是谓逆则成仙,即恩生于害;由天性降落人心,为生老病死,是谓顺则生人,即害生于恩。


【白话文】


统摄万物者,制造万物者,都在于一气。万物有生必有死,则生乃死之根源;有死必有生,则死又是生的根源。人间社会也是如此,无害则无恩,因救害而有了恩;若知恩不报,则害又生于恩。


\begin{yuanwen}
愚人以天地文理圣,我以时物文理哲。人以愚虞圣,我以不愚虞圣。人以奇期圣,我以不奇期圣。
\end{yuanwen}




【字解】愚人,常人,一般的人。我,指修道之人。


【释义】愚人以天文地理为神圣,我以随机应变为原则。人以愚弄欺骗为神圣,我以不言而信为神圣。人以惊世骇俗为神圣,我以和光同尘为神圣。这些都是修德之要,无德便不能培道。


道家认为,道在我身上就是德;没有德也就失去了道。有人做功出魔,或功夫停滞,就因为不注重修德之缘故。


【白话文】


愚蠢的人认为,天地万物都是无形的,是不可知的神圣之物;而我却认为,天地万象都是有形的,是可知的。有的人用愚蠢的办法揣测、预料天地的表面现象,以为自得,自称为圣人;我却认为,聪明智慧,能体察万物之理的人为圣人。人们以为能够推度出神奇事物的出现是圣人;我却认为,能体察天地、成就万物“不奇期”者是圣人。

\begin{yuanwen}
故曰:沉水入火,自取灭亡。自然之道静,故天地万物生。天地之道浸,故阴阳胜。阴阳相推,而变化顺矣。是故圣人知自然之道不可违,因而制之。至静之道,律历所不能契。爰有奇器,是生万象。八卦甲子,神机鬼藏。阴阳相胜之术,昭昭乎进乎象矣。
\end{yuanwen}


【字解】水,指肾水,在易象为坎卦。火,指心火,在易象为离卦。


【释义】以坎水填入离火之中,使后天坎离复为先天乾坤,则人心灭亡,而天性复现,到此筑基完成。


【白话文】


故古语说:“(追求“愚虞”和“奇期”的)如同于投于水火之中,自取灭亡。”





【字解】浸,浸润,充满。胜,主宰。


【释义】自然之道,主静立极,空空洞洞,无中生有,《老子》曰:"清静为天下正。"天地万物遂得以生化。天地之道充满其中,天为阳,地为阴,因此阴阳之道主宰于万事万物之中。静极生动,阴极生阳,阳极消长,互资互根。如此相推,则天地万物生生化化,顺其自然,不失其序也。


【白话文】


天地日月依其自然规律静静地运行着,故天地万物得以生存、生长。天地之道是浸润渐进的,阴极生阳,阳极生阴,阴阳相互转化,而四季成序,万物生长,都按照一定的规律自然顺畅的运动。




【字解】违,违抗,改移。制,制订,采用。契,契合,规定。


【释义】圣人明白自然之道不可随意违抗,因而采用至静之法。只有静才能体悟天道,才能识别五贼,才能天人合发,才能反夺造化。一切修为,都是从静中自然生出。能静片刻,可以攒簇一年之气候,这是律历所不能规定的。


【白话文】


所以,圣人知道天地自然规律是不可违背的,人应该与天地合,顺其道而行。至静之道是无形的,而天文历法是有形的,却无法揭示和包括无形的“至静之道”。





【字解】奇器,奇异之器。万象,万象变化。象物象,指事物的本来面目。



【释义】有奇异之器,才能产生万象。八卦甲子之中,藏有鬼神莫测之机。阴阳相胜的法则,昭昭然可以揭示事物的本来面目了。


《参同契》、《悟真篇》等类丹经,多以卦象干支描述自身修练中的阴阳变化,亦宗此义也。


【白话文】


于是,圣人就发明了一种神奇的东西,用以昭示天地万物之象,这就是阴阳八卦和六十甲子,神之申机,鬼之屈藏,无不包括在内。阴阳相生相克之术,与天地之道暗合(阴符),故能昭示天地之间的万象。


译文:“眼盲者善长听,耳聋者善长看。(因此,如果能)断绝或助利其一(或眼或耳),就会增强十倍之能力;如果能每天断绝耳、目、口(勿听、勿视、勿言),就会增强万倍之能力。心因万物而躁生,因万物而寂灭,关键在于眼。(要知道,)上天不施恩德(无声无言),(因)而能产生大恩德;(而)响雷暴风(指外物)只会使万物发生骚动。”

译文:“至乐在于知足,至静在于无私。上天因无恩而至私,故能大恩而至公(施惠于万物)。统摄的法式在于调和其气。”

译文:“生为死之根源,死为生之根源。利因害而生,害亦因利而生。”

译文:“愚昧之人常以懂得天地之准则为智慧,我却以遵循时令、洞悉外物为聪明;俗人以欺诈为智慧,我却不以欺诈为聪明;俗人以奇异为智慧,我却不以奇异为聪明。所以说:以欺诈与奇异行事,如水入火,自取灭亡。”

译文:“自然之道为静,所以能生天地万物。天地的运行遵循自然,所以能使阴阳相胜。阴阳相胜相生,则变化和谐。”







译文:“所以,圣人懂得自然之道不可违背,因而制订了各种法则。然而,至静之道是乐律和历法所不能契合的。于是就有了奇妙的《易》,它产生了各种象征,是以八种卦象为本,并贯以六十甲子,来演化种种玄机的。这样一来,阴阳循环相生也就能很清楚地蕴涵于各种象征之中了。”(最后这几段是宋明理学家或内丹家所加,意在尊孔易(周易)贬老道(归藏)) [6]

\backmatter

\end{document}