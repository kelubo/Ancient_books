% 黄帝阴符经
% 黄帝阴符经.tex

\documentclass[12pt,UTF8]{ctexbook}

% 设置纸张信息。
\usepackage[a4paper,twoside]{geometry}
\geometry{
	left=25mm,
	right=25mm,
	bottom=25.4mm,
	bindingoffset=10mm
}

% 设置字体,并解决显示难检字问题。
\xeCJKsetup{AutoFallBack=true}
\setCJKmainfont{SimSun}[BoldFont=SimHei, ItalicFont=KaiTi, FallBack=SimSun-ExtB]

% 目录 chapter 级别加点(.)。
\usepackage{titletoc}
\titlecontents{chapter}[0pt]{\vspace{3mm}\bf\addvspace{2pt}\filright}{\contentspush{\thecontentslabel\hspace{0.8em}}}{}{\titlerule*[8pt]{.}\contentspage}

% 设置 part 和 chapter 标题格式。
\ctexset{
	chapter/name={},
	chapter/number={}
}

% 设置古文原文格式。
\newenvironment{yuanwen}{\bfseries\zihao{4}}

% 设置署名格式。
\newenvironment{shuming}{\hfill\bfseries\zihao{4}}

% 注脚每页重新编号,避免编号过大。
\usepackage[perpage]{footmisc}

\title{\heiti\zihao{0} 黄帝阴符经}
\author{}
\date{}

\begin{document}

\maketitle
\tableofcontents

\frontmatter
\chapter{前言}

《黄帝阴符经》又称《阴符经》、《黄帝天机经》。李筌分为神仙抱一之道、富国安人之法、强兵战胜之术。全书以隐喻论述养生,愚者不察,谓兵法权谋等说或谓苏秦之“太公阴符之谋”皆离旨甚远。

作者,旧题黄帝撰。疑似作者,黄帝、苏秦、寇谦之、李筌。

创作年代,商朝/战国/北魏/唐朝

《黄帝阴符经》与《混元阳符经》相配,论涉养生要旨、气功、八卦、天文历法等方面。关于成书有人说黄帝,有人说是战国时的苏秦,近代学者多认为其成书于南北朝。

作为一部高度精炼的道教经书,《黄帝阴符经》正如其他许多具有理性精神之道教学者所撰之作品一样,不是简单因袭易学义理派的言辞,而是运用其义理思维,以《易》通《老》,演述神仙抱一之道、富国安人之法、强兵战胜之术,全书以隐喻论述养生,愚者不查谓兵法权谋等说或谓苏秦之“太公阴符之谋”皆离旨甚远。如道教《纯阳演正孚佑帝君既济真经》,通篇全部以军事术语写成,不知者初见会认定是一篇兵书。因此李筌、张果老、朱熹等人曾先后为《阴符经》作注。朱熹虽然认为其伪但认为“非深于道者不能作”。 [1]

《阴符经》旧题黄帝撰,所以也叫做《黄帝阴符经》。因而有题称伊尹、太公、范蠡、鬼谷子、张良、诸葛亮等注解。这一说,最不合理。宋黄庭坚说:“《阴符经》出于唐李筌。熟读其文,知非黄帝书也”,“又妄托子房、孔明诸贤训注,尤可笑。惜不经柳子厚一掊击也”①。好事者说黄帝撰经,并且假托太公、张良等作注,这些都是显明的依托古人说法,不可置信。自唐李筌为《阴符经》作注,以后累朝均不乏好事者步其后尘,迨至晚清,《阴符经》注解本已不下百余种,今仅存于明《正统道藏》的便有二十四种。注解虽多,但众说纷纭,见解芜杂。

《阴符经》的作者,历来说法不一,共有四种说法:

第一种观点认为是黄帝所撰,伊尹、太公、范蠡、鬼谷子等注。

第二种观点认为是北魏寇谦之所作,其根据是因为杜光庭《神仙感遇传》谓此书是“上清道士寇谦之藏诸名山”。

第三种观点认为是唐代中期的李荃所作,持这种观点的有宋代的黄庭坚、朱熹等。

第四种观点认为是南北朝时一位“深于道者”所作。

朱熹在《阴符经考异序》引:“

邵子曰:《阴符经》七国时书也。

伊川程子曰:《阴符经》何时书 ?非商末则周末。“

但是朱熹并不同意他们的说法,宋朱熹《阴符经考异》中说:"《阴符经》三百言,李筌得于石室中,云寇谦之所藏,出于黄帝。河南邵氏以为战国时书,程子以为非商末即周末。世数久远,不得而详言。以文字气象言之,必非古书,然非深于道者不能作也。 ……或曰此书即筌之所为,得于石室者伪也。其词支而晦,故人各得以其所见为说耳。筌本非深于道者也。是果然欲?吾不得而知也。"

《道藏》中收录的《黄帝阴符经疏序》词上有差异,即言“魏真君”而不言的记载与此基本相同,只是在个别字“大魏真君”。《阴符经》全称《黄帝阴符经》,古以为出自黄帝之手,此当属托名。不过,其说却自有来历。宋代以来学者,始对此书之作者及产生年代提出质疑,但意见颇不一致。宋人黄庭坚《山谷题跋》及朱熹《阴符经考异》以为此书是李筌假托黄帝名自造;清人姚际恒、全祖望等学者认为此书乃魏寇谦之伪托;今人余嘉锡及王明先生均对此有考辨。

余氏《四库提要辨证》指出:“昔晋哀帝兴宁二年紫虚元君上真司命南岳魏夫人下降,授弟子杨羲以《上清真经》,使作隶字写出,以传句容许谧并第(弟)三息许翔,事见《真诰·运题象》。于时所出道经甚多,《黄庭经》即出于是时,……其后杜京产将诸经书往剡南,吾疑《阴符经》即为此辈所作。以其有强兵战胜之术,故京产弟子孙恩遂因之以作乱。”杜京产为魏晋时人,余氏以为此书为魏晋人杜京产所作。王明先生认为:《阴符经》的一个重要思想“天地,万物之盗;万物,人之盗”,不见于古籍,最早出自《列子·天瑞篇》。

王氏引述了《列子》书多条行文来证实《阴符经》之思想来源出自《列子》,又据一些学者关于《列子》属“伪书”的观点,认定《阴符经》当出南北朝时道门中人或当时隐者之手。虽然《阴符经》之思想来源出自《列子》是无疑的,但近年来许多学者对《列子》一书之年代重新进行考证,如许抗生先生所作《列子考辨》,根据先秦与两汉不少典籍引用《列子》文句的事实,认为《列子》当属战国时代之作品,但在许多地方经过后人增改。⑧《列子》为早期黄老道家典籍。如此一来,则《阴符经》之出世年代是否属于南北朝则尚待进一步研究。

注解及杂著

《黄帝阴符经》在宋郑樵《通志》上所载书目共有39种之多,明《正统道藏》所收的成书也不下20种,后之学者纵然把这些注解都阅遍了,恐怕仍旧不能明白《黄帝阴符经》是怎么一回事。《战国策》言:苏秦得“太公阴符之谋,伏而读之”;而《史记·苏秦传》则言: “得周书阴符,伏而读之”。西汉国家藏书目录《汉书·艺文志》道家类曾有著录曰:“《太公》237篇,《谋》81篇,《言》71篇,《兵》85篇。”班固注“吕望为周师尚父,本有道者。”清沈钦韩说:《谋》者即太公之《阴谋》,《言》者即太公之《金匮》,《兵》者即《太公兵法》,苏秦曾得《太公谋》八十一篇,从中悟出纵横术 [2]。至于《道藏》中现存之《黄帝阴符经》,很容易看出它是专门修炼家言,与兵家无涉,凡以兵家的权谋术数作注者皆文不对题,其间亦有不谈兵而泛论国家政治及人事得失者,都与《黄帝阴符经》的宗旨相去甚远。宋儒朱熹虽不识《黄帝阴符经》作用,但也有几句好评的。

《黄帝阴符经》原文有300余字的,也有400余字的,何种版本为可靠?已无从断定。所幸其中要紧的话在各种版本上都一致保存,大体尚无妨碍。惟注解总嫌芜杂,阅之徒乱人意。有些地方,经文并不难懂,如果看了注解以后,再和经文两相对照,就觉得满纸都是荆棘。不知它在那里说什么话?即如经文“君子得之固躬,小人得之轻命”,本意是说正派人得到这个法子,能够使自己身体坚固;邪派人得到这个法子,反而轻易促短自己的寿命。有些版本把“固躬”改作“固穷”,或许因为《论语》有“君子固穷”之说,遂妄改之,但不思与上文“其盗机也,天下莫能见,莫能知“三句怎样可以连在一起?“盗机”的作用和“固穷”的品格究竟有什么相干?又如经文“天人合发。’’一句,本是修炼家的专门术语,注家不得其解,把它改作“天人合德”,一字之差,竟至点金成铁。而且“天人合发”的“发”字是根据上文“天发杀机、人发杀机”两句而来,若把“发”字改为“德”字,试问有何根据?又如“天发杀机,龙蛇起陆”,原文只有两句,后来各种版本把两句改成四句,而改法又不相同:(1)“天发杀机,移星易宿;地发杀机,龙蛇起陆”。(2)“天发杀机,龙蛇起陆,地发杀机,星辰陨伏”。(3)“天发杀机,星辰陨伏;地发杀机,龙蛇起陆”。他们所添改的四句都不及原来的两句好,反而觉得累赘。原文是“天”与“人”相对待,不需要把“地”排列进去;他们把“天、地、人”三才并列,遂失却原文的意旨。原文“龙蛇起陆”是隐语,不是真有这件事,而他们当真地认为龙蛇在地下潜藏不住,都跑到地面上来了,因此就凭自己的理想,加入“地发杀机”一句;又因‘‘天发杀机’’没有下文,变成孤立的句子,于是再用“星辰陨伏”或“移星易宿,,以补足原文语气,读者更莫名其妙。又如经文‘‘其盗机也,天下莫能见,莫能知”,而李筌的注本上则多了两个‘‘不’’字,作“天下莫不能见,莫不能知”,这是显然的错误,但李筌并未加以校正,而且将错就错的曲为之说,原来是很容易懂的话,竞弄得非常难懂,所以后人读《黄帝阴符经》,最好不要看各家注解。 [3]

关于《阴符经》又称《黄帝阴符经》。经文很短,共有400余字;但据一般说,从“观天之道”起,至“我以时物文理哲”为止,是它的原文,仅300余字,所以《悟真篇》云:“阴符宝字逾三百。”自“我以时物文理哲”以下100余字,说是后人增补,但这一段文字,是宋代以来即已经有了的,如朱熹在注《阴符经》时,即非常赞赏其中的“自然之道静,故天地万物生;天地之道浸,故阴阳胜”几句话,他说:“四句说得极妙”。又说:“浸字下得最好”。也有人说:这一段最早见于柳公权书《阴符》(《宣和书谱》有唐柳公权书《阴符经》),如《黄帝阴符经注解》引高氏《纬略》说:“蔡端明云:柳书《阴符经》之最精者,善藏笔锋”。那么,应当更早了。至于它究竟是多少字?因为各家传本不同,我们也不能肯定。

它的内容,各家看法并不一致,悬殊很大。有的认为它是谈道家修养方法的书,但其中又有谈“道”和谈“丹”之分;有的认为它是纵横家的书,所谈都是权谋术数;也有人认为它是兵家的书。比较来说,以第一种看法为多,因为在《阴符经》上篇中是很清楚地说出“知之修炼是谓圣人”。可见它的宗旨所在,是说道家的修养方法,主要是“观天之道,执天之行”,并认为能够做到这一点就可以“宇宙在乎手,万化生乎身”,也就是掌握了长生久视的自主之权。宋代的学者,像周敦颐、程颐、程颢、朱熹他们都很喜欢《阴符经》,对这一部书十分推重。但当时也有一些学者则不同意他们的看法,如黄震说:“经以符言.既异矣;符以阴言,尤异矣”。又说它“言用兵而不能明其所以用兵,言修炼而不能明其所以修炼,言鬼神而不能明其所以鬼神,盖异端之士掇拾异说而本无所定见者,此其所以为阴符欤!” [4]

成书背景

《阴符经》传说是轩辕黄帝所写,但实质上不可能,有人说出于先秦,最早给它写注的李筌说是寇谦之所传并藏之于名山,这些都是传说,现代学者认为是北朝的人所写,而且最初与道教无关。事实在唐代,《阴符经》没有受到主流道教的关注,虽然李筌之后,张果也曾经作注,柳公权有《阴符经》的书法作品,但直到唐末五代杜光庭注《阴符经》,这部经才算正式被道教吸纳,因为它不是由道教内的人写的,那么被道教接受就需要一个过程。但是之后,内丹学和宋明理学都比较看重这部经,甚至认为这部经可以跟《老子》相比,所以后来《阴符经》地位比较高。 [5]

版本

《道藏》丛本
《四库全书》丛本
《广汉魏丛书》本
《墨海金壶》丛本


历代评价

历代经注者有:太公、范蠡、鬼谷子、张良、诸葛亮、李筌及朱熹。
唐 李筌《黄帝阴符经疏序》:“少室山达观子李筌,好神仙之道,常历名山,博采方术。至嵩山虎口岩石壁中,得阴符本,绢素书,朱漆轴,以绛缯缄之,封云:“魏真君二年七月七日,上清道士寇谦之藏诸名山,用传同好。”其本糜烂,应手灰灭。筌略抄记,虽诵在口,竟不能晓其义理。因入秦,至骊山下,逢一老母,髽髻当顶,余发倒垂,敝衣扶杖路旁。见遗火烧树,自语曰:“火生于木,祸发必克。” 筌惊而问之曰:“此是《黄帝阴符》上文,母何得而言?” 母曰:“吾受此符三元六甲周甲子矣。谨按《太一遁甲经》云: ‘一元六十岁行一甲子;三元行一百八十岁,三甲子为一周;六周积算,一千八十岁。’年少从何而知?” 筌稽首再拜,具告得处。母笑曰:“年少颧颊贯于生门,命轮齐于月角,血脑未减,心影不偏,性贤而好法,神勇而乐智,是吾弟子也。然五十六年当有大厄。”因出丹书符,冠杖端,刺筌口,令跪而吞之,曰:“天地相保。” 乃坐树下,说《阴符》玄义。言竟,诫筌曰:“《黄帝阴符》三百言,百言演道,百言演法,百言演术。参演其三,混而为一,圣贤智愚,各量其分,得而学之矣。上有神仙抱一之道,中有富国安民之法,下有强兵战胜之术。圣人学之得其道,贤人学之得其法,智人学之得其术,小人学之受其殃。识分不同也。皆内出于天机,外合于人事,若巨海之朝百谷,止水之含万象。其机张,包宇宙,括九夷,不足以为大;其机弥,隐微尘,纳芥子,不足以为小。观其精微,《黄庭》八景不足以为学;察其至要,经传子史不足以为文;任其巧智,孙吴韩白不足以为奇。是以动植之性,成败之数,死生之理,无非机者,一名《黄帝天机之书》。九窍四肢不具,悭贪、愚痴、风痫、狂诳者,并不得闻。如传同好,必清斋三日,不择卑幼,但有本者为师,不得以富贵为重、贫贱为轻,违者夺二十纪。
《河图》、《洛书》云:‘黄帝曰:圣人生,天帝赐算三万六千七百二十纪,主一岁。若有过,司命辄夺算,算尽夺纪,纪尽则身死;有功德,司命辄与算,算得与纪,纪得则身不死,长生矣。’每年七月七日写一卷,藏诸名山岩石间,得算一千二百。本命日诵七遍,令人多智慧,益心机,去邪魅,销灾害,出三尸,下九虫。所以圣人藏之金匮,不妄传也。” 母语毕,日已晡矣。曰:“吾有麦饭,相与为食。”因袖中出一瓠,令筌取水。筌往谷中盛水,其瓠忽重,可百余斤,力不能制,便沉于泉,随觅不得,久而却来,已失母所在,唯留麦饭一升。筌悲泣号诉,至夕不复见。筌乃食麦饭而归,渐觉不饥,至令能数日不食,亦能一日数食,气力自倍。筌所注《阴符》,并依骊山母所说,非筌自能。后来同好,敬尔天机,无妄传也。”
宋蹇昌辰《阴符经解》序:“……黄帝始祖,道家者流。欲广真风,得玄女三百余言,复系以一百余字,综合万化之机,权统群灵之妙,藏微隐妙,赅天括地,其经简,其意深,理归于自得者也。”
宋任照《黄帝阴符经注解》序:“阴者暗也,符者合也。故天道显而彰乎大理,人道通乎妙而不知,是以黄帝修《阴符经》以明道,与人道有暗合大理之妙,故谓之阴符焉。”
宋袁淑真《黄帝阴符经集解》序:“黄帝智穷恍惚,思极杳冥,辨天人合变之机,演阴阳动静之妙。经云:‘知之修炼,谓之圣人。’所以黄帝得之以登云天,信其明矣。黄帝阐弘道义,务救世人,诚恐后来昧于修习,乃集其要三百余言,洞启真源,传示于世。”
宋张伯端《悟真篇》:“阴符宝字逾三百,道德灵文止五千。”
宋黄庭坚:“《阴符经》出于唐李筌。熟读其文,知非黄帝书也”,“又妄托子房、孔明诸贤训注,尤可笑。惜不经柳子厚一掊击也”①。
宋伊川程子《阴符经考异》:“《阴符经》何时书 ?非商末则周末。”
宋邵子(邵雍)《阴符经考异》:“《阴符经》七国时书也。”
宋朱熹《阴符经考异》:"《阴符经》三百言,李筌得于石室中,云寇谦之所藏,出于黄帝。河南邵氏以为战国时书,程子以为非商末即周末。世数久远,不得而详言。以文字气象言之,必非古书,然非深于道者不能作也。 ……或曰此书即筌之所为,得于石室者伪也。其词支而晦,故人各得以其所见为说耳。筌本非深于道者也。是果然欲?吾不得而知也。"
宋黄震《黄氏日钞》:“经以符言.既异矣;符以阴言,尤异矣”,“言用兵而不能明其所以用兵,言修炼而不能明其所以修炼,言鬼神而不能明其所以鬼神,盖异端之士掇拾异说而本无所定见者,此其所以为阴符欤!”
宋晁公武《郡斋读书志 ·阴符经一卷》:右唐少室山布衣李筌云:《阴符经》者,黄帝之书。或曰受之于广成子或曰受之玄女。或曰黄帝与风后玉女论阴阳六甲,退而自著其书。
孟绰然《黄帝阴符经注》序:“《阴符经》三百字,言简而意详,文深而事备,天地生杀之机,阴阳造化之理,妙用真功,包涵总括尽在其中也。昔轩辕黄帝,万机之暇,渊默冲虚,获遇真经,就崆峒山而问天真皇人广成子先生,得其真趣,勤而行之。一旦鼎湖,乘火龙而登天。斯文遂传于后世也。”
王道渊《黄帝阴符经夹颂解注》序: “阴符之所以作也,昔黄帝慕道心切,故往崆峒山拜广成子而问至道,授以返还长生之诀,复于峨眉山又拜天真皇人。”
明胡应麟《笔丛·四部正伪》:……杨用修直云:筌作非也。或以唐永徽初褚遂良尝写一百本,今墨迹尚存。夫曰:遂良书则既盛行当世,筌何得托于轩辕?意世无传本,遂良奉敕录于秘书,人不恒靓也。余按《国策》,苏秦于诸侯不遂,因读阴符至刺股,则此书自战国以前有之,而《汉书艺文志》不载,盖毁于兵火。故《隋志》有《太公阴符钤录》一卷,又《周书阴符》九卷,未知孰是,当居一于斯。或疑季子所攻必权术,而《阴符》兼养生。夫《阴符》实兵家之祖,非养生可概也。此书固匪黄帝,亦匪太公,其为苏子所读则了然。
清姚际恒《古今伪书考 ·阴符经》:此书言虚无之道,言修炼之术,以 “气”作“炁”,乃道家书,必寇谦之所作而筌得之耳。其云得于石壁中,则妄也。……或谓即筌所为,亦非也,褚遂良书之以传于世。
清黄云眉《古今伪书考补证 ·阴符经》:……此乱世之书也,奈何欲上污古圣也哉! 《史记》:“苏秦得《周书阴符》而读之。” 《索隐》引《战国策》谓:“《太公阴符》之谋。”则《阴符》或即《太公兵法》?然《风后握奇经》传有吕尚增字本,此《阴符经》义殊不类,而以为出于黄帝,殆所谓无稽之言也。(《此君园文集》卷二十五)眉按: ……杨慎谓:“《阴符经》盖出后汉末。唐人文章引用者,惟吴武陵《上韩舍人行军书》有“禽之制在气”一语;梁肃《受命宝赋》有“天人合发,区宇乐推”一语;冯用之《权论》、《机论》两引用之。此外绝无及之者。”(《升庵全集》卷四十六)可知唐人见此书者极少,而慎犹疑为汉末人作,何也。
清余嘉锡《四库提要辩证·道家类·阴符经解一卷》: “案《隋书 ·经籍志》,有《太公阴符钤录》,又《周书阴符》九卷,皆不云黄帝。《集仙传》始称唐李筌于虎口岩石室得此书。题曰:“大魏真君二年七月七日,道士寇谦之藏之名山,用传同好。”已糜烂。筌抄读数千篇,竟不露其意,后于骊山遇老母,乃传授微旨,为之作注。其说怪诞不足信。胡应麟《笔丛》,乃谓苏秦所读即此书,故其书非伪,而托于黄帝,则李筌之伪。考《战国策》载,苏秦发箧得《太公阴符》具有明文。又历代史志,皆以《周书阴符》著录兵家,而《黄帝阴符》入道家,亦足为判然两书之证。应麟假借牵合,殊为未确。嘉锡案: ……昔晋哀帝兴宁二年,紫虚元君上真司命南岳魏夫人下降,授弟子杨羲以上清真经,使作隶写出,以传句容许谧并第三息许拥,事见《真诰·运题象》,于时所出道经甚多,《黄庭经》即出于是时。……其后杜京产将诸经书往剡南,吾疑《阴符经》即为此辈所作。以其有强兵战胜之术,故京产弟子孙恩,遂因之以作乱。”
清梁启超《古书真伪及其年代 ·阴符经》:“……清眺际恒曰:“必寇谦之所作,而筌得之耳。”……王谟“《阴符》是太公书兵法,以为黄帝书固谬。余则谓其文简洁,不似唐人文字,姚、王所言甚是。特亦未必太公或寇谦之所作,置之战国末,与《系辞》、《老子》同时可耳。盖其思想与二书相近也。”
清徐大椿曰:“阴符赞易之书也。”
清杨文会《阴符经发隐》曰:“隐微难见,故名为阴;妙合大道,名之为符。经者,万古之常法也,后人撰述如纬。”略补注:黄者中央之色,帝者晦明之先,中以统五行,帝以先万物,调合万有,诚乎中庸也。
历代名流学者,根据著作与历史条件,内容与风格,站在学术研究立场上分析此书,认为有成于周初、春秋战国或汉晋等朝代的黄老学派之手,判断各异。 [7]

据说《阴符经》是唐朝著名道士李筌在河南省境内的登封嵩山少室虎口岩石壁中发现的,此后才传抄流行于世。根据李筌对本经典的解释著作《黄帝阴符经疏》,可以把它的内容概括为两个部分:首先讲述观察自然界及其发展变化的客观规律,所以,天性运行为自然规律,人心则顺应自然规律;其次阐明了天、地、人生杀的变化情况,人的生杀之气的放和收,应与自然同步,才能把握好事物成功的机遇。然后,阐明人后天禀性巧拙的生成和耳目口鼻的正确运用,主要效法自然五行相生原则,修炼自身。


\mainmatter

\chapter{黄帝阴符经}

\begin{yuanwen}
观\footnote{观察,体悟。}天之道,执\footnote{按照,执行。}天之行,尽矣。
\end{yuanwen}

观察天道运行之规律,并按照天之运行修炼自身,一切修道的内容都可以包括在其中了。

观察天体的运行规律,掌握其规律并按其规律去做,则天地阴阳动静之道就全包括在内了。

修道所追求的就是"天人合一"的境界,天人合一的程度越深,修炼的层次也就越高。故此明白了天人合一,也就等于明白了修道。这句话点明了全经主题,是全篇总纲。

\begin{yuanwen}
天有五贼\footnote{贼害,戕害。},见\footnote{识别,发现。}之者昌\footnote{精进,成功。}。
\end{yuanwen}

天上有金木水火土五大行星,喻指五行。五行生克制化,莫不戕害我身,使我堕入其中,尝受生老病死之苦,不能做自己命运的主人。而在修练之人,则可识其贼性,探得造化之根源,使五行颠倒,造化逆行,自能反夺五行之造化,使"贼"化为"昌",反而促使我之道成。

天体有五行之气,五行相生顺时而行,则万物昌盛。

\begin{yuanwen}
五贼在心\footnote{指修练人之心。},施行于天\footnote{指身外之宇宙。}。
\end{yuanwen}

反夺五行造化,在于一心之运用。此乃空空洞洞,不执不失之道心,非世俗顽恶之人心。其所反夺之源,在于体外之宇宙。由于色身有限,宇宙无限,要从宇宙之中施行反夺,才能获取无穷之造化。修成恒古不灭之先天元神,长生久视,"天地有坏,这个不坏"。大道之奥妙,早已揭示无遗矣。

了解和掌握了五行生克制化之理,合天而行。

\begin{yuanwen}
宇宙在乎手\footnote{手掌,手通心,亦指自心。},万物生乎身。
\end{yuanwen}

人能认清五贼,追根溯源,还归本来,求得宇宙总持之门,自然成为造化主人。此时无穷宇宙,如同在我掌中;万物变化,亦好似生于自身。又手通心,亦指宇宙变化,自心了然可知。这等气魄。若非修道之士其谁人能之。

则宇宙就掌握在手中,万物就生乎身上。

\begin{yuanwen}
天性,人也。人心,机\footnote{时机(此经重在"机"字,包括所有的道功、法、道妙。吾人修道,采药得丹,全在火候,全在掌握时机)。}也。立\footnote{遵循。}天之道,以定人也。
\end{yuanwen}

吾人未生之前,不过元神混沌之体,谓之天性;既生我后,化为后天气质之性,谓之人心。天性既可化为人心,吾人自可明通此机,遵循天道,去掉人心,返归天性。老子谓之"归根复命",大道之根源在此。

天性即是人性,人性即是天机,天人合一。天道定了,人道也就定了。

\begin{yuanwen}
天发杀机,移星易\footnote{变易,变化。}宿。地发杀机,龙蛇\footnote{指水患地震。}起陆。人发杀机,天地反覆。天人合发,万变(化)定基。
\end{yuanwen}

天发杀机,日月相蚀,陨星坠落。地发杀机,洪水地震,起于四野。人发杀机,天翻地覆,灾异横起。要在人能合乎天道,天人齐发,则万种变化,可以定其基矣,以上虽言杀机,但生杀互根,杀机即是生机。人能发杀机于天地,即是反夺生机于自身。丹道谓之"大死再活"置之死地而后生,是也。

以上虽言杀机,但是生杀互根,修练人须由此悟去,杀机即是生机。人能发杀机于天地,即是反夺生机于自身。丹经谓之"大死再活","置之死地而后生"。

五行逆行则天发杀机,星体移位,黑白颠倒,灾难将至;地发杀机,则山崩水溢,龙蛇不安其位;人发杀机,则翻天覆地,山河动摇。若是人合天机同发,则万物将在一个新的基础上定下来。

\begin{yuanwen}
性\footnote{指人心。}有巧拙,可以伏藏。九窍之邪\footnote{邪妄。},在乎三要\footnote{指耳、目、口三宝。},可以动静。
\end{yuanwen}

修练之人,要在杀机中反夺生机,必须人天合发,即人性合乎天性。但人性有巧有拙,务使伏巧为拙,使外拙而内巧,拙中藏巧,才合乎天性。但是人心有巧有拙,务使巧伏为拙,使外拙而内巧,拙中藏巧,方才合乎天性。

伏藏之道,在于九窍(即耳、目、口、鼻、脐、外肾、谷道。)九窍皆邪妄出入之门户,而关键更在于耳、目、口三者。精通于耳,气通于口,神通于目,动则外漏,静则内藏,使动化为静,则三要皆成为三宝矣。

人有聪明智慧的一面,也有愚蠢笨拙的一面,但都不要显示出来,要善于隐藏。人有九窍,能招惹邪恶是非,其中唯有耳、目、口这三者是最重要的。耳能听,目能视,口能说,它们可以动也可以静。

\begin{yuanwen}
火\footnote{指人之心火。}生于木\footnote{木能生火,喻为元神。},祸发必克。奸生于国,时动必溃。知之修炼,谓之圣人。
\end{yuanwen}

钻木取火,古人经验。但火性太炽,则木反为火伤。比喻人之心火过旺,必伤无神。推之治国,其理亦同,国家出了奸臣,祸国殃民,动荡之时,必然崩溃。犹人炼意不净,滋生妄念,定有伤丹之度。可见祸福生杀,太过不及,差之毫厘,谬之千里。识得其机,修之炼之,才是圣人。

火生于木,火燃烧起来木就变成灰烬了。奸贼生于国内,若奸贼得逞则国家就要灭亡。唯有圣人能修身炼性,防微杜渐。

\begin{yuanwen}
天生天杀,道之理也。
\end{yuanwen}

天生天杀,阴阳消长,乃顺行之自然。

天生万物(春生夏长),天亦杀万物(秋敛冬藏),这是自然规律。



\begin{yuanwen}
天地,万物之盗\footnote{逆取,反夺。}。万物,人之盗。人,万物之盗(也)。三盗既宜\footnote{平衡,协调。},三才既安。故曰:“食其时,百骸埋。动其机,万化安。”人知其神而神,不知其不神(之)所以神也。日月有数,大小有定。圣功生焉,神明出焉。其盗机也,天下莫能见,莫能知。君子得之固穷(躬),小人得之轻命。
\end{yuanwen}

但天杀之机,即是反夺生气之机也,又为逆行修道之枢要。天地从万物中反夺,万物从人中反夺,人从万物中反夺。三者互相反夺,合平衡,才合乎生杀之道,成为自然。

天地生万物亦杀万物,万物生人亦杀人,人生万物亦杀万物。三者相互为盗又寓相生之理,使天、地、人各得其位,各司其职,万物育生。





译文:“看上天运行的轨迹,做上天赋予的使命,(万事万物的奥妙)就尽了。天有金木水火土(五行相克),看见的人会昌盛。五行在心中体会,施行合天的行动。这样,宇宙虽大,仍在一掌之中(天地都来一掌中),千变万化,不出一身之外(人身为一小天地)。”

译文:“上天之性是人的根本,人心却是诈伪。所以要以上天之道来定人心。”

译文:“上天若出现五行相克,就会使星宿移位;大地若出现五行相克,就会使龙蛇飞腾;人体内若出现五行相克,就能使小天地颠倒。倘若人能顺应自然而同时发生五行相克,就能使各种变化稳定下来。”

译文:“人性虽有巧有拙,却可以隐藏起来。九窍是否沾惹外邪,关键在于耳、目、口三窍之动静。三窍动则犹如木头着火,灾祸发生必被攻克;如x有奸邪,时间一到必致溃亡。懂得如此修炼,称为圣人。”



译文:“万物顺应天地之规律而自然生长;人利用万物而富足;万物依靠人而昌盛。只要天地、万物与人之间各得其宜,那么它们就会安定下来。所以说:休养要遵循时令,身体才会得到调理;行动要把握时机,万物才会变得安定。人们只懂得“盗”的神妙莫测而以为神(世人只知偷盗不被查觉,谓之‘神’),却不知“盗”不神妙莫测才是最神妙莫测的(却不知顺天地、万物之规律而公开盗之,方为‘神’)。要知道,太阳与月亮各有规律,大与小都有定规,只有懂得这些道理,才会有大功产生,才会有神明护佑。这些“盗”的机巧是天下之人所不能见、所不能知的。有悟性的人得到它,就会躬行(能顺应自然);无悟性的人得到它,却会丧命(因违法偷盗)。”


\begin{yuanwen}
瞽者善听,聋者善视。绝利一源,用师十倍。三返昼夜,用师万倍。心生于物,死于物,机在目。天之无恩,而大恩生。迅雷烈风,莫不蠢然。
\end{yuanwen}



故曰食其时,百骸理。动其机,万化安。人知其神之神,不知不神所以神也。


【字解】食,掌握,采取。动,发动。


【释义】欲求修炼,贵在能知生杀予夺之时机。按时采取,从天地万物中反夺生机,陶铸自身筋骨,才能成为乾健之躯。乘机发动,借生杀变化之机,反夺造化,安定自身。丹功每次提高阶段,都在掌握时机。


平常人只知后天思虑之神为神,不知先天不神之神,才是真神。大要修道,先使后天识神归于先天不神,空空洞洞,虚灵不昧,才能时至神知,机动觉随,反夺造化,调理百骸,得成修炼之动。


【白话文】


所以古语说:“饮食得其时,则人体得到调养生息;行动符合天机,则万物安泰。”人只知道万物从阴阳而生,将这看不见、摸不着的东西称之为“神”;却不知道这个“神”是从至道虚无的“不神”而来的。


图片​


日月有数,小大有定。圣功生焉,神明出焉。


【字解】数,定数。定,周期。圣功,即修练之功。


【释义】太阳东升西降,月亮晦朔弦望,皆有定数。小往则大来,大往则小来,阳大阴小,与日月之出没相同。我能知往推来,食其时而动其机,采日精月华,夺天地正气,自可完成修真成圣之动,神明由此而出。


【白话文】


日月之行必有常数,日月之大小也有定数。圣人掌握其规律推而测之,就能昭示神明之道。


图片​


其盗机也,天下莫能见,莫能知。君子得之固躬,小人得之轻命。


【字解】固,固然。躬,躬敬,谨慎。轻,丧失,夭折。命,生命。


【释义】盗机即反夺之机也,反夺造化之功,皆无形象可言,若有形象,便落后天,故天下无见之知之者。先有见知,便失真机。采炼之时,若为"寂然不动,感而遂通"之先天元肖,无形无象,不可得见,则可成丹。此时若动情识,迅即化为后天浊质,可以见如,必有走漏之危,即使追回采炼,亦不能成丹。此反夺之机,君子得之固然谨慎,倍受奉行,可以长生久视。小人得之轻视造化,修功差驰,反促其寿也。


【白话文】


三盗的实现都是在寂静中完成的,形迹未露,人们看不见、摸不着,求知者对它认识不一。君子得之能顺时而行,用以健壮身体、修身养性;小人得之则违时而行,恃才妄为,反而害身。


图片​


瞽者善听,聋者善视。绝利一源,用师十倍:三反昼夜,用师万倍。


【字解】师,兵事。修道与用兵理同。


【释义】双目失明的人,视不外漏,专一于耳,所以听觉灵敏。两耳失聪的人,听不外漏,专一于目,因而视觉灵敏。专心用于一处,便可得到用兵十倍的效力。反复昼夜地不断用心,则可得到用兵万倍的效力。丹法与用兵相同,二者一理,运用之妙,都在专一。


【白话文】


瞎子有目不能视,却擅长于听;聋子有耳不能听,却擅长于视。不能听或不能视就杜绝了外界的种种诱惑,胜于众人十倍;如若再能昼夜反省自己,则就胜于众人万倍了。

图片​


心生于物,死于物,机在目。


【字解】心,指人心。目,指眼睛。


【释义】人生之初,心本虚空,渐为外物所扰,因而产生各种念想,损人心性,损尽则死。修道下手,还虚第一,盖"魔由心生,境由心造",心若不虚,反而自惹魔障,坏我功修。故须收心离境,聚性止念。其机在目,神生于心,发于二目,乃丹动之枢机。内视、采药、烹炼、养胎及至出神等等,均以目力机。


【白话文】


心(人的思想)来源于客观事物,并随着客观事物的变化而变化、发展而发展、灭亡而灭亡,其机关就在于目(目是“三要”之首)。


图片​


天之无恩而大恩生,迅雷烈风莫不蠢然。


【字解】天,指天空,于天道。


【释义】天本空空洞洞,无识无知,毫无施恩之意,而其行四时,育万物,大恩遂生焉。迅雷烈风受其驱使,而蠢蠢然不能自主。此乃大道隐含之力量,不可思议,修真悟道之士,当由此参证之。


【白话文】


宇宙天体按着自己的规律运行着,无意施恩于万物,而万物却得其恩泽雨露而生长。阴阳相交,产生了雷电风雨,使万物自由自在地发育生长。


图片​


至乐性余,至静性廉。天之至私,用之至公。


【字解】至,到,真正。余,余闲。廉,清廉。


【释义】至乐的人,心胸坦荡,性有余闲。至静的人,心性收敛,廉而不失。修炼的人悟到虚静之时,心忽开朗,舒适畅快,妙不可言,就是达到至乐至静的境界了。


天道驱风使雷,运行四时,看似至私,而作用于万物生化,却无偏无倚,一视同仁,实为至公。犹天性降之于人,虽为个人所私,实际贤愚皆同,人人均有。天性与太虚等量,大公无私,若至私而实至公也。


【白话文】


人的性格有“至乐”“至静”之分,至乐者性格开朗,宽裕优容;至静者思维缜密,廉洁无染。天也有“至公”和“至私”两个方面,它将天地万物都包容于一身,似其自私;然而万物又无偿地使用它,则又大公无私。


图片​


禽之制在杰。生者死之根,死者生之根。恩生于害,害生于恩。


【字解】禽,通"擒",制服之意。


【释义】制服的诀窍在于肖,肖聚则生,肖散则死。生与死互为本根,生于何处,死于何处,人由男女而生,亦因男女而死。恩害相生,亦同于生死。由人心返还天性,为死处求生,是谓逆则成仙,即恩生于害;由天性降落人心,为生老病死,是谓顺则生人,即害生于恩。


【白话文】


统摄万物者,制造万物者,都在于一气。万物有生必有死,则生乃死之根源;有死必有生,则死又是生的根源。人间社会也是如此,无害则无恩,因救害而有了恩;若知恩不报,则害又生于恩。


图片​


愚人以天地文理圣,我以时物文理哲。人以愚虞圣,我以不愚虞圣。人以奇其圣,我以不奇其圣。


【字解】愚人,常人,一般的人。我,指修道之人。


【释义】愚人以天文地理为神圣,我以随机应变为原则。人以愚弄欺骗为神圣,我以不言而信为神圣。人以惊世骇俗为神圣,我以和光同尘为神圣。这些都是修德之要,无德便不能培道。


道家认为,道在我身上就是德;没有德也就失去了道。有人做功出魔,或功夫停滞,就因为不注重修德之缘故。


【白话文】


愚蠢的人认为,天地万物都是无形的,是不可知的神圣之物;而我却认为,天地万象都是有形的,是可知的。有的人用愚蠢的办法揣测、预料天地的表面现象,以为自得,自称为圣人;我却认为,聪明智慧,能体察万物之理的人为圣人。人们以为能够推度出神奇事物的出现是圣人;我却认为,能体察天地、成就万物“不奇期”者是圣人。
图片​


沉水入火,自取灭亡。


【字解】水,指肾水,在易象为坎卦。火,指心火,在易象为离卦。


【释义】以坎水填入离火之中,使后天坎离复为先天乾坤,则人心灭亡,而天性复现,到此筑基完成。


【白话文】


故古语说:“(追求“愚虞”和“奇期”的)如同于投于水火之中,自取灭亡。”


自然之道静,故天地万物生。天地之道浸,故阴阳胜,阴阳相推而变化顺矣。


【字解】浸,浸润,充满。胜,主宰。


【释义】自然之道,主静立极,空空洞洞,无中生有,《老子》曰:"清静为天下正。"天地万物遂得以生化。天地之道充满其中,天为阳,地为阴,因此阴阳之道主宰于万事万物之中。静极生动,阴极生阳,阳极消长,互资互根。如此相推,则天地万物生生化化,顺其自然,不失其序也。


【白话文】


天地日月依其自然规律静静地运行着,故天地万物得以生存、生长。天地之道是浸润渐进的,阴极生阳,阳极生阴,阴阳相互转化,而四季成序,万物生长,都按照一定的规律自然顺畅的运动。

图片​


圣人知自然之道不可违,因而制之至静之道,律历所不能契。


【字解】违,违抗,改移。制,制订,采用。契,契合,规定。


【释义】圣人明白自然之道不可随意违抗,因而采用至静之法。只有静才能体悟天道,才能识别五贼,才能天人合发,才能反夺造化。一切修为,都是从静中自然生出。能静片刻,可以攒簇一年之气候,这是律历所不能规定的。


【白话文】


所以,圣人知道天地自然规律是不可违背的,人应该与天地合,顺其道而行。至静之道是无形的,而天文历法是有形的,却无法揭示和包括无形的“至静之道”。


图片​


爰有奇器,是生万象。八卦甲子,神机鬼藏。阴阳相胜之术,昭昭乎近乎象矣。


【字解】奇器,奇异之器。万象,万象变化。象物象,指事物的本来面目。



【释义】有奇异之器,才能产生万象。八卦甲子之中,藏有鬼神莫测之机。阴阳相胜的法则,昭昭然可以揭示事物的本来面目了。


《参同契》、《悟真篇》等类丹经,多以卦象干支描述自身修练中的阴阳变化,亦宗此义也。


【白话文】


于是,圣人就发明了一种神奇的东西,用以昭示天地万物之象,这就是阴阳八卦和六十甲子,神之申机,鬼之屈藏,无不包括在内。阴阳相生相克之术,与天地之道暗合(阴符),故能昭示天地之间的万象。


译文:“眼盲者善长听,耳聋者善长看。(因此,如果能)断绝或助利其一(或眼或耳),就会增强十倍之能力;如果能每天断绝耳、目、口(勿听、勿视、勿言),就会增强万倍之能力。心因万物而躁生,因万物而寂灭,关键在于眼。(要知道,)上天不施恩德(无声无言),(因)而能产生大恩德;(而)响雷暴风(指外物)只会使万物发生骚动。”

译文:“至乐在于知足,至静在于无私。上天因无恩而至私,故能大恩而至公(施惠于万物)。统摄的法式在于调和其气。”

译文:“生为死之根源,死为生之根源。利因害而生,害亦因利而生。”

译文:“愚昧之人常以懂得天地之准则为智慧,我却以遵循时令、洞悉外物为聪明;俗人以欺诈为智慧,我却不以欺诈为聪明;俗人以奇异为智慧,我却不以奇异为聪明。所以说:以欺诈与奇异行事,如水入火,自取灭亡。”

译文:“自然之道为静,所以能生天地万物。天地的运行遵循自然,所以能使阴阳相胜。阴阳相胜相生,则变化和谐。”


\begin{yuanwen}
至乐性愚,至静性廉。天之至私,用之至公。禽之制在气。
\end{yuanwen}


\begin{yuanwen}
生者,死之根。死者,生之根。恩生于害,害生于恩。
\end{yuanwen}


\begin{yuanwen}
愚人以天地文理圣,我以时物文理哲。人以愚虞圣,我以不愚虞圣。人以奇期圣,我以不奇期圣。(故曰:)沉水入火,自取灭亡。
\end{yuanwen}


\begin{yuanwen}
自然之道静,故天地万物生。天地之道浸,故阴阳胜。阴阳推而变化顺矣。
\end{yuanwen}


\begin{yuanwen}
(是故)圣人知自然之道不可违,因而制之。至静之道,律历所不能契。爰有奇器,是生万象,八卦甲子,神机鬼藏。阴阳相胜之术,昭昭乎尽(进)乎象矣。
\end{yuanwen}




译文:“所以,圣人懂得自然之道不可违背,因而制订了各种法则。然而,至静之道是乐律和历法所不能契合的。于是就有了奇妙的《易》,它产生了各种象征,是以八种卦象为本,并贯以六十甲子,来演化种种玄机的。这样一来,阴阳循环相生也就能很清楚地蕴涵于各种象征之中了。”(最后这几段是宋明理学家或内丹家所加,意在尊孔易(周易)贬老道(归藏)) [6]

\backmatter

\end{document}