% 吕氏春秋
% 吕氏春秋.tex

\documentclass[12pt,UTF8]{ctexbook}

% 设置纸张信息。
\usepackage[a4paper,twoside]{geometry}
\geometry{
	left=25mm,
	right=25mm,
	bottom=25.4mm,
	bindingoffset=10mm
}

% 设置字体,并解决显示难检字问题。
\xeCJKsetup{AutoFallBack=true}
\setCJKmainfont{SimSun}[BoldFont=SimHei, ItalicFont=KaiTi, FallBack=SimSun-ExtB]

% 目录 chapter 级别加点(.)。
\usepackage{titletoc}
\titlecontents{chapter}[0pt]{\vspace{3mm}\bf\addvspace{2pt}\filright}{\contentspush{\thecontentslabel\hspace{0.8em}}}{}{\titlerule*[8pt]{.}\contentspage}

% 设置 part 和 chapter 标题格式。
\ctexset{
	part/name= {第,卷},
	part/number={\chinese{part}},
	chapter/name={第,篇},
	chapter/number={\chinese{chapter}}
}

% 设置古文原文格式。
\newenvironment{yuanwen}{\bfseries\zihao{4}}

% 设置署名格式。
\newenvironment{shuming}{\hfill\bfseries\zihao{4}}

% 注脚每页重新编号,避免编号过大。
\usepackage[perpage]{footmisc}

\title{\heiti\zihao{0} 吕氏春秋}
\author{}
\date{}

\begin{document}

\maketitle
\tableofcontents

\frontmatter
\chapter{前言、序言}

[汉]高诱 注
[清]毕沅 校

《吕氏春秋》又名《吕览》,是先秦的一部重要典籍,旧题秦吕不韦撰,其实是秦相吕不韦主编而由其门下宾客集体撰著的学术总集。

《史记·吕不韦传》记载:“吕不韦乃使其客人人著所闻,集论以为八览、六论、十二纪,二十余万言。以为备天地万物古今之事,号曰《吕氏春秋》。布咸阳市门,悬千金其上,延诸侯游士宾客有能增损一字者予千金。”吕不韦广罗学者,编撰巨著,公开买错,求改文章,这实在是中国编书史上的一大盛事,于此亦可见这位权倾一时的秦相的政治魄力及其称霸学术的气势。

《吕氏春秋》约成书于秦始皇八年。《序意》篇说:“维秦八年,岁在涒滩,秋甲子朔,良人请问十二纪。”秦始皇八年即前239年,距前221年秦统一六国仅十九年。秦的统一天下,在中国历史上是空前的大业,其影响并及于今天。《吕氏春秋》的撰成,显示出统一前政治上的要求,思想上的动员,理论上的准备,是适应历史潮流的产物。

《吕氏春秋》以十二纪为首,“以春为喜气而言生,夏为乐气而言养,秋为怒气而言杀,冬为哀气而言死,所谓春生夏长秋收冬藏也”。每一纪以月令开头,“以第一篇言天地之道,而以四篇言人事”,每纪五篇,最后一篇是《序意》,共六十一篇。其次是八览。第一篇《有始览》说:“天斟万物,圣人览焉,以观其类。”明示八览的宗旨,并从开天辟地天地“有始”出发,讲“孝行”、“慎大”、“先识”、“审分”、“审应”、“离俗”、“恃君”,本之于天,探究国家祸福的由来,博论治国之术。每览八篇,《有始览》缺一篇,共计六十三篇。最后是六论,共三十六篇,“盖博言君臣氓庶之所当务者也”(吕思勉《先秦学术概论》)。殿后的《士容论》,主要论述士的气质、修养、品格和知识结构,其后四篇《上农》、《任地》、《辩土》、《审时》则反映了先秦农家学说,盖以农业为国家的基础,士不可不知也,所以其作用不仅仅是保留了极有价值的农家学说资料。《吕氏春秋》以当时的现实政治为纲,纲举目张,全书编次条贯统系,是任何一部先秦著作所不能企及的。

战国后期,先秦各派学术思想空前发达,已到了一个总结阶段,《吕氏春秋》对诸子百家学说作了一次大规模的综合性概括,不别门户,集其大成,以有利于国家的统治。对于此书属于哪一家,历来有不同的看法。班固的《汉书·艺文志》将其归入杂家;《四库全书总目提要》谓“大抵以儒为主,而参以道家、墨家”;清人卢文弨说“大约宗墨氏之学,而缘饰以儒术”;陈奇猷先生在其《吕氏春秋校释》中则认为“吕不韦的主导思想是阴阳家之学”;《辞海》(1988年版)称其“内容以儒道思想为主”。而早在汉末,高诱在《吕氏春秋序》中已说此书“以道法为标的,以无为为纲纪”,肯定了它以老庄哲学思想为主导。

《吕氏春秋》根本旨归在于圣人之治,所以虽倡导法天地、无为而治,仍然集纳儒家的别贵贱、重教化,墨家的首功利、主尚贤,法家的尊主卑臣、一断于法,乃至阴阳家论阴阳五行等一切有益于当时现实政治的学说。此外,《吕氏春秋》又崇尚自然,重视生命,讲全性之道。有的学者将它归入秦汉道家类,自是比较妥当的。

《吕氏春秋》富有中国重实践、讲辩证的思维特色,涵盖哲学、政治、经济、历史、道德、军事等领域,引《诗经》、《尚书》、《周易》、《礼记》、《春秋》及诸子等典籍,为秦统一天下提供了理论基础,为后世展示了一整套综合的文化学说,并保存了大量可贵的先秦学术资料。一百多年后,汉刘安主编的《淮南子》又进一步全面发扬了《吕氏春秋》的基本思想。

吕不韦相秦十三年,为秦统一天下建有不小的功绩,从其主持编成的《吕氏春秋》,“你可以发觉它的每一篇每一节差不多都是和秦国的政治传统相反对,尤其是和秦始皇后来的政见和作风作正面的冲突”(郭沫若《十批判书·吕不韦与秦王政的批判》)。因而,他在生前不幸被迫害而自杀,死后又没有得到公允的评价,是很自然的。而今天,作为对先秦哲学思想进行系统探讨的《吕氏春秋》在先秦哲学史中放射的异彩,正愈来愈吸引着人们的重视与研究。

《吕氏春秋》旧有汉末高诱训解,经二千多年辗转流传,书中有不少讹脱、错简以致晦涩难解之处,有些篇文字前后重复,或内容缺乏联系,甚至相互牴牾,但已无从考订。现存最早版本是元至正年间嘉兴路儒学刊本。明代有弘治十一年李瀚刻等十余种本子。清乾隆间,著名学者毕沅据元人大字本等八种悉心校勘,厘成《吕氏春秋》校正本。光绪元年,浙江书局在印行毕氏本时又作过校正。本书采用浙江书局本作底本,除明显错误外,一般不作校改,顺应原校注内容标点,以为读者提供一种比较简练可读的本子。

\chapter{吕氏春秋新校正序}


毕沅

《汉书·艺文志》杂家,《吕氏春秋》二十六卷,秦相吕不韦辑智略士作。原夫六经以后,九流竞兴,虽醇醨有间,原其意旨,要皆有为而作。降如虞卿诸儒,或因穷愁,托于造述,亦皆有不获已之故焉。其著一书,专觊世名,又不成于一人,不能名一家者,实始于不韦,而《淮南》内外篇次之。然淮南王后不韦几二百年,其采用诸书,能详所自出者,十尚四五。即如今《道藏》中《文子》十二篇,淮南王书前后采之殆尽,间有增省一二字、移易一二语以成文者,类皆当时宾客所为,而淮南王又不暇深考与?

不韦书在秦火以前,故其采缀,原书类亡,不能悉寻其所本。今观其《至味》一篇,皆述伊尹之言,而汉儒如许慎、应劭等间引其文,一则直称伊尹曰,一则又称伊尹书。今考《艺文志》道家《伊尹》五十一篇,不韦所本,当在是矣。又《上农》、《任地》、《辨土》等篇,述后稷之言,与《亢仓子》所载略同,则亦周、秦以前农家者流相传为后稷之说无疑也。他如采《老子》、《文子》之说,亦不一而足。是以其书沈博绝丽,汇儒墨之旨,合名法之源,古今帝王天地名物之故,后人所以探索而靡尽与!

《隋书·经籍志》杂部,《吕氏春秋》二十六卷,高诱注。诱序自言尝为《孟子章句》及《孝经解》等,今已不见。世所传诱注《国策》,亦非真本,唯此书及淮南王书注最为可信。诱注二书,亦间有不同,《有始览》篇“大汾冥阨”,解云“大汾,处未闻。冥阨、荆阮、方城皆在楚”,而淮南王书注则云“大汾在晋”,“冥阨”《淮南》作“渑阨”,注云“今弘农渑池是也”。《先识览》篇“男女切倚”,解云“切,磨;倚,近也”,淮南王书“倚”作“踦”,注又云“踦,足也”。《知分》篇解云“鱼满二千斤为蛟”,而淮南王书又作“二千五百斤”。至于音训,亦时时不同。此盖随文生义,或又各依先师旧训为解,故错出而不相害与?

暇日取元人大字本以下,悉心校勘,同志如抱经前辈等又各有所订正,遂据以付梓。鸠工于戊申之夏,逾年而告成。若淮南王书,则及门庄知县炘已取《道藏》足本刊于西安,故不更及云。

乾隆五十四年岁在己酉孟夏月吉序。

新校吕氏春秋所据旧本


元人大字本脱误与近时本无异。

李瀚本明弘治年刻,篇题尚是古式,今皆仍之。

许宗鲁本从宋贺铸旧校本出,字多古体。嘉靖七年刻。

宋启明本不刻年月。有王世贞序。

刘如宠本神庙丙申刻。

汪一鸾本神庙乙巳刻。

朱梦龙本每用他书之文以改本书,为最劣。

陈仁锡奇赏汇编本。





书内审正参订姓氏


余姚卢文弨绍弓

嘉善谢墉昆城

嘉定钱大昕晓徵

仁和孙志祖诒穀

金坛段玉裁若膺

江阴赵曦明敬夫

嘉定钱塘学源

阳湖孙星衍渊如

阳湖洪亮吉穉存

仁和梁玉绳燿北

钱塘梁履绳处素

武进臧镛堂在东

\chapter{吕氏春秋序}

高诱

吕不韦者,濮阳人也,为阳翟之富贾,家累千金。

秦昭襄王者,孝公之曾孙,惠文王之孙,武烈王之子也。太子死,以庶子安国君柱为太子。柱有子二十余人,所幸妃号曰华阳夫人,无子。安国君庶子名楚,其母曰夏姬,不甚得幸。令楚质于赵,而不能顾质,数东攻赵,赵不礼楚。时不韦贾于邯郸,见之,曰:“此奇货也,不可失。”乃见楚曰:“吾能大子之门。”楚曰:“何不大君之门,乃大吾之门邪?”不韦曰:“子不知也,吾门待子门大而大之。”楚默幸之。不韦曰:“昭襄王老矣,而安国君为太子。窃闻华阳夫人无子,能立適嗣者独华阳夫人耳。请以千金为子西行,事安国君,令立子为適嗣。”不韦乃以宝玩珍物献华阳夫人,因言:“楚之贤,以夫人为天母,日夜涕泣,思夫人与太子。”夫人大喜,言于安国君,于是立楚为適嗣,华阳夫人以为己子,使不韦傅之。不韦取邯郸姬,已有身,楚见说之,遂献其姬。至楚所,生男,名之曰正,楚立之为夫人。

暨昭襄王薨,太子安国君立,华阳夫人为后,楚为太子。安国君立一年薨,谥为孝文王。太子楚立,是为庄襄王,以不韦为丞相,封为文信侯,食河南洛阳十万户。庄襄王立三年而薨,太子正立,是为秦始皇帝,尊不韦为相国,号称仲父。

不韦乃集儒书,使著其所闻,为十二纪、八览、六论,训解各十余万言, (1) 备天地万物古今之事,名为《吕氏春秋》,暴之咸阳市门,悬千金其上,有能增损一字者与千金。 (2) 时人无能增损者。诱以为时人非不能也,盖惮相国,畏其势耳。然此书所尚,以道德为标的,以无为为纲纪,以忠义为品式,以公方为检格,与孟轲、孙卿、淮南、扬雄相表里也,是以著在《录略》。诱正《孟子》章句,作《淮南》、《孝经》解毕讫,家有此书,寻绎案省,大出诸子之右。既有脱误,小儒又以私意改定,犹虑传义失其本真,少能详之,故复依先师旧训,辄乃为之解焉,以述古儒之旨,凡十七万三千五十四言。若有纰缪不经,后之君子,断 (3) 而裁之,比其义焉。

(1) 【校】梁伯子云:“《史记·十二诸侯年表序》及《吕不韦传》并云著八览、六论、十二纪,以纪居末,故世称《吕览》,举其居首者言之。今《吕氏春秋》以十二纪为首,似非本书序次。”愚案:以十二纪居首,此“春秋”之所由名也。《汉书·艺文志》杂家载《吕氏春秋》二十六篇,不称《吕览》。郑康成注《礼记·礼运》“故圣人作则必以天地为本”一节云:“天地以至于五行,其制作所取象也。礼义人情,其政治也。四灵者,其征报也。此则《春秋》始于元、终于麟包之矣。《吕氏》说月令而谓之‘春秋’,事类相近焉。”《正义》疏之云:“吕不韦说十二月之令谓为《吕氏春秋》,事之伦类,与孔子所修《春秋》相附近焉。《月令》亦载天地阴阳四时日月星辰五行礼义之属,故云相近也。”据此,则自汉以来皆以《吕氏春秋》为正名。至于行文之便,则容有不拘耳。

(2) 【校】梁伯子云:“《太平御览》八百九卷引《史记》同此序,而百九十一卷引《史》云,吕不韦撰《春秋》成,榜于秦市曰‘有人能改一字者赐金三十斤’,岂别有所据乎?”

(3) 一作斫。

\mainmatter

% 增加空行
~\\

% 增加字间间隔,适用于三字经、诗文等。
 \qquad  

\part{孟春纪}



孟春


一曰:

孟春之月,日在营室, (1) 昏参中,旦尾中。 (2) 其日甲乙,其帝太皞, (3) 其神句芒。 (4) 其虫鳞,其音角, (5) 律中太蔟,其数八。 (6) 其味酸,其臭膻, (7) 其祀户,祭先脾。 (8) 东风解冻,蛰虫始振, (9) 鱼上冰,獭祭鱼。 (10) 候雁北。 (11) 天子居青阳左个, (12) 乘鸾辂,驾苍龙, (13) 载青旂,衣青衣,服青玉, (14) 食麦与羊,其器疏以达。 (15)

(1) 孟,长。春,时。夏之正月也。营室,北方宿,卫之分野。是月,日躔此宿。

(2) 参,西方宿,晋之分野。尾,东方宿,燕之分野。是月昏旦时,皆中于南方。

(3) 甲乙,木日也。太皞,伏羲氏以木德王天下之号,死,祀于东方,为木德之帝。

(4) 句芒,少暤氏之裔子曰重,佐木德之帝,死为木官之神。

(5) 东方少阳,物去太阴,甲散为鳞。鳞,鱼属也,龙为之长。角,木也,位在东方。

(6) 太蔟,阳律也。竹管音与太蔟声和,太阴气衰,少阳气发,万物动生,蔟地而出,故曰“律中太蔟”。五行数五,木第三,故数八。

(7) 春,东方木王,木味酸。酸者,钻也。万物应阳,钻地而出。膻,木香膻也。

(8) 蛰伏之类始动生,出由户,故祀户也。脾属土。陈俎豆,脾在前,故曰“祭先脾”。春,木胜土,先食所胜也。一说,脾属木,自用其藏也。

(9) 蛰,读如《诗·文王之什》。东方木,木,火母也。火气温,故东风解冻,冰泮释地。蛰伏之虫乘阳,始振动苏生也。

(10) 鱼,鲤鲋之属也。应阳而动,上负冰。獭 ,水禽也,取鲤鱼置水边,四面陈之,世谓之祭鱼为时候者。

(11) 候时之雁,从彭蠡来,北过至北极之沙漠也。

【校】案:《礼记·月令》作“鸿雁来”,郑注云:“今《月令》‘鸿’皆为‘候’。”《正义》云:“《月令》出有先后,入《礼记》者为古,不入《礼记》者为今。”则《吕氏春秋》是也。卢案:“仲秋,雁自北徼外而入中国,可以言‘来’,若自南往北,非由南徼外也,似不可以言‘来’。《吕氏》作‘候雁北’,当矣。”

(12) 青阳者,明堂也。中方外圜,通达四出,各有左右房谓之个。个,犹隔也。东出谓之青阳,南出谓之明堂,西出谓之总章,北出谓之玄堂。是月,天子朝日告朔,行令于左个之房,东向堂,北头室也。

【校】案:明堂之制,中外皆方,不得如注所云。“个犹隔也”,旧本缺一“个”字,今补。

(13) 辂,车也。鸾鸟在衡,和在轼,鸣 [1] 相应和。后世不能复致,铸铜为之,饰以金,谓之鸾辂也。《周礼》“马八尺以上为龙,七尺以上为 ,六尺以上为马”也。

【校】案:“鸾”字与《月令》同,唯刘本作“銮”,注“鸾鸟在衡”作“銮在镳”,案《诗·蓼萧》毛传“在镳曰鸾”,郑于《驷 笺》云:“置鸾于镳,异于常车。”若据郑说,则刘本非是。但《说文》“銮”字从“金”,云“人君乘车,四马,镳八銮。铃,象鸾鸟声”,高氏之解,或异于郑,未可知也,亦不得竟以刘本为非。

(14) 旂,旗名,交龙为旂。载者若今之鸡翘车是也。服,佩也。所衣服佩玉皆青者,顺木色也。

【校】案:蔡邕《独断》云:“鸾旗车,编羽毛列系橦旁,俗人名之鸡翘车,非也。”《续汉·舆服志》同。刘昭引胡广曰“以铜作鸾鸟车衡上”,则与高诱注合。

(15) 麦属金,羊属土。是月也,金土以老,食所胜也。宗庙所用之器,皆疏镂通达,以象阳气之射出。

是月也,以立春。 (1) 先立春三日,太史谒之天子曰:“某日立春,盛德在木。” (2) 天子乃斋。 (3) 立春之日,天子亲率三公、九卿、诸侯、大夫以迎春于东郊。 (4) 还,乃赏卿、诸侯、大夫于朝。 (5) 命相布德和令,行庆施惠,下及兆民。 (6) 庆赐遂行,无有不当。 (7) 乃命太史,守典奉法,司天日月星辰之行, (8) 宿离不忒,无失经纪,以初为常。 (9)

(1) 冬至后四十六日而立春,立春之节多在是月也。

(2) 谒,告也。《周礼》“太史掌国之六典,正岁时以序事”,故告天子以立春日也。盛德在木,王东方也。

(3) 《论语》曰:“斋必变食,居必迁坐。”自禋洁也。

(4) 率,使也。迎春木气于东方八里之郊。

(5) 赏,爵禄之赏也。三公至尊,坐而论道,不嫌不赏,故但言卿、诸侯、大夫者也。

【校】旧本“卿”上衍“公”字,乃后人据《月令》增入,而不知其与注不合也。

(6) 相,三公也。出为二伯,一相处于内也。布阳德和柔之令,行其庆善,施其泽惠,下至于兆民,无不被之也。

(7) 各得其所也。

(8) 典,六典。法,八法。日月五星,行度迟速,太史之职也,故命使司知之也。

(9) 忒,差也。星辰宿度,司知其度,以起牵牛之初为常。

【校】案:冬至十一月中起牵牛一度。

是月也,天子乃以元日祈谷于上帝。 (1) 乃择元辰,天子亲载耒耜,措之参于保介之御间, (2) 率三公、九卿、诸侯、大夫躬耕帝籍田, (3) 天子三推,三公五推,卿、诸侯、大夫九推。 (4) 反,执爵于太寝, (5) 三公、九卿、诸侯、大夫皆御,命曰“劳酒”。 (6)

(1) 日,从甲至癸也。元,善也。祈,求也。上帝,天帝也。

(2) 元,善也。辰,十二辰,从子至亥也。耒耜,耕器也。措,置也。保介,副也。御,致也。择善辰之日,载耒耜之具于籍田,致于保介之间施用之也。

【校】《月令》“参于”作“于参”。注“元善也”三字衍,所解于文义不甚顺。郑以保介为车右,此云“副也”,当谓副车。

(3) 躬,亲也。天子籍田千亩,以供上帝之粢盛,故曰“帝籍”。

【校】《月令》“帝籍”下无“田”字,此书《上农》篇亦有之。

(4) 礼以三为文,故天子三推,谓一发也。《国语》曰:“王耕一发,班三之。”班,次也。谓公、卿、大夫各三,其上公三发,卿九发,大夫二十七发也。

【校】正文“大夫”,《月令》无。案:《周语》作“王耕一墢”,墢有钵、跋二音,《说文》作“坺”,云“一臿土也”。

(5) 爵,饮爵。太寝,祖庙也。示归功于先祖,故于庙饮酒也。

(6) 御致天子之命,劳群臣于太庙,饮之以酒。

是月也,天气下降,地气上腾,天地和同,草木繁动。 (1) 王布农事,命田舍东郊, (2) 皆修封疆,审端径术, (3) 善相丘陵阪险原隰, (4) 土地所宜,五谷所殖, (5) 以教道民,必躬亲之。 (6) 田事既饬,先定准直,农乃不惑。 (7)

(1) 是月也,泰卦用事,乾下坤上,天地和同。繁,众。动,挺而生也。

【校】“繁动”,《月令》作“萌动”。

(2) 命,令也。东郊,农郊也。命农大夫舍止东郊,监视田事。

(3) 修,治也。封,界也。起其疆畔,纠督惰窳于疆下也。《诗》云:“中田有庐,疆埸有瓜。”无休废也。端正其径路,不得邪行败稼穑也。

【校】《汉书·五行志》载谣曰“邪径败良田”,“灭明不由径”,亦当是不随众人穿田取捷耳。

(4) 相,视也。阪险,倾危也。广平曰原。下湿曰隰。

(5) 殖,长。

(6) 《诗》云:“弗躬弗亲,庶民弗信。”

(7) 饬,读作敕。敕督田事,准定其功,农夫正直不疑惑。

是月也,命乐正入学习舞。 (1) 乃修祭典,命祀山林川泽,牺牲无用牝。 (2) 禁止伐木, (3) 无覆巢,无杀孩虫胎夭飞鸟,无麛无卵, (4) 无聚大众,无置城郭, (5) 掩骼霾髊。 (6)

(1) 乐正,乐官之长也。入学官,教国子讲习羽籥之舞。《周礼》:“大胥掌学士之版,以六乐之会正舞位也。”

(2) 典,掌也。功施于民则祀之。山林川泽,百物所生,又能兴云雨以殖嘉苗,故祀之。无用牝,尚蠲洁也。

(3) 春,木王,尚长养也。

(4) 蕃庶物也。麋子曰夭,鹿子曰麛也。

【校】案:《月令正义》云:“胎谓在腹中者,夭谓生而已出者。”此及《淮南注》皆云“麋子曰夭”,本《尔雅·释兽》文。彼“夭”字作“ ”。

(5) 置,立也。

(6) 髊,读水渍物之渍。白骨曰骼,有肉曰髊。掩霾者,覆藏之也。顺木德而尚仁恩也。

是月也,不可以称兵,称兵必有天殃。 (1) 兵戎不起,不可以从我始。 (2) 无变天之道, (3) 无绝地之理, (4) 无乱人之纪。 (5) 孟春行夏令,则风雨不时,草木早槁,国乃有恐。 (6) 行秋令,则民大疫,疾风暴雨数至,藜莠蓬蒿并兴。 (7) 行冬令,则水潦为败,霜雪大挚,首种不入。 (8)

(1) 称,举也。殃,咎也。

(2) 春当行仁,非兴兵征伐时也,故曰“不可以从我始”。

(3) 变犹戾也。

(4) 绝犹断也。

(5) 人反德为乱。纪,道也。

(6) 春,木也。夏,火也。木德用事,法当宽仁,而行火令,火性炎上,故使草木槁落,不待秋冬,故曰天气不和,国人惶恐也。

【校】“风雨”,《月令》作“雨水”。

(7) 木,仁;金,杀;而行其令,气不和,故民疫病也。金生水,与水相干,故风雨数至,荒秽滋生,是以藜莠蓬蒿并兴。

【校】《月令》“疾风”作“猋风”,“数至”作“总至”。

(8) 春,阳;冬,阴也;而行其令,阴乘阳,故水潦为败,雪霜大挚,伤害五谷。春为岁始,稼穑应之不成熟也,故曰“首种不入”。

【校】案:《月令注》云:“旧说首种谓稷。”





本生


二曰:

始生之者,天也;养成之者,人也。 (1) 能养天之所生而勿撄之,谓之天子。 (2) 天子之动也,以全天为故者也。 (3) 此官之所自立也。 (4) 立官者以全生也。 (5) 今世之惑主, (6) 多官而反以害生,则失所为立之矣。 (7) 譬之若修兵者,以备寇也。今修兵而反以自攻,则亦失所为修之矣。 (8)

(1) 始,初也。

(2) 撄犹戾也。

【校】旧本作“谓天子”,无“之”字,孙据《太平御览》七十七增。

(3) 全犹顺也。天,性也。故,事也。

(4) 官,正也。自,从也。

(5) 生,性也。

(6) 主,谓王也。

(7) 多立官,致任不肖人,乱象干度,故以害生也,失其所为立官之法也。

(8) 若秦筑长城以备患,不知长城之所以自亡也,亦失其所为修兵之法也。

夫水之性清,土者抇之,故不得清; (1) 人之性寿,物者抇之,故不得寿。 (2) 物也者所以养性也,非所以性养也。 (3) 今世之人,惑者多以性养物, (4) 则不知轻重也。 (5) 不知轻重,则重者为轻,轻者为重矣。若此,则每动无不败。以此为君,悖;以此为臣,乱;以此为子,狂。三者国有一焉,无幸必亡。 (6)

(1) 抇,读曰骨。骨,浊也。

【校】注似衍一“骨”字,《说文》“淈,浊也”,与汩、滑义同,并音骨。

(2) 抇,乱也。乱之使夭折也。

(3) 物者,货贿,所以养人也。世人贪欲过制者,多所以取祸,故曰“非所以性养也”。

(4) 夫无为者,不以身役物,有为者,则以物役身,故曰“惑者多以性养物”也。

(5) 轻,喻物。重,喻身。

(6) 假令有幸,且犹危危病者也。

今有声于此,耳听之必慊, (1) 已听之则使人聋,必弗听; (2) 有色于此,目视之必慊,已视之则使人盲,必弗视; (3) 有味于此,口食之必慊,已食之则使人瘖,必弗食。 (4) 是故圣人之于声色滋味也,利于性则取之,害于性则舍之,此全性之道也。世之贵富者,其于声色滋味也多惑者, (5) 日夜求,幸而得之则遁焉。 (6) 遁焉,性恶得不伤? (7) 万人操弓,共射其一招,招无不中; (8) 万物章章,以害一生,生无不伤, (9) 以便一生,生无不长。 (10) 故圣人之制万物也,以全其天也, (11) 天全则神和矣,目明矣,耳聪矣,鼻臭矣,口敏矣,三百六十节皆通利矣。若此人者,不言而信, (12) 不谋而当,不虑而得, (13) 精通乎天地,神覆乎宇宙, (14) 其于物无不受也,无不裹也, (15) 若天地然; (16) 上为天子而不骄, (17) 下为匹夫而不惛, (18) 此之谓全德之人。 (19)

(1) 慊,快也。

(2) 以聋,故不当听也。

(3) 以盲,故不当视也。

(4) 以瘖,故不当食也。《老子》曰“五声乱耳,使耳不聪;五色乱目,使目不明;五味实口,使口爽伤”也。

【校】案:《老子道经》云“五音令人耳聋,五色令人目盲,五味令人口爽”,此约略其文耳。“实口”,后注亦同,非误。

(5) 惑,眩。

(6) 遁,流逸不能自禁也。

(7) 恶,安也。伤,病也。

(8) 招,埻的也。众人所见,会弓射之,故曰“无不中”也。

【校】“共射一招”中间“其”字衍。注“埻”与“准”音义同。

(9) 章章,明美貌。故生陨也。

(10) 便,利也。利其生性,故生长久也。

(11) 天,身也。

(12) 法天不言,四时行焉,是其信也。

(13) 《诗》云“不识不知,顺帝之则”,故曰不谋虑而当,合得事实。

(14) 宇宙,区宇之内。言其德大,皆覆被也。

(15) 受犹承也。裹犹囊也。

(16) 其德如天无不覆,如地无不载,故曰“若天地然”也。

(17) 常战栗也,故《尧戒》曰:“战战栗栗,日慎一日。”

(18) 惛,读忧闷之闷,义亦然也。

(19) 其德行升降,无所亏阙,故曰“全”。

贵富而不知道,适足以为患, (1) 不如贫贱。贫贱之致物也难,虽欲过之奚由? (2) 出则以车,入则以辇,务以自佚, (3) 命之曰招 之机; (4) 肥肉厚酒,务以自强,命之曰烂肠之食; (5) 靡曼皓齿,郑、卫之音,务以自乐,命之曰伐性之斧: (6) 三患者,贵富之所致也。故古之人有不肯贵富者矣,由重生故也, (7) 非夸以名也,为其实也。 (8) 则此论之不可不察也。 (9)

(1) 不知持盈止足之道,以至破亡,故曰“适足以为患”也。

(2) 贫贱无势,不能致情欲之物,故曰“难”也。于礼无为,于身无阙,故曰“虽欲过之奚由”也。

(3) 人引车曰辇。出门乘车,入门用辇,此骄佚之务也。

(4) 招,至也。蹷机,门内之位也。乘辇于宫中游翔,至于蹷机,故曰“务以自佚”也。《诗》云:“不远伊尔,薄送我畿。”此不过蹷之谓。

【校】案:此注全不谙文义而妄说。盖招,致也。蹷者,痿蹷。过佚则血脉不周通,骨干不坚利,故为致蹷之机括。高误以蹷为门橛,又误以机即《诗》之畿,故有斯讹。黄东发亦言其误。又案:李善注《文选》枚乘《七发》引此“招”作“佁”,嗣理切,孤文无证,亦不可从。

(5) 《论语》曰“肉虽多,不使胜食气”,又曰“不为酒困”,《老子》曰“五味实口,使口爽伤”,故谓之“烂肠之食”也。

【校】“务以自强”,旧作“相强”,孙据《御览》八百四十五改,与前后句法正同。卢云:“案《贾谊书·傅职》云‘饮酒而醉,食肉而饱,饱而强食’,正自强之谓也。”

(6) 靡曼,细理弱肌,美色也。皓齿,《诗》所谓“齿如瓠犀”者也。郑国淫辟,男女私会于溱、洧之上,有“询 ”之乐,“勺药”之和。昔者殷纣使乐师作《朝歌》、《北鄙》靡靡之乐,以为淫乱。武王伐纣,乐师抱其乐器自投濮水之中。暨卫灵公北朝于晋,宿于濮上,夜闻水中有琴瑟之音,乃使师涓以琴写其音。灵公至晋国,晋平公作乐,公曰:“寡人得新声,请以乐君。”遂使涓作之,平公大说。师旷止之曰:“此亡国之音也。纣之太师以此音自投于濮水,得此声必于濮水之上。”地在卫,因曰“郑、卫之音”。以其淫辟灭亡,故曰“伐性之斧”者也。

【校】梁仲子案:“《意林》所载作‘伐命之斧’。”注“细理弱肌”,本多无“理弱”二字,今从朱本,与洪兴祖补注《楚辞·招魂》所引合。

(7) 古人,谓尧时许由、方回、善绻,舜时雄陶,周时伯夷,汉时四皓,皆不肯富贵者。高位实疾颠,故曰“重生故也”。

【校】注“方回”,旧本皆误作“方因”。“善绻”或“善卷”之驳文。“雄陶”误作“皋陶”。案:《国策》齐颜斶曰“舜有七友”,陶潜《四八目》具载其名,以雄陶为首,盖本《尸子》,今从之。《汉书·古今人表》作“雒陶”。“高位实疾颠”,《周语》文,今本依宋庠之说改作“偾”字。案注“颠,陨也”,正是陨坠之义。宋误为“殒”,故云“宜从偾”。若是“偾”,注当言“踣”乃合。诱注《知分》篇亦是“颠”字。

(8) 夸,虚也。非以为轻富贵求虚名也,以为其可以全生保性之实也。

(9) 论此上句贵贱祸福,不可不察也。





重己


三曰:

倕,至巧也。人不爱倕之指,而爱己之指,有之利故也。 (1) 人不爱昆山之玉、江汉之珠, (2) 而爱己之一苍璧小玑,有之利故也。 (3) 今吾生之为我有,而利我亦大矣。 (4) 论其贵贱,爵为天子,不足以比焉; (5) 论其轻重,富有天下,不可以易之; (6) 论其安危,一曙失之,终身不复得: (7) 此三者,有道者之所慎也。 (8) 有慎之而反害之者,不达乎性命之情也。 (9) 不达乎性命之情,慎之何益? (10) 是师者之爱子也,不免乎枕之以糠;是聋者之养婴儿也,方雷而窥之于堂,有殊弗知慎者! (11) 夫弗知慎者,是死生存亡可不可,未始有别也。 (12) 未始有别者,其所谓是未尝是,其所谓非未尝非,是其所谓非,非其所谓是,此之谓大惑。 (13) 若此人者,天之所祸也。 (14) 以此治身,必死必殃;以此治国,必残必亡。 (15) 夫死殃残亡,非自至也,惑召之也。 (16) 寿长至常亦然。 (17) 故有道者,不察所召,而察其召之者, (18) 则其至不可禁矣。 (19) 此论不可不熟。 (20)

(1) 倕,尧之巧工也。虽巧无益于己,故不爱之也。己指虽不如倕指巧,犹自为用,故言“有之利故也”。

(2) 昆山之玉,燔以炉炭,三日三夜,色泽不变,玉之美者也。江汉有夜光之明珠,珠之美者也。

(3) 苍璧,石多玉少也,珠之不圜者曰玑,皆喻不好也。而爱之者,有之为己用,得其利故也。

(4) 吾生我有,有我身也,天下之利有我,如我之爱苍璧与小玑,有之利故也,故曰“利我亦大矣”。

(5) 论其所贵所贱,人虽尊为天子,不足以比己之所贱。

(6) 论其所轻所重,人虽富有天下之财,不肯以己易之。

(7) 贫贱所以安也,富贵所以危也。曙,明日也。言一日失其所以安,终身不能复得之也。

(8) 道尚无为,不尚此三者,故曰“有道者之所慎”。

(9) 守慎无为,轻贵重身,当时行则行,时止则止,而反有害之者,故曰“不达乎性命之情”者也。

(10) 虽慎之犹见害,故曰“何益”。

(11) 师,瞽师,目无见者也,故枕子以糠,糠易眯子目,非利之者也。聋者不闻雷之声,不顿颡自拍解谢咎过,而反徐步窥儿于堂,故曰“有殊弗知慎者”也。殊犹甚也。

【校】注“易眯”,旧作“其盲眯”,讹。

(12) 言不能别知也。

(13) 己之所是,众人之所非也,故曰“未尝是”。己之所非,众人之所是也,故曰“未尝非”。是己之所是,非己之所非,而以此求同于己者也,故谓之“大惑”。

(14) 祸,咎也。

(15) 以其天之所祸也,不死不亡者,未之有也,故曰“必”。

(16) 召,致也。以惑致之也。

(17) 亦以仁义召之也。

(18) 所召,仁与义也。推行仁义,寿长自至,故曰“不察所召”也。召之者,不行仁义,残亡应行而至,故曰“察其召之”也。

(19) 禹、汤罪己,其兴也勃焉;桀、纣罪人,其亡也忽焉。皆己自召之,何可禁御?

(20) 熟犹知也。

使乌获疾引牛尾,尾绝力勯,而牛不可行,逆也; (1) 使五尺竖子引其棬,而牛恣所以之,顺也。 (2) 世之人主、贵人, (3) 无贤不肖,莫不欲长生久视, (4) 而日逆其生,欲之何益? (5) 凡生之长也, (6) 顺之也,使生不顺者,欲也, (7) 故圣人必先适欲。 (8)

(1) 乌获,秦武王力士也,能举千钧。勯,读曰单。单,尽也。

(2) 恣,从也。之,至也。

(3) 人主,谓王者、诸侯也。贵人,谓公卿大夫也。

(4) 视,活也。

(5) 王者、贵人所行,淫侈纵欲暴虐,反戾天常,不顺生道,日所施行,无不到逆其生,虽欲长生,若乌获多力,到引牛尾,尾绝不能行,故曰“欲之何益”也。

【校】注“到”字从李本,古“倒”字。

(6) 【校】“之”字旧本缺,孙据《御览》七百二十增。

(7) 欲,情欲也。

(8) 适犹节也。

室大则多阴,台高则多阳,多阴则蹷, (1) 多阳则痿, (2) 此阴阳不适之患也。 (3) 是故先王不处大室, (4) 不为高台, (5) 味不众珍, (6) 衣不 热。 (7) 热则理塞, (8) 理塞则气不达; (9) 味众珍则胃充, (10) 胃充则中大鞔; (11) 中大鞔而气不达, (12) 以此长生可得乎? (13) 昔先圣王之为苑囿园池也,足以观望劳形而已矣; (14) 其为宫室台榭也,足以辟燥湿而已矣; (15) 其为舆马衣裘也,足以逸身暖骸而已矣; (16) 其为饮食酏醴也,足以适味充虚而已矣; (17) 其为声色音乐也,足以安性自娱而已矣。 (18) 五者,圣王之所以养性也,非好俭而恶费也,节乎性也。 (19)

(1) 蹷,逆寒疾也。

(2) 痿,躄不能行也。

(3) 患,害也。

(4) 为蹷疾也。

(5) 为痿疾也。

(6) 为伤胃也。

(7) ,读曰亶。亶,厚也。

(8) 理塞,脉理闭结也。

【校】“塞”字旧本作“寒”,孙据《御览》作“塞”,下同。

(9) 达,通也。

(10) 充,满也。

(11) 鞔,读曰懑。不胜食气为懑病也。肥肉厚酒,烂肠之食,此之谓也。

(12) 不达,壅闭也。

(13) 言不得也。

【校】《御览》作“以此求长生,其可得乎”。

(14) 畜禽兽所,大曰苑,小曰囿,《诗》云“王在灵囿”。树果曰园,《诗》曰“园有树桃”。有水曰池。可以游观娱志,故曰足以劳形而已。

(15) 宫,庙也。室,寝也。《尔雅》曰:“宫谓之室,室谓之宫。”土方而高曰台。有屋曰榭。燥谓阳炎,湿谓雨露,故曰足以备之而已。

【校】旧校云“‘辟’一作‘备’”。

(16) 逸,安也。

(17) 酏,读如《诗》“蛇蛇硕言”之蛇。《周礼》“浆人掌王之六饮,水浆醴凉医酏”也。又《酒正》“二曰醴齐”。醴者,以糵与黍相体,不以曲也,浊而甜耳。

【校】注“相体”,旧作“相醴”,误,今改正。

(18) 声,五音宫商角徵羽也。色,青黄赤白黑也。

(19) 节犹和也。和适其情性而已,不过制也。





贵公


四曰:

昔先圣王之治天下也,必先公, (1) 公则天下平矣。 (2) 平得于公。 (3) 尝试观于上志, (4) 有得天下者众矣,其得之以公, (5) 其失之必以偏。 (6) 凡主之立也,生于公。 (7) 故《鸿范》曰:“无偏无党,王道荡荡; (8) 无偏无颇,遵王之义; (9) 无或作好,遵王之道; (10) 无或作恶,遵王之路。” (11)

(1) 公,正也。

(2) 平,和也。

(3) 得犹出也。

(4) 上志,古记也。

(5) 【校】孙云:“《御览》七十七作‘有天下’,无‘得’字,‘得之’下有‘必’字。”

(6) 偏,私,不正也。

(7) 生,性也。

(8) 荡荡,平易也。《诗》云“鲁道有荡”。

(9) 义,法也。

【校】案:义,古音俄,正与颇协。而唐《孝明诏》改从《易》泰卦九三之“无平不陂”,非是。观此与《宋世家》犹作“颇”字,乃古书之未经窜改者。梁伯子云:“王逸注《离骚》‘循绳墨而不颇’,引《易》作‘不颇’,知《易》本不作‘陂’也。‘义’,古作‘谊’。案:宜有何音,亦与颇协。”

(10) 或,有也。好,私好,鬻公平于曲惠也。

(11) 恶,擅作威也。

天下非一人之天下也,天下之天下也。 (1) 阴阳之和,不长一类;甘露时雨,不私一物; (2) 万民之主,不阿一人。 (3) 伯禽将行,请所以治鲁, (4) 周公曰:“利而勿利也。” (5) 荆人有遗弓者而不肯索, (6) 曰:“荆人遗之,荆人得之,又何索焉?”孔子闻之,曰:“去其荆而可矣。” (7) 老聃闻之,曰:“去其人而可矣。”故老聃则至公矣。 (8) 天地大矣,生而弗子,成而弗有, (9) 万物皆被其泽,得其利,而莫知其所由始, (10) 此三皇、五帝之德也。 (11)

(1) 《书》曰“皇天无亲,惟德是辅”,故曰“天下之天下也”。

(2) 私犹异也。

(3) 阿亦私也。

(4) 伯禽,周公子也,成王封之于鲁,《诗》云:“建尔元子,俾侯于鲁。”

(5) 务在利民,勿自利也。

(6) 遗,失也。

(7) 言人得之而已,何必荆人也。

(8) 公,正也。言天下得之而已,何必人,故曰“至公”,无所私为也。

(9) 天大地大,生育民人,不以为己子,成遂万物,不以为己有也。

(10) 由,从也。万物皆蒙天地之泽而得其利,若尧时父老无徭役之劳,击壤于里陌,自以为当然,故曰莫知其所从始也。

(11) 三皇、五帝德大,能法天地,民人被其泽而得其利,亦不知其所从始也。《老子》云:“圣人不仁,以百姓为刍狗。”此之谓也。

管仲有病,桓公往问之,曰:“仲父之病矣, (1) 渍甚,国人弗讳, (2) 寡人将谁属国?” (3) 管仲对曰:“昔者臣尽力竭智,犹未足以知之也, (4) 今病在于朝夕之中,臣奚能言?” (5) 桓公曰:“此大事也,愿仲父之教寡人也。” (6) 管仲敬诺,曰:“公谁欲相?” (7) 公曰:“鲍叔牙可乎?”管仲对曰:“不可,夷吾善鲍叔牙。 (8) 鲍叔牙之为人也,清廉洁直,视不己若者不比于人, (9) 一闻人之过,终身不忘。” (10) “勿已,则隰朋其可乎”?“隰朋之为人也,上志而下求, (11) 丑不若黄帝,而哀不己若者。 (12) 其于国也,有不闻也; (13) 其于物也,有不知也; (14) 其于人也,有不见也。 (15) 勿已乎,则隰朋可也。” (16) 夫相,大官也。处大官者,不欲小察, (17) 不欲小智。 (18) 故曰:大匠不斫, (19) 大庖不豆, (20) 大勇不斗, (21) 大兵不寇。 (22) 桓公行公去私恶, (23) 用管子而为五伯长; (24) 行私阿所爱,用竖刀而虫出于户。 (25)

(1) 病,困也。

【校】孙云:“本书《知接》篇作‘仲父之疾病矣’,《列子·力命》篇倒作‘病疾’,又《庄子·徐无鬼》篇作‘仲父之病病矣’。”

(2) 渍,亦病也。按《公羊传》曰:“大眚者何?大渍也。”国人弗讳,言死生不可讳也。

【校】《御览》六百三十二作“如渍甚”。注“大眚”,《公羊》本作“大灾”,见《庄二十年传》,此“眚”字当是后人因后有“肆大眚”之文而误改之。

(3) 属,托也。

(4) 未足以知人也。

(5) 奚,何也。

(6) 教犹告也。

(7) 言欲用谁为相。

(8) 夷吾,管仲名。善犹和也。

(9) 比,方也。

(10) 念人之过,必亡人之功,不可为霸者之相也。

【校】注“亡”,似当作“忘”。

(11) 志上世贤人而模之也。求犹问也。《论语》曰:“孔文子不耻下问,是以谓之文也。”

(12) 自丑其德不如黄帝。《诗》云“高山仰止,景行行止”,乡昔人也。哀不如己者,欲教育训厉,使与己齐也。

【校】丑,耻也。“黄帝”,刘本作“皇帝”,“皇”、“黄”古通用。

(13) 不求闻其善也,志在利国而已矣。

(14) 物,事也。非其职事,不求知之也。

(15) 务在济民,不求见之。《孝经》曰:“非家至而见之也。”此总说隰朋所行。

(16) 言可用也。

(17) 察,苛也。

(18) 小智则好知小事以自矜伐也。

(19) 但视模范而已,不复自斫削也。

(20) 但调和五味,使神人享之而已,不复自列簠簋籩豆也。

(21) 大勇之人,折冲千里,而能服远,不复自斗也。

(22) 寇,害也。若武王之伐纣,扫除无道,释箕子之囚,朝成汤之庙,抚殷之民,不寇害之也。

(23) 于人之过,无所念、无所私也,故曰“去私”也。

(24) 长,上也。

(25) 阿竖刀、易牙之谀,不正適长。其死也,国乱民扰,五子争立,无主丧,六十日乃殡,至使虫流出户也。

【校】刀本有貂音,后人始作“刁”字,今从古。

人之少也愚,其长也智。故智而用私,不若愚而用公。 (1) 日醉而饰服, (2) 私利而立公,贪戾而求王,舜弗能为。 (3)

(1) 用私以败,用公则济。

(2) 饰,读曰敕。《礼》“丧不饮酒食肉”,而日醉于酒,欲整丧纪,犹无目欲视青黄,无耳欲听宫商也。

(3) 舜犹不能为,况凡人乎。





去私


五曰:

天无私覆也,地无私载也,日月无私烛也,四时无私行也, (1) 行其德而万物得遂长焉。 (2) 黄帝言曰:“声禁重, (3) 色禁重, (4) 衣禁重, (5) 香禁重, (6) 味禁重, (7) 室禁重。” (8) 尧有子十人,不与其子而授舜; (9) 舜有子九人,不与其子而授禹:至公也。 (10)

(1) 【校】旧校云:“‘行’一作‘为’。”孙案:“《御览》四百二十九正作‘为’。”

(2) 遂,成也。

(3) 不欲虚名过其实也。

【校】《黄氏日抄》云:“此禁声色大过耳。”注非。

(4) 不欲好色至淫纵也。

(5) 不欲衣服逾僭,若子臧好聚鹬冠也。

(6) 不欲奢侈芬香闻四远也。

(7) 不欲厚味胜食气伤性也。

(8) 不欲宫室崇侈,使土木胜也。

(9) 《孟子》曰:“尧使九男二女事舜。”此曰“十子”,殆丹朱为胤子,不在数中。

(10) 《国语》曰:“舜有商均。”此曰“九子”,不知出于何书也。

晋平公问于祁黄羊曰:“南阳无令,其谁可而为之?” (1) 祁黄羊对曰:“解狐可。” (2) 平公曰:“解狐非子之仇邪?” (3) 对曰:“君问可,非问臣之仇也。”平公曰:“善。”遂用之。国人称善焉。居有间, (4) 平公又问祁黄羊曰:“国无尉,其谁可而为之?”对曰:“午可。” (5) 平公曰:“午非子之子邪?”对曰:“君问可,非问臣之子也。”平公曰:“善。”又遂用之。国人称善焉。孔子闻之曰:“善哉!祁黄羊之论也。外举不避仇,内举不避子,祁黄羊可谓公矣。”

(1) 南阳,晋山阳河北之邑,今河内温阳、樊州之属皆是也。令,君也。而,能。为,治。

【校】注“州”,旧本讹作“川”。案:州为汉河内郡之县,今改正。

(2) 黄羊,晋大夫祁奚之字。

(3) 平公,晋悼公之子彪。

【校】平公、黄羊,不于始见下注,何也?

(4) 间,顷也。

(5) 《传》曰:“祁奚请老,晋侯问嗣焉,称解狐,其仇也。将立之而卒,又问,对曰‘午也可’。”

【校】案:《左传》在鲁襄三年,晋悼公之四年也,此云“平公”,误。注引《传》文虽略,亦足以正吕氏所记之谬。

墨者有鉅子腹 ,居秦, (1) 其子杀人,秦惠王曰:“先生之年长矣,非有它子也,寡人已令吏弗诛矣, (2) 先生之以此听寡人也。”腹 对曰:“墨者之法曰:杀人者死,伤人者刑。此所以禁杀伤人也。夫禁杀伤人者,天下之大义也。王虽为之赐, (3) 而令吏弗诛,腹 不可不行墨子之法。” (4) 不许惠王,而遂杀之。子,人之所私也, (5) 忍所私以行大义, (6) 鉅子可谓公矣。

(1) 鉅,姓。子,通称。腹 ,字也。 ,读曰车 之 。

【校】鉅子,犹鉅儒、鉅公之称,腹乃其姓耳。《庄子·天下》篇“以巨子为圣人”,向、崔本作“鉅”。向云:“墨家号其道理成者为鉅子,若儒家之硕儒。”“ ”与《檀弓下》“孺子 ”实同一字,彼《释文》“音吐孙反”,此音车 《淮南子·精神训》“守其篅 ”,盖竹簟席所为。《玉篇》“音徒本切”,与今人所呼合。旧本作“ ”,盖书家“屯”字往往作“ ”,而此又误从“乇”也。

(2) 惠王,秦孝公子驷。

(3) 受赐也。

【校】案:赐犹惠也,注似误。

(4) 欲必行之,杀其子也。

(5) 私,爱也。

(6) 忍,读曰仁,行之忍也。

【校】注“曰仁”,李本作“仁行”,俱未详。

庖人调和而弗敢食,故可以为庖。若使庖人调和而食之,则不可以为庖矣。王伯之君亦然,诛暴而不私,以封天下之贤者,故可以为王伯;若使王伯之君诛暴而私之,则亦不可以为王伯矣。 (1)

(1) 《传》曰:“作事威,克其爱,虽小必济。”故曰“诛暴而弗私”也。假令有所私枉,则不可以为王伯君矣。




————————————————————

[1] 鸣:原本作“鸟”,蒋维乔曰:浙刊本注“鸣”误“鸟”。





第二卷 仲春纪



仲春


一曰:

仲春之月,日在奎, (1) 昏弧中,旦建星中。 (2) 其日甲乙,其帝太皞,其神句芒。其虫鳞,其音角,律中夹钟, (3) 其数八。其味酸,其臭膻,其祀户,祭先脾。始雨水,桃李华, (4) 苍庚鸣,鹰化为鸠。 (5) 天子居青阳太庙, (6) 乘鸾辂,驾苍龙,载青旂,衣青衣,服青玉,食麦与羊,其器疏以达。 (7)

(1) 仲春,夏之二月。奎,西方宿,鲁之分野也。是月,日躔此宿。

(2) 弧星在舆鬼南,建星在斗上。是月昏旦时,皆中于南方。

(3) 夹钟,阴律也。是月,万物去阴,夹阳而生,故竹管音中夹钟也。

【校】卢云:“案:注旧本作‘去阳夹阴’,讹。《淮南注》作‘去阴夹阳,聚地而生’。”今据改正。又《初学记》引高注云“是月万物去阴而生,故候管者中夹钟”,可以互证。其不并引“竹管”之语者,以正月已用郑注“管以铜为之”,故不欲互异也。“钟”、“鍾”得两通。

(4) 自冬冰雪至此土发而耕,故曰“始雨水”也。桃李之属皆舒华也。

【校】《礼记·月令》作“桃始华”。

(5) 苍庚,《尔雅》曰“商庚、黎黄,楚雀”也。齐人谓之抟黍,秦人谓之黄离,幽、冀谓之黄鸟,《诗》云“黄鸟于飞,集于灌木”是也。至是月而鸣。鹰化为鸠,喙正直,不鸷击也。鸠,盖布谷鸟。

【校】案:《尔雅》“黎黄”作“ 黄”,郭璞注“皇黄鸟”下云“俗呼黄离留”,《淮南注》作“秦人谓之黄流离”,此作“黄离”,三者皆可通,无烦补字。

(6) 青阳,东向堂。太庙,中央室。

(7) 说在《孟春》。

是月也,安萌牙,养幼少,存诸孤。 (1) 择元日,命人社。 (2) 命有司,省囹圄,去桎梏,无肆掠,止狱讼。 (3)

(1) 顺春阳,长养幼少,存恤孤寡。萌牙诸当生者不扰动,故曰“安”。

(2) 元,善也。日,从甲至癸也。社祭后土,所以为民祈谷也。嫌日有从否,重农事,故卜择之。

(3) 有司,理官,主狱者也。囹圄,法室。省之者,赦轻微也。在足曰桎,在手曰梏。肆,极。掠,笞也。言“无”者,须立秋也。止,禁。

是月也,玄鸟至。至之日,以太牢祀于高禖。 (1) 天子亲往,后妃率九嫔御, (2) 乃礼天子所御,带以弓 ,授以弓矢,于高禖之前。 (3)

(1) 玄鸟,燕也,春分而来,秋分而去。《传》曰:“玄鸟氏,司启者也。”《周礼》“媒氏以仲春之月,合男女于时也。奔则不禁。”因祭其神于郊,谓之“郊禖”。郊音与高相近,故或言“高禖”。王者后妃以玄鸟至日祈继嗣于高禖。三牲具曰太牢。

【校】案:《周礼》本作“于是时也,奔者不禁”。

(2) 王者,一后、三夫人、九嫔、二十七世妇。但后、夫人率九嫔祀高禖耳。御,见天子于高禖中也。

(3) 礼,后妃之侍见于天子者,于高禖祠之前。 ,弓韬也。授以弓矢,示服猛,得男象也。

是月也,日夜分,雷乃发声,始电。 (1) 蛰虫咸动,开户始出。 (2) 先雷三日,奋铎以令于兆民曰:“雷且发声, (3) 有不戒其容止者,生子不备,必有凶灾。” (4) 日夜分,则同度量,钧衡石,角斗桶,正权概。 (5)

(1) 分,等,昼夜钧也。冬阴闭固,阳伏于下,是月阳升,雷始发声。震气为雷,激气为电。

(2) 蛰伏之虫始动苏,开蛰之户始出生。

(3) 铎,木铃也,金口木舌为木铎,金舌为金铎。所以振告兆民,使知将雷也。

(4) 有不戒慎容止者,以雷电合房室者,生子必有瘖躄通精狂痴之疾,故曰“不备,必有凶灾”。

【校】通精,未详。

(5) 度,尺丈也。量,鬴钟也。钧,铨。衡石,称也。石,百二十斤。角,平。斗桶,量器也。称锤曰权。概,平斗斛者。令钧等也。

【校】案:《月令》“角斗甬”,“桶”与“甬”通用,《史记·商君传》“平斗桶”,郑康成音勇,小司马音统;《淮南》作“称”,亦“桶”之讹。李善注《文选》陆佐公《新刻漏铭》引作“角升桶”,“升”字误。

是月也,耕者少舍, (1) 乃修阖扇,寝庙必备。 (2) 无作大事,以妨农功。 (3)

(1) 少舍,皆耕在野,少有在都邑者也。《尚书》曰“厥民析”,散布在野。《传》曰“阴阳分布,震雷出滞,土地不备垦,辟在司寇”之谓也。

(2) 阖扇,门扇也。民所由出,故治之也。寝以安身,庙以事祖,故曰必无堕顿也。

【校】“必”,《月令》作“毕”,古通用,注自从“必”字作解。

(3) 大事,兵戈征伐也。

是月也,无竭川泽,无漉陂池,无焚山林。 (1) 天子乃献羔开冰,先荐寝庙。 (2) 上丁,命乐正,入舞舍采, (3) 天子乃率三公、九卿、诸侯亲往视之。 (4) 中丁,又命乐正,入学习乐。 (5)

(1) 皆为尽类夭物。

(2) 开冰室取冰,以治鉴,以祭庙。春荐韭卵。《诗》云:“二之日凿冰冲冲,三之日纳于凌阴,四之日其早,献羔祭韭。”此之谓也。

【校】注“治鉴”二字,旧作“凿”,讹,今据《周礼》改正。

(3) 是月上旬丁日,命乐官正,率卿大夫之子入学官习舞也。舍犹置也。初入学官,必礼先师,置釆帛于前以贽神也。《周礼》“春入学,舍釆合舞,秋颁学,合声,以六乐之会正舞位”,此之谓也。

【校】“入舞舍采”,《月令》作“习舞释菜”。郑注《学记》“菜谓芹藻之类”,与此注异。注“入学官”,各本多作“学宫”,唯李本作“官”。案《贾子·保傅》云:“学者所学之官也。”此官盖谓官寺,《正月纪注》中正作“学官”。

(4) 常事曰视。

【校】《月令》“诸侯”下有“大夫”。

(5) 中旬丁日又入学官习乐。乐所以移风易俗,协和民人也。谓六代之乐《云门》、《咸池》、《大韶》、《大頀》、《大夏》、《大武》也。《周礼》曰“以乐教和,则民不乖”,此之谓也。

【校】注“大頀”,注本作“大护”,与“濩”并通用。

是月也,祀不用牺牲,用圭璧,更皮币。 (1) 仲春行秋令,则其国大水,寒气总至,寇戎来征。 (2) 行冬令,则阳气不胜,麦乃不熟,民多相掠。 (3) 行夏令,则国乃大旱,暖气早来,虫螟为害。 (4)

(1) 是月尚生育,故不用牺牲。更,代也,以圭璧代牺牲也。皮币,鹿皮玄 束帛也。《记》曰“币帛皮圭,告于祖祢”,此之谓也。

(2) 仲春,阳中也。阳气长养而行秋金杀戮之令,故寒气猥至,寇害之兵来伐其国也。

(3) 冬阴肃杀,而行其令,阴气乘阳,阳气不胜,故麦不成熟,民饥穷,故相劫掠也。

(4) 夏气炎阳,而行其令,故大旱。火气热,故旱暖也。极阳生阴,故虫螟作害也。虫食稼心谓之螟。





贵生


二曰:

圣人深虑天下,莫贵于生。夫耳目鼻口,生之役也。 (1) 耳虽欲声,目虽欲色,鼻虽欲芬香,口虽欲滋味,害于生则止。在四官者, (2) 不欲,利于生者则弗为。 (3) 由此观之,耳目鼻口不得擅行,必有所制。 (4) 譬之若官职,不得擅为, (5) 必有所制。 (6) 此贵生之术也。

(1) 役,事也。

【校】案:有君之者,故曰“役”,观下文自明。

(2) 止,禁也。四官,耳目鼻口也。

(3) 则不治此四官之欲。

(4) 擅,专也。制,制于心也。

(5) 为,作。

(6) 制于君也。

尧以天下让于子州支父, (1) 子州支父对曰:“以我为天子犹可也。虽然,我适有幽忧之病,方将治之,未暇在天下也。” (2) 天下,重物也, (3) 而不以害其生,又况于它物乎? (4) 惟不以天下害其生者也,可以托天下。 (5)

(1) 子州支父,古贤人也。

【校】旧作“子州友父”,讹。《太平御览》八十引作“子州支父”,与《庄子·让王》篇、《汉书·古今人表》皆合。

(2) 幽,隐也。《诗》云“如有隐忧”。我心不悦,未暇在于治天下。

【校】案:《尔雅》云:“在,察也。”

(3) 重,大。物,事。

(4) 它犹异也。

(5) 托,付。

越人三世杀其君,王子搜患之, (1) 逃乎丹穴。 (2) 越国无君,求王子搜而不得,从之丹穴。王子搜不肯出,越人薰之以艾,乘之以王舆。王子搜援绥登车,仰天而呼曰:“君乎!独不可以舍我乎!” (3) 王子搜非恶为君也,恶为君之患也。 (4) 若王子搜者,可谓不以国伤其生矣,此固越人之所欲得而为君也。 (5)

(1) 王子搜,《淮南子》云“越王翳也”。

【校】案《竹书纪年》,翳之前唯有不寿见杀,次朱句立,即翳之父也。翳为子所弑,越人杀其子,立无余,又见弑,立无颛。是无颛之前方可云“三世杀其君”,王子搜似非翳也。

(2) 《淮南》云“山穴”也。

(3) 舍,置也。

(4) 患,害。

(5) 欲得王子搜为君也。

鲁君闻颜阖得道之人也,使人以币先焉。颜阖守闾,鹿布之衣而自饭牛。鲁君之使者至,颜阖自对之。使者曰:“此颜阖之家邪?”颜阖对曰:“此阖之家也。”使者致币,颜阖对曰:“恐听缪而遗使者罪,不若审之。” (1) 使者还反审之,复来求之,则不得已。 (2) 故若颜阖者,非恶富贵也,由重生恶之也。世之人主多以富贵骄得道之人,其不相知, (3) 岂不悲哉? (4)

(1) 恐缪误致币得罪,故劝令审之。

(2) 颜阖逾坯而逃之,故不得。

(3) 骄,泰也。《淮南记》曰:“鱼相忘乎江湖,人相忘乎道术。”言各得其志,故不相知之也。

(4) 悲于富贵而骄人也。

故曰:道之真以持身,其绪余以为国家, (1) 其土苴以治天下。 (2) 由此观之,帝王之功,圣人之余事也, (3) 非所以完身养生之道也。 (4) 今世俗之君子,危身弃生以徇物, (5) 彼且奚以此之也? (6) 彼且奚以此为也? (7) 凡圣人之动作也,必察其所以之 (8) 与其所以为。 (9) 今有人于此,以随侯之珠,弹千仞之雀,世必笑之。是何也?所用重,所要轻也。 (10) 夫生岂特随侯珠之重也哉?

(1) 以持身之余绪,以治国家。

(2) 土,瓦砾也。苴,草蒯也。土鼓、蒯桴,伊耆氏之乐也。《孝经》曰:“安上治民,莫善于礼;移风易俗,莫善于乐。”故可以治天下。苴,音同鲊。

【校】《庄子释文》:“土,敕雅反,又如字。苴,侧雅反。”观此注意,土自作如字读。

(3) 圣人治之,优有余裕,故曰“余事”。

(4) 尧、舜、禹、汤之治天下,黎黑瘦瘠,过家门而不入,故曰“非所以完身养生之道”,趋济民而已。

【校】案:“趋”与“取”同,如杨子取为我;《史记·酷吏传》“取为小治”之意相似。

(5) 徇犹随也。

(6) 此,此物也。之,至也。

(7) 彼,谓今世俗人云“君子优之也”,何以物为也。

(8) 之,至也。

(9) 为,作也。

(10) 重,谓随侯珠也。要,得也。轻,谓雀也。

子华子曰:“全生为上, (1) 亏生次之, (2) 死次之, (3) 迫生为下。” (4) 故所谓尊生者,全生之谓。 (5) 所谓全生者,六欲皆得其宜也。 (6) 所谓亏生者,六欲分得其宜也。 (7) 亏生则于其尊之者薄矣。其亏弥甚者也,其尊弥薄。 (8) 所谓死者,无有所以知,复其未生也。 (9) 所谓迫生者,六欲莫得其宜也,皆获其所甚恶者,服是也,辱是也。 (10) 辱莫大于不义,故不义,迫生也, (11) 而迫生非独不义也,故曰迫生不若死。 (12) 奚以知其然也?耳闻所恶,不若无闻;目见所恶,不若无见。故雷则掩耳,电则掩目,此其比也。凡六欲者,皆知其所甚恶,而必不得免,不若无有所以知。无有所以知者,死之谓也,故迫生不若死。嗜肉者,非腐鼠之谓也;嗜酒者,非败酒之谓也;尊生者,非迫生之谓也。

(1) 子华子,古体道人。无欲,故全其生。长生是行之上也。

(2) 少亏其生,和光同尘,可以次全生者。

(3) 守死不移其志,可以次亏生者。

(4) 迫,促也。促欲得生,尸素宠禄,志不高洁,人之下也。

(5) 于身无所亏,于义无所损,故曰“全生”。

(6) 六欲:生、死、耳、目、口、鼻也。

(7) 分,半也。

(8) 弥,益。

(9) 死君亲之难,义重于生,视死如归,故曰“无有所以知,复其未生也”。

(10) 服,行也。行不义,是故辱。

(11) 不能蹈义而死,迫于苟生。《语》曰:“水火,吾见蹈死者矣,未见蹈仁而死者也。”

(12) 迫,促。急于苟生,不仁义,不如蹈仁义死为贵。





情欲


三曰:

天生人而使有贪有欲。欲有情,情有节, (1) 圣人修节以止欲, (2) 故不过行其情也。 (3) 故耳之欲五声,目之欲五色,口之欲五味,情也。此三者,贵贱愚智贤不肖欲之若一, (4) 虽神农、黄帝,其与桀、纣同。 (5) 圣人之所以异者,得其情也。 (6) 由贵生动则得其情矣,不由贵生动则失其情矣。 (7) 此二者死生存亡之本也。 (8)

(1) 节,适也。

(2) 【校】旧校云:“‘止’,一作‘制’。”

(3) 不过其适。

(4) 三,谓耳目口也。一犹等也。

(5) 有天下同也。

【校】案:此足上文“欲之若一”耳。

(6) 圣人得其不过节之情。

(7) 失其不过节之情。

(8) 圣人得其情,乱人失其情。得情生存,失情死亡,故曰生死存亡之本。

俗主亏情,故每动为亡败。 (1) 耳不可赡,目不可厌,口不可满,身尽府种,筋骨沉滞,血脉壅塞,九窍寥寥,曲失其宜, (2) 虽有彭祖,犹不能为也。 (3) 其于物也,不可得之为欲, (4) 不可足之为求, (5) 大失生本。 (6) 民人怨谤,又树大雠; (7) 意气易动,蹻然不固; (8) 矜势好智,胸中欺诈; (9) 德义之缓,邪利之急。 (10) 身以困穷,虽后悔之,尚将奚及? (11) 巧佞之近,端直之远, (12) 国家大危,悔前之过,犹不可反。 (13) 闻言而惊,不得所由。 (14) 百病怒起,乱难时至,以此君人,为身大忧。 (15) 耳不乐声,目不乐色,口不甘味,与死无择。 (16)

(1) 俗主,凡君也。败,灭亡也。

(2) 府,腹疾也。种,首疾也。极三关之欲,以病其身,故九窍皆寥寥然虚。曲,过其适,以害其性也。

【校】孙云:“案《玉篇》‘疛,除又切,心腹疾’也,引此作‘身尽疛种’,然则‘府’字误也。后《尽数》篇亦同此误。”卢云:“案《尽数》篇‘郁处头则为肿为风,处腹则为张为府’,‘府’当为‘疛’,《玉篇》之说可从。”此处注虽以腹疾首疾分解,而“种”之为首疾亦当作“肿”。此云“身尽府种”,则举全体言之,又何必分腹与首邪?案《西山经》云“竹山有草,名曰黄雚,可以已疥,又可以已胕”,郭氏注云“治胕肿也,音符”,此“府种”即“胕肿”字假借耳。钱学源云:“《素问·五常政大论》‘少阳司天有寒热胕肿,太阳司天亦有筋脉不利,甚则胕肿’之语。”

(3) 彭祖,殷之贤臣,治性清静,不欲于物,盖寿七百岁。《论语》所谓“述而不作,信而好古,窃比于我老彭”是也。言虽彭祖之无欲,不能化治俗主,使之无欲,故曰“虽有彭祖,犹不能为”。

(4) 贵不可得之物,宝难得之货,此之谓欲,故曰“为欲”。

(5) 规求无足,不知纪极,不可盈厌,此之为求,故曰“为求”。

(6) 《老子》曰“出生入死”,故曰“大失生本”。

(7) 俗主求欲,民人皆怨而谤讪,如仇雠也。

(8) 蹻谓乘蹻之蹻,谓其流行速疾不坚固之貌,故其志气易动也。

【校】注疑是“读乘蹻之蹻”。禹山行乘桥,亦作“蹻”。《类篇》云:“以铁如锥,施之履下。音脚,亦音乔。”

(9) 矜大其宠势(原为“契”,据许维遹本改),好尚其所行,自谓为智,胸臆之中,欺诈不诚,所行暴虐,犹语民言恩惠也。

(10) 缓犹后,急犹先。

(11) 困犹危。奚,何也。

(12) 巧佞者亲近之,正直者疏远之。

(13) 反,见。

【校】注疑是“反,复”。

(14) 所行残暴,闻将危败灭亡之言而乃始惊怖,行不仁不义之所致也,故曰“不得所由”。由,用也。

(15) 此非恤民之道,故身大忧。

(16) 声色美味,死者所不得说,人不能乐甘之,故曰“与死无择”。择,别也。

古人得道者,生以寿长, (1) 声色滋味,能久乐之。奚故?论早定也。 (2) 论早定则知早啬, (3) 知早啬则精不竭。 (4) 秋早寒则冬必暖矣,春多雨则夏必旱矣,天地不能两,而况于人类乎?人之与天地也同, (5) 万物之形虽异,其情一体也。 (6) 故古之治身与天下者,必法天地也。 (7) 尊酌者众则速尽, (8) 万物之酌大贵之生者众矣, (9) 故大贵之生常速尽,非徒万物酌之也, (10) 又损其生以资天下之人, (11) 而终不自知。 (12) 功虽成乎外,而生亏乎内。 (13) 耳不可以听,目不可以视,口不可以食, (14) 胸中大扰,妄言想见,临死之上,颠倒惊惧,不知所为,用心如此,岂不悲哉! (15)

(1) 体道无欲象天,天予之福,故必寿长,终其性命。

(2) 体道者生而能行之,故曰“论早定”。

(3) 啬,爱。

(4) 爱精神,故不竭。

(5) 同于不能两也。

(6) 体,性也。情皆好生,故曰“一体”。

(7) 法,象也。

(8) 尊,酒也。酌揖之者多,故酒速尽也。

【校】“揖”与“挹”同。

(9) 万物酌揖阴阳以生。阴阳谕君。大贵君者,爱君之德以生者众也。

【校】注“爱”疑是“受”。梁仲子云:“朱本作‘万物酌君之德以生者众也’。”

(10) 酌,取之也。

(11) 资犹给。

(12) 知犹觉也。

(13) 《幽通记》曰“张修襮而内逼”,故曰亏生乎内。

【校】案班固《幽通赋》有此语,此与《必己》篇注皆作“幽通记”,当仍之。张谓张毅,事见《庄子》、《淮南》。“修襮”旧作“循襮”,今依后注,与班赋合。

(14) 【校】此下旧提行。今案中间文亦无缺,岂注有脱邪?

(15) 悲情欲而不知所为用心之人。

世人之事君者,皆以孙叔敖之遇荆庄王为幸。 (1) 自有道论之则不然,此荆国之幸。 (2) 荆庄王好周游田猎,驰骋弋射,欢乐无遗, (3) 尽傅其境内之劳与诸侯之忧于孙叔敖。 (4) 孙叔敖日夜不息,不得以便生为故, (5) 故使庄王功迹著乎竹帛,传乎后世。 (6)

(1) 孙叔敖,楚令尹,薳贾之子也。

【校】近时毛检讨大可辨叔敖非楚公族,并非 氏,乃期思之鄙人。卢云:“窃案《左氏宣十二年传》‘随武子云: 敖为宰,择楚国之令典,军行右辕’云云,下‘令尹南辕反旆’,又云‘王告令尹,改乘辕而北之’,是 敖即令尹孙叔敖,军事皆主之。前一年令尹 艾猎城沂,比年之间,楚令尹不闻置两人,《知分》篇虽有‘孙叔敖三为令尹而不喜,三去令尹而不忧’之语,乃是子文之事误记耳。况在军中必无轻易废置之理,其为一人无可疑者。与其信诸子,不如信《传》。”

(2) 言孙叔敖贤,能事君以道,致之于霸,荆国得之,幸也。

(3) 遗,废。

(4) 事功曰劳。尽俾付孙叔敖,使忧之也。

【校】“傅”与“付”通,旧作“传”,误,钱校改。

(5) 休息也,不得以便利生性,故不休息也。

(6) 庄王之霸,功传于后世,乃孙叔敖之日夜不息以广其君,君德之所以成也。





当染


四曰:

墨子见染素丝者而叹曰 (1) :“染于苍则苍,染于黄则黄,所以入者变,其色亦变,五入而以为五色矣。” (2) 故染不可不慎也。

(1) 墨子名翟,鲁人,作书七十二篇。

(2) 一入一色。

非独染丝然也, (1) 国亦有染。舜染于许由、伯阳, (2) 禹染于皋陶、伯益, (3) 汤染于伊尹、仲虺, (4) 武王染于太公望、周公旦, (5) 此四王者所染当,故王天下, (6) 立为天子,功名蔽天地。 (7) 举天下之仁义显人,必称此四王者。 (8) 夏桀染于干辛、歧踵戎, (9) 殷纣染于崇侯、恶来, (10) 周厉王染于虢公长父、荣夷终, (11) 幽王染于虢公鼓、祭公敦, (12) 此四王者所染不当,故国残身死,为天下僇。 (13) 举天下之不义辱人,必称此四王者。 (14) 齐桓公染于管仲、鲍叔, (15) 晋文公染于咎犯、郄偃, (16) 荆庄王染于孙叔敖、沈尹蒸, (17) 吴王阖庐染于伍员、文之仪, (18) 越王句践染于范蠡、大夫种, (19) 此五君者所染当,故霸诸侯,功名传于后世。范吉射染于张柳朔、王生, (20) 中行寅染于黄藉秦、高强, (21) 吴王夫差染于王孙雄、太宰嚭, (22) 智伯瑶染于智国、张武, (23) 中山尚染于魏义、椻长, (24) 宋康王染于唐鞅、田不禋。 (25) 此六君者所染不当,故国皆残亡,身或死辱,宗庙不血食,绝其后类,君臣离散,民人流亡。举天下之贪暴可羞人,必称此六君者。凡为君非为君而因荣也,非为君而因安也,以为行理也。行理生于当染。故古之善为君者,劳于论人, (26) 而佚于官事,得其经也。 (27) 不能为君者,伤形费神,愁心劳耳目,国愈危,身愈辱,不知要故也。 (28) 不知要故则所染不当, (29) 所染不当,理奚由至? (30) 六君者是已。六君者非不重其国爱其身也,所染不当也。存亡故不独是也,帝王亦然。 (31)

(1) 【校】“然”,旧误作“纱”,今据《墨子·所染》篇改正。

(2) 舜,颛顼五世之孙,瞽瞍之子也,名重华。许由,阳城人,尧聘之不至。伯阳,盖老子也,舜时师之者也。

(3) 禹,颛顼六世孙,鲧之子也,名文命。伯益,皋陶之子也。

【校】案:皋陶子乃伯翳,非益也,益乃高阳之第三子名 者,《路史》有辨,甚明。

(4) 汤,契后十二世孙王癸之子也,名天乙。伊尹,汤相,《诗》云:“实惟阿衡,实左右商王。”仲虺居薛,为汤之左相,皆贤德也。《孟子》曰:“王者师臣也。”

【校】当出《外书》,或约与景、丑语。

(5) 武王,周文王之子,名发。太公望,河内汲人也,佐武王伐纣,成王封之于齐。周公旦,武王之弟也,辅成王,封之于鲁。

【校】梁伯子云:“齐、鲁皆武王所封,此与《长见》篇注同误。”

(6) 所从染得其人,故曰“当”。

(7) 蔽犹极也。

(8) 称美其德以为喻也。

(9) 桀,夏后皋之孙,癸之子也。干辛、歧踵戎,桀之邪臣。

【校】“干辛”,旧本作“羊辛”,《知度》篇亦同。案《墨子》及《古今人表》、《抱朴子·良规》篇与此书《慎大》篇皆作“干辛”,《说苑·尊贤》篇作“干莘”,今据改正。又“歧踵戎”、《墨子》及诸书多作“推哆”,亦作“推侈”。

(10) 纣,帝乙之子,名辛。崇,国,侯,爵,名虎;恶来,嬴姓,飞廉之子;纣之谀臣。

【校】案:《书》称“商王受”,或云“字受德”,亦见《书》及《逸周书》。此云“名辛”,与《史》同。

(11) 厉王,周夷王之子,名胡。虢、荣,二卿士也。《传》曰:“荣夷公好专利,而不知大难。”

(12) 幽王,周厉王之孙,宣王之子,名官皇。虢公、祭公,二卿士也。《传》曰:“虢石父谗谄巧佞之人也,以此教王,其能久乎?”

【校】《墨子》作“染于傅公夷、蔡公穀”。注“官皇”,诸书多作“宫涅”。梁伯子云:“当从刘恕《外纪》、子由《古史》作‘宫湦’,《史记集解》‘徐广曰一作生’,惟名湦,故又作‘生’也。”

(13) 不当者,不得其人。僇,辱也。

(14) 称其恶以为戒也。

(15) 桓公,齐僖公之子,名小白。管、鲍,其二卿也。

(16) 文公,晋献公之子,名重耳。咎犯、郄偃者,其二大夫。

【校】“郄”乃“郤”之俗字。《墨子》作“高偃”,《御览》六百二十作“郭偃”。

(17) 庄王,楚穆王之子,名旅。孙、沈,其二大夫。

(18) 阖庐,吴王夷昧之子,名光。伍、文,其二大夫。

【校】“文”,旧本讹作“父”,今据《尊师》篇改正。《墨子》作“文义”。

(19) 句践,允常之子。范蠡,楚三户人也,字少伯。大夫种,姓文氏,字禽,楚之邹人。

【校】案:《越绝》云:“范蠡始居楚,生于宛橐,或伍户之虚。”“伍户”疑即“三户”,它书引《吴越春秋》有云:“文种为宛令,之三户之里见蠡。”案邹是时尚未属楚,《尊师》篇注又作“楚鄞人”,皆误,当作“楚之郢人”。钱詹事晓徵云:“《太平寰宇记·江陵府·人物》云‘文种,楚南郢人’,此必本于高氏注,北宋本犹未 也。种本楚郢人,故得为宛令,若邹若鄞,皆非楚地矣。王伯厚引《吕览注》以种为鄞人,则南宋本已误。然虞仲翔、朱育历数会稽先贤,初不及种,乾道《四明图经》、宝庆《四明志》叙人物亦无及种者,当依《寰宇记》改正。”

(20) 吉射,晋范献子鞅之子昭子也。张柳朔、王生二人者,吉射家臣也。

【校】《墨子》作“长柳朔、王胜”。

(21) 寅,晋大夫中行穆子之子荀子也。黄藉秦、高强,其家臣。高强,齐子尾之子,奔晋为中行氏之臣。

【校】《墨子》无“黄”字。

(22) 夫差,吴王阖庐子也。雄与嚭二人,其大夫也。嚭,晋伯宗之孙,楚州犁之子。

【校】“王孙雄”,《墨子》作“王孙雒”,《越绝》、《吴越春秋》皆作“王孙骆”,《说苑》作“公孙雒”,《国语》旧本亦作“雒”,宋庠《补音》从《史记》定作“雄”,且为之说曰:“汉改‘洛’为‘雒’,疑‘洛’字非吴人所名。”今案宋说误也。“有 有雒”,见于《鲁颂》;《春秋·文八年经》书“公子遂会雒戎”,《传》作“伊雒之戎”;《宣三年传》“楚子伐陆浑之戎,遂至于雒”;《汉书》弘农郡上雒;非后汉时始改也。今不若各从本书为得。

(23) 智瑶,宣子甲之子襄子也。国、武二人,其家臣。

(24) 尚,魏公子牟之后,魏得中山以邑之也。义、长,其二臣也。

【校】“椻”,《墨子》作“偃”。

(25) 唐、田,宋康王之二臣。

【校】《荀子·解蔽》篇杨倞注亦作“田不禋”。《古今人表》作“田不礼”,《御览》亦同。《墨子》作“伷不礼”。

(26) 论犹择也。

(27) 经,道。

(28) 愈,益也。益危辱者,不知所行之要约也。

(29) 所从染不得其人也。

(30) 至犹得也。

(31) 为帝王者,亦当知所从染也。

非独国有染也。孔子学于老聃、孟苏、夔靖叔。 (1) 鲁惠公使宰让请郊庙之礼于天子, (2) 桓王使史角往,惠公止之。 (3) 其后在于鲁,墨子学焉。 (4) 此二士者,无爵位以显人,无赏禄以利人, (5) 举天下之显荣者必称此二士也。 (6) 皆死久矣,从属弥众,弟子弥丰,充满天下。 (7) 王公大人从而显之,有爱子弟者,随而学焉,无时乏绝。子贡、子夏、曾子学于孔子。田子方学于子贡,段干木学于子夏,吴起学于曾子。禽滑 学于墨子,许犯学于禽滑 , (8) 田系学于许犯。孔墨之后学显荣于天下者众矣,不可胜数,皆所染者得当也。

(1) 三人皆体道者,亦染孔子。

(2) 惠公,鲁孝公之子,隐公之父。

(3) 止,留。

(4) 其后,史角之后也,亦染墨翟。

(5) 二士,谓孔子、墨翟。

(6) 称,说也。

(7) 弥,益;丰,盛也。言二士之徒,显荣者益盛,散布,故曰“充满天下”。

(8) 【校】梁仲子云:“疑当作‘禽滑釐’,《列子·汤问》篇、《庄子·天下》篇、《说苑·反质》篇皆作‘釐’字,此书《尊师》篇作‘禽滑黎’,《列子·杨朱》篇作‘禽骨釐’,《人表》作‘禽屈釐’,《列子》殷敬顺本亦同。”





功名 (1)


(1) 【校】一作“由道”。

五曰:

由其道,功名之不可得逃, (1) 犹表之与影,若呼之与响。 (2) 善钓者,出鱼乎十仞之下,饵香也; (3) 善弋者,下鸟乎百仞之上,弓良也; (4) 善为君者,蛮夷反舌殊俗异习皆服之,德厚也。 (5) 水泉深则鱼鳖归之,树本盛则飞鸟归之,庶草茂则禽兽归之,人主贤则豪桀归之。 (6) 故圣王不务归之者,而务其所以归。 (7)

(1) 《淮南记》曰“人甘,非正为蹠也,蹠而焉往”,故曰“不可得逃”。

【校】案:《缪称训》曰“人之甘甘,非正为蹠也,而蹠焉往”,彼注云“臣之死君,子之死父,非以求蹠也,而蹠焉往,言蹠乃往至也”,彼后又注云“蹠,愿也”。

(2) 影,晷也。行则晷随之,呼则响应之。推此言之,故功名何可得逃也。

(3) 七尺曰仞。下犹底也。

(4) 弋,缴射之也。《诗》云“弋凫与雁”。下犹陨也。

(5) 东方曰夷,南方曰蛮,其在四表皆为夷也。戎狄言语与中国相反,因谓“反舌”。一说南方有反舌国,舌本在前,末倒向喉,故曰“反舌”。

【校】注“舌本”,旧脱“舌”字,孙据李善注《文选》陆佐公《石阙铭》补。

(6) 才过百人曰豪,千人曰桀。

(7) 务人使归之,末也;而务其所行可归,本也;故曰“务其所以归”也。

强令之笑不乐,强令之哭不悲。 (1) 强令之为道也,可以成小,而不可以成大。 (2) 缶醯黄,蜹聚之,有酸, (3) 徒水则必不可。 (4) 以狸致鼠,以冰致蝇,虽工,不能。 (5) 以茹鱼去蝇,蝇愈至, (6) 不可禁, (7) 以致之之道去之也。 (8) 桀、纣以去之之道致之也, (9) 罚虽重,刑虽严,何益? (10) 大寒既至,民暖是利;大热在上,民清是走。故民无常处,见利之聚,无之去。 (11) 欲为天子,民之所走,不可不察。 (12) 今之世,至寒矣,至热矣,而民无走者,取则行钧也。 (13) 欲为天子,所以示民,不可不异也。 (14) 行不异乱虽信今,民犹无走。 (15) 民无走,则王者废矣, (16) 暴君幸矣,民绝望矣。 (17)

(1) 无其中心,故不乐不悲。

(2) 虚称可以伪制,显实难以诈成,虚小实大也,故曰“不可以成大”也。

(3) 黄,美也。黄故能致酸,酸故能致蜹。

(4) 水无酸,故不可以致蜹也。

(5) 不能致也。

(6) 茹,读茹船漏之茹字。茹,臭也。愈,益也。

【校】案:《易·既济》六四“ 有衣袽”,子夏《易》作“茹”,又通作“帤”,《韵会》引《黄庭经》云:“人间纷纷臭如帤。”

(7) 禁,止也。

【校】孙云:“李善注《文选》左太冲《魏都赋》引‘以茹鱼驱蝇,蝇愈至而不可禁’。”

(8) 致之者茹也,去之不可也。

(9) 去之,残暴也。以致暴之道致治,不治也。

(10) 《淮南记》曰:“急辔利 ,非千里之御也;严刑峻法,非百王之治也。”故曰“何益”。

(11) 处,居也。去,移也。

(12) 察犹知也。

(13) 钧,等也。等于乱暴也。

(14) 若殷纣暴乱,武王以仁义伐之,故曰“不可不异”。

(15) 《传》曰:“以化平化谓之治。”以乱止乱,何治之有?故行不异乱,虽欲信利民,无肯归走也。

(16) 夫民以王者为命,王者以民为本,本无所走,命无所制,而不废者,未之有也。

(17) 无明天子,故暴乱诸侯以为幸也。民无所于救命,故绝望。

故当今之世,有仁人在焉,不可而不此务; (1) 有贤主,不可而不此事。 (2) 贤不肖,不可以不相分, (3) 若命之不可易, (4) 若美恶之不可移。 (5) 桀、纣贵为天子,富有天下,能尽害天下之民,而不能得贤名之。 (6) 关龙逢、王子比干能以要领之死,争其上之过, (7) 而不能与之贤名。 (8) 名固不可以相分,必由其理。 (9)

(1) 务其仁义。

(2) 事其仁义。

(3) 分犹异也。

【校】旧本“异”作“与”,讹,今以上文正之。

(4) 命短不可为使长也。

(5) 尧、舜为美,桀、纣为恶,故曰“不可移”也。

(6) 残义损善曰桀,贱仁多累曰纣。贤主于行,何可虚得。

【校】案:《独断》“残人多垒曰桀,残义损善曰纣”,《史记集解》作“贼人多杀曰桀”,李石《续博物志》又作“残民多舋曰桀”。

(7) 关龙逢,桀忠臣也。王子比干,纣诸父也。争,谏也。桀、纣皆杀之,故曰“能以要领之死,争其上之过”也。

【校】“关龙逢”,如字。李本作“逄”,非。

(8) 不能致桀、纣使享贤名。若后稷好稼,不能使禾自生。

(9) 为善得善名,为恶得恶名,故曰“必由其理”。





第三卷 季春纪



季春


一曰:

季春之月,日在胃, (1) 昏七星中,旦牵牛中。 (2) 其日甲乙,其帝太暤,其神句芒。其虫鳞,其音角,律中姑洗, (3) 其数八。其味酸,其臭膻,其祀户,祭先脾。桐始华,田鼠化为 , (4) 虹始见,萍始生。 (5) 天子居青阳右个, (6) 乘鸾辂,驾苍龙,载青旂,衣青衣,服青玉,食麦与羊,其器疏以达。 (7)

(1) 季春,夏之三月。胃,西方宿,赵之分野。是月,日躔此宿。

【校】案:《淮南·天文训》:“胃,魏之分野。”

(2) 七星,南方宿,周之分野。牵牛,北方宿,越之分野。是月昏旦时,皆中于南方也。

(3) 姑洗,阳律也。姑,故。洗,新。是月阳气发生,去故就新,竹管音中姑洗也。

【校】注“发”,旧本作“养”,讹。《初学记》引作“是月阳气发,故去故就新”,今定作“发”字。其“生”字似不误,仍之。

(4) 桐,梧桐也,是月生叶,故曰“始华”。田鼠,鼸鼠也。 , ,青州谓之 母,周、雒谓之 ,幽州谓之 也。

【校】案:此注多讹脱,《夏小正传》云“田鼠者,嗛鼠也”,《尔雅》作“鼸”,盖颊里藏食之鼠也。注脱“鼸”字,今补。又“ ”,旧讹作“鹊”,“ 母”讹作“ ”。案《小正传》云:“ ,鹌也。”《尔雅》“ , 母”,郭注云“ 也,青州呼 母”;《列子释文》引《夏小正》“田鼠化为 ”,作“ ”;“ ”与“鹌”、“ ”并同,“鹊”以形近而讹,故定为“ ”字。“ ”亦以形近讹“ ”,今据郭注改正。 母读为牟无,《说文》云“ ,牟母也”。

(5) 虹, 也,兖州谓之虹,《诗》曰“ 在东,莫之敢指”是也。萍,水藻,是月始生。

【校】注“虹”,旧讹“订”,谢校改。“萍”,《月今》作“蓱”,郑注:“蓱,萍也。”今《月令》亦作“萍”,误。

(6) 右个,南头室也。

(7) 说在《孟春》。

是月也,天子乃荐鞠衣于先帝, (1) 命舟牧覆舟,五覆五反,乃告舟备具于天子焉。 (2) 天子焉始乘舟。荐鲔于寝庙,乃为麦祈实。 (3)

(1) 《周礼·司服》章曰:“王祀昊天上帝则服大裘而冕,祀五帝亦如之。”又《内司服》章王后之六服有菊衣,衣黄如菊花,故谓之菊衣。春王东方,色皆尚青,此云荐菊衣,诱未达也。

【校】案:《内司服》郑注云:“鞠衣,黄桑服也,色如麴尘,象桑叶始生。”盖后妃服以躬桑者。

(2) 舟牧,主舟官也。是月天子将乘舟始渔,恐有穿漏,反覆视之,五覆五反,慎之至也。

(3) 焉犹于此。自冬至此,于是始乘舟。荐,进也。鲔鱼似鲤而小,《诗》曰“鱣鲔泼泼”,进此鱼于寝庙,祷祈宗祖,求麦实也。前曰庙,后曰寝,《诗》云“寝庙奕奕”,言相连也。

【校】注“泼泼”,《诗》作“发发”。《鲁颂》“路寝孔硕,新庙奕奕”,此引作“寝庙奕奕”,蔡邕《独断》所引亦同。“相连”,旧作“后连”,据《独断》改。《周礼·隶仆》注“奕奕”作“绎绎”,云“相连貌”也。

是月也,生气方盛,阳气发泄,生者毕出 (1) ,萌者尽达,不可以内。 (2) 天子布德行惠,命有司,发仓窌,赐贫穷,振乏绝。 (3) 开府库,出币帛,周天下,勉诸侯, (4) 聘名士,礼贤者。 (5)

(1) 【校】旧校云:“‘生’一作‘牙’。”案“牙”字是,《月令》作“句”。

(2) 发泄犹布散也。象阳达物,亦当散出货贿,不可赋敛以内之。

(3) 方者曰仓,穿地曰窌。无财曰贫,鳏寡孤独曰穷。行而无资曰乏,居而无食曰绝。振,救也。

【校】《月令》“窌”作“廪”。

(4) 府库,币帛之藏也。周,赐。勉,进。

(5) 聘,问之也。有明德之士、大贤之人,聘而礼之,将与兴化致理者也。

【校】注首“聘问之也”四字旧本缺,孙据李善注《文选》枣道彦《杂诗》增入。

是月也,命司空曰:“时雨将降,下水上腾,循行国邑,周视原野, (1) 修利堤防,导达沟渎,开通道路,无有障塞。 (2) 田猎罼弋,罝罘罗网,喂兽之药,无出九门。” (3)

(1) 司空,主土官也。是月下水上腾,恐有浸渍,害伤五稼,故使循行遍视之。广平曰原,郊外曰野。

(2) 障,壅。塞,绝也。

(3) 罼,掩网也。弋,缴射飞也,《诗》云“弋凫与雁”。罝,兔网也,《诗》云“肃肃兔罝”。罗,鸟网也,《诗》云“鸳鸯于飞,罼之罗之”。罘,射鹿罟也。网,其总名也。天子城门十二,东方三门,王气所在处,尚生育,明喂兽之药所不得出也,嫌余三方九门得出,故特戒之如言“无”也。

【校】“罼弋”,《月令》作“毕翳”,注云“翳或作弋”。“九门”,旧本作“国门”,云“一作‘九’”,今案注作“九”为是。注“如言无也”,李本“如”作“加”,谢云:“如,而也。李本不可从。”

是月也,命野虞无伐桑柘。 (1) 鸣鸠拂其羽,戴任降于桑, (2) 具栚曲 筐。 (3) 后妃斋戒,亲东乡躬桑。 (4) 禁妇女无观, (5) 省妇使,劝蚕事。 (6) 蚕事既登, (7) 分茧称丝效功, (8) 以共郊庙之服,无有敢堕。 (9)

(1) 野虞,主材官。桑与柘皆可以养蚕,故命其官使禁民不得斫伐。

(2) 鸣鸠,班鸠也。是月拂击其羽,直刺上飞数十丈乃复者是也。戴任,戴胜,鸱也。《尔雅》曰“ 鸠”,部生于桑。是月其子强飞,从桑空中来下,故曰“戴任降于桑”也。

【校】“戴任”,《月令》作“戴胜”,《淮南》作“戴 ”,注不当训鸱。但旧本《月令正义》引《尔雅》亦作“鸱鸠”,此作“ 鸠”,究属“ ”二字之误。“部生于桑”云云,不知所出。

(3) 栚,读曰朕。栚, 也。三辅谓之栚,关东谓之 。曲,薄也。青、徐谓之曲。圆底曰 ,方底曰筐,皆受桑器也。是月立夏,蚕生,故敕具也。

【校】《月令》作“具曲植籧筐”,《淮南》作“具扑曲莒筐”,此书旧本作“具挟曲蒙筐”。“挟”与“扑”皆“栚”之讹文也。《说文》云:“栚,槌之横者也。”槌即植也。《方言》:“槌,宋、魏、陈、楚、江、淮之间谓之植,自关而西谓之槌,齐谓之样。其横,关西曰 ,宋、魏、陈、楚、江、淮之间谓之 ,齐部谓之 。”今据此并注皆改正。“栚”从“朕”省,《方言》不省作“ ”。注“栚, 也”,旧本脱,今从《淮南注》补,则下文方有所承。 ,丁革反。旧本作“关东谓之得”,讹。“曲”,《说文》作“ ”,云“蚕薄也”,《广雅》又从“竹”作“筁”。段云:“‘蒙’乃‘ ’字之误,即《记》之‘籧’也,亦即‘筥’也。”今依改正。案郭璞注《方言》云:“‘ ’,古‘筥’字。”

(4) 王者一后三夫人。妃即夫人,与后参职,配王兼众事。王者亲耕,故后妃亲桑也,以为天下先,劝众民也。

(5) 观,游。

(6) 省其他使,劝其趋蚕事。

(7) 登,成也。

(8) 效,致也。丝多为上功。

(9) 郊祭天,庙祭祖。《周礼·内宰 [1] 》章“仲春,诏后率内外命妇蚕于北郊,以为祭服”,此之谓也。

【校】“堕”,《月令》作“惰”,同。

是月也,命工师令百工审五库之量,金铁、 (1) 皮革筋、 (2) 角齿、 (3) 羽箭干、 (4) 脂胶丹漆, (5) 无或不良。 (6) 百工咸理,监工日号,无悖于时, (7) 无或作为淫巧,以荡上心。 (8) 是月之末,择吉日大合乐, (9) 天子乃率三公、九卿、诸侯、大夫亲往视之。 (10)

(1) 句。

(2) 句。

(3) 句。

(4) 句。

(5) 句。

(6) 良,善。

(7) 监工,工官之长。悖,逆也。时可用作器,无逆之也。不作为逆也。

(8) 淫巧,非常诡怪。若宋人以玉为楮叶,三年而成,乱之楮叶之中不可别知之类也,故曰“以荡上心”。荡,动也。

【校】注旧本“诡”上衍“说”字,今删。

(9) 乐以和民,故择于是月下旬吉日,大合六乐,八音克谐,《箫韶》九成。《周礼·大司 [2] 乐》章“以乐舞教国子,舞《云门大卷》、《大咸》、《大韶》、《大夏》、《大护》、《大武》,大合乐以和邦国,以谐万民,以安宾客,以悦远人”,此之谓也。

(10) 视其乐也。

是月也,乃合纍牛、腾马游牝于牧, (1) 牺牲驹犊,举书其数。 (2) 国人傩,九门磔禳,以毕春气。 (3) 行之是令,而甘雨至三旬。 (4) 季春行冬令,则寒气时发,草木皆肃,国有大恐。 (5) 行夏令,则民多疾疫,时雨不降,山陵不收。 (6) 行秋令,则天多沉阴,淫雨早降,兵革并起。 (7)

(1) 纍,读如《诗》“葛纍”之纍。纍牛,父牛也。腾马,父马也。皆将群游从牝于牧之野,风合之。

【校】“纍”,《月令》作“累”,《淮南》作“ ”,《淮南》注“读葛藟之藟”。

(2) 举其犊驹在牺牲者,皆簿领书其头数也。

(3) 傩,读《论语》“乡人傩”同。命国人傩,索宫中区隅幽暗之处,击鼓大呼,驱逐不祥,如今之正岁逐除是也。九门,三方九门也。嫌非王气所在,故磔犬羊以禳木气尽之,故曰“以毕春气”也。

【校】“国人傩”,《月令》作“命国傩”,《淮南》作“令国难”。此疑倒误。“傩”疑本作“难”,故注读从《论语》之傩。“同”字疑后人所增。“区隅”亦作“沤隅”,又一作“欧隅”。

(4) 行之是令,行是之令也。十日曰旬。

【校】《月令》无此句,《淮南》有,下同。

(5) 行冬寒杀气之令,故寒气早发,草本肃棘,木不曲直也。气不和,故国大惶恐也。

【校】注“行冬”下旧本有“令”字,衍,今删。

(6) 行夏炎阳之令,火干木,故民疾疫,雨泽不降,故山陵所殖不收入。

(7) 秋阴气用事,水之母也,而行其令,故多沉阴为淫雨也。阴为兵器,故并起。





尽数


二曰:

天生阴阳寒暑燥湿,四时之化,万物之变,莫不为利,莫不为害。 (1) 圣人察阴阳之宜,辨万物之利以便生,故精神安乎形,而年寿得长焉。 (2) 长也者,非短而续之也,毕其数也。 (3) 毕数之务,在乎去害。何谓去害?大甘、大酸、大苦、大辛、大咸,五者充形则生害矣。大喜、大怒、大忧、大恐、大哀,五者接神则生害矣。大寒、大热、大燥、大湿、大风、大霖、大雾,七者动精则生害矣。 (4) 故凡养生,莫若知本,知本则疾无由至矣。 (5)

(1) 顺者利时,逆者害时。

(2) 精神内守,无所贪欲,故形性安。形性安则寿命长也。

(3) 毕,尽也。平其无欲之情,不夭陨,故尽其长久之数。

(4) 诸言大者,皆过制也。

(5) 《传》曰“人受天地之中以生”,所谓命也。《孟子》曰:“人性无不善。”本其善性,闭塞利欲,疾无由至矣。

精气之集也,必有入也。集于羽鸟,与为飞扬; (1) 集于走兽,与为流行;集于珠玉,与为精朗;集于树木,与为茂长;集于圣人,与为敻明。 (2) 精气之来也,因轻而扬之,因走而行之,因美而良之,因长而养之,因智而明之。 (3)

(1) 【校】旧校云:“‘养’一作‘翔’。”

(2) 集,皆成也。敻,大也,远也。敻,读如《诗》云“于嗟敻兮”。

【校】此《韩诗》。

【校】旧校云:“‘养’一作‘善’。”案此段用韵,“善”字非也。

(3) 因,依也。明,智也。

流水不腐, (1) 户枢不蝼, (2) 动也。形气亦然。形不动则精不流,精不流则气郁。郁处头则为肿、为风, (3) 处耳则为挶、为聋, (4) 处目则为 、为盲, (5) 处鼻则为鼽、为窒, (6) 处腹则为张、为疛, (7) 处足则为痿、为蹷。 (8)

(1) 腐,臭败也。

(2) 【校】《意林》作“不蠹”。

(3) 肿与风,皆首疾。

(4) 皆耳疾也。

(5) ,眵也;盲,无见。皆目疾也。

【校】孙云:“李善注《文选》宋玉《风赋》引‘ ’作‘蔑’,‘高诱曰:蔑,眵’。”此注旧本皆作“ 肝 ”,误,今从彼注改正。“善又云:‘蔑与 古字通,亡结切;眵,亡支切。’”

(6) 鼽,齆鼻。窒,不通。

(7) 疛,跳动。皆腹疾。

【校】“疛”,旧本作“府”,误也。《说文》:“疛,小腹疾。”此云“跳动”者,《诗·小雅·小弁》云“惄焉如 ”,《释文》云“本或作‘ ’,《韩诗》作‘疛’,除又反,义同”,此所训正合。

(8) 痿,不能行。蹷,逆疾也。

轻水所,多秃与瘿人; (1) 重水所,多尰与躄人; (2) 甘水所,多好与美人; (3) 辛水所,多疽与痤人; (4) 苦水所,多尪与伛人。 (5)

(1) 秃,无发。瘿,咽疾。

【校】所,即处。下放此。

(2) 肿足曰尰。躄,不能行也。

(3) 美,亦好也。

(4) 疽、痤,皆恶疮也。

(5) 尪,突胸仰向疾也。伛,伛脊疾也。

凡食,无强厚味,无以烈味 (1) 重酒, (2) 是以谓之疾首。 (3) 食能以时,身必无灾。 (4) 凡食之道,无饥无饱,是之谓五藏之葆。 (5) 口必甘味,和精端容,将之以神气。 (6) 百节虞欢,咸进受气。饮必小咽,端直无戾。

(1) 烈,犹酷也。

(2) 重,酒厚也。

(3) 疾首,头痛疾也。

【校】疾首,犹言致疾之端,注非是。

(4) 时,节也。不过差,故身无灾疾也。

(5) 葆,安也。

(6) 端,正。将,养。

今世上卜筮祷祠,故疾病愈来。譬之若射者,射而不中,反修于招,何益于中? (1) 夫以汤止沸,沸愈不止,去其火则止矣。故巫医毒药,逐除治之,故古之人贱之也,为其末也。 (2)

(1) 于招,埻艺也。患射不能中,不知循彀精艺,而反修其标的,故曰“何益于中”也。

【校】旧校云:“‘修’一作‘循’,‘招’一作‘的’。”注“埻”,旧误作“ ”。梁仲子云:“《本生》篇注云‘招,埻的也’,《外传越语》韦注云‘艺,射的也’;‘于招’盖连文。”

(2) 古之人治正性,保天命者也。不然,则邪气乘之以疾病,使巫医毒药除逐治之,故谓贱之也。若止沸以汤,不去其火,故曰“为其末也”。





先己


三曰:

汤问于伊尹曰:“欲取天下,若何?” (1) 伊尹对曰:“欲取天下,天下不可取;可取,身将先取。” (2) 凡事之本,必先治身,啬其大宝。 (3) 用其新,弃其陈,腠理遂通。 (4) 精气日新,邪气尽去,及其天年。 (5) 此之谓真人。 (6)

(1) 汤为诸侯时也。

(2) 言不可取天下,身将先为天下所取也。

(3) 啬,爱也。大宝,身也。

【校】旧校云:“‘治’一作‘取’。”

(4) 用药物之新,弃去其陈以疗疾,则腠理肌脉遂通利不闭也。

【校】赵云:“注非也,此即《庄子》所云‘吐故纳新’也。”梁仲子云:“《淮南·泰族训》‘呼而出故,吸而入新’,亦相似。”

(5) 【校】孙云:“《御览》七百二十‘及’作‘反’。”

(6) 真德之人。

昔者先圣王成其身而天下成, (1) 治其身而天下治。 (2) 故善响者不于响于声, (3) 善影者不于影于形, (4) 为天下者不于天下于身。 (5) 《诗》曰:“淑人君子,其仪不忒。其仪不忒,正是四国。” (6) 言正诸身也。故反其道而身善矣, (7) 行义则人善矣, (8) 乐备君道而百官已治矣, (9) 万民已利矣。 (10) 三者之成也,在于无为。无为之道曰胜天, (11) 义曰利身, (12) 君曰勿声。 (13) 勿身督听, (14) 利身平静, (15) 胜天顺性。 (16) 顺性则聪明寿长, (17) 平静则业进乐乡, (18) 督听则奸塞不皇。 (19) 故上失其道,则边侵于敌; (20) 内失其行,名声堕于外。 (21) 是故百仞之松,本伤于下,而末槁于上; (22) 商、周之国,谋失于胸, (23) 令困于彼。 (24) 故心得而听得, (25) 听得而事得,事得而功名得。 (26) 五帝先道而后德, (27) 故德莫盛焉; (28) 三王先教而后杀, (29) 故事莫功焉; (30) 五伯先事而后兵, (31) 故兵莫强焉。 (32) 当今之世,巧谋并行,诈术递用, (33) 攻战不休,亡国辱主愈众, (34) 所事者末也。 (35)

(1) 王道成也。

(2) 詹何曰“未闻身治而国乱,身乱而国治者”,此之谓也。

(3) 声善则响善也。

(4) 形正则影正。

(5) 身正则天下治。

(6) 忒,差也。

(7) 体道无欲,故身善。

(8) 行仁义于所宜,则人善之矣。

(9) 乐服行君人无为之道,则百官承使化职事也。

【校】注当云“则百官承化,职事已治也”,旧本有脱误。

(10) 君无为则万民安利。

(11) 天无为而化,君能无为而治民,以为胜于天。

(12) 能行仁义,则可以利其身。

(13) 为君之道,务在利民;勿自利身,故曰“勿身”。

(14) 督,正也。正听,不倾听也。

【校】旧本作“倾不听也”,讹,今乙正。

(15) 行仁义,故能平静也。

(16) 无为而不欲,故能顺性也。

(17) 顺法天性,则聪明也。《虞书》云“天聪明,自我民聪明”,此之谓也。法天无为,故寿长久也。

(18) 行仁义则民业进而乐乡其化。

(19) 正听万法,赏罚分明,故奸轨塞断于不皇。皇,暇也。

(20) 君无道则敌国侵削其边,俘其民也。《论语》曰“上失其道,民散久矣”,此之谓也。

(21) 内失抚民之行则邻国贱之,故曰“名声堕于外”也。若晋惠公背外内之赂,杀李克之党,内无忠臣之辅,外无诸侯之助,与秦穆公战而败亡。

【校】赵云:“内失其行,不能反道以善身,故名声堕于外也。‘李克’,《内》《外传》作‘里克’,古‘李’、‘里’通用。”

(22) 本,根也。君亦国之本。

(23) 商、周二王之季也。胸犹内。

(24) 彼亦外也。

(25) 得犹知也。

(26) 事事必得之则功成名立,故功名得也。

(27) 五帝:黄帝、高阳、高辛、尧、舜。先犹尚也。

(28) 德之大者,无出于五帝。

(29) 三王,夏、商、周也。

(30) 成王事之功,无过于三王。

【校】孙云:“《御览》七十七作‘三王先德而后事,故功莫大焉’。”

(31) 五伯:昆吾、大彭、豕韦、齐桓、晋文。

(32) 兵之强者,无强于五伯者也。

(33) 递,代。

(34) 愈,益。众,多。

(35) 事,治。

夏后伯启与有扈战于甘泽而不胜, (1) 六卿请复之。 (2) 夏后伯启曰:“不可。吾地不浅, (3) 吾民不寡,战而不胜,是吾德薄而教不善也。”于是乎处不重席,食不贰味,琴瑟不张, (4) 钟鼓不修, (5) 子女不饬, (6) 亲亲长长, (7) 尊贤使能。期年而有扈氏服。 (8) 故欲胜人者,必先自胜;欲论人者,必先自论; (9) 欲知人者,必先自知。 (10) 《诗》曰:“执辔如组。” (11) 孔子曰:“审此言也,可以为天下。” (12) 子贡曰:“何其躁也!”孔子曰:“非谓其躁也,谓其为之于此,而成文于彼也。”圣人组修其身,而成文于天下矣。故子华子曰:“丘陵成而穴者安矣, (13) 大水深渊成而鱼鳖安矣, (14) 松柏成而涂之人已荫矣。” (15)

(1) 有扈,夏同姓诸侯。《传》曰“启伐有扈”,《书》曰“大战于甘,乃召六卿。王曰:‘六事之人,予誓告汝。有扈氏威侮五行,怠弃三正,天用剿绝其命,今予惟龚行天之罚。’”此之谓也。

【校】“夏后伯启”,旧本作“夏后相”。孙云:“如果为相,注不应但据启事为证,考《御览》八十二帝启事中引此作‘夏后伯启’,乃知今本误也。然《困学纪闻》亦引作‘夏后相’,则南宋时本已误矣。”卢云:“案‘伯’,古多作‘柏’,后人疑为‘相’,因并误删‘启’字。”

(2) 请复战也。

(3) 浅,褊。

(4) 张,施。

(5) 修,设。

(6) 不文饬也。

【校】“饬”与“饰”通,《御览》二百七十九作“饰”。

(7) 长长,敬长。

(8) 服,从。

(9) 《传》曰“惟无瑕者可以戮人”,亦由无阙者可以论人,身有阙而论人,是为自论也。

【校】赵云:“必先自论,与上自胜、下自知一例,注并非。”

(10) 知人则哲,惟帝其难之,故当先自知而后求知人也。

(11) 组,读组织之组。夫组织之匠,成文于手,犹良御执辔于手而调马足,以致万里也。

【校】注“足以”,旧本作“口以”,讹。

(12) 审,实也。为,治也。

(13) 穴而处之。

(14) 沉而居之。

(15) 成,茂。

孔子见鲁哀公, (1) 哀公曰:“有语寡人曰:‘为国家者,为之堂上而已矣。’ (2) 寡人以为迂言也。” (3) 孔子曰:“此非迂言也。丘闻之:得之于身者得之人,失之于身者失之人。 (4) 不出于门户而天下治者,其惟知反于己身者乎!” (5)

(1) 哀公,定公宋之子蒋也。

(2) 夫人皆治堂以行礼,治国亦当以礼,故曰“为之堂上而已矣”。

【校】《说苑·政理》篇、《家语·贤君》篇俱作“卫灵公问”。

(3) 迂,远。

(4) 《论语》曰“君子求诸己”,故曰得之身者得诸人,失之身则失之人也。

(5) 反者大也。





论人


四曰:

主道约,君守近。 (1) 太上反诸己,其次求诸人。其索之弥远者,其推之弥疏; (2) 其求之弥强者,失之弥远。

(1) 近者,守之于身也。

(2) 索,求。弥,益也。

【校】注“求”下旧衍“之”字。

何谓反诸己也?适耳目,节嗜欲,释智谋,去巧故, (1) 而游意乎无穷之次, (2) 事心乎自然之涂, (3) 若此则无以害其天矣。 (4) 无以害其天则知精, (5) 知精则知神,知神之谓得一。 (6) 凡彼万形,得一后成。 (7) 故知一,则应物变化,阔大渊深,不可测也; (8) 德行昭美,比于日月,不可息也; (9) 豪士时之,远方来宾,不可塞也; (10) 意气宣通,无所束缚,不可收也。 (11) 故知知一,则复归于朴。 (12) 嗜欲易足,取养节薄,不可得也; (13) 离世自乐,中情洁白,不可量也; (14) 威不能惧,严不能恐,不可服也。 (15) 故知知一,则可动作当务,与时周旋,不可极也。 (16) 举错以数,取与遵理,不可惑也; (17) 言无遗者,集肌肤,不可革也; (18) 谗人困穷,贤者遂兴,不可匿也。 (19) 故知知一,则若天地然,则何事之不胜, (20) 何物之不应? (21) 譬之若御者,反诸己,则车轻马利,致远复食而不倦。 (22) 昔上世之亡主,以罪为在人,故日杀僇而不止,以至于亡而不悟。 (23) 三代之兴王,以罪为在己,故日功而不衰,以至于王。 (24)

(1) 释亦去也。巧故,伪诈也。

(2) 次,舍。

(3) 事,治也。自然,无为。涂,道也。

(4) 天,身。

(5) 精,明微。

(6) 一,道也。

(7) 天道生万物,万物得一乃后成也。

(8) 测,尽极也。

(9) 息,灭也。

(10) 塞,遏也。

(11) 收,守。

【校】“收”,疑当作“牧”,与韵叶,牧亦训守。

(12) 朴,本也。

(13) 不可得使多欲,厚自养也。一曰:若此人者不可得。

(14) 离世,不群。量,行也。

【校】“量”字亦疑误。

(15) 不可无威得威力服。

【校】注“不可”二字疑衍,盖言无威而使威力皆服也。

(16) 极,穷。

(17) 惑,眩。

(18) 遗,失也。《孝经》曰“言满天下无口过”,此之谓也。革,更也。

【校】正文有脱字。

(19) 匿犹伏也。

【校】注“伏”,旧讹“任”,今改正。

(20) 胜犹任也。

(21) 应,当也。

(22) 倦,罢。

【校】复食二字未详。

(23) 亡主,若桀、纣者也。以罪为在他人,故多杀僇,是灭亡之道也,而不自觉知也。

(24) 三代,禹、汤、文王也。日行其人民之功不衰倦,以至于王有天下也。

何谓求诸人?人同类而智殊, (1) 贤不肖异,皆巧言辩辞以自防御, (2) 此不肖主之所以乱也。 (3) 凡论人,通则观其所礼, (4) 贵则观其所进, (5) 富则观其所养,听则观其所行, (6) 止则观其所好,习则观其所言, (7) 穷则观其所不受,贱则观其所不为。 (8) 喜之以验其守, (9) 乐之以验其僻, (10) 怒之以验其节, (11) 惧之以验其特, (12) 哀之以验其人, (13) 苦之以验其志。 (14) 八观六验,此贤主之所以论人也。 (15) 论人者又必以六戚四隐。 (16) 何谓六戚?父、母、兄、弟、妻、子。何谓四隐?交友、故旧、邑里、门郭。内则用六戚四隐,外则用八观六验,人之情伪贪鄙美恶,无所失矣, (17) 譬之若逃雨污,无之而非是, (18) 此先圣王之所以知人也。

(1) 殊,异。

(2) 防御仇也。

【校】注疑有误。

(3) 乱,惑。

【校】“主”旧作“王”,案下有“贤主”,则此当作“不肖主”明矣,今改正。

(4) 通,达也。《孟子》曰“达则兼善天下”,故观其所宾礼。

(5) 进,荐也。尧荐舜,舜荐禹。《传》曰:“善进善,不善蔑由至矣;不善进不善,善亦蔑由至矣。”故曰“观其所进”也。

(6) 养则养贤也,行则行仁也,故观之也。

【校】听,谓听言也。

(7) 好则好义,言则言道。

(8) 不受非其类也。不为谄谀。

【校】不受非分之财,不为非义之事。

(9) 守,清守也。

(10) 僻,邪。

(11) 节,性。

(12) 特,独也。虽独不恐。

(13) 人人可哀,不忍之也。

(14) 钻坚攻难,不成不止,故曰“以验其志”也。

(15) 论犹论量也。

(16) 六戚,六亲也。四隐,相隐而扬长蔽短也。

【校】注“短”字旧阙,今案文义补。

(17) 言尽知之。

(18) 皆是雨也。





圜道


五曰:

天道圜,地道方,圣王法之,所以立上下。 (1) 何以说天道之圜也?精气一上一下,圜周复杂,无所稽留,故曰天道圜。 (2) 何以说地道之方也?万物殊类殊形,皆有分职,不能相为,故曰地道方。 (3) 主执圜,臣处方,方圜不易,其国乃昌。日夜一周,圜道也。 (4) 月躔二十八宿,轸与角属,圜道也。 (5) 精行四时,一上一下,各与遇,圜道也。 (6) 物动则萌,萌而生,生而长,长而大,大而成,成乃衰,衰乃杀,杀乃藏,圜道也。 (7) 云气西行,云云然, (8) 冬夏不辍; (9) 水泉东流,日夜不休; (10) 上不竭,下不满; (11) 小为大,重为轻;圜道也。 (12) 黄帝曰:“帝无常处也, (13) 有处者乃无处也。” (14) 以言不刑蹇,圜道也。 (15) 人之窍九,一有所居则八虚, (16) 八虚甚久则身毙。 (17) 故唯而听,唯止; (18) 听而视,听止。 (19) 以言说一。 (20) 一不欲留,留运为败, (21) 圜道也。一也齐至贵, (22) 莫知其原,莫知其端,莫知其始,莫知其终,而万物以为宗。 (23) 圣王法之,以令其性,以定其正, (24) 以出号令。令出于主口,官职受而行之, (25) 日夜不休,宣通下究, (26) 瀸于民心,遂于四方, (27) 还周复归,至于主所,圜道也。令圜则可不可,善不善,无所壅矣。 (28) 无所壅者,主道通也。 (29) 故令者,人主之所以为命也,贤不肖安危之所定也。 (30)

(1) 上,君。下,臣。

(2) 杂犹匝。无所稽留,运不止也。

【校】《御览》二及十五俱作“圜通周复无杂”,此出后人所附益,不可信也。

(3) 不能相为,不能相兼。

(4) 圜,天道也。

(5) 躔,舍也。轸,南方鹑尾。角,东方苍龙。行度所经也。

【校】赵云:“二十八宿,始角终轸,轸角相接,注不分晓。”

(6) 精,日月之光明也。

(7) 藏,潜也。

(8) 云,运也。周旋运布,肤寸而合,西行则雨也。

【校】注云“运也”,旧本作“游也”,误,今改正。

(9) 辍,止也。

(10) 休,息也。

(11) 水从上流而东,不竭尽也。下至海,受而不满溢也。

(12) 小者泉之源也,流不止也,集于海,是为大也。水湿而重,升作为云,是为轻也。

(13) 无常处,言无为而化,乃有处也。

(14) 有处,有为也。有为则不能化,乃无处为也。

(15) 刑,法也。言无刑法,故蹇难也。天道正刑不法,故曰“圜道也”。

(16) 居,读曰居处之居,居犹壅闭也。

(17) 虚,病。毙,死。

(18) 听则唯止矣。

(19) 视则听止矣。

(20) 一,道本。

(21) 留,滞。

(22) 道无匹敌,故曰“至贵”也。

【校】孙云:“李善注《文选》江文通《拟孙廷尉诗》引作‘一也者,至贵者’也。”

(23) 道无形,其原始终极莫能知之。道生万物,以为宗本。

(24) 【校】旧校云:“‘令’一作‘全’,‘正’一作‘生’。”

(25) 官职,职官之长。

【校】注似当作“官职,百官之职”。

(26) 宣,遍布也。

(27) 瀸,洽。遂,达。

【校】注旧本作“遂远”,讹,今改正。

(28) 不可者能令之可,不善者能令之善,化使然也。皆通之,故曰“无所壅”。

(29) 言纳忠受谏,臣情上达,无所壅蔽,是为君之道通也。

(30) 君者法天,天无私,故所以为命也。赋命各得其中,安与危无怨憾,故曰“定”也。

【校】正文“安”下旧本衍“之”字,今删。

人之有形体四枝,其能使之也,为其感而必知也。 (1) 感而不知,则形体四枝不使矣。 (2) 人臣亦然。号令不感,则不得而使矣。 (3) 有之而不使,不若无有。 (4) 主也者,使非有者也, (5) 舜、禹、汤、武皆然。先王之立高官也,必使之方, (6) 方则分定,分定则下不相隐。 (7) 尧、舜,贤主也,皆以贤者为后,不肯与其子孙,犹若立官必使之方。 (8) 今世之人主,皆欲世勿失矣, (9) 而与其子孙,立官不能使之方,以私欲乱之也,何哉?其所欲者之远,而所知者之近也。 (10) 今五音之无不应也,其分审也。 (11) 宫、徵、商、羽、角,各处其处,音皆调均,不可以相违,此所以无不受也。 (12) 贤主之立官,有似于此。百官各处其职,治其事以待主,主无不安矣;以此治国,国无不利矣;以此备患,患无由至矣。 (13)

(1) 感者,痛恙也。手足必知其处所,故使之也。

(2) 不能相使,则形体疾也。

(3) 不可得而使则国乱。

(4) 不若无臣。

(5) 汤使桀臣,武王使纣臣,皆非其有也。

(6) 方,正。

(7) 隐,私也。君臣上下无私邪相壅蔽之。

(8) 以贤者为后,谓禅位也。尧传舜,舜传禹,故曰“不肯与其子孙”也。方,正,不私邪之谓也。

(9) 父死子继曰世。

(10) 自传子孙,冀世世不失,是其所欲者之远也。子孙不肖,骄淫暴虐,必见改置,不得长久,是所知者之近也。

(11) 各守其声,集以成和,故曰“其分审”。

(12) 受亦应也。

【校】旧本脱“无”字,则义相反,今依上文补之。注“也”字旧作“之”,亦改正。

(13) 【校】“患”字,本亦有不叠者,今从许本、汪本。




————————————————————

[1] 宰:原本作“子”,据《周礼》改。

[2] “司”上原衍“胥”字,今删。





第四卷 孟夏纪



孟夏


一曰:

孟夏之月,日在毕, (1) 昏翼中,旦婺女中。 (2) 其日丙丁,其帝炎帝, (3) 其神祝融。 (4) 其虫羽,其音徵, (5) 律中仲吕,其数七。 (6) 其性礼,其事视, (7) 其味苦,其臭焦, (8) 其祀灶,祭先肺。 (9) 蝼蝈鸣,丘蚓出, (10) 王菩生,苦菜秀。 (11) 天子居明堂左个, (12) 乘朱辂,驾赤 , (13) 载赤旂,衣赤衣,服赤玉, (14) 食菽与鸡,其器高以觕。 (15)

(1) 孟夏,夏之四月也。毕,西方宿,秦之分野。是月,日躔此宿也。

【校】案:《淮南·天文训》“毕,魏之分野”,与此注不同。

(2) 翼,南方宿,楚之分野。婺女,北方宿,越之分野。是月昏旦时,皆中于南方。

【校】案:注“婺女北方宿”,旧作“南方”,讹。《淮南》作“须女吴”,此与《季冬纪》注皆云越,不同。

(3) 丙丁,火日也。炎帝,少典之子,姓姜氏,以火德王天下,是为炎帝,号曰神农,死托祀于南方,为火德之帝。

(4) 祝融,颛顼氏后,老童之子吴回也,为高辛氏火正,死为火官之神。

(5) 盛阳用事,鳞散而羽,故曰“其虫羽”。羽虫,凤为之长。徵,火也,位在南方。

(6) 仲吕,阴律也。阳散在外,阴实在中,所以旅阳成功也,故曰“仲吕”。五行数五,火第二,故曰“七”。

【校】旧本“在中”作“其中”,“旅阳”作“类阳”,“成功”二字脱在下,作“其数成功五”,梁仲子据《初学记》所引改正。“五行数五”,亦据前后文改。

(7) 【校】《月令》无此二句,此书前后亦无此例,当为衍文。

(8) 火味苦。火臭焦。

(9) 吴囘,囘禄之神,托于灶。是月火王,故祀之也。肺,金也。祭礼之先进肺,用其胜也。一曰:肺,火,自用其藏。

【校】注“吴囘”,旧作“吴国”,讹,今改正。

(10) 蝼蝈,虾蟆也。是月阴气动于下,故阴类鸣,丘蚓从土中出。

【校】注“丘蚓”下旧本有“虾蟆”二字,乃衍文,今删。

(11) “菩”或作“瓜”, 也,是月乃生。《尔雅》云:“不荣而实曰秀,荣而不实曰英。”苦菜当言英者也。

【校】“王菩”,旧本并注皆讹作“王善”。案《月令》“王瓜生”,注云“今《月令》云‘王 生’”,此书必本作“菩”,古“菩”、“ ”通用,郭璞注《穆天子传》“茅 ”云:“‘ ’,今‘菩’字,音倍。”《集韵》“音蓓,与‘ ’通”。此书刘本疑“王善”误,径依《月令》作“王瓜生”,并改注云“王瓜即今栝楼也”,大违阙疑之义。

(12) 明堂,南乡堂。左个,东头室。

(13) 顺火德也。骍马黑尾曰 。

(14) 皆赤,顺火也。

(15) 菽,豆也。觕,大也。器高大以象火性。

是月也,以立夏。 (1) 先立夏三日,太史谒之天子曰:“某日立夏,盛德在火。” (2) 天子乃斋。 (3) 立夏之日,天子亲率三公、九卿、大夫以迎夏于南郊。 (4) 还,乃行赏,封侯庆赐,无不欣说。 (5) 乃命乐师,习合礼乐。 (6) 命太尉,赞杰俊,遂贤良,举长大, (7) 行爵出禄,必当其位。 (8)

(1) 春分后四十六日立夏。立夏多在是月。

(2) 太史,说在《孟春》。以盛德在火,火王南方也。

(3) 说在《孟春》。

(4) 南郊,七里之郊。

(5) 还,从南郊还也。封侯,命以茅土。《传》曰“赏以春夏,刑以秋冬”,此之谓也。无不欣说,咸赖其所赐。

(6) 礼,所以经国家,定社稷,利人民;乐,所以移风易俗,荡人之邪,存人之正性;故命乐师使习合之。

(7) 命,使。赞,白也。千人为俊,万人为杰。遂,达也。有贤良长大之人,皆当白达举用之,故齐桓公命于子之乡,有孝于父母,聪慧质仁秀出于众者,则以告,有不以告,谓之蔽贤而罪之,此之谓也。

【校】注“白达”,旧讹作“自达”,又“于子之乡”作“于天子之乡”,“聪慧质仁”作“聪慧质直仁”,《齐语》无“天”字、“质”字,今皆删正。

(8) 当,直也。

是月也,继长增高,无有坏隳。 (1) 无起土功,无发大众,无伐大树。 (2)

(1) 象阳长养物也。

【校】“隳”,《月令》作“堕”,《释文》云:“又作‘隳’。”

(2) 所以顺阳气。

是月也,天子始 。 (1) 命野虞出行田原,劳农劝民,无或失时; (2) 命司徒循行县鄙, (3) 命农勉作,无伏于都。 (4)

(1) ,细葛也。《论语》曰“当暑袗 绤”,此之谓也。

(2) 劳,勉。劝,教。使民不失其时。

【校】《月令》“劳农”上有“为天子”三字。

(3) 县,畿内之县。县,二千五百家也。鄙,五百家也。司徒主民,故使循行。

(4) 伏,藏。都,国。

【校】《月令》“伏”作“休”。

是月也,驱兽无害五谷,无大田猎, (1) 农乃升麦。 (2) 天子乃以彘尝麦,先荐寝庙。 (3)

(1) 为夭物也。

(2) 升,献。

【校】《月令》作“农乃登麦”。升犹登也。旧本作“收”,今据注定作“升”。

(3) 麦始熟,故言尝。彘,水畜,夏所宜食也。先寝庙,孝之至。

是月也,聚蓄百药,糜草死。 (1) 麦秋至,断薄刑,决小罪,出轻系。 (2) 蚕事既毕,后妃献茧,乃收茧税,以桑为均, (3) 贵贱少长如一,以给郊庙之祭服。

(1) 是月阳气极,药草成,故聚积之也。糜草,荠、亭历之类。

【校】“糜”,《月令》作“靡”。

(2) 是月阳气盛于上,及五月阴气伏于下,故断薄刑、决小罪,顺杀气也。轻系,不及于刑者解出之。

(3) 均,平也。桑多税多,桑少税少。

是月也,天子饮酎,用礼乐。 (1) 行之是令,而甘雨至三旬。 (2) 孟夏行秋令,则苦雨数来,五谷不滋,四鄙入保。 (3) 行冬令,则草木早枯,后乃大水,败其城郭。 (4) 行春令,则虫蝗为败,暴风来格,秀草不实。 (5)

(1) 酎,春酝也。是月天子乃与群臣饮酒作乐。《诗》云:“为此春酒,以介眉寿。”

(2) 行之是令,行此之令也。旬,十日也。十日一雨,三旬三雨也。

(3) 孟夏盛阳而行金气杀戮之令,水生于金,故苦雨杀谷不滋茂也。四境之民,畏寇贼来,入城郭以自保守也。

(4) 行冬寒固闭之令,故草木早枯,大水坏其城郭,奸时逆行之徵也。

(5) 是月当继长增高,助阳长养,而行春启蛰之令,故有虫蝗之败。春木气,多风,故暴疾之风应气而至,使当秀之草不长茂。





劝学 (1)


(1) 【校】一作“观师”。

二曰:

先王之教,莫荣于孝,莫显于忠。忠孝,人君人亲之所甚欲也;显荣,人子人臣之所甚愿也。然而人君人亲不得其所欲,人子人臣不得其所愿,此生于不知理义。 (1) 不知义理,生于不学。 (2)

(1) 不知理义,在君父则不仁不慈,在臣子则不忠不孝。不忠不孝,故君父不得其所欲也。不仁不慈,故臣子不得其所愿也。

(2) 生犹出。

【校】“义理”,亦当同上文作“理义”。

学者师达而有材,吾未知其不为圣人。 (1) 圣人之所在,则天下理焉。 (2) 在右则右重,在左则左重, (3) 是故古之圣王未有不尊师者也。尊师则不论其贵贱贫富矣。 (4) 若此则名号显矣,德行彰矣。故师之教也,不争轻重尊卑贫富, (5) 而争于道。其人苟可,其事无不可。 (6) 所求尽得,所欲尽成,此生于得圣人。圣人生于疾学。 (7) 不疾学而能为魁士名人者,未之尝有也。 (8) 疾学在于尊师。师尊则言信矣,道论矣。 (9) 故往教者不化,召师者不化; (10) 自卑者不听, (11) 卑师者不听。 (12) 师操不化不听之术而以强教之,欲道之行、身之尊也,不亦远乎? (13) 学者处不化不听之势而以自行,欲名之显、身之安也,是怀腐而欲香也,是入水而恶濡也。 (14)

(1) 学者师道通达其义,而有材秀,言圣人之言,行圣人之行,是则圣人矣,故曰“吾未知其不为圣人”也。

(2) 理,治。

(3) 重,尊也。德大行可顺移也。

(4) 言道重人轻。

(5) 《论语》曰“人能弘道,非道弘人”,故曰“不争轻重尊卑”。

(6) 《易·系辞》曰“苟非其人,道不虚行”,故曰“其人苟可,其事无不可”。

(7) 疾,趋也。

(8) 魁大之士,名德之人。

(9) 信,从也。言从则其道见讲论矣。

(10) 《易》曰“匪我求童蒙,童蒙来求我”,故往教之师不见化从也。童蒙当求师而反召师,亦不宜化,师之道也。

【校】梁仲子云:“案《周易释文》‘童蒙求我,一本作来求我’,此注所引,从或本也。”又“而反召师”,旧本“师”讹“也”,今改正。

(11) 言往教之师不见听也。

(12) 谓召师而学,亦不听师言也。

(13) 言愈远于尊也。

(14) 腐烂必臭,怀而欲其香;入水必濡,而恶之;皆不可得也。

凡说者,兑之也,非说之也。 (1) 今世之说者,多弗能兑,而反说之。夫弗能兑而反说,是拯溺而硾之以石也, (2) 是救病而饮之以堇也, (3) 使世益乱,不肖主重惑者,从此生矣。故为师之务,在于胜理,在于行义。 (4) 理胜义立则位尊矣, (5) 王公大人弗敢骄也, (6) 上至于天子,朝之而不惭。 (7) 凡遇,合也,合不可必; (8) 遗理释义,以要不可必, (9) 而欲人之尊之也,不亦难乎? (10) 故师必胜理行义然后尊。

(1) 【校】旧校云:“一作‘本’。”

(2) 硾,沉也。能没杀人,何拯之有?

【校】旧校云:“‘拯’一作‘承’。”案“拯”、“承”通。

(3) 救,治也。堇,毒药也。能毒杀人,何治之有?

(4) 行尊道贵德之义。

(5) 王者不臣师,是位尊也。

【校】孙云:“以上下文参校,‘义立’当作‘义行’。”

(6) 不敢骄侮轻慢师道。

(7) 天子朝师,尊有德,故不惭。

(8) 师道与天子,遭时见尊,不可必常也。

(9) 要,求也。

(10) 为师如是,不见尊之道也,故曰“不亦难乎”。

曾子曰:“君子行于道路,其有父者可知也,其有师者可知也。夫无父而无师者,余若夫何哉!”此言事师之犹事父也。曾点使曾参,过期而不至, (1) 人皆见曾点曰:“无乃畏邪?” (2) 曾点曰:“彼虽畏,我存,夫安敢畏?”孔子畏于匡,颜渊后,孔子曰:“吾以汝为死矣。”颜渊曰:“子在,回何敢死?”颜回之于孔子也,犹曾参之事父也。古之贤者与, (3) 其尊师若此,故师尽智竭道以教。 (4)

(1) 曾点,曾参父也。《诗》云“期逝不至,而多为恤”,此之谓也。

(2) 畏,犹死也。

(3) 句。

(4) 尊师犹尊父,则师不为之爱道也,故曰“尽智竭道以教”也。





尊师


三曰:

神农师悉诸,黄帝师大挠, (1) 帝颛顼师伯夷父,帝喾师伯招,帝尧师子州支父, (2) 帝舜师许由,禹师大成贽, (3) 汤师小臣, (4) 文王、武王师吕望、周公旦,齐桓公师管夷吾, (5) 晋文公师咎犯、随会, (6) 秦穆公师百里奚、公孙枝, (7) 楚庄王师孙叔敖、沈尹巫, (8) 吴王阖闾师伍子胥、文之仪, (9) 越王句践师范蠡、大夫种。 (10) 此十圣人、六贤者,未有不尊师者也。今尊不至于帝,智不至于圣,而欲无尊师,奚由至哉? (11) 此五帝之所以绝,三代之所以灭。 (12)

(1) 悉,姓;诸,名也。大挠作甲子。

【校】《汉书·古今人表》亦作“悉诸”,《新序·杂事》五引《吕子》作“悉老”,“大挠”作“大真”,《人表》作“大填”。

(2) 【校】旧本无“支”字,校云“一作‘友’”,则于文无所丽。孙据《御览》四百四所引补“支”字,与《庄子》、《汉书人表》、皇甫谧《高士传》皆合。《贵生》篇作“子州友父”,嵇康《高士传》亦同,见《御览》五百九,此即旧校者所据本也。

(3) 【校】《新序》作“执”。

(4) 小臣,谓伊尹。

(5) 【校】《新序》有“隰朋”。

(6) 咎犯,狐偃也。随会,范武子。

【校】案:随会在文公后,此与《说苑·尊贤》篇“晋文侯行地登隧,随会不扶”,皆记者之误也。梁伯子云:“《列子·说符》又以随会与赵文子并时,亦非。”

(7) 百里奚,故虞臣也。公孙枝,大夫子桑也。

(8) 沈县大夫。

【校】旧本“尹”作“申”,讹。其名多不同,《当染》篇作“沈尹蒸”,《察传》篇作“沈尹筮”,《赞能》篇作“沈尹茎”,此又作“巫”,《新序》作“竺”,《渚宫旧事》作“华”,文皆相近。

(9) 文,氏;之仪,名。

(10) 范蠡,字少伯,楚人也。大夫种,姓文,字禽,楚郢人。

【校】注“郢”,旧本讹作“鄞”,今改正,说见《当染》篇。

(11) 至于道。

(12) 言五帝、三代之后,不复重道尊师,故所以绝灭。

且天生人也,而使其耳可以闻,不学,其闻不若聋; (1) 使其目可以见,不学,其见不若盲; (2) 使其口可以言,不学,其言不若爽; (3) 使其心可以知,不学,其知不若狂。 (4) 故凡学,非能益也, (5) 达天性也。能全天之所生而勿败之,是谓善学。 (6) 子张,鲁之鄙家也; (7) 颜涿聚,梁父之大盗也;学于孔子。段干木,晋国之大驵也, (8) 学于子夏。 (9) 高何、县子石, (10) 齐国之暴者也,指于乡曲, (11) 学于子墨子。 (12) 索卢参,东方之巨狡也, (13) 学于禽滑黎。 (14) 此六人者,刑戮死辱之人也。今非徒免于刑戮死辱也,由此为天下名士显人,以终其寿, (15) 王公大人从而礼之,此得之于学也。 (16)

(1) 聋,无所闻也。

(2) 盲,无所见也。

【校】梁仲子云:“《意林》作‘耳有所闻,不学而不如聋;目有所见,不学而不如盲’,马氏盖以意节之耳。”

(3) 爽,病;无所别也。

【校】《新序》“爽”作“喑”。孙云:“《御览》三百六十六作‘其言曲以爽’。”

(4) 暗行妄发之谓狂。

【校】孙云:“《御览》作‘其知暗以狂’。”

(5) 【校】《御览》“能益”上有“为”字,《新序》“能益”下有“之”字。

(6) 败,毁也。

(7) 鄙,小。

(8) 驵, 人也。

【校】注“ ”,疑与“侩”通。

(9) 子夏,孔子弟子卜商之字。

(10) 【校】《墨子》书,弟子有高石子,不见此二人。

(11) 其暴虐为乡曲人所斥也。

(12) 墨翟。

(13) 巨,大。狡,猾。

(14) 禽滑黎,墨子弟子。

【校】此注末有“一作籥滑”四字,当出旧校者之辞,但“滑”字各书或作“骨”,或作“屈”,“黎”字或作“氂”,或作“釐”,至“禽”字各书俱同,未见有作“籥”者。《墨子·耕柱》篇有骆滑氂好勇,闻乡有勇士必杀之,墨子谓非好勇是恶勇,则非墨子弟子也。

(15) 寿,年也。

(16) 学以致之,无鬼神也,故曰得之。

凡学,必务进业,心则无营; (1) 疾讽诵, (2) 谨司闻; (3) 观欢愉,问书意; (4) 顺耳目,不逆志; (5) 退思虑,求所谓; (6) 时辨说,以论道; (7) 不苟辨,必中法; (8) 得之无矜,失之无惭, (9) 必反其本。 (10) 生则谨养,谨养之道,养心为贵; (11) 死则敬祭,敬祭之术,时节为务。 (12) 此所以尊师也。治唐圃,疾灌寖,务种树; (13) 织葩屦, (14) 结罝网,捆蒲苇;之田野,力耕耘,事五谷; (15) 如山林,入川泽, (16) 取鱼鳖,求鸟兽。此所以尊师也。视舆马,慎驾御; (17) 适衣服,务轻暖;临饮食,必蠲絜; (18) 善调和,务甘肥;必恭敬,和颜色,审辞令;疾趋翔, (19) 必严肃。此所以尊师也。

(1) 营,惑。

(2) 疾,力。

(3) 司,候。

【校】“司”,古“伺”字。

(4) 视师欢悦以问书意。

(5) 不自干逆力学之志。

(6) 求所思虑,是而行之。

(7) 辨别道之义理。

(8) 不苟口辨,反是为非,言中法制。

(9) 矜,自伐。无惭恡也。

(10) 本,谓本性也。

(11) 贵,尚也。

【校】所谓养志是也。

(12) 四时之节。

【校】旧校云:“‘时’一作‘崇’。”

(13) 唐,堤,以壅水。圃,农圃也。树,稼也。

(14) 【校】案:“葩”疑“萉”字之误。《说文》“萉,枲实也”,或作“ ”。盖葩屦即后人所谓麻鞵耳。案《晏子·问下》篇有“治唐园,考菲履”之语,萉音与菲亦相近,益明为“萉”字无疑。

(15) 事,治也。

(16) 如,往也。川泽有水,故言入也。

(17) 【校】旧校云:“‘慎’一作‘顺’。”

(18) 蠲,读曰圭也。

【校】旧校云:“‘絜’字一作‘祭’。”

(19) 【校】“翔”与“跄”同。

君子之学也,说义必称师以论道, (1) 听从必尽力以光明。 (2) 听从不尽力,命之曰背;说义不称师,命之曰叛。 (3) 背叛之人,贤主弗内之于朝, (4) 君子不与交友。 (5) 故教也者,义之大者也;学也者,知之盛者也。义之大者,莫大于利人,利人莫大于教; (6) 知之盛者,莫大于成身,成身莫大于学。 (7) 身成,则为人子弗使而孝矣,为人臣弗令而忠矣,为人君弗强而平矣,有大势可以为天下正矣。 (8) 故子贡问孔子曰:“后世将何以称夫子?”孔子曰:“吾何足以称哉?勿已者,则好学而不厌,好教而不倦,其惟此邪!”天子入太学 [1] 祭先圣,则齿尝为师者弗臣,所以见敬学与尊师也。 (9)

(1) 论,明。

(2) 听从师所行。

(3) 背,戾也。叛,换也。言学者听从不尽其力,犹民背国;说义不称其师,犹臣叛君。

【校】注以换训叛。换,易也。《诗·卷阿》“伴奂”,徐邈音“畔换”,笺云“自纵弛之意”。学者以己臆见易师之说,即是自放纵叛其师也。

(4) 贤,明。

(5) 不与背叛之人为交友。

(6) 以仁义利之,教然后知,故曰“莫大于教”也。

(7) 成身遂为君子,以致之,故曰“莫大于学”。

(8) 天下正者,正天下也。

(9) 太学,明堂也。





诬徒 (1)


(1) 【校】一作“诋役”。

四曰:

达师之教也, (1) 使弟子安焉,乐焉,休焉,游焉,肃焉,严焉。此六者得于学,则邪辟之道塞矣, (2) 理义之术胜矣。 (3) 此六者不得于学,则君不能令于臣,父不能令于子,师不能令于徒。 (4) 人之情,不能乐其所不安,不能得于其所不乐。为之而乐矣,奚待贤者?虽不肖者犹若劝之。为之而苦矣,奚待不肖者?虽贤者犹不能久。 (5) 反诸人情,则得所以劝学矣。子华子曰:“王者乐其所以王, (6) 亡者亦乐其所以亡, (7) 故烹兽不足以尽兽,嗜其脯则几矣。” (8) 然则王者有嗜乎理义也, (9) 亡者亦有嗜乎暴慢也。所嗜不同,故其祸福亦不同。 (10)

(1) 达,通也。

(2) 塞,断也。

(3) 术,道也。胜犹行也。

(4) 【校】旧云:“此篇一名‘诋役’,凡篇中‘徒’字皆作‘役’,徒与役谓弟子也。”案此段疑非高氏之文。

(5) 久,长也。

(6) 子华子,古之体道人。乐其所以王,故得王,汤、武是也。

(7) 乐其所以亡,故得亡,桀、纣是也。

(8) 几,近也。

(9) 嗜犹乐。乐行理义。

(10) 嗜理义则获福,嗜暴慢则获祸,故曰“祸福亦不同”。

不能教者,志气不和,取舍数变,固无恒心,若晏阴喜怒无处; (1) 言谈日易,以恣自行,失之在己,不肯自非, (2) 愎过自用,不可证移; (3) 见权亲势及有富厚者,不论其材,不察其行,驱而教之,阿而谄之,若恐弗及; (4) 弟子居处修洁,身状出伦, (5) 闻识疏达,就学敏疾,本业几终者,则从而抑之, (6) 难而悬之,妒而恶之;弟子去则冀终, (7) 居则不安, (8) 归则愧于父母兄弟, (9) 出则惭于知友邑里;此学者之所悲也, (10) 此师徒相与异心也。人之情恶异于己者,此师徒相与造怨尤也。 (11) 人之情不能亲其所怨,不能誉其所恶,学业之败也,道术之废也,从此生矣。 (12)

(1) 晏阴,喻残害也。处,常也。

(2) 谓若桀、纣罪人。

(3) 愎,戾。证,谏。

(4) 见权势及富厚者,故不论其材行,阿意谄之,恐不见及。

(5) 伦,匹。

(6) 几,近也。

(7) 弟子欲去,则冀终其业,且由豫也。

(8) 居,近也。苦其恶不安也。

(9) 愧,惭。

(10) 悲,悼。

(11) 造,作。

(12) 废,失。

善教者则不然,视徒如己, (1) 反己以教,则得教之情也。 (2) 所加于人,必可行于己, (3) 若此则师徒同体。 (4) 人之情,爱同于己者,誉同于己者,助同于己者,学业之章明也,道术之大行也,从此生矣。

(1) 徒,谓弟子也。

(2) 情,理。

【校】朱本“也”作“矣”。

(3) 所施于人者,人乐也,故曰“必可行于己”。

(4) 体,行也。

不能学者,从师苦而欲学之功也, (1) 从师浅而欲学之深也。 (2) 草木、鸡狗、牛马不可谯诟遇之,谯诟遇之,则亦谯诟报人, (3) 又况乎达师与道术之言乎? (4) 故不能学者,遇师则不中,用心则不专, (5) 好之则不深,就业则不疾, (6) 辩论则不审, (7) 教人则不精; (8) 于师愠, (9) 怀于俗, (10) 羁神于世; (11) 矜势好尤,故湛于巧智, (12) 昏于小利,惑于嗜欲; (13) 问事则前后相悖, (14) 以章则有异心, (15) 以简则有相反; (16) 离则不能合,合则弗能离, (17) 事至则不能受; (18) 此不能学者之患也。 (19)

(1) 苦,读如盬会之盬。苦,不精至也。功,名也。欲得为名。

【校】注“盬”,旧作“监”,讹。此以盬恶训苦,但会字未详,亦恐有讹。精至,即精致。其云“功,名也”误。功与苦相反,与下文浅、深一例。《齐语》云“工辨其功苦”,注云:“坚曰功,脆曰苦。”

(2) 欲人谓之学深也。

(3) 谯诟犹祸恶也。

【校】谯诟疑即贾谊疏之“奊诟”,谓遇之不如其分也。彼颜注云“无志分”,此注云“祸恶”,亦各以意解耳。旧校云“‘谯’一作‘护’”,更难通。

(4) 达,通也。

(5) 不中,不正也。不专,不壹也。

(6) 不心好之,故不能深。就业不疾速也。

(7) 不能明是非。

(8) 教,效也。效人别是非,不能精核。

(9) 愠,怒也。不能别是非,故怨于师。

(10) 怀,安也。

(11) 羁,牵也。神,御也。世,时也。

【校】盖谓其精神萦扰于世务而不能脱然也。注训神为御,未详。

(12) 矜大其权势,好为尤过之事,湛没于巧诈之智。

(13) 昏,迷;惑,悖也。

(14) 悖,乱。

(15) 心犹义也。

【校】旧校云:“‘章’一作‘军’。”

(16) 反,易。

【校】旧校云:“‘简’一作‘文’。”

(17) 离,别。

(18) 受犹成也。

(19) 患,害也。





用众 (1)


(1) 【校】一作“善学”。

五曰:

善学者若齐王之食鸡也,必食其跖数千而后足, (1) 虽不足,犹若有跖。 (2) 物固莫不有长,莫不有短,人亦然。 (3) 故善学者,假人之长以补其短,故假人者遂有天下。无丑不能,无恶不知。 (4) 丑不能,恶不知,病矣。 (5) 不丑不能,不恶不知,尚矣。 (6) 虽桀、纣犹有可畏可取者,而况于贤者乎? (7) 故学士曰:“辩议不可不为。” (8) 辩议而苟可为,是教也。教,大议也。辩议而不可为,是被褐而出,衣锦而入。 (9)

(1) 跖,鸡足踵。喻学者取道众多然后优也。跖,读如捃摭之摭。

【校】《淮南·说山训》“数千”作“数十”。注“取道”,旧本作“之道”,亦从彼注改。

(2) 食鸡跖众而后足也。若有博学多艺,如食鸡跖,道乃深也。

【校】正文难晓。注重释上文,于此句殊不比附。窃疑正文“不”字乃衍文。谓虽足而犹若有跖未尽食者,此则学如不及,唯恐有闻,为足以形容好学者贪多务得之意耳。

(3) 亦有长短。

(4) 故孔子入太庙每事问,是不丑不能,不恶不知。

(5) 病,困。

(6) 尚,上也。

(7) 桀作瓦,纣作胡粉,今人业之,尚可取之一隅。

(8) 不可为者,不可施也。

(9) 被褐在外,衣锦盛内,故不可。

戎人生乎戎、长乎戎而戎言,不知其所受之;楚人生乎楚、长乎楚而楚言,不知其所受之。今使楚人长乎戎,戎人长乎楚,则楚人戎言,戎人楚言矣。 (1) 由是观之,吾未知亡国之主不可以为贤主也, (2) 其所生长者不可耳。故所生长不可不察也。

(1) 《孟子》曰:“有楚大夫,欲其子之齐言也,使一齐人傅之,众楚人咻之,虽日挞而求其齐也,不可得矣。引而置之庄岳之间数年,虽日挞而求其楚,亦不可得矣。”此之谓也。

(2) 欲以楚人戎言、戎人楚言化移之。

天下无粹白之狐,而有粹白之裘, (1) 取之众白也。夫取于众,此三皇五帝之所以大立功名也。 (2) 凡君之所以立,出乎众也。立已定而舍其众,是得其末而失其本。得其末而失其本,不闻安居。 (3) 故以众勇无畏乎孟贲矣, (4) 以众力无畏乎乌获矣, (5) 以众视无畏乎离娄矣, (6) 以众知无畏乎尧、舜矣。 (7) 夫以众者,此君人之大宝也。 (8) 田骈谓齐王曰:“孟贲庶乎患术,而边境弗患。 (9) 楚、魏之王辞言不说, (10) 而境内已修备矣,兵士已修用矣,得之众也。”

(1) 粹,纯。

(2) 三皇:伏羲、神农、女娲也。五帝:黄帝、帝喾、颛顼、帝尧、帝舜也。

【校】注“女娲”当在“神农”前。

(3) 不闻得末失本能有安定之居也。

(4) 孟贲,古大勇士。

(5) 乌获,有力人,能举千钧。

【校】注“千钧”,旧本误作“千金”,今据前《重己》篇注改正。

(6) 离娄,黄帝时明目人,能见针末于百步之外。

(7) 尧、舜,圣帝也。言百发之中,必有羿、逢蒙之功,众知之中,必有与圣人同,故曰无畏于尧、舜也。

【校】注“功”,疑当作“巧”。

(8) 《淮南记》曰:“万人之众无废功,千人之众无绝良。”故人君以众为大宝也。

(9) 齐之边境,不以孟贲为患者,众也。

(10) 不以言辞为说。




————————————————————

[1] 学:原本作“庙”,据许维遹本改。





第五卷 仲夏纪



仲夏


一曰:

仲夏之月,日在东井, (1) 昏亢中,旦危中。 (2) 其日丙丁,其帝炎帝,其神祝融。其虫羽,其音徵,律中蕤宾, (3) 其数七。其味苦,其臭焦,其祀灶,祭先肺。小暑至,螳螂生, (4) 始鸣,反舌无声。 (5) 天子居明堂太庙, (6) 乘朱辂,驾赤 ,载赤旂,衣朱衣,服赤玉,食菽与鸡,其器高以觕,养壮狡。 (7)

(1) 仲夏,夏之五月。东井,南方宿,秦之分野,是月日躔此宿。

(2) 亢,东方宿,卫之分野;危,北方宿,齐之分野;是月昏旦时,皆中于南方也。

【校】案:《淮南·天文训》亢为郑之分野。

(3) 蕤宾,阳律也。是月阴气萎蕤在下,象主人;阳气在上,象宾客。竹管音中蕤宾也。

(4) 小暑,夏至后六月节也,螳螂于是生。螳螂一曰天马,一曰龁疣,兖州谓之拒斧也。

【校】注“龁疣”,《月令正义》郑答王瓒问作“食肬”;俗本作“食 ”,误。《淮南》注作“齿肬”,当是脱其半耳。《初学记》引此注正作“龁疣”,又云“兖、豫谓之巨斧”。

(5) ,伯劳也。是月阴作于下,阳发于上,伯劳夏至后应阴而杀蛇,磔之于棘而鸣于上。《传》曰:“伯赵氏,司至者也。”反舌,伯舌也,能辨反其舌,变易其声,效百鸟之鸣,故谓之百舌。承上微阴,伯赵起于下,后应阴,故无声。

【校】注“阳发于上”,《初学记》作“阳散于上”;又“磔之”句作“乃磔之棘上而始鸣也”。案辨反即遍反,古“辨”、“遍”通。

(6) 明堂,南向堂也。太庙,中央室也。

(7) 壮狡,多力之士,养之慎阳施也,盖所谓旱则资舟,夏则资皮,备之也。

【校】“壮狡”,《月令》作“壮佼”。此书《听言》篇作“壮狡”,《禁塞》篇作“壮佼”,二字通。郑《诗·狡童》传云“昭公有壮狡之志”,亦作“狡”字。

是月也,命乐师,修鞀鞞鼓,均琴瑟管箫, (1) 执干戚戈羽, (2) 调竽笙埙篪, (3) 饬钟磬柷敔。 (4) 命有司,为民祈祀山川百原,大雩帝,用盛乐。 (5) 乃命百县,雩祭祀百辟卿士有益于民者,以祈谷实。 (6) 农乃登黍。 (7)

(1) 师,乐官之长也。鞀鞞,所以节乐也,故修之。琴瑟管箫,所以宣音也,故均平之。管,六孔,似篪。箫,今之歌竹箫也。

【校】注“管六孔似篪”,旧本作“一孔似蘧”,讹,今据《广雅》改正。

(2) 干,楯。戚,斧。戈,戟,长六尺六寸。羽以为 ,舞者执之以指麾也。春夏干戚,秋冬羽籥。

(3) 竽,笙之大者,古皆以瓠为之。竽,三十六簧,笙,十七簧。埙,以土为之,大如雁子,其上为六孔。篪,以竹,大二寸,长尺二寸,七孔,一孔上伏,横吹之。声音上和,故言调。《诗》云“伯氏吹埙,仲氏吹篪”,此之谓也。

【校】“埙篪”,《月令》作“竾簧”。注“竽笙之大者”,旧脱“者”字,今补。郭璞注《尔雅》大笙云“十九簧”,小笙“十三簧”,《广雅》但云“笙十三管”,今此云“十七簧”,恐字误。

(4) 钟,金。磬,石。柷,如漆桶,中有木椎,左右击以节乐。敔,木虎,脊上有鉏铻,以杖栎之以止乐。乐以和成,故饬整之也。

(5) 名山大川,泉源所出。非一,故言百。能兴雨者,皆祈祀之。雩,旱祭也。帝,五帝也。为民祈雨,重之,故用盛乐。六代之乐也。

(6) 百县,畿内之百县大夫也。祀前世百君卿士功施于民者。雩祭之,求福助成谷实。

【校】“祭”字衍,《月令》无。注首“百县”,旧作“百辟”,讹,今改正。

(7) 登,进。稙黍熟,先进之。

是月也,天子以雏尝黍, (1) 羞以含桃,先荐寝庙。 (2) 令民无刈蓝以染, (3) 无烧炭, (4) 无暴布。 (5) 门闾无闭,关市无索。 (6) 挺重囚,益其食。 (7) 游牝别其群,则絷腾驹,班马正。 (8)

(1) 雏,春鹨也。不言尝雏而言尝黍,重谷也。

(2) 羞,进。含桃, 桃。 鸟所含食,故言含桃。是月而熟,故进之。先致寝庙,孝而且敬。

(3) 为蓝青未成也。

(4) 为草木未成,不欲夭物。

【校】《月令》作“毋烧灰”。

(5) 是月炎气盛猛,暴布则脆伤之。

(6) 门,城门。闾,里门也。民顺阳气,布散在外,人当出入,故不闭也。关,要塞也,市,人聚也。无索,不征税。

(7) 挺,缓也。

(8) 是月牝马怀妊已定,故放之则别其群,不欲驹蹄逾趯其胎育,故絷之也。班,告也。马正,掌马之官。《周礼》“五尺曰驹”。“马正”,《月令》作“马政”。注“逾”,疑当作“踊”。

是月也,日长至, (1) 阴阳争,死生分。 (2) 君子斋戒,处必掩, (3) 身欲静无躁,止声色,无或进, (4) 薄滋味,无致和, (5) 退嗜欲,定心气,百官静,事无刑,以定晏阴之所成。 (6) 鹿角解,蝉始鸣, (7) 半夏生,木堇荣。 (8)

(1) 夏至之日,昼漏水上刻六十五,夜漏水上刻三十五,故曰“长至”。

【校】旧本作“长日至”,《黄氏日抄》已言其误,今依《月令》移正。

(2) 是月阴气始起于下,盛阳盖覆其上,故曰“争”也。品物滋生,荠、麦、亭历、棘刺之属死,故曰“死生分”。分,别也。

【校】注“覆”字旧本脱在“起于”下,今移正。

(3) 句。

(4) 掩,深也。声,五音。色,五色。止节之,无有进御也。

【校】《月令》无“欲静”二字,郑注云“今《月令》‘毋躁’为‘欲静’”,然则此又出“无躁”二字,非本文也。“掩”亦与“奄”同,注皆训为深。盖夏避暑气,冬避寒气,皆以居处言也。今人多读“处必掩身”为句。考《月令正义》引正文已如此,但其所释亦是以居处言,并不谓身之不当亵露,故疑《正义》“处必掩”下之“身”字亦后人所加也。

(5) 薄犹损也。和,齐和也。

(6) 退,止也。事无刑,当精详而后行也。晏安阴,微阴。

【校】《月令》“退”作“止”。

(7) 夏至,鹿角解堕,蝉鼓翼始鸣。

(8) 半夏,药草。木堇,朝荣暮落,是月荣华,可用作蒸,杂家谓之朝生,一名蕣,《诗》云“颜如蕣华”是也。

是月也,无用火南方, (1) 可以居高明,可以远眺望,可以登山陵,可以处台榭。 (2) 仲夏行冬令,则雹霰伤谷,道路不通,暴兵来至; (3) 行春令,则五谷晚熟,百螣时起,其国乃饥; (4) 行秋令,则草木零落,果实早成,民殃于疫。 (5)

(1) 火王南方,为扬火气。

(2) 明,显也。积土四方而高曰台,台加木为榭,皆所以顺阳宣明之。

【校】观此,则郑注“处必掩”为“隐翳”,高注为“深”,皆与此相反,故仲夏言掩身,理可通也。

(3) 冬寒,冰冻,故雹霰伤害五谷也。冬阴,闭藏,多雹霰,道路陷坏,不通利也。暴害之兵横来至。

【校】《月令》“霰”作“冻”。

(4) 行春木王生育之令,故五谷晚熟也。百螣,动股之属也,时起为害,故五谷不时,国饥也。螣,读近殆,兖州人谓蝗为螣。

(5) 有核曰果,无核曰蓏。仲夏行秋成熟之令,故草木零落,果实早成熟。非其时气,故民疾疫。





大乐


二曰:

音乐之所由来者远矣, (1) 生于度量,本于太一。太一出两仪,两仪出阴阳。 (2) 阴阳变化,一上一下,合而成章。 (3) 浑浑沌沌,离则复合,合则复离, (4) 是谓天常。 (5) 天地车轮, (6) 终则复始,极则复反,莫不咸当。 (7) 日月星辰,或疾或徐,日月不同,以尽其行。 (8) 四时代兴,或暑或寒,或短或长,或柔或刚。 (9) 万物所出,造于太一,化于阴阳。 (10) 萌芽始震,凝 以形。 (11) 形体有处,莫不有声。声出于和,和出于适。和适,先王定乐,由此而生。 (12)

(1) 远,久。

(2) 两仪,天地也。出,生也。

(3) 章犹形也。

(4) 浑,读如衮冕之衮。沌,读近屯。离,散。合,会。

(5) 天之常道。

(6) 轮,转。

【校】李善注《文选》木玄虚《海赋》引作“天地如车轮”,《御览》一、又五百六十六皆无“如”字。

(7) 极,穷。咸,皆。当,合。

(8) 不同,度有长短也。以尽其行度也。起牵牛至周于牵牛,故曰“以尽其行”。

【校】《御览》五百六十六作“宿日不同”。

(9) 冬寒,夏暑。冬至短,夏至长。春柔而秋刚。

(10) 造,始也。太一,道也。阴阳,化成万物者也。

【校】旧校云:“‘造’一作‘本’。”案《御览》“造”、“本”二字皆有。

(11) 震,动也。谓动足以成形也。

【校】《御览》作“萌芽始厥,凝寒以刑”,注“厥,动也”。案字书本无“ ”字,此误。“刑”与“形”通。

(12) 由和生也。

【校】正文“和适”二字疑衍。注“由和”下似当有“适”字。

天下太平,万物安宁, (1) 皆化其上, (2) 乐乃可成。成乐有具,必节嗜欲, (3) 嗜欲不辟, (4) 乐乃可务。 (5) 务乐有术,必由平出,平出于公, (6) 公出于道。故惟得道之人,其可与言乐乎! (7) 亡国戮民,非无乐也,其乐不乐。 (8) 溺者非不笑也, (9) 罪人非不歌也, (10) 狂者非不武也, (11) 乱世之乐有似于此。君臣失位,父子失处,夫妇失宜,民人呻吟,其以为乐也,若之何哉? (12)

(1) 【校】“物”,《御览》作“民”。

(2) 化犹随也。

(3) 节,止。

(4) 辟,开。

(5) 务,成。

(6) 公,正。

(7) 言,说。

(8) 不和于雅,故不乐也。

【校】旧本作“不乐其乐”,孙云“《御览》五百六十九作‘其乐不乐’。”案下篇及《明理》篇俱作“其乐不乐”,今移正。

(9) 《传》曰:“溺人必笑。”虽笑不欢。

(10) 当死强歌,虽歌不乐。

【校】注“强歌”二字,旧本作“者”,今从《御览》补正。

(11) 狂悖之人,虽武不足畏。

(12) 以民人呻吟叹戚,不可为乐也,故曰“若之何哉”。

凡乐,天地之和,阴阳之调也。始生人者,天也,人无事焉。天使人有欲,人弗得不求; (1) 天使人有恶,人弗得不辟。 (2) 欲与恶所受于天也, (3) 人不得兴焉, (4) 不可变,不可易。 (5) 世之学者,有非乐者矣,安由出哉? (6)

(1) 欲,贪也。人情欲,故弗得不有求也。

(2) 恶,憎;辟,远也。故曰“弗得不辟”,人情有所憎恶,辟远之也。

(3) 受之于天。

(4) 不得为天之为也。

【校】注“不得为”下旧衍一“焉”字,今删。

(5) 天所为,故不可变易。

(6) 非犹讥。出犹生。

【校】案:《墨子》书有《非乐》篇。

大乐,君臣、父子、长少之所欢欣而说也。欢欣生于平, (1) 平生于道。道也者,视之不见,听之不闻,不可为状。 (2) 有知不见之见、不闻之闻、无状之状者,则几于知之矣。 (3) 道也者,至精也, (4) 不可为形,不可为名,强为之,谓之太一。 (5) 故一也者制令,两也者从听, (6) 先圣择两法一, (7) 是以知万物之情。故能以一听政者,乐君臣,和远近,说黔首, (8) 合宗亲;能以一治其身者,免于灾, (9) 终其寿,全其天; (10) 能以一治其国者,奸邪去,贤者至,成大化;能以一治天下者,寒暑适,风雨时, (11) 为圣人。故知一则明,明两则狂。 (12)

(1) 平,和。

(2) 言道无形,不可为状。

(3) 几,近也。有人能是,近于知道也。

(4) 精,微。

(5) 【校】“强为之”下疑脱一“名”字。

(6) 从听,听从。

(7) 择,弃也。法,用也。

(8) 秦谓民为黔首。

(9) 灾,害。

(10) 天,身。

(11) 适,和也。时,不差忒。

(12) 【校】疑当叠“知一”二字。





侈乐


三曰:

人莫不以其生生,而不知其所以生; (1) 人莫不以其知知,而不知其所以知。知其所以知,之谓知道;不知其所以知,之谓弃宝。弃宝者必离其咎。 (2) 世之人主,多以珠玉戈剑为宝,愈多而民愈怨,国人愈危,身愈危累, (3) 则失宝之情矣。 (4) 乱世之乐与此同。 (5) 为木革之声则若雷,为金石之声则若霆,为丝竹歌舞之声则若噪。 (6) 以此骇心气、动耳目、摇荡生则可矣, (7) 以此为乐则不乐。 (8) 故乐愈侈,而民愈郁, (9) 国愈乱,主愈卑,则亦失乐之情矣。凡古圣王之所为贵乐者,为其乐也。夏桀、殷纣作为侈乐,大鼓钟磬管箫之音,以巨为美, (10) 以众为观,俶诡殊瑰,耳所未尝闻,目所未尝见, (11) 务以相过,不用度量。 (12) 宋之衰也,作为千钟; (13) 齐之衰也,作为大吕; (14) 楚之衰也,作为巫音。 (15) 侈则侈矣,自有道者观之,则失乐之情。失乐之情,其乐不乐。 (16) 乐不乐者,其民必怨,其生必伤。 (17) 其生之与乐也,若冰之于炎日,反以自兵。 (18) 此生乎不知乐之情,而以侈为务故也。

(1) 以,用。

(2) 宝,重也。咎,殃也。

(3) 《老子》曰“多藏厚亡”,故曰“愈危累”。

(4) 情,实也。

(5) 同于危累。

(6) 噪,叫。

(7) 生,性。

(8) 不乐,不和。

(9) 侈,淫。郁,怨。

(10) 巨,大。

(11) 俶,始也。始作诡异瑰奇之乐,故耳未尝闻,目未尝见。

【校】案:“俶诡”亦作“ 诡”。《庄子·德充符》释文云:“ ,尺叔反。李云:‘ 诡,奇异也。’”又见《天下》篇。此注训 为始,非也。

(12) 不用乐之法则,故曰务相过。

(13) 钟律之名。

【校】“千钟”,《御览》五百六十六作“十秩”。

(14) 大吕,阴律,十二月也。

【校】此注非也。《贵直论》“无使齐之大吕陈之廷”,注云:“齐之钟律也。”案《史记索隐》云:“大吕,齐钟名。”王厚斋云:“此即乐毅书所云‘大吕陈于元英’者。”

(15) 男曰觋,女曰巫。

【校】旧本注无“男曰觋”三字,今从《初学记》十五所引补。梁仲子云:“《尚书》‘是谓巫风’,不特属之女也。《周礼·春官·神仕》疏云:‘男子阳,有两称,名巫名觋;女子阴,不变,直名巫,无觋。’所谓散文则通也。”

(16) 非正乐,故曰“不乐”也。

(17) 怨,悲。伤,痛。

(18) 兵,灾也。

【校】“生”,旧本讹作“王”,从《御览》改正。“炎日”,《御览》作“炭”。注“兵灾也”,或作“兵灾兵也”,非。

乐之有情,譬之若肌肤形体之有情性也,有情性则必有性养矣。寒、温、劳、逸、饥、饱,此六者非适也。 (1) 凡养也者,瞻非适而以之适者也。能以久处其适,则生长矣。 (2) 生也者,其身固静,感而后知,或使之也。遂而不返, (3) 制乎嗜欲, (4) 制乎嗜欲无穷,则必失其天矣。 (5) 且夫嗜欲无穷,则必有贪鄙悖乱之心,淫佚奸诈之事矣。 (6) 故强者劫弱,众者暴寡,勇者凌怯,壮者慠幼,从此生矣。 (7)

(1) 适,中适也。

(2) 长,久。

(3) 返,还。

(4) 为嗜欲所制。

(5) 天,身。

(6) 【校】“悖乱”,旧作“浮乱”,讹,今改正。此与《乐记》文相似。

(7) 从欲生也。





适音 (1)


(1) 【校】一作“和乐”。

四曰:

耳之情欲声, (1) 心不乐,五音在前弗听; (2) 目之情欲色, (3) 心弗乐,五色在前弗视; (4) 鼻之情欲芬香, (5) 心弗乐,芬香在前弗嗅; (6) 口之情欲滋味, (7) 心弗乐,五味在前弗食。欲之者,耳目鼻口也;乐之弗乐者,心也。心必和平,然后乐;心必乐,然后耳目鼻口有以欲之。故乐之务在于和心,和心在于行适。 (8) 夫乐有适,心亦有适。 (9) 人之情,欲寿而恶夭,欲安而恶危,欲荣而恶辱,欲逸而恶劳。四欲得,四恶除,则心适矣。四欲之得也,在于胜理。胜理以治身,则生全,以生全则寿长矣。胜理以治国,则法立,法立则天下服矣。故适心之务在于胜理。

(1) 欲闻音声。

(2) 心不乐,声音虽在前,耳不听之。

(3) 欲视五色。

(4) 心不欲视之也。

(5) 欲芬香之韬藉也。

【校】注“韬藉”疑是“酝藉”。

(6) 不嗅味也。

(7) 欲美味也。

(8) 适,中适也。

(9) 【校】旧本“夫乐”下衍“之”字,又“亦”字作“非”,孙并从《御览》五百六十九删正。

夫音亦有适。太巨则志荡, (1) 以荡听巨则耳不容,不容则横塞,横塞则振。 (2) 太小则志嫌,以嫌听小 (3) 则耳不充,不充则不詹, (4) 不詹则窕。 (5) 太清则志危,以危听清则耳谿极, (6) 谿极则不鉴,不鉴则竭。 (7) 太浊则志下,以下听浊则耳不收, (8) 不收则不抟,不抟则怒。 (9) 故太巨、太小、太清、太浊,皆非适也。 (10)

(1) 【校】孙云:“‘太巨’,《御览》作‘大巨’,以下凡‘太’字并作‘大’。”

(2) 振,动。

【校】旧本作“横塞则振动”,无注,今从《御览》改正。

(3) 嫌听譬自嫌之嫌。

【校】注有误字,似本为嫌字作音,而后人妄改之。

(4) 詹,足也。詹,读如澹然无为之澹。

【校】《御览》作“詹音澹也”,疑是。盖“澹”,古“赡”字。注既训詹为足,则自读从澹足之澹。《汉书·食货志》“犹未足以澹其欲也”,师古曰:“‘澹’,古‘赡’字。赡,给也。”当读时艳切。若依此注则如字,读徒滥切矣。恐亦是后人妄改也。

(5) 窕,不满密也。

(6) 谿,虚;极,病也。不闻和声之故也。

(7) 鉴,察也。太清无和,耳不能察,则竭病也。

【校】“鉴”,《御览》并作“监”。注末“也”字,旧本讹作“之”。

(8) 不收,越散。

(9) 不抟,入不专一也。故惑怒也。

【校】“抟”,旧本皆误作“特”,孙从《御览》改正。案“抟”与“专”同。注“入”字亦从《御览》补。

(10) 不巨、不小、不清、不浊,得四者之中乃为适。此四者,皆言其太,故曰“非适”。

【校】旧本“太小”在“太清”下,从《御览》乙正。

何谓适?衷,音之适也。何谓衷?大不出钧,重不过石,小大轻重之衷也。 (1) 黄钟之宫,音之本也, (2) 清浊之衷也。衷也者,适也。以适听适则和矣。乐无太,平和者是也。故治世之音安以乐,其政平也; (3) 乱世之音怨以怒,其政乖也;亡国之音悲以哀,其政险也。 (4) 凡音乐,通乎政而移风平俗者也。 (5) 俗定而音乐化之矣。故有道之世,观其音而知其俗矣,观其政而知其主矣。故先王必托于音乐以论其教。 (6) 《清庙》之瑟,朱弦而疏越,一唱而三叹,有进乎音者矣。 (7) 大飨之礼,上玄尊而俎生鱼, (8) 大羹不和,有进乎味者也。 (9) 故先王之制礼乐也,非特以欢耳目、极口腹之欲也, (10) 将以教民平好恶、行理义也。 (11)

(1) 三十斤为钧,百二十斤为石。

(2) 本始于黄钟,十一月律。

(3) 民闻其乐,安之日喜。

(4) 险犹危。

(5) 风犹化。

(6) 论,明。

(7) 文王之庙,肃然清静,贵其乐和,故曰“有进音”。

【校】案:《礼记·乐记》作“有遗音者矣”,下亦作“遗味”,郑注:“遗,余也。”今此俱作“进”,文不同。

(8) 大飨,飨上帝于明堂也。玄尊,明水也;俎生鱼,皆上质贵本。

【校】注“明水”,旧本作“酒水”讹,今改正。

(9) 大羹,肉湆而未之和,贵本古得礼也,故曰“有进乎味”。

(10) 特,但也。

【校】旧本于“将”字下注“特也”二字,误。案“将”字当属下文,据《乐记》当作“将以”,今并补正。

(11) 平,正也。行犹通也。





古乐


五曰:

乐所由来者尚也, (1) 必不可废。有节有侈,有正有淫矣。 (2) 贤者以昌,不肖者以亡。 (3)

(1) 尚,曩。

(2) 节,适也。侈,大也。正,雅也。淫,乱也。

(3) 昌,盛也。亡,灭也。

昔古朱襄氏之治天下也, (1) 多风而阳气畜积,万物散解,果实不成, (2) 故士达作为五弦瑟,以来阴气,以定群生。 (3) 昔葛天氏之乐,三人操牛尾,投足以歌八阕: (4) 一曰《载民》,二曰《玄鸟》,三曰《遂草木》,四曰《奋五谷》,五曰《敬天常》,六曰《建帝功》,七曰《依地德》,八曰《总禽兽之极》。 (5) 昔陶唐氏之始,阴多滞伏而湛积, (6) 水道壅塞,不行其原, (7) 民气郁阏而滞著, (8) 筋骨瑟缩不达, (9) 故作为舞以宣导之。 (10)

(1) 朱襄氏,古天子,炎帝之别号。

(2) 解,落也。有核曰果。

(3) 士达,朱襄氏之臣。

【校】“来”,旧本作“采”, ,今从《御览》五百七十六改正。《日抄》同。

(4) 葛天氏,古帝名。投足犹蹀足。阕,终。

【校】张揖曰:“葛天氏,三皇时君号也。”见《文选·上林赋》注。“操”旧作“掺”,俗字,今从《初学记》九、《御览》五百六十六、陈祥道《礼书》改正。

(5) 上皆乐之八篇名也。

【校】旧本“建帝功”作“达帝功”。案:《文选·上林赋》注张揖引作“彻帝功”。李善谓以“建”为“彻”,误,则当作“建”也。又旧本作“总万物之极”,校云:“一作‘禽兽之极’。”今案《初学记》十五、《史记·司马相如传》索隐及《选注》皆作“总禽兽之极”,今据改正。

(6) 陶唐氏,尧之号。

【校】孙云:“‘陶唐’乃‘阴康’之误。颜师古注《汉书·司马相如传》云:‘《古今人表》有葛天氏、阴康氏,诱不观《古今人表》,妄改《吕氏》本文。’案李善注《文选》竟沿其误,唯章怀注《后汉书·马融传》引作‘阴康’。”

(7) 故有洪水之灾。

【校】旧校云:“一作‘阳道壅塞,不行其次’。”孙云:“李善注《文选》傅武仲《舞赋》、张景阳《七命》俱引作‘阳道壅塞’。”

(8) 阏,读曰遏止之遏。

(9) 【校】《七命注》作“筋骨挛缩”。

(10) 宣,通。

昔黄帝令伶伦作为律。 (1) 伶伦自大夏之西, (2) 乃之阮隃之阴, (3) 取竹于嶰溪之谷,以生空窍厚钧者,断两节间, (4) 其长三寸九分而吹之,以为黄钟之宫, (5) 吹曰“舍少”。次制十二筒, (6) 以之阮隃之下,听凤皇之鸣,以别十二律。其雄鸣为六,雌鸣亦六,以比黄钟之宫,适合。 (7) 黄钟之宫皆可以生之,故曰黄钟之宫,律吕之本。 (8) 黄帝又命伶伦与荣将 (9) 铸十二钟,以和五音,以施《英韶》。以仲春之月,乙卯之日,日在奎,始奏之,命之曰《咸池》。 (10)

(1) 伶伦,黄帝臣。

【校】《说苑·修文》篇作“泠伦”,《古今人表》作“冷沦”。

(2) 大夏,西方之山。

(3) 阮隃,山名。山北曰阴。

【校】“阮隃”,《汉书·律志》作“昆仑”,《说苑·修文》篇、《风俗通·音声》篇、《左氏成九年正义》皆作“昆仑”,《世说·言语 [1] 》篇引《吕》亦同。

(4) 竹生溪谷者,取其厚钧,断两节间,以为律管。

【校】《汉志》作“取竹之解谷,生其窍厚均者”,《说苑》、《风俗通》亦同。《世说注》“厚”上增“薄”字,赘。

(5) 断竹长三寸九分,吹之,音中黄钟之宫。

【校】“其长三寸九分”,《汉志》无,《说苑》及《御览》五百六十五作“其长九寸”。钱詹事云:“三寸九分,不必改作九寸。安溪李文贞谓‘黄钟长八寸一分,应钟长四寸二分,此三寸九分,即二律相较之数’,是也。案此三寸九分,备有十二律,非谓黄钟止长三寸九分。下云‘以为黄钟之长’者,即长于应钟之数。盖应钟十月律,秦岁首所中也。增长三寸九分而得黄钟,方是十一月律。《吕纪》本用秦法,追考上古,知安溪之说不谬。”

(6) 六律六吕各有管,故曰十二筒。舍,成舍矣。

【校】《说苑》无“吹”字。旧本“曰”作“日”,《说苑》作“曰”,又“舍”作“含”,今“曰”字已据改正,其“舍”字亦讹,注“舍成舍矣”四字亦不可晓,因有此注“舍”字姑仍之。考《晋志》及《御览》五百六十五并作“含少”。明弘治中莆人李文利主“含少”之说,谓黄钟实止三寸九分,其说与古背,不可用。《御览》竟改作其“长九寸”,又近人谓当作“四寸五分”,皆非是。“筒”,《说苑》、《风俗通》、《御览》俱作“管”,李善注《文选》丘希范《侍宴》诗作“筩”,与“筒”实一字。善又别引作“箫”,误也。

(7) 合,和谐。

【校】“比”,旧本误作“此”,李善注马季长《长笛赋》引作“比”,《汉书》、《说苑》皆同。

(8) 法凤之雌雄,故律有阴阳,上下相生,故曰“黄钟之宫,皆可以生之”。

(9) 【校】旧校云:“一作‘援’。”今案《御览》作“营援”;《路史》作“荣猨”,注引《隋志》及《国朝会要》皆作“荣猨”。

(10) 奏十二钟乐,名之为《咸池》。

帝颛顼生自若水,实处空桑, (1) 乃登为帝。惟天之合,正风乃行, (2) 其音若熙熙凄凄锵锵。帝颛顼好其音,乃令飞龙作效八风之音, (3) 命之曰《承云》,以祭上帝。 (4) 乃令鱓先为乐倡, (5) 鱓乃偃寝,以其尾鼓其腹, (6) 其音英英。 (7)

(1) 处,居。空桑。

(2) 惟天之合,德与天合。风,化也。

【校】赵云:“言八方之风,各得其正也。”

(3) 八风,八卦之风。

(4) 上帝,昊天上帝。

(5) 倡,始也。

【校】“乃令”,《初学记》作“乃命”。乐倡,乐人也,似不当训为始。

(6) 鼓,击。

【校】“寝”,旧本讹“浸”。

(7) 英英,和盛之貌。

【校】旧本“英英”不重,误,与上文皆依《初学记》、《御览》改正。

帝喾命咸黑作为《声歌》: (1) 《九招》、《六列》、《六英》。 (2) 有倕作为鼙、鼓、钟、磬、吹苓、管、埙、篪、鞀、椎、锺。 (3) 帝喾乃令人抃, (4) 或鼓鼙,击钟磬,吹苓,展管篪。因令凤鸟、天翟舞之。帝喾大喜,乃以康帝德。 (5)

(1) 【校】旧校云:“‘声’一作‘唐’。”案《御览》、《路史》俱作“唐”。

(2) 【校】此六字衍,说见下。

(3) 【校】“有倕”,《御览》倒作“倕有”。有,当读为又。

(4) 两手相击曰抃。

(5) 康,安。

帝尧立,乃命质为乐。质乃效山林溪谷之音以歌, (1) 乃以麋 置缶而鼓之, (2) 乃拊石击石,以象上帝玉磬之音,以致舞百兽。瞽叟乃拌五弦之瑟, (3) 作以为十五弦之瑟。命之曰《大章》,以祭上帝。

(1) “质”当为“夔”。

【校】《路史》以质与夔非一人。“质”亦作“ ”。

(2) 鼓,击。

(3) 拌,分。

舜立,命延 (1) 乃拌瞽叟之所为瑟,益之八弦,以为二十三弦之瑟。帝舜乃令质修《九招》、《六列》、《六英》,以明帝德。 (2)

(1) 【校】“命”,旧本作“仰”,误,据《路史》改正。

(2) 《招》、《列》、《英》,皆乐名也。帝,谓舜。

【校】《招》、《列》、《英》至此始见,故诱于此下注,则上乃衍文明矣。

禹立,勤劳天下,日夜不懈。 (1) 通大川,决壅塞,凿龙门,降通漻水以导河, (2) 疏三江五湖,注之东海,以利黔首。于是命皋陶作为《夏籥》九成,以昭其功。 (3)

(1) 勤,忧。

(2) 决壅塞,故凿龙门也。降,大。漻,流。

(3) 九成,九变。昭,明。

殷汤即位,夏为无道,暴虐万民,侵削诸侯,不用轨度,天下患之。汤于是率六州以讨桀罪。 (1) 功名大成,黔首安宁。汤乃命伊尹作为《大护》,歌《晨露》,修《九招》、《六列》,以见其善。 (2)

(1) 【校】旧校云:“‘讨’一作‘诛’。”案《御览》作“以诛桀之罪”。

(2) 《大护》、《晨露》、《九招》、《六列》,皆乐名。善,美。

周文王处岐,诸侯去殷三淫而翼文王。 (1) 散宜生曰:“殷可伐也。”文王弗许。 (2) 周公旦乃作诗曰:“文王在上,于昭于天。周虽旧邦,其命维新。”以绳文王之德。 (3)

(1) 文王,古公亶父之孙,王季历之子也。古公避獯鬻之难,邑于岐,谓岐山之阳有周地,及受命,因为天下号也。淫,过。翼,佐。三淫,谓剖比干之心,断材士之股,刳孕妇之胎者,故诸侯去之而佐文王也。

【校】古文《秦誓》有“斮朝涉之胫”语,究不知何出?《春秋繁露·王道》篇云:“斮朝涉之足,视其挴。”《水经注》九淇水下云:“老人晨将渡水,而沈吟难济。纣问其故,左右曰‘老者髓不实,故晨寒也’,纣乃于此斮胫而视髓。”是相传有此事也。今此云“断材士之股”,《先识览》注亦同,《淮南·俶真训》亦有此语。

(2) 散宜生,文王四臣之一也。《论语》曰:“文王为西伯,三分天下有其二,以服事殷。”故弗许。

(3) 【校】绳,誉也。见《左氏庄十四年传正义》云:“字书‘绳’作‘譝’。”

武王即位,以六师伐殷。六师未至,以锐兵克之于牧野。 (1) 归,乃荐俘馘于京太室,乃命周公为作《大武》。 (2)

(1) 未至殷都,而胜纣于牧野。

(2) 《大武》,周乐。

【校】“为作”,《御览》倒。

成王立,殷民反, (1) 王命周公践伐之。 (2) 商人服象,为虐于东夷。 (3) 周公遂以师逐之,至于江南。乃为《三象》,以嘉其德。

(1) 反,叛。

(2) 践,往。

【校】《尚书大传》云:“周公摄政三年,践奄。践之者,籍之也。籍之,谓杀其身,执其家,猪其宫。”

(3) 象,兽名也。

故乐之所由来者尚矣,非独为一世之所造也。 (1)

(1) 《三象》,周公所作乐名。嘉,美也。尚,久也。自黄帝以来,功成作乐,故曰“非独为一世之所造也”。




————————————————————

[1] 言语:原本作“德行”,误,据《世说新语》改正。





第六卷 季夏纪



季夏


一曰:

季夏之月,日在柳, (1) 昏心中,旦奎中。 (2) 其日丙丁,其帝炎帝,其神祝融。其虫羽,其音徵,律中林钟, (3) 其数七。其味苦,其臭焦,其祀灶,祭先肺。凉风始至,蟋蟀居宇, (4) 鹰乃学习,腐草化为蚈。 (5) 天子居明堂右个, (6) 乘朱辂,驾赤 ,载赤旂,衣朱衣,服赤玉,食菽与鸡,其器高以觕。

(1) 季夏,夏之六月也。柳,南方宿,周之分野。是月,日躔北宿。

(2) 心,东方宿,宋之分野。奎,西方宿,鲁之分野。是月昏旦时,皆中于南方。

(3) 林,众;钟,聚;阴律也。阳气衰,阴气起,万物众聚而成,竹管之音应林钟也。

(4) 夏至后四十六日立秋节,故曰“凉风始至”。蟋蟀,蜻 ,《尔雅》谓之 。阴气应,故居宇,鸣以促织。

【校】《月令》“凉风”作“温风”,“居宇”作“居壁”。

(5) 秋节将至,故鹰顺杀气自习肄,为将搏鸷也。蚈,马蚿也。蚈,读如蹊径之蹊。幽州谓之秦渠,一曰萤火也。

【校】《月令》作“腐草为萤”。此书旧本作“腐草化为萤蚈”,衍“萤”字。《淮南》无,观注当与《淮南》同。盖昔人读此书,偶旁记异同之文,而因以误入也。《说文》引《明堂月令》曰“腐草为蠲”,蠲,即蚈也。“化”亦衍字。

(6) 明堂,向南堂。右个,西头室。

是月也,令渔师,伐蛟取鼍,升龟取鼋。 (1) 乃命虞人入材苇。 (2)

(1) 渔师,掌鱼官也。渔,读若相语之语。蛟、鼍、鼋皆鱼属。鼍皮可作鼓。《诗》曰“鼍鼓 ”;鼋可为羹,《传》曰“楚人献鼋于郑灵公,灵公不与公子宋鼋羹,公子怒,染指于鼎,尝之而出”是也;皆不害人,易得,故言“取”也。蛟有鳞甲,能害人,难得,故言“伐”也。龟,神,可以决吉凶,入宗庙,尊之也,故曰“升”也。

【校】渔,高读牛倨切,《季冬》云“音《论语》之语”亦同。《月令》“登龟”,此作“升”,义同。

(2) 虞人,掌山泽之官。材苇供国用也。

【校】“虞人”,《月令》作“泽人”。

是月也,令四监大夫合百县之秩刍,以养牺牲。 (1) 令民无不咸出其力, (2) 以供皇天上帝、名山大川、四方之神,以祀宗庙社稷之灵,为民祈福。 (3)

(1) 周制,天子畿内方千里,分为百县,县有四郡,郡有鄙,故《春秋传》曰:“上大夫受县,下大夫受郡。”周时县大郡小,至秦始皇兼天下,初置三十六郡以监县耳。此云“百县”,说周制畿内之县也。四监,监四郡大夫也。秩,常也。常所当刍,故聚之以养牺牲。

【校】《月令》作“大合”,无“夫”字。

(2) 咸,皆也。出其力以聚刍而用之。

(3) 祈,求也。

【校】《月令》“为民”上有“以”字。

是月也,命妇官染采,黼黻文章,必以法故,无或差忒,黑黄苍赤,莫不质良, (1) 勿敢伪诈, (2) 以给郊庙祭祀之服, (3) 以为旗章,以别贵贱等级之度。 (4)

(1) 妇人善别五色,故命其官使染采也。白与黑谓之黼。黑与青谓之黻。青与赤谓之文。赤与白谓之章。修其法章,不有差忒,故黑黄苍赤之色皆美善。

【校】《月令》“忒”作“贷”。旧校云:“‘差’一作‘迁’。”注“修其法章”,疑是“法制”。

(2) 勿,无也。

【校】《月令》作“毋敢诈伪”。

(3) 郊祀天,庙祀祖。

(4) 熊虎为旗。章,服也。贵有长尊,贱有等威,故曰“度”。

【校】“等威”,旧误作“等卑”,今依《左氏宣十二年传》文改正。

是月也,树木方盛,乃命虞人入山行木,无或斩伐。 (1) 不可以兴土功,不可以合诸侯,不可以起兵动众,无举大事,以摇荡于气。 (2) 无发令而干时,以妨神农之事。 (3) 水潦盛昌,命神农将巡功,举大事则有天殃。 (4)

(1) 虞人,掌山林之官。行,察也。视山木,禁民不得斩伐。

【校】“无或”,《月令》作“无有”,或亦训有也。

(2) 土功,筑台穿池。合诸侯,造盟会也。举动兵众,思启封疆也。大事,征伐也。于时不时,故曰“摇荡于气”。

【校】《月令》作“以摇养气”。注“思启封疆”,用《左氏成八年传》文,旧本作“息封疆”,误,今改正。

(3) 无发干时之令,畜聚人功,以妨害神农耘耨之事。

【校】“干时”,《月令》作“待”,无“干”字。

(4) 昔炎帝神农,能殖嘉谷,神而化之,号为神农。后世因名其官为神农。巡行堰亩修治之功。于此时或举大事,妨害农事,禁戒之,云有天殃之罚。

【校】《月令》“神农”上无“命”字,“巡”作“持”。

是月也,土润溽暑,大雨时行,烧薙行水,利以杀草,如以热汤,可以粪田畴,可以美土疆。 (1) 行之是令,是月甘雨三至,三旬二日。 (2) 季夏行春令,则谷实解落,国多风咳,人乃迁徙。 (3) 行秋令,则丘隰水潦,禾稼不熟,乃多女灾。 (4) 行冬令,则寒气不时,鹰隼早鸷,四鄙入保。 (5)

(1) 夏至后三十日大暑节,火王也。润溽而漯重,又有时雨。烧薙,行水灌之,如以热汤,可以成粪田畴,美土疆。疆,界畔。

(2) 行之是令,行是之令也。十日为旬。二日者,阴晦朔日也。月十日一雨,又二十日一雨,一月中得二日耳,故曰“三旬二日”。

(3) 春,木王。木性堕落,阳发多雨,而行其令,故谷实散落,民病风咳上气也。民迁徙移家,春阳布散也。

【校】“解落”,《月令》作“鲜落”。

(4) 丘,高;隰,下也。言高下有水潦,象金气也。故杀禾稼,使不成熟也。金干火,故多女灾,生子不育也。

(5) 冬阴闭固,而行其令,故寒风不节也。鹰隼早鸷,象冬气杀戮。四界之民畏寇贼之来,故入城郭自保守也。

【校】“寒气”,《月令》作“风寒”。

中央土,其日戊己, (1) 其帝黄帝,其神后土。 (2) 其虫倮,其音宫, (3) 律中黄钟之宫,其数五。 (4) 其味甘,其臭香, (5) 其祀中霤,祭先心。 (6) 天子居太庙太室, (7) 乘大辂,驾黄 ,载黄旂,衣黄衣,服黄玉, (8) 食稷与牛, (9) 其气圜以掩。 (10)

(1) 戊己,土日,土王中央也。

(2) 黄帝,少典之子,以土德王天下,号轩辕氏,死,托祀为中央之帝。后土,官。共工氏子句龙,能平九土,死,托祀为后土之神。

(3) 阳发散越,而属倮虫。倮虫,麒麟为之长。宫,土也,位在中央,为之音主。

(4) 黄钟,阳律也。竹管音中黄钟之宫也。其数五,五行之数,土第五也。

(5) 土味甘。土臭香。

(6) 土王中央,故祀中霤。霤,室中之祭,祭后土也。祭祀之肉先进心,心,火也,用所胜也。一曰:心,土,自用其藏也。

(7) 南向中央室曰太庙。又处其中央,故曰“太室”。

(8) 土色黄,故尚黄色。

(9) 稷、牛皆属土。

(10) 掩,象土含养万物。

【校】《月令》作“圜以闳”。旧校云:“一作‘掩以闳’。”





音律


二曰:

黄钟生林钟, (1) 林钟生太蔟, (2) 太蔟生南吕, (3) 南吕生姑洗, (4) 姑洗生应钟, (5) 应钟生蕤宾, (6) 蕤宾生大吕, (7) 大吕生夷则, (8) 夷则生夹钟, (9) 夹钟生无射, (10) 无射生仲吕。 (11) 三分所生,益之一分以上生。三分所生,去其一分以下生。黄钟、大吕、太蔟、夹钟、姑洗、仲吕、蕤宾为上,林钟、夷则、南吕、无射、应钟为下。 (12) 大圣至理之世,天地之气,合而生风。日至则月钟其风,以生十二律。 (13) 仲冬日短至, (14) 则生黄钟。季冬生大吕。孟春生太蔟。仲春生夹钟。季春生姑洗。孟夏生仲吕。仲夏日长至, (15) 则生蕤宾。季夏生林钟。孟秋生夷则。仲秋生南吕。季秋生无射。孟冬生应钟。天地之风气正,则十二律定矣。

(1) 黄钟,十一月律。林钟,六月律。

(2) 太蔟,正月律。

(3) 南吕,八月律。

(4) 姑洗,三月律。

(5) 应钟,十月律。

(6) 蕤宾,五月律。

(7) 大吕,十二月律。

(8) 夷则,七月律。

(9) 夹钟,二月律。

(10) 无射,九月律。

(11) 仲吕,四月律。

【校】案:《说苑·修文》篇云“黄钟生林钟,林钟生大吕,大吕生夷则,夷则生太蔟,太蔟生南吕,南吕生夹钟,夹钟生无射,无射生姑洗,姑洗生应钟,应钟生蕤宾”,无“蕤宾生大吕”句。《御览》五百六十五引《吕氏》亦与《说苑》同,皆非隔入相生之义。《晋书·律志》引《吕氏》则皆与今本合,知不可信《御览》以改此文。

(12) 律吕相生,上者上生,下者下生。

【校】案:蕤宾不当为上,当在林钟之首。考《周礼》《大司乐》、《大师》两章注,蕤宾皆重上生。即朱子《钟律》篇亦并不误。而近人反据误本谓蕤宾亦下生,谬之甚者。《晋志》俗本亦误作蕤宾下生,《宋志》则不误,可以正之。此注当作“上者下生,下者上生”,如此方所谓律吕相生。今本疑亦传写之误。

(13) 【校】《御览》“月钟”作“日行”,盖亦依《说苑》之文以改《吕氏》。

(14) 冬至日,日极短,故曰“日短至”。

(15) 夏至日,日极长,故曰“日长至”。

黄钟之月,土事无作,慎无发盖,以固天闭地,阳气且泄。 (1)

(1) 黄钟,十一月也。且,将也。

【校】《月令》作“以固而闭”。又“且泄”作“沮泄”。

大吕之月,数将几终, (1) 岁且更起,而农民无有所使。 (2)

(1) 大吕,十二月。几,近。终,尽。

(2) 使,役。

【校】《礼记·月令》“而农民”上有“专”字。

太蔟之月,阳气始生, (1) 草木繁动, (2) 令农发土,无或失时。 (3)

(1) 太蔟,正月。冬至后四十六日立春,故曰“阳气始生”。

(2) 动,生。

(3) 发土而耕。

【校】此月去芒种尚远,而必亟于发土者,盖所谓勿震勿渝,脉其满眚,谷乃不殖,故数劳之地,苗乃易于滋长也。

夹钟之月,宽裕和平,行德去刑, (1) 无或作事,以害群生。 (2)

(1) 夹钟,二月也。行仁德,去刑戮也。

(2) 事,兵戎事也。故曰“以害群生”。

姑洗之月,达道通路,沟渎修利, (1) 申之此令,嘉气趣至。 (2)

(1) 姑洗,三月也。时雨将降,故修利沟渎。

(2) 顺其阳德,故嘉喜之气至。

仲吕之月,无聚大众,巡劝农事, (1) 草木方长,无携民心。 (2)

(1) 仲吕,四月。大众,谓军旅工役也。顺阳长养,无役大众,妨废农功,故戒之曰“无”也。必循行农事劝率之。

(2) 民当务农,长养谷木,徭役聚则心携离,逆上命也,故戒之曰“无”也。

蕤宾之月,阳气在上,安壮养侠, (1) 本朝不静,草木早槁。 (2)

(1) 蕤宾,五月。壮,盛;侠,少也。皆安养之,助阳也。

【校】“在上”,旧本作“在土”。案是月阴始生于下,则当云“阳气在上”,今改正。《月令》是月“养壮佼”,此“养侠”亦当是“养佼”之误。

(2) 静,安。朝政不宁,故草木变动堕落早枯槁也。

林钟之月,草木盛满,阴将始刑, (1) 无发大事,以将阳气。 (2)

(1) 林钟,六月。刑,杀也。夏至后四十六日立秋。秋则行刑戮,故曰阴气将始杀也。

【校】“盛满”疑本是“盛盈”,与下文皆两句为韵。

(2) 发,起。将犹养。

夷则之月,修法饬刑,选士厉兵, (1) 诘诛不义,以怀远方。 (2)

(1) 夷则,七月也。饬,读如敕。饬正刑法,所以行法也。简选武士,厉利其兵。

(2) 怀,柔也。《诗》云“柔远能迩,以定我王”也。

南吕之月,蛰虫入穴, (1) 趣农收聚, (2) 无敢懈怠,以多为务。 (3)

(1) 南吕,八月也。蛰,读如《诗·文王之什》。

【校】旧本“文王”下有一“什”字,非,《孟春纪》注可证。

(2) 仲秋大雨,故收聚。

(3) 务犹事也。

无射之月,疾断有罪,当法勿赦, (1) 无留狱讼,以亟以故。 (2)

(1) 无射,九月。有罪当断,故勿赦。

(2) 亟,疾。故,事。

应钟之月,阴阳不通,闭而为冬, (1) 修别丧纪, (2) 审民所终。 (3)

(1) 应钟,十月。阳伏在下,阴闭于上,故不通。

(2) 【校】旧校云:“‘别’一作‘辨’。”

(3) 审,慎。终,卒。修别丧服亲疏轻重之纪,故曰“审民所终”也。





音初


三曰:

夏后氏孔甲田于东阳 山, (1) 天大风,晦盲, (2) 孔甲迷惑,入于民室。主人方乳, (3) 或曰:“后来,是良日也, (4) 之子是必大吉。” (5) 或曰:“不胜也,之子是必有殃。”后乃取其子以归,曰:“以为余子,谁敢殃之?”子长成人,幕动坼橑,斧斫斩其足, (6) 遂为守门者。 (7) 孔甲曰:“呜呼!有疾命矣夫!”乃作为《破斧》之歌,实始为东音。 (8)

(1) 孔甲,禹后十四世皋之父,发之祖,桀之宗。田,猎也。

【校】注“宗”,曾也,谓曾祖。

(2) 盲,暝也。

(3) 乳,产。

(4) 【校】“是”,旧本作“见”。孙云:“《御览》三百六十一及七百六十三‘见’俱作‘是’。”今据改。

(5) 之,其。

(6) 【校】“斫斩”疑衍“斩”,《御览》作“破”。

(7) 以其无足,遂为守门之官,向谓之子有殃也。

(8) 为东阳之音。

禹行功 (1) 见涂山之女, (2) 禹未之遇而巡省南土。 (3) 涂山氏之女乃令其妾候禹于涂山之阳。 (4) 女乃作歌,歌曰:“候人兮猗。” (5) 实始作为南音。 (6) 周公及召公取风焉,以为《周南》、《召南》。 (7)

(1) 【校】孙云:“李善注《文选》张平子《南都赋》、左太冲《吴都赋》并引作‘禹行水’,《御览》一百三十五同。”

(2) 【校】梁仲子云:“《水经注》《淮水》及《江水》引此并作‘嵞山’。”案宋柳佥本元作“涂山”。

(3) 遇,礼也。禹未之礼而巡狩南行也,省南方之土。

(4) 涂山在九江,近当涂也。山南曰阳也。

【校】“候”,旧本作“待”,今从《初学记》十改。善注《吴都赋》引作“往候”。注“九江”,旧作“九回”,误,今据《汉书·地理志》改正。

(5) 【校】《选》注无“兮”字。

(6) 南方国风之音。

(7) 取涂山氏女南音以为乐歌也。

周昭王亲将征荆, (1) 辛馀靡长且多力,为王右。 (2) 还反涉汉,梁败, (3) 王及蔡公抎于汉中。 (4) 辛馀靡振王北济,又反振蔡公。 (5) 周公乃侯之于西翟,实为长公。 (6) 殷整甲徙宅西河, (7) 犹思故处, (8) 实始作为西音。长公继是音以处西山, (9) 秦缪公取风焉,实始作为秦音。 (10)

(1) 周昭王,康王之子,穆王之父。荆,楚也。秦庄王讳楚,避之曰“荆”。

【校】《左氏僖四年传正义》引“荆”下有“蛮”字。

(2) 右,兵车之右也。

(3) 【校】案:《左传正义》云:“旧说皆言汉滨之人,以胶胶船,故得水而坏,昭王溺焉。”不知本出何书?此言“梁败”,又互异也。

(4) 抎,坠,音曰颠陨之陨。

【校】注“曰”字衍。

(5) 振,救也。《传》曰:“齐桓公伐楚,让之曰:‘尔贡苞茅不入,王祭不供,无以缩酒,寡人是徵。昭王南征,没而不复,寡人是问。’对曰:‘贡之不入,寡君之罪,敢不共乎?昭王之不复,君其问诸水滨。’”由此言之,昭王为没于汉,辛馀靡焉得振王北济哉?

【校】孙云:“振者,振其尸也。注非。”

(6) 西翟,西方也。以辛馀靡有振王之功,故赏之为长公。

【校】注“功”,旧本作“力”,非是,今改正。

(7) 【校】案:《竹书纪年》“河亶甲名整,元年自嚣迁于相”,即其事也。旧校云:“‘河’一作‘阿’。”

(8) 处,居也。

(9) 西音,周之音。

(10) 取西音以为秦国之乐音。

有娀氏有二佚女,为之九成之台, (1) 饮食必以鼓。 (2) 帝令燕往视之, (3) 鸣若谥隘。 (4) 二女爱而争搏之,覆以玉筐。少选,发而视之, (5) 燕遗二卵,北飞遂不反。 (6) 二女作歌,一终曰“燕燕往飞”,实始作为北音。 (7)

(1) 成犹重。

【校】孙云:“王逸注《离骚》引‘有娀氏有美女,为之高台而饮食之’。李善注《文选》《鲁灵光殿赋》、《齐故安陆昭王碑文》两引此文,‘为’下皆无‘之’字。”

(2) 鼓,乐。

(3) 【校】旧校云:“‘视’一作‘劾’。”

(4) 【校】孙云:“《安陆昭王碑文》注引作‘隘隘’。”

(5) 少选,须臾。

【校】梁仲子云:“《一切经音义》十三引《吕氏》作‘小选’,古‘少’、‘小’通用。”案:今《吕氏》本皆作“少选”,此与《荡兵》、《执一》诸篇皆然,无作“小”者,当亦由后人改之矣。

(6) 帝,天也。天令燕降卵于有娀氏女,吞之生契。《诗》云:“天命玄鸟,降而生商。”又曰:“有娀方将,立子生商。”此之谓也。

【校】案:《列女传》一引《诗》“有娀方将,立子生商”,亦无“帝”字。旧本作“有娀氏女方将”,因上文误衍二字,今删去。

(7) 北国之音。

凡音者,产乎人心者也。感于心则荡乎音, (1) 音成于外而化乎内。 (2) 是故闻其声而知其风, (3) 察其风而知其志, (4) 观其志而知其德。盛衰、贤不肖、君子小人皆形于乐,不可隐匿。故曰:乐之为观也,深矣。

(1) 荡,动。

(2) 内化生内心。

(3) 风,俗。

(4) 【校】旧校云:“一作‘意’,下同。”

土弊则草木不长, (1) 水烦则鱼鳖不大, (2) 世浊则礼烦而乐淫。 (3) 郑、卫之声,桑间之音, (4) 此乱国之所好,衰德之所说。 (5) 流辟、 越、慆滥之音出, (6) 则滔荡之气、邪慢之心感矣,感则百奸众辟从此产矣。故君子反道以修德, (7) 正德以出乐,和乐以成顺。 (8) 乐和而民乡方矣。 (9)

(1) 弊,恶。

(2) 扰,浑。

【校】据此注则正文本作“水扰”,后人以《乐记》之文改之。

(3) 烦,乱。淫,邪。

(4) 说见《孟春纪》。

(5) 说,乐。

(6) 出,生也。

【校】“ ”与“佻”同。

(7) 修,治也。

(8) 乐以和为成顺。

(9) 乡,仰。方,道。





制乐


四曰:

欲观至乐,必于至治。 (1) 其治厚者,其乐治厚;其治薄者,其乐治薄; (2) 乱世则慢以乐矣。今窒闭户牖,动天地,一室也。故成汤之时,有谷生于庭,昏而生,比旦而大拱。 (3) 其吏请卜其故。 (4) 汤退卜者曰:“吾闻祥者,福之先者也,见祥而为不善,则福不至。妖者,祸之先者也,见妖而为善,则祸不至。” (5) 于是早朝晏退,问疾吊丧,务镇抚百姓。三日而谷亡。 (6) 故祸兮福之所倚,福兮祸之所伏。圣人所独见,众人焉知其极? (7)

(1) 至乐,至和之乐。至治,至德之治。

(2) 【校】孙云:“李善注《文选》潘安仁《笙赋》引此‘其乐厚’、‘其乐薄’,无两‘治’字。”

(3) 《书叙》云:“伊陟相太戊,毫有桑谷祥,共生于朝。”太戊,太甲之孙,太康之子也,号为中宗。满两手曰拱。汤生仲丁,仲丁生太甲,太甲生太康,太康生太戊,凡五君矣。此云汤之时,不亦谬乎?由此观之,曝咸阳市门,无敢增损一字者,明畏不韦之势耳。故扬子云恨不及其时,车载其金而归也。

【校】“而大拱”,旧本讹作“其大拱”,梁仲子据《御览》八十三改,与《韩诗外传》正同。梁伯子云:“昏生旦拱,与《史记》言‘一暮大拱’,并理所难信。《书大传》、《汉书·五行志》、《说苑·敬慎》篇、《论衡·异虚》篇并作‘七日大拱’,《韩诗外传》三作‘三日’,当以‘七日’为是。伪孔传及《家语·五仪》篇亦作‘七日’。”

(4) 灼龟曰卜。

【校】《御览》“吏”作“史”。

(5) 为善则福应之,故祸无从至也。

【校】《外传》三以此为伊尹之言。

(6) 亡,灭。

【校】旧本“亡”讹“止”,今据《御览》改。《外传》亦作“亡”。

(7) 极犹终。

周文王立国八年, (1) 岁六月,文王寝疾五日而地动,东西南北不出国郊。 (2) 百吏皆请曰:“臣闻地之动,为人主也。今王寝疾五日而地动,四面不出周郊,群臣皆恐,曰请移之。” (3) 文王曰:“若何其移之也?”对曰:“兴事动众,以增国城,其可以移之乎!”文王曰:“不可。夫天之见妖也,以罚有罪也。我必有罪,故天以此罚我也。今故兴事动众以增国城,是重吾罪也。不可。” (4) 文王曰: (5) “昌也请改行重善以移之,其可以免乎!”于是谨其礼秩、皮革,以交诸侯;饬其辞令、 (6) 币帛,以礼豪士; (7) 颁其爵列、等级、田畴,以赏群臣。 (8) 无几何,疾乃止。 (9) 文王即位八年而地动,已动之后四十三年,凡文王立国五十一年而终。此文王之所以止殃翦妖也。 (10)

(1) 【校】《外传》三“立”作“莅”。

(2) 邑外曰郊。

(3) 【校】孙疑“曰”字衍,《外传》无。

(4) 重犹益也。移咎徵于他人,是重吾罪,故曰“不可”。

(5) 【校】语毕而更起也。《外传》作“以之”,连上“不可”为文。

(6) 饬,读如敕。饬正其辞令也。

(7) 币,圭璧。帛,玄 也。材倍百人曰豪也。

(8) 【校】旧校云:“‘赏’一作‘宾’。”

(9) 止,除也。

(10) 翦,除。

宋景公之时,荧惑在心。 (1) 公惧,召子韦而问焉,曰:“荧惑在心,何也?” (2) 子韦曰:“荧惑者,天罚也;心者,宋之分野也。祸当于君。虽然,可移于宰相。”公曰:“宰相,所与治国家也,而移死焉,不祥。” (3) 子韦曰:“可移于民。”公曰:“民死,寡人将谁为君乎?宁独死。” (4) 子韦曰:“可移于岁。”公曰:“岁害则民饥, (5) 民饥必死。为人君而杀其民以自活也,其谁以我为君乎? (6) 是寡人之命固尽已,子无复言矣。”子韦还走,北面载拜曰:“臣敢贺君。天之处高而听卑。君有至德之言三,天必三赏君。今昔荧惑其徙三舍, (7) 君延年二十一岁。”公曰:“子何以知之?”对曰:“有三善言,必有三赏,荧惑必三徙舍。 (8) 舍行七星, (9) 星一徙当一 [1] 年,三七二十一,臣故曰君延年二十一岁矣。 (10) 臣请伏于陛下以伺候之。荧惑不徙,臣请死。”公曰:“可。”是夕荧惑果徙三舍。

(1) 景公,元公佐之子栾。荧惑,五星之一,火之精也。心,东方宿,宋之分野。

(2) 子韦,宋之太史,能占宿度者,故问之。

(3) 祥,吉。

【校】注“吉”,疑本是“善”字。

(4) 《传》曰:“后非众无以守邑。”故曰“将谁为君乎”?

【校】案:“众非元后何戴,后非众罔与守邦”,此晚出《古文尚书·大禹谟》文也。汉时未有此,故诱皆以为《传》。

(5) 谷不熟为饥也。

(6) 《传》曰:“众非元后何戴。”故曰“其谁以我为君”。

(7) 【校】“今昔”本多作“今夕”,今依李本作“今昔”。昔训夜。

(8) 【校】“必三徙舍”,旧作“有三徙舍”,讹,今据《淮南·道应训》及《新序》四改正。

(9) 星,宿也。

(10) 以德复星也。徙三舍固其理也。死生有命,不可益矣。而延二十一岁,诱无闻也。





明理


五曰:

五帝三王之于乐,尽之矣。 (1) 乱国之主未尝知乐者,是常主也。 (2) 夫有天赏得为主,而未尝得主之实, (3) 此之谓大悲。 (4) 是正坐于夕室也, (5) 其所谓正,乃不正矣。 (6)

(1) 尽,极。

(2) 非贤主也。

(3) 未尝得为贤主之实。

(4) 此之为大悲哀之人。

(5) 夕室,以喻悲人也。言其室邪夕不正,徒正其坐也。

【校】梁仲子云:“《晏子春秋》六曰:‘景公新成柏寝之室,使师开鼓琴。师开左抚宫,右弹商,曰室夕云云。公曰:先君太公以营丘之封立城,曷为夕?晏子对曰:古之立国者,南望南斗,北戴枢星,彼安有朝夕哉?然而以今之夕者,周之建国,国之西方,以尊周也。’”

(6) 悲人所为,如坐夕室,自以为正,乃不正之谓也。

凡生,非一气之化也;长,非一物之任也;成,非一形之功也。故众正之所积,其福无不及也; (1) 众邪之所积,其祸无不逮也。其风雨则不适, (2) 其甘雨则不降,其霜雪则不时, (3) 寒暑则不当, (4) 阴阳失次, (5) 四时易节, (6) 人民淫烁不固, (7) 禽兽胎消不殖, (8) 草木庳小不滋, (9) 五谷萎败不成。 (10) 其以为乐也,若之何哉? (11) 故至乱之化,君臣相贼, (12) 长少相杀,父子相忍,弟兄相诬,知交相倒, (13) 夫妻相冒,日以相危,失人之纪, (14) 心若禽兽,长邪苟利, (15) 不知义理。 (16)

(1) 及,至也。

(2) 适,时也。

(3) 不当霜雪而霜雪,故曰“不时”。

(4) 不当寒而寒,不当暑而暑。

(5) 【校】旧校云:“一作‘易次’。”

(6) 谓不得其所。

【校】旧校云:“‘节’一作‘位’。”

(7) 淫邪销烁不一也。不固,不执正道。

(8) 销烁不成,不得长殖也。

(9) 滋亦长。

【校】“庳”与“卑”同。旧本作“痺”,讹,今改正。

(10) 成,熟也。

(11) 言不可以为乐,故曰“若之何哉”。

(12) 君不君,臣不臣,故相贼。

(13) 倒,逆。

(14) 冒,嫉。危,疑。相嫉则相猜疑,故失人道之纲纪。

【校】案:“日以相危,失人之纪”,乃统承上文,不专以夫妻言。注非。

(15) 【校】旧校云:“一作‘苟且’。”

(16) 乱政之化也,心如禽兽,焉知义理。

其云状有若犬、若马、若白鹄、若众车; (1) 有其状若人,苍衣赤首,不动,其名曰天衡; (2) 有其状若悬旍而赤,其名曰云旍; (3) 有其状若众马以斗,其名曰滑马; (4) 有其状若众植华以长, (5) 黄上白下,其名蚩尤之旗。 (6) 其日有斗蚀,有倍僪,有晕珥, (7) 有不光,有不及景, (8) 有众日并出,有昼盲, (9) 有霄见。 (10) 其月有薄蚀, (11) 有晖珥,有偏盲,有四月并出,有二月并见, (12) 有小月承大月,有大月承小月,有月蚀星,有出而无光。其星有荧惑, (13) 有彗星,有天棓,有天欃,有天竹,有天英,有天干, (14) 有贼星,有斗星,有宾星。其气有上不属天,下不属地, (15) 有丰上杀下,有若水之波,有若山之楫, (16) 春则黄,夏则黑,秋则苍,冬则赤。其妖孽有生如带,有鬼投其陴, (17) 有菟生雉,雉亦生 , (18) 有螟集其国,其音匈匈, (19) 国有游蛇西东, (20) 马牛乃言, (21) 犬彘乃连, (22) 有狼入于国, (23) 有人自天降, (24) 市有舞鸱,国有行飞, (25) 马有生角, (26) 雄鸡五足, (27) 有豕生而弥, (28) 鸡卵多毈, (29) 有社迁处, (30) 有豕生狗。 (31) 国有此物,其主不知惊惶亟革,上帝降祸,凶灾必亟。 (32) 其残亡死丧,殄绝无类,流散循饥无日矣。 (33) 此皆乱国之所生也,不能胜数,尽荆、越之竹,犹不能书。 (34) 故子华子曰:“夫乱世之民,长短颉 ,百疾, (35) 民多疾疠,道多褓襁,盲秃伛尪,万怪皆生。” (36) 故乱世之主,乌闻至乐? (37) 不闻至乐,其乐不乐。 (38)

(1) 云气形状如物之形也。

(2) 衡物之气。

【校】《御览》八百七十七作“天冲”。

(3) 云气之象旍旗者。

【校】“悬旍”,旧本作“悬釜”,讹。案:《御览》作“悬旌”,“旌”与“旍”同,今定为“旍”字。

(4) 《五行传》为马妖也。

(5) 【校】旧校云:“‘华’一作‘藿’。”

(6) 【校】旧本作“蚩尤之旍”,又作“蚩尤之旍旗”,皆讹。今据《史记·天官书》、《汉书·天文志》改正。《集解》及师古注并引晋灼曰:“《吕氏春秋》云其色黄上白下。”

(7) 斗蚀,两日共斗而相食。倍僪、晕珥,皆日旁之危气也;在两旁反出为倍,在上反出为僪,在上内向为冠,两旁内向为珥。晕,读为君国子民之君。气围绕日周匝,有似军营相围守,故曰“晕”也。

【校】“倍僪”亦作“背鐍”,又作“背谲”,《汉志》作“背穴”。

(8) 【校】旧校云:“‘及’一作‘反’。”

(9) 盲,冥也。

(10) 霄,夜。见,明。

【校】“霄”,当是“宵”之借。

(11) 薄,迫也。日月激会相掩,名为薄蚀。

【校】“其月”,旧本作“其日”,误,今改正。

(12) 并犹俱也。

(13) 荧惑,火精。

(14) 干,楯也。

(15) 属犹至。

(16) 楫,林木也。

(17) 陴,脚也,音“杨子爱骭一毛”之骭。

【校】案:陴字音义皆可疑,或是骨幹之幹,则是脊胁也,与骭音正同,但不当训为脚耳。

(18) ,一名冠爵,于《五行传》羽虫之孽。

(19) 食心为螟。音声飞匈匈,惊动众人,集其国都也。

(20) 于《五行传》为蛇妖也。西东,示民流迁,国不安宁也。

(21) 言,语。皆妖也。

(22) 连,合。皆妖也。

【校】汉孝景二年有此。

(23) 国,都也。《河图》曰:“野鸟入,主人亡也。”

(24) 降,下。人,妖也。

(25) 【校】旧校云:“一作‘彘’。”

(26) 于《五行传》为马祸。

(27) 羽虫之孽。

(28) 弥,蹄不甲也。于《五行传》为青黑之祥也。

【校】注旧本“青黑”上有“墨”字衍。

(29) 【校】案:《说文》“毈,卵不孚也,徒玩切”。旧本作“假”,讹,今改正。《淮南·原道训》、《法言·先知》篇俱有“毈”字。

(30) 迁,移。

【校】案:《史记·六国表》:“秦惠文君二年宋太丘社亡。”

(31) 于《五行传》为豕祸。

(32) 乱惑之主,见妖孽之怪,不知惊惶疾自革更,共御厥罚,故上帝降之祸,凶灾必至。

【校】“共御”,旧本作“共卫”,讹,今从《书大传》改正。

(33) 循,大也。谷不熟曰饥。无复有期日也。

(34) 楚、越,竹所出也,尚不能胜书者,妖多也。

(35) 疾,病也。长短者,无节度也。颉犹大。 ,逆也。百疾,变诈也。既无节度,大逆为变诈之疾也。

【校】案:《庄子·徐无鬼》篇“颉滑有实”,向秀注:“颉滑,错乱也。”此“颉 ”疑与“颉滑”义同。注“颉犹大”,旧本作“ 犹大”,讹。又“逆”作“迎”,亦讹。今并改正。

(36) 褓,小儿被也。襁,褛格绳也。言民襁负其子走道,跛而散去。盲,无见。秃,无发。伛,偻俯者也。尪,短仰者也。怪物妄生非一类,故言万怪者也。

【校】注“襁,褛格绳也”,旧本“格”作“袷”,又作“拾”,下又衍一“上”字,皆讹。案:褛格即缕络,《方言》“络谓之格”,义得通也,后《直谏》篇注作“缕格”。段若膺云:“织缕为络,其绳谓之襁。”梁仲子云:“《论语》‘襁负’,疏引《博物志》云‘织缕为之’。又《汉书·宣帝纪》注:‘李奇曰:襁,络也。’”

(37) 乌,安也。

【校】旧校云:“‘乌’一作‘焉’。”

(38) 乱国之乐怨以悲,不闻至德之乐,故曰“其乐不乐”也。




————————————————————

[1] 一:原本作“七”,误,据许维遹本改。





第七卷 孟秋纪



孟秋


一曰:

孟秋之月, (1) 日在翼, (2) 昏斗中,旦毕中。 (3) 其日庚辛,其帝少皞, (4) 其神蓐收。 (5) 其虫毛,其音商, (6) 律中夷则,其数九。 (7) 其味辛,其臭腥, (8) 其祀门,祭先肝。 (9) 凉风至,白露降, (10) 寒蝉鸣,鹰乃祭鸟,始用行戮。 (11) 天子居总章左个, (12) 乘戎路,驾白骆, (13) 载白旂,衣白衣,服白玉, (14) 食麻与犬,其器廉以深。 (15)

(1) 【校】旧此下有“长日至四旬六日”七字,又注云“夏至后,日尚长,至四十六日立秋,昼夜等,故曰长日至四旬六日”二十五字,于文不类。且后文自有注,不应预出。立秋时亦不得云昼夜等。谢以辞义俱浅陋,定为俗师所加。今从《月令》删去。

(2) 孟秋,夏之七月。翼,南方宿,楚之分野。是月,日躔此宿。

(3) 斗,北方宿,吴之分野。毕,西方宿,赵之分野。是月昏旦时,皆中于南方。

【校】正文旧又衍“则立秋”三字,《月令》无,今并删去。又注“毕,赵之分野”,旧“赵”讹作“越”。案《淮南·天文训》则属魏。

(4) 庚辛,金日也。少暤,帝喾之子挚兄也。以金德王天下,号为金天氏,死配金,为西方金德之帝。

(5) 少暤氏裔子曰该,皆有金德,死托祀为金神。

【校】注“皆”字疑当作“实”。

(6) 金气寒,裸者衣毛。毛,虫之属,而虎为之长。商,金也,其位在西方。

【校】注“裸者衣毛”,旧本脱“毛”字,今从《淮南》注补。

(7) 夷则,阳律也,竹管音与夷则和,太阳气衰,太阴气发,万物肃然,应法成性,故曰“律中夷则”。其数九,五行数五,金第四,故曰“九”。

【校】梁仲子云:“《初学记》引注‘气衰’作‘力衰’,‘肃然’作‘雕伤’。”

(8) 五行,金味辛,金臭腥。

(9) 孟秋始内,由门入,故祀门也。肝,木也。祭祀之肉用其胜,故先进肝。又曰:肝,金也,自用其藏也。

(10) 凉风,坤卦之风,为损。降下。

(11) 寒蝉得寒气,鼓翼而鸣,时候应也。是月鹰挚杀鸟于大泽之中,四面陈之,世谓之祭鸟。于是时乃始行戮,刑罚顺秋气。

【校】“始用”,《月令》、《淮南》皆作“用始”。此误倒也。高注《淮南》云“用是时乃始行戮”,语尤明。

(12) 总章,西向堂也。西方总成万物,章明之也,故曰“总章”。左个,南头室也。

(13) 戎路,白路也。白马黑鬣曰骆。

(14) 白,顺金也。

(15) 犬,金畜也。廉,利也,象金断割。深,象阴闭藏。

是月也,以立秋。先立秋三日,大史谒之天子, (1) 曰:“某日立秋,盛德在金。”天子乃斋。 (2) 立秋之日,天子亲率三公、九卿、诸侯、大夫以迎秋于西郊。 (3) 还,乃赏军率武人于朝。 (4) 天子乃命将帅,选士厉兵,简练桀俊, (5) 专任有功,以征不义, (6) 诘诛暴慢,以明好恶,巡彼远方。 (7)

(1) 夏至后四十六日立秋,多在是月。谒,告也。

(2) 盛德在金,金主西方也。斋,自禋洁。

(3) 九里之郊。

(4) 金气用事,治兵讨暴,非率不整,非武不齐,故赏军将与武人于朝,与众共之。

(5) 材过万人曰桀,千人曰俊。

【校】旧本“选”误“还”,又脱“士”字,今从汪本据《月令》补正。《淮南》作“选卒”。

(6) 征,正也。

(7) 巡,行也。远方,天下也。

【校】“巡”,《月令》、《淮南》作“顺”。

是月也,命有司,修法制, (1) 缮囹圄,具桎梏,禁止奸, (2) 慎罪邪,务搏执。命理瞻伤察创,视折审断, (3) 决狱讼,必正平,戮有罪,严断刑。 (4) 天地始肃,不可以赢。 (5)

(1) 禁令。

(2) 囹圄,法室。桎梏谓械,在足曰桎,在手曰梏,所以禁止人之奸邪。

(3) 慎,戒。有奸罪者搏执之也。理,狱官也。使视伤创毁折者可断之,故曰“审断”。

(4) 争罪曰狱,争财曰讼。决之者必得其正平,不直者戮而刑之。

【校】“正平”,《月令》作“端平”,此反不为始皇讳。

(5) 肃,杀。素气始行,不可以骄赢。犯令必诛,故曰“不可以赢”。

【校】注“犯令必诛”以下乃后人所妄加。高氏本以“赢”与“盈”同。夏日长赢,今当秋收敛之候,不可以骄盈也。《淮南》注“赢,盛也”,义亦相似。《月令》郑注云“解也”,以肃为严急,故不可以舒缓,与骄盈意亦未尝不相近也。

是月也,农乃升谷,天子尝新,先荐寝庙。 (1) 命百官,始收敛。 (2) 完堤防,谨壅塞,以备水潦。 (3) 修宫室,附墙垣,补城郭。 (4)

(1) 升,进也。先致寝庙,《孝经》曰:“四时祭祀,不忘亲也。”

(2) 收敛,孟秋始内。

(3) 是月,月丽于毕,俾雨滂沱,故预完堤防,备水潦。

(4) 附,读如符。附犹培也。

【校】《月令》“附”作“坏”。

是月也,无以封侯、立大官, (1) 无割土地,行重币,出大使。 (2)

(1) 封侯,裂土封之邑也。大官,谓上公九命之官。

(2) 无割土地,以地赐人。重币,金帛之币。大使,使命也。方金气之收藏,皆所不宜行也。

行之是令,而凉风至三旬。孟秋行冬令,则阴气大胜,介虫败谷,戎兵乃来。 (1) 行春令,则其国乃旱,阳气复还,五谷不实。 (2) 行夏令,则多火灾,寒热不节,民多疟疾。 (3)

(1) 冬,水王,而行其令,故阴气大胜也。介虫,龟属。冬,玄武,故介甲之虫败其谷也。金水相并,则戎兵来侵为害。

(2) 春阳亢燥,而行其令,故枯旱也。是月凉风用事,而行春暖之令,而谷更生,故害而不成实也。

【校】“复还”本或作“后还”,误,今从汪本,与《月令》、《淮南》皆合。

(3) 夏,火王,而行其令,故多火灾。金气,火气。寒热相干不节,使民病疟疾。寒热所生。

【校】《月令》作“则国多火灾”。《淮南》作“冬多火灾”。





荡兵 (1)


(1) 【校】一作“用兵”。

二曰:

古圣王有义兵而无有偃兵。 (1) 兵之所自来者上矣, (2) 与始有民俱。 (3) 凡兵也者,威也;威也者,力也。民之有威力,性也。性者所受于天也,非人之所能为也。武者不能革, (4) 而工者不能移。 (5) 兵所自来者久矣,黄、炎故用水火矣, (6) 共工氏固次作难矣, (7) 五帝固相与争矣,递兴废,胜者用事。人曰 (8) “蚩尤作兵”,蚩尤非作兵也,利其械矣。 (9) 未有蚩尤之时,民固剥林木以战矣,胜者为长。 (10) 长则犹不足治之,故立君。 (11) 君又不足以治之,故立天子。天子之立也出于君,君之立也出于长,长之立也出于争。 (12) 争斗之所自来者久矣,不可禁,不可止。 (13) 故古之贤王有义兵而无有偃兵。

(1) 偃,止。

(2) 自,从。上,古。

【校】旧校云:“‘上’一作‘久’。”

(3) 俱,皆。

(4) 革,更。

(5) 移,易。

(6) 黄,黄帝;炎,炎帝也。炎帝为火灾,黄帝灭之也。

【校】《御览》二百七十一“故”作“固”,下文皆作“固”。案“故”、“固”古亦通用。

(7) 共工之治九州,有异高辛氏,争为帝而亡,故曰“次作难也”。

【校】《御览》“次”作“欲”。

(8) 【校】旧本作“又曰”,今从《御览》改。

(9) 蚩尤,少皞氏之末,九黎之君名也。始作乱,伐无罪,杀无辜,善用兵,为之无道,非始造之也,故曰“非作兵也”。

【校】《御览》“矣”作“也”。

(10) 长,率。

(11) 立,置也。

(12) 战胜而为长,故曰“出于争”。

(13) 天生五材,民并用之,废一不可,谁能去兵?兵之来久矣,圣人以治,乱人以亡,废兴、存亡、昏明之术也,故曰“不可禁,不可止”。

【校】注本子罕语,见襄廿七年《左传》。

家无怒笞,则竖子婴儿之有过也立见; (1) 国无刑罚,则百姓之悟相侵也立见; (2) 天下无诛伐,则诸侯之相暴也立见。 (3) 故怒笞不可偃于家,刑罚不可偃于国,诛伐不可偃于天下,有巧有拙而已矣。 (4) 故古之圣王有义兵而无有偃兵。

(1) 家无严亲怒笞之威,则小子好争上下,过立著见也。

(2) 无刑罚可畏,臣下故有相侵凌夺掠之罪。

【校】“悟”与“忤”、“牾”并通用。《史记·韩非传》“大忠无所拂悟”,《索隐》云:“不拂‘牾’于君。”《正义》云:“‘拂悟’,当为‘咈忤’,古字假借耳。”今本《史记》作“拂辞”,误也。朱本于此书又删去“悟”字,轻改古书,最不可训。

(3) 无诛伐可畏,故相暴,大兼小也。

(4) 巧者以治,拙者以乱。

夫有以 死者,欲禁天下之食,悖; (1) 有以乘舟死者,欲禁天下之船,悖;有以用兵丧其国者,欲偃天下之兵,悖。夫兵不可偃也,譬之若水火然, (2) 善用之则为福,不能用之则为祸; (3) 若用药者然,得良药则活人,得恶药则杀人。义兵之为天下良药也亦大矣。 (4) 且兵之所自来者远矣,未尝少选不用。贵贱长少、贤者不肖相与同,有巨有微而已矣。 (5) 察兵之微,在心而未发,兵也;疾视,兵也;作色,兵也;傲言,兵也;援推, (6) 兵也;连反, (7) 兵也;侈斗, (8) 兵也;三军攻战,兵也:此八者皆兵也,微巨之争也。今世之以偃兵疾说者,终身用兵而不自知悖,故说虽强,谈虽辨,文学虽博,犹不见听。 (9) 故古之圣王有义兵而无有偃兵。

(1) 悖,惑。

(2) 水以疗汤,火以熟食,兵以除乱,夫何偃也?故曰“若水火然”。

【校】注“熟食”,旧本“熟”多作“热”,讹,唯朱本作“熟”,此可从。

(3) 《传》曰“能者养之以求福,不能者败之以取祸”,此之谓。

【校】案:《左氏成十三年传》刘子言“能者养以之福,不能者败以取祸”。《汉书》《律志》、《五行志》,汉《酸枣令刘熊碑》,皆作“养以之福”,孔疏、颜注莫不同。今本则作“养之以福”,此注颇与今本同。凡注家引书,诚不必尽符本文,然此颇有后人妄改痕迹。缘高氏本作“养以之福”,读者不解,因改为“求福”,而以“之”字移于上,又于次句亦增一“之”字,以成对文。末句“此之谓也”,删去“也”字,则必刻成之后,就板增两字,而末句只有一字之空,故并“也”字去之,始整齐耳。元和陈芳林云:“改‘之福’为‘求福’,则非定命矣。”斯言允哉。

(4) 义兵除天下之凶残,解百姓之倒悬而生育之,故方之于良药。

(5) 少选,须臾也。贤不肖者用兵,皆欲得胜,是其同也。巨,觕略。微,要妙,睹未萌之萌也。

(6) 【校】案:援推,义当与推挽同,或援之使来,或推之使去,有分别,见即兵象矣。旧校云“一作‘挂刺’”,所未能详也。

(7) 【校】“连反”当出《易·蹇》爻辞。连,与人也;反,自守也;有同有异而兵兴矣。旧校云“‘连’一作‘速’”,疑误。

(8) 【校】犹斗侈也,谓以豪侈相争胜也。

(9) 虽以辨文博学,力说偃兵,不自知博者,不听从也。

【校】注“博者”字讹,或“博”是“悖”字,下亦当有一“人”字。

兵诚义,以诛暴君而振苦民, (1) 民之说也,若孝子之见慈亲也,若饥者之见美食也;民之号呼而走之, (2) 若强弩之射于深溪也,若积大水而失其壅堤也。中主犹若不能有其民,而况于暴君乎? (3)

(1) 【校】旧校云:“一作‘弱民’。”

(2) 走,归。

(3) 中主,非贤君。





振乱


三曰:

当今之世浊甚矣, (1) 黔首之苦不可以加矣。 (2) 天子既绝,贤者废伏, (3) 世主恣行,与民相离,黔首无所告愬。 (4) 世有贤主、秀士,宜察此论也,则其兵为义矣。 (5) 天下之民,且死者也而生, (6) 且辱者也而荣, (7) 且苦者也而逸。 (8) 世主恣行,则中人将逃其君,去其亲,又况于不肖者乎? (9) 故义兵至,则世主不能有其民矣,人亲不能禁其子矣。 (10)

(1) 浊,乱也。君肆害,不可禁卫,故乱甚。

【校】注“禁卫”疑亦是“禁御”。

(2) 民人之苦毒,不可复增加。

(3) 绝,若三代之末,祚数尽也。贤者不见用,故废伏。

【校】赵云:“天子既绝,谓周已亡而秦未称帝之时也。”

(4) 世主,乱主也。乱政亟行,与民相违,黔首怀怨,无所控告。

(5) 贤主,治主也。秀士,治士也。宜察恣行之主与民相离怨而舍之也,必举兵诛之。诛其君,吊其民,故曰“其兵为之义”也。

【校】注“为之义”,疑“之”字衍,或“为”字当作“谓”。

(6) 且,将也。治主之兵救其民,故曰将生也。

【校】且,将也。旧本作“行也”,讹,今改正。

(7) 荣,光明也。

(8) 民见吊恤安逸。

(9) 遭恣行之君,中凡之人将逃而去之,不能顾其亲戚也,又况下愚不肖之人,能保守其君而不逃去其亲者也?

(10) 世主暴乱,若桀、纣者也,民去之而归汤、武,故不能得其有也,其亲不能禁止其子。

凡为天下之民长也,虑莫如长有道而息无道,赏有义而罚不义。今之世学者多非乎攻伐。非攻伐而取救守,取救守则乡之所谓长有道而息无道、赏有义而罚不义之术不行矣。天下之长民,其利害在 (1) 察此论也。攻伐之与救守一实也, (2) 而取舍人异。 (3) 以辨说去之,终无所定论。固不知,悖也;知而欺心,诬也。 (4) 诬悖之士,虽辨无用矣。 (5) 是非其所取而取其所非也,是利之而反害之也,安之而反危之也。 (6) 为天下之长患,致黔首之大害者,若说为深。 (7) 夫以利天下之民为心者,不可以不熟察此论也。 (8)

(1) 【校】旧校云:“一本下有‘此’字。”朱本从之。今案:“在察此论也”连下读为是,观下文可见。

(2) 攻伐欲陷人,救守欲完人,其实一也。

(3) 攻伐欲破人,救守欲全人,故曰“取舍人异”。

(4) 论说事情,固不知之,是为悖;实知之而自欺其心,是为诬。

(5) 辨无所能施,故谓之“无用”。

【校】赵云:“言说虽若可听,而断不可用也。下文申言其故。”

(6) 民以为安,而安之以礼义也;危之,乃危亡之道也;故曰安之反危也。

【校】言非攻伐欲以安利之,而不知其适相反。

(7) 说若是者,为天下之患,为黔首之害深而大也。

(8) 论犹别也。

【校】“别”即“辩”,古通用。

夫攻伐之事,未有不攻无道而罚不义也。攻无道而伐不义,则福莫大焉,黔首利莫厚焉。 (1) 禁之者, (2) 是息有道而伐有义也,是穷汤、武之事,而遂桀、纣之过也。 (3) 凡人之所以恶为无道不义者,为其罚也; (4) 所以蕲有道行有义者,为其赏也。 (5) 今无道不义存,存者,赏之也; (6) 而有道行义穷,穷者,罚之也。 (7) 赏不善而罚善,欲民之治也,不亦难乎? (8) 故乱天下害黔首者,若论为大。 (9)

(1) 厚,重也。

(2) 禁,止也。

(3) 遂犹长也。

(4) 恶犹畏。

(5) 蕲,读曰祈。或作“勤”。

(6) 虽存,幸耳,赏之非也。

(7) 虽穷,不幸耳,罚之非也。

【校】注皆不得本意,此所云赏罚,岂真赏之罚之也哉?使无道者安全,即不啻赏之;使有道者不得伸天讨,即不啻罚之矣。

(8) 治,整也。

(9) 论若是者,赏所当罚者,罚所当赏者,是以乱天下而害黔首最为大也。

【校】案:此篇之论,其谓天下攻伐人者之皆义兵乎?苟非义兵,则能救守者,正《春秋》之所深嘉而乐予也,而此非之,是与圣贤之意相违矣。下篇虽稍持平,然亦偏主攻伐意多。





禁塞


四曰:

夫救守之心,未有不守无道而救不义也。守无道而救不义,则祸莫大焉, (1) 为天下之民害莫深焉。 (2) 凡救守者,太上以说, (3) 其次以兵。 (4) 以说则承从多群, (5) 日夜思之,事心任精,起则诵之,卧则梦之。自今单唇乾肺,费神伤魂, (6) 上称三皇五帝之业以愉其意, (7) 下称五伯名士之谋以信其事, (8) 早朝晏罢,以告制兵者, (9) 行说语众,以明其道。道毕说单而不行, (10) 则必反之兵矣。 (11) 反之于兵,则必斗争之情,必且杀人, (12) 是杀无罪之民,以兴无道与不义者也。无道与不义者存,是长天下之害, (13) 而止天下之利, (14) 虽欲幸而胜,祸且始长。 (15) 先王之法曰,“为善者赏,为不善者罚。”古之道也,不可易。 (16) 今不别其义与不义,而疾取救守,不义莫大焉,害天下之民者莫甚焉。故取攻伐者不可,非攻伐不可, (17) 取救守不可, (18) 非救守不可。 (19) 取惟义兵为可。 (20) [1] 兵苟义,攻伐亦可, (21) 救守亦可; (22) 兵不义,攻伐不可, (23) 救守不可。 (24)

(1) 莫,无也。无有大之者。

(2) 深,重也。无有重之者。

(3) 说,说言也。

【校】注当是“说以言也”,次“说”字讹。

(4) 以兵威之。

(5) 【校】旧校云:“‘从’一作‘徒’。”

(6) 单,尽。乾,晞。费,损。神,人之神也。魂,人之阳精也。阳精为魂,阴精为魄。

【校】“自今”,疑本是“自令”。

(7) 愉,悦。

(8) 信,明也。其说救守之事。

(9) 制者,主也。谓敌之主兵者。

(10) 毕、单皆尽。不行,不见从。

(11) 说不见从,故反之以兵威之。

(12) 【校】“斗争”二字当叠。

(13) 为天下之害者得滋长。

(14) 晋献公曰:“物不两施。”害长故利止者也。

(15) 晋献公伐丽戎,史苏曰:“胜而不吉。”故曰祸乃始长也。

(16) 易犹违。

(17) 于义可攻可伐,故不可非也。

(18) 于义不可攻不可伐,故不可取,惟义所在。

(19) 于义当救当守,故不可非。

(20) 于义当守当救,不可取而有之也。

【校】此“救守不可取”五字乃衍文。注亦无异前说,皆当删去。

(21) 以有道攻伐无道,故《司马法》曰“以战去战,虽战可也”,此之谓也。

(22) 谓诸侯思启封疆,以无道攻有道,虽救之可也,极困设守亦可也。

(23) 若以桀、纣之兵攻伐汤、武,曷当可乎?

(24) 桀、纣坚守而往救之,亦不可也。

使夏桀、殷纣无道至于此者,幸也;使吴夫差、智伯瑶侵夺至于此者,幸也; (1) 使晋厉、陈灵、宋康不善至于此者,幸也。 (2) 若令桀、纣知必国亡身死,殄无后类,吾未知其厉为无道之至于此也。吴王夫差、智伯瑶知必国为丘墟,身为刑戮,吾未知其为不善无道侵夺之至于此也。 (3) 晋厉知必死于匠丽氏, (4) 陈灵知必死于夏徵舒, (5) 宋康知必死于温,吾未知其为不善之至于此也。 (6) 此七君者,大为无道不义,所残杀无罪之民者,不可为万数; (7) 壮佼、老幼、胎 之死者, (8) 大实平原,广堙深溪、大谷,赴巨水,积灰,填沟洫险阻,犯流矢,蹈白刃,加之以冻饿饥寒之患,以至于今之世,为之愈甚。故暴骸骨无量数, (9) 为京丘若山陵。 (10) 世有兴主仁士,深意念此,亦可以痛心矣,亦可以悲哀矣。 (11)

(1) 夫差,吴王阖闾之子。智伯,智宣子之子襄子也。

(2) 晋厉公,景公之子州蒲也。陈灵公,共公之子平国也。宋康王,在春秋后,当战国时,僭称王。

【校】案:厉公实名州满,《史记》作“寿曼”,声同耳。梁伯子云:“《左传成十年正义》引应劭《讳议》云:‘周穆王名满,而有晋侯州满。’”

(3) 夫差、智伯为无道,侵夺无厌。夫差为越王句践所灭,智伯为襄子所杀于晋阳之下也。

(4) 匠丽氏,晋大夫家也。厉公无道,栾书、中行偃杀之于匠丽氏也。

(5) 夏徵舒,陈大夫御叔之子,夏姬所生也。灵公通于夏姬,与孔宁、仪行父饮酒于夏氏。徵舒过之,公谓行父曰:“徵舒似汝。”对曰:“亦似君。”徵舒病之。公出自其厩,射而杀之,故曰“死于夏徵舒”。

(6) 温,魏邑也。宋康王,名偃,宋元公佐六世之孙,辟兵之子也。立十一年,自为王。东败齐,取五城;南败楚,取三百里;西败魏军于温;与齐、楚、魏为敌国。以韦囊盛血,悬而射之,号曰射天。诸侯患之,咸曰宋复为纣,不可不诛。即位四十七年,齐湣王与楚、魏伐宋,遂灭之,而三分其地,故曰“死于温”。

【校】“宋康”,《荀子·王霸》篇作“宋献”,杨倞云:“国灭之后,其臣子各私为谥,故不同。”案:此注依《宋世家》为说。“四十七年”,《年表》偃立止四十三年。梁伯子云“宋实无取齐、楚地及败魏军之事”,详所著《史记刊误》中。

(7) 万人一数之,言多不可胜数。

【校】“大为无道”,旧本“为”误作“而”,今改正。

(8) 【校】“ ”与“ ”同。

(9) 言多。

(10) 战斗杀人,合土筑之,以为京观,故谓之京丘,若山林高大也。

(11) 哀亦痛也。

察此其所自生,生于有道者之废,而无道者之恣行。 (1) 夫无道者之恣行,幸矣。 (2) 故世之患,不在救守,而在于不肖者之幸也。 (3) 救守之说出,则不肖者益幸也,贤者益疑矣。 (4) 故大乱天下者,在于不论其义而疾取救守。 (5)

(1) 恣,放也。

(2) 无道者恣其情欲而见信用,不得诛灭,是乃幸也。

(3) 【校】正文似讹,当云“故世之患,在于救守,而为不肖者之幸也”,如此方与上下文顺。

(4) 疑怪其何以益幸也。

(5) 疾犹争。





怀宠


五曰:

凡君子之说也,非苟辨也;士之议也,非苟语也。必中理然后说, (1) 必当义然后议。 (2) 故说义而王公大人益好理矣,士民黔首益行义矣。 (3) 义理之道彰,则暴虐奸诈侵夺之术息也。 (4) 暴虐奸诈之与义理反也,其势不俱胜,不两立。故兵入于敌之境, (5) 则民知所庇矣, (6) 黔首知不死矣。 (7) 至于国邑之郊,不虐五谷,不掘坟墓,不伐树木,不烧积聚,不焚室屋,不取六畜。得民虏奉而题归之, (8) 以彰好恶; (9) 信与民期,以夺敌资。 (10) 若此而犹有忧恨冒疾遂过不听者,虽行武焉亦可矣。

(1) 理,义。

(2) 议,言。

(3) 一命为士民。士民之说为士者也。

(4) 彰,明。息,灭。

(5) 境,壤。

(6) 庇,依荫也。

(7) 知义兵救民之命,不杀害。

(8) 奉,送也。

(9) 好其颛民,恶其恶君也。《传》曰:“其君是恶,其民何罪?”此之谓也。

(10) 以信与民期,不违之也。资,用也。敌以暴虐用其民,故以信义夺其民也。

先发声出号曰: (1) 兵之来也,以救民之死。 (2) 子之在上无道, (3) 据傲荒怠,贪戾虐众,恣睢自用也,辟远圣制,謷丑先王,排訾旧典,上不顺天, (4) 下不惠民, (5) 征敛无期,求索无厌, (6) 罪杀不辜,庆赏不当。若此者,天之所诛也,人之所仇也,不当为君。今兵之来也,将以诛不当为君者也,以除民之仇而顺天之道也。 (7) 民有逆天之道,卫人之仇者,身死家戮不赦。 (8) 有能以家听者,禄之以家; (9) 以里听者,禄之以里; (10) 以乡听者,禄之以乡; (11) 以邑听者,禄之以邑; (12) 以国听者,禄之以国。 (13) 故克其国,不及其民, (14) 独诛所诛而已矣。 (15) 举其秀士 (16) 而封侯之, (17) 选其贤良而尊显之, (18) 求其孤寡而振恤之, (19) 见其长老而敬礼之。 (20) 皆益其禄,加其级。 (21) 论其罪人而救出之。 (22) 分府库之金,散仓廪之粟, (23) 以镇抚其众,不私其财。问其丛社大祠,民之所不欲废者而复兴之, (24) 曲加其祀礼。是以贤者荣其名,而长老说其礼,民怀其德。 (25)

(1) 号,令。

(2) 死,命。

(3) 子,谓所伐国之君。

【校】“据”当与“倨”通,朱本即作“倨”。

(4) 顺,承。

(5) 惠,爱。

(6) 期,度。厌,足。

【校】注旧作“其度厌之”,讹,今改正。

(7) 【校】旧校云:“‘天’一作‘民’。”

(8) 卫犹护助也。救无道之君,则身死家戮不赦贷也。

【校】孙云:“‘不赦’,旧本误作‘不救’。注‘赦贷’,旧本误作‘救贰’。”今并从孙说改正。

(9) 以一家禄之。

(10) 里,闾也。《周礼》“五家为比,五比为闾”,闾,二十五家。

(11) 《周礼》“二千五百家为州,五州为乡”,乡,万二千五百家。

(12) 《周礼》“八家为井,四井为邑”,三十二家也。此上乡邑皆不从《周礼》。

(13) 国,都也。《周礼》“二千五百家为县,四县为都”,然则国都万家也。

(14) 克,胜。及,罪。

(15) 所诛,君也。

(16) 【校】旧校云:“一作‘俊’。”案:非是。

(17) 秀士,俊士。

(18) 授以上位。

(19) 无子曰孤,无夫曰寡。振,赡。矜,恤。

(20) 尊高年。

(21) 禄,食。级,等。

(22) 论犹理。

【校】“救”,亦当作“赦”。

(23) 金,铁也,可以为田器,皆布散以与人民。

(24) 兴之,举其祀。

(25) 怀,安也。

今有人于此,能生死一人, (1) 则天下必争事之矣。 (2) 义兵之生一人亦多矣, (3) 人孰不说?故义兵至,则邻国之民归之若流水, (4) 诛国之民望之若父母,行地滋远,得民滋众, (5) 兵不接刃而民服若化。 (6)

(1) 生,活也。

(2) 事此一人。

(3) 【校】案:“一”字衍。

(4) 民归之若流水,不可壅御也。

(5) 所诛国之民,睎望义兵之至,若望其父母。滋,益;众,多也。《孟子》曰:“百姓箪食壶浆以迎王师,奚为后予?”此之谓也。

(6) 接,交。若被其化也。

【校】“若化”,本多作“其化”,今从宋邦乂本。




————————————————————

[1] 按:陈昌齐曰:据文义当以四“不可”截句,高氏误读误注。





第八卷 仲秋纪



仲秋


一曰:

仲秋之月,日在角, (1) 昏牵牛中,旦觜巂中。 (2) 其日庚辛,其帝少皞,其神蓐收,其虫毛,其音商, (3) 律中南吕, (4) 其数九。其味辛,其臭腥,其祀门,祭先肝。凉风生, (5) 候雁来,玄鸟归,群鸟养羞。 (6) 天子居总章太庙, (7) 乘戎路,驾白骆,载白旂,衣白衣,服白玉,食麻与犬,其器廉以深。 (8)

(1) 仲秋,夏之八月。角,东方宿,韩、郑之分野。是月,日躔此宿。

(2) 牵牛,北方宿,越之分野。觜巂,西方宿,魏之分野。是月昏旦时,皆中于南方。

【校】案《淮南·天文训》,觜巂属赵。

(3) 说在《孟秋》。

(4) 南吕,阴律。是月,阳气内藏,阴旅于阳,任其成功,竹管音中南吕。

(5) 说在《孟秋》。

【校】《月令》作“盲风至”,郑注:“盲风,疾风也。”孙云:“《孟秋》已云凉风至,此何以又云凉风生?应从《记》。”

(6) 是月候时之雁从北漠中来,南过周、雒之彭蠡。玄鸟,燕也,春分而来,秋分而去,归蛰所也。《传》曰:“玄鸟氏,司分者也。”寒气将至,群鸟养进其毛羽御寒也,故曰“群鸟养羞”。

【校】注“北漠”各本作“北汉”,讹,今从汪本,与《淮南》注合。郑注《月令》云“羞谓所食也”,此注又别。

(7) 总章,西向堂。太庙,中央室也。

(8) 说在《孟秋》。

是月也,养衰老,授几杖,行麋粥饮食。 (1) 乃命司服,具饬衣裳,文绣有常,制有小大,度有短长,衣服有量,必循其故,冠带有常。 (2) 命有司,申严百刑,斩杀必当, (3) 无或枉桡,枉桡不当,反受其殃。 (4)

(1) 阴气发,老年衰,故共养之。授其几杖,赋行饮食麋粥之礼。今之八月,比户赐高年鸠杖粉粢是也。《周礼》,大罗氏掌献鸠杖以养老,又伊耆氏掌共老人之杖。

【校】“麋”与“糜”同,本亦作“糜”。《周礼》罗氏掌献鸠以养国老,《礼记·郊特牲》有大罗氏,此参用彼文,衍“杖”字,缺“国”字。《周礼》伊耆氏共王之齿杖,郑注:“王之所以赐老者之杖。”

(2) 司服,主衣服之官。将饬正衣服,故命之也。上曰衣,下曰裳。青与赤五色备谓之绣。《周礼》:“司服掌王之吉服。祀昊天上帝则大裘而冕,祀五帝亦如之。享先王则衮冕,享先公飨射则 冕,祀四望山川则毳冕,祭社稷五祀则 冕,群小祀则玄冕,凡兵事韦弁服,视朝则皮弁服。”皮者鹿皮冠,服者素积也,故曰小大短长,冠带有常也。

【校】旧注多脱误,今考《礼》注补正。

(3) 有司,理官。刑非一,故言百。军刑斩,狱刑杀,皆重其事,故曰“必当”。

(4) 凌弱为枉,违强为桡。反,还。殃,咎。

是月也,乃命宰祝,巡行牺牲,视全具,案刍豢, (1) 瞻肥瘠,察物色, (2) 必比类,量小大,视长短,皆中度。五者者备当,上帝其享。天子乃傩,御佐疾,以通秋气。 (3) 以犬尝麻,先祭寝庙。 (4)

(1) 宰,于《周礼》为充人,掌养祭祀之牺牲。系于牢,刍之三月也。祝,太祝。以骍牷事神,祈福祥也。巡行牺牲,视其全具者,恐其毁伤。案其刍豢之薄厚。牛羊曰刍,犬豕曰豢。

(2) 物,毛也。

(3) 傩,逐疫除不祥也。《语》曰:“乡人傩,朝服立于阼阶。”御,止也。佐疾谓疗也,傩以止之也。以通达秋气,使不壅闭。

【校】《月令》无“御佐疾”三字。

(4) 犬,金畜也。麻始熟,故尝之。

是月也,可以筑城郭,建都邑, (1) 穿窦窌,修囷仓。 (2) 乃命有司,趣民收敛,务蓄菜,多积聚。 (3) 乃劝种麦,无或失时,行罪无疑。 (4)

(1) 国有先君宗庙曰都,无曰邑。

(2) 穿水通窦,不欲地泥湿也。穿窌所以盛谷也。修治囷仓,仲秋大内,谷当入也。圆曰囷,方曰仓。

(3) 有司,于《周礼》为场人。场,协入也。蓄菜,乾苴之属也。《诗》云:“亦有旨蓄,以御冬”也。

(4) 罪,罚也。

【校】“无或”当从《淮南》作“若或”。如从《月令》作“无或失时”,则下“其有失时”句亦不可去。

是月也,日夜分,雷乃始收声,蛰虫俯户。 (1) 杀气浸盛,阳气日衰,水始涸。 (2) 日夜分,则一度量, (3) 平权衡,正钧石,齐斗甬。 (4)

(1) 是月秋分。分,等也。昼漏五十刻,夜漏五十刻,故曰“日夜分”也。雷乃始收藏,其声不震也。将蛰之虫,俯近其所蛰之户。

【校】《月令》作“雷始收声”,此“乃”、“始”二字,当衍其一。“俯户”,《月令》作“坏户”。

(2) 杀气,阴气。涸,竭。

(3) 一,同也。度,尺丈。量,釜钟也。

(4) 权,秤衡也。三十斤为钧。百二十斤为石。斗、甬,皆量器也。

【校】“斗甬”,旧本作“升角”,讹,今从《月令》改正。

是月也,易关市,来商旅,入货贿,以便民事。 (1) 四方来杂,远乡皆至, (2) 则财物不匮,上无乏用,百事乃遂。 (3) 凡举事无逆天数, (4) 必顺其时, (5) 乃因其类。 (6)

(1) 易关市,不征税也,故商旅来。市贱鬻贵曰商。旅者,行商也。货贿,财赂也。以所有易所无,民得其求,故曰“以便民事”。

(2) 杂,会也。关市不征,故远乡皆至。

【校】“杂”,《月令》作“集”。

(3) 上无乏用,所求得也。事非一,故言“百事”。遂,成也。

(4) 天数,天道。

【校】“举事”,《月令》作“举大事”,“天数”作“大数”。

(5) 其时,天时。

(6) 因顺其事类不干逆。

【校】“乃因”,《月令》作“慎因”。

行之是令,白露降三旬。 (1) 仲秋行春令,则秋雨不降,草木生荣,国乃有大恐。 (2) 行夏令,则其国旱,蛰虫不藏,五谷复生。 (3) 行冬令,则风灾数起,收雷先行,草木早死。 (4)

(1) 行之是令,行是之令也,故白露降三旬,成万物也。

(2) 天阳亢燥,而行温仁之令,故雨不降。尚生育,故草木荣华,李、梅之属冬实也。金木相干,有兵象,故曰民有大惶恐也。

(3) 夏气盛阳,故炎旱,使蛰伏之虫不潜藏,五谷复萌生也,于《洪范》五行为“恒燠”之徵。

【校】“其国旱”,必本是“其国乃旱”,上节必本是“国有大恐”。后人以《月令》参校,遂记一“乃”字于“有大恐”之上,写时因误入,后来校者本欲去上“乃”字,而反误去此节之“乃”字,一剩一脱,其所以致误之由,尚可推求而得也。

(4) 冬寒严猛,故风灾数发。收藏之雷先动,行未当行,故曰“先”也。





论威 (1)


(1) 【校】“论”疑本是“谕”字。

二曰:

义也者,万事之纪也。君臣上下,亲疏之所由起也; (1) 治乱安危,过胜之所在也。 (2) 过胜之,勿求于他,必反于己。人情欲生而恶死, (3) 欲荣而恶辱。死生荣辱之道一,则三军之士可使一心矣。 (4) 凡军,欲其众也; (5) 心,欲其一也。三军一心,则令可使无敌矣。令能无敌者,其兵之于天下也亦无敌矣。古之至兵,民之重令也, (6) 重乎天下,贵乎天子。其藏于民心,捷于肌肤也,深痛执固, (7) 不可摇荡, (8) 物莫之能动。 (9) 若此则敌胡足胜矣。 (10) 故曰其令强者其敌弱,其令信者其敌诎。 (11) 先胜之于此,则必胜之于彼矣。 (12)

(1) 上,长。下,幼。

(2) 得纪则治而安,失纪则乱而危也。过犹取也。胜,有所胜也。

(3) 欲,贪也。

(4) 一于纪。

(5) 众,多也。以多击寡,虽拙者犹以克胜,故欲其众。

(6) 至兵,至德君之兵也。令无不化,故谓之至重也。

【校】注“至重”,似不当有“至”字。

(7) 捷,养也。

【校】注疑未是。“捷”,或当为“浃”。

(8) 荡,动也。

(9) 动,移也。

(10) 如此者胜敌不足以为武,故言“胡足胜矣”,小之也。

(11) 令强者,不可犯也;令信者,赏不僭,刑不滥也;故能使其敌弱而屈服也。

(12) 此,近,谓庙堂。彼,远,谓原野。

凡兵,天下之凶器也;勇,天下之凶德也。 (1) 举凶器,行凶德,犹不得已也。 (2) 举凶器必杀,杀,所以生之也; (3) 行凶德必威,威,所以慑之也。 (4) 敌慑民生,此义兵之所以隆也。 (5) 故古之至兵,才民未合, (6) 而威已谕矣, (7) 敌已服矣, (8) 岂必用枹鼓干戈哉? (9) 故善谕威者,于其未发也,于其未通也,窅窅乎冥冥,莫知其情, (10) 此之谓至威之诚。 (11)

(1) 兵者战斗有负败,勇者凌傲有死亡,故皆谓之凶。

(2) 已,止也。

(3) 杀无道所以生有道也。《司马法》曰:“有故杀人,虽杀人可也。”

(4) 威,畏也。慑,惧也。以威畏敌人,使之畏惧己也。

(5) 隆,盛也。

(6) 合,交。

【校】孙云:“‘才民’,《御览》二百七十一、又三百三十九俱作‘士民’。”

(7) 谕犹行。

(8) 服,降。

(9) 鼓以进士。干,楯也。戈,戟也。

(10) 窅,音窈。

【校】“窅窅乎冥冥”,疑“窅”字不当叠。

(11) 诚,实也。

凡兵,欲急疾捷先。欲急疾捷先之道,在于知缓徐迟后而急疾捷先之分也。 (1) 急疾捷先,此所以决义兵之胜也,而不可久处。知其不可久处,则知所兔起凫举死 之地矣。 (2) 虽有江河之险则凌之, (3) 虽有大山之塞则陷之, (4) 并气专精, (5) 心无有虑, (6) 目无有视,耳无有闻,一诸武而已矣。冉叔誓必死于田侯,而齐国皆惧; (7) 豫让必死于襄子,而赵氏皆恐; (8) 成荆致死于韩主,而周人皆畏; (9) 又况乎万乘之国,而有所诚必乎,则何敌之有矣? (10) 刃未接而欲已得矣。 (11) 敌人之悼惧惮恐,单荡精神尽矣,咸若狂魄, (12) 形性相离, (13) 行不知所之,走不知所往,虽有险阻要塞,铦兵利械,心无敢据,意无敢处,此夏桀之所以死于南巢也。今以木击木则拌, (14) 以水投水则散,以冰投冰则沉,以涂投涂则陷,此疾徐先后之势也。

(1) 【校】孙云:“‘而’字,《御览》作‘缓徐迟后’四字。”

(2) 起,走;举,飞也。兔走凫趋,喻急疾也。 音闷,谓绝气之闷。

【校】注“谓”字非衍即误。

(3) 凌,越也。

(4) 陷,坏也。

(5) 【校】卢云:“案《御览》二百七十一作‘抟精’。”“抟”与“专”同,前卷五《适音》篇“不收则不抟,不抟则怒”,注云:“不抟,不专一也。”则知《吕氏》书多用“抟”字。《御览》所见,尚仍是古本。后人不知,乃径改为“专”字。余尝考《易》、《左传》、《管子》、《史记》,而知“抟”之即“专”,文繁不录。

(6) 无有由豫之虑。

(7) 冉叔,仪工。田侯,齐君也。

【校】事无考,注亦不明。

(8) 豫让,晋毕阳之孙,因族以为氏。让欲报仇杀赵襄子,故赵氏恐也。

(9) 畏其义。

(10) 言无有敢敌者。

(11) 已得欲杀也。

(12) 咸,皆。魄飞荡若狂人。

(13) 离,违也。

(14) 拌,析也。

夫兵有大要,知谋物之不谋之不禁也, (1) 则得之矣。专诸是也, (2) 独手举剑至而已矣,吴王壹成。 (3) 又况乎义兵,多者数万,少者数千,密其躅路,开敌之涂,则士岂特与专诸议哉!

(1) 【校】句疑。

(2) 专诸,吴之勇人,为阖庐刺吴王僚也。

(3) 专诸一举而成阖庐为王,故曰“吴王一成”。成,谓专诸能成吴王也。





简选


三曰:

世有言曰:“驱市人而战之,可以胜人之厚禄教卒; (1) 老弱罢民,可以胜人之精士练材; (2) 离散係系, (3) 可以胜人之行陈整齐; (4) 锄櫌白梃,可以胜人之长铫利兵。” (5) 此不通乎兵者之论。 (6) 今有利剑于此,以刺则不中,以击则不及,与恶剑无择, (7) 为是斗因用恶剑则不可。 (8) 简选精良,兵械铦利,发之则不时,纵之则不当,与恶卒无择, (9) 为是战因用恶卒则不可。王子庆忌、陈年犹欲剑之利也。 (10) 简选精良,兵械铦利,令能将将之, (11) 古者有以王者、有以霸者矣,汤、武、齐桓、晋文、吴阖庐是矣。 (12)

(1) 厚禄,大将也。教卒,习战也。

(2) 练材,拳勇有力之材。

(3) 【校】疑“系”为“絫”字之误。

(4) 行陈,五列也。整齐,周旋进退也。

【校】注“五列”即“伍列”。

(5) 櫌,椎;梃,杖也。长铫,长矛也。铫,读曰苇苕之苕。

(6) 通,达也。

(7) 择,别。

(8) 言不可用也。

(9) 恶卒,怯卒。

(10) 庆忌,吴王僚之子也;陈年,齐人;皆勇捷有力也。

【校】梁仲子云:“陈年即《吴越春秋》之陈音,善射者,楚人也。古年、音声相近。”

(11) 能将,上将。

(12) 汤,殷主癸之子天乙也。武,周文王之子发也。齐桓,僖公之子小白也。晋文,献公之子重耳也。吴阖庐,夷昧之子光也。

殷汤良车七十乘,必死六千人, (1) 以戊子战于郕,遂禽推移、大牺, (2) 登自鸣条,乃入巢门,遂有夏。 (3) 桀既奔走,于是行大仁慈,以恤黔首,反桀之事, (4) 遂其贤良,顺民所喜,远近归之,故王天下。 (5)

(1) 【校】孙云:“《御览》三百二十五‘必死’下有‘士’字。”

(2) 桀多力,能推移大牺,因以为号,而禽克之。

【校】“移”上旧本缺“推”字,据《御览》补。注“推”下缺“移”字,亦补之。梁仲子云:“《淮南·主术训》‘桀之力能推移大牺’,此注所本也。据《墨子·明鬼下》篇‘禽推哆、大戏’,则皆人名。此推移即推哆也。《所染》篇云‘夏桀乐于干辛、推哆’,此下又云‘推哆、大戏,主别兕虎,指画杀人’,此大牺即大戏也。诱不参考,而以大牺为桀号,误甚。”卢云:“案下文云‘桀奔走’,则何尝成禽哉?汤之待桀有礼,见于他书者多矣,从未有言禽桀者。”

(3) 殷汤遂有夏之天下。

(4) 桀为残贼,汤为仁惠,故曰“反桀之事”。

(5) 殷之王,古之帝也。

【校】“故王”之王,于况反。注读如字,误。

武王虎贲三千人,简车三百乘,以要甲子之事于牧野,而纣为禽。 (1) 显贤者之位,进殷之遗老,而问民之所欲,行赏及禽兽,行罚不辟天子, (2) 亲殷如周,视人如己,天下美其德,万民说其义,故立为天子。 (3)

(1) 要,成也。甲子之日,获纣于牧野。

(2) 谓杀纣也。

(3) 武王为天所子。

【校】语极明白,而注故迂曲。

齐桓公良车三百乘,教卒万人,以为兵首, (1) 横行海内,天下莫之能禁, (2) 南至石梁, (3) 西至酆郭, (4) 北至令支。 (5) 中山亡邢,狄人灭卫, (6) 桓公更立邢于夷仪,更立卫于楚丘。

(1) 首,始也。

(2) 禁,止也。

(3) 石梁,在彭城也。

(4) 酆郭,在长安西南。

(5) 令支,在辽西。

(6) 中山,狄国也,一名鲜虞,在今卢奴西。中山伐邢而亡之。邢国今在赵襄国也。狄杀卫懿公于荧泽,故曰“灭”也。

【校】梁仲子云:“齐桓因狄伐邢,遂迁之。狄未尝亡邢也。邢为卫灭,见《左传》僖廿五年。中山为白狄别种,伐邢者为赤狄。诱不之驳,何也?”

晋文公造五两之士五乘, (1) 锐卒千人,先以接敌, (2) 诸侯莫之能难。反郑之埤,东卫之亩, (3) 尊天子于衡雍。 (4)

(1) 两,技也。五技之人,兵车五乘,七十五人也。

【校】以技训两,未知何出。“五乘”下当叠一“乘”字。

(2) 在车曰士,步曰卒。

(3) 反,覆。覆郑城埤而取之,使卫耕者皆东亩,以遂晋兵也。

(4) 文公率诸侯朝天子于衡雍。衡雍践土,今之河阳。

吴阖庐选多力者五百人,利趾者三千人,以为前陈, (1) 与荆战,五战五胜,遂有郢。 (2) 东征至于庳庐, (3) 西伐至于巴、蜀,北迫齐、晋,令行中国。 (4)

(1) 趾,足也。陈,列也。

(2) 郢,楚都。

(3) 国名也。

(4) 中国,诸华。

故凡兵势险阻,欲其便也;兵甲器械,欲其利也;选练角材,欲其精也; (1) 统率士民,欲其教也。 (2) 此四者,义兵之助也。时变之应也,不可为而不足专恃, (3) 此胜之一策也。 (4)

(1) 角犹量也。精犹锐利。

(2) 教,习也。

(3) 专,独也。

(4) 策,谋术。





决胜


四曰:

夫兵有本干:必义,必智,必勇。义则敌孤独, (1) 敌孤独则上下虚, (2) 民解落; (3) 孤独则父兄怨,贤者诽,乱内作。 (4) 智则知时化,知时化则知虚实盛衰之变,知先后远近纵舍之数。 (5) 勇则能决断,能决断则能若雷电飘风暴雨,能若崩山破溃、别辨霣坠,若鸷鸟之击也, (6) 搏攫则殪, (7) 中木则碎。此以智得也。

(1) 孤独,无助。

(2) 【校】旧校云:“一作‘乘’。”

(3) 解,散。

(4) 诽,谤也。

(5) 数,术也。

(6) 谓如鹰隼感秋霜之节奋击也。

(7) 殪,死也。

夫民无常勇,亦无常怯。有气则实,实则勇;无气则虚,虚则怯。怯勇虚实,其由甚微,不可不知。 (1) 勇则战,怯则北。 (2) 战而胜者,战其勇者也;战而北者,战其怯者也。怯勇无常,倏忽往来,而莫知其方, (3) 惟圣人独见其所由然。故商、周以兴, (4) 桀、纣以亡。巧拙之所以相过, (5) 以益民气与夺民气,以能斗众与不能斗众。军虽大,卒虽多,无益于胜。 (6) 军大卒多而不能斗,众不若其寡也。夫众之为福也大,其为祸也亦大。譬之若渔深渊,其得鱼也大,其为害也亦大。 (7) 善用兵者,诸边之内莫不与斗,虽厮舆白徒,方数百里皆来会战,势使之然也。 (8) 幸也者,审于战期而有以羁诱之也。 (9)

(1) 当知之也。

(2) 北,走也。

(3) 方,道也。

(4) 商,汤也。周,武也。

(5) 过,绝也。

(6) 多而不能以克,故曰“无益于胜”。

(7) 为霣溺则死,故害大。

(8) 厮,役。舆,众。白衣之徒。

(9) 羁,牵。诱,导。

凡兵,贵其因也。因也者,因敌之险以为己固,因敌之谋以为己事。能审因而加,胜则不可穷矣。 (1) 胜不可穷之谓神,神则能不可胜也。 (2) 夫兵,贵不可胜。 (3) 不可胜在己,可胜在彼。圣人必在己者,不必在彼者,故执不可胜之术以遇不胜之敌,若此,则兵无失矣。凡兵之胜,敌之失也。胜失之兵,必隐必微,必积必抟。隐则胜阐矣, (4) 微则胜显矣,积则胜散矣,抟则胜离矣。诸搏攫柢噬之兽,其用齿角爪牙也,必托于卑微隐蔽,此所以成胜。 (5)

(1) 穷,极。

(2) 能胜不能所以胜,故曰“不可胜”。

(3) 【校】孙云:“《御览》二百二十五作‘夫兵不贵胜,而贵不可胜’,此脱四字。”

(4) 阐,布也。

【校】上“必抟”与此“抟”字,旧本皆作“搏”,盖因下文“搏”字而误。案抟之义为专壹,正与分离相反,故今定作“抟”字。

(5) 若狐之搏雉,俯伏弭毛以喜说之,雉见而信之,不惊惮远飞,故得禽之。军戎亦皆如此,故曰“所以成胜”。

【校】注“军戎”,旧本作“军贼”,讹,今改作“戎”。亦或是“战”字误。





爱士 (1)


(1) 【校】一作“慎穷”。

五曰:

衣人以其寒也,食人以其饥也。饥寒,人之大害也。救之,义也。 (1) 人之困穷,甚如饥寒,故贤主必怜人之困也,必哀人之穷也。如此则名号显矣,国士得矣。 (2)

(1) 大仁义也。

(2) 得国士也。

昔者秦缪公乘马而车为败,右服失而野人取之。 (1) 缪公自往求之, (2) 见野人方将食之于岐山之阳。 (3) 缪公叹曰 (4) :“食骏马之肉而不还饮酒,余恐其伤女也!”于是遍饮而去。处一年,为韩原之战。 (5) 晋人已环缪公之车矣,晋梁由靡已扣缪公之左骖矣, (6) 晋惠公之右路石奋投而击缪公之甲,中之者已六札矣。 (7) 野人之尝食马肉于岐山之阳者三百有余人,毕力为缪公疾斗于车下, (8) 遂大克晋,反获惠公以归。 (9) 此《诗》之所谓曰“君君子则正,以行其德; (10) 君贱人则宽,以尽其力”者也。 (11) 人主其胡可以无务行德爱人乎? (12) 行德爱人,则民亲其上; (13) 民亲其上,则皆乐为其君死矣。 (14)

(1) 四马车,两马在中为服,《诗》曰“两服上襄”是也;两马在边为骖,《诗》曰“两骖如舞”是也。

(2) 【校】旧本脱此句,孙据李善注《文选》曹子建《求自试表》所引补。梁仲子云:“《韩诗外传》十作‘求三日而得之’,《淮南·氾论训》作‘追而及之’,《说苑·复恩》篇亦有‘自往求之’句,皆于语义为合,此文脱无疑。”

(3) 【校】《外传》作“茎山”。

(4) 【校】《选注》、《御览》四百七十九、又八百九十六俱作“笑曰”。

(5) 处一年,饮食肉人酒之明年也。伐晋惠公,战于晋地之韩原。

(6) 环,围。扣,持。

(7) 甲,铠也。陷之六札。

【校】孙云:“《御览》作‘其甲之抎者已六札矣’,注‘抎者,配 也,文有所失也’。《说文系传》手部‘抎’字亦引之。此文疑已为后人窜改,并注亦删去。”卢云:“案‘抎者,配 也’,语不可晓,疑或是‘抎音颠陨也’,下‘有所失也’是《说文》语。高未必引《说文》,殆后人所益,又脱去‘说’字耳。”

(8) 毕,尽。疾,急。

(9) 克,胜也。胜晋,执惠公归于秦。

(10) 为君子作君,正法以行德,无德不报。

(11) 此逸诗也。为贱人作君,宽饶之以尽其力,故缪公战以胜晋。

(12) 胡,何也。

【校】旧本“行德”下有“人”字,今从《御览》删。

(13) 【校】“行德”二字旧脱,从《御览》补。

(14) 食马肉人为缪公死战,不爱其死,以获惠公是也。

赵简子有两白骡而甚爱之。阳城胥渠处 (1) 广门之官,夜款门而谒曰:“主君之臣胥渠有疾, (2) 医教之曰:‘得白骡之肝,病则止; (3) 不得则死。’”谒者入通。董安于御于侧,愠曰:“嘻!胥渠也,期吾君骡,请即刑焉。” (4) 简子曰:“夫杀人以活畜,不亦不仁乎?杀畜以活人,不亦仁乎?”于是召庖人杀白骡,取肝以与阳城胥渠。处无几何, (5) 赵兴兵而攻翟。广门之官左七百人,右七百人,皆先登而获甲首。 (6) 人主其胡可以不好士?

(1) 阳城,姓。胥渠,名。处犹病也。

【校】注以处训病,未见所出。《贾谊书·耳痺》篇有“渠如处车裂回泉”语,彼是人名,则此亦正相类。《汉书·人表》载“胥渠”,无“处”字。

(2) 广门,邑名也。官,小臣也。款,扣也。赵简子,晋大夫也,大夫称主者也。

(3) 止,愈也。

(4) 安于,简子家臣。愠,怒。即,就也。谓就胥渠而刑之也。

(5) 【校】《御览》四十九无“处”字。梁仲子云:“‘处’字属下,与上文‘处一年’文义相似。”

(6) 获衣甲者之首。

凡敌人之来也,以求利也。今来而得死, (1) 且以走为利。敌皆以走为利, (2) 则刃无与接。 (3) 故敌得生于我,则我得死于敌; (4) 敌得死于我,则我得生于敌。 (5) 夫得生于敌,与敌得生于我,岂可不察哉? (6) 此兵之精者也。存亡死生决于知此而已矣。 (7)

(1) 是不得利而进。

(2) 且,将也。《传》曰:“见可而进,知难而退,武之善经也。”故以走为利。

(3) 接,交战也。

(4) 敌克,故得生也。己负,故为死也。

(5) 敌负,故我得杀敌也。能杀敌,故己得生也。

【校】此段正文及注,宋邦乂本脱去,别本皆有。

(6) 得胜则生,负则败,故不可不察而知。

(7) 言能用兵,胜负死生之本,所由克败,故曰此兵之精妙矣。





第九卷 季秋纪



季秋


一曰:

季秋之月,日在房, (1) 昏虚中,旦柳中。 (2) 其日庚辛,其帝少皞,其神蓐收,其虫毛,其音商,律中无射, (3) 其数九。其味辛,其臭腥,其祀门,祭先肝。 (4) 候雁来,宾爵入大水为蛤。 (5) 菊有黄华,豺则祭兽戮禽。 (6) 天子居总章右个, (7) 乘戎路,驾白骆,载白旂,衣白衣,服白玉,食麻与犬,其器廉以深。 (8)

(1) 季秋,夏之九月。房,东方宿,宋之分野。是月,曰躔此宿。

(2) 虚,北方宿,齐之分野。柳,南方宿,周之分野。是月昏旦时,皆中于南方。

(3) 无射,阳律也。竹管音与无射和也。阴气上升,阳气下降,故万物随而藏,无射出见也。

(4) 说在《孟秋》。

(5) 是月,候时之雁从北方来,南之彭蠡,盖以为八月来者其父母也。其子羽翼稚弱,未能及之,故于是月来过周雒也。宾爵者,老爵也,栖宿于人堂宇之间,有似宾客,故谓之宾爵。大水,海也。《传》曰“爵入于海为蛤”,此之谓也。

【校】《月令》郑注以“鸿雁来宾”为句,与此异。

(6) 豺,兽也,似狗而长毛,其色黄,于是月杀兽,四围陈之,世所谓祭兽。戮者,杀也。

(7) 右个,北头室也。

(8) 说在《孟秋》。

是月也,申严号令,命百官贵贱无不务入, (1) 以会天地之藏, (2) 无有宣出。命冢宰,农事备收,举五种之要。 (3) 藏帝籍之收于神仓,祗敬必伤。 (4)

(1) 季秋毕内,故务入也。

(2) 会,合也。

(3) 冢宰,于《周礼》为天官。冢,大;宰,治也。主治万事,故命之也。举书五种之要,具文簿也。

(4) 天子籍田千亩,其所收谷也,故谓之帝籍之收。于仓受 ,以供上帝神祇之祀,故谓之神仓。饬,正也。祗敬必正,不倾邪也。

【校】案:“ ”为“穀”之异文,《尚书大传》、《山海经》、《论衡》、《齐民要术》皆有此字。或从“木”,误,今从《篇海》从“禾”。

是月也,霜始降, (1) 则百工休。 (2) 乃命有司曰:“寒气总至,民力不堪,其皆入室。” (3) 上丁,入学习吹。 (4)

(1) 秋分后十五曰寒露,寒露后十五曰霜降,故曰“始”也。

(2) 霜降天寒,朱漆不坚,故百工休,不复作器。

(3) 有司,于《周礼》为司徒。司徒主众,故命之使民入室也。《诗》云:“穹窒熏鼠,塞向墐户。嗟我妇子,曰为改岁,入此室处。”此之谓也。

(4) 是月上旬丁日,入学吹笙习礼乐。《周礼》“籥师掌教国子舞羽吹籥”,《诗》云“吹笙鼓簧,承筐是将”,此之谓也。

【校】《月令》作“命乐正入学习吹”,此脱三字。注“吹籥”,旧作“吹笙竽籥”,今据《周礼》删正。

是月也,大飨帝,尝牺牲,告备于天子。 (1) 合诸侯,制百县, (2) 为来岁受朔日,与诸侯所税于民,轻重之法,贡职之数,以远近土地所宜为度, (3) 以给郊庙之事.无有所私。 (4)

(1) 大飨上帝,尝牺牲一日,先杀毛以告全,故告备于天子也。

【校】此注似有讹脱。案:《周礼·大宰职》论祭天礼云:“及纳亨,赞王牲事。”郑注“纳牲将告杀,谓乡祭之晨”,则非先一日杀也。《诗·信南山》篇云:“执其鸾刀,以启其毛,取其血 。”笺云:“毛以告纯,血以告杀。”此注“告全”即“告纯”也。旧本误作“告令”,今改正。

(2) 合,会。诸侯之制度,车服之级,各如其命数。百县,畿内之县也。五家为邻,五邻为里,四里为攒,五攒为鄙,五鄙为县,然则谓县者二千五百家也。

【校】案:《周礼·遂人》“攒”作“酂”。旧本“五鄙”讹作“四鄙”,今改正。

(3) 来岁,明年也。秦以十月为正,故于是月受明年历日也。由此言之,《月令》为秦制也。诸侯所税轻重,职贡多少之数,远者贡轻,近者贡重,各有所宜。

【校】卢云:“案若以十月为来岁,而于九月始受朔日,则仅就百县言为可。若远方诸侯,则有不能逮者矣。注据此即为秦制,吾未之信。”

(4) 郊祀天,庙祀祖,取共事而已。无有所私,多少不如法制也。

是月也,天子乃教于田猎,以习五戎獀马。 (1) 命仆及七驺咸驾,载旍旐, (2) 舆受车以级,整设于屏外, (3) 司徒搢扑,北向以誓之。 (4) 天子乃厉服厉饬,执弓操矢以射, (5) 命主祠,祭禽于四方。 (6)

(1) 五戎,五兵,谓刀、剑、矛、戟、矢也。獀,择也。为将田,故习肄五兵。选择田马,取堪乘也。

【校】“獀马”,《月令》作“班马政”。旧本“獀”下有“一作搜”三字,乃校者之辞。此无“政”字,避始皇讳,而《月令》不讳,则《月令》之非秦制益明矣。

(2) 仆,于《周礼》为田仆,掌御田辂。七驺,于《周礼》当为趣马,掌良马驾税之任,无七驺之官也。田仆掌佐马之政,令获者植旍,故载旍也。

【校】“旍”与“旌”同。“令获者植旍”,旧本作“令猎者扬旍”,误,今改正。

(3) 舆,众也。众当受田车者,各以等级陈于屏外也。天子外屏。屏,树垣也。《尔雅》云“屏谓之树”,《论语》曰“树塞门”者也。

【校】《月令》无“舆”字,又“受”作“授”。

(4) 搢,插也。扑,所以教也。插置带间,誓告其众。

(5) 是月天子尚武,乃服猛,厉其所佩之饬以射禽也。《周礼·司服》章:“凡田,冠弁服。”戎服,垂衣也。

【校】案:《月令》作“天子乃厉饰,执弓挟矢以猎”,古“饰”、“饬”亦或通用。注“戎服垂衣也”亦似有讹,《月令正义》引熊氏云:“春夏田,冠弁服,秋冬韦弁服,韦弁服即所谓戎服也。”郑云:“以 韦为弁,又以为衣裳。”然则“垂衣”乃“韦衣”之误也。

(6) 主祠,掌祀之官也。祭始设禽兽者于四方,报其功也。不知其神所在,故博求于四方。

是月也,草木黄落,乃伐薪为炭, (1) 蛰虫咸俯在穴,皆墐其户。 (2) 乃趣狱刑,无留有罪, (3) 收禄秩之不当者,共养之不宜者。 (4)

(1) 草本节解,斧入山林,故伐木作炭。

【校】注“伐木”,旧作“伐林”,讹。

(2) 咸,皆。俯,伏。藏于穴,墐塞其户也。墐,读如斤斧之斤也。

【校】“穴”,《月令》作“内”,古书往往互用。

(3) 阴气杀僇,故刑狱当者决之,故曰“无留有罪”也。

(4) 不当者,谓无功德而受禄秩也。不宜者,谓若屈到嗜芰,曾晳嗜羊枣,非礼之养,故收去之也。一说:言所养无勋于国,其先人无贤,所不宜养,故收敛之也。

【校】注末旧作“所宜养故收敛者也”,脱“不”字,“者”当作“之”,今补正。

是月也,天子乃以犬尝稻,先荐寝庙。 (1)

(1) 稻始升,故尝之。先进于庙,孝敬亲也。

季秋行夏令,则其国大水,冬藏殃败,民多鼽窒; (1) 行冬令,则国多盗贼,边境不宁,土地分裂; (2) 行春令,则暖风来至,民气解堕,师旅必兴。 (3)

(1) 秋,金气,水之母也。夏阳布施,多淋雨。二气相并,故大水也。火气热,故冬藏殃败也。火金相干,故民鼽窒,鼻不通也。鼽,读曰仇怨之仇。

【校】“鼽窒”,《月令》作“鼽嚏”。

(2) 冬令纯阴,奸谋所生之象,故多盗贼,使边境之人不宁也,则土地见侵削,为邻国所分裂。

(3) 春阳仁,故暖风至,民解堕也。木干金,故师旅并兴。二千五百人为师,五百人为旅。

【校】“师旅必兴”,《月令》作“师兴不居”。





顺民


二曰:

先王先顺民心,故功名成。 (1) 夫以德得民心以立大功名者,上世多有之矣。 (2) 失民心而立功名者,未之曾有也。 (3) 得民必有道,万乘之国,百户之邑,民无有不说。 (4) 取民之所说而民取矣,民之所说岂众哉?此取民之要也。 (5)

(1) 治天下之功,圣人之名也。

【校】注“名”字,旧本作“功”,讹,今改正。

(2) 神农、黄帝、尧、舜、禹、汤、文、武皆是也,故上世多有之。

(3) 蚩尤、夷昕、桀、纣下至周厉、幽王、晋厉、宋康、卫懿、楚灵之属,皆有灭亡,故曰“未之曾有”也。

【校】注“夷昕”盖夷羿也,未知高氏有所本,抑字误?

(4) 说其仁与义也。

(5) 要,约置也。

昔者汤克夏而正天下。 (1) 天大旱,五年不收, (2) 汤乃以身祷于桑林, (3) 曰:“余一人有罪,无及万夫。万夫有罪,在余一人。无以一人之不敏, (4) 使上帝鬼神伤民之命。” (5) 于是翦其发, 其手, (6) 以身为牺牲, (7) 用祈福于上帝。民乃甚说,雨乃大至。则汤达乎鬼神之化、人事之传也。 (8)

(1) 正,治也。

(2) 谷不熟,无所收。

【校】梁仲子云:“《论衡·感虚》篇‘书传言汤遭七年旱,或言五年’,知此言五年亦非误。李善注《文选》应休琏《与广川长书》亦作‘五年’。”

(3) 祷,求也。桑林,桑山之林,能兴云作雨也。

(4) 不敏,不材。

(5) 上帝,天也。天神曰神,人神曰鬼。谷者,民命也,旱不收,故曰“伤民之命。”

(6) 【校】李善注引此亦作“ ”,音郦。后《精通》篇“刃若新 研”,注“ ,砥也”,窃意 若作历音,则似当从 得声。善又注刘孝标《辩命论》引此竟作“磨”字,恐是“磿”字之误,从邑本无义。《战国·燕策》“故鼎反乎磿室”,磿室犹《楚辞·招魂》之所谓“砥室”,王逸注“砥,石名也”,引《诗》“其平如砥”,诱之注非取此义乎?而音又同,故余以“磿”字为是。孙侍御主《辩命论》注作“磨”,与“刃若新磨”较合,但不读郦耳。《蜀志·郤正传》注引作“ 其手”,《论衡》又作“丽其手”。

(7) 【校】《蜀志》注引作“自以为牺牲”,《文选》注及《御览》二百七十三皆同。

(8) 达,通。化,变。传,至。

文王处岐事纣,冤侮雅逊,朝夕必时, (1) 上贡必适,祭祀必敬。 (2) 纣喜,命文王称西伯,赐之千里之地。文王载拜稽首而辞曰:“愿为民请炮烙之刑。” (3) 文王非恶千里之地,以为民请炮烙之刑,必欲得民心也。得民心则贤于千里之地, (4) 故曰文王智矣。

(1) 雅,正;逊,顺也。纣虽冤枉文王而侮慢之,文王正顺诸侯之礼,不失其时。

(2) 贡,职贡也。

(3) 纣常熨烂人手,因作铜烙,布火其下,令人走其上,人堕火而死,观之以为娱乐,故名为“炮烙之刑。”

【校】案:“炮烙”当作“炮格”。江邻幾《杂志》引陈和叔云“《汉书》作‘炮格’”,乃今本亦尽改作“炮烙”矣。此注云“作铜烙”,乃显是“铜格”之误。格是庋格,亦作“庋阁”,小司马《索隐》于《史记·殷本纪》引邹诞生云“一音阁”,又杨倞注《荀子·议兵》篇音古责反,此二音皆是格,非烙。烙乃烧灼,安得言“铜烙”,且使罪人行其上乎?郑康成注《周礼·牛人》云:“互,若今屠家悬肉格。”据《列女传》云“膏铜柱”,则与康成所言,要亦不大相远耳。

(4) 贤犹多也。

越王苦会稽之耻, (1) 欲深得民心,以致必死于吴, (2) 身不安枕席,口不甘厚味, (3) 目不视靡曼, (4) 耳不听钟鼓, (5) 三年苦身劳力,焦唇乾肺,内亲群臣,下养百姓,以来其心。 (6) 有甘脆不足分,弗敢食; (7) 有酒流之江,与民同之。 (8) 身亲耕而食,妻亲织而衣。味禁珍, (9) 衣禁袭, (10) 色禁二。 (11) 时出行路,从车载食,以视孤寡老弱之渍病困穷颜色愁悴不赡者, (12) 必身自食之。 (13) 于是属诸大夫而告之曰 (14) :“愿一与吴徼天下之衷。 (15) 今吴、越之国相与俱残,士大夫履肝肺,同日而死,孤与吴王接颈交臂而偾, (16) 此孤之大愿也。若此而不可得也,内量吾国不足以伤吴, (17) 外事之诸侯不能害之, (18) 则孤将弃国家,释群臣,服剑臂刃,变容貌,易姓名,执箕帚而臣事之, (19) 以与吴王争一旦之死。 (20) 孤虽知要领不属, (21) 首足异处,四枝布裂,为天下戮,孤之志必将出焉!” (22) 于是,异日果与吴战于五湖,吴师大败。遂大围王宫,城门不守,禽夫差,戮吴相, (23) 残吴二年而霸。此先顺民心也。 (24)

(1) 耻,辱也。

(2) 必死战以报吴,欲以灭会稽耻也。

(3) 【校】旧本“甘厚”二字倒,今据李善注《文选》东方曼倩《非有先生论》乙正。

(4) 靡曼,好色。

(5) 不欲闻音乐。

(6) 欲得其欢心。

(7) 不敢独食。

(8) 投醪同味。

(9) 珍异。

(10) 袭,重。

(11) 二,青、黄也。

(12) 渍亦病也。《公羊传》曰:“大渍者,大病也。”

【校】案:《公羊庄二十年经》“齐大灾”,《传》曰:“大灾者何?大瘠也。大渍者何? 也。”“瘠”亦作“渍”。郑注《曲礼》引之,此似所见本异。高注《贵公》篇亦引《公羊》“大眚者何?大渍也”,又不同。或“眚”字后人所妄改。

(13) 赡犹足也。

(14) 属,会。

(15) 徼,求。衷,善。

【校】“下”字疑衍。

(16) 偾,僵也。

(17) 伤,败。

(18) 不能以之害吴。

(19) 服,带。臂,手。

(20) 争,决。旦,朝。

(21) 属,连。

(22) 将出,必死以伐吴也。

(23) 夫差,吴王阖庐之子。相,吴臣也。

(24) 越王先顺说民心,二年故能灭吴立霸功也。

齐庄子请攻越,问于和子。和子曰:“先君有遗令曰:‘无攻越。越,猛虎也。’” (1) 庄子曰:“虽猛虎也,而今已死矣。” (2) 和子曰 [1] 以告鸮子。 (3) 鸮子曰:“已死矣,以为生。” (4) 故凡举事,必先审民心,然后可举。 (5)

(1) 齐庄子,齐臣也。和子,齐田常之孙田和也,后为齐侯,因曰和子也。猛虎,言越王武勇多力,不可伐也。

(2) 言越王衰老,不能复致力战也,故曰“而今已死矣”。

(3) 鸮子,齐相。

(4) 以为生,为民所说。

(5) 审,定也。定民心所系,而举大事以攻伐也。





知士


三曰:

今有千里之马于此,非得良工,犹若弗取。 (1) 良工之与马也,相得则然后成, (2) 譬之若枹之与鼓。 (3) 夫士亦有千里,高节死义,此士之千里也。能使士待千里者,其惟贤者也。 (4)

(1) 良工,相马工也。

(2) 成良马。

(3) 枹待鼓,鼓待枹,乃发声也。良马亦然。

(4) 犹贤者能之也。

【校】《御览》八百九十六“待”作“行”,“也”作“乎”。

静郭君善剂貌辨。 (1) 剂貌辨之为人也多訾, (2) 门人弗说。 (3) 士尉以証静郭君, (4) 静郭君弗听,士尉辞而去。孟尝君窃以谏静郭君, (5) 静郭君大怒曰:“刬而类! (6) 揆吾家,苟可以傔剂貌辨者,吾无辞为也!” (7) 于是舍之上舍,令长子御,朝暮进食。 (8) 数年,威王薨,宣王立。 (9) 静郭君之交,大不善于宣王, (10) 辞而之薛,与剂貌辨俱。 (11) 留无几何, (12) 剂貌辨辞而行,请见宣王。静郭君曰:“王之不说婴也甚, (13) 公往,必得死焉。”剂貌辨曰:“固非求生也。”请必行,静郭君不能止。 (14)

(1) 静郭君,田婴也,孟尝君田文之父也,为薛君,号曰静郭君。

【校】案:《国策》作“靖郭君”。“齐貌辩”,《古今人表》作“昆辨”。“昆”或是“皃”之讹,然据《元和姓纂》有“昆姓,夏诸侯昆吾之后,齐有昆弁,见《战国策》”。今当各依本文可也。

(2) 【校】《国策》作“疵”,高诱注“疵,阙病也”;鲍彪注“疵,病也,谓过失”。

(3) 静郭君门人不说也。

(4) 証,谏。

【校】“証”,旧作“證”,注同。案《说文》証训谏,證训告,不同。此当作“証”,今改正。

(5) 窃,私。私谏静郭君,使听士尉之言,而止其去。

(6) 刬,灭;而,汝也。

(7) 傔,足也。揆度吾家,诚可以足剂貌辨者,吾不辞也。

【校】“揆”,《国策》作“破”,又“傔”作“慊”。

(8) 上舍,甲第也。御,侍也。以馆貌辨也。旦暮也。

(9) 威王之子。

(10) 交,接也。大不为王所善也。

(11) 俱,偕。

(12) 留于薛。

(13) 甚犹深。

(14) 止,禁止也。

剂貌辨行,至于齐。宣王闻之,藏怒以待之。 (1) 剂貌辨见, (2) 宣王曰:“子静郭君之所听爱也?”剂貌辨答曰:“爱则有之,听则无有。 (3) 王方为太子之时,辨谓静郭君曰:‘太子之不仁,过 涿视,若是者倍反。 (4) 不若革太子,更立卫姬婴儿校师。’ (5) 静郭君泫而曰 (6) :‘不可,吾弗忍为也。’且静郭君听辨而为之也,必无今日之患也。此为一也。 (7) 至于薛,昭阳请以数倍之地易薛,辨又曰:‘必听之。’ (8) 静郭君曰:‘受薛于先王,虽恶于后王,吾独谓先王何乎? (9) 且先王之庙在薛,吾岂可以先王之庙予楚乎?’”又不肯听辨。此为二也。 (10) 宣王太息,动于颜色,曰:“静郭君之于寡人,一至此乎!寡人少,殊不知此。 (11) 客肯为寡人少来静郭君乎?” (12) 剂貌辨答曰:“敬诺。” (13) 静郭君来,衣威王之服,冠其冠,带其剑。宣王自迎静郭君于郊,望之而泣。静郭君至,因请相之。 (14) 静郭君辞,不得已而受。 (15) 十日,谢病强辞,三日而听。 (16)

(1) 藏,怀。

(2) 句。

(3) 徒见爱耳,言则不见从也。

(4) 涿,不仁之人也。过犹甚也。太子不仁,甚于 涿,视如此者倍反,不循道理也。

【校】字书无“ ”字。注训 涿为不仁之人,不知何据。《国策》作“过颐豕视”,刘辰翁曰:“过颐,即俗所谓耳后见腮;豕视,即相法所谓下邪偷视。”

(5) 婴儿,幼少之称,卫姬所生,校师其名也,威王之庶子也。劝静郭君令废太子,更立校师为太子也。

【校】“校师”,《国策》作“郊师”。

(6) 【校】旧校云:“‘泫’一作‘泣’。”案:《国策》作“泣”。

(7) 言静郭君听辨之言,则无今日见逐之患也。此一不见听也。

(8) 昭阳,楚相也。求以倍地易薛之少,辨劝之可也。

(9) 先王,威王也。见恶于后王,先王其谓我何?

(10) 二不见听。

(11) 动,变也。一犹乃也。少,小,故不知此也。

(12) 言犹可也。

(13) 诺,顺。

(14) 请以为相也。

(15) 受为相。

(16) 听,许。

当是时也,静郭君可谓能自知人矣。 (1) 能自知人,故非之弗为阻。 (2) 此剂貌辨之所以外生乐、趋患难故也。 (3)

(1) 知人,知剂貌辨也。

(2) 阻,止。

(3) 外弃其生命,乐解人之患,往见宣王,不辟难之故也。

【校】《国策》作“外生乐患趣难者也”。孙云:“观注,似此亦本与《国策》同。”





审己


四曰:

凡物之然也,必有故。 (1) 而不知其故,虽当与不知同,其卒必困。 (2) 先王、名士、达师之所以过俗者,以其知也。水出于山而走于海, (3) 水非恶山而欲海也,高下使之然也。稼生于野而藏于仓,稼非有欲也,人皆以之也。 (4) 故子路掩雉而复释之。 (5)

(1) 故,事。

(2) 当,合;同,等也。困于不知其故也。

(3) 走,归。

(4) 以,用也。

(5) 所得者小,不欲夭物,故释之也。

子列子常射中矣,请之于关尹子。 (1) 关尹子曰:“知子之所以中乎?”答曰:“弗知也。”关尹子曰:“未可。” (2) 退而习之三年,又请。 (3) 关尹子曰:“子知子之所以中乎?”子列子曰:“知之矣。” (4) 关尹子曰:“可矣,守而勿失。” (5) 非独射也,国之存也,国之亡也,身之贤也,身之不肖也,亦皆有以。 (6) 圣人不察存亡、贤不肖,而察其所以也。

(1) 子列子,贤人,体道者,请问其射所以中于关尹喜。关尹喜师老子也。

(2) 弗知射所以中者未可语。

(3) 习,学也。又复请问于关尹子。

(4) 知射心平体正然后能中,自求诸己,不求诸人,故曰“知之”。

(5) 守求诸己,不求诸人,勿失也。

(6) 求诸己则存,求诸人则亡。

齐攻鲁,求岑鼎。鲁君载他鼎以往。齐侯弗信而反之,为非, (1) 使人告鲁侯曰:“柳下季以为是,请因受之。” (2) 鲁君请于柳下季。 (3) 柳下季答曰:“君之赂以欲岑鼎也, (4) 以免国也。臣亦有国于此, (5) 破臣之国以免君之国,此臣之所难也。”于是鲁君乃以真岑鼎往也。 (6) 且柳下季可谓此能说矣, (7) 非独存己之国也,又能存鲁君之国。 (8)

(1) 反,还也。以为非岑鼎,故还也。

(2) 齐侯使人告鲁君,言柳下季以为是岑鼎,请因受之也。疑鲁君欺之,而信柳下季。

(3) 欲令柳下季证之为岑鼎。

(4) 【校】犹言赂以其所欲之岑鼎。《新序·节士》篇作“君之欲以为岑鼎也。”

(5) 亦有国于此,言己有此信以为国也。

(6) 【校】《韩非·说林下》“岑鼎”作“谗鼎”,又属之乐正子春。若是两事,则各是一鼎,名各不同,否则传者互异,岑与谗声通转耳。

(7) 【校】《新序》作“可谓守信矣”。

(8) 《论语》云:“非信不立。”柳下季有信,故能存鲁君之国。

齐湣王亡居于卫, (1) 昼日步足,谓公玉丹曰:“我已亡矣,而不知其故。吾所以亡者,果何故哉?我当已。” (2) 公玉丹答曰:“臣以王为已知之矣,王故尚未之知邪?王之所以亡也者,以贤也。天下之王皆不肖,而恶王之贤也,因相与合兵而攻王,此王之所以亡也。”湣王慨焉太息曰:“贤固若是其苦邪?”此亦不知其所以也, (3) 此公玉丹之所以过也。 (4)

(1) 亡,出奔。

(2) 不自知为何故而亡。果亦竟也。竟为何等故亡哉?

【校】案:《史记·孝武本纪》索隐云:“《风俗通》齐湣王臣有公玉冉,音语录反。”又引《三辅决录》云:“杜陵有玉氏,音肃。”“今读公玉与《决录》音同”。卢云:“案‘丹’与‘冉’字形相近,实一人。”《贾谊书》所载虢君事略与此同。注“亦竟也”,李本作“一竟也”。

(3) 湣王不自知其所为亡之故,愚惑之甚也,故曰“亦不知其所以也”。

(4) 过,谓不忠也。湣王愚惑,阿顺而说之也。

越王授有子四人。越王之弟曰豫,欲尽杀之,而为之后。 (1) 恶其三人而杀之矣,国人不说,大非上。 (2) 又恶其一人而欲杀之,越王未之听。其子恐必死,因国人之欲逐豫,围王宫。越王太息曰:“余不听豫之言,以罹此难也。”亦不知所以亡也。 (3)

(1) 越王授,勾践五世之孙。其弟欲杀王之四子,而以己代为之后也。

【校】案:“勾践五世孙则王翳也,为太子诸咎所弑,见《纪年》,与此略相合。”前《贵生》篇有王子搜,疑一人。注“其弟”二字旧缺,案文义增。

(2) 非犹咎也。

(3) 愚既愚也,其惑固亦甚也,故曰“亦不知所以亡”。

【校】正文“亦不知”下,李本有“其”字。注首疑有脱误。





精通


五曰:

人或谓兔丝无根。兔丝非无根也,其根不属也,伏苓是。 (1) 慈石召铁,或引之也。 (2) 树相近而靡,或 之也。 (3) 圣人南面而立,以爱利民为心, (4) 号令未出,而天下皆延颈举踵矣,则精通乎民也。 (5) 夫贼害于人,人亦然。 (6) 今夫攻者,砥厉五兵,侈衣美食,发且有日矣。所被攻者不乐,非或闻之也,神者先告也。 (7) 身在乎秦,所亲爱在于齐,死而志气不安,精或往来也。 (8)

(1) 属,连也。《淮南记》曰:“下有茯苓,上有兔丝。”一名女罗,《诗》曰:“葛与女罗,施于松上。”

【校】案:注所引与今《诗》异。

(2) 石,铁之母也。以有慈石,故能引其子。石之不慈者,亦不能引也。

(3) 【校】案:《淮南·氾论训》“相戏以刃者,太祖 其肘”,音读茸,注“挤”也。

(4) 心在利民。

(5) 天下皆延颈企踵,立而望之,不遑坐也,其精诚能通洞于民使之然也。

(6) 为贼害人,故人亦延颈举踵,襁负而去之,不遑安坐也,故曰“人亦然”。

(7) 非闻将见攻也,神先告之,令其志意愁戚不乐。

(8) 《淮南记》曰:“慈母在于燕,適子念于荆。”言精相往来者也。

德也者,万民之宰也。 (1) 月也者,群阴之本也。月望则蚌蛤实,群阴盈; (2) 月晦则蛘蛤虚,群阴亏。 (3) 夫月形乎天,而群阴化乎渊; (4) 圣人行德乎己,而四荒咸饬乎仁。 (5)

(1) 宰,主也。

(2) 月十五日盈满,在西方与日相望也。蚌蛤,阴物,随月而盛,其中皆实满也。

(3) 虚,蚌蛤肉随月亏而不盈满也。

(4) 形,见也。群阴,蚌蛤也。随月盛衰虚实也。

(5) 四表荒裔之民,法圣人之德,皆饬正其仁义,化使之然。

养由基射 ,中石,矢乃饮羽,诚乎 也。 (1) 伯乐学相马,所见无非马者,诚乎马也。 (2) 宋之庖丁好解牛,所见无非死牛者,三年而不见生牛。用刀十九年,刃若新 研, (3) 顺其理,诚乎牛也。钟子期夜闻击磬者而悲, (4) 使人召而问之曰:“子何击磬之悲也?”答曰:“臣之父不幸而杀人,不得生;臣之母得生,而为公家为酒;臣之身得生,而为公家击磬。臣不睹臣之母三年矣。昔为舍氏睹臣之母,量所以赎之则无有, (5) 而身固公家之财也,是故悲也。” (6) 钟子期叹嗟曰:“悲夫,悲夫!心非臂也,臂非椎非石也,悲存乎心而木石应之。”故君子诚乎此而谕乎彼,感乎己而发乎人,岂必强说乎哉?

(1) 饮羽,饮矢至羽。诚以为真 也。

【校】“ ”乃“兕”之或体。旧误作“先”,校者欲改为“虎”,非也。日本山井鼎《毛诗考文》云:“‘兕觥’古本作‘ ’。”

(2) 伯乐善相马,秦穆公之臣也。所见无非马者,亲之也。

(3) ,砥也。

(4) 钟,姓也。子,通称。期,名也。楚人钟仪之族。

(5) 量,度。

(6) 【校】《新序》四载此微不同,云“昨日为舍市而睹之,意欲赎之无财,身又公家之有也。”孙云:“《新序》义较长。”

周有申喜者,亡其母,闻乞人歌于门下而悲之,动于颜色。谓门者内乞人之歌者,自觉而问焉, (1) 曰:“何故而乞?”与之语,盖其母也。故父母之于子也,子之于父母也,一体而两分, (2) 同气而异息。若草莽之有华实也,若树木之有根心也,虽异处而相通,隐志相及,痛疾相救,忧思相感, (3) 生则相欢,死则相哀,此之谓骨肉之亲。神出于忠 (4) 而应乎心,两精相得,岂待言哉!

(1) 【校】《御览》五百七十一“自觉”作“自见”。

(2) 【校】李善注《文选》曹子建《求自试表》、谢希逸《宣贵妃诔》皆作“一体而分形”。

(3) 感,动。

(4) 神,性。




————————————————————

[1] 陶鸿庆曰:“和子”下不当有“曰”字,盖“因”字之误。





第十卷 孟冬纪



孟冬


一曰:

孟冬之月,日在尾, (1) 昏危中,旦七星中。 (2) 其日壬癸, (3) 其帝颛顼,其神玄冥。 (4) 其虫介,其音羽, (5) 律中应钟,其数六。 (6) 其味咸,其臭朽, (7) 其祀行,祭先肾。 (8) 水始冰,地始冻, (9) 雉入大水为蜃,虹藏不见。 (10) 天子居玄堂左个, (11) 乘玄辂,驾铁骊, (12) 载玄旂,衣黑衣,服玄玉, (13) 食黍与彘, (14) 其器宏以弇。 (15)

(1) 孟冬,夏之十月。尾,东方宿燕之分野。是月,日躔此宿。

(2) 危,北方宿,齐之分野。七星,南方宿,周之分野,是月昏旦时,皆中于南方。

(3) 壬癸,水日。

(4) 颛顼,黄帝之孙,昌意之子,以水德王天下,号高阳氏,死祀为北方水德之帝。玄冥,官也。少皞氏之子曰循,为玄冥师,死祀为水神。

【校】注“高阳氏”,旧本作“汤氏”讹,今改正。又“循”,《左转》作“脩”。

(5) 介,甲也,象冬闭固,皮漫胡也。羽,水也,位在北方。

【校】注“漫”与“曼”、“ ”音义同。皮漫胡,谓皮长而下垂,亦似闭固之象。

(6) 应钟,阴律也。竹管音与应钟和也。阴应于阳,转成其功,万物聚藏,故曰“律中应钟”。其数六,五行数五,水第一,故曰“六”也。

(7) 水之臭味也,凡咸朽者皆属焉。气之若有若无者为朽也。

(8) 行,门内地也,冬守在内,故祀之。“行”或作“井”,水给人,冬水王,故祀之也。祭祀之内先进肾,属水,自用其藏也。

【校】案:《淮南·时则训》作“祀井”。

(9) 秋分后三十日霜降,后十五日立冬,水冰地冻也,故曰“始”也。

(10) 蜃,蛤也。大水,淮也。《传》曰:“雉入于淮为蜃。”虹,阴阳交气也,是月阴壮,故藏不见。

(11) 玄堂,北向堂也。左个,西头室也。

(12) 玄辂,黑辂;铁骊亦黑。象北方也。

(13) 玄,黑,顺水色。

(14) 彘,水属也。

(15) 宏,大。弇,深。象冬闭藏也。

是月也,以立冬。先立冬三日,太史谒之天子曰 (1) :“某日立冬,盛德在水。”天子乃斋。 (2) 立冬之日,天子亲率三公、九卿、大夫以迎冬于北郊。 (3) 还,乃赏死事,恤孤寡。 (4)

(1) 秋分四十六日而立冬,故多在是月也。谒,告也。

(2) 盛德在水,王北方也。

(3) 六里之郊。

(4) 先人有死王事以安社稷者,赏其子孙;有孤寡者,矜恤之。

是月也,命太卜,祷祠龟策,占兆审卦吉凶。 (1) 于是察阿上乱法者则罪之,无有掩蔽。 (2)

(1) 《周礼》“太卜掌三兆之法,一曰玉兆,二曰瓦兆,三曰原兆”,又“掌三《易》之法,一曰《连山》,二曰《归藏》,三曰《周易》”。龟曰兆,筮曰卦,故命太卜祷祠龟策,占兆审卦以知吉凶。

【校】《月令》作“命太史衅龟 ”。

(2) 阿意曲从,取容于上,以乱法度,必察知之,则行其罪罚,无敢强匿者。

【校】《月令》作“是察阿党,则罪无有掩蔽”。古本《月令》“是”下有“月也”二字,宋本《正义》标题亦有“是月”字。

是月也,天子始裘。 (1) 命有司曰:“天气上腾,地气下降,天地不通,闭而成冬。” (2) 命百官,谨盖藏。命司徒,循行积聚,无有不敛;附城郭, (3) 戒门闾,修楗闭,慎关籥,固封玺, (4) 备边境,完要塞,谨关梁,塞蹊径; (5) 饬丧纪,辨衣裳,审棺椁之厚薄, (6) 营丘垄之小大、高卑、薄厚之度,贵贱之等级。 (7)

(1) 始犹先也。裘,温服。优尊者,故先服之。

(2) 天地闭,冰霜凛烈成冬也。

【校】《月令》“闭”下有“塞”字。

(3) 附,益也,令高固也。

【校】“附”,《月令》作“坏”。

(4) 玺,读曰移徙之徙。门闾,里门。关,籥。固,坚。玺,印封也。

【校】《月令》“楗”作“键”,“关”作“管”,“玺”作“疆”。郑注云:“今《月令》‘疆’或作‘玺’。”

(5) 要塞,所以固国也。关梁,所以通涂也。塞绝蹊径,为其败田。

(6) 纪,数也。正二十五月之服数,遣送衣裳棺椁,尊者厚,卑者薄,各有等差,故别之。审,慎也。

【校】注“正二十五月之服数”,举重者则其余皆正可知也。“之服数”,旧作“服之数”,今案文义乙正。

(7) 营,度也。丘,坟;垄,冢也。度其制度,贵者高大,贱者卑小,故曰“等级”也。

是月也,工师效功,陈祭器,按度程, (1) 无或作为淫巧,以荡上心, (2) 必功致为上。物勒工名,以考其诚。 (3) 工有不当,必行其罪,以穷其情。 (4)

(1) 程,法也。

【校】《月令》“工师”上有“命”字。

(2) 荡,动也。

(3) 物,器也。勒铭工姓名著于器,使不得诈巧,故曰“以考其诚”。

(4) 不当,不功致也,故行其罪,以穷断其诈巧之情。

【校】《月令》“工”作“功”。

是月也,大饮蒸,天子乃祈来年于天宗。 (1) 大割,祠于公社及门闾,飨先祖五祀,劳农夫以休息之。 (2) 天子乃命将率讲武,肄射御、角力。 (3)

(1) 是月农功毕矣,天子诸侯与其群臣大饮酒,班齿列也。蒸,俎实体解节折谓肴蒸也。祈,求也。求明年于天宗之神。宗,尊也。凡天地四时,皆为天宗。万物非天不生,非地不载,非春不动,非夏不长,非秋不成,非冬不藏,《书》曰“禋于六宗”,此之谓也。

【校】注“班齿列”即《周礼》之“正齿位”也,旧本倒作“列齿”,误;又“体解”亦缺“体”字,又“求明年于天宗之神”倒作“之神于天宗”,今皆改正。

(2) 大割,杀牲也。祠于公社、国社、后土也。生为上公,死祀为贵神也。先祠公社,乃及门闾先祖,先公后私之义也。五祀:木正句芒其祀户,火正祝融其祀灶,土正后土其祀中霤,后土为社,金正蓐收其祀门,水正玄冥其祀井,故曰“五祀”。社为土官,稷为木官,俱在五祀中,以其功大,故别言社稷耳。是月农夫空闲,故劳犒休息之,不役使也。

【校】旧本“大割”下有“牲”字,《月令》无,案注亦与《月令》同,今删。“飨”,《月令》作“腊”。旧本“先祖”作“祷祖”,亦据《月令》及本注改正。

(3) 肄,习也。角犹试。

是月也,乃命水虞渔师,收水泉池泽之赋, (1) 无或敢侵削众庶兆民, (2) 以为天子取怨于下, (3) 其有若此者,行罪无赦。 (4)

(1) 虞,官也。师,长也。赋,税也。

(2) 削,刻也。天子曰兆民。兆,大数也。

(3) 税敛重则民怨,故取怨于下。

(4) 此为天子取怨于下者,故行其罪罚无赦贷也。

孟冬行春令,则冻闭不密,地气发泄,民多流亡。 (1) 行夏令,则国多暴风,方冬不寒,蛰虫复出。 (2) 行秋令,则雪霜不时,小兵时起,土地侵削。 (3)

(1) 春阳散越,故冻不密,地气发泄,使民流亡,象阳布散。

【校】“发泄”,《月令》作“上泄”。

(2) 冬法当闭藏,反行夏盛阳之令,故多暴疾之风。阳气炎温,故盛冬不寒,蛰伏之虫复出也,于《洪范》五行“豫,恒燠若”之征也。

(3) 秋,金气,干水,不当霜而霜,不当雪而雪,故曰“不时”。小兵数起,邻国来伐,侵削土地,于《洪范》五行“急,恒寒若”之征也。





节丧


二曰:

审知生,圣人之要也;审知死,圣人之极也。知生也者,不以害生,养生之谓也;知死也者,不以害死,安死之谓也。 (1) 此二者,圣人之所独决也。 (2) 凡生于天地之间,其必有死,所不免也。 (3) 孝子之重其亲也, (4) 慈亲之爱其子也, (5) 痛于肌骨,性也。所重所爱,死而弃之沟壑,人之情不忍为也,故有葬死之义。 (6) 葬也者,藏也,慈亲孝子之所慎也。 (7) 慎之者,以生人之心虑。 (8) 以生人之心为死者虑也,莫如无动,莫如无发。无发无动,莫如无有可利,则此之谓重闭。 (9)

(1) 【校】《续汉书·礼仪志下》注引此“不以物害生”、“不以物害死”两句,皆有“物”字。

(2) 决,知。

(3) 《庄子》曰:“生,寄也;死,归也。”故曰“所不免”。

(4) 重,尊。

(5) 爱,心不能忘也。

【校】《续志》注“慈”作“若”,以下文观之,“慈”字是。

(6) 言情不忍弃之沟壑,故有葬送之义。

(7) 慎,重也。

(8) 虑,计也。

(9) 无有可利,若杨王孙倮葬,人不发掘,不见动摇,谓之重闭也。

古之人,有藏于广野深山而安者矣,非珠玉国宝之谓也,葬不可不藏也。葬浅则狐狸抇之, (1) 深则及于水泉。故凡葬必于高陵之上,以避狐狸之患、水泉之湿。此则善矣,而忘奸邪、盗贼、寇乱之难,岂不惑哉? (2) 譬之若瞽师之避柱也,避柱而疾触杙也。狐狸、水泉、奸邪、盗贼、寇乱之患,此杙之大者也。慈亲孝子避之者,得葬之情矣。 (3) 善棺椁,所以避蝼蚁蛇虫也。今世俗大乱之主愈侈其葬,则心非为乎死者虑也,生者以相矜尚也。 (4) 侈靡者以为荣, (5) 俭节者以为陋,不以便死为故, (6) 而徒以生者之诽誉为务。此非慈亲孝子之心也。父虽死,孝子之重之不怠; (7) 子虽死,慈亲之爱之不懈。夫葬所爱所重,而以生者之所甚欲,其以安之也,若之何哉? (8)

(1) 抇,读曰掘。

(2) 厚葬,人利之,必有此难。故谓之惑也。

(3) 得薄葬之情也。

【校】旧校云:“‘避’一作‘备’。”下同。

(4) 虑,计也。以厚葬奢侈相高大、不为葬者避发掘之计也,故曰“生者以相矜尚也”。

(5) 荣,誉也。

(6) 故,事。

(7) 重,尊。怠,懈。

(8) 甚欲,欲厚葬也。厚葬必见发掘,故曰“其以安之也,若之何哉”?言不安也。

民之于利也,犯流矢、蹈白刃,涉血 肝以求之。 (1) 野人之无闻者,忍亲戚、兄弟、知交以求利。 (2) 今无此之危,无此之丑, (3) 其为利甚厚,乘车食肉,泽及子孙。虽圣人犹不能禁,而况于乱? (4) 国弥大, (5) 家弥富,葬弥厚。含珠鳞施, (6) 夫玩好货宝,钟鼎壶滥, (7) 舆马衣被戈剑,不可胜其数, (8) 诸养生之具,无不从者。 (9) 题凑之室, (10) 棺椁数袭, (11) 积石积炭,以环其外。 (12) 奸人闻之,传以相告。 (13) 上虽以严威重罪禁之,犹不可止。 (14) 且死者弥久,生者弥疏;生者弥疏,则守者弥怠;守者弥怠,而葬器如故, (15) 其势固不安矣。世俗之行丧,载之以大 , (16) 羽旄旌旗如云,偻翣以督之,珠玉以备之,黼黻文章以饬之, (17) 引绋者左右万人以行之, (18) 以军制立之然后可。 (19) 以此观世, (20) 则美矣,侈矣,以此为死,则不可也。 (21) 苟便于死,则虽贫国劳民, (22) 若慈亲孝子者之所不辞为也。

(1) “ ”,古“抽”字。

(2) 无闻礼义。

(3) 丑,耻。

(4) 【校】卢云:“疑此下当有‘世’字。盖言圣人在上,治平之世,犹有贪利而冒禁者,况于四海鼎沸之日,其又谁为禁之哉?”

(5) 弥犹益也。

(6) 含珠,口实也。鳞施,施玉于死者之体如鱼鳞也。

(7) 以冰置水浆于其中为滥,取其冷也。

【校】梁仲子云:“‘壶滥’,刘本作‘壶鑑’,注同。案《集韵》‘鑑,胡暂切’。《周礼》‘春始治鑑’,或从‘水’,亦作‘ ’、‘ ’,故《左传襄九年正义》引《周礼》作‘ ’。”卢云:“案《墨子·节葬》篇云:‘又必多为屋幕,鼎鼓几梴,壶滥戈剑,羽毛齿革,寝而埋之。’凡两见,盖亦器名,注似臆说。《慎势》篇作‘壶鑑’,云‘功名著乎盘盂,铭篆著乎壶鑑’。”

(8) 【校】“其”字衍。

(9) 诸养生之具无不从。从,送也。以送死人。

(10) 室,椁藏也。题凑,複絫。

【校】案:《汉书·霍光传》“便房黄肠题凑”,注引苏林曰:“以柏木黄心致絫棺外,故曰‘黄肠’。木头皆内向,故曰‘题凑’。”

(11) 袭,重。

(12) 石以其坚,炭以御湿。环,绕也。

【校】案:积炭非但御湿,亦使树木之根不穿入也。

(13) 告,语也。

【校】“传”,《续志》注作“转”。

(14) 不能止其发掘。

(15) 言宝赂不渝变。

(16) 大 ,车也。

(17) 丧车有羽旄旌旗之饬,有云气之画。偻,盖也。翣,棺饬也。画黼黻之状如扇翣于偻边,天子八,诸侯六,大夫四也。

【校】案:《礼记·檀弓下》云:“制绞衾,设蒌翣,为使人勿恶也。”注云:“蒌翣,棺之墙饰也。”此作“偻”,或音同可借用。此“饬”字义皆是“饰”。

(18) 绋,引棺索也。礼,送葬皆执绋。

(19) 制,法。

(20) 观世犹示人也。

(21) 于死人不可也。

(22) 【校】旧校云:“一作‘身’。”





安死


三曰:

世之为丘垄也,其高大若山,其树之若林, (1) 其设阙庭、为宫室、造宾阼也若都邑。 (2) 以此观世示富则可矣,以此为死则不可也。夫死,其视万岁犹一瞚也。 (3) 人之寿,久之不过百, (4) 中寿不过六十。以百与六十为无穷者之虑, (5) 其情必不相当矣。以无穷为死者之虑,则得之矣。

(1) 木藂生曰林也。

【校】《续志》注“山”下有“陵”字,“林”下有“薮”字。

(2) 宾阶,阼阶也。若为都邑之制。

(3) 瞚者,颍川人相视曰瞚也。一曰瞚者,谓人卧始觉也。

【校】“瞚”与“瞬”同。李善注《文选》陆士衡《文赋》引作“万世犹一瞬”。

(4) 【校】“久之”,《续志》注作“久者”。

(5) 虑,谋也。

今有人于此,为石铭置之垄上,曰:“此其中之物,具珠玉、玩好、财物、宝器甚多,不可不抇, (1) 抇之必大富,世世乘车食肉。” (2) 人必相与笑之,以为大惑。 (3) 世之厚葬也,有似于此。 (4) 自古及今,未有不亡之国也;无不亡之国者,是无不抇之墓也。以耳目所闻见,齐、荆、燕尝亡矣,宋、中山已亡矣,赵、魏、韩皆亡矣,其皆故国矣。 (5) 自此以上者,亡国不可胜数, (6) 是故大墓无不抇也。而世皆争为之,岂不悲哉? (7)

(1) 抇,发也。

(2) 谓抇墓富而得爵禄,故乘车食肉,世世相传也。

(3) 惑,悖也。

(4) 【校】《续志》注作“而为之阙庭以自表,此何异彼哉”。

(5) 【校】《续志》注作“赵、韩、魏皆失其故国矣”。

(6) 上犹前也。不可胜数,亡国多也。

【校】“者”字《续志》无。

(7) 【校】《续志》注“世”作“犹”。

君之不令民, (1) 父之不孝子,兄之不悌弟,皆乡里之所釜 者而逐之。 (2) 惮耕稼采薪之劳,不肯官人事, (3) 而祈美衣侈食之乐, (4) 智巧穷屈,无以为之。 (5) 于是乎聚群多之徒,以深山广泽林薮,扑击遏夺,又视名丘大墓葬之厚者,求舍便居,以微抇之, (6) 日夜不休,必得所利,相与分之。夫有所爱所重,而令奸邪、盗贼、寇乱之人卒必辱之,此孝子、忠臣、亲父、交友之大事。 (7)

(1) 令,善。

【校】《续志》注句上有“今夫”二字。

(2) 以釜 食之人,皆欲讨逐之。

【校】“ ”,旧“鬲”旁作“几”,字书无考。顾亭林引作“ ”,注云“鬲同”,今从之。《史记·蔡泽传》“遇夺釜鬲于涂”。

(3) 既惮耕稼,又不肯居官循治人事也。

【校】注“循治”,疑当作“修治”。

(4) 祈,求。

(5) 穷,极。屈,尽。

(6) 【校】有人自关中来者,为言奸人掘墓,率于古贵人冢旁相距数百步外为屋以居,人即于屋中穿地道以达于葬所,故从其外观之,未见有发掘之形也,而藏已空矣。噫!孰知今人之巧,古已先有为之者。小人之求利,无所不至,初无古今之异也。

(7) 《传》曰:“宋文公卒,始厚葬,用蜃炭,益车马,始用殉,重器备,椁有四阿,棺有翰桧,君子谓华元、乐吕于是不臣。臣治烦去惑者也,是以伏死而争。今二子者,君生则纵其惑,死也又益其侈,是弃君于恶也,何臣之为?”此之谓也。

尧葬于穀林,通树之; (1) 舜葬于纪市,不变其肆; (2) 禹葬于会稽,不变人徒。 (3) 是故先王以俭节葬死也,非爱其费也, (4) 非恶其劳也, (5) 以为死者虑也。先王之所恶,惟死者之辱也。发则必辱,俭则不发。故先王之葬,必俭,必合,必同。何谓合?何谓同?葬于山林则合乎山林,葬于阪隰 (6) 则同乎阪隰。此之谓爱人。夫爱人者众,知爱人者寡。 (7) 故宋未亡而东冢抇, (8) 齐未亡而庄公冢抇。 (9) 国安宁而犹若此,又况百世之后而国已亡乎?故孝子、忠臣、亲父、交友不可不察于此也。夫爱之而反危之,其此之谓乎! (10) 《诗》曰:“不敢暴虎,不敢冯河。人知其一,莫知其他。”此言不知邻类也。 (11) 故反以相非,反以相是。其所非方其所是也,其所是方其所非也。 (12) 是非未定,而喜怒斗争反为用矣。吾不非斗,不非争, (13) 而非所以斗,非所以争。故凡斗争者,是非已定之用也。今多不先定其是非,而先疾斗争,此惑之大者也。 (14)

(1) 通林以为树也,《传》曰“尧葬成阳”,此云穀林,成阳山下有穀林。

【校】尧葬成阳,《水经注》言之甚晰。又案:刘向云“葬济阴丘陇山”,《续征记》“在小成阳南九里”,《通典》“曹州界有尧冢,尧所居”,其说皆非。罗苹《路史注》以《墨子》云“尧葬蛩山之阴”,王充云“葬冀州”,《山海经》云“葬狄山,或云葬崇山”,皆妄之甚。

(2) 市肆如故,言不烦民也。《传》曰“舜葬苍梧九疑之山”,此云于纪市,九疑山下亦有纪邑。

【校】《墨子》云“舜葬南己之市”,《御览》五百五十五作“南纪”,引《尸子》作“南己”。案:《路史注》云:“纪即冀,故纪后为冀后。今河东皮氏东北有冀亭。冀,子国也。鸣条在安邑西北,其地相近。《记》谓舜葬苍梧,《皇览》谓在零陵营浦县,尤失之。”梁伯子云:“《困学纪闻》五引薛氏言苍梧在海州界,近莒之纪城,亦非。阎伯诗云:‘海州苍梧山,即《山海经》之郁州,无舜葬于此之说。’”

(3) 变,动也。言无所兴造,不扰民也。会稽山在会稽山阴县南。

(4) 费,财也。

(5) 恶犹患也。

(6) 【校】旧校云:“一作‘阪阬’。”

(7) 谓凡爱死人者之众,多厚葬之。知所以爱之者寡,言能俭葬者少也。

(8) 东冢,文公冢也。文公厚葬,故冢被发也。冢在城东,因谓之东冢。

(9) 庄公名购,僖公之父。以葬厚,冢见发。

(10) 使见发掘之谓。

【校】《续志》注作“欲爱而反害之,欲安而反危之,忠臣孝子亦不可以厚葬矣”。

(11) 《诗·小雅·小旻》之卒章也。无兵搏虎曰暴。无舟渡河曰冯。喻小人而为政,不可以不敬,不敬之则危,犹暴虎冯河之必死也。人知其一,莫知其他。一,非也,人皆知小人之为非,不知不敬小人之危殆,故曰“不知邻类也”。

(12) 方,比。

(13) 非犹罪也。

(14) 【校】“故反以相非”以下,似《不二》篇之文误脱于此。

鲁季孙有丧,孔子往吊之。入门而左,从客也。主人以玙璠收, (1) 孔子径庭而趋,历级而上, (2) 曰:“以宝玉收,譬之犹暴骸中原也。” (3) 径庭历级,非礼也;虽然,以救过也。 (4)

(1) 丧,季平子意如之丧也。桓子斯在丧位,孔子吊之,入门而左行,故曰从客位也。主人以玙璠收,收,敛者也。

(2) 上堂。

(3) 玙璠,君佩玉也。昭公在外,平子行君事,入宗庙佩玙璠,故用之。孔子以平子逐昭公出之,其行恶,不当以敛,而反用之,肆行非度,人又利之,必见发掘,故犹暴骸中原也。

(4) 孔子“拜下,礼也。今拜乎上,泰也。虽违众,吾从下”,言不欲违礼,亦不欲人之失礼,故历级也。





异宝


四曰:

古之人非无宝也,其所宝者异也。孙叔敖疾,将死,戒其子曰:“王数封我矣,吾不受也。 (1) 为我死,王则封汝,必无受利地。 (2) 楚、越之间有寝之丘者,此其地不利, (3) 而名甚恶。 (4) 荆人畏鬼,而越人信 。 (5) 可长有者,其唯此也。” (6) 孙叔敖死,王果以美地封其子,而子辞, (7) 请寝之丘,故至今不失。孙叔敖之知,知不以利为利矣。知以人之所恶为己之所喜,此有道者之所以异乎俗也。 (8)

(1) 孙叔敖,楚大夫 贾之子,庄王之令尹也。

(2) 人所贪利之地。

【校】“为”字衍,《后汉书·郭丹传》注引此无。

(3) 人不利之。

【校】《列子·说符》篇、《淮南·人间训》皆作“寝邱”,无“之”字,《史记·滑稽传》正义引此同。

(4) 恶,谓丘名也。

【校】《史记正义》引作“而前有垢谷,后有戾邱,其名恶,可长有也”。此见《淮南》注。此注自谓寝邱名恶,非有缺文。

(5) 言荆人畏鬼神,越人信吉凶之 祥,此地名丘畏恶之名,终不利也。

(6) 唯,独也。

(7) 【校】《后汉书》作“其子辞”。

(8) 众人利利,孙叔敖病利,故曰“所以异于俗”也。

五员亡,荆急求之,登太行而望郑曰:“盖是国也,地险而民多知, (1) 其主俗主也,不足与举。” (2) 去郑而之许,见许公而问所之。许公不应,东南向而唾。 (3) 五员载拜受赐曰:“知所之矣。”因如吴。过于荆,至江上,欲涉, (4) 见一丈人, (5) 刺小船,方将渔,从而请焉。丈人度之,绝江。 (6) 问其名族, (7) 则不肯告。 (8) 解其剑以予丈人, (9) 曰:“此千金之剑也,愿献之丈人。” (10) 丈人不肯受,曰:“荆国之法,得五员者,爵执圭,禄万檐, (11) 金千镒。昔者子胥过,吾犹不取, (12) 今我何以子之千金剑为乎?” (13) 五员过于吴, (14) 使人求之江上,则不能得也。每食必祭之,祝曰:“江上之丈人!天地至大矣,至众矣,将奚不有为也?而无以为。为矣 (15) 而无以为之,名不可得而闻, (16) 身不可得而见, (17) 其惟江上之丈人乎!”

(1) 登,升也。太行,山名,处则未闻。多知,将问所以自窜也。

【校】案:高氏注《淮南·地形训》云:“太行,在今上党太行关,直河内野王县是也。”此何以云“处则未闻”?此山今在河南辉县西北,与山西泽州相邻也。

(2) 举犹谋也。俗主,不肖凡君。

(3) 欲令之吴也。

(4) 涉,渡。

(5) 丈人,长老称也。

(6) 绝,过。

(7) 族,姓。

(8) 丈人不肯告。

(9) 【校】旧校云:“‘予’一作‘献’。”

(10) 献,上也。

(11) 【校】“檐”与“儋”古通用,今作“担”。

(12) 执圭,《周礼》“侯执信圭”,言爵之为侯也。万檐,万石也。金千镒,二十两为一镒。不取子胥以受赏也,故曰“我何以欲子之千金剑为”。

【校】旧校云:“‘犹’一作‘尚’。”

(13) 【校】旧校云:“‘何’一作‘曷’。”梁伯子云:“此江上丈人伪言也。因揣知必五员,故作此言以拒之耳。”

(14) 过犹至也。

(15) 何不有为,言无不为也。江上丈人无以为矣,无以为,乃大有于五员也,故曰“而无以为”也。

【校】案:注当云“乃大有为于五员也,故曰而无以为为也”,脱两“为”字。

(16) 闻,知也。

(17) 求之江上,不能得也。

宋之野人耕而得玉,献之司城子罕。子罕不受。 (1) 野人请曰:“此野人之宝也,愿相国为之赐而受之也。”子罕曰:“子以玉为宝,我以不受为宝。”故宋国之长者曰:“子罕非无宝也,所宝者异也。”

(1) 司城,官名。

今以百金与抟黍以示儿子, (1) 儿子必取抟黍矣;以和氏之璧与百金以示鄙人,鄙人必取百金矣;以和氏之璧、道德之至言以示贤者,贤者必取至言矣。其知弥精,其所取弥精;其知弥觕,其所取弥觕。 (2)

(1) 儿子,小儿。

(2) 精,微妙也。觕,粗疏也。





异用


五曰:

万物不同,而用之于人异也,此治乱、存亡、死生之原。 (1) 故国广巨,兵强富, (2) 未必安也;尊贵高大,未必显也;在于用之。桀、纣用其材而以成其亡,汤、武用其材而以成其王。

(1) 原,本。

(2) 【校】旧校云:“一作‘充富’。”

汤见祝网者,置四面, (1) 其祝曰:“从天坠者, (2) 从地出者,从四方来者,皆离吾网。”汤曰:“嘻!尽之矣。非桀,其孰为此也?” (3) 汤收其三面, (4) 置其一面,更教祝曰:“昔蛛蝥作网罟,今之人学纾。 (5) 欲左者左,欲右者右,欲高者高,欲下者下,吾取其犯命者。”汉南之国闻之曰:“汤之德及禽兽矣!” (6) 四十国归之。 (7) 人置四面,未必得鸟;汤去其三面,置其一面,以网其四十国,非徒网鸟也。 (8)

(1) 置,设。

(2) 坠,陨也。

(3) 孰,谁也。

(4) 【校】旧校云:“‘收’一作‘放’。”孙云:“李善注《文选》张平子《东京赋》、扬子云《羽猎赋》引此‘收’并作‘拔’,旧校当是‘一作拔’。”

(5) 纾,缓。

【校】《贾谊书·谕诚》篇“蛛蝥作网,今之人循绪”。旧本“蝥”作“螯”,误。“纾”疑与“杼”通,注训为缓,非是。

(6) 汉南,汉水之南。

(7) 【校】梁仲子云:“李善注《东京赋》作‘三十国’。”

(8) 徒犹但也。

周文王使人抇池,得死人之骸。吏以闻于文王,文王曰:“更葬之。”吏曰:“此无主矣。”文王曰:“有天下者,天下之主也;有一国者,一国之主也。今我非其主也?” (1) 遂令吏以衣棺更葬之。天下闻之曰:“文王贤矣!泽及髊骨, (2) 又况于人乎?”或得宝以危其国,文王得朽骨以喻其意, (3) 故圣人于物也无不材。 (4)

(1) 【校】“也”与“邪”古通用。《御览》八十四作“邪”。

(2) 骨有肉曰髊,无曰枯。

(3) 喻,说。说民意也。

(4) 材,用也。

孔子之弟子从远方来者,孔子荷杖而问之曰:“子之公不有恙乎?”搏杖而揖之,问曰:“子之父母不有恙乎?”置杖而问曰:“子之兄弟不有恙乎?”杙步而倍之,问曰:“子之妻子不有恙乎?” (1) 故孔子以六尺之杖,谕贵贱之等,辨疏亲之义,又况于以尊位厚禄乎?

(1) 【校】孙云:“《御览》七百十‘公’作‘父’,下无‘父’字,‘搏杖’作‘持杖’,‘杙步而倍之’作‘杖步而倚之’。《广韵》‘杖’字下引云‘孔子见弟子,抱杖而问其父母,柱杖而问其兄弟,曳杖而问其妻子,尊卑之差也’,盖约此文。”

古之人贵能射也,以长幼养老也。 (1) 今之人贵能射也,以攻战侵夺也。其细者以劫弱暴寡也,以遏夺为务也。仁人之得饴, (2) 以养疾侍老也。 (3) 跖与企足得饴,以开闭取楗也。 (4)

(1) 礼,射中饮不中,故所以长幼养老也。

(2) 饴,饧。

(3) 侍亦养也。

(4) 跖,盗跖;企足,庄 也;皆大盗人名也。以饴取人楗牡,开人府藏,窃人财物者也。

【校】案:《淮南·说林训》:“柳下惠见饴曰‘可以养老’,盗跖见饴曰‘可以黏牡’,见物同而用之异。”注:“牡,门户籥牡。”此云楗即牡也。黏牡使之无声,又开之滑易也。





第十一卷 仲冬纪



仲冬


一曰:

仲冬之月,日在斗, (1) 昏东壁中,旦轸中。 (2) 其日壬癸,其帝颛顼,其神玄冥。其虫介,其音羽, (3) 律中黄钟, (4) 其数六。其味咸,其臭朽,其祀行,祭先肾。冰益壮,地始坼, (5) 鹖 不鸣,虎始交。 (6) 天子居玄堂太庙, (7) 乘玄辂,驾铁骊,载玄旂,衣黑衣,服玄玉,食黍与彘,其器宏以弇。 (8) 命有司曰:“土事无作,无发盖藏,无起大众,以固而闭。” (9) 发盖藏,起大众,地气且泄,是谓发天地之房。 (10) 诸蛰则死,民多疾疫,又随以丧, (11) 命之曰畅月。 (12)

(1) 仲冬,夏之十一月。斗,北方宿,吴之分野。是月,日躔此宿。

【校】案《淮南·天文训》,斗属越。

(2) 东壁,北方宿,卫之分野。轸,南方宿,楚之分野。是月昏旦时,皆中于南方。

(3) 说在《孟冬》。

(4) 黄钟,阳律也。竹管音与黄钟和也。阳气聚于下,阴气盛于上,万物萌聚于黄泉之下,故曰“黄钟”也。

(5) 立冬后三十日大雪节,故冰益壮。地始坼,冻裂也。

(6) 鹖 ,山鸟,阳物也。是月阴盛,故不鸣也。虎乃阳中之阴也,阴气盛,以类发也。

【校】“鹖 ”,《月令》古本作“曷旦”,今本作“鹖旦”,《淮南》作“ ”。

(7) 太庙,中央室也。

(8) 说在《孟冬》。

(9) 有司,于《周礼》为司徒,掌建邦之土地,主地图与民人之教,故命之也。

(10) 房,所以闭藏也。

【校】“且泄”,古本《月令》同,今本作“沮泄”,《释文》不为“沮”作音,注、疏亦无解,然则“沮”字非也。《音律》篇亦作“阳气且泄”。

(11) 发泄阴气,故蛰伏者死,民疾以丧亡也。

(12) 阴气在上,民人空闲,无所事作,故命之曰畅月也。

是月也,命阉尹,申宫令,审门闾,谨房室,必重闭。 (1) 省妇事,毋得淫,虽有贵戚近习,无有不禁。 (2) 乃命大酋,秫稻必齐,曲糵必时, (3) 湛 必洁,水泉必香, (4) 陶器必良,火齐必得,兼用六物,大酋监之,无有差忒。 (5) 天子乃命有司祈祀四海、大川、名原、渊泽、井泉。 (6)

(1) 阉,宫官。尹,正也。于《周礼》为宫人,掌王之六寝,故命之。申宫令,审门闾,谨房室,必重闭,皆所以助阴气也。

【校】“门闾”,蔡邕《月令说》作“门闱”,云:“阉尹者,内官也,主宫室出入宫门。宫中之门曰闱,阉尹之职也。闾,里门,非阉尹所主,知当作‘闱’。”见《月令问答》。

(2) 淫则禁之,尊卑一者也。

(3) 大酋,主酒官也。酋酝米曲,使之化熟,故谓之酋。于《周礼》为酒正,“掌酒之政令,以式法度授酒材,辨五齐之名”。秫与稻必得其齐,曲与糵必得其时,则酒善也。

【校】注“酋酝米曲”及“故谓之酋”,两“酋”字旧本皆作“酒”,讹。又“曲与糵必得其时”,旧无“与”字。案上云“秫与稻”,则此亦当相配,且与下注数六物相合也。又旧本叠“得其时”三字,案亦衍文,今去之。

(4) 湛,渍也。 ,炊也。香,美也。炊必清洁,水泉善则酒美也。湛,读瀋釜之瀋。 ,读炽火之炽也。

【校】瀋釜未详,陆德明音子廉反,异于高读。

(5) 陶器,瓦器也。六物:秫、稻、曲、糵、水、火也。大酋监之,皆得其齐,故无有差忒也。

(6) 皆有功于人,故祈祀之也。

是月也,农有不收藏积聚者,牛马畜兽有放佚者,取之不诘。 (1) 山林薮泽, (2) 有能取疏食田猎禽兽者,野虞教导之。 (3) 其有侵夺者,罪之不赦。 (4)

(1) 诘,诛也。

(2) 无水曰薮,有水曰泽。

(3) 草实曰疏食。野虞,掌山泽之官也,故教导之也。

(4) 必罚之也。

是月也,日短至, (1) 阴阳争,诸生荡。 (2) 君子斋戒,处必弇, (3) 身欲宁,去声色,禁嗜欲,安形性, (4) 事欲静,以待阴阳之所定。 (5) 芸始生,荔挺出,蚯蚓结,麋角解,水泉动。 (6) 日短至,则伐林木,取竹箭。 (7)

(1) 冬至之日,昼漏水上刻四十五,夜水上刻五十五,故曰“日短至”。在牵牛一度也。

(2) 阴气在上,微阳动升,故曰“争”也。诸蛰伏当生者皆动摇也。

【校】案:郑注《月令》云:“荡谓物动将萌牙也。”

(3) 句。

(4) 弇,深邃也。宁,静也。声,五声也。色,五色也。屏去之,崇寂静也。阴阳方争,嗜欲咸禁绝之,所以安形性也。

【校】处必弇,以其所居言之。今《月令》作“处必掩身”,盖与仲夏文相涉而更误矣。

(5) 定犹成也。

(6) 芸,蒿菜名也。荔,马荔。挺,生出也。蚯蚓,虫也。结,纡也。麋角解堕,水泉涌动,皆应微阳气也。

【校】郑注《月令》云“荔挺,马 也”,与此异。

(7) 是月也,竹木调牣,又斧斤入山林之时也,故伐取之也。

【校】案:《周礼·地官》“山虞,仲冬斩阳木,仲夏斩阴木”,郑注云“坚濡调”,此注“调”,意正同。又“牣”与“韧”、“刃”、“忍”古皆通用,有取柔韧者,此则取其坚韧也。汪本乃改作“调均”,非是。

是月也,可以罢官之无事者,去器之无用者,涂阙庭门闾, (1) 筑囹圄,此所以助天地之闭藏也。

(1) 阙,门阙也,于《周礼》为象魏。门闾皆涂塞,使坚牢也。

仲冬行夏令,则其国乃旱,气雾冥冥,雷乃发声。 (1) 行秋令,则天时雨汁,瓜瓠不成,国有大兵。 (2) 行春令,则虫螟为败,水泉减竭,民多疾疠。 (3)

(1) 夏火炎上,故其国旱也。清浊相干,气雾冥冥也。夏气发泄,故雷动声也。

【校】“气雾”,《月令》作“氛雾”,此疑讹。

(2) 秋,金,水之母也。冬节白露,故雨汁也。金用事以干水,故瓜瓠不成,有大兵来伐之也。

(3) 春,木气。木生虫,故虫螟为败。食谷心曰螟。阳气炕燥,故水泉减竭也。水木相干,气不和,故民多疾疠也。

【校】《月令》“减”作“咸”,古通用。《左传》“咸黜不端”,《正义》云“诸本或作‘减’”。又“不为末减”,王肃注《家语》云“《左传》作‘咸’”。梁仲子云:“《群经音辨》咸有胡斩切,一音消也。《史记索隐》《司马相如传》‘上减五,下登三’,韦昭说作‘咸’。”又“疾疠”,《月令》作“疥疠”。





至忠


二曰:

至忠逆于耳,倒于心, (1) 非贤主其孰能听之? (2) 故贤主之所说,不肖主之所诛也。 (3) 人主无不恶暴劫者,而日致之,恶之何益? (4) 今有树于此,而欲其美也, (5) 人时灌之,则恶之, (6) 而日伐其根,则必无活树矣。夫恶闻忠言,乃自伐之精者也。 (7)

(1) 倒亦逆也。

(2) 听,受也。

(3) 贤主说忠言也,不肖主反之。《春秋传》曰:“忠为令德,非其人则不可,况不令之尤者乎?”故被不肖主之所诛也。

(4) 日致为暴劫之政也。《孟子》曰“恶湿而居下”,故曰“恶之何益”也。

(5) 美,成也。

(6) 恶其灌之。

(7) 精犹甚。甚于自伐其根者也。

荆庄哀王猎于云梦, (1) 射随兕,中之。申公子培劫王而夺之。 (2) 王曰:“何其暴而不敬也?”命吏诛之。 (3) 左右大夫皆进谏曰:“子培,贤者也,又为王百倍之臣,此必有故,愿察之也。” (4) 不出三月,子培疾而死。 (5) 荆兴师,战于两棠,大胜晋, (6) 归而赏有功者。申公子培之弟进请赏于吏曰:“人之有功也于军旅,臣兄之有功也于车下。” (7) 王曰:“何谓也?”对曰:“臣之兄犯暴不敬之名,触死亡之罪于王之侧,其愚心将以忠于君王之身,而持千岁之寿也。 (8) 臣之兄尝读故记,曰:‘杀随兕者,不出三月。’ (9) 是以臣之兄惊惧而争之, (10) 故伏其罪而死。” (11) 王令人发平府而视之,于故记果有,乃厚赏之。 (12) 申公子培,其忠也可谓穆行矣。 (13) 穆行之意,人知之不为劝,人不知不为沮, (14) 行无高乎此矣。

(1) 荆庄哀王,考烈王之子,在春秋后。云梦,楚泽也,在南郡华容也。

【校】此楚庄王也,不当有“哀”字。《说苑·立节》篇、《渚宫旧事》、《御览》八百九十皆作“楚庄王”,是穆王子也。或有作“庄襄王”者,亦误。

(2) 随兕,恶兽名也。子培,申邑宰也。楚僭称王,邑宰称公也。以杀随兕者之凶,故劫夺王,代王受殃也。

【校】“随兕”,《说苑》作“科雉”。

(3) 下陵其上谓之暴。诛之,诛子培也。

(4) 子培之贤,百倍于人,必有所为故也,故曰愿王察之也。

(5) 为代王杀随兕,故死也。

(6) 两棠,地名也。荆克晋负,故曰“大胜”。

(7) 于王车下,夺王随兕,所以代王死之,兄有是功。

【校】旧本“请赏于”下脱“吏曰人之有功也于”八字,又“军旅”下衍“曰”字,今据《御览》删补。

(8) 忠犹爱也。持犹得也。忠爱君上,犯夺随兕,是代君王受死亡之殃,使君王得千岁之寿也。

(9) 故记,古书也。比三月必死,故曰“不出”也。

(10) 惊惧王寿之不长,故与王争随兕而夺王也。

(11) 罪,殃也。

(12) 平府,府名也。赏之,赏子培之弟也。

(13) 穆,美也。

(14) 劝,进。沮,止也。

齐王疾痏, (1) 使人之宋迎文挚。文挚至,视王之疾,谓太子曰:“王之疾必可已也。 (2) 虽然,王之疾已,则必杀挚也。”太子曰:“何故?”文挚对曰:“非怒王, (3) 则疾不可治, (4) 怒王则挚必死。”太子顿首强请曰:“苟已王之疾,臣与臣之母以死争之于王。王必幸臣与臣之母, (5) 愿先生之勿患也。”文挚曰:“诺。请以死为王。” (6) 与太子期而将往不当者三, (7) 齐王固已怒矣。文挚至,不解履登床,履王衣,问王之疾,王怒而不与言。 (8) 文挚因出辞以重怒王,王叱而起,疾乃遂已。 (9) 王大怒不说,将生烹文挚。太子与王后急争之而不能得,果以鼎生烹文挚。爨之三日三夜,颜色不变。 (10) 文挚曰:“诚欲杀我,则胡不覆之,以绝阴阳之气?”王使覆之,文挚乃死。夫忠于治世易,忠于浊世难。 (11) 文挚非不知活王之疾而身获死也, (12) 为太子行难以成其义也。 (13)

(1) 齐王,湣王也,宣王之子。痏,病痏也。

【校】梁仲子云:“《论衡·道虚》篇作‘齐王病痟’。痟盖即《周礼·天官·疾医》之所谓‘痟首’也。”卢云:“案痟首常有之疾,未必难治,此或与消渴之消同。李善注《文选》张景阳《七命》又引作‘病瘠’。”

(2) 已犹愈也。

(3) 怒,读如强弩之弩。

【校】《日抄》引作“弩激之弩”。

(4) 【校】孙云:“《御览》六百四十五‘治’作‘活’,与下‘文挚非不知活王之疾’合。”

(5) 幸,哀也。

(6) 为,治也。

(7) 三不如期也。

(8) 故不解屦以履王衣,欲令王怒也。王果甚怒,不与文挚言也。

(9) 已,除愈也。

(10) 变,毁也。

(11) 贤君赏忠臣,故曰“易”也。乱主杀之,故曰“难”也。

(12) 获,得也。

(13) 为太子故,行其所难也。死之以成太子孝敬之义也。

【校】此事姑妄听之而已。





忠廉


三曰:

士议之不可辱者大之也, (1) 大之则尊于富贵也,利不足以虞其意矣。 (2) 虽名为诸侯,实有万乘,不足以挺其心矣。 (3) 诚辱则无为乐生。 (4) 若此人也,有势则必不自私矣,处官则必不为污矣,将众则必不挠北矣。 (5) 忠臣亦然。苟便于主,利于国,无敢辞违,杀身出生以徇之。 (6) 国有士若此,则可谓有人矣。若此人者固难得, (7) 其患虽得之有不智。 (8)

(1) 议,平也。平之不可得污辱者,士之大者也。

(2) 虞犹回也。

(3) 挺犹动也。

(4) 言诚可欲得辱,则无用生为,故曰“无为乐生”也。

【校】注“欲”字疑衍。

(5) 北,走也。

(6) 出犹去。去生必死也。徇犹卫也。

【校】注“卫也”疑“从也”之讹,见下注。

(7) 言得之难。

(8) 其患者,当其难也,虽得践其难,践其难必死,故曰“有不智”也。

【校】若此士者,得之固难,幸而得之矣,又患在于人主不能知之,所谓以众人遇之也。注殊失本意。“有”与“又”同,智读曰知,《墨子》书皆如是。

吴王欲杀王子庆忌而莫之能杀, (1) 吴王患之。要离曰:“臣能之。”吴王曰:“汝恶能乎? (2) 吾尝以六马逐之江上矣,而不能及;射之矢,左右满把,而不能中。今汝拔剑则不能举臂,上车则不能登轼,汝恶能?”要离曰:“士患不勇耳,奚患于不能?王诚能助,臣请必能。”吴王曰:“诺。”明旦,加要离罪焉,挚执妻子,焚之而扬其灰。 (3) 要离走,往见王子庆忌于卫。 (4) 王子庆忌喜曰:“吴王之无道也,子之所见也,诸侯之所知也。今子得免而去之,亦善矣。”要离与王子庆忌居有间,谓王子庆忌曰:“吴之无道也愈甚,请与王子往夺之国。”王子庆忌曰:“善。”乃与要离俱涉于江。 (5) 中江,拔剑以刺王子庆忌。王子庆忌捽之,投之于江,浮则又取而投之。 (6) 如此者三,其卒曰:“汝,天下之国士也,幸汝以成而名。” (7) 要离得不死,归于吴。吴王大说,请与分国。要离曰:“不可。臣请必死!”吴王止之。要离曰:“夫杀妻子焚之而扬其灰,以便事也,臣以为不仁。 (8) 夫为故主杀新主,臣以为不义。 (9) 夫捽而浮乎江,三入三出,特王子庆忌为之赐而不杀耳, (10) 臣已为辱矣。夫不仁不义,又且已辱,不可以生。”吴王不能止,果伏剑而死。 (11) 要离可谓不为赏动矣。故临大利而不易其义,可谓廉矣。廉,故不以贵富而忘其辱。 (12)

(1) 吴王,阖庐光,篡庶父僚而即其位。庆忌者,僚之子也,故欲杀之。庆忌有力捷疾,而人皆畏之,无能杀之者。

(2) 恶,安也。

(3) 吴王伪加要离罪,烧其妻子,扬其灰。

【校】孙云:“李善注《文选·邹阳狱中上书》作‘执其妻子,燔而扬其灰’。”

(4) 【校】《左氏哀廿年传》云“庆忌适楚”,此与《吴越春秋》皆云在卫。

(5) 涉,渡也。

(6) 【校】孙云:“李善注《文选》郭景纯《江赋》‘捽之’作‘捽而’,‘浮则’作‘浮出’。”

(7) 幸,活。而,汝。

(8) 便犹成也。

(9) 【校】此文讹,案《吴越春秋》“为新君而杀故君之子,非义也”。

(10) 特犹直也。

(11) 果,终也。

(12) 不忘其妻子烧死之辱,以取吴国之贵富也。

卫懿公有臣曰弘演,有所于使。 (1) 翟人攻卫,其民曰:“君之所予位禄者,鹤也;所贵富者,宫人也。君使宫人与鹤战,余焉能战?” (2) 遂溃而去。翟人至,及懿公于荣泽, (3) 杀之,尽食其肉,独舍其肝。弘演至,报使于肝,毕,呼天而啼,尽哀而止,曰:“臣请为襮。”因自杀先出其腹实,内懿公之肝。 (4) 桓公闻之曰:“卫之亡也,以为无道也。今有臣若此,不可不存。”于是复立卫于楚丘。弘演可谓忠矣,杀身出生以徇其君; (5) 非徒徇其君也,又令卫之宗庙复立,祭祀不绝,可谓有功矣。

(1) 懿公,卫惠公之子赤也。演,读如胤子之胤。

(2) 《鲁闵二年传》曰:“狄人伐卫。卫懿公好鹤,鹤有乘轩者。将战,国人受甲者皆曰:‘使鹤,鹤有禄位,余焉能战?’”此之谓也。

(3) 【校】《左传》、《韩诗外传》七并作“荧泽”,当从之。

(4) 襮,表也。纳公之肝于其腹中,故曰“臣请为襮”者也。

(5) 出,去也。去生就死,以徇从其君。





当务


四曰:

辨而不当论,信而不当理,勇而不当义,法而不当务,惑而乘骥也,狂而操吴干将也,大乱天下者,必此四者也。 (1) 所贵辨者,为其由所论也;所贵信者,为其遵所理也;所贵勇者,为其行义也;所贵法者,为其当务也。

(1) 四者,辨、信、勇、法也。惑而乘骥,必失其道。吴干将,利剑也,狂而操之,必杀害人。故曰“乱天下者,必此四者也”。

跖之徒问于跖曰:“盗有道乎?” (1) 跖曰:“奚啻其有道也?夫妄意关 (2) 内中藏,圣也; (3) 入先,勇也;出后,义也;知时,智也;分均,仁也。不通此五者而能成大盗者,天下无有。” (4) 备说非六王、五伯, (5) 以为尧有不慈之名, (6) 舜有不孝之行, (7) 禹有淫湎之意, (8) 汤、武有放杀之事, (9) 五伯有暴乱之谋, (10) 世皆誉之,人皆讳之,惑也。 (11) 故死而操金椎以葬,曰:“下见六王、五伯,将敲其头矣!”辨若此,不如无辨。 (12)

(1) 跖,大盗之人。徒,其弟子。

(2) 关,闭也。

(3) 以外知内,此几于圣也。

【校】案:“妄意关内”,于文已足,不当复有“中藏”字,《淮南·道应训》作“意而中藏者圣也”,疑后人以《淮南》之文旁注“关内”下后遂误入正文。

(4) 无有成大盗者。

(5) 备,具也。说,道也。非者,讥呵其阙也。六王,谓尧、舜、禹、汤、文、武也。五伯,齐桓、晋文、宋襄、楚庄、秦缪也。

(6) 不以天下与胤子丹朱,而反禅舜,故曰有“不慈之名”也。

(7) 《诗》云:“娶妻如之何?必告父母。”尧妻舜,舜遂不告而娶,故曰“有不孝之行”也。

(8) 禹甘旨酒而饮之,故曰“有淫湎之意”。

(9) 成汤放桀于南巢,周武杀殷纣于宣室,故曰“有放杀之事”。

(10) 五伯争国,骨肉相杀,以大兼小,故曰“有暴乱之谋”。

(11) 世称六王之圣,五伯之贤,而人讳其放杀暴乱之谋。《论语》曰:“爱之欲其生,恶之欲其死。既欲其生,又欲其死,惑也。”此之谓也。

【校】注引《论语》殊不切。

(12) 敲音 ,击也。辨说六王、五伯之阙,而欲见敲其头。辨如此,不若无辨也。

【校】“敲”,旧本作“穀”,注“音 ”作“音 ”,又一本作“音 ”,并讹。段云:“《说文》‘敲,击头也’,口卓切。”钱詹事云:“‘ ’不成字,当为‘ ’之 ,《说文》‘ ,从上击下也’。”孙氏说同。卢案:“《广韵》 、敲并苦角切,是其音正同也。”今俱改正。

楚有直躬者,其父窃羊而谒之上。 (1) 上执而将诛之,直躬者请代之。将诛矣,告吏曰:“父窃羊而谒之,不亦信乎?父诛而代之,不亦孝乎?信且孝而诛之,国将有不诛者乎?” (2) 荆王闻之,乃不诛也。孔子闻之曰:“异哉!直躬之为信也,一父而载取名焉。”故直躬之信,不若无信。 (3)

(1) 谒,告也。上,君也。《语》曰:“叶公告孔子曰:‘吾党有直躬者,其父攘羊而子证之。’”此之谓也。

(2) 言淫刑以逞,谁能免之,故曰“国将有不诛者乎”。

(3) 父为子隐,子为父隐,直在其中矣。信而证父,故曰“不若无信”也。

齐之好勇者,其一人居东郭,其一人居西郭,卒然相遇于涂,曰:“姑相饮乎?”觞数行, (1) 曰:“姑求肉乎?”一人曰:“子肉也,我肉也,尚胡革求肉而为? (2) 于是具染而已。” (3) 因抽刀而相啖,至死而止。勇若此,不若无勇。 (4)

(1) 觞,爵也。

(2) 革,更也。

(3) 染,豉酱也。

(4) 《传》曰:“酒以成礼,弗继以淫。”勇而相噬,无礼之甚,故曰“不若无勇”。

【校】注迂甚。

纣之同母三人,其长曰微子启,其次曰中衍,其次曰受德。受德乃纣也,甚少矣。 (1) 纣母之生微子启与中衍也,尚为妾,已而为妻而生纣。纣之父、纣之母欲置微子启以为太子,太史据法而争之曰:“有妻之子,而不可置妾之子。”纣故为后。 (2) 用法若此,不若无法。 (3)

(1) 少,小也。

(2) 置,立也。

(3) 太子所以继世树德化下也,法当以法,纣为淫虐以乱天下,故曰“不若无法”也。

【校】注“法当以法”句有脱误,其意盖谓立长建善,不当徒法也。





长见


五曰:

智所以相过,以其长见与短见也。 (1) 今之于古也,犹古之于后世也;今之于后世,亦犹今之于古也。故审知今则可知古,知古则可知后, (2) 古今前后一也。故圣人上知千岁,下知千岁也。

(1) 长,远也。短,近也。

(2) 古,昔也。后,来也。

荆文王曰:“苋譆数犯我以义,违我以礼, (1) 与处则不安,旷之而不穀得焉。 (2) 不以吾身爵之,后世有圣人,将以非不穀。” (3) 于是爵之五大夫。 (4) “申侯伯善持养吾意,吾所欲则先我为之, (5) 与处则安,旷之而不穀丧焉。 (6) 不以吾身远之,后世有圣人,将以非不穀。”于是送而行之。 (7) 申侯伯如郑,阿郑君之心,先为其所欲, (8) 三年而知郑国之政也, (9) 五月而郑人杀之。是后世之圣人使文王为善于上世也。 (10)

(1) 文王,武王之子也。犯我,使从义也。违我,使入礼也。

【校】“苋譆”,《说苑·君道》篇作“筦饶”,《新序》一作“筦苏”。

(2) 与之居,不安之也。旷察之,使我从义入礼,则不穀得不危亡焉。

【校】案:旷犹久也。

(3) 非犹罪也。

(4) 爵苋譆为五大夫也。

(5) 意,志也。先意承志,《传》所谓“从而不违”也。

(6) 与处则安者,臧武仲曰“季孙之爱我,疾疹也;孟孙之恶我,药石也。美疹不如恶石也”,故曰“而不穀丧焉”也。

【校】注“疾疹”,《左传》作“疾疢”。

(7) 《鲁僖七年传》曰:“初,申侯之出也,有宠于楚文王。文王将死,与之璧使行,曰:‘惟我知汝,汝专利而不厌,予取予求,不汝玼瑕也。后之人将求多于汝,汝必不免。我死,汝速行,毋适小国,将不汝容焉。’”此之谓也。

(8) 阿,从也。

(9) 知犹为也。

(10) 上犹前也。

晋平公铸为大钟,使工听之,皆以为调矣。 (1) 师旷曰:“不调,请更铸之。”平公曰:“工皆以为调矣。”师旷曰:“后世有知音者,将知钟之不调也,臣窃为君耻之。”至于师涓而果知钟之不调也。是师旷欲善调钟,以为后世之知音者也。

(1) 平公,悼公之子。调,和也。

吕太公望封于齐, (1) 周公旦封于鲁。 (2) 二君者甚相善也,相谓曰:“何以治国?”太公望曰:“尊贤上功。”周公旦曰:“亲亲上恩。”太公望曰:“鲁自此削矣。” (3) 周公旦曰:“鲁虽削,有齐者亦必非吕氏也。”其后,齐日以大,至于霸,二十四世而田成子有齐国。 (4) 鲁公以削,至于觐存, (5) 三十四世而亡。 (6)

(1) 太公望,炎帝之后。四岳佐禹治水有功,锡姓为姜,氏曰有吕,故曰吕望。遭纣之乱,闻西伯善养老者,遂奔于周,钓于渭滨。文王出田而见之,曰:“吾望公之久矣。”乃载与俱归,号为太公望,使为太师。文王薨,佐武王伐纣。成王封之于齐,故《传》曰“齐,大岳之胤”。

【校】注“吾望公之久矣”,《史记·齐世家》作“吾太公望子久矣”。《宋书·符瑞志》“太公望本名吕尚。文王至磻溪之水,尚钓于涯,王下趋拜曰:‘望公七年,乃今见光景于斯。’尚立变名,答曰‘望钓得玉璜’”云云,盖本《尚书纬·帝命验》之文。梁仲子云:“注盖引《左氏庄廿二年传》‘姜,太岳之后也’,而偶涉隐十一年之文。”

(2) 周公旦,文王之子,武王之弟也。武王崩,成王幼少,代摄政七年,致太平,成王封之于鲁也。

(3) 亲亲上恩,恩多则威武不行,威武不行,故削弱也。

(4) 尊贤敬德,故能霸也。上功则臣权重,故能夺君国也。田成子恒杀简公,适二十四世也。

(5) 觐,裁也。

(6) 自鲁公伯禽至顷公雠为楚考烈王所灭,适三十四世也。

吴起治西河之外,王错谮之于魏武侯, (1) 武侯使人召之。吴起至于岸门, (2) 止车而望西河, (3) 泣数行而下。其仆谓吴起曰:“窃观公之意,视释天下若释 。 (4) 今去西河而泣,何也?”吴起抿泣而应之曰:“子不识。 (5) 君知我而使我毕能,西河可以王。 (6) 今君听谗人之议而不知我, (7) 西河之为秦取不久矣,魏从此削矣。” (8) 吴起果去魏入楚。有间,西河毕入秦,秦日益大。 (9) 此吴起之所先见而泣也。

(1) 吴起,卫人,为魏将,善用兵,故能治西河之外,谓北边也。武侯,文侯之子。

(2) 岸门,邑名。

【校】案:《史记·魏世家》正义引《括地志》云:“岸门在许州长社县西北十八里。”

(3) 【校】后《观表》篇“止车而”下有“休”字。

(4) 释,弃。

(5) 识,知也。

【校】“抿”与“抆”同。《观表》篇作“雪”,注“拭也”。

(6) 能,力也。尽力为之,可以致君于王也。

(7) 谗人,王错也。

(8) 秦将取之,不复久也。魏失西河,故从此削弱也。

(9) 毕由尽也。

魏公叔座疾,惠王往问之, (1) 曰:“公叔之病甚矣! (2) 将奈社稷何?”公叔对曰:“臣之御庶子鞅,愿王以国听之也。 (3) 为不能听, (4) 勿使出境。” (5) 王不应,出而谓左右曰:“岂不悲哉? (6) 以公叔之贤,而今谓寡人必以国听鞅,悖也夫!”公叔死,公孙鞅西游秦,秦孝公听之,秦果用强,魏果用弱。非公叔座之悖也,魏王则悖也。夫悖者之患,固以不悖为悖。 (7)

(1) 惠王,武侯之子。

【校】“座”,旧作“痤”,与《魏策》同。据《御览》四百四十四、又六百三十二两引皆作“座”,与《史记·商君传》合,今从之。

(2) 【校】旧本作“公叔之疾嗟疾甚矣”,案《御览》两引皆作“公叔之病甚矣”,今据改正。

(3) 御庶子,爵也。鞅,卫之公孙也,故曰公孙鞅,或曰卫鞅。言其智计足以相社稷,能使用而从也。

(4) 【校】“为”,《御览》作“若”。

(5) 言不能用鞅者必杀之,无令他国得用之也,故曰“勿使出境”。

(6) 出,王视公叔疾而出也。

(7) 悖者不自知为悖,故谓不悖者为悖。





第十二卷 季冬纪



季冬


一曰:

季冬之月,日在婺女, (1) 昏娄中,旦氐中。 (2) 其日壬癸,其帝颛顼,其神玄冥。其虫介,其音羽,律中大吕, (3) 其数六。其味咸,其臭朽,其祀行,祭先肾。雁北乡,鹊始巢, (4) 雉雊鸡乳。 (5) 天子居玄堂右个, (6) 乘玄骆,驾铁骊,载玄旂,衣黑衣,服玄玉,食黍与彘,其器宏以弇。命有司大傩,旁磔,出土牛,以送寒气。 (7) 征鸟厉疾。乃毕行山川之祀,及帝之大臣、天地之神祇。 (8)

(1) 季冬,夏之十二月。婺女,北方宿,越之分野。是月,日躔此宿也。

【校】此书“婺”,旧并从“務”,案《说文》从“敄”,今并改正。

(2) 娄,西方宿,鲁之分野。氐,东方宿,韩之分野。是月昏旦时,皆中于南方。

【校】案:《淮南·天文训》氐属宋。

(3) 大吕,阴律也。竹管音与大吕和也。万物萌生动于黄泉,未能达见。吕,旅也。所以旅阴即阳,助其成功,故曰“大吕”也。

【校】注“所以旅阴即阳”,旧本“旅”下有“去”字,衍,今删去。

(4) 雁在彭蠡之泽,是月皆北乡,将来至北漠也。鹊,阳鸟,顺阳而动,是月始为巢也。

(5) 《诗》云:“雉之朝雊,尚求其雌。”乳,卵也。

【校】旧本作“乳雉雊”,误,今案注当与《月令》文同,今改正。

(6) 玄堂,北向堂。右个,东头室也。

(7) 大傩,逐尽阴气为阳导也。今人腊岁前一日,击鼓驱疫,谓之逐除是也。《周礼》“方相氏掌蒙熊皮,黄金四目,玄衣朱裳,执戈扬楯,率百隶而时傩,以索室驱疫鬼”,此之谓也。旁磔犬羊于四方以攘,其毕冬之气也。出土牛,令之乡县,得立春节,出劝耕土牛于东门外是也。

【校】注“其毕冬之气也”,“其”字衍。又“令之乡县”,疑是“今之郡县”。案《续汉·礼仪志》亦于季冬出土牛,此云立春节,说又异也。

(8) 征犹飞也。厉,高也。言是月群鸟飞行高且疾也。帝之大臣,功施于民,若禹、稷之属也。天曰神,地曰祇。是月岁终报功,载祀典诸神毕祀之也。

【校】《月令》无“行”字、“地”字。

是月也,命渔师始渔,天子亲往, (1) 乃尝鱼,先荐寝庙。冰方盛,水泽复, (2) 命取冰,冰已入。 (3) 令告民,出五种。 (4) 命司农,计耦耕事, (5) 修耒耜,具田器。命乐师大合吹而罢。 (6) 乃命四监收秩薪柴,以供寝庙及百祀之薪燎。 (7)

(1) 渔,读如《论语》之语。是月也将捕鱼,故命其长也。天子自行观之。

(2) 复亦盛也。“复”或作“複”,冻重絫也。

【校】《月令》作“水泽腹坚”。旧本于此下又有一“坚”字,乃后人以《月令》之文益之,今删去。

(3) 入凌室也。《诗》云:“二之日凿冰冲冲,三之日纳于凌阴。”此之谓也。

(4) 出之于窌,简择之也。

(5) 计,会也。耦,合也。

【校】《月令》作“命农”,无“司”字。

(6) 《周礼·籥章》:“仲春,昼击土鼓,吹《邠诗》以逆暑;仲秋,夜逆寒,亦如之。”举春、秋,省文也,则冬、夏可知。

(7) 四监者,周制,天子畿方千里之内分为百县,县有四郡,郡有一大夫监之,故命四监使收掌薪柴也。燎者,积聚柴薪,置璧与牲于上而燎之,升其烟气,故曰“以供寝庙及百祀之薪燎”也。

【校】“寝庙”,《月令》作“郊庙”。案注所云燔柴之礼,是郊也。下文“寝庙”始注云“祖庙”,则此处正文亦必本与《月令》同。

是月也,日穷于次,月穷于纪,星回于天, (1) 数将几终,岁将更始。 (2) 专于农民,无有所使。 (3) 天子乃与卿大夫饬国典,论时令,以待来岁之宜。 (4) 乃命太史,次诸侯之列,赋之牺牲, (5) 以供皇天上帝社稷之享。 (6) 乃命同姓之国,供寝庙之刍豢。 (7) 令宰历卿大夫至于庶民土田之数,而赋之牺牲,以供山林名川之祀。 (8) 凡在天下九州之民者,无不咸献其力, (9) 以供皇天上帝、社稷寝庙、山林名川之祀。

(1) 次,宿也。是月,日周于牵牛,故曰“日穷于次”也。月遇日相合为纪。月终纪,光尽而复生曰朔,故曰“月穷于纪”。日有常行,行于中道,五星随之,故曰“星回于天”也。一说:十二次穷于牵牛,故曰“穷于次”也。纪,道也。月穷于故宿,故曰“穷于纪”。星回于天,谓二十八宿更见于南方,是月回于牵牛,故曰“星回于天”也。

(2) 夏以十三月为正。夏数得天,言天时者皆从夏正也,故于是月十二月之数近终,岁将更始于正月也。

(3) 农事将起,独于农民无所役使也。

(4) 饬,读曰敕。敕正国法,论时令所宜者而行之。

【校】《月令》“与公卿大夫共饬国典”,多“公”字、“共”字。

(5) 次,列也。诸侯异姓者,太史乃次其列位、国之大小,赋敛其牺牲也。

(6) 皇天上帝,五帝也。社,后土之神,谓句龙也。稷,田官之神,谓列山氏子柱与周弃也。享,祀也。

(7) 寝庙,祖庙也。亲同姓,故使供之也。牛羊曰刍,犬豕曰豢。

(8) 宰历,于《周礼》为太宰,掌建邦之六典八法,以御其众,故命之也。

【校】“令”,《月令》作“命”。《正义》云:“宰,小宰。”郑注云:“历犹次也。”此注以“宰历”连文,似误,或“历”字衍。“掌”字旧本脱,今补。

(9) 咸,皆也。献,致也。

行之是令,此谓一终,三旬二日。 (1) 季冬行秋令,则白露早降,介虫为妖,四邻入保。 (2) 行春令,则胎夭多伤,国多固疾,命之曰逆。 (3) 行夏令,则水潦败国,时雪不降,冰冻消释。 (4)

(1) 行之是令,行是之令也。终,一岁十二月终也。三旬二日者,十日一旬也,二十日为二旬,后一旬在新月,故曰“三旬二日”。

(2) 金气白,故白露早降,介甲之虫为妖灾也。金为兵革,故四境之民入城郭以自保守也。

【校】“四邻”,《月令》作“四鄙”。

(3) 季冬大寒,而行春温仁之令,气不和调,故胎养夭伤。国多逆气之由,故命曰逆。

(4) 火气炎阳,又多淋雨,故水潦败国也。时雪当降而不降,冰冻不当消释而消释,火气温,干时之征也。





士节


二曰:

士之为人,当理不避其难, (1) 临患忘利, (2) 遗生行义, (3) 视死如归。 (4) 有如此者,国君不得而友,天子不得而臣。 (5) 大者定天下,其次定一国,必由如此人者也。 (6) 故人主之欲大立功名者,不可不务求此人也。 (7) 贤主劳于求人,而佚于治事。 (8)

(1) 理,义也。杀身成义,何难之避也?

(2) 道而用之。

(3) 惟义所在,不必生也,故曰“遗生”也。

(4) 易也。

(5) 以其义高任大,一国之君不能得友,天子不能得臣也。尧不能屈许由,周不能移伯夷,汉高不能致四皓,此之类也。

(6) 定天下,舜、禹、周弃是也。定一国,蘧伯玉、段干木是也。

(7) 务,勉也。

(8) 得贤而仕之,故佚于治事也。

齐有北郭骚者,结罘网,捆蒲苇,织萉屦, (1) 以养其母。犹不足, (2) 踵门见晏子曰:“愿乞所以养母。”晏子之仆谓晏子曰:“此齐国之贤者也。其义不臣乎天子,不友乎诸侯,于利不苟取,于害不苟免。 (3) 今乞所以养母,是说夫子之义也,必与之。”晏子使人分仓粟、分府金而遗之, (4) 辞金而受粟。有间,晏子见疑于齐君, (5) 出奔,过北郭骚之门而辞。 (6) 北郭骚沐浴而出见晏子曰:“夫子将焉适?” (7) 晏子曰:“见疑于齐君,将出奔。” (8) 北郭子曰:“夫子勉之矣。”晏子上车,太息而叹曰:“婴之亡岂不宜哉?亦不知士甚矣。”晏子行。 (9) 北郭子召其友而告之曰:“说晏子之义,而尝乞所以养母焉。 (10) 吾闻之曰:‘养及亲者,身伉其难。’ (11) 今晏子见疑,吾将以身死白之。” (12) 著衣冠,令其友操剑奉笥而从,造于君庭,求复者曰:“晏子,天下之贤者也,去则齐国必侵矣。必见国之侵也,不若先死。请以头托白晏子也。”因谓其友曰:“盛吾头于笥中,奉以托。”退而自刎也。其友因奉以托。其友谓观者曰:“北郭子为国故死,吾将为北郭子死也。”又退而自刎。齐君闻之,大骇,乘驲而自追晏子,及之国郊, (13) 请而反之。晏子不得已而反,闻北郭骚之以死白己也,曰:“婴之亡岂不宜哉?亦愈不知士甚矣。” (14)

(1) 【校】旧本作“屦履”,校云:“一作‘葩履’。”今据《尊师》篇定作“萉屦”。

(2) 犹,尚也。

(3) 于不义之利,不苟且而取也。当义能死,故不苟免。

(4) 【校】次“分”字衍,《说苑·复恩》篇无。

(5) 有间,无几间也。

(6) 辞者,别也。

(7) 适,之也。

(8) 奔,走也。

(9) 行,去也。

(10) 【校】“尝”,旧本作“当”,讹,今从《说苑》改正;“焉”,彼作“者”。

(11) 伉,当。

(12) 白,明也。

(13) 驲,传车也。郊,境也。

【校】“驲”,各本多作“驿”,李本作“驲”。案文十六年《左氏传》“楚子乘驲”,杜注“驲,传车也”,与此合,今从之。

(14) 晏子自谓施北郭骚不得其人为不知士也,又不知北郭骚能为其杀身以明己,故曰“婴之亡岂不宜哉?亦愈不知士甚矣”,自责深也。

【校】旧本正文“婴之亡”上有“晏”字,衍,今据注删去。





介立 (1)


(1) 【校】一作“立意”。

三曰:

以贵富有人易,以贫贱有人难。今晋文公出亡, (1) 周流天下,穷矣,贱矣, (2) 而介子推不去,有以有之也;反国有万乘,而介子推去之,无以有之也。能其难, (3) 不能其易, (4) 此文公之所以不王也。 (5) 晋文公反国, (6) 介子推不肯受赏,自为赋诗曰:“有龙于飞,周遍天下。五蛇从之,为之丞辅。 (7) 龙反其乡,得其处所。四蛇从之,得其露雨。 (8) 一蛇羞之,桥死于中野。”悬书公门,而伏于山下。 (9) 文公闻之,曰:“嘻!此必介子推也。”避舍变服,令士庶人曰:“有能得介子推者,爵上卿,田百万。” (10) 或遇之山中,负釜盖簦, (11) 问焉曰:“请问介子推安在?”应之曰:“夫介子推苟不欲见而欲隐,吾独焉知之?”遂背而行,终身不见。人心之不同,岂不甚哉?今世之逐利者,早朝晏退,焦唇干嗌,日夜思之,犹未之能得;今得之而务疾逃之,介子推之离俗远矣。

(1) 文公名重耳,晋献公之太子申生异母弟也。遭丽姬之乱,太子申生见杀,重耳避难奔翟十二年,自翟经于诸国也。

(2) 【校】旧校云:“‘穷’一作‘贫’。”

(3) 能以贫贱有人也。

(4) 不能以富贵有人也。

(5) 力能霸,德不能王也。

(6) 【校】旧校云:“一作‘反入’。”

(7) 丞,佐也。辅,相也。龙,君也,以喻文公。五蛇,以喻赵衰、狐偃、贾他、魏犨、介子推也。

(8) 露雨,膏泽。

(9) 【校】案:《传》载介子推之言曰“身将隐,焉用文”,安有自为诗而悬于公门之事?《说苑·复恩》篇以为从者怜之,乃悬书宫门,说尚可通。歌辞与此及《史记·晋世家》、《新序·节士》篇所载各不同。梁仲子云:“桥死疑是槁死。《御览》九百二十九无‘桥’字。”

(10) 百万亩也。

(11) 【校】旧本“簦”误从“艸”,又注“音登”二字,亦与高注不似。

东方有士焉曰爰旌目, (1) 将有适也,而饿于道。狐父之盗曰丘,见而下壶餐以 之。爰旌目三 之而后能视,曰:“子何为者也?”曰:“我狐父之人丘也。”爰旌目曰:“嘻!汝非盗邪?胡为而食我?吾义不食子之食也。”两手据地而吐之,不出,喀喀然遂伏地而死。 (2) 郑人之下 也, (3) 庄蹻之暴郢也, (4) 秦人之围长平也, (5) 韩、荆、赵此三国者之将帅贵人皆多骄矣,其士卒众庶皆多壮矣, (6) 因相暴以相杀,脆弱者拜请以避死, (7) 其卒递而相食,不辨其义,冀幸以得活。如爰旌目已食而不死矣,恶其义而不肯不死。今此相为谋,岂不远哉?

(1) 【校】梁仲子云:“《列子·说符》篇亦作‘爰旌目’。《后汉书·张衡传》作‘旌瞀’,注云‘一作爰精目’,并引《列子》亦作‘精目’。又《新序·节士》篇作‘族目’,讹。”

(2) 昔者齐饥,黔敖为食于路。有人戢其履,瞢瞢而来。黔敖呼之曰:“嗟!来食。”扬其目而应之曰:“吾惟不食嗟来之食,以至于此。”黔敖随而谢之。遂去,不食而死。君子以为其嗟也可去,其谢也可食。亓介相似,旌目其类也。

【校】“瞢瞢而来”,《礼记·檀弓下》作“贸贸然来”。

(3) ,邑名也,义则未闻。

【校】吴志伊《字汇补》云:“ 音未闻,一本作‘ ’。”梁仲子云:“《说文》‘婚’,籀文作‘ ’,略相似。《古音附录》以革旁作者,云‘古昏字’,未详。”卢云:“韩哀侯灭郑而徙都之,改号曰郑。此昏疑即《汉志》陈留郡之东昏县,正郑地。郑人下昏,或即说韩灭郑一事。观下云‘韩、荆、赵’,更可见郑人之即韩矣。”

(4) 庄蹻,楚成王之大盗。郢,楚都。

【校】梁伯子云:“《商子·弱民》篇、《荀子·议兵》篇、《韩诗外传》四、《补史记礼书》并有‘庄蹻起而楚分’之语,皆不言在楚何时。《韩非·喻老》篇载楚庄王欲伐越,杜子说曰‘庄蹻为盗于境内’,以为在庄王时。而高氏以为楚成王时,则又在前,未知何据。若《史》、《汉》则以蹻为庄王苗裔,在楚威王之世,而杜氏《通典·边防三》、马氏《通考·南蛮二》辨其误,以范史谓在顷襄王时为定。独《困学纪闻·考史》据《韩非》、《汉书》以将军庄蹻与盗名氏相同,是二人,此未敢信。”卢云:“案《后汉书·西南夷传》‘楚顷襄王时,遣将军庄豪伐夜郎,因留王滇池’,杜氏言即庄蹻。《华阳国志·南中志》云‘楚威王遣将军庄蹻伐夜郎,克之,会秦夺楚黔中地,无路得反,遂留王滇池’,此本非楚之境内地。今此言‘暴郢’,《韩非》言‘为盗于境内’,《荀子》言‘庄蹻起,楚分为三四’,皆与言将军事不合。《荀子》以唐蔑之死与蹻并言,案秦杀唐眛,‘眛’即‘蔑’,在楚怀王二十八年,则蹻当威、怀时。亦可见此注或本作‘威’,因形近而误‘成’,未可知也。”

(5) 秦使白起围赵括军于长平,坑其四十万众。

(6) 【校】卢云:“壮,伤也。”

(7) 避犹免也。





诚廉


四曰:

石可破也,而不可夺坚; (1) 丹可磨也,而不可夺赤。 (2) 坚与赤,性之有也。 (3) 性也者,所受于天也,非择取而为之也。豪士之自好者,其不可漫以污也,亦犹此也。 (4)

(1) 性坚。

(2) 【校】旧校云:“‘磨’一作‘靡’,注亦同。”今案:不见所为注,岂脱漏欤?

(3) 【校】各本多脱“也”字,唯朱本有。

(4) 倍百人为豪。

【校】旧校云:“‘豪士’一作‘人豪’。”

昔周之将兴也,有士二人,处于孤竹,曰伯夷、叔齐。 (1) 二人相谓曰:“吾闻西方有偏伯焉,似将有道者,今吾奚为处乎此哉?”二子西行如周,至于岐阳,则文王已殁矣。武王即位,观周德,则王使叔旦就胶鬲于次四内, (2) 而与之盟曰:“加富三等,就官一列。”为三书同辞,血之以牲,埋一于四内,皆以一归。又使保召公就微子开于共头之下, (3) 而与之盟曰:“世为长侯,守殷常祀,相奉桑林,宜私孟诸。” (4) 为三书同辞,血之以牲,埋一于共头之下,皆以一归。伯夷、叔齐闻之,相视而笑曰:“嘻,异乎哉!此非吾所谓道也。昔者神农氏之有天下也,时祀尽敬而不祈福也。 (5) 其于人也,忠信尽治而无求焉。 (6) 乐正与为正,乐治与为治,不以人之坏自成也, (7) 不以人之庳自高也。今周见殷之僻乱也,而遽为之正与治, (8) 上谋而行货,阻丘而保威也。 (9) 割牲而盟以为信,因四内与共头以明行,扬梦以说众, (10) 杀伐以要利,以此绍殷,是以乱易暴也。 (11) 吾闻古之士,遭乎治世,不避其任; (12) 遭乎乱世,不为苟在。今天下暗,周德衰矣。与其并乎周以漫吾身也, (13) 不若避之以洁吾行。”二子北行,至首阳之下而饿焉。

(1) 孤竹国在辽西,殷诸侯国也。

(2) 四内,地名。

(3) 共头,水名。

【校】案:共头即共首,山名,在汉之河内共县。

(4) 相犹使也。使奉桑林之乐。孟诸,泽名也,为私邑也。

(5) 时,四时。祈,求也。

(6) 无所求于民也。

(7) 【校】“坏”,宋邦乂本作“壤”。壤亦伤也。

(8) 遽,疾也。

(9) 行货,谓与胶鬲盟加富三等也。阻,依。保,持。

【校】“阻丘”疑是“阻兵”。杜注《左传》:“阻,恃也。”保亦当训恃。

(10) 宣扬武王灭殷之梦,以喜众民。

【校】案:事见《周书·程寤》篇,今已亡。《御览》五百三十三载其略云:“文王去商在程。正月既生魄,太姒梦见商之庭产棘,小子发取周庭之梓,树于阙间,化为松柏棫柞。寤,惊以告文王。文王曰:‘召发于明堂拜吉梦,受商之大命于皇天上帝。’”此其事也。

(11) 绍,续。

(12) 任,职也。力所能。

(13) 漫,污。

人之情,莫不有重,莫不有轻。 (1) 有所重则欲全之,有所轻则以养所重。 (2) 伯夷、叔齐此二士者,皆出身弃生以立其意,轻重先定也。 (3)

(1) 莫不有重,于天下也。莫不有轻,义重身也。

(2) 养所重,不污于武王,为以全其忠也。

【校】注“忠”疑当作“重”。

(3) 伯夷、叔齐让国而去,轻身重名,故曰“轻重先定”。





不侵


五曰:

天下轻于身,而士以身为人。 (1) 以身为人者,如此其重也, (2) 而人不知以奚道相得。 (3) 贤主必自知士,故士尽力竭智,直言交争,而不辞其患。 (4) 豫让、公孙宏 (5) 是矣。当是时也,智伯、孟尝君知之矣。 (6) 世之人主,得地百里则喜,四境皆贺, (7) 得士则不喜,不知相贺,不通乎轻重也。 (8) 汤、武,千乘也,而士皆归之。 (9) 桀、纣,天子也,而士皆去之。 (10) 孔、墨,布衣之士也, (11) 万乘之主、千乘之君不能与之争士也。 (12) 自此观之,尊贵富大不足以来士矣, (13) 必自知之然后可。 (14)

(1) 轻于身,重于义也。以身为人者,为人杀身。

(2) 《淮南记》曰:“左手据天下之图,右手刎其喉,愚夫不为也。”今以义为人杀身,故曰“如此其重也”。

(3) 奚,何也。不知以何道得人,乃令之为己死也。

(4) 士为知己者死,故尽力竭智,何患之辞也?

(5) 【校】避改。

(6) 智伯知豫让,故为之报仇,言士为知己者死也。孟尝君知公孙宏,故为之不受折于秦也。

(7) 举国皆贺,国中喜可知也。

(8) 不但不知相贺也,乃不知贤也,故曰“不通乎轻重也”。

(9) 汤,殷受命之王,名天乙,商主癸之子也。武王,周文王之子,名发。

(10) 桀,夏失天下之王,帝皋之孙,帝发之子。纣,殷失天下之王,文丁之孙,帝乙之子也。

【校】注“文丁”,旧本作“太丁”,讹,今据《竹书纪年》改正。

(11) 孔子、墨翟。

(12) 万乘,天子也。千乘,诸侯也。士不归之而归孔、墨,故曰“不能与之争士也”。

(13) 来犹致也。

(14) 可者,可致也。

豫让之友谓豫让曰:“子之行何其惑也?子尝事范氏、中行氏,诸侯尽灭之,而子不为报,至于智氏,而子必为之报,何故?”豫让曰:“我将告子其故。 (1) 范氏、中行氏,我寒而不我衣,我饥而不我食,而时使我与千人共其养,是众人畜我也。夫众人畜我者,我亦众人事之。至于智氏则不然,出则乘我以车,入则足我以养,众人广朝,而必加礼于吾所, (2) 是国士畜我也。 (3) 夫国士畜我者,我亦国士事之。”豫让,国士也,而犹以人之于己也为念, (4) 又况于中人乎?

(1) 告,语也。故,事也。

(2) 句。

(3) 【校】“是”,旧本多作“谓”,则当以“所谓”连读。今从李本作“是”,义长。

(4) 于犹厚也。

孟尝君为从, (1) 公孙宏谓孟尝君曰:“君不若使人西观秦王。意者秦王帝王之主也,君恐不得为臣,何暇从以难之? (2) 意者秦王不肖主也,君从以难之未晚也。” (3) 孟尝君曰:“善。愿因请公往矣。” (4) 公孙宏敬诺,以车十乘之秦。秦昭王闻之,而欲丑之以辞,以观公孙宏。 (5) 公孙宏见昭王,昭王曰:“薛之地小大几何?”公孙宏对曰:“百里。”昭王笑曰:“寡人之国,地数千里,犹未敢以有难也。今孟尝君之地方百里,而因欲以难寡人犹可乎?”公孙宏对曰:“孟尝君好士,大王不好士。”昭王曰:“孟尝君之好士何如?”公孙宏对曰:“义不臣乎天子,不友乎诸侯,得意则不惭为人君,不得意则不屑为人臣, (6) 如此者三人。 (7) 能治可为管、商之师, (8) 说义听行,其能致主霸王, (9) 如此者五人。 (10) 万乘之严主辱其使者,退而自刎也,必以其血污其衣,有如臣者七人。” (11) 昭王笑而谢焉,曰:“客胡为若此?寡人善孟尝君,欲客之必谨谕寡人之意也。” (12) 公孙宏敬诺。公孙宏可谓不侵矣。昭王,大王也; (13) 孟尝君,千乘也。立千乘之义而不可凌, (14) 可谓士矣。 (15)

(1) 关东曰从。

(2) 言不能成从以难秦也。

(3) 晚,后。

(4) 往,行。

(5) 昭王,秦惠王之子,武王之弟也。“丑”或作“耻”。耻,辱也。观公孙宏云何也。

(6) 【校】旧本“惭”上脱“不”字,又“屑”讹作“肖”。案《战国·齐策》云:“得志不惭为人主,不得志不肯为人臣。”今据补正。

(7) 有此者三人也。

(8) 管仲、商鞅。

(9) 【校】《策》作“能致其主霸王”,句顺。

(10) 有此者五人也。

(11) 臣,公孙宏自谓也,故言“有如臣者七人”也。

【校】“七人”,《策》作“十人”。注殊赘。

(12) 谕,明。

(13) 【校】《策》作“大国也”。

(14) 凌,侮。

(15) 孔子曰:“使于四方,不辱君命,可谓士矣。”此之谓也。

【校】《策》作“可谓足使矣”。





序意 (1)


(1) 【校】旧云:“一作‘廉孝’。”案“廉孝”二字与此无涉,必尚有脱文。

维秦八年,岁在涒滩, (1) 秋,甲子朔,朔之日,良人请问十二纪。 (2) 文信侯曰: (3) “尝得学黄帝之所以诲颛顼矣,爰有大圜在上,大矩在下, (4) 汝能法之,为民父母。盖闻古之清世, (5) 是法天地。凡十二纪者,所以纪治乱存亡也,所以知寿夭吉凶也。上揆之天,下验之地,中审之人,若此则是非、可不可无所遁矣。天曰顺,顺维生;地曰固,固维宁;人曰信,信维听。三者咸当,无为而行。行也者,行其理也。行数,循其理,平其私。夫私视使目盲,私听使耳聋,私虑使心狂,三者皆私设精则智无由公。 (6) 智不公,则福日衰,灾日隆, (7) 以日倪而西望知之。” (8)

(1) 八年,秦始皇即位八年也。岁在申名涒滩。涒,大也。滩,循也。万物皆大循其情 也。涒滩,夸人短舌不能言为涒滩也。

【校】案:今谓始皇即位之年岁在乙卯,钱氏塘以超辰之法推之,知在癸丑,再加七年是庚申,是年又当超辰,则为辛酉。而此犹云涒滩者,失数超辰之岁耳。超辰亦谓之跳辰,《周礼》、《冯相》、《保章》注、疏中详言之。自东汉以后,不明此理,故武帝太初元年,班固谓之丙子者,后人却谓之丁丑矣。

(2) 良人,君子也。

(3) 吕不韦封洛阳,号文信侯。

(4) 圜,天也。矩,方,地也。

(5) 清,平。

(6) 公,正也。

(7) 隆,盛。

(8) 日中而盛,跌而衰,人之盛衰于此。西望,日暮也,故曰倪而西望之也。

【校】“倪”与“睨”同,李本作“兒”。注“跌”与“昳”同,《周礼·大司徒》“日东则景夕多风”,郑司农云:“景夕,谓日跌景乃中。”《史记·天官书》“日昳”,《汉书·天文志》作“日跌”。谢云:“此句文与上不属,又下一段亦不当在此篇。”

赵襄子游于囿中,至于梁,马却不肯进。青荓为参乘, (1) 襄子曰:“进视梁下,类有人。” (2) 青荓进视梁下。豫让却寝,佯为死人,叱青荓曰:“去!长者吾且有事。” (3) 青荓曰:“少而与子友,子且为大事, (4) 而我言之,是失相与友之道;子将贼吾君,而我不言之,是失为人臣之道。如我者惟死为可。” (5) 乃退而自杀。青荓非乐死也,重失人臣之节,恶废交友之道也。青荓、豫让可谓之友也。

(1) 【校】旧校云:“一作‘青蓱’。”案李善注《文选》陈琳《答东阿王笺》引作“青蓱”。梁仲子云:“《汉书人表》作‘青荓子’,《水经·汾水注》作‘清洴’,今新刻亦改作‘青荓’矣。”

(2) 类,象也。

(3) 言将杀襄子。

【校】《选》注无“吾”字,是。长者,让自谓也。

(4) 【校】《选》注作“子今日为大事”。

(5) 适可得死也。





第十三卷 有始览



有始


一曰:

天地有始,天微以成,地塞以形。 (1) 天地合和,生之大经也。 (2) 以寒暑日月昼夜知之, (3) 以殊形殊能异宜说之。 (4) 夫物合而成,离而生。知合知成,知离知生,则天地平矣。 (5) 平也者,皆当察其情,处其形。 (6)

(1) 始,初也。天,阳也。虚而能施,故微以生万物。地,阴也,实而能受,故塞以成形兆也。

(2) 经犹道也。

(3) 知犹别也。

【校】旧本“以寒”下衍“以”字,今去之。

(4) 形能各有所施,故说译之也。

(5) 合,和也。平,成也。

(6) 【校】旧校云:“一作‘平也者,皆反其情,变其形也’。”

天有九野,地有九州,上 [1] 有九山,山有九塞,泽有九薮, (1) 风有八等,水有六川。 (2)

(1) 险阻曰塞。有水曰泽。无水曰薮。

(2) 【校】《淮南·地形训》作“水有六品”,后“六川”作“六水”。

何谓九野?中央曰钧天,其星角、亢、氐。 (1) 东方曰苍天,其星房、心、尾。 (2) 东北曰变天,其星箕、斗、牵牛。 (3) 北方曰玄天,其星婺女、虚、危、营室。 (4) 西北曰幽天,其星东壁、奎、娄。 (5) 西方曰颢天,其星胃、昴、毕。 (6) 西南曰朱天,其星觜巂、参、东井。 (7) 南方曰炎天,其星舆鬼、柳、七星。 (8) 东南曰阳天,其星张、翼、轸。 (9)

(1) 钧,平也。为四方主,故曰“钧天”。角、亢、氐,东方宿,韩、郑分野。

(2) 东方二月建卯,木之中也。木色青,故曰“苍天”。房、心、尾,东方宿。房、心,宋分野。尾、箕,燕分野。

(3) 东北,水之季,阴气所尽,阳气所始,万物向生,故曰“变天”。斗、牛,北方宿。尾、箕,一名析木之津,燕之分野。斗、牛,吴、越分野。

(4) 北方十一月建子,水之中也。水色黑,故曰“玄天”也。婺女,亦越之分野。虚、危,齐分野。营室,卫分野。

(5) 西北,金之季也,将即太阴,故曰“幽天”。东壁,北方宿,一名豕韦,卫之分野。奎、娄,西方宿,一名降、娄,鲁之分野。

(6) 西方八月建酉,金之中也。金色白,故曰“颢天”。昴、毕,西方宿,一名大梁,赵之分野。

【校】注“昴毕”上当有“胃,鲁之分野”五字。

(7) 西南,火之季也,为少阳,故曰“朱天”。巂、参,西方宿,一名实沈,晋之分野。东井,南方宿,一名鹑首,秦之分野。

(8) 南方五月建午,火之中也。火曰炎上,故曰“炎天”。舆鬼,南方宿,秦之分野。柳、七星,南方宿,一名鹑火,周之分野。

(9) 东南,木之季也,将即太阳,纯乾用事,故曰“阳天”。张、翼、轸,南方宿。张,周之分野。翼、轸,一名鹑尾,楚之分野。

【校】注“张、翼、轸,南方宿”,旧脱“轸”字,又“南”讹作“北”,今改正。

何谓九州?河、汉之间为豫州,周也。 (1) 两河之间为冀州,晋也。 (2) 河、济之间为兖州,卫也。 (3) 东方为青州,齐也。泗上为徐州,鲁也。 (4) 东南为扬州,越也。南方为荆州,楚也。西方为雍州,秦也。北方为幽州,燕也。

(1) 河在北,汉在南,故曰“之间”。

(2) 东至清河,西至西河。

(3) 河出其北,济经其南。

(4) 泗,水名也。

何谓九山?会稽,太山, (1) 王屋,首山,太华, (2) 岐山,太行,羊肠,孟门。 (3)

(1) 会稽山在今会稽郡。太山在今太山郡,是为东岳也。

(2) 王屋在河东垣县东北,济水所出也。首山在蒲阪之南,河曲之中,伯夷所隐。太华在弘农华阴县,是为西岳也。

(3) 岐山在右扶风美阳县西北,周家所邑。太行在河内野王县北。羊肠,其山盘纡譬如羊肠,在太原晋阳县北。

【校】注末七字旧本缺,据李善注《文选》魏武帝《苦寒行》所引补。又诱注《淮南·地形训》云:“孟门,太行之限也。”此不注,疑文脱。

何谓九塞?大汾,冥阨,荆阮,方城, (1) 殽,井陉,令疵,句注,居庸。 (2)

(1) 大汾,处未闻。冥阨、荆阮、方城皆在楚。鲁定四年,吴伐楚,楚左司马请塞直辕、冥阨以击吴人者也。

【校】“大汾”,《淮南》作“太汾”,注云“在晋”,此何以云未闻?“冥阨”,《淮南》作“渑阨”,彼注云:“今宏农渑池是也。”皆与此不同。岂彼乃许慎注欤?又“塞”字旧本脱,今案《传》文增。

(2) 殽在弘农渑池县西。井陉在常山井陉县,通太原关。令疵,处则未闻。句注在雁门。居庸在上谷沮阳之东,通军都关也。

【校】《淮南》“殽”下有“阪”字。“令疵”,旧本讹作“疵处”,据注是“令疵”。《淮南注》云“令疵在辽西”,则即是令支,乃齐桓所刜者。又“军都关”,旧讹作“居都关”,《淮南注》作“运都关”。钱云:“运乃军之讹,军都亦上谷县,在居庸之东。”今皆改正。

何谓九薮? (1) 吴之具区, (2) 楚之云梦, (3) 秦之阳华, (4) 晋之大陆, (5) 梁之圃田, (6) 宋之孟诸, (7) 齐之海隅, (8) 赵之巨鹿, (9) 燕之大昭。 (10)

(1) 薮,泽也。有水曰泽,无水曰薮。

(2) 具区在吴、越之间。

【校】《淮南》“吴”作“越”。

(3) 云梦在南郡华容。

(4) 阳华在凤翔,或曰在华阴西。

【校】《尔雅》作“阳陓”,《淮南》作“阳纡”,注云:“阳纡在冯翊池阳,一名具圃。”

(5) 魏献子所畋,犹楚之华容也。

【校】注“畋”,旧讹作“居”,据《左氏定元年传》改正。

(6) 圃田在今河南中牟。

【校】“梁”,《淮南》作“郑”。

(7) 孟诸在梁国睢阳之东南。

【校】《淮南注》作“东北”,郭注《尔雅》亦同。此讹。

(8) 隅犹崖也。

(9) 广阿泽也。

【校】郭璞注《尔雅》“晋有大陆”云:“今巨鹿北广阿泽是也。”《尔雅》本无“赵之巨鹿”,而有“鲁之大野,周之焦护”为十薮。

(10) 大昭,今太原郡是也。

【校】“大昭”,《淮南》作“昭余”,《尔雅》作“昭余祁”。

何谓八风?东北曰炎风, (1) 东方曰滔风, (2) 东南曰熏风, (3) 南方曰巨风, (4) 西南曰凄风, (5) 西方曰 风, (6) 西北曰厉风, (7) 北方曰寒风。 (8)

(1) 炎风,艮气所生。一曰融风。

(2) 震气所生。一曰明庶风。

【校】《淮南》作“条风”。

(3) 巽气所生。一曰清明风。

【校】旧校云:“‘熏风’或作‘景风’。”案《淮南》作“景风”。

(4) 离气所生。一曰凯风。《诗》曰“凯风自南”。

【校】孙云:“李善注《文选》木玄虚《海赋》、王子渊《洞箫赋》、潘安仁《河阳县作诗》引俱作‘凯风’。”

(5) 坤气所生。一曰凉风。

【校】《淮南》作“凉风”。

(6) 兑气所生。一曰阊阖风。

(7) 干气所生。一曰不周风。

【校】《淮南》作“丽风”。

(8) 坎气所生。一曰广莫风。

何谓六川?河水,赤水,辽水,黑水,江水,淮水。 (1)

(1) 河出昆仑东北陬。赤水出其东南陬。辽水出砥石山,自塞北东流,直至辽东之西南入海。黑水出昆仑西北陬。江水出岷山,在蜀西徼外。淮水出桐柏山,在南阳平氏县也。

【校】注“自塞北东流”,《水经注》“北”作“外”,又下作“直辽东”,无“至”字。

凡四海之内,东西二万八千里,南北二万六千里, (1) 水道八千里,受水者亦八千里,通谷六,名川六百,陆注三千, (2) 小水万数。 (3)

(1) 子午为经,卯酉为纬。四海之内,纬长经短。

(2) 【校】《淮南》“注”作“径”。

(3) 陆无水,水盛内乃注之也。

凡四极之内,东西五亿有九万七千里,南北亦五亿有九万七千里。 (1) 极星与天俱游,而天极不移。 (2) 冬至日行远道,周行四极,命曰玄明。 (3) 夏至日行近道,乃参于上。当枢之下无昼夜。 (4) 白民之南,建木之下,日中无影,呼而无响,盖天地之中也。 (5) 天地万物,一人之身也,此之谓大同。 (6) 众耳目鼻口也,众五谷寒暑也,此之谓众异,则万物备也。天斟万物, (7) 圣人览焉,以观其类。 (8) 解在乎天地之所以形, (9) 雷电之所以生, (10) 阴阳材物之精, (11) 人民禽兽之所安平。 (12)

(1) 海东西长,南北短。极内等。

(2) 极星,辰星也。《语》曰“譬如北辰,居其所而众星拱之”,故曰“不移”。

(3) 远道,外道也,故曰“周行四极”。玄明,大明也。

(4) 近道,内道也。乃参倍于上下曰高也。当极之下分明不置曜统一也,故曰“无昼夜”。

【校】注“下曰”疑是“夏日”,“不置”疑是“不冥”。

(5) 白民之国,在海外极内。建木在广都南方,众帝所从上下也,复在白民之南。建木状如牛,引之有皮,黄叶若罗也。日正中将下,日直人下,皆无影;大相叫呼,又无音响人声;故谓盖天地中也。

【校】注“引”,旧作“豕”字,讹。案《海内南经》云:“有木,其状如牛,引之有皮,若缨黄蛇,其叶如罗,其实如栾,其木若 ,其名曰建木。在窫窳西。”

(6) 以亓人身喻天地万物。《易》曰“近取诸身,远取诸物”,故曰“大同”也。

(7) 【校】旧校云:“‘斟’一作‘堪’。”注亦同。案“堪”或是“斟”字,会集也,盛也。

(8) 天斟输万物,圣人总览以知人也。

(9) 天地之初始成形也。

(10) 震气为雷,激气为电,始生时也。

(11) 阴阳皆由天地。阴阳例万物也。

(12) 人民禽兽,动作万物,皆由天地阴阳以生,各得其所乐,故曰“之所安平”也。





应同 (1)


(1) 【校】旧作“名类”,乃“召类”之讹,然与卷二十篇目复。旧校云“一名‘应同’”,今即以“应同”题篇。

二曰:

凡帝王者之将兴也,天必先见祥乎下民。 (1) 黄帝之时,天先见大螾大蝼。 (2) 黄帝曰土气胜。土气胜,故其色尚黄,其事则土。 (3) 及禹之时,天先见草木秋冬不杀。禹曰木气胜。木气胜,故其色尚青,其事则木。 (4) 及汤之时,天先见金刃生于水。汤曰金气胜。金气胜,故其色尚白,其事则金。 (5) 及文王之时,天先见火赤乌衔丹书集于周社。文王曰火气胜。火气胜,故其色尚赤,其事则火。 (6) 代火者必将水,天且先见水气胜。水气胜,故其色尚黑,其事则水。 (7) 水气至而不知,数备,将徙于土。 (8) 天为者时,而不助农于下。 (9) 类固相召,气同则合,声比则应。 (10) 鼓宫而宫动,鼓角而角动。 (11) 平地注水,水流湿;均薪施火,火就燥。 (12) 山云草莽,水云鱼鳞, (13) 旱云烟火,雨云水波,无不皆类其所生以示人。 (14) 故以龙致雨,以形逐影。师之所处,必生棘楚。 (15) 祸福之所自来,众人以为命,安知其所。 (16)

(1) 祥,征应也。

(2) 蝼,蝼蛄。螾,蚯蚓。皆土物。

【校】注“蝼,蝼蛄”,旧本作“蛄蝼”,今补正。

(3) 则,法也。法土色尚黄。

(4) 法木色青。

(5) 法金色白。

(6) 法火色赤。

(7) 法水色黑。

(8) 【校】旧校云:“‘徙’一作‘见’。”

(9) 助犹成也。

(10) 应,和。

(11) 鼓,击也。击大宫而小宫应,击大角而小角和,言类相感也。

(12) 水流湿者先濡,火就燥者先然。

(13) 【校】旧本误作“角 ”,吴志伊《字汇补》载之,徐仲山谓“鱼鳞”之讹。今案唐、宋人类部所引皆作“鱼鳞”,《淮南·览冥训》亦同,今改正。

(14) 【校】《御览》八“皆”作“比”。

(15) 军师训罚,以杀伐为首,棘楚以戮人,喜生战地,故生其处也。

【校】案:《老子》曰:“师之所处,荆棘生焉。”此偏不为孝文王讳,何也?注亦不明。“训罚”疑“讨罚”。“戮人”旧作“战人”,讹,今改正。

(16) 自,从也。凡人以为天命,不知其所由也。

夫覆巢毁卵,则凤凰不至; (1) 刳兽食胎,则麒麟不来;干泽涸渔,则龟龙不往。 (2) 物之从同,不可为记。子不遮乎亲,臣不遮乎君。 (3) 君同则来,异则去。故君虽尊,以白为黑,臣不能听; (4) 父虽亲,以黑为白,子不能从。黄帝曰:“芒芒昧昧, (5) 因天之威, (6) 与元同气。” (7) 故曰同气贤于同义,同义贤于同力,同力贤于同居,同居贤于同名。帝者同气, (8) 王者同义, (9) 霸者同力, (10) 勤者同居则薄矣, (11) 亡者同名则觕矣。 (12) 其智弥觕者,其所同弥粗;其智弥精者,其所同弥精。 (13) 故凡用意不可不精。夫精,五帝三王之所以成也。成齐类同皆有合,故尧为善而众善至,桀为非而众非来。 (14) 《商箴》云:“天降灾布祥,并有其职。”以言祸福人或召之也。 (15) 故国乱非独乱也,又必召寇。独乱未必亡也,召寇则无以存矣。 (16)

(1) 【校】案:“覆巢”旧误倒,今乙正。

(2) 【校】疑当作“不住”,此有韵。

(3) 遮,后遏也。

(4) 听,从。

(5) 【校】旧本皆不重。案:《文子·符言》、《上仁》篇,《淮南·缪称》、《泰族训》,及《御览》七十七引皆重,此注亦然,今据改正。

(6) 【校】旧校云:“一作‘道’。”

(7) 芒芒昧昧,广大之貌。天之威无不敬也,非同气不协。

(8) 同元气也。

(9) 同仁义也。

(10) 同武力也。

【校】《文子》、《淮南》并作“同功”。

(11) 同居于世。

(12) 同名,不仁不义。粗,恶也。

(13) 精,微妙也。

(14) 【校】旧校云:“一本作‘桀为恶而众恶来’。”

(15) 职,主也。召,致也。

(16) 存,在也。

凡兵之用也,用于利,用于义。攻乱则脆,脆则攻者利;攻乱则义,义则攻者荣。荣且利,中主犹且为之,况于贤主乎?故割地宝器,卑辞屈服,不足以止攻,惟治为足。 (1) 治则为利者不攻矣,为名者不伐矣。凡人之攻伐也,非为利则因为名也。名实不得,国虽强大者,曷为攻矣?解在乎史墨来而辍不袭卫,赵简子可谓知动静矣! (2)

(1) 足止人攻。

(2) 【校】事见《召类》篇,“史墨”作“史默”。





去尤


三曰:

世之听者,多有所尤。多有所尤,则听必悖矣。所以尤者多故, (1) 其要必因人所喜,与因人所恶。东面望者不见西墙,南乡视者不睹北方,意有所在也。

(1) 句。

人有亡 者,意其邻之子。视其行步,窃 也;颜色,窃 也;言语,窃 也;动作态度,无为而不窃 也。抇其谷而得其 , (1) 他日,复见其邻之子,动作态度,无似窃 者。其邻之子非变也,己则变矣。变也者无他,有所尤也。

(1) 【校】“抇”,旧讹作“相”,今从《列子·说符》篇改正。

邾之故法,为甲裳以帛。 (1) 公息忌 (2) 谓邾君曰:“不若以组。凡甲之所以为固者,以满窍也。今窍满矣,而任力者半耳。且组则不然,窍满则尽任力矣。”邾君以为然,曰:“将何所以得组也?”公息忌对曰:“上用之则民为之矣。”邾君曰:“善。”下令,令官为甲必以组。公息忌知说之行也,因令其家皆为组。人有伤之者曰:“公息忌之所以欲用组者,其家多为组也。”邾君不说,于是复下令,令官为甲无以组。 (3) 此邾君之有所尤也。为甲以组而便,公息忌虽多为组,何伤也?以组不便,公息忌虽无组, (4) 亦何益也?为组与不为组,不足以累公息忌之说,用组之心,不可不察也。

(1) 以帛缀甲。

(2) 【校】旧校云:“一作‘忘’。”

(3) 以,用。

(4) 【校】孙云:“《御览》三百五十六作‘虽无为组’。”

鲁有恶者, (1) 其父出而见商咄,反而告其邻曰:“商咄不若吾子矣。”且其子至恶也,商咄至美也。彼以至美不如至恶,尤乎爱也。故知美之恶,知恶之美,然后能知美恶矣。《庄子》曰:“以瓦殶者翔,以钩殶者战,以黄金殶者殆。 (2) 其祥一也,而有所殆者,必外有所重者也。外有所重者泄,盖内掘。” (3) 鲁人可谓外有重矣。解在乎齐人之欲得金也,及秦墨者之相妒也, (4) 皆有所乎尤也。老聃则得之矣,若植木而立乎独,必不合于俗,则何可扩矣。

(1) 恶,丑。

(2) 【校】《庄子·达生》篇:“以瓦注者巧,以钩注者惮,以黄金注者殙。”《列子·黄帝》篇“注”并作“抠”,“殙”作“涽”,文义各小异。此“殶”字无考。《淮南·说林训》又作“ ”。

(3) 【校】《淮南》作“是故所重者在外,则内为之掘”,注云:“掘,律气不安详。”《列子》作“凡重外者拙内”,语更简而明。

(4) 【校】两事皆见《去宥篇》。





听言


四曰:

听言不可不察,不察则善不善不分。善不善不分,乱莫大焉。三代分善不善,故王。今天下弥衰,圣王之道废绝, (1) 世主多盛其欢乐, (2) 大其钟鼓,侈其台榭苑囿,以夺人财;轻用民死,以行其忿;老弱冻馁,夭膌壮狡,汔尽穷屈, (3) 加以死虏;攻无罪之国以索地,诛不辜之民以求利;而欲宗庙之安也,社稷之不危也,不亦难乎?今人曰:“某氏多货,其室培湿,守狗死,其势可穴也。”则必非之矣。曰:“某国饥,其城郭庳,其守具寡,可袭而篡之。”则不非之。乃不知类矣。 (4) 《周书》曰:“往者不可及,来者不可待,贤明其世,谓之天子。”故当今之世,有能分善不善者,其王不难矣。

(1) 【校】旧校云:“‘圣王’一作‘圣人’。”

(2) 【校】旧校云:“‘欢’一作‘观’。”

(3) 【校】狡与佼同,说见《仲夏纪》。

(4) 【校】与《墨子·非攻》篇意同。

善不善本于义,不于爱。爱利之为道大矣。夫流于海者,行之旬月,见似人者而喜矣。及其期年也,见其所尝见物于中国者而喜矣。夫去人滋久,而思人滋深欤?乱世之民,其去圣王亦久矣。其愿见之,日夜无间。故贤王秀士之欲忧黔首者,不可不务也。 (1)

(1) 务,勉也。

功先名,事先功,言先事。不知事,恶能听言?不知情,恶能当言? (1) 其与人谷言也,其有辩乎,其无辩乎? (2) 造父始习于大豆,蜂门始习于甘蝇, (3) 御大豆,射甘蝇,而不徙人以为性者也。 (4) 不徙之,所以致远追急也,所以除害禁暴也。 (5) 凡人亦必有所习其心,然后能听说。不习其心,习之于学问。不学而能听说者,古今无有也。解在乎白圭之非惠子也, (6) 公孙龙之说燕昭王以偃兵及应空洛之遇也, (7) 孔穿之议公孙龙,翟翦之难惠子之法。此四士者之议,皆多故矣,不可不独论。 (8)

(1) 安能使其言当合于事乎?

(2) 谷言,善言。辩,别也。

(3) 习,学也。大豆、甘蝇,盖御射人姓名。

【校】梁仲子云:“《列子·汤问》篇‘造父之师曰泰豆氏’,此大豆当读泰。”案:蜂门即逢蒙,《荀子·王霸》篇、《史记·龟策传》皆同。《汉书人表》作“逢门子”,《庄子》作“蓬蒙”,《法言·学行》篇作“逢蒙”,音薄红切,《盐铁论·能言》篇作“逢须”,唯今本《孟子》乃作“逄蒙”。

(4) 专学不徙,以得深术。

(5) 专学大豆、甘蝇之法而不徙之,故御射得。御可以致远追急,射而发中可以除害禁暴也。

(6) 白圭,周人也。惠子,惠施,仕魏。

【校】见《不屈》篇。

(7) 【校】说偃兵见《应言》篇。梁仲子云:“空洛之遇事,见后《淫辞》篇,作‘空雄’,地名,岂亦‘空雒’之误欤?”

(8) 公孙龙、孔穿、翟翦皆辩人。

【校】二事亦见《淫辞》篇。





谨听


五曰:

昔者禹一沐而三捉发,一食而三起, (1) 以礼有道之士,通乎己之不足也。 (2) 通乎己之不足,则不与物争矣。 (3) 愉易平静以待之,使夫自得之; (4) 因然而然之,使夫自言之。 (5) 亡国之主反此,乃自贤而少人。少人则说者持容而不极, (6) 听者自多而不得, (7) 虽有天下,何益焉?是乃冥之昭,乱之定,毁之成,危之宁。 (8) 故殷、周以亡,比干以死,悖而不足以举。 (9) 故人主之性, (10) 莫过乎所疑,而过于其所不疑; (11) 不过乎所不知,而过于其所以知。 (12) 故虽不疑,虽已知,必察之以法,揆之以量,验之以数。 (13) 若此则是非无所失,而举措无所过矣。 (14)

(1) 【校】梁仲子云:“《淮南·氾论训》作‘一馈而十起’。”

(2) 欲以问知所不知也,故曰“通乎己之不足”。

(3) 情欲之物不争。

(4) 【校】旧校云:“‘得’一作‘以’。”

(5) 【校】旧校云:“‘言’一作‘宁’。”

(6) 极,至。

(7) 自多,自贤也。

(8) 以冥为明,以乱为定,以毁为成,以危为宁也。

(9) 殷、周以乱而亡,比干以忠而死。不当乱而乱,不可为忠而忠,故悖不可胜举。

(10) 【校】旧校云:“一作‘任’。”

(11) 所疑者,不敢行,故不过也。其所不疑者,不可而行之,故以为过。

(12) 所不知者,不敢施,故不为。所以知者,不可施而必为,故曰“过于其所以知”。

(13) 其所不疑,其所已知,俗主所专用。而贤主能以法制行之,以度量揆之,以数术验之。

(14) 其慎所不疑,审所已知,故不失过也。

夫尧恶得贤天下而试舜?舜恶得贤天下而试禹? (1) 断之于耳而已矣。耳之可以断也,反性命之情也。 (2) 今夫惑者,非知反性命之情, (3) 其次非知观于五帝三王之所以成也, (4) 则奚自知其世之不可也?奚自知其身之不逮也? (5) 太上知之,其次知其不知。 (6) 不知则问,不能则学。《周箴》曰:“夫自念斯,学德未暮。” (7) 学贤问,三代之所以昌也。 (8) 不知而自以为知,百祸之宗也。 (9)

(1) 恶,安;试,用也。何以得贤于天下能用舜、禹?

(2) 反,本。

(3) 惑,眩惑也。

(4) 成,成其治。

(5) 奚,何也。逮,及也。

(6) 生自知其上也,其次知其不知也。

(7) 暮,晚。

(8) 学贤知,昌盛。

(9) 宗,本也。《论语》曰:“不知为不知。”夫不知者而自以为知,则反于道,百祸归之,故曰“百祸之宗也”。

名不徒立,功不自成,国不虚存,必有贤者。 (1) 贤者之道,牟而难知,妙而难见。 (2) 故见贤者而不耸,则不惕于心。不惕于心,则知之不深。 (3) 不深知贤者之所言,不祥莫大焉。 (4) 主贤世治,则贤者在上; (5) 主不肖世乱 [2] ,则贤者在下。今周室既灭,而天子已绝。 (6) 乱莫大于无天子, (7) 无天子则强者胜弱,众者暴寡,以兵相残,不得休息, (8) 今之世当之矣。 (9) 故当今之世,求有道之士,则于四海之内,山谷之中,僻远幽闲之所, (10) 若此则幸于得之矣。得之,则何欲而不得?何为而不成? (11) 太公钓于滋泉,遭纣之世也,故文王得之而王。 (12) 文王,千乘也;纣,天子也。天子失之,而千乘得之,知之与不知也。 (13) 诸众齐民,不待知而使,不待礼而令。 (14) 若夫有道之士,必礼必知,然后其智能可尽。 (15) 解在乎胜书之说周公,可谓能听矣;齐桓公之见小臣稷,魏文侯之见田子方也,皆可谓能礼士矣。 (16)

(1) 惟贤者然后立名成功而存其国也。《传》曰:“不有君子,其能国乎?”此之谓也。

(2) 牟犹大也。贤者之道,磥落不凡,惟义所在,非不肖所及,故难知也。其仁爱物,本于中心,精妙幽微,亦非不肖所及,故难见也。

(3) 不深知贤者师法之也。

(4) 祥,善也。

(5) 位在上。

【校】自“主贤世治”已下,又见后《观世》篇。

(6) 周厉王无道,流于彘而灭,无天子十一年,故曰“已绝”。

【校】秦昭王五十二年西周亡,十年而始皇帝继为王,又二十六年始为皇帝。所云“天子已绝”者,在始皇未为皇帝之时。注非是。

(7) 【校】“乱”字旧本脱在上注内,今据《观世》篇改正。

(8) 【校】旧校云:“‘休’一作‘暂’。”

(9) 当其时也。

(10) 所,处也。

(11) 得贤则欲而得,为而成也。

(12) 【校】梁仲子云:“《水经·渭水上注》引作‘太公钓兹泉’。”孙云:“《御览》七十、又八百三十四并作‘兹泉’。”旧本句末“王”字脱,亦从《御览》补。

(13) 文王知太公贤,是以得之;纣不知贤,是以失之;故曰知与不知也。

(14) 齐民,凡民。非一,故言诸众。

【校】旧校云:“‘令’一作‘合’。”案《观世》篇亦作“令”,注“令犹使也”。

(15) 可尽得而用也。

(16) 能礼士,故曰得士。商纣不能礼士,故失太公以灭亡也。

【校】案:胜书说周公见《精谕》篇,齐桓、魏文二事皆见《下贤》篇。此田子方乃段干木之讹。





务本


六曰:

尝试观上古记,三王之佐,其名无不荣者,其实无不安者,功大也。 (1) 《诗》云:“有晻凄凄,兴云祁祁。雨我公田,遂及我私。” (2) 三王之佐,皆能以公及其私矣。俗主之佐,其欲名实也,与三王之佐同,而其名无不辱者,其实无不危者,无公故也。 (3) 皆患其身不贵于国也,而不患其主之不贵于天下也;皆患其家之不富也,而不患其国之不大也。此所以欲荣而愈辱,欲安而益危。 (4) 安危荣辱之本在于主,主之本在于宗庙,宗庙之本在于民,民之治乱在于有司。 (5) 《易》曰:“复自道,何其咎,吉。” (6) 以言本无异,则动卒有喜。 (7) 今处官则荒乱,临财则贪得, (8) 列近则持谏, (9) 将众则罢怯, (10) 以此厚望于主,岂不难哉! (11)

(1) 上古记,上世古书也。名者,爵位名也。实者,功实也。

(2) 《诗·小雅·大田》之三章也。晻,阴雨也。阴阳和,时雨祁祁然不暴疾也。古者井田,十一而税,公田在中,私田在外。民有礼让之心,故愿先公田而及私也。

【校】案:《颜氏家训·书证》篇辨“兴云”当作“兴雨”,以班孟坚《灵台》诗“祁祁甘雨”为证。钱詹事晓征作《汉书考异》据《韩奕》篇“祁祁如云”,谓经师传授有异,非转写有讹。又段明府若膺云:“古人言雨,止言‘降雨’‘下雨’,无有言‘兴雨’者。‘兴云祁祁,雨我公田’,犹《白华》诗之‘英英白云,露彼菅茅’,语意正相似。”案钱、段二说极是,然观注意亦似本作“兴雨”。

(3) 【校】“无公”,后《务大》篇作“无功”,公亦功也,古通用。

(4) 【校】旧校云:“‘益’一作‘愈’。”

(5) 有司,于《周礼》为太宰,掌建国之六典,以佐王治邦国,以治官府,以纪万民,此之谓也。

(6) 乾下巽上,小畜,“初九,复自道,何其咎,吉”。乾为天,天道转运,为乾初得其位。既天行周匝复始,故曰“复自道”也。复自进退,又何咎乎?动而无咎,故吉也。

(7) 乾动,反其本,终复始,无有异,故“卒有喜”也。

(8) 欲多。

【校】“临财”,各本作“临射”,今从刘本。

(9) 列,位也。持谏,不公正。

(10) 罢,劳也。怯,无勇。

(11) 厚,多。

今有人于此,修身会计则可耻, (1) 临财物资尽则为己, (2) 若此而富者,非盗则无所取。 (3) 故荣富非自至也,缘功伐也。今功伐甚薄而所望厚,诬也; (4) 无功伐而求荣富,诈也。 (5) 诈诬之道,君子不由。 (6) 人之议多曰:“上用我,则国必无患。”用己者未必是也,而莫若其身自贤。 (7) 而己犹有患,用己于国,恶得无患乎? (8) 己,所制也;释其所制而夺乎其所不制,悖, (9) 未得治国,治官可也。 (10) 若夫内事亲,外交友,必可得也。苟事亲未孝,交友未笃,是所未得,恶能善之矣?故论人无以其所未得,而用其所已得,可以知其所未得矣。 (11)

(1) 【校】旧校云:“‘可’一作‘不’。”

(2) 尽犹略也。无不充仞以为己有。

(3) 《诗》云:“不稼不穑,胡取禾三百亿兮?不狩不猎,胡瞻尔庭有县特兮。”故曰“非盗则无所取”。

(4) 以薄获厚为诬也。

(5) 以虚取之为诈。

(6) 由,用也。

(7) 有人于此,言用我者则国无患,而使用之,未必然也。使无患,莫若自修其身之贤也。

(8) 犹,尚。恶,安。

(9) 言身者己所自制也,释己而不修,故曰夺乎所不制,乃悖谬之道也。

(10) 官,小政也。推此言之,若此人者,未任为大臣,但可小政也。

(11) 以其孝得于亲,则知必忠于君也;以其所行能高仁义,知必轻身;故可以知其未得也。

古之事君者,必先服能,然后任; (1) 必反情,然后受。 (2) 主虽过与,臣不徒取。 (3) 《大雅》曰:“上帝临汝,无贰尔心。”以言忠臣之行也。 (4) 解在郑君之问被瞻之义也, (5) 薄疑应卫嗣君以无重税。此二士者,皆近知本矣。 (6)

(1) 服其能堪任也。

(2) 反情,常内省也。受,受禄也。

(3) 过,多。

(4) 《大雅·大明》之七章也。言天临命武王,伐纣必克之,不敢有疑心。喻君命臣齐一专心输力,不敢惑忠臣之行也。

(5) 见《务大论》。被瞻知齐国衰乱,桓公之薨,虫流出户,盖不听管仲临终之言,因讽郑君。

【校】案:《务大论》郑君问被瞻义不死君、不亡君,殊不如注所言。

(6) 嗣君,平侯之子也,秦贬称君。薄疑劝嗣君以王者富民,故曰“无重税”也。

【校】薄疑事见《审应览》。





谕大


七曰:

昔舜欲旗古今而不成, (1) 既足以成帝矣;禹欲帝而不成,既足以正殊俗矣; (2) 汤欲继禹而不成,既足以服四荒矣; (3) 武王欲及汤而不成,既足以王道矣;五伯欲继三王而不成,既足以为诸侯长矣;孔丘、墨翟欲行大道于世而不成,既足以成显名矣。 (4) 夫大义之不成,既有成矣已。 (5) 《夏书》曰:“天子之德广运,乃神,乃武乃文。” (6) 故务在事,事在大。 (7) 地大则有常祥、不庭、歧母、群抵、天翟、 (8) 不周,山大则有虎豹熊螇蛆, (9) 水大则有蛟龙鼋鼍鳣鲔。 (10) 《商书》曰:“五世之庙,可以观怪。 (11) 万夫之长,可以生谋。” (12) 空中之无泽陂也,井中之无大鱼也, (13) 新林之无长木也。 (14) 凡谋物之成也,必由广大众多长久,信也。

(1) 旗,覆也。

【校】“旗”当与“綦”同,乃极尽之义。旧校云:“‘旗’一作‘ ’,一作‘揭’。”梁伯子云:“观注训覆,则作‘ ’为是。‘ ’即‘冒’也。”

(2) 殊俗,异方之俗也。

(3) 四表之荒服也。

(4) 名,圣贤之名。

(5) 【校】二字当衍其一。

(6) 逸《书》也。

(7) 事,为。

(8) 常祥、不庭、群抵、歧母、天翟,皆兽名也。

【校】此虽山名,然不应独举,当亦与上文为一类。

(9) 皆兽名。不周山在翟。

【校】螇蛆未详所出,或是猿狙,亦可作“虫”旁。

(10) 鱼二千斤为蛟。鼋可作羹,《传》曰:“楚人献鼋于郑灵公,不与公子宋鼋羹。公子怒,染指于鼎,尝之而出。”鼍鱼皮可作鼓,《诗》云“鼍鼓 ”。鳣鲔皆大鱼,长丈余,《诗》云“鳣鲔发发”。

(11) 逸《书》。喻山大水大生大物。庙者鬼神之所在,五世久远,故于其所观魅物之怪异也。

(12) 长,大也。大故可以成奇谋也。

(13) 《淮南记》曰“蜂房不能容鹤卵”,此之谓也。

(14) 言未久也。

季子曰 (1) :“燕雀争善处于一室之下,子母相哺也,姁姁焉相乐也, (2) 自以为安矣。灶突决,则火上焚栋,燕雀颜色不变,是何也?乃不知祸之将及己也。为人臣免于燕雀之智者,寡矣。夫为人臣者,进其爵禄富贵,父子兄弟相与比周于一国,姁姁焉相乐也,以危其社稷, (3) 其为灶突近也,而终不知也,其与燕雀之智不异矣。故曰:‘天下大乱,无有安国;一国尽乱,无有安家;一家皆乱,无有安身。’此之谓也。故小之定也必恃大,大之安也必恃小。小大贵贱,交相为恃, (4) 然后皆得其乐。”定贱小在于贵大, (5) 解在乎薄疑说卫嗣君以王术, (6) 杜赫说周昭文君以安天下, (7) 及匡章之难惠子以王齐王也。 (8)

(1) 【校】后《务大》篇作“孔子曰”。梁仲子云:“案《孔丛子·论势》篇子顺引‘先人有言’云云,则作‘孔子’为是。”

(2) 【校】“姁姁”后作“区区”,《孔丛》作“喣喣”。

(3) 【校】后句上有“而”字,此脱。

(4) 【校】后作“赞”。

(5) 《淮南记》曰“牛马之气烝生虮虱,虮虱气蒸不能生牛马”,小不能生大,故曰“定贱小在于贵大”。

(6) 见《务大论》。

(7) 杜赫,周人,杜伯之后。昭文君,周末世分东西之后君号也。说见《务大论》。

(8) 匡章乃孟轲所谓通国称不孝者,能王齐王亦大也。

【校】此见《爱类》篇。




————————————————————

[1] 上:他本均作“土”。

[2] 乱:原本作“辞”,据乾隆本改。





第十四卷 孝行览



孝行


一曰:

凡为天下,治国家,必务本而后末。 (1) 所谓本者,非耕耘种殖之谓,务其人也。 (2) 务其人,非贫而富之,寡而众之, (3) 务其本也。务本莫贵于孝。 (4) 人主孝,则名章荣,下服听,天下誉; (5) 人臣孝,则事君忠,处官廉,临难死; (6) 士民孝,则耕芸疾,守战固,不罢北。 (7) 夫孝,三皇五帝之本务,而万事之纪也。 (8)

(1) 詹何曰:“身治而国不治者,未之有也。”故曰“必务本”。

(2) 务犹求也。

(3) 众,多也。

(4) 孝为行之本也。行于孝者,故圣人贵之。

(5) 誉,乐也。孔子曰:“昔者明王之以孝治天下也,不敢遗小国之臣,而况于公侯伯子男乎?故得万国之欢心。”

(6) 孝于亲,故能忠于君,《孝经》曰“以孝事君则忠”,此之谓也。处官廉,《孝经》曰“修身慎行,恐辱先也”,此之谓也。临难死,君父之难,视死如归,义重身轻也。

(7) 耕芸疾,用天之道,分地之利。衣食足,知荣辱,故守则坚,战必克,无退走者。

【校】孙云:“《御览》七十七‘罢’作‘败’。”

(8) 三皇:伏羲、神农、女娲也。五帝:轩辕、帝颛顼、帝喾高辛、帝尧陶唐、帝舜有虞也。纪犹贯因也。

【校】案:《初学记》十七引“纪”上有“纲”字。注“女娲”当在“神农”前。所纪五帝,文有讹脱,当云“黄帝轩辕、帝颛顼高阳”,方与下相配。“贯因”,刘本无“因”字。

夫执一术而百善至,百邪去,天下从者,其惟孝也! (1) 故论人必先以所亲,而后及所疏; (2) 必先以所重,而后及所轻。 (3) 今有人于此,行于亲重,而不简慢于轻疏,则是笃谨孝道, (4) 先王之所以治天下也。 (5) 故爱其亲,不敢恶人;敬其亲,不敢慢人。爱敬尽于事亲,光耀加于百姓, (6) 究于四海, (7) 此天子之孝也。

(1) 一术,孝术。

(2) 先本后末,先近后远。

(3) 所重,谓其亲。所轻,谓他人。

(4) 有人行孝敬于其亲,以及人之亲,故不敢简慢于轻疏者,是厚慎孝道之谓也。

(5) 先王以孝治天下。

(6) 加,施也。

(7) 究,极也。

曾子曰:“身者,父母之遗体也。行父母之遗体,敢不敬乎? (1) 居处不庄,非孝也; (2) 事君不忠,非孝也; (3) 莅官不敬,非孝也; (4) 朋友不笃,非孝也; (5) 战陈无勇,非孝也。 (6) 五行不遂,灾及乎亲,敢不敬乎?” (7)

(1) 敬,畏慎。

(2) 庄,敬。

(3) 忠,正也。

(4) 莅,临也。

(5) 笃,信也。

(6) 扬子曰:“孟轲勇于义。”勇而立义,扬名于后世,孝之终也。

(7) 遂,成。

《商书》曰:“刑三百,罪莫重于不孝。” (1)

(1) 商汤所制法也。

曾子曰:“先王之所以治天下者五:贵德、贵贵、贵老、敬长、慈幼。此五者,先王之所以定天下也。 (1) 所谓贵德,为其近于圣也; (2) 所谓贵贵,为其近于君也;所谓贵老,为其近于亲也;所谓敬长,为其近于兄也;所谓慈幼,为其近于弟也。”

(1) 定,安也。

(2) 【校】案:《礼记·祭义》“圣”作“道”。

曾子曰:“父母生之,子弗敢杀;父母置之, (1) 子弗敢废;父母全之,子弗敢阙。 (2) 故舟而不游,道而不径,能全支体,以守宗庙,可谓孝矣。” (3)

(1) 置,立。

(2) 阙犹毁。

(3) 济水载舟不游涉,行道不从邪径,为免没溺畏险之害,故曰“能全支体,以守宗庙”。

【校】注“免”字,旧本作“逸”,讹,今改正。

养有五道:修宫室,安床笫,节饮食,养体之道也; (1) 树五色,施五采,列文章,养目之道也; (2) 正六律, (3) 和五声, (4) 杂八音,养耳之道也; (5) 熟五谷,烹六畜,和煎调,养口之道也; (6) 和颜色,说言语,敬进退,养志之道也。 (7) 此五者,代进而厚用之,可谓善养矣。 (8)

(1) 节饮食,肉虽多,不使胜食气;修宫室,不上漏下湿;故曰“养体之道”也。

(2) 列,别也。青与赤谓之文。赤与白谓之章。以极目观,故曰“养目之道”也。

(3) 六律:黄钟、太蔟、姑洗、蕤宾、夷则、无射。

(4) 五声:宫、商、角、徵、羽。

(5) 八音,八卦之音。杂会之以听耳,故曰“养耳之道”。

(6) 熟五谷,烹刍豢,和快口腹,故曰“养口之道”。

(7) 和颜色,以说父母之志意,故曰“养志之道”。

(8) 代,更。更次用之,以便亲性,可谓为善养亲也。

乐正子春下堂而伤足,瘳而数月 (1) 不出,犹有忧色。门人问之曰:“夫子下堂而伤足,瘳而数月不出,犹有忧色,敢问其故?” (2) 乐正子春曰:“善乎而问之! (3) 吾闻之曾子,曾子闻之仲尼:父母全而生之,子全而归之,不亏其身,不损其形,可谓孝矣。君子无行咫步而忘之。余忘孝道,是以忧。”故曰,身者非其私有也, (4) 严亲之遗躬也。 (5)

(1) 【校】旧校云:“一作‘三月’,下同。”案:《祭义》亦作“数月”。

(2) 故,事也。

(3) 而,汝也。

(4) 私犹独。

(5) 躬,体。

民之本教曰孝, (1) 其行孝曰养。养可能也,敬为难; (2) 敬可能也,安为难; (3) 安可能也,卒为难。 (4) 父母既没,敬行其身,无遗父母恶名,可谓能终矣。仁者,仁此者也; (5) 礼者,履此者也; (6) 义者,宜此者也;信者,信此者也;强者,强此者也。乐自顺此生也, (7) 刑自逆此作也。 (8)

(1) 本,始。

(2) 行敬之难。

(3) 安宁其亲难。

(4) 卒,终。

(5) 【校】此皆《祭义》之文,旧本独少此一句,脱耳,今补之。

(6) 履,行。

(7) 【校】旧校云:“‘顺’一作‘慎’。”

(8) 能顺行,无遗父母恶名,故乐生也。逆之则刑辟作也。





本味


二曰:

求之其本,经旬必得;求之其末,劳而无功。 (1) 功名之立,由事之本也,得贤之化也。 (2) 非贤,其孰知乎事化? (3) 故曰其本在得贤。

(1) 虽久无所得。

(2) 得贤人与之共治,以立其功名,故曰“得贤之化也”。

(3) 【校】“事化”承上文之言。旧校云“‘化’一作‘民’”,本又作“名”,皆讹。

有侁氏女子采桑,得婴儿于空桑之中, (1) 献之其君。其君令烰人养之。 (2) 察其所以然, (3) 曰“其母居伊水之上,孕, (4) 梦有神告之曰:‘臼出水而东走,毋顾!’明日,视臼出水,告其邻,东走十里而顾,其邑尽为水,身因化为空桑”, (5) 故命之曰伊尹, (6) 此伊尹生空桑之故也。 (7) 长而贤。汤闻伊尹,使人请之有侁氏,有侁氏不可。伊尹亦欲归汤,汤于是请取妇为婚。有侁氏喜,以伊尹媵女。 (8) 故贤主之求有道之士,无不以也; (9) 有道之士求贤主,无不行也; (10) 相得然后乐。 (11) 不谋而亲,不约而信,相为殚智竭力,犯危行苦, (12) 志欢乐之,此功名所以大成也。固不独, (13) 士有孤而自恃,人主有奋而好独者,则名号必废熄, (14) 社稷必危殆。故黄帝立四面,尧、舜得伯阳、续耳然后成。 (15) 凡贤人之德,有以知之也。 (16)

(1) 侁,读曰莘。

(2) 烰犹庖也。

(3) 察,省。

(4) 任身为孕。

(5) 伊尹母化作空桑。

(6) 【校】以其生于伊水,故名之伊尹,非有讹也。而黄氏东发所见本作“故命之曰空桑”,以为地名,且为之辨曰:“此书第五纪云‘颛顼生自若水,实处空桑’,则前乎伊尹之未生已有空桑之地矣。”卢云:“案黄氏所据本非也。同一因地命名,不若伊尹之确。张湛注《列子·天瑞 [1] 》篇‘伊尹生于空桑’引传记与今本同,尤为明证。”

(7) 【校】旧校云:“‘生’一作‘出’。”

(8) 【校】旧本作“以伊尹为媵送女”。段云:“《说文》‘ ’字下引吕不韦曰‘有侁氏以伊尹 女’。 ,送也。则‘为’、‘送’二字明是后人所增入。”媵已是送,无烦重絫言之,今删正。

(9) 以,用也。

【校】“以也”旧作“在以”。孙云:“《御览》四百二作‘无不以也’。”又此下旧本有一“为”字,衍。并依《御览》删正。

(10) 为媵言必行。

(11) 贤主得贤臣,贤臣得贤主,故曰“相得然后乐”也。

(12) 殚、竭,皆尽也。危,难也。苦,勤也。

(13) 固,必也。

(14) 熄,灭也。

(15) 黄帝使人四面出求贤人,得之立以为佐,故曰“立四面”也。伯阳、续耳皆贤人,尧用之以成功也。

【校】“续耳”,《尸子》、《韩非子》作“续牙”,《汉书人表》作“续身”,皆隶转失之。

(16) 知其贤乃得而用之。

【校】旧校云:“‘之德’一作‘道德’。”

伯牙鼓琴,钟子期听之。方鼓琴而志在太山,钟子期曰:“善哉乎鼓琴!巍巍乎若太山。”少选之间,而志在流水, (1) 钟子期又曰:“善哉乎鼓琴!汤汤乎若流水。”钟子期死,伯牙破琴绝弦,终身不复鼓琴,以为世无足复为鼓琴者。 (2) 非独琴若此也,贤者亦然。 (3) 虽有贤者,而无礼以接之,贤奚由尽忠?犹御之不善,骥不自千里也。 (4)

(1) 少选,须臾之间也。志在流水,进而不解也。

(2) 伯,姓;牙,名,或作“雅”。钟,氏;期,名;子,皆通称。悉楚人也。少善听音,故曰为世无足为鼓琴也。

(3) 世无贤者,亦无所从受礼义法则与共治国也。

(4) 言不肖者无礼以接贤者,贤者何用尽其忠乎?若不知御者御骥,骥亦不为之从千里也。

汤得伊尹,祓之于庙, (1) 爝以爟火,衅以牺猳。 (2) 明日,设朝而见之,说汤以至味。 (3) 汤曰:“可对而为乎?” (4) 对曰:“君之国小,不足以具之,为天子然后可具。夫三群之虫, (5) 水居者腥,肉玃者臊,草食者膻。 (6) 臭恶犹美,皆有所以。 (7) 凡味之本,水最为始。五味三材, (8) 九沸九变,火为之纪。 (9) 时疾时徐,灭腥去臊除膻,必以其胜,无失其理。 (10) 调和之事,必以甘酸苦辛咸,先后多少,其齐甚微,皆有自起。 (11) 鼎中之变,精妙微纤,口弗能言,志不能喻, (12) 若射御之微,阴阳之化,四时之数。 (13) 故久而不弊,熟而不烂, (14) 甘而不哝, (15) 酸而不酷, (16) 咸而不减,辛而不烈,澹而不薄,肥而不 。 (17) 肉之美者:猩猩之唇。獾獾之炙。 (18) 隽觾之翠。 (19) 述荡之 。 (20) 旄象之约。 (21) 流沙之西,丹山之南,有凤之丸, (22) 沃民所食。 (23) 鱼之美者:洞庭之 。东海之鲕。 (24) 醴水之鱼,名曰朱鳖,六足,有珠百碧。 (25) 雚水之鱼,名曰鳐,其状若鲤而有翼, (26) 常从西海夜飞,游于东海。 (27) 菜之美者:昆仑之蘋。 (28) 寿木之华。 (29) 指姑之东, (30) 中容之国,有赤木、玄木之叶焉。 (31) 余瞀之南, (32) 南极之崖, (33) 有菜,其名曰嘉树,其色若碧。 (34) 阳华之芸。 (35) 云梦之芹。 (36) 具区之菁。 (37) 浸渊之草,名曰土英。 (38) 和之美者:阳檏之姜。招摇之桂。 (39) 越骆之菌。鳣鲔之醢。 (40) 大夏之盐。宰揭之露。其色如玉, (41) 长泽之卵。 (42) 饭之美者:玄山之禾。不周之粟。 (43) 阳山之穄。南海之秬。 (44) 水之美者:三危之露。 (45) 昆仑之井。 (46) 沮江之丘,名曰摇水。 (47) 曰山之水。高泉之山,其上有涌泉焉,冀州之原。 (48) 果之美者:沙棠之实。 (49) 常山之北,投渊之上,有百果焉,群帝所食。 (50) 箕山之东,青鸟之所,有甘栌焉。 (51) 江浦之橘。云梦之柚。 (52) 汉上石耳。所以致之, (53) 马之美者,青龙之匹,遗风之乘。 (54) 非先为天子,不可得而具。天子不可强为,必先知道。 (55) 道者止彼在己, (56) 己成而天子成, (57) 天子成则至味具。 (58) 故审近所以知远也,成己所以成人也。圣王之道要矣,岂越越多业哉!” (59)

(1) 【校】《风俗通·祀典》引此句下有“薰以萑苇”四字,《续汉书·礼仪志》中注亦同,今本脱去耳。

(2) 《周礼》“司爟掌行火之政令”。火者所以祓除其不祥,置火于桔皋,烛以照之。衅,以牲血涂之曰衅。爟,读曰权衡之权。

(3) 为汤说美味。

(4) 【校】“对”字讹,当作“得”。《御览》八百四十九作“可得为之乎”。

(5) 三群,谓水居、肉玃、草食者也。

(6) 水居者,川禽鱼鳖之属,故其臭腥也。肉玃者,玃拿肉而食之,谓鹰雕之属,故其臭臊也。草食者,食草木,谓獐鹿之属,故其臭膻也。

(7) 臭恶犹美,若蜀人之作羊腊,以臭为美,各有所用也。

(8) 五行之数,水第一,故曰水最为始。五味:咸、苦、酸、辛、甘。三才:水、木、火。

(9) 纪犹节也。品味待火然后成,故曰火为之节。

【校】旧本正文作“火之为纪”,今从《御览》乙正,与注合。

(10) 用火熟食,或炽或微,治除臊腥,胜去其臭,故曰“必以其胜”也。齐和之节,得其中适,故曰“无失其理”也。

(11) 齐,和分也。自,从也。

(12) 鼎中品味,分齐纤微,故曰不能言也。志意揆度,不能谕说。

(13) 射者望毫毛之近,而中艺于远也;御者执辔于手,调马口之和,而致万里;故曰“若射御之微”也。阴阳之化而成万物也。四时之数,春生夏长,秋收冬藏,物有异功也。

【校】注“马口”似当作“马足”。

(14) 弊,败也。烂,失饪也。《论语》云:“失饪不食。”

(15) 【校】“哝”乃“噮”字之讹,后《审时》篇“得时之黍,食之不噮而香”,《玉篇》“于县切”,又《酉阳杂俎》亦云“酒食甘而不噮”。

(16) 【校】案:《玉篇》引伊尹曰“酸而不嚛”,《酉阳杂俎》亦是“嚛”字。

(17) 言皆得其中适。

【校】“ ”,字书无考。案今人言味过厚而难入口者,有虚侯、虚交二音,岂本此欤?

(18) 猩猩,兽名也,人面狗躯而长尾。獾獾,鸟名,其形未闻。

【校】旧校云:“‘獾’一作‘获’。”今案:《南山经》云“青邱之山有鸟焉,其状如鸠,其音若呵,名曰灌灌”,注“或作‘濩濩’”,则此“獾”当作“灌”,“获”亦当作“濩”。若“貛”从“豸”,则是兽名。今注云“鸟名”,则当如《山海经》所说也。

(19) 鸟名也。翠,厥也,形则未闻也。

【校】“觾”乃“燕”字之讹,《初学记》与《文选·七命》注皆作“燕”。《选》注“隽”作“嶲”,则子规也。《礼记·内则》有“舒雁翠”、“舒凫翠”,注“尾肉也”,皆不可食者,今闽、广人以此为美。“翠”亦作“臎”。《广雅》“臎,髁 也”,《说文》作“ , 骨也”,训皆相合。《玉篇》“臎,鸟尾上肉也”。

(20) 兽名。 ,读如棬碗之碗。 者,踏也,形则未闻。

【校】《初学记》引作“迷荡”。

(21) 旄,旄牛也,在西方。象,象兽也,在南方。约,饰也。以旄牛之尾,象兽之齿,以饰物也。一曰:约,美也。旄象之肉美,贵异味也。

【校】案:此论味之美者,何忽及于饰乎?《楚辞·招魂》“土伯九约”,王逸注:“约,屈也。”九屈难解,“屈”必是“ ”之讹,《玉篇》云“短尾也”。今时牛尾、鹿尾皆为珍品,但象尾不可知耳。《说文》无“屈”,有“ ”,云“无尾也”,疑“无”字亦误衍。

(22) “丸”,古“卵”字也。流沙,沙自流行,故曰流沙,在燉煌西八百里。丹山在南方,丹泽之山也。二处之表,有凤皇之卵。

(23) 食凤卵也。沃之国在西方。

【校】见《大荒西经》。

(24) 洞庭,江水所经之泽名也。 、鲕,鱼名也。一云鱼子也。

(25) 醴水在苍梧,环九疑之山,其鱼六足,有珠如蛟皮也。

【校】《东山经》注引“澧水之鱼,名曰朱鳖,六足,有珠”。梁仲子云:“此注不解‘百碧’,疑当从下文作‘若碧’,盖青色珠也。”

(26) 雚水在西极。若,如也。翼,羽翼也。

【校】《西山经》“泰品之山,观水出焉,是多文鳐鱼”,形状与此同。

(27) 鳐从西海至东海,乘云气而飞。

(28) 昆仑,山名,在西北,其高九万八千里。蘋,大蘋,水藻也。

【校】郭璞以蘋即《西山经》之 草,其状如葵,其味如葱,食之可以已劳。

(29) 寿木,昆仑山上木也。华,实也。食其实者不死,故曰“寿木”。

(30) 【校】旧校云:“‘指’一作‘枯’。”案《齐民要术》十引作“括姑”,则“枯”亦“括”之讹。

(31) 指姑乃姑余,山名也,在东南方,《淮南记》曰“轶 鸡于姑余”是也。赤木、玄木,其叶皆可食,食之而仙也。

【校】注“ 鸡”,旧讹作“题难”,今据《淮南·览冥训》改正。

(32) 【校】旧校云:“‘瞀’一作‘督’。”

(33) 【校】旧校云:“一作‘旁’。”

(34) 余瞀,南方山名也。有嘉美之菜,故曰嘉树,食之而灵。若碧,青色。

【校】注“灵”字,旧作“虚”,今据《齐民要术》十改正。

(35) 阳华乃华阳,山名也。芸,芳菜也。在吴、越之间。

(36) 云梦,楚泽。芹生水涯。

【校】孙云:“《说文》艸部‘ ’字云‘菜之美者,云梦之 ’,徐锴云:‘此《吕氏春秋》伊尹对汤之辞,其为状未闻。’”卢云:“案《说文》有‘菦’字,云‘菜类蒿,《周礼》有菦菹’;又有‘芹’字,云‘楚葵也’;俱巨巾切。又出‘ ’字,驱喜切。今案:‘ ’亦是‘芹’。凡真、文韵中字俱与支、微、齐相通,不胜枚举。但以从‘斤’者言之,如沂、圻、旂、祈、颀、蕲等字皆可见。《祭法》‘相近于坎坛’读为‘禳祈’;《左氏传》‘公子欣时’,《公羊传》作‘喜时’;《谥法》‘治典不杀曰祈’,‘祈’亦作‘震’;则可知‘ ’之即为‘芹’无疑矣。”

(37) 具区,泽名,吴、越之间。菁,菜名。

(38) 浸渊,深渊也,处则未闻。英,言其美善。土英,华也。

(39) 阳檏,地名,在蜀郡。招摇,山名,在桂阳。《礼记》曰:“草木之滋,姜桂之谓也。”故曰“和之美”。

(40) 越骆,国名。菌,竹笋也。鳣鲔,大鱼也,以为醢酱。无骨曰醢,有骨曰臡。

(41) 大夏,泽名,或曰山名,在西北。盐,形盐。宰揭,山名,处则未闻。

【校】梁仲子云:“《初学记》引作‘揭雩之露,其色紫’,《御览》十二同。”

(42) 长泽,大泽,在西方。大鸟之卵,卵大如瓮也。

(43) 饭,食也。玄山,处则未闻。不周,山名,在西北方,昆仑之西北。

(44) 山南曰阳,昆仑之南,故曰“阳山”。南海,南方之海。穄,关西谓之 ,冀州谓之 。秬,黑黍也。

【校】孙云:“案《说文》禾部‘秏’字注‘伊尹曰:饭之美者,玄山之禾,南海之秏’。”注“ ”,旧讹“糜”,又“ ”,旧讹“坚”,今皆改正。

(45) 三危,西极山名。

(46) 井,泉。

(47) 沮渐如江旁之泉水。

(48) 皆西方之山泉也。冀州在中央,水泉东流,经于冀州,故曰“之原”。原,本也。

【校】“曰山”当是“白山”。“高泉”,《中山经》作“高前”。

(49) 沙棠,木名也,昆仑山有之。

【校】见《西山经》。

(50) 有核曰果,无核曰蓏。群帝,众帝,先升遐者。

(51) 箕山,许由所隐也,在颍川阳城之西。青鸟,昆仑山之东。二处皆有甘栌之果。

【校】《史记·司马相如传》索隐引应劭曰:“《伊尹书》云‘箕山之东,青鸟之所,有卢橘,夏熟’。”此或误记。《说文》“栌”字下引作“青凫”,师古《汉书注》讹作“青马”,《海外北经》注引作“‘有甘柤焉’,柤音柤梨之柤”,又不同。

(52) 浦,滨也,橘所生也。生江北则为枳。云梦,楚泽,出柚。

(53) 汉,水名,出于嶓冢,东注于江。石耳,菜名也。所以致之,致备味也。

(54) 匹、乘,皆马名。《周礼》“七尺以上为龙”。行迅谓之遗风。

(55) 言当顺天命而受之,不可以强取也。道,谓仁义天下之道。

(56) 彼谓他人。

(57) 己成仁义之道而成为天子。《孟子》曰:“得乎丘民为天子。”

(58) 天下贡珍,故至味具。

(59) 要,约也。越越,轻易之貌。业,事也。圣王得仁义约要之道以化天下,天下化之,岂必越越然轻易多为民之事也。





首时 (1)


(1) 【校】一作“胥时”。

三曰:

圣人之于事,似缓而急、 (1) 似迟而速以待时。 (2) 王季历困而死,文王苦之, (3) 有不忘羑里之丑。时未可也。 (4) 武王事之,夙夜不懈,亦不忘王门之辱, (5) 立十二年,而成甲子之事, (6) 时固不易得。 (7) 太公望,东夷之士也, (8) 欲定一世而无其主。 (9) 闻文王贤, (10) 故钓于渭以观之。 (11)

(1) 似缓,谓无为也。急,谓成功也。

(2) 谓若武王会于孟津,八百诸侯皆曰:“纣可伐矣。”武王曰:“汝未知天命也。”还归二年,似迟也。甲子之日克纣于牧野,故曰“待时”。

(3) 王季历,文王之父也。勤劳国事,以至薨没,故文王哀思苦痛也。

(4) 纣为无道,拘文王于羑里。不忘其丑耻也,所以不伐纣者,天时之未可也。

(5) 武王继位,虽臣事纣,不忘文王为纣所拘于羑里之辱。文王得归,乃筑灵台,作王门,相女童,击钟鼓,示不与纣异同也。武王以此为耻而不忘也。

【校】“王门”即“玉门”,古以中画近上为“王”字,“王”三画正均即“玉”字。《淮南·道应训》注云:“以玉饰门也。”注“击”字旧本缺,据《淮南注》补。又下脱“异”字亦案文义补。

(6) 立为天子也。甲子之日克纣牧野,故曰“成甲子之事”。

(7) 固,常也。

(8) 太公望,河内人也。于周丰、镐为东,故曰“东夷之士”。

【校】《史记》“太公望,东海上人也”,此云河内,不知何本。

(9) 主,谓贤君。

(10) 文,谥也。经天纬地曰文。

(11) 渭,水名,近丰、镐,文王所邑也。观视文王之德,能有天下也。

伍子胥欲见吴王而不得, (1) 客有言之于王子光者,见之而恶其貌,不听其说而辞之。 (2) 客请之王子光,王子光曰:“其貌适吾所甚恶也。” (3) 客以闻伍子胥,伍子胥曰:“此易故也。 (4) 愿令王子居于堂上,重帷而见其衣若手,请因说之。”王子许。 (5) 伍子胥说之半,王子光举帷,搏其手而与之坐; (6) 说毕,王子光大说。 (7) 伍子胥以为有吴国者,必王子光也,退而耕于野七年。王子光代吴王僚为王,任子胥。子胥乃修法制,下贤良,选练士,习战斗,六年,然后大胜楚于柏举, (8) 九战九胜,追北千里, (9) 昭王出奔随,遂有郢, (10) 亲射王宫,鞭荆平之坟三百。 (11) 乡之耕,非忘其父之仇也,待时也。 (12)

(1) 吴王,王僚也,王子光之庶长子。

【校】此注讹舛显然。刘本、汪本改“子光”二字为“夷昧”,似顺而实非也。梁伯子云:“《史记》以吴王僚为夷昧之子,光为诸樊之子,《汉书人表》亦以僚为夷昧子,而《公羊襄廿九年传》谓僚者长庶,《左传昭廿七年正义》据《世本》以僚为寿梦庶子,夷昧庶兄,而光为夷昧子。先儒皆从《史记》,不从《世本》。乃高氏于《当染》、《简选》、《察微》三篇注云‘夷昧子光’,于《忠廉》篇云‘光庶父僚’,皆依《世本》为说。此处若依刘、汪改本,是又依《史记》为说,且误解《公羊》‘长庶’一语,以为夷昧之庶子,而不自知其矛盾矣。”卢云:“案此注但当改‘庶长子’为‘庶父’,便与前后注合,且下文王子光即于此注内带见亦是,今去‘子光’而改‘夷昧’,尚剩一‘王’字未去,所改未为得也。”

(2) 光恶子胥之颜貌,不受其言,辞谢之也。

(3) 请,问也。恶,憎也。

(4) 故,事。

(5) 言于重帷中见衣若手者,为说霸国之说也。许,诺。

(6) 搏执子胥之手,与之俱坐,听其说。

(7) 子胥说霸术毕,子光大说,其将必用之也。

(8) 柏举,楚南鄙邑。

(9) 北,走也。

(10) 郢,楚都。《传》云“五战及郢”。

(11) 平王,恭王之子弃疾也,后改名熊居,听费无忌之谗,杀伍子胥父兄,故子胥射其宫、鞭其坟也。

(12) 乡,曩者。始之吴时,耕于吴境,待天时,须楚之罪熟也。

墨者有田鸠,欲见秦惠王, (1) 留秦三年而弗得见。客有言之于楚王者,往见楚王。楚王说之,与将军之节以如秦。 (2) 至,因见惠王。告人曰:“之秦之道,乃之楚乎!”固有近之而远,远之而近者。 (3) 时亦然。有汤、武之贤,而无桀、纣之时,不成; (4) 有桀、纣之时,而无汤、武之贤,亦不成。圣人之见时,若步之与影不可离。 (5)

(1) 田鸠,齐人,学墨子术。惠王,孝公之子驷也。

(2) 如,之也。

(3) 留秦三年,不得见惠王,近之而远也。从楚来,至而得见,远之而近也。

(4) 不成其王。

(5) 步行日中,影乃逐之,不可得远之也。人从得时,如影之随人,亦不可离之也。

故有道之士未遇时,隐匿分窜,勤以待时。 (1) 时至,有从布衣而为天子者, (2) 有从千乘而得天下者, (3) 有从卑贱而佐三王者, (4) 有从匹夫而报万乘者。 (5) 故圣人之所贵,唯时也。水冻方固, (6) 后稷不种,后稷之种必待春。故人虽智而不遇时,无功。 (7) 方叶之茂美,终日采之而不知; (8) 秋霜既下,众林皆羸。 (9) 事之难易,不在小大,务在知时。 (10)

(1) 分,大。窜,藏。勤,劳。

【校】注“大”字疑“ ”之讹,即“别”字。

(2) 舜是也。

(3) 汤、武是也。

(4) 太公望、伊尹、傅说是也。

(5) 豫让是也。赵襄子兼土拓境,有兵车万乘。豫让为智伯报之,襄子高其义而不杀。豫让卒不止,终得斩襄子衬身之衣然后就死也。

(6) 固,坚也。

(7) 五稼非春不生。智者之功,非时不成。

(8) 不知其叶之尽也。

(9) 羸,叶尽也。

(10) 圣人时行则行,时止则止,与万物终始也。

郑子阳之难,猘狗溃之; (1) 齐高国之难,失牛溃之。众因之以杀子阳、高国。 (2) 当其时,狗牛犹可以为人唱,而况乎以人为唱乎?饥马盈厩,嗼然, (3) 未见刍也。饥狗盈窖, (4) 嗼然,未见骨也。见骨与刍,动不可禁。 (5) 乱世之民,嗼然,未见贤者也。见贤人,则往不可止。往者非其形心之谓乎?齐以东帝困于天下,而鲁取徐州; (6) 邯郸以寿陵困于万民,而卫取茧氏。 (7) 以鲁卫之细,而皆得志于大国,遇其时也。 (8) 故贤主秀士之欲忧黔首者,乱世当之矣。 (9) 天不再与,时不久留,能不两工,事在当之。 (10)

(1) 溃,乱也。子阳,郑相,或曰郑君。好行严猛,人家有猘狗者诛之,人畏诛,国人皆逐猘狗也。

(2) 众因之以杀二子。逐失牛之乱,如逐猘狗之乱也,故祸同。

(3) 嗼然,无声。

(4) 【校】《御览》八百九十六作“宫”字。

(5) 动犹争也。

(6) 齐湣王僭号于东,民不顺之,故困于天下,是以鲁国略取徐州也。

(7) 寿陵,魏邑,赵兼有之,万民不附,是以卫人取其茧氏之邑也。

(8) 细,小也。遇大国之民皆欲之,则取之也。

(9) 当乱世忧而济之者。

(10) 天不再与,一姓不再兴。时不久留,日中则昃者也。





义赏


四曰:

春气至则草木产,秋气至则草木落,产与落或使之,非自然也。故使之者至,物无不为;使之者不至,物无可为。 (1) 古之人审其所以使,故物莫不为用。 (2) 赏罚之柄,此上之所以使也。其所以加者义,则忠信亲爱之道彰。 (3) 久彰而愈长,民之安之若性,此之谓教成。教成,则虽有厚赏严威弗能禁。 (4) 故善教者不以赏罚而教成,教成而赏罚弗能禁。用赏罚不当亦然。 (5) 奸伪贼乱贪戾之道兴, (6) 久兴而不息,民之雠之若性, (7) 戎、夷、胡、貉、巴、越之民是以,虽有厚赏严罚弗能禁。 (8) 郢人之以两版垣也,吴起变之而见恶, (9) 赏罚易而民安乐。 (10) 氐羌之民,其虏也, (11) 不忧其系累,而忧其死不焚也。 (12) 皆成乎邪也。 (13) 故赏罚之所加,不可不慎,且成而贼民。 (14)

(1) 未春无可为生,未秋无可为落。

(2) 使之者以其时生则生,时落则落,故曰“莫不为用”。

(3) 彰,明也。

(4) 言德教一成,虽复赏罚之使为不忠不信,人人自为忠信,若性自然,不可禁止也。

(5) 言民为不忠不信,亦不能禁。

(6) 兴,作也。

(7) 雠,用也。

(8) 禁,止也。

(9) 郢,楚都也。楚人以两版筑垣。吴起,卫人也,楚以为将。变其两版,教之用四,楚俗习久见怨也,《公羊传》曰:“文公逆祀,去者三人;定公顺祀,叛者五人。”此之谓久习也。

(10) 易其邪而施其正,民去邪从正,故安乐也。

(11) 氐与羌二种夷民。言氐羌之民为寇贼,为人执虏也。

(12) 焚,烧也。

(13) 不得天之正气。

(14) 赏罚正而民正,赏罚不正而民邪,故曰“且成而贼民”,是以君人慎之也。

昔晋文公将与楚人战于城濮, (1) 召咎犯而问曰:“楚众我寡,奈何而可?” (2) 咎犯对曰:“臣闻繁礼之君,不足于文;繁战之君,不足于诈。 (3) 君亦诈之而已。”文公以咎犯言告雍季,雍季曰:“竭泽而渔,岂不获得?而明年无鱼。焚薮而田,岂不获得?而明年无兽。 (4) 诈伪之道,虽今偷可,后将无复, (5) 非长术也。”文公用咎犯之言, (6) 而败楚人于城濮。 (7) 反而为赏,雍季在上。 (8) 左右谏曰:“城濮之功,咎犯之谋也。君用其言而赏后其身,或者不可乎!”文公曰:“雍季之言,百世之利也。咎犯之言,一时之务也。 (9) 焉有以一时之务先百世之利者乎?”孔子闻之,曰:“临难用诈,足以却敌;反而尊贤,足以报德。文公虽不终,始足以霸矣。”赏重则民移之,民移之则成焉。 (10) 成乎诈,其成毁, (11) 其胜败。 (12) 天下胜者众矣,而霸者乃五。 (13) 文公处其一,知胜之所成也。 (14) 胜而不知胜之所成,与无胜同。 (15) 秦胜于戎,而败乎殽; (16) 楚胜于诸夏,而败乎柏举。 (17) 武王得之矣, (18) 故一胜而王天下。 (19) 众诈盈国,不可以为安,患非独外也。 (20)

(1) 城濮,楚北境之地名。

(2) 咎犯,狐偃也,字子犯,文公之舅也,因曰“咎犯”。

【校】古“咎”与“舅”同。

(3) 足犹厌也。诈者,谓诡变而用奇也。

【校】旧校云:“一本作‘以力战之君,不足于力;以诈战之君,不足于诈’。”

(4) 言尽其类。

(5) 不可复行。

(6) 言,谋也。

(7) 败,破也。

(8) 上,首也。

(9) 务犹事。

(10) 移犹归。

(11) 虽成必毁。

(12) 虽胜后必毁败。

(13) 乃犹裁也。

(14) 居五霸之一。

(15) 同,等也。

(16) 秦缪公破西戎而霸,使孟明、白乙丙、西乞术将师东袭郑。郑人知之,还,晋襄公御之殽,大破之,获其三帅。

(17) 庄王服郑胜晋于邲,故曰胜乎诸夏也。及昭王南与吴人战,吴破之柏举。此皆不知胜之所成也,故曰“与无胜同”。

(18) 得犹知。

(19) 一胜,克纣。

(20) 亦从内发之也。

赵襄子出围,赏有功者五人,高赦为首。 (1) 张孟谈曰:“晋阳之中,赦无大功,赏而为首,何也?”襄子曰:“寡人之国危,社稷殆,身在忧约之中,与寡人交而不失君臣之礼者,惟赦。 (2) 吾是以先之。”仲尼闻之曰:“襄子可谓善赏矣!赏一人,而天下之为人臣莫敢失礼。” (3) 为六军则不可易, (4) 北取代,东迫齐,令张孟谈逾城潜行,与魏桓、韩康期而击智伯,断其头以为觞, (5) 遂定三家, (6) 岂非用赏罚当邪? (7)

(1) 智伯求地于襄子,襄子不与,智伯率韩、魏之君围赵襄子于晋阳三月。张孟谈私与韩、魏构谋,韩、魏反智伯军,使赵襄子杀之,故曰“出围”。

【校】《韩非·难一》,《淮南·氾论》、《人间训》,《说苑·复恩》篇,《古今人表》,“高赦”并作“高赫”,《史记·赵世家》作“高共”,徐广曰:“一作‘赫’。”

(2) 惟,独。

(3) 一人,谓高赦。

【校】王伯厚云:“赵襄子事在孔子后,孔鲋已辩其妄。”

(4) 易,轻。

(5) 觞,酒器也。

【校】孙云:“案此可证饮器之为酒器。”

(6) 韩、魏、赵也。

(7) 当,正也。





长攻


五曰:

凡治乱存亡,安危强弱,必有其遇,然后可成,各一则不设。 (1) 故桀、纣虽不肖,其亡遇汤、武也,遇汤、武,天也,非桀、纣之不肖也。汤、武虽贤,其王遇桀、纣也,遇桀、纣,天也,非汤、武之贤也。若桀、纣不遇汤、武,未必亡也;桀、纣不亡,虽不肖,辱未至于此。 (2) 若使汤、武不遇桀、纣,未必王也;汤、武不王,虽贤,显未至于此。 (3) 故人主有大功不闻不肖, (4) 亡国之主不闻贤。 (5) 譬之若良农,辩土地之宜,谨耕耨之事,未必收也;然而收者,必此人也。 (6) 始在于遇时雨,遇时雨,天地也,非良农所能为也。

(1) 遇犹遭也。各有一乱,不能相治。《传》曰:“以乱平乱,何治之有?”故不设攻战相攻伐也。

(2) 至于此,灭亡也。

(3) 显,荣。此,天下。

(4) 功名掩也。

(5) 乱以掩也。

(6) 收由耕耨始也,故曰“必此人也”。

越国大饥, (1) 王恐,召范蠡而谋。范蠡曰:“王何患焉? (2) 今之饥,此越之福,而吴之祸也。夫吴国甚富而财有余,其王年少,智寡才轻,好须臾之名,不思后患。 (3) 王若重币卑辞以请籴于吴,则食可得也。 (4) 食得,其卒越必有吴,而王何患焉?” (5) 越王曰:“善!”乃使人请食于吴。吴王将与之,伍子胥进谏曰:“不可与也!夫吴之与越,接土邻境,道易人通, (6) 仇雠敌战之国也,非吴丧越,越必丧吴。若燕、秦、齐、晋,山处陆居,岂能逾五湖九江,越十七阨以有吴哉? (7) 故曰非吴丧越,越必丧吴。今将输之粟,与之食,是长吾雠而养吾仇也。 (8) 财匮而民恐, (9) 悔无及也。不若勿与而攻之,固其数也, (10) 此昔吾先王之所以霸。且夫饥,代事也, (11) 犹渊之与阪,谁国无有?”吴王曰:“不然。 (12) 吾闻之,义兵不攻服,仁者食饥饿。今服而攻之,非义兵也;饥而不食,非仁体也。不仁不义,虽得十越,吾不为也。”遂与之食。不出三年,而吴亦饥,使人请食于越。越王弗与,乃攻之,夫差为禽。 (13)

(1) 谷不熟。

(2) 【校】《说苑·权谋》篇四水进谏语与下文略同。

(3) 其王,吴王夫差也。

【校】正文“其王”,旧本脱“其”字,今据注增。

(4) 王,越王句践也。

(5) 得其籴,终必得其国,王何忧焉?

(6) 【校】《说苑》无“人”字。

(7) 逾,度也。越,历也。谓彼险难也。

【校】“九江”,《说苑》作“三江”。

(8) 【校】《御览》八百四十“养”作“豢”。

(9) 【校】《说苑》作“怨”。

(10) 数,术。

(11) 先王,谓阖闾也。代,更也。

(12) 吴王,夫差。

(13) 夫差,吴王也。禽,为越所获。

楚王欲取息与蔡, (1) 乃先佯善蔡侯,而与之谋曰:“吾欲得息,奈何?”蔡侯曰:“息夫人,吾妻之姨也。 (2) 吾请为飨息侯与其妻者,而与王俱,因而袭之。” (3) 楚王曰:“诺。”于是与蔡侯以飨礼入于息,因与俱,遂取息。旋舍于蔡,又取蔡。 (4)

(1) 楚王,文王也。息、蔡,二国名。

(2) 蔡侯,昭侯也。妻之女弟为姨,《传》曰“吾姨也”,此之谓也。

【校】案:此乃蔡哀侯也,注误。又“女弟”当作“女兄弟”。

(3) 【校】旧校云:“‘而’一作‘以’。”

(4) 不劳师徒而得之曰取,《传》曰“易也”。

赵简子病,召太子而告之曰:“我死,已葬,服衰而上夏屋之山以望。” (1) 太子敬诺。简子死,已葬,服衰,召大臣而告之曰:“愿登夏屋以望。”大臣皆谏曰:“登夏屋以望,是游也。服衰以游,不可。”襄子曰:“此先君之命也,寡人弗敢废。”群臣敬诺。襄子上于夏屋,以望代俗, (2) 其乐甚美。于是襄子曰:“先君必以此教之也。”及归, (3) 虑所以取代,乃先善之。代君好色,请以其弟姊妻之, (4) 代君许诺。弟姊已往,所以善代者乃万故。 (5) 马郡宜马,代君以善马奉襄子。 (6) 襄子谒于代君而请觞之,马郡尽, (7) 先令舞者置兵其羽中数百人, (8) 先具大金斗。代君至,酒酣, (9) 反斗而击之,一成,脑涂地。 (10) 舞者操兵以斗,尽杀其从者。因以代君之车迎其妻,其妻遥闻之状, (11) 磨笄以自刺。故赵氏至今有刺笄之证 (12) 与“反斗”之号。

(1) 赵简子,晋大夫赵景子成之子鞅也。太子,赵无恤襄子也。服衰,谓期年,勿复三年也。夏屋山,代之南山也。观望,欲令取代也。

(2) 俗,土也。

(3) 【校】旧校云:“一作‘反归’。”

(4) 【校】案:“弟姊”二字不当连文。据《赵世家》,襄子之姊前为代王夫人,是“弟”字衍。

(5) 善,好也。襄子所好于代者非一事,故言“万故”也。

(6) 《传》曰“冀州之北土,马之所生也”,故谓代为马郡也。言代君以马奉襄子也。

【校】《传》无“州”字。

(7) 谒,告也。觞,飨也。襄子告代君而请饮之酒,醉而杀之,尽取其国也,故曰“马郡尽”也。

【校】“马郡尽”似当在上节之下,言善马俱尽也。注欠顺。

(8) 羽,舞者所执持也。置兵其中,不欲代君觉之也。

(9) 金斗,酒斗也。金重,大,作之可以杀人。酣,饮酒合乐之时。

(10) 一成,一下也。首碎,故脑涂地也。

(11) 【校】疑“之”字衍。

(12) 【校】旧校云:“一作‘山’。”

此三君者,其有所自而得之,不备遵理, (1) 然而后世称之,有功故也。有功于此,而无其失,虽王可也。 (2)

(1) 三君,越王句践、楚文王、赵襄子也。自,从也。遵,循也。理,道也。

(2) 此三君有功名,假令无其阙失,虽为王可也。





慎人 (1)


(1) 【校】一作“顺人”。

六曰:

功名大立,天也。为是故,因不慎其人,不可。 (1) 夫舜遇尧,天也。舜耕于历山,陶于河滨,钓于雷泽, (2) 天下说之,秀士从之,人也。夫禹遇舜,天也。禹周于天下,以求贤者,事利黔首, (3) 水潦川泽之湛滞壅塞可通者,禹尽为之,人也。夫汤遇桀,武遇纣,天也。汤、武修身积善为义,以忧苦于民,人也。 (4)

(1) 推之于天,不复慎其为人、修仁义,故曰“不可”也。

(2) 陶,作瓦器。

(3) 事,治也。黔首,民也。

(4) 苦,劳也。

舜之耕渔,其贤不肖与为天子同。 (1) 其未遇时也,以其徒属堀地财,取水利, (2) 编蒲苇,结罘网,手足胼胝不居, (3) 然后免于冻 之患。 (4) 其遇时也,登为天子,贤士归之,万民誉之,丈夫女子,振振殷殷,无不戴说。 (5) 舜自为诗曰:“普天之下,莫非王土;率土之滨,莫非王臣。”所以见尽有之也。 (6) 尽有之,贤非加也; (7) 尽无之,贤非损也。 (8) 时使然也。

(1) 同,辞也。

【校】注“辞”疑“等”之误。

(2) 地财,五谷。水利,濯灌。

【校】“以”、“与”同。“堀”当作“掘”。

(3) 居,止。

(4) 患,难也。

(5) 振振殷殷,众友之盛。

【校】孙云:“‘振振’,王元长《曲水诗序》‘殷殷均乎姚泽’,李善注先引此作‘陈陈殷殷,无不戴说。高诱曰:殷,盛也’,后又引此作‘辄辄敐敐,莫不戴说。高诱曰:敐敐,动而喜貌也。殷殷或为敐敐,故两引之。辄,知叶切。敐,仕勤切’。案此所引盖《吕览》别本。又《广韵》一先有‘ ’字在田字纽下,引‘天子 敐敐,莫不载悦’,注‘喜悦之貌’;又十九臻有‘敐’字,引《吕氏春秋注》云‘ ,动而喜貌’。‘辄’、‘ ’、‘敐’、‘ ’皆与《吕氏》今本不同,而又互异。《说文》欠部‘ ’云‘指而笑也’,然则从‘攴’从‘殳’皆非。”

(6) 【校】王伯厚云:“疑与咸邱蒙同一说而托之于舜。”

(7) 加,益也。

(8) 损,减。

百里奚之未遇时也,亡虢而虏晋, (1) 饭牛于秦,传鬻以五羊之皮。公孙枝得而说之, (2) 献诸缪公,三日,请属事焉。 (3) 缪公曰:“买之五羊之皮而属事焉,无乃天下笑乎?”公孙枝对曰:“信贤而任之,君之明也;让贤而下之,臣之忠也。 (4) 君为明君,臣为忠臣。彼信贤,境内将服,敌国且畏,夫谁暇笑哉?”缪公遂用之。谋无不当,举必有功, (5) 非加贤也。使百里奚虽贤,无得缪公,必无此名矣。今焉知世之无百里奚哉?故人主之欲求士者,不可不务博也。

(1) “虢”当为“虞”。百里奚,虞臣也。《传》曰:“伐虞,获其大夫井伯以媵秦缪姬。”《孟子》曰:“百里奚,虞人也。晋人以垂棘之璧假道于虞以伐虢,宫之奇谏之,百里奚知虞公之不可谏也而去之秦。”此云亡虢,误矣。扬子云恨不及其时,车载其金。

(2) 公孙枝,秦大夫子桑。

(3) 献,进也。请以大夫职事属付百里奚也。

(4) 下,避也。

(5) 【校】《御览》四百二此下有“号曰五羖大夫”六字。

孔子穷于陈、蔡之间,七日不尝食,藜羹不糁。宰予备矣。 (1) 孔子弦歌于室,颜回择菜于外,子路与子贡相与而言曰:“夫子逐于鲁,削迹于卫,伐树于宋, (2) 穷于陈、蔡。杀夫子者无罪,藉夫子者不禁。 (3) 夫子弦歌鼓舞未尝绝音,盖君子之无所丑也若此乎?” (4) 颜回无以对,入以告孔子。孔子憱然推琴,喟然而叹曰:“由与赐,小人也!召,吾语之。”子路与子贡入,子贡曰 (5) :“如此者,可谓穷矣!”孔子曰:“是何言也?君子达于道之谓达,穷于道之谓穷。 (6) 今丘也拘仁义之道, (7) 以遭乱世之患,其所也,何穷之谓? (8) 故内省而不疚于道,临难而不失其德。大寒既至,霜雪既降,吾是以知松柏之茂也。 (9) 昔桓公得之莒,文公得之曹,越王得之会稽。 (10) 陈、蔡之厄,于丘其幸乎!”孔子烈然返瑟而弦, (11) 子路抗然执干而舞。 (12) 子贡曰:“吾不知天之高也,不知地之下也。” (13) 古之得道者,穷亦乐, (14) 达亦乐, (15) 所乐非穷达也, (16) 道得于此,则穷达一也, (17) 为寒暑风雨之序矣。 (18) 故许由虞乎颍阳, (19) 而共伯得乎共首。 (20)

(1) “备”当作“惫”。惫,极也。《论语》曰:“卫灵公问陈于孔子,对曰:‘俎豆之事,则尝闻之矣。军旅之事,未之学也。’明日遂行。在陈绝粮,从者病,莫能兴。”此之谓也,故曰“宰予惫矣”。

(2) 【校】旧校云:“‘伐’一作‘拔’。”案:《风俗通·穷通》篇作“拔”。

(3) 藉犹辱也。

【校】案:藉,陵藉也。

(4) 丑犹耻也。

(5) 【校】《庄子·让王》篇及《风俗通》俱作“子路曰”。

(6) 《论语》曰:“君子亦有穷乎?子曰:‘君子固穷,小人穷斯滥矣。’”

(7) 【校】“拘”,《庄子》、《风俗通》并作“抱”。

(8) 言不穷于道也。

(9) 众木遇霜雪皆凋,喻小人遭乱世无以自免。松柏喻君子而能茂盛也。《论语》曰:“岁寒然后知松柏之后凋。”此之谓也。

(10) 齐桓公遭无知之乱出奔莒,晋文公遇丽姬之谗出过曹,越王句践与吴战而败,栖于会稽之山,卒皆享国,克复其耻,为霸君,故曰“得之”。

(11) 返,更也。更取瑟而弦歌。

【校】“烈然返瑟”,《庄子》作“削然反琴”。

(12) 干,楯也。

【校】“抗然”,《庄子》作“扢然”。

(13) 高下喻广大也。言不能知孔子圣德之如天地。

(14) 乐其道也。

(15) 乐兼善天下也。

(16) 言乐道也。

(17) 此,近,喻身也。言得道之人,不为穷极,不为达显,故一之也。

(18) 寒暑,阴阳也。阴阳和,风雨序也。圣人法天地,顺阴阳,故能不为穷达变其节也。

(19) 虞,乐也。颍水之北曰阳。轻天下而不屈于尧,养志于箕山,山在颍水之北,故曰乐乎颍阳也。

(20) 共,国;伯,爵也。弃其国,隐于共首山而得其志也。不知出何书也。

【校】梁伯子云:“共伯值厉王之难,摄政十四年,乃率诸侯会二相而立宣王,共伯归共国,得乎共首,所谓‘逍遥得志乎共山之首’云尔,安得有弃国隐山之事?《开春论》注又以共伯为夏时诸侯,大误。”卢云:“案诱时《竹书纪年》犹未出,故云不知出何书,而所言皆误也。”





遇合


七曰:

凡遇,合也。时 (1) 不合,必待合而后行。故比翼之鸟死乎木,比目之鱼死乎海。孔子周流海内,再干世主,如齐至卫,所见八十余君,委质为弟子者三千人,达徒七十人。七十人者,万乘之主得一人用可为师,不为无人。以此游仅至于鲁司寇, (2) 此天子之所以时绝也,诸侯之所以大乱也。 (3) 乱则愚者之多幸也,幸则必不胜其任矣。 (4) 任久不胜,则幸反为祸。其幸大者,其祸亦大,非祸独及己也。故君子不处幸,不为苟, (5) 必审诸己然后任,任然后动。 (6)

(1) 【校】句下当叠一“时”字。

(2) 仅犹裁也。孔子有圣德,不见大用,裁至于司寇也。

(3) 言不知圣人,不能用之,所以绝、所以乱也。

(4) 多幸爱不肖之人而宠用之,故不胜其任。

(5) 处,居也。不为苟易邀于俗、取容说也。

(6) 任则处德,动则量力。

凡能听说者,必达乎论议者也。世主之能识论议者寡,所遇恶得不苟? (1) 凡能听音者,必达于五声。 (2) 人之能知五声者寡,所善恶得不苟? (3) 客有以吹籁见越王者,羽、角、宫、徵、商不缪, (4) 越王不善;为野音,而反善之。 (5) 说之道亦有如此者也。 (6) 人有为人妻者,人告其父母曰:“嫁不必生也。 (7) 衣器之物,可外藏之,以备不生。”其父母以为然,于是令其女常外藏。 (8) 姑妐知之曰:“为我妇而有外心, (9) 不可畜。”因出之。 (10) 妇之父母以谓为己说者以为忠,终身善之,亦不知所以然矣。 (11) 宗庙之灭,天下之失,亦由此矣。 (12) 故曰遇合也无常。说,适然也。若人之于色也,无不知说美者,而美者未必遇也。故嫫母执乎黄帝, (13) 黄帝曰:“厉女德而弗忘,与女正而弗衰,虽恶奚伤?” (14) 若人之于滋味,无不说甘脆,而甘脆未必受也。文王嗜昌蒲菹, (15) 孔子闻而服之,缩頞而食之,三年然后胜之。 (16) 人有大臭者,其亲戚、兄弟、妻妾、知识无能与居者,自苦而居海上。 (17) 海上人有说其臭者,昼夜随之而弗能去。 (18) 说亦有若此者。

(1) 恶,安也。

(2) 达,通也。

(3) 【校】旧校云:“‘善’一作‘喜’。”

(4) 籁,二孔籥也。不缪,五声无失。

(5) 野,鄙也。

(6) 说贤人而不用,言不肖而归之,故曰“亦有如此者”也。

(7) 不必生,谓终死。

(8) 藏私财于外也。

(9) 【校】《释名》:“俗或谓舅曰章,又曰妐。”旧校云:“‘外心’一作‘异心’。”

(10) 以为盗窃,犯七出,故出之也。

(11) 不知其女之所以见出由此也。

(12) 亦由此不理者,故宗庙灭没,以失其天下也。

(13) 黄帝说之。

(14) 恶,丑也。奚,何也。言敕厉女以妇德而不忘失,付与女以内正而不衰疏,故曰虽丑何伤,明说恶也。

【校】“厉”旧作“属”,案“属”与下“付与”意复,观注以敕为训,则当作“厉”字,因形近而讹,今并注俱改正。

(15) 昌本之菹。

(16) 胜,服。

(17) 苦,伤也。

(18) 去,离也。

陈有恶人焉,曰敦洽雠麋,椎颡广颜,色如漆赭, (1) 垂眼临鼻, (2) 长肘而盭。 (3) 陈侯见而甚说之, (4) 外使治其国,内使制其身。 (5) 楚合诸侯,陈侯病,不能往,使敦洽雠麋往谢焉。楚王怪其名而先见之, (6) 客有进状有恶其名言有恶状。楚王怒,合大夫而告之, (7) 曰:“陈侯不知其不可使,是不知也; (8) 知而使之,是侮也。 (9) 侮且不智,不可不攻也。”兴师伐陈,三月然后丧。 (10) 恶足以骇人,言足以丧国, (11) 而友之足于陈侯而无上也,至于亡而友不衰。 (12) 夫不宜遇而遇者,则必废。 (13) 宜遇而不遇者,此国之所以乱,世之所以衰也。 (14) 天下之民,其苦愁劳务从此生。 (15) 凡举人之本,太上以志,其次以事,其次以功。 (16) 三者弗能,国必残亡,群孽大至,身必死殃,年得至七十、九十犹尚幸。 (17) 贤圣之后,反而孽民,是以贼其身, (18) 岂能独哉? (19)

(1) 【校】“糜”旧作“麋”。案李善注左太冲《魏都赋》、刘孝标《辩命论》并作“麋”,《御览》三百八十二同,《初学记》作“眉”,与“麋”同,今定作“麋”。“椎”旧本作“雄”,校云“一作‘推’”。案《魏都赋》注作“椎”,今从之。《广韵》作“狭颡广额,颜色如漆”,今“漆赭”旧本作“浃赬”,校云“一作‘沬赭’”。“沬”或“柒”字之误,“柒”即“漆”字,《辩命论》注作“漆赭”,今从之。《初学记》作“色如漆”,无“赭”字。

(2) 【校】旧校云:“‘眼’一作‘发’。”

(3) 盭,胝也。

【校】“盭”即“戾”字,不当训胝。案《选》注引正文作“盭股”,今脱“股”字,误为“胝”入注中,而又误增二字也。

(4) 【校】《选》注引高诱曰“丑而有德也”,今本缺。下注“丑恶无德”,正相反。

(5) 制陈侯身。

(6) 【校】旧校云:“‘怪’一作‘知’。”

(7) 合,会。

(8) 不知,无所知也。

(9) 侮,慢。

(10) 丧,灭之也。

(11) 雠麋貌恶足以惊人,其言足以亡国也。

(12) 友爱敦洽雠麋无有出上者也。楚怒而伐之,以至于灭,而爱之不衰废也。

(13) 若敦洽雠麋,丑恶无德,不宜见遇而反见遇,如此者不必久,故曰“必废”也。

(14) 贤者至道,宜一遇明世,佐时理物,不遇之,故国不治,所以乱也。世不知贤不肖,所以衰也。

(15) 从此宜遇而不遇也。

(16) 举,用也。志,德也。

(17) 所遇不当,而无此三者,身必死殃也。得至七十、九十者,乃大幸耳。

(18) 【校】旧校云:“‘贼’一作‘残’。”

(19) 陈,舜之苗胤也,故曰“贤圣之后”也。孽,病也。所遇不当,为楚所灭,以残其身也,并病其民,故曰“岂能独哉”。





必己 (1)


(1) 【校】一作“本知”,一作“不遇”。

八曰:

外物不可必,故龙逄诛,比干戮, (1) 箕子狂,恶来死, (2) 桀、纣亡。 (3) 人主莫不欲其臣之忠,而忠未必信,故伍员流乎江, (4) 苌弘死,藏其血三年而为碧。 (5) 亲莫不欲其子之孝,而孝未必爱,故孝己疑,曾子悲。 (6)

(1) 龙逄谏桀而桀杀之。比干,纣之诸父也。谏纣,纣剖其心视之,故曰“戮”。

【校】此处“龙逄”,各本皆不作“逢”,仍之。

(2) 箕子,纣之庶父也,见纣之乱而佯狂也。恶来,飞廉之子,纣谀臣也,武王杀之。

(3) 杀忠臣,故灭亡。

(4) 伍子胥谏吴王夫差,不欲与越籴,夫差未信之,不从其言,以鸱夷置子胥而投之江也。

(5) 苌弘,周敬王大夫,号知天道,欲城成周,支天之所坏,故卫奚知其不得没也。及范吉射、荀寅叛其君,苌弘与知之。周刘氏、范氏世为婚姻,苌弘事刘文公,故周人与范氏。晋人让周,周为之杀苌弘。不当其罪,故血三年而为碧也。

【校】“卫奚”,《左传》作“卫徯”。

(6) 孝己,殷王高宗子也。曾参,以其至孝,见疑于其父,故为之伤悲也。

【校】注“以”字旧脱,今补。

庄子行于山中, (1) 见木甚美,长大,枝叶盛茂, (2) 伐木者止其旁而弗取。问其故,曰:“无所可用。”庄子曰:“此以不材得终其天年矣。”出于山,及邑,舍故人之家。 (3) 故人喜,具酒肉,令竖子为杀雁飨之。竖子请曰:“其一雁能鸣,一雁不能鸣,请奚杀?”主人之公曰:“杀其不能鸣者。”明日,弟子问于庄子曰:“昔者山中之木以不材得终天年,主人之雁以不材死, (4) 先生将何以处?”庄子笑曰:“周将处于材不材之间。材不材之间,似之而非也,故未免乎累。若夫道德则不然,无讶无訾, (5) 一龙一蛇,与时俱化,而无肯专为; (6) 一上一下,以禾为量, (7) 而浮游乎万物之祖, (8) 物物而不物于物,则胡可得而累? (9) 此神农、黄帝之所法。 (10) 若夫万物之情,人伦之传则不然, (11) 成则毁,大则衰,廉则锉, (12) 尊则亏,直则骩, (13) 合则离,爱则隳, (14) 多智则谋,不肖则欺, (15) 胡可得而必?”

(1) 【校】旧校云:“‘行’一作‘过’。”

(2) 庄子名周,宋之蒙人也,轻天下,细万物,其术尚虚无,著书五十二篇,名之曰《庄子》。

【校】五十二篇本《汉志》,今本十卷三十三篇。

(3) 舍,止也。故人,知旧也。

(4) 【校】旧校云:“一作‘以不能鸣死’。”

(5) 【校】《庄子·山木》篇作“无誉无疵”。

(6) 专,一。

(7) 禾三变,故以为法也。一曰:禾,中和。

【校】注“禾三变”,谓始于粟,生于苗,成于穗也。见《淮南子·缪称训》高诱注。旧本“三”上有“两”字,衍,今删去。

(8) 祖,始。

(9) 物物而不物,言制作。喻天地不在万物中,故曰“不物”。若制礼者不制于礼也,不以物自累之也。

(10) 法,则也。神农,少典之子赤帝也,居三皇之中,农殖嘉谷而化之,号曰神农。黄帝,轩辕氏也,得道而仙。言二帝以此为法则者也。

(11) 传犹转。

(12) 廉,利也。锉,缺伤。

(13) 尊,高也。《传》曰“高位疾颠”,故曰“则亏”。骩,曲也。直不可久,故曰“直则骩”。《诗》云“草木死无不萎”,此之谓也。

【校】此约《小雅·谷风》之诗“无草不死,无木不萎”二语而失之。

(14) 隳,废也。

(15) 多智则人谋料之,不肖则人欺诈之。

牛缺居上地,大儒也,下之邯郸,遇盗于耦沙之中。 (1) 盗求其橐中之载则与之,求其车马则与之,求其衣被则与之。牛缺出而去, (2) 盗相谓曰:“此天下之显人也,今辱之如此,此必愬我于万乘之主, (3) 万乘之主必以国诛我,我必不生,不若相与追而杀之,以灭其迹。” (4) 于是相与趋之, (5) 行三十里,及而杀之。此以知故也。 (6) 孟贲过于河,先其五。 (7) 船人怒,而以楫虓其头, (8) 顾不知其孟贲也。中河,孟贲瞋目而视船人,发植,目裂,鬓指, (9) 舟中之人尽扬播入于河。 (10) 使船人知其孟贲,弗敢直视, (11) 涉无先者, (12) 又况于辱之乎?此以不知故也。 (13) 知与不知,皆不足恃,其惟和调近之。犹未可必, (14) 盖有不辨和调者,则和调有不免也。 (15) 宋桓司马有宝珠,抵罪出亡。 (16) 王使人问珠之所在,曰:“投之池中。” (17) 于是竭池而求之,无得,鱼死焉。此言祸福之相及也。纣为不善于商,而祸充天地, (18) 和调何益? (19)

(1) 牛,姓也。缺,其名。秦人也。秦在西方,故称“下之邯郸”。淤沙为耦,盖地名也。

(2) 【校】《列子·说符》作“步而去”。

(3) 劫夺其财。不以礼为辱。愬,告也。

(4) 迹,踪也。

(5) 趋,逐。

(6) 盗知牛缺为贤人故。

【校】卢云:“知与不知,注皆不得本意。当云‘牛缺使盗知其为贤人故也’,下注当云‘孟贲不使船人知其为勇士故也’,此则与上文一意相承,所谓如此如彼,皆不可必也。”

(7) 【校】章怀注《后汉书·郑太传》引“孟贲过河,先于其伍”,古“伍”字作“五”。

(8) 先其伍,超越次弟也。虓,暴辱。

(9) 植,竖。指,直。

【校】《御览》三百六十六“鬓”作“须”。

(10) 扬,动也。播,散也。入犹投也。

(11) 直,正。

(12) 无敢先孟贲也。

(13) 船人不知孟贲为勇士故也。

(14) 近之,近无愁难,犹未可必也。

(15) 【校】卢云:“此二句颇似注中语误入正文,若直接上注‘犹未可必’之下,正相吻合。注末一‘也’字,当为衍文。”

(16) 桓司马,桓魋。抵,当也。

(17) 《春秋鲁哀十四年传》曰:“宋桓魋之有宠,欲害公。公知之,攻桓魋。魋出奔卫。”公则宋景公也。春秋时宋未僭称王也,此云“王使人问珠”,复妄言者也。

(18) 充犹大。

(19) 和调,善之者也。纣不能行之,故曰“何益”也。

【校】卢云:“此注又错说。本意谓当纣之时,善人亦不得免焉,如鱼之安处于池,而适遭求珠之害,故曰‘和调何益’。终篇皆言处世之难必耳。高氏意常歆羡秦市之金,岂亦知己之亦多误乎?”

张毅好恭,门闾帷 (1) 薄聚居众无不趋, (2) 舆隶姻媾小童无不敬,以定其身; (3) 不终其寿,内热而死。 (4) 单豹好术,离俗弃尘, (5) 不食谷实,不衣芮温, (6) 身处山林岩堀,以全其生;不尽其年,而虎食之。 (7) 孔子行道而息, (8) 马逸,食人之稼,野人取其马。子贡请往说之,毕辞,野人不听。有鄙人始事孔子者,曰请往说之,因谓野人曰:“子不耕于东海,吾不耕于西海也,吾马何得不食子之禾?” (9) 其野人大说,相谓曰:“说亦皆如此其辩也!独如向之人?” (10) 解马而与之。说如此其无方也而犹行, (11) 外物岂可必哉?

(1) 【校】旧校云:“‘帷’一作‘帐’。”

(2) 过之必趋。

(3) 定,安也。

(4) 《幽通记》曰:“张毅修襮而内逼。”此之谓也。

(5) 【校】旧校云:“一作‘弃世谓不群也’。”

(6) 不食谷实,行气道引也。芮,絮也。

(7) 《幽通记》曰:“单豹治里而外凋。”此之谓也。

【校】旧本作“治衷不外调”,讹,今据班孟坚赋改正。

(8) 【校】李善注《文选》、陆士衡《演连珠》“东野有不释之辩”引此作“孔子行于东野”。

(9) 【校】《选》注引作“子耕东海至于西海”,与《淮南·人间训》同。

(10) 独犹孰也。向之人,谓子贡也。

(11) 方,术。

君子之自行也,敬人而不必见敬,爱人而不必见爱。敬爱人者,己也;见敬爱者,人也。君子必在己者,不必在人者也。必在己,无不遇矣。




————————————————————

[1] 天瑞:原本作“黄帝”,误,据《列子》改。





第十五卷 慎大览



慎大


一曰:

贤主愈大愈惧,愈强愈恐。 (1) 凡大者,小邻国也;强者,胜其敌也。 (2) 胜其敌则多怨,小邻国则多患。多患多怨,国虽强大,恶得不惧?恶得不恐? (3) 故贤主于安思危, (4) 于达思穷, (5) 于得思丧。 (6) 《周书》曰:“若临深渊,若履薄冰。”以言慎事也。 (7)

(1) 愈,益。

(2) 夫大者,侵削邻国使小也;强者,强以克弱,故能胜其敌也。

(3) 恶,安也。

(4) 安不忘危。

(5) 显不忘约。

(6) 丧,亡也。有得有失,故思之。

(7) 《周书》,周文公所作也。若临深渊,恐陨坠也;如履薄冰,恐陷没也;故曰“以言慎事”。

桀为无道,暴戾顽贪, (1) 天下颤恐而患之, (2) 言者不同,纷纷分分,其情难得。 (3) 干辛任威,凌轹诸侯,以及兆民, (4) 贤良郁怨。杀彼龙逢,以服群凶。众庶泯泯,皆有远志, (5) 莫敢直言,其生若惊。 (6) 大臣同患,弗周而畔。 (7) 桀愈自贤,矜过善非, (8) 主道重塞,国人大崩。 (9) 汤乃惕惧,忧天下之不宁,欲令伊尹往视旷夏,恐其不信,汤由亲自射伊尹。 (10) 伊尹奔夏三年,反报于亳, (11) 曰:“桀迷惑于末嬉,好彼琬、琰, (12) 不恤其众。众志不堪,上下相疾,民心积怨,皆曰:‘上天弗恤,夏命其卒。’” (13) 汤谓伊尹曰:“若告我旷夏尽如诗。” (14) 汤与伊尹盟,以示必灭夏。伊尹又复往视旷夏,听于末嬉。末嬉言曰:“今昔天子梦西方有日,东方有日,两日相与斗,西方日胜,东方日不胜。”伊尹以告汤。商涸旱, (15) 汤犹发师,以信伊尹之盟,故令师从东方出于国,西以进。未接刃而桀走,逐之至大沙,身体离散,为天下戮。不可正谏,虽后悔之,将可奈何!汤立为天子,夏民大说,如得慈亲,朝不易位,农不去畴, (16) 商不变肆, (17) 亲郼如夏。 (18) 此之谓至公,此之谓至安,此之谓至信。尽行伊尹之盟,不避旱殃,祖伊尹世世享商。 (19)

(1) 心不则德义之经为顽,求无厌足为贪。

(2) 颤,惊也。患,忧也。

(3) 纷纷,殽乱也。分分,恐恨也。其情难得知也。

(4) 干辛,桀之谀臣也,专桀无道之威以致灭亡。

(5) 龙逢忠而桀杀之,故众庶泯泯然乱。有远志,离散也。

(6) 惊,乱貌。民不敢保其生也。

【校】旧校云:“‘惊’或作‘梦’。”

(7) 患,忧也。心惧尽见诛,故同忧也。不周于义,而将背畔也。

(8) 其所行者非,而反善也。

【校】旧校云:“‘矜’一作‘给’。”

(9) 崩,坏散。

(10) 恐夏不信伊尹,故由扬言而亲自射伊尹,示伊尹有罪而亡,令夏信之也。

【校】梁伯子云:“旷,空也。或云是‘犷’之讹,言其猛不可附也,古猛切。”卢云:“旷夏似言间夏。汤令伊尹为间于夏,而恐其不信,故亲射之。诸子书有言尹与末喜比而亡夏者,此出战国荒唐之言。观此下云‘若告我旷夏尽如志’,又云‘往视旷夏,听于末嬉’云云,亦即此意,是明明以伊尹为间谍也。”

(11) 亳,汤都。

(12) “琬”当作“婉”。婉顺阿意之人。或作“琬琰”,美玉也。

【校】观注意,则高所见本或有脱“琰”字者。案《竹书纪年》注云:“后桀十四年,命扁伐岷山,岷山女于桀二人,曰琬曰琰。后爱之,无子,斫其名于苕华之玉,苕是琬,华是琰,而弃其元妃于洛。曰妹喜以与伊尹交,遂以亡夏。”今本《纪年》末有讹字,此参用马骕所引文。据此则琬、琰不但为二玉名也。

(13) 卒,卒尽也。

(14) 诗,志也。

(15) 涸,枯也。

(16) 畴,亩也。

(17) 安其所也。

(18) 郼读如衣,今兖州人谓殷氏皆曰衣。言桀民亲殷如夏氏也。

【校】《书·武成》“殪戎殷”,《礼记·中庸》作“壹戎衣”,二字声本相近。

(19) 祖用伊尹之贤。世世享商,享之尽商世也。

武王胜殷,入殷,未下舆,命封黄帝之后于铸, (1) 封帝尧之后于黎, (2) 封帝舜之后于陈。下舆,命封夏后之后于杞,立成汤之后于宋,以奉桑林。 (3) 武王乃恐惧,太息流涕,命周公旦进殷之遗老,而问殷之亡故,又问众之所说,民之所欲。殷之遗老对曰:“欲复盘庚之政。” (4) 武王于是复盘庚之政, (5) 发巨桥之粟, (6) 赋鹿台之钱,以示民无私。 (7) 出拘救罪,分财弃责,以振穷困。 (8) 封比干之墓, (9) 靖箕子之宫, (10) 表商容之闾, (11) 士过者趋,车过者下。 (12) 三日之内,与谋之士封为诸侯, (13) 诸大夫赏以书社, (14) 庶士施政去赋。 (15) 然后济于河, (16) 西归报于庙。 (17) 乃税马于华山,税牛于桃林, (18) 马弗复乘,牛弗复服, (19) 衅鼓旗甲兵, (20) 藏之府库,终身不复用。此武王之德也。故周明堂外户不闭,示天下不藏也。唯不藏也可以守至藏。 (21) 武王胜殷,得二虏而问焉,曰:“若国有妖乎?” (22) 一虏对曰:“吾国有妖。昼见星而天雨血,此吾国之妖也。”一虏对曰:“此则妖也。虽然,非其大者也。吾国之妖,甚大者, (23) 子不听父,弟不听兄,君令不行,此妖之大者也。”武王避席再拜之。此非贵虏也,贵其言也。故《易》曰:“愬愬履虎尾,终吉。” (24)

(1) 铸,国名。

【校】《乐记》云“封帝尧之后于祝”,铸与祝声相近,此云封黄帝之后,殆误也。梁仲子云:“《淮南·俶真训》‘冶工之铸器’,注云‘铸读如唾祝之祝’;祝不读如字,《周礼·疡医》注云‘祝读如注病之注’;则知铸、祝同一音也。”

(2) 【校】《御览》二百一作“犁”。案《乐记》云“封黄帝之后于蓟”,黎与蓟声亦相近,此皆互易。

(3) 桑山之林,汤所祷也,故使奉之。

(4) 盘庚,太甲后十七世祖丁之子,殷之中兴王也,故欲复行其政也。

【校】注“十七世”当作“十五世”。

(5) 不违民欲。

(6) 巨桥,纣仓名。

(7) 鹿台,纣钱府。赋,布也。私,爱也。

(8) 分财,分有与无也。弃责,责己不责彼也。振,救也。矜寡孤独曰穷,无衣食曰困。

【校】“救罪”疑是“赦罪”。谢云:“弃责即《左传》所云‘已责’,‘责’古‘债’字,注非也。”

(9) 以其忠谏而见杀,故封崇其墓,以章贤也。

(10) 以箕子避乱,佯狂而奔,故清净其宫,以异之也。

【校】“靖”似当作“清”,七性切。

(11) 商容,殷之贤人,老子师也,故表异其闾里。

(12) 过商容之里者趋,车载者下也。

(13) 与谋委质于武王之士,封以为诸侯也。

(14) 大夫与谋为国,以书社赏之。二十五家为社也。

(15) 施之于政事,去其徭赋也。

(16) 【校】旧书“济于”倒,从《绎史》乙转。究疑“于”字乃衍文。

(17) 还济孟津河,西归于丰、镐,报功于文王庙。《传》曰“振旅凯入,饮至策勋”,此之谓也。

(18) 税,释也。华山在华阴南,西岳也。桃林,秦、晋之塞也,盖在华阴西长城是也。

(19) 【校】旧本作“牛弗服”,今亦从《绎史》增“复”字。

(20) 杀牲祭以血涂之曰衅。鼓以进众。旗,军械也。熊虎为旗。甲,铠。兵,戈戟箭矢也。

(21) 至德之藏。

(22) 若,汝。妖,怪。

(23) 【校】《新序·杂事二》“甚”作“其”。

(24) 愬愬,惧也。居之以礼,行之以恭,恐惧戒慎,如履虎尾,终必吉也。喻二虏见于武王,有履虎尾之危,以言所知,武王拜之,是终吉也。

【校】旧校云:“‘愬’一作‘遡’字,读如 。”谢云:“引《易》以戒人君,岂为二虏哉?注非是。”

赵襄子攻翟,胜老人、中人,使使者来谒之。 (1) 襄子方食抟饭,有忧色。左右曰:“一朝而两城下,此人之所以喜也, (2) 今君有忧色何?”襄子曰:“江河之大也, (3) 不过三日。 (4) 飘风暴雨,日中不须臾。 (5) 今赵氏之德行无所于积, (6) 一朝而两城下,亡其及我乎!” (7) 孔子闻之曰:“赵氏其昌乎?” (8) 夫忧所以为昌也,而喜所以为亡也。胜非其难者也,持之其难者也。 (9) 贤主以此持胜,故其福及后世。齐、荆、吴、越,皆尝胜矣,而卒取亡, (10) 不达乎持胜也。唯有道之主能持胜。孔子之劲,举国门之关,而不肯以力闻。 (11) 墨子为守攻,公输般服,而不肯以兵加。 (12) 善持胜者,以术强弱。 (13)

(1) 襄子,赵简子之子无恤也。使辛穆子伐翟,胜之,下老人、中人城,使使者来谒襄子。谒,告也。今卢奴西山中有老人、中人城也。

【校】案:《晋语》九、《列子·说符》及《御览》三百二十一皆作“左人中人”,《淮南·道应训》作“尤人终人”。

(2) 【校】《列子》无“以”字。

(3) 大,长。

(4) 三日则消也。

(5) 《易》曰“日中则仄”,故曰“日中不须臾”。

【校】旧校云:“‘飘风’一作‘焱风’。”案:日中不须臾,谓一日之中不过顷刻即过耳,即指风雨言,注非是。然如《列子·说符》篇“飘风暴雨”下有“不终朝”三字,则“日中”句当如注所云耳。

(6) 言无积德积行。

(7) 《传》曰:“知惧如此,斯不亡矣。”

(8) 昌,盛也。

【校】案:孔子卒时,简子尚在,此与《义赏》篇同误。

(9) 持犹守。

(10) 卒,终也。

【校】旧校云:“‘取’一作‘败’。”

(11) 劲,强也。孔子以一手捉城门关显而举之,不肯以有力闻于天下。

【校】此殆即孔子之父事也。《左氏襄十年传》“偪阳人启门,诸侯之士门焉,县门发,郰人纥抉之以出门者”,非孔子也。注“显”疑本是“翘”字。

(12) 公输般在楚,楚王使设云梯为攻宋之具,墨子闻而往说之。楚王曰:“公输般,天下之巧工也。寡人使攻宋之城,何为不得?”墨子曰:“使公输般攻宋之城,臣请为宋守之备。”公输般九攻之,墨子九却之。又令公输般守备,墨子九下之。不肯以善用兵见知于天下也。墨子,名翟,鲁人也,著书七十篇,以墨道闻也。

【校】案:《墨子》书本七十一篇,今缺者十六篇。注末“闻也”,旧作“闻之”,误。

(13) 言能以术强其弱也。

【校】旧校云:“一本作‘善持胜者,不以强弱’。”案:《列子》作“以强为弱”。





权勋


二曰:

利不可两,忠不可兼。 (1) 不去小利,则大利不得;不去小忠,则大忠不至。 (2) 故小利,大利之残也; (3) 小忠,大忠之贼也。圣人去小取大。

(1) 兼,并也。

(2) 至犹成也。

(3) 残,害也。

昔荆龚王与晋厉公战于鄢陵,荆师败,龚王伤。 (1) 临战,司马子反渴而求饮,竖阳穀操黍酒而进之。 (2) 子反叱曰:“訾! (3) 退!酒也。”竖阳穀对曰:“非酒也。”子反曰:“亟退却也!” (4) 竖阳穀又曰:“非酒也。”子反受而饮之。子反之为人也嗜酒,甘而不能绝于口,以醉。 (5) 战既罢,龚王欲复战而谋, (6) 使召司马子反,子反辞以心疾。龚王驾而往视之,入幄中, (7) 闻酒臭而还,曰:“今日之战,不穀亲伤,所恃者司马也。而司马又若此,是忘荆国之社稷,而不恤吾众也。不穀无与复战矣。”于是罢师去之,斩司马子反以为戮。故竖阳穀之进酒也,非以醉子反也, (8) 其心以忠也, (9) 而适足以杀之。故曰:小忠,大忠之贼也。

(1) 晋大夫吕锜射龚王,中其目,故曰“伤”。

(2) 酒器受三升曰黍。

【校】梁伯子云:“《内》、《外传》,《韩子》《十过》、《饰邪》二篇,《汉书人表》,并是‘穀阳’,而《史记·晋楚世家》、《淮南·人间训》、《说苑·敬慎》篇与此并倒作‘阳穀’。”案:黍酒是酿黍所成者,《说文》“酏,黍酒也”,注非。《十过》篇作“觞酒”,《饰邪》篇作“卮酒”。

(3) 【校】《韩非》作“嘻”。

(4) 【校】《说苑》作“酒也”,是。

(5) 绝,止也。

(6) 【校】《饰邪》篇作“而谋事”。

(7) 幄,帐也。

(8) 【校】《十过》篇作“不以仇子反也”,《饰邪》篇作“非以端恶子反也”,《说苑》作“非以妒子反也”,皆较“醉”字胜。

(9) 忠,爱也。

昔者晋献公使荀息假道于虞以伐虢。荀息曰:“请以垂棘之璧与屈产之乘,以赂虞公,而求假道焉,必可得也。” (1) 献公曰:“夫垂棘之璧,吾先君之宝也;屈产之乘,寡人之骏也。若受吾币而不吾假道,将奈何?”荀息曰:“不然。彼若不吾假道,必不吾受也; (2) 若受我而假我道,是犹取之内府而藏之外府也,犹取之内皂而著之外皂也。 (3) 君奚患焉?” (4) 献公许之。乃使荀息以屈产之乘为庭实, (5) 而加以垂棘之璧,以假道于虞而伐虢。虞公滥于宝与马而欲许之。 (6) 宫之奇谏曰:“不可许也。虞之与虢也,若车之有辅也。车依辅,辅亦依车,虞、虢之势是也。 (7) 先人有言曰:‘唇竭而齿寒。’ (8) 夫虢之不亡也恃虞,虞之不亡也亦恃虢也。若假之道,则虢朝亡而虞夕从之矣。奈何其假之道也?”虞公弗听,而假之道。荀息伐虢,克之。还反伐虞,又克之。荀息操璧牵马而报。 (9) 献公喜曰:“璧则犹是也,马齿亦薄长矣。”故曰:小利,大利之残也。 (10)

(1) 垂棘,美璧所出之地,因以为名也。屈产之乘,屈邑所生,四马曰乘。今河东北屈骏马者是也。

(2) 【校】旧校云:“一作‘必不敢受也’。”

(3) 皂,枥也。

(4) 患犹难也。

(5) 为虞庭中之实。

(6) 滥,贪。

(7) 车,牙也。辅,颊也。车、辅相依凭得以近喻也。

(8) 竭,亡也。

【校】梁伯子云:“案《左传》‘唇亡齿寒’之语,《战国》《齐》、《赵策》俱引之,而《韩策》作‘唇揭齿寒’,注:‘揭犹反也。’‘揭’字似胜‘亡’字。《庄子·胠箧》篇作‘唇竭’,此与《淮南·说林训》亦并作‘竭’。疑皆因‘揭’而误也。”

(9) 报,白也。

(10) 残,害也。

中山之国有 繇者,智伯欲攻之而无道也, (1) 为铸大钟,方车二轨以遗之。 繇之君将斩岸堙溪以迎钟。赤章蔓枝谏曰:“诗云‘唯则定国’, (2) 我胡以得是于智伯? (3) 夫智伯之为人也,贪而无信,必欲攻我而无道也,故为大钟,方车二轨以遗君。君因斩岸堙溪以迎钟,师必随之。”弗听。有顷谏之,君曰:“大国为欢,而子逆之,不祥。子释之。” (4) 赤章蔓枝曰:“为人臣不忠贞,罪也;忠贞不用,远身可也。”断毂而行, (5) 至卫七日而 繇亡。 (6) 欲钟之心胜也,欲钟之心胜则安 繇之说塞矣。 (7) 凡听说,所胜不可不审也,故太上先胜。 (8)

(1) 繇,国之近晋者也,或作“仇酋”。智伯,晋大夫智襄子瑶也。

【校】“ ”,旧本作“夙”。何屺瞻云:“当作‘ ’。”梁仲子云:“《韩非·说林下》作‘仇由’,《战国·西周策》作‘厹由’,《史记·樗里子传》作‘仇犹’,《索隐》云‘高诱注《国策》以仇犹为厹由’,《说文系传》口部‘叴’云‘《吕氏春秋》有叴犹国,智伯欲伐者也’。”

(2) 【校】《左氏僖四年传》公孙支对秦穆公曰“臣闻之,唯则定国”,下两引《诗》,则知此语是逸诗也。

(3) 赤章蔓枝, 繇之臣也。

【校】“我胡”下旧有“则”字,因上文而衍,今删去。

(4) 释,置。

(5) 山中道狭,故断车毂而行去。

(6) 智伯灭之。

【校】《韩非》作“至于齐七月而仇由亡矣”。

(7) 塞,不行也。

(8) 先犹上也。

昌国君将五国之兵以攻齐。 (1) 齐使触子将,以迎天下之兵于济上。 (2) 齐王欲战,使人赴触子,耻而訾之曰:“不战,必刬若类,掘若垄!” (3) 触子苦之, (4) 欲齐军之败,于是以天下兵战,战合,击金而却之, (5) 卒北。 (6) 天下兵乘之, (7) 触子因以一乘去,莫知其所,不闻其声。 (8) 达子又帅其余卒, (9) 以军于秦周,无以赏,使人请金于齐王。 (10) 齐王怒曰:“若残竖子之类, (11) 恶能给若金?” (12) 与燕人战,大败,达子死,齐王走莒。 (13) 燕人逐北入国,相与争金于美唐甚多。 (14) 此贪于小利以失大利者也。 (15)

(1) 昌国君,乐毅也,为燕昭王将伐齐。五国,谓燕、秦、韩、魏、赵也。

【校】梁伯子云:“时攻齐者尚有楚,高氏因本文五国,故不数楚,然非也。”

(2) 济,水也。

(3) 刬,灭也。若,汝也。垄,冢也。言不堪敌而战克破燕军,必刬灭汝种类,平掘汝先人之冢也。

(4) 苦,病。

(5) 【校】旧校云:“‘却’一作‘退’。”

(6) 北,走也。

(7) 乘犹胜也。

【校】案:乘犹陵也。

(8) 【校】旧校云:“一作‘问’。”

(9) 达子,齐人也。帅,将也。

(10) 军,屯也。秦周,齐城门名也。请金,将以赏有功也。

(11) 残,余也。竖子,谓达子也。

(12) 恶,安也。给,与也。

(13) 走,奔也。莒,邑也。

(14) 美唐,金藏所在。

(15) 小利,金也。大利,国也。言湣王贪金不给达子以失国,乃大惑者也。





下贤


三曰:

有道之士固骄人主。人主之不肖者亦骄有道之士,日以相骄,奚时相得?若儒、墨之议与齐、荆之服矣。贤主则不然,士虽骄之,而己愈礼之,士安得不归之?士所归,天下从之帝。 (1) 帝也者,天下之适也; (2) 王也者,天下之往也。得道之人,贵为天子而不骄倨, (3) 富有天下而不骋夸, (4) 卑为布衣而不瘁摄, (5) 贫无衣食而不忧慑, (6) 豤乎其诚自有也, (7) 觉乎其不疑有以也, (8) 桀乎其必不渝移也, (9) 循乎其与阴阳化也,悤悤乎其心之坚固也, (10) 空空乎其不为巧故也, (11) 迷乎其志气之远也, (12) 昏乎其深而不测也, (13) 确乎其节之不庳也,就就乎其不肯自是, (14) 鹄乎其羞用智虑也, (15) 假乎其轻俗诽誉也, (16) 以天为法,以德为行,以道为宗, (17) 与物变化而无所终穷, (18) 精充天地而不竭, (19) 神覆宇宙而无望, (20) 莫知其始,莫知其终,莫知其门,莫知其端,莫知其源, (21) 其大无外,其小无内,此之谓至贵。 (22) 士有若此者,五帝弗得而友,三王弗得而师,去其帝王之色,则近可得之矣。 (23)

(1) 句。

(2) 适,主也。

(3) 倨,傲也。

(4) 夸,诧而自大也。

(5) 瘁,病也。摄犹屈也。

(6) 慑,惧也。

(7) 自有,有道。

【校】“豤”即“恳”字,旧本作“狠”,讹,今改正。

(8) 《诗》云:“何其久也,必有以也。”

(9) 桀,特也。渝,变也。移,易也。

(10) 悤悤,明貌。

(11) 空空,悫也。巧故,伪诈。

(12) 志在江海之上。

(13) 测,尽也。言深不可尽。

【校】正文“也”字旧脱,案当有。孙云:“李善注《文选》曹子建《杂诗》引‘风乎其高无极也’,疑此处脱文。”

(14) 就就,读如由与之与。

【校】注“由与”即“犹豫”。案《尔雅·释兽》释文“犹,羊周、羊救二反”,《字林》“弋又反”,此就字读从之也。

(15) 鹄,读如浩浩昊天之浩,大也。

(16) 皆谓体道之人也。

(17) 宗,本也。

(18) 穷,极也。

(19) 充,实。竭,尽。

(20) 四方上下曰宇,以屋喻天地也。往古来今曰宙,言其神而包覆之。无望,无界畔也。

(21) 道不可得知也。

(22) 道在大能大,故无复有外;在小能小,故无复有内。道所贵之也。

(23) 去犹除也。除其尊宠盈满之色,则近得师友矣。

【校】旧校云:“‘可’一作‘于’。”

尧不以帝见善绻,北面而问焉。 (1) 尧,天子也;善绻,布衣也。何故礼之若此其甚也?善绻,得道之士也。得道之人,不可骄也。 (2) 尧论其德行达智而弗若, (3) 故北面而问焉。此之谓至公。非至公其孰能礼贤? (4)

(1) 善绻,有道之士也。尧不敢以自尊,北面而问焉。

【校】“善绻”,《庄子》作“善卷”。

(2) 人轻道重也。

(3) 若,如也。

(4) 孰,谁也。

周公旦,文王之子也,武王之弟也,成王之叔父也,所朝于穷巷之中、瓮牖之下者七十人。 (1) 文王造之而未遂, (2) 武王遂之而未成,周公旦抱少主而成之, (3) 故曰成王,不唯以身下士邪?

(1) 瓮牖,以破瓮蔽牖,言贫陋也。

(2) 造,始也。遂,成也。

(3) 抱,奉。

齐桓公见小臣稷,一日三至弗得见。 (1) 从者曰:“万乘之主见布衣之士,一日三至而弗得见,亦可以止矣。” (2) 桓公曰:“不然。士骜禄爵者,固轻其主; (3) 其主骜霸王者,亦轻其士。纵夫子骜禄爵,吾庸敢骜霸王乎?” (4) 遂见之,不可止。 (5) 世多举桓公之内行,内行虽不修,霸亦可矣。 (6) 诚行之此论,而内行修,王犹少。 (7)

(1) 稷不见之也。

(2) 止,休也。

(3) 骜亦轻也。

(4) 庸,用也。

(5) 【校】《新序·杂事五》作“五往而后得见”,《韩非·难》一作“于是五往乃得见之”。

(6) 霸功大,亦可以灭内行之阙也。

(7) 犹,尚也。

子产相郑, (1) 往见壶丘子林,与其弟子坐必以年,是倚其相于门也。 (2) 夫相万乘之国而能遗之, (3) 谋志论行,而以心与人相索, (4) 其唯子产乎? (5) 故相郑十八年,刑三人,杀二人,桃李之垂于行者莫之援也, (6) 锥刀之遗于道者莫之举也。 (7)

(1) 郑大夫子国之子公孙乔也。

【校】《左传》作“侨”。

(2) 年,齿也。子产,壶丘子弟子。坐以齿长少相亚,不以尊位而上之,倚置其相之宠于壶丘之门外,不以加于坐也,故曰“倚其相于门”也。

(3) 遗犹舍也。郑国北迫晋,南近楚,爵则伯也,赋千乘耳,而云万乘,复妄言也。

【校】注“遗犹舍也”,旧作“全也”,讹,今改正。

(4) 索,尽也。孔子曰:“子产有君子之道四焉:其行己也恭,其事上也敬,其养民也惠,其使民也义。”推其志行,以忠心与人相极尽,知其情实。一曰:索,法。与人为法则。

(5) 唯,独也。

(6) 援,攀也。

(7) 举犹取也。

魏文侯见段干木,立倦而不敢息, (1) 反见翟黄,踞于堂而与之言。 (2) 翟黄不说。 (3) 文侯曰:“段干木官之则不肯,禄之则不受。今女欲官则相位,欲禄则上卿,既受吾实, (4) 又责吾礼,无乃难乎?”故贤主之畜人也,不肯受实者其礼之。 (5) 礼士莫高于节欲,欲节则令行矣,文侯可谓好礼士矣。好礼士,故南胜荆于连堤,东胜齐于长城,虏齐侯,献诸天子,天子赏文侯以上闻。 (6)

(1) 倦,罢也。

(2) 反,从干木所还也。

(3) 以文侯敬干木而慢己也。

(4) 实犹爵禄也。

(5) 礼,敬也。

(6) 文侯,毕公高之后,与周同姓,魏桓子之孙,始立为侯。文,谥也。

【校】梁伯子云:“《国策》、《史记》皆不见文侯胜荆、齐之事。”“上闻”,旧本作“上卿”,讹。案《史》、《汉》《樊哙传》“上闻爵”,如淳注引此语作“上闻”,张晏曰“得径上闻也”,晋灼曰“名通于天子也”。今《史记》多讹为“上閒”,唯《索隐》本是“上闻”,又引此作“上閒”云“閒音中间”,恐讹也。





报更


四曰:

国虽小,其食足以食天下之贤者,其车足以乘天下之贤者,其财足以礼天下之贤者。与天下之贤者为徒, (1) 此文王之所以王也。 (2) 今虽未能王,其以为安也,不亦易乎? (3) 此赵宣孟之所以免也, (4) 周昭文君之所以显也, (5) 孟尝君之所以却荆兵也。 (6) 古之大立功名与安国免身者,其道无他,其必此之由也。 (7) 堪士不可以骄恣屈也。 (8)

(1) 徒,党也。

(2) 《诗》云“济济多士,文王以宁”,此之谓也。

(3) 立王功大,保安其国差小,故曰“不亦易”。

(4) 宣孟,晋卿赵盾也,履行仁义,束脯以食翳桑之饿人,以免灵公伏甲之难。

(5) 昭文君,周后所分立东周君也,宾礼张仪,欲与分国。张仪重之于秦,秦尊奉之,故曰“所以显也”。

【校】注“重之”,旧作“胜之”,讹,今案下文改正。

(6) 孟尝君,齐公子田婴之子田文也,下士礼贤,养客三千人,行仁义而强,故荆兵却偃,不敢攻之也。

(7) 古立功名安国免身无咎殃者,皆以此仁义之道也。

(8) 堪,乐也。乐士当以礼卑谦,若魏公子之虚己,故不可以骄恣屈而有之也。

【校】孙云:“‘堪士’疑是‘湛士’。”旧校云:“‘屈’一作‘有’。”

昔赵宣孟将上之绛,见骩桑之下, (1) 有饿人卧不能起者。宣孟止车,为之下食,蠲而 之,再咽而后能视。宣孟问之曰:“女何为而饿若是?”对曰:“臣宦于绛,归而粮绝,羞行乞而憎自取,故至于此。” (2) 宣孟与脯二朐, (3) 拜受而弗敢食也。问其故,对曰:“臣有老母,将以遗之。” (4) 宣孟曰:“斯食之,吾更与女。” (5) 乃复赐之脯二束与钱百,而遂去之。处二年,晋灵公欲杀宣孟,伏士于房中以待之,因发酒于宣孟。 (6) 宣孟知之,中饮而出。灵公令房中之士疾追而杀之。一人追疾,先及宣孟,之面曰:“嘻!君舆, (7) 吾请为君反死。” (8) 宣孟曰:“而名为谁?” (9) 反走对曰:“何以名为?臣骩桑下之饿人也。”还斗而死。 (10) 宣孟遂活。此书之所谓“德几无小”者也。 (11) 宣孟德一士,犹活其身,而况德万人乎?故诗曰“赳赳武夫,公侯干城”, (12) “济济多士,文王以宁”。 (13) 人主胡可以不务哀士? (14) 士其难知,唯博之为可, (15) 博则无所遁矣。 (16)

(1) 【校】《后汉书·赵壹传》注云:“‘骩’,古‘委’字。”《淮南·人间训》作“委桑”,《左传》作“翳桑”。

(2) 羞于行乞,自憎至此也。

【校】注谬。憎自取,言憎恶径自取之,亦不肯也。

(3) 【校】旧本作“一朐”,案《北堂书钞》百四十五、《初学记》二十六及《赵壹传》注俱是“二朐”,今据改正。

(4) 【校】《御览》八百三十六“将”作“请持”二字,《初学记》、《后汉书》注“将”亦作“持”。

(5) 斯犹尽也。

【校】《诗·大雅·皇矣》篇“王赫斯怒”,郑笺云:“斯,尽也。”《释文》:“郑读斯为赐。”

(6) 发犹致也。

(7) 舆,车也。教宣孟使就车也。

(8) 反,还也。

(9) 而,汝也。

(10) 【校】梁伯子云:“桑下饿人是灵辄,斗死者是提弥明,此误合二人为一。《史记·晋世家》亦同此误,《索隐》言之矣。《水经注》四亦误。”

(11) 【校】案:《墨子·明鬼》篇“禽艾之道之曰:‘得玑无小,灭宗无大。’”翟氏灏谓《逸周书·世俘解》有禽艾侯之语,当即此禽艾,但二语尚未见所出。此德几无小,犹所谓惠不期多寡,期于当阨云耳。未知禽艾之言,意相同否?“得”与“德”古字通用。

(12) 此《周南》之风《兔罝》之首章也,言其贤可为公侯扞难其城藩也,以喻骩桑下之人扞赵盾之难也。

(13) 此《大雅·文王》之三章也。文王以多士而造周,赵盾以桑下之人去患也。

【校】注首九字旧本多缺,依朱本补。又“造周”二字亦脱,今案文义补。

(14) 哀,爱也。

(15) 博,广也。

(16) 遁,失也。

张仪,魏氏余子也。 (1) 将西游于秦,过东周。客有语之于昭文君者曰:“魏氏人张仪,材士也, (2) 将西游于秦,愿君之礼貌之也。”昭文君见而谓之曰:“闻客之秦,寡人之国小,不足以留客。虽游,然岂必遇哉?客或不遇, (3) 请为寡人而一归也。国虽小,请与客共之。”张仪还走,北面再拜。 (4) 张仪行, (5) 昭文君送而资之。至于秦,留有间,惠王说而相之。 (6) 张仪所德于天下者,无若昭文君。 (7) 周,千乘也,重过万乘也, (8) 令秦惠王师之。 (9) 逢泽之会,魏王尝为御,韩王为右, (10) 名号至今不忘,此张仪之力也。

(1) 大夫庶子为余,受氏为张。

(2) 【校】孙云:“《文选》袁阳源诗‘荆魏多壮士’,李善注引此作‘壮士’,《御览》四百七十五同。”

(3) 【校】旧校云:“‘或’一作‘訾’。”訾犹叹也。

(4) 拜昭文君之言也。

(5) 行,去也。

(6) 惠王,孝公之子,始称王也。说张仪而相之。

(7) 德犹恩也。

(8) 张仪重之。

(9) 师昭文君。

(10) 秦会诸侯于逢泽,魏王为昭文君御,韩王为之右也。

孟尝君前在于薛,荆人攻之。淳于髡为齐使于荆,还反,过于薛,孟尝君令人礼貌而亲郊送之, (1) 谓淳于髡曰:“荆人攻薛,夫子弗为忧,文无以复侍矣。” (2) 淳于髡曰:“敬闻命矣。”至于齐,毕报, (3) 王曰:“何见于荆?”对曰:“荆甚固, (4) 而薛亦不量其力。”王曰:“何谓也?”对曰:“薛不量其力,而为先王立清庙。荆固而攻薛,薛清庙必危, (5) 故曰薛不量其力,而荆亦甚固。”齐王知颜色, (6) 曰:“嘻!先君之庙在焉。”疾举兵救之,由是薛遂全。颠蹶之请,坐拜之谒, (7) 虽得则薄矣。 (8) 故善说者,陈其势,言其方,见人之急也,若自在危厄之中, (9) 岂用强力哉?强力则鄙矣。说之不听也,任不独在所说,亦在说者。

(1) 【校】《齐策》“礼貌”作“体貌”。

(2) 文,孟尝名也。侍,侍见也。

【校】“侍”,旧作“待”,讹,今从《齐策》改,注同。

(3) 反命毕也。

(4) 固,护,以侵兼人。

(5) 【校】衍下“薛”字。《齐策》作“荆固而攻之,清庙必危”。

(6) 齐王,宣王也,威王之子。知犹发也。

【校】《齐策》作“和其颜色”。

(7) 【校】“坐拜”,《策》作“望拜”。

(8) 薄,轻少也。

【校】“得”,旧讹作“薄”,今从《策》改正。

(9) 【校】“危厄”,《策》作“隘窘”。





顺说


五曰:

善说者若巧士,因人之力以自为力,因其来而与来,因其往而与往, (1) 不设形象,与生与长,而言之与响,与盛与衰,以之所归; (2) 力虽多,材虽劲, (3) 以制其命。顺风而呼,声不加疾也; (4) 际高而望,目不加明也。所因便也。

(1) 与犹助也。

(2) 归,终也。

(3) 劲,强也。

(4) 加,益也。

惠盎见宋康王,康王蹀足謦欬, (1) 疾言曰:“寡人之所说者,勇有力也,不说为仁义者。 (2) 客将何以教寡人?”惠盎对曰:“臣有道于此, (3) 使人虽勇,刺之不入;虽有力,击之弗中。大王独无意邪?” (4) 王曰:“善!此寡人所欲闻也。”惠盎曰:“夫刺之不入,击之不中,此犹辱也。臣有道于此,使人虽有勇弗敢刺,虽有力不敢击。大王独无意邪?”王曰:“善!此寡人之所欲知也。”惠盎曰:“夫不敢刺,不敢击,非无其志也。臣有道于此,使人本无其志也。 (5) 大王独无意邪?”王曰:“善!此寡人之所愿也。”惠盎曰:“夫无其志也,未有爱利之心也。臣有道于此,使天下丈夫女子莫不欢然皆欲爱利之。此其贤于勇有力也, (6) 居四累之上。大王独无意邪?” (7) 王曰:“此寡人之所欲得。” (8) 惠盎对曰:“孔、墨是也。 (9) 孔丘、墨翟,无地为君, (10) 无官为长。 (11) 天下丈夫女子莫不延颈举踵,而愿安利之。 (12) 今大王,万乘之主也,诚有其志, (13) 则四境之内皆得其利,其贤于孔、墨也远矣。” (14) 宋王无以应。 (15) 惠盎趋而出,宋王谓左右曰:“辨矣!客之以说服寡人也。”宋王,俗主也,而心犹可服,因矣。 (16) 因则贫贱可以胜富贵矣,小弱可以制强大矣。 (17)

(1) 【校】旧本讹作“惠盎见宋康成公而谓足声速”,今据《列子·黄帝》篇、《淮南·道应训》及李善注《文选》谢惠连《咏牛女》诗所引改正。

(2) 惠盎者,宋人,惠施族也。康王,宋昭公曾孙辟公之子,名侵,立十一年,僭号称王,四十五年,大为不道,故曰宋子不足仁义者也。齐湣王伐灭之。

【校】正文“也不说”三字,旧本作“而无”,今从《列子》、《淮南》改。梁伯子云:“注‘名侵’当是‘偃’字之讹,‘四十五年’与《禁塞篇》注‘四十七年’又异,其实六十一年也。”

(3) 有道于此,勇有力者也。

(4) 不可入,不可中,如此者,大王独无意欲之邪。

(5) 本无有击刺之志也。

(6) 言以仁义之德,使民皆欲爱利之也,故贤于勇有力。

(7) 四累,谓卿、大夫、士及民四等也。君处四分之上,故曰“四累之上”,喻尊高也。临下以德,则下爱利之矣。大王意独无欲之邪?

【校】四累,即指上所言层累而上凡四等,注非是,而张湛注《列子》亦与之同。

(8) 欲得人爱利也。

【校】正文句末,《列子》、《淮南》皆有“也”字。

(9) 言当为孔丘、墨翟之德,则得所欲也,故曰“是也”,当法则之也。

(10) 以德见尊。

(11) 以道见敬。

(12) 延颈,引领也。举踵,企望之也。愿其尊高安而利也。

(13) 有孔、墨之志。

(14) 得贤名过于孔、墨。远犹多也。

(15) 应,答也。

(16) 因犹便也。

(17) 惠盎是也。

田赞衣补衣而见荆王, (1) 荆王曰:“先生之衣,何其恶也?”田赞对曰:“衣又有恶于此者也。”荆王曰:“可得而闻乎?”对曰:“甲恶于此。” (2) 王曰:“何谓也?”对曰:“冬日则寒,夏日则暑,衣无恶乎甲者。赞也贫,故衣恶也。 (3) 今大王,万乘之主也,富贵无敌,而好衣民以甲,臣弗得也。 (4) 意者为其义邪?甲之事,兵之事也,刈人之颈,刳人之腹,隳人之城郭,刑人之父子也。 (5) 其名又甚不荣。 (6) 意者为其实邪?苟虑害人,人亦必虑害之; (7) 苟虑危人,人亦必虑危之。其实人则甚不安。之 (8) 二者,臣为大王无取焉。” (9) 荆王无以应。说虽未大行,田赞可谓能立其方矣。 (10) 若夫偃息之义,则未之识也。 (11)

(1) 田赞,齐人也。补衣,弊衣也。

(2) 甲,铠也。此,恶衣也。

(3) 【校】《御览》三百五十六引叠一“贫”字。

(4) 得犹取也。

(5) 隳,坏也。刑,杀也。

(6) 兵杀人,以逆名,不得为荣。

(7) 不得财宝也。为财利广出,苟谋害人,人亦必谋害之。《传》曰“晋侯诬人,人亦诬之”,其此之谓也。

(8) 其为事如此,甚不得安也。

【校】旧校云:“‘人则’一作‘久则’。”

(9) 二者,害与危。臣为大王计,无取此二者也。

(10) 方,道也。

(11) 段干木偃息以安魏,田赞辩说以服荆,比之偃息,故曰未知谁贤之也。

管子得于鲁,鲁束缚而槛之,使役人载而送之齐,其讴歌而引。 (1) 管子恐鲁之止而杀己也,欲速至齐,因谓役人曰:“我为汝唱,汝为我和。”其所唱适宜走,役人不倦,而取道甚速。管子可谓能因矣。 (2) 役人得其所欲,己亦得其所欲,以此术也。 (3) 是用万乘之国,其霸犹少,桓公则难与往也。 (4)

(1) 役人皆讴歌而挽其车以送之也。

【校】《意林》作“皆讴歌而引车”,《御览》五百七十一同。

(2) 因役人用势欲走,而为唱歌欢之令走也。

【校】注“欢之”疑当作“劝之”。

(3) 以,用。此术,道也。

(4) 往,王也。言其难与致于王也。





不广


六曰:

智者之举事必因时。时不可必成, (1) 其人事则不广, (2) 成亦可,不成亦可。以其所能托其所不能,若舟之与车。 (3) 北方有兽,名曰蹶, (4) 鼠前而兔后,趋则跲,走则颠,常为蛩蛩距虚取甘草以与之。 (5) 蹶有患害也,蛩蛩距虚必负而走。此以其所能托其所不能。 (6)

(1) 必成犹必得也。

(2) 广,博也。

(3) 舟不能陆,车不能浮,然更相载,故曰“以其所能托其所不能”也。

(4) 【校】《说苑·复恩》篇作“蟨”,《尔雅注》同。《淮南·道应训》作“蹷”。

(5) 【校】《尔雅》作“岠虚”,《说苑》作“巨虚”,《淮南》作“ ”。

(6) 托,寄也。

鲍叔、管仲、召忽三人相善,欲相与定齐国,以公子纠为必立。召忽曰:“吾三人者于齐国也,譬之若鼎之有足,去一焉则不成。且小白则必不立矣, (1) 不若三人佐公子纠也。”管仲曰:“不可。夫国人恶公子纠之母,以及公子纠;公子小白无母,而国人怜之。事未可知,不若令一人事公子小白。夫有齐国,必此二公子也。” (2) 故令鲍叔傅公子小白,管子、召忽居公子纠所。公子纠外物则固难必。 (3) 虽然,管子之虑近之矣。 (4) 若是而犹不全也,其天邪!人事则尽之矣。

(1) 小白,齐桓公名。

(2) 二公子,齐僖公之子,襄公之弟也。

(3) 物,事也。纠在外,不可谓必得主,故曰“固难必”。

(4) 虑,谋也。

齐攻廪丘。赵使孔青将死士而救之,与齐人战,大败之。齐将死。得车二千,得尸三万,以为二京。 (1) 甯越谓孔青曰:“惜矣!不如归尸以内攻之。 (2) 越闻之,古善战者,莎随贲服。 (3) 却舍延尸, (4) 彼得尸而财费乏, (5) 车甲尽于战,府库尽于葬,此之谓内攻之。” (6) 孔青曰:“敌齐不尸则如何?” (7) 甯越曰:“战而不胜,其罪一;与人出而不与人入,其罪二;与之尸而弗取,其罪三。民以此三者怨上, (8) 上无以使下,下无以事上,是之谓重攻之。”甯越可谓知用文武矣。用武则以力胜,用文则以德胜。文武尽胜,何敌之不服! (9)

(1) 古者军伐克败,于其所获尸,合土葬之,以为京观,故孔青欲以齐尸为二京也。

(2) 甯越,赵之中牟人也。言不如归尸于齐,齐人必怨,其将使葬送以尽其财,是所以内攻之也。

【校】梁仲子云:“《孔丛·论势》篇以归尸为子顺语,余亦小同大异。”

(3) 莎随犹相守,不进不却。贲,置也。服,退也。

(4) 军行三十里为一舍。却舍以缓其尸,使齐人得收之。

(5) 【校】七字旧本讹在上句中,又“乏”作“之”,今依孙校改正。

(6) 齐人战败,尽其车甲。府库,财所藏也,葬死者以尽之,令其贫穷且相怨,此所谓内攻之术也。

(7) 言与齐为敌,不收其尸为京则如何?

【校】注谬甚。敌齐,指齐人为敌人也。我缓之使得收,而彼不收,将如之何?下文甚明,何以妄说?

(8) 【校】旧校云:“‘怨’一作‘罪’。”

(9) 能尽服之。

晋文公欲合诸侯。咎犯曰:“不可。天下未知君之义也。”公曰:“何若?”咎犯曰:“天子避叔带之难,出居于郑。君奚不纳之,以定大义,且以树誉。” (1) 文公曰:“吾其能乎?”咎犯曰:“事若能成,继文之业,定武之功,辟土安疆,于此乎在矣。事若不成,补周室之阙,勤天子之难, (2) 成教垂名,于此乎在矣。 (3) 君其勿疑。”文公听之,遂与草中之戎、 (4) 骊土之翟,定天子于成周。 (5) 于是天子赐之南阳之地, (6) 遂霸诸侯。举事义且利,以立大功,文公可谓智矣,此咎犯之谋也。出亡十七年,反国四年而霸,其听皆如咎犯者邪!

(1) 树,立也。

(2) 勤,忧也。

(3) 成仁义之教,勤天子之名,以示诸侯,于此在矣。

(4) 【校】旧校云:“‘与’一作‘兴’。”

(5) 天子,周襄王也。避母弟叔带之难,出奔在郑,晋文纳之于成周,故曰“定”也。成周,今洛阳也。

(6) 襄王赐之南阳之地,在河之北,晋之山南,故言南阳,今河内阳樊、温之属是也。

管子、鲍叔佐齐桓公举事, (1) 齐之东鄙人有常致苦者。管子死,竖刀、易牙用,国之人常致不苦,不知致苦,卒为齐国良工,泽及子孙。知大礼,知大礼虽不知国可也。 (2)

(1) 举犹用也。

(2) 礼,国之本。君子务本,本立而道生,故曰“不知国可也”。





贵因


七曰:

三代所宝莫如因,因则无敌。禹通三江五湖,决伊阙,沟回陆,注之东海,因水之力也。 (1) 舜一徙成邑,再徙成都,三徙成国, (2) 而尧授之禅位,因人之心也。 (3) 汤、武以千乘制夏、商,因民之欲也。 (4) 如秦者立而至,有车也; (5) 适越者坐而至,有舟也。 (6) 秦、越,远涂也,竫立安坐而至者,因其械也。 (7)

(1) 回,通也。

(2) 《周礼》“四井为邑”,邑方二里也;“四县为都”,都方二十二里也。邑有封,都有成,然则邑小都大。《传》曰:“都城过百雉,国之害也。”成国,成千乘之国也。

(3) 授之禅位,与之天下也。人皆喜之,故曰“因人之心”也。

(4) 《传》曰“众曹所好,鲜其不济”,汤、武是也;“众曹所恶,鲜其不败”,桀、纣是也;故曰“因民之欲”也。

【校】案:《周语》下泠州鸠对周景王曰:“民所曹好,鲜其不济也;其所曹恶,鲜其不废也。”

(5) 立犹行也。车行陆而至也。

【校】古者车皆立乘,故云“立”,与下“坐”字对文。注非也。

(6) 适,之也。

(7) 竫,正也。械,器也。

武王使人候殷, (1) 反报岐周曰:“殷其乱矣!”武王曰:“其乱焉至?”对曰:“谗慝胜良。” (2) 武王曰:“尚未也。”又复往,反报曰:“其乱加矣!”武王曰:“焉至?”对曰:“贤者出走矣。” (3) 武王曰:“尚未也。”又往,反报曰:“其乱甚矣!”武王曰:“焉至?”对曰:“百姓不敢诽怨矣。” (4) 武王曰:“嘻!”遽告太公。 (5) 太公对曰:“谗慝胜良,命曰戮; (6) 贤者出走,命曰崩; (7) 百姓不敢诽怨,命曰刑胜。 (8) 其乱至矣,不可以驾矣。” (9) 故选车三百,虎贲三千,朝要甲子之期,而纣为禽, (10) 则武王固知其无与为敌也。因其所用,何敌之有矣!

(1) 候,视也。

(2) 谗,邪也,慝,恶也,而皆进用之,忠良黜远之,故曰“胜良”也。

(3) 谓箕子奔朝鲜。

(4) 言百姓畏纣无道刑戮之诛,皆闭口无诽怨之言。

(5) 遽,疾。

(6) 戮,暴也。

(7) 崩,坏也。

(8) 《传》曰“厉王虐,国人谤王,王使卫巫监谤者,得而杀之,乃不敢言而道路以目”,刑辟胜也。

(9) 驾,加也。

(10) 朝,早朝也。与诸侯要期甲子之日也。

武王至鲔水,殷使胶鬲候周师,武王见之。胶鬲曰:“西伯将何之?无欺我也!”武王曰:“不子欺,将之殷也。”胶鬲曰:“朅至?” (1) 武王曰:“将以甲子至殷郊,子以是报矣!” (2) 胶鬲行。天雨,日夜不休, (3) 武王疾行不辍。 (4) 军师皆谏曰:“卒病,请休之。” (5) 武王曰:“吾已令胶鬲以甲子之期报其主矣。今甲子不至,是令胶鬲不信也。胶鬲不信也,其主必杀之。吾疾行,以救胶鬲之死也。”武王果以甲子至殷郊,殷已先陈矣。至殷,因战,大克之。此武王之义也。人为人之所欲,己为人之所恶,先陈何益? (6) 适令武王不耕而获。 (7)

(1) 朅,何也。言以何日来至殷也。

(2) 报,白也。

(3) 行犹还也。不休止降雨,天地和同也,武王所以克纣也。

(4) 辍,止也。

(5) 休,息也。

(6) 人,谓武王也。人之所欲,天必从之,顺天诛也。己,谓纣也。人之所恶,天必坏之,所坏不可支,故曰“先陈何益”。

(7) 不耕而获,不战而克也。故《孙子》曰:“不战而屈人之兵,善之善者也。”此之谓也。

武王入殷,闻殷有长者,武王往见之,而问殷之所以亡。殷长者对曰:“王欲知之,则请以日中为期。”武王与周公旦明日早要期,则弗得也。武王怪之,周公曰:“吾已知之矣。此君子也,取不能其主,有以其恶告王,不忍为也。若夫期而不当,言而不信,此殷之所以亡也,已以此告王矣。”

夫审天者,察列星而知四时,因也; (1) 推历者,视月行而知晦朔,因也;禹之裸国,裸入衣出, (2) 因也;墨子见荆王,锦衣吹笙,因也; (3) 孔子道弥子瑕见釐夫人,因也; (4) 汤、武遭乱世,临苦民,扬其义,成其功,因也。故因则功,专则拙。 (5) 因者无敌, (6) 国虽大,民虽众,何益? (7)

(1) 【校】旧校云:“一本此句下有‘动作因日光而治万事,因也’十一字。”案此浅陋,必非本文。

(2) 【校】旧校云:“一本作‘入衣出否’。”

(3) 墨子好俭非乐,锦与笙非其所服也,而为之,因荆王之所欲也。

(4) 弥子瑕,卫灵公之幸臣也,孔子因之欲见灵公夫人南子,《论语》云“子见南子,子路不悦。夫子矢之曰‘予所不者,天厌之,天厌之’”是也。此釐夫人,未之闻;或云为谥。《谥法》“小心畏忌曰釐”。若南子淫佚,与宋朝通;太子蒯聩过宋野,野人歌之曰“既定尔娄猪,盍归我艾豭”。推此言之,不得谥为釐明矣。

【校】梁仲子云:“《淮南·泰族训》云‘孔子欲行王道,东西南北七十说而无所偶,故因卫夫人、弥子瑕而欲通其道’,语义政合,此似有脱误,然此皆战国时人所为也。”注“过宋野”,旧作“于野”,讹,今依《左传》改正。

(5) 因则成,故曰“功”。专则败,故曰“拙”。

(6) 因民之欲,道以义,故无与之敌者,汤、武是也。

【校】注“道”旧作“遵”,上文“道弥子瑕”,旧校云“‘道’一作‘遵’”,案皆讹,今改作“道”。

(7) 民虽众多,不能使之不亡,故曰“何益”,桀、纣是也。





察今


八曰:

上胡不法先王之法,非不贤也,为其不可得而法。 (1) 先王之法,经乎上世而来者也,人或益之,人或损之,胡可得而法?虽人弗损益,犹若不可得而法。东夏之命,古今之法,言异而典殊。 (2) 故古之命多不通乎今之言者,今之法多不合乎古之法者。 (3) 殊俗之民,有似于此。其所为欲同,其所为异。 (4) 口惽之命不愉,若舟车衣冠滋味声色之不同,人以自是,反以相诽。天下之学者多辩,言利辞倒,不求其实,务以相毁,以胜为故。 (5) 先王之法,胡可得而法?虽可得,犹若不可法。凡先王之法,有要于时也,时不与法俱至,法虽今而至,犹若不可法。故择先王之成法,而法其所以为法。 (6) 先王之所以为法者何也?先王之所以为法者人也;而己亦人也,故察己则可以知人,察今则可以知古,古今一也,人与我同耳。有道之士,贵以近知远,以今知古,以益所见知所不见。 (7) 故审堂下之阴, (8) 而知日月之行,阴阳之变;见瓶水之冰,而知天下之寒,鱼鳖之藏也;尝一脟肉,而知一镬之味,一鼎之调。 (9)

(1) 胡,何也。

(2) 东夏,东方也。命,令也。

【校】旧校云:“‘言’一作‘世’。”

(3) 【校】旧校云:“‘合’一作‘同’。”

(4) 【校】旧本“异”上亦有“欲”字,系误衍,李本无,今从之。

(5) 故,事也。

(6) 【校】旧校云:“‘择’一作‘释’。”

(7) 【校】案《意林》无“益”字。

(8) 阴,日夕昃也。

【校】注“夕昃”疑“晷”之误。孙云:“李善注陆士衡《演连珠》引高诱曰‘阴,晷影之候也’。”

(9) 调,和也。

【校】“一脟”,旧本作“一脬”,讹。卢云:“‘脟’与‘脔’同。旧本讹其下,而《日抄》引作‘肘’,又脱其上。”今案:《史记·司马相如传》载《子虚赋》有“脟割轮焠”之语,《集解》引郭璞曰“脟音脔”,李善注《文选》亦同;又《汉书·相如传》师古曰“‘脟’与‘脔’同”。今定为“脟”字。《意林》及《北堂书钞》百四十五、《御览》八百六十三皆作“一脔”,他书亦皆作“一脔”,知“一脟”之即为“一脔”者少矣。

荆人欲袭宋,使人先表澭水。 (1) 澭水暴益, (2) 荆人弗知,循表而夜涉,溺死者千有余人,军惊而坏都舍。向其先表之时可导也, (3) 今水已变而益多矣,荆人尚犹循表而导之,此其所以败也。今世之主,法先王之法也,有似于此。 (4) 其时已与先王之法亏矣, (5) 而曰“此先王之法也”,而法之以为治,岂不悲哉!故治国无法则乱,守法而弗变则悖,悖乱不可以持国。世易时移,变法宜矣。譬之若良医,病万变,药亦万变。病变而药不变,向之寿民,今为殇子矣。 (6) 故凡举事必循法以动, (7) 变法者因时而化,若此论则无过务矣。 (8) 夫不敢议法者,众庶也;以死守者,有司也; (9) 因时变法者,贤主也。是故有天下七十一圣,其法皆不同。非务相反也,时势异也。故曰良剑期乎断,不期乎镆铘; (10) 良马期乎千里,不期乎骥骜。 (11) 夫成功名者,此先王之千里也。

(1) 【校】旧校云:“‘澭’一作‘灌’。”

(2) 暴,卒。益,长。

(3) 导,涉也。向其施表时水可涉也。

(4) 似此表澭水而不知其长益也。

(5) 亏,毁也。

(6) 向,曩也。未成人夭折曰殇子也。

(7) 动,作也。

(8) 务犹事也。

(9) 【校】“守”下亦当有“法”字。

(10) 镆铘,良剑也。取其能断,无取于名也,故曰“不期乎镆铘”。

(11) 骜,千里马名也。王者乘之游骜,因曰骥骜也。

楚人有涉江者, (1) 其剑自舟中坠于水,遽契其舟曰:“是吾剑之所从坠。” (2) 舟止,从其所契者入水求之。舟已行矣,而剑不行,求剑若此,不亦惑乎!以此故法为其国与此同。 (3) 时已徙矣,而法不徙,以此为治,岂不难哉?有过于江上者,见人方引婴儿而欲投之江中,婴儿啼。人问其故,曰:“此其父善游。”其父虽善游,其子岂遽善游哉?此任物亦必悖矣。 (4) 荆国之为政,有似于此。 (5)

(1) 涉,渡也。

(2) 遽,疾也。疾刻舟识之于此下坠剑者也。

【校】旧校云:“‘契’一作‘刻’。”

(3) 为,治也。与此契舟求剑者同也。

(4) 任,用也。

(5) 似此悖也。





第十六卷 先识览



先识


一曰:

凡国之亡也,有道者必先去,古今一也。 (1) 地从于城, (2) 城从于民, (3) 民从于贤。 (4) 故贤主得贤者而民得,民得而城得,城得而地得。夫地得岂必足行其地、人说其民哉?得其要而已矣。 (5)

(1) 《传》曰“君子见几而作,不俟终日”,故必先去也。孔子曰“贤者避世,其次避地,其次避人,其次避言”,故曰“古今一也”。

【校】案:《子华子·神气》篇“吾闻之,太上违世,其次违地,其次违人”,与此避人正相合。

(2) 城不下,地不迁。

(3) 民不溃,城不坏。

(4) 亶父处邠,狄人攻之,杖策而去,邑乎岐周,邠人襁负而随之,故曰民从贤也。

【校】所谓“天下之父归之,其子焉往”是也。下文终古、向挚、屠黍诸人,亦是说在下之贤人。注尚未切。

(5) 《孝经》曰“非家至而日见之也”,以德化耳,故曰“得其要而已矣”。

夏太史令终古出其图法,执而泣之。夏桀迷惑,暴乱愈甚,太史令终古乃出奔如商。汤喜而告诸侯曰:“夏王无道,暴虐百姓,穷其父兄,耻其功臣,轻其贤良,弃义听谗,众庶咸怨,守法之臣,自归于商。” (1) 殷内史向挚见纣之愈乱迷惑也,于是载其图法,出亡之周。武王大说,以告诸侯曰:“商王大乱,沈于酒德,辟远箕子,爰近姑与息, (2) 妲己为政,赏罚无方, (3) 不用法式,杀三不辜, (4) 民大不服,守法之臣,出奔周国。” (5)

(1) 知桀之必亡也。

(2) 箕子忠臣而疏远之,姑息之臣而与近之。

【校】案:《尸子》曰“弃黎老之言,用姑息之语”,注云:“姑,妇也。息,小儿也。”与此意同。

(3) 方,道。

(4) 剖比干之心,折材士之股,刳孕妇而观其胞。

【校】注“股”,旧本作“肝”,误,今据《古乐篇》注改正。

(5) 周国在丰、镐也。

晋太史屠黍见晋之乱也,见晋公之骄而无德义也,以其图法归周。 (1) 周威公见而问焉,曰:“天下之国孰先亡?” (2) 对曰:“晋先亡。”威公问其故,对曰:“臣比在晋也,不敢直言,示晋公以天妖,日月星辰之行多以不当,曰:‘是何能为?’ (3) 又示以人事多不义,百姓皆郁怨,曰:‘是何能伤?’又示以邻国不服,贤良不举,曰:‘是何能害?’如是,是不知所以亡也。故臣曰晋先亡也。”居三年,晋果亡。 (4) 威公又见屠黍而问焉,曰:“孰次之?”对曰:“中山次之。”威公问其故,对曰:“天生民而令有别。有别,人之义也,所异于禽兽麋鹿也,君臣上下之所以立也。中山之俗,以昼为夜,以夜继日,男女切倚,固无休息, (5) 康乐,歌谣好悲, (6) 其主弗知恶,此亡国之风也。 (7) 臣故曰中山次之。”居二年,中山果亡。威公又见屠黍而问焉,曰:“孰次之?”屠黍不对。威公固问焉,对曰:“君次之。”威公乃惧,求国之长者,得义莳、田邑而礼之, (8) 得史 、赵骈以为谏臣, (9) 去苛令三十九物, (10) 以告屠黍。对曰:“其尚终君之身乎!” (11) 曰 (12) :“臣闻之,国之兴也,天遗之贤人与极言之士; (13) 国之亡也,天遗之乱人与善谀之士。” (14) 威公薨,肂九月不得葬,周乃分为二。 (15) 故有道者之言也,不可不重也。

(1) 屠黍,晋出公之太史也。出公,顷公之孙,定公之子也。《史记》曰:“智伯攻出公,出公奔齐而道死焉。”

【校】“屠黍”,《说苑·权谋》篇作“屠余”。

(2) 周敬王后五世,考烈王封其弟于河南为桓公。威公,桓公之孙也。

【校】谢云:“敬王五传为考王,《人表》作‘考哲’,此误‘考烈’。西周威公为桓公之子,非孙也。”

(3) 不敢直言其乱也,但语以日月星辰之行多不当其宿度也,而云是无能为也。

【校】《说苑》作“多不当,曰:是何能然”。

(4) 屠黍居周三年也。

(5) 切,磨;倚,近也。无休息,夜淫不足,续以昼日。

【校】“切倚”,《淮南·齐俗训》作“切踦”,注:“踦,足也。”《说苑》同。

(6) 康,乐也。安淫酒之乐,乐极则继之以悲也。

【校】“康乐”上《说苑》有“淫昏”二字。

(7) 风,化也。

(8) 二人贤者也。

【校】“义莳”,《说苑》作“锜畴”。

(9) 二人直人。

【校】《说苑》作“史理、赵巽”。

(10) 物,事。

(11) 其尚,尚也。

【校】旧本“君”下衍“子”字,今从《黄氏日抄》所引去之,《说苑》亦无。

(12) 【校】《说苑》无。

(13) 极,尽。

(14) 谀,诌也。

【校】次“遗”字,旧校云“一作‘予’”。

(15) 下棺置地中谓之肂。

周鼎著饕餮,有首无身,食人未咽,害及其身,以言报更也。 (1) 为不善亦然。白圭之中山,中山之王欲留之,白圭固辞,乘舆而去。又之齐, (2) 齐王欲留之仕,又辞而去。人问其故,曰:“之二国者皆将亡,所学有五尽。何谓五尽?曰:莫之必,则信尽矣; (3) 莫之誉,则名尽矣;莫之爱,则亲尽矣;行者无粮、居者无食,则财尽矣;不能用人,又不能自用,则功尽矣。国有此五者,无幸必亡。中山、齐皆当此。” (4) 若使中山之王与齐王闻五尽而更之,则必不亡矣。 (5) 其患不闻,虽闻之又不信。然则人主之务,在乎善听而已矣。夫五割而与赵,悉起而距军乎济上,未有益也。 (6) 是弃其所以存,而造其所以亡也。 (7)

(1) 【校】《广雅·释言》云:“更,偿也。”

(2) 白圭,周人。

(3) 【校】《说苑》作“莫之必忠,则言尽矣”,下“誉”字、“爱”字上皆有“必”字。

(4) 当此五尽。

【校】“无幸”,旧本作“无辜”,误,今从《本生》篇改正。《说苑》亦作“毋幸”。

(5) 更犹革也。

(6) 中山五割地与赵,赵卒亡之;齐悉起军以距燕人于济上,燕卒破之;不能自存,故曰“未有益也”。

(7) 保地养民,所以存也,弃而不修。割地与赵,弃民于燕,不能自卫,而众破亡,故曰“造其所以亡也”。





观世


二曰:

天下虽有有道之士,国犹少。千里而有一士,比肩也;累世而有一圣人,继踵也。士与圣人之所自来,若此其难也,而治必待之,治奚由至? (1) 虽幸而有,未必知也, (2) 不知则与无贤同。 (3) 此治世之所以短,而乱世之所以长也。 (4) 故王者不四,霸者不六,亡国相望,囚主相及。 (5) 得士则无此之患。 (6) 此周之所封四百余, (7) 服国八百余,今无存者矣。虽存,皆尝亡矣。贤主知其若此也,故日慎一日,以终其世。 (8) 譬之若登山,登山者,处已高矣,左右视,尚巍巍焉山在其上。贤者之所与处,有似于此。身已贤矣,行已高矣,左右视,尚尽贤于己。故周公旦曰:“不如吾者,吾不与处,累我者也; (9) 与我齐者,吾不与处,无益我者也。” (10) 惟贤者必与贤于己者处。贤者之可得与处也,礼之也。

(1) 《淮南记》曰:“欲治之君不世出,可与治之臣不万一,以不万一待不世出,何由遇哉?”故曰“治奚由至”。

(2) 未必知其为贤也。

(3) 不知其贤而不用之,故不治,则与无贤同。

(4) 短,少;长,多也。

(5) 言不绝也。

(6) 无亡囚之患也。

(7) 封,建。

【校】“此”疑“比”。

(8) 没世为世。

【校】疑是“没身为世”。贤主时以其亡其亡为忧也。

(9) 【校】“不如吾者”,旧本作“吾不如者”,误,今从《意林》改正。《大戴·曾子制言》中卢注亦作“不如我者”。

(10) 齐,等也。等则不能胜己,故曰“无益我者也”。

主贤世治,则贤者在上; (1) 主不肖世乱,则贤者在下。今周室既灭,天子既废, (2) 乱莫大于无天子,无天子则强者胜弱,众者暴寡,以兵相刬, (3) 不得休息,而佞进, (4) 今之世当之矣。 (5) 故欲求有道之士,则于江海之上,山谷之中,僻远幽闲之所,若此则幸于得之矣。太公钓于滋泉, (6) 遭纣之世也,故文王得之。文王,千乘也;纣,天子也。天子失之,而千乘得之,知之与不知也。 (7) 诸众齐民,不待知而使,不待礼而令。 (8) 若夫有道之士,必礼必知,然后其智能可尽也。 (9)

(1) 上,上位也。

(2) 【校】“天子”,旧本作“天下”,讹。此段与前《谨听》篇同,彼云“而天子已绝”。

(3) 刬,灭。

(4) 佞谄者进而升用也。

(5) 今,谓衰周无天子之世,故曰“当之”。

(6) 【校】说见《谨听》篇。卢云:“《说文》‘兹,黑也’,引《春秋传》曰‘何故使吾水滋’,今《左传》作‘兹’,则‘兹’乃本字,后人加以水旁,实则一字耳。”

(7) 纣不知太公贤,故失之也。

(8) 令亦使也。

(9) 可尽得而用也。

晏子之晋,见反裘负刍息于涂者,以为君子也。 (1) 使人问焉,曰:“曷为而至此?”对曰:“齐人累之,名为越石父。” (2) 晏子曰:“嘻!”遽解左骖以赎之,载而与归。至舍,弗辞而入。越石父怒,请绝。晏子使人应之曰:“婴未尝得交也, (3) 今免子于患,吾于子犹未邪?” (4) 越石父曰:“吾闻君子屈乎不己知者,而伸乎己知者。吾是以请绝也。” (5) 晏子乃出见之,曰:“向也见客之容而已,今也见客之志。 (6) 婴闻察实者不留声, (7) 观行者不讥辞, (8) 婴可以辞而无弃乎?” (9) 越石父曰:“夫子礼之,敢不敬从。”晏子遂以为客。 (10) 俗人有功则德,德则骄。今晏子功免人于厄矣,而反屈下之,其去俗亦远矣。此令功之道也。 (11)

(1) 晏子,齐大夫晏平仲也。

(2) 累之,累然有罪。

【校】“累”,《新序·节士》篇作“纍”,即《史记》所云“在缧绁中”也。

(3) 【校】旧校云:“‘交’一作‘友’。”

(4) 【校】旧本下复有一“也”字。古“也”字亦与“邪”通,后人注“邪”字于旁以代音,而传写遂误入正文。今去“也”留“邪”,盖以便读者使不致惑耳。

(5) 【校】案:《史记·晏子传》载石父之言云:“方吾在缧绁中,彼不知我也。夫子既已感寤而赎我,是知己。知己而无礼,固不如在缧绁之中。”如此则所以绝之意方明。

(6) 【校】《晏子·杂上》篇作“意”,《新序》同。

(7) 实,功实也。言欲察人之功实,不复留意考其名声也。

(8) 欲观人之至行,不讥刺之以辞。

(9) 辞,谢也。谢不敏而可以弗弃也。

(10) 客,敬。

(11) 【校】《晏子》、《新序》“令功”俱作“全功”。

子列子穷,容貌有饥色。 (1) 客有言之于郑子阳者, (2) 曰:“列御寇,盖有道之士也, (3) 居君之国而穷,君无乃为不好士乎?”郑子阳令官遗之粟数十秉。子列子出见使者,再拜而辞。使者去,子列子入,其妻望而拊心曰:“闻为有道者妻子,皆得逸乐。今妻子有饥色矣,君过而遗先生食,先生又弗受也。岂非命也哉?”子列子笑而谓之曰 (4) :“君非自知我也,以人之言而遗我粟也,至已而罪我也,有罪且以人言, (5) 此吾所以不受也。”其卒民果作难,杀子阳。 (6) 受人之养而不死其难则不义,死其难则死无道也,死无道,逆也。子列子除不义去逆也岂不远哉?且方有饥寒之患矣,而犹不苟取,先见其化也。先见其化而已动,远乎性命之情也。 (7)

(1) 子列子,御寇,体道人也,著书八篇,在庄子前,庄子称之也。

(2) 子阳,郑相也。一曰郑君。

(3) 【校】旧本“列御寇”上衍一“子”字。案《列子·说符》、《庄子·让王》俱无“子”字,《新序》作“子列子圄寇”。

(4) 【校】旧校云:“‘笑’一作‘歎’。”

(5) 【校】“有”下“罪”字衍。“有”与“又”同。《庄子》作“至其罪我也,又且以人之言”,《列子》同。

(6) 子阳严猛,刑无所赦。家人有折弓者,畏诛,因国人逐猘狗之乱而杀子阳也。

(7) 孔子曰“贫观其所取”,此之谓也。

【校】“远”疑“达”字之误。





知接


三曰:

人之目以照见之也,以瞑则与不见同, (1) 其所以为照、所以为瞑异。 (2) 瞑士未尝照,故未尝见,瞑者目无由接也。 (3) 无由接而言见,詤。 (4) 智亦然,其所以接智、所以接不智同, (5) 其所能接、所不能接异。 (6) 智者其所能接远也, (7) 愚者其所能接近也。 (8) 所能接近而告之以远化,奚由相得?无由相得,说者虽工,不能喻矣。 (9) 戎人见暴布者而问之曰:“何以为之莽莽也?” (10) 指麻而示之。怒曰:“孰之壤壤也,可以为之莽莽也?” (11) 故亡国非无智士也,非无贤者也, (12) 其主无由接故也。无由接之患,自以为智, (13) 智必不接。今不接而自以为智,悖。 (14) 若此则国无以存矣,主无以安矣。智无以接, (15) 而自知弗智,则不闻亡国,不闻危君。 (16)

(1) 同一目也。

【校】谓目本非有异。

(2) 谓见与不见,故曰“异”。

(3) 接,见。

(4) 詤,读诬妄之诬,亿不详审也。

【校】旧本“詤”作“ ”。段云:“当作‘詤’。《说文》‘詤,梦言也,从言亡声’,正如‘亡’‘无’、‘荒’‘ ’通用,故可读诬。”又惠氏于《左氏襄廿九年传》“只见疏也”,亦谓当为“詤”。

(5) 一同智也。

【校】亦当作“同一智也”。

(6) 异,谓能与不能。

(7) 智者达于明,见未萌之前,故曰“接远”。

(8) 愚者蔽于明,祸至而不知,故曰“接近”。

(9) 虽子贡辩敏,无由何如,故曰弗能喻。

(10) 为,作也。莽莽,长大貌也。

(11) 壤壤犹养治之。莽莽,均长貌。

【校】注不明。壤壤,纷错之貌。《史记·货殖传》“天下壤壤,皆为利往”。此指麻之未治者。戎人见其纷乱难理,言孰有如此而可以成长大之幅乎?疑人之欺己也。

(12) 谓虽有贤智之士,不能为昏主谋以在将亡之国也。

(13) 【校】旧校云:“‘为智’一作‘长智’。”

(14) 悖,惑。

(15) 【校】李本作“由接”。

(16) 言人君自知不智,则求贤而任之,故不闻亡国危君也。桀、纣所以国亡身灭,不自知不智故也。

管仲有疾,桓公往问之曰:“仲父之疾病矣, (1) 将何以教寡人?”管仲曰:“齐鄙人有谚曰:‘居者无载,行者无埋。’ (2) 今臣将有远行,胡可以问?” (3) 桓公曰:“愿仲父之无让也。”管仲对曰:“愿君之远易牙、竖刀、常之巫、卫公子启方。” (4) 公曰:“易牙烹其子以慊寡人, (5) 犹尚可疑邪?”管仲对曰:“人之情,非不爱其子也。其子之忍,又将何有于君?” (6) 公又曰:“竖刀自宫以近寡人, (7) 犹尚可疑邪?”管仲对曰:“人之情,非不爱其身也。其身之忍,又将何有于君?”公又曰:“常之巫审于死生,能去苛病, (8) 犹尚可疑邪?”管仲对曰:“死生命也,苛病失也。 (9) 君不任其命守其本,而恃常之巫,彼将以此无不为也。” (10) 公又曰:“卫公子启方事寡人十五年矣,其父死而不敢归哭,犹尚可疑邪?”管仲对曰:“人之情,非不爱其父也。其父之忍,又将何有于君?”公曰:“诺。”管仲死,尽逐之。食不甘,宫不治,苛病起,朝不肃。居三年,公曰:“仲父不亦过乎?孰谓仲父尽之乎?” (11) 于是皆复召而反。明年,公有病,常之巫从中出曰:“公将以某日薨。”易牙、竖刀、常之巫相与作乱,塞宫门,筑高墙,不通人,矫以公令。 (12) 有一妇人逾垣入,至公所。公曰:“我欲食。”妇人曰:“吾无所得。”公又曰:“我欲饮。”妇人曰:“吾无所得。” (13) 公曰:“何故?”对曰:“常之巫从中出曰:‘公将以某日薨。’ (14) 易牙、竖刀、常之巫相与作乱,塞宫门,筑高墙,不通人,故无所得。 (15) 卫公子启方以书社四十下卫。” (16) 公慨焉叹,涕出曰:“嗟乎!圣人之所见岂不远哉?若死者有知,我将何面目以见仲父乎?”蒙衣袂而绝乎寿宫。 (17) 虫流出于户,上盖以杨门之扇, (18) 三月不葬。 (19) 此不卒听管仲之言也。 (20) 桓公非轻难而恶管子也, (21) 无由接见也。 (22) 无由接,固却其忠信, (23) 而爱其所尊贵也。 (24)

(1) 病,困也。

(2) 谓臣居职有谋计,皆当宣之于君,无有载藏之于心也。行谓即世也,亦当输写所知,使君行之,无有怀藏埋之地中。

(3) 言不足问。

(4) 远犹疏也。无令相近。

【校】“竖刀”,旧本作“竖刁”,字俗。刀亦有貂音。

(5) 慊,快。

(6) 子,所爱也,而忍杀之,何能有爱于君?

(7) 宫,割阴为奄人。

(8) 苛,鬼病,魂下人病也。

(9) 精神失其守,魍魉鬼物乘以下人,故曰“失”。

【校】孙云:“《御览》四百四十六作‘苛病本也’。观下文‘守其本’之言,似‘本’字是。”

(10) 为妖惑也。

(11) 谁谓仲父言尽可用乎。

(12) 令矫公命为不通人之命。

【校】注“矫公”二字当在“令,命”之下,盖先以命释令也。

(13) 言无从得饮食与公。

(14) 【校】此十三字疑衍文。

(15) 无使得饮食也。

(16) 下,降也。社,二十五家也。四十社凡千家,以降归于卫。

(17) 蒙,冒也。袂,衣袖也。以衣覆面而绝。寿宫,寝堂也。

(18) 杨门,门名。扇,屏也。邪臣争权,莫能举丧事,六十日而殡,虫流出户,不欲人见,故掩以杨门之扇也。

(19) 【校】《史记·齐世家》正义引作“二月不殡”。

(20) 【校】旧校云:“‘言’一作‘败’。”

(21) 轻,易。

(22) 【校】疑“见”字衍。

(23) 接,知也。却,不用。

【校】案:“固”与“故”通用。刘本作“见”字,属上句,非。

(24) 爱其所尊所贵,谓竖刀、易牙、常之巫、卫公子启方之属也。





悔过


四曰:

穴深寻,则人之臂必不能极矣, (1) 是何也?不至故也。智亦有所不至。所不至,说者虽辩,为道虽精,不能见矣。 (2) 故箕子穷于商, (3) 范蠡流乎江。 (4)

(1) 八尺曰寻。

【校】极,《意林》作“及”。

(2) 精,微妙也。

(3) 为纣所困。

(4) 佐越王句践灭吴,雪会稽之耻,功成而还,轻舟浮于江而去也。

【校】孙云:“《离谓》篇云:‘范蠡、子胥以此流。’意少伯乘扁舟出入三江五湖,不知所终,传闻异辞遂有流江之说欤?”卢云:“案《贾谊书·耳痺》篇,建宁本作‘范蠡负室而归五湖’,潭本作‘负石而蹈五湖’。潭本与流江之说颇相似,疑当时相传有此言也。”

昔秦缪公兴师以袭郑。 (1) 蹇叔谏曰:“不可。臣闻之,袭国邑,以车不过百里,以人不过三十里, (2) 皆以其气之 与力之盛至,是以犯敌能灭,去之能速。 (3) 今行数千里,又绝诸侯之地以袭国,臣不知其可也。 (4) 君其重图之。” (5) 缪公不听也。蹇叔送师于门外而哭曰:“师乎!见其出而不见其入也。”蹇叔有子曰申与视, (6) 与师偕行。蹇叔谓其子曰:“晋若遏师必于殽。 (7) 女死不于南方之岸,必于北方之岸,为吾尸女之易。” (8) 缪公闻之,使人让蹇叔曰:“寡人兴师,未知何如。今哭而送之,是哭吾师也。”蹇叔对曰:“臣不敢哭师也。臣老矣,有子二人,皆与师行,比其反也,非彼死则臣必死矣,是故哭。” (9) 师行过周, (10) 王孙满要门而窥之, (11) 曰:“呜呼!是师必有疵。 (12) 若无疵,吾不复言道矣。夫秦非他,周室之建国也。 (13) 过天子之城,宜橐甲束兵, (14) 左右皆下,以为天子礼。今袀服回建,左不轼,而右之 (15) 超乘者五百乘, (16) 力则多矣,然而寡礼,安得无疵?” (17) 师过周而东。郑贾人弦高、奚施 (18) 将西市于周,道遇秦师,曰:“嘻!师所从来者远矣,此必袭郑。”遽使奚施归告,乃矫郑伯之命以劳之, (19) 曰:“寡君固闻大国之将至久矣。大国不至,寡君与士卒窃为大国忧,日无所与焉,惟恐士卒罢弊与糗粮匮乏。何其久也。使人臣犒劳以璧,膳以十二牛。”秦三帅对曰:“寡君之无使也,使其三臣丙也、术也、视也于东边候 之道, (20) 过是,以迷惑陷入大国之地。” (21) 不敢固辞,再拜稽首受之。三帅乃惧而谋曰:“我行数千里,数绝诸侯之地以袭人,未至而人已先知之矣,此其备必已盛矣。” (22) 还师去之。当是时也,晋文公适薨,未葬。先轸言于襄公曰 (23) :“秦师不可不击也。臣请击之。”襄公曰:“先君薨,尸在堂,见秦师利而因击之,无乃非为人子之道欤?”先轸曰:“不吊吾丧,不忧吾哀,是死吾君而弱其孤也。若是而击,可大强。 (24) 臣请击之。”襄公不得已而许之。先轸遏秦师于殽而击之,大败之,获其三帅以归。缪公闻之,素服庙临, (25) 以说于众曰:“天不为秦国,使寡人不用蹇叔之谏,以至于此患。”此缪公非欲败于殽也,智不至也。 (26) 智不至则不信, (27) 言之不信,师之不反也从此生。 (28) 故不至之为害大矣。 (29)

(1) 不鸣钟鼓,密声曰袭。

(2) 军行三十里一舍。

(3) ,壮也。故进能灭敌,去之能疾也。

(4) 绝,过也。过诸侯之土地,远行袭国,必不能以克,故曰“不知其可也”。

(5) 重,深。

【校】戒其勿轻易也。

(6) 申,白乙丙也。视,孟明视也。皆蹇叔子也。

【校】案:《左氏》“蹇叔之子与师”,则必非三帅明矣。《史记·秦本纪》云“百里傒子孟明视,蹇叔子西乞术、白乙丙”,孙云“均属传讹”。

(7) 殽,渑池县西崤塞是也。

(8) 识之易也。

(9) 彼,谓其子。

(10) 周,今河南城,所谓王城也。《公羊传》曰:“王城者,西周。”襄王时也。

(11) 王孙满,周大夫。要,徼也。

(12) 疵,病。

(13) 周家所封立也。

(14) 【校】梁仲子云:“《左传僖卅三年正义》引作‘櫜甲束兵’。”

(15) 袀,同也。兵服上下无别,故曰“袀服”。回建者,兵车四乘也。左,君位也。君不载而车右之不轼。

【校】袀服即《左传》之“均服”,旧本作“初服”,讹。回建,注所释殊不明,此似言车上所建者。《考工记》有六建,谓五兵与人也。“君不载”以下字亦多讹,窃疑“右之超乘者五百乘”本连下为句,高氏误分之。时秦伯不自行,亦不当言“左,君位也”。盖将在左,御居中,御主车可不下,今左并不轼,右既下,复超乘以上,与《左氏传》微异。

(16) 【校】《左传》作“三百乘”。

(17) 超乘,巨踊车上也。不下车为天子礼,故曰力多而寡礼。

【校】注“巨踊”之“巨”,当从《左传》“距跃曲踊”之“距”。车中如何跳踊?《左传》所载“左右免胄而下”为是。盖既下而即跃以上车,示其有勇。

(18) 【校】《淮南·人间训》作“蹇他”。

(19) 擅称君命曰矫。

(20) 候,视也。 ,晋国也。

【校】案:李善注《文选》谢灵运《述祖德》诗引此作“使臣”,无“人”字。旧本“ ”讹作“晋”,注亦讹。今从善注改正,而删去旧校“一作 ,注亦同”六字。

(21) 【校】旧校云:“‘陷入’一作‘以及’。”

(22) 盛,强。

(23) 襄公,文公之子 。

(24) 强,霸也。

【校】旧本注又有“一作若是而弗击,不可大强”十一字,乃校者之辞。

(25) 哭也。

(26) 言但虑袭郑之利,不知将有殽之败也,故曰“智不至也”。

(27) 蹇叔哭其子云“晋人遏师必于殽”,缪公不信。

【校】正文旧本作“智至”。案:语当承上文,今增正。

(28) 蹇叔言信,不可不信也。师之不反,败殽也。《穀梁传》曰“匹马只轮无反者”,从蹇叔言信生也。

【校】首句旧多作“而言不可不信”,今从朱本改。注末句讹,当云“从不信蹇叔言生也”。

(29) 师败帅执,故害大也。





乐成


五曰:

大智不形,大器晚成,大音希声。禹之决江水也,民聚瓦砾。事已成,功已立,为万世利。禹之所见者远也,而民莫之知。故民不可与虑化举始, (1) 而可以乐成功。

(1) 始,首也。

孔子始用于鲁,鲁人鹥诵之曰:“麛裘而 ,投之无戾。 而麛裘,投之无邮。” (1) 用三年,男子行乎涂右,女子行乎涂左,财物之遗者,民莫之举。 (2) 大智之用,固难逾也。 (3) 子产始治郑,使田有封洫,都鄙有服。 (4) 民相与诵之曰:“我有田畴,而子产赋之。我有衣冠,而子产贮之。 (5) 孰杀子产,吾其与之。” (6) 后三年,民又诵之曰:“我有田畴,而子产殖之。 (7) 我有子弟,而子产诲之。 (8) 子产若死,其使谁嗣之?” (9) 使郑简、鲁哀当民之诽 也而因弗遂用,则国必无功矣, (10) 子产、孔子必无能矣。 (11) 非徒不能也,虽罪施,于民可也。 (12) 今世皆称简公、哀公为贤,称子产、孔子为能。此二君者,达乎任人也。 (13) 舟车之始见也,三世然后安之。 (14) 夫开善岂易哉? (15) 故听无事治,事治之立也,人主贤也。 (16)

(1) 孔子衣麛裘。投,弃也。“邮”字与“尤”同。言投弃孔子无罪尤也。

【校】鹥盖鲁人名,《孔丛子》作“谤”,《御览》同。“ ”字旧讹“鞞”,案当作“ ”,与“芾”、“韍”、“绂”字同,《孔丛子·陈士义》篇正作“芾”。

(2) 举,取也。

(3) 逾,迈也。

【校】卢云:“‘逾’当本是‘喻’字。言大智之用,固不能使人易晓也。注就讹文为释,非是。”

(4) 封,界;洫,沟也。服,法服也。君子小人各有制。

(5) 【校】《左氏襄卅年传》“贮”作“褚”,同。卢云:“案《周礼·廛人》注‘ ,藏’,《释文》云‘本或作贮,或作褚’。”梁仲子云:“《一切经音义·四分律第四十一》引《传》亦作‘贮’。”

(6) 与犹助也。《左传》曰“郑子产作丘赋,国人谤之”此之谓也。

(7) 殖,长也。

(8) 诲,教也。

(9) 嗣,续也。

(10) 言二国人民诽 仲尼、子产之时,二君国不复用,则二国亦无用贤圣之功。

(11) 若二人不见用,则必无所能为也。

(12) 言非但不能有为也,虽施二人罪罚,于民意亦可。

【校】注“施”,旧作“此”,讹。案王肃注《家语·正论解》:“施生,施犹行也,行生者之罪也。”杜预注昭十四年《左氏传》亦云:“施,行罪也。”今改正。

(13) 任,用也。

(14) 安,习也。

(15) 开,通也。

(16) 听无事,谓民谤子产、孔子,无用之为事也,乃贤主所以为事也,谤之无治也,又贤主能听之,故曰“听无事治,事治之立也”。

魏攻中山,乐羊将, (1) 已得中山,还反报文侯, (2) 有贵功之色。 (3) 文侯知之,命主书曰:“群臣宾客所献书者,操以进之。”主书举两箧以进。 (4) 令将军视之,书尽难攻中山之事也。 (5) 将军还走,北面再拜曰:“中山之举,非臣之力,君之功也。”当此时也,论士殆之日几矣, (6) 中山之不取也,奚宜二箧哉?一寸而亡矣。 (7) 文侯,贤主也,而犹若此,又况于中主邪?中主之患,不能勿为,而不可与莫为。 (8) 凡举无易之事, (9) 气志视听动作无非是者,人臣且孰敢以非是邪疑为哉?皆壹于为,则无败事矣。此汤、武之所以大立功于夏、商, (10) 而句践之所以能报其仇也。 (11) 以小弱皆壹于为而犹若此,又况于以强大乎? (12)

(1) 乐羊为将以伐中山。

(2) 报,白也。

(3) 【校】旧校云:“‘贵’一作‘责’。”卢云:“疑是‘负功’。”

(4) 【校】《秦策》作“谤书一箧”。

(5) 难,说。

(6) 论士,议士也。殆,危。几,近。

(7) 中山之不取,谓乐羊不敢取以为己功,一方寸之书则亡矣,何乃二箧也?

(8) 夫唯贤主能无为耳。中庸之主不能无为,故不可与为无为也。

(9) 【校】旧校云:“‘易’一作‘为’。”

(10) 成汤得夏,武王得商,故曰“立功”也。

(11) 越王句践破吴于五湖,故曰“能报其仇也”。

(12) 汤、武以百里,越王臣事吴王夫差,为之前马,故称“小弱”。

魏襄王与群臣饮,酒酣,王为群臣祝,令群臣皆得志。 (1) 史起兴而对曰:“群臣或贤或不肖,贤者得志则可。不肖者得志则不可。” (2) 王曰:“皆如西门豹之为人臣也。”史起对曰:“魏氏之行田也以百亩,邺独二百亩,是田恶也。漳水在其旁,而西门豹弗知用,是其愚也。知而弗言,是不忠也。愚与不忠,不可效也。” (3) 魏王无以应之。明日,召史起而问焉,曰:“漳水犹可以灌邺田乎?”史起对曰:“可。”王曰:“子何不为寡人为之?”史起曰:“臣恐王之不能为也。”王曰:“子诚能为寡人为之,寡人尽听子矣。” (4) 史起敬诺,言之于王曰:“臣为之,民必大怨臣,大者死,其次乃藉臣。臣虽死藉,愿王之使他人遂之也。” (5) 王曰:“诺。”使之为邺令。史起因往为之。邺民大怨,欲藉史起。史起不敢出而避之。王乃使他人遂为之。水已行,民大得其利,相与歌之曰:“邺有圣令,时为史公。决漳水,灌邺旁。终古斥卤,生之稻粱。” (6) 使民知可与不可,则无所用矣。 (7) 贤主忠臣,不能导愚教陋,则名不冠后,实不及世矣。史起非不知化也,以忠于主也。魏襄王可谓能决善矣。诚能决善,众虽喧哗,而弗为变。功之难立也,其必由哅哅邪。国之残亡,亦犹此也。 (8) 故哅哅之中,不可不味也。中主以之哅哅也止善,贤主以之哅哅也立功。 (9)

(1) 魏襄王,孟子所见梁惠王之子也。祝,愿也。

(2) 贤者得志则忠,故曰“可”也。不肖得志则骄,骄则乱,故曰“不可”。公孙丑曰:“伊尹放太甲于桐宫,太甲贤,又反之。贤者之为人臣,其君不贤则可放欤?”孟子曰:“有伊尹之志则可,无伊尹之志则篡也。”

(3) 【校】梁伯子云:“《史记·河渠书》‘西门豹引漳水溉邺’,《后汉书·安帝纪》‘初元二年修西门豹所分漳水为支渠以溉田’,《水经·浊漳水》注亦云‘豹引漳以溉邺’,《吕氏》所言不足据,《汉书·沟洫志》乃误仍之。左太冲《魏都赋》云‘西门溉其前,史起灌其后’,斯得其实。”

(4) 听,从也。

(5) 遂,成也。

(6) 【校】案:《汉书·沟洫志》“民歌之曰‘邺有贤令兮为史公,决漳水兮灌邺旁,千古舄卤兮生稻粱’”,数字不同。

(7) 【校】案:“无所用”下似脱一“贤”字。

(8) 【校】“犹”与“由”同。

(9) 按《魏王世家》,文侯生武侯,武侯生惠王,惠王生襄王。西门豹,文侯用为邺令,史起亚之,不得为四世之君臣也。又孟子见梁襄王,出,语人曰:“望之而不似人君,就之而不见所畏焉。”何能决善哉?此言复谬也。

【校】注“魏世家”,“王”字衍。以一见定其终身不能从善,此言亦过。梁仲子云:“《左氏传襄廿五年正义》引此书云‘魏文侯时,史起为邺令,引漳水以灌田’,与今本异。”





察微


六曰:

使治乱存亡若高山之与深溪, (1) 若白垩之与黑漆,则无所用智,虽愚犹可矣。且治乱存亡则不然,如可知,如可不知;如可见,如可不见。 (2) 故智士贤者相与积心愁虑以求之, (3) 犹尚有管叔、蔡叔之事与东夷八国不听之谋。 (4) 故治乱存亡,其始若秋毫。 (5) 察其秋毫,则大物不过矣。 (6)

(1) 有水曰涧,无水曰溪。

(2) 【校】孙疑两“可不”文倒,据李善注《文选》东方曼倩《非有先生论》作“不可”为是。

(3) 积累其仁心,思虑其善政,以求致治也。

(4) 成王幼少,周公摄政,勤心国家,以致太平。管叔,周公弟也;蔡叔,周公兄也;流言作乱。东夷八国,附从二叔,不听王命。周公居摄三年,伐奄,八国之中最大,著在《尚书》,余七国小,又先服,故不载于经也。

【校】梁伯子以诸书皆言管、蔡是周公弟,唯《孟》、《荀》及《史记》以管叔为周公兄,此又言蔡叔为周公兄,益不可信。全谢山以皋鼬之会,将长蔡于卫,不闻长蔡于鲁,安得如此注所言乎?

(5) 喻微细也。

(6) 过,失也。

鲁国之法,鲁人为人臣妾于诸侯,有能赎之者,取其金于府。子贡赎鲁人于诸侯来而让不取其金。孔子曰:“赐失之矣。自今以往,鲁人不赎人矣。取其金则无损于行, (1) 不取其金则不复赎人矣。” (2) 子路拯溺者,其人拜之以牛,子路受之。孔子曰:“鲁人必拯溺者矣。” (3) 孔子见之以细,观化远也。 (4)

(1) 言无所损于德行也。

(2) 《淮南记》曰“子贡让而止善”,此之谓也。

【校】“止善”,旧本误作“亡义”,今据《淮南·齐俗训》本文改正。

(3) 《淮南记》曰“子路受而劝德”,此之谓也。

(4) 见其始,知其终,故曰“观化远也”。

楚之边邑曰卑梁, (1) 其处女与吴之边邑处女桑于境上,戏而伤卑梁之处女。卑梁人操其伤子以让吴人,吴人应之不恭,怒杀而去之。吴人往报之,尽屠其家。卑梁公怒, (2) 曰:“吴人焉敢攻吾邑?”举兵反攻之, (3) 老弱尽杀之矣。吴王夷昧闻之怒,使人举兵侵楚之边邑,克夷而后去之。 (4) 吴、楚以此大隆。 (5) 吴公子光又率师与楚人战于鸡父, (6) 大败楚人,获其帅潘子臣、小帷子、陈夏啮, (7) 又反伐郢, (8) 得荆平王之夫人以归, (9) 实为鸡父之战。凡持国,太上知始,其次知终,其次知中。三者不能,国必危,身必穷。 (10) 《孝经》曰:“高而不危,所以长守贵也;满而不溢,所以长守富也。富贵不离其身,然后能保其社稷而和其民人。”楚不能之也。 (11)

(1) 【校】梁伯子云:“卑梁是吴边邑,《史记·十二侯表》及《楚世家》、《伍子胥传》皆同。楚边邑乃钟离也。此与《吴世家》所载皆误。”

(2) 公,卑梁大夫也。楚僭称王,守邑大夫皆称公,若周之单襄公、成肃公、刘文公也。

(3) 反,更也。

(4) 夷,平。

(5) “隆”当作“格”。格,斗也。

(6) 公子光,夷昧之子也。

(7) 潘子臣、小帷子,楚二大夫也。鸡父之战,胡、沈、陈、蔡皆佐楚战,故吴获之。夏,姓;啮,名;陈大夫。

【校】案:鸡父之战,获陈夏啮,在鲁昭廿三年;吴太子终累败楚舟师,获潘子臣、小帷子,在定六年;此误合为一。《释文》云:“‘惟’,本又作‘帷’。”《群经音辨》云:“小惟子,楚人也,音帷。”

(8) 又,复也。郢,楚国都也。

(9) 【校】卢云:“案《左氏昭廿三年传》云:‘楚太子建之母在郹,召吴人而启之。冬十月甲申,吴太子诸樊入郹,取楚夫人与其宝器以归。’与鸡父之战同一年事。”

(10) 言楚不知始与终,又不知中,故国危身穷也。

(11) 【校】黄东发云:“观此所引,然则《孝经》固古书也。”

郑公子归生率师伐宋。 (1) 宋华元率师应之大棘, (2) 羊斟御。明日将战,华元杀羊飨士,羊斟不与焉。 (3) 明日战,怒谓华元曰:“昨日之事,子为制; (4) 今日之事,我为制。” (5) 遂驱入于郑师。宋师败绩,华元虏。 (6) 夫弩机差以米则不发。战,大机也。飨士而忘其御也,将以此败而为虏,岂不宜哉? (7) 故凡战必悉熟偏备,知彼知己,然后可也。 (8)

(1) 《鲁宣二年传》曰“郑公子归生受命于楚伐宋”,言受命于楚与晋争盟也。

(2) 应,击也。大棘,宋邑,今陈留襄邑南大棘是也。

(3) 与,及也。

(4) 昨日之事,杀羊事也。

(5) 今日之事,御事也。

【校】陈氏树华《春秋内传考正》云“《左传》‘子为政’、‘我为政’,此或因始皇名改”,但他卷不尽然。

(6) 为郑虏。

(7) 《传》曰:“羊斟非人也,以其私憾,败国殄民,刑孰大焉。”此之谓也。

(8) 古之良将,人遗之单醪,输之于川,与士卒从下流饮之,示不自独享其味也。华元羊肉不及羊斟而身见虏,故曰“凡战必悉熟偏备,知彼知己”。

【校】注“单醪”亦作“箪醪”,李善注《文选》张景阳《七命》引《黄石公记》曰“昔良将之用兵也,人有馈一箪之醪,投河,令众迎流而饮之。夫一箪之醪,不味一河,而三军思为致死者,以滋味及之也”,或以为楚庄王事。“独享”,宋邦乂本作“独周”,形近而讹,今改正。

鲁季氏与郈氏斗鸡,郈氏介其鸡, (1) 季氏为之金距。 (2) 季氏之鸡不胜,季平子怒,因归郈氏之宫而益其宅。 (3) 郈昭伯怒,伤之于昭公, (4) 曰:“禘于襄公之庙也,舞者二人而已,其余尽舞于季氏。 (5) 季氏之无道无上久矣,弗诛,必危社稷。”公怒,不审, (6) 乃使郈昭伯将师徒以攻季氏,遂入其宫。仲孙氏、叔孙氏相与谋曰:“无季氏,则吾族也死亡无日矣。”遂起甲以往,陷西北隅以入之,三家为一,郈昭伯不胜而死。昭公惧,遂出奔齐,卒于乾侯。 (7) 鲁昭听伤而不辩其义, (8) 惧以鲁国不胜季氏,而不知仲、叔氏之恐而与季氏同患也,是不达乎人心也。不达乎人心,位虽尊,何益于安也?以鲁国恐不胜一季氏,况于三季?同恶固相助。 (9) 权物若此其过也,非独仲、叔氏也,鲁国皆恐。鲁国皆恐,则是与一国为敌也,其得至乾侯而卒犹远。 (10)

(1) 介,甲也。作小铠著鸡头也。

【校】案:《淮南·人间训》注云“介,以芥菜涂其鸡翅也”,与此互异。

(2) 以利铁作锻距, 其距上。

(3) 平子,名意如,悼子纥之子也。侵郈氏宫以益己宅。

【校】《淮南》“归”作“侵”,又下句作“而筑之宅”。

(4) 郈氏,鲁孝公子惠伯华之后也,以字为氏,因曰郈氏。昭,谥也。伤犹谮也。

【校】梁仲子云:“惠伯华,《礼记·檀弓上》注作‘惠伯巩’,《正义》引《世本》作‘革’,字形并相近。‘以字为氏’当作‘以邑为氏’,孝公八世孙成叔为郈大夫,因以为氏。”

(5) 禘,大祭也。襄公,昭公之父也。礼,天子八佾,诸侯六佾。六佾者,四十八人。于襄公庙二人,余在季氏,季氏僭也。

【校】“二人”,《左传》、《淮南》并同。吴斗南《两汉刊误补遗》曰:“‘人’当作‘八’。舞必以八人成列,故郑人赂晋以女乐二八。若四人尚不成乐,况二人乎?”卢云:“案秦遗戎王女乐亦是二八,齐遗鲁女乐八十人,《御览》引《家语》作‘二八’,知此‘二人’断然字误。鲁自隐公初用六羽,当有六八。季氏大夫,本有四八,今又取公之四佾以往,故公止有二八。观高氏注亦本不误,乃转写之失也。”

(6) 审,详也。

(7) 乾侯,晋邑。

(8) 即辨,别。义,宜。

(9) 同恶昭公。

(10) 不薨国内,乃至乾侯,故以为远也。





去宥


七曰:

东方之墨者谢子,将西见秦惠王。 (1) 惠王问秦之墨者唐姑果。唐姑果恐王之亲谢子贤于己也, (2) 对曰:“谢子,东方之辩士也。其为人甚险,将奋于说,以取少主也。” (3) 王因藏怒以待之。谢子至,说王,王弗听。谢子不说,遂辞而行。 (4) 凡听言以求善也,所言苟善,虽奋于取少主,何损?所言不善,虽不奋于取少主,何益?不以善为之悫,而徒以取少主为之悖, (5) 惠王失所以为听矣。用志若是,见客虽劳,耳目虽弊,犹不得所谓也。此史定所以得行其邪也, (6) 此史定所以得饰鬼以人,罪杀不辜,群臣扰乱,国几大危也。人之老也,形益衰, (7) 而智益盛。 (8) 今惠王之老也,形与智皆衰邪? (9)

(1) 谢子,关东人也,学墨子之道。惠王,秦孝公之子驷也。

【校】《说苑·杂言》篇作“祁射子”,古“谢”、“射”通。

(2) 【校】《说苑》“唐姑”无“果”字。旧校云:“‘亲’一作‘视’。”

(3) 奋,强也。少主,惠王也。

(4) 行,去也。

(5) 悫,诚也。

(6) 史定,秦史。

(7) 衰,肌肤消也。

(8) 老者见事多,所闻广,故智益盛。

(9) 皆,俱也。

荆威王学书于沈尹华,昭釐恶之。威王好制, (1) 有中谢佐制者,为昭釐谓威王曰:“国人皆曰王乃沈尹华之弟子也。” (2) 王不说,因疏沈尹华。中谢,细人也, (3) 一言而令威王不闻先王之术,文学之士不得进,令昭釐得行其私,故细人之言不可不察也。且数怒人主,以为奸人除路,奸路以除而恶壅却,岂不难哉? (4) 夫激矢则远,激水则旱, (5) 激主则悖,悖则无君子矣。夫不可激者,其唯先有度。 (6)

(1) 威王,楚怀王之父也。制,术数也。

(2) 中谢,官名也。佐王制法制也。

【校】梁仲子云:“楚官有中射士,见《韩非·十过》篇,此作‘中谢’,亦通用。”卢云:“《史记·张仪传》后陈轸举中谢对楚王云云,《索隐》云‘中谢,盖谓侍御之官’,则知楚之官实有中谢,与此正同。”

(3) 细,小人也。

(4) 除犹开通也,故曰“而恶壅却,岂不难”也。

(5) 【校】案:《淮南·兵略训》、《鹖冠子·世兵》篇俱作“水激则悍,矢激则远”,《史记·贾谊传》索隐引此正作“旱”,以言水激则去疾,不能浸润也,与两家作“悍”不同。但近所行陆佃注《鹖冠子》本亦作“旱”,小司马又云“《说文》‘旱’与‘悍’同音”,则亦可通用也。

(6) 度,法也。

邻父有与人邻者,有枯梧树,其邻之父言梧树之不善也,邻人遽伐之。邻父因请而以为薪,其人不说曰:“邻者若此其险也,岂可为之邻哉?”此有所宥也。 (1) 夫请以为薪与弗请,此不可以疑枯梧树之善与不善也。

(1) 宥,利也,又云为也。

【校】注颇难通。疑“宥”与“囿”同,谓有所拘碍而识不广也。以下文观之,犹言蔽耳。

齐人有欲得金者,清旦,被衣冠,往鬻金者之所,见人操金,攫而夺之。吏搏而束缚之,问曰:“人皆在焉,子攫人之金,何故?”对吏曰:“殊不见人,徒见金耳。”此真大有所宥也。

夫人有所宥者,固以昼为昏,以白为黑,以尧为桀,宥之为败亦大矣。亡国之主,其皆甚有所宥邪!故凡人必别宥然后知, (1) 别宥则能全其天矣。 (2)

(1) 句。

(2) 天,身也。

【校】“则能”旧本作“别能”,今案文义改。





正名


八曰:

名正则治,名丧则乱。使名丧者,淫说也。说淫则可不可而然不然,是不是而非不非。 (1) 故君子之说也,足以言贤者之实、不肖者之充而已矣, (2) 足以喻治之所悖、乱之所由起而已矣, (3) 足以知物之情、人之所获以生而已矣。

(1) 不可者而可之也,不然者而然之也,不是者而是之也,不非者而非之也,故曰“淫说”也。

(2) 充亦实也。

(3) 喻,明。悖,惑。

【校】卢云:“《左氏庄十一年传》云‘禹、汤罪己,其兴也悖焉’,杜注云‘悖,盛貌’,《释文》云‘悖一作勃’。此当以‘治之所悖’为句,不当训惑,疑是‘盛’字之讹。”

凡乱者,刑名不当也。人主虽不肖,犹若用贤,犹若听善,犹若为可者。其患在乎所谓贤从不肖也, (1) 所为善而从邪辟, (2) 所谓可从悖逆也, (3) 是刑名异充而声实异谓也。夫贤不肖、善邪辟、可悖逆, (4) 国不乱、身不危奚待也? (5) 齐湣王是以知说士而不知所谓士也, (6) 故尹文问其故, (7) 而王无以应。此公玉丹之所以见信而卓齿之所以见任也,任卓齿而信公玉丹,岂非以自仇邪? (8)

(1) 从,使人从不肖自谓贤。

(2) 使人从邪辟自谓善,故曰“其患”也。

(3) 可者,乃从悖逆之道也。

(4) 不肖者贤之,邪辟者善之,悖逆者可之也。

(5) 言乱亡立至,无所复待也。

(6) 湣王,齐田常之孙田和立为宣王,湣王,宣王之子也。言知当敬义士,不能知其所行,徒谓之士也。

【校】梁仲子云:“前《乐成》篇‘义士’作‘议士’。”

(7) 问所以为士之故也。

(8) 公玉丹,齐臣。卓齿,楚人,亦为湣王臣。其毙由在此二人,非欲以自毙也,然二人卒毙之。湣王无道,齿杀之而擢其筋,悬之于东庙终日,以自毙者也。

【校】梁仲子云:“‘卓齿’,《齐策》作‘淖齿’,颜师古注《人表》‘淖,音女教反,字或作卓’。”梁伯子云:“《潜夫论》作‘踔齿’,《史记·田单传》徐广作‘悼齿’。注‘东庙’,后《行论篇》注亦同,《国策》作‘庙梁’。”

尹文见齐王, (1) 齐王谓尹文曰:“寡人甚好士。”尹文曰:“愿闻何谓士?”王未有以应。尹文曰:“今有人于此,事亲则孝,事君则忠,交友则信,居乡则悌。有此四行者,可谓士乎?”齐王曰:“此真所谓士已。” (2) 尹文曰:“王得若人,肯以为臣乎?” (3) 王曰:“所愿而不能得也。”尹文曰:“使若人于庙朝中, (4) 深见侮而不斗,王将以为臣乎?”王曰:“否。大夫见侮而不斗,则是辱也, (5) 辱则寡人弗以为臣矣。”尹文曰:“虽见侮而不斗,未失其四行也。未失其四行者,是未失其所以为士一矣。未失其所以为士一,而王以为臣,失其所以为士一,而王不以为臣,则向之所谓士者,乃士乎?”王无以应。尹文曰:“今有人于此,将治其国,民有非则非之,民无非则非之,民有罪则罚之,民无罪则罚之,而恶民之难治,可乎?”王曰:“不可。”尹文曰:“窃观下吏之治齐也,方若此也。”王曰:“使寡人治信若是,则民虽不治,寡人弗怨也。 (6) 意者未至然乎?” (7) 尹文曰:“言之不敢无说,请言其说。王之令曰:‘杀人者死,伤人者刑。’民有畏王之令,深见侮而不敢斗者,是全王之令也, (8) 而王曰:‘见侮而不敢斗,是辱也。’夫谓之辱者,非此之谓也,以为臣不以为臣者罪之也,此无罪而王罚之也。”齐王无以应。论皆若此,故国残身危,走而之穀 (9) 如卫。 (10) 齐湣王,周室之孟侯也, (11) 太公之所以老也。桓公尝以此霸矣,管仲之辩名实审也。 (12)

(1) 尹文,齐人,作《名书》一篇,在公孙龙前,公孙龙称之。

(2) 【校】旧校云:“一作‘矣’。”

(3) 【校】旧校云:“‘肯’一作‘用’。”

(4) 【校】旧校云:“‘庙’一作‘广’。”

(5) 【校】“大夫”疑衍“大”字。

(6) 虽不可治,言不怨也。

(7) 王言,意以为未至如是。

【校】此注各本脱,李本有。

(8) 【校】李本无“之”字。

(9) 穀,齐邑也。

(10) 如,之也。

(11) 孟,长也。

(12) 桓公以继绝存亡,率义以霸,管子辅而成之,不以土地之大也。今此湣王继篡国之胄僭号,不义之人,无管子之辅,假有之,又不能用,喻以桓公,山头井底,不得方之者也。





第十七卷 审分览



审分


一曰:

凡人主必审分,然后治可以至, (1) 奸伪邪辟之涂可以息, (2) 恶气苛疾无自至。 (3) 夫治身与治国,一理之术也。 (4) 今以众地者,公作则迟,有所匿其力也; (5) 分地则速,无所匿迟也。 (6) 主亦有地,臣主同地,则臣有所匿其邪矣, (7) 主无所避其累矣。 (8)

(1) 主,谓君也。分,谓仁义礼律杀生与夺之分也。至者,至于治也。

(2) 息,灭也。

(3) 自,从也。君德合则祥瑞应,故苛疾无从来至也。

(4) 身治则国治,故曰“一理之术也”。

(5) 作,为也。迟,徐也。迟用其力而不勤也。

(6) 分地,独也。速,疾也。获稼穑则入己分而有之,各自欲得疾成,无藏匿,无舒迟也。

(7) 邪,私也。不欲君知,故蔽之也。

(8) 累犹负也。谓主不以正临之,令臣自欲容私,故君无所避其负也。

凡为善难,任善易。奚以知之?人与骥俱走,则人不胜骥矣;居于车上而任骥,则骥不胜人矣。人主好治人官之事,则是与骥俱走也, (1) 必多所不及矣。 (2) 夫人主亦有车,居无去车, (3) 则众善皆尽力竭能矣,谄谀诐贼巧佞之人无所窜其奸矣。 (4) 坚 (5) 穷廉直忠敦之士毕竞劝骋骛矣。 (6) 人主之车,所以乘物也。察乘物之理,则四极可有。 (7) 不知乘物,而自怙恃,夺其智能,多其教诏,而好自以, (8) 若此则百官恫扰, (9) 少长相越,万邪并起,权威分移, (10) 不可以卒,不可以教,此亡国之风也。 (11)

(1) 言君好为人臣之官事,是谓与骥俱走,无以胜之也。

【校】旧校云:“‘人官’一作‘人臣’。”

(2) 言力不赡也。好自治人臣之所官事亦如之。

(3) 去犹释也。去,读去就之去。

【校】案:“居”字旧在“车”字上,系误倒,“居”字当属下句,今乙正。

(4) 窜犹容也。

(5) 坚,刚也。

(6) 毕,尽。

(7) 察,明也。有之易也。

(8) 诏亦教。以,用也。

(9) 恫,动。扰,乱。

【校】“恫”,《玉篇》作“挏”。

(10) 政在家门。

(11) 风,化。

王良之所以使马者,约审之以控其辔,而四马莫敢不尽力。 (1) 有道之主,其所以使群臣者亦有辔。其辔何如?正名审分,是治之辔已。故按其实而审其名,以求其情;听其言而察其类,无使放悖。 (2) 夫名多不当其实,而事多不当其用者,故人主不可以不审名分也。不审名分,是恶壅而愈塞也。 (3) 壅塞之任,不在臣下,在于人主。 (4) 尧、舜之臣不独义,汤、禹之臣不独忠,得其数也; (5) 桀、纣之臣不独鄙,幽、厉之臣不独辟,失其理也。 (6)

(1) 王良,晋大夫邮无正邮良也,以善御之功,死托精于星,天文“王良策驷”是也。

【校】邮无正,见《国语》,即《左传》之邮无恤。旧本“邮”作“孙”,意即孙阳。

(2) 放,纵也。悖,乱也。

(3) 名,虚实爵号之名也。分,杀生与夺之分也。《传》曰:“唯器与名,不可以假人,君之所慎也。”故曰不可不审。愈,益也。不审之而欲治,犹恶湿而居下也,故曰“恶壅而愈塞也”。

(4) 君明则臣忠,臣忠则政无壅塞,故曰“在于人主”。

(5) 御之得其术。

(6) 幽王,周宣王之子。厉王,周宣王之父。言先幽、厉,偶文耳。杀戮不辜曰厉,壅过不达曰幽,皆恶谥也。

【校】“壅过”,《逸周书》、《独断》、苏明允并作“壅遏”。

今有人于此,求牛则名马,求马则名牛,所求必不得矣; (1) 而因用威怒,有司必诽怨矣,牛马必扰乱矣。百官,众有司也;万物, (2) 群牛马也。不正其名,不分其职,而数用刑罚,乱莫大焉。夫说以智通,而实以过悗; (3) 誉以高贤,而充以卑下; (4) 赞以洁白,而随以污德; (5) 任以公法,而处以贪枉; (6) 用以勇敢,而堙以罢怯。 (7) 此五者,皆以牛为马,以马为牛,名不正也。故名不正,则人主忧劳勤苦,而官职烦乱悖逆矣。国之亡也,名之伤也,从此生矣。白之顾益黑, (8) 求之愈不得者,其此义邪! (9)

(1) 失其名,故不得也。

(2) 【校】旧校云:“一作‘邦’。”

(3) 以,用。

【校】旧校云:“‘过’一作‘遇’。”又本“悗”作“悦”。今案:“遇”、“悦”皆非也。悗音瞒,又音懑,《玉篇》“惑也”,《庄子·大宗师》释文“废忘也”。

(4) 充,实。

(5) 以污秽之德,随洁白之踪,里谚所谓“牛头而卖马脯”,此理之谓也。

(6) 与上“卖马脯”义同。

(7) 将行罢怯,以充勇敢之用,故芎穷之似藁本,蛇床之类薇芜,碧卢之乱美玉,非猗顿不能别也。暗主之于名实,亦不能知也,是以赵高壅蔽二世,以鹿为马,此之类也。

【校】“薇芜”,《博物志》作“蘼芜”。

(8) 顾,反。

(9) 此牛名马之类也。

故至治之务,在于正名,名正则人主不忧劳矣,不忧劳则不伤其耳目之主。 (1) 问而不诏, (2) 知而不为, (3) 和而不矜, (4) 成而不处, (5) 止者不行,行者不止,因刑而任之,不制于物,无肯为使, (6) 清静以公, (7) 神通乎六合,德耀乎海外, (8) 意观乎无穷,誉流乎无止, (9) 此之谓定性于大湫, (10) 命之曰无有。 (11) 故得道忘人,乃大得人也,夫其非道也。 (12) 知德忘知,乃大得知也,夫其非德也。 (13) 至知不幾,静乃明幾也,夫其不明也。 (14) 大明不小事,假乃理事也,夫其不假也。 (15) 莫人不能,全乃备能也,夫其不全也。 (16) 是故于全乎去能,于假乎去事,于知乎去幾,所知者妙矣。 (17) 若此则能顺其天,意气得游乎寂寞之宇矣,形性得安乎自然之所矣。全乎万物而不宰, (18) 泽被天下而莫知其所自始, (19) 虽不备五者,其好之者是也。 (20)

(1) 主犹性也。

【校】案注,似“主”本是“生”字。

(2) 诏,教也。好问而行之,不自专独为教诏。

(3) 虽知之,不与为名其功也。

(4) 和则成矣,不自矜伐。

(5) 处,居也。《老子》曰“功成而弗居”,此之谓也。

(6) 止者不行,谓土也。行者不止,谓水也。因形而任之,不令土行,不令水止也。不制于物者,不为物所制,物不能制之也。若此人者,王公不能屈,何肯为人之使令者乎?

(7) 公,正。

(8) 六合,四方上下也。海外,四海之外。

(9) 流,行。

(10) 性,命也。大湫犹大窦。

(11) 无有,无形也。道无形。无形,言得道也。

(12) 得道,澹然无所思虑,故忘人也。而人慕之,此乃所以大得人也。夫其非道也,亦在其人也。不能使人人得之也,故曰“夫其非道也”。

【校】旧本作“夫非其道也”,注同。今案下数句皆“其”字在“非”字上,今亦依例乙转。

(13) 自知有德,忘人知之,而人仰之,此乃所以大得知也。夫其非德也,亦在其人也。不能使人人知之也,故曰“夫其非德也”。

(14) 幾,近也。至有德,虽万里人犹知之,故曰“不幾”也。静,安也。安处其德,乃所以使人明之也。望远若近,故曰“静乃明幾也”。夫其不明也,亦在其人也。明不能使人人见之,故曰“夫其不明也”。

【校】卢云:“此所言幾,即今人所谓机警也。此与圣人言不逆诈、不亿不信、先觉乃贤意相似,注似非也。”

(15) 大明者,垂拱无为而化流行,不治小事也。假,摄。若周公、鲁隐勤理成致之也。夫其不假也,亦在其人也。久假不归,恶乃知非,故曰“夫其不假也”。

【校】旧本正文“夫其不能”下缺“也”字,今依注补。

(16) 假摄者,务济国事,事济归之,故曰“莫人不能,全乃备能也”。夫其不全也,亦在其人也。周公有流言之谤,鲁隐有钟巫之难,故曰“夫其不全也”。推其本情,但管、蔡倾邪,不达圣人之旨也,其大乎子翚有欲太宰之志,于是生之也。

【校】注“其大乎”三字衍,仍当有一“公”字。又“生之”疑是“生心”。

(17) 妙,微也。

(18) 宰,主。

(19) 自,从。始,首。

(20) 人于此五者,虽不能备有,但能好慕则幾矣。





君守


二曰:

得道者必静。静者无知,知乃无知,可以言君道也。故曰中欲不出谓之扃,外欲不入谓之闭。 (1) 既扃而又闭,天之用密,有准不以平,有绳不以正。 (2) 天之大静,既静而又宁,可以为天下正。 (3)

(1) 【校】二语见《文子·上仁》篇、《淮南·主术训》。

(2) 准,法。正,直。

【校】“准”,《说文》本作“準”,从水,隼声,而诸子书多省作“准”,《五经文字》云“《字林》作‘准’”。今姑仍旧本。

(3) 宁,安。正,主。

身以盛心,心以盛智,智乎深藏而实莫得窥乎! (1) 《鸿范》曰:“惟天阴骘下民。”阴之者,所以发之也。 (2) 故曰不出于户而知天下,不窥于牖而知天道。 (3) 其出弥远者,其知弥少, (4) 故博闻之人、强识之士阙矣, (5) 事耳目、深思虑之务败矣, (6) 坚白之察、无厚之辩外矣。 (7) 不出者,所以出之也;不为者,所以为之也。 (8) 此之谓以阳召阳,以阴召阴。 (9) 东海之极,水至而反; (10) 夏热之下,化而为寒。 (11) 故曰天无形,而万物以成; (12) 至精无象,而万物以化; (13) 大圣无事,而千官尽能。 (14) 此乃谓不教之教,无言之诏。故有以知君之狂也,以其言之当也; (15) 有以知君之惑也,以其言之得也。 (16) 君也者,以无当为当,以无得为得者也。当与得不在于君,而在于臣。 (17) 故善为君者无识,其次无事。有识则有不备矣, (18) 有事则有不恢矣。 (19) 不备不恢,此官之所以疑,而邪之所从来也。今之为车者,数官然后成。 (20) 夫国岂特为车哉? (21) 众智众能之所持也,不可以一物一方安车也。 (22) 夫一能应万,无方而出之务者, (23) 唯有道者能之。

(1) 窥,见。

(2) 阴阳升陟也,言天覆生下民,王者助天举发,明之以仁义也。

(3) 因人之知以知之。

【校】“故曰”者,本《老子·德经》之言,下二语亦是。

(4) 不知人而恃己明,不能察偏远,故弥少也。

(5) 阙,短。

(6) 败,伤。

(7) 外,弃所以为也。

(8) 不出户庭而知天下,与出无异,故曰“所以出之”。不为而有所成,与为无异,故曰“所以为之”。

(9) 召,致也。

(10) 反,还。

(11) 寒暑更也。

(12) 天无所制,而物自成。

(13) 说与“昊天”同。

(14) 官得其人,人任其职,故尽能也。

(15) 君狂言,臣不敢谏之,而自以其言为当也,是以知其言之狂。

(16) 狂言而自得,所以知其惑也。

(17) 待臣匡正。

(18) 物不可悉识,备识其物则为不备也。

【校】注“则为”,朱本作“则反有”。

(19) 恢亦备也。

(20) 轮舆辕轴,各自有材,故曰“数官然后成”。

(21) 特,但。

(22) 方,道也。

(23) 一者,道也。

鲁鄙人遗宋元王闭, (1) 元王号令于国,有巧者皆来解闭。人莫之能解。兒说之弟子请往解之, (2) 乃能解其一,不能解其一,且曰:“非可解而我不能解也,固不可解也。”问之鲁鄙人,鄙人曰:“然,固不可解也,我为之而知其不可解也。今不为而知其不可解也,是巧于我。”故如兒说之弟子者,以“不解”解之也。 (3) 郑大师文终日鼓瑟而兴,再拜其瑟前曰:“我效于子,效于不穷也。”故若大师文者,以其兽者先之,所以中之也。 (4)

(1) 鄙人,小人也。闭,结不解者也。

(2) 【校】《韩非·外储说左上》云:“兒说,宋人,善辩者也。”《淮南·人间训》注云:“宋大夫。”

(3) 言此不可以解也乃能解。

(4) 徼射其兽,走与矢会,故中之也。

故思虑自心伤也, (1) 智差自亡也, (2) 奋能自殃, (3) 其有处自狂也。故至神逍遥倏忽,而不见其容;至圣变习移俗,而莫知其所从;离世别群,而无不同; (4) 君民孤寡,而不可障壅。 (5) 此则奸邪之情得, (6) 而险陂谗慝谄谀巧佞之人无由入。 (7) 凡奸邪险陂之人,必有因也。何因哉?因主之为。 (8) 人主好以己为, (9) 则守职者舍职而阿主之为矣。 (10) 阿主之为,有过则主无以责之,则人主日侵,而人臣日得。 (11) 是宜动者静,宜静者动也。尊之为卑,卑之为尊,从此生矣。此国之所以衰,而敌之所以攻之者也。

(1) 思虑劳精神而乱于心,故自伤也。

(2) 差,过也。用智过差,极其情欲,以自消亡也。

(3) 奋,强也。夏桀强其能以肆无道,自取破灭之殃。

(4) 同,和。

(5) 孤寡,人君之谦称也。能自卑谦名誉者,不可防障。

(6) 得犹知也。

(7) 无从自入而见用也。

(8) 因犹顺也。

(9) 己所好、情所欲则为也。

(10) 阿,从。

(11) 得其阿主之志也。

奚仲作车, (1) 苍颉作书, (2) 后稷作稼, (3) 皋陶作刑, (4) 昆吾作陶, (5) 夏鮌作城。 (6) 此六人者所作当矣, (7) 然而非主道者。 (8) 故曰作者忧,因者平。惟彼君道,得命之情,故任天下而不强,此之谓全人。 (9)

(1) 奚仲,黄帝之后,任姓也,《传》曰:“为夏车正,封于薛。”

(2) 苍颉生而知书,写仿鸟迹以造文章。

(3) 后,君;稷,官也。烈山氏子曰柱,能植百谷蔬菜以为稷。

【校】案:柱在舜臣之稷之前。又下云“非至道者”,故不数弃,而以柱当之。

(4) 《虞书》曰:“皋陶!蛮夷猾夏,寇贼奸宄,女作士师,五刑有服。”

(5) 昆吾,颛顼之后,吴回之孙,陆终之子,己姓也,为夏伯制作陶冶埏埴为器。

【校】旧本注“吴回”下衍“黎”字,今删。

(6) 鮌,禹父也。筑作城郭。

(7) 当,合也,合其宜。

(8) 【校】旧校云:“‘主’一作‘至’。”

(9) 全人,全德之人,无亏阙也。





任数


三曰:

凡官者,以治为任,以乱为罪。今乱而无责,则乱愈长矣。 (1) 人主以好暴示能, (2) 以好唱自奋, (3) 人臣以不争持位, (4) 以听从取容, (5) 是君代有司为有司也, (6) 是臣得后随以进其业。 (7) 君臣不定, (8) 耳虽闻不可以听, (9) 目虽见不可以视, (10) 心虽知不可以举, (11) 势使之也。 (12) 凡耳之闻也藉于静, (13) 目之见也藉于昭, (14) 心之知也藉于理。 (15) 君臣易操,则上之三官者废矣。 (16) 亡国之主,其耳非不可以闻也,其目非不可以见也,其心非不可以知也,君臣扰乱 [1] ,上下不分别,虽闻曷闻,虽见曷见,虽知曷知, (17) 驰骋而因耳矣,此愚者之所不至也。 (18) 不至则不知,不知则不信。 (19) 无骨者不可令知冰。 (20) 有土之君能察此言也,则灾无由至矣。

(1) 长,大。

(2) 以能暴示众。

【校】旧校云:“‘暴’一作‘为’。”今案:“为”字是也。注“暴示”乃表暴之意。若作能为威严解,正文与注并窒碍。

(3) 奋,强。

(4) 《孝经》云:“臣不可以不争于君。”此不争持位,非忠臣也。

(5) 阿意曲从以自容。

(6) 有司,大臣也。大臣匡君,进思尽忠,退思补过。此听从取容,无有正君者,君当自正耳,是为代有司为有司。

(7) 后随,随后也。其业,不争取容定业也。

(8) 君不君,臣不臣,故不定也。

(9) 不可以听五音。

(10) 不可以视五色。

(11) 不可举取。

(12) 言其人不忠不正,苟取容说,志意倾邪,故曰“势使之也”。

(13) 藉,假也。静无声,乃有所闻,故藉于静。

(14) 昭,明也。非明目无所见,故藉明以见物。

(15) 去物断义,非理不决,故藉于理以决物。

(16) 三官,耳、目、心。不得其正,故曰“废”。

(17) 虽知就利避害,不知仁义与就利避害之本也。去其本而求之于末,故曰“虽知曷知”。其闻、见之义亦然。

(18) 驰骋,田猎也。田猎禽兽,亡国之主所乐及,修其本者弗为也,故曰“愚者之所不至也”。

(19) 言不知其君,不信修仁义,无欲为可以致治安国之本。

(20) 亡国之主,不知去贪暴、施仁惠,若无骨之虫,春生秋死,不知冬寒之有冰雪。

且夫耳目知巧固不足恃,惟修其数、行其理为可。 (1) 韩昭釐侯视所以祠庙之牲,其豕小, (2) 昭釐侯令官更之。 (3) 官以是豕来也,昭釐侯曰:“是非向者之豕邪?”官无以对。命吏罪之。从者曰:“君王何以知之?”君曰:“吾以其耳也。” (4) 申不害闻之, (5) 曰:“何以知其聋?以其耳之聪也。 (6) 何以知其盲?以其目之明也。何以知其狂?以其言之当也。故曰去听无以闻则聪,去视无以见则明,去智无以知则公。去三者不任则治,三者任则乱。” (7) 以此言耳目心智之不足恃也。耳目心智,其所以知识甚阙, (8) 其所以闻见甚浅。以浅阙博居天下,安殊俗,治万民,其说固不行。 (9) 十里之间,而耳不能闻;帷墙之外,而目不能见;三亩之宫,而心不能知。其以东至开梧, (10) 南抚多 , (11) 西服寿靡, (12) 北怀儋耳, (13) 若之何哉? (14) 故君人者,不可不察此言也。治乱安危存亡,其道固无二也。故至智弃智,至仁忘仁,至德不德。无言无思,静以待时,时至而应,心暇者胜。凡应之理,清净公素,而正始卒。焉此治纪,无唱有和,无先有随。古之王者,其所为少,其所因多。因者,君术也;为者,臣道也。为则扰矣,因则静矣。因冬为寒,因夏为暑,君奚事哉?故曰君道无知无为,而贤于有知有为,则得之矣。 (15)

(1) 理,道。

(2) 昭釐,谥也。晋宣子起之后也。起生贞子,居平阳。生康子,与赵襄子共灭智伯而分其地。生武子,都宜阳。生景侯处,徙阳翟。釐侯,景侯子也。

【校】梁伯子云:“《史记·韩世家》贞子生简子,简子生庄子,庄子生康子。徐广谓《史记》多无简子、庄子。《人表》亦同。然韩简子见《左传》及《史·晋赵世家》,惟庄子无考。今《史记》据《世本》,诱似未见此也。昭釐侯,《史》作‘昭侯’,乃懿侯子,非景侯子也。”

(3) 以豕小,使官更易大者。

(4) 言识其耳。

(5) 申不害,郑之京人,昭釐侯之相。

(6) 【校】“聪”,旧本“听”,讹,今案下文改。聪与聋韵协。

(7) 任,用也。

(8) 阙,短。

(9) 博,旷。固,必。

(10) 东极之国。

【校】“其以”,《意林》作“而欲”。

(11) 南极之国。

【校】“ ”,《意林》作“ ”。

(12) 西极之国。“靡”亦作“麻”。

【校】《大荒西经》作“南服寿麻”,“南”字讹,注引亦作“麻”。

(13) 北极之国。

【校】《大荒西经》作“阘耳”。

(14) 【校】《意林》作“何以得哉”。

(15) 贤,愈。得,知。

有司请事于齐桓公,桓公曰:“以告仲父。”有司又请,公曰:“告仲父。”若是三。习者曰:“一则仲父,二则仲父,易哉为君?” (1) 桓公曰:“吾未得仲父则难,已得仲父之后,曷为其不易也?”桓公得管子,事犹大易,又况于得道术乎?

(1) 习,近习,所亲臣也。

孔子穷乎陈、蔡之间,藜羹不斟,七日不尝粒。 (1) 昼寝。颜回索米,得而爨之,几熟,孔子望见颜回攫其甑中而食之。选间食熟, (2) 谒孔子而进食。孔子佯为不见之。孔子起曰:“今者梦见先君,食洁而后馈。” (3) 颜回对曰:“不可。向者煤炱入甑中,弃食不祥,回攫而饮之。” (4) 孔子叹曰:“所信者目也,而目犹不可信;所恃者心也,而心犹不足恃。 (5) 弟子记之,知人固不易矣。” (6) 故知非难也,孔子之所以知人难也。

(1) 无藜羹可斟,无粒可食,故曰“不斟”、“不尝”。

【校】“斟”乃“糂”之讹。《说文》“糂,以米和羹也。”前《慎人》篇作“不糁”。

(2) 选间,须臾。

(3) 【校】孙云:“《御览》八百三十八‘后’作‘欲’,李善注《文选》陆士衡《君子行》作‘食洁故馈’。”

(4) 煤炱,烟尘也。入犹堕也。

【校】“煤炱”,旧本讹作“煤室”。孙云:“《选》注作‘炱煤’。”梁仲子云:“卢玉川诗‘当天一搭如煤炱’,政用此。”“室”与“炱”形近致讹,今定作“煤炱”。旧本注“烟尘”下多“之煤”二字,乃衍文,又“堕”作“坠”,今皆依《选》注删正。

(5) 目见妄,不可信。心忆妄,不足恃。

(6) 记,识。





勿躬


四曰:

人之意苟善,虽不知,可以为长。 (1) 故李子曰:“非狗不得兔,兔化而狗,则不为兔。”人君而好为人官,有似于此。 (2) 其臣蔽之,人时禁之; (3) 君自蔽,则莫之敢禁。夫自为人官,自蔽之精者也。 (4) 祓篲日用而不藏于箧, (5) 故用则衰,动则暗,作则倦。 (6) 衰、暗、倦三者,非君道也。

(1) 长,上。

(2) 作君而好治人官职,似兔化而为狗也。

(3) 人时有止之者。

(4) 精,甚。

(5) 祓篲,贱物也。日用扫除,故不藏于箧。喻人君好治人臣之职,与祓篲何异。

(6) 君用思臣识则志衰也。举动作臣安社稷利民之事,未必能独当,是自见蒙暗也。代臣作趋走力役之事则心倦。

大桡作甲子,黔如作虏首, (1) 容成作历,羲和作占日,尚仪作占月, (2) 后益作占岁,胡曹作衣,夷羿作弓,祝融作市,仪狄作酒,高元作室,虞姁作舟,伯益作井,赤冀作臼,乘雅作驾, (3) 寒哀作御, (4) 王冰作服牛,史皇作图,巫彭作医,巫咸作筮。 (5) 此二十官者,圣人之所以治天下也。圣王不能二十官之事,然而使二十官尽其巧,毕其能,圣王在上故也。 (6) 圣王之所不能也,所以能之也; (7) 所不知也,所以知之也。 (8) 养其神、修其德而化矣, (9) 岂必劳形愁弊耳目哉?是故圣王之德,融乎若日之始出,极烛六合而无所穷屈; (10) 昭乎若日之光,变化万物而无所不行。神合乎太一,生无所屈而意不可障; (11) 精通乎鬼神,深微玄妙而莫见其形。今日南面,百邪自正,而天下皆反其情, (12) 黔首毕乐其志、安育其性,而莫为不成。 (13) 故善为君者,矜服性命之情,而百官已治矣,黔首已亲矣,名号已章矣。 (14)

(1) 【校】旧校云:“‘虏’一作‘虑’。”案:虏与虑皆不可解。《世本》云“隶首作数”,或是此误。亦疑“虏首”当是“蔀首”。

(2) 【校】尚仪即常仪。古读仪为何,后世遂有嫦娥之鄙言。

(3) 【校】旧校云:“‘雅’一作‘持’。”案:《荀子·解蔽》篇云“乘杜作乘马”,杨倞注云:“《吕氏春秋》作‘一驾’。”“一”字或衍文。疑旧校“持”字乃“杜”字之误,杜即相土也。

(4) 【校】寒哀即《世本》之韩哀,古“寒”、“韩”通。“哀”,旧本作“衰”,误。孙云:“《蜀志·郤正传》注引作‘韩哀’。”

(5) 蓍筮。

(6) 圣王在上,官使人人任其事也,故尽毕其巧能也。

(7) 用其人,得其任,故所以能。

(8) 《老子》曰“不知乃知之”,此之谓。

(9) 无所思虑劳神,是养神也。无状而能化,化育万物谓也。

【校】“谓”疑衍,否或上当有“之”字。

(10) 极,北极,天太阴也,日能烛之。

【校】“日”旧讹“月”,注同。赵云:“极烛犹言遍烛,注非。”

(11) 大通也。神与通合生道,乃无诎厌;志意通达,不可障塞。

(12) 南面,当阳而治,谓之天子也。反,本。

【校】朱本注末有“也”字。

(13) 莫,无。

(14) 章,明也。

管子复于桓公曰 (1) :“垦田大邑, (2) 辟土艺粟,尽地力之利,臣不若甯遬,请置以为大田。 (3) 登降辞让,进退闲习,臣不若隰朋,请置以为大行。 (4) 蚤入晏出,犯君颜色,进谏必忠,不辟死亡,不重贵富,臣不若东郭牙,请置以为大谏臣。 (5) 平原广城, (6) 车不结轨,士不旋踵, (7) 鼓之,三军之士视死如归,臣不若王子城父, (8) 请置以为大司马。 (9) 决狱折中,不杀不辜,不诬无罪,臣不若弦章, (10) 请置以为大理。 (11) 君若欲治国强兵,则五子者足矣;君欲霸王,则夷吾在此。”桓公曰:“善。”令五子皆任其事,以受令于管子。 (12) 十年,九合诸侯,一匡天下,皆夷吾与五子之能也。管子,人臣也,不任己之不能, (13) 而以尽五子之能,况于人主乎?人主知能不能之可以君民也,则幽诡愚险之言无不职矣,百官有司之事毕力竭智矣。五帝三王之君民也,下固不过毕力竭智也。

(1) 复,白。

(2) 【校】《新序》“大”作“创”,《韩诗外传》作“垦田仞邑”。

(3) 甯遬,甯戚。

【校】古戚、速同音,“遬”即“速”。

(4) 大行,官名也。《周礼》:“大行人掌大宾客之礼,以亲诸侯。”

(5) 楚有箴尹之官,亦谏臣。

【校】《外传》、《新序》皆无“大”字。《御览》二百七十三无“臣”字。梁仲子云:“《管子·小匡》篇作‘鲍叔牙为大谏’。”

(6) 【校】“城”疑“域”,《新序》作“囿”。

(7) 结,交也。车两轮间曰轨。

(8) 【校】《新序》作“成甫”,《外传》亦作“成”。

(9) 司马,主武之官也。《周礼》“大司马之职,掌建国之九法,以佐王平邦国”也。

(10) 【校】《管子》作“宾须无”。王厚斋云:“案《说苑》,弦章在景公时,当以《管子》为正。”梁仲子云:“《小匡》篇作‘子旗为大理’,子旗盖弦章之字。”孙云:“《韩非·外储说左下》作‘弦商’,《新序》四作‘弦宁’。”

(11) 大理,治狱官。

(12) 受管子之令。

(13) 【校】《黄氏日抄》引作“不任己之能”。

夫君人而知无恃其能、勇、力、诚、信,则近之矣。凡君也者,处平静,任德化,以听其要,若此则形性弥羸而耳目愈精,百官慎职而莫敢愉綖, (1) 人事其事,以充其名。 (2) 名实相保,之谓知道。

(1) 愉,解。綖,缓。

【校】旧校云:“‘慎’一作‘顺’。”

(2) 上“事”,治也。





知度


五曰:

明君者,非遍见万物也,明于人主之所执也。有术之主者,非一自行之也,知百官之要也。知百官之要,故事省而国治也。明于人主之所执,故权专而奸止。奸止则说者不来而情谕矣,情者不饰而事实见矣。 (1) 此谓之至治。至治之世,其民不好空言虚辞,不好淫学流说, (2) 贤不肖各反其质, (3) 行其情不雕其素; (4) 蒙厚纯朴,以事其上。若此则工拙、愚智、勇惧可得以故易官,易官则各当其任矣。故有职者安其职,不听其议; (5) 无职者责其实,以验其辞。 (6) 此二者审,则无用之言不入于朝矣。君服性命之情,去爱恶之心, (7) 用虚无为本, (8) 以听有用之言,谓之朝。 (9) 凡朝也者,相与召理义也, (10) 相与植法则也。 (11) 上服性命之情,则理义之士至矣,法则之用植矣,枉辟邪挠之人退矣, (12) 贪得伪诈之曹远矣。 (13) 故治天下之要存乎除奸,除奸之要存乎治官,治官之要存乎治道,治道之要存乎知性命。 (14) 故子华子曰:“厚而不博,敬守一事, (15) 正性是喜。群众不周,而务成一能。 (16) 尽能既成,四夷乃平。 (17) 唯彼天符,不周而周。 (18) 此神农之所以长,而尧舜之所以章也。” (19)

(1) 饰,虚。

(2) 不学正道为淫学。邪说谓之流说。

(3) 反,本。质,正。

(4) 素,朴也。本性纯朴,不雕饰之以为华藻也。

【校】“行其情”,旧作“其行情”,孙云“李善注《文选·齐竟陵王行状》引作‘行其情’”,今依乙正。

(5) 有乱众干度之议者不听之。

(6) 验,功。

【校】案:“功”字必误,疑当为“效”,又疑是“劾”。

(7) 爱恶,好憎。

(8) 虚无,无所爱恶也。无所爱恶则公正,治之本也。

(9) 有用之言,谓忠正有益于国者。

(10) 召,致。

(11) 植,立。

(12) 挠,曲。

(13) 曹,众。

(14) 知性命则不珍难得之物,不为无益之事,唯道是从,利民而已。

(15) 子华子,体道人也。一事,正事。

(16) 一能,专一之能,言公正。

(17) 平,和。

(18) 忠信为周。

(19) 长犹盛也。章,著明也。以,用也。

人主自智而愚人,自巧而拙人, (1) 若此则愚拙者请矣, (2) 巧智者诏矣。 (3) 诏多则请者愈多矣, (4) 请者愈多,且无不请也。主虽巧智,未无不知也。 (5) 以未无不知,应无不请,其道固穷。 (6) 为人主而数穷于其下,将何以君人乎?穷而不知其穷,其患又将反以自多, (7) 是之谓重塞之主,无存国矣。故有道之主,因而不为, (8) 责而不诏, (9) 去想去意,静虚以待,不伐之言,不夺之事,督名审实,官复自司,以不知为道,以奈何为实。 (10) 尧曰:“若何而为及日月之所烛?” (11) 舜曰:“若何而服四荒之外?” (12) 禹曰:“若何而治青北,化九阳、奇怪之所际?” (13)

(1) 自智谓人愚,自巧谓人拙。《诗》云:“惟彼不顺,自独俾臧。自有肺肠,俾民卒狂。”愚拙者此之谓也。

【校】注“此”字疑衍。

(2) 君自谓智而巧,故愚拙者从之请也。

(3) 诏,教。

(4) 听益乱。

(5) 未能尽无所不知也。

(6) 固,必。

(7) 反,更。多,大。

(8) 因循旧法,不改为。

(9) 责臣成功,不妄以偏见教诏。

(10) 道尚不知,不知乃知也。以不知为贵,因循长养,不戾自然之性,故以不可奈何为实也。

【校】自“有道之主”以下,亦见《淮南·主术训》,一二文异,不复别出。此为“实”,旧校云“一作‘宝’”,则正与《淮南》合。观此注意,似亦当作“宝”为是。

(11) 烛,照。

(12) 荒,裔远也。

(13) 皆四夷之远国。际,至也。

赵襄子之时,以任登为中牟令。 (1) 上计,言于襄子曰:“中牟有士曰胆、胥己,请见之。” (2) 襄子见而以为中大夫。 (3) 相国曰:“意者君耳而未之目邪?为中大夫若此其易也? (4) 非晋国之故。” (5) 襄子曰:“吾举登也,已耳而目之矣。登所举,吾又耳而目之, (6) 是耳目人终无已也。”遂不复问,而以为中大夫。襄子何为任人,则贤者毕力。 (7) 人主之患,必在任人而不能用之,用之而与不知者议之也。绝江者托于船,致远者托于骥,霸王者托于贤。伊尹、吕尚、管夷吾、百里奚,此霸王者之船骥也。释父兄与子弟,非疏之也; (8) 任庖人钓者与仇人仆虏,非阿之也。持社稷立功名之道,不得不然也。 (9) 犹大匠之为宫室也,量小大而知材木矣,訾功丈而知人数矣。 (10) 故小臣、吕尚听,而天下知殷、周之王也; (11) 管夷吾、百里奚听, (12) 而天下知齐、秦之霸也。岂特骥远哉? (13)

(1) 【校】《韩非·外储说左上》“任登”作“王登”。

(2) 【校】《韩非》作“中章、胥己”,是二人。下云“一日而见二中大夫”。

(3) 以,用也。

(4) 【校】“易”旧作“见”,讹,今案文义改正。

(5) 故,法。

(6) 谓耳任登之名,目任登之实,登之所举,岂复假耳目哉?

【校】旧本“吾又耳而目之”下亦有“矣”字,今从《韩非》去之。

(7) 毕,尽也。

(8) 言其父兄子弟不肖,不能为霸王之船骥,故释之,非苟远也。

(9) 庖人即伊尹,钓者即吕尚,仇人即管夷吾,仆虏即百里奚之辈。非阿之,取其可以为社稷功名之道。

(10) 訾,相也。相功力丈尺,而知用人数多少也。

【校】《说苑·尊贤》篇作“比功校而知人数矣”。

(11) 殷之尽,周之兴。

【校】此注误。小臣,汤之师也,谓伊尹,见《尊师》篇。

(12) 【校】旧校云:“一作‘任’。”案:《说苑》作“任”。

(13) 【校】当作“岂特船骥哉”。《说苑》作“岂特船乘哉”。

夫成王霸者固有人,亡国者亦有人。桀用羊辛, (1) 纣用恶来,宋用唐鞅, (2) 齐用苏秦,而天下知其亡。 (3) 非其人而欲有功,譬之若夏至之日而欲夜之长也, (4) 射鱼指天而欲发之当也。 (5) 舜、禹犹若困,而况俗主乎? (6)

(1) 【校】说见《当染》篇。

(2) 【校】从《说苑》作“唐鞅”,亦见《当染》篇,旧本作“ 唐”,误。

(3) 【校】旧本无“知”字,又“其”字讹作“甚”,今亦从《说苑》改正。

(4) 【校】“若”,《说苑》作“苦”。

(5) 当,中。

(6) 【校】“若”,《说苑》作“亦”。





慎势


六曰:

失之乎数,求之乎信,疑。 (1) 失之乎势,求之乎国,危。 (2) 吞舟之鱼,陆处则不胜蝼蚁。 (3) 权钧则不能相使,势等则不能相并,治乱齐则不能相正。故小大、轻重、少多、治乱,不可不察, (4) 此祸福之门也。

(1) 失诚信之数,欲人信之,故疑。

(2) 失居上之势,以恃有国,故危也。

(3) 蝼蚁食也。

(4) 察,知也。

凡冠带之国,舟车之所通, (1) 不用象译狄鞮,方三千里。 (2) 古之王者,择天下之中而立国, (3) 择国之中而立宫,择宫之中而立庙。天下之地,方千里以为国,所以极治任也。非不能大也,其大不若小,其多不若少。 (4) 众封建,非以私贤也,所以便势全威, (5) 所以博义。义博利则无敌, (6) 无敌者安。故观于上世,其封建众者,其福长,其名彰。神农十七世有天下,与天下同之也。 (7)

(1) 通,达。

(2) 《周礼》:“象胥掌蛮、夷、闽、越、戎、狄之国使,传通其言也。”东方曰羁,南方曰象,西方曰狄鞮,北方曰译。《国语》所谓曰羁南三千里内,被服五常,华夏之盛明,胡不用象译狄鞮也。

【校】注“象胥”下旧本衍“古”字,今删。“闽越”,《周礼》作“闽貊”。《王制》“东方曰寄”,此作“羁”,未详何出。“《国语》所谓曰羁南”七字疑衍文。“胡”字亦疑衍。

(3) 国,千里之畿。

(4) 在德不在人,《传》曰“楚子观兵于周疆,问鼎之大小轻重焉。王孙满对曰‘在德不在鼎。德之休明,虽小,重;其奸回昏乱,虽大,轻’”是也。故曰“其大不若小,其多不若少”。

【校】注旧本作“在德之休明,虽大,轻”,文有脱漏,今依《传》补十二字。

(5) 众,多。

(6) 【校】孙云:“李善注《文选》陆士衡《五等论》引作‘所以博利博义也,利博义博则无敌也’。”

(7) 神农,炎帝也。农植嘉谷,化养兆民,天下号之曰“神农”。

王者之封建也,弥近弥大,弥远弥小。 (1) 海上有十里之诸侯。 (2) 以大使小,以重使轻,以众使寡,此王者之所以家以完也。 (3) 故曰,以滕、费则劳,以邹、鲁则逸, (4) 以宋、郑则犹倍日而驰也, (5) 以齐、楚则举而加纲旃而已矣。 (6) 所用弥大,所欲弥易。 (7) 汤其无郼,武其无岐,贤虽十全,不能成功。 (8) 汤、武之贤,而犹藉知乎势,又况不及汤、武者乎?故以大畜小吉,以小畜大灭, (9) 以重使轻从, (10) 以轻使重凶。 (11) 自此观之,夫欲定一世,安黔首之命,功名著乎槃盂,铭篆著乎壶鉴,其势不厌尊,其实不厌多。多实尊势,贤士制之,以遇乱世,王犹尚少。 (12)

(1) 近国大,远国小,强干弱枝。

(2) 海上,四海之上,言远也。十里,小国。

(3) 家,室也。王者以天下为家,故所以天下为国。

(4) 滕、费小,故劳也。邹、鲁大,故逸也。

(5) 倍日而驰,以行其威易也。

(6) 齐、楚最大,举纲纪加之于小国,无大劳,故曰“而已矣”。

(7) 用大使小,欲尽济,故曰“弥易”。

(8) 郼、岐,汤、武之本国。假令无之,贤虽十倍,不能以成功业。

【校】郼,说见《慎大》篇。

(9) 灭,亡也。

(10) 从,顺。

(11) 凶,逆也。

(12) 以尊势贤士之佐,遇乱世,而王尚为少。

天下之民穷矣苦矣。民之穷苦弥甚,王者之弥易。 (1) 凡王也者,穷苦之救也。水用舟,陆用车,涂用輴,沙用鸠,山用樏, (2) 因其势也者令行。 (3) 位尊者其教受, (4) 威立者其奸止,此畜人之道也。故以万乘令乎千乘易,以千乘令乎一家易,以一家令乎一人易。尝识及此,虽尧、舜不能。 (5) 诸侯不欲臣于人而不得已,其势不便,则奚以易臣? (6) 权轻重,审大小,多建封,所以便其势也。王也者,势也。王也者,势无敌也。势有敌则王者废矣。有知小之愈于大、少之贤于多者,则知无敌矣。知无敌,则似类嫌疑之道远矣。故先王之法,立天子不使诸侯疑焉,立诸侯不使大夫疑焉,立適子不使庶孽疑焉。 (7) 疑生争,争生乱。是故诸侯失位则天下乱,大夫无等则朝廷乱,妻妾不分则家室乱,適孽无别则宗族乱。慎子曰:“今一兔走,百人逐之, (8) 非一兔足为百人分也,由未定。 (9) 由未定,尧且屈力,而况众人乎? (10) 积兔满市,行者不顾, (11) 非不欲兔也,分已定矣。分已定,人虽鄙不争。”故治天下及国,在乎定分而已矣。 (12)

(1) 苦纣之民纣之乱,与武王陈其牧野,倒矢而射,横戈而战,武王由是弥易。

(2) 【校】案:《文子·自然》篇“水用舟,沙用 ,泥用 ,山用樏”,《释音》云“ ,乃鸟切,推版具”。又《淮南·齐俗训》“譬若舟车 穷庐”,叶林宗本作“ ”,俗本作“鸠”,至《修务训》叶本亦作“鸠”矣。

(3) 【校】案:“因其势也”下,似当云“因其势者其令行”,补四字语气方完。

(4) 受,因。

(5) 不能以行其化。

【校】“尝识及此”,疑是“尝试反此”。

(6) 奚,何也。

(7) 尊卑皆有别。

(8) 慎子名到,作《法书》四十二篇,在申不害、韩非前,申、韩称之也。

【校】注旧本作“四十一篇”,今据《汉书·艺文志》改。

(9) 未定者,人欲望之也。

(10) 屈,竭也。

(11) 顾,视。

(12) 分土画界,各守其封,故定分也。

【校】注“定分”似当作“分定”。

庄王围宋九月, (1) 康王围宋五月, (2) 声王围宋十月。 (3) 楚三围宋矣,而不能亡。非不可亡也,以宋攻楚,奚时止矣? (4) 凡功之立也,贤不肖强弱治乱异也。

(1) 庄王,楚穆王子,共王父也。围宋在鲁宣十五年。

【校】《春秋》围宋在宣十四年之秋,逾年而始与平,故高注每云十五年。

(2) 康王,楚共王审之子,庄王之孙也。宋君病,不以告,故不书于《经》。

(3) 声王,楚惠王熊章之孙,简王之子,在春秋后。

(4) 宋无德,楚亦无德,故曰“以宋攻楚”也。

齐简公有臣曰诸御鞅,谏于简公曰:“陈成常与宰予,之二臣者,甚相憎也。 (1) 臣恐其相攻也,相攻唯固则危上矣。愿君之去一人也。” (2) 简公曰:“非而细人所能识也。” (3) 居无几何,陈成常果攻宰予于庭,即简公于庙。 (4) 简公喟焉太息曰:“余不能用鞅之言,以至此患也。”失其数,无其势,虽悔无听鞅也与无悔同, (5) 是不知恃可恃而恃不恃也。周鼎著象,为其理之通也。理通,君道也。 (6)

(1) 简公,悼公阳生之子壬也。陈成常,陈乞之子恒也。宰予字子我。

【校】注“壬”,旧本作“王子”,讹,今改正。阚止字子我,诸子遂误以为宰予。

(2) 相憎不可并也,故愿去一人。

(3) 【校】旧校云:“‘而’一作‘汝’,‘识’一作‘议’。”

(4) 【校】《说苑·正谏》篇作“贼简公于朝”。

(5) 悔,恨。

(6) 【校】周鼎著象,详见《先识览》。





不二


七曰:

听群众人议以治国,国危无日矣。 (1) 何以知其然也?老耽贵柔,孔子贵仁,墨翟贵廉,关尹贵清, (2) 子列子贵虚, (3) 陈骈贵齐, (4) 阳生贵己, (5) 孙膑贵势, (6) 王廖贵先,兒良贵后。 (7) 此十人者,皆天下之豪士也。 (8) 有金鼓,所以一耳; (9) 必同法令,所以一心也;智者不得巧,愚者不得拙,所以一众也;勇者不得先,惧者不得后,所以一力也。故一则治,异则乱;一则安,异则危。夫能齐万不同,愚智工拙皆尽力竭能,如出乎一穴者, (10) 其唯圣人矣乎!无术之智,不教之能,而恃强速贯习,不足以成也。

(1) 听,从也。听从众人之议,人心不同,如其面焉,故国不能安宁也。《诗》曰“如彼筑室于道谋,是用不溃于成”,此之谓也。

(2) 关尹,关正也,名喜,作《道书》九篇。能相风角,知将有神人,而老子到,喜说之,请著《上至经》五千言,而从之游也。

【校】“老耽”,《困学纪闻》十引仍作“老聃”。

(3) 体道人也,壶子弟子。

(4) 陈骈,齐人也,作《道书》二十五篇。贵齐,齐死生,等古今也。

【校】注旧本作“一十五篇”,今据《汉书·艺文志》改。

(5) 轻天下而贵己。《孟子》曰:“阳子拔体一毛以利天下弗为也。”

【校】李善注《文选》谢灵运《述祖德》诗引作杨朱。“阳”、“杨”古多通用。

(6) 孙膑,楚人,为齐臣,作《谋》八十九篇,权之势也。

【校】梁伯子云:“《史》、《汉》皆以孙膑为齐人,此独以为楚人,当别有据。”

(7) 王廖谋兵事贵先建策也。兒良作兵谋贵后。

(8) 【校】旧本无此十一字,孙云:“李善注《文选》贾谊《过秦论》、陆士衡《豪士赋》序皆有。”今据补。卢云:“此下疑所脱尚多,引此十人,必不如是而止,应有断制语,前《安死》篇‘故反以相非’一段颇似此处文,又此下段亦必别有发端语,而今无从考补矣。”

(9) 金,钟也。击金则退,击鼓则进。

(10) 【校】旧校云:“‘穴’一作‘空’。”案:“空”与“孔”同。





执一


八曰:

天地阴阳不革,而成万物不同。 (1) 目不失其明,而见白黑之殊。耳不失其听,而闻清浊之声。 (2) 王者执一,而为万物正。 (3) 军必有将,所以一之也; (4) 国必有君,所以一之也;天下必有天子,所以一之也;天子必执一,所以抟之也。 (5) 一则治,两则乱。今御骊马者,使四人人操一策,则不可以出于门闾者,不一也。 (6)

(1) 革,改也。不同,区以别也。

(2) 清,商。浊,宫。

(3) 一者平。正者主。

(4) 将,主。

(5) 【校】“抟”与“专”同。说见前。旧作“搏”,讹。

(6) 骊马, 马也。在中曰服,在边曰 。策,辔策也。御四马者六辔,乃四人持,故曰“不一”。

楚王问为国于詹子, (1) 詹子对曰:“何闻为身,不闻为国。” (2) 詹子岂以国可无为哉?以为为国之本,在于为身。身为而家为,家为而国为,国为而天下为。故曰以身为家,以家为国,以国为天下。 (3) 此四者,异位同本。故圣人之事,广之则极宇宙,穷日月, (4) 约之则无出乎身者也。慈亲不能传于子,忠臣不能入于君,唯有其材者为近之。 (5)

(1) 詹何,隐者。

(2) 身治国乱,未之有也,故曰“为身”。

【校】为训治也。《意林》两“为”字即改作“治”。

(3) 为,治。

(4) 穷亦极也。

(5) 近犹知也。

田骈以道术说齐,齐王应之曰:“寡人所有者,齐国也,愿闻齐国之政。”田骈对曰:“臣之言,无政而可以得政。譬之若林木,无材而可以得材。 (1) 愿王之自取齐国之政也。”骈犹浅言之也,博言之,岂独齐国之政哉?变化应求而皆有章,因性任物而莫不宜当, (2) 彭祖以寿,三代以昌, (3) 五帝以昭,神农以鸿。 (4)

(1) 材从林生。

(2) 当,合。

(3) 彭祖,殷贤大夫,治性,寿益七百。《论语》曰“窃比于我老彭”,此之谓也。三代,夏、殷、周,以治性而昌盛。

(4) 五帝:黄帝轩辕、颛顼高阳、帝喾高辛、帝尧陶唐、帝舜有虞。神农炎帝,三皇之一也。皆以治世体道。昭,明;鸿,盛也。

吴起谓商文曰:“事君果有命矣夫!” (1) 商文曰:“何谓也?”吴起曰:“治四境之内,成训教,变习俗,使君臣有义,父子有序,子与我孰贤?”商文曰:“吾不若子。” (2) 曰:“今日置质为臣,其主安重; (3) 今日释玺辞官,其主安轻。子与我孰贤?”商文曰:“吾不若子。” (4) 曰:“士马成列,马与人敌,人在马前,援桴一鼓,使三军之士乐死若生,子与我孰贤?”商文曰:“吾不若子。”吴起曰:“三者子言不吾若也,位则在吾上,命也夫事君!” (5) 商文曰:“善。子问我,我亦问子。世变主少,群臣相疑,黔首不定, (6) 属之子乎?属之我乎?”吴起默然不对,少选曰:“与子。” (7) 商文曰:“是吾所以加于子之上已!”吴起见其所以长,而不见其所以短;知其所以贤,而不知其所以不肖。故胜于西河,而困于王错, (8) 倾造大难,身不得死焉。 (9) 夫吴胜于齐, (10) 而不胜于越。 (11) 齐胜于宋, (12) 而不胜于燕。 (13) 故凡能全国完身者,其唯知长短赢绌之化邪!

(1) 吴起,卫人,为楚将,又相魏,为西河太守。商文,盖魏臣也。

【校】梁仲子云:“商文,《史记·吴起传》作‘田文’,与孟尝君同姓名。”

(2) 若,如也。

(3) 置犹委也。

(4) 子谓吴起。

【校】此可不注,又不应在次见下,得无后人所为乎?

(5) 言事君由天命。

(6) 【校】孙云:“《御览》四百四十六字此下有‘当此之时’四字。”

(7) 少选,须臾也。与犹归。

(8) 王错谮之于武侯,故曰“困于王错”。

(9) 大难,车裂之难。

【校】卢云:“起后在楚,事悼王。王死,贵人相与射起,起伏王尸而死。见《史记》本传。此书后《贵卒》篇亦同。至《战国·秦策》,《韩非》《难言》、《问田》两篇,《史记·蔡泽传》,皆言起支解,此亦可信。既攒射矣,何必不臠割?唯此注言车裂则非是。”

(10) 吴王夫差破齐于艾陵。

(11) 越王句践破吴王夫差于五湖。

(12) 齐宣王伐宋而胜。

【校】案《史表》,灭宋者,齐湣王也。

(13) 燕昭使乐毅伐齐,下其城七十二也。




————————————————————

[1] 扰乱:原本作“乱扰”,据乾隆本改。





第十八卷 审应览



审应


一曰:

人主出声应容,不可不审。凡主有识,言不欲先。 (1) 人唱我和,人先我随,以其出为之入,以其言为之名,取其实以责其名, (2) 则说者不敢妄言, (3) 而人主之所执其要矣。 (4)

(1) 《淮南记》曰:“先唱者穷之路,后动者达之原也。”故言动欲后。

(2) 实,德行之实也;名,德行之名也。盖虚名可以伪致,显实难以诈成,故以其实考责其名也。

【校】注“盖虚名可以伪致”,旧本多作“虚称不可以为致”,今从刘本改正。

(3) 其为名实不相当也。

(4) 要,约也。

孔思请行,鲁君曰:“天下主亦犹寡人也,将焉之?” (1) 孔思对曰:“盖闻君子犹鸟也,骇则举。”鲁君曰:“主不肖而皆以然也,违不肖,过不肖,而自以为能论天下之主乎?凡鸟之举也,去骇从不骇。 (2) 去骇从不骇,未可知也。去骇从骇,则鸟曷为举矣?”孔思之对鲁君也,亦过矣。

(1) 孔思,子思,伯鱼之子也。行,去。之,他也。

(2) 骇,扰也。

魏惠王使人谓韩昭侯曰:“夫郑乃韩氏亡之也,愿君之封其后也。 (1) 此所谓存亡继绝之义。君若封之,则大名。”昭侯患之。公子食我曰:“臣请往对之。”公子食我至于魏,见魏王曰:“大国命弊邑封郑之后,弊邑不敢当也。弊邑为大国所患,昔岀公之后声氏为晋公,拘于铜鞮,大国弗怜也,而使弊邑存亡继绝,弊邑不敢当也。” (2) 魏王惭曰:“固非寡人之志也,客请勿复言。” (3) 是举不义以行不义也。魏王虽无以应,韩之为不义愈益厚也。 (4) 公子食我之辩,适足以饰非遂过。 (5)

(1) 惠王,魏武侯子也,孟子所见梁惠王也。韩哀侯灭郑,初兼其国。昭侯,哀侯之孙也,故适使封郑之后。

(2) 大国,谓魏国也。言韩当为大国所忧。出公、声氏,韩之先君也。曾为晋公所执于铜鞮,魏国不救,故曰大国不怜也。欲使韩封郑之后,故曰“弊邑不敢当也”。

(3) 言封郑非寡人意也,故令客勿复言也。

(4) 厚,多也。

(5) 饰好其非,遂成其过。

魏昭王问于田诎曰:“寡人之在东宫之时, (1) 闻先生之议曰‘为圣易’,有诸乎?” (2) 田诎对曰:“臣之所举也。” (3) 昭王曰:“然则先生圣于?” (4) 田诎对曰:“未有功而知其圣也,是尧之知舜也;待其功而后知其舜也,是市人之知圣也。今诎未有功,而王问诎曰‘若圣乎’,敢问王亦其尧邪?”昭王无以应。田诎之对,昭王固非曰“我知圣也”耳,问曰“先生其圣乎”,己因以知圣对昭王, (5) 昭王有非其有,田诎不察。 (6)

(1) 昭王,哀王之子也。东宫,世子也。《诗》云:“东宫之妹,邢候之姨。”

【校】注旧本作“昭王,襄王之子”,讹,据《魏世家》改正。

(2) 有是言不?

【校】注末旧衍“可”字,今删。

(3) 言有是言。

(4) 于,乎也。

【校】卢云:“古‘于’、‘乎’通。《列子·黄帝》篇‘今汝之鄙至此乎’,殷敬顺《释文》云‘本又作于’。”

(5) 己,谓田诎。

(6) 察,知也。

赵惠王谓公孙龙曰:“寡人事偃兵十余年矣而不成,兵不可偃乎?” (1) 公孙龙对曰:“偃兵之意,兼爱天下之心也。兼爱天下,不可以虚名为也,必有其实。 (2) 今蔺、离石入秦, (3) 而王缟素布总; (4) 东攻齐得城,而王加膳置酒。 (5) 秦得地而王布总, (6) 齐亡地而王加膳, (7) 所非兼爱之心也。 (8) 此偃兵之所以不成也。”今有人于此,无礼慢易而求敬,阿党不公而求令,烦号数变而求静,暴戾贪得而求定,虽黄帝犹若困。 (9)

(1) 惠王,赵襄子后七世武灵王之子,吴娃所生。事,治。偃,止也。

【校】注“吴娃”,旧本作“吴姬”,讹,今改正。

(2) 虚,空。实,诚。

(3) 二县叛赵自入于秦也,今属西河。

(4) 缟素布总,丧国之服。

【校】旧本“布”作“出”,校云“一作‘布’”。今案“出”明是讹字,故竟定作“布”。

(5) 得国之乐也。言王不兼爱也。

(6) 秦得蔺、离石也。

(7) 置酒而为欢。

(8) 【校】“所非”疑是“此非”。

(9) 困,不能谐。

卫嗣君欲重税以聚粟,民弗安,以告薄疑曰:“民甚愚矣。 (1) 夫聚粟也,将以为民也。其自藏之与在于上,奚择?” (2) 薄疑曰:“不然。其在于民而君弗知, (3) 其不如在上也; (4) 其在于上而民弗知,其不如在民也。” (5) 凡听必反诸己,审则令无不听矣。 (6) 国久则固,固则难亡。今虞、夏、殷、周无存者,皆不知反诸己也。

(1) 嗣君,蒯聩后八世平侯之子也,秦贬其号为君。薄疑其臣也,故以重税告之,谓民为愚。

【校】注旧本“后”下衍一“也”字,今删。以蒯聩后为君者谓之则八世,以序次言之实六世也。

(2) 言民自藏粟于家与藏之于官何择。择,失也。

【校】注“失也”似当作“异也”,见下注。

(3) 知犹得也。

(4) 为官言,不如其在上。上谓官。

(5) 为民言,不如在于民。

(6) 听,从。

公子沓相周,申向说之而战。 (1) 公子沓訾之曰:“申子说我而战,为吾相也夫?” (2) 申向曰:“向则不肖,虽然,公子年二十而相,见老者而使之战,请问孰病哉? (3) ”公子沓无以应。 (4) 战者,不习也; (5) 使人战者,严驵也。 (6) 意者恭节而人犹战,任不在贵者矣。故人虽时有自失者,犹无以易恭节。自失不足以难,以严驵则可。 (7)

(1) 申向,周人,申不害之族也。为公子沓相,说,见公子而战。战,惧也。

(2) 訾,毁也。说我,我说之也,而战惧。毁之,言不任为吾相也夫。不满之辞。

【校】此两节注皆非是。公子沓为周之相,非申向相公子沓也。毁其说我而战惧,将以我为相尊严之故而然欤?如是与下文皆吻合。今注乃言公子沓以申向不任为吾相,大谬。

(3) 孰,谁也。

(4) 应,答也。

(5) 不惯习见尊者,故惧而战。

(6) 严,尊。驵,骄。

【校】案:“驵”与“怚”、“姐”同。

(7) 言以严驵者,失则可也。





重言


二曰:

人主之言,不可不慎。高宗,天子也,即位谅暗,三年不言。 (1) 卿大夫恐惧,患之。 (2) 高宗乃言曰:“以余一人正四方,余唯恐言之不类也,兹故不言。” (3) 古之天子,其重言如此,故言无遗者。 (4)

(1) 高宗,殷王盘庚之弟小乙之子也,德义高美,殷人尊之,故曰“高宗”。谅暗,三年不言,在小乙之丧也。《论语》曰:“高宗谅暗,三年不言,何谓也。孔子曰:‘古之人皆然。君薨,百官总己,听于冢宰三年。’”此之谓也。

(2) 患,忧。

(3) 类,善。兹,此。

(4) 遗,失也。

成王与唐叔虞燕居,援梧叶以为珪,而授唐叔虞曰:“余以此封女。” (1) 叔虞喜,以告周公。周公以请曰:“天子其封虞邪?”成王曰:“余一人与虞戏也。” (2) 周公对曰:“臣闻之,天子无戏言。天子言,则史书之,工诵之,士称之。”于是遂封叔虞于晋。 (3) 周公旦可谓善说矣,一称而令成王益重言,明爱弟之义,有辅王室之固。 (4)

(1) 削桐叶以为珪,冒以授叔虞。《周礼》“侯执信圭,七寸”,故曰“余以此封女”。

(2) 戏,不诚也。

【校】《说苑·君道》篇无“人”字,是。

(3) 叔虞,成王之母弟也。《传》曰:“当武王邑姜方娠太叔,梦天帝谓己曰:‘余命而子曰虞,将与之唐。’及生,有文在其手曰‘虞’,遂以命之。及成王灭唐,而封太叔为晋侯。”此之谓也。

(4) 辅,正。

荆庄王立三年,不听而好 。 (1) 成公贾入谏, (2) 王曰:“不穀禁谏者,今子谏,何故?” (3) 对曰:“臣非敢谏也,愿与君王 也。”王曰:“胡不设不穀矣?” (4) 对曰:“有鸟止于南方之阜,三年不动不飞不鸣,是何鸟也?”王射之 (5) 曰:“有鸟止于南方之阜,其三年不动,将以定志意也;其不飞,将以长羽翼也;其不鸣,将以览民则也。 (6) 是鸟虽无飞,飞将冲天,虽无鸣,鸣将骇人。 (7) 贾出矣,不穀知之矣。”明日朝,所进者五人,所退者十人。群臣大说,荆国之众相贺也。故《诗》曰:“何其久也,必有以也。何其处也,必有与也。”其庄王之谓邪!成公贾之 也,贤于太宰嚭之说也。太宰嚭之说,听乎夫差,而吴国为墟; (8) 成公贾之 ,喻乎荆王,而荆国以霸。 (9)

(1) 庄王,楚缪王商臣之子旅也。 ,谬言。

【校】案: ,廋辞也。《史记·滑稽传》作“喜隐”。

(2) 【校】孙云:“《史记·楚世家》作‘五举’,《新序·杂事二》作‘士庆’,《滑稽传》又以为淳于髡说齐威王。”

(3) 禁,止也。

(4) 设,施也。何不施 言于不穀也。

(5) 使王射不动不鸣何意也。

(6) 览,观。

(7) 冲,至也。骇,惊也。

(8) 嚭,晋柏州犁之子。州犁奔楚,嚭自楚之吴,以为太宰。

(9) 庄王霸。

齐桓公与管仲谋伐莒,谋未发而闻于国。 (1) 桓公怪之曰:“与仲父谋伐莒,谋未发而闻于国,其故何也?”管仲曰:“国必有圣人也。”桓公曰:“嘻!日之役者,有执蹠 而上视者, (2) 意者其是邪?”乃令复役,无得相代。少顷,东郭牙至。 (3) 管仲曰:“此必是已。”乃令宾者延之而上,分级而立。 (4) 管子曰:“子邪言伐莒者?”对曰:“然。” (5) 管仲曰:“我不言伐莒,子何故言伐莒?”对曰:“臣闻君子善谋,小人善意。臣窃意之也。”管仲曰:“我不言伐莒,子何以意之?”对曰:“臣闻君子有三色:显然喜乐者,钟鼓之色也;湫然清静者,衰绖之色也;艴然充盈,手足矜者,兵革之色也。 (6) 日者臣望君之在台上也,艴然充盈,手足矜者,此兵革之色也。君呿而不唫, (7) 所言者莒也;君举臂而指,所当者莒也。臣窃以虑诸侯之不服者,其惟莒乎!臣故言之。”凡耳之闻以声也。今不闻其声,而以其容与臂,是东郭牙不以耳听而闻也。桓公、管仲虽善匿,弗能隐矣。 (8) 故圣人听于无声,视于无形,詹何、田子方、老耽是也。 (9)

(1) 发,行。闻,知。

(2) 蹠,逾。

【校】 字无考。注以逾训蹠,亦难晓。《说苑·权谋》篇作“执柘杵”。梁仲子云:“《墨子·备穴》篇云‘用 若松为穴户’, 不知何物,字与 相似。”

(3) 【校】《说苑》作“东郭垂”。

(4) 延,引。级,阶陛。

(5) 子,谓东郭牙。牙曰“然”也。

【校】“管子”亦当作“管仲”。“子邪言伐莒者”,文似倒而实顺。注“牙”字旧本不重,今案文义补之。

(6) 矜,严也。

【校】“显然喜乐”,《意林》作“欢然喜乐”,旧本《吕氏》作“善乐”。又“清静”,《意林》作“清净”,本亦多同,唯李本作“静”。又“艴”作“沸”。《说苑》字句亦间不同,今不悉记。

(7) 呿,开。唫,闭。

【校】“唫”,本或作“ ”,《说苑》作“吁而不吟”。

(8) 匿,藏。隐,蔽。

(9) 詹何,体道人也。田子方学于子贡,尚贤仁而贵礼义,魏文侯友之。老耽学于无为而贵道德,周史伯阳也,三川竭,知周将亡,孔子师之也。





精谕


三曰:

圣人相谕不待言,有先言言者也。海上之人有好蜻者, (1) 每居海上, (2) 从蜻游,蜻之至者百数而不止,前后左右尽蜻也, (3) 终日玩之而不去。 (4) 其父告之曰:“闻蜻皆从女居, (5) 取而来,吾将玩之。”明日之海上,而蜻无至者矣。 (6)

(1) 【校】《列子·黄帝》篇作“有好沤鸟者”,下并同。

(2) 【校】孙云:“李善注《文选》江文通《拟阮步兵》诗作‘每朝居海上’,《御览》九百五十同。”

(3) 蜻,蜻蜓,小虫,细腰四翅,一名白宿。

(4) 玩,弄。

(5) 居,所。

【校】注颇僻,似不若训处。或本作古“処”字,而传写讹“所”。

(6) 【校】孙云:“《选》注沈休文《咏湖中雁》诗作‘群蜻翔而不下’。”

胜书说周公旦曰 (1) :“廷小人众,徐言则不闻,疾言则人知之。徐言乎?疾言乎?”周公旦曰:“徐言。”胜书曰:“有事于此,而精言之而不明,勿言之而不成。精言乎?勿言乎?” (2) 周公旦曰:“勿言。”故胜书能以不言说,而周公旦能以不言听。此之谓不言之听。不言之谋,不闻之事,殷虽恶周,不能疵矣。 (3) 口 不言,以精相告,纣虽多心,弗能知矣。 (4) 目视于无形,耳听于无声,商闻虽众,弗能窥矣。 (5) 同恶同好,志皆有欲,虽为天子,弗能离矣。

(1) 【校】《韩诗外传》四但作“客”。《说苑·指武》篇作“王满生”。

(2) 精,微。勿,无。

(3) 疵,病。

【校】《外传》、《说苑》皆作诛管、蔡事。

(4) 纣多恶周之心,不能知周必病。

【校】注“必病”下似当有一“己”字。

(5) 窥犹见。

孔子见温伯雪子,不言而出。 (1) 子贡曰:“夫子之欲见温伯雪子好矣, (2) 今也见之而不言,其故何也?”孔子曰:“若夫人者,目击而道存矣,不可以容声矣。” (3) 故未见其人而知其志,见其人而心与志皆见,天符同也。 (4) 圣人之相知,岂待言哉?

(1) 伯雪子,得道人。

(2) 【校】孙云:“《庄子·田子方》篇‘子贡’作‘子路’,‘好矣’作‘久矣’。”

(3) 【校】旧校云:“‘击’一作‘解’。”

(4) 符,道也。同,合也。

白公问于孔子曰:“人可与微言乎?”孔子不应。 (1) 白公曰:“若以石投水,奚若?” (2) 孔子曰:“没人能取之。” (3) 白公曰:“若以水投水,奚若?”孔子曰:“淄、渑之合者,易牙尝而知之。” (4) 白公曰:“然则人不可与微言乎?”孔子曰:“胡为不可?唯知言之谓者为可耳。” (5) 白公弗得也。 (6) 知谓则不以言矣。 (7) 言者,谓之属也。 (8) 求鱼者儒,争兽者趋, (9) 非乐之也。故至言去言, (10) 至为无为。 (11) 浅智者之所争则末矣,此白公之所以死于法室。 (12)

(1) 白公,楚平王之孙,太子建之子胜也。白,楚县也。楚僭称王,守县大夫皆称公。太子建为费无极所谮,出奔郑,与晋通谋,欲反郑于晋,郑人杀之。胜与庶父令尹子西、司马子期伐郑,报父之仇,许而未行。晋人伐郑,子西、子期率师救郑。胜怒曰:“郑人在此,仇不远矣。”欲杀子西、子期,故问微言。微言,阴谋密事也。孔子知之,故不应之。

【校】注“胜与庶父”,当作“胜请庶父”。

(2) 喻微言若石沉没水中,人不知。

(3) 没行水中之人能取之。

(4) 淄、渑,齐之两水名也。易牙,齐桓公识味臣也,能别淄、渑之味也。

(5) 知言,言仁言义。言忠信仁义大行于民,民欣而戴之,则可用也。

(6) 弗得,不得知言之言。

(7) 不欲白公以微言言。

(8) 谓不仁不义之言。

(9) 【校】《列子·说符》篇作“争鱼者濡,逐兽者趋”,《文子·微明》篇亦同。

(10) 去不仁不义之言。

(11) 至德之人,为乃无为。无为因天无为,天无为而万物成,乃有为也,故至德之人能体之也。

(12) 末,小也。白公不能蹈无为,遂行其志,杀子西、子期而有荆国。叶公子高率方城外众攻白公,九日而杀之法室。法室,司寇也。一曰浴室,澡浴之室也。

【校】《列子》及《淮南·道应训》俱作“浴室”。

齐桓公合诸侯, (1) 卫人后至。公朝而与管仲谋伐卫,退朝而入,卫姬望见君,下堂再拜,请卫君之罪。公曰:“吾于卫无故,子曷为请?”对曰:“妾望君之入也,足高气强,有伐国之志也。见妾而有动色,伐卫也。”明日君朝,揖管仲而进之。管仲曰:“君舍卫乎?”公曰:“仲父安识之?”管仲曰:“君之揖朝也恭,而言也徐,见臣而有惭色,臣是以知之。”君曰:“善。仲父治外,夫人治内,寡人知终不为诸侯笑矣。”桓公之所以匿者不言也,今管子乃以容貌音声,夫人乃以行步气志,桓公虽不言,若暗夜而烛燎也。

(1) 合,会也。

晋襄公使人于周曰:“弊邑寡君寝疾,卜以守龟,曰:‘三涂为祟。’弊邑寡君使下臣愿藉途而祈福焉。” (1) 天子许之。 (2) 朝,礼使者事毕,客出。苌弘谓刘康公曰:“夫祈福于三涂而受礼于天子,此柔嘉之事也,而客武色,殆有他事,愿公备之也。” (3) 刘康公乃儆戎车卒士以待之。晋果使祭事先,因令杨子将卒十二万而随之,涉于棘津,袭聊、阮、梁蛮氏,灭三国焉。此形名不相当,圣人之所察也,苌弘则审矣。故言不足以断小事,唯知言之谓者可为。

(1) 三涂之山,陆浑之南,故假道于周也。襄公,文公之子 也。按《春秋经》,襄公以鲁僖三十三年即位,至鲁文公六年卒,无卜三涂为祟之言也。《鲁昭十七年传》曰:“晋侯使屠蒯如周,请事于洛与三涂。苌弘谓刘子:‘客容猛,非祥也,其伐戎乎?陆浑睦于楚,必是故也。君其备之。’乃儆戎备。九月丁卯,晋荀吴帅师涉自棘津,使祭史先用牲于洛,陆浑人不知师从之。庚午,遂灭陆浑,数之以其贰于楚也。”计襄公卒至此,乃九十六年,历世亡失。按《传》,晋顷公也。此云襄公,复妄言也。

【校】注引《传》多讹,今悉据《传》文改正。唯“非祭也”作“非祥也”,误涉《昭十五年传》“非祭祥也”之文。

(2) 天子,周景王。

(3) 晋襄公,周襄王时也。苌弘乃景王、敬王之大夫,春秋之末也。以世推之,当为晋顷公,其不得为襄公明矣。





离谓


四曰:

言者以谕意也,言意相离,凶也。乱国之俗,甚多流言,而不顾其实,务以相毁,务以相誉,毁誉成党, (1) 众口熏天, (2) 贤不肖不分,以此治国,贤主犹惑之也, (3) 又况乎不肖者乎?惑者之患,不自以为惑,故惑 (4) 惑之中有晓焉,冥冥之中有昭焉。 (5) 亡国之主,不自以为惑,故与桀、纣、幽、厉皆也。然有亡者国, (6) 无二道矣。

(1) 【校】旧校云:“‘毁誉’一作‘巧辞’。”

(2) 熏,感动也。

(3) 分,别。惑,疑。

(4) 句。

(5) 【校】“昭”字当重。

(6) 句。

郑国多相县以书者。子产令无县书,邓析致之。子产令无致书,邓析倚之。令无穷,则邓析应之亦无穷矣。是可不可无辨也。 (1) 可不可无辨,而以赏罚,其罚愈疾,其乱愈疾,此为国之禁也。 (2) 故辨而不当理则伪, (3) 知而不当理则诈。诈伪之民,先王之所诛也。理也者,是非之宗也。 (4)

(1) 辨,别。

(2) 为,治。禁,法。

(3) 伪,巧也。

(4) 宗,本也。

洧水甚大,郑之富人有溺者,人得其死者。 (1) 富人请赎之,其人求金甚多,以告邓析, (2) 邓析曰:“安之,人必莫之卖矣。” (3) 得死者患之,以告邓析,邓析又答之曰:“安之,此必无所更买矣。” (4) 夫伤忠臣者有似于此也。夫无功不得民,则以其无功不得民伤之;有功得民,则又以其有功得民伤之。 (5) 人主之无度者,无以知此,岂不悲哉?比干、苌弘以此死, (6) 箕子、商容以此穷, (7) 周公、召公以此疑, (8) 范蠡、子胥以此流, (9) 死生存亡 [1] 安危,从此生矣。 (10)

(1) 【校】“死”与“尸”同。《史记·秦本纪》“晋、楚流死河二万人”,《汉书·酷吏传》“安所求子死,桓东少年场”,此书《期贤》篇“扶伤舆死”亦是。《意林》作“有人得富者尸”。

(2) 【校】《意林》作“富人党以告邓析”。

(3) 【校】《意林》作“必无买此者”。

(4) 【校】《意林》作“必无人更买,义必无不赎”,下五字疑是注。

(5) 此邓析之谗辩,所以车裂而死。

(6) 以世诡辩,反白为黑,而主不知,故死。

(7) 箕子,纣之庶父也。商容,纣时贤人,老子所从学者也。以主不知,故穷。

(8) 以管、蔡流言,故疑也。《论语》曰“虽有周亲,不如仁人”,此之谓也。

【校】此引《论语》,不解所用意。

(9) 流,放。

(10) 此谗辩无理若邓析。

子产治郑,邓析务难之,与民之有狱者约:“大狱一衣,小狱襦袴。” (1) 民之献衣襦袴而学讼者,不可胜数。以非为是,以是为非,是非无度,而可与不可日变。 (2) 所欲胜因胜,所欲罪因罪。郑国大乱,民口喧哗。子产患之,于是杀邓析而戮之,民心乃服,是非乃定,法律乃行。今世之人,多欲治其国,而莫之诛邓析之类, (3) 此所以欲治而愈乱也。

(1) 【校】旧校云:“一作‘ ’,下同。”案《玉篇》:“ ,子愦切, 衣也。”

(2) 【校】旧校云:“‘日’一作‘因’。”

(3) 有如邓析者无能诛。

【校】案:《列子·力命》篇亦云“子产杀邓析”。考《左氏定九年传》“郑驷歂杀邓析而用其竹刑”,驷歂乃代子太叔为政者,则邓析、子产并不同时。张湛注《列子》云:“子产卒后二十年而邓析死也。”

齐有事人者,所事有难而弗死也。遇故人于涂,故人曰:“固不死乎?”对曰:“然。凡事人以为利也,死不利,故不死。”故人曰:“子尚可以见人乎?”对曰:“子以死为顾可以见人乎?” (1) 是者数传。不死于其君长,大不义也,其辞犹不可服,辞之不足以断事也明矣。夫辞者,意之表也。鉴其表而弃其意,悖。 (2) 故古之人得其意则舍其言矣。听言者以言观意也,听言而意不可知,其与桥言无择。 (3)

(1) 顾,反。

(2) 悖,惑。

(3) 桥,戾也。择犹异。

齐人有淳于髡者,以从说魏王。魏王辩之, (1) 约车十乘,将使之荆。辞而行,有以横说魏王,魏王乃止其行。 (2) 失从之意,又失横之事,夫其多能不若寡能, (3) 其有辩不若无辩。周鼎著倕而龁其指,先王有以见大巧之不可为也。 (4)

(1) 关东六国为从也。魏王以为辩达。

(2) 关西为横。髡以合关东从为未足,复说欲连关西之横,王多其言,故辍不使行之也。

【校】有以读为又以。

(3) 寡,少。

(4) 倕,尧之巧工也,以巧闻天下。周家铸鼎,著倕于鼎,使自啮其指,明不当大巧为也。一说:周铸鼎象百物,技巧绝殊,假令倕见之,则自衔啮其指,不能复为,故言“大巧之不可为也”。

【校】注前说是也。《淮南》《本经训》、《道应训》皆有此语。





淫辞


五曰:

非辞无以相期,从辞则乱。乱辞之中又有辞焉,心之谓也。言不欺心,则近之矣。凡言者以谕心也。言心相离,而上无以参之,则下多所言非所行也,所行非所言也。言行相诡,不祥莫大焉。

空雄之遇,秦、赵相与约, (1) 约曰:“自今以来,秦之所欲为,赵助之;赵之所欲为,秦助之。”居无几何,秦兴兵攻魏,赵欲救之。秦王不说,使人让赵王曰:“约曰:‘秦之所欲为,赵助之;赵之所欲为,秦助之。’今秦欲攻魏,而赵因欲救之,此非约也。”赵王以告平原君, (2) 平原君以告公孙龙,公孙龙曰:“亦可以发使而让秦王曰:‘赵欲救之,今秦王独不助赵,此非约也。’”

(1) 空雄,地名。遇,会也。约,盟也。

【校】“空雄”,前《听言》篇作“空洛”。此疑本是“空雒”,写者误耳。

(2) 赵王,赵惠王也。平原君,赵公子胜也。

孔穿、公孙龙相与论于平原君所,深而辩,至于藏三牙,公孙龙言藏之三牙甚辩, (1) 孔穿不应,少选,辞而出。 (2) 明日,孔穿朝, (3) 平原君谓孔穿曰:“昔者公孙龙之言甚辩。” (4) 孔穿曰:“然,几能令藏三牙矣。虽然,难。 (5) 愿得有问于君,谓藏三牙甚难而实非也,谓藏两牙甚易而实是也, (6) 不知君将从易而是者乎, (7) 将从难而非者乎?”平原君不应。明日,谓公孙龙曰:“公无与孔穿辩。” (8)

(1) 公孙龙、孔穿皆辩士也。论,相易夺也。龙言藏之三牙。辩,说也。若乘白马禁不得度关,因言马白非白马,此之类也,故曰“甚辩”也。

【校】谢云:“臧二耳,见《孔丛子·公孙龙》篇。‘耳’字篆文近‘牙’,故传写致误。愚意‘臧’、‘ ’古字通用,谓羊也。此作‘藏’,尤误。”卢云:“作‘三耳’是也。龙意两耳,形也,又有一司听者以君之,故为三耳。但此下又言马齿,则此书似是作‘三牙’。又案《新论》言‘龙乘白马无符传,关吏不听出关,此虚言难以夺实也’,今此注意又相反,非也。”

(2) 少选,须臾。

(3) 朝,见也。

(4) 昔,昨日也。其辩,谓藏三牙之说也。

(5) 言藏三牙之说近难成也。

(6) 难易之说未闻。

(7) 【校】旧“者乎”上有“也”字,衍,今删去。

(8) 辩,相易夺也。

【校】《孔丛子》有“其人理胜于辞,公辞胜于理”二语,亦当并引。

荆柱国庄伯令其父视, (1) 曰:“日在天。”“视其奚如?”曰:“正圆。”视其时,曰:“当今。”令谒者驾,曰:“无马。”令涓人取冠,“进上。”问马齿,圉人曰:“齿十二与牙三十。” (2) 人有任臣不亡者,臣亡,庄伯决之任者无罪。 (3)

(1) 柱国,官名,若秦之有相国。

(2) 马上下齿十二,牙上下十八,合为三十。谓若公孙龙灭去其三牙,多而偏,不可均,故难也;藏去其二,少而均,故易。

【校】正文与注皆难晓。

(3) 断之便无罪,析言破律之刑。

【校】注“便”似当作“使”。

宋有澄子者,亡缁衣,求之涂, (1) 见妇人衣缁衣,援而弗舍,欲取其衣,曰:“今者我亡缁衣。”妇人曰:“公虽亡缁衣,此实吾所自为也。”澄子曰:“子不如速与我衣。昔吾所亡者,纺缁也;今子之衣, 缁也。以 缁当纺缁,子岂不得哉?” (2)

(1) 涂,道也。

(2) 得犹便也。澄子横认路妇缁衣,计其 与纺以为便,非其理也,言宋乱无法也。

宋王谓其相唐鞅曰:“寡人所杀戮者众矣,而群臣愈不畏,其故何也?” (1) 唐鞅对曰:“王之所罪,尽不善者也。罪不善,善者故为不畏。 (2) 王欲群臣之畏也,不若无辨其善与不善而时罪之,若此则群臣畏矣。”居无几何,宋君杀唐鞅。唐鞅之对也,不若无对。 (3)

(1) 宋王,康王也。言何故不畏我。

(2) 【校】杨倞注《荀子·解蔽》篇引《论衡》作“善者胡为畏”。

(3) 鞅令宋王善与不善皆罪之以立威,王是以杀唐鞅,故曰“唐鞅之对,不若无对”。

惠子为魏惠王为法。为法已成,以示诸民人, (1) 民人皆善之。 (2) 献之惠王,惠王善之,以示翟翦,翟翦曰:“善也。” (3) 惠王曰:“可行邪?”翟翦曰:“不可。”惠王曰:“善而不可行,何故?”翟翦对曰:“今举大木者,前乎舆 ,后亦应之,此其于举大木者善矣。 (4) 岂无郑、卫之音哉?然不若此其宜也。 (5) 夫国亦木之大者也。” (6)

(1) 【校】旧校云:“一作‘良人’。”

(2) 惠子,惠施,宋人也,仕魏,为惠王相也,孟子所见梁惠王也。

(3) 翟翦,翟黄之后也。

(4) “舆 ”或作“邪 ”。前人倡,后人和,举重劝力之歌声也。

(5) 郑、卫之音皆新声,非雅乐,凡人所说也,不如呼“舆 ”宜于举大木也。

(6) 言惠子之法若郑、卫之音,宜于众人之耳,于治国之法,未可用也,故曰“善而不可行”也。





不屈


六曰:

察士以为得道则未也。虽然,其应物也,辞难穷矣。辞难穷,其为祸福犹未可知。 (1) 察而以达理明义,则察为福矣;察而以饰非惑愚,则察为祸矣。 (2) 古者之贵善御也,以逐暴禁邪也。

(1) 犹,尚也。

(2) 惑,误。

魏惠王谓惠子曰:“上世之有国,必贤者也。今寡人实不若先生,愿得传国。” (1) 惠子辞。 (2) 王又固请曰:“寡人莫有之国于此者也,而传之贤者,民之贪争之心止矣。欲先生之以此听寡人也。” (3) 惠子曰:“若王之言,则施不可而听矣。王固万乘之主也,以国与人犹尚可。今施,布衣也,可以有万乘之国而辞之,此其止贪争之心愈甚也。”惠王谓惠子曰“古之有国者,必贤者也”,夫受而贤者,舜也,是欲惠子之为舜也;夫辞而贤者,许由也,是惠子欲为许由也;传而贤者,尧也,是惠王欲为尧也。尧、舜、许由之作,非独传舜而由辞也,他行称此。今无其他,而欲为尧、舜、许由,故惠王布冠而拘于鄄, (4) 齐威王几弗受; (5) 惠子易衣变冠,乘舆而走,几不出乎魏境。 (6) 凡自行不可以幸,为必诚。 (7)

(1) 传,授。

(2) 谢不受之。

(3) 听,从。

(4) 鄄,邑名也。自拘于鄄,将服于齐也。

(5) 威王,田和之孙,孟子所见宣王之父。几,危。危不受魏惠王也。

(6) 言几不免难于魏境内也。

(7) 言惠王幸享传国之名 [2] ,惠子幸享以不受之名,以为必诚也。

匡章谓惠子于魏王之前曰:“蝗螟,农夫得而杀之,奚故?为其害稼也。 (1) 今公行,多者数百乘,步者数百人;少者数十乘,步者数十人。此无耕而食者,其害稼亦甚矣。” (2) 惠王曰:“惠子施也,难以辞与公相应。 (3) 虽然,请言其志。”惠子曰:“今之城者,或者操大筑乎城上,或负畚而赴乎城下,或操表掇以善睎望。若施者,其操表掇者也。 (4) 使工女化而为丝,不能治丝;使大匠化而为木,不能治木;使圣人化而为农夫,不能治农夫。施而治农夫者也, (5) 公何事比施于螣螟乎?”惠子之治魏为本,其治不治。当惠王之时,五十战而二十败,所杀者不可胜数,大将、爱子有禽者也。 (6) 大术之愚,为天下笑,得举其讳, (7) 乃请令周太史更著其名。 (8) 围邯郸三年而弗能取,士民罢潞, (9) 国家空虚, (10) 天下之兵四至, (11) 罪庶诽谤, (12) 诸侯不誉。 (13) 谢于翟翦,而更听其谋,社稷乃存。 (14) 名宝散出,土地四削,魏国从此衰矣。 (15) 仲父,大名也;让国,大实也。说以不听不信。听而若此,不可谓工矣。不工而治,贼天下莫大焉。 (16) 幸而独听于魏也。 (17) 以贼天下为实,以治之为名,匡章之非,不亦可乎? (18)

(1) 匡章,孟子弟子也。蝗,螽也。食心曰螟,食叶曰 。今兖州谓蝗为螣。谕王与惠子擅相禅受,害于义者也。

【校】梁伯子云:“高氏注《淮南·氾论训》以陈仲子为孟子弟子,此以匡章为孟子弟子,均妄说也。”

(2) 甚于蝗螟。

(3) 公,谓匡章。

(4) 施,惠子名也。表掇,仪度。

(5) 而,能也。

(6) 言惠王用惠子之谋,为土地之故,麋烂其民而战之,大败,又将复之,恐不胜,用乃驱其所爱子弟以殉之,此谓以其所不爱及其所爱,故曰“大将、爱子有禽者”矣。

(7) 天下人笑之,得举书其讳恶。

(8) 言惠王比惠子于管夷吾,欲更著其名。名,仲父之名也。

(9) 潞,羸也。

【校】“潞”与“露”同。

(10) 府藏竭也。

(11) 救邯郸之兵从四方来至也。

(12) 怨望多也。

(13) 皆道其恶也。

(14) 翟翦言惠子之法善而不可行,又为惠王说举大木,前呼舆 ,后亦和之,岂无郑、卫之音,不若此其宜也。尝谢负于翟翦而从其谋,社稷乃存也。

【校】注“尝”疑是“当”。末“也”字旧作“之”,误,今改正。

(15) 名宝散出,以赂邻国也。土地为四方所侵削,故曰“魏国从此衰”。

(16) 贼,害。

(17) 言惠子之言独见听用于魏者幸也。

(18) 匡章之非惠子,不亦可也?

白圭新与惠子相见也,惠子说之以强, (1) 白圭无以应。惠子出,白圭告人曰:“人有新取妇者,妇至,宜安矜烟视媚行。 (2) 竖子操蕉火而巨,新妇曰:‘蕉火大巨。’ (3) 入于门,门中有歛陷, (4) 新妇曰:‘塞之!将伤人之足。’此非不便之家氏也, (5) 然而有大甚者。今惠子之遇我尚新, (6) 其说我有大甚者。”惠子闻之曰:“不然。《诗》曰:‘恺悌君子,民之父母。’恺者大也,悌者长也。君子之德,长且大者,则为民父母。父母之教子也,岂待久哉?何事比我于新妇乎?《诗》岂曰‘恺悌新妇’哉?”诽污因污,诽辟因辟,是诽者与所非同也。白圭曰“惠子之遇我尚新,其说我有大甚者”,惠子闻而诽之,因自以为为之父母,其非有甚于白圭亦有大甚者。

(1) 以强力也。

(2) 媚行,徐行。

(3) 【校】蕉,薪樵也。

(4) 歛,读曰胁。

【校】歛从“欠”,呼滥切。疑即坎窞。注不可晓。旧校云:“‘陷’一作‘堛’。”梁仲子疑“歛”为“欿”字之误。

(5) 家氏,妇氏。

【校】此与《卫策》灭灶徙臼之事相似。

(6) 遇,见。





应言


七曰:

白圭谓魏王曰:“市丘之鼎以烹鸡,多洎之则淡而不可食, (1) 少洎之则焦而不熟, (2) 然而视之蝺焉美无所可用。 (3) 惠子之言,有似于此。” (4) 惠子闻之,曰:“不然。使三军饥而居鼎旁,适为之甑,则莫宜之此鼎矣。”白圭闻之,曰:“无所可用者,意者徒加其甑邪?”白圭之论自悖,其少魏王大甚。以惠子之言蝺焉美无所可用,是魏王以言无所可用者为仲父也,是以言无所用者为美也。

(1) 市丘,魏邑也。鼎,大鼎,不宜烹小也。能知五味也。肉汁曰洎。淡者,洎多无味,故不可食之也。

【校】梁仲子云:“市邱之为魏邑,无考。‘巿’疑是‘巿’,读若贝,与‘市’字异。沛邱,齐地,见《史记·齐世家》。《左氏庄八年传》作‘贝邱’。沛、贝同音,省文作‘巿’。”卢云:“昭廿年《传》‘齐侯田于沛’,《释文》‘沛,音贝’,是则沛邱之即贝邱信矣。”余案:《史记·孟荀列传》索隐引《吕氏春秋》作“函牛之鼎,不可以烹鸡”,疑当以“函牛”为是。函牛之鼎,大鼎也,与喻意似更切。又案《蔡邕集》载《荐边让书》引传曰“函牛之鼎以烹鸡,多汁则澹而不可食,少汁则燋而不熟”,其文与此正同。市丘、沛丘俱不闻以大鼎著名。今欲言大鼎,何必定取某地所出?然《蔡集》旧本亦注云“一曰市丘之鼎”,故并载梁说,以俟后来择焉。又注“能知五味也”上疑有脱文。

(2) 焦,燥。鸡难臑熟。

(3) 蝺,读龋齿之龋。龋,鼎好貌。

【校】“蝺”字无考,疑是“竬”,与“偊”、“踽”皆同。

(4) 似此鼎,好而不可用。

公孙龙说燕昭王以偃兵, (1) 昭王曰:“甚善。寡人愿与客计之。”公孙龙曰:“窃意大王之弗为也。”王曰:“何故?”公孙龙曰:“日者大王欲破齐,诸天下之士其欲破齐者,大王尽养之;知齐之险阻要塞、君臣之际者,大王尽养之;虽知而弗欲破者,大王犹若弗养。其卒果破齐以为功。今大王曰:‘我甚取偃兵。’诸侯之士在大王之本朝者,尽善用兵者也。臣是以知大王之弗为也。”王无以应。

(1) 龙,魏人也。昭王,燕王哙之子也。偃,止也。

司马喜难墨者师于中山王前以非攻, (1) 曰:“先生之所术非攻夫?”墨者师曰:“然。” (2) 曰:“今王兴兵而攻燕,先生将非王乎?”墨者师对曰:“然则相国是攻之乎?”司马喜曰:“然。”墨者师曰:“今赵兴兵而攻中山,相国将是之乎?”司马喜无以应。

(1) 司马喜,赵之相国也。

(2) 然,如是。

路说谓周颇曰:“公不爱赵,天下必从。”周颇曰:“固欲天下之从也,天下从则秦利也。”路说应之曰:“然则公欲秦之利夫?”周颇曰:“欲之。”路说曰:“公欲之,则胡不为从矣?”

魏令孟卬割绛、 、安邑之地以与秦王。 (1) 王喜,令起贾为孟卬求司徒于魏王。 (2) 魏王不说,应起贾曰:“卬,寡人之臣也。寡人宁以臧为司徒,无用卬。 (3) 愿大王之更以他人诏之也。” (4) 起贾出,遇孟卬于廷,曰:“公之事何如?”起贾曰:“公甚贱于公之主, (5) 公之主曰:‘宁用臧为司徒,无用公。’” (6) 孟卬入见,谓魏王曰:“秦客何言?”王曰:“求以女为司徒。”孟卬曰:“王应之谓何?”王曰:“宁以臧,无用卬也。”孟卬太息曰:“宜矣王之制于秦也!王何疑秦之善臣也?以绛、 、安邑令负牛书与秦,犹乃善牛也。 (7) 卬虽不肖,独不如牛乎?且王令三将军为臣先,曰‘视卬如是身’, (8) 是重臣也。令二轻臣也, (9) 令臣责, (10) 卬虽贤,固能乎?” (11) 居三日,魏王乃听起贾。 (12) 凡人主之与其大官也,为有益也。今割国之锱锤矣,而因得大官, (13) 且何地以给之? (14) 大官,人臣之所欲也。孟卬令秦得其所欲, (15) 秦亦令孟卬得其所欲, (16) 责以偿矣,尚有何责?魏虽强,犹不能责无责,又况于弱?魏王之令乎孟卬为司徒,以弃其责,则拙也。

(1) 【校】“孟卬”乃“孟卯”之误。《淮南子》注云:“孟卯,齐人。”《战国策》作“芒卯”。案《魏策》“芒卯谓秦王曰‘王有所欲于魏者,长羊、王屋、洛林之地也。王能使臣为魏之司徒,则臣能使魏献之’”,今此云“割绛、 、安邑之地”,“ ”疑即“汾”之异文,字书不载。梁仲子云:“安邑,魏都也。奈何割其国都以与人?此殊不可信。”

(2) 【校】起贾疑即须贾。

(3) 臧亦魏臣。

(4) 诏,告。

(5) 公之主甚贱公。

(6) 公谓卬。

(7) 言王使负牛持绛、 、安邑之书致之于秦,秦犹善牛。

【校】负牛,当亦是魏臣,在孟卬之下者。旧校云:“‘乃’一作‘之’。”

(8) 王身。

(9) 二,疑也。臣见疑则不重矣。

(10) 令秦责臣。

(11) 言不能也。

(12) 听起贾言,用卬为司徒。

(13) 割,分也。锱锤,铢两也。谓分绛、 、安邑而得大官。大官,司徒也。

(14) 给,足。

(15) 所欲田邑。

(16) 所欲司徒。

秦王立帝,宜阳 [3] 许绾诞魏王, (1) 魏王将入秦。魏敬谓王曰 (2) :“以河内孰与梁重?”王曰:“梁重。”又曰:“梁孰与身重?”王曰:“身重。”又曰:“若使秦求河内,则王将与之乎?”王曰:“弗与也。”魏敬曰:“河内,三论之下也; (3) 身,三论之上也。秦索其下而王弗听,索其上而王听之,臣窃不取也。”王曰:“甚然。” (4) 乃辍行。 (5) 秦虽大胜于长平,三年然后决, (6) 士民倦,粮食。 (7) 当此时也,两周全,其北存。魏举陶削卫,地方六百,有之势是, (8) 而入大蚤, (9) 奚待于魏敬之说也? (10) 夫未可以入而入,其患有将可以入而不入, (11) 入与不入之时,不可不熟论也。 (12)

(1) 诞,诈也。许绾,秦臣也。秦实未为帝也,诈魏王,言帝欲令魏王入朝也。

(2) 【校】“魏敬”,《魏策》作“周 ”。

(3) 三论,谓河内与梁及身也。

(4) 甚善。

【校】旧本注二字在“甚”字之下,误,今移正。

(5) 辍,止。不入秦。

【校】旧本“辍”上有“辄”字,系误衍,今删。

(6) 秦将白起攻赵三年,坑其卒四十万众于长平,故曰“大胜”也。

(7) 【校】此二字下脱一字。

(8) 有之势是,有是之势。

(9) 入秦大蚤。

(10) 言何必待魏敬之说乃不入秦邪。

(11) 【校】旧本作“夫未可以入而入,其患有将可以入而入,其患有将可以入而不入”,衍正文九字,又于两“将”字下俱注“将大”二字,殊谬。“其患有将可以入而不入”本是一句,“有”与“又”同。诱岂不谙文义而以两“将”字为句乎?今削去。

(12) 论,辩也。





具备


八曰:今有羿、蠭蒙、繁弱于此,而无弦,则必不能中也。 (1) 中非独弦也,而弦为弓中之具也。夫立功名亦有具,不得其具,贤虽过汤、武,则劳而无功矣。汤尝约于 薄矣, (2) 武王尝穷于毕裎矣, (3) 伊尹尝居于庖厨矣,太公尝隐于钓鱼矣。贤非衰也,智非愚也,皆无其具也。故凡立功名,虽贤必有其具,然后可成。

(1) 羿,夏之诸侯,有穷之君也,善射,百发百中。蠭蒙,羿弟子也,亦能百中。繁弱,良弓所出地也,因以为弓名。

【校】孙宣公音《孟子》“蠭蒙”作“逄蒙”,音薄江反,似未考乎此。

(2) “薄”或作“亳”。

(3) 毕裎,毕丰。

【校】“裎”与“程”同。孙宣公《孟子音义》“裎音程,亦作‘程’”。注“毕裎,毕丰”,盖以丰即程也。毕、丰皆在咸阳。案《周书·大匡解》“维周王宅程三年”,孔晁注云:“程,地名,在岐州左右,后以为国,初王季之子文王因焉,而遭饥馑,乃徙丰焉。”是丰、程不得为一地。《雍录》云:“丰在鄠县,程在咸阳东北。”案《孟子》云:“文王卒于毕郢。”文王墓在今西安府咸宁县。毕裎,疑当即毕郢。

宓子贱治亶父,恐鲁君之听谗人而令己不得行其术也, (1) 将辞而行,请近吏二人于鲁君, (2) 与之俱至于亶父。邑吏皆朝,宓子贱令吏二人书。吏方将书,宓子贱从旁时掣摇其肘,吏书之不善,则宓子贱为之怒。吏甚患之,辞而请归。宓子贱曰:“子之书甚不善,子勉归矣!” (3) 二吏归报于君, (4) 曰:“宓子不得为书。”君曰:“何故?”吏对曰:“宓子使臣书,而时掣摇臣之肘,书恶而有甚怒,吏皆笑宓子。 (5) 此臣所以辞而去也。”鲁君太息而叹曰:“宓子以此谏寡人之不肖也。寡人之乱子,而令宓子不得行其术,必数有之矣。微二人,寡人几过。”遂发所爱, (6) 而令之亶父,告宓子曰:“自今以来,亶父非寡人之有也,子之有也。有便于亶父者,子决为之矣。五岁而言其要。” (7) 宓子敬诺,乃得行其术于亶父。三年,巫马旗短褐衣弊袭,而往观化于亶父,见夜渔者,得则舍之。巫马旗问焉,曰:“渔为得也,今子得而舍之,何也?”对曰:“宓子不欲人之取小鱼也。 (8) 所舍者小鱼也。”巫马旗归,告孔子曰:“宓子之德至矣,使民暗行若有严刑于旁。 (9) 敢问宓子何以至于此?”孔子曰:“丘尝与之言曰:‘诚乎此者刑乎彼。’ (10) 宓子必行此术于亶父也。”夫宓子之得行此术也,鲁君后得之也。鲁君后得之者,宓子先有其备也。先有其备,岂遽必哉?此鲁君之贤也。

(1) 子贱,孔子弟子宓不齐。

【校】“谗”,旧本作“说”,讹,今改正。

(2) 【校】《家语·屈节解》“吏”作“史”,下文“邑吏吏皆外”并同。

(3) 勉犹趣也。

(4) 报鲁君也。

(5) 吏,邑吏也。

(6) 发,遣。

(7) 要,约最簿书。

(8) 古者鱼不尺不升于俎。宓子体圣人之化,为尽类也,故不欲人取小鱼。

(9) 暗,夜。

(10) 施至诚于近以化之,使刑行于远。

三月婴儿,轩冕在前,弗知欲也;斧钺在后,弗知恶也;慈母之爱谕焉,诚也。故诚有诚乃合于情,精有精乃通于天。乃通于天, (1) 水木石之性,皆可动也,又况于有血气者乎?故凡说与治之务莫若诚。 (2) 听言哀者,不若见其哭也;听言怒者,不若见其斗也。说与治不诚,其动人心不神。 (3)

(1) 【校】五字疑误衍。

(2) 以诚说则信著之,以诚治则化行之。

(3) 动,感。神,化。言不诚不能行其化也。




————————————————————

[1] “亡”原作“以”,据乾隆本改。

[2] “名”,原为“子”,据乾隆本改。

[3] 乾隆本“宜阳”下有“令”。





第十九卷 离俗览



离俗


一曰:

世之所不足者,理义也; (1) 所有余者,妄苟也。 (2) 民之情,贵所不足,贱所有余。 (3) 故布衣人臣之行,洁白清廉中绳,愈穷愈荣; (4) 虽死,天下愈高之,所不足也。 (5) 然而以理义斫削,神农、黄帝犹有可非,微独舜、汤。 (6) 飞兔、要褭,古之骏马也,材犹有短。 (7) 故以绳墨取木,则宫室不成矣。 (8)

(1) 人能蹈之者少,故曰“不足”。

(2) 妄作苟为,不尊理义,君子少,小人多,故有余也。

(3) 所不足者理与义也,故贵之。所有余者妄与苟也,故贱之。

(4) 绳,正也。行如此者,益穷困益有荣名。

(5) 高,贵也。所洁白中正,若周时伯夷,卫之弘演。身虽死亡,天下闻之而益贵。

(6) 微亦非也。舜有卑父之谤,汤有放弑之事,然以通义斫削,神农、黄帝之行犹有可苛者,非独舜与汤也。言虽圣不能无阙,况贤者乎?

【校】注“卑父之谤”,见下《举难》篇及《淮南·氾论训》。

(7) 飞兔、要褭,皆马名也,日行万里,驰若兔之飞,因以为名也。材犹有短,力有所不足。“褭”字读如曲挠之挠也。

(8) 正材难得,故宫室不成也。

舜让其友石户之农,石户之农曰:“棬棬乎,后之为人也! (1) 葆力之士也。”以舜之德为未至也,于是乎夫负妻戴, (2) 携子以入于海,去之终身不反。舜又让其友北人无择,北人无择曰:“异哉,后之为人也!居于甽亩之中,而游入于尧之门。不若是而已, (3) 又欲以其辱行漫我,我羞之。” (4) 而自投于苍领之渊。 (5) 汤将伐桀,因卞随而谋,卞随辞曰:“非吾事也。”汤曰:“孰可?”卞随曰:“吾不知也。”汤又因务光而谋, (6) 务光曰:“非吾事也。”汤曰:“孰可?”务光曰:“吾不知也。”汤曰:“伊尹何如?”务光曰:“强力忍 , (7) 吾不知其他也。”汤遂与伊尹谋夏伐桀, (8) 克之。以让卞随,卞随辞曰:“后之伐桀也,谋乎我,必以我为贼也;胜桀而让我,必以我为贪也。吾生乎乱世,而无道之人再来 我,吾不忍数闻也。”乃自投于颍水而死。 (9) 汤又让于务光,曰:“智者谋之, (10) 武者遂之, (11) 仁者居之, (12) 古之道也。吾子胡不位之? (13) 请相吾子。” (14) 务光辞曰:“废上,非义也; (15) 杀民,非仁也; (16) 人犯其难,我享其利,非廉也。吾闻之,非其义,不受其利;无道之世,不践其土。况于尊我乎?吾不忍久见也。”乃负石而沈于募水。 (17) 故如石户之农、北人无择、卞随、务光者,其视天下若六合之外,人之所不能察; (18) 其视贵富也,苟可得已,则必不之赖; (19) 高节厉行,独乐其意,而物莫之害 (20) ;不漫于利,不牵于势, (21) 而羞居浊世;惟此四士者之节。 (22) 若夫舜、汤,则苞裹覆容,缘不得已而动,因时而为,以爱利为本,以万民为义。譬之若钓者,鱼有小大,饵有宜适,羽有动静。 (23)

(1) 【校】“棬棬”,《庄子·让王》篇作“捲捲”,《释文》云:“音权,郭音眷,用力貌。”

(2) 【校】“戴”,旧本作“妻”,讹,今依《庄子》改正。

(3) 已,止也。

(4) 漫,污也。

(5) 投犹沈也。“苍领”或作“青令”。

【校】《庄子》作“清泠”,《淮南·齐俗训》亦同。

(6) 【校】《庄子》作“瞀光”,《荀子·成相》篇作“牟光”。

(7) ,辱也。

【校】《庄子》“ ”作“垢”。

(8) 【校】《庄子》无“夏”字。

(9) 以汤伐桀,故谓之无道之人也。以受汤之让为贪辱也。不忍闻之,故投水而死。颍出于颍川阳城西山中也。

【校】梁仲子云:“《水经·颍水注》引云‘卞随耻受汤让,自投此水而死’,张显《逸民传》、嵇叔夜《高士传》并言‘投泂水而死’,未知其孰是也。”案《庄子》作“椆水”,《释文》云:“本又作‘桐水’,司马本作‘泂水’。”

(10) 图之也。

(11) 遂,成也。

【校】旧校云:“‘武’一作‘贤’。”

(12) 居,处也。

(13) 【校】《庄子》作“立乎”。

(14) 胡,何。何不位天子之位也。言己请为吾子为相。

【校】注下“为”字疑衍。

(15) 上,天子,谓桀。废之,非礼义也。

(16) 战伐杀民,非仁心。

(17) 募,水名也,音千伯之伯。

【校】募无伯音,疑“ ”之讹。《庄子》作“庐水”,司马本作“卢水”。

(18) 察,见也。

(19) 不之赖,不赖之也。赖,利也,一曰善也。

(20) 不欲于物,故物无能害。

(21) 漫,污;牵,拘也。

(22) 四士,谓石户之农、北人无择、卞随、务光。羞居乱世,皆远引而去,或自投而死,此四人,介之大者。

(23) 羽,钓浮也。

齐、晋相与战,平阿之余子亡戟得矛, (1) 却而去,不自快, (2) 谓路之人曰:“亡戟得矛,可以归乎?”路之人曰:“戟亦兵也,矛亦兵也,亡兵得兵,何为不可以归?”去行,心犹不自快,遇高唐之孤叔无孙,当其马前曰:“今者战,亡戟得矛,可以归乎?” (3) 叔无孙曰:“矛非戟也,戟非矛也,亡戟得矛,岂亢责也哉?” (4) 平阿之余子曰:“嘻!还反战,趋尚及之。”遂战而死。叔无孙曰:“吾闻之,君子济人于患,必离其难。” (5) 疾驱而从之,亦死而不反。 (6) 令此将众,亦必不北矣; (7) 令此处人主之旁,亦必死义矣。今死矣而无大功,其任小故也。任小者,不知大也。今焉知天下之无平阿余子与叔无孙也?故人主之欲得廉士者,不可不务求。

(1) 失戟得矛,心不平。平阿,齐邑也。余子,官氏也。与晋人战,亡其所执戟,而得晋人之矛也。

(2) 失戟得矛,心不自安。

【校】旧校云:“‘却’一作‘退’。”案《御览》三百五十三作“退而不自快”。

(3) 高唐,齐邑也。孤,孤特,位尊。叔,姓。无孙,名。守高唐之大夫也。余子当其马前而问之。

(4) 亢,当也。

(5) 济,入也。

(6) 反,还也。

(7) 北,走也。

齐庄公之时, (1) 有士曰宾卑聚,梦有壮子,白缟之冠,丹绩之 , (2) 东布之衣,新素履,墨剑室,从而叱之,唾其面,惕然而寤,徒梦也。 (3) 终夜坐,不自快。明日,召其友而告之曰:“吾少好勇,年六十而无所挫辱。今夜辱,吾将索其形,期得之则可,不得将死之。”每朝与其友俱立乎衢,三日不得,却而自殁。 (4) 谓此当务则未也,虽然,其心之不辱也,有可以加乎! (5)

(1) 庄公,名光,顷公之孙,灵公之子,景公之兄。

(2) ,缨也。

【校】“绩”疑“缋”。

(3) 寤,觉。徒,但。

(4) 【校】旧校云:“‘却’一作‘退’。”

(5) 加,上也。





高义


二曰:

君子之自行也, (1) 动必缘义,行必诚义, (2) 俗虽谓之穷,通也。 (3) 行不诚义,动不缘义,俗虽谓之通,穷也。然则君子之穷通,有异乎俗者也。故当功以受赏,当罪以受罚。赏不当,虽与之必辞; (4) 罚诚当,虽赦之不外。 (5) 度之于国,必利长久。长久之于主,必宜内反于心, (6) 不惭然后动。

(1) 【校】旧校云:“‘自’一作‘为’。”

(2) 所行诚义也。

(3) 通,达也。

(4) 辞,不敢受也。

(5) 不敢远也。

(6) 【校】旧本“反”作“及”,孙据李善注《文选》崔子玉《座右铭》所引改。

孔子见齐景公, (1) 景公致廪丘以为养,孔子辞不受,入谓弟子曰:“吾闻君子当功以受禄。今说景公,景公未之行而赐之廪丘,其不知丘亦甚矣!”令弟子趣驾,辞而行。 (2) 孔子布衣也,官在鲁司寇, (3) 万乘难与比行,三王之佐不显焉,取舍不苟也夫! (4)

(1) 景公,名杵臼,庄公光之弟,灵公环之子。

(2) 行,去也。

(3) 为鲁定公之司寇。

(4) 【校】旧校云:“一作‘不苟且也’。”

子墨子游公上过于越。 (1) 公上过语墨子之义, (2) 越王说之,谓公上过曰:“子之师苟肯至越, (3) 请以故吴之地,阴江之浦,书社三百以封夫子。” (4) 公上过往复于子墨子, (5) 子墨子曰:“子之观越王也,能听吾言、用吾道乎?”公上过曰:“殆未能也。” (6) 墨子曰:“不唯越王不知翟之意,虽子亦不知翟之意。若越王听吾言、用吾道,翟度身而衣,量腹而食,比于宾萌,未敢求仕。 (7) 越王不听吾言、不用吾道,虽全越以与我,吾无所用之。 (8) 越王不听吾言、不用吾道,而受其国, (9) 是以义翟也。义翟何必越,虽于中国亦可。” (10) 凡人不可不熟论。秦之野人,以小利之故,弟兄相狱,亲戚相忍。今可得其国,恐亏其义而辞之,可谓能守行矣。其与秦之野人相去亦远矣。

(1) 公上过,子墨子弟子也。

【校】《墨子·鲁问》篇作“公尚过”。

(2) 义,道。

(3) 苟,诚也。

(4) 社,二十五家也。三百社,七千五百家。

(5) 复,白也。

(6) 殆,近也。

(7) 宾,客也。萌,民也。

(8) 无用越为之也。

(9) 【校】旧校云:“‘受’一作‘爱’。”

(10) 【校】《墨子》作“是我以义 也,钧之 ,亦于中国耳,何必于越哉”。此两“翟”字讹。“ ”字无考,当是“糶”字之误。

荆人与吴人将战,荆师寡,吴师众。荆将军子囊曰:“我与吴人战,必败。败王师,辱王名,亏壤土,忠臣不忍为也。”不复于王而遁。 (1) 至于郊,使人复于王曰:“臣请死。”王曰:“将军之遁也,以其为利也。今诚利,将军何死?”子囊曰:“遁者无罪,则后世之为王臣者, (2) 将皆依不利之名而效臣遁。若是,则荆国终为天下挠。” (3) 遂伏剑而死。王曰:“请成将军之义。” (4) 乃为之桐棺三寸,加斧锧其上。 (5) 人主之患,存而不知所以存,亡而不知所以亡,此存亡之所以数至也。 、岐之广也, (6) 万国之顺也,从此生矣。 (7) 荆之为四十二世矣,尝有乾溪、白公之乱矣, (8) 尝有郑襄、州侯之避矣, (9) 而今犹为万乘之大国,其时有臣如子囊与!子囊之节,非独厉一世之人臣也。 (10)

(1) 复,白也。遁,走也。

(2) 【校】旧本缺“臣”字,今据《说苑·立节》篇补。《渚宫旧事》作“则后之为将者”,此处“者将”二字若乙转,可不添“臣”字。

(3) 挠,弱也。

(4) 【校】“之”字从《渚宫旧事》补,此脱在下句,下句可无“之”字。

(5) 【校】梁仲子云:“案此即《左传》襄十四年楚子囊还自伐吴卒之事。检《传》上文言伐吴之役,为吴所败,未能全师而还。《吕览》大与《传》违。盖子囊之死,适当旋师之时,遂相传异说。夫见可知难,军之善政,子囊何至自讨,王亦何至忍与子玉、子反同诛?殆不可信。”

(6) ,汤所居也。岐,武王所居也。

(7) 顺,从。

【校】旧校云:“‘生’一作‘至’。”

(8) 灵王作乾溪之台,百姓愁怨,公子弃疾弑之而立,是为平王;白公胜,平王太子建之子也,出奔郑,郑人杀之,胜请令尹子西、司马子旗伐郑复仇,许而未行,晋人伐郑,子西、子旗率师救郑,胜怒,杀令尹子西、司马子旗;故曰“乾溪、白公之乱”也。

【校】注旧本“杀之”作“杀报”,讹,今改正,并补“胜请”二字。

(9) 郑襄、州侯事晋而伐楚,楚人避之也。

(10) 言子囊之忠,虽百世犹不可忘,故曰“非独厉一世之人臣”。

荆昭王之时,有士焉曰石渚, (1) 其为人也公直无私,王使为政。 (2) 道有杀人者, (3) 石渚追之,则其父也。还车而反,立于廷曰:“杀人者,仆之父也。以父行法,不忍; (4) 阿有罪,废国法,不可。 (5) 失法伏罪,人臣之义也。”于是乎伏斧锧,请死于王。 (6) 王曰:“追而不及,岂必伏罪哉?子复事矣。” (7) 石渚辞曰:“不私其亲,不可谓孝子;事君枉法,不可谓忠臣。君令赦之, (8) 上之惠也;不敢废法,臣之行也。”不去斧锧,殁头乎王廷。正法枉必死,父犯法而不忍,王赦之而不肯,石渚之为人臣也,可谓忠且孝矣。

(1) 【校】《韩诗外传》二、《新序·节士》篇、《史记·循吏传》皆作“石奢”,《渚宫旧事》与此同。

(2) 昭王,楚平王弃疾之子熊轸。

(3) 【校】“道”旧作“廷”,《新序》同,皆误也,今从《外传》、《史记》作“道”,方与下“追之”及“反立于廷”相合。

(4) 不忍行刑于父,孝也。

(5) 阿,私也。

(6) 免父杀身,忠孝之义。

(7) 事,职事也。

(8) 【校】旧校云:“‘君令’一作‘令吏’。”案《渚宫旧事》作“令吏舍之”。





上德


三曰:

为天下及国, (1) 莫如以德,莫如行义。以德以义,不赏而民劝, (2) 不罚而邪止。此神农、黄帝之政也。以德以义,则四海之大,江河之水,不能亢矣;太华之高, (3) 会稽之险, (4) 不能障矣; (5) 阖庐之教,孙、吴之兵,不能当矣。 (6) 故古之王者,德回乎天地, (7) 澹乎四海, (8) 东西南北,极日月之所烛,天覆地载,爱思不臧, (9) 虚素以公, (10) 小民皆之, (11) 其之敌而不知其所以然,此之谓顺天;教变容改俗,而莫得其所受之, (12) 此之谓顺情。 (13) 故古之人,身隐而功著,形息而名彰, (14) 说通而化奋,利行乎天下 (15) 而民不识。 (16) 岂必以严罚厚赏哉?严罚厚赏,此衰世之政也。

(1) 为,治也。

(2) 劝,善也。

(3) 西岳也。

(4) 山名,在吴郡。

(5) 障,防也。

(6) 孙、吴,吴起、孙武也。吴王阖庐之将也,兵法五千言是也。

(7) 回,通。

(8) 澹,之也。

【校】注疑未是。刘本作“泊也”,亦是妄改,或是“安也”,与“憺”义同。

(9) 臧,匮也。

【校】“思”旧作“恶”,校云“‘恶’一作‘思’”,今从旧校改。臧之训匮,未知何出。

(10) 素,质也。恶其质以奉公,王之实也。

【校】注“恶”疑当作“虚”,“王”疑当作“正”。

(11) 皆公己也。

【校】“己”亦疑是“正”之误。

(12) 得犹知也。

(13) 情,性也。顺其天性也。

(14) 身没于前,名明于后世。

(15) 利民之化行满天下。

(16) 识,知也。尧时民不知尧德,以季世视之则睹也。

三苗不服,禹请攻之。 (1) 舜曰:“以德可也。”行德三年而三苗服。 (2) 孔子闻之,曰:“通乎德之情,则孟门、太行不为险矣。 (3) 故曰德之速,疾乎以邮传命。”周明堂,金在其后,有以见先德后武也。 (4) 舜其犹此乎! (5) 其臧武通于周矣。

(1) 三苗,远国,在豫章之彭蠡也。

(2) 【校】孙云:“李善注《文选》王元长《曲水诗序》‘行德’作‘修德’。”

(3) 孟门,太行之险也。太行塞在河内野王之北上党关也。

【校】注“之险也”疑是“皆险地”。

(4) 作乐,金镈在后,故曰“先德后武”。

(5) 【校】旧校云:“‘此’一作‘上’。”

晋献公为丽姬远太子。太子申生居曲沃,公子重耳居蒲,公子夷吾居屈。丽姬谓太子曰:“往昔君梦见姜氏。”太子祠而膳于公, (1) 丽姬易之。 (2) 公将尝膳,姬曰:“所由远,请使人尝之。” (3) 尝人,人死;食狗,狗死;故诛太子。太子不肯自释, (4) 曰:“君非丽姬,居不安,食不甘。”遂以剑死。 (5) 公子夷吾自屈奔梁。公子重耳自蒲奔翟,去翟过卫,卫文公无礼焉; (6) 过五鹿如齐,齐桓公死;去齐之曹,曹共公视其骈胁,使袒而捕池鱼; (7) 去曹过宋,宋襄公加礼焉; (8) 之郑,郑文公不敬, (9) 被瞻谏曰:“臣闻贤主不穷穷。今晋公子之从者,皆贤者也。君不礼也,不如杀之。”郑君不听;去郑之荆,荆成王慢焉; (10) 去荆之秦,秦缪公入之。 (11) 晋既定,兴师攻郑,求被瞻。被瞻谓郑君曰:“不若以臣与之。”郑君曰:“此孤之过也。”被瞻曰:“杀臣以免国,臣愿之。”被瞻入晋军,文公将烹之,被瞻据镬而呼曰:“三军之士皆听瞻也:自今以来,无有忠于其君,忠于其君者将烹。”文公谢焉,罢师,归之于郑。且被瞻忠于其君,而君免于晋患也;行义于郑,而见说于文公也;故义之为利博矣。 (12)

(1) 姜氏,申生母也。膳,胙之也。

【校】注“之”字疑衍。

(2) 易犹毒也。

【校】梁仲子疑是“易以毒也”。汪本改作“置也”,义不足。

(3) 太子自曲沃归膳,故曰“所由远”。姬施酖于酒,置毒于肉,故先使人尝之。

(4) 释,理也。

(5) 【校】案《传》云“缢于新城”。

(6) 文公名毁,宣公庶子顽烝宣姜而生之。无礼,不礼重耳也。

(7) 共公名襄,昭公之子。

【校】此与《淮南·人间训》同。《黄氏日抄》云“恐无此理”。

(8) 襄公名兹父,桓公御说之子。

(9) 文公名捷,郑厉公之子。

(10) 慢,易,不敬也。《传》曰:“及楚,楚子飨之曰:‘公子若反晋国,则何以报不穀?’对曰:‘子女玉帛则君有之,羽毛齿革则君地生焉,其波及晋国者,君之余也,其何以报君?’曰:‘虽然,则何以报我?’对曰:‘若以君之灵得反晋国,晋、楚治兵,遇于中原,其避君三舍;若不获命,其左执鞭弭,右属櫜鞬,以与君周还。’子玉请杀之。楚子曰:‘晋公子廉而俭,文而有礼。其从者肃而宽,忠而能力。晋侯无亲,外内恶之。吾闻姬姓,唐叔之后,其后衰者也。其将由晋公子重耳乎?天将与之,谁能废之?违天必有大咎。’乃送诸秦。”推此言之,不得为慢之也。

【校】注本《左传》。“虽然”下“则”字衍。“廉而俭”,《传》作“广而俭”,无“重耳”二字。

(11) 入,晋纳也。

【校】注当云“纳之晋也”。

(12) 博,大也。

墨者巨子孟胜,善荆之阳城君。 (1) 阳城君令守于国,毁璜以为符,约曰:“符合听之。”荆王薨,群臣攻吴起,兵于丧所,阳城君与焉,荆罪之。阳城君走,荆收其国。孟胜曰:“受人之国,与之有符。今不见符,而力不能禁,不能死,不可。”其弟子徐弱谏孟胜曰:“死而有益阳城君,死之可矣;无益也,而绝墨者于世,不可。”孟胜曰:“不然。吾于阳城君也,非师则友也,非友则臣也。不死,自今以来,求严师必不于墨者矣,求贤友必不于墨者矣,求良臣必不于墨者矣。死之,所以行墨者之义而继其业者也。 (2) 我将属巨子于宋之田襄子。 (3) 田襄子,贤者也,何患墨者之绝世也?” (4) 徐弱曰:“若夫子之言,弱请先死以除路。”还殁头前于。孟胜因使二人传巨子于田襄子。 (5) 孟胜死,弟子死之者百八十三人。以致令于田襄子, (6) 欲反死孟胜于荆。田襄子止之曰:“孟子已传巨子于我矣。”不听, (7) 遂反死之。 (8) 墨者以为不听巨子不察。严罚厚赏,不足以致此。今世之言治,多以严罚厚赏,此上世之若客也。 (9)

(1) 巨子孟胜,二人学墨道者也,为阳城君所善。

【校】《庄子·天下》释文引向秀云:“墨家号其道理成者为巨子,若儒家之硕儒。”此注非。下云“我将属巨子于宋之田襄子”,亦以名归之,而使其弟子皆从之受学也。

(2) 义,道;继,续也。

(3) 我,谓孟胜也。属,托也。

(4) 田襄子亦墨者也。

(5) 二人,孟胜之弟子也。传,送也。

(6) 【校】句上当有“二人”二字。“以”犹“已”也。

(7) 【校】旧本作“当听”,非,今改正。

(8) 反死孟胜于荆。

(9) 【校】义未详。





用民


四曰:

凡用民,太上以义,其次以赏罚。其义则不足死,赏罚则不足去就,若是而能用其民者,古今无有。民无常用也,无常不用也,唯得其道为可。 (1)

(1) 可用也。

阖庐之用兵也不过三万, (1) 吴起之用兵也不过五万。 (2) 万乘之国,其为三万、五万尚多,今外之则不可以拒敌,内之则不可以守国,其民非不可用也,不得所以用之也。不得所以用之, (3) 国虽大,势虽便,卒虽众,何益? (4) 古者多有天下而亡者矣,其民不为用也。 (5) 用民之论,不可不熟。剑不徒断,车不自行,或使之也。夫种麦而得麦,种稷而得稷,人不怪也。用民亦有种,不审其种,而祈民之用,惑莫大焉。 (6)

(1) 阖庐,吴王光也。

(2) 吴起,卫人,为楚将。

(3) 【校】孙云:“《御览》二百七十一‘守国’作‘守固’,两‘用之’下皆有‘术’字,然案下文似不当有。”

(4) 不知用之,何益于?不能以克敌也。

(5) 自古以来,有天下者多,而多无遗,民不为之用,故灭亡。

(6) 祈,求。

当禹之时,天下万国,至于汤而三千余国,今无存者矣,皆不能用其民也。民之不用,赏罚不充也。 (1) 汤、武因夏、商之民也,得所以用之也。管、商亦因齐、秦之民也,得所以用之也。 (2) 民之用也有故, (3) 得其故,民无所不用。用民有纪有纲,壹引其纪,万目皆起;壹引其纲,万目皆张。为民纪纲者何也?欲也,恶也。何欲何恶?欲荣利,恶辱害。辱害所以为罚充也,荣利所以为赏实也。赏罚皆有充实,则民无不用矣。 (4)

(1) 当赏不赏,当罚不罚,则民不怀不威,故不为之用也。

(2) 管,管仲。商,商鞅。

(3) 故,事也。

(4) 无不可用也。

阖庐试其民于五湖,剑皆加于肩,地流血几不可止。 (1) 句践试其民于寑宫, (2) 民争入水火, (3) 死者千余矣,遽击金而却之。 (4) 赏罚有充也。莫邪不为勇者兴,惧者变, (5) 勇者以工,惧者以拙,能与不能也。

(1) 试,用,习肄之也。

(2) 【校】旧作“寑官”,刘本作“寑宫”,案刘勰《新论·阅武》篇正作“寑宫”,今从刘本。

(3) 【校】《韩非·内储说上》“越王将复吴而试其教,燔台而鼓之,使民赴火者,赏在火也;临江而鼓之,使人赴水者,赏在水也”,即此事。

(4) 却犹止也。

【校】旧校云:“‘却’一作‘退’。”案《新论》正作“退”。

(5) 莫邪,良剑也。不为勇者利、怯者钝也。

夙沙之民,自攻其君而归神农。 (1) 密须之民,自缚其主而与文王。 (2) 汤、武非徒能用其民也,又能用非己之民。能用非己之民,国虽小,卒虽少,功名犹可立。 (3) 古昔多由布衣定一世者矣, (4) 皆能用非其有也。用非其有之心,不可察之本。 (5) 三代之道无二,以信为管。 (6)

(1) 夙沙,大庭氏之末世也。其君无道,故自攻之。神农,炎帝。

(2) 《诗》云“密人不共,敢距大邦”,此之谓也。

(3) 立,成也。

(4) 终一人之身为世。

(5) 本,始也。

【校】似当云“不可不察之本”,少一“不”字。

(6) 管,准法。

宋人有取道者,其马不进,倒而投之 水。 (1) 又复取道,其马不进,又倒而投之 水。如此者三。虽造父之所以威马,不过此矣。不得造父之道,而徒得其威,无益于御。 (2) 人主之不肖者,有似于此。不得其道,而徒多其威。威愈多,民愈不用。 (3) 亡国之主,多以多威使其民矣。故威不可无有,而不足专恃。譬之若盐之于味,凡盐之用,有所托也;不适,则败托而不可食。威亦然,必有所托,然后可行。 (4) 恶乎托?托于爱利。爱利之心谕,威乃可行。威太甚则爱利之心息,爱利之心息,而徒疾行威,身必咎矣。此殷、夏之所以绝也。君利势也,次官也。处次官,执利势,不可而不察于此。夫不禁而禁者,其唯深见此论邪。

(1) 倒,杀。投,弃之。

【校】梁仲子云:“《水经·淮水注》引作‘投之鸡水’。”

(2) 无益于不知御之道。

(3) 民不为之用。

(4) 行之也。





适威


五曰:

先王之使其民,若御良马,轻任新节, (1) 欲走不得,故致千里。善用其民者亦然。民日夜祈用而不可得, (2) 若得为上用,民之走之也,若决积水于千仞之溪, (3) 其谁能当之?《周书》曰:“民,善之则畜也,不善则仇也。” (4) 有仇而众,不若无有。厉王,天子也, (5) 有仇而众,故流于彘,祸及子孙, (6) 微召公虎而绝无后嗣。 (7) 今世之人主,多欲众之, (8) 而不知善,此多其仇也。不善则不有, (9) 有必缘其心爱之谓也, (10) 有其形不可谓有之。 (11) 舜布衣而有天下。桀,天子也,而不得息,由此生矣。 (12) 有无之论,不可不熟。 (13) 汤、武通于此论,故功名立。 (14)

(1) 节,节也。

【校】注疑“节,饰也”,或是“节,节其力也”。

(2) 祈,求也。

(3) 七尺曰仞。

(4) 《周书》周公所作。畜,好。

(5) 厉王名胡,《谥法》“杀戮不辜曰厉”,周夷王之子,宣王之父。

(6) 流,放也。彘,地名,今河东永安是也。

(7) 微,无也。虎臣宣王。《诗》云“王命召虎,式辟四方,彻我疆土”,此之谓也。

【校】赵云:“此注大谬。《周本纪》云‘厉王太子静匿召公家,国人围之,召公以己子代太子,太子得免,是为宣王’是也。‘虎臣宣王’似当作‘虎,宣王臣’。”

(8) 【校】似当作“多欲民众”。

(9) 不得有其位。

【校】注“位”当作“众”,下同。

(10) 缘其仁心,故曰“心爱之谓也”。

【校】似当作“故曰爱之谓也”。

(11) 形,体也。不可谓有天下之位也。

(12) 息,安也。不得安其位,由此多其仇生矣。

(13) 熟犹知。

(14) 立犹见也。

古之君民者,仁义以治之,爱利以安之,忠信以导之, (1) 务除其灾,思致其福。故民之于上也,若玺之于涂也,抑之以方则方,抑之以圜则圜。若五种之于地也,必应其类,而蕃息于百倍。此五帝、三王之所以无敌也。 (2) 身已终矣,而后世化之如神, (3) 其人事审也。 (4)

(1) 导犹先也。

(2) 无能敌之也。

(3) 从其化有如神也。

(4) 其所施行皆可为人法式,故曰“审也”。

魏武侯之居中山也, (1) 问于李克曰:“吴之所以亡者何也?” (2) 李克对曰:“骤战而骤胜。” (3) 武侯曰:“骤战而骤胜,国家之福也。其独以亡,何故?”对曰:“骤战则民罢,骤胜则主骄。以骄主使罢民,然而国不亡者,天下少矣。骄则恣,恣则极物; (4) 罢则怨,怨则极虑。 (5) 上下俱极,吴之亡犹晚。 (6) 此夫差之所以自殁于干隧也。” (7)

(1) 【校】《韩诗外传》十、《新序·杂事五》俱作“魏文侯”。

(2) 武侯,文侯之子也。乐羊伐中山得中山,故武侯居之也。

(3) 骤,数也。

(4) 极尽可欲之物。

(5) 极其巧欺不臣之虑。

(6) 犹,尚。

(7) 为越所破,自刭于干遂。

东野稷以御见庄公,进退中绳, (1) 左右旋中规。 (2) 庄公曰:“善。”以为造父不过也, (3) 使之钩百而少及焉。 (4) 颜阖入见, (5) 庄公曰:“子遇东野稷乎?”对曰:“然,臣遇之。 (6) 其马必败。”庄公曰:“将何败?”少顷,东野之马败而至。庄公召颜阖而问之曰:“子何以知其败也?”颜阖对曰:“夫进退中绳,左右旋中规,造父之御无以过焉。乡臣遇之,犹求其马,臣是以知其败也。” (7)

(1) 【校】旧校云:“‘退’一作‘却’。”下同。

(2) 东野,姓;稷,其名。

(3) 过犹胜也。

(4) 不达也。

(5) 见,谒也。

(6) 按《鲁世家》,庄公,桓公之子同也。颜阖在春秋后,盖鲁穆公时人也,在庄公后十二世矣。若实庄公,颜阖为妄矣。若实颜阖,庄公为妄矣。由此观之,咸阳市门之金,固得载而归也。

【校】梁伯子云:“东野稷事,此本于《庄子·达生》篇,《释文》曰‘李云鲁庄公,或云颜阖不与鲁庄公同时,当是卫庄公’。余考《庄子·人间世》言‘颜阖将傅卫灵公太子’,《让王》言‘鲁君致币颜阖’,李云‘鲁哀公’,亦见本书《贵生》篇,又《庄子·列御寇》篇言‘鲁哀公问颜阖’,则此为卫庄公是也。而《荀子·哀公》篇、《韩诗外传》二、《新序·杂事五》、《家语·颜回》篇皆云‘鲁定公问颜回,东野之御’,盖传闻异辞耳。高氏未加详考,误以为鲁庄公,訾吕子妄说,思载咸阳市门之金而归,何其陋也。又《荀》、《韩》、《新序》、《人表》、《家语》‘稷’字并作‘毕’。”

(7) 善当自求于心,而反求于御马速疾,故知其败也。

【校】此注非是。犹求其马,即下所云“极”是也。

故乱国之使其民,不论人之性,不反人之情,烦为教而过不识, (1) 数为令而非不从, (2) 巨为危而罪不敢, (3) 重为任而罚不胜。 (4) 民进则欲其赏,退则畏其罪,知其能力之不足也,则以为继矣。以为继知, (5) 则上又从而罪之, (6) 是以罪召罪, (7) 上下之相仇也,由是起矣。故礼烦则不庄,业烦则无功, (8) 令苛则不听,禁多则不行。 (9) 桀、纣之禁,不可胜数,故民因而身为戮, (10) 极也,不能用威适。 (11) 子阳极也好严,有过而折弓者,恐必死,遂应猘狗而弑子阳,极也。 (12) 周鼎有窃, (13) 曲状甚长,上下皆曲,以见极之败也。 (14)

(1) 过,责。识,知。

(2) 令不可从,而非人不从之也。

(3) 不敢登其危者而罪之也。

(4) 不能胜其所任者而罪之也。

(5) 【校】此二句疑当作“则难以为继矣,难以为继”,脱两“难”字,下“知”字衍。

(6) 罪之,罪其为也。

(7) 召,致也。

(8) 【校】旧校云:“一作‘准’。”

(9) 设禁而不禁,为不行也。

(10) 【校】旧校云:“一作‘用’。”案:当是“困”字。

(11) 适,宜也。

(12) 子阳,郑君也,一曰郑相也。好严猛,于罪刑无所赦。家人有折弓者,恐诛,因国人有逐狡狗之扰而杀子阳。极于刑之故也。

(13) 【校】旧校云:“一作‘穷’。”

(14) 未闻。





为欲


六曰:

使民无欲,上虽贤,犹不能用。 (1) 夫无欲者,其视为天子也与为舆隶同, (2) 其视有天下也与无立锥之地同, (3) 其视为彭祖也与为殇子同。 (4) 天子至贵也,天下至富也,彭祖至寿也,诚无欲则是三者不足以劝。 (5) 舆隶至贱也,无立锥之地至贫也,殇子至夭也,诚无欲则是三者不足以禁。会有一欲,则北至大夏,南至北户,西至三危,东至扶木,不敢乱矣; (6) 犯白刃,冒流矢,趣水火, (7) 不敢却也; (8) 晨寤兴,务耕疾庸, (9) 为烦辱,不敢休矣。故人之欲多者,其可得用亦多;人之欲少者,其得用亦少;无欲者,不可得用也。人之欲虽多,而上无以令之,人虽得其欲,人犹不可用也。令人得欲之道,不可不审矣。

(1) 民无欲,不为物动,虽有贤君,不能得用之也。

(2) 舆,众也。

(3) 同,等也。

(4) 彭祖,殷贤大夫也,盖寿七百余岁。九岁以下为下殇,七岁以下为无服殇。

(5) 劝,乐也。

(6) 乱犹难也。

【校】钱詹事云:“扶木即蟠木。古音扶如酺,声转为蟠。《汉书·天文志》‘奢为扶’,郑氏云‘扶当为蟠’。”

(7) 【校】旧校云:“‘趣’一作‘赴’。”

(8) 却犹止也。

(9) “ ”,古“耕”字。

【校】案:上既云“务耕疾庸”,则“ ”必非“耕”字。又似属下句,阙疑可也。

善为上者,能令人得欲无穷,故人之可得用亦无穷也。蛮夷反舌殊俗异习之国, (1) 其衣服冠带、宫室居处、舟车器械、声色滋味皆异,其为欲使一也。 (2) 三王不能革,不能革而功成者,顺其天也; (3) 桀、纣不能离,不能离而国亡者,逆其天也。逆而不知其逆也,湛于俗也。久湛而不去则若性。性异非性,不可不熟。不闻道者,何以去非性哉?无以去非性,则欲未尝正矣。欲不正,以治身则夭,以治国则亡。故古之圣王,审顺其天而以行欲,则民无不令矣,功无不立矣。圣王执一,四夷皆至者,其此之谓也!执一者,至贵也,至贵者无敌。圣王托于无敌,故民命敌焉。

(1) 反舌,夷语,与中国相反,故曰“反舌”也。

(2) 一,同也。

(3) 天,身也。

群狗相与居,皆静无争。投以炙鸡,则相与争矣。 (1) 或折其骨,或绝其筋,争术存也。争术存,因争;不争之术存,因不争。取争之术而相与争,万国无一。凡治国,令其民争行义也;乱国,令其民争为不义也。强国,令其民争乐用也;弱国,令其民争竞不用也。夫争行义乐用与争为不义竞不用,此其为祸福也,天不能覆,地不能载。 (2)

(1) 炙鸡,狗所欲之,故斗争之。

【校】注两“之”字皆衍。

(2) 言其大也。

晋文公伐原, (1) 与士期七日。七日而原不下, (2) 命去之。谋士言曰:“原将下矣。”师吏请待之。公曰:“信,国之宝也。得原失宝,吾不为也。”遂去之。明年,复伐之, (3) 与士期必得原然后反。原人闻之,乃下。卫人闻之,以文公之信为至矣,乃归文公。故曰“攻原得卫”者,此之谓也。文公非不欲得原也,以不信得原,不若勿得也,必诚信以得之,归之者非独卫也。文公可谓知求欲矣。

(1) 原,晋邑。文公复国,原不从,故伐之。今河内轵县北原城是也。

(2) 下,降。

【校】僖廿五年《左氏传》、《淮南·道应训》俱作“三日”,《韩非·外储说左上》作“十日”,《新序·杂事四》作“五日”。

(3) 【校】与《左传》、《韩非》不合。





贵信


七曰:

凡人主必信。信而又信,谁人不亲? (1) 故《周书》曰“允哉允哉”,以言非信则百事不满也。 (2) 故信之为功大矣。信立,则虚言可以赏矣。虚言可以赏,则六合之内皆为己府矣。信之所及,尽制之矣。制之而不用,人之有也; (3) 制之而用之,己之有也。己有之,则天地之物毕为用矣。 (4) 人主有见此论者,其王不久矣;人臣有知此论者,可以为王者佐矣。

(1) 谁犹何也。

(2) 《周书》,逸书也。满犹成。

(3) 人之有,他人之有也。

(4) 毕,尽也。

天行不信,不能成岁;地行不信,草木不大。 (1) 春之德风,风不信,其华不盛,华不盛,则果实不生。 (2) 夏之德暑,暑不信,其土不肥,土不肥,则长遂不精。 (3) 秋之德雨,雨不信,其谷不坚,谷不坚,则五种不成。 (4) 冬之德寒,寒不信,其地不刚,地不刚,则冻闭不开。 (5) 天地之大,四时之化,而犹不能以不信成物,又况乎人事? (6) 君臣不信,则百姓诽谤,社稷不宁;处官不信,则少不畏长,贵贱相轻;赏罚不信,则民易犯法,不可使令; (7) 交友不信,则离散鬱怨,不能相亲; (8) 百工不信,则器械苦伪,丹漆染色不贞。 (9) 夫可与为始,可与为终,可与尊通,可与卑穷者,其唯信乎!信而又信,重袭于身,乃通于天。以此治人,则膏雨甘露降矣,寒暑四时当矣。 (10)

(1) 不信,气节阴阳皆不交,故不成岁也。

(2) 在木曰实,在地曰蓏。

(3) 遂,成也。

(4) 坚,好;成,熟也。

(5) 不开,气不通也。

(6) 乎,于也。

(7) 易,轻也。

(8) 亲,比也。

(9) 贞,正也。

(10) 当犹应也。

齐桓公伐鲁,鲁人不敢轻战,去鲁国五十里而封之。鲁请比关内侯以听, (1) 桓公许之。曹翙谓鲁庄公曰 (2) :“君宁死而又死乎?其宁生而又生乎?”庄公曰:“何谓也?”曹翙曰:“听臣之言,国必广大,身必安乐,是生而又生也;不听臣之言,国必灭亡,身必危辱,是死而又死也。”庄公曰:“请从。”于是明日将盟,庄公与曹翙皆怀剑至于坛上。庄公左搏桓公,右抽剑以自承, (3) 曰:“鲁国去境数百里,今去境五十里,亦无生矣。钧其死也,戮于君前。” (4) 管仲、鲍叔进,曹翙按剑当两陛之间,曰:“且二君将改图,毋或进者!”庄公曰:“封于汶则可,不则请死。”管仲曰:“以地卫君,非以君卫地,君其许之。”乃遂封于汶南,与之盟。归而欲勿予,管仲曰:“不可。人特劫君而不盟,君不知,不可谓智; (5) 临难而不能勿听,不可谓勇;许之而不予,不可谓信。不智不勇不信,有此三者,不可以立功名。予之,虽亡地亦得信。以四百里之地见信于天下,君犹得也。”庄公,仇也;曹翙,贼也。信于仇贼,又况于非仇贼者乎? (6) 夫九合之而合,壹匡之而听,从此生矣。管仲可谓能因物矣。以辱为荣,以穷为通,虽失乎前,可谓后得之矣。物固不可全也。

(1) 【校】梁仲子云:“关内侯,秦爵也。刘昭注《续汉书·百官志》引刘劭《爵制》曰:‘秦都山西,以关内为王畿,故曰关内侯。’然则齐安得有关内侯乎?《管子·大匡》篇载此事云:‘鲁不敢战,去国五十里而为之关,鲁请比于关内以从于齐。’据此,疑‘侯’字衍。”卢云:“案曹沬事出于战国之人所撰造,事既不实,辞亦鄙诞不经,但以耳目所见,施之上世,而不知其有不合也。”

(2) 【校】“曹翙”,《左传》作“曹刿”,《公羊》、《国策》、《史记》并作“曹沬”。

(3) 承,佐也。

【校】梁仲子云:“注非也。《左氏昭廿一年传》‘子皮承宜僚以剑’,哀十六年《传》‘承之以剑’,杜云‘拔剑指其喉’。盖曹翙以剑自向,故下云‘戮于君前’,即以颈血湔衣之意。”

(4) 钧,等也。戮,亦死也。

(5) 【校】《御览》四百三十作“人将劫君而不知,不可谓智”,此“不盟君”三字剩。

(6) 《公羊传》曰:“庄公升坛,曹子手剑而从之,请复汶阳之田。管子曰:‘君许之。’桓公曰:‘诺。’曹子请盟。桓公下,与之盟。要盟可犯,而桓公不欺;曹子可仇,而桓公不怨。桓公之信著乎天下,自柯之盟始焉。”此之谓也。





举难


八曰:

以全举人固难,物之情也。 (1) 人伤尧以不慈之名,舜以卑父之号,禹以贪位之意,汤、武以放弑之谋,五伯以侵夺之事。 (2) 由此观之,物岂可全哉?故君子责人则以人, (3) 自责则以义。责人以人则易足,易足则得人;自责以义则难为非,难为非则行饰。 (4) 故任天地而有余。 (5) 不肖者则不然,责人则以义,自责则以人。责人以义责难瞻,难瞻则失亲; (6) 自责以人则易为,易为则行苟。 (7) 故天下之大而不容也,身取危,国取亡焉。此桀、纣、幽、厉之行也。尺之木必有节目,寸之玉必有瑕瓋。先王知物之不可全也, (8) 故择务而贵取一也。 (9)

(1) 物,事。事难全也。

(2) 伤,毁也。

(3) 【校】梁仲子云:“此即以众人望人之意。”

(4) 饰,读曰敕。敕,正也。

(5) 德饶也。

(6) 难瞻则恐,恐则离叛,故失所亲也。

【校】梁仲子云:“‘瞻’疑当作‘赡’。”

(7) 苟且,不从礼义也。

【校】《管子·水地》篇云“夫玉瑕適皆见,精也”,注云:“瑕適,玉病也。”今此加“玉”旁,乃俗作,字书不载。

(8) 【校】“不可全”,旧本“全”上衍一“不”字,今删。

(9) 一,分。

季孙氏劫公家,孔子欲谕术则见外, (1) 于是受养而便说, (2) 鲁国以訾。 (3) 孔子曰:“龙食乎清而游乎清,螭食乎清而游乎浊,鱼食乎浊而游乎浊。 (4) 今丘上不及龙,下不若鱼,丘其螭邪!”夫欲立功者,岂得中绳哉?救溺者濡,追逃者趋。 (5)

(1) 季孙氏,武子季文子子也。劫夺公家政事而自专之也。孔子欲以道而见远外。

【校】旧校云:“‘谕’一作‘论’。”案:注误,当云“桓子,季平子子也”。末疑有文脱,似当云“孔子欲以道术谕之而虑见远外也”。

(2) 孔子受其养,而季氏便之。

【校】注非也。受其养则不见远外,于以谕道术则便矣。

(3) 訾,毁也。毁孔子也。

(4) 螭,龙之别也。

(5) 趋,走也。

魏文侯弟曰季成,友曰翟璜。 (1) 文侯欲相之,而未能决,以问季充。 (2) 季充对曰:“君欲置相, (3) 则问乐腾与王孙苟端孰贤?” (4) 文侯曰:“善。”以王孙苟端为不肖,翟璜进之;以乐腾为贤,季成进之。 (5) 故相季成。凡听于主,言人不可不慎。季成,弟也,翟璜,友也,而犹不能知,何由知乐腾与王孙苟端哉?疏贱者知,亲习者不知,理无自然。自然而断相,过。季充之对文侯也,亦过。 (6) 虽皆过,譬之若金之与木,金虽柔,犹坚于木。

(1) 【校】亦作“黄”。

(2) 【校】乃李克也,因形近而讹。

(3) 置,立。

(4) 孰,谁。

【校】《新序》四“乐腾”作“乐商”,下同。

(5) 【校】“为不肖”,旧本作“而不肖”,“贤”作“贵”,今并从《新序》改正。

(6) 过,长也。《论语》曰“过犹不及”,言俱不得其适。

孟尝君问于白圭曰:“魏文侯名过桓公,而功不及五伯,何也?” (1) 白圭对曰:“文侯师子夏,友田子方,敬段干木,此名之所以过桓公也。卜相曰‘成与璜孰可’?此功之所以不及五伯也。 (2) 相也者,百官之长也。择者欲其博也。今择而不去二人,与用其仇亦远矣。且师友也者,公可也;戚爱也者,私安也。以私胜公,衰国之政也。然而名号显荣者,三士羽翼之也。” (3)

(1) 孟尝君,齐公子田婴之子田文也。白圭,周人。问文侯功何以不及五伯也。

(2) 卜,择也。成,季成。璜,翟璜也。

(3) 羽翼,佐之。

【校】旧本脱“翼”字,今据李善注《文选》王子渊《四子讲德论》补。《新序》四作“三士翊之也”。注“羽翼”旧倒,《选》注枚叔《七发》引作“羽翼,佐也”。

甯戚欲干齐桓公,穷困无以自进,于是为商旅,将任车以至齐, (1) 暮宿于郭门之外。桓公郊迎客,夜开门,辟任车,爝火甚盛,从者甚众。甯戚饭牛居车下,望桓公而悲,击牛角疾歌。 (2) 桓公闻之,抚其仆之手曰:“异哉!之歌者非常人也。” (3) 命后车载之。桓公反,至,从者以请。 (4) 桓公赐之衣冠,将见之。甯戚见,说桓公以治境内。明日复见,说桓公以为天下。 (5) 桓公大说,将任之。 (6) 群臣争之曰:“客,卫人也。卫之去齐不远,君不若使人问之,而固贤者也, (7) 用之未晚也。”桓公曰:“不然。问之,患其有小恶。以人之小恶,亡人之大美,此人主之所以失天下之士也已。”凡听必有以矣,今听而不复问,合其所以也。且人固难全,权而用其长者, (8) 当举也。桓公得之矣。

(1) 任亦将也。

【校】注非是,与下“辟任车”不可通。《淮南·道应训》注云:“任,载也。《诗》曰‘我任我辇’。”此则是已。

(2) 歌《硕鼠》也。其诗曰“硕鼠硕鼠,无食我黍!三岁贯女,莫我肯顾。逝将去女,适彼乐土,乐土乐土,爰得我所。硕鼠硕鼠,无食我麦!三岁贯女,莫我肯得。逝将去女,适彼乐国,乐国乐国,爰得我直。硕鼠硕鼠,无食我苗!三岁贯女,莫我肯逃。逝将去女,适彼乐郊,乐郊乐郊,谁之永号”者是也。

【校】孙云:“《后汉书·马融传》注引《说苑》曰‘甯戚饭牛于康衢,击车辐而歌《硕鼠》’,与此正合。”梁仲子云:“今《说苑·善说》篇云:‘甯戚饭牛康衢,击车辐而歌《顾见》,桓公得之,霸也。’以上下文义求之,‘顾见’当是‘硕鼠’之讹。”卢云:“案《史记·邹阳传》集解引应劭曰:‘齐桓公夜出迎客,而甯戚疾击其牛角商歌曰:南山矸,白石烂,生不遭尧与舜禅。短布单衣适至骭,从昏饭牛薄夜半,长夜曼曼何时旦。’此歌出《三齐记》。《艺文类聚》又载一篇云:‘沧浪之水白石粲,中有鲤鱼长尺半。縠布单衣裁至骭,清朝饭牛至夜半。黄犊上坂且休息,吾将舍汝相齐国。’李善注《文选》成公子安《啸赋》又载一篇云:‘出东门兮厉石班,上有松柏清且兰。粗布衣兮缊缕,时不遇兮尧舜主。牛兮努力食细草,大臣在尔侧,吾当与尔适楚国。’三歌真赝虽不可知,合之亦自成章法。仁和陈嗣倩云:‘疾商歌,殆非一歌也。’今故具录之,以备参考焉。”

(3) 【校】《新序》五“之”作“此”。

(4) 请所置。

(5) 为,治。

(6) 任,用也。

(7) 【校】“而”与“如”同。

(8) 【校】《新序》作“当此举也”。





第二十卷 恃君览



恃君


一曰:

凡人之性,爪牙不足以自守卫, (1) 肌肤不足以扞寒暑, (2) 筋骨不足以从利辟害, (3) 勇敢不足以却猛禁悍, (4) 然且犹裁万物,制禽兽,服狡虫, (5) 寒暑燥湿弗能害, (6) 不唯先有其备,而以群聚邪?群之可聚也,相与利之也。利之出于群也,君道立也。 (7) 故君道立则利出于群, (8) 而人备可完矣。昔太古尝无君矣, (9) 其民聚生群处,知母不知父,无亲戚兄弟夫妻男女之别,无上下长幼之道,无进退揖让之礼,无衣服履带宫室畜积之便,无器械舟车城郭险阻之备,此无君之患。 (10) 故君臣之义,不可不明也。 (11) 自上世以来,天下亡国多矣,而君道不废者,天下之利也。 (12) 故废其非君,而立其行君道者。 (13) 君道何如?利而物利章。 (14)

(1) 卫,扞也。

(2) 扞,御也。

(3) 从,随也。

(4) 禁,止也。

(5) 狡虫,虫之狡害者。

(6) 古人知为之备。

(7) 众之所奉戴,故道立。

(8) 群,众也。

(9) 太古,上古。两仪之始,未有君臣之制。

(10) 上苟所无者,无以化下,故以无君为患。

(11) 明,知也。

(12) 君施庆赏威刑以正之,故天下之利也。

(13) 行,奉也。

(14) 熊虎为旗章,明识也。

非滨之东, (1) 夷秽之乡, (2) 大解、陵鱼、其、鹿野、摇山、扬岛、大人之居,多无君; (3) 扬、汉之南, (4) 百越之际, (5) 敝凯诸、夫风、余靡之地,缚娄、阳禺、 兜之国,多无君; (6) 氐、羌、呼唐、离水之西,僰人、野人、 (7) 篇窄之川,舟人、送龙、突人之乡,多无君; (8) 雁门之北,鹰隼、所鸷、须窥之国,饕餮、穷奇之地,叔逆之所,儋耳之居,多无君。 (9) 此四方之无君者也。其民糜鹿禽兽, (10) 少者使长,长者畏壮,有力者贤, (11) 暴傲者尊,日夜相残,无时休息,以尽其类。 (12) 圣人深见此患也,故为天下长虑, (13) 莫如置天子也; (14) 为一国长虑,莫如置君也。置君非以阿君也, (15) 置天子非以阿天子也,置官长非以阿官长也。德衰世乱,然后天子利天下, (16) 国君利国,官长利官,此国所以递兴递废也,乱难之所以时作也。 (17) 故忠臣廉士,内之则谏其君之过也, (18) 外之则死人臣之义也。 (19)

(1) 朝鲜乐浪之县,箕子所封,滨于东海也。

【校】“非”疑当作“北”,犹言北海之东也。

(2) 东方曰夷。秽,夷国名。

(3) 东方之夷,无有君长。

(4) 扬州、汉水南。

(5) 越有百种。

(6) 皆南越之夷无君者。

(7) 僰,读如匍匐之匐。

(8) 西方之戎无君者。先言氐、羌,后言突人,自近及远也。

(9) 北方狄无君者也。孔子曰“夷狄之有君,不如诸夏之亡”,故曰“多无君”也。

(10) 不知礼义,无长幼之别,如麋鹿禽兽也。

(11) 贤,豪者也。

(12) 类,种也。

(13) 虑,计也。

(14) 置,立也。

(15) 阿犹私为也。

(16) 幼奉长,卑事尊,强不得陵弱,众不得暴寡,以此利之。

【校】卢云:“注非是。利天下,言以天下为己利也。古之圣王有天下而不与后世,则以天下为己利,故有兴有废,而乱难时作。如此方与下文意相承接。”

(17) 不得常,施时盗作耳。

(18) 谏,止也。

(19) 义重于身。

豫让欲杀赵襄子, (1) 灭须去眉,自刑以变其容,为乞人而往乞于其妻之所。其妻曰:“状貌无似吾夫者,其音何类吾夫之甚也?”又吞炭以变其音。其友谓之曰:“子之所道甚难而无功。 (2) 谓子有志则然矣,谓子智则不然。以子之材而索事襄子, (3) 襄子必近子。子得近而行所欲,此甚易而功必成。”豫让笑而应之曰:“是先知报后知也,为故君贼新君矣,大乱君臣之义者无此,失吾所为为之矣。 (4) 凡吾所为为此者,所以明君臣之义也,非从易也。”

(1) 欲为智伯杀赵襄子也,已说在上篇。

(2) 【校】所道犹言所由。《赵策》无“所”字。

(3) 索,求也。

(4) 【校】《赵策》作“是为先知报后知,为故君贼新君,大乱君臣之义者无此矣”,无“失吾所为为之”六字。两本皆可通。无此犹言无如此。吴师道疑其有缺字,非也。

柱厉叔事莒敖公, (1) 自以为不知,而去居于海上。 (2) 夏日则食菱芡, (3) 冬日则食橡栗。 (4) 莒敖公有难,柱厉叔辞其友而往死之。 (5) 其友曰:“子自以为不知故去,今又往死之,是知与不知无异别也。” (6) 柱厉叔曰:“不然。自以为不知故去,今死而弗往死,是果知我也。 (7) 吾将死之,以丑后世人主之不知其臣者也, (8) 所以激君人者之行,而厉人主之节也。 (9) 行激节厉,忠臣幸于得察。 (10) 忠臣察则君道固矣。” (11)

(1) 莒,子国也。敖,公谥。公,君也。

【校】案:此与《列子·说符》篇同。《说苑·立节》篇作“莒穆公有臣曰朱厉附”。

(2) 柱厉叔自以不为敖公之所知,而远去居于海上也。

(3) 菱,芰也;芡,鸡头也,一名雁头;生水中。

(4) 橡,皂斗也,其状似栗。

(5) 往死敖公之难也。

(6) 言叔为不见知于敖公而舍之去,今复往死其难,是与见知、不见知无别异也。

(7) 今不死其难,是为使敖公果知我为不良臣也。

(8) 丑,愧也。唯明君能知忠臣耳,敖公弗及也。死其难,可以使后世不知良臣之君惭于不知人也。

(9) 激,发也。所以发起君人之行。厉,高也。人君务在知人,知人则哲,所以厉人主之志节也。

【校】“人主”,《御览》六百二十一作“人臣”,非是。下云“行激节厉,忠臣幸于得察”,则“节厉”正指人主言。

(10) 察,知也。

(11) 臣见知则尽忠以卫上,故君道安固不危殆也。





长利


二曰:

天下之士也者,虑天下之长利,而固处之以身若也。利虽倍于今,而不便于后,弗为也; (1) 安虽长久,而以私其子孙,弗行也。 (2) 自此观之,陈无宇之可丑亦重矣, (3) 其与伯成子高、周公旦、戎夷也,形虽同,取舍之殊,岂不远哉? (4)

(1) 为,施也。

(2) 私,利也。

(3) 陈无宇,齐大夫,陈须无之子桓子也。丑,谓其贪也。与鲍文子俱伐栾、高氏,战于稷,栾、高氏败,又败于庄,国人追之,又败于鹿门,栾施、高强出奔,陈、鲍分其室,是其贪禄也。

(4) 伯成子高辞诸侯而耕。周公旦股肱周室、辅翼成王而致太平。戎夷,齐之仁人也。陈无宇虽身形与之同,然其行贪欲,相去绝远也。

尧治天下,伯成子高立为诸侯。尧授舜,舜授禹,伯成子高辞诸侯而耕。禹往见之,则耕在野。禹趋就下风而问曰:“尧理天下,吾子立为诸侯。今至于我而辞之,故何也?” (1) 伯成子高曰:“当尧之时,未赏而民劝,未罚而民畏。民不知怨,不知说,愉愉其如赤子。今赏罚甚数,而民争利且不服,德自此衰,利自此作, (2) 后世之乱自此始。 (3) 夫子盍行乎?无虑吾农事。” (4) 协而耰,遂不顾。 (5) 夫为诸侯,名显荣,实佚乐,继嗣皆得其泽,伯成子高不待问而知之,然而辞为诸侯者,以禁后世之乱也。 (6)

(1) 【校】《庄子·天地》篇作“其故何也”。

(2) 作,起也。

【校】《庄子》作“刑自此立”,《新序·节士》篇作“刑自此繁”。

(3) 始,首也。

(4) 盍,何不也。行,去也。虑犹乱也。

【校】《庄子》作“无落吾事”。虑、落声相近。

(5) 协,和悦也。耰,覆种也。顾,视也。

(6) 以止后世争荣之乱也。

辛宽见鲁缪公,曰:“臣而今而后知吾先君周公之不若太公望封之知也。昔者太公望封于营丘,之渚海阻山高,险固之地也, (1) 是故地日广,子孙弥隆。 (2) 吾先君周公封于鲁,无山林溪谷之险,诸侯四面以达, (3) 是故地日削,子孙弥杀。” (4) 辛宽出,南宫括入见。公曰:“今者宽也非周公,其辞若是也。”南宫括对曰:“宽少者,弗识也。 (5) 君独不闻成王之定成周之说乎?其辞曰:‘惟余一人,营居于成周。惟余一人,有善易得而见也,有不善易得而诛也。’ (6) 故曰善者得之,不善者失之,古之道也。 (7) 夫贤者岂欲其子孙之阻山林之险以长为无道哉?小人哉,宽也!”今使燕爵为鸿鹄凤皇虑,则必不得矣。 (8) 其所求者,瓦之间隙,屋之翳蔚也, (9) 与一举则有千里之志,德不盛、义不大则不至其郊。 (10) 愚庳之民,其为贤者虑,亦犹此也。固妄诽訾,岂不悲哉? (11)

(1) 【校】孙云:“李善注《文选》司马相如《子虚赋》引‘辛宽曰:太公望封于营邱,渚海阻山’,无‘之’字、‘高’字。‘渚’属下读,是。营邱恐不得言渚也。”梁仲子云:“赋云‘齐东陼巨海’,注引此,则‘渚’当为‘陼’。”卢云:“案韦昭注《越语》云‘水边曰陼’,此正言边海耳。‘山高’疑本是一‘嵩’字误分,《尔雅》‘山大而高嵩中岳’,盖依此名,《尔雅》本非专为中岳作释,故齐亦可言嵩。余当从《选》注。”

(2) 广,大也。隆,盛也。

(3) 达,通也。

(4) 削,小也。杀,衰也。

(5) 少,小也。不知也。

(6) 言恃德不恃险也。

(7) 得之者,若汤、武也;失之者,若桀、纣。故曰“古之道也”。

(8) 燕爵谕辛宽也。言宽亦不能为贤者虑也。

(9) 燕爵志小而近也。

(10) 为圣德之君至其郊也。

(11) 亦如燕爵为鸿鹄凤皇虑,何时能得?既不得,又妄诽谤訾毁之,故曰“岂不悲哉”,痛伤之也。

戎夷违齐如鲁,天大寒而后门, (1) 与弟子一人宿于郭外,寒愈甚,谓其弟子曰:“子与我衣,我活也;我与子衣,子活也。我国士也,为天下惜死; (2) 子不肖人也,不足爱也。 (3) 子与我子之衣。”弟子曰:“夫不肖人也,又恶能与国士之衣哉?” (4) 戎夷太息叹曰:“嗟乎!道其不济夫。” (5) 解衣与弟子,夜半而死。弟子遂活。谓戎夷其能必定一世,则未之识; (6) 若夫欲利人之心,不可以加矣。 (7) 达乎分,仁爱之心识也,故能以必死见其义。 (8)

(1) 违,去。去齐至鲁也。后门,日夕门已闭也。

(2) 惜,爱也。

(3) 爱亦惜也。

(4) 恶,安也。不肖人亦自爱其死,安能与国士之衣哉?

(5) 死之,道其不济也。

(6) 识,知也。

(7) 加,上也。

(8) 诱以戎夷不义之义耳。欲求弟子之衣以惜其死,是不义也;弟子拒之以不肖人恶能与国士之衣,计不能两生,穷乃解衣,是不义之义也。《淮南记》曰:“楚有卖其母者,而谓其买者曰:‘此母老矣,幸善食之。’”不亦不义也?

【校】注末“也”字当与“邪”同,犹言此岂可谓之义?所引《淮南记》见《说山训》。





知分


三曰:

达士者,达乎死生之分。 (1) 达乎死生之分,则利害存亡弗能惑矣。 (2) 故晏子与崔杼盟而不变其义; (3) 延陵季子,吴人愿以为王而不肯; (4) 孙叔敖三为令尹而不喜, (5) 三去令尹而不忧; (6) 皆有所达也, (7) 有所达则物弗能惑。 (8)

(1) 君子死义,不求苟生,不义而生弗为也,故曰“达乎死生之分”。《淮南记》曰:“左手据天下之图,右手刎其喉,愚夫弗为,生贵于天下也。”死君亲之难者,则当视死如归,盖义重于身也,此之谓达于死生之分者也。

(2) 不为利存而遂苟生,不为害亡而辞死,故曰利害存亡弗能惑移也。

(3) 崔子盟,国人曰:“所不与崔、庆者不祥。”晏子仰天叹曰:“婴所不惟忠于君,利社稷者是与。”故曰“不变其义”。

【校】旧本注多讹,今从许本参以《左传》改正。“是与”下《左传》有“有如上帝”四字。

(4) 季子,吴寿梦子札也,不肯为王,去之延陵,不入吴国,故曰“延陵季子”也。

【校】注“子札”旧本作“孙子”,讹,今改正。

(5) 叔敖, 贾伯盈之子。

(6) 令尹,楚卿也。《论语》曰“令尹子文”,不云叔敖。

(7) 达于高位疾颠、厚味腊毒者也。

(8) 惑,动也。

荆有次非者,得宝剑于干遂。 (1) 还反涉江, (2) 至于中流,有两蛟夹绕其船。 (3) 次非谓舟人曰:“子尝见两蚊绕船能两活者乎?”船人曰:“未之见也。”次非攘臂祛衣,拔宝剑曰:“此江中之腐肉朽骨也,弃剑以全己,余奚爱焉!”于是赴江刺蛟, (4) 杀之而复上船,舟中之人皆得活。荆王闻之,仕之执圭。 (5) 孔子闻之曰:“夫善哉!不以腐肉朽骨而弃剑者,其次非之谓乎!”

(1) 干遂,吴邑。

【校】“次非”,《汉书·宣帝纪》注如淳引作“兹非”,《后汉书》马融、蔡邕等传注及《北堂书钞》百三十七并引作“佽飞”,李善注《文选》郭景纯《江赋》作“佽非”,唯杨倞注《荀子·劝学》篇所引同。“干遂”,如淳作“干将”,杨倞作“于越”。

(2) 涉,度也。

(3) 鱼满二千斤为蛟。

【校】《淮南》注作“二千五百斤”。

(4) 赴,入也。

(5) 《周礼》“侯执信圭”,楚以次非勇武而侯之。

禹南省,方济乎江,黄龙负舟。舟中之人五色无主。禹仰视天而叹曰:“吾受命于天,竭力以养人。生,性也;死,命也。余何忧于龙焉?” (1) 龙俯耳低尾而逝。 (2) 则禹达乎死生之分、利害之经也。 (3) 凡人物者,阴阳之化也。阴阳者,造乎天而成者也。天固有衰嗛废伏,有盛盈蚠息; (4) 人亦有困穷屈匮,有充实达遂。 (5) 此皆天之容物理也,而不得不然之数也。古圣人不以感私伤神, (6) 俞然而以待耳。 (7)

(1) 忧,惧也。

(2) 逝,去也。

(3) 经,道。

(4) 【校】“蚠”,梁仲子疑“坌”。案《贾谊书》“坌冒楚棘”,一作“蚠”。

(5) 达,通。遂,成。

(6) 感念私邪,伤神性也。

(7) 俞,安。

晏子与崔杼盟,其辞曰:“不与崔氏而与公孙氏者,受其不祥!” (1) 晏子俯而饮血,仰而呼天曰:“不与公孙氏而与崔氏者,受此不祥!” (2) 崔杼不说,直兵造胸,句兵钩颈, (3) 谓晏子曰:“子变子言, (4) 则齐国吾与子共之;子不变子言,则今是已!” (5) 晏子曰:“崔子,子独不为夫《诗》乎?《诗》曰:‘莫莫葛藟,延于条枚。凯弟君子,求福不回。’ (6) 婴且可以回而求福乎?子惟之矣!” (7) 崔杼曰:“此贤者,不可杀也。”罢兵而去。晏子援绥而乘, (8) 其仆将驰,晏子抚其仆之手曰 (9) :“安之,毋失节。疾不必生,徐不必死。鹿生于山,而命悬于厨。今婴之命有所悬矣。”晏子可谓知命矣。命也者,不知所以然而然者也,人事智巧以举错者不得与焉。故命也者,就之未得,去之未失。 (10) 国士知其若此也,故以义为之决而安处之。 (11)

(1) 公孙氏,齐群公子之子,故曰“公孙氏”。公党之 [1] 不与崔杼同者也,故曰“不祥”也。

(2) 反其盟也。

(3) 直,矛也。句,戟也。

(4) 变,更。

(5) 已,竟也。言今竟子。

【校】注“竟”旧本作“竞”,误。杼欲置晏子于死,则是终竟之。今俗间恶詈人语尚有相似者。

(6) 《诗·大雅·旱麓》之卒章。莫莫,葛藟之貌。延蔓于条枚之上,得其性也。乐易之君子,求福不以邪道,顺于天性,以正直受大福。

【校】“延于条枚”,此《韩诗》,见《外传》二,《后汉书·黄琬传》注同。“岂弟”作“凯弟”,《礼记·表记》同。注“旱麓”,李本作“干麓”。

(7) 惟,宜也。

【校】梁仲子云:“当训为思。”

(8) 【校】“援”旧多作“授”,汪本作“受”。案《意林》作“援”,今从之。

(9) 【校】“抚”,旧本作“无良”,讹,案《晏子·杂上》及《韩诗外传》二俱作“抚”,《新序·义勇》篇作“拊”,俱无“良”字,今据删正。

(10) 蹈义就死,未必死也,故曰“就之未得”。苟从不义,以去死求生,未必生,故曰“去之未失”也。

(11) 处,居也。

白圭问于邹公子夏后启曰 (1) :“践绳之节,四上之志,三晋之事,此天下之豪英。 (2) 以处于晋,而迭闻晋事,未尝闻践绳之节、四上之志。 (3) 愿得而闻之。” (4) 夏后启曰:“鄙人也,焉足以问?” (5) 白圭曰:“愿公子之毋让也。”夏后启曰:“以为可为,故为之。为之,天下弗能禁矣。 (6) 以为不可为,故释之。释之,天下弗能使矣。” (7) 白圭曰:“利弗能使乎?威弗能禁乎?”夏后启曰:“生不足以使之,则利曷足以使之矣? (8) 死不足以禁之,则害曷足以禁之矣?” (9) 白圭无以应。夏后启辞而出。 (10) 凡使贤不肖异, (11) 使不肖以赏罚, (12) 使贤以义。 (13) 故贤主之使其下也必义,审赏罚,然后贤不肖尽为用矣。 (14)

(1) 夏后启,邹公子之名。

(2) 践绳之节,正直也。四上,谓君也。卿、大夫、士与君为四,四者之中,君处其上,故曰“四上之志”。晋之三卿韩、魏、赵氏,皆以豪英之才专制晋国,三分之为诸侯,卒皆称王,故曰“三晋之事,此天下之豪英”、万人为英,百人为豪。

(3) 处,居。居于晋,数闻三晋之事。

【校】旧校云:“‘迭’,一作‘亟’。”今案注,作“亟”为是。

(4) 愿闻践绳之节、四上之志也。

(5) 言不足问。

(6) 禁,止也。

(7) 释,舍。

(8) 生重利轻。言令必生犹不可使也,但以所利谕之,何足以使之?

(9) 死重害轻也。言为义者,虽死为之,故曰“不足以禁之”。死且犹弗禁,何况害也,何足以禁之也?

(10) 出,去。

(11) 使贤以义,使不肖以利,故曰“异”也。

(12) 言赏必生、罚必死,不肖者喜生恶死,则可使矣。

(13) 贤者不畏义死,不好不义生,唯义之所在,死生一也。

(14) 尽可得使为己用也。





召类


四曰:

类同相召, (1) 气同则合, (2) 声比则应。 (3) 故鼓宫而宫应, (4) 鼓角而角动。 (5) 以龙致雨,以形逐影。 (6) 祸福之所自来,众人以为命,焉不知其所由。故国乱非独乱,有必召寇。 (7) 独乱未必亡也,召寇则无以存矣。

(1) 召,致也。

(2) 合,会也。

(3) 应,和也。

(4) 鼓大宫,小宫应。

(5) 击大角,小角动。

(6) 龙,水物也,故致雨。影出于形,形行日中则影随之,故曰“以形逐影”。

(7) 召,致。

【校】有读曰又。

凡兵之用也,用于利,用于义。 (1) 攻乱则服,服则攻者利; (2) 攻乱则义,义则攻者荣。 (3) 荣且利,中主犹且为之,有况于贤主乎? (4) 故割地宝器,戈剑卑辞屈服,不足以止攻,唯治为足。 (5) 治则为利者不攻矣, (6) 为名者不伐矣。 (7) 凡人之攻伐也,非为利则固为名也。名实不得,国虽强大,则无为攻矣。 (8) 兵所自来者久矣。尧战于丹水之浦,以服南蛮; (9) 舜却苗民,更易其俗; (10) 禹攻曹魏、屈骜、有扈,以行其教。 (11) 三王以上,固皆用兵也。乱则用,治则止。治而攻之,不祥莫大焉;乱而弗讨,害民莫长焉。此治乱之化也, (12) 文武之所由起也。文者爱之徵也,武者恶之表也。爱恶循义,文武有常,圣人之元也。 (13) 譬之若寒暑之序,时至而事生之。圣人不能为时,而能以事适时。事适于时者,其功大。 (14)

(1) 《传》曰:“利,义之和也。”

(2) 得其利。

(3) 得荣名也。

(4) 【校】有读曰又。

(5) 足以止人攻。

(6) 为利动者不来攻己。

(7) 为武移者不来伐己。

(8) 无名实之国虽强大,则无为往攻之矣。《传》曰“取乱侮亡”此是也。

(9) 丹水在南阳。浦,岸也,一曰崖也。

【校】梁仲子云:“《水经·丹水注》引作‘尧有丹水之战,以服南蛮’。”

(10) 苗民,有苗也。却犹止。更,改。

(11) 《春秋传》曰“启伐有扈”,言屈骜,不知出何书也。

【校】案:《路史·国名纪》:“夏后攻曹魏、屈骜,《吕览》云‘启’。《潜夫论》‘曹,姜姓’。詹伯曰‘祖自夏,以稷、魏、骀为吾西土’。《盟会图》云‘嬴姓。隰之吉乡北有古屈城,北屈也’。”旧本“禹攻曹魏”下有小注“攻伐”二字,此殊可省,且其离句亦非也。

(12) 化,变也。

(13) 元,宝。

(14) 事之适得其时,则无不成,故功大。

士尹池为荆使于宋,司城子罕觞之。 (1) 南家之墙犨于前而不直, (2) 西家之潦径其宫而不止。 (3) 士尹池问其故, (4) 司马子罕曰:“南家工人也,为鞔者也。 (5) 吾将徙之,其父曰:‘吾恃为鞔以食三世矣, (6) 今徙之,是宋国之求鞔者不知吾处也,吾将不食。 (7) 愿相国之忧吾不食也。’为是故,吾弗徙也。西家高,吾宫庳,潦之经吾宫也利,故弗禁也。”士尹池归荆,荆王适兴兵而攻宋,士尹池谏于荆王曰:“宋不可攻也。其主贤, (8) 其相仁。 (9) 贤者能得民, (10) 仁者能用人。 (11) 荆国攻之,其无功而为天下笑乎!”故释宋而攻郑。孔子闻之曰:“夫修之于庙堂之上,而折冲乎千里之外者,其司城子罕之谓乎!” (12) 宋在三大万乘之间, (13) 子罕之时,无所相侵,边境四益, (14) 相平公、元公、景公以终其身,其唯仁且节与! (15) 故仁节之为功大矣。 (16) 故明堂茅茨蒿柱,土阶三等,以见节俭。 (17)

(1) 司城,司空,卿官。宋武公名司空,故改为司城。觞,爵饮尹池酒也。

【校】“士尹池”,《御览》四百十九引作“工尹他”,《新序·刺奢》篇与此同。

(2) 犨犹出。曲出子罕堂前也。

(3) 西家地高,潦东流经子罕之宫而不禁。

【校】“径”,《新序》、《御览》作“经”。旧校云:“一作‘注’。”孙云:“李善注《文选》张景阳《杂诗》引作‘注于庭下而不止’。”

(4) 问不直墙、不止潦之故。

(5) 鞔,履也。作履之工也。一曰:鞔,靷也。作车靷之工也。

【校】“者也”,旧本作“百也”,讹,今改正。《说文》云“鞔,履空也”,徐曰“履殻”。

(6) 作鞔以共食。

(7) 鞔不售,无以自食。

(8) 主,君。

(9) 相,子罕。

(10) 得民欢心。

(11) 人为之用也。

(12) 冲车所以冲突敌之军,能陷破之也。有道之国,不可攻伐,使欲攻己者折还其冲车于千里之外,不敢来也。

(13) 南有楚,北有晋,东有齐,故曰“三大万乘之间”也。

(14) 四境不侵削则为益。

(15) 节,俭也。

(16) 按《春秋》,子罕杀宋昭公,不但相三君以终身。

【校】梁伯子云:“春秋时,子罕是乐喜,乃宋贤臣,奈何以为杀君乎?战国时,宋亦有昭公,其时亦有子罕,逐君擅政,如《韩非子》、《韩诗外传》、《淮南》、《说苑》诸书所说耳。”

(17) 等,级也。茅可覆屋,蒿非柱任也,虽云俭节,实所未闻。

【校】案:《大戴·盛德》篇云:“周时德泽洽和,蒿茂大,以为宫柱,名蒿宫也。”

赵简子将袭卫,使史默往睹之, (1) 期以一月,六月而后反。 (2) 赵简子曰:“何其久也?”史默曰:“谋利而得害,犹弗察也。 (3) 今蘧伯玉为相,史鰌佐焉, (4) 孔子为客,子贡使令于君前,甚听。 (5) 《易》曰:‘涣其群,元吉。’涣者贤也,群者众也,元者吉之始也。‘涣其群元吉’者,其佐多贤也。” (6) 赵简子按兵而不动。凡谋者,疑也。疑则从义断事,从义断事则谋不亏,谋不亏则名实从之。 (7) 贤主之举也,岂必旗偾将毙而乃知胜败哉?察其理而得失荣辱定矣。故三代之所贵,无若贤也。 (8)

(1) 睹,视。

【校】《御览》四百二引作“瞆之”,注“瞆,视也,音贵”。案:睹,见也,疑非视义。

(2) 反,还也。

(3) 察,知。

(4) 伯玉,卫大夫蘧庄子无咎之子瑗,谥曰成子。史鰌亦卫之大夫,字子鱼。《论语》云“直哉史鱼”。

(5) 君从其言。

(6) 谓孔子、子贡之客也。吴公子札适卫,说蘧瑗、史鰌、公子荆、公叔发、公子翚曰“卫多君子,未有患也”,故曰“其佐多贤也”。

【校】案:《左传》“蘧瑗”下有“史狗”,陆德明作“史朝”,此公子翚疑是“鼂”之讹,即“朝”也。但公子朝通于宣姜,惧而作乱,不得为贤。梁伯子云“或是公孙朝”。

(7) 既有美名,又有其实,故曰“名实从之”。

(8) 若,如也。





达郁


五曰:

凡人三百六十节、九窍、五藏、六府,肌肤欲其比也, (1) 血脉欲其通也, (2) 筋骨欲其固也, (3) 心志欲其和也, (4) 精气欲其行也, (5) 若此则病无所居而恶无由生矣。病之留、恶之生也,精气郁也。 (6) 故水郁则为污, (7) 树郁则为蠹, (8) 草郁则为蒉。 (9) 国亦有郁。生德不通, (10) 民欲不达,此国之郁也。国郁处久,则百恶并起而万灾丛至矣, (11) 上下之相忍也,由此出矣。 (12) 故圣王之贵豪士与忠臣也,为其敢直言而决郁塞也。

(1) 比犹致也。

【校】谓致密。

(2) 通,利。

(3) 固,坚。

(4) 和,调也。

(5) 精气以行血脉,荣卫三百六十节,故曰“欲其行也”。

(6) 郁,滞,不通也。

(7) 水浅不流,污也。

(8) 蠹,蝎,木中之虫也。

(9) 蒉,秽。

【校】梁仲子云:“《续汉书·郡国志三》注引《尔雅》‘木立死曰菑’,又引此‘草郁即为菑’,疑‘蒉’本是‘ ’字,即‘菑’也,因形近而讹。”

(10) 【校】“生德”疑“主德”。

(11) 丛,聚也。

(12) 出,生也。

周厉王虐民,国人皆谤。 (1) 召公以告曰:“民不堪命矣!”王使卫巫监谤者, (2) 得则杀之。国莫敢言,道路以目。 (3) 王喜,以告召公曰:“吾能弭谤矣!” (4) 召公曰:“是障之也,非弭之也。 (5) 防民之口,甚于防川。川壅而溃,败人必多。夫民犹是也。是故治川者决之使导,治民者宣之使言。是故天子听政,使公卿列士正谏,好学博闻献诗,矇箴师诵, (6) 庶人传语, (7) 近臣尽规, (8) 亲戚补察,而后王斟酌焉。 (9) 是以下无遗善, (10) 上无过举。 (11) 今王塞下之口而遂上之过,恐为社稷忧。”王弗听也。三年,国人流王于彘。 (12) 此郁之败也。郁者,不阳也。周鼎著鼠,令马履之,为其不阳也。不阳者,亡国之俗也。

(1) 谤,怨。

(2) 召公,周大夫召公奭也。监,视。

【校】召公奭未必至厉王时尚在。据韦昭注《周语》,以为召康公之后穆公虎也。

(3) 以目相视而已,不敢失言。

(4) 弭,止也。

(5) 障,防。

(6) 目不见曰矇。师,瞽师。《诗》云“矇叟奏功”。

【校】《周语》云:“使公卿至于列士献诗,瞽献曲,史献书,师箴,瞍赋,矇诵,百工谏。”注引《诗》与今《毛诗》异。案《诗释文》云:“‘瞍’,依字作‘叟’。”又案《史记·屈原传》集解亦引作“奏功”。

(7) 庶人,无官者,不得见王,故传语,因人以通。

(8) 规,谏。

(9) 斟酌,取其善而行。

(10) 善皆达王所。

(11) 过,失。

(12) 流,放也。彘,河东永安是也。

管仲觞桓公。日暮矣,桓公乐之而徵烛。 (1) 管仲曰:“臣卜其昼,未卜其夜。君可以出矣。” (2) 公不说曰:“仲父年老矣,寡人与仲父为乐将几之?请夜之。” (3) 管仲曰:“君过矣。夫厚于味者薄于德,沉于乐者反于忧。壮而怠则失时, (4) 老而解则无名。 (5) 臣乃今将为君勉之, (6) 若何其沉于酒也?”管仲可谓能立行矣。凡行之墯也于乐, (7) 今乐而益饬; (8) 行之坏也于贵, (9) 今主欲留而不许。伸志行理,贵乐弗为变,以事其主。此桓公之所以霸也。 (10)

(1) 觞,飨也。徵,求也。

【校】“日暮”旧作“曰暮”,讹,今改正。

(2) 出,罢。

【校】疑是“几何”。

(3) 以夜继昼。

(4) 怠,懈。

(5) 无善终之名。

【校】注旧本作“之始”,讹。

(6) 勉,励。励君使不沉于夜乐。

(7) 墯,坏。酣乐。

(8) 饬,正也。

(9) 贵则骄。

(10) 管仲不与桓公烛,不留桓公夜乐,所以能致桓公于霸也。

【校】梁伯子云:“《管子·中匡》篇所载略同。又《说苑·反质》篇以为景公、晏子事,恐皆由《左传》而附会耳”。

列精子高听行乎齐湣王, (1) 善衣东布衣,白缟冠,颡推之履,特会朝雨袪步堂下,谓其侍者曰:“我何若?” (2) 侍者曰:“公姣且丽。” (3) 列精子高因步而窥于井,粲然恶丈夫之状也, (4) 喟然叹曰:“侍者为吾听行于齐王也,夫何阿哉! (5) 又况于所听行乎万乘之主?人之阿之亦甚矣, (6) 而无所镜其残,亡无日矣。 (7) 孰当可而镜? (8) 其唯士乎!” (9) 人皆知说镜之明己也,而恶士之明己也。 (10) 镜之明己也功细, (11) 士之明己也功大。 (12) 得其细,失其大,不知类耳。 (13)

(1) 列精子高,六国时贤人也。听行,其德行见敬于齐王也。湣王,宣王之子。

(2) 颡推之履,弊履也。袪步,举衣而步也。列精子高自谓其从者曰:我好丑如何也?

【校】郑注《礼记》“深衣曰善衣,朝祭之服也”,然则颡推之履必非弊履可知。列精子高方且自矜其容以问侍者,恶有著弊履者乎?高不能注,不若阙诸。

(3) 姣、丽,皆好貌也。

【校】孙云:“李善注《文选》陆士衡《日出东南隅行》‘高台多妖丽’引此‘姣’作‘妖’。”

(4) 临井自照,见不好,故曰“恶丈夫之状也”。

(5) 阿,曲媚也。列精子高言侍者以我为齐王所听而敬,谓我美丽,不言恶,故曰阿我也。

【校】注“以我”,旧本缺“以”字,今补。

(6) 万乘之主,谓齐王。从者且犹阿我而云美且丽也,人之阿齐王,齐王实不良而言其良,甚于己侍者之言也。

【校】此又影合“邹忌修”事。

(7) 言齐王无所用自见其残暴也,亡无期日矣。

(8) 孰,能。镜,照。

(9) 独士履礼蹈正,不阿于俗,而能镜之也。

(10) 镜明见人之丑,而人不椎镜破之,而扢以玄锡,摩以白旃,是说镜之明己也。士有明己者,陈己之短,欲令改之,以除其病,而不德之,反欲杀之,是恶士之明己也。

【校】注“丑”旧作“首”,又“改”作“长”,皆讹,今案文义改正。

(11) 细,小。

(12) 正己之服而以匡君致治,安定社稷,故功之大也。

(13) 类,事。

赵简子曰:“厥也爱我,铎也不爱我。 (1) 厥之谏我也,必于无人之所; (2) 铎之谏我也,喜质我于人中, (3) 必使我丑。” (4) 尹铎对曰:“厥也爱君之丑也, (5) 而不爱君之过也; (6) 铎也爱君之过也,而不爱君之丑也。臣尝闻相人于师,敦颜而土色者忍丑。 (7) 不质君于人中,恐君之不变也。” (8) 此简子之贤也,人主贤则人臣之言刻。 (9) 简子不贤,铎也卒不居赵地, (10) 有况乎在简子之侧哉? (11)

(1) 厥,赵厥,赵简子家臣也,铎,尹铎,亦家臣也。《传》曰:“季孙之爱我,疾疹也。孟孙之恶我,药石也。美疹不如恶石。”此之谓也。

【校】梁仲子云:“《说苑·臣术》篇作‘尹绰’、‘赦厥’,此注云‘赵厥’,未知所本。又‘疹’,《左传》作‘疢’。”

(2) 所,处也。

(3) 质,正。

(4) 丑,恶。

【校】案:丑当训耻。

(5) 爱,惜。

(6) 过,明也。

【校】案:过当训失。

(7) 敦,厚也。土色,黄色也。土为四时五行之主,多所戴受,故能辱忍丑也。谓简子之色也。

【校】注“戴受”疑是“载受”,别本“受”作“爱”,今从许本作“受”。

(8) 变,改。

(9) 刻,尽。

(10) 居,处。

(11) 侧,犹在左右也。





行论


六曰:

人主之行与布衣异, (1) 势不便,时不利,事仇以求存, (2) 执民之命。执民之命,重任也,不得以快志为故。 (3) 故布衣行此指于国,不容乡曲。 (4)

(1) 布衣,匹夫。

(2) 仇,周也。

【校】旧校云:“‘存’一作‘全’。”

(3) 故,事也。

(4) 指犹志。布衣之人行此志于国,不能自容于乡曲。

尧以天下让舜。 (1) 鮌为诸侯,怒于尧曰:“得天之道者为帝,得帝之道者为三公。今我得地之道,而不以我为三公。”以尧为失论, (2) 欲得三公,怒甚猛兽,欲以为乱,比兽之角能以为城, (3) 举其尾能以为旌, (4) 召之不来,仿佯于野以患帝。舜于是殛之于羽山,副之以吴刀。 (5) 禹不敢怨而反事之,官为司空, (6) 以通水潦,颜色黎黑,步不相过,窍气不通,以中帝心。 (7)

(1) 让犹予也。

(2) 论犹理也。

(3) 以为城池之固。

(4) 以为旌旗之表也。

(5) 羽山,东极之山也。《书》云“鮌乃殛死”,先殛后死也。

【校】副,当读如“为天子削瓜者副之”之副。梁仲子云:“《海内经》郭注引《启筮》‘副’作‘剖’。”

(6) 禹,鮌子也。不敢怨舜而还事舜,治水土者也。

【校】案:注“者”字衍。

(7) 中犹得。

昔者纣为无道,杀梅伯而醢之,杀鬼侯而脯之,以礼诸侯于庙。 (1) 文王流涕而咨之。 (2) 纣恐其畔,欲杀文王而灭周。文王曰:“父虽无道,子敢不事父乎?君虽不惠,臣敢不事君乎?孰王而可畔也?”纣乃赦之。天下闻之,以文王为畏上而哀下也。《诗》曰:“惟此文王,小心翼翼。昭事上帝,聿怀多福。” (3)

(1) 肉酱为醢。肉熟为脯。梅伯、鬼侯皆纣之诸侯也。梅伯说鬼侯之女美,令纣取之,纣听妲己之谮曰以为不好,故醢梅伯、脯鬼侯,以其脯燕诸侯于庙中。

【校】注“曰”字疑是“因”。

(2) 咨,嗟叹辞。

(3) 《诗·大雅·大明》之三章。言文王小心翼翼然敬慎,明于事上,不敢携贰,所以得众福也。

齐攻宋,燕王使张魁将燕兵以从焉,齐王杀之。燕王闻之,泣数行而下,召有司而告之曰:“余兴事而齐杀我使,请令举兵以攻齐也。” (1) 使受命矣。凡繇进见,争之曰:“贤王故愿为臣。今王非贤主也,愿辞不为臣。” (2) 昭王曰:“是何也?”对曰:“松下乱,先君以不安弃群臣也。王苦痛之,而事齐者,力不足也。 (3) 今魁死而王攻齐,是视魁而贤于先君。”王曰:“诺。” (4) “请王止兵。” (5) 王曰:“然则若何?”凡繇对曰:“请王缟素辟舍于郊,遣使于齐,客而谢焉,曰:‘此尽寡人之罪也。大王贤主也,岂尽杀诸侯之使者哉?然而燕之使者独死,此弊邑之择人不谨也。愿得变更请罪。’” (6) 使者行至齐, (7) 齐王方大饮,左右官实,御者甚众,因令使者进报。 (8) 使者报言燕王之甚恐惧而请罪也,毕,又复之,以矜左右官实。 (9) 因乃发小使以反令燕王复舍。 (10) 此济上之所以败, (11) 齐国以虚也。七十城,微田单,固几不反。 (12) 湣王以大齐骄而残,田单以即墨城而立功。 (13) 诗曰:“将欲毁之,必重累之;将欲踣之,必高举之。”其此之谓乎! (14) 累矣而不毁,举矣而不踣, (15) 其唯有道者乎! (16)

(1) 【校】“请令”疑当作“请今”。

(2) 辞,去也。

(3) 昭王,燕王子哙之子。先君,谓子哙也。松下,地名也。齐伐燕,子哙与松下战,为齐所获,故曰“弃群臣也”。王苦伤之而奉事齐者,盖力不足以伐齐。

(4) 从凡繇谏也。

(5) 请王出令止兵也。

(6) 更,改更也。

(7) 行,还也。

(8) 使其使者进报燕使之至也。

(9) 说燕王谓伏罪,讫,又复使说之,以自矜大于左右官实。官,长也。使闻知也。

(10) 小使,微者也。反燕王使复舍也。

(11) 此齐所以为燕军所败于济上也。

(12) 虚,弱也。燕昭王使乐毅伐齐,得七十余城,事未讫,使骑劫代之,田单率即墨市民击骑劫军,尽破之,悉反其城,故曰无田单几不反矣。

【校】“不反”,旧作“不及”,注末作“几不及免矣”,两“及”字皆当作“反”,又“免”字衍,今并删正。

(13) 湣王骄暴,淖齿杀之,擢其筋,悬之东庙,故曰“而残”也。田单以即墨市民大破燕军,故曰“而立功”也。

(14) 诗,逸诗也。

(15) 累之重,乃易毁也。踣,破也。举之高,乃易破也。以喻湣王骄乱甚,乃易破也。燕军攻高亦易破,使田单序其名也。

【校】据注,踣当读剖,与举为韵。“序其名”,“序”字必误,疑是“成其名”。

(16) 有道者,能满而不溢,高而不危,故曰“其唯有道者乎”也。

楚庄王使文无畏于齐,过于宋,不先假道。 (1) 还反,华元言于宋昭公曰:“往不假道,来不假道,是以宋为野鄙也。 (2) 楚之会田也,故鞭君之仆于孟诸。 (3) 请诛之。”乃杀文无畏于扬梁之堤。 (4) 庄王方削袂,闻之曰:“嘻!” (5) 投袂而起,履及诸庭, (6) 剑及诸门, (7) 车及之蒲疏之市。 (8) 遂舍于郊, (9) 兴师围宋九月。 (10) 宋人易子而食之,析骨而爨之。宋公肉袒执牺, (11) 委服告病曰 (12) :“大国若宥图之,唯命是听。”庄王曰:“情矣,宋公之言也! (13) 乃为却四十里, (14) 而舍于卢门之阖, (15) 所以为成而归也。 (16) 凡事之本在人主, (17) 人主之患在先事而简人,简人则事穷矣。今人臣死而不当,亲帅士民以讨其故, (18) 可谓不简人矣。宋公服以病告而还师, (19) 可谓不穷矣。夫舍诸侯于汉阳 (20) 而饮至者,其以义进退邪? (21) 强不足以成此也。 (22)

(1) 庄王,楚穆王商臣之子,恭王之父也。无畏申周,楚大夫也,使如齐,不假道于宋也。

【校】申周即申舟,古字通。

(2) 昭公,宋成公王臣之子杵臼。往来不假道,欲以宋为鄙邑。

(3) 言往日与楚会田于孟诸,无畏挞宋公之仆。

(4) 【校】梁仲子云:“案:扬梁,宋地,见《左氏襄十二年传》。又《水经注》‘涣水又东径杨亭北,即春秋杨梁也’,近水,故有堤防。‘杨’、‘扬’古通用。”“堤”,李本作“腹”。

(5) 嘻,怒貌也。

【校】孔太史广森《经学卮言》曰:“削,裁也。投袂,投其所削之袂也。《左氏宣十四年传》文未备,杜氏遂以投为振,壹若拂袖之义,误已。”

(6) 《传》曰“履及于绖皇”也。

(7) 《传》曰“剑及寝门”。

(8) 【校】“蒲疏”,《左传》作“蒲胥”,二字通。

(9) 邑外曰郊。

(10) 围宋在鲁宣公十四年。

(11) 牺,牲也。

(12) 病,困。

(13) 【校】旧校云:“‘情’一作‘殆’。”

(14) 【校】《左传》作“三十里”。

(15) 卢门,宋城门。阖,扉也。

(16) 成,平。

(17) 【校】旧此下有“之患”二字,乃因下文而衍,今删。

(18) 讨,伐也。

(19) 还,反也。

(20) 水北曰阳。

【校】“舍”疑“合”字误。

(21) 叛而讨之,以义进也。服而舍之,以义退也。

(22) 《传》曰“强而不义,其毙必速”,唯义以济,故曰强不足以成也。

【校】注“毙”旧作“弊”,今据昭元年《左氏传》改正。





骄恣


七曰:

亡国之主,必自骄,必自智,必轻物。 (1) 自骄则简士, (2) 自智则专独, (3) 轻物则无备。 (4) 无备召祸,专独位危,简士壅塞。 (5) 欲无壅塞必礼士,欲位无危必得众,欲无召祸必完备。三者,人君之大经也。 (6)

(1) 自谓有过人之智,故曰“轻物”。

(2) 简,傲也。

(3) 不咨忠臣。

(4) 《传》曰“无备而官办者,犹拾潘也”,此之谓也。

【校】旧本无“办者”二字,今从哀三年《左传》文补。又“潘”,《传》作“瀋”。

(5) 士不尽规,故壅塞无闻知。

(6) 经,道也。

晋厉公侈淫,好听谗人,欲尽去其大臣而立其左右。胥童谓厉公曰:“必先杀三郄。 (1) 族大多怨,去大族不逼。” (2) 公曰:“诺。”乃使长鱼矫杀郄犨、郄锜、郄至于朝而陈其尸。于是厉公游于匠丽氏,栾书、中行偃劫而幽之, (3) 诸侯莫之救,百姓莫之哀, (4) 三月而杀之。人主之患,患在知能害人,而不知害人之不当而反自及也。 (5) 是何也?智短也。智短则不知化,不知化者举自危。 (6)

(1) 三郄,锜、犨、至也。

(2) 不逼迫公室。

(3) 栾书,武子也。中行偃,荀偃,荀伯游献子也。幽,囚也。

【校】偃字伯游。

(4) 言厉公之恶。

(5) 不当,谓害贤近不肖。自及,死于匠丽氏。

(6) 危,败。

魏武侯谋事而当,攘臂疾言于庭曰:“大夫之虑莫如寡人矣!” (1) 立有间,再三言。 (2) 李悝趋进曰 (3) :“昔者楚庄王谋事而当,有大功,退朝而有忧色。左右曰:‘王有大功,退朝而有忧色,敢问其说?’王曰:‘仲虺有言,不穀说之, (4) 曰:诸侯之德,能自为取师者王,能自取友者存,其所择而莫如己者亡。 (5) 今以不穀之不肖也,群臣之谋又莫吾及也,我其亡乎?’ (6) 曰:“此霸王之所忧也,而君独伐之,其可乎?” (7) 武侯曰:“善。”人主之患也,不在于自少,而在于自多。自多则辞受, (8) 辞受则原竭。 (9) 李悝可谓能谏其君矣,壹称而令武侯益知君人之道。

(1) 武侯,文侯之子也。疾言于庭,伐智自大也。

(2) 言自多也。

(3) 【校】《荀子·尧问》篇、《新序·杂事一》“李悝”皆作“吴起”。

(4) 仲虺,汤左相也。不穀,自谓也。

(5) 择,取也。孔子曰:“无友不如己者,过则勿惮改。”故曰取无如己者亡。

【校】《困学纪闻》二引此,“取友”上亦有“为”字。

(6) 今以不穀之名不肖,群臣之谋又无如吾,无能相匡以济道,故曰“我其亡乎”。

【校】注“名”字似衍。

(7) 霸王唯此之忧,忧不得友而自存也,而独自务伐,言不可。

(8) 辞受,当受言而不受。

(9) 不受谋臣之言而自谋之,则谋虑之言竭尽也。

【校】卢云:“原,水之原也。川仰浦而后大,君受言而后圣,原其可竭乎?”

齐宣王为大室, (1) 大益百亩, (2) 堂上三百户。以齐之大,具之三年而未能成。 (3) 群臣莫敢谏王。 (4) 春居问于宣王曰 (5) :“荆王释先王之礼乐而乐为轻, (6) 敢问荆国为有主乎?”王曰:“为无主。” (7) “贤臣以千数而莫敢谏,敢问荆国为有臣乎?”王曰:“为无臣。” (8) “今王为大室,其大益百亩,堂上三百户。以齐国之大,具之三年而弗能成。群臣莫敢谏,敢问王为有臣乎?”王曰:“为无臣。” (9) 春居曰:“臣请辟矣!”趋而出。 (10) 王曰:“春子,春子,反!何谏寡人之晚也?寡人请今止之。”遽召掌书曰:“书之, (11) 寡人不肖,而好为大室,春子止寡人。”箴谏不可不熟。莫敢谏若,非弗欲也。春居之所以欲之与人同,其所以入之与人异。宣王微春居,几为天下笑矣。 (12) 由是论之,失国之主,多如宣王,然患在乎无春居。故忠臣之谏者,亦从入之,不可不慎。此得失之本也。 (13)

(1) 【校】“大”旧作“太”,今从《新序·刺奢》篇校改。

(2) 【校】“益”,《新序》作“盖”,下同。《御览》一百七十四同。

(3) 宣王,齐威王之子,孟子所见易衅钟之牛者也。成,立也。

(4) 莫,无。

(5) 【校】“春居”,《新序》作“香居”。

(6) 《语》曰“君子不重则不威”,而反自乐,何以为贤也?

【校】注“反自”旧本倒,今乙正。

(7) 为无贤主。

(8) 为无贤臣。

(9) 【校】“臣”字旧本缺,从《新序》补。

(10) 出,去也。

(11) 【校】“掌”,《新序》作“尚”。尚,主也。

(12) 微,无。几,近。

(13) 本,原也。

赵简子沉鸾徼于河, (1) 曰:“吾尝好声色矣,而鸾徼致之;吾尝好宫室台榭矣,而鸾徼为之;吾尝好良马善御矣,而鸾徼来之。 (2) 今吾好士六年矣,而鸾徼未尝进一人也。是长吾过而绌善也。” (3) 故若简子者,能厚以理督责于其臣矣。 (4) 以理督责于其臣,则人主可与为善,而不可与为非;可与为直,而不可与为枉。此三代之盛教。

(1) 【校】《说苑·君道》篇作“栾激”,《水经·河水四注》同。

(2) 【校】《说苑》“来”作“求”。

(3) 所得者皆过,所不进者乃善,故曰“长吾过而绌善也”。

【校】《说苑》作“而黜吾善也”。

(4) 【校】“厚”,旧本作“后”,今从《水经注》四引改正。





观表


八曰:

凡论人心,观事传,不可不熟,不可不深。天为高矣,而日月星辰云气雨露未尝休也; (1) 地为大矣,而水泉草木毛羽裸鳞未尝息也。 (2) 凡居于天地之间、六合之内者,其务为相安利也,夫为相害危者,不可胜数。人事皆然。事随心,心随欲。欲无度者,其心无度。心无度者,则其所为不可知矣。人之心隐匿难见,渊深难测。 (3) 故圣人于事志焉。圣人之所以过人以先知,先知必审徵表。 (4) 无徵表而欲先知,尧、舜与众人同等。 (5) 徵虽易,表虽难,圣人则不可以飘矣, (6) 众人则无道至焉。 (7) 无道至则以为神,以为幸。 (8) 非神非幸,其数不得不然。 (9) 郈成子、吴起近之矣。 (10)

(1) 休,止也。

【校】“休也”,旧本作“休矣”,今从《意林》作“也”。

(2) 毛虫,虎狼之属也。羽虫,凤皇鸿鹄鹤鹜之属也。裸虫,麒麟麋鹿牛羊之属也,蹄角裸见,皆为裸虫。鳞虫,蛇鳞之属。

(3) 测犹知也。

(4) 徵,应。表,异。一曰奇表。

(5) 圣人以徵表为异也。

(6) 飘,疾也。必翔而后集,故不可以疾也。

(7) 徵无表以道以至先也。

(8) 无表之道,能过绝于人以先知者,则以为有神有幸。

(9) 言非有神非有幸者必须表,故曰“其数不得不然”。

(10) 【校】旧校云:“‘近’一作‘有’。”

郈成子为鲁聘于晋,过卫, (1) 右宰穀臣止而觞之,陈乐而不乐,酒酣而送之以璧。 (2) 顾反,过而弗辞。 (3) 其仆曰:“曏者右宰穀臣之觞吾子,吾子也甚欢。 (4) 今侯渫过而弗辞?” (5) 郈成子曰:“夫止而觞我,与我欢也。陈乐而不乐,告我忧也。酒酣而送我以璧, (6) 寄之我也。若由是观之,卫其有乱乎!”倍卫三十里, (7) 闻甯喜之难作,右宰穀臣死之。 (8) 还车而临,三举而归。 (9) 至,使人迎其妻子,隔宅而异之, (10) 分禄而食之。其子长而反其璧。 (11) 孔子闻之,曰:“夫智可以微谋,仁可以托财者, (12) 其郈成子之谓乎!”郈成子之观右宰穀臣也,深矣,妙矣。不观其事而观其志,可谓能观人矣。

(1) 郈成子,鲁大夫也,郈敬子国之子,郈青孙也。适晋,道经卫。

【校】梁仲子云:“《外传·鲁语上》注‘国’作‘同’。”

(2) 右宰穀臣,卫大夫也。以璧送郈成子。

【校】李善注《文选》刘孝标《广绝交论》“穀臣”作“毂臣”。

(3) 反,还也。自晋还,过卫,不辞右宰穀臣。

(4) 曏,曩也。甚,厚也。

(5) 侯,何也。重过为渫过。何为不辞右宰。

(6) 【校】旧本作“送之我以璧”,《孔丛子·陈士义》篇及《广绝交论注》皆无“之”字,今据删。

(7) 【校】《孔丛》、《选注》“倍”皆作“背”。

(8) 甯喜,卫大夫甯惠子殖之子悼子也。惠子与孙林父共逐献公出之。惠子疾,临终,谓悼子曰:“吾得罪于君,名载诸侯之策。君入则掩之。若能掩之,则吾子也。”悼子许诺。鲁襄二十六年,杀卫侯剽而纳献公,故曰“甯喜之难作”也。

(9) 临,哭也。右宰息如是者三,故曰“三举”。

【校】注“右宰息”三字有讹脱,疑当作“右宰一哭一息”。

(10) 【校】《孔丛》“异”作“居”。

(11) 反,还也。

(12) 【校】《孔丛》作“仁可与托孤,廉可与寄财者”。

吴起治西河之外, (1) 王错谮之于魏武侯,武侯使人召之。吴起至于岸门,止车而休,望西河,泣数行而下。其仆谓之曰:“窃观公之志,视舍天下若舍屣。 (2) 今去西河而泣,何也?”吴起雪泣而应之曰 (3) :“子弗识也。君诚知我,而使我毕能, (4) 秦必可亡,而西河可以王。 (5) 今君听谗人之议,而不知我,西河之为秦也不久矣, (6) 魏国从此削矣。” (7) 吴起果去魏入荆,而西河毕入秦,魏日以削,秦日益大。此吴起之所以先见而泣也。

(1) 吴起,卫人,仕于魏文侯,为治西河。

【校】注旧本作“魏侯”,今补“文”字。

(2) 屣,弊履。

【校】前《长见》篇已载此事,两“舍”字皆作“释”。

(3) 雪,拭也。

(4) 毕,尽。

(5) 可以立王政也。

(6) 言西河畔魏入于秦也。

(7) 削,弱也。

古之善相马者,寒风是相口齿, (1) 麻朝相颊,子女厉相目,卫忌相髭,许鄙相 , (2) 投伐褐相胸胁,管青相 肳, (3) 陈悲相股脚,秦牙相前,赞君相后。 (4) 凡此十人者,皆天下之良工也。若赵之王良,秦之伯乐、九方堙,尤尽其妙矣。 (5) 其所以相者不同, (6) 见马之一徵也, (7) 而知节之高卑,足之滑易,材之坚脆,能之长短。非独相马然也,人亦有徵,事与国皆有徵。圣人上知千岁,下知千岁,非意之也,盖有自云也。绿图幡薄,从此生矣。 (8)

(1) 【校】“寒风”,《淮南·齐俗训》作“韩风”,又“是”字朱本作“氏”。案:“寒”、“韩”,“是”、“氏”,古皆通用。

(2) ,后窍也。 字读如穷穹之穹。

【校】“ ”乃“尻”之俗体,《玉篇》“苦刀切”,此音读未详。

(3) 【校】李善注《文选》张景阳《七命》作“唇吻”,《御览》八百九十六同。

(4) 【校】“赞”,《御览》作“贲”。

(5) 【校】以上十七字旧本无,据《七命》注补。孙云:“又见《七发》及《荐祢衡表》、《与吴季重书》注,无‘九方堙’。”

(6) 以,用。

(7) 徵,验也。

(8) 幡亦薄也,锻作铁物,言薄令薄也。

【校】语未详,当出纬书。注亦欠明。“言薄”或是“言幡”。梁仲子云:“《淮南·俶真训》有‘洛出丹书,河出绿图’语。”




————————————————————

[1] 之:原本作“也”,据许维遹本改。





第二十一卷 开春论



开春


一曰:

开春始雷,则蛰虫动矣。 (1) 时雨降,则草木育矣。 (2) 饮食居处适,则九窍百节千脉皆通利矣。 (3) 王者厚其德,积众善,而凤皇圣人皆来至矣。 (4) 共伯和修其行,好贤仁,而海内皆以来为稽矣。 (5) 周厉之难,天子旷绝, (6) 而天下皆来谓矣。 (7) 以此言物之相应也,故曰行也成也。善说者亦然。言尽理而得失利害定矣,岂为一人言哉? (8)

(1) 动,苏也。

(2) 育,长也。

(3) 通利,不壅闭,无疾病矣。

(4) 雄曰凤,雌曰皇,三代来至门庭,周室至于山泽。《诗》云“凤皇鸣矣,于彼高冈”,此之谓也。圣人皆来至,谓尧得夔、龙、稷、契,舜得益,汤得伊尹,武丁得傅说之属是也。

(5) 共,国;伯,爵;夏时诸侯也。以好贤仁而人归之,皆以来附为稽迟也。

【校】案《竹书纪年》,厉王十二年奔彘,十三年共伯和摄行天子事,至二十六年宣王立,共伯和遂归国。诱时《竹书》未出,故说此多讹。

(6) 难,厉王流于彘也。周无天子十一年,故曰“旷绝”也。

(7) 谓天子也。

(8) 善说者大言天下之事,得其分理,爱之不助,憎之不枉,故曰“岂为一人言哉”。

魏惠王死,葬有日矣。 (1) 天大雨雪,至于牛目。群臣多谏于太子者,曰:“雪甚如此而行葬,民必甚疾之, (2) 官费又恐不给, (3) 请弛期更日。” (4) 太子曰:“为人子者,以民劳与官费用之故,而不行先王之葬,不义也。子勿复言。”群臣皆莫敢谏,而以告犀首。 (5) 犀首曰:“吾未有以言之。 (6) 是其唯惠公乎?请告惠公。” (7) 惠公曰:“诺。”驾而见太子,曰:“葬有日矣?”太子曰:“然。”惠公曰:“昔王季历葬于涡山之尾,灓水啮其墓, (8) 见棺之前和。 (9) 文王曰:‘嘻!先君必欲一见群臣百姓也,天故使灓水见之。’ (10) 于是出而为之张朝,百姓皆见之,三日而后更葬。此文王之义也。今葬有日矣,而雪甚,及牛目,难以行。太子为及日之故,得无嫌于欲亟葬乎?愿太子易日。先王必欲少留而抚社稷、安黔首也,故使雨雪甚。 (11) 因弛期而更为日,此文王之义也。若此而不为,意者羞法文王也?”太子曰:“甚善。敬弛期,更择葬日。”惠子不徒行说也,又令魏太子未葬其先君而因有说文王之义。 (12) 说文王之义以示天下,岂小功也哉?

(1) 孟子所见梁惠王也。秦伐魏,魏徙都大梁。梁在陈留浚仪西大梁城是也。

(2) 【校】《战国·魏策》作“甚病之”。

(3) 给,足也。

(4) 更,改也。

(5) 犀首,魏人公孙衍也。佩五国相印,能合从连横,号为“犀首”。

(6) 未犹无也。

(7) 言唯惠公能谏之也。惠公,惠王相惠施也。

(8) 【校】梁仲子云:“《魏策》作‘楚山之尾’,《论衡·死伪》篇作‘滑山之尾’,《初学记》十四引作‘涡水之尾’。”“灓”从“水”,旧本讹从“木”。吴师道《国策》注:“姚宏云:‘灓音鸾,《说文》云漏流也,一曰渍也。’”

(9) 棺题曰和。

【校】“题”旧本作“头”,据李善注《文选》谢惠连 [1] 《祭古冢文》所引改。《说文》云:“题,额也。”

(10) 见犹出也。

【校】“天”,《国策》、《论衡》皆作“夫”。又“灓水”,《初学记》引作“明水”,《国策》注同。

(11) 【校】《国策》无“雨”字。

(12) 【校】“因有”当作“有因”,“有”与“又”同。《国策》作“又因”。

韩氏城新城,期十五日而成。 (1) 段乔为司空,有一县后二日,段乔执其吏而囚之。囚者之子走告封人子高曰:“唯先生能活臣父之死, (2) 愿委之先生。”封人子高曰:“诺。”乃见段乔,自扶而上城。封人子高左右望曰:“美哉城乎!一大功矣,子必有厚赏矣。自古及今,功若此其大也,而能无有罪戮者,未尝有也。”封人子高出, (3) 段乔使人夜解其吏之束缚也而出之。故曰封人子高为之言也,而匿己之为而为也;段乔听而行之也,匿己之行而行也。说之行若此其精也,封人子高可谓善说矣。

(1) 韩氏本都弘农宜阳,其后都颍川阳翟。新城,今河南新城是也。故戎蛮子之国也。

(2) 子高,贤者也。封人,田大夫,职在封疆,故谓之封人。《周礼》亦有封人之官。《传》曰“颍考叔为颍谷封人”也。

(3) 出,去也。

叔向之弟羊舌虎善栾盈。 (1) 栾盈有罪于晋,晋诛羊舌虎,叔向为之奴而朡。 (2) 祈奚曰:“吾闻小人得位,不争不祥; (3) 君子在忧,不救不祥。” (4) 乃往见范宣子而说也, (5) 曰:“闻善为国者,赏不过而刑不慢。赏过则惧及淫人,刑慢则惧及君子。与其不幸而过,宁过而赏淫人,毋过而刑君子。故尧之刑也,殛鮌于虞而用禹; (6) 周之刑也,戮管、蔡而相周公; (7) 不慢刑也。”宣子乃命吏出叔向。救人之患者,行危苦,不避烦辱,犹不能免。今祈奚论先王之德,而叔向得免焉。学岂可以已哉?类多若此。

(1) 栾盈,晋大夫栾书之孙、栾黡之子怀子也。

(2) 奴,戮也。律坐父兄没入为奴。《周礼》曰“其奴,男子入于罪隶”,此之谓也。朡,系也。

【校】案:字书无“朡”字,疑是“朘”,缩肉之意也。

(3) 当谏君退之,故不争不祥也。

(4) 忧,厄也。当谏君免之,故不救不祥也。

(5) 祈奚,高梁伯之子祈黄羊也。为范宣子说叔向也。范宣子,范文子之子丐也。

【校】“丐”乃“匄”之或体。

(6) 殛,诛也。于舜用禹。禹,鮌之子也。

(7) 管叔,周公弟,蔡叔其兄也。二人流言,欲乱周室,而戮之。周公相成王而尹天下也。

【校】注以蔡叔为周公兄,误,说已见《察微》篇。





察贤


二曰:

今有良医于此,治十人而起九人,所以求之万也。 (1) 故贤者之致功名也,比乎良医,而君人者不知疾求,岂不过哉? (2) 今夫塞者, (3) 勇力、时日、卜筮、祷祠无事焉,善者必胜。立功名亦然,要在得贤。 (4) 魏文侯师卜子夏,友田子方,礼段干木, (5) 国治身逸。 (6) 天下之贤主,岂必苦形愁虑哉?执其要而已矣。 (7) 雪霜雨露时,则万物育矣, (8) 人民修矣,疾病妖厉去矣。 (9) 故曰 之容若委衣裘,以言少事也。

(1) 以术之良,故人多求之也。

(2) 人皆知求良医以治病,人君不知求贤臣以治国,故曰“岂不过哉”。

(3) 【校】“塞”,旧本作“寒”,赵云“当作‘塞’”,今从之。“塞”亦作“簺”,先代切,《说文》云“行棊相塞也”。

(4) 要,约也。

(5) 礼,式其闾也。

(6) 逸,不劳也。

(7) 要,谓师贤友明,敬有德而已也。

(8) 育,成也。

(9) 妖,怪。厉,恶。去犹除也。

宓子贱治单父, (1) 弹鸣琴,身不下堂,而单父治。巫马期以星出,以星入,日夜不居,以身亲之,而单父亦治。巫马期问其故于宓子,宓子曰:“我之谓任人,子之谓任力。任力者故劳,任人者故逸。” (2) 宓子则君子矣。逸四肢,全耳目,平心气,而百官以治,义矣,任其数而已矣。 (3) 巫马期则不然,弊生事精, (4) 劳手足,烦教诏,虽治犹未至也。

(1) 子贱,孔子弟子宓不齐也。

【校】孙云:“李善注《文选》潘正叔《赠河阳》诗‘宓’作‘虙’。”今案:“虙”字是。虙羲字作此。

(2) 【校】《说苑·政理》篇两“故”字作“固”,古通用。

(3) 数,术也。

(4) 【校】《说苑》作“弊性事情”。





期贤


三曰:

今夫爚蝉者,务在乎明其火,振其树而已。火不明,虽振其树,何益? (1) 明火不独在乎火,在于暗。 (2) 当今之时,世暗甚矣,人主有能明其德者,天下之士,其归之也,若蝉之走明火也。 (3) 凡国不徒安,名不徒显,必得贤士。 (4)

(1) 虽振树,蝉飞去,不能得之,故曰“何益”也。

(2) 暗冥无所见,火乃光耳,故曰“在于暗”也。

(3) 走,趋也。

【校】孙云:“李善注《文选》干令升《晋纪总论》引作‘赴明火’,《御览》九百五十二亦同。”

(4) 《传》曰:“不有君子,其能国乎?”故曰“必得贤士”。

赵简子昼居,喟然太息曰:“异哉!吾欲伐卫十年矣,而卫不伐。” (1) 侍者曰:“以赵之大而伐卫之细,君若不欲则可也,君若欲之,请令伐之。” (2) 简子曰:“不如而言也。 (3) 卫有士十人于吾所, (4) 吾乃且伐之,十人者其言不义也,而我伐之,是我为不义也。”故简子之时,卫以十人者按赵之兵, (5) 殁简子之身。卫可谓知用人矣,游十士而国家得安。简子可谓好从谏矣,听十士而无侵小夺弱之名。

(1) 不伐,不果伐也。

(2) 【校】“令”疑“今”。

(3) 而,汝。

(4) 于犹在也。

(5) 按,止也。

魏文侯过段干木之闾而轼之。 (1) 其仆曰:“君胡为轼?”曰:“此非段干木之闾欤?段干木盖贤者也,吾安敢不轼?且吾闻段干木未尝肯以己易寡人也, (2) 吾安敢骄之? (3) 段干木光乎德,寡人光乎地; (4) 段干木富乎义,寡人富乎财。”其仆曰:“然则君何不相之?” (5) 于是君请相之,段干木不肯受,则君乃致禄百万,而时往馆之。 (6) 于是国人皆喜,相与诵之曰:“吾君好正,段干木之敬;吾君好忠,段干木之隆。” (7) 居无几何,秦兴兵欲攻魏,司马唐谏秦君曰 (8) :“段干木贤者也,而魏礼之,天下莫不闻,无乃不可加兵乎?” (9) 秦君以为然,乃按兵,辍不敢攻之。 (10) 魏文侯可谓善用兵矣。尝闻君子之用兵,莫见其形,其功已成,其此之谓也。野人之用兵也,鼓声则似雷,号呼则动地,尘气充天,流矢如雨,扶伤舆死, (11) 履肠涉血,无罪之民,其死者量于泽矣, (12) 而国之存亡、主之死生犹不可知也。其离仁义亦远矣!

(1) 闾,里也。《周礼》“二十五家为闾”。轼,伏轼也。礼:国君轼马尾;兵车不轼,尚威武也。

(2) 谓以己之德易寡人之处不肯也。

(3) 骄慢之也。

(4) 【校】孙云:“李善注左太冲《魏都赋》‘地’作‘势’。”

(5) 何不以段干木为辅相也。

(6) 时往诣其馆也。

(7) 隆,高也。

(8) 【校】《古今人表》有司马庾,与魏文侯相接。《淮南》正作“庾”,注云:“秦大夫。或作‘唐’。”

(9) 【校】《选》注“兵乎”二字倒。

(10) 辍,止也。

【校】“敢”字疑衍。

(11) 【校】“死”与“尸”同。

(12) 量犹满也。





审为


四曰:

身者所为也,天下者所以为也,审所以为,而轻重得矣。 (1) 今有人于此断首以易冠,杀身以易衣,世必惑之。 (2) 是何也?冠所以饰首也,衣所以饰身也,杀所饰,要所以饰,则不知所为矣。 (3) 世之走利,有似于此。危身伤生、刈颈断头以徇利,则亦不知所为也。

(1) 身所重,天下所轻也。得犹知也。

(2) 惑,怪也。

(3) 为谓相为之为。

【校】注“谓”疑“读”。

太王亶父居邠,狄人攻之。 (1) 事以皮帛而不受,事以珠玉而不肯, (2) 狄人之所求者,地也。 (3) 太王亶父曰:“与人之兄居而杀其弟,与人之父处而杀其子,吾不忍为也。 (4) 皆勉处矣!为吾臣与狄人臣,奚以异? (5) 且吾闻之,不以所以养害所养。”杖策而去。 (6) 民相连而从之,遂成国于岐山之下。 (7) 太王亶父可谓能尊生矣。 (8) 能尊生,虽贵富不以养伤身,虽贫贱不以利累形。今受其先人之爵禄,则必重失之。生之所自来者久矣,而轻失之。岂不惑哉? (9)

(1) 太王亶父,公祖之子,王季之父,文王之祖,号曰古公。《诗》曰:“古公亶父,来朝走马,率西水浒,至于岐下。”避狄难也。狄人,猃狁,今之匈奴也。

【校】注“公祖”,《史记·本纪》作“公叔祖类”,《索隐》引皇甫谧云:“公祖,一名祖绀诸盩,字叔类,号曰太公也。”旧本脱“诗曰古公”四字,今补。

(2) 【校】《庄子·让王》篇“皮帛”句下有“事之以犬马而不受”一句,此“肯”字亦作“受”。《淮南·道应训》云:“事之以皮帛珠玉而弗受。”则“犬马”句可不增。《诗·大雅·绵》正义云:“《毛传》言‘不得免焉’,《书传略说》云‘每与之不止’,《吕氏春秋》云‘不受’。”据此,则此“肯”字定误。

(3) 【校】《淮南》句上有“曰”字,此亦可不增。

(4) 言忍争土地,与狄人战斗,杀人之子弟也。

(5) 勉,务。处,居也。教邠人务安居,为臣等耳,故曰“奚以异”。

【校】案:《庄子》云“子皆勉居矣”,则此疑亦当有“子”字。

(6) 所以养者,土地也。所养者,谓民人也。策,棰也。

(7) 连,结也。民相与结檐随之众多,复成为国也。岐山在右扶风美阳之北,其下有周地,周家因之以为天下号也。

(8) 尊,重也。

(9) 言今人重失其先人之爵禄,争土地而失其生命,故曰“岂不惑哉”。

韩、魏相与争侵地。子华子见昭釐侯,昭釐侯有忧色。 (1) 子华子曰:“今使天下书铭于君之前,书之曰:‘左手攫之则右手废,右手攫之则左手废,然而攫之必有天下。’君将攫之乎?亡其不与?” (2) 昭釐侯曰:“寡人不攫也。”子华子曰:“甚善。自是观之,两臂重于天下也,身又重于两臂。韩之轻于天下远,今之所争者,其轻于韩又远。 (3) 君固愁身伤生以忧之戚不得也?” (4) 昭釐侯曰:“善。教寡人者众矣,未尝得闻此言也。”子华子可谓知轻重矣。知轻重,故论不过。 (5)

(1) 子华子,体道人也。昭釐,复谥也。韩武子五世之孙哀侯之子也。

【校】昭釐,已说见《任数》篇。此“五世”当作“六世”,“哀侯”当作“懿侯”也。

(2) 【校】音否欤。

(3) 远犹多也。

(4) 戚,近也。

【校】旧本“戚”作“臧”,案臧不当训近,《庄子·让王》篇作“戚”此应不异。

(5) 过,失也。

中山公子牟谓詹子曰:“身在江海之上,心居乎魏阙之下,奈何?” (1) 詹子曰:“重生,重生则轻利。” (2) 中山公子牟曰:“虽知之,犹不能自胜也。” (3) 詹子曰:“不能自胜则纵之,神无恶乎。 (4) 不能自胜而强不纵者,此之谓重伤。重伤之人无寿类矣。” (5)

(1) 子牟,魏公子也,作书四篇。魏伐得中山,公以邑子牟,因曰“中山公子牟”也。詹子,古得道者也。身在江海之上,言志放也。魏阙,心下巨阙也。心下巨阙,言神内守也。一说:魏阙,象魏也。悬教象之法,浃日而收之,魏魏高大,故曰“魏阙”。言身虽在江海之上,心存王室,故在天子门阙之下也。

【校】案:后一说得本意。

(2) 言不以利伤生也。

(3) 言人虽知重生当轻利,犹不能自胜其情欲也。

(4) 言人不能自胜其情欲则放之,放之神无所憎恶,言当宁神以保性也。

【校】“纵之”下当再叠“纵之”二字。《文子·下德》篇、《淮南·道应训》俱叠作“从之从之”,又下“不纵”作“不从”,又“恶乎”,《淮南》作“怨乎”,《文子》作“则神无所害也”。

(5) 言人不能自胜其情欲而不放之,则重伤其神也。神伤则夭殒札瘥,故曰“无寿类”也。重,读复重之重。

【校】案:此“重”不当读平声,当从《庄子释文》音直用反。





爱类


五曰:

仁于他物,不仁于人,不得为仁。不仁于他物,独仁于人,犹若为仁。仁也者,仁乎其类者也。故仁人之于民也,可以便之,无不行也。 (1) 神农之教曰 (2) :“士有当年而不耕者,则天下或受其饥矣; (3) 女有当年而不绩者,则天下或受其寒矣。” (4) 故身亲耕,妻亲织, (5) 所以见致民利也。贤人之不远海内之路,而时往来乎王公之朝,非以要利也, (6) 以民为务故也。 (7) 人主有能以民为务者,则天下归之矣。王也者,非必坚甲利兵选卒练士也,非必隳人之城郭杀人之士民也。上世之王者众矣,而事皆不同,其当世之急、忧民之利、除民之害同。 (8)

(1) 便,利也。行,为也。

(2) 神农,炎帝也。

(3) 当其丁壮之年,故不耕植,则谷不丰,故有受其饥者也。

(4) 《诗》云:“不绩其麻,市也婆娑。”衣服不供,有受其寒者。

【校】旧本作“不绩其麻布也”,误,案当全引《诗》文,今补正。

(5) 身,神农之身也。

(6) 要,徼也。

(7) 以利民为务。

(8) 同,等也。

公输般为高云梯,欲以攻宋。 (1) 墨子闻之,自鲁往,裂裳裹足,日夜不休,十日十夜而至于郢, (2) 见荆王曰:“臣,北方之鄙人也, (3) 闻大王将攻宋,信有之乎?”王曰:“然。”墨子曰:“必得宋乃攻之乎?亡其不得宋且不义犹攻之乎?” (4) 王曰:“必不得宋, (5) 且有不义,则曷为攻之?”墨子曰:“甚善。臣以宋必不可得。” (6) 王曰:“公输般,天下之巧工也,已为攻宋之械矣。” (7) 墨子曰:“请令公输般试攻之,臣请试守之。”于是公输般设攻宋之械,墨子设守宋之备。公输般九攻之, (8) 墨子九却之,不能入。 (9) 故荆辍不攻宋。墨子能以术御荆免宋之难者,此之谓也。

(1) 公输,鲁般之号也。在楚为楚王设攻宋之具也。

(2) 郢,楚都也。

(3) 鄙,小也。

(4) 犹,尚也。

(5) 【校】旧校云:“‘必’一作‘既’。”

(6) 臣以为攻宋必不可得也。

(7) 械,器也。

(8) 【校】旧本此句无“公输般”三字,今据《御览》三百二十所引补。

(9) 入犹下也。

圣王通士不出于利民者无有。 (1) 昔上古龙门未开,吕梁未发, (2) 河出孟门,大溢逆流, (3) 无有丘陵沃衍、平原高阜,尽皆灭之, (4) 名曰鸿水。 (5) 禹于是疏河决江,为彭蠡之障, (6) 干东土,所活者千八百国, (7) 此禹之功也。 (8) 勤劳为民,无苦乎禹者矣。 (9)

(1) 言皆欲利民也。

(2) 龙门,河之阨,在左冯翊夏阳之北。吕梁,在彭城吕县,大石在水中,禹决而通之,号曰吕梁。发,通也。

(3) 昔龙门、吕梁未通,河水稸积,其深乃出于孟门山 [2] 之上。大溢逆流,无有涯畔也。

(4) 灭,没也。

(5) 鸿,大也。

(6) 彭蠡泽在豫章。障,防也。

【校】《黄氏日抄》云:“此于地里不合。”卢云:“此‘为彭蠡之障’不必承上为文,且亦不必连下‘干东土’也。”

(7) 干,燥也。禹致群臣于会稽,执玉帛者万国,此曰千八百者,但谓被水灾之国耳。言使民得居燥土不溺死,故曰活之也。

(8) 功,治水之功也。

(9) 事功曰劳。其治水,凿龙门,辟伊阙,决江疏河,其勤苦无如禹者也。

匡章谓惠子曰:“公之学去尊,今又王齐王,何其到也?” (1) 惠子曰:“今有人于此,欲必击其爱子之头,石可以代之。 (2) 匡章曰 [3] 公取之代乎?其不与? (3) 施取代之。子头所重也,石所轻也,击其所轻以免其所重,岂不可哉?” (4) 匡章曰:“齐王之所以用兵而不休,攻击人而不止者,其故何也?” (5) 惠子曰:“大者可以王,其次可以霸也。今可以王齐王而寿黔首之命,免民之死,是以石代爱子头也,何为不为?” (6) 民寒则欲火,暑则欲冰,燥则欲湿,湿则欲燥。寒暑燥湿相反,其于利民一也。利民岂一道哉?当其时而已矣。 (7)

(1) 去尊,弃尊位也。今王事齐王,居其尊位,谓惠子言行何其到逆相违背也。

【校】古“倒”字皆作“到”。

(2) 爱子,所爱之子也。舍爱子头而击石也,故曰石可以代子也。

(3) 言公取石以代子头乎?其不与邪?

(4) 言其可也。

【校】施,惠子名。此段乃惠子语。

(5) 为何等故也。

(6) 言何为不用兵也。

(7) 冬寒欲温,夏暑欲凉,故曰“当其时而已矣”。





贵卒


六曰:

力贵突,智贵卒。 (1) 得之同则速为上,胜之同则湿为下。 (2) 所为贵骥者,为其一日千里也, (3) 旬日取之,与驽骀同。 (4) 所为贵镞矢者,为其应声而至, (5) 终日而至,则与无至同。 (6)

(1) 【校】音仓卒之卒。

(2) 湿犹迟,久之也。

【校】案:《荀子·修身》篇“卑湿重迟”,作“湿”字为是,音他合切。

(3) 贵其疾也。

(4) 十日为旬。驽骀十日亦至千里,故曰“与驽骀同”也。

(5) 镞矢轻利也。小曰镞矢,大曰篇矢。

(6) 射三百步,终一日乃至,是为与无所至同也。

【校】旧校云“‘无至’一作‘无矢’。”

吴起谓荆王曰:“荆所有余者,地也;所不足者,民也。今君王以所不足益所有余,臣不得而为也。” (1) 于是令贵人往实广虚之地,皆甚苦之。 (2) 荆王死,贵人皆来。尸在堂上,贵人相与射吴起。吴起号呼曰:“吾示子吾用兵也。”拔矢而走,伏尸插矢而疾言曰:“群臣乱王!”吴起死矣。 (3) 且荆国之法,丽兵于王尸者尽加重罪,逮三族。吴起之智可谓捷矣。 (4)

(1) 臣无所得为君计耳。

(2) 贵人,贵臣也。皆不欲往实广虚之地,苦病之也。

(3) 吴起拔人所射之矢以插王尸,因言曰群臣谓王为乱而射王尸,欲令群臣被诛,以自为报也。

(4) 捷,疾也。言发谋以报其仇之速疾也。

齐襄公即位,憎公孙无知,收其禄。 (1) 无知不说,杀襄公。公子纠走鲁,公子小白奔莒。既而国杀无知,未有君。 (2) 公子纠与公子小白皆归,俱至,争先入公家。 (3) 管仲扞弓射公子小白,中钩。 (4) 鲍叔御公子小白僵。 (5) 管子以为小白死,告公子纠曰:“安之,公子小白已死矣。”鲍叔因疾驱先入,故公子小白得以为君。鲍叔之智应射而令公子小白僵也,其智若镞矢也。 (6)

(1) 齐襄公,庄公购之孙,僖公禄父之子诸儿也。公孙无知,僖公之弟夷仲年之子,故曰孙,于襄公为从弟。

(2) 公孙无知自立为君,故国人杀之,未有其君也。

(3) 公家,公之朝也。

(4) 钩,带钩也。

(5) 御犹使也。僵犹偃也。

(6) 镞矢,言其捷疾也。

周武君使人刺伶悝于东周。伶悝僵, (1) 令其子速哭曰:“以谁刺我父也?”刺者闻,以为死也。 (2) 周以为不信,因厚罪之。 (3)

(1) 周武君,西周之君。伶悝,东周之臣也。僵,毙也。

【校】案:此“僵”与上小白佯死之“僵”一也,上训偃,此不当又训毙,似当删去。

(2) 刺者闻伶悝已死,因报西周武君曰伶悝已死矣。

(3) 罪所使刺伶悝者也。

赵氏攻中山。中山之人多力者曰吾丘 , (1) 衣铁甲、操铁杖以战,而所击无不碎,所冲无不陷,以车投车,以人投人也,几至将所而后死。 (2)

(1) 【校】吾丘即虞丘,《汉书》“吾丘寿王”,《说苑》作“虞丘”。“ ”当即“ ”之或体,《集韵》音戎用切,从“穴”得声,未必然也。孙云“《御览》三百十三、又三百五十六并作‘鸠’。”

(2) 将,赵氏之将也。近至其将所然后死,言吾丘 力有余也。




————————————————————

[1] 谢惠连:原本作“谢灵运”,误,据《文选》改。

[2] 原为“清门山”,误,据乾隆本改正。

[3] “匡章曰”三字疑衍。





第二十二卷 慎行论



慎行


一曰:

行不可不孰。不孰,如赴深溪,虽悔无及。 (1) 君子计行虑义, (2) 小人计行其利乃不利。 (3) 有知不利之利者,则可与言理矣。 (4)

(1) 孰犹思也。有水曰涧,无水曰溪。不可不思行仁如入深溪,不可使满而平也。虽悔行不纯淑,陷入刑辟,无所复及也。

(2) 虑,度也。度义而后行之也。

(3) 《传》曰“蕴利生孽”,故曰“乃不利”也。

(4) 理,道也。

荆平王有臣曰费无忌, (1) 害太子建,欲去之。 (2) 王为建取妻于秦而美, (3) 无忌劝王夺。 (4) 王已夺之,而疏太子。 (5) 无忌说王曰:“晋之霸也,近于诸夏,而荆僻也, (6) 故不能与争。 (7) 不若大城城父而置太子焉,以求北方, (8) 王收南方,是得天下也。” (9) 王说,使太子居于城父。居一年,乃恶之曰:“建与连尹将以方城外反。” (10) 王曰:“已为我子矣,又尚奚求?” (11) 对曰:“以妻事怨,且自以为犹宋也, (12) 齐、晋又辅之, (13) 将以害荆,其事已集矣。” (14) 王信之,使执连尹, (15) 太子建出奔。 (16) 左尹郄宛,国人说之,无忌又欲杀之,谓令尹子常曰:“郄宛欲饮令尹酒。” (17) 又谓郄宛曰:“令尹欲饮酒于子之家。”郄宛曰:“我贱人也,不足以辱令尹。令尹必来辱, (18) 我且何以给待之?”无忌曰:“令尹好甲兵, (19) 子出而寘之门, (20) 令尹至,必观之已,因以为酬。” (21) 及飨日,惟门左右而寘甲兵焉。 (22) 无忌因谓令尹曰:“吾几祸令尹。郄宛将杀令尹,甲在门矣。”令尹使人视之,信, (23) 遂攻郄宛,杀之。国人大怨,动作者莫不非令尹。 (24) 沈尹戍谓令尹曰:“夫无忌,荆之谗人也, (25) 亡夫太子建, (26) 杀连尹奢,屏王之耳目。 (27) 今令尹又用之,杀众不辜,以兴大谤,患几及令尹。” (28) 令尹子常曰:“是吾罪也,敢不良图?”乃杀费无忌,尽灭其族,以说其国。动而不论其义,知害人而不知人害己也,以灭其族,费无忌之谓乎! (29)

(1) 【校】宋邦乂本从《左传》作“极”,各本俱作“忌”,与《史记》、《吴越春秋》同。

(2) 平王,楚恭王之子弃疾也。

(3) 美,好也。

(4) 夺,取也。

(5) 疏,远也。

(6) 僻,远也。

(7) 争,霸也。

(8) 城父,楚北境之邑,今属沛国。北方,宋、郑、鲁、卫也。

(9) 南方,谓吴、越也。

(10) 连尹,伍奢,子胥之父也。方城,楚之阨塞也。反,叛也。

(11) 子,太子也。

(12) 犹,如也。

【校】《左传》作“犹宋、郑也”。

(13) 辅,助也。

(14) 集,合也。

(15) 执,囚也。

(16) 出奔郑也。

(17) 子常,名囊瓦,令尹子囊之孙。郄尹,光唐之子也。宛,字也。

【校】注“光唐”无考,高或据《世本》为说。宛字子恶,注“也”字讹。

(18) 辱,屈辱也。

(19) 甲,铠也。兵,戟也。

(20) 寘,置也。

(21) 酬,报也。《诗》云“献酬交错”,此之谓也。

【校】案:古者燕饮,于酬之时皆有物,以致劝侑之意,故曰“因以为酬”。注“报也”,旧讹作“执也”,今据《诗·彤弓》传改正。

(22) 【校】《左氏昭廿七年传》作“帷诸门左”。梁仲子云:“‘惟’、‘帷’形声俱相近,古多通借。《左氏定六年释文》‘小惟子,本又作帷’,《庄子·渔父》释文‘缁帷,本或作惟’。”

(23) 信有甲也。

(24) 非,咎也。

【校】“动作者”,《左传》作“进胙者”。

(25) 沈尹戍,庄王之孙,沈诸梁叶公子高之父也。

【校】“戍”,《左传》作“戌”,庄玉之曾孙也。

(26) 【校】“夫”衍字。案昭廿七年《左氏传》作“丧太子建”。

(27) 屏,蔽也。

(28) 几,近也。

(29) 以谗邪害人,人以公正害之,故族灭也。

崔杼与庆封谋杀齐庄公,庄公死,更立景公,崔杼相之。 (1) 庆封又欲杀崔杼而代之相,于是 崔杼之子,令之争后。崔杼之子相与私 。 (2) 崔杼往见庆封而告之,庆封谓崔杼曰:“且留,吾将兴甲以杀之。”因令卢满嫳兴甲以诛之, (3) 尽杀崔杼之妻、子及枝属,烧其室屋,报崔杼曰:“吾已诛之矣。”崔杼归,无归,因而自绞也。 (4) 庆封相景公,景公苦之。庆封出猎,景公与陈无宇、公孙灶、公孙虿诛封。 (5) 庆封以其属斗,不胜,走如鲁。齐人以为让, (6) 又去鲁而如吴,王予之朱方。 (7) 荆灵王闻之,率诸侯以攻吴,围朱方,拔之, (8) 得庆封,负之斧质,以徇于诸侯军,因令其呼之曰:“毋或如齐庆封,弑其君而弱其孤,以亡其大夫。”乃杀之。 (9) 黄帝之贵而死, (10) 尧、舜之贤而死,孟贲之勇而死,人固皆死,若庆封者,可谓重死矣, (11) 身为僇,支属不可以见,行忮之故也。 (12)

(1) 庄公名光,灵公之子也。景公名杵臼,庄公之弟也。

(2) ,斗也。 ,读近鸿,缓气言之。

【校】“ ”与“椓”同。《左氏哀十七年传》“太子又使椓之”,旧训诉,于此不切,义当与嗾同,今人言挑拨,意颇近之。“ ”旧本“门”内作“卷”,字书无此字。《广韵》一送“鬨”字下云“兵鬭也,又下降切,俗作闀”。《集韵》、《类篇》皆同。《韵会》“鬨”依《说文》从“鬥”,谓《广韵》“今与门户字同”之说为非,今“ ”字亦从之。

(3) 【校】“卢满嫳”,《左传》作“卢蒲嫳”。“蒲”、“满”二字形近,古书多互出。“嫳”旧本作“婺”,讹,今改正。

(4) 绞,经也。

(5) 无宇,陈须无之子桓子也。公孙灶,惠公之孙,公子栾坚之子子雅也。趸,惠公之孙,公子高祈之子子尾也。与共诛庆封也。

【校】坚,子栾名。祈,子高名。旧本“子雅”作“子射”,讹,今改正。

(6) 责让鲁,为其受庆封。

(7) 朱方,吴邑,以封庆封也。

【校】“吴”字当重。

(8) 灵王,恭王庶子围也。覆取之曰拔。

(9) 亡其大夫,谓崔杼强而死。

【校】“以亡”,《左氏昭四年传》作“以盟”。

(10) 黄帝得道仙而可贵,然终归于死。

(11) 【校】死而又死,谓之重死。

(12) 忮,恶也。

凡乱人之动也,其始相助,后必相恶。为义者则不然,始而相与,久而相信,卒而相亲,后世以为法程。 (1)

(1) 程,度也。





无义


二曰:

先王之于论也极之矣, (1) 故义者,百事之始也, (2) 万利之本也。 (3) 中智之所不及也, (4) 不及则不知,不知趋利, (5) 趋利固不可必也,公孙鞅、郑平、续经、公孙竭是已。 (6) 以义动则无旷事矣。 (7) 人臣与人臣谋为奸,犹或与之,又况乎人主与其臣谋为义,其孰不与者?非独其臣也,天下皆且与之。

(1) 极,尽也。

(2) 始,首也。

(3) 本,原也。《传》曰:“利,义之和也。”故曰“利之本也”。

(4) 不能及知之也。

(5) 【校】似当作“不知则趋利”,脱一“则”字。

(6) 公孙鞅,商鞅也。郑平,秦臣也。续经,赵人也。公孙竭,亦秦之臣也。并下自解。

(7) 旷,废也。

公孙鞅之于秦,非父兄也,非有故也,以能用也,欲堙之责,非攻无以, (1) 于是为秦将而攻魏。魏使公子卬将而当之。 (2) 公孙鞅之居魏也,固善公子卬,使人谓公子卬曰:“凡所为游而欲贵者,以公子之故也。今秦令鞅将,魏令公子当之,岂且忍相与战哉?公子言之公子之主,鞅请亦言之主,而皆罢军。”于是将归矣,使人谓公子曰:“归未有时相见, (3) 愿与公子坐而相去别也。”公子曰:“诺。”魏吏争之曰:“不可。”公子不听,遂相与坐。公孙鞅因伏卒与车骑以取公子卬。秦孝公薨,惠王立,以此疑公孙鞅之行,欲加罪焉。公孙鞅以其私属与母归魏,襄庛不受,曰:“以君之反公子卬也,吾无道知君。”故士自行不可不审也。 (4)

(1) 堙,塞也。鞅欲报塞相秦之责,非攻伐无以塞责。

(2) 当,应也。

(3) 言归相见无有时也。

(4) 惠王杀鞅,车裂之,何得以其私族与母归魏而不见受乎?公子卬家何以不取而杀之?鞅执公子卬,有罪于魏,推此言之,复归魏妄矣。《战国策》曰:“鞅欲归魏。秦人曰:‘商君之法急,不得出也。’惠王得而车裂之。”襄庛,魏人也。

【校】襄庛即穰庛,《竹书纪年》梁惠成王二十八年“穰庛帅师及郑孔夜战于梁赫”,本或作“疵”者讹。

郑平于秦王,臣也;其于应侯,交也。欺交反主,为利故也。方其为秦将也,天下所贵之无不以者,重也。重以得之,轻必失之。去秦将,入赵、魏,天下所贱之无不以也,所可羞无不以也。行方可贱可羞,而无秦将之重,不穷奚待? (1)

(1) 待,恃也。

赵急求李欬,李言、续经与之俱如卫,抵公孙与,公孙与见而与入, (1) 续经因告卫吏使捕之, (2) 续经以仕赵五大夫。 (3) 人莫与同朝, (4) 子孙不可以交友。 (5)

(1) 抵,主也。入犹纳也。

【校】案:《史记·张耳传》“去抵父客”,《索隐》云:“抵,归也。”此训最惬。《广雅》则云“至也”。

(2) 捕李欬也。

(3) 五大夫,爵也。

(4) 贱续经之行也。

(5) 人不交友之也。

公孙竭与阴君之事,而反告之樗里相国, (1) 以仕秦五大夫。功非不大也,然而不得入三都, (2) 又况乎无此其功而有行乎? (3)

(1) 樗里疾也。

(2) 三都,赵、卫、魏也。

(3) 无有交友受寄托之功,而有其相输告之行也。

【校】正文“其”字疑当在“有”字下。





疑似


三曰:

使人大迷惑者,必物之相似也。玉人之所患,患石之似玉者;相剑者之所患,患剑之似吴干者; (1) 贤主之所患,患人之博闻辩言而似通者。 (2) 亡国之主似智,亡国之臣似忠。相似之物,此愚者之所大惑,而圣人之所加虑也。 (3) 故墨子见歧道而哭之。 (4)

(1) 吴干,吴之干将者也。

(2) 通,达也。

(3) 虑则知之也。

(4) 为其可以南可以北。言乖别也。

周宅酆镐近戎人,与诸侯约,为高葆祷于王路, (1) 置鼓其上,远近相闻,即戎寇至,传鼓相告,诸侯之兵皆至救天子。戎寇当至, (2) 幽王击鼓,诸侯之兵皆至,褒姒大说,喜之。 (3) 幽王欲褒姒之笑也,因数击鼓,诸侯之兵数至而无寇。至于后戎寇真至,幽王击鼓,诸侯兵不至,幽王之身乃死于丽山之下,为天下笑。 (4) 此夫以无寇失真寇者也。贤者有小恶以致大恶。 (5) 褒姒之败,乃令幽王好小说以致大灭。 (6) 故形骸相离,三公九卿出走,此褒姒之所用死,而平王所以东徙也, (7) 秦襄、晋文之所以劳王劳而赐地也。 (8)

(1) 【校】《御览》三百三十八“葆”作“堡”,无下四宇。

(2) 【校】“当至”别本作“尝至”,今从元本。《御览》三百九十一作“戎尝寇周”。

(3) 【校】《御览》作“大说而笑”。

(4) 【校】旧本无“幽王击鼓,诸侯兵不至”九字,“之身”倒作“身之”,今并从《御览》补正。

(5) 恶积足以灭身,故曰“以致大恶”。

(6) 《诗》云“赫赫宗周,褒姒灭之”也。

(7) 平王,幽王之太子宜臼也。东徙于洛邑,今河南县也。

(8) 秦襄公,秦仲之孙,庄公之子也。幽王为犬戎所败:平王东徙,襄公将兵救周有功,受周故地酆镐,列为诸侯。晋文侯仇,穆侯之子也。《传》曰“平王东迁,晋、郑焉依”,此之谓也。

【校】“焉依”旧误倒,今从《左氏隐六年传》乙正。

梁北有黎丘部,有奇鬼焉, (1) 喜效人之子侄昆弟之状。 (2) 邑丈人有之市而醉归者,黎丘之鬼效其子之状,扶而道苦之。丈人归,酒醒而诮其子 (3) 曰:“吾为汝父也,岂谓不慈哉? (4) 我醉,汝道苦我,何故?”其子泣而触地曰:“孽矣!无此事也。昔也往责于东邑,人可问也。”其父信之,曰:“嘻!是必夫奇鬼也。我固尝闻之矣。”明日端复饮于市,欲遇而刺杀之。明旦之市而醉,其真子恐其父之不能反也, (5) 遂逝迎之。 (6) 丈人望其真子, (7) 拔剑而刺之。丈人智惑于似其子者,而杀其真子。夫惑于似士者而失于真士, (8) 此黎丘丈人之智也。

(1) 【校】孙云:“章怀注《后汉书·张衡传》‘部’引作‘乡’。”

(2) 【校】孙云“李善注《文选》张平子《思玄赋》‘喜’引作‘善’。”案:子侄之称,始见于此。

(3) 诮,让。

(4) 【校】《御览》八百八十三“谓”作“为”。

(5) 反,还也。

(6) 逝,往也。

(7) 【校】《选注》作“丈人望见之”。

(8) 【校】“其真子”,旧本作“于真子”,今从《选》注改正。

疑似之迹,不可不察,察之必于其人也。舜为御,尧为左,禹为右,入于泽而问牧童,入于水而问渔师,奚故也?其知之审也。夫孪子之相似者,其母常识之,知之审也。





壹行


四曰:

先王所恶,无恶于不可知。不可知,则君臣、父子、兄弟、朋友、夫妻之际败矣。十际皆败,乱莫大焉。凡人伦,以十际为安者也,释十际则与麋鹿虎狼无以异,多勇者则为制耳矣。不可知,则知无安君、无乐亲矣,无荣兄、无亲友、无尊夫矣。强大未必王也,而王必强大。王者之所藉以成也何?藉其威与其利。非强大则其威不威,其利不利。其威不威则不足以禁也, (1) 其利不利则不足以劝也, (2) 故贤主必使其威利无敌,故以禁则必止,以劝则必为。 (3) 威利敌而忧苦民、行可知者王,威利无敌而以行不知者亡。 (4) 小弱而不可知,则强大疑之矣。 (5) 人之情不能爱其所疑,小弱而大不爱,则无以存。 (6) 故不可知之道,王者行之废, (7) 强大行之危, (8) 小弱行之灭。 (9)

(1) 禁,止也。

(2) 劝,进也。

(3) 为,治也。

(4) 无仁义之行见知,故亡也。

(5) 小而不小,弱而不弱,故强国大国疑之也。

(6) 小国弱国而为强大者,不为大国所爱,则无以自存。

(7) 废,坏也。

(8) 危,倾陨也。

(9) 灭,破亡也。

今行者见大树,必解衣县冠倚剑而寝其下,大树非人之情亲知交也,而安之若此者信也。 (1) 陵上巨木,人以为期,易知故也。 (2) 又况于士乎?士义可知故也,则期为必矣。 (3) 又况强大之国?强大之国诚可知,则其王不难矣。 (4) 人之所乘船者,为其能浮而不能沉也。世之所以贤君子者,为其能行义而不能行邪辟也。

(1) 大树不欺诈人,故信之。

(2) 巨木,人所同见也。期会其下,荫休之也。故曰“易知故也”。

(3) 聚人复期会于其所而咨诹之。

(4) 《孟子》曰“以齐王犹反手也”,故曰“不难矣”。

孔子卜,得贲。孔子曰:“不吉。” (1) 子贡曰:“夫贲亦好矣,何谓不吉乎?”孔子曰:“夫白而白,黑而黑,夫贲又何好乎?”故贤者所恶于物,无恶于无处。 (2)

(1) 贲,色不纯也。《诗》云:“鹑之贲贲。”

【校】案:《诗》作“奔奔”。“贲”与“奔”古通用。《左传》僖五年、襄廿七年、《礼记·表记》皆作“贲贲”。

(2) 恶物之无目,恶其无处可名之也。

夫天下之所以恶,莫恶于不可知也。夫不可知,盗不与期,贼不与谋。盗贼大奸也,而犹所得匹偶, (1) 又况于欲成大功乎?夫欲成大功,令天下皆轻劝而助之, (2) 必之士可知。

(1) 【校】“所得”二字疑倒。

(2) 劝,进也。





求人


五曰:

身定、国安、天下治,必贤人。 (1) 古之有天下也者,七十一圣。观于《春秋》,自鲁隐公以至哀公十有二世,其所以得之,所以失之,其术一也。得贤人,国无不安,名无不荣;失贤人,国无不危,名无不辱。先王之索贤人,无不以也, (2) 极卑极贱,极远极劳。虞用宫之奇、吴用伍子胥之言,此二国者,虽至于今存可也,则是国可寿也。有能益人之寿者,则人莫不愿之。今寿国有道,而君人者而不求,过矣。

(1) 身者国之本也。詹子曰:“未闻身乱而国治者也。”故曰身定国安而治,须贤人也。

(2) 以,用也。

尧传天下于舜,礼之诸侯,妻以二女,臣以十子,身请北面朝之,至卑也。 (1) 伊尹,庖厨之臣也,傅说,殷之胥靡也, (2) 皆上相天子,至贱也。禹东至榑木之地,日出、九津、青羌之野, (3) 攒树之所, 天之山, (4) 鸟谷、青丘之乡,黑齿之国, (5) 南至交阯、孙朴、续樠之国,丹粟、漆树、沸水、漂漂、九阳之山, (6) 羽人、裸民之处,不死之乡, (7) 西至三危之国,巫山之下,饮露、吸气之民,积金之山, (8) 其肱、一臂、三面之乡,北至人正之国,夏海之穷,衡山之上, (9) 犬戎之国,夸父之野,禺强之所,积水、积石之山,不有懈堕, (10) 忧其黔首,颜色黎黑,窍藏不通, (11) 步不相过, (12) 以求贤人,欲尽地利,至劳也。 (13) 得陶、化益、真窥、横革、之交五人佐禹, (14) 故功绩铭乎金石, (15) 著于盘盂。 (16)

(1) 舜,布衣也,故曰“至卑”。

(2) 胥靡,刑罪之名也。

(3) 榑木,大木也。津,崖也。《淮南子》曰“日出阳谷”。青羌,东方之野也。

【校】榑木即扶木,《为欲》篇“东至扶木”。

(4) 山高至天也。

【校】 音民,抚也。疑亦与“扪”同音义。

(5) 东方其人齿黑,因曰“黑齿之国”也。

(6) 南方积阳,阳数极于九,故曰“九阳之山”也。

(7) 羽人,鸟喙,背上有羽翼。裸民,不衣衣裳也。乡亦国也。

(8) 饮露、吸气,养形人也。西方刚气所在,故曰“积金之山”也。

(9) 今正,北极之国也。夏海,大冥也。北方纯阴,故曰大冥之中处。衡山者,北极之山也。

【校】其肱,疑即《海外西经》之奇肱,所谓一臂三目者是也。注首“今正”与正文“人正”不知孰是。又“之中处”疑是“之穷处”,或三字是衍文。

(10) 犬戎,西戎之别也。夸父,兽名也。禺强,天神也。之所,处也。积水,谓海也。积石,山名也。经营行之,不懈堕休息也。

【校】郭璞注《海外北经》云:“夸父者,盖神人之名也。”经云:“北方禺强,人面鸟身,珥两青蛇,践两赤蛇。”

(11) 病也。

(12) 罢也。

(13) 地利,嘉谷也。至,大也。事功曰劳。

(14) 【校】王厚斋云:“《荀子·成相》曰‘禹得益、皋陶、横革、直成为辅’,此陶即皋陶也,化益即伯益也,真窥即直成也,‘真’与‘直’字相类,横革名同,唯之交未详。”卢云:“案‘窥’或本是‘竀’字,与‘成’音近。”

(15) 金,钟鼎也。石,丰碑也。

(16) 盘盂之器,皆铭其功。

昔者尧朝许由于沛泽之中,曰:“十日出而焦火不息,不亦劳乎? (1) 夫子为天子而天下已定矣, (2) 请属天下于夫子。”许由辞曰:“为天下之不治与?而既已治矣。自为与?啁噍巢于林,不过一枝; (3) 偃鼠饮于河,不过满腹。归已君乎! (4) 恶用天下?” (5) 遂之箕山之下,颍水之阳,耕而食, (6) 终身无经天下之色。 (7) 故贤主之于贤者也,物莫之妨, (8) 戚爱习故, (9) 不以害之,故贤者聚焉。贤者所聚,天地不坏,鬼神不害,人事不谋, (10) 此五常之本事也。

(1) 【校】梁仲子云:“《庄子·逍遥游》‘焦火’作‘爝火’,《释文》云‘本亦作燋,音爵’。此‘焦’下已从‘火’,则不必更加火旁。”

(2) 夫子,谓许由也。

(3) 自为,为己也。与,即也。啁噍,小鸟也。巢,蔟也。偃,息也。啁音超。

【校】“啁噍”,《庄子》作“鹪鹩”。注“与即也”,疑误,两“与”字皆语辞。又“偃息也”衍。“啁音超”亦非高注。

(4) 满腹,不求余也。归,终也。

(5) 恶,安也。

(6) 箕山在颍川阳城之西。水北曰阳也。

(7) 经,横理也。

(8) 不以物故,妨害贤者。

(9) 戚,亲也。

(10) 人不以奸邪谋之也。

皋子众疑取国,召南宫虔、孔伯产而众口止。 (1)

(1) 皋子,贤者也。其取国,告虔、产,口乃止。虔、产,其徒之贤者也。其事不与许由相连也。皋子众疑许由欲取国也。

【校】此注上下异说。“其取国”上当有“众疑”二字。末云“皋子众疑许由欲取国也”,或当有“一云”二字。以众为皋子之名,然于“众口止”仍难强通。

晋人欲攻郑,令叔向聘焉,视其有人与无人。 (1) 子产为之诗曰:“子惠思我,褰裳涉洧;子不我思,岂无他士?”叔向归曰:“郑有人,子产在焉,不可攻也。秦、荆近,其诗有异心,不可攻也。” (2) 晋人乃辍攻郑。 (3) 孔子曰:“《诗》云‘无竞惟人’,子产一称而郑国免。” (4)

(1) 视其有无贤人也。

(2) 郑近秦与荆也。其诗云“子不我思,岂无他人”,将事秦、荆,故曰“有异心,不可攻也”。

(3) 辍,止也。

(4) 《诗·大雅·抑》之二章也,“无竞惟人,四方其训之”。无竞,竞也。国之强,惟在得人,故曰郑国免其难也。





察传


六曰:

夫得言不可以不察,数传而白为黑,黑为白。故狗似玃,玃似母猴,母猴似人,人之与狗则远矣。 (1) 此愚者之所以大过也。

(1) 玃,猳玃,兽名也。

闻而审则为福矣,闻而不审不若无闻矣。齐桓公闻管子于鲍叔,楚庄闻孙叔敖于沈尹筮,审之也,故国霸诸侯也。 (1) 吴王闻越王句践于太宰嚭,智伯闻赵襄子于张武,不审也,故国亡身死也。 (2)

(1) 鲍叔牙说管仲于桓公,沈尹筮说叔敖于庄王,察其贤明审也。

(2) 太宰嚭,吴王夫差臣也。张武,智伯臣也。不审句践、襄子之智能,故越攻吴,吴王夫差死于干遂,智伯围赵襄子于晋阳,襄子与韩、魏通谋,杀智伯于高梁之东,故曰“国亡身死也”。

凡闻言必熟论,其于人必验之以理。 (1) 鲁哀公问于孔子曰:“乐正夔一足,信乎?”孔子曰:“昔者舜欲以乐传教于天下,乃令重黎举夔于草莽之中而进之,舜以为乐正。 (2) 夔于是正六律,和五声,以通八风,而天下大服。 (3) 重黎又欲益求人, (4) 舜曰:‘夫乐,天地之精也,得失之节也,故唯圣人为能和,乐之本也。夔能和之,以平天下, (5) 若夔者一而足矣。’故曰‘夔一足’,非‘一足’也。”宋之丁氏,家无井而出溉汲,常一人居外。及其家穿井,告人曰:“吾穿井得一人。”有闻而传之者曰:“丁氏穿井得一人。”国人道之,闻之于宋君。宋君令人问之于丁氏,丁氏对曰:“得一人之使,非得一人于井中也。”求能之若此, (6) 不若无闻也。 (7) 子夏之晋,过卫, (8) 有读史记者曰:“晋师三豕涉河。” (9) 子夏曰:“非也,是己亥也。夫‘己’与‘三’相近,‘豕’与‘亥’相似。” (10) 至于晋而问之,则曰“晋师己亥涉河”也。

(1) 验,效也。理,道理也。

(2) 乐官之正也。

(3) 六律,六气之律。阳为律,阴为吕,合十二也。五声,五行之声,宫商角徵羽也。八风,八卦之风也。通和阴阳,故天下大服也。

【校】“和五声”,《风俗通·正失》篇引作“和均五声”,李善注《文选》马季长《长笛赋》亦有“均”字。

(4) 益求如夔者也。

(5) 和,调也。

(6) 【校】孙疑是“求闻若此”。

(7) 无闻则不妄言也。

(8) 子夏,孔子弟子卜商也。

(9) 【校】《意林》作“渡河”。

(10) 【校】案“己”古文作“ ”,“亥”古文作“ ”。

辞多类非而是,多类是而非。是非之经,不可不分。 (1) 此圣人之所慎也。然则何以慎?缘物之情及人之情以为所闻,则得之矣。 (2)

(1) 经,理也。分,明也。

(2) 物之所不得然者,推之以人情,则夔不得一足、穿地作井不得一人明矣,故曰以为所闻得之矣。





第二十三卷 贵直论



贵直


一曰:

贤主所贵莫如士。所以贵士,为其直言也,言直则枉者见矣。 (1) 人主之患,欲闻枉而恶直言,是障其源而欲其水也, (2) 水奚自至? (3) 是贱其所欲而贵其所恶也,所欲奚自来? (4)

(1) 睹玉之白,别漆之黑也,故枉者见矣。

(2) 障,塞也。

【校】孙云:“《御览》四百二十八作‘是障水源而欲其流也’。”

(3) 奚,何也。自,从也。

(4) 所欲,欲闻己枉也。贵其所恶,恶闻直言,则己枉何从来至?《淮南子》曰:“塞其耳而欲闻五音,掩其目而欲詧青黄,不可得也。”此之谓也。

能意见齐宣王。宣王曰:“寡人闻子好直,有之乎?” (1) 对曰:“意恶能直?意闻好直之士,家不处乱国,身不见污君。身今得见王,而家宅乎齐,意恶能直?” (2) 宣王怒曰:“野士也!” (3) 将罪之。能意曰:“臣少而好事,长而行之。王胡不能与野士乎,将以彰其所好耶?” (4) 王乃舍之。 (5) 能意者,使谨乎论于主之侧,亦必不阿主。 (6) 不阿主之所得岂少哉?此贤主之所求,而不肖主之所恶也。 (7)

(1) 能,姓也。意,名也。齐士也。宣王,威王之子也。

(2) 宅,居也。恶,安也。

(3) 言鄙野之士也。

(4) 与犹用也。彰,明也。上有明君,下乃有直臣,王胡为不能用意之好直也?

(5) 舍,不诛也。

(6) 阿,曲也。

(7) 恶,疾也。

狐援说齐湣王曰:“殷之鼎陈于周之廷, (1) 其社盖于周之屏, (2) 其干戚之音在人之游。 (3) 亡国之音不得至于庙,亡国之社不得见于天,亡国之器陈于廷,所以为戒。 (4) 王必勉之!其无使齐之大吕陈之廷, (5) 无使太公之社盖之屏, (6) 无使齐音充人之游。”齐王不受。 (7) 狐援出而哭国三日, (8) 其辞曰:“先出也, (9) 衣 ;后出也,满囹圄。吾今见民之洋洋然东走,而不知所处。”齐王问吏曰:“哭国之法若何?”吏曰:“斮。” (10) 王曰:“行法!”吏陈斧质于东闾,不欲杀之而欲去之。狐援闻而蹶往过之。 (11) 吏曰:“哭国之法斮,先生之老欤?昏欤?” (12) 狐援曰:“曷为昏哉?”于是乃言曰:“有人自南方来,鲋入而鲵居, (13) 使人之朝为草而国为墟。 (14) 殷有比干,吴有子胥,齐有狐援。已不用若言, (15) 又斮之东闾,每斮者以吾参夫二子者乎!” (16) 狐援非乐斮也,国已乱矣,上已悖矣,哀社稷与民人,故出若言。出若言非平论也,将以救败也,固嫌于危。 (17) 此触子之所以去之也,达子之所以死之也。 (18)

(1) 狐援,齐臣也。湣王,齐宣王之子也。殷纣灭亡,鼎迁于周,故陈其庭也。

【校】“狐援”,《齐策》作“狐咺”,《古今人表》作“狐爰”。

(2) 屏,障也。言周存殷社而屋其上,屏之以为戒也。

(3) 干,楯。戚,斧。舞者所执以舞也。游,乐也。

(4) 戒惧灭亡。

(5) 大吕,齐之钟律也。陈,列也。

(6) 太公,田常之孙田和也,始代吕氏为齐侯,田氏宗之,号为太公。

(7) 湣王不受狐援之言。

(8) 狐援哭也。

【校】案:合两注观之,正文本无“狐援”二字。“三日”,《困学纪闻·考史》引作“五日”,或笔误。

(9) 出,去也。

(10) 斮,斩。

(11) 蹶,颠蹶走往也。过犹见也。

(12) 昏,乱也。

(13) 鲋,小鱼。鲵,大鱼,鱼之贼也,啖食小鱼。而鲵居人国,喻为人害。

(14) 墟,丘墟也。

(15) 若言犹直言也。

(16) 每犹当也。斮狐援者,比比干、子胥而三之也,故曰以参夫二子者。

(17) 固,必也。嫌犹近也。

(18) 乐毅为燕昭王将,伐齐,齐使触子应之。齐湣王不礼触子,触子欲齐军败,触子乘车而去,故曰“所以去之”。达子代触子将,又为燕败,故曰“达子之所以死”也。

【校】事见《权勋》篇。

赵简子攻卫附郭,自将兵,及战,且远立, (1) 又居于犀蔽屏橹之下, (2) 鼓之而士不起。简子投桴而叹曰 (3) :“呜呼!士之速弊一若此乎。” (4) 行人烛过免胄横戈而进曰:“亦有君不能耳,士何弊之有?” (5) 简子艴然作色曰:“寡人之无使,而身自将是众也。 (6) 子亲谓寡人之无能,有说则可,无说则死!”对曰:“昔吾先君献公即位五年,兼国十九, (7) 用此士也。惠公即位二年,淫色暴慢,身好玉女, (8) 秦人袭我,逊去绛七十, (9) 用此士也。 (10) 文公即位二年,厎之以勇,故三年而士尽果敢,城濮之战,五败荆人,围卫取曹,拔石社, (11) 定天子之位, (12) 成尊名于天下, (13) 用此士也。亦有君不能取, (14) 士何弊之有?”简子乃去犀蔽屏橹,而立于矢石之所及, (15) 一鼓而士毕乘之。 (16) 简子曰:“与吾得革车千乘也,不如闻行人烛过之一言。”行人烛过可谓能谏其君矣,战斗之上,枹鼓方用,赏不加厚,罚不加重,一言而士皆乐为其上死。 (17)

(1) 附郭,近郭也。远立,立于矢石所不及也。

(2) 【校】孙云:“《御览》三百五十一作‘屏蔽犀橹’,又三百十三亦作‘犀橹’。《说文系传》广部‘庰’字引‘赵简子立于庰蔽之下’。盖今本‘犀’与‘屏’互易也。”

(3) 投,弃也。

(4) 速犹化也。一犹皆也。言士之变化弊恶皆如此乎。

(5) 【校】旧本脱“士”字,今从《御览》补,与下文合。

(6) 【校】“而”旧讹“汝”,今从《御览》改正。

(7) 【校】《韩非·难二》作“并国十七”。

(8) 玉女,美女也。

(9) 【校】《韩非》作“秦人来侵,去绛十七里”。

(10) 陨于韩,为秦所获也。

(11) 【校】梁仲子云:“《淮南·齐俗训》‘殷人之礼,其社用石’,详《陈氏礼书》九十二。”

(12) 天子,周襄王也,避子带之乱,出居于郑,文公纳之,故曰“定天子之位”也。

(13) 尊名,霸诸侯之名也。

(14) 【校】《韩非》作“耳”,《御览》三百十三同。

(15) 矢,箭;石,砮也。及,至也。

(16) 毕,尽也。乘,陵也。

【校】“陵”,旧讹“后”,今案文义改。

(17) 烛过之谏,简子能行。





直谏


二曰:

言极则怒, (1) 怒则说者危,非贤者孰肯犯危?而非贤者也,将以要利矣。 (2) 要利之人,犯危何益?故不肖主无贤者。无贤则不闻极言,不闻极言则奸人比周,百邪悉起, (3) 若此则无以存矣。凡国之存也,主之安也,必有以也。 (4) 不知所以,虽存必亡,虽安必危, (5) 所以不可不论也。 (6)

(1) 极,尽也。人能受逆耳之尽言者少,故怒之。

(2) 要,求也。

(3) 起,兴也。

(4) 《诗》云:“何其久也,必有以也。”此之谓也。

(5) 《书》曰“于安思危”,此之谓也。

【校】“于安思危”,《周书·程典解》文。刘本作“居安思危”,出《左氏襄十一年传》,魏绛亦引《书》以告晋悼公者。

(6) 论犹知也。

齐桓公、管仲、鲍叔、甯戚相与饮,酒酣, (1) 桓公谓鲍叔曰:“何不起为寿?”鲍叔奉杯而进曰:“使公毋忘出奔在于莒也, (2) 使管仲毋忘束缚而在于鲁也, (3) 使甯戚毋忘其饭牛而居于车下。” (4) 桓公避席再拜曰:“寡人与大夫能皆毋忘夫子之言,则齐国之社稷幸于不殆矣!” (5) 当此时也,桓公可与言极言矣。可与言极言,故可与为霸。

(1) 酣,乐也。

(2) 桓公遭公孙无知杀襄公之乱也,出奔莒。毋忘之者,欲令其在上不骄也。

(3) 不死公子纠之难,出奔于鲁,鲁人束缚之以归于齐。

(4) 甯戚,卫人也,为商旅,宿于齐郭门之外。桓公夜出郊迎客,甯戚于其车下饭牛,疾商歌。桓公知其贤,举以为大夫也。

(5) 避席,下席也。殆,危也。

荆文王得茹黄之狗,宛路之矰, (1) 以畋于云梦, (2) 三月不反;得丹之姬, (3) 淫,期年不听朝。 (4) 葆申曰:“先王卜以臣为葆,吉。 (5) 今王得茹黄之狗,宛路之矰,畋三月不反;得丹之姬,淫,期年不听朝。王之罪当笞。”王曰:“不穀免衣襁褓而齿于诸侯, (6) 愿请变更而无笞。”葆申曰:“臣承先王之令,不敢废也。王不受笞,是废先王之令也。臣宁抵罪于王,毋抵罪于先王。”王曰:“敬诺。”引席,王伏。 (7) 葆申束细荆五十, (8) 跪而加之于背,如此者再,谓:“王起矣。”王曰:“有笞之名一也。”遂致之。 (9) 申曰:“臣闻君子耻之,小人痛之。耻之不变,痛之何益?”葆申趣出, (10) 自流于渊,请死罪。文王曰:“此不穀之过也。葆申何罪?”王乃变更,召葆申,杀茹黄之狗,析宛路之矰, (11) 放丹之姬。后荆国兼国三十九。 (12) 令荆国广大至于此者,葆申之力也,极言之功也。

(1) 文王,荆武王之子。矰,弋射短矢。

【校】《说苑·正谏》篇“茹黄”作“如黄”,“宛路”作“箘簬”。《御览》二百六亦作“如黄”。

(2) 畋,猎也。云梦,楚泽,在南郡华容也。

(3) 【校】《说苑》“丹”作“舟”。

(4) 淫,惑也。朝,正也。

【校】注似以政训朝,不当作“正”。

(5) 葆,太葆,官也。申,名也。

【校】《说苑》“葆”俱作“保”。《淮南·说山训》作“鲍申”,非。

(6) 襁,缕格绳;褓,小儿被也。齿,列也。

【校】旧本“缕”讹“楼”,“被”讹“补”。案《明理》篇注云“襁,缕格上绳也”,此少一“上”字。“缕”字、“被”字据改正。

(7) 【校】《说苑》作“乃席王,王伏”。

(8) 【校】《说苑》“荆”作“箭”。

(9) 遂痛致之。

(10) 【校】《说苑》作“趋出”。

(11) 【校】《说苑》“析”作“折”,当从之。

(12) 【校】《说苑》作“兼国三十”。





知化


三曰:

夫以勇事人者以死也,未死而言死,不论, (1) 以虽知之与勿知同。 (2) 凡智之贵也,贵知化也。人主之惑者则不然。 (3) 化未至则不知,化已至,虽知之与勿知一贯也。事有可以过者,有不可以过者。而身死国亡,则胡可以过?此贤主之所重,惑主之所轻也。所轻,国恶得不危?身恶得不困?危困之道,身死国亡,在于不先知化也。吴王夫差是也。 (4) 子胥非不先知化也,谏而不听,故吴为丘墟,祸及阖庐。 (5)

(1) 诈言已死,不可为人论说。

【校】此注未明。事人以死,谓扞敌御难而致死,死有益于人国也。未得死所而徒以言死,其言又不用,是不论也。下“知之”指君言,下文甚明。注皆非。

(2) 《诗》云:“既明且哲,以保其身。”《传》曰:“生,好物也。死,恶物也。好物,乐也。恶物,哀也。”虽知以死事人,是为乐可哀也,故与勿知同。勿,无也。

(3) 不然,不知化也。

(4) 夫差,吴王阖庐光之子也。夫差不知胜越,而为越所灭也。

(5) 越王句践报吴,灭其社稷,故为丘墟也。宗庙破灭,不得血食,故曰“祸及阖庐”也。

吴王夫差将伐齐。子胥曰:“不可。夫齐之与吴也,习俗不同,言语不通。我得其地不能处, (1) 得其民不得使。 (2) 夫吴之与越也,接土邻境,壤交通属, (3) 习俗同,言语通。我得其地能处之,得其民能使之。越于我亦然。夫吴、越之势不两立,越之于吴也,譬若心腹之疾也,虽无作,其伤深而在内也。夫齐之于吴也,疥癣之病也,不苦其已也,且其无伤也。今释越而伐齐,譬之犹惧虎而刺猏, (4) 虽胜之,其后患无央。” (5) 太宰嚭曰:“不可。君王之令所以不行于上国者,齐、晋也。君王若伐齐而胜之,徙其兵以临晋,晋必听命矣,是君王一举而服两国也,君王之令必行于上国。” (6) 夫差以为然,不听子胥之言,而用太宰嚭之谋。子胥曰:“天将亡吴矣,则使君王战而胜;天将不亡吴矣,则使君王战而不胜。”夫差不听。子胥两袪高蹶而出于廷, (7) 曰:“嗟乎!吴朝必生荆棘矣。” (8) 夫差兴师伐齐,战于艾陵大败齐师, (9) 反而诛子胥。子胥将死,曰:“与!吾安得一目以视越人之入吴也?”乃自杀。夫差乃取其身而流之江, (10) 抉其目,著之东门,曰:“女胡视越人之入我也?”居数年,越报吴,残其国,绝其世,灭其社稷,夷其宗庙, (11) 夫差身为擒。 (12) 夫差将死,曰:“死者如有知也,吾何面以见子胥于地下?”乃为幎以冒而死。 (13) 夫患未至,则不可告也;患既至,虽知之无及矣。故夫差之知惭于子胥也,不若勿知。

(1) 处,居也。

(2) 使,役也。

(3) 属,连也。

(4) 兽三岁曰猏也。

(5) 虎之患未能央。

【校】央亦训尽。后患不必指虎言。

(6) 上国,中国也。

(7) 两手举衣而行。蹶,蹈也。《传》曰“鲁人之皋,使我高蹈”,嗔怒貌,此之谓也。

【校】此与举趾高正相似,《哀廿一年传》注“高蹈,远行也”,无嗔怒意。

(8) 嗟,叹辞也。子胥谓太宰嚭劝王伐齐,国必破亡,故朝生荆棘也。

(9) 艾陵,齐地也。

(10) 《传》曰“子胥自杀,吴王盛之鸱夷,投之江”,故曰“流”。

(11) 夷,平也。

(12) 为越所擒也。

(13) 冒,覆面也,惭见于子胥也。

【校】“以冒而死”,旧本作“以冒面死”,案注云“冒,覆面也”,则正文不当有“面”字,今改正。





过理


四曰:

亡国之主一贯, (1) 天时虽异,其事虽殊,所以亡同者,乐不适也,乐不适则不可以存。糟丘酒池,肉圃为格, (2) 雕柱而桔诸侯,不适也。 (3) 刑鬼侯之女而取其环, (4) 截涉者胫而视其髓, (5) 杀梅伯而遗文王其醢,不适也。 (6) 文王貌受以告诸侯。 (7) 作为琁室,筑为顷宫, (8) 剖孕妇而观其化, (9) 杀比干而视其心,不适也。 (10) 孔子闻之曰:“其窍通则比干不死矣。” (11) 夏、商之所以亡也。 (12)

(1) 一,道也。贯,同也。其所以亡之道同,同于不仁,且不知足也。

(2) 格以铜为之,布火其下,以人置上,人烂堕火而死,笑之以为乐,故谓之乐不适也。

【校】“炮格”,各书俱讹作“炮烙”,得此可以正之。

(3) 雕画高柱,施桔槔于其端,举诸侯而上下之,故曰“不适”。

(4) 听妲己之谮,杀鬼侯之女以为脯,而取其所服之环也。

【校】“环”,旧本作“瓌”,讹,今改正。

(5) 以其涉水能寒也,故视其髓,欲知其与人有异不也。

【校】注“能寒”,能读曰耐。

(6) 梅伯,纣之诸侯也,说鬼侯之女美好。纣受妲己之谮,以为不好,故杀梅伯以为醢。醢,肉酱也,以遗文王。故曰“不适也”。

(7) 貌受,心不受也,故曰“告诸侯”也。

(8) 琁室,以琁玉文饰其室也。顷宫,筑作宫墙,满一顷田中,言博大也。

【校】书传多云桀作璇室,纣作倾宫。今举属之纣,以言其土木之侈,固不必细为分别也。梁仲子云:“《淮南·本经训》注:‘琁或作旋。言室施机关,可转旋也。’顷宫,此注作如字读,它书俱作‘倾’字。”

(9) 化,育也。视其胞褢。

【校】注旧本作“胞裏”,“裏”当作“褢”,亦疑是“裹”字。

(10) 比干,纣之诸父也,数谏纣之非,纣不能听,故视其心,欲知其何以不与人同也。

(11) 圣人心达性通。纣性不仁,心不通,安于为恶,杀比干,故孔子言其一窍通则比干不见杀也。

(12) 桀杀关龙逄,纣杀比干,故曰此“夏、商之所以亡也”。

晋灵公无道,从上弹人而观其避丸也, (1) 使宰人臑熊 ,不熟, (2) 杀之,令妇人载而过朝以示威,不适也。赵盾骤谏而不听,公恶之,乃使沮麛。 (3) 沮麛见之,不忍贼, (4) 曰:“不忘恭敬,民之主也。贼民之主,不忠; (5) 弃君之命,不信。 (6) 一于此,不若死。” (7) 乃触廷槐而死。 (8)

(1) 灵公,襄公之子,文公之孙也。从高台上引弹,观其走而避丸以为乐也。

(2) 【校】《左氏宣二年传》作“宰夫胹熊蹯不熟”。

(3) 盾,赵成子之子宣子也。

【校】《左传》“使鉏麑贼之”,今此“贼之”二字亦当有,或下文“见之”字误,而又误入下文耳。

(4) 贼,杀也。

(5) 大夫称主,因曰民之主也。

(6) 违命不信。

(7) 不忠不信,若行之必有其一也。

【校】正文“一”上,《左传》有“有”字

(8) 触,畜也。

【校】“畜”疑“撞”字之误。

齐湣王亡居卫, (1) 谓公王丹曰:“我何如主也?” (2) 王丹对曰:“王,贤主也。臣闻古人有辞天下而无恨色者,臣闻其声, (3) 于王而见其实。 (4) 王名称东帝,实辨天下, (5) 去国居卫,容貌充满,颜色发扬, (6) 无重国之意。” (7) 王曰:“甚善!丹知寡人。寡人自去国居卫也,带益三副矣。” (8)

(1) 湣王,宣王之子。

(2) 公王丹,湣王臣也。

【校】公王丹即公玉丹,古“玉”字作“王”,三画匀。

(3) 声,名也。

(4) 所行之实。

(5) 辨,治也。

(6) 光明也。

(7) 言轻之也。

(8) “副”或作“倍”。度湣王之亡国宜也。但湎涎无忧耻辱,喜于公王丹巧佞之言,因云“丹知寡人”也。带益三倍,苟活者肥,令腹大耳。

宋王筑为蘖帝,鸱夷血,高悬之,射著甲胄,从下,血坠流地。 (1) 左右皆贺曰:“王之贤过汤、武矣。汤、武胜人,今王胜天,贤不可以加矣。” (2) 宋王大说,饮酒。室中有呼万岁者,堂上尽应。堂上已应,堂下尽应。门外庭中闻之,莫敢不应。不适也。 (3)

(1) 宋王,康王也。“蘖”当作“ ”,“帝”当作“臺”。蘖与 其音同,帝与臺字相似,因作“蘖帝”耳。《诗》云“庶姜 ”,高长 也。言康王筑为台,革囊之大者为鸱夷,盛血于台上,高悬之以象天,著甲胄,自下射之,血流堕地,与之名,言中天神下其血也。

【校】注“ ”,旧本作“类”,讹。“与之名言”四字,刘本作“谓之”二字。

(2) 加,上也。

(3) 不僭不滥、动中礼义之谓适。今此畏无道,不敢不应耳,故曰“不适也”。





壅塞


五曰:

亡国之主,不可以直言。不可以直言,则过无道闻, (1) 而善无自至矣。无自至则壅。 (2)

(1) 不可以直言谏正也,则其过成。以无道远闻,人皆闻之。

【校】过无道闻,言过无路以闻于主也。注非是。

(2) 自,从也。《传》曰“善进善,不善蔑由至矣;不善进不善,善亦蔑由至矣”,故曰“壅”。

【校】注“传曰”下文有脱,今据《论人篇》注增补。

秦缪公时,戎强大。秦缪公遗之女乐二八与良宰焉。戎主大喜,以其故,数饮食,日夜不休。左右有言秦寇之至者,因扜弓而射之。 (1) 秦寇果至,戎主醉而卧于樽下,卒生缚而擒之。未擒则不可知, (2) 已擒则又不知。 (3) 虽善说者,犹若此何哉? (4)

(1) 寇,兵也。扜,引也。

【校】“扜”旧讹作“扞”,注同。案《大荒南经》“有人方扜弓射黄蛇,名曰蜮人”,郭璞注:“扜,挽也,音纡。”今据改正。扜亦音乌。

(2) 不知将见擒也。

(3) 醉不自知也。

【校】旧校云“一本作‘既擒则无及矣’”,李本“矣”作“也”。

(4) 言说无如之何。

齐攻宋, (1) 宋王使人候齐寇之所至。 (2) 使者还,曰:“齐寇近矣,国人恐矣。”左右皆谓宋王曰:“此所谓‘肉自生虫’者也。 (3) 以宋之强,齐兵之弱,恶能如此?” (4) 宋王因怒而诎杀之。 (5) 又使人往视齐寇,使者报如前,宋王又怒诎杀之。如此者三。其后又使人往视,齐寇近矣,国人恐矣。使者遇其兄,曰:“国危甚矣,若将安适?” (6) 其弟曰:“为王视齐寇, (7) 不意其近而国人恐如此也。今又私患乡之先视齐寇者,皆以寇之近也报而死。今也报其情,死; (8) 不报其情?又恐死。 (9) 将若何?”其兄曰:“如报其情,有且先夫死者死,先夫亡者亡。” (10) 于是报于王曰:“殊不知齐寇之所在,国人甚安。”王大喜。左右皆曰:“乡之死者宜矣。”王多赐之金。寇至,王自投车上驰而走,此人得以富于他国。夫登山而视牛若羊,视羊若豚。牛之性不若羊,羊之性不若豚, (11) 所自视之势过也,而因怒于牛羊之小也,此狂夫之大者。狂而以行赏罚,此戴氏之所以绝也。 (12)

(1) 齐湣王攻宋,灭之也。

(2) 候,视也。

(3) 【校】“生”旧本作“至”,讹,今改正。

(4) 言宋强盛,齐兵之弱,安能来至此也。

(5) 诎,枉也。无罪而杀之曰枉。

(6) 适,之也。

(7) 【校】“为王”,旧本作“为兄”,讹,今改正。

(8) 以齐寇至之情实告宋王,必诛死也。

(9) 不以寇至之情报而设备,齐寇至杀人,是又恐死。

(10) 【校】有,读与又同。

(11) 性犹体也。若犹如也。

(12) 戴氏子罕,戴公子孙也,别为乐氏。《传》曰:“宋之乐,其与宋升降乎?”宋国衰,子罕后子孙亦衰,赏罚失中,故曰“此戴氏之所以绝也”。

【校】旧校云:“‘戴氏’一本作‘叔世’。”

齐王欲以淳于髡傅太子,髡辞曰:“臣不肖,不足以当此大任也,王不若择国之长者而使之。”齐王曰:“子无辞也。寡人岂责子之令太子必如寡人也哉?寡人固生而有之也。子为寡人令太子如尧乎,其如舜也?”凡说之行也,道 (1) 不智听智,从自非受是也。 (2) 今自以贤过于尧、舜, (3) 彼且胡可以开说哉?说必不入,不闻存君。 (4)

(1) 句。

(2) 【校】道谓有道也。“自”字疑衍。

(3) 【校】旧校云:“‘过’一作‘远’。”

(4) 不纳忠言之说,鲜不危亡,故曰“不闻存君”也。

齐宣王好射, (1) 说人之谓己能用强弓也。 (2) 其尝所用不过三石,以示左右。左右皆试引之,中关而止, (3) 皆曰:“此不下九石,非王其孰能用是?” (4) 宣王之情, (5) 所用不过三石,而终身自以为用九石,岂不悲哉? (6) 非直士其孰能不阿主?世之直士,其寡不胜众,数也。 (7) 故乱国之主,患存乎用三石为九石也。 (8)

(1) 好,喜也。

(2) 示有力也。

【校】“用”旧作“则”,孙据《御览》三百四十七改正。

(3) 关,谓关弓。弦正半而止也。

(4) 言九石之弓,独王用之耳。

(5) 情,实也。

(6) 伤其自轻,而不知其实。

【校】注“自轻”疑“用轻”之误。

(7) 数,道数也。

(8) 力不足,而自以为有余也。其功德,其治理,皆亦如之也。





原乱


六曰:

乱必有弟, (1) 大乱五,小乱三, 乱三。 (2) 故《诗》曰“毋过乱门”,所以远之也。 (3) 虑福未及,虑祸之,所以皃之也。 (4) 武王以武得之,以文持之,倒戈弛弓,示天下不用兵,所以守之也。

(1) 弟,次也。

【校】“弟”,本一作“第”,今从汪本,乃古“第”字。

(2) 大乱五,谓晋国废长立少,立而复杀之也。小乱三,谓杀里克之党也。 乱三,谓于朝栾盈以兵昼入于绛也。

【校】 >字或音喧声也,或云与“訆”同,义皆不当,注亦不明了。此似皆指骊姬之乱,安得忽及栾盈?又“于朝”上似尚有缺文。窃疑“ >”或是“讨”字之讹。惠公杀里克,文公杀吕郤,是讨乱三也。

(3) 逸《诗》也。

【校】案:《左氏昭十九年传》子产引作谚。

(4) 【校】“皃”疑“免”字之误。

晋献公立骊姬以为夫人,以奚齐为太子,里克率国人以攻杀之。 (1) 荀息立其弟公子卓,已葬,里克又率国人攻杀之。 (2) 于是晋无君。公子夷吾重赂秦以地而求入, (3) 秦缪公率师以纳之,晋人立以为君,是为惠公。惠公既定于晋,背秦德而不予地。 (4) 秦缪公率师攻晋,晋惠公逆之,与秦人战于韩原。晋师大败,秦获惠公以归,囚之于灵台。十月,乃与晋成, (5) 归惠公而质太子圉。太子圉逃归也。惠公死,圉立为君,是为怀公。秦缪公怒其逃归也,起奉公子重耳以攻怀公,杀之于高梁, (6) 而立重耳,是为文公。文公施舍,振废滞,匡乏困,救灾患,禁淫慝,薄赋敛,宥罪戾, (7) 节器用,用民以时,败荆人于城濮, (8) 定襄王, (9) 释宋,出穀戍, (10) 外内皆服, (11) 而后晋乱止。故献公听骊姬,近梁五、优施,杀太子申生,而大难随之者五,三君死,一君虏, (12) 大臣卿士之死者以百数,离咎二十年。

(1) 杀奚齐也。

(2) 复杀公子卓也。

(3) 地,河外之城五。求入为晋君也。

(4) 《传》曰“入而背秦赂”,此之谓也。

(5) 成,平也。

(6) 高梁,晋地。

(7) 宥,宽也。

(8) 荆人,成王。

(9) 周襄王辟子带之难,出居于郑,文公纳之,故曰“定”也。

(10) 楚子围宋,又使申公叔侯守齐之穀邑。晋文伐曹、卫,将平之。楚爱曹、卫,与晋俱成,解宋之围,召穀戍而去之也。

(11) 外,诸侯;内,卿大夫也。皆服文公之德也。

(12) 三君死,申生、奚齐、公子卓也。一君虏,惠公为秦所执,囚之灵台也。

【校】谢云“三君死,谓奚齐、卓子、怀公。注误。”

自上世以来,乱未尝一。而乱人之患也,皆曰一而已,此事虑不同情也。事虑不同情者,心异也。故凡作乱之人,祸希不及身。 (1)

(1) 希,鲜也。





第二十四卷 不苟论



不苟


一曰:

贤者之事也,虽贵不苟为, (1) 虽听不自阿, (2) 必中理然后动, (3) 必当义然后举, (4) 此忠臣之行也。贤主之所说, (5) 而不肖主之所不说, (6) 非恶其声也。人主虽不肖, (7) 其说忠臣之声与贤主同, (8) 行其实则与贤主有异。 (9) 异,故其功名祸福亦异。 (10) 异,故子胥见说于阖闾而恶乎夫差, (11) 比干生而恶于商, (12) 死而见说乎周。 (13)

(1) 虽欲尊贵,不苟为也。不如礼曰苟为也。

(2) 虽言见听,当以忠正,不自阿媚以取容也。

(3) 非理不移也。

(4) 非义不行也。

(5) 说犹敬也。

(6) 【校】旧作“而不肖主虽不肖其说”,乃因下文而讹,今改正。

(7) 句。

(8) 同,等也。

(9) 贤主能用忠臣之言,不肖主能刑杀之,故曰“有异”也。

(10) 贤主受大福,不肖主获大祸,故曰“亦异”也。

(11) 夫差恶子胥也。

(12) 商纣恶之也。

(13) 周武王说其忠也。

武王至殷郊,系堕。 (1) 五人御于前,莫肯之为, (2) 曰:“吾所以事君者非系也。”武王左释白羽,右释黄钺,勉而自为系。孔子闻之曰:“此五人者之所以为王者佐也,不肖主之所弗安也。”故天子有不胜细民者,天下有不胜千乘者。 (3)

(1) 【校】《韩非·外储说左下》云“文王伐崇,至凤黄虚,韈系解,因自结”,一事而传者异。

(2) 【校】疑是“为之系”,倒二字,脱一字。

(3) 天下,海内也。千乘,一国也。

秦缪公见戎由余,说而欲留之,由余不肯。缪公以告蹇叔,蹇叔曰:“君以告内史廖。”内史廖对曰:“戎人不达于五音与五味,君不若遗之。”缪公以女乐二八人与良宰遗之。 (1) 戎王喜,迷惑大乱,饮酒,昼夜不休。由余骤谏而不听,因怒而归缪公也。蹇叔非不能为内史廖之所为也,其义不行也。缪公能令人臣时立其正义,故雪殽之耻而西至河雍也。 (2)

(1) 【校】“人”字疑衍。 宰,谓膳宰。

(2) 雪,除也。

秦缪公相百里奚。 (1) 晋使叔虎、齐使东郭蹇如秦,公孙枝请见之。 (2) 公曰:“请见客,子之事欤?”对曰:“非也。”“相国使子乎?” (3) 对曰:“不也。”公曰:“然则子事非子之事也。 (4) 秦国僻陋戎夷,事服其任,人事其事,犹惧为诸侯笑。今子为非子之事,退!将论而罪。” (5) 公孙枝出,自敷于百里氏。百里奚请之,公曰:“此所闻于相国欤?枝无罪,奚请?有罪,奚请焉?” (6) 百里奚归,辞公孙枝。公孙枝徙,自敷于街,百里奚令吏行其罪。定分官,此古人之所以为法也。今缪公乡之矣,其霸西戎,岂不宜哉?

(1) 以百里奚为相也。

(2) 【校】梁仲子云:“叔虎即下文郤子虎,晋大夫郤芮子父郤豹也,见韦昭《晋语》注。” 公孙枝,秦大夫子桑也。

(3) 相国,百里奚也。

(4) 事,见客事也。

【校】上“子”字疑衍。

(5) 而,汝也。

(6) 奚,何也。

晋文公将伐邺,赵衰言所以胜邺之术,文公用之,果胜。还,将行赏。衰曰:“君将赏其本乎?赏其末乎?赏其末则骑乘者存,赏其本则臣闻之郤子虎。”文公召郤子虎曰:“衰言所以胜邺,邺既胜,将赏之,曰:‘盖闻之于子虎,请赏子虎。’” (1) 子虎曰:“言之易,行之难。臣言之者也。”公曰:“子无辞。”郤子虎不敢固辞,乃受矣。凡行赏欲其博也,博则多助。今虎非亲言者也,而赏犹及之,此疏远者之所以尽能竭智者也。晋文公亡久矣,归而因大乱之余,犹能以霸,其由此欤? (2)

(1) 【校】《新序》四、《御览》六百三十三皆无两“虎”字,是。

(2) 亡久,谓避骊姬之乱,在狄十二年,历行诸侯五年,凡十七年。归晋国,因大乱之后,能建霸功,皆由用此术也。





赞能


二曰:

贤者善人以人,中人以事, (1) 不肖者以财。 (2) 得十良马,不若得一伯乐; (3) 得十良剑,不若得一欧冶; (4) 得地千里,不若得一圣人。 (5) 舜得皋陶而舜受之, (6) 汤得伊尹而有夏民, (7) 文王得吕望而服殷商。 (8) 夫得圣人,岂有里数哉? (9)

(1) 贤者以人,以人之德也。中人任人,以人之力也。

(2) 不肖者任人,以人之财贿也。《传》曰“政以贿成”,此之谓也。

(3) 伯乐善得马,得伯乐则得良马,不但十也,故曰“不若得一伯乐”也。

(4) 欧冶善为剑工也,义与伯乐同。

(5) 义与欧冶同。

【校】孙云:“《初学记》十七《贤类》引作‘不如得一贤士’,《意林》及《御览》四百二皆作‘贤人’。《御览》八百九十六作‘圣人’,当由后来传本误也。”

(6) 受,用也。

【校】注“受”字,旧本作“授”。今案:“受之”即《书》所谓“俾予从欲以治”也,不当训用。舜未授皋陶以天下,亦不当作“授”。

(7) 有夏桀之民也,王天下也。

(8) 殷纣之众服从文王之德也。

(9) 言得其用多不可数也,故曰“岂有里数哉”。

管子束缚在鲁, (1) 桓公欲相鲍叔。 (2) 鲍叔曰:“吾君欲霸王,则管夷吾在彼, (3) 臣弗若也。”桓公曰:“夷吾,寡人之贼也,射我者也,不可。” (4) 鲍叔曰:“夷吾为其君射人者也。 (5) 君若得而臣之,则彼亦将为君射人。”桓公不听, (6) 强相鲍叔。固辞让而相, (7) 桓公果听之。于是乎使人告鲁曰:“管夷吾,寡人之仇也,愿得之而亲加手焉。” (8) 鲁君许诺,乃使吏鞹其拳, (9) 胶其目,盛之以鸱夷,置之车中。至齐境, (10) 桓公使人以朝车迎之,祓以爟火,衅以牺猳焉, (11) 生与之如国, (12) 命有司除庙筵几而荐之, (13) 曰:“自孤之闻夷吾之言也,目益明,耳益聪。孤弗敢专,敢以告于先君。” (14) 因顾而命管子曰:“夷吾佐予!” (15) 管仲还走,再拜稽首,受令而出。 (16) 管子治齐国,举事有功,桓公必先赏鲍叔,曰:“使齐国得管子者,鲍叔也。”桓公可谓知行赏矣。凡行赏欲其本也,本则过无由生矣。 (17)

(1) 为鲁所束缚也。

(2) 欲以鲍叔为齐相也。

(3) 彼,鲁也。

(4) 《传》曰“乾时之役,申孙之矢射于桓公,中钩”,故曰“不可”。

(5) 其君,公子纠也。

(6) 不从鲍叔之言。

(7) 固,必也。

【校】“鲍叔”当重,“而相”二字衍文。

(8) 言欲得管仲,亲手自杀之,以为辞也。

(9) 鞹,革也。以革囊其手也。

(10) 境,界也。

(11) 火所以祓除不祥也。《周礼》“司爟掌行火之政令”,故以爟火祓之也。杀牲以血涂之为衅。小事不用大牲,故以猳豚也。《传》曰“郑伯使卒出猳,行出犬鸡”,此之谓也。爟,读如权衡。

【校】“权衡”,旧本误作“权字”,今依《本味篇》注改正。

(12) 如,至也。

(13) 荐,进也。

(14) 告,白也。

(15) 予,我也。

(16) 出于庙也。

(17) 过,失也。

孙叔敖、沈尹茎相与友。 (1) 叔敖游于郢三年,声问不知, (2) 修行不闻。 (3) 沈尹茎谓孙叔敖曰:“说义以听,方术信行,能令人主上至于王,下至于霸,我不若子也。耦世接俗,说义调均,以适主心,子不若我也。子何以不归耕乎?吾将为子游。” (4) 沈尹茎游于郢五年,荆王欲以为令尹,沈尹茎辞曰:“期思之鄙人有孙叔敖者,圣人也。 (5) 王必用之,臣不若也。”荆王于是使人以王舆迎叔敖,以为令尹,十二年而庄王霸,此沈尹茎之力也。功无大乎进贤。

(1) 【校】当作“筮”,下同。

(2) 【校】旧校云:“‘问’一作‘晦’。”

(3) 郢,楚都也。

(4) 欲令孙叔敖隐也。

【校】游谓游扬也。

(5) 【校】梁仲子云:“《左传》文十年杜注:‘楚期思邑,今弋阳期思县。’杨倞注《荀子·非相》篇云:‘鄙人,郊野之人也。’”





自知


三曰:

欲知平直,则必準绳; (1) 欲知方圆,则必规矩; (2) 人主欲自知,则必直士。 (3) 故天子立辅弼,设师保,所以举过也。 (4) 夫人故不能自知,人主犹其。 (5) 存亡安危,勿求于外, (6) 务在自知。

(1) 準,平;绳,直也。

【校】李本“準”皆作“准”。

(2) 规,圆;矩,方也。

(3) 唯直士能正言也。

(4) 举犹正也。

(5) 【校】孙云:“《御览》七十七作‘夫人固不能自知,人主独甚’。此‘犹其’二字讹。”

(6) 言皆在己也。

尧有欲谏之鼓, (1) 舜有诽谤之木, (2) 汤有司过之士, (3) 武王有戒慎之鞀, (4) 犹恐不能自知。 (5) 今贤非尧、舜、汤、武也,而有掩蔽之道,奚繇自知哉?荆成、齐庄不自知而杀, (6) 吴王、智伯不自知而亡, (7) 宋、中山不自知而灭, (8) 晋惠公、赵括不自知而虏, (9) 钻荼、庞涓、太子申不自知而死, (10) 败莫大于不自知。 (11)

(1) 欲谏者击其鼓也。

【校】《淮南·主术训》作“尧置敢谏之鼓”。

(2) 书其过失以表木也。

【校】注“以”字,《淮南》注作“于”。

(3) 司,主也。主,正也。正其过阙也。

(4) 欲戒者摇其鞀鼓之。

(5) 犹尚恐己不能自知其过失也。

(6) 荆成王为公子商臣所杀,齐庄公为崔杼所杀,皆不自知之咎也。

(7) 吴王,夫差也。智伯,晋卿智襄子也。夫差为越所破,死于干隧;智伯为赵襄子所破,死于高梁之东;故曰“而亡”也。

(8) 宋康王无道,为齐所灭。中山乱男女之别,为魏所灭也。

(9) 惠公为秦所虏。赵括以军降,秦坑其兵四十万于长平也。

(10) 钻荼、庞涓,魏惠王之将。申,魏惠王之太子也,与庞涓东伐齐,战于马陵,齐人尽杀之。故惠王谓孟子曰:“晋国,天下莫强焉,叟之所知也。及寡人身,东败于齐,长子死。”此之谓也。

(11) 莫,无也。

范氏之亡也, (1) 百姓有得钟者,欲负而走,则钟大不可负,以椎毁之,钟况然有音, (2) 恐人闻之而夺己也,遽掩其耳。 (3) 恶人闻之,可也;恶己自闻之,悖矣。为人主而恶闻其过,非犹此也? (4) 恶人闻其过尚犹可。

(1) 范氏,晋卿范武子之后也。谓简子率师逐范吉射也。一曰:智伯伐范氏而灭之,故曰“亡”也。

(2) 【校】李善注《文选》任彦昇《百辟劝进笺》“况然”作“怳然”,《淮南·说山训》作“鎗然有声”。

(3) 遽,疾也。

(4) 此,自掩其耳也。

【校】案:“非犹此也”,“也”与“邪”通用。《选》注作“亦犹此也”,则如字。

魏文侯燕饮,皆令诸大夫论己。 (1) 或言君之智也。 (2) 至于任座,任座曰:“君,不肖君也。得中山不以封君之弟,而以封君之子,是以知君之不肖也。”文侯不说,知于颜色。 (3) 任座趋而出。次及翟黄,翟黄曰:“君,贤君也。臣闻其主贤者,其臣之言直。今者任座之言直,是以知君之贤也。”文侯喜曰:“可反欤?” (4) 翟黄对曰:“奚为不可?臣闻忠臣毕其忠, (5) 而不敢远其死。座殆尚在于门。” (6) 翟黄往视之,任座在于门,以君令召之。任座入,文侯下阶而迎之,终座以为上客。 (7) 文侯微翟黄,则几失忠臣矣。 (8) 上顺乎主心以显贤者,其唯翟黄乎! (9)

(1) 【校】李善注《文选》孔文举《荐祢衡表》引作“问诸大夫,寡人何如主也”。

(2) 【校】孙云:“《御览》六百二十二作‘或言君仁,或言君义,或言君智’,疑此有脱文。”

(3) 知犹见也。

(4) 欤,邪也。谓任座可反邪?

(5) 毕,尽也。

(6) 殆犹必也。

(7) 客,敬也。

(8) 微,无也。几,近也。

(9) 【校】《新序》一前作翟黄语,后作任座语,与此互异。





当赏


四曰:

民无道知天,民以四时寒暑日月星辰之行知天。 (1) 四时寒暑日月星辰之行当,则诸生有血气之类皆为得其处而安其产。 (2) 人臣亦无道知主, (3) 人臣以赏罚爵禄之所加知主。 (4) 主之赏罚爵禄之所加者宜, (5) 则亲疏远近贤不肖皆尽其力而以为用矣。 (6)

(1) 以,用也。

(2) 产,生也。

【校】《日抄》作“皆得其处”,无“为”字。

(3) 主,君也。

(4) 加,施也。

(5) 宜犹当也。

(6) 为君用也。

晋文公反国,赏从亡者,而陶狐不与。 (1) 左右曰:“君反国家,爵禄三出,而陶狐不与。敢问其说。” (2) 文公曰:“辅我以义、导我以礼者,吾以为上赏;教我以善、强我以贤者,吾以为次赏;拂吾所欲、数举吾过者,吾以为末赏。三者所以赏有功之臣也。若赏唐国之劳徒,则陶狐将为首矣。” (3) 周内史兴闻之曰:“晋公其霸乎! (4) 昔者圣王先德而后力,晋公其当之矣!” (5)

(1) 赏不及之也。

【校】梁仲子云:“‘陶狐’,《史记·晋世家》作‘壶叔’,《外传》三、《说苑·复恩》篇作‘陶叔狐’。”

(2) 欲知之也。

(3) 唐国,晋国也。勤劳之徒,则陶狐也,故不与三赏中也。

【校】注“故”字旧作“欲”,讹,今改正。

(4) 内史兴,周大夫也,奉使来赐文公命闻之。

(5) 当先德而后力也。

秦小主夫人用奄变,群贤不说自匿,百姓郁怨非上。 (1) 公子连亡在魏,闻之,欲入,因群臣与民从郑所之塞。 (2) 右主然守塞,弗入, (3) 曰:“臣有义,不两主。公子勉去矣!” (4) 公子连去,入翟,从焉氏塞, (5) 菌改入之。 (6) 夫人闻之,大骇, (7) 令吏兴卒,奉命曰:“寇在边。”卒与吏其始发也, (8) 皆曰:“往击寇。”中道因变曰:“非击寇也,迎主君也。” (9) 公子连因与卒俱来,至雍,围夫人,夫人自杀。 (10) 公子连立,是为献公,怨右主然而将重罪之, (11) 德菌改而欲厚赏之。 (12) 监突争之曰:“不可。 (13) 秦公子之在外者众, (14) 若此则人臣争入亡公子矣。此不便主。” (15) 献公以为然,故复右主然之罪, (16) 而赐菌改官大夫, (17) 赐守塞者人米二十石。献公可谓能用赏罚矣。凡赏非以爱之也,罚非以恶之也,用观归也。所归善,虽恶之,赏;所归不善,虽爱之,罚; (18) 此先王之所以治乱安危也。 (19)

(1) 小主,秦君也,秦厉公曾孙惠公之子也。夫人用奄变,为惑乱也。

【校】以《史记·秦本纪》考之,小主即出子也。

(2) 公子连一名元,秦厉公曾孙灵公之子也,于小主为从父昆弟也。

【校】公子连即献公,于小主为从祖昆弟。《索隐》云“名师隰”,殆据《世本》。

(3) 右主然,秦守塞吏也。弗内公子连也。

(4) 内公子连则两主矣。劝之使疾去。

(5) 塞在安定。将之北翟。

【校】注“将”、“翟”二字疑衍。

(6) 菌改亦守塞吏也。入之,内公子连也。

(7) 小主夫人也。骇,惊也。

(8) 发,行也。

(9) 主君,谓公子连。

(10) 雍,秦都也。

(11) 怨其不入己也。

(12) 德其入己也。

(13) 监突,秦大夫也。

(14) 众,多也。

(15) 如此则诸臣争内亡公子。亡公子得入,则争为君,故于主不便也。

(16) 复,反也。反其罪,不复罪也。

(17) 官大夫,秦爵也。

(18) 《传》曰:“善有章,虽贱,赏也;恶有衅,虽贵,罚也。”此之谓也。

(19) 乱者能治之也,危者能安之也。





博志


五曰:

先王有大务,去其害之者,故所欲以必得,所恶以必除,此功名之所以立也。 (1) 俗主则不然,有大务而不能去其害之者,此所以无能成也。夫去害务与不能去害务,此贤不肖之所以分也。 (2) 使獐疾走,马弗及至,已而得者,其时顾也。 (3) 骥一日千里,车轻也;以重载则不能数里,任重也。 (4) 贤者之举事也,不闻无功, (5) 然而名不大立、利不及世者,愚不肖为之任也。 (6) 冬与夏不能两刑, (7) 草与稼不能两成,新谷熟而陈谷亏,凡有角者无上齿,果实繁者木必庳, (8) 用智褊者无遂功,天之数也。 (9) 故天子不处全,不处极,不处盈。全则必缺,极则必反,盈则必亏。先王知物之不可两大,故择务,当而处之。

(1) 立,成也。

(2) 分,别也。

(3) 反顾稽其行,故见得也。

(4) 任,载也。

(5) 言有功也。

(6) 愚不肖人为之任政事,故使其君贤名不立,福利不及后世子孙也。

(7) 《传》曰“火中而寒暑退”,故曰“不能两刑”。

【校】案:刑犹成也。

(8) 有核曰果。物莫能两大,故戴角者无上齿,果实繁者木为之庳小也。

【校】《大戴礼·易本命》篇“戴角者无上齿”,又《战国·秦策》引诗曰“木实繁者披其枝”,亦是此义。梁仲子云:“齿、庳为韵。”

(9) 遂,成也。

孔、墨、甯越,皆布衣之士也,虑于天下,以为无若先王之术者, (1) 故日夜学之。有便于学者,无不为也;有不便于学者,无肯为也。盖闻孔丘、墨翟,昼日讽诵习业,夜亲见文王、周公旦而问焉。 (2) 用志如此其精也, (3) 何事而不达?何为而不成?故曰:“精而熟之,鬼将告之。”非鬼告之也,精而熟之也。 (4) 今有宝剑良马于此,玩之不厌,视之无倦;宝行良道,一而弗复。欲身之安也,名之章也,不亦难乎?

(1) 孔子、墨翟也。甯越,中牟人也,知道术之士也。

(2) 夜则梦见文王、周公而问其道也。《论语》曰:“吾衰久矣,吾不复梦见周公。”

【校】案:“吾衰久矣”,尚是朱子以前读法,宋本句读亦如此。

(3) 精,微密也。

(4) 史曰“日精所学,致无鬼神”,故曰有鬼告之也。

甯越,中牟之鄙人也,苦耕稼之劳,谓其友曰:“何为而可以免此苦也?”其友曰:“莫如学。学三十岁则可以达矣。”甯越曰:“请以十五岁。 (1) 人将休,吾将不敢休;人将卧,吾将不敢卧。” (2) 十五岁而周威公师之。 (3) 矢之速也,而不过二里止也;步之迟也,而百舍不止也。今以甯越之材而久不止,其为诸侯师,岂不宜哉?

(1) 【校】“五”字旧本脱,据李善注《文选》韦宏嗣《博弈论》补,《御览》六百十一同。

(2) 【校】案:“吾”下两“将”字皆疑衍。

(3) 威公,西周君也。师之者,以甯越为师也。

养由基、尹儒,皆文艺之人也。 (1) 荆廷尝有神白猿,荆之善射者莫之能中,荆王请养由基射之。养由基矫弓操矢而往,未之射而括中之矣,发之则猿应矢而下,则养由基有先中中之者矣。 (2) 尹儒学御三年而不得焉,苦痛之, (3) 夜梦受秋驾于其师。明日往朝其师, (4) 望而谓之曰:“吾非爱道也,恐子之未可与也。今日将教子以秋驾。” (5) 尹儒反走,北面再拜,曰:“今昔臣梦受之。”先为其师言所梦,所梦固秋驾已。 (6) 上二士者可谓能学矣,可谓无害之矣,此其所以观后世已。 (7)

(1) 【校】“尹儒”一作“尹需”。“文艺”,本或作“六艺”,今从李本,与下篇合。

(2) 《幽通记》曰“养流睇而猿号”,此之谓也。

【校】注“流”字,旧作“由基”二字,讹,今改正。

(3) 痛,悼也。

(4) 【校】“师”字当重。

(5) 秋驾,御法也。

(6) 句。

(7) 二士,甯越、尹儒也。观,示也。





贵当


六曰:

名号大显,不可强求,必繇其道。 (1) 治物者不于物于人,治人者不于事于君, (2) 治君者不于君于天子,治天子者不于天子于欲, (3) 治欲者不于欲于性。性者,万物之本也,不可长,不可短,因其固然而然之,此天地之数也。窥赤肉而鸟鹊聚,狸处堂而众鼠散, (4) 衰绖陈而民知丧,竽瑟陈而民知乐,汤、武修其行而天下从, (5) 桀、纣慢其行而天下畔, (6) 岂待其言哉?君子审在己者而已矣。

(1) 繇,用也。

(2) 治,饬也。君,侯也。

(3) 欲,贪欲也。不贪欲则天子安乐也。

(4) 窥,见也。散,走也。

(5) 修其仁义之行,故天下顺从之也。

(6) 慢,易也。

荆有善相人者,所言无遗策, (1) 闻于国。 (2) 庄王见而问焉,对曰:“臣非能相人也,能观人之友也。观布衣也,其友皆孝悌纯谨畏令,如此者,其家必日益, (3) 身必日荣矣,所谓吉人也。观事君者也,其友皆诚信有行好善,如此者,事君日益,官职日进,此所谓吉臣也。 (4) 观人主也,其朝臣多贤,左右多忠,主有失,皆交争证谏, (5) 如此者,国日安,主日尊,天下日服, (6) 此所谓吉主也。臣非能相人也,能观人之友也。”庄王善之,于是疾收士,日夜不懈,遂霸天下。故贤主之时见文艺之人也,非特具之而已也,所以就大务也。 (7) 夫事无大小,固相与通。田猎驰骋,弋射走狗,贤者非不为也,为之而智日得焉,不肖主为之而智日惑焉。志曰:“骄惑之事,不亡奚待?” (8)

(1) 遗犹失也。

(2) 国人闻之也。

(3) 益,富也。

(4) 吉,善也。

(5) 交,俱也。

【校】《外传》九、《新序》五作“正谏”。案:证亦谏也,见《说文》。

(6) 服其德也。

(7) 就,成也。

(8) 志,古记也。

齐人有好猎者, (1) 旷日持久而不得兽,入则愧其家室,出则愧其知友州里。惟其所以不得之故,则狗恶也。欲得良狗,则家贫无以。 (2) 于是还疾耕,疾耕则家富,家富则有以求良狗,狗良则数得兽矣,田猎之获常过人矣。 (3) 非独猎也,百事也尽然。霸王有不先耕而成霸王者,古今无有。此贤者不肖之所以殊也。 (4) 贤不肖之所欲与人同,尧、桀、幽、厉皆然,所以为之异。故贤主察之,以为不可,弗为;以为可,故为之。为之必繇其道,物莫之能害,此功之所以相万也。 (5)

(1) 【校】“齐人”,旧本或作“君”,或作“尹”,皆讹,今从《日抄》改正。孙云:“《御览》八百三十二,又九百五并作‘齐’字。”

(2) 无以买狗。

(3) 过犹多也。

(4) 殊,异也。

(5) 万倍也。





第二十五卷 似顺论



似顺


一曰:

事多似倒而顺,多似顺而倒。 (1) 有知顺之为倒、倒之为顺者,则可与言化矣。 (2) 至长反短,至短反长,天之道也。 (3)

(1) 倒,逆也。

(2) 化,道也。

(3) 夏至极长,过至则短,故曰“至长反短”;冬至极短,过至则长,故曰“至短反长”也。天道有盈缩之数,故曰“天之道也”。

荆庄王欲伐陈, (1) 使人视之。使者曰:“陈不可伐也。”庄王曰:“何故?”对曰:“城郭高,沟洫深,蓄积多也。”宁国曰:“陈可伐也。 (2) 夫陈,小国也,而蓄积多,赋敛重也,则民怨上矣;城郭高,沟洫深,则民力罢矣。兴兵伐之,陈可取也。”庄王听之,遂取陈焉。 (3)

(1) 庄王,楚穆王之子也。

(2) 宁国,楚臣。

【校】《说苑·权谋》篇“蓄积多”下云“其国宁也。王曰:陈可伐也”,后“庄王听之”作“兴兵伐之”。

(3) 《传》曰“伐而言取,易也。”

【校】注“传曰”旧作“陈曰”,讹,今改正。

田成子之所以得有国至今者,有兄曰完子,仁且有勇。 (1) 越人兴师诛田成子,曰:“奚故杀君而取国?” (2) 田成子患之。完子请率士大夫以逆越师,请必战,战请必败,败请必死。田成子曰:“夫必与越战可也。战必败,败必死,寡人疑焉。” (3) 完子曰:“君之有国也,百姓怨上,贤良又有死之,臣蒙耻。以完观之也,国已惧矣。今越人起师,臣与之战,战而败,贤良尽死,不死者不敢入于国。君与诸孤处于国,以臣观之,国必安矣。”完子行,田成子泣而遣之。夫死败,人之所恶也,而反以为安,岂一道哉?故人主之听者与士之学者,不可不博。 (4)

(1) 成子,田常也。有国,齐国也。

(2) 杀君,杀齐简公而取其国也。

(3) 疑焉,不欲其死也。

(4) 听博则达义,学博则达道也。

尹铎为晋阳,下,有请于赵简子。 (1) 简子曰:“往而夷夫垒。我将往,往而见垒,是见中行寅与范吉射也。” (2) 铎往而增之。 (3) 简子上之晋阳,望见垒而怒曰:“嘻!铎也欺我。”于是乃舍于郊,将使人诛铎也。孙明进谏曰:“以臣私之,铎可赏也。 (4) 铎之言固曰:‘见乐则淫侈,见忧则诤治,此人之道也。今君见垒念忧患,而况群臣与民乎?夫便国而利于主,虽兼于罪,铎为之。 (5) 夫顺令以取容者,众能之,而况铎欤?’ (6) 君其图之。” (7) 简子曰:“微子之言,寡人几过。” (8) 于是乃以免难之赏赏尹铎。人主太上喜怒必循理, (9) 其次不循理必数更,虽未至大贤,犹足以盖浊世矣, (10) 简子当此。 (11) 世主之患,耻不知而矜自用,好愎过而恶听谏, (12) 以至于危。耻无大乎危者。 (13)

(1) 尹铎者,赵简子家臣也。晋阳,简子邑。为,治也。

(2) 夷,平也。中行文子与范昭子专晋君权,伐赵简子,围之晋阳,所作垒壁培堙也。简子不欲见之,故使尹铎平除之也。

【校】《晋语》九“垒”下有“培”字,观此注似亦本有“培”字。又“是”字下旧本脱“见”字,据《晋语》补。

(3) 增益其垒璧令高大也。

(4) 孙明,简子臣孙无政邮良也。私,惟也。

【校】《晋语》邮无正字伯乐,《左传》邮无恤亦名邮良,即王良也。此云孙明,当即孙阳。注云孙无政,亦见前。

(5) 【校】旧注云:“‘兼’或作‘谦’。”疑亦校者之辞。“谦”字无义,或当为“嫌”。

(6) 容,说也。况铎为贤人也。

(7) 图,议之也。

(8) 过,失也。

(9) 太上,上德之君。

(10) 更,革也。变革不循危亡之迹,虽未至大贤,尚足以盖浊世专欲之人也。

(11) 简子之行与此相值也。

(12) 鄙耻于不知而矜大于自用,愎过恶谏,固败是求,世主之大病也。

【校】注旧本缺“求”字,案“固败是求”见《左传》庆郑语,此用其成文,今补。

(13) 危败则灭亡,耻但惭辱耳,故无大于危者也。





别类


二曰:

知不知,上矣。过者之患,不知而自以为知。物多类然而不然,故亡国僇民无已。夫草有莘有藟, (1) 独食之则杀人,合而食之则益寿。 (2) 万堇不杀。 (3) 漆淖水淖, (4) 合两淖则为蹇, (5) 湿之则为干。 (6) 金柔锡柔,合两柔则为刚,燔之则为淖。 (7) 或湿而干,或燔而淖,类固不必可推知也。 (8) 小方,大方之类也;小马,大马之类也;小智,非大智之类也。 (9)

(1) 【校】《御览》九百九十四“莘”作“华”,《日抄》作“萃”。

(2) 合药而服,愈人病,故曰益人寿也。

(3) 【校】堇,乌头也,毒药,能杀人。万堇则不能杀,未详。

(4) 【校】“水”下旧无“淖”字,今案文义补。

(5) 蹇,强也。言水漆相得则强而坚也。

(6) 干,燥也。

(7) 火炽金流,故为淖也。

(8) 漆得湿而干燥,金遇燔而流淖,皆非其类也,故曰“不必可推知也”。

(9) 大智知人所不知,见一隅则以三隅反,小智闻十裁通其一,故不可以为类也。

鲁人有公孙绰者,告人曰:“我能起死人。” (1) 人问其故,对曰:“我固能治偏枯, (2) 今吾倍所以为偏枯之药,则可以起死人矣。”物固有可以为小,不可以为大;可以为半,不可以为全者也。 (3)

(1) 《淮南记》曰“王孙绰”。

【校】见《淮南·览冥训》。彼注云:“盖周人,一曰卫人。王孙贾之后也。”

(2) 【校】旧校云:“‘治’一作‘为’。”为亦治也。

(3) 半谓偏枯,全谓死人也。

【校】梁仲子云:“小、大、半、全,乃概论物情。注太泥。”

相剑者曰:“白所以为坚也,黄所以为牣也, (1) 黄白杂则坚且牣,良剑也。”难者曰:“白所以为不牣也,黄所以为不坚也,黄白杂则不坚且不牣也。又柔则錈, (2) 坚则折。剑折且錈,焉得为利剑?”剑之情未革,而或以为良,或以为恶,说使之也。故有以聪明听说,则妄说者止;无以聪明听说,则尧、桀无别矣。 (3) 此忠臣之所患也, (4) 贤者之所以废也。 (5)

(1) 【校】“牣”与“韧”、“忍”、“刃”、“纫”古皆通用。李善注《王文宪集序》引作“纫”。

(2) 【校】字书无此字,当与“卷”同。

(3) 无聪明以听说,不能知贤不肖,故尧、桀无有所别也。

(4) 患,忧也。

(5) 不见别白黑,故废弃也。

义,小为之则小有福,大为之则大有福。于祸则不然,小有之不若其亡也。 (1) 射招者欲其中小也,射兽者欲其中大也。物固不必,安可推也? (2)

(1) 祸虽微小,积小成大,以危身亡国,故曰小有之不若无也。

(2) 招,埻艺也。中小,谓剖微不失毫分,射之工也。射兽欲其中大者,得肉多,故以中为工也。射则同也,中之小大异,故曰“物固不必,安可推也”。

高阳应将为室,家匠对曰:“未可也。木尚生,加涂其上,必将挠。 (1) 以生为室,今虽善,后将必败。” (2) 高阳应曰:“缘子之言,则室不败也。木益枯则劲, (3) 涂益干则轻,以益劲任益轻则不败。” (4) 匠人无辞而对,受令而为之。室之始成也善,其后果败。高阳应好小察,而不通乎大理也。

(1) 高阳,宋邑,因以为氏。应,名也。或作“高魋”。宋大夫也。家匠,家臣也。挠,弱曲也,故曰“未可也”。

【校】梁仲子云:“《淮南·人间训》作‘高阳魋’,《广韵》‘阳’字下引《吕氏》有辩士‘高阳魋’,此注内脱一‘阳’字。”

(2) 家臣所谓,直于辞而合事实者也。

(3) 劲,强也。

(4) 此俛于辞,而后必败,其言不合事实者也。

【校】俛当是勉强之义。

骥骜绿耳背日而西走,至乎夕则日在其前矣。 (1) 目固有不见也,智固有不知也,数固有不及也。不知其说所以然而然,圣人因而兴制,不事心焉。

(1) 日东行,天西旋。日行迟,天旋疾。及夕,日入于虞渊之北,骥不能及,故曰在前矣。

【校】注说迂曲。





有度


三曰:

贤主有度而听,故不过。 (1) 有度而以听,则不可欺矣, (2) 不可惶矣,不可恐矣,不可喜矣。以凡人之知,不昏乎其所已知,而昏乎其所未知, (3) 则人之易欺矣,可惶矣,可恐矣,可喜矣,知之不审也。

(1) 度,法也。

(2) 欺,误也。

(3) 昏,暗也。

客有问季子曰:“奚以知舜之能也?” (1) 季子曰:“尧固已治天下矣,舜言治天下而合己之符, (2) 是以知其能也。”“若虽知之,奚道知其不为私?” (3) 季子曰:“诸能治天下者,固必通乎性命之情者,当无私矣。夏不衣裘,非爱裘也,暖有余也。冬不用 , (4) 非爱 也,清有余也。 (5) 圣人之不为私也,非爱费也,节乎己也。 (6) 节己,虽贪污之心犹若止,又况乎圣人?”

(1) 季子,户季子,尧时诸侯也。

(2) 己,尧也。

(3) 私,邪也。

【校】此二句,客又问也。

(4) ,扇也。

【校】“ ”与“箑”同。

(5) 清,寒。

(6) 【校】“费”,旧本误作“贵”,孙云:“《重己》篇云‘非好俭而恶费也,节乎性也’,与此正相同。《御览》四百二十九亦作‘费’。”今改正。

许由非强也,有所乎通也, (1) 有所通则贪污之利外矣。 (2) 孔、墨之弟子徒属充满天下,皆以仁义之术教导于天下,然而无所行。教者术犹不能行,又况乎所教? (3) 是何也?仁义之术外也。夫以外胜内,匹夫徒步不能行,又况乎人主? (4) 唯通乎性命之情,而仁义之术自行矣。

(1) 通于无为也。

(2) 外,弃也。

(3) 所教,谓孔、墨弟子之弟子也。

(4) 人主,谓俗主,又不能行也。

先王不能尽知,执一而万物治。 (1) 使人不能执一者,物感之也。 (2) 故曰:通意之悖,解心之缪,去德之累,通道之塞。 (3) 贵、富、显、严、名、利六者,悖意者也。 (4) 容、动、色、理、气、意六者,缪心者也。 (5) 恶、欲、喜、怒、哀、乐六者,累德者也。 (6) 智、能、去、就、取、舍六者,塞道者也。 (7) 此四六者不荡乎胸中则正, (8) 正则静,静则清明,清明则虚,虚则无为而无不为也。 (9)

(1) 不能尽知万物也,执守一道而万物治理矣。

(2) 感,惑也。

(3) 悖、缪、累、塞四者所以为人病也,唯执一者能解去道之塞,不壅闭也。

(4) 此六者人情所欲也。孔子曰“富与贵,人之所欲也,不以其道,得之不居”,故曰“悖意”。悖,乱也。

【校】案:古读皆以“不以其道”为句,此注亦当尔。《论语》“不处”,此作“不居”,《论衡》《问孔》、《刺孟》两篇并同。

(5) 此六者不节,所以惑人心者也。

(6) 此六者不节,所以为德累者也。

(7) 此六者宜适难中,所以窒塞道,使不通者也。

(8) 荡,动也。此四六者皆得其适,不倾邪荡动于胸臆之中则正矣。《诗》云:“静恭尔位,正直是与。”此之谓也。

(9) 虚者道也。道尚空虚,无为而无不为,人能行之亦无不为也。





分职


四曰:

先王用非其有如己有之, (1) 通乎君道者也。 (2) 夫君也者,处虚素服而无智,故能使众智也;智反无能,故能使众能也;能执无为,故能使众为也。无智、无能、无为,此君之所执也。 (3) 人主之所惑者则不然,以其智强智,以其能强能,以其为强为,此处人臣之职也。处人臣之职而欲无壅塞,虽舜不能为。 (4)

(1) 【校】孙云:“《御览》六百二十作‘如己之有’。案下文皆作‘如己有之’,《御览》非也。”

(2) 桀、纣有天下,非汤、武之有也,而汤、武有之,此之类也,故曰“通乎君道者也”。

(3) 君执一以为化之也。

【校】注“之”字疑衍。

(4) 若此者,虽舜之圣不能无壅塞,况惑主乎?

武王之佐五人。 (1) 武王之于五人者之事无能也,然而世皆曰取天下者武王也。故武王取非其有如己有之,通乎君道也。通乎君道,则能令智者谋矣,能令勇者怒矣,能令辩者语矣。夫马者,伯乐相之,造父御之, (2) 贤主乘之,一日千里。无御相之劳而有其功,则知所乘矣。 (3)

(1) 五人者,周公旦、召公奭、太公望、毕公高、苏公忿生也。

(2) 伯乐善相马,秦缪公臣也。造父,嬴姓,飞廉之子,善御,周穆王臣也。

(3) 功,千里之功也,故曰知乘也。

今召客者,酒酣, (1) 歌舞鼓瑟吹竽,明日不拜乐己者, (2) 而拜主人,主人使之也。先王之立功名有似于此, (3) 使众能与众贤,功名大立于世,不予佐之者,而予其主,其主使之也。 (4) 譬之若为宫室,必任巧匠,奚故? (5) 曰:匠不巧则宫室不善。夫国,重物也,其不善也岂特宫室哉? (6) 巧匠为宫室,为圆必以规,为方必以矩,为平直必以准绳。功已就, (7) 不知规矩绳墨,而赏匠巧匠之。宫室已成, (8) 不知巧匠,而皆曰:“善,此某君某王之宫室也。”此不可不察也。 (9) 人主之不通主道者则不然,自为人则不能, (10) 任贤者则恶之,与不肖者议之。此功名之所以伤, (11) 国家之所以危。 (12)

(1) 召,请也。饮酒合乐为酣。

(2) 拜,谢也。乐己者,谓倡优也。

(3) 有似于主人使之者也。

(4) 【校】“其主”二字旧本不重,今据《困学纪闻》十所引补。

(5) 奚,何也。

(6) 特犹直也。

(7) 就,成也。

【校】李本作“准”,别本作“準”。

(8) 【校】《困学纪闻》“赏匠巧”下有“也”字,又有“巧”字。卢云:“案‘也’字当有,下‘匠之’二字系衍文,当删。”

(9) 察犹知也。

(10) 【校】“自为人”疑是“自为之”。

(11) 伤,败也。

(12) 危,亡也。

枣,棘之有;裘,狐之有也。食棘之枣,衣狐之皮,先王固用非其有而己有之。汤、武一日而尽有夏、商之民,尽有夏、商之地,尽有夏、商之财。以其民安,而天下莫敢之危; (1) 以其地封,而天下莫敢不说;以其财赏,而天下皆竞。 (2) 无费乎 与岐周,而天下称大仁,称大义,通乎用非其有。 (3)

(1) 【校】“敢之”二字似当乙转。

(2) 竞,进也。

(3) 通,达也。

白公胜得荆国, (1) 不能以其府库分人。七日,石乞曰:“患至矣。 (2) 不能分人则焚之,毋令人以害我。”白公又不能。 (3) 九日,叶公入,乃发太府之货予众, (4) 出高库之兵以赋民, (5) 因攻之。十有九日而白公死。国非其有也而欲有之,可谓至贪矣。不能为人,又不能自为,可谓至愚矣。譬白公之啬,若枭之爱其子也。 (6)

(1) 杀令尹子西、司马子期而得荆国也。

(2) 石乞,白公臣也。

(3) 不能焚之也。

(4) 叶公,楚叶县大夫沈诸梁子高也。

(5) 赋,予也。

(6) 枭爱养其子,子长而食其母也。白公爱荆国之财而杀其身也。

卫灵公天寒凿池。 (1) 宛春谏曰:“天寒起役,恐伤民。” (2) 公曰:“天寒乎?”宛春曰:“公衣狐裘,坐熊席,陬隅有灶, (3) 是以不寒。今民衣弊不补,履决不组。 (4) 君则不寒矣,民则寒矣。”公曰:“善。”令罢役。左右以谏曰:“君凿池,不知天之寒也,而春也知之。以春之知之也而令罢之,福将归于春也, (5) 而怨将归于君。”公曰:“不然。夫春也,鲁国之匹夫也,而我举之, (6) 夫民未有见焉, (7) 今将令民以此见之。曰春也有善,于寡人有也, (8) 春之善非寡人之善欤?”灵公之论宛春,可谓知君道矣。君者固无任,而以职受任。工拙,下也;赏罚,法也;君奚事哉?若是则受赏者无德,而抵诛者无怨矣,人自反而已。此治之至也。 (9)

(1) 灵公,襄公之子。

(2) 伤,病也。

(3) 【校】《新序·刺奢》篇“陬隅”作“隩隅”。

(4) 【校】《新序》作“苴”。

(5) 【校】《新序》“福”作“德”,《御览》三十四同。

(6) 举,用也。

(7) 未见其德。

(8) 【校】“曰”,《新序》作“且”。

(9) 抵,当也。





处方


五曰:

凡为治必先定分,君臣父子夫妇。君臣父子夫妇六者当位,则下不逾节而上不苟为矣,少不悍辟而长不简慢矣。 (1) 金木异任,水火殊事?阴阳不同,其为民利一也。 (2) 故异所以安同也,同所以危异也。 (3) 同异之分,贵贱之别,长少之义。此先王之所慎,而治乱之纪也。 (4)

(1) 悍,凶也。辟,邪也。简,惰也。慢,易也。

(2) 六者皆所以为民用,故曰“为民利一也”。

(3) 言同异更相成。

(4) 圣人以治,乱人以乱,在所以由之也。

今夫射者仪豪而失墙, (1) 画者仪发而易貌, (2) 言审本也。 (3) 本不审,虽尧、舜不能以治。 (4) 故凡乱也者,必始乎近而后及远,必始乎本而后及末。 (5) 治亦然。 (6) 故百里奚处乎虞而虞亡,处乎秦而秦霸; (7) 向挚处乎商而商灭,处乎周而周王。 (8) 百里奚之处乎虞,智非愚也;向挚之处乎商,典非恶也;无其本也。 (9) 其处于秦也,智非加益也;其处于周也,典非加善也;有其本也。 (10) 其本也者,定分之谓也。 (11)

(1) 仪,望也。睎望毫毛之微,而不视堵墙之大,故能中也。

(2) 画者睎毫发,写人貌,仪之于象,不失其形,故曰“易貌”也。

(3) 射必能中,画必象人,故曰“审本”。

(4) 本,身;审,正也。身不正而欲治者,尧、舜且犹不能,况凡人乎?

(5) 近喻小,远喻大也。为乱之君先小后大也。本谓身,末谓国也。詹何曰“未闻身乱而国治也”,故曰“始乎本而后及末”。

(6) 未闻身治而国乱也,故曰“亦然”。

(7) 虞公贪璧马之赂,不从其言,为晋所灭,故亡也。秦缪公用其谋而兼西戎,故霸也。

(8) 向挚,纣之太史令也。纣不从其言而奔周,期年而纣灭,周武王用其谋而王天下也。

(9) 本谓虞、商之君。身不治,自取灭亡也。

(10) 有其本,言秦、周之君身正而治也。

(11) 言其为君治理分定,不悖惑也。

齐令章子将而与韩、魏攻荆,荆令唐蔑将而应之。 (1) 军相当,六月而不战。齐令周最趣章子急战,其辞甚刻。 (2) 章子对周最曰:“杀之免之,残其家,王能得此于臣。不可以战而战,可以战而不战,王不能得此于臣。”与荆人夹沘水而军, (3) 章子令人视水可绝者,荆人射之,水不可得近。 (4) 有刍水旁者,告齐候者 (5) 曰:“水浅深易知。荆人所盛守,尽其浅者也;所简守,皆其深者也。”候者载刍者与见章子。章子甚喜,因练卒以夜奄荆人之所盛守,果杀唐蔑。章子可谓知将分矣。

(1) 应,击也。

【校】“唐蔑”,《楚世家》作“唐眛”。“应之”旧作“拒之”,注“‘拒’一作‘应’”,梁仲子云“《水经·沘水注》引作‘荆使唐蔑应之’”,则“应”字正是本文,今改正。

(2) 趣,督也。刻亦急也。

(3) 【校】“沘”旧作“泚”,梁仲子云:“旧本《水经》‘泚水’,何氏焯改作‘沘水’,注引此文。新校本从《汉·地理志》改作‘比水’,引此作‘夹比而军’。”

(4) 近犹迫也。

(5) 候,视也。

韩昭釐侯出弋,靷偏缓。 (1) 昭釐侯居车上,谓其仆:“靷不偏缓乎?”其仆曰:“然。”至,舍,昭釐侯射鸟,其右摄其一靷适之。 (2) 昭釐侯已射,驾而归。上车,选间 (3) 曰:“乡者靷偏缓,今适,何也?”其右从后对曰:“今者臣适之。”昭釐侯至,诘车令, (4) 各避舍。 (5) 故擅为妄意之道虽当,贤主不由也。 (6)

(1) 弋,猎也。《论语》曰“戈不射宿”。

(2) 适犹等也。

(3) 选间犹选顷也。

(4) 诘,让也。

(5) 【校】句上似当有“与右”二字。

(6) 由,用也。

今有人于此,擅矫行则免国家,利轻重则若衡石,为方圜则若规矩,此则工矣巧矣,而不足法。 (1) 法也者,众之所同也,贤不肖之所以其力也。 (2) 谋出乎不可用, (3) 事出乎不可同,此为先王之所舍也。 (4)

(1) 巧而不足法者,以其不循规矩故也。

(2) 【校】“其力”疑当作“共力”。

(3) 【校】旧校云:“一作‘行’。”

(4) 舍而不为也。





慎小


六曰:

上尊下卑。卑则不得以小观上。 (1) 尊则恣,恣则轻小物, (2) 轻小物则上无道知下,下无道知上,上下不相知则上非下,下怨上矣。人臣之情,不能为所怨; (3) 人主之情,不能爱所非。 (4) 此上下大相失道也。故贤主谨小物以论好恶。 (5) 巨防容蝼而漂邑杀人, (6) 突泄一熛而焚宫烧积, (7) 将失一令而军破身死, (8) 主过一言而国残名辱,为后世笑。 (9)

(1) 观,视也。上,君也。

(2) 小物,凡小事也。

(3) 不能为之竭力尽节也。

(4) 方非罪之,何能爱也?

(5) 好,善也。恶,恶也。

(6) 巨,大;防,堤也。如堤有孔穴容蝼蛄,则溃漏窍决,至于漂没闾邑,溺杀人民也。

(7) 灶突烟泄出,则火滥炎上,烧人之宫室积委也。

【校】“突”亦作“堗”。《广雅》:“灶窗谓之堗。”或谓“突”当作“ ”。案《说文》:“ ,深也,一曰灶突。”然则 特灶突之一名,《说文》亦但云“一曰灶 ”,不云“灶 ”,何得以“突”为“ ”之误?故今仍作“突”字。又“熛”旧本讹作“烟”,今从《日抄》改正。

(8) 教令不当为失。失令不从,士无先登之心,而怀奔北之志,故军破败,将见禽获而身死也。

(9) 主过一言犹将失一令,故国残亡,恶名著闻,以自污辱,乃为后世之人所非笑也。

卫献公戒孙林父、甯殖食。 (1) 鸿集于囿,虞人以告, (2) 公如囿射鸿。二子待君,日晏,公不来至。 (3) 来,不释皮冠而见二子。二子不说,逐献公,立公子黚。 (4)

(1) 林父,孙文子也。甯殖,惠子也。

(2) 畜禽兽大曰苑,小曰囿。虞人,主囿之官也。以告,以鸿告也。

(3) 晏,暮也。

(4) 《传》曰:“卫人立公孙剽,孙林父、甯殖相之。”此云立公子黚,复误矣。案《卫世家》,公子黚乃灵公之子,太子蒯聩之弟也,是为悼公,于献公为曾孙也,焉得立之乎?

卫庄公立,欲逐石圃。 (1) 登台以望,见戎州而问之曰:“是何为者也?”侍者曰:“戎州也。” (2) 庄公曰:“我姬姓也,戎人安敢居国?”使夺之宅,残其州。晋人适攻卫,戎州人因与石圃杀庄公,立公子起。 (3) 此小物不审也。 (4) 人之情不蹶于山, (5) 而蹶于垤。 (6)

(1) 庄公,灵公之子蒯聩也。石圃,卫卿石恶之子也。蒯聩在外,圃不欲纳之,故立而逐之也。

(2) 戎州,戎之邑也。

(3) 公子起,卫灵公子,庄公之弟也。

(4) 审,慎也。

(5) 蹶,踬颠顿也。

(6) 垤,蚁封也。蚁封卑小,人轻之,故踬颠也。

齐桓公即位,三年三言,而天下称贤,群臣皆说。去肉食之兽,去食粟之鸟,去丝罝之网。 (1)

(1) 是三言也。

吴起治西河,欲谕其信于民, (1) 夜日置表于南门之外, (2) 令于邑中曰:“明日有人能偾南门之外表者,仕长大夫。” (3) 明日日晏矣,莫有偾表者。 (4) 民相谓曰:“此必不信。” (5) 有一人曰:“试往偾表,不得赏而已,何伤?” (6) 往偾表,来谒吴起。 (7) 吴起自见而出,仕之长大夫。夜日又复立表,又令于邑中如前。邑人守门争表,表加植,不得所赏。 (8) 自是之后,民信吴起之赏罚。 (9) 赏罚信乎民,何事而不成,岂独兵乎? (10)

(1) 吴起,卫人也,为魏武侯西河守。谕,明也。

(2) 置,立也。表,柱也。

(3) 偾,僵也。长大夫,上大夫也。

【校】“能”字旧本缺,孙据《纪闻》十补,《御览》四百三十同。

(4) 莫,无也。

(5) 不信其有赏也。

(6) 言不敢必得其赏也。

【校】“而已”,《纪闻》作“则已”。言纵不得赏,非有害也。注不得解。

(7) 谒,告也。

(8) 如前,与前令同也。邑人贪赏,争往偾表,表深植而不能偾,不得其所赏也。

(9) 吴起赏罚不欺民,民信之也。

(10) 言非独信用兵以成功也,大信用赏罚以成事,故使秦人不敢东向犯盗西河也。

【校】旧校云:“‘岂独兵乎’一作‘非独兵也’。”案:注“大”,刘本作“亦”。





第二十六卷 士容论



士容


一曰:

士不偏不党,柔而坚,虚而实。 (1) 其状朖然不儇,若失其一。 (2) 傲小物而志属于大, (3) 似无勇而未可恐狼, (4) 执固横敢而不可辱害, (5) 临患涉难而处义不越, (6) 南面称寡而不以侈大, (7) 今日君民而欲服海外,节物甚高而细利弗赖, (8) 耳目遗俗而可与定世, (9) 富贵弗就而贫贱弗朅, (10) 德行尊理而羞用巧卫, (11) 宽裕不訾而中心甚厉, (12) 难动以物而必不妄折, (13) 此国士之容也。 (14)

(1) 而,能也。

(2) 一谓道也。能柔坚虚实之士,其状貌朖然舒大,不儇给巧伪为之,畏失其道也。

(3) 傲,轻也。轻略丛脞翳蔑之事,而志属连于有大成功也。

(4) 未可恐以非义之事也。

(5) 狼,贪兽也。所搏执坚固。横犹勇敢。之士若此者不可辱,亦不可害也。

【校】注“犹”疑“犷”。

(6) 越,失也。

(7) 南面,君位也。孤寡,谦称也。士之如此者,使即南面之君位,亦处义而已,不以奢侈广大也。

【校】注“位”字阙,今案文义补。

(8) 海外,四海之外。而欲服之,化广大也。节物,事也。行事甚高,细小之利不恃赖之也。

(9) 耳目视听,礼义是则,故能遗弃流俗,可与大定于一世也。

(10) 轻富贵,甘贫贱。

【校】朅,去也。宋玉《九辩》云“车既驾兮朅而归”。

(11) 尊重道理而行,羞以巧媚自荣卫也。

【校】注“荣”疑“营”。

(12) 不訾,毁败人也。甚厉,至高远也。

(13) 不为物动,唯义所在,不妄屈折也。

(14) 容犹法也。

齐有善相狗者,其邻假以买取鼠之狗。 (1) 期年乃得之,曰:“是良狗也。”其邻畜之数年,而不取鼠,以告相者。相者曰:“此良狗也。其志在獐麋豕鹿,不在鼠。欲其取鼠也则桎之。”其邻桎其后足, (2) 狗乃取鼠。 (3) 夫骥骜之气,鸿鹄之志,有谕乎人心者,诚也。人亦然,诚有之则神应乎人矣。言岂足以谕之哉?此谓不言之言也。 (4)

(1) 假犹请也。请善相狗者买取鼠之狗也。

【校】旧校云:“一本作‘其邻借之买鼠狗’。”借犹请也。今案:《御览》九百五作“其邻藉之买鼠狗”,则当作“藉”字。

(2) 桎,械也。著足曰桎,著手曰梏。

(3) 【校】旧校云:“一本作‘狗则取鼠矣’。”

(4) 不言之言,以道化也。

客有见田骈者, (1) 被服中法,进退中度,趋翔闲雅,辞令逊敏。 (2) 田骈听之毕而辞之。 (3) 客出,田骈送之以目。 (4) 弟子谓田骈曰:“客士欤?”田骈曰:“殆乎非士也。 (5) 今者客所弇敛,士所术施也;士所弇敛,客所术施也。 (6) 客殆乎非士也。”故火烛一隅,则室偏无光; (7) 骨节早成,空窍哭历,身必不长; (8) 众无谋方,乞谨视见,多故不良; (9) 志必不公, (10) 不能立功; (11) 好得恶予,国虽大不为王, (12) 祸灾日至。故君子之容,纯乎其若钟山之玉,桔乎其若陵上之木, (13) 淳淳乎慎谨畏化而不肯自足, (14) 乾乾乎取舍不悦而心甚素朴。 (15)

(1) 田骈,齐人也,作《道书》二十五篇。

(2) 逊,顺也。敏,材也。

(3) 辞,遣也。

(4) 以目送而视之也。

(5) 殆,近也。

(6) 【校】旧校云:“‘术’皆当作‘述’。”今案:古亦通用。

(7) 烛,照也。偏,半也。

(8) 长,大也。

(9) 良,善也。

(10) 公,正也。

(11) 立,成也。

(12) 好得,厚敛也。恶予,恡啬也。多藏厚亡,故必不为王。

(13) 纯,美也。钟山之玉,燔以炉炭,三日三夜,色泽不变。陵上之木鸿且大,皆天性也。君子天性纯敏,故以此为喻也。

(14) 化,教也。常畏而奉之,不肯自足。其智思以事,必问详而后行之也。

(15) 乾乾,进不倦也。取舍不悦,常敬慎也。心甚素朴,精洁专一,情不散欲也。

唐尚敌年为史, (1) 其故人谓唐尚愿之, (2) 以谓唐尚。唐尚曰:“吾非不得为史也,羞而不为也。”其故人不信也。 (3) 及魏围邯郸,唐尚说惠王而解之围,以与伯阳, (4) 其故人乃信其羞为史也。居有间,其故人为其兄请。 (5) 唐尚曰:“卫君死,吾将汝兄以代之。”其故人反兴再拜而信之。夫可信而不信,不可信而信,此愚者之患也。 (6) 知人情不能自遗,以此为君,虽有天下何益? (7) 故败莫大于愚。愚之患,在必自用,自用则戆陋之人从而贺之。有国若此,不若无有。古之与贤从此生矣。 (8) 非恶其子孙也,非徼而矜其名也,反其实也。 (9)

(1) 史,国史也。

(2) 故人者,唐尚知旧也。以唐尚明习天文宿度,审咎徵之应,故为愿之也。

(3) 不信其羞为史。

(4) 惠王,魏文侯之孙,武侯之子,孟子所见梁惠王也。解邯郸围也。以与伯阳,以伯阳邑资之也。

(5) 请于唐尚,欲仕其兄。

(6) 可信,谓唐尚羞为史。不可信,谓唐尚欲以其兄代卫君。卫君不可得也,而信为可得,故曰“不可信而信也”。患者,犹病也。

(7) 不能自遗亡其贪欲之情,必危亡也,故曰“虽有天下何益”。

(8) 古人传位于贤,以子不肖,不可予也。

(9) 徼,求也。矜,大也。以国予贤则兴,子孙不肖,予其国必灭亡,故曰“反其实也”。





务大


二曰:

尝试观于上志, (1) 三王之佐,其名无不荣者, (2) 其实无不安者,功大故也。 (3) 俗主之佐,其欲名实也与三王之佐同, (4) 其名无不辱者,其实无不危者,无功故也。 (5) 皆患其身不贵于其国也,而不患其主之不贵于天下也,此所以欲荣而逾辱也, (6) 欲安而逾危也。

(1) 上志,古记也。

(2) 荣,显也。

(3) 实犹终也。

(4) 同,等也。

(5) 无大功故也。

(6) 逾,益也。

孔子曰:燕爵争善处于一屋之下,母子相哺也,区区焉相乐也, (1) 自以为安矣。灶突决,上栋焚,燕爵颜色不变,是何也?不知祸之将及之也。不亦愚乎? (2) 为人臣而免于燕爵之智者寡矣。夫为人臣者,进其爵禄富贵,父子兄弟相与比周于一国,区区焉相乐也。而以危其社稷,其为灶突近矣,而终不知也,其与燕爵之智不异。故曰:天下大乱,无有安国;一国尽乱,无有安家;一家尽乱,无有安身。此之谓也。故细之安, (3) 必待大;大之安,必待小。 (4) 细大贱贵,交相为赞, (5) 然后皆得其所乐。 (6)

(1) 区区,得志貌也。

【校】“区区”当作“呕呕”,下同。前《谕大》篇作“姁姁”。

(2) 【校】“及之”当作“及己”。

(3) 细,小也。

(4) 言相须也。

【校】两“待”字,前《谕大》篇俱作“恃”,下“赞”字亦作“恃”。

(5) 交,更也。赞,助也。

(6) 乐,愿也。

薄疑说卫嗣君以王术, (1) 嗣君应之曰:“所有者千乘也,愿以受教。” (2) 薄疑对曰:“乌获奉千钧,又况一斤?” (3) 杜赫以安天下说周昭文君, (4) 昭文君谓杜赫曰:“愿学所以安周。” (5) 杜赫对曰:“臣之所言者不可,则不能安周矣;臣之所言者可,则周自安矣。” (6) 此所谓以弗安而安者也。 (7)

(1) 嗣君,卫平侯之子也,秦贬其号曰君。

(2) 卫君国之赋,兵车千乘耳。王者万乘,故愿以受教也。

【校】案:《淮南·道应训》“所有”上有“予”字。此注非是。愿以受教者,愿以千乘之国受教也。薄疑之对,以千钧谕王术,一斤喻治国。言王术可为,于治国乎何有?注皆不得本意。

(3) 千钧,三万斤也。喻卫君之贤,为王术,若乌获之力以举一斤,言其易也。

【校】《淮南》“奉”作“举”。

(4) 杜赫,周人杜伯之后也。周昭文君,周分为二,东周之君也。

(5) 以,用也。

(6) 所言安行仁与义也。

(7) 当昭文君时,人不安行仁义,而仁义不行也。然仁义,必安之本也,故曰“以弗安而安者也”。

郑君问于被瞻曰:“闻先生之义,不死君,不亡君,信有之乎?” (1) 被瞻对曰:“有之。夫言不听,道不行,则固不事君也。若言听道行,又何死亡哉?” (2) 故被瞻之不死亡也,贤乎其死亡者也。 (3)

(1) 郑君,穆公也。被瞻事郑文公,故穆公即位,问瞻所行之义信不乎?

(2) 言从贤臣之言,不死亡也。

(3) 使君无道,臣不能正,乃死亡耳。被瞻言听道行,不死不亡,故曰“贤乎死亡者也”。

昔有舜欲服海外而不成,既足以成帝矣。禹欲帝而不成,既足以王海内矣。汤、武欲继禹而不成,既足以王通达矣。五伯欲继汤、武而不成,既足以为诸侯长矣。孔、墨欲行大道于世而不成,既足以成显荣矣。夫大义之不成,既有成已,故务事大。





上农


三曰:

古先圣王之所以导其民者,先务于农。民农非徒为地利也,贵其志也。民农则朴,朴则易用,易用则边境安,主位尊。 (1) 民农则重,重则少私义, (2) 少私义则公法立,力专一。民农则其产复, (3) 其产复则重徙,重徙则死其处而无二虑。 (4) 民舍本而事末则不令, (5) 不令则不可以守,不可以战。 (6) 民舍本而事末则其产约,其产约则轻迁徙,轻迁徙则国家有患,皆有远志,无有居心。 (7) 民舍本而事末则好智,好智则多诈,多诈则巧法令, (8) 以是为非,以非为是。

(1) 尊,重也。

【校】次“易用”,旧本脱“用”字,据《御览》七十七补,《亢仓子·农道》篇作“易用则边境安,安则主位尊”,又多“安则”二字。

(2) 【校】“重”,《亢仓子》作“童”,亦如《大戴》之《王言》篇与《家语》“童”、“重”互异也。

(3) 【校】《御览》“复”作“厚”,《亢仓子》作“複”,下并同。

(4) 处,居。

(5) 令,善。

(6) 战,攻。

(7) 居,安也。

(8) 巧,读如巧智之巧。

【校】《亢仓子》有“巧法令则”四字在下句首。

后稷曰:“所以务耕织者,以为本教也。”是故天子亲率诸侯耕帝籍田,大夫、士皆有功业。 (1) 是故当时之务,农不见于国, (2) 以教民尊地产也。 (3) 后妃率九嫔蚕于郊,桑于公田。是以春秋冬夏皆有麻枲丝茧之功,以力妇教也。 (4) 是故丈夫不织而衣,妇人不耕而食,男女贸功以长生, (5) 此圣人之制也。 (6) 故敬时爱日,非老不休, (7) 非疾不息,非死不舍。 (8)

(1) 《传》曰“王耕一发,班三之,庶人终于千亩”,故曰“皆有功业”也。

【校】“皆有功业”,《亢仓子》作“第有功级”。注“一发”,《周语》作“一墢”,此作“发”,讹。韦昭注:“一墢,一 之发也。”玩注意,似《亢仓子》本是。

(2) 当启蛰耕农之务,农民不见于国都也。《孟春纪》曰“王布农事,命田舍东郊”,故农民不得见于国也。

(3) 地产,嘉谷也。

(4) 力,任其力、效其功也。

【校】《亢仓子》作“劝人力妇教也”。

(5) 贸,易也。

【校】“以长生”,《亢仓子》作“资相为业”。

(6) 制,法也。

(7) 休,止也。

(8) 舍,置也。

上田夫食九人,下田夫食五人,可以益,不可以损。 (1) 一人治之,十人食之,六畜皆在其中矣。此大任地之道也。故当时之务,不兴土功,不作师徒,庶人不冠弁 (2) 、娶妻、嫁女、享祀,不酒醴聚众, (3) 农不上闻,不敢私籍于庸,为害于时也。然后制野禁,苟非同姓, (4) 农不出御, (5) 女不外嫁,以安农也。 (6) 野禁有五:地未辟易,不操麻,不出粪; (7) 齿年未长,不敢为园囿;量力不足,不敢渠地而耕; (8) 农不敢行贾, (9) 不敢为异事。 (10) 为害于时也。然后制四时之禁:山不敢伐材下木, (11) 泽人不敢灰僇, (12) 缳网罝罦不敢出于门,罛罟不敢入于渊, (13) 泽非舟虞,不敢缘名。为害其时也。 (14) 若民不力田,墨乃家畜。

(1) 损,减也。

(2) 弁,鹿皮冠。《诗》云“冠弁如星”。

【校】冠弁不见《诗考》,恐是字误。

(3) 《礼》“取妇之家,三日不举乐;嫁女之家,三日不绝烛”,故不以酒醴聚众也。

(4) 苟,诚也。

(5) 御妻也。

(6) 异姓之女不出闾邑而嫁也。

(7) 出犹捐也。

(8) 渠,沟也。

(9) 守其疆亩也。

(10) 异犹他也。

(11) 伐,斫也。

(12) 烧灰不以时多僇。

(13) 罝,兽罟也。《诗》云“肃肃兔罝”。罟,鱼罟也。《诗》云“施罛 ,鳣鲔发发”。

(14) 舟虞,主舟官。

国家难治,三疑乃极, (1) 是谓背本反则, (2) 失毁其国。凡民自七尺以上,属诸三官。 (3) 农攻粟,工攻器,贾攻货。 (4) 时事不共,是谓大凶。夺之以土功,是谓稽,不绝忧唯,必丧其秕。夺之以水事,是谓籥,丧以继乐, (5) 四邻来虚。夺之以兵事,是谓厉, (6) 祸因胥岁,不举铚艾。数夺民时,大饥乃来。野有寝耒,或谈或歌,旦则有昏,丧粟甚多。皆知其末,莫知其本真。 (7)

(1) 【校】义未详。

(2) 则,法也。

(3) 三官,农、工、贾也。

(4) 攻,治也。

(5) 继,续也。

(6) 厉,摩也。

(7) 不敏也。

【校】三字疑亦正文。





任地


四曰:

后稷曰:子能以窐为突乎? (1) 子能藏其恶而揖之以阴乎? (2) 子能使吾士靖而甽浴士乎? (3) 子能使保湿安地而处乎?子能使雚夷毋淫乎? (4) 子能使子之野尽为泠风乎? (5) 子能使藳数节而茎坚乎?子能使穗大而坚均乎? (6) 子能使粟圜而薄糠乎?子能使米多沃而食之强乎?无之若何?

(1) 窐,容污,下也。突,理出,丰高也。

(2) 阴犹润泽也。

(3) “士”当作“土”。

【校】古“士”、“土”间亦通用。

(4) 淫,延生也。

(5) 泠风,和风,所以成谷也。

(6) 《诗》云:“实发实秀,实坚实好。”此之谓也。

凡耕之大方:力者欲柔,柔者欲力;息者欲劳,劳者欲息;棘者欲肥,肥者欲棘; (1) 急者欲缓,缓者欲急; (2) 湿者欲燥,燥者欲湿。 (3) 上田弃亩,下田弃甽。五耕五耨,必审以尽。其深殖之度,阴土必得,大草不生, (4) 又无螟蜮。 (5) 今兹美禾,来兹美麦。 (6) 是以六尺之耜,所以成亩也;其博八寸,所以成甽也; (7) 耨柄尺,此其度也; (8) 其耨六寸,所以间稼也。 (9) 地可使肥,又可使棘。人肥必以泽, (10) 使苗坚而地隙;人耨必以旱,使地肥而土缓。 (11)

(1) 棘,羸瘠也。《诗》云“棘人之栾栾”,言羸瘠也。土亦有瘠土。

(2) 急者,谓强垆刚土也,故欲缓。缓者,谓沙堧弱土也,故欲急。和二者之中,乃能殖谷。

(3) 湿,谓下湿近污泉,故欲燥。燥,谓高明暵干,故欲湿。不燥不湿,取其中适,乃成黍稷也。

(4) 草,秽也。

(5) “蜮”,或作“螣”。食心曰螟,食叶曰蜮。兖州谓蜮为螣,音相近也。

【校】惠氏栋云:“‘蜮’,当为‘ ’。”

(6) 兹,年也。

(7) 耜六尺,其刃广八寸。古者以耜耕,广六尺为亩,三尺为甽,辽西之人谓之“ ”也。

【校】《周礼》“广尺深尺曰 ”,此云“三尺”,黄东发谓于正文不合。其言曰:“耜者,今之犁,广六尺,旋转以耕土,其块彼此相向,亦广六尺,而成一疄,此之谓亩。而百步为亩,总亩之四围总名。‘其博八寸,所以成甽’者,犁头之刃,逐块随刃而起,其长竟亩,其起而空之处与刃同其阔,此之谓甽。”案此所云,则与《周礼》相近。“ ”,字书无考。

(8) 度,制也。

(9) 耨,所以耘苗也。刃广六寸,所以入苗间也。

(10) 地耕熟则肥,肥即得谷多,不则瘠,瘠则得谷少,故曰“可使”也。人肥则颜色润泽。

(11) 缓,柔也。

草諯大月。 (1) 冬至后五旬七日,菖始生。 (2) 菖者,百草之先生者也。于是始耕。 (3) 孟夏之昔,杀三叶而获大麦。 (4) 日至,苦菜死而资生, (5) 而树麻与菽, (6) 此告民地宝尽死。凡草生藏日中出,狶首生而麦无叶, (7) 而从事于蓄藏, (8) 此告民究也。 (9) 五时见生而树生,见死而获死。 (10) 天下时,地生财,不与民谋。 (11)

(1) 大月,孟冬月也。

(2) 菖,菖蒲,水草也,冬至后五十七日而挺生。

(3) 《传》曰“土发而耕”,此之谓也。

(4) 昔,终也。三叶,荠、亭历、菥蓂也,是月之季枯死,大麦熟而可获。大麦,旋麦也。

【校】《初学记》二十七引《吕氏》“孟夏之山百谷三叶而获大麦”。

(5) 菜名也。

【校】“资”疑即“ ”,蒺藜也。

(6) 树,种也。菽,豆也。

(7) 凡草,庶草也。日中,春分也。众草生而出也。狶首,草名也,至其生时,麦无叶,皆成熟也。

(8) 藏之于仓也。

(9) 究,毕也。刈麦毕也。

(10) 五时,五行生杀之时也。见生,谓春夏种稼而生也。见死,谓秋冬获刈收死者也。

(11) 天降四时,地出稼穑,自然之道也,故曰“不与民谋”。

有年瘗土,无年瘗土。 (1) 无失民时,无使之治下。知贫富利器,皆时至而作,渴时而止。 (2) 是以老弱之力可尽起, (3) 其用日半,其功可使倍。 (4) 不知事者,时未至而逆之,时既往而慕之, (5) 当时而薄之, (6) 使其民而郄之。 (7) 民既郄,乃以良时慕,此从事之下也。操事则苦,不知高下,民乃逾处。种稑禾不为稑,种重禾不为重, (8) 是以粟少而失功。 (9)

(1) 祭土曰瘗。年,谷也。有谷祭土,报其功也。无谷祭土,禳其神也。

【校】“禳”,旧作“让”,讹,赵改正。

(2) 利用之器,有其时而为之,无其时而止之。

(3) 【校】《亢仓子》作“可使尽起”。

(4) 一辟曰倍。

【校】注“一辟”疑是“一倍”。

(5) 慕,思也。

(6) 薄,轻也,言不重时也。“薄”或作“怠”。

(7) 郄,逆之也。

(8) 晚种早熟为稑,早种晚熟为重。《诗》云:“黍稷重稑,植穉菽麦。”此之谓也。

(9) 不当其时,故粟少也。食之少气力,故曰“少而失功”也。





辩土


五曰:

凡耕之道,必始于垆, (1) 为其寡泽而后枯; (2) 必厚其靹, (3) 为其唯厚而及。 (4) 者 之,坚者耕之,泽其靹而后之;上田则被其处,下田则尽其污。无与三盗任地:夫四序参发,大甽小亩,为青鱼胠,苗若直猎,地窃之也。既种而无行,耕而不长,则苗相窃也。弗除则芜, (5) 除之则虚, (6) 则草窃之也。故去此三盗者,而后粟可多也。

(1) 垆,埴垆地也。

(2) 言土燥湿也。

【校】注“燥湿”下疑当有一“均”字。

(3) 厚,深也。

【校】靹,音义缺。

(4) “ ”或作“选”。

【校】梁仲子云:“‘ ’,疑即‘ ’字。《集韵》‘饱,或从缶’。” ,音义并缺。

(5) 芜,秽也。

(6) 虚,动稼根。

所谓今之耕也,营而无获者。 (1) 其早者先时,晚者不及时,寒暑不节,稼乃多菑实。其为亩也,高而危则泽夺,陂则埒,见风则 , (2) 高培则拔, (3) 寒则雕, (4) 热则修, (5) 一时而五六死,故不能为来。 (6) 不俱生而俱死,虚稼先死, (7) 众盗乃窃。望之似有余,就之则虚。 (8) 农夫知其田之易也, (9) 不知其稼之疏而不适也; (10) 知其田之际也,不知其稼居地之虚也。 (11) 不除则芜,除之则虚,此事之伤也。 (12) 故亩欲广以平,甽欲小以深, (13) 下得阴, (14) 上得阳, (15) 然后咸生。 (16)

(1) “获”或作“种”。

(2) ,仆也。

(3) 培田侧也。

(4) 雕,不实也。

(5) 修,长也。

(6) 来丕成也。

(7) 虚,根不实。

(8) 虚,不颖不栗。《诗》云“实颖实栗,有邰家室”也。

(9) 易,治也。易,读如易纲之易也。

【校】注“易纲”,梁仲子疑是“易畴”。

(10) 疏,希也,不中适也。

(11) 虚,亦希也。

(12) 伤,败也。

(13) 【校】孙云:“李善注《文选》王元长《策秀才文》‘清甽泠风’引此‘深’作‘凊’。”今案:“深”字是,《亢仓子》作“ 欲深以端”。

(14) 阴,湿也。

(15) 阳,日也。

(16) 咸,皆也。

稼欲生于尘,而殖于坚者。 (1) 慎其种,勿使数,亦无使疏。于其施土,无使不足, (2) 亦无使有余。 (3) 熟有耰也, (4) 必务其培。其耰也植,植者其生也必先。 (5) 其施土也均,均者其生也必坚。 (6) 是以亩广以平则不丧本, (7) 茎生于地者五分之以地。 (8) 茎生有行,故速长;弱不相害,故速大。 (9) 衡行必得,纵行必术。正其行,通其风, (10) 夬心中央,帅为泠风。 (11) 苗其弱也欲孤, (12) 其长也欲相与居, (13) 其熟也欲相扶。 (14) 是故三以为族,乃多粟。 (15)

(1) 殖,长也。

(2) 土,壤也。

(3) 余犹多也。

(4) 耰,覆种也。

(5) 先犹速也。

(6) 坚,好也。

(7) 本,根也。

(8) 分,别也。

(9) 速,疾也。

(10) 行,行列也。

(11) 夬,决也。心于苗中央。帅,率也。啸泠风以摇长之也。“夬”或作“使”。

【校】《选》注引作“夬必中央,师为泠风”,又引注云“必于苗中央,师师然肃泠风以摇长也”。

(12) 弱,小也。苗始生小时,欲得其孤特疏数适中则茂好也。

(13) 言相依植,不偃仆。

【校】旧本无“其”字,又注作“相依助不僵仆”,皆讹脱,今据《齐民要术》所引补正。《亢仓子》亦有“其”字。《要术》“居”作“俱”。今案《亢仓》作“居”,与此同。

(14) 扶,相扶持,不可伤折也。

【校】《齐民要术》作“相扶持,不伤折”,此亦衍二字。

(15) 族,聚也。

【校】《亢仓子》作“稼乃多谷”。

凡禾之患,不俱生而俱死。是以先生者美米,后生者为秕。 (1) 是故其耨也,长其兄而去其弟。 (2) 树肥无使扶疏,树 不欲专生而族居。 (3) 肥而扶疏则多秕, (4) 而专居则多死。 (5) 不知稼者,其耨也去其兄而养其弟, (6) 不收其粟而收其秕,上下不安则禾多死, (7) 厚土则孽不通, (8) 薄土则蕃 而不发。垆埴冥色,刚土柔种,免耕杀匿,使农事得。

(1) 秕,不成粟也。

(2) 养大杀小。

(3) 专,独也。

(4) 根扇迫也。

(5) 专,独。不能自荫润其根,故多枯死也。

(6) 杀其大者,养其小者也。

(7) 【校】旧本“秕”作“粗”,下“不”字脱,并依《亢仓子》补正。

(8) 壤深不能自达,故多孽死也。





审时


六曰:

凡农之道,厚之为宝。斩木不时,不折必穗; (1) 稼就而不获, (2) 必遇天灾。 (3) 夫稼为之者人也, (4) 生之者地也,养之者天也。是以人稼之容足,耨之容耨,据之容手。 (5) 此之谓耕道。

(1) 折犹坚也。

(2) 获,得也。

(3) 灾,害也。

(4) 为,治也。

(5) 谓根苗疏数之间也。

【校】《亢仓子》作“耨之容耰,耘之容手”。

是以得时之禾,长秱长穗,大本而茎杀, (1) 疏穖而穗大, (2) 其粟圆而薄糠, (3) 其米多沃而食之强。 (4) 如此者不风。 (5) 先时者,茎叶带芒以短衡,穗巨而芳夺,秮米而不香。 (6) 后时者,茎叶带芒而末衡,穗阅而青零, (7) 多秕而不满。 (8)

(1) “杀”或作“小”。本,根也。茎稍小,鼠尾桑条谷也。

(2) 穖,禾穗果羸也。

(3) 圆,丰满也。薄糠,言米大也。

(4) 强,有势力也。

(5) 风,落也。

(6) “夺”或作“奋”字。

【校】旧校云:“‘末’一作‘小’。”案:《亢仓子》作“小茎”。

(7) 青零,未熟而先落。

【校】“阅”,《亢仓子》作“锐”。

(8) 满,成也。

得时之黍,芒茎而徼下,穗芒以长,抟米而薄糠,舂之易,而食之不噮而香。 (1) 如此者不饴。先时者,大本而华,茎杀而不遂, (2) 叶藳短穗。后时者,小茎而麻长,短穗而厚糠,小米钳而不香。 (3)

【校】《亢仓》“穗”下有“不”字。

(1) 香,美也。噮,读如 厌之 。

【校】《御览》八百四十二作“ ”。窃疑上注“读如 厌之 ”当在此句下。据《御览》,噮音北县切,决不当读 也。

(2) 遂,长。

【校】“藳”,《御览》作“高”。

(3) 小米故厚糠也。

【校】“米钳”,《御览》作“米令”,注云“令,新也”。

得时之稻,大本而茎葆,长秱疏穖,穗如马尾,大粒无芒,抟米而薄糠,舂之易而食之香。如此者不益。 (1) 先时者,本大而茎叶格对, (2) 短秱短穗,多秕厚糠,薄米多芒。后时者,纤茎而不滋,厚糠多秕, 辟米不得恃定熟, (3) 卬天而死。

(1) 益,息也。

【校】旧校云:“‘益’一作‘蒜’。”案:《御览》八百三十九作“ ”。注“益,息也”,义亦难晓。

(2) 对,等也。

(3) 辟,小也。“恃”或作“待”。

【校】《御览》无“ ”字,字书无考,下作“辟米不大”,注止“辟小”二字,正文“得恃”及注“恃或作待”皆无。

得时之麻,必芒以长,疏节而色阳,小本而茎坚,厚枲以均,后熟多荣,日夜分复生。如此者不蝗。 (1)

(1) 蝗虫不食麻节也。

得时之菽,长茎而短足,其荚二七以为族,多枝数节,竞叶蕃实, (1) 大菽则圆,小菽则抟以芳,称之重,食之息以香。如此者不虫。 (2) 先时者,必长以蔓,浮叶疏节,小荚不实。后时者,短茎疏节,本虚不实。

(1) 二七,十四实也。

【校】“荚”旧讹作“ ”,今从《初学记》、《御览》改,下讹作“英”,亦并改。

(2) 虫不啮其荚芒也。

得时之麦,秱长而颈黑,二七以为行,而服,薄 而赤色,称之重,食之致香以息,使人肌泽且有力。 (1) 如此者不蚼蛆。先时者,暑雨未至, (2) 胕动,蚼蛆而多疾, (3) 其次羊以节。后时者,弱苗而穗苍狼,薄色而美芒。

(1) “肌”或作“肥”。

(2) “至”或作“上”。

(3) 胕动,病心。胕,读如疛。

【校】洪氏亮吉《汉魏音》引此注云:“胕读如疛。”案:“肘”如“疛”,音同,知“胕”“肘”本一字也。今本“疛”作“痛”,误,从旧本改正。《亢仓》“胕动”作“胕肿”。

【校】案:苍狼,青色也。在竹曰“苍筤”,在天曰“仓浪”,在水曰“沧浪”,字异而义皆同。

是故得时之稼兴, (1) 失时之稼约。 (2) 茎相若称之,得时者重,粟之多。量粟相若而舂之,得时者多米。量米相若而食之, (3) 得时者忍饥。 (4) 是故得时之稼,其臭香,其味甘,其气章, (5) 百日食之, (6) 耳目聪明,心意睿智, (7) 四卫变强, (8) 气不入,身无苛殃。 (9) 黄帝曰:“四时之不正也,正五谷而已矣。” (10)

(1) 兴,昌也。

(2) 约,青病也。

(3) 【校】旧校云:“一作‘以为食’。”

(4) 忍犹能也。能,耐也。

(5) 气,力也。章,盛也。

(6) “百日食之”者,食之百日也。

(7) 睿,明也。

(8) 四卫,四枝也。

(9) 苛,病。殃,咎。

(10) 五谷正时,食之无病,故曰“正五谷而已”。





吕氏春秋附考



序说


《吕氏春秋·序意》:维秦八年,岁在涒滩, (1) 秋,甲子朔,朔之日,良人请问十二纪。文信侯曰:“尝得学黄帝之所以诲颛顼矣,爰有大圜在上,大矩在下,汝能法之,为民父母。盖闻古之清世,是法天地。凡十二纪者,所以纪治乱存亡也,所以知寿夭吉凶也。上揆之天,下验之地,中审之人,若此则是非可不可无所遁矣。天曰顺,顺维生;地曰固,固维宁;人曰信,信维听。三者咸当,无为而行。行也者,行其理也。行数,循其理,平其私。夫私视使目盲,私听使耳聋,私虑使心狂,三者皆私设精 (2) 则智无由公。智不公,则福日衰,灾日隆,以日倪而西望知之。” (3)

(1) 高诱注:秦始皇即位八年。

(2) 疑情。

(3) 此《吕氏》十二纪原序,且其言近道,故以为冠冕。

《史记·吕不韦列传》:吕不韦者,阳翟大贾人也……太子政立为王,尊吕不韦为相国,号称“仲父”……当是时,魏有信陵君,楚有春申君,赵有平原君,齐有孟尝君,皆下士喜宾客以相倾。吕不韦以秦之强,羞不如,亦招致士,厚遇之,至食客三千人。是时诸侯多辩士,如荀卿之徒,著书布天下。吕不韦乃使其客人人著所闻,集论以为八览、六论、十二纪,二十余万言。以为备天地万物古今之事,号曰《吕氏春秋》。布咸阳市门,悬千金其上,延诸侯游士宾客有能增损一字者予千金。 (1)

(1) 案:不韦著书之由,惟此最详且确。太史公曰:“孔子之所谓‘闻’者,其吕子乎!”真能灼见不韦本意。后之言《吕氏春秋》者多失之。

《十二诸侯年表》,吕不韦者,秦庄襄王相,亦上观尚古,删拾《春秋》,集六国时事,以为八览、六论、十二纪,为《吕氏春秋》。

《太史公自序》:不韦迁蜀,世传《吕览》。 (1)

(1) 《正义》曰:即《吕氏春秋》。

《汉书·司马迁传》:不韦迁蜀,世传《吕览》。 (1)

(1) 苏林曰:《吕氏春秋》篇名八览、六论。

郑康成曰:《月令》……本《吕氏春秋》十二月纪之首章也。以礼家好事抄合之,后人因题之名曰《礼记》。 (1)

(1) 《三礼目录》。

又曰:吕氏说月令而谓之“春秋”,事类相近焉。 (1)

(1) 《礼运》注。

蔡邕曰:《周书》七十一篇,而《月令》第五十三。秦相吕不韦著书,取“月令”为纪号。淮南王安亦取以为第四篇,改名曰“时则”。故偏见之徒,或云“《月令》,吕不韦作”,或云“淮南”,皆非也。 (1)

(1) 《蔡中郎集》。

司马贞曰:八览者,《有始》、《孝行》、《慎大》、《先识》、《审分》、《审应》、《离俗》、《时君》也。 (1) 六论者,《开春》、《慎行》、《贵直》、《不苟》、《以顺》、《士容》也。 (2) 十二纪者,记十二月也,其书有《孟春》等纪。二十余万言,三十余卷也。 (3)

(1) 本书作“恃”。

(2) 本书作“似”。

(3) 《史记索隐》。

【校】案《汉志》及《隋》《唐志》皆“二十六”,此及《子钞》与《书录解题》俱作“三十”,误也。

唐马总曰:吕不韦,始皇时相国,乃集儒士为十二纪、八览、六论,暴于咸阳市,有能增损一字与千金。无敢易者。 (1)

(1) 《意林》。

宋黄震曰:《吕氏春秋》者,秦相吕不韦耻以贵显而不及荀卿子之徒著书布天下,使其宾客共著八览、六论、十二纪,窃名《春秋》。高诱为之训解。淳熙五年冬,尚书韩彦直为之序,谓:“士之传于天下后世者,非徒以其书。夫子之圣则书宜传,孟子之亚圣则书宜传。过是而以书传者,老聃以虚无传,庄周以假寓传,屈原以骚传,荀卿以刑名传, (1) 司马迁以史传,扬雄以《法言》传,班孟坚以续史迁传。然概之孔、孟宜无传,而皆得并传者,其人足与也。《吕氏春秋》言天地万物之故,其书最为近古,今独无传焉,岂不以吕不韦而因废其书邪?愈久无传,恐天下无有识此书者,于是序而传之。”括苍蔡伯尹又跋其书之后曰:“汉兴,高堂生、后仓、二戴之徒取此书之十二纪为《月令》,河间献王与其客取其《大乐》、《适音》为《乐记》,司马迁多取其说为《世家》、《律历书》,孝武藏书以预九家之学,刘向集书以系《七略》之数。今其书不得与诸子争衡者,徒以不韦病也,然不知不韦固无与焉者也。” (2)

(1) 此句似有讹脱。或是“荀卿以性恶传,韩非以刑名传”。

(2) 《黄氏日抄》。

宋高似孙曰:淮南王尚奇谋,募奇士,庐馆一开,天下隽绝驰骋之流无不雷奋云集,蜂议横起,瑰诡作新,可谓一时杰出之作矣。及观《吕氏春秋》,则淮南王书殆岀于此者乎?不韦相秦,盖始皇之政也。始皇不好士,不韦则徕英茂,聚畯豪,簪履充庭,至以千计;始皇甚恶书也,不韦乃极简册,攻笔墨,采精录异,成一家言。吁!不韦何为若此者也?不亦异乎?《春秋》之言曰:“十里之间,耳不能闻;帷墙之外,目不能见;三亩之间,心不能知。而欲东至开晤,南抚多 ,西服寿靡,北怀儋耳,何以得哉?” (1) 此所以讥始皇也,始皇顾不察哉!不韦以此书暴之咸阳门曰“有能损益一字者与千金”,人卒无一敢易者,是亦愚黔之甚矣。秦之士其贱若此,可不哀哉!虽然,是不特人可愚也,虽始皇亦为之愚矣!异时亡秦者,又皆屠沽负贩不一知书之人,呜呼! (2)

(1) 语见《任数》篇,“开晤”作“开梧”,“多 ”作“多 ”。《意林》所载作“开悟”、“多 ”也。

(2) 《子略》。

宋马端临曰:《吕氏春秋》暴咸阳门,有能增损一字者予千金,时人无增损者。高诱以为非不能也,畏其势耳。昔《张侯论》为世所贵,崔浩《五经注》,学者尚之。二人之势犹能使其书传如此,况不韦权位之盛,学者安能牾其意而有所更易乎?诱之言是也。然十二纪者,本周公书,后置于《礼记》,善矣,而目之为吕令者,误也。 (1)

(1) 《文献通考》。

宋王应麟曰:《书目》……是书凡百六十篇,以月纪为首,故以“春秋”名书。十二纪篇首与《月令》同。 (1)

(1) 《玉海》。

元陈澔曰:吕不韦相秦十余年,此时已有必得天下之势,故大集群儒,损益先王之礼而作此书,名曰“春秋”,将欲为一代兴王之典礼也,故其间亦多有未见与《礼经》合者。其后徙死。始皇并天下,李斯作相,尽废先王之制,而《吕氏春秋》亦无用矣。然其书也,亦当时儒生学士有志者所为,犹能仿佛古制,故记礼者有取焉。 (1)

(1) 《礼记集说》。

明方孝孺曰:《吕氏春秋》十二纪、八览、六论,凡百六十篇。吕不韦为秦相时,使其宾客所著者也。太史公以为不韦徙蜀乃作《吕览》。夫不韦以见疑去国,岁余即饮鸩死,何有宾客,何暇著书哉?史又称不韦书成,悬之咸阳市,置千金其上,有易一字者辄与之。不韦已徙蜀,安得悬书于咸阳?由此而言,必为相时所著,太史公之言误也。 (1) 不韦以大贾乘势市奇货、致富贵而行不谨,其功业无足道者,特以宾客之书显其名于后世,况乎人君任贤以致治者乎?然其书诚有足取者,其《节丧》、《安死》篇讥厚葬之弊,其《勿躬》篇言人君之要在任人,《用民》篇言刑罚不如德礼,《达郁》、《分职》篇皆尽君人之道,切中始皇之病。其后秦卒以是数者偾败亡国,非知幾之士,岂足以为之哉?第其时去圣人稍远,论德皆本黄老,书出于诸人之所传闻,事多舛谬,如以桑谷共生为成汤,以鲁庄与颜阖论马,与齐桓伐鲁,鲁请比关内侯,皆非实事,而其时竟无敢易一字者,岂畏不韦势而然耶?然予独有感焉,世之谓严酷者,必曰秦法,而为相者,乃广致宾客以著书,书皆诋訾时君为俗主,至数秦先王之过无所惮,若是者皆后世之所甚讳,而秦不以罪。呜呼!然则秦法犹宽也。

(1) 本传不误。

卢文弨曰:《玉海》云:“《书目》,是书凡百六十篇。”今书篇数与《书目》同。然《序意》旧不入数,则尚少一篇。此书分篇极为整齐,十二纪,纪各五篇;六论,论各六篇;八览,览当各八篇,今第一览止七篇,正少一。考《序意》本明十二纪之义,乃末忽载豫让一事,与《序意》不类,且旧校云“一作‘廉孝’”,与此篇更无涉,即豫让亦难专有其名。 (1) 因疑《序意》之后半篇俄空焉,别有所谓“廉孝”者,其前半篇亦简脱,后人遂强相附合,并《序意》为一篇,以补总数之缺。然《序意》篇首无“六曰”二字,后人于目中专辄加之,以求合其数,而不知其迹有难掩也。今故略为分别,正以明不敢妄作之意云耳。

(1) 黄氏震云“十二纪终而缀之以《序意》,主豫让”云,则在宋时本已如此,然以为主豫让者,其说亦误也。





卷帙


《汉书·艺文志》杂家:《吕氏春秋》二十六篇,秦相吕不韦辑智略士作。

梁庾仲容《子钞》:《吕氏春秋》三十六卷。 (1)

(1) 《子略》。

《隋书·经籍志》杂部:《吕氏春秋》二十六卷,秦相吕不韦撰,高诱注。

马总《意林》:《吕氏春秋》二十六卷。

《旧唐书·经籍志》杂家:《吕氏春秋》二十六卷,吕不韦撰。

《新唐书·艺文志》杂家:《吕氏春秋》二十六卷,吕不韦撰,高诱注。

《文献通考·经籍》杂家:《吕氏春秋》二十卷。 (1)

(1) 此脱“六字”。

《通志·艺文略》杂家:《吕氏春秋》二十六卷,秦相吕不韦撰,高诱注。

《郡斋读书志》杂家类:《吕氏春秋》二十六卷,右秦相吕不韦撰,后汉高诱注。

《直斋书录解题》杂家类:《吕氏春秋》三十六卷,秦相吕不韦撰,后汉高诱注。 (1)

(1) 此与《子钞》卷数皆误。

《宋史·艺文志》杂家类:吕不韦《吕氏春秋》二十六卷,高诱注。







\backmatter

\end{document}