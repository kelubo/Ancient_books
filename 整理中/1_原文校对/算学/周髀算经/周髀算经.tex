% 周髀算经
% 周髀算经.tex

\documentclass[a4paper,12pt,UTF8,twoside]{ctexbook}

% 设置纸张信息。
\RequirePackage[a4paper]{geometry}
\geometry{
	%textwidth=138mm,
	%textheight=215mm,
	%left=27mm,
	%right=27mm,
	%top=25.4mm, 
	%bottom=25.4mm,
	%headheight=2.17cm,
	%headsep=4mm,
	%footskip=12mm,
	%heightrounded,
	inner=1in,
	outer=1.25in
}

% 设置字体,并解决显示难检字问题。
\xeCJKsetup{AutoFallBack=true}
\setCJKmainfont{SimSun}[BoldFont=SimHei, ItalicFont=KaiTi, FallBack=SimSun-ExtB]

% 目录 chapter 级别加点(.)。
\usepackage{titletoc}
\titlecontents{chapter}[0pt]{\vspace{3mm}\bf\addvspace{2pt}\filright}{\contentspush{\thecontentslabel\hspace{0.8em}}}{}{\titlerule*[8pt]{.}\contentspage}

% 设置 part 和 chapter 标题格式。
\ctexset{
	part/name= {第,卷},
	part/number={\chinese{part}},
	chapter/name={第,篇},
	chapter/number={\chinese{chapter}}
}

% 设置古文原文格式。
\newenvironment{yuanwen}{\bfseries\zihao{4}}

% 设置署名格式。
\newenvironment{shuming}{\hfill\bfseries\zihao{4}}

% 注脚每页重新编号,避免编号过大。
\usepackage[perpage]{footmisc}

\title{\heiti\zihao{0} 周髀算经}
\author{}
\date{}

\begin{document}

\maketitle
\tableofcontents

\frontmatter
\chapter{前言、序言}

\mainmatter

% 增加空行
~\\

% 增加字间间隔,适用于三字经、诗文等。
 \qquad  

\part{}

\chapter{}

昔者周公問于商高曰。竊聞乎大夫善數也。

請問古者包犧立周天歷度。  夫天不可階而升。地不可得尺寸而度。

請問數安從出。

商高曰。數之法。出于圓方。  圓出于方。方出于矩。

矩出于九九八十一。

故折矩。

以為句。廣三。  股修四。  徑隅五。

既方其外。半之一矩。

環而共盤。得成三四五。  兩矩共長二十有五。是謂積矩。

故禹之所以治天下者。此數之所生也。

周公曰。大哉言數。  請問用矩之道。

商高曰。平矩以正繩。  偃矩以望高。覆矩以測深。臥矩以知遠。

環矩以為圓。合矩以為方。

方屬地。圓屬天。天圓地方。

方數為典。以方出圓。  笠以寫天。  天青黑。地黃赤。天數之為笠也。青黑為表。丹黃為裏。以象天地之位。

是故。知地者智。知天者聖。

智出于句。

句出于矩。

夫矩之于數。其裁制萬物。惟所為耳。  周公曰。善哉。

周髀算經卷上之二昔者。榮方問于陳子。

曰。今者竊聞夫子之道。

知日之高大。

光之所照。一日所行。遠近之數。

人所望見。

四極之窮。

列星之宿。

天地之廣袤。

夫子之道。皆能知之。其信有之乎。  陳子曰。然。

榮方曰。方雖不省。願夫子幸而說之。

今若方者。可教此道耶。

陳子曰。然。  此皆算術之所及。

子之于算。足以知此矣。若誠累思之。

于是榮方歸而思之。數日不能得。

復見陳子曰。方、思之不能得。敢請問之。陳子曰。思之未熟。

此亦望遠起高之術。而子?能得。則子之於數。未能通類。

是智有所不及。而神有所窮。

夫道術、言約而用博者。智類之明。

問一類而以萬事達者。謂之知道。

今子所學。

算數之術。是用智矣。而尚有所難。是子之智類單。  夫道術所以難通者。既學矣。患其不博。  既博矣。患其不習。  既習矣。患其不能知。

故同術相學。  同事相觀。此列士之愚智。

賢不肖之所分。

是故能類以合類。此賢者業精習智之質也。

夫學同業而不能入神者。此不肖無智。而業不能精習。

是故算不能精習。吾豈以道隱子哉。固復熟思之。

榮方復歸思之。數日不能得。復見陳子曰。方思之以精熟矣。智有所不及。而神有所窮。知不能得。願終請說之。

陳子曰。復坐。吾語汝。于是榮方復坐而請陳子之說。曰夏至南萬六千里。冬至南十三萬五千里。  日中立竿測影。

此一者。天道之數。

周髀長八尺。夏至之日晷一尺六寸。

髀者。股也。正晷者。句也。

正南千里。句一尺五寸。正北千里。句一尺七寸。

日益表。南晷日益長。候句六尺。

即取竹空徑一寸。長八尺。捕影而視之。空正掩日。

而日應空之孔。

由此觀之。率八十寸。而得徑一寸。

故以句為首。以髀為股。  從髀至日下六萬里。而髀無影。從此以上至日。則八萬里。

以率率之。八十里得徑一里。十萬里得徑千二百五十里。  故曰。日晷徑。千二百五十里。

若求邪至日者。以日下為句。日高為股。句股各自乘。并而開方除之。得邪至日。從髀所旁至日所。十萬里。

法曰。周髀長八尺。句之損益。寸千里。

故曰。極者天廣袤也。

今立表高八尺以望極。其句一丈三寸。由此觀之。則從周北十萬三千里而至極下。

榮方曰。周髀者何。陳子曰。古時天子治周。  此數望之從周。故曰周髀。

髀者。表也。  日夏至南萬六千里。日冬至南十三萬五十里。日中無影。以此觀之。從南至夏至之日中十一萬九千里。  北至其夜半亦然。

凡徑。二十三萬八千里。  此夏至日道之徑也。其周。七十一萬四千里。

從夏至之日中。至冬至之日中。十一萬九千里。

北至極下亦然。則從極南至冬至之日中。二十三萬八千里。從極北至其夜半亦然。凡徑四十七萬六千里。此冬至日道徑也。其周百四十二萬八千里。從春秋分之日中北至極下。十七萬八千五百里。

從極下北至其夜半亦然。凡徑三十五萬七千里。周一百七萬一千里。故曰月之道常緣宿。日道亦與宿正。

南至夏至之日中。北至冬至之夜半。南至冬至之日中。北至夏至之夜半。亦徑三十五萬七千里。周一百七萬一千里。  春分之日夜分。以至秋分之日夜分。極下常有日光。

秋分之日夜分。以至春分之日夜分。極下常無日光。

故春秋分之日夜分之時。日光所照。適至極。陰陽之分等也。冬至夏至者。日道發斂之所生也。至晝夜長短之所極。

春秋分者。陰陽之修。晝夜之象。

晝者陽。夜者陰。

春分以至秋分。晝之象。

秋分至春分。夜之象。故春秋分之日中。光之所照北極下。夜半日光之所照亦南至極。此日夜分之時也。故曰日照四旁。各十六萬七千里。

人所望見遠近。宜如日光所照。

從周所望見。北過極六萬四千里。

南過冬至之日三萬二千里。

夏至之日中光。南過冬至之日中光四萬八千里。

南過人所望見萬六千里。

北過周十五萬一千里。北過極四萬八千里。

冬至之夜半日光。南不至人目所見七千里。

不至極下七萬一千里。

夏至之日中與夜半日光九萬六千里。過極相接。

冬至之日中與夜半日光。不相及十四萬二千里。不至極下七萬一千里。

夏至之日。正東西望。直周東西日下至周五萬九千五百九十八里半。冬至之日。正東西方不見日。  以算求之。日下至周二十一萬四千五百五十七里半。

凡此數者。日道之發斂。

冬至夏至。觀律之數。聽鐘之音。

冬至晝。夏至夜。

差數及日光所還觀之。

四極徑八十一萬里。周二百四十三萬里。

從周南至日照處三十萬二千里。

周北至日照處五十萬八千里。

東西各三十九萬一千六百八十三里半。

周在天中南十萬三千里。故東西短中徑二萬六千六百三十二里有奇。  周北五十萬八千里。冬至日十三萬五千里。冬至日道徑四十七萬六千里。周百四十二萬八千里。日光四極。當周東西各三十九萬一千六百八十三里有奇。

此方圓之法。

\chapter{}

凡為此圖。以丈為尺。以尺為寸。以寸為分。分、一千里。凡用繒方八尺一寸。今用繒方四尺五分。分、為二千里。

呂氏曰。凡四海之內。東西二萬八千里。南北二萬六千里。

凡為日月運行之圓周。七衡周而六閒。以當六月。

節六月為百八十二日八分日之五。

故日夏至在東井極內衡。日冬至在牽牛極外衡也。  衡復更。終冬至。

故曰一歲三百六十五日四分日之一。歲一內極一外極。

三十日十六分日之七。月一外極一內極。

是故。一衡之閒。萬九千八百三十三里三分里之一。即為百步。

欲知次衡徑。倍而增內衡之徑。

二之。以增內衡徑。

次衡放此。

內一衡徑二十三萬八千里。周七十一萬四千里。分為三百六十五度四分度之一。度得一千九百五十四里二百四十七步千四百六十一分步之九百三十三。

次二衡徑二十七萬七千六百六十六里二百步。周八十三萬三千里。分里為度。度得二千二百八十里百八十八步千四百六十一分步之千三百三十二。  次三衡徑三十一萬七千三百三十三里一百步。周九十五萬二千里。分為度。度得二千六百六里百三十步千四百六十一分步之二百七十。  次四衡徑三十五萬七千里。周一百七萬一千里。分為度。度得二千九百三十二里七十一步四千百六十一分步之六百六十九。

次五衡徑三十九萬六千六百六十六里二百步。周百一十九萬里。分為度。度得三千二百五十八里十二步千四百六十一分步之千六十八。  次六衡徑四十三萬六千三百三十三里一百步。周百三十萬九千里。分為度。度得三千五百八十三里二百五十四步千四百六十一分步之六。

次七衡徑四十七萬六千里周百四十二萬八千里。分為度。度得三千九百九里一百九十五步千四百六十一分步之四百五。

其次曰。冬至所北照過北衡十六萬七千里。  為徑八十一萬里。

周二百四十三萬里。

分為三百六十五度四分度之一。度得六千六百五十二里二百九十三步千四百六十一分步之三百二十七。過北而往者。未之或知。  或知者。或疑其可知。或疑其難知。此言上聖不學而知之。

故冬至日晷丈三尺五寸。夏至日晷尺六寸。冬至日晷長。夏至日晷短。日晷損益寸。差千里。故冬至夏至之日。南北遊十一萬九千里。四極徑八十一萬里。周二百四十三萬里。分為度。度得六千六百五十二里二百九十三步千四百六十一分步之三百二十七。此度之相去也。

其南北遊日六百五十一里一百八十二步一千四百六十一分步之七百九十八。

術曰。置十一萬九千里為實。以半歲一百八十二日八分日之五為法。

而通之。

得九十五萬二千為實。

所得一千四百六十一為法。除之。  實如法得一里。不滿法者。三之。如法得百。步。

不滿法者十之。如法得十。步。

不滿法者十之。如法得一。步。

不滿法者。以法命之。

\part{}

\chapter{}
凡日月運行。四極之道。  極下者。其地高人所居六萬里。滂沱四隤而下。

天之中央。亦高四旁六萬里。

故日光外所照。經八十一萬里。周二百四十三萬里。

故日運行處極北。北方日中。南方夜半。日在極東。東方日中。西方夜半。日在極南。南方日中。北方夜半。日在極西。西方日中。東方夜半。凡此四方者。天地四極四和。

晝夜易處。

加四時相及。  然其陰陽所終。冬夏所極。皆若一也。

天象蓋笠。地法覆槃。

天離地八萬里。

冬至之日。雖在外衡。常出極下地上二萬里。

故日兆月。

月光乃出。故成明月。

星辰乃得行列。

是故秋分以往到冬至。三光之精微。以成其道遠。  此天地陰陽之性自然也。

欲知北極樞。旋周四極。

當以夏至夜半時。北極南遊所極。

冬至夜半時。北遊所極。

冬至日加酉之時。西遊所極。  日加卯之時。東遊所極。  此北極璇璣四遊。  正北極樞。璇璣之中。正北。天之中。  正極之所遊。冬至日加酉之時。立八尺表。以繩繫表顛。希望北極中大星。引繩計地而識之。  又到旦明日加卯之時。復引繩希望之。首及繩致地。而識其端相去二尺三寸。

故東西極二萬三千里。

其兩端相去。正東西。  中折之。以指表。正南北。

加此時者。皆以漏揆度之。此東西南北之時。

其繩致地。所識去表丈三寸。故天之中去周十萬三千里。

何以知其南北極之時。以冬至夜半北遊所極也。北過天中萬一千五百里。以夏至南遊所極。不及天中萬一千五百里。此皆以繩繫表顛而希望之。北極至地所識丈一尺四寸半。故去周十一萬四千五百里。

過天中萬一千五百里。其南極至地所識九尺一寸半。故去周九萬一千五百里。其南不及天中萬一千五百里。此璇璣四極南北過不及之法。東西南北之正句。  周去極十萬三千里。日去人十六萬七千里。夏至去周萬六千里。夏至日道徑二十三萬八千里。周七十一萬四千里。春秋分日道徑三十五萬七千里。周百七萬一千里。冬至日道徑四十三萬六千里。周百四十二萬八千里。日光四極八十一萬里。周二百四十三萬里。從周南三十萬二千里。  璇璣徑二萬三千里。周六萬九千里。此陽絕陰彰。故不生萬物。

其術曰。立正句定之。

以日始出。立表而識其晷。日入復識其晷。晷之兩端相直者。正東西也。中折之。指表者。正南北也。極下不生萬物。何以知之。

冬至之日。去夏至十一萬九千里。萬物盡死。夏至之日。去北極十一萬九千里。是以知極下不生萬物。北極左右。夏有不釋之冰。

春分秋分。日在中衡。春分以往。日益北五萬九千五百里而夏至。秋分以往。日益南五萬九千五百里而冬至。

中衡去周七萬五千五百里。  中衡左右。冬有不死之草。夏長之類。

此陽彰陰微。故萬物不死。五穀一歲再熟。

凡北極之左右。物有朝生暮獲。  立二十八宿。以周天歷度之法。

術曰。倍正南方。

以正句定之。即平地徑二十一步。周六十三步。令其平矩以水正。

則位徑一百二十一尺七寸五分。因而三之。為三百六十五尺四分尺之一。  以應周天三百六十五度四分度之一。審定分之。無令有纖微。  分度以定。則正督經緯。而四分之一。合各九十一度十六分度之五。

于是圓定而正。

則立表正南北之中央。以繩繫顛。希望牽牛中央星之中。

則復候須女之星先至者。

如復以表繩。希望須女先至定中。

即以一遊儀。希望牽牛中央星。出中正表西幾何度。

各如遊儀所至之尺。為度數。

遊在于八尺之上。故知牽牛八度。

其次星。放此。以盡二十八宿度。則定矣。

立周度者。  各以其所先至遊儀度上。

車輻引繩就中央之正以為轂。則正矣。  日所以入。亦以周定之。  欲知日之出入。

以東井夜半中。牽牛之初臨子之中。

東井出中正表西三十度十六分度之七而臨未之中。牽牛初亦當臨丑之中。

于是天與地協。

乃以置周二十八宿。

置以定。乃復置周度之中央。立正表。

以冬至夏至之日。以望日始出也。立一遊儀于度上。以望中央表之晷。

晷參正。則日所出之宿度。

日入放此。

\chapter{}
牽牛。去北極百一十五度千六百九十五里二十一步千四百六十一分步之八百一十九。  術曰。置外衡去北極樞二十三萬八千里。除璇璣萬一千五百里。  其不除者。二十二萬六千五百里。以為實。

以內衡一度數千九百五十四里二百四十七步千四百六十一分步之九百三十三以為法。

實如法得一。度。

不滿法。求里步。

約之。合三百得一。以為實。

以千四百六十一分為法。得一。里。

不滿法者。三之。如法得百。步。

不滿法者。又上十之。如法得一。步。

不滿法者。以法命之。

次、放此。

婁與角。去北極九十一度六百一十里二百六十四步千四百六十一分步之千二百九十六。  術曰。置中衡去北極樞十七萬八千五百里。以為實。

以內衡一度數為法。實如法得一。度。不滿法者。求里步。不滿法者。以法命之。

東井去北極六十六度千四百八十一里百五十五步千四百六十一分步之千二百四十五。

術曰、置內衡去北極樞十一萬九千里。加璇璣萬一千五百里。

得十三萬五百里。以為實。

以內衡一度數為法。實如法得一。度。不滿法者。求里步。不滿法者。以法命之。

凡八節二十四氣。氣損益九寸九分六分分之一。冬至晷長一丈三尺五寸。夏至晷長一尺六寸。問次節損益寸數長短各幾何。  冬至晷長丈三尺五寸。  小寒丈二尺五寸。小分五。  大寒丈一尺五寸一分。小分四。

立春丈五寸二分。小分三。

雨水九尺五寸三分。小分二。

啟蟄八尺五寸四分。小分一。

春分七尺五寸五分。

清明六尺五寸五分。小分五。

穀雨五尺五寸六分。小分四。

立夏四尺五寸七分。小分三。

小滿三尺五寸八分。小分二。

芒種二尺五寸九分。小分一。

夏至一尺六寸。

小暑二尺五寸九分。小分。

大暑三尺五寸八分。小分二。

立秋四尺五寸七分。小分三。

處暑五尺五寸六分。小分四。

白露六尺五寸五分。小分五。

秋分七尺五寸五分。小分一。

寒露八尺五寸四分。小分一。

霜降九尺五寸三分。小分二。

立冬丈五寸二分。小分三。小雪丈一尺五寸一分。小分四。

大雪丈二尺五寸。小分五。  凡為八節二十四氣。氣損益九寸九分六分分之一。

冬至夏至。為損益之始。

術曰。置冬至晷。以夏至晷減之。餘為實。以十二為法。

實如法得一。寸。不滿法者。十之。以法除之。得一。分。  不滿法者。以法命之。

月後天十三度十九分度之七。  術曰。置章月二百三十五。以章歲十九除之。加日行一度。得十三度十九分度之七。此月一日行之數。即後天之度及分。

小歲。月不及故舍三百五十四度萬七千八百六十分度之六千六百一十二。  術曰。置小歲三百五十四日九百四十分日之三百四十八。  以月後天十三度十九分度之七乘之。為實。

又以度分母乘日分母。為法。實如法。得積後天四千七百三十七度萬七千八百六十分度之六千六百一十二。

以周天三百六十五度萬七千八百六十分度之四千四百六十五除之。

其不足除者。

三百五十四度萬七千八百六十分度之六千六百一十二。

此月不及故舍之分度數。他皆放此。

大歲。月不及故舍十八度萬七千八百六十分度之萬一千六百二十八。  術曰。置大歲三百八十三日九百四十分日之八百四十七。

以月後天十三度十九分度之七乘之。為實。又以度分母乘日分母。為法。實如法。得積後天五千一百三十二度萬七千八百六十分度之二千六百九十八。

以周天除之。  其不足除者。

此月不及故舍之分度數。  經歲。月不及故舍百三十四度萬七千八百六十分度之萬一百五。

術曰。置經歲三百六十五日九百四十分日之二百三十五。

以月後天十三度十九分度之七乘之。為實。又以度分母乘日分母。為法。實如法。得積後天四千八百八十二度萬七千八百六十分度之萬四千五百七十。

以周天除之。

其不足除者。  此月不及故舍之分度數。

小月。不及故舍二十二度萬七千八百六十分度之七千七百五十五。

術曰。置小月二十九日。

以月後天十三度十九分度之七乘之。為實。又以度分母乘日分母。為法。實如法。得積後天三百八十七度萬七千八百六十分度之萬二千二百二十。  以周天分除之。

其不足除者。此月不及故舍之分度數。

大月。不及故舍三十五度萬七千八百六十分度之萬四千三百三十五。

術曰。置大月三十日。  以月後天十三度十九分度之七乘之。為實。又以度分母乘日分母。為法。實如法。得積後天四百一度萬七千八百六十分度之九百四十。

以周天除之。

其不足除者。

此月不及故舍之分度數。

經月。不及故舍二十九度萬七千八百六十分度之九千四百八十一。

術曰。置經月二十九日九百四十分日之四百九十九。

以月後天十三度十九分度之七乘之為實。又以度分母乘日分母。為法。實如法。得積後天三百九十四度萬七千八百六十分度之萬三千九百四十六。

以周天除之。

其不足除者。

此月不及故舍之分度數。  六百五十二萬三千三百六十五除之。得一周。餘分五十二萬七千四百二十一。即不及故舍之分。以一萬七千八百六十除之。得經月不及故舍二十九度。不盡九千四百八十一。即以命分。

周髀算經卷下之三冬至晝極短。日出辰而入申。

陽照三。不覆九。

東西相當。正南方。

夏至晝極長。日出寅而入戌。陽照九。不覆三。

東西相當。正北方。

日出左而入右。南北行。  故冬至從坎陽在子。日出巽而入坤。見日光少。故曰寒。

夏至從離陰在午。日出艮而入乾。見日光多。故曰暑。日月失度。而寒暑相姦。  往者詘。來者信也。故詘信相感。

故冬至之後。日右行。夏至之後。日左行。左者往。右者來。

故月與日合。為一月。

日復日。為一日。

日復星。為一歲。

外衡冬至。

內衡夏至。

六氣復返。皆謂中氣。  陰陽之數。日月之法。十九歲為一章。

四章為一蔀。七十六歲。

二十蔀為一遂。遂千五百二十歲。

三遂為一首。首四千五百六十歲。

七首為一極。極三萬一千九百二十歲。生數皆終。萬物復始。

天以更元作紀歷。  何以知天三百六十五度四分度之一。而日行一度。而月後天十三度十九分度之七。二十九日九百四十分日之四百九十九。為一月。十二月十九分月之七。為一歲。  周天除之。

其不足除者。如合朔。古者包犧神農。制作為歷。度元之始。見三光未如其則。

日月列星。未有分度。

日主晝。月主夜。晝夜為一日。日月俱起建星。

月度疾。日度遲。

日月相逐于二十九日三十日閒。

而日行天二十九度餘。

未有定分。

于是三百六十五日南極影長。明日反短。以歲終日影反長。故知之三百六十五日者三。三百六十六日者一。

故知一歲三百六十五日四分日之一。歲終也。月積後天十三周。又與百三十四度餘。

無慮後天十三度十九分度之七。未有定。

于是日行天七十六周。月行天千一十六周。及合于建星。

置月行後天之數。以日後天之數除之。得十三度十九分度之七。則月一日行天之度。

復置七十六歲之積月。

以七十六歲除之。得十二月十九分月之七。則一歲之月。  置周天度數。以十二月十九分月之七除之。得二十九日九百四十分日之四百九十九。則一月日之數。 


\backmatter

\end{document}