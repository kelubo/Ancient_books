% 九章算术
% 九章算术.tex

\documentclass[12pt,UTF8]{ctexbook}

% 设置纸张信息。
\usepackage[a4paper,twoside]{geometry}
\geometry{
	left=25mm,
	right=25mm,
	bottom=25.4mm,
	bindingoffset=10mm
}

% 设置字体,并解决显示难检字问题。
\xeCJKsetup{AutoFallBack=true}
\setCJKmainfont{SimSun}[BoldFont=SimHei, ItalicFont=KaiTi, FallBack=SimSun-ExtB]

% 目录 chapter 级别加点(.)。
\usepackage{titletoc}
\titlecontents{chapter}[0pt]{\vspace{3mm}\bf\addvspace{2pt}\filright}{\contentspush{\thecontentslabel\hspace{0.8em}}}{}{\titlerule*[8pt]{.}\contentspage}

% 设置 part 和 chapter 标题格式。
\ctexset{
	part/name= {第,卷},
	part/number={\chinese{part}},
	chapter/name={第,篇},
	chapter/number={\chinese{chapter}}
}

% 图片相关设置。
\usepackage{graphicx}
\graphicspath{{Images/}}

% 设置古文原文格式。
\newenvironment{yuanwen}{\bfseries\zihao{4}}

% 设置署名格式。
\newenvironment{shuming}{\hfill\bfseries\zihao{4}}

% 注脚每页重新编号,避免编号过大。
\usepackage[perpage]{footmisc}

\title{\heiti\zihao{0} 九章算术}
\author{}
\date{}

\begin{document}

\maketitle
\tableofcontents

\frontmatter
\chapter{前言}

中国传统数学源远流长,有其自身特有的思想体系与发展途径,从远古以至宋元,在很长一段时间内成为世界数学发展的主流,但自明代以来,由于政治社会等种种原因,特别如明末徐光启所指出的那样,一方面“名理之儒,土苴天下之实事”,另方面“妖妄之术,谬言数有神理”,致使中国传统数学濒于灭绝,以后全为西方欧几里得传统所凌替以至垄断,虽然康乾之世曾有一度重视,但仅止于发掘阐释古籍而已,循至20世纪中叶,李俨、钱宝琮先生撰写中国数学史专门著作进行介绍,使中国古算得以不绝如缕。到70年代特别是改革开放以来,全国兴起了研习中国传统数学的高潮,论著迭出,仅就对《九章算术》与注者刘徽的各种形式的专著,就在10种以上,其它方面论著之多,更难以统计,这些研究使中国传统数学的固有特色,如构造性、机械化、以及离散型的算法形式等,与西方欧几里得传统迥然异趣,得以贻然在目,甚至国外数学史家,也表示了对中国古算的浓厚兴趣,李约瑟的中国科技史巨著固不待论,此外还酝酿了《九章算术》与刘徽注的英文与法文编译,尤其值得一提的是:《九章算术》刘徽注中关于阳马术的一段术文,过去认为有脱漏舛误而难以理解。丹麦的 Wagner 先生却给予了正确的解释,使中国古算中一段辉煌成就,得以大白于世。

《九章算术》博大精深,美誉四方,是我国古代最重要的数学经典。湖北省荆州地区江陵张家山出土的西汉“算数书”,因某些原因,在它出土(1985年)后11年,虽经努力,我们仍没有看到全文,所见仅仅是一鳞半爪而已。“算数书”无疑是《九章算术》的母本,理应列专章作出评述。

《九章算术》接触领域很广,古称“算术”,并不局限于今之算术,当代科学技术史泰斗英国李约瑟博士就译《九章算术》为Nine Chapters on the MathematicalArt。一个世纪以来,特别是

《九章算术》是传统中国乃至古代东方极其重要的数学著作,在西学东渐之前一直是中国与东亚国家的数学教科书,历千年而不衰。

此书不明作者姓氏,而其中某些内容可上溯至先秦时期,据此或可认为此书为历代不断补充、更新、修订的“集体作品”。全书问题共分九类,故名《九章算术》。

今传《九章算术》为西汉初年重编。成书后,历经刘徽、祖冲之父子、李淳风等大数学家的注释。

《九章算术》全书内容可简单概括如下:

第一章方田:论述不同形状田地面积的计算方法,并详细介绍了分数的加法、减法、乘法和除法法则。

第二章粟米:论述在商品交易过程中的折算方法,重点介绍了在折算过程中广泛应用的今有法则。

第三章衰分:介绍了在手工业和农业领域按比例分配的处理方法,详细讲解了衰分和返衰法则,涉及基础的数列知识。

第四章少广:论述方和圆的面积和体积的计算方法,并介绍了开平方和开立方法则的使用和演算,并涉及开圆和开立圆(球)的计算和应用。

第五章商功:论述土方工程中的体积计算,详细介绍了各种形状的土方的处理方法。

第六章均输:介绍了在赋税、服役等情况下合理负担的处理方法,并涉及一些相遇问题的解决方法。

第七章盈不足:介绍了盈亏问题的处理方法,详细讲解了双假设法的应用。

第八章方程:介绍了利用方程组解决实际问题的方法,详细讲解了线性方程组的解法,并提出了正负数的概念。

第九章勾股:介绍了利用勾股定理求解高度、长度等实际问题的方法。

早在20世纪三十年代,日本著名数学与数学史家藤原松三郎就曾在一次关于“东洋数学史”的学术演讲中,发表过极为精辟的见解:研究东洋数学史,首先要研究中国数学的发展史;而要研究中国数学史,则必须从研读《九章算术》开始。\footnote{业师刘书琴教授早年东渡日本,投师藤原先生门下,曾聆听先生的精彩演讲。这段话便是根据他的回忆记录的。}

19世纪八十年代,英国科学技术史专家李约瑟(Joseph Terence Montgomery Needham)在他的《中国科学技术史》这部巨著的篇首中写道:至于远东的文明、特别是其中最古老而又最重要的中国文明对科学、科学思想和技术的贡献,直到今天还仍然为云翳所遮蔽,而没有被人们所认识。“远东”这个名词本身,就说明了欧洲人有一种根深蒂固的偏见,甚至连那些怀有良好意愿的欧洲人,也很难排除这种偏见。\footnote{见李约瑟《中国科学技术史》第一卷导论,第一章序言。}

我国著名数学家吴文俊先生在1990年发表的《关于研究数学在中国的历史与现状》中预言道:《九章》与“刘注”所贯串的机械化思想,不仅曾经深刻影响了数学的历史进程,而且对数学的现状也正在发扬它日益显著的影响。它在进入21世纪后在数学中的地位,几乎可以预卜。\footnote{见吴文俊《关于研究数学在中国的历史与现状》。《自然辩证法通讯》,1990年第4期,37—39页。}《九章算术》与《几何原本》是大约同一时代的东、西方数学成就的总结。早在七八十年以前,日本数学史家小仓金之助就把《九章算术》与《几何原本》(以下亦称《原本》)相提并论,认为《九章》是“中国的欧几里得”。\footnote{小仓金之助,《支那数学社会性》,《改造》,1933年1月;收入小仓金之助《数学史研究》第一辑。}作为古代东、西方数学的代表作,《九章》与《原本》在数学发展史上的产生与流传确有许多相似之处。

然而当人们把中国古代科技发展的史实与近代科学的产生相联系而思索时,便萌生了一个使人困惑不解的问题:近代自然科学为何不发生在中国?

这就是举世瞩目的“李约瑟难题”。为寻求问题的答案,一种似是而非的推论颇为流行:近代自然科学未能发生在中国,是因为中国传统数学没有发展成近代数学;中国古代传统数学未能发展成近代数学,乃是由于中国传统数学本身的弱点所决定的。

所谓中国传统数学的弱点,质言之即指中国古代数学没有形成如同古希腊数学那样的公理化演绎体系。长期以来,西方学者视古希腊学术为人类科学及科学思想的根源。欧几里得的《几何原本》被奉为几何学的“圣经”;为《几何原本》所建立的由定义、公理、定理、证明构成的演绎系统,成为近代数学推理论证的典范。尤其从20世纪三十年代始法国的布尔巴基(Bourbaki)学派提出了用结构这一概念来贯串整个数学,并以其鸿篇巨制《数学原理》对数学发展产生了巨大影响。直至“李约瑟难题”提出的20世纪五十年代,布尔巴基的影响已波及整个数学界。在当时的历史条件下,以西方数学公理化体系为“标准”去评判中国传统数学的短长,从中找出某些“弱点”与“缺陷”,用以论证中国传统数学之所以未能发展为近代数学的原因是不足为怪的。这种“拘泥于西方数学的先入之见”的论证,自然最终无法摆脱“西方中心论”的偏见。

关于“东方数学的算法体系”的观点之提出,是近年来中国数学史研究的重大成就;与此相应,人们对于公理化体系与方法的意义与局限性,也开始了冷静的思索。

由《九章算术》研究而引起的对古代东、西方数学体系的比较,是一个极有意义的论题。事实上,在整个数学科学发展的历程中,始终存在着算法与演绎两种倾向,它们代表着东、西方数学传统的基本特征。回溯数学发展的历史,我们就会发现:数学的发展并非始终是演绎倾向独霸天下,而总是算法倾向与演绎倾向交替取得主导地位。古代巴比伦和埃及式的原始算法,被希腊式的演绎几何所接替;而在中世纪希腊数学衰落之时,算法倾向在中国、印度等地区继续繁荣,以至17、18世纪在欧洲产生无穷小算法时期;19世纪以来,随着分析的严格化运动,演绎倾向再度兴起,它在比古希腊高得多的水准上远远超越几何学的范围而扩展到数学的其他领域,成为现代数学的中流砥柱。\footnote{参见李文林《算法、演绎倾向与数学史的分期》。《自然辩证法通讯》,1986年第2期。}吴文俊先生在《关于研究数学在中国的历史与现状》中总结道:“世界古代数学分为东、西方两大流派。古代西方数学是以古希腊欧几里得《几何原本》为典范的公理化演绎体系;古代东方数学则是以我国《九章》及其刘徽注为代表的机械化算法体系。在世界数学发展的历史长河中,这两种体系互为消长,交替成为主流,推动着这门学科不断向前进展。”

20世纪中叶以来,随着电子计算机的出现,计算技术不仅在社会生活中的作用显著提高,也使数学的发展产生了根本性变革,与公理化演绎体系大相径庭的机械化算法体系随之兴起,它已越来越为数学家所认识和重视。当人们开始注意算法史的研究之时,数学史家才深刻地认识到,那种肇始于古代中国而不同于古希腊传统的东方数学,正是典型的机械化算法体系。以《九章算术》为代表的机械化算法体系,在经过明代以来近几百年的相对消沉之后,由于电子计算机的兴起而重新活跃起来。近年来,不少著名的数学家纷纷转向计算机代数、计算性几何一类新兴学科。一个令人惊喜的成就是几何定理证明的机械化。20世纪七十年代末期,著名数学家吴文俊先生从中国数学史研究领域转入数学机械化领域,在短短的几年间便获得了突破性进展。他所创立的机器证明理论在国际上被誉为“吴方法”,不仅被成功地应用于初等几何、微分几何、非欧几何等领域,而且迄今已证明了大量的数学定理,并使自动推理研究应用于高科技的诸多领域。追溯这一方法的来源,吴文俊解释说:“(它)直接导源于我国传统数学的思维方式,也就是从公元前1世纪成型的《九章算术》开始经祖冲之到元代大数学家朱世杰形成的以解方程为特色的机械算法体系。”

另一重要的《九章》史实表明,作为近代数学主要标志的微积分学,并非希腊演绎数学传统的发展,而是东方算法传统胜利的产物。东方数学的算法精神早在文艺复兴时期前就通过阿拉伯人传播到欧洲,并为欧洲学者所吸收。微积分学的发展史表明,从16世纪中期开始的一百多年间,为解决力学与几何学等领域提出的一系列实际问题,许多大数学家都致力于寻求这种特殊的“无穷小算法”,而并不注意算法的证明。这种倾向一直延续到18世纪末。“极限的概念,作为微分学的真正基础,对于希腊头脑来说完全像是一个外国人。”\footnote{见Scott,A history of mathematics, 1958。}然而,极限的方法作为中算传统数量观与数系理论的自然发展,早已为刘徽在圆面积与角锥体积计算中成功地运用。\footnote{参见李继闵《东方数学典籍〈九章算术〉及其刘徽注研究》。}在微积分的创造过程中起着重大作用而为西方学者盛称的所谓卡瓦列里(Cavalieri)原理,事实上早已为古代中算家所应用并为刘徽、祖暅成功地应用于球积计算,它于中算当称之为“刘祖原理”。中国古代数学的研究证明,我国古代早已“具备了西欧17世纪发明微积分前夕的许多条件,不妨说我们已经接近了微积分的大门”。\footnote{由于篇幅原因,这里不能详细阐述《九章》在数量观、实数系,以及极限论上的重大贡献,感兴趣的读者可参阅李继闵《〈九章算术〉导读与译注》。}

研究数学史的目的在于“古为今用”“以史为鉴”。用现代数学的观点研究《九章算术》始于20世纪初。对《九章》的早期研究,主要是日本数学史家三上义夫\footnote{三上义夫于1910年出版的《中日数学之发展》第一部分介绍中国古代数学,有专章论述《九章》及《海岛算算经》。他的《中国算学之特色》(1926)、《数学史丛话》(1933)及博士论文(1933)中对《九章》及刘徽注都有出色的专题研究。}(1875—1950)、小仓金之助(10)(1885—1962),美国数学史家D. E.史密斯(11)(D. E. Smith, 1860—1944)及F.卡约黎(F. Cajori, 1859-1930)(12),我国数学史家李俨(13)(1892—1963)、钱宝琮(14)(1892—1974)等人的工作,当以中日学者的贡献称著。20世纪五十年代以后,苏联与东欧各国对《九章算术》及刘徽注的研究得以展开。A. Л.尤什凯维奇(A.л.ющкeвиц)(15)和Э. И.别辽兹金娜(Э. И.Ьepeзкинa)(16)等苏联数学史家的工作占据着显要的地位。与此同时,美国与西欧对《九章》与“刘注”的研究亦日趋深入。英国科学史家李约瑟(17)(1900—1995)和他的合作者王铃(18)(1917—1994)是其中的杰出人物。李约瑟的《中国科学技术史》的数学章,是当时西方世界中唯一的一本系统论述中国古代数学的专册。对中国古代数学的成就及其在世界数学史上的地位做出较为客观公允的评述。书中列举了十进位位值制记数法、圆周率计算、贾宪三角形与高次数字方程求根法等中算家的卓越成就,来说明在宋元时期以前中国人在数学的许多领域处于领先于欧洲的地位。另外,西方学者在某些方面也取得了值得称赞的成就。丹麦学者华道安(D. B. Wagner)关于“刘徽阳马术注”和“刘徽、祖暅之论球积”的研究(19),就是其中显著的例证。

(10) 小仓金之助于1933年发表博士论文《中国数学的社会性》,其副标题为“通过《九章算术》看秦汉时代的社会状态”。(11)  D. E. Smith,与三上义夫合著《日本数学史》(1914),又于1925年出版著名的两卷本《数学史》,都对《九章算术》作了评介。(12)  F.Cajori于1922年出版其名著《数学史》,在介绍中国古典数学概貌中,对《九章算术》有简短说明。(13) 李俨《中国数学源流考略》(1919)、《中国数学史大纲》(1928)、《中国算学史》(1936)以及《中算史论丛》1—4册(1935)都对《九章》有所论述。(14) 钱宝琮《九章问题分类考》(1921)、《方程算法源流考》(1921)、《中国算术中之周率研究》(1923)、《〈九章算术〉盈不足术流传欧洲考》(1927)等文是他早期研究《九章》的代表作。(15)  A.Л.尤什凯维奇的著名论文《中国学者在数学领域中的成就》(1955)对《九章》在代数学的成就有详细的评述。他的《中世纪数学史》(1961)有近20页篇幅介绍《九章》及刘徽注的一些工作。(16) Э.И.别辽兹金娜于1957年将《九章算术》译成俄文;她的专著《中国古代数学》(1980)中有相当多的篇幅论述《九章》及其刘徽注的成就。(17) 李约瑟在1959年出版了《中国科学技术史》的第三卷,其中前168页为数学部分,书中对《九章算术》有较详细的论述。(18) 王铃于1956年把《九章算术》译成英文在英国出版,并同时发表了《〈九章算术〉和汉代期间的中国数学史》一书。(19) 见D.B.Wagner, An Early Chinese Derivation of the Volume of a Pyramid: Liu Hui, Third Century A. D., Historia Mathematica, 6(1979), pp. 164—188.及Liu Hui and Tsu Keng-chih on the Volume of a Sphere, Chinese Science pp. 59—79.(20) 见严敦杰为《九章算术汇校本》所作序言诗句。(21) 参见李继闵《调日法源流考》《中算家的分数近似法探究》《通其率考释》等论文。(22) 秦九韶《数书九章序》云:“今数之书,尚三十余家。天象历度,谓之缀术;太乙、壬甲,谓之三式,皆曰内算,言其秘也。《九章》所载,即《周官》九数,系于方圆者为里术,皆曰外算,对内而言也。其用相通,不可歧二。”(23) 参见Douglas Bridges,Ray Mines,《什么是构造数学?》。(24) 按照构造性的观点,这种证明只是“表明X不可能不存在,但并未给出寻找X的方法”。甚至构造数学之极端主张不能接受逻辑的排中律。(25) 参见李继闵《东方数学典籍〈九章算术〉及其刘徽注研究》,以及论文《十进位值制记数与筹算起源初探》。(26) 其后的《夏侯阳算经》有更为详细的记述:“一从十横,百立千僵,千十相望,万百相当。满六已上,五在上方,六不积算,五不单张。”(27) 参见李继闵《东方数学典籍〈九章算术〉及其刘徽注研究》。(28) 李约瑟认为,“实际上,它是一种‘修辞的’和位置的代数学”(见《中国科学技术史》卷三)。(29) 参见李继闵《从“演纪之法”与“大衍总数术”看秦九韶在算法上的成就》。(30) 参见李继闵《东方数学典籍〈九章算术〉及其刘徽注研究》。(31) 参见李继闵《东方数学典籍〈九章算术〉及其刘徽注研究》;郭书春《汇校九章算术》;李学勤《〈汇校九章算术〉跋》;莫绍揆《有关〈九章算术〉的一些讨论》。(32) 在古代注释《九章》著称于世者有:东汉刘洪(2世纪中叶)、徐岳(2世纪末);三国阚泽(3世纪中叶)、刘徽;南北朝祖冲之、祖暅父子(5世纪);唐初李淳风(7世纪);北宋贾宪(11世纪);南宋杨辉(13世纪)等。(33) 参见李继闵《东方数学典籍〈九章算术〉及其刘徽注研究》。(34) 见郭书春《关于〈九章算术〉及其刘徽注》《刘徽祖籍及其思想》。(35) 参见李继闵《刘徽对整勾股数的研究》。(36) 参见李继闵《从勾股比率论到重差术》;梅荣照《刘徽的勾股理论—关于勾股定理及其有关的几个公式的证明》。(37) 引自吴文俊《我国古代测望之学重差理论评介,兼评数学史研究中某些方法问题》。(38) 此书清初藏南京黄虞稷家,现藏上海图书馆。1980年文物出版社原式影印,收入《宋刻算经六种》。(39) 《永乐大典》散失,“算”字条下仅存卷16343—16344两卷,现存英国剑桥大学。1960年中华书局影印的《永乐大典》包括上述两卷在内。(40) 清代嘉定毛岳生家藏残本《详解九章算法》“石研斋抄本”一部。1842年宜稼堂主人郁松年刻刊《宜稼堂丛书》时收入该书。1932年商务印书馆出版《丛书集成(初编)》亦收入该书。(41) 其中商功缺第1~7问,第20~27问和刍童术;均输缺第26问。(42) 《四库全书》于1789年抄完七部,分贮于文渊等七阁。后来文源阁(北京圆明园)、文汇阁(扬州)、文宗阁(镇江)毁于战火;文澜阁(杭州)毁一部分,后又补齐。今存文渊阁(北京故宫)、文津阁(承德)、文溯阁(沈阳)藏本。文津阁本现藏北京国家图书馆,文渊阁本现藏台湾。(43) 经乾隆下诏对《四库全书》择其应刊刻者令镌版通行,后又批准改用木活字版印刷。因嫌活字版名称不雅,钦定为“聚珍版”。(44) 其中有乾隆间福建影雕聚珍版初版、1893年又据李潢《九章算术细草图说》修改而成的“补刊本”、1899年广州的广雅书局本、1936年排印的《丛书集成初编》本等。(45) 这个影宋的残本《九章算术》于乾隆中转入清宫,作为天禄琳琅阁藏书,今存故宫博物院。1932年由该院影印为《天禄琳琅阁丛书》的一种。(46) 此据郭书春的考证。




特别值得注意的是20世纪后半期以来,《九章算术》先后被译成多种文字出版。继1956年王铃将《九章算术》译为英文之后,1957年Э.И.别辽兹金娜将它译成了俄文,1968年K. Vogel把《九章算术》译为德文,1975年大矢真一与清水达雄各自发表了《九章》的日文译本。但遗憾的是上述各译本都未翻译刘徽的注释。1978年日本的朝日新闻社出版的《科学名著》第二册《中国天文学·数学集》收入了川原秀成所译的《九章算术》日文版,它是首个包括刘徽注在内的完整的《九章算术》的外文译本。(另有1999年出版的沈康身的英文译本,2004年出版的林力娜(Karine Chemla)的法文译本。)

对比近代关于《九章》与《原本》的研究进程,无疑《九章》的研究起步较晚,亦不如《原本》研究之广泛、深入。然而20世纪七十年代后期兴起的“《九章》热”,却大有后来居上之势。由于现代计算机所需数学的方式方法,正与《九章》传统的算法体系若合符节,因此《九章》的研究更加具有重大的现实意义。《九章算术》内容博大精深,与《几何原本》相比确有许多优胜之处。“若把《原本》比《算数》,此中翘楚是《九章》。”(20)

二 《九章算术》的篇章结构及风格特点

任何一个民族的科学文化都有其发生发展的历史渊源,因而表现出迥然不同的风格与特色。数学在古希腊哲学体系中占有重要的地位。古希腊的哲学具有重视抽象、崇尚逻辑、追求理想的传统。受这种传统思想的影响,古希腊的数学家尤其注重数学的推理论证,追求理论的系统完美。于是自然形成了其数学研究的传统:一切数学结论必须根据明确规定的公理以无懈可击的演绎法推导出来。经过几代数学家的努力,最终由欧几里得《几何原本》的公理化体系构建了西方数学理论的“范式”。着重抽象概念与逻辑思维,以及概念与概念之间的逻辑关系,以定义、公理、定理、证明构成其表达形式。与古希腊形成鲜明的对照,中国先秦诸子的思想则多具有注重实践、推崇经验、讲求实用的倾向。无论是儒家内在的人文主义还是道家传统的自然主义,虽然也具有思辩性与逻辑性,但它更偏重于直觉和经验的因素。对中国古代数学有深刻影响的墨家学说,它一方面促进了数学逻辑因素的发展,另一方面墨家所代表的手工艺者阶层注重生产实践的思想体系也会带来数学技术化的倾向。法家管仲把“计数”列为他的“七法”之一,强调计算与数学对于社会生活的重要性,其着眼点也在于把它看作是成就任何一件大事的必要手段。如此的先秦时期思想与文化形成了中算家的科学传统:从各个不同的实际应用领域中抽象出具有普遍意义的数学问题与模型,经过分析提高而提炼出一般的原理、原则与方法,运用这些基本的数学原理构造其求解的简捷而能行的机械化算法。与这一独特的算法体系相适应,我国传统数学乃是由问、答、术、注、草等几个彼此相关联的项目构成其独特的表达形式。《九章算术》表达的这种数学机械化算法体系,可以说是与《几何原本》异其旨趣的东方数学理论的“范式”,形成了欧几里得《几何原本》纯粹数学的公理化体系与中国古代《九章算术》应用数学的机械化算法体系的鲜明对比。
(一)篇章结构与理论系统《九章算术》是以应用问题解法集成的体例编纂成书的。全书246个题目按其应用范围与解题方法划分为九章,按应用领域来分科,具有浓厚的“应用数学”的色彩。《九章》中的每一类题目之下又包含着若干条目的内容。《九章》本文一般由三部分组成:一是“问”;二是“答”;三是“术”。除本经之外,传本还有后世学者的注释。在《九章》以后的历代算经大都遵循它的这种体例,有的还增加了演草、比类、演段等项目。
作为《九章》篇章名称的方田、粟米、衰分、少广、商功、均输、盈不足、方程、勾股,代表了中国传统数学的分科。它肇源于古老的《周礼》“九数”,这与汉末郑玄《周礼注》之说相合:“九数:方田、粟米、差分、商功、均输、方程、赢不足、旁要;今有重差、勾股。”“方田”,以御田畴界域,即是应用于田亩地域大小丈量的数学分科,大致是最初的测地学。“粟米”,以御交质变易,即是应用于谷物之类商品交易的数学分科,当属古代的商业数学。“衰分”,以御贵贱禀税,即是应用于粮食、税收等经济管理部门的数学分科,当属古代的经济数学。“少广”,以御积幂方圆,是处理几何学中面积与体积类问题的计算,广泛应用于多种学科,如测地学。“商功”,以御功程积实,即是应用于工程管理部门的数学分科,当属古代的工程数学。“均输”,以御远近劳费,即是应用于赋税徭役摊派方面的数学分科,当属古代的管理数学。“盈不足”,以御隐杂互见,是一种线性插值算法,通过两次假设获得的数据,求取两个未知量。“方程”,以御错糅正负,它是由“程禾”发展而来,即是应用于谷物测产方面的数学分科,似乎属于古代的农用数学了。至于“勾股”,以御高深广远,它是由古老的旁要发展而来;旁要,从旁腰取,即是应用于布点测量的数学分科,也就是古代的测量学了。
然而,按应用领域分科并不意味着这种数学就没有理论系统,《九章算术》便具有其内在独特的理论结构。其特点就在于它是以基本的算法与数学模型为其理论构成单位;而这各种各样的算法又以为数不多的基本原理或法则为“纲纪”,贯串成一个完整的算法理论体系。《九章》全书246问分属于53种算法。其实,《九章》的章名亦是一种基本算法的名称。
方田章主要讲各种图形面积的计算,它以方田即直田面积算法为基础,采用割邪补直、化圆为方、以直代曲等直观性方法,将各种图形化为方田算得其面积的精确或近似数值。因此方田术即为本章的基本算法。由于度量精密化的要求而产生的分数算法,也作为一个完整的数学基本内容附置于全书的首章之中。
粟米章的基础为“粟米之法”,即依各种谷物的交换率而相互推求的比例算法,古称“今有术”。由此引导出计算物品单价的经率术;进而讨论单价的近似整值与贵贱物价的分配,发展出颇为独特的其率术与反其率术。
衰分,即按比例分配,它是古代应用相当广泛的基本算法。衰分讨论成正比关系的量,而返衰则讨论成反比关系的量。在衰分章中化返衰为列衰,即将正、反比关系统一处理,两者化归同一算法。
少广章讨论的问题与方田章相逆,即由已知面积反求边长或周长。由正方形面积求边,由圆(球)之积求周(径),则为开平方或开立方算法。开方本于除法,它是除数待定条件下的除法。少广即是由除法到开方的运算理论。
商功是讨论各种柱、锥、台体的求积,其中心是阳马术(即四棱锥体积计算),它奠定了中算家多面体体积理论的基础。
均输术是由正、反比关系复合而成分配问题之解法,它作为衰分术之发展而广泛用于徭役分配与调节运输。
盈不足章讲双假设法及其实际应用,它最终归结为“盈不足”“两盈、两不足”“盈适足、不足适足”三种模式的计算。
“方程”术相当现今线性方程组的矩阵解法,其中行列相加减自然地推广到正负数的范围。
勾股章的前部分是讲解勾股形的各种算法,它是其后部分勾股测量的理论根据。《九章》的数学体系表现为算法的集合,这些算法前后关联、井然有序,反映出算法理论发展的来龙去脉。如果将《原本》的理论结构称做“逻辑链”的话,那么《九章》的理论结构当称之为“算法链”。自然“逻辑链”前因后果关系显然;而“算法链”的联结纽带却很隐蔽。联结“算法链”的纽带主要是数学的基本原理与法则,即“算之纲纪”;贯穿《九章》各部分算法的一条总纲便是率的运算法则。
“凡数相与者谓之率。”比率是古代算家最常见的数量关系。用以表示一组比率的数可以“粗者俱粗,细者俱细”,从而可以“乘以散之,约以聚之,齐同以通之”。比率的这种基本运算性质被中算家化为筹式的演算规则,成了贯串《九章》算法理论的一条总的纲纪。

《九章》中形形色色的数学应用问题都可以通过分析其中数量间的比率关系,最终用“今有术”,即四项比例算法求得问题的解答。《九章》中所建立的列衰、返衰、均输、盈朒、方程等各种筹算模式,皆以比率关系为基础。《九章》中数与式的演算差不多都可以归结为比率的遍乘、通约与齐同。在古代算家对于数量关系,即“势”的认识中,把比率看成一切。正如《九章算术》所反映出来的,宇宙之内,天地人物,似乎无一不是比率关系:天圆地方,而圆、方以三、四为率;物物交换,以率相通;得禄出税,依爵次衰分,而列衰,相与率也;徭役摊派,均而输之,而均输者,以行道日约户数为衰,乃衰分之别术也;形与形间,以率相关,圆台之于方台,圆方之率也;“丸居立方,十六分之九也”。凡此种种,天地、人际、物际、形际之间,“其相与之势”(即关系)无一不可用率来表示。虽然在《九章算术》“盈不足”章中,中算家已经遇到了数量间更为复杂的(高次与超越)“关系”,但是他们都毫不例外地将其当作比率关系而用盈不足术去处理。即使对几何量的考察也以比率关系为根本,不仅方圆、周径之率为中算家所特别关注,而且勾股不失本率原理代替了西方的相似形理论成为几何测量的基础,甚至整数勾股弦的一般公式也是用比率的形式来表述的。
中国古代传统数学理论是一种“纲目结构”,纲举目张:目是组成理论之网的眼孔;纲是联结细目的总绳。《九章》以术为目,以率为纲,即是依算法划分理论单元,而用基本的数量关系把它们联结成一个整体。
(二)问题类型与数学模式
众所周知,要应用数学去处理实际问题,首先要把现实问题数学化。由现实问题到数学模型是一个抽象的思维过程,中国传统数学以寻求应用问题的一般解法为宗旨,因而从现实问题中抽象出一般的数学模式并设计出求解的机械化算法,这就构成了中国古算的基本框架。《九章算术》中的数学问题具有典型性,它们往往是现实生活中应用相当广泛的一类问题的代表,经过数学加工、提炼,成为一种特定的数学模型,问题的解法亦具有示范性,有举一反三之效。这与古希腊丢番图的《算术》“几乎没有得出求解的普遍法则”是大不相同的。
例如“方程”就是古代描述多元一次关系的数学模式,它标志着两千年前中算家在发展数学模式化方面所达到的高度。古代的“方程”即是用算筹布列成的数码方阵,每行上列诸数为物率,最下列之数为总实,其行数适与物数相等,它相当于现代多元线性方程组的增广矩阵。盈不足术是古代解一般数学应用问题的别开生面的算法,传入西方被称之为“双假设法”。这种方法的基本思想是,通过两次假设试验将实际应用问题转化为盈朒类数学模型,从而运用相应类型的机械化算法求解,即“现实问题→数学形式化→数学解答”的数学处理过程。
在中国古算中类似于盈不足术这样,用简单的趣味问题来描述一类数学模型是屡见不鲜的。《孙子算经》中的“物不知数”问,是古代著名的一次同余问题的数学模型,它来源于天文测算,古历法中的上元积年推算是它的现实原型。中国古代数学一开始便同天文历法结下不解之缘。在中国数学史上最有影响的“算经十书”,其中最早的《周髀》就是一部天文数学著作。中算史上许多具有世界意义的杰出成就是来自历法推算的。举世闻名的“大衍求一术”(一次同余式组解法)产生于历法上元积年推算;由于推算日月、五星行度的需要,中算家创立了“招差术”(高次内插法);而由于选择历法数据的要求,历算家发展了分数近似法。(21)历法中的算法与太乙术数之类一样,被称为“内算”,秘不外传。正因为如此,流传现今的历代《律历志》中,一般只有历法数据而无算法。(22)不过,历法中的各种算法无疑都能在《九章》中找到它们的理论根源。
(三)中算的构造性与几何算术化的特点《九章》中的题目可谓“有问必答”,它是中国古代传统数学表达方式中不可或缺的组成部分。一般说来,一个问题给出一组数值解,对于不定分析问题则给出多组解。从现代数学的观点来看,这种“有问必答”正反映了构造数学的主要特征。西方传统数学的公理化方法是非构造性的,这种非构造性观点,往往着眼于数学对象的存在性、唯一性和可能性等问题的讨论,而不大关心如何具体求出解答,或将能行的方法付诸有效的实现。构造性观点则要求用有限的方法将所需的数学对象构造出来。这两种观点的根本差异在于对数学对象存在性的不同理解。按照构造性的观点,为了证明存在一个具有性质P的事物X这样的断言,我们必须找出一个有限的方法来构造X,以及找出一个有限的方法来证明X具有性质P。(23)与此截然不同,按公理化方法所做的“纯粹的存在性证明”,即是一个事物X的“存在性”是通过采用指出假设“X不存在”就会导致矛盾的归谬法来证明的。(24)在纯粹的理论研究中,有许多问题一时难以给出构造性处理,而首先讨论其存在性、可能性等问题,自然是有意义的。但是问题最终的解决并付诸实用,还应当是构造性的。《九章算术》中的答案都是由已知的数据按术文给出的算法,经有限步骤的运算而求得的。答案中虽然只列出数值解的数字,但从注释文字可以看出,中算家对于解的存在性与唯一性等问题是有所讨论的。事实上在“方程”求解的过程中,古代算家不可避免地要遇到有解无解,有唯一组解或多组解的不同情况,并且从演算中总结出存在性与唯一性的条件。在这方面刘徽的“九章注”有明确的记述。中算家虽然对于各类应用问题解的性质与构造具有相当深入的认识,但更专注于构造出此类问题求解的一般算法程序。中算“方程术”程序之科学合理,剩余定理构造之精妙完美,都堪称古代构造数学之典范。《九章算术》采用的“有问必答”表达方式,也反映出中国传统数学几何算术化的特点。纯数学是以现实世界的数量关系与空间形式为其研究对象的。中国古代数学包含有丰富的几何内容,中算家在面积、体积和勾股理论方面取得了卓越的成就。然而,与古代希腊几何学迥然不同,中国古代的图形研究表现为数量的计算,它以长度、面积和体积等度量为主要对象,而一般不注重图形性质与位置关系的研究,甚至中国古代几何学不讨论角的性质与度量。几何对象的度量化,使中算“以算为主”的特点得以充分展现。虽然形数结合一般表现为几何方法与代数方法的相互渗透,但中国传统数学中几何算术化始终成为一种主要倾向。在中算家看来,一切几何对象都是可以计算的,几何的结论也可表现为几何对象间的数量关系,因而常常用其擅长的算法来解几何问题,这与古代希腊往往用几何作图的方法来处理代数问题正好相反。如果说古希腊几何学的主要方法是逻辑论证,那么中国古代几何学的基本手段就是数量计算。
(四)算法机械化与寓理于算的特点
中国古代数学称为“算术”,其本义是运用算筹的技术。算筹是我国古代特有的计算工具,它起初是人们随处可取的竹木细枝,后来发展为形制规整、做工精致的骨牙算筹。中国传统数学自始至终都与算器的应用密不可分,对算器有明显的依赖性,以致可以用“筹算”二字来代表中国古代的数学。中国古代从未有过西方那样的笔算;后来算经中的演草也只是筹算推演过程的简要书面记录。算筹是在计算机发明以前我国所独创并且是最有效的计算工具。(25)
由于《九章》并非一部算术的启蒙读物,因此它对记数法没有专门的论述。关于筹算制度较早的记载见于《孙子算经》:“凡算之法,先识其位。一从十横,百立千僵,千十相望,万百相当。”(26)它强调算筹记数纵横相间与位值制原则。筹码记数有纵横二式;它从1至9的数码依次摆成下列形状:
纵式 
横式 个、百、万……等位上的数码用纵式;十、千……等位上的数码用横式;筹码无“零”的符号,而用空位表示。例如“30745”记为“”。筹算记数除了纵横相间与空位表“零”之外,在本质上与现今通行的阿拉伯数字记法没有什么不同。纵横相间有利于辨认数位,以空位表“零”对于算器记数并不是缺点。(27)
中国古代的筹算决不限于单纯的数值计算,而是发展了一套内容十分丰富的“筹式”演算。中算家不仅利用筹码所在不同的“位”来表示不同的“值”,发明了十进位位值制记数法,而且还利用筹在算板上的各种相对位置关系来表示各种特定的数量关系,用以描述某种类型的实际应用问题。演算

对象由“数”发展到“式”,即由数量进到数量关系的研究,后者具有更为一般的代数的性质。中国古代的筹式本身就具有代数符号的一般性的品格,是一种特殊的代数系统。(28)
“术”是中国传统数学表达方式的核心部分。它记述解决所提出的一类问题的普遍方法,实际上就相当于现在计算机科学中的“算法”,在简单的情形下也相当于一个公式或一个定理。术文的内容通常包括如何布筹列式以及对筹式施行的演算程序。如果说算筹是电子计算机的“硬件”,那么中国古代的“算术”就是程序设计的“软件”。中国的筹算不用运算符号,无须保留运算的中间过程,只要求通过筹式的逐步变换而最终获得问题的解答。因此,中国古代数学著作中的“术”,都是用一套一套的“程序语言”所描写的程序化算法。各种不同的筹式都有其基本的变换法则和固定的演算程序,中算家善于运用演算的对称性、循环性等特点,将演算程序设计得十分简捷而巧妙。(29)
“寓理于算”,可以说是中国传统数学理论在表现形式上的一个特点。中算家经常将其依据的算理蕴涵于演算的步骤之中,起到“不言而喻,不证自明”的作用。例如,《九章》中的“方程术”,由于将“方程”的每行视为一组比率(所谓“令每行为率”),于是施行于行列之间的乘除、并减运算便成为比率性质的自然应用,而消元求解的道理也就不言而喻了。(30)此外,算经中篇目的划分、题序的安排,一般也都或多或少地体现出其理论归属或内在的逻辑联系。
如果说西方数学公理化体系贯串一条明显的逻辑链,容易为人们所掌握,那么中国古算的机械化算法体系这种“隐蔽的”理论结构便难于为后世学者所认识。因此,研读《九章》的数学表达方式与理论结构极其重要。正所谓“《九章》翘楚宜详览,‘算术’精微总根源”!
三 《九章算术》的作者、注者及流传版本《九章》与《原本》是大约同一时代的东、西方数学成就的总结。由于年代久远,这两部划时代数学巨著成书的确切年代均已无法断定。《九章》的成书年代曾是中国数学史研究的一大争鸣问题,众说纷纭,以早期钱宝琮主张的“东汉初年成书说”颇具影响。然而,近年来综合中国数学史的深入研究与考古发掘秦汉简帛的整理断定,魏人刘徽关于《九章算术》源流的叙述是有据而可信的。(31)《九章》并非一人一时之作,它集从西周迄秦汉我国古代数学之大成。作为我国古代流传至今的最早一部算经,《九章》是由先秦之遗残经西汉数学家张苍(约公元前250—前152)、耿寿昌(公元前1世纪中叶)等人几番删补而在西汉中期始成定本的。
中国古代数学由于其表达方式与早期书写条件的限制,数学的理论与原理未能诉诸文字,主要靠口授师传。自东汉以来,研习与注释《九章算术》者蜂起,他们或口授,或笔传,师弟相承,世代不绝,代有其人(32),魏晋间人刘徽便是此中杰出的代表。他于魏陈留王景元四年(263)前后所撰的《九章算术注》成为不朽的传世之作。《九章》所采用的“问、答、术”的表达方式,显于“法”而隐于“理”,其中蕴涵的深邃的数学思想与精湛的数学理论只有通过刘徽的注释才得以阐明。而且,与《九章》的经文相比,刘徽的注释语句简略,用字深奥,内容博大精深。
关于刘徽的身世履历、生卒年代均无可详考。根据文献记载推断刘徽为魏晋间人,于263年前后注释《九章算术》(33)。由于他在生前没有显赫的社会地位,因此《晋书》中未能为之立传,据此今人推断刘徽是一位“布衣数学家”。《宋史·礼记》记载,北宋末算学祀典中刘徽曾被追封为“淄乡男”;另外,淄川临近渤海,而刘徽著有测望海岛的《海岛算经》一卷;由此推测他是现今山东淄川(今属山东淄博)一带人。也有文章考证认为祀典中对刘徽的封号是按其籍贯而来的,淄乡在今山东滨州邹平县境内,由此推断刘徽祖籍在今山东邹平,为汉淄乡侯的后代。(34)
刘徽是中国古代杰出的数学家,在中算史上有着重要的地位,他最重要的贡献就是注释《九章算术》。刘徽所撰写的《九章算术注》(以下亦称《九章注》)十卷与《九章重差图》一卷,是中国数学史上划时代的著作。《九章算术注》十卷到唐代演变为《九章算术注》九卷与《海岛算经》一卷,《九章重差图》可惜在宋代已经失传。刘徽的《九章算术注》自问世以来便一直受到后世学人的推崇,被视为研习《九章》所必读。
刘徽的学术贡献可以概括为三大项:注《九章》,撰“重差”,创“割圆”。如果说第一项主要是对前贤学术传统的继承与阐发,那么后两项当归之为刘徽个人的发展与创新了。
刘徽是集前人之大成者,《九章注》中固然不乏其个人的独创,但其中的基本原理与论证方法大多还是前人已经有的。从理论的总体上看,刘徽注最杰出的成就便是它揭示出《九章算术》是一个“以率为纲”的算法理论体系。刘徽注“以率为纲”,用比率来解释《九章》中的各种算法,使其算理显豁而自然,收到了“纲举目张”之效。虽然,比率理论决非刘徽之首创,不过从留传后世的文献看,阐发得最为系统精辟的当推刘徽注了。
从传统几何学方面看,刘徽注的又一杰出成就在于它提炼出关于图形的基本数学原理,用以论证《九章》中的度量几何学公式。“出入相补”原理是对明显几何事实的概括,刘徽注用“出入相补”来论证《九章·方田》中各种简单直线形的面积公式。用“方”的分割来说明开方术的演算步骤,其根据仍在“出入相补”。虽然出入相补原理不能归之于刘徽所首创,但刘徽在应用这一原理方面有其发展与创新,其对整数勾股弦公式的几何论证就是精彩的一例。(35)他将多面体体积计算转化为其中所含立方个数的计算,从而实现了几何的算术化,其关键在于证明基本几何体之间的简单倍数关系,刘徽注运用极限观念完成了这一理论的严格证明。
截面原理素为中算家所熟悉与应用,成为中国古算求积理论的重要基础,刘徽与祖暅对这一原理都有广泛而精妙的应用。叠线成面,叠面成体乃是中算家传统的几何观念。对截面原理最成功的应用乃是关于球体积的计算。刘徽凭借对截面原理的深知与对图形性质的谙熟,发明了“牟合方盖”为球积公式的探求设计出正确的方案。可惜他功亏一篑未能计算出“方盖”之积,这一永载史册的创造最后由南北朝时代的祖暅最终完成。
刘徽的第二个学术贡献是撰“重差”。汉代的重差术其时似已失传,经过他的反复探究,终于“辄造重差,并为注解,以究古人之意,缀于勾股之下”。他的《海岛算经》原是作为《九章注》的延展部分撰写的,唐初才另本单行。《九章注·序》中以大半篇幅论述“重差”的意义及其由来,足见刘徽对它的重视。刘徽将源于窥天之重差施之于测地,类推衍化发展成《海岛》九术,把古代测量技术推向一个高峰。其重差造术之根据在于勾股比率,以两个差率代替勾、股之率是其中的关键。(36)“而且,重差理论中以量长代替角的测量的这一方法所隐含的以多次简易测量代替较难测量的原理,在现代的各种技术问题上可能还是有现实意义的。”(37)
“割圆术”是《九章》中最长的一条注释,它用以论证“圆田术”,即圆的面积公式,并由此推算出较为精密的圆周率。刘徽术最可宝贵的创新在于“割圆拼方”与极限方法的运用。割圆拼方是刘徽割圆术的基本思想,他从中算家所熟悉的六觚,即圆内接正六边形出发,逐次等分圆周,将圆裁为边数依次倍增的觚形,而它又可以移补成半觚周为从、边心距为广的长方形。显然这种割圆拼方是一个近似的过程,圆面积在裁割中是有所失的,但当割圆为觚达到极限时,作为边心距与半径之差的“余径”便消失为零。于是边心距伸展为半径,半觚周伸展为半圆周,以半径为广、半周为从的长方形便与圆面积相等。这便是刘徽用极限方法证明圆积公式的过程大要。此外,刘徽依割圆所得筝形中所含小勾股弦与圆径、觚面之关系,列出依次计算边心距及余径,觚面与觚幂的循环递推程序,并由此逐步算得一百九十二觚之幂,从而推出圆周率,进而又“以率消息”得到更精确之圆率。把“割圆术”列为刘徽三大学术贡献之一,足见数学史界对此项古代出色创作的推崇。
唐初以后,《九章》不仅作为国家颁行的主要数学教科书为莘莘学子传诵研习,而且远传朝鲜、日本及南亚诸国。明代中叶以后,随着中国传统数学的衰微,《九章》与刘徽注的内容已鲜为人知。自清代中叶戴震、李潢等以来对《九章》的整理与研究,实际上已经属于数学史的范畴了。《九章算术》在我国历经两千多年辗转抄录、翻刻,其版本流传情况已不可详考。早在北宋元丰七年(1084)便有官家秘书省刻本,而流存迄今的最早善本是南宋嘉定六年(1213)鲍澣之在汀州的翻刻本(简称“南宋本”或“鲍刻本”)(38)。明代永乐六年(1408)编成的《永乐大典》,在“算”字条下分类抄录了《九章算术》的内容(简称“大典本”)(39)。南宋末年数学家杨辉的《详解九章算法》是《九章》的另一种版本,它“录经、注原文于前,而以其所撰题解、释注、比类、图说分附各条之后”(简称“杨辉本”)(40)。南宋本、大典本和杨辉本是《九章》流传迄今的三个明代以前的古本,可惜皆残缺不全。南宋本仅存方田、粟米、衰分、少广、商功五章;大典本现已大部分散佚;杨辉本只保存了商功、均输、盈不足、方程、勾股这后五章的内容(41)。
清代以来广为流传的《九章算术》的几个版本,主要是经著名学者戴震(1724—1777)辑录校勘的。清乾隆三十八年(1773)开四库全书馆,戴氏充《四库全书》纂修及分校官。次年,他从《永乐大典》中抄集《九章算术》九卷,并做了一番校勘工作,从而在辑录校注大典本的基础上,形成了四库全书本(简称“四库本”,成于1784年)(42)和武英殿聚珍版本(简称“殿本”,初刊于1774年)(43)《九章算术》。后者几经翻刻,并逐渐掺入了他人的校改,虽仍冠以“武英殿聚珍版”的名号,但前后各版多有不同(44)。1684年常熟汲古阁主人毛扆从南京黄虞稷处借得半部南宋本《九章算术》,影雕成汲古阁本《九章算经》(45)。1776年孔继涵得到汲古阁本,尔后成为戴震先后校订屈刻本和孔刻本的根据。屈刻本即乾隆四十一年(1776)常熟算家屈曾发刊刻的《九章算术》与《海岛算经》的合刊本,题称“豫簪堂藏版”。孔刻本即曲阜孔继涵刻《九章算术》,它是微波榭本《算经十书》之一。戴震校订屈、孔二氏的刻本大约是在1776年仲秋至1777年春他逝世之前。(46)此后依据微波榭本翻刻的《九章算术》有南昌梅启照的《算经十书》本和商务印书馆的《万有文库》本、《四部丛刊》本,等等。
在戴震之后,著名算家李潢(?—1812)以微波榭本为蓝本,撰成《九章算术细草图说》,在戴氏工作的基础上又校正了许多错误文字,并对难读的部分作图说、演细草。其中采用了李锐、戴敦元等人的校订稿的一部分,沈钦裴于李潢去世后遵嘱代为校算编辑,付梓前又做了个别的校正。李潢本后来亦多次被翻刻,是19世纪颇有影响的一个《九章》版本。
20世纪六十年代,钱宝琮以恢复唐代立于学宫的刘、李注本《九章算术》为宗旨,根据《天禄琳琅丛书》本和宜稼堂本《详解九章算法》,重新校点而收入中华书局出版的《算经十书》之中,成为20世纪中出现的第一个新的《九章算术》校勘本(简称“钱校本”)。1983年科学出版社出版了白尚恕的《〈九章算术〉注释》(简称“白注本”),它以钱校本为蓝本,参考各家之说,用通俗语言、近代数学术语对《九章》及刘、李注文详加注释。1991年仲夏,在本书初稿写成之后又见到了郭书春的汇校本《九章算术》,不久又得见白尚恕的《九章算术今译》(简称“今译本”)。这些著作作为《九章算术》的新版本,对于提高古算典籍整理的水平和推动《九章》及刘注研究向深层次发展无疑都产生了积极的作用。




在很长时间内,西方主流学术界一直认为东方(主要是中国)并没有真正意义上的数学,即没有一个严格的、成系统的公理化演绎体系,正如他们认为古代中国的哲思只是一些道德训诫,至多有一些思辨概念的萌芽。近代西风东渐以来,中国的知识分子因大的政经局势与民族富强动力使然,无暇深入探究中西致思方式内在、深刻的差异,就接受了西方学界的主流观念,认为研究中学的终南捷径在于掌握一整套来自于西方哲学与科学的概念范畴体系,以便把中学的某一学科(如数学、哲学、中医等)套入其间,以能入其窠臼为能事。这种研究、思考方式当然有其不可忽视的现实原因,但由于它完全不考虑,更不立足于中西方思维方式的内在差异,一味以中学向西学比附、看齐,最后只能是缘木求鱼,不得结果,或以己之短比人之长,以西学的思维模式湮没中学的致思方式。

这种研究中学的思考方式自近代以来发轫,至近些年为止仍未有根本的扭转。为今之计在于认真研读原典,努力接近和体会古人的致思方式。对于中国古代数学(算学),我们尤应研读其本源之作《九章算术》。《九章算术》向我们昭示,中国古代的数学是一个完全不同于西方数学公理化演绎体系的自成系统的体系。在饱受西方数学训练的学人看来,中国古代数学似乎缺失一个严格、完整的从公理开始渐次推演、下降的逻辑体系。但这并不意味着古人缺乏创立这样体系的智能水平,而是他们并不追求这样的体系。西方科学继承了自古希腊而来的追根究底的形而上学冲动,其彻底的、反思的特质自有其庄严、动人之处,但其理论前提不断被质疑、冲击,又从反面说明了其试图以公理化的逻辑体系统摄一学科中全部现象的冲动,是一个“不可完成的任务”。而以《九章算术》为代表的中国古代数学,以西学的思维方式来看,似乎不太像科学意义上的“数学”,而有些近似于算术“游戏”。它的着力点完全不在于建立能够完全涵盖某一确定领域的抽象程度甚高的公理、定理、公式,而是在于具体情境中的“涵泳”“玩味”。正如尤为体现中国式致思方式的围棋,其对于人类智能的要求与开掘,实不逊于哲学、数学、物理学等基础理论学科,但的确很难为其推导或总结出具有严格适用范围的公式、定理。其所谓“定式”不过只是在某一具体局部形势中目前发现的较为合理、有效的下法。一切有关围棋的智慧都发生在对当下具体形势的直观中。有时,它在某一范围内的贯通性,并非凭借抽象的公理、定理和公式,而是对具体情境或形势的直觉能力。这种直觉力恰恰是在境域式的算术“游戏”中涵养濡育而成的。这种发生于境域中的直觉力直面一切具体而微的数学活动,在某种意义上比西方数学更能够进入到数学活动的本质深处。我们如果仔细研究《九章算术》,会看到它在数学上的创见当不止于发现了可与西方相比拟的勾股定理之类,而沿着这一中国独特的致思方式继续“玩味”下去,我们会在数学以及人类思想活动的诸基本领域内不断贡献出真正有价值的智慧成果。

本书为全本全译,以郭书春的《九章算术译注》(上海古籍出版社,“中国古代科技名著译注丛书”)为底本。由于个人水平有限,其中不免有讹误之处,还请方家批评指正。

豫簪堂清乾隆41年[1776]

\chapter{九章算术序}

\begin{shuming}
刘徽撰
\end{shuming}

刘徽:魏晋期间伟大的数学家,中国古典数学理论的奠基者之一,著有《九章算术注》和《海岛算经》。

刘徽是《九章算术》成书后第一个重要的注释者。魏晋之后,所有《九章算术》的版本,都采用了刘徽的注释。刘徽在《九章算术》之前写了本篇序言,后世的所有版本也都予以收录。

刘徽的序言首先说明了《九章算术》的版本来源。《九章算术》的基本框架,源于周公的“九数”。秦始皇焚书坑儒后,西周流传的《九章算术》经文散乱缺失。西汉时期,经过张苍与耿寿昌的编辑、校补,大体上形成了后世流传的《九章算术》的版本。

接着,刘徽陈述了他注释《九章算术》的两条主要原则。

其一,“事类相推,各有攸归”。《九章算术》是一部问题集,针对一些同类的问题,会设计一个算法,称之为“术”。这些“术”看起来似乎是孤立的,但是,就其数学本质而言,很多“术”的构造原理是相同的。例如,《九章算术》的很多算法,都是建立在“率”这个更基本的概念之下构造出来的。刘徽注的一个重要的贡献,是深刻地揭示了不同算法间所共同拥有的基本原理。

其二,“析理以辞,解体用图”。《九章算术》的所有“术”,基本上都是算法的陈述,没有给出这些“术”的构造原理或证明过程。刘徽针对几乎所有的重要的“术”,都采用逻辑语言与模型图解,补充给出了严格、清晰的推导或证明,如割圆术、开方术、阳马术等。

刘徽序言的最后一个部分,主要阐述了他自己在数学上的一个重要的创建,即“重差术”。为了说明“重差术”的应用,刘徽撰写了《重差》一卷,作为《九章算术》的第十卷,缀于“勾股”之后。唐代李淳风编辑《算经十书》时,将《重差》作为算经之一种独立成书,因《重差》的第一问是测算海岛的高远,因之命名为《海岛算经》。

\begin{yuanwen}
昔在包犠氏\footnote{包(páo)牺氏:即伏羲氏,古代传说中的三皇之一,风姓。相传其始画八卦,又教民渔猎,取牺牲以供庖厨,因称庖牺。亦作“伏犠”、“伏戏”。他与女娲婚配而产生人类。他们同被尊为人类始祖。《周易·系辞下》:“古者包犠氏之王天下也,仰则观象于天,俯则观法于地……于是始作八卦。”包,通“庖”。}始\footnote{开始,最初。}画八卦\footnote{八卦:《周易》中的八种具有象征意义的基本图形,每个图形用三个分别代表阳的“—”(阳爻)和代表阴的“--”(阴爻)符号组成。名称是:乾(☰)、坤(☷)、震(☳)、巽(☴)、坎(☵)、离(☲)、艮(☶)、兑(☱)。《易传》作者认为八卦主要象征天、地、雷、风、水、火、山、泽八种自然现象,并认为乾、坤两卦在八卦中占有特别重要的地位,是自然界和人类社会一切现象的最初根源。八卦中,乾与坤、震与巽、坎与离、艮与兑是四个矛盾对立的形态。传说周文王将八卦互相组合,又得六十四卦,用来象征自然现象和社会现象的发展变化。我国古代的八种有象征意义的符号,组成《易》的基本图像。由阳爻和阴爻排列而成,每三根爻组成一卦,其名称为:☰(乾),☷(坤),☳(震),☴(巽),☵(坎),☲(离),☶(艮),☱(兑)。},以通神明之德,以类万物之情,作九九之术\footnote{即九九乘法法则。古时由“九九”自上而下,而至“一一”,故称“九九乘法”。},以合六爻\footnote{为了表示更多的事物或现象,将八卦中的两卦按照一上一下的方式组合,构成“复卦”。复卦共有六个爻位,因此又称“六爻”。}之变。暨\footnote{至,到。}于黄帝神而化之,引而伸之,于是建历纪\footnote{历数纲纪。},协律吕\footnote{中国古代律制“十二律”,又名“正律”,简称“律吕”。},用稽\footnote{考核,计数。}道原,然后两仪\footnote{即阴、阳。}四像\footnote{即太阴(水)、少阳(木)、少阴(金)、太阳(火)。}精微之气可得而效焉。记称隶首\footnote{黄帝史官,始作算数。后世以“隶首之学”指算学。}作数,其详未之闻也。按周公\footnote{姓姬名旦,因封地在周,故称周公。西周初期杰出的政治家、军事家和思想家。}制礼\footnote{《周礼》,儒家经典。}而有九数\footnote{指古代数学功课的九个细目。东汉的郑玄在他的《周礼注疏·地官司徒·保氏》中引郑司农(郑众)所言:“九数:方田、粟米、差分、少广、商功、均输、方程、赢不足、旁要;今有重差、夕桀、勾股也。”},九数之流,则《九章》是矣。

往者暴秦焚书\footnote{公元前213年,秦始皇接受丞相李斯的建议,下令除《秦纪》、医药、卜筮、种树之书外,其他如百家语、《诗》、《书》等书限期交官府烧毁。焚书事件对中华文化造成了极大的破坏。},经术散坏。自时厥后,汉北平侯张苍\footnote{西汉丞相,曾校正《九章算术》,制定历法。}、大司农\footnote{汉代官名,负责掌管赋税、盐、铁、酒的制作专卖,漕运、调拨物资和国家财政。}中丞耿寿昌\footnote{西汉天文学家,理财家。汉宣帝时任大司农中丞,精通数学,修订《九章算术》。用铜铸造浑天仪观天象。著有《月行帛图》232卷,《月行度》2卷,今已佚。}皆以善算命世。苍等因旧文之遗残,各称删补。故校其目,则与古或异,而所论者多近语也。
\end{yuanwen}

以前庖牺氏最先画出八卦,用来通达并获得天地神明的美好品质,模仿世间万物的情状。后来又作九九之术,来配合六爻的变化。直到黄帝神妙地将其变化引申,于是建立历法纲纪、调正音律,用来考察道的本原,而后两仪四象的精髓可以被获得并且效法。曾有记载说“隶首创立了算学”,但我没有听说过其中的详细情节。按:周公制定礼乐制度,其中有九数,九数后来发展成《九章算术》。

过去残暴的秦始皇焚书,造成经术散坏。后来,汉代的北平侯张苍、大司农中丞耿寿昌都以擅长算学闻名于世。张苍等根据旧时的残缺遗文,进行删减补充。所以,它的目录与古代版本有些许不同,论述则多采用近代语言。

\begin{yuanwen}
徽幼习《九章》,长再详览。观阴阳\footnote{古人为了解释自然界中各种对立又相关联的大自然现象,如天地、日月、昼夜、寒暑、男女、上下等,归纳出“阴阳”的概念。}之割裂\footnote{区别。},总算术之根源,探赜\footnote{幽深,玄妙。}索隐,遂悟其意。是以敢竭顽鲁,采其所见,为之作注。事类相推,各有攸\footnote{所。}归\footnote{归属。},故枝条虽分而同本榦者知,发其一端而已。又所析理以辞,解体用图,庶亦约而能周\footnote{周密。},通而不黩\footnote{过多。},览之者,思过半矣。且算在六艺\footnote{古代儒家要求学生掌握的六种基本才能,即礼、乐、射、御、书、数。},古者以宾兴\footnote{周代举贤之法。谓乡大夫自乡小学荐举贤能而宾礼之,以升入国学。兴,兴举。}贤能,教习国子\footnote{公卿大夫的子弟。};虽曰九数,其能穷纤入微,探测无方;至于以法相传,亦犹规矩度量可得而共,非特难为也。当今好之者寡,故世虽多通才达学,而未必能综于此耳。
\end{yuanwen}

我幼年时学习《九章算术》,年长后又详细钻研。观察事物的正反区别,总结了算数的根源,在探索这些幽深玄妙的道理之余,逐渐领悟到其中的思想。于是冒昧地竭尽愚钝,收集我见到的资料,为它作注释。事物之间可以相互类推,分别有各自的归属。所以枝条虽然分离却具有同一个主干,原因是它们发于同一开端。再加上用文辞分析数理,用图形解释立体,就会使它简明且周密,通顺且不烦琐,阅读的人可以懂得一半以上的内容。算学属于六艺之一,古代用来兴举贤能之人和教育贵族子弟。虽然称为九数,却可以尽到极小极微,探索到无穷无尽。至于流传下来的方法,就像规矩度量一样存在而有共性,所以学习它并非很困难的事情。目前喜爱算学的人很少,所以世上虽然有很多学识渊博的人,却未必对它精通。
============================
\begin{yuanwen}
《周官\footnote{即《周礼》,儒家经典。}·大司徒\footnote{六官之一,管理土地及户口。}》职\footnote{掌管。},夏至日中立八尺之表\footnote{测量日影的标杆。}。其景\footnote{y\v{i}ng ,影。}尺有五寸,谓之地中。说云,南戴日下万五千里。夫云尔者,以术推之。\footnote{“说云”两句:郑玄《周礼注》中的文字。}案(按):《九章》立四表望远及因木望山之术,皆端旁互见,无有超邈若斯之类。然则苍等为术犹未足以博尽群数也。徽寻九数有重差之名,原其指趣\footnote{宗旨,意义。}乃所以施于此也。

凡望极高、测绝深而兼知其远者必用重差、勾股,则必以重差为率,故曰重差也。立两表于洛阳之城,令高八尺,南北各尽平地。同日度其正中之时。以景差为法\footnote{除数。},表高乘表间为实\footnote{被除数。},实如法而一。所得加表高,即日去\footnote{距离。}地也。以南表之景乘表间为实,实如法而一,即为从南表至南戴日下也。以南戴日下及日去地为勾、股,为之求弦,即日去人也。以径寸之筒南望日,日满筒空,则定筒之长短以为股率,以筒径为勾率,日去人之数为大股,大股之句即日径也。虽夫圆穹之象犹曰可度,又况泰山之高与江海之广哉。徽以为今之史籍且略举天地之物,考论厥数,载之于志,以阐世术之美,辄造《重差》,并为注解,以究古人之意,缀于勾股之下。度高者重表,测深者累矩,孤离者三望,离而又旁求者四望。触类而长之,则虽幽遐\footnote{僻远;深幽。}诡伏\footnote{隐藏不露。},靡\footnote{无。}所不入,博物\footnote{通晓众物。}君子,详而览焉。
\end{yuanwen}

《周礼》中规定大司徒的职责之一,就是在夏至日的正午立一根8尺长的标杆,将影长是1尺5寸的地方,定为大地的中心。《周礼注》中说,此刻太阳在南方15000里处。这个结论可以由术推算出来。按:《九章算术》中有立四根标杆求距离和根据树木求山高的方法,都是在近处设参照物相对应,还没有距离这么遥远的。如此看来,张苍等人的计算方法不足以涵盖所有的数学方法。我发现九数中有“重差”这一项目,它原本的宗旨是为了解答这类问题。凡是测量极高、极深又求它们的远近距离的情况,必须用重差、勾股。由于要以两次直角边相当边的差数作为率,所以称为“重差”。在洛阳城立两根标杆,高8尺,使它们处于南北方向的同一水平面上。同一天正午的时候测量它们的影长。以它们的影长之差作为除数,标杆的高度乘标杆间的距离作为被除数,被除数除以除数,所得之数加上标杆的高度,即为太阳到地面的距离。以南标杆的影长乘标杆间的距离为被除数,除数不变,所得即为南标杆到太阳直射处的距离。分别以南标杆到太阳直射处的距离和太阳到地面的距离作为勾、股,求得的弦即为太阳到人的距离。以直径1寸的竹筒向南观望太阳,阳光充满竹筒内的空间。以筒的长度为股率,筒的直径为勾率,太阳到人的距离为大股,与大股相对应的勾即为太阳的直径。即使是天象都可以测量,更何况是泰山的高度和江海的宽度呢。我认为当今的史籍已经有了一些对天地间事物的记录,并且考论它们的数量,记载在志书中,展示了世间算学的美妙。于是我写《重差》,并为之作注解,以探究古人的本意,附于《勾股》之后。测量高度用两根标杆,测量深度用多次矩尺,对孤立的测量点需要观测三次,对孤立且要求解决其他问题的需要观测四次。如果触类旁通,即使问题深远隐秘,也没有不能解决的。博学的君子们,请仔细阅读这本书吧。

\mainmatter
\chapter{方田}

(以御田畴界域) 今有田广十五步,从十六步。问为田几何?答曰:一亩。

又有田广十二步,从十四步。问为田几何?答曰:一百六十八步。

〔图:从十四,广十二。〕 方田术曰:广从步数相乘得积步。

〔此积谓田幂。凡广从相乘谓之幂。

淳风等按:经云广从相乘得积步,注云广从相乘谓之幂。观斯注意,积幂义 同。以理推之,固当不尔。何则?幂是方面单布之名,积乃众数聚居之称。循名责实,二者全殊。虽欲同之,窃恐不可。今以凡言幂者据广从之一方;其言积者 举众步之都数。经云相乘得积步,即是都数之明文。注云谓之为幂,全乖积步之 本意。此注前云积为田幂,于理得通。复云谓之为幂,繁而不当。今者注释,存 善去非,略为料简,遗诸后学。〕 以亩法二百四十步除之,即亩数。百亩为一顷。

〔淳风等按:此为篇端,故特举顷、亩二法。余术不复言者,从此可知。一 亩之田,广十五步,从而疏之,令为十五行,则每行广一步而从十六步。又横而 截之,令为十六行,则每行广一步而从十五步。此即从疏横截之步,各自为方, 凡有二百四十步。一亩之地,步数正同。以此言之,则广从相乘得积步,验矣。

二百四十步者,亩法也;百亩者,顷法也。故以除之,即得。〕 今有田广一里,从一里。问为田几何?答曰:三顷七十五亩。

又有田广二里,从三里。问为田几何?答曰:二十二顷五十亩。

里田术曰:广从里数相乘得积里。以三百七十五乘之,即亩数。

〔按:此术广从里数相乘得积里。方里之中有三顷七十五亩,故以乘之,即 得亩数也。〕 今有十八分之十二,问约之得几何?答曰:三分之二。

又有九十一分之四十九,问约之得几何?答曰:十三分之七。

○约分 〔按:约分者,物之数量,不可悉全,必以分言之;分之为数,繁则难用。

设有四分之二者,繁而言之,亦可为八分之四;约而言之,则二分之一也,虽则 异辞,至于为数,亦同归尔。法实相推,动有参差,故为术者先治诸分。〕 术曰:可半者半之;不可半者,副置分母、子之数,以少减多,更相减损, 求其等也。以等数约之。

〔等数约之,即除也。其所以相减者,皆等数之重叠,故以等数约之。〕 今有三分之一,五分之二,问合之得几何?答曰:十五分之十一。

又有三分之二,七分之四,九分之五,问合之得几何?答曰:得一、六十三 分之五十。

又有二分之一,三分之二,四分之三,五分之四,问合之得几何?答曰:得 二、六十分之四十三。

○合分 〔淳风等按:合分知,数非一端,分无定准,诸分子杂互,群母参差。粗细 既殊,理难从一,故齐其众分,同其群母,令可相并,故曰合分。〕 术曰:母互乘子,并以为实。母相乘为法。

〔母互乘子。约而言之者,其分粗;繁而言之者,其分细。虽则粗细有殊, 然其实一也。众分错杂,非细不会。乘而散之,所以通之。通之则可并也。凡母 互乘子谓之齐,群母相乘谓之同。同者,相与通同,共一母也;齐者,子与母齐, 势不可失本数也。方以类聚,物以群分。数同类者无远;数异类者无近。远而通 体知,虽异位而相从也;近而殊形知,虽同列而相违也。然则齐同之术要矣:错 综度数,动之斯谐,其犹佩觿解结,无往而不理焉。乘以散之,约以聚之,齐同 以通之,此其算之纲纪乎?其一术者,可令母除为率,率乘子为齐。〕 实如法而一。不满法者,以法命之。

〔今欲求其实,故齐其子,又同其母,令如母而一。其余以等数约之,即得 知,所谓同法为母,实余为子,皆从此例。〕 其母同者,直相从之。

今有九分之八,减其五分之一,问余几何?答曰:四十五分之三十一。

又有四分之三,减其三分之一,问余几何?答曰:十二分之五。

○减分 〔淳风等按:诸分子、母数各不同,以少减多,欲知余几,减余为实,故曰 减分。〕 术曰:母互乘子,以少减多,余为实。母相乘为法。实如法而一。

〔母互乘子知,以齐其子也。以少减多知,齐故可相减也。母相乘为法者, 同其母也。母同子齐,故如母而一,即得。〕 今有八分之五,二十五分之十六,问孰多?多几何?答曰:二十五分之十六 多,多二百分之三。

又有九分之八,七分之六,问孰多?多几何?答曰:九分之八多,多六十三 分之二。

又有二十一分之八,五十分之十七,问孰多?多几何?答曰:二十一分之八 多,多一千五十分之四十三。

○课分 〔淳风等按:分各异名,理不齐一,较其相近之数,故曰课分也。〕 术曰:母互乘子,以少减多,余为实。母相乘为法。实如法而一,即相多也。

〔淳风等按:此术母互乘子,以少分减多分,与减分义同;惟相多之数,意 与减分有异:减分知,求其余数有几;课分知,以其余数相多也。〕 今有三分之一,三分之二,四分之三。问减多益少,各几何而平?答曰:减 四分之三者二,三分之二者一,并,以益三分之一,而各平于十二分之七。

又有二分之一,三分之二,四分之三。问减多益少,各几何而平?答曰:减 三分之二者一,四分之三者四、并,以益二分之一,而各平于三十六分之二十三。

○平分 〔淳风等按:平分知,诸分参差,欲令齐等,减彼之多,增此之少,故曰平 分也。〕 术曰:母互乘子, 〔齐其子也。〕 副并为平实。

〔淳风等按:母互乘子,副并为平实知,定此平实主限,众子所当损益知, 限为平。〕 母相乘为法。

〔母相乘为法知,亦齐其子,又同其母。〕 以列数乘未并者各自为列实。亦以列数乘法。

〔此当副置列数除平实,若然则重有分,故反以列数乘同齐。

淳风等按:问云所平之分多少不定,或三或二,列位无常。平三知,置位三 重;平二知,置位二重。凡此之例,一准平分不可豫定多少,故直云列数而已。〕 以平实减列实,余,约之为所减。并所减以益于少。以法命平实,各得其平。

今有七人,分八钱三分钱之一。问人得几何?答曰:人得一钱二十一分钱之 四。

又有三人三分人之一,分六钱三分钱之一、四分钱之三。问人得几何?答曰: 人得二钱八分钱之一。

○经分 〔淳风等按:经分者,自合分已下,皆与诸分相齐,此乃直求一人之分。以 人数分所分,故曰经分也。〕 术曰:以人数为法,钱数为实,实如法而一。有分者通之。

〔母互乘子知,齐其子;母相乘者,同其母。以母通之者,分母乘全内子。

乘,散全则为积分,积分则与子相通,故可令相从。凡数相与者谓之率。率知, 自相与通。有分则可散,分重叠则约也;等除法实,相与率也。故散分者,必令 两分母相乘法实也。〕 重有分者同而通之。

〔又以法分母乘实,实分母乘法。此谓法、实俱有分,故令分母各乘全分内 子,又令分母互乘上下。〕 今有田广七分步之四,从五分步之三,问为田几何?答曰:三十五分步之十 二。

又有田广九分步之七,从十一分步之九,问为田几何?答曰:十一分步之七。

又有田广五分步之四,从九分步之五,问为田几何?答曰:九分步之四。

○乘分 〔淳风等按:乘分者,分母相乘为法,子相乘为实,故曰乘分。〕 术曰:母相乘为法,子相乘为实,实如法而一。

〔凡实不满法者而有母、子之名。若有分,以乘其实而长之,则亦满法,乃 为全耳。又以子有所乘,故母当报除。报除者,实如法而一也。今子相乘则母各 当报除,因令分母相乘而连除也。此田有广从,难以广谕。设有问者曰:马二十 匹,直金十二斤。今卖马二十匹,三十五人分之,人得几何?答曰:三十五分斤 之十二。其为之也,当如经分术,以十二斤金为实,三十五人为法。设更言马五 匹,直金三斤。今卖马四匹,七人分之,人得几何?答曰:人得三十五分斤之十 二。其为之也,当齐其金、人之数,皆合初问入于经分矣。然则分子相乘为实者, 犹齐其金也;母相乘为法者,犹齐其人也。同其母为二十,马无事于同,但欲求 齐而已。又,马五匹,直金三斤,完全之率;分而言之,则为一匹直金五分斤之 三。七人卖四马,一人卖七分马之四。金与人交 互相生。所从言之异,而计数则 三术同归也。〕 今有田广三步三分步之一,从五步五分步之二,问为田几何?答曰:十八步。

又有田广七步四分步之三,从十五步九分步之五,问为田几何?答曰:一百 二十步九分步之五。

又有田广十八步七分步之五,从二十三步十一分步之六,问为田几何?答曰: 一亩二百步十一分步之七。

○大广田 〔淳风等按:大广田知,初术直有全步而无余分;次术空有余分而无全步; 此术先见全步,复有余分,可以广兼三术,故曰大广。〕 术曰:分母各乘其全,分子从之, 〔分母各乘其全,分子从之者,通全步内分子。如此则母、子皆为实矣。〕 相乘为实。分母相乘为法。

〔犹乘分也。〕 实如法而一。

〔今为术广从俱有分,当各自通其分。命母入者,还须出之,故令分母相乘 为法而连除之。〕 今有圭田广十二步,正从二十一步,问为田几何?答曰:一百二十六步。

又有圭田广五步二分步之一,从八步三分步之二,问为田几何?答曰:二十 三步六分步之五。

术曰:半广以乘正从。

〔半广知,以盈补虚为直田也。亦可半正从以乘广。按:半广乘从,以取中 平之数,故广从相乘为积步。亩法除之,即得也。〕 今有邪田,一头广三十步,一头广四十二步,正从六十四步。问为田几何? 答曰:九亩一百四十四步。

又有邪田,正广六十五步,一畔从一百步,一畔从七十二步。问为田几何? 答曰:二十三亩七十步。

术曰:并两斜而半之,以乘正从若广。又可半正从若广,以乘并。亩法而一。

〔并而半之者,以盈补虚也。〕 今有箕田,舌广二十步,踵广五步,正从三十步,问为田几何?答曰:一亩 一百三十五步。

又有箕田,舌广一百一十七步,踵广五十步,正从一百三十五步,问为田几 何?答曰:四十六亩二百三十二步半。

术曰:并踵、舌而半之,以乘正从。亩法而一。

〔中分箕田则为两邪田,故其术相似。又可并踵、舌,半正从,以乘之。〕 今有圆田,周三十步,径十步。

〔淳风等按:术意以周三径一为率,周三十步,合径十步。今依密率,合径 九步十一分步之六。〕 问为田几何?答曰:七十五步。

〔此于徽术,当为田七十一步一百五十七分步之一百三。

淳风等按:依密率,为田七十一步二十三分步之一十三。〕 又有圆田,周一百八十一步,径六十步三分步之一。

〔淳风等按:周三径一,周一百八十一步,径六十步三分步之一。依密率, 径五十七步二十二分步之一十三。〕 问为田几何?答曰:十一亩九十步十二分步之一。

〔此于徽术,当为田十亩二百八步三百一十四分步之一百十三。

淳风等按:依密率,当为田十亩二百五步八十八分步之八十七。〕 术曰:半周半径相乘得积步。

〔按:半周为从,半径为广,故广从相乘为积步也。假令圆径二尺,圆中容 六觚之一面,与圆径之半,其数均等。合径率一而外周率三也。

又按:为图,以六觚之一面乘一弧半径,三之,得十二觚之幂。若又割之, 次以十二觚之一面乘一弧之半径,六之,则得二十四觚之幂。割之弥细,所失弥 少。割之又割,以至于不可割,则与圆周合体而无所失矣。觚面之外,又有余径。

以面乘余径,则幂出觚表。若夫觚之细者,与圆合体,则表无余径。表无余径, 则幂不外出矣。以一面乘半径,觚而裁之,每辄自倍。故以半周乘半径而为圆幂。

此一周、径,谓至然之数,非周三径一之率也。周三者,从其六觚之环耳。以推 圆规多少之觉,乃弓之与弦也。然世传此法,莫肯精核;学者踵古,习 其谬失。

不有明据,辩之斯难。凡物类形象,不圆则方。方圆之率,诚著于近,则虽远可 知也。由此言之,其用博矣。谨按图验,更造密率。恐空设法,数昧而难譬,故 置诸检括,谨详其记注焉。

割六觚以为十二觚术曰:置圆径二尺,半之为一尺,即圆里觚之面也。令 半径一尺为弦,半面五寸为句,为之求股。以句幂二十五寸减弦幂,余七十五寸, 开方除之,下至秒、忽。又一退法,求其微数。微数无名知以为分子,以十为分 母,约作五分忽之二。故得股八寸六分六厘二秒五忽五分忽之二。以减半径,余 一寸三分三厘九毫七秒四忽五分忽之三,谓之小句。觚之半面又谓之小股。为之 求弦。其幂二千六百七十九亿四千九百一十九万三千四百四十五忽,余分弃之。

开方除之,即十二觚之一面也。

割十二觚以为二十四觚术曰:亦令半径为弦,半面为句,为之求股。置上 小弦幂,四而一,得六百六十九亿八千七百二十九万八千三百六十一忽,余分弃之, 即句幂也。以减弦幂,其余开方除之,得股九寸六分五厘九毫二秒五忽五分忽之 四。以减半径,余三分四厘七秒四忽五分忽之一,谓之小句。觚之半面又谓之小 股。为之求小弦。其幂六百八十一亿四千八百三十四万九千四百六十六忽,余分 弃之。开方除之,即二十四觚之一面也。

割二十四觚以为四十八觚术曰:亦令半径为弦,半面为句,为之求股。置上 小弦幕,四而一,得一百七十亿三千七百八万七千三百六十六忽,余分弃之,即 句幂也。以减弦幂,其余,开方除之,得股九寸九分一厘四毫四秒四忽五分忽之 四。以减半径,余八厘五毫五秒五忽五分忽之一,谓之小句。觚之半面又谓之小 股。为之求小弦。其幂一百七十一亿一千二十七万八千八百一十三忽,余分弃之。

开方除之,得小弦一寸三分八毫六忽,余分弃之,即四十八觚之一面。以半径一 尺乘之,又以二十四乘之,得幂三万一千三百九十三亿四千四百万忽。以百亿除 之,得幂三百一十三寸六百二十五分寸之五百八十四,即九十六觚之幂也。

割四十八觚以为九十六觚术曰:亦令半径为弦,半面为句,为之求股。置次 上弦幂,四而一,得四十二亿七千七百五十六万九千七百三忽,余分弃之,即句 幂也。以减弦幂,其余,开方除之,得股九寸九分七厘八毫五秒八忽十分忽之九。

以减半径,余二厘一毫四秒一忽十分忽之一,谓之小句。觚之半面又谓之小股。

为之求小弦。其幂四十二亿八千二百一十五万四千一十二忽,余分弃之。开方除 之,得小弦六分五厘四毫三秒八忽,余分弃之,即九十六觚之一面。以半径一尺 乘之,又以四十八乘之,得幂三万一千四百一十亿二千四百万忽,以百亿除之, 得幂三百一十四寸六百二十五分寸之六十四,即一百九十二觚之幂也。以九十六 觚之幂减之,余六百二十五分寸之一百五,谓之差幂。倍之,为分寸之二百一十, 即九十六觚之外弧田九十六所,谓以弦乘矢之凡幂也。加此幂于九十六觚之幂, 得三百一十四寸六百二十五分寸之一百六十九,则出圆之表矣。故还就一百九十 二觚之全幂三百一十四寸以为圆幂之定率而弃其余分。以半径一尺除圆幂,倍之, 得六尺二寸八分,即周数。令径自乘为方幂四百寸,与圆幂相折,圆幂得一百五 十七为率,方幂得二百为率。方幂二百其中容圆幂一百五十七也。圆率犹为微少。

案:弧田图令方中容圆,圆中容方,内方合外方之半。然则圆幂一百五十七,其 中容方幂一百也。又令径二尺与周六尺二寸八分相约,周得一百五十七,径得五 十,则其相与之率也。周率犹为微少也。晋武库中汉时王莽作铜斛,其铭曰:律 嘉量斛,内方尺而圆其外,庣旁九厘五毫,幂一百六十二寸,深一尺,积一千六 百二十寸,容十斗。以此术求之,得幂一百六十一寸有奇,其数相近矣。此术微 少。而觚差幂六百二十五分寸之一百五。以一百九十二觚之幂为率消息,当取此 分寸之三十六,以增于一百九十二觚之幂,以为圆幂,三百一十四寸二十五分寸 之四。置径自乘之方幂四百寸,令与圆幂通相约,圆幂三千九百二十七,方幂得 五千,是为率。方幂五千中容圆幂三千九百二十七;圆幂三千九百二十七中容方 幂二千五百也。以半径一尺除圆幂三百一十四寸二十五分寸之四,倍之,得六尺 二寸八分二十五分分之八,即周数也。全径二尺与周数通相约,径得一千二百五 十,周得三千九百二十七,即其相与之率。若此者,盖尽其纤微矣。举而用之, 上法仍约耳。当求一千五百三十六觚之一面,得三千七十二觚之幂,而裁其微分, 数亦宜然,重其验耳。

淳风等案:旧术求圆,皆以周三径一为率。若用之求圆周之数,则周少径多。

用之求其六觚之田,乃与此率合会耳。何则?假令六觚之田,觚间各一尺为面, 自然从角至角,其径二尺可知。此则周六径二与周三径一已合。恐此犹为难晓, 今更引物为喻。设令刻物作圭形者六枚,枚别三面,皆长一尺。攒此六物,悉使 锐头向里,则成六觚之周,角径亦皆一尺。更从觚角外畔,围绕为规,则六觚之 径尽达规矣。当面径短,不至外规。若以径言之,则为规六尺,径二尺,面径皆 一尺。面径股不至外畔,定无二尺可知。故周三径一之率于圆周乃是径多周少。

径一周三,理非精密。盖术从简要,举大纲,略而言之。刘徽特以为疏,遂改张 其率。但周、径相乘,数难契合。徽虽出斯二法,终不能究其纤毫也。祖冲之以 其不精,就中更推其数。今者修撰,捃摭诸家,考其是非,冲之为密。故显之于 徽术之下,冀学者知所裁焉。〕 又术曰:周、径相乘,四而一。

〔此周与上觚同耳。周、径相乘,各当一半。而今周、径两全,故两母相乘 为四,以报除之。于徽术,以五十乘周,一百五十七而一,即径也。以一百五十 七乘径,五十而一,即周也。新术径率犹当微少。据周以求径,则失之长;据径 以求周,则失之短。诸据见径以求幂者,皆失之于微少;据周以求幂者,皆失之 于微多。

淳风等按:依密率,以七乘周,二十二而一,即径;以二十二乘径,七而一, 即周。依术求之,即得。〕 又术曰:径自相乘,三之,四而一。

〔按:圆径自乘为外方,三之,四而一者,是为圆居外方四分之三也。若令 六觚之一面乘半径,其幂即外方四分之一也。因而三之,即亦居外方四分之三也。

是为圆里十二觚之幂耳。取以为圆,失之于微少。于徽新术,当径自乘,又以一 百五十七乘之,二百而一。

淳风等按:密率,令径自乘,以十一乘之,十四而一,即圆幂也。〕 又术曰:周自相乘,十二而一。

〔六觚之周,其于圆径,三与一也。故六觚之周自相乘为幂,若圆径自乘者 九方。九方凡为十二觚者十有二,故曰十二而一,即十二觚之幂也。今此令周自 乘,非但若为圆径自乘者九方而已。然则十二而一,所得又非十二觚之幂也。若 欲以为圆幂,失之于多矣。以六觚之周,十二而一可也。于徽新术,直令圆周自 乘,又以二十五乘之,三百一十四而一,得圆幂。其率:二十五者,周幂也;三 百一十四者,周自乘之幂也。置周数六尺二寸八分,令自乘,得幂三十九万四千 三百八十四分。又置圆幂三万一千四百分。皆以一千二百五十六约之,得此率。

淳风等按:方面自乘即得其积。圆周求其幂,假率乃通。但此术所求用三、 一为率。圆田正法,半周及半径以相乘。今乃用全周自乘,故须以十二为母。何 者?据全周而求半周,则须以二为法。就全周而求半径,复假六以除之。是二、 六相乘,除周自乘之数。依密率,以七乘之,八十八而一。〕 今有宛田,下周三十步,径十六步。问为田几何?答曰:一百二十步。

又有宛田,下周九十九步,径五十一步。问为田几何?答曰:五亩六十二步 四分步之一。

术曰:以径乘周,四而一。

〔此术不验,故推方锥以见其形。假令方锥下方六尺,高四尺。四尺为股, 下方之半三尺为句。正面邪为弦,弦五尺也。令句弦相乘,四因之,得六十尺, 即方锥四面见者之幂。若令其中容圆锥,圆锥见幂与方锥见幂,其率犹方幂之与 圆幂也。按:方锥下六尺,则方周二十四尺。以五尺乘而半之,则亦锥之见幂。

故求圆锥之数,折径以乘下周之半,即圆锥之幂也。今宛田上径圆穹,而与圆锥 同术,则幂失之于少矣。然其术难用,故略举大较,施之大广田也。求圆锥之幂, 犹求圆田之幂也。今用两全相乘,故以四为法,除之,亦如圆田矣。开立圆术说 圆方诸率甚备,可以验此。〕 今有弧田,弦二十步,矢十五步。问为田几何?答曰:一亩九十七步半。

又有弧田,弦七十八步二分步之一,矢十三步九分步之七。问为田几何?答 曰:二亩一百五十五步八十一分步之五十六。

术曰:以弦乘矢,矢又自乘,并之,二而一。

〔方中之圆,圆里十二觚之幂,合外方之幂四分之三也。中方合外方之半, 则朱青合外方四分之一也。弧田,半圆之幂也。故依半圆之体而为之术。以弦乘 矢而半之,则为黄幂,矢自乘而半之,则为二青幂。青、黄相连为弧体,弧体法 当应规。今觚面不至外畔,失之于少矣。圆田旧术以周三径一为率,俱得十二觚 之幂,亦失之于少也,与此相似。指验半圆之幂耳。若不满半圆者,益复疏阔。

宜句股锯圆材之术,以弧弦为锯道长,以矢为锯深,而求其径。既知圆径,则弧 可割分也。割之者,半弧田之弦以为股,其矢为句,为之求弦,即小弧之弦也。

以半小弧之弦为句,半圆径为弦,为之求股。以减半径,其余即小弦之矢也。割 之又割,使至极细。但举弦、矢相乘之数,则必近密率矣。然于算数差繁,必欲 有所寻究也。若但度田,取其大数,旧术为约耳。〕 今有环田,中周九十二步,外周一百二十二步,径五步。

〔此欲令与周三径一之率相应,故言径五步也。据中、外周,以徽术言之, 当径四步一百五十七分步之一百二十二也。

淳风等按:依密率,合径四步二十二分步之十七。〕 问为田几何?答曰:二亩五十五步。

〔于徽术,当为田二亩三十一步一百五十七分步之二十三。

淳风等按:依密率,为田二亩三十步二十二分步之十五。〕 术曰:并中、外周而半之,以径乘之,为积步。

〔此田截而中之周则为长。并而半之知,亦以盈补虚也。此可令中、外周各 自为圆田,以中圆减外圆,余则环实也。〕 又有环田,中周六十二步四分步之三,外周一百一十三步二分步之一,径十 二步三分步之二。

〔此田环而不通匝,故径十二步三分步之二。若据上周求径者,此径失之于 多,过周三径一之率,盖为疏矣。于徽术,当径八步六百二十八分步之五十一。

淳风等按:依周三径一考之,合径八步二十四分步之一十一。依密率,合径 八步一百七十六分步之一十三。〕 问为田几何?答曰:四亩一百五十六步四分步之一。

〔于徽术,当为田二亩二百三十二步五千二十四分步之七百八十七也。依周 三径一,为田三亩二十五步六十四分步之二十五。

淳风等按:密率,为田二亩二百三十一步一千四百八分步之七百一十七也。〕 术曰:置中、外周步数,分母子各居其下。母互乘子,通全步内分子。以中 周减外周,余半之,以益中周。径亦通分内子,以乘周为实。分母相乘为法。除 之为积步。余,积步之分。以亩法除之,即亩数也。

〔按:此术,并中、外周步数于上,分母子于下,母互乘子者,为中外周俱 有余分,故以互乘齐其子,母相乘同其母。子齐母同,故通全步,内分子。半之 知,以盈补虚,得中平之周。周则为从,径则为广,故广从相乘而得其积。既合 分母,还须分母出之。故令周、径分母相乘而连除之,即得积步。不尽,以等数 除之而命分。以亩法除积步,得亩数也。〕 

\chapter{粟米}
(以御交 质变易) 粟米之法 〔凡此诸率相与大通,其时相求,各如本率。可约者约之。别术然也。〕 粟率五十大抃五十四稻六十 粝米三十粝饭七十五豉六十三 粺米二十七粺饭五十四飧九十 米二十四饭四十八熟菽一百三半 御米二十一御饭四十二糵一百七十五 小<麦啇>十三半菽荅麻麦各四十五 今有 〔此都术也。凡九数以为篇名,可以广施诸率。所谓告往而知来,举一隅而 三隅反者也。诚能分诡数之纷杂,通彼此之否塞,因物成率,审辨名分,平其偏 颇,齐其参差,则终无不归于此术也。〕 术曰:以所有数乘所求率为实。以所有率为法。

〔少者多之始,一者数之母,故为率者必等之于一。据粟率五、粝率三,是 粟五而为一,粝米三而为一也。欲化粟为米者,粟当先本是一。一者,谓以五约 之,令五而为一也。讫,乃以三乘之,令一而为三。如是,则率至于一,以五为 三矣。然先除后乘,或有余分,故术反之。又完言之知,粟五升为粝米三升;以 分言之知,粟一斗为粝米五分斗之三,以五为母,三为子。以粟求粝米者,以子 乘,其母报除也。然则所求之率常为母也。

淳风等按:“宜云所求之率常为子,所有之率常为母。”今乃云“所求之率 常为母”知,脱错也。〕 实如法而一。

今有粟一斗,欲为粝米。问得几何?答曰:为粝米六升。

术曰:以粟求粝米,三之,五而一。

〔淳风等按:都术:以所求率乘所有数,以所有率为法。此术以粟求米,故 粟为所有数。三是米率,故三为所求率。五为粟率,故五为所有率。粟率五十, 米率三十,退位求之,故惟云三、五也。〕 今有粟二斗一升,欲为粺米。问得几何?答曰:为粺米一斗一升五十分 升之十七。

术曰:以粟求粺米,二十七之,五十而一。

〔淳风等按:粺米之率二十有七,故直以二十七之,五十而一也。〕 今有粟四斗五升,欲为米。问得几何?答曰:为米二斗一升五 分升之三。

术曰:以粟求米,十二之,二十五而一。

〔淳风等按:米之率二十有四,以为率太繁,故因而半之。半所求之 率,以乘所有之数。所求之率既减半,所有之率亦减半。是故十二乘之,二十五 而一也。〕 今有粟七斗九升,欲为御米。问得几何?答曰:为御米三斗三升五十分升之 九。

术曰:以粟求御米,二十一之,五十而一。

今有粟一斗,欲为小<麦啇>。问得几何?答曰:为小<麦啇>二升一十分升之 七。

术曰:以粟求小<麦啇>,二十七之,百而一。

〔淳风等按:小<麦啇>之率十三有半。半者二为母,以二通之,得二十七, 为所求率。又以母二通其粟率,得一百,为所有率。凡本率有分者,须即乘除也。

他皆仿此。〕 今有粟九斗八升,欲为大<麦啇>。问得几何?答曰:为大<麦啇>一十斗五升 二十五分升之二十一。

术曰:以粟求大<麦啇>,二十七之,二十五而一。

〔淳风等按:大<麦啇>之率五十有四。因其可半,故二十七之,亦如粟求 米,半其二率。〕 今有粟二斗三升,欲为粝饭。问得几何?答曰:为粝饭三斗四升半。

术曰:以粟求粝饭,三之,二而一。

〔淳风等按:粝饭之率七十有五,粟求粝饭,合以此数乘之。今以等数二十 有五约其二率,所求之率得三,所有之率得二,故以三乘二除。〕 今有粟三斗六升,欲为粺饭。问得几何?答曰:为粺饭三斗八升二十五 分升之二十二。

术曰:以粟求粺饭,二十七之,二十五而一。

〔淳风等按:此术与大<麦啇>多同。〕 今有粟八斗六升,欲为饭。问得几何?答曰:为饭八斗二升二 十五分升之一十四。

术曰:以粟求饭,二十四之,二十五而一。

〔淳风等按:<麦啇>饭率四十八。此亦半二率而乘除。〕 今有粟九斗八升,欲为御饭。问得几何?答曰:为御饭八斗二升二十五分升 之八。

术曰:以粟求御饭,二十一之,二十五而一。

〔淳风等按:此术半率,亦与饭多同。〕 今有粟三斗少半升,欲为菽。问得几何?答曰:为菽二斗七升一十分升之三。

今有粟四斗一升太半升,欲为荅。问得几何?答曰:为荅三斗七升半。

今有粟五斗太半升,欲为麻。问得几何?答曰:为麻四斗五升五分升之三。

今有粟一十斗八升五分升之二,欲为麦。问得几何?答曰:为麦九斗七升二 十五分升之一十四。

术曰:以粟求菽、荅、麻、麦,皆九之,十而一。

〔淳风等按:四术率并四十五,皆是为粟所求,俱合以此率乘其本粟。术欲 从省,先以等数五约之,所求之率得九,所有之率得十,故九乘十除,义由于此。〕 今有粟七斗五升七分升之四,欲为稻。问得几何?答曰:为稻九斗三十五分 升之二十四。

术曰:以粟求稻,六之,五而一。

〔淳风等按:稻率六十,亦约二率而乘除。〕 今有粟七斗八升,欲为豉。问得几何?答曰:为豉九斗八升二十五分升之七。

术曰:以粟求豉,六十三之,五十而一。

今有粟五斗五升,欲为飧。问得几何?答曰:为飧九斗九升。

术曰:以粟求飧,九之,五而一。

〔淳风等按:飧率九十,退位,与求稻多同。〕 今有粟四斗,欲为熟菽。问得几何?答曰:为熟菽八斗二升五分升之四。

术曰:以粟求熟菽,二百七之,百而一。

〔淳风等按:熟菽之率一百三半。半者,其母二,故以母二通之。所求之率 既被二乘,所有之率随而俱长,故以二百七之,百而一。〕 今有粟二斗,欲为糵。问得几何?答曰:为糵七斗。

术曰:以粟求糵,七之,二而一。

〔淳风等按:糵率一百七十有五,合以此数乘其本粟。术欲从省,先以等数 二十五约之,所求之率得七,所有之率得二,故七乘二除。〕 今有粝米十五斗五升五分升之二,欲为粟。问得几何?答曰:为粟二十五斗 九升。

术曰:以粝米求粟,五之,三而一。

〔淳风等按:上术以粟求米,故粟为所有数,三为所求率,五为所有率。今 此以米求粟,故米为所有数,五为所求率,三为所有率。准都术求之,各合其数。

以下所有反求多同,皆准此。〕 今有粺米二斗,欲为粟。问得几何?答曰:为粟三斗七升二十七分升之一。

术曰:以粺米求粟,五十之,二十七而一。

今有米三斗少半升,欲为粟。问得几何?答曰:为粟六斗三升三十六 分升之七。

术曰:以米求粟,二十五之,十二而一。

今有御米十四斗,欲为粟。问得几何?答曰:为粟三十三斗三升少半升。

术曰:以御米求粟,五十之,二十一而一。

今有稻一十二斗六升一十五分升之一十四,欲为粟。问得几何?答曰:为粟 一十斗五升九分升之七。

术曰:以稻求粟,五之,六而一。

今有粝米一十九斗二升七分升之一,欲为粺米。问得几何?答曰:为粺 米一十七斗二升一十四分升之一十三。

术曰:以粝米求粺米,九之,十而一。

〔淳风等按:粺米率二十七,合以此数乘粝米。术欲从省,先以等数三约 之,所求之率得九,所有之率得十,故九乘而十除。〕 今有粝米六斗四升五分升之三,欲为粝饭。问得几何?答曰:为粝饭一十六 斗一升半。

术曰:以粝米求粝饭,五之,二而一。

〔淳风等按:粝饭之率七十有五,宜以本粝米乘此率数。术欲从省,先以等 数十五约之,所求之率得五,所有之率得二,故五乘二除,义由于此。〕 今有粝饭七斗六升七分升之四,欲为飧。问得几何?答曰:为飧九斗一升三 十五分升之三十一。

术曰:以粝饭求飧,六之,五而一。

〔淳风等按:飧率九十,为粝饭所求,宜以粝饭乘此率。术欲从省,先以等 数十五约之,所求之率得六,所有之率得五。以此,故六乘五除也。〕 今有菽一斗,欲为熟菽。问得几何?答曰:为熟菽二斗三升。

术曰:以菽求熟菽,二十三之,十而一。

〔淳风等按:熟菽之率一百三半。因其有半,各以母二通之,宜以菽数乘此 率。术欲从省,先以等数九约之,所求之率得一十一半,所有之率得五也。〕 今有菽二斗,欲为豉。问得几何?答曰:为豉二斗八升。

术曰:以菽求豉,七之,五而一。

〔淳风等按:豉率六十三,为菽所求,宜以菽乘此率。术欲从省,先以等数 九约之,所求之率得七,而所有之率得五也。〕 今有麦八斗六升七分升之三,欲为小<麦啇>。问得几何?答曰:为小<麦啇> 二斗五升一十四分升之一十三。

术曰:以麦求小<麦啇>,三之,十而一。

〔淳风等按:小<麦啇>之率十三半,宜以母二通之,以乘本麦之数。术欲从 省,先以等数九约之,所求之率得三,所有之率得十也。〕 今有麦一斗,欲为大<麦啇>。问得几何?答曰:为大抃一斗二升。

术曰:以麦求大<麦啇>,六之,五而一。

〔淳风等按:大<麦啇>之率五十有四,合以麦数乘此率。术欲从省,先以等 数九约之,所求之率得六,所有之率得五也。〕 今有出钱一百六十,买瓴甓十八枚。

〔瓴甓,砖也。〕 问枚几何?答曰:一枚八钱九分钱之八。

今有出钱一万三千五百,买竹二千三百五十个。问个几何?答曰:一个,五 钱四十七分钱之三十五。

经率术曰:以所买率为法,所出钱数为实,实如法得一。

〔此术犹经分。

淳风等按:今有之义,以所求率乘所有数,合以瓴甓一枚乘钱一百六十为实。

但以一乘不长,故不复乘,是以径将所买之率与所出之钱为法、实也。又按:此 今有之义。出钱为所有数,一枚为所求率,所买为所有率,而今有之,即得所求 数。一乘不长,故不复乘,是以径将所买之率为法,以所出之钱为实,实如法得 一枚钱。不尽者,等数而命分。〕 今有出钱五千七百八十五,买漆一斛六斗七升太半升。欲斗率之,问斗几何? 答曰:一斗,三百四十五钱五百三分钱之一十五。

今有出钱七百二十,买缣一匹二丈一尺。欲丈率之,问丈几何?答曰:一丈, 一百一十八钱六十一分钱之二。

今有出钱二千三百七十,买布九匹二丈七尺。欲匹率之,问匹几何?答曰: 一匹,二百四十四钱一百二十九分钱之一百二十四。

今有出钱一万三千六百七十,买丝一石二钧一十七斤。欲石率之,问石几何? 答曰:一石,八千三百二十六钱一百九十七分钱之百七十八。

术曰:以求所率乘钱数为实,以所买率为法,实如法得一。

〔淳风等按:今有之义,钱为所求率,物为所有数,故以乘钱,又以分母乘 之为实。实如法而一,有分者通之。所买通分内子为所有率,故以为法。得钱数 不尽而命分者,因法为母,实余为子。实见不满,故以命之。〕 今有出钱五百七十六,买竹七十八个。欲其大小率之,问各几何?答曰:其 四十八个,个七钱;其三十个,个八钱。

今有出钱一千一百二十,买丝一石二钧十八斤。欲其贵贱斤率之,问各几何? 答曰:其二钧八斤,斤五钱;其一石一十斤,斤六钱。

今有出钱一万三千九百七十,买丝一石二钧二十八斤三两五铢。欲其贵贱石 率之,问各几何?答曰:其一钧九两一十二铢,石八千五十一钱;其一石一钧二 十七斤九两一十七铢,石八千五十二钱。

今有出钱一万三千九百七十,买丝一石二钧二十八斤三两五铢。欲其贵贱钧 率之,问各几何?答曰:其七斤一十两九铢,钧二千一十二钱;其一石二钧二十 斤八两二十铢,钧二千一十三钱。

今有出钱一万三千九百七十,买丝一石二钧二十八斤三两五铢。欲其贵贱斤 率之,问各几何?答曰:其一石二钧七斤十两四铢,斤六十七钱;其二十斤九两 一铢,斤六十八钱。

今有出钱一万三千九百七十,买丝一石二钧二十八斤三两五铢。欲其贵贱两 率之,问各几何?答曰:其一石一钧一十七斤一十四两一铢,两四钱;其一钧一 十斤五两四铢,两五钱。

其率术曰:各置所买石、钧、斤、两以为法,以所率乘钱数为实,实如法 而一。不满法者,反以实减法。法贱实贵。其求石、钧、斤、两,以积铢各除法、 实,各得其积数,余各为铢。

〔其率知,欲令无分。按:出钱五百七十六,买竹七十八个,以除钱,得七, 实余三十,是为三十个复可增一钱。然则实余之数即是贵者之数,故曰实贵也。

本以七十八个为法,今以贵者减之,则其余悉是贱者之数。故曰法贱也。其求石、 钧、斤、两,以积铢各除法、实,各得其积数,余各为铢者,谓石、钧、斤、两 积铢除实,又以石、钧、斤、两积铢除法,余各为铢,即合所问。〕 今有出钱一万三千九百七十,买丝一石二钧二十八斤三两五铢。欲其贵贱铢 率之,问各几何?答曰:其一钧二十斤六两十一铢,五铢一钱;其一石一钧七斤 一十二两一十八铢,六铢一钱。

今有出钱六百二十,买羽二千一百翭。

〔翭,羽本也。数羽称其本,犹数草木称其根株。〕 欲其贵贱率之,问各几何?答曰:其一千一百四十翭,三翭一钱; 其九百六十翭,四翭钱。

今有出钱九百八十,买矢榦五千八百二十枚。欲其贵贱率之,问各几何?答 曰:其三百枚,五枚一钱;其五千五百二十枚,六枚一钱。

反其率术曰:以钱数为法,所率为实,实如法而一。不满法者,反以实减 法。法少实多。二物各以所得多少之数乘法、实,即物数。

〔按:其率:出钱六百二十,买羽二千一百翭。反之,当二百四十钱, 一钱翭;其三百八十钱,一钱三翭。是钱有二价,物有贵贱。故以羽乘 钱,反其率也。

淳风等按:其率者,钱多物少;反其率知,钱少物多;多少相反,故曰反其 率也。其率者,以物数为法,钱数为实。反之知,以钱数为法,物数为实。不满 法知,实余也。当以余物化为钱矣。法为凡钱,而今以化钱减之,故以实减法。

法少知,经分之所得,故曰法少;实多者,余分之所益,故曰实多。乘实宜以多, 乘法宜以少,故曰各以其所得多少之数乘法、实,即物数。〕

\chapter{衰分}
(以御贵贱禀税) 衰分 〔衰分,差也。〕 术曰:各置列衰; 〔列衰,相与率也。重叠,则可约。〕 副并为法,以所分乘未并者,各自为实。实如法而一。

〔法集而衰别。数,本一也。今以所分乘上别,以下集除之,一乘一除,适 足相消,故所分犹存,且各应率而别也。于今有术,列衰各为所求率,副并为所 有率,所分为所有数。又以经分言之,假令甲家三人,乙家二人,丙家一人,并 六人,共分十二,为人得二也。欲复作逐家者,则当列置人数,以一人所得乘之。

今此术先乘而后除也。〕 不满法者,以法命之。

今有大夫、不更、簪袅、上造、公士,凡五人,共猎得五鹿。欲以爵次分之, 问各得几何?答曰:大夫得一鹿三分鹿之二;不更得一鹿三分鹿之一;簪袅得一 鹿;上造得三分鹿之二;公士得三分鹿之一。

术曰:列置爵数,各自为衰。

〔爵数者,谓大夫五,不更四,簪袅三,上造二,公士一也。《墨子·号令 篇》以爵级为赐,然则战国之初有此名也。〕 副并为法。以五鹿乘未并者各自为实。实如法得一鹿。

〔今有术,列衰各为所求率,副并为所有率,今有鹿数为所有数,而今有之, 即得。〕 今有牛、马、羊食人苗。苗主责之粟五斗。羊主曰:“我羊食半马。”马主 曰:“我马食半牛。”今欲衰偿之,问各出几何?答曰:牛主出二斗八升七分升 之四;马主出一斗四升七分升之二;羊主出七升七分升之一。

术曰:置牛四、马二、羊一,各自为列衰,副并为法。以五斗乘未并者各自 为实。实如法得一斗。

〔淳风等按:此术问意,羊食半马,马食半牛,是谓四羊当一牛,二羊当一 马。今术置羊一、马二、牛四者,通其率以为列衰。〕 今有甲持钱五百六十,乙持钱三百五十,丙持钱一百八十,凡三人俱出关, 关税百钱。欲以钱数多少衰出之,问各几何?答曰:甲出五十一钱一百九分钱之 四十一;乙出三十二钱一百九分钱之一十二;丙出一十六钱一百九分钱之五十六。

术曰:各置钱数为列衰,副并为法。以百钱乘未并者,各自为实。实如法得 一钱。

〔淳风等按:此术甲、乙、丙持钱数以为列衰,副并为所有率,未并者各为 所求率,百钱为所有数,而今有之,即得。〕 今有女子善织,日自倍,五日织五尺。问日织几何?答曰:初日织一寸三十 一分寸之十九;次日织三寸三十一分寸之七;次日织六寸三十一分寸之十四;次 日织一尺二寸三十一分寸之二十八;次日织二尺五寸三十一分寸之二十五。

术曰:置一、二、四、八、十六为列衰,副并为法。以五尺乘未并者,各自 为实。实如法得一尺。

今有北乡算八千七百五十八,西乡算七千二百三十六,南乡算八千三百五十 六。凡三乡发徭三百七十八人。欲以算数多少衰出之,问各几何?答曰:北乡遣 一百三十五人一万二千一百七十五分人之一万一千六百三十七;西乡遣一百一十 二人一万二千一百七十五分人之四千四;南乡遣一百二十九人一万二千一百七十 五分人之八千七百九。

术曰:各置算数为列衰, 〔淳风等按:三乡算数,约,可半者,为列衰。〕 副并为法。以所发徭人数乘未并者,各自为实。实如法得一人。

〔按:此术,今有之义也。〕 今有禀粟,大夫、不更、簪袅、上造、公士,凡五人,一十五斗。今有大夫 一人后来,亦当禀五斗。仓无粟,欲以衰出之,问各几何?答曰:大夫出一斗四 分斗之一;不更出一斗;簪袅出四分斗之三;上造出四分斗之二;公士出四分斗 之一。

术曰:各置所禀粟斛,斗数、爵次均之,以为列衰。副并而加后来大夫亦五 斗,得二十以为法。以五斗乘未并者,各自为实。实如法得一斗。

〔禀前五人十五斗者,大夫得五斗,不更得四斗,簪袅得三斗,上造得二斗, 公士得一斗。欲令五人各依所得粟多少减与后来大夫,即与前来大夫同。据前来 大夫已得五斗,故言亦也。各以所得斗数为衰,并得十五,而加后来大夫亦五斗, 凡二十,为法也。是为六人共出五斗,后来大夫亦俱损折。今有术,副并为所有 率,未并者各为所求率,五斗为所有数,而今有之,即得。〕 今有禀粟五斛,五人分之。欲令三人得三,二人得二,问各几何?答曰:三 人,人得一斛一斗五升十三分升之五;二人,人得七斗六升十三分升之十二。

术曰:置三人,人三;二人,人二,为列衰。副并为法。以五斛乘未并者各 自为实。实如法得一斛。

反衰术曰:列置衰而令相乘,动者为不动者衰。

今有大夫、不更、簪袅、上造、公士凡五人,共出百钱。欲令高爵出少,以 次渐多,问各几何?答曰:大夫出八钱一百三十七分钱之一百四;不更出一十钱 一百三十七分钱之一百三十;簪袅出一十四钱一百三十七分钱之八十二;上造出 二十一钱一百三十七分钱之一百二十三;公士出四十三钱一百三十七分钱之一百 九。

术曰:置爵数,各自为衰,而反衰之。副并为法。以百钱乘未并者,各自为 实。实如法得一钱。

〔以爵次言之,大夫五、不更四。欲令高爵得多者,当使大夫一人受五分, 不更一人受四分。人数为母,分数为子。母同则子齐,齐即衰也。故上衰分宜以 五、四为列焉。今此令高爵出少,则当大夫五人共出一人分,不更四人共出一人 分,故谓之反衰。人数不同,则分数不齐。当令母互乘子。母互乘子,则动者为 不动者衰也。亦可先同其母,各以分母约,其子为反衰。副并为法。以所分乘未 并者,各自为实。实如法而一。〕 今有甲持粟三升,乙持粝米三升,丙持粝饭三升。欲令合而分之,问各几何? 答曰:甲二升一十分升之七;乙四升一十分升之五;丙一升一十分升之八。

术曰:以粟率五十、粝米率三十、粝饭率七十五为衰,而反衰之。副并为法。

以九升乘未并者,各自为实。实如法得一升。

〔按:此术,三人所持升数虽等,论其本率,精粗不同。米率虽少,令最得 多;饭率虽多,反使得少。故令反之,使精得多而粗得少。于今有术,副并为所 有率,未并者各为所求率,九升为所有数,而今有之,即得。〕 今有丝一斤,价直二百四十。今有钱一千三百二十八,问得丝几何?答曰: 五斤八两一十二铢五分铢之四。

术曰:以一斤价数为法,以一斤乘今有钱数为实。实如法得丝数。

〔按:此术今有之义,以一斤价为所有率,一斤为所求率,今有钱为所有数, 而今有之,即得。〕 今有丝一斤,价直三百四十五。今有丝七两一十二铢,问得钱几何?答曰: 一百六十一钱三十二分钱之二十三。

术曰:以一斤铢数为法,以一斤价数乘七两一十二铢为实。实如法得钱数。

〔淳风等按:此术亦今有之义。以丝一斤铢数为所有率,价钱为所求率,今 有丝为所有数,而今有之,即得。〕 今有缣一丈,价直一百二十八。今有缣一匹九尺五寸,问得钱几何?答曰: 六百三十三钱五分钱之三。

术曰:以一丈寸数为法,以价钱数乘今有缣寸数为实。实如法得钱数。

〔淳风等按:此术亦今有之义。以缣一丈寸数为所有率,价钱为所求率,今 有缣寸数为所有数,而今有之,即得。〕 今有布一匹,价直一百二十五。今有布二丈七尺,问得钱几何?答曰:八十 四钱八分钱之三。

术曰:以一匹尺数为法,今有布尺数乘价钱为实。实如法得钱数。

〔淳风等按:此术亦今有之义。以一匹尺数为所有率,价钱为所求率,今有 布为所有数,今有之,即得。〕 今有素一匹一丈,价直六百二十五。今有钱五百,问得素几何?答曰:得素 一匹。

术曰:以价直为法,以一匹一丈尺数乘今有钱数为实。实如法得素数。

〔淳风等按:此术亦今有之义。以价钱为所有率,五丈尺数为所求率,今有 钱为所有数,今有之,即得。〕 今有与人丝一十四斤,约得缣一十斤。今与人丝四十五斤八两,问得缣几何? 答曰:三十二斤八两。

术曰:以一十四斤两数为法,以一十斤乘今有丝两数为实。实如法得缣数。

〔淳风等按:此术亦今有之义。以一十四斤两数为所有率,一十斤为所求率, 今有丝为所有数,而今有之,即得。〕 今有丝一斤,耗七两。今有丝二十三斤五两,问耗几何?答曰:一百六十三 两四铢半。

术曰:以一斤展十六两为法。以七两乘今有丝两数为实。实如法得耗数。

〔淳风等按:此术亦今有之义。以一斤为十六两为所有率,七两为所求率, 今有丝为所有数,而今有之,即得。〕 今有生丝三十斤,干之,耗三斤十二两。今有干丝一十二斤,问生丝几何? 答曰:一十三斤一十一两十铢七分铢之二。

术曰:置生丝两数,除耗数,余,以为法。

〔馀四百二十两,即干丝率。〕 三十斤乘干丝两数为实。实如法得生丝数。

〔凡所得率,如细则俱细,粗则俱粗,两数相抱而已。故品物不同,如上缣、 丝之比,相与率焉。三十斤凡四百八十两,今生丝率四百八十两,今干丝率四百 二十两,则其数相通。可俱为铢,可俱为两,可俱为斤,,无所归滞也。若然, 宜以所有干丝斤数乘生丝两数为实。今以斤、两错互而亦同归者,使干丝以两数 为率,生丝以斤数为率,譬之异类,亦各有一定之势。

淳风等按:此术,置生丝两数,除耗数,余即干丝之率,于今有术为所有率; 三十斤为所求率,干丝两数为所有数。凡所为率者,细则俱细,粗则俱粗。今有 一斤乘两知,干丝即以两数为率,生丝即以斤数为率,譬之异物,各有一定之率 也。〕 今有田一亩,收粟六升太半升。今有田一顷二十六亩一百五十九步,问收粟 几何?答曰:八斛四斗四升一十二分升之五。

术曰:以亩二百四十步为法。以六升太半升乘今有田积步为实。实如法得粟 数。

〔淳风等按:此术亦今有之义。以一亩步数为所有率,六升太半升为所求率, 今有田积步为所有数,而今有之,即得。〕 今有取保,一岁价钱二千五百。今先取一千二百,问当作日几何?答曰:一 百六十九日二十五分日之二十三。

术曰:以价钱为法,以一岁三百五十四日乘先取钱数为实。实如法得日数。

〔淳风等按:此术亦今有之义。以价为所有率,一岁日数为所求率,取钱为 所有数,而今有之,即得。〕 今有贷人千钱,月息三十。今有贷人七百五十钱,九日归之,问息几何?答 曰:六钱四分钱之三。

术曰:以月三十日乘千钱为法。

〔以三十日乘千钱为法者,得三万,是为贷人钱三万,一日息三十也。〕 以息三十乘今所贷钱数,又以九日乘之,为实。实如法得一钱。

〔以九日乘今所贷钱为今一日所有钱,于今有术为所有数,息三十为所求率; 三万钱为所有率。此又可以一月三十日约息三十钱,为十分一日,以乘今一日所 有钱为实;千钱为法。为率者,当等之于一也。故三十日或可乘本,或可约息, 皆所以等之也。〕 

\chapter{少广}

○(以御积幂方圆) 少广 〔淳风等按:一亩之田,广一步,长二百四十步。今欲截取其从少,以益其 广,故曰少广。〕 术曰:置全步及分母子,以最下分母遍乘诸分子及全步, 〔淳风等按:以分母乘全步者,通其分也;以母乘子者,齐其子也。〕 各以其母除其子,置之于左,命通分者,又以分母遍乘诸分子及已通者,皆 通而同之。并之为法。

〔淳风等按:诸子悉通,故可并之为法。亦宜用合分术,列数尤多,若用乘 则算数至繁,故别制此术,从省约。〕 置所求步数,以全步积分乘之为实。

〔此以田广为法,以亩积步为实。法有分者,当同其母,齐其子,以同乘法 实,而并齐于法。今以分母乘全步及子,子如母而一,并以并全法,则法实俱长, 意亦等也。故如法而一,得从步数。〕 实如法而一,得从步。

今有田广一步半。求田一亩,问从几何?答曰:一百六十步。

术曰:下有半,是二分之一。以一为二,半为一,并之,得三,为法。置田 二百四十步,亦以一为二乘之,为实。实如法得从步。

今有田广一步半、三分步之一。求田一亩,问从几何?答曰:一百三十步一 十一分步之一十。

术曰:下有三分,以一为六,半为三,三分之一为二,并之,得一十一,为 法。置田二百四十步,亦以一为六乘之,为实。实如法得从步。

今有田广一步半、三分步之一、四分步之一。求田一亩,问从几何?答曰: 一百一十五步五分步之一。

术曰:下有四分,以一为一十二,半为六,三分之一为四,四分之一为三, 并之,得二十五,以为法。置田二百四十步,亦以一为一十二乘之,为实。实如 法而一,得从步。

今有田广一步半、三分步之一、四分步之一、五分步之一。求田一亩,问从 几何?答曰:一百五步一百三十七分步之一十五。

术曰:下有五分,以一为六十,半为三十,三分之一为二十,四分之一为一 十五,五分之一为一十二,并之,得一百三十七,以为法。置田二百四十步,亦 以一为六十乘之,为实。实如法得从步。

今有田广一步半、三分步之一、四分步之一、五分步之一、六分步之一。求 田一亩,问从几何?答曰:九十七步四十九分步之四十七。

术曰:下有六分,以一为一百二十,半为六十,三分之一为四十,四分之一 为三十,五分之一为二十四,六分之一为二十,并之,得二百九十四,以为法。

置田二百四十步,亦以一为一百二十乘之,为实。实如法得从步。

今有田广一步半、三分步之一、四分步之一、五分步之一、六分步之一、七 分步之一。求田一亩,问从几何?答曰:九十二步一百二十一分步之六十八。

术曰:下有七分,以一为四百二十,半为二百一十,三分之一为一百四十, 四分之一为一百五,五分之一为八十四,六分之一为七十,七分之一为六十,并 之,得一千八十九,以为法。置田二百四十步,亦以一为四百二十乘之,为实。

实如法得从步。

今有田广一步半、三分步之一、四分步之一、五分步之一、六分步之一、七 分步之一、八分步之一。求田一亩,问从几何?答曰:八十八步七百六十一分步 之二百三十二。

术曰:下有八分,以一为八百四十,半为四百二十,三分之一为二百八十, 四分之一为二百一十,五分之一为一百六十八,六分之一为一百四十,七分之一 为一百二十,八分之一为一百五,并之,得二千二百八十三,以为法。置田二百 四十步,亦以一为八百四十乘之,为实。实如法得从步。

今有田广一步半、三分步之一、四分步之一、五分步之一、六分步之一、七 分步之一、八分步之一、九分步之一。求田一亩,问从几何?答曰:八十四步七 千一百二十九分步之五千九百六十四。

术曰:下有九分,以一为二千五百二十,半为一千二百六十,三分之一为八 百四十,四分之一为六百三十,五分之一为五百四,六分之一为四百二十,七分 之一为三百六十,八分之一为三百一十五,九分之一为二百八十,并之,得七千 一百二十九,以为法。置田二百四十步,亦以一为二千五百二十乘之,为实。实 如法得从步。

今有田广一步半、三分步之一、四分步之一、五分步之一、六分步之一、七 分步之一、八分步之一、九分步之一、十分步之一。求田一亩、问从几何?答曰: 八十一步七千三百八十一分步之六千九百三十九。

术曰:下有一十分,以一为二千五百二十,半为一千二百六十,三分之一为 八百四十,四分之一为六百三十,五分之一为五百四,六分之一为四百二十,七 分之一为三百六十,八分之一为三百一十五,九分之一为二百八十,十分之一为 二百五十二,并之,得七千三百八十一,以为法。置田二百四十步,亦以一为二 千五百二十乘之,为实。实如法得从步。

今有田广一步半、三分步之一、四分之步一、五分步之一、六分步之一、七 分步之一、八分步之一、九分步之一、十分步之一、十一分步之一。求田一亩, 问从几何?答曰:七十九步八万三千七百一十一分步之三万九千六百三十一。

术曰:下有一十一分,以一为二万七千七百二十,半为一万三千八百六十, 三分之一为九千二百四十,四分之一为六千九百三十,五分之一为五千五百四十 四,六分之一为四千六百二十,七分之一为三千九百六十,八分之一为三千四百 六十五,九分之一为三千八十,一十分之一为二千七百七十二,一十一分之一为 二千五百二十,并之,得八万三千七百一十一,以为法。置田二百四十步,亦以 一为二万七千七百二十乘之,为实。实如法得从步。

今有田广一步半、三分步之一、四分步之一,五分步之一、六分步之一、七 分步之一、八分步之一、九分步之一、十分步之一、十一分步之一、十二分步之 一。求田一亩,问从几何?答曰:七十七步八万六千二十一分步之二万九千一百 八十三。

术曰:下有一十二分,以一为八万三千一百六十,半为四万一千五百八十, 三分之一为二万七千七百二十,四分之一为二万七百九十,五分之一为一万六千 六百三十二,六分之一为一万三千八百六十,七分之一为一万一千八百八十,八 分之一为一万三百九十五,九分之一为九千二百四十,一十分之一为八千三百一 十六,十一分之一为七千五百六十,十二分之一为六千九百三十,并之,得二十 五万八千六十三,以为法。置田二百四十步,亦以一为八万三千一百六十乘之, 为实。实如法得从步。

〔淳风等按:凡为术之意,约省为善。宜云“下有一十二分,以一为二万七 千七百二十,半为一万三千八百六十,三分之一为九千二百四十,四分之一为六 千九百三十,五分之一为五千五百四十四,六分之一为四千六百二十,七分之一 为三千九百六十,八分之一为三千四百六十五,九分之一为三千八十,十分之一 为二千七百七十二,十一分之一为二千五百二十,十二分之一为二千三百一十, 并之,得八万六千二十一,以为法。置田二百四十步,亦以一为二万七千七百二 十乘之,以为实。实如法得从步。”其术亦得知,不繁也。〕 今有积五万五千二百二十五步,问为方几何?答曰:二百三十五步。

又有积二万五千二百八十一步,问为方几何?答曰:一百五十九步。

又有积七万一千八百二十四步,问为方几何?答曰:二百六十八步。

又有积五十六万四千七百五十二步四分步之一,问为方几何?答曰:七百五 十一步半。

又有积三十九亿七千二百一十五万六百二十五步,问为方几何?答曰:六万 三千二十五步。

○开方 〔求方幂之一面也。〕 术曰:置积为实。借一算,步之,超一等。

〔言百之面十也。言万之面百也。〕 议所得,以一乘所借一算为法,而以除。

〔先得黄甲之面,上下相命,是自乘而除也。〕 除已,倍法为定法。

〔倍之者,豫张两面朱幂定袤,以待复除,故曰定法。〕 其复除,折法而下。

〔欲除朱幂者,本当副置所得成方,倍之为定法,以折、议、乘,而以除。

如是当复步之而止,乃得相命。故使就上折下。〕 复置借算,步之如初。以复议一乘之, 〔欲除朱幂之角黄乙之幂,其意如初之所得也。〕 所得副以加定法,以除。以所得副从定法。

〔再以黄乙之面加定法者,是则张两青幂之袤。〕 复除,折下如前。若开之不尽者,为不可开,当以面命之。

〔术或有以借算加定法而命分者,虽粗相近,不可用也。凡开积为方,方之 自乘当还复有积分。令不加借算而命分,则常微少;其加借算而命分,则又微多。

其数不可得而定。故惟以面命之,为不失耳。譬犹以三除十,以其余为三分之一, 而复其数可以举。不以面命之,加定法如前,求其微数。微数无名者以为分子, 其一退以十为母,其再退以百为母。退之弥下,其分弥细,则朱幂虽有所弃之数, 不足言之也。〕 若实有分者,通分内子为定实,乃开之。讫,开其母,报除。

〔淳风等按:分母可开者,并通之积先合二母。既开之后,一母尚存,故开 分母,求一母为法,以报除也。〕 若母不可开者,又以母乘定实,乃开之。讫,令如母而一。

〔淳风等按:分母不可开者,本一母也。又以母乘之,乃合二母。既开之后, 亦一母存焉,故令一母而一,得全面也。

又按:此术“开方”者,求方幂之面也。借一算者,假借一算,空有列位之 名,而无除积之实。方隅得面,是故借算列之于下。“步之超一等”者,方十自 乘,其积有百,方百自乘,其积有万,故超位,至百而言十,至万而言百。“议 所得,以一乘所借算为法,而以除”者,先得黄甲之面,以方为积者两相乘,故 开方除之,还令两面上下相命,是自乘而除之。“除已,倍法为定法”者,实积 未尽,当复更除,故豫张两面朱幂袤,以待复除,故曰定法。“其复除,折法而 下”者,欲除朱幂,本当副置所得成方,倍之为定法,以折、议、乘之,而以除, 如是,当复步之而止,乃得相命。故使就上折之而下。“复置借算,步之如初, 以复议一乘之,所得副以加定法,以定法除”者。欲除朱幂之角黄乙之幂。“以 所得副从定法”者,再以黄乙之面加定法,是则张两青幂之袤,故如前开之,即 合所问。〕 今有积一千五百一十八步四分步之三。问为圆周几何?答曰:一百三十五步。

〔于徽术,当周一百三十八步一十分步之一。

淳风等按:此依密率,为周一百三十八步五十分步之九。〕 又有积三百步,问为圆周几何?答曰:六十步。

〔于徽术,当周六十一步五十分步之十九。

淳风等按:依密率,为周六十一步一百分步之四十一。〕 开圆术曰:置积步数,以十二乘之,以开方除之,即得周。

〔此术以周三径一为率,与旧圆田术相返覆也。于徽术,以三百一十四乘积, 如二十五而一,所得,开方除之,即周也。开方除之,即径。是为据见幂以求周, 犹失之于微少。其以二百乘积,一百五十七而一,开方除之,即径,犹失之于微 多。

淳风等按:此注于徽术求周之法,其中不用“开方除之,即径”六字,今 本有者,衍剩也。依密率,八十八乘之,七而一。按周三径一之率,假令周六径 二,半周半径相乘得幂三,周六自乘得三十六。俱以等数除幂,得一周之数十二 也。其积:本周自乘,合以一乘之,十二而一,得积三也。术为一乘不长,故以 十二而一,得此积。今还原,置此积三,以十二乘之者,复其本周自乘之数。凡 物自乘,开方除之,复其本数,故开方除之,即周。〕 今有积一百八十六万八百六十七尺, 〔此尺谓立方尺也。凡物有高、深而言积者,曰立方。〕 问为立方几何?答曰:一百二十三尺。

又有积一千九百五十三尺八分尺之一,问为立方几何?答曰:一十二尺半。

又有积六万三千四百一尺五百一十二分尺之四百四十七,问为立方几何?答 曰:三十九尺八分尺之七。

又有积一百九十三万七千五百四十一尺二十七分尺之一十七,问为立方几何? 答曰:一百二十四尺太半尺。

开立方 〔立方适等,求其一面也。〕 术曰:置积为实。借一算,步之,超二等。

〔言千之面十,言百万之面百。〕 议所得,以再乘所借一算为法,而除之。

〔再乘者,亦求为方幂。以上议命而除之,则立方等也。〕 除已,三之为定法。

〔为当复除,故豫张三面,以定方幂为定法也。〕 复除,折而下。

〔复除者,三面方幂以皆自乘之数,须得折、议,定其厚薄尔。开平幂者, 方百之面十;开立幂者,方千之面十。据定法已有成方之幂,故复除当以千为百, 折下一等也。〕 以三乘所得数,置中行。

〔设三廉之定长。〕 复借一算,置下行。

〔欲以为隅方。立方等未有定数,且置一算定其位。〕 步之,中超一,下超二等。

〔上方法,长自乘而一折,中廉法,但有长,故降一等;下隅法,无面长, 故又降一等也。〕 复置议,以一乘中, 〔为三廉备幂也。〕 再乘下, 〔令隅自乘,为方幂也。〕 皆副以加定法。以定法除。

〔三面、三廉、一隅皆已有幂,以上议命之而除,去三幂之厚也。〕 除已,倍下,并中,从定法。

〔凡再以中、三以下,加定法者,三廉各当以两面之幂连于两方之面,一隅 连于三廉之端,以待复除也。言不尽意,解此要当以棋,乃得明耳。〕 复除,折下如前。开之不尽者,亦为不可开。

〔术亦有以定法命分者,不如故幂开方,以微数为分也。〕 若积有分者,通分内子为定实。定实乃开之。讫,开其母以报除。

〔淳风等按:分母可开者,并通之积先合三母。既开之后一母尚存,故开分 母,求一母,为法,以报除也。〕 若母不可开者,又以母再乘定实,乃开之。讫,令如母而一。

〔淳风等按:分母不可开者,本一母也。又以母再乘之,令合三母。既开之 后,一母犹存,故令一母而一,得全面也。

按:“开立方”知,立方适等,求其一面之数。“借一算,步之,超二等” 者,但立方求积,方再自乘,就积开之,故超二等,言千之面十,言百万之面百。

“议所得,以再乘所借算为法,而以除”知,求为方幂,以议命之而除,则立方 等也。“除已,三之为定法”,为积未尽,当复更除,故豫张三面已定方幂为定 法。“复除,折而下”知,三面方幂皆已有自乘之数,须得折、议定其厚薄。据 开平方,百之面十,其开立方,即千之面十。而定法已有成方之幂,故复除之者, 当以千为百,折下一等。“以三乘所得数,置中行”者,设三廉之定长。“复借 一算,置下行”者,欲以为隅方,立方等未有数,且置一算定其位也。“步之, 中超一,下超二”者,上方法长自乘而一折,中廉法但有长,故降一等,下隅法 无面长,故又降一等。“复置议,以一乘中”者,为三廉备幂。“再乘下”,当 令隅自乘为方幂。“皆副以加定法,以定法除者,三面、三廉、一隅皆已有幂, 以上议命之而除,去三幂之厚。“除已,倍下、并中,从定法”者,三廉各当以 两面之幂连于两方之面,一隅连于三廉之端,以待复除。其开之不尽者,折下如 前,开方,即合所问。“有分者,通分内子开之。讫,开其母以报除”,“可开 者,并通之积,先合三母;既开之后,一母尚存,故开分母”者,“求一母为法, 以报除。”“若母不可开者,又以母再乘定实,乃开之。讫,令如母而一”,分 母不可开者,本一母,又以母再乘,令合三母,既开之后,亦一母尚存。故令如 母而一,得全面也。〕 今有积四千五百尺。

〔亦谓立方之尺也。〕 问为立圆径几何?答曰:二十尺。

〔依密率,立圆径二十尺,计积四千一百九十尺二十一分尺之一十。〕 又有积一万六千四百四十八亿六千六百四十三万七千五百尺。问为立圆径几 何?答曰:一万四千三百尺。

〔依密率,为径一万四千六百四十三尺四分尺之三。〕 开立圆术曰:置积尺数,以十六乘之,九而一,所得,开立方除之,即立 圆径。

〔立圆,即丸也。为术者,盖依周三径一之率。令圆幂居方幂四分之三,圆 囷居立方亦四分之三。更令圆囷为方率十二,为丸率九,丸居圆囷又四分之三也。

置四分自乘得十六,三分自乘得九,故丸居立方十六分之九也。故以十六乘积, 九而一,得立方之积。丸径与立方等,故开立方而除,得径也。然此意非也。何 以验之?取立方棋八枚,皆令立方一寸,积之为立方二寸。规之为圆囷,径二寸, 高二寸。又复横因之,则其形有似牟合方盖矣。八棋皆似陽马,圆然也。按:合 盖者,方率也,丸居其中,即圆率也。推此言之,谓夫圆囷为方率,岂不阙哉? 以周三径一为圆率,则圆幂伤少;令圆囷为方率,则丸积伤多,互相通补,是以 九与十六之率偶与实相近,而丸犹伤多耳。观立方之内,合盖之外,虽衰杀有渐, 而多少不掩。判合总结,方圆相缠,浓纤诡互,不可等正。欲陋形措意,惧失正 理。敢不阙疑,以俟能言者。

黄金方寸,重十六两;金丸径寸,重九两,率生于此,未曾验也。《周官· 考工记》:“朅氏为量,改煎金锡则不耗,不耗然后权之,权之然后准之,准之 然后量之。”言炼金使极精,而后分之则可以为率也。令丸径自乘,三而一,开 方除之,即丸中之立方也。假令丸中立方五尺,五尺为句,句自乘幂二十五尺。

倍之得五十尺,以为弦幂,谓平面方五尺之弦也。以此弦为股,亦以五尺为句, 并句股幂得七十五尺,是为大弦幂。开方除之,则大弦可知也。大弦则中立方之 长邪,邪即丸径。故中立方自乘之幂于丸径自乘之幂,三分之一也。今大弦还乘 其幂,即丸外立方之积也。大弦幂开之不尽,令其幂七十五再自乘之,为面,命 得外立方积,四十二万一千八百七十五尺之面。又令中立方五尺自乘,又以方乘 之,得积一百二十五尺,一百二十五尺自乘,为面,命得积,一万五千六百二十 五尺之面。皆以六百二十五约之,外立方积,六百七十五尺之面,中立方积,二 十五尺之面也。

张衡算又谓立方为质,立圆为浑。衡言质之与中外之浑:六百七十五尺之面, 开方除之,不足一,谓外浑积二十六也;内浑,二十五之面,谓积五尺也。今徽 令质言中浑,浑又言质,则二质相与之率犹衡二浑相与之率也。衡盖亦先二质之 率推以言浑之率也。衡又言:“质,六十四之面;浑,二十五之面。”质复言浑, 谓居质八分之五也。又云:方,八之面;圆,五之面。”圆浑相推,知其复以圆 囷为方率,浑为圆率也,失之远矣。衡说之自然欲协其陰陽奇偶之说而不顾疏密 矣。虽有文辞,斯乱道破义,病也。置外质积二十六,以九乘之,十六而一,得 积十四尺八分尺之五,即质中之浑也。以分母乘全内子,得一百一十七。又置内 质积五,以分母乘之,得四十,是谓质居浑一百一十七分之四十,而浑率犹为伤 多也。假令方二尺,方四面,并得八尺也,谓之方周。其中令圆径与方等,亦二 尺也。圆半径以乘圆周之半,即圆幂也。半方以乘方周之半,即方幂也。然则方 周知,方幂之率也;圆周知,圆幂之率也。按:如衡术,方周率八之面,圆周率 五之面也。令方周六十四尺之面,圆周四十尺之面也。又令径二尺自乘,得径四 尺之面,是为圆周率十之面,而径率一之面也。衡亦以周三径一之率为非,是故 更著此法,然增周太多,过其实矣。

淳风等按:祖暅之谓刘徽、张衡二人皆以圆囷为方率,丸为圆率,乃设新 法。祖暅之开立圆术曰:“以二乘积,开立方除之,即立圆径。其意何也?取 立方棋一枚,令立枢于左后之下隅,从规去其右上之廉;又合而衡规之,去其前 上之廉。于是立方之棋分而为四,规内棋一,谓之内棋;规外棋三,谓之外棋。

规更合四棋,复横断之。以句股言之,令余高为句,内棋断上方为股,本方之数, 其弦也。句股之法:以句幂减弦幂,则余为股幂。若令余高自乘,减本方之幂, 余即内棋断上方之幂也。本方之幂即此四棋之断上幂。然则余高自乘,即外三棋 之断上幂矣。不问高卑,势皆然也。然固有所归同而途殊者尔。而乃控远以演类, 借况以析微。按:陽马方高数参等者,倒而立之,横截去上,则高自乘与断上幂 数亦等焉。夫叠棋成立积,缘幂势既同,则积不容异。由此观之,规之外三棋旁 蹙为一,即一陽马也。三分立方,则陽马居一,内棋居二可知矣。合八小方成一 大方,合八内棋成一合盖。内棋居小方三分之二,则合盖居立方亦三分之二,较 然验矣。置三分之二,以圆幂率三乘之,如方幂率四而一,约而定之,以为丸率。

故曰丸居立方二分之一也。”等数既密,心亦昭晢。张衡放旧,贻哂于后,刘徽 循故,未暇校新。夫岂难哉,抑未之思也。依密率,此立圆积,本以圆径再自乘, 十一乘之,二十一而一,得此积。今欲求其本积,故以二十一乘之,十一而一。

凡物再自乘,开立方除之,复其本数。故立方除之,即丸径也。〕

\chapter{商功}
(以御功程积实) 今有穿地,积一万尺。问为坚、壤各几何?答曰:为坚七千五百尺;为壤一 万二千五百尺。

术曰:穿地四为壤五, 〔壤谓息土。〕 为坚三, 〔坚谓筑土。〕 为墟四。

〔墟谓穿坑。此皆其常率。〕 以穿地求壤,五之;求坚,三之;皆四而一。

〔今有术也。〕 以壤求穿,四之;求坚,三之;皆五而一。以坚求穿,四之;求壤,五之; 皆三而一。

〔淳风等按:此术并今有之义也。重张穿地积一万尺,为所有数,坚率三、 壤率五各为所求率,穿率四为所有率,而今有之,即得。〕 城、垣、堤、沟、堑、渠皆同术。

术曰:并上下广而半之, 〔损广补狭。〕 以高若深乘之,又以袤乘之,即积尺。

〔按:此术“并上下广而半之”者,以盈补虚,得中平之广。“以高若深乘 之”,得一头之立幂。“又以袤乘之”者,得立实之积,故为积尺。〕 今有穿地,袤一丈六尺,深一丈,上广六尺,为垣积五百七十六尺。问穿地 下广几何?答曰:三尺五分尺之三。

术曰:置垣积尺,四之为实。

〔穿地四,为坚三。垣,坚也。以坚求穿地,当四之,三而一也。〕 以深、袤相乘, 〔为深、袤之立实也。〕 又三之,为法。

〔以深、袤乘之立实除垣积,即坑广。又三之者,与坚率并除之。〕 所得,倍之。

〔为坑有两广,先并而半之,即为广狭之中平。今先得其中平,故又倍之知, 两广全也。〕 减上广,余即下广。

〔按:此术穿地四,为坚三。垣即坚也。今以坚求穿地,当四乘之,三而一。

深、袤相乘者,为深袤立幂。以深袤立幂除积,即坑广。又三之,为法,与坚率 并除。所得,倍之者,为坑有两广,先并而半之,为中平之广。今此得中平之广, 故倍之还为两广并。故减上广,余即下广也。〕 今有城下广四丈,上广二丈,高五丈,袤一百二十六丈五尺。问积几何?答 曰:一百八十九万七千五百尺: 今有垣下广三尺,上广二尺,高一丈二尺,袤二十二丈五尺八寸。问积几何? 答曰:六千七百七十四尺。

今有堤下广二丈,上广八尺,高四尺,袤一十二丈七尺。问积几何?答曰: 七千一百一十二尺。

冬程人功四百四十四尺,问用徒几何?答曰:一十六人二百一十一分人之二。

术曰:以积尺为实,程功尺数为法,实如法而一,即用徒人数。

今有沟,上广一丈五尺,下广一丈,深五尺,袤七丈。问积几何?答曰:四 千三百七十五尺。

春程人功七百六十六尺,并出土功五分之一,定功六百一十二尺五分尺之四。

问用徒几何?答曰:七人三千六十四分人之四百二十七。

术曰:置本人功,去其五分之一,余为法。

〔“去其五分之一”者,谓以四乘,五除也。〕 以沟积尺为实,实如法而一,得用徒人数。

〔按:此术“置本人功,去其五分之一”者,谓以四乘之,五而一,除去出 土之功,取其定功。乃通分内子以为法。以分母乘沟积尺为实者,法里有分,实 里通之,故实如法而一,即用徒人数。此以一人之积尺除其众尺,故用徒人数。

不尽者,等数约之而命分也。〕 今有堑,上广一丈六尺三寸,下广一丈,深六尺三寸,袤一十三丈二尺一寸。

问积几何?答曰:一万九百四十三尺八寸。

〔八寸者,谓穿地方尺,深八寸。此积余有方尺中二分四厘五毫,弃之。文 欲从易,非其常定也。〕 夏程人功八百七十一尺,并出土功五分之一,沙砾水石之功作太半,定功二 百三十二尺一十五分尺之四。问用徒几何?答曰:四十七人三千四百八十四分人 之四百九。

术曰:置本人功,去其出土功五分之一,又去沙砾水石之功太半,余为法。

以堑积尺为实。实如法而一,即用徒人数。

〔按:此术“置本人功,去其出土功五分之一”者,谓以四乘,五除。“又 去沙砾水石作太半”者,一乘,三除,存其少半,取其定功。乃通分内子以为法。

以分母乘堑积尺为实者,为法里有分,实里通之,故实如法而一,即用徒人数。

不尽者,等数约之而命分也。〕 今有穿渠,上广一丈八尺,下广三尺六寸,深一丈八尺,袤五万一千八百二 十四尺。问积几何?答曰:一千七万四千五百八十五尺六寸。

秋程人功三百尺,问用徒几何?答曰:三万三千五百八十二人,功内少一十 四尺四寸。

一千人先到,问当受袤几何?答曰:一百五十四丈三尺二寸八十一分寸之八。

术曰:以一人功尺数乘先到人数为实。

〔以一千人一日功为实。立实为功。〕 并渠上下广而半之,以深乘之,为法。

〔以渠广深之立实为法。〕 实如法得袤尺。

今有方堡壔, 〔堡者,堡城也;壔,音丁老反,又音纛,谓以土拥木也。〕 方一丈六尺,高一丈五尺。问积几何?答曰:三千八百四十尺。

术曰:方自乘,以高乘之,即积尺。

今有圆堡瑽,周四丈八尺,高一丈一尺。问积几何?答曰:二千一百一十二 尺。

〔于徽术,当积二千一十七尺一百五十七分尺之一百三十一。

淳风等按:依密率,积二千一十六尺。〕 术曰:周自相乘,以高乘之,十二而一。

〔此章诸术亦以周三径一为率,皆非也。于徽术当以周自乘,以高乘之,又 以二十五乘之,三百一十四而一。此之圆幂亦如圆田之幂也。求幂亦如圆田,而 以高乘幂也。

淳风等按:依密率,以七乘之,八十八而一。〕 今有方亭,下方五丈,上方四丈,高五丈。问积几何?答曰:一十万一千六 百六十六尺太半尺。

术曰:上下方相乘,又各自乘,并之,以高乘之,三而一。

〔此章有堑堵、陽马,皆合而成立方。盖说算者乃立棋三品,以效高深之积。

假令方亭,上方一尺,下方三尺,高一尺。其用棋也,中央立方一,四面堑堵四, 四角陽马四。上下方相乘为三尺,以高乘之,得积三尺,是为得中央立方一,四 面堑堵各一。下方自乘为九,以高乘之,得积九尺。是为中央立方一、四面堑堵 各二、四角陽马各三也。上方自乘,以高乘之,得积一尺,又为中央立方一。凡 三品棋皆一而为三,故三而一,得积尺。用棋之数:立方三、堑堵陽马各十二, 凡二十七,棋十三。更差次之,而成方亭者三,验矣。为术又可令方差自乘,以 高乘之,三而一,即四陽马也;上下方相乘,以高乘之,即中央立方及四面堑堵 也。并之,以为方亭积数也。〕 今有圆亭,下周三丈,上周二丈,高一丈。问积几何?答曰:五百二十七尺 九分尺之七。

〔于徽术,当积五百四尺四百七十一分尺之一百一十六也。

淳风等按:依密率,为积五百三尺三十三分尺之二十六。〕 术曰:上下周相乘,又各自乘,并之,以高乘之,三十六而一。

〔此术周三径一之义。合以三除上下周,各为上下径。以相乘,又各自乘, 并,以高乘之,三而一,为方亭之积。假令三约上下周俱不尽,还通之,即各为 上下径。令上下径相乘,又各自乘,并,以高乘之,为三方亭之积分。此合分母 三相乘得九,为法,除之。又三而一,得方亭之积。从方亭求圆亭之积,亦犹方 幂中求圆幂。乃令圆率三乘之,方率四而一,得圆亭之积。前求方亭之积,乃以 三而一;今求圆亭之积,亦合三乘之。二母既同,故相准折,惟以方幂四乘分母 九,得三十六,而连除之。于徽术,当上下周相乘,又各自乘,并,以高乘之, 又二十五乘之,九百四十二而一。此方亭四角圆杀,比于方亭,二百分之一百五 十七。为术之意,先作方亭,三而一。则此据上下径为之者,当又以一百五十七 乘之,六百而一也。今据周为之,若于圆堡昪,又以二十五乘之,三百一十四而 一,则先得三圆亭矣。故以三百一十四为九百四十二而一,并除之。

淳风等按:依密率,以七乘之,二百六十四而一。〕 今有方锥,下方二丈七尺,高二丈九尺。问积几何?答曰:七千四十七尺。

术曰:下方自乘,以高乘之,三而一。

〔按:此术假令方锥下方二尺,高一尺,即四陽马。如术为之,用十二陽马 成三方锥。故三而一,得方锥也。〕 今有圆锥,下周三丈五尺,高五丈一尺。问积几何?答曰:一千七百三十五 尺一十二分尺之五。

〔于徽术,当积一千六百五十八尺三百一十四分尺之十三。

淳风等按:依密率,为积一千六百五十六尺八十八分尺之四十七。〕 术曰:下周自乘,以高乘之,三十六而一。

〔按:此术圆锥下周以为方锥下方。方锥下方令自乘,以高乘之,令三而一, 得大方锥之积。大锥方之积合十二圆矣。今求一圆,复合十二除之,故令三乘十 二,得三十六,而连除。于徽术,当下周自乘,以高乘之,又以二十五乘之,九 百四十二而一。圆锥比于方锥亦二百分之一百五十七。令径自乘者,亦当以一百 五十七乘之,六百而一。其说如圆亭也。

淳风等按:依密率,以七乘之,二百六十四而一。〕 今有堑堵,下广二丈,袤一十八丈六尺,高二丈五尺。问积几何?答曰:四 万六千五百尺。

术曰:广袤相乘,以高乘之,二而一。

〔邪解立方,得两堑堵。虽复橢方,亦为堑堵。故二而一。此则合所规棋。

推其物体,盖为堑上叠也。其形如城,而无上广,与所规棋形异而同实。未闻所 以名之为堑堵之说也。〕 今有陽马,广五尺,袤七尺,高八尺。问积几何?答曰:九十三尺少半尺。

术曰:广袤相乘,以高乘之,三而一。

〔按:此术陽马之形,方锥一隅也。今谓四柱屋隅为陽马。假令广袤各一尺, 高一尺,相乘,得立方积一尺。邪解立方,得两堑堵;邪解堑堵,其一为陽马, 一为鳖臑。陽马居二,鳖臑居一,不易之率也。合两鳖臑成一陽马,合三陽马而 成一立方,故三而一。验之以棋,其形露矣。悉割陽马,凡为六鳖臑。观其割分, 则体势互通,盖易了也。其棋或修短、或广狭、立方不等者,亦割分以为六鳖臑。

其形不悉相似。然见数同,积实均也。鳖臑殊形,陽马异体。然陽马异体,则不 纯合。不纯合,则难为之矣。何则?按:邪解方棋以为堑堵者,必当以半为分; 邪解堑堵以为陽马者,亦必当以半为分,一从一横耳。设以陽马为分内,鳖臑为 分外。棋虽或随修短广狭,犹有此分常率知,殊形异体,亦同也者,以此而已。

其使鳖臑广、袤、高各二尺,用堑堵、鳖臑之棋各二,皆用赤棋。又使陽马之广、 袤、高各二尺,用立方之棋一,堑堵、陽马之棋各二,皆用黑棋。棋之赤、黑, 接为堑堵,广、袤、高各二尺。于是中攽其广、袤,又中分其高。令赤、黑堑堵 各自适当一方,高一尺,方一尺,每二分鳖臑,则一陽马也。其余两端各积本体, 合成一方焉。是为别种而方者率居三,通其体而方者率居一。虽方随棋改,而固 有常然之势也。按:余数具而可知者有一、二分之别,则一、二之为率定矣。其 于理也岂虚矣。若为数而穷之,置余广、袤、高之数,各半之,则四分之三又可 知也。半之弥少,其余弥细,至细曰微,微则无形。由是言之,安取余哉?数而 求穷之者,谓以情推,不用筹算。鳖臑之物,不同器用;陽马之形,或随修短广 狭。然不有鳖臑,无以审陽马之数,不有陽马,无以知锥亭之数,功实之主也。〕 今有鳖臑,下广五尺,无袤;上袤四尺,无广;高七尺。问积几何?答曰: 二十三尺少半尺。

术曰:广袤相乘,以高乘之,六而一。

〔按:此术臑者,臂节也。或曰:半陽马,其形有似鳖肘,故以名云。中破 陽马,得两鳖臑。鳖臑之见数即陽马之半数。数同而实据半,故云六而一,即得。〕 今有羡除,下广六尺,上广一丈,深三尺;末广八尺,无深;袤七尺。问积 几何?答曰:八十四尺。

术曰:并三广,以深乘之,又以袤乘之,六而一。

〔按:此术羡除,实隧道也。其所穿地,上平下邪,似两鳖臑夹一堑堵,即 羡除之形。假令用此棋:上广三尺,深一尺,下广一尺;末广一尺,无深;袤一 尺。下广、末广皆堑堵之广。上广者,两鳖臑与一堑堵相连之广也。以深、袤乘, 得积五尺。鳖臑居二,堑堵居三,其于本棋皆一为六,故六而一。合四陽马以为 方锥。邪画方锥之底,亦令为中方。就中方削而上合,全为中方锥之半。于是陽 马之棋悉中解矣。中锥离而为四鳖臑焉。故外锥之半亦为四鳖臑。虽背正异形, 与常所谓鳖臑参不相似,实则同也。所云夹堑堵者,中锥之鳖臑也。凡堑堵上袤 短者,连陽马也。下袤短者,与鳖臑连也。上、下两袤相等知,亦与鳖臑连也。

并三广,以高、袤乘,六而一,皆其积也。今此羡除之广即堑堵之袤也。按: 此本是三广不等,即与鳖臑连者。别而言之:中央堑堵广六尺,高三尺,袤七尺。

末广之两旁,各一小鳖臑,皆与堑堵等。令小鳖臑居里,大鳖臑居表,则大鳖臑 皆出橢方锥:下广二尺,袤六尺,高七尺。分取其半,则为袤三尺。以高、广乘 之,三而一,即半锥之积也。邪解半锥得此两大鳖臑。求其积,亦当六而一,合 于常率矣。按:陽马之棋两邪,棋底方。当其方也,不问旁角而割之,相半可知 也。推此上连无成不方,故方锥与陽马同实。角而割之者,相半之势。此大小鳖 臑可知更相表里,但体有背正也。〕 今有刍甍,下广三丈,袤四丈;上袤二丈,无广;高一丈。问积几何?答曰: 五千尺。

术曰:倍下袤,上袤从之,以广乘之,又以高乘之,六而一。

〔推明义理者:旧说云:“凡积刍有上下广曰童,甍,谓其屋盖之苫也。” 是故甍之下广、袤与童之上广、袤等。正解方亭两边,合之即刍甍之形也。假令 下广二尺,袤三尺;上袤一尺,无广;高一尺。其用棋也,中央堑堵二,两端陽 马各二。倍下袤,上袤从之,为七尺。以下广乘之,得幂十四尺。陽马之幂各居 二,堑堵之幂各居三。以高乘之,得积十四尺。其于本棋也,皆一而为六。故六 而一,即得。亦可令上下袤差乘广,以高乘之,三而一,即四陽马也;下广乘上 袤而半之,高乘之,即二堑堵;并之,以为甍积也。〕 刍童、曲池、盘池、冥谷皆同术。

术曰:倍上袤,下袤从之;亦倍下袤,上袤从之;各以其广乘之,并,以高 若深乘之,皆六而一。

〔按:此术假令刍童上广一尺,袤二尺;下广三尺,袤四尺;高一尺。其用 棋也,中央立方二,四面堑堵六,四角陽马四。倍下袤为八,上袤从之,为十, 以高、广乘之,得积三十尺。是为得中央立方各三,两端堑堵各四,两旁堑堵各 六,四角陽马亦各六。复倍上袤,下袤从之,为八,以高、广乘之,得积八尺。

是为得中央立方亦各三,两端堑堵各二。并两旁,三品棋皆一而为六。故六而一, 即得。为术又可令上下广袤差相乘,以高乘之,三而一,亦四陽马;上下广袤 互相乘,并,而半之,以高乘之,即四面六堑堵与二立方;并之,为刍童积。又 可令上下广袤互相乘而半之,上下广袤又各自乘,并,以高乘之,三而一,即得 也。〕 其曲池者,并上中、外周而半之,以为上袤;亦并下中、外周而半之,以为 下袤。

〔此池环而不通匝,形如盘蛇,而曲之。亦云周者,谓如委谷依垣之周耳。

引而伸之,周为袤。求袤之意,环田也。〕 今有刍童,下广二丈,袤三丈;上广三丈,袤四丈;高三丈。问积几何?答 曰:二万六千五百尺。

今有曲池,上中周二丈,外周四丈,广一丈;下中周一丈四尺,外周二丈四 尺,广五尺;深一丈。问积几何?答曰:一千八百八十三尺三寸少半寸。

今有盘池,上广六丈,袤八丈;下广四丈,袤六丈,深二丈。问积几何?答 曰:七万六百六十六尺太半尺。

负土往来七十步,其二十步上下棚除,棚除二当平道五;踟蹰之间十加一; 载输之间三十步,定一返一百四十步。土笼积一尺六寸。秋程人功行五十九里半。

问人到积尺及用徒各几何?答曰:人到二百四尺。用徒三百四十六人一百五十三 分人之六十二。

术曰:以一笼积尺乘程行步数,为实。往来上下棚除二当平道五。

〔棚,阁;除,斜道;有上下之难,故使二当五也。〕 置定往来步数,十加一,及载输之间三十步,以为法。除之,所得即一人所 到尺。以所到约积尺,即用徒人数。

〔按:此术棚,阁;除,斜道;有上下之难,故使二当五。置定往来步数, 十加一,及载输之间三十步,是为往来一返凡用一百四十步。于今有术为所有率, 笼积一尺六寸为所求率,程行五十九里半为所有数,而今有之,即所到尺数。以 所到约积尺,即用徒人数者,此一人之积除其众积尺,故得用徒人数。为术又 可令往来一返所用之步约程行为返数,乘笼积为一人所到。以此术与今有术相 反覆,则乘除之或先后,意各有所在而同归耳。〕 今有冥谷,上广二丈,袤七丈;下广八尺,袤四丈;深六丈五尺。问积几何? 答曰:五万二千尺。

载土往来二百步,载输之间一里。程行五十八里;六人共车,车载三十四尺 七寸。问人到积尺及用徒各几何?答曰:人到二百一尺五十分尺之十三。用徒二 百五十八人一万六十三分人之三千七百四十六。

术曰:以一车积尺乘程行步数,为实。置今往来步数,加载输之间一里,以 车六人乘之,为法。除之,所得即一人所到尺。以所到约积尺,即用徒人数。

〔按:此术今有之义。以载输及往来并得五百步,为所有率,车载三十四尺 七寸为所求率,程行五十八里,通之为步,为所有数,而今有之,所得即一车所 到。欲得人到者,当以六人除之,即得。术有分,故亦更令乘法而并除者,亦用 以车尺数以为一人到土率,六人乘五百步为行率也。又亦可五百步为行率,令六 人约车积尺数为一人到土率,以负土术入之。入之者,亦可求返数也。要取其会 通而已。术恐有分,故令乘法而并除。以所到约积尺,即用徒人数者,以一人所 到积尺除其众积,故得用徒人数也。〕 今有委粟平地,下周一十二丈,高二丈。问积及为粟几何?答曰:积八千尺。

〔于徽术,当积七千六百四十三尺一百五十七分尺之四十九。

淳风等按:依密率,为积七千六百三十六尺十一分尺之四。〕 为粟二千九百六十二斛二十七分斛之二十六。

〔于徽术,当粟二千八百三十斛一千四百一十三分斛之一千二百一十。

淳风等按:依密率,为粟二千八百二十八斛九十九分斛之二十八。〕 今有委菽依垣,下周三丈,高七尺。问积及为菽各几何?答曰:积三百五十 尺。

〔依徽术,当积三百三十四尺四百七十一分尺之一百八十六。

淳风等按:依密率,为积三百三十四尺十一分尺之一。〕 为菽一百四十四斛二百四十三分斛之八。

〔依徽术,当菽一百三十七斛一万二千七百一十七分斛之七千七百七十一。

淳风等按:依密率,为菽一百三十七斛八百九十一分斛之四百三十三。〕 今有委米依垣内角,下周八尺,高五尺。问积及为米各几何?答曰:积三十 五尺九分尺之五。

〔于徽术,当积三十三尺四百七十一分尺之四百五十七。

淳风等按:依密率,当积三十三尺三十三分尺之三十一。〕 为米二十一斛七百二十九分斛之六百九十一。

〔于徽术,当米二十斛三万八千一百五十一分斛之三万六千九百八十。

淳风等按:依密率,为米二十斛二千六百七十三分斛之二千五百四十。〕 委粟术曰:下周自乘,以高乘之,三十六而一。

〔此犹圆锥也。于徽术,亦当下周自乘,以高乘之,又以二十五乘之,九百 四十二而一也。〕 其依垣者, 〔居圆锥之半也。〕 十八而一。

〔于徽术,当令此下周自乘,以高乘之,又以二十五乘之,四百七十一而一。

依垣之周,半于全周。其自乘之幂居全周自乘之幂四分之一,故半全周之法以为 法也。〕 其依垣内角者, 〔角,隅也,居圆锥四分之一也。〕 九而一。

〔于徽术,当令此下周自乘,而倍之,以高乘之,又以二十五乘之,四百七 十一而一。依隅之周,半于依垣。其自乘之幂居依垣自乘之幂四分之一,当半依 垣之法以为法。法不可半,故倍其实。又此术亦用周三径一之率。假令以三除周, 得径;若不尽,通分内子,即为径之积分。令自乘,以高乘之,为三方锥之积分。

母自相乘得九,为法,又当三而一,得方锥之积。从方锥中求圆锥之积,亦犹方 幂求圆幂。乃当三乘之,四而一,得圆锥之积。前求方锥积,乃以三而一;今求 圆锥之积,复合三乘之。二母既同,故相准折。惟以四乘分母九,得三十六而连 除,圆锥之积。其圆锥之积与平地聚粟同,故三十六而一。

淳风等按:依密率,以七乘之,其平地者,二百六十四而一;依垣者,一百 三十二而一;依隅者,六十六而一也。〕 程粟一斛积二尺七寸; 〔二尺七寸者,谓方一尺,深二尺七寸,凡积二千七百寸。〕 其米一斛积一尺六寸五分寸之一; 〔谓积一千六百二十寸。〕 其菽、荅、麻、麦一斛皆二尺四寸十分寸之三。

〔谓积二千四百三十寸。此为以精粗为率,而不等其概也。粟率五,米率三, 故米一斛于粟一斛,五分之三;菽、荅、麻、麦亦如本率云。故谓此三量器为概, 而皆不合于今斛。当今大司农斛,圆径一尺三寸五分五厘,正深一尺,于徽术, 为积一千四百四十一寸,排成余分,又有十分寸之三。王莽铜斛于今尺为深九寸 五分五厘,径一尺三寸六分八厘七毫。以徽术计之,于今斛为容九斗七升四合有 奇。《周官·考工记》:朅氏为量,深一尺,内方一尺而圆外,其实一釜。于徽 术,此圆积一千五百七十寸。《左氏传》曰:“齐旧四量:豆、区、釜、钟。四 升曰豆,各自其四,以登于釜。釜十则钟。”钟六斛四斗。釜六斗四升,方一尺, 深一尺,其积一千寸。若此方积容六斗四升,则通外圆积成旁,容十斗四合一龠 五分龠之三也。以数相乘之,则斛之制:方一尺而圆其外,庣旁一厘七毫,幂一 百五十六寸四分寸之一,深一尺,积一千五百六十二寸半,容十斗。王莽铜斛与 《汉书·律历志》所论斛同。〕 今有仓,广三丈,袤四丈五尺,容粟一万斛。问高几何?答曰:二丈。

术曰:置粟一万斛积尺为实。广、袤相乘为法。实如法而一,得高尺。

〔以广袤之幂除积,故得高。按:此术本以广袤相乘,以高乘之,得此积。

今还元,置此广袤相乘为法,除之,故得高也。〕 今有圆囷, 〔圆囷,廪也,亦云圆囤也。〕 高一丈三尺三寸少半寸,容米二千斛。问周几何?答曰:五丈四尺。

〔于徽术,当周五丈五尺二寸二十分寸之九。

淳风等按:依密率,为周五丈五尺一百分尺之二十七。〕 术曰:置米积尺, 〔此积犹圆堡昪之积。〕 以十二乘之,令高而一。所得,开方除之,即周。

〔于徽术,当置米积尺,以三百一十四乘之,为实。二十五乘囷高为法。所 得,开方除之,即周也。此亦据见幂以求周,失之于微少也。晋武库中有汉时王 莽所作铜斛,其篆书字题斛旁云:律嘉量斛,方一尺而圆其外,庣旁九厘五毫, 幂一百六十二寸;深一尺,积一千六百二十寸,容十斗。及斛底云:律嘉量斗, 方尺而圆其外,庣旁九厘五毫,幂一尺六寸二分。深一寸,积一百六十二寸,容 一斗。合、龠皆有文字。升居斛旁,合、龠在斛耳上。后有赞文,与今律历志同, 亦魏晋所常用。今粗疏王莽铜斛文字、尺、寸、分数,然不尽得升、合、勺之文 字。按:此术本周自相乘,以高乘之,十二而一,得此积。今还元,置此积,以 十二乘之,令高而一,即复本周自乘之数。凡物自乘,开方除之,复其本数。故 开方除之,即得也。

淳风等按:依密率,以八十八乘之,为实。七乘囷高为法。实如法而一。开 方除之,即周也。〕 

\chapter{均输}
(以御远近劳费) 今有均输粟,甲县一万户,行道八日;乙县九千五百户,行道十日;丙县一 万二千三百五十户,行道十三日;丁县一万二千二百户,行道二十日,各到输所。

凡四县赋当输二十五万斛,用车一万乘。欲以道里远近、户数多少衰出之,问粟、 车各几何?答曰:甲县粟八万三千一百斛,车三千三百二十四乘。乙县粟六万三 千一百七十五斛,车二千五百二十七乘。丙县粟六万三千一百七十五斛,车二千 五百二十七乘。丁县粟四万五百五十斛,车一千六百二十二乘。

术曰:令县户数各如其本行道日数而一,以为衰。

〔按:此均输,犹均运也。令户率出车,以行道日数为均,发粟为输。据甲 行道八日,因使八户共出一车;乙行道十日,因使十户共出一车。计其在道,则 皆户一日出一车,故可为均平之率也。

淳风等按:县户有多少之差,行道有远近之异。欲其均等,故各令行道日数 约户为衰。行道多者少其户,行道少者多其户。故各令约户为衰。以八日约除甲 县,得一百二十五,乙、丙各九十五,丁六十一。于今有术,副并为所有率。未 并者各为所求率,以赋粟车数为所有数,而今有之,各得车数。一旬除乙,十三 除丙,各得九十五;二旬除丁,得六十一也。〕 甲衰一百二十五,乙、丙衰各九十五,丁衰六十一,副并为法。以赋粟车数 乘未并者,各自为实。

〔衰,分科率。〕 实如法得一车。

〔各置所当出车,以其行道日数乘之,如户数而一,得率:户用车二日四十 七分日之三十一,故谓之均。求此户以率,当各计车之衰分也。〕 有分者,上下辈之。

〔辈,配也。车、牛、人之数不可分裂,推少就多,均赋之宜。今按:甲分 既少,宜从于乙。满法除之,有余从丙。丁分又少,亦宜就丙。除之适尽。加乙、 丙各一,上下辈益,以少从多也。〕 以二十五斛乘车数,即粟数。

今有均输卒:甲县一千二百人,薄塞;乙县一千五百五十人,行道一日;丙 县一千二百八十人,行道二日;丁县九百九十人,行道三日;戊县一千七百五十 人,行道五日。凡五县赋输卒一月一千二百人。欲以远近、人数多少衰出之,问 县各几何?答曰:甲县二百二十九人。乙县二百八十六人。丙县二百二十八人。

丁县一百七十一人。戊县二百八十六人。

术曰:令县卒各如其居所及行道日数而一,以为衰。

〔按:此亦以日数为均,发卒为输。甲无行道日,但以居所三十日为率。言 欲为均平之率者,当使甲三十人而出一人,乙三十一人而出一人。出一人者,计 役则皆一人一日,是以可为均平之率。〕 甲衰四,乙衰五,丙衰四,丁衰三,戊衰五,副并为法。以人数乘未并者各 自为实。实如法而一。

〔为衰,于今有术,副并为所有率,未并者各为所求率,以赋卒人数为所有 数。此术以别,考则意同,以广异闻,故存之也。各置所当出人数,以其居所及 行道日数乘之,如县人数而一。得率:人役五日七分日之五。〕 有分者,上下辈之。

〔辈,配也。今按:丁分最少,宜就戊除。不从乙者,丁近戊故也。满法除 之,有余从乙。丙分又少,亦就乙除,有余从甲。除之适尽。从甲、丙二分,其 数正等,二者于乙远近皆同,不以甲从乙者,方以下从上也。〕 今有均赋粟:甲县二万五百二十户,粟一斛二十钱,自输其县;乙县一万二 千三百一十二户,粟一斛一十钱,至输所二百里;丙县七千一百八十二户,粟一 斛一十二钱,至输所一百五十里;丁县一万三千三百三十八户,粟一斛一十七钱, 至输所二百五十里;戊县五千一百三十户,粟一斛一十三钱,至输所一百五十里。

凡五县赋输粟一万斛。一车载二十五斛,与僦一里一钱。欲以县户赋粟,令费劳 等,问县各粟几何?答曰:甲县三千五百七十一斛二千八百七十三分斛之五百一 十七。乙县二千三百八十斛二千八百七十三分斛之二千二百六十。丙县一千三百 八十八斛二千八百七十三分斛之二千二百七十六。丁县一千七百一十九斛二千八 百七十三分斛之一千三百一十三。戊县九百三十九斛二千八百七十三分斛之二千 二百五十三。

术曰:以一里僦价乘至输所里, 〔此以出钱为均也。问者曰:“一车载二十五斛,与僦一里一钱。”一钱, 即一里僦价也。以乘里数者,欲知僦一车到输所所用钱也。甲自输其县,则无取 僦价也。〕 以一车二十五斛除之, 〔欲知僦一斛所用钱。〕 加一斛粟价,则致一斛之费。

〔加一斛之价于一斛僦直,即凡输粟取僦钱也:甲一斛之费二十,乙、丙各 十八,丁二十七,戊十九也。〕 各以约其户数,为衰。

〔言使甲二十户共出一斛,乙、丙十八户共出一斛。计其所费,则皆户一钱, 故可为均赋之率也。计经赋之率,既有户算之率,亦有远近、贵贱之率。此二率 者,各自相与通。通则甲二十,乙十二,丙七,丁十三,戊五。一斛之费谓之钱 率。钱率约户率者,则钱为母,户为子。子不齐,令母互乘为齐,则衰也。若其 不然。以一斛之费约户数,取衰。并有分,当通分内子,约之,于算甚繁。此一 章皆相与通功共率,略相依似。以上二率、下一率亦可放此,从其简易而已。又 以分言之,使甲一户出二十分斛之一,乙一户出十八分斛之一,各以户数乘之, 亦可得一县凡所当输,俱为衰也。乘之者,乘其子,母报除之。以此观之,则以 一斛之费约户数者,其意不异矣。然则可置一斛之费而反衰之。约户,以乘户率 为衰也。合分注曰:“母除为率,率乘子为齐。”反衰注曰:“先同其母,各以 分母约,其子为反衰。”以施其率,为算既约,且不妨处下也。〕 甲衰一千二十六,乙衰六百八十四,丙衰三百九十九,丁衰四百九十四,戊 衰二百七十,副并为法。所赋粟乘未并者,各自为实。实如法得一。

〔各置所当出粟,以其一斛之费乘之,如户数而一,得率:户出三钱二千八 百七十三分钱之一千三百八十一。按:此以出钱为均。问者曰:“一车载二十五 斛,与僦一里一钱。”一钱即一里僦价也。以乘里数者,欲知僦一车到输所用钱。

甲自输其县,则无取僦之价。以一车二十五斛除之者,欲知僦一斛所用钱。加一 斛之价于一斛僦直,即凡输粟取僦钱:甲一斛之费二十,乙、丙各十八,丁二十 七,戊一十九。各以约其户,为衰:甲衰一千二十六,乙衰六百八十四,丙衰三 百九十九,丁衰四百九十四,戊衰二百七十。言使甲二十户共出一斛,乙、丙十 八户共出一斛。计其所费,则皆户一钱,故可为均赋之率也。于今有术,副并为 所有率,未并者各为所求率,赋粟一万斛为所有数。此今有、衰分之义也。〕 今有均赋粟:甲县四万二千算,粟一斛二十,自输其县;乙县三万四千二百 七十二算,粟一斛一十八,佣价一日一十钱,到输所七十里;丙县一万九千三百 二十八算,粟一斛一十六,佣价一日五钱,到输所一百四十里;丁县一万七千七 百算,粟一斛一十四,佣价一日五钱,到输所一百七十五里;戊县二万三千四十 算,粟一斛一十二,佣价一日五钱,到输所二百一十里;己县一万九千一百三十 六算,粟一斛一十,佣价一日五钱,到输所二百八十里。凡六县赋粟六万斛,皆 输甲县。六人共车,车载二十五斛,重车日行五十里,空车日行七十里,载输之 间各一日。粟有贵贱,佣各别价,以算出钱,令费劳等,问县各粟几何?答曰: 甲县一万八千九百四十七斛一百三十三分斛之四十九。乙县一万八百二十七斛一 百三十三分斛之九,丙县七千二百一十八斛一百三十三分斛之六。丁县六千七百 六十六斛一百三十三分斛之一百二十二。戊县九千二十二斛一百三十三分斛之七 十四。己县七千二百一十八斛一百三十三分斛之六。

术曰:以车程行空、重相乘为法,并空、重,以乘道里,各自为实,实如法 得一日。

〔按:此术重往空还,一输再行道也。置空行一里用七十分日之一,重行一 里用五十分日之一。齐而同之,空、重行一里之路,往返用一百七十五分日之六。

完言之者,一百七十五里之路,往返用六日也。故并空、重者,齐其子也;空、 重相乘者,同其母也。于今有术,至输所里为所有数,六为所求率,一百七十五 为所有率,而今有之,即各得输所用日也。〕 加载输各一日, 〔故得凡日也。〕 而以六人乘之, 〔欲知致一车用人也。〕 又以佣价乘之, 〔欲知致车人佣直几钱。〕 以二十五斛除之, 〔欲知致一斛之佣直也。〕 加一斛粟价,即致一斛之费。

〔加一斛之价于致一斛之佣直,即凡输一斛粟取佣所用钱。〕 各以约其算数为衰, 〔今按:甲衰四十二,乙衰二十四,丙衰十六,丁衰十五,戊衰二十,己衰 十六。于今有术,副并为所有率,未并者各自为所求率,所赋粟为所有数。此今 有、衰分之义也。〕 副并为法,以所赋粟乘未并者,各自为实。实如法得一斛。

〔各置所当出粟,以其一斛之费乘之,如算数而一,得率:算出九钱一百三 十三分钱之三。又载输之间各一日者,即二日也。〕 今有粟七斗,三人分舂之,一人为粝米,一人为粺米,一人为米, 令米数等。问取粟、为米各几何?答曰:粝米取粟二斗一百二十一分斗之一十。

粺米取粟二斗一百二十一分斗之三十八。米取粟二斗一百二十一分斗之 七十三。为米各一斗六百五分斗之一百五十一。

术曰:列置粝米三十,粺米二十七,米二十四,而反衰之。

〔此先约三率:粝为十,粺为九,为八。欲令米等者,其取粟:粝 率十分之一,粺率九分之一,率八分之一。当齐其子,故曰反衰也。

淳风等按:米有精粗之异,粟有多少之差。据率,粺、少而粝多; 用粟,则粺、多而粝少。米若依本率之分,粟当倍率,故今反衰之,使 精取多而粗得少。〕 副并为法。以七斗乘未并者,各自为取粟实。实如法得一斗。

〔于今有术,副并为所有率,未并者各为所求率,粟七斗为所有数,而今有 之,故各得取粟也。〕 若求米等者,以本率各乘定所取粟为实,以粟率五十为法,实如法得一斗。

〔若径求为米等数者,置粝米三,用粟五;粺米二十七,用粟五十; 米十二,用粟二十五。齐其粟,同其米,并齐为法。以七斗乘同为实。所得,即 为米斗数。〕 今有人当禀粟二斛。仓无粟,欲与米一、菽二,以当所禀粟。问各几何?答 曰;米五斗一升七分升之三。菽一斛二升七分升之六。

术曰:置米一、菽二,求为粟之数。并之,得三、九分之八,以为法。亦置 米一、菽二,而以粟二斛乘之,各自为实。实如法得一斛。

〔淳风等按:置粟率五,乘米一,米率三除之,得一、三分之二,即是米一 之粟也;粟率十,以乘菽二,菽率九除之,得二、九分之二,即是菽二之粟也。

并全,得三。齐子,并之,得二十四;同母,得二十七;约之,得九分之八。故 云“并之,得三、九分之八”。米一、菽二当粟三、九分之八,此其粟率也。于 今有术,米一、菽二皆为所求率,当粟三、九分之八,为所有率,粟二斛为所有 数。凡言率者,当相与。通之,则为米九、菽十八,当粟三十五也。亦有置米 一、菽二,求其为粟之率,以为列衰。副并为法,以粟乘列衰为实。所得即米一、 菽二所求粟也。以米、菽本率而今有之,即合所问。〕 今有取佣,负盐二斛,行一百里,与钱四十。今负盐一斛七斗三升少半升, 行八十里。问与钱几何?答曰:二十七钱一十五分钱之一十一。

术曰:置盐二斛升数,以一百里乘之为法。

〔按:此术以负盐二斛升数乘所行一百里,得二万里。是为负盐一升行二万 里,得钱四十。于今有术,为所有率。〕 以四十钱乘今负盐升数,又以八十里乘之,为实。实如法得一钱。

〔以今负盐升数乘所行里,今负盐一升凡所行里也。于今有术以所有数,四 十钱为所求率也。衰分章“贷人千钱”与此同。〕 今有负笼重一石,行百步,五十返。今负笼重一石一十七斤,行七十六步, 问返几何?答曰:五十七返二千六百三分返之一千六百二十九。

术曰:以今所行步数乘今笼重斤数,为法。

〔此法谓负一斤一返所行之积步也。〕 故笼重斤数乘故步,又以返数乘之,为实。实如法得一返。

〔按:此法,负一斤一返所行之积步;此实者一斤一日所行之积步。故以一 返之课除终日之程,即是返数也。

淳风等按:此术,所行步多者得返少,所行步少者得返多。然则故所行者今 返率也。故令所得返乘今返之率,为实,而以故返之率为法,今有术也。按:此 负笼又有轻重,于是为术者因令重者得返少,轻者得返多。故又因其率以乘法、 实者,重今有之义也。然此意非也。按:此笼虽轻而行有限,笼过重则人力遗。

力有遗而术无穷,人行有限而笼轻重不等。使其有限之力随彼无穷之变,故知此 术率乖理也。若故所行有空行返数,设以问者,当因其所负以为返率,则今返之 数可得而知也。假令空行一日六十里,负重一斛行四十里。减重一斗进二里半, 负重二斗以下与空行同。今负笼重六斗,往返行一百步,问返几何?答曰:一百 五十返。术曰:置重行率,加十里,以里法通之,为实。以一返之步为法。实如 法而一,即得也。〕 今有程传委输,空车日行七十里,重车日行五十里。今载太仓粟输上林,五 日三返,问太仓去上林几何?答曰:四十八里一十八分里之一十一 术曰:并空、重里数,以三返乘之,为法。令空、重相乘,又以五日乘之, 为实。实如法得一里。

〔此亦如上术。率:一百七十五里之路,往返用六日也。于今有术,则五日 为所有数,一百七十五里为所求率,六日为所有率。以此所得,则三返之路。今 求一返,当以三约之,因令乘法而并除也。为术亦可各置空、重行一里用日之率, 以为列衰,副并为法。以五日乘列衰为实。实如法,所得即各空、重行日数也。

各以一日所行以乘,为凡日所行。三返约之,为上林去太仓之数。按:此术重往 空还,一输再还道。置空行一里用七十分日之一,重行一里用五十分日之一。齐 而同之,空、重行一里之路,往返用一百七十五分日之六。完言之者,一百七十 五里之路,往返用六日。故并空、重者,并齐也;空、重相乘者,同其母也。于 今有术,五日为所有数,一百七十五为所求率,六为所有率。以此所得,则三返 之路。今求一返者,当以三约之。故令乘法而并除,亦当约之也。〕 今有络丝一斤为练丝一十二两,练丝一斤为青丝一斤一十二铢。今有青丝一 斤,问本络丝几何?答曰:一斤四两一十六铢三十三分铢之一十六。

术曰:以练丝十二两乘青丝一斤一十二铢为法。以青丝一斤铢数乘练丝一斤 两数,又以络丝一斤乘,为实。实如法得一斤。

〔按:练丝一斤为青丝一斤十二铢,此练率三百八十四,青率三百九十六也。

又络丝一斤为练丝十二两,此络率十六,练率十二也。置今有青丝一斤,以练率 三百八十四乘之,为实。实如青丝率三百九十六而一。所得,青丝一斤,练丝之 数也。又以络率十六乘之,所得为实;以练率十二为法。所得,即练丝用络丝之 数也。是谓重今有也。虽各有率,不问中间。故令后实乘前实,后法乘前法而并 除也。故以练丝两数为实,青丝铢数为法。一曰:又置络丝一斤两数与练丝十 二两,约之,络得四,练得三。此其相与之率。又置练丝一斤铢数与青丝一斤一 十二铢,约之,练得三十二,青得三十三。亦其相与之率。齐其青丝、络丝,同 其二练,络得一百二十八,青得九十九,练得九十六,即三率悉通矣。今有青丝 一斤为所有数,络丝一百二十八为所求率,青丝九十九为所有率。为率之意犹此, 但不先约诸率耳。凡率错互不通者,皆积齐同用之。放此,虽四五转不异也。言 同其二练者,以明三率之相与通耳,于术无以异也。又一术:今有青丝一斤铢 数乘练丝一斤两数,为实;以青丝一斤一十二铢为法。所得,即用练丝两数。以 络丝一斤乘所得为实,以练丝十二两为法,所得,即用络丝斤数也。〕 今有恶粟二十斗,舂之,得粝米九斗。今欲求粺米一十斗,问恶粟几何? 答曰:二十四斗六升八十一分升之七十四。

术曰:置粝米九斗,以九乘之,为法。亦置粺米十斗,以十乘之,又以恶 粟二十斗乘之,为实。实如法得一斗。

〔按:此术置今有求粺米十斗,以粝米率十乘之,如粺率九而一,即 粺化为粝,又以恶粟率二十乘之,如粝率九而一,即粝亦化为恶粟矣。此亦重 今有之义。为术之意犹络丝也。虽各有率,不问中间。故令后实乘前实,后法乘 前法而并除之也。〕 今有善行者行一百步,不善行者行六十步。今不善行者先行一百步,善行者 追之。问几何步及之?答曰:二百五十步。

术曰:置善行者一百步,减不善行者六十步,余四十步,以为法。以善行者 之一百步乘不善行者先行一百步,为实。实如法得一步。

〔按:此术以六十步减一百步,余四十步,即不善行者先行率也;善行者行 一百步,追及率。约之,追及率得五,先行率得二。于今有术,不善行者先行一 百步为所有数,五为所求率,二为所有率,而今有之,得追及步也。〕 今有不善行者先行一十里,善行者追之一百里,先至不善行者二十里。问善 行者几何里及之?答曰:三十三里少半里。

术曰:置不善行者先行一十里,以善行者先至二十里增之,以为法。以不善 行者先行一十里乘善行者一百里,为实。实如法得一里。

〔按:此术不善行者既先行一十里,后不及二十里,并之,得三十里也,谓 之先行率。善行者一百里为追及率。约之,先行率得三,三为所有率,而今有之, 即得也。其意如上术也。〕 今有兔先走一百步,犬追之二百五十步,不及三十步而止。问犬不止,复行 几何步及之?答曰:一百七步七分步之一。

术曰:置兔先走一百步,以犬走不及三十步减之,余为法。以不及三十步乘 犬追步数为实。实如法得一步。

〔按:此术以不及三十步减先走一百步,余七十步,为兔先走率。犬行二百 五十步为追及率。约之,先走率得七,追及率得二十五。于今有术,不及三十步 为所有数,二十五为所求率,七为所有率,而今有之,即得也。〕 今有人持金十二斤出关,关税之,十分而取一。今关取金二斤,偿钱五千。

问金一斤值钱几何?答曰:六千二百五十。

术曰:以一十乘二斤,以十二斤减之,余为法。以一十乘五千为实。实如法 得一钱。

〔按:此术置十二斤,以一乘之,十而一,得一斤五分斤之一,即所当税者 也。减二斤,余即关取盈金。以盈除所偿钱,即金值也。今术既以十二斤为所税, 则是以十为母,故以十乘二斤及所偿钱,通其率。于今有术,五千钱为所有数, 十为所求率,八为所有率,而今有之,即得也。〕 今有客马,日行三百里。客去忘持衣。日已三分之一,主人乃觉。持衣追及, 与之而还;至家视日四分之三。问主人马不休,日行几何?答曰:七百八十里。

术曰:置四分日之三,除三分日之一, 〔按:此术“置四分日之三,除三分日之一”者,除,其减也。减之余,有 十二分之五,即是主人追客还用日率也。〕 半其余,以为法。

〔去其还,存其往。率之者,子不可半,故倍母,二十四分之五。是为主人 与客均行用日之率也。〕 副置法,增三分日之一。

〔法二十四分之五者,主人往追用日之分也。三分之一者,客去主人未觉之 前独行用日之分也。并连此数,得二十四分日之十三,则主人追及前用日之分也。

是为客用日率也。然则主人用日率者,客马行率也;客用日率者,主人马行率也。

母同则子齐,是为客马行率五,主人马行率十三。于今有术,三百里为所有数, 十三为所求率,五为所有率,而今有之,即得也。〕 以三百里乘之,为实。实如法,得主人马一日行。

〔欲知主人追客所行里者,以三百里乘客用日分子十三,以母二十四而一, 得一百六十二里半。以此乘客马与主人均行日分母二十四,如客马与主人均行用 日分子五而一,亦得主人马一日行七百八十里也。〕 今有金棰,长五尺,斩本一尺,重四斤;斩末一尺,重二斤。问次一尺各重 几何?答曰:末一尺重二斤。次一尺重二斤八两。次一尺重三斤。次一尺重三斤 八两。次一尺重四斤。

术曰:令末重减本重,余,即差率也。又置本重,以四间乘之,为下第一衰。

副置,以差率减之,每尺各自为衰。

〔按:此术五尺有四间者,有四差也。今本末相减,余即四差之凡数也。以 四约之,即得每尺之差。以差数减本重,余即次尺之重也。为术所置,如是而已。

今此率以四为母,故令母乘本为衰,通其率也。亦可置末重,以四间乘之,为上 第一衰。以差重率加之,为次下衰也。〕 副置下第一衰,以为法。以本重四斤遍乘列衰,各自为实。实如法得一斤。

〔以下第一衰为法,以本重乘其分母之 数,而又反此率乘本重,为实。一乘 一除,势无损益,故惟本存焉。众衰相推为率,则其余可知也。亦可副置末衰为 法,而以末重二斤乘列衰为实。此虽迂回,然是其旧。故就新而言之也。〕 今有五人分五钱,令上二人所得与下三人等,问各得几何?答曰:甲得一钱 六分钱之二。乙得一钱六分钱之一。丙得一钱。丁得六分钱之五。戊得六分钱之 四。

术曰:置钱,锥行衰。

〔按:此术“锥行”者,谓如立锥:初一、次二、次三、次四、次五,各均, 为一列者也。〕 并上二人为九,并下三人为六。六少于九,三。

〔数不得等,但以五、四、三、二、一为率也。〕 以三均加焉,副并为法。以所分钱乘未并者,各自为实。实如法得一钱。

〔此问者,令上二人与下三人等,上、下部差一人,其差三。均加上部,则 得二三;均加下部,则得三三。下部犹差一人,差得三,以通于本率,即上、下 部等也。于今有术,副并为所有率,未并者各为所求率,五钱为所有数,而今有 之,即得等耳。假令七人分七钱,欲令上二人与下五人等,则上、下部差三人。

并上部为十三,下部为十五。下多上少,下不足减上。当以上、下部列差而后均 减,乃合所问耳。此可仿下术:令上二人分二钱半为上率,令下三人分二钱半为 下率。上、下二率以少减多,余为实。置二人、三人,各半之,减五人,余为法。

实如法得一钱,即衰相去也。下衰率六分之五者,丁所得钱数也。〕 今有竹九节,下三节容四升,上四节容三升。问中间二节欲均容,各多少? 答曰:下初一升六十六分升之二十九。次一升六十六分升之二十二。次一升六十 六分升之一十五。次一升六十六分升之八。次一升六十六分升之一。次六十六分 升之六十。次六十六分升之五十三。次六十六分升之四十六。次六十六分升之三 十九。

术曰:以下三节分四升为下率,以上四节分三升为上率。

〔此二率者,各其平率也。〕 上、下率以少减多,余为实。

〔按:此上、下节各分所容为率者,各其平率。上、下以少减多者,余为中 间五节半之凡差,故以为实也。〕 置四节、三节,各半之,以减九节,余为法。实如法得一升。即衰相去也。

〔按此术法者,上下节所容已定之节,中间相去节数也;实者,中间五节半 之凡差也。故实如法而一,则每节之差也。〕 下率一升少半升者,下第二节容也。

〔一升少半升者,下三节通分四升之平率。平率即为中分节之容也。〕 今有凫起南海,七日至北海;雁起北海,九日至南海。今凫、雁俱起,问何 日相逢?答曰:三日十六分日之十五。

术曰:并日数为法,日数相乘为实,实如法得一日。

〔按:此术置凫七日一至,雁九日一至。齐其至,同其日,定六十三日凫九 至,雁七至。今凫、雁俱起而问相逢者,是为共至。并齐以除同,即得相逢日。

故“并日数为法”者,并齐之意;“日数相乘为实”者,犹以同为实也。一曰: 凫飞日行七分至之一,雁飞日行九分至之一。齐而同之,凫飞定日行六十三分至 之九,雁飞定日行六十三分至之七。是为南北海相去六十三分,凫日行九分,雁 日行七分也。并凫、雁一日所行,以除南北相去,而得相逢日也。〕 今有甲发长安,五日至齐;乙发齐,七日至长安。今乙发已先二日,甲乃发 长安,问几何日相逢?答曰:二日十二分日之一。

术曰:并五日、七日,以为法。

〔按:此术“并五日、七日为法”者,犹并齐为法。置甲五日一至,乙七日 一至。齐而同之,定三十五日甲七至,乙五至。并之为十二至者,用三十五日也。

谓甲、乙与发之率耳。然则日化为至,当除日,故以为法也。〕 以乙先发二日减七日, 〔“减七日”者,言甲、乙俱发,今以发为始发之端,于本道里则余分也。〕 也。

余,以乘甲日数为实。

〔七者,长安去齐之率也;五者,后发相去之率也。今问后发,故舍七用五。

以乘甲五日,为二十五日。言甲七至,乙五至,更相去,用此二十五日也。

实如法得一日。

〔一日甲行五分至之一,乙行七分至之一。齐而同之,甲定日行三十五分至 之七,乙定日行三十五分至之五。是为齐去长安三十五分,甲日行七分,乙日行 五分也。今乙先行发二日,已行十分,余,相去二十五分。故减乙二日,余,令 相乘,为二十五分。〕 今有一人一日为牝瓦三十八枚,一人一日为牡瓦七十六枚。今令一人一日作 瓦,牝、牡相半,问成瓦几何?答曰:二十五枚少半枚。

术曰:并牝、牡为法,牝、牡相乘为实,实如法得一枚。

〔此意亦与凫雁同术。牝、牡瓦相并,犹如凫、雁日飞相并也。按:此术 “并牝、牡为法”者,并齐之意;“牝、牡相乘为实”者,犹以同为实也。故实 如法,即得也。〕 今有一人一日矫矢五十,一人一日羽矢三十,一人一日摐矢十五。今令一人 一日自矫、羽、摐,问成矢几何?答曰:八矢少半矢。

术曰:矫矢五十,用徒一人;羽矢五十,用徒一人太半人;摐矢五十,用徒 三人少半人。并之,得六人,以为法。以五十矢为实。实如法得一矢。

〔按:此术言成矢五十,用徒六人,一日工也。此同工其作,犹凫、雁共至 之类,亦以同为实,并齐为法。可令矢互乘一人为齐,矢相乘为同。今先令同于 五十矢。矢同则徒齐,其归一也。——以此术为凫雁者,当雁飞九日而一至,凫 飞九日而一至七分至之二。并之,得二至七分至之二,以为法。以九日为实。— —实如法而一,得一人日成矢之数也。〕 今有假田,初假之岁三亩一钱,明年四亩一钱,后年五亩一钱。凡三岁得一 百。问田几何?答曰:一顷二十七亩四十七分亩之三十一。

术曰:置亩数及钱数。令亩数互乘钱数,并,以为法。亩数相乘,又以百钱 乘之,为实。实如法得一亩。

〔按:此术令亩互乘钱者,齐其钱;亩数相乘者,同其亩。同于六十,则初 假之岁得钱二十,明年得钱十五,后年得钱十二也。凡三岁得钱一百,为所有数, 同亩为所求率,四十七钱为所有率,今有之,即得也。齐其钱,同其亩,亦如凫 雁术也。于今有术,百钱为所有数,同亩为所求率,并齐为所有率。

淳风等按:假田六十亩,初岁得钱二十,明年得钱十五,后年得钱十二。

并之,得钱四十七。是为得田六十亩,三岁所假。于今有术,百钱为所有数,六 十亩为所求率,四十七为所有率,而今有之,即合问也。〕 今有程耕,一人一日发七亩,一人一日耕三亩,一人一日耰种五亩。今令一 人一日自发、耕、耰种之,问治田几何?答曰:一亩一百一十四步七十一分步之 六十六。

术曰:置发、耕、耰亩数,令互乘人数,并,以为法。亩数相乘为实。实如 法得一亩。

〔此犹凫雁术也。

淳风等按:此术亦发、耕、耰种亩数互乘人者,齐其人;亩数相乘者,同 其亩。故并齐为法,以同为实。计田一百五亩,发用十五人,耕用三十五人,种 用二十一人。并之,得七十一工。治得一百五亩,故以为实。而一人一日所治, 故以人数为法除之,即得也。〕 今有池,五渠注之。其一渠开之,少半日一满,次一日一满,次二日半一满, 次三日一满,次五日一满。今皆决之,问几何日满池?答曰:七十四分日之十五。

术曰:各置渠一日满池之数,并,以为法。

〔按:此术其一渠少半日满者,是一日三满也;次一日一满;次二日半满者, 是一日五分满之二也;次三日满者,是一日三分满之一也;次五日满者,是一日 五分满之一也。并之,得四满十五分满之十四也。〕 以一日为实,实如法得一日。

〔此犹矫矢之术也。先令同于一日,日同则满齐。自凫雁至此,其为同齐有 二术焉,可随率宜也。〕 其一术:各置日数及满数。

〔其一渠少半日满者,是一日三满也;次一日一满;次二日半满者,是五日 二满;次三日一满,次五日一满。此谓之列置日数及满数也。〕 令日互相乘满,并,以为法。日数相乘为实。实如法得一日。

〔亦如凫雁术也。按:此其一渠少半日满池者,是一日三满池也;次一日一 满;次二日半满者,是五日再满;次三日一满;次五日一满。此谓列置日数于右 行,及满数于左行。以日互乘满者,齐其满;日数相乘者,同其日。满齐而日同, 故并齐以除同,即得也。〕 今有人持米出三关,外关三而取一,中关五而取一,内关七而取一,余米五 斗。问本持米几何?答曰:十斗九升八分升之三。

术曰:置米五斗,以所税者三之,五之,七之,为实。以余不税者二、四、 六相互乘为法。实如法得一斗。

〔此亦重今有也。所税者,谓今所当税之。定三、五、七皆为所求率,二、 四、六皆为所有率。置今有余米五斗,以七乘之,六而一,即内关未税之本米也。

又以五乘之,四而一,即中关未税之本米也。又以三乘之,二而一,即外关未税 之本米也。今从末求本,不问中间,故令中率转相乘而同之,亦如络丝术。

又一术:外关三而取一,则其余本米三分之二也。求外关所税之余,则当置 一,二分乘之,三而一。欲知中关,以四乘之,五而一。欲知内关,以六乘之, 七而一。凡余分者,乘其母、子:以三、五、七相乘得一百五,为分母;二、四、 六相乘,得四十八,为分子。约而言之,则是余米于本所持三十五分之十六也。

于今有术,余米五斗为所有数,分母三十五为所求率,分子十六为所有率也。〕 今有人持金出五关,前关二而税一,次关三而税一,次关四而税一,次关五 而税一,次关六而税一。并五关所税,适重一斤。问本持金几何?答曰:一斤三 两四铢五分铢之四。

术曰:置一斤,通所税者以乘之,为实。亦通其不税者,以减所通,余为法。

实如法得一斤。

〔此意犹上术也。“置一斤,通所税者”,谓令二、三、四、五、六相乘, 为分母,七百二十也。“通其所不税者”,谓令所税之余一、二、三、四、五相 乘,为分子,一百二十也。约而言之,是为余金于本所持六分之一也。以子减母, 凡五关所税六分之五也。于今有术,所税一斤为所有数,分母六为所求率,分子 五为所有率。此亦重今有之义。又虽各有率,不问中间,故令中率转相乘而连除 之,即得也。置一以为持金之本率,以税率乘之、除之,则其率亦成积分也。〕 

\chapter{盈不足}
(以御隐杂互见) 今有共买物,人出八,盈三;人出七,不足四。问人数、物价各几何?答曰: 七人。物价五十三。

今有共买鸡,人出九,盈一十一;人出六,不足十六。问人数、鸡价各几何? 答曰:九人。鸡价七十。

今有共买琎,人出半,盈四;人出少半,不足三。问人数、琎价各几何?答 曰:四十二人。琎价十七。

〔注云“若两设有分者,齐其子,同其母”,此问两设俱见零分,故齐其子, 同其母。又云“令下维乘上。讫,以同约之”,不可约,故以乘,同之。〕 今有共买牛,七家共出一百九十,不足三百三十;九家共出二百七十,盈三 十。问家数、牛价各几何?答曰:一百二十六家。牛价三千七百五十。

〔按:此术并盈不足者,为众家之差,故以为实。置所出率,各以家数除之, 各得一家所出率。以少减多者,得一家之差。以除,即家数。以出率乘之,减盈, 故得牛价也。〕 术曰:置所出率,盈不足各居其下。令维乘所出率,并,以为实。并盈、不 足,为法。实如法而一。

〔按:盈者,谓朓;不足者,谓之朒;所出率谓之假令。盈、朒维乘两 设者,欲为同齐之意。据“共买物,人出八,盈三;人出七,不足四”,齐其假 令,同其盈、朒,盈、朒俱十二。通计齐则不盈不朒之正数,故可并之为 实,并盈、不足为法。齐之三十二者,是四假令,有盈十二;齐之二十一者,是 三假令,亦朒十二;并七假令合为一实,故并三、四为法。〕 有分者通之。

〔若两设有分者,齐其子,同其母。令下维乘上,讫,以同约之。〕 盈不足相与同其买物者,置所出率,以少减多,余,以约法、实。实为物价, 法为人数。

〔“所出率以少减多”者,余,谓之设差,以为少设。则并盈、朒,是为 定实。故以少设约定实,则法,为人数;适足之实故为物价。盈朒当与少设相 通。不可遍约,亦当分母乘,设差为约法、实。〕 其一术曰:并盈、不足为实。以所出率,以少减多,余为法。实如法得一人。

以所出率乘之,减盈、增不足,即物价。

〔此术意谓盈不足为众人之差。以所出率以少减多,余为一人之差。以一人 之差约众人之差,故得人数也。〕 今有共买金,人出四百,盈三千四百;人出三百,盈一百。问人数、金价各 几何?答曰:三十三人。金价九千八百。

今有共买羊,人出五,不足四十五;人出七,不足三。问人数、羊价各几何? 答曰:二十一人。羊价一百五十。

术曰:置所出率,盈、不足各居其下。令维乘所出率,以少减多,余为实。

两盈、两不足以少减多,余为法。实如法而一。有分者,通之。两盈两不足相与 同其买物者,置所出率,以少减多,余,以约法、实。实为物价,法为人数。

〔按:此术两不足者,两设皆不足于正数。其所以变化,犹两盈。而或有势 同而情违者。当其为实,俱令不足维乘相减,则遗其所不足焉。故其余所以为实 者,无朒数以损焉。盖出而有余,两盈。两设皆逾于正数。假令与共买物,人 出八,盈三;人出九,盈十。齐其假令,同其两盈。两盈俱三十。举齐则兼去。

其余所以为实者,无盈数。两盈以少减多,余为法。齐之八十者,是十假令;而 凡盈三十者,是十,以三之;齐之二十七者,是三假令;而凡盈三十者,是三, 以十之。今假令两盈共十、三,以三减十,余七,为一实。故令以三减十,余七 为法。所出率以少减多,余谓之设差。因设差为少设,则两盈之差是为定实。故 以少设约法得人数,约实即得金数。〕 其一术曰:置所出率,以少减多,余为法。两盈、两不足以少减多,余为实。

实如法而一,得人数。以所出率乘之,减盈、增不足,即物价。

〔“置所出率,以少减多”,得一人之差。两盈、两不足相减,为众人之差。

故以一人之差除之,得人数。以所出率乘之,减盈、增不足,即物价。〕 今有共买犬,人出五,不足九十;人出五十,适足。问人数、犬价各几何? 答曰:二人。犬价一百。

今有共买豕,人出一百,盈一百;人出九十,适足。问人数、豕价各几何? 答曰:一十人。豕价九百。

术曰:以盈及不足之数为实。置所出率,以少减多,余为法。实如法得一人。

其求物价者,以适足乘人数,得物价。

〔此术意谓以所出率,以少减多者,余是一人不足之差。不足数为众人之差。

以一人差约之,故得人之数也。以盈及不足数为实者,数单见,即众人差,故以 为实。所出率以少减多,即一人差,故以为法。以除众人差,得人数。以适足乘 人数,即得物价也。〕 今有米在十斗桶中,不知其数。满中添粟而舂之,得米七斗。问故米几何? 答曰:二斗五升。

术曰:以盈不足术求之。假令故米二斗,不足二升;令之三斗,有余二升。

〔按:桶受一斛,若使故米二斗,须添粟八斗以满之。八斗得粝米四斗八升, 课于七斗,是为不足二升。若使故米三斗,须添粟七斗以满之。七斗得粝米四斗 二升,课于七斗,是为有余二升。以盈不足维乘假令之数者,欲为齐同之意。为 齐同者,齐其假令,同其盈朒。通计齐即不盈不朒之正数,故可以并之为实, 并盈、不足为法。实如法,即得故米斗数,乃不盈不朒之正数也。〕 今有垣高九尺。瓜生其上,蔓日长七寸;瓠生其下,蔓日长一尺。问几何日 相逢?瓜、瓠各长几何?答曰:五日十七分日之五。瓜长三尺七寸一十七分寸之 一。瓠长五尺二寸一十七分寸之一十六。

术曰:假令五日,不足五寸;令之六日,有余一尺二寸。

〔按:“假令五日,不足五寸”者,瓜生五日,下垂蔓三尺五寸;瓠生五日, 上延蔓五尺;课于九尺之垣,是为不足五寸。“令之六日,有余一尺二寸”者, 若使瓜生六日,下垂蔓四尺二寸;瓠生六日,上延蔓六尺;课于九尺之垣,是为 有余一尺二寸。以盈、不足维乘假令之数者,欲为齐同之意。齐其假令,同其盈 朒。通计齐即不盈不朒之正数,故可并以为实,并盈、不足为法。实如法而 一,即设差不盈不朒之正数,即得日数。以瓜、瓠一日之长乘之,故各得其长 之数也。〕 今有蒲生一日,长三尺;莞生一日,长一尺。蒲生日自半,莞生日自倍。问 几何日而长等?答曰:二日十三分日之六。各长四尺八寸一十三分寸之六。

术曰:假令二日,不足一尺五寸;令之三日,有余一尺七寸半。

〔按:“假令二日,不足一尺五寸”者,蒲生二日,长四尺五寸;莞生二日, 长三尺;是为未相及一尺五寸,故曰不足。“令之三日,有余一尺七寸半”者, 蒲增前七寸半,莞增前四尺,是为过一尺七寸半,故曰有余。以盈不足乘除之。

又以后一日所长各乘日分子,如日分母而一者,各得日分子之长也。故各增二日 定长,即得其数。〕 今有醇酒一斗,直钱五十;行酒一斗,直钱一十。今将钱三十,得酒二斗。

问醇、行酒各得几何?答曰:醇酒二升半。行洒一斗七升半。

术曰:假令醇酒五升,行酒一斗五升,有余一十;令之醇酒二升,行酒一斗 八升,不足二。

〔据醇酒五升,直钱二十五;行酒一斗五升,直钱一十五;课于三十,是为 有余十。据醇酒二升,直钱一十;行酒一斗八升,直钱一十八;课于三十,是为 不足二。以盈不足术求之。此问已有重设及其齐同之意也。〕 今有大器五,小器一,容三斛;大器一,小器五,容二斛。问大、小器各容 几何?答曰:大器容二十四分斛之十三。小器容二十四分斛之七。

术曰:假令大器五斗,小器亦五斗,盈一十斗;令之大器五斗五升,小器二 斗五升,不足二斗。

〔按:大器容五斗,大器五容二斛五斗。以减三斛,余五斗,即小器一所容。

故曰“小器亦五斗”。小器五容二斛五斗,大器一,合为三斛。课于两斛,乃多 十斗。令之大器五斗五升,大器五合容二斛七斗五升。以减三斛,余二斗五升, 即小器一所容。故曰小器二斗五升”。大器一容五斗五升,小器五合容一斛二斗 五升,合为一斛八斗。课于二斛,少二斗。故曰“不足二斗”。以盈不足维乘, 除之。〕 今有漆三得油四,油四和漆五。今有漆三斗,欲令分以易油,还自和余漆。

问出漆、得油、和漆各几何?答曰:出漆一斗一升四分升之一。得油一斗五升。

和漆一斗八升四分升之三。

术曰:假令出漆九升,不足六升;令之出漆一斗二升,有余二升。

〔按:此术三斗之漆,出九升,得油一斗二升,可和漆一斗五升,余有二斗 一升,则六升无油可和,故曰“不足六升”。令之出漆一斗二升,则易得油一斗 六升,可和漆二斗。于三斗之中已出一斗二升,余有一斗八升。见在油合和得漆 二斗,则是有余二升。以盈、不足维乘之,为实。并盈、不足为法。实如法而一, 得出漆升数。求油及和漆者,四、五各为所求率,三、四各为所有率,而今有之, 即得也。〕 今有玉方一寸,重七两;石方一寸,重六两。今有石立方三寸,中有玉,并 重十一斤。问玉、石重各几何?答曰:玉一十四寸,重六斤二两。石一十三寸, 重四斤一十四两。

术曰:假令皆玉,多十三两;令之皆石,不足一十四两。不足为玉,多为石。

各以一寸之重乘之,得玉、石之积重。

〔立方三寸是一面之方,计积二十七寸。玉方一寸重七两,石方一寸重六两, 是为玉、石重差一两。假令皆玉,合有一百八十九两。课于一十一斤,有余一十 三两。玉重而石轻,故有此多。即二十七寸之中有十三寸,寸损一两,则以为石 重,故言多为石。言多之数出于石以为玉。假令皆石,合有一百六十二两。课于 十一斤,少十四两,故曰不足。此不足即以重为轻。故令减少数于并重,即二十 七寸之中有十四寸,寸增一两也。〕 今有善田一亩,价三百;恶田七亩,价五百。今并买一顷,价钱一万。问善、 恶田各几何?答曰:善田一十二亩半。恶田八十七亩半。

术曰:假令善田二十亩,恶田八十亩,多一千七百一十四钱七分钱之二;令 之善田一十亩,恶田九十亩,不足五百七十一钱七分钱之三。

〔按:善田二十亩,直钱六千;恶田八十亩,直钱五千七百一十四、七分钱 之二,课于一万,是多一千七百一十四、七分钱之二。令之善田十亩,直钱三千; 恶田九十亩,直钱六千四百二十八、七分钱之四;课于一万,是为不足五百七十 一、七分钱之三。以盈不足术求之也。〕 今有黄金九枚,白银一十一枚,称之重,适等。交 易其一,金轻十三两。问 金、银一枚各重几何?答曰:金重二斤三两一十八铢。银重一斤一十三两六铢。

术曰:假令黄金三斤,白银二斤一十一分斤之五,不足四十九,于右行。令 之黄金二斤,白银一斤一十一分斤之七,多一十五,于左行。以分母各乘其行内 之数。以盈、不足维乘所出率,并,以为实。并盈、不足为法。实如法,得黄金 重。分母乘法以除,得银重。约之得分也。

〔按:此术假令黄金九,白银一十一,俱重二十七斤。金,九约之,得三斤; 银,一十一约之,得二斤一十一分斤之五;各为金、银一枚重数。就金重二十七 斤之中减一金之重,以益银,银重二十七斤之中减一银之重,以益金,则金重二 十六斤一十一分斤之五,银重二十七斤一十一分斤之六。以少减多,则金轻一十 七两一十一分两之五。课于一十三两,多四两一十一分两之五。通分内子言之, 是为不足四十九。又令之黄金九,一枚重二斤,九枚重一十八斤;白银一十一, 亦合重一十八斤也。乃以一十一除之,得一斤一十一分斤之七,为银一枚之重数。

今就金重一十八斤之中减一枚金,以益银;复减一枚银,以益金,则金重一十七 斤一十一分斤之七,银重一十八斤一十一分斤之四。以少减多,即金轻一十一分 斤之八。课于一十三两,少一两一十一分两之四。通分内子言之,是为多一十五。

以盈不足为之,如法,得金重。分母乘法以除者,为银两分母,故同之。须通法 而后乃除,得银重。余皆约之者,术省故也。〕 今有良马与驽马发长安,至齐。齐去长安三千里。良马初日行一百九十三里, 日增一十三里,驽马初日行九十七里,日减半里。良马先至齐,复还迎驽马。问 几何日相逢及各行几何?答曰:一十五日一百九十一分日之一百三十五而相逢。

良马行四千五百三十四里一百九十一分里之四十六。驽马行一千四百六十五里一 百九十一分里之一百四十五。

术曰:假令十五日,不足三百三十七里半;令之十六日,多一百四十里。以 盈、不足维乘假令之数,并而为实。并盈、不足为法。实如法而一,得日数。不 尽者,以等数除之而命分。求良马行者:十四乘益疾里数而半之,加良马初日之 行里数,以乘十五日,得十五日之凡行。又以十五日乘益疾里数,加良马初日之 行。以乘日分子,如日分母而一。所得,加前良马凡行里数,即得。其不尽而命 分。求驽马行者:以十四乘半里,又半之,以减驽马初日之行里数,以乘十五日, 得驽马十五日之凡行。又以十五日乘半里,以减驽马初日之行,余,以乘日分子, 如日分母而一。所得,加前里,即驽马定行里数。其奇半里者,为半法。以半法 增残分,即得。其不尽者而命分。

〔按:“令十五日,不足三百三十七里半”者,据良马十五日凡行四千二百 六十里,除先去齐三千里,定还迎驽马一千二百六十里;驽马十五日凡行一千四 百二里半,并良、驽二马所行,得二千六百六十二里半。课于三千里,少三百三 十七里半。故曰不足。“令之十六日,多一百四十里”者,据良马十六日凡行四 千六百四十八里;除先去齐三千里,定还迎驽马一千六百四十八里,驽马十六日 凡行一千四百九十二里。并良、驽二马所行,得三千一百四十里。课于三千里, 余有一百四十里。故谓之多也。以盈不足之,实如法而一,得日数者,即设差不 盈不朒之正数。以二马初日所行里乘十五日,为一十五日平行数。求初末益疾 减迟之数者,并一与十四,以十四乘而半之,为中平之积。又令益疾减迟里数乘 之,各为减益之中平里。故各减益平行数,得一十五日定行里。若求后一日,以 十六日之定行里数乘日分子,如日分母而一,各得日分子之定行里数。故各并十 五日定行里,即得。其驽马奇半里者,法为全里之分,故破半里为半法,以增残 分,即合所问也。〕 今有人持钱之蜀贾,利十,三。初返归一万四千,次返归一万三千,次返归 一万二千,次返归一万一千,后返归一万。凡五返归钱,本利俱尽。问本持钱及 利各几何?答曰:本三万四百六十八钱三十七万一千二百九十三分钱之八万四千 八百七十六。利二万九千五百三十一钱三十七万一千二百九十三分钱之二十八万 六千四百一十七。

术曰:假令本钱三万,不足一千七百三十八钱半;令之四万,多三万五千三 百九十钱八分。

〔按:假令本钱三万,并利为三万九千;除初返归留,余,加利为三万二千 五百;除二返归留,余,又加利为二万五千三百五十;除第三返归留,余,又加 利为一万七千三百五十五;除第四返归留,余,又加利为八千二百六十一钱半; 除第五返归留,合一万钱,不足一千七百三十八钱半。若使本钱四万,并利为五 万二千;除初返归留,余,加利为四万九千四百;除第二返归留,余,又加利为 四万七千三百二十;除第三返归留,余,又加利为四万五千九百一十六;除第四 返归留,余,又加利为四万五千三百九十钱八分;除第五返归留,合一万,余三 万五千三百九十钱八分,故曰多。

又术:置后返归一万,以十乘之,十三而一,即后所持之本。加一万一千, 又以十乘之,十三而一,即第四返之本。加一万二千,又以十乘之,十三而一, 即第三返之本。加一万三千,又以十乘之,十三而一,即第二返之本。加一万四 千,又以十乘之,十三而一,即初持之本。并五返之钱以减之,即利也。〕 今有垣厚五尺,两鼠对穿。大鼠日一尺,小鼠亦日一尺。大鼠日自倍,小鼠 日自半。问几何日相逢?各穿几何?答曰:二日一十七分日之二。大鼠穿三尺四 寸十七分寸之一十二,小鼠穿一尺五寸十七分寸之五。

术曰:假令二日,不足五寸;令之三日,有余三尺七寸半。

〔大鼠日倍,二日合穿三尺;小鼠日自半,合穿一尺五寸;并大鼠所穿,合 四尺五寸。课于垣厚五尺,是为不足五寸。令之三日,大鼠穿得七尺,小鼠穿得 一尺七寸半。并之,以减垣厚五尺,有余三尺七寸半。以盈不足术求之,即得。

以后一日所穿乘日分子,如日分母而一,即各得日分子之中所穿。故各增二日定 穿,即合所问也。〕

\chapter{方程}
(以御错糅正负) 今有上禾三秉,中禾二秉,下禾一秉,实三十九斗;上禾二秉,中禾三秉, 下禾一秉,实三十四斗;上禾一秉,中禾二秉,下禾三秉,实二十六斗。问上、 中、下禾实一秉各几何?答曰:上禾一秉九斗四分斗之一。中禾一秉四斗四分斗 之一。下禾一秉二斗四分斗之三。

方程 〔程,课程也。群物总杂,各列有数,总言其实。令每行为率。二物者再程, 三物者三程,皆如物数程之。并列为行,故谓之方程。行之左右无所同存,且为 有所据而言耳。此都术也,以空言难晓,故特系之禾以决之。又列中、左行如右 行也。〕 术曰:置上禾三秉,中禾二秉,下禾一秉,实三十九斗于右方。中、左禾列 如右方。以右行上禾遍乘中行,而以直除。

〔为术之意,令少行减多行,反复相减,则头位必先尽。上无一位,则此行 亦阙一物矣。然而举率以相减,不害余数之课也。若消去头位,则下去一物之实。

如是叠令左右行相减,审其正负,则可得而知。先令右行上禾乘中行,为齐同之 意。为齐同者,谓中行直减右行也。从简易虽不言齐同,以齐同之意观之,其义 然矣。〕 又乘其次,亦以直除。

〔复去左行首。〕 然以中行中禾不尽者遍乘左行,而以直除。

〔亦令两行相去行之中禾也。〕 左方下禾不尽者,上为法,下为实。实即下禾之实。

〔上、中禾皆去,故余数是下禾实,非但一秉。欲约众秉之实,当以禾秉数 为法。列此,以下禾之秉数乘两行,以直除,则下禾之位皆决矣。各以其余一位 之秉除其下实。即计数矣用算繁而不省。所以别为法,约也。然犹不如自用其旧。

广异法也。〕 求中禾,以法乘中行下实,而除下禾之实。

〔此谓中两禾实,下禾一秉实数先见,将中秉求中禾,其列实以减下实。而 左方下禾虽去一,以法为母,于率不通。故先以法乘,其通而同之。俱令法为母, 而除下禾实。以下禾先见之实令乘下禾秉数,即得下禾一位之列实。减于下实, 则其数是中禾之实也。〕 余,如中禾秉数而一,即中禾之实。

〔余,中禾一位之实也。故以一位秉数约之,乃得一秉之实也。〕 求上禾,亦以法乘右行下实,而除下禾、中禾之实。

〔此右行三禾共实,合三位之实。故以二位秉数约之,乃得一秉之实。今中 下禾之实其数并见,令乘右行之禾秉以减之。故亦如前各求列实,以减下实也。〕 余,如上禾秉数而一,即上禾之实。实皆如法,各得一斗。

〔三实同用,不满法者,以法命之。母、实皆当约之。〕 今有上禾七秉,损实一斗,益之下禾二秉,而实一十斗;下禾八秉,益实一 斗,与上禾二秉,而实一十斗。问上、下禾实一秉各几何?答曰:上禾一秉实一 斗五十二分斗之一十八。下禾一秉实五十二分斗之四十一。

术曰:如方程。损之曰益,益之曰损。

〔问者之辞虽?今按:实云上禾七秉,下禾二秉,实一十一斗;上禾二秉, 下禾八秉,实九斗也。“损之曰益”,言损一斗,余当一十斗;今欲全其实,当 加所损也。“益之曰损”,言益实以一斗,乃满一十斗;今欲知本实,当减所加, 即得也。〕 损实一斗者,其实过一十斗也;益实一斗者,其实不满一十斗也。

〔重谕损益数者,各以损益之数损益之也。〕 今有上禾二秉,中禾三秉,下禾四秉,实皆不满斗。上取中、中取下、下取 上各一秉而实满斗。问上、中、下禾实一秉各几何?答曰上禾一秉实二十五分斗 之九。中禾一秉实二十五分斗之七。下禾一秉实二十五分斗之四。

术曰:如方程。各置所取。

〔置上禾二秉为右行之上,中禾三秉为中行之中,下禾四秉为左行之下,所 取一秉及实一斗各从其位。诸行相借取之物皆依此例。〕 以正负术入之。

正负术曰: 〔今两算得失相反,要令正负以名之。正算赤,负算黑,否则以邪正为异。

方程自有赤、黑相取,法、实数相推求之术。而其并减之势不得广通,故使赤、 黑相消夺之,于算或减或益。同行异位殊为二品,各有并、减之差见于下焉。著 此二条,特系之禾以成此二条之意。故赤、黑相杂足以定上下之程,减、益虽殊 足以通左右之数,差、实虽分足以应同异之率。然则其正无入以负之,负无入以 正之,其率不妄也。〕 同名相除, 〔此谓以赤除赤,以黑除黑,行求相减者,为去头位也。然则头位同名者, 当用此条,头位异名者,当用下条。〕 异名相益, 〔益行减行,当各以其类矣。其异名者,非其类也。非其类者,犹无对也, 非所得减也。故赤用黑对则除,黑;无对则除,黑;黑用赤对则除,赤;无对则 除,赤;赤黑并于本数。此为相益之,皆所以为消夺。消夺之与减益成一实也。

术本取要,必除行首。至于他位,不嫌多少,故或令相减,或令相并,理无同异 而一也。〕 正无入负之,负无入正之。

〔无入,为无对也。无所得减,则使消夺者居位也。其当以列实或减下实, 而行中正负杂者亦用此条。此条者,同名减实,异名益实,正无入负之,负无入 正之也。〕 其异名相除,同名相益,正无入正之,负无入负之。

〔此条异名相除为例,故亦与上条互取。凡正负所以记其同异,使二品互相 取而已矣。言负者未必负于少,言正者未必正于多。故每一行之中虽复赤黑异算 无伤。然则可得使头位常相与异名。此条之实兼通矣,遂以二条反覆一率。观其 每与上下互相取位,则随算而言耳,犹一术也。又,本设诸行,欲因成数以相去 耳。故其多少无限,令上下相命而已。若以正负相减,如数有旧增法者,每行可 均之,不但数物左右之也。〕 今有上禾五秉,损实一斗一升,当下禾七秉;上禾七秉,损实二斗五升,当 下禾五秉。问上、下禾实一秉各几何?答曰:上禾一秉五升。下禾一秉二升。

术曰:如方程。置上禾五秉正,下禾七秉负,损实一斗一升正。

〔言上禾五秉之实多,减其一斗一升,余,是与下禾七秉相当数也。故互其 算,令相折除,以一斗一升为差。为差者,上禾之余实也。〕 次置上禾七秉正,下禾五秉负,损实二斗五升正。以正负术入之。

〔按:正负之术,本设列行,物程之数不限多少,必令与实上下相次,而以 每行各自为率。然而或减或益,同行异位,殊为二品,各自并、减,之差见于下 也。〕 今有上禾六秉,损实一斗八升,当下禾一十秉;下禾一十五秉,损实五升, 当上禾五秉。问上、下禾实一秉各几何?答曰:上禾一秉实八升。下禾一秉实三 升。

术曰:如方程。置上禾六秉正,下禾一十秉负,损实一斗八升正。次,上禾 五秉负,下禾一十五秉正,损实五升正。以正负术入之。

〔言上禾六秉之实多,减损其一斗八升,余是与下禾十秉相当之数。故亦互 其算,而以一斗八升为差实。差实者,上禾之余实。〕 今有上禾三秉,益实六斗,当下禾一十秉;下禾五秉,益实一斗,当上禾二 秉。问上、下禾实一秉各几何?答曰:上禾一秉实八斗。下禾一秉实三斗。

术曰:如方程。置上禾三秉正,下禾一十秉负,益实六斗负。次置上禾二秉 负,下禾五秉正,益实一斗负。以正负术入之。

〔言上禾三秉之实少,益其六斗,然后于下禾十秉相当也。故亦互其算,而 以六斗为差实。差实者,下禾之余实。〕 今有牛五,羊二,直金十两;牛二,羊五,直金八两。问牛、羊各直金几何? 答曰:牛一直金一两二十一分两之一十三。羊一直金二十一分两之二十。

术曰:如方程。

〔假令为同齐,头位为牛,当相乘。右行定,更置牛十,羊四,直金二十两; 左行:牛十,羊二十五,直金四十两。牛数等同,金多二十两者,羊差二十一使 之然也。以少行减多行,则牛数尽,惟羊与直金之数见,可得而知也。以小推大, 虽四五行不异也。〕 今有卖牛二,羊五,以买一十三豕,有余钱一千;卖牛三,豕三,以买九羊, 钱适足;卖六羊,八豕,以买五牛,钱不足六百。问牛、羊、豕价各几何?答曰 牛价一千二百。羊价五百。豕价三百。

术曰:如方程。置牛二,羊五正,豕一十三负,余钱数正;次,牛三正,羊 九负,豕三正;次五牛负,六羊正,八豕正,不足钱负。以正负术入之。

〔此中行买、卖相折,钱适足,故但互买卖算而已。故下无钱直也。设欲以 此行如方程法,先令二牛遍乘中行,而以右行直除之。是故终于下实虚缺矣。故 注曰正无实负,负无实正,方为类也。方将以别实加适足之数与实物作实。

盈不足章“黄金白银”与此相当。“假令黄金九,白银一十一,称之重适等。

交 易其一,金轻十三两。问金、银一枚各重几何?”与此同。〕 今有五雀六燕,集称之衡,雀俱重,燕俱轻。一雀一燕交 而处,衡适平。并 雀、燕重一斤。问雀、燕一枚各重几何?答曰:雀重一两一十九分两之一十三。

燕重一两一十九分两之五。

术曰:如方程。交 易质之,各重八两。

〔此四雀一燕与一雀五燕衡适平,并重一斤,故各八两。列两行程数。左行 头位其数有一者,令右行遍除。亦可令于左行而取其法、实于左。左行数多,以 右行取其数。左头位减尽,中、下位算当燕与实。右行不动。左上空,中法,下 实,即每枚当重宜可知也。按:此四雀一燕与一雀五燕其重等,是三雀、四燕重 相当。雀率重四,燕率重三也。诸再程之率皆可异术求也,即其数也。〕 今有甲、乙二人持钱不知其数。甲得乙半而钱五十,乙得甲太半而亦钱五十。

问甲、乙持钱各几何?答曰:甲持三十七钱半。乙持二十五钱。

术曰:如方程。损益之。

〔此问者言一甲,半乙而五十;太半甲,一乙亦五十也。各以分母乘其全, 内子。行定:二甲,一乙而钱一百;二甲,三乙而钱一百五十。于是乃如方程。

诸物有分者放此。〕 今有二马,一牛,价过一万,如半马之价;一马,二牛,价不满一万,如半 牛之价。问牛、马价各几何?答曰:马价五千四百五十四钱一十一分钱之六。牛 价一千八百一十八钱一十一分钱之二。

术曰:如方程。损益之。

〔此一马半与一牛价直一万也,二牛半与一马亦直一万也。一马半与一牛直 钱一万,通分内子,右行为三马,二牛,直钱二万。二牛半与一马直钱一万,通 分内子,左行为二马,五牛,直钱二万也。〕 今有武马一匹,中马二匹,下马三匹,皆载四十石至阪,皆不能上。武马借 中马一匹,中马借下马一匹,下马借武马一匹,乃皆上。问武、中、下马一匹各 力引几何?答曰:武马一匹力引二十二石七分石之六。中马一匹力引一十七石七 分石之一。下马一匹力引五石七分石之五。

术曰:如方程。各置所借,以正负术入之。

今有五家共井,甲二绠不足,如乙一绠。乙三绠不足,以丙一绠;丙四绠不 足,以丁一绠;丁五绠不足,以戊一绠;戊六绠不足,以甲一绠。如各得所不足 一绠,皆逮。问井深、绠长各几何?答曰:井深七丈二尺一寸。甲绠长二丈六尺 五寸。乙绠长一丈九尺一寸。丙绠长一丈四尺八寸。丁绠长一丈二尺九寸。戊绠 长七尺六寸。

术曰:如方程。以正负术入之。

〔此率初如方程为之,名各一逮井。其后,法得七百二十一,实七十六,是 为七百二十一绠而七十六逮井,并用逮之数。以法除实者,而戊一绠逮井之数定, 逮七百二十一分之七十六。是故七百二十一为井深,七十六为戊绠之长,举率以 言之。〕 今有白禾二步,青禾三步,黄禾四步,黑禾五步,实各不满斗。白取青、黄, 青取黄、黑,黄取黑、白,黑取白、青,各一步,而实满斗。问白、青、黄、黑 禾实一步各几何?答曰:白禾一步实一百一十一分斗之三十三。青禾一步实一百 一十一分斗之二十八。黄禾一步实一百一十一分斗之一十七。黑禾一步实一百一 十一分斗之一十。

术曰:如方程。各置所取,以正负术入之。

今有甲禾二秉,乙禾三秉,丙禾四秉,重皆过于石。甲二重如乙一,乙三重 如丙一,丙四重如甲一。问甲、乙、丙禾一秉各重几何?答曰:甲禾一秉重二十 三分石之一十七。乙禾一秉重二十三分石之一十一。丙禾一秉重二十三分石之一 十。

术曰:如方程。置重过于石之物为负。

〔此问者言甲禾二秉之重过于一石也。其过者何云?如乙一秉重矣。互其算, 令相折除,而一以石为之差实。差实者,如甲禾余实。故置算相与同也。〕 以正负术入之。

〔此入,头位异名相除者,正无入正之,负无入负之也。〕 今有令一人,吏五人,从者一十人,食鸡一十;令一十人,吏一人,从者五 人,食鸡八;令五人,吏一十人,从者一人,食鸡六。问令、吏、从者食鸡各几 何?答曰令一人食一百二十二分鸡之四十五。吏一人食一百二十二分鸡之四十一。

从者一人食一百二十二分鸡之九十七。

术曰:如方程。以正负术入之。

今有五羊,四犬,三鸡,二兔,直钱一千四百九十六;四羊,二犬,六鸡, 三兔,直钱一千一百七十五;三羊,一犬,七鸡,五兔,直钱九百五十八;二羊, 三犬,五鸡,一兔,直钱八百六十一。问羊、犬、鸡、兔价各几何?答曰:羊价 一百七十七。犬价一百二十一。鸡价二十三。兔价二十九。

术曰:如方程。以正负术入之。

今有麻九斗,麦七斗,菽三斗,荅二斗,黍五斗,直钱一百四十;麻七斗, 麦六斗,菽四斗,荅五斗,黍三斗,直钱一百二十八;麻三斗,麦五斗,菽七斗, 荅六斗,黍四斗,直钱一百一十六;麻二斗,麦五斗,菽三斗,荅九斗,黍四斗, 直钱一百一十二;麻一斗,麦三斗,菽二斗,荅八斗,黍五斗,直钱九十五。问 一斗直几何?荅曰:麻一斗七钱。麦一斗四钱。菽一斗三钱。荅一斗五钱。黍一 斗六钱。

术曰:如方程。以正负术入之。

〔此麻麦与均输、少广之章重衰、积分皆为大事。其拙于精理徒按本术者, 或用算而布毡,方好烦而喜误,曾不知其非,反欲以多为贵。故其算也,莫不暗 于设通而专于一端。至于此类,苟务其成,然或失之,不可谓要约。更有异术者, 庖丁解牛,游刃理间,故能历久其刃如新。夫数,犹刃也,易简用之则动中庖丁 之理。故能和神爱刃,速而寡尤。凡九章为大事,按法皆不尽一百算也。虽布算 不多,然足以算多。世人多以方程为难,或尽布算之象在缀正负而已,未暇以论 其设动无方,斯胶柱调瑟之类。聊复恢演,为作新术,著之于此,将亦启导疑意。

网罗道精,岂传之空言?记其施用之例,著策之数,每举一隅焉。

方程新术曰:以正负术入之。令左、右相减,先去下实,又转去物位,则其 求一行二物正负相借者,是其相当之率。又令二物与他行互相去取,转其二物相 借之数,即皆相当之率也。各据二物相当之率,对易其数,即各当之率也。更置 成行及其下实,各以其物本率今有之,求其所同。并,以为法。其当相并而行中 正负杂者,同名相从,异名相消,余,以为法。以下置为实。实如法,即合所问 也。一物各以本率今有之,即皆合所问也。率不通者,齐之。

其一术曰:置群物通率为列衰。更置成行群物之数,各以其率乘之,并,以 为法。其当相并而行中正负杂者,同名相从,异名相消,余为法。以成行下实乘 列衰,各自为实。实如法而一,即得。

以旧术为之。凡应置五行。今欲要约,先置第三行,减以第四行,又减第五 行;次置第二行,以第二行减第一行,又减第四行。去其头位;余,可半;次置 右行及第二行。去其头位;次以右行去第四行头位,次以左行去第二行头位,次 以第五行去第一行头位;次以第二行去第四行头位;余,可半;以右行去第二行 头位,以第二行去第四行头位。余,约之为法、实。实如法而一,得六,即有黍 价。以法治第二行,得荅价,右行得菽价,左行得麦价,第三行麻价。如此凡用 七十七算。

以新术为此。先以第四行减第三行;次以第三行去右行及第二行、第四行下 位,又以减左行下位,不足减乃止;次以左行减第三行下位,次以第三行去左行 下位。讫,废去第三行。次以第四行去左行下位,又以减右行下位;次以右行去 第二行及第四行下位;次以第二行减第四行及左行头位;次以第四行减左行菽位, 不足减乃止;次以左行减第二行头位,余,可再半;次以第四行去左行及第二行 头位,次以第二行去左行头位,余,约之,上得五,下得三,是菽五当荅;次以 左行去第二行菽位,又以减第四行及右行菽位,不足减乃止;次以右行减第二行 头位,不足减乃止;次以第二行去右行头位,次以左行去右行头位;余,上得六, 下得五,是为荅六当黍五;次以左行去右行荅位,余,约之,上为二,下为一; 次以右行去第二行下位,以第二行去第四行下位,又以减左行下位;次,左行去 第二行下位,余,上得三,下得四,是为麦三当菽四;次以第二行减第四行下位; 次以第四行去第二行下位;余,上得四,下得七,是为麻四当麦七。是为相当之 率举矣。据麻四当麦七,即麻价率七而麦价率四;又麦三当菽四,即为麦价率四 而菽价率三;又菽五当荅三,即为菽价率三而荅价率五;又荅六当黍五,即为荅 价率五而黍价率六;而率通矣。更置第三行,以第四行减之,余有麻一斗,菽四 斗正,荅三斗负,下实四正。求其同为麻之数,以菽率三、荅率五各乘其斗数, 如麻率七而一,菽得一斗七分斗之五正,荅得二斗七分斗之一负。则菽、荅化为 麻。以并之,令同名相从,异名相消,余得定麻七分斗之四,以为法。置四为实, 而分母乘之,实得二十八,而分子化为法矣以法除得七,即麻一斗之价。置麦率 四、菽率三、荅率五、黍率六,皆以麻乘之,各自为实。以麻率七为法。所得即 各为价。亦可使置本行实与物同通之,各以本率今有之,求其本率所得。并, 以为法。如此,即无正负之异矣,择异同而已。又可以一术为之。置五行通率, 为麻七、麦四、菽三、荅五、黍六,以为列衰。成行麻一斗,菽四斗正,荅三斗 负,各以其率乘之。讫,令同名相从,异名相消,余为法。又置下实乘列衰,所 得各为实。此可以置约法,则不复乘列衰,各以列衰为价。如此则凡用一百二十 四算也。〕 

\chapter{勾股}
(以御高深广远) 今有句三尺,股四尺,问为弦几何?答曰:五尺。

今有弦五尺,句三尺,问为股几何?答曰:四尺。

今有股四尺,弦五尺,问为句几何?答曰:三尺。

句股 〔短面曰句,长面曰股,相与结角曰弦。句短其股,股短其弦。将以施于诸 率,故先具此术以见其源也。〕 术曰:句、股各自乘,并,而开方除之,即弦。

〔句自乘为朱方,股自乘为青方。令出入相补,各从其类,因就其余不移动 也,合成弦方之幂。开方除之,即弦也。〕 又,股自乘,以减弦自乘。其余,开方除之,即句。

〔淳风等按:此术以句、股幂合成弦幂。句方于内,则句短于股。令股自乘, 以减弦自乘,余者即句幂也。故开方除之,即句也。〕 又,句自乘,以减弦自乘。其余,开方除之,即股。

〔句、股幂合以成弦幂,令去其一,则余在者皆可得而知之。〕 今有圆材,径二尺五寸。欲为方版,令厚七寸,问广几何?答曰:二尺四寸。

术曰:令径二尺五寸自乘,以七寸自乘,减之。其余,开方除之,即广。

〔此以圆径二尺五寸为弦,版厚七寸为句,所求广为股也。〕 今有木长二丈,围之三尺。葛生其下,缠木七周,上与木齐。问葛长几何? 答曰:二丈九尺。

术曰:以七周乘围为股,木长为句,为之求弦。弦者,葛之长。

〔据围广,求从为木长者其形葛卷裹袤。以笔管,青线宛转,有似葛之缠木。

解而观之,则每周之间自有相间成句股弦。则其间葛长,弦。七周乘围,并合众 句以为一句;木长而股,短;术云木长谓之股,言之倒。句与股求弦,亦无围。

弦之自乘幂出上第一图。句、股幂合为弦幂,明矣。然二幂之数谓倒在于弦幂之 中而已。可更相表里,居里者则成方幂,其居表者则成矩幂。二表里形讹而数均。

又按:此图句幂之矩青,卷白表,是其幂以股弦差为广,股弦并为袤,而股幂方 其里。股幂之矩青,卷白表,是其幂以句弦差为广,句弦并为袤,而句幂方其里。

是故差之与并用除之,短、长互相乘也。〕 今有池方一丈,葭生其中央,出水一尺。引葭赴岸,适与岸齐。问水深、葭 长各几何?答曰:水深一丈二尺。葭长一丈三尺。

术曰:半池方自乘, 〔此以池方半之,得五尺为句;水深为股;葭长为弦。以句、弦见股,故令 句自乘,先见矩幂也。〕 以出水一尺自乘,减之。

〔出水者,股弦差。减此差幂于矩幂则除之。〕 余,倍出水除之,即得水深。

〔差为矩幂之广,水深是股。令此幂得出水一尺为长,故为矩而得葭长也。〕 加出水数,得葭长。

〔淳风等按:此葭本出水一尺,既见水深,故加出水尺数而得葭长也。〕 今有立木,系索其末,委地三尺。引索却行,去本八尺而索尽。问索长几何? 答曰:一丈二尺六分尺之一。

术曰:以去本自乘, 〔此以去本八尺为句,所求索者,弦也。引而索尽、开门去阃者,句及股弦 差,同一术。去本自乘者,先张矩幂。〕 令如委数而一。

〔委地者,股弦差也。以除矩幂,即是股弦并也。〕 所得,加委地数而半之,即索长。

〔子不可半者,倍其母。加差者并,则两长。故又半之。其减差者并,而半 之,得木长也。〕 今有垣高一丈,倚木于垣,上与垣齐。引木却行一尺,其木至地。问木长几 何?答曰:五丈五寸。

术曰:以垣高一十尺自乘,如却行尺数而一。所得,以加却行尺数而半之, 即木长数。

〔此以垣高一丈为句,所求倚木者为弦,引却行一尺为股弦差。为术之意与 系索问同也。〕 今有圆材埋在壁中,不知大小。以锯锯之,深一寸,锯道长一尺。问径几何? 答曰:材径二尺六寸。

术曰:半锯道自乘, 〔此术以锯道一尺为句,材径为弦,锯深一寸为股弦差之一半。锯道长是半 也。

淳风等按:下锯深得一寸为半股弦差。注云为股差差者,锯道也。〕 如深寸而一,以深寸增之,即材径。

〔亦以半增之。如上术,本当半之,今此皆同半,故不复半也。〕 今有开门去阃一尺,不合二寸。问门广几何?答曰:一丈一寸。

术曰:以去阃一尺自乘。所得,以不合二寸半之而一。所得,增不合之半, 即得门广。

〔此去阃一尺为句,半门广为弦,不合二寸以半之,得一寸为股弦差。求弦, 故当半之。今次以两弦为广数,故不复半之也。〕 今有户高多于广六尺八寸,两隅相去适一丈。问户高、广各几何?答曰:广 二尺八寸。高九尺六寸。

术曰:令一丈自乘为实。半相多,令自乘,倍之,减实。半其余,以开方除 之。所得,减相多之半,即户广;加相多之半,即户高。

〔令户广为句,高为股,两隅相去一丈为弦,高多于广六尺八寸为句股差。

按图为位,弦幂适满万寸。倍之,减句股差幂,开方除之。其所得即高广并数。

以差减并而半之,即户广。加相多之数,即户高也。今此术先求其半。一丈自乘 为朱幂四、黄幂一。半差自乘,又倍之,为黄幂四分之二,减实,半其余,有朱 幂二、黄幂四分之一。其于大方者四分之一。故开方除之,得高广并数半。减差 半,得广;加,得户高。又按:此图幂:句股相并幂而加其差幂,亦减弦幂,为 积。盖先见其弦,然后知其句与股。今适等,自乘,亦各为方,合为弦幂。令半 相多而自乘,倍之,又半并自乘,倍之,亦合为弦幂。而差数无者,此各自乘之, 而与相乘数,各为门实。及股长句短,同源而分流焉。假令句、股各五,弦幂五 十,开方除之,得七尺,有余一,不尽。假令弦十,其幂有百,半之为句、股二 幂,各得五十,当亦不可开。故曰:圆三、径一,方五、斜七,虽不正得尽理, 亦可言相近耳。其句股合而自相乘之幂者,令弦自乘,倍之,为两弦幂,以减之, 其余,开方除之,为句股差。加于合而半,为股;减差于合而半之,为句。句、 股、弦即高、广、邪。其出此图也,其倍弦为袤。令矩句即为幂,得广即句股差。

其矩句之幂,倍句为从法,开之亦句股差。以句股差幂减弦幂,半其余,差为从 法,开方除之,即句也。〕 今有竹高一丈,末折抵地,去本三尺。问折者高几何?答曰:四尺二十分尺 之一十一。

术曰:以去本自乘, 〔此去本三尺为句,折之余高为股,以先令句自乘之幂。〕 令如高而一。

〔凡为高一丈为股弦并,以除此幂得差。〕 所得,以减竹高而半余,即折者之高也。

〔此术与系索之类更相反覆也。亦可如上术,令高自乘为股弦并幂,去本自 乘为矩幂,减之,余为实。倍高为法,则得折之高数也。〕 今有二人同所立,甲行率七,乙行率三。乙东行,甲南行十步而斜东北与乙 会。问甲、乙行各几何?答曰:乙东行一十步半,甲斜行一十四步半及之。

术曰:令七自乘,三亦自乘,并而半之,以为甲斜行率。斜行率减于七自乘, 余为南行率。以三乘七为乙东行率。

〔此以南行为句,东行为股,斜行为弦,并句弦率七。欲引者,当以股率自 乘为幂,如并而一,所得为句弦差率。加并之半为弦率,以差率减,余为句率。

如是或有分,当通而约之乃定。术以同使无分母,故令句弦并自乘为朱、黄相连 之方。股自乘为青幂之矩,以句弦并为袤,差为广。今有相引之直,加损同上。

其图大体以两弦为袤,句弦并为广。引黄断其半为弦率。列用率七自乘者,句弦 并之率。故弦减之,余为句率。同立处是中停也,皆句弦并为率,故亦以句率同 其袤也。〕 置南行十步,以甲斜行率乘之;副置十步,以乙东行率乘之;各自为实。实 如南行率而一,各得行数。

〔南行十步者,所有见句求见弦、股,故以弦、股率乘,如句率而一。〕 今有句五步,股十二步。问句中容方几何?答曰:方三步十七分步之九。

术曰:并句、股为法,句、股相乘为实。实如法而一,得方一步。

〔句、股相乘为朱、青、黄幂各二。令黄幂袤于隅中,朱、青各以其类,令 从其两径,共成修之幂:中方黄为广,并句、股为袤。故并句、股为法。幂图: 方在句中,则方之两廉各自成小句股,而其相与之势不失本率也。句面之小句、 股,股面之小句、股各并为中率,令股为中率,并句、股为率,据见句五步而今 有之,得中方也。复令句为中率,以并句、股为率,据见股十二步而今有之,则 中方又可知。此则虽不效而法,实有法由生矣。下容圆率而似今有、衰分言之, 可以见之也。〕 今有句八步,股一十五步。问句中容圆径几何?答曰:六步。

术曰:八步为句,十五步为股,为之求弦。三位并之为法。以句乘股,倍之 为实。实如法,得径一步。

〔句、股相乘为图本体,朱、青、黄幂各二。倍之,则为各四。可用画于小 纸,分裁邪正之会,令颠倒相补,各以类合,成修幂:圆径为广,并句、股、弦 为袤。故并句、股、弦以为法。又以圆大体言之,股中青必令立规于横广,句、 股又邪三径均。而复连规,从横量度句、股,必合而成小方矣。又画中弦以规 除会,则句、股之面中央小句股弦:句之小股、股之小句皆小方之面,皆圆径之 半。其数故可衰。以句、股、弦为列衰,副并为法。以句乘未并者,各自为实。

实如法而一,得句面之小股可知也。以股乘列衰为实,则得股面之小句可知。言 虽异矣,及其所以成法之实,则同归矣。则圆径又可以表之差并:句弦差减股 为圆径;又,弦减句股并,余为圆径;以句弦差乘股弦差而倍之,开方除之,亦 圆径也。〕 今有邑方二百步,各中开门。出东门一十五步有木。问出南门几何步而见木? 答曰:六百六十六步大半步。

术曰:出东门步数为法, 〔以句率为法也。〕 半邑方自乘为实,实如法得一步。

〔此以出东门十五步为句率,东门南至隅一百步为股率,南门东至隅一百步 为见句步。欲以见句求股,以为出南门数。正合半邑方自乘者,股率当乘见句, 此二者数同也。〕 今有邑东西七里,南北九里,各中开门。出东门一十五里有木。问出南门几 何步而见木?答曰:三百一十五步。

术曰:东门南至隅步数,以乘南门东至隅步数为实。以木去门步数为法。实 如法而一。

〔此以东门南至隅四里半为句率,出东门一十五里为股率,南门东至隅三里 半为见股。所问出南门即见股之句。为术之意,与上同也。〕 今有邑方不知大小,各中开门。出北门三十步有木,出西门七百五十步见木。

问邑方几何?答曰:一里。

术曰:令两出门步数相乘,因而四之,为实。开方除之,即得邑方。

〔按:半邑方,令半方自乘,出门除之,即步。令二出门相乘,故为半方邑 自乘,居一隅之积分。因而四之,即得四隅之积分。故为实,开方除,即邑方也。〕 今有邑方不知大小,各中开门。出北门二十步有木,出南门一十四步,折而 西行一千七百七十五步见木。问邑方几何?答曰:二百五十步。

术曰:以出北门步数乘西行步数,倍之,为实。

〔此以折而西行为股,自木至邑南一十四步为句,以出北门二十步为句率, 北门至西隅为股率,半广数。故以出北门乘折西行股,以股率乘句之幂。然此幂 居半,以西行。故又倍之,合东,尽之也。〕 并出南、北门步数,为从法,开方除之,即邑方。

〔此术之幂,东西如邑方,南北自木尽邑南十四步之幂,各南北步为广,邑 方为袤,故连两广为从法,并,以为隅外之幂也。〕 今有邑方一十里,各中开门。甲、乙俱从邑中央而出:乙东出;甲南出,出 门不知步数,邪向东北,磨邑隅,适与乙会。率:甲行五,乙行三。问甲、乙行 各几何?答曰:甲出南门八百步,邪东北行四千八百八十七步半,及乙。乙东行 四千三百一十二步半。

术曰:令五自乘,三亦自乘,并而半之,为邪行率;邪行率减于五自乘者, 余为南行率;以三乘五为乙东行率。

〔求三率之意与上甲乙同。〕 置邑方,半之,以南行率乘之,如东行率而一,即得出南门步数。

〔今半方,南门东至隅五里。半邑者,谓为小股也。求以为出南门步数。故 置邑方,半之,以南行句率乘之,如股率而一。〕 以增邑方半,即南行。

〔半邑者,谓从邑心中停也。〕 置南行步,求弦者,以邪行率乘之;求东行者,以东行率乘之,各自为实。

实如法,南行率,得一步。

〔此术与上甲乙同。〕 今有木去人不知远近。立四表,相去各一丈,令左两表与所望参相直。从后 右表望之,入前右表三寸。问木去人几何?答曰:三十三丈三尺三寸少半寸。

术曰:令一丈自乘为实,以三寸为法,实如法而一。

〔此以入前右表三寸为句率,右两表相去一丈为股率,左右两表相去一丈为 见句。所问木去人者,见句之股。股率当乘见句,此二率俱一丈,故曰自乘之。

以三寸为法。实如法得一寸。〕 今有山居木西,不知其高。山去木五十三里,木高九丈五尺。人立木东三里, 望木末适与山峰斜平。人目高七尺。问山高几何?答曰:一百六十四丈九尺六寸 太半寸。

术曰:置木高,减人目高七尺, 〔此以木高减人目高七尺,余有八丈八尺,为句率;去人目三里为股率;山 去木五十三里为见股,以求句。加木之高,故为山高也。〕 余,以乘五十三里为实。以人去木三里为法。实如法而一。所得,加木高, 即山高。

〔此术句股之义。〕 今有井,径五尺,不知其深。立五尺木于井上,从木末望水岸,入径四寸。

问井深几何?答曰:五丈七尺五寸。

术曰:置井径五尺,以入径四寸减之,余,以乘立木五尺为实。以入径四寸 为法。实如法得一寸。

〔此以入径四寸为句率,立木五尺为股率,井径之余四尺六寸为见句。问井 深者,见句之股也。〕 今有户不知高、广,竿不知长短。横之不出四尺,从之不出二尺,邪之适出。

问户高、广、邪各几何?答曰:广六尺。高八尺。邪一丈。

术曰:从、横不出相乘,倍,而开方除之。所得,加从不出,即户广; 〔此以户广为句,户高为股,户邪为弦。凡句之在股,或矩于表,或方于里。

连之者举表矩而端之。又从句方里令为青矩之表,未满黄方。满此方则两端之邪 重于隅中,各以股弦差为广,句弦差为袤。故两端差相乘,又倍之,则成黄方之 幂。开方除之,得黄方之面。其外之青知,亦以股弦差为广。故以股弦差加,则 为句也。〕 加横不出,即户高;两不出加之,得户邪。

\end{document}