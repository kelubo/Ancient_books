%-*- coding: UTF-8 -*-
% 孙子兵法
% 孙子兵法.tex

\documentclass[12pt,UTF8]{ctexbook}

% 设置纸张信息。
\usepackage[a4paper,twoside]{geometry}
\geometry{
	left=25mm,
	right=25mm,
	bottom=25.4mm,
	bindingoffset=10mm
}

% 设置字体,并解决显示难检字问题。
\xeCJKsetup{AutoFallBack=true}
\setCJKmainfont{SimSun}[BoldFont=SimHei, ItalicFont=KaiTi, FallBack=SimSun-ExtB]

% 目录 chapter 级别加点(.)。
\usepackage{titletoc}
\titlecontents{chapter}[0pt]{\vspace{3mm}\bf\addvspace{2pt}\filright}{\contentspush{\thecontentslabel\hspace{0.8em}}}{}{\titlerule*[8pt]{.}\contentspage}

% 设置 part 和 chapter 标题格式。
\ctexset{
	chapter/name={},
	chapter/number={}
}

% 设置古文原文格式。
\newenvironment{yuanwen}{\bfseries\zihao{4}}

% 设置署名格式。
\newenvironment{shuming}{\hfill\bfseries\zihao{4}}

% 注脚每页重新编号,避免编号过大。
\usepackage[perpage]{footmisc}

\title{\heiti\zihao{0} 孙子兵法}
\author{孙武}
\date{}

\begin{document}
	
\maketitle
\tableofcontents

\frontmatter
\chapter{前言}
    
    3. 译文
    
《孙子兵法》,俗称为《孙子》,《汉书·艺文志》著录为《吴孙子兵法》。是现存我国历史上的第一部兵书。

现在的传世本共有《计》、《作战》、《谋攻》、《形》、《势》、《虚实》、《军争》、《九变》、《行军》、《地形》、《九地》、《火攻》、《用间》十三篇。其内容博大精深,理论高度概括,逻辑缜密严谨,实践层出不穷,不但是我国古典军事文化遗产中的璀璨瑰宝,而且也是我国优秀传统文化宝库中的重要组成部分。

关于作者问题,历来众说纷纭,有认为是春秋时期客居吴国的齐人孙武所著,有认为是孙武的后世子孙孙膑整理而成,有认为是战国初年某位山林处士编写,还有的说是三国时代曹操编撰而成等等。

余嘉锡先生在其《古书通例》中说,“古人著书既多单篇别行,不自编次”,“本无专集,往往随作数篇即已行世”,“古书书名,本非作者自题,后人既为之编次成书,知其为某家之学,则题其氏若名以为识别”。

故认为《孙子兵法》既非出自一人之手,也非成书于一时。它的编辑成书,当为孙武的后学所辑。据李零先生考证认为:“《孙子》一书作为‘孙子学派’军事思想的记录,其中某些内容可能在春秋末期就已产生,但是它作为一部完整的书却是由战国时期的人整理完成的。”

孙武的一生,除了其赫赫战功以外,更主要的就是他给后人留下了不少珍贵的论兵、论政的篇章,其中尤以流传下来的《孙子兵法》最为著名。在这短短的十三篇五千字中,体现了孙武完整的军事思想体系。

关于它的成书问题,李零先生认为:《孙子兵法》的成书过程可能正与竹简本《齐孙子》所说“明之吴越,言之于齐”相合,大约是从春秋末期的吴国开始到战国时期的齐国,经过长期整理最后完成。《孙子》书中虽然有某些字句叙述了春秋末期的史实,但就其总体来说,却应是完成于战国时期。

该书自问世以来,对中国古代军事学术的发展产生了巨大而深远的影响,被人们尊奉为“兵经”、“百世谈兵之祖”。它不仅是历代军事家用于指导战争实践的必读之书,而且对于当今多数中国人来说也并不陌生,“知彼知己,百战不殆”、“不战而屈人之兵”、“避实击虚”等,这些古老的军事格言已成为现代社会生活的智慧而广为流传。作为至今已有两千多年的古代兵书,其中多数作战方式早已不适用于现代战争,但作为思想,其基本原则已渗透到现代军事,乃至商业竞争、企业管理、体育竞赛、外交谈判等诸多领域。

《孙子兵法》有一个非常全面而完整的体系,体现了战争与政治、经济、文化等各方面的关系。黄朴民先生研究认为:《孙子兵法》一书至少包含了四个层次的内容,即战争观念、战略思想、战术思想和治军思想。

战争观念即对战争的基本态度。在这一问题上,《孙子兵法》提出了以“慎战”、“备战”为核心内涵的“安国全军之道”。它对战争持十分慎重的态度,开宗明义地表示“兵者,国之大事,死生之地,存亡之道,不可不察也”。它坚决反对在战争问题上轻举妄动,穷兵黩武。强调“主不可以怒而兴师,将不可以愠而致战”。要做到“非利不动,非得不用,非危不战”,“合于利而动,不合于利而止”。但孙武同时也要求加强备战,立足于未雨绸缪、有备无患。“慎战为主”、“备战为辅”,他认为惟有“慎战”与“备战”并重,方可“安国全军”,在当时日趋激烈的争霸兼并战争中牢牢把握战争的主动权,方可立于不败之地。

在战略思想上,《孙子兵法》推崇“不战而屈人之兵”的全胜攻略,提倡以最小的军事代价取得最大的政治成果。因此在实行方式上,重视“伐谋”与“伐交”,即依靠谋略取胜,或依靠外交取胜。在作战行动上,它强调主动进攻,突然袭击,速战速决的刚性战略。即“用兵贵在神速”。在谋略的背后,它又强调实力政策。“先为不可胜”,要有强大的军事实力,以压倒性的优势超过对方。

《孙子兵法》的战术思想是全书的精髓,唐太宗李世民曾评价:“观诸兵书,无出孙武;孙武十三篇,无出虚实。”这其中最精要之句便是“致人而不致于人”,即要调动别人而不被别人所调动,要牢牢掌握主动权。

《孙子兵法》的治军思想至今在军事、商业等多种领域仍有广泛的借鉴意义。所谓治军思想即如何选拔将领、训练部队、调动士兵的积极性以及如何建设和谐军队。它的核心治军思想可用“令文齐武”来概括。所谓“文”就是用政治训导、物质鼓励、精神奖掖来教育部队;所谓“武”就是通过严格的军纪军法来约束部队,有过则杀。文武并用,以确保军队的战斗力致最大的发挥。

孙武的军事思想具有朴素的唯物论和辩证法观点,他强调战争的胜负不取决于鬼神,而是与政治清明、经济发展、外交努力、军事实力、自然条件诸因素有着密切的关系,决定战争胜负的主要原因就是要看以上这些条件具备的如何,这就体现了他朴素的唯物论观点。此外孙武认为世界是客观存在的,而且世界上的事物都在不停地变化着,他强调在战争中应积极创造条件,发挥人的主观能动性,促使敌人朝着有利于自己的方向转化,表明孙武已掌握了基本朴素的辩证法。正是因为孙武在这门具体学科中概括和总结出了异常丰富、多方面的哲学道理,才确立了他在春秋末期思想界中与孔子、老子的并列地位,被并称为春秋末期思想界的三颗明星。

自《汉书·艺文志》以下,《孙子兵法》历代都有著录。最值得庆幸的是1972年4月在山东临沂银雀山一号汉墓中出土了竹简本《孙子兵法》,为迄今最早的传世写本,可惜为残简,不能窥其全貌。竹简本《孙子兵法》除有传世本十三篇残简外,尚有五篇佚文。十三篇残简,除传世本《地形》篇外,各篇都有所发现,共有两千七百余字。由于竹简出土时严重残损,正文篇题仅存《作战》、《刑(形)》、《執(势)》、《虚实》、《九地》、《火攻》、《用间》七个,皆抄写在各篇篇首第一简简背上端。此外该墓还同时出土了一块《孙子》篇题的木牍残片,虽然上面的篇题文字有些残缺(只残存七个),但据正文中保存的篇题也可以补出一些。据学者研究,该木牍应为《孙子》十三篇的全部篇题。简本《孙子》的文字内容与今本大同小异,但也有些不同如篇次的安排和今本也不尽相同,文辞比今本古奥,句子比今本简短等。篇题木牍的发现,为复原简本《孙子》的编次和推估其卷帙规模提供了依据,同时也可借以考求今本《孙子》的流传源委。

银雀山汉简《孙子兵法》的佚文共存五篇,即《吴问》、《四变》、《黄帝伐赤帝》、《地形二》、《见吴王》,其中除《四变》和《见吴王》篇题为整理者所加外,其余皆为原书篇题,都书写在各篇第一简简背。《吴问》篇记载了吴王与孙子关于晋国六卿军事、政治制度的问答,这类问答形式在传世本十三篇中没有发现,但见《通典》等书所引《孙子》佚文中。《四变》是解释传世本《九变》篇的某些字句的。《黄帝伐赤帝》的内容与传世本《行军》有关,也有人认为是解释《行军》中四种“处军之利”的内容。《地形二》疑为传世本以外的另一篇论述地形的文字。从内容上看,好像是发挥、解释《行军》、《九地》等篇的内容的。《见吴王》的内容和《史记·孙子吴起列传》中所记孙子见吴王阖庐以兵法试诸妇人之事大致相同。李零先生认为:“这几篇几乎都是解释发挥十三篇,让人觉得比十三篇晚出。但我们也不能排除,在战国时期,还有早于十三篇或与十三篇同时的孙子书存在,传世十三篇只是‘选萃’”。

传世《孙子兵法》的注释本,最早应推东汉未年曹操的《孙子略解》(即传世的《魏武帝注孙子》)。曹操以后到宋代比较著名的注家主要有三国吴沈友、南朝梁孟氏,隋张子尚、萧吉,唐李荃、杜牧、陈皞、贾林,宋梅尧臣、王皙、何延锡、张预、宋奇等人。上述各家,除沈友、张子尚、萧吉、宋奇四家注已佚外,其余九家注与曹操注、唐杜佑《通典》之《孙子》引文注被合为《十一家注孙子》保存下来。虽然这些注本还存在有不少缺点,但由于它们的时代较早,为我们保存了许多古文异本、校说和古代训话,具有不可替补的价值。现存重要的版本为南宋宁宗时所刻《十一家注孙子》、宋刻与宋抄《武经七书》本,其中宋本《十一家注孙子》经清代孙星衍校定考辨后,成了近世流传最广,影响最大,而且最被大家所推崇的读本。

\mainmatter

\chapter{计篇第一}
	
《说文》云:计,会也,筭\footnote{su\`an}(算)也。“筭”是一种原始的计数工具,与筹、策是同类性质的东西。古代出兵打仗之前都要先在庙堂上用这种工具计算敌我优势,叫“庙算”。战争是关系到国家生死存亡的大事,所以在出兵作战前,一定要认真分析敌我作战的基本条件。本篇以“计”名篇,论述了决定战争胜败的五个基本原则。最后着重指出,对站前的作战计划、作战意图必须深思熟虑。

\begin{yuanwen}
孙子曰:兵\footnote{指战争。}者,国之大事,死生之地,存亡之道,不可不察\footnote{考察,研究。}也。
\end{yuanwen}
    
孙子说:战争是国家的大事,地形上的死地、生地,战场上的存亡胜败,是不能不慎重分析研究的。
    
\begin{yuanwen}
故经\footnote{衡量。这里是分析研究的意思。}之以五事,校\footnote{ji\`ao,比较。}之以计,而索\footnote{探索。}其情\footnote{指实际情况。}。一曰道\footnote{道义、品德。这里指在政治方面是否得民心。},二曰天\footnote{指天时(气象、时令)方面的条件。},三曰地\footnote{指地理条件。如距离远近、险要平坦、广阔狭隘等地形、地势。},四曰将\footnote{将令。这里指统率作战的将帅的谋略、智能以及指挥士兵作战的方法等。},五曰法\footnote{军法、法令。这里指军队的编制、将帅的职掌、军备物资的供给等。}。道者,令民与上\footnote{指国君。}同意\footnote{同心。}也,故可以与之死,可以与之生,而不畏危\footnote{不惧怕危险。}也。天者,阴阳\footnote{这里指昼夜、晴雨等气象变化。}、寒暑、时制也\footnote{指春、夏、秋、冬四季。}。地者,远近、险易\footnote{地势的险阻平坦。}、广狭、死生\footnote{这里指死地与生地,即地形是否有利于攻守进退。}也。将者,智、信、仁、勇、严也。法者,曲制\footnote{指军队编制制度。}、官道\footnote{指各级将吏的职责划分以及统辖管理制度。}、主\footnote{掌管。}用\footnote{物资费用。}\footnote{掌管物资费用的后勤管理制度。}也。凡此五者,将莫不闻\footnote{知道。},知之者胜,不知者不胜。
\end{yuanwen}
	
所以,要通过以下五个方面来研究,比较分析双方的各种条件,考察双方的实际情况,来预测战争胜负的可能性。一是道义,二是天时,三是地理,四是将帅,五是法规。所谓“道义”,就是使民众和君主的心意相同,这样才可以同生共死,而不惧怕危险。所谓“天时”,就是指阴阳、寒暑、春夏秋冬四季。所谓“地形”,就是指路程的远近、地势的险要平坦、战场的广阔狭窄、是生地还是死地等地理条件。所谓“将帅”,就是指将帅的智谋才能、赏罚有信、对部下仁慈关爱、果断勇敢、军纪严明。所谓“法规”,就是指军队组织的编制、将吏责权的划分、军需物资的掌管和供给。对于这五个方面,身为将领要深刻了解。了解的就能胜利,否则就不能胜利。

\begin{yuanwen}
故校之以计,而索其情。曰:主孰\footnote{疑问代词,谁,哪一方。}有道?将孰有能?天地孰得?法令孰行?兵众孰强?士卒孰练?赏罚孰明?吾以此知胜负矣。
\end{yuanwen}
	
所以,要通过双方的考察分析,掌握实际情况,并据此加以比较,从而来预测战争胜负的情形。就要问:哪一方的君主能得民心?哪一方的将领更有能力?哪一方占有天时地利?哪一方的法规、法令更能严格执行?哪一方的兵力更强大?哪一方的士卒训练更加有素?哪一方的赏罚更公正严明?我根据这些分析比较,就可以判明双方的胜负了。
	
\begin{yuanwen}
将\footnote{虚词,表示假设。}听吾计,用之必胜,留之;将不听吾计,用之必败,去\footnote{离开。}之。
\end{yuanwen}
	
如果采用我的计策,指挥作战必胜,我就留下;如果不采用我的计策,指挥作战必败,我就离开。
	
\begin{yuanwen}
计利\footnote{分析后有优势,有利于我。}以听\footnote{听从,采纳。},乃为之势\footnote{含有态势之意。如战略形势、战术态势、战场优势等。它与“形”相反,多指随机的、能动的东西,如指挥的灵活、士气的勇怯等。},以佐\footnote{辅助。}其外\footnote{指国境之外。古时用兵多在境外,如《管子·七法》里说:“故凡攻伐以为道也,计必先定于内,然后出兵乎境。计未定于内而兵出乎境,是则战之自胜,攻之自毁也。”}。势者,因利而制权\footnote{权变,灵活处置。}\footnote{指根据实际利害关系而灵活应变。}也。
\end{yuanwen}
	
筹划有利并且能得到执行,还要设法造“势”,来协助在外的军事行动。所谓“势”,就是根据实际利害关系而采取相应的措施。
	
\begin{yuanwen}
兵者,诡\footnote{诡秘、诡诈。意思是说用兵打仗是一种诡诈的行为。}道也。故能而示\footnote{显示、表示。这里也含有伪装的意思。}之不能,用而示之不用,近而示之远,远而示之近。利而诱之,乱而取\footnote{指攻取。}之,实而备之,强而避之,怒而挠\footnote{打扰,扰乱。}之,卑\footnote{这里可理解为卑弱而谨慎。}而骄之,佚\footnote{通“逸”,安逸、安稳。这里是指休整充分的意思。}而劳\footnote{使动用法,使疲劳。}之,亲而离\footnote{离间。}之。攻其无备,出其不意。此兵家之胜,不可先传\footnote{传授,在这里可引申为“规定”。}也。
\end{yuanwen}
	
用兵作战,就是一种诡诈的行为。因此,能攻却要装出不能攻,要打却要装出不去打;欲从近处攻打却要装出从远处攻打,欲从远处攻打却要装出从近处攻打;对方贪利就要用利益来诱惑他,对方混乱就要趁机攻取他;对方充实就要防备他,对方强大就要躲避他;丢分暴躁易恼怒就要骚扰他,对方自卑谨慎就要使他骄傲自大;对方休整充分就要使其劳累,对方内部团结就要设法离间他。攻打对方没有防备之处,在对方没有料到的时机发动进攻。这些都是军事家克敌制胜的诀窍,要在战争中根据实际情况灵活应用,不可能事先作出死板的规定。
	
\begin{yuanwen}
夫未战而庙算\footnote{古时出兵作战之前,都要到宗庙里举行仪式,商讨作战计划,这就叫“庙算”。}胜者,得算\footnote{本指计数用的筹码,这里引申指取得胜利的条件。}多\footnote{得算多,指取得胜利的条件多。}也;未战而庙算不胜者,得算少也。多算胜,少算不胜,而况于无算乎!吾以此观之,胜负见矣。
\end{yuanwen}
	
还没有出兵交战,就在“庙算”上先已获胜,是由于得到的“算筹”较多。还没有出兵交战,就在“庙算”上先已失败,是由于得到的“算筹”较少。得到“算筹”较多的就可能取胜,得到“算筹”较少的就不会取胜,更何况那些没有得到“算筹”的呢?我根据“庙算”的结果来观察,胜负之分就显而易见了。
	
\chapter{作战篇第二}
	
本篇主要从战争对人力、物力、财力的依赖关系出发,论述了“持久作战”会给国家带来危害的观点,提出了“兵贵胜,不贵久”的速战速决的军事思想。任何一场战争都是双方军事实力的较量,而军事实力所依赖的则是综合国力的强弱。军队长期在外作战,国家的财力终会枯竭,长期消耗,必然会导致战争的失败。为减轻作战的负担,进一步提出“取用于国,因粮于敌”的主张,减少远程运输,节约作战开支,这在当时来说是一种很了不起的军事思想。
	
\begin{yuanwen}
孙子曰:凡用兵之法,驰车\footnote{一种轻型战车。}千驷\footnote{古代四马拉一车为“驷”。},革车\footnote{古代运载辎重的战车。}千乘\footnote{辆。},带甲\footnote{指用甲胄武装起来的士卒。}十万,千里馈\footnote{运送。}粮,则内外之费,宾客之用,胶漆之材\footnote{泛指制作和维护作战器械所需的材料。},车\footnote{车辆。}甲\footnote{盔甲。}\footnote{这里泛指各种军事装备。}之奉\footnote{供给,补充。},日费千金,然后十万之师举\footnote{出兵作战。}矣。
\end{yuanwen}
	
孙子说:凡兴兵作战,需要出动轻车千辆,重车千乘,兵士十万,并千里迢迢运送粮食。这样一来,前方后方的各种开支,招待宾客策士的费用,物资器材,战车、甲胄的供给等,每天要花费千金,之后十万大军方才出兵作战。
	
\begin{yuanwen}
其用战\footnote{用兵作战。}也胜,久则钝兵\footnote{钝了刃的刀。}挫锐\footnote{挫了尖的矛。}\footnote{这里比喻军队疲惫,锐气挫伤。},攻城则力屈\footnote{指战斗力衰竭。},久暴\footnote{p\`u}师\footnote{军队长期在外作战。}则国用\footnote{国家的开支。}不足。夫钝兵挫锐,屈力殚\footnote{枯竭。}货\footnote{财货。}\footnote{耗尽物力财力。},则诸侯乘其弊\footnote{指疲惫。}而起,虽有智者,不能善其后矣。故兵闻拙\footnote{笨拙。}速,未睹巧\footnote{巧妙。}之久也。夫兵久\footnote{指作战时间长。}而国利\footnote{对国家有利。}者,未之有也。故不尽知用兵之害者,则不能尽知用兵之利也。
\end{yuanwen}
	
用这样庞大的军队作战务求速胜,持久就会使军队疲惫,锐气挫伤。攻城就会耗尽兵力,军队长期在外作战,必然导致国家财用不足。如果军队疲惫不堪、锐气受挫、军队实力耗尽、国内资源枯竭,那么诸侯就会乘机向我发起进攻,即使有智谋之士也无法挽救如此危局。所以在实际作战中只听说过宁拙而求速胜,没见过求巧而久拖的。战争旷日持久而对国家有利的,从来没有过。所以,不能完全了解用兵危害的将领,就不能完全了解用兵的益处。

\begin{yuanwen}
善用兵者,役不再籍\footnote{指户籍。古代按户籍征兵。},粮不三载\footnote{运输。};取用于国,因\footnote{依靠。}粮于敌,故军食可足也。
\end{yuanwen}
	
善于用兵的人,不用再三从国内征兵,不用再三从国内运粮。武器装备由国内供应,从敌人那里夺取粮食,这样,军队的粮草就可以充足了。
	
\begin{yuanwen}
国之贫于师者远输\footnote{远道运输。},远输则百姓贫。近于师者贵卖,贵卖则百姓财竭,财竭则急于丘役\footnote{按丘征集的赋税徭役。丘,是古代的地方行政单位。据《周礼》,在古代九家为井,四井为邑,四邑为丘。}。力屈、财殚,中原\footnote{这里指国内。}内虚于家。百姓之费,十去其七;公家之费,破车罢\footnote{p\'i,通“疲”。}马,甲胄矢弩,戟楯\footnote{同“盾”。}蔽橹\footnote{大盾牌。},丘牛\footnote{指从“丘”征集来的牛。}大车\footnote{指辎重车。},十去其六。
\end{yuanwen}
	
国家因作战而贫困,是由于军队远途运输,远途运输就会导致百姓贫穷。靠近驻军的地方物价必然会暴涨,物价暴涨就会导致百姓的财物枯竭,财物枯竭国家就会急于征集赋税和劳役。军力耗尽,财源枯竭,国内空虚。百姓的私家财产损耗掉十分之七;公家的财产,由于战车破损,战马疲惫,甲胄、弓箭、矛戟、盾牌、拉辎重的牛车也损耗掉十分之六。
	
\begin{yuanwen}
故智将务\footnote{必须。}食于敌。食敌一钟\footnote{古代的容量单位。六十四斗为一钟。},当吾二十钟;芑\footnote{q\'i,通“萁”,豆秸。}秆\footnote{禾茎。}一石\footnote{古代的重量单位。一百二十斤为一石。},当吾二十石。
\end{yuanwen}
	
所以明智的将军一定要靠敌国解决粮草,从敌国搞到一钟的粮食就相当于从本国运来二十钟,在当地取得草料一石,就相当于从本国运来二十石。

\begin{yuanwen}
故杀敌者,怒\footnote{指激起士兵对敌人的愤怒。}也;取敌之利者,货\footnote{财货。指用来奖赏士兵的财物。}也。故车战得车十乘已上,赏其先得者。而更\footnote{更换。}其旌旗\footnote{旗帜。},车杂\footnote{混合,搀杂。}而乘之,卒善而养之,是谓胜敌而益强。
\end{yuanwen}
	
所以,要使士兵拼死杀敌,就必须激起士兵对敌人的愤怒。要使士兵勇于夺取敌方的军需物资,就必须用缴获的财物来奖赏士兵。所以,在车战中,凡夺取敌军战车十辆以上的,就奖赏最先夺取战车的人。而夺得的战车要立即换去上面的旗帜,编入我方车队而夹杂在一起乘用。对俘虏来的士卒要优待他们、供养他们,这就是所谓战胜敌人而使自己日益强大的原因。

\begin{yuanwen}
故兵贵胜,不贵久。故知兵之将,生民之司命,国家安危之主\footnote{主宰。}也。
\end{yuanwen}
	
所以,用兵作战贵在速胜,最不宜的是旷日持久。深知用兵之道的将帅,是民众命运的掌握者,是国家安危的主宰者。
	
\chapter{谋攻篇第三}
	
“谋攻”就是用计谋来征服敌人。孙武认为理想的作战结果是“全国为上”、“全军为上”、“全旅为上”、“劝阻微商”、“全伍为上”,“破国”、“破军”、“破旅”、“破卒”、“破伍”则次之。但最理想的作战结果是“不战而屈人之兵”。如何才能达到最佳的作战效果?就是要用计谋去战胜敌人。本篇主要论述了“上兵伐谋”的思想、国君与将帅的关系、致胜的条件和致败的原因,最后提出了著名的“知己知彼,百战不殆”的思想,作战的谋略必须建立在了解敌我双方情况的基础上。
	
\begin{yuanwen}
孙子曰:凡用兵之法,全国为上,破国次之;全军\footnote{古代军队的编制单位。据《周礼》,一万两千五百人为一军。}为上,破军次之;全旅\footnote{古代军队的编制单位。据《周礼》,五百人为一旅。}为上,破旅次之;全卒\footnote{古代军队的编制单位。据《周礼》,百人为一卒。}为上,破卒次之;全伍\footnote{古代军队的编制单位。据《周礼》,五人为一伍。}为上,破伍次之。是故百战百胜,非善之善者也;不战而屈人之兵,善之善者也。
\end{yuanwen}
	
孙子说:大凡作战的原则是:使整个敌国屈服是上策,用武力攻破敌国使之屈服就差一些;使敌人全军降服是上策,击破敌军就差一些;使敌人全旅降服是上策,击破敌旅就差一些;使敌人全卒降服是上策,击破敌卒就差一些;使敌人全伍降服是上策,击破敌伍就差一些。所以说,百战百胜,算不上是好中之最好的;不通过交战就使敌人的军队降服,这才是好中之最好的。

\begin{yuanwen}
故上兵\footnote{最好的军事手段。}伐谋\footnote{用谋略讨伐。},其次伐交\footnote{用外交手段去讨伐。},其次伐兵\footnote{用武力去讨伐。},其下攻城。攻城之法为不得已。修\footnote{制造。}橹\footnote{古代的一种攻城工具,即“楼橹”。}轒辒\footnote{f\'en,古代攻城用的一种四轮车具。},具\footnote{准备。}器械,三月而后成;距闉\footnote{y\=in,是指堆积攻城用的土山。闉,通“堙”,土山。},又三月而后已。将不胜其忿而蚁附\footnote{指士兵像蚂蚁一样的爬城。}之,杀士\footnote{指士卒伤亡。}三分之一而城不拔\footnote{指城被攻下。}者,此攻之灾也。
\end{yuanwen}
	
所以,好中之最好的军事行动是用谋略挫败敌人,其次就是用外交手段战胜敌人,再次就是用武力击败敌军,最下之策是攻打敌人的城池。攻城是迫不得已采取的方法。制造橹、轒辒等各种攻城工具,准备所有的攻城器械,三个月才能完成。堆筑攻城的土山,又得三个月才能完成。如果将领难以抑制焦躁情绪,命令士兵像蚂蚁一样爬梯攻城,尽管士兵死伤三分之一,而城池仍然攻不下来,这就是攻城所带来的灾难。

\begin{yuanwen}
故善用兵者,屈人之兵而非战\footnote{指不用交战的方法。}也,拔人之城而非攻\footnote{指不用强攻的方法。}也,毁人之国而非久\footnote{指战争不要旷日持久。}也,必以全争于天下。故兵不顿\footnote{通“钝”,疲惫,受挫。}而利可全,此谋攻之法也。
\end{yuanwen}
	
所以善于用兵打仗的人,不通过打仗就使敌人屈服,不通过攻城就使敌城投降,摧毁敌国不需长期作战;一定要用“全胜”的策略争胜于天下,这样既不使国力兵力疲惫,又获得了全面胜利的利益,这就是谋攻的法则。

\begin{yuanwen}
故用兵之法,十则围之,五则攻之,倍则分之,敌\footnote{指与敌人兵力相等,势均力敌。}则能战之,少则能逃\footnote{摆脱,逃离。}之,不若\footnote{指条件不如敌人。}则能避之。故小敌之坚,大敌之擒也。
\end{yuanwen}
	
所以,用兵作战的原则是:十倍于敌就围歼敌人,五倍于敌就进攻敌人,一倍于敌就要设法分散敌人,势均力敌就要设法战胜敌人,兵力少于敌就设法摆脱敌人,如果各种条件不如敌人就要避免作战。所以,弱小的军队如果坚持硬拼,那就会被强大的敌人所俘虏。

\begin{yuanwen}
夫将者,国\footnote{指国君。}之辅\footnote{辅佐。}也。辅周\footnote{辅佐周到。},则国必强;辅隙\footnote{辅佐有漏洞、缺陷。},则国必弱。
\end{yuanwen}
	
将帅是国君的辅佐,辅佐得缜密周祥,国家就必然会强大,辅佐得有疏漏失当,国家就必然会衰弱。

\begin{yuanwen}
故君之所以患\footnote{危害,贻害。}于军者三:不知军之不可以进而谓\footnote{告诉。这里有命令的意思。}之进,不知军之不可以退而谓之退,是谓縻\footnote{m\'i,羁縻,牵制。这里是指军队受到束缚。}军;不知三军之事,而同\footnote{参与。这里有干涉的意思。}三军之政\footnote{行政。指军队的工作。}者,则军士惑矣;不知三军之权\footnote{权谋,权变。},而同三军之任\footnote{指挥。},则军士疑矣。三军既惑且疑,则诸侯之难至矣,是谓乱军\footnote{把自己的军心扰乱。}引胜\footnote{导致敌人胜利。}。
\end{yuanwen}
	
所以,国君给军队造成的危害有三种:不知道军队不可以前进而下令前进,不知道军队不可以后退而下令后退,这叫束缚牵制军队;不懂得三军战守之事而要参与和干涉三军之政,将士们就会迷惑而无所适从;不懂得三军战略战术的权宜变化而要参与和干涉三军的指挥,将士们就会产生疑虑。军队既无所适从,又疑虑重重,各诸侯就会趁机兴兵作难。这就会把自己的军心扰乱而导致敌人胜利。

\begin{yuanwen}
故知胜有五:知可以战与不可以战者胜,识众寡\footnote{指军队力量配备的多少。}之用者胜,上下同欲\footnote{指同心、齐心。}者胜,以虞\footnote{准备。}待不虞者胜,将能而君不御\footnote{驾御。这里引申为牵制、干预的意思。}者胜。此五者,知胜之道也。
\end{yuanwen}
	
所以,有五个方面可以预见胜利:能够准确判断仗能打或不能打的会取得胜利;能够知道根据敌我双方而配备兵力的会取得胜利;全军上下同心协力的会取得胜利;有充分准备的对付毫无准备的会取得胜利;将领的才能精通军事、精于权变而君主又不加干涉的会取得胜利。以上五条就是预见胜利的方法。

\begin{yuanwen}
故曰:知彼知己者,百战不殆\footnote{危险,失败。};不知彼而知己,一胜一负;不知彼,不知己,每战必殆。
\end{yuanwen}
    
所以说:了解对方也了解自己,每次战斗都不会失败;不了解对方只了解自己,就可能胜负各占一半;既不了解对方又不了解自己,那就会每战必败。
	
\chapter{军形篇第四}
	
该篇以“形”字命题,所谓“形”,与下篇所论的“势”是一对矛盾的概念。据《汉书·艺文志·兵书略》中记载,任宏当年论次兵书有四种,其中第二种为“形势”,而“形势”则属于战术学的范畴,其特征是“雷动风举,后发而先至,离合背乡,变化无常,以轻疾制敌者也”。任宏所说的“形势”是一个合成词,“形”和“势”似乎无别,都是指人为造成的态势。但在《孙子兵法》中“形”和“势”是有明显区别的。“形”含有形象、形体等义,是指战争中客观、经常、易见的诸多因素。从本篇内容来看,主要是指军事的实力、力量的强弱等。所以本篇反复论述的内容主要是“胜可知而不可为”,“故善战者立于不败之地,而不失敌之败也”。并且把战争的物资准备(“地生度,度生量,量生数,数生称,称生胜”)看作是取胜的根本条件。决定战争胜利的因素,主要是军事实力和战略战术的谋划。善战者首先是创造出不被敌人打败的条件,然后再伺机打败敌人。
	
\begin{yuanwen}
孙子曰:昔之善战者,先为不可胜\footnote{不可战胜。},以待敌之可胜\footnote{可战胜。}。不可胜在己,可胜在敌。故善战者,能为不可胜,不能使敌之可胜。故曰:胜可知而不可为\footnote{指在条件不具备的情况下不能硬做。}。
\end{yuanwen}
	
孙子说:过去善于用兵作战的人,总是首先创造自己不可战胜的条件,然后等待可以战胜敌人的机会。使不被敌人战胜的主动权掌握在自己手中,能否战胜敌人,在于敌人是否给以可乘之机。所以,善于作战的人能做到自己不被敌人战胜,却不能做到是敌人一定会为我所胜。从这个意义上说,胜利可以预见,但在条件不具备的情况下不能强为。

\begin{yuanwen}
不可胜者,守\footnote{采取防守。}也;可胜者,攻\footnote{采取进攻。}也。守则不足,攻则有余\footnote{竹简为:守则有余,攻则不足。}。善守者,藏于九地之下\footnote{是说隐藏的深不可知。古人常用“九”来表示数的极点。},善攻者,动于九天\footnote{极言高不可及。}之上,故能自保而全胜也。
\end{yuanwen}

当敌人不能战胜时,应采取防守的战术;当可以乘机战胜敌人时,就采取进攻的战术。采取防守是因为我方兵力不足,采取进攻是因为我方兵力有余。善于防守的,把自己的兵力隐藏在深不可测的地方;善于进攻的,就像部队从高不可及的天空而降,因此能既保全自己又获得全胜。

\begin{yuanwen}
见胜不过\footnote{指没有超过。}众人之所知,非善之善者也;战胜而天下曰善,非善之善者也。故举秋毫\footnote{本指秋天鸟兽的细毛。这里比喻非常细微的东西。}不为多力,见日月不为明目\footnote{眼睛明亮。},闻雷霆不为聪耳\footnote{指耳朵灵。}。古之所谓善战者,胜于易胜者\footnote{指容易被战胜的人。}也。故善战者之胜也,无智名\footnote{指有智慧的名声。},无勇功\footnote{指勇敢杀敌的名声。}。故其战胜不忒\footnote{t\`e,没有差错。}。不忒者,其所措\footnote{指作战措施。}必胜,胜已败者\footnote{指已经处于失败地位的敌人。}也。故善战者,立于不败之地,而不失敌之败\footnote{指使敌人致败的时机。}也。是故胜兵先胜\footnote{指先造成的取胜条件。}而后求战,败兵先战而后求胜。善用兵者,修道而保法,故能为胜败之政\footnote{这里指主宰战争的胜负。}。
\end{yuanwen}

能预见胜利但没有超过大家的见识,就不能算是好中最好的;打了胜仗而天下人都称赞,也不能算是高明中最高明的。正如举得起秋毫之重的东西称不上是力大,能看见日月算不上是眼明,听见雷鸣算不上是耳聪。古代所谓善于用兵的人,只是战胜了那些容易战胜的敌人。所以,真正善于用兵的人取得胜利,并没有智慧过人的名声,也没有勇武杀敌的战功,是因为他在打胜仗时没有出现任何差错。不出现任何差错,其原因就在于他的措施能确保胜利,他所战胜的是已经处于失败地位的敌人。所以,善于打仗的人,能使自己处于不败之地,并且不会放过任何使敌人致败的时机。所以,打胜仗的军队,总是在先造成取胜条件之后才去交战,而打败仗的部队,总是先去交战而在战争中企图侥幸取胜。会用兵的人,善于修明政治并且遵循致胜的法度,所以能够成为胜败的主宰者。

\begin{yuanwen}
兵法:一曰度\footnote{长度。这里指国土面积的大小。一说“度”指忖度、判断。},二曰量\footnote{容量。这里指物产数量的多少。},三曰数\footnote{数量。这里指兵员的多少。},四曰称\footnote{权衡轻重。这里指力量的对比。},五曰胜\footnote{胜利。}。地生度,度生量,量生数,数生称,称生胜。故胜兵若以镒称铢\footnote{比喻力量相差很大。“镒y\`i”和“铢”都是古代的重量单位,一镒等于二十四两,一铢等于一两的二十四分之一。据出土战国衡器和记重铜器,镒与铢的比为1:576。},败兵若以铢称镒。胜者之战民\footnote{指挥士卒作战。}也,若决积水于千仞\footnote{比喻非常高的意思。仞,古代的长度单位,古人有“七尺一仞”、“八尺一仞”等几种说法。}之谿者,形\footnote{从本篇内容看,主要指军事实力。}也。
\end{yuanwen}

兵法上说:一是要估算土地的面积,二是要推算物产数量的多少,三是要统计兵员的数量,四是要比较双方的军事实力,五是要得出胜负的判断。有了土地就有了土地面积,有了土地面积就能推算出物产数量的多少,有了物产数量的多少就能决定投人兵员的数目,有了投入兵员的数目就能比较双方的军事实力,知道了双方的军事实力就能得出胜负的判断。所以获胜的军队对于失败的一方就如同用“镒”来称“铢”,具有绝对优势,而失败的军队对于获胜的一方就如同用“铢”来称“镒”,处于绝对劣势。胜利者指挥军队打仗,就像从千仞高的山润中放泻积水,其势猛不可挡,这就是军事实力的表现。
	
\chapter{(兵)势篇第五}
	
“势”即《汉书·艺文志·兵书略》中记载的任宏当年论次兵书中的第二种“形势”之“势”,即指人为造成的一种事态。本篇主要论述了“势”的形成和利用以及“势”和作战的关系等问题。“势”是以“奇正”之术(兵力的战术配置)为主要内容的,要正确运用“奇正”的变化,以出奇制胜;此外就是要“择人而任势”,即强调充分发挥将帅杰出的指挥才能。孙武认为,一个人聪明的将帅应随着情况的变化而改变奇正的战法,犹如天地一样变化无穷、江河一样奔流不竭,总是善出奇兵,打败敌人。
	
\begin{yuanwen}
孙子曰:凡治\footnote{管理,治理。}众如治寡,分数\footnote{指军队的组织编制。}是也;斗众如斗寡,形名\footnote{本指事物的形体和名称,是先秦时形名家(也叫名家)的术语,但也被当时的兵家和法家所采用。这里的“形名”泛指指挥军队作战的工具及联络信号,如金、鼓、旌、旗之类。}是也;三军之众,可使毕受敌\footnote{四面受敌。“毕”原作“必”,误,据王晳注及竹简本改。}而无败者,奇正\footnote{古代军队作战的方法、奇ji1,指变化无端、出敌不意的作战方法。正,指正规的和一般的作战方法。}是也;兵之所加,如以碫\footnote{磨刀石,这里泛指石头。}投卵\footnote{鸡蛋。}\footnote{用石头投向鸡蛋,比喻实力强的军队进攻实力弱的军队就如同用石头砸鸡蛋一样容易。}者,虚实\footnote{指兵力的集中和分散。一说指兵力的强弱。}是也。
\end{yuanwen}

孙子说:要做到治理人数多的军队就像治理人数少的军队一样,关键是组织、编制的问题;指挥人数多的军队作战就像指挥人数少的军队作战一样,关键是指挥作战用的工具及联络信号问题;三军将士四面受敌而不会失败,关键是正确运用“奇正”变化的问题;向敌军发起进攻,如同用石头砸鸡蛋一样容易,关键是以实击虚的问题。

\begin{yuanwen}
凡战者,以正合\footnote{会合交战。},以奇胜。故善出奇者,无穷如天地,不竭\footnote{枯竭。}如江河。终而复始,日月是也。死而复生,四时是也。声\footnote{古代以宫、商、角、徵、羽五个基本音阶为五声。}不过五,五声之变,不可胜\footnote{尽。}听也;色\footnote{古代以青、赤、黄、白、黑五种颜色为正色。}不过五,五色之变,不可胜观也;味\footnote{古代以酸、甜、苦、辣、咸五种基本味道为正味。}不过五,五味之变,不可胜尝也;战势\footnote{这里指战略、战术的态势。}不过奇正,奇正之变,不可胜穷也。奇正相生,如循环之无端\footnote{即无头无尾。},孰能穷之?
\end{yuanwen}

一般作战,都是以“正”兵会合交战,而用“奇”兵出奇制胜的。所以善于运用奇兵的人,其战术的变化就像天地变化一样无穷无尽,像江河一样永不枯竭。终而复始,像日月起落运行一样;死而复生,像四季更迭往复无穷一样。声音不过宫、商、角、徵、羽五个音阶,然而用五种音阶变化组合,就能产生出永远也听不完的乐音来;颜色不过红、黄、蓝、白、黑五种,然而用五种色调变化组合,就能产生出永远看不完的色彩来;味道不过酸、甜、苦、辣、咸五种,然而用五种味道变化调合,就能产生出永远也尝不完的味道来。战争中的势态不过“奇”、“正”两种,然而用“奇”、“正”变化组合,就能产生出变化无穷的战略战术奇正可以相互转化,就好比顺着圆环旋转,永无尽头,谁能够穷尽它呢?

\begin{yuanwen}
激水之疾\footnote{急速。},至于漂石者,势也;鸷鸟\footnote{zh\`i,指凶猛的鸟。}之疾,至于毁折者,节\footnote{节奏。}也。故善战者,其势险,其节短。势如彍弩\footnote{gu\=o,把弓拉满。},节如发机\footnote{扣动弩机。}。
\end{yuanwen}

湍急的流水飞快地奔泻,以至于能漂动大石,这是借助了强大的水势;鸷鸟飞速凶猛,以致能迅速捕杀鸟兽,是因为它掌握了急促的节奏。所以善于作战的指挥者,他所造成的态势是险峻的,进攻的节奏是短促的。所造成的“势险”就如同拉满弓弩一样险峻,进攻的“节奏”就如同搏动弩机那样突然。
	
\begin{yuanwen}
纷纷纭纭\footnote{这里形容旗帜纷杂混乱。},斗乱\footnote{指战斗混乱。}而不可乱也;浑浑沌沌\footnote{这里形容战车转动,人马奔驰。},形圆\footnote{指阵型是圆形的,即圆阵。四面八方都能应付自如。}而不可败也。乱生于治,怯生于勇,弱生于强。治乱,数\footnote{即上文的“分数”,指军队的编制。}也;勇怯,势\footnote{指军事态势。}也;强弱,形也。故善动\footnote{调动。}敌者,形\footnote{指以假象欺骗敌人。}之,敌必从之;予\footnote{给予。}之,敌必取之。以利动之,以卒\footnote{这里指重兵。}待之。
\end{yuanwen}

旌旗纷杂,局势混乱,但自己组织指挥的军队不紊乱;战车转动,人马奔驰,但自己组织指挥的军队应付自如,立于不败。在一定的条件下,“乱”可以由“治”产生,“怯”可以由“勇”产生,“弱”可以由“强”产生。“治乱”是组织编制的问题;“勇怯”是势态优劣的问题;“强弱”是力量大小的问题。所以善于调动敌军的人,向敌军示以假象,敌军一定会为其所骗;给敌军一点好处,敌军一定会为其所诱。用小利引诱敌军,部署重兵来严阵以待。

\begin{yuanwen}
故善战者,求之于势,不责\footnote{这里为苛求的意思。}于人,故能择人而任势。任势者,其战人\footnote{指挥士卒作战。}也如转木石。木石之性,安则静,危\footnote{危险。这里指地势倾斜。}则动,方则止,圆则行。故善战人之势,如转圆石于千仞之山者,势\footnote{即《势篇》之势,指将帅在指挥作战时所造成的有利态势。}也。
\end{yuanwen}

所以善于指挥作战的人追求的是有利的“势”,而不是去苛求士兵,因此能选择人才去利用已形成的“势”善于利用“势”的将领,其指挥部队作战就像转动木头和石头。木头石头的特性是处于平坦地势上就静止不动,处于地势倾斜的地方就滚动,方形的就容易静止,圆形的就容易滚动。所以,善于指挥作战的人所造就的“势”,就像从很高的山上把圆石滚下来一样,来势凶猛,不可阻挡。这就是军事上所谓的“势”。
	
\chapter{虚实篇第六}
	
“虚”是指兵力相对分散而薄弱,“实”是指兵力相对集中而强大,“虚实”就是指在战场上通过分散、集中兵力的战术变化来造成我强敌若(“我专而敌分”、“我众敌寡”)的战势来战胜敌人。本篇是《势》篇中“任势”战略思想的一种发挥。在本篇中孙武主要论述了“善战者致人而不致于人”,善于迷惑、调动、分散敌人,“避实而就虚”、“因敌而致胜”的作战指导思想。文中阐述了“先发制人,以逸待劳”的观点,强调避实而击虚,最后孙武指出“兵无常势,水无常形”,“能因敌变化而取胜者谓之神”,这就告诉我们,一个善于指挥作战的将领在战场上必须根据敌我双方的具体情况来确定作战的战略战术,这样才能做到“百战不殆”,成为战场上的战神。	
	
\begin{yuanwen}
孙子曰:凡先处\footnote{处,居止。}\footnote{这里是先期到达、占据的意思。}战地而待敌者佚\footnote{安逸。这里是从容的意思。},后处战地而趋\footnote{疾行。}战\footnote{仓促应战。}者劳\footnote{疲劳。这里也有被动的意思。}。故善战者,致人\footnote{这里指调动敌人。}而不致于人。能使敌人自至者,利之也;能使敌人不得至者,害\footnote{妨害。这里也有阻挠的意思。}之也。故敌佚能劳之,饱能饥之,安能动之。
\end{yuanwen}

孙子说:凡先到达并占据战地等待敌人的就主动从容,而后期到达战地匆忙投人战斗的就被动疲劳。所以善于指挥作战的人,能调动敌人而不被敌人所调动。能使敌人自动进人预定地域的,是用小利引诱的结果;能使敌人不能到达其预定地域的,是阻挠的结果。当敌人休整较好时,能设法使之疲劳;当敌人给养充足时,能设法使之饥饿;当敌人安稳不动时,能设法使他行动起来。

\begin{yuanwen}
出其所必趋\footnote{急行,奔赴。}\footnote{原作“不趋”,据银雀山汉简改正。},趋其所不意\footnote{出其不意。}。行千里而不劳者,行于无人之地也。攻而必取者,攻其所不守也。守而必固者,守其所不攻也。故善攻者,敌不知其所守;善守者,敌不知其所攻。微乎微乎,至于无形;神乎神乎,至于无声,故能为敌之司命\footnote{命运的主宰者。}。
\end{yuanwen}

向敌人不设防的地区进军,急速到达他预料不到的地点攻击。行军千里而不疲惫的原因,是因为走在敌军无人抵抗或没有设防的地区。进攻就一定会获胜的原因,是因为攻击的是敌人不设防的地方。防守一定会稳固的原因,是因为守住了敌人不进攻的地方。所以善于进攻的人,能使敌人不知道怎样防守。而善于防守的人:能使敌人不知道从哪里或怎样进攻。微妙啊微妙,竟然见不到一点形迹;神奇啊神奇,居然漏不出一点消息。这样,就能成为敌人命运的主宰者。
	
\begin{yuanwen}
进而不可御\footnote{抵御,防御。}者,冲其虚也;退而不可追者,速而不可及\footnote{到。这里是说追上的意思。}也。故我欲战,敌虽高垒深沟,不得不与我战者,攻其所必救也。我不欲战,画地而守之,敌不得与我战者,乖\footnote{背离,违背。}其所之也。
\end{yuanwen}

进攻时敌人无法抵御,是由于攻击了敌人兵力空虚的地方;撤退时敌人无法追击,是由于行动迅速而敌人无法追上。所以我军要想交战,哪怕敌军有高垒深沟,也不得不出来与我军交战的原因,是因为我军攻击了它非救不可的要害之处。我军不想与敌军交战,哪怕是在地上画界防守,敌军也无法与我军交战的原因,是因为我军已改变了敌军的进攻方向。
	
\begin{yuanwen}
故形人\footnote{这里指诱使敌人暴露形迹。}而我无形,则我专\footnote{专一。这里指集中。}而敌分\footnote{分散。}。我专为一,敌分为十,是以十攻其一也,则我众敌寡。能以众击寡者,则吾之所与战者约\footnote{少}矣。吾所与战之地不可知,不可知则敌所备者多。敌所备者多,则吾所与战者寡矣。故备前则后寡,备后则前寡,备左则右寡,备右则左寡,无所不备,则无所不寡。寡者,备人者也;众者,使人备己者也。
\end{yuanwen}

所以,诱使敌军暴露形迹而我军处于隐蔽状态,这样我军的兵力就可以集中而敌军的兵力就会分散。我把兵力集中在一点,而敌人分散在十处,就相当于我军以十信的兵力攻击敌人,这样就出现我众敌寡的态势。能做到以众击寡,是因为与我军直接交战的敌军少了的缘故,我军所设定的战场在哪里,敌军不知道,敌军不知道就会处处分兵设防。敌军所防备的地方越多,那么和我军交战的敌军就会越少。所以防备了前面,后面的兵力就不足;防备了后面,前面的兵力就不足;防备了左边,右边的兵力就不足;防备了右边,左边的兵力就不足;处处都防备,就处处兵力不足。之所以兵力薄弱,就是因为处处去防备别人;之所以兵力充足,就是因为迫使敌人处处防备自己。

\begin{yuanwen}
故知战之地,知战之日,则可千里而会战。不知战地,不知战日,则左不能救右,右不能救左,前不能救后,后不能救前,而况远者数十里,近者数里乎?以吾度\footnote{du\'o,推测判断。}之,越人\footnote{即越国人。春秋时越国和吴国经常相互征伐,孙武经常为吴王讲论兵法。}之兵虽多,亦奚\footnote{疑问词,相当于“何”的意思。}益于胜哉?故曰:胜可为也。敌虽众,可使无斗\footnote{没有战斗力,无法战斗。}。
\end{yuanwen}

所以,能预知同敌人交战的地点,又能预知同敌人交战的时间,即使行军千里也可以与敌人交战;如果不能预知与敌人交战的地点,又不能预知与敌人交战的时间,仓促遇敌,那就会左翼不能救右翼,右翼不能救左翼,前面不能救后面,后面不能救前面,更何况远的有数十里,近的也有好几里呢?依我的推测判断,越国的军队虽然很多,但对他的取胜又有什么帮助呢?所以说:胜利是可以创造的,敌人虽然兵多,却可以使其没有战斗力,无法参与战斗。

\begin{yuanwen}
故策\footnote{指分析判断。}之而知得失之计,作\footnote{指诱使敌人行动。}之而知动静之理,形\footnote{指陈师布阵的态势。}之而知死生之地,角\footnote{比较。这里指试探性的进攻。}之而知有余不足之处。故形兵之极\footnote{最高境界。},至于无形。无形,则深间\footnote{间谍。}不能窥\footnote{偷看。},智者不能谋。因形而错胜\footnote{制胜。错,通“措”,放置。}于众,众不能知;人皆知我所以胜之形,而莫知吾所以制胜之形。故其战胜不复\footnote{不重复。},而应形于无穷。
\end{yuanwen}

所以通过分析可以判断敌人作战计划的优劣得失诱使敌人行动可以了解敌方的动静规律,通过陈师布阵就可以知道地形是生地还是死地,通过试探性进攻就可以探明敌人兵力布置的强弱多寡。所以,陈师布阵的方法运用得极其巧妙时,可以达到一点形迹也没有。如果一点形迹也不暴露,即使隐藏再深的间谍也不能探明我的真实情况,足智多谋的人也想不出对付我的办法。即使把根据敌情采取策略而取得的胜利,摆在众人面前,人们也不知道是怎样打胜的。人们都知道我之所以克敌制胜的方法,却不知道我是怎样运用这些方法制胜的。所以每次战胜都不是重复前一次的方法,而是根据不同的情况采取了变化无穷的战术。

\begin{yuanwen}
夫兵形象水,水之形,避高而趋下。兵之形,避实而击虚。水因地而制流,兵因敌而制胜。故兵无常势\footnote{固定不变的常态。},水无常形。能因敌变化而取胜者,谓之神\footnote{这里指用兵如神。}。故五行\footnote{金、木、水、火、土。古人认为五行“相生相胜”。这种相生相克的结果就是没有一个常能胜的。}无常胜,四时无常位,日有短长,月有死生\footnote{指月有盈亏。古人叫“生霸”、“死霸”。“生霸”是指月亮有光明。“死霸”是指月亮的光明由明转晦。}。
\end{yuanwen}

作战的方式方法就像流水一样,水流动时是避开高处而向低处流,作战取胜的方法是避开设防坚实之处而攻其薄弱的环节。水根据地势来决定流向,军队根据敌情来采取制胜的方略。所以用兵作战没有一成不变的态势,就像流水没有固定的形状和去向一样。能够根据敌情的变化而取胜的,就称得上用兵如神。所以,用兵作战就像金、木、水、火、土相生相克,没有哪个常胜;春、夏、秋、冬四季依次交替,没有哪个季节固定不移,白天有短有长,月亮有缺有圆,永远处于变化之中。
	
\chapter{军争篇第七}
	
“军争”就是指作战的双方争夺取胜的有利条件,也就是争夺战场上的主动权。孙武认为要想夺得战场上的主动权,必须“先知迁直之计”,就是故意迂回绕道,并用小利把敌人引诱到别的方向去,然后倍道兼行,出敌不意。“军争”有“利”、“危”两种情况,如果正确应用“迂直之计”就会争取到有利的作战条件;如果不能正确应用“迂直之计”,不但不能争取到有利的作战条件,还会反受其“危”。孙武“迂直之计”里的“以患为利”、“后人发,先人至”的战术,也是孙武“军争”思想的精华。迂与直,后与先,患与利,本来是矛盾的双方,但在一定的条件下是可以相互转化的。此外,孙武在本篇里还提出“避其锐气,击其惰归”等著名的作战原则。
	
\begin{yuanwen}
孙子曰:凡用兵之法,将受命于君,合军聚众,交和而舍\footnote{交和,营垒之门相对。和,即和门,也叫军门、垒门。}\footnote{舍,驻扎。}\footnote{指两军对峙驻扎。},莫难于军争\footnote{两军争夺制胜的条件。}。军争之难者,以迂\footnote{迂回曲折。}为直\footnote{指直道。}\footnote{通过迂回曲折的弯路达到近直的目的。},以患\footnote{祸患,不利。}为利。故迂其途而诱之以利,后人发,先人至,此知迂直之计者也。
\end{yuanwen}

孙子说:大凡用兵的方法,从将帅接受国君的命令,征集民众、组织军队,到同敌人对阵,这中间,没有比率先争得制胜的先机更难的事了。要想争得制胜的先机,最困难的莫过于把迁回曲折的弯路变为捷径,把不利的条件转变为有利的条件。所以用迁回绕道的方法,再用小利引诱敌人,这样就能比敌人后出发而抢先占领阵地,这就是懂得了“以迂为直”的道理。

\begin{yuanwen}
故军争为\footnote{这里作“有”、“是”讲。}利\footnote{指有利的一面。},军争为危。举军\footnote{全军。}而争利,则不及;委\footnote{丢弃,抛弃。}军而争利,则辎重\footnote{指粮秣、军械等军需物资。}捐\footnote{损失,丢弃。}。是故卷甲\footnote{铠甲。}而趋,日夜不处\footnote{停止,休息。},倍\footnote{加倍。}道兼行,百里而争利,则擒\footnote{这里指被敌所擒。}三军将\footnote{三军之帅皆可称将军。},劲者\footnote{健壮的士卒。}先,疲者后,其法十一而至。五十里而争利,则蹶\footnote{ju\'e,挫折,失败。}上将军\footnote{前军将领。},其法半至。三十里而争利,则三分之二至。是故军无辎重则亡,无粮食则亡,无委积\footnote{指军需物资储备。}则亡。
\end{yuanwen}

军争有有利的一面,军争也有危险的一面。如果全军带着所有辎重去争利,就不能及时到达预定位置;如果丢弃辎重轻装去争利,装备辎重就会损失。所以卷甲快速前进,日夜不息,加倍地急行军,奔跑百里去争利,三军的将领就可能会被俘获。强壮的士兵先到达,疲弱的士兵掉了队,采用这种方法,只会有十分之一的兵力赶到;走五十里去争利,上军的将领会受挫折,其结果只会有半数的兵力赶到;走三十里去争利,其结果也只会有三分之二的兵力赶到。因此,部队没有辎重就会失败,没有粮食供应就不能生存,没有物资储备就无法坚持作战。

\begin{yuanwen}
故不知诸侯之谋者,不能豫交\footnote{和诸侯结交。豫为“预”的本字,参与。};不知山林、险阻\footnote{指山水险要阻隔的地方。}、沮泽\footnote{j\`u,指水草丛生的沼泽地。}之形者,不能行军;不用乡导\footnote{指向导,指给军队带路的人。乡,通“向”。}者,不能得地利。
\end{yuanwen}

所以,不了解诸侯列国的战略企图,就不能与之结交;不熟悉山林、险阻、沼泽等地形,就不能行军;不使用向导,就不能有效地利用有利的地形。
	
\begin{yuanwen}
故兵以诈立\footnote{靠诡诈而存在。这里“诈”也有变化多端的意思。},以利动,以分合为变\footnote{指作战时军队的集中与分散的变化。}者也。故其疾\footnote{快速。}如风,其徐\footnote{缓慢。}如林,侵掠如火,不动如山,难知如阴,动如雷震。掠乡\footnote{古代地方行政组织。}分众,廓\footnote{扩大。}地分利,悬权\footnote{悬挂秤锤,称量东西。这里指要权衡利害得失。}而动。先知迂直之计者胜,此军争之法也。
\end{yuanwen}

所以,用兵作战要善于用“诈”,要根据是否有利来采取行动,要根据双方情势或分散或集中。所以,军队行动迅速时就像狂风骤至,行动舒缓时就像严整的树林,攻击敌人时就像燎原的烈火,按兵不动时就像巍然屹立的山岳,隐蔽时就像阴天看不清日月星辰,行动时就像雷霆万钧之势。要掠取敌人的作战物资并分其民众,要扩张地域并分兵把守要地,要权衡利害得失并相机而动。谁先懂得以迁为直的方法,谁就能取得胜利。这就是军争的法则。

\begin{yuanwen}
《军政》\footnote{古代的兵书。}曰:“言不相闻,故为金鼓\footnote{古代夜战时用来指挥作战、传递信号的工具。据《周礼》,“鼓人”掌六鼓四金之音声。六鼓指雷鼓、灵鼓、路鼓、鼖鼓、鼛鼓、晋鼓。四金指錞、镯、铙、铎。古代作战,鼓以作气,金以抑怒。鼓法有五:一持兵,二结阵,三行,四背,五急背。};视不相见,故为旌旗\footnote{泛指指挥作战用的各种旗帜。旗法有五:一赤南方,二玄北方,三青东方,四白西方,五黄中央。}。”夫金鼓旌旗者,所以一\footnote{统一。}人之耳目也。人既专一,则勇者不得独进,怯者不得独退,此用众之法也。故夜战多火鼓\footnote{当为“金鼓”之误,《武经》作“金鼓”,银雀山汉简作“鼓金”。},昼战多旌旗\footnote{银雀山汉简《孙子兵法》作“昼战多旌旗,夜战多金鼓。”},所以变人之耳目也。
\end{yuanwen}

《军政》里说:“作战时用话语传递指挥信息是难以听到的,所以设置了金鼓来充当指挥作战、传递信号的工具;作战时用动作来指挥会看不清或看不见,所以用旌旗来指挥作战。”金鼓和旌旗都是用来统一士兵的视听、统一作战行动的。全军的士兵都能服从统一指挥,那么勇敢的将士就不会单独冒进,胆怯的将士也不会独自退却这就是指挥大军作战的方法。所以,夜间作战多用金鼓指挥,白天打仗多用旌旗指挥。这些是用来适应士卒的视听的。

\begin{yuanwen}
故三军可夺\footnote{剥夺,这里指打击、动摇、挫伤。}气\footnote{挫伤士气。},将军可夺心\footnote{动摇决心。}。是故朝\footnote{造成。}气锐\footnote{这里指气盛。}\footnote{早晨士气饱满,锐不可当。},昼气惰\footnote{懒惰,懈怠。},暮气归\footnote{这里指气竭。}。故善用兵者,避其锐气,击其惰归,此治气\footnote{这里指掌握士气。}者也。以治待乱,以静待哗,此治心者也。以近待远,以佚待劳,以饱待饥,此治力者也。无邀\footnote{截击。}正正之旗,勿击堂堂之陈\footnote{zh\`en,通“阵”,这里指阵容、阵势。},此治变者也。
\end{yuanwen}

所以说,可以挫伤三军士卒的锐气,可以动摇其将军的决心。军队初战时,士气饱满,锐不可当;过了一段时间之后,士气就会低落;到了后期,士气就会衰竭。所以,善于用兵的人,总是避开敌人的锐气,趁其士气低落衰竭时就发起猛攻,这就是正确掌握士气的方法。用严整来对待敌人的混乱,用镇定沉着来对待敌人的躁动喧哗,这就是正确掌握军心的方法。用自己靠近的战场来等待远道跋涉的敌人,用自己的从容休整来等待疲惫不堪的敌人,用自己的粮足食饱来等待粮尽人饥的敌人,这就是正确掌握军力的方法。不要去迎击旗帜整齐、部伍统一的军队,不要去攻击阵容整肃、士气饱满的军队,这是正确掌握随机应变的方法。

\begin{yuanwen}
故用兵之法,高陵\footnote{山陵。}勿向\footnote{指从上向下仰攻。},背\footnote{背靠,背依。}丘勿逆\footnote{指迎面进攻。},佯\footnote{假装。}北\footnote{败北。}勿从\footnote{跟从,跟踪。},锐卒\footnote{指锐气强盛的士卒。}勿攻,饵兵\footnote{指引诱我军的敌军。}勿食\footnote{这里是吃掉、消灭的意思。},归师勿遏\footnote{阻止,阻拦。},围师必阙\footnote{通“缺”,空缺。}\footnote{古人认为陷于包围之中的士兵必将作困兽之斗,因此在包围敌人时一定要留出放生的缺口,以减少伤亡。},穷寇\footnote{穷途末路的残敌。}勿迫,此用兵之法也。
\end{yuanwen}

所以,用兵的方法是:敌军占领山地时不要仰攻,敌军背靠高地时不要正面迎击,对于假装败退的敌人不要跟踪追击,对于敌人的精锐部队不要去强攻,对于敌人的诱兵不要去消灭,对于撒退的部队不要去阻截,包围敌军时一定要留出缺口,对于陷人绝境的敌人不要过分逼迫这些都是用兵的基本原则。
	
\chapter{九变篇第八}
	
“九变”一词历来有多种解释,李零先生认为实即《九地》篇里的“九地之变”,即“散地”“轻地”“争地”“交地”“衢地”“重地”“圮地”“围地”“死地”之变,本篇里的“绝地”也见于《九地》篇。孙武认为“九地之变,屈伸之利,人情之理,不可不察”(见《九地》篇),意即九种地形的应变处置、攻防进退的利害得失、全军上下的心理状态,这些都是不能不认真研究和周密考察的。在本篇中作者主要论述了指挥作战应根据不同的地形、敌情,采取灵活机动的处置原则,见利思害,见害思利,从利害两个方面综合考虑问题,并提出了有备无患的备战思想。	
	
\begin{yuanwen}
孙子曰:凡用兵之法,将受命于君\footnote{古代将军受命于君时举行隆重的仪式。如《淮南子·兵略训》里说:“凡国有难,君自宫召将,诏之曰:'社稷之命在将军,即今国有难,愿请子将而应之。'将军受命,乃令祝史太卜斋宿三日,之太庙,钻灵龟,卜吉日,以受鼓旗。”是古代将军受命于君的过程。},合军聚众。圮地无舍\footnote{圮地当为“氾f\`an地”之误,银雀山竹简本《九地篇》作“泛地”,“泛”通“氾”。氾地,就是指山林、险阻、沮泽等难行的道路。},衢地交合,绝地无留,围地则谋,死地则战,涂\footnote{通“途”,道路。}有所不由\footnote{经过。},军有所不击,城有所不攻,地有所不争,君命有所不受\footnote{接受。这里也有不执行的意思。}。
\end{yuanwen}

孙子说:大凡用兵的方法,将帅接受国君的命令,征集兵员组建军队,在山林、险阻和水草杂生的地方不要扎营驻军,在四通八达的地区要与四邻结交,在与后方隔绝、难以生存的地区不要停留,在被包围的地方要巧于计谋,在死地就要决一死战。有的道路可以不从那里经过,有些敌军不可以攻击,有些城池不可以占领,有些地域不可以争夺,君主的命令有时可以不接受。

\begin{yuanwen}
故将通于九变之地利者,知用兵矣;将不通于九变之利者,虽知地形,不能得地之利矣。治兵不知九变之术,虽知五利,不能得人之用矣。
\end{yuanwen}

所以,将帅能够精通“九变”的具体运用,可以说是懂得用兵了。如果将帅不精通“九变”的具体运用,即使了解地形,也不能得地利。治兵却不懂得“九变”的方法,虽然知道“五利”,也不能充分发挥军队的作用。

\begin{yuanwen}
是故智者\footnote{指明智的将帅。}之虑,必杂\footnote{掺杂,这里有兼顾的意思。}于利害。杂于利而务\footnote{指战斗任务。}可信\footnote{通“伸”,这里是顺利发展的意思。}也,杂于害而患可解\footnote{解除,免除。}也。
\end{yuanwen}

明智的将帅考虑问题时,一定要兼顾到利、害关系。在有利的情况下考虑不利的因素,事情就能顺利发展,在不利的情况下考虑有利的因素,祸患就可以排除。
	
\begin{yuanwen}
是故屈\footnote{屈服。这里是使动用法。}诸侯者以害\footnote{这里指害怕、忌讳、厌恶的事。},役\footnote{役使。}诸侯者以业\footnote{事业,这里指自己的实力。},趋\footnote{这里是归附的意思。}诸侯者以利。
\end{yuanwen}

因此,要通过“害”来迫使诸侯屈服,要通过自己的实力来役使诸侯忙乱,要通过“利”来诱使诸侯归附。
	
\begin{yuanwen}
故用兵之法,无恃\footnote{依靠,依赖。这里也有希望的意思。}其不来,恃吾有以待之;无恃其不攻,恃吾有所不可攻也。
\end{yuanwen}

所以,用兵的方法是:不要寄希望于敌人不来,而要依靠我方有充分的准备,严阵以待;不要寄希望于敌人不进攻,而要依靠自己有使敌人无法攻破的力量。

\begin{yuanwen}
故将有五危:必死\footnote{这里指有勇无谋,只知死拼。},可杀\footnote{就可能被杀。}也;必生\footnote{指贪生怕死。},可虏\footnote{可能被俘虏。}也;忿速\footnote{速,急躁。}\footnote{指愤怒急躁。},可侮\footnote{可能被欺侮。}也;廉洁\footnote{这里指廉洁好名,过于自尊。},可辱\footnote{可能被羞辱。}也;爱民,可烦也。凡此五者,将之过也,用兵之灾也。覆军杀将,必以五危,不可不察也。
\end{yuanwen}

所以,将领有五种致命的危险:如果去死拼硬打,就可能招致杀身之祸;如果贪生怕死,就可能被俘虏;如果性情暴躁易怒,就可能受到敌人的轻每而妄动:如果廉洁好名,过于自尊,就可能会被羞辱而失去理智;如果爱民众,就可能导致烦扰而陷于被动。所有这五种情况,都是将领易犯的过失,给指挥用兵带来灾难。全军覆没,将领被杀,一定是因为这五种致命弱点引起的,因此不可不充分地去认真考察和了解。
	
\chapter{行军篇第九}

孙武在本篇中主要论述了行军作战、处置军队、判断敌情等几个问题。他在本篇中提出了三十三种观察、判断敌情的方法,只有通过这些方法,把侦察到的情况加以分析、判断,才能真正掌握敌情,才能真正制定出切实可行的作战方案从而取得最后的胜利。孙武在本篇中还提出了“令之以文,齐之以武”的文武兼用的治军原则:赏罚要适时适度,这也是孙武军事思想的一个重要组成部分。
	
\begin{yuanwen}
孙子曰:凡处军\footnote{指行军作战中对军队的处置。}相敌\footnote{指观察、判断敌情。},绝\footnote{《淮南子·时则》“自昆仑东绝两恒山”,注云:“绝,过也。”古代凡穿越山地或穿越其他地形都可称之为“绝”。}山依\footnote{靠近。}谷\footnote{通过山地时要靠近山谷。},视生处\footnote{居。}高\footnote{居高向阳。曹操注云:“生者,阳也。”作战的地形有“生地”和“死地”之分,“生地”即指向阳开阔的地方。},战隆\footnote{指高地。}无登\footnote{攀登。},此处山之军也。绝水\footnote{绝,《广雅·释诂二》:“绝,渡也。”绝水,横渡江河。}必远水;客\footnote{古代交战时,把进攻的一方称为“客”,防守的一方称为“主”。银雀山汉简《孙膑兵法》有《客主人分》篇。}绝水而来,勿迎之于水内,令半济\footnote{渡。}而击之\footnote{《吴子·料敌》篇也说:“涉水半渡,可击。”战国以来,“半济而击”已成为基本的战略战术,历史上以此取胜的战例也有很多。但也有不肯“半济而击”而招致败绩的,如《左传·僖公二十二年》记载的宋楚泓之战,因宋襄公坚持古代战法而招致失败,就是典型例子。},利;欲战者,无附\footnote{靠近,接近。}于水而迎\footnote{逆。}客;视生处高,无迎水流,此处水上之军也。绝斥\footnote{盐碱地。}泽\footnote{沼泽之地。},惟亟\footnote{急切,赶快。}去无留;若交军于斥泽之中,必依水草而背众树\footnote{古代兵家处军面向开阔,背有依托。斥泽之中没有高地可以依托,故背靠树林作为依托。},此处斥泽之军也。平陆\footnote{平原。}处易\footnote{平坦。}而右背高\footnote{古代兵家处军,前面和左侧要平坦开阔,后面和右侧面要有高险可依。阴阳家也认为左前(东南)为阳,右背(西北)为阴,应背阴向阳。},前死后生\footnote{前低后高。低地为死地,高地为生地。前与敌战,不战则死;后依高山,故称“后生”。},此处平陆之军也。凡此四军之利,黄帝之所以胜四帝也。
\end{yuanwen}

孙子说:凡是部署军队和判断敌情时,应该注意:通过山地时要靠近有水草的溪谷,应该在居高向阳的地方驻扎,敌人占领高地时不要仰攻,这是在山地部署军队的原则。横渡江河以后,应在远离水流的地方驻扎,敌人渡水来战,不要在江河中迎击,而要在敌人部分已渡、部分未渡时发起攻击,这样最为有利。如果要与敌人决战,不要紧靠水边迎敌,要居高向阳,不要面迎水流,这是在江河地带部署军队的原则。通过盐碱沼泽地带时,要迅速离开,不要停留;如果在盐碱沼泽地带与敌军相遇,那就一定要占据靠近水草而背靠树林的地方,这是在盐碱沼泽地带部署军队的原则。在平原地带要选择平坦的地方设营,而右侧要依托高地,前低后高,这是在平原地带部署军队的原则。以上四种处置军队的好处,就是黄帝之所以能战胜其他四帝的重要原因。

\begin{yuanwen}
凡军好\footnote{h\`ao,喜欢。}高而恶\footnote{厌恶,讨厌。}下,贵\footnote{这里是重视的意思。}阳而贱\footnote{轻视,这里有避开的意思。}阴,养生\footnote{指据有水草之利。}而处实\footnote{指依托高地而处。},军无百疾,是谓必胜。丘陵堤防,必处其阳而右背之。此兵之利,地之助\footnote{得到地形的辅助。}也。上雨,水沫至\footnote{银雀山竹简本《孙子兵法》作“上雨水,水流至”。“沫”字当为“流”字之误。},欲涉\footnote{徒步趟水。}者,待其定也。
\end{yuanwen}

大凡驻军都是喜欢干燥的高地,厌恶潮湿的洼地;重视向阳之处,避开阴暗之地;军队要驻扎在接近水草、地势较高的地方,将士百病不生,这样才会有战胜敌人的把握。在丘陵堤防地行军,一定要占领向阳的一面,并且背靠高地,这些对作战有利的条件,是得利于地形的辅助。上游下雨,洪水突至,如果要徒步涉水,应等待水势平稳以后再过河。

\begin{yuanwen}
凡地有绝涧\footnote{两山险峻,水流其间的地方。}、天井\footnote{四周高峻,中间低洼,形若深井的地方。}、天牢\footnote{三面绝壁,易进难出的地方。曹操注:深山所过若朦胧者为“天牢”。}、天罗\footnote{林深草茂,形若网罗,进出两难的地方。}、天陷\footnote{地势低洼,沼泽连绵,泥泞易陷的地方。}、天隙\footnote{地形狭窄如缝的地方。银雀山竹简本作“天郄”,“郄”同“郤”,“郤”与“隙”古为通假字。},必亟\footnote{急速。}去\footnote{离开。}之,勿近也。吾远之,敌近之;吾迎之,敌背之。军行有险阻、潢井\footnote{hu\'ang,指内涝积水,地势洼陷的地方。}、葭苇\footnote{芦苇,这里指长满芦苇的地方。}、山林翳荟\footnote{y\`ihu\`i,指草木长得很茂盛的山林地带。}者,必谨\footnote{仔细。}复\footnote{反复。}索\footnote{搜索。}之,此伏奸之所处也。
\end{yuanwen}

凡是遇到“绝涧”“天井”、“天牢”“天罗”“天陷”天隙”这几种地形时,一定要迅速离开,不要接近。我们应该远离这些地形,而让敌人去靠近它;我们应面向这些地形,而让敌人去背靠它。行军在山川险阻、芦苇从生的低洼地、草木茂盛的山林地区,一定要仔细反复地搜索:这些地方都可能是敌人埋设伏兵和隐伏奸细之处。

\begin{yuanwen}
敌近而静者,恃\footnote{依仗。}其险也;远而挑战者,欲人之进也;其所居易\footnote{平地。}者,利\footnote{指地利。}也。众树动者,来也;众草多障\footnote{指障碍物。}者,疑\footnote{疑惑。}也\footnote{曹操注云:结草为障,欲使我疑。};鸟起者,伏也;兽骇\footnote{惊骇。}者,覆\footnote{覆盖。}也\footnote{曹操注:敌广陈张翼,来覆我也。}。尘高而锐\footnote{尖。这里指尘土飞扬得高而尖。}者,车来也;卑而广者,徒来也;散而条达\footnote{条理通达。这里指尘土飞扬得散乱而细长。}者,樵采\footnote{打柴。}也;少而往来者,营军\footnote{设营驻军。}也。辞卑\footnote{言辞谦卑。}而益备\footnote{加紧备战。}者,进也;辞强而进驱者,退也;轻车先出居其侧者,陈\footnote{通“阵”。这里作动词将,指布阵。}也;无约而请和者,谋也;奔走而陈兵车者,期\footnote{也叫“期会”,指按照交合作战。}也;半进半退者,诱也。杖而立者,饥\footnote{饥饿。}也;汲\footnote{从井里打水。}而先饮者,渴也;见利而不进者,劳也;鸟集者,虚也;夜呼者,恐也;军扰者,将不重也;旌旗动者,乱也;吏怒者,倦也;粟马肉食,军无悬缻\footnote{f\v{o}u,同“缶”,陶制的炊具,这里泛指一切炊具。},不返其舍者,穷寇也;谆谆\footnote{zh\=un,恳切。}翕翕\footnote{x\=i,和顺。这里指士卒们在一起絮絮不休的低声议论。},徐\footnote{慢慢地。}与人言者,失众也;数赏\footnote{不断地奖赏。}者,窘\footnote{窘迫,没有办法。}也;数罚\footnote{不断地惩罚。}者,困\footnote{指陷入困境。}也;先暴而后畏其众者,不精\footnote{精明。}之至也;来委谢\footnote{指敌人派使者来委婉谢罪。古代相见,馈赠礼物叫“委质”。谢,告。}者,欲休息\footnote{这里是休兵息战的意思。}也。
\end{yuanwen}

敌人离我很近而很安静,是依仗它占领了险要的地形;敌人离我很远却向我挑战,是企图引诱我前进;敌人占据了平坦的地方,是因为它拥有了地利。许多树木枝叶摇动,是敌人隐蔽前来;草丛中设有许多遮障物,是敌人布下的疑阵;群鸟惊飞,是下面有埋伏;野兽骇奔,是敌人大举来偷袭;尘土飞扬高而尖,是敌人的战车向我驶来;尘土低而宽广,是敌人徒步向我开来;尘土飞扬散乱而细长,是敌人正在打柴;尘土少而时起时落,是敌人正在安营扎寨。敌人使者的言辞谦卑却又在加紧战备,是准备要进攻作战;敌人使者的言辞强硬而军队又做出前进姿态的,是准备要撤退;敌人的轻车先出动并且部署在两侧的,是在布置作战的阵势;敌人事先没有约定而突然前来讲和,是其中另有阴谋;敌人急速奔走并排列阵势,是准备按期与我交合作战;敌人若半进半退,是企图引诱我军。敌兵拄着兵器站立的,是饥饿的表现;负责供水的士兵打了水先自己喝的,是干渴的表现;敌人见利而不进兵争夺的,是疲劳的表现;敌人营寨上集聚鸟雀的,这表明军营是空营;敌人夜间有人惊叫的,是恐惧的表现;敌营浮动纷乱的,是敌将没有威严的表现;敌人的旗帜摇动不整齐的,是敌军队伍混乱的表现;敌军军更容易发怒的,是疲倦的表现;敌军用粮食喂马然后宰杀战马吃肉,收拾起一切炊具,不回营舍的,是陷入绝境的表现;士卒们絮絮不休地低声议论的,是敌将失去了人心的表现;不断地犒赏士卒的,是敌军一筹莫展的表现;不断地处罚部属的,是敌军陷人窘境的表现;敌军将帅先强暴而后又害怕部下的,是最不精明将领的表现;派使者来委婉谢罪的,是敌人想休兵息战的表现。

\begin{yuanwen}
兵怒而相迎,久而不合\footnote{指交战,交锋。},又不相去,必谨察之。
\end{yuanwen}

敌人盛怒而来,但总是不肯交锋,又不撤退的,遇到这种情况一定要谨慎地观察其意图。
	
\begin{yuanwen}
兵非益多也,惟\footnote{只要。}无武进\footnote{轻举妄动,盲目冒进。},足以并\footnote{集中。}力、料敌\footnote{判断敌情。}、取人\footnote{取胜于敌人。}而已。夫惟无虑而易敌\footnote{轻视敌人。}者,必擒于人。
\end{yuanwen}

用兵打仗并非兵力越多越好,只要不轻敌冒进,并集中兵力,判明敌情,就能取胜于敌人。只有那种无深谋远虑而又轻敌妄动的人,一定会被敌人所擒获。

\begin{yuanwen}
卒未亲附\footnote{亲近依附,真心拥戴。}而罚之则不服,不服则难用也。卒已亲附而罚不行,则不可用也。故令之以文,齐之以武\footnote{文指赏,武指罚。《管子·禁藏》:“赏诛为文武。”注云:“赏则文,诛则武。”曹操注云:“文,仁也;武,法也。”},是谓必取\footnote{必定取胜。}。令素\footnote{平素。}行\footnote{指平素就一贯执行。}以教其民,则民服;令素不行以教其民,则民不服。令素行者,与众相得\footnote{相投合。}也。
\end{yuanwen}

如果士卒还没有真心依附就惩罚他们,那么他们就会不服从,不服从就难以使用。如果士卒已经真心依附而惩罚仍不能执行,也不能用来作战。所以,要用奖赏来团结士卒,用军纪军法统一步调,这样的军队战之必胜,平素严格执行军令,并用奖赏来团结士卒,士卒就会服从;平素不严格执行军令,也不用奖赏来团结士卒,士卒就会不服从。平时军令能够彻底执行的,就能与士卒相处得非常融洽。
	
\chapter{地形篇第十}
	
本篇主要从“地形”的角度论述了军队在不同地形条件下的行动原则,并强调将帅要重视对地形的研究和利用。因为“地形”是战争中经常要遇到的客观条件,地形之高低险阻,战况之复杂多变,指挥战争的人不可不认真考虑。孙武在文中从不同的角度说明了地形与作战有着密切的关系,他明确地指出“地形者,兵之助也”,行军打仗如能“知彼知己”,则“胜乃不殆”;如能“知天知地”则“胜乃不穷”。

\begin{yuanwen}
孙子曰:地形有通\footnote{通达。这里指四通八达之地。}者,有挂\footnote{挂碍,牵阻。这里指易往难返之地。}者,有支\footnote{有相持之意。这里指谁先出发都不利之地。}者,有隘\footnote{指出口狭窄的地方。曹操注云:“隘形者,两山之间通谷也,敌势不得扰我也。”}者,有险者,有远者。我可以往,彼可以来,曰通;通形者,先居高阳,利粮道,以战则利。可以往,难以返,曰挂;挂形者,敌无备,出而胜之;敌若有备,出而不胜,难以返,不利。我出而不利,彼出而不利,曰支;支形者,敌虽利我,我无出也;引而去之,令敌半出而击之,利。隘形者,我先居之,必盈\footnote{满,堵的意思。}之以待敌;若敌先居之,盈而勿从,不盈而从之。险形\footnote{形势险要的地方。}者,我先居之,必居高阳以待敌;若敌先居之,引而去之,勿从也。远形\footnote{指敌我相距较远的地方。}者,势均\footnote{势均力敌。}难以挑战,战而不利。凡此六者,地之道\footnote{道理,原则。}也,将之至\footnote{极,最,重要。}任\footnote{责任。},不可不察也。
\end{yuanwen}

孙子说:地形有“通形”、“挂形”、“支形”、“隘形”、“险形”、“远形”等六种。凡是我们可以去、敌人也可以来的地域,叫做“通形”。在“通形”地域里,应抢先占领开阔向阳的高地,有利于粮道畅通,这样与敌人交战则利。凡是可以前进、难以返回的地域,称作“挂形”。在“挂形”的地域里,敌人如果没有防备,我们突击就能取胜。如果敌人有所防备,出击又不能取胜,就难以返回,这就对作战不利。凡是我军出击不利,敌人出击也不利的地域叫做“支形”。在“支形”的地域里,敌人虽然以利相诱我军,我们也不要出击。应该率军撤离,诱使敌人出击一半时再回师反击,这样才会对我有利。在“隘形”的地域里,我们应该抢先占领,一定要用重兵完全控制隘口,以等待敌人的到来;如果敌人抢先占据了隘口,并用重兵把守,我们就不要进击;如果敌人没有用重兵扼守隘口,那么就迅速攻取它。在“险形”的地域里,如果我军抢先占领,就一定要占据开阔向阳的高地来等待敌人;如果敌人抢先占领,就要率军主动撤离,不要去跟他交锋。在“远形”的地域里,敌我双方势均力敌,不宜挑战,如果勉强求战将对我不利。以上六点,是有效利用地形的原则。这是将帅们至关重要的责任,不可不认真考察研究。

\begin{yuanwen}
故兵\footnote{这里指打了败仗的士卒。}有走\footnote{败走。}者,有弛\footnote{废弛。}者,有陷\footnote{陷败。}者,有崩\footnote{崩溃。}者,有乱者,有北\footnote{败北,也是败退的意思。}者。凡此六者,非天之灾,将之过也。夫势均,以一击十,曰走。卒强吏弱,曰弛。吏强卒弱,曰陷。大吏\footnote{指比较小的军官。曹操注云:“大吏,小将也。”}怒而不服,遇敌怼\footnote{冤家对头的意思。}而自战,将不知其能,曰崩。将弱不严,教道\footnote{教育士卒的方法和原则。}不明,吏卒无常,陈兵纵横\footnote{这里指出兵作战 列阵,队形纵横},曰乱。将不能料敌\footnote{判断敌情。},以少合众,以弱击强,兵无选锋\footnote{指精选出有战斗力的士卒组成的先锋部队。},曰北。凡此六者,败之道也,将之至任,不可不察也。
\end{yuanwen}

所以军队打了败仗后有“走”、“弛”、“陷”“崩”、“乱”、“北”六种情况。这六种情况的发生,并不是由天灾所造成的,而是由将领的过失所致。双方势均力敌的情况下,以一击十而导致失败的,叫做“走”。士卒强悍而军官懦弱造成失败的,叫做“弛”。将领强悍而士卒懦弱造成失败的,叫做“陷”。副将怨怒而不服从指挥,遇到敌人擅自出战,主将又不了解他们的能力而加以控制,因此而造成失败的,叫做“崩”。将领懦弱又缺乏威严,治军没有章法,吏卒无所遵循,列兵布阵又杂乱无常,因此而造成失败的,叫做“乱”。将领不能正确判断敌情,以少击众,以弱击强,又没有精锐先锋部队,因而造成失败的,叫做“北”。以上六种情况,都是导致失败的原因。将领要负重大责任,是不可不认真去考察研究的。

\begin{yuanwen}
夫地形者,兵之助\footnote{辅助。}也。料敌制胜,计险厄\footnote{地势险要的地方。}远近,上将\footnote{大将,主将。}之道\footnote{指职责。}也。知此而用战者必胜,不知此而用战者必败。故战道\footnote{指战场上的实际情况。}必胜,主曰无战,必战可也;战道不胜,主曰必战,无战可也。故进不求名,退不避罪,唯民\footnote{唐代避李世民讳改“民”为“人”,宋代从之。}是保,而利合于主\footnote{指国君。},国之宝也。
\end{yuanwen}

地形,是用兵打仗的辅助条件。正确判断敌情而夺取胜利,考察地形的险易,计算道路的远近,这是上将的职责。知道这些然后去指挥作战的,必定能取得胜利,不了解这些而去指挥作战的,必定要失败。所以,掌握了战场情形就一定能取得胜利,即使君主说不要打,也可以坚持去打;根据战场情形不能取得胜利,即使君主说一定要打,也可以不去打。所以,作为一个将帅,应该进不企求战胜的名声,退不回避违命的责任,只求保全百姓,符合国君的利益,这样的将帅才是国家的宝贵人才。

\begin{yuanwen}
视\footnote{这里是看作、对待的意思。}卒如婴儿,故可与之赴深谿\footnote{很深的溪涧。泛指危险的地带。};视卒如爱子,故可与之俱死。厚\footnote{指待遇比较丰厚。}而不能使,爱\footnote{这里是贬义,指溺爱。}而不能令,乱而不能治,譬若骄子\footnote{娇生惯养的孩子。},不可用也。
\end{yuanwen}

对待士卒像体贴婴儿一样,士卒就可以跟随你共同赴汤蹈火;对待士卒像对待爱子一样,士卒就可以与你同生共死。但是,对士卒待遇比较丰厚而不能使用,对士卒溺爱而不能指挥,违反纪律而不能惩治,这就好像娇惯坏
的孩子一样,是不能用来打仗的。

\begin{yuanwen}
知吾卒之可以击,而不知敌之不可击,胜之半也;知敌之可击,而不知吾卒之不可以击,胜之半也;知敌之可击,知吾卒之可以击,而不知地形之不可以战,胜之半也。故知兵\footnote{指通晓用兵作战的人。}者,动而不迷,举而不穷。故曰:知彼知己,胜乃不殆\footnote{不会有危险。};知天知地,胜乃不穷。
\end{yuanwen}

只知道自己的部队可以进攻,而不知道敌人不可以进攻,取胜的可能只有一半;只了解敌人可以进攻,而不了解自己的部队不可以进攻,取胜的可能也只有一半。知道敌人可以进攻,也知道自己的部队可以进攻,然而不了解地形不利于作战,取胜的可能性仍然只有一半。所以通晓用兵作战的人,他行动起来决不会盲动,他的战略战术的变化是无穷的。所以说,了解对方,了解自己,争取胜利就不会有危险;懂得天时,懂得地利,胜利就不可穷尽了。
	
\chapter{九地篇第十一}

所谓“九地”就是指九种不同的作战地形,从内容上看,似乎是上篇《地形》的姊妹篇。在本篇中,孙武主要论述了在这九种地域作战的用兵原则,并强调要善于掌握官兵在不同作战地域的不同心理状态,以便采取相应的作战策略。孙武在本篇中提出了“兵之情主速,乘人之不及,由不虞之道,攻其所不戒也”,“齐勇若一”“刚柔皆得”,“携手若使一人”,“并敌一向,千里杀将”等作战原则,一直为古今中外的军事家所推崇。
	
\begin{yuanwen}
孙子曰:用兵之法,有散地,有轻地,有争地,有交地,有衢地,有重地,有圮地,有围地,有死地。诸侯自战其地者\footnote{下文各句有“者”字,文通意顺。此句缺,《魏武帝注孙子》本有“者”字,据补。},为散地\footnote{在自己的领土上作战叫“散地”。散,相对“专”而言,孙武认为士卒在本土作战则思土恋家,容易涣散,入敌境作战则思想专一。}。入人之地而不深者,为轻地\footnote{进入敌境不深的地方叫“轻地”。轻,相对“重”而言,孙武认为进入敌境不深,危急时士卒就会易于轻返。}。我得亦利,彼得亦利者,为争地\footnote{谁先占领谁就有利的必争之地。}。我可以往,彼可以来者,为交地\footnote{道路四通八达之地。}。诸侯之地三属\footnote{指敌我和其他邻国连接的地方。},先至而得天下众者,为衢地\footnote{与多国接壤之地。}。入人之地深,背城邑多者,为重地\footnote{深入敌境较深而且背后有很多城邑的地方。}。行山林、险阻、沮泽,凡难行之道者,为圮地\footnote{有山林、险阻、沮泽等难行之地。}。所由入者隘\footnote{狭窄。},所从归者迂\footnote{迂回曲折。},彼寡可以击吾之众者,为围地。疾战\footnote{指拼死作战。}则存,不疾战则亡者,为死地。是故散地则无战,轻地则无止\footnote{不能停留。},争地则无攻,交地则无绝\footnote{不要断绝联络,隔绝队伍。},衢地则合交\footnote{指结交邻国,搞好关系。},重地则掠\footnote{掠夺。这里是说进入敌境较深,容易造成给养不继,所以要掠夺敌国。},圮地则行,围地则谋,死地则战。
\end{yuanwen}

孙子说:根据用兵作战原则,战地在地理上有散地轻地、争地、交地、衢地、重地、圮地、围地、死地之分。在自己的领土上作战叫“散地”。进人敌境不深的地方叫“轻地”。我方占领有利,敌人占领也有利的地方叫“争地”。我军可以前往,敌军也可以前来的地方叫“交地”。诸候的国土与多国相毗邻,先到者就可以获得多方支援的地方叫“衢地”。深人敌境较深而且背后有很多城邑的地方叫“重地”。山林、险阻、沼泽等难于通行的地区叫“圮地”。进入的道路狭窄,退出的道路迂曲,敌人可以用少量兵力攻击我方众多兵力的地方叫“围地”。拼死奋战就能生存,不拼死奋战就会覆灭的地方叫“死地”。所以处于散地就不要作战;处于轻地就不要停留;遇争地应先敌占领,如果敌人已先占领,就不要强攻;处于交地就不要断绝联络;处于衢地就要结交邻国;处于重地就要掠取敌国;处于圮地就要迅速通过;陷人围地就要用计谋突围;处于死地就要迅猛奋战,死中求生。

\begin{yuanwen}
所谓古之善用兵者,能使敌人前后不相及\footnote{相互照顾。及,顾及。},众寡不相恃\footnote{互相依靠协同。},贵贱\footnote{身份高的和身份低的。这里指军官和士卒。}不相救,上下不相收\footnote{聚集,收拢,统属。},卒离而不集,兵合而不齐。合于利而动,不合于利而止。敢问:“敌众整而将来,待之若何?”曰:“先夺其所爱\footnote{这里指要害的、重要的,敌人所喜爱的地方。},则听矣。”兵之情\footnote{情理。}主\footnote{重在。}速,乘人之不及,由不虞\footnote{指料想不到的。}之道,攻其所不戒\footnote{戒备,防备。}也。
\end{yuanwen}

古代善于用兵作战的人,能使敌人前后部队无法相互顾及,主力和小部队无法相互依靠协同,军官和士卒之间无法相互救援,,上级和下级之间不能互相统属,士兵分散后不能集中,兵力集中了又不能整齐。对我有利就打对我无利就停止行动。试问:“如果敌人人多势众且又阵势严整前来与我作战,该用什么办法对付呢?”回答是:“先夺取敌人喜爱的要害之地就能使其就范。”用兵之理贵在神速,要乘敌人措手不及的时候,由敌人意料不到的道路前进,去攻击敌人没有戒备的地方。

\begin{yuanwen}
凡为客\footnote{客军。指离开本土进入别国作战的军队。}之道,深入则专\footnote{专心一志。}。主人不克\footnote{指被进攻的一方不能战胜进攻的一方。主,指在本土作战的一方。},掠于饶野\footnote{指富饶的乡村。},三军足食;谨养而勿劳,并气积力;运兵\footnote{指部署兵力。}计谋,为不可测。投\footnote{置。}之无所往,死且不北。死焉\footnote{文言疑问词,怎么。}不得,士人尽力。兵士甚陷则不惧,无所往则固\footnote{稳固,坚定。这里指军心稳定。},深入则拘\footnote{拘束。},不得已则斗。是故其兵不修\footnote{整治。}而戒\footnote{戒备。},不求而得,不约而亲,不令而信\footnote{信从。}。禁祥\footnote{妖祥,这里指占卜等活动。}去疑,至死无所之。
\end{yuanwen}

大凡进入敌国境内作战的原则是:深入敌境则军心就会专心一致,敌人就无法战胜我们。在敌国富饶的地区掠取粮草,三军就有了足够的给养。注意休养士卒,不要使其劳累,鼓足士气,积聚力量。部署兵力要巧设计谋,使敌人无法判断我军的意图。把部队置于无路可走的绝境,士卒就会拼死不败退。士卒既能拼死尽力,怎么会不取胜呢!士卒深陷险境,就无所畏惧,无路可走,军心就会稳固。士卒深人敌境就不易涣散,迫不得已就会殊死决战。因此,这样的军队不待整治督促就能加强戒备,不用强求就能执行任务,无须约束就能亲密团结,不用命令就能信守服从。禁止迷信活动,消除士卒的疑虑他们至死也不会逃避。
	
\begin{yuanwen}
吾士无余财,非恶货\footnote{厌恶财货。恶,厌恶,不喜欢。}也;无余命,非恶寿也。令发之日,士卒坐者涕\footnote{眼泪。}沾襟\footnote{衣襟。},偃\footnote{仰倒。}卧者涕交颐\footnote{面颊。}。投之无所往,诸\footnote{专诸,春秋末期吴国的刺客,曾为吴国的公子光(即阖庐)刺杀吴王僚。事见《史记·刺客列传》、《左传》昭公二十年、二十七年等书。}、刿\footnote{曹刿,又名曹沫,春秋时期鲁国的勇士,曾在齐鲁两国的盟会上劫持了齐桓公。}之勇也。
\end{yuanwen}

我军士卒没有多余的钱财,并不是他不喜欢钱财;士卒置生死于度外,并不是他们不想长寿。当作战命令下达的时候,坐着的士卒泪沾衣襟,躺着的士卒泪流满面把士卒置于无路可走的绝境,他们就都会像专诸、曹刿一样勇敢。

\begin{yuanwen}
故善用兵者,譬如率然\footnote{古代传说中的一种蛇。《神异经·西荒经》里说:“西方山中有蛇,头尾差大,有色五彩,人物触之者,中头则尾至,中尾则头至,中腰则头尾并至,名曰率然。”};率然者,常山\footnote{恒山,避汉文帝刘恒讳,改“恒”为“常”。在今河北省曲阳县西北与山西接壤处。}之蛇也。击其首则尾至,击其尾则首至,击其中则首尾俱至。敢问:“兵可使如率然乎?”曰:“可。”夫吴人与越人相恶也,当其同舟而济\footnote{渡。},遇风,其相救也如左右手。是故方\footnote{并。}马埋轮,未足恃\footnote{依靠}也\footnote{曹操注云:“方马,缚马也;埋轮,示不动也。”};齐勇若一,政\footnote{这里指御兵之术。}之道也;刚柔\footnote{强弱。}皆得,地之理也。故善用兵者,携手若使一人,不得已也。
\end{yuanwen}

所以善于用兵作战的人,能使部队像“率然”蛇一样。“率然”是常山上的一种蛇,打它的头部,尾巴就来救应;打它的尾部,头部就来救应;打它的腰部,头尾都来救应。试问:“用兵作战可以使军队像'率然’一样吗?”答曰:“可以。”吴国人和越国人虽然互相仇视,但当他们同舟共济而遇上大风时,他们之间相互救援就如同人的左右手一样。所以,把马并列地缚在一起,把车轮埋起来,以此来防止士卒溃散是靠不住的。要想使部队齐心奋战如同一个人一样,关键在于驾驭士卒的方法。要使“刚柔”相得益彰,在于恰当地利用地形。所以善于用兵作战的人,能使全军上下携起手来团结得如同一个人一样,这是因为客观形势迫使士卒不得不这样。

\begin{yuanwen}
将军之事,静\footnote{沉着冷静。}以\footnote{“以”与“而”通。}幽\footnote{幽深莫测。},正\footnote{严正。}以治\footnote{不乱。}。能愚\footnote{愚弄,蒙蔽。}士卒之耳目,使之无知。易\footnote{改变。}其事,革\footnote{变更。}其谋,使人无识;易其居,迂其途,使人不得虑。帅与之\footnote{代词,指士卒。}期,如登高而去其梯。帅与之深入诸侯之地,而发其机\footnote{弩机。},焚舟破釜,若驱群羊,驱而往,驱而来,莫知所之。聚三军之众,投之于险,此谓将军之事也。九地之变,屈伸\footnote{伸展。}之利,人情之理,不可不察。
\end{yuanwen}

将军带兵作战,要沉着冷静而幽深莫测,要端庄持重,有条不紊。要能蒙蔽士卒的耳目,使他们对于军事行动毫无所知。要变更作战任务,改变作战计划,使他们无法知道为什么要改变;要不时地变换驻地,有意迁回行军,使他们无法推测出行动的意图。将帅与士卒如期去作战,要像登高而抽去梯子一样,断其退路。将帅率领士卒深人诸侯的境地,要像击发誓机、破釜沉舟一样一往无前,就像驱赶羊群一样,驱过来又赶过去,使他们不知道要到何处去。集结三军士卒,把他们置于危险的境地,让他们拼死作战,这就是将军带兵作战的诀窍。不同地区作战方法的变化、各种变通之利、士卒的心理状态,这些都是将帅不能不认真去研究和考察的问题。

\begin{yuanwen}
凡为客之道,深则专,浅则散。去国越境而师\footnote{这里用作动词,指打仗。}者,绝地也;四达者,衢地也;入深者,重地也;入浅者,轻地也;背固前隘者,围地也;无所往者,死地也。是故散地,吾将一其志;轻地,吾将使之属\footnote{联属,连续。因为入人之境不深,士卒心里未能专一,所以说“使之属”。};争地,吾将趋其后\footnote{意谓在争取有利地形时,不可从正面攻击敌人,宜速抄其后,所以说“趋其后”。};交地,吾将谨其守;衢地,吾将固其结\footnote{指结交诸侯。};重地,吾将继其食;圮地,吾将进其塗\footnote{占据通道。圮地难行,宜择可行之道,所以说“进其塗”。塗通“途”,道路。};围地,吾将塞其阙\footnote{通“缺”,缺口。};死地,吾将示之以不活。故兵之情,围则御,不得已则斗,过\footnote{指深陷险境。}则从。
\end{yuanwen}

大凡进人敌国境内作战的规律是:深入敌境则军心专一,浅人敌境则军心涣散。离开国土越过边境去别国作战的称为“绝地”;四通八达的地区叫做“衢地”;进人敌境纵深地区的叫做“重地”;进入敌境浅的地区叫做“轻地”;背后有险阻前面有隘口的地区叫“围地”;无路可走的地区就是“死地”。因此,在散地要使军队意志专一;在轻地要使军队紧密连属;在争地要迅速赶到敌人的后面;在交地要固守;在衢地要巩固与诸侯国的结盟;在重地要保障军需供应;在圮地就必须占据通道;在围地就要堵塞缺口;在死地就要显示出决一死战的信念。所以,士卒的心理状态是:陷人包围时就会竭力抵御,迫不得已时就会拼死战斗,深陷险境时就会言听计从。

\begin{yuanwen}
是故不知诸侯之谋者,不能预\footnote{通“与”。}交;不知山林、险阻、沮泽之形者,不能行军;不用乡导\footnote{向导。乡通“向”。}者,不能得地利。四五者\footnote{曹操注云:“谓九地之利害。”泛指上述各类情况,犹如“这些”、“这类”。},不知一,非霸王\footnote{银雀山汉简作“王霸之兵”。}之兵也。夫霸王之兵,伐大国,则其众不得聚;威加于敌,则其交\footnote{外交。}不得合\footnote{联合。}。是故不争天下之交,不养\footnote{指事奉。}天下之权,信\footnote{通“伸”。李荃注云:“惟得伸己之私志。”}己之私,威加于敌,故其城可拔,其国可隳\footnote{通“毁”,毁灭。}。
\end{yuanwen}

所以不了解诸侯国的战略意图,就不要和他结交;不熟悉山林、险阻、沼泽等地形情况,就不能行军;不使用向导,就不能得到地利。上述各类,如有一样不了解,就不能算是王、霸的军队。凡是王、霸的军队讨伐大国,能使敌国的军民来不及聚集;给敌军施加压力,能够使敌方的外交不能联合其他诸候。因此,不必争着同天下诸侯结交,也用不着去事奉天下的霸权,只要能伸张、施展自己的意志,给敌军施加压力,就可以攻取敌人的城邑,摧毁敌人的国家。

\begin{yuanwen}
施无法之赏,悬\footnote{悬挂,这里指颁发。}无政之令,犯\footnote{这里是约束、驱使之意。}三军之众,若使一人。犯之以事,勿告以言;犯之以利,勿告以害。投之亡地然后存,陷之死地然后生。夫众陷于害,然后能为胜败。故为兵之事,在于顺详敌之意,并敌一向,千里杀将,此谓巧能成事者也。
\end{yuanwen}

施行超出规定的奖赏,颁发超出规定的命令,驱使三军之众就如同用一个人一样。驱使士卒执行任务,而不告诉他们其中的意图;以利益来驱使士卒,只告诉他利益的一面,而不告诉他危害的一面。把他们置于绝境然后才能得以保存;把他们置于死地然后才能起死回生。三军陷人绝境,然后才能赢得胜利。所以,指挥作战的事情,在于详细地观察敌人的意图,集中兵力朝一个方向进攻,这样,即使长驱千里,也可以斩杀敌将,这就是所谓应用巧妙的方法而能够达到制胜目的的人。

\begin{yuanwen}
是故政举\footnote{这里指决定战事。}之日,夷关\footnote{封关。}折\footnote{毁坏的意思。}符\footnote{古代的一种通行凭证。},无通其使\footnote{使节。};厉\footnote{通“励”,磨砺。这里是反复计议的意思。}于廊庙\footnote{即庙堂。}之上,以诛\footnote{曹操注云:“诛,治也。”引申为商议、决定。}其事。敌人开阖\footnote{开门。这里指有机可乘。阖,门扇。},必亟入之。先其所爱\footnote{所爱之处。这里指关键、要害。},微\footnote{隐藏。}与之期。践墨\footnote{本指木工在木材上先画墨线,然后再随着墨线去加工物件。这里引申为随着敌情的变化而变化。}随敌,以决战事。是故始如处女,敌人开户;后如脱兔\footnote{放开的兔子。这里比喻像脱逃的野兔一样行动迅速。},敌不及拒。
\end{yuanwen}

所以,在决定作战的时候,就要封锁关口,废除通行凭证,禁止使节往来,要在庙堂上仔细研究敌情,决定战略决策。一旦有机可乘,一定要迅速潜人。首先夺取敌人战略要地,隐蔽与之作战的时间。实施作战计划要随着敌情的变化而变化,来决定自己的作战行动。所以,战争开始时就像处女那样显得沉静,一旦有机可乘,就要像脱逃的野兔一样行动迅速,使敌人来不及抗拒。
	
\chapter{火攻篇第十二}

“火攻”就是用“火”来帮助士卒进攻敌人,是古代的作战方式之一。孙武专辟一篇,详细地论述了火攻的种类、条件以及实施方法等问题。文中孙武提出了火攻有五种形式:一是焚烧敌军人马,二是焚烧敌军委积,三是焚烧敌军辎重,四是焚烧敌军军库,五是焚烧敌军攻城用的地道。他认为:实施火攻必须具备一定的条件,火攻器材必须平素就准备好。放火要看准天时,起火要选好日子。更重要的是火攻“必因五火之变而应之”,就是说必须根据火攻后敌情的变化而适时策应,根据具体情况适时地对敌发起进攻,以求扩大战果。孙武在本篇中还特别指出君主和将帅对待战争要慎重从事,提出“主不可以怒而兴师,将不可以愠而致战”战争的主导者切不可感情用事“合于利而动,不合于利而止”这才是安国全军之道”。
	
\begin{yuanwen}
孙子曰:凡火攻有五:一曰火人\footnote{指焚烧敌军的人马。火,用作动词,即焚烧。},二曰火积\footnote{焚烧敌军的委积(粮、草)。},三曰火辎\footnote{焚烧敌军的辎重。},四曰火库\footnote{焚烧敌军的武库。},五曰火队\footnote{队,通“隧”,指攻城用的地道。}。行火必有因\footnote{条件。},烟火必素\footnote{平素,平时。}具。发火有时,起火有日。时者,天之燥也;日者,月在箕\footnote{二十八宿之一。下同。}、壁、翼、轸也。凡此四宿者,风起之日也。
\end{yuanwen}

孙子说:火攻有五种形式:一是焚烧敌军人马,二是焚烧敌军委积,三是焚烧敌军辎重,四是焚烧敌军军库五是焚烧敌军攻城用的地道。实施火攻必须具备一定的条件,火攻器材必须平素就准备好。放火要看准天时,起火要选好日期。天时是指气候干燥的季节,日期是指月亮运行到“箕”、“壁”“翼”“轸”四个星宿位置的时候月亮运行到这四个星宿的时候,就是起风的日子

\begin{yuanwen}
凡火攻,必因\footnote{根据。}五火之变而应\footnote{策应。}之。火发于内,则早应之于外。火发兵静者,待而勿攻,极\footnote{尽。}其火力,可从而从\footnote{跟从。这里指进攻。}之,不可从而止。火可发于外,无待于内,以时发之。火发上风,无攻下风。昼风久,夜风止。凡军必知有五火之变,以数守之。
\end{yuanwen}

凡用火攻,必须根据五种火攻后敌情的变化而适时策应。在敌营内部放火,就要及时派兵从外面接应。火起之后而敌军依然毫无动静,就应该等待而不可贸然进攻,让火尽量燃烧后,再根据情况可以进攻就进攻,不可以进攻就停止。火可从外面烧起,就不必等待内应,只要时机成熟点火就行。火从上风点起,不要从下风进攻。
白天风刮得久了,夜晚就会停止。军队必须懂得这五种火攻的变化形式,然后耐心等候条件,实施火攻。

\begin{yuanwen}
故以火佐\footnote{帮助。}攻者明\footnote{王引之《经义述闻》引王念孙说谓“争明”即“争强”,是“明”可训“强”。“明”与下文“强”为互文,当为同义。},以水佐攻者强\footnote{指增强其威力。}。水可以绝\footnote{分割、断绝。},不可以夺\footnote{这里是赶走的意思。}。
\end{yuanwen}

用火来帮助进攻可以壮大其声势,用水来辅助军队进攻可以增强其威力。水可以分割、隔绝敌军,但却不能赶走敌军。
    
\begin{yuanwen}
夫战胜攻取,而不修\footnote{修治,这里引申为巩固。}其功者凶\footnote{祸,这里是危险的意思。},命曰\footnote{叫做。}费留\footnote{曹操注云:“若水之留(流),不复还。”水不复还即“白流”。}。故曰:明主虑\footnote{考虑。}之,良将修\footnote{这里有研究的意思。}之。非利不动,非得\footnote{得到,收获,这里有取胜的意思。}不用\footnote{指用兵作战。},非危不战。主不可以怒而兴师\footnote{兴兵作战。},将不可以愠\footnote{愤怒,恼怒。}而致战。合于利而动,不合于利而止。怒可以复喜,愠可以复悦。亡国不可以复存,死者不可以复生。故明君慎之,良将警之,此安国全军\footnote{保全军队。}之道也。
\end{yuanwen}

凡是打了胜仗,攻取了土地城邑,而不能巩固其胜利成果的就很危险,这种情况叫做“费留”。所以说,明智的君主要慎重地考虑这个问题,贤良的将帅要认真研究这个问题。形势不利就不行动,没有取胜的把握就不用兵:不到危急关头就不开战。君主不能因一时之怒而发动战争,将帅不可因一时之忿而出兵作战。符合国家利益就行动,不符合国家利益就停止。愤怒可以重新变为欢喜,恼怒可以重新转为高兴,但是国家灭亡了就不能复存,士卒死了也不能再生。所以,聪明的君主应该慎重对待战争,优良的将帅应该警惕战争,这才是安定国家和保全军队的重要原则。
	
\chapter{用间篇第十三}

“用间”即使用间谍。在本篇中,孙武主要论述了使用间谍侦察敌情在作战中的重要意义。文中提出间谍有五种:即乡间、内间、反间、死间、生间。如果五种间谍同时起用,使敌人摸不清我军的行动规律,这就是使用间谍神妙的道理,也是君主克敌制胜的法宝。孙武十分重视间谍的作用,他认为使用间谍是作战取胜的一个关键。在本篇中孙武还同时提出“非圣智不能用间,非仁义不能使间,非微妙不能得间之实”只有“明君贤将”才能“取于人”“动而胜人,成功出于众者”,这些都是使用间谍“先知”的结果。
	
\begin{yuanwen}
孙子曰:凡兴师十万,出征千里,百姓之费,公家之奉\footnote{通“俸”,指国家开支。},日费千金;内外骚动,怠\footnote{疲惫,懈怠。}于道路,不得操\footnote{操作,从事。}事者,七十万家\footnote{曹操注云:“古者八家为邻,一家从军,七家奉之,言十万之师举,不事耕稼者七十万家。”}。相守\footnote{相持。}数年,以争一日之胜,而爱\footnote{吝惜。}爵\footnote{爵位。}禄\footnote{俸禄。}百金,不知敌之情者,不仁\footnote{没有仁爱之心。}之至也,非人之将也,非主之佐也,非胜之主也。
\end{yuanwen}

孙子说:凡兴兵十万,出征千里,百姓的耗费、公家的开支,每天要花费千金;国内外动乱不安,军民疲惫地在路上奔波,不能从事正常生产的有七十万家。双方相持了数年,只是为了夺取最后一天的胜利,而吝惜爵禄和钱财,不能掌握敌情而导致失败的,那就是不仁到极点了,这种人不配做军队的统帅,不配做君主的助手,不配做胜利的主宰者。
	
\begin{yuanwen}
故明君贤将,所以动而胜人,成功出于众\footnote{出类拔萃。}者,先知\footnote{先知先觉。这里是指事先掌握了敌情。}也。先知者,不可取\footnote{求助。}于鬼神,不可象于事\footnote{指筮占之事。},不可验\footnote{验证。}于度\footnote{日月星辰运行的度数。},必取于人,知敌之情者也。
\end{yuanwen}

所以,贤明的君主和优秀的将领之所以一出兵就能战胜敌人,其功业出类拔萃,就在于事先能掌握敌情。事先掌握了敌情,就不用求神问鬼,不用从象数占卜,也不用日月星辰运行的度数来验证吉凶祸福,一定要从那些熟悉敌情的人的口中去获取。

\begin{yuanwen}
故用间有五:有因间\footnote{即下文的“乡间”。指利用敌国的乡野之民当间谍。},有内间,有反间,有死间,有生间\footnote{指到敌方刺探情况后能生还的间谍。}。五间俱起\footnote{起用。},莫知其道\footnote{途径,规律。},是谓神纪\footnote{即“道”,这里有“道理”的意思。},人君之宝也。乡间者,因其乡人\footnote{指敌国的详见百姓。}而用之。内间者,因其官人而用之。反间者,因其敌间\footnote{敌方派来的间谍。}而用之。死间者,为诳事于外,令吾间知之,而传于敌间也。生间者,反\footnote{通“返”,返回。}报也。
\end{yuanwen}

所以,使用的间谍有五种:即因间、内间、反间、死间、生间。五种间谍同时起用,能使敌人摸不清我军的行动规律,这就是使用间谍神妙的道理,也是君主克敌制胜的法宝。所谓“乡间”,就是指利用敌国的乡野百姓当间谍:所谓“内间”,就是利用敌方的官吏做间谍;所谓“反间”就是使敌方的间谍反为我军所用;所谓“死间”,就是指故意向外散布虚假的情况,并让潜人敌营的我方间谍知道而传给敌人的间谍;所谓“生间”,就是探知敌人情报后能够生还的间谍。

\begin{yuanwen}
故三军之事,莫亲\footnote{亲信,亲密。}于间,赏莫厚于间,事莫密\footnote{秘密。}于间,非圣智\footnote{才智过人。}不能用间,非仁义\footnote{这里指深仁厚义。}不能使间,非微妙\footnote{精细奥妙。微,细心。妙,巧妙。这里指谋虑精细的人。}不能得间之实。微哉!微哉!无所不用间也。间事未发而先闻者,间与所告者皆死。
\end{yuanwen}

所以在三军中,没有比间谍更亲信的了,没有比给间谍奖赏更优厚的了,没有比间谍的事更秘密的了。不是才智超群的人不能使用间谍,不是深仁厚义的人不能使用间谍,不是谋虑精细的人不能得到间谍的真实情报。微妙啊微妙!简直是没有什么地方不可以使用间谍的了。起用间谍的事还未实施而泄漏消息的,那么间谍和泄露消息的人都要处死。

\begin{yuanwen}
凡军之所欲击,城之所欲攻,人之所欲杀,必先知其守将\footnote{指守城之将。},左右\footnote{指守将身边的近侍之臣。},谒者\footnote{指守城的门卫。},门者\footnote{指把守城门的官吏。},舍人\footnote{指把守寝舍的官吏。}之姓名,令吾间必索\footnote{求。这里有“刺探”的意思。}知之。
\end{yuanwen}

凡是要出击敌方的军队,要攻占敌方的城邑,要刺杀敌方的人员,都必须事先探知他们的驻守将领、守将身边的近侍之臣、守城的门卫、把守城门的官更和把守寝舍的官吏的姓名,命令我方间谍务必将这些情况刺探清楚

\begin{yuanwen}
必索敌人之间来间我者,因而利\footnote{利用。}之,导而舍之\footnote{引导他并且把他放回去。},故反间可得而用也。因是\footnote{代词,指“乡间”、“内间”。}而知之,故乡间、内间可得而使也。因是而知之,故死间为诳事,可使告敌。因是而知之,故生间可使如期\footnote{指如期返回。}。五间之事,主必知之,知之必在于反间,故反间不可不厚\footnote{指待遇优厚。}也。
\end{yuanwen}

一定要查清楚前来侦察我军的敌方间谍,要收买利用他,引导他,然后再把他放回去,这样,反间就可以得到并为我所用了。通过反间了解了敌情,所以乡间、内间也就可以利用起来了。通过反间了解了敌情,就能使死间故意向外散布虚假的情况,并且让他报告敌人。通过反间了解了敌情,生间就可以如期返回通报敌情。五种间谍的使用,君主都必须掌握了解,了解情况的关键在于能使用反间计,所以对反间不可不给予优厚的待遇。

\begin{yuanwen}
昔殷之兴也,伊挚\footnote{伊尹,原为夏桀之臣,商汤灭夏时用他为相,消灭了夏朝。}在夏;周之兴也,吕牙在殷。故惟明君贤将,能以上智\footnote{高超的智慧。}为间者,必成大功。此兵之要,三军之所恃\footnote{依靠。}而动也。	
\end{yuanwen}

从前殷商兴起时,伊挚在夏朝为臣;周朝兴起时,吕牙在殷朝为臣。所以,明智的君主和贤能的将领,能用有高超智慧的人充当间谍,就一定能成就大功。这是用兵的关键,整个军队都要依靠间谍提供的情报来决定军事行动。

\end{document}