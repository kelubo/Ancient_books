% 吴子
% 吴子.tex

\documentclass[a4paper,12pt,UTF8,twoside]{ctexbook}

% 设置纸张信息。
\RequirePackage[a4paper]{geometry}
\geometry{
	%textwidth=138mm,
	%textheight=215mm,
	%left=27mm,
	%right=27mm,
	%top=25.4mm, 
	%bottom=25.4mm,
	%headheight=2.17cm,
	%headsep=4mm,
	%footskip=12mm,
	%heightrounded,
	inner=1in,
	outer=1.25in
}

% 设置字体,并解决显示难检字问题。
\xeCJKsetup{AutoFallBack=true}
\setCJKmainfont{SimSun}[BoldFont=SimHei, ItalicFont=KaiTi, FallBack=SimSun-ExtB]

% 目录 chapter 级别加点(.)。
\usepackage{titletoc}
\titlecontents{chapter}[0pt]{\vspace{3mm}\bf\addvspace{2pt}\filright}{\contentspush{\thecontentslabel\hspace{0.8em}}}{}{\titlerule*[8pt]{.}\contentspage}

% 设置 part 和 chapter 标题格式。
\ctexset{
	part/name= {第,卷},
	part/number={\chinese{part}},
	chapter/name={第,篇},
	chapter/number={\chinese{chapter}}
}

% 设置古文原文格式。
\newenvironment{yuanwen}{\bfseries\zihao{4}}

% 设置署名格式。
\newenvironment{shuming}{\hfill\bfseries\zihao{4}}

\title{\heiti\zihao{0} 吴子}
\author{吴起}
\date{战国}

\begin{document}

\maketitle
\tableofcontents

\frontmatter
\chapter{前言}

《吴子》又称《吴子兵法》、《吴起兵法》,是一部兵法著作。提出以治为胜,赏罚严明,主张在军队实行“进有重赏,退有重刑”,做到“令行禁止,严不可犯”。提出“用兵之法,教戒为先”,主张通过严格的军事训练,使士卒掌握各种作战本领,提高整个军队的战斗力。强调“简募良材”,根据士卒体力、技能等条件的不同,合理分工和编组,实现军队的优化组合。

《吴子》相传为战国时吴起所著。《韩非·五蠢篇》:“境内皆言兵,藏孙、吴之书者家有之。”《史记·孙子吴起列传》:“世俗所称师旅,皆道《孙子 十三篇》,《吴起兵法》世多有。”《汉书·艺文志》著录有《吴起》四十八篇。可见《吴起兵法》在战国末年和汉初还在流传。但1972年,山东临沂银雀山汉墓出土竹简中,兵书很多,却未见该书。(参见《临沂银雀山汉墓发掘简报》)据郭沫若同志考证:“但可惜这书(指《吴起兵法》)是亡了”。“故今存《吴子》实可断言为伪。以笔调觇之,大率西汉中叶时人之所依托。”(见郭著《青铜时代·述吴起》第三节)。书中提到的“笳”(见《应变 第五》)、“鞍”(见《治兵 第三》)等都是吴起时代还没有的东西,特别是“骑三千匹”(见《励士 第六》)这样大规模地使用骑兵部队更是那时所没有的。

《隋书·经藉志》著录:《吴起兵法》一卷。按《隋书》为唐初魏征所撰。而魏征在其《群书治要》中所摘录的《吴子》的《图国》、《论将》、《治兵》、《励土》四篇内容,大体与今本《吴子》相同。这样,原书亡佚时间至少在唐初以前了。

今本《吴子》分:《图国》、《料敌》、《治兵》、《论将》、《应变》、《励土》六篇。虽非吴起原著,但就其内容来看,仍不失为一部较有价值的兵书,不管是谁所著,都是前人留下的宝贵遗产。宋神宗年间(公元1078-1085),将其列入《武经七书》,颁行武学,为将校所必读,颇受重视。

现有英、日、法、俄等文字译本。

\mainmatter

\chapter{图国}

图国,即筹划治理国家。本篇共分八节,大体是论述治国、治军、兴兵作战、亲民用贤等国家大计。

\begin{yuanwen}
吴起儒服,以兵机见魏文侯。

文侯曰:“寡人不好军旅之事。”

起曰:“臣以见占隐,以往察来,主君何言与心违?今君四时,使斩离皮革,掩以朱漆,画以丹青,炼以犀象,冬日衣之则不温,夏日衣之则不凉。为长戟二丈四尺,短戟一丈二尺。革车掩户,缦轮笼毂,观之于目则不丽,乘之于国则不轻,不识主君安用此也?若以备进战退守,而不求能用者,譬犹伏鸡之搏狸,乳犬之犯虎,虽有斗心,随之死矣。昔承桑氏之君,修德废武,以灭其国。有扈氏之君,恃众好勇,以丧其社稷。明主鉴兹,必内修文德,外治武备。故当敌而不进,无逮于义矣。僵尸而哀之,无逮于仁矣。”

于是文侯身自布席,夫人捧觞,醮吴起于庙,立为大将崇西河。与诸侯大战七十六,全胜六十四,馀则钧解。辟土四面,拓地千里,皆起之功也。

吴子曰:“昔之图国家者,必先教百姓而亲万民。有四不和:不和于国,不可以出军;不和于军,不可以出陈;不和于陈,不可以进战;不和于战,不可以决胜。是以道之主,将用其民,先和而造大事。不敢信其私谋,必告于祖庙,启于元龟,参之天时,吉乃后举。民知爱其命,惜其死,若此之至,而与之临难,则士以进死为荣,退生为辱矣。”

吴子曰:“夫道者,所以反本复始;义者,所以行事立功;谋者,所以违害就利;要者,所以保业守成。若行不合道,举不合义,而处大居贵,患必及之。是以圣人绥之以道,理之以义,动之以礼,抚之以仁。此四德者,修之则兴,废之则衰。故成汤讨桀而夏民喜说,周武伐纣而殷人不非;举顺天人,故能然矣。”

吴子曰;“凡制国治军,必教之以礼,励之以义,使有耻也。夫人有耻,在大,足以战;在小,跳以守矣。然战胜易,守胜难。故曰:‘天下战国,五胜者祸,四胜者弊,三胜者霸,二胜者王,一胜者帝。’是以数胜得天下者稀,以亡者众。”

吴子曰:“凡兵之所起者有五:一曰争名,二曰争利,三曰积德恶,四曰内乱,五曰因饥。其名又有五:一曰义兵,二曰强兵,三曰刚兵,四曰暴兵,五曰逆兵。禁暴救乱曰义,恃众以伐曰强,因怒兴师曰刚,弃礼贪利曰暴,国乱人疲,举事动众曰逆。五者之数,各有其道:义必以礼服,强必以谦服,刚必以辞服,暴必以诈服,逆必以权服。”

武侯问曰:“愿闻治兵、料人、固国之道。”

起对曰:“古之明王,必谨君臣之礼,饰上下之仪,安集吏民,顺俗而教,简募良材,以备不虞。昔齐桓募士五万,以霸诸侯。晋文召为前行四万,以获其志。秦缪置陷陈三万,以服邻敌。故强国之君,必料其民。民有胆勇气力者,聚为一卒。乐以进战效力以显其忠勇者,聚为一卒。能逾高超远轻足善走者,聚为一卒。王臣失位而欲见功于上者,聚为一卒。弃城去守,欲除其丑者,聚为一卒。此五者军之练锐也。有此三千人,内出可以决围,外入可以屠城矣。”

武侯曰:“愿闻陈必定,守必固,战必胜之道。”

起对曰:“立见且可,岂直闻乎!君能使贤者居上,不肖者处下,则陈已定矣。民安其田宅,亲其有司,则守已固矣。百姓皆是吾君而非邻国,则战已胜矣。”

武侯尝谋事,群臣莫能及,罢朝而有喜色。

起进曰:“昔楚庄王尝谋事,群臣莫能及,罢朝而有忧色。申公问曰:‘君有忧色,何也?’曰:‘寡人闻之,也不绝圣,国不乏贤,能得其师者五,能得其友者霸。今寡人不才,而群臣莫及者,楚国其殆矣。’此楚庄王之所忧,而君说之,臣窃惧矣。”于是武侯有惭色。
\end{yuanwen}

\chapter{料敌}
\begin{yuanwen}
武侯谓吴起曰:“今秦胁吾西,楚带吾南,赵冲吾北,齐临吾东,燕绝吾后,韩据吾前。六国兵四守,势甚不便,忧此奈何?”

起对曰:“夫安国家之道,先戒为宝。今君已戒,祸其远矣。臣请论六国之俗:夫齐陈重而不坚,秦陈散而自斗,楚陈整而不久,燕陈守而不走,三晋陈治而不用。”

夫齐性刚,其国富,君臣骄奢而简于细民,其政宽而禄不均,一陈两心,前重后轻,故重而不坚。击此之道,必三分之,列其左右,胁而从之,其陈可坏。

秦性强,其地险,其政严,其赏罚信,其人不让,皆有斗心,故散而自战。击此之道,必先示之以利而引去之,士贪于得而离其将,乘乖猎散,设伏投机,其将可取。

楚性弱,其地广,其政骚,其民疲,故整而不久。击此之道,袭乱其屯,先夺其气,轻进速退,弊而劳之,勿与战争,其军可败。

燕性悫,其民慎,好勇义,寡诈谋,故守而不走。击此之道,触而迫之,陵而远之,驰而后之,则上疑而下惧,谨我车骑必避之路,其将可虏。

三晋者,中国也,其性和,其政平,其民疲于战,习于兵,轻其将,薄其禄,士无死志,故治而不用。击此之道,阻陈而压之,众来则拒之,去则追之,以倦其师。此其势也。

“然则一军之中必有虎贲之士,力轻扛鼎,足轻戎马,搴旗取将,必有能者。若此之等,选而别之,爱而贵之,是谓军命。其有工用五兵材力健疾,志在吞敌者,必加其爵列,可以决胜。厚其父母妻子,劝赏畏罚。此坚阵之士,可与持久。能审料此;或以击倍。”

武侯曰“善!”

吴子曰:“凡料敌,有不卜而与之战者八:一曰疾风大寒,早兴寤迁,刊木济水,不惮艰难;二曰盛夏炎热,晏兴无间,行驱饥渴,务于取远;三曰师既淹久,粮食无有,百姓怨怒,妖祥数起,上不能止;四曰军资既渴,薪刍既寡,天多阴雨,欲掠无所;五曰徒众不多,水地不利,人马疾疫,四邻不至;六曰道远日暮,士众劳惧,倦而未食,解甲而息;七曰将薄吏轻,士卒不固,三军数惊,师徒无助;八曰陈而未定,舍而未毕,行孤涉险,半隐半出。诸如此者,击之勿疑。有不占而避之者六:一曰土地广大,人民富众;二曰上爱萁下,惠施流布;三曰赏信刑察,发必得时,四曰陈功居列,任贤使能;五曰师徒之众,兵甲之精;六曰四邻之助,大国之援。凡此不如敌人,避之勿疑,所谓见可而进,知难而退也。”

武侯问曰:“吾欲观敌之外以知其内,察其进以知其止,以定胜负,可得闻乎?”

起对曰:“敌人之来,荡荡无虑,旌旗烦乱,人马数顾,一可击十,必使无措。诸侯未会,君臣未和,沟垒未成,禁令未施,三军匈匈,欲前不能,欲去不敢,以半击倍,百战不殆。”

武侯问敌必可击之道,起对曰:“用兵必须审敌虚实而趋其危。敌人远来新至,行列未定,可击;既食未设备;可击;奔走,可击;勤劳,可击;未得地利,可击;失时不从,可击;旌旗乱动,可击;涉长道,后行未息,可击;涉水半渡,可击;险道狭路,可击;陈数移动,可击;将离士卒,可击;心怖,可击。凡若此者,选锐冲之,分兵继之,急击勿疑。”
\end{yuanwen}

\chapter{治兵}
\begin{yuanwen}
武侯问曰:“进兵之道何先?”

起对曰:“先明四轻、二重、一信。”

曰:“何谓也?”

对曰:“使地轻马,马轻车,车轻人,人轻战。明知阴阳,则地轻马;刍秣以时,则马轻车;膏锏有余,则车轻人;锋锐甲坚,则人轻战;进有重赏,退有重刑,行之以信,令制远,此胜之主也。”

武侯问曰:“兵何以为胜?”

起对曰:“以治为胜。”

又问曰:“不在众寡?”

对曰“:“若法令不明,赏罚不信,金之不止,鼓之不进,虽有百万何益于用?所谓治者,居则有礼,动则有威,进不可当,退不可追,前却有节,左右应麾,虽绝成陈,虽散成行。与之安,与之危,其众可合不可离,可用而不可疲,投之所往,天下莫当。名曰父子之兵。”

吴子曰:“凡行军之道,无犯进止之节,无失饮食之适,无绝人马之力。此三者,所以任其上令。任其上令,则治之所由生也。若进止不度,饮食不适,马疲人倦而不解舍,所以不任其上令。上令既废,以居则乱,以战则败。”

吴子曰:“凡兵战之场,立尸之地,必死则生,幸生则死。其善将者,如坐漏船之中,伏烧屋之下,使智者不及谋,勇者不及怒,受敌可也。故曰,用兵之害,犹豫最大;三军之灾,生于狐疑。”

吴子曰:“夫人当死其所不能,败其所不便。故用兵之法,教戒为先。一人学战,教成十人。十人学战,教成百人。百人学战,教成千人。千人学战,教成万人。万人学战,教成三军。以近待远,以佚待劳,以饱待饥。圆而方之,坐而起之,行而止之,左而右之,前而后之,分而合之,结而解之,每变皆习,乃用授其兵。是谓将事。”

吴子曰:“教战之令,短者持矛戟,长者持弓弩,强者持旌旗,勇者持金鼓,弱者给厮养,智者为谋主。乡里相比,什伍相保,一鼓整兵,二鼓习陈,三鼓趋食,四鼓严办,五鼓就行。闻鼓声合,然后举旗。”

武侯问曰:“三军进止,岂有道乎?”

起对曰:“无当天灶,无当龙差别。天灶者,大谷之口;龙头者,大山之端。必左青龙,右白虎,前朱雀,后玄武,招摇在上,从事于下。将战之时,审候风所从
来。风顺致呼而从之,风逆坚以待之。”

琥侯问曰:“凡畜卒骑,岂有方乎?”

起对曰:“夫马,必安其处所,适其水草,节其饥饱。冬则温烧,夏则凉庑。刻剔毛魆;谨落四下。戢其耳目,无令惊骇。习其驰逐,闲其进止。人马相亲,然后
可使。车骑之具,鞍、勒、衔、辔,必令完坚。凡马不伤于末,必伤于始;不伤于饥,必伤于饱。日暮道远,必数上下;宁劳于人,慎勿劳马;常令有余,备敌覆我。能明此者,横行天下。”
\end{yuanwen}

\chapter{论将}
\begin{yuanwen}
吴子曰:“夫总文武者,军之将也。兼刚柔者,兵之事也。凡人论将,常观于勇。勇之于将,乃数分之一耳。夫勇者必轻合,轻合而不知利;未可也。故将之所慎者五:一曰理,二曰备,三曰果,四曰戒,五曰约。理者,治众如治寡。备者,出门如见敌。果者,临敌不怀生。戒者,虽克如始战。约者,法令省而不烦。受命而不辞敌,破而后言返,将之礼也。故师出之日,有死之荣,无生之辱。”

吴子曰:“凡兵有四机:一曰气机,二曰地机,三曰事机,四曰力机。三军之众,百万之师,张设轻众,在于一人,是谓气机。路狭道险,名山大寨,士夫所守,千夫不过,是谓地机。善行间谍,轻兵往来,分散其众,使其君臣相怨,上下相咎,是谓事机。车坚管辖,舟利橹辑,士习战陈,马闲驰逐,是谓力机。知此四者,乃可为将。然其威、德、仁、勇,必足以率下安众,怖敌决疑,施令而下不敢犯,所在寇不敢敌。得之国强,去之国亡,是谓良将。”

吴子曰:“夫鼙鼓金铎,所以威耳;旌旗麾帜,所以威目;禁令刑罚,所以威心。耳威于声,不可不清;目威于色,不可不明;心威于刑,不可不严。三者不立,虽有其国,必败于敌。故曰:将之所麾,莫不从移;将之所指,莫不前死。”

吴子曰:“凡战之要,必先战其将而察其才,因形用权,则不劳而功举。其将愚而信人,可诈而诱;贪而忽名,可货而赂;轻变无谋,可劳而困,上富而骄,下贫而怨,可离而间,进退多疑,其众无依,可震而走;士轻其将而有归志,塞易开险,可邀而取;进道易,退道难,可来而前,进道险,退道易,可薄而击;居军
下湿,水无所通,霖雨数至,可灌而沉;居军荒泽,草楚幽秽,风飚数至,可焚而灭,停久不移,将士懈怠,其军不备,可潜而袭。”

武侯问曰:“两军相望,不知其将,我欲相对之,其术如何?”

起对曰:“令贱而勇者,将轻锐以尝之,务于北,无务于得。观敌之来,一坐一起,其政以理。其追北佯为不及,见其利佯为不知。如此将者,名为智将,勿与战也。若其众权哗,旌旗烦乱,其卒自行自止,其兵或纵或横,其追北恐不及,见利恐不得,此为愚将,虽众可获。”
\end{yuanwen}

\chapter{应变}
\begin{yuanwen}
武侯问曰:“车坚马良,将勇兵强,卒遇敌人,乱而失行,则如之何?”

起对曰:“凡战之法,昼以旌旗幡麾为节,夜以金鼓笳笛为节。麾左而左,麾右而右,鼓之则进,金之则止,一吹而行,再吹而聚,不从令者诛。三军服威,士卒用命,则战无强敌,攻无坚陈矣。”

武侯问曰:“若敌众我寡,为之奈何?”

起对曰:“避之于易,邀之于阨。故曰以一击十,莫善于阨;以士击百,莫善于险,以千击万,莫善于阻。今有少年卒起,击金鸣鼓于阨路,虽有大众,莫不惊
动。故曰:‘用众者务易,用少者务隘。”

武侯问曰:“有师甚众,既武且勇,背大险阻,右山左水;深沟高垒,守以强弩;退如山移,进如风雨;粮食又多,难与长守。”

对曰:“大哉问乎!非此车骑之力,圣人之谋也。能备千乘万骑,兼之徒步,分为五军,各军一衢。夫五军五衢,敌人必惑,莫之所加。敌人若坚守,以固其兵,急行间谍,以观其虑。彼听吾说,解之而去;不听吾说,斩使焚书。分为五战,战胜勿追,不胜疾归。如是佯北,安行疾斗,一结其前,一绝其后,两军衔枚,或左或右,而袭其处,五军交至,必有其利。此击强之道也。”

武侯问曰:“敌近而薄我,欲去无路;我众甚惧,为之奈何?”

对曰:“为此之术,若我众彼寡,分而乘之;彼众我寡,以方从之;从之无息,虽众可服。”

武侯问曰:“若遇敌于溪谷之间,谤多险阻,彼众我寡,为之奈何?”

起对曰:“遇诸丘陵、林谷、深山、大泽,疾行亟去,勿得从容。若高山深谷,卒然相遇,必选鼓噪而乘之。进弓与弩,且射且虏。审察其政,乱则击之勿疑。”

武侯问曰:“左右高山,地甚狭迫,卒遇敌人,击之不敢,去之不得,为之奈何?”

起对曰:“此谓谷战,虽众不用。募吾材士,与敌相当,轻足利兵,以为前行,分车列骑,隐于四旁,相去数里,无见其兵,敌必坚陈,进退不敢。于是出旌列旆,行出山外营之。敌人必惧,车骑挑之,勿令得休。此谷战之法也。”

武侯问曰:“吾与敌相遇大水之泽,倾轮没辕,水薄车骑,舟楫不设,进退不得,为之奈何?”

起对曰:“此谓水战,无用车骑,且留其傍。登高四望,必得水情。知其广狭,尽其浅深,乃可为奇以胜之。敌若绝水,半渡而薄之。”

武侯问曰:“天久连雨,马陷车止,四面受敌,三军惊骇,为之奈何?”

起对曰:“凡用车者,阴湿则停,阳燥则起,贵高贱下,驰其强车,若进若止,必从其道。敌人若起,必逐其迹。”

武侯问曰:“暴寇卒来,掠吾田野,取吾牛羊,则如之何?”

起对曰:“暴寇之来,必虑其强,善守勿应。彼将暮去,其装必重,其心必恐,还退务速,必有不属,追而击之,其兵可覆。

吴子曰:“凡攻敌围城之道,城邑既破,各入其宫。御其禄秩,收其器物。军之所至,无刊其木、发其屋、取其栗、杀其六畜、燔其积聚,示民无残心。其有请降,许而安之。”
\end{yuanwen}

\chapter{励士}
\begin{yuanwen}
武侯问曰:“严刑明赏,足以胜乎?”

起对曰:“严明之事,臣不能悉,虽然,非所恃也。夫发号布令而人乐闻,兴师动众而人乐战,交兵接刃而人乐死,此三者,人主之所恃也。”

武侯曰:“致之奈何?”

对曰:“君举有功而进飨之,无功而励之。”

于是武侯设坐庙廷,为三行飨士大夫。上功坐前行,肴席,兼重器上牢;次功坐中行,肴席,器差减;无功坐后行,肴席无重器。

飨毕而出,又颁赐有功者父母妻子于庙门外,亦以功为差。有死事之家,岁被使者劳赐其父母,著不忘于心。行之三年,秦人兴师,临于西河,魏士闻之,不待吏令,介胄而奋击之者以万数。

武侯召吴起而谓曰:“子前日之教行矣。”

起对曰:“臣闻人有短长,气有盛衰。君试发无功者五万人,臣请率以当之脱其不胜,取笑于诸侯,失权于天下矣。今使一死贼伏于旷野,千人追之,莫不枭
视狼顾。何者?忌其暴起害己也。是以一人投命,足惧千夫。今臣以五万之众为一死贼,率以讨之,固难敌矣。”

于是武侯从之,兼车五百乘,骑三千匹,而破秦五十万众,此励土之功也。

先战一日,吴起令三军曰;“诸吏士当从,受驰车、骑与徒,若车不得车,骑不得骑,徒不得徒,虽破军,皆无功。”故战之日,其令不烦而威镇天下。
\end{yuanwen}

\end{document}