% 灵枢

\documentclass[12pt,UTF8]{ctexbook}

% 设置纸张信息。
\usepackage[a4paper,twoside]{geometry}
\geometry{
	left=25mm,
	right=25mm,
	bottom=25.4mm,
	bindingoffset=10mm
}

\usepackage{tcolorbox}

% 目录 chapter 级别加点(.)。
\usepackage{titletoc}
\titlecontents{chapter}[0pt]{\vspace{3mm}\bf\addvspace{2pt}\filright}{\contentspush{\thecontentslabel\hspace{0.8em}}}{}{\titlerule*[8pt]{.}\contentspage}

% 设置 part 和 chapter 标题格式。
\ctexset{
	part/name= {第,卷},
	part/number={\chinese{part}},
	chapter/name={第,篇},
	chapter/number={\chinese{chapter}}
}

% 设置字体,并解决显示难检字问题。
\xeCJKsetup{AutoFallBack=true}
\setCJKmainfont{SimSun}[BoldFont=SimHei, ItalicFont=KaiTi, FallBack=SimSun-ExtB]

% 设置古文原文格式。
\newenvironment{yuanwen}{\bfseries\zihao{4}}

% 设置署名格式。
\newenvironment{shuming}{\hfill\bfseries\zihao{4}}

% 注脚每页重新编号,避免编号过大。
\usepackage[perpage]{footmisc}

\title{\heiti\zihao{0} 灵枢}
\date{}

\begin{document}

\maketitle
	
\tableofcontents
	
\frontmatter
	
\chapter{序}
	
\begin{yuanwen}
昔黄帝作《内经》十八卷,《灵枢》九卷,《素问》九卷,乃其数焉。世所奉行唯《素问》耳。越人得其一二而述《难经》,皇甫谧次而为《甲乙》,诸家之说,悉自此始。其间或有得失,未可为后世法。则谓如南阳活人书称:咳逆者,哕也。谨按《灵枢经》曰:新谷气入于胃,与故寒气相争,故曰哕。举而并之,则理可断矣。又如《难经》第六十五篇,是越人标指《灵枢》本输之大略,世或以为流注。谨按《灵枢经》曰:所言节者,神气之所游行出人也,非皮肉筋骨也;又曰:神气者,正气也。神气之所游行出人者,流注也;井荥输经合者,本输也。举而并之,则知相去不啻天壤之异。但恨《灵枢》不传久矣,世莫能究。夫为医者,在读医书耳,读而不能为医者有矣,未有不读而能为医者也。不读医书,又非世业,杀人尤毒于梃刃。是故古人有言曰:为人子而不读医书,犹为不孝也。仆本庸昧,自髫迄壮,潜心斯道,颇涉其理。辄不自揣,参对诸书,再行校正家藏旧本《灵枢》九卷,共八十一篇,增修音释,附于卷末,勒为二十四卷。庶使好生之人,开卷易明,了无差别。除已具状经所属申明外,准使府指挥依条申转运司选官详定,具书送秘书省国子监。今崧专访请名医,更乞参详,免误将来。利益无穷,功实有自。
\end{yuanwen}
	
% 增加空行。
~\\
	
\begin{shuming}
时宋绍兴乙亥仲夏望日锦官史崧题
\end{shuming}
	
\mainmatter

\part{}
	
\chapter{九针十二原}
	
九针,古代用来针治的九种针具,即镵针、员针、鍉针、锋针、铍针、员利针、毫针、长针、大针。针具何以为九种,这与古人的数字崇拜有关。古人在生产生活的实践中发现客观世界存在着数量关系,这种数量关系式世界的本质,决定着万物的存在方式。数,不仅是单纯的计算工具,而且是自然规律的反映。于是产生了数理哲学,来指导人类的社会实践。《素问·三部九侯论》云:“天地之至数,始于一,终于九焉。”“九”为数之极,所以针具也有九种。十二原,指十二原穴。具体指五脏各二原穴,合膏之原、盲之原各一,共十二穴。“原”即源,本源之义。所以篇中云:“五脏有疾,应出十二原”,五脏之病在十二原穴上有反映,因此“五脏有疾,当取之十二原。”本篇主要内容有三部分,首先论述了针刺中经气的微妙变化及针刺的疾、徐、迎、随、开、阖等手法和补泻作用。其次,详论了九针之形制及各自适宜的主治病证。最后叙述了分布在肘、膝、胸、脐等处的十二个原穴及脏腑疾病分别取用十二原穴的道理。取其论述所及内容“九针”和“十二原”而名篇。
	
\begin{yuanwen}
黄帝问于岐伯曰:余子万民\footnote{爱万民。},养百姓\footnote{百官。},而收其租税;余哀其不给而属\footnote{zh\v{u},连续。}有疾病。余欲勿使被\footnote{受。}毒药\footnote{治病药物。古人以药能治病,谓之毒药。},无用砭石,欲以微针通其经脉,调其血气,荣其逆顺出入之会。令可传于后世,必明为之法,令终而不灭,久而不绝,易用难忘,为之经纪\footnote{条理。},异其章,别其表里,为之终始。令各有形,先立《针经》。愿闻其情。
\end{yuanwen}
	
\begin{yuanwen}
岐伯答曰:臣请推而次之,令有纲纪,始于一,终于九焉。请言其道!小针\footnote{也叫微针,即今之毫针。}之要,易陈而难入。粗守形,上守神。神乎神,客在门。未睹其疾,恶知其原?刺之微,在速迟。粗守关,上守机,机之动,不离其空\footnote{通“孔”,腧穴。}。空中之机,清静而微。其来不可逢,其往不可追。知机之道者,不可挂以发\footnote{此处以发射弓弩的技术比喻针刺。“不可挂以发”诸家解释都认为是针刺技术精深之义。但对其本意未有确解。“不可挂以发”与“叩之不发”意正相反。后者意为虽箭在弦上却不能射出。窃以为“不可挂以发”,意指不将箭与弦挂得过紧,则可较容易地把箭射出。}。不知机道,叩之不发。知其往来,要与之期。粗之(闇)暗乎,妙哉,工独有之。往\footnote{指经气去。}者为逆,来\footnote{指经气至。}者为顺,明知逆顺,正行无问。逆而夺之,恶得无虚?追而济之,恶得无实?迎之随之,以意和之,针道毕矣。
\end{yuanwen}
	
\begin{yuanwen}
凡用针者,虚则实之,满则泄之,宛\footnote{y\`u,通“郁”,积聚。}陈则除之,邪胜则虚之。《大要》曰:徐而疾则实,疾而徐则虚。言实与虚,若有若无\footnote{有气指针刺后在刺穴周围产生的酸麻胀痛之感,甚至沿经脉传导,在医生手下有紧滞感。无气则为针刺后没有感觉,医生下针如刺豆腐。气本无形,故云若有若无。}。察后与先,若存若亡。为虚与实,若得若失\footnote{形容针刺补泻手法的作用。实证,泻而取之,使患者若有所失;虚证,补而实之,使患者若有所得。}。虚实之要,九针最妙。补泻之时,以针为之。泻曰:必持内之,放而出之,排阳得针\footnote{有三说。一,阳指皮肤浅表部位,排开浅表部位,使邪气随针外泄。二,阳指表阳,排开表阳,以去邪气。三,排阳,推扬,转针。},邪气得泄。按而引针,是谓内温\footnote{气血内蕴。温,同“蕴”。},血不得散,气不得出也。补曰:随之,随之,意若妄之,若行\footnote{行针导气。}若按\footnote{按压孔穴以下针。},如蚊虻止,如留如还,去如弦绝。令左属右,其气故止,外门已闭,中气乃实。必无留血,急取诛之。持针之道,坚者为宝。正指直刺,无针左右。神在秋毫,属意病者,审视血脉,刺之无殆。方刺之时,必在悬阳\footnote{凡刺时必举阳气为主,故曰悬阳。悬,举。阳,神气。},及与两卫\footnote{卫气在阳,肌表之卫。脾气在阴,脏腑之卫。二者皆神气所居,不可伤犯。凡用针首宜顾此。},神属勿去,知病存亡。血脉者,在腧横居,视之独澄,切之独坚。
\end{yuanwen}
	
\begin{yuanwen}
九针之名,各不同形。一曰镵针\footnote{因其针形尖锐,得名。ch\'an,锐。},长一寸六分;二曰员针,长一寸六分;三曰鍉针\footnote{因其针形似箭而得名。d\=i。},长三寸半;四曰锋针,长一寸六分;五曰铍针\footnote{因其针锋如剑而得名。p\=i,两刃小刀。},长四寸,广二分半;六曰员利针,长一寸六分;七曰毫针,长三寸六分;八曰长针,长七寸;九曰大针,长四寸。镵针者,头大末锐,去泻阳气;员针者,针如卵形,揩摩分间,不得伤肌肉者,以泻分气;鍉针者,锋如黍粟之锐,主按脉勿陷,以致其气;锋针者,刃三隅,以发痼疾;铍针者,末如剑锋,以取大脓;员利针者,尖如氂\footnote{m\'ao,牦牛尾,也指马尾。},且员且锐,中身微大,以取暴气;毫针者,尖如蚊虻喙,静以徐往,微以久留之而养,以取痛痹;长针者,锋利身长,可以取远痹;大针者,尖如梃\footnote{t\v{i}ng,专指竹梃。},其锋微员,以泻机关之水也。九针毕矣。
\end{yuanwen}
	
\begin{yuanwen}
夫气之在脉也,邪气在上,浊气\footnote{饮食积滞之气。}在中,清气在下。故针陷脉则邪气出,针中脉\footnote{中部阳明之合穴,即足三里穴。}则浊气出,针太深则邪气反沉,病益。故曰:皮肉筋脉,各有所处,病各有所宜,各不同形,各以任其所宜。无实无虚,损不足而益有余,是谓甚病,病益甚,取五脉\footnote{五脏腧穴。}者死,取三脉\footnote{手足三阳脉。}者恇\footnote{ku\=ang};夺阴者死,夺阳者狂,针害毕矣。刺之而气不至,无问其数。刺之而气至,乃去之,勿复针。针各有所宜,各不同形,各任其所为,刺之要,气至而有效,效之信,若风之吹云,明乎若见苍天。刺之道毕矣。
\end{yuanwen}
	
\begin{yuanwen}
黄帝曰:愿闻五脏六腑所出之处。
	
岐伯曰:五脏五腧,五五二十五腧\footnote{每脏有井、荥、输、经、合之五腧穴,五脏共二十五穴。};六腑六腧,六六三十六腧\footnote{每腑有井、荥、输、原、经、合六腧,六腑共三十六腧穴。}。经脉十二,络脉十五\footnote{十二经各有一络脉,加任、督及脾之大络,共十五络。}。凡二十七气,以上下。所出为井\footnote{古代以泉源出水之处为井。人之血气,出于四肢,故脉出处,为井。},所溜为荥\footnote{形容脉气流过的地方,像刚从泉源流出的小水流。《说文·水部》:“荥(x\'ing),绝小水也”。},所注为输\footnote{形容脉气流注到此后又灌注到彼。注:灌注。输,运输。},所行为经\footnote{通。},所入为合。二十七气所行,皆在五腧也。节之交\footnote{人体关节等部交接处的间隙。},三百六十五会。知其要者,一言而终;不知其要,流散无穷。所言节者,神气之所游行出入也,非皮肉筋骨也。
\end{yuanwen}
	
\begin{yuanwen}
睹其色,察其目,知其散复。一其形,听其动静,知其邪正。右主推之,左持而御之,气至而去之。凡将用针,必先诊脉,视气之剧易,乃可以治也。五脏之气已绝于内,而用针者反实其外,是谓重竭。重竭必死,其死也静。治之者辄反其气,取腋与膺。五脏之气已绝于外,而用针者反实其内,是谓逆厥。逆厥则必死,其死也躁。治之者反取四末。刺之害,中而不去,则精泄;不中而去,则致气。精泄则病益甚而恇,致气则生为痈疡。
\end{yuanwen}
	
\begin{yuanwen}
五脏有六腑,六腑有十二原,十二原出于四关\footnote{即两肘两膝之四个关节。},四关主治五脏。五脏有疾,当取之十二原。十二原者,五脏之所以禀三百六十五节之会也。五脏有疾也,应出十二原,而原各有所出,明知其原,睹其应,而知五脏之害矣。
\end{yuanwen}
	
	\begin{yuanwen}
	阳中之少阴,肺也,其原出于太渊,太渊二。阳中之太阳,心也,其原出于大陵,大陵二。阴中之少阳,肝也,其原出于太冲,太冲二。阴中之至阴,脾也,其原出于太白,太白二。阴中之太阴,肾也,其原出于太溪,太溪二。膏之原,出于鸠尾,鸠尾一。肓之原,出于脖胦,脖胦一。凡此十二原者,主治五脏六腑之有疾者也。
	
	胀取三阳,飧泄取三阴。
\end{yuanwen}

\begin{yuanwen}
今夫五脏之有疾也,譬犹刺也,犹污也,犹结也,犹闭也。刺虽久,犹可拔也。污虽久,犹可雪也,结虽久,犹可解也。闭虽久,犹可决也。或言久疾之不可取者,非其说也。夫善用针者,取其疾也,犹拔刺也,犹雪污也,犹解结也,犹决闭也。疾虽久,犹可毕也。言不可治者,未得其术也。
\end{yuanwen}
	
	\begin{yuanwen}
	刺诸热者,如以手探汤;刺寒清者,如人不欲行。阴有阳疾者,取之下陵三里,正往无殆,气下乃止,不下复始也。疾高而内者,取之阴之陵泉;疾高而外者,取之阳之陵泉也。
	\end{yuanwen}

	\chapter{本输}
	
	\begin{yuanwen}
	黄帝问于岐伯曰:凡刺之道,必通十二经络之所终始,络脉之所别处,五俞之所留,六腑之所与合,四时之所出入,五脏之所溜处,阔数之度,浅深之状,高下所至。愿闻其解。
	
	岐伯曰:请言其次也。肺出于少商,少商者,手大指端内侧也,为井木;溜于鱼际,鱼际者,手鱼也,为荥;注于太渊,太渊鱼后一寸陷者中也,为俞;行于经渠,经渠寸口中也,动而不居为经;入于尺泽,尺泽肘中之动脉也,为合。手太阴经也。
	
	心出于中冲,中冲,手中指之端也,为井木;流于劳宫,劳宫掌中中指本节之内间也,为荥;注于大陵,大陵掌后两骨之间方下者也,为俞;行于间使,间使之道,两筋之间,三寸之中也,有过则至,无过则止,为经;入于曲泽,曲泽,肘内廉下陷者之中也,屈而得之,为合。手少阴也。
	
	肝出于大敦,大敦者,足大趾之端,及三毛之中也,为井木;溜于行间,行间足大趾间也,为荥;注于太冲,太冲行间上二寸陷者之中也,为俞;行于中封,中封内踝之前一寸半,陷者之中,使逆则宛,使和则通,摇足而得之,为经;入于曲泉,曲泉辅骨之下,大筋之上也,屈膝而得之,为合。足厥阴也。
	
	脾出于隐白,隐白者,足大趾之端内侧也,为井木;溜于大都,大都本节之后下陷者之中也,为荥;注于太白,太白腕骨之下也,为俞;行于商丘,商丘内踝之下陷者之中也,为经;入于阴之陵泉,阴之陵泉,辅骨之下陷者之中也,伸而得之,为合。足太阴也。
	
	肾出于涌泉,涌泉者足心也,为井木;溜于然谷,然谷,然骨之下者也,为荥;注于太溪,太溪内踝之后跟骨之上陷中者也,为俞;行于复溜,复溜,上内踝二寸,动而不休,为经;入于阴谷,阴谷,辅骨之后,大筋之下,小筋之上也,按之应手,屈膝而得之,为合。足少阴经也。
	
	膀胱出于至阴,至阴者,足小趾之端也,为井金;溜于通谷,通谷,本节之前外侧也,为荥;注于束骨,束骨,本节之后陷者中也,为俞;过于京骨,京骨,足外侧大骨之下,为原;行于昆仑,昆仑,在外踝之后,跟骨之上,为经;入于委中,委中,腘中央,为合,委而取之。足太阳也。
	
	胆出于窍阴,窍阴者,足小趾次趾之端也,为井金;溜于侠溪,侠溪,足小趾次趾之间也,为荥;注于临泣,临泣,上行一寸半,陷者中也,为俞;过于丘墟,丘墟,外踝之前下陷者中也,为原。行于阳辅,阳辅外踝之上辅骨之前及绝骨之端也,为经;入于阳之陵泉,阳之陵泉,在膝外陷者中也,为合,伸而得之。足少阳也。
	
	胃出于厉兑,厉兑者,足大趾内次趾之端也,为井金;溜于内庭,内庭,次趾外间也,为荥;注于陷谷,陷谷者,上中指内间上行二寸陷者中也,为俞;过于冲阳,冲阳,足跗上五寸陷者中也,为原,摇足而得之;行于解溪,解溪,上冲阳一寸半陷者中也,为经;入于下陵,下陵,膝下三寸胻骨外三里也,为合;复下三里三寸,为巨虚上廉,复下上廉三寸,为巨虚下廉也;大肠属上,小肠属下,足阳明胃脉也。大肠小肠,皆属于胃,是足阳明也。
	
	三焦者,上合手少阳,出于关冲,关冲者,手小指次指之端也,为井金;溜于液门,液门,小指次指之间也,为荥;注于中渚,中渚,本节之后陷者中也,为俞;过于阳池,阳池,在腕上陷者之中也,为原;行于支沟,支沟,上腕三寸两骨之间陷者中也,为经;入于天井,天井,在肘外大骨之上陷者中也,为合,屈肘而得之;三焦下腧在于足大趾之前,少阳之后,出于腘中外廉,名曰委阳,是太阳络也,手少阳经也。三焦者,足少阳太阴之所将,太阳之别也,上踝五寸,别入贯腨肠,出于委阳,并太阳之正,入络膀胱,约下焦,实则闭癃,虚则遗溺,遗溺则补之,闭癃则泻之。
	
	手太阳小肠者,上合手太阳,出于少泽,少泽,小指之端也,为井金;溜于前谷,前谷,在手外廉本节前陷者中也,为荥;注于后溪,后溪者,在手外侧本节之后也,为俞;过于腕骨,腕骨,在手外侧腕骨之前,为原;行于阳谷,阳谷,在锐骨之下陷者中也,为经;入于小海,小海,在肘内大骨之外,去端半寸,陷者中也,伸臂而得之,为合。手太阳经也。
	
	大肠上合手阳明,出于商阳,商阳,大指次指之端也,为井金;溜于本节之前二间,为荥;注于本节之后三间,为俞;过于合谷,合谷,在大指岐骨之间,为原;行于阳溪,阳溪,在两筋间陷者中也,为经;入于曲池,在肘外辅骨陷者中,屈臂而得之,为合。手阳明也。
	
	是谓五脏六腑之俞,五五二十五俞,六六三十六俞也。六腑皆出足之三阳,上合于手者也。
	
	缺盆之中,任脉也,名曰天突。一次,任脉侧之动脉足阳明也,名曰人迎;二次脉,手阳明也,名曰扶突;三次脉,手太阳也,名曰天窗;四次脉,足少阳也,名曰天容;五次脉,手少阳也,名曰天牖;六次脉,足太阳也,名曰天柱;七次脉,颈中央之脉,督脉也,名曰风府。腋内动脉手太阴也,名曰天府。腋下三寸手心主也,名曰天池。
	
	刺上关者,呿不能欠。刺下关者,欠不能呿。刺犊鼻者,屈不能伸。刺两关者,伸不能屈。
	
	足阳明,侠喉之动脉也,其俞在膺中。手阳明,次在其俞外,不至曲颊一寸。手太阳当曲颊。足少阳在耳下曲颊之后。手少阳出耳后上加完骨之上。足太阳侠项大筋之中,发际。
	
	阴尺动脉,在五里,五俞之禁也。
	
	肺合大肠,大肠者,传道之腑。心合小肠,小肠者,受盛之腑。肝合胆,胆者中精之腑。脾合胃,胃者五谷之腑。肾合膀胱,膀胱者津液之腑也。少阳属肾,肾上连肺,故将两脏。三焦者,中渎之腑也,水道出焉,属膀胱,是孤之腑也,是六腑之所与合者。
	
	春取络脉诸荥大经分肉之间,甚者深取之,间者浅取之。夏取诸俞孙络肌肉皮肤之上。秋取诸合,余如春法。冬取诸井诸俞之分,欲深而留之。此四时之序,气之所处,病之所舍,脏之所宜。转筋者,立而取之,可令遂已。痿厥者,张而刺之,可令立快也。
	\end{yuanwen}
\begin{yuanwen}
	
\end{yuanwen}
\begin{yuanwen}
	
\end{yuanwen}
\begin{yuanwen}
	
\end{yuanwen}
\begin{yuanwen}
	
\end{yuanwen}
\begin{yuanwen}
	
\end{yuanwen}
\begin{yuanwen}
	
\end{yuanwen}
\begin{yuanwen}
	
\end{yuanwen}
\begin{yuanwen}
	
\end{yuanwen}
\begin{yuanwen}
	
\end{yuanwen}
\begin{yuanwen}
	
\end{yuanwen}
\begin{yuanwen}
	
\end{yuanwen}
\begin{yuanwen}
	
\end{yuanwen}
\begin{yuanwen}
	
\end{yuanwen}
\begin{yuanwen}
	
\end{yuanwen}
\begin{yuanwen}
	
\end{yuanwen}
\begin{yuanwen}
	
\end{yuanwen}
\begin{yuanwen}
	
\end{yuanwen}
\begin{yuanwen}
	
\end{yuanwen}
\begin{yuanwen}
	
\end{yuanwen}
\begin{yuanwen}
	
\end{yuanwen}
\begin{yuanwen}
	
\end{yuanwen}
\begin{yuanwen}
	
\end{yuanwen}
\begin{yuanwen}
	
\end{yuanwen}
\begin{yuanwen}
	
\end{yuanwen}
\begin{yuanwen}
	
\end{yuanwen}
\begin{yuanwen}
	
\end{yuanwen}
\begin{yuanwen}
	
\end{yuanwen}
\begin{yuanwen}
	
\end{yuanwen}
\begin{yuanwen}
	
\end{yuanwen}
\begin{yuanwen}
	
\end{yuanwen}
\begin{yuanwen}
	
\end{yuanwen}
\begin{yuanwen}
	
\end{yuanwen}
\begin{yuanwen}
	
\end{yuanwen}
\begin{yuanwen}
	
\end{yuanwen}
\begin{yuanwen}
	
\end{yuanwen}
\begin{yuanwen}
	
\end{yuanwen}
\begin{yuanwen}
	
\end{yuanwen}
\begin{yuanwen}
	
\end{yuanwen}




\begin{yuanwen}
	
\end{yuanwen}

\begin{yuanwen}
	
\end{yuanwen}

\begin{yuanwen}
	
\end{yuanwen}
\begin{yuanwen}
	
\end{yuanwen}
\begin{yuanwen}
	
\end{yuanwen}
\begin{yuanwen}
	
\end{yuanwen}
\begin{yuanwen}
	
\end{yuanwen}
\begin{yuanwen}
	
\end{yuanwen}
\begin{yuanwen}
	
\end{yuanwen}
\begin{yuanwen}
	
\end{yuanwen}
\begin{yuanwen}
	
\end{yuanwen}
\begin{yuanwen}
	
\end{yuanwen}
\begin{yuanwen}
	
\end{yuanwen}
\begin{yuanwen}
	
\end{yuanwen}
\begin{yuanwen}
	
\end{yuanwen}
\begin{yuanwen}
	
\end{yuanwen}
\begin{yuanwen}
	
\end{yuanwen}
\begin{yuanwen}
	
\end{yuanwen}
\begin{yuanwen}
	
\end{yuanwen}
\begin{yuanwen}
	
\end{yuanwen}
\begin{yuanwen}
	
\end{yuanwen}
\begin{yuanwen}
	
\end{yuanwen}
\begin{yuanwen}
	
\end{yuanwen}
\begin{yuanwen}
	
\end{yuanwen}
\begin{yuanwen}
	
\end{yuanwen}
\begin{yuanwen}
	
\end{yuanwen}
\begin{yuanwen}
	
\end{yuanwen}
\begin{yuanwen}
	
\end{yuanwen}
\begin{yuanwen}
	
\end{yuanwen}
\begin{yuanwen}
	
\end{yuanwen}
\begin{yuanwen}
	
\end{yuanwen}
\begin{yuanwen}
	
\end{yuanwen}
\begin{yuanwen}
	
\end{yuanwen}
\begin{yuanwen}
	
\end{yuanwen}
\begin{yuanwen}
	
\end{yuanwen}
\begin{yuanwen}
	
\end{yuanwen}
\begin{yuanwen}
	
\end{yuanwen}
\begin{yuanwen}
	
\end{yuanwen}
\begin{yuanwen}
	
\end{yuanwen}
\begin{yuanwen}
	
\end{yuanwen}
\begin{yuanwen}
	
\end{yuanwen}
\begin{yuanwen}
	
\end{yuanwen}
\begin{yuanwen}
	
\end{yuanwen}
\begin{yuanwen}
	
\end{yuanwen}
\begin{yuanwen}
	
\end{yuanwen}

	
	
	\chapter{小针解}
	
	\begin{yuanwen}
	所谓易陈者,易言也。难入者,难着于人也。粗守形者,守刺法也。上守神者,守人之血气有余不足可补泻也。神客者,正邪共会也。神者,正气也,客者邪气也。在门者,邪循正气之所出入也。未睹其疾者,先知邪正何经之疾也。恶知其原者,先知何经之病所取之处也。
	
	刺之微在数迟者,徐疾之意也。粗守关者,守四支而不知血气正邪之往来也。上守机者,知守气也。机之动不离其空中者,知气之虚实,用针之徐疾也。空中之机,清静以微者,针以得气,密意守气勿失也。其来不可逢者,气盛不可补也。其往不可追者,气虚不可泻也。不可挂以发者,言气易失也。扣之不发者,言不知补泻之意也。血气已尽而气不下也。
	
	知其往来者,知气之逆顺盛虚也。要与之期者,知气之可取之时也。粗之闇者,冥冥不知气之微密也。妙哉!工独有之者,尽知针意也。往者为逆者,言气之虚而小,小者逆也。来者为顺者,言形气之平,平者顺也。明知逆顺正行无问者,言知所取之处也。迎而夺之者,泻也;追而济之者,补也。
	
	所谓虚则实之者,气口虚而当补之也。满则泄之者,气口盛而当泻之也。宛陈则除之者,去血脉也。邪胜则虚之者,言诸经有盛者,皆泻其邪也。徐而疾则实者,言徐内而疾出也。疾而徐则虚者,言疾内而徐出也。言实与虚若有若无者,言实者有气,虚者无气也。察后与先若亡若存者,言气之虚实,补泻之先后也,察其气之已下与常存也。为虚为实,若得若失者,言补者佖然若有得也,泻则恍然若有失也。
	
	夫气之在脉也,邪气在上者,言邪气之中人也高,故邪气在上也。浊气在中者,言水谷皆入于胃,其精气上注于肺,浊溜于肠胃,言寒温不适,饮食不节,而病生于肠胃,故命曰浊气在中也。清气在下者,言清湿地气之中人也,必从足始,故曰清气在下也。针陷脉,则邪气出者取之上,针中脉则浊气出者,取之阳明合也。针太深则邪气反沉者,言浅浮之病,不欲深刺也。深则邪气从之入,故曰反沉也。皮肉筋脉各有所处者,言经络各有所主也。取五脉者死,言病在中气不足,但用针尽大泻其诸阴之脉也。取三阳之脉者,唯言尽泻三阳之气,令病人恇然不复也。夺阴者死,言取尺之五里五往者也。夺阳者狂,正言也。
	
	睹其色,察其目,知其散复,一其形,听其动静者,言上工知相五色于目。有知调尺寸小大缓急滑涩以言所病也。知其邪正者,知论虚邪与正邪之风也。右主推之,左持而御之者,言持针而出入也。气至而去之者,言补泻气调而去之也。调气在于终始一者,持心也。节之交三百六十五会者,络脉之渗灌诸节者也。
	
	所谓五脏之气,已绝于内者,脉口气内绝不至,反取其外之病处,与阳经之合,有留针以致阳气,阳气至则内重竭,重竭则死矣。其死也,无气以动,故静。所谓五脏之气,已绝于外者,脉口气外绝不至,反取其四末之输,有留针以致其阴气,阴气至则阳气反入,入则逆,逆则死矣。其死也,阴气有余,故躁。
	
	所以察其目者,五脏使五色循明。循明则声章。声章者,则言声与平生异也。
	\end{yuanwen}
	
	\begin{yuanwen}
		
	\end{yuanwen}
	
	\begin{yuanwen}
		
	\end{yuanwen}
	
	\begin{yuanwen}
		
	\end{yuanwen}
	\begin{yuanwen}
		
	\end{yuanwen}
	\begin{yuanwen}
		
	\end{yuanwen}
	\begin{yuanwen}
		
	\end{yuanwen}
	\begin{yuanwen}
		
	\end{yuanwen}
	\begin{yuanwen}
		
	\end{yuanwen}
	\begin{yuanwen}
		
	\end{yuanwen}
	\begin{yuanwen}
		
	\end{yuanwen}
	\begin{yuanwen}
		
	\end{yuanwen}
	\begin{yuanwen}
		
	\end{yuanwen}
	\begin{yuanwen}
		
	\end{yuanwen}
	\begin{yuanwen}
		
	\end{yuanwen}
	\begin{yuanwen}
		
	\end{yuanwen}
	\begin{yuanwen}
		
	\end{yuanwen}
	\begin{yuanwen}
		
	\end{yuanwen}
	\begin{yuanwen}
		
	\end{yuanwen}
	\begin{yuanwen}
		
	\end{yuanwen}
	\begin{yuanwen}
		
	\end{yuanwen}
	\begin{yuanwen}
		
	\end{yuanwen}
	\begin{yuanwen}
		
	\end{yuanwen}
	\begin{yuanwen}
		
	\end{yuanwen}
	\begin{yuanwen}
		
	\end{yuanwen}
	\begin{yuanwen}
		
	\end{yuanwen}
	\begin{yuanwen}
		
	\end{yuanwen}
	\begin{yuanwen}
		
	\end{yuanwen}
	\begin{yuanwen}
		
	\end{yuanwen}
	\begin{yuanwen}
		
	\end{yuanwen}
	\begin{yuanwen}
		
	\end{yuanwen}
	\begin{yuanwen}
		
	\end{yuanwen}
	\begin{yuanwen}
		
	\end{yuanwen}
	\begin{yuanwen}
		
	\end{yuanwen}
	\begin{yuanwen}
		
	\end{yuanwen}
	\begin{yuanwen}
		
	\end{yuanwen}
	\begin{yuanwen}
		
	\end{yuanwen}
	\begin{yuanwen}
		
	\end{yuanwen}
	\begin{yuanwen}
		
	\end{yuanwen}
	\begin{yuanwen}
		
	\end{yuanwen}
	\begin{yuanwen}
		
	\end{yuanwen}
	\begin{yuanwen}
		
	\end{yuanwen}
	\begin{yuanwen}
		
	\end{yuanwen}
	\begin{yuanwen}
		
	\end{yuanwen}
	\begin{yuanwen}
		
	\end{yuanwen}
	\begin{yuanwen}
		
	\end{yuanwen}
	
	
	
\chapter{邪气藏府病形}

本篇主要论述了邪气中人的不同原因和部位,以及中阴中阳的区别;阐述了察色、按脉、问病、诊尺肤等诊法在诊断上的重要性,以及色与脉、脉与尺肤的相应情况;列举了五脏病变的缓、急、大、小、滑、涩六脉及其症状和针刺治疗原则;列举了六腑病变的症状和取穴法与针刺法。是论述邪气与脏腑及疾病症状关系的重要篇章。本篇的关键词是“邪气”,疾病的病因;“脏腑”,疾病伤及的部位;“病形”,疾病的表现症状,因此以《邪气脏腑病形》名篇。
	
\begin{yuanwen}
黄帝问于岐伯曰:邪气之中人也,奈何?
	
岐伯答曰:邪气之中人高也。
	
黄帝曰:高下有度乎?
	
岐伯曰:身半以上者,邪中之也。身半已下者,湿中之也。故曰:邪之中人也,无有常。中于阴则溜于腑,中于阳则溜\footnote{流,淌。}于经。
\end{yuanwen}
	
\begin{yuanwen}
黄帝曰:阴之与阳也,异名同类,上下相会,经络之相贯,如环无端。邪之中人,或中于阴,或中于阳,上下左右,无有恒常,其故何也?
	
岐伯曰:诸阳之会,皆在于面。中人也,方乘虚时,及新用力,若饮食汗出,腠理开,而中于邪。中于面,则下阳明。中于项,则下太阳。中于颊,则下少阳。中于膺背两胁亦中其经。
\end{yuanwen}
	
\begin{yuanwen}
黄帝曰:其中于阴,奈何?
	
岐伯曰:中于阴者,常从臂胻始\footnote{h\'eng,足胫。}。夫臂与胻,其阴皮薄,其肉淖泽\footnote{润泽。在此作“柔软”解。n\`ao。},故俱受于风,独伤其阴。
\end{yuanwen}
	
\begin{yuanwen}
黄帝曰:此故伤脏乎?
	
岐伯答曰:身之中于风也,不必动脏。故邪入于阴经,则其脏气实,邪气入而不能客,故还之于腑。故中阳则溜于经,中阴则溜于腑。
\end{yuanwen}
	
\begin{yuanwen}
黄帝曰:邪之中人脏,奈何?
	
岐伯曰:愁忧恐惧则伤心。形寒寒饮则伤肺\footnote{喻昌说:“肺气外达皮毛,内行水道。形寒则外寒,从皮毛而入;饮冷则水冷从肺上溢,遏抑肺气,不令外扬下达,其治节不行,周身之气,无所禀仰而肺病矣。”}。以其两寒相感,中外皆伤,故气逆而上行。有所堕坠,恶血留内,若有所大怒,气上而不下,积于胁下则伤肝。有所击仆,若醉入房,汗出当风则伤脾。有所用力举重,若入房过度,汗出浴水则伤肾。
	
黄帝曰:五脏之中风,奈何?
	
岐伯曰:阴阳俱感,邪乃得往。
	
黄帝曰:善哉。
\end{yuanwen}
	
\begin{yuanwen}
黄帝问于岐伯曰:首面与身形也,属骨连筋,同血合于气耳。天寒则裂地凌冰,其卒寒,或手足懈惰,然而其面不衣,何也?
	
岐伯答曰:十二经脉,三百六十五络,其血气皆上于面而走空窍。其精阳气上走于目而为睛。其别气走于耳而为听。其宗气上出于鼻而为臭。其浊气出于胃,走唇舌而为味。其气之津液,皆上熏于面,而皮又厚,其肉坚,故热甚,寒不能胜之也。
\end{yuanwen}
	
\begin{yuanwen}
黄帝曰:邪之中人,其病形何如?
	
岐伯曰:虚邪\footnote{四时反常的邪风,即虚邪贼风。}之中身也,洒淅动形。正邪\footnote{四时正常的风气,也能乘人之虚,侵袭人体而引起疾病。}之中人也微,先见于色,不知于身,若有若无,若亡若存,有形无形,莫知其情。
	
黄帝曰:善哉。
\end{yuanwen}
	
\begin{yuanwen}
黄帝问于岐伯曰:余闻之,见其色,知其病,命曰明。按其脉,知其病,命曰神。问其病,知其处,命曰工。余愿见而知之,按而得之,问而极之,为之奈何?
	
岐伯答曰:夫色脉与尺之相应也,如桴鼓\footnote{比喻事物相应,就像用鼓槌击鼓而有声一样。桴,f\'u,鼓槌。}影响之相应也,不得相失也,此亦本末根叶之殊候也,故根死则叶枯矣。色脉形肉不得相失也。故知一则为工,知二则为神,知三则神且明矣。
\end{yuanwen}
	
\begin{yuanwen}
黄帝曰:愿卒闻之。
	
岐伯答曰:色青者,其脉弦\footnote{弦脉端直以长,如张弓弦,为肝脉。}也,赤者,其脉钩\footnote{钩脉来盛去衰,为心脉。}也,黄者,其脉代\footnote{此处为脾之平脉,有更代的意思。}也,白者,其脉毛\footnote{毛脉轻虚而浮,为肺脉。}也,黑者,其脉石\footnote{石脉沉濡而滑,为肾脉。}。见其色而不得其脉,反得其相胜之脉\footnote{相胜就说相克,如肝病见肺之毛脉,是金克木,这就是相胜之脉。}则死矣;得其相生之脉\footnote{如肝病见肾之石脉,是水生木,即为相生之脉。}则病已矣。
\end{yuanwen}
	
\begin{yuanwen}
黄帝问于岐伯曰:五脏之所生,变化之病形,何如?
	
岐伯答曰:先定其五色五脉之应,其病乃可别也。
	
黄帝曰:色脉已定,别之奈何?
	
岐伯说:调其脉之缓急、小大、滑涩,而病变定矣。
\end{yuanwen}
	
\begin{yuanwen}
黄帝曰:调之奈何?
	
岐伯答曰:脉急者,尺之皮肤亦急;脉缓者,尺之皮肤亦缓;脉小者,尺之皮肤亦减而少;脉大者,尺之皮肤亦贲\footnote{大。}而起;脉滑者,尺之皮肤亦滑;脉涩者,尺之皮肤亦涩。凡此变者,有微有甚,故善调尺者,不待于寸;善调脉者,不待于色。能参合而行之者,可以为上工,上工十全九。行二者为中工,中工十全七。行一者为下工,下工十全六。
\end{yuanwen}
	
	\begin{yuanwen}
	
	黄帝曰:请问脉之缓、急,小、大,滑、涩之病形何如?
	
	岐伯曰:臣请言五藏之病变也。心脉急甚者为瘈瘲;征急,为心痛引背,食不下。缓甚,为狂笑;微缓,为伏梁,在心下,上下行,时唾血。大甚,为喉吤;微大,为心痹引背,善泪出。小甚为善哕;微小为消病。滑甚为善渴;微滑为心疝,引脐,小腹鸣。涩为为瘖;微涩为血溢,维厥耳鸣,颠疾。
	
	肺脉急甚,为癫疾;微急,为肺寒热,怠惰,咳唾血,引腰背胸,若鼻息肉不通。缓甚,为多汗;微缓,为痿,痿,偏风,头以下汗出不可止。大甚,为胫肿;微大,为肺痹,引胸背,起恶见日光。小甚,为泄;微小,为消痹。滑甚,为息贲上气;微滑,为上下出血。涩甚,为呕血;微涩,为鼠?,在颈支腋之间,下不胜其上,其应善酸矣。
	
	肝脉急甚者为恶言;微急为肥气在胁下,若复杯。缓甚为善呕,微缓为水瘕痹也。大甚为内痈,善呕衄;微大为肝痹,阴缩,咳引小腹。小甚为多饮;微小为消瘅。滑甚为(疒贵)疝;微滑为遗溺。涩甚为溢饮;微涩为瘈挛筋痹。
	
	脾脉急甚为瘈瘲;微急为膈中,食饮入而还出,后沃沫。缓甚为痿厥;微缓为风痿,四肢不用,心慧然若无病。大甚为击仆;微大为疝气,腹里大脓血在肠胃之外。小甚为寒热;微小为消瘅。滑甚为(疒贵)癃;微滑为虫毒蛕蝎腹热。涩甚为肠(疒贵);微涩为内(疒贵),多下脓血。
	
	肾脉急甚为骨癫疾;微急为沉厥奔豚,足不收,不得前后。缓甚为折脊;微缓为洞,洞者,食不化,下嗌逐出。大甚为阴痿;微大为石水,起脐已下至小腹睡睡然,上至胃脘,死不治。小甚为洞泄;微小为消瘅。滑甚为癃(疒贵);微滑为骨痿,坐不能起,起则目无所见。涩甚为大痈;微涩为不月,沉痔。
	\end{yuanwen}
	
\begin{yuanwen}
黄帝曰:病之六变者,刺之奈何?
	
岐伯答曰:诸急\footnote{紧脉。}者多寒,缓者多热,大者多气少血,小者血气皆少,滑者阳气盛,微有热,涩者多血少气,微有寒。是故刺急者,深内\footnote{深刺。内,同“纳”,指进针。}而久留之;刺缓者,浅内\footnote{浅刺。}而疾发针,以去其热;刺大者,微泻其气,无出其血;刺滑者,疾发针而浅内之,以泻其阳气而去其热;刺涩者,必中其脉,随其逆顺而久留之,必先按而循之,已发针,疾按其痏\footnote{w\v{e}i,泛指针孔。},无令其血出,以和其脉;诸小者,阴阳形气俱不足,勿取以针,而调以甘药也。
\end{yuanwen}
	
\begin{yuanwen}
黄帝曰:余闻五脏六腑之气,荥输所入为合,令何道从入,入安连过?愿闻其故。
	
岐伯答曰:此阳脉之别入于内,属于腑者也。
	
黄帝曰:荥输与合,各有名乎?
	
岐伯答曰:荥输治外经,合治内腑。
\end{yuanwen}
	
\begin{yuanwen}
黄帝曰:治内腑奈何?
	
岐伯曰:取之于合。
	
黄帝曰:合各有名乎?
	
岐伯答曰:胃合于三里,大肠合入于巨虚上廉,小肠合入于巨虚下廉,三焦合入于委阳,膀胱合入于委中央,胆合入于阳陵泉。
\end{yuanwen}
	
\begin{yuanwen}
黄帝曰:取之奈何?
	
岐伯答曰:取之三里者,低跗\footnote{马莳说:“取三里者,将足之跗面低下著地而取之,不使之举足。”跗,f\=u,足背部。}取之;巨虚者,举足取之;委阳者,屈伸而索之;委中者,屈而取之;阳陵泉者,正竖膝,予之齐,下至委阳之阳取之;取诸外经者,揄申而从之\footnote{y\'ao。周学海说:“《骨空论》注云:揄,摇也。谓或摇或伸而寻之。”}。
\end{yuanwen}
	
\begin{yuanwen}
黄帝曰:愿闻六腑之病。
	
岐伯答曰:面热者,足阳明病;鱼络血者\footnote{是说手鱼的部位血脉郁滞或有淤斑。},手阳明病;两跗之上脉竖陷者,足阳明病。此胃脉也。
\end{yuanwen}
	
\begin{yuanwen}
大肠病者,肠中切痛而鸣濯濯\footnote{zhu\'o,偿命的声音。}。冬日重感于寒即泄,当脐而痛,不能久立。与胃同候,取巨虚上廉。
\end{yuanwen}
	
\begin{yuanwen}
胃病者,腹䐜\footnote{ch\=en}胀,胃脘当心而痛,上支两胁,膈咽不通,食饮不下,取之三里也。
\end{yuanwen}
	
\begin{yuanwen}
小肠病者,小腹痛,腰脊控睾而痛,时窘之后\footnote{大便的避讳语。},当耳前热,若寒甚,若独肩上热甚,及手小指次指之间热,若脉陷者,此其候也。手太阳病也,取之巨虚下廉。
\end{yuanwen}
	
\begin{yuanwen}
三焦病者,腹气满,小腹尤坚,不得小便,窘急,溢则水,留即为胀。候在足太阳之外大络,大络在太阳少阳之间,亦见于脉,取委阳。
\end{yuanwen}
	
\begin{yuanwen}
膀胱病者,小腹\footnote{中医以脐下三寸以下为小腹。}偏肿而痛,以手按之,即欲小便而不得,肩上热,若脉陷,及足小趾外廉及胫踝后皆热,若脉陷,取委中央。
\end{yuanwen}
	
\begin{yuanwen}
胆病者,善太息,口苦,呕宿汁,心下澹澹\footnote{d\`an,水波动貌。这里指心慌心跳。}恐人将捕之,嗌中吤吤然\footnote{ji\`e。咽喉中如有物作梗,咯吐不舒。},数唾。在足少阳之本末,亦视其脉之陷下者灸之,其寒热者取阳陵泉。
\end{yuanwen}
	
\begin{yuanwen}
黄帝曰:刺之有道乎?
	
岐伯答曰:刺此者,必中气穴\footnote{腧穴。腧穴和经气相通,故称气穴。},无中肉节\footnote{肌肉之间的节界。张景岳:“肉有节界,是谓肉节。”}。中气穴则针游于巷\footnote{即刺中穴位后,即沿着经脉循行路线出现针感。张景岳:“巷,道也。中其气穴,则针着脉道而经络通。”};中肉节即皮肤痛。补泻反则病益笃,中筋则筋缓,邪气不出,与其真相搏,乱而不去,反还内著。用针不审,以顺为逆也。
\end{yuanwen}
	
	
	\part{}
	
	\chapter{根结}
	
	\begin{yuanwen}
	岐伯曰:天地相感,寒暖相移,阴阳之道,孰少孰多,阴道偶,阳道奇。发于春夏,阴气少,阳气多,阴阳不调,何补何泻。发于秋冬,阳气少,阴气多;阴气盛而阳气衰,故茎叶枯槁,湿雨下归,阴阳相移,何泻何补。奇邪离经,不可胜数,不知根结,五脏六腑,折关败枢,开合而走,阴阳大失,不可复取。九针之玄,要在终始;故能知终始,一言而毕,不知终始,针道咸绝。
	
	太阳根于至阴,结于命门。命门者,目也。阳明根于厉兑,结于颡大。颡大者,钳耳也。少阳根于窍阴,结于窗笼。窗笼者,耳中也。太阳为开,阳明为合,少阳为枢,故开折,则肉节渎而暴病起矣。故暴病者,取之太阳,视有余不足。渎者,皮肉宛膲而弱也。合折,则气无所止息而痿疾起矣。故痿疾者,取之阳明,视有余不足。无所止息者,真气稽留,邪气居之也。枢折,即骨繇而不安于地。故骨繇者,取之少阳,视有余不足。骨繇者,节缓而不收也。所谓骨繇者,摇故也。当窃其本也。
	
	太阴根于隐白,结于太仓。少阴根于涌泉,结于廉泉。厥阴根于大敦,结于玉英,络于膻中。太阴为合,少阳为枢。故开折,则仓廪无所输,膈洞。膈洞者,取之太阴,视有余不足,故开折者,气不足而生病也。合折,即气绝而喜悲。悲者取之厥阴,视有余不足。枢折,则脉有所结而不通。不通者,取之少阴,视有余不足,有结者,皆取之不足。
	
	足太阳根于至阴,溜于京骨,注于昆仑,入于天柱、飞扬也。足少阳根于窍阴,溜于丘墟,注于阳辅,入于天容、光明也。足阳明根于厉兑,溜于冲阳,注于下陵,入于人迎,丰隆也。手太阳根于少泽,溜于阳谷,注于小海,入于天窗,支正也。少阳根于关冲,溜于阳池,注于支沟,入于天牖、外关也。手阳明根于商阳,溜于合谷,注于阳溪,入于扶突、偏历也。此所谓十二经者,盛络皆当取之。
	
	一日一夜五十营,以营五脏之精,不应数者,名曰狂生。所谓五十营者,五脏皆受气,持其脉口,数其至也。五十动而不一代者,五脏皆受气。四十动一代者,一脏无气。三十动一代者,二脏无气。二十动一代者,三脏无气。十动一代者,四脏无气。不满十动一代者,五脏无气。予之短期,要在终始。所谓五十动而不一代者,以为常也。以知五脏之期,予之短期者,乍数乍疏也。
	
	黄帝曰:逆顺五体者,言人骨节之大小,肉之坚脆,皮之厚薄,血之清浊,气之滑涩,脉之长短,血之多少,经络之数,余已知之矣,此皆布衣匹夫之士也。夫王公大人,血食之君,身体柔脆,肌肉软弱,血气慓悍滑利,其刺之徐疾浅深多少,可得同之乎。
	
	岐伯答曰:膏梁菽藿之味,何可同也?气滑即出疾,其气涩则出迟,气悍则针小而入浅,气涩则针大而入深,深则欲留,浅则欲疾。以此观之,刺布衣者,深以留之,刺大人者,微以徐之,此皆因气慓悍滑利也。
	
	黄帝曰:形气之逆顺奈何?
	
	岐伯曰:形气不足,病气有余,是邪胜也,急泻之;形气有余,病气不足,急补之;形气不足,病气不足,此阴阳气俱不足也,不可刺之,刺之则重不足。重不足则阴阳俱竭,血气皆尽,五脏空虚,筋骨髓枯,老者绝灭,壮者不复矣。形气有余,病气有余,此谓阴阳俱有余也。急泻其邪,调其虚实。故曰:有余者泻之,不足者补之,此之谓也。
	
	故曰:刺不知逆顺,真邪相搏。满而补之,则阴阳四溢,肠胃充郭,肝肺内(月真),阴阳相错。虚而泻之,则经脉空虚,血气竭枯,肠胃(亻耳耳耳)辟,皮肤薄着,毛腠夭膲,予之死期。
	
	故曰:用针之要,在于知调阴与阳。调阴与阳,精气乃光,合形与气,使神内藏。
	
	故曰:上工平气,中工乱脉,下工绝气危生。
	
	故曰:下工不可不慎也,必审五藏变化之病,五脉之应,经络之实虚,皮之柔麤,而后取之也。
	\end{yuanwen}
	
	\begin{yuanwen}
		
	\end{yuanwen}
	
	\begin{yuanwen}
		
	\end{yuanwen}
	
	\begin{yuanwen}
		
	\end{yuanwen}
	\begin{yuanwen}
		
	\end{yuanwen}
	\begin{yuanwen}
		
	\end{yuanwen}
	\begin{yuanwen}
		
	\end{yuanwen}
	\begin{yuanwen}
		
	\end{yuanwen}
	\begin{yuanwen}
		
	\end{yuanwen}
	\begin{yuanwen}
		
	\end{yuanwen}
	\begin{yuanwen}
		
	\end{yuanwen}
	\begin{yuanwen}
		
	\end{yuanwen}
	\begin{yuanwen}
		
	\end{yuanwen}
	\begin{yuanwen}
		
	\end{yuanwen}
	\begin{yuanwen}
		
	\end{yuanwen}
	\begin{yuanwen}
		
	\end{yuanwen}
	\begin{yuanwen}
		
	\end{yuanwen}
	\begin{yuanwen}
		
	\end{yuanwen}
	\begin{yuanwen}
		
	\end{yuanwen}
	\begin{yuanwen}
		
	\end{yuanwen}
	\begin{yuanwen}
		
	\end{yuanwen}
	\begin{yuanwen}
		
	\end{yuanwen}
	\begin{yuanwen}
		
	\end{yuanwen}
	\begin{yuanwen}
		
	\end{yuanwen}
	\begin{yuanwen}
		
	\end{yuanwen}
	\begin{yuanwen}
		
	\end{yuanwen}
	\begin{yuanwen}
		
	\end{yuanwen}
	\begin{yuanwen}
		
	\end{yuanwen}
	\begin{yuanwen}
		
	\end{yuanwen}
	\begin{yuanwen}
		
	\end{yuanwen}
	\begin{yuanwen}
		
	\end{yuanwen}
	\begin{yuanwen}
		
	\end{yuanwen}
	\begin{yuanwen}
		
	\end{yuanwen}
	\begin{yuanwen}
		
	\end{yuanwen}
	\begin{yuanwen}
		
	\end{yuanwen}
	\begin{yuanwen}
		
	\end{yuanwen}
	\begin{yuanwen}
		
	\end{yuanwen}
	\begin{yuanwen}
		
	\end{yuanwen}
	\begin{yuanwen}
		
	\end{yuanwen}
	\begin{yuanwen}
		
	\end{yuanwen}
	\begin{yuanwen}
		
	\end{yuanwen}
	\begin{yuanwen}
		
	\end{yuanwen}
	\begin{yuanwen}
		
	\end{yuanwen}
	\begin{yuanwen}
		
	\end{yuanwen}
	\begin{yuanwen}
		
	\end{yuanwen}
	\begin{yuanwen}
		
	\end{yuanwen}
	
	
	
\chapter{寿夭刚柔}

本篇主要论述人的体质有刚柔的不同,而“刚”和“柔”可以从形体的缓急、正气的盛衰、骨骼的大小、肌肉的坚脆、皮肤的厚薄等方面进行分辨。体质刚柔不但与发病和治疗密切相关,而且与人的寿命长短有着直接联系,因此观察形气是否相称也可以预测寿命的长短。由于文中内容以“寿夭刚柔”为主,故以此名篇。本篇特别详尽地论述了“形”与“气”的关系。形气是中医学及中国哲学的一对重要范畴。中医和中国哲学认为事物包括“形”“气”两方面。“形”为事物的载体,“气”为事物生存的动力,形气应该和谐相称。在两者之中,气是事物的本质,决定事物的性质和状态以及存亡。因此,中医学极为重视形气的相称、和谐。特别看重气对人体生命的意义,强调气对治疗和养生的意义。

\begin{yuanwen}
黄帝问于少师曰\footnote{相传为黄帝之臣。}:余闻人之生也,有刚有柔,有弱有强,有短有长,有阴有阳,愿闻其方。
	
少师答曰:阴中有阴,阳中有阳,审知阴阳,刺之有方。得病所始,刺之有理。谨度\footnote{du\'o,推测,衡量。}病端\footnote{有“本”、“始”的含义。},与时相应。内合于五脏六腑,外合于筋骨皮肤。是故内有阴阳,外亦有阴阳。在内者,五脏为阴,六腑为阳,在外者,筋骨为阴,皮肤为阳。故曰,病在阴之阴者\footnote{病变部位在脏。内为阴,五脏为阴中之阴。},刺阴之荥输,病在阳之阳者\footnote{病变部位在皮肤。外为阳,皮肤为外之阳,故云阳之阳。},刺阳之合,病在阳之阴者\footnote{病变部位在筋骨。外为阳,筋骨为外之阴。},刺阴之经,病在阴之阳者\footnote{病变部位在腑。内为阴,六腑为阴中之阳。},刺络脉。故曰,病在阳者命曰风,病在阴者命曰痹,阴阳俱病命曰风痹。病有形而不痛者,阳之类也;无形而痛者,阴之类也。无形而痛者,其阳完而阴伤之也,急治其阴,无攻其阳。有形而不痛者,其阴完而阳伤之也,急治其阳,无攻其阴。阴阳俱动,乍有形,乍无形,加以烦心,命曰阴胜其阳。此谓不表不里,其形不久\footnote{即预后不良。}。
\end{yuanwen}
	
\begin{yuanwen}
黄帝问于伯高\footnote{相传为黄帝之臣。}曰:余闻形气,病之先后、外内之应,奈何?
	
伯高答曰:风寒伤形,忧恐忿怒伤气;气伤脏,乃病脏,寒伤形,乃应形;风伤筋脉,筋脉乃应。此形气外内之相应也。
\end{yuanwen}
	
\begin{yuanwen}
黄帝曰:刺之奈何?
	
伯高答曰:病九日者,三刺而已;病一月者,十刺而已;多少远近,以此衰之\footnote{意谓按比数递减。马元台:“人之感病不同,日数各有多少远近,以此大略,病三日而刺一次者之法,等而杀之。”衰之,在此有“减少”的含义。}。久痹不去身者,视其血络,尽出其血。
	
黄帝曰:外内之病,难易之治,奈何?
	
伯高答曰:形先病而未入脏者,刺之半其日。脏先病而形乃应者,刺之倍其日。此外内难易之应也。
\end{yuanwen}
	
\begin{yuanwen}
黄帝问于伯高曰:余闻形有缓急,气有盛衰,骨有大小,肉有坚脆,皮有厚薄,其以立寿夭,奈何?
	
伯高答曰:形与气相任\footnote{相当,相称。}则寿,不相任则夭。皮与肉相裹则寿,不相裹则夭,血气经络胜形\footnote{血气经络不但与外形相称,而且要更为强盛才能长寿。}则寿,不胜形则夭。
\end{yuanwen}
	
\begin{yuanwen}
黄帝曰:何谓形之缓急?
	
伯高答曰:形充而皮肤缓者则寿,形充而皮肤急者则夭。形充而脉坚大者顺也,形充而脉小以弱者气衰,衰则危矣。若形充而颧不起者骨小,骨小则夭矣。形充而大肉䐃\footnote{j\`un,肌肉突起处。}坚而有分者肉坚,肉坚则寿矣;形充而大肉无分理不坚者肉脆,肉脆则夭矣。此天之生命,所以立形定气而视寿夭者,必明乎此立形定气,而后以临病人,决死生。
\end{yuanwen}
	
\begin{yuanwen}
黄帝曰:余闻寿夭,无以度之。
	
伯高答曰:墙基卑,高不及其地者\footnote{以比喻的方法来说明面部形态。墙基,在此指耳边下部。地,指耳前肌肉。大意是说面部肌肉陷下,四周骨骼显露。},不满三十而死。其有因加疾者,不及二十而死也。
	
黄帝曰:形气之相胜,以立寿夭奈何?
	
伯高答曰:平人而气胜形者寿;病而形肉脱,气胜形者死,形胜气者危矣。
	\end{yuanwen}
	
	\begin{yuanwen}
	黄帝曰:余闻刺有三变,何谓三变?伯高答曰:有刺营者,有刺卫者,有刺寒痹之留经者。
	
	黄帝曰:刺三变者奈何?伯高答曰:刺营者出血,刺卫者出气,刺寒痹者内热。
	
	黄帝曰:营卫寒痹之为病奈何?伯高答曰:营之生病也,寒热少气,血上下行。卫之生病也,气痛时来时去,怫忾贲响,风寒客于肠胃之中。寒痹之为病也,留而不去,时痛而皮不仁。
	
	黄帝曰:刺寒痹内热奈何?伯高答曰:刺布衣者,以火焠之;刺大人者,以药熨之。
	
	黄帝曰:药熨奈何?伯高答曰:用淳酒二十斤,蜀椒一斤,干姜一斤,桂心一斤,凡四种,皆嚼咀,渍酒中,用绵絮一斤,细白布四丈,并内酒中,置酒马矢熅中,封涂封,勿使泄。五日五夜,出绵絮曝干之,干复溃,以尽其汁。每渍必晬其日,乃出干。干,并用滓与绵絮,复布为复巾,长六七尺,为六七巾,则用之生桑炭炙巾,以熨寒痹所刺之处,令热入至于病所,寒复炙巾以熨之,三十遍而止。汗出以巾拭身,亦三十遍而止。起步内中,无见风。每刺必熨,如此病已矣。
	\end{yuanwen}

	
	\chapter{官针}
	\begin{yuanwen}
	凡刺之要,官针最妙。九针之宜,各有所为,长、短、大、小,各有所施也。不得其用,病弗能移。疾浅针深,内伤良肉,皮肤为痈;病深针浅,病气不泻,支为大脓。病小针大,气泻太甚,疾必为害;病大针小,气不泄泻,亦复为败。失针之宜。大者泻,小者不移。已言其过,请言其所施。
	
	病在皮肤无常处者,取以镵针于病所,肤白勿取。病在分肉间,取以圆针于病所。病在经络痼痹者,取以锋针。病在脉,气少,当补之者,取以鍉针于井荥分俞。病为大脓者,取以铍针。病痹气暴发者,取以圆利针。病痹气痛而不去者,取以毫针。病在中者,取以长针。病水肿不能通关节者,取以大针。病在五脏固居者,取以锋针,泻于井荥分俞,取以四时。
	
	凡刺有九,以应九变。一曰俞刺,俞刺者,刺诸经荥俞脏俞也;二曰远道刺,远道刺者,病在上,取之下,刺腑俞也;三曰经刺,经刺者,刺大经之结络经分也;四曰络刺,络刺者,刺小络之血脉也;五曰分刺,分刺者,刺分肉之间也;六曰大泻刺,大泻刺者,刺大脓以铍针也;七曰毛刺,毛刺者,刺浮痹皮肤也;八曰巨刺,巨刺者,左取右,右取左;九曰焠刺,焠刺者,刺燔针则取痹也。
	
	凡刺有十二节,以应十二经。一曰偶刺,偶刺者,以手直心若背,直痛所,一刺前,一刺后,以治心痹。刺此者,傍针之也。二曰报刺,报刺者,刺痛无常处也。上下行者,直内无拔针,以左手随病所按之,乃出针,复刺之也。三曰恢刺,恢刺者,直刺傍之,举之前后,恢筋急,以治筋痹也。四曰齐刺,齐刺者,直入一,傍入二,以治寒气小深者;或曰三刺,三刺者,治痹气小深者也。五曰扬刺,扬刺者,正内一,傍内四,而浮之,以治寒气之搏大者也。六曰直针刺,直针刺者,引皮乃刺之,以治寒气之浅者也。七曰输针,输刺者,直入直出,稀发针而深之,以治气盛而热者也。八曰短刺,短刺者,刺骨痹,稍摇而深之,致针骨所,以上下摩骨也。九曰浮刺,浮刺者,傍入而浮之,以治肌急而寒者也。十曰阴刺,阴刺者,左右率刺之,以治寒厥;中寒厥,足踝后少阴也。十一曰傍针刺,傍针刺者,直刺傍刺各一,以治留痹久居者也。十二曰赞刺,赞刺者,直入直出,数发针而浅之,出血是谓治痈肿也。
	
	脉之所居,深不见者,刺之微内针而久留之,以致其空脉气也。脉浅者,勿刺,按绝其脉乃刺之,无令精出,独出其邪气耳。
	
	所谓三刺,则谷气出者。先浅刺绝皮,以出阳邪,再刺则阴邪出者,少益深绝皮,致肌肉,未入分肉间也;已入分肉之间,则谷气出。故刺法曰:始刺浅之,以逐邪气,而来血气,后刺深之,以致阴气之邪,最后刺极深之,以下谷气。此之谓也。
	
	故用针者,不知年之所加,气之盛衰,虚实之所起,不可以为工也。
	
	凡刺有五,以应五脏,一曰半刺,半刺者,浅内而疾发针,无针伤肉,如拔毛状,以取皮气,此肺之应也。
	
	二曰豹文刺,豹文刺者,左右前后针之,中脉为故,以取经络之血者,此心之应也。
	
	三曰关刺,关刺者,直刺左右尽筋上,以取筋痹,慎无出血,此肝之应也;或曰渊刺;一曰岂刺。
	
	四曰合谷刺,合谷刺者,左右鸡足,针于分肉之间,以取肌痹,此脾之应也。
	
	五曰输刺,输刺者,直入直出,深内之至骨,以取骨痹,此肾之应也。
	\end{yuanwen}
	
	\begin{yuanwen}
		
	\end{yuanwen}
	\begin{yuanwen}
		
	\end{yuanwen}
	\begin{yuanwen}
		
	\end{yuanwen}
	\begin{yuanwen}
		
	\end{yuanwen}
	\begin{yuanwen}
		
	\end{yuanwen}
	\begin{yuanwen}
		
	\end{yuanwen}
	\begin{yuanwen}
		
	\end{yuanwen}
	\begin{yuanwen}
		
	\end{yuanwen}
	\begin{yuanwen}
		
	\end{yuanwen}
	\begin{yuanwen}
		
	\end{yuanwen}
	\begin{yuanwen}
		
	\end{yuanwen}
	\begin{yuanwen}
		
	\end{yuanwen}
	\begin{yuanwen}
		
	\end{yuanwen}
	\begin{yuanwen}
		
	\end{yuanwen}
	\begin{yuanwen}
		
	\end{yuanwen}
	
	
	\begin{yuanwen}
		
	\end{yuanwen}
	
	\begin{yuanwen}
		
	\end{yuanwen}
	
	\begin{yuanwen}
		
	\end{yuanwen}
	
	\begin{yuanwen}
		
	\end{yuanwen}
	\begin{yuanwen}
		
	\end{yuanwen}
	\begin{yuanwen}
		
	\end{yuanwen}
	\begin{yuanwen}
		
	\end{yuanwen}
	\begin{yuanwen}
		
	\end{yuanwen}
	\begin{yuanwen}
		
	\end{yuanwen}
	\begin{yuanwen}
		
	\end{yuanwen}
	\begin{yuanwen}
		
	\end{yuanwen}
	\begin{yuanwen}
		
	\end{yuanwen}
	\begin{yuanwen}
		
	\end{yuanwen}
	\begin{yuanwen}
		
	\end{yuanwen}
	\begin{yuanwen}
		
	\end{yuanwen}
	\begin{yuanwen}
		
	\end{yuanwen}
	\begin{yuanwen}
		
	\end{yuanwen}
	\begin{yuanwen}
		
	\end{yuanwen}
	\begin{yuanwen}
		
	\end{yuanwen}
	\begin{yuanwen}
		
	\end{yuanwen}
	\begin{yuanwen}
		
	\end{yuanwen}
	\begin{yuanwen}
		
	\end{yuanwen}
	\begin{yuanwen}
		
	\end{yuanwen}
	\begin{yuanwen}
		
	\end{yuanwen}
	\begin{yuanwen}
		
	\end{yuanwen}
	\begin{yuanwen}
		
	\end{yuanwen}
	\begin{yuanwen}
		
	\end{yuanwen}
	\begin{yuanwen}
		
	\end{yuanwen}
	\begin{yuanwen}
		
	\end{yuanwen}
	\begin{yuanwen}
		
	\end{yuanwen}
	\begin{yuanwen}
		
	\end{yuanwen}
	\begin{yuanwen}
		
	\end{yuanwen}
	\begin{yuanwen}
		
	\end{yuanwen}
	\begin{yuanwen}
		
	\end{yuanwen}
	\begin{yuanwen}
		
	\end{yuanwen}
	\begin{yuanwen}
		
	\end{yuanwen}
	\begin{yuanwen}
		
	\end{yuanwen}
	\begin{yuanwen}
		
	\end{yuanwen}
	\begin{yuanwen}
		
	\end{yuanwen}
	\begin{yuanwen}
		
	\end{yuanwen}
	\begin{yuanwen}
		
	\end{yuanwen}
	\begin{yuanwen}
		
	\end{yuanwen}
	\begin{yuanwen}
		
	\end{yuanwen}
	\begin{yuanwen}
		
	\end{yuanwen}
	\begin{yuanwen}
		
	\end{yuanwen}
	\begin{yuanwen}
		
	\end{yuanwen}
	
	
	
	
	
	\begin{yuanwen}
		
	\end{yuanwen}
	
	\begin{yuanwen}
		
	\end{yuanwen}
	
	\begin{yuanwen}
		
	\end{yuanwen}
	\begin{yuanwen}
		
	\end{yuanwen}
	\begin{yuanwen}
		
	\end{yuanwen}
	\begin{yuanwen}
		
	\end{yuanwen}
	\begin{yuanwen}
		
	\end{yuanwen}
	\begin{yuanwen}
		
	\end{yuanwen}
	\begin{yuanwen}
		
	\end{yuanwen}
	\begin{yuanwen}
		
	\end{yuanwen}
	\begin{yuanwen}
		
	\end{yuanwen}
	\begin{yuanwen}
		
	\end{yuanwen}
	\begin{yuanwen}
		
	\end{yuanwen}
	\begin{yuanwen}
		
	\end{yuanwen}
	\begin{yuanwen}
		
	\end{yuanwen}
	\begin{yuanwen}
		
	\end{yuanwen}
	\begin{yuanwen}
		
	\end{yuanwen}
	\begin{yuanwen}
		
	\end{yuanwen}
	\begin{yuanwen}
		
	\end{yuanwen}
	\begin{yuanwen}
		
	\end{yuanwen}
	\begin{yuanwen}
		
	\end{yuanwen}
	\begin{yuanwen}
		
	\end{yuanwen}
	\begin{yuanwen}
		
	\end{yuanwen}
	\begin{yuanwen}
		
	\end{yuanwen}
	\begin{yuanwen}
		
	\end{yuanwen}
	\begin{yuanwen}
		
	\end{yuanwen}
	\begin{yuanwen}
		
	\end{yuanwen}
	\begin{yuanwen}
		
	\end{yuanwen}
	\begin{yuanwen}
		
	\end{yuanwen}
	\begin{yuanwen}
		
	\end{yuanwen}
	\begin{yuanwen}
		
	\end{yuanwen}
	\begin{yuanwen}
		
	\end{yuanwen}
	\begin{yuanwen}
		
	\end{yuanwen}
	\begin{yuanwen}
		
	\end{yuanwen}
	\begin{yuanwen}
		
	\end{yuanwen}
	\begin{yuanwen}
		
	\end{yuanwen}
	\begin{yuanwen}
		
	\end{yuanwen}
	\begin{yuanwen}
		
	\end{yuanwen}
	\begin{yuanwen}
		
	\end{yuanwen}
	\begin{yuanwen}
		
	\end{yuanwen}
	\begin{yuanwen}
		
	\end{yuanwen}
	\begin{yuanwen}
		
	\end{yuanwen}
	\begin{yuanwen}
		
	\end{yuanwen}
	\begin{yuanwen}
		
	\end{yuanwen}
	\begin{yuanwen}
		
	\end{yuanwen}
	
	
	
\chapter{本神}

本,这里是动词,探究本原、本质的意思。神,一般指精神活动,是心的主要功能,并主宰着整个人体的生命活动。广义的神,还包括肝、肺、脾、肾等脏所主的魂、魄、意、志,以及思、虑、智、忆等精神思维活动在内。本篇对于精神活动的产生、变化,与五脏的关系,以及发病后的症状表现等,都一一作了阐述,特别提出“凡刺之法,先必本于神”的论点,故以《本神》名篇。神是中国文化和哲学的重要范畴之一。《周易》认为“阴阳不测之谓神”,神既是天地阴阳之道变化的内在动力,又是其外在的极致表现。中国哲学注重对宇宙变化之神的探求。中医学重视人身之神,在养生上强调“养神”;在治疗上强调“治神”;医学上的最高成就者称为“神医”。中国的文学、艺术强调“神韵”,艺术上追求“出神入化”。总之,“神”是把握中国文化和中医学的关键范畴之一。
	
\begin{yuanwen}
黄帝问于岐伯曰:凡刺之法,先必本于神\footnote{这是广义的神,概括了人体整个生命活动现象。包括下文所讲“血、脉、营、气、精、神”等生理活动的内容。}。血、脉、营、气、精、神,此五脏之所藏也。至其淫泆\footnote{y\`i,恣纵。}离脏则精失,魂\footnote{精神活动之一。}魄\footnote{是先天的本能,如感觉、运动等。}飞扬,志意恍乱\footnote{思想混乱,茫然无主。},智虑去身者,何因而然乎?天之罪与?人之过乎?何谓德、气、生、精、神、魂、魄、心、意、志、思、智、虑?请问其故。
	
岐伯答曰:天之在我者,德也,地之在我者,气也。德流气薄\footnote{迫近,附着。}而生者也。故生之来谓之精,两精相搏\footnote{张景岳:“两精者,阴阳之精也。搏,交结也。”即男女交媾,两精结合。搏,结合。}谓之神;随神往来者谓之魂;并精而出入者谓之魄;所以任\footnote{负担,主持。}物者谓之心;心有所忆谓之意;意之所存谓之志;因志而存变谓之思;因思而远慕谓之虑;因虑而处物谓之智。
\end{yuanwen}
	
\begin{yuanwen}
故智者之养生也,必顺四时而适寒暑,和喜怒而安居处,节阴阳而调刚柔,如是则僻邪不至,长生久视\footnote{是寿命延长,不易衰老之意。《吕氏春秋》有“莫不欲长生久视”,注云:“视,活也。”《老子》有“是谓深根固柢,长生久视之道”。}。
\end{yuanwen}
	
\begin{yuanwen}
是故怵惕\footnote{恐惧的样子。怵,ch\`u,恐惧。惕,t\`i,敬畏。}思虑者则伤神,神伤则恐惧,流淫而不止\footnote{张景岳:“流淫谓流泄淫溢。如下文所云恐惧而不解则伤精,精时自下者是也。”}。因悲哀动中者,竭绝而失生\footnote{张景岳:“悲则气消,悲哀太甚则胞络绝,故至失生。竭者绝之渐,绝则尽绝无余矣。”}。喜乐者,神惮散而不藏\footnote{张景岳:“喜发于心,乐散在外,暴喜伤阳,故神气惮散而不藏。惮(d\`an),惊惕也。”意谓神气耗散而不能归藏于心。}。愁忧者,气闭塞而不行。盛怒者,迷惑而不治\footnote{张景岳:“怒则气逆,甚者心乱,故至昏迷惶惑而不治。不治,乱也。”}。恐惧者,神荡惮而不收\footnote{张景岳:“恐惧则神志惊散,故荡惮而不收。上文言喜乐者,神惮散而不藏,与此稍同。但彼云不藏者,神不能持而流荡也;此云不收者,神为恐惧而散失也。所当详辨。”}。
\end{yuanwen}
	
\begin{yuanwen}
心,怵惕思虑则伤神,神伤则恐惧自失。破䐃脱肉,毛悴色夭,死于冬。
\end{yuanwen}
	
\begin{yuanwen}
脾,愁忧不解则伤意,意伤则悗\footnote{m\'an,闷也。胸膈苦闷。}乱\footnote{烦乱。},四肢不举,毛悴色夭,死于春。
\end{yuanwen}
	
\begin{yuanwen}
肝,悲哀动中则伤魂,魂伤则狂忘不精,不精则不正,当人阴缩而挛筋,两胁骨不举,毛悴色夭,死于秋。
\end{yuanwen}
	
\begin{yuanwen}
肺,喜乐无极则伤魄,魄伤则狂,狂者意不存人,皮革焦,毛悴色夭,死于夏。
\end{yuanwen}
	
\begin{yuanwen}
肾,盛怒而不止则伤志,志伤则喜忘其前言,腰脊不可以俯仰屈伸,毛悴色夭,死于季夏。
\end{yuanwen}
	
\begin{yuanwen}
恐惧而不解则伤精,精伤则骨痠\footnote{su\=an}痿厥,精时自下。是故五脏主藏精者也,不可伤,伤则失守而阴虚;阴虚则无气,无气则死矣。是故用针者,察观病人之态,以知精神魂魄之存亡,得失之意,五者以伤,针不可以治之也。
\end{yuanwen}
	
\begin{yuanwen}
肝藏血,血舍魂\footnote{意即魂的功能凭依于血。舍,有住宿、寄居的含义。}。肝气虚则恐,实则怒。脾藏营,营舍意。脾气虚则四肢不用,五脏不安,实则腹胀,经溲不利\footnote{大小便不利。经,《甲乙经》作“泾”。}。心藏脉,脉舍神,心气虚则悲,实则笑不休。肺藏气,气舍魄,肺气虚,则鼻塞不利,少气,实则喘喝,胸盈仰息。肾藏精,精舍志,肾气虚则厥,实则胀,五脏不安。必审五脏之病形,以知其气之虚实,谨而调之也。
\end{yuanwen}

\chapter{终始}

终始,是中国古代哲学的重要范畴。中国哲学关注的是包括人类在内的天地万物的生生化化,是关乎生命的学问。中国的医学与哲学一样也是关乎生命的科学而不仅仅是治病祛疾之术。生命是在时间中展开的过程,对于时间的关注,成为中国哲学和医学的根本特征。古人认为生命是在阳变阴合的大化流行中永不停息,循环往复的过程。标志这一循环往复过程的范畴就说终始。生命活动以及生命活动过程中正常和异常的变化都有这种终而复始的规律。抓住了终始范畴就掌握了事物发展变化的关键。正如《大学》所说:“物有本末,事有终始,知所先后,则近道矣。”“终始”范畴见于《内经》的诸多篇章,是贯穿于《内经》中的重要思想线索之一。本篇以《终始》名篇,来组织有关材料,对临床医家有重要的提示作用。本篇的中心内容,是从脉口、人迎的脉象对比,来诊察十二经气血阴阳的变化;根据病证情况,以确定针刺治疗的原则和方法。篇首以“明知终始,五脏为纪”开篇,篇末以六经终绝的症状结尾,前后呼应,层次分明,以示读者掌握这些自始而终的规律,所以篇名《终始》。
	
\begin{yuanwen}
凡刺之道,毕于《终始》。明知终始,五脏为纪\footnote{总要。},阴阳定矣。阴者主脏,阳者主腑。阳受气于四末,阴受气于五脏\footnote{马元台:“阳在外,受气于四肢;阴在内,受气于五脏。”四末,即四肢。}。故泻者迎之,补者随之。知迎知随,气可令和。和气之方,必通阴阳。五脏为阴,六腑为阳。传之后世,以血为盟\footnote{是古人盟誓时一种极其郑重的仪式。即宰杀牲畜取血,由参加订盟的人共同吸饮或涂于口旁,以此表示决不背信弃约。}。敬之者昌,慢之者亡。无道行私,必得夭殃\footnote{张景岳:“不明至道,而强不知以为知,即无道行私也。”夭殃,夭折死亡的祸害。}。
\end{yuanwen}
	
\begin{yuanwen}
谨奉天道,请言终始!终始者,经脉为纪。持其脉口人迎,以知阴阳,有余不足,平与不平。天道毕矣。所谓平人者不病。不病者,脉口人迎应四时也,上下相应而俱往来也,六经之脉不结动也,本末之寒温之相守司也,形肉血气必相称也。是谓平人。少气者,脉口人迎俱少而不称尺寸也。如是者,则阴阳俱不足。补阳则阴竭,泻阴则阳脱。如是者,可将以甘药,不可饮以至剂。如此者,弗灸。不已者,因而泻之,则五脏气坏矣。
\end{yuanwen}
	
	\begin{yuanwen}
	人迎一盛,病在足少阳,一盛而躁,病在手少阳。人迎二盛,病在足太阳,二盛而躁,病在手太阳,人迎三盛,病在足阳明,三盛而躁,病在手阳明。人迎四盛,且大且数,名曰溢阳,溢阳为外格。
	
	脉口一盛,病在足厥阴;厥阴一盛而躁,在手心主。脉口二盛,病在足少阴;二盛而躁,在手少阴。脉口三盛,病在足太阴;三盛而躁,在手太阴。脉口四盛,且大且数者,名曰溢阴。溢阴为内关,内关不通,死不治。人迎与太阴脉口俱盛四倍以上,名曰关格。关格者,与之短期。
	
	人迎一盛,泻足少阳而补足厥阴,二泻一补,日一取之,必切而验之,疏取之,上气和乃止。人迎二盛,泻足太阳补足少阴,二泻一补,二日一取之,必切而验之,疏取之,上气和乃止。人迎三盛,泻足阳明而补足太阴,二泻一补,日二取之,必切而验之,疏取之,上气和乃止。
	
	脉口一盛,泻足厥阴而补足少阳,二补一泻,日一取之,必切而验之,疏而取,上气和乃止。脉口二盛,泻足少阴而补足太阳,二补一泻,二日一取之,必切而验之,疏取之,上气和乃止。脉口三盛,泻足太阴而补足阳明,二补一泻,日二取之,必切而验之,疏而取之,上气和乃止。所以日二取之者,太、阳主胃,大富于谷气,故可日二取之也。
	
	人迎与脉口俱盛三倍以上,命曰阴阳俱溢,如是者不开,则血脉闭塞,气无所行,流淫于中,五脏内伤。如此者,因而灸之,则变易而为他病矣。
	\end{yuanwen}
	
\begin{yuanwen}
凡刺之道,气调而止。补阴泻阳,音气益彰,耳目聪明。反此者,血气不行。
\end{yuanwen}
	
\begin{yuanwen}
所谓气至而有效者\footnote{中医以针刺治病取效的关键在于得气,即“气至”。人体生命活动的关键在于气血的畅通周流,疾病之所以发生就是因为气血出了问题,治疗时也是以调动和恢复气血的功能为目标。所以只有“气至”,即有了酸麻胀痛及循经感传的现象,才会有疗效。},泻则益虚。虚者,脉大如其故而不坚也。坚如其故者,适虽言快,病未去也。补则益实。实者,脉大如其故而益坚也。夫如其故而不坚者,适虽言快,病未去也。故补则实,泻则虚。痛虽不随针,病必衰去。必先通十二经脉之所生病,而后可得传于终始矣。故阴阳不相移,虚实不相倾,取之其经。
\end{yuanwen}
	
\begin{yuanwen}
凡刺之属,三刺\footnote{指针刺皮肤、肌肉、分肉三种深浅不同的刺法。}至谷气。邪僻妄合\footnote{指不正之气即邪气与血气混合。},阴阳易居。逆顺相反,沉浮异处\footnote{脉气当沉而反浮之在表,当浮而反沉之在里。杨上善:“春脉或沉,冬脉或浮,故曰异处。”}。四时不得\footnote{脉气不能与四时顺应。张志聪:“四时不得者,不得其升降浮沉也。”},稽留淫泆。须针而去。故一刺则阳邪出,再刺则阴邪出,三刺则谷气至,谷气至而止。所谓谷气至者,已补而实,已泻而虚,故以知谷气至也。邪气独去者,阴与阳未能调,而病知愈也。故曰:补则实,泻则虚。痛虽不随针,病必衰去矣。
\end{yuanwen}
	
\begin{yuanwen}
阴盛而阳虚,先补其阳,后泻其阴而和之。阴虚而阳盛,先补其阴,后泻其阳而和之。
\end{yuanwen}
	
\begin{yuanwen}
三脉\footnote{指足阳明、足厥阴、足少阳三脉。马元台:“阳明动于大指次指之间,凡厉兑、陷谷、冲阳、解溪,皆在足跗上也。厥阴动于大指次指之间,正以大敦、行间、太冲、中封,在足跗内也。少阴则动于足心,其穴涌泉,乃足跗之下也。”}动于足大指之间,必审其实虚。虚而泻之,是谓重虚。重虚,病益甚。凡刺此者,以指按之。脉动而实且疾者则泻之,虚而徐者则补之。反此者,病益甚。其动也,阳明在上,厥阴在中,少阴在下。膺腧中膺,背腧中背。肩膊虚者,取之上\footnote{张景岳:“凡肩膊之虚软而痛者,病有阴经阳经之异。阴经在膺,故治阴病者,当取膺腧而必中其膺;阳经在背,故治阳病者,当取背腧而必中其背。病在手经,故取之上。上者,手也。如手太阴之中府、云门,手厥阴之天池,皆膺腧也。手少阳之肩髎(li\'ao),天髎,手太阳之天宗、曲垣、肩外俞,皆背腧也。咸主肩膊虚痛等病。”}。重舌\footnote{舌下的血脉胀起,形如小舌,似为两舌相重,故称重舌。},刺舌柱以铍针也\footnote{即舌下的筋,像柱一样,故称舌柱。}。手屈而不伸者,其病在筋;伸而不屈者,其病在骨。在骨守骨,在筋守筋。
\end{yuanwen}
	
\begin{yuanwen}
泻一方实,深取之,稀按其痏\footnote{杨上善:“希,迟也。迟按针伤之处,使气泄也。”痏,w\v{e}i,针孔。},以极出其邪气;补一方虚,浅刺之,以养其脉,疾按其痏\footnote{杨上善:“按针伤之处,急关其门,使邪气不入,正气不出也。”},无使邪气得入。邪气来也紧而疾,谷气来也徐而和。脉实者,深刺之,以泄其气;脉虚者,浅刺之,使精气无得出,以养其脉,独出其邪气。刺诸痛者,其脉皆实。
\end{yuanwen}
	
\begin{yuanwen}
故曰:从腰以上者,手太阴阳明皆主之;从腰以下者,足太阴阳明皆主之。病在上者下取之;病在下者高取之;病在头者取之足;病在腰者取之腘。病生于头者,头重;生于手者,臂重;生于足者,足重。治病者,先刺其病所以生者也。
\end{yuanwen}
	
\begin{yuanwen}
春,气在毛;夏,气在皮肤;秋,气在分肉;冬,气在筋骨。刺此病者,各以其时为齐\footnote{同“剂”。在此可理解为“标准”。}。故刺肥人者,以秋冬之齐;刺瘦人者,以春夏之齐。病痛者,阴也。痛而以手按之不得者,阴也,深刺之\footnote{张景岳:“凡病痛者,多由寒邪滞逆于经,及深居筋骨之间,凝聚不散,故病痛者为阴也。按之不得者,隐藏深处也,是为阴邪,故刺亦宜深。然则痛在浮浅者,由属阳邪可知也。但诸痛属阴者多耳。”}。病在上者,阳也。病在下者,阴也。痒者\footnote{张景岳:“痒者,散动于肤腠,故为阳。”},阳也,浅刺之。
\end{yuanwen}
	
\begin{yuanwen}
病先起阴者,先治其阴而后治其阳;病先起阳者,先治其阳而后治其阴。刺热厥者,留针,反为寒;刺寒厥者,留针,反为热。刺热厥者,二阴一阳;刺寒厥者,二阳一阴。所谓二阴者,二刺阴也;一阳者,一刺阳也。久病者,邪气入深。刺此病者,深内而久留之,间日而复刺之。必先调其左右,去其血脉。刺道毕矣。
\end{yuanwen}
	
\begin{yuanwen}
凡刺之法,必察其形气。形肉未脱,少气而脉又躁,躁疾者,必为缪刺之。散气可收,聚气可布\footnote{杨上善:“缪刺之益,正气散而收聚,邪气聚而可散也。”}。深居静处,占神往来,闭户塞牖,魂魄不散。专意一神,精气之分,毋闻人声,以收其精,必一其神,令志在针。浅而留之,微而浮之,以移其神,气至乃休。男内女外,坚拒勿出。谨守勿内,是谓得气。
\end{yuanwen}
	
	\begin{yuanwen}
	凡刺之禁:新内勿刺,新刺勿内;已醉勿刺,已刺勿醉;新怒勿刺,已刺勿怒;新劳勿刺,已刺勿劳;已饱勿刺,已刺勿饱;已饥勿刺,已刺勿饥;已渴勿刺,已刺勿渴;大惊大恐,必定其气乃刺之。乘车来者,卧而休之,如食顷乃刺之。出行来者,坐而休之,如行千里顷乃刺之。凡此十二禁者,其脉乱气散,逆其营卫,经气不次,因而刺之,则阳病入于阴,阴病出为阳,则邪气复生。粗工勿察,是谓伐身,形体淫乱,乃消脑髓,津液不化,脱其五味,是谓失气也。
	
	太阳之脉,其终也。戴眼,反折,瘈瘲,其色白,绝皮乃绝汗,绝汗则终矣。
	
	少阳终者,耳聋,百节尽纵,目系绝,目系绝,一日半则死矣。其死也,色青白,乃死。
	
	阳明终者,口目动作,喜惊、妄言、色黄;其上下之经盛而不行,则终矣。
	
	少阴终者,面黑,齿长而垢,腹胀闭塞,上下不通而终矣。
	
	厥阴终者,中热溢干,喜溺,心烦,甚则舌卷,卵上缩而终矣。
	
	太阴终者,腹胀闭,不得息,气噫,善呕,呕则逆,逆则面赤,不逆则上下不通,上下不通则面黑,皮毛憔而终矣。
	\end{yuanwen}
	
	
\part{}
	
\chapter{经脉}

本篇详细叙述了十二经脉的起止点、循行部位、发病症候及治疗原则,并分别说明十二络脉的循行和病候,五阴经气绝所出现的特征和预后。因篇中重点是论述十二经脉,篇首即着重指出经脉在决死生、处百病、调虚实等方面的重要作用,故以《经脉》名篇,是中国经络学说的重要文献。
	
\begin{yuanwen}
雷公问于黄帝曰:《禁脉》之言,凡刺之理,经脉为始。营其所行,制其度量。内次五脏,外别六腑,愿尽闻其道。
	
黄帝曰:人始生,先成精,精成而脑髓生。骨为干,脉为营,筋为刚,肉为墙。皮肤坚而毛发长。谷入于胃,脉道以通,血气乃行。
	
雷公曰:愿卒闻经脉之始生。
	
黄帝曰:经脉者,所以能决死生,处百病,调虚实,不可不通。
\end{yuanwen}
	
\begin{yuanwen}
肺手太阴之脉,起于中焦\footnote{指中脘部位。},下络\footnote{联络。凡萦绕于与本经相表里的脏腑均称络。}大肠,还\footnote{指经脉循行去而复回。}循\footnote{沿着。}胃口\footnote{指胃上口贲门与下口幽门。},上膈属\footnote{隶属。凡经脉连于其本经的脏腑均称属。}肺。从肺系\footnote{指与肺连接的气管、喉咙等组织。}横出腋下,下循臑\footnote{n\`ao,上臂。}内,行少阴心主之前,下肘中,循臂内,上骨下廉\footnote{边缘。},入寸口,上鱼\footnote{手大指本节后掌侧肌肉隆起处,形状如鱼,故名。},循鱼际\footnote{“鱼”的边缘。},出大指之端;其支者,从腕后直出次指内廉,出其端。
\end{yuanwen}
	
	\begin{yuanwen}
	
	是动则病肺胀满,膨胀而喘咳,缺盆中痛,甚则交两手而瞀,此为臂厥。是主肺所生病者,咳上气,喘渴,烦心,胸满,臑臂内前廉痛厥,掌中热。气盛有余,则肩背痛,风寒汗出中风,小便数而欠。气虚则肩背痛,寒,少气不足以息,溺色变。为此诸病,盛则泻之,虚则补之,热则疾之,寒则留之,陷下则灸之,不盛不虚,以经取之。盛者,寸口大三倍于人迎,虚者,则寸口反小于人迎也。
	\end{yuanwen}
	
\begin{yuanwen}
大肠手阳明之脉,起于大指次指之端,循指上廉,出合谷两骨之间\footnote{即第一、二掌骨之间,俗名虎口,又名合谷。},上入两筋之中\footnote{指手腕背侧,拇长伸肌腱与拇短伸肌腱两筋间陷中,有穴名叫阳溪。},循臂上廉,入肘外廉,上臑外前廉,上肩,出髃骨\footnote{y\'u,为肩胛骨与锁骨相连接的地方,即肩髃穴处。}之前廉,上出于柱骨之会上\footnote{肩胛骨上,颈骨隆起处,即大椎穴处。因诸阳脉会于大椎,故称会上。},下入缺盆\footnote{锁骨窝。}络肺,下膈属大肠。其支者,从缺盆上颈贯颊,入下齿中,还出挟口,交人中,左之右,右之左,上挟鼻孔。
\end{yuanwen}
	
	\begin{yuanwen}
	是动则病齿痛,颈肿。是主津液所生病者,目黄,口干,鼽衄,喉痹,肩前臑痛,大指次指痛不用,气有余则当脉所过者热肿;虚则寒栗不复。为此诸病,盛则泻之,虚则补之,热则疾之,寒则留之,陷下则灸之,不盛不虚,以经取之。盛者,人迎大三倍于寸口;虚者,人迎反小于寸口也。
	\end{yuanwen}
	
\begin{yuanwen}
胃足阳明之脉,起于鼻之交頞\footnote{\`e,鼻梁。}中,旁纳太阳之脉,下循鼻外,入上齿中,还出挟口,环唇,下交承浆,却循颐\footnote{在口角的外下方,腮的前下方。}后下廉,出大迎,循颊车,上耳前,过客主人,循发际,至额颅\footnote{前额骨部,在发下眉上处。};其支者,从大迎前下人迎,循喉咙,入缺盆,下膈,属胃,络脾;其直者,从缺盆下乳内廉,下挟脐,入气街\footnote{又叫“气冲”。在少腹下方,毛际两旁。}中;其支者,起于胃口,下循腹里,下至气街中而合,以下髀关,抵伏兔,下膝膑中,下循胫外廉,下足跗,入中指内间;其支者,下廉三寸而别,下入中指外间;其支者,别跗上,入大指间,出其端。
\end{yuanwen}
	
	\begin{yuanwen}
	是动则病洒洒振寒,善呻,数欠,颜黑,病至则恶人与火,闻木声则惕然而惊,心欲动,独闭户塞牖而处。甚则欲上高而歌,弃衣而走,贲向腹胀,是为骭厥。是主血所生病者,狂疟温淫,汗出,鼽衄,口喎,唇胗,颈肿,喉痹,大腹水肿,膝膑肿痛,循膺乳、气冲、股、伏兔、骭外廉、足跗上皆痛,中趾不用,气盛则身以前皆热,其有余于胃,则消谷善饥,溺色黄;气不足则身以前皆寒栗,胃中寒则胀满。为此诸病,盛则泻之,虚则补之,热则疾之,寒则留之,陷下则灸之,不盛不虚,以经取之。盛者,人迎大三倍于寸口,虚者,人迎反小于寸口也。
	\end{yuanwen}
	
\begin{yuanwen}
脾足太阴之脉,起于大指之端,循指内侧白肉际\footnote{又称赤白肉际,是手足两侧阴阳界面的分界处。阳面赤色,阴面白色。},过核骨后\footnote{是足大趾本节后内侧凸出的圆骨。形如果核,故名。},上内踝前廉,上踹\footnote{chu\v{a}n,小腿肚。}内,循胫骨后,交出厥阴之前,上膝股内前廉,入腹属脾络胃,上膈,挟咽,连舌本\footnote{舌根。},散舌下;其支者,复从胃,别上膈,注心中。
\end{yuanwen}

    \begin{yuanwen}	
	是动则病舌本强,食则呕,胃脘痛,腹胀,善噫,得后与气,则快然如衰,身体皆重。是主脾所生病者,舌本痛,体不能动摇,食不下,烦心,心下急痛,溏瘕泄,水闭,黄疸,不能卧,强立,股膝内肿厥,足大趾不用。为此诸病,盛则泻之,虚则补之,热则疾之,寒则留之,陷下则灸之,不盛不虚,以经取之。盛者,寸口大三倍于人迎,虚者,寸口反小于人迎。
	\end{yuanwen}
	
\begin{yuanwen}
心手少阴之脉,起于心中,出属心系\footnote{指心脏与其他脏器相联系的脉络。张景岳:“心当五椎之下,其系有五,上系连肺,肺下系心,心下三条,连脾肝肾,故心通五脏之气而为之主也。”},下膈络小肠;其支者,从心系上挟咽,系目系\footnote{眼球内连于脑的脉络。};其直者,复从心系却上肺,下出腋下,下循臑内后廉,行太阴心主之后,下肘内,循臂内后廉,抵掌后锐骨\footnote{指掌后小指侧的高骨。}之端,入掌内后廉,循小指之内出其端。
\end{yuanwen}

	\begin{yuanwen}
	是动则病嗌干,心痛,渴而欲饮,是为臂厥。是主心所生病者,目黄,胁痛,臑臂内后廉痛厥,掌中热痛。为此诸病,盛则泻之,虚则补之,热则疾之,寒则留之,陷下则灸之,不盛不虚,以经取之。盛者,寸口大再倍于人迎,虚者,寸口反小于人迎也。
	\end{yuanwen}
	
\begin{yuanwen}
小肠手太阳之脉,起于小指之端,循手外侧上腕,出踝\footnote{此处指手腕后方小指侧的高骨。}中,直上循臂骨下廉,出肘内侧两筋之间,上循臑外后廉,出肩解\footnote{肩后骨缝。},绕肩胛,交肩上,入缺盆络心,循咽下膈,抵胃属小肠;其支者,从缺盆循颈上颊,至目锐眦\footnote{眼外角。z\`i。},却入耳中;其支者,别颊上䪼\footnote{zhu\=o,眼眶的下方,包括颧骨内连及上牙床的部位。}抵鼻,至目内眦\footnote{眼内角。},斜络于颧。
\end{yuanwen}

	\begin{yuanwen}
	是动则病嗌痛,颔肿,不可以顾,肩似拔,臑似折。是主液所生病者,耳聋、目黄,颊肿,颈、颔、肩、臑、肘、臂外后廉痛。为此诸病,盛则泻之,虚则补之,热则疾之,寒则留之,陷下则灸之,不盛不虚,以经取之。盛者,人迎大再倍于寸口,虚者,人迎反小于寸口也。
	\end{yuanwen}
	
\begin{yuanwen}
膀胱足太阳之脉,起于目内眦,上额交巅\footnote{头顶正中最高点,当百会穴处。};其支者,从巅至耳上角\footnote{耳壳的上部。};其直者,从巅入络脑,还出别下项,循肩髃\footnote{y\'u,肩胛骨。}内,挟脊抵腰中,入循膂\footnote{l\v{\"u},挟脊两旁的肌肉。},络肾属膀胱;其支者,从腰中下挟脊贯臀,入腘中;其支者,从髆内左右,别下,贯胛,挟脊内,过髀枢\footnote{股骨上端的关节,即环跳穴处。为髀骨所嵌入的地方,有转枢作用,故称髀枢。},循髀外,从后廉下合腘中,以下贯踹内,出外踝之后,循京骨\footnote{足外侧小趾本节后突出的半圆骨,又穴名。京,本意为高地、高处。},至小指外侧。
\end{yuanwen}

\begin{yuanwen}
	是动则病冲头痛,目似脱,项如拔,脊痛,腰似折,髀不可以曲,腘如结,踹(腨)如裂,是为踝厥。是主筋所生病者,痔、瘧、狂、癲疾、頭䪼項痛,目黃、淚出,鼽衄,項、背、腰、尻、膕踹(腨)、腳皆痛,小趾不用。为此诸病,盛则泻之,虚则补之,热则疾之,寒则留之,陷下则灸之,不盛不虚,以经取之。盛者,人迎大再倍于寸口,虚者,人迎反小于寸口也。
	\end{yuanwen}
	
\begin{yuanwen}
肾足少阴之脉,起于小指之下,邪\footnote{偏斜。}走足心,出于然谷之下,循内踝之后,别入跟中,以上踹内,出腘内廉,上股内后廉,贯脊,属肾,络膀胱;其直者,从肾上贯肝膈,入肺中,循喉咙,挟舌本;其支者,从肺出络心,注胸中。
\end{yuanwen}

\begin{yuanwen}
	是动则病饥不欲食,面如漆柴,咳唾则有血,喝喝而喘,坐而欲起,目(盳盳)如无所见,心如悬若饥状。气不足则善恐,心惕惕如人将捕之,是为骨厥。是主肾所生病者,口热,舌干,咽肿,上气,嗌干及痛,烦心,心痛,黄疸,肠澼,脊股内后廉痛,痿厥,嗜卧,足下热而痛。为此诸病,盛则泻之,虚则补之,热则疾之,寒则留之,陷下则灸之,不盛不虚,以经取之。灸则强食生肉,缓带披发,大杖重履而步。盛者,寸口大再倍于人迎,虚者,寸口反小于人迎也。
	\end{yuanwen}
	
\begin{yuanwen}
心主手厥阴心包络之脉,起于胸中,出属心包络,下膈,历络三焦\footnote{自胸至腹依次联络上中下三焦。};其支者,循胸出胁,下腋三寸,上抵腋,下循臑\footnote{n\`ao,上臂。}内,行太阴少阴之间,入肘中,下臂行两筋之间,入掌中,循中指出其端;其支者,别掌中,循小指次指出其端。
\end{yuanwen}

\begin{yuanwen}
	是动则病手心热,臂肘挛急,腋肿,甚则胸胁支满,心中憺憺大动,面赤,目黄,喜笑不休。是主脉所生病者,烦心,心痛,掌中热。为此诸病,盛则泻之,虚则补之,热则疾之,寒则留之,陷下则灸之,不盛不虚,以经取之。盛者,寸口大一倍于人迎,虚者,寸口反小于人迎也。
	\end{yuanwen}
	
\begin{yuanwen}
三焦手少阳之脉,走于小指次指之端,上出两指之间,循手表腕\footnote{手与腕的背面。},出臂外两骨之间,上贯肘,循臑外,上肩,而交出足少阳之后,入缺盆,布膻中,散落心包,下膈,循属三焦;其支者,从膻中上出缺盆,上项,系耳后直上,出耳上角,以屈下颊至䪼;其支者,从耳后入耳中,出走耳前,过客主人前,交颊,至目锐眦。
\end{yuanwen}

\begin{yuanwen}
	是动则病耳聋浑浑焞焞,嗌肿,喉痹。是主气所生病者,汗出,目锐眦痛,颊痛,耳后、肩、臑、肘、臂外皆痛,小指次指不用。为此诸病,盛则泻之,虚则补之,热则疾之,寒则留之,陷下则灸之,不盛不虚,以经取之。盛者,人迎大一倍于寸口,虚者,人迎反小于寸口也。
	\end{yuanwen}
	
\begin{yuanwen}
胆足少阳之脉,起于目锐眦,上抵头角,下耳后,循颈行手少阳之前,至肩上,却交出手少阳之后,入缺盆;其支者,从耳后入耳中,出走耳前,至目锐眦后;其支者,别锐眦,下大迎,合于手少阳,抵于䪼,下加颊车,下颈合缺盆,以下胸中,贯膈络肝属胆,循胁里,出气街,绕毛际\footnote{耻骨部生阴毛之处。},横入髀厌\footnote{髀枢,即环跳部。}中;其直者,从缺盆下腋,循胸过季胁,下合髀厌中,以下循髀阳\footnote{大腿的外侧。},出膝外廉,下外辅骨\footnote{腓骨。小腿骨有胫骨、腓骨两支,胫骨为主,腓骨为辅,且在外侧,故称外辅骨。}之前,直下抵绝骨\footnote{在外踝直上三寸许腓骨的凹陷处。腓骨至此似乎绝断,故称绝骨。}之端,下出外踝之前,循足跗上,入小指次指之间;其支者,别跗上,入大指之间,循大指歧骨内出其端,还贯爪甲,出三毛\footnote{足大趾爪甲后生毛处。}。
\end{yuanwen}

\begin{yuanwen}
	是动则病口苦,善太息,心胁痛,不能转侧,甚则面微有尘,体无膏泽,足外反热,是为阳厥。是主骨所生病者,头痛,颔痛,目锐眦痛,缺盆中肿痛,腋下肿,马刀侠瘿,汗出振寒,疟,胸、胁、肋、髀、膝外至胫、绝骨、外踝前及诸节皆痛,小趾次趾不用。为此诸病,盛则泻之,虚则补之,热则疾之,寒则留之,陷下则灸之,不盛不虚,以经取之。盛者,人迎大一倍于寸口,虚者,人迎反小于寸口也。
	\end{yuanwen}
	
\begin{yuanwen}
肝足厥阴之脉,起于大指丛毛\footnote{即上文“三毛”。}之际,上循足跗上廉,去内踝一寸,上踝八寸,交出太阴之后,上腘内廉,循股阴入毛中,过阴器,抵小腹,挟胃属肝络胆,上贯膈,布胁肋,循喉咙之后,上入颃颡,连目系,上出额,与督脉会于巅;其支者,从目系下颊里,环唇内;其支者,复从肝别贯膈,上注肺。
\end{yuanwen}

\begin{yuanwen}
	是动则病腰痛不可以俛仰,丈夫(疒贵)疝,妇人少腹肿,甚则嗌干,面尘,脱色。是主肝所生病者,胸满,呕逆,飧泄,狐疝,遗溺,闭癃。为此诸病,盛则泻之,虚则补之,热则疾之,寒则留之,陷下则灸之,不盛不虚,以经取之。盛者,寸口大一倍于人迎,虚者,寸口反小于人迎也。
	\end{yuanwen}
	
	\begin{yuanwen}
	手太阴气绝,则皮毛焦。太阴者,行气温于皮毛者也。故气不荣,则皮毛焦;皮毛焦,则津液去皮节;津液去皮节者,则爪枯毛折;毛折者,则毛先死。两笃丁死,火胜金也。
	\end{yuanwen}

	\begin{yuanwen}
	手少阴气绝,则脉不通;脉不通,则血不流;血不流,则发色不泽,故其面黑如漆柴者,血先死。壬笃癸死,水胜火也。
	\end{yuanwen}
	
	\begin{yuanwen}
	足太阴气绝者,则脉不荣肌肉。唇舌者,肌肉之本也。脉不荣,则肌肉软;肌肉软,则舌萎人中满;人中满,则唇反;唇反者,肉先死。甲笃乙死,木胜土也。
	\end{yuanwen}
	
	\begin{yuanwen}
	足少阴气绝,则骨枯。少阴者,冬脉也,伏行而濡骨髓者也,故骨不濡,则肉不能着也;骨肉不相亲,则肉软却;肉软却,故齿长而垢,发无泽;发无泽者,骨先死。戊笃己死,土胜水也。
	\end{yuanwen}
	
	\begin{yuanwen}
	足厥阴气绝,则筋绝。厥阴者,肝脉也,肝者,筋之合也,筋者,聚于阴气,而脉络于舌本也。故脉弗荣,则筋急;筋急则引舌与卵,故唇青舌卷卵缩,则筋先死。庚笃辛死,金胜木也。
	\end{yuanwen}
	
	\begin{yuanwen}
	五阴气俱绝,则目系转,转则目运;目运者,为志先死;志先死,则远一日半死矣。六阳气绝,则阴与阳相离,离则腠理发泄,绝汗乃出,故旦占夕死,夕占旦死。
	\end{yuanwen}
	
\begin{yuanwen}
经脉十二者,伏行分肉之间,深而不见;其常见者,足太阴过于外踝之上,无所隐故也。诸脉之浮而常见者,皆络脉也。六经络手阳明少阳之大络,起于五指间,上合肘中。饮酒者,卫气先行皮肤,先充络脉,络脉先盛。故卫气已平,营气乃满,而经脉大盛。脉之卒然动者,皆邪气居之,留于本末,不动则热,不坚则陷且空,不与众同,是以知其何脉之动也。
\end{yuanwen}
	
\begin{yuanwen}
雷公曰:何以知经脉之与络脉异也?
	
黄帝曰:经脉者常不可见也,其虚实也,以气口知之。脉之见者,皆络脉也。
	
雷公曰:细子无以明其然也。
	
黄帝曰:诸络脉皆不能经大节之间,必行绝道\footnote{指经脉不到的间道(偏僻的小路)。}而出,入复合于皮中,其会皆见于外。故诸刺络脉者,必刺其结上\footnote{络脉有血液淤结之处。}。甚血者虽无结,急取之以泻其邪而出其血。留之发为痹也。凡诊络脉,脉色青,则寒且痛;赤则有热。胃中寒,手鱼之络多青矣;胃中有热,鱼际络赤。其暴黑者,留久痹也。其有赤、有黑、有青者,寒热气也。其青短者,少气也。凡刺寒热者皆多血络,必间日而一取之,血尽而止,乃调其虚实。其小而短者少气,甚泻之则闷,闷甚则仆,不得言。闷则急坐之也。
\end{yuanwen}
	
	\begin{yuanwen}
	手太阴之别,名曰列缺。起于腕上分间,并太阴之经,直入掌中,散入于鱼际。其病实则手锐掌热;虚则欠(去欠),小便遗数。取之去腕寸半。别走阳明也。
	
	手少阴之别,名曰通里。去腕一寸半,别而上行,循经入于心中,系舌本,属目系。其实则支膈,虚则不能言。取之掌后一寸,别走太阳也。
	
	手心主之别,名曰内关。去腕二寸,出于两筋之间,循经以上,系于心包络。心系实则心痛,虚则为头强。取之两筋间也。
	
	手太阳之别,名曰支正。上腕五寸,内注少阴;其别者,上走肘,络肩髃。实则节弛肘废;虚则生(月尤),小者如指痂疥。取之所别也。
	
	手阳明之别,名曰偏历。去腕三寸,别入太阴;其别者,上循臂,乘肩髃,上曲颊伤齿;其别者,入耳,合于宗脉。实则龋聋;虚则齿寒痹隔。取之所别也。
	
	手少阳之别,名曰外关。去腕二寸,外绕臂,注胸中,合心主。病实则肘挛,虚则不收。取之所别也。
	
	足太阳之别,名曰飞扬。去踝七寸,别走少阴。实则鼽窒,头背痛;虚则鼽衄。取之所别也。
	
	足少阳之别,名曰光明,去踝五寸,别走厥阴,下络足跗。实则厥,虚则痿躄,坐不能起。取之所别也。
	
	足阳明之别,名曰丰隆。去踝八寸。别走太阴;其别者,循胫骨外廉,上络头项,合诸经之气,下络喉嗌。其病气逆则喉痹瘁瘖。实则狂巅,虚则足不收,胫枯。取之所别也。
	
	足太阴之别,名曰公孙。去本节之后一寸,别走阳明;其别者,入络肠胃,厥气上逆则霍乱,实则肠中切痛;虚则鼓胀。取之所别也。
	
	足少阴之别,名曰大钟。当踝后绕跟,别走太阳;其别者,并经上走于心包下,外贯腰脊。其病气逆则烦闷,实则闭癃,虚则腰痛。取之所别者也。
	
	足厥阴之别,名曰蠡沟。去内踝五寸,别走少阳;其别者,经胫上睪,结于茎。其病气逆则睪肿卒疝。实则挺长,虚则暴痒。取之所别也。
	
	任脉之别,名曰尾翳。下鸠尾,散于腹。实则腹皮痛,虚则痒搔。取之所别也。
	
	督脉之别,名曰长强。挟膂上项,散头上,下当肩胛左右,别走太阳,入贯膂。实则脊强,虚则头重,高摇之,挟脊之有过者。取之所别也。
	
	脾之大络,名曰大包。出渊腋下三寸,布胸胁。实则身尽痛,虚则百节尽皆纵。此脉若罢络之血者,皆取之脾之大络脉也。
	
	凡此十五络者,实则必见,虚则必下。视之不见,求之上下。人经不同,络脉亦所别也。
	\end{yuanwen}
	
	\chapter{经别}
	
	\begin{yuanwen}
	黄帝问于岐伯曰:余闻人之合于天地道也,内有五脏,以应五音、五色、五时、五味、五位也;外有六腑,以应六律。六律建阴阳诸经而合之十二月、十二辰、十二节、十二经水、十二时、十二经脉者,此五脏六腑之所以应天道。夫十二经脉者,人之所以生,病之所以成,人之所以治,病之所以起,学之所始,工之所止也。粗之所易,上之所难也。请问其离合,出入奈何?岐伯稽首再拜曰:明乎哉问也!此粗之所过,上之所息也,请卒言之。
	
	足太阳之正,别入于腘中,其一道下尻五寸,别入于肛,属于膀胱,散之肾,循膂,当心入散;直者,从膂上出于项,复属于太阳,此为一经也。足少阴之正,至腘中,别走太阳而合,上至肾,当十四椎出属带脉;直者,系舌本,复出于项,合于太阳,此为一合。成以诸阴之别,皆为正也。
	
	足少阳之正,绕髀入毛际,合于厥阴,别者入季胁之间,循胸里属胆,散之上肝,贯心以上挟咽,出颐颌中,散于面,系目系,合少阳于外眦也。足厥阴之正,别跗上,上至毛际,合于少阳,与别俱行,此为二合也。
	
	足阳明之正,上至脾,入于腹里属胃,散之脾,上通于心,上循咽出于口,上頞?,还系目系,合于阳明也。足太阴之正,上至髀,合于阳明,与别俱行,上结于咽,贯舌中,此为三合也。
	
	手太阳之正,指地,别于肩解,入腋走心,系小肠也。手少阴之正,别入于渊腋两筋之间,属于心,上走喉咙,出于面,合目内眦,此为四合也。
	
	手少阳之正,指天,别于巅,入缺盆,下走三焦,散于胸中也。手心主之正,别下渊腋三寸,入胸中,别属三焦,出循喉咙,出耳后,合少阳完骨之下,此为五合也。
	
	手阳明之正,从手循膺乳,别于肩?,入柱骨,下走大肠,属于肺,上循喉咙,出缺盆,合于阳明也。手太阴之正,别入渊腋少阴之前,入走肺,散之大阳(肠),上出缺盆,循喉咙,复合阳明,此六合也。
	\end{yuanwen}
	
	\chapter{经水}
	
	\begin{yuanwen}
	黄帝问于岐伯曰:经脉十二者,外合于十二经水,而内属于五脏六腑。夫十二经水者,其有大小、深浅、广狭、远近各不同;五脏六腑之高下、大小、受谷之多少亦不等,相应奈何?夫经水者,受水而行之;五脏者,合神气魂魄而藏之;六腑者,受谷而行之,受气而扬之;经脉者,受血而营之。合而以治,奈何?刺之深浅,灸之壮数,可得闻乎?
	
	岐伯答曰:善哉问也!天至高不可度,地至广不可量,此之谓也。且夫人生于天地之间,六合之内,此天之高,地之广也,非人力之所能度量而至也。若夫八尺之士,皮肉在此,外可度量切循而得之,其死可解剖而视之。其藏之坚脆,腑之大小,谷之多少,脉之长短,血之清浊,气之多少,十二经之多血少气,与其少血多气,与其皆多血气,与其皆少血气,皆有大数。其治以针艾,各调其经气,固其常有合乎。
	
	黄帝曰:余闻之,快于耳不解于心,愿卒闻之。
	
	岐伯答曰:此人之所以参天地而应阴阳也,不可不察。足太阳外合清水,内属于膀胱,而通水道焉。足少阳外合于渭水,内属于胆。足阳明外合于海水,内属于胃。足太阴外合于湖水,内属于脾。足少阴外合于汝水,内属于肾。足厥阴外合于渑水,内属于肝。手太阳外合于淮水,内属于小肠,而水道出焉。手少阳外合于漯水,内属于三焦。手阳明外合于江水,内属于大肠。手太阴外合于河水,内属于肺。手少阴外合济水,内属于心。手心主外合于漳水,内属于心包。凡此五脏六腑十二经水者,外有源泉,而内有所禀,此皆内外相贯,如环无端,人经亦然。故天为阳,地为阴,腰以上为天,腰以下为地。故海以北者为阴,湖以北者为阴中之阴;漳以南者为阳,河以北至漳者为阳中之阴;漯以南至江者,为阳中之太阳,此一隅之阴阳也,所以人与天地相参也。
	
	黄帝曰:夫经水之应经脉也,其远近浅深,水血之多少,各不同,合而以刺之奈何?
	
	岐伯答曰:足阳明,五脏六腑之海也,其脉大,血多气盛,热壮,刺此者不深勿散,不留不泻也。足阳明刺深六分,留十呼。足太阳深五分,留七呼。足少阳深四分,留五呼。足太阴深三分,留四呼。足少阴深二分,留三呼。足厥阴深一分,留二呼。手之阴阳,其受气之道近,其气之来疾,其刺深者,皆无过二分,其留,皆无过一呼。其少长、大小、肥瘦,以心擦之,命曰法天之常,灸之亦然。灸而过此者,得恶火则骨枯脉涩,刺而过此者,则脱气。
	
	黄帝曰:夫经脉之大小,血之多少,肤之厚薄,肉之坚脆及腘之大小,可为量度乎?
	
	岐伯答曰:其可为度量者,取其中度也。不甚脱肉,而血气不衰也。若夫度之人,消瘦而形肉脱者,恶可以度量刺乎。审、切、循、扪、按,视其寒温盛衰而调之,是谓因适而为之真也。
	\end{yuanwen}
	
	\part{}
	\chapter{经筋}
	
	\begin{yuanwen}
	足太阳之筋,起于足小趾,上结于踝,邪上结于膝,其下循足外侧,结于踵,上循跟,结于腘;其别者,结于腨外,上腘中内廉,与腘中并上结于臀,上挟脊上项;其支者,别入结于舌本;其直者,结于枕骨,上头,下颜,结于鼻;其支者,为目上网,下结于頄;其支者,从腋后外廉结于肩髃;其支者,入腋下,上出缺盆,上结于完骨;其支者,出缺盆,邪上出于頄。其病小趾支跟肿痛,腘挛,脊反折,项筋急,肩不举,腋支缺盆中纽痛,不可左右摇。治在燔针劫刺,以知为数,以痛为输,名曰仲春痹也。
	
	足少阳之筋,起于小指次指,上结外踝,上循胫外廉,结于膝外廉;其支者,别起外辅骨,上走髀,前者结于伏兔之上,后者,结于尻;其直者,上乘沙季胁,上走腋前廉,系于膺乳,结于缺盆;直者,上出腋,贯缺盆,出太阳之前,循耳后,上额角,交巅上,下走颔,上结于頄;支者,结于目眦为外维。其病小指次指支转筋,引膝外转筋,膝不可屈伸,腘筋急,前引髀,后引尻,即上乘(月少)季胁痛,上引缺盆、膺乳、颈维筋急。从左之右,右目不开,上过右角,并蹻脉而行,左络于右,故伤左角,右足不用,命曰维筋相交。治在燔针劫刺,以知为数,以痛为输,名曰孟春痹也。
	
	足阳明之筋,起于中三指,结于跗上,邪外上加于辅骨,上结于膝外廉,直上结于髀枢,上循胁属脊;其直者,上循(骨干),结于膝;其支者,结于外辅骨,合少阳;其直者,上循伏兔,上结于髀,聚于阴器,上腹而布,至缺盆而结,上颈,上挟口,合于頄,下结于鼻,上合于太阳。太阳为目上网,阳明为目下网;其支者,从颊结于耳前。其病足中指支胫转筋,脚跳坚,伏兔转筋,髀前踵,(疒贵)疝,腹筋急,引缺盆及颊,卒口僻;急者,目不合,热则筋纵,目不开,颊筋有寒,则急,引颊移口,有热则筋弛纵,缓不胜收,故僻。治之以马膏,膏其急者;以白酒和桂,以涂其缓者,以桑钩钩之,即以生桑炭置之坎中,高下以坐等。以膏熨急颊,且饮美酒,敢美炙肉,不饮酒者,自强也,为之三拊而已。治在燔针劫刺,以知为数,以痛为输,名曰季春痹也。
	
	足太阴之筋,起于大指之端内侧,上结于内踝;其直者,络于膝内辅骨,上循阴股,结于髀,聚于阴器,上腹结于脐,循腹里,结于肋,散于胸中;其内者,着于脊。其病足大指支内踝痛,转筋痛,膝内辅骨痛,阴股引髀而痛,阴器纽痛,上引脐两胁痛,引膺中脊内痛。治在燔针劫刺,以知为数,以痛为输,命曰孟秋痹也。
	
	足少阴之筋,起于小指之下,并足太阴之筋,邪走内踝之下,结于踵,与太阳之筋合,而上结于内辅之下,并太阴之筋,而上循阴股,结于阴器,循脊内挟膂上至项,结于枕骨,与足太阳之筋合。其病足下转筋,及所过而结者皆痛及转筋。病在此者,主癎瘈及痉,在外者不能挽,在内者不能仰。故阳病者,腰反折不能俛,阴病者,不能仰。治在燔针劫刺,以知为数,以痛为输。在内者熨引饮药,此筋折纽,纽发数甚者死不治,名曰仲秋痹也。
	
	足厥阴之筋,起于大指之上,上结于内踝之前,上循胫,上结内辅之下,上循阴股,结于阴器,络诸筋。其病足大指支内踝之前痛,内辅痛,阴股痛转筋,阴器不用,伤于内则不起,伤于寒则阴缩入,伤于热则纵挺不收,治在行水清阴气;其病转筋者,治在燔针劫刺,以知为数,以痛为输,命曰季秋痹也。
	
	手太阳之筋,起于小指之上,结于腕,上循臂内廉,结于肘内锐骨之后,弹之应小指之上,入结于腋下;其支者,后走腋后廉,上绕肩胛,循颈出走太阳之前,结于耳后完骨;其支者,入耳中;直者,出耳上,下结于颔,上属目外眦。其病小指支肘内锐骨后廉痛,循臂阴,入腋下,腋下痛,腋后廉痛,绕肩胛引颈而痛,应耳中鸣痛引颔,目瞑良久乃得视,颈筋急,则为筋痿颈肿,寒热在颈者。治在燔针劫刺之,以知为数,以痛为输。其为肿者,复而锐之。本支者,上曲牙,循耳前属目外眦,上颔结于角,其痛当所过者支转筋。治在燔针劫刺,以知为数,以痛为输,名曰仲夏痹也。
	
	手少阳之筋,起于小指次指之端,结于腕,中循臂,结于肘,上绕臑外廉、上肩、走颈,合手太阳;其支者,当曲颊入系舌本;其支者,上曲牙,循耳前,属目外眦,上乘颔,结于角。其病当所过者,即支转筋,舌卷。治在燔针劫刺,以知为数,以痛为输,名曰季夏痹也。
	
	手阳明之筋,起于大指次指之端,结于腕,上循臂,上结于肘外,上臑,结于髃;其支者,绕肩胛,挟脊;直者,从肩髃上颈;其支者,上颊,结于頄;直者,上出手太阳之前,上左角,络头,下右颔。其病当所过者,支痛及转筋,肩不举,颈不可左右视。治在燔针劫刺,以知为数,以痛为输,名曰孟夏痹也。
	
	手太阴之筋,起于大指之上,循指上行,结于鱼后,行寸口外侧,上循臂,结肘中,上臑内廉,入腋下,出缺盆,结肩前髃,上结缺盆,下结胸里,散贯贲,合贲下抵季胁。其病当所过者,支转筋,痛甚成息贲,胁急吐血。治在燔针劫刺,以知为数,以痛为输。名曰仲冬痹也。
	
	手心主之筋,起于中指,与太阴之筋并行,结于肘内廉,上臂阴,结腋下,下散前后挟胁;其支者,入腋,散胸中,结于臂。其病当所过者,支转筋前及胸痛息贲。治在燔针劫刺,以知为数,以痛为输,名曰孟冬痹也。
	
	手太阴之筋,起于小指之内侧,结于锐骨,上结肘内廉,上入腋,交太阴,挟乳里,结于胸中,循臂下系于脐。其病内急心承伏梁,下为肘网。其病当所过者,支转筋,筋痛。治在燔针劫刺,以知为数,以痛为输。其成伏梁唾血脓者,死不治。经筋之病,寒则反折筋急,热则筋弛纵不收,阴痿不用。阳急则反折,阴急则俯不伸。焠刺者,刺寒急也,热则筋纵不收,无用燔针,名曰季冬痹也。
	
	足之阳明,手之太阳,筋急则口目为僻,眦急不能卒视,治皆如右方也。
	\end{yuanwen}
	
	\chapter{骨度}
	
	\begin{yuanwen}
	黄帝问于伯高曰:脉度言经脉之长短,何以立之?
	
	伯高曰:先度其骨节之大小、广狭、长短,而脉度定矣。
	
	黄帝曰:愿闻众人之度。人长七尺五寸者,其骨节之大小长短各几何?
	
	伯高曰:头之大骨围,二尺六寸,胸围四尺五寸。腰围四尺二寸。发所覆者颅至项,尺二寸。发以下至颐,长一尺,君子终折。
	
	结喉以下至缺盆中,长四寸。缺盆以下至(骨曷)(骨亏),长九寸,过则肺大,不满则肺小。(骨曷)(骨亏)以下至天枢,长八寸,过则胃大,不及则胃小。天枢以下至横骨,长六寸半,过则回肠广长,不满则狭短。横骨,长六寸半。横骨上廉以下至内辅之上廉,长一尺八寸。内辅之上廉以下至下廉,长三寸半。内辅下廉,下至内踝,长一尺三寸。内踝以下至地,长三寸。膝腘以下至附属,长一尺六寸。附属以下至地,长三寸。故骨围大则太过,小则不及。
	
	角以下至柱骨,长一尺。行腋中不见者,长四寸。腋以下至季胁,长一尺二寸。季胁以下至髀枢,长六寸,髀枢以下至膝中,长一尺九寸。膝以下至外踝,长一尺六寸。外踝以下至京骨,长三寸。京骨以下至地,长一寸。
	
	耳后当完骨者,广九寸。耳前当耳门者,广一尺三寸。两颧之间,相去七寸。两乳之间,广九寸半。两髀之间,广六寸半。
	
	足长一尺二寸,广四寸半。肩至肘,长一尺七寸;肘至腕,长一尺二寸半。腕至中指本节,长四寸。本节至其末,长四寸半。
	
	项发以下至背骨,长二寸半,膂骨以下至尾骶,二十一节,长三尺,上节长一寸四分分之一,奇分在下,故上七节至于膂骨,九寸八分分之七。此众人骨之度也,所以立经脉之长短也。是故视其经脉之
	
	在于身也,其见浮而坚,其见明而大者,多血,细而沉者,多气也。
	\end{yuanwen}
	
	\chapter{五十营}
	
	\begin{yuanwen}
	黄帝曰:余愿闻五十营奈何?
	
	岐伯答曰:天周二十八宿,宿三十六分;人气行一周,千八分,日行二十八宿。人经脉上下左右前后二十八脉,周身十六丈二尺,以应二十八宿,漏水下百刻,以分昼夜。故人一呼脉再动,气行三寸,呼吸定息,气行六寸;十息,气行六尺,日行二分。二百七十息,气行十六丈二尺,气行交通于中,一周于身,下水二刻,日行二十五分。五百四十息,气行再周于身,下水四刻,日行四十分。二千七百息,气行十周于身,下水二十刻,日行五宿二十分。一万三千五百息,气行五十营于身,水下百刻,日行二十八宿,漏水皆尽脉终矣。所谓交通者,并行一数也。故五十营备,得尽天地之寿矣,凡行八百一十丈也。
	\end{yuanwen}
	
	\chapter{营气}
	
	\begin{yuanwen}
	黄帝曰:营气之道,内谷为宝。谷入于胃,乃传之肺,流溢于中,布散于外,精专者,行于经隧,常营无已,终而复始,是谓天地之纪。故气从太阴出注手阳明,上行注足阳明,下行至跗上,注大指间,与太阴合;上行抵髀,从脾注心中;循手少阴,出腋中臂,注小指,合手太阳;上行乘腋,出?内,注目内眦,上巅,下项,合足太阳;循脊,下尻,下行注小指之端,循足心,注足少阴;上行注肾,从肾注心外,散于胸中;循心主脉,出腋,下臂,出两筋之间,入掌中,出中指之端,还注小指次指之端,合手少阳;上行注膻中,散于三焦,从三焦注胆,出胁,注足少阳;下行至跗上,复从跗注大指间,合足厥阴,上行至肝,从肝上注肺,上循喉咙,入颃颡之窍,究于畜门。其支别者,上额,循巅,下项中,循脊,入骶,是督脉也;络阴器,上过毛中,入脐中,上循腹里,入缺盆,下注肺中,复出太阴。此营气之所行也,逆顺之常也。
	\end{yuanwen}
	
	\chapter{脉度}
	
	\begin{yuanwen}
	黄帝曰:愿闻脉度。
	
	岐伯答曰:手之六阳,从手至头,长五尺,五六三丈。手之六阴,从手至胸中,三尺五寸,三六一丈八尺,五六三尺,合二丈一尺。足之六阳,从足上至头,八尺,六八四丈八尺。足之六阴,从足至胸中,六尺五寸,六六三丈六尺,五六三尺合三丈九尺。蹻脉从足至目,七尺五寸,二七一丈四尺,二五一尺,合一丈五尺。督脉、任脉,各四尺五寸,二四八尺,二五一尺,合九尺。凡都合一十六丈二尺,此气之大经隧也。
	
	经脉为里,支而横者为络,络之别者为孙,盛而血者疾诛之,盛者泻之,虚者饮药以补之。五藏常内阅于上七窍也。故肺气通于鼻,肺和则鼻能知臭香矣;心气通于舌,心和则舌能知五味矣;肝气通于目,肝和则目能辨五色矣;脾气通于口,脾和则口能知五谷矣;肾气通于耳,肾和则耳能闻五音矣。五脏不和,则七窍不通;六腑不合则留为痈。故邪在腑则阳脉不和,阳脉不和则气留之,气留之则阳气盛矣。阳气太盛,则阴不利,阴脉不利则血留之,血留之则阴气盛矣。阴气太盛则阳气不能荣也,故曰关。阳气太盛,则阴气弗能荣也,故曰格。阴阳俱盛,不得相荣,故曰关格。关格者,不得尽期而死也。
	
	黄帝曰:蹻脉安起安止,何气荣水?
	
	岐伯答曰:蹻脉者,少阴之别,起于然骨之后。上内踝之上,直上循阴股,入阴,上循胸里,入缺盆,上出人迎之前,入頄,属目内眦,合于太阳,阳蹻而上行,气并相还,则为濡,目气不荣,则目不合。
	
	黄帝曰:气独行五脏,不荣六腑,何也?
	
	岐伯答曰:气之不得无行也,如水之流,如日月之行不休,故阴脉荣其脏,阳脉荣其腑,如环之无端,莫知其纪,终而复始,其流溢之气,内溉脏腑,外濡腠理。
	
	黄帝曰:蹻脉有阴阳,何脉当其数?
	
	岐伯曰:男子数其阳,女子数其阴,当数者为阴,其不当数者为络也。
	\end{yuanwen}
	
\chapter{营卫生会}
	
营卫来源于水谷,生成于脾胃,分为两条道路:清纯的为营气,行于脉中;慓悍的为卫气,行于脉外。一昼夜之间,两者各行于阳二十五周次,行于阴亦二十五周次,当黎明与日落的时候,交相出入,至半夜大会于手太阴。由于本篇主要论述营卫的生成和会合,故命名《营卫生会》。

\begin{yuanwen}
黄帝问于岐伯曰:人焉受气?阴阳焉会?何气为营?何气为卫?营安从生?卫于焉会?老壮不同气,阴阳异位,愿闻其会。
	
岐伯答曰:人受气于谷。谷入于胃,以传于肺,五脏六腑,皆以受气。其清者为营,浊者为卫\footnote{张景岳:“谷气出于胃,而气有清浊之分。清者,水谷之精气也;浊者,水谷之悍气也。诸家以上下焦言清浊者皆非。清者属阴,其性精专,故化生血脉,而周行于经隧之中,是为营气;浊者属阳,其性慓疾滑利,故不循经络,而直达肌表,充实于皮毛分肉之间,是为卫气。”},营在脉中,卫在脉外。营周不休,五十而复大会。阴阳相贯,如环无端。卫气行于阴二十五度,行于阳二十五度,分为昼夜。故气至阳而起,至阴而止。故曰:日中而阳陇为重阳,夜半而阴陇为重阴。故太阴主内,太阳主外,各行二十五度分为昼夜。夜半为阴陇,夜半后而为阴衰,平且阴尽,而阳受气矣。日中而阳陇\footnote{隆盛的意思。},日西而阳衰。日入阳尽,而阴受气矣。夜半而大会,万民皆卧,命曰合阴。平旦阴尽而阳受气。如是无已,与天地同纪。
\end{yuanwen}
	
\begin{yuanwen}
黄帝曰:老人之不夜瞑者,何气使然?少壮之人不昼瞑者,何气使然?
	
岐伯答曰:壮者之气血盛,其肌肉滑,气道通,营卫之行,不失其常,故昼精\footnote{指神清气爽,精神饱满。}而夜瞑。老者之气血衰,其肌肉枯,气道涩,五脏之气相博,其营气衰少而卫气内伐\footnote{衰败。},故昼不精,夜不瞑。
\end{yuanwen}
	
\begin{yuanwen}
黄帝曰:愿闻营卫之所行,皆何道从来?
	
岐伯答曰:营出于中焦,卫出于下焦\footnote{张景岳:“营气者,由谷入于胃,中焦受气取汁,化其精微,而上注于肺,乃自手太阴始,周行于经隧之中,故营气出于中焦。卫气者,出其悍气之慓疾,而先行于四末分肉皮肤之间,不入于脉,故于平旦阴尽,阳气出于目,循头项下行,始于足太阳膀胱经,而行于阳分,日西阳尽,则始于足少阴肾经,而行于阴分,其气自膀胱与肾由下而出,故卫气出于下焦。”}。
	
黄帝曰:愿闻三焦之所出。
	
岐伯答曰:上焦出于胃上口,并咽以上,贯膈而布胸中,走腋,循太阴之分而行,还至阳明,上至舌,下足阳明。常与营俱行于阳二十五度,行于阴亦二十五度,一周也。故五十度而复大会于手太阴矣\footnote{张景岳:“上焦之气,常与营气俱行于阳二十五度,阴亦二十五度。阳阴者,言昼夜也。昼夜周行五十度,至次日寅时,复会于手太阴肺经,是为一周,然则营气虽出于中焦,而施化则由于上焦也。”}。
	
黄帝曰:人有热,饮食下胃,其气未定,汗则出,或出于面,或出于背,或出于身半,其不循卫气之道而出,何也?
	
岐伯曰:此外伤于风,内开腠理,毛蒸理泄,卫气走之,固不得循其道。此气慓悍滑疾,见开而出,故不得从其道,故命曰漏泄。
\end{yuanwen}
	
\begin{yuanwen}
黄帝曰:愿闻中焦之所出。
	
岐伯答曰:中焦亦并胃中,出上焦之后。此所受气者,泌糟粕,蒸津液,化其精微,上注于肺脉,乃化而为血。以奉生身,莫贵于此。故独得行于经隧,命曰营气。
	
黄帝曰:夫血之与气,异名同类。何谓也?
	
岐伯答曰:营卫者,精气也;血者,神气也。故血之与气,异名同类焉。故夺血者无汗,夺汗者无血。故人生有两死,而无两生\footnote{人体夺血会致死亡,夺汗也会致死亡,所以说“有两死”。血与汗两者缺一则不能生,所以说“无两生”。}。
\end{yuanwen}

\begin{yuanwen}
黄帝曰:愿闻下焦之所出。
	
岐伯答曰:下焦者,别回肠,注于膀胱,而渗入焉。故水谷者,常并居于胃中,成糟粕而俱下于大肠,而成下焦。渗而俱下。济泌别汁,循下焦而渗入膀胱焉。
	
黄帝曰:人饮酒,酒亦入胃,谷未熟而小便独先下,何也?
	
岐伯答曰:酒者,熟谷之液也。其气悍以清,故后谷而入,先谷而出焉。
	
黄帝曰:善。余闻上焦如雾,中焦如沤,下焦如渎,此之谓也。
\end{yuanwen}
	
	
	\chapter{四时气}
	
	\begin{yuanwen}
	黄帝问于岐伯曰:夫四时之气,各不同形,百病之起,皆有所生,灸刺之道,何者为定?
	
	岐伯答曰:四时之气,各有所在,灸刺之道,得气穴为定。故春取经、血脉、分肉之间,甚者,深刺之,间者,浅刺之;夏取盛经孙络,取分间绝皮肤;秋取经俞。邪在腑,取之合;冬取井荥,必深以留之。
	
	温疟汗不出,为五十九痏,风(疒水)肤胀,为五十痏。取皮肤之血者,尽取之。飧泄补三阴之上,补阴陵泉,皆久留之,热行乃止。
	
	转筋于阳,治其阳;转筋于阴,治其阴。皆卒刺之。徒(疒水)先取环谷下三寸,以铍针针之,已刺而筩之,而内之,入而复之,以尽其(疒水),必坚。来缓则烦悗,来急则安静,间日一刺之,(疒水)尽乃止。饮闭药,方刺之时徒饮之,方饮无食,方食无饮,无食他食,百三十五日。
	
	着痹不去,久寒不已,卒取其三里。骨为干。肠中不便,取三里,盛泻之,虚补之。疠风者,素刺其肿上。已刺,以锐针针其处,按出其恶气,肿尽乃止。常食方食,无食他食。
	
	腹中常鸣,气上冲胸,喘不能久立。邪在大肠,刺肓之原,巨虚上廉、三里。小腹控睪,引腰脊,上冲心。邪在小肠者,连睪系,属于脊,贯肝肺,络心系。气盛则厥逆,上冲肠胃,熏肝,散于肓,结于脐,故取之肓原以散之,刺太阴以予之,取厥阴以下之,取巨虚下廉以去之,按其所过之经以调之。
	
	善呕,呕有苦,长太息,心中憺憺,恐人将捕之;邪在胆,逆在胃,胆液泄,则口苦,胃气逆,则呕苦,故曰呕胆。取三里以下。胃气逆,则刺少阳血络,以闭胆逆,却调其虚实,以去其邪。
	
	饮食不下,膈塞不通,邪在胃脘,在上脘,则刺抑而下之,在下脘,则散而去之。小腹痛肿,不得小便,邪在三焦,约取之太阳大络,视其络脉与厥阴小络结而血者,肿上及胃脘,取三里。
	
	睹其色,察其以,知其散复者,视其目色,以知病之存亡也。一其形,听其动静者,持气口人迎以视其脉,坚且盛且滑者,病日进,脉软者,病将下,诸经实者,病三日已。气口候阴,人迎候阳也。
	\end{yuanwen}
	
	\part{}
	
	\chapter{五邪}
	
	\begin{yuanwen}
	邪在肺,则病皮肤痛,寒热,上气喘,汗出,欬动肩背。取之膺中外喻,背三节五脏之傍,以手疾按之,快然,乃刺之。取之缺盆中以越之。
	
	邪在肝,则两胁中痛,寒中,恶血在内,行善掣节,时脚肿。取之行间,以引胁下,补三里以温胃中,取血脉以散恶血;取耳间青脉,以去其掣。
	
	邪在脾胃,则病肌肉痛,阳气有余,阴气不足,则热中善饥;阳气不足,阴气有余,则寒中肠鸣、腹痛;阴阳俱有余,若俱不足,则有寒有热,皆调于三里。
	
	邪在肾,则病骨痛,阴痹。阴痹者,按之而不得,腹胀,腰痛,大便难,肩背颈项痛,时眩。取之涌泉、昆仑。视有血者,尽取之。
	
	邪在心,则病心痛,喜悲时眩仆;视有余不足而调之其输也。
	\end{yuanwen}
	
	\chapter{寒热病}
	
	皮寒热者,不可附席,毛发焦,鼻槁腊。不得汗,取三阳之络,以补手太阴。肌寒热者,肌痛,毛发焦而唇槁腊。不得汗,取三阳于下,以去其血者,补足太阴,以出其汗。
	
	骨寒热者,病无所安,汗注不休。齿未槁,取其少阴于阴股之络;齿已槁,死不治。骨厥亦然。骨痹,举节不用而痛,汗注、烦心。取三阴之经,补之。
	
	身有所伤,血出多及中风寒,若有所堕坠,四肢懈惰不收,名曰体惰。取其小腹脐下三结交。三结交者,阳明太阴也,脐下三寸关元也。厥痹者,厥气上及腹。取阴阳之络,视主病也,泻阳补阴经也。
	
	颈侧之动脉人迎。人迎,足阳明也,在婴筋之前。婴筋之后,手阳明也,名曰扶突。次脉,足少阳脉也,名曰天牖。次脉,足太阳也,名曰天柱。腋下动脉,臂太阴也,名曰天府。
	
	阳迎头痛,胸满不得息,取之人迎。暴瘖气鞭,取扶突与舌本出血。暴袭气蒙,耳目不明,取天牖。暴挛?眩,足不任身,取天柱。暴痹内逆,肝肺相搏,血溢鼻口,取天府。此为天牖五部。
	
	臂阳明,有入頄遍齿者,名曰大迎。下齿龋,取之臂。恶寒补之,不恶寒泻之。足太阳有入頄遍齿者,名曰角孙。上齿龋,取之在鼻与頄前。方病之时,其脉盛,盛则泻之,虚则补之。一曰取之出鼻外。
	
	足阳明有挟鼻入于面者,名曰悬颅。属口,对入系目本,视有过者取之。损有余,益不足,反者益其。足太阳有通项入于脑者,正属目本,名曰眼系。头目苦痛,取之在项中两筋间。入脑乃别阴蹻、阳蹻,阴阳相交,阳入阴,阴出阳,交于目锐眦,阳气盛则瞋目,阴气盛则瞑目。
	
	热厥取足太阴、少阳,皆留之;寒厥取足阳明、少阴于足,皆留之。舌纵涎下,烦悗,取足少阴。振寒洒洒鼓颔,不得汗出,腹胀烦悗,取手太阴,刺虚者,刺其去也;刺实者,刺其来也。
	
	春取络脉,夏取分腠,秋取气口,冬取经输。凡此四时,各以时为齐。络脉治皮肤,分腠治肌肉,气口治筋脉,经输治骨髓。五脏,身有五部:伏兔一;腓二,腓者腨也;背三,五脏之输四;项五。此五部有痈疽者死。
	
	病始手臂者,先取手阳明、太阴而汗出;病始头首者,先取项太阳而汗出;病始足胫者,先取足阳明而汗出。臂太阴可汗出,足阳明可汗出,故取阴而汗出甚者,止之于阳,取阳而汗出甚者,止之于阴。
	
	凡刺之害,中而不去则精泄;不中而去则致气。精泄则病甚而怄,致气则生为痈疽也。
	\chapter{癫狂病}
	
	目眦外决于面者,为锐眦;在内近鼻者,为内眦;上为外眦,下为内眦。
	
	癫疾始生,先不乐,头重痛,视举目赤,甚作极,已而烦心。候之于颜。取手太阳、阳明、太阴,血变为止。
	
	癫疾始作,而引口啼呼喘悸者,候之手阳明、太阳。左强者,攻其右;右强者,攻其左,血变为止。癫疾始作,先反僵,因而脊痛,候之足太阳、阳明、太阴、手太阳,血变为止。
	
	治癫疾者,常与之居,察其所当取之处。病至,视之有过者泻之,置其血于瓠壶之中,至其发时,血独动矣,不动,灸穷骨二十壮。穷骨者,骶骨也。
	
	骨癫疾者,顑、齿诸腧、分肉皆满而骨居,汗出、烦悗,呕多沃沫,气下泄,不治。
	
	筋癫疾者,身倦挛急大,刺项大经之大杼脉,呕多沃沫,气下泄,不治。
	
	脉癫疾者,暴仆,四肢之脉皆胀而纵,脉满,尽刺之出血,不满,灸之项太阳,灸带脉于腰相去三寸,诸分肉本输。呕吐沃沫,气下泄,不治。癫疾者,疾发如狂者,死不治。
	
	狂始生,先自悲也,喜忘、苦怒、善恐者得之忧饥,治之取手太阳、阳明,血变而止,及取足太阴、阳明。狂始发,少卧不饥,自高贤也,自辩智也,自尊贵也,善骂詈,日夜不休,治之取手阳明太阳太阴舌下少阴,视之盛者,皆取之,不盛,释之也。
	
	狂言,惊,善笑,好歌乐,妄行不休者,得之大恐,治之取手阳明太阳太阴。狂,目妄见,耳妄闻,善呼者,少气之所生也;治之取手太阳太阴阳明,足太阴头两顑。
	
	狂者多食,善见鬼神,善笑而不发于外者,得之有所大喜,治之取足太阴太阳阳明,后取手太阴太阳阳明。狂而新发,未应如此者,先取曲泉左右动脉,及盛者见血,有顷已,不已,以法取之,灸骨骶二十壮。
	
	风逆,暴四肢肿,身漯漯,唏然时寒,饥则烦,饱则善变,取手太表里,足少阴阳明之径,肉清取荥,骨清取井、经也。
	
	厥逆为病也,足暴清,胸若将裂,肠若将以刀切之,烦而不能食,脉大小皆涩,暖取足少阴,清取足阳明,清则补之,温则泻之。厥逆腹胀满,肠鸣,胸满不得息,取之下胸二胁,咳而动手者,与背输,以手按之,立快者是也。
	
	内闭不得溲,刺足少阴太阳,与抵上以长针。气逆,则取其太阴、阳明、厥阴,甚取少阴、阳明,动者之经也。
	
	少气,身漯漯也,言吸吸也,骨酸体重,懈惰不能动,补足少阴。短气息短,不属,动作气索,补足少阴,去血络也。
	\chapter{热病}
	
	偏枯,身偏不用而痛,言不变,志不乱,病在分腠之间,巨针取之,益其不足,损其有余,乃可复也。
	
	痱之为病也,身无痛者,四肢不收;智乱不甚,其言微知,可治;甚则不能言,不可治也。病先起于阳,复入于阴者,先取其阳,后取其阴,浮而取之。
	
	热病三日,而气口静、人迎躁者,取之诸阳,五十九刺,以泻其热,而出其汗,实其阴,以补其不足者。身热甚,阴阳皆静者,勿刺也;其可刺者,急取之,不汗出则泄。所谓勿刺者,有死征也。
	
	热病七日八日,脉口动,喘而短者,急刺之,汗且自出,浅刺手大指间。
	
	热病七日八日,脉微小,病者溲血,口中干,一日半而死。脉代者,一日死。
	
	热病已得汗出,而脉尚躁,喘且复热,勿刺肤,喘甚者死。
	
	热病七日八日,脉不躁,躁不散改,后看中有汗;三日不汗,四日死。未曾汗者,勿腠刺之。
	
	热病先肤痛,窒鼻充面,取之皮,以第一针,五十九,苛轸鼻,索皮于肺,不得,索之火,火者,心也。
	
	热病先身涩倚而热,烦俛,干唇口溢,取之皮,以第一针,五十九;肤胀口干,寒汗出,索脉于心,不得,索之水,水者,肾也。
	
	热病溢干多饮,善惊,卧不能起,取之肤肉,以第六针,五十九,目眦青,索肉于脾,不得,索之水,木者,肝也。
	
	热病面青,脑痛,手足躁,取之筋间,以第四针于四逆;筋躄目浸,索筋于肝,不得,索之金,金者,肺也。
	
	热病数惊,瘈瘲而狂,取之脉,以第四针,急泻有余者,癫疾毛发去,索血于心,不得,索之水,水者,肾也。
	
	热病身重骨痛,耳聋而好瞑,取之骨,以第四针,五十九,刺骨;病不食,啮齿耳青,索骨于肾,不得,索之土,土者,脾也。
	
	热病不知所痛,耳聋,不能自收,口干,阳热甚,阴颇有寒者,热在髓,死不可治。
	
	热病头痛,颞颥,目(疒丰勹手)脉痛,善衄,厥热病也,取之以第三针,视有余不足,寒热痔。
	
	热病,体重,肠中热,取之以第四针,于其俞,及下诸趾间,索气于胃胳(应作络)得气也。
	
	热病挟脐急痛,胸胁满,取之涌泉与阴陵泉,取以第四针,针嗌里。
	
	热病,而汗且出,及脉顺可汗者,取之鱼际、太渊、大都、太白。泻之则热去,补之则汗出,汗出大甚,取内踝上横脉以止之。
	
	热病已得汗而脉尚躁盛,此阴脉之极也,死;其得汗而脉静者,生。
	
	热病者,脉尚盛躁而不得汗者,此阳脉之极也,死;脉盛躁得汗静者,生。
	
	热病不可刺者有九:一曰:汗不出,大颧发赤秽者死;二曰:泄而腹满甚者死;三曰:目不明,热不已者死;四曰:老人婴儿热而腹满者死;五曰:汗不出呕下血者死;六曰:舌本烂,热不已者死;七曰:咳而衄,汗不出,出不至足者死;八曰:髓热者死;九曰:热而痉者死。腰折,病病,齿噤也。凡此九者,不可刺也。
	
	所谓五十九刺者,两手外内侧各三,凡十二痏。五指间各一,凡八痏,足亦如是。头入发一寸旁三分各三,凡六痏。更入发三寸边五,凡十痏。耳前后口下者各一,项中一,凡六痏。巅上一,聪会一,发际一,廉泉一,风池二,天柱二。
	
	气满胸中喘息,取足太阴大趾之端,去爪甲如薤叶,寒则留之,热则疾之,气下乃止。
	
	心疝暴痛,取足太阴厥阴,尽刺去其血络。
	
	喉痹舌卷,口中干,烦心,心痛,臂内廉痛,不可及头,取手小指次指爪甲下,去端如韭叶。
	
	目中赤痛,从内眦始,取之阴蹻。
	
	风痉身反折,先取足太阳及腘中及血络出血,中有寒,取三里。
	
	癃,取之阴蹻及三毛上及血络出血。
	
	男子如蛊,女子如怚,身体腰脊如解,不欲饮食,先取涌泉见血,视跗上盛者,尽见血也。
	\chapter{厥病}
	
	厥头痛,面若肿起而烦心,取之足阳明太阴。厥头痛,头脉痛,心悲,善泣,视头动脉反盛者,刺尽去血,后调足厥阴。
	
	厥头痛,贞贞头重而痛,写头上五行,行五,先取手少阴,后取足少阴。厥头痛,意善忘,按之不得,取头面左右动脉,后取足太阴。
	
	厥头痛,项先痛,腰脊为应,先取天柱,后取足太阳。厥头痛,头痛甚,耳前后脉涌有热,泻出其血,后取足少阳。
	
	真头痛,头痛甚,脑尽痛,手足寒至节,死不治。
	
	头痛不可取于腧者,有所击堕,恶血在于内,若肉伤,痛未已,可则刺,不可远取也。头痛不可刺者,大痹为恶,日作者,可令少愈,不可已。头半寒痛,先取手少阳阳明,后取足少阳阳明。
	
	厥心痛,与背相控,善瘈,如从后触其心,伛偻者,肾心痛也,先取京骨、昆仑,发狂不已,取然谷。厥心痛,腹胀胸满,心尤痛甚,胃心痛也,取之大都、大白。
	
	厥心痛,痛如以锥针刺其心,心痛甚者,脾心痛也,取之然谷、太溪。
	
	厥心痛,色苍苍如死状,终日不得太息,肝心痛也,取之行间、太冲。
	
	厥心痛,卧若徒居,心痛间,动作,痛益甚,色不变,肺心痛也,取之鱼际、太渊。
	
	真心痛,手足清至节,心痛甚,日发夕死,夕发旦死。心痛不可刺者,中有盛聚,不可取于腧。
	
	肠中有虫瘕及蛟蛕,皆不可取以小针;心腸痛,憹作痛,腫聚,往來上下行,痛有休止,腹熱喜渴涎出者,是蛟蛕也。以手聚按而坚持之,无令得移,以大针刺之,久持之,虫不动,乃出针也。恐腹浓痛,形中上者。
	
	耳聋无闻,取耳中;耳鸣,取耳前动脉;耳痛不可刺者,耳中有脓,若有干聆聊,耳无闻也;耳聋取手小指次指爪甲上与肉交者,先取手,后取足;耳鸣取手中指爪甲上,左取右,右取左,先取手,后取足。
	
	足髀不可举,侧而取之,在枢合中,以员利针,大针不可刺。病注下血,取曲泉。
	
	风痹淫砾,病不可已者,足如履冰,时如入汤中,股胫淫砾,烦心头痛,时呕时悗,眩已汗出,久则目眩,悲以喜恐,短气,不乐,不出三年死也。
	\chapter{病本}
	
	先病而后逆者,治其本;先逆而后病者,治其本;先寒而后生病者,治其本;先病而后生寒者,治其本;先热而后生病者,治其本。
	
	先泄而后生他病者,治其本,必且调之,乃治其他病。先病而后中满者,治其标;先病后泄者,治其本;先中满而后烦心者,治其本。
	
	有客气,有同气。大小便不利治其标,大小便利,治其本。
	
	病发而有余,本而标之,先治其本,后治其标;病发而不足,标而本之,先治其标,后治其本,谨详察间甚,以意调之,间者并行,甚为独行;先小大便不利而后生他病者,治其本也。
	\chapter{杂病}
	
	厥挟脊而痛者,至顶,头沉沉然,目(目巟)(目巟)然,腰脊强。取足太阳腘中血络。
	
	厥胸满面肿,唇漯漯然,暴言难,甚则不能言,取足阳明。
	
	厥气走喉而不能言,手足清,大便不利,取足少阴。
	
	厥而腹向向然,多寒气,腹中谷谷,便溲难,取足太阴。
	
	嗌干,口中热如胶,取足少阴。
	
	膝中痛,取犊鼻,以员利针,发而间之。针大如牦,刺膝无疑。
	
	喉痹不能言,取足阳明;能言,取手阳明。
	
	疟不渴,间日而作,取足阳明;渴而日作,取手阳明。
	
	齿痛,不恶清饮,取足阳明;恶清饮,取手阳明。
	
	聋而不痛者,取足少阳;聋而痛者,取手阳明。
	
	衄而不止,衄血流,取足太阳;衄血,取手太阳。不已,刺宛骨下;不已,刺腘中出血。
	
	腰痛,痛上寒,取足太阳阳明;痛上热,取足厥阴;不可以俛仰,取足少阳。中热而喘,取足少阴腘中血络。
	
	喜怒而不欲食,言益小,刺足太阴;怒而多言,刺足少阳。
	
	顑痛,刺手阳明与顑之盛脉出血。
	
	项痛不可俛仰,刺足太阳;不可以顾,刺手太阳也。
	
	小腹满大,上走胃,至心,淅淅身时寒热,小便不利,取足厥阴。
	
	腹满,大便不利,腹大,亦上走胸嗌,喘息喝喝然,取足少阴。
	
	腹满食不化,腹向向然,不能大便,取足太阴。
	
	心痛引腰脊,欲呕,取足少阴。
	
	心痛,腹胀,墙墙然,大便不利,取足太阴。
	
	心痛,引背不得息,刺足少阴;不已,取手少阳。
	
	心痛引小腹满,上下无常处,便溲难,刺足厥阴。
	
	心痛,但短气不足以息,刺手太阴。
	
	心痛,当九节刺之,按,已刺按之,立已;不已,上下求之,得之立已。
	
	顑痛,刺足阳明曲周动脉,见血,立已;不已,按人迎于经,立已。
	
	气逆上,刺膺中陷者,与下胸动脉。
	
	腹痛,刺脐左右动脉,已刺按之,立已;不已,刺气街,已刺按之,立已。
	
	痿厥为四末束悗,乃疾解之,日二;不仁者,十日而知,无休,病已止。
	
	岁以草刺鼻,嚏,嚏而已;无息,而疾迎引之,立已;大惊之,亦可已。
	\chapter{周痹}
	
	黄帝问于岐伯曰:周痹之在身也,上下移徒随脉,其上下左右相应,间不容空,愿闻此痛,在血脉之中邪?将在分肉之间乎?何以致是?其痛之移也,间不及下针,其慉痛之时,不及定治,而痛已止矣。何道使然?愿闻其故?岐伯答曰:此众痹也,非周痹也。
	
	黄帝曰:愿闻众痹。岐伯对曰:此各在其处,更发更止,更居更起,以右应左,以左应右,非能周也。更发更休也。黄帝曰:善。刺之奈何?岐伯对曰:刺此者,痛虽已止,必刺其处,勿令复起。
	
	帝曰:善。愿闻周痹何如?岐伯对曰:周痹者,在于血脉之中,随脉以上,随脉以下,不能左右,各当其所。黄帝曰:刺之奈何?岐伯对曰:痛从上下者,先刺其下以过之,后刺其上以脱之。痛从下上者,先刺其上以过之,后刺其下以脱之。
	
	黄帝曰:善。此痛安生?何因而有名?岐伯对曰:风寒湿气,客于外分肉之间,迫切而为沫,沫得寒则聚,聚则排分肉而分裂也,分裂则痛,痛则神归之,神归之则热,热则痛解,痛解则厥,厥则他痹发,发则如是。帝曰:善。余已得其意矣。此内不在脏,而外未发于皮,独居分肉之间,真气不能周,故名曰周痹。故刺痹者,必先切循其下之六经,视其虚实,及大络之血结而不通,及虚而脉陷空者而调之,熨而通之。其瘈坚转引而行之。黄帝曰:善。余已得其意矣,亦得其事也。九者经巽之理,十二经脉阴阳之病也。
	\chapter{口问}
	
	黄帝闲居,辟左右而问于岐伯曰:余已闻九针之经,论阴阳逆顺,六经已毕,愿得口问。岐伯避席再拜曰:善乎哉问也,此先师之所口传也。黄帝曰:愿闻口传。岐伯答曰:夫百病之始生也,皆生于风雨寒暑,阴阳喜怒,饮食居处,大惊卒恐。则血气分离,阴阳破败,经络厥绝,脉道不通,阴阳相逆,卫气稽留,经脉虚空,血气不次,乃失其常。论不在经者,请道其方。
	
	黄帝曰:人之欠者,何气使然?岐伯答曰:卫气昼日行于阳,夜半则行于阴,阴者主夜,夜者卧;阳者主上,阴者主下;故阴气积于下,阳气未尽,阳引而上,阴引而下,阴阳相引,故数欠。阳气尽,阴气盛,则目瞑;阴气尽而阳气盛,则寤矣。泻足少阴,补足太阳。
	
	黄帝曰:人之哕者,何气使然?岐伯曰:谷入于胃,胃气上注于肺。今有故寒气与新谷气,俱还入于胃,新故相乱,真邪相攻,气并相逆,复出于胃,故为哕。补手太阴,泻足少阴。
	
	黄帝曰:人之唏者,何气使然?岐伯曰:此阴气盛而阳气虚,阴气疾而阳气徐,阴气盛而阳气绝,故为唏。补足太阳,泻足少阴。
	
	黄帝曰:人之振寒者,何气使然?岐伯曰:寒气客于皮肤,阴气盛,阳气虚,故为振寒寒栗,补诸阳。
	
	黄帝曰:人之噫者,何气使然?岐伯曰:寒气客于胃,厥逆从下上散,复出于胃,故为噫。补足太阴阳明,一曰补眉本也。
	
	黄帝曰:人之嚏者,何气使然?岐伯曰:阳气和利,满于心,出于鼻,故为嚏。补足太阳荣眉本,一曰眉上也。
	
	黄帝曰:人之亸者,何气使然?岐伯曰:胃不实则诸脉虚;诸脉虚则筋脉懈惰;筋脉懈惰则行阴用力,气不能复,故为亸。因其所在,补分肉间。
	
	黄帝曰:人之哀而泣涕出者,何气使然?岐伯曰:心者,五脏六腑之主也;目者,宗脉之所聚也,上液之道也;口鼻者,气之门户也。故悲哀愁忧则心动,心动则五脏六腑皆摇,摇则宗脉感,宗脉感则液道开,液道开,故泣涕出焉。液者,所以灌精濡空窍者也,故上液之道开则泣,泣不止则液竭;液竭则精不灌,精不灌则目无所见矣,故命曰夺精。补天柱经侠颈。
	
	黄帝曰:人之太息者,何气使然?岐伯曰:忧思则心系急,心系急则气道约,约则不利,故太息以伸出之,补手少阴心主,足少阳留之也。
	
	黄帝曰:人之涎下者,何气使然?岐伯曰:饮食者,皆入于胃,胃中有热则虫动,虫动则胃缓,胃缓则廉泉开,故涎下,补足少阴。
	
	黄帝曰:人之耳中鸣者,何气使然?岐伯曰:耳者,宗脉之所聚也,故胃中空则宗脉虚,虚则下溜,脉有所竭者,故耳鸣,补客主人,手大指爪甲上与肉交者也。
	
	黄帝曰:人之自啮舌者,何气使然?岐伯曰:此厥逆走上,脉气辈至也。少阴气至则啮舌,少阳气至则啮颊,阳明气至则啮唇矣。视主病者,则补之。
	
	凡此十二邪者,皆奇邪之走空窍者也。故邪之所在,皆为不足。故上气不足,脑为之不满,耳为之苦鸣,头为之苦倾,目为之眩。中气不足,溲便为之变,肠为之苦鸣。下气不足,则乃为痿厥心悗。补足外踝下留之。
	
	黄帝曰:治之奈何?岐伯曰:肾主为欠,取足少阴;肺主为哕,取手太阴、足少阴;唏者,阴与阳绝,故补足太阳,泻足少阴;振寒者,补诸阳;噫者,补足太阴阳明;嚏者,补足太阳眉本;亸,因其所在,补分肉间;泣出补天柱经侠颈,侠颈者,头中分也;太息,补手少阴、心主、足少阳,留之;涎下补足少阴;耳鸣补客主人,手大指爪甲上与肉交者;自啮舌,视主病者,则补之。目眩头倾,补足外踝下留之;痿厥心悗,刺足大趾间上二寸,留之,一曰足外踝下留之。
	
\part{}
	
\chapter{师传}

本篇首先强调了医生临床思维方法的重要性,提出了“顺”与“便”两个对临证具有一般指导意义的范畴。认为无论治国与治家,还是治身都必须以“顺”为最高的原则。这一思想史老子“道法自然”、“无为而无不为”思想在医学上的发挥。老子认为依道而生的自然万物包括人类,都依靠道的法则自然生化发展,人类作为有智慧的存在,虽然有自由行动的能力,但人类的行动必须因顺道的自然法则,才能成功,否则必然失败。这就是“无为而无不为”。“无为”不是无所作为,而是不以人的私意妄为,因外物变化之道而为。在医学上就要求医家认真研究病人的人情和疾病的自然规律,顺之而为,以获十全之功。所谓“顺者,非独阴阳脉论气之逆顺也,百姓人民皆欲顺其志也”。“临病人问所便”,“便”为病人人情所喜爱,或“相宜”于疾病之情,是“顺”这一原则的具体体现。“便”有三种具体的运用。对病情,既有“便寒”、“便热”之常,又有“便其相逆”之变;在人情上,还有王公大人、血食之君,“禁之则逆其志,顺之则加其病”这种难以应对的情况。都必须予以妥善的处理才能取得理想的疗效。可见,作者对“顺”和“便”的认识是非常全面深刻而富于辩证精神的。作者认为,这些知识和智慧,来源于前人的经验积累,因此必须从临床实践中,接受先师传授下来的宝贵经验,故以《师传》名篇。最后叙述了“从外知内”的诊断机理,即根据肢体、五官的形态及功能改变,来测候内脏的大小、强弱和预后吉凶等,以说明望诊的重要性。
	
\begin{yuanwen}
黄帝曰:余闻先师,有所心藏,弗著于方\footnote{方版,古代书写用的木板。}。余愿闻而藏之,则而行之。上以治民,下以治身,使百姓无病。上下和亲,德泽下流。子孙无优,传于后世。无有终时,可得闻乎?
	
岐伯曰:远乎哉问也。夫治民与自治,治彼与治此,治小与治大,治国与治家,未有逆而能治之也,夫惟顺而已矣。顺者,非独阴阳脉论气之逆顺也,百姓人民皆欲顺其志也。
\end{yuanwen}
	
\begin{yuanwen}
黄帝曰:顺之奈何?
	
岐伯曰:入国问俗,入家问讳,上堂问礼,临病人问所便\footnote{可理解为病者“喜爱”或“相宜”的意思。张景岳:“便者,相宜也。有居处之宜否,有动静之宜否,有阴阳之宜否,有寒热之宜否,有性情之宜否,有味气之宜否。临病人而失其宜,施治必相左矣。故必问病人之所便,是皆取顺之道也。”}。
	
黄帝曰:便病人奈何?
	
岐伯曰:夫中热消瘅则便寒,寒中之属则便热。胃中热则消谷,令人悬心善饥。脐以上皮热,肠中热,则出黄如糜。脐以下皮寒,肠中寒,则肠鸣飧泄。胃中寒,肠中热,则胀而且泄。胃中热,肠中寒,则疾饮,小腹痛胀。
\end{yuanwen}
	
\begin{yuanwen}
黄帝曰:胃欲寒饮,肠欲热饮,两者相逆,便之奈何?且夫王公大人血食之君,骄恣从欲,轻人,而无能禁之,禁之则逆其志,顺之则加其病,便之奈何?治之何先?
	
岐伯曰:人之情,莫不恶死而乐生。告之以其败,语之以其善,导之以其所便,开之以其所苦。虽有无道之人,恶有不听者乎?
\end{yuanwen}
	
\begin{yuanwen}
黄帝曰:治之奈何?
	
岐伯曰:春夏先治其标,后治其本;秋冬先治其本,后治其标。
\end{yuanwen}
	
\begin{yuanwen}
黄帝曰:便其相逆\footnote{张景岳:“谓于不可顺之中,而复有不得不委曲,以便其情者也。”}者奈何?
	
岐伯曰:便此者,食饮衣服,亦欲适寒温。寒无凄怆\footnote{形容寒冷重。},暑无出汗。食饮者,热无灼灼\footnote{形容食物过热。灼,烧。},寒无沧沧\footnote{形容食物过凉。沧,寒冷。}。寒温中适。故气将持。乃不致邪僻也。
\end{yuanwen}
	
	\begin{yuanwen}
	黄帝曰:本藏以身形肢节(月囷)肉,候五脏六腑之大小焉。今夫王公大人,临朝即位之君,而问焉,谁可扪循之,而后答乎?
	
	岐伯曰:身形肢节者,藏府之盖也,非面部之阅也。
	
	黄帝曰:五藏之气,阅于面者,余已知之矣,以肢节知而阅之,奈何?
	
	岐伯曰:五藏六府者,肺为之盖,巨肩陷咽,候见其外。
	
	黄帝曰:善。
	
	岐伯曰:五藏六府,心为之主,缺盆为之道,骷骨有余,以候(骨曷)(骨亏)。
	
	黄帝曰:善。
	
	岐伯曰:肝者,主为将,使之候外,欲知坚固,视目小大。
	
	黄帝曰:善。
	
	岐伯曰:脾者,主为卫,使之迎粮,视唇舌好恶,以知吉凶。
	
	黄帝曰:善。
	
	岐伯曰:肾者,主为外,使之远听,视耳好恶,以知其性。黄帝曰:
	
	善。愿闻六府之候。
	
	岐伯曰:六府者,胃为之海,庞骸、大颈、张胸,五谷乃容。鼻隧以长,以候大肠。唇厚、人中长,以候小肠。目下果大,其胆乃横。鼻孔在外,膀胱漏泄。鼻柱中央起,三焦乃约,此所以候六府者也。上下三等,藏安且良矣。
	\end{yuanwen}
	
	
	\chapter{决气}
	
	黄帝曰:余闻人有精、气、津、液、血、脉,余意以为一气耳,今乃辨为六名,余不知其所以然。岐伯曰:两神相搏,合而成形,常先身生,是谓精。何谓气?岐伯曰:上焦开发,宣五谷味,熏肤、充身、泽毛,若雾露之溉,是谓气。何谓津?岐伯曰:腠理发泄,汗出溱溱,是谓津。何谓液?岐伯曰:谷入气满,淖泽注于骨,骨属屈伸,泄泽补益脑髓,皮肤润泽,是谓液。何谓血?岐伯曰:中焦受气,取汁变化而赤,是谓血。何谓脉?岐伯曰:壅遏营气,令无所避,是谓脉。
	
	黄帝曰:六气有,有余不足,气之多少,脑髓之虚实,血脉之清浊,何以知之?岐伯曰:精脱者,耳聋;气脱者,目不明;津脱者,腠理开,汗大泄;液脱者,骨属屈伸不利,色夭,脑髓消,胫酸,耳数鸣;血脱者,色白,夭然不泽,其脉空虚,此其候也。
	
	黄帝曰:六气者,贵贱何如?岐伯曰:六气者,各有部主也,其贵贱善恶,可为常主,然五谷与胃为大海也。
	\chapter{肠胃}
	
	黄帝问于伯高曰:余愿闻六府传谷者,肠胃之大小长短,受谷之多少奈何?伯高曰:请尽言之,谷所从出入浅深远近长短之度:唇至齿长九分,口广二寸半;齿以后至会厌,深三寸半,大容五合;舌重十两,长七寸,广二寸半;咽门重十两,广一寸半。至胃长一尺六寸,胃纡曲屈,伸之,长二尺六寸,大一尺五寸,径五寸,大容三斗五升。小肠后附脊,左环回日迭积,其注于回肠者,外附于脐上。回运环十六曲,大二寸半,径八分分之少半,长三丈三尺。回肠当脐左环,回周叶积而下,回运还反十六曲,大四寸,径一寸寸之少半,长二丈一尺。广肠传脊,以受回肠,左环叶脊上下,辟大八寸,径二寸寸之大半,长二尺八寸。肠胃所入至所出,长六丈四寸四分,回曲环反,三十二曲也。
	\chapter{平人绝谷}
	
	黄帝曰:愿闻人之不食,七日而死,何也?伯高曰:臣请言其故。
	
	胃大一尺五寸,径五寸,长二尺六寸,横屈受水谷三斗五升,其中之谷,常留二斗,水一斗五升而满,上焦泄气,出其精微,慓悍滑疾,下焦下溉诸肠。
	
	小肠大二寸半,径八分分之少半,长三丈二尺,受谷二斗四升,水六升三合合之大半。
	
	回肠大四寸,径一寸寸之少半,长二丈一尺,受谷一斗,水七升半。
	
	广肠大八寸,径二寸寸之大半,长二尺八寸,受谷九升三合八分合之一。
	
	肠胃之长,凡五丈八尺四寸,受水谷九斗二升一合合之大半,此肠胃所受水谷之数也。平人则不然,胃满则肠虚,肠满则胃虚,更虚更满,故气得上下,五脏安定,血脉和利,精神乃居,故神者,水谷之精气也。故肠胃之中,当留谷二斗,水一斗五升;故平人日再后,后二升半,一日中五升,七日五七三斗五升,而留水谷尽矣;故平人不食饮七日而死者,水谷精气津液皆尽故也。

\chapter{海论}

\begin{tcolorbox}
古人在人身小宇宙,宇宙大人体,天人相应哲学观念指引下,认为人与自然界无论是在结构形态还是在生理功能上都有着相通相应的关系,可以用自然界的形态功能来比拟说明人体的形态结构和功能。由此而导出了“取类比象”或曰“取象比类”的基本方法。自然界有十二经水,人体有十二经脉与之相应。自然界的十二经水,有东南西北四海为之调节。人体十二经脉中营卫气血的生成和运行,同样有四海作为汇聚之所。本篇详论髓海(脑)、血海(冲脉)、气海(膻中)、水谷之海(胃)的生理,以及有余不足的病理,因而以《海论》名篇。
\end{tcolorbox}

\begin{yuanwen}
黄帝问于岐伯曰:余闻刺法于夫子,夫子之所言,不离于营卫血气。夫十二经脉,内属于腑脏,外络于肢节,夫子乃合之于四海乎?
	
岐伯答曰:人亦有四海\footnote{古人认为海为江河之水汇聚之处,海有四。人身髓、气、血以及饮食物也有其所汇聚之处,故比称为“四海”。},十二经水。经水者,皆注于海,海有东西南北,命曰四海。
	
黄帝曰:以人应之奈何?
	
岐伯曰:人有髓海,有血海,有气海,有水谷之海,凡此四者,以应四海也。
\end{yuanwen}
	
\begin{yuanwen}
黄帝曰:远乎哉!夫子之合人天地四海也。愿闻应之奈何?
	
岐伯答曰:必先明知阴阳表里荥腧\footnote{在此作流转、输注解。}所在,四海定矣。
\end{yuanwen}

\begin{yuanwen}
黄帝曰:定之奈何?
	
岐伯曰:胃者,水谷之海\footnote{胃能容纳饮食物,故称“水谷之海”。},其输上在气街,下至三里;冲脉者,为十二经之海,其输上在于大杼,下出于巨虚之上下廉;膻中\footnote{在此指胸中部位。}者,为气之海\footnote{张景岳:“膻中,胸中也,肺之所居。诸气者,皆属于肺,是为真气,亦曰宗气。宗气积于胸中,出于喉咙,以贯心脉,而行呼吸,故膻中为之气海。”},其输上在于柱骨之上下\footnote{指项后的哑门与大椎二穴。柱骨,亦称“天柱骨”,系指全部颈椎。},前在于人迎;脑为髓之海\footnote{张景岳:“凡骨之有髓,惟脑为最巨,故诸髓皆属于脑,而脑为髓之海。”},其输上在于其盖\footnote{脑盖骨。张景岳:“盖,脑盖骨也。即督脉之囟会、风府,亦督脉穴,此皆髓海之上下前后输也。”},下在风府。
\end{yuanwen}
	
\begin{yuanwen}
黄帝曰:凡此四海者,何利何害?何生何败?
	
岐伯曰:得顺者生,得逆者败;知调者利,不知调者害。
\end{yuanwen}
	
\begin{yuanwen}
黄帝曰:四海之逆顺\footnote{保持正常,或虽有病而趋向好转者为顺;发生病变,甚至逐渐恶化的为逆。}奈何?
	
岐伯曰:气海有余者,气满胸中,悗息面赤;气海不足,则气少不足以言。血海有余,则常想其身大,怫然\footnote{f\'u,郁闷貌。}不知其所病\footnote{形容病势进展缓慢,自己不觉得有病。};血海不足,亦常想其身小,狭然\footnote{狭小貌。张景岳:“狭,隘狭也,索然不广之貌。”}不知其所病。水谷之海有余,则腹满;水谷之海不足,则饥不受谷食。髓海有余,则轻劲多力,自过其度\footnote{超过常人一般的水平。四海之有余不足共八条,惟有“髓海有余”而见“轻劲有力,自过其度”一条,诸家都认为是无病之象。};髓海不足,则脑转耳鸣,胫痠眩冒,目无所见,懈怠安卧。
\end{yuanwen}
	
\begin{yuanwen}
黄帝曰:余已闻逆顺,调之奈何?
	
岐伯曰:审守其输\footnote{审察和掌握四海所流注部位的输穴。},而调其虚实,无犯其害。顺者得复,逆者必败。
	
黄帝曰:善。
\end{yuanwen}
	
	\chapter{五乱}
	
	黄帝曰:经脉十二者,别为五行,分为四时,何失而乱?何得而治?岐伯曰:五行有序,四时有分,相顺则治,相逆则乱。
	
	黄帝曰:何谓相顺?岐伯曰:经脉十二者,以应十二月。十二月者,分为四时。四时者,春秋冬夏,其气各异,营卫相随,阴阳已知,清浊不相干,如是则顺之而治。
	
	黄帝曰:何为逆而乱,岐伯曰:清气在阴,浊气在阳,营气顺脉,卫气逆行,清浊相干,乱于胸中,是谓大悗。故气乱于心,则烦心密嘿,俛首静伏;乱于肺,则俛仰喘喝,接手以呼;乱于肠胃,是为霍乱;乱于臂胫,则为四厥;乱于头,则为厥逆,头重眩仆。
	
	黄帝曰:五乱者,刺之有道乎?岐伯曰:有道以来,有道以去,审知其道,是谓身宝。黄帝曰:善。愿闻其道。岐伯曰:气在于心者,取之手少阴心主之俞;气在于肺者,取之手太阴荥,足少阴俞,气在于肠胃者,取之足太阴阳明,不下者,取之三里,气在于头者,取之天柱大杼,不知,取足太阳荥俞;气在于臂足,取之先去血脉,后取其阳明少阳之荥俞。
	
	黄帝曰:补泻奈何?岐伯曰:徐入徐出,谓之导气。补泻无形,谓之同精。是非有余不足也,乱气之相逆也。黄帝曰:允乎哉道,明乎哉论,请着之玉版,命曰治乱也。
	\chapter{胀论}
	
	黄帝曰:脉之应于寸口,如何而胀?岐伯曰:其脉大坚以涩者,胀也。黄帝曰:何以知藏府之胀也。岐伯曰:阴为藏,阳为府。
	
	黄帝曰:夫气之令人胀也,在于血脉之中耶,脏腑之内乎?岐伯曰:三者皆存焉,然非胀之舍也。黄帝曰:愿闻胀之舍。岐伯曰:夫胀者,皆在于脏腑之外,排脏腑而郭胸胁,胀皮肤,故命曰胀。
	
	黄帝曰:脏腑之在胸胁腹里之内也,若匣匮之藏禁器也,名有次舍,异名而同处,一域之中,其气各异,愿闻其故。黄帝曰:未解其意,再问。岐伯曰:夫胸腹,脏腑之郭也。膻中者,心主之宫城也;胃者,太仓也;咽喉、小肠者,传送也;胃之五窍者,闾里门户也;廉泉、玉英者,津液之道也。故五脏六腑者,各有畔界,其病各有形状。营气循脉,卫气逆为脉胀;卫气并脉循分为肤胀。三里而泻,近者一下,远者三下,无问虚实,工在疾泻。
	
	黄帝曰:愿闻胀形。岐伯曰:夫心胀者烦心短气,卧不安;肺胀者,虚满而喘咳;肝胀者,胁下满而痛引小腹;脾胀者,善秽,四肢烦俛,体重不能胜衣,卧不安;肾胀者,腹满引背央央然,腰髀痛。六府胀,胃胀者,腹满,胃脘痛,鼻闻焦臭,妨于食,大便难;大肠胀者,肠鸣而痛濯濯,冬日重感于寒,则餐泄不化;小肠胀者,少腹慎胀,引腰而痛;膀胱胀者,少腹而气癃;三焦胀者,气满于皮肤中,轻轻然而不坚;胆胀者,胁下痛胀,口中苦,善太息。
	
	凡此诸胀者,其道在一,明知逆顺,针数不失,泻虚补实,神去其室,致邪失正,真不可定,麤之所败,谓之天命;补虚泻实,神归其室,久塞其空,谓之良工。
	
	黄帝曰:胀者焉生?何因而有?岐伯曰:卫气之在身也,常然并脉,循分肉,行有逆顺,阴阳相随,乃得天和,五脏更始,四时循序,五谷乃化。然后厥气在下,营卫留止,寒气逆上,真邪相攻,两气相搏,乃合为胀也。黄帝曰:善。何以解惑?岐伯曰:合之于真,三合而得。帝曰:善。
	
	黄帝问于岐伯曰:胀论言无问虚实,工在疾泻,近者一下,远者三下,今有其三而不下者,其过焉在?岐伯对曰:此言陷于肉肓,而中气穴者也。不中气穴,则气内闭,针不陷肓,则气不行,上越中肉,则卫气相乱,阴阳相逐。其于胀也,当泻不泻,气故不下,三而不下,必更其道,气下乃止,不下复始,可以万全,乌有殆者乎?其于胀也,必审其弥,当泻则泻,当补则补,如鼓应桴,恶有不下者乎?
	\chapter{五癃津液别}
	
	黄帝问于岐伯曰:水谷入于口,输于肠胃,其液别为五,天寒衣薄,则为溺与气,天热衣厚则为汗,悲哀气并则为泣,中热胃缓则为唾。邪气内逆,则气为之闭塞而不行,不行则为水胀,余知其然也,不知其何由生?愿闻其道。
	
	岐伯曰:水谷皆入于口,其味有五,各注其海。津液各走其道,故三焦出气,以温肌肉,充皮肤,为其津,其流而不行者为液。
	
	天暑衣厚则腠理开,故汗出,寒留于分肉之间,聚沫则为痛。
	
	天寒则腠理闭,气湿不行,水下留于膀胱,则为溺与气。
	
	五脏六腑,心为之主,耳为之听,目为之候,肺为之相,肝为之将,脾为之卫,肾为之主外。故五脏六腑之津液,尽上渗于目,心悲气并,则心系急。心系急则肺举,肺举则液上溢。夫心系与肺,不能常举,乍上乍下,故欬而泣出矣。
	
	中热则胃中消谷,消谷则虫上下作。肠胃充郭,故胃缓,胃缓则气逆,故唾出。
	
	五谷之津液,和合而为膏者,内渗入于骨空,补益脑髓,而下流于阴股。
	
	阴阳不和,则使液溢而下流于阴,髓液皆减而下,下过度则虚,虚,故腰背痛而胫酸。
	
	阴阳气道不通,四海闭塞,三焦不泻,津液不化,水谷并行肠胃之中,别于回肠,留于下焦,不得渗膀胱,则下焦胀,水溢则为水胀,此津液五别之逆顺也。
	\chapter{五阅五使}
	
	黄帝问于岐伯曰:余闻刺有五官五阅,以观五气。五气者,五脏之使也,五时之副也。愿闻其五使当安出?岐伯曰:五官者,五脏之阅也。黄帝曰:愿闻其所出,令可为常。岐伯曰:脉出于气口,色见于明堂,五色更出,以应五时,各如其常,经气入脏,必当治理。
	
	帝曰:善。五色独决于明堂乎?岐伯曰:五官已辨,阙庭必张,乃立明堂,明堂广大,蕃蔽见外,方壁高基,引垂居外,五色乃治,平搏广大,寿中百岁,见此者,刺之必已,如是之人者,血气有余,肌肉坚致,故可苦以针。
	
	黄帝曰:愿闻五官。岐伯曰:鼻者,肺之官也;目者,肝之官也;口唇者,脾之官也;舌者,心之官也;耳者,肾之官也。
	
	黄帝曰:以官何候?岐伯曰:以候五脏。故肺病者,喘息鼻张;肝病者,眦青;脾病者,唇黄;心病者,舌卷短,颧赤;肾病者,颧与颜黑。
	
	黄帝曰:五脉安出,五色安见,其常色殆者如何?岐伯曰:五官不辨,阙庭不张,小其明堂,蕃蔽不见,又埤其墙,墙下无基,垂角去外。如是者,虽平常殆,况加疾哉。
	
	黄帝曰:五色之见于明堂,以观五脏之气,左右高下,各有形乎?岐伯曰:脏腑之在中也,各以次舍,左右上下,各如其度也。
	
\chapter{逆顺肥瘦}

逆顺是中国哲学和中国医学的重要范畴。所谓“逆”即与自然之势相逆反,顺即与自然之势相顺应。《易传》说:“数往者顺,知来者逆。”逆顺称为中国古代哲人考察自然之道的重要范畴之一。逆顺作为医学和中国哲学的范畴更是贯穿于《内经》的主要思想线索之一。逆顺运用于疾病预后,指顺证、逆证。所谓“顺证”,指预后良好的疾病,而逆证则是预后较差或可能死亡的病证。就本篇来说,逆顺指十二经脉走向与气血运行的逆顺规律。此外,本篇探讨了针刺的深浅、快慢、次数,必须根据人体的胖瘦以及年龄大小、皮肤黑白、体质强弱等来酌量决定。因以《逆顺肥瘦》名篇。
	
\begin{yuanwen}
黄帝问于岐伯曰:余闻针道于夫子,众多毕悉矣。夫子之道应若失,而据未有坚然\footnote{形容病证顽固。}者也。夫子之问学熟乎,将审察于物而心生之乎?
	
岐伯曰:圣人之为道者,上合于天,下合于地,中合于人事。必有明法,以起度数,法式\footnote{方式,方法。}检押\footnote{规则。},乃后可传焉。故匠人不能释尺寸而意短长,废绳墨而起平木也;工人不能置规而为圆,去矩而为方。知用此者,固自然之物,易用之教,逆顺之常也。
\end{yuanwen}
	
\begin{yuanwen}
黄帝曰:愿闻自然奈何?
	
岐伯曰:临深决水,不用功力,而水可竭也。循拙决冲\footnote{沿着窟处来开要塞之意。掘,通“堀”。“堀”同“窟”。},而经\footnote{路径。}可通也。此言气之滑涩,血之清浊,行之逆顺也。
\end{yuanwen}
	
\begin{yuanwen}
黄帝曰:愿闻人之白黑肥瘦少长,各有数乎?
	
岐伯曰:年质壮大,血气充盈,肤革坚固,因加以邪。刺此者,深而留之,此肥人也。广肩腋项,肉薄厚皮而黑色,唇临临然\footnote{形容口唇肥厚下垂。《广雅·释诂》:“临,大也。”大,引申有厚意。},其血黑以浊,其气涩以迟。其为人也,贪于取与。刺此者,深而留之,多益其数也。
\end{yuanwen}
	
\begin{yuanwen}
黄帝曰:刺瘦人奈何?
	
岐伯曰:瘦人者,皮薄色少,肉廉廉然\footnote{形容肌肉瘦薄。},薄唇轻言。其血清气滑,易脱于气,易损于血。刺此者,浅而疾之。
\end{yuanwen}
	
\begin{yuanwen}
黄帝曰:刺常人奈何?
	
岐伯曰:视其白黑,各为调之。其端正敦厚者,其血气和调,刺此者,无失常数也。
\end{yuanwen}
	
\begin{yuanwen}
黄帝曰:刺壮士真骨\footnote{坚固的骨骼。}者,奈何?
	
岐伯曰:刺壮士真骨,坚肉\footnote{结实的肌肉。}缓节\footnote{筋骨坚强,关节舒缓。}监监然\footnote{形容坚强有力。},此人重则气涩血浊,刺此者,深而留之,多益其数;劲则气滑血清,刺此者,浅而疾之。
\end{yuanwen}
	
\begin{yuanwen}
黄帝曰:刺婴儿奈何?
	
岐伯曰:婴儿者,其肉脆血少气弱,刺此者,以豪刺,浅刺而疾发针,日再可也。
\end{yuanwen}
	
\begin{yuanwen}
黄帝曰:临深决水,奈何?
	
岐伯曰:血清气浊,疾泻之,则气竭焉。
	
黄帝曰:循拙决冲,奈何?
	
岐伯曰:血浊气涩,疾泻之,则经可通也。
\end{yuanwen}
	
	\begin{yuanwen}
	黄帝曰:脉行之逆顺,奈何?
	
	岐伯曰:手之三阴,从脏走手;手之三阳,从手走头;足之三阳,从头走足;足之三阴,从足走腹。
	
	黄帝曰:少阴之脉独下行,何也?
	
	岐伯曰:不然,夫冲脉者,五脏六腑之海也,五脏六腑皆禀焉。其上者,出于颃颡,渗诸阳,灌诸精;其下者,注少阴之大络,出于气街,循阴股内廉入腘中,伏行骭骨内,下至内踝之后属而别。其下者,并于少阴之经,渗三阴;其前者,伏行出跗属,下循跗,入大趾间,渗诸络而温肌肉。故别络结则附上不动,不动则厥,厥则寒矣。
	
	黄帝曰:何以明之?
	
	岐伯曰:以言导之,切而验之,其非必动,然后仍可明逆顺之行也。
	
	黄帝曰:窘乎哉!圣人之为道也。明于日月,微于毫厘,其非夫子,孰能道之也。
	\end{yuanwen}
	
	
	\chapter{血络论}
	
	黄帝曰:愿闻其奇邪而不在经者。岐伯曰:血络是也。
	
	黄帝曰:刺血络而仆者,何也?血出而射者,何也?血少黑而浊者,何也?血出清而半为汁者,何也?拔针而肿者,何也?血出若多若少而面色苍苍者,何也?拔针而面色不变而烦悗者,何也?多出血而不动摇者,何也?愿闻其故。
	
	岐伯曰:脉气盛而血虚者,刺之则脱气,脱气则仆。
	
	血气俱盛而阴气多者,其血滑,刺之则射;阳气蓄积,久留而不泻者,其血黑以浊,故不能射。
	
	新饮而液渗于络,而未合和于血也,故血出而汁别焉;其不新饮者,身中有水,久则为肿。
	
	阴气积于阳,其气因于络,故刺之血未出而气先行,故肿。
	
	阴阳之气,其新相得而未和合,因而泻之,则阴阳俱脱,表里相离,故脱色而苍苍然。
	
	刺之血出多,色不变而烦悗者,刺络而虚经,虚经之属于阴者,阴脱,故烦悗。
	
	阴阳相得而合为痹者,此为内溢于经,外注于络。如是者,阴阳俱有余,虽多出血而弗能虚也。
	
	黄帝曰:相之奈何?岐伯曰:血脉者,盛坚横以赤,上下无常处,小者如针,大者如筋,则而泻之万全也,故无失数矣。失数而反,各如其度。
	
	黄帝曰:针入而肉着者,何也?岐伯曰:热气因于针,则针热,热则内着于针,故坚焉。
	\chapter{阴阳清浊}
	
	黄帝曰:余闻十二经脉,以应十二经水者,其五色各异,清浊不同,人之血气若一,应之奈何?岐伯曰:人之血气,苟能若一,则天下为一矣,恶有乱者乎?黄帝曰:余问一人,非问天下之众。岐伯曰:夫一人者,亦有乱气,天下之象,亦有乱人,其合为一耳。
	
	黄帝曰:愿闻人气之清浊。岐伯曰:受谷者浊,受气者清。清者注阴,浊者注阳。浊而清者,上出于咽,清而浊者,则下行。清浊相干,命曰乱气。
	
	黄帝曰:夫阴清而阳浊,浊者有清,清者有浊,清浊别之奈何?岐伯曰:气之大别,清者上注于肺,浊者下走于胃。胃之清气,上出于口;肺之浊气,下注于经,内积于海。
	
	黄帝曰:诸阳皆浊,何阳浊甚乎?岐伯曰:手太阳独受阳之浊,手太阴独受阴之清;其清者上走空窍,其浊者下行诸经。诸阴皆清,足太阴独受其浊。
	
	黄帝曰:治之奈何?岐伯曰:清者其气滑,浊者其气涩,此气之常也。故刺阴者,深而留之;刺阳者,浅而疾之;清浊相干者,以数调之也。
	
	\part{}
	\chapter{阴阳系日月}
	
	黄帝曰:余闻天为阳,地为阴,日为阳,月为阴,其合之于人,奈何?岐伯曰:腰以上为天,腰以下为地,故天为阳,地为阴,故足之十二经脉,以应为十二月,月生于水,故在下者为阴;手之十指,以应十日,日主火,故在上者为阳。
	
	黄帝曰:合之于脉,奈何?岐伯曰:寅者,正月之生阳也,主左足之少阳;未者,六月,主右足之少阳。卯者,二月,主左足之太阳;午者,五月,主右足之太阳。辰者,三月,主左足之阳明;巳者,四月,主右足之阳明。此两阳合于前,故曰阳明。申者,七月之生阴也,主右足之少阴;丑者,十二月,主左足之少阴;酉者,八月,主右足之太阴;子者,十一月,主左足之太阴;戌者,九月,主右足之厥阴;亥者,十月,主左足之厥阴;此两阴交尽,故曰厥阳。
	
	甲主左手之少阳;己主右手之少阳;乙主左手之太阳,戊主右手之太阳;丙主左手之阳明,丁主右手之阳明,此两火并合,故为阳明。庚主右手之少阴,癸主左手之少阴,辛主右手之太阴,壬主左手之太阴。
	
	故足之阳者,阴中之少阳也;足之阴者,阴中之太阴也。手之阳者,阳中之太阳也;手之阴者,阳中之少阴也。腰以上者为阳,腰以下者为阴。
	
	其于五脏也,心为阳中之太阳,肺为阴中之少阴,肝为阴中少阳,脾为阴中之至阴,肾为阴中之太阴。
	
	黄帝曰:以治之奈何?岐伯曰:正月二月三月,人气在左,无刺左足之阳;四月五月六月,人气在右,无刺右足之阳,七月八月九月,人气在右,无刺右足之阴,十月十一月十二月,人气在左,无刺左足之阴。
	
	黄帝曰:五行以东方为甲乙木主春。春者,苍色,主肝,肝者,足厥阴也。今乃以甲为左手之少阳,不合于数,何也?岐伯曰:此天地之阴阳也,非四时五行之以次行也。且夫阴阳者,有名而无形,故数之可十,离之可百,散之可千,推之可万,此之谓也。
	
\chapter{病传}

本篇论述疾病由外而内逐步入侵脏腑的情况;说明了脏腑疾病的传变规律以及不同的传变方式对疾病预后的影响。故以《病传》名篇。
	
\begin{yuanwen}
黄帝曰:余受九针于夫子,而私览于诸方。或有导引行气\footnote{凡人自摩自捏,伸缩手足,除劳去烦,名为导引。通过导引,以达到行气活血,养筋壮骨的目的,故曰“导引行气”。},按摩、灸、熨、刺、焫\footnote{ru\`o}、饮药之一者,可独守耶,将尽行之乎?
	
岐伯曰:诸方者,众人之方也,非一人之所尽行也。	
\end{yuanwen}
	
\begin{yuanwen}
黄帝曰:此乃所谓守一勿失,万物毕者也\footnote{马元台:“诸方虽行于众病,而医工当知乎守一。守一者,合诸方而尽明之,各守其一而勿失也。庶于万物之病,可以毕治而无误矣。”}。今余已闻阴阳之要,虚实之理,倾移之过,可治之属。愿闻病之变化,淫传绝败而不可治者,可得闻乎?
	
岐伯曰:要乎哉问!道,昭乎其如日醒,窘乎其如夜瞑。能被而服之,神与俱成。毕将服之,神自得之。生神之理,可著于竹帛,不可传于子孙。
\end{yuanwen}
	
\begin{yuanwen}
黄帝曰:何谓日醒?
	
岐伯曰:明于阴阳,如惑之解,如醉之醒。
	
黄帝曰:何谓夜瞑?
	
岐伯曰:瘖乎其无声,漠乎其无形。折毛发理,正气横倾。淫邪\footnote{指偏胜的病邪。}泮衍\footnote{p\`an。泮衍:扩散,蔓延。},血脉传溜。大气入藏\footnote{此谓严重病邪入侵于内脏。张景岳:“大气,大邪之气也。”},腹痛下淫\footnote{下焦脏气逆乱。淫,乱。}。可以致死,不可以致生。
\end{yuanwen}
	
\begin{yuanwen}
黄帝曰:大气入藏,奈何?
	
岐伯曰:病先发于心,一日而之肺,三日而之肝,五日而之脾。三日不已,死。冬夜半,夏日中。
\end{yuanwen}
	
\begin{yuanwen}
病先发于肺,三日而之肝,一日而之脾,五日而之胃。十日不已,死。冬日入,夏日出。
\end{yuanwen}
	
\begin{yuanwen}
病先发于肝,三日而之脾,五日而之胃,三日而之肾。三日不已,死。冬日入,夏早食。
\end{yuanwen}
	
\begin{yuanwen}
病先发于脾,一日而之胃,二日而之肾,三日而之膀胱。十日不已,死。冬人定\footnote{戌时。相当19--21时。},夏晏食。
\end{yuanwen}
	
\begin{yuanwen}
病先发于胃,五日而之肾,三日而之膀胱,五日而上之心。二日不已,死。冬夜半,夏日昳\footnote{text}。
\end{yuanwen}
	
	\begin{yuanwen}
	病先发于肾,三日而之膀胱,三日而上之心,三日而之小肠。三日不已,死。冬大晨\footnote{text},夏晏晡\footnote{text}。
	\end{yuanwen}
	
	\begin{yuanwen}
	病先发于膀胱,五日而之肾,一日而之小肠,一日而之心。二日不已,死。冬鸡鸣,夏下晡\footnote{text}。
	\end{yuanwen}
	
	\begin{yuanwen}
	诸病以次相传,如是者,皆有死期,不可刺也!间一脏及至三四脏者\footnote{text},乃可刺也。
	\end{yuanwen}
	
	\chapter{淫邪发梦}
	
	黄帝曰:愿闻淫邪泮衍,奈何?岐伯曰:正邪从外袭内,而未有定舍,反淫于脏,不得定处,与营卫俱行,而与魂魄飞扬,使人卧不得安而喜梦;气淫于腑,则有余于外,不足于内;气淫于脏,则有余于内,不足于外。
	
	黄帝曰:有余不足,有形乎?岐伯曰:阴气盛,则梦涉大水而恐惧;阳气盛,则梦大火而燔?;阴阳俱盛,则梦相杀。上盛则梦飞,下盛则梦堕;甚饥则梦取,甚饱则梦予;肝气盛,则梦怒,肺气盛,则梦恐惧、哭泣、飞扬;心气盛,则梦善笑恐畏;脾气盛,则梦歌、身体重不举;肾气盛,则梦腰脊两解不属。凡此十二盛者,至而泻之,立已。
	
	厥气客于心,则梦见丘山烟火;客于肺,则梦飞扬,见金铁之奇物;客于肝,则梦山林树木;客于脾,则梦见丘陵大泽,坏屋风雨;客于肾,则梦临渊,没居水中;客于膀胱,则梦游行;客于胃,则梦饮食;客于大肠,则梦田野;客于小肠,则梦聚邑冲衢;客于胆,则梦斗讼自刳;客于阴器,则梦接内;客于项,则梦斩首;客于胫,则梦行走而不能前,及居深地窌苑中;客于股肱,则梦礼节拜起;客于胞?,则梦溲便。凡此十五不足者,至而补之立已也。
	\chapter{顺气一日分为四时}
	
	黄帝曰:夫百病之所始生者,必起于燥温寒暑风雨阴阳喜怒饮食居处,气合而有形,得脏而有名,余知其然也。夫百病者,多以旦慧昼安,夕加夜甚,何也?岐伯曰:四时之气使然。
	
	黄帝曰:愿闻四时之气。岐伯曰:春生,夏长,秋收,冬藏,是气之常也,人亦应之,以一日分为四时,朝则为春,日中为夏,日入为秋,夜半为冬。朝则人气始生,病气衰,故旦慧;日中人气长,长则胜邪,故安;夕则人气始衰,邪气始生,故加;夜半人气入脏,邪气独居于身,故甚也。
	
	黄帝曰:有时有反者何也?岐伯曰:是不应四时之气,脏独主其病者,是必以脏气之所不胜时者甚,以其所胜时者起也。黄帝曰:治之奈何?岐伯曰:顺天之时,而病可与期。顺者为工,逆者为麤。
	
	黄帝曰:善,余闻刺有五变,以主五输。愿闻其数。岐伯曰:人有五脏,五脏有五变。五变有五输,故五五二十五输,以应五时。
	
	黄帝曰:愿闻五变。岐伯曰:肝为牡藏,其色青,其时春,其音角,其味酸,其日甲乙;心为牡藏,其色赤,其时夏,其日丙丁,其音征,其味苦;脾为牝藏,其色黄,其时长夏,其日戊己,其音宫,其味甘;肺为牝藏,其色白,其音商,其时征,其日庚辛,其味辛;肾为牝藏,其色黑,其时冬,其日壬癸,其音羽,其味咸。是为五变。
	
	黄帝曰:以主五输奈何?藏主冬,冬刺井;色主春,春刺荥;时主夏,夏刺输;音主长夏,长夏刺经;味主秋,秋刺合。是谓五变,以主五输。
	
	黄帝曰:诸原安和,以致五输。岐伯曰:原独不应五时,以经合之,以应其数,故六六三十六输。
	
	黄帝曰:何谓藏主冬,时主夏,音主长夏,味主秋,色主春。愿闻其故。岐伯曰:病在藏者,取之井;病变于色者,取之荥;病时间时甚者,取之输;病变于音者,取之经;经满而血者,病在胃;及以饮食不节得病者,取之于合,故命曰味主合。是谓五变也。

	\chapter{外揣}

	\begin{yuanwen}
	黄帝曰:余闻九针九篇,余亲受其词\footnote{text},颇得其意。夫九针者,始于一而终于九\footnote{text},然未得其要道也。夫九针者,小之则无内,大之则无外,深不可为下,高不可为盖。恍惚无穷,流溢无极。余知其合于天道、人事、四时之变也。然余愿杂之毫毛,浑束为一,可乎?
	
	岐伯曰:明乎哉问也!非独针道焉,夫治国亦然。
	\end{yuanwen}
	
	\begin{yuanwen}
	黄帝曰:余愿闻针道,非国事也。
	
	岐伯曰:夫治国者,夫惟道焉。非道,何可小大深浅,杂合而为一乎?
	\end{yuanwen}
	
	\begin{yuanwen}
	黄帝曰:愿卒闻之。
	
	岐伯曰:日与月焉,水与镜焉,鼓与响焉。夫日月之明,不失其影。水镜之察,不失其形。鼓响之应,不后其声。动摇则应和,尽得其情。
	\end{yuanwen}

	\begin{yuanwen}
	黄帝曰:窘乎哉!昭昭之明不可蔽。其不可蔽,不失阴阳也。合而察之,切而验之,见而得之,若清水明镜之不失其形也。五音不彰,五色不明,五脏波荡,若是则内外相袭\footnote{text},若鼓之应桴,响之应声,影之似形。故远者司外揣内\footnote{text},近者司内揣外。是谓阴阳之极,天地之盖。请藏之灵兰之室\footnote{text},弗敢使泄也。
	\end{yuanwen}
	
	\chapter{五变}
	
	黄帝问于少俞曰:余闻百疾之始期也,必生于风雨寒暑,循毫毛而入腠理,或复还,或留止,或为风肿汗出,或为消瘅,或为寒热,或为留痹,或为积聚。奇邪淫溢,不可胜数,愿闻其故。夫同时得病,或病此,或病彼,意者天之为人生风乎,何其异也?少俞曰:夫天之生风者,非以私百姓也,其行公平正直,犯者得之,避者得无殆,非求人而人自犯之。
	
	黄帝曰:一时遇风,同时得病,其病各异,愿闻其故。少俞曰:善乎其问!请论以比匠人。匠人磨斧斤,砺刀削断材木。木之阴阳,尚有坚脆,坚者不入,脆者皮弛,至其交节,而缺斤斧焉。夫一木之中,坚脆不同,坚者则刚,脆者易伤,况其材木之不同,皮之厚薄,汁之多少,而各异耶。夫木之蚤花先生叶者,遇春霜烈风,则花落而叶萎;久曝大旱,则脆木薄皮者,枝条汁少而叶萎;久阴淫雨,则薄皮多汁者,皮漉而浅;卒风暴起,则刚脆之木,根摇而叶落。凡此五者,各有所伤,况于人乎!
	
	黄帝曰:以人应木,奈何?少俞答曰:木之所伤也,皆伤其枝。枝之刚脆而坚,未成伤也。人之有常病也,亦因其骨节皮肤腠理之不坚固者,邪之所舍也,故常为病也。
	
	黄帝曰:人之善病风厥漉汗者,何以候之?少俞答曰:内不坚,腠理疏,则善病风。黄帝曰:何以候肉之不坚也?少俞答曰:腘肉不坚,而无分理。理者麤理,麤理而皮不致者,腠理疏。此言其浑然者。
	
	黄帝曰:人之善病消瘅者,何以候之?少俞答曰:五脏皆柔弱者,善病消瘅。黄帝曰:何以知五脏之柔弱也?少俞答曰:夫柔弱者,必有刚强,刚强多怒,柔者易伤也。黄帝曰:何以候柔弱之与刚强?少俞答曰:此人薄皮肤,而目坚固以深者,长冲直肠,其心刚,刚则多怒,怒则气上逆,胸中蓄积,血气逆留,髋皮充肌,血脉不行,转而为热,热则消肌肤,故为消瘅。此言其人暴刚而肌肉弱者也。
	
	黄帝曰:人之善病寒热者,何以候之?少俞答曰:小骨弱肉者,善病寒热。黄帝曰:何以候骨之小大,肉之坚脆,色之不一也?少俞答曰:颧骨者,骨之本也。颧大则骨大,颧小则骨小。皮肤薄而其肉无(月囷),其臂懦懦然,其地色殆然,不与其天同色,污然独异,此其候也。然后臂薄者,其髓不满,故善病寒热也。
	
	黄帝曰:何以候人之善病痹者?少俞答曰:麤理而肉不坚者,善病痹。黄帝曰:痹之高下有处乎?少俞答曰:欲知其高下者,各视其部。
	
	黄帝曰:人之善病肠中积聚者,何以候之?少俞答曰:皮肤薄而不泽,肉不坚而淖泽。如此,则肠胃恶,恶则邪气留止,积聚乃伤脾胃之间,寒温不次,邪气稍至。蓄积留止,大聚乃起。
	
	黄帝曰:余闻病形,已知之矣!愿闻其时。少俞答曰:先立其年,以知其时。时高则起,时下则殆,虽不陷下,当年有冲道,其病必起,是谓因形而生病,五变之纪也。

	\chapter{本藏}
	
	\begin{yuanwen}
	黄帝问于岐伯曰:人之血气精神者,所以奉生而周于性命者也\footnote{text}。经脉者,所以行血气而营阴阳,濡筋骨,利关节者也;卫气者,所以温分肉,充皮肤,肥腠理,司开阖者也\footnote{text};志意者,所以御精神,收魂魄,适寒温,和喜怒者也。是故血和则经脉流行,营复阴阳,筋骨劲强,关节清利矣。卫气和则分肉解利,皮肤调柔,腠理致密矣。志意和则精神专直\footnote{text},魂魄不散,悔怒不起,五脏不受邪矣。寒温和则六腑化谷,风痹不作,经脉通利,肢节得安矣。此人之常平也。五脏者,所以藏精神血气魂魄者也;六腑者,所以化水谷而行津液者也。此人之所以具受于天也,无愚智贤不肖,无以相倚\footnote{text}也。然有其独尽天寿,而无邪僻之病,百年不衰,虽犯风雨卒寒大暑,犹有弗能害也;有其不离屏蔽室内\footnote{text},无怵惕之恐,然犹不免于病,何也?愿闻其故。
	\end{yuanwen}
	
	\begin{yuanwen}
	岐伯对曰:窘乎哉问也!五脏者,所以参天地,副阴阳\footnote{text},而连四时,化五节者也\footnote{text};五脏者,固有小大、高下、坚脆、端正、偏倾者;六腑亦有小大、长短、厚薄、结直、缓急。凡此二十五者\footnote{text},各不同,或善或恶,或吉或凶。请言其方。
	\end{yuanwen}
	
	\begin{yuanwen}
	心小则安,邪弗能伤,易伤以忧;心大则忧不能伤,易伤于邪。心高则满于肺中,悗而善忘,难开以言;心下则藏外\footnote{text},易伤于寒,易恐以言。心坚则藏安守固;心脆则善病消瘅热中。心端正则和利难伤;心偏倾则操持不一,无守司也。
	\end{yuanwen}
	
	\begin{yuanwen}
	肺小则少饮,不病喘喝;肺大则多饮,善病胸痹、喉痹、逆气。肺高则上气肩息咳;肺下则居贲迫肺,善胁下痛。肺坚则不病咳上气;肺脆则苦病消瘅易伤。肺端正则和利难伤;肺偏倾则胸偏痛也。
	\end{yuanwen}
	
	\begin{yuanwen}
	肝小则脏安,无胁下之病;肝大则逼胃迫咽,迫咽则苦膈中,且胁下痛。肝高则上支贲切\footnote{text},胁挽,为息贲;肝下则逼胃,胁下空,胁下空则易受邪。肝坚则藏安难伤;肝脆则善病消瘅易伤。肝端正则和利难伤;肝偏倾则胁下痛也。
	\end{yuanwen}
	
	\begin{yuanwen}
	脾小则脏安,难伤于邪也;脾大则苦凑䏚而痛\footnote{text},不能疾行。脾高,则䏚引季胁而痛\footnote{text};脾下则下加于大肠,下加于大肠则脏苦受邪。脾坚则脏安难伤;脾脆则善病消瘅易伤。脾端正则和利难伤;脾偏倾则善满善胀也。
	\end{yuanwen}
	
	\begin{yuanwen}
	肾小则脏安难伤;肾大,则善病腰痛,不可以俯仰,易伤以邪。肾高则苦背膂痛,不可以俯仰;肾下则腰尻痛\footnote{text},不可以俯仰,为狐疝。肾坚则不病腰背痛;肾脆则善病消瘅易伤。肾端正则和利难伤;肾偏倾则苦尻痛也。凡此二十五变者,人之所苦常病。
	\end{yuanwen}
	
	\begin{yuanwen}
	黄帝曰:何以知其然也?
	
	岐伯曰:赤色小理者,心小,粗理者,心大。无𩩲𩨗者,心高;𩩲𩨗小、短、举者,心下。𩩲𩨗长者,心下坚;𩩲𩨗弱以薄者,心脆。𩩲𩨗直下不举者,心端正;𩩲𩨗倚一方者,心偏倾也。
	\end{yuanwen}
	
	\begin{yuanwen}
	白色小理者,肺小;粗理者,肺大。巨肩反膺陷喉者,肺高\footnote{text};合腋张胁者,肺下\footnote{text}。好肩背厚者,肺坚;肩背薄者,肺脆。背膺厚者,肺端正;胁偏疏者,肺偏倾也。
	\end{yuanwen}
	
	\begin{yuanwen}
	青色小理者,肝小;粗理者,肝大。广胸反骹者,肝高\footnote{text};合胁兔骹者,肝下\footnote{text}。胸胁好者,肝坚;胁骨弱者,肝脆。膺腹好相得者,肝端正;胁骨偏举者,肝偏倾也。
	\end{yuanwen}
	
	\begin{yuanwen}
	黄色小理者,脾小;粗理者,脾大。揭唇者\footnote{text},脾高;唇下纵者,脾下。唇坚者,脾坚;唇大而不坚者,脾脆。唇上下好者,脾端正;唇偏举者,脾偏倾也。
	\end{yuanwen}
	
	\begin{yuanwen}
	黑色小理者,肾小;粗理者,肾大。高耳者,肾高;耳后陷者,肾下。耳坚者,肾坚;耳薄不坚者,肾脆。耳好前居牙车者,肾端正\footnote{text};耳偏高者,肾偏倾也。凡此诸变者,持则安,减则病也。
	\end{yuanwen}
	
	\begin{yuanwen}
	帝曰:善。然非余之所问也。愿闻人之有不可病者,至尽天寿,虽有深忧大恐,怵惕之志,犹不能感\footnote{text}也,甚寒大热,不能伤也;其有不离屏蔽室内,又无怵惕之恐,然不免于病者,何也?愿闻其故。
	
	岐伯曰:五脏六腑,邪之舍也,请言其故。五脏皆小者,少病,苦燋心\footnote{text},大愁忧;五脏皆大者,缓于事,难使以忧。五脏皆高者,好高举措;五脏皆下者,好出人下。五脏皆坚者,无病;五脏皆脆者,不离于病。五脏皆端正者,和利得人心;五脏皆偏倾者,邪心而善盗,不可以为人,卒反复言语也。
	\end{yuanwen}
	
	\begin{yuanwen}
	黄帝曰:愿闻六腑之应。
	
	岐伯答曰:肺合大肠,大肠者,皮其应。心合小肠,小肠者,脉其应。肝合胆,胆者,筋其应。脾合胃,胃者,肉其应。肾合三焦膀胱,三焦膀胱者,腠理毫毛其应。
	\end{yuanwen}
	
	\begin{yuanwen}
	黄帝曰:应之奈何?
	
	岐伯曰:肺应皮。皮厚者大肠厚,皮薄者大肠薄。皮缓,腹裹大者,大肠大而长\footnote{text};皮急者,大肠急而短。皮滑者,大肠直\footnote{text};皮肉不相离者,大肠结\footnote{text}。
	\end{yuanwen}
	
	\begin{yuanwen}
	心应脉。皮厚者脉厚,脉厚者小肠厚。皮薄者脉薄,脉薄者小肠薄。皮缓者脉缓,脉缓者,小肠大而长。皮薄而脉冲小者\footnote{text},小肠小而短。诸阳经脉皆多纡屈者,小肠结。
	\end{yuanwen}
	
	\begin{yuanwen}
	脾应肉。肉䐃坚大者胃厚;肉䐃幺者胃薄\footnote{text}。肉䐃小而幺者胃不坚;肉䐃不称身者胃下,胃下者下管约不利\footnote{text}。肉䐃不坚者胃缓;肉䐃无小裹累者胃急。肉䐃多少裹累者胃结,胃结者,上管约不利也\footnote{text}。
	\end{yuanwen}
	
	\begin{yuanwen}
	肝应爪。爪厚色黄者胆厚;爪薄色红者胆薄;爪坚色青者胆急;爪濡色赤者胆缓;爪直色白无约者胆直;爪恶色黑多纹者胆结也。
	\end{yuanwen}
	
	\begin{yuanwen}
	肾应骨,密理厚皮者,三焦膀胱厚\footnote{text};粗理薄皮者,三焦膀胱薄。疏腠理者,三焦膀胱缓;皮急而无毫毛者,三焦膀胱急。毫毛美而粗者,三焦膀胱直,稀毫毛者,三焦膀胱结也。
	\end{yuanwen}
	
	\begin{yuanwen}
	黄帝曰:厚薄美恶皆有形,愿闻其所病。
	
	岐伯答曰:视其外应,以知其内脏,则知所病矣。
	\end{yuanwen}
	
	
	\part{}
	\chapter{禁服}
	
	雷公问于黄帝曰:细子得受,通于九针六十篇,旦暮勤服之,近者编绝,久者简垢,然尚讽诵弗置,未尽解于意矣。「外揣」言浑束为一,未知所谓也。夫大则无外,小则无内,大小无极,高下无度,束之奈何?士之才力,或有厚薄,智虑褊浅,不能博大深奥,自强于学若细子。细子恐其散于后世,绝于子孙,敢问约之奈何?黄帝曰:善乎哉问也。此先师之所禁,坐私传之也,割臂歃血之盟也,子若欲得之,何不斋乎。
	
	雷公再拜而起曰:请闻命于是也,乃斋宿三日而请曰:敢问今日正阳,细子愿以受盟。黄帝乃与俱入斋室,割臂歃血,黄帝亲祝曰:今日正阳,歃血传方,有敢背此言者,反受其殃。雷公再拜曰:细子受之。黄帝乃左握其手,右授之书曰:慎之慎之,吾为子言之,凡刺之理,经脉为始,营其所行,知其度量,内刺五脏,外刺六腑,审察卫气,为百病母,调其虚实,虚实乃止,泻其血络,血尽不殆矣。
	
	雷公曰:此皆细子之所以通,未知其所约也。黄帝曰:夫约方者,犹约囊也,囊满而弗约,则输泄,方成弗约,则神与弗俱。雷公曰:愿为下材者,勿满而约之。黄帝曰:未满而知约之以为工,不可以为天下师。
	
	雷公曰:愿闻为工。黄帝曰:寸口主中,人迎主外,两者相应,俱往俱来,若引绳大小齐等。春夏人迎微大,秋冬寸口微大,如是者,名曰平人。
	
	人迎大一倍于寸口,病在足少阳,一倍而躁,在手少阳。人迎二倍,病在足太阳,二倍而躁,病在手太阳。人迎三倍,病在足阳明,三倍而躁,病在手阳明。盛则为热,虚则为寒,紧则为痛痹,代则乍甚乍间。盛则泻之,虚则补之,紧痛则取之分肉,代则取血络,且饮药,陷下则灸之,不盛不虚,以经取之,名曰经刺。人迎四倍者,且大且数,名曰溢阳,溢阳为外格,死不治。必审按其本末,察其寒热,以验其脏腑之病。
	
	寸口大于人迎一倍,病在足厥阴,一倍而躁,在手心主。寸口二倍,病在足少阴,二倍而躁,在手少阴。寸口三倍,病在足太阴,三倍而躁,在手太阴。盛则胀满,寒中,食不化,虚则热中、出糜、少气、溺色变,紧则痛痹,代则乍痛乍止。盛则泻之,虚则补之,紧则先刺而后灸之,代则取血络,而后调之,陷下则徒灸之,陷下者,脉血结于中,中有着血,血寒,故宜灸之,不盛不虚,以经取之。寸口四倍者,名曰内关,内关者,且大且数,死不治。必审察其本末之寒温,以验其脏腑之病。
	
	通其营输,乃可传于大数。大数曰:盛则徒泻之,虚则徒补之,紧则灸刺,且饮药,陷下则徒灸之,不盛不虚,以经取之。所谓经治者,饮药,亦曰灸刺,脉急则引,脉大以弱,则欲安静,用力无劳也。

	\chapter{五色}

	\begin{yuanwen}
	雷公问于黄帝曰:五色独决于明堂乎?小子未知其所谓也\footnote{text}。
	
	黄帝曰:明堂者,鼻也;阙者,眉间也;庭者,颜也;蕃者,颊侧也;蔽者,耳门也。其间欲方大\footnote{text},去之十步,皆见于外。如是者寿,必中百岁。
	\end{yuanwen}
	
	\begin{yuanwen}
	雷公曰:五言之辨,奈何?
	
	黄帝曰:明堂骨高以起,平以直。五脏次于中央\footnote{text},六腑挟其两侧\footnote{text}。首面上于阙庭,王宫在于下极\footnote{text}。五脏安于胸中,真色以致,病色不见。明堂润泽以清。五官恶得无辨乎?
	
	雷公曰:其不辨者,可得闻乎?
	
	黄帝曰:五色之见也,各出其色部。部骨陷者,必不免于病矣。其色部乘袭者\footnote{text},虽病甚,不死矣。
	
	雷公曰:官五色奈何?
	
	黄帝曰:青黑为痛,黄赤为热,白为寒。是谓五官。
	\end{yuanwen}
	
	\begin{yuanwen}
	雷公曰:病之益甚,与其方衰,如何?
	
	黄帝曰:外内皆在焉。切其脉口滑小紧以沉者,病益甚,在中;人迎气大紧以浮者,其病益甚,在外。其脉口浮滑者,病日进;人迎沉而滑者,病日损。其脉口滑以沉者,病日进,在内;其人迎脉滑盛以浮者,其病日进,在外。脉之浮沉及人迎与寸口气小大等者,病易已;病之在藏,沉而大者,易已,小为逆;病在腑,浮而大者,其病易已。人迎盛坚者,伤于寒;气口盛坚者,伤于食。
	\end{yuanwen}
	
	\begin{yuanwen}	
	雷公曰:以色言病之间甚,奈何?
	
	黄帝曰:其色粗以明\footnote{text},沉夭者为甚。其色上行者,病益甚;其色下行,如云彻散者,病方已。五色各有脏部\footnote{text},有外部,有内部也。色从外部走内部者,其病从外走内;其色从内走外者,其病从内走外。病生于内者,先治其阴,后治其阳,反者益甚。其病生于阳者,先治其外,后治其内,反者益甚。其脉滑大以代而长者,病从外来。目有所见,志有所恶。此阳气之并也,可变而已。
	\end{yuanwen}
	
	\begin{yuanwen}
	雷公曰:小子闻风者,百病之始也;厥逆者,寒湿之起也。别之奈何?
	
	黄帝曰:常候阙中,薄泽为风\footnote{text},冲浊为痹\footnote{text}。在地为厥\footnote{text}。此其常也。各以其色言其病。
	\end{yuanwen}
	
	\begin{yuanwen}
	雷公曰:人不病卒死,何以知之?
	
	黄帝曰:大气入于脏腑者\footnote{text},不病而卒死矣。
	
	雷公曰:病小愈而卒死者,何以知之?
	
	黄帝曰:赤色出两颧,大如母指者,病虽小愈,必卒死。黑色出于庭,大如母指,必不病而卒死。
	\end{yuanwen}
	
	\begin{yuanwen}
	雷公再拜曰:善哉!其死有期乎?
	
	黄帝曰:察色以言其时。
	
	雷公曰:善乎!愿卒闻之。
	
	黄帝曰:庭者,首面也;阙上者,咽喉也;阙中者,肺也;下极者\footnote{text},心也;直下者\footnote{text},肝也;肝左者,胆也;下者\footnote{text},脾也;方上者\footnote{text},胃也;中央者\footnote{text},大肠也;挟大肠者,肾也;当肾者,脐也;面王以上者\footnote{text},小肠也,面王以下者,膀胱、子处也;颧者,肩也;颧后者,臂也;臂下者,手也;目内眦上者,膺乳也;挟绳而上者\footnote{text},背也;循牙车以下者\footnote{text},股也;中央者,膝也;膝以下者,胫也;当胫以下者,足也;巨分者\footnote{text},股里也;巨屈者\footnote{text},膝膑也。此五脏六腑肢节之部也,各有部分。有部分,用阴和阳,用阳和阴。当明部分,万举万当。能别左右,是谓大道;男女异位,故曰阴阳。审察泽夭,谓之良工。
	\end{yuanwen}
	
	\begin{yuanwen}
	沉浊为内,浮泽为外。黄赤为风,青黑为痛,白为寒。黄而膏润为脓,赤甚者为血。痛甚为挛,寒甚为皮不仁。五色各见其部,察其浮沉,以知浅深;察其泽夭,以观成败。察其散抟\footnote{tu\'an},以知远近。视色上下,以知病处。积神于心,以知往今。故相气不微,不知是非。属意勿去,乃知新故。色明不粗,沉夭为甚,不明不泽,其病不甚。其色散,驹驹然\footnote{text},未有聚;其病散而气痛,聚未成也。
	\end{yuanwen}
	
	\begin{yuanwen}
	肾乘心,心先病,肾为应,色皆如是。
	\end{yuanwen}
	
	\begin{yuanwen}
	男子色在于面王,为小腹痛,下为卵痛。其圜直为茎痛\footnote{text}。高为本,下为首\footnote{text}。狐疝㿉阴之属也\footnote{text}。
	\end{yuanwen}

	\begin{yuanwen}
	女子在于面王,为膀胱、子处之病。散为痛,抟为聚。方员左右,各如其色形。其随而下至胝为淫\footnote{text}。有润如膏状,为暴食不洁。
	\end{yuanwen}
	
	\begin{yuanwen}
	左为左,右为右。其色有邪,聚散而不端。面色所指者也。色者,青、黑、赤、白、黄,皆端满有别乡\footnote{text}。别乡赤者,其色赤,大如榆荚,在面王为不日。其色上锐,首空上向,下锐下向,在左右如法。以五色命脏,青为肝,赤为心,白为肺,黄为脾,黑为肾。肝合筋,心合脉,肺合皮,脾合肉,肾合骨也。
	\end{yuanwen}
	
	
	
	\chapter{论勇}
	
	黄帝问于少俞曰:有人于此,并行并立,其年之长少等也,衣之厚薄均也,卒然遇烈风暴雨,或病或不病,或皆病,或皆不病,其故何也?少俞曰:帝问何急?黄帝曰:愿尽闻之。少俞曰:春青风夏阳风,秋凉风,冬寒风。凡此四时之风者,其所病各不同形。
	
	黄帝曰:四时之风,病人如何?少俞曰:黄色薄皮弱肉者,不胜春之虚风;白色薄皮弱肉者,不胜夏之虚风;青色薄皮弱肉,不胜秋之虚风;赤色薄皮弱肉,不胜冬之虚风也。黄帝曰:黑色不病乎?少俞曰:黑色而皮厚肉坚,固不伤于四时之风;其皮薄而肉不坚,色不一者,长夏至而有虚风者,病矣。其皮厚而肌肉坚者,长夏至而有虚风,不病矣。其皮厚而肌肉坚者,必重感于寒,外内皆然,乃病。黄帝曰:善。
	
	黄帝曰:夫人之忍痛与不忍痛,非勇怯之分也。夫勇士之不忍痛者,见难则前,见痛则止;夫怯士之忍痛者,闻难则恐,遇痛不动。夫勇士之忍痛者,见难不恐,遇痛不动;夫怯士之不忍痛者,见难与痛,目转面盻,恐不能言,失气,惊,颜色变化,乍死乍生。余见其然也,不知其何由,愿闻其故。少俞曰:夫忍痛与不忍痛者,皮肤之薄厚,肌肉之坚脆,缓急之分也,非勇怯之谓也。
	
	黄帝曰:愿闻勇怯之所由然。少俞曰:勇士者,目深以固,长冲直扬,三焦理横,其心端直,其肝大以坚,其胆满以傍,怒则气盛而胸张,肝举而胆横,眦裂而目扬,毛起而面苍,此勇士之由然者也。
	
	黄帝曰:愿闻怯士之所由然。少俞曰:怯士者,目大而不减,阴阳相失,其焦理纵,(骨曷)(骨亏)短而小,肝系缓,其胆不满而纵,肠胃挺,胁下空,虽方大怒,气不能满其胸,肝肺虽举,气衰复下,故不能久怒,此怯士之所由然者也。
	
	黄帝曰:怯士之得酒,怒不避勇士者,何脏使然?少俞曰:酒者,水谷之精,熟谷之液也,其气慓悍,其入于胃中,则胃胀,气上逆,满于胸中,肝浮胆横,当是之时,固比于勇士,气衰则悔。与勇士同类,不知避之,名曰酒悖也。
	\chapter{背腧}
	
	黄帝问于岐伯曰:愿闻五脏之腧,出于背者。岐伯曰:背中大腧,在杼骨之端,肺腧在三焦之间,心腧在五焦之间,膈腧在七焦之间,肝腧在九焦之间,脾腧在十一焦之间,肾腧在十四焦之间。皆挟脊相去三寸所,则欲得而验之,按其处,应在中而痛解,乃其输也。灸之则可刺之则不可。气盛则泻之,虚则补之。以火补者,毋吹其火,须自灭也;以火泻之,疾吹其火,传其艾,须其火灭也。
	\chapter{卫气}
	
	黄帝曰:五脏者,所以藏精神魂魄者也;六腑者,所以受水谷而行化物者也。其气内干五脏,而外络肢节。其浮气之不循经者,为卫气;其精气之行于经者,为营气。阴阳相随,外内相贯,如环之无端。亭亭淳淳乎,孰能窃之。然其分别阴阳,皆有标本虚实所离之处。能别阴阳十二经者,知病之所生;候虚实之所在者,能得病之高下;知六腑之气街者,能知解结契绍于门户;能知虚实之坚软者,知补泻之所在;能知六经标本者,可以无惑于天下。
	
	岐伯曰:博哉!圣帝之论。臣请尽意悉言之。足太阳之本,在根以上五寸中,标在两络命门。命门者,目也。足少阳之本,在窍阴之间,标在窗笼之前。窗笼者,耳也。足少阴之本,在内踝下上三寸中,标在背输与舌下两脉也。足厥阴之本,在行间上五寸所,标在背腧也。足阳明之本,在厉兑,标在人迎,颊挟颃颡也。足太阴之本,在中封前上四寸之中,标在背腧与舌本也。
	
	手太阳之本,在外踝之后,标在命门之上一寸也。手少阳之本,在小指次指之间上二寸,标在耳后上角下外眦也。手阳明之本,在肘骨中,上至别阳,标在颜下合钳上也。手太阴之本,在寸口之中,标在腋内动也。手少阴之本,在锐骨之端,标在背腧也。手心主之本,在掌后两筋之间二寸中,标在腋下下三寸也。
	
	凡候此者,下虚则厥,下盛则热;上虚则眩,上盛则热痛。故石者,绝而止之,虚者,引而起之。
	
	请言气街,胸气有街,腹气有街,头气有街,胫气有街。故气在头者,止之于脑;气在胸者,止之膺与背腧;气在腹者,止之背腧,与冲脉于脐左右之动脉者;气在胫者,止之于气街,与承山踝上以下。取此者,用毫针,必先按而在久应于手,乃刺而予之。所治者,头痛眩,腹痛中满暴胀,及有新。痛可移者,易已也;积不痛,难已也。
	\chapter{论痛}
	
	黄帝问于少俞曰:筋骨之强弱,肌肉之坚脆,皮肤之厚薄,腠理之疏密,各不同,其于针石火焫之痛何如?肠胃之厚薄坚脆亦不等,其于毒药何如?愿尽闻之。少俞曰:人之骨强、筋弱、肉缓、皮肤厚者,耐痛,其于针石之痛火焫亦然。
	
	黄帝曰:其耐火煤者,何以知之?少俞答曰:加以黑色而美骨者,耐火焫。黄帝曰:其不耐针石之痛者,何以知之?少俞曰:坚肉薄皮者,不耐针石之痛,于火焫亦然。
	
	黄帝曰:人之病,或同时而伤,或易已,或难已,其故何如?少俞曰:同时而伤,其身多热者,易已;多寒者,难已。
	
	黄帝曰:人之胜毒,何以知之?少俞曰:胃厚、色黑、大骨及肥骨者,皆胜毒;故其瘦而薄胃者,皆不胜毒也。

	\chapter{天年}

	\begin{yuanwen}
	黄帝问于岐伯曰:愿闻人之始生,何气筑为基?何立而为楯?何失而死?何得而生?
	
	岐伯曰:以母为基,以父为楯\footnote{text};失神者死,得神者生也。
	
	黄帝曰:何者为神?
	
	岐伯曰:血气已和,营卫已通,五脏已成,神气舍心\footnote{text},魂魄毕具,乃成为人。
	\end{yuanwen}
	
	\begin{yuanwen}
	黄帝曰:人之寿夭各不同,或夭或寿,或卒死,或病久,愿闻其道。
	
	岐伯曰:五脏坚固,血脉和调。肌肉解利\footnote{text},皮肤致密。营卫之行,不失其常。呼吸微徐\footnote{text},气以度行。六腑化谷,津液布扬。各如其常,故能长久。
	\end{yuanwen}
	
	\begin{yuanwen}
	黄帝曰:人之寿百岁而死,何以致之?
	
	岐伯曰:使道隧以长\footnote{text},基墙高以方\footnote{text},通调营卫,三部三里起\footnote{text}。骨高肉满,百岁乃得终。
	\end{yuanwen}
	
	\begin{yuanwen}
	黄帝曰:其气之盛衰,以至其死,可得闻乎?
	
	岐伯曰:人生十岁,五脏始定,血气已通,其气在下,故好走\footnote{text}。二十岁,血气始盛,肌肉方长,故好趋\footnote{text}。三十岁,五脏大定,肌肉坚固,血脉盛满,故好步\footnote{text}。四十岁,五脏六腑十二经脉,皆大盛以平定。腠理始疏,荣华颓落,发颇斑白,平盛不摇,故好坐。五十岁,肝气始衰,肝叶始薄,胆汁始减,目始不明。六十岁,心气始衰,若忧悲,血气懈惰,故好卧。七十岁,脾气虚,皮肤枯。八十岁,肺气衰,魄离,故言善误。九十岁,肾气焦,四脏经脉空虚。百岁,五脏皆虚,神气皆去,形骸独居而终矣。
	\end{yuanwen}
	
	\begin{yuanwen}
	黄帝曰:其不能终寿而死者,何如?
	
	岐伯曰:其五脏皆不坚,使道不长。空外以张,喘息暴疾。又卑基墙,薄脉少血,其肉不石。数中风寒,血气虚,脉不通。真邪相攻,乱而相引。故中寿而尽也。
	\end{yuanwen}
	
	
	\chapter{逆顺}
	
	黄帝问于伯高曰:余闻气有逆顺,脉有盛衰,刺有大约,可得闻乎?伯高曰:气之逆顺者,所以应天地阴阳四时五行也;脉之盛衰者,所以候血气之虚实有余不足;刺之大约者,必明知病之可刺,与其未可刺,与其已不可刺也。
	
	黄帝曰:候之奈何?伯高曰:兵法曰无迎逢逢之气,无击堂堂之阵。刺法曰:无刺熇熇之热,无刺漉漉之汗,无刺浑浑之脉,无刺病与脉相逆者。
	
	黄帝曰:候其可刺奈何?伯高曰:上工,刺其未生者也;其次,刺其未盛者也;其次,刺其已衰者也。下工,刺其方袭者也;与其形之盛者也;与其病之与脉相逆者也。故曰:方其盛也,勿敢毁伤,刺其已衰,事必大昌。故曰:上工治未病,不治已病,此之谓也。
	\chapter{五味}
	
	黄帝曰:愿闻谷气有五味,其入五脏,分别奈何?伯高曰:胃者,五脏六腑之海也,水谷皆入于胃,五脏六腑,皆禀气于胃。五味各走其所喜,谷味酸,先走肝,谷味苦,先走心,谷味甘,先走脾,谷味辛,先走肺,谷味咸,先走肾。谷气津液已行,营卫大通,乃化糟粕,以次传下。
	
	黄帝曰:营卫之行奈何?伯高曰:谷始入于胃,其精微者,先出于胃之两焦,以溉五脏,别出两行,营卫之道。其大气之搏而不行者,积于胸中,命曰气海,出于肺,循咽喉,故呼则出,吸则入。天地之精气,其大数常出三入一,故谷不入,半日则气衰,一日则气少矣。
	
	黄帝曰:谷之五味,可得闻乎?伯高曰:请尽言之。五谷:糠米甘,麻酸,大豆咸,麦苦,黄黍辛。五果:枣甘,李酸,栗咸,杏苦,桃辛。五畜:牛甘,犬酸,猪咸,羊苦,鸡辛。五菜:葵甘,韭酸,藿咸,薤苦,葱辛。
	
	五色:黄色宜甘,青色宜酸,黑色宜咸,赤色宜苦,白色宜辛。凡此五者,各有所宜。五宜所言五色者,脾病者,宜食糠米饭,牛肉枣葵;心病者,宜食麦羊肉杏薤;肾病者,宜食大豆黄卷猪肉栗藿;肝病者,宜食麻犬肉李韭;肺病者,宜食黄黍鸡肉桃葱。
	
	五禁:肝病禁辛,心病禁咸,脾病禁酸,肾病禁甘,肺病禁苦。
	
	肝色青,宜食甘,糠米饭、牛肉、枣、葵皆甘。心色赤,宜食酸,犬肉、麻、李、韭皆酸。脾黄色,宜食咸,大豆、猪肉、栗、藿皆咸。肺白色,宜食苦,麦、羊肉、杏、薤皆苦。肾色黑,宜食辛,黄黍、鸡肉、桃、葱皆辛。
	
	\part{}
	\chapter{水胀}
	
	黄帝问于岐伯曰:水与肤胀、鼓胀、肠覃、石瘕、石水,何以别之?岐伯曰:水始起也,目窠上微肿,如新卧起之状,其颈脉动,时咳,阴股间寒,足胫肿,腹乃大,其水已成矣。以手按其腹,随手而起,如裹水之状,此其候也。
	
	黄帝曰:肤胀何以候之?岐伯曰:肤胀者,寒气客于皮肤之间,冬冬然不坚,腹大,身尽肿,皮厚,按其腹,窅而不起,腹色不变,此其候也。
	
	鼓胀何如?岐伯曰:腹胀身皆大,大与肤胀等也,色苍黄,腹筋起,此其候也。
	
	肠覃何如?岐伯曰:寒气客于肠外,与卫气相搏,气不得荣,因有所系,癖而内着,恶气乃起,瘜肉乃生。其始生也,大如鸡卵,稍以益大,至其成,如怀子之状,久者离岁,按之则坚,推之则移,月事以时下,此其候也。
	
	石瘕何如?岐伯曰:石瘕生于胞中,寒气客于子门,子门闭塞,气不得通,恶血当泻不泻,衄以留止,日以益大,状如怀子,月事不以时下,皆生于女子,可导而下。
	
	黄帝曰:肤胀鼓胀,可刺邪?岐伯曰:先泻其胀之血络,后调其经,刺去其血血络也。
	
	\chapter{贼风}

	\begin{yuanwen}
	黄帝曰:夫子言贼风邪气之伤人也,令人病焉。今有其不离屏蔽,不出空穴之中,卒然病者,非不离贼风邪气,其故何也?
	
	岐伯曰:此皆尝有所伤于湿气,藏于血脉之中,分肉之间,久留而不去。若有所堕坠,恶血在内而不去。卒然喜怒不节,饮食不适,寒温不时,腠理闭而不通。其开而遇风寒,则血气凝结,与故邪相袭,则为寒痹。其有热则汗出,汗出则受风。虽不遇贼风邪气,必有因加而发焉。
	\end{yuanwen}
	
	\begin{yuanwen}
	黄帝曰:今夫子之所言者,皆病人之所自知也。其毋所遇邪气,又毋怵惕之所志,卒然而病者,其故何也?唯有因鬼神之事乎?
	
	岐伯曰:此亦有故邪留而未发,因而志有所恶,及有所慕,血气内乱,两气相搏。其所从来者微,视之不见,听而不闻,故似鬼神。
	\end{yuanwen}
	
	\begin{yuanwen}
	黄帝曰:其祝而已者\footnote{text},其故何也?
	
	岐伯曰:先巫者,因知百病之胜,先知其病之所从生者,可祝而已也\footnote{text}。
	\end{yuanwen}
	
	
	\chapter{卫气失常}
	
	黄帝曰:卫气之留于腹中,搐积不行,菀蕴不得常所,使人支胁胃中满,喘呼逆息者,何以去之?伯高曰:其气积于胸中者,上取之,积于腹中者,下取之,上下皆满者,旁取之。
	
	黄帝曰:取之奈何?伯高对曰:积于上,泻人迎、天突、喉中;积于下者,泻三里与气街;上下皆满者,上下取之,与季胁之下一寸;重者,鸡足取之。诊视其脉大而弦急,及绝不至者,及腹皮急甚者,不可刺也。黄帝曰:善。
	
	黄帝问于伯高曰:何以知皮肉气血筋骨之病也?伯高曰:色起两眉薄泽者,病在皮;唇色青黄赤白黑者,病在肌肉;营气濡然者,病在血气;目色青黄赤白黑者,病在筋;耳焦枯受尘垢,病在骨。
	
	黄帝曰:病形何如,取之奈何?伯高曰:夫百病变化,不可胜数,然皮有部,肉有桂,血气有输,骨有属。黄帝曰:愿闻其故。伯高曰:皮之部,输于四末;肉之柱,有臂胫诸阳分肉之间,与足少阴分间;血气之输,输于诸络,气血留居,则盛而起,筋部无阴无阳,无左无右,候病所在;骨之属者,骨空之所以受益而益脑者也。
	
	黄帝曰:取之奈何?伯高曰:夫病变化,浮沉深浅,不可胜究,各在其处,病间者浅之,甚者深之,间者小之,甚者众之,随变而调气,故曰上工。
	
	黄帝问于伯高曰:人之肥瘦大小温寒,有老壮少小,别之奈何?伯高对曰:人年五十已上为老,二十已上为壮,十八已上为少,六岁已上为小。
	
	黄帝曰:何以度知其肥瘦?伯高曰:人有肥、有膏、有肉。黄帝曰:别此奈何?伯高曰:腘肉坚,皮满者,肥。腘肉不坚,皮缓者,膏。皮肉不相离者,肉。
	
	黄帝曰:身之寒温何如?伯高:膏者,其肉淖而粗理者,身寒,细理者,身热。脂者,其肉坚,细理者热,粗理者寒。
	
	黄帝曰:其肥瘦大小奈何?伯高曰:膏者,多气而皮纵缓,故能纵腹垂腴。肉者,身体容大。脂者,其身收小。
	
	黄帝曰:三者之气血多少何如?伯高曰:膏者,多气,多气者,热,热者耐寒。肉者,多血则充形,充形则平。脂者,其血清,气滑少,故不能大。此别于众人者也。
	
	黄帝曰:众人奈何?伯高曰:众人皮肉脂膏,不能相加也,血与气,不能相多,故其形不小不大,各自称其身,命曰众人。
	
	黄帝曰:善。治之奈何?伯高曰:必先别其三形,血之多少,气之清浊,而后调之,治无失常经。是故膏人纵腹垂腴,肉人者,上下容大,脂人者,虽脂不能大者。
	\chapter{玉版}
	
	黄帝曰:余以小针为细物也,夫子乃言上合之于天,下合之于地,中合之于人,余以为过针之意矣,愿闻其故。岐伯曰:何物大于天乎?夫大于针者,惟五兵者焉,死之备也,非生之具。且夫人者,天地之镇也,其不可不参乎?夫治民者,亦唯针焉。夫针之与五兵,其孰小乎?
	
	黄帝曰:病之生时,有喜怒不测,饮食不节,阴气不足,阳气有余,营气不行,乃发为痈疽。阴阳不通,两热相搏,乃化为浓,小针能取之乎?岐伯曰:圣人不能使化者为之,邪不可留也。故两军相当,旗帜相望,白刃陈于中野者,此非一日之谋也。能使其民令行,禁止士卒无白刃之难者,非一日之教也,须臾之得也。夫至使身被痈疽之病,脓血之聚者,不亦离道远乎?夫痈疽之生,脓血之成也,不从天下,不从地出,积微之所生也,故圣人自治于未有形也,愚者遭其已成也。
	
	黄帝曰:其已形,不予遭,脓已成,不予见;为之奈何?岐伯曰:脓已成,十死一生,故圣人弗使已成,而明为良方,着之竹帛,使能者踵而传之后世,无有终时者,为其不予遭也。
	
	黄帝曰:其已有脓血而后遭乎?不导之以小针治乎?岐伯曰:以小治小者,其功小,以大治大者,多害,故其已成脓血者,其唯砭石铍锋之所取也。
	
	黄帝曰:多害者其不可全乎?岐伯曰:其在逆顺焉。黄帝曰:愿闻逆顺。岐伯曰:以为伤者,其白眼青,黑眼小,是一逆也;内药而呕者,是二逆也;腹痛渴甚,是三逆也;肩项中不便,是四逆也;音嘶色脱,是五逆也。除此五者,为顺矣。
	
	黄帝曰:诸病皆有逆顺,可得闻乎?岐伯曰:腹胀、身热、脉大,是一逆也;腹鸣而满,四肢清泄,其脉大,是二逆也;衄而不止,脉大,是三逆也;咳而溲血脱形,其脉小劲,是四逆也;咳脱形,身热,脉小以疾,是谓五逆也。如是者,不过十五日而死矣。
	
	其腹大胀,四末清,脱形,泄甚,是一逆也;腹胀便血,其脉大,时绝,是二逆也;咳溲血,形肉脱,脉搏,是三逆也;呕血,胸满引背,脉小而疾,是四逆也;咳呕,腹胀且飧泄,其脉绝,是五逆也。如是者,不及一时而死矣。工不察此者而刺之,是谓逆治。
	
	黄帝曰:夫子之言针甚骏,以配天地,上数天文,下度地纪,内别五脏,外次六腑,经脉二十八会,尽有周纪。能杀生人,不能起死者,子能反之乎?岐伯曰:能杀生人,不能起死者也。黄帝曰:余闻之,则为不仁,然愿闻其道,弗行于人。岐伯曰:是明道也,其必然也,其如刀剑之可以杀人,如饮酒使人醉也,虽勿诊,犹可知矣。
	
	黄帝曰:愿卒闻之。岐伯曰:人之所受气者,谷也。谷之所注者,胃也。胃者,水谷气血之海也。海之所行云气者,天下也。胃之所出气血者,经隧也。而隧者,五脏六腑之大络也,迎而夺之而已矣。
	
	黄帝曰:上下有数乎?岐伯曰:迎之五里,中道而止,五至而已,五往而脏之气尽矣,故五五二十五,而竭其输矣,此所谓夺其天气者也,非能绝其命而倾其寿者也。黄帝曰:愿卒闻之。岐伯曰:窥门而刺之者,死于家中;入门而刺之者,死于堂上。黄帝曰:善乎方,明哉道,请着之玉版,以为重宝,传之后世,以为刺禁,令民勿敢犯也。
	\chapter{五禁}
	
	黄帝问于岐伯曰:余闻刺有五禁,何谓五禁?岐伯曰:禁其不可刺也。黄帝曰:余闻刺有五夺。岐伯曰:无泻其不可夺者也。黄帝曰:余闻刺有五过。岐伯曰:补泻无过其度。黄帝曰:余闻刺有五逆。岐伯曰:病与脉相逆,命曰五逆。黄帝曰:余闻刺有九宜。岐伯曰:明知九针之论,是谓九谊。
	
	黄帝曰:何谓五禁,愿闻其不可刺之时。岐伯曰:甲乙日自乘,无刺实,无发蒙于耳内。丙丁日自乘,无振埃于肩喉廉泉。戊己日自乘四季,无刺腹,去爪泻水。庚辛日自乘,无刺关节于股膝。壬癸日自乘,无刺足胫,是谓五禁。
	
	黄帝曰:何谓五夺?岐伯曰:形肉已夺,是一夺也;大夺血之后,是二夺也;大汗出之后,是三夺也;大泄之后,是四夺也;新产及大血之后,是五夺也。此皆不可泻。
	
	黄帝曰:何谓五逆?岐伯曰:热病脉静,汗已出,脉盛躁,是一逆也;病泄,脉洪大,是二逆也;着痹不移,(月囷)肉破,身热,脉偏绝,是三逆也;淫而夺形、身热,色夭然白,乃后下血衄,血衄笃重,是谓四逆也;寒热夺形,脉坚搏,是谓五逆也。
	\chapter{动输}
	
	黄帝曰:经脉十二,而手太阴、足少阴、阳明,独动不休,何也?岐伯曰:是明胃脉也。胃为五脏六腑之海,其清气上注于肺,肺气从太阴而行之,其行也,以息往来,故人一呼,脉再动,一吸脉亦再动,呼吸不已,故动而不止。
	
	黄帝曰:气之过于寸口也,上十焉息,下八焉伏,何道从还?不知其极。
	
	岐伯曰:气之离脏也,卒然如弓弩之发,如水之下岸,上于鱼以及衰,其余气衰散以逆上,故其行微。
	
	黄帝曰:足之阳明,何因而动?岐伯曰:胃气上注于肺,其悍气上冲头者,循咽,上走空窍,循眼系,入络脑,出顑,下客主人,循牙车,合阳明,并下人迎,此胃气别走于阳明者也。故阴阳上下,其动也若一。故阳病而阳脉小者,为逆;阴病而阴脉大者,为逆。故阴阳俱静俱动,若引绳相倾者病。
	
	黄帝曰:足少阴何因而动?岐伯曰:冲脉者,十二经之海也,与少阴之大络,起于肾下,出于气街,循阴股内廉,邪入腘中,循胫骨内廉,并少阴之经,下入内踝之后。入足下,其别者,邪入踝,出属附上,入大指之间,注诸络,以温足胫,此脉之常动者也。
	
	黄帝曰:营卫之行也,上下相贯,如环之无端,今有其卒然遇邪风,及逢大寒,手足懈惰,其脉阴阳之道,相输之会,行相失也,气何由还?岐伯曰:夫四末阴阳之会者,此气之尤络也;四街者,气之径路也。故络绝则径通,四末解则气从合,相输如环。黄帝曰:善。此所谓如环无端,莫知其纪,终而复始,此之谓也。
	
	\chapter{五味论}

	\begin{yuanwen}
	黄帝问于少俞曰:五味入于口也,各有所走,各有所病。酸走筋,多食之,令人癃。咸走血,多食之,令人渴。辛走气,多食之,令人洞心。苦走骨,多食之,令人变呕。甘走肉,多食之,令人悗心。余知其然也,不知其何由,愿闻其故。
	
	少俞答曰:酸入于胃,其气涩以收,上之两焦\footnote{text},弗能出入也。不出即留于胃中,胃中和温,则下注膀胱。膀胱之胞薄以懦\footnote{text},得酸则缩绻,约而不通,水道不行,故癃。阴者,积筋之所终也\footnote{text},故酸入而走筋矣。
	\end{yuanwen}
	
	\begin{yuanwen}
	黄帝曰:咸走血,多食之,令人渴,何也?
	
	少俞曰:咸入于胃;其气上走中焦,注于脉,则血气走之。血与咸相得则凝,凝则胃中汁注之。注之则胃中竭,竭则咽路焦\footnote{text},故舌本干而善渴。血脉者,中焦之道也,故咸入而走血矣。
	\end{yuanwen}
	
	\begin{yuanwen}
	黄帝曰:辛走气,多食之,令人洞心,何也?
	
	少俞曰:辛入于胃,其气走于上焦,上焦者,受气而营诸阳者也。姜韭之气熏之,营卫之气不时受之,久留心下,故洞心。辛与气俱行,故辛入而与汗俱出。
	\end{yuanwen}
	
	\begin{yuanwen}
	黄帝曰:苦走骨,多食之,令人变呕,何也?
	
	少俞曰:苦入于胃,五谷之气,皆不能胜苦。苦入下脘,三焦之道皆闭而不通,故变呕。齿者,骨之所终也,故苦入而走骨,故入而复出,知其走骨也。
	\end{yuanwen}
	
	\begin{yuanwen}
	黄帝曰:甘走肉,多食之。令人悗心,何也?
	
	少俞曰:甘入于胃,其气弱小,不能上至于上焦,而与谷留于胃中者,令人柔润者也。胃柔则缓,缓则虫动,虫动则令人悗心。其气外通于肉,故甘走肉。
	\end{yuanwen}
	
	\chapter{阴阳二十五人}
	
	黄帝曰:余闻阴阳之人何如?伯高曰:天地之间,六合之内,不离于五,人亦应之。故五五二十五人之政,而阴阳之人不与焉。其态又不合于众者五,余已知之矣。愿闻二十五人之形,血气之所生,别而以候,从外知内,何如?岐伯曰:「悉乎哉问也,此先师之秘也,虽伯高犹不能明之也。黄帝避席遵循而却曰:余闻之得其人弗教,是谓重失,得而泄之,天将厌之,余愿得而明之,金柜藏之,不敢扬之。岐伯曰:先立五形金木水火土,别其五色,异其五形之人,而二十五人具矣。黄帝曰:愿卒闻之。岐伯曰:慎之慎之,臣请言之。
	
	木形之人,比于上角似于苍帝,其为人苍色,小头,长面大肩背直身小,手足好。有才,劳心少力多忧,劳于事,能春夏不能秋冬感而病生。足厥阴,佗佗然,大角之人比于左足少阳,少阳之上遗遗然。左角之人比于右足少阳,少阳之下随随然。钛角之人,比于右足少阳,少阳之上推推然。判角之人比于左足少阳,少阳之下枯枯然。
	
	火形之人,比于上征,似于赤帝。其为人赤色广胤,脱面,小头,好肩背,髀腹小手足,行安地疾心,行摇肩背肉满。有气轻财少信多虑,见事明好颜,急心不寿暴死。能春夏不能秋冬,秋冬感而病生,手少阴核核然。质征之人,比于左手太阳,太阳之上,肌肌然,少征之人比于右手太阳,太阳之下??然,右征之人比于右手太阳,太阳之上鲛鲛然。质判之人,比于左手太阳,太阳之下支支颐颐然。
	
	形于之人,比于上宫,似于上古黄帝,其为人黄色圆面、大头、美肩背、大腹、美股胫、小手足、多肉、上下相称行安地,举足浮。安心,好利人不喜权势,善附人也。能秋冬不能春夏,春夏感而病生,足太阴,敦敦然。大宫之人比于左足阳明,阳明之上婉婉然。加宫之人,比于左足阳明,阳明之下坎坎然。少宫之人,比于右足阳明,阳明之上,枢枢然。左宫之人,比于右足阳明,阳明之下,兀兀然。
	
	金形之人比于上商,似于白帝,其为人方面白色、小头、小肩背小腹、小手足如骨发踵外,骨轻。身清廉,急心静悍,善为吏,能秋冬,不能春夏,春夏感而病生。手太阴敦敦然,釱商之人比于左手阳明,阳明之上,廉廉然。右商之人,比于左手阳明,阳明之下脱脱然。左商之人比于右手阳明,阳明之上监监然。少商之人,比于右手阳明,阳明之下,严严然。
	
	水形之人,比于上羽,似于黑帝,其为人,黑色面不平,大头廉颐,小肩大腹动手足,发行摇身下尻长,背延延然。不敬畏善欺绐人,戮死。能秋冬不能春夏,春夏感而病生。足少阴汗汗然。大羽之人,比于右足太阳,太阳之上,颊颊然。少羽之人,比于左足太阳,太阳之下洁洁然。桎之为人,比于左足太阳,太阳之上安安然。是故五形之人二十五变者,众之所以相欺者是也。
	
	黄帝曰:得其形,不得其色何如?岐伯曰:形胜色,色胜形者,至其胜时年加,感则病行,失则忧矣。形色相得者,富贵大乐。黄帝曰:其形色相当胜之时,年加可知乎?岐伯曰:凡年忌下上之人,大忌常加七岁,十六岁、二十五岁、三十四岁、四十三岁、五十二岁、六十一岁皆人之大忌,不可不自安也,感则病行,失则忧矣,当此之时,无为奸事,是谓年忌。
	
	黄帝曰:夫子之言脉之上下,血气之候似知形气,奈何?岐伯曰:足阳明之上血气盛则髯美长,血少气多则髯短,故气少血多则髯少,血气皆少则无髯。两吻多画,足阳明之下血气盛则下毛美长至胸,血多气少则下毛美短至脐,行则善高举足,足趾少肉足善寒,血少气多则肉而善瘃,血气皆少则无毛有则稀、枯悴,善痿厥,足痹。
	
	足少阳之上,气血盛则通髯美长,血多气少则通髯美短,血少气多则少髯,血气皆少则无须,感于寒湿则善痹。骨痛爪枯也。足少阳之下,血气盛则胫毛美长,外踝肥;血多气少则胫毛美短,外踝皮坚而厚,血少气多则胻毛少,外踝皮薄而软,血气皆少则无毛,外踝瘦无肉。
	
	足太阳之上,血气盛则美眉,眉有毫毛血多气少则恶眉,面多少理,血少气多则面多肉,血气和则美色,足太阳之下,血气盛则肉满,踵坚,气少血多则瘦,跟空,血气皆少则善转筋,踵下痛。
	
	手阳明之上,血气盛则髭美。血少气多则髭恶,血气皆少则无髭。手阳明之下血气盛则腋下毛美,手鱼肉以温,气血皆少则手瘦以寒。
	
	手少阴之上,血气盛则眉美以长,耳色美,血气皆少则耳焦恶色。手少阳之下,血气盛则手卷多肉以温,血气皆少则寒以瘦,气少血多则瘦以多脉。
	
	手太阳之上,血气盛则多须,面多肉以平,血气皆少则面瘦恶色。手太阳之下,血气盛则掌肉充满,血气皆少则掌瘦以寒。
	
	黄帝曰:二十五人者,刺之有约乎?岐伯曰:美眉者,足太阳之脉,气血多,恶眉者,血气少,其肥而泽者,血气有余,肥而不泽者,气有余,血不足,瘦而无泽者,气血俱不足,审察其形气有余不足而调之,可以知逆顺矣。
	
	黄帝曰:刺其诸阴阳奈何?岐伯曰:按其寸口人迎,以调阴阳,切循其经络之凝涩,结而不通者,此于身皆为痛痹,甚则不行,故凝涩,凝涩者,致气以温之血和乃止。其结络者,脉结血不和,决之乃行,故曰:气有余于上者,导而下之,气不足于上者,推而休之,其稽留不至者,因而迎之,必明于经隧,乃能持之,寒与热争者,导而行之,其宛陈血不结者,则而予之,必先明知二十五人则血气之所在,左右上下,刺约毕也。
	
	\part{}
	\chapter{五音五味}
	
	右征与少征,调右手太阳二,左商与左征,调左手阳明上。少征与大宫,调左手阳明上,右角与大角,调右手少阳下。大征与少征,调左手太阳上,众羽与少羽,调右足太阳下,少商与右商调右手太阳下,桎羽与众羽,调右足太阳下,少宫与大宫,调右足阳明下,判角与少角,调右足少阳下,釱商与上商,调右足阳明下,釱商与上角,调左足太阳下。
	
	上征与右征同谷麦、畜羊、果杏,手少阴藏心,色赤味苦,时夏。上羽与大羽,同谷大豆,畜彘,果栗,足少阴藏肾,色黑味咸,时冬。上宫与大宫同谷稷,畜牛,果枣,足太阴藏脾,色黄味甘,时季夏。上商与右商同谷黍,畜鸡,果桃,手太阴藏肺,色白味辛,时秋。上角与大角,同谷麻、畜犬、果李,足厥阴藏肝,色青味酸,时春。
	
	大宫与上角,同右足阳明上,左角与大角,同左足阳明上,少羽与大羽同右足太阳下,左商与右商,同左手阳明上,加宫与大宫同左足少阳上,质判与大宫,同左手太阳下,判角与大角同左足少阳下,大羽与大角,同右足太阳上,大角与大宫同右足少阳上,右征、少征、质征、上征、判征、右角、釱角、上角、大角、判角。右商、少商、釱商、上商、左商。少宫、上宫、大宫、加宫、左角宫。众羽、桎羽、上羽、大羽、少羽。
	
	黄帝曰:妇人无须者,无血气乎?岐伯曰:冲脉任脉皆起于胞中,上循背里,为经络之海,其浮而外者,循腹右上行,会于咽喉,别而络唇口,血气盛则充肤热肉,血独盛者澹渗皮肤,生毫毛。今妇人之生有余于气,不足于血以其数脱血也,冲任之脉,不荣口唇,故须不生焉。
	
	黄帝曰:士人有伤于阴,阴气绝而不起,阴不用,然其须不去,其故何也?宦者独去何也?愿闻其故。岐伯曰:宦者去其宗筋,伤其冲脉,血泻不复,皮肤内结,唇口内荣故须不生。
	
	黄帝曰:其有天宦者,未尝被伤,不脱于血,然其须不生其故何也?岐伯曰:此天之所不足也,其任冲不盛、宗筋不成,有气无血,唇口不荣,故须不生。
	
	黄帝曰:善乎哉!圣人之通万物也,若日月之光影,音声鼓响,闻其声而知其形,其非夫子,孰能明万物之精。是故圣人,视其颜色黄赤者,多热气,青白者少热气,黑色者多血少气,美眉者,太阳多血;通髯极须者,少阳多血,美须者阳明多血,此其时然也。
	
	夫人之常数,太阳常多血少气,少阳常多气少血,阳明常多血多气,厥阴常多气少血,少阴常多血少气,太阴常多血少气,此天之常数也。
	
	\chapter{百病始生}
	
	黄帝问于岐伯曰:夫百病之始生也,皆于风雨寒暑,清湿喜怒,喜怒不节则伤脏,风雨则伤上,清湿则伤下。三部之气所伤异类,愿闻其会,岐伯曰:三部之气各不同或起于阴或起于阳请言其方,喜怒不节则伤脏,脏伤则病起于阴也,清湿袭虚,则病起于下,风雨袭虚,则病起于上,是谓三部,至于其淫泆,不可胜数。
	
	黄帝曰:余固不能数,故问先师愿卒闻其道,岐伯曰:风雨寒热不得虚,邪不能独伤人。卒然逢疾风暴雨而不病者,盖无虚,故邪不能独伤人。此必因虚邪之风,与其身形,两虚相得,乃客其形。两实相逢,众人肉坚,其中于虚邪也因于天时,与其身形,参以虚实,大病乃成,气有定舍,因处为名,上下中外,分为三员。
	
	是故虚邪之中人也,始于皮肤,皮肤缓则腠理开,开则邪从毛发入,入则抵深,深则毛发立,毛发立则淅然,故皮肤痛。留而不去,则传舍于络脉,在络之时,痛于肌肉,故痛之时息,大经代去,留而不去,传舍于经,在经之时,洒淅喜惊。留而不去,传舍于俞,在俞之时,六经不通四肢,则肢节痛,腰脊乃强,留而不去,传舍于伏冲之脉,在伏冲之时体重身痛,留而不去,传舍于肠胃,在肠肾之时,贲响腹胀,多寒则肠鸣飧泄,食不化,多热则溏出糜。留而不去,传舍于肠胃之外,募原之间,留着于脉,稽留而不去,息而成积,或着孙脉,或着络脉,或着经脉,或着俞脉,或着于伏冲之脉,或着于膂筋,或着于肠胃之募原,上连于缓筋,邪气淫泆,不可胜论。
	
	黄帝曰:愿尽闻其所由然。岐伯曰:其着孙络之脉而成积者,其积往来上下,臂小孙络之居也,浮而缓,不能句积而止之,故往来移行肠胃之间,水凑渗注灌,濯濯有音,有寒则(月真)(月真)满雷引,故时切痛,其着于阳明之经则挟脐而居,饱食则益大,饥则益小。其着于缓筋也,似阳明之积,饱食则痛,饥则安。其着于肠胃之募原也,痛而外连于缓筋,饱食则安,饥则痛。其着于伏冲之脉者,揣之应手而动,发手则热气下于两股,如汤沃之状。其着于膂筋,在肠后者饥则积见,饱则积不见,按之不得。其着于输之脉者,闭塞不通,津液不下,孔窍干壅,此邪气之从外入内,从上下也。
	
	黄帝曰:积之始生,至其已成,奈何?岐伯曰:积之始生,得寒乃生,厥乃成积也,黄帝曰:其成积奈何?岐伯曰:厥气生足悗,悗生胫寒,胫寒则血脉凝涩,血脉凝涩则寒气上入于肠胃,入于肠胃则(月真)胀,(月真)胀则肠外之汁沫迫聚不得散,日以成积。卒然多食饮,则肠满,起居不节,用力过度,则络脉伤,阳络伤则血外溢,血外溢则衄血,阴络伤则血内溢,血内溢则后血。肠胃之络伤则血溢于肠外,肠外有寒,汁沫与血相搏,则并合凝聚不得散,而积成矣。卒然中外于寒,若内伤于忧怒,则气上逆,气上逆则六俞不通,温气不行,凝血蕴里而不散,津液涩渗,着而不去,而积皆成矣。
	
	黄帝曰:其生于阴者,奈何?岐伯曰:忧思伤心,重寒伤肺,忿怒伤肝,醉以入房,汗出当风伤脾,用力过度,若入房汗出洛,则伤肾,此内外三部之所生病者也。
	
	黄帝曰:善治之奈何?岐伯答曰:察其所痛,以知其应,有余不足,当补则补,当泻则泻,毋逆天时,是谓至治。
	
	\chapter{行针}
	
	黄帝问于岐伯曰:余闻九针于夫子,而行之于百姓,百姓之血气,各不同形,或神动而气先针行;或气与针相逢;或针已出,气独行;或数刺乃知;或发针而气逆;或数刺病益剧。凡此六者,各不同形,愿闻其方。
	
	岐伯曰:重阳之人,其神易动,其气易往也。黄帝曰:何谓重阳之人?岐伯曰:重阳之人,熇熇高高,言语善疾,举足善高,心肺之脏气有余,阳气滑盛而扬,故神动而气先行。
	
	黄帝曰:重阳之人而神不先行者,何也?岐伯曰:此人颇有阴者也。黄帝曰:何以知其颇有阴者也。岐伯曰:多阳者,多喜;多阴者,多怒,数怒者,易解,故曰颇有阴。其阴阳之离合难,故其神不能先行也。
	
	黄帝曰:其气与针相逢,奈何?岐伯曰:阴阳和调,而血气淖泽滑利,故针入而气出,疾而相逢也。
	
	黄帝曰:针已出而气独行者,何气使然?岐伯曰:其阴气多而阳气少,阴气沉而阳气浮者内藏,故针已出,气乃随其后,故独行也。
	
	黄帝曰:数刺乃知,何气使然?岐伯曰:此人之多阴而少阳,其气沉而气往难,故数刺乃知也。
	
	黄帝曰:针入而气逆者,何气使然?岐伯曰:其气逆与其数刺病益甚者,非阴阳之气,浮沉之势也。此皆麤之所败,工之所失,其形气无过焉。
	
	\chapter{上膈}
		
	黄帝曰:气为上膈者,食饮入而还出,余已知之矣。虫为下膈。下膈者,食焠时乃出,余未得其意,愿卒闻之。岐伯曰:喜怒不适,食饮不节,寒温不时,则寒汁流于肠中。流于肠中则虫寒,虫寒则积聚,守于下管,则肠胃充郭,卫气不营,邪气居之。人食则虫上食,虫上食则下管虚,下管虚则邪气胜之,积聚以留,留则痈成,痈成则下管约。其痈在管内者,即而痛深,其痈在外者,则痈外而痛浮,痈上皮热。
	
	黄帝曰:刺之奈何?岐伯曰:微按其痈,视气所行,先浅刺其傍,稍内益深,逐而刺之,毋过三行,察其沉浮,以为深浅。已刺必熨,令热入中,日使热内,邪气益衰,大痈乃溃。伍以参禁,以除其内,恬憺无为,乃能行气,后以咸苦,化谷乃下矣。
	
	\chapter{忧恚无言}
	
	黄帝问于少师曰:人之卒然忧恚,而言无音者,何道之塞?何气出行?使音不彰?愿闻其方。少师答曰:咽喉者,水谷之道也。喉咙者,气之所以上下者也。会厌者,声音之户也。口唇者,声音之扇也。舌者,声音之机也。悬壅垂者,声音之关者。颃颡者,分气之所泄也。横骨者,神气所使主发舌者也。故人之鼻洞涕出不收者,颃颡不开,分气失也。是故厌小而疾薄,则发气疾,其开阖利,其出气易,其厌大而厚,则开阖难,其气出迟,故重言也。人卒然无音者,寒气客于厌,则厌不能发,发不能下,至其开阖不致,故无音。
	
	黄帝曰:刺之奈何?岐伯曰:足之少阴,上系于舌,络于横骨,终于会厌。两泻其血脉,浊气乃避。会厌之脉,上络任脉,取之天突,其厌乃发也。
	
	\chapter{寒热}
	
	黄帝问于岐伯曰:寒热瘰?在于颈腋者,皆何气使生?岐伯曰:此皆鼠?寒热之毒气也,留于脉而不去者也。
	
	黄帝曰:去之奈何?岐伯曰:鼠?之本,皆在于脏,其末上出于颈腋之间,其浮于脉中,而未内着于肌肉,而外为脓血者,易去也。
	
	黄帝曰:去之奈何?岐伯曰:请从其本引其末,可使衰去,而绝其寒热。审按其道以予之,徐往徐来以去之,其小如麦者,一刺知,三刺而已。
	
	黄帝曰:决其生死奈何?岐伯曰:反其目视之,其中有赤脉,上下贯瞳子,见一脉,一岁死;见一脉半,一岁半死;见二脉,二岁死;见二脉半,二岁半死;见三脉,三岁而死。见赤脉不下贯瞳子,可治也。
	
	\chapter{邪客}
	
	黄帝问于伯高曰:夫邪气之客人也,或令人目不瞑不卧出者,何气使然?伯高曰:五谷入于胃也,其糟粕津液宗气,分为三隧。故宗气积于胸中,出于喉咙,以贯心脉,而行呼吸焉。营气者,泌其津液,注之于脉,化以为血,以荣四末,内注五脏六腑,以应刻数焉。卫气者,出其悍气之慓疾,而先行于四末分肉皮肤之间,而不休者也。昼日行于阳,夜行于阴,常从足少阴之分间,行五脏六腑,今厥气客于五脏六腑,则卫气独卫其外,行于阳,不得入于阴。行于阳则阳气盛,阳气盛则阳桥陷,不得入于阴,阴虚,故目不瞑。
	
	黄帝曰:善。治之奈何?伯高曰:补其不足,泻其有余,调其虚实,以通其道,而去其邪。饮以半夏汤一剂,阴阳已通,其卧立至。黄帝曰:善。此所谓决渎壅塞,经络大通,阴阳和得者也。愿闻其方。伯高曰:其汤方以流水千里以外者八升,扬之万遍,取其清五升,煮之,炊以苇薪火,沸置秫米一升,治半夏五合,徐炊,令竭为一升半,去其滓,饮汁一小杯,日三稍益,以知为度,故其病新发者,复杯则卧,汗出则已矣。久者,三饮而已也。
	
	黄帝问于伯高曰:愿闻人之肢节以应天地奈何?伯高答曰:天圆地方,人头圆足方以应之。天有日月,人有两目;地有九州,人有九窍;天有风雨,人有喜怒;天有雷电,人有声音;天有四时,人有四肢;天有五音,人有五脏;天有六律,人有六腑;天有冬夏,人有寒热;天有十曰,人有手十指;辰有十二,人有足十指,茎垂以应之,女子不足二节,以抱人形;天有阴阳,人有夫妻;岁有三百六十五日,人有三百六十五节;地有高山,人有肩膝;地有深谷,人有腋腘;地有十二经水,人有十二经脉;地有泉脉,人有卫气;地有草蓂,人有毫毛;天有昼夜,人有卧起;天有列星,人有牙齿;地有小山,人有小节;地有山石,人有高骨;地有林木,人有募筋;地有聚邑,人有腘肉;岁有十二月,人有十二节;地有四时不生草,人有无子。此人与天地相应者也。
	
	黄帝问于岐伯曰:余愿闻持针之数,内针之理,纵舍之意,扞皮开腠理,奈何?脉之屈折,出入之处,焉至而出,焉至而止,焉至而徐,焉至而疾,焉至而入,六腑之输于身者,余愿尽闻其方。岐伯曰:帝之所问,针道毕矣。
	
	黄帝曰:愿卒闻之。岐伯曰:手太阴之脉,出于大指之端,内屈,循白肉际,至本节之后太渊,留以澹,外屈,上于本节下,内屈,与阴诸络会于鱼际,数脉并注,其气滑利,伏行壅骨之下,外屈,出于寸口而行,上至于肘内廉,入于大筋之下,内屈,上行臑阴,入腋下,内屈,走肺。此顺行逆数之屈折也。心主之脉,出于中指之端,内屈,循中指内廉以上,留于掌中,伏行两骨之间,外屈,出两筋之间,上至肘内廉,入于小筋之下,留两骨之会,上入于胸中,内络于心脉。
	
	黄帝曰:手太阴之脉,独无俞,何也?岐伯曰:少阴,心脉也。心者,五脏六腑之大主也,精神之所舍也,其脏坚固,邪弗能容也。容之则心伤,心伤则神去,神去则死矣。故诸邪之在于心者,皆在于心之包络。包络者,心主之脉也,故独无俞焉。
	
	黄帝曰:少阴独无俞者,不病乎?岐伯曰:其外经病而藏不病,故独取其经于掌后锐骨之端。其余脉出入屈折,其行之徐疾,皆如手少阴心主之脉行也。故本俞者,皆因其气之虚实疾徐以取之,是谓因冲而泻,因衰而补,如是者,邪气得去,真气坚固,是谓因天之序。
	
	黄帝曰:持针纵舍奈何?岐伯曰:必先明知十二经脉之本末,皮肤之寒热,脉之盛衰滑涩。其脉滑而盛者,病日进;虚而细者,久以持;大以涩者,为痛痹。阴阳如一者,病难治。其本末尚热者,病尚在;其热以衰者,其病亦去矣。持其尺,察其肉之坚脆,大小滑涩,寒温燥湿。因视目之五色,以知五脏,而决死生。视其血脉,察其色,以知其寒热痛痹。
	
	黄帝曰:持针纵舍,余未得其意也。岐伯曰:持针之道,欲端以正,安以静。先知虚实而行疾徐。左手执骨,右手循之。无与肉果。泻欲端以正,补必闭肤。辅针导气,邪得淫泆,真气得居。
	
	黄帝曰:扞皮开腠理奈何?岐伯曰:因其分肉,左别其肤,微内而徐端之,适神不散,邪气得去。
	
	黄帝问于岐伯曰:人有八虚,各何以候?
	
	岐伯答曰:以候五脏。
	
	黄帝曰:候之奈何?
	
	岐伯曰:肺心有邪,其气留于两肘;肝有邪,其气流于两腋;脾有邪,其气留于两髀;肾有邪,其气留于两腘。凡此八虚者,皆机关之室,真气之所过,血络之所游。邪气恶血,固不得住留。住留则伤筋络骨节;机关不得屈伸,故痀挛也。
	
	\chapter{通天}

	\begin{yuanwen}
	黄帝问于少师曰:余尝闻人有阴阳,何谓阴人?何谓阳人?
	
	少师曰:天地之间,六合之内,不离于五,人亦应之,非徒一阴一阳而已也。而略言耳,口弗能遍明也。
	
	黄帝曰:愿略闻其意,有贤人圣人,心能备而行之乎?
	
	少师曰:盖有太阴之人,少阴之人,太阳之人,少阳之人,阴阳和平之人。凡五人者\footnote{text},其态不同,其筋骨气血各不等。
	\end{yuanwen}
	
	\begin{yuanwen}
	黄帝曰:其不等者,可得闻乎?
	
	少师曰:太阴之人,贪而不仁,下齐湛湛\footnote{text},好内而恶出\footnote{text},心和而不发\footnote{text},不务于时,动而后之\footnote{text},此太阴之人也。
	\end{yuanwen}
	
	\begin{yuanwen}
	少阴之人,小贪而贼心,见人有亡\footnote{text},常若有得,好伤好害,见人有荣,乃反愠怒,心疾而无恩\footnote{text},此少阴之人也。
	\end{yuanwen}
	
	\begin{yuanwen}
	太阳之人,居处于于\footnote{text},好言大事,无能而虚说,志发于四野\footnote{text},举措不顾是非,为事如常自用\footnote{text},事虽败而常无悔。此太阳之人也。
	\end{yuanwen}
	
	\begin{yuanwen}
	少阳之人,諟谛好自责\footnote{text},有小小官,则高自宜,好为外交而不内附。此少阳之人也。
	\end{yuanwen}
	
	\begin{yuanwen}
	阴阳和平之人,居处安静,无为惧惧,无为欣欣,婉然从物\footnote{text},或与不争,与时变化,尊则谦谦,谭而不治\footnote{text},是谓至治\footnote{text}。古之善用针艾者,视人五态乃治之。盛者泻之,虚者补之。
	\end{yuanwen}
	
	\begin{yuanwen}
	黄帝曰:治人之五态奈何?
	
	少师曰:太阴之人,多阴而无阳。其阴血浊,其卫气涩。阴阳不和,缓筋而厚皮。不之疾泻,不能移之。
	\end{yuanwen}
	
	\begin{yuanwen}
	少阴之人,多阴少阳,小胃而大肠\footnote{text},六腑不调。其阳明脉小而太阳脉大,必审调之。其血易脱,其气易败也。
	\end{yuanwen}
	
	\begin{yuanwen}
	太阳之人,多阳而少阴。必谨调之,无脱其阴,而泻其阳。阳重脱者易狂\footnote{text},阴阳皆脱者,暴死,不知人也\footnote{text}。
	\end{yuanwen}
	
	\begin{yuanwen}
	少阳之人,多阳少阴,经小而络大\footnote{text}。血在中而气在外,实阴而虚阳,独泻其络脉,则强气脱而疾,中气不足,病不起也。
	\end{yuanwen}
	
	\begin{yuanwen}
	阴阳和平之人,其阴阳之气和,血脉调。谨诊其阴阳,视其邪正,安容仪。审有余不足。盛则泻之,虚则补之,不盛不虚,以经取之。此所以调阴阳,别五态之人者也。
	\end{yuanwen}
	
	\begin{yuanwen}
	黄帝曰:夫五态之人者,相与毋故,卒然新会,未知其行也,何以别之?
	
	少师答曰:众人之属\footnote{text},不知五态之人者,故五五二十五人,而五态之人不与焉。五态之人,尤不合于众者也。
	\end{yuanwen}
	
	\begin{yuanwen}
	黄帝曰:别五态之人,奈何?
	
	少师曰:太阴之人,其状黮黮然黑色\footnote{text},念然下意\footnote{text},临临然长大\footnote{text},腘然未偻\footnote{text},此太阴之人也。
	
	少阴之人,其状清然窃然\footnote{text},固以阴贼,立而躁崄,行而似伏。此少阴之人也。
	
	太阳之人,其状轩轩储储\footnote{text},反身折腘\footnote{text}。此太阳之人也。
	
	少阳之人,其状立则好仰,行则好摇,其两臂两肘则常出于背。此少阳之人也。
	
	阴阳和平之人,其状委委然\footnote{text},随随然\footnote{text},颙颙然\footnote{text},愉愉然\footnote{text},䁢䁢然\footnote{text},豆豆然\footnote{text},众人皆曰君子。此阴阳和平之人也。
	\end{yuanwen}
	
	
	\part{}
	\chapter{官能}
	
	\begin{yuanwen}
	黄帝问于岐伯曰:余闻九针于夫子,众多矣不可胜数,余推而论之,以为一纪。余司诵之,子听其理,非则语余,请正其道,令可久传后世无患,得其人乃传,非其人勿言。
	
	岐伯稽首再拜曰:请听圣王之道。
	
	黄帝曰:用针之理,必知形气之所在,左右上下,阴阳表里,血气多少,行之逆顺,出入之合,谋伐有过。知解结,知补虚泻实,上下气门,明通于四海。审其所在,寒热淋露以输异处,审于调气,明于经隧,左右肢络,尽知其会。寒与热争,能合而调之,虚与实邻,知决而通之,左右不调,把而行之,明于逆顺,乃知可治,阴阳不奇,故知起时。审于本末,察其寒热,得邪所在,万刺不殆。知官九针,刺道毕矣。
	
	明于五俞徐疾所在,屈伸出入,皆有条理。言阴与阳,合于五行,五脏六腑,亦有所藏,四时八风,尽有阴阳。各得其位,合于明堂,各处色部,五脏六腑。察其所痛,左右上下,知其寒温,何经所在。审皮肤之寒温滑涩,知其所苦,膈有上下,知其气所在。先得其道,稀而疏之,稍深以留,故能徐入之。大热在上,推而下之;从上下者,引而去之;视前痛者,常先取之。大寒在外,留而补之;入于中者,从合泻之。针所不为,灸之所宜。上气不足,推而扬之;下气不足,积而从之;阴阳皆虚,火自当之。厥而寒甚,骨廉陷下,寒过于膝,下陵三里。阴络所过,得之留止,寒入于中,推而行之;经陷下者,火则当之;结络坚紧,火所治之。不知所苦,两蹻之下,男阴女阳,良工所禁,针论毕矣。
	
	用针之服,必有法则,上视天光,下司八正,以辟奇邪,而观百姓,审于虚实,无犯其邪。是得天之灵,遇岁之虚,救而不胜,反受其殃,故曰必知天忌,乃言针意。
	
	法于往古,验于来今,观于窈冥,通于无穷。麤之所不见,良工之所贵。莫知其形,若神髣佛。
	
	邪气之中人也,洒淅动形;正邪之中人也,微先见于色,不知于其身,若有若无,若亡若存,有形无形,莫知其情。是故上工之取气,乃救其萌芽;下工守其已成,因败其形。
	
	是故工之用针也,知气之所在,而守其门户,明于调气,补泻所在,徐疾之意,所取之处。泻必用员,切而转之,其气乃行,疾而徐出,邪气乃出,伸而迎之,遥大其穴,气出乃疾。补必用方,外引其皮,令当其门,左引其枢,右推其肤,微旋而徐推之,必端以正,安以静,坚心无解,欲微以留,气下而疾出之,推其皮,盖其外门,真气乃存。用针之要,无忘其神。
	
	雷公问于黄帝曰:针论曰:得其人乃传,非其人勿言,何以知其可传?
	
	黄帝曰:各得其人,任之其能,故能明其事。
	
	雷公曰:愿闻官能奈何?
	
	黄帝曰:明目者,可使视色;聪耳者,可使听音;捷疾辞语者,可使传论;语徐而安静,手巧而心审谛者,可使行针艾,理血气而调诸逆顺,察阴阳而兼诸方。缓节柔筋而心和调者,可使导引行气;疾毒言语轻人者,可使唾痈?病;爪苦手毒,为事善伤者,可使按积抑痹。各得其能,方乃可行,其名乃彰。不得其人,其功不成,其师无名。故曰:得其人乃言,非其人勿传,此之谓也。手毒者,可使试按龟,置龟于器下,而按其上,五十日而死矣,手甘者,复生如故也。
	\end{yuanwen}

	\chapter{论疾诊尺}
	
	\begin{yuanwen}
	黄帝问岐伯曰:余欲无视色持脉,独调其尺,以言其病,从外知内,为之奈何?
	
	岐伯曰:审其尺之缓急小大滑涩,肉之坚脆,而病形定矣。
	
	视人之目窠上微痈,如新卧起状,其颈脉动,时咳,按其手足上,窅而不起者,风水肤胀也。
	
	尺肤滑,其淖泽者,风也。尺肉弱者,解并,安卧脱肉者,寒热,不治。尺肤滑而泽脂者,风也。尺肤涩者,风痹也。尺肤麤如枯鱼之鳞者,水泆饮也。尺肤热甚,脉盛躁者,病温也,其脉甚而滑者,病且出也。尺肤寒,其脉小者,泄、少气。尺肤炬然,先热后寒者,寒热也;尺肤先寒,久大之而热者,亦寒热也。
	
	肘所独热者,腰以上热;手所独热者,腰以下热。肘前独热者,膺前热;肘后独热者,肩背热。臂中独热者,腰腹热;肘后麤以下三四寸热者,肠中有虫。掌中热者,腹中热;掌中寒者,腹中寒。鱼上白肉有青血脉者,胃中有寒。
	
	尺炬然热,人迎大者,当本血;尺坚大,脉小甚,少气,免有加,立死。
	
	目赤色者病在心,白在肺,青在肝,黄在脾,黑在肾。黄色不可名者,病在胸中。
	
	诊目痛,赤脉从上下者,太阳病;从下上者,阳明病;从外走内者,少阳病。
	
	诊寒热,赤脉上下至瞳子,见一脉一岁死;见一脉半,一岁半死;见二脉,二岁死;见二脉半,二岁半死;见三脉,三岁死。
	
	诊龋齿痛,按其阳之来,有过者独热,在左左热,在右右热,在上上热,在下下热。
	
	诊血脉者,多赤多热,多青多痛,多黑为久痹,多赤、多黑、多青皆见者,寒热。
	
	身痛而色微黄,齿垢黄,爪甲上黄,黄疸也。安卧小便黄赤,脉小而涩者不嗜食。
	
	人病,其寸口之脉,与人迎之脉小大等,及其浮沉等者,病难已也。
	
	女子手少阴脉动甚者妊子。
	
	婴儿病,其头毛皆逆上者必死。耳间青脉起者掣痛。大便赤瓣飧泄,脉小者,手足寒,难已;飧泄,脉小,手足温,泄易也。
	
	四时之变,寒暑之胜,重阴必阳,重阳必阴;故阴主寒,阳主热,故寒甚则热,热甚则寒,故曰寒生热,热生寒,此阴阳之变也。
	
	故曰:冬伤于寒,春生病热;春伤于风,夏生飧泄肠僻,夏伤于暑,秋生疟;秋伤于湿,冬生咳嗽。是谓四时之序也。
	\end{yuanwen}

	\chapter{刺节真邪}
	
	\begin{yuanwen}
	黄帝问于岐伯曰:余闻刺有五卫,奈何?
	
	岐伯曰:固有五卫,一曰振埃,二曰发蒙,三曰去爪,四曰彻衣,五曰解惑。
	
	黄帝曰:夫子言五卫,余未知其意。
	
	岐伯曰:振埃者,刺外经去阳病也;发蒙者,刺腑俞,去腑病也;去爪者,刺关节肢络也;彻衣者,尽刺诸阳之奇俞也;解惑者,尽知调阴阳,补泻有余不足,相倾移也。
	
	黄帝曰:刺卫言振埃,夫子乃言刺外经,去阳病,余不知其所谓也。愿卒闻之。
	
	岐伯曰:振埃者,阳气大逆,上满于胸中,愤瞋肩息,大气逆上,喘喝坐伏,病恶埃烟,饲不得息,请言振埃,尚疾于振埃。
	
	黄帝曰:善。取之何如?
	
	岐伯曰:取之天容。
	
	黄帝曰:其咳上气穷拙胸痛者,取之奈何?
	
	岐伯曰:取之廉泉。
	
	黄帝曰:取之有数乎?
	
	岐伯曰:取天容者,无过一里,取廉泉者,血变而止。
	
	帝曰:善哉。
	
	黄帝曰:刺卫言发蒙,余不得其意。夫发蒙者,耳无所闻,目无所见,夫子乃言刺腑俞,去腑病,何输使然,愿闻其故。
	
	岐伯曰:妙乎哉问也。此刺之大约,针之极也,神明之类也,口说书卷,犹不能及也,请言发蒙耳,尚疾于发蒙也。
	
	黄帝曰:善。愿卒闻之。
	
	岐伯曰:刺此者,必于日中,刺其听宫,中其眸子,声闻于耳,此其输也。
	
	黄帝曰:善。何谓声闻于耳?
	
	岐伯曰:刺邪以手坚按其两鼻窍,而疾偃其声,必应于针也。
	
	黄帝曰:善。此所谓弗见为之,而无目视,见而取之,神明相得者也。
	
	黄帝曰:刺卫言去爪,夫子乃言刺关节肢络,愿卒闻之。
	
	岐伯曰:腰脊者,身之大关节也;肢胫者,人之管以趋翔也;茎垂者,身中之机,阴精之候,津液之道也。故饮食不节,喜怒不时,津液内溢,乃下留于睪,血道不通,日大不休,俛仰不便,趋翔不能。此病荣然有水,不上不下,铍石所取,形不可匿,常不得蔽,故命曰去爪。
	
	帝曰:善。
	
	黄帝曰:刺卫言彻衣,夫子乃言尽刺诸阳之奇俞,未有常处也。愿卒闻之。
	
	岐伯曰:是阳气有余,而阴气不足,阴气不足则内热,阳气有余则外热,内热相搏,热于怀炭,外畏绵帛近,不可近身,又不可近席。腠理闭塞,则汗不出,舌焦唇槁,腊干益燥,饮食不让美恶。
	
	黄帝曰:善。取之奈何?
	
	岐伯曰:取之于其天府大杼三痕,又刺中膂,以去其热,补足手太阴,以去其汗,热去汗稀,疾于彻衣。
	
	黄帝曰:善。
	
	黄帝曰:刺卫言解惑,夫子乃言尽知调阴阳,补泻有余不足,相倾移也,惑何以解之?
	
	岐伯曰:大风在身,血脉偏虚,虚者不足,实者有余,轻重不得,倾侧宛伏,不知东西,不知南北,乍上乍下,乍反乍复,颠倒无常,甚于迷惑。
	
	黄帝曰:善。取之奈何?
	
	岐伯曰:泻其有余,补其不足,阴阳平复,用针若此,疾于解惑。
	
	黄帝曰:善。请藏之灵兰之室,不敢妄出也。
	
	黄帝曰:余闻刺有五邪,何谓五邪?
	
	岐伯曰:病有持痈者,有容大者,有狭小者,有热者,有寒者,是谓五邪。
	
	黄帝曰:刺五邪奈何?
	
	岐伯曰:凡刺五邪之方,不过五章,瘅热消灭,肿聚散亡,寒痹益温,小者益阳;大者必去,请道其方。
	
	凡刺痈邪,无迎陇,易俗移性。不得脓,脆道更行,去其乡,不安处所乃散亡,诸阴阳过痈者,取之其输泻之。
	
	凡刺大邪,日以小,泄夺其有余,乃益虚。剽其通,针其邪,肌肉亲视之,毋有反其真,刺诸阳分肉间。
	
	凡刺小邪,日以大,补其不足,乃无害。视其所在,迎之界,远近尽至,其不得外侵而行之,乃自费,刺分肉间。
	
	凡刺热邪,越而苍,出游不归,乃无病。为开通,辟门户,使邪得出,病乃已。
	
	凡刺寒邪,日以温,徐往徐来,致其神。门户已闭,气不分,虚实得调,其气存也。
	
	黄帝曰:官针奈何?
	
	岐伯曰:刺痈者,用铍针;刺大者,用锋针;刺小者,用员利针;刺热者,用纔针;刺寒者,用毫针也。
	
	请言解论,与天地相应,与四时相副,人参天地,故可为解。下有渐洳,上生苇蒲,此所以知形气之多少也。阴阳者,寒暑也,热则滋雨而在上,根茎少汁,人气在外,皮肤缓,腠理开,血气减,汗大泄,皮淖泽。寒则地冻水冰,人气在中,皮肤致,腠理闭,汗不出,血气强,肉坚涩。当是之时,善行水者,不能往冰,善穿地者,不能击冻,善用针者,亦不能取四厥,血脉凝结,坚搏不往来者,亦未可即柔。故行水者,必待天温,冰释冻解,而水可行,地可穿也。人脉犹是也。治厥者,必先熨调和其经,掌与腋,肘与脚,项与脊以调之,火气已通,血脉乃行。然后视其病,脉淖泽者,刺而平之;坚紧者,破而散之,气下乃止,此所谓以解结者也。
	
	用针之类,在于调气,气积于胃,以通营卫,各行其道。宗气留于海,其下者,注于气街,其上者,走于息道。故厥在于足,宗气不下,脉中之血,凝而留止,弗之火调,弗能取之。
	
	用针者,必先察其经络之实虚,切而循之,按而弹之,视其应动者,乃后取之而下之。六经调者,谓之不病,虽病,谓之自已也。一经上实下虚而不通者,此必有横络盛加于大经,令之不通,视而泻之,此所谓解结也。
	
	上寒下热,先刺其项太阳,久留之,已刺则熨项与肩胛,令热下合乃止,此所谓推而上之者也。上热下寒,视其虚脉而陷之于经络者,取之,气下乃止,此所谓引而下之者也。
	
	大热遍身,狂而妄见妄闻妄言,视足阳明及大络取之,虚者补之,血而实者泻之。因其偃卧,居其头前,以两手四指挟按颈动脉,久持之,卷而切,推下至缺盆中,而复止如前,热去乃止,此所谓推而散之者也。
	
	黄帝曰:有一脉生数十病者,或痛,或痈,或热,或寒,或痒,或痹,或不仁,变化无穷,其故何也?
	
	岐伯曰:此皆邪气之所生也。
	
	黄帝曰:余闻气者,有真气,有正气,有邪气。何谓真气?
	
	岐伯曰:真气者,所受于天,与谷气并而充身也。正气者,正风也,从一方来,非实风,又非虚风也。邪气者,虚风之贼伤人也,其中人也深,不能自去。正风者,其中人也浅,合而自去,其气来柔弱,不能胜真气,故自去。
	
	虚邪之中人也,洒晰动形,起毫毛而发腠理。其入深,内搏于骨,则为骨痹;搏于筋,则为筋挛;搏于脉中,则为血闭,不通则为痈。搏于肉,与卫气相搏,阳胜者,则为热,阴胜者,则为寒。寒则真气去,去则虚,虚则寒搏于皮肤之间。其气外发,腠理开,毫毛摇,气往来行,则为痒。留而不去,则痹。卫气不行,则为不仁。
	
	虚邪偏容于身半,其入深,内居荣卫,荣卫稍衰,则真气去,邪气独留,发为偏枯。其邪气浅者,脉偏痛。
	
	虚邪之入于身也深,寒与热相搏,久留而内着,寒胜其热,则骨疼肉枯;热胜其寒,则烂肉腐肌为脓,内伤骨,内伤骨为骨蚀。有所疾前筋,筋屈不得伸,邪气居其间而不反,发为筋溜。有所结,气归之,卫气留之,不得反,津液久留,合而为肠溜。久者,数岁乃成,以手按之柔,已有所结,气归之,津液留之,邪气中之,凝结日以易甚,连以聚居,为昔瘤。以手按之坚,有所结,深中骨,气因于骨,骨与气并,日以益大,则为骨疽。有所结,中于肉,宗气归之,邪留而不去,有热则化而为脓,无热则为肉疽。凡此数气者,其发无常处,而有常名也。
	\end{yuanwen}	
	
	\chapter{卫气行}
	
	\begin{yuanwen}
	黄帝问于岐伯曰:愿闻卫气之行,出入之合,何如?
	
	岐伯曰:岁有十二月,日有十二辰,子午为经,卯酉为纬。天周二十八宿,而一面七星,四七二十八星。房昴为纬,虚张为经。是故房至毕为阳,昴至心为阴。阳主昼,阴主夜。故卫气之行,一日一夜五十周于身,昼日行于阳二十五周,夜行于阴二十五周,周于五藏。
	
	是故平旦阴尽,阳气出于目,目张则气上行于头,循项下足太阳,循背下至小趾之端。其散者,别于目锐眦,下手太阳,下至手小指之间外侧。其散者,别于目锐眦,下足少阳,注小趾次趾之间。以上循手少阳之分侧,下至小指之间。别者以上至耳前,合于颔脉,注足阳明以下行,至跗上,入五趾之间。其散者,从耳下下手阳明,入大指之间,入掌中。其至于足也,入足心,出内踝,下行阴分,复合于目,故为一周。
	
	是故日行一舍,人气行一周与十分身之八;日行二舍,人气行三周于身与十分身之六;日行三舍,人气行于身五周与十分身之四;日行四舍,人气行于身七周与十分身之二;日行五舍,人气行于身九周;日行六舍,人气行于身十周与十分身之八;日行七舍,人气行于身十二周在身与十分身之六;日行十四舍,人气二十五周于身有奇分与十分身之二,阳尽于阴,阴受气矣。其始入于阴,常从足少阴注于肾,肾注于心,心注于肺,肺注于肝,肝注于脾,脾复注于肾为周。是故夜行一舍,人气行于阴藏一周与十分藏之八,亦如阳行之二十五周,而复合于目。阴阳一日一夜,合有奇分十分身之四,与十分藏之二,是故人之所以卧起之时,有早晏者,奇分不尽故也。
	
	黄帝曰:卫气之在于身也,上下往来不以期,候气而刺之,奈何?
	
	伯高曰:分有多少,日有长短,春秋冬夏,各有分理,然后常以平旦为纪,以夜尽为始。是故一日一夜,水下百刻,二十五刻者,半日之度也,常如是毋已,日入而止,随日之长短,各以为纪而刺之。谨候其时,病可与期,失时反候者,百病不治。故曰:刺实者,刺其来也,刺虚者,刺其去也。此言气存亡之时,以候虚实而刺之,是故谨候气之所在而刺之,是谓逢时。在于三阳,必候其气在于阳而刺之,病在于三阴,必候其气在阴分而刺之。
	
	水下一刻,人气在太阳;水下二刻,人气在少阳;水下三刻,人气在阳明;水下四刻,人气在阴分。水下五刻,人气在太阳;水下六刻,人气在少阳;水下七刻,人气在阳明;水下八刻,人气在阴分。水下九刻,人气在太阳;水下十刻,人气在少阳;水下十一刻,人气在阳明;水下十二刻,人气在阴分。水下十三刻,人气在太阳;水下十四刻,人气在少阳;水下十五刻,人气在阳明;水下十六刻,人气在阴分。水下十七刻,人气在太阳;水下十八刻,人气在少阳;水下十九刻,人气在阳明;水下二十刻,人气在阴分。水下二十一刻,人气在太阳;水下二十二刻,人气在少阳;水下二十三刻,人气在阳明;水下二十四刻,人气在阴分。水下二十五刻,人气在太阳,此半日之度也。从房至毕一十四舍水下五十刻,日行半度,回行一舍,水下三刻与七分刻之四。大要曰:常以日之加于宿上也,人气在太阳,是故日行一舍,人气行三阳行与阴分,常如是无已,天与地同纪,纷纷纷纷,终而复始,一日一夜水下百刻而尽矣。
	\end{yuanwen}

	\chapter{九宫八风}
	
	\begin{yuanwen}
	太一常以冬至之日,居叶蛰之宫四十六日,明日居天留四十六日,明日居仓门四十六日,明日居阴洛四十五日,明日居天宫四十六日,明日居玄委四十六日,明日居仓果四十六日,明日居新洛四十五日,明日复居叶蛰之宫,曰冬至矣。
	
	太一日游,以冬至之日,居叶蛰之宫,数所在日,从一处至九日,复返于一。常如是无已,终而复始。
	
	太一移日,天必应之以风雨,以其日风雨则吉,岁美民安少病矣。先之则多雨,后之则多汗。太一在冬至之日有变,占在君;太一在春分之日有变,占在相;太一在中宫之日有变,占在吏;太一在秋分之日有变,占在将;太一在夏至之日有变,占在百姓。所谓有变者,太一居五宫之日,病风折树木,扬沙石,各以其所主,占贵贱。因视风所从来而占之,风从其所居之乡来为实风,主生,长养万物;从其冲后来为虚风,伤人者也,主杀,主害者。谨候虚风而避之,故圣人日避虚邪之道,如避矢石然,邪弗能害,此之谓也。
	
	是故太一入徙立于中宫,乃朝八风,以占吉凶也。风从南方来,名曰大弱风,其伤人也,内舍于心,外在于脉,气主热。风从西南方来,名曰谋风,其伤人也,内舍于脾,外在于肌,其气主为弱。风从西方来,名曰刚风,其伤人也,内舍于肺,外在于皮肤,其气主为燥。风从西北方来,名曰折风,其伤人也,内舍于小肠,外在于手太阳脉,脉绝则溢,脉闭则结不通,善暴死。风从北方来,名曰大刚风,其伤人也,内舍于肾,外在于骨与肩背之膂筋,其气主为寒也。风从东北方来,名曰凶风,其伤人也,内舍于大肠,外在于两胁腋骨下及肢节。风从东方来,名曰婴兀风,其伤人也,内舍于肝,外在于筋纽,其气主为身湿。风从东南方来,名曰弱风,其伤人也,内舍于胃,外在肌肉,其气主体重。此八风皆从其虚之乡来,乃能病人。三虚相搏,则为暴病卒死。两实一虚,病则为淋露寒热。犯其两湿之地,则为痿。故圣人避风,如避矢石焉。其有三虚而偏中于邪风,则为仆偏枯矣。
	\end{yuanwen}
		
	
	
	\part{}
	
	\chapter{九针论}
	
	\begin{yuanwen}
	黄帝曰:余闻九针于夫子,众多博大矣,余犹不能寤,敢问九针焉生,何因而有名?
	
	岐伯曰:九针者,天地之大数也,始于一而终于九。故曰:一以法天,二以法地,三以法人,四以法时,五以法音,六以法律,七以法星,八以法风,九以法野。
	
	黄帝曰:以针应九之数,奈何?
	
	岐伯曰:夫圣人之起天地之数也,一而九之,故以立九野。九而九之,九九八十一,以起黄钟数焉,以针应数也。
	
	一者,天也。天者,阳也。五藏之应天者肺,肺者,五藏六府之盖也,皮者,肺之合也,人之阳也。故为之治针,必以大其头而锐其末,令无得深入而阳气出。
	
	二者,地也。人之所以应土者,肉也。故为之治针,必筩其身而员其末,令无得伤肉分,伤则气得竭。
	
	三者,人也。人之所以成生者,血脉也。故为之治针,必大其身而员其末,令可以按脉物陷,以致其气,令邪气独出。
	
	四者,时也。时者,四时八风之客于经络之中,为瘤病者也。故为之治针,必筩其身而锋其末,令可以泻热出血,而痼病竭。
	
	五者,音也。音者,冬夏之分,分于子午,阴与阳别,寒与热争,两气相搏,合为痈脓者也。故为之治针,必令其末如剑锋,可以取大脓。
	
	六者,律也。律者,调阴阳四时而合十二经脉,虚邪客于经络而为暴痹者也。故为之治针,必令尖如厘,且员其锐,中身微大,以取暴气。
	
	七者,星也。星者,人之七窍,邪之所客于经,而为痛痹,合于经络者也。故为之治针,令尖如蚊虻喙,静以徐往,微以久留,正气因之,真邪俱往,出针而养者也。
	
	八者,风也。风者,人之股肱八节也。八正之虚风,八风伤人,内舍于骨解腰脊节腠理之间为深痹也。故为之治针,必长其身,锋其末,可以取深邪远痹。
	
	九者,野也。野者,人之节解皮肤之间也。淫邪流溢于身,如风水之状,而留不能过于机关大节者也。故为之治针,令尖如挺,其锋微员,以取大气之不能过于关节者也。
	
	黄帝曰:针之长短有数乎?
	
	岐伯曰:一曰铁针者,取法于巾针,去末寸半,卒锐之,长一寸六分,主热在头身也。二曰员针,取法于絮针,其身而卵其锋,长一寸六分,主治分间气。三曰提针,取法于黍粟之锐,长三寸半,主按脉取气,令邪出。四曰锋针,取法于絮针,其身,锋其末,长一寸六分,主痈热出血。五曰铍针,取法于剑锋,广二分半,长四寸,主大痈脓,两热争者也。六曰员利针,取法于厘针,微大其末,反小其身,令可深内也,长一寸六分。主取痈痹者也。七曰毫针,取注于毫毛,长一寸六分,主寒热痛痹在络者也。八曰长针,取法于綦针,长七寸,主取深邪远痹者也。九曰大针,取法于锋针,其锋微员,长四寸,主取大气不出关节者也。针形毕矣,此九针大小长短法也。
	
	黄帝曰:愿闻身形,应九野,奈何?
	
	岐伯曰:请言身形之应九野也,左足应立春,其日戊寅己丑。左胁应春分,其日乙卯。左手应立夏,其日戊辰己巳。膺喉首头应夏至,其日丙午。右手应立秋,其中戊申己末。右胁应秋分,其日辛酉。右足应立冬,其日戊戌己亥。腰尻下窍应冬至,其日壬子。六腑下三脏应中州,其大禁,大禁太一所在之日,及诸戊己。凡此九者,善候八正所在之处。所主左右上下身体有痈肿者,欲治之,无以其所直之日溃治之,是谓天忌日也。
	
	形东志苦,病生于脉,治之于灸刺。形苦志东,病生于筋,治之以熨引。形东志东,病生于肉,治之以针石。形苦志苦,病生于咽喝,治之以甘药。形数惊恐,筋脉不通,病生于不仁,治之以按摩谬药。是谓形。
	
	五脏气,心主噫,肺主咳,肝主语,脾主吞,肾主欠。
	
	六腑气,胆为怒,胃为气逆秽,大肠小肠为泄,膀胱不约为遗溺,下焦溢为水。
	
	五味:酸入肝,辛入肺,苦入心,甘入脾,咸入肾,淡入胃,是谓五味。
	
	五并:精气并肝则忧,并心则喜,并肺则悲,并肾则恐,并脾则畏,是谓五精之气,并于脏也。
	
	五恶:肝恶风,心恶热,肺恶寒,肾恶燥,脾恶湿,此五脏气所恶也。
	
	五液:心主汗,肝主泣,肺主涕,肾主唾,脾主液,此五液所出也。
	
	五劳:久视伤血,久卧伤气,久坐伤肉,久立伤骨,久行伤筋,此五久劳所病也。
	
	五走:酸走筋,辛走气,苦走血,咸走骨,甘走肉,是谓五走也。
	
	五裁:病在筋,无食酸;病在气,无食辛;病在骨,无食咸;病在血,无食苦;病在肉,无食甘。口嗜而欲食之,不可多也,必自裁也,命曰五裁。
	
	五发:阴病发于骨,阳病发于血,阴病发于肉,阳病发于冬,阴病发于夏。
	
	五邪:邪入于阳,则为狂;邪入于阴,则为血瘅;邪入于阳,转则为癫疾;邪入于阴,转则为瘖;阳入于阴,病静;阴出之于阳,病喜怒。
	
	五藏:心藏神,肺藏魄,肝藏魂,脾藏意,肾藏精志也。
	
	五主:心主脉,肺主皮,肝主筋,脾主肌,肾主骨。
	
	阳明多血多气,太阳多血少气,少阳多气少血,太阴多血少气,厥阴多血少气,少阴多气少血。故曰刺阳明出血气,刺太阳出血恶气,刺少阳出气恶血,刺太阴出血恶气,刺厥阴出血恶气,刺少阴出气恶血也。
	
	足阳明太阴为里表,少阳厥阴为表里,太阳少阴为表里,是谓足之阴阳也。手阳明太阴为表里,少阳心主为表里,太阳少阴为表里,是谓手之阴阳也。
	\end{yuanwen}
	
	\chapter{岁露论}
	
	\begin{yuanwen}
    黄帝问于岐伯曰:经言夏日伤暑,秋病疟,疟之发以时,其故何也?
    
    岐伯对曰:邪客于风府,病循膂而下,卫气一日一夜,常大会于风府,其明日日下一节,故其日作晏,此其先客于脊背也。故每至于风府则腠理开,腠理开则邪气入,邪气入则病作,此所以日作尚晏也。卫气之行风府,日下一节,二十一日下至尾底,二十二日入脊内,注于伏冲之脉,其行九日,出于缺盆之中,其气上行,故其病稍益至。其内搏于五脏,横连募原,其道远,其气深,其行迟,不能日作,故次日乃蓄积而作焉。
    
    黄帝曰:卫气每至于风府,腠理乃发,发则邪入焉。其卫气日下一节,则不当风府,奈何?
    
    岐伯曰:风府无常,卫气之所应,必开其腠理,气之所舍节,则其府也。
    
    黄帝曰:善。夫风之与疟也,相与同类,而风常在,而疟特以时休,何也?
    
    岐伯曰:风气留其处,疟气随经络,沉以内搏,故卫气应,乃作也。
    
    帝曰:善。
    
    黄帝问于少师曰:余闻四时八风之中人也,故有寒暑,寒则皮肤急而腠理闭;暑则皮肤缓而腠理开。贼风邪气,因得以入乎?将必须八正虚邪,乃能伤人乎?
    
    少师答曰:不然。贼风邪气之中人也,不得以时,然必因其开也,其入深,其内极病,其病人也,卒暴。因其闭也,其入浅以留,其病也,徐以迟。
    
    黄帝曰:有寒温和适,腠理不开,然有卒病者,其故何也?
    
    少师答曰:帝弗知邪入乎。虽平居其腠理开闭缓急,其故常有时也。
    
    黄帝曰:可得闻乎?
    
    少师曰:人与天地相参也,与日月相应也。故月满则海水西盛,人血气积,肌肉充,皮肤致,毛发坚,腠理郗,烟垢着,当是之时,虽遇贼风,其入浅不深。至其月郭空,则海水东盛,人气血虚,其卫气去,形独居,肌肉减,皮肤纵,腠理开,毛发残,胶理薄,烟垢落,当是之时,遇贼风则其入深,其病人也,卒暴。
    
    黄帝曰:其有卒然暴死暴病者,何也?
    
    少师答曰:三虚者,其死暴疾也;得三实者邪不能伤人也。
    
    黄帝曰:愿闻三虚。少师曰:乘年之衰,逢月之空,失时之和,因为贼风所伤,是谓三虚。故论不知三虚,工反为粗。
    
    帝曰:愿闻三实。
    
    少师曰:逢年之盛,遇月之满,得时之和,虽有贼风邪气,不能危之也。
    
    黄帝曰:善乎哉论!明乎哉道!请藏之金匮,命曰三实。然,此一夫之论也。
    
    黄帝曰:愿闻岁之所以皆同病者,何因而然?
    
    少师曰:此八正之候也。
    
    黄帝曰:候之奈何?
    
    少师曰:候此者,常以冬至之日,太一立于叶蛰之宫,其至也,天必应之以风雨者矣。风雨从南方来者,为虚风,贼伤人者也。其以夜半至也,万民皆卧而弗犯也,故其岁民少病。其以昼至者,万民懈惰而皆中于虚风,故万民多病。虚邪入客于骨而不发于外,至其立春,阳气大发,腠理开,因立春之日,风从西方来,万民又皆中于虚风,此两邪相搏,经气结代者矣。故诸逢其风而遇其雨者,命曰遇岁露焉,因岁之和,而少贼风者,民少病而少死。岁多贼风邪气,寒温不和,则民多病而死矣。
    
    黄帝曰:虚邪之风,其所伤贵贱何如,候之奈何?
    
    少师答曰:正月朔日,太一居天留之宫,其日西北风,不雨,人多死矣。正月朔日,平旦北风,春,民多死。正月朔日,平旦北风行,民病多者,十有三也。正月朔日,日中北风,夏,民多死。正月朔日,夕时北风,秋,民多死。终日北风,大病死者十有六。正月朔日,风从南方来,命曰旱乡;从西方来,命曰白骨,将国有殃,人多死亡。正月朔日,风从东方来,发屋,扬沙石,国有大灾也。正月朔日,风从东南方行,春有死亡。正月朔日,天和温不风粜贱,民不病;天寒而风,粜贵,民多病。此所谓候岁之风,残伤人者也。二月丑不风,民多心腹病;三月戌不温,民多寒热;四月已不暑,民多瘅病;十月申不寒,民多暴死。诸所谓风者,皆发屋,折树木,扬沙石起毫毛,发腠理者也。
	\end{yuanwen}
	
	
	\chapter{大惑论}
	
	\begin{yuanwen}
	黄帝问于岐伯曰:余尝上于清冷之台,中阶而顾,匍匐而前,则惑。余私异之,窃内怪之,独瞑独视,安心定气,久而不解。独博独眩,披发长跪,俛而视之,后久之不已也。卒然自上,何气使然?
	
	岐伯对曰:五脏六腑之精气,皆上注于目而为之精。精之窠为眼,骨之精为瞳子,筋之精为黑眼,血之精为络,其窠气之精为白眼,肌肉之精为约束,裹撷筋骨血气之精,而与脉并为系。上属于脑,后出于项中。故邪中于项,因逢其身之虚,其入深,则随眼系以入于脑。入于脑则脑转,脑转则引目系急。目系急则目眩以转矣。邪其精,其精所中不相比也,则精散。精散则视歧,视歧见两物。目者,五脏六腑之精也,营卫魂魄之所常营也,神气之所生也。故神劳则魂魄散,志意乱。是故瞳子黑眼法于阴,白眼赤脉法于阳也。故阴阳合传而精明也。目者,心使也。心者,神之舍也,故神精乱而不转。卒然见非常处精神魂魄,散不相得,故曰惑也。
	
	黄帝曰:余疑其然。余每之东苑,未曾不惑,去之则复,余唯独为东苑劳神乎?何其异也?
	
	岐伯曰:不然也。心有所喜,神有所恶,卒然相惑,则精气乱,视误,故惑,神移乃复。是故间者为迷,甚者为惑。
	
	黄帝曰:人之善忘者,何气使然?
	
	岐伯曰:上气不足,下气有余,肠胃实而心肺虚。虚则营卫留于下,久之不以时上,故善忘也。
	
	黄帝曰:人之善饥而不嗜食者,何气使然?
	
	岐伯曰:精气并于脾,热气留于胃,胃热则消谷,谷消故善饥。胃气逆上,则胃脘寒,故不嗜食也。
	
	黄帝曰:病而不得卧者,何气使然?
	
	岐伯曰:卫气不得入于阴,常留于阳。留于阳则阳气满,阳气满则阳蹻盛,不得入于阴则阴气虚,故目不瞑矣。
	
	黄帝曰:病目而不得视者,何气使然?
	
	岐伯曰:卫气留于阴,不得行于阳,留于阴则阴气盛,阴气盛则阴蹻满,不得入于阳则阳气虚,故目闭也。
	
	黄帝曰:人之多卧者,何气使然?
	
	岐伯曰:此人肠胃大而皮肤湿,而分肉不解焉。肠胃大则卫气留久;皮肤湿则分肉不解,其行迟。夫卫气者,昼日常行于阳,夜行于阴,故阳气尽则卧,阴气尽则寤。故肠胃大,则卫气行留久;皮肤湿,分肉不解,则行迟。留于阴也久,其气不清,则欲瞑,故多卧矣。其肠胃小,皮肤滑以缓,分肉解利,卫气之留于阳也久,故少瞑焉。
	
	黄帝曰:其非常经也,卒然多卧者,何气使然?
	
	岐伯曰:邪气留于上焦,上焦闭而不通,已食若饮汤,卫气留久于阴而不行,故卒然多卧焉。
	
	黄帝曰:善。治此诸邪,奈何?
	
	岐伯曰:先其脏腑,诛其小过,后调其气,盛者泻之,虚者补之,必先明知其形志之苦乐,定乃取之。
	\end{yuanwen}	
	
	
	\chapter{痈疽}
	
	\begin{yuanwen}
	黄帝曰:余闻肠胃受谷,上焦出气,以温分肉,而养骨节,通腠理。中焦出气如露,上注溪谷,而渗孙脉,津液和调,变化而赤为血。血和则孙脉先满溢,乃注于络脉,皆盈,乃注于经脉,阴阳已张,因息乃行。行有经纪,周有道理,与天合同,不得休止。切而调之,从虚去实,泻则不足,疾则气减,留则先后。从实去虚,补则有余,血气已调,形气乃持。余已知血气之平与不平,未知痈疽之所从生,成败之时,死生之期,有远近,何以度之,可得闻乎?
	
	岐伯曰:经脉留行不止,与天同度,与地合纪。故天宿失度,日月薄蚀;地经失纪,水道流溢,草萓不成,五谷不殖;径路不通,民不往来,巷聚邑居,则别离异处。血气犹然,请言其故。夫血脉营卫,周流不休,上应星宿,下应经数。寒邪客于经络之中,则血泣,血泣则不通,不通则卫气归之,不得复反,故痈肿。寒气化为热,热胜则腐肉,肉腐则为脓。脓不泻则烂筋,筋烂则伤骨,骨伤则髓消,不当骨空,不得泄泻,血枯空虚,则筋骨肌肉不相荣,经脉败漏,熏于五脏,藏伤故死矣。
	
	黄帝曰:愿尽闻痈疽之形,与忌曰名。
	
	岐伯曰:痈发于嗌中,名曰猛疽。猛疽不治,化为脓,脓不泻,塞咽,半日死。其化为脓者,泻则合豕膏,冷食,三日而已。
	
	发于颈,名曰夭疽。其痈大以赤黑,不急治,则热气下入渊腋,前伤任脉,内熏肝肺。熏肝肺,十余日而死矣。
	
	阳留大发,消脑留项,名曰脑烁。其色不乐,项痛而如刺以针。烦心者,死不可治。
	
	发于肩及臑,名曰疵痈。其状赤黑,急治之,此令人汗出至足,不害五脏。痈发四五日,逞焫之。
	
	发于腋下赤坚者,名曰米疽。治之以砭石,欲细而长,疏砭之,涂以豕膏,六日已,勿裹之。其痈坚而不溃者,为马刀挟瘿,急治之。
	
	发于胸,名曰井疽。其状如大豆,三四日起,不早治,下入腹,不治,七日死矣。
	
	发于膺,名曰甘疽。色青,其状如谷实柧楼,常苦寒热,急治之,去其寒热,十岁死,死后出脓。
	
	发于胁,名曰败疵。败疵者,女子之病也,灸之,其病大痈脓,治之,其中乃有生肉,大如赤小豆,坐陵翘草根各一升,以水一斗六升煮之,竭为取三升,则强饮厚衣,坐于釜上,令汗出至足已。
	
	发于股胫,名曰股胫疽。其状不甚变,而痈脓搏骨,不急治,三十日死矣。
	
	发于尻,名曰锐疽。其状赤坚大,急治之,不治,三十日死矣。
	
	发于股阴,名曰赤施。不急治,六十日死。在两股之内,不治,十日而当死。
	
	发于膝,名曰疵痈。其状大,痈色不变,寒热,如坚石,勿石,石之者死,须其柔,乃石之者,生。
	
	诸痈疽之发于节而相应者,不可治也。发于阳者,百日死;发于阴者,三十日死。
	
	发于胫,名曰兔啮,其状赤至骨,急治之,不治害人也。
	
	发于内踝,名曰走缓。其状痈也,色不变,数石其输,而止其寒热,不死。
	
	发于足上下,名曰四淫。其状大痈,急治之,百日死。
	
	发于足傍,名曰厉痈。其状不大,初如小指,发,急治之,去其黑者;不消辄益,不治,百日死。
	
	发于足趾,名脱痈。其状赤黑,死不治;不赤黑,不死。不衰,急斩之,不则死矣。
	
	黄帝曰:夫子言痈疽,何以别之?
	
	岐伯曰:营卫稽留于经脉之中,则血泣而不行,不行则卫气从之而不通,壅遏而不得行,故热。大热不止,热胜,则肉腐,肉腐则为脓。然不能陷,骨髓不为焦枯,五脏不为伤,故命曰痈。
	
	黄帝曰:何谓疽?
	
	岐伯曰:热气淳盛,下陷肌肤,筋髓枯,内连五脏,血气竭,当其痈下,筋骨良肉皆无余,故命曰疽。疽者,上之皮夭以坚,上如牛领之皮。痈者,其皮上薄以泽。此其候也。
	\end{yuanwen}
	
	
\end{document}