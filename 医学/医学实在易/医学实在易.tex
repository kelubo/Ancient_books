%-*- coding: UTF-8 -*-
% 医学实在易.tex

\documentclass[a4paper,12pt,UTF8,twoside]{ctexbook}

% 设置纸张信息。
\RequirePackage[a4paper]{geometry}
\geometry{
	%textwidth=138mm,
	%textheight=215mm,
	%left=27mm,
	%right=27mm,
	%top=25.4mm, 
	%bottom=25.4mm,
	%headheight=2.17cm,
	%headsep=4mm,
	%footskip=12mm,
	%heightrounded,
	inner=1in,
	outer=1.25in
}

% 目录 chapter 级别加点(.)。
\usepackage{titletoc}
\titlecontents{chapter}[0pt]{\vspace{3mm}\bf\addvspace{2pt}\filright}{\contentspush{\thecontentslabel\hspace{0.8em}}}{}{\titlerule*[8pt]{.}\contentspage}

% 设置 part 和 chapter 标题格式。
\ctexset{
	part/name= {第,卷},
	part/number={\chinese{part}},
	chapter/name={第,篇},
	chapter/number={\chinese{chapter}}
}

% 设置古文原文格式。
\newenvironment{yuanwen}{\noindent\bfseries\zihao{4}}

\title{\heiti\zihao{0} 医学实在易}
\author{陈修园}
\date{清▪1809年}

\begin{document}

\maketitle
\tableofcontents

\frontmatter
\chapter{前言、序言}

\mainmatter
\part{卷一}

十二官
六脏六腑纳甲诗
内景说
心说
肝说
脾说
肺说
肾说
胃说
胆说
大肠小肠说
三焦说
手心主说( 即心包络 )
膀胱说
命门说
附录高士宗部位说
经络易知
十二经脉起止图
十六络穴图
经络说
十二经诗
十六络诗
十二经气血流注诗(旧本)
十二经气血多少诗
四诊易知
望色说
望色诗
辨舌说
辨舌诗
闻声诗
其二(僧自性着) 
问证说
问证诗
切脉说
《内经》分发脏腑
王叔和分发脏腑
李濒湖分发脏腑
张景岳分发脏腑
寸关尺分诊三焦
脉法统论
持脉秘旨
新着八脉四言诗
七怪脉四言诗
妇人科诊脉四言诗
小儿验纹按额诊脉四言诗
诊脉别解一
诊脉别解二
《内经》诊法
附录徐灵胎诊脉决死生论
拟补徐灵胎诊脉论诗
运气易知
司天在泉图说
张飞畴运气不足凭说

\part{卷二}
运气易知 节
太阳表证诗
阳明表证诗
少阳表证诗
感冒诗
疟疾诗
瘟疫诗
盗汗自汗诗 节
中风证歌 节
治中风八法歌 节
附中风应灸俞穴 节
历节风诗 节
痹诗 节
鹤膝风诗 节
脚气诗
暑证诗
附录高士宗中暑论 节
湿证诗
肿证诗
头痛诗
续论 节
眩晕诗
咳嗽诗
咳嗽续论
又诗一首
\part{卷三}
里证
伤寒里证诗
心腹诸痛诗
痰饮诗
附录
痢疾诗
痢疾救逆诗三首
阴虚下痢诗
奇恒痢疾诗
泄泻诗
秘结诗
隔证诗
反胃诗
隔证余论
腰痛诗
不寐诗
不能食诗
谷劳诗
食亦诗
黄疸诗
伤寒条
伤寒证诗
霍乱诗
续论
录《千金》孙真人治霍乱吐下治中汤 节
臌胀诗
蛊胀诗
疝气诗
厥证诗
寒症统论
\part{卷四}
伤寒条
伤寒表热诗
伤寒中热里热诗
附引三条
口糜龈烂诗
血证诗
下血久不止用断红丸诗
血证穷极用当归补血汤诗
喘促诗
哮症诗
癃闭诗
浊证诗
呕吐哕呃逆诗
吞酸诗
三消诗
附录张隐庵消渴论
续论
实证
内外俱实病诗
阻塞清道诗
阻塞浊道诗
积聚诗
癫狂痫诗
伤食诗
伤酒诗
久服地黄暴脱诗
室女经闭诗
祟病诗
砂证诗
虚证八条
理中丸汤诗
附子汤诗 
炙甘草汤诗
虚痨诗
怔忡诗
痿证诗
遗精诗
遗溺诗
房痨伤寒诗
素盛一条 
素盛服药诗
素衰一条 
素衰服药诗
\part{卷五}
表证诸方
伤寒
感冒
疟疾
瘟疫
阴虚盗汗症
盗汗自汗
汗出不治症
黄汗备方
中风症
历节风
痹症
鹤膝风
脚气
暑证
湿症
肿症
头痛
眩晕
咳嗽
伤寒
心腹诸痛
肝火胁痛
息积
痰饮
痢疾
泄泻
秘结症
膈症反胃方
腰痛
不寐症
不能食
食亦症
黄疸症
寒症
霍乱吐泻
鼓胀单腹胀方
疝气
厥证
\part{卷六}
血证方
哮症
五淋癃闭
尤氏治小便不利
赤白浊 节
呕吐哕呃
吞酸
三消
实症诸方
积聚痞气奔豚方
癫狂痫
伤食
伤酒
久服地黄暴脱
室女闭经
祟病
\part{卷七}
虚症诸方
伤寒
虚痨
怔忡(惊悸、健忘同治) 
痿症
遗精
遗溺
房劳伤寒
素盛诸方
素衰诸方
\part{卷八}
妇人科
催生歌


\chapter{1}
\section{1}
\section{2}

\end{document}