% 洗冤集录
% 洗冤集录.tex

\documentclass[12pt,UTF8]{ctexbook}

% 设置纸张信息。
\usepackage[a4paper,twoside]{geometry}
\geometry{
	left=25mm,
	right=25mm,
	bottom=25.4mm,
	bindingoffset=10mm
}

% 设置字体,并解决显示难检字问题。
\xeCJKsetup{AutoFallBack=true}
\setCJKmainfont{SimSun}[BoldFont=SimHei, ItalicFont=KaiTi, FallBack=SimSun-ExtB]

% 目录 chapter 级别加点(.)。
\usepackage{titletoc}
\titlecontents{chapter}[0pt]{\vspace{3mm}\bf\addvspace{2pt}\filright}{\contentspush{\thecontentslabel\hspace{0.8em}}}{}{\titlerule*[8pt]{.}\contentspage}

% 设置 part 和 chapter 标题格式。
\ctexset{
	part/name= {卷之,},
	part/number={\chinese{part}},
	chapter/name={,},
	chapter/number={\chinese{chapter}}
}

% 设置古文原文格式。
\newenvironment{yuanwen}{\bfseries\zihao{4}}

% 设置署名格式。
\newenvironment{shuming}{\hfill\bfseries\zihao{4}}

% 注脚每页重新编号,避免编号过大。
\usepackage[perpage]{footmisc}

\title{\heiti\zihao{0} 洗冤集录}
\author{宋慈}
\date{南宋}

\begin{document}

\maketitle
\tableofcontents

\frontmatter

\chapter{前言}

《洗冤集录》由南宋宋慈撰写,成书于1247年。是当今中外学者公认的世界最早的、系统的法医学专著,比欧洲公认最早的一部系统法医学专著(意大利医生福尔图纳托·费代莱的《医生的报告》)要早350多年。

南宋是中国历史上经济发达、文化繁荣、法律完善、科技进步,但政权动荡不稳、军事冲突不断的朝代。《洗冤集录》成书于这个时代,完成于宋慈之手,是有原因的。

一是法律制度已较为完备。宋代已形成了较为严密的法典,为了保证实体法的有效实施,宋代构建了完备的审判、复核、监督检查机构体系,规定了详细的起诉形式,建立了收集、辨别、运用证据的制度。为防止司法官吏在审判活动中滥用职权、徇私舞弊等造成刑狱冤滥,其从制度层面对审判权进行了限制。随着中央集权的加强,在司法上,中央越来越广泛地行使审判权,提刑官制度就是中央对地方直接干预的司法审判活动。宋真宗景德四年(1007),“复置诸路提点刑狱官”,真宗言:“所虑四方刑狱官吏,未尽得人。一夫受冤,即召灾沴。”并亲自挑选朝官中“性度平和有执守者”为各地提刑官,设立提刑官公允断狱,完善对府县监督的司法制度。

二是法医检验已具备一定的经验基础。法医检验在我国有着悠久的历史,在战国年代已有专门的治狱之官,即根据伤、创、折、断的深浅及大小来确定罪行轻重。1975年出土的云梦秦简中,记载了当时的一些案件和检验情况,其细致程度令人赞叹。同时,我国检验史上也不乏优秀人才和经典案例,如“庄遵疑哭”、“张举烧猪”、“王臻辨葛”等,类似的案例被收录于诸多著作,如《内恕录》、《疑狱集》、《折狱龟鉴》、《棠阴比事》等,供官员断案参考。另外,此时也已在尸检中积累了丰富的技术手段,如红伞验尸、白梅洗敷等,为形成系统的法医学著作提供了基础。

三是作者宋慈的个人经历与性格特点。作为朱熹的再传弟子,宋慈受到朱熹“仁政民本”“视民如伤”思想的影响。他在本书序言中谈到,有时案情是信是疑难以决断,必定会反复深思寻找答案,生怕轻率行事,让死者白白地被翻动检验。可见,宋慈将案件受害者的利益、普通百姓的安危放在了心上。同时,朱熹“格物致知”的思想也深深影响了宋慈。他记录在《洗冤集录》中的检验方法,无不是通过长期实践总结出的宝贵经验。在序言的最后,他还恳请看到此书的官员们如果遇到书中未提及的检验方法或案例,一定要写信告诉他,以便补充完善。正是这种对真理不懈追求的态度,指引着宋慈完成了这部法医学著作。
结合自身的断案经历,宋慈本人对法医检验工作的认识也较前人更为深刻。他认为:“狱事莫重于大辟,大辟莫重于初情,初情莫重于检验。”法医鉴定结论是刑事诉讼案件的重要依据,是案件的重中之重。判罪量刑,没有证据就无从谈起。宋慈所主张的“洗冤”,即是法医鉴定,有两层意思:一是通过鉴定来洗除冤枉;二是洗除误鉴、误判,这也是法医鉴定的本质所在。

《洗冤集录》的作者宋慈,在《宋史》中无传,在宋元时期的重要典籍《文献通考》中,也未曾留下任何记载。在相关的地方志中,对宋慈的记载也极为简略。如明代的《嘉靖建阳县志》,对宋慈只有百余字的记载。直到晚清时,史学家陆心源编撰《宋史翼》以补充《宋史》时,才将宋慈列入了《循吏传》。陆心源为宋慈立传,则基本上是抄录南宋刘克庄所撰《宋经略墓志铭》。
宋慈(1186—1249),字惠父,号自牧,出生于福建建阳的一个书香门第之家,具体日期不详。他的先祖宋咸是宋仁宗天圣二年(1024)进士,曾知邵武军,任广西转运判官等职,曾为《周易》《扬子法言》作注。在这种家庭环境中,宋慈接受了良好的教育。他自幼拜同乡吴雉为师,吴雉乃朱熹高足,也就是说,宋慈是朱熹的再传弟子。20岁时,宋慈赴杭州,拜主持南宋最高学府太学的闽北浦城人真德秀为师。真德秀以朱熹为宗,其格言“律己以廉、抚民以仁、存心以公、莅事以勤”对宋慈产生了终身影响,更是为宋慈一生铺垫了思想基石。宋慈求学多年,在仕途上却并非一帆风顺。直至宋宁宗嘉定十年(1217),32岁的宋慈才考中乙科进士,谋取了浙江鄞县县尉一职。遗憾的是,就在上任前,他的父亲病逝。按照当时的规定,宋慈居家守孝。
大约在嘉定十七年(1224),宋慈才调任信丰县主簿,不久赣州知州郑性之招他为幕僚。任期结束以后,江西南部发生了三峒叛乱,以南雄、赣州、南安三地为中心的数百里的地区都被攻占,石门寨和高平寨成为据点。江西提刑叶宰创建节制司,派宋慈平叛。宋慈立即奔赴山前,先赈济六堡的饥民,让他们不参与三峒叛乱。继而率兵攻占石门寨,俘虏敌方首脑。宋慈向叶宰报告,并率领义兵积极战斗,最终攻陷高平寨,捉到叛军首领。三峒之乱平定后,幕府向上请功,这成为宋慈正式进入官场的契机。
当时福建地区的汀州、南剑州、邵武军出现叛军,陈为招捕使,真德秀给陈写了一封信,推荐宋慈。于是,陈上奏后,传命宋慈与李华一同商议军事。宋慈率军从竹洲出发行进了三百多里,如期到达老虎寨会师。之后直取招贤、招德等地,收服邵武军,叛军首领无一漏网。汀州士兵囚禁知州陈孝严,据城抵抗。宋慈等人赶到汀州,暗中写下招安文书。他与李华一同坐堂,命令州中的士兵前来支取赏赐,士兵们都带着兵器前来。宋慈面色如常,下令杀了七个参与叛乱的士兵头目,同时出示招安文书,宽恕了余党,众人信服。陈认为宋慈有奇才,闽中平叛乱有功,推荐他担任长汀知县。
自此,宋慈历任福建长汀知县(1231)、邵武军通判(1237)、南剑州通判(1238)等职。嘉熙三年(1239)升任提点广东刑狱,次年移任提点江西刑狱兼知赣州。淳祐五年(1245)转任常州知州。淳祐七年(1247),提点湖南刑狱并兼大使行府参议官。后任宝谟阁直学士,奉命巡回四路。
淳祐八年(1248)冬,宋慈升任焕章阁直学士、知广州、兼任广东经略安抚使。执政期间顾全大局而宽恕末节,恩威并用。开置府衙两个月后,一日宋慈忽然感到四肢不协,但仍然亲自处理政务。州立学校举行祭祀孔丘的仪式,属下请求宋慈让其他官员代为行礼,宋慈却坚持亲自前往。之后,宋慈身体愈发不适,于淳祐九年三月七日(1249年4月21日)在任内去世,享年64岁。第二年,宋慈归葬于福建建阳崇雒里昌茂村。朝廷追封宋慈为朝议大夫,宋理宗赵昀在他的墓门上亲自题字以旌表他的功绩,并评价他是“分忧中外之臣”。
宋慈长期从事刑狱断案工作,对于决狱理刑的态度十分严肃认真。例如,原先广东官吏多不奉行法令,“有留狱数年未详覆者”。担任提点广东刑狱后,宋慈定下办案规约,责令官员限期执行,在八个月内审理了二百多名囚犯,为其中被陷害和冤屈的犯人平冤昭雪。由于宋慈“听讼清明,决事刚果”,“以民命为重”,因此,在民众中间赢得了清官的名声。
在处理狱讼的过程中,宋慈特别重视现场勘验。他对当时传世的尸伤检验著作加以总结,结合自己丰富的检验经验,写就了一部完整而又系统的法医学名著—《洗冤集录》。该书刊行两年后,宋慈病逝,因此也可以说,《洗冤集录》是宋慈毕其一生精力完成的巨著。
二《洗冤集录》的宋刊本迄今尚未发现,现存最早的版本为元刻本《宋提刑洗冤集录》。内容自“条令”起,至“验状说”终,共五卷、五十三条。
书中的主要内容包括:宋朝关于检验的条令、验尸的方法及注意事项、法医现场学、尸体现象、生前伤与死后伤区别、机械性窒息、机械性损伤、碎尸检验、交通损伤、狱中死亡、火烧死和汤泼死、中毒死、病死、针灸致死、尸体发掘、救死方以及法医妇产科学、法医昆虫学等。可以说,其涉及了现代法医学的大部分领域,不仅记载了案例和检验方法,而且全面、系统地阐述了相关检验原理和经验,是一部早期的、系统的法医学著作。《洗冤集录》的成就主要包括以下几个方面。
其一,对一些主要的尸体现象,已有较为明确的认识。宋慈说:“凡死人项后、背上、两肋、后腰、腿内、两臂上、两腿后、两曲䐐、两脚肚子上下有微赤色,验是本人身死后一向仰卧停泊,血脉坠下,致有此微赤色,即不是别致他故身死。”这里所称“血坠”,即指的是现代法医学中的“尸斑”。
其二,对自缢、勒死、溺死、外物压塞口鼻死四种机械性窒息进行了较详细的描述。关于缢死:自缢伤痕“脑后分八字,索子不交”“用细紧麻绳、草索在高处自缢,悬头顿身致死,则痕迹深;若用全幅勒帛及白练、项帕等物,又在低处,则痕迹浅”。又指出:“若勒喉上即口闭,牙关紧,舌抵齿不出;若勒喉下则口开,舌尖出齿门二分至三分。”“口吻两角及胸前有吐涎沫。”关于勒死,指出它与缢死的不同之处在于“项下绳索交过”“多是于项后当正或偏左右系定,须有系不尽垂头处”。关于溺死,强调“腹肚胀,拍着响”“手脚爪缝有沙泥”“口鼻内有水沫”等。
其三,探讨了机械性损伤,将其分为常见的“手足他物伤”与“刃伤”两大类。他物类似现在所说的钝器。宋慈详细介绍了皮下出血的形状、大小与凶器性状的关系以及根据损伤位置判断凶手与被害者的位置关系等。
其四,书中对中暑死、冻死、汤泼死与烧死等高低温所致的死亡征象做了描述,对现场尸体检查的注意事项做了系统的归纳。
其五,书中明确提出了动物对尸体的破坏及其与生前伤的鉴别。“凡人死后被虫、鼠伤,即皮破无血,破处周围有虫鼠啮痕,纵迹有皮肉不齐去处。若狗咬,则痕迹粗大。”《洗冤集录》在检验依据方面做了比较系统的研究,这是宋慈在法律依据和检验原理方面探索法医实践的体现。这一点正是宋慈超越了其他人,使《洗冤集录》区别于历史上其他检验书籍的地方,也是《洗冤集录》被后世学者重视、公认、流行并传播到世界各地,成为传世之书的原因所在。其中主要体现了以下6个方面的思维与方法。
1.律。验尸前心中要有一个准绳,这就是以法律条文为依据。哪些尸体该验,具体该怎么验,检验官要精通法律,对各种尸体的检验及其法律规定了如指掌。
2.问。针对不同案件、不同尸体、不同场所、不同人群,“问”的内容和流程不同。“问”官员、老百姓、报案人,“问之又问”“审之又审”,问得“一清二楚”,审后“明明白白”。《洗冤集录》里非常详细地记载了应“问”的具体问题,是书中的精彩之处。
3.看。通过尸体上的种种痕迹,宋慈仿佛能够亲眼看到上吊者拿着绳索扣好绳结上吊,看到凶手把人绞死后伪装上吊,看到死者被谋杀后推入水中。“看”与“问”结合,恢复事件的本来面貌,从而得出结论,称为“事件重建”,形成了法医现场学。具体来说,就是了解案情、案由、经过,结合现场、痕迹遗留、尸体位置、损伤情况综合分析,看出破绽、查出真伪,使案件水落石出。
4.借。宋慈善于借助不同学科、不同门类、不同手段的科学原理和研究成果进行尸体检验,如借助苍蝇嗜血的习性破获杀人案,利用了法医昆虫学的知识;如拿着油伞验尸,实质上是利用光学原理使尸体上的伤痕清晰可辨,这与现代法医学中用紫外线照射检查伤痕的原理一致。
5.鉴。鉴定就是要像镜子一样反映真实,找出不合理的地方。不合理的地方包括:与法不符,如应验而不验、不亲临检视、定而不当等,这涉及“法医法学”的问题;与情理不符,如自杀上吊颈部却发现勒痕或扼痕等,提出了“法医学死后造作或伪装”的课题;与损伤部位不符,如创口方向、起刀、收刀等损伤有自身的规律,也就是现在所说的“法医损伤学”内容;与窒息征象不符,书中介绍了吊死、勒死出现面部淤肿、胀红和眼睑出血等窒息征象及缢痕、勒沟、扼痕等的特征,涉及“法医学机械性窒息”的内容;与尸体腐败现象不符,书中不但介绍了一般尸体的腐败规律,也记载了与常规情形不同的各类保存型尸体,以及消瘦、体弱、年老等尸体不易腐败的现象,涉及“法医学尸体现象”的内容;与正常人不符,书中介绍了鸡胸、纹身、刺字等,涉及早期“法医人类学”的内容;利用不同疾病和不同中毒死亡的症状来研究猝死和毒死,涉及早期“法医死亡学和中毒学”的内容。
6.理。即指法医鉴定结论要有依据,也就是现在所说的证据意识。如生前烧死者由于呼吸作用会从口腔和鼻孔吸入大量烟灰,而死后被焚尸者口、鼻腔内没有烟灰;生前勒死、吊死应留下相应索沟或扼痕,有“面赤淤肿”的窒息征象,而死后伪装吊死者则没有上述征象;生前被钝器打击致皮下出血会出现红斑、肿胀,触硬;生前刀伤有出血、皮肉卷缩的现象,而死后割伤则肉白、皮肉不紧缩等。这是《洗冤集录》的精华所在。
当然,限于当时的科学技术水平、检验手段以及对各种死亡认识的不足,特别是不能进行尸体解剖,从现代法医学角度来看,《洗冤集录》的缺陷也是很明显的,但不能因此否认其在法医学上的历史地位和对现代法医学的深远影响。

《洗冤集录》在初次刊行后的几百年间屡有修订、增补,各种版本的卷数、小节标题甚至书名亦有不同。《洗冤集录》初刊于南宋淳祐丁未(1247),由宋慈自刻。但该版本与宋代其他版本现皆不存。当时宋慈正值湖南提刑充大使行府参议官任上,故后世翻刻时又将此书题为《宋提刑洗冤集录》。
现存最早的《宋提刑洗冤集录》版本是元大德年间建阳余氏勤有堂刊本,现藏于北京大学图书馆。该本共五卷,半叶十六行,行二十七字,黑口,卷端题“朝散大夫新除直秘阁湖南提刑充大使行府参议官宋慈惠父编”,有宋慈自序。该本成为日后众多版本的祖本。2005年北京图书馆出版社影印出版了建阳余氏勤有堂刊本《宋提刑洗冤集录》。
明初沿用元大德年间建阳余氏勤有堂《洗冤集录》刊本,共五卷五十三篇,现存南京图书馆。清兰陵孙星衍于嘉庆十二年(1807)依据元大德年间建阳余氏勤有堂本校刊,由顾广圻复校刊印,后又被收入《岱南阁丛书》中,称《岱南阁丛书》本,卷首有牌记“兰陵孙氏元椠重刊”字样。该本为五卷五十三篇,存宋慈作原序一篇,是目前流传最广的本子。商务印书馆发行的王云五《丛书集成初编》中收录的版本便是据《岱南阁丛书》本排印。
清嘉庆十七年(1812),藏书家吴鼒将宋慈《洗冤集录》、赵逸斋《平冤录》和王与《无冤录》三书汇集,并刻印成《宋元检验三录》。因当时孙氏与吴氏的《洗冤集录》重刊本都由顾广圻所刻,故可推测吴鼒重刻本是按孙星衍重刻所用元刊本覆刻。清陆心源曾藏宋慈《洗冤集录》影宋钞本。清代藏书家许梿于咸丰四年(1854)编刻《洗冤录详义》时曾用过此影宋钞本作校本,并校录了宋慈《洗冤集录》。
此外,《洗冤集录》还有明初杨士奇《文渊阁书目》(卷十四著录《洗冤集录》)、明代建阳闽潭城书林萃庆堂刻本《附刻宋提刑洗冤录》、明万历胡文焕覆刻本(收录在胡文焕万历年间所刻《格致丛书》)、万历《建阳县志·艺文志》“梓书门”著录明代建阳书坊《洗冤集录》刻本、附刊于《御制新颂大明律例注释招拟折狱指南》(其内容以宋慈《洗冤集录》为主,结合王与《无冤录》内容而成)、清康熙三十三年(1694)大清律例馆组织修订的《洗冤集录》(定本为朝廷正式颁发的官书《律例馆校正洗冤录》)、清康熙三十四年(1695)于琨辑注《祥刑要览》四卷(其中卷二、卷三为《洗冤集录》上下二册,以元刊《洗冤集录》为主)、清《四库全书》二卷本《洗冤集录》(从明《永乐大典》辑出)、咸丰四年(1854)许梿编刻《洗冤录详义》四卷(以《洗冤集录》为蓝本)等诸多版本。
在众多版本中,清代孙星衍依元刻本校刊的《宋提刑洗冤集录》、许梿的《洗冤录详义》等影响很大。本书采用的底本,即清嘉庆十二年(1807)孙星衍依元刻本校刊的《宋提刑洗冤集录》,其被认为是宋本《洗冤集录》的重刊。

作为一部系统的法医学著作,《洗冤集录》在世界范围内也有相当大的影响。元代的王与曾以《洗冤集录》为蓝本,增补编撰而成《无冤录》,1438年,高丽使臣李朝成将《无冤录》带回朝鲜,加注刊行,取名《新注无冤录》。在很长一段时间里,此书一直是朝鲜法医检验领域的标准著作。1736年,日本日源尚久将《新注无冤录》翻译成日文,其在短短的10年间6次再版,影响极大。
经由朝鲜、日本或越南,欧洲的一些国家也先后接触到《洗冤集录》,并将其翻译出版。如1779年,法国人将《洗冤集录》节译于巴黎的《中国历史、科学、艺术》论丛。1863年,荷兰译本刊出。1882年,法国医生马丁在《远东评论》发表了《洗冤集录》提要论文。1908年法文本正式出版。同年,德国人霍夫曼又将法文本翻译成德文出版。《洗冤集录》在欧洲的传播不只经由上述途径,还有来华学者直接传到欧洲的。如1873年英国剑桥大学东方文化教授嘉尔斯来到宁波,并拜见宁波官员。当他来到官府后,看见官员升堂时案桌上摆着一本书,被派到现场验尸时也带着这本书,随时翻阅参考,就问这是什么书。官员告诉他,这本书叫《洗冤集录》,于是,他以极大的兴趣着手翻译,并将译名定为《洗冤录—验尸官教程》。完成后分期刊于英国《皇家医学杂志》,并有单行本。据不完全统计,《洗冤集录》传至邻邦及欧、美,各种译本达21种之多。其中,朝鲜3种、日本8种、越南1种、荷兰1种、德国2种、法国3种、英国1种、美国1种、俄罗斯(评介)1种。
鉴于《洗冤集录》对法医学的贡献,宋慈被公认为世界法医学奠基人。《洗冤集录》问世后,“官司检验奉为金科玉律”,“入官佐幕无不肄习”,凡“士君子学古入官,听讼决狱,皆奉《洗冤集录》为圭臬”。《洗冤集录》成为法医检验的指南,是司法官员手中必备的检验书籍。清代许梿在《洗冤录详义序》中这样表述:“检验之有《洗冤集录》,犹谳狱之有律例也。”西方医学史家对《洗冤集录》也有很高的评价。1956年,苏联教授契利法珂夫《法医学史及法医检验》一书,把宋慈的画像印在了卷首,称他为“法医学奠基人”。1981年美国密歇根大学中国研究中心麦克奈特翻译出版了《洗除错误:十三世纪的中国法医学》,该书序言中明确指出:“《洗冤集录》被认为是世界上现存最早的法医学著作,完成于南宋时期的1247年。这部作品,领先于在文艺复兴时期出现的欧洲法医学著作,如意大利人福尔图纳托·费代莱和保罗·扎基亚的著作。前者于1602年编著的《医生的报告》,被称为是欧洲第一部系统的法医学著作;而后者在1635年发表的《法医学问题》中第一次提出了‘法医学’这一术语。”
值得一提的是,英国著名的历史学家李约瑟(1900—1995)在他的《中国科学技术史》(第6卷第6分册医学)中把宋慈的《洗冤集录》称为“科学革命之前最伟大的法医学著作”,并专门介绍了中国的法医学:①宋慈和他的时代;②《洗冤集录》;③宋慈之前的中国法医学;④秦简;⑤早期的证据;⑥元明时期的法医学发展;⑦清代的法医学发展;⑧与医学有关的有趣问题;⑨与欧洲的一些比较;附录,《洗冤集录》的版本及译本。该书把“宋慈之前的中国法医学”和宋慈《洗冤集录》出现以后的中国法医学与欧洲法医学进行比较,认为宋慈及其《洗冤集录》对中国乃至世界法医学做出了开创性的贡献。2017年笔者参与伯克哈德·马代亚编著的《世界法医学史》一书第四章《中国法医学史》的撰写,其中专门介绍了世界法医学奠基人宋慈及其《洗冤集录》。
由于《洗冤集录》成书于十三世纪,加之涉及许多生僻的古代医学、法律术语,相当艰涩难懂,因此在本次整理过程中,我们尽量在题解、注释中加入了一些宋朝司法检验制度、检验方法的科学解释等内容,以提高本书的可读性。
本书的题解部分由陈新山完成,注释和译文部分由黄瑞亭完成。感谢中华书局领导的信任与支持,感谢李丽雅、周梓翔二位编辑对本书倾注的心血,使得本书更具严肃性、科学性和可读性。
切望广大读者及专家学者不吝金玉,惠予指正。

《洗冤集录》六卷,是中国古代第一部比较系统地总结尸体检查经验的法医学名著,也是世界上最早的一部较完整的法医学专著。本书自南宋后成为历代官府尸伤检验的蓝本,曾定为宋、元、明、清各代刑事检验的准则,在中国古代司法实践中,起过重大作用。本书曾被译成多种外国文字,深受世界各国重视,在世界法医学史上占有十分重要的地位。此次整理,以上海图书馆藏孤本元刊大字本为底本,不仅对原书进行了校勘,对各种术语、名词作了科学的注释、客观的评价,而且用白话作了今译,从而有助于读者阅读、理解这部在中国科学史上有一定价值的著作。

《洗冤集录》是中国古代法医学著作。南宋宋慈著,刊于宋淳祐七年(1247),是世界上现存第一部系统的法医学专著,比公元1602年意大利人福寞乃·法特里所写的法医专著要早355年。该书的最早版本,当属宋淳祐丁未宋慈于湖南宪治的自刻本,继又奉旨颁行天下,但均已不传。现存最早的版本是元刻本《宋提刑洗冤集录》;兰陵孙星衍元椠重刊本或称《岱南阁丛书》本;此外又有从《永乐大典》中辑出的 2 卷本;清代多种刻本与元刻本完全相同。还有 1937 年商务印书馆的《丛书集成(初编)》本。现较通行的有:法律出版社 1958 年的《洗冤集录点校本》;群众出版社 1980 年出版杨奉琨校译本《洗冤录校译》;上海科学技术出版社 1981 年出版贾静涛点校本。

《洗冤集录》六卷,是中国古代第一部比较系统地总结尸体检查经验的法医学名著,也是世界上最早的一部较完整的法医学专著。本书自南宋后成为历代官府尸伤检验的蓝本,曾定为宋、元、明、清各代刑事检验的准则,在中国古代司法实践中,起过重大作用。本书曾被译成多种外国文字,深受世界各国重视,在世界法医学史上占有十分重要的地位。全书从生理、药理、诊断、治疗、预防、急救、检验等方面进行论述,有不少内容至今仍可借鉴,如在治疗骨折用夹板固定伤断部位,包扎创伤田活剥鸡皮作绷带,它富于弹性,从而能缩短创伤愈合时间。《洗冤集录》汇集了不少前人的成果,如《内恕录》、《折狱龟鉴》、《棠阴比事》、《检验法》等(有的已经失传)。不仅是我国,也是世界法医学史上最优秀的文化遗产之一。
本书 5 卷 53 目,约 7 万字。前有作者自序。卷 1 包括条令、检覆总说、疑难杂说等目;卷 2—卷 5 分列各种尸伤的检验区别等项。《条令》目下辑有宋代历年公布的条令 29 则,都是对检验官员规定的纪律和注意事项。其余52 目,排列分卷不甚有序,各目下内容亦有穿插交错,但细加缕析,其内容大致可分三方面:1、检验官员应有的态度和原则;2、各种尸伤的检验和区分方法;3、保辜和各种救急处理。
本书对尸体现象、窒息、损伤、现场检查、尸体检查等方面都有较科学的观察和归纳,有的达到相当精细的程度。主要成就有:尸斑的发生与分布;腐败的表现和影响条件;尸体现象与死后经过时间的关系;棺内分娩的发现;缢死的绳套分类;缢沟的特征及影响的条件;自缢、勒死与死后假作自缢的鉴别;溺死与外物压塞口鼻而死的尸体所见;窒息性玫瑰齿的发现;骨折的生前死后鉴别;各种刃伤的损伤特征;生前死后及自杀、他杀的鉴别;致命伤的确定;焚死与焚尸的区别;各种死亡情况下的现场勘验方法等。第 52 目“救死方”下,收集了自缢、水溺、暍死、冻死、杀伤及胎动等抢救办法及单方数十则,都是通过经验证明是行之有效的。

 宋慈(1186—1249),字惠父,南宋福建建阳人,法医学家。少受业于同邑“考亭高第”吴稚门下,受朱熹的考亭学派(又称闽学)影响很深。南宋宁宗嘉定十年(1217)进士,历任主簿、县令、通判兼摄郡事。嘉熙六年(1239),升提点广东刑狱,后又移任江西提点刑狱兼知赣州。淳祐年间,除直秘阁,提点湖南刑狱并兼大使行府参议官,协助湖南安抚大使陈 处理大使行府一切军政要务。宋慈居官清廉刚正,体恤民情,不畏权豪,决事果断。20 余年官宦生涯中,大部分时间与刑狱方面有关,深知“狱事莫重于大辟,大辟莫重于初情,初情莫重于检验”,认为检验乃是整个案件“死生出入之权舆,直枉屈伸之机括”,因而对于狱案总是审之又审,“不敢生一毫慢易心”。发现吏仵奸巧欺侮,则亟予驳正;若疑信未决,必反复深思,决不率然而行。认真审慎的实践,得出一条重要经验,“狱情之失,多起于发端之差;定验之误,皆原于历试之浅”,于是博采近世所传诸书如《内恕录》、《折狱龟鉴》等数家,荟萃厘正,参以自己的实际经验,总为一编,名曰《洗冤集录》,刊于湖南宪治,供省内检验官吏参考,以指导狱事的检验,达到“洗冤泽物”的目的。宋慈死后,理宗为表彰他的功绩,曾为其御书墓门。其挚友刘克庄(后村)在墓志铭中赞他“奉使四路,皆司臬事,听讼清明,决事刚果,抚善良甚恩,临豪滑甚威,属部官吏以至穷闾委巷,深山幽谷之民,咸若有一宋提刑之临其前。”

\chapter{洗冤集录序}

\begin{yuanwen}
狱事\footnote{刑狱之事,与刑事案件的判决相关的事。}莫重于大辟\footnote{古代五刑之一,指死刑。},大辟莫重于初情\footnote{原本的犯罪事实。},初情莫重于检验\footnote{指检验死伤。我国古代实行官验制度,秦代令史负责检验,自汉代至唐代由县令负责检验,宋代由州差司理参军、县差县尉负责检验。官府检验时令仵作、行人协助。}。盖死生出入\footnote{这里指出罪、入罪。“出罪”指把有罪判为无罪或重罪判为轻罪。“入罪”则指把无罪判为有罪或轻罪判为重罪。}之权舆\footnote{起始。},幽枉\footnote{冤屈。}屈伸之机括\footnote{弩上发射箭矢的机件。比喻事物的关键。},于是乎决。法中所以通差令佐\footnote{县令及其佐官。依下文“条令”,凡检验尸体,县一级应派县尉,县尉不在,依次派主簿、县丞。这些官员都不在,则县令亲往。}、理掾\footnote{宋朝司理参军的别称,是州一级掌管狱讼的官员。掾,yuàn 。}者,谨之至也。
\end{yuanwen}

刑事案件没有比死刑判决更为重要的,死刑判决没有比查清原本的犯罪事实更为重要的,查清原本的犯罪事实没有比检验更为重要的。因为罪犯的生与死、罪行轻与重的最初依据,蒙冤与昭雪的关键,都由此决定。法律中规定派县令及其佐官、司理参军处理检验之事,是十分谨慎的。

\begin{yuanwen}

\end{yuanwen}

年来州县,悉以委之初官,付之右选,更历未深,骤然尝试,重以仵作之欺伪,吏胥之奸巧,虚幻变化,茫不可诘。纵有敏者,一心两目,亦无所用其智,而况遥望而弗亲,掩鼻而不屑者哉!慈四叨臬寄,他无寸长,独于狱案,审之又审,不敢萌一毫慢易心。若灼然知其为欺,则亟与驳下;或疑信未决,必反复深思,惟恐率然而行,死者虚被涝漉。每念狱情之失,多起于发端之差;定验之误,皆原于历试之浅。遂博采近世所传诸书,自《内恕录》以下,凡数家,会而稡之,厘而正之,增以己见,总为一编,名曰《洗冤集录》,刊于湖南宪治,示我同寅,使得参验互考,如医师讨论古法,脉络表里先已洞澈,一旦按此以施针砭,发无不中。则其洗冤泽物,当与起死回生同一功用矣。淳祐丁末嘉平节前十日,朝散大夫、新除直秘阁、湖南提刑充大使行府参议官宋慈惠父序。
贤士大夫或有得于见闻及亲所历涉,出于此集之外者,切望片纸录赐,以广未备。慈拜禀


在所有案件的审理中,最重要的就是死刑的判决。而要对犯人判处死刑,最要紧的就是要查明案情的线索及实情,而要弄清案子的线索和实情,首要的就是要依靠检验勘查的手段。因为人犯是生是死,断案是曲是直,冤屈是伸张还是铸成,全都取决于根据检验勘查而下的结论。这也就是法律中规定的州县审理案情的所有刑事官员必须亲身参与检验勘查的道理之所在,一定要无比谨慎小心才行啊!
近年来各地方衙门,却把如此重大的事项交给一些新任官员或是武官去办理,这些官员没有多少经验,便骤然接手案子,如果再有勘验人员从中欺瞒,衙门中的低级办事人员居中作奸捣鬼,那么案情的扑朔迷离,仅仅靠审问是很难弄清楚的。这中间即使有一些干练的官员,但仅凭着一个脑袋两只眼,也很难把他的聪明才智发挥出来,何况那些远远望着非亲非故的尸体不肯近前、对血腥恶臭避之犹恐不及的官吏们呢!我宋慈这个人四任执法官,别的本事没有,惟独在断案上非常认真,必要审理了再审理,不敢有一丝一毫的马虎。如果发现案情中存在欺诈行为,必然厉言驳斥矫正,决不留情;如果有谜团难以解开,也一定要反复思考找出答案,生怕独断专行、让死者死不瞑目。
我常常在想,案狱之所以会出现误判,很多都是缘于细微之处出现的偏差;而勘查验证失误,则是因为办案马虎、经验不足造成。有鉴于此,我广采博引近世流传的法医学著作,从《内恕录》一路下来共有好几种,认真消化,汲其精华,去其谬误,再加上自己长期司法实践的经验积累,编成一本书,起名《洗冤集录》,在我湖南任上刊印出来,给我的同僚们研读,以便他们在审理案子时参照。这就如同医生学习古代医书处方一样,在诊治病人之前,事先就能够厘清脉络,做到有章可循,再对症施药,则没有不见效的。而就审案来说,其所起的洗清冤屈、还事实于本来面目的结果,与医生治病救人、起死回生的道理也是完全相同的。

\mainmatter

% 增加空行
~\\

\part{}

\chapter{条令}

诸尸应验而不验;初复同。或受差过两时不发;遇夜不计,下条准此;或不亲临视;或不定要害致死之因;或定而不当,谓以非理死为病死,因头伤为胁伤之类。各以违制论。即凭验状致罪已出入者,不在自首觉举之例。其事状难明定而失当者,杖一百。吏人、行人一等科罪。

诸被差验复,非系经隔日久而辄称尸坏不验者,坐以应验不验之罪。淳祐详定。
诸验尸,报到过两时不请官者;请官违法或受请违法而不言;或牒至应受而不受;或初复检官吏、行人相见及漏露所验事状者,各杖一百。若验讫,不当日内申所属者,准此。
诸县承他处官司请官验尸,有官可那而称阙;若阙官而不具事因申牒;或探伺牒至而托故在假被免者,各以违制论。
诸行人因验尸受财,依公人法。
诸检复之类应差官者,差无亲嫌干碍之人。
诸命官所任处,有任满赏者,不得差出,应副检验尸者听差。
诸验尸,州差司理参军,本院囚别差官,或止有司理一院,准此。县差尉,县尉阙即以次差簿、丞,县丞不得出本县界。监当官皆缺者,县令前去。若过十里或验本县囚,牒最近县,其郭下县皆申州。应复验者,并于差初验日,先次申牒差官。应牒最近县而百里内无县者,听就近牒巡检或都巡检。内复检应止牒本县官而独员者,准此。谓非见出巡捕者。
诸监当官出城验尸者,县差手力、伍人当直。
诸死人未死前,无缌麻以上亲在死所,若禁囚责出十日内及部送者,同。并差官验尸。人力、女使经取口词者,差公人。囚及非理致死者,仍复验。验复讫,即为收瘗。仍差人监视;亲戚收瘗者,付之。若知有亲戚在他所者,仍报知。
诸尸应复验者,在州申州;在县,于受牒时牒尸所最近县。状牒内各不得具致死之因。相去百里以上而远于本县者,止牒本县官。独员即牒他县。
诸请官验尸者,不得越黄河、江、湖,江河谓无桥梁,湖谓水涨不可度者。及牒独员县。郭下县听牒,牒至,即申州差官前去。
诸验尸,应牒近县而牒远县者,牒至亦受。验毕,申所属。
诸尸应牒邻县验复,而合请官在别县,若百里外,或在病假不妨本职非。无官可那者,受牒县当日具事因在假者具日时。保明,申本州及提点刑狱司,并报元牒官司,仍牒以次县。
诸初、复检尸格目,提点刑狱司依式印造。每副初、复各三纸,以《千字文》为号凿定,给下州县。遇检验,即以三纸先从州县填讫,付被差官。候检验讫,从实填写。一申州县,一付被害之家,无,即缴回本司。一具日时字号入急递,径申本司点检。遇有第三次后检验,准此。
诸因病死谓非在囚禁及部送者。应验尸,而同居缌麻以上亲,或异居大功以上亲至死所而愿免者,听。若僧道有法眷,童行有本师未死前在死所,而寺观主首保明各无他故者,亦免。其僧道虽无法眷,但有主首或徒众保明者,准此。
诸命官因病亡,谓非在禁及部送者。若经责口词,或因卒病,而所居处有寺观主首,或店户及邻居并地分合干人,保明无他故者,官司审察,听免检验。
诸县令、丞、簿虽应差出,须当留一员在县。非时俱阙,州郡差官权。
诸称违制论者,不以失论。《刑统?制》曰:“谓奉制有所施行而违者,徒二年,若非故违而失错旨意者,杖一百”。
诸监临主司受财枉法二十匹,无禄者二十五匹,绞。若罪至流及不枉法赃五十匹,配本城。
诸以毒物自服,或与人服而诬告人罪,不至死者,配千里。若服毒人已死,而知情诬告人者,并许人捕捉,赏钱五十贯。
诸缌麻以上亲,因病死辄以他故诬人者,依诬告法,谓言殴死之类,致官司信凭已经检验者。不以荫论,仍不在引虚减等之例。即缌麻以上亲,自相诬告,及人力女使病死,其亲辄以他故诬告主家者,准此。尊长诬告卑幼,荫赎减等自依本法。
诸有诈病及死伤受使检验不实者,各依所欺减一等。若实病死及伤不以实验者,以“故入人罪”论。《刑统?议》曰:“上条诈疾病者杖一百;检验不实同诈妄,减一等杖九十。”
诸尸虽经验而系妄指他尸告论,致官司信凭推鞠,依诬告法。即亲属至死所妄认者,杖八十。被诬人在禁致死者,加三等。若官司妄勘者,依“入人罪法”。
《刑统?疏》:“以‘他物’殴人者,杖六十。见血为伤。非手足者其余皆为他物,即兵不用刃,亦是。”
《申明刑统》:“以靴鞋踢人伤,从官司验定:坚硬即从他物,若不坚硬,即难作他物例。”
诸保辜者,手足限十日,他物殴伤人者二十日,以刃及汤火三十日折日,折跌肢体及破骨者三十日。限内死者,各依杀人论。诸啮人者,依他物法。限内堕胎者,堕后别保三十日,仍通本殴伤限,不得过五十日。其在限外及虽在限内以他故死者,各依本殴伤法。他故,谓别增余患而死。假殴人头伤,风从头疮而入、因风致死之类,仍依杀人论。若不因头疮得风而死,是为他故,各依本殴伤法。
乾道六年,尚书省此状:“州县检验之官,并差文官,如有阙官去处,复检官方差右选。○本所看详:“检验之官自合依法差文臣。如边远□小县,委的阙文臣处,复检官权差识字武臣。今声说照用。”
嘉定十六年二月十八日
敕:“臣僚奏:‘检验不定要害致命之因,法至严矣。而检验失实,则为觉举,遂以苟免。欲望睿旨下刑部看详,颁示遵用。’刑寺长贰详议:‘检验不当,觉举自有见行条法,今检验不实,则乃为觉举,遂以苟免。今看详:命官检验不实或失当,不许用觉举原免。余并依旧法施行。奉圣旨依’。”



(一)凡有尸体应当检验而不检验的(初验、覆验相同);或受到差遣超过两个时辰不出发的(碰到夜间不算,下条同此);或不亲到现场验看的;或不验定出要害致死原因的;或验定得不恰当的(指把非正常死定为病死,由于头伤致死而定为胁伤致死之类的情况),各按「违制罪」论处。若凭验单判罪已构成「出入」的,不属于「自首觉举」的范围。由于情况难明,定得不恰当的,处杖刑一百。吏役人员和仵作行人同等论罪。
(二)凡受差进行检验或覆验,不是经隔时间太久,就称说尸体已坏不验的,处以「应验不验」的罪(淳佑时审查修定)。
(三)凡验尸,报到后过两个时辰不请官的;请官违法或受请违法而不言的;或请官验尸的公文到来应当接受而不接受的;或初验和覆验的官员、吏役、仵作行人相见及透露所检验的情况的,各处杖刑一百(如果检验完毕,不当天内申报所属上级的,同此)。
(四)凡县受到其它地方官府请官验尸,有官可以挪出却称说缺官的;如果缺官,却不备文申明情况的;或探知有请官公文到来,却假托正在请假期中避免应差的,各按「违制罪」论处。
(五)凡仵作行人因验尸受人财物的,按照「公人法」办理。
(六)凡检验、覆验需要差官的,差同本案没有亲故嫌怨关系的人。
(七)凡命官所任职处,有任期届满受赏的,不得差出,由应付验尸的官员听候差遣。
(八)凡验尸,州差司理参军(本院囚犯,另外差官,或只有司理院一院的,同此)。县差县尉。县尉缺,就依次差主簿、县丞(县丞不得出本县界)。尉、簿、丞等监当官都缺的,县令亲自前去。如果超过县界十里,或是检验本县的囚犯,发公文请距离现场最近的县差官。州城脚下的县,都报请州差官。应当覆验的案件,都要在差初验官的同时,先备文申请有关单位差官。应当发公文给最近的县请官,但在百里之内没有县的,听凭就近发文请巡检或都巡检(其中覆验应当只请本县官,但属于独员的,同此,并指不是现正外出巡捕的)。
(九)凡尉、簿、丞等监当官出城验尸的,县差手力五人跟班听差。
(十)凡死人在未死之前,没有缌麻服以上的亲属在场的(如在监牢中的囚犯受责打后已过十天及正在押送中的,也相同),都差官验尸(男佣女仆,经过录取口词的,差公人)。囚犯及非正常死亡,检验后还必须覆验。覆验完毕,即代为收埋(须差人监视,死者亲戚收埋的,可交付之)。如果知道死者有亲戚在其它地方的,仍要告知。
(十一)凡尸体应当进行覆验的,属于州直接管辖范围的申报州派官,属于县管的,在接受初验公文的同时发公文到尸体所在地最近的县派官(公文中,都不得写出致死的原因)。最近的县相距百里以外而远于本县的,只发文请本县的官覆验(本县为独员者即发公文到其它县)。
(十二)凡请官验尸的,不得越过黄河、江、湖(江河是说没有桥梁可通的,湖是说水涨不可渡过的)及发公文到独员县(州城脚下的独员县,听凭发去公文,文到后即立刻申报州府,差官前去)。
(十三)凡验尸,应当发公文到靠近的县却发到远县的,文到也要接受,派官检验完毕申报所属上级。
(十四)凡遇尸体应当发文到邻近县请官检验或覆检,而所要请的官却在别县,如果远在百里之外,或正在病假之中(不妨碍执行所任职务的不算),没有其它官员可以挪派的,接到公文的县要当天写明事由(在假期中的写明假期的起止日时)负责申报本州及提点刑狱司,并告知原发文单位,再发文到按顺次应派官的县派官前往。
(十五)凡初验、覆验的验尸表格,由提点刑狱司按照规定格式印制,每副初检、覆验各三份,用千字文作为排号编定,下发到各州县。遇有检验,就用三份表格先从州县填写有关事项毕,交付被差验尸的官。等检验完毕,再由验官把检验情况据实填写。一份呈报州县,一份交付被害人家属(没有家属的即缴回本司),一份写明日时字号交由邮传站的快递,直接申报本司审查(遇有第三次以后覆验,同此)。
(十六)凡由于病死(指不是正在囚禁中及押送中的)应当验尸,但死者的同居缌麻服以上亲属,或分居大功服以上亲属到死所而愿意免验的,听从免验。如果和尚道士有眷属,小道士小和尚有本师,未死前在死所,而寺观的主持人负责证明各无其它问题的,也予以免验。有的和尚道士虽无眷属,但有寺观主持人或僧道群众负责证明没有问题的,同此。
(十七)凡命官因病死亡(指不是在监禁及押送中的),如已经录取口词,或因突然病死,而所住的地方有寺观主持人,或店户及邻居,以及本地有关人等负责证明无其它问题的,经所管官府审查,听从免验。
(十八)凡县令、县丞、主簿虽然都应差出,但必须留下一员在县(特别情况下全缺的,州郡差官暂时代理县务)。
(十九)凡称按违制罪论处的,不按过失处理(《刑统?诸被制书》……条疏议:违背制书是指奉皇帝命令有所施行而违背的,徒刑二年,如果不是故意违背,而是理解错命令的意旨以致有所违背的,杖刑一百)。
(二十)凡居于统辖监督地位和居于主管地位的官吏,受贿枉法达二十匹绢的;没有俸禄的吏人,受贿枉法达二十五匹绢的,处以绞刑。如果所犯的罪只够流罪以及受贿而未枉法屈人赃物为五十匹绢的,配到本地牢城管制服苦役。
(二十一)凡以毒物自服,或给别人服,而诬告他人罪不至于处死刑的,配到一千里远的地方管制服苦役。如果服毒的人已经死亡,而知晓内情诬告他人的,准许人随地捕捉,给赏钱五十贯。
(二十二)凡缌麻服以上的亲属因病死亡,而以其它原故诬告别人的,按诬告法论处(指把病死说成被殴打死之类,以致官府听信已经进行检验了的),不适用官荫减刑赎罪规定,也不在引虚减等从轻处理的范围之内。如果缌麻服以上亲属间自相诬告,以及男佣、女仆病死他们亲属乃以其它原故诬告主家的,同此(尊长诬告卑幼,有关荫赎减等的规定,当然按照本法处理)。
(二十三)凡由伪装疾病、死亡和损伤造成检验不实的,按各该诈欺者所犯罪减轻一等论处。如把真病、死和伤验成假病、死和伤的,按「故入人罪」论处(《刑统》疏议:上条规定,伪装疾病者,处杖刑一百,检验不实,同于上述诈欺罪减轻一等,处杖刑九十)。
(二十四)凡有尸体虽经检验了,但是出于妄指他人的尸体提出控告,致使官府听信而据以进行审问的,按照诬告法论处。如亲属到死所妄认的,处杖刑八十。被诬告的人在监禁中造成死亡的,对妄认者罪加三等。如果官府妄审的,按「入人罪法」论处。
(二十五)《刑统》疏议:「用他物打人的,处杖刑六十(见血的就算伤。除了手脚之外,其余的都算作他物,就是兵器不使用锋刃的,也是)。
(二十六)《申明刑统》:用靴、鞋踢人伤,听从官府检验确定,靴、鞋坚硬,就按他物伤的规定论处,如不坚硬,就难于按照他物伤的规定论处。
(二十七)凡负殴伤担保责任的,手脚打伤人的担保期限十天;用他物打伤人的,二十天;用刀类及汤火伤人的,三十天;损折肢体及伤骨的,五十天。在保限之内受伤者死亡的,各按杀人罪论处(凡用牙齿咬伤人的,各按他物伤人法论处。妇女在保限内堕胎的,堕后另加保限三十天,但连同原殴伤保限加在一起,不得超过五十天)。如在保限之外,以及虽在保限之内由于其它原故死亡的,各按原殴伤罪论处(所谓「其它原故」,是指殴伤之外另增患其它病症而死的。假如殴人伤头,风从头部伤口侵入,因风造成死亡之类,仍按杀人罪论处。如果不是因头部伤口得风致死的,就是属于「其它原故」死亡,各按原殴伤罪论处)。
(二十八)干道六年八月十六日,尚书省批示州县:「检验官员,都差文官,如有缺官地方,覆验官才差武官。」
敕令所审定:「检验官员,自当依法差遣文官,如是边远小县,实在缺少文官地方,覆验官权且差遣识字武官。现申明照用。」
(二十九)嘉定十六年二月十八日敕令:「臣僚奏称:对检验尸伤不定出要害致死原因者,刑罚是很严厉的,但对检验失之真实者,却实行『觉举』从宽,遂使得以逃脱罪责。希望圣旨下刑部审定,发布指示遵行。」刑部、大理寺正副长官审议:「检验不当的,『觉举』从宽,自有现行条令,现检验不实的,竟也实行『觉举』从宽,遂使得以逃脱罪责。现审定:凡命官检验不实或不当的,一律不准援用『觉举』规定加以宽免。其余都按旧法办理。接到圣旨后遵照执行。」


\chapter{检覆总说上}

 凡验官,多是差厅子、虞候,或以亲随作公人、家人各目前去,追集邻人保伍,呼为先牌,打路排保、打草踏路、先驰看尸之类,皆是搔扰乡众,此害最深,切须戒忌。
凡检验,承牒之后不可接见在近官员、秀才、术人、僧道,以防奸欺及招词诉。仍未得凿定日时于牒,前到地头约度程限方可书凿,庶免稽迟。仍约束行吏等人不得少离官员,恐有乞觅。遇夜,行吏须要勒令供状,方可止宿。凡承牒检验,须要行凶人随行,差土着、有家累田产、无过犯节级、教头、部押公人看管。如到地头,勒令行凶人当面,对尸子细检喝;勒行人公吏对众邻保当面供状;不可下司,恐有过度走弄之弊。如未获行凶人,以邻保为众证。所有尸帐,初复官不可漏露,仍须是躬亲诣尸首地头,监行人检喝,免致出脱重伤处。
凡检官,遇夜宿处,须问其家是与不是凶身血属亲戚,方可安歇,以别嫌疑。
凡血属入状乞免检,多是暗受凶身买和,套合公吏入状,检官切不可信凭便与备申,或与缴回格目。虽得州县判下,明有公文照应,犹须审处。恐异时亲属争钱不平,必致生词,或致发觉,自亦例被污秽难明。
凡行凶器仗,索之少缓则奸囚之家藏匿移易,妆成疑狱可以免死,干系甚重。初受差委,先当急急收索。若早出官,又可参照痕伤大小、阔狭,定验无差。
凡到检所,未要自向前,且于上风处坐定,略唤死人骨属或地主、湖南有地主,他处无。竞主,审问事因了,点数干系人及邻保,应是合于检状着字人。齐足,先令扎下硬四至,始同人吏向前看验。若是自缢,切要看吊处及项上痕,更看系处尘土曾与不曾移动及吊处高下,元踏甚处、是甚物上得去系处。更看垂下长短,项下绳带大小对痕阔狭,细看是活套头、死套头,有单挂十字系、有缠绕系,各要看详。若是临高扑死,要看失脚处土痕踪迹、高下。若是落水渰死,亦要看失脚处土痕、高下及量水浅深。
其余杀伤、病患诸般非理死人,扎四至了,但令扛  明净处,且未用汤水酒醋。先于检一遍,子细看脑后、顶心、头发内,恐有火烧钉子钉入骨内。其血不出,亦不见痕损。更切点检眼睛、口、齿、舌、鼻、大小便二处,防有他物。然后用温水洗了,先使酒醋蘸纸,搭头面上、胸胁、两乳、脐腹、两肋间,更用衣被盖罨了,浇上酒醋,用荐席罨一时久方检。不得信令行人只将酒醋泼过,痕损不出也。



(一)凡验官下去验尸,多是差遣厅子虞候,或以亲随充作公差、家丁各种名目,前去召集邻人保伍,使令打前站,带路找人,踏荒开路,先行看尸之类,都是他们。在骚扰乡民上,这种害处最大,务要严加禁戒。
(二)凡验官接受验尸公文之后,不可接见出事地点附近的官员、秀才、江湖术士、和尚道士,以防止奸人诈骗及招惹是非。也不可在公文上写定检验时间,前去验尸地头约估一下实际所需时间,才能写定,以免延误受责。还必须约束仵作行人和吏役等,不许稍离官员身边,恐有不法活动。遇到夜间,仵作吏役等必须勒令他们作出保证,才可住宿。
(三)凡接受公文出去检验,需要随带凶手的,要差遣本地土著、有家室牵累和田产、没有过错的头目率领公差看押。到达验尸地点,要勒令仵作当着凶手的面对尸仔细检验喝报。要勒令仵作、吏役等对众邻人保伍当面作出保证。不要把凶手下到县监狱,恐有串通倒弄的弊端发生。如果凶手尚未捕获,就以邻人保伍作为公众见证。所有验尸记录,初、覆验官不可泄露。验官还必须亲临验尸现场,监视仵作行人检验喝报,以免漏掉重要伤损去处。
(四)凡验官遇到夜间需要住宿的地方,必须问明所要住宿的人家,是不是凶手的血属亲戚,以避嫌疑。
(五)凡死者的血亲递状请求免除检验的,多是暗受凶手买通,串同公吏呈递的,验官切不可听信,便给备文呈报免验,或让缴回验尸状。即使州、县有批示下来,明明有公文照应,也仍然需要慎重处理,恐怕以后死者的亲属因争钱不平,必将引起诉讼,问题倘或发觉,验官自然也要被牵累。这时,污秽满身,也就难于辩明了。
(六)凡行凶器物,搜索得少许缓慢,那些奸猾的囚犯之家,就要藏匿倒换,装扮成疑案,可因此免死,关系十分重大。验官一受到差委,首先应当紧急搜索凶器。如能及早搜出交官,又可以参照伤痕的大小宽窄,使得验定不出差误。
(七)凡验官到验尸现场,不要立即上前看验,且在上风的地方坐定,传唤死者的骨肉亲属,或尸体所在地的土地主人(湖南有地主到场,其它地方没有)、争讼人,粗略询问一下事情发生的经过,点数一下关系人和邻人保伍等,应在验尸单上签字的人都齐全了,先令仵作吏役等丈量记录下尸体所在的东西南北四至,然后再同仵作吏役等人一起上前看验。如果是上吊死的,务必要看吊的地方和死者颈上的索痕。要看系绳地方的尘土,绳子有没有移动过,吊的地方离地面有多少高下,原来踏的是什么地方,是什么东西才能上得去系绳处;要看绳套垂下有多少长短,项下绳带有多少粗细,对照一下勒痕的宽窄;要细看绳套是活套头还是死套头,有的绳套是单挂十字系,有的是缠绕系,各要仔细看验。如果是从高处扑下跌死的,要看失脚处土痕踪迹的高下。如果是落水淹死的,也要看失脚处土痕的高下,以及测量一下水的深浅。
其它的杀伤病患,各种横死的人,札记下四至以后,就教人扛抬到明亮干净的地方,暂且不要使用汤水酒醋洗尸,先行干检一遍。要仔细看验脑后、顶心头发内,恐怕会有火烧的钉子钉入骨内(这种烧钉钉人,血不流出,也看不到伤痕)。更要切实检验眼睛、口齿、舌、鼻、大小便二处,提防有其它东西。然后用温水冲洗后,先用酒醋蘸纸搭盖在尸体的头面上、胸胁、两乳、脐腹、两肋间,更用衣服覆盖好,浇上酒醋,用草席盖一个时辰之久,再进行检验。不可听任仵作行人只拿酒醋泼过便验,那样,伤损处是显现不出来的。



\chapter{检覆总说下}

凡检验,不可信凭行人。须令将酒醋洗净,子细检视。如烧死,口内有灰;溺死,腹胀、内有水;以衣物或湿纸搭口鼻上死,即腹干胀;若被人勒死,项下绳索交过,手指甲或抓损;若自缢,即脑后分八字,索子不交,绳在喉下,舌出;喉上,舌不出。切在详细,自余伤损致命即无可疑。如有疑虑,即且捉贼。捉贼不获,犹是公过。若被人打杀却作病死,后如获贼,不免深谴。
凡检验文字,不得作“皮破血出”,大凡皮破即血出。当云:“皮微损,有血出。”
凡定致命痕,虽小,当微广其分寸。定致命痕,内骨折,即声说;骨不折,不须言,骨不折却重害也。或行凶器杖未到,不可分毫增减,恐他日索到异同。
凡伤处多,只指定一痕系要害致命。
凡聚众打人,最难定致命痕。如死人身上有两痕皆可致命,此两□痕若是一人下手则无害;若是两人,则一人偿命,一人不偿命。须是两痕内斟酌得最重者为致命。
凡官守,戒访外事。惟检验一事,若有大段疑难,须更广布耳目以合之,庶几无误。如斗殴,限内身死,痕损不明,若有病色、曾使医人、师巫救治之类,即多因病患死。若不访问则不知也。虽广布耳目,不可任一人,仍在善使之;不然,适足自误。
凡行凶人,不得受他通吐,一例收人解送,待他到县通吐后,却勾追。恐手脚下人妄生事,搔扰也。
凡初、复检讫,血属、耆正副、邻人并责状看守尸首,切不可混同解官,徒使被扰。但解凶身、干证。若狱司要人,自会追呼。
凡检复后,体访得行凶事因不可见之公文者,面白长官,使知曲折,庶易勘鞠。
近年诸路宪司行下,每于初、复检官内,就差一员兼体究。凡体究者,必须先唤集邻保,反复审问。如归一,则合款供;或见闻参差,则令各供一款;或并责行凶人供吐大略,一并缴申本县及宪司,县狱凭此审勘,宪司凭此详复;或小有差互,皆受重责;簿、尉既无刑禁,邻里多已惊奔。若凭吏卒开口,即是私意。须是多方体访。务令参会归一。切不可凭一二人口说,便以为信,及备三两纸供状,谓可塞责。况其中不识字者,多出吏人代书。其邻证内,或又与凶身是亲故及暗受买嘱符合者,不可不察。
随行人吏及合干人,多卖弄四邻,先期纵其走避,只捉远邻及老人、妇人及未成丁人塞责。或不得已而用之,只可参互审问,终难凭以为实,全在斟酌。又有行凶人,恐要切干证人真供,有所妨碍,故令藏匿;自以亲密人或地客佃客出官,合套诬证,不可不知。
顽凶多不伏于格目内凶身下填写姓名、押字。公吏有所取受,反教令别撰名色,写作“被诬”或“干连”之类,欲乘此走弄出入。近江西宋提刑重定格目,申之朝省,添入被执人一项。若虚实未定者,不得已与之,就下书填。其确然是实者,须勒令佥押于正行凶字下,不可姑息诡随,全在检验官自立定见。



(一)凡检验,不能听任仵作行人行事,必须叫他们用酒醋将尸体洗净,仔细验看。如果是烧死的,嘴里有灰;溺死的肚腹膨胀,肚里有水;用衣物或湿纸按在口鼻上捂死的,便肚腹干胀;如果是被人勒死的,脖项下绳索相交而过,手指甲可能有抓损之处;如果是自己上吊死的,就脑后分八字,索子不相交,绳索勒在喉头之下部位的,舌头伸出口外,绳在喉上的,舌头不伸出,务要仔细看验。其余的损伤致命,便无多大可疑。如有疑虑之处,就且待捉到凶犯后再定。捉拿凶犯不着,那还是属于公事上的过错,如果死者本是被人打杀的,却作为病死验定,将来一旦凶犯捉到了,真象大白,那就免不了要受重罚了。
(二)凡检验书上的文字,不可写做「皮破血出」字样。因为太凡皮破就血出。应当写做:皮微损,有血出。
(三)凡定致命伤痕,虽小也不可稍为扩大它的分寸。定致命伤,有骨折,就说明,骨不折,不必说,骨不折,却也依然是要害啊(即或行凶器物未找到,对伤痕程度的确定,也不可有分毫的增减,恐怕他日找到后有所异同)。
(四)凡伤处多的,只指定一处伤痕为要害致命伤。
(五)凡聚众打死的人,最难定致命伤。如果死人身上有两处伤痕,都可以致命,而这两处伤痕如果是由一个人下手打的,那倒还无妨;如果是两个人打的,就要出现一个人偿命,一个人不偿命的情况了。所以必须在两处伤痕内,斟酌出一个最重的作为致命伤。
(六)凡居官守职常规是禁戒探问外事,惟独检验这件事,如果遇有大的疑难案件,必须更广布尔目探访情况加以印证,才有可能避免错误。比如有人被殴打在担保限期内死去,伤痕不明,如果死者面有病色曾经请医生、巫师救治过之类,就多半是由于病患而死的。如果不去访问,那就无从得知了。只是广布尔目,不可专任一人,仍在善于使用他们,不然,恰足以自误。
(七)凡行凶人,捕到后不要让他供吐,一律派人押解去县。等他到县里全部供出后,就立即追捕同案犯。恐怕吏役下人,妄自生事,骚扰乡民。
(八)凡初、覆验完毕,尸亲、耆长、保正副、邻人等,都要责令他们具结看守尸首。切不可混同在一起解送到官府,白白使他们受到骚扰。只是解送凶手和证人就行了。如果司狱司需要找人,自然会传唤他们的。
(九)凡检覆后,访察到行凶事由不可公开写到公文上的,当面向长官报告,使他能够了解案件的曲折情况,以便易于审理。
(十)近年各路宪司指示下级,每在初、覆验官当中,就便差委一人兼任「体究」。凡担任「体究」的,必须先召集邻人保伍,反复审问。如果他们所说的趋于一致,就合为一款供述;如果他们的所见所闻参差不一,就叫他们分别各供一款,或同时责令行凶人供述一个大略,一并呈交到本县和宪司。县狱就凭此审理,宪司就凭此复核。或小有差错,都要受到重责。簿、尉既无明令禁止,乡邻多已惊怕逃避,若凭吏卒开口,那就是他们一己的私意。应当多方深入访问,务要使各方面的材料相互参证归于一致。切不可凭一二人的口说,便以为确实可信,以及有了三两纸供状,就认为可以敷衍了事。况且乡邻中不识字的,口供多是出于吏人代写,其中有的又可能与凶手是亲戚故旧,以及暗中受到收买作假证的,不可不仔细考察。
(十一)随行的吏役及其它有关人员,常利用职务,徇私卖放近邻,事先纵使他们逃避,只捉一些远邻或老人、妇女以及未成年的人拿来敷衍塞责(或不得已而使用他们的口供,也只可与审问互相参证,终究难于据以为实,全在于仔细斟酌)。又有的行凶人怕紧要的见证人照直供述,对他有所妨碍,而故意叫躲藏起来,自以亲信人或庄客佃户等到官出庭,串通捏造假证,不可不知。
(十二)顽固的囚犯多不认罪,在验尸表格「凶身」项下填写姓名押字,公吏人员有所勒索受贿,反教罪犯别出花样,写成为被诬陷或受牵连之类,企图乘此机会变动案情,制造出入。近来江西的宋提刑重新修订了验尸表格,上报朝廷和尚书省备案,增添进了「被拘捕人」一项。如果遇到虚实未定的罪犯,不得已就给他在这一项下填写;对于那些确属实在的罪犯,就必须使令他们签名画押在「正行凶人」那一项下面。不可苟且盲从,全在检验官自己拿定主见。


\chapter{疑难杂说上}

 凡验尸,不过刀刃杀伤与他物斗打、拳手欧击、或自缢、或勒杀、或投水、或被人弱杀、或病患,数者致命而已。然有勒杀类乎自缢;溺死类乎投水;斗殴有在限内致命而实因病患身死;人力女使因被捶挞,在主家自害自缢之类。理有万端,并为疑难。临时审察,切勿轻易。差之毫厘,失之千里。
凡检验疑难尸首,如刃物所伤,透过者须看内外疮口,大处为行刃处,小处为透过处。如尸首烂,须看其元衣服比伤着去处。
尸或覆卧,其右手有短刃物及竹头之类自喉至脐下者,恐是酒醉撺倒,自压自伤。
如近有登高处或泥,须看身上有无钱物,有无损动处,恐因取物失脚自伤水类。
检妇人,无伤损处须看阴门,恐自此入刀于腹内,离皮浅则脐上下微有血沁;深则无。多是单独人求食妇人。
如男子,须看顶心,恐有平头钉。粪门恐有硬物自此入。多是同行人因丈夫年老、妇人年少之类也。
凡尸,在身无痕损,唯面色有青黯,或一边似肿,多是被人以物搭口鼻及罨捂杀。或是用手巾、布袋之类绞杀不见痕,更看顶上肉硬即是。切要者,手足有无系缚痕,舌上恐有嚼破痕,大小便二处恐有踏肿痕。若无此类,方看口内有无涎唾,喉间肿与不肿,如有涎及肿,恐患缠喉风死,宜详。
若究得行凶人,当来有窥谋、事迹分明、又已招伏,方可检出。若无影迹,即恐是酒醉卒死。
多有人相斗殴了,各自分散。散后或有去近江河池塘边洗头面上血,或取水吃,却为方相打了,尚困乏;或因醉,相打后头旋落水渰死。落水时尚活,其尸腹肚膨胀,十指甲内有沙泥,两手向前,验得只是落水渰死分明。其尸上有殴击痕损,更不可定作致命去处,但一一扎上验状,只定作落水致命最捷。缘打伤虽在要害处,尚有辜限在,法虽在辜限内及限外,以他故死者,各依本殴伤法。注:他故,谓别增余患而死者。今既是落水身死,则虽有痕伤,其实是以他故致死分明。曾有验官,为见头上伤损,却定作因打伤迷闷不觉倒在水内,却将打伤处作致命,致招罪人翻异不绝。
更有相打散,乘高扑下卓死。亦然。但验失脚处高下、扑损痕瘢、致命要害处,仍须根究曾见相打分散证佐人。
凡验因争斗致死,虽二主分明而尸上并无痕损,何以定要害致命处?此必是被伤人旧有宿患气疾;或是未争斗以前先曾饮酒至醉,至争斗时有所触犯致气绝而死也。如此者,多是肾子或一个、或两个缩上不见,须用温醋汤蘸衣服或绵絮之类罨一饭久,令仵作行人以手按小腹下,其肾子自下,即其验也。然后子细看要害致命处。
昔有甲乙同行,乙有随身衣物而甲欲谋取之。甲呼乙行路,至溪河欲渡。中流,甲执乙就水而死,是无痕也,何以验之?先验其尸瘦劣、大小,十指甲各黑黯色,指甲及鼻孔内各有沙泥,胸前赤色,口唇青班,腹肚胀。此乃乙劣而为甲之所执于水而致死也。当究甲之元情,须有赃证以观此验,万无一失。
又有年老人,以手捂之而气亦绝,是无痕而死也。
有一乡民,令外甥并邻人子,将锄头同开山种粟,经再宿不归。及往观焉,乃二人俱死在山,遂闻官。随身衣服并在。牒官验尸,验官到地头,见一尸在小茅舍外,后项骨断,头、面各有刃伤痕;一尸在茅舍内,左项下、右脑后各有刃伤痕。在外者,众曰:“先被伤而死。”在内者,众曰:“后自刃而死。”官司但以各有伤,别无财物,定两相并杀。一验官独曰:“不然!若以情度情,作两相并杀而死可矣。其舍内者,右脑后刃痕可疑,岂有自用刃于脑后者?手不便也。”不数日间,乃缉得一人,挟仇并杀两人。县案明,遂闻州,正极典。不然,二冤永无归矣。大凡相并杀,余痕无疑,即可为检验,贵在精专,不可失误。
嘉庆丁卯山东督粮道孙星衍依元本校刊,元和县学生员顾广圻复校


(一)大凡验尸,不过是刀刃杀伤与他物斗打、拳手殴击、或自己上吊、或被人勒杀、或自行投水、或被人溺杀、或由于病患而造成的死亡数种而已。但是有的被人勒杀却类似自己上吊;被人溺死却类似自行投水;斗殴受伤在担保限期内死亡而实际上却是由于病患而死;男佣、女仆因被责打而在主家自害自缢之类,道理有多种多样,都是疑难。临时须详细验看检查,切不可存有轻易的思想,要知道差之毫厘,就会失之千里。
(二)凡检验疑难尸首,如果是尖刀物所伤穿透过肢体的,要验看内外伤口的情况,伤口大的地方是穿入处,伤口小的地方是透过处。如果尸体已烂,要验看死者原来穿的衣服,对照伤到的地方。
尸体如果是覆卧,右手握有短的带刃物以及竹头之类,伤在喉至脐下一带的,恐怕是酒醉掼倒,自压自伤而死。如果尸体附近有登高之处或泥,要看验死者身上有没有钱物,有没有损动去处,恐怕是因为取物失脚自伤自死之类。
(三)检验妇女尸体不见伤损的地方,要验看阴门,恐怕有人从这里插刀入腹内。刀离皮浅的,便肚脐上下微有血晕出现,深的便没有。这种情况多发生在单身谋生的妇人身上。
如果是男尸,要看验顶心,恐怕有平头钉:要看验粪门,恐怕有硬的东西从这里插入。这多是同床干的,因为丈夫年老妇人年少之类而谋害的。
(四)凡尸体周身没有伤痕,只是面色有些青黯,或脸的一边有些像肿的样子,这多是被人用东西搭在口鼻上捂死。或是用手巾布袋之类绞死,不见痕迹,更看项上肉硬就是。务必要看手脚上有没有被捆绑的痕迹;舌头上恐怕有嚼破的痕迹;大小便二处恐怕有被脚踏肿的痕迹。如果没有这一类情况,方才看嘴里有没有涎唾,喉咙中肿与不肿。如果有口涎及喉肿现象,恐怕是患缠喉疯而死的,应当详细考察。
如果查究出行凶人近来有窥伺图谋,事迹分明,又已招认伏罪,方可检出。如果没有什么形迹,就恐怕是酒醉后突然死亡的。
(五)多有人相斗殴罢,各自分散,散后,有的人去近处江河、池塘边洗濯头脸上的血迹,或取水吃,但因为方才相打罢,还很疲劳;或因酒醉相打后头旋,落水淹死。由于落水时还是活的,因而尸体的肚腹膨胀,十指甲内有沙泥,两手向前,验得只是落水淹死。分明尸体上有殴打伤痕存在,切不可再定作致命去处,但一一札记在验尸状上,只定作落水致命,最为简捷了当。因为打伤,虽在要害地方,还有担保限期,在法律上虽在保限内以及保限外由于其它原故死了的,各按照殴伤法论处(注,其它原故是指另增其它毛病而死的)。现在既然是落水身死,纵使有伤痕,其实是因为其它原故致死是十分清楚的。曾经有位验官因看到死者头上的伤损,便定作因打伤后昏迷,不觉倒在水内致死。竟将打伤地方当成了致命所在,以致招来罪人翻案不绝。
更有相打分散后,乘高扑下失足跌的,也是这样。只要验看失脚处高下、扑伤的痕瘢、致命要害所在,还必须查问曾看到相打分散的见证人。
(六)凡检验因争斗致死的尸体,虽然双方曾经互相殴打的事很清楚,但是尸体上并不见有伤损,这要怎样来定要害致死之处呢?这一定是被伤人本来就患有气疾,或是未争斗以前,先曾饮酒至醉,到争斗时有所触犯,以致气绝而死了的。这样的情况,多是睾丸一个或两个缩上不见,须用温醋蘸衣服或绵絮之类热敷一顿饭之久,叫仵作行人用手按压死者的小腹下部,其睾丸自下,便是验证。然后仔细验看要害致命所在。
(七)从前有甲乙二人同行,乙有随身衣物,而甲想设法取得它,于是招呼乙一道走,路至溪河,刚要渡过中流的时候,甲便捉住乙按到水中淹死。这自然是没有伤痕的,怎样检验呢?先验看死者尸身瘦弱,大小十指指甲各呈黯色,指甲及鼻孔内各有沙泥,胸前呈现赤色,嘴唇有青斑,肚腹鼓胀,这就是乙比较瘦弱被甲按到水里而致死的了。要审问查明甲作案时的原始情节,要有赃证加以验证,就会万无一失。
又有的年老人,被人用手捂住口鼻,也便气绝身死。这也是没有伤痕而死亡的一种情况。
(八)有一个乡民,叫自己的外甥跟邻人的儿子携带锄头一起去开山种粟。经过两宿不归。待前往探看时,竟发现两个人都死在山上了。遂报告到官府。经查死者衣服都在,发出公文请官验尸。验官到达地头,看到一尸在小茅屋外面,后脖颈骨被砍断,头部和面部各有刃伤痕;一尸在茅屋里面,左项下、右脑后各有刃伤痕。在屋外的,众人说是先被杀伤而死的;在屋内的,众人说是自杀而死的。官府但以两尸各有伤痕,别无财物,定作两相并杀而死。一验官独说:「不然。如果拿一般情况来推度案情,作为两相并杀而死是可以的;但是那屋内尸上的右脑后刀痕很可怀疑,那有自己拿刀从脑后自杀的呢?手不方便啊。」后来没有隔几天,就捕获到了一个怀仇并杀两人的凶手。悬案大白,遂报告到州府,将凶手处了极刑。如果不是这样,那两个人的冤仇就要永无归宿了!大凡两相并杀,所有伤痕都无可疑,即可予以检验定案。贵在精细专心,不可疏忽差错。


\part{}

\chapter{疑难杂说下}

有检验被杀尸在路傍,始疑盗者杀之。及点检,沿身衣物俱在,遍身镰刀斫伤十余处。检官曰:“盗只欲人死取财,今物在伤多,非冤仇而何?”遂屏左右,呼其妻问曰:“汝夫自来与甚人有冤仇最深?”应曰:“夫自来与人无冤仇,只近日有某甲来做债不得,曾有克期之言,然非冤仇深者。”检官默识其居,遂多差人分头告示侧近居民:“各家所有镰刀尽底将来,只今呈验,如有隐藏,必是杀人贼,当行根勘!”俄而,居民赍到镰刀七八十张,令布列地上。时方盛暑,内镰刀一张,蝇子飞集。检官指此镰刀问为谁者?忽有一人承当,乃是做债克期之人。就擒讯问,犹不伏。检官指刀令自看:“众人镰刀无蝇子,今汝杀人血腥气犹在,蝇子集聚,岂可隐耶?”左右环视者失声叹服,而杀人者叩首服罪。
昔有深池中溺死人,经久,事属大家因仇事发。体究官见皮肉尽无,惟髑髅骨尚在。累委官不肯验。上司督责至数人,独一官员承当。即行就地检骨。先点检,见得其他并无痕迹,乃取髑髅净洗,将净热汤瓶细细斟汤灌,从脑门穴入,看有无细泥沙屑自鼻孔窍中出,以此定是与不是生前溺水身死。盖生前落水,则因鼻息取气,吸入沙土;死后则无。
广右有凶徒,谋死小童行而夺其所赍。发觉,距行凶日已远。囚已招伏:“打夺就,推入水中。”尉司打捞,已得尸于下流,肉已溃尽,仅留骸骨,不可辨验,终未免疑其假合,未敢处断。后因阅案卷,见初焉体究官缴到血属所供,称其弟元是龟胸而矮小。遂差官复验,其胸果然,方敢定刑。
南方之民,每有小小争竞,便自尽其命而谋赖人者多矣。先以榉树皮罨成痕损,死后如他物所伤。何以验之?但看其痕,里面须深黑色,四边青赤,散成一痕而无虚肿者,即是生前以榉树皮罨成也。盖人生即血脉流行,与榉相扶而成痕。若以手按着痕损处,虚肿,即非榉皮所罨也。若死后以榉皮罨者,即苦无散远青赤色,只微有黑色。而按之不紧硬者,其痕乃死后罨之也。盖人死后血脉不行,致榉不能施其效。更在审详元情,尸首痕损,那边长短能合他物大小,临时裁之,必无疏误。
凡有死尸,肥壮无痕损,不黄瘦,不得作病患死。又有尸首,无痕损,只是黄瘦,亦不得据所见只作病患死检了。切须子细验定因何致死。唯此等检验,最误人也。
凡疑难检验及两争之家稍有事力,须选惯熟仵作人,有行止、畏谨守分、贴司,并随马行。饮食水火,令人监之。少休,以待其来。不知是,则私请行矣。假使验得甚实,吏或受赂,其事亦变。官吏获罪犹庶几,变动事情、枉致人命,事实重焉。
应检验死人,诸处伤损并无,不是病状,难为定验者,先须勒下骨肉次第等人状讫,然后剃除死人发髻,恐生前彼人将刃物钉入囟门或脑中,杀害性命。
被残害死者,须检齿、舌、耳、鼻内或手足指甲中,有签制算害之类。
凡检验尸首,指定作被打后服毒身死、及被打后自缢身死、被打后投水身死之类,最须见得亲切方可如此申上。世间多有打死人后,以药灌入口中,诬以自服毒药;亦有死后用绳吊起,假作生前自缢者;亦有死后推在水中,假作自投水者。一或差互,利害不小。今须子细点检死人在身痕伤,如果不是要害致命去处,其自缢、投水及自服毒,皆有可凭实迹,方可保明。


(一)有位验官验一个被杀在路旁的尸体,起初怀疑是强盗杀的。待到检点全身,发现衣物全在,遍身有镰刀砍伤十多处。验官说:「强盗要杀人只为取财,现在财物在而伤痕多,不是仇杀是什么?」于是叫左右退下,传唤死者的妻子来问道:「你丈夫平日跟什么人有冤仇最深?」回答说:「我丈夫向来与人没有冤仇。只是近日有某甲前来借债,没有借到,曾有限定日期的言语,但说不上是冤仇深的。」验官默记下了某甲的住处,遂多差人分头告示某甲住地附近的居民:「各家所有镰刀尽数拿出来,立即呈交验看,如有隐藏,必是杀人贼,将予追究查办!」不一会儿,居民送到了镰刀七、八十张。令按次陈列在地上。当时正值盛暑天气,内有镰刀一张,苍绳飞集其上。验官指着这把镰刀问是谁的,忽有一人出来承当,原来就是那个借债未遂克期而去的人。当即逮捕审问,仍不认罪。验官指着镰刀叫他自己看,对他说:「众人的镰刀上都没有苍蝇。现在你杀人留下的血腥气仍在,所以苍蝇集聚。难道能隐瞒得了吗?」左右围观的人都为之失声叹服,那个杀人的人也叩头承认了自己的罪行。
(二)从前有深池中淹死了人,经过很长时间,事情被有势力的人家因为仇事揭发了出来。体究官下来,见皮肉都没有了,只有髑髅和骸骨还在。屡次派官,都不肯验。上级督责至数人,独有一位官员出来承当。当即进行就地验骨。先点检一遍,发现并没有什么伤害痕迹。于是取来髑髅加以净洗,用干净的水瓶斟水细细从脑门穴灌入,看有没有细泥沙屑自鼻窍中流出,以此来判定是否是生前溺水身死的。这是因为生前溺水死,就会因鼻孔吸气,吸入泥沙,死后入水的便没有。
(三)广西地方,有凶徒谋害死一个小童行,而夺去了他所携带的财物。待被发觉,距离行凶时间已远。囚犯已经招认:「劫夺完毕,把人推入水中。」经县尉司打捞,也在河下流涝到了尸体。肉已烂尽,只留骸骨,不可辨认。官府终不免怀疑它是属于一种偶合,不敢决断处理。后因翻阅案卷,看到最初体究官交到的一份尸亲所作的供述,说其弟本是一个龟胸而矮小的人。遂即派官前往覆验,尸骸的胸骨果然是这样,才敢定刑。
(四)南方之民,每为小小争竞,便自尽其命以图诬赖他人的是很多的。办法是先用榉树皮在身上罨敷成一种伤痕,死后就像是用他物打伤。这要怎样来辨验呢?只要看其痕里面是深黑色,四边青赤,散成一痕,而又没有浮肿的,就是生前用榉树皮罨敷成的了。这是由于人活着血脉流行,与榉皮相辅而成痕的原故(如果用手按下,痕损处虚肿,那就不是榉皮罨敷成的了)。如果是死后用榉皮罨敷的,就苦于没有扩散远伸的青赤色。只是微有黑色,按之不坚硬的,这样的伤痕就是死后罨敷出来的了。这是由于人死以后,血脉不行,致使榉皮不能发挥效用的原故。更在于详细观察研究案件的始初情节,尸体伤痕的那边长短,能合乎什么器物大小,临时斟酌,定无差误。
(五)凡有死尸肥壮、没有伤痕、不黄瘦的,不得作为病患死;又有尸首无伤痕,只是黄瘦,也不得只根据所见就当作病患死检验了事。务要仔细验定致死原因是什么,唯有这类检验最易误人啊!
(六)凡是疑难的检验,以及争讼双方稍有势力的,要挑选业务纯熟的仵作行人,有品行、小心谨慎的书吏,并跟随验官的马后行走,饮食大小便,都要令人监视,少停,以待其来。不这样做,那私下求情请托之类的事就要发生了。即使验得很真实,吏役人等如果受了贿赂,事情也要变样的。官吏获罪倒还罢了,变动了案情,冤枉了人命,事情实在重大啊。
(七)应检验的死人,各处并无损伤,也不是患病的样子,难于定验的,需要勒令死者的骨肉亲属人等依次供述,然后剃去死者的发髻验看,恐生前被人用尖刃的东西钉入顖门或脑中,杀害了性命。
(八)被残害死的,要检验齿、舌、耳、鼻内或手足指甲中,可能有签子刺入谋害之类的情况。
(九)凡检验的尸首,指定作被打后服毒身死以及被打后自缢身死、被打后投水身死之类的,最需要检验得确切实在,才能照此呈报上去。社会上常有在打死人后,用药物灌入死者口中,诬为自己服毒死的;也有在人死后用绳吊起,假作生前自己上吊死的;也有人死后推入水中,假作生前自行投水死的。一有差误,利害不小。必须仔细点验死者在身伤痕,如果不是要害致命的去处,其自缢、投水及自服毒等,都要有可靠的凭据,方可证明。


\chapter{初检}

告状切不可信,须是详细检验,务要从实。
有可任公吏,使之察访。或有非理等说,且听来报,自更裁度。
戒左右人,不得卤莽。
初检,不得称尸首坏烂不任检验,并须指定要害致死之因。
凡初检时,如体问得是争斗分明,虽经多日,亦不得定作无凭检验,招上司问难。须子细定当痕损致命去处。若委是经日久变动,方称尸首不任摆拨。初检尸有无伤损讫,就验处衬簟,尸首在物上,复以物盖。候毕,周围用灰印记,有若干枚,交与守尸弓手、耆正副、邻人看守。责状附案,交与复检,免至被人残害伤损尸首也。若是疑难检验,仍不得远去,防复检异同。


(一)告状,切不可信,必须经过详细检验,务要根据事实。
(二)有可信任的吏役人员,使令进行访察,或有关于凶横死等方面的说法,且听他回来汇报,自己再作裁酌定夺。
(三)要告戒左右手下人,不可鲁莽从事。
(四)初检,不得称尸首坏烂,不堪检验,并要指定出要害所在和致死原因。
(五)凡初检时,如已调查了解到是因争斗而死,事实分明的话,虽然经过多日了,也不可定作「无凭检验」,致讨上级的责备非难。必须仔细验定损伤致命的去处。如果实在是经隔日久腐败了,才可以称尸体不堪摆弄拨动。
(六)初检尸体有无伤损毕,就验处将尸体衬垫在东西上,再用东西覆盖起来。等垫盖好了,在四周围打上石灰印,记下有若干枚数,然后交给弓手、耆长、保正副、邻人等看守,立下责任状附卷,交与覆验者,以免被人残害伤损了尸体。如果是疑难的检验,仍不得远去,以防覆验中出现不一致。


\chapter{覆检}

与前检无异,方可保明具申。万一致命处不明,痕损不同,如以药死作病死之类,不可概举。前检受弊,复检者乌可不究心察之,恐有连累矣。
检得与前验些小不同。迁就改正。果有大段违戾,不可依随,更再三审问干系等人,如众称可变,方据检得异同事理供申。不可据己见便变易。
复检,如尸经多日,头面胖胀,皮发脱落,唇口翻张,两眼迭出,蛆虫咂食,委实坏烂不通措手。若系刃伤、他物、拳手足踢痕虚处,方可作无凭复检状申。如是他物及刃伤骨损,宜冲洗子细验之,即须于状内声说致命,岂可作无凭检验申上。
复检官验讫,如无争论,方可给尸与亲属。无亲属者,责付本都埋瘗,勒令看守,不得火化及散落。如有争论,未可给尸。且掘一坑,就所簟物尸安顿坑内,上以门扇盖,用土罨瘗作堆,周回用灰印印记,防备后来官司再检复,仍责看守状附案。


(一)覆验结果与前检无异,才能负责证明,备文上报。万一致命处不清楚,伤痕不一样,如将药死了的作病死了的之类,不可粗略呈上。前检有了弊病,覆验者怎么可以不细心考察?那恐怕是要受到连累的了。
(二)覆验结果与前检只有细小的不同,可以迁就改正;果真有重大出入,就不可依从。更要再三审问本案的关系人等,如众人都说可变,方根据所检验到的与前检不一致的事实和理由加以改定申报。不可根据一己之见,就加以改变。
(三)覆验时,如果尸体已经过多日,头面膨胀,皮发脱落,口唇翻张,两睛突出,蛆虫咂食,着实坏烂不堪,无从下手,若是刃物伤、他物伤、拳打足踢伤等伤处已虚空了,才能作「无凭覆验」备文申报。如果是他物伤及刃物伤造成骨损了的,应当加以冲洗仔细检验,必须于验尸状内申说致命原因,岂可作「无凭检验」申报上级?
(四)覆验官检验完毕,如无争论,方可把尸体交给尸亲。没有尸亲的,责成本地都保加以掩埋,勒令看守,不得火化及散落。如有争论,不可发还尸体。且掘一坑,就所垫物抬尸安放坑内,上面用门扇盖好,用土掩埋成坟堆,周围用石灰打上印记,防备后来官府再检验覆查。还要责成看守人员立下看守状附案。


\chapter{验尸}

身上件数:○正头面:有无髻子?发长、若干?顶心、囟门、发际、额、两眉、两眼、或开或闭。如闭,擘开验眼睛全与不全。鼻、两鼻孔。口、或开或闭。齿、舌、如自缢,舌有无抵齿。胲、喉、胸、两乳、妇人两奶膀。心腹、脐、小肚、玉茎、阴囊、次后,捻两肾子全与不全。妇人言产门,女子言阴门。两脚大腿、膝、两脚臁肕、两脚胫、两脚面、十指爪。
翻身:脑后、乘枕、项、两胛、背脊、腰、两臀瓣、有无杖疤。谷道、后腿、两曲 、两腿肚、两脚跟、两脚板。
左侧:○左顶下、脑角、太阳穴、耳、面脸、颈、肩、膊、肘、腕、臂、手、五指爪、全与不全,或拳、或不拳。曲腋、胁肋、胯、外腿、外膝、外臁肕、脚踝。右侧,亦如之。四缝尸首须躬亲看验。顶心、囟门、两额角、两太阳、喉下、胸前、两乳、两胁肋、心腹、脑后、乘枕、阴囊、谷道,并系要害致命之处。妇人看阴门、两奶膀。
于内若一处有痕损在要害,或非致命,即令仵作指定喝起。
众约死人年几岁,临时须子细看颜貌供写,或问血属尤真。
凡检尸,先令多烧苍术、皂角,方诣尸前。检毕,约三五步,令人将醋泼炭火上,行从上过,其秽气自然去矣。
多备葱、椒、盐、白梅,防其痕损不见处,藉以拥罨。仍带一砂盆,并捶研上件物。
凡检复,须在专一,不可避臭恶。切不可令仵作行人遮闭玉茎、产门之类,大有所误。仍子细验头发内、谷道、产门内,虑有铁钉或他物在内。
检出致命要害处,方可押两争及知见、亲属令见。切不可容令近前,恐损害体尸。
被伤处,须子细量长、阔、深、浅、小、大,定致死之由。
仵作行人受嘱,多以芮一作茜草投醋内,涂伤损处,痕皆不见。以甘草汁解之,则见。
人身本赤黑色,死后变动作青 色,其痕未见。有可疑处,先将水洒湿,后将葱白拍碎令开,涂痕处,以醋蘸纸盖上,候一时久,除去,以水洗,其痕即见。
若尸上有数处青黑,将水滴放青黑处,是,痕则硬,水住不流;不是,痕处软,滴水便流去。
验尸并骨伤损处,痕迹未见,用糟、醋泼罨尸首,于露天以新油绢或明油雨伞覆欲见处,迎日隔伞看,痕即见。若阴雨,以熟炭隔照,此良法也。或更隐而难见,以白梅捣烂摊在欲见处,再拥罨看。犹未全见,再以白梅取肉加葱、椒、盐、糟一处研,拍作饼子火上煨,令极热,烙损处,下先用纸衬之,即见其损。
昔有二人斗殴,俄顷,一人仆地气绝,见证分明。及验出,尸乃无痕损,检官甚挠。时方寒,忽思得计,遂令掘一坑,深二尺余,依尸长短,以柴烧热得所,置尸坑内,以衣物覆之。良久,觉尸温,出尸,以酒、醋泼纸贴,则致命痕伤遂出。
拥罨检讫,仵作行人喝四缝尸首。谓尸仰卧,自头喝:顶心、囟门全,额全,两额角全,两太阳全,两眼、两眉、两耳、两腮、两肩并全,胸、心、脐、腹全,阴肾全,妇人云产门全,女人云阴门全。两髀、腰、膝、两臁肕、两脚面、十指爪并全。
左手臂、肘、腕并指甲全,左肋并胁全,左腰、胯及左腿、脚并全。右亦如之。
翻转尸:脑后、乘枕全,两耳后发际连项全,两背胛连脊全,两腰眼、两臀并谷道全,两腿、两后、两腿肚、两脚跟、两脚心并全。



(一)身上应检件数:
正面:(有无髻子?)发长(若干?)、顶心、顖门、发际、额、两眉、两眼(或开或闭,如有的开有的闭,如果是闭的,要擘开眼睑验看眼睛全还是不全)、鼻(两鼻孔)、口(有的开有的闭)、齿、舌(如果是自缢死的,要看舌头有没有抵齿)、颔颏、喉、胸、两乳(妇人两乳房)、心腹、肚脐、小肚、阴茎、阴囊(外部验看以后,还要用手捻捏两睪丸看全与不全。妇人阴叫产门,女子阴叫阴门)、两大腿、两膝、两小腿、两脚腕、两脚面、十脚指。
翻身(背后):脑后、乘枕、颈项、两肩胛、背、腰、两臀瓣(有没有杖打的伤痕?)、肛门、大腿后侧、两腿弯、两腿肚、两脚跟、两脚板。
左侧:左项下、脑角、太阳穴、耳、面脸、脖颈、肩膊、肘、腕、臂、手、五指(齐全不齐全?拳握不拳握?)、腋窝、胁肋、胯外、膝外、外臁肕、脚踝。
右侧:与左侧相同。
按前后左右由上而下顺序对尸体各部位进行全身普检,验官必须亲自看验:顶心、顖门、两额角、两太阳、喉下、胸前、两乳、两胁肋、心腹、脑后、乘枕、阴囊、肛门各部位,都是要害致命之处(妇人看阴门、两乳房)。
其中如果一处有伤损在要害部位,即或不是致命伤,就要令仵作指定喝报出来。
(二)众人约估死人年纪多少岁,临时必须仔细看尸体的颜貌供写,或询问死者血亲,尤为真实可靠。
(三)凡验尸,先要令人多烧苍朮、皂角,方到尸前。检验完毕,在大约三五步远的地方,令人将醋泼在炭火上,打从上面走过,身上的秽臭之气就会自然去掉了。
(四)验尸时要多备葱、椒、盐、白梅等物,以防尸体上的伤痕不见处,借以拥罨。还要带一只砂盆,用以捣研上开药物。
(五)凡检验覆验,需要专心一意,不可回避秽臭。切不可令仵作行人遮蔽阴茎、产门之类,大有所误。还要仔细看验头发内,肛门、产门内,怕有铁钉或其它东西在内。
(六)检验出致命要害处了,方可押双方当事人和证人、亲属等使令看见。切不可容许他们近前,恐损害尸体。
(七)被伤处需要仔细量度长、阔、深、浅、小、大,定出致死的原因。
(八)仵作行人受人买通嘱托,多以芮(一作茜)草投入醋内,用它涂在伤损地方,伤痕便都会隐而不见。用甘草汁解之,便见。
(九)人体本为赤黑色,死后变化作青膒色,尸体上的伤痕不易显现。遇有可疑地方,可先用水洒湿,再用葱白拍碎使开,涂在可疑处,以醋蘸纸覆盖在上面。等候一个时辰之久除去,用水洗,伤痕就会现出。
(十)假如尸体上有数处青黑,可用水滴到青黑的地方,是伤痕便硬,水止住不流;不是伤痕处软,滴水即流去。
(十一)检验尸及骨伤损处,痕迹不见,可用糟醋泼罨尸体,在露天处用新油绢或明油雨伞罩在想要见到的地方,迎日隔伞看,伤痕即见。如遇阴雨天可用炭火隔照。这是一种好方法。或有更加隐伏难于看见的,用白梅捣烂,摊盖在想看到的地方,再拥罨看。如果还不能完全看清楚,那就再用白梅取肉,加上葱、椒、盐、糟等放在一起磨碎,拍成饼子,用火煨令极热,烙伤处,下面先用纸衬垫起来,便能见到伤损。
(十二)从前有两个人斗殴,顷刻一个人扑倒在地气绝身死。见证分明。及至检验,尸体全身并无伤痕。验官甚感困扰。当时正值寒天,忽然想到一个办法,即令人挖掘一坑,深二尺多,依照尸体的长短,用柴火烧热到适当程度,放尸到坑内,用衣物覆盖起来。良久,觉尸温,抬出尸体用酒醋泼纸敷贴,致命伤痕于是现了出来。
(十三)拥罨检验完毕,仵作行人要喝报尸体的全身应检部位,说:尸首仰卧,从头喝报起。顶心、顖门全,额全,两额角全,两太阳全,两眼、两眉、两耳、两腮、两肩并全,胸、心、脐、腹全,阴肾全(妇人为产门全,女人为阴门全),两髀、腰、膝、两臁肕、两脚面、十脚指并全。
左手臂、肘、腕并指甲全,左肋并胁全,左腰、胯及左腿、脚并全。
右也同左一样。
翻转尸,脑后、乘枕全,两耳后、发际连项全,两背胛连脊全,两腰眼、两臀并肛门全,两腿、两后〈月秋〉、两腿肚、两脚跟、两脚心并全。


\chapter{妇人}

\begin{yuanwen}
凡验妇人,不可羞避。
若是处女,劄四至讫,劄出光明平稳处,先令坐婆剪去中指甲,用绵札。先勒死人母亲及血属并邻妇二三人同看,验是与不是处女。令坐婆以所剪甲指头入阴门内,有黯血出,是;无即非。
若妇人有胎孕不明致死者,勒坐婆验腹内委实有无胎孕。如有孕,心下至肚脐以手拍之,坚如铁石;无即软。
若无身孕,又无痕损,勒坐婆定验产门内,恐有他物。
有孕妇人被杀。或因产子不下(禁止)死,尸经埋地窖,至检时却有死孩儿。推详其故,盖尸埋顿地窖,因地水、火风吹,死人尸首胀满,骨节缝开,故逐出腹内胎孕孩子。亦有脐带之类,皆在尸脚下,产门有血水、恶物流出。
若富人家女使,先量死处四至了,便扛出大路上,检验有无痕损,令众人见,以避嫌疑。
附小儿尸并胞胎
有因争斗因而杀子谋人者,将子手足捉定,用脚跟于喉下踏死。只令仵作行人,以手按其喉必塌,可验真伪。
凡定当小儿骸骨,即云:“十二三岁小儿”。若驳问:“如何不定是男是女?”即解云:“某当初只指定十二三岁小儿,即不曾说是男是女。盖律称‘儿’,不定作‘儿’是男女也。”
堕胎者准律:“未成形像,杖一百;堕胎者,徒三年。”律云“堕”,谓打而落,谓胎子落者。按《五藏神论》:“怀胎一月如白露,二月如桃花,三月男女分,四月形像具,五月筋骨成,六月毛发生,七月动右手,是男于母左;八月动左手,是女于母右,九月三转身,十月满足。”
若验得未成形像,只验所堕胎作血肉一片或一块。若经日坏烂,多化为水。若所堕胎已成形像者,谓头脑、口、眼、耳、鼻、手、脚、指甲等全者,亦有脐带之类,令收生婆定验月数,定成人形或未成形,责状在案。
堕胎儿在母腹内被惊后死胎下者,衣胞紫黑色,血荫软弱,生下腹外死者,其尸淡红赤,无紫黑色及胞衣白。
\end{yuanwen}

(一)凡验妇人尸体,不可怕羞回避。
(二)如果是检验处女,丈量札记四至完毕后,抬出到光明平稳的地方,先令稳婆剪去中指的指甲,用绵絮扎裹指头,勒令死者的母亲及其它血亲并邻妇二三人一同看验。是与不是处女,令稳婆用所剪甲的指头插入阴户内,有黯血出就是,没有就不是。
(三)如果怀有胎孕,不明致死原因的,要勒令稳婆先检验腹内确实有没有胎孕。如有孕,从心口以下到肚脐,以手拍之,硬如铁石;没有孕,便软。
(四)如果没有胎孕,又没有伤损的,勒令稳婆一定要验看阴户内,恐怕其中有他物。
(五)有的怀孕妇人被杀,或因产子不下身死,尸体经过埋进地窖,到检验时却有了死孩儿了。仔细研究它的原故,这大概是因为尸体埋在地窖,由于地水火风自然力的作用,死人的尸体膨胀了起来,骨节缝松了开来,因此逐出了腹内的胎儿。这种胎儿也有脐带之类,都是在尸体的胯下。妇人的阴户有血水、秽物流出。
(六)如果是富人家的使女,先量死处四至完了,便抬出去到大路上检验。有没有伤损,使大家看见,以避免嫌疑。

【附】小儿尸并胞胎
(一)有因争斗而杀死己子以谋害他人的,把自己孩子手脚捉住,用脚跟踩在喉下踏死。对此只要令仵作行人用手按他的喉部,必定塌陷,便可以验出真伪来。
(二)凡检验鉴定小儿骸骨,就说是十二三岁小儿。如果有人质问为什么不定是男是女?便解释说:某当初只指定是十二三岁小儿,就不曾说是男是女,因为律上只称儿,不定出「儿」是男是女啊。
(三)对于堕胎的,比照刑律未成形象的处杖刑一百,堕胎儿的处徒刑三年。刑律上所说的堕,是指打而落的意思,是指胎儿被打落而言的。按照五藏神论的说法:怀胎一个月的如白露,二个月的如桃花,三个月的男女分,四个月的形象备,五个月的筋骨成,六个月的毛发生,七个月时动右手,是男偏在母左,八个月时动左手,是女偏在母右,九个月三转身,十个月满足。
如果所检验的胎孕是未成形象的,只验所堕胎作血肉一片或一块——如经隔日久坏烂了,多化为水。如果所堕胎已成形象的,是说头脑、口、眼、耳、鼻、手、脚、指甲等齐全的,也有脐带之类。要令收生婆验定月数,定成人形或未成人形,责成具状在案。
(四)堕胎儿在母腹内被惊后死胎下的,衣胞紫黑色,血荫软弱。生下到腹外死的,其尸淡红或赤色,无紫黑色,以及胞衣发白。


\chapter{四时变动}

春三月、尸经两三日,口、鼻、肚皮、两胁、胸前肉色微青。经十日则鼻、耳内有恶汁流出。胖匹缝切,胀臭也胀肥人如此。久患瘦劣人,半月后方有此证。
夏三月,尸经一两日,先从面上、肚皮、两胁、胸前肉色变动。○经三日,口、鼻内汁流蛆出,遍身胖胀,口唇翻,皮肤脱烂,疱胗起。○经四五日,发落。
暑月罨尸,损处浮皮多白,不损处却青黑,不见的实痕。设若避臭秽,据见在检过,往往误事。稍或疑处,浮皮须令剥去,如有伤损,底下血荫分明。更有暑月,九窍内未有蛆虫,却于太阳穴、发际内、两胁、腹内先有蛆出,必此处有损。
秋三月,尸经二三日,亦先从面上、肚皮、两胁、胸前肉色变动。
经四五日,口、鼻内汁流蛆出,遍身胖胀,口唇翻,疱胗起。
经六七日,发落。
冬三月,尸经四五日,身体肉色黄紧,微变。
经半月以后,先从面上、口、鼻、两胁、胸前变动。
或安在湿地、用荐席裹角埋瘗其尸,卒难变动。更详月头月尾,按春秋节气定之。
盛热,尸首经一日即皮肉变动,作青黯色,有气息。
经三四日,皮肉渐坏,尸胀,蛆出,口、鼻汁流,头发渐落。
盛寒五日,如盛热一日时,半月如盛热三四日时。
春秋气候和平,两三日可比夏一日,八九日可比夏三四日。
○然人有肥、瘦、老、少,肥、少者易坏,瘦、老者难坏。
○又南北气候不同,山内寒暄不常。更在临时通变审察。



(一)在春季的三个月中,尸体经过两三天,口、鼻、肚皮、两胁、胸前,肉色微青。经过十天,便鼻、耳内有臭水流出,尸体膨胀发臭。肥胖的人是这样,久患疾病和身体瘦劣的人,半月后,才有这样的证象。
(二)在夏季的三个月中,尸体经过一两天,先从面上、肚皮、两胁、胸前肉色发生变化。经过三天,口鼻内液体外流,蛆虫生,周身膨胀发臭,口唇翻张,皮肤脱烂,疱胗起。经过四五天,毛发脱落。
在暑热月份里洗罨尸体,伤损处浮皮多发白,不伤损处却青黑,看不到确实的伤痕所在。假如怕臭避脏,只根据当时所能看到的表面情况检过了事,往往误事。稍有可疑之处,浮皮须令仵作剥去,如有伤损,底下血荫分明。
更有暑热月份里尸体的九窍内没有蛆虫,却在太阳穴、发际内,两胁、腹内等处先有蛆虫的,必定是这些地方有伤损存在。
(三)在秋季的三个月中,尸体经过两三天,也先从面上、肚皮、两胁、胸前肉色发生变化。经过四五天,口鼻内液体外流,蛆虫生,全身膨胀发臭,口唇翻张,疱胗起。经过六七天,毛发脱落。
(四)在冬季的三个月中,尸体经过四五天,身上肉色黄紧微变。经过半个月以后,先从面上、口、鼻、两胁、胸前等处开始腐败。
或有安放在湿地,用草席包裹着的,其尸一时难于腐败。要仔细研究季度的月头和月尾情况,按照春秋节气的变化来确定尸体的时间。
(五)当盛热的天气,尸体经过一天,便皮肉腐败,变作青黯色,有气味。经过三四天,皮肉逐渐坏烂,尸身膨胀,生蛆,口鼻液体外流,头发逐渐脱落。
(六)当盛寒的天气,尸体变化五天相当于盛热一天时的情况,半个月相当于盛热三四天时的情况。
春秋两季气候温和,尸体变化情况,两三天可比夏天一天,八九天可比夏季三四天。
但是人有肥、瘦、老、少,肥胖、年少的容易腐败,瘦劣,年老的难于腐败。
再有,南北方气候不同,山区气候寒暖无常,更在于临时灵活运用,细心考察。


\chapter{洗罨}

 宜多备糟、醋。○衬尸纸惟有藤连纸、白抄纸可用。若竹纸,见盐、醋多烂,恐侵损尸体。
尸于平稳光明地上,先干检一遍。用水冲洗,次挼皂角洗涤尸垢腻,又以水冲荡洁净。
洗时下用门扇簟席衬,不惹尘土。洗了,如法用糟、醋拥罨尸首。仍以死人衣物尽盖,用煮醋淋,又以荐席罨一时久,候尸体透软,即去盖物,以水冲去糟、醋方验。不得信行人说,只将酒醋泼过,痕损不出。
初春与冬月,宜热煮醋及炒糟令热。○仲春与残秋,宜微热。○夏秋之内,糟、醋微热,以天气炎热,恐伤皮肉。○秋将深。则用热尸左右手肋,相去三四尺,加火熁,以气候差凉。○冬雪寒凛,尸首僵冻,糟、醋虽极热,被衣重叠,拥罨亦不得尸体透软。当掘坑长阔于尸,深三尺,取炭及木柴遍铺坑内,以火烧令通红,多以醋沃之,气勃勃然,方连拥罨法物衬簟, 尸置于坑内,仍用衣被覆盖,再用热醋淋遍。坑两边相去二三尺,复以火烘。约透,去火,移尸出验。○冬残春初,不必掘坑,只用火烘两边。看节候详度。湖南风俗,检死人皆于尸傍开一深坑,用火烧红。去火,入尸在坑内,泼上糟、醋,又四面有火逼。良久,扛出尸。或行凶人争痕损,或死人骨属相争,不肯认。至于有三四次扛入火坑重检者,人尸至三四次经火,肉色皆焦赤,痕损愈不分明,行吏因此为奸。未至一两月间,肉皆溃烂。及其家有论诉,差到聚检官时已是数月,止有骨殖,肉上痕损并不得而知。火炕法,独湖南如此,守官者宜知之。


(一)洗罨尸体,应当多准备糟、醋。
衬尸纸只有藤连纸、白抄纸好用,像竹纸之类,遇到盐、醋多烂,恐怕要浸坏尸体。
(二)抬尸到平稳光明的地上,先检验一遍,用水进行冲洗。其次搓皂角洗涤尸体上的垢腻,再用水冲洗干净(洗时,下面要用门扇、席子等垫上,不使尸体沾惹尘土)。洗毕,按照通常方法用糟、醋拥罨尸体。仍用死人的衣物覆裹尸体,以煮热的醋浇淋,再用草席覆盖一个时辰左右,待尸体透软了,便拿掉覆盖物,用水冲去糟、醋,开始检验。不可听信仵作行人说,只用洒、醋泼过,那样,伤损不会出来。
(三)初春与冬月,适合于用热煮醋和热炒糟使尸体发热。
仲春与残秋,适合微热。
夏秋之间,糟、醋稍微热一点即可,因为天气炎热,恐伤皮肉。
秋将深,则要用热,尸体的左右手和胁肋两边相去三四尺远的地方,加用火烘,因气候略凉。
冬日雪天,寒气凛冽,尸体僵冻,糟、醋虽极热,衣被重叠围裹,也不能达到尸体透软。应当挖掘一坑,长阔于尸体,深三尺,取炭及木柴遍铺坑底,用火烧使通红,然后多用醋浇泼,使之热气腾腾,方才连用覆盖所用各物、垫席等抬尸放到坑内,仍用衣被覆盖,再用热醋浇遍。在坑两边相去二三尺远的地方,再用火烘。大约透软了,去火移尸出验。
冬残春初,不必掘坑,只用火烘两边,要看季节、气候情况酌处。
(四)湖南的风俗,检验死人都在尸旁开一深坑,用火烧红,去火放尸到坑内,泼上糟、醋,又四面用火逼烘良久,抬出尸体。有时行凶人对伤痕有争议,或死人的血亲相争不肯认可,以至有三四次抬进火坑重验的。一个人的尸体多达三四次经火,肉色便都焦赤,伤损所在就更不清楚,仵作行人和吏役等便乘此机会行私舞弊。不到一两个月的时间,皮肉都溃烂无余了,及至死者家属有争议上诉,差下来会合检验的官员时,已是数月之久,止剩下骨架,肉体上的伤痕并不得而知。火坑验尸法独有湖南是这样,做提刑官的人是应当知道的。


\chapter{验未埋瘗尸首}

未埋尸首,或在屋内地上或床上,或屋前后露天地上,或在山岭、溪涧、草木上,并先打量顿尸所在,四至高低,所离某处若干。在溪涧之内,上去山脚或岸几许?系何人地上?地名甚处?若屋内,系在何处及上下有无物色盖簟?讫,方可  尸出验。
先剥脱在身衣服或妇人首饰,自头上至鞋袜,逐一抄劄。或是随身行李,亦具名件。讫,且以温水洗尸一遍了验。未要便用酒醋。
剥烂衣服洗了,先看其尸有无军号,或额角、面脸上所刺大小字体计几行,或几字?是何军人?若系配隶人,所配隶何州军字?亦须计行数。如经刺环,或方或圆,或在手臂、项上,亦记几个。内是刺字或环子?曾艾灸或用药取,痕迹黯漤及成疤瘢,可取竹,削一篦子,于灸处挞之可见。○辨验色目人讫,即看死人身上甚处有雕青、有灸瘢,系新旧疮疤?有无脓血?计共几个?及新旧官杖疮疤,或背或臀?并新旧荆杖子痕,或腿或脚底?甚处有旧疮疖瘢,甚处是见患?须量见分寸及何处有黯记之类,尽行声说。如无,亦开写。○打量尸首身长若干?发长若干?年颜若干?


(一)未经掩埋的尸体,或者是屋内的地上或床上,或者是在屋前屋后的露天地上,或者是在山岭溪涧中的草木上,都先要打量停尸所在的四至、高下,距离某处多少远近。如在溪涧之中,上面距离山脚或者崖岸多少远近?是在什么人的地上?地名是什么?如在屋内,是在什么地方?以及尸身上下有没有东西盖垫着?打量完了,才可以抬尸出来检验。
(二)验时,先要剥掉尸体的全身衣服,或妇人首饰,自头上到脚下的鞋袜,逐件点检登记。或者是随身行李,也要开列出名称件数。完毕后,且用温水洗尸一遍,乃验,未可便用酒醋。
(三)剥完衣服,冲洗尸体完毕,先看死者身上有没有军号,或额角上、脸孔上所刺的大小字体共计有几行或几字,是什么军的军人。如果是配隶人,是配隶到什么州?所刺的军字也要计算行数。如果是经过刺环的,是方或是圆,刺在手背上或是项上,也要计算是几个。其中如有刺字或环子曾经用艾灸过或用药起过,痕迹黯淡及成疤痕了的,可取竹削一篦子,打击灸过的地方,可以看见原刺字或环子。
在辨明死者的身分后,即开始看验死人身上什么地方有雕青,有灸瘢;是新旧疮疤,有没有脓血,共计有几个;以及是新旧官杖创疤,是在背部或是臀部;还有是新旧荆杖子伤痕,是在腿上或是脚底;什么是旧的疮疖瘢,什么地方是现患的,必须量出分寸。以及什么地方有黯记之类,全都要加以说明。如果没有,也要开写清楚。
还要打量尸首,身长有多少,发长有多少,年岁有多少。


\chapter{验坟内及屋下攒殡尸}

先验坟系何人地上?地名甚处?土堆一个,量高及长阔,并各计若干尺寸,及尸见 殡在何人至下,亦如前量之。
次看尸头、脚所向,谓如头东脚西之类,头离某处若干?脚离某处若干?左右亦如之。对众爬开浮土,或取去 砖,看其尸用何物盛簟。谓棺木有无漆饰、席有无沿禒及蕟蕈之类。舁出开拆,取尸于光明处地上验之。



(一)首先,对于坟内尸要验看坟山是在什么人的地上,地名什么去处。土堆一个,要量出高和长阔都各有多少尺寸。对于临时厝柩待葬的攒殡尸要验看现安厝在什么人的屋下,也如前量出高、长、阔的尺寸。
(二)其次,要验看尸体的头脚所向,例如头东脚西之类。头离某处多少,脚离某处多少,左右也是这样。是坟内尸当众扒开浮土,或是攒殡尸拆去厝丘的砖块,验看尸体是用什么东西盛殓的,如是棺木,有没有漆饰?如是席子,有没有饰边和粗衬席之类?然后,抬出开拆,取出尸体到光明的地上进行验验。


\chapter{验坏烂尸}

若避臭秽不亲临,往往误事。
尸首变动,臭不可近,当烧苍术、皂角辟之,用麻油涂鼻,或作纸摅子揾油塞两鼻孔,仍以生姜小块置口内。遇检,切用猛闭口,恐秽气冲入。○量扎四至讫,用水冲去蛆虫秽污,皮肉干净方可验。未须用糟、醋。频令新汲水浇尸首四面。
尸首坏烂,被打或刃伤处痕损皮肉作赤色,深重作青黑色,贴骨不坏,虫不能食。


(一)验坏烂尸体,如果怕臭避脏,不亲临现场检验,往往要误事。
(二)尸体发生腐败,臭气逼人,不可接近,可烧苍朮、皂角以袪除臭气。用香油涂抹鼻端,或做成空心纸捻沾油塞进两鼻孔,还要用生姜小块含放口内。临验,切要猛闭住口,恐秽气冲入。
丈量札记四至完毕,用水冲去蛆虫秽污,皮肉干净了,才好验。不要用糟醋可连续令人用新从井中打出来的水浇淋尸体四面。
(三)尸体坏烂,被打或刃伤地方的伤痕,皮肉作赤色,深重的作青黑色,贴骨不坏,虫不能食。


\chapter{无凭检验}

凡检验无凭之尸,宜说头发褪落,曲鬓、头面、遍身皮肉并皆一概青黑, 皮坏烂,及被蛆虫咂破骨殖显露去处。
如皮肉消化,宜说骸骨显露,上下皮肉并皆一概消化,只有些小消化不及筋肉与骨殖相连,今来委是无凭检复本人生前沿身上下有无伤损它故,及定夺年颜、形状、致死因依不得,兼用手揣捏得沿身上下并无骨损去处。


(一)凡验「无从检验」的尸体,宜说死者头发已经褪落,鬓角、脸面、周身皮肉,都已一概青黑,〈皮达〉皮坏烂,以及被蛆虫咂破,骨殖显露去处。
(二)如果皮肉腐烂了,宜说尸体骸骨显露,全身上下皮肉并皆一概腐烂,只有很少的地方没有烂掉,筋肉与骨相连。现在确实是无从检验覆验。死者本人生前周身上下皮肉有没有伤损其它原故,以及确定年龄、相貌形状、致死原因等,都不可能了。并经过用手按捏全身上下,也并无骨损去处。


\chapter{白僵死瘁死}

先铺炭火,约与死人长阔,上铺薄布,可与炭等。以水喷微湿,卧尸于上。仍以布覆盖头面、肢体讫,再用炭火铺拥令遍,再以布覆之,复用水遍洒。一时久,其尸皮肉必软起。乃揭所铺布与炭看,若皮肉软起,方可以热醋洗之。于验损处,以葱、椒、盐同白梅和糟研烂,拍作饼子,火内煨令热,先于尸上用纸搭了,次以糟饼罨之,其痕损必见。


先铺一层热炭灰,长阔约与死人相当,上铺薄布,可与炭灰大小相等,用水喷使微湿,平放尸体在上面。更用布覆盖尸体的头面肢体毕,再用热炭灰铺拥一遍,再覆盖上一层布,复用水遍洒使湿。过一个时辰左右,尸体的皮肉定会软化,乃揭开所铺布与炭察看。如果皮肉软化了,方可用热醋冲洗。在所要验的疑有伤损之处,用葱、椒、盐同白梅和糟研烂,拍成饼子,放在火内煨热,先在尸体上用纸衬垫好,再以糟饼罨敷上,伤痕定会现出。


\part{}

\chapter{验骨}

人有三百六十五节,按一年三百六十五日。
男子骨白,妇人骨黑。妇人生,骨出血如河水,故骨黑。如服毒药,骨黑。须子细详定。
髑髅骨,男子自顶及耳并脑后共八片,蔡州人有九片。脑后横一缝。当正直下至发际,别有一直缝。妇人只六片,脑后横一缝。当正直下无缝。
牙有二十四,或二十八,或三十二,或三十六。
胸前骨三条。
心骨一片,嫩如钱大。
项与脊骨,各十二节。
自项至腰共二十四 骨,上有一大 骨。
肩井及左右饭匙骨各一片。
左右肋骨,男子各十二条,八条长,四条短。
妇人各十四条。
男女腰间各有一骨大如手掌,有八孔,作四行。样:
手脚骨各二段。男子左右手腕及左右臁肕骨边皆有捭骨,妇人无。两脚膝头各有  骨隐在其间,如大指大。手掌、脚板各五缝,手脚大拇指并脚第五指各二节,余十四指并三节。
尾蛆骨若猪腰子,仰在骨节下。
男子者,其缀脊处凹,两边皆有尖瓣,如棱角,周布九窍。
妇人者,其缀脊处平直,周布六窍。
大小便处,各一窍。
骸骨各用麻、草小索或细篾串讫,各以纸签标号某骨,检验时不至差误。


(一)人的全身有骨头三百六十五节,按照一年三百六十五日之数。
(二)男子的骨头白,女人的骨头黑(女人生骨出血如河水,故骨黑。如服毒药骨黑,必须仔细检查验定)。
(三)髑髅骨(即脑颅骨),男子自头顶到两耳连同脑后共八片(蔡州人有九片),脑后横有一缝,当正直下到发际另有一条直缝;女人只有六片,脑后横一缝,当正直下没有缝。
(四)牙齿有二十四只,或二十八只,或三十二只,或三十六只。
(五)胸前骨(即胸骨)三条;
(六)心骨(即胸骨剑突)一片,软脆,像铜钱大。
(七)颈项骨与脊骨(即脊柱骨)各十二节。
(八)自颈项到腰共有二十四块椎骨,上有一块大椎骨。
(九)左右肩井骨(即锁骨)及左右饭匙骨(即肩胛骨)各一块。
(十)左右肋骨,男子各十二条,八条长,四条短,女人各十四条。
(十一)男女腰间各有一骨(即〈骨氐〉骨),大如手掌,有八孔作四行(样)。
(十二)手脚骨各二段,男子左右手腕及左右臁肕骨旁边,都有捭骨(女人没有);两腿的膝头各有〈奄页〉骨一块,隐在其间,如大拇指那样大;手掌脚板各有五缝,手脚大拇指和脚的第五指各二节,其余十四指都是三节。
(十三)尾蛆骨(即尾骨)像猪腰子,仰接在脊椎骨的末梢,男子的,联结脊椎骨处凹洼,两边都有尖瓣,像菱角,周围分布有九个孔窍;女人的,联结脊椎骨处平直,周围分布有六个孔窍。
(十四)大小便两处各有一个孔窍。
(十五)对于骸骨,各要用麻、草小绳或细竹篾串好,然后用纸签一一标明某骨,以使检验时不至发生差错。


\chapter{论沿身骨脉及要害去处}

夫人两手指甲相连者小节,小节之后中节,中节之后者本节,本节之后肢骨之前生掌骨,掌骨上生掌肉。掌肉后可屈曲者腕,腕左起高骨者手外踝,右起高骨者右手踝,二踝相连生者臂骨,辅臂骨者髀骨,三骨相继者肘骨,前可屈曲者曲肘,曲肘上生者臑骨,臑骨上生者肩髃,肩髃之前者横髃骨,横髃骨之前者髀骨,髀骨之中陷者缺盆。缺盆之上者颈,颈之前者颡喉,颡喉之上者结喉,结喉之上者胲,胲两傍者曲颔,曲颔两傍者颐。颐两傍者颊车,颊车上者耳,耳上者曲鬓,曲鬓上行者顶,顶前者囟门,囟门之下者发际,发际正下者额,额下者眉,眉际之末者太阳穴,太阳穴前者目,目两傍者两小眥,两小眥上者上脸,下者下脸,正位能瞻视者目瞳子,瞳近鼻者两大訾,近两大眥者鼻山根,鼻山根上印堂,印堂上者脑角,脑角下者承枕骨。脊骨下横生者髋骨,髋骨两傍者 骨, 下中者腰门骨, 骨上连生者腿骨,腿骨下可屈曲者曲 ,曲   上生者膝盖骨,膝盖骨下生者胫骨,胫骨傍生者 骨, 骨下左起高大者两足外踝,右起高大者两足右踝,胫骨前垂者两足跂骨,跂骨前者足本节,本节前者小节,小节相连者足指甲,指甲后生者足前趺,跌后凹陷者足心,下生者足掌骨。掌骨后生者踵肉。踵肉后者脚跟也。
检滴骨亲法,谓如某甲是父或母,有骸骨在,某乙来认亲生男或女,何以验之?试令某乙就身刺一两点血滴骸骨上,是的生亲则血沁入骨内,否则不入。俗云“滴骨亲”盖谓此也。
检骨须是晴明。先以水净洗骨,用麻穿定形骸次第,以簟子盛定。却锄开地窖一穴,长五尺、阔三尺、深二尺,多以柴炭烧煅,以地红为度。除去火,却以好酒二升、酸醋五升泼地窖内,乘热气扛骨入穴内,以藁荐遮定,烝骨一两时,候地冷取去荐,扛出骨殖向平明处,将红油伞遮尸骨验。○若骨上有被打处,即有红色路微荫,骨断处其接续两头各有血晕色。再以有痕骨照日看,红活乃是生前被打分明。○骨上若无血荫,踪有损折乃死后痕,切不可以酒醋煮骨,恐有不便处。此项须是晴明方可,阴雨则难见也。○如阴雨,不得已则用煮法:以瓮一口,如锅煮物,以炭火煮醋,多入盐、白梅同骨煎,须着亲临监视,候千百滚取出水洗,向日照,其痕即见,血皆浸骨损处,赤色、青黑色,仍子细验有无破裂。
煮骨不得见锡,用则骨多黯,○若有人作弊,将药物置锅内,其骨有伤处反白不见。解法见验尸门。
若骨或经三两次洗罨,其色白与无损同,何以辨之?当将合验损处骨以油灌之,其骨大者有缝,小者有窍,候油溢出,则揩令干,向明照:损处油到即停住不行,明亮处则无损。
一法:浓磨好墨涂骨上,候干,即洗去墨。若有损处则墨必浸入,不损则墨不浸。
又法:用新绵于骨上拂拭,遇损处必牵惹线丝起。折者其色在骨断处两头。又看折处其骨芒刺向里或外,殴打折者芒刺在里;在外者非。
髑髅骨,有他故处骨青,骨折处带淤血。
子细看骨上,有青晕或紫黑晕,长是他物,圆是拳,大是头撞。小是脚尖。四缝骸骨内,一处有损折系致命所在,或非要害,即令仵作行人指定喝起。
拥罨检讫,仵作行人喝四缝骸骨,调尸仰卧,自髑髅喝:顶心至囟门骨、鼻梁骨、胲、颔骨并口骨并全。两眼眶、两额角、两太阳、两耳、两腮 骨并全。两肩井、两臆骨全。胸前龟子骨、心坎骨全。
左臂、腕、手及脾骨全。右肋骨全。左胯、左腿、左臁肕并脾骨、及左脚踝骨、脚掌骨并全。右亦如之。
翻转喝:脑后、乘枕骨、脊下至尾蛆骨并全。
凡验元被伤、杀死人,经日尸首坏、蛆虫咂食、只存骸骨者,元被伤痕,血粘骨上,有干黑血为证。若无伤骨损,其骨上有破损如头发露痕,又如瓦器龟裂,沉淹损路为验。
殴死者死,伤处不至骨损。则肉紧贴在骨上,用水冲激亦不去,指甲蹙之方脱,肉贴处其痕损即可见。
验骨讫,自髑髅、肩井、臆骨并臂、腕、手骨,及胯骨、腰、腿骨、臁肕、膝盖并髀骨,并摽号左右。其肋骨共二十四茎、左右各十二茎。分左右,系左在第一、左第二,右第一、右第二之类,茎茎依资次题讫。内脊骨二十四节,亦自上题一、二、三、四,连尾蛆骨处号之。并胸前龟子骨、心坎骨亦号之,庶易于检凑。两肩、两胯、两腕皆有盖骨,寻常不系在骨之数,经打伤损方入众骨系数,不若拘收在数为良也。先用纸数重包定,次用油单纸三四重裹了,用索子交眼扎,系作三四处,封头印押讫,用桶一只盛之,上以板盖,掘坑埋瘗,作堆标记,仍用灰印。
行在有一种毒草,名曰贱草,煎作膏子售人。若以染骨,其色必变黑黯,粗可乱真。然被打若在生前,打处自有晕痕,如无晕而骨不损,即不可指以为痕。切须子细辨别真伪。



(一)从人的两手起,和指甲相连的是手指小节,小节之后是中节,中节之后是本节,本节之后前臂之前生着掌骨,掌骨的上面生着掌肉,掌肉之后可弯曲的地方是手腕,手腕外侧高起的骨头是手外踝,内侧高起的骨头是手内踝,和二踝相连生的是臂骨(挠骨),辅臂骨而生的是髀骨(即尺骨),三骨相接连的是肘骨,他的前面可弯曲的地方是曲肘,曲肘之上生着的是臑骨(即肱骨),臑骨之上生着的是肩颙(即肩头),肩颙之前是横颙骨,横颙骨之前的髀骨,髀骨之中凹陷的是缺盆(即肩窝),缺盆之上是脖颈,脖颈的前面是颡喉,颡喉之上是结喉,结喉之上是颏,颏的两旁是曲颔,曲颌的两旁是颐,颐的两旁是颊车,颊车之上是耳,耳之上是曲鬓,曲鬓向上到头顶,头顶之前是顖门,顖门之下是发际,发际的正下是额,额下是眉,眉际的末梢是太阳穴,太阳穴的前面是目,目的两外角是两小眦,两小眦的上面是上眼睑,下面是下眼睑,生在正中能够看视的是目瞳子,瞳子内侧靠近鼻根的眼角是两大眦,两大眦之间是鼻山根,鼻山根之上是印堂,印堂之上是脑角,脑角之下是承枕骨,脊柱骨的下面横生的是髋骨,髋骨两旁是钗骨,钗骨下中是腰门骨(即〈骨氐〉骨),钗骨上连生的是腿骨(即股骨),腿骨下可以弯曲的地方是曲〈月秋〉(即腘窝),曲〈月秋〉上面生的是膝盖骨,膝盖骨下生的是胫骨,胫骨旁生的是〈骨行〉骨(即腓骨),〈骨行〉骨下外起高大的是两足外踝,内起高大的是两足内踝,胫骨前面下垂的是两足跂骨(即跖骨),跂骨之前是足本节,本节之前是小节,小节的前上端生足指甲,指甲之后是足前跌,跌后凹陷的是足心,下面生足掌骨,掌骨之后生踵肉,踵肉后面是脚跟。
(二)滴骨验亲法,是说如果某甲是父或母,只余骸骨存在,有某乙前来认亲说自己是死者的亲生儿子或女儿,这要怎样来验定呢?可试令某乙就身上刺出一两点血,滴在骸骨上,如果是的亲生则血沁入骨内,否则不入,俗说滴骨亲,大约就是指这个说的了。
(三)检骨须是晴朗的日子。先以水净洗尸骨,用麻绳穿定身体各部骸骨的次第,用席盛好,可开掘地窖一个,长五尺,宽三尺,深二尺,多用柴炭烧煅,以地红为标准,除去火,再用好酒二升,酸醋五升,泼到地窖里面,乘热气抬骨放入坑内,以草垫盖好,蒸骨一两个时辰,等地冷了取去草垫,抬出骨殖向平坦明亮的地方,用红油伞罩尸骨进行检验。如果骨上有被打的地方,就会有红色纹路微印,骨折的地方,它的接续处两头各有血晕色,再以有痕的骨照着日光看,如果红活,便是生前被打分明。骨上如无血印踪迹,有损折,乃是死后痕。切不可用酒醋煮骨,恐有不便处。这种检骨的方法必须天气晴明才行,如果阴雨天那就难于检出了。
如阴雨天不得已,就用煮法。用瓮一口,如同用锅煮东西一样,用炭火来煮醋,多放入盐、白梅同骨一道煮。必须亲临现场监视,等煮到千百滚后取出,用水洗了,对着日头照,有伤痕即可见到。血都浸集在骨损的地方,赤色或青黑色。还要仔细验看,有没有破裂。
(四)煮骨不得见锡,接触到锡骨多变黯。
如果有人作弊,把药物放到锅里,那骨上有伤的地方,反而变白什么都不见。(解法见验尸部分)
如果骨头有的经过三两次洗罨,其色变白有伤也同没有伤一样了,这要怎样来辨验?可将应验伤损处的骨头,以油灌之,这种骨头大的有缝,小的有窍,候油溢出,便揩使干,向光明处照看,凡伤损的地方,油到便停住不动,明亮的地方便没有伤损。
一法,浓磨好墨涂抹在骨上,候干即洗去墨,如有伤损的地方,黑墨就一定浸入,没有伤损黑墨即不浸入。
又法,用新绵絮于骨上拂拭,遇伤损地方必定牵惹绵丝起。骨折的,其色在骨断处两头,还要看折断地方,其芒刺向里还是向外,殴打折断的,芒刺在里,在外的便不是。
头颅骨有其它问题的地方骨青,骨折的地方带有淤血。
仔细看骨上有青晕或紫黑晕的形状,长形的是他物伤,圆形的是拳伤,大块的是头撞伤,小块的是脚尖踢伤。
在全身上下前后左右所有的骸骨中,有一处损折是在致命部位,即或不是要害,就要叫仵作行人指定喝报出来。
(五)蒸骨检验完了后,仵作行人按前后左右由上而下部位顺序喝报全身骸骨,说:尸仰卧,自头颅喝起,顶心至顖门骨、鼻梁骨、颏颔骨和口骨都全,两眼眶、两额角、两太阳、两耳、两~骨都全,两肩井、两肊骨全,胸前龟子骨、心坎骨全。
左臂、腕、手及髀骨全,左肋骨全,左胯、左腿、左臁肕和髀骨以及左脚踝骨、脚掌骨都全。
右侧也如同左侧。
翻转身来喝报,脑后乘枕骨、脊下至尾蛆骨都全。
(六)凡验原被伤杀死的人,经过一些时日,尸首坏烂,蛆虫咂食,只剩下骸骨的,原被伤的地方,血粘骨上,有干黑血为证。如果无伤而骨损的,其骨上的破损,有如头发丝样的痕迹,又像瓦器龟裂,损伤的纹路深沉不显,可为验证。
(七)殴死的人,致死的伤处不至骨损的,便肉紧贴在骨上,用水冲激也不下来,以指甲剐剔才掉,肉贴的地方,其伤痕便可看出。
(八)检骨完了,自头颅、肩井肊骨,臂、腕、手骨,以及胯骨、腰腿骨,臁肕、膝盖和髀骨,都要标明左右。肋骨共二十四根,左右各十二根,分成左右,就是:左第一、左第二、右第一、右第二之类,要根根按照次序标写好。脊柱骨二十四节,也要自上而下一、二、三、四……,连尾蛆骨标明号数。胸前龟子骨、心坎骨也要写上标识,以便易于检点拼凑。两肩、两胯、两腕皆有盖骨,平常不计算在全身骨数之内,只在被打损伤的时候,才计算在众骨数内,实际上不如计算在全身骨数之内为好。对于标号过了的骨头,先用纸数重包好,再用油单纸三四层包了,用绳子交眼扎缚作三四处,盖上封头印,用桶一只盛起,上面用板盖好,挖坑掩埋,堆起坟堆,作上标记,并打上石灰印。
(九)行都临安一带地方,有一种毒草名叫贱草,熬成膏子出卖与人。如果用以染骨,骨色就变得黯黑,几乎可以和真伤相混。不过被打是在生前,伤处自有晕痕,如果没有晕痕而骨又不损,就不可指以为伤痕,务要仔细辨别真伪。


\chapter{自缢}

自缢身死者,两眼合、唇口黑、皮开露齿。若勒喉上,即口闭牙关紧,舌抵齿不出。又云:齿微咬舌。若勒喉下,则口开、舌尖出齿门二分至三分,面带紫赤色,口吻两甲及胸前有吐涎沫。两手须握大拇指,两脚尖直垂下,腿上有血荫,如火灸班痕,及肚下至小腹并坠下,青黑色。大小便自出。大肠头或有一两点血。喉下痕紫赤色或黑淤色,直至左右耳后发际,横长九寸以上至一尺以来。一云:丈夫合一尺一寸,妇人合一尺。脚虚则喉下勒深,实则浅。人肥则勒深,瘦则浅。用细紧麻绳、草索在高处自缢,悬头顿身致死则痕迹深;若用全幅勒帛及白练项帕等物,又在低处,则痕迹浅。低处自缢,身多卧于下,或侧或覆。侧卧,其痕斜起横喉下。覆卧,其痕正起在喉下,起于耳边,多不至脑后发际下。
自缢处须高八尺以上,两脚悬虚,所踏物须倍高。如悬虚处或在床、椅、火炉、船仓内,但高二三尺以来亦可自缢而死。
若经泥雨,须看死人赤脚或着鞋,其踏上处有无印下脚迹。
自缢有活套头、死套头、单系十字、缠绕系。须看死人踏甚物入头在绳套内,须垂得绳套宽入头方是。活套头,脚到地并膝跪地,亦可死。
死套头,脚到地并膝跪地,亦可死。
单系十字,悬空方可死;脚尖稍到地亦不死。
单系十字,是死人先自用绳带自系项上后,自以手系高处。须是先看上头系处尘土,及死人踏甚处物,自以手攀系得上向绳头着方是。上面系绳头处或高或大,手不能攀及不能上,则是别人吊起。更看所系处物伸缩,须是头坠下去上头系处一尺以上方是。若是头紧抵上头,定是别人吊起。
缠绕系,是死人先将绳带缠绕项上两遭,自踏高,系在上面垂身致死。或是先系绳带在梁栋或树枝上,双?垂下,踏高入头在?内。更缠过一两遭,其痕成两路,上一路缠过耳后斜入发际,下一路平绕项。行吏畏避驳杂,必告检官,乞只申一痕。切不可信。若除了上一痕,不成自缢;若除下一痕,正是致命要害去处。或复检官不肯相同书填格目,血属有词,再差官复检出,为之奈何?须是据实,不可只作一条痕检。其相叠与分开处,作两截量,尽取头了,更重将所系处绳带缠过比并,阔狭并同,任从复检,可无后患。
凡因患在床仰卧,将绳带等物自缢者,则其尸两眼合,两唇皮开、露齿,咬舌出一分至二分,肉色黄,形体瘦,两手拳握,臀后有粪出,左右手内多是把自缢物色,至系紧死后只在手内,须量两□手拳相去几寸以来,喉下痕迹紫赤,周围长一尺余。结缔在喉下,前面分数较深,曾被救解则其尸肚胀,多口不咬舌,臀后无粪。
若真自缢,开掘所缢脚下穴三尺以来,究得火炭方是。
○或在屋下自缢,先看所缢处楣梁枋桁之类,尘土衮乱至多方是。如只有一路无尘,不是自缢。
○先以杖子于所系绳索上轻轻敲,如紧直乃是。或宽慢即是移尸。大凡移尸别处吊挂,旧痕挪动便有两痕。
凡验自缢之尸,先要见得在甚地分、甚街巷、甚人家?何人见本人?自用甚物?于甚处搭过?或作十字死?系定,或于项下作活?套,却验所着衣新旧,打量身四至东西南北至甚物?面觑甚处?背向甚处?其死人用甚物踏上?上量头悬去所吊处相去若干尺寸。下量脚下至地相去若干尺寸。或所缢处虽低,亦看头上悬挂索处下至所离处,并量相去若干尺寸,对众解下,扛尸于露明处,方解脱自缢套绳,通量长若干尺寸,量围喉下套头绳围长若干,项下交围,量到耳后发际起处阔狭、横斜、长短,然后依法检验。
凡验自缢人,先问元申人,其身死人是何色目人?见时早晚?曾与不曾解下救应?申官时早晚?如有人识认,即问自缢人年若干?作何经纪?家内有甚人?却因何在此间自缢?若是奴仆,先问雇主讨契书辨验。仍看契上有无亲戚?年多少?更看元吊挂踪迹去处。如曾解下救应,即问解下时有气脉无气脉?解下约多少时死?切须子细。
大凡检验,未可便作自缢致命,未辨子细。凡有此,只可作其人生前用绳索系咽喉下或上要害,致命身死,以防死人别有枉横。且如有人睡着,被人将索勒死吊起所在,其检官如何见得是自缢致死?宜子细也!
多有人家女使人力或外人,于家中自缢,其人不晓法,避见臭秽及避检验,遂移尸出外吊挂,旧痕移动,致有两痕。旧痕紫赤有血荫,移动痕只白色无血荫,移尸事理甚分明,要公行根究,开坐生前与死后痕,盖移尸不过杖罪,若漏落不具,复检官不相照应,申作两痕,官司必反见疑,益重干连人之祸。
尸首日久坏烂,头吊在上,尸侧在地,肉溃见骨,但验所吊头,其绳若入槽,谓两耳连颔下深向骨本者。及验两手腕骨、头脑骨皆赤色者是。一云:齿赤色,及十指尖骨赤色者是。



(一)自缢身死的人,两眼闭合,嘴唇青黑,唇开露齿。如果是勒在喉头之上,就口闭,牙关咬紧,舌头抵住牙齿不出来(有的说齿微咬舌)。如果是勒在喉头之下,便口开,舌头伸出口外约二分至三分。面带紫赤色,口吻两角以及胸前有吐出的涎沫。两手虚握,大拇指、两脚尖直垂下。腿上有血荫,如同火灸的斑痕,以及肚下至小腹都因血液下坠成为青黑色。大小便自出,大肠头(即肛门)有的有一两点血。脖子上的勒痕呈紫色,或黑淤色,直到左右耳后的发际,横长约九寸以上到一尺以来(一说男子合一尺一寸,妇人合一尺)。脚下悬空,脖子上勒的沟就深;脚不悬空,就浅。人肥就勒的深;瘦就浅。用细紧麻绳、草索在高的地方上吊,身悬半空,勒的痕迹就深;如用全幅勒帛及白绢项帕等物,又在低的地方,勒的痕迹就浅。在低的地方上吊,身体多卧在下面,或者侧身卧,或者覆身卧。侧身卧的,索痕斜起,旁经喉下;覆身卧的,索痕正起,横在喉下,从两耳边上去,大多不到脑后发际下。
上吊挂绳的地方需要高八尺以上,两脚悬空,所踏着上去的东西须倍高于悬空的高度。或在床、椅、火炉、船舱内,只要高二三尺以来,也可以自缢而死。
如果死者去上吊处经过一段烂泥地,需要看死人赤脚还是穿鞋,他所踏着上去的东西上有没有印下脚迹。
(二)上吊,有活套头、死套头、单系十字、缠绕系等。要看死人踏什么东西入头在绳套内。垂下的绳套必须宽裕得使死者站在踏物上足以放头进去的程度,才是自行上吊死的。
活套头,脚着地及膝跪地,也可以吊死。
死套头,脚着地及膝跪地,也可以吊死。
单系十字,悬空,才可以吊死,脚尖稍为着地,也吊不死。
单系十字,是死者先用绳带自行系在项上后,自己用手系挂到高的地方。检验时,需要先看上头系挂地方的尘土,以及死者上去系绳时踏的什么东西,要自己用手能够够得着系绳的地方,才是自行吊死的。上面系绳的地方,或者太高,或者太大,手不能攀得着以及不能上得去的,那就是别人吊上去的。其次,还要看所系挂处绳带物的伸缩,需要套头坠下距上头系挂处约长一尺以上,才是自己上吊死的。如果套头紧抵着上头系处,一定是别人吊上去的。
缠绕系,是死者先用绳带在颈项上缠绕两遭,自己踏着东西系在高的地方,身体下坠致死。或者是先系绳带在梁栋或树枝上,扣套垂下,踏高入头在扣套内,再缠上一两遭。缠绕系的索痕成上下两路,上一路绕过耳后,斜入发际;下一路平绕项行。吏人畏惧复杂艰难,定会禀告验官,请求只申报一条索痕,切不可听信。如果除去上一痕,便不成为上吊,如果除去下一痕,却正是致命的要害去处。倘或覆检官不肯也和初检一样填写验尸单,死者的亲属因而有所上诉,上级再差官覆检出来,那将怎么办呢?必须据实办理,不可只作为一条索痕定验。索痕的相叠和分开的地方,要作两截量,定出头尾,画取样子,更要重新用原来系吊的绳带缠过,比较阔狭完全相同,这样,任凭覆检,可无后患。
(三)凡因病患在床仰卧,用绳带等物上吊死的,这种尸体两眼闭合,唇开露齿,咬舌出一分至二分,肤色黄,形体瘦,两手拳握,臀后有大便出。左右手内,大多数是握着自缢用物到勒紧以后,死后只在手内。要量一量两手拳相距几寸以来。喉下痕迹紫赤,周围长约一尺多,结缔在喉下,前面的痕迹比较深。曾经被解救过,则其尸体肚胀,大多口不咬舌,臀后没有大便。
(四)如果是真自缢,在吊死者脚下开掘穴深三尺以来,找到火炭,才是。
或在屋下自缢,先看所缢处的横梁等处,尘土滚乱极多,才是。如果只有一路无尘,不是自缢。
可先用杖子在所系吊的绳索上轻轻敲打,如果紧直,乃是自缢。如宽松,就是移尸。大凡移尸到别处吊挂,旧痕挪动,便有两痕。
(五)凡验上吊死的尸体,先要弄清在什么地区,什么街巷,什么人家,什么人看见,本人自用什么东西,在什么地方搭挂,是作十字死扣系定,还是在项下作活扣套。再验死者所着的衣服新旧,打量尸体所在的四至,东西南北各至什么所在,面向什么地方,背向什么地方。死人是踏什么东西上去的,上量吊者头部去系绳处相距若干尺寸,下量死者脚到地面相去若干尺寸。或所吊处虽低,也要看头上悬挂绳索处,下至所离处,并量相去若干尺寸。当众解下,抬尸到明亮的地方,才解去上吊者的绳套,通量长若干尺寸。量围在脖子上的套头绳围长若干,项下交围的,量到耳后发际起处,量索痕阔狭、横斜长短,然后按照常法进行检验。
(六)凡验上吊死的人,先要问原报案人,该身死人是什么身分的人?发现时早晚?有没有解下救应过?报告官府时早晚?如果有人认识,即问上吊人年岁多少?作什么营生?家里有什么人?因为什么在这里上吊?如果是奴仆,先向雇主索讨契约辨验,看契约上奴仆有没有亲戚,年岁多少,更看原吊挂踪迹去处。如果曾经解下救过,即问解下时有气脉无气脉,解下约多少时间死去,切须仔细。
(七)大凡检验,不可便作上吊致命,不加仔细辨别。凡遇这类情况,只可作其人生前用绳索系咽喉下或咽喉上的要害部位,致命身死,以防死者别有屈枉情节。况且如有人在睡熟的时候,被人用绳索勒死吊将起来,检验官怎么见得是自行上吊致死?一定要仔细啊。
(八)多有人家女使、力役,或外人在家中吊死,家主不通晓法律,因避见臭秽及逃避检验,遂移尸到外面吊挂,旧的索痕移动了,以致有了两条索痕。旧痕紫赤有血荫,移动后的索痕只白色,没有血荫。移尸的事理很清楚,要当众查究明白,开写出生前与死后痕。因为移尸不过杖罪,如果漏脱不写清楚,覆检官不相照应,报为两痕,上级必然反而见疑,愈益加重了本案有关人等的灾祸。
(九)遇有尸首日久坏烂,头吊在上,身体侧斜在地,肉烂见骨的情况,但验所吊头。如果绳已入糟(指自两耳连颔下深向骨本的),以及两手腕骨、头脑骨都呈现为赤色的,是吊死的。(一说牙齿及十指尖骨赤色的为是。)


\chapter{被打勒死假作自缢}

 自缢、被人勒杀或算杀假作自缢,甚易辨。真自缢者,用绳索、帛之类系缚处,交至左右耳后,深紫色,眼合、唇开、手握、齿露,缢在喉上则舌抵齿,喉下则舌多出,胸前有涎滴沫,臀后有粪出。若被人打勒杀假作自缢,则口、眼开,手散,发慢,喉下血脉不行,痕迹浅淡,舌不出,亦不抵齿,项上肉有指爪痕,身上别有致命伤损去处。
惟有生勒,未死间即时吊起,诈作自缢,此稍难辨。如迹状可疑,莫若检作勒杀,立限捉贼也。
凡被人隔物,或窗棂、或林木之类勒死,伪作自缢,则绳不交喉下,痕多平过却极深,黑黯色,亦不起于耳后发际。
绞勒喉下死者,结缔在死人项后,两手不垂下。纵垂下亦不直。项后结交却有背倚柱等处。或把衫襟 着,即喉下有衣衫领黑迹,是要害处,气闷身死。
凡检被勒并死人,将项下勒绳索或是诸般带系,临时子细声说缠绕过遭数。多是于项后当正或偏左右系定,须有系不尽垂头处。其尸合面、地卧,为被勒时争命,须是揉扑得头发或角子散慢,或沿身上有搕擦着痕。
凡被勒身死人,须看觑尸身四畔,有扎磨踪迹去处。
又有死后被人用绳索系扎手脚及项下等处,其人已死气血不行,虽被系缚,其痕不紫赤,有白痕可验。死后系缚者无血荫,系缚痕虽深入皮,即无青紫赤色,但只是白痕。
有用火篦烙成痕,但红色或焦赤,带湿不干。



(一)自缢的和被人勒杀或谋害后伪装成自缢的,是很容易辨别的。真自缢的,用绳索、帛之类系缚处,交至左右耳后,索痕深紫色,眼合唇开,手握露齿,绳索勒在喉上的则舌抵齿,勒在喉下的则舌多出,胸前有涎滴沫,臀后有粪便出。假如是被人打勒杀,假作自缢的,则口眼开,手散发乱,喉下血脉不行,痕迹浅淡,舌不出,也不抵齿,项肉上有指爪痕,身上别有致命伤损去处。
惟有勒到将死未死的当儿,实时吊起,诈作自缢的,这要稍难辨别。如果迹状可疑,莫如即验成勒杀,立下限期捕捉凶犯为好。
(二)凡被人隔物,或者窗棂或者林木之类勒死,伪作自缢的,则绳不交,喉下痕多平过,但极深,黯黑色,也不起于耳后发际。
绞勒喉下死的,结缔在死人项后,两手不垂下,即使垂下也不直,项后结交,却有背倚柱等处或把衫襟搊着。假如喉下有衣衫领黑迹,是咽喉要害地方被压迫以致气闷身死的。
(三)凡检验被勒身死的人,要把项下勒的绳索,或是各种带系,临时仔细说明。绳带缠绕过遭数,多是在项后当正或偏左右系定,须有系不尽的垂头处。其尸应当覆身面朝地卧,因为被勒时争命,一定是揉扑得头发或角子散慢,或沿身上有搕擦着痕。
(四)凡被勒身死的人,须观察其尸身的四侧,当有绑挣扎的踪迹去处。
(五)又有死后被人用绳索绑扎手脚及项下等处的,其人已死,气血不行,虽被系绑,其痕迹也不紫赤,有白痕可作验证。死后系绑的,无血荫,系绑痕虽深入皮,便无青紫赤色,但只是白痕。
有用火篦烙成痕的,只红色或焦赤,带湿不干。


\chapter{溺死}



 若生前溺水尸首,男仆卧、女仰卧。头面仰,两手两脚俱向前。口合,眼开闭不定,两手拳握,腹肚胀,拍作响,落水则手开、眼微开、肚皮微胀;投水则手握、眼合、腹内急胀。两脚底皱白不胀,头髻紧,头与发际、手脚爪缝,或脚着鞋则鞋内各有沙泥,口、鼻内有水沫及有些小淡色血污,或有搕擦损处,此是生前溺水之验也。盖其人未死,必须争命,气脉往来搐水入肠,故两手自然拳曲,脚罅缝各有沙泥,口、鼻有水沫流出,腹内有水胀也。
若检复迟,即尸首经风日吹晒,遍身上皮起,或生白疱。
若身上无痕,面色赤,此是被人倒提水揾死。
若尸面色微赤,口、鼻内有泥水沫,肚内有水,腹肚微 胀,真是渰水身死。
若因病患溺死,则不计水之深浅可以致死,身上别无它故。

若疾病身死,被人抛掉在水内,即口、鼻无水沫,肚内无水不胀,面色微黄,肌肉微瘦。
若因患倒落泥渠内身死者,其尸口、眼开,两手微握。身上衣裳并口、鼻、耳、发际并有青泥污者,须脱下衣裳用水淋洗,酒喷其尸,被泥水淹浸处即肉色微白,肚皮微胀,指甲有泥。
若被人殴打杀死推在水内,入深则胀,浅则不甚胀。其尸肉色带黄不白,口、眼开,两手散,头发宽慢,肚皮不胀,口、眼、耳、鼻无水沥流出,指爪罅缝并无沙泥,两手不拳缩,两脚底不皱白却虚胀。身上有要害致命伤损处,其痕黑色,尸有微瘦。临时看验。若检得身上有损伤处,录其痕迹。虽是投水,亦合押合干人赴官司推究。
诸自投井、被人推入井、自失脚落井尸首,大同小异,皆头目有被砖石磕擦痕,指甲、毛发有沙泥,腹胀,侧覆卧之则口内水出,别无它故,只作落井身死,即投井、推入在其间矣。所谓落井,小异者:推入与自落井则手开、眼微开,腰身间或有钱物之类;自投井则眼合、手握、身间无物。
大凡有故入井,须脚直下。若头在下,恐被人赶逼或它人推送入井。若是失脚,须看失脚处土痕。
自投河、被人推入河,若水稍深阔,则无磕擦沙泥等事。若水浅狭,亦与投井、落井无异。大抵水深三四尺皆能渰杀人,验之果无它故,只作落水身死,则自投、推入在其间矣。若身有绳索及微有痕损可疑,则宜检作被人谋害置水身死,不过立限捉贼,切勿恤一捕限而贻罔测之忧。
诸溺河池,行运者谓之河,不行运者谓之池。检验之时先问元申人:早晚见尸在水内?见时便只在今处或自漂流而来?若是漂流而来,即问是东西南北?又如何流到此便住?如何申官?如称见其人落水,即问当时曾与不曾救应?若曾救应,其人未出水时已死或救应上岸才死?或即申官或经几时申官?若在江河、陂潭、池塘间,难以打量四至,只看尸所浮在何处。如未浮,打捞方出,声说在何处打捞见尸。池塘或坎阱有水处可以致命者,须量见浅深丈尺,坎阱则量四至。江河、陂潭,尸起浮或见处地岸并池塘、坎阱,系何人所管?地名何处?
诸溺井之人,检验之时亦先问元申人:如何知得井内有人?初见有人时其人死未?既知未死,因何不与救应?其尸未浮,如何知得井内有人?若是屋下之井,即问身死人自从早晚不见?却如何知在井内?凡井内有人,其井面自然先有水沫,以此为验。
量井之四至,系何人地上?其地名甚处?若溺尸在底则不必量,但约深若干丈尺,方摝尸出。
尸在井内,满胀则浮出尺余,水浅则不出。若出,看头或脚在上在下,先量尺寸。不出,亦以丈竿量到尸近边尺寸,亦看头或脚在上在下。
检溺死之尸,水浸多日,尸首胖胀,难以显见致死之因,宜申说头发脱落、头目胖胀、唇口番张,头面连遍身上下皮血,并皆一概青黑褪皮。验是本人在井或河内死后,水浸经隔日数致有此,今来无凭检验本人沿身有无伤损它故,又定夺年颜形状不得,只检得本人口鼻内有沫、腹胀,验得前件尸首委是某处水溺身死。其水浸更多日,无凭检验,即不用申说致命因依。
初春雪寒,经数日方浮,与春夏秋末不侔。
凡溺死之人,若是人家奴婢或妻女,未落水先已曾被打,在身有伤,今次又的然见得是自落水或投井身死,于格目内亦须分明具出伤痕,定作被打复溺水身死。
投井死人,如不曾与人交争,验尸时面目、头额有利刃痕,又依旧带血,似生前痕,此须看井内有破瓷器之属以致伤着人,初入井时,气尚未绝,其痕依旧带血,若验作生前刃伤,岂不利害?


(一)如果是生前溺水的尸首,男仆卧,女仰卧,头面后仰,两手两脚俱向前,口合,眼开闭不定,两手拳握,肚腹鼓胀,拍着发响,(落水的则手开,眼微开,肚皮微胀。投水的则手握,眼合,腹内急胀。)两脚底皱白不胀,发髻紧,头与发际、手脚爪缝,或脚着鞋则鞋内各有泥沙,口鼻内有水沫及有些小淡色血污,或有搕擦损处。这是生前溺水的验证。(因为溺水的人未死前必须争命,由于呼吸关系,便要吸水入肚,所以死者两手拳曲,脚罅缝各有沙泥,口鼻有水沫流出,肚子里有水胀。)
(二)如果检验覆验迟了,就会因尸首经受风日吹晒,遍身上皮起,或生白疱。
(三)如果尸首身上无痕,面色发赤,这是被人倒提起来用水闷死的。
(四)如果尸首面色微赤,口鼻内有泥水沫,肚内有水,肚腹微胀的,是真的淹水身死。
(五)如因病患溺死,则不论水的深浅,可以致死,身上别无其它情况。
(六)如果是疾病身死,被人拋掷在水里的,就口鼻没有水沫,肚内没有水,不鼓胀,面色微黄,肌肉微瘦。
(七)如果是因为病患倒落泥塘里身死的,其尸口眼开,两手微握。身上衣裳并口、鼻、耳、发际同有青泥污的,要脱下衣裳,用水淋洗,以酒喷尸,被泥水淹浸的地方,即肉色微白,肚皮微胀,指甲中有泥。
(八)如果是被人殴打杀死后,推在水里的,入水深的则胀,浅的则不大胀,其尸肉色带黄不白,口眼开,两手散,头发散慢,肚皮不胀,口、眼、耳、鼻没有水沥流出,指爪罅缝没有沙泥,两手不拳缩,两脚底不皱白,但虚胀。身上有要害致命的伤损去处,其痕黑色,尸首有的微瘦。临时看验,如果检出身上有伤损地方,便把痕迹记录下来,虽是投水,也应该押送所有关系人等到官府查问追究。
(九)凡自投井,被人推入井,自失脚落井的尸首,大同小异,都是头目有被砖石磕擦痕,指甲、毛发中有沙泥,肚子胀。使尸体侧覆卧,便口内水出。如果别无他故,就只定作落井身死,这样,投井、推入井的都包括在其中了。所谓落井小有不同的,只在于被推入井与自落井的,便手开眼微开,腰身间或有钱物之类;自投井的,便眼合手握,身上没有钱物。
大凡有原故入井的,须脚直下,如果头在下,恐怕是被入赶逼,或他人推送入井的。如果是失脚的,要看失脚处的土痕。
(十)自投河、被人推入河的,如果水稍深阔,便没有磕擦、沙泥等现象;如果水浅狭,也和投井、落井相同。大概水深三四尺的光景,都能淹死人,验了确实没有其它原因,就只定作落水身死,自投水被推入水的便都包括在这里面了。如果尸体上有绳索或略有伤痕可疑之处,就应当验作被人谋害,放在水中弄死,不过立下捕限捉拿贼犯,切不要顾及一个捕限,而留下不可预测的后患。
(十一)凡溺于河池(通行航运的叫做河,不通行航运的叫做池)的,检验时候,先讯问原报案人,早晚看到尸体在水里的?看到时便只在现在地方,或是从其它地方漂流而来的?如果是漂流而来,即问是东、西、南、北方向?又怎样流到这里便停住的?怎样来报官的?假如称说曾看到其人落水,就问当时有没有救应过?如果曾经救应过,其人没有出水的时候就已死了,还是救应上岸后才死的?是看到后立即报官,或是过了几时报官的?
(十二)如果在江、河、陂、潭、池塘里面,难以打量四至的,只看尸首浮起在什么地方,如果尸未浮起打捞才出的,说明在什么地方打捞见尸。池塘或坎穽有水地方可以致命的,要量出浅深尺寸,坎穽则量明四至。江、河、陂、潭尸体起浮或发现处地岸,以及池塘坎穽,为什么人所管?地名何处?
(十三)凡溺井的人,检验时候,也先问原报案人,怎样知道井里有人的?初见有人时,其人死了没有?既知没有死,为什么不给以救应?死者的尸体没有浮起,怎样知道井里有人的?如果是屋下之井,即问身死人自从什么时候不见的?却又怎样知道在井里?凡井内有人,其井内自然先有水沫,以此为验。
量井的四至,是在什么人的地上?其地名甚处?如果溺尸在井底,就不必量四至,但约量井深若干丈尺,才捞尸出。
尸体在井内,满胀则浮出尺余,水浅则不出。如果浮出,要看头或脚在上在下,先量尺寸。不浮出,也要用丈竿量到尸体近边的尺寸,也要看头或脚在上在下。
(十四)检验溺死之尸,由于水浸多日,尸体膨胀,难以明显看出致死原因的,应当详细说明尸体头发脱落,头脸膨胀,口唇翻张,头脸连同周身上下皮肉,并皆一概青黑褪皮的情况。验明是死者本人在井内或河内,死后水浸,经隔多日,以致这样。现在没有凭据检验尸体沿身有无伤损及其它问题,也决定不了死者的年龄面貌形状,只检得本人口鼻内有沫,腹胀。验得前件尸者,确实是在某处水溺身死。那种水浸日数更多没有凭据检验的,便不用详细说明致命原由。
(十五)初春雪寒,尸体经过数天才浮,和春夏秋末不相同。
(十六)凡溺死的人,如果是人家奴婢或妻女,未落水前先已曾经被打,本身有伤,现在又的确证实是自落水或投井身死的,在验尸单内也要明白填写出伤痕,定作被打后又溺水身死的。
(十七)投井死的人,如果不曾和人争斗过,而验尸时面孔头额有利刃痕,又依旧带血,好象生前伤痕的,这要看井内有没有破瓷器之类的东西,以致伤着了。人初入井时气还未绝,伤着了,其伤痕依旧带血,如果验成了生前刃伤,岂不利害!


\part{}

\chapter{验他物及手足伤死}

律云:“见血为伤。非手足者,其余皆为他物,即兵不用刃亦是。”
○伤损条限:“手足十日,他物二十日。”
斗讼敕:“诸啮人者,依他物法。”
元符敕《申明刑统》:“以靴鞋踢人伤,从官司验定:坚硬即从他物;若不坚硬即难作他物例。”
○或额、肘、膝拶,头撞致死,并作他物痕伤。
○诸“他物”,是铁鞭、尺、斧头、刃背、木杆、棒、马鞭、木柴、砖、石、瓦、粗布鞋、衲底鞋、皮鞋、草鞋之类。
若被打死者,其尸口、眼开,发髻乱,衣服不齐整,两手不拳,或有溺污内衣。
若在辜限外死,须验伤处是与不是在头,及因破伤风灌注致命身死。
应验他物及手足殴伤痕损,须在头面上、胸前、两乳、胁肋傍、脐腹间、大小便二处,方可作要害致命去处。手足折损亦可死。其痕周匝有血荫方是生前打损。
诸用他物及头、额、拳手、脚足坚硬之物撞打,痕损颜色,其至重者紫黯微肿,次重者紫赤微肿,又其次紫赤色,又其次青色。其出限外痕损者,其色微青。
凡他物打着,其痕即斜长或横长。如拳手打着即方圆。如脚足踢,比如拳寸分寸较大。凡伤痕大小,定作掌、足、他物,当以上件物比定,方可言分寸。凡打着两日身死,分寸稍大,毒气蓄积向里,可约得一两日后身死。若是打着当下(禁止)死,则分寸深重,毒气紫黑,即时向里,可以当下(禁止)死。
诸以身去就物谓之“磕”,虽着无破处,其痕方圆,虽破亦不至深。其被他物及手足伤,皮虽伤而血不出者,其伤痕处有紫赤晕。
凡行凶人若用棒杖等行打,则多先在实处,其被伤人或经一两时辰,或一两日、或三五日以至七八日、十余日身死。又有用坚硬他物行打便至身死者,更看痕迹轻重。若是先驱捽被伤人头髻,然后散拳踢打,则多在虚怯要害处,或一拳一脚便致命。若因脚踢着要害处致命,切要子细验认行凶人脚上有无鞋履,防日后问难。
凡他物伤,若在头脑者,其皮不破,即须骨肉损也。若在其他虚处。即临时看验。若是尸首左边损,即是凶身行右物致打,顺故也。若是右边损,即损处在近后,若在右前即非也。若在后,即又虑凶身自后行他物致打。贵在审之无失。
看其痕大小,量见分寸,又看几处皆可致命,只指一重害处,定作虚怯要害致命身死。
打伤处,皮膜相离,以手按之即响。以热醋罨,罨则有痕。
凡被打伤杀死人,须定最是要害处致命身死。若打折脚手,限内或限外死时,要详打伤分寸,阔狭,后定是将养不较致命身死。面颜、岁数临时声说。凡验他物及拳、踢痕,细认斜长、方圆。皮有微损,未洗尸前用水洒湿,先将葱白捣烂涂,后以醋、糟,候一时除,以水洗,痕即出。
若将榉木皮罨成痕假作他物痕,其痕内烂损、黑色,四围青色,聚成一片而无虚肿,捺不坚硬。
又有假作打死,将青竹篦火烧烙之,却只有焦黑痕,又浅而光平。更不坚硬。



(一)律上说:见血为伤,除了手脚打踢的而外其余都为他物伤,就是兵器不用锋刃的也是。
伤损的法定责任担保期限,手足打的伤十天,他物打的伤二十天。
斗讼敕:凡咬人的,依照「他物法」处理。
元符敕申明刑统:用靴鞋踢人的,由官吏检验确定,如果靴鞋坚硬,就按「他物」处理,如果不坚硬,就难作「他物」看待。
或额、肘、膝抵压、头撞致死的,并作「他物」伤痕。
诸他物是铁鞭、尺、斧头、刀背、木杆棒、马鞭、木柴、砖、石、瓦、粗布鞋、衲底鞋、皮鞋、草鞋之类。
(二)如果被打死的,其尸口眼开,发髻乱,衣服不齐整,两手不拳,有的有便溺沾污内衣。
(三)如果在责任担保期限之外死的,要验伤的部位是不是在头部,及因破伤风感染,以致身死的。
(四)应验他物和手足殴伤,伤痕必须是在头脸上、胸前、两乳、胁肋旁、脐腹间、大小便二处的,才可定作要害致命的去处。手足折损的也可以死,其伤痕周围有血荫的,才是生前打损的。
(五)凡是用他物和头、拳、脚等坚硬之物撞打的,伤痕颜色最重的紫黯微肿,次重的紫赤微肿,又其次的紫赤色,再其次的青色,那种出了责任担保期限以外的伤痕,其颜色微青。
凡是他物打的,其伤痕即斜长或横长,如果是拳打的即方圆,如果是脚踢的,则比拳打的分寸较大。(凡伤痕大小定作拳足他物打的,应当以上项各物比定,才可谈分寸。)凡打着两天身死的,则伤痕分寸较大,毒气蓄积向里,可约得一两天后身死;如果是打着当下身死的,则分寸深重,毒气紫黑,实时向里,可以当下身死。
(六)凡是以身去就物的叫做磕。虽磕着,没有破处,其伤痕方圆,虽磕破也不至于深。那种被他物和手足所伤,皮虽受伤而血不出的,其伤痕处有紫赤晕。
(七)凡行凶人假如是用棒杖等打的,则伤处多在实在的地方,其被伤的人,或经过一两个时辰,或经过一两天、三五天以至七八天、十多天身死。又有用坚硬的他物打,立即便致身死的,更要看验伤痕情况轻重。如果是先揪住被伤人的头髻,然后散拳踢打的,则伤处多在虚弱要害的地方,或一拳一脚便致命。如果是因为脚踢着要害地方致命的,切要仔细验认行凶人脚上有无鞋履,防备日后质问非难。
(八)凡他物伤,如果是在头脑部位的,其皮不破,即要骨肉伤损。如果是在其它虚弱地方,即临时看验。如果是尸首左边伤损,即是行凶人用右手拿东西所打,这是因为手顺的原故。如果是右边伤损,即损处在近后,如在右前,即不是。如果是在后面,便又要考虑到是凶手从身后用他物所打。贵在审察无误。
(九)要看伤痕大小,量出分寸,又要看几处皆可致命,只指定一处重害地方,作为虚弱要害致命身死之处。
(十)打伤地方皮膜相剥离的,用手按之即作响,用热醋罨敷之便有伤痕出现。
(十一)凡被打伤致死的人,必须定出致命身死的最要害的地方。如果是打折了脚手,在责任担保期限内或限外死去时,要仔细考察研究打伤的分寸阔狭后,定验是将养不好,致命身死。面貌年岁等,也要临时加以说明。
(十二)凡验他物及拳打脚踢的伤,要仔细辨验伤痕是斜长方圆,表皮有否微损。如果伤痕不见,在未洗尸前,用水洒湿,先将葱白捣烂涂上,然后再用醋糟罨敷,经过一个时辰除去,以水洗尸,伤痕即现出。
(十三)如果是用榉树罨敷成痕,假作为他物伤痕的,则其痕内烂损黑色,四围青色,聚作一片,而没有虚肿现象,用手按捺也不坚硬。
(十四)又有假作打死的,在尸上用青竹篦火烧烙成伤痕,但只有焦黑痕,又浅而光平,更不坚硬。


\chapter{自刑}

 凡自割喉下死者,其尸口、眼合,两手拳握,臂曲而缩,死人用手把定刃物,似作力势,其手自然拳握。肉色黄,头髻紧。
若用小刀子自割,只可长一寸五分至二寸。用食刀,即长三寸至四寸以来,若用磁器,分数不大。逐件器刃自割,并下刃一头尖小,但伤着气喉即死。若将刃物自斡着喉下、心前、腹上、两胁肋、太阳、顶门要害处,但伤着膜,分数虽小即便死。如割斡不深及不系要害,虽两三处未得致死。若用左手,刃必起自右耳后,过喉一二寸。用右手,必起自左耳后。伤在喉骨上难死,盖喉骨坚也。在喉骨下易死,盖喉骨下虚而易断也。○其痕起手重、收手轻。假如用左手把刃而伤,则喉右边下手处深,左边收刃处浅,其中间不如右边,盖下刃大重,渐渐负痛缩手,因而轻浅,及左手须似握物是也。右手亦然。
凡自割喉下,只是一出刀痕。若当下(禁止)死时,痕深一寸七分,食系气系并断。如伤一日以下(禁止)死,深一寸五分,食系断,气系微破。如伤三五日以后死者,深一寸三分,食系断,须头髻角子散慢。
更看其人面愁而眉皱,即是自割之状。此亦难必。
若自用刀剁下手并指节者,其皮头皆齐,必用药物封扎。虽是刃物自伤,必不能当下(禁止)死,必是将养不较致死。其痕肉皮头卷向里。如死后伤者,即皮不卷向里。以此为验。
又有人因自用口齿咬下手指者,齿内有风着于痕口,多致身死,少有生者。其咬破处疮口一道,周回骨折,必有脓水淹浸,皮肉损烂,因此将养不较致命身死。其痕有口齿迹及有皮血不齐去处。
验自刑人,即先问元申人:其身死人是何色目人?自刑时或早或晚?用何刃物?若有人来识认,即问身死人年若干?在生之日使左手使右手?如是奴婢,即先讨契书看,更问有无亲戚?及已死人使左手使右手?并须子细看验痕迹去处。
更须看验,在生前刃伤即有血行,死后即无血行。



(一)凡自割喉下死的,其尸口眼闭合,两手拳握,胳膊弯曲而拳缩,(死人用手把定刃伤,像用力的样子,他的手自然拳握。)肤色黄,头髻紧。
(二)如果是用小刀子自割,伤痕只可长一寸五分至二寸,用食刀即长三寸至四寸以来,如果是用磁器,割破的分寸不会大。每件器物刃物自割,并且下刃的一头尖小的,但伤着气喉即死。
如果是用刃物自扎着喉下、心前、腹上、两胁肋、太阳、顶门等要害地方,只要伤着脉膜,分寸虽小,即会立时死去;如果刺的不深以及不是要害地方,虽有三两处,也不会致死。
如果是用左自刎的,伤口必定起自右耳后,过喉一二寸;用右手的,必定起自左耳后。伤在喉骨上的难死,因为喉骨比较坚硬。伤在喉骨下的易死,因为喉骨下软弱而易于割断。其伤痕起手重,收手轻。(假如用左手把刃而伤,则喉右边下手处深,左边收刃处浅,其中间不如右边,因为下刃太重渐渐负痛缩手,因而轻浅,以及左手须似握物状便是。右手的也是这样。)
凡自割喉下,只有一处刀痕,如果当下身死时,痕深一寸七分,食管气管并断;如果伤后一日以下身死,深一寸五分,食管断,气管微破;如果伤了三五日以后死的,深一寸三分,食管断,须头髻角子散慢。
更看其人面愁而眉皱,即是自割的形状(这也难定)。
(三)如果自己用刀剁下手和指节的,其皮头皆齐,定用药物包扎,虽是用刃物自伤的,必不能当下身死,一定是将养不好致死的。其断处的皮头卷向里。如果是死后伤的,即皮不卷向里,以此为验。
(四)又有人因自用牙齿咬下手指的,齿上有风着于伤口,多致身死,少有生存的。其咬破处疮口一道,周围骨折,必然有脓水淹浸,皮肉损烂。因此,将养不到,致命身死。这种伤口上有牙齿痕迹,以及有皮肉不整齐去处。
(五)检验用刃物自杀的人,即先问元报案人,这个身死人是什么身分的人?自杀时或早或晚?用什么刃物?如果有人来认识死者,即问身死人年龄若干?在生之日,使左手使右手?如果是奴婢,即先讨契书看,更问死者有没有亲戚,以及生前使左手使右手?并须仔细看验伤痕地方。
更要看验,在生前刃伤,即有血流出,死后即无血流出。


\chapter{杀伤}

凡被人杀伤死者,其尸口、眼开,头髻宽或乱,两手微握,所被伤处要害分数较大,皮肉多卷凸,若透膜,肠脏必出。
其被伤人见行凶人用刃物来伤之时,必须争竞,用手来遮截,手上必有伤损。或有来护者,亦必背上有伤着处。若行凶人于虚怯要害处一刃直致命者,死人手上无伤,其疮必重。若行凶人用刃物斫着脑上、顶门、脑角后、发际,必须斫断头发,如用刃剪者。若头顶骨折,即是尖物刺着,须用手捏着其骨损与不损。
若尖刃斧痕;上阔长,内必狭。大刀痕浅必狭,深必阔。刀伤处其痕两头尖小,无起手收手轻重。枪刺痕浅则狭,深必透。簳,其痕带圆。或只用竹枪,尖竹担斡着要害处,疮口多不齐整,其痕方、圆不等。
凡验被快利物伤死者,须看元着衣衫有无破伤处,隐对痕、血点可验。○又如刀剔伤肠肚出者,其被伤处须有刀刃撩划三两痕。且一刀所伤。如何却有三两痕?盖凡人肠脏盘在左右胁下,是以撩划着三两痕。
凡检刀、枪刃斫剔,须开说尸在甚处向当?着甚衣服,上有无皿迹,伤处长、阔、深分寸?透肉不透肉?或肠肚出、膋膜出作致命处?仍检刃伤衣服穿孔。如被竹枪、尖物剔伤致命,便说尖硬物剔伤致死。
凡验杀伤,先看是与不是刀刃等物,及生前死后痕伤。如生前被刃伤,其痕肉阔、花文交出;若肉痕齐截,只是死后假作刃伤痕。
如生前刃伤即有血汁,及所伤痕疮口、皮肉、血多花,鲜色,所损透膜即死。若死后用刀刃割伤处,肉色即干白,更无血花也。盖人死后血脉不行,是以肉色白也。
此条仍责取行人定验,是与不是生前、死后伤痕。
活人被刃杀伤死者,其被刃处皮肉紧缩,有血荫四畔。若被支解者,筋骨皮肉稠粘,受刃处皮肉骨露。
死人被割截尸首,皮肉如旧,血不灌荫,被割处皮不紧缩,刃尽处无血流,其色白,纵痕下有血,洗检挤捺,肉内无清血出,即非生前被刃。
更有截下头者,活时斩下,筋缩入。死后截下,项长,并不伸缩。
凡检验被杀身死尸首,如是尖刃物,方说被刺要害。若是齐头刃物即不说“刺”字。如被伤着肚上、两肋下或脐下,说长阔分寸后,便说斜深透内脂膜,肚肠出,有血污,验是要害被伤割处致命身死。若是伤着心前肋上,只说斜深透内,有血污,验是要害致命身死。如伤着喉下,说深至项,锁骨损,兼周回所割得有方圆不齐去处,食系、气系并断,有血污,致命身死,可说要害处。如伤着头面上或太阳穴、脑角后、发际内,如行凶人刃物大,方说骨损。若脑浆出时有血污,亦定作要害处致命身死。如斫或刺着沿身,不拘那里,若经隔数日后身死,便说将养不较致命身死。
 凡验被杀伤人,未到验所,先问元申人曾与不曾收捉得行凶人?是何色目人?使是何刃物?曾与不曾收得刃物?如收得,取索看大小,着纸画样。如不曾收得,则问刃物在甚处?亦令元申人画刃物样,画讫,令元申人于样下书押字。更问元申人,其行凶人与被伤人是与不是亲戚?有无冤仇?



(一)凡是被人杀伤死的,其尸口眼,头髻宽松或散乱,两手微握,所被伤处要害分数较大,皮肉多卷凸,如果穿透肚皮,肠脏必出。
其被伤人见行凶人用刃物来伤的时候,必然要挣扎,用手来遮拦,手上一定有伤损,或有来护庇的,也一定背上有伤着地方。如果行凶人在虚弱要害处一刃径行致命的,死人手上没有伤,刃的创伤必重。如果行凶人用刃物砍着脑上顶门、脑角后发际,必须砍断头发,如同用剪刀剪的。如果头顶骨折,即是尖的东西刺着了,要用手按捏其骨看损与不损。
(二)如果是尖刃斧伤,上面阔长,内里必狭。大刀伤,浅必狭,深必阔。刀伤处,它的伤痕两头尖小,没有起手收手轻重的分别。枪刺的伤,浅则狭,深则穿透,伤痕带圆形。或只用竹枪尖、竹担扎着要害地方,疮口多不齐整,伤痕方圆不等。
(三)凡验被快利物伤死的,要看死者原著衣衫上有没有破伤的地方,隐对伤痕血点可验。又比如刀挑伤肚肠脱出了的,其被伤的地方,须有刀刃撩划三两痕。那么一刀所伤怎么竟会有三两痕了呢?这是因为人的肚肠是盘在左右胁下的,所以会撩划着三两痕。
(四)凡验刀枪刃砍挑的,要写明尸体在什么地方,什么方向,穿什么衣服,上面有没有血迹,伤的地方长阔深分寸,透肉不透肉,或肚肠出,脂膜出,作致命处。还要检验刃伤衣服的穿孔。如果是被竹枪尖物挑伤致命的,便写是尖硬物挑伤致死的。
(五)凡验杀伤,先看是否是刀刃等物杀伤,以及是生前、死后的伤痕。如果是生前被刃物所伤的,其痕肉开阔,收缩参差不齐,花纹交错。如果伤痕的肉截齐,就只是死后假造的刃伤痕。如果是生前的刃伤,即有血汁,以及所伤的疮口皮肉血多花鲜色,所伤透过脉膜即死。如果是死后用刀割出来的伤,肉色即干白,更没有血花。(因为人死后血脉不行,所以肉色是白的。)
此条仍责成仵作行人定验,是与不是生前、死后伤痕。
(六)活人被刃物杀伤死的,其被刃处皮肉紧缩,有血荫四畔。如果是被支解的,筋骨皮肉稠粘,受刃处皮肉紧缩骨头露出。
(七)死人被割截的,则尸首的皮肉如旧,血不灌荫,被割地方皮不紧缩,刃物尽处无血流,其色白。纵然伤痕下面有血,洗检挤捺,肉内没有清血出的,即不是生前被刃伤的。
(八)更有截下头的,活时斩下,则筋缩入,死后截下,则项长,并不收缩。
(九)凡检验被杀身死的尸首,如果凶器是尖刃物,方始说被刺要害,如果是齐头刃物,就不说刺字。如被伤着肚腹上、两胁下或脐下,说长阔分寸后,便说,斜深透内脂膜,肚肠出,有血污,验是要害处被割伤致命身死。如果是伤着心前肋上,只说斜深透内,有血污,验是要害致命身死。如果是伤着喉下,则说深至项,锁骨损,并且所割伤处周围有方圆不齐的去处,食管气管并断,有血污,致命身死。可说要害处。如果伤着头面上,或太阳穴、脑角后发际内,如果行凶人刃物大,方始说骨损,如果脑浆出时,有血污,也定作要害处致命身死。如果是砍或刺着,周身不拘那里,如果经隔数天后身死的,便说是将养不好,致命身死的。
(十)凡验被杀伤人,未到验所,先问原报案人曾不曾捉获得行凶人?是什么身分的人?使的是什么刃物?曾不曾收得了刃物?如收得了,即索取看大小,着人用纸画样,如果不曾收得,则问刃物在什么地方,也令原报案人画刃物样。画好了,令原报案人在图样下签名画押。还要问原报案人,本案的行凶人同被伤人是不是亲戚,有没有冤仇。


\chapter{尸首异处}

凡验尸首异处,勒家属先辨认尸首,务要子细。打量尸首顿处四至讫,次量首级离尸远近,或左或右,或去肩脚若干尺寸。支解手臂、脚腿,各量别计,仍各写相去尸远近。却随其所解肢体与尸相凑,提捧首与项相凑,围量分寸。一般系刃物斫落。若项下皮肉卷凸,两肩井耸 ,系生前斫落;皮肉不卷凸,两肩井不耸 ,系死后斫落。



凡验身首异处的尸体,使令家属先辨认尸首。务必要打量尸首放处的四至。完了后,再量首级离开尸体远近,或左或右,或距离肩、脚若干尺寸。支解手臂脚腿的,分头量,各计算,仍要各写相距尸体远近。并且依照所支解肢体的原来位置同尸体相凑起来,提捧首级同脖项相凑起来,围量尺寸。一般都是刃物砍落的。如果项下皮卷肉凸,两锁骨耸起皮剥,当是生前砍落的;如果项下皮肉不卷凸,两锁骨不耸起皮剥,就是死后砍落的。


\chapter{火死}

凡生前被火烧死者,其尸口、鼻内有烟灰,两手脚皆拳缩。缘其人未死前,被火逼奔争,口开气脉往来,故呼吸烟灰入口鼻内。若死后烧者,其人虽手、足拳缩,口内即无烟灰。若不烧着两肘骨及膝骨,手、脚亦不拳缩。
若因老病失火烧死,其尸肉色焦黑或卷,两手拳曲、臂曲在胸前,两膝亦曲,口、眼开,或咬齿及唇,或有脂膏黄色突出皮肉。
若被人勒死抛掉在火内,头发焦黄,头面浑身烧得焦黑,皮肉搐皱,并无揞浆蟽皮去处,项下有被勒着处痕迹。
又若被刃杀死却作火烧死者,勒仵作拾起白骨,扇去地下灰尘,于尸首下净地上用酽米醋、酒泼。若是杀死,即有血入地,鲜红色。须先问尸首生前宿卧所在?却恐杀死后移尸往他处,即难验尸下血色。
大凡人屋,或瓦或茅盖,若被火烧,其死尸在茅、瓦之下。或因与人有仇,乘势推入烧死者,其死尸则在茅、瓦之下。兼验头、足,亦有向至。
如尸被火化尽,只是灰,无条段骨殖者,勒行人邻证供状:缘上件尸首,或失火烧毁、或被人烧毁,即无骸骨存在,委是无凭检验。方与备申。
凡验被火烧死人,先问元申人:火从何处起?火起时其人在甚处?因甚在彼?被火烧时曾与不曾救应?仍根究曾与不曾与人作闹?见得端的方可检验。或检得头发焦拳,头面连身一概焦黑,宜申说:今来无凭检验本人沿身上下有无伤损他故,及定夺年颜形状不得,只检得本人口鼻内有无灰烬,委是火烧身死。如火烧深重,实无可凭,即不要说口、鼻内灰烬。



(一)凡生前被火烧死的,其尸体口鼻内有烟灰,两手两脚都拳缩(由于被烧的人未死前被火逼的奔跑挣扎,口开喘气,所以呼吸烟灰入口鼻内);如果是死后烧的,被烧的人虽然手脚拳缩,口内便没有烟灰。如果不烧着两肘骨及膝骨,手脚也不拳缩。
(二)如果因为老病失火烧死的,他的尸体肉色焦黑或卷,两手拳曲,胳臂曲在胸前,两膝也曲,口眼开,或咬齿及唇,或有黄色脂膏,凝积在皮肉上面。
(三)如果是被人勒死后拋在火内的,尸体头发焦黄,头面浑身烧得焦黑,皮肉搐皱,并没有起泡〈皮达〉皮的去处,项下有被勒的痕迹。
(四)又如被刀杀死,却作火烧死的,令仵作行人拾起白骨,扇去地下灰尘,在尸体下面的净地上用酽米醋酒洒泼,如果是杀死的,即会有血入地因而现出鲜红色。必须先问死者生前宿卧在什么地方,恐怕是杀死后移尸往他处,即难于验出尸体身下的血色。
大凡人的住屋或是用瓦盖或是用茅盖,如果到火烧,屋里被烧死的人尸体在茅瓦之下;或者因为与人有仇怨,被乘势推入烧死的,其尸体则在茅瓦之上。兼验头足,方向上也当有所不同。
如果尸体火化尽,只剩下灰而没有成条成段骨殖的,令仵作邻证等供状:由于上件尸首,或失火烧毁,或被人烧毁,已无骸骨存在,的确是没有凭证可验。才给予备文申报。
(五)凡验被火烧死的人,先问原报案人,火是从什么地方起的?火起的时候死者在什么地方?因为什么在那里?被火烧的时候,曾不曾被救应过?还要彻底查究死者死前曾否与人争闹过,查问出究竟了,方可检验。
或检到死者头发焦拳,头面连同全身一概焦黑的情况,宜详细说明,现已没有依据检验死者全身上下有没有伤损其它原故,及判断年龄面貌形状不得。只验到死者口鼻内有无灰烬,确实是火烧身死的。如果火烧得严重,实在没有可凭依的了,即不要说口鼻内有无灰烬了。


\chapter{汤泼死}

 凡被热汤泼伤者,其尸皮肉皆拆,皮脱白色,着肉者亦白,肉多烂赤。
如在汤火内,多是倒卧,伤在手、足、头面、胸前。如因斗打或头撞、脚踏、手推在汤火内,多是两后 与臀、腿上,或有打损处,其疱不甚起,与其他所烫不同。


凡是被热汤泼伤而死的,这种尸体皮肉都开裂,皮脱处呈白色,着肉处也呈白色,肉多坏烂红赤。
如果在汤火内,多是倒卧,伤在手足、头面、胸前。如果是由于斗打或头撞、脚踢、手推在汤火内的,多是伤在两后腿弯和大腿到臀部上。或有打伤处,上面疱不大起,和其它所烫地方不同。


\chapter{服毒}

凡服毒死者,尸口、眼多开,面紫黯或青色,唇紫黑,手足指甲俱青黯,口、眼、耳、鼻间有血出。
甚者遍身黑肿,面作青黑色,唇卷发疱,舌缩或裂拆、烂肿、微出,唇亦烂肿或裂拆,指甲尖黑,喉、腹胀作黑色、生疱,身或青班,眼突,口、鼻、眼内出紫黑血,须发浮不堪洗。未死前须吐出恶物或泻下黑血,谷道肿突或大肠穿出。
有空腹服毒,惟腹肚青胀而唇、指甲不青者;亦有食饱后服毒,惟唇、指甲青而腹肚不青者;又有腹脏虚弱、老病之人,略服毒而便死,腹肚、口唇、指甲并不青者,却须参以他证。
生前中毒而遍身作青黑,多日皮肉尚有,亦作黑色。若经久,皮肉腐烂见骨,其骨黪黑色。
死后将毒药在口内假作中毒,皮肉与骨只作黄白色。
 凡服毒死,或时即发作,或当日早晚,若其药慢,即有一日或二日发。或有翻吐,或吐不绝,仍须于衣服上寻余药,及死尸坐处寻药物器皿之类。
中虫毒,遍身上下、头面、胸心并深青黑色,肚胀,或口内吐血,或粪门内泻血。
鼠莽草毒,江南有之。亦类中虫,加之唇裂,齿龈青黑色。此毒经一宿一日,方见九窍有血出。
食果实、金石药毒者,其尸上下或有一二处赤肿,有类拳手伤痕;或成大片青黑色,爪甲黑,身体肉缝微有血;或腹胀,或泻血。酒毒,腹胀或吐、泻血。
砒霜、野葛毒,得一伏时,遍身发小疱,作青黑色,眼睛耸出,舌上生小刺疱绽出,口唇破裂,两耳胀大,腹肚膨胀,粪门胀绽,十指甲青黑。
金蚕蛊毒,死尸瘦劣,遍身黄白色,眼睛塌,口齿露出,上下唇缩,腹肚塌。将银钗验,作黄浪色,用皂角水洗不去。○一云如是:只身体胀,皮肉似汤火疱起,渐次为脓,舌头、唇、鼻皆破裂,乃是中金蚕蛊毒之状。○手脚指甲及身上青黑色,口、鼻内多出血,皮肉多裂,舌与粪门皆露出,乃是中药毒、菌蕈毒之状。
如因吐泻瘦弱,皮肤微黑不破裂,口内无血与粪门不出,乃是饮酒相反之状。
若验服毒,用银钗,皂角水揩洗过,探入死人喉内,以纸密封,良久取出,作青黑色,再用皂角水揩洗,其色不去。如无,其色鲜白。
如服毒中毒死人,生前吃物压下入肠脏内,试验无证,即自谷道内试,其色即见。
凡检验毒死尸,间有服毒已久、蕴积在内试验不出者,须先以银或铜钗探入死人喉讫,却用热糟醋自下盦洗,渐渐向上,须令气透,其毒气熏蒸,黑色始现。如便将热糟、醋自上而下,则其毒气逼热气向下,不复可见。或就粪门上试探,则用糟、醋当反是。
又一法,用大米或占米三升炊饭;用净糯米一升淘洗了,用布袱盛就所炊饭上炊 。取(又鸟)子一个,鸭子亦可。打破取白,拌糯米饭令匀,依前袱起,着在前大米占米饭上。以手三指,紧握糯米饭,如鸭子大,毋令冷,急开尸口齿外放着,及用小纸三五张搭遮尸口、耳、鼻、臀、阴门之处,仍用新绵絮三五条,酽醋三五升,用猛火煎数沸,将棉絮放醋锅内煮半时取出,仍用糟盘罨尸,却将棉絮盖覆。若是死人生前被毒,其尸即肿胀,口内黑臭恶汁喷来棉絮上,不可近。后除去棉絮,糯米饭被臭恶之汁亦黑色而臭,此是受毒药之状。如无,则非也。试验糯米饭封起申官府之时,分明开说。此检验诀,曾经大理寺看定。
广南人小有争怒赖人。自服胡蔓草,一名断肠草,形如阿魏,叶长尖,条蔓生,服三叶以上即死。干者或收藏经久作末食,亦死。如方食未久,将大粪汁灌之可解。其草近人则叶动。将嫩叶心浸水,涓滴入口即百窍溃血,其法急取抱卵不生(又鸟)儿研细,和麻油开口灌之,乃尽吐出恶物而苏。如少迟,无可救者。



(一)凡是服毒死的,尸体口眼多开,面部紫黯或青色,嘴唇紫黑,手脚指甲皆黑黯,口眼耳鼻中有血出。严重的,遍身黑肿,面部作青黑色,嘴唇翻卷起疱,舌头收缩或裂拆烂肿微出,嘴唇也烂肿或裂拆,指甲尖黑,喉、腹肿胀作黑色,生疱,身上或有青斑,眼睛突出,口鼻眼内出紫黑血,须发浮乱不堪梳洗。未死前须吐出恶物,或泻下黑血,肛门浮肿突出,或大肠头脱出。
有空腹服毒,只是肚腹青胀而嘴唇、指甲不青的;也有吃饱后服毒,只是嘴唇、指甲青而肚腹不青的;又有腹脏虚弱老病的人,略服毒便死,而肚腹、口唇、指甲并不青的,却须参以其它旁证。
生前中毒,而遍身作青黑色,多日皮肉尚有,也作黑色,如果经过时间长久,皮肉腐烂见骨,其骨浅青黑色。
死后把毒药放在口内假作中毒的,皮肉与骨只作黄白色。
(二)凡服毒死的,或立时就发作,或当天早晚发作,如果毒性较慢,即有一天或二天发作的。毒性发后,或有翻吐,或呕吐不绝,仍须在死者的衣服上寻找余药,及死尸坐的地方寻找药物器皿之类。
中虫毒的,遍身上下、头面、胸心都是青黑色,肚胀,或口内吐血,或肛门内泻血。
中鼠莽草毒的(江南有之),也类似中虫毒,再加上嘴唇裂,齿龈青黑色。这种毒经过一夜一天,方见,九窍有血出。
吃果实金石药物中毒的,这种尸体上下或有一二处赤肿,有些像拳头打的伤痕,或成大片青黑色,指甲黑,身体的毛孔微有血出,或腹胀,或泻血。
中酒毒的,腹胀,或吐泻血。
中砒霜野葛毒的,得一昼夜十二个时辰发,遍身起小疱,作青黑色,眼睛突出,舌上生小刺疱绽出,嘴唇破裂,两耳肿大,肚腹膨胀,肛门肿胀破裂,十指甲青黑。
中金蚕蛊毒的,死尸瘦劣,遍身黄白色,眼睛塌陷,牙齿露出,上下唇缩,肚腹塌下。拿银钗验,作黄浪色,用皂角水洗不去。
也有的说是这样:只身体肿胀,皮肉像汤火烫的起疱,渐渐化脓,舌头嘴唇鼻头皆破裂的,乃是中金蚕蛊毒的形状。手脚指甲及身上青黑色,口鼻内多出血,皮肉多裂,舌头伸出和肛门突出的,乃是中药毒、菌蕈毒的形状。
如果由于吐泻瘦弱,皮肤微黑不破裂,口内无血和肛门不突出的,乃是饮酒相反的形状。
(三)如验服毒,用银钗,着皂角水洗揩过,探入死人的喉内,口鼻以纸密封,良久取出,如果有毒,银钗作青黑色,再用皂角水揩洗,其色不去。如果无毒,钗的颜色鲜白。
如果服毒中毒死的人,生前吃东西把毒物压下,进入肠脏内了,因而从喉内试验没有验证的,即自肛门内试验,毒的气色即会现出。
(四)凡检验毒死的尸体,间或有服毒已久,蕴积在内,试验不出的,须先以银或铜钗探入死人喉内后,却用热糟醋自下罨洗,渐渐向上,须令气透,使毒气上行,熏蒸钗物,黑色方才现出。如果开始便将热糟醋自上而下罨洗,则其热气便逼使毒气向下,上头便不再能看到。或就肛门中试验的,则用糟醋的方向应和上述的相反。
(五)又一法,用大米或粘米三升炊饭,用净糯米掏洗了,用布袱盛起来,就所炊饭上炊蒸,取鶏蛋一个(鸭蛋也可以),打破取白,拌糯米饭使匀,依旧包起,着在前大米粘米饭上,用手的三个指头紧握糯米饭如鸭蛋大,勿令冷,急开尸口,齿外放着,及用小纸三五张,搭盖在尸体的口、耳、鼻、肛门和阴户地方,再用新绵絮三五条,酽醋三五升,猛火煎数沸,将绵絮放到醋锅内煮半个时辰取出,仍用糟围罨尸,却将绵絮覆盖。如果是死者生前被毒,其尸即肿胀,口内黑臭恶汁喷出到绵絮上,不可近。然后除去绵絮,糯米饭为臭恶之汁所喷,也黑色而臭,这是受毒药所毒的现象。如果没有这些,就不是。用作试验的糯米饭,在封起来上报官府的时候,要开写明白。这一检验方法,曾经过大理寺审定的。
(六)广南人小有争斗,愤怒赖人,自服胡蔓草,——一名断肠草,形状如同阿魏,叶长尖,条蔓生。服三叶以上即死。干的或收藏经久,作末服食也死。如果刚吃下未久,将大粪汁灌之可解。其草靠近人则叶动。将嫩叶心浸水,点滴入口,即百窍流血。其解法,急取抱孵未生的蛋中鸡儿,细研和麻油开口灌之,乃尽吐出恶物而活过来。如果少迟,没有能够救治的。


\chapter{病死}

凡因病死者,形体羸瘦,肉色痿黄,口、眼多合,腹肚低陷,两眼通黄,两拳微握,发髻解脱,身上或有新旧针灸瘢痕,余无他故,即是因病死。
凡病患求乞在路死者,形体瘦劣,肉色痿黄,口、眼合,两手微握,口齿焦黄,唇不着齿。
邪魔中风卒死,尸多肥,肉色微黄,口、眼合,头髻紧,口内有涎沫,遍身无他故。
卒死,肌肉不陷,口、鼻内有涎沫,面色紫赤。盖其人未死时涎壅于上,气不宣通,故面色及口、鼻如此。
卒中死,眼开、睛白,口齿开,牙关紧,间有口眼涡斜并口两角、鼻内涎沫流出,手脚拳曲。
中暗风,尸必肥,肉色滉白色,口、眼皆闭,涎唾流溢;卒死于邪崇,其尸不在于肥瘦,两手皆握,手、足爪甲多青;或暗风如发惊搐死者,口、眼多?斜,手、足必拳缩,臂、腿、手、足细小,涎沫亦流。已上三项大略相似,更须检时子细分别。
伤寒死,遍身紫赤色,口、眼开,有紫汗流,唇亦微绽,手不握拳。
时气死者,眼开、口开,遍身黄色,量有薄皮起,手、足俱伸。
中暑死,多在五六七月,眼合,舌与粪门俱不出,面黄白色。
冻死者,面色痿黄,口内有涎沫,牙齿硬,身直,两手紧抱胸前,兼衣□服单薄。检时用酒、醋洗,得少热气则两腮红,面如芙蓉色,口有涎沫出,其涎不粘,此则冻死证。
饥饿死者,浑身黑瘦硬直,眼闭、口开,牙关紧禁,手、脚俱伸。
或疾病死,值春夏秋初,申得迟,经隔两三日,肚上,脐下,两胁肋、骨缝有微青色,此是病人死后经日变动,腹内秽污发作攻注皮肤,致有此色。不是生前有他故,切宜子细。
凡验病死之人,才至检所,先问元申人:其身死人来自何处?几时到来?几时得病?曾与不曾申官取责口词?有无人识认?如收得口词,即须问元患是何疾病?年多少?病得几日方申官取问口词?既得口词之后几日身死?如无口词,则问如何取口词不得?若是奴婢,则须先讨契书看,问有无亲戚?患是何病?曾请是何医人?吃甚药?曾与不曾申官取口词?如无,则问不责口词因依?然后对众证定。如别无它故,只取众定验状,称说遍身黄色,骨瘦,委是生前因患是何疾致死,仍取医人定验疾色状一纸。如委的众证因病身死分明,元初虽不曾取责口词,但不是非理致死,不须牒请复验。



(一)凡是因病死的,身体瘦弱,肤色痿黄,口眼多闭合,肚腹多塌陷,两眼通黄,两拳微握,发髻解脱,身上或有新旧针灸瘢痕,别无其它原故,即是因病死了的。
(二)凡是患病求乞在路死的,身体瘦劣,肤色痿黄,口眼闭合,两手微握,牙齿焦黄,唇不着齿。
邪魔中风猝然死亡的,尸体多肥,肤色微黄,口眼闭合,头髻紧,口内有涎沫,遍身没有其它原故。
(三)猝然死亡的,肌肉不陷,口鼻内有涎沫,面色紫赤。因为其人未死时,涎痰壅到上面来,气不得通,所以面色和口鼻这个样子。
猝中死的,眼开睛白,口齿开,牙关紧,间或有口眼歪斜的,同时口两角、鼻内涎沫流出,手脚拳曲。
(四)中暗风的,尸体必肥,皮肤多滉白色,口眼都闭,涎唾外流。猝然死于邪祟的,其尸不在于肥瘦,两手都握,手脚指甲多青。或中暗风,如发惊搐死的,口眼多歪斜,手脚一定拳缩,臂腿手脚细小,口鼻也外流涎沫(以上三项大略相似,还须检验时仔细分别)。
(五)伤寒病死的,遍身紫赤色,口眼张开,有紫汗流出,唇也微裂,手不握拳。
(六)时气死的,眼开口开,遍身黄色,检查有薄皮起,手脚都伸开。
(七)中暑死的,多在五、六、七月,眼闭,舌头和肛门都不出,面黄白色。
(八)冻死的,面色痿黄,口内有涎沫,牙关硬,身体僵直,两手紧抱胸前,兼之衣服单薄。检验时,用酒醋洗得少有热气,则两腮泛红,面如芙蓉色,口有涎沫出。这种涎沫不粘,就是冻死的证明。
(九)饥饿死的,浑身黑瘦,硬直,眼闭口开,牙关紧闭,手脚都伸开。
(十)或有疾病死的,正当春夏和秋初时候,申报的慢,经隔三两日,肚上脐下,两胁肋骨缝有微青色。这是病人死后,由于时间变化,肚子里的秽污发作,攻注到皮肤,以致有这种颜色。不是生前有什么其它原故,切要仔细。
(十一)凡检验病死的人,才到检验所在地,先问原报案人,这个身死人是从什么地方来的?几时到来的?几时得病的?曾与不曾报官录取口词?有没有人认识他?如果取得口词了,就要问原来患的是什么疾病?这人年纪多大了?病到几天才报官录取口词的?既取得口词之后,几天身死的?如果没有口词,则问为什么取口词不到?如果是奴婢,便要先讨卖身契书看,问死者有没有亲戚?患的是什么病?曾请过那个医生?吃什么药?曾与不曾报官录取口词?如果没有,则问明不取口词的原因,然后对众证实定案。如果别无其它原故,那就采取众定验状的方式,称说,遍身黄色,骨瘦,确实是生前因患是何疾致死。还要取得医生定验疾色状一纸。如果确实众证因病身死分明,当初虽不曾录取口词,但不是非命致死的,不须请官覆验。


\chapter{针灸死}

须勾医人,验针灸处是与不是穴道。虽无意致杀,亦须说显是针灸杀,亦可科医“不应为”罪。


必须拘提医生检验针灸处是不是穴道,虽然是无意致杀,也要说显然是针灸杀,也可以科处医生以「不应为罪」。


\chapter{札口词}

凡抄扎口词,恐非正身,或以它人伪作病状代其饰说,一时不可辨认。合于所判状内云:“日后或死亡,申官从条检验。”庶使豪强之家,预知所警。


凡记录口词,恐怕不是本人,有的是以他人伪作病状,代其饰说,一时不能辨认清楚。应在所判状内申明:日后如或死亡报官,根据有关条令检验。这或许可使那些豪强之家,预知有所警悟。


\part{}

\chapter{验罪囚死}

凡验诸处狱内非理致死囚人,须当径申提刑司,即时入发 铺。


凡检验各处监狱里非命致死的囚犯,须得直接申报提刑司,实时把申报文件交付发递铺递送。


\chapter{受杖死}

定所受杖处疮痕阔狭,看阴囊及妇人阴门,并两胁肋、腰、小腹等处有无血荫痕。
小杖痕,左边横长三寸,阔二寸五分。右边横长三寸五分,阔三寸。各深三分。
大杖痕,左右各方圆三寸至三寸五分,各深三分,各有脓水。兼疮周回亦有脓水淹浸、皮肉溃烂去处。
背上杖疮,横长五寸,阔三寸,深五分。如日浅时,宜说兼疮周回,有毒气攻注、青赤 皮紧硬去处。如日数多时,宜说兼疮周回亦有脓水淹浸、皮肉溃烂去处,将养不较致命身死。
又有讯腿杖,而荆杖侵及外肾而死者,尤须细验。



对于受杖刑致死的人要定验所受杖处疮痕的阔狭,看验阴囊和妇人的阴户,以及两胁肋、腰、小腰等处,有没有血荫痕。
小杖的伤痕,左边横长三寸,阔二寸五分;右边横长三寸五分,阔三寸。各深三分。
大杖的伤痕,左右各方圆三寸到三寸五分各深三分,各有脓水,兼疮伤的周围也有脓水淹浸、皮肉溃烂的去处。
背上的杖疮,横长五寸,阔三寸,深五分。如果日数少时,宜说,兼疮伤的周围有毒气攻注,青赤〈皮达〉皮紧硬的去处。如果日数多时,宜说,兼疮伤的周围也有脓水淹浸、皮肉溃烂的去处。将养不好,致命身死。
又有审问时用腿杖,而荆杖侵及外肾而死的,尤其须要细验。


\chapter{跌死}

凡从树及屋临高跌死者,看枝柯挂掰所在并屋高低、失脚处踪迹,或土痕高下及要害处,须有抵隐或物擦磕痕瘢。若内损致命痕者,口、眼、耳、鼻内定有血出。若伤重分明,更当子细验之,仍量扑落处高低丈尺。



凡从树上和屋上高处失足跌死的,要看验树枝上挂绊的痕迹和屋的高矮,失脚处的踪迹或土痕的高下,以及身上要害处须有被硬物抵垫或外物擦磕的瘢痕。如果是内部有致命损伤的,则口眼耳鼻内一定有血出。如果伤重分明了,更应当仔细检验,并要量跌落处的高低丈尺。


\chapter{塌压死}

凡被塌压死者,两腿 出、舌亦出,两手微握,遍身死血淤紫黯色。或鼻有血,或清水出,伤处有血荫、赤肿,皮破处四畔赤肿。或骨并筋皮断折。须压着要害致命,如不压着要害不致死。死后压即无此状。
凡检舍屋及墙倒石头脱落压着身死人,其尸沿身虚怯要害去处若有痕损,须说长阔分寸,作坚硬物压痕。仍看骨损与不损。若树木压死,要见得所倒树木斜伤着痕损分寸。



(一)凡被塌压死的,两眼脱出,舌也出,两手微握,遍身死淤血紫黯色,或鼻中有血或清水出。在受伤的地方有血荫赤肿,皮破地方四周赤肿,或骨并筋皮断折。必须压着要害,才会致命,如不压着要害,不致死。死后压的,便无这些情况。
(二)凡验房屋和墙倒石头脱落压着身死的,其尸沿身虚弱要害去处如有伤损,必须说长阔分寸,作坚硬物压伤,并看骨损与否。如是树木压死的,要验出所倒树木斜伤着的伤痕的分寸。


\chapter{外物压塞口鼻死}

 凡被人以衣服或湿纸搭口、鼻死,则腹干胀。
若被人以外物压塞口鼻,出气不得后命绝死者,眼开睛突,口、鼻内流出清血水,满面血荫赤黑色,粪门突出及便溺污坏衣服。



(一)凡被人用衣服或湿纸搭在口鼻上闷死的,即腹部干胀。
(二)如果是被人用东西压塞口鼻,出气不得而命绝身死的,眼开睛突,口鼻内流出清血水,满面血荫赤黑色,肛门突出,以及便溺污坏衣服。


\chapter{硬物瘾痁死}

 凡被外物瘾痁死者,肋后有瘾痁着紫赤肿,方圆三寸四寸以来,皮不破,用手揣捏得筋骨伤损,此最为虚怯要害致命去处。


凡被外物瘾坫?瓦死的,肋后有瘾坫?瓦着的紫赤肿块,方圆约三寸四寸开外,皮不破,用手揣捏得到肋骨伤损,这最是虚弱要害致命的去处。


\chapter{牛马踏死}

凡被马踏死者,尸色微黄,两手散,头发不慢,口、鼻中多有血出,痕黑色。被踏要害处便死,骨折、肠脏出。若只筑倒或踏不着要害处,即有皮破、瘾赤黑痕,不致死。○驴足痕小。
牛角触着,若皮不破,伤亦赤肿。触着处多在心头、胸前,或在小腹、胁肋亦不可拘。


凡被马踏死的,尸体的颜色微黄,两手散开,头发不散乱,口鼻中多有血出痕迹,黑色。被踏着要害地方便死,骨折肠脏出;如果只是撞倒,或踏不到要害处,即使有皮破瘾赤黑痕,不致死。
驴踏的,足痕较小。
牛角抵触着的,若皮不破,伤也是赤肿。触到的地方,在心头胸前,或在小腹胁肋,也不可拘一。


\chapter{车轮拶死}

凡被车轮拶死者,其尸肉色微黄,口、眼开,两手微握,头髻紧。
凡车轮头拶着处,多在心头、胸前并两胁肋要害处便死。不是要害不致死。



(一)凡被车轮碾压死的,其肉色微黄,口眼开,两手微握,头髻紧。
(二)凡车轮迎面碾撞着处,多在心头胸前,并两胁肋。碾撞着要害处便死,不是要害处不致死。


\chapter{雷震死}
凡被雷震死者,其尸肉色焦黄,浑身软黑,两手拳散、口开、眼(兑皮),耳后、发际焦黄,头髻披散,烧着处皮肉紧硬而挛缩,身上衣服被天火烧烂。或不火烧。伤损痕迹多在脑上及脑后,脑缝多开,鬓发如焰火烧着。从上至下,时有手掌大片浮皮,紫赤,肉不损,胸、项、背、膊上或有似篆文痕。


凡是被雷震死的,其尸肉色焦黄,浑身软黑,两手拳散,口开眼突,耳后发际焦黄,头髻披散,烧着处皮肉紧硬而挛缩,身上衣服被雷火烧烂(或者不烧烂)。伤损痕迹,多在脑上及脑后,脑缝多裂开,鬓发像被焰火烧着,从上到下,常有手掌大的大片浮皮紫赤,肉不损,胸前、项上、背后、胳膊上,或者有类似篆文的痕迹。


\chapter{虎咬死}
凡被虎咬死者,尸肉色黄,口、眼多开,两手拳握,发髻散乱,粪出,伤处多不齐整,有舌舐齿咬痕迹。
虎咬人多咬头项上,身上有爪痕掰损痕,伤处成窟或见骨,心头、胸前、臂、腿上有伤处,地上有虎迹。勒画匠画出虎迹,并勒村甲及伤人处邻人供责为证。一云:虎咬人月初咬头项,月中咬腹背,月尽咬两脚。猫儿咬鼠亦然。


凡是被虎咬死的,尸体的肉色发黄,口眼多开,两手拳握,发髻散乱,粪便出。伤处多不齐整,有舌舐齿咬的痕迹。
虎咬人多咬头项上,身上有爪痕,攀伤痕。伤处成窟窿,或见骨,心口、胸前、臂腿上有伤处,地上有虎的足迹。令画匠画出虎的足迹,并令村长及伤人处邻人负责供述作证(也有的说,虎咬人月初咬头项,月中咬腹背,月终咬两脚。猫儿咬鼠也是这样)。


\chapter{蛇虫伤死}

凡被蛇虫伤致死者,其被伤处微有啮损黑痕,四畔青肿,有青黄水流,毒气灌注四肢,身体光肿、面黑。如检此状,即须定作毒气灌着甚处致死。


凡是被蛇虫咬伤致死的,其被伤处微有咬伤的黑痕,四周青肿,有青黄水流出来,毒气灌注四肢,身体光肿,面黑。如果检到这种情况,便要定为毒气灌注着什么地方致死。


\chapter{酒食醉饱死}
凡验酒食醉饱致死者,先集会首等,对众勒仵作行人用醋汤洗检。在身如无痕损,以手拍死人肚皮,膨胀而响者,如此即是因酒食醉饱过度,腹胀心肺致死。仍取本家亲的骨肉供状,述死人生前常吃酒多少致醉,及取会首等状,今来吃酒多少数目,以验致死因依。


凡验酒食醉饱致死的,先召集会餐的东道人等当面令仵作行人用醋、热水洗检。周身如果没有伤痕,用手拍死人肚皮膨胀而响的,这样便是由于酒食醉饱过度,胀满心肺致死。还要取死者的亲属的供状,陈明死者平时常吃酒多少致醉,以及取会餐的东道人等的供状,陈明这次死者吃酒多少数目,以验证致死的前因后果。


\chapter{醉饱后筑踏内损死}

凡人吃酒食至饱,被筑踏内损亦可致死。其状甚难明。其尸外别无他故,唯口、鼻、粪门有饮食并粪带血流出,遇此形状,须子细体究曾与人交争,因而筑踏?见人照证分明,方可定死状。


凡人吃酒饭至饱,被顶撞践踏,造成内部损伤的,也可致死。它的形状很难定,尸体外表别无其它情况,只是口鼻有饮食,肛门有粪便,带血流出。遇有这种形状,必须仔细观察查究死者生前曾否与人争斗,因而冲撞践踏。要见证人对证分明,才可以填定验尸状。


\chapter{男子作过死}

 凡男子作过太多,精气耗尽、脱死于妇人身上者,真伪不可不察。真则阳不衰,伪者则痿。


凡男子淫欲太多,精气耗尽,脱死在妇人身上的,真假不可不仔细考察。真的则阴茎不衰,假的则痿。


\chapter{遗路死}

或是被打死者扛在路傍,耆正只申官作遗路死尸,须是子细。如有痕迹,合申官多方体访。


或是被打死的,扛在路旁,耆正只报官当作遗路死尸,需要仔细检验。如有痕迹,应当报官,多方面查访。


\chapter{死后仰卧停泊有微赤色}

凡死人项后、背上、两肋后、腰、腿内、两臂上、两腿后、两曲 、两脚肚子上,下有微赤色。○验是本人身死后,一向仰卧停泊,血脉坠下,致有此微赤色,即不是别致他故身死。


凡死人项后、背上、两肋后、腰、腿内、两臂上、两腿后、两腘窝、两腿肚子上,下有微赤色。
验是本人身死后,一直仰卧停着,血液沉下,以致有这种微赤色的,便不是别因其它原故致死。


\chapter{死后虫鼠犬伤}
凡人死后被虫鼠伤,即皮破无血,破处周回有虫鼠啮痕踪迹,有皮肉不齐去处。若狗咬则痕迹粗大。


凡人死后被虫鼠伤的,即皮破无血,破处周围有虫鼠啃咬的痕迹,有皮肉不整齐的去处。如果是狗咬的,则齿痕粗大。


\chapter{发冢}
验是甚向坟、围长阔多少?被贼人开锄,坟上狼藉,锹锄开深尺寸?见板或开棺见尸?勒所报人具出死人元装着衣服物色,有甚不见被贼人偷去?


检验被掘冢是什么山向,坟围的长阔多少,被贼人开掘,坟土狼藉,开掘深多少尺寸,见棺板或开棺见尸。令报案人具报出死人原著衣服等对象数,有什么不见,被贼人偷去了。


\chapter{验邻悬尸}
凡邻县有尸在山林荒僻处,经久损坏,无皮肉,本县已作病死检了,却牒邻县复,盖为他前检不明,于心未安,相攀复检。有如此类,莫若据直申:其尸见有白骨一副,手、足、头全,并无皮肉、肠胃,验是死经多日,即不见得因何致死,所有尸骨未敢给付埋殡,申所属施行。不可被公人给作“无凭检验”。
凡被牒往他县复检者,先具承牒时辰,起离前去事状,申所属官司。值夜止宿。及到地头,次弟取责于连人罪状,致死今经几日方行检验?如经停日久,委的皮肉坏烂不任看验者,即具仵作行人等众状,称尸首头、项、口、眼、耳、鼻、咽喉上下至心胸、肚脐、小腹、手脚等并遍身上下,尸胀臭烂,蛆虫往来咂食,不任检验。如稍可验,即先用水洗去浮蛆虫,子细依理检验。


(一)凡邻县有尸在山林荒僻的地方,经久损坏,没有皮肉,本县已经当作病死检验了,却发出公文请邻县覆验,这是因为他们感到前面的检验不够明白,于心未安,所以相请覆验。遇到这类情况,不如据实直捷申明:该尸现在只余白骨一副,手、足、头齐全,没有皮肉、肠胃,验是尸体经过多日,已经验不出是因为什么致死的了。所有尸骨,未敢给付尸亲埋殡,报请所属上级指示办理。不可被吏役等所欺诳,作为「无凭检验」处理。
(二)凡被公文召往他县进行覆验的,先要写具接奉公文的时刻和起程前去的具体情况,虽报所属上级机关。遇夜住宿。及到地头,按次讯问录取有关牵连人的罪状,致死到今经隔几天方才检验。如果经过停放日久,确实已是皮肉坏烂,不堪看验了的,即写具仵作行人等众状,宣称:尸首的头、项、口、眼、耳、鼻、咽喉、手、脚等,并遍身上下,膨胀臭烂,蛆虫往来咂食,不堪检验了。如果稍微可验,即先用水洗去尸上的浮蛆,仔细依据常理检验。


\chapter{辟秽方}
【三神汤】能辟死气
苍术二两。米泔浸两宿,焙干白术半两甘草半两。炙右为细末,每服二钱,入盐少许,点服。
【辟秽丹】能辟秽气
麝香少许细辛半两甘松一两川芎二两
右为细末,蜜圆如弹子大,久窨为妙,每用一圆烧之。
【苏合香圆】每一圆含化,尤能辟恶。


三神汤
【三神汤】能辟尸气。
苍朮二两。淘米水浸二宿,焙干。 白朮半两。
甘草半两。炕干。
以上药做成细末,每次服二钱,放入食盐少许,点白汤服。
辟秽丹
【辟秽丹】能辟秽气。
麝香少许 细辛半两 甘泉一两 川芎一两
以上药做成细末,和蜜团成丸子如弹子大,久窖的为好,每次用一丸焚烧。
苏合香圆
【苏合香圆】每次用一丸含化,尤其能辟恶气。


\chapter{救死方}
若缢,从早至夜虽冷亦可救;从夜至早稍难。若心下温,一日以上犹可救,不得截绳,但款款抱解放卧,令一人踏其两肩,以手拔其发常令紧;一人微微捻整喉咙,依先以手擦胸上散动之;一人磨搦臂、足屈伸之,若已僵,但渐渐强屈之;又按其腹。如此一饭久即气从口出,复呼吸、眼开。勿苦劳动。又以少官桂汤及粥饮与之,令润咽喉。更令二人以笔管吹其耳内。若依此救,无有不活者。
又法:紧用手罨其口,勿令通气,两时许气急即活。
又用皂角、细辛等分为末,如大豆许吹两鼻孔。
水溺一宿者尚可救,捣皂角以棉裹纳下部内,须臾出水即活。
又屈死人两足着人肩上,以死人背贴生人背担走,吐出水即活。
又先打壁泥一堵置地上,却以死者仰卧其上,更以壁土覆之,止露口、眼,自然水气翕入泥间,其人遂苏。洪丞相在番阳,有溺水者身僵气绝,用此法救即苏。
又炒热沙覆死人面,上下着沙,只留出口、耳、鼻,沙冷湿又换,数易即苏。又醋半盏灌鼻中,
又绵裹石灰纳下部中,水出即活。又倒悬,以好酒灌鼻中及下部。又倒悬,解去衣,去脐中垢,令两人以笔管吹其耳。
又急解死人衣服,于脐上灸百壮。
暍死于行路上,旋以刀器掘开一穴,入水捣之,却取烂浆以灌死者即活。中暍不省人事者,与冷水吃即死,但且急取灶间微热灰壅之,复以以稍热汤蘸手巾熨腹胁间,良久苏醒,不宜便与冷物吃。
冻死,四肢直、口噤、有微气者,用大锅炒灰,令暖袋盛,熨心上,冷即换之,候目开,以温酒及清粥稍稍与之。若不先温其心便以火炙,即冷气与火争,必死。
又用毡或藁荐卷之,以索系,令二人相对踏令兖转、往来如衦古旱切,摩展衣也。毡法,候四肢温即止。
魇死,不得用灯火照,不得近前急唤,多杀人。但痛咬其足根及足拇指畔及唾其面必活。
魇不省者,移动些小卧处,徐徐吃之即省。夜间魇者,元有灯即存,元无灯切不可用灯照。又用笔管吹两耳,及取病人头发二七茎捻作绳,刺入鼻中。又盐汤灌之。
又研韭汁半盏灌鼻中,冬用根亦得。
又灸两足大拇指聚毛中三七壮,聚毛,乃脚指向上生茅处。○又皂角末如大豆许吹两鼻内,得嚏则气通,三四日者尚可救。
中恶客忤卒死。凡卒死,或先病,及睡卧间忽然而绝,皆是中恶也。用韭黄心于男左女右鼻内,刺入六七寸,令目间血出即活。○视上唇内沿,有如粟米粒,以针挑破。
又用皂角或生半夏末,如大豆许吹入两鼻。又用羊屎烧烟薰鼻中。又绵浸好酒半盏,手按令汁入鼻中,及提其两手,勿令惊,须臾即活。
又灸脐中百壮,鼻中吹皂角末,或研韭汁灌耳中。
又用生菖蒲,研取汁一盏灌之。
杀伤,凡杀伤不透膜者,乳香、没药各一,皂角子大,研烂,以小便半盏、好酒半盏同煎,通口服。然后用花蕊石散或乌贼鱼骨,或龙骨为末,傅疮口上,立止。
推官宋瑑定验两处杀伤,气偶未绝,亟令保甲各取葱白热锅炒熟,遍傅伤处,继而呻吟,再易葱而伤者无痛矣。曾以语乐平知县鲍旂,及再会,鲍曰:“葱白甚妙。乐平人好斗多伤,每有杀伤公事,未暇诘问,先将葱白傅伤损处,活人甚多,大辟为之减少。”出张声道《经验方》。
胎动不安,凡妇人因争斗胎不安,腹内气刺、痛胀、上喘者:
川芎一两半当归半两
右为细末,每服二钱。酒一大盏煎六分,炒生姜少许在内尤佳。又用苎麻根一大把净洗,入生姜三五片、水一大盏煎至八分,调粥饭与服。
惊怖死者,以温酒一两杯,灌之即活。
五绝及堕、打、卒死等,但须心头温暖,虽经日亦可救。先将“死人”盘屈在地上,如僧打坐状,令一人将“死人”头发控放低,用生半夏末以竹筒或纸简、笔管吹在鼻内。如活,却以生姜自然汁灌之,可解半夏毒。五绝者,产、“魅”、缢、压、溺。治法:单方,半夏一味。
卒暴、堕氵颠 、筑倒及鬼魇死,若肉未冷,急以酒调苏合香圆灌入口,若下喉去可活。


(一)如果吊死是从早到夜的,虽然尸体冷了也可救,从夜到早的,稍难。如果心头还温,一日以上的还可救。解救时不可截绳,但慢慢抱住解下,放之使仰卧,令一人用脚登住死者两肩,以手揪住其头发把头向上拉令紧使脖颈平直通顺,一人微微揉弄其喉咙,摩擦其胸上使它散动,一人按摩其臂、腿使它曲伸。如果已经僵硬了,但渐渐强使它弯曲。又按其腹,这样经过一顿饭时之久,就会气从口出,恢复呼吸。等眼睁开了,便不用拨弄他,再以少许官桂汤及粥给他吃,使润喉咙,更令二人以笔管吹其耳内。如能按照这样办法救治,没有不活的。
又法,紧用手按住死者的口,勿使通气,约两个时辰左右,气憋急了即活。
又法,用皂角、细辛等分研成细末,像一粒大豆那么多,吹入两鼻孔。
(二)水淹死一宿的还可以救,用皂角捣烂以绵絮包裹纳入肛门中,须臾出水即活。
又法,拳屈死人两腿,着人肩上,让死人背贴活人背,扛着走动,吐出水即活。
又法,先打泥壁一堵,置地上,却以死者仰卧在它的上面,再以壁土覆盖死者,止露口眼,自然水气被吸收到泥土中去,死者便能复活。洪承相在番阳,有溺水的,身僵气绝,用这个法子救治,即复活。
又法,炒热沙覆在死人面上,上下着沙,只留出口、耳、鼻,沙变冷湿了,再换,更换数次,即复活。
又法,用醋半盏,灌入死者的鼻中。
又法,用绵包裹石灰纳入死者的肛门中,水出即活。
又法,将死者倒悬起来,以好酒灌入鼻中和肛门。
又法,将死者倒悬解去衣服,去掉脐中垢,令两人以笔管吹死者的两耳。
又法,急解去死人的衣服,在肚脐上灸百壮。
(三)中暑死在行路上的,速用刀器掘开一穴,放水进去捣,却取烂泥浆,以灌死者,即活。中暑昏迷不省人事的,与冷水吃即死。但急取灶间微热灰壅偎之,再用稍热的水蘸手巾熨腹胁间,良久苏醒。不宜便与冷物吃。
(四)冻死的,四肢直,口紧闭,有微气出的,用大锅炒灰使暖,用袋盛好熨心上,冷了就换,等眼睁开,以温酒及清粥稍稍给些吃。如果不先温其心,便以火烤,则冷气与火相争必死。
又法,用毡或草席把冻死者卷起来,以绳系定,令二人相对,踏令滚转往来,如衦法,候四肢温即止。
(五)魇死,不得用灯火照,不得近前急唤,用灯火照、近前急唤多致人于死。但痛咬其脚跟,及脚拇指畔,以及唾其面,必活。魇不省的,少许移动卧处,徐徐唤之即省。夜间魇的,原来有灯的即保留灯,原来无灯的切不可用灯照。
又法,用笔管吹两耳,及取病人头发二七茎,捻作绳刺入鼻中。
又法,用盐汤灌之。
又法,研y汁半盏,灌鼻中。冬季用y根也可以。
又法,灸两脚大拇指聚毛中三七壮(聚毛是脚指向上生毛的地方)。
又法,用皂角末,如一粒大豆那样多,吹入两鼻内,打喷嚏便气通,三四天的还可以救活。
(六)中恶客忤暴死。凡暴死,或先病及睡卧间忽然而死的,都是中恶。用韭黄心于男左女右鼻内,刺入六七寸,使目间血出即活。
看上唇内沿,有像粟米粒样的小疙疸,用针挑破。
又法,用皂角或生半夏末,像一粒大豆那样多,吹入两鼻中。
又法,用羊屎烧烟熏鼻中。
又法,用绵浸蘸好酒半盏,手捏使汁滴入鼻中,及提其两手,勿令惊,片刻即活。
又法,灸脐中百壮,鼻中吹皂角末,或研韭汁灌耳中。
又法,用生菖蒲研取汁一盏,灌之。
(七)杀伤。凡杀伤不透过内膜的,用乳香没药各一,如皂角子大研烂,以小便半盏,好酒半盏同煎,通温服,然后用花蕊石散,或乌贼鱼骨,或龙骨为末,敷疮口上立止。
推官宋瑑,定验一个两处杀伤的人,气偶未绝,急令保甲各取葱白放在热锅里炒熟,遍敷伤处,继而呻吟,再调换一次葱,而伤者居然没有疼痛了。后来他曾把这件事告诉乐平知县鲍旗。及再会,鲍说:葱白很妙。乐平地方的人好斗多伤,每当遇到杀伤的案件,犯人未及审问,便先将葱白敷伤处,救活人命很多,死刑因此减少。张榜称道,推广为经验方。
(八)胎动不安。凡妇女因争斗致使胎孕不安,腹内刺痛发胀气喘的,用
川芎一两半 当归半两
研成细末,每服二钱,酒一大盏煎成六分,炒生姜少许放在里面,尤其好。
又法,用苎麻根一大把,净洗,放入生姜三五片,水一大盏煎成八分,调粥饭给吃。
(九)惊吓死的,用温酒一两杯灌之,即活。
(十)五绝及堕打急死的,但须心头温暖,虽经日也可以救。先将死人盘屈在地上,如同和尚打坐的样子,令一人将死人头发控放低,用生半夏末以竹筒或纸筒笔管吹在鼻内。如果活了,再以生姜纯汁灌之,可解半夏毒。(五绝,指产、魅、缢、压、溺。治法:单方半夏一味。)
(十一)暴死、堕跌、撞倒及鬼魇死的,如果尸身未冷,急以酒调苏合香丸灌入口中,如能下喉去,可以复活。


\chapter{验状说}

凡验状,须开具死人尸首元在甚处?如何顿放?彼处四至?有何衣服在彼?逐一各检劄名件。其尸首有无雕青、灸瘢?旧有何缺折肢体及伛偻、拳跛、秃头、青紫、黑色、红志、肉瘤、蹄踵诸般疾状,皆要一一于验状声载,以备证验诈伪,根寻本原推勘。及有不得姓名人尸首,后有骨肉陈理者,便要验状证辨观之。今之验状若是简略,具述不全,致妨久远照用。况验尸首,本缘非理、狱囚、军人、无主死人,则委官定验,兼官司信凭检验状推勘,何可疏略?又况验尸失当,致罪非轻,当是任者切宜究之。


凡验尸状必须开具:死人尸体原来在什么地方,怎样安放,那里的四至,有什么衣服在那里,要逐一检验札记下名称件数。该尸体身上有没有雕专灸瘢,旧有什么缺折肢体以及伛偻、拳跛、秃头、青紫、黑色、红志、肉瘤、蹄踵等各种疾状,都要在验尸状中一一记明,以备证验诈伪,追寻本原,推究审问;以及遇有不知姓名人的尸体,后来有骨肉亲属出来申诉理论的,便要用验尸状辨证查看。今天的验状,这样简略,记述不全,以致妨碍长远应用。况且检验尸体,本是因为凶死、狱囚、军人、无主死人等,才派官定验,兼以官府信凭验尸状推究审讯,怎么可以疏略?又何况验尸失当,获罪不轻。担当此项职务者,切要很好研究啊!


\backmatter

\end{document}