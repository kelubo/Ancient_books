% 肘后备急方
% 肘后备急方.tex

\documentclass[12pt,UTF8]{ctexbook}

% 设置纸张信息。
\usepackage[a4paper,twoside]{geometry}
\geometry{
	left=25mm,
	right=25mm,
	bottom=25.4mm,
	bindingoffset=10mm
}

% 设置字体,并解决显示难检字问题。
\xeCJKsetup{AutoFallBack=true}
\setCJKmainfont{SimSun}[BoldFont=SimHei, ItalicFont=KaiTi, FallBack=SimSun-ExtB]

% 目录 chapter 级别加点(.)。
\usepackage{titletoc}
\titlecontents{chapter}[0pt]{\vspace{3mm}\bf\addvspace{2pt}\filright}{\contentspush{\thecontentslabel\hspace{0.8em}}}{}{\titlerule*[8pt]{.}\contentspage}

% 设置 part 和 chapter 标题格式。
\ctexset{
	part/name= {卷,},
	part/number={\chinese{part}},
	chapter/name={第,篇},
	chapter/number={\chinese{chapter}}
}

% 设置古文原文格式。
\newenvironment{yuanwen}{\bfseries\zihao{4}}

% 设置署名格式。
\newenvironment{shuming}{\hfill\bfseries\zihao{4}}

% 注脚每页重新编号,避免编号过大。
\usepackage[perpage]{footmisc}

\title{\heiti\zihao{0} 肘后备急方}
\author{葛洪}
\date{晋}

\begin{document}

\maketitle
\tableofcontents

\frontmatter
\chapter{前言、序言}

《肘后备急方》为晋代葛洪所撰。本书选《玉函方》书中精要之方,重点选取常见病、急性病的简便治疗方药,编撰成书,以备临证急用。

第一卷到第四卷为“内病”,主治方包括中恶、心腹痛、伤寒、时气、中风、水病、黄疸、虚损、上气咳嗽等疾病,大约相当于内科疾病,第五卷到第六卷为“外发病”,主治方包括痈疽、疮毒、癣疥、耳目等疾病,大约相当于外科类疾病,第七卷为“他犯病”,主治方包括虫兽伤、中毒等疾病,第八卷为百病备急丸散及牲畜病等。

本次整理以文渊阁四库全书本《肘后备急方》为底本。





整理说明

《肘后备急方》为晋代葛洪所撰,又称《肘后救卒方》,简称《肘后方》。原书三卷,南北朝齐梁间道教名士陶弘景增补为一百零一方。后至金代,汴京国子监博士杨用道又再次予以增补,补入了宋代医家唐慎微《证类本草》中的许多附方,名《附广肘后备急方》,共八卷,内容更为丰富而实用。现将校注中有关情况说明如下:

1.本次整理以文渊阁四库全书本《肘后备急方》为底本,以清乾隆五十九年甲寅(1794)於然室刻本为校本。他校本有:《备急千金要方》《外台秘要》《太平圣惠方》《证类本草》等。

2.采用现代标点方法,对原书进行标点。将原书中繁体字竖排改为规范简体字横排。原书中“右”字用以代表前文者,改为“上”字;代表后文的“左”字,改为“下”字。

3.异体字、古字、通假字,予以径改。

4.原书文字误、脱、衍、倒者,有校本或他校资料可据者,则据校本或他校资料改,无校本或他校资料可据者,予以保留。

5.原书中药名用字一律径改为现行通用药名用字,如“黄耆”改为“黄芪”、“柭葜”改为“菝葜”。





导 读

《肘后备急方》为晋代葛洪所著,又称《肘后救卒方》,简称《肘后方》。葛洪(283—343),字稚川,号抱朴子,晋代著名医药学家、道教代表人物。葛洪有着丰富的医药学知识,在其行医实践中,常注意总结自身临床心得与民间医疗经验。他曾著有《葛洪玉函方》(一作《金匮药方》,该著已亡佚),该书卷帙宏大,达一百卷之巨,后自感《葛洪玉函方》过于浩繁,不便于携带,不利于医者临床急用时查阅,遂选取书中精要之方,重点选取常见病、急性病的简便治疗方药,编撰成书,以备临证急用,其书名为《肘后救卒方》。“肘后”之意,意为易于携带,可置之于袖中肘后,以便随时查阅。葛洪在《肘后方》序中说:“诸家各作备急,既不能穷诸病状,兼多珍贵之药,岂贫家野居所能立办”,故选录“率多易得之药,其不获已,须买之者,亦皆贱价,草石所在皆有”,从而撰为本书。由此可见该书简、便、廉、验的宗旨和特色,充分体现了实用性与临床急救性质,可以说是我国最早的临床急救手册。

《肘后方》的内容

葛洪《肘后方》原书三卷,后经南北朝齐梁间道教名士陶弘景增补为一百零一方。陶弘景学贯儒、释、道三教,一百零一方,取意于佛经“人用四大成身,一大辄有一百一病”之说,故名之曰《肘后百一方》。《肘后百一方》仍为三卷,上卷为治“内病”三十五方,中卷为治“外发病”三十五方,下卷为治“为物所苦病”三十一方。后至金代,汴京国子监博士杨用道又再次予以增补,补入了宋代医家唐慎微《证类本草》中的许多附方,名《附广肘后备急方》,共八卷,内容更为丰富而实用。

现存《肘后方》虽有部分内容亡佚,但仍保持八卷的体例。第一卷到第四卷为“内病”主治方,包括中恶、心腹痛、伤寒、时气、中风、水病、黄疸、虚损、上气咳嗽等疾病,大约相当于内科疾病。第五卷到第六卷为“外发病”主治方,包括痈疽、疮毒、癣疥、耳目等疾病,大约相当于外科类疾病。第七卷为“他犯病”主治方,包括虫兽伤、中毒等疾病。第八卷为百病备急丸散及牲畜病等。

《肘后方》的学术成就

《肘后方》对传染病的认识,达到了很高的水平,很多对疾病的认识是世界医学中记载最早的。

如书中详细描述了天花的临床症状,“比岁有病时行,仍发疮,头面及身,须臾周匝,状如火疮,皆戴白浆,随决随生,不即治,剧者多死,治得瘥后,疮瘢紫黑,弥岁方灭”,这是世界医学史上最早关于天花的认识,比国外阿拉伯医生累塞斯对天花的描述早近六百年。

《肘后方》卷七中还最早记载了恙虫病(书中称之为“沙虱”)的临床特征,并指出本病见于岭南。恙虫病至今还在我国华南地区、西南地区以及日本、东南亚等地局部流行,是由立克次体感染所致。葛洪对恙虫病(沙虱)临床特征的描述确实较为符合临床实际。而国外一直到公元19世纪,才由日本人做出了一定的研究,比葛洪晚了一千六百年。

此外,《肘后方》还指出了鬼注、尸注(相当于肺结核一类的疾病)的低热、慢性消耗性症状以及“乃至灭门”的传染性。书中对疥虫的发现和对恶脉(淋巴结炎)、溧疽(坏疽)、卒心腹痛(急腹症)等疾病的认识也是较早的。

《肘后方》中对很多疾病的治疗及预防方法在世界医学史上也具有很重要的意义。如《肘后方》中治疗狂犬病,首创用狂犬脑组织敷贴在狂犬咬伤的创口上,以防治狂犬病,实具有免疫学思想。德国发现白喉抗毒素的细菌学家贝林(1854—1917)对此盛赞说:“中国人远在两千年前,即知以毒攻毒之医理,这是合乎现代科学的一句古训。”

《肘后方》还发现以鲜青蒿绞汁服可以治疗疟疾,这为我国当代药理研究提取出治疗疟疾的有效成分青蒿素,提供了宝贵的资料,也是我国当代医学的一大成就。

《肘后方》书中还记载了吹气人工呼吸法、烧灼法止血、腹穿、导尿、灌肠、清疮、引流、骨折外固定、关节脱位整复、水银软膏治蛲虫等治疗技术,这些记载也是较早的,在一千七百年前就有了这样的认识水平和治疗水平,是十分难能可贵的。

《肘后方》以急救便携为目的,以简、便、廉、验为特色,在很多疾病的认识和治疗方法上都有很高的成就,甚至达到了古代世界医学史的领先水平。

《肘后方》作者葛洪

葛洪(283—363),字稚川,自号抱朴子,丹阳句容(今属江苏)人,东晋道教学者、著名炼丹家、医药学家,道教灵宝派著名代表人物。葛洪为三国方士葛玄的侄孙,世称小仙翁。本出身江南士族,其祖在三国吴时历任御史中丞、吏部尚书等要职,封寿县侯,其父悌也曾为晋朝邵陵太守。葛洪为悌之第三子,颇受其父之娇宠。葛洪13岁时,其父去世,从此家道中落。16岁时,开始读儒家经典,尤喜“神仙导养之法”。后从郑隐学炼丹秘术,颇受器重。

西晋太安元年(302),其师郑隐知季世之乱,江南将鼎沸,乃携入室弟子东投霍山,唯葛洪仍留丹阳。太安二年,张昌、石冰于扬州起义,大都督秘任葛洪为将兵都尉,由于镇压起义军有功,迁伏波将军。后又师事鲍靓,继修道术,深得鲍靓器重。建兴四年(316),东晋朝廷赐爵葛洪为关内侯,并授以官职,葛洪皆固辞不就。后听闻交趾郡产丹砂,遂率子侄同行。南行至广州,为刺史邓岳所留,乃止于罗浮山炼丹。于东晋兴宁元年(363)卒,享年81岁。

葛洪继承并改造了早期道教的神仙理论,在《抱朴子内篇》中,他不仅全面总结了晋以前的神仙理论,并系统地总结了晋以前的神仙方术,包括守一、行气、导引和房中术等;同时又将神仙方术与儒家的纲常名教相结合,强调“欲求仙者,要当以忠孝和顺仁信为本,若德行不修,而但务方术,皆不得长生也”。他认为神仙养生为内,儒术应世为外。提出治乱世应用重刑,以严刑峻法匡时佐世。葛洪在坚信炼制和服食金丹可得长生成仙的思想指导下,长期从事炼丹,在其炼丹实践中,积累了丰富的经验,认识了物质的某些特征及其化学反应。他在《抱朴子内篇》中的《金丹》和《黄白》两篇中,系统地总结了晋以前的炼丹成就,记载了大量的古代丹经和丹法,勾画了中国古代炼丹的历史梗概,对隋唐炼丹术的发展具有重大影响,成为炼丹史上一位承前启后的著名炼丹家。

葛洪精晓医学和药物学,主张道士应该兼修医术。认为“古之初为道者,莫不兼修医术,以救近祸焉”,修道者如不兼习医术,一旦“病痛及己”,便“无以攻疗”,不仅不能长生成仙,甚至连自己的性命也难保。葛洪撰有《肘后救卒方》和《玉函方》等医学著作。他说:“余所撰百卷,名曰《玉函方》,皆分别病名,以类相续,不相杂错,其《救卒》三卷,皆单行径易,约而易验,篱陌之间,顾眄皆药,众急之病,无不毕备,家有此方,可不用医。”关于两书情况,本章于前已有相关内容,此处不予赘述。葛洪在《抱朴子内篇·仙药》中对许多药用植物的形态特征、生长习性、主要产地、入药部分及治病作用等,均做了详细的记载和说明,对我国后世医药学的发展也产生了很大的影响。葛洪的妻子鲍姑擅长灸法,是我国医学史上有记载的第一位女灸家。

葛洪一生著作宏富,撰有《抱朴子内篇》二十卷、《抱朴子外篇》五十卷、《碑颂诗赋》一百卷、《军书檄移章表笺记》三十卷、《神仙传》十卷、《隐逸传》十卷、《玉函方》一百卷、《肘后备急方》三卷。但多亡佚,《正统道藏》和《万历续道藏》共收其著作十三种,以后人误题或伪托者居多。

\chapter{序}

医有方古也。古以来著方书者,无虑数十百家,其方殆未可以数计,篇帙浩瀚,苟无良医师,安所适从?况穷乡远地,有病无医,有方无药,其不罹夭折者几希。丹阳葛稚川,夷考古今医家之说,验其方简要易得,针灸分寸易晓,必可以救人于死者,为《肘后备急方》。使有病者得之,虽无韩伯休,家自有药;虽无封君达,人可为医,其以备急固宜。华阳陶弘景曰:葛之此制,利世实多,但行之既久,不无谬误。乃著《百一方》,疏于《备急》之后,讹者正之,缺者补之,附以炮制服食诸法,纤悉备具,仍区别内、外、他犯为三条,可不费讨寻,开卷见病,其以备急益宜。葛、陶二君,世共知为有道之士,于学无所不贯,于术无所不通,然犹积年仅成此编,盖一方一论,已试而后录之,非徒采其简易而已。人能家置一帙,遇病得方,方必已病。如历卞和之肆举皆美玉,入伯乐之厩无非骏足,可以易而忽之邪。

葛自序云,人能起信,可免夭横,意可见矣。自天地大变,此方湮没几绝,间一存者,以自宝,是岂制方本意?连帅乌侯,夙多疹疾,宦学之余,留心于医药,前按察河南北道,得此方于平乡郭氏,郭之妇翁得诸汴之掖庭,变乱之际,与身存亡,未尝轻以示人,迨今而出焉,天也,侯命上刻之,以趣其成,唯恐病者见方之晚也。虽然方之显晦,而人之生死休戚系焉。出自有时,而隐痛恻怛,如是其急者,不忍人之心也。有不忍人之心,斯有不忍人之政矣,则侯之仁斯民也,岂直一方书而已乎?方之出,乃吾仁心之发见者也,因以序见命,特书其始末,以告夫未知者。

至元丙子季秋稷亭段成巳题

\mainmatter

% 增加空行
~\\

\part{}

\chapter{救卒中恶死方第一}
\chapter{救卒死尸蹶方第二}
\chapter{救卒客忤死方第三}
\chapter{治卒得鬼击方第四}
\chapter{治卒魇寐不寤方第五}
\chapter{治卒中五尸方第六}
\chapter{治尸注鬼注方第七}
\chapter{治卒心痛方第八}
\chapter{治卒腹痛方第九}
\chapter{治心腹俱痛方第十}
\chapter{治卒心腹烦满方第十一}

\part{}

\chapter{治卒霍乱诸急方第十二}
\chapter{治伤寒时气温病方第十三}
\chapter{治时气病起诸劳复方第十四}
\chapter{治瘴气疫疠温毒诸方第十五}

\part{}

\chapter{治寒热诸疟方第十六}
\chapter{治卒发癫狂病方第十七}
\chapter{治卒得惊邪恍惚方第十八}
\chapter{治中风诸急方第十九}
\chapter{治卒风喑不得语方第二十}
\chapter{治风毒脚弱痹满上气方第二十一}
\chapter{治服散卒发动困笃方第二十二}
\chapter{治卒上气咳嗽方第二十三}
\chapter{治卒身面肿满方第二十四}

\part{}

\chapter{治卒大腹水病方第二十五}
\chapter{治卒心腹癥坚方第二十六}
\chapter{治心腹寒冷食饮积聚结癖方第二十七}
\chapter{治胸膈上痰诸方第二十八}
\chapter{治卒患胸痹痛方第二十九}
\chapter{治卒胃反呕方第三十}
\chapter{治卒发黄疸诸黄病第三十一}
\chapter{治卒患腰胁痛诸方第三十二}
\chapter{治虚损羸瘦不堪劳动方第三十三}
\chapter{治脾胃虚弱不能饮食方第三十四}
\chapter{治卒绝粮失食饥惫欲死方第三十五}

\part{}

\chapter{治痈疽妒乳诸毒肿方第三十六}
\chapter{治肠痈肺痈方第三十七}
\chapter{治卒得癞皮毛变黑方第四十}
\chapter{治卒得虫鼠诸瘘方第四十一(后有瘰疬)}
\chapter{治卒阴肿痛卵方第四十二}

\part{}


\chapter{治目赤痛暗昧刺诸病方第四十三}
\chapter{治卒耳聋诸病方第四十四}
\chapter{治耳为百虫杂物所入方第四十五}
\chapter{治卒食噎不下方第四十六}
\chapter{治卒诸杂物鲠不下方第四十七}
\chapter{治卒误吞诸物及患方第四十八}
\chapter{治面皰发秃身臭心昏鄙丑方第四十九}

\part{}

\chapter{治为熊虎爪牙所伤毒痛方第五十}
\chapter{治卒有犬所咬毒方第五十一}
\chapter{治卒毒及狐溺棘所毒方第五十二}
\chapter{治卒青蛙蝮虺众蛇所螫方第五十三}
\chapter{治蛇疮败蛇骨刺人入口绕身诸方第五十四}
\chapter{治卒入山草禁辟众蛇药术方第五十五}
\chapter{治卒蜈蚣蜘蛛所螫方第五十六}
\chapter{治卒虿螫方第五十七}
\chapter{治卒蜂所螫方第五十八}
\chapter{治卒蝎所螫方第五十九}
\chapter{治中蛊毒方第六十}
\chapter{治卒中溪毒方第六十一}
\chapter{治卒中射工水弩毒方第六十二}
\chapter{治卒中沙虱毒方第六十三}
\chapter{治卒服药过剂烦闷方第六十四}
\chapter{治卒中诸药毒救解方第六十五}
\chapter{治食中诸毒方第六十六}
\chapter{治防避饮食诸毒方第六十七}
\chapter{治卒饮酒大醉诸病方第六十八}

\part{}

\chapter{治百病备急丸散膏诸要方第六十九}
\chapter{治牛马六畜水谷疫疠诸病方第七十}



救卒中恶死方第一

救卒死,或先病痛,或常居寝卧,奄忽而绝,皆是中死,救之方。

一方,取葱黄心刺其鼻,男左、女右,入七八寸。若使目中血出,佳。扁鹊法同。是后吹耳条中。葛当言此云吹鼻,故别为一法。

又方,令二人以衣壅口,吹其两耳,极则易,又可以筒吹之,并捧其肩上,侧身远之,莫临死人上。

又方,以葱刺耳,耳中、鼻中血出者莫怪,无血难治,有血是候。时当捧两手忽放之,须臾死人自当举手捞人,言痛乃止。男刺左鼻、女刺右鼻中,令入七八寸余,大效。亦治自缢死。与此扁鹊方同。

又方,以绵渍好酒中,须臾,置死人鼻中,手按令汁入鼻中,并持其手足,莫令惊。

又方,视其上唇里弦弦者,有白如黍米大,以针决去之。

又方,以小便灌其面,数回即能语。此扁鹊方法。

又方,取皂荚如大豆,吹其两鼻中,嚏则气通矣。

又方,灸其唇下宛宛中承浆穴十壮,大效矣。

又方,割雄鸡颈取血,以涂其面,干复涂,并以灰营死人一周。

又方,以管吹下部,令数人互吹之,气通则活。

又方,破白犬以拓心上。无白犬,白鸡亦佳。

又方,取雄鸭,就死人口上断其头,以热血沥口中,并以竹筒吹其下部,极则易人,气通下即活。

又方,取牛马粪尚湿者,绞取汁,灌其口中,令入喉。若口已噤者,以物强发之;若不可强者,乃扣齿下;若无新者,以人溺解干者,绞取汁。此扁鹊云。

又方,以绳围其死人肘腕,男左女右,毕伸绳从背上大槌度以下,又从此灸,横行各半绳。此法三灸各三,即起。

又方,令爪其病患人中,取醒。不者,卷其手,灸下文头,随年。

又方,灸鼻人中,三壮也。

又方,灸两足大指爪甲聚毛中,七壮。此华佗法。一云三七壮。

又方,灸脐中,百壮也。

扁鹊法又云:断豚尾,取血饮之,并缚豚以枕之,死人须臾活。

又云:半夏末如大豆,吹鼻中。

又方,捣女背屑重一钱匕,开口纳喉中,以水苦酒,立活。

按,此前救卒死四方,并后尸蹶事,并是魏大夫传中正一真人所说扁鹊受长桑公子法,寻此传出世,在葛后二十许年,无容知见,当是斯法久已在世,故或言楚王,或言赵王,兼立语次第,亦参差故也。

又张仲景诸要方,捣薤汁以灌鼻中。

又方,割丹雄鸡冠血,管吹纳鼻中。

又方,以鸡冠及血涂面上,灰围四边,立起。

又方,猪脂如鸡子大,苦酒一升煮沸,以灌喉中。

又方,大豆二七枚,以鸡子白并酒和,尽以吞之。

救卒死而壮热者。矾石半斤,水一斗半,煮消以渍脚,令没踝。

救卒死而目闭者。骑牛临面,捣薤汁灌之耳中,吹皂荚鼻中,立效。

救卒死而张目及舌者。灸手足两爪后十四壮了,饮以五毒诸膏散有巴豆者。

救卒死而四肢不收,矢便者。马屎一升,水三斗,煮取二斗以洗之。又取牛洞一升,温酒灌口中。洞者,稀粪也。灸心下一寸、脐上三寸、脐下四寸各一百壮,瘥。

若救小儿卒死而吐利,不知是何病者。马屎一丸,绞取汁以吞之。无湿者,水煮取汁。

又有备急三物丸散及裴公膏,并在后备急药条中,救卒死尤良,亦可临时合用之。凡卒死、中恶及尸蹶,皆天地及人身自然阴阳之气忽有乖离否隔,上下不通,偏竭所致,故虽涉死境,犹可治而生,缘气未都竭也。当尔之时,兼有鬼神于其间,故亦可以符术而获济者。

附方

扁鹊云,中恶与卒死鬼击亦相类,已死者为治,皆参用此方。

捣菖蒲生根,绞汁灌之,立瘥。

尸厥之病,卒死脉犹动,听其耳中如微语声,股间暖是也,亦此方治之。

孙真人治卒死方,以皂角末吹鼻中。

救卒死尸蹶方第二

尸蹶之病,卒死而脉犹动,听其耳中,循循如啸声,而股间暖是也,耳中虽然啸声而脉动者,故当以尸蹶,救之方。

以管吹其左耳中极三度,复吹右耳三度,活。

又方,捣干菖蒲,以一枣核大,着其舌下。

又方,灸鼻人中,七壮,又灸阴囊下去下部一寸,百壮。若妇人,灸两乳中间。又云,爪刺人中良久,又针人中至齿,立起。

此亦全是《魏大夫传》中扁鹊法,即赵太子之患。又张仲景云,尸一蹶,脉动而无气,气闭不通,故静然而死也。

以菖蒲屑纳鼻两孔中,吹之,令人以桂屑着舌下。又云扁鹊法。治楚王效。

又方,剔左角发,方二寸,烧末,以酒灌,令入喉,立起也。

又方,以绳围其臂腕,男左女右,绳从大椎上度下行脊上,灸绳头五十壮,活。此是扁鹊秘法。

又方,熨其两胁下,取灶中墨如弹丸,浆水和饮之,须臾,三四,以管吹耳中,令三四人更互吹之。又小管吹鼻孔,梁上尘如豆,着中吹之,令入,瘥。

又方,白马尾二七茎,白马前脚目二枚,合烧之,以苦酒丸如小豆。开口吞二丸,须臾,服一丸。

又方,针百会,当鼻中,入发际五寸,针入三分。补之针足大指甲下肉侧去甲三分;又针足中指甲上,各三分,大指之内,去端韭叶;又针手少阴、锐骨之端各一分。

又方,灸膻中穴二十八壮。

救卒客忤死方第三

客忤者,中恶之类也,多于道门门外得之,令人心腹绞痛胀满,气冲心胸,不即治,亦杀人,救之方。

灸鼻人中三十壮,令切鼻柱下也。以水渍粳米,取汁一二升,饮之。口已噤者,以物强发之。

又方,捣墨,水和,服一钱匕。

又方,以铜器若瓦器贮热汤,器着腹上;转冷者,撤去衣,器亲肉;大冷者,易以热汤,取愈则止。

又方,以三重衣着腹上,铜器着衣上,稍稍,少许茅于器中烧之,茅尽益之,勿顿多也,取愈乃止。

又方,以绳横度其人口,以度其脐,去四面各一处,灸各三壮,令四火俱起,瘥。

又方,横度口中,折之,令上头着心下,灸下头五壮。

又方,真丹方寸匕,蜜三合,和服。口噤者,折齿下之。

扁鹊治忤有救卒符并服盐汤法,恐非庸世所能,故不载。而此病即今人所谓中恶者,与卒死、鬼击亦相类,为治皆参取而用之已死者。

捣生菖蒲根,绞取汁,含之,立瘥。

卒忤,停尸不能言者,桔梗(烧)二枚,末之,服。

又方,末细辛、桂,分等,纳口中。

又方,鸡冠血和真珠,丸如小豆,纳口中,与三四枚,瘥。

若卒口噤不开者,末生附子,置管中,吹纳舌下,即瘥矣。

又方,人血和真珠,如梧桐子大,二丸,折齿纳喉中,令下。

华佗卒中恶、短气欲死。灸足两拇指上甲后聚毛中,各十四壮,即愈。未瘥,又灸十四壮。前救卒死方,三七壮,已有其法。

又张仲景诸要方,麻黄四两,杏仁七十枚,甘草一两,以水八升,煮取三升,分令咽之,通治诸感忤。

又方,韭根一把,乌梅二十个,茱萸半斤,以水一斗煮之,以病患栉纳中三沸,栉浮者生,沉者死。煮得三升,与饮之。

又方,桂一两,生姜三两,栀子十四枚,豉五合,捣,以酒三升,搅,微煮之,味出去滓,顿服取瘥。

飞尸走马汤,巴豆二枚,杏仁二枚,合,绵缠椎,令碎,着热汤二合中,指捻令汁出,便与饮之,炊间顿下饮,瘥。小量之,通治诸飞尸鬼击。

又有诸丸散,并在备急药中。客者,客也;忤者,犯也,谓客气犯人也。此盖恶气,治之多愈,虽是气来鬼鬼毒厉之气,忽逢触之其衰歇,故不能如自然恶气治之,入身而侵克脏腑经络,瘥后,犹宜更为治,以消其余势。不尔,亟终为患,令有时辄发。

附方

《外台秘要》治卒客忤,停尸不能言。细辛、桂心,等分,纳口中。

又方,烧桔梗二两,末,米饮服,仍吞麝香如大豆许,佳。

《广利方》治卒中客忤垂死。麝香一钱,重研,和醋二合,服之,即瘥。

治卒得鬼击方第四

鬼击之病,得之无渐,卒着如人力刺状,胸胁腹内,绞急切痛,不可抑按,或即吐血,或鼻中出血,或下血,一名鬼排。治之方。

灸鼻下人中一壮,立愈。不瘥,可加数壮。

又方,升麻、独活、牡桂,分等。末,酒服方寸匕,立愈。

又方,灸脐下一寸,二壮。

又方,灸脐上一寸,七壮,及两踵白肉际,取瘥。

又方,熟艾如鸭子大,三枚,水五升,煮取二升,顿服之。

又方,盐一升,水二升,和搅饮之,并以冷水噀之,勿令即得吐,须臾吐,即瘥。

又方,以粉一撮,着水中搅,饮之。

又方,以淳酒吹纳两鼻中。

又方,断白犬一头,取热犬血一升。饮之。

又方,割鸡冠血以沥口中,令一咽,仍破此鸡,以搨心下,冷乃弃之于道边。得乌鸡弥佳,妙。

又方,牛子屎一升,酒三升,煮服之。大牛亦可用之。

又方,刀鞘三寸,烧末,水饮之。

又方,烧鼠屎,末,服如黍米。不能饮之,以少水和,纳口中。

又有诸丸散,并在备急药条中。今巫实见人忽有被鬼神所摆拂者,或犯其行伍,或遇相触突,或身神散弱,或愆负所贻,轻者因而获免,重者多见死亡,犹如燕简辈事,非为虚也。必应死,亦不可,要自不得不救尔。

附方

《古今录验》疗妖魅猫鬼病患不肯言鬼方,鹿角屑捣散,以水服方寸匕,即言实也。

治卒魇寐不寤方第五

卧忽不寤,勿以火照,火照之杀人。但痛啮其踵及足拇指甲际,而多唾其面,即活。

又治之方,末皂角,管吹两鼻中,即起。三四日犹可吹。又以毛刺鼻孔中,男左女右,辗转进之。

又方,以芦管吹两耳,并取病患发二七茎,作绳纳鼻孔中。割雄鸡冠取血,以管吹入咽喉中,大效。

又方,末灶下黄土,管吹入鼻中。末雄黄并桂,吹鼻中,并佳。

又方,取井底泥涂目毕,令人垂头于井中,呼其姓名,即便起也。

又方,取韭捣,以汁吹鼻孔。冬月可掘取根,取汁灌于口中。

又方,以盐汤饮之,多少约在意。

又方,以其人置地,利刀画地,从肩起,男左女右,令周面以刀锋刻病患鼻,令入一分,急持勿动,其人当鬼神语求哀,乃问,阿谁,何故来?当自乞去,乃以指灭向所画地,当肩头数寸,令得去,不可不具诘问之也。

又方,以瓦甑覆病患面上,使人疾打,破甑,则寤。

又方,以牛蹄或马蹄,临魇人上,亦可治卒死。青牛尤佳。

又方,捣雄黄,细筛,管吹纳两鼻中。桂亦佳。

又方,菖蒲末,吹两鼻中,又末纳舌下。

又方,以甑带左索缚其肘后,男左女右,用余稍急绞之,又以麻缚脚,乃诘问其故,约敕解之。令一人坐头守,一人于户内呼病患姓名,坐人应曰诺,在便苏。

卒魇不觉。灸足下大趾聚毛中,二十一壮。

人喜魇及恶梦者。取火死灰,着履中,合枕。

又方,带雄黄,男左女右。

又方,灸两足大趾上聚毛中,灸二十壮。

又方,用真麝香一字于头边。

又方,以虎头枕尤佳。

辟魇寐方。取雄黄如枣核,系左腋下,令人终身不魇寐。

又方,真赤罽方一赤以枕之。

又方,作犀角枕佳。以青木香纳枕中,并带。

又方,魇治卒魇寐久。书此符于纸,烧令黑,以少水和之,纳死人口中,悬鉴死者耳前打之,唤死者名,不过半日,即活。

魇卧寐不寤者,皆魂魄外游,为邪所执录,欲还未得所,忌火照,火照遂不复入。而有灯光中魇者,是本由明出,但不反身中故耳。

附方

《千金方》治鬼魇不寤。皂荚末刀圭,起死人。

治卒中五尸方第六

五尸者(飞尸、遁尸、风尸、沉尸、尸注也,今所载方兼治之),其状腹痛,胀急,不得气息,上冲心胸,旁攻两胁,或磥块涌起,或挛引腰脊,兼治之方。

灸乳后三寸,十四壮,男左女右。不止,更加壮数,瘥。

又方,灸心下三寸,六十壮。

又方,灸乳下一寸,随病左右,多其壮数,即瘥。

又方,以四指尖其痛处,下灸指下际数壮,令人痛,上爪其鼻人中,又爪其心下一寸,多其壮,取瘥。

又方,破鸡子白,顿吞之。口闭者,纳喉中摇顿令下,立瘥。

又方,破鸡子白,顿吞七枚。不可,再服。

又方,理商陆根,熬,以囊贮,更番熨之,冷复易。虽有五尸之名,其例皆相似,而有小异者(飞尸者,游走皮肤,洞穿脏腑,每发刺痛,变作无常也;遁尸者,附骨入肉,攻凿血脉,每发不可得近,见尸丧、闻哀哭便作也;风尸者,淫跃四肢,不知痛之所在,每发昏恍,得风雪便作也;沉尸者,缠结脏腑,冲心胁,每发绞切,遇寒冷便作也;尸注者,举身沉重,精神错杂,常觉昏废,每节气改变,辄致大恶,此一条别有治后熨也)。凡五尸,即身中尸鬼接引也,共为病害,经术甚有消灭之方,而非世徒能用,今复撰其经要,以救其敝方。

雄黄一两,大蒜一两,令相和似弹丸许,纳二合热酒中,服之,须臾,瘥。未瘥,更作。已有疢者,常蓄此药也。

又方,干姜、桂,分等,末之,盐三指撮,熬令青,末,合水服之,即瘥。

又方,捣蒺藜子,蜜丸,服,如胡豆,二丸,日三。

又方,粳米二升,水六升,煮一沸服之。

又方,猪肪八合,铜器煎,小沸,投苦酒八合,相和,顿服,即瘥。

又方,掘地作小坎,水满中,熟搅,取汁服之。

又方,取屋上四角茅,纳铜器中,以三赤,布覆腹,着器布上,烧茅令热,随痛追逐,蹠下痒,即瘥。若瓦屋,削取四角柱烧之,亦得极大神良者也。

又方,桂一赤,姜一两,巴豆三枚,合捣末,苦酒和如泥,以敷尸处,燥,即瘥。

又方,乌臼根(锉)二升,煮令浓,去滓,煎汁,凡五升,则入水一两,服五合至一升,良。

又方,忍冬茎叶(锉)数斛,煮令浓,取汁煎之,服如鸡子一枚,日二三服,佳也。

又方,烧乱发、熬杏仁,等分,捣膏,和丸之,酒服桐子大三丸,日五六服。

又方,龙骨三分,藜芦二分,巴豆一分,捣,和井花水,服如麻子大,如法丸。

又方,漆叶曝干,捣末,酒服之。

又方,鼍肝一具,熟煮,切,食之令尽,亦用蒜薤。

又方,断鳖头烧末,水服,可分为三度,当如肉者,不尽,后发更作。

又方,雄黄一分,栀子十五枚,芍药一两,水三升,煮取一升半,分再服。

又方,栀子二七枚,烧末服。

又方,干姜、附子各一两,桂二分,巴豆三十枚(去心,并生用),捣筛,蜜和捣万杵,服二丸,如小豆大。此药无所不治。

又飞尸入腹刺痛死方。凡犀角、射罔、五注丸,并是好药,别在大方中。治卒有物在皮中,如虾蟆,宿昔下入腹中,如杯不动摇,掣痛不可堪,过数日即煞人方。

巴豆十四枚,龙胆一两,半夏、土瓜子各一两,桂一斤半。合捣碎,以两布囊贮,蒸热,更番以熨之,亦可煮饮,少少服之。

此本在杂治中,病名曰阴尸,得者多死。

治尸注鬼注方第七

尸注、鬼注病者,葛云,即是五尸之中尸注,又挟诸鬼邪为害也。其病变动,乃有三十六种至九十九种,大略使人寒热、淋沥、恍恍、默默,不的知其所苦,而无处不恶,累年积月,渐就顿滞,以至于死,死后复传之旁人,乃至灭门。觉知此候者,便宜急治之,方。

取桑树白皮,曝干,烧为灰,得二斗许,着甑中蒸,令气浃便下,以釜中汤三四斗,淋之又淋,凡三度,极浓止。澄清取二斗,以渍赤小豆二斗一宿,曝干,干复渍灰,汁尽止。乃湿蒸令熟,以羊肉若鹿肉作羹,进此豆饭,初食一升至二升取饱满。微者三四斗愈,极者七八斗。病去时体中自觉疼痒淫淫。或若根本不拔,重为之,神验也。

又方,桃仁五十枚,破,研,以水煮取四升,一服尽当吐。吐病不尽,三两日更作。若不吐,非注。

又方,杜蘅一两,茎一两,人参半两许,瓠子二七枚,松萝六铢,赤小豆二七枚,捣末散,平旦温服方寸匕,晚当吐百种物。若不尽,后更服之也。

又方,獭肝一具,阴干,捣末,水服方寸匕,日三。一具未瘥,更作。姚云神良。

又方,朱砂、雄黄各一两,鬼臼、罔草各半两,巴豆四十枚(去心皮),蜈蚣两枚,捣,蜜和丸,服如小豆。不得下,服二丸亦长将行之。姚氏,烧发灰、熬杏仁紫色,分等,捣如脂,猪脂和酒服,梧桐子大,日三服,瘥。

又有华佗狸骨散、龙牙散、羊脂丸诸大药等,并在大方中。及成帝所受淮南丸,并疗疰易灭门。女子小儿多注车、注船、心闷乱、头痛,吐,有此疹者,宜辟方。

车前子、车下李根皮、石长生、徐长卿各数两,分等,粗捣,作方囊,贮半合,系衣带及头。若注船,下暴惨,以和此共带之,又临入船,刻取此船,自烧作屑,以水服之。

附方

《子母秘录》治尸注。烧乱发如鸡子大,为末,水服之,瘥。

《食医心镜》主传尸、鬼气、咳嗽、痃癖、注气,血气不通,日渐羸瘦。方:桃仁一两,去皮尖,杵碎,以水一升半煮汁,着米煮粥,空心食之。

治卒心痛方第八

治卒心痛。桃白皮煮汁,宜空腹服之。

又方,桂末若干、姜末,二药并可单用,温酒服方寸匕,须臾,六七服,瘥。

又方,驴屎,绞取汁五六合,及热顿服,立定。

又方,东引桃枝一把,切,以酒一升,煎取半升,顿服,大效。

又方,生油半合,温服,瘥。

又方,黄连八两,以水七升,煮取一升五合,去滓,温服五合,每日三服。

又方,当户以坐,若男子病者,令妇人以一杯水以饮之;若妇人病者,令男子以一杯水以饮之。得新汲水尤佳。又以蜜一分,水二分,饮之,益良也。

又方,败布裹盐如弹丸,烧令赤,末,以酒一盏服之。

又方,煮三沸汤一升,以盐一合搅,饮之。若无火作汤,亦可用水。

又方,闭气忍之数十度,并以手大指按心下宛宛中,取愈。

又方,白艾(成熟者)三升,以水三升,煮取一升,去滓,顿服之。若为客气所中者,当吐之虫物。

又方,苦酒一杯,鸡子一枚,着中合搅,饮之。好酒亦可用。

又方,取灶下热灰,筛去炭分,以布囊贮,令灼灼尔,便更番以熨痛上,冷,更熬热。

又方,蒸大豆,若煮之,以囊贮,更番熨痛处,冷复易之。

又方,切生姜若干姜半升,以水二升,煮取一升,去滓,顿服。

又方,灸手中央长指端三壮。

又方,好桂,削去皮,捣筛,温酒服三方寸匕。不瘥者,须臾,可六七服。无桂者,末干姜,佳。

又方,横度病患口折之,以度心厌下,灸度头三壮。

又方,尽地作五行字,撮中央土,以水一升,搅饮之也。

又方,吴茱萸二升,生姜四两,豉一升,酒六升,煮三升半,分三服。

又方,人参、桂心、栀子(擘)、甘草(炙)、黄芩各一两,水六升,煮取二升,分三服,奇效。

又方,桃仁七枚,去皮尖,熟,研,水合顿服,良。亦可治三十年患。

又方,附子二两(炮),干姜一两,捣,蜜丸。服四丸如梧子大,日三。

又方,吴茱萸一两半,干姜,准上桂心一两,白术二两,人参、橘皮、椒(去闭口及子、汗)、甘草(炙)、黄芩、当归、桔梗各一两,附子一两半(炮)。捣,筛,蜜和为丸,如梧子大,日三,稍加至十丸、十五丸,酒饮下,饭前食后任意,效验。

又方,桂心八两,水四升,煮取一升,分三服。

又方,苦参三两,苦酒升并,煮取八合,分再服。亦可用水无煮者,生亦可用。

又方,龙胆四两,酒三升,煮取一升半,顿服。

又方,吴茱萸五合,桂一两,酒二升半,煎取一升,分二服,效。

又方,吴茱萸二升,生姜四两,豉一升,酒六升,煮取二升半,分为三服。

又方,白鸡一头,治之如食法,水三升,煮取二升,去鸡,煎汁,取六合,纳苦酒六合,入珍珠一钱,复煎取六合,纳末麝香如大豆二枚,顿服之。

又方,桂心、当归各一两,栀子十四枚,捣为散,酒服方寸匕,日三五服。亦治久心病发作有时节者也。

又方,桂心二两,乌头一两,捣,筛,蜜和为丸,一服如梧子大三丸,渐加之。

暴得心腹痛如刺方,苦参、龙胆各二两,升麻、栀子各三两,苦酒五升,煮取二升,分二服,当大吐,乃瘥。

治心疝发作有时,激痛难忍方。真射罔、吴茱萸分等,捣末,蜜和丸,如麻子,服二丸,日三服。勿吃热食。

又方,灸心鸠尾下一寸,名巨阙,及左右一寸,并百壮。又与物度颈及度脊如之,令正相对也,凡灸六处。

治久患常痛,不能饮食,头中疼重方。乌头六分,椒六分,干姜四分,捣末,蜜丸,酒饮服,如大豆四丸,稍加之。

又方,半夏五分,细辛五分,干姜二分,人参三分,附子一分,捣末,苦酒和丸如梧子大,酒服五丸,日三服。

治心下牵急懊痛方。桂心三两,生姜三两,枳实五枚,水五升,煮取三升,分三服。亦可加术二两、胶饴半斤。

治心肺伤动冷痛方。桂心二两,猪肾二枚,水八升,煮取三升,分三服。

又方,附子二两,干姜一两,蜜丸,服四丸如梧子大,日三服。

治心痹心痛方。蜀椒一两(熬令黄),末之,以狗心血丸之如梧子,服五丸,日五服。

治心下坚痛,大如碗边,如旋柈,名为气分饮水所结方。枳实七枚(炙),术三两,水一斗,煮取三升,分为三服,当稍软也。

若心下百结积,来去痛者方。吴茱萸末一升,真射罔如弹丸一枚,合捣,以鸡子白和丸,丸如小豆大,服二丸,即瘥。

治心痛多唾,似有虫方。取六畜心,生切作十四脔,刀纵横各割之,以真丹一两,粉肉割中,旦悉吞之。入雄黄、麝香,佳。

饥而心痛者,名曰饥疝。龙胆、附子、黄连分等,捣筛,服一钱匕,日三度服之。

附方

《药性论》主心痛、中恶或连腰脐者。盐如鸡子大。青布裹烧赤,纳酒中,顿服,当吐恶物。

《拾遗》序,延胡索止心痛,末之,酒服。

《圣惠方》治久心痛,时发不定,多吐清水,不下饮食。以雄黄二两,好醋二升,慢火煎成膏,用干蒸饼,丸如梧桐子大,每服七丸,姜汤下。

又方,治九种心痛妨闷。用桂心一分为末,以酒一大盏,煎至半盏,去滓,稍热服,立效。

又方,治寒疝心痛,四肢逆冷,全不饮食。用桂心二两为散。不计时候,热酒调下一钱匕。

《外台秘要》治卒心痛。干姜为末。水饮调下一钱。

又方,治心痛。当归为末,酒服方寸匕。

又《必效》治 心痛。熊胆如大豆,和水服,大效。

又方,取鳗鲡鱼,淡炙令熟,与患人食一二枚,永瘥。饱食弥佳。

《经验方》治四十年心痛不瘥。黍米,淘汁,温服,随多少。

《经验后方》治心痛。姜黄一两,桂穰三两,为末,醋汤下一钱匕。

《简要济众》治九种心痛,及腹胁积聚滞气。筒子干漆二两,捣碎,炒烟出,细研,醋煮,面糊和丸,如梧桐子大,每服五丸至七丸,热酒下,醋汤亦得,无时服。

姚和众治卒心痛。郁李仁三七枚,烂嚼,以新汲水下之,饮温汤尤妙,须臾痛止,却煎薄盐汤热呷之。

《兵部手集》治心痛不可忍,十年五年者,随手效。以小蒜,酽醋煮,顿服之,取饱。不用着盐。

治卒腹痛方第九

治卒腹痛方。书舌上作风字,又画纸上作两蜈蚣相交,吞之。

又方,捣桂末,服三寸匕。苦酒、人参、上好干姜亦佳。

又方,粳米二升,以水六升,煮二七沸,饮之。

又方,食盐一大把,多饮水送之,忽当吐,即瘥。

又方,掘土作小坎,水满饮中,熟搅,取汁饮之。

又方,令人骑其腹,溺脐中。

又方,米粉一升,水二升,和饮。

又方,使病患伏卧,一人跨上,两手抄举其腹,令病患自纵重轻举抄之,令去床三尺许,便放之,如此二七度止。拈取其脊骨皮深取痛引之,从龟尾至顶乃止。未愈,更为之。

又方,令卧,枕高一尺许,拄膝使腹皮蹙气入胸,令人抓其脐上三寸便愈。能干咽吞气数十遍者弥佳。此方亦治心痛,此即伏气。

治卒得诸疝,小腹及阴中相引,痛如绞,自汗出,欲死方。捣沙参末,筛,服方寸匕,立瘥。

此本在杂治中,谓之寒疝,亦名阴疝。此治不瘥,可服诸利丸下之,作走马汤亦佳。

治寒疝腹痛,饮食下唯不觉其流行方。椒二合,干姜四两,水四升,煮取二升,去滓,纳饴一斤,又煎取半分,再服,数数服之。

又方,半夏一升,桂八两,生姜一升,水六升,煮取二升,分为三服。

治寒疝,来去每发绞痛方。吴茱萸三两,生姜四两,豉二合,酒四升,煮取二升,分为二服。

又方,附子一枚,椒二百粒,干姜半两,半夏十枚,大枣三十枚,粳米一升,水七升,煮米熟,去滓,一服一升,令尽。

又方,肉桂一斤,吴茱萸半升,水五升,煮取一升半,分再服。

又方,牡蛎、甘草、桂各二两,水五升,煮取一升半,再服。

又方,宿乌鸡一头(治如食法),生地黄七斤,合细锉之,着甑蔽中蒸,铜器承,须取汁,清旦服,至日晡令尽。其间当下诸寒癖,讫,作白粥渐食之。久疝者,下三剂。

附方

《博济方》治冷热气不和,不思饮食,或腹痛 刺。山栀子、川乌头等分,生捣为末,以酒糊丸,如梧桐子大,每服十五丸,炒生姜汤下。如小肠气痛,炒茴香、葱酒任下二十丸。

《经验方》治元藏气发,久冷腹痛虚泻。

应急大效玉粉丹。生硫黄五两,青盐一两,以上衮细研,以蒸饼为丸,如绿豆大,每服五丸,热酒空心服,以食压之。

《子母秘录》治小腹疼,青黑,或亦不能喘。苦参一两。醋一升半,煎八合,分二服。

《圣惠方》治寒疝,小腹及阴中相引痛,自汗出。以丹参一两,杵为散,每服热酒调下二钱匕,佳。

治心腹俱痛方第十

治心腹俱胀痛,短气欲死或已绝方。取栀子十四枚,豉七合,以水二升,先煮豉取一升二合,绞去滓,纳栀子,更煎取八合,又绞去滓,服半升。不愈者尽服之。

又方,浣小衣,饮其汁一二升,即愈。

又方,桂二两(切),以水一升二合,煮取八合,去滓,顿服。无桂者,着干姜亦佳。

又方,乌梅二七枚,以水五升煮一沸,纳大钱二七枚,煮得二升半。强人可顿服,羸人可分为再服,当下便愈。

又方,茱萸二两,生姜四两,豉三合,酒四升,煮取二升,分为三服,即瘥。

又方,干姜一两,巴豆二两,捣,蜜丸,一服如小豆二丸,当吐下,瘥。

治心腹相连常胀痛方。野狼毒二两,附子半两,捣,筛,蜜丸如梧子大,日一服一丸,二日二丸,三日后服三丸,再一丸,至六日,服三丸,自一至三,以常服,即瘥。

又方,吴茱萸一合,干姜四分,附子、细辛、人参各二分,捣筛,蜜丸如梧子大,服五丸,日三服。

凡心腹痛,若非中恶,霍乱,则是皆宿结冷热所为。今此方可采以救急,瘥后,要作诸大治,以消其根源也。

附方

《梅师方》治心腹胀坚,痛闷不安,虽未吐下欲死。以盐五合,水一升,煎令消,顿服,自吐下,食出即定。不吐更服。

《孙真人方》治心腹俱痛。以布裹椒,薄注上火熨,令椒汗出,良。

《十全方》,心脾痛。以高良姜(细锉,炒)杵末,米饮调下一钱匕立止。

治卒心腹烦满方第十一

治卒心腹烦满,又胸胁痛欲死方。以热汤令灼灼尔,渍手足,复易,秘方。

又方,青布方寸,鹿角三分,乱发灰二钱匕,以水二升,煮令得一升五合,去滓,尽服之。

又方,锉薏苡根,浓煮取汁,服三升。

又方,取比轮钱二十枚,水五升,煮取三沸,日三服。

又方,捣香菜汁,服一二升。水煮干姜亦佳。

又方,即用前心痛栀子豉汤法,瘥。

又方,黄芩一两,杏仁二十枚,牡蛎一两,水三升,煮取一升,顿服。

治厥逆烦满,常欲呕方。小草、桂、细辛、干姜、椒各二两,附子二两(炮),捣,蜜和丸,服如桐子大四丸。

治卒吐逆方。灸乳下一寸,七壮,即愈。

又方,灸两手大拇指内边爪后第一文头各一壮,又灸两手中央长指爪下一壮,愈。

此本杂治中,其病亦是痰壅霍乱之例,兼宜依霍乱条法治之。人卒在此上条患者亦少,皆因他病兼之耳。或从伤寒未复,或从霍乱吐下后虚燥,或者劳损服诸补药痞满,或触寒热邪气,或食饮协毒,或服药失度,并宜各循其本源为治,不得专用此法也。

附方

《千金方》治心腹胀,短气。以草豆蔻一两,去皮为末,以木瓜生姜汤下半钱匕。

《斗门方》治男子女人久患气胀心闷,饮食不得,因饮食不调,冷热相击,致令心腹胀满方。厚朴火上炙令干,又蘸姜汁炙,直待焦黑为度。捣筛,如面,以陈米饮调下二钱匕,日三服,良。亦治反胃,止泻甚妙。

《经验方》治食气遍身黄肿,气喘,食不得,心胸满闷。

不蛀皂角(去皮子,涂好醋,炙令焦,为末)一钱匕,巴豆七枚(去油膜),二件以淡醋及研好墨为丸,如麻子大,每服三丸,食后陈橘皮汤下,日三服,隔一日增一丸,以利为度。如常服,消酒食。

《梅师方》治腹满不能服药。煨生姜,绵裹,纳下部中,冷即易之。

《圣惠方》治肺脏壅热烦闷。新百合四两,蜜半盏,和蒸令软,时时含一枣大,咽津。


卷 二


\backmatter

\end{document}