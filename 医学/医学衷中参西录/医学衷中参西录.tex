% 医学衷中参西录
% 医学衷中参西录.tex

\documentclass[a4paper,12pt,UTF8,twoside]{ctexbook}

% 设置纸张信息。
\RequirePackage[a4paper]{geometry}
\geometry{
	%textwidth=138mm,
	%textheight=215mm,
	%left=27mm,
	%right=27mm,
	%top=25.4mm, 
	%bottom=25.4mm,
	%headheight=2.17cm,
	%headsep=4mm,
	%footskip=12mm,
	%heightrounded,
	inner=1in,
	outer=1.25in
}

% 设置字体,并解决显示难检字问题。
\xeCJKsetup{AutoFallBack=true}
\setCJKmainfont{SimSun}[BoldFont=SimHei, ItalicFont=KaiTi, FallBack=SimSun-ExtB]

% 目录 chapter 级别加点(.)。
\usepackage{titletoc}
\titlecontents{chapter}[0pt]{\vspace{3mm}\bf\addvspace{2pt}\filright}{\contentspush{\thecontentslabel\hspace{0.8em}}}{}{\titlerule*[8pt]{.}\contentspage}

% 设置 part 和 chapter 标题格式。
\ctexset{
	part/name= {第,卷},
	part/number={\chinese{part}},
	chapter/name={第,篇},
	chapter/number={\chinese{chapter}}
}

% 设置古文原文格式。
\newenvironment{yuanwen}{\bfseries\zihao{4}}

% 设置署名格式。
\newenvironment{shuming}{\hfill\bfseries\zihao{4}}

\title{\heiti\zihao{0} 医学衷中参西录}
\author{张锡纯}
\date{清·公元1909年 }

\begin{document}

\maketitle
\tableofcontents

\frontmatter
\chapter{前言、序言}






\mainmatter

\part{医方}

\chapter{治阴虚劳热方}

\section{资生汤}
属性:治劳瘵羸弱已甚,饮食减少,喘促咳嗽,身热脉虚数者。亦治女子血枯不月。

生山药(一两) 玄参(五钱) 于术(三钱) 生鸡内金(二钱,捣碎) 牛蒡子(三钱,炒,捣) 

热甚者,加生地黄五六钱。脾为后天之本,能资生一身。脾胃健壮,多能消化饮食,则全身自然健壮,何曾见有多饮多食,而病劳瘵者哉?《内经》阴阳别论曰∶“二阳之病发心脾,有不得隐曲,在女子为不月,其传为风
 
以其先不过阳明,胃腑不能多纳饮食也,而原其饮食减少之故。曰发于心脾,原其发于心脾之 
故。曰有不得隐曲者何居?盖心为神明之府,有时心有隐曲,思想不得自遂,则心神拂郁, 
心血亦遂不能濡润脾土,以成过思伤脾之病。脾伤不能助胃消食,变化津液,以溉五脏,在男 
子已隐受其病,而尚无显征;在女子则显然有不月之病。此乃即女以征男也。至于传为风消 
,传为息贲,无论男女病证至此,人人共见,劳瘵已成,挽回实难,故曰不治。然医者以活 
人为心,病证之危险,虽至极点,犹当于无可挽回之中,尽心设法以挽回之。而其挽回之法 
,仍当遵二阳之病发心脾之旨。戒病者淡泊寡欲,以养其心,而复善于补助其脾胃,使饮食 
渐渐加多,其身体自渐渐撤消。如此汤用于术以健脾之阳,脾土健壮,自能助胃。山药以滋 
胃之阴,胃汁充足,自能纳食(胃化食赖有酸汁)。特是脾为统血之脏,《内经》谓“血生 
脾”,盖谓脾系血液结成,故中多函血。西人亦谓脾中多回血管 
为血汇萃之所。此证因心思拂郁,心血不能调畅,脾中血管遂多闭塞 
,或如烂炙,或成丝膜,此脾病之由。而脾与胃相助为理,一气贯通,脏病不能助腑,亦即胃 
不能纳食之由也。鸡内金为鸡之脾胃,中有瓷、石、铜、铁,皆能消化,其善化有形郁积可 
知。且其性甚和平,兼有以脾胃补脾胃之妙。故能助健补脾胃之药,特立奇功,迥非他药所 
能及也。方中以此三味为不可挪移之品。 
玄参《神农本草经》谓其微寒,善治女子产乳余疾,且其 
味甘胜于苦,不至寒凉伤脾胃可知,故用之以去上焦之浮热,即以退周身之烧热;且其色黑 
多液,《神农本草经》又谓能补肾气,故以治劳瘵之阴虚者尤宜也。牛蒡子体滑气香,能润肺又能 
利肺,与山药、玄参并用,大能止嗽定喘,以成安肺之功,故加之以为佐使也。 
地黄生用,其凉血退热之功,诚优于玄参。西人谓其中函铁质,人之血中,又实有铁锈。地 
黄之善退热者,不但以其能凉血滋阴,实有以铁补铁之妙,使血液充足,而蒸热自退也。又 
劳瘵之热,大抵因真阴亏损,相火不能潜藏。 
地黄善引相火下行,安其故宅。《神农本草经》列之上品,洵良药也。然必烧热过甚而始 
加之者,以此方原以健补脾胃为主,地黄虽系生用,经水火煎熬,其汁浆仍然粘泥,恐于脾 
胃有不宜也。至热甚者,其脾胃必不思饮食,用地黄退其热,则饮食可进,而转有辅助脾胃 
生山药,即坊间所鬻之干山药,而未经火炒者也。此方若用炒熟山药,则分毫无效 
于术色黄气香,乃浙江于潜所产之白术也。色黄则属土,气香则醒脾,其健补脾胃之功,迥 
异于寻常白术。若非于潜产而但观 
其色黄气香,用之亦有殊效,此以色、味为重,不以地道为重也。 
西人谓∶胃之所以能化食者,全赖中有酸汁。腹饥思食时,酸汁自然从胃生出。若忧思过度 
,或恼怒过度,则酸汁之生必少,或分毫全无,胃中积食,即不能消化。此论与《内经》“二 
阳之病发心脾”、“过思则伤脾”之旨暗合。 
或问曰∶《内经》谓脾主思,西人又谓思想发于脑部,子则谓思发于心者何也?答曰∶《内 
经》所谓脾主思者,非谓脾自能思也。盖脾属土,土主安静,人安静而后能深思,至西人谓思发于脑部,《内经》早寓其 
理。脉要精微论曰∶“头 
者精明之府。”夫头之中心点在脑,头为精明之府,即脑为精明之府矣。既曰精明,岂有不 
能思之理,然亦非脑之自能思也。试观古文“思”字作“ ”,囟者脑也,心者心也,是知 
思也者,原心脑相辅而成,又须助以脾土镇静之力也。 
或问曰∶子解二阳之病发心脾一节,与王氏《内经》之注不同,岂王氏之注解谬欤?答曰∶ 
愚实不敢云然。然由拙解以绎经文,自觉经文别有意味,且有实用也。夫二阳之病发心脾, 
与下三阳为病发寒热,一阳发病、少气、善咳、善泄,句法不同,即讲法可以变通。盖二阳之 
病发心脾,谓其病自心脾而来也。三阳为病发寒热,是形容三阳之病状也,故将之病“之” 
字易作“为’字。至一阳发病数句,其句法又与三阳为病句不同,而其理则同也。 
或又问∶三阳一阳病,皆形容其发病之状,二阳病,独推究其发病之原因者何居?答曰∶三 
阳、一阳,若不先言其病发之状,人即不知何者为三阳、一阳病。至二阳胃腑,原主饮食, 
人人皆知。至胃腑有病,即不能饮食,此又人人皆知。然其所以不能饮食之故,人多不能知 
也。故发端不言其病状,而先发明其得病之由来也。 

或又问∶胃与大肠皆为二阳,经文既浑曰二阳,何以知其所指者专在于胃、答曰∶胃为足阳 
明,大肠为手阳明,人之足经长、手经短,足经原可以统手经,论六经者原当以足经为主。 
故凡《内经》但曰某经,而不别其为手与足者,皆指足经而言,或言足经而手经亦统其中。 
若但言手经,则必别之曰手某经矣。经文俱在,可取而细阅也。 
一九一三年,客居大名。治一室女,劳瘵年余,月信不见,羸弱不起。询方于愚,为拟此汤。 
连服数剂,饮食增多。身犹发热,加生地黄五钱,五六剂后热退,渐能起床,而腿疼不能行 
动。又加丹参、当归各三钱,服至十剂腿愈,月信亦见。又言有白带甚剧,向忘言及。遂去 
丹参加生牡蛎六钱,又将于术加倍,连服十剂带证亦愈。遂将此方邮寄家中,月余门人高 
促异常,饮食减少,脉甚虚数,投以资生汤十剂全愈。”审斯则知此方治劳瘵,无论男女, 
服之皆有捷效也。 

女子月信,若日久不见,其血海必有坚结之血。治此等证者,但知用破血通血之药,往往病 
犹未去,而人已先受其伤。鸡内金性甚和平,而善消有形郁积,服之既久,瘀血之坚结者, 
自然融化。矧此方与健脾滋阴之药同用,新血活泼滋长,生新自能化瘀也。 

附录∶ 

直隶青县张××来函∶ 

津埠宋氏妇,年将四旬,身体羸弱,前二年即咳嗽吐痰,因不以为事未尝调治。今春证浸加剧,屡次服药无效。诊 
其脉, 
左部弦细,右部微弱,数近六至。咳嗽,吐痰白色,气腥臭,喘促自汗,午后发热,夜间尤甚,胸膈满闷,饮食减少, 
大便秘结,知其已成痨瘵而兼肺病也。从前所服药十余纸,但以止嗽药治其肺病,而不知子虚补母之义,所以无效。为 
疏方用资生汤加减,生山药八钱,玄参、大生地、净萸肉各六钱,生牡蛎、生杭芍、生赭石各四钱,于术、生鸡内金、 
甘草各二钱。煎服二剂,汗止喘轻,发热咳嗽稍愈。遂将前方去牡蛎,加蒌仁、地骨皮各三钱,山药改用一两,赭石改 
用六钱。连服十剂,诸病皆愈,为善后计,用生山药细末八钱煮粥,调白糖服之,早晚各一次。后月余。与介绍人晤面, 
言此时宋氏妇饮食甚多,身体较前健壮多矣。 
又∶族嫂年三十五岁,初患风寒咳嗽,因懒于服药,不以为事。后渐至病重,始延医延医。所服之药,皆温散燥烈之品, 
不知风寒久而化热,故越治越剧,几至不起。后生于腊底回里,族兄邀为诊视。脉象虚而无力,身瘦如柴,咳嗽微喘, 
饮食减少,大便泄泻,或兼白带,午后身热, 
颧红,确系痨瘵已成。授以资生汤,加炒薏仁、茯苓片、生龙骨、生牡蛎各三钱,茵陈、炙甘草各钱半。服二剂,身热、 
颧红皆退,咳嗽泄泻亦见愈。后仍按此方加减,又服六剂,诸病皆痊。嘱其每日用生怀山药细末煮粥,调以白糖服之, 
以善其后。 

四川泾南周××来函∶ 

杨姓女,年十九岁。出嫁二载,月事犹未见,身体羸瘦,饮食减少,干咳无痰,五心烦热,诊其脉细数有力。仿用资生 
汤方,用生山药一两,于术二钱,牛蒡子三钱,玄参五钱,生地黄四钱,生鸡内金一钱。连服五剂,热退咳减,食欲增 
加。遂于原方中去生地,倍于术。又服三剂, 潮忽至。共服二十剂全愈。 

奉天法库县万××来函∶ 

族弟妇产后虚羸少食,迁延月余,渐至发灼、自汗、消瘦、乏气、干呕、头晕等证,此方书所谓蓐劳也。经医四人治不 
效,并添颧红作泻。适生自安东归,为之诊视,六脉虚数。检阅所服之方,有遵《医宗金鉴》三合饮者,有守用养荣汤 
者,要皆平淡无奇。然病势至此,诚难入手,幸脉虽虚数,未至无神,颧虽红,犹不抟聚(若抟聚则阴阳离矣,不抟聚 
是其阴阳犹未离),似尚可治。此盖素即阴虚,又经产后亡血,气亦随之,阴不中守,阳不外固,故汗出气乏;其阴阳不 
相维系,阴愈亏而阳愈浮,故发烧咳嗽头晕;其颧红者,因其部位应肾,肾中真阳上浮,故发现于此,而红且热也;其 
消瘦作泻者,以二阳不纳,无以充肌肉,更不特肾阴虚,而脾阴胃液均虚,中权失司,下陷不固,所必然者。此是病之 
原委欤?再四思维,遂处方,用生怀山药二两,于术三钱,玄参四钱,鸡内金、牛蒡子各二钱(此系资生汤原方稍加重), 
外加净萸肉、龙骨、牡蛎各五钱,止汗并以止泻。五剂后,汗与泻均止,饮食稍进,惟干咳与发热仅去十之二三。又照 
原方加粉甘草、天冬、生地等味,连服七剂。再照方减萸肉,加党参二钱,服四剂后,饮食大进,并能起坐矣,惟经尚 
未行。更按资生汤原方,加当归四钱。服数剂后,又复少有加减,一月经脉亦通。 

\section{十全育真汤}
属性:治虚劳,脉弦、数、细、微,肌肤甲错,形体羸瘦,饮食不壮筋力,或自汗,或咳逆,或喘促,或寒热不时,或多梦纷纭,精气不固。

野台参(四钱) 生黄 (四钱) 生山药(四钱) 知母(四钱) 玄参(四钱) 生龙骨(四钱,捣细) 生 
牡蛎(四钱,捣细) 丹参(二钱) 三棱(钱半) 莪术(钱半) 
气分虚甚者,去三棱、莪术,加生鸡内金三钱;喘者,倍山药,加牛蒡子三钱;汗多者,以 
白术 
骨、牡蛎、萸肉各一两煎服,不过两剂其汗即止。汗止后再服原方。若先冷后热而汗出者 
,其脉或更兼微弱不起,多系胸中大气下陷,细阅拙拟升陷汤后跋语,自知治法。 
仲景治劳瘵,有大黄 虫丸,有百劳丸,皆多用破血之药。诚以人身经络,皆有血融贯其 
间,内通脏腑,外溉周身,血一停滞,气化即不能健运,劳瘵 
恒因之而成。是故劳瘵者肌肤甲错,血不华色,即日食珍馐服参苓,而分毫不能长肌肉、壮 
筋力。或转消瘦支离,日甚一日,诚以血瘀经络阻塞其气化也。玉田王清任着《医林改错》 
一书,立活血逐瘀诸汤,按上中下部位,分消瘀血,统治百病,谓瘀血去而诸病自愈。其立言不 
无偏处,然其大旨则确有主见,是以用其方者,亦多效验。今愚因治劳瘵,故拟十全育真汤 
,于补药剂中,加三棱,莪术以通活气血,窃师仲景之大黄 虫丸、百劳丸之意也。且仲景 
于《金匮》列虚劳一门,特以血痹虚劳四字标为提纲。益知虚劳者必血痹,而血痹之甚,又 
未有不虚劳者。并知治虚劳必先治血痹,治血痹亦即所以治虚劳也。 
或问∶治劳瘵兼用破血之药,诚为确当之论,但破血用三棱、莪术,将毋其力过猛乎?答曰 
∶仲景之大黄 虫丸,与百劳丸所用破血之药,若大黄、干漆、水蛭,皆猛于 
三棱、莪术,而方中不用三棱、莪术者、诚以三棱、莪术《神农本草经》不载。至梁陶 
弘景着《名医别录》于《神农本草经》外增药品 
三百六十五味,皆南北朝以前,名医所用之药,亦未载三棱、莪术。是当仲景时犹无三棱、 
莪术,即有之,亦未经试验可知。而愚于破血药中,独喜用三棱、莪术者,诚以其既善破血 
,尤善调气。补药剂中以为佐使, 
将 
资生纳谷为宝。无论何病,凡服药后饮食渐增者易治,饮食渐减者难治。三棱、莪术与参 
、术、 诸药并用,大能开胃进食,又愚所屡试屡效者也。 
或问∶劳瘵之证,阴虚发热者居其强半。故钱仲阳之减味地黄丸;张景岳 
之左归饮,皆为对证良方,以其皆以熟地黄为君,大能滋真阴退虚热也。子方中何以独不用 
也?答曰∶若论用熟地,我固过来人也。忆初读方书时,曾阅赵氏《医贯》、《张氏八阵》 
、《冯氏锦囊》诸书,遂确信其说。临证最喜用熟地,曾以八味地黄丸作汤,加苏子、白芍, 
治吸不归根之喘逆;加陈皮、白芍,治下虚上盛之痰涎;加苏子、浓朴,治肾不摄气,以致 
冲气上逆之胀满(时病患服之觉有推荡之力,后制参赭镇气汤治此证更效,又尝减茯苓、泽泻三分之二,治女子消渴小 
便频数(《金匮》谓 
治男子消渴以治女子亦效,案详玉液汤下),又尝去附子,加知母、白芍 
,治阴虚不能化阳,致小便不利积成水肿;又尝用六味地黄丸作汤,加川芎、知母,以治如破 
之头疼;加胆草、青黛,以治非常之眩晕;加五味、枸杞、柏子仁,以敛散大之瞳子,且信 
其煎汁数碗,浩荡饮之之说;用熟地四两、茯苓一两,以止下焦不固之滑泻;用熟地四两、 
白芍一两,以通阴虚不利之小便;又尝于一日之中用熟地斤许,治外感大病之后,忽然喘逆 
,脉散乱欲脱之险证(此证当用后来复汤,彼时其方未拟出,惟知用熟地亦幸成功 
,是知冯楚瞻谓熟地能大补肾中元气诚有所试也),且不独治内伤也;又尝用熟地 
、阿胶大滋真阴之类,治温病脉阳浮而阴不应,不能作汗,一日连服二剂,济阴以应其阳, 
使之自汗(详案在寒解汤下);并一切伤寒 外感,因下元虚惫而 
邪深陷者,莫不重用熟地,补其下元,即以托邪外出。惟用以治阴虚劳热之证,轻者可效, 
若脉数至七八至鲜有效者。彼时犹不知改图,且以为地黄丸,即《金匮》之肾气丸,自古推 
为良方,此而不效,则他方更无论矣,不知肾气丸原用干地黄,即药坊间之生地也,其桂用 
桂枝,即《神农本草经》之牡桂也,与今之地黄丸迥不侔矣。其方《金 
匮》凡五见,一治“香港脚上入少腹不仁”;一治“虚劳腰痛,少腹拘急,小便不利”;一治 
“短气有微饮,当从小便去之”;一治“男子消渴,小便反多,饮一斗,小便一斗”;一治 
“妇人转胞,胞系了戾,不得溺”。统观五条,原治少腹膀胱之疾居多,非正治劳瘵之药, 
况后世之修制,又失其本然乎。 
后治一妇人,年近五旬。身热劳嗽,脉数几至八至。先用六 
味地黄丸加减作汤服不效,继用左归饮加减亦不效。愚忽有会悟,改用生黄 六钱、知母八 
钱为方,数剂见轻,又加丹参、当归各三钱,连服十剂全愈。以后凡遇阴虚有热之证,其稍 
有根柢可挽回者,于方中重用黄 、知母,莫不随手奏效。始知叔和脉法谓数至七八至为不 
治之脉者,非确论也。盖人禀天地之气以生,人身之气化即天地之气化,天地将雨之时,必 
阳气温暖上升,而后阴云会合大雨随之。黄 温升补气,乃将雨时上升之阳气也;知母寒润 
滋阴,乃将雨时四合之阴云也。二药并用,大具阳升阴应云行雨施之妙。膏泽优渥烦热自退 
,此不治之治也。况劳瘵者多损肾 
,黄 能大补肺气,以益肾水之源,使气旺自能生水,而知母又大能滋肺中津液,俾阴阳不 
至偏胜,即肺脏调和,而生水之功益普也(黄 、知母虽可并用以退虚热,然遇 
阴虚热甚者,又必须加生地黄八钱或至一两,方能服之有效)。 
或又问∶肾气丸虽非专治虚劳之药,而《金匮》虚劳门,明载其治虚劳腰疼,似虚者皆可服 
之,子独谓无甚效验,岂古方不可遵欤?答曰∶肾气丸若果按古方修制,地黄用干地黄,桂 
用桂枝,且止为丸剂,而不作汤剂,用之得当,诚有效验。盖生地能逐血痹(《神 
农本草经》),而熟地无斯效也。桂枝能调营卫,而肉桂无斯效也。血痹逐,则瘀血 
自消,营卫调,则气血自理。至于山萸肉之酸温,亦能逐痹(《神龙本草经》山茱萸逐寒 
湿痹)。牡丹皮之辛凉,亦能破血。附子之大辛大温,又能温通血脉,与地黄之寒 
凉相济,以共成逐血痹之功。是肾气丸为补肾之药,实兼为开瘀血之药,故列于《金匮》虚 
劳门,而为要方也。其止为丸剂,而不作汤剂者,诚以地黄经水火煎熬,则汁浆稠粘、性近熟 
地,其逐血痹之力必减,是以《神农本草经》,谓地黄生者尤良 
也。后贤徐灵胎曾治一人,上盛下虚,胸次痰火壅滞,喘不能卧,将人参切作小块,用清水 
理痰之药煎汤,送服而愈。后其病复发,病家自用原方,并人参亦煎服,病益甚,灵胎仍教以 
根据从前 
,可以悟古人制肾气丸之精义矣。 
或又问∶肾气丸既按古方修制,可以有效,而《金匮》虚劳门,肾气丸与大黄 虫丸之外,又有 
七方,皆可随证采择,则子之十全育真汤,似亦可以不拟欤?答曰∶《金匮》虚劳门诸方, 
虽皆有效,而一方专治虚劳门一证。若拙拟十全育真汤,实兼治虚劳门诸证。如方中用黄 
以补气,而即用人参以培元气之根本。用知母以滋阴,而即用山药、元参以壮真阴之渊源。 
用三棱、莪术以消瘀血,而即用丹参以化瘀血之渣滓。至龙骨、牡蛎,若取其收涩之性,能 
助黄 以固元气;若取其凉润之性,能助知母以滋真阴;若取其开通之性(《神农本草经 
》龙骨主 瘕,后世本草亦谓牡蛎消血),又能助三棱、莪术以 
消融瘀滞也。至于疗肺虚之咳逆、肾虚之喘促,山药最良。治多梦之纷纭,虚汗之淋漓,龙骨、牡蛎尤胜。此 
方中意也,以寻常药饵十味,汇集成方,而能补助人身之真阴阳、真气血、真精神,故曰十全育真也。 
劳瘵者多兼瘀血,其证原有两种∶有因劳瘵而瘀血者,其人或调养失宜,或纵欲过度,气血 
亏损,流通于周身者必然迟缓,血即因之而瘀,其瘀多在经络;有因瘀血而成劳瘵者,其人 
或有跌伤碰伤,或力小任重,或素有吐衄证,服药失宜,以致先有瘀血,日久浸成劳 
瘵,其瘀血多在脏腑。此二者服十全育真汤皆可愈。而瘀血在脏腑者,尤须多用破血之药。又 
瘀在经络者,亦可用前方资生汤,加当归、丹参。瘀在脏腑之剧者,又宜用拙拟理冲汤,或理 
冲丸。此数方可参变汇通,随时制宜也。 
世俗医者,遇脉数之证,大抵责之阴虚血涸。不知元气虚极莫支者,其脉可至极数。设有人 
或力作,或奔驰,至气力不能支持之时,其脉必数。乃以力倦之不能支持,以仿气虚之不能 
支持,其事不同而其理同也。愚临证细心体验,凡治虚劳之证,固不敢纯用补药,然理气药 
多于补气药,则脉即加数,补气药多于理气药,则脉即渐缓。是知脉之数与不数,固视乎血 
分之盈亏,实尤兼视乎气分之强弱。故此十全育真汤中,台参、黄 各四钱,而三棱、莪术 
各钱半,补气之药原数倍于理气之药。若遇气分虚甚者,犹必以鸡内金易三棱、莪术也。 
药性之补、破、寒、热,虽有一定,亦视乎服药者之资禀为转移。尝权衡黄 之补力,与三 
棱、莪术之破力,等分用之原无轩轾。尝用三棱、莪术各三钱,治脏腑间一切 
瘕积聚,恐其伤气,而以黄 六钱佐之,服至数十剂,病去而气分不伤,且有愈服而愈觉强壮者。若遇气 
分甚虚者,才服数剂,即觉气难支持,必须加黄 ,或减三棱、莪术,方可久服。盖虚极之 
人,补药难为攻,而破药易见过也。若其人气壮而更兼郁者,又必须多用三棱、莪术,或少 
用黄 ,而后服之不至满闷。又尝权衡黄 之热力,与知母之寒力,亦无轩轾,等分用之可久 
服无寒热也(此论汤剂作丸剂则知母寒力胜于黄 热力)。而素畏热者 
,服之必至增热,素畏寒者,服之又转增寒,其寒热之力无定,亦犹补破之力无定也。故临 
证调方者,务须细心斟酌,随时体验,息息与病机相符,而后百用不至一失也。 
附录∶ 
直隶沧县李××来函∶ 
弟长男媳,年二十四岁,于本年(丙寅)正月间患寒热往来,自原素畏服药,故隐忍不肯言。至四月初,家人来迓 
弟, 
言儿媳病剧。回家视之,虽未卧床不起。而瘦弱实难堪矣。诊其脉,弦而浮数。细询病情,言每逢午后先寒后热,时而 
微咳无痰,日夜作泻十余次,黎明则头汗出,胸间绵绵作疼,食一下咽即胀满难堪,而诸虚百损之状,显然尽露。筹思 
良久,为立逍遥散方。服两剂无效。因复至沧取药,适逢张××自津来沧,遂将儿媳之病细述本末。张××曰∶“以弟之 
意,将用何方以治之?”答曰∶“余拟将资生汤、十全育真汤二方,汇通用之,可乎?”张××曰∶“得之矣。此良方也, 
服之必效。”弟遂师二方之义,用生怀山药八钱,生白术、净萸肉、生鸡内金、生龙骨、生牡蛎、鲜石斛各三钱,丹参四 
钱。连服四剂,诸证皆大轻减。又于原方加三棱、莪术(十全育真汤中用此二药者,因虚劳之证多血痹也。)各一钱,粉 
丹皮、地骨皮各二钱。又连服八剂,诸病悉退,饮食增加,今已完全成功矣。 


<篇名>3.醴泉饮
属性:治虚劳发热,或喘或嗽,脉数而弱。 
生山药(一两) 大生地(五钱) 人参(四钱) 玄参(四钱) 生赭石(四钱,轧细) 牛蒡子(三钱,炒,捣) 
天冬(四钱) 甘草(二钱) 
劳热之证,大抵责之阴虚。有肺阴虚者,其人因肺中虚热熏蒸,时时痒而作嗽, 
甚或肺中有所损伤,略一动作,辄发喘促,宜滋补肺阴,兼清火理痰之品,有肾阴虚者,其 
人因肾虚不能纳气,时时咳逆上气,甚或喘促,宜填补下焦真阴。兼用收降之品。若其脉甚 
数者,陈修园谓,宜滋养脾阴。盖以脾脉原主和缓,脉数 
者必是脾阴受伤,宜于滋阴药中,用甘草以引之归脾,更兼用味淡之药,如薏米、石斛之类 
。特是人身之阴,所盖甚广,凡周身之湿处皆是也。 
故阴虚之甚者,其周身血脉津液,皆就枯涸。必用汁浆最多之药,滋脏腑之阴,即以溉周身 
之液,若方中之山药、地黄是也。然脉之数者,固系阴虚,亦系气分虚弱,有不能支持之象 
,犹人之任重而体颤也。故用人参以补助气分,与玄参、天冬之凉润者并用,又能补助阴分。 
且虑其升补之性,与咳嗽上逆者不宜,故又佐以赭石之压力最胜者,可使人参补益之 
力下行直至涌泉,而上焦之逆气浮火,皆随之顺流而下;更可使下焦真元之气,得人参之峻 
补而顿旺,自能吸引上焦之逆气浮火下行也。至于牛蒡子与山药并用,最善止嗽,甘草与天冬 
并用,最善润肺,此又屡试屡效者也。 
初制此方时,原无赭石,有丹参三钱,以运化人参之补力。后治一年少妇人,信水数月不行 
,时作寒热,干嗽连连,且兼喘逆,胸隔满闷,不思饮食,脉数几至七至。治以有丹参原方 
不效,遂以赭石易丹参,一剂咳与喘皆愈强半,胸次开通,即能饮食,又服数剂脉亦和缓, 
共服二十剂,诸病皆愈。以后凡治妇女月闭血枯,浸至虚劳,或兼咳嗽满闷者,皆先 
投以此汤,俾其饮食加多,身体强壮,经水自通。间有瘀血暗阻经道,或显有 
瘕可据者,继服拙拟理冲汤,或理冲丸以消融之,则妇女无难治之病矣。若 
其人胸中素觉短气,或大便易滑泻者,又当预防其大气下陷(大气下陷详升 
陷汤)。用醴泉饮时,宜减赭石、牛蒡子,并一切苏子、蒌仁、紫菀、杏仁,治咳 
按∶短气与喘原迥异。短气者难于呼气不上达也。喘者难于吸气不下降也。而不善述病情者 
,往往谓喘为“上不来气”,是以愚生平临证,凡遇自言上不来气者,必细细询问,确知其 
果系呼气难与吸气难,而后敢为施治也。 
又按∶方书名咳喘曰“咳逆”,喘曰“喘逆”,因二证多由逆气上干也。而愚临证实验以来 
,知因大气下陷而咳喘者,亦复不少。盖肺悬胸中,必赖大气以包举之,而后有所附丽;大 
气以鼓舞之,而后安然呼吸。大气一陷,则包举之力微,肺即无所附丽,而咳嗽易生。鼓舞 
之机滞,肺必努力呼吸,而喘促易作。曾治一少年,泄泻半载方愈。后因劳力过度,觉喉中之气不舒,五六呼吸之间, 
必咳 
以拙拟升陷汤,数剂而愈。 
又∶曾治一人,年近五旬,素有喘疾。因努力任重,旧证复发。延 
医服药罔效。后愚诊视其脉,数近六至,而兼有沉濡之象。愚疑其阴虚不能纳气,因其脉兼 
沉濡,不敢用降气之药。遂用熟地、生山药、枸杞、玄参大滋真阴之药,大剂煎汤,送下人 
参小块二钱,连服三剂脉即不数,仍然沉濡,喘虽见轻,仍不能愈。因思此证得之努力任重 
,胸中大气因努力而陷,所以脉现沉濡,且其背恶寒而兼发紧,此亦大气下陷之征也。亦治 
以升陷汤,方中升麻、柴胡、桔梗皆不敢用,以桂枝尖三钱代之。因其素有不纳气之 
证,桂枝能升大气,又能纳气归肾也(理详参赭镇气汤下)。又外 
加滋阴之药,数剂全愈(详案在升陷汤下)。按此二证之病因,与 
醴泉饮所主之病迥异,而其咳喘则同。必详观升陷汤后跋语,及所载诸案,始明治此二证之 
理。而附载于此者,恐临证者审证不确,误以醴泉饮治之也。 
沈阳娄××,年二十二,虚劳咳嗽,其形羸弱,脉数八至,按之即无。细询之,自言 
曾眠热炕之上,晨起觉心中发热,从此食后即吐出,夜间咳嗽甚剧,不能安寝。因二十余日 
寝食俱废, 遂觉精神恍惚,不能支持。愚闻之,知脉象虽危,仍系新证,若久病至此,诚 
难挽回矣。遂投以醴泉饮,为其呕吐,将赭石改用一两(重用赭石之理详参 
赭镇气汤下),一剂吐即止,可以进食,嗽亦见愈。从前五六日未大便,至此大便 
亦通下。如此加减服之,三日后脉数亦见愈。然犹六至余,心中犹觉发热,遂将玄参、生地 
皆改用六钱,又每日于午时,用白蔗糖冲水,送服西药阿斯匹林(药性详参麦汤 
下)七厘许。数日诸病皆愈,脉亦复常。 
沈阳苏××,年三十许,劳嗽二年不愈,动则作喘,饮食减少。更医十余人,服药数百剂, 
分毫无效,羸弱转甚。来院延医。其脉数六至,虽细弱仍有根柢,知其可治。自言上焦恒觉发热,大便三四日一行 
,时或干燥。遂投以醴泉饮,为其便迟而燥,赭石改用六钱,又加鸡内金二钱(捣 
细),恐其病久脏腑经络多瘀滞也。数剂后饭量加增,心中仍有热时,大便已不燥 
,间日 一行。遂去赭石二钱,加知母二钱,俾于晚间服汤药后,用白蔗糖水,送服阿斯匹林 
四分之一瓦,得微汗。后令于日间服之,不使出 
汗,数日不觉发热,脉亦复常,惟咳嗽未能全愈。又用西药几阿苏六分,薄荷冰四分,和以绿豆 
粉为丸,梧桐子大,每服三丸,日两次,汤药仍照方服之,五六日后咳嗽亦愈,身体从此康 
健。 

<篇名>4.一味薯蓣饮
属性:治劳瘵发热,或喘或嗽,或自汗,或心中怔忡,或因小便不利,致大便滑泻,及一切阴分亏 
生怀山药(四两,切片) 煮汁两大碗,以之当茶,徐徐温饮之。 
山药之性,能滋阴又能利湿,能滑润又能收涩。是以能补肺补肾兼补脾胃。且其含蛋白质最 
多,在滋补药中诚为无上之品,特性甚和平,宜多服常服耳。 
陈修园谓∶山药为寻常服食之物,不能治大病,非也。若果不治大病,何以《金匮》治劳瘵有 
薯蓣丸。尝治一室女,温病痰喘,投以小青龙加石膏汤,又遵《伤寒论》加减法,去麻黄加 
杏仁,喘遂定。时已近暮,一夜安稳。至黎明喘大作,脉散乱如水上浮麻,不分至数,此将 
脱之候也。取药不及,适有生山药两许,急煮汁饮之,喘稍定,脉稍敛,可容取药,方中仍 
重用山药而愈(详案在仙露汤下)。 
一室女,月信年余未见,已成劳瘵,卧床不起。治以拙拟资生汤,复俾日用生山 
药四两,煮汁当茶饮之,一月之后,体渐复初,月信亦通。 
一妇人,产后十余日,大喘大汗,身热劳嗽。医者用黄 、熟地、白芍等药,汗出愈多。后 
愚诊视,脉甚虚弱,数至七至,审证论脉,似在不治。俾其急用生山药六两,煮汁徐徐 
饮之,饮完添水重煮,一昼夜所饮之水,皆取于山药中。翌日又换山药六两,仍如此煮饮之 
。三日后诸病皆愈。 
一人,年四十余,得温病十余日,外感之火已消十之八九。大便忽然滑下,喘息迫促,且有 
烦渴之意。其脉甚虚,两尺微按即无。亦急用生山药六两,煎汁两大碗,徐徐温饮下,以之 
当茶,饮完煎渣再饮,两日共享山药十八两,喘与烦渴皆愈,大便亦不滑泻。 
附录∶ 
直隶青县张××来函∶ 
侄女××,已于归数载,因患瘰 证成痨,喘嗽不休,或自汗,或心中怔忡,来函索方。余揣此系阴分亏损已极所 
致。 
俾先用一味薯蓣饮,每日用生怀山药四两,煮汁两大碗,当茶频频温饮之。不数剂,喘定汗止,咳嗽亦见轻。继又兼服 
薯蓣粥,作点心用之,渐渐全愈。 
奉天法库县万××来函∶ 
家慈患痰喘咳嗽病,三十年于兹矣。百方不效,且年愈高,病愈进。乃于今春宿病既发,又添发灼、咽干、头汗出、食 
不下等证。生虽习医,此时惟战兢不敢处方,遂请一宿医诊视,云是痰盛有火,孰知是肺气与脾阴肾阴将虚竭也。与人 
参清肺汤,加生地、丹皮等味,服二剂,非特未效,遂发灼如火,更添泄泻,有不可终日之势。于是不敢延医,自选用 
资生汤方,服一剂,亦无显效。转思此时方中于术、牛蒡子、鸡内金等味有未合也。因改用一味薯蓣饮,用生怀山药四 
两,加玄参三钱。服一剂见效,二剂大见效,三剂即病愈强半矣。后乃改用薯蓣粥,用生怀山药一两为细末,煮作粥, 
少调以白糖,每日两次当点心服之。又间进开胃之药。旬余而安。 

<篇名>5.参麦汤
属性:治阴分亏损已久,浸至肺虚有痰,咳嗽劳喘,或兼肺有结核者。 
人参(三钱) 干麦冬(四钱,带心) 生山药(六钱) 清半夏(二钱) 牛蒡子(三钱,炒,捣) 苏子(二钱,炒,捣) 
生杭芍(三钱) 甘草(钱半) 
人参为补肺之主药,而有肺热还伤肺之虞,有麦冬以佐之,则转能退热。麦冬为 
润肺之要品,而有咳嗽忌用之说,有半夏以佐之,则转能止嗽。至于山药,其收涩也能助人 
参以补气,其粘润也,能助麦冬以滋液。虽多服久服,或有壅滞,而牛蒡子之滑利,实又可以 
相济。且牛蒡子能降肺气之逆,半夏能降胃气、冲气之逆,苏子与人参同用,又能降逆气 
之因虚而逆。平其逆气,则喘与嗽不治自愈矣。用白芍者,因肝 
为肺之对宫,肺金虚损,不能清肃下行以镇肝木,则肝火恒恣横而上逆,故加芍药以敛戢其 
火。且芍药与甘草同用,甘苦化合味近人参,即功近人参,而又为补肺之品也。 
按∶古方多以麦冬治肺虚咳嗽,独徐灵胎谓嗽者断不宜用。盖以其汁浆胶粘太甚,肺中稍有 
客邪,即可留滞不散,惟济以半夏之辛燥开通,则不惟治嗽甚效。即治喘亦甚效。故仲景治 
伤寒解后,虚羸少气,气逆欲吐,有竹叶石膏汤,麦冬与半夏同用。治火逆上气,有麦门冬汤,以麦冬为君 
,亦佐以半夏也。又肺虚劳嗽者,医者多忌用半夏,是未知半夏之性者也。徐灵胎曰∶“肺 
属金喜敛而不喜散。”盖敛则肺叶垂而气顺,散则肺叶张而气逆。半夏之辛,与姜、桂之辛 
迥别,入喉则闭不能言,涂金疮则血不复出,辛中滞涩,故能疏又能敛也。又辛之敛与酸之 
敛不同,酸则一主于敛,辛则敛中有发散之意,尤与肺投合也。又喻嘉言赞麦门冬汤中用半夏曰∶“于大建中气,大生津 
液药中,增入半夏之辛温一味,以利咽下气,此非半夏之功,实善用半夏之功也。” 
愚对于此证,悉心研究,知其治法,当细分为数种。肾传肺者,以大滋真阴之药为主,以清肺理痰之药为佐, 
若拙拟之醴泉饮是也;肺传肾者,以清肺理痰之药为主,以滋补 
真阴之药为佐,若此参麦汤是也;其因肺肾俱病,而累及脾胃者,宜肺肾双补,而兼顾其 
脾胃,若拙拟之滋培汤、珠玉二宝粥是也。如此分途施治,斟酌咸宜,而又兼服阿斯匹林,凡其脉之稍有根柢可挽回者, 
需以时日皆愈也。至于但肺有结核,而未累及他脏者,可于斯编治肺病方中,酌其治法(论肺病治法,实合虚劳肺病详 
细论之 
也,凡治虚劳及肺病者皆宜参观) 
阿斯匹林,其性凉而能散,善退外感之热,初得外感风热,服之出凉汗即愈。兼能退内 
伤之热,肺结核者,借之以消除其热,诚有奇效。又善治急性关节肿疼,发表痘毒、麻疹及 
肠胃炎、肋膜炎诸证,西药中之最适用者也。 
特其发汗之力甚猛。若结晶坚而大者,以治外感,半瓦即可出汗;若当天气寒凉,或近寒带 
之地,须服至一瓦,或至瓦半。若其略似白粉,微有结晶者,药力薄弱,服至一瓦方能出汗 
,多可服至瓦半或二瓦。是在临证者,相其药力之优劣,而因时、因地、因人制宜也。 
至用阿斯匹林治内伤,其分量尤须少用。因内伤发热之人,阴虚阳浮,最易发汗。西人用治 
肺结核之热,日服三瓦,其在欧洲地寒,且其人自幼多肉食,脏腑营卫壮固,或者犹可,在 
吾中华则定然不可。是以丁仲祜用阿斯匹林治肺结核,一日三次共服一瓦半,则视西人所用之 
愚用阿斯匹林治肺结核,视西人所用之数,则减之又减。曾治一少年,染肺结核,咳嗽食少 
,身体羸弱,半载不愈,求为延医。遂投以理肺清痰、健胃滋阴之药,又于晚间临睡时,用 
白蔗糖冲水,送服阿斯匹林三分之一瓦。须臾周身即得大汗,过三点钟其汗始止,翌日觉周 
身酸懒,盖因汗太过也。而咳嗽则较前见轻,食欲亦少振,继服滋补之药数剂,每日只用阿 
斯匹林六分之一瓦,作一次服下,或出微汗,或不出汗,从此精神渐渐清爽,调治月余而愈 
。自此以后,用阿斯匹林治肺结核,必先少少试服,初次断不敢稍多也。 
至西人谓防其出汗,可与止汗之药同服,亦系善法。然仍恐服后止汗之药不效,而阿斯匹林 
之发汗,仍然甚效也。愚治肺结核证,若一日用至一瓦,或一瓦强,恒作十次,或十余次服 
下。勿须用止汗之药,亦可不出汗。即有时微见汗,亦系佳兆。 
凡劳瘵阴虚之证,其脉之急数者,无论肺结核与不结核,于每服滋补剂外,皆宜服阿斯匹 
林,或半瓦,或至一瓦。恐其出汗多,分几次服下,其初日服之俾微见汗,后日日常服,以 
或出汗或不出汗为适宜。如此旬日之间,脉之数者可渐和缓。 
附录∶ 
广西柳州宾××来函∶ 
治一妇人,年四十三岁,素因家务劳心,又兼伤心,遂患吐血。后吐血虽愈,而喘嗽殊甚,夜不能卧。诸医率用枇 
杷叶、 
款冬花、杏仁、紫菀、贝母等药治之。其后右边面颧淡红肿起,嗽喘仍不少愈。后仆为延医,先投以王清任少腹逐瘀汤 
加苏子沉香二剂,继服书中参麦汤八剂,喘嗽皆愈。 
又∶治一男子,年四十六岁,心中发热作喘,医治三年无效。仆为诊视,先投以书中首方资生汤,遵注加生地黄六钱。 
一剂见轻,数剂病愈强半。继用参麦汤数剂,病愈十之八九。然病已数年,身体羸弱,非仓猝所能撤消;望先生赐惠, 
为拟一善后之方,则幸甚矣。 

<篇名>6.珠玉二宝粥
属性:治脾肺阴分亏损,饮食懒进,虚热劳嗽,并治一切阴虚之证。 
生山药(二两) 生薏米(二两) 柿霜饼(八钱) 
上三味,先将山药、薏米捣成粗渣,煮至烂熟,再将柿霜饼切碎,调入融化,随 
意服之。山药、薏米皆清补脾肺之药。然单用山药,久则失于粘腻;单用薏米,久则失于淡 
渗,惟等分并用,乃可久服无弊。又用柿霜之凉可润肺、甘能归脾者,以为之佐使。病患服 
之不但疗病,并可充饥,不但充饥,更可适口。用之对证,病自渐愈,即不对证,亦无他患 
。柿霜饼,即柿霜熬成者,为柿霜白而净者甚少,故用其熬成饼者。然熬此饼时恒有掺以薄荷水 
者,其性即不纯良。遇阴虚汗多之证用之即有不宜,若果有白净柿霜尤胜于饼。 
一少年,因感冒懒于饮食,犹勤稼穑,枵腹力作,遂成劳嗽。过午发热,彻夜咳吐痰涎。医 
者因其年少,多用滋阴补肾之药,间有少加参、 者。调治两月不效,饮食减少,痰涎转增 
,渐至不起,脉虚数兼有弦象,知其肺脾皆有伤损也。授以此方,俾一日两次服之,半月全愈 
或问∶脉现弦象,何以即知其脾肺伤损?答曰∶脉虽分部位,而其大致实不分部位,今此证 
左右之脉皆弦,夫弦为肝脉,肝盛必然侮脾,因肝属木脾属土也。且五行之中,惟土可以包 
括四行,即脾气可以包括四脏。故六部脉中,皆以和缓为贵,以其饶有脾土之气也。今其脉 
不和缓而弦硬,其脾气受伤,不能包括四脏可知。又肺属金,所以镇肝木者也,故肺金清肃 
之气下行,肝木必不至恣横,即脉象不至于弦。今其脉既现如此弦象,则肺金受伤,不能镇 
肝木更可知也。 

<篇名>7.沃雪汤
属性:治同前证,更兼肾不纳气作喘者。 
生山药(一两半) 牛蒡子(炒捣,四钱) 柿霜饼(冲服,六钱) 
一人,年四十余,素有喘证,薄受外感即发。医者投以小青龙汤,一剂即愈,习 
以为常。一日喘证复发,连服小青龙汤三剂不愈。其脉五至余,右寸浮大,重按即无。知其 
从前服小青龙即愈者,因其证原受外感;今服之而不愈者,因此次发喘原无外感也,盖其薄 
受外感即喘;肺与肾原有伤损,但知治其病标,不知治其病本,则其伤损必益甚,是以此 
次不受外感亦发喘也。为拟此汤服两剂全愈,又服数剂以善其后。 

<篇名>8.水晶桃
属性:治肺肾两虚,或咳嗽,或喘逆,或腰膝酸疼,或四肢无力,以治孺子尤佳 
核桃仁(一斤) 柿霜饼(一斤) 
先将核桃仁饭甑蒸熟,再与柿霜饼同装入瓷器内蒸之,融化为一,晾冷,随意服 
凡果核之仁,具补益之性者,皆能补肾。核桃乃果核之最大者,其仁既 
多脂,味更香美,为食中佳品,性善补肾可知。柿霜色白入肺,而甘凉滑润,其甘也能益肺 
气,其凉也能清肺热,其滑也能利肺痰,其润也能滋肺燥,与核 
桃同用,肺肾同补,金水相生,虚者必易壮实。且食之又甚适口,饥时可随便服之,故以治 
小儿尤佳也。 
【附方】俗传治劳嗽方,秋分日取鲜莱菔十余枚去叶,自叶中心穿以鲜槐条,令槐条头透出 
根外,悬于茂盛树上满百日,至一百零一日取下。用时去槐条,将莱菔切片煮烂,调红 
沙糖服之,每服一枚,数服即愈。 
按∶莱菔色白入肺,槐条色黑入肾,如此作用,盖欲导引肺气归肾。其悬于茂盛树上者,因 
茂树之叶多吐氧气,莱菔借氧气酝酿,其补益之力必增也。悬之必满百日者,欲其饱经霜露 
,借金水之气,以补金水之脏也。邑中孙姓叟,年近六旬,劳喘,百药不效,后得此方服之 
而 

<篇名>9.既济汤
属性:治大病后阴阳不相维系。阳欲上脱,或喘逆,或自汗,或目睛上窜,或心中摇摇如悬旌;阴 
欲下脱,或失精,或小便不禁,或大便滑泻。一切阴阳两虚,上热下凉之证。 
大熟地(一两) 萸肉(去净核,一两) 生山药(六钱) 生龙骨(捣细,六钱) 
生牡蛎(捣细,六钱) 茯苓(三钱) 生杭芍(三钱) 乌附子(一钱) 
或问∶既济汤原为救脱之药,方中何以不用人参?答曰∶人参之性补而兼升,以治上脱,转有气高不返之虞。喻嘉 
言《寓意草》中论之甚详。惟与赭石同用,始能纳气归根。而证兼下脱者,赭石又不宜用,为不用赭石,所以不敢用人 
参。且阳之上脱也,皆因真阴虚损,不能潜藏元阳,阳气始无所系恋而上奔。故方中重用熟地、山药以峻补真阴,俾阴 
足自能潜阳。而佐以附子之辛热,原与元阳为同气,协同芍药之苦降(《神农本草经》味苦),自能引浮越之元阳下归其 
宅。更有萸肉、龙骨、牡蛎以收敛之,俾其阴阳固结,不但元阳不复上脱,而真阴亦永不下脱矣。 
一人,年二十余,禀资素羸弱,又耽烟色,于秋初患疟,两旬始愈。一日大便滑 
泻数次,头面汗出如洗,精神颓愦,昏昏似睡。其脉上盛下虚,两寸摇摇,两尺欲无,数至 
七至。延医二人皆不疏方。愚后至为拟此汤,一剂而醒,又服两剂遂复初。 
友人张××,曾治一少年,素患心疼,发时 
昼夜号呼。医者屡投以消通之药,致大便滑泻,虚气连连下泄,汗出如洗,目睛上泛,心神 
惊悸,周身 动,须人手按,而心疼如故。延医数人皆不敢疏方。张××投以此汤,将方中萸 
肉倍作二两,连服两剂,诸病皆愈,心疼竟从此除根。 
或问∶此方能治脱证宜矣,而并能治心疼者何也?答曰∶凡人身内外有疼处,皆其气血痹而 
不通。《神农本草经》谓“山茱萸主心下邪气、寒热、温中、逐寒湿痹”,是萸肉不但酸敛,而更 
善开通可知。李士材治肝虚作疼,萸肉与当归并用。愚治肝虚腿疼,曾重用萸肉随手奏效( 
详案在曲直汤下)。盖萸肉得木气最浓,酸敛之中大具条畅之性 
,故善于治脱,尤善于开痹也。大抵其证原属虚痹,气血因虚不能流通而作疼。医者不知,惟事 
开破,迨开至阴阳将脱,而其疼如故,医者亦束手矣。而投以此汤,惟将萸肉加倍,竟能于 
救脱之外,更将心疼除根。此非愚制方之妙,实张××之因证施用,而善于加减也。 

<篇名>10.来复汤
属性:治寒温外感诸证,大病瘥后不能自复,寒热往来,虚汗淋漓;或但热不寒,汗出而热解,须 
臾又热又汗,目睛上窜,势危欲脱;或喘逆,或怔忡,或气虚不足以息,诸证若见一端,即 
宜急服。 
萸肉(去净核,二两) 生龙骨(捣细,一两) 生牡蛎(捣细,一两) 生杭芍(六钱) 野台参(四钱) 甘草(二钱, 
蜜炙) 
萸肉救脱之功,较参、术、 更胜。盖萸肉之性,不独补肝也,凡人身之阴阳气血将散者,皆能敛之。故救脱之药, 
当 
以萸肉为第一。而《神农本草经》载于中品,不与参、术、 并列者,窃忆古书竹简韦编,易于错简,此或错简之误欤! 
凡人元气之脱,皆脱在肝。故人虚极者,其肝风必先动,肝风动,即元气欲脱之兆也。又肝与胆脏腑相根据,胆为少 
阳, 
有病主寒热往来;肝为厥阴,虚极亦为寒热往来,为有寒热,故多出汗。萸肉既能敛汗,又善补肝,是以肝虚极而元气 
将脱者服之最效。愚初试出此药之能力,以为一己之创见,及详观《神农本草经》山茱萸原主寒热,其所主之寒热,即 
肝经虚极之寒热往来也。特从前涉猎观之,忽不加察,且益叹《神农本草经》之精当,实非后世本草所能及也。又《神 
农本草经》谓山茱萸能逐寒湿痹,是以本方可用以治心腹疼痛。曲直汤用以治肢体疼痛,以其味酸能敛。补络补管汤, 
用之以治咳血吐血。再合以此方重用之,最善救脱敛汗。则山茱萸功用之妙,真令人不可思议矣。 
一人,年二十余,于孟冬得伤寒证,调治十余日,表里皆解。忽遍身发热,顿饭 
顷,汗出淋漓,热顿解,须臾又热又汗。若是两昼夜,势近垂危,仓猝迎愚延医。及至,见 
汗出浑身如洗,目上窜不露黑睛,左脉微细模糊,按之即无,此肝胆虚极,而元气欲脱也, 
盖肝胆虚者,其病象为寒热往来,此证之忽热忽汗,亦即寒热往来之意。急用净萸肉二两煎 
服,热与汗均愈其半,遂为拟此方,服两剂而病若失。 
赵叟,年六十三岁,于仲冬得伤寒证,痰喘甚剧。其脉浮而弱,不任循按,问其平素,言有劳病,冬日恒发喘嗽。 
再三 
筹思,强治以小青龙汤去麻黄,加杏仁、生石膏,为其脉弱,俾预购补药数种备用。服药后喘稍愈,再诊其脉微弱益甚, 
遂急用净萸肉一两,生龙骨、生牡蛎各六钱,野台参四钱,生杭芍三钱为方,皆所素购也。煎汤甫成,此时病患呼吸俱 
微,自觉气息不续,急将药饮下,气息遂能接续。 
李××,年五旬,偶相值,求为诊脉,言前月有病服药已愈,近觉身体清爽,未知脉象何如?诊之,其脉尺部无根, 
寸 
部摇摇有将脱之势,因其自谓病愈,若遽悚以危语,彼必不信,姑以脉象平和答之。遂秘谓其侄曰∶“令叔之脉甚危险, 
当服补敛之药,以防元气之暴脱。”其侄向彼述之,果不相信后二日,忽遣人迎愚,言其骤然眩晕不起,求为延医。既至, 
见其周身颤动,头上汗出,言语错乱,自言心怔忡不能支持,其脉上盛下虚之象较前益甚,急投以净萸肉两半,生龙骨、 
生牡蛎、野台参、生赭石各五钱,一剂即愈。继将萸肉改用一两,加生山药八钱,连服数剂,脉亦复常。按∶此方赭石 
之分量,宜稍重于台参。 
附录∶ 
湖北天门县崔××来函∶ 
张港朱××之儿媳,产后角弓反张,汗出如珠,六脉散乱无根,有将脱之象,迎为延医。急用净萸肉二两,俾煎汤 
服之, 
一剂即愈。 

<篇名>11.镇摄汤
属性:治胸膈满闷,其脉大而弦,按之似有力,非真有力,此脾胃真气外泄,冲脉逆气上干之证, 
慎勿作实证治之。若用开通之药,凶危立见。服此汤数剂后脉见柔和,即病有转机,多服自 
愈。 
野台参(五钱) 生赭石(轧细,五钱) 生芡实(五钱) 生山药(五钱) 萸肉(去净核,五钱) 清半夏(二钱) 茯苓 
(二钱) 
服药数剂后,满闷见轻,去芡实加白术二钱。 
脉之真有力者,皆有洪滑之象。洪者如波涛叠涌,势作起伏;滑者指下滑润,累 
累如贯珠。此脉象弦直,既无起伏之势,又无贯珠之形,虽大而有力,实非真有力之象。 
和缓者脾胃之正脉,弦长者肝胆之正脉。然脾胃属土,其脉象原宜包括金、木、水、火诸脏 
腑,故六部之脉皆有和缓,乃为正象。今其脉弦而有力,乃肝木横恣,侵侮脾土之象,故知 
其脾胃虚也。 
冲脉上隶阳明,故冲气与胃气原相贯通。今因胃气虚而不降,冲气即易于上干。此时脾胃气 
化不固,既有外越之势,冲气复上干而排挤之,而其势愈外越,故其脉又兼大也。 
一媪,年过六旬,胸腹满闷,时觉有气自下上冲,饮食不能下行。其子为书贾,且知医。曾 
因卖书至愚书校,述其母病证,且言脉象大而弦硬。为拟此汤,服一剂满闷即减,又服数 
剂全愈。 
一人,年近五旬,心中常常满闷,呕吐痰水。时觉有气起自下焦,上冲胃口。其 
脉弦硬而长,右部尤甚,此冲气上冲,并迫胃气上逆也。问其大便,言甚干燥。遂将方中赭 
石改作一两,又加知母、生牡蛎各五钱,浓朴、苏子各钱半,连服六剂全愈。 
附录: 
直隶盐山李××来函∶ 
天津王媪,得病月余,困顿已极,求治于弟。诊其脉,六部皆弦硬有力,更粗大异常,询其病,则胸膈满闷,食已 
即吐,月余以来,未得一饭不吐,且每日大便两三次,所便少许有如鸡矢,自云心中之难受,莫可言喻,不如即早与世 
长辞,脱此苦恼。细思胸膈满闷,颇似实证者;,然而脉象弦硬粗大,无一点柔和之象,遂忆镇摄汤下注云,治胸膈满闷, 
其脉大而弦,按之有力,此脾胃真气外泄,冲脉逆气上干之证,慎勿以实证治之云云。即抄镇摄汤原方予之。服一剂, 
吐即见减,大便次数亦见减,脉遂有柔和之象。四五剂,即诸病全愈。以后遇此等脉象,即按此汤加减治之,无不效如 
桴鼓。 


<篇名>敦复汤
属性:(附∶服硫黄法) 
治下焦元气虚惫,相火衰微,致肾弱不能作强(《内经》云肾者作强之官),脾弱不能健运,或腰膝酸疼, 
或黎明泄泻,一切虚寒诸证。 
野台参(四钱) 乌附子(三钱) 生山药(五钱) 补骨脂(四钱,炒捣) 核桃仁(三钱) 萸肉(四钱, 
去净核) 茯苓(钱半) 生鸡内金(钱半,捣细) 
敦复汤,原为补相火之专方,而方中以人参为君,与萸肉、茯苓并用,借其收敛下行之力,能大补肾中元 
气,元气既旺相火自生。又用乌附子、补骨脂之大热纯阳,直达下焦,以助相火之热力,核桃仁之温润多脂, 
峻补肾脏,以浓相火之基址。且附子与人参同用名参附汤,为回元阳之神丹,补骨脂与核桃仁并用名青蛾丸, 
为助相火之妙品(核桃仁属木,补骨脂属火,并用之,有木火相生之妙),又恐药性太热,于下焦真阴久而有 
碍,故又重用生山药,取其汁浆稠粘,能滋下焦真阴,其气味甘温,又能固下焦气化也。至于鸡内金,其健运 
脾胃之力,既能流通补药之滞,其收涩膀胱之力,又能逗留热药之性也。 
人身之热力,方书恒责重相火,而不知君火之热力,较相火尤胜。盖生育子女以相火为主,消化饮食以君 
火为主。君火发于心中,为阳中之火,其热下济,大能温暖脾胃,助其消化之力,此火一衰,脾胃消化之力顿 
减。若君火旺而相火衰者,其人仍能多饮多食可享大寿,是知君火之热力,关于人身者甚大也。愚自临证实验 
以来,遇君火虚者不胜计,其人多廉于饮食,寒饮留滞为恙,投以辛热升补之剂,即随手奏效(拙拟理饮汤为 
治是病的方)。彼谓心脏恶热,用药惟宜寒凉者,犹是一偏之论。曾治一人,年二十余,嗜睡无节,即动作饮 
食之时,亦忽然昏倒鼾睡。诊其脉两尺洪滑有力,知其肾经实而且热也。遂用黄柏、知母各八钱,茯苓、泽泻 
各四钱,数剂而愈。是知人之资禀不齐∶心脏多恶热,而亦有宜温补者;肾脏多恶寒,而亦有宜凉泻者。是在 
临证时细心与之消息,不可拘于成见也。 
欲明心火之热力,今又得一确实证验。愚资禀素强壮,心火颇旺而相火少衰,饮食不忌寒凉,恒畏坐凉处。 
因此,数年来,常于食前,服生硫黄如黑豆大一块,约有四厘,甚见效验。 

\chapter{治阳虚方}
<篇名>敦复汤
属性:尝观葛稚川《肘后方》,首载扁鹊玉壶丹,系硫黄一味九转而成。治一切阳分衰惫之病。而其转法所需之 
物颇难备具,今人鲜有服者。愚临证实验以来,觉服制好之熟硫黄,犹不若径服生者其效更捷。盖硫黄制熟则 
力减,少服无效,多服又有燥渴之弊,服生硫黄少许,即有效而又无他弊也。十余年间,用生硫黄治愈沉寒锢 
冷之病不胜计。盖硫黄原无毒,其毒也即其热也,使少服不令觉热,即于人分毫无损,故不用制熟即可服,更 
可常服也。且自古论硫黄者,莫不谓其功胜桂、附,惟径用生者系愚之创见,而实由自家徐徐尝验,确知其功 
效甚奇,又甚稳妥,然后敢以之治病。今邑中日服生硫黄者数百人,莫不饮食加多,身体强壮,皆愚为之引导 
也。今略举生硫黄治验之病数则于下∶ 
一孺子三岁失乳。频频滑泻,米谷不化,瘦弱异常。俾嚼服生硫黄如绿豆粒大两块,当日滑泻即愈,又服 
数日,饮食加多,肌肉顿长。后服数月,严冬在外嬉戏,面有红光,亦不畏寒。 
一叟年近六旬,得水肿证。小便不利,周身皆肿,其脉甚沉细,自言素有疝气,下焦常觉寒凉。愚曰∶欲 
去下焦之寒,非服硫黄不可。且其性善利水,施之火不胜水而成水肿者尤为对证。为开苓桂术甘汤加野台参三 
钱、威灵仙一钱,一日煎渣再服,皆送服生硫黄末二分。十日后,小便大利,肿消三分之二。下焦仍觉寒凉, 
遂停汤药单服硫黄试验,渐渐加多,一月共服生硫黄四两,周身肿尽消,下焦亦觉温暖。 
一人年十八九,常常呕吐涎沫,甚则吐食。诊其脉象甚迟濡,投以大热之剂毫不觉热,久服亦无效验。俾 
嚼服生硫黄如黄豆粒大,徐徐加多,以服后移时觉微温为度。后一日两次服,每服至二钱,始觉温暖。共服生 
硫黄四斤,病始除根。 
一数月孺子,乳汁不化,吐泻交作,常常啼号,日就羸瘦。其啼时蹙眉,似有腹疼之意。俾用生硫黄末三 
厘许,乳汁送服,数次而愈。 
一人年四十许,因受寒腿疼不能步履。投以温补宣通之剂,愈后,因食猪头(猪头咸寒与猪肉不同)反复 
甚剧,疼如刀刺,再服前药不效。俾每于饭前嚼服生硫黄如玉秫粒大,服后即以饭压之。试验加多,后每服至 
钱许,共服生硫黄二斤,其证始愈。 
一叟年六十有一,频频咳吐痰涎,兼发喘逆。人皆以为劳疾,未有治法。诊其脉甚迟,不足三至,知其寒 
饮为恙也。投以拙拟理饮汤加人参、附子各四钱,喘与咳皆见轻而脉之迟仍旧。因思脉象如此,非草木之品所 
能挽回。俾服生硫黄少许,不觉温暖,则徐徐加多,两月之间,服生硫黄斤余,喘与咳皆愈,脉亦复常。 
一妇人年五旬,上焦阳分虚损,寒饮留滞作嗽,心中怔忡,饮食减少,两腿畏寒,卧床不起者已二年矣。 
医者见其咳嗽怔忡,犹认为阴分虚损,复用熟地、阿胶诸滞泥之品,服之病益剧。后愚诊视,脉甚弦细,不足 
四至,投以拙拟理饮汤加附子三钱,服七八日咳嗽见轻,饮食稍多而仍不觉热,知其数载沉 ,非程功半载不 
能愈也。俾每日于两餐之前服生硫黄三分,体验加多,后服数月,其病果愈。 
按∶古方中硫黄皆用石硫黄,而今之硫黄皆出于石,其色黄而亮,砂粒甚大,且无臭气者即堪服食。且此 
物燃之虽气味甚烈,嚼之实无他味。无论病在上在下,皆宜食前嚼服,服后即以饭压之。若不能嚼服者,为末 
开水送服亦可,且其力最长,即一日服一次,其热亦可昼夜不歇。 

\chapter{治大气下陷方}
<篇名>1.升陷汤
属性:治胸中大气下陷,气短不足以息。或努力呼吸,有似乎喘。或气息将停,危在顷刻。其兼证,或寒热往来, 
或咽干作渴,或满闷怔忡,或神昏健忘,种种病状,诚难悉数。其脉象沉迟微弱,关前尤甚。其剧者,或六脉 
不全,或参伍不调。 
生箭 (六钱) 知母(三钱) 柴胡(一钱五分) 桔梗(一钱五分) 升麻(一钱) 
气分虚极下陷者,酌加人参数钱,或再加山萸肉(去净核)数钱,以收敛气分之耗散,使升者不至复陷更 
佳。若大气下陷过甚,至少腹下坠,或更作疼者,宜将升麻改用钱半,或倍作二钱。 
升陷汤,以黄 为主者,因黄 既善补气,又善升气。惟其性稍热,故以知母之凉润者济之。柴胡为少阳 
之药,能引大气之陷者自左上升。升麻为阳明之药,能引大气之陷者自右上升。桔梗为药中之舟楫,能载诸药 
之力上达胸中,故用之为向导也。至其气分虚极者,酌加人参,所以培气之本也。或更加萸肉,所以防气之涣 
也。至若少腹下坠或更作疼,其人之大气直陷至九渊,必需升麻之大力者,以升提之,故又加升麻五分或倍作 
二钱也。方中之用意如此,至随时活泼加减,尤在临证者之善变通耳。 
大气者,充满胸中,以司肺呼吸之气也。人之一身,自飞门以至魄门,一气主之。然此气有发生之处, 
有培养之处,有积贮之处。天一生水,肾脏先成,而肾系命门之中(包肾之膜油连于脊椎自下上数七节处)有 
气息息萌动,此乃干元资始之气,《内经》所谓“少火生气”也。此气既由少火发生,以徐徐上达。培养于后 
天水谷之气,而磅因礴 
之势成。绩贮于膺胸空旷之府,而盘据之根固。是大气者,原以元气为根本,以水谷之气为养料,以胸中之地 
为宅窟者也。夫均是气也,至胸中之气,独名为大气者,诚以其能撑持全身,为诸气之纲领,包举肺外,司呼 
吸之枢机,故郑而重之曰大气。夫大气者,内气也。呼吸之气,外气也。人觉有呼吸之外气与内气不相接续者, 
即大气虚而欲陷,不能紧紧包举肺外也。医者不知病因,犹误认为气郁不舒,而开通之。其剧者,呼吸将停, 
努力始能呼吸,犹误认为气逆作喘,而降下之。则陷者益陷,凶危立见矣。其时作寒热者,盖胸中大气,即上 
焦阳气,其下陷之时,非尽下陷也,亦非一陷而不升也。当其初陷之时,阳气郁而不畅则作寒,既陷之后,阳 
气蓄而欲宣则作热。迨阳气蓄极而通,仍复些些上达,则又微汗而热解。其咽干者,津液不能随气上潮也。其 
满闷者,因呼吸不利而自觉满闷也。其怔忡者,因心在膈上,原悬于大气之中,大气既陷,而心无所附丽也。 
其神昏健忘者,大气因下陷,不能上达于脑,而脑髓神经无所凭借也。其证多得之力小任重或枵腹力作,或病 
后气力未复,勤于动作,或因泄泻日久,或服破气药太过,或气分虚极自下陷,种种病因不同。而其脉象之微 
细迟弱,与胸中之短气,实与寒饮结胸相似。然诊其脉似寒凉,而询之果畏寒凉,且觉短气者,寒饮结胸也; 
诊其脉似寒凉,而询之不畏寒凉,惟觉短气者,大气下陷也。且即以短气论,而大气下陷之短气,与寒饮结胸 
之短气,亦自有辨。寒饮结胸短气,似觉有物压之;大气下陷短气,常觉上气与下气不相接续。临证者当细审 
之(寒饮结胸详理饮汤下)。 
肺司呼吸,人之所共知也。而谓肺之所以能呼吸者,实赖胸中大气,不惟不业医者不知,即医家知者亦鲜, 
并方书亦罕言及。所以愚初习医时,亦未知有此气。迨临证细心体验,始确知于肺气呼吸之外,别有气贮于胸 
中,以司肺脏之呼吸。而此气, 
且能撑持全身,振作精神,以及心思脑力、官骸动作,莫不赖乎此气。此气一虚,呼吸即觉不利,而且肢体酸 
懒,精神昏愦,脑力心思,为之顿减。若其气虚而且陷,或下陷过甚者,其人即呼吸顿停,昏然罔觉。愚既实 
验得胸中有此积气与全身有至切之关系,而尚不知此气当名为何气。涉猎方书,亦无从考证。惟《金匮》水气 
门,桂枝加黄 汤下,有“大气一转,其气乃散”之语。后又见喻嘉言《医门法律》谓∶“五脏六腑,大经小 
络,昼夜循环不息,必赖胸中大气,斡旋其间”。始知胸中所积之气,当名为大气。因忆向读《内经》热论篇 
有“大气皆去病日已矣”之语,王氏注大气,为大邪之气也。若胸中之气,亦名为大气,仲景与喻氏果何所本? 
且二书中亦未尝言及下陷。于是复取《内经》,挨行逐句细细研究,乃知《内经》所谓大气,有指外感之气言 
者,有指胸中之气言者。且知《内经》之所谓宗气,亦即胸中之大气。并其下陷之说,《内经》亦尝言之。 
今试取《内经》之文释之。《灵枢》五味篇曰∶“谷始入于胃,其精微者,先出于胃之两焦,以溉五脏, 
别出两行营卫之道,其大气之抟而不行者,积于胸中,命曰气海。出于肺,循喉咽,故呼则出,吸则入。天地 
之精气,其大数常出三入一,故谷不入半日则气衰,一日则气少矣。”愚思肺悬胸中,下无透窍,胸中大气, 
包举肺外,上原不通于喉,亦并不通于咽,而曰出于肺循喉咽,呼则出,吸则入者,盖谓大气能鼓动肺脏使之 
呼吸,而肺中之气,遂因之出入也。所谓天地之精气常出三入一者,盖谓吸入之气,虽与胸中不相通,实能隔 
肺膜通过四分之一以养胸中大气,其余三分吐出,即换出脏腑中混浊之气,此气化之妙用也。然此篇专为五味 
养人而发,故第言饮食能养胸中大气,而实未发明大气之本源。愚尝思之,人未生时,皆由脐呼吸,其胸中原 
无大气,亦无需乎大气。迨胎气日盛,脐下元气渐充,遂息息 
上达胸中而为大气。大气渐满,能鼓动肺膜使之呼吸,即脱离母腹,由肺呼吸而通天地之气矣(西人谓肺之呼吸 
延髓主之,胸中大气实又为延髓之原动力)。 
至大气即宗气者,亦尝深考《内经》而得之。《素问》平人气象论曰∶“胃之大络名虚里,出于左乳下, 
其动应衣,脉宗气也。” 
按∶虚里之络,即胃输水谷之气于胸中,以养大气之道路。而其贯膈络肺之余,又出于左乳下为动脉。是 
此动脉,当为大气之余波,而曰宗气者,是宗气即大气,为其为生命之宗主,故又尊之曰宗气。其络所以名虚 
里者,因其贯膈络肺游行于胸中空虚之处也。 
又∶《灵枢》客邪篇曰∶“五谷入于胃,其糟粕、津液、宗气,分为三隧。故宗气积于胸中,出于喉咙, 
以贯心脉,而行呼吸焉。”观此节经文,则宗气即为大气,不待诠解。且与五味篇同为伯高之言,非言出两人, 
而或有异同。且细审“以贯心脉,而行呼吸”之语,是大气不但为诸气之纲领,并可为周身血脉之纲领矣。至 
大气下陷之说,《内经》虽无明文,而其理实亦寓于《内经》中。《灵枢》五色篇雷公问曰∶“人无病卒死, 
何以知之?”黄帝曰∶“大气入于脏腑者,不病而卒死。”夫人之膈上,心肺皆脏,无所谓腑也。经既统言脏 
腑,指膈下脏腑可知。以膈上之大气,入于膈下之脏腑,非下陷乎?大气既陷,无气包举肺外以鼓动其 辟之 
机,则呼吸顿停,所以不病而猝死也。观乎此,则大气之关于人身者,何其重哉! 
愚深悯大气下陷之证医多误治,因制升陷汤一方,又有回阳升陷汤、理郁升陷汤二方,皆由升陷汤加减而 
成。此三升陷汤后,附载治愈之案,其病之现状∶有呼吸短气者,有心中怔忡者,有淋漓大汗者,有神昏健忘 
者,有声颤身动者,有寒热往来者,有胸中满闷者(此因呼吸不利而自觉满闷,若作满闷治之立危),有努力 
呼吸似喘者(此种现状尤多,乃肺之呼吸将停,其人努力呼吸以自救,若作喘证治之立危),有咽干作渴者, 
有常常呵 
欠者,有肢体痿废者,有食后易饥者,有二便不禁者,有癃闭身肿者,有张口呼气外出而气不上达,肛门突出 
者,在女子有下血不止者,更有经水逆行者(证因气逆者多,若因气陷致经水逆行者曾见有两人,皆投以升陷 
汤治愈),种种病状实难悉数。其案亦不胜录。治愈大气下陷之案,略登数则于下,以备考征。 
有兄弟二人,其兄年近六旬,弟五十余。冬日畏寒,共处一小室中,炽其煤火,复严其户牖。至春初,二 
人皆觉胸中满闷,呼吸短气。盖因户牖不通外气,屋中氧气全被煤火着尽,胸中大气既乏氧气之助,又兼受炭 
气之伤,日久必然虚陷,所以呼吸短气也。因自觉满闷,医者不知病因,竟投以开破之药。迨开破益觉满闷, 
转以为药力未到,而益开破之。数剂之后,其兄因误治,竟至不起。其弟服药亦增剧,而犹可支持,遂延愚诊 
视。其脉微弱而迟,右部尤甚,自言心中发凉,少腹下坠作疼,呼吸甚觉努力。知其胸中大气下陷已剧,遂投 
以升陷汤,升麻改用二钱,去知母,加干姜三钱。两剂,少腹即不下坠,呼吸亦顺。将方中升麻、柴胡、桔梗 
皆改用一钱,连服数剂而愈。 
一人,年四十八。素有喘病,薄受外感即发,每岁反复两三次,医者投以小青龙加石膏汤辄效。一日反复 
甚剧,大喘昼夜不止。医者投以从前方两剂,分毫无效。延愚诊视,其脉数至六至,兼有沉濡之象。疑其阴虚 
不能纳气,故气上逆而作喘也。因其脉兼沉濡,不敢用降气之品。遂用熟地黄、生山药、枸杞、玄参、大滋真 
阴之品,大剂煎汤,送服人参小块二钱。连服三剂,喘虽见轻,仍不能止。复诊视时,见令人为其椎背,言背 
常发紧,椎之则稍轻,呼吸亦稍舒畅。此时,其脉已不数,仍然沉濡。因细询,此次反复之由,言曾努力搬运 
重物,当时即觉气分不舒,迟两三日遂发喘。乃恍悟,此证因阴虚不能纳气,故难于吸。因用力太过,大气下 
陷,故难于呼。其呼吸皆须努力,故呼 
吸倍形迫促。但用纳气法治之,止治其病因之半,是以其喘亦止愈其半也。遂改用升陷汤,方中升麻、柴胡、 
桔梗,皆不敢用,以桂枝尖三钱代之。又将知母加倍,再加玄参四钱,连服数剂全愈。 
按∶此证虽大气下陷,而初则实兼不纳气也。升麻、柴胡、桔梗,虽能升气,实与不纳气之证有碍,用之 
恐其证仍反复。惟桂枝性本条达,能引脏腑之真气上行,而又善降逆气。仲景苓桂术甘汤,用之以治短气,取 
其能升真气也。桂枝加桂汤,用之以治奔豚,取其能降逆气也。且治咳逆上气吐吸(喘也)《神农本草经》 
原有明文。既善升陷,又善降逆,用于此证之中,固有一无二之良药也。 
或问∶桂枝一物耳,何以既能升陷又能降逆?答曰∶其能升陷者,以其枝直上而不下垂,且色赤属火,而 
性又温也。其能降逆者,以其味辛,得金气而善平肝木,凡逆气之缘肝而上者(逆气上升者多由于肝),桂枝 
皆能镇之。大抵最良之药,其妙用恒令人不测。拙拟参赭镇气汤后,有单用桂枝治一奇病之案,可以参观。 
一人,年二十余。动则作喘,时或咳嗽。医治数年,病转增剧,皆以为劳疾不可治。其脉非微细,而指下 
若不觉其动。知其大气下陷,不能鼓脉外出,以成起伏之势也。投以升陷汤,加人参、天冬各三钱,连服数剂 
而愈。因其病久,俾于原方中减去升麻,为末炼蜜作丸药,徐服月余,以善其后。 
一人,年二十四。胸中满闷,昼夜咳嗽,其咳嗽时,胁下疼甚。诊其脉象和平,重按微弦无力。因其胁疼, 
又兼胸满,疑其气分不舒,少投以理气之药。为其脉稍弱,又以黄 佐之,而咳嗽与满闷益甚,又兼言语声颤 
动。乃细问病因,知其素勤稼穑,因感冒懒食,犹枵腹力作,以致如此。据此病因,且又服理气之药不受,其 
为大气下陷无疑。遂投以升陷汤,四剂,其病脱然。 
按∶此证之形状,似甚难辨,因初次未细诘问,致用药少有 
差错,犹幸迷途未远,即能醒悟,而病亦旋愈。由斯观之,临证者,甚勿自矜明察,而不屑琐琐细问也。 
一人,年四十许。失音半载,渐觉咽喉发紧,且常溃烂,畏风恶寒,冬日所着衣服,至孟夏犹未换。饮食 
减少,浸成虚劳,多方治疗,病转增剧。诊其脉,两寸微弱,毫无轩起之象,知其胸中大气下陷也。投以升陷 
汤,加玄参四钱,两剂,咽喉即不发紧。遂减去升麻,又连服十余剂,诸病皆愈。 
西丰县张××,年十八九,患病数年不愈,来院延医。其证夜不能寐,饮食减少,四肢无力,常觉短气。其 
脉关前微弱不起。知系胸中大气下陷,故现种种诸证。投以升陷汤,为其不寐,加熟枣仁、龙眼肉各四钱,数 
剂全愈。 
奉天于氏女,出嫁而孀,根据居娘门。因病还家中,夜忽不能言,并不能息。其同院住者王××,系愚门生, 
急来院扣门,求为援救。原素为诊脉调药,知其大气虚损,此次之证,确知其为大气下陷,遂为疏方,用生箭 
一两,当归四钱,升麻二钱,煎服,须臾即能言语。翌晨舁至院中,诊其脉沉迟微弱,其呼吸仍觉短气。遂 
将原方减升麻一钱,又加生山药、知母各三钱,柴胡、桔梗各一钱,连服数剂全愈。按此证,脉迟而仍用知母 
者,因大气下陷之脉大抵皆迟,非因寒凉而迟也,用知母以济黄 之热,则药性和平,始能久服无弊。 
奉天袁姓少妇,小便处常若火炙,有时觉腹中之气下坠,则炙热益甚。诊其脉关前微弱,关后重按又似有 
力。其呼吸恒觉短气,心中时或发热。知其素有外感伏邪,久而化热,又因胸中大气下陷,伏邪亦随之下陷也。 
治以升陷汤加生石膏八钱,后渐加至二两,服药旬日全愈。 
或疑大气下陷者,气不上达也,喘者,气不下降也,何以历述大气下陷之病状,竟有努力呼吸有似乎喘者? 
答曰∶此理不易骤解,仍宜以治愈之案证之。一人,年二十余。因力田劳苦过度,致胸中大气下陷,四肢 
懒动,饮食减少,自言胸中满闷,其实非满闷乃短气也,病患不善述病情,往往如此。医者不能自审病因,投 
以开胸理气之剂,服之增重。又改用半补半破之剂,服两剂后,病又增重。又延他医,投以桔梗、当归、木香 
各数钱,病大见愈,盖全赖桔梗升提气分之力也,医者不知病愈之由,再服时,竟将桔梗易为苏梗,升降易性, 
病骤反复。自此不敢服药。迟延二十余日,病势垂危,喘不能卧,昼夜倚壁而坐;假寐片时,气息即停,心下 
突然胀起,急呼醒之,连连喘息数口,气息始稍续;倦极偶卧片时,觉腹中重千斤,不能转侧,且不敢仰卧; 
其脉乍有乍无,寸关尺或一部独见,或两部同见,又皆一再动而止。此病之危,已至极点。因确知其为大气下 
陷,遂放胆投以生箭 一两,柴胡、升麻、净萸肉各二钱。煎服片时,腹中大响一阵,有似昏愦,苏息片时, 
恍然醒悟。自此呼吸复常,可以安卧,转侧轻松。其六脉皆见,仍有雀啄之象。自言百病皆除,惟觉胸中烦热, 
遂将方中升麻、柴胡皆改用钱半,又加知母、玄参各六钱,服后脉遂复常。惟左关三五不调,知其气分之根柢 
犹未实也,遂用野台参一两,玄参、天冬、麦冬(带心)各三钱,两剂全愈。 
盖人之胸中大气,实司肺脏之呼吸。此证因大气下陷过甚,呼吸之机关将停,遂勉强鼓舞肺气,努力呼吸 
以自救,其迫促之形有似乎喘,而实与气逆之喘有天渊之分。观此证假寐片时,肺脏不能努力呼吸,气息即无, 
其病情可想也。设以治气逆作喘者治此证之喘,以治此证之喘者治气逆作喘,皆凶危立见。然欲辨此二证,原 
有确实征验∶凡喘证,无论内伤外感,其剧者必然肩息(《内经》谓喘而肩上抬者为肩息);大气下陷者,虽 
至呼吸有声,必不肩息。盖肩息者,因喘者之吸气难,不肩息者,因大气下陷者之呼气难 
也。欲辨此证,可作呼气难与吸气难之状,以默自体验,临证自无差谬。又喘者之脉多数,或有浮滑之象,或 
尺弱寸强,大气下陷之脉,皆与此成反比例,尤其明征。 
一人,年四十许。每岁吐血两三次,如此四年,似有一年甚于一年之势。其平素常常咳嗽,痰涎壅滞,动 
则作喘,且觉短气。其脉沉迟微弱,右部尤甚。知其病源系大气下陷,投以升陷汤,加龙骨、牡蛎(皆不用 )、 
生地黄各六钱,又将方中知母改用五钱,连服三剂,诸病皆愈。遂减去升麻,又服数剂以善其后。 
或问∶吐血之证,多由于逆气上干,而血随气升。此证既大气下陷,当有便血溺血之证,何以竟吐血乎? 
答曰∶此证因大气陷后,肺失其养,劳嗽不已,以致血因嗽甚而吐出也。究之胸中大气,与上逆之气原迥异。 
夫大气为诸气之纲领,大气陷后,诸气无所统摄,或更易于上干。且更有逆气上干过甚,排挤胸中大气下陷者 
(案详参赭镇气汤下)。至便血溺血之证,由于大气下陷者诚有之,在妇女更有因之血崩者(案详固冲汤下)。 
又转有因大气下陷,而经血倒行,吐血衄血者(案详加味麦门冬汤下)。是知大气既陷,诸经之气无所统摄, 
而或上或下错乱妄行,有不能一律论者。 
或问∶龙骨、牡蛎为收涩之品,大气陷者宜升提,不宜收涩。今方中重用二药,皆至六钱,独不虑其收 
涩之性,有碍大气之升乎?答曰∶龙骨、牡蛎最能摄血之本源。此证若但知升其大气,恐血随升气之药复妄动, 
于升陷汤中,加此二药,所以兼顾其血也。且大气下陷后,虑其耗散,有龙骨、牡蛎以收敛之,转能辅升陷汤 
之所不逮。况龙骨善化瘀血(《神农本草经》主 瘕),牡蛎善消坚结(观其治瘰 可知)。二药并用,能使 
血之未离经者,永安其宅,血之已离经者,尽化其滞。加于升陷汤中,以治气陷兼吐血之证,非至稳善之妙药乎。 
按∶吐血证最忌升麻。此证兼吐血,服升陷汤时,未将升麻 
减去者,因所加之龙骨、牡蛎原可监制之,而服药之时,吐血之证,犹未反复也。若恐升麻有碍血证时,亦可 
减去之,多加柴胡一钱。 
一人,年四十余。小便不利,周身漫肿,自腰以下,其肿尤甚。上焦痰涎杜塞,剧时几不能息。咳嗽痰中 
带血,小便亦有血色。迁延半载,屡次延医服药,病转增剧。其脉滑而有力,疑是湿热壅滞,询之果心中发热。 
遂重用滑石、白芍以渗湿清热,佐以柴胡、乳香、没药以宣通气化。为其病久,不任疏通,每剂药加生山药两 
许,以固气滋阴。又用药汁,送服三七末二钱,以清其血分。数剂热退血减,痰涎亦少,而小便仍不利。偶于 
诊脉时,见其由卧起坐,因稍费力,连连喘息十余口,呼吸始顺。且其脉从前虽然滑实,究在沉分。此时因火 
退,滑实既减,且有濡象。恍悟此证确系大气下陷。遂投以升陷汤,知母改用六钱,又加玄参五钱,木通二钱, 
一剂小便即利。又服数剂,诸病全愈。 
一人,年四十七。咳嗽短气,大汗如洗,昼夜不止,心中怔忡,病势危急。遣人询方,俾先用山萸肉 
(去净核)二两煎服,以止其汗。翌日迎愚诊视,其脉微弱欲无,呼吸略似迫促。自言大汗虽止,而仍有出汗 
之时,怔忡见轻,仍觉短气。知其确系大气下陷,遂投以升陷汤,为其有汗,加龙骨、牡蛎(皆不用 )各五 
钱,三剂而愈。 
一妇人,年二十余。资禀素羸弱,因院中失火,惊恐过甚,遂觉呼吸短气,心中怔忡,食后更觉气不上 
达,常作太息。其脉近和平,而右部较沉。知其胸中大气,因惊恐下陷,《内经》所谓恐则气陷也。遂投以升 
陷汤,为心中怔忡,加龙眼肉五钱,连服四剂而愈。 
一妇人,年二十余。因境多拂郁,常作恼怒,遂觉呼吸短气,咽干作渴,剧时,觉气息将停,努力始能 
呼吸。其脉左部如 
常,右部来缓去急,分毫不能鼓指。《内经》谓宗气贯心脉,宗气即大气也。此证盖因常常恼怒,致大气下陷, 
故不能鼓脉外出,以成波澜也。遂投以升陷汤,为其作渴,将方中知母改用六钱,连服三剂,病愈强半,右脉 
亦较前有力,遂去升麻,又服数剂全愈。 
或问∶《内经》谓恐则气陷,前案中已发明之。然《内经》又谓怒则气逆也,何以与此案中之理,相矛盾 
乎?答曰∶《内经》所谓怒则气逆者,指肝胆之气而言,非谓胸中大气也。然肝胆之气上逆,上冲大气亦上逆 
者,故人当怒急之时,恒有头目眩晕,其气呼出不能吸入,移时始能呼吸,此因大气上逆也。有肝胆之气上逆, 
排挤大气转下陷者,拙拟参赭镇气汤下,有治验之案可考也。况大气原赖谷气养之,其人既常恼怒,纳谷必少, 
大气即暗受其伤,而易下陷也。 
一妇人,因临盆努力过甚,产后数日,胁下作疼,又十余日,更发寒热。其翁知医,投以生化汤两剂,病 
大见愈。迟数日,寒热又作。遂延他医调治,以为产后瘀血为恙,又兼受寒,于活血化瘀药中,重加干姜。数 
剂后,寒热益甚,连连饮水,不能解渴。时当仲夏,身热如炙,又复严裹浓被,略以展动,即觉冷气侵肤。后 
愚诊视,左脉沉细欲无,右脉沉紧,皆有数象。知其大气下陷,又为热药所伤也。其从前服生化汤觉轻者,全 
得芎 升提之力也。治以升陷汤,将方中知母改用八钱,又加玄参六钱,一剂而寒热已,亦不作渴。从前两日 
不食,至此遂能饮食。惟胁下微疼,继服拙拟理郁升陷汤,二剂全愈。 
按∶产后虽有实热,若非寒温外感之热,忌用知母,而不忌用玄参,以玄参原为治产乳之药,《神农本草 
经》有明文也。此证虽得之产后,时已逾月,故敢放胆重用知母。 
或问∶紧为受寒之脉,故伤寒麻黄汤证,其脉必紧。此证既为热药所伤,何以其右脉沉紧?答曰∶脉沉紧 
者,其脉沉而有力 
也。夫有力当作洪象,此证因大气下陷,虽内有实热,不能鼓脉作起伏之势,故不为洪而为紧,且为沉紧也。 
其独见于右部者,以所服干姜之热,胃先受之也。 
按∶脉无起伏为弦,弦而有力,即紧脉也。若但弦,则为寒矣。仲景平脉篇谓“双弦者寒,偏弦者饮。” 
究之饮为稀涎,亦多系因寒而成也。 
一妇人,年三十余。得下痿证,两腿痿废,不能屈伸,上半身常常自汗,胸中短气,少腹下坠,小便不利, 
寝不能寐。延医治疗数月,病势转增。诊其脉细如丝,右手尤甚。知其系胸中大气下陷,欲为疏方。病家疑而 
问曰∶“大气下陷之说,从前医者,皆未言及。然病之本源,既为大气下陷,何以有种种诸证乎?”答曰∶人 
之大气虽在胸中,实能统摄全身,今因大气下陷,全身无所统摄,肢体遂有废而不举之处,此两腿之所以痿废 
也。其自汗者,大气既陷,外卫之气亦虚也。其不寐者,大气既陷,神魂无所根据附也。小便不利者,三焦之气 
化,不升则不降,上焦不能如雾,下焦即不能如渎也。至于胸中短气,少腹下坠,又为大气下陷之明征也。遂 
治以升陷汤,因其自汗,加龙骨、牡蛎(皆不用 )各五钱,两剂汗止,腿稍能屈伸,诸病亦见愈。继服拙拟 
理郁升陷汤数剂,两腿渐能着力。然痿废既久,病在筋脉,非旦夕所能脱然。俾用舒筋通脉之品,制作丸药, 
久久服之,庶能全愈。 
一妇人,年三十许。胸中满闷,不能饮食。医者纯用开破之药数剂,忽然寒热,脉变为迟。医者见脉迟, 
又兼寒热,方中加黄 、桂枝、干姜各数钱,而仍多用破气之药。购药未服,愚应其邻家延请,适至其村,病 
家求为诊视,其脉迟而且弱,问其呼吸觉短气乎?答曰∶今于服药数剂后,新添此证。知其胸中大气因服破气 
之药下陷。时医者在座,不便另为疏方,遂谓医曰∶子方中所加之药,极为对证,然此时其胸中大气下陷,破 
气药分毫不可再用。遂单将所加之黄 、桂枝、干姜煎服。寒热顿已,呼吸亦觉畅舒。后医者即方略为加减, 
又服数剂全愈。 
门人高××曾治一人,年三十余。因枵腹劳力过度,致大气下陷。寒热往来,常常短气,大汗淋漓,头疼咽 
干,畏凉嗜睡,迁延日久,不能起床。医者误认为肝气郁结,投以鳖甲、枳实、麦芽诸药,病益剧。诊其脉, 
左寸关尺皆不见,右部脉虽见,而微弱欲无。知其为大气下陷,投以升陷汤,加人参三钱,一剂左脉即见,又 
将知母改用五钱,连服数剂全愈。 
高××又治一妇人,年三十许。胸中短气,常常出汗,剧时觉气不上达,即昏不知人,移时始苏,睡时恒自 
惊寤。诊其脉,微弱异常,知其胸中大气下陷甚剧。遂投以升陷汤,知母改用五钱,又加人参、萸肉(去净核) 
各三钱,连服数剂全愈。 
大气下陷之证,不必皆内伤也,外感证亦有之。一人年四十许,于季春得温证,延医调治不愈,留连两旬, 
病益沉重。后愚诊视,其两目清白无火,竟昏愦不省人事,舌干如磋,却无舌苔。问之亦不能言语,周身皆凉, 
其五六呼吸之顷,必长出气一口。其脉左右皆微弱,至数稍迟,此亦胸中大气下陷也。盖大气不达于脑中则神 
昏,大气不潮于舌本则舌干,神昏舌干,故问之不能言也。其周身皆凉者,大气陷后,不能宣布于营卫也。其 
五六呼吸之顷,必长出气者,大气陷后,胸中必觉短气,故太息以舒其气也。遂用野台参一两、柴胡二钱,煎 
汤灌之,一剂见轻,两剂全愈。 
按∶此证从前原有大热,屡经医者调治,大热已退,精神愈惫。医者诿为不治,病家亦以为气息奄奄,待 
时而已。乃迟十余日,而病状如故,始转念或可挽回,而迎愚诊视。幸投药不瘥,随手奏效,是知药果对证, 
诚有活人之功也。 
又按∶此证若不知为大气下陷,见其舌干如斯,但知用熟地、阿胶、枸杞之类滋其津液,其滞泥之性, 
填塞膺胸,既陷之大气将何由上达乎?愚愿业医者,凡遇气分不舒之证,宜先存一大气下陷理想,以细心体察, 
倘遇此等证,庶可挽回人命于顷刻也。 
一人,年三十余。于初夏得温病,医者用凉药清解之,兼用枳实、青皮破气诸品,连服七八剂,谵语不省 
人事,循衣摸床,周身颤动。再延他医,以为内风已动,辞不治。后愚诊视,其脉五至,浮分微弱,而重按似 
有力,舌苔微黄,周身肌肤不热,知其温热之邪,随破气之药下陷已深,不能外出也。遂用生石膏二两,知母、 
野台参各一两,煎汤两茶杯,分二次温服。自午至暮,连进二剂,共服药四次,翌日精神清爽,能进饮食,半 
日进食五次,犹饥而索食。看护者不敢复与,则周身颤动,复发谵语,疑其病又反复,求再诊视。其脉象大致 
和平,而浮分仍然微弱。恍悟其胸中大气,因服破气之药下陷,虽用参数次,至此犹未尽复,故亟亟求助于水 
谷之气,且胃中之气,因大气下陷无所统摄,或至速于下行,而饮食亦因之速下也。遂用野台参两许,佐以麦 
门冬(带心)三钱、柴胡二钱,煎汤饮下,自此遂愈。 
或问∶子所治大气下陷证,有两日不食者,有饮食减少者。此证亦大气下陷,何以转能多食?答曰∶事有 
常变,病亦有常变。王清任《医林改错》载有所治胸中瘀血二案∶一则胸不能着物,一则非以物重压其胸不安, 
皆治以血腑逐瘀汤而愈。夫同一胸中瘀血,其病状竟若斯悬殊,故同一大气之下陷也,其脾胃若因大气下陷, 
而运化之力减者,必然少食;若大气下陷,脾胃之气亦欲陷者,或转至多食。曾治一少妇,忽然饮食甚多,一 
时觉饥不食,即心中怔忡。医者以为中消证,屡治不效。向愚询方,疑其胸中大气下陷,为开升陷汤方,加龙 
骨、牡蛎(皆不用 )各五钱,数剂而愈。盖病因虽同,而病之情状,恒因人之资禀不同,而有变易。斯在临 
证者,细心体察耳。 
按∶此证与前证,虽皆大气下陷,而实在寒温之余,故方中不用黄 ,而用人参。因寒温之热,最能铄耗 
津液,人参能补气,兼能生津液,是以《伤寒论》方中,凡气虚者,皆用人参,而不用黄 也。 
上所列者,皆大气下陷治验之案也。然此证为医者误治及失于不治者甚多,略登数则于下,以为炯戒。 
庚戍秋,在沧州治病,有赵姓,忽过访,言有疑事,欲质诸先生。问何疑?曰∶予妹半月前来归宁,数日 
间,无病而亡,未知何故?愚曰∶此必有病,子盖未知耳。×曰∶其前一日,觉咽喉发闷,诊其脉沉细,疑其 
胸有郁气,俾用开气之药一剂,翌日不觉轻重,惟自言不再服药,斯夕即安坐床上而逝。其咽喉中发闷,并不 
甚剧,故曰无病也。愚曰∶此胸中大气下陷耳。时行箧中有治大气下陷诸案,因出示之,且为剖析其理。×泫 
然流涕曰,斯诚为药误矣。 
一人,年三十余。呼吸短气,胸中满闷。医者投以理气之品,似觉稍轻,医者以为药病相投,第二剂,遂 
放胆开破其气分。晚间服药,至夜如厕,便后遂不能起。看护者,扶持至床上,昏昏似睡,呼之不应,须臾张 
口呼气外出,若呵欠之状,如斯者日余而亡。后其兄向愚述之,且问此果何病?因历举大气下陷之理告之。其 
兄连连太息,既自悔择医不慎,又痛恨医者误人,以后不敢轻于延医服药。 
一农家媪,年五十余。因麦秋农家忙甚,井臼之事皆自任之,渐觉呼吸不利,气息迫促。医者误认为气逆 
作喘,屡投以纳气降气之药,气息遂大形迫促,其努力呼吸之声,直闻户外。延愚诊视,及至,诊其脉左右皆 
无,勉为疏方,取药未至而亡,此 
亦大气下陷也。其气息之迫促,乃肺之呼吸将停,努力呼吸以自救也,医者又复用药,降下其气,何其谬哉! 
一诸生,年五十六,为学校教员,每讲说后,即觉短气,向愚询方。愚曰,此胸中大气,虚而欲陷,为至 
紧要之证,当多服升补气分之药。彼欲用烧酒炖药,谓朝夕服之甚便。愚曰,如此亦可,然必须将药炖浓,多 
饮且常饮耳。遂为疏方,用生黄 四两、野台参二两,柴胡、桔梗各八钱,先用黄酒斤许,煎药十余沸,再用 
烧酒二斤,同贮瓶中,置甑中炖开,每饭前饮之,旬日而愈。后因病愈,置不复饮。隔年,一日步行二里许, 
自校至家,似有气息迫促之状,不能言语,倏忽而亡。盖其身体素胖,艰于行步,胸中大气,素有欲陷之机, 
因行动劳苦,而遂下陷,此诚《内经》所谓“大气入于脏腑,不病而猝死”者也。方书有气厥,中气诸名目, 
大抵皆大气下陷之证,特未窥《内经》之旨,而妄为议论耳。 
按∶《内经》原有气厥二字,乃谓气厥逆上行, 
非后世所谓气厥也。 
或问∶案中所载大气下陷证,病因及其病状,皆了如指掌矣。然其脉之现象,或见于左部,或见于右部, 
或左右两部皆有现象可征,且其脉多迟,而又间有数者,同一大气之下陷也,何以其脉若是不同乎?答曰∶胸 
中大气包举肺外,原与肺有密切之关系,肺之脉诊在右部,故大气下陷,右部之脉多微弱者其常也。然人之元 
气自肾达肝,自肝达于胸中,为大气之根本。其人或肝肾素虚,或服破肝气之药太过,其左脉或即更形微弱, 
若案中左部寸关尺皆不见,左脉沉细欲无,左关参伍不调者是也。至其脉多迟,而又间有数者,或因阴分虚损, 
或兼外感之热,或为热药所伤,乃兼证之现脉,非大气下陷之本脉也。 
或问∶李东垣补中益气汤所治之证,若身热恶寒、心烦懒言,或喘、或渴、或阳虚自汗,子所治大气下 
陷案中,类皆有之。至其内伤外感之辨,谓内伤则短气不足以息,尤为大气下陷之明征。至其方中所用之药, 
又与子之升陷汤相似。何以其方名为补中益气,但治中气之虚陷,而不言升补大气乎?答曰∶大气之名,虽见 
于《内经》,然《素问》中所言之大气,乃指外感之邪气而言,非胸中之大气也。至《灵枢》所言,虽系胸中 
大气,而从来读《内经》者,恒目《灵枢》为针经而不甚注意。即王氏注《内经》,亦但注《素问》而不注 
《灵枢》。后人为其不易索解,则更废而不读。至仲景《伤寒》、《金匮》两书,惟《金匮》水气门,有“大 
气一转,其气乃散”之语。他如《难经》、《千金》、《外台》诸书,并未言及大气。是以东垣于大气下陷证, 
亦多误认为中气下陷,故方中用白术以健补脾胃,而后来之调补脾胃者,皆以东垣为法。夫中气诚有下陷之时, 
然不若大气下陷之尤属危险也。间有因中气下陷,泄泻日久,或转致大气下陷者,可仿补中益气汤之意,于拙 
拟升陷汤中,去知母加白术数钱。若但大气下陷,而中气不下陷者,白术亦可不用,恐其气分或有郁结,而 
、术并用,易生胀满也。 
按∶补中益气汤所治之喘证,即大气下陷者之努力呼吸也。若果系真喘,桔梗尚不宜用,况升麻乎?愚少 
时观东垣书,至此心尝疑之,后明大气下陷之理,始觉豁然,而究嫌其立言欠妥。设医者,真以为补中益气汤 
果能治喘,而于气机上逆之真喘亦用之,岂不足偾事乎?此有关性命之处,临证者当审辨之。 
附录∶ 
直隶青县张××来函∶ 
湖朝鲜××妻,年六旬,素多肝郁,浸至胸中大气下陷。其气短不足以息,因而努力呼吸,有似乎喘;喉干 
作渴;心中满闷怔忡;其脉甚沉微。知其胸中大气下陷过甚,肺中呼吸几有将停之 
势,非投以升陷汤,以升补其大气不可。为录出原方,遵注大气;陷之甚者将升麻加倍服。一剂后,吐出粘涎 
数碗,胸中顿觉舒畅。又于方中加半夏、陈皮,连服三剂,病遂霍然。盖此证因大气下陷,其胸肺胃脘无大气 
以斡旋之,约皆积有痰涎,迨服药后,大气来复,故能运转痰涎外出,此《金匮》水气门所谓“大气一转,其 
气(水气即痰涎)乃散”也。后大气下陷证数见不鲜,莫不用升陷汤加减治愈。 
河北省沦县张××来函∶ 
族婶母,年四十余岁,身体素弱。因境遇不顺,又多抑郁。癸亥十月下旬,忽患头疼甚剧,已三日矣, 
族叔来舍,俾生往诊。及至,闻呻吟不已,卧床不起,言已针过百会及太阳两处,均未见效。其左脉微细如丝, 
按之即无,右脉亦无力,自言气息不接,胸闷不畅,不思饮食,自觉精神恍惚,似难支持,知其胸中之大气下 
陷也。其头疼者,因大气陷后,有他经之逆气乘虚上干也。遵用升陷汤原方,升提其下陷之大气,连服数剂全愈。 
四川泾南周××来函∶ 
程姓男孩,年五岁,乳哺不足,脱肛近四载,医不能治。其面白神疲,身体孱弱;大肠坠出二寸许,用 
手塞入,旋又坠出;其脉濡弱无力;呼吸促短,状若不能接续。知其胸中大气下陷,下焦之气化因之不能固摄 
也。仿用升陷汤方,用生箭 四钱,知母二钱,桔梗、柴胡、升麻各一钱,潞参、净萸肉各三钱煎汤一盅,分 
两次温饮下。连服二剂,肛即收缩。乃减去升麻,再服三剂,全愈。 
直隶唐山张××来函∶ 
××之内,以其夫病势沉重,深恐难起,忧虑成疾。心内动悸,痞塞短气。医者以为痰郁,用二陈汤加 
减清之,病益加剧。因鉴其父为药所误(其父因下痢十余日,医用大黄四钱降之复杯而卒),遂停药不敢服。 
此际愚正在城中为其夫调治余病。俟愚来家求诊,见其满面油光,两手尺寸之脉皆极沉,惟关脉坚而有力。愚 
曰∶“此乃胸中大气下陷,何医者不明如是,而用清痰之二陈也。今两关脉之坚弦,乃彼用药推荡之力。”诊 
际,大气一陷,遂全身一战,冷汗满额,心即连次跳动十余次。遂用升陷汤,再仿逍遥散、炙甘草汤之意,提 
其下陷之气,散其中宫之滞,并以交其心肾。一剂而三部平,大气固。嗣因尺中太微,而理气药及升柴等药皆 
不敢用,遂按治大气下陷方之意及治虚劳之法,精心消息,调治而愈。 
安徽绩溪章××来函∶ 
有冯××,务农而家小康。其母章氏,年正八秩,体丰善饭。一日忽觉左手麻痹,渐至不能持碗。越朝方食 
面饼,倏然僵厥,坐向下堕,肢冷额汗,气息仅属。人皆以为猝中也,聚商救治,自午至晡,逐见危殆,来请 
吾为筹挽救简方,以老人素不服药,且口噤鼻塞,恐药汁亦难下咽耳。吾意谓年老久厥,讵能回阳?姑嘱以红 
灵丹少许吹鼻中,倘嚏气能宣通,再议用药。乃药甫入而嚏作,似渐苏醒。然呼吸甚微,如一线游丝,恐风吹 
断。先按口鼻,温度甚低,音在喉中,犹言誓不服药。诊其脉,则沉微,察其瞳,亦涣散。遂确定为大气下陷。 
但值耄年,势难遽投重峻之剂,爰照升陷汤方而小其剂,用生箭 一钱五分,知母八分,净萸肉一钱,柴胡四 
分,升麻三分。煎服须臾,即渐有转机。续进两剂,逐次平复。继俾服潞党参,每日二钱,加五味子五粒,广 
陈皮少许,频饮代茶。今春见之,较未病前更倍康强矣。 
山东德州卢××来函∶ 
五家嫂及内子两人,系因家务心力煎劳,自觉无日不病者。五家嫂怔忡异常,每犯此病,必数日不能起床, 
须人重按其心,终日面目虚浮,无病不有。而内子则不但怔忡,寒热往来,少腹重坠,自汗、盗汗,亦无定时, 
面目手足及右腿无日不肿。而两人丸药 
日不离口,不但无效,更渐加剧。后侄查大气下陷一切方案,确知两人皆系大气下陷无疑。服升陷汤数剂,并 
加滋补之味,而各病若失,现今均健壮如常矣。 
直隶盐山孙××来函∶ 
一九二六年冬,河东友人翟××之母乳部生疮,疼痛难忍,同事王××约往诊视。翟××言,昨日请医诊 
治,服药一剂,亦不觉如何,惟言誓不再服彼医方药。生诊视时,其脉左关弦硬,右寸独微弱,口不能言,气 
息甚微,病势已危险万分。生断为年高因病疮大气下陷,为开升陷汤,以升举其气,又加连翘、丹参诸药,以 
理其疮。一剂能言,病患喜甚,非按原方再服一剂不可。后生又诊数次,即方略为加减,数服全愈。后遇此证 
数次,亦皆用升陷汤加减治愈。 
奉天桓仁县阎××来函∶ 
客岁家慈得大气下陷证,吾以向未行医,未敢率尔用药,遂聘本县名流再三延医,终无效验。迟至今岁正 
月初二日,气息奄奄,迫不及待,遂急用升陷汤,遵方后所注更番增减,按证投药,数月沉 ,数日全愈。 
江苏平台王××来函∶ 
一妇人,产后乳上生痈,肿疼殊甚。延西医治不效,继延吾延医。其脓已成,用针刺之,出脓甚多,第二 
日已眠食俱安矣。至第三日,忽神昏不食,并头疼。其母曰∶“此昨日受风寒以致如此。”诊其脉,微细若无, 

<篇名>2.回阳升陷汤
属性:治心肺阳虚,大气又下陷者。其人心冷、背紧、恶寒,常觉短气。 
生黄 (八钱) 干姜(六钱) 当归身(四钱) 桂枝尖(三钱) 甘草(一钱) 
周身之热力,借心肺之阳,为之宣通,心肺之阳,尤赖胸中大气,为之保护。大气一陷,则心肺阳分素虚 
者,至此而益虚,欲助心肺之阳,不知升下陷之大气,虽日服热药无功也。 
或问∶心脏属火,西人亦谓周身热力皆发于心,其能宣通周身之热宜矣。今论周身热力不足,何以谓心肺 
之阳皆虚?答曰∶肺与心同居膈上,左心房之血脉管,右心房之回血管,皆与肺循环相通,二脏之宣通热力, 
原有相助为理之妙。然必有大气以斡旋之,其功用始彰耳。 
按∶喻嘉言《医门法律》最推重心肺之阳,谓心肺阳旺,则阴分之火自然潜伏。至陈修园推展其说,谓心 
肺之阳下济,大能温暖脾胃消化痰饮。皆确论也。 
一童子,年十三四,心身俱觉寒凉,饮食不化,常常短气,无论服何热药,皆分毫不觉热。其脉微弱而迟, 
右部兼沉。知其心肺阳分虚损,大气又下陷也。为制此汤,服五剂,短气已愈,身心亦不若从前之寒凉。遂减 
桂枝之半,又服数剂全愈。俾停药,日服生硫黄分许,以善其后。 
一人,年五十余。大怒之后,下痢月余始愈。自此胸中常觉满闷,饮食不能消化。数次延医服药,不外通 
利气分之品,即间有温补脾胃者,亦必杂以破气之药,愈服病愈增重。后愚诊视,其脉沉细微弱,至数甚迟。 
询其心中,常有觉凉之时。知其胸中大气下陷,兼上焦阳分虚损也。遂投以此汤,十剂全愈。后因怒,病又反 
复,医者即愚方,加浓朴二钱,服后少腹下坠作疼,彻夜不能寐,复求为延医,仍投以原方而愈。 
赵姓媪,年近五旬,忽然昏倒不语,呼吸之气大有滞碍,几不能息,其脉微弱而迟。询其生平,身体羸弱, 
甚畏寒凉,恒觉胸 
中满闷,且时常短气。即其素日资禀及现时病状以互戡病情,其为大气下陷兼寒饮结胸无疑。然此时情势将成 
痰厥,住在乡村取药无及,遂急用胡椒二钱捣碎煎两三沸,澄取清汤灌下。须臾胸中作响,呼吸顿形顺利。继用 
干姜八钱煎汤一盅,此时已自能饮下。须臾气息益顺,精神亦略清爽,而仍不能言,且时作呵欠,十余呼吸之顷 
必发太息,知其寒饮虽开,大气之陷者犹未复也。遂投以拙拟回阳升陷汤。服数剂,呵欠与太息皆愈,渐能言语。 
按∶此证初次单用干姜开其寒饮,而不敢佐以赭朴诸药以降下之者,以其寒饮结胸又兼大气下陷也。设 
若辨证不清而误用之,必至凶危立见,此审证之当细心也。 


<篇名>3.理郁升陷汤
属性:治胸中大气下陷,又兼气分郁结,经络湮淤者。 
生黄 (六钱) 知母(三钱) 当归身(三钱) 桂枝尖(钱半) 柴胡(钱半) 乳香(三钱,不 
去油) 没药(三钱,不去油) 
胁下撑胀,或兼疼者,加龙骨、牡蛎(皆不用 )各五钱;少腹下坠者,加升麻一钱。 
一妇人,年三十许。胸中满闷,时或作疼,鼻息发热,常常作渴。自言得之产后数日,劳力过度。其脉 
迟而无力,筹思再三,莫得病之端绪。姑以生山药一两,滋其津液,鸡内金二钱、陈皮一钱,理其疼闷,服后 
忽发寒热。再诊其脉,无力更甚,知其气分郁结,又下陷也。遂为制此汤,一剂诸病皆觉轻,又服四剂全愈。 
一少女,年十五。脐下左边起一 瘕,沉沉下坠作疼,上连腰际,亦下坠作疼楚,时发呻吟。剧时,常 
觉小便不通,而非不通也。诊其脉,细小而沉。询其得病之由,言因小便不利,便时 
努力过甚,其初腰际坠疼,后遂结此 瘕。其方结时,揉之犹软,今已五阅月,其患处愈坚结。每日晚四点钟, 
疼即增重,至早四点钟,又渐觉轻。愚闻此病因,再以脉象参之,知其小便时努力过甚。上焦之气,陷至下焦 
而郁结也。遂治以理郁升陷汤,方中乳香、没药皆改用四钱,又加丹参三钱、升麻钱半,二剂而坠与疼皆愈。 
遂去升麻,用药汁送服朱血竭末钱许,连服数剂, 瘕亦消。 
或问∶龙骨、牡蛎为收涩之品,兼胁下胀疼者,何以加此二药?答曰∶胁为肝之部位,胁下胀疼者,肝 
气之横恣也,原当用泻肝之药,又恐与大气下陷者不宜。用龙骨、牡蛎,以敛戢肝火,肝气自不至横恣,此敛 
之即以泻之,古人治肝之妙术也。且黄 有膨胀之力,胀疼者原不宜用,有龙骨、牡蛎之收敛,以缩其膨胀之 
力,可放胆用之无碍,此又从体验而知道也。尝治一少妇,经水两月不见,寒热往来,胁下作疼,脉甚微弱而 
数至六至。询之常常短气,投以理郁升陷汤,加龙骨、牡蛎各五钱,为脉数,又加玄参、生地、白芍各数钱, 
连服四剂。觉胁下开通,瘀血下行,色紫黑,自此经水调顺,诸病皆愈。盖龙骨、牡蛎性虽收涩,而实有开通 
之力,《神农本草经》谓龙骨消 瘕,而又有牡蛎之咸能软坚者以辅之,所以有此捷效也。 


<篇名>4.醒脾升陷汤
属性:治脾气虚极下陷,小便不禁。 
生箭 (四钱) 白术(四钱) 桑寄生(三钱) 川续断(三钱) 萸肉(四钱,去净核) 龙骨 
(四钱, 捣) 牡蛎(四钱, 捣) 川萆 (二钱) 甘草(二钱,蜜炙) 
《内经》曰∶“饮入于胃,游溢精气,上输于脾,脾气散精,上归于肺,通调水道,下输膀胱。”是 
脾也者,原位居中焦,为水 
饮上达下输之枢机,枢机不旺,则不待上达而即下输,此小便之所以不禁也。然水饮降下之路不一,《内经》 
又谓“肝热病者,小便先黄”,又谓“肝壅两 (胁也)满,卧则惊悸,不得小便。”且芍药为理肝之主药, 
而善利小便。由斯观之,是水饮又由胃入肝,而下达膀胱也。至胃中所余水饮,传至小肠渗出,此又人所共知。 
故方中用黄 、白术、甘草以升补脾气,即用黄 同寄生、续断以升补肝气,更用龙骨、牡蛎、萸肉、萆 以 
固涩小肠也。又人之胸中大气旺,自能吸摄全身气化,不使下陷,黄 与寄生并用,又为填补大气之要药也。 
或问∶黄 为补肺脾之药,今谓其能补肝气何也?答曰∶肝属木而应春令,其气温而性喜条达,黄 性 
温而升,以之补肝,原有同气相求之妙用。愚自临证以来,凡遇肝气虚弱,不能条达,一切补肝之药不效者, 
重用黄 为主,而少佐以理气之品,服之,复杯之顷,即见效验(治验详黄 解)。是知谓肝虚无补法者,非 
见道之言也。 
或问∶《神农本草经》谓桑寄生能治腰疼、坚齿发、长须眉,是当为补肝肾之药,而谓其能补胸中大气 
何也?答曰∶寄生根不着土,寄生树上,最善吸空中之气,以自滋生,故其所含之气化,实与胸中大气为同类。 
尝见有以补肝肾,而多服久服,胸中恒觉满闷,无他,因其胸中大气不虚,故不受寄生之补也。且《神农本草 
经》不又谓其治痈肿乎?然痈肿初起,服之必无效,惟痈肿溃后,生肌不速,则用之甚效。如此而言,又与黄 
之主痈疽败证者相同,则其性近黄 ,更可知矣。 
或问∶萆 世医多用以治淋,夫淋以通利为主,盖取萆 能利小便也。此方中用之以固小便,其性果固 
小便乎,抑利小便乎?答曰∶萆 为固涩下焦之要药,其能治失溺,《名医别录》原有明文。时医因古方有萆 
解厘清饮,遂误认萆 为利小便之要 
药,而于小便不利、淋涩诸证多用之。尝见有以利小便,而小便转癃闭者,以治淋证,竟致小便滴沥不通者, 
其误人可胜道哉!盖萆 厘清饮之君萆 ,原治小便频数,溺出旋白如油,乃下焦虚寒,气化不固之证,观其 
佐以缩小便之益智,温下焦之乌药,其用意可知。特当日命名时,少欠斟酌,遂致庸俗医辈,错有会心,贻害 
无穷,可不慎哉! 

\chapter{治喘息方}
<篇名>1.参赭镇气汤
属性:治阴阳两虚,喘逆迫促,有将脱之势。亦治肾虚不摄,冲气上干,致胃气不降作满闷。 
野台参(四钱) 生赭石(六钱,轧细) 生芡实(五钱) 生山药(五钱) 萸肉(六钱,去净核) 
生龙骨(六钱,捣细) 生牡蛎(六钱,捣细) 生杭芍(四钱) 苏子(二钱,炒捣) 
生赭石压力最胜,能镇胃气冲气上逆,开胸膈,坠痰涎,止呕吐,通燥结,用之得当,诚有捷效。虚者可 
与人参同用。 
仲景旋复代赭石汤,赭石、人参并用。治“伤寒发汗,若吐若下解后,心下痞硬,噫气不除者”。参、赭 
镇气汤中人参,借赭石下行之力,挽回将脱之元气,以镇安奠定之,亦旋复代赭石汤之义也。 
一妇人,年三十余,劳心之后兼以伤心,忽喘逆大作,迫促异常。其翁知医,以补敛元气之药治之,觉胸 
中窒碍不能容受。更他医以为外感,投以小剂青龙汤喘益甚。延愚诊视,其脉浮而微数,按之即无,知为阴阳 
两虚之证。盖阳虚则元气不能自摄,阴虚而肝肾又不能纳气,故作喘也。为制此汤,病患服药后,未 
及复杯曰∶“吾有命矣。”询之曰∶“从前呼吸惟在喉间,几欲脱去,今则转落丹田矣。”果一剂病愈强半, 
又服数剂全愈。 
一妇人,年二十余,因与其夫反目,怒吞鸦片。已经救愈,忽发喘逆,迫促异常。须臾又呼吸顿停,气息 
全无,约十余呼吸之顷,手足乱动,似有蓄极之势,而喘复如故。若是循环不已,势近垂危,延医数人,皆不 
知为何病。后愚诊视其脉,左关弦硬,右寸无力,精思良久,恍然悟曰∶此必怒激肝胆之火,上冲胃气。夫胃 
气本下行者也,因肝胆之火冲之,转而上逆,并迫肺气亦上逆,此喘逆迫促所由来也。逆气上干,填塞胸膈, 
排挤胸中大气,使之下陷。夫肺悬胸中,须臾无大气包举之,即须臾不能呼吸,此呼吸顿停所由来也(此理参 
观升陷汤后跋语方明)。迨大气蓄极而通,仍上达胸膈,鼓动肺脏,使得呼吸、逆气遂仍得施其击撞,此又病 
势之所以循环也。《神农本草经》载,桂枝主上气咳逆、结气、喉痹、吐吸(吸不归根即吐出),其能降逆气 
可知。其性温而条达,能降逆气,又能升大气可知。遂单用桂枝尖三钱,煎汤饮下,须臾气息调和如常。夫以 
桂枝一物之微,而升陷降逆,两擅其功,以挽回人命于顷刻,诚天之生斯使独也。然非亲自经验者,又孰信其 
神妙如是哉!继用参赭镇气汤,去山药、苏子,加桂枝尖三钱、知母四钱,连服数剂,病不再发。此喘证之特 
异者,故附记于此。 
附录∶ 
直隶青县张××来函∶ 
定县吴××妻病,服药罔效。弟诊其脉,浮而无力。胸次郁结,如有物杜塞,饮食至胃间,恒觉烧热不下。 
仿参赭镇气汤之义,用野台参六钱,赭石细末二两。将二药煎服,胸次即觉开通。服至二剂,饮食下行无碍。 
因其大便犹燥,再用当归、肉苁蓉各四钱,俾煎服。病若失。 
安徽绩溪章××来函∶ 
洪××,年五十余,家素贫苦,曾吸鸦片,戒未多年,由咳而成喘疾,勉强操劳,每届冬令则加剧,然病发 
时亦往往不服药而自愈。兹次发喘,初由外感,兼发热头痛。医者投以二活、防、葛,大剂表散,遂汗出二日 
不止,喘逆上冲,不能平卧,胸痞腹胀,大便旬余未行,语不接气,时或螈 ,种种见证,已濒极险。诊其脉, 
微细不起。形状颓败殊甚。详细勘视,诚将有阴阳脱离之虞。适日前阅赭石解,记其主治,揣之颇合。但恐其 
性太重镇而正气将随以下陷也,再四踌躇,因配以真潞党参、生怀山药、野茯神、净萸肉、广桔红、京半夏、 
龙骨、牡蛎、苏子、蒡子等,皆属按证而拟,竟与参赭镇气汤大致相同。一剂病愈大半,两剂即扶杖起行,三 
剂则康复如恒矣。前月遇之,自言冬不知寒,至春亦未反复。 


<篇名>2.薯蓣纳气汤
属性:治阴虚不纳气作喘逆。 
生山药(一两) 大熟地(五钱) 萸肉(五钱,去净核) 柿霜饼(四钱,冲服) 生杭芍(四钱) 
牛蒡子(二钱,炒捣) 苏子(二钱,炒捣) 甘草(二钱,蜜炙) 生龙骨(五钱,捣细) 
前方,治阴阳两虚作喘,此方,乃专治阴虚作喘者也。方书谓肝肾虚者,其人即不能纳气,此言亦近理, 
然须细为剖析。空气中有氧气,乃养物之生气也。人之肺脏下无透窍,而吸入之养气,实能隔肺胞,息息通过, 
以下达腹中,充养周身。肝肾居于腹中,其气化收敛,不至膨胀,自能容纳下达之气,且能导引使之归根。有 
时肾虚气化不摄,则上注其气于冲,以冲下连肾也。夫冲为血海,实亦主气,今因为肾气贯注,则冲气又必上逆于 
胃,以冲上连胃也。由是,冲气兼挟胃气上逆,并迫肺气亦上逆矣,此喘之所由来也。又《内经》谓肝主疏, 
泄肾主闭藏。夫肝之疏泄,原以济肾之闭藏,故二便之通行,相火之萌动,皆与肝气有关,方书所以有肝行肾 
气之说。今因肾失其闭藏之性,肝遂不能疏泄肾气使之下行,更迫于肾气之膨胀,转而上逆。由斯,其逆气可 
由肝系直透膈上,亦能迫肺气上逆矣,此又喘之所由来也。方中用地黄、山药以补肾,萸肉、龙骨补肝即以敛 
肾,芍药、甘草甘苦化阴,合之柿霜之凉润多液,均为养阴之妙品,苏子、牛蒡又能清痰降逆,使逆气转而下 
行,即能引药力速于下达也。至方名薯蓣纳气汤者,因山药补肾兼能补肺,且饶有收敛之力,其治喘之功最弘也。 
或问∶养气虽能隔肺胞通过,亦甚属些些无多,何以当吸气内入之时,全腹皆有膨胀之势?答曰∶若明此 
理,益知所以致喘之由。人之脏腑皆赖气以撑悬,是以膈上有大气,司肺呼吸者也;膈下有中气,保合脾胃者 
也,脐下有元气,固性命之根蒂者也。当吸气入肺之时,肺胞膨胀之力,能鼓舞诸气,节节运动下移,而周身 
之气化遂因之而流通。且喉管之分支下连心肝,以通于奇经诸脉,当吸气内入之时,所吸之气原可由喉管之分 
支下达,以与肺中所吸之气,相助为理也。下焦肝肾(奇经与肾相维系)属阴,阴虚气化不摄则内气膨胀,遂 
致吸入之气不能容受而急于呼出,此阴虚者所以不纳气而作喘也。 


<篇名>3.滋培汤
属性:治虚劳喘逆,饮食减少,或兼咳嗽,并治一切阴虚羸弱诸证。 
生山药(一两) 于术(三钱,炒) 广陈皮(二钱) 牛蒡子(二钱,炒捣) 生杭芍(三钱) 玄参 
(三钱) 生赭石(三钱,轧细) 炙甘草(二钱) 
痰郁肺窍则作喘,肾虚不纳气亦作喘。是以论喘者恒责之肺、肾二脏,未有责之于脾、胃者。不知胃气宜 
息息下行,有时不下行而转上逆,并迫肺气亦上逆即可作喘。脾体中空,能容纳诸回血管之血,运化中焦之气, 
以为气血宽闲之地,有时失其中空之体,或变为紧缩,或变为胀大,以致壅激气血上逆迫肺,亦可作喘。且脾 
脉缓大,为太阴湿土之正象,虚劳喘嗽者,脉多弦数,与缓大之脉反对,乃脾土之病脉也。故重用山药以滋脾 
之阴,佐以于术以理脾之阳,脾脏之阴阳调和,自无或紧缩、或涨大之虞。特是,脾与胃脏腑相根据,凡补脾之 
药皆能补胃。而究之脏腑异用,脾以健运磨积,宣通津液为主;胃以熟腐水谷、传送糟粕为主。若但服补药, 
壅滞其传送下行之机,胃气或易于上逆,故又宜以降胃之药佐之,方中之赭石、陈皮、牛蒡是也。且此数药之 
性,皆能清痰涎利肺气,与山药、玄参并用,又为养肺止嗽之要品也。用甘草、白芍者,取其甘苦化合,大有 
益于脾胃,兼能滋补阴分也。并治一切虚劳诸证者,诚以脾胃健壮,饮食增多,自能运化精微以培养气血也。 
一人,年二十二,喘逆甚剧,脉数至七至,用一切治喘药皆不效,为制此方。将药煎成,因喘剧不能服, 
温汤三次始服下,一剂见轻,又服数剂全愈。 
或问∶药之健脾胃者,多不能滋阴分,能滋阴分者,多不能健脾胃,此方中芍药、甘草同用,何以谓能兼 
此二长?答曰∶《神农本草经》谓芍药味苦,后世本草谓芍药味酸。究之,芍药之味苦酸皆有。陈修园笃信 
《神农本草经》谓芍药但苦不酸。然嚼服芍药钱许,恒至 齿,兼有酸味可知。若取其苦味与甘草相合,有甘 
苦化阴之妙(甘苦化阴说始于叶天士),故能滋阴分。若取其酸味与甘草相合,有甲己化土之妙(甲木味酸己 
土味甘),故能益脾胃。此皆取其化出之性以为用也。又陈修园曰∶芍药苦平破滞,本泻药非补药也。 
若与甘草同用,则为滋阴之品,与生姜、大枣、桂枝同用,则为和营卫之品,与附子、干姜同用,则能收敛元 
阳,归根于阴,又为补肾之品。本非补药,昔贤往往取为补药之主,其旨微矣。按此论甚精,能示人用药变化 
之妙,故连类及之。 
西人谓∶心有病可以累肺作喘,此说诚信而有征。盖喘者之脉多数,夫脉之原动力发于心,脉动数则心 
动亦数可知。心左房之赤血与右房之紫血,皆与肺循环相通。若心动太急,逼血之力过于常度,则肺脏呼吸亦 
因之速过常度,此自然之理也。然心与肾为对待之体,心动若是之急数,肾之真阴不能上潮,以靖安心阳可知。 
由是言之,心累肺作喘之证,亦即肾虚不纳气之证也。 
西人又谓∶喘证因肺中小气管,痰结塞住,忽然收缩,气不通行,呼吸短促,得痰出乃减。有日日发作 
者,又数日或因辛苦寒冷而发作者,又有因父母患此病传延者。发作时,苦剧不安,医治无良法。应用纸浸火 
硝水内,取出晒干,置盆内燃点,乘烟焰熏腾时,以口吸养气入肺(火硝多含养气)。或用醉仙桃干叶当烟吸 
之,内服樟脑鸦片酒壹贰钱、更加姜末一分半、白矾七厘共为散,水调服。虽未必能除根,亦可渐轻。按∶此 
证乃劳疾之伤肺者,当名为肺劳。虽发作时甚剧,仍可久延岁月。其治法当用拙拟黄 膏。 
肺劳之证,因肺中分支细管多有瘀滞,热时肺胞松客气化犹可宣通,故病则觉轻;冷时肺胞紧缩其痰涎 
恒益杜塞,故病则加重。此乃肺部之锢疾,自古无必效之方。惟用曼陀罗熬膏,和以理肺诸药,则多能治愈。 
爰将其方详开于下∶ 
曼陀罗正开花时,将其全科切碎,榨取原汁四两,入锅内熬至若稠米汤;再加入硼砂二两,熬至融化; 
再用远志细末、甘草细末各四两,生石膏细末六两,以所熬之膏和之,以适可为丸为度,分作小丸。每服钱半, 
若不效可多至二钱,白汤送下,一日两次。久服病可除根。若服之觉热者,石膏宜加重。 
肺脏具 辟之机。其 辟之机自如,自无肺劳病证。远志、硼砂最善化肺管之瘀,甘草末服,不经火炙、 
水煮,亦善宣通肺中气化,此所以助肺脏之辟也。曼陀罗膏大有收敛之力,此所以助肺脏之 也。用石膏者, 
因曼陀罗之性甚热,石膏能解其热也。且远志、甘草、硼砂皆为养肺之品,能多生津液,融化痰涎,俾肺脏 
辟之机灵活无滞,则肺劳之喘嗽自愈也。 
按∶醉仙桃即曼陀罗花也。其花白色,状类牵牛而大,其叶大如掌而有尖,结实大如核桃,实蒂有托盘如 
钱,皮有芒刺如包麻,中含细粒,如火麻仁。渤海之滨生殖甚多,俗呼为洋金花。李时珍谓∶“服之令人昏昏 
如醉,可作麻药。”又谓∶“熬水洗脱肛甚效。”盖大有收敛之功也。西人药学谓∶用醉仙桃花实叶,俱要鲜 
者榨汁,或熬干,或晒干作膏。每服三厘,能补火止疼、令人熟睡,善疗喘嗽。正与时珍之说相似。然此物有 
毒不可轻用。今人治劳喘者,多有取其花与叶,作烟吸之者,实有目前捷效,较服其膏为妥善也。 

\chapter{治痰饮方}
<篇名>1.理饮汤
属性:治因心肺阳虚,致脾湿不升,胃郁不降,饮食不能运化精微,变为饮邪。停于胃口为满闷,溢于膈上为短 
气,渍满肺窍为喘促,滞腻咽喉为咳吐粘涎。甚或阴霾布满上焦,心肺之阳不能畅舒,转郁而作热。或阴气逼 
阳外出为身热,迫阳气上浮为耳聋。然必诊其脉,确乎弦迟细弱者,方能投以此汤。 
于术(四钱) 干姜(五钱) 桂枝尖(二钱) 炙甘草(二钱) 茯苓片(二钱) 生杭芍(二钱) 
桔红(钱半) 川浓朴(钱半) 
服数剂后,饮虽开通,而气分若不足者,酌加生黄 数钱。 
方中用桂枝、干姜,以助心肺之阳,而宣通之。白术、茯苓、甘草,以理脾胃之湿,而淡渗之(茯苓甘草 
同用最泻湿满)。用浓朴者,叶天士谓∶“浓朴多用则破气,少用则通阳”,欲借温通之性,使胃中阳通气降, 
运水谷速于下行也。用桔红者,助白术、茯苓、甘草以利痰饮也。至白芍,若取其苦平之性,可防热药之上僭 
(平者主降),若取其酸敛之性,可制虚火之浮游(《神农本草经》谓芍药苦平,后世谓芍药酸敛,其味实苦 
而微酸)。且药之热者,宜于脾胃,恐不宜于肝胆,又取其凉润之性,善滋肝胆之阴,即预防肝胆之热也。况 
其善利小便,小便利而痰饮自减乎。 
一妇人,年四十许。胸中常觉满闷发热,或旬日,或浃辰之间,必大喘一两日。医者用清火理气之药,初 
服稍效,久服转增剧。后愚诊视,脉沉细几不可见。病家问∶系何病因?愚曰∶此乃心肺阳虚,不能宣通脾胃, 
以致多生痰饮也。人之脾胃属土,若地舆然。心肺居临其上正当太阳部位(膈上属太阳,观《伤寒论》太阳篇 
自知),其阳气宣通,若日丽中天暖光下照。而胃中所纳水谷,实借其阳气宣通之力,以运化精微而生气血, 
传送渣滓而为二便。清升浊降,痰饮何由而生?惟心肺阳虚,不能如离照当空,脾胃即不能借其宣通之力,以 
运化传送,于是饮食停滞胃口,若大雨之后,阴雾连旬,遍地污淖,不能干渗,则痰饮生矣。痰饮既生,日积 
月累,郁满上焦则作闷,渍满肺窍则作喘,阻遏心肺阳气,不能四布则作热。医者不识病源,犹用凉药清之, 
勿怪其久而增剧也。遂为制此汤,服之一剂,心中热去,数剂后转觉凉甚。遂去白芍,连服二十余剂,胸次豁 
然,喘不再发。 
一妇人,年三十许。身形素丰。胸中痰涎郁结,若碍饮食,上焦时觉烦热。偶服礞石滚痰丸有效,遂日日 
服之。初则饮食加多,继则饮食渐减,后则一日不服,即不能进饮食。又久服之。 
竟分毫无效,日仅一餐,进食少许,犹不能消化。且时觉热气上腾,耳鸣欲聋,始疑药不对证。求愚延医,其 
脉象浮大,按之甚软。愚曰∶此证心肺阳虚,脾胃气弱,为服苦寒攻泻之药太过,故病证脉象如斯也。拟治以 
理饮汤。病家谓,从前医者,少用桂附,即不能容受,恐难再用热药。愚曰∶桂附原非正治心肺脾胃之药,况 
又些些用之,病重药轻,宜其不受。若拙拟理饮汤,与此证针芥相投,服之必无他变。若畏此药,不敢轻服, 
单用干姜五钱,试服亦可。病家根据愚言,煎服干姜后,耳鸣即止,须臾觉胸次开通。继投以理饮汤,服数剂, 
心中亦觉凉甚。将干姜改用一两,又服二十余剂,病遂除根。 
一妇人,年四十许。上焦满闷烦躁,思食凉物,而偶食之,则满闷益甚。且又黎明泄泻,日久不愈,满闷 
益甚,将成臌胀。屡次延医服药,多投以半补半破之剂,或佐以清凉,或佐以收涩,皆分毫无效。后愚诊视,脉 
象弦细而迟。知系寒饮结胸,阻塞气化。欲投以理饮汤,病家闻而迟疑,似不敢服。亦俾先煎干姜数钱服之,胸中 
烦躁顿除。为其黎明泄泻,遂将理饮汤去浓朴、白芍,加生鸡内金钱半,补骨脂三钱,连服十余剂,诸病皆愈。 
一妇人,年近五旬,常觉短气,饮食减少。屡次延医服药,或投以宣通,或投以升散,或投以健补脾胃, 
兼理气之品,皆分毫无效。浸至饮食日减,羸弱不起,奄奄一息,病家亦以为不治之证矣。后闻愚在其邻村, 
屡救危险之证,复延愚诊视。其脉弦细欲无,频吐稀涎。询其心中,言觉有物杜塞胃口,气不上达,知其为寒 
饮凝结也。遂投以理饮汤,方中干姜改用七钱,连服三剂,胃口开通。又觉呼吸无力,遂于方中加生黄 三钱, 
连服十余剂,病全愈。方书谓,饮为水之所结,痰为火之所凝,是谓饮凉而痰热也。究之饮证亦自分凉热,其 
热者,多由于忧思过度, 
甚则或至癫狂,虽有饮而恒不外吐。其凉者,则由于心肺阳虚,如方名下所言种种诸情状。且其证,时吐稀涎, 
常觉短气,饮食廉少,是其明征也(后世谓痰之稀者为饮,稠者为痰,与《金匮》所载四饮名义不同)。 
邑,韩××医学传家,年四十有四,偶得奇疾。卧则常常发搐,旋发旋止,如发寒战之状,一呼吸之间即愈。 
即不发搐时,人偶以手抚之,又辄应手而发。自治不效,广求他医治疗皆不效。留连半载,病势浸增。后愚诊 
视,脉甚弦细。询其饮食甚少,知系心肺脾胃阳分虚惫,不能运化精微,以生气血。血虚不能荣筋,气虚不能 
充体,故发搐也。必发于卧时者,卧则气不顺也。人抚之而辄发者,气虚则畏人按也。授以理饮汤方,数剂, 
饮食加多,搐亦见愈。二十剂后,病不再发。 


<篇名>2.理痰汤
属性:治痰涎郁塞胸膈,满闷短气。或渍于肺中为喘促咳逆。停于心下为惊悸不寐。滞于胃口为胀满哕呃。溢于 
经络为肢体麻木或偏枯。留于关节,着于筋骨,为俯仰不利,牵引作疼。随逆气肝火上升,为眩晕不能坐立。 
生芡实(一两) 清半夏(四钱) 黑芝麻(三钱,炒捣) 柏子仁(二钱,炒捣) 生杭芍(二钱) 
陈皮(二钱) 茯苓片(二钱) 
世医治痰,习用宋《局方》二陈汤,谓为治痰之总剂。不知二陈汤能治痰之标,不能治痰之本。何者?痰 
之标在胃,痰之本原在于肾。肾主闭藏,以膀胱为腑者也。其闭藏之力,有时不固,必注其气于膀胱。膀胱膨 
胀,不能空虚若谷,即不能吸引胃中水饮,速于下行而为小便,此痰之所由来也。又肾之上为血海,奇经之冲 
脉也。其脉上隶阳明,下连少阴。为其下连少阴也,故肾中气化不摄,则冲气易于上干。为其上隶阳明也,冲 
气上干,胃 
气亦多上逆,不能息息下行以运化水饮,此又痰之所由来也。此方以半夏为君,以降冲胃之逆。即重用芡实, 
以收敛冲气,更以收敛肾气,而浓其闭藏之力。肾之气化治,膀胱与冲之气化,自无不治,痰之本原清矣。用 
芝麻、柏实者,润半夏之燥,兼能助芡实补肾也。用芍药、茯苓者,一滋阴以利小便,一淡渗以利小便也。用 
陈皮者,非借其化痰之力,实借其行气之力,佐半夏以降逆气,并以行芡实、芝麻、柏实之滞腻也。 
友人毛××,曾治一妇人,年四十余。上盛下虚,痰涎壅滞,饮食减少,动则作喘。他医用二陈汤加减治之, 
三年,病转增剧。后延毛××诊视,投以此汤,数剂病愈强半。又将芡实减去四钱,加生山药五钱,连服二十余 
剂,痰尽消,诸病皆愈。至今数年,未尝反复。 
毛××又尝治一少妇,患痫风。初两三月一发,浸至两三日一发。脉滑、体丰,知系痰涎为恙。亦治以此 
汤,加赭石三钱,数剂竟能拔除病根。后与愚觌面述之。愚喜曰∶向拟此汤时,原不知能治痫风,经兄加赭石 
一味,即建此奇功,大为此方生色矣。 
按∶此方若治痫风,或加朱砂,或加生铁落,或用磨刀水煎药,皆可。 


<篇名>3.龙理痰汤
属性:治因思虑生痰,因痰生热,神志不宁。 
清半夏(四钱) 生龙骨(六钱,捣细) 生牡蛎(六钱,捣细) 生赭石(三钱,轧细) 朴硝(二钱) 
黑芝麻(三钱,炒捣) 柏子仁(三钱,炒捣) 生杭芍(三钱) 陈皮(二钱) 茯苓(二钱) 
此方,即理痰汤,以龙骨、牡蛎代芡实,又加赭石、朴硝也。其所以如此加减者,因此方所主之痰,乃虚 
而兼实之痰。实 
痰宜开,礞石滚痰丸之用硝黄者是也;虚痰宜补,肾虚泛作痰,当用肾气丸以逐之者是也。至虚而兼实之痰, 
则必一药之中,能开痰亦能补虚,其药乃为对证,若此方之龙骨、牡蛎是也。盖人之心肾,原相助为理。肾虚 
则水精不能上输以镇心,而心易生热,是由肾而病及心也;心因思虑过度生热,必暗吸肾之真阴以自救,则肾 
易亏耗,是由心而病及肾也。于是心肾交病,思虑愈多,热炽液凝,痰涎壅滞矣。惟龙骨、牡蛎能宁心固肾, 
安神清热,而二药并用,陈修园又称为治痰之神品,诚为见道之言。故方中用之以代芡实,而犹恐痰涎过盛, 
消之不能尽消,故又加赭石、朴硝以引之下行也。 
一人,年三十余。常觉胆怯,有时心口或少腹 动后,须臾觉有气起自下焦,上冲胸臆,郁而不伸,连作 
呃逆,脖项发热,即癫狂唱呼。其夹咽两旁内,突起若瘰 ,而不若瘰 之硬。且精气不固,不寐而遗,上焦 
觉热,下焦觉凉。其脉左部平和,微嫌无力,右部直上直下(李士材脉诀云直上直下冲脉昭昭),仿佛有力, 
而按之非真有力。从前屡次医治皆无效。此肾虚,致冲气挟痰上冲,乱其心之神明也。投以此汤,减浓朴之半, 
加山萸肉(去净核)五钱,数剂诸病皆愈,惟觉短气。知系胸中大气下陷(理详升陷汤下),投以拙拟升陷汤, 
去升麻、柴胡,加桂枝尖二钱,两剂而愈。盖此证,从前原有逆气上干,升麻、柴胡能升大气,恐兼升逆气。 
桂枝则升大气,兼降逆气,故以之代升、柴也。 
一媪,年六十二,资禀素羸弱。偶当外感之余,忽然妄言妄见,惊惧异常,手足扰动,饥渴不敢饮食,少 
腹塌陷,胸膈突起。脉大于平时一倍,重按无力。知系肝肾大虚,冲气上逆,痰火上并,心神扰乱也。投以此 
汤,去朴硝,倍赭石,加生山药、山萸肉(去净核)、生地黄各六钱。又磨取铁锈水煎药(理详一味铁养汤下), 
一剂即愈。又服一剂,以善其后。 


<篇名>4.健脾化痰丸
属性:治脾胃虚弱,不能运化饮食,以至生痰。 
生白术(二两) 生鸡内金(二两,去净瓦石糟粕) 
上药二味,各自轧细过罗,各自用慢火焙熟(不可焙过),炼蜜为丸梧桐子大。每服三钱,开水送下。白 
术为健补脾胃之主药,然土性壅滞,故白术多服久服,亦有壅滞之弊;有鸡内金之善消瘀积者以佐之,则补益 
与宣通并用。俾中焦气化,壮旺流通,精液四布,清升浊降,痰之根柢蠲除矣。又此方不但治痰甚效,凡廉于 
饮食者,服之莫不饮食增多。且久服之,并可消融腹中一切积聚。 
初拟此方时,原和水为丸。而久服者间有咽干及大便燥结之时。后改用蜜丸,遂无斯弊。 


<篇名>5.期颐饼
属性:治老人气虚,不能行痰,致痰气郁结,胸次满闷,胁下作疼。凡气虚痰盛之人,服之皆效,兼治疝气。 
生芡实(六两) 生鸡内金(三两) 白面(半斤) 白沙糖(不拘多少) 
先将芡实用水淘去浮皮,晒干,轧细,过罗。再将鸡内金(中有瓦石糟粕去净分量还足)轧细,过罗, 
置盆内浸以滚水,半日许。再入芡实、白糖、白面,用所浸原水,和作极薄小饼,烙成焦黄色,随意食之。 
鸡内金,以补助脾胃,大能运化饮食,消磨瘀积。食化积消,痰涎自除。再者,老人痰涎壅盛,多是下焦 
虚惫,气化不摄,痰涎随冲气上泛。芡实大能敛冲固气,统摄下焦气化。且与麦面同用,一补心,一补肾,使 
心肾相济,水火调和,而痰气自平矣。 
或问∶老人之痰,既由于气虚不行,何不加以补助气分之品?答曰∶凡补气之药,久服转有他弊。此方所 
用药品,二谷食,一肉食,复以沙糖调之,可作寻常服食之物,与他药饵不同。且食之,能令人饮食增多, 
则气虚者自实也。 
此方去芡实,治小儿疳积痞胀,大人 瘕积聚。 


<篇名>6.治痰点天突穴法
属性:(附∶捏结喉法、明矾汤、麝香香油灌法) 
点天突穴以治痰厥,善针灸者,大抵知之。而愚临证体验,尤曲尽点法之妙。穴在结喉(颈间高骨)下宛 
宛中。点时屈手大指(指甲长须剪之)以指甲贴喉,指端着穴,直向下用力(勿斜向里),其气即通。指端, 
当一起一点,令痰活动,兼频频挠动其指端,令喉痒作嗽,其痰即出。 
一妇人,年二十许。数日之前,觉胸中不舒,一日忽然昏昏似睡,半日不醒。适愚自他处归,过其村。 
病家见愚喜甚,急求延医。其脉沉迟,兼有闭塞之象。唇 动。凡唇动者,为有痰之征。脉象,当系寒痰壅滞 
上焦过甚。遂令人扶之坐,以大指点其天突穴,俾其喉痒作嗽。约点半点钟,咳嗽十余次,吐出凉痰一碗,始 
能言语。又用干姜六钱,煎汤饮下而愈。 
岁在甲寅,客居大名之金滩镇。时当孟春,天寒,雨且雪,一兵士衣装尽湿,因冻甚,不能行步,其伙舁 
之至镇,昏不知人。呼之不应,用火烘之,且置于温暖之处,经宿未醒。闻愚在镇,曾用点天突穴法,治愈一 
人,求为延医。见其僵卧不动,呼吸全无。按其脉,仿佛若动。以手掩其口鼻,每至呼吸之顷,微觉有热,知犹 
可救。遂令人扶起俾坐,冶以点天突穴之法,兼捏其结喉。约两点钟,咳嗽二十余次,共吐凉痰碗半,始能呻 
吟。亦饮以干姜而愈。 
【捏结喉法】得之沧州友人张××,其令人喉痒作嗽之力尤速。欲习其法者,可先自捏其结喉,如 
何捏法即可作嗽,则得其法矣。然当气塞不通时。以手点其天突穴,其气即通。捏结喉,必痒嗽吐痰后,其气 
乃通。故二法宜相辅并用也。 
按∶西人谓,冻死者若近火,则寒气内迫,难救。宜置寒冷室中,或树阴无风处,将衣服脱除,用雪团 
或冷水,周身摩擦;或将身置冷水中,周身摩擦。及四肢渐次柔软,行人工呼吸法,此时摩擦,更不宜间断。 
迨患者自能呼吸,先被以薄衾,继用稍浓之被,渐移入暖室。 
按∶此法必周身血肉,冻至冰凝,呼吸全无者,方宜用之。若冻犹不至若是之剧,用其法者又宜斟酌变通。 
究之其法虽善,若果有寒痰杜塞,必兼用点天突穴,捏结喉法,方能挽救。人工呼吸法,即患者呼吸全无, 
以法复其呼吸之谓也。其法,先将患者仰卧,俾其头及胸稍高。启其口,将舌周遭缠以细布条,紧结之,防 
舌退缩,及口之收闭。救护者跪于头之旁,以两手握患者之两肘,上提过头,俾空气流入肺中,以助其吸后, 
须臾,将两肘放下,紧压于胸胁之际,以助其呼(助其呼时更有人以两手心按其胸及心窝更佳)。如此往复, 
行至患者自能呼吸而止。此为救急之良方,凡呼吸暴停者,皆可用此方救之。 
【明矾汤】生白矾,长于治顽痰热痰,急证用之,诚有捷效。惟凉痰凝滞者,断不可用。一妇人,年二 
十余。因悲泣过度,痰涎杜塞胃口,其胃气蓄极上逆,连连干呕。形状又似呃逆,气至咽喉不能上达。剧时, 
浑身抖战,自掇其发,有危在顷刻之状。医者,用生姜自然汁灌之,益似不能容受。愚诊视之,其脉左手沉濡, 
右三部皆无。然就其不受生姜观之,仍当是热痰杜塞,其脉象如此者,痰多能瘀脉也。且其面有红光,亦系热 
证。遂用生白矾二钱,化水俾饮之,即愈。此方愚用之屡次,审知其非寒痰 
杜塞,皆可随手奏效。即痰厥至垂危者,亦能救愈。 
【麝香香油灌法】严用和云∶“中风不醒者,麝香清油灌之。”曾治一人,年二十余。因夫妻反目,身 
躯忽然后挺,牙关紧闭,口出涎沫。及愚诊视,已阅三点钟矣。其脉闭塞不全,先用痧药吹鼻,得嚏气通, 
忽言甚渴。及询之,仍昏昏如故,惟牙关微开,可以进药。因忆严用和麝香清油灌法,虽治中风不醒,若治痰 
厥不醒,亦当有效。况此证形状,未必非内风掀动。遂用香油二两炖热,调麝香一分,灌之即醒。 
又∶硼砂四钱化水,治痰厥可代白矾,较白矾尤稳妥。若治寒痰杜塞,用胡椒三钱捣碎,煎汤灌之, 
可代生姜自然汁,与干姜汤。 
附录∶ 
沧县董××来函∶ 
朱姓妇,产后旬余,甚平顺。适伊芳弟来视,午后食煮包一大碗,伊芳弟去后,竟猝然昏倒,四肢抽搐, 
不省人事。延为诊视,六脉皆伏。当系产后五内空虚,骤而饱食填息,胸中大气不能宣通,诸气亦因之闭塞, 
故现此证。取药不及,急用点天突穴及捏结喉法,又用针刺十宣及少商穴,须臾咳吐稠痰若干,气顺腹响,微 
汗而愈。 

\chapter{治肺病方}
<篇名>1.黄膏
属性:治肺有劳病,薄受风寒即喘嗽,冬时益甚者。 
生箭 (四钱) 生石膏(四钱捣细) 净蜂蜜(一两) 粉甘草(二钱细末) 
生怀山药(三钱细末) 鲜茅根(四钱,锉碎如无鲜者可用干者二钱代之) 
上药六味,先将黄 、石膏、茅根,煎十余沸去渣,澄取清汁二杯,调入甘草、山药末同煎,煎时以箸搅 
之,勿令二末沉锅底,一沸其膏即成。再调入蜂蜜,令微似沸,分三次温服下,一日服完,如此服之,久而 
自愈。然此乃预防之药,喘嗽未犯时,服之月余,能祓除病根。 
肺胞之体,原玲珑通彻者也。为其玲珑通彻,故具 辟之机,而司呼吸之气。其 辟之机无碍,即呼吸之 
气自如也。有时肺脏有所损伤,其微丝血管及肺胞涵津液之处,其气化皆湮淤凝滞,致肺失其玲珑之体,即有 
碍于 辟之机,呼吸即不能自如矣。然当气候温和时,肺叶舒畅,呼吸虽不能自如,犹不至甚剧。有时薄受风寒, 
及令届冱寒之时,肺叶收缩,则瘀者益瘀,能 而不能辟,而喘作矣。肺中之气化,瘀而且喘,痰涎壅滞,而 
嗽亦作矣。故用黄 以补肺之阳,山药以滋肺之阴,茅根以通肺之窍,俾肺之阴阳调和,窍络贯通,其 
辟之力自适均也。用石膏者,因其凉而能散,其凉也能调黄 之热,其散也能助茅根之通也。用甘草者,因其 
味甘,归脾益土,即以生金也。用蜂蜜者,因其甘凉滑润,为清肺润肺,利痰宁嗽之要品也。 
茅根禀初春少阳之气,升而能散,原肺脏对宫,肝家之药也。夫肺金主敛,肝木主散,此证因肺金之敛太 
过,故用茅根导引肝木之气,入肺以宣散之,俾其 辟之机自若,而喘嗽均不作矣。 
或问∶凡药之名膏者,皆用其药之原汁,久经熬炼而成膏。今仅取黄 、石膏、茅根之清汁,而调以山药、 
甘草之末与蜜,以成膏者何也?答曰∶古人煎药,皆有火候,及药之宜先入后入,或浸水掺入;及药之宜汤、 
宜膏、宜丸、宜散之区别,然今人不讲久矣。如此方黄 、茅根过炼,则宣通之力微,石膏过炼, 
则清凉之力减,此三味所以不宜熬膏也。然犹恐药入胃之后,由中焦而直趋下焦,其力不能灌注于肺。故加山 
药、蜂蜜之润而粘,甘草之和而缓者,调入成膏。使人服之,能留恋胃中不遽下,俾其由胃输脾,由脾达肺也。 
或问∶调之成膏者,恃山药、蜂蜜也。至甘草何不与黄 、石膏同煎取汁,而亦为末调入?答曰∶西人谓, 
甘草微有苛(苛即薄荷)辣之味,煎之则甘味减,而苛辣之味转增。是以西人润肺之甘草水,止以开水浸之, 
取其味甘、且清轻之气上升也。此方将甘草调入汤中,止煎一沸,亦犹西人作甘草水之意也。 


<篇名>2.清金益气汤
属性:治 羸少气,劳热咳嗽,肺痿失音,频吐痰涎,一切肺金虚损之病。 
生黄 (三钱) 生地黄(五钱) 知母(三钱) 粉甘草(三钱) 玄参(三钱) 沙参(三钱) 
川贝母(二钱去心) 牛蒡子(三钱炒捣) 
一妇人,年四十,上焦发热,咳吐失音,所吐之痰自觉腥臭,渐渐羸瘦,其脉弦而有力。投以清火润肺之 
药,数剂不效。为制此汤,于大队清火润肺药中,加生黄 一味以助元气,数剂见轻,十余剂后,病遂全愈。 
或问∶脉既有力矣,何以复用补气之药?答曰∶脉之有力,有真有假。凡脉之真有力者,当于敦浓和缓中 
见之,此脾胃之气壮旺,能包括诸脏也(脾胃属土能包括金木水火诸脏腑)。其余若脉象洪而有力,多系外 
感之实热。若滑而有力,多系中焦之热痰。若弦而有力,多系肝经之偏盛,尤为有病之脉,此证之脉是也。 
盖肺属金、肝属木,金病不能镇木,故脉现弦而有力之象。此肝木横恣,转欲侮金之象也。凡肺痿、肺痈之病, 
多有胁下疼者,亦系肝木偏胜所致。 
一人,年三十余,肺中素郁痰火,又为外感拘束,频频咳嗽,吐痰腥臭。恐成肺痈,求为延医。其脉浮而 
有力,关前兼滑。遂先用越婢汤,解其外感,咳嗽见轻,而吐痰腥臭如故。次用葶苈(生者三钱纱袋装之)大 
枣(七枚擘开)汤,泻其肺中壅滞之痰,间日一服。又用三七、川贝、粉甘草、金银花为散,鲜地骨皮煎汤, 
少少送服,日三次。即用葶苈大枣汤之日,亦服一次。如此调治数日,葶苈大枣汤用过三次,痰涎顿少,亦不 
腥臭。继用清金益气汤、贝母、牛蒡子各加一钱,连服十余剂,以善其后。 


<篇名>3.清金解毒汤
属性:治肺脏损烂,或将成肺痈,或咳嗽吐脓血者,又兼治肺结核。 
生明乳香(三钱) 生明没药(三钱) 粉甘草(三钱) 生黄 (三钱) 玄参(三钱) 沙参(三钱) 牛蒡 
子(三钱炒捣) 贝母(三钱) 知母(三钱) 三七(二钱,捣细药汁送服) 
将成肺痈者去黄 ,加金银花三钱。 
一人,年四十八,咳吐痰涎甚腥臭,夜间出汗,日形羸弱。医者言不可治,求愚诊视。脉数至六至,按之 
无力,投以此汤,加生龙骨六钱,又将方中知母加倍,两剂汗止,又服十剂全愈。 
肺结核之治法,曾详载于参麦汤下。然所论者,因肺结核而成劳瘵之治法,此方及下方,乃治肺结核而未 
成劳瘵者也。 


<篇名>4.安肺宁嗽丸
属性:治肺郁痰火及肺虚热作嗽,兼治肺结核。 
嫩桑叶(一两) 儿茶(一两) 蓬砂(一两) 苏子(一两炒捣) 粉甘草(一两) 
上药五味为细末,蜜作丸三钱重,早晚各服一丸,开水送下。 
肺脏具 辟之机,治肺之药,过于散则有碍于 ,过于敛则有碍于辟。桑得土之精气而生(根皮甚黄燧应 
夏季是其明征),故长于理肺家之病,以土生金之义也。至其叶凉而宣通,最解肺中风热,其能散可知。又善 
固气化,治崩带脱肛(肺气旺自无诸疾),其能敛可知。敛而且散之妙用,于肺脏 辟之机尤投合也。蓬砂之 
性凉而滑,能通利肺窍,儿茶之性凉而涩,能安敛肺叶。二药并用,与肺之 辟亦甚投合。又佐以苏子之降气 
定喘,甘草之益土生金,蜂蜜之润肺清燥,所以治嗽甚效也。 
按∶蓬砂、儿茶,医者多认为疮家专药。不知其理痰宁嗽,皆为要品。且二药外用,能解毒化腐生肌,故 
内服亦治肺结核,或肺中损烂,亦甚有效验。 


<篇名>5.清凉华盖饮
属性:治肺中腐烂,浸成肺痈,时吐脓血,胸中隐隐作疼,或旁连胁下亦疼者。 
甘草(六钱) 生明没药(四钱,不去油) 丹参(四钱) 知母(四钱) 
病剧者加三七二钱(捣细送服)。脉虚弱者,酌加人参、天冬各数钱。 
肺痈者,肺中生痈疮也。然此证肺中成疮者,十之一二,肺中腐烂者,十之八九。故治此等证,若葶苈、 
皂荚诸猛烈之药,古人虽各有专方,实不可造次轻用,而清火解毒化腐生肌之品,在所必需也。甘草为疮家解 
毒之主药,且其味至甘,得土气最浓,故能生金益肺,凡肺中虚损糜烂,皆能愈之。是以治肺痈便方,有单用 
生粉草四两煎汤,频频饮之者。而西人润肺药水,亦单有用甘草制成者。特其性微温,且有壅滞之意,而调以 
知母之寒滑,则甘草虽多用无碍,且可借甘草之甘温,以化知母之苦寒,使之滋阴退热,而不伤胃也。丹参性 
凉清热,色赤活血,其质轻松,其味微辛,故能上达于肺,以宣通脏腑之毒血郁热而消融之。乳香、没药同为 
疮家之要药,而消肿止疼之力,没药尤胜,故用之以参赞丹参,而痈疮可以内消。三七化瘀解毒之力最优,且 
化瘀血而不伤新血,其解毒之力,更能佐生肌药以速于生肌,故于病之剧者加之。至脉虚者,其气分不能运化 
药力,方虽对证无功,又宜助以人参。而犹恐有肺热还伤肺之虞,是以又用天冬,以解其热也。 
一人,年三十余,昼夜咳嗽,吐痰腥臭,胸中隐隐作疼,恐成肺痈,求为延医。其脉浮而有力,右胜于左, 
而按之却非洪实。投以清金解毒汤,似有烦躁之意,大便又滑泻一次。自言从前服药,略补气分,即觉烦躁, 
若专清解,又易滑泻,故屡次延医无效也。遂改用粉甘草两半、金银花一两、知母、牛蒡子各四钱,煎汤一大 
碗,分十余次温饮下,俾其药力常在上焦,十剂而愈。后两月,因劳力过度旧证复发,胸中疼痛甚于从前,连 
连咳吐,痰中兼有脓血。再服前方不效,为制此汤,两剂疼止。为脉象虚弱,加野台参三钱,天冬四钱,连服 
十剂全愈。 
邑曾××,精通医学,曾告愚曰∶治肺痈方,林屋山人犀黄丸最效。余用之屡次,皆随手奏功,今录其方于 
下,以备参观。 
《外科证治全生集》(王洪绪所着)犀黄丸,用乳香、没药末各一两,麝香钱半,犀牛黄三分,共研细。 
取黄米饭一两捣烂,入药再捣为丸,莱菔子大,晒干(忌火烘)。每服三钱,热陈酒送下。 
徐灵胎曰∶“苏州钱复庵咳血不止,诸医以血证治之,病益剧。余往诊,见其吐血满地,细审血中似有 
脓而腥臭,因谓之曰∶此肺痈也,脓已成矣。《金匮》云‘脓成则死’,然有生者。余遂多方治之,病家亦始 
终相信,一月而愈。盖余平日,因此证甚多,集唐人以来验方,用清凉之药以清其火,滋肺之药以养其血,滑 
降之药以祛其痰,芳香之药以通其气,更以珠黄之药解其 
毒,金石之药填其空,兼数法而行之,屡试必效。今治复庵,亦兼此数法而痊。”按∶此论诚为治肺痈者之准 
绳,故录之以备参观。 
西人、东人,对于肺结核,皆视为至险之证。愚治以中药汤剂,辅以西药阿斯匹林,恒随手奏效,参麦汤 
下论之甚详。而于近今,又得一治法。奉天宿××之兄,年近五旬,素有肺病。东人以为肺结核,屡次医治皆 
无效。一日忽给其弟来电报,言病势已革,催其速还。宿××因来院中,求为疏方,谓前数日来信言,痰嗽较 
前加剧,又添心中发热,今电文未言及病情,大约仍系前证,而益加剧也。夫病势至此,诚难挽回,因其相求 
恳切,遂为疏方∶玄参、生山药各一两,而佐以川贝、牛蒡、甘草诸药。至家将药煎服,其病竟一汗而愈。始 
知其病之加剧者,系有外感之证。外感传里,阳明燥热,得凉润之药而作汗,所以愈也。其从前肺病亦愈者, 
因肺中之毒热随汗外透,暂觉愉快,而其病根实犹伏而未除也。后旬余其肺病复发,咳嗽吐痰腥臭。宿××复来 
询治法,手执一方,言系友人所赠,问可服否。视之林屋山人犀黄丸也。愚向者原拟肺结核可治以犀黄丸,及 
徐氏所论治肺痈诸药。为其价皆甚昂,恐病者辞费,未肯轻于试用。今有所见与愚同者,意其方必然有效。怂 
恿制其丸,服之未尽剂而愈。 
奉天赵××,年四十许。心中发热、懒食、咳嗽、吐痰腥臭,羸弱不能起床。询其得病之期,至今已迁延三 
月矣。其脉一分钟八十五至,左脉近平和,右脉滑而实,舌有黄苔满布,大便四五日一行且甚燥。知其外感, 
稽留于肺胃,久而不去,以致肺脏生炎,久而欲腐烂也。西人谓肺结核证至此,已不可治。而愚慨然许为治愈, 
投以清金解毒汤,去黄 ,加生山药六钱、生石膏一两,三剂后热大清减,食量加增,咳嗽吐痰皆见愈。遂去 
山药,仍加黄 三钱,又去石膏,以花粉六钱代之,每日兼服阿斯匹林 
四分之一瓦,如此十余日后,病大见愈。身体康健,而间有咳嗽之时,因忙碌遂停药不服。二十日后,咳嗽又 
剧,仍吐痰有臭,再按原方加减治之,不甚效验。亦俾服犀黄丸病遂愈。 

\chapter{治吐衄方}
<篇名>1.寒降汤
属性:治吐血、衄血,脉洪滑而长,或上入鱼际,此因热而胃气不降也,以寒凉重坠之药,降其胃气则血止矣。 
生赭石(六钱,轧细) 清半夏(三钱) 蒌仁(四钱,炒捣) 生杭芍(四钱) 竹茹(三钱) 牛蒡子(三钱, 
炒捣) 粉甘草(钱半) 
《金匮》治心气不足吐衄,有泻心汤,大黄与黄连、黄芩并用,后世未窥仲景制方之意,恒多误解。不知 
所谓心气不足者,非不足也,若果不足,何又泻之?盖此证因阳明胃腑之热,上逆冲心,以致心中怔忡不安, 
若有不足之象。仲景从浅处立说,冀人易晓,遂以心气不足名之。故其立方,独本《内经》吐血、衄血,责重 
阳明不降之旨,用大黄直入阳明之腑,以降其逆上之热,又用黄芩以清肺金之热,使其清肃之气下行,以助阳 
明之降力,黄连以清心火之热,使其元阳潜伏,以保少阴之真液,是泻之实所以补之也。且黄连之性肥肠止泻, 
与大黄并用,又能逗留大黄之力,使之不至滑泻,故吐衄非因寒凉者,服之莫不立愈。且愈后而瘀血全消,更 
无他患,真良方也。即使心气果系不足,而吐衄不止将有立危之势,先用泻心汤以止其吐衄,而后从容调补, 
徐复其正,所谓急则治标,亦医家之良图也。乃世人竟畏大黄力猛,不敢轻用,即或用之,病家亦多骇疑。是 
以愚不得已, 
拟此寒降汤,重用赭石,以代大黄降逆之力,屡次用之,亦可随手奏效也。 
或问∶后世本草谓血证忌用半夏,以其辛而燥也。子所拟寒降汤,治吐衄之因热者,何以方中仍用半夏, 
独不虑其辛燥伤血乎?答曰∶血证须有甄别,若虚劳咳嗽,痰中带血,半夏诚为所忌。若大口吐血,或衄血不 
止,虽虚劳证,亦可暂用半夏以收一时之功,血止以后,再徐图他治。盖吐血之证,多由于胃气挟冲气上逆, 
衄血之证,多由于胃气冲气上逆,并迫肺气亦上逆。《内经》厥论篇曰∶阳明厥逆、喘咳身热、善惊衄、呕血。 
煌煌圣言,万古不易。是治吐衄者,原当以降阳明之厥逆为主,而降阳明胃气之逆者,莫半夏若也。 
一童子,年十四,陡然吐血,一昼夜不止,势甚危急,求为诊视。其脉洪长,右部尤重按有力。知其胃气 
因热不降,血随逆气上升也。为拟此汤,一剂而愈,又服一剂,脉亦和平。 
一人,年十八,偶得吐血证,初不甚剧。因医者误治,遂大吐不止。诊其脉如水上浮麻,莫辨至数,此虚 
弱之极候也。若不用药立止其血,危可翘足而待。遂投以此汤,去竹茹,加生山药一两,赭石改用八钱,一剂 
血止。再诊其脉,左右皆无,重按亦不见。愚不禁骇然。询之心中亦颇安稳,惟觉酸懒无力。忽忆吕沧洲曾治 
一发斑证,亦六脉皆无,沧洲谓∶脉者血之波澜,今因发斑伤血,血伤不能复作波澜,是以不见,斑消则脉出 
矣。遂用白虎加人参汤,化其斑毒,脉果出(详案在青盂汤下)。今此证大吐亡血,较之发斑伤血尤甚,脉之 
重按不见,或亦血分虚极,不能作波澜欤?其吐之时,脉如水上浮麻者,或因气逆火盛,强迫其脉外现欤?不 
然闻其诊毕还里(相距十里),途中复连连呕吐,岂因路间失血过多欤?踌躇久之,乃放胆投以大剂六味地黄 
汤,减茯苓、泽泻三分之二,又加人参、赭石各数钱,一剂脉出。又服平补之药二十余剂,始复初。 
按∶因寒因热,皆可使胃气不降。然因热胃气不降者,人犹多知之,因寒胃气不降者,则知者甚鲜。黄氏 
论胃气不降,专主因寒一面,盖有所感触而言也。曾有一少妇,上焦烦热,不能饮食,频频咳吐,皆系稀涎, 
脉象弦细无力。知系脾胃湿寒,不能运化饮食下行,致成留饮为恙也。询其得病之初,言偶因咳嗽懒食,延本 
处名医投以栝蒌、贝母、麦冬之类,旋愈旋即反复,服药月余竟至如此。遂为开苓桂术甘汤,加干姜、半夏 
(细观理饮汤后跋语自知),且细为剖析用药之意。及愚旋里,其药竟不敢服,复请前医治之,月余而亡。夫 
世之所谓名医者,其用药大抵如此,何不读黄氏之论,而反躬自省也哉! 
门人高××实验一方∶赭石、滑石等分研细,热时新汲井泉水送服,冷时开水送服,一两或至二两,治吐 
衄之因热者甚效。高××又在保阳,治一吐血证甚剧者,诸药皆不效,诊其脉浮而洪,至数微数,重按不实。初 
投以拙拟保元寒降汤,稍见效,旋又反复。高××遂放胆投以赭石二两、台参六钱、生杭芍一两,一剂而愈。 
附录∶ 
直隶青县张××来函∶ 
天津曹××,年二十五岁,自春日患吐血证,时发时愈,不以介意。至仲冬,忽吐血较前剧,咳嗽音哑, 
面带贫血,胸中烦热,食少倦怠,屡治罔效,来寓求诊。左脉细弱,右脉则弦而有力,知其病久生热,其胃气 
因热上逆,血即随之上升也。为开寒降汤方,为其咳嗽音哑,加川贝三钱,连服二剂,病大轻减。又服二剂, 
不但吐血已止,而咳嗽音哑诸病皆愈。 
安徽当阳吴××来函∶ 
孟夏二十三日,赤日晴天,铄人脏腑。有陶××者,因业商,斯日出外买粮,午后忽于路中患吐血,迨抵家 
尚呕不止。凌晨来院求治。诊其脉象洪滑,重按甚实,知其为热所迫而胃气不降也。因夫子尝推《金匮》泻心 
汤为治吐衄良方,遂俾用其方煎汤,送服黑山栀细末二钱。服后病稍愈而血仍不止,诊其脉仍然有力,遂为开 
寒降汤,加广三七细末三钱,俾将寒降汤煎一大盅,分两次将三七细末送服。果一剂而愈。 


<篇名>2.温降汤
属性:治吐衄脉虚濡而迟,饮食停滞胃口,不能消化,此因凉而胃气不降也,以温补开通之药,降其胃气,则血止矣。 
白术(三钱) 清半夏(三钱) 生山药(六钱) 干姜(三钱) 生赭石(六钱,轧细) 生杭芍 
(二钱) 川浓朴(钱半) 生姜(二钱) 
或问∶此汤以温降为名,用药宜热不宜凉矣。乃既用干姜之热,复用芍药之凉,且用干姜而更用生姜者何 
也?答曰∶脾胃与肝胆,左右对待之脏腑也。肝胆属木,中藏相火,其性恒与热药不宜。用芍药者,所以防干 
姜之热力入肝也。且肝为藏血之脏,得芍药之凉润者以养之,则宁谧收敛,而血不妄行。更与生姜同用,且能 
和营卫,调经络,引血循经,此所以用干姜又用生姜也。 
《内经》厥论篇谓阳明厥逆衄呕血,所谓阳明者,指胃腑而言也,所谓厥逆者,指胃腑之气上行而言也。 
盖胃以消化饮食,传送下行为职,是以胃气以息息下行为顺,设或上行则为厥逆;胃气厥逆,可至衄血、呕血, 
因血随胃气上行也。然胃气厥逆因热者固多,因寒者亦间有之。岁在壬寅,邑之北境,有学生刘××者,年十三 
岁,一日之间衄血四次。诊其脉甚和平,询其心中不觉凉热。因思吐衄之证热者居多,且以童子少阳 
之体,时又当夏令,遂略用清凉止血之品。衄益甚,脉象亦现微弱,知其胃气因寒不降,转迫血上逆而为衄也。 
投以拙拟温降汤,一剂即愈。隔数日又有他校学生,年十四岁,吐血数日不愈,其吐之时,多由于咳嗽。诊其 
脉,甚迟濡,右关尤甚。疑其脾胃虚寒,不能运化饮食,询之果然。盖吐血之证多由于胃气不降,饮食不能运 
化,胃气即不能下降。咳嗽之证,多由于痰饮入肺。饮食迟于运化,又必多生痰饮,因痰饮而生咳嗽,因咳嗽 
而气之不降者更转而上逆,此吐血之所由来也。亦投以温降汤,一剂血止,接服数剂,饮食运化,咳嗽亦愈。 
近在沈阳论及此事,李××谓,从前有老医徐××者,曾用理中汤治愈历久不愈之吐血证,是吐血诚有因寒 
者之明征也。然徐××但用理中汤以暖胃补胃,而不知用赭石、半夏佐之以降胃气,是处方犹未尽善也。特是 
药局制药,多不如法,虽清半夏中亦有矾,以治吐衄及呕吐,必须将矾味用微温之水淘净。淘时,必于方中原 
定之分量外,多加数钱,以补其淘去矾味所减之分量及药力。 
又∶薛立斋原有血因寒而吐者,治用理中汤加当归之说。特其因寒致吐血之理,未尝帮助,是以后世间有 
驳其说者。由斯知着医书者宜将病之原因仔细发透,俾读其书者易于会悟,不至生疑为善。 
不惟吐衄之证有因寒者,即便血之证亦有因寒者,特其证皆不多见耳。邻村高某,年四十余,小便下血久 
不愈,其脉微细而迟,身体虚弱,恶寒,饮食减少。知其脾胃虚寒,中气下陷,黄坤载所谓“血之亡于便溺者, 
太阴不升也。”为疏方∶干姜、于术各四钱,生山药、熟地黄各六钱,乌附子、炙甘草各三钱。煎服一剂,血 
即见少。连服十余剂,全愈。此方中不用肉桂者,恐其动血分也。 


<篇名>3.清降汤
属性:治因吐衄不止,致阴分亏损,不能潜阳而作热,不能纳气而作喘。甚或冲气因虚上干,为呃逆、为眩晕。 
心血因虚甚不能内荣,为怔忡、为惊悸不寐。或咳逆、或自汗诸虚证蜂起之候。 
生山药(一两) 清半夏(三钱) 净萸肉(五钱) 生赭石(六钱,轧细) 牛蒡子(二钱,炒捣) 
生杭芍(四钱) 甘草(钱半) 


<篇名>4.保元寒降汤
属性:治吐血过多,气分虚甚,喘促咳逆,血脱而气亦将脱。其脉上盛下虚,上焦兼烦热者。 
生山药(一两) 野台参(五钱) 生赭石(八钱,轧细) 知母(六钱) 大生地(六钱) 生杭芍(四钱) 
牛蒡子(四钱,炒捣) 三七(二钱,细轧药汁送服) 
一叟,年六十四,素有劳疾,因劳嗽太甚,呕血数碗。其脉摇摇无根,或一动一止,或两三动一止。此气 
血虚极,将脱之候也。诊脉时见其所嗽吐者,痰血相杂。询其从前呕吐之时心中发热。为制此汤,一剂而血止, 
又服数剂脉亦调勺。 
附录∶ 
直隶青县张××来函∶ 
河间裘××,年二十八岁。患咳嗽吐血,且咯吐甚多,气分太虚,喘息迫促,上焦烦热,其脉大而无力,右 
部尤甚,盖血脱而气亦将脱也。急用保元寒降汤,加青竹茹、麦门冬各三钱。一剂血止。至第二剂,将台参五 
钱易为西洋参一钱,服之而愈。方病相投,效如影响,洵不误也。 


<篇名>5.保元清降汤
属性:治吐衄证,其人下元虚损,中气衰惫,冲气胃气因虚上逆,其脉弦而硬急,转似有力者。 
野台参(五钱) 生赭石(八钱,轧细) 生芡实(六钱) 生山药(六钱) 生杭芍(六钱) 牛蒡子(二钱, 
炒捣) 甘草(钱半) 
附录∶ 
友人毛××曾治一少年吐血证。其人向经医者治愈,旋又反复。毛××诊其脉弦而有力,知其为冲胃之气上逆 
也。遂于治吐血方中,重用半夏、赭石以降逆,白芍、牡蛎(不 )以敛冲泻热,又加人参以补其中气,使中 
气健旺以斡旋诸药成功。有从前为治愈之医者在座,颇疑半夏不可用,毛××力主服之。一剂血止,再剂脉亦 
和平,医者讶为异事。毛××晓知曰∶“此证乃下元虚损,冲气因虚上逆,并迫胃气亦上逆,脉似有力而非真有 
力,李士材四字脉诀所谓∶直上直下,冲脉昭昭者,即此谓也。若误认此脉为实热,而恣用苦寒之药凉其血分, 
血分因凉而凝,亦可止而不吐,而异日瘀血为恙,竟成劳瘵者多矣。今方中用赭石、半夏以镇冲气,使之安其 
故宅,而即用白芍、牡蛎以敛而固之,使之永不上逆。夫血为气之配,气为血之主,气安而血自安矣,此所以 
不治吐血,而吐血自止也。况又有人参之大力者,以参赞诸药,使诸药之降者、敛者,皆得有所凭借以成功乎。” 


<篇名>6.秘红丹
属性:治肝郁多怒,胃郁气逆,致吐血、衄血及吐衄之证屡服他药 
不效者,无论因凉因热,服之皆有捷效。 
川大黄(一钱细末) 油肉桂(一钱细末) 生赭石(六钱细末) 
上药三味,将大黄、肉桂末和匀,用赭石末煎汤送下。 
一妇人,年近三旬,咳嗽痰中带血,剧时更大口吐血,常觉心中发热。其脉一分钟九十至,按之不实。 
投以滋阴宁嗽降火之药数剂无效。因思此证,若用药专止其嗽,嗽愈其吐血亦当愈。遂用川贝九钱,煎取清汤 
四茶盅,调入生山药细末一两,煮作稀粥。俾于一日连进二剂,其嗽顿止(此方可为治虚嗽良方),吐血证亦 
遂愈。数日后,觉血气上潮,肺复作痒而嗽,因此又复吐血。自言夜间睡时,常作生气恼怒之梦,怒极或梦中 
哭泣,醒后必然吐血。据所云云,其肝气必然郁遏,遂改用舒肝(连翘薄荷不可多用)泻肝(龙胆楝子)之品, 
而以养肝(柏子仁生阿胶)镇肝(生龙骨生牡蛎)之药辅之,数剂病稍轻减。而犹间作恼怒之梦,梦后仍复吐 
血。欲辞不治,病家又信服难却。再四踌躇,恍悟平肝之药,以桂为最要,肝属木,木得桂则枯也(以桂作钉 
钉树,其树立枯),而单用之则失于热。降胃止血之药,以大黄为最要(观《金匮》治吐衄有泻心汤重用大黄 
可知),胃气不上逆,血即不逆行也,而单用之又失于寒。若二药并用,则寒热相济,性归和平,降胃平肝, 
兼顾无遗。况俗传方,原有用此二药为散,治吐血者,用于此证当有捷效。而再以重坠之药辅之,则力专下行, 
其效当更捷也。遂用大黄、肉桂细末各一钱和匀,更用生赭石细末煎汤送下,吐血顿愈,恼怒之梦,亦从此不 
作。后又遇吐血者数人,投以此方,皆随手奏效。至其人身体壮实而暴得吐血者,又少变通其方∶大黄、肉桂 
细末各用钱半,将生赭石细末六钱与之和匀,分三次服,白开水送下,约点半钟服一次。 


<篇名>7.二鲜饮
属性:治虚劳证,痰中带血。 
鲜茅根(四两,切碎) 鲜藕(四两,切片) 
煮汁常常饮之,旬日中自愈。若大便滑者,茅根宜减半。再用生山药细末两许,调入药汁中,煮作茶汤服之。 
茅根善清虚热而不伤脾胃,藕善化瘀血而兼滋新血,合用之为涵养真阴之妙品。且其形皆中空,均能利水, 
血亦水属,故能引泛滥逆上之血徐徐下行,安其部位也。 
至于藕以治血证,若取其化瘀血,则红莲者较优。若用以止吐衄,则白莲者胜于红莲者。 
堂兄××,年五旬,得吐血证,延医治疗不效。脉象滑数,摇摇有动象,按之不实。时愚在少年,不敢轻于 
疏方,因拟此便方,煎汤两大碗,徐徐当茶温饮之,当日即见愈,五六日后病遂脱然。自言未饮此汤时,心若 
虚悬无着,既饮后,觉药力所至,若以手按心,使复其位,此其所以愈也。 


<篇名>8.三鲜饮
属性:治同前证兼有虚热者。 
即前方加鲜小蓟根二两。 
大、小蓟皆能清血分之热,以止血热之妄行,而小蓟尤胜。凡因血热妄行之证,单用鲜小蓟根数两煎汤, 
或榨取其自然汁,开水冲服,均有捷效,诚良药也。 
小蓟茎中生虫,即结疙瘩如小枣。若取其鲜者十余枚捣烂,开水冲服,治吐衄之因热者甚效。邻村李××曾 
告愚曰∶“余少年曾得吐血证,屡次服药不效,后得用小蓟疙瘩便方,服一次即愈。因呼之谓清凉如意珠,真 
药中之佳品也。” 


<篇名>9.化血丹
属性:治咳血,兼治吐衄,理瘀血,及二便下血。 
花蕊石(三钱, 存性) 三七(二钱) 血余(一钱, 存性) 
共研细,分两次,开水送服。 
世医多谓三七为强止吐衄之药,不可轻用,非也。盖三七与花蕊石,同为止血之圣药,又同为化血之圣 
药,且又化瘀血而不伤新血,以治吐衄,愈后必无他患。此愚从屡次经验中得来,故敢确实言之。即单用三七 
四五钱,或至一两,以治吐血、衄血及大、小便下血皆效。常常服之,并治妇女经闭成 瘕。至血余,其化瘀 
血之力不如花蕊石、三七,而其补血之功则过之。以其原为人身之血所生,而能自还原化,且 之为炭,而又 
有止血之力也。 
曾治一童子,年十五,大便下血,数月不愈,所下者若烂炙,杂以油膜,医者诿谓不治。后愚诊视其脉, 
弦数无力。俾用生山药轧细作粥,调血余炭六七分服之,日二次,旬日全愈。 


<篇名>10.补络补管汤
属性:治咳血吐血,久不愈者。 
生龙骨(一两,捣细) 生牡蛎(一两,捣细) 萸肉(一两,去净核) 三七(二钱,研细药汁送服) 
服之血犹不止者,可加赭石细末五六钱。 
张景岳谓∶“咳嗽日久,肺中络破,其人必咳血。”西人谓∶胃中血管损伤破裂,其人必吐血。龙骨、 
牡蛎、萸肉,性皆收涩,又兼具开通之力(三药之性详既济汤来复汤与理郁升陷汤清带汤下),故能补肺络, 
与胃中血管,以成止血之功,而又不至有遽止之患,致留瘀血为恙也。又佐以三七者,取其化腐生新,使损伤 
之处易愈,且其性善理血,原为治衄之妙品也。 
咳血之原由于肺,吐血之原由于胃,人之所共知也。而西人于吐血,论之尤详。其说谓∶胃中多回血管, 
有时溃裂一二处而血出,其故或因胃本体自生炎证,烂坏血管,或因跌打外伤,胃 
中血管断裂,其血棕黑而臭秽,危险难治,但此类甚少。常见之证,大概血管不曾溃裂,其血亦可自管中溢出, 
其血多带黑色。因回血管之血色原紫黑,而溢出在胃,胃中酸汁又能令血色变黑也。若血溢自胃中血管,实时 
吐出,其色亦可鲜红。其病原,或因胃致病,或因身虚弱血质稀薄,皆能溢出。有胃自不病,或因别经传入于 
胃,如妇女倒经,是子宫之血传入于胃。又如肝脾胀大,血不易通行,回血管满溢,入胃则吐出,入大小肠则 
便出。便与吐之路不同,其理一也。 
或问∶《内经》谓∶阳明厥逆,则吐衄。西人谓∶胃中血管损伤破裂出血,则吐血。此二说亦相通乎?答 
曰∶阳明厥逆,胃腑气血必有膨胀之弊,此血管之所以易破也。降其逆气,血管之破者自闭。设有不闭,则用 
龙骨、牡蛎诸收涩之药以补之,防其溃烂,佐以三七、乳香、没药诸生肌之品以养之。此拙拟补络补管汤所以 
效也。设使阳明未尝厥逆,胃中血管或因他故而破裂,则血在胃中,亦恒随饮食下行自大便出,不必皆吐出也。 
此方原无三七,有乳香、没药各钱半。偶与友人××谈及,××谓∶“余治吐血,亦用兄补络补管汤,以三 
七代乳香、没药,则其效更捷。”愚闻之遂欣然易之。 
××又谓∶“龙骨、牡蛎能收敛上溢之热,使之下行,而上溢之血,亦随之下行归经。至萸肉为补肝之 
妙药,凡因伤肝而吐血者,萸肉又在所必需也。且龙骨、牡蛎之功用神妙无穷。即脉之虚弱已甚,日服补药毫 
无起象,或病虚极不受补者,投以大剂龙骨、牡蛎,莫不立见功效,余亦不知其何以能然也。”愚曰∶人身阳 
之精为魂,阴之精为魄。龙骨能安魂,牡蛎能强魄。魂魄安强,精神自足,虚弱自愈也。是龙骨、牡蛎,固为 
补魂魄精神之妙药也。 
邑有吐血久不愈者,有老医于××,重用赤石脂二两,与诸 
止血药治之,一剂而愈。后其子××向愚述其事,因诘之曰∶“重用赤石脂之义何据?”其子曰∶“凡吐血多因 
虚火上升,然人心中之火,亦犹炉中之火,其下愈空虚,而火上升之力愈大,重用赤石脂,以填补下焦,虚火 
自不上升矣。”愚曰∶“兄之论固佳,然犹有剩义。赤石脂重坠之力,近于赭石,故能降冲胃之逆,其粘涩之 
力,近于龙骨、牡蛎,故能补血管之破。兼此二义,重用石脂之奥妙,始能尽悉。是以愚遇由外伤内,若跌碰 
致吐血久不愈者,料其胃中血管必有伤损,恒将补络补管汤去萸肉,变汤剂为散剂,分数次服下,则龙骨、牡 
蛎,不但有粘涩之力,且较煎汤服者,更有重坠之力,而吐血亦即速愈也。”其子闻之欣然曰∶“先严用此方 
时,我年尚幼,未知详问,今闻兄言贶我多矣。” 
曾治沧州马氏少妇,咳血三年,百药不效,即有愈时,旋复如故。后愚为诊视,其夜间多汗,遂用净萸肉、 
生龙骨、生牡蛎各一两,俾煎服,拟先止其汗,果一剂汗止,又服一剂咳血亦愈。盖从前之咳血久不愈者,因 
其肺中之络,或胃中血管有破裂处,萸肉与龙骨、牡蛎同用以涩之敛之,故咳血亦随之愈也。 
又治本村表弟张×,年三十许,或旬日,或浃辰之间,必吐血数口,浸至每日必吐,亦屡治无效。其脉近 
和平,微有芤象,亦治以此方,三剂全愈。后又将此方加三七细末三钱,煎药汤送服,以治咳血吐血久不愈者, 
约皆随手奏效。若遇吐血之甚者,宜再加赭石五六钱,与此汤前三味同煎汤,送服三七细末更效。 
邑张某家贫佣力,身挽鹿车运货远行,因枵腹努力太过,遂致大口吐血。卧病旅邸,恐即不起,意欲还里, 
又乏资斧。乃勉强徒步徐行,途中又复连吐不止,目眩心慌,几难举步。腹中觉饥,怀有干饼,又难下咽。偶 
拾得山楂十数枚,遂和干饼食之,觉精神顿爽,其病竟愈。盖酸者能敛,而山楂则酸敛之中,兼有 
化瘀之力。与拙拟补络补管汤之意相近,故获此意外之效也。 
附录∶ 
江苏崇明县蔡××来函∶ 
回忆毕业中学时,劳心过度,致患吐血,虽家祖世医,终难疗治。遍求名医延医,亦时止时吐。及肄业大 
学时,吐血更甚,医者多劝辍学静养,方可望痊。乃辍学家居,服药静养,病仍如旧。计无所施,自取数世所 
藏医书遍阅之,又汗牛充栋,渺茫无涯。况玉石混杂,瑜瑕莫辨,徒增望洋之叹也。幸今秋自周小农处购得 
《衷中参西录》,阅至吐衄方补络补管汤,知为治仆病的方。抄出以呈家祖父,命将药剂减半煎服,颇见效验。 
遂放胆照原方,兼取寒降汤之义加赭石六钱,连服三剂全愈。从前半月之间,必然反复,今已月余安然无恙, 
自觉身体渐强,精神倍加。 


<篇名>11.化瘀理膈丹
属性:治力小任重,努力太过,以致血瘀膈上,常觉短气。若吐血未愈者,多服补药或凉药,或多用诸药炭,强 
止其血,亦可有此病,皆宜服此药化之。 
三七(二钱,捣细) 鸭蛋子(四十粒,去皮) 
上药二味,开水送服,日两次。凡服鸭蛋子,不可嚼破,若嚼破即味苦不能下咽,强下咽亦多呕出。 
一童子,年十四,夏日牧牛野间。众牧童嬉戏,强屈其项背,纳头 中,倒缚其手,置而弗顾,戏名为看 
瓜。后经人救出,气息已断。俾盘膝坐,捶其腰背,多时方苏。惟觉有物填塞胸膈,压其胸中大气,妨碍呼吸。 
剧时气息仍断,两目上翻,身躯后挺。此必因在 中闷极之时努挣不出,热血随努挣之气力上溢,而停于膈上 
也。俾单用三七三钱捣细,开水送服,两次全愈。 
一人,年四十七,素患吐血。医者谓其虚弱,俾服补药,连服十余剂,觉胸中发紧,而血益不止。后有人 
语以治吐血便方,大黄、肉桂各五分轧细,开水送服,一剂血止。然因从前误服补药,胸中常觉不舒,饮食减 
少,四肢酸懒无力。愚诊之,脉似沉牢,知其膈上瘀血为患也。俾用鸭蛋子五十粒去皮,糖水送服,日两次, 
数日而愈。 

\chapter{治心病方}
<篇名>1.定心汤
属性:治心虚怔忡。 
龙眼肉(一两) 酸枣仁(五钱,炒捣) 萸肉(五钱,去净核) 柏子仁(四钱,炒捣) 生龙骨 
(四钱,捣细) 生牡蛎(四钱,捣细) 生明乳香(一钱) 生明没药(一钱) 
心因热怔忡者,酌加生地数钱,若脉沉迟无力者,其怔忡多因胸中大气下陷,详观拙拟升陷汤后跋语及诸 
案自明治法。 
《内经》谓“心藏神”,神既以心为舍字,即以心中之气血为保护,有时心中气血亏损,失其保护之职, 
心中神明遂觉不能自主而怔忡之疾作焉。故方中用龙眼肉以补心血,枣仁、柏仁以补心气,更用龙骨入肝以安 
魂,牡蛎入肺以定魄。魂魄者心神之左辅右弼也,且二药与萸肉并用,大能收敛心气之耗散,并三焦之气化亦 
可因之团聚。特是心以行血为用,心体常有舒缩之力,心房常有启闭之机,若用药一于补敛,实恐于舒缩启闭 
之运动有所妨碍,故又少加乳香、没药之流通气血者以调和之。其心中兼热用生地者,因生地既能生血以补虚, 
尤善凉血而清热,故又宜视热之轻重而斟酌加之也。 
西人曰∶人身心肺关系尤重,与脑相等。凡关系重者、护持 
之尤谨,故脑则有头额等八骨以保护之,而心肺亦有胸胁诸骨以保护之。心肺体质相连,功用亦相倚赖,心之 
功用关系全体,心病则全体皆受害,心之重如此。然论其体质,不过赤肉所为,其能力专主舒缩,以行血脉。 
有左右上下四房;左上房主接肺经赤血;右上房主接周身回血;左下房主发赤血,营运周身;右下房主接上房 
回血过肺,更换赤血而回左上房;左上房赤血,落左下房入总脉管,以养全体;右上房回血,落右下房上注于 
肺,以出炭气而接养气。故人一身之血,皆经过于心肺。心能运血周流一身,无一息之停,实时接入,实时发 
出,其跳跃即其逼发也,以时辰表验试,一 (即一分钟)跳七十五次,每半时跳四千五百次,一昼夜计跳 
十万八千次。然平人跳不自觉,若觉心跳即是心经改易常度。心房之内左浓于右,左下房浓于右下房几一倍, 
盖左房主接发赤血,功用尤劳,故亦加浓也。心位在胸中居左,当肋骨第四至第七节,尖当肋骨第五第六之间, 
下于乳头约一寸至半寸,横向胸骨。病则自觉周遭皆跳,凡心经本体之病,或因心房变薄变浓,或心房之门有 
病,或夹膜有病,或总管有病。亦如眼目之病,或在明角罩,或在瞳人,或在睛珠,非必处处皆病也。大概心 
病左多于右因左房功用尤劳故耳。心病约有数端∶一者,心体变大,有时略大,或大过一半。因心房之户,有 
病拦阻,血出入不便,心舒缩之劳过常度。劳多则变大,亦与手足过劳则肿大之理相同。大甚,则逼血舒缩之 
用因之不灵矣。一者,心房门户变小、或变大、或变窄、或变阔,俱为非宜。盖心血自上房落下房之门,开张 
容纳血入后,门即翕闭,不令血得回旋上出。其自下房入总管处亦有门,血至则开张使之上出,血出后门即翕 
闭,不令血得下返。若此处太窄太小,则血不易出。太大太阔,则血逼发不尽,或已出复返,营运不如常度矣。 
再者心跳∶凡无病之人心跳每不自觉。若因病而跳时时自觉,抚之或觉动。然此 
证有真有假∶真者心自病而跳也,或心未必有病,但因身虚而致心跳,亦以真论;若偶然心跳,其人惊惧,防 
有心病,其实心本无病,即心跳亦临时之事,是为假心跳证,医者均须细辨。凡心匀跳无止息,侧身而卧,可 
左可右,呼吸如常,大概心自不病。所虑跳跃不定,或三四次一停,停后复跳不能睡卧,左半身着床愈觉不安, 
当虑其门户有病,血不回运如常。有停滞妄流而为膨胀者,有累肺而咳嗽、难呼吸或喘者,有累脑而昏蒙头疼, 
中风慌怯者,有累肝而血聚积满溢者,有累胃不易消化,食后不安,心更跳者,皆心病之关系也。若胸胁骨之 
下有时动悸,人或疑为心跳,其实因胃不消化、内有风气,与心跳病无涉,虚弱人及妇女患者最多,略服补胃 
及微利药可也。 
按∶西人论心跳证有真假,真者手扪之实觉其跳,假者手扪之不觉其跳。其真跳者又分两种∶一为心体 
自病,若心房门户变大小窄阔之类,可用定心汤,将方中乳香、没药皆改用三钱,更加当归、丹参各三钱;一 
为心自不病,因身弱而累心致跳,当用治劳瘵诸方治之。至假心跳即怔忡证也,其收发血脉之动力,非大于常 
率,故以手扪之不觉其跳。特因气血虚而神明亦虚,即心之寻常舒缩,徐徐跳动,神明当之,亦若有冲激之势, 
多生惊恐,此等证治以定心汤时,磨取铁锈水煎药更佳。至于用铁锈之说,不但如西人之说,取其能补血分, 
实借其镇重之力以安心神也。载有一味铁养汤,细观方后治验诸案,自知铁锈之妙用。惟怔忡由于大气下陷者, 
断不宜用。 


<篇名>2.安魂汤
属性:治心中气血虚损,兼心下停有痰饮,致惊悸不眠。 
龙眼肉(六钱) 酸枣仁(四钱,炒捣) 生龙骨(五钱,捣末) 生牡蛎(五钱,捣末) 清半夏 
(三钱) 茯苓片(三钱) 生赭石(四钱,轧细) 
若服一两剂后无效者,可于服汤药之外,临睡时用开水送服西药臭剥一瓦,借其麻痹神经之力,以收一 
时之效,俾汤剂易于为力也。 
方书谓∶痰饮停于心下,其人多惊悸不寐。盖心,火也,痰饮,水也,火畏水刑,故惊悸至于不寐也。 
然痰饮停滞于心下者,多由思虑过度,其人心脏气血,恒因思虑而有所伤损。故方中用龙眼肉以补心血,酸枣 
仁以敛心气,龙骨、牡蛎以安魂魄,半夏、茯苓以清痰饮,赭石以导引心阳下潜,使之归藏于阴,以成瞌睡之功也。 
一媪,年五十余,累月不能眠,屡次服药无效。诊其脉有滑象,且其身形甚丰腴。知其心下停痰也。为 
制此汤,服两剂而愈。 
一妇人,年三十许,一月之间未睡片时,自言倦极仿佛欲睡,即无端惊恐而醒。诊其脉左右皆有滑象, 
遂用苦瓜蒂十枚,焙焦轧细,空心时开水送服,吐出胶痰数碗,觉心中异常舒畅,于临眠之先又送服熟枣仁 
细末二钱,其夜遂能安睡。后又调以利痰养心安神之药,连服十余剂,其证永不反复矣。 
《内经》邪客篇有治目不得瞑方。用流水千里以外者八升,扬之万遍,取其清五升煮之,炊以苇薪。水 
沸,置秫米一升,制半夏(制好之半夏)五合,徐炊令竭为一升半。去其渣饮汁一小杯,日三稍益,以知为度 
(知觉好也)。故其病新发者,复杯则卧,汗出而已矣,久则三饮而已也。观此方之义,其用半夏,并非为其 
利痰,诚以半夏生当夏半,乃阴阳交换之时,实为由阳入阴之候,故能通阴阳和表里,使心中之阳渐渐潜藏于 
阴,而入睡乡也。秫米即芦稷之米(俗名高粱),取其汁浆稠润甘缓,以调和半夏之辛烈也。水用长流水,更 
扬之万遍,名曰∶“劳水”,取其甘缓能滋养也。薪用苇薪,取其能畅发肾气上升,以接引心气下降,而交其 
阴阳也。观古人每处一方,并其所用之薪与水及其煎法、服法,莫不 
详悉备载,何其用心之周至哉! 
按∶《内经》之方多奇验,半夏秫米汤,取半夏能通阴阳,秫米能和脾胃,阴阳通、脾胃和,其人即可 
安睡。故《内经》谓“饮药后,复杯即瞑”,言其效之神速也。乃后世因其药简单平常,鲜有用者,则良方竟 
埋没矣。门生高××治天津刘姓,年四十二,四月未尝少睡,服药无效,问治法于愚,告以半夏秫米汤方。高×× 
因其心下发闷,遂变通经方,先用鲜莱菔四两切丝,煎汤两茶杯,再用其汤煎清半夏四钱服之。时当晚八点钟, 
其人当夜即能安睡,连服数剂,心下之满闷亦愈。 

\chapter{治癫狂方}
<篇名>1.荡痰汤
属性:治癫狂失心,脉滑实者。 
生赭石(二两,轧细) 大黄(一两) 朴硝(六钱) 清半夏(三钱) 郁金(三钱) 


<篇名>2.荡痰加甘遂汤
属性:治前证,顽痰凝结之甚者,非其证大实不可轻投。其方,即前方加甘遂末二钱,将他药煎好,调药汤中服。 
凡用甘遂,宜为末,水送服。或用其末,调药汤中服。若入汤剂煎服,必然吐出。又凡药中有甘遂,不 
可连日服之,必隔两三日方可再服,不然亦多吐出。又其性与甘草相犯,用者须切记。 
甘遂性猛烈走窜,后世本草,称其以攻决为用,为下水之圣药。痰亦水也,故其行痰之力,亦百倍于他 
药。曾治一少年癫狂,医者投以大黄六两,连服两剂,大便不泻。后愚诊视,为 
开此方,惟甘遂改用三钱。病家谓,从前服如许大黄,未见行动,今方中止用大黄两许,岂能效乎?愚曰∶但 
服,无虑也。服后,大便连泻七八次,降下痰涎若干,癫狂顿愈。见者以为奇异,彼盖不知甘遂三钱之力,远 
胜于大黄六两之力也。 
痰脉多滑,然非顽痰也。愚治此证甚多。凡癫狂之剧者,脉多瘀塞,甚或六脉皆不见,用开痰药通之, 
其脉方出,以是知顽痰之能闭脉也。 
癫狂之证,乃痰火上泛,瘀塞其心与脑相连窍络,以致心脑不通,神明皆乱。故方中重用赭石,借其重 
坠之力,摄引痰火下行,俾窍络之塞者皆通,则心与脑能相助为理,神明自复其旧也。是以愚治此证之剧者, 
赭石恒有用至四两者,且又能镇甘遂使之专于下行,不至作呕吐也。 
癫者,性情颠倒,失其是非之明。狂者,无所畏惧,妄为妄言,甚或见闻皆妄。大抵此证初起,先微露 
癫意,继则发狂。狂久不愈,又渐成癫,甚或知觉全无。盖此证,由于忧思过度,心气结而不散,痰涎亦即随 
之凝结。又加以思虑过则心血耗,而暗生内热。痰经热炼,而胶粘益甚,热为痰锢,而消解无从。于是痰火充 
溢,将心与脑相通之窍络,尽皆瘀塞,是以其神明淆乱也。其初微露癫意者,痰火犹不甚剧也,迨痰火积而益 
盛,则发狂矣。是以狂之甚者,用药下其痰,恒作红色,痰而至于红,其热可知。迨病久,则所瘀之痰,皆变 
为顽痰。其神明淆乱之极,又渐至无所知觉,而变为癫证。且其知觉欲无,从前之忧思必减,其内热亦即渐消, 
而无火以助其狂,此又所以变为癫也。然其初由癫而狂易治,其后由狂而癫难治。故此证,若延至三四年者, 
治愈者甚少。 
人之神明,原在心与脑两处。金正希曰∶“人见一物必留一影于脑中,小儿善忘者,脑髓未满也,老人 
健忘者,脑髓渐空 
也。”汪 庵释之曰∶“凡人追忆往事,恒闭目上瞪,凝神于脑,是影留于脑之明征。”由斯观之,是脑原主 
追忆往事也。其人或有思慕不遂,而劳神想象,或因从前作事差误,而痛自懊 ,则可伤脑中之神。若因研究 
理解工夫太过,或有将来难处之事,而思患预防,踌躇太过,苦心思索,则多伤心中之神。究之,心与脑,原 
彻上彻下,共为神明之府。一处神明伤,则两处神俱伤。脑中之神明伤,可累及脑气筋。心中之神明伤,亦可 
累及脑气筋。且脑气筋伤,可使神明颠倒狂乱,心有所伤,亦可使神明颠倒狂乱也。 
曾治一少妇癫狂,强灌以药,不能下咽。遂俾以朴硝代盐,每饭食之,病患不知,月余而愈。诚以朴硝咸 
寒属水,为心脏对宫之药,以水胜火,以寒胜热,能使心中之火热,消解无余,心中之神明,自得其养,非仅 
取朴硝之能开痰也。 


<篇名>3.调气养神汤
属性:治其人思虑过度,伤其神明。或更因思虑过度,暗生内热,其心肝之血,消耗日甚,以致心火肝气,上冲 
头部,扰乱神经,致神经失其所司,知觉错乱,以是为非,以非为是,而不至于疯狂过甚者。 
龙眼肉(八钱) 柏子仁(五钱) 生龙骨(五钱,捣碎) 生牡蛎(五钱,捣碎) 远志(二钱,不炙) 
生地黄(六钱) 天门冬(四钱) 甘松(二钱) 生麦芽(三钱) 菖蒲(二钱) 甘草(钱半) 镜 
面朱砂(三分,研细用头次煎药汤两次送服) 磨取铁锈浓水煎药。 
此乃养神明、滋心血、理肝气、清虚热之方也。龙眼肉色赤入心,且多津液,最能滋补血分,兼能保合心 
气之耗散,故以之为主药∶柏子仁多含油质,故善养肝,兼能镇肝,又与龙骨、牡蛎 
之善于敛戢肝火肝气者同用,则肝火肝气自不挟心火上升,以扰乱神经也;用生地黄者,取其能泻上焦之虚热, 
更能助龙眼肉生血也;用天门冬者,取其凉润之性,能清心宁神,即以开燥痰也;用远志、菖蒲者,取其能开 
心窍、利痰涎,且能通神明也;用朱砂、铁锈水者,以其皆能镇安神经,又能定心平肝也;用生麦芽者,诚以 
肝为将军之官,中寄相火,若但知敛之镇之,或激动其响应之力,故又加生麦芽,以将顺其性,盖麦芽炒用能 
消食,生用则善舒肝气也。至于甘松,其性在中医用之以清热、开瘀、逐痹;在西医则推为安养神经之妙药, 
而兼能治霍乱转筋,盖神经不失其所司,则筋可不转,此亦足见安养神经之效也。 

\chapter{治痫风方}
<篇名>1.加味磁朱丸
属性:治痫风。 
磁石(二两,能吸铁者,研极细水飞出,切忌火 ) 赭石(二两) 清半夏(二两) 朱砂(一两) 
上药各制为细末。再加酒曲半斤,轧细过罗,可得细曲四两。炒熟二两,与生者二两,共和药为丸,桐 
子大。铁锈水煎汤,送服二钱,日再服。 
磁石为铁、养二种原质化合,含有磁气。其气和异性相引,同性相拒,颇类电气,故能吸铁。 之则磁 
气全无,不能吸铁,用之即无效。然其石质甚硬,若生用入丸散中,必制为极细末,再以水飞之,用其随水飞 
出者方妥。或和水研之,若拙拟磨翳散之研飞炉甘石法,更佳。 
朱砂无毒,而 之则有毒。按化学之理,朱砂原硫黄、水银二 
原质合成。故古方书,皆谓朱砂内含真汞,汞即水银也。若 之,则仍将分为硫黄、水银二原质,所以有毒。 
又原方原用神曲,而改用酒曲者,因坊间神曲窨发皆未能如法,多带酸味,转不若造酒曲者,业有专门,曲发 
甚精,用之实胜于神曲也。 
磁朱丸方,乃《千金方》中治目光昏耗、神水宽大之圣方也。李濒湖解曰∶磁石入肾,镇养真阴,使肾水 
不外移。朱砂入心,镇养心血,使邪火不上侵。佐以神曲,消化滞气,温养脾胃生发之气。然从前但知治眼疾 
而不知治痫风,至柯韵伯称此方治痫风如神,而愚试之果验,然不若加赭石、半夏之尤为效验也。原方止此三 
味,为加赭石、半夏者,诚以痫风之证,莫不气机上逆,痰涎上涌。二药并用,既善理痰,又善镇气降气也。 
送以铁锈汤者,以相火生于命门,寄于肝胆,相火之暴动实于肝胆有关。此肝胆为木脏,即为风脏,内风之煽 
动,亦莫不于肝胆发轫。铁锈乃金之余气,故取金能制木之理,镇肝胆以熄内风;又取铁能引电之理,借其重 
坠之性,以引相火下行也。 
附录∶ 
友人祁××之弟患痫风,百药不效。后得一方,用干熊胆若黄豆粒大一块(约重分半),凉水少许浸开服之 
(冬月宜温水浸开温服),数次而愈。祁××向愚述之,因试其方果效。 


<篇名>2.通变黑锡丹
属性:治痫风。 
铅灰(二两,研细) 硫化铅(一两,研细) 麦曲(两半,炒熟) 
上三味,水和为丸,桐子大。每服五六丸,多至十丸。用净芒硝四五分冲水送服。若服药后,大便不利者 
(铅灰硫化铅皆能涩大便),芒硝又宜多用。 
古方有黑锡丹,用硫黄与铅化合,以治上热下凉,上盛下虚之证,洵为良方。而犹未尽善者,因其杂以 
草木诸热药,其性易升浮,即不能专于下达。向曾变通其方,专用硫化铅,和熟麦曲为丸,以治痫风数日一发 
者,甚有效验。乃服至月余,因觉热停服,旬余病仍反复。遂又通变其方,多用铅灰,少用硫化铅,俾其久服 
不致生热,加以累月之功,痫风自能除根。更佐以健脾,利痰、通络、清火之汤剂,治法尤为完善。 
取铅灰法 用黑铅数斤,熔化后,其面上必有浮灰。屡次熔化,即可屡次取之。 
制硫化铅法 用黑铅四两,铁锅内熔化。再用硫黄细末四两,撒于铅上。硫黄皆着,急用铁铲拌炒。铅 
经硫黄烧炼,结成砂子,取出晾冷,碾轧成饼者(系未化透之铅)去之,余者,再用乳钵研极细。 


<篇名>3.一味铁养汤
属性:治痫风,及肝胆之火暴动,或胁疼,或头疼目眩,或气逆喘吐,上焦烦热,至一切上盛下虚之证皆可。 
用其汤煎药,又兼能补养血分。 
方用长锈生铁,和水磨取其锈,磨至水皆红色,煎汤服之。 
化学家名铁锈为铁养,以铁与氧气化合而成锈也。其善于镇肝胆者,以其为金之余气,借金以制木也。 
其善治上盛下虚之证者,因其性重坠,善引逆上之相火下行。相火为阴中之火,与电气为同类,此即铁能引电 
之理也。其能补养血分者,因人血中原有铁锈,且取铁锈嗅之,又有血腥之气,此乃以质补质,以气补气之理。 
且人身之血,得氧气则赤,铁锈原铁与氧气化合,故能补养血分也。西人补血之药,所以有铁酒。 
一六岁幼女,初数月一发痫风,后至一日数发,精神昏昏若 
睡,未有醒时。且两目露睛,似兼慢惊。遂先用《福幼编》治慢惊之方治之,而露睛之病除。继欲治其痫风, 
偶忆方书有用三家磨刀水洗疮法,因铁锈能镇肝,以其水煎药,必能制肝胆上冲之火,以熄内风。乃磨水者, 
但以水贮罐中,而煎药者,误认为药亦在内,遂但煎其水服之,其病竟愈。后知药未服,仍欲煎服。愚曰∶磨 
刀水既对证,药可不服。自此日煎磨刀水服两次,连服数日,痫风永不再发。 
一人,年三十许,痫风十余年不愈,其发必以夜。授以前加味磁朱丸方,服之而愈。年余其病又反复, 
然不若从前之剧。俾日磨浓铁锈水,煎汤服之,病遂除根。 
族家嫂,年六旬。夜间忽然呕吐头疼,心中怔忡甚剧,上半身自汗。其家人以为霍乱证。诊其脉,关前 
浮洪,摇摇而动。俾急磨浓铁锈水,煎汤服下即愈。 
友人韩××曾治一人,当恼怒之后,身躯忽然后挺,气息即断,一日数次。韩××诊其脉,左关虚浮。遂投 
以萸肉(去净核)、龙骨、牡蛎(皆不用 )、白芍诸药,用三家磨刀水煎之,一日连服二剂,病若失。 
西药治痫风者,皆系麻醉脑筋之品,强制脑筋使之不发,鲜能祓除病根。然遇痫风之剧而且勤,身体羸 
弱,不能支持者,亦可日服其药两次,以图目前病不反复,而徐以健脾、利痰、通络、清火之药治之。迨至身 
形强壮,即可停止西药,而但治以健脾、利痰、通络、清火之品。或更佐以镇惊(若朱砂、磁石类)、祛风 
(若蜈蚣、全蝎类)、透达脏腑(若麝香、牛黄类)之品,因证制宜,病根自能祓除无余也。 

\footnote{治小儿风证方}
<篇名>1.定风丹
属性:治初生小儿绵风,其状逐日抽掣,绵绵不已,亦不甚剧。 
生明乳香(三钱) 生明没药(三钱) 朱砂(一钱) 全蜈蚣(大者一条) 全蝎(一钱) 
共为细末,每小儿哺乳时,用药分许,置其口中,乳汁送下,一日约服药五次。 
此方以治小儿绵风或惊风,大抵皆效。而能因证制宜,再煮汤剂以送服此丹,则尤效。 
一小儿,生后数日即抽绵风。一日数次,两月不愈。为拟此方,服药数日而愈。所余之药,又治愈小儿三人。 
附录∶ 
宗弟张××喜用此丹以治小儿惊风。又恒随证之凉热虚实,作汤剂以送服此丹。其所用之汤药方,颇有可 
采。爰录其治验之原案二则于下。 
(原案一)己巳端阳前,友人黄××幼子,生六月,头身胎毒终未愈。禀质甚弱,忽肝风内动,抽掣绵绵不 
休。囟门微凸,按之甚软,微有赤色。指纹色紫为爪形。目睛昏而无神,或歪。脉浮小无根。此因虚气化不固, 
致肝阳上冲脑部扰及神经也。黄××云∶此证西医已诿为不治,不知尚有救否?答曰∶此证尚可为,听吾用药, 
当为竭力治愈。遂先用定风丹三分,水调灌下。继用生龙骨、生牡蛎、生石决明以潜其阳;钩藤钩、薄荷叶、 
羚羊角(锉细末三分)以熄其风;生箭 、生山药、山萸肉、西洋参以 
补其虚;清半夏、胆南星、粉甘草以开痰降逆和中。共煎汤多半杯,调入定风丹三分,频频灌之。二剂肝风止, 
又增损其方,四剂全愈。 
按∶黄 治小儿百病,明载《神农本草经》。惟此方用之,微有升阳之嫌。然《神农本草经》又谓其主大 
风,肝风因虚内动者,用之即能熄风可知。且与诸镇肝敛肝之药并用,若其分量止用二三钱,原有益而无损也。 
(原案二)天津聂姓幼子,生七月,夜间忽患肝风,抽动喘息,不知啼。时当仲夏,天气亢旱燥热。察 
其风关、气关纹红有爪形,脉数身热,知系肝风内动。急嘱其乳母,将小儿置床上,不致怀抱两热相并。又嘱 
其开窗,以通空气。先用急救回生丹吹入鼻中,以镇凉其脑系。遂灌以定风丹三分。又用薄荷叶、黄菊花、钩 
藤钩、栀子、羚羊角以散风清热,生龙骨、生牡蛎、生石决明以潜阳镇逆,天竹黄、牛蒡子、川贝母以利痰定 
喘。将药煎好,仍调入定风丹三分,嘱其作数次灌下,勿扰其睡。嗣来信,一剂风熄而病愈矣。 
按∶此二证,虽皆系肝风内动抽掣,而病因虚实迥异。张××皆治以定风丹,而其煎汤送服之药,因证各 
殊。如此善用成方,可为妙手灵心矣。 
又∶献县刘姓之婴孩,抽绵风不已,夜半询方。知病危急,适存有按小儿风证方所制定风丹,与以少许, 
服之立止,永未再犯。后屡用此方皆效。 
【附方】 鲍云韶《验方新编》预防小儿脐风散。方用枯矾、蓬砂各二钱半、朱砂二分、冰片、麝香各五 
厘,共为末。凡小儿降生后,洗过,即用此末擦脐上。每小儿换襁布时,仍擦此末。脐带落后,亦仍擦之。擦 
完一料,永无脐风之证。 
按∶此方最妙,愚用之多次皆效。真育婴之灵丹也。 


<篇名>2.镇风汤
属性:治小儿急惊风。其风猝然而得,四肢搐溺,身挺颈痉,神昏面热,或目睛上窜,或痰涎上壅,或牙关紧闭, 
或热汗淋漓。 
钩藤钩(三钱) 羚羊角(一钱,另炖兑服) 龙胆草(二钱) 青黛(二钱) 清半夏(二钱) 生赭石(二钱, 
轧细) 茯神(二钱) 僵蚕(二钱) 薄荷叶(一钱) 朱砂(二分,研细送服) 
磨浓生铁锈水煎药。 
小儿得此证者,不必皆由惊恐。有因外感之热,传入阳明而得者,方中宜加生石膏;有因热疟而得者,方 
中宜加生石膏、柴胡。 
急惊之外,又有所谓慢惊者,其证皆因寒,与急惊之因热者,有冰炭之殊。方书恒以一方治急、慢惊风二 
证,殊属差谬。慢惊之证,惟庄在田《福幼编》辨之最精,用方亦最妙。其辨慢惊风,共十四条∶一、慢惊吐 
泻,脾胃虚寒也。一、慢惊身冷,阳气抑遏不出也。一、慢惊鼻风煽动,真阴失守,虚火烧肺也。一、慢惊面 
色青黄及白,气血两虚也。一、慢惊口鼻中气冷,中寒也。一、慢惊大小便清白,肾与大肠全无火也。一、慢 
惊昏睡露睛,神气不足也。一、慢惊手足抽掣,血不行于四肢也。一、慢惊角弓反张,血虚筋急也。一、慢惊 
乍寒乍热,阴血虚少,阴阳错乱也。一、慢惊汗出如洗,阴虚而表不固也。一、慢惊手足螈 ,血不足养筋也。 
一、慢惊囟门下陷,虚至极也。一、慢惊身虽发热、口唇焦裂出血却不喜饮冷茶水,进以寒凉愈增危笃,以及 
所吐之乳、所泻之物皆不甚消化,脾胃无火可知。唇之焦黑,乃真阴之不足也明矣。 
脾风之证,亦小儿发痉之证,即方书所谓慢惊风也。因慢惊 
二字欠解,近世方书有改称慢脾风者,有但称脾风者。二名较之,似但称脾风较妥,因其证之起点由于脾胃虚 
寒也。盖小儿虽为少阳之体,而少阳实为稚阳,有若草木之萌芽,娇嫩畏寒。是以小儿或饮食起居多失于凉, 
或因有病过服凉药,或久疟、久痢,即不服凉药亦可因虚生凉,浸成脾风之证。其始也,因脾胃阳虚,寒饮凝 
滞于贲门之间,阻塞饮食不能下行,即下行亦不能消化,是以上吐而下泻。久之,则真阴虚损,可作灼热,其 
寒饮充盛,迫其身中之阳气外浮,亦可作灼热,浸至肝虚风动,累及脑气筋,遂至发痉,手足抽掣。此证庄在 
田《福幼编》论之最详,其所拟之逐寒荡惊汤及加味理中地黄汤二方亦最善。先用逐寒荡惊汤,大辛大热之剂, 
冲开胸中寒痰,可以受药不吐,然后接用加味理中地黄汤,诸证自愈。愚用其方救人多矣,而因证制宜又恒有 
所变通,方能随手奏效。 
【附方】 
(1)逐寒荡惊汤 
胡椒、炮姜、肉桂各一钱,丁香十粒,共捣成细渣。以灶心土三两煮汤,澄清,煎药大半茶杯(药皆捣碎, 
不可久煎,肉桂又忌久煎,三四沸即可),频频灌之。接服加味理中地黄汤,定获奇效。 
按∶此汤当以胡椒为君。若遇寒痰结胸之甚者,当用二钱,而稍陈者,又不堪用。族侄××六岁时,曾患 
此证。饮食下咽,胸膈格拒,须臾吐出。如此数日,昏睡露睛,身渐发热。投以逐寒荡惊汤原方,尽剂未吐。 
欲接服加味理中地黄汤,其吐又作。恍悟,此药取之乡间小药坊,其胡椒必陈。且只用一钱,其力亦小。遂于 
食料铺中,买胡椒二钱,炮姜、肉桂、丁香,仍按原方,煎服一剂。而寒痰开豁,可以受食。继服加味理中地 
黄汤,一剂而愈。 
方中所用灶心土,须为更改。凡草木之质,多含碱味。草木烧化,其碱味皆归灶心土中。若取其土煎汤, 
碱味浓浓,甚是难服,且与脾胃不宜。以灶圹内周遭火燎红色之土代之,则无碱味,其功效远胜于灶心土。 
(2)加味理中地黄汤 
熟地五钱,焦白术三钱,当归、党参、炙 、故纸(炒捣)、枣仁(炒捣)、枸杞各二钱,炮姜、萸肉 
(去净核)、炙草、肉桂各一钱,生姜三片,红枣三枚(捭开),胡桃二个(用仁)打碎为引。仍用灶心土 
(代以灶圹土)二两,煮水煎药。取浓汁一茶杯,加附子五分,煎水搀入。量小儿大小,分数次灌之。 
如咳嗽不止者,加米壳、金樱子各一钱。如大热不退者,加生白芍一钱。泄泻不止,去当归加丁香七粒。 
隔二三日,止用附子二三分。盖因附子大热,中病即宜去之。如用附子太多,则大小便闭塞不出。如不用附子, 
则脏腑沉寒,固结不开。若小儿虚寒至极,附子又不妨用一二钱。若小儿但泻不止,或微见惊搐,尚可受药吃 
乳便利者,并不必服逐寒荡惊汤,只服此汤一剂,而风定神清矣。若小儿尚未成慢惊,不过昏睡发热,或有时 
热止,或昼间安静,夜间发热,均宜服之。若新病壮实之小儿,眼红口渴者,乃实火之证,方可暂行清解。但 
果系实火,必大便闭结,气壮声洪,且喜多饮凉水。若吐泻交作,则非实火可知。倘大虚之后,服一剂无效, 
必须大剂多服为妙。方书所谓天吊风、慢脾风皆系此证。 
按∶此原方加减治泻不止者,但加丁香,不去当归。而当归最能滑肠,泻不止者,实不宜用。若减去当 
归,恐滋阴之药少,可多加熟地一二钱(又服药泻仍不止者,可用高丽参二钱捣为末,分数次用药汤送服,其 
泻必止。)。 
慢惊风,不但形状可辨,即其脉亦可辨。族侄××七八岁时,疟疾愈后,忽然吐泻交作。时霍乱盛行,其 
家人皆以为霍乱 
证。诊其脉弦细而迟,六脉皆不闭塞。愚曰∶此非霍乱。吐泻带有粘涎否?其家人谓偶有带时。愚曰∶此寒痰 
结胸,格拒饮食,乃慢惊风将成之兆也。投以逐寒荡惊汤、加味理中地黄汤各一剂而愈。 
又∶此二汤治慢惊风,虽甚效验。然治此证者,又当防之于预,乃为完全之策。一孺子,年五六岁,秋夏 
之交,恣食瓜果当饭。至秋末,其行动甚迟,正行之时,或委坐于地。愚偶见之,遂恳切告其家人曰∶此乃慢 
惊风之先兆也。小儿慢惊风证,最为危险,而此时调治甚易,服药两三剂,即无患矣。其家人不以为然。至冬 
初,慢惊之形状发现,呕吐不能受食,又不即治。迁延半月,病势垂危,始欲调治。而服药竟无效矣。 
又∶有状类急惊,而病因实近于慢惊者。一童子,年十一二,咽喉溃烂。医者用吹喉药吹之,数日就愈。 
忽然身挺,四肢搐搦,不省人事,移时始醒,一日数次。诊其脉甚迟濡。询其心中,虽不觉凉,实畏食凉物。 
其呼吸,似觉短气。时当仲夏,以童子而畏食凉,且征以脉象病情,其为寒痰凝结,瘀塞经络无疑。投以《伤 
寒论》白通汤,一剂全愈。 
又∶治一五岁幼童,先治以逐寒荡惊汤,可进饮食矣,而滑泻殊甚。继投以加味理中地黄汤,一日连进两 
剂,泄泻不止,连所服之药亦皆泻出。遂改用红高丽参大者一支,轧为细末,又用生怀山药细末六钱煮作粥, 
送服参末一钱强。如此日服三次,其泻遂止。翌日仍用此方,恐作胀满,又于所服粥中调入西药百布圣六分, 
如此服至三日,病全愈。 
又治一未周岁小孩,食乳即吐,屡次服药亦吐出,囟门下陷,睡时露晴,将成脾风。俾其于每吃乳时,用 
生硫黄细末一捻,置儿口中,乳汁送下,其吐渐稀,旬日全愈。 

\chapter{治内外中风方}
<篇名>1.搜风汤
属性:治中风 
防风(六钱) 真辽人参(四钱,另炖同服,或用野台参七钱代之,高丽参不宜用) 清半夏(三钱) 
生石膏(八钱) 僵蚕(二钱) 柿霜饼(五钱,冲服) 麝香(一分,药汁送服) 
中风之证,多因五内大虚,或秉赋素虚,或劳力劳神过度,风自经络袭入,直透膜原而达脏腑,令脏腑各 
失其职。或猝然昏倒,或言语謇涩,或溲便不利,或溲便不觉,或兼肢体痿废偏枯,此乃至险之证。中之轻者, 
犹可迟延岁月,中之重者,治不如法,危在翘足间也。故重用防风引以麝香,深入脏腑以搜风。犹恐元气虚弱, 
不能运化药力以逐风外出,故用人参以大补元气,扶正即以胜邪也。用石膏者,因风蕴脏腑多生内热,人参补 
气助阳分亦能生热,石膏质重气轻性复微寒,其重也能深入脏腑,其轻也能外达皮毛,其寒也能祛脏腑之热, 
而即解人参之热也。用僵蚕者,徐灵胎谓邪之中人,有气无形,穿经入络,愈久愈深,以气类相反之药投之则 
拒而不入,必得与之同类者和入诸药使为向导,则药至病所,而邪与药相从,药性渐发,邪或从毛孔出,从二 
便出,不能复留,此从治之法也。僵蚕因风而僵,与风为同类,故善引祛风之药至于病所成功也。用半夏、柿 
霜者,诚以此证皆痰涎壅滞,有半夏以降之,柿霜以润之,而痰涎自息也。 
此证有表不解,而浸生内热者,宜急用发汗药,解其表,而兼清其内热。又兼有内风煽动者,可与后内 
中风治法汇通参观, 
于治外感之中兼有熄内风之药,方为完善。 
中风之证,有偏寒者,有偏热者,有不觉寒热者。拙拟此方治中风之无甚寒热者也。若偏热者,宜《金 
匮》风引汤加减(干姜、桂枝宜减半)。若偏寒者,愚别有经验治法。曾治一媪,年五十许,于仲冬忽然中风 
昏倒,呼之不应,其胸中似有痰涎壅滞,大碍呼吸。诊其脉,微细欲无,且迟缓,知其素有寒饮,陡然风寒袭 
入,与寒饮凝结为恙也。急用胡椒三钱捣碎,煎两三沸,取浓汁多半茶杯灌之,呼吸顿觉顺利。继用干姜六钱, 
桂枝尖、当归各三钱,连服三剂,可作呻吟,肢体渐能运动,而左手足仍不能动。又将干姜减半,加生黄 五 
钱,乳香、没药各三钱,连服十余剂,言语行动遂复其常。 
若其人元气不虚,而偶为邪风所中,可去人参,加蜈蚣一条、全蝎一钱。若其证甚实,而闭塞太甚者,或 
二便不通,或脉象郁涩,可加生大黄数钱,内通外散,仿防风通圣散之意可也。 
徐灵胎曾治一人,平素多痰,手足麻木,忽昏厥遗尿、口噤手拳、痰声如锯。医者进参附、熟地等药, 
煎成末服。诊其脉,洪大有力,面赤气粗。此乃痰火充实,诸窍皆闭,服参附立危。遂以小续命汤去桂附,加 
生军一钱为末,假称他药纳之,恐旁人之疑骇也。三剂而有声,五剂而能言。然后以养血消痰之药调之,一月 
后,步履如初。此案与愚所治之案对观,则凉热之间昭然矣。又遗尿者多属虚,而此案中之遗尿则为实,是知 
审证者,不可拘于一端也。然真中风证极少,类中风者极多,中风证百人之中真中风不过一二人。审证不确即 
凶危立见,此又不可不慎也。 


<篇名>2.逐风汤
属性:治中风抽掣及破伤后受风抽掣者。 
生箭 (六钱) 当归(四钱) 羌活(二钱) 独活(二钱) 
全蝎(二钱) 全蜈蚣(大者两条) 
蜈蚣最善搜风,贯串经络、脏腑无所不至,调安神经又具特长。而其性甚和平,从未有服之觉瞑眩者。 
曾治一媪,年六旬。其腿为狗咬破受风,周身抽掣。延一老医调治,服药十余日,抽掣愈甚。所用之药, 
每剂中皆有全蝎数钱,佐以祛风、活血、助气之药,仿佛此汤而独未用蜈蚣。遂为拟此汤,服一剂而抽掣即止。 
又服一剂,永不反复。 
又治一人,年三十余,陡然口眼歪斜,其受病之边,目不能瞬。俾用蜈蚣二条为末、防风五钱,煎汤送 
服,三次全愈。审斯,则蜈蚣逐风之力,原迥异于他药也。且其功效,不但治风也,愚于疮痈初起甚剧者,恒 
加蜈蚣于托药之中,莫不随手奏效。虽《神农本草经》谓有坠胎之弊,而中风抽掣,服他药不效者,原不妨用。 
《内经》所谓“有故无殒,亦无殒也”。况此汤中,又有黄 、当归以保摄气血,则用分毫何损哉。 


<篇名>3.加味黄五物汤
属性:治历节风证,周身关节皆疼,或但四肢作疼,足不能行步,手不能持物。 
生箭 (一两) 于术(五钱) 当归(五钱) 桂枝尖(三钱) 秦艽(三钱) 广陈皮(三钱) 
生杭芍(五钱) 生姜(五片) 
热者加知母,凉者加附子,脉滑有痰者加半夏。 
《金匮》桂枝芍药知母汤,治历节风之善方也。而气体虚者用之,仍有不效之时,以其不胜麻黄、防风 
之发也。今取《金匮》治风痹之黄 五物汤,加白术以健脾补气,而即以逐痹(《神农本草经》逐寒湿痹)。 
当归以生其血,血活自能散风(方书谓血活风自去)。秦艽为散风之润药,性甚和平,祛风而不伤血。陈皮为 
黄之佐使,又能引肌肉经络之风达皮肤由毛孔而出也。广橘红其大者皆柚也,非 
橘也。《神农本草经》原橘、柚并称,故用于药中,橘、柚似无须分别(他处柚皮不可入药)。且名为橘红, 
其实皆不去白,诚以原不宜去也。 
附录∶ 
直隶青县张××来函∶ 
湖北张某,患历节风证,西医名偻麻质斯,服其药年余无效。步履艰难,天未凉即着皮裤。诊其脉,浮 
数有力,知为经络虚而有热之象。遂用加味黄 五物汤,遵注热者加知母,又加生薏米、鲜桑枝、牛膝、木通。 
服一剂觉轻减,三剂离杖,五剂痊愈。近年用此方治痛风、历节证,愈者甚多。若无热者,即用书中原方,亦 
甚效验。 
江苏平台王××来函∶ 
顾××,患肢体痿废,时当溽暑,遍延中西医延医无效。用加味黄 五物汤治之,连服数剂全愈。 


<篇名>4.加味玉屏风散
属性:治破伤后预防中风,或已中风而螈 ,或因伤后房事不戒以致中风。 
生箭 (一两) 白术(八钱) 当归(六钱) 桂枝尖(钱半) 防风(钱半) 黄蜡 
(三钱) 生白矾(一钱) 作汤服。 
此方原为预防中风之药,故用黄 以固皮毛,白术以实肌肉,黄蜡、白矾以护膜原。犹恐破伤时微有感 
冒,故又用当归、防风、桂枝以活血散风。其防风、桂枝之分量特轻者,诚以此方原为预防中风而设,故不欲 
重用发汗之药以开腠理也。 
盖《神农本草经》原谓黄 主大风,方中重用黄 一两,又有他药以为之佐使,宜其风证皆可治也。若 
已中风抽掣者,宜加全蜈蚣两条。若更因房事不戒以致中风抽风者,宜再加真鹿角胶三钱 
(另煎兑服),独活一钱半。若脉象有热者,用此汤时,知母、天冬皆可酌加。 
自拟此方以来,凡破伤后恐中风者,俾服药一剂,永无意外之变,用之数十年矣。 
表侄高××之族人,被人用枪弹击透手心,中风抽掣,牙关紧闭。自牙缝连灌药无效,势已垂危。从前,其 
庄有因破伤预防中风,服此方者,高××见而录之。至此,高××将此方授族人,一剂而愈。 
又一人,被伤后,因房事不戒,中风抽掣,服药不效。友人毛××治之,亦投以此汤而愈。夫愚拟此方, 
原但为预防中风,而竟如此多效,此愚所不及料者也。 


<篇名>5.镇肝熄风汤
属性:治内中风证(亦名类中风,即西人所谓脑充血证),其脉弦长有力(即西医所谓血压过高),或上盛下虚, 
头目时常眩晕,或脑中时常作疼发热,或目胀耳鸣,或心中烦热,或时常噫气,或肢体渐觉不利,或口眼渐形 
歪斜,或面色如醉,甚或眩晕,至于颠仆,昏不知人,移时始醒,或醒后不能撤消,精神短少,或肢体痿废, 
或成偏枯。 
怀牛膝(一两) 生赭石(一两,轧细) 生龙骨(五钱,捣碎) 生牡蛎(五钱,捣碎) 生龟板 
(五钱,捣碎) 生杭芍(五钱) 玄参(五钱) 天冬(五钱) 川楝子(二钱,捣碎) 生麦芽(二钱) 
茵陈(二钱) 甘草(钱半) 
心中热甚者,加生石膏一两。痰多者,加胆星二钱。尺脉重按虚者,加熟地黄八钱、净萸肉五钱。大便不 
实者,去龟板、赭石,加赤石脂(喻嘉言谓石脂可代赭石)一两。 
风名内中,言风自内生,非风自外来也。《内经》谓“诸风掉眩,皆属于肝”。盖肝为木脏,木火炽盛, 
亦自有风。此因肝木失和风自肝起。又加以肺气不降,肾气不摄,冲气胃气又复上逆,于斯,脏腑之气化皆上 
升太过,而血之上注于脑者,亦因之 
太过,致充塞其血管而累及神经。其甚者,致令神经失其所司,至昏厥不省人事。西医名为脑充血证,诚由剖 
解实验而得也。是以方中重用牛膝以引血下行,此为治标之主药。而复深究病之本源,用龙骨、牡蛎、龟板、 
芍药以镇熄肝风,赭石以降胃降冲,玄参、天冬以清肺气,肺中清肃之气下行,自能镇制肝木。至其脉之两尺虚 
者,当系肾脏真阴虚损,不能与真阳相维系。其真阳脱而上奔,并挟气血以上冲脑部,故又加熟地、萸肉以补 
肾敛肾。从前所拟之方,原止此数味。后因用此方效者固多,间有初次将药服下转觉气血上攻而病加剧者,于 
斯加生麦芽、茵陈、川楝子即无斯弊。盖肝为将军之官,其性刚果,若但用药强制,或转激发其反动之力。茵陈 
为青蒿之嫩者,得初春少阳生发之气,与肝木同气相求,泻肝热兼舒肝郁,实能将顺肝木之性。麦芽为谷之 
萌芽,生用之亦善将顺肝木之性使不抑郁。川楝子善引肝气下达,又能折其反动之力。方中加此三味,而后用此 
方者,自无他虞也。心中热甚者,当有外感,伏气化热,故加石膏。有痰者,恐痰阻气化之升降,故加胆星也。 
内中风之证,曾见于《内经》。而《内经》初不名为内中风,亦不名为脑充血,而实名之为煎厥、大厥、 
薄厥。今试译《内经》之文以明之。《内经》脉解篇曰∶“肝气当治而未得,故善怒,善怒者名曰煎厥。”盖肝 
为将军之官,不治则易怒,因怒生热,煎耗肝血,遂致肝中所寄之相火,掀然暴发,挟气血而上冲脑部,以致 
昏厥。此非因肝风内动,而遂为内中风之由来乎? 
《内经》调经论曰∶“血之与气,并走于上,此为大厥,厥则暴死。气反则生,气不反则死。”盖血不 
自升,必随气而上升,上升之极,必至脑中充血。至所谓气反则生,气不反则死者,盖气反而下行,血即随之 
下行,故其人可生。若其气上行不反,血必随之充而益充,不至血管破裂不止,犹能望其复苏乎。 
读此节经文,内中风之理明,脑充血之理亦明矣。 
《内经》生气通天论曰∶“阳气者大怒则形绝,血宛(即郁字)于上,使人薄厥。”观此节经文,不待诠 
解,即知其为肝风内动,以致脑充血也。其曰薄厥者,言其脑中所宛之血,激薄其脑部,以至于昏厥也。细思 
三节经文,不但知内中风即西医所谓脑充血,且更可悟得此证治法,于经文之中,不难自拟对证之方,而用之 
必效也。 
特是证名内中风,所以别外受之风也。乃自唐、宋以来,不论风之外受、内生,浑名曰中风。夫外受之风 
为真中风,内生之风为类中风,其病因悬殊,治法自难从同。若辨证不清,本系内中风,而亦以祛风之药发表 
之,其脏腑之血,必益随发表之药上升,则脑中充血必益甚,或至于血管破裂,不可救药。此关未透,诚唐、 
宋医学家一大障碍也。迨至宋末刘河间出,悟得风非皆由外中,遂创为五志过极动火而猝中之论,此诚由《内 
经》“诸风掉眩皆属于肝”句悟出。盖肝属木,中藏相火,木盛火炽,即能生风也。大法,以白虎汤、三黄汤 
沃之,所以治实火也。以逍遥散疏之,所以治郁火也(逍遥散中柴胡能引血上行最为忌用,是以镇肝熄风汤中 
止用茵陈、生麦芽诸药疏肝)。以通圣散(方中防风亦不宜用)、凉膈散双解之,所以治表里之邪火也。以六 
味汤滋之,所以壮水之主,以制阳光也。以八味丸引之,所谓从治之法,引火归源也(虽曰引火归源,而桂、 
附终不宜用)。细审河间所用之方,虽不能丝丝入扣,然胜于但知治中风不知分内外者远矣。且其谓有实热者, 
宜治以白虎汤,尤为精确之论。愚治此证多次,其昏仆之后,能自苏醒者多,不能苏醒者少。其于苏醒之后, 
三四日间,现白虎汤证者,恒十居六七。因知此证,多先有中风基础,伏藏于内,后因外感而激发,是以从前 
医家,统名为中风。不知内风之动,虽由于外感之激发,然非激发于外感之风,实激发于外感之因风生热,内 
外两热相并,遂致内风暴动。此时但宜治外感之热,不可再散外 
感之风,此所以河间独借用白虎汤,以泻外感之实热,而于麻桂诸药概无所用。盖发表之药,皆能助血上行, 
是以不用,此诚河间之特识也。吾友张山雷(江苏嘉定人),着有《中风 诠》一书,发明内中风之证,甚为 
精详。书中亦独有取于河间,可与拙论参观矣。 
后至元李东垣、朱丹溪出,对于内中风一证,于河间之外,又创为主气、主湿之说。东垣谓人之元气不 
足,则邪凑之,令人猝倒僵仆,如风状。夫人身之血,原随气流行,气之上升者过多,可使脑部充血,排挤脑 
髓神经。至于昏厥,前所引《内经》三节文中已言之详矣。若气之上升者过少,又可使脑部贫血,无以养其脑 
髓神经,亦可至于昏厥。是以《内经》又谓∶“上气不足,脑为之不满,耳为之苦鸣,头为之倾,目为之眩。” 
观《内经》如此云云,其剧者,亦可至于昏厥,且其谓脑为之不满,实即指脑中贫血而言也。由斯而论,东垣 
之论内中风,由于气虚邪凑,原于脑充血者之中风无关,而实为脑贫血者之中风,开其治法也。是则河间之主 
火,为脑充血,东垣之主气,为脑贫血,一实一虚,迥不同也。至于丹溪则谓东南气温多湿,有病风者,非风 
也,由湿生痰,痰生热,热生风,此方书论中风者所谓丹溪主湿之说也。然其证原是痰厥,与脑充血、脑贫血 
皆无涉。即使二证当昏厥之时,间有挟痰者,乃二证之兼证,非二证之本病也。 
按∶其所谓因热生风之见解,似与河间主火之意相同,而实则迥异。盖河间所论之火生于燥,故所用之 
药,注重润燥滋阴。丹溪所论之热生于湿,其所用之药,注重去湿利痰。夫湿非不可以生热,然因湿生热,而 
动肝风者甚少矣(肝风之动多因有燥热)。是则二子之说,仍以河间为长也。 
至清中叶王勋臣出,对于此证,专以气虚立论。谓人之元气,全体原十分,有时损去五分,所余五分,虽 
不能充体,犹可支持全身。而气虚者经络必虚,有时气从经络虚处通过,并于一 
边,彼无气之边,即成偏枯。爰立补阳还五汤,方中重用黄 四两,以峻补气分,此即东垣主气之说也。然王 
氏书中,未言脉象何如。若遇脉之虚而无力者,用其方原可见效。若其脉象实而有力,其人脑中多患充血,而 
复用黄 之温而升补者,以助其血愈上行,必至凶危立见,此固不可不慎也。前者邑中某人,右手废不能动, 
足仍能行。其孙出门,遇一在津业医者甫归,言此证甚属易治,遂延之诊视。所立病案言脉象洪实,已成痪证 
无疑。其方仿王氏补阳还五汤,有黄 八钱。服药之后,须臾昏厥不醒矣。夫病本无性命之忧,而误服黄 八 
钱,竟至如此,可不慎哉! 
刘××丁卯来津后,其脑中常觉发热,时或眩晕,心中烦躁不宁,脉象弦长有力,左右皆然,知系脑充血证。 
盖其愤激填胸,焦思积虑者已久,是以有斯证也。为其脑中觉热,俾用绿豆实于囊中作枕,为外治之法。又治 
以镇肝熄风汤,于方中加地黄一两,连服数剂,脑中已不觉热。遂去川楝子,又将生地黄改用六钱,服过旬日, 
脉象和平,心中亦不烦躁,遂将药停服。 
又∶天津于氏所娶新妇,过门旬余,忽然头疼。医者疑其受风,投以发表之剂,其疼陡剧,号呼不止。延 
愚为之诊视。其脉弦硬而长,左部尤甚。知其肝胆之火上冲过甚也。遂投以镇肝熄风汤,加龙胆草三钱,以泻 
其肝胆之火。一剂病愈强半,又服两剂,头已不疼,而脉象仍然有力。遂去龙胆草,加生地黄六钱,又服数剂, 
脉象如常,遂将药停服。 


<篇名>6.加味补血汤
属性:治身形软弱,肢体渐觉不遂,或头重目眩,或神昏健忘,或觉脑际紧缩作疼。甚或昏仆移时苏醒致成偏枯, 
或全身痿废,脉象迟弱,内中风证之偏虚寒者(肝过盛生风,肝虚极亦可生风),此即西人所谓脑贫血病也。 
久服此汤当愈。 
生箭 (一两) 当归(五钱) 龙眼肉(五钱) 真鹿角胶(三钱,另炖同服) 丹参(三钱) 明乳香 
(三钱) 明没药(三钱) 甘松(二钱) 
服之觉热者,酌加天花粉、天冬各数钱。觉发闷者,加生鸡内金钱半或二钱。服数剂后,若不甚见效, 
可用所煎药汤送服麝香二厘或真冰片半分亦可。若服后仍无甚效,可用药汤,送制好马钱子二分。 
脑充血者,其脑中之血过多,固能伤其脑髓神经。脑贫血者其脑中之血过少,又无以养其脑髓神经。是 
以究其终极,皆可使神经失其所司也。古方有补血汤,其方黄 、当归同用,而黄 之分量,竟四倍于当归, 
诚以阴阳互为之根,人之气壮旺者,其血分自易充长。况人之脑髓神经,虽赖血以养之,尤赖胸中大气上升以 
斡旋之。是以《内经》谓“上气不足,脑为之不满,耳为之苦鸣,头为之倾,目为之眩。”所谓上气者,即胸 
中大气上升于脑中者也。因上气不足,血之随气而注于脑者必少,而脑为之不满,其脑中贫血可知。且因上气 
不足,不能斡旋其神经,血之注于脑者少,无以养其神经,于是而耳鸣、头倾、目眩,其人可忽至昏仆可知。 
由此知因脑部贫血以成内中风证者,原当峻补其胸中大气,俾大气充足,自能助血上升,且能斡旋其脑部,使 
不至耳鸣、头倾、目眩也。是以此方不以当归为主药,而以黄 为主药也。用龙眼肉者,因其味甘色赤,多含 
津液,最能助当归以生血也。用鹿角胶者,因鹿之角原生于头顶督脉之上,督脉为脑髓之来源,故鹿角胶之性 
善补脑髓。凡脑中血虚者,其脑髓亦必虚,用之以补脑髓,实可与补血之药相助为理也。用丹参、乳香、没药 
者,因气血虚者,其经络多瘀滞,此于偏枯痿废亦颇有关系,加此通气活血之品,以化其经络之瘀滞,则偏枯 
痿废者自易愈也。用甘松者,为其能助心房运动有力,以多输血于脑,且又为调养神经之要品,能引诸药至脑 
以调养其神经也。用麝香、梅 
片者,取其香能通窍以开闭也。用制过马钱子者,取其能 动脑髓神经使之灵活也。 
甘松,即西药中之缬草,其气香,味微酸。《神农本草经》谓其治暴热、火疮、赤气、疥瘙、疽痔、马鞍、 
热气。《名医别录》谓其治痈肿、浮肿、结热、风痹、不足、产后痛。甄权谓其治毒风, 痹,破多年凝血, 
能化脓为水,产后诸病,止腹痛、余疹、烦渴。大明谓其除血气心腹痛、破 结、催生、落胞、血晕、鼻血、 
吐血、赤白带下、眼障膜、丹毒、排脓、补痿。西人则以为兴奋之品,善治心脏麻痹、霍乱转筋。东人又以为 
镇静神经之特效药,用治癫狂、痫痉诸病。盖为其气香,故善兴奋心脏,使不至于麻痹,而其馨香透窍之力, 
亦自能开痹通瘀也。为其味酸,故能保安神经,使不至于妄行,而酸化软坚之力,又自能化多年之 结,使尽 
消融也。至于其能补痿,能治霍乱转筋者,即心脏不麻痹,神经不妄行之功效外着者也。孰谓中西医理不相贯通哉? 
高姓臾,年过六旬,渐觉两腿乏力,浸至时欲眩仆,神昏健忘。恐成痿废,求为延医。其脉微弱无力。为 
制此方服之,连进十剂,两腿较前有力,健忘亦见愈,而仍有眩晕之时。再诊其脉,虽有起色,而仍不任重按。 
遂于方中加野台参、天门冬各五钱,威灵仙一钱,连服二十余剂始愈。用威灵仙者,欲其运化参、 之补力, 
使之灵活也。 
门人张××曾治一人,年三十余。于季冬负重贸易,日行百余里,歇息时,又屡坐寒地。后觉腿疼,不能 
行步,浸至卧床不能动转,周身筋骨似皆痿废,服诸药皆不效。张××治以加味补血汤,将方中乳香、没药皆 
改用六钱,又加净萸肉一两。数剂后,腿即不疼。又服十余剂,遂全愈。 
按∶加味补血汤,原治内中风之气血两亏者,而略为变通, 
即治腿疼如此效验,可谓善用成方者矣。 

\chapter{治肢体痿废方}
<篇名>1.振颓汤
属性:治痿废。 
生黄 (六钱) 知母(四钱) 野台参(三钱) 于术(三钱) 当归(三钱) 生明乳香(三钱) 生明 
没药(三钱) 威灵仙(钱半) 干姜(二钱) 牛膝(四钱) 
热者,加生石膏数钱,或至两许。寒者去知母,加乌附子数钱。筋骨受风者,加明天麻数钱。脉弦硬而大 
者,加龙骨、牡蛎各数钱,或更加山萸肉亦佳。骨痿废者,加鹿角胶、虎骨胶各二钱(另炖同服)。然二胶伪 
者甚多,若恐其伪,可用续断、菟丝子各三钱代之。手足皆痿者,加桂枝尖二钱。 
痿证之大旨,当分为三端∶有肌肉痹木,抑搔不知疼痒者。其人或风寒袭入经络,或痰涎郁塞经络,或 
风寒痰涎,互相凝结经络之间,以致血脉闭塞,而其原因,实由于胸中大气虚损。盖大气旺,则全体充盛,气 
化流通,风寒痰涎,皆不能为恙。大气虚,则腠理不固,而风寒易受,脉管湮淤,而痰涎易郁矣;有周身之筋 
拘挛,而不能伸者。盖人身之筋,以宗筋为主,而能荣养宗筋者,阳明也。其人脾胃素弱,不能化谷生液,以 
荣养宗筋,更兼内有蕴热以铄耗之,或更为风寒所袭,致宗筋之伸缩自由者,竟有缩无伸,浸成拘挛矣;有筋 
非拘挛,肌肉非痹木,惟觉骨软不能履地者,乃骨髓枯涸,肾虚不能作强也。 
方中用黄 以补大气,白术以健脾胃,当归、乳香、没药以流通血脉,灵仙以祛风消痰,恐其性偏走泄, 
而以人参之气血兼补者 
佐之,干姜以开气血之痹,知母以解干姜、人参之热,则药性和平,可久服而无弊。其阳明有实热者,加石膏 
以清阳明之热,仿《金匮》风引汤之义也。营卫经络有凝寒者,加附子以解营卫经络之寒,仿《金匮》近效术 
附汤之义也。至其脉弦硬而大,乃内风煽动,真气不固之象,故加龙骨、牡蛎以熄内风敛真气。骨痿者加鹿胶、 
虎胶取其以骨补骨也。筋骨受风者,加明天麻取其能搜筋骨之风,又能补益筋骨也。若其痿专在于腿,可但用 
牛膝以引之下行。若其人手足并痿者,又宜加桂枝兼引之上行。 
或问∶此方治痿之因热者,可加生石膏至两许,其证有实热可知,而方中仍用干姜何也?答曰∶《金匮》 
风引汤治热瘫痫之的方,原石膏、寒水石与干姜并用。盖二石性虽寒而味则淡,其寒也能胜干姜之热,其淡也 
不能胜干姜之辣。故痿证之因热者,仍可借其异常之辣味,以开气血之痹也。 


<篇名>2.振颓丸
属性:前证之剧者,可兼服此丸,或单服此丸亦可。并治偏枯、痹木诸证。 
人参(二两) 于术(二两炒) 当归(一两) 马钱子(一两,法制) 乳香(一两) 没药(一两) 
全蜈蚣(大者五条,不用炙) 穿山甲(一两,蛤粉炒) 
共轧细过罗,炼蜜为丸如桐子大。每服二钱,无灰温酒送下,日再服。 
马钱子,即番本鳖,其毒甚烈,而其毛与皮尤毒。然治之有法,则有毒者,可至无毒。而其开通经络, 
透达关节之力,实远胜于他药也。今将制马钱子法,详载于下。 
法∶将马钱子先去净毛,水煮两三沸即捞出。用刀将外皮皆刮净,浸热汤中,旦、暮各换汤一次,浸足 
三昼夜取出。再用香 
油煎至纯黑色,擘开视其中心微有黄意,火候即到。将马钱子捞出,用温水洗数次,将油洗净。再用沙土,同 
入锅内炒之,土有油气,换土再炒,以油气尽净为度。 
马钱子为健胃妙药。马钱子性虽有毒,若制至无毒,服之可使全身 动,以治肢体麻痹(此兴奋神经之作 
用);若少少服之,但令胃腑 动有力,则胃中之食必速消。此非但凭理想,实有所见而云然也。沧州朱媪, 
年过六旬,素有痫风证,医治数十年,先服中药无效,继服西药麻醉脑筋之品,虽见效,然必日日服之始能强 
制不发。因诸药性皆咸寒,久服伤胃,浸至食量减少,身体羸弱。后有人授以王勋臣龙马自来丹方,其方原以 
马钱子为主药,如法制好,服之数日,食量顿增,旬余身体渐壮,痫病虽未即除根,而已大轻减矣。由斯知马 
钱子健胃之功效迥异乎他药也。 
特是龙马自来丹,马钱子伍以地龙,为治痫风设也。若用以健胃,宜去地龙,加炒白术细末,其健胃之效 
益着。爰拟定其方于下∶ 
炒白术(四两,细末) 制好马钱子(一两,细末) 
二药调勺,水和为丸一分重(干透足一分),饭后服五丸,一日再服,旬余自见功效。 


<篇名>3.姜胶膏
属性:治肢体受凉疼痛,或有凝寒阻遏血脉,麻木不仁。 
鲜姜自然汁(一斤) 明亮水胶(四两) 
上二味同熬成稀膏,摊于布上,贴患处,旬日一换。凡因受寒肢体疼痛,或因受寒肌肉麻木不仁者,贴 
之皆可治愈。即因受风,而筋骨疼痛,或肌肉麻木者,贴之亦可治愈。惟有热肿疼者,则断不可用。 
盖此等证心中无病,原宜外治。鲜姜之辛辣开通,热而能散,故能温暖肌肉,深透筋骨,以除其凝寒痼 
冷,而涣然若冰释 
也。用水胶者,借其粘滞之力,然后可熬之成膏也。若证因受风而得者,拟用细辛细末掺于膏药之中,或用他 
祛风猛悍之药,掺于其中,其奏效当更捷也。 
有人因寝凉炕之上,其右腿外侧时常觉凉,且有时疼痛。用多方治之不效。语以此方,贴至二十日全愈。 
又有人常在寒水中捕鱼,为寒水所伤。自膝下被水浸处皆麻木,抑搔不知疼痒,渐觉行动乏力。语以此方, 
俾用长条布摊药膏缠于腿上,其足趺、足底皆贴以此膏,亦数换而愈。 

\chapter{治膈食方}
<篇名>参赭培气汤
属性:治膈食(“论胃病噎膈治法及反胃治法”宜参看)。 
潞党参(六钱) 天门冬(四钱) 生赭石(八钱,轧细) 清半夏(三钱) 淡苁蓉(四钱) 知母(五钱) 
当归身(三钱) 柿霜饼(五钱,服药后含化徐徐咽之) 
人之一身,自飞门以至魄门,一气主之,亦一气悬之。故人之中气充盛,则其贲门(胃之上口)宽展,自 
能容受水谷,下通幽门(胃之下口)以及小肠、大肠,出为二便,病何由而作?若中气衰惫,不能撑悬于内, 
则贲门缩小,以及幽门、小肠、大肠皆为之紧缩。观膈证之病剧者,大便如羊矢,固因液短,实亦肠细也。况 
中气不旺,胃气不能息息下降,而冲气转因胃气不降,而乘虚上干,致痰涎亦随逆气上并,以壅塞贲门。夫此 
时贲门已缩如藕孔,又加逆气痰涎以壅塞其间,又焉能受饮食以下达乎?故治此证者,当以大补中气为主,方 
中之人参是也。以降逆安冲为佐,以清痰理气为使,方中之赭石、半夏、柿霜是也。又虑人参性热、半夏性燥, 
故又加知母、天冬、当归、柿霜、以清热润燥、生津生血也。用苁蓉者,以其能补肾,即能敛冲,冲气不上冲, 
则胃气易于下降。且患此证者,多有便难之虞,苁蓉与当归、赭石并用,其润便通结之功,又甚效也。若服数 
剂无大效,当系贲门有瘀血,宜加三棱、桃仁各二钱。 
仲景《伤寒论》有旋复代赭石汤,原治伤寒发汗,若吐若下解后,心下痞硬,噫气不除者。周扬俊、喻嘉 
言皆谓治膈证甚效。拙拟此方,重用赭石,不用旋复花者,因旋复花《神农本草经》原言味咸,今坊间所鬻旋 
复花,苦而不咸,用之似无效验。惟邑武帝台为汉武帝筑台望海之处,地多咸卤,周遭所产旋复花,大于坊间 
鬻者几一倍。其味咸而兼辛,以治膈食甚效。 
或问∶《神农本草经》旋复花,未言苦亦未言辛。药坊之苦者,既与《神农本草经》之气味不合,岂武帝 
台之辛者,独与《神农本草经》之气味合乎?答曰∶古人立言尚简,多有互文以见义者。《神农本草经》为有 
文本后第一书,其简之又简可知。故读《神农本草经》之法,其主治未全者,当于气味中求之;其气味未全 
者,即可于主治中求之。旋复花《神农本草经》载其主结气,胁下满,惊悸、除水、去五脏间寒热,补中下气。 
三复《神农本草经》主治之文,则复花当为平肝降气之要药,应借辛味,以镇肝木,其味宜咸而兼辛明矣。至 
于苦味,性多令人涌吐,是以旋复花不宜兼此味也。其味不至甚苦,亦可斟酌加入也。 
一叟,年六十余得膈证,向愚求方。自言犹能细嚼焦脆之物,用汤水徐徐送下,然一口咽之不顺,即呕吐 
不能再食,且呕吐之时,带出痰涎若干。诊其脉关后微弱,关前又似滑实,知其上焦痰涎壅滞也。用此汤加邑 
武帝台所产旋复花二钱,连服四剂而愈。 
一人,年四十六,素耽叶子戏,至废寝食。初觉有气上冲咽喉,浸至妨碍饮食,时或呕吐不能下行。其脉 
弦长而硬,左右皆 
然。知系冲气挟胃气上冲。治以此汤,加武帝台旋复花二钱、生芡实四钱,降其冲逆之气而收敛之,连服十剂而愈。 
族家姑,年五旬有六,初觉饮食有碍,后浸增重,惟进薄粥,其脉弦细无力。盖生平勤俭持家,自奉甚薄, 
劳心劳力又甚过。其脉之细也,因饮食菲薄而气血衰,其脉之弦也,因劳心过度而痰饮盛也。姑上有两姊,皆 
以此疾逝世,气同者其病亦同,惴惴自恐不愈。愚毅然以为可治,投以此汤,加白术二钱、龙眼肉三钱,连服 
十余剂全愈。 
堂侄女,年四十八岁,素羸弱多病。因自理家务,劳心过度,恒彻夜不寐。于癸卯夏日得膈证。时愚远出, 
遂延他医调治,屡次无效。及愚旋里,病势已剧。其脉略似滑实,重按无力。治以此汤,加龙眼肉五钱,两剂 
见轻,又服十余剂全愈。 
奉天北镇县萧叟,年六十七岁,得膈证延医治不愈。迁延五六月,病浸加剧,饮水亦间有难下之时。来院 
求为延医。其脉弦长有力,右部尤甚。知其冲气上冲过甚,迫其胃气不下降也。询其大便,干燥不易下,多日 
不行,又须以药通之。投以参赭培气汤,赭石改用一两。数剂后,饮食见顺,脉亦稍和,觉胃口仍有痰涎杜塞。 
为加清半夏三钱,连服十剂,饮食大顺,脉亦复常,大便亦较易。遂减赭石之半,又服数剂,大便一日两次。 
遂去赭石、柿霜饼、当归、知母,加于术三钱,数剂后自言,觉胃中消化力稍弱,此时痰涎已清,又觉胃口似 
有疙瘩,稍碍饮食之路。遂将于术改用六钱,又加生鸡内金(捣细)二钱,佐于术以健运脾胃,即借以消胃口 
之障碍,连服十余剂全愈。 
友人吴××治姜姓叟,年六十余,得膈食证。屡次延医调治,服药半载,病转增进。吴××投以参赭培气汤, 
为其脉甚弦硬,知其冲气上冲,又兼血液枯少也,遂加生芡实以收敛冲气,龙眼肉以滋润血液,一剂能进饮食, 
又连服七八剂,饮食遂能如常。 
附录∶ 
奉天义县马××来函∶ 
去岁(乙丑)舍侄××患膈食,延医延医,年余无效。及病至垂危,诸医束手无策,有旧戚赠一良方,言 
系《衷中参西录》所载之方,名参赭培气汤,服之立见功效。连服十剂,其病全愈。 
奉天法库县万××来函∶ 
邱××之女,年十五,天癸已至,因受惊而经闭。两阅月,发现心热、心跳、膨胀等证,经医治疗未效, 
更添翻胃吐食、便燥、自汗等证。又经两月,更医十数,病益剧。适友人介绍为之诊视,脉浮数而濡,尺弱于 
寸,面色枯槁,肢体消瘦,不能起床;盖两月间食入即吐,或俟半日许亦必吐出,不受水谷之养,并灼热耗阴, 
无怪其支离若是也。思之再四,此必因受惊气乱而血亦乱,遂至遏其生机,且又在童年,血分未充,即不能应 
月而潮,久之不下行,必上逆,气机亦即上逆,况冲为血海,隶属阳明,阳明有升无降,冲血即随之上逆,瘀 
而不行,以至作灼作胀。其心跳者,为上冲之气血所扰也。其出汗吐食者,为上冲之气血所迫也。其津液因汗 
吐过多而消耗,所以大便干燥也。势非降逆、滋阴,镇心、解瘀之药并用不可。查参赭镇气汤及参赭培气汤二 
方,实为治斯证之津梁,爰即二方加减,赭石两半,当归、净萸肉、龙骨、牡蛎各五钱,白芍、肉苁蓉、党参、 
天冬、生鸡内金各三钱,磨取铁锈之水煎服。一剂病似觉甚,病家哗然,以为药不对证,欲另延医。惟介绍人 
主持甚力,勉又邀生再诊,此中喧变生固未之知也。既诊脉如故,决无病进之象。后闻有如此情形,生亦莫解。 
因反复思之,恍悟∶此必胃虚已极,兼胃气上逆过甚,遽投以如此重剂,其胃虚不能运化,气逆更多冲激,想 
有一番瞑眩,故病似加重也。于斯将原方减半,煎汤一盅,又分 
两次温服下,并送服柿霜三钱。其第一次服,仍吐药一半,二次即不吐,服完此剂后,略进薄粥,亦未吐,病 
家始欢然相信。又连服三剂,汗与吐均止,心跳膨胀亦大见轻。惟灼热犹不甚减,遂去净萸肉、龙骨、牡蛎, 
加生地、玄参各四钱,服五剂后,灼热亦愈强半。如此加减服之,一月后遂能起床矣。嘱其仍守服原方,至诸 
病全愈后可停药勿服,月事至期亦当自至也。 

\chapter{治呕吐方}
<篇名>1.镇逆汤
属性:治呕吐,因胃气上逆,胆火上冲者。 
生赭石(六钱,细轧) 青黛(二钱) 清半夏(三钱) 生杭芍(四钱) 龙胆草(三钱) 吴茱萸(一钱) 
生姜(二钱) 野台参(二钱) 


<篇名>2.薯蓣半夏粥
属性:治胃气上逆,冲气上冲,以致呕吐不止,闻药气则呕吐益甚,诸药皆不能下咽者。 
生山药(一两,轧细) 清半夏(一两) 
上二味,先将半夏用微温之水淘洗数次,不使分毫有矾味。用煮菜小锅(勿用药瓶)煎取清汤约两杯半, 
去渣调入山药细末,再煎两三沸,其粥即成,和白沙糖食之。若上焦有热者,以柿霜代沙糖,凉者用粥送服干 
姜细末半钱许。 
吐后口舌干燥,思饮水者热也。吐后口舌湿润,不思饮水者凉也。若呕吐既久,伤其津液,虽有凉者亦可 
作渴,又当细审其脉,滑疾为热,弦迟为凉。滑而无力,为上盛下虚,上则热而下或凉。弦而有力,为冲胃气 
逆,脉似热却非真热。又当问其所饮食 
者,消化与否,所呕吐者,改味与否,细心询问体验,自能辨其凉热虚实不误也。 
从来呕吐之证,多因胃气冲气,并而上逆。半夏为降胃安冲之主药,故《金匮》治呕吐,有大小半夏汤。 
特是呕者,最忌矾味,而今之坊间鬻者,虽清半夏亦有矾,故必将矾味洗净,而后以治呕吐,不至同于抱薪救 
火也。其多用至一两者,诚以半夏味本辛辣,因坊间治法太过,辣味全消,又经数次淘洗,其力愈减,必额外 
多用之,始能成降逆止呕之功也。而必与山药作粥者,凡呕吐之人,饮汤则易吐,食粥则借其稠粘留滞之力, 
可以略存胃腑,以待药力之施行。且山药,在上大能补肺生津,则多用半夏,不虑其燥,在下大能补肾敛冲, 
则冲气得养,自安其位。且与半夏皆无药味,故用于呕吐甚剧,不能服药者尤宜也。 
有因“胆倒”而呕吐不止者,《续名医类案》载∶许宣治一儿十岁,从戏台倒跌而下,呕吐苦水,绿如菜 
汁。许曰∶此“胆倒”也,胆汁倾尽则死矣。方用温胆汤,加枣仁、代赭石,正其胆腑。可名正胆汤,一服吐 
止。按∶此证甚奇异,附载于此,以备参考。 

\chapter{治霍乱方}
<篇名>1.急救回生丹
属性:治霍乱吐泻转筋,诸般痧证暴病,头目眩晕,咽喉肿疼,赤痢腹疼,急性淋证。 
朱砂(顶高者一钱五分) 冰片(三分) 薄荷冰(二分) 粉甘草(一钱细末) 
上药四味共研细,分作三次服,开水送下,约半点钟服一次。若吐剧者,宜于甫吐后急服之。若于将吐时 
服之,恐药未暇 
展布即吐出。服后温复得汗即愈。服一次即得汗者,后二次仍宜服之。若服完一剂未全愈者,可接续再服一剂。 
若其吐泻已久,气息奄奄有将脱之势,但服此药恐不能挽回,宜接服后急救回阳汤。 
朱砂能解心中窜入之毒,且又重坠,善止呕吐,俾服药后不致吐出。此方中冰片,宜用樟脑炼成者。因樟 
脑之性,原善振兴心脏,通活周身血脉,尤善消除毒菌。特其味稍劣,炼之为冰片,味较清馥,且经炼,而其 
力又易上升至脑,以清脑中之毒也。薄荷冰善解虎列拉之毒,西人屡发明之。且其味辛烈香窜,无窍不通,无 
微不至,周身之毒皆能扫除。矧与冰片,又同具发表之性,服之能作汗解,使内蕴之邪由汗透出。且与冰片皆 
性热用凉,无论症之因凉因热,投之咸宜也。粉甘草最善解毒,又能调和中宫,以止吐泻。且又能调和冰片、 
薄荷冰之气味,使人服之不致过于苛辣也。 
霍乱之证,西人所谓虎列拉也。因空气中有时含有此毒,而地面积秽之处,又酿有毒瓦斯与之混合(观此证 
起点多在大埠不洁之处可知),随呼吸之气入肺,由肺传心胞(即心肺相连之脂膜),由心胞传三焦(上焦心 
下膈膜,中焦包脾连胃脂膜,下焦络肠包肾脂膜),为手厥阴、少阳脏腑之相传。然其毒入三焦,其人中气 
充盛,无隙可乘,犹伏而不动。有时或饮食过量,或因寒凉伤其脾胃,将有吐泻之势,毒即乘虚内袭,盘据胃 
肠,上下不通,遂挥霍撩乱,而吐泻交作矣。吐泻不已,其毒可由肠胃而入心,更由心而上窜于脑,致脑髓神 
经与心俱病。左心房输血之力与右心房收血之力,为之顿减,是以周身血脉渐停,而通体皆凉也。其证多发于 
秋际者,因此毒瓦斯酿成多在夏令。人当暑热之时,周身时时有汗,此毒之伏于三焦者,犹得随汗些些外出。迨 
至秋凉汗闭,其毒不得外出,是以蓄极而动,乘脾胃之虚而内攻也。故治此症者,当以解毒之药为主,以助心 
活血之药为佐,以调阴阳奠中土之药为使。爰拟此方,名之曰急救回生丹。 
附录∶ 
直隶故城县袁××来函∶ 
前次寄来急救回生丹方,不知何以斟酌尽善。初故城闹疫,按方施药六十剂,皆随手辄效。后故城外镇郑 
家口闹疫,又施药二百剂,又莫不全活。 


<篇名>2.卫生防疫宝丹
属性:治霍乱吐泻转筋,下痢腹疼,及一切痧症。平素口含化服,能防一切疠疫传染。 
粉甘草(十两,细末) 细辛(两半,细末) 香白芷(一两,细末) 薄荷冰(四钱,细末) 冰片 
(二钱,细末) 朱砂(三两,细末) 
先将前五味和匀,用水为丸如桐子大,晾干(不宜日晒)。再用朱砂为衣,勿令余剩。装以布袋,杂以琉 
珠,来往撞荡,务令光滑坚实。如此日久,可不走气味。若治霍乱证,宜服八十丸,开水送服。余证宜服四五 
十丸。服后均宜温复取微汗。若平素含化以防疫疠,自一丸至四五丸皆可。此药又善治头疼、牙疼(含化), 
心下、胁下及周身关节经络作疼,气郁、痰郁、食郁、呃逆、呕哕。醒脑养神,在上能清,在下能温,种种利 
益,不能悉数。 
以上二方,后方较前方多温药两味。前方性微凉,后方则凉热平均矣。用者斟酌于病因,凉热之间,分途 
施治可也。后方若临证急用,不暇为丸,可制为散,每服一钱,效更速。 
附录∶ 
直隶盐山孙××来函∶ 
一九二四年六月,友人杜××之母得霍乱证,上吐下泻,转 
筋腹疼,六脉闭塞。生诊视后,为开卫生防疫宝丹方,共研作粉,每次服一钱。服第一次,吐泻稍止。服第二 
次,病即痊愈。 
斯年初冬,邓××之儿媳得霍乱证,时已夜半,请为诊视。吐泻转筋,六脉皆无,心中迷乱,时作谵语。治 
以卫生防疫宝丹,初服仍吐,服至二次,脉即徐出而愈。 


<篇名>3.急救回阳汤
属性:治霍乱吐泻已极,精神昏昏,气息奄奄,至危之候。 
潞党参(八钱) 生山药(一两) 生杭芍(五钱) 山萸肉(八钱,去净核) 炙甘草(三钱) 赭石(四钱, 
研细) 朱砂(五分,研细) 
先用童便半盅炖热,送下朱砂,继服汤药。 
以上二方,皆为治霍乱之要药矣。然彼以祛邪为主,此以扶正为主。诚以得此证者,往往因治不如法,致 
日夜吐泻不已,虚极将脱,危在目前。病势至此,其从前之因凉因热皆不暇深究,惟急宜重用人参以回阳,山 
药、芍药以滋阴,山萸肉以敛肝气之脱(此证吐泻之始肝木助邪侮土,吐泻之极而肝气转先脱),炙甘草以和 
中气之漓,此急救回阳汤所以必需也。用赭石者,不但取其能止呕吐,俾所服之药不致吐出,诚以吐泻已久, 
阴阳将离,赭石色赤入心,能协同人参,助心气下降。而方中山药,又能温固下焦,滋补真阴,协同人参以回 
肾气之下趋,使之上行也。用朱砂且又送以童便者,又以此时百脉闭塞,系心脏为毒瓦斯所伤,将熄其鼓动之机, 
故用朱砂直入心以解毒,又引以童便使毒瓦斯从尿道泻出,而童便之性又能启发肾中之阳上达,以应心脏也。是 
此汤为回阳之剂,实则交心肾和阴阳之剂也。服此汤后,若身温脉出,觉心中发热有烦躁之意者,宜急滋其阴分 
,若玄参、生芍药之类,加甘草以和之,煎一大剂,分数次温饮下。此《伤寒论》太阳篇,先用甘草干姜汤继用 
甘草芍药汤之法也。 
门人高××,曾治一少妇。吐泻一昼夜,甚是困惫,浓煎人参汤,送服益元散而愈。盖独参汤能回阳,益 
元散能滋阴,又能和中(滑石甘草能和中以止吐泻)解毒(甘草朱砂能解毒),且可引毒瓦斯自小便出,是以应 
手奏效。此亦拙拟急救回阳汤之意也。 
此证之转筋者,多因吐泻不已,肝木乘脾气之虚而侮土。故方书治转筋多用木瓜,以其酸能敛肝,即所 
以平肝也。然平肝之药,不必定用木瓜。壬寅秋际,霍乱流行,曾单用羚羊角三钱。治愈数人。因羚羊角善解 
热毒,又为平肝之妙药也。又曾有一人,向愚询治泄泻之方。告以酸石榴连皮捣烂,煎汤服之。后值霍乱发生, 
其人用其方治霍乱初起之泄泻者,服之泻愈,而霍乱亦愈。由是观之,石榴亦为敛肝之要药,而敛肝之法,又 
实为治霍乱之要着也。 
霍乱之证,有实热者居多,其真寒凉者,不过百中之一二。即百脉闭塞,周身冰冷,但其不欲复被,思 
饮凉水,即不可以凉断,当先少少与以凉水,若饮后病增重者,其人虽欲复饮,而不至急索者,凉水可勿与也。 
若饮后病不增重,须臾不与,有不能忍受之状,可尽量与之,任其随饮随吐,借凉水将内毒换出,亦佳方也。 
曾遇有恣饮凉水而愈者,问之,言当病重之时,若一时不饮凉水,即觉不能复活,则凉水之功用可知矣。然凉 
水须用新汲井泉水方效,无井泉水处,可以冰水代之,或吞服小冰块亦佳。 
王孟英曰∶鸡矢白散,为《金匮》治霍乱转筋入腹之方。愚仿其意,拟得蚕矢汤,治霍乱转筋、腹疼、 
口渴、烦躁,危急之证甚效。方用晚蚕砂、木瓜各三钱,生薏仁、大豆芽(如无可代以生麦芽)各四钱,川黄 
连、炒山栀各二钱,醋炒半夏、酒炒黄芩、吴茱萸各一钱,以阴阳水煎,稍凉,徐徐服之。丁酉八九月间,吾 
杭盛行霍 
乱转筋之证。有沈氏妇者,夜深患此,继即音哑肢寒。比晓,其夫皇皇求为救治。诊其脉弦细以涩,两尺如无, 
口极渴而沾饮即吐不已,腓坚硬如石,其时疼楚异常。因拟此方治之,徐徐凉饮,药入口竟得不吐。外以好烧 
酒令人用力摩擦转筋坚硬之处,擦将一时许,其硬块始渐软散,而筋不转吐泻亦减。甫时复与前药半剂,夜间 
居然安寐矣。后治相类者多人,悉以是法获效。 
陆九芝曰∶霍乱一证,有寒有热,热者居其九,寒者居其一。凡由高楼大厦,乘凉饮冷而得之者,仲景则 
有理中、四逆诸方,后世亦有浆水、大顺、复元、冷香饮子诸方,病多属寒,药则皆宜热。若夫春分以后,秋 
分以前,少阳相火,少阴君火,太阴湿土,三气合行其令。天之热气则下降,地之湿气则上腾,人在气交之中, 
清气在阴,浊气在阳,阴阳反戾,清浊相干,气乱于中,而上吐下泻。治此者,宜和阴阳,厘清浊,以定其乱, 
乱定即无不愈。此则病非寒也,而亦非尽用寒药也。即如薷藿、平陈、胃苓等汤习用之剂,亦皆温通,特不用 
姜附丁萸之大辛大热者耳。又有不吐不泻而挥霍撩乱者,则多得之饱食之后。凡夏月猝然冒暑,惟食填太阴, 
亦曰饱食填息。此证为病最速,为祸最酷,而人多忽之。即有知者,亦仅以停食为言,绝不信其为闭证之急者。 
闭则手足肢冷,六脉俱伏,甚则喜近烈日。此乃邪闭而气道不宣,其畏寒也正其热之甚也。此等证,只欠一吐 
法耳。自吐法之不讲,本属一吐即愈之病,而竟不知用也。此外更有四肢厥逆,甚至周身如冰,而竟不恶寒, 
反有恶热者,此更是内真热,外假寒,即厥阴中热深厥深之象。岂独不可用四逆、理中,即姜汤、米饮及五苓 
散中之桂枝,亦不可用。而且宜苦寒之剂,佐以挑痧、刮痧等法,刺出其恶血以泄热毒者。同治壬戌,江苏沪 
渎,时疫盛行,绵延而至癸亥。余尝以石膏、芩、连,清而愈之者,则暑湿热之霍乱也。以凉水调胆矾吐而愈 
之者,则饱食填息之霍 
乱也。其肢皆冷,而其脉皆伏,维时大医,竞用丁萸桂附,日误数人,而竟不知改图,岂不深可惜哉。 
上所录二则,皆于霍乱之证,有所发明,故详志之,以备采择。 
霍乱之证,宜兼用外治之法,以辅药饵所不逮。而外治之法,当以针灸为最要。至应针之处,若十宣、 
中脘、尺泽、足三里、阴陵、承山、太溪、太仓、太冲、公孙等穴(约略举之,未能悉数),习针灸者大抵皆 
知。惟督脉部分,有素 穴,刺同身寸之三分出血,最为治霍乱之要着。凡吐泻交作,心中撩乱者,刺之皆效。 
诸针灸之书,皆未言其能治霍乱。世之能针灸者,间有知刺其处者,而或刺鼻准之尖,或刺鼻柱中间,又多不 
能刺其正穴。两鼻孔中间为鼻柱,《内经》王注,谓此穴在鼻柱之上端,则非鼻准之尖,及鼻柱中间可知。然 
刺未中其正穴者,犹恒有效验,况刺中其正穴乎。盖此穴通督脉,而鼻通任脉,刺此一处,则督、任二脉,可 
互相贯通,而周身之血脉,亦因之可贯通矣。 
又宜佐以刮痧之法。盖此证病剧之时,周身冰冷,回血管之血液凝滞不行。当用细口茶碗,将碗边一处少 
涂香油,两手执定其无油之处,先刮其贴脊两旁,脊椎上亦可轻刮,以刮处尽红为度。盖以脏腑之系皆连于脊, 
而诸脏腑 穴,亦贴脊两旁,故以刮此处为最要。而刮时,又宜自上而下挨次刮之,可使毒瓦斯下行。次刮其胸 
与胁,次刮其四肢曲处(尺泽委中)及腿内外 ,至头额项肩,亦可用钱刮之。又当兼用放痧之法∶将四肢回 
血管之血,用手赶至腿臂曲处,用带上下扎紧,于尺泽、委中两旁回血管,用扁针刺出其血,以助其血脉之流 
通,且又放出炭气,俾霍乱之毒菌,从此轻减也。 
又宜佐以温体之法。用滚水煮新砖八个,以熨腋下及四肢曲处,及两脚涌泉穴。或水煮粗浓之布,乘热迭 
数层,复于转筋之 
处。即不转筋者,亦可复于少腹及腿肚之上,凉则易之。或以茶壶及水笼袋,满贮热水,以熨各处。或醋炒葱 
白(切丝)、或醋炒干艾叶(揉碎)熨之。或用手醮火酒或烧酒,急速擦摩其周身及腿肚发硬之处。种种助暖 
之法不一,临证者随事制宜可也。 

\chapter{治泄泻方}
<篇名>1.益脾饼
属性:治脾胃湿寒,饮食减少,长作泄泻,完谷不化。 
白术(四两) 干姜(二两) 鸡内金(二两) 熟枣肉(半斤) 
上药四味,白术、鸡内金皆用生者,每味各自轧细焙熟(先轧细而后焙者,为其焙之易匀也)。再将干姜 
轧细,共和枣肉,同捣如泥,作小饼。木炭火上炙干,空心时,当点心,细嚼咽之。曾为友人制此方,和药一 
料,服之而愈者数人。后屡试此方,无不效验。 
一妇人,年三十许,泄泻数月。用一切治泻诸药皆不效。其脉不凉,亦非完谷不化。遂单用白术、枣肉, 
如法为饼,服之而愈。此证并不用鸡内金者,因鸡内金虽有助脾胃消食之力,而究与泻者不宜也。 
附录∶ 
直隶青县张××来函∶ 
胞妹××,年四十余岁,体素瘦弱,久患脾胃湿寒,胃脘时觉疼痛,饮食减少,常作泄泻,完谷不化。因照 
益脾饼原方,为制一料,服之即愈。为善后计,又服一料,永久祓除病根。 


<篇名>2.扶中汤
属性:治泄泻久不止,气血俱虚,身体羸弱,将成劳瘵之候。 
于术(一两,炒) 生山药(一两) 龙眼肉(一两) 
小便不利者加椒目(三钱,炒捣) 
龙眼肉,味甘能补脾,气香能醒脾,诚为脾家要药。且心为脾母,龙眼肉色赤入心,又能补益心脏,俾母 
旺自能荫子也。愚治心虚怔忡,恒俾单购龙眼肉斤许,饭甑蒸熟,徐徐服之,皆大有功效,是能补心之明征。 
又大便下血者,多因脾虚不能统血,亦可单服龙眼肉而愈,是又补脾之明征也。 
一妇人,年四十许。初因心中发热,气分不舒,医者投以清火理气之剂,遂泄泻不止。更延他医,投以温 
补之剂,初服稍轻,久服,则泻仍不止。一日夜四五次,迁延半载,以为无药可治。后愚为诊视,脉虽濡弱, 
而无弦数之象,知犹可治。但泻久身弱,虚汗淋漓,心中怔忡,饮食减少,踌躇久之,为拟此方,补脾兼补心 
肾。数剂泻止,而汗则加多。遂于方中加龙骨、牡蛎(皆不用 )各六钱,两剂汗止,又变为漫肿。盖从前泻 
时,小便短少,泻止后,小便仍少,水气下无出路,故蒸为汗,汗止又为漫肿也。斯非分利小便,使水下有出 
路不可。特其平素常觉腰际凉甚,利小便之药,凉者断不可用。遂用此方,加椒目三钱,连服十剂全愈。 


<篇名>3.薯蓣粥
属性:治阴虚劳热,或喘,或嗽,或大便滑泻,小便不利,一切羸弱虚损之证。 
生怀山药(一斤,轧细过罗) 
上药一味,每服用药七八钱,或至一两。和凉水调入锅内, 
置炉上,不住以箸搅之,两三沸即成粥服之。若小儿服,或少调以白糖亦可。 
此粥多服久服间有发闷者,掺以西药百布圣一瓦同服,则无此弊,且更多进饮食。 
门生吴××,年二十一。羸弱发热,脉象虚数,不能饮食。俾早晚服山药粥,加百布圣,晌午单服玄参三钱, 
煎汤服。如此数日,食量增加,发热亦愈,自此健壮。 
一妇人,年三十余。泄泻数月不止,病势垂危。倩人送信于其父母,其父将往瞻视,询方于愚。言从前屡 
次延医治疗,百药不效。因授以山药煮粥方,日服三次,两日全愈。又服数日,身亦康健。 
一娠妇,日发痫风。其脉无受娠滑象,微似弦而兼数。知阴分亏损,血液短少也。亦俾煮山药粥服之即愈。 
又服数次,永不再发。 
奉天关氏少妇,素有劳疾,因产后暴虚,喘嗽大作。治以此粥,日服两次,服至四五日,喘嗽皆愈。又服 
数日,其劳疾自此除根。 
奉天郑××之女,年五岁。秋日为风寒所束,心中发热。医者不知用辛凉表散,而纯投以苦寒之药,连服 
十余剂,致脾胃受伤,大便滑泻,月余不止,而上焦之热益炽。医者皆辞不治,始求愚为诊视,其形状羸弱已 
甚,脉象细微浮数,表里俱热,时时恶心,不能饮食,昼夜犹泻十余次。治以此粥,俾随便饮之,日四五次, 
一次不过数羹匙,旬日全愈。 
农村小儿,于秋夏之交,多得滑泻证。盖农家此时多饮凉水,而小儿尤喜饮之。农家此时多食瓜果,而小 
儿尤喜食之。生冷之物,皆伤脾胃,脾胃伤,则滑泻随之,此自然之理也。而滑泻之证,在小儿为最难治。盖 
小儿少阳之体,阴分未足,滑泻不 
止,尤易伤阴分。往往患此证者,数日即浑身发热,津短燥渴,小便不利,干呕懒食,唯嗜凉物。当此之际, 
欲滋其阴,而脾胃愈泥,欲健其脾,而真阴愈耗,凉润温补,皆不对证。而小儿又多苦服药,病家又多姑息, 
以婉随小儿之意,以致迁延岁月,竟成不治者多矣。惟山药脾肾双补,在上能清,在下能固,利小便而止大便, 
真良药也。且又为寻常服食之物,以之作粥,少加沙糖调和,小儿必喜食之。一日两次煮服,数日必愈。若系 
哺乳稚子,不能食粥,即食之亦不能多者,但浓煮生山药汁,饮之亦可。愚以此方治小儿多矣。志在救人者, 
甚勿以为寻常服食之物,而忽之也。 


<篇名>4.薯蓣鸡子黄粥
属性:治泄泻久,而肠滑不固者。 
即前薯蓣粥,加熟鸡子黄三枚。 
一人,年近五旬。泄泻半载不愈,羸弱已甚。遣人来询方,言屡次延医服药,皆分毫无效。授以薯蓣粥方, 
数日又来言,服之虽有效验,泻仍不止。遂俾用鸡子数枚煮熟,取其黄捏碎,调粥中服之,两次而愈。盖鸡子 
黄,有固涩大肠之功,且较鸡子白,易消化也。以后此方用过数次,皆随手奏效。 
附录∶ 
直隶青县张××来函∶ 
河间刘××,年五十余岁。漏疮甚剧,屡治不痊,后兼泄泻不止,盖肠滑不固,故医药无灵。诊其脉甚小弱, 
渐已成痨。嘱其用薯蓣鸡子黄粥。一剂泻止。三服,精神焕发。十数日后,身体撤消。此后凡遇虚泻久不愈者, 
用之屡收特效。 


<篇名>5.薯蓣苜汤
属性:治阴虚肾燥,小便不利,大便滑泻,兼治虚劳有痰作嗽。 
生山药(一两,轧细) 生车前子(四钱) 
上二味,同煮作稠粥服之,一日连服三次,小便自利,大便自固。盖山药能固大便,而阴虚小便不利者服 
之,又能利小便。车前子能利小便,而性兼滋阴,可为补肾药之佐使(五子衍宗丸中用之),又能助山药以止 
大便。况二药皆汁浆稠粘,同作粥服之,大能留恋肠胃,是以效也。治虚劳痰嗽者,车前宜减半。盖用车前者, 
以其能利水,即能利痰,且性兼滋阴,于阴虚有痰者尤宜。而仍不敢多用者,恐水道过利,亦能伤阴分也。 
按∶车前子能利小便,而骤用之亦无显然功效。惟将车前子炒熟(此药须买生者自家经手炒,以微熟为度, 
过熟则无力),嚼服少许,须臾又服,约六点钟服尽一两,小便必陡然利下,连连不止。此愚实验而得之方也。 
又∶单用车前子两半,煮稠粥,顿服之,治大便滑泻亦甚效验。黄姓媪,大便滑泻,百药不效。或语以此 
方,一服即愈。然必用生者煮之,始能成粥,若炒熟者,则不能成粥矣。 


<篇名>6.加味天水散
属性:治暑日泄泻不止,肌肤烧热,心中燥渴,小便不利,或兼喘促。小儿尤多此证,用此方更佳。 
生山药(一两) 滑石(六钱) 粉甘草(三钱) 作汤服 
此久下亡阴,又兼暑热之证也。故方中用天水散以清溽暑之热。而甘草分量,三倍原方(原方滑石六、甘草 
一,故亦名六一散),其至浓之味,与滑石之至淡者相济,又能清阴虚之热。又重用山药之大滋真阴,大固元 
气者,以参赞之。真阴足,则小便自利,元气固,则泄泻自止。且其汁浆稠粘,与甘草之甘缓者同用,又能逗 
留滑石,不至 
速于淡渗。俾其清凉之性由胃输脾,由脾达肺,水精四布,下通膀胱,则周身之热与上焦之燥渴喘促,有不倏 
然顿除者乎? 
小儿少阳之体,最不耐热,故易伤暑。而饮食起居,喜贪寒凉,故又易泄泻。泻久则亡阴作热,必愈畏暑 
气之热,病热循环相因,所以治之甚难也。此方药止三味,而用意周匝,内伤外感,兼治无遗。一两剂后,暑热 
渐退,即滑石可以渐减,随时斟酌用之,未有不应手奏效者。小儿暑月泻久,虚热上逆,与暑热之气相并, 
填塞胃口,恒至恶心呕吐,不受饮食。此方不但清暑滋阴,和中止泻,其重坠之性,又能镇胃安冲,使上逆之 
热与暑气之热,徐徐下行,自小便出,而其恶心呕吐自止。 
初定此方时,授门人高××录之。翌日,高××还里,遇一孺子,泄泻月余,身热燥渴,嗜饮凉水,强与饮 
食,即恶心呕吐,多方调治不愈。高××投以此汤,一剂,燥渴与泄泻即愈其半。又服一剂,能进饮食,诸病皆愈。 


<篇名>7.加味四神丸
属性:治黎明腹疼泄泻。 
补骨脂(六两酒炒) 吴茱萸(三两,盐炒) 五味子(四两,炒) 肉豆蔻(四两,面裹煨) 
花椒(一两微焙) 生硫黄(六钱) 大枣(八十一枚) 生姜(六两,切片) 
先煮姜十余沸,入枣同煮,至烂熟去姜,余药为细末,枣肉为丸,桐子大。 
人禀天地之气而生,人身一小天地也。天地之一阳生于子,故人至夜半之时,肾系命门之处,有气息息萌 
动,即人身之阳气也。至黎明寅时,为三阳之候,人身之阳气,亦应候上升,自下焦而将达中焦。其人或元阳 
之根柢素虚,当脐之处,或兼有凝寒遮蔽,即互相薄激,致少腹作疼。久之阳气不胜凝寒,上升之机转为下 
降,大便亦即溏下,此黎明作泻之所由来也。夫下焦之阳气少火也,即相火也,其火生于命门,而寄于肝胆。 
故四神方中,用补骨脂以补命门,吴茱萸以补肝胆,此培火之基也。然泻者关乎下焦,实又关乎中焦,故又用 
肉豆蔻之辛温者,以暖补脾胃。且其味辛而涩,协同五味之酸收者,又能固涩大肠,摄下焦气化。且姜枣同煎, 
而丸以枣肉,使辛甘化合,自能引下焦之阳,以达于中焦也。然此药病轻者可愈,病重者服之,间或不愈,以 
其补火之力犹微也。故又加花椒、硫黄之大补元阳者以助之,而后药力始能胜病也。 

\chapter{治痢方}
<篇名>1.化滞汤
属性:治下痢赤白,腹疼,里急后重初起者。若服药后病未全愈,继服后方。 
生杭芍(一两) 当归(五钱) 山楂(六钱) 莱菔子(五钱,炒捣) 甘草(二钱) 生姜(二钱) 
若身形壮实者,可加大黄、朴硝各三钱下之。 
方中之意∶用芍药以泄肝之热,甘草以缓肝之急,莱菔子以开气分之滞,当归、山楂以化血分之滞,生姜 
与芍药并用又善调寒热之互相凝滞,且当归之汁液最滑,痢患滞下而以当归滑之,其滞下愈而痢自愈也。 


<篇名>2.燮理汤
属性:治下痢服前药未全愈者。若下痢已数日,亦可迳服此汤。又治噤口痢。 
生山药(八钱) 金银花(五钱) 生杭芍(六钱) 牛蒡子(二钱,炒捣) 甘草(二钱) 黄连(钱半) 肉 
桂 
(一钱半,去粗皮将药煎至数十沸再入) 
单赤痢加生地榆二钱,单白痢加生姜二钱,血痢加鸭蛋子二十粒(去皮),药汁送服。 
痢证古称滞下。所谓滞下者,诚以寒火凝结下焦,瘀为脓血,留滞不下,而寒火交战之力又逼迫之,以使 
之下也。故方中黄连以治其火,肉桂以治其寒,二药等分并用,阴阳燮理于顷刻矣。用白芍者,《伤寒论》诸 
方,腹疼必加芍药协同甘草,亦燮理阴阳之妙品。且痢证之噤口不食者,必是胆火逆冲胃口,后重里急者,必 
是肝火下迫大肠,白芍能泻肝胆之火,故能治之。矧肝主藏血,肝胆火戢,则脓血自敛也。用山药者,滞下久 
则阴分必亏,山药之多液,可滋脏腑之真阴。且滞下久,则气化不固,山药之收涩,更能固下焦之气化也。又 
白芍善利小便,自小便以泻寒火之凝结。牛蒡能通大便,自大便以泻寒火之凝结。金银花与甘草同用,善解热 
毒,可预防肠中之溃烂。单白痢则病在气分,故加生姜以行气。单赤痢则病在血分,故加生地榆以凉血。至痢 
中多带鲜血,其血分为尤热矣,故加鸭蛋子,以大清血分之热。拙拟此方以来,岁遇患痢者不知凡几,投以此 
汤,即至剧者,连服数剂亦必见效。 
痢证,多因先有积热,后又感凉而得。或饮食贪凉,或寝处贪凉,热为凉迫,热转不散。迨历日既多,又 
浸至有热无凉,犹伤于寒者之转病热也。所以此方虽黄连、肉桂等分并用,而肉桂之热,究不敌黄连之寒。况 
重用白芍,以为黄连之佐使,是此汤为燮理阴阳之剂,而实则清火之剂也。 
或问∶以此汤治痢,虽在数日之后,或服化滞汤之后,而此时痢邪犹盛,遽重用山药补之,独无留邪之患 
乎?答曰∶山药虽 
饶有补力,而性略迟钝,与参、 之迅速者不同。在此方中,虽与诸药同服,约必俟诸药之凉者、热者、通者、 
利者,将痢邪消融殆尽,而后大发其补性,以从容培养于诸药之后,俾邪去而正已复,此乃完全之策,又何至 
留邪乎?且山药与芍药并用,大能泻上焦之虚热,与痢之噤口者尤宜。是以愚用此汤,遇痢之挟虚与年迈者, 
山药恒用至一两,或至一两强也。 
或问∶地榆方书多炒炭用之,取其黑能胜红,以制血之妄行。此方治单赤痢加地榆,何以独生用乎?答曰∶ 
地榆之性,凉而且涩,能凉血兼能止血,若炒之则无斯效矣,此方治赤痢所以必加生地榆也。且赤痢之证,其 
剧者,或因肠中溃烂。林屋山人治汤火伤,皮肤溃烂,用生地榆末和香油敷之甚效。夫外敷能治皮肤因热溃烂, 
而内服亦当有此效可知也。鸭蛋子一名鸦胆子,苦参所结之子也。不但善治血痢,凡诸痢证皆可用之。即纯白 
之痢,用之亦有效验,而以治噤口痢、烟后痢、尤多奇效,并治大小便因热下血。其方单用鸭蛋子(去皮), 
择成实者五六十粒,白沙糖化水送服,日两次,大有奇效。若下血因凉者,亦可与温补之药同用。其善清血热, 
而性非寒凉,善化瘀滞,而力非开破,有祛邪之能,兼有补正之功,诚良药也。坊间将鸭蛋子去皮,用益元散 
为衣,治二便下血如神,名曰菩提丹。 
一人,年五十余,素吸鸦片。当霍乱盛行之时,忽然心中觉疼、恶心呕吐、下痢脓血参半。病家惧甚,以 
为必是霍乱暴证。诊其脉毫无闭塞之象,惟弦数无力,左关稍实。愚曰∶此非霍乱,乃下焦寒火交战,故腹中 
作疼,下痢脓血。上焦虚热壅迫,故恶心呕吐,实系痢证之剧者。遂投以白芍六钱,竹茹、清半夏各三钱,甘 
草、生姜各二钱,一剂呕吐即愈,腹疼亦轻,而痢独不愈,不思饮食。俾单用鸭蛋子五十粒,一日连服两次, 
病若失。审斯,鸭蛋子不但善理下焦,即上焦虚热,用之亦妙,此所以治噤口痢而有捷效也。 
一人,年四十八,资禀素弱,亦吸鸦片。于季秋溏泻不止。一日夜八九次,且带红色,心中怔忡,不能饮 
食。日服温补之药,分毫无效。延愚延医,其脉左右皆微弱,而尺脉尤甚,知系下焦虚寒。为其便带红色,且 
从前服温补之药无效,俾先服鸭蛋子四十粒,泻愈其半,红色亦略减,思饮食。继用温补下焦之药煎汤,送服 
鸭蛋子三十粒,后渐减至十粒,十剂全愈。盖此证虽下焦虚寒,而便带红色,实兼有痢证也。故单服鸭蛋子, 
而溏泻已减半。然亦足征鸭蛋子虽善清热化瘀,而实无寒凉开破之弊,洵良药也。 
沧州友人滕××,壬寅之岁,于中秋下赤痢,且多鲜血。医治两旬不愈。适愚他出新归。过访之,求为诊 
治。其脉象洪实,知其纯系热痢。遂谓之曰∶此易治。买苦参子百余粒,去皮,分两次服下即愈矣。翌日愚复 
他出,二十余日始归。又访之,言曾遍问近处药坊,皆无苦参子。后病益剧,遣人至敝州取来,如法服之,两 
次果愈。后滕××旋里,其族人有适自奉天病重归来者,大便下血年余,一身悉肿,百药不效。滕××授以此方, 
如法服之,三次全愈。 
附录∶ 
直隶青县张××来函∶ 
芦台李××,年四十二岁,壬戌五月间,因劳碌暑热,大便下血,且腹疼。医者多用西洋参、野于术、地榆 
炭、柏叶炭温涩之品投之,愈服愈危。小站王××,余友也,代寄函询方,并将病源暨前方开示。余阅毕,遂为 
邮去菩提丹四服。每服六十粒,日服一次。未几,接复函,谓服毕血止,腹疼亦愈,极赞药之神妙。近年用此 
丹治赤痢及二便下血,愈者甚多。 
山东德州卢××来函∶ 
××族嫂,因逃荒惊恐,又兼受暑,致患痢两月余,服药无 
效,益加沉重。侄为开乌梅六个,山楂两半,煎汤送服益元散四钱,去皮鸭蛋子四十粒。煎药渣时,亦如此送 
服。一剂病若失。 


<篇名>3.解毒生化丹
属性:治痢久郁热生毒,肠中腐烂,时时切疼,后重,所下多似烂炙,且有腐败之臭。 
金银花(一两) 生杭芍(六钱) 粉甘草(三钱) 三七(二钱捣细) 鸭蛋子(六十粒,去皮拣成实者) 
上药五味,先将三七、鸭蛋子,用白沙糖化水送服。次将余药煎汤服。病重者,一日须服两剂始能见效。 
此证,乃痢之最重者。若初起之时,气血未亏,用拙拟化滞汤,或加大黄、朴硝下之即愈。若未全愈,继 
服燮理汤数剂,亦可全愈。若失治迁延日久,气血两亏,浸至肠中腐烂,生机日减,致所下之物,色臭皆腐败, 
非前二方所能愈矣。此方则重在化腐生肌,以救肠中之腐烂,故服之能建奇效也。 
一人,年五十二,因大怒之后,中有郁热,又寝于冷屋之中,内热为外寒所束,愈郁而不散,大便下血。 
延医调治,医者因其得于寒凉屋中,谓系脾寒下陷,投以参、 温补之药,又加升麻提之。服药两剂,病益增 
重,腹中切疼,常常后重,所便之物,多如烂炙。更延他医,又以为下元虚寒,而投以八味地黄丸,作汤服之, 
病益加重。后愚诊视,其脉数而有力,两尺愈甚。确知其毒热郁于肠中,以致肠中腐烂也。为拟此方,两剂而愈。 
一妇人,年五十许,素吸鸦片。又当恼怒之余,初患赤痢,滞下无度。因治疗失宜,渐至血液腐败,间如 
烂炙,恶心懒食,少腹切疼。其脉洪数,纯是热象。亦治以此汤,加知母、白头翁各四钱,煎汤服。又另取鸭 
蛋子六十粒、三七二钱,送服。每 
日如此服药两次,三日全愈。 


<篇名>4.天水涤肠汤
属性:治久痢不愈,肠中浸至腐烂,时时切疼,身体因病久羸弱者。 
生山药(一两) 滑石(一两) 生杭芍(六钱) 潞党参(三钱) 白头翁(三钱) 粉甘草(二钱) 
一媪,年六十一岁,于中秋痢下赤白,服药旋愈,旋又反复。如此数次,迁延两月。因少腹切疼,自疑寒 
凉,烧砖熨之。初熨时稍觉轻,以为对证。遂日日熨之,而腹中之疼益甚。昼夜呻吟,噤口不食。所下者痢与 
血水相杂,且系腐败之色。其脉至数略数,虽非洪实有力,实无寒凉之象。舌上生苔,黄而且浓。病患自谓下 
焦凉甚,若用热药温之疼当愈。愚曰∶前此少腹切疼者,肠中欲腐烂也,今为热砖所熨而腹疼益甚,败血淋漓, 
则肠中真腐烂矣。再投以热药,危可翘足而待。病患亦似会悟,为制此方。因河间天水散(即六一散),原为 
治热痢之妙药,此方中重用滑石、甘草,故名之天水涤肠汤。连服四剂,疼止,痢亦见愈。减去滑石四钱,加 
赤石脂四钱,再服数剂,病愈十之八九。因上焦气微不顺,俾用鲜藕四两,切细丝煎汤,频频饮之,数日而愈。 
此证亦痢中至险之证。而方中用党参者,因痢久体虚,所下者又多腐败,故于滋阴清火解毒药中,特加党 
参以助其生机。而其产于潞者,性平不热,于痢证尤宜也。 
此证若服此汤不效,则前方之三七、鸭蛋子、金银花亦可酌加,或加生地榆亦可。试观生地榆为末、香油 
调,涂汤火伤神效,其能治肠中因热腐烂可知也。 


<篇名>5.通变白头翁汤
属性:治热痢下重腹疼,及患痢之人,从前曾有鸦片之嗜好者。 
生山药(一两) 白头翁(四钱) 秦皮(三钱) 生地榆(三钱) 生杭芍(四钱) 甘草(二钱) 旱三七(三 
钱, 
轧细) 鸭蛋子(六十粒,去皮拣成实者) 
上药共八味,先将三七、鸭蛋子,用白蔗糖水送服一半,再将余煎汤服。其相去之时间,宜至点半钟。所 
余一半,至煎汤药渣时,仍如此服法。 
《伤寒论》治厥阴热痢下重者,有白头翁汤。其方,以白头翁为主,而以秦皮、黄连、黄柏佐之。 
愚用此方,而又为之通变者,因其方中尽却病之药,而无扶正之药,于证之兼虚者不宜。且连、柏并用, 
恐其苦寒之性妨碍脾胃,过侵下焦也。矧《伤寒论》白头翁汤,原治时气中初得之痢,如此通变之,至痢久而 
肠中腐烂者,服之亦可旋愈也。 
奉天王××,年四十许。己未孟秋,自郑州病归,先泻后痢,腹疼重坠,赤白稠粘,一日夜十余次。先入奉 
天东人所设医院中,东人甚畏此证,处以隔离所,医治旬日无效。遂出院归寓,求为延医。其脉弦而有力,知 
其下久阴虚,肝胆又蕴有实热也。投以此汤,一剂痢愈。仍变为泻,日四五次,自言腹中凉甚。愚因其疾原先 
泻,此时痢愈又泻,且恒以温水袋自熨其腹,疑其下焦或有伏寒,遂少投以温补之药。才服一剂,又变为痢, 
下坠腹疼如故,惟次数少减。知其病原无寒,不受温补。仍改用通变白头翁汤。一剂痢又愈,一日犹泻数次。 
继用生山药一两,龙眼、莲子各六钱,生杭芍三钱,甘草、茯苓各二钱,又少加酒曲、麦芽、白蔻消食之品, 
调补旬日全愈。 
奉天李××,年近四旬。因有事,连夜废寝。陡然腹疼,继而泄泻,兼下痢。其痢赤多于白,上焦有热,不 
能饮食。其脉弦而浮,按之不实。先投以三宝粥方,腹疼与泻痢皆见轻,仍不能饮食。继用通变白头翁汤方, 
连服两剂,痢愈可进饮食,腹疼泄泻犹未全愈。后仍用三宝粥方,去鸭蛋子,日服两次,数日病全愈。 


<篇名>6.三宝粥
属性:治痢久,脓血腥臭,肠中欲腐,兼下焦虚惫,气虚滑脱者。 
生山药(一两,轧细) 三七(二钱,轧细) 鸭蛋子(五十粒,去皮) 
上药三味,先用水四盅,调和山药末煮作粥。煮时,不住以箸搅之,一两沸即熟,约得粥一大碗。即用其 
粥送服三七末、鸭蛋子。 
己巳之岁,愚客居德州,有卢姓妇,年五十六。于季夏下痢赤白,迁延至仲冬不愈。延医十余人,服药百 
剂,皆无效验,亦以为无药可医矣。后求愚延医,脉象微弱,至数略数,饮食减少,头目时或眩晕,心中微觉 
烦热,便时下坠作疼,然不甚剧。询其平素,下焦畏凉。是以从前服药,略加温补,上即烦热,略为清理,下 
又腹疼泄泻也。为拟此方,一日连服两次,其病遂愈。后旬余,因登楼受凉,旧证陡然反复,日下十余次,腹 
疼觉剧。其脉象微弱如前,至数不数。俾仍用山药粥,送服生硫黄末三分,亦一日服两次,病愈强半。翌日又 
服一次,心微觉热。继又改用前方,两剂全愈。 
戊午秋日,愚初至奉天,有李××,年二十八。下痢四十余日,脓血杂以脂膜,屡次服药,病益增剧,羸弱 
已甚。诊其脉,数而细弱,两尺尤甚。亦治以此方。服后两点钟腹疼一阵,下脓血若干。病家言∶从前腹疼不 
若是之剧,所下者亦不若是之多,似疑药不对证。愚曰∶腹中瘀滞下尽即愈矣。伸再用白蔗糖化水,送服去皮 
鸭蛋子五十粒。此时已届晚九点钟,一夜安睡,至明晨,大便不见脓血矣。后间日大便,又少带紫血,俾仍用 
山药粥送服鸭蛋子二十粒,数次全愈。 
又∶斯秋中元节后,有刘××者,下痢两月不愈,求为延医。 
其脉近和平,按之无力,日便五六次,血液腐败,便时不甚觉疼,后重亦不剧。亦治以此方,一剂病愈强半。 
翌日将行,嘱以再接原方服两剂当愈。后至奉,接其来函言∶服第二剂,效验不如从前;至三剂,病转似增重。 
因恍悟,此证下痢两月,其脉毫无数象,且按之无力,其下焦当系寒凉。俾仍用山药粥送服炒熟小茴香末一钱, 
连服数剂全愈。 
或问∶西人谓痢为肠中生炎。所谓炎者,红热肿疼,甚则腐烂也。观此案与治卢姓之案,皆用热药成功, 
亦可谓之肠炎乎?既非肠炎,何以其肠亦欲腐烂乎?答曰∶痢证,原有寒有热。热证不愈,其肠可至腐烂。寒 
证久不愈,其肠亦可腐烂。譬如疮疡,红肿者阳而热,白硬者阴而寒,其究竟皆可变为脓血。赏观《 园随笔 
录》,言其曾患牙疳,医者治以三黄犀角纯寒之品,满口肉烂尽,而色白不知疼。后医者,改用肉桂、附子等 
品,一服知疼,连服十余剂而愈。夫人口中之肌肉,犹肠中之肌肉也。口中之肌肉,可因寒而腐烂,肠中之肌 
肉,独不可因寒而腐烂乎?曾治一人,因久居潮湿之地,致下痢,三月不愈。所下者紫血杂以脂膜,腹疼后重。 
或授以龙眼肉包鸭蛋子方,服之,下痢与腹疼益剧。后愚诊视,其脉微弱而沉,左部几不见。俾用生硫黄研细, 
掺熟面少许,作丸。又重用生山药、熟地、龙眼肉煎浓汤送服。连服十余剂,共计服生硫黄两许,其痢始愈。 
由是观之,即纯系赤痢,亦诚有寒者,然不过百中之二三耳。且尝实验痢证,若因寒者,虽经久不愈,犹可支 
持。且其后重、腹疼,较因热者亦轻也。且《伤寒论》有桃花汤,治少阴病下利、便脓血者,原赤石脂与干姜 
并用,此为以热药治寒痢之权舆。注家不知,谓少阴之火伤阴络所致,治以桃花汤,原系从治之法。又有矫诬 
药性,谓赤石脂性凉,重用至一斤,干姜虽热,止用一两,其方仍以凉论者。今试取其药十分之一,煎汤服之, 
果凉乎热 
乎?此皆不知《伤寒论》此节之义,而强为注解者也。 


<篇名>7.通变白虎加人参汤
属性:治下痢,或赤、或白、或赤白参半,下重腹疼,周身发热,服凉药而热不休,脉象确有实热者。 
生石膏(二两,捣细) 生杭芍(八钱) 生山药(六钱) 人参(五钱,用野党参按此分量,若辽东真野 
参宜减半,至高丽参则断不可用) 甘草(二钱) 
上五味,用水四盅,煎取清汤两盅,分二次温饮之。 
此方,即《伤寒论》白虎加人参汤,以芍药代知母、山药代粳米也。痢疾身热不休,服清火药而热亦不休 
者,方书多诿为不治。夫治果对证,其热焉有不休之理?此乃因痢证夹杂外感,其外感之热邪,随痢深陷,永 
无出路,以致痢为热邪所助,日甚一日而永无愈期。惟治以此汤,以人参助石膏,能使深陷之邪,徐徐上升外 
散,消解无余。加以芍药、甘草以理下重腹疼,山药以滋阴固下,连服数剂,无不热退而痢愈者。 
按∶外感之热已入阳明胃腑,当治以苦寒,若白虎汤、承气汤是也。若治以甘寒,其病亦可暂愈,而恒将 
余邪锢留胃中,变为骨蒸劳热,永久不愈(《世补斋医书》论之甚详)。石膏虽非苦寒,其性寒而能散,且无 
汁浆,迥与甘寒粘泥者不同。而白虎汤中,又必佐以苦寒之知母。即此汤中,亦必佐以芍药,芍药亦味苦 
(《神家本草经》)微寒之品,且能通利小便。故以佐石膏,可以消解阳明之热而无余也。 
一叟,年六十七,于中秋得痢证,医治二十余日不效。后愚诊视,其痢赤白胶滞,下行时,觉肠中热而且 
干,小便亦觉发热,腹痛下坠并迫。其脊骨尽处,亦下坠作痛。且时作眩晕,其脉洪长有力,舌有白苔甚浓。 
愚曰∶此外感之热挟痢毒之热下 
迫,故现种种病状,非治痢兼治外感不可。遂投以此汤,两剂,诸病皆愈。其脉犹有余热,拟再用石膏清之, 
病家疑年高,石膏不可屡服,愚亦应征他往。后二十余日,痢复作。延他医治疗,于治痢药中,杂以甘寒濡润 
之品,致外感之余热,永留肠胃不去,其痢虽愈,而屡次反复。延至明年仲夏,反复甚剧。复延愚延医,其脉 
象、病证皆如旧。因谓之曰,去岁若肯多服石膏数两,何至有以后屡次反复,今不可再留邪矣。仍投以此汤, 
连服三剂,病愈而脉亦安和。 
一人,年四十二,患白痢,常觉下坠,过午尤甚,心中发热,间作寒热。医者于治痢药中,重用黄连一 
两清之,热如故,而痢亦不愈。留连两月,浸至不起。诊其脉,洪长有力,亦投以此汤。为其间作寒热,加柴 
胡二钱,一剂热退痢止,犹间有寒热之时。再诊其脉,仍似有力,而无和缓之致。知其痢久,而津液有伤也, 
遂去白芍、柴胡,加玄参、知母各六钱,一剂寒热亦愈。 
一媪,年六旬,素多疾病。于夏季晨起,偶下白痢,至暮十余次。秉烛后,忽然浑身大热,不省人事, 
循衣摸床,呼之不应。其脉洪而无力,肌肤之热烙指。知系气分热痢,又兼受暑,多病之身,不能支持,故精 
神昏愦如是也。急用生石膏三两、野台参四钱,煎汤一大碗,徐徐温饮下,至夜半尽剂而醒,痢亦遂愈。诘朝 
煎渣再服,其病脱然。 
一人,年五十余,于暑日痢而且泻,其泻与痢俱带红色,下坠腹疼,噤口不食。医治两旬,病势浸增, 
精神昏愦,气息奄奄。诊其脉,细数无力,周身肌肤发热。询其心中亦觉热,舌有黄苔,知其证夹杂暑温。暑 
气温热,弥漫胃口,又兼痢而且泻,虚热上逆,是以不能食也。遂用生山药两半、滑石一两、生杭芍六钱、粉 
甘草三钱,一剂诸病皆见愈,可以进食。又服一剂全愈。 
此证用滑石不用石膏者,以其证兼泻也。为不用石膏,即不敢用人参,故倍用山药以增其补力。此就通变之方, 
而又为通变也。 
痢证,又有肝胆肠胃先有郁热,又当暑月劳苦于烈日之中,陡然下痢,多带鲜血,脉象洪数,此纯是一团 
火气。宜急用大苦大寒之剂,若芩、连、知、柏、胆草、苦参之类,皆可选用。亦可治以白虎汤,方中生石膏 
必用至二两,再加生白芍一两。若脉大而虚者,宜再加人参三钱。若其脉洪大甚实者,可用大承气汤下之,而 
佐以白芍、知母。 
有痢久而清阳下陷者,其人或间作寒热,或觉胸中短气。当于治痢药中,加生黄 、柴胡以升清阳。脉 
虚甚者,亦可酌加人参。又当佐以生山药以固下焦,然用药不可失于热也。有痢初得,兼受外感者,宜于治痢 
药中,兼用解表之品。其外邪不随痢内陷,而痢自易治。不然,则成通变白虎加人参汤所主之证矣。 
痢证初得虽可下之,然必确审其无外感表证,方可投以下药。其身体稍弱,又宜少用参、 佐之。 
痢证忌用滞泥之品,然亦不可概论。外祖母,年九旬。仲夏下痢赤白甚剧,脉象数而且弦。愚用大熟地、 
生杭芍各一两煎汤,服下即愈。又服一剂,脉亦和平。 
痢证间有凉者,然不过百中之一耳,且又多系纯白之痢。又必脉象沉迟,且食凉物,坐凉处则觉剧者。 
治以干姜、白芍、小茴香各三钱,山楂四钱,生山药六钱,一两剂即愈。用白芍者,诚以痢证必兼下坠腹疼。 
即系凉痢,其凉在肠胃,而其肝胆间必有伏热,亦防其服热药,而生热也。 
凡病患酷嗜之物,不可力为禁止。尝见患痢者,有恣饮凉水而愈者,有饱食西瓜而愈者。总之,人之 
资禀不齐,病之变态多端,尤在临证时,精心与之消息耳。曾治一少年,下痢,昼夜无数,里急后重。投以清 
火通利之药数剂,痢已减半,而后重分毫 
不除。疑其肠中应有阻隔,投以大承气汤,下燥粪长数寸而愈。设此证,若不疑其中有阻隔,则燥粪不除,病 
将何由愈乎? 
有奇恒痢者,张隐庵谓,其证三阳并至,三阴莫当,九窍皆塞,阳气旁溢,咽干喉塞痛,并于阴则上下无 
常,薄为肠 。其脉缓小迟涩,血温身热者死,热见七日者死。盖因阳气偏盛,阴气受伤,是以脉小迟涩。此 
证急宜用大承气汤泻阳养阴,缓则不救。若不知奇恒之因,见脉气平缓,而用平易之剂,必至误事。 

\chapter{治燥结方}
<篇名>1.硝菔通结汤
属性:治大便燥结久不通,身体兼羸弱者。 
净朴硝(四两) 鲜莱菔(五斤) 
将莱菔切片,同朴硝和水煮之。初次煮,用莱菔片一斤,水五斤,煮至莱菔烂熟捞出。就其余汤,再入莱 
菔一斤。如此煮五次,约得浓汁一大碗,顿服之。若不能顿服者,先饮一半,停一点钟,再温饮一半,大便即 
通。若脉虚甚,不任通下者,加人参数钱,另炖同服。 
软坚通结,朴硝之所长也。然其味咸性寒,若遇燥结甚实者,少用之则无效,多用之则咸寒太过,损肺伤 
肾。其人或素有劳疾或下元虚寒者,尤非所宜也。惟与莱菔同煎数次,则朴硝之咸味,尽被莱菔提出,莱菔之 
汁浆,尽与朴硝融化。夫莱菔味甘,性微温,煨熟食之,善治劳嗽短气(方附水晶桃下),其性能补益可知。 
取其汁与朴硝同用,其甘温也,可化朴硝之咸寒,其补益也,可缓朴硝之攻破。若或脉虚不任通下,又借人参 
之大力者,以为之扶持保护。然后师有节制,虽猛悍亦可用也。 
一媪,年近七旬,伤寒初得,无汗,原是麻黄汤证。因误服桂枝汤,遂成白虎汤证。上焦烦热太甚,闻药 
气即呕吐。但饮所煎石膏清水,亦吐。俾用鲜梨片蘸生石膏细末,嚼咽之。约用石膏两半,阳明之大热遂消, 
而大便旬日未通,其下焦余热,仍无出路,欲用硝黄降之,闻药气仍然呕吐。且其人素患劳嗽,身体羸弱,过 
用咸寒,尤其所忌。为制此方,煎汁一大碗,仍然有朴硝余味,复用莱菔一个,切成细丝,同葱 油醋,和药 
汁调作羹。病患食之香美,并不知是药,大便得通而愈。 
一媪,年七旬,劳嗽甚剧,饮食化痰涎,不化津液,致大便燥结,十余日不行,饮食渐不能进。亦拟投以 
此汤,为羸弱已甚,用人参三钱,另炖汁,和药服之。一剂便通,能进饮食。复俾煎生山药稠汁,调柿霜饼服 
之,劳嗽亦见愈。 


<篇名>2.赭遂攻结汤
属性:治宿食结于肠间,不能下行,大便多日不通。其证或因饮食过度,或因恣食生冷,或因寒火凝结,或因呕 
吐既久,胃气冲气,皆上逆不下降。 
生赭石(二两,轧细) 朴硝(五钱) 干姜(二钱) 甘遂(一钱半,轧细药汁送服) 
热多者,去干姜。寒多者,酌加干姜数钱。呕多者,可先用赭石一两、干姜半钱煎服,以止其呕吐。呕 
吐止后,再按原方煎汤,送甘遂末服之。 
朴硝虽能软坚,然遇大便燥结过甚,肠中毫无水气者,其软坚之力,将无所施。甘遂辛窜之性,最善行 
水,能引胃中之水直达燥结之处,而后朴硝因水气流通,乃得大施其软坚之力,燥结虽久,亦可变为溏粪,顺 
流而下也。特是甘遂力甚猛悍,以攻决为用,能下行亦能上达,若无以驾驭之,服后恒至吐泻交作。况此证多 
得之涌吐之余,或因气机不能下行,转而上逆,未得施其 
攻决之力,而即吐出者。故以赭石之镇逆,干姜之降逆,协力下行,以参赞甘遂成功也。且干姜性热,朴硝性 
寒,二药并用,善开寒火之凝滞。寒火之凝滞于肠间者开,宿物之停滞于肠间者亦易开也。愚用此方救人多矣, 
即食结中脘下脘,亦未有不随手奏效者。 
乙卯之岁,客居广平,忽有车载病患,造寓求诊者。其人年过五旬,呻吟不止,言自觉食物结于下脘,甚 
是痛楚,数次延医调治,一剂中大黄用至两半不下。且凡所服之药,觉行至所结之处,即上逆吐出,饮食亦然。 
此时上焦甚觉烦躁,大便不通者已旬日矣。诊其脉,虽微弱,至数不数,重按有根。知犹可任攻下,因谓之曰∶ 
此病易治,特所服药中,有猛悍之品,服药时,必吾亲自监视方妥。然亦无须久淹,能住此四点钟,结处即通 
下矣。遂用此汤去干姜,方中赭石改用三两,朴硝改用八钱。服后须臾,腹中作响,迟两点半钟,大便通下而 
愈。后月余,又患结证如前,仍用前方而愈。 
附录∶ 
山东德州卢××来函∶ 
族侄孙××,患肠结证,缠绵两月有余。更医数十人,服药百余剂,不但无效,转大增剧。伊芳亦以为无人 
能治,无药可医。气息奄奄,殓服已备。后接先生来信(曾为去信服衷中参西录中赭遂攻结汤),即携《衷中 
参西录》往视,幸伊芳心神未昏,将赭遂攻结汤方查出示之。伊芳素知医,卧观一小时,即猛起一手拍腑,言我病 
即愈,幸不当死。立急派人取药,服后片刻,腹中大响一阵,自觉其结已开,随即大泻两三盆,停约两句钟, 
又泻数次,其病竟愈。随即食山药粉稀粥两茶杯,继用补益濡润之药数剂以善其后。 


<篇名>3.通结用葱白熨法
属性:治同前证。 
大葱白(四斤,切作细丝) 干米醋(多备待用) 
将葱白丝和醋炒至极热,分作两包,乘热熨脐上。凉则互换,不可间断。其凉者,仍可加醋少许,再炒热。 
然炒葱时,醋之多少,须加斟酌。以炒成布包后,不至有汤为度。熨至六点钟,其结自开。 
一孺子,年六岁。因食肉过多,不能消化,郁结肠中。大便不行者六七日,腹中胀满,按之硬如石,用一 
切通利药皆不效。为用此法熨之,至三点钟,其腹渐软。又熨三点钟,大便通下如羊矢,其胀遂消。 
一童子,年十五六。因薄受外感,腹中胀满,大便数日不通,然非阳明之实热燥结也。医者投以承气汤, 
大便仍不通,而腹转增胀。自觉为腹胀所迫,几不能息,且时觉心中怔忡。诊其脉,甚微细,按之即无。脉虚 
证实,几为束手。亦用葱白熨法,腹胀顿减。又熨三点钟,觉结开,行至下焦。继用猪胆汁导法,大便得通而愈。 
一人,年四十许,素畏寒凉。愚俾日服生硫黄,如黑豆粒大两块,大见功效,已年余矣。偶因暑日劳碌, 
心中有火,恣食瓜果,又饱餐肉食,不能消化,肠中结而不行,且又疼痛,时作呕吐。医者用大黄附子细辛汤 
降之,不效。又用京都薛氏保赤万应散,三剂并作一剂服之,腹疼减去,而仍不通行。后愚诊视,其脉近和平, 
微弦无力。盖此时不食数日,不大便十日矣。遂治以葱白熨法,觉腹中松畅,且时作开通之声。而仍然恶心, 
欲作呕吐。继用赭石二两,干姜钱半,俾煎服以止其恶心。仍助以葱白熨法,通其大便。外熨内攻,药逾五点 
钟,大便得通而愈。 
按∶《金匮》大黄附子细辛汤,诚为开结良方。愚尝用以治肠结腹疼者甚效。即薛氏保赤万应散,三剂作 
一剂服之,以治大人,亦为开结良方。愚用过屡次皆效。而以治此证,二方皆不效者,以其证兼呕吐,二方皆 
不能止其呕吐故也。病患自言,从前所服之药,皆觉下行未至病所,即上逆吐出。独此次服药,则沉重下达, 
直抵病结之处,所以能攻下也。 
一人,年四十三。房事后,恣食生冷,忽然少腹抽疼,肾囊紧缩。大便四日不通,上焦兼有烦躁之意。医 
者投以大黄附子细辛汤,两胁转觉疼胀。诊其脉,弦而沉,两尺之沉尤甚。先治以葱白熨法,腹中作响,大有 
开通之意。肾囊之紧缩见愈,而大便仍未通。又用赭石二两,附子五钱,当归、苏子各一两,煎汤,甫饮下, 
即觉药力下坠。俾复煎渣饮之,有顷,降下结粪若干,诸病皆愈。 
按∶此证用葱白熨之,虽未即通,而肠中之结已开。至所服之药,重用赭石者,因此证,原宜用热药以温 
下焦,而上焦之烦躁,与大便之燥结,又皆与热药不宜。惟重用赭石以佐之,使其热力下达,自无僭上之患。 
而其重坠之性,又兼有通结之功。上焦之浮热,因之归根,下焦之凝寒,因之尽化矣。 
附∶ 
(1)猪胆汁导法,乃《伤寒论》下燥结之法也。原用猪胆汁,和醋少许,以灌谷道中。今变通其法,用 
醋灌猪胆中,手捻令醋与胆汁融和,再用以通气长竹管,一端装猪胆中,用细绳扎住,一端纳谷道中。用手将 
猪胆汁,由竹管挤入谷道。若谷道离大便犹远,宜将竹管深探至燥粪之处。若结之甚者,又必连用两三个。若 
畏猪胆汁凉,或当冷时,可将猪胆置水中温之。若无鲜猪胆,可将干者,用醋泡开,再将醋灌猪胆中,以手捻 
至胆汁之凝结者皆融化,亦可用。若有灌肠注射器,则用之更便。 
(2)古方治小便忽然不通者,有葱白炙法。用葱白一握,捆作一束,将两端切齐,中留二寸。以一端安 
脐上,一端用炭火炙之。待炙至脐中发热,小便自通。此盖借其温通之性,自脐透达,转入膀胱,以启小便之 
路也。然仅以火炙其一端,则热力之透达颇难。若以拙拟葱白熨法代之。则小便之因寒不通,或因气滞不通者, 
取效当更速也。 
按∶此熨法,不但可通二便,凡疝气初得,用此法熨之,无不愈者。然须多熨几次,即熨至疝气消后,仍 
宜再熨两三次。或更加以小茴香、胡椒诸末,同炒亦佳(用胡椒末时,不宜过五钱,小茴香可多用)。 

\chapter{治消渴方}
<篇名>1.玉液汤
属性:治消渴。消渴,即西医所谓糖尿病,忌食甜物。 
生山药(一两) 生黄 (五钱) 知母(六钱) 生鸡内金(二钱,捣细) 葛根(钱半) 五味子(三钱) 
天花粉(三钱) 
消渴之证,多由于元气不升,此方乃升元气以止渴者也。方中以黄 为主,得葛根能升元气。而又佐以 
山药、知母、花粉以大滋真阴。使之阳升而阴应,自有云行雨施之妙也。用鸡内金者,因此证尿中皆含有糖质, 
用之以助脾胃强健,化饮食中糖质,为津液也。用五味者,取其酸收之性,大能封固肾关,不使水饮急于下趋也。 
方书消证,分上消、中消、下消。谓上消口干舌燥,饮水不能解渴,系心移热于肺,或肺金本体自热不 
能生水,当用人参白虎汤;中消多食犹饥,系脾胃蕴有实热,当用调胃承气汤下之;下消谓饮一斗溲亦一斗, 
系相火虚衰,肾关不固,宜用八味肾气 
丸。 
白虎加人参汤,乃《伤寒论》治外感之热,传入阳明胃腑,以致作渴之方。方书谓上消者宜用之,此借用 
也。愚曾试验多次,然必胃腑兼有实热者,用之方的。中消用调胃承气汤,此须细为斟酌,若其右部之脉滑而 
且实,用之犹可,若其人饮食甚勤,一时不食,即心中怔忡,且脉象微弱者,系胸中大气下陷,中气亦随之下 
陷,宜用升补气分之药,而佐以收涩之品与健补脾胃之品,拙拟升陷汤后有治验之案可参观。若误用承气下之, 
则危不旋踵。至下消用八味肾气丸,其方《金匮》治男子消渴,饮一斗溲亦一斗。而愚尝试验其方,不惟治男 
子甚效,即治女子亦甚效。曾治一室女得此证,用八味丸变作汤剂,按后世法,地黄用熟地、桂用肉桂,丸中 
用几两者改用几钱,惟茯苓、泽泻各用一钱,两剂而愈。后又治一少妇得此证,投以原方不效,改遵古法,地 
黄用干地黄(即今生地),桂用桂枝,分量一如前方,四剂而愈。此中有宜古宜今之不同者,因其证之凉热, 
与其资禀之虚实不同耳。 
消渴证,若其肺体有热,当治以清热润肺之品。若因心火热而铄肺者,更当用清心之药。若肺体非热,因腹 
中气化不升,轻气即不能上达于肺,与吸进之养气相合而生水者,当用升补之药,补其气化,而导之上升,此 
拙拟玉液汤之义也。然消渴之证,恒有因脾胃湿寒、真火衰微者,此肾气丸所以用桂、附。而后世治消渴,亦 
有用干姜、白术者。尝治一少年,咽喉常常发干,饮水连连,不能解渴。诊其脉微弱迟濡。投以四君子汤,加 
干姜、桂枝尖,一剂而渴止矣。又有湿热郁于中焦作渴者,苍柏二妙散、丹溪越鞠丸,皆可酌用。 
邑人某,年二十余,贸易津门,得消渴证。求津门医者,调治三阅月,更医十余人不效,归家就医于愚。 
诊其脉甚微细,旋饮水旋即小便,须臾数次。投以玉液汤,加野台参四钱,数剂渴 
见止,而小便仍数,又加萸肉五钱,连服十剂而愈。 


<篇名>2.滋饮
属性:治消渴。 
生箭 (五钱) 大生地(一两) 生怀山药(一两) 净萸肉(五钱) 生猪胰子(三钱,切碎) 
上五味,将前四味煎汤,送服猪胰子一半,至煎渣时,再送服余一半。若遇中、上二焦积有实热,脉象洪 
实者,可先服白虎加人参汤数剂,将实热消去强半,再服此汤,亦能奏效。 
消渴一证,古有上中下之分,谓其证皆起于中焦而极于上下。究之无论上消、中消、下消,约皆渴而多饮 
多尿,其尿有甜味。是以《圣济总录》论消渴谓∶“渴而饮水多,小便中有脂,似麸而甘。”至谓其证起于中 
焦,是诚有理,因中焦 病,而累及于脾也。盖 为脾之副脏,在中医书中,名为散膏,即扁鹊《难经》所谓 
脾有散膏半斤也( 尾衔接于脾门,其全体之动脉又自脾脉分支而来,故与脾有密切之关系)。有时 脏发酵, 
多酿甜味,由水道下陷,其人小便遂含有糖质。迨至 病累及于脾,致脾气不能散精达肺(《内经》谓脾气散 
精上达于肺)则津液少,不能通调水道(《内经》谓通调水道下归膀胱)则小便无节,是以渴而多饮多溲也。 
尝阅报,有患消渴,延中医治疗,服药竟愈者。所用方中,以黄 为主药,为其能助脾气上升,还其散精达肺 
之旧也。《金匮》有肾气丸,善治消渴。其方以干地黄(即生地黄)为主,取其能助肾中之真阴,上潮以润肺, 
又能协同山萸肉以封固肾关也。又向因治消渴,曾拟有玉液汤,方中以生怀山药为主,屡试有效。近阅医报且 
有单服山药以治消渴而愈者。以其能补脾固肾,以止小便频数,而所含之蛋白质,又能滋补 脏,使其散膏充 
足,且又色白入肺,能润肺生水,即以止渴也。又俗传治消渴方,单服生猪胰子可愈。盖猪胰子即猪之 ,是 
人之 病,而可补以物之 
也。此亦犹鸡内金,诸家本草皆谓其能治消渴之理也。鸡内金与猪胰子,同为化食之物也。愚因集诸药,合为 
一方,以治消渴,屡次见效。 

\chapter{治癃闭方}
<篇名>1.宣阳汤
属性:治阳分虚损,气弱不能宣通,致小便不利。 
野台参(四钱) 威灵仙(钱半) 寸麦冬(六钱,带心) 地肤子(一钱) 


<篇名>2.济阴汤
属性:治阴分虚损,血亏不能濡润,致小便不利。 
怀熟地(一两) 生龟板(五钱,捣碎) 生杭芍(五钱) 地肤子(一钱) 
阴分阳分俱虚者,二方并用,轮流换服,如下案所载服法。小便自利。 
一媪,年六十余,得水肿证,延医治不效。时有专以治水肿名者,其方秘而不传。服其药自大便泻水数桶, 
一身肿尽消,言忌咸百日,可保永愈。数日又见肿,旋复如故。服其药三次皆然,而病患益衰惫矣。盖未服其 
药时,即艰于小便,既服药后,小便滴沥全无,所以旋消而旋肿也。再延他医,皆言服此药,愈后复发者,断 
乎不能调治。后愚诊视,其脉数而无力。愚曰∶脉数者阴分虚也,无力者阳分虚也。膀胱之腑,有下口无上口, 
水饮必随气血流行,而后能达于膀胱,出为小便。《内经》所谓“州都之官,津液藏焉,气化则能出”者是也。 
此脉阴阳俱虚,致气化伤损,不能运化水饮以达膀胱,此小便所以滴沥全无也。一方,以人参为君,辅以麦冬 
以济参之热,灵仙以行参之滞,少 
加地肤子为向导药,名之曰宣阳汤。一方以熟地为君,辅以龟板以助熟地之润,芍药以行熟地之滞(芍药善利小 
便,故能行熟地之泥),亦少加地肤子为向导药,名之曰济阴汤。二方轮流服之,先服济阴汤,取其贞下起元 
也。服至三剂,小便稍利。再服宣阳汤,亦三剂小便大利。又再服济阴汤,小便直如泉涌,肿遂尽消。 
一妇人,年三十许,因阴虚小便不利,积成水肿甚剧,大便亦旬日不通,一老医投以八正散不效。友人高 
××为出方,用生白芍六两,煎汁两大碗,再用阿胶二两,熔化其中,俾病患尽量饮之。老医甚为骇疑,高×× 
力主服之。尽剂而二便皆通,肿亦顿消。后老医与愚觌面,为述其事,且问此等药何以能治此病?答曰∶此必 
阴虚不能化阳,以致二便闭塞。白芍善利小便,阿胶能滑大便,二药并用,又大能滋补真阴,使阴分充足,以 
化其下焦偏胜之阳,则二便自能通利也。 
子××治一水肿证。其人年六旬,二便皆不通利,心中满闷,时或烦躁。知其阴虚积有内热,又兼气分不舒 
也,投以生白芍三两,橘红、柴胡各三钱,一剂二便皆通。继服滋阴理气、少加利小便之药而愈。 
一妇人,年四十许,得水肿证,百药不效。偶食绿豆稀饭,觉腹中松畅,遂连服数次,小便大利而愈。 
有人向愚述其事,且问所以能愈之故。答曰∶绿豆与赤小豆同类,故能行水利小便,且其性又微凉,大能滋阴 
退热。凡阴虚有热,致小便不利者,服之皆有效也。 


<篇名>3.白茅根汤
属性:治阳虚不能化阳,小便不利,或有湿热壅滞,以致小便不利,积成水肿。 
白茅根(一斤,掘取鲜者去净皮与节间小根细切) 
将茅根用水四大碗煮一沸,移其锅置炉旁,候十数分钟,视其茅根若不沉水底,再煮一沸,移其锅置炉旁, 
须臾视其根皆沉水底,其汤即成。去渣温服多半杯,日服五六次,夜服两三次,使药力相继,周十二时,小便自利。 
茅根鲜者煮稠汁饮之,则其性微凉,其味甘而且淡。为其凉也,故能去实火。为其甘也,故能清虚热。为 
其淡也,故能利小便。又能宣通脏腑,畅达经络,兼治外感之热,而利周身之水也。然必须如此煮法,服之方 
效。若久煎,其清凉之性及其宣通之力皆减,服之即无效矣。所煮之汤,历一昼夜即变绿色,若无发酵之味, 
仍然可用。 
一妇人,年四十余,得水肿证。其翁固诸生,而精于医者,自治不效,延他医延医亦不效。偶与愚遇,问 
有何奇方,可救此危证。因细问病情,知系阴虚有热,小便不利。遂俾用鲜茅根煎浓汁,饮旬日全愈。 
一媪,年六十余,得水肿证。医者用药,治愈三次皆反复,再服前药不效。其子商于梓匠,欲买棺木,梓 
匠固其亲属,转为求治于愚。因思此证反复数次,后服药不效者,必是病久阴虚生热,致小便不利。细问病情, 
果觉肌肤发热,心内作渴,小便甚少。俾单用鲜白茅根煎汤,频频饮之,五日而愈。 
一妇人,年四十许,得水肿证。其脉象大致平和,而微有滑数之象。俾浓煎鲜茅根汤饮之,数日病愈强 
半。其子来送信,愚因嘱之曰∶有要紧一言,前竟忘却。患此证者,终身须忌食牛肉。病愈数十年,食之可以 
复发。孰意其子未返,已食牛肉。且自觉病愈,出坐庭中,又兼受风。其证陡然反复,一身尽肿,两目因肿甚 
不能开视。愚用越婢汤发之,以滑石易石膏(用越婢汤原方,常有不汗者,若以滑石易石膏则易得汗),一剂 
汗出,小便顿利,肿亦见消。再饮白茅根 
汤,数日病遂全愈。 
白茅根,拙拟二鲜饮与三鲜饮,用以治吐衄。此方又用以治水肿,而其功效又不止此也。愚治伤寒温病, 
于大便通后,阳明之盛热已消,恒俾浓煮鲜茅根汤,渴则饮之,其人病愈必速,且愈后即能饮食,更无反复之 
患。盖寒温愈后,其人不能饮食与屡次复病者,大抵因余热未尽,与胃中津液未复也。白茅根甘凉之性,既能 
清外感余热,又能滋胃中津液。至内有郁热,外转觉凉者,其性又善宣通郁热使达于外也。 
按∶凡膨胀,无论或气、或血、或水肿。治愈后,皆终身忌食牛肉。盖牛肉属土,食之能壅滞气血,且其 
彭亨之形,有似腹胀,故忌之也。医者治此等证,宜切嘱病家,慎勿误食。 


<篇名>4.温通汤
属性:治下焦受寒,小便不通。 
椒目(八钱,炒捣) 小茴香(二钱,炒捣) 威灵仙(三钱) 
人之水饮,由三焦而达膀胱。三焦者,身内脂膜也。曾即物类验之,其脂膜上皆有微丝血管,状若红绒毛, 
即行水之处。此管热则膨涨,凉则凝滞,皆能闭塞水道。若便浊兼受凉者,更凝结稠粘杜塞溺管,滴沥不通。故 
以椒目之滑而温、茴香之香而热者,散其凝寒,即以通其窍络。更佐以灵仙温窜之力,化三焦之凝滞,以达膀胱 
,即化膀胱之凝滞,以达溺管也。凉甚者,肉桂、附子、干姜皆可酌加。气分虚者,更宜加人参助气分以行药力。 


<篇名>5.加味苓桂术甘汤
属性:治水肿小便不利,其脉沉迟无力,自觉寒凉者。 
于术(三钱) 桂枝尖(二钱) 茯苓片(二钱) 甘草(一钱) 干姜(三钱) 人参(三钱) 乌附子(二钱) 
威灵仙(一钱五分) 
肿满之证,忌用甘草,以其性近壅滞也。惟与茯苓同用,转能泻湿满,故方中未将甘草减去。若肿胀甚剧, 
恐其壅滞者,去之亦可。 
服药数剂后,小便微利;其脉沉迟如故者,用此汤送服生硫黄末四五厘。若不觉温暖,体验渐渐加多,以 
服后移时觉微温为度。 
人之水饮,非阳气不能宣通。上焦阳虚者,水饮停于膈上。中焦阳虚者,水饮停于脾胃。下焦阳虚者,水 
饮停于膀胱。水饮停蓄既久,遂渐渍于周身,而头面肢体皆肿,甚或腹如抱瓮,而膨胀成矣。此方用苓桂术甘 
汤,以助上焦之阳。即用甘草协同人参、干姜,以助中焦之阳。又人参同附子,名参附汤(能固下焦元阳将脱) 
协同桂枝,更能助下焦之阳(桂枝上达胸膈,下通膀胱故肾气丸用桂枝不用肉桂)。三焦阳气宣通,水饮亦随 
之宣通,而不复停滞为患矣。至灵仙与人参并用,治气虚小便不利甚效(此由实验而知,故前所载宣阳汤并用 
之)。而其通利之性,又能运化术、草之补力,俾胀满者服之,毫无滞碍,故加之以为佐使也。若药服数剂后, 
脉仍如故,病虽见愈,实无大效,此真火衰微太甚,恐非草木之品所能成功。故又用生硫黄少许,以补助相火。 
诸家本草谓其能使大便润,小便长,补火之中大有行水之力,故用之因凉成水肿者尤良也。服生硫黄法,其中 
有治水肿之验案宜参观。 
脉沉水肿,与脉浮水肿迥异。脉浮者,多系风水,腠理闭塞,小便不利。当以《金匮》越婢汤发之,通身 
得汗,小便自利。若浮而兼数者,当是阴虚火动,宜兼用凉润滋阴之药。脉沉水肿,亦未可遽以凉断。若沉而 
按之有力者,系下焦蕴热未化,仍当用凉润之药,滋阴以化其阳,小便自利。惟其脉沉而且迟,微弱欲无,询 
之更自觉寒凉者,方可放胆用此汤无碍。或但服生硫黄,试验渐渐加多,亦可奏效。特是肿之剧者,脉之部位皆 
肿,似难辨其沉浮与有力无力,必重按移时,使按处成凹,始能细细辨认。 
苓桂术甘汤,为治上焦停饮之神方。《金匮》曰∶“短气有微饮,当从小便去之,苓桂术甘汤主之,肾 
气丸亦主之。”喻嘉言注云∶“呼气短,宜用苓桂术甘汤,以化太阳(膈上)之气;吸气短,宜用肾气丸,以 
纳少阴(肾经)之气。”推喻氏之意,以为呼气短,则上焦阳虚,吸气短,则下焦阴虚,故二方分途施治。然 
以之为学人说法,以自明其别有会心则可;以之释《金匮》,谓其文中之意本如是则不可。愚临证体验多年, 
见有膈上气旺而膺胸开朗者,必能运化水饮,下达膀胱,此用苓桂术甘汤治饮之理也。见有肾气旺,而膀胱流 
通者,又必能吸引水饮,下归膀胱,此用肾气丸治饮之理也。故仲景于上焦有微饮而短气者,并出两方,任人 
取用其一,皆能立建功效。况桂枝为宣通水饮之妙药,茯苓为淡渗水饮之要品,又为二方之所同乎。且《金匮》 
之所谓短气,乃呼气短,非吸气短也。何以言之,吸气短者,吸不归根即吐出,《神农本草经》所谓吐吸,即 
喘之替言也。《金匮》之文,有单言喘者,又有短气与喘并举者。若谓短气有微饮句,当兼呼气短与吸气短 
而言,而喘与短气并举者,又当作何解耶(惟论溢饮变其文曰气短似言吸气短)? 
用越婢汤治风水,愚曾经验,遇药病相投,功效甚捷。其方《金匮》以治风水恶风,一身悉肿,脉浮不 
渴,续自汗出,无大热者。而愚临证体验以来,即非续自汗出者,用之亦可,若一剂而汗不出者,可将石膏易 
作滑石(分量须加重)。 


<篇名>6.寒通汤
属性:治下焦蕴蓄实热,膀胱肿胀,溺管闭塞,小便滴沥不通。 
滑石(一两) 生杭芍(一两) 知母(八钱) 黄柏(八钱) 
一人,年六十余,溺血数日,小便忽然不通,两日之间滴沥全无。病患不能支持,自以手揉挤,流出血 
水少许,稍较轻松。揉挤数次,疼痛不堪揉挤。 徨无措,求为延医。其脉沉而有力,时当仲夏,身复浓被, 
犹觉寒凉。知其实热郁于下焦,溺管因热而肿胀不通也。为拟此汤,一剂稍通,又加木通、海金沙各二钱,服 
两剂全愈。 


<篇名>7.升麻黄汤
属性:治小便滴沥不通。偶因呕吐咳逆,或侧卧欠伸,可通少许,此转胞也。用升提药,提其胞而转正之,胞 
系不了戾,小便自利。 
生黄 (五钱) 当归(四钱) 升麻(二钱) 柴胡(二钱) 
一妇人,产后小便不利,遣人询方。俾用生化汤加白芍,治之不效,复来询方。言有时恶心呕吐,小便 
可通少许。愚恍悟曰,此必因产时努力太过,或撑挤太甚,以致胞系了戾,是以小便不通。恶心呕吐,则气机 
上逆,胞系有提转之势,故小便可以稍通也。遂为拟此汤,一剂而愈。 
三焦之气化不升则不降。小便不利者,往往因气化下陷,郁于下焦,滞其升降流行之机也。故用一切利 
小便之药不效,而投以升提之药恒多奇效。是以拙拟此汤,不但能治转胞,并能治小便癃闭也。 
古方有但重用黄 ,治小便不利,积成水肿者(参阅陆定圃《冷庐医话》)。 
水肿之证,有虚有实,实者似不宜用黄 。然其证实者甚少,而虚者居多。至其证属虚矣,又当详辨其 
为阴虚阳虚,或阴阳俱虚。阳虚者气分亏损,可单用、重用黄 。阴虚者其血分枯耗,宜重用滋阴之药,兼取 
阳生阴长之义,而以黄 辅之。至阴阳俱虚者,黄 与滋阴之药,可参半用之。医者不究病因,痛诋为不可用, 
固属卤莽,至其连用除湿猛剂,其卤莽尤甚。盖病至 
积成水肿,即病因实者,其气血至此,亦有亏损。猛悍药,或一再用犹可。若不得已而用至数次,亦宜以补气 
血之药辅之。况其证原属重用黄 治愈之虚证乎。至今之医者,对于此证,纵不用除湿猛剂,亦恒多用利水之 
品。不知阴虚者,多用利水之药则伤阴;阳虚者,多用利水之药亦伤阳。夫利水之药,非不可用,然贵深究其 
病因,而为根本之调治,利水之药,不过用作向导而已。 
【附方】葛稚川《肘后方》治小便不通,用大蝼蛄二枚,取下体,以水一升渍饮,须臾即通。 
又《寿域方》用土狗后半,焙研调服半钱,小便即通,生研亦可。 
又《唐氏经验方》用土狗后截和麝香捣,纳脐中缚定,即通。 
按∶土狗即蝼蛄,《日华诸家本草》谓其治水肿,头面肿。李时珍谓其通大小便,治石淋,诚为利小便 
要药。凡小便不通者,无论凉热虚实,皆可加于药中以为向导。即单服之,亦甚有效验。然观古方,皆用其后 
半截。盖其前半,开破之力多,后半利水力多。若治二便皆不通者,当全用之。 
俗传∶治小便不通闻药方。用明雄黄一钱,蟾酥五分(焙发),麝香六厘,共研细,鼻闻之,小便即通。 


<篇名>8.鸡汤
属性:治气郁成臌胀,兼治脾胃虚而且郁,饮食不能运化。 
生鸡内金(四钱,去净瓦石糟粕捣碎) 于术(三钱) 生杭芍(四钱) 柴胡(二钱) 广陈皮(二钱) 生姜 
(三钱) 
《内经》谓∶“诸湿肿满,皆属于脾。”诚以脾也者,与胃相连以膜,能代胃行其津液。且地居中焦 
(为中焦油膜所包),更能为四旁宣其气化。脾若失其所司,则津液气化凝滞,肿满即随之矣。是臌胀者,当 
以理脾胃为主也。西人谓脾体中虚,内多回血管。若 
其回血管之血,因脾病不能流通,瘀而成丝成块,原非草木之根 所能消化。鸡内金为鸡之脾胃,中有瓦石铜 
铁皆能消化,其善化有形瘀积可知。故能直入脾中,以消回血管之瘀滞。而又以白术之健补脾胃者以驾驭之, 
则消化之力愈大。柴胡,《神农本草经》谓“主肠胃中饮食积聚,能推陈致新”,其能佐鸡内金消瘀可知。且 
与陈皮并用,一升一降,而气自流通也。用芍药者,因其病虽系气臌,亦必挟有水气,芍药善利小便,即善行 
水,且与生姜同用,又能调和营卫,使周身之气化流通也。夫气臌本为难治之证,从拟此方之后,连治数证皆效。 
治一叟年六旬,腹胀甚剧。治以此汤数剂,其效不速。用黑丑一钱炒研细,煎此汤送下,两剂大见功效。 
又去黑丑,再服数剂全愈。若小便时觉热,且色黄赤者,宜酌加滑石数钱。 
鸡内金虽饶有消化之力,而诸家本草,实有能缩小便之说,恐于证之挟有水气者不宜。方中用白芍以利 
小便,所以济鸡内金之短也。 
《内经》谓∶“按之 而不起者,风水也。”愚临证体验以来,知凡系水臌,按之皆不能即起。气臌则 
按之举手即起。或疑若水积腹中,不行于四肢,如方书所谓单腹胀者,似难辨其为气为水。不知果为水证,重 
按移时,举手则有微痕,而气证则无也。且气臌证,小便自若,水臌证,多小便不利,此又其明征也。 


<篇名>9.鸡茅根汤
属性:治水臌气臌并病,兼治单腹胀,及单水臌胀,单气臌胀。 
生鸡内金(五钱,去净瓦石糟粕轧细) 生于术(分量用时斟酌) 鲜茅根(二两,锉细) 
先将茅根煎汤数茶盅(不可过煎,一两沸后慢火温至茅根沉水底汤即成)。先用一盅半,加生姜五片, 
煎鸡内金末,至半盅时,再添茅根汤一盅,七八沸 
后,澄取清汤(不拘一盅或一盅多)服之。所余之渣,仍用茅根汤煎服。日进一剂,早晚各服药一次。初服小 
便即多,数日后大便亦多。若至日下两三次,宜减鸡内金一钱,加生于术一钱。又数日,胀见消,大便仍勤, 
可减鸡内金一钱,加于术一钱。又数日,胀消强半,大便仍勤,可再减鸡内金一钱,加于术一钱。如此精心随 
病机加减,俾其补破之力,适与病体相宜,自能全愈。若无鲜茅根,可用药局中干茅根一两代之。无鲜茅根即 
可不用生姜。所煎茅根汤,宜当日用尽,煎药后若有余剩,可当茶温饮之。 
鸡内金之功效,前方下已详论之矣。至于茅根最能利水,人所共知。而用于此方,不但取其利水也,茅根 
春日发生最早,是禀一阳初生之气,而上升者也。故凡气之郁而不畅者,茅根皆能畅达之。善利水又善理气, 
故能佐鸡内金,以奏殊功也。加生姜者,恐鲜茅根之性微寒也。且其味辛能理气,其皮又善利水也。继加于术, 
减鸡内金者,因胀已见消,即当扶正以胜邪,不敢纯用开破之品,致伤其正气也。或疑此方,初次即宜少加 
于术者。而愚曾经试验,早加于术,固不若晚加之有效也。 
或问∶茅根能清热利小便,人所共知。至谓兼理气分之郁,诸家本草皆未言及,子亦曾单用之,而有确实 
之征验乎?答曰∶此等实验,已不胜记。曾治一室女,心中常觉发热,屡次服药无效。后愚为诊视,六脉皆沉 
细,诊脉之际,闻其太息数次,知其气分不舒也。问其心中胁下,恒隐隐作疼。遂俾剖取鲜茅根,锉细半斤, 
煎数沸当茶饮之。两日后,复诊其脉,已还浮分,重诊有力,不复闻其太息。问其胁下,已不觉疼,惟心中仍 
觉发热耳。再饮数日,其心中发热亦愈。 
又尝治少年,得肺鼠疫病。其咽喉唇舌,异常干燥。精神昏昏似睡。周身肌肤不热。脉象沉微。问其心中, 
时常烦闷。此鼠疫之邪,闭塞其少阴,致肾气不能上达也。问其大便,四日未 
行。遂投以大剂白虎加人参汤,先用茅根数两煎汤,以之代水煎药,取汁三盅,分三次饮下。其脉顿起,变作 
洪滑之象。精神已复,周身皆热,诸病亦皆见愈。俾仍按原方将药煎出,每饮一次,调入生鸡子黄一枚,其病 
遂全愈。盖茅根生于水边,原兼禀寒水之气。且其出地之时,作尖锐之锥形,故能直入少阴,助肾气上达,与 
心相济,则心即跳动有力,是以其脉,遂洪滑外现也。再加生鸡子黄,以滋少阴之液,俾其随气上升,以解上 
焦之因燥生热,因热生烦,是以诸病皆愈也。此二案皆足征茅根理气之效也。 

\chapter{治淋浊方}
<篇名>1.理血汤
属性:治血淋及溺血、大便下血,证之由于热者。 
生山药(一两) 生龙骨(六钱,捣细) 生牡蛎(六钱,捣细) 海螵蛸(四钱,捣细) 茜草(二钱) 
生杭芍(三钱) 白头翁(三钱) 真阿胶(三钱,不用炒) 
溺血者,加龙胆草三钱。大便下血者,去阿胶,加龙眼肉五钱。 
血淋之症,大抵出之精道也。其人或纵欲太过而失于调摄,则肾脏因虚生热。或欲盛强制而妄言采补,则 
相火动无所泄,亦能生热。以致血室(男女皆有,男以化精女以系胞)中血热妄动,与败精混合化为腐浊之物, 
或红、或白、成丝、成块,溺时杜塞牵引作疼。故用山药、阿胶以补肾脏之虚,白头翁其性寒凉,其味苦而兼 
涩,凉血之中大有固脱之力,故以清肾脏之热,茜草、螵蛸以化其凝滞而兼能固其滑脱,龙骨、牡蛎以固其滑 
脱而兼能化其凝滞,芍药 
以利小便而兼能滋阴清热,所以投之无不效也。此证,间有因劳思过度而心热下降,忿怒过甚而肝火下移以成 
者,其血必不成块,惟溺时牵引作疼。此或出之溺道,不必出自精道也。投以此汤亦效。 
溺血之证,不觉疼痛,其证多出溺道,间有出之精道者。大抵心移热于小肠,则出之溺道。肝移热于血室, 
则出之精道。方中加生地黄者,泻心经之热也。若系肝移热于血室者,加龙胆草亦可。 
一人,年三十许,患血淋。溲时血块杜塞,努力始能溲出,疼楚异常。且所溲者上多浮油,胶粘结于器底, 
是血淋而兼膏淋也。从前延医调治,经三十五人,服药年余,分毫无效, 羸已甚。后愚诊视,其脉弦细,至 
数略数,周身肌肤甲错,足骨凸处,其肉皮皆成旋螺高寸余,触之甚疼。盖卧床不起者,已半载矣。细询病因, 
谓得之忿怒之余误坠水中,时当秋夜觉凉甚,遂成斯证。知其忿怒之火,为外寒所束,郁于下焦而不散,而从 
前居室之间,又有失保养处也。拟投以此汤,为脉弦,遂以柏子仁(炒捣)八钱,代方中山药,以其善于养肝 
也。疏方甫定,其父出所服之方数十纸,欲以质其同异。愚曰∶无须细观,诸方与吾方同者,惟阿胶白芍耳, 
阅之果然。其父问何以知之?愚曰∶吾所用之方,皆苦心自经营者,故与他方不同。服三剂血淋遂愈,而膏淋 
亦少减。改用拙拟膏淋汤,连服二十余剂,膏淋亦愈,而小便仍然频数作疼。细询其疼之实状,谓少腹常觉疼 
而且坠,时有欲便之意,故有尿即不能强忍,知其又兼气淋也。又投以拙拟气淋汤,十剂全愈。周身甲错,足 
上旋螺尽脱。 
溺血之证,热者居多,而间有因寒者,则此方不可用矣。曾治一人,年三十余,陡然溺血,其脉微弱而迟, 
自觉下焦凉甚。知其中气虚弱,不能摄血,又兼命门相火衰微,乏吸摄之力,以 
致肾脏不能封固,血随小便而脱出也。投以四君子汤,加熟地、乌附子,连服二十余剂始愈。又有非凉非热, 
但因脾虚不能统血而溺血者。方书所谓失于便溺者,太阴之不升也。仍宜用四君子汤,以龙骨、牡蛎佐之。 
大便下血者,大抵由于肠中回血管或血脉管破裂。方中龙骨、牡蛎之收涩,原可补其破裂之处。而又去阿 
胶者,防其滑大肠也。加龙眼肉者,因此证间有因脾虚不能统血而然者,故加龙眼肉以补脾。若虚甚者,又当 
重用白术,或更以参、 佐之。若虚而且陷者,当兼佐以柴胡、升麻。若虚而且凉者,当兼佐以干姜、附子, 
减去芍药、白头翁。一少妇,大便下血月余,屡次服药不效。愚为诊视,用理血汤,去阿胶,加龙眼肉五钱治 
之。而僻处药坊无白头翁,权服一剂,病稍见愈。翌日至他处药坊,按方取药服之,病遂全愈。则白头翁之功 
效,何其伟哉! 
附录∶ 
直隶唐山张××来函∶ 
张××,年二十八岁,于冬月初,得膏淋,继之血淋。所便者,或血条,或血块,后则继以鲜血,溺频茎疼。 
屡经医者调治,病转加剧。其气色青黑,六脉坚数,肝脉尤甚。与以理血汤,俾连服三剂,血止,脉稍平,他 
证仍旧。继按治淋浊方诸方加减治之,十余剂全愈。 


<篇名>2.膏淋汤
属性:治膏淋。 
生山药(一两) 生芡实(六钱) 生龙骨(六钱,捣细) 生牡蛎(六钱,捣细) 大生地(六钱,切片) 
潞党参(三钱) 生杭芍(三钱) 
膏淋之证,小便混浊,更兼稠粘,便时淋涩作疼。此证由肾脏亏损,暗生内热。肾脏亏损则蛰藏不固, 
精气易于滑脱。内热暗生,则膀胱熏蒸,小便改其澄清。久之,三焦之气化滞其升降之机,遂至便时牵引作疼, 
而混浊稠粘矣。故用山药、芡实以补其虚,而兼有收摄之功。龙骨、牡蛎以固其脱,而兼有化滞之用。地黄、 
芍药以清热利便。潞参以总提其气化,而斡旋之也。若其证混浊,而不稠粘者,是但出之溺道,用此方时,宜 
减龙骨、牡蛎之半。 


<篇名>3.气淋汤
属性:治气淋。 
生黄 (五钱) 知母(四钱) 生杭芍(三钱) 柴胡(二钱) 生明乳香(一钱) 生明没药(一钱) 
气淋之证,少腹常常下坠作疼,小便频数,淋涩疼痛。因其人下焦本虚,素蕴内热,而上焦之气化又复下 
陷,郁而生热,则虚热与湿热,互相结于太阳之腑,滞其升降流通之机而气淋之证成矣。故以升补气化之药为 
主,而以滋阴利便流通气化之药佐之。 


<篇名>4.劳淋汤
属性:治劳淋。 
生山药(一两) 生芡实(三钱) 知母(三钱) 真阿胶(三钱,不用炒) 生杭芍(三钱) 
劳淋之证,因劳而成。其人或劳力过度、或劳心过度、或房劳过度,皆能暗生内热,耗散真阴。阴亏热 
炽,熏蒸膀胱,久而成淋,小便不能少忍,便后仍复欲便,常常作疼。故用滋补真阴之药为主,而少以补气之 
药佐之,又少加利小便之药作向导。然 
此证得之劳力者易治,得之劳心者难治,得之房劳者尤难治。又有思欲无穷,相火暗动而无所泄,积久而成淋 
者,宜以黄柏、知母以凉肾,泽泻、滑石以泻肾,其淋自愈。 
或问∶以上治淋四方中,三方以山药为君,将山药之性与淋证最相宜乎?答曰∶阴虚小便不利者,服山药 
可利小便。气虚小便不摄者,服山药可摄小便。盖山药为滋阴之良药,又为固肾之良药,以治淋证之淋涩频数, 
诚为有一无二之妙品。再因证而加以他药辅佐之,所以投之辄效也。 


<篇名>5.砂淋丸
属性:治砂淋,亦名石淋。 
黄色生鸡内金(一两,鸡鸭皆有肫皮而鸡者色黄宜去净砂石) 生黄 (八钱) 知母(八钱) 生杭芍 
(六钱) 蓬砂(六钱) 朴硝(五钱) 硝石(五钱) 
共轧细,炼蜜为丸,桐子大,食前开水送服三钱,日两次。 
石淋之证,因三焦气化瘀滞,或又劳心劳力过度,或房劳过度,膀胱暗生内热。内热与瘀滞煎熬,久而结 
成砂石,杜塞溺道,疼楚异常。其结之小者,可用药化之,若大如桃、杏核以上者,不易化矣。须用西人剖 
取之法,此有关性命之证,剖取之法虽险,犹可于险中求稳也。 
鸡内金为鸡之脾胃,原能消化砂石。蓬砂可为金、银、铜焊药,其性原能柔五金、治骨鲠,故亦善消硬物。 
朴硝,《神农本草经》谓其能化七十二种石。硝石,《神农本草经》不载,而《名医别录》载之,亦谓其能化 
七十二种石。想此二物性味相近,古原不分,即包括于朴硝条中,至陶隐居始别之,而其化石之能则同也。然 
诸药皆消破之品,恐于元气有伤,故加黄 以补助气分,气分壮旺,益能运化药力。犹恐黄 性热,与淋证不 
宜,故又加知母、 
芍药以解热滋阴,而芍药之性,又善引诸药之力至膀胱也。 
按∶此证有救急之法。当石杜塞不通时,则仰卧溺之可通。若仍不通,或侧卧、或立、或以手按地,俾石 
离其杜塞之处即可通。 


<篇名>6.寒淋汤
属性:治寒淋。 
生山药(一两) 小茴香(二钱,炒捣) 当归(三钱) 生杭芍(二钱) 椒目(二钱,炒捣) 
上所论五淋,病因不同而证皆兼热。此外,实有寒热凝滞,寒多热少之淋。其证喜饮热汤,喜坐暖处,时 
常欲便,便后益抽引作疼,治以此汤服自愈。 


<篇名>7.秘真丹
属性:治诸淋证已愈,因淋久气化不固,遗精白浊者。 
五倍子(一两,去净虫粪) 粉甘草(八钱) 
上二味共轧细,每服一钱,竹叶煎汤送下,日再服。 
曾治一人,从前患毒淋,服各种西药两月余,淋已不疼,白浊亦大见轻,然两日不服药,白浊仍然反复。 
愚俾用膏淋汤,送服秘真丹,两次而愈。 


<篇名>8.毒淋汤
属性:治花柳毒淋,疼痛异常,或兼白浊,或兼溺血。 
金银花(六钱) 海金沙(三钱) 石韦(二钱) 牛蒡子(二钱,炒捣) 甘草梢(二钱) 生杭芍(三钱) 
三七(二钱,捣细) 鸭蛋子(三十粒,去皮) 
上药八味,先将三七末、鸭蛋子仁用开水送服,再服余药所煎之汤。 
此证若兼受风者,可加防风二三钱。若服药数剂后,其疼瘥减,而白浊不除,或更遗精者,可去三七、鸭 
蛋子,加生龙骨、生牡蛎各五钱。 
鸭蛋子味至苦,而又善化瘀解毒清热,其能消毒菌之力,全在于此。又以三七之解毒化腐生肌者佐之,以 
加于寻常治淋药中,是以治此种毒淋,更胜于西药也。 


<篇名>9.清毒二仙丹
属性:治花柳毒淋,无论初起、日久,凡有热者,服之皆效。 
丈菊子(一两,捣碎) 鸭蛋子(四十粒,去皮仁破者勿用服时宜囫囵吞下) 
上药二味,将丈菊子煎汤一盅,送服鸭蛋子仁。 
丈菊俗名向日葵,其花善催生,子善治淋。邻村一少年患此证,便时膏淋与血液相杂,疼痛颇剧,语以此 
方,数次全愈。 


<篇名>10.鲜小蓟根汤
属性:治花柳毒淋,兼血淋者。 
鲜小蓟根(一两,洗净锉细) 
上一味,用水煎三四沸,取清汤一大茶盅饮之,一日宜如此饮三次。若畏其性凉者,一次用六七钱亦可。 
曾治一少年患此证,所便者血溺相杂,其血成丝、成块,间有脂膜,疼痛甚剧,且甚腥臭。屡次医治无效, 
授以此方,连服五日全愈。 
按∶如毒淋之兼血淋者,而与鸭蛋子、三七、鲜小蓟根并用则效。 
小蓟于三鲜饮下曾言之。然彼则用治吐血,此则用治毒淋中 
之血淋,皆极效验,而其功用实犹不止此也。一十五六岁童子,项下起疙瘩数个,大如巨栗,皮色不变,发热 
作疼。知系阳证,俾浓煎鲜小蓟根汤,连连饮之,数日全消。盖其善消血中之热毒,又能化瘀开结,故有如此 
功效也。 


<篇名>11.澄化汤
属性:治小便频数,遗精白浊,或兼疼涩,其脉弦数无力,或咳嗽、或自汗、或阴虚作热。 
生山药(一两) 生龙骨(六钱,捣细) 牡蛎(六钱,捣细) 牛蒡子(三钱,炒捣) 生杭芍(四钱) 
粉甘草(钱半) 生车前子(三钱,布包) 


<篇名>12.清肾汤
属性:治小便频数疼涩,遗精白浊,脉洪滑有力,确系实热者。 
知母(四钱) 黄柏(四钱) 生龙骨(四钱,捣细) 生牡蛎(三钱,炒捣) 海螵蛸(三钱,捣细) 
茜草(二钱) 生杭芍(四钱) 生山药(四钱) 泽泻(一钱半) 
或问∶龙骨、牡蛎收涩之品也。子治血淋,所拟理血汤中用之,前方治小便频数或兼淋涩用之,此方治 
小便频数疼涩亦用之,独不虑其收涩之性有碍于疼涩乎?答曰∶龙骨、牡蛎敛正气而不敛邪气,凡心气耗散、 
肺气息贲、肝气浮越、肾气滑脱,用之皆有捷效。即证兼瘀、兼疼或兼外感,放胆用之,毫无妨碍。拙拟补络 
补管汤、理郁升陷汤、从龙汤、清带汤,诸方中论之甚详,皆可参观。 
一叟,年七十余,遗精白浊、小便频数,微觉疼涩。诊其六脉平和,两尺重按有力,知其年虽高,而肾 
经确有实热也。投以此汤,五剂全愈。 
一人,年三十许,遗精白浊,小便时疼如刀 ,又甚涩数。诊其脉滑而有力,知其系实热之证。为其年 
少,疑兼花柳毒淋,遂投以此汤,加没药(不去油)三钱、鸭蛋子(去皮)四十粒(药汁送服),数剂而愈。 


<篇名>13.舒和汤
属性:治小便遗精白浊,因受风寒者,其脉弦而长,左脉尤甚。 
桂枝尖(四钱) 生黄 (三钱) 续断(三钱) 桑寄生(三钱) 知母(三钱) 
服此汤数剂后病未全愈者,去桂枝,加龙骨、牡蛎(皆不用 )各六钱。 
东海渔者,年三十余,得骗白证甚剧。旬日之间,大见衰惫,惧甚,远来求方。其脉左右皆弦,而左部 
弦而兼长。夫弦长者,肝木之盛也。木与风为同类,人之脏腑,无论何处受风,其风皆与肝木相应。《内经》 
阴阳应象论所谓“风气通于肝”者是也。脉之现象如此,肝因风助,倍形其盛,而失其和也。况病患自言,因 
房事后小盒饭风,从此外肾微肿,遂有此证,尤为风之明征乎。盖房事后,肾脏经络虚而不闭,风气乘虚袭入, 
鼓动肾脏不能蛰藏(《内经》谓肾主蛰藏),而为肾行气之肝木,又与风相应,以助其鼓动,而大其疏泄 
(《内经》肝主疏泄),故其病若是之剧也。为拟此汤,使脉之弦长者,变为舒和。服之一剂见轻,数剂后遂 
全愈。以后凡遇此等症,其脉象与此同者,投以此汤无不辄效。 

\chapter{治伤寒方}
<篇名>1.麻黄加知母汤
属性:治伤寒无汗。 
麻黄(四钱) 桂枝尖(二钱) 甘草(一钱) 杏仁(二钱,去皮炒) 知母(三钱) 
先煮麻黄五六沸,去上沫,纳诸药煮取一茶盅。温服复被,取微似汗,不须啜粥,余如桂枝法将息。 
麻黄汤原方,桂枝下有去皮二字,非去枝上之皮也。古人用桂枝,惟取梢尖嫩枝折视之,内外如一,皮 
骨不分。若见有皮骨可分辨者,去之不用,故曰去皮。陈修园之侄鸣岐曾详论之。 
《伤寒论》太阳篇中麻黄汤,原在桂枝汤后。而麻黄证多,桂枝证不过十中之一二,且病名伤寒,麻 
黄汤为治伤寒初得之主方,故先录之。 
伤寒之证,先自背受之,背者足太阳所辖之部位也。是以其证初得,周身虽皆恶寒,而背之恶寒尤甚, 
周身虽皆觉疼,而背下连腿之疼痛尤甚。其脉阴阳俱紧者,诚以太阳为周身外卫之阳,陡为风寒所袭,逼其阳 
气内陷,与脉相并,其脉当有力,而作起伏迭涌之势。而寒气之缩力(凡物之体热则涨,寒则缩),又将外卫 
之气缩紧,逼压脉道,使不得起伏成波澜,而惟现弦直有力之象。甚或因不能起伏,而至左右弹动。 
凡脉之紧者必有力。夫脉之跳动,心脏主之。而其跳动之有力,不但心主之也;诸脏腑有热皆可助脉之 
跳动有力,营卫中有热亦可助脉之跳动有力。特是脉之有力者,恒若水之有浪,大有起伏之势。而紧脉虽有力, 
转若无所起伏,诚以严寒束其外表, 
其收缩之力能逼营卫之热内陷与脉相并,以助其有力;而其收缩之力又能遏抑脉之跳动,使无起伏。是紧脉之 
真相,原于平行中见其有力也。至于紧脉或左右弹者,亦蓄极而旁溢之象也。仲师治以麻黄汤,所以解外表所 
束之寒也。 
方中用麻黄之性热中空者,直走太阳之经,外达皮毛,借汗解以祛外感之寒。桂枝之辛温微甘者,偕同甘 
草以温肌肉、实腠理,助麻黄托寒外出。杏仁之苦降者,入胸中以降逆定喘。原方止此四味,而愚为加知母者, 
诚以服此汤后,间有汗出不解者,非因汗出未透,实因余热未清也。佐以知母于发表之中,兼寓清热之意,自 
无汗后不解之虞。此乃屡经试验,而确知其然,非敢于经方轻为加减也。 
或问∶喘为肺脏之病,太阳经于肺无涉,而其证多兼微喘者何也?答曰∶胸中亦太阳部位,其中所积之大 
气,原与周身卫气,息息相通。卫气既为寒气所束,则大气内郁,必膨胀而上逆冲肺,此喘之所由来也。又风 
寒袭于皮毛,必兼入手太阴肺经,挟痰涎凝郁肺窍,此又喘之所由来也。麻黄能兼入手太阴经,散其在经之风 
寒,更能直入肺中,以泻其郁满。所以能发太阳之汗者不仅麻黄,而仲景独取麻黄,为治足经之药,而手经亦 
兼顾无遗,此仲景制方之妙也。 
凡利小便之药,其中空者,多兼能发汗, 蓄、木通之类是也。发汗之药,其中空者,多兼能利小便,麻 
黄、柴胡之类是也。太阳经病,往往兼及于膀胱,以其为太阳之腑也。麻黄汤治太阳在经之邪,而在腑者亦兼 
能治之。盖在经之邪,由汗而解,而在腑之邪,亦可由小便而解。彼后世自作聪明,恒用他药以代麻黄汤者, 
于此义盖未之审也。 
大青龙汤,治伤寒无汗烦躁。是胸中先有内热,无所发泄,遂郁而作烦躁,故于解表药中,加石膏以清内 
热。然麻黄与石膏 
并用,间有不汗之时。若用此方,将知母加重数钱,其寒润之性,能入胸中化合而为汁,随麻、桂以达于外, 
而烦躁自除矣。 
伤寒与温病,始异而终同。为其始异也,故伤寒发表,可用温热,温病发表必须辛凉。为其终同也,故病 
传阳明之后,无论寒温,皆宜治以寒凉,而大忌温热。兹编于解表类中,略取《伤寒论》太阳篇数方,少加疏 
解,俾初学知伤寒初得治法,原异于温病,因益知温病初得治法,不同于伤寒。至于伤寒三阴治法,虽亦与温 
病多不同,然其证甚少。若扩充言之,则凡因寒而得之霍乱、痧证,又似皆包括其中。精微浩繁,万言莫罄, 
欲精其业者,取原书细观可也。 


<篇名>2.加味桂枝代粥汤
属性:治伤寒有汗。 
桂枝尖(三钱) 生杭芍(三钱) 甘草(钱半) 生姜(三钱) 大枣(三枚,掰开) 生黄 (三钱) 
知母(三钱) 防风(二钱) 
煎汤一茶盅,温服复被,令一时许,遍身 微似有汗者益佳。不可如水流漓,病必不除。禁生冷、粘滑、 
肉面、五辛、酒酪及臭恶等物。 
桂枝汤为治伤风有汗之方。释者谓风伤营则有汗,又或谓营分虚损,即与外邪相感召。斯说也,愚尝疑之。 
人之营卫,皆为周身之外廓。卫譬则郭也,营譬则城也,有卫以为营之外围,外感之邪,何能越卫而伤营乎? 
盖人之胸中大气,息息与卫气相关,大气充满于胸中,则饶有吸力,将卫气吸紧,以密护于周身,捍御外感, 
使不得着体,即或着体,亦止中于卫,而不中于营,此理固显然也。有时胸中大气虚损,不能吸摄卫气,卫气 
散漫,不能捍御外邪,则外邪之来,直可透卫而入营矣。且愚临证实验以来,凡胸中大气虚损,或更下陷者, 
其人恒大汗淋漓,拙拟升陷汤下,载有数案,可参观也。是知凡桂枝汤证,皆因大气 
虚损,其汗先有外越之机,而外邪之来,又乘卫气之虚,直透营分,扰其营中津液,外泄而为汗也。究之,风 
寒原不相离,即系伤风,其中原挟有寒气,若但中于卫则亦能闭汗矣。故所用桂枝汤中,不但以祛风为务,而 
兼有散寒之功也。 
陈古愚曰∶“桂枝辛温,阳也。芍药苦平,阴也。桂枝又得生姜之辛,同气相求,可恃之调周身之阳气。 
芍药而得大枣、甘草之甘苦化合,可恃之以滋周身之阴液。既取大补阴阳之品,养其汗源,为胜邪之本,又啜 
粥以助之,取水谷之津以为汗,汗后毫不受伤,所谓立身于不败之地,以图万全也。”按∶此解甚超妙,而于啜 
粥之精义,犹欠发挥。如谓取水谷之津,以为汗,而人无伤损,他发汗药,何以皆不啜粥?盖桂枝汤所主之证, 
乃外感兼虚之证,所虚者何?胸中大气是也。《内经》曰∶“谷始入于胃,其精微者,先出于胃之两焦,以溉 
五脏,别出两行营卫之道,其大气之抟而不行者,积于胸中,命曰气海。”由斯观之,大气虽本于先天,实赖 
后天水谷之气培养而成。桂枝汤证,既因大气虚损,致卫气漫散,邪得越卫而侵营,故于服药之后,即啜热粥, 
能补助胸中大气以胜邪,兼能宣通姜、桂以逐邪,此诚战则必胜之良方也。乃后世医者忽不加察,虽用其方, 
多不啜粥,致令服后无效,病转深陷,故王清任《医林改错》深诋桂枝汤无用,非无用也,不啜粥故也。是以 
愚用此方时,加黄 升补大气,以代粥补益之力,防风宣通营卫,以代粥发表之力,服后啜粥固佳,即不啜粥, 
亦可奏效。而又恐黄 温补之性,服后易至生热,故又加知母,以预为之防也。 
按∶凡服桂枝汤原方,欲其出汗者,非啜粥不效。赵晴初曰∶族侄柏堂,二十一岁时,酒后寐中受风,遍 
身肌肤麻痹,搔之不知疼痒,饮食如常。时淮阴吴鞠通适寓伊芳家,投以桂枝汤,桂枝五钱、白芍四钱、甘草三 
钱、生姜三片、大枣两枚,水三 
杯,煎二杯,先服一杯,得汗止后服,不汗再服。并嘱弗夜膳,临睡腹觉饥,服药一杯,须臾啜热稀粥一碗, 
复被取汗。柏堂如其法,只一服,便由头面至足,遍身 得微汗,汗到处,一手搔之,辄知疼痒,次日病若 
失。观此医案,知欲用桂枝汤原方发汗者,必须啜粥,若不啜粥,即能发汗,恐亦无此功效。 
或问∶桂枝汤证,其原因既为大气虚损,宜其阳脉现微弱之象,何以其脉转阳浮而阴弱乎?答曰∶人之 
一身,皆气之所撑悬也。此气在下焦为元气,在中焦为中气,在上焦为大气,区域虽分,而实一气贯注。故一 
身之中,无论何处气虚,脉之三部,皆现弱象。今其关前之脉,因风而浮,转若不见其弱,而其关后之脉,仍 
然微弱,故曰阳浮而阴弱也。如谓阴弱为下焦阴虚,则其脉宜兼数象。而愚生平所遇此等证,其脉多迟缓,不 
及四至,其为气分虚损,而非阴分虚损可知。即所谓啬啬恶寒、淅淅恶风,翕翕发热,亦皆气分怯弱之形状也。 


<篇名>3.从龙汤
属性:治外感痰喘,服小青龙汤,病未全愈,或愈而复发者,继服此汤。 
龙骨(一两,不用 捣) 牡蛎(一两,不用 捣) 生杭芍(五钱) 清半夏(四钱) 苏子(四钱,炒捣) 
牛蒡子(三钱,炒捣) 
热者,酌加生石膏数钱或至一两。 
从来愚治外感痰喘,遵《伤寒论》小青龙汤加减法,去麻黄加杏仁,热者更加生石膏,莫不随手而愈。 
然间有愈而复发,再服原方不效者,自拟得此汤后,凡遇此等证,服小青龙汤一两剂即愈者,继服从龙汤一剂, 
必不再发。未全愈者,服从龙汤一剂或两剂,必然全愈。名曰从龙汤者,为其最宜用于小青龙汤后 
也。 
或疑,方中重用龙骨、牡蛎,收涩太过,以治外感之证,虽当发表之余,仍恐余邪未尽,被此收涩之药固 
闭于中,纵一时强制不喘,恐病根益深,异日更有意外之变。答曰∶若是以品龙骨、牡蛎,浅之乎视龙骨、牡 
蛎者也,斯可征之以前哲之说。 
徐灵胎曰∶龙骨最粘涩,能收敛正气,凡心神耗散,肠胃滑脱之疾,皆能已之。此药但敛正气,而不敛邪 
气。所以仲景于伤寒邪气未尽者,亦恒与牡蛎同用,若仲景之柴胡加龙骨牡蛎汤,桂枝、甘草、龙骨、牡蛎汤 
诸方是也。愚于伤寒、温病,热实脉虚,心中怔忡,精神骚扰者,恒龙骨与萸肉、生石膏并用,即可随手奏效。 
门人高××曾治一外感痰喘,其喘剧脉虚,医皆诿为不治。高××投以小青龙汤,去麻黄,加杏仁,又加生 
石膏一两、野台参五钱,一剂而喘定。恐其反复,又继投以从龙汤,亦加人参与生石膏,其病霍然顿愈。 
又∶子××治曲姓叟,年六十余,外感痰喘,十余日不能卧。医者投以小清龙汤两剂,病益加剧(脉有热而不 
敢多加生石膏者其病必加剧)。××视之,其 
脉搏一息六至,上焦烦躁,舌上白苔满布,每日大便两三次,然非滑泻。审证论脉,似难挽回。而××仍投以小 
青龙汤,去麻黄,加杏仁,又加野台参三钱,生龙骨、生牡蛎各五钱,生石膏一两半。一剂病愈强半,又服一 
剂全愈。 
按∶前案但加补气之药于小青龙汤中,后案并加敛气之药于小青龙汤中,似近于少年卤莽,而皆能挽回至 
险之证,亦可为用小青龙汤者多一变通之法矣。 
邑,郑××,年五十许。感冒风寒,痰喘甚剧,服表散、清火、理痰之药皆不效,留连二十余日,渐近垂危。 
其甥刘××,从愚读书,与言医学,颇能记忆。闻其舅病革,往省之,既至, 
则衣冠竟属纩矣。刘××用葶苈(四钱生者布包)大枣(五枚擘开)汤,加五味子二钱,煎汤灌之,豁然顿醒, 
继服从龙汤一剂全愈。盖此证乃顽痰郁塞肺之窍络,非葶苈大枣汤,不能泻之。且喘久则元气必虚,加五味子 
二钱,以收敛元气,并可借葶苈下行之力,以纳气归肾也。可知拙拟从龙汤,固宜于小青龙汤后,而服过发表 
之药者,临时制宜,皆可酌而用之,不必尽在小青龙汤后也。 


<篇名>4.馏水石膏饮
属性:治胸中先有蕴热,又受外感,胸中烦闷异常,喘息迫促,其脉浮洪有力,按之未实,舌苔白而未黄者。 
生石膏(二两,轧细) 甘草(三钱) 麻黄(二钱) 
上药三味,用蒸汽水煎两三沸,取清汤一大碗,分六次温服下。前三次,一点钟服一次,后三次,一点半 
钟服一次。病愈则停服,不必尽剂。下焦觉凉者,亦宜停服。僻处若无汽水,可用甘澜水代之。 
作甘澜水法∶用大盆盛水,以杓扬之,扬久水面起有若干水泡,旁有人执杓逐取之,即甘澜水。 
若以治温病中,似此证者,不宜用麻黄。宜用西药阿斯匹林一瓦,融化于汤中以代之。若僻处药局无阿斯 
匹林,又可代以薄荷叶二钱。 
奉天钱姓妇于仲冬得伤寒证,四五日间,喘不能卧,胸中烦闷异常,频频呼唤,欲自开其胸。诊其脉浮洪 
而长,重按未实,舌苔白浓。知其证虽入阳明,而太阳犹未罢也(胸中属太阳)。此时欲以小青龙汤治喘,则 
失于热。欲以白虎汤治其烦热,又遗却太阳之病,而喘不能愈。踌躇再三,为拟此方,取汽水轻浮之力,能引 
石膏上升,以解胸中之烦热。甘草甘缓之性,能逗留石膏不使下趋,以专其上行之力。又少佐以麻黄解散太阳 
之余邪,兼借以 
泻肺定喘,而胸中满闷可除也。汤成后,俾徐徐分六次服之。因病在上焦,若顿服,恐药力下趋,则药过病所, 
而病转不愈也。服至三次,胸间微汗,病顿见愈,服至尽剂,病愈十之八九。再诊其脉,关前犹似浮洪,喘息 
已平,而从前兼有咳嗽未愈,继用玄参一两,杏仁(去皮)二钱,蒌仁、牛蒡子各三钱,两剂全愈。 


<篇名>5.通变大柴胡汤
属性:治伤寒温病,表证未罢,大便已实者。 
柴胡(三钱) 薄荷(三钱) 知母(四钱) 大黄(四钱) 
此方若治伤寒,以防风易薄荷。 
《伤寒论》大柴胡汤,治少阳经与阳明府同病之方也。故方中用柴胡以解在经之邪,大黄以下阳明在府之 
热,方中以此二药为主,其余诸药,可加可减,不过参赞以成功也。然其方宜于伤寒,而以治温病、与表证不 
在少阳者,又必稍为通变,而后所投皆宜也。 
或问∶其表果系少阳证,固宜用柴胡矣。若非少阳证,既加薄荷、防风以散表邪,何须再用柴胡乎?答曰∶ 
凡表证未罢,遽用降药下之,恒出两种病证∶一为表邪乘虚入里,《伤寒论》所载,下后胸满心下痞硬,下后 
结胸者是也;一为表邪乘虚入里且下陷,《伤寒论》所谓,下之利不止者是也。此方中用防风、薄荷以散之, 
所以防邪之内陷,用柴胡以升之,所以防邪之下陷也。 
一人,年二十余。伤寒六七日,头疼恶寒,心中发热,咳吐粘涎。至暮尤寒热交作,兼眩晕,心中之热亦 
甚。其脉浮弦,重按有力,大便五日未行。投以此汤,加生石膏六钱、芒硝四钱,下大便二次。上半身微见汗, 
诸病皆见轻。惟心中犹觉发热,脉象不若从前之浮弦,而重按仍有力。拟投以白虎加人参汤,恐当 
下后,易作滑泻,遂以生山药代粳米,连服两剂全愈。 


<篇名>6.加味越婢加半夏汤
属性:治素患劳嗽,因外感袭肺,而劳嗽益甚,或兼喘逆,痰涎壅滞者。 
麻黄(二钱) 石膏(三钱, 捣) 生山药(五钱) 寸麦冬(四钱,带心) 清半夏(三钱) 牛蒡子(三钱, 
炒捣) 
玄参(三钱) 甘草(一钱五分) 大枣(三枚,擘开) 生姜(三片) 
《伤寒论》有桂枝二越婢一汤,治太阳病发热恶寒,热多寒少。《金匮》有越婢汤,治受风水肿。有越 
婢加半夏汤,治外感袭肺,致肺中痰火壅滞,胀而作喘。今因其人素患劳嗽,外感之邪与肺中蕴蓄之痰,互相 
胶漆,壅滞肺窍而劳嗽益甚。故用越婢加半夏汤,以祛外袭之邪,而复加山药、玄参、麦冬,牛蒡子,以治其 
劳嗽。此内伤外感兼治之方也。 
一叟,年近七旬。素有劳嗽,初冬宿病发动,又兼受外感,痰涎壅滞胸间,几不能息。剧时昏不知人,身 
躯后挺。诊其脉,浮数无力。为制此汤,一剂气息通顺,将麻黄、石膏减半,又服数剂而愈。 
或问∶子尝谓石膏宜生用,不宜 用。以石膏寒凉之中,原兼辛散, 之则辛散之力,变为收敛,服之转 
可增病。乃他方中,石膏皆用生者,而此独用 者何也?答曰∶此方所主之病,外感甚轻,原无大热。方中用 
麻黄以祛肺邪,嫌其性热,故少加石膏佐之。且更取 者,收敛之力,能将肺中痰涎凝结成块,易于吐出。此 
理从用 石膏点豆腐者悟出,试之果甚效验。后遇此等证,无论痰涎如何壅盛、如何杜塞,投以此汤,须臾, 
药力行后,莫不将痰涎结成小块,连连吐出,此皆 石膏与麻黄并用之 
效也。若以治寒温大热,则断不可 。若更多用,则更不可 也( 石膏用于此方,且止三钱,自无妨碍。然愚后 
来志愿,欲全国药局,皆不备 石膏,后有用此方者,若改用生石膏四钱更佳)。 

\chapter{治温病方}
<篇名>1.清解汤
属性:治温病初得,头疼,周身骨节酸疼,肌肤壮热,背微恶寒无汗,脉浮滑者。 
薄荷叶(四钱) 蝉蜕(三钱,去足土) 生石膏(六钱,捣细) 甘草(一钱五分) 
《伤寒论》曰∶“太阳病,发热而渴,不恶寒者,为温病。若发汗已,身灼热者,名曰风温。风温为病, 
脉阴阳俱浮,自汗出,身重,多眠睡,息必鼾,言语难出。”此仲景论温病之提纲也。乃提纲详矣,而后未 
明言治温病之方。及反复详细观之,乃知《伤寒论》中,原有治温病方,且亦明言治温病方,特涉猎观之不知 
耳。六十一节云∶“发汗后,不可更行桂枝汤。汗出而喘,无大热者,可与麻黄、杏仁、甘草、石膏汤主之。”夫 
此证既汗后不解,必是用辛热之药,发不恶寒证之汗,即温病提纲中,所谓若发汗已也(提纲中所谓若发汗, 
是用辛热之药强发温病之汗)。其汗出而喘,无大热者,即温病提纲中,所谓若发汗已,身灼热及后所谓自汗 
出、多眠睡、息必鼾也。睡而息鼾,醒则喘矣。此证既用辛热之药,误发于前,仲景恐医者见其自汗,再误认 
为桂枝汤证,故特戒之曰∶不可更行桂枝汤,而宜治以麻杏甘石汤。此节与温病提纲遥遥相应,合读之则了如 
指掌。然麻杏甘石汤,诚为治温病初得之的方矣。而愚于发表药中不用麻黄,而用薄荷、蝉蜕者,曾于葛根黄 
芩黄连汤解后详论之,兹不再赘。 
今者论温病之书甚伙,而郑卫红紫,适足乱真。愚本《内 
经》、仲景,间附以管见,知温病大纲,当分为三端。今逐端详论,胪列于下,庶分途施治,不至错误。 
一为春温。其证因冬月薄受外感,不至即病。所受之邪,伏于膜原之间,阻塞脉络,不能宣通,暗生内热。 
迨至春日阳生,内蕴之热,原有萌动之机,而复薄受外感,与之相触,则陡然而发,表里俱热,《内经》所谓 
“冬伤于寒,春必病温”者是也, 
宜治以拙拟凉解汤。热甚者,治以拙拟寒解汤。有汗者,宜仲景葛根黄连黄芩汤,或拙拟和解汤,加生石膏。 
若至发于暑月,又名为暑温,其热尤甚。初得即有脉洪长,渴嗜凉水者,宜投以大剂白虎汤,或拙拟仙露汤。 
一为风温。犹是外感之风寒也,其时令已温,外感之气已转而为温,故不名曰伤寒、伤风,而名风温,即 
《伤寒论》中所谓风温之为病者是也。然其证有得之春初者,有得之春暮者,有得之夏秋者,当随时序之寒热, 
参以脉象,而分别治之。若当春初秋末,时令在寒温之间,初得时虽不恶寒,脉但浮而无热象者,宜用拙拟清 
解汤,加麻黄一二钱,或用仲景大青龙汤。若当暑热之日,其脉象浮而且洪者,用拙拟凉解汤,或寒解汤。若 
有汗者,用拙拟和解汤,或酌加生石膏。 
一为湿温。其证多得之溽暑。阴雨连旬,湿气随呼吸之气,传入上焦,窒塞胸中大气。因致营卫之气不相 
贯通,其肌表有似外感拘束,而非外感也。其舌苔白而滑腻,微带灰色。当用解肌利便之药,俾湿气由汗与小 
便而出,如拙拟宣解汤是也。仲景之猪苓汤,去阿胶,加连翘亦可用。至湿热蓄久,阳明府实,有治以白虎汤, 
加苍术者,其方亦佳。而愚则用白虎汤,以滑石易知母,又或不用粳米,而以生薏米代之。至于“冬不藏精, 
春必病温”,《内经》虽有明文,其证即寓于风温、春温之中。盖内虚之人,易受外感,而阴虚蕴热之人,尤 
易受温病。故无论风温、春 
温之兼阴虚者,当其发表、清解、降下之时,皆宜佐以滋阴之品,若生山药、生地黄、玄参、阿胶、生鸡子黄 
之类均可酌用,或宜兼用补气之品,若白虎汤之加人参,竹叶石膏汤之用人参,诚以人参与凉润之药并用,不 
但补气,实大能滋阴也。 
上所论温病,乃别其大纲及其初得治法。至其证之详悉,与治法之随证变通,皆备于后之方案中。至于疫 
病,乃天地之疠气,流行传染,与温病迥异。 
方中薄荷叶,宜用其嫩绿者。至其梗,宜用于理气药中,若以之发汗,则力减半矣。若其色不绿而苍,则 
其力尤减。若果嫩绿之叶,方中用三钱即可。 
薄荷气味近于冰片,最善透窍。其力内至脏腑筋骨,外至腠理皮毛,皆能透达,故能治温病中之筋骨作疼 
者。若谓其气质清轻,但能发皮肤之汗,则浅之乎视薄荷矣。 
蝉蜕去足者,去其前之两大足也。此足甚刚硬,有开破之力。若用之退目翳消疮疡,带此足更佳。若用之 
发汗,则宜去之,盖不欲其于发表中,寓开破之力也。 
蝉蜕性微凉、味淡,原非辛散之品,而能发汗者,因其以皮达皮也。此乃发汗中之妙药,有身弱不任发表 
者,用之最佳。且温病恒有兼瘾疹者,蝉蜕尤善托瘾疹外出也。 
石膏性微寒,《神农本草经》原有明文。虽系石药,实为平和之品。且其质甚重,六钱不过一大撮耳。其 
凉力,不过与知母三钱等。而其清火之力则倍之,因其凉而能散也。尝观后世治温之方,至阳明府实之时,始 
敢用石膏五六钱,岂能知石膏者哉!然必须生用方妥, 者用至一两,即足偾事。又此方所主之证,或兼背微 
恶寒,乃热郁于中,不能外达之征,非真恶寒也。白虎汤证中,亦恒有如此者,用石膏透达其热,则不恶寒矣。 
或问∶外感中于太阳则恶寒,中于阳明则不恶寒而发热。时 
至春、夏,气候温热,故外感之来,不与寒水相感召,而与燥金相感召,直从身前阳明经络袭入,而为温病。 
后世论温病者,多是此说。而《伤寒论》温病提纲,冠之以太阳病者何也?答曰∶温病初得,亦多在太阳,特 
其转阳明甚速耳。 
曾治一人,年二十余。当仲夏夜寝,因夜凉,盖单衾冻醒,发懒,仍如此睡去。须臾又冻醒,晨起微觉恶 
寒。至巳时已觉表里大热,兼喘促,脉洪长而浮。投以清解汤,方中生石膏,改用两半,又加牛蒡子(炒捣) 
三钱,服后得汗而愈。由斯观之,其初非中于太阳乎,然不专在太阳也。人之所以觉凉者,由于衣衾之薄。其 
气候究非寒凉,故其中于人不专在太阳,而兼在阳明。且当其时,人多蕴内热,是以转阳明甚速也,然此所论 
者风温耳。若至冬受春发,或夏发之温,恒有与太阳无涉者。故《伤寒论》温病提纲中,特别之曰∶风温之为 
病,明其异于“冬伤于寒,春必病温”之温病也。又杏仁与牛蒡子,皆能降肺定喘,而杏仁性温、牛蒡子性 
凉,伤寒喘证,皆用杏仁,而温病不宜用温药,故以牛蒡子代之。 
附录∶ 
直隶盐山孙××来函∶ 
一九二五年春,一人来津学木工。因身体单薄,又兼天热,得温病,请为诊视。脉浮数而滑,舌苔白浓, 
时时昏睡。为开清解汤,生石膏用一两,为其脉数,又加玄参五钱,一剂病愈。 
斯年仲春,俞××之三位女儿皆出瘟疹。生为诊视,皆投以清解汤,加连翘、生地、滑石而愈。同时之患此 
证者,势多危险。惟生投以此方,皆能随手奏效。 


<篇名>2.凉解汤
属性:治温病,表里俱觉发热,脉洪而兼浮者。 
薄荷叶(三钱) 蝉蜕(二钱,去足土) 生石膏(一两,捣细) 甘草(一钱五分) 
春温之证,多有一发而表里俱热者,至暑温尤甚,已详论之于前矣。而风温证,两三日间,亦多见有此 
脉、证者。此汤皆能治之,得汗即愈。 
西人治外感,习用阿斯匹林法。用阿斯匹林一瓦,和乳糖(可代以白蔗糖)服之,得汗即愈。愚屡次试之, 
其发汗之力甚猛,外感可汗解者,用之发汗可愈。若此凉解汤,与前清解汤,皆可以此药代之,以其凉而能散 
也。若后之寒解汤,即不可以此药代之,盖其发汗之力有余,而清热之力,仍有不足也。 


<篇名>3.寒解汤
属性:治周身壮热,心中热而且渴,舌上苔白欲黄,其脉洪滑。或头犹觉疼,周身犹有拘束之意者。 
生石膏(一两,捣细) 知母(八钱) 连翘(一钱五分) 蝉蜕(一钱五分,去足土) 
或问∶此汤为发表之剂,而重用石膏、知母,微用连翘、蝉蜕,何以能得汗?答曰,用此方者,特恐其诊 
脉不真,审证不确耳。果如方下所注脉证,服之复杯可汗,勿庸虑此方之不效也。盖脉洪滑而渴,阳明府热已 
实,原是白虎汤证。特因头或微疼,外表犹似拘束,是犹有一分太阳流连未去。故方中重用石膏、知母以清 
胃府之热;而复少用连翘、蝉蜕之善达表者,引胃中化而欲散之热,仍还太阳作汗而解。斯乃调剂阴阳,听其 
自汗,非强发其汗也。况石膏性凉(《神农本草经》谓其微寒即凉也)味微辛,有实热者,单服之即能汗乎? 
曾治一少年,孟夏长途劳役,得温病,医治半月不效。后愚诊视,其两目清白,竟无所见,两手循衣摸床, 
乱动不休, 语不省人事。其大便从前滑泻,此时虽不滑泻,每日仍溏便一两次。脉浮数,右寸之浮尤甚,两 
尺按之即无。因此证目清白无见者,肾阴将竭也。手循衣摸床者,肝风已动也。病势之危,已至 
极点。幸喜脉浮,为病还太阳。右寸浮尤甚,为将汗之势。其所以将汗而不汗者,人身之有汗,如天地之有雨。 
天地阴阳和而后雨,人身亦阴阳和而后汗。此证尺脉甚弱,阳升而阴不能应,汗何由作?当用大润之剂,峻补 
真阴,济阴以应其阳,必能自汗。遂用熟地、玄参、阿胶、枸杞之类,约重六七两,煎汤一大碗,徐徐温饮下, 
一日连进二剂,即日大汗而愈。审是,则发汗原无定法。当视其阴阳所虚之处,而调补之,或因其病机而利导 
之,皆能出汗,非必发汗之药始能汗也。按∶寒温之证,原忌用粘腻滋阴、甘寒清火,以其能留邪也。而用以 
为发汗之助,则转能逐邪外出,是药在人用耳。 
一人,年四十余。为风寒所束不得汗,胸中烦热,又兼喘促。医者治以苏子降气汤,兼散风清火之品,数 
剂病益进。诊其脉,洪滑而浮,投以寒解汤,须臾上半身即出汗。又须臾,觉药力下行,至下焦及腿亦皆出汗, 
病若失。 
一人,年三十许。得温证,延医治不效,迁延十余日。愚诊视之,脉虽洪而有力,仍兼浮象。问其头疼乎? 
曰∶然!渴欲饮凉水乎?曰∶有时亦饮凉水,然不至燥渴耳。知其为日虽多,而阳明之热,犹未甚实,太阳之 
表,犹未尽罢也。投以寒解汤,须臾汗出而愈。 
一人,年三十余。于冬令感冒风寒,周身恶寒无汗,胸间烦躁。原是大青龙汤证,医者投以麻黄汤。服后汗 
无分毫,而烦躁益甚,几至疯狂。诊其脉,洪滑异常,两寸皆浮,而右寸尤甚。投以寒解汤,复杯之顷,汗出 
如洗而愈。审是则寒解汤不但宜于温病,伤寒现此脉者,投之亦必效也。 
一叟,年七旬。素有劳疾,薄受外感,即发喘逆,投以小青龙汤,去麻黄,加杏仁、生石膏辄愈。上元节 
后,因外感甚重,旧病复发,五六日间,热入阳明之府。脉象弦长浮数,按之有力, 
而无洪滑之象(此外感兼内伤之脉)。投以寒解汤,加潞参三钱,一剂汗出而喘愈。再诊其脉,余热犹炽,继 
投以白虎加人参以山药代粳米汤一大剂,分三次温饮下,尽剂而愈。 
一妊妇,伤寒两、三日。脉洪滑异常,精神昏愦,间作 语,舌苔白而甚浓。为开寒解汤方,有一医者在 
座,问方中之意何居?愚曰∶欲汗解耳。曰∶此方能汗解乎?愚曰∶此方遇此证,服之自能出汗,若泛作汗解 
之药服之,不能汗也。饮下须臾,汗出而愈。 
一妇人,年二十余,得温病。咽喉作疼,舌强直,几不能言,心中热而且渴,频频饮水,脉竟沉细异常, 
肌肤亦不发热。遂舍脉从证,投以寒解汤,得微汗,病稍见愈。明晨又复如故,舌之强直更甚。知药原对证, 
而力微不能胜病也。遂仍投以寒解汤,将石膏加倍,煎汤两盅,分二次温饮下,又得微汗,病遂愈。 
按∶伤寒脉若沉细,多系阴证。温病脉若沉细,则多系阳证。盖温病多受于冬,至春而发,其病机自内 
向外。有时病机郁而不能外达,其脉或即现沉细之象,误认为凉,必至误事。又此证,寒解汤既对证见愈矣, 
而明晨,舌之强直更甚,乃将方中生石膏倍作二两,分两次前后服下,其病即愈。由是观之,凡治寒温之热者, 
皆宜煎一大剂,分数次服下,效古人一剂三服之法也。 
门人高××曾治一媪,年近七旬。于春初得伤寒证,三四日间,烦热异常。又兼白痢,昼夜滞下无度,其 
脉洪滑兼浮。高××投以寒解汤,加生杭芍三钱,一剂微汗而热解,痢亦遂愈。 
又∶吴又可曰∶“里证下后,脉浮而微数,身微热,神思或不爽。此邪热浮于肌表,里无壅滞也。虽无汗, 
宜白虎汤,邪可从汗而解。若下后,脉空虚而数,按之豁然如无者,宜白虎加人参汤,复杯则汗解。”按∶白 
虎汤与白虎加人参汤,皆非解表之药,而用之得当,虽在下后,犹可须臾得汗,况在未下之前乎。不但此也, 
即承气汤,亦可为汗解之药,亦视乎用之何如耳。 
又洪吉人曰∶“余尝治热病八、九日,用柴葛解之、芩连清之、硝黄下之,俱不得汗。昏愦扰乱,撮空摸床, 
危在顷刻。以大剂地黄汤(必系减去桂附者),重加人参、麦冬进之。不一时,通身大汗淋漓,恶证悉退,神 
思顿清。” 
按∶此条与愚用补阴之药发汗相似,所异者,又加人参以助其气分也。上所论者皆发汗之理,果能汇通参 
观,发汗之理,无余蕴矣。 
附录∶ 
直隶盐山李××来函∶ 
天津××,得温病,先服他医清解之药数剂无效。弟诊其脉象,沉浮皆有力,表里壮热无汗。投以寒解汤 
原方,遍身得汗而愈。山斯知方中重用生石膏、知母以清热,少加连翘、蝉蜕以引热透表外出,制方之妙远 
胜于银翘散、桑菊饮诸方矣。且由此知石膏生用诚为妙药。从治愈此证之后,凡遇寒温实热诸证,莫不遵书中 
方论,重用生石膏治之。其热实脉虚者,亦莫不遵书中方论,用白虎加人参汤,或用白虎加人参以生山药代粳 
米汤,皆能随手奏效。 
直隶盐山孙××来函∶ 
斯年初冬,适郭姓之女得伤寒证,三四日间阳明热势甚剧,面赤气粗,六脉洪数,时作谵语。为开寒解汤, 
因胸中觉闷,加栝蒌仁一两,一剂病愈。 


<篇名>4.石膏阿斯匹林汤
属性:治同前证。 
生石膏(二两,轧细) 阿斯匹林(一瓦) 
上药二味,先用白蔗糖冲水,送服阿斯匹林。再将石膏煎汤 
一大碗,待周身正出汗时,乘热将石膏汤饮下三分之二,以助阿斯匹林发表之力。迨至汗出之后,过两三点钟, 
犹觉有余热者,可仍将所余石膏汤温饮下。若药服完,热犹未尽者,可但用生石膏煎汤,或少加粳米煎汤,徐 
徐温饮之,以热全退净为度,不用再服阿斯匹林也。 
又∶此汤不但可以代寒解汤,并可以代凉解汤。若以代凉解汤时,石膏宜减半。 
附录∶ 
江苏平台王××来函∶ 
小儿××,秋夏之交,陡起大热,失常神呆,闭目不食。家慈见而骇甚。吾因胸有成竹定见,遂曰∶“此 
无忧。”即用书中石膏阿斯匹林汤,照原方服法,服后即神清热退。第二日午际又热,遂放胆再用原方,因其 
痰多而咳,为加清半夏、牛蒡子,服之全愈。 


<篇名>5.和解汤
属性:治温病表里俱热,时有汗出,舌苔白,脉浮滑者。 
连翘(五钱) 蝉蜕(二钱,去足土) 生石膏(六钱,捣细) 生杭芍(五钱) 甘草(一钱) 
若脉浮滑,而兼有洪象者,生石膏当用一两。 


<篇名>6.宣解汤
属性:治感冒久在太阳,致热蓄膀胱,小便赤涩。或因小便秘,而大便滑泻。兼治湿温初得,憎寒壮热,舌苔 
灰色滑腻者。 
滑石(一两) 甘草(二钱) 连翘(三钱) 蝉蜕(三钱,去足土) 生杭芍(四钱) 
若滑泻者,甘草须加倍。 
一叟,年六十五,得风温证。六七日间,周身悉肿,肾囊肿大似西瓜,屡次服药无效。旬日之外,求为诊 
视。脉洪滑微浮,心中热渴,小便涩热,痰涎上泛,微兼喘息,舌苔白浓。投以此汤,加生石膏一两,周身微 
汗,小便通利,肿消其半,犹觉热渴。遂将方中生石膏加倍,服后又得微汗,肿遂尽消,诸病皆愈。按∶此乃 
风温之热,由太阳经入于膀胱之府,阻塞水道,而阳明胃府亦将实也。由是观之,彼谓温病入手经、不入足经 
者,何其谬哉! 


<篇名>7.滋阴宣解汤
属性:治温病,太阳未解,渐入阳明。其人胃阴素亏,阳明府证未实,已燥渴多饮,饮水过多,不能运化,遂成 
滑泻,而燥渴益甚。或喘,或自汗,或小便秘。温疹中多有类此证者,尤属危险之候,用此汤亦宜。 
其方即宣解汤加生山药一两,甘草改用三钱。 
此乃胃府与膀胱同热,又兼虚热之证也。滑石性近石膏,能清胃府之热,淡渗利窍,能清膀胱之热,同甘 
草生天一之水,又能清阴虚之热,一药而三善备,故以之为君。而重用山药之大滋真阴,大固元气者,以为之 
佐使。且山药生用,则汁浆稠粘,同甘草之甘缓者,能逗留滑石于胃中,使之由胃输脾,由脾达肺,水精四布。 
循三焦而下通膀胱,则烦热除,小便利,而滑泻止矣。又兼用连翘、蝉蜕之善达表者,以解未罢之太阳,使膀 
胱蓄热,不为外感所束,则热更易于消散。且蝉之性,饮而不食,有小便无大便,故其蜕,又能利小便,而止 
大便也。愚自临证以来,遇此等证,不知凡几。医者率多束手,而投以此汤,无不愈者。若用于温疹兼此证者, 
尤为妥善,以连翘、蝉蜕,实又表散 
温疹之妙药也。 
一媪,年近七旬,素患漫肿。为调治月余,肿虽就愈,而身体未复。忽于季春得温病,上焦烦热,病家自 
剖鲜地骨皮,煮汁饮之稍愈,又饮数次,遂滑泻不止,而烦热益甚。其脉浮滑而数,重诊无力。病家因病者年 
高,又素有疾病,加以上焦烦热,下焦滑泻,惴惴惟恐不愈,而愚毅然以为可治。投以滋阴宣解汤,一剂泻止, 
烦热亦觉轻。继用拙拟白虎加人参以山药代粳米汤,煎汁一大碗,一次只温饮一大口,防其再滑泻也。尽剂而愈。 
一室女,感冒风热,遍身瘾疹,烦渴滑泻,又兼喘促。其脉浮数无力。愚踌躇再四,亦投以滋阴宣解汤, 
两剂诸病皆愈。 
按∶服滋阴宣解汤,皆不能出大汗,且不宜出大汗,为其阴分虚也。间有不出汗者,病亦可愈。 


<篇名>8.滋阴清燥汤
属性:治同前证。外表已解,其人或不滑泻,或兼喘息,或兼咳嗽,频吐痰涎,确有外感实热,而脉象甚虚数者。 
若前证,服滋阴宣解汤后,犹有余热者,亦可继服此汤。 
其方即滋阴宣解汤,去连翘、蝉蜕。 
一妇人,受妊五月,偶得伤寒。三四日间,胎忽滑下。上焦燥渴,喘而且呻,痰涎壅盛,频频咳吐。延医 
服药,病未去,而转添滑泻,昼夜十余次。医者辞不治,且谓危在旦夕。其家人惶恐,迎愚诊视。其脉似洪滑, 
重诊指下豁然,两尺尤甚。本拟治以滋阴清燥汤,为小产才四五日,不敢遽用寒凉。遂先用生山药二两、酸石 
榴一个,连皮捣烂,同煎汁一大碗,分三次温饮下。滑泻见愈,他病如故。再诊其脉,洪滑之力较实,因思此 
证虽虚,确有外感实热,若不先解其实热,他病何以得愈?时届晚三点钟, 
病患自言,每日此时潮热,又言精神困倦已极,昼夜苦不得睡。遂于斯日,复投以滋阴清燥汤。方中生山药重 
用两半,煎汁一大碗,徐徐温饮下,一次只饮药一口,诚以产后,脉象又虚,不欲寒凉侵下焦也。斯夜遂得安 
睡,渴与滑泻皆愈,喘与咳亦愈其半。又将山药、滑石各减五钱,加龙骨、牡蛎各八钱,一剂而愈。 
一室女,伤寒过两旬矣,而瘦弱支离,精神昏愦,过午发热,咳而且喘,医者辞不治。诊其脉,数至七至, 
微弱欲无。因思此证,若系久病至此,不可为矣。然究系暴虚之证,生机之根柢当无损。勉强投以滋阴清燥汤, 
将滑石减半,又加玄参、熟地黄各一两,野台参五钱,煎汤一大碗,徐徐温饮下。饮完煎滓重饮,俾药力昼夜 
相继。两日之间,连服三剂,滑石渐减至二钱,其病竟愈。 
按∶此证始终不去滑石者,恐当伤寒之余,仍有余邪未净。又恐补药留邪,故用滑石引之下行,使有出路 
也。又凡煎药若大剂,必需多煎汤数杯,徐徐服之。救险证宜如此,而救险证之阴分亏损者,尤宜如此也。 
汲××之母,年近七旬。身体羸弱,谷食不能消化,惟饮牛乳,或间饮米汤少许,已二年卧床,不能起坐矣。 
于戊午季秋,受温病。时愚初至奉天,自锦州邀愚诊视。脉甚细数,按之微觉有力。发热咳嗽,吐痰稠粘,精 
神昏愦,气息奄奄。投以滋阴清燥汤,减滑石之半,加玄参五钱,一剂病愈强半。又煎渣取清汤一茶盅,调入 
生鸡子黄一枚,服之全愈。 
奉天一孺子年四岁,得温病,邪犹在表,医者不知为之清解,遽投以苦寒之剂,服后滑泻,四五日不止。 
上焦燥热,闭目而喘,精神昏愦。延为延医,病虽危险,其脉尚有根柢,知可挽回。俾用滋阴清燥汤原方,煎 
汁一大茶杯,为其幼小,俾徐徐温饮下,尽剂而愈。然下久亡阴,余有虚热,继用生山药、玄参各 
一两以清之,两剂热尽除。大抵医者遇此等证,清其燥热,则滑泻愈甚,补其滑泻,其燥热亦必愈甚。惟此方, 
用山药以止滑泻,而山药实能滋阴退热,滑石以清燥热,而滑石实能利水止泻,二药之功用,相得益彰。又佐 
以芍药之滋阴血、利小便,甘草之燮阴阳、和中宫,亦为清热止泻之要品。汇集成方,所以效验异常。愚用此 
方,救人多矣,即势至垂危,投之亦能奏效。 
奉天刘××,年二十五六,于季冬得伤寒,经医者误治,大便滑泻无度,而上焦烦热,精神昏愦,时作谵语, 
脉象洪数,重按无力。遂重用生山药两半、滑石一两、生杭芍六钱、甘草三钱,一剂泻止。上焦烦热不退,仍 
作谵语,爰用玄参、沙参诸凉润之药清之,仍复滑泻,再投以前方一剂泻又止,而上焦之烦热益甚,精神亦益 
昏愦,毫无知觉。此时其家人毕至,皆以为不可复治。诊其脉虽不实,仍有根柢,至数虽数,不过六至,知犹 
可治,遂慨切谓其家人曰∶“果信服余药,此病尚可为也”,其家人似领悟。为疏方,用大剂白虎加人参汤, 
更以生山药一两代粳米,大生地一两代知母,煎汤一大碗,嘱其药须热饮,一次止饮一口,限以六句钟内服完, 
尽剂而愈。 
津市钱姓小儿四岁,灼热滑泻,重用滋阴清燥汤治愈。 
附录∶ 
奉天铁岭杨××来函∶ 
治李姓妇人膨胀证。先经他医用苍术、槟榔、浓朴、枳实、香附、紫蔻之类辛燥开破,初服觉轻,七八剂 
后病转增剧,烦渴泄泻。又更他医,投以紫朴琥珀丸,烦渴益甚,一日夜泄泻十五六次,再诊时,医者辞不治。 
又延医数人,皆诿为不治。后乃一息奄奄,舁至床上两次,待时而已。其姻家有知生者强生往视。其脉如水上 
浮麻,不分至数,按之即无,惟两尺犹似有根,言语不 
真,仿佛可辨,自言心中大渴,少饮水即疼不可忍,盖不食者已三日矣。先投以滋阴清燥汤,为脉象虚甚,且 
气息有将脱之意,又加野台参、净萸肉,一剂,诸病皆愈,可以进食。遂俾用一味薯蓣粥,送服生鸡内金细末 
及西药百布圣,取其既可作药,又可作饭也。又即前方加减,日服一剂,旬日全愈。 


<篇名>9.滋阴固下汤
属性:治前证服药后,外感之火已消,而渴与泻仍未全愈。或因服开破之药伤其气分,致滑泻不止。其人或兼 
喘逆,或兼咳嗽,或自汗,或心中怔忡者,皆宜急服此汤。 
生山药(两半) 怀熟地(两半) 野台参(八钱) 滑石(五钱) 生杭芍(五钱) 甘草(二钱) 
酸石榴(一个,连皮捣烂) 
上药七味,用水五盅,先煎酸石榴十余沸,去滓再入诸药,煎汤两盅,分二次温饮下。若无酸石榴,可 
用牡蛎( 研)一两代之。汗多者,加山萸肉(去净核)六钱。 
寒温诸证,最忌误用破气之药。若心下或胸胁疼痛,加乳香、没药、楝子、丹参诸药,腹疼者加芍药,皆 
可止疼。若因表不解,束其郁热作疼者,解表清热,其疼自止。若误服槟榔、青皮、郁金、枳壳诸破气之品, 
损其胸中大气,则风寒乘虚内陷,变成结胸者多矣。即使传经已深,而肠胃未至大实,可降下者,则开破与寒 
凉并用,亦易使大便滑泻,致变证百出。愚屡见此等医者误人,心甚恻怛。故与服破气药而结胸者,制荡胸汤 
以救其误。服破气药而滑泻者,制此汤以救其误。究之,误之轻者可救,误之重者实难挽回于垂危之际也。 


<篇名>10.犹龙汤
属性:治胸中素蕴实热,又受外感。内热为外感所束,不能发泄。 
时觉烦躁,或喘、或胸胁疼,其脉洪滑而长者。 
连翘(一两) 生石膏(六钱,捣细) 蝉蜕(二钱,去足土) 牛蒡子(二钱,炒捣) 
喘者,倍牛蒡子。胸中疼者加丹参、没药各三钱。胁下疼者,加柴胡、川楝子各三钱。 
此方所主之证,即《伤寒论》大青龙汤所主之证也。然大青龙汤宜于伤寒,此则宜于温病。至伤寒之病, 
其胸中烦躁过甚者,亦可用之以代大青龙,故曰犹龙也。 
一妇,年三十余。胸疼连胁,心中发热。服开胸、理气、清火之药不效。后愚诊视,其脉浮洪而长。知其 
上焦先有郁热,又为风寒所束,则风寒与郁热相搏而作疼也。治以此汤,加没药、川楝子各四钱,一剂得汗而愈。 
一叟,年过七旬。素有劳病。因冬令伤寒,劳病复发,喘而且咳,两三日间,痰涎壅盛,上焦烦热。诊其 
脉,洪长浮数。投以此汤,加玄参、潞参各四钱,一剂汗出而愈。 
门人刘××,曾治一人,年四十。外感痰喘甚剧。四五日间,脉象洪滑,舌苔白而微黄。刘××投以此汤, 
方中石膏用一两,连翘用三钱。一剂周身得汗,外感之热已退,而喘未全愈。再诊其脉,平和如常,微嫌无力。 
遂用拙拟从龙汤,去苏子,加潞参三钱,一剂全愈。愚闻之喜曰∶外感痰喘,小青龙汤所主之证也。拙拟犹龙 
汤,原以代大青龙汤,今并可代小青龙汤,此愚之不及料也。将方中药味轻重,略为加减,即能另建奇功,以 
斯知方之运用在人,慧心者自能变通也。 
按∶连翘原非发汗之药,即诸家本草,亦未有谓其能发汗者。惟其人蕴有内热,用至一两必然出汗,且其 
发汗之力缓而长。为其力之缓也,不至为汪洋之大汗,为其力之长也,晚睡时服之,可使通夜微觉解肌。且能 
舒肝气之郁,泻肺气之实,若但目为疮家要药,犹未识连翘者也。用连翘发汗,必色青者方有 
力。盖此物嫩则青,老则黄。凡物之嫩者,多具生发之气,故凡发汗所用之连翘,必须青连翘。 

\chapter{治伤寒温病同用方}
<篇名>1.仙露汤
属性:治寒温阳明证,表里俱热,心中热,嗜凉水,而不至燥渴,脉象洪滑,而不至甚实,舌苔白浓,或白而微 
黄,或有时背微恶寒者。 
生石膏(三两,捣细) 玄参(一两) 连翘(三钱) 粳米(五钱) 
上四味,用水五盅,煎至米熟,其汤即成。约可得清汁三盅,先温服一盅。若服完一剂,病犹在者,可仍 
煎一剂,服之如前。使药力昼夜相继,以病愈为度。然每次临服药,必详细问询病患,若腹中微觉凉,或欲大 
便者,即停药勿服。候两三点钟,若仍发热未大便者,可少少与服之。若已大便,即非溏泻而热犹在者,亦可 
少少与服。 
《伤寒论》白虎汤,为阳明府病之药,而兼治阳明经病。此汤为阳明经病之药,而兼治阳明府病。为其 
所主者,责重于经,故于白虎汤方中,以玄参之甘寒(《神农本草经》言苦寒,细嚼之实甘而微苦,古今药或 
有不同),易知母之苦寒,又去甘草,少加连翘。欲其轻清之性,善走经络,以解阳明在经之热也。方中粳米, 
不可误用糯米(俗名浆米)。粳米清和甘缓,能逗留金石之药于胃中,使之由胃输脾,由脾达肺,药力四布, 
经络贯通。糯米质粘性热,大能固闭药力,留中不散,若错用之,即能误事。 
一叟年七十有一,因感冒风寒,头疼异常,彻夜不寝。其脉 
洪大有力,表里俱发热,喜食凉物,大便三日未行,舌有白苔甚浓。知系伤寒之热,已入阳明之府。因头疼甚 
剧,且舌苔犹白,疑犹可汗解。治以拙拟寒解汤,加薄荷叶一钱。头疼如故,亦未出汗,脉益洪实。恍悟曰∶ 
此非外感表证之头疼,乃阳明经府之热,相并上逆,而冲头部也。为制此汤,分三次温饮下,头疼愈强半,夜 
间能安睡,大便亦通。复诊之,脉象余火犹炽,遂用仲景竹叶石膏汤,生石膏仍用三两,煎汁一大碗,分三次 
温饮下,尽剂而愈。 
按∶竹叶石膏汤,原寒温大热退后,涤余热、复真阴之方。故其方不列于六经,而附载于六经之后。其所 
以能退余热者,不恃能用石膏,而恃石膏与参并用。盖寒温余热,在大热铄涸之余,其中必兼有虚热。石膏得 
人参,能使寒温后之真阴顿复,而余热自消,此仲景制方之妙也。又麦冬甘寒粘滞,虽能为滋阴之佐使,实能 
留邪不散,致成劳嗽。而惟与石膏、半夏并用,则无忌,诚以石膏能散邪,半夏能化滞也。或疑炙甘草汤(亦 
名复脉汤)中亦有麦冬,却无石膏、半夏。然有桂枝、生姜之辛温宣通者,以驾驭之,故亦不至留邪。彼惟知 
以甘寒退寒温之余热者,安能援以为口实哉! 
温病中,有当日得之,即宜服仙露汤者。一童子,年十六。暑日力田于烈日之中,午饭后,陡觉发热,无 
汗,烦渴引饮。诊其脉,洪而长,知其暑而兼温也。投以此汤,未尽剂而愈。 
按∶此证初得,而胃府之热已实。彼谓温病入手经,不入足经者,何梦梦也! 
上焦烦热太甚者,原非轻剂所能疗。而投以重剂,又恐药过病所,而病转不愈。惟用重剂,徐徐饮下,乃 
为合法。曾治一人,年四十余。素吸鸦片,于仲冬得伤寒,两三日间,烦躁无汗。原是大青龙汤证,因误服桂 
枝汤,烦躁益甚。迎愚诊视,其脉关前洪滑,两尺无力。为开仙露汤,因其尺弱,嘱其徐徐饮下,一 
次只饮药一口,防其寒凉侵下焦也。病家忽愚所嘱,竟顿饮之,遂致滑泻数次,多带冷沫。上焦益觉烦躁,鼻 
如烟熏,面如火炙。其关前脉,大于前一倍,又数至七至。知其已成戴阳之证,急用人参一两,煎好兑童便半 
茶蛊,将药碗置凉水盆中,候冷顿饮之。又急用玄参、生地、知母各一两,煎汤一大碗,候用。自服参后,屡 
诊其脉,过半点钟,脉象渐渐收敛,至数似又加数。遂急将候用之药炖热,徐徐饮下,一次饮药一口,阅两点 
钟尽剂,周身微汗而愈。此因病家不听所嘱,致有如此之失,幸而救愈,然亦险矣。审是,则凡药宜作数次服 
者,慎勿顿服也。盖愚自临证以来,无论内伤、外感,凡遇险证,皆煎一大剂,分多次服下。此以小心行其放 
胆,乃万全之策,非孤注之一掷也。 
《伤寒论》阳明篇中,白虎汤后,继以承气汤,以攻下肠中燥结,而又详载不可攻下诸证。诚以承气力 
猛,倘或审证不确,即足误事。愚治寒温三十余年,得一避难就易之法。凡遇阳明应下证,亦先投以大剂白虎 
汤一两剂。大便往往得通,病亦即愈。即间有服白虎汤数剂,大便犹不通者,而实火既消,津液自生,肠中不 
致干燥,大便自易降下。用玄明粉三钱,加蜂蜜或柿霜两许,开水冲调服下,大便即通。若仍有余火未尽,而 
大便不通者,单用生大黄末一钱(若凉水调服生大黄末一钱,可抵煮服者一两),蜜水调服,通其大便亦可。 
且通大便于服白虎汤后,更无下后不解之虞。盖下证略具,而脉近虚数者,遽以承气下之,原多有下后不解者, 
以其真阴亏、元气虚也。惟先服白虎汤或先服白虎加人参汤,去其实火,即以复其真阴,培其元气,而后微用 
降药通之,下后又何至不解乎。此亦愚百用不至一失之法也。 
间有用白虎汤润下大便,病仍不解,用大黄降之而后解者,以其肠中有匿藏之结粪也。曾治一媪,年七十 
余,季冬得伤寒证,七八日间,延愚诊视。其脉洪长有力,表里俱热,烦渴异常, 
大便自病后未行。投以白虎加人参汤二剂,大便遂通,一日降下三次,病稍见愈,而脉仍洪长。细审病情,当 
有结粪未下,遂单用大黄三钱,煮数沸服之,下结粪四五枚,病遂见愈,仍非脉净身凉,又用拙拟白虎加人参 
以山药代粳米汤,服未尽剂而愈。然此乃百中之一二也。临证者,不可因此生平仅遇之证,遂执为成法,轻视 
白虎,而重视承气也。 
重用石膏以退火之后,大便间有不通者,即可少用通利之药通之。此固愚常用之法,而随证制宜,又不可 
拘执成见。曾治一少年,伤寒已过旬日,阳明火实,大便燥结,投一大剂白虎汤,一日连进二剂,共享生石膏 
六两,至晚九点钟,火似见退,而精神恍惚,大便亦未通行,再诊其脉,变为弦象,夫弦主火衰,亦主气虚。 
知此证清解已过,而其大便仍不通者,因其元气亏损,不能营运白虎汤凉润之力也。遂单用人参五钱,煎汤俾 
服之,须臾大便即通,病亦遂愈。盖治此证的方,原是白虎加人参汤,因临证时审脉不确,但投以白虎汤,遂 
致病有更改。幸迷途未远,犹得急用人参,继所服白虎汤后以成功。诚以日间所服白虎汤,尽在腹中,得人参 
以助之,始能运化。是人参与白虎汤,前后分用之,亦无异于一时同用之也。益叹南阳制方之神妙,诚有令人 
不可思议者也。吴又可谓∶“如人方肉食而病适来,以致停积在胃,用承气下之,惟是臭水稀粪而已,于承气 
汤中,单加人参一味,虽三四十日停积之物于是方下。盖承气借人参之力鼓舞胃气,宿物始动也。”又可此论, 
亦即愚用人参于白虎汤后,以通大便之理也。 
附录∶ 
湖北天门县崔××来函∶ 
丁卯仲夏,何某,身染温病。他医以香薷饮、藿香正气散治 
之,不效。迎仆诊视,遵用清解汤,一剂而愈。时因温病盛行,以书中清解汤、凉解汤、寒解汤、仙露汤、从 
龙汤、馏水石膏饮,有呕者,兼用代赭石。本此数方,变通而用,救愈三千余人,共享生石膏一千余斤,并未 
偾事。 


<篇名>2.石膏粳米汤
属性:治温病初得,其脉浮而有力,身体壮热。并治一切感冒初得,身不恶寒而心中发热者。若其热已入阳明 
之腑,亦可用代白虎汤。 
生石膏(二两,轧细) 生粳米(二两半) 
上二味,用水三大碗,煎至米烂熟,约可得清汁两大腕。乘热尽量饮之,使周身皆汗出,病无不愈者。 
若阳明腑热已实,不必乘热顿饮之,徐徐温饮下,以消其热可也。 
或问∶外感初得,即中有蕴热,阳明胃腑,不至燥实,何至速用生石膏二两?答曰∶此方妙在将石膏同 
粳米煎汤,乘热饮之。俾石膏寒凉之性,随热汤发散之力,化为汗液尽达于外也。西人谓,胃本无化水之能, 
亦无出水之路。而壮实之人,饮水满胃,须臾水气旁达,胃中即空。盖胃中原多微丝血管,能引水气以入回血 
管,由回血管过肝入心,以营运于周身,由肺升出为气,由皮肤渗出为汗,余透肾至膀胱为溺。石膏煎汤,毫 
无气味,毫无汁浆,直与清水无异,且又乘热饮之,则敷布愈速,不待其寒性发作,即被胃中微丝血管吸去, 
化为汗、为气,而其余为溺,则表里之热,亦随之俱化。此寒因热用,不使伤胃之法也。且与粳米同煮,其冲 
和之气,能助胃气之发达,则发汗自易。其稠润之汁,又能逗留石膏,不使其由胃下趋,致寒凉有碍下焦。不 
但此也,清水煎开后,变凉甚速,以其中无汁浆,不能留热也。此方粳米多至二两半,汤成之后,必然汁浆甚 
稠。饮至 
胃中,又善留蓄热力,以为作汗之助也。是以人之欲发汗者,饮热茶不如啜热粥也。 
初拟此方时,惟用以治温病。实验既久,知伤寒两三日后,身不恶寒而发热者,用之亦效。丙辰正月上旬, 
愚自广平移居德州。自邯郸上火车,自南而北,复自北而南,一昼夜绕行千余里。车窗多破,风寒彻骨。至德 
州,同行病者五六人,皆身热无汗。遂用生石膏、粳米各十余两,饭甑煮烂熟,俾病者尽量饮其热汤,皆周身 
得汗而愈,一时称快。 
沈阳朱姓妇,年五旬。于戊午季秋,得温病甚剧。时愚初至奉天,求为延医。见其以冰囊作枕,复悬冰囊, 
贴面之上侧。盖从前求东人调治,如此治法,东人之所为也。合目昏昏似睡,大声呼之,毫无知觉。其脉洪大 
无伦,按之甚实。愚谓其夫曰∶此病阳明腑热,已至极点。外治以冰,热愈内陷。然此病尚可为,非重用生石 
膏不可。其夫韪愚言,遂用生石膏细末四两、粳米八钱,煎取清汁四茶杯,徐徐温灌下。约历十点钟,将药服 
尽,豁然顿醒。后又用知母、花粉、玄参、白芍诸药,少加连翘以清其余热,服两剂全愈。 
附录∶ 
江苏崇明县蔡××来函∶ 
季秋,敝处张氏之女得瘟病甚剧,服药无效,医言不治,病家以为无望。其母求人强仆往视,见其神昏如 
睡,高呼不觉;脉甚洪实。用先生所拟之石膏粳米汤,生石膏用三两,粳米用五钱。见者莫不惊讶诽笑。且有 
一老医扬言于人曰∶“蔡某年仅二十,看书不过年余,竟大胆若此!石膏重用三两,纵 透用之亦不可,况生 
者乎?此药下咽,人即死矣。”有人闻此言,急来相告,仆曰∶“此方若用 石膏,无须三两,即一两亦断送人命而 
有余。若用生者,即再多数两亦无碍,况仅三两乎。”遂急催病家购药,亲自监视,煎取清汤一大碗,徐徐温 
灌下。病患霍然顿醒。其家人惊喜异常,直以为死后重生矣。继而热疟流行,经仆重用生石膏治愈者不胜计。 


<篇名>3.镇逆白虎汤
属性:治伤寒、温病邪传胃腑,燥渴身热,白虎证俱。其人胃气上逆,心下满闷者。 
生石膏(三两,捣细) 知母(两半) 清半夏(八钱) 竹茹粉(六钱) 
用水五盅,煎汁三盅,先温服一盅。病已愈者,停后服。若未全愈者,过两点钟,再温服一盅。《伤寒 
论》白虎汤,治阳明腑热之圣药也。盖外邪炽盛,势若燎原,胃中津液,立就枯涸,故用石膏之辛寒以祛外感 
之邪,知母之凉润以滋内耗之阴。特是石膏质重(虽煎作汤性亦下坠),知母味苦,苦降与重坠相并,下行之 
力速,胃腑之热或难尽消。且恐其直趋下焦而为泄泻也,故又借粳米之浓汁、甘草之甘味,缓其下趋之势。以 
待胃中微丝血管徐徐吸去,由肺升出为气,由皮肤渗出为汗,余入膀胱为溺,而内蕴之热邪随之俱清,此仲景 
制方之妙也。然病有兼证,即用药难拘成方。犹是白虎汤证也,因其人胃气上逆,心下胀满,粳米、甘草不可 
复用,而以半夏、竹茹代之,取二药之降逆,以参赞石膏、知母成功也。 
一妇人,年三十余,得温证。始则呕吐,五六日间,心下满闷,热而且渴。脉洪滑有力,舌苔黄浓。闻 
其未病之先,曾有郁怒未伸,因得斯证,俗名夹恼伤寒。然时当春杪,一得即不恶寒,乃温病,非伤寒也。为 
疏此方,有一医者在座,疑而问曰∶此证因胃气上逆作胀满,始将白虎汤方,另为更定。何以方中不用开通气 
分之药,若承气汤之用浓朴、枳实,而惟用半夏、竹茹 
乎?答曰∶白虎汤用意,与承气迥异。盖承气汤,乃导邪下行之药,白虎汤乃托邪外出之药。故服白虎汤后, 
多有得汗而解者。间有服后未即得汗,而大热既消,其饮食之时,恒得微汗,余热亦由此尽解。若因气逆胀满, 
恣用破气之药,伤其气分,不能托邪外出,将邪陷愈深,胀满转不能消,或更增剧。试观《伤寒论》多有因误 
下伤其气分,成结胸,成心下痞硬证,不可不知也。再试观诸泻心,不轻用破气之品,却有半夏泻心汤。又仲 
景治“伤寒解后,气逆欲呕”有竹叶石膏汤,半夏与石膏并用;治“妇人乳、中虚、烦乱、呕逆”有竹皮大丸, 
竹茹与石膏并用,是半夏、竹茹善降逆气可知也。今师二方之意,用之以易白虎汤中之甘草、粳米,降逆气而 
不伤正气,服后仍可托邪外出,由汗而解,而胀满之证,亦即消解无余。此方愚用之屡矣,未有不随手奏效者。 
医者闻言省悟,听愚用药,服后,病患自觉胀满之处,如以手推排下行,病亦遂愈。 


<篇名>4.白虎加人参以山药代粳米汤
属性:治寒温实热已入阳明之府,燥渴嗜饮凉水,脉象细数者。 
生石膏(三两,捣细) 知母(一两) 人参(六钱) 生山药(六钱) 粉甘草(三钱) 
上五味,用水五盅,煎取清汁三盅,先温服一盅。病愈者,停后服。若未全愈者,过两点钟,再服一盅。 
至其服法详细处,与仙露汤同。 
伤寒法,白虎汤用于汗、吐、下后当加人参。究之脉虚者,即宜加之,不必在汗、吐、下后也。愚自临证 
以来,遇阳明热炽,而其人素有内伤,或元气素弱,其脉或虚数,或细微者,皆投以白虎加人参汤。实验既久, 
知以生山药代粳米,则其方愈稳妥,见效亦愈速。盖粳米不过调和胃气,而山药兼能固摄下焦元气,使 
元气素虚者,不至因服石膏、知母而作滑泻。且山药多含有蛋白之汁,最善滋阴。白虎汤得此,既祛实火,又 
清虚热,内伤外感,须臾同愈。愚用此方救人多矣。略列数案于下,以资参考。 
一叟,年近六旬。素羸弱,劳嗽,得伤寒证,三日,昏愦不知人。诊其脉甚虚数,而肌肤烙手,确有实热。 
知其脉虚证实,邪火横恣,元气又不能支持,故传经犹未深入,而即昏愦若斯也。踌躇再四,乃放胆投以此汤。 
将药煎成,乘热徐徐灌之,一次只灌下两茶匙。阅三点钟,灌药两盅,豁然顿醒。再尽其余,而病愈矣。 
一叟,年六旬。素亦羸弱多病,得伤寒证,绵延十余日。舌苔黄浓而干,心中热渴,时觉烦躁。其不烦躁 
之时,即昏昏似睡,呼之,眼微开,精神之衰惫可知。脉象细数,按之无力。投以凉润之剂,因其脉虚,又加 
野台参佐之。大便忽滑泻,日下数次。因思此证,略用清火之药,即滑泻者,必其下焦之气化不固。先用药固 
其下焦,再清其上焦、中焦未晚也。遂用熟地黄二两,酸石榴一个,连皮捣烂,同煎汤一大碗。分三次温饮下, 
大便遂固。间日投以此方,将山药改用一两,以生地黄代知母,煎汤成,徐徐温饮下,一次只饮药一大口。阅 
八点钟,始尽剂,病愈强半。翌日,又按原方,如法煎服,病又愈强半。第三日,又按其方服之,尽剂而愈。 
按∶熟地黄原非治寒温之药,而病至极危时,不妨用之,以救一时之急。故仲景治脉结代,有炙甘草汤,亦用 
干地黄(即今生地),结代亦险脉也。如无酸石榴时,可用龙骨( 捣)、牡蛎( 捣)各五钱代之。 
一叟,年六旬余。素吸鸦片,羸弱多病,于孟冬感冒风寒,其脉微弱而浮。愚用生黄 数钱,同表散之 
药治之,得汗而愈。间日,因有紧务事,冒寒出门,汗后重感,比前较剧。病卧旅邸,不能旋里。因延彼处医 
者延医,时身热饮水,病在阳明之 
府。医者因其脉微弱,转进温补,病益进。更延他医,以为上有浮热,下有实寒,用附子、吴茱萸,加黄连治 
之。服后,齿龈尽肿,且甚疼痛,时觉烦躁,频频饮水,不能解渴。不得已复来迎。愚至,诊其脉细而数,按 
之略实。遂投以此汤,加玄参六钱,以散其浮游之热。一剂牙疼即愈,烦躁与渴亦见轻。翌日,用原方去玄参, 
将药煎成,调入生鸡子黄三枚,作三次温饮下,大便得通而愈。 
一人,年二十,资禀素弱。偶觉气分不舒,医者用三棱、延胡等药破之。自觉短气,遂停药不敢服。隔两 
日,忽发喘逆,筋惕肉动,精神恍惚。脉数至六至,浮分摇摇,按之若无。肌肤甚热,上半身时出热汗,自言 
心为热迫,甚觉怔忡。其舌上微有白苔,中心似黄。统观此病情状,虽陡发于一日,其受外感,已非一日。盖 
其气分不舒时,即受外感之时,特其初不自觉耳。为其怔忡太甚,不暇取药,急用生鸡子黄四枚,温开水调和, 
再将其碗置开水盆中,候温服之,喘遂止,怔忡亦见愈。继投以此汤,煎汁一大碗,仍调入生鸡子黄三枚,徐 
徐温饮下。自晚十点钟至早七点钟,尽剂而病若失。因其从前服药伤气,俾用玄参一两、潞参五钱,连服数剂 
以善其后。 
一童子,年十七。于孟夏得温证,八九日间,呼吸迫促,频频咳吐,痰血相杂。其咳吐之时,疼连胸胁, 
上焦微嫌发闷。诊其脉,确有实热,而数至七至,摇摇无根。盖其资禀素弱,又兼读书劳心,其受外感又甚剧, 
故脉象若是之危险也。为其胸胁疼闷兼吐血,遂减方中人参之半,加竹茹、三七(捣细冲服)各二钱。用三七 
者,不但治吐血,实又兼治胸胁之疼也。一剂血即不吐,诸病亦见愈。又服一剂全愈。 
一农家孺子,年十一。因麦秋农家忙甚,虽幼童亦作劳田间,力薄不堪重劳,遂得温病。手足扰动,不能 
安卧,谵语不休,所言者皆劳力之事,昼夜目不能瞑。脉象虽实,却非洪滑。拟 
投以此汤,又虑小儿少阳之体,外邪方炽,不宜遽用人参,遂用生石膏两半、蝉蜕一钱,煎服后,诸病如故。 
复来询方,且言其苦于服药,昨所服者,呕吐将半。愚曰,单用生石膏二两,煎取清汁,徐徐温饮之,即可不 
吐。乃如言服之,病仍不愈。再为诊视,脉微热退,谵语益甚,精神昏昏,不省人事。急用野台参两半、生石 
膏二两,煎汁一大碗,分数次温饮下。身热脉起,目遂得瞑,手足稍安,仍作谵语。又于原渣加生石膏、麦冬 
各一两,煎汁二盅,分两次温饮下。降大便一次,其色甚黑,病遂愈。 
按∶此证若早用人参,何至病势几至莫救。幸即能省悟,犹能竭力挽回,然亦危而后安矣。愚愿世之用 
白虎汤者,宜常存一加人参之想也。又按∶此案与前案观之,凡用白虎汤而宜加人参者,不必其脉现虚弱之象 
也。凡谂知其人劳心过度,或劳力过度,或在老年,或有宿疾,或热已入阳明之府,脉象虽实,而无洪滑之象, 
或脉有实热,而至数甚数者,用白虎汤时,皆宜酌加人参。 
寒温证表里皆虚,汗出淋漓,阳明胃腑,仍有实热者,用此汤时,宜加龙骨、牡蛎。一童子,年十六, 
于季冬得伤寒证。因医者用发表药太过,周身时时出汗,仍表里大热,心中怔忡,精神恍惚。脉象洪数,按之 
无力。遂用此汤,加龙骨、牡蛎(皆不 )各一两,煎汁一大碗,分数次温饮下,尽剂而愈。 
仲景治伤寒脉结代者,用炙甘草汤,诚佳方也。愚治寒温,若其外感之热不盛,遇此等脉,即遵仲景之 
法。若其脉虽结代,而外感之火甚实者,亦用白虎加人参以山药代粳米汤。曾治一叟,年六旬余。于孟冬得伤 
寒证,五六日间,延愚诊视。其脉洪滑,按之亦似有力。表里俱觉发热,间作呻吟,又兼喘逆,然不甚剧。投 
以白虎汤,一剂大热稍减。再诊其脉,或七八动一止,或十余动一止,两手皆然,而重按无力。遂于原方中加 
人参八钱,兼师炙甘草汤中用干地黄之意,以生地代知母。煎汁两盅,分二 
次温饮下。脉即调匀,且较前有力,而热仍如故。从前方中生石膏二两遂加倍为四两,煎汁一大碗,俾徐徐温 
饮下,尽剂而愈。 
按∶治此证时,愚习用白虎汤,而犹未习用白虎汤加参也。自此以后,凡年过六旬之人,即脉甚洪实, 
用白虎汤时,亦必少加人参二三钱。 
结代之脉虽并论,究之结脉轻于代脉,故结脉间有宜开通者。曾治一叟,年六十余,大便下血,医治三 
十余日,病益进。日下血十余次,且多血块,精神昏愦。延为诊视,脉洪实异常,至数不数,惟右部有止时, 
其止无定数,乃结脉也。其舌苔纯黑,知系温病大实之证。从前医者,但知治其便血,不知治其温病可异也。 
投以白虎加人参以山药代粳米汤,将石膏改用四两,煎汤三盅,分三次温饮下。每次送服旱三七细末一钱。如 
此日服一剂,两日血止,大便仍滑泻,脉象之洪实减半,而其结益甚,且腹中觉胀。询其病因,知得诸恼怒之 
后。遂改用莱菔子六钱,而佐以白芍、滑石、花粉、茅根、甘草诸药,一剂胀消。脉之至数调匀,仍稍有洪实 
之象,滑泻亦减。再投以滋阴清燥汤,一剂泻止,脉亦平和。 
寒温之证,最忌舌干。至舌苔薄而干,或干而且缩者,尤为险证。而究其原因,却非一致。有因真阴亏损 
者,有因气虚不上潮者,有因气虚更下陷者,皆可治以白虎加人参以山药代粳米汤。盖人参之性,大能补气, 
元气旺而上升,自无下陷之虞。而与石膏同用,又大能治外感中之真阴亏损,况又有山药、知母,以濡润之乎。 
若脉象虚数者,又宜多用人参,减石膏一两,再加玄参、生地滋阴之品。煎汁三四茶盅,徐徐温饮下,一次只 
饮一大口,防其寒凉下侵致大便滑泻。又欲其药力息息上达,助元气以生津液,饮完一剂,再煎一剂,使药力 
昼夜相继,数日舌润火退,其病自愈。一人年二十余,素劳力太过,即觉气分下陷。一 
岁之间,为治愈三次。至秋杪感冒时气,胸中烦热满闷,燥渴引饮,滑泻不止,微兼喘促。舌上无苔,其色鲜 
红,兼有砂粒。延医调治,投以半补半破之剂。意欲止其滑泻兼治其满闷也。服药二剂,滑泻不止。后愚为诊 
视,其脉似有实热,重按无力。遂先用拙拟加味天水散止其滑泻。方中生山药用两半、滑石用一两,一剂泻 
止。继服滋阴清火之剂,数剂喘促亦愈,火亦见退。唯舌干连喉,几不能言,频频饮水,不少濡润,胸中仍觉 
满闷。愚恍悟曰∶此乃外感时气,挟旧病复发,故其脉象虽热,按之不实。其舌干如斯者,津液因气分下陷而 
不上潮也。其胸中满闷者,气分下陷,胸中必觉短气,病患不善言病情,故漫言满闷也。此时大便不行已五日。 
遂投以白虎加人参以山药代粳米汤,一剂病愈十之七八,而舌之干亦减半。又服一剂,大便得通,病觉全愈。 
舌上仍无津液,又用潞参一两、玄参两半,日服一剂,三日后舌上津液滋润矣。 
一童子,年十三,于孟冬得伤寒证。七八日间,喘息鼻煽动,精神昏愦,时作谵语,所言者皆劳力之事。 
其脉微细而数,按之无力。欲视其舌,干缩不能外伸,启齿探视,舌皮有瘢点作黑色,似苔非苔,频饮凉水, 
毫无濡润之意。愚曰∶此病必得之劳力之余,胸中大气下陷,故津液不能上潮,气陷不能托火外出,故脉道瘀 
塞。不然,何以脉象若是,恣饮凉水而不滑泻乎?遂治以白虎加人参以山药代粳米汤,煎汁一大碗,徐徐温饮 
下,一昼夜间连进二剂,其病遂愈。 
按∶脉虚数而舌干者,大便虽多日不行,断无可下之理,即舌苔黄而且黑亦不可下。惟按上所载治法, 
使其大便徐徐自通,方为稳善。若大便通后,而火犹炽,舌仍干者,可用潞参一两、玄参二两煮汁,徐徐饮之, 
以舌润火退为度。若或因服药失宜,大便通后,遂滑泻,其虚火上逆,舌仍干者,可用拙拟滋阴固下汤去滑石, 
加沙参数钱。若其为日既久,外感之火全消,而舌干 
神昏,或呼吸之间,常若气不舒,而时作太息者,此大气因服药下陷,病虽愈而不能自复也。宜单用人参两许 
煎汤服之,或少加柴胡亦可(此证有案在升陷汤下宜参观)。若微有余热,可加玄参佐之。 
寒温下后不解,医者至此,恒多束手。不知《伤寒论》原有治此证的方,即白虎加人参汤也。其一百六十 
八节云∶“伤寒病,若吐、若下后,七、八日不解,热结在里,表里俱热,时时恶风,大渴、舌上干燥而烦, 
欲饮水数升者,白虎加人参汤主之。”愚生平治寒温,未有下后不解者,于仙露汤后曾详论之。然恒有经他医 
下后不解,更延愚为延医者,其在下后多日,大便未行,脉象不虚弱者,即按《伤寒论》原方。若在甫下之后, 
或脉更兼虚弱,即以山药代粳米,或更以生地代知母,莫不随手奏效。盖甫下之后,大便不实,骤用寒凉,易 
至滑泻。而山药收涩,地黄粘润,以之代粳米、知母,实有固下之力,而于脉之兼虚弱者,则尤宜也。况二药 
皆能滋真阴,下后不解,多系阴分素虚之人,阴分充足,自能胜外感之余热也。 
寒温之证,过十余日大热已退,或转现出种种危象。有宜单治以人参,不必加人参于白虎汤中者。王宇泰 
曰∶余每治伤寒温热等证,为庸医妄汗、误下已成坏证,危在旦夕者,以人参二两,童子小便煎之,水浸冰冷, 
饮之立效。又张致和曾治一伤寒坏证,势近垂危,手足俱冷,气息将断。用人参一两、附子一钱,于石HT 内煎 
至一碗,新汲水浸之冰冷,一服而尽。少顷病患汗出,鼻梁尖上涓涓如水。盖鼻梁应脾,若鼻端有汗者可救, 
以土在人身之中周遍故也。 
又∶愚曾治一温证,已过两旬,周身皆凉,气息奄奄。确知其因误治,胸中大气下陷。遂用人参一两、柴 
胡二钱,作汤灌之,两剂全愈。此证详案,在拙拟升陷汤下可参观。 
白虎汤加人参,又以山药代粳米,既能补助气分托邪外出, 
更能生津止渴,滋阴退热,洵为完善之方。间有真阴太虚,又必重用滋阴之药以辅翼之,始能成功者。一媪, 
年过七旬,于孟夏得温证,五六日间,身热燥渴,精神昏愦,舌似无苔,而舌皮数处作黑色,干而且缩。脉细 
数,按之无力。当此高年,审证论脉,似在不治。而愚生平临证,明明见不可治之证,亦必苦心研究而设法治 
之,此诚热肠所迫,不能自已,然亦往往多有能救者。踌躇再四,为疏两方。一方即白虎加人参以山药代粳米 
汤,一方用熟地黄二两,生山药、枸杞各一两,真阿胶(不炒)五钱,煎汤后,调入生鸡子黄四枚。二方各煎 
汁一大碗,徐徐轮流温服,阅十点钟,尽剂而愈。自言从前服药,皆不知觉,此时则犹如梦醒。视其舌上犹干 
黑,然不缩矣。其脉至数仍数,似有余热。又用玄参二两、潞参一两煎汤一大碗,徐徐温服,一日一剂,两日 
大便得通。再视其舌,津液满布,黑皮有脱去者矣。 
隔数日,其夫年与相等,亦受温病。四五日间,烦热燥渴。遣人于八十里外致冰一担,日夜食之,烦渴如 
故。复迎愚延医,其脉洪滑而长,重按有力,舌苔白浓,中心微黄。知其年虽高而火甚实也。遂投以白虎加人 
参以山药代粳米汤,将方中石膏改用四两,连进两剂,而热渴俱愈。其家人疑而问曰∶此证从前日食冰若干, 
热渴分毫不退,今方中用生石膏数两,连进两剂而热渴俱愈,是石膏之性凉于冰远矣。愚曰∶非也。石膏原不 
甚凉,然尽量食冰不愈而重用生石膏即愈者,因石膏生用能使寒温之热有出路也。西人不善治寒温,故遇寒温 
实热证最喜用冰,然多有不愈者。至石膏生用,性能发汗,其热可由汗解。即使服后无汗,亦可宣通内蕴之热, 
由腠理毛孔息息达出,人自不觉耳。 
按∶此证与前证,年岁同,受病之时亦同。而一则辅以熟地、枸杞之类,以滋真阴;一则重加生石膏,以 
清大热。此乃随病、脉之虚实,活泼加减,所以投之辄效也。 
忆五年前,族家姊,年七旬有三,忽得瘫痪证,迎愚诊视。既至见有医者在座。用药一剂,其方系散风补 
气理痰之品,甚为稳善,愚亦未另立方。翌日,脉变洪长,知其已成伤寒证。先时,愚外祖家近族有病者,订 
于斯日迎愚,其车适至,愚将行,谓医者曰∶此证乃瘫痪基础预伏于内,今因伤寒而发,乃两病偕来之证。然 
瘫痪病缓,伤寒病急。此证阳明实热,已现于脉,非投以白虎加人参汤不可,君须放胆用之,断无差谬。后医 
者终畏石膏寒凉,又疑瘫痪证不可轻用凉药。迟延二日,病势垂危,复急迎愚。及至,则已夜半矣。诊其脉, 
洪而且数,力能搏指,喘息甚促,舌强直,几不能言。幸喜药坊即在本村,急取白虎加人参汤一剂,方中生石 
膏用三两,煎汤两盅,分二次温饮下,病稍愈。又单取生石膏四两,煮汁一大碗,亦徐徐饮下,至正午尽剂而 
愈。后瘫痪证调治不愈,他医竟归咎于愚。谓从前用过若干石膏,所以不能调治。吁!年过七旬而瘫痪者,愈 
者几人?独不思愚用石膏之时,乃挽回已尽之人命也。且《金匮》治热瘫痫,有风引汤,原石膏与寒水石并用。 
彼谤愚者,生平盖未见《金匮》也。 
尝治一少年,素羸弱多病。于初夏得温证,表里俱热,延医调治不愈。适愚自他处治病归,经过其处, 
因与其父素稔,入视之。其脉数近六至,虽非洪滑鼓指,而确有实热。舌苔微黄,虽不甚干,毫无津液。有煎 
就药一剂未服,仍系发表之剂,乃当日延医所疏方,其医则已去矣。愚因谓其父曰∶此病外感实热,已入阳明 
之府。其脉象不洪滑者,元气素虚故也。阳明府热之证,断无发表之理。况其脉数液短,兼有真阴虚损之象尤 
忌发汗乎。其父似有会悟,求愚另为疏方。本拟用白虎加人参汤,又思用人参,即须多用石膏,其父素小心过 
度,又恐其生疑不敢服,遂但为开白虎汤,方中生石膏用二两。嘱其煎汁两茶盅,分二次温饮下,服后若余火 
不净,仍宜再服清火之药。言毕,愚即旋里。后 
闻其服药后,病亦遂愈。迟十余日,大便又燥结,两腿微肿,将再迎愚延医。而其父友人有自谓知医者,言其 
腿肿,系多服生石膏之过。而孰知系服石膏犹少之过哉!病家竟误听其言,改延他医,投以大剂承气汤,服后 
其人即不语矣,迁延数日而亡。夫自谓知医者,不过欲炫己之长,而妄指他人之短。岂知其言之一出,即足误 
人性命哉! 
至产后之证,忌用寒凉。而果系产后温证,心中燥热,舌苔黄浓,脉象洪实,亦宜投以白虎加人参以山药 
代粳米汤,而更以玄参代知母则尤妥善。盖愚于产后温证之轻者,其热虽入阳明之府,脉象不甚洪实,恒重用 
玄参一两或至二两,辄能应手奏效;若系剧者,必白虎加人参以山药代粳米汤,而更以玄参代知母方能有效。 
诚以石膏、玄参,《神农本草经》皆明载其治产乳。故于产后温病之轻者,可单用玄参。至温病之剧者,不妨 
石膏、玄参并用也。然用石膏必须佐以人参,因其时当产后,其热虽实,而体则虚也。不用知母者,《神农本 
草经》未载其治产乳,不敢师心自用,漫以凉药治产后也。 
友人吴××,深通医学,其侄××亦知医,有戚家延之治产后病。临行,吴××嘱之曰∶果系产后温热,阳 
明胃府大实,非用《衷中参西录》中白虎加人参以山药代粳米汤,更以玄参代知母不可。及至诊之,果系产后 
温证,病脉皆甚实。××遵所嘱,开方取药。而药坊皆不肯与,谓产后断无用生石膏之理,病家因此生疑,×× 
辞归。病家又延医治数日,病势垂危,复求为延医。携药而往,如法服之,一剂而愈。 
附录∶ 
沧县董××来函∶ 
赵××,患温病。医者投以桂枝汤,觉热渴气促。又与柴胡 
汤,热尤甚,且增喘嗽,频吐痰涎,不得卧者六七日。医者谓病甚重,不能为矣。举家闻之,惶恐无措。伊芳弟 
××延为延医。既至,见病患喘促肩息,头汗自出,表里皆热,舌苔深灰,缩不能言。急诊其脉,浮数有力,重 
按甚空。因思此证阳明热极,阴分将竭,实为误服桂枝、柴胡之坏证。急投以白虎加人参以山药代粳米汤,更 
以玄参代知母。连服两剂,渴愈喘止,脉不浮数,仍然有力,舌伸能言,而痰嗽不甚见轻。继投以从龙汤,去 
苏子,加人参四钱,天冬八钱,服七剂全愈。 
又∶一赵姓妇,年二十余,产后八九日,忽得温病。因误用热药发汗,致热渴喘促,舌苔干黑,循衣摸床, 
呼索凉水,病家不敢与。脉弦数有力,一息七至。急投以白虎加人参以山药代粳米汤,为系产后,更以玄参代 
知母。方中生石膏,重用至四两。又加生地、白芍各数钱。煎汤一大碗,分四次温饮下,尽剂而愈。当时有知 
医者在座,疑而问曰∶“产后忌用寒凉,何以如此放胆,重用生石膏?且知母、玄参皆系寒凉之品,何以必用 
玄参易知母?”答曰∶“此理俱在《衷中参西录》中”,遂于行箧中出书示知,医者细观移时,始喟然叹服。 
又∶外祖家表妹,因产后病温,服补药二十余剂,致大热、大渴、大汗,屡索凉水。医者禁勿与饮,急欲 
投井。及生视之,舌黑唇焦,目睛直视,谵语发狂。诊其脉,细数有力。问其小便赤涩,大便紫黑粘滞,不甚 
通利。盖以产后血虚,又得温病,兼为补药所误,以致外邪无由而出,内热如焚,阴血转瞬告罄。急投以白虎 
加人参汤,仍用山药、玄参代粳米、知母。服后,一夜安稳。黎明,旋又反复,热渴又如从前。细思产后血室 
空虚,邪热乘虚而入,故大便紫黑,宜调以桃仁承气汤,以下其瘀血,邪热当随之俱下。因小便赤涩,膀胱蓄 
热,又加滑石四钱,甘草钱半。乃开药局者系其本族,谓此药断不可服。病家疑甚,复延前 
医相质。前医谓,此病余连治三次,投以温补药转剧,昨服白虎加人参汤,既稍见轻,想服承气汤亦无妨也。 
病家闻之,始敢煎服。因方中大黄重用六钱,俾煎汤一盅半,分三次温饮下。逾三点钟,降下大便如胶漆者二 
次,鲜红色者一次,小便亦清利,脉净身凉而愈。 
又∶赵××之子,年十九岁,偶得温病,医者下之太早,大便转不通者十八日,热渴喘满,舌苔干黑,牙龈 
出血,目盲谵语,腹胀如鼓,脐突出二寸,屡治不效。忽大便自利,完谷不化,随食随即泻出。诊其脉,尽伏。 
身冷厥逆,气息将无。乍临茫然不知所措,细询从前病状及所服之药,始悟为阳极似阴,热深厥亦深也。然须 
用药将其滑泻止住,不复热邪旁流,而后能治其热厥。遂急用野台参三钱,大熟地、生山药、滑石各六钱。煎 
服后,泻止脉出,洪长滑数,右部尤甚。继拟以大剂白虎加人参汤,生石膏重用至八两。竟身热厥回,一夜 
甚安。至明晨,病又如故。试按其腹中,有坚块,重按眉皱似疼,且其腹胀脐突若此,知其内有燥粪甚多。遂 
改用大黄一两,芒硝六钱,赭石、蒌仁各八钱,煎汤一大盅,分两次温饮下。下燥粪二十七枚而愈。 
奉天铁岭杨××来函∶ 
本村张氏妇,得温病,继而小产,犹不以为意。越四五日,其病大发。遍请医生,均谓瘟病小产,又兼 
邪热太甚,无方可治。其夫造门求为延医。生至其家,见病患目不识人;神气恍惚;渴嗜饮水,大便滑泻;脉 
数近八至,且微细无力;舌苔边黄中黑,缩不能伸。为其燥热,又兼滑泻,先投以滋阴清燥汤,一剂泻止,热 
稍见愈。继投以大剂白虎加人参以山药代粳米汤,为其产后,以玄参代知母,为其舌缩脉数,阴分大亏,又加 
枸杞、生地。煎汤一大碗,调入生鸡子黄三枚,分数次徐徐温饮下。精神清爽,舌能伸出,连服三剂全愈。 


<篇名>5.宁嗽定喘饮
属性:治伤寒温病,阳明大热已退,其人或素虚或在老年,至此益形怯弱,或喘,或嗽,或痰涎壅盛,气息似甚不足者。 
生怀山药(两半) 甘蔗自然汁(一两) 酸石榴自然汁(六钱) 生鸡子黄(四个) 
先将山药煎取清汤一大碗,再将余三味调入碗中,分三次温饮下,约两点钟服一次。若药已凉,再服时须 
将药碗置开水中温之。然不可过热,恐鸡子黄熟,服之即无效。 
一周姓叟,年近七旬,素有劳疾,且又有鸦片嗜好,于季秋患温病,阳明府热炽盛,脉象数而不实,喘而 
兼嗽,吐痰稠粘。投以白虎加人参汤,以生山药代粳米,一剂,大热已退,而喘嗽仍不愈,且气息微弱,似不 
接续。其家属惶恐,以为难愈。且言如此光景,似难再进药。愚曰∶勿须用药,寻常服食之物即可治愈矣。为 
开此方,病家视之,果系寻常食物,知虽不对证,亦无妨碍。遂如法服之,二剂全愈。 


<篇名>6.荡胸汤
属性:治寒温结胸,其证胸膈痰饮,与外感之邪互相凝结,上塞咽喉,下滞胃口,呼吸不利,满闷短气,饮水不 
能下行,或转吐出。兼治疫证结胸。 
蒌仁(二两,新炒者捣) 生赭石(二两,研细) 苏子(六钱,炒捣) 芒硝(四钱,冲服) 
用水四盅,煎取清汁两盅,先温服一盅。结开,大便通行,停后服。若其胸中结犹未开,过两点钟,再温 
服一盅。若胸中之结已开,而大便犹未通下,且不觉转矢气者,仍可温服半盅。 
伤寒下早成结胸,至温病未经下者,亦可成结胸。至疫病自口鼻传入,遇素有痰饮者,其疹疠之气,与上 
焦痰饮,互相胶漆, 
亦成结胸。《伤寒论》陷胸汤、丸三方,皆可随证之轻重高下借用。特是大陷胸汤、丸中皆有甘遂,世俗医者, 
恒望而生畏。至小陷胸汤,性虽平和,又有吴又可瘟疫忌用黄连之说存于胸中,遂亦不肯轻用。及遇此等证, 
而漫用开痰、破气、利湿之品,若橘红、莱菔、苍术、白芥、茯苓、浓朴诸药,汇集成方,以为较陷胸诸汤、 
丸稳,而且病家服之,以为药性和平,坦然无疑。不知破其气而气愈下陷,利其湿而痰愈稠粘。如此用药,真 
令人长太息者也。愚不得已,将治结胸诸成方变通汇萃之,于大陷胸汤中取用芒硝,于小陷胸汤中取用蒌实, 
又于治心下痞硬之旋复代赭石汤中取用赭石,而复加苏子以为下行之向导,可以代大陷胸汤、丸。少服之,亦 
可代小陷胸汤。非欲与《伤寒论》诸方争胜也,亦略以便流俗之用云尔。 
一媪,年六十余。当孟夏晨饭之际,忽闻乡邻有斗者,出视之,见强者凌弱太甚,心甚不平;又兼饭后有 
汗受风,遂得温证。表里俱热,胃口杜塞,腹中疼痛,饮水须臾仍吐出。七八日间,大便不通。其脉细数,按 
之略实。自言心中燥渴,饮水又不能受,从前服药止吐,其药亦皆吐出。若果能令饮水不吐,病犹可望愈。愚 
曰∶易耳。为开此汤,加生石膏二两、野台参五钱,煎汤一大碗,分三次温饮下。晚间服药,翌晨大便得通而 
愈。当大便未通时,曾俾用山萸肉(去净核)二两煎汤,以备下后心中怔忡及虚脱,及大便通后,微觉怔忡, 
服之即安。 
一室女得温病。两三日间,痰涎郁塞,胸膈满闷异常,频频咳吐,粘若胶漆,且有喘促之意,饮水停滞胃 
口,间或吐出,其脉浮滑。问之微觉头疼,知其表证犹未罢也。遂师河间双解散之意,于荡胸汤中加连翘、蝉 
蜕各三钱。服后微汗,大便得通而愈。 


<篇名>7.一味莱菔子汤
属性:治同前证。 
莱菔子(生者一两,熟者一两) 共捣碎,煎汤一大茶杯,顿服之。 
奉天许××,年二十余。得温病。三四日觉中脘郁结,饮食至其处不下行,仍上逆吐出。来院求为延医。 
其脉沉滑而实,舌苔白而微黄。表里俱觉发热,然不甚剧。自言素多痰饮,受外感益甚。因知其中脘之郁结, 
确系外感之邪与痰饮相凝滞也。先投以荡胸汤,两点钟后,仍复吐出。为拟此方,一剂结开,可受饮食。继投 
以清火理痰之品,两剂全愈。 
按∶此证若服荡胸汤,将方中赭石细末留出数钱,开水送下,再服汤药亦可不吐,其结亦必能开。非莱菔 
子汤之力胜于荡胸汤也,而试之偶效,尤必载此方者,为药性较荡胸汤尤平易,临证者与病家,皆可放胆用之 
而无疑也。若此方不效者,亦可改用荡胸汤,先将赭石细末送下数钱之法。 


<篇名>8.镇逆承气汤
属性:治寒温阳明府实,大便燥结,当用承气下之,而呕吐不能受药者。 
芒硝(六钱) 赭石(二两,研细) 生石膏(二两,捣细) 潞党参(五钱) 
上药四味,用水四盅,先煎后三味,汤将成,再加芒硝,煎一两沸。取清汁二盅,先温服一盅。过三点钟, 
若腹中不觉转动,欲大便者,再温服余一盅。 
一邻妇,年二十余。得温病已过十日,上焦燥热,呕吐,大便燥结,自病后未行。延医数次服药皆吐出。 
适愚自他处归,诊其脉,关前甚洪实,一息五至余,其脉上盛于下一倍,所以作呕吐。其至数数者,吐久伤津 
液也。为拟此汤,一剂热退呕止,大便得通而愈。 
或问∶此证胃腑热实大肠燥结,方中何以复用党参?答曰∶此证多有呕吐甚剧,并水浆不能存者,又有初 
病即呕吐,十数日不 
止者,其胃气与胃中津液,必因呕吐而大有伤损,故用党参补助胃中元气,且与凉润之石膏并用,大能滋胃中 
津液,俾胃中气足液生,自能运转药力下至魄门以通大便也。愚用此方救人多矣,果遇此等证,放胆投之,无 
不效者。 
一人,年四十许。二便不通,呕吐甚剧,不受饮食。倩人询方。疑系外感之热所致,问其心中发热否?言 
来时未尝言及。遂为约略疏方,以赭石二两以止其呕吐,生杭芍一两以通小便,芒硝三钱以通大便。隔日,其 
人复来,言服后呕吐即止,二便亦通,此时心中发热且渴如故。既曰如故,是其从前原有热渴之病,阳明之腑 
证已实,特其初次遣人未尝详言也。投以大剂白虎加人参汤,一剂而愈。 
按∶此证亦镇逆承气汤证,因其证两次始述明,遂致将方中药品前后两次分用之,其病亦即前后两次而愈矣。 

\chapter{治瘟疫瘟疹方}
<篇名>1.青盂汤
属性:治瘟疫表里俱热,头面肿疼,其肿或连项及胸。亦治阳毒发斑疹。 
荷叶(一个用周遭边浮水者良鲜者尤佳) 生石膏(一两,捣细) 真羚羊角(二钱,另煎兑服) 
知母(六钱) 蝉蜕(三钱,去足土) 僵蚕(二钱) 金线重楼(二钱,切片) 粉甘草(钱半) 
荷叶禀初阳上升之气,为诸药之舟楫,能载清火解毒之药上至头面,且其气清郁,更能解毒逐秽,施于疫 
毒诸证尤宜也。至于叶宜取其浮水者,以贴水而生,得水面轻气最多,故善发表。 
如浮萍之生于水面,而善发汗也。 
金线重楼,一名蚤休,一名紫河车草。味甘而淡,其解毒之功,可仿甘草。然甘草性温,此药性凉,以解 
一切热毒,尤胜于甘草,故名蚤休。言若中一切蛊毒,或蝎螫蛇咬、或疮疡用之而皆可早早止住。古蚤与早, 
原相通也。古谚赞蚤休曰∶“七叶一枝花,深山是我家。痈疽遇着我,一似手捻拿。”盖此物七叶对生茎腰, 
状如莲花一朵,自叶中心出茎,至巅开花一朵,形扁而黄,花上有黄丝下垂,故又名金线重楼。重楼者,其叶 
与花似各作一层也。其名紫河车草者,盖紫河为初生之地点,其处蕃多,可采之盈车,俗名为草河车误矣。其 
形状皮色皆如干姜。若皮不黄,而微带紫色者,其味必微辣而不甘,含毒性,即不可用。若无佳者,方中不用 
此味亦可。 
羚羊角与犀角,皆性凉而解毒。然犀禀水土之精气而生,为其禀土之精,故能入胃,以消胃腑之实热。为 
其禀水之精,故又能以水胜火兼入心中,以消心脏本体之热力。而疫邪之未深入者,转因服犀角后,心气虚冷, 
不能捍御外邪,致疫邪之恣横,竟犯君主之宫,此至紧要之关系,医者不可不知。羚羊角善清肝胆之火,兼清 
胃腑之热。其角中天生木胎,性本条达,清凉之中,大具发表之力。与石膏之辛凉,荷叶、连翘之清轻升浮者 
并用,大能透发温疫斑疹之毒火郁热,而头面肿处之毒火郁热,亦莫不透发消除也。曾治一六岁孺子,出疹三 
四日间,风火内迫,喘促异常。单投以羚羊角三钱,须臾喘止,其疹自此亦愈。夫疹之毒热,最宜表散清解, 
乃至用他药表散清解无功,势已垂危,而单投以一味羚羊角,即能挽回,其最能清解而兼能表散可知也。且其 
能避蛊毒,《神农本草经》原有明文。疫病发斑,皆挟有毒疠之气也。 
僵蚕乃蚕将脱皮时,因受风不能脱下,而僵之蚕。因其病风 
而僵,故能为表散药之向导,而兼具表散之力。是以痘疹不出者,僵蚕最能表出之。不但此也,僵蚕僵而不腐, 
凡人有肿疼之处,恐其变为腐烂,僵蚕又能治之,此气化相感之妙也。 
疫与寒温不同。寒温者,感时序之正气。因其人卫生之道,于时序之冷暖失宜,遂感其气而为病。其病 
者,偶有一二人,而不相传染。疫者,感岁运之戾气。因其岁运失和,中含毒瓦斯,人触之即病。《内经》刺法 
论所谓无问大小,病状相似者是也。其病者,挨户挨村,若摇役然,故名曰疫,且又互相传染也。《内经》本 
病论有五疫之名,后世约分为寒疫、温疫。治温疫,世习用东垣普济消毒饮。治寒疫,世习用巢谷世圣散子。 
然温疫多而寒疫少,拙拟之清盂汤,实专为治温疫设也。 
一妇人,年四十许,得大头瘟证。头面肿大疼痛,两目肿不能开,上焦烦热,心中怔忡。彼家误为疮毒, 
竟延疡医治疗。医者自出药末,敷头面,疼稍愈。求其出方治烦热怔忡,彼言专习外科,不管心中之病。时愚 
应他家延请,适至其村,求为延医。其脉洪滑有力,关前益甚,投以清盂汤,将方中石膏改用二两,煎汁两茶 
盅,分二次温饮下,尽剂而愈。 
一人,年二十余,得温疫。三四日间头面悉肿,其肿处,皮肤内含黄水,破后且溃烂。身上间有斑点, 
闻人言,此证名大头瘟。其溃烂之状,又似瓜瓤瘟,最不易治。惧甚,求为诊视。其脉洪滑而长,舌苔白而微 
黄。问其心中,惟觉烦热,嗜食凉物。遂晓之曰,此证不难治。头面之肿烂,周身之斑点,无非热毒入胃而随 
胃气外现之象。能放胆服生石膏,可保全愈。遂投以青盂汤,方中石膏改用三两,知母改用八钱,煎汁一大碗, 
分数次温饮下。一剂病愈强半。翌日,于方中减去荷叶、蝉蜕,又服一剂全愈。 
按∶发斑之证异于疹者,以其发处不高,以手拂之,与肤平也。其证有阳毒、阴毒之分。阳毒发斑,系 
阳明毒热伤血所致。 
阴毒发斑,或为寒疫之毒,或因汗吐下后中气虚乏,或因过服凉药,遂成阴证,寒伏于下,逼其无根之火上独 
熏肺而发斑。其色淡红,隐隐见于肌表,与阳证发斑色紫赤者不同。愚生平所治发斑,皆系阳证。至阴证实未 
之见,其证之甚少可知。然正不可因阴证者甚少,而阴阳之际不详辨也。今采古人阳毒阴毒发斑治验之案数条 
于下,以备参观。庶几胸有定见,临证时不至误治也。 
吕沧洲云∶一人伤寒十余日,身热而静,两手脉尽伏。医者以为坏证,弗与药。余诊之,三部脉举按皆无, 
舌苔滑,两颧赤如火,语言不乱,因告之曰∶此子必大发赤斑,周身如锦纹。夫血脉之波澜也,今血为邪热所 
搏,掉而为斑,外现于皮肤,呼吸之气无形可倚,犹沟渠之水虽有风不能成波澜也,斑消则脉出矣。及揭其衾, 
而赤斑烂然。与白虎加人参汤,化其斑脉乃复常。按∶发斑至于无脉,其证可谓险矣。即遇有识者,细诊病情, 
以为可治,亦必谓毒火郁热盘踞经络之间以阻塞脉道之路耳。而沧洲独断为发斑则伤血,血伤则脉不见。是诚 
沧洲之创论,然其言固信而有征也。忆己亥春,尝治一少年吐血证。其人大口吐血,数日不止,脉若有若无, 
用药止其血后,脉因火退,转分毫不见。愚放胆用药调补之,竟得无恙(此证详案在寒降汤下)。夫吐血过多 
可至无脉,以征沧洲血伤无脉之说确乎可信。此阳毒发斑也。 
许叔微治一人,内寒外热而发斑。六脉沉细,肩背胸胁斑出数点,随出随隐,旋更发出,语言狂乱,非谵 
语也,肌表虽热,以手按之,须臾冷透如冰。与姜附等药数服后,得大汗而愈。此阴毒发斑也。 
吴仁斋治一人,伤寒七八日,因服凉药太过,遂变身冷,手足厥逆,通身黑斑,惟心头温暖,乃伏火也。 
诊其六脉沉细,昏沉不知人事,亦不能言语,状似尸厥。遂用人参三白汤,加熟附子半枚、干姜二钱,水煎服 
下。待一时许,斑色渐红,手足渐暖。 
而苏醒后,复有余热不清,此伏火后作也。以黄连解毒汤、竹叶石膏汤调之而愈。此阴毒发斑中有伏阳也。 
虞天民曰∶有内伤证,亦出斑疹,但微见红。此胃气极虚,一身之火游行于外。当补益气血,则中有主而 
气不外游,荣有养而血不外散,此证尤当慎辨。洪吉人解之曰∶按此证与阳毒发斑不同,亦与阴毒发斑不同, 
其方当用补中益气汤,加归、芍之类。 
瘟毒之病,有所谓羊毛瘟者(亦名羊毛疹),其证亦系瘟疫,而心中兼有撩乱之证。若视其前后对心处有 
小痤(俗名疙瘩),以针鼻点之,其顶陷而不起,其中即有白毛,当以针挑出之。若恐挑之不净,可用发面馍 
馍去皮,杂以头发,少蘸香油,周身搓擦。再审其证之虚实凉热,投以治疫病之药,即愈。此证古书不载,而 
今人患此证者甚多,其白毛,即周身之汗毛,大抵因有汗受风闭其毛孔,而汗毛不能外出,因不外出,所以作 
白色(若用黄酒和荞麦面擦之更好)。 


<篇名>2.护心至宝丹
属性:治瘟疫自肺传心,其人无故自笑,精神恍惚,言语错乱。 
生石膏(一两,捣细) 人参(二钱) 犀角(二钱) 羚羊角(二钱) 朱砂(三分,研细) 牛黄(一分,研 
细) 
将药前四味共煎汤一茶盅,送服朱砂、牛黄末。 
此证属至危之候,非寻常药饵所能疗治。故方中多用珍异之品,借其宝气以解入心之热毒也。 
瘟疫之毒未入心者,最忌用犀角。而既入心之后,犀角又为必须之药。 
瘟疫之毒,随呼吸之气传入,原可入肺。心与肺同居膈上,且左心房之血脉管与右心房之回血管,又皆与 
肺循环相通,其相传似甚易。而此证不常有者,因有包络护于心上代心受邪,由包 
络下传三焦,为手厥阴,少阳脏腑之相传,此心所以不易受邪也。愚临证二十余年,仅遇一媪患此证,为拟此 
方,服之而愈。 


<篇名>3.清疹汤
属性:治小儿出疹,表里俱热。或烦躁引饮,或喉疼声哑,或喘逆咳嗽。 
生石膏(一两,捣细) 知母(六钱) 羚羊角(二钱) 金线重楼(钱半,切片) 薄菏叶(二钱) 青连 
翘(二钱) 蝉蜕(钱半,去足土) 僵蚕(二钱) 
用水煎取清汤一盅半,分二次温饮下,以服后得微汗为佳。若一次得微汗者,余药仍可再服。若服一次即 
得大汗者,余药当停服。此药分量,系治七八岁以上者,若七八岁以下者,可随其年之大小,斟酌少用。或将 
药减半或用三分之一皆可。 
喉疼声哑者,可将石膏加重五钱,合前得两半。若疹出不利者,用鲜苇根(活水中者更佳)一大握去节水 
煎沸,用其水煎药。 
疹证多在小儿,想小儿脏腑间原有此毒,又外感时令之毒瓦斯而发,则一发表里俱热。若温病初得之剧者, 
其阳明经府之间,皆为热毒之所弥漫。故治此证,始则发表,继则清解,其有实热者,皆宜用石膏。至喉疼声 
哑者,尤为热毒上冲,石膏更宜放胆多用。惟大便滑泻者,石膏、知母皆不宜用,可去此二药,加滑石一两、 
甘草三钱。盖即滑泻亦非凉证,因燥渴饮水过多,脾胃不能运化故也,故加滑石以利其小便,甘草以和其脾胃, 
以缓水饮下趋之势。若其滑泻之甚者,可用拙拟滋阴宣解汤,既可止泻,又可表疹外出也。然此证最忌滑泻, 
恐其毒因滑泻内陷即不能外出。若服以上方而滑泻不止,可用生山药两许,轧细煮作粥,再将熟鸡子黄两三枚 
捏碎调粥中服之,其滑泻必止。泻止后,再徐徐以凉药清补之。 
奉天友人朱××之子,年五岁。于庚申立夏后,周身壮热,出疹甚稠密,脉甚洪数,舌苔白浓,知其疹而兼 
瘟也。欲以凉药清解之,因其素有心下作疼之病,出疹后,贪食鲜果,前一日犹觉疼,又不敢投以重剂。遂勉 
用生石膏、玄参各六钱,薄荷叶、蝉蜕各一钱,连翘二钱。晚间服药,至翌日午后视之,其热益甚,喉疼,气 
息甚粗,鼻翅煽动,且自鼻中出血少许,有烦躁不安之意。愚不得已,重用生石膏三两,玄参、麦冬(带心) 
各四钱,仍少佐以薄荷叶、连翘诸药。俾煎汤二茶盅,分三次温饮下。至翌日视之,则诸证皆轻减矣。然余热 
犹炽,而大便虽下一次,仍系燥粪。询其心犹发热,脉仍有力。遂于凉解药中,仍用生石膏一两,连服两剂, 
壮热始退。继用凉润清解之剂调之全愈。 
按∶此证初次投以生石膏、玄参各六钱,其热不但不退而转见增加,则石膏之性原和平,确非大凉可知也。 
至其证现种种危象,而放胆投以生石膏三两,又立能挽回,则石膏对于有外感实热诸证,直胜金丹可知。此证 
因心下素有疼病,故石膏、玄参初止用六钱。若稍涉游移,并石膏、玄参亦不敢用,再认定疹毒,宜托之外出 
而多用发表之品,则翌日现证之危险,必更加剧,即后投以大剂凉药,亦不易挽回也。目睹耳闻,知孺子罹瘟 
疹之毒,为俗医药误者甚多,故于记此案时,而再四详为申明。 
瘟疫之证,虽宜重用寒凉,然须谨防其泄泻。若泄泻,则气机内陷,即无力托毒外出矣。是以愚用大剂寒 
凉,治此等证时,必分三四次徐徐温服下,俾其药力长在上焦,及行至下焦,其寒凉之性已为内热所化,自无泄 
泻之弊。而始终又须以表散之药辅之,若薄荷、连翘、蝉蜕、僵蚕之类,则火消毒净,疹愈之后亦断无他患矣。 
至若升麻、羌活之药,概不敢用。友人刘××,精通医学。曾治一孺子,出疹刚见点即回。医者用一切药,皆不能 
表出。毒瓦斯内攻,势甚危急,众皆束手。刘××投以《伤寒论》麻杏甘石汤,一剂疹皆发出,自此遂愈。夫麻杏 
甘石汤,为汗后、下后、汗出而喘无大热者之方,刘××用以治疹,竟能挽回人命于顷刻,可为善用古方者矣 
(用此方者,当视其热度之高低,热度高者石膏用一两,麻黄用一钱,热度低者石膏用一两,麻黄用二钱)。 

\chapter{治疟疾方}
<篇名>加味小柴胡汤
属性:治久疟不愈,脉象弦而无力。 
柴胡(三钱) 黄芩(二钱) 知母(三钱) 潞参(三钱) 鳖甲(三钱,醋炙) 清半夏(二钱) 
常山(钱半,酒炒) 草果(一钱) 甘草(一钱) 酒曲(三钱) 生姜(三钱) 大枣(两枚,捭开) 
疟初起者减潞参、鳖甲。热甚者,加生石膏五六钱或至一两。寒甚者,再加草果五分或至一钱。(神曲皆 
发不好故方中用酒曲) 
疟邪不专在少阳,而实以少阳为主,故其六脉恒露弦象。其先寒者,少阳之邪外与太阳并也,其后热者, 
少阳之邪内与阳明并也。故方中用柴胡以升少阳之邪,草果、生姜以祛太阳之寒,黄芩、知母以清阳明之热。 
又疟之成也,多挟痰挟食,故用半夏、常山以豁痰,酒曲以消食也。用人参,因其疟久气虚,扶其正即所以逐 
邪外出。用鳖甲者,因疟久则胁下结有痞积(方书名疟母实由肝脾胀大),消其痞积,然后能断疟根株。用甘 
草、大枣者,所以化常山之猛烈而服之不至瞑眩也。 
或问∶叶天士医案,其治疟之方,多不用柴胡。其门人又有根传之说,谓不宜用柴胡治疟。若误用之,实 
足偾事。其说果可 
信乎?答曰∶叶氏当日声价甚高,疟原小疾,初起之时,鲜有延之延医者。迨至疟久,而虚证歧出,恒有疟邪 
反轻,而他病转重,但将其病之重者治愈,而疟亦可随愈,此乃临证通变之法,非治疟之正法也。至于病在厥 
阴,亦有先寒后热,出汗少愈,形状类疟之证。此系肝气虚极将脱,若误认为疟,用柴胡升之,凶危立见。此 
当重用山萸肉,以敛而补之,是以《神农本草经》山茱萸,亦主寒热也。叶氏门人所谓,误用柴胡足偾事者, 
大抵指此类耳。 
或问∶叶氏治疟,遇其人阴虚燥热者,恒以青蒿代柴胡。后之论者,皆赞其用药,得化裁通变之妙。不知 
青蒿果可以代柴胡乎?答曰∶疟邪伏于胁下两板油中,乃足少阳经之大都会。柴胡之力,能入其中,升提疟邪 
透膈上出,而青蒿无斯力也。若遇阴虚者,或热入于血分者,不妨多用滋阴凉血之药佐之。若遇燥热者,或热 
盛于气分者,不妨多用清燥散火之药佐之。曾治一人,疟间日一发,热时若燔,即不发疟之日,亦觉心中发热, 
舌燥口干,脉象弦长(凡疟脉皆弦)重按甚实,知其阳明火盛也。投以大剂白虎汤,加柴胡三钱。服后顿觉心 
中清爽,翌晨疟即未发。又煎前剂之半,加生姜三钱,服之而愈。 
又尝治一人得温病,热入阳明之府,舌苔黄浓,脉象洪长,又间日一作寒热,此温而兼疟也。然其人素 
有鸦片嗜好,病虽实,而身体素虚。投以拙拟白虎加人参以麦冬代知母、山药代粳米汤,亦少加柴胡,两剂而愈。 
西人治疟,恒用金鸡纳霜,于未发疟之日,午间、晚间各服半瓦,白糖水送下。至翌晨又如此服一次,其疟即愈。 
方书谓冬冷多温病,夏热多疟疾。此言冬日过冷,人身有伏寒,至春随春阳化热,即多成温病;夏日过 
热,人身有伏暑,至秋为薄寒所激发,即多生疟疾也。丁卯季夏,暑热异常,京津一带因热而死者甚多,至秋 
果多疟疾。服西药金鸡纳霜亦可愈,而 
愈后恒屡次反复。姻家王姓少年,寄居津门,服金鸡纳霜愈疟三次后,又反复。连服前药数次,竟毫无效验。 
诊其脉,左右皆弦长有力。夫弦为疟脉,其长而有力者,显系有伏暑之热也。为开白虎汤方,重用生石膏二两, 
又加柴胡、何首乌各二钱,一剂而疟愈。恐未除根,即原方又服一剂,从此而病不反复矣。此方用白虎汤以解 
伏暑,而又加柴胡、何首乌者,凡外感之证其脉有弦象者,必兼有少阳之病,宜用柴胡清之;而外邪久在少阳, 
其经必虚,又宜用何首乌补之。二药并用,一扶正,一逐邪也。少阳与阳明并治,是以伏暑愈而疟亦随愈也。 

\chapter{治气血郁滞肢体疼痛方}
<篇名>1.升降汤
属性:治肝郁脾弱,胸胁胀满,不能饮食。宜与论肝病治法参看。 
野台参(二钱) 生黄 (二钱) 白术(二钱) 广陈皮(二钱) 川浓朴(二钱) 生鸡内金(二钱,捣细) 
知母(三钱) 生杭芍(三钱) 桂枝尖(一钱) 川芎(一钱) 生姜(二钱) 
世俗医者,动曰平肝,故遇肝郁之证,多用开破肝气之药。至遇木盛侮土,以致不能饮食者,更谓伐肝即 
可扶脾。不知人之元气,根基于肾,而萌芽于肝。凡物之萌芽,皆嫩脆易于伤损,肝既为元气萌芽之脏,而开 
破之若是,独不虑损伤元气之萌芽乎?《内经》曰“厥阴(肝经)不治,求之阳明(胃经)”,《金匮》曰“见 
肝之病,当先实脾”。故此方,惟少用桂枝、川芎以舒肝气,其余诸药,无非升脾降胃,培养中土,俾中宫气 
化敦浓,以听肝气之自理。实窃师《内经》求之阳明,与《金匮》当先实脾之奥旨耳。 
按∶“见肝之病,当先实脾”二句,从来解者,谓肝病当传脾,实之所以防其相传,如此解法固是,而实 
不知实脾,即所以理肝也。兼此二义,始能尽此二句之妙。 
一媪,年近六旬。资禀素弱,又兼家务劳心,遂致心中怔忡,肝气郁结,胸腹胀满,不能饮食,舌有黑苔, 
大便燥结,十数日一行。广延医者为治,半载无效,而羸弱支离,病势转增。后愚诊视,脉细如丝,微有弦意, 
幸至数如常,知犹可治。遂投以升降汤,为舌黑便结,加鲜地骨皮一两,数剂后,舌黑与便结渐愈,而地骨皮 
亦渐减。至十剂病愈强半,共服百剂,病愈而体转健康。 


<篇名>2.培脾舒肝汤
属性:治因肝气不舒、木郁克土,致脾胃之气不能升降,胸中满闷,常常短气。 
于术(三钱) 生黄 (三钱) 陈皮(二钱) 川浓朴(二钱) 桂枝尖(钱半) 柴胡(钱半) 
生麦冬(二钱) 生杭芍(四钱) 生姜(二钱) 
脾主升清,所以运津液上达。胃主降浊,所以运糟粕下行。白术、黄 ,为补脾胃之正药,同桂枝、柴 
胡,能助脾气之升,同陈皮、浓朴,能助胃气之降。清升浊降满闷自去,无事专理肝气,而肝气自理,况桂枝、 
柴胡与麦芽,又皆为舒肝之妙品乎。用芍药者,恐肝气上升,胆火亦随之上升,且以解黄 、桂枝之热也。用 
生姜者,取其辛散温通,能浑融肝脾之气化于无间也。 
从来方书中,麦芽皆是炒熟用之,惟陈修园谓麦芽生用,能升发肝气,可谓特识。盖人之元气,根基于 
肾,萌芽于肝,培养于脾,积贮于胸中为大气以斡旋全身。麦茅为谷之萌芽,与肝同气相求,故能入肝经,以 
条达肝气,此自然之理,无庸试验而可信其必 
然者也。然必生煮汁饮之,则气善升发,而后能遂其条达之用也。 
附录∶ 
直隶青县张××来函∶ 
族侄妇,年二十余,素性谨言,情志抑郁。因气分不舒,致四肢痉挛颤动,呼吸短促,胸中胀闷,约一昼 
夜。先延针科医治,云是鸡爪风,为刺囟门及十指尖,稍愈,旋即复作如故。其脉左部弦细,右部似有似无, 
一分钟数至百至。其两肩抬动,气逆作喘。询知其素不健壮,廉于饮食。盖肝属木而主筋,肝郁不舒则筋挛, 
肝郁恒侮其所胜,故脾土受伤而食少。遂为开培脾舒肝汤。为有逆气上干,又加生赭石细末五钱。嘱服二剂。 
痉挛即愈,气息亦平。遂去赭石,照原方又服数剂,以善其后。 


<篇名>3.金铃泻肝汤
属性:治胁下掀疼。 
川楝子(五钱,捣) 生明乳香(四钱) 生明没药(四钱) 三棱(三钱) 莪术(三钱) 甘草(一钱) 
刘河间有金铃子散(即楝子之核)与玄胡索等分,为末服之,以治心腹胁下作疼。其病因,由于热者甚效。 
诚以金铃子能引心包之火及肝胆所寄之相火下行,又佐以玄胡索以开通气血,故其疼自止也。而愚用其方,效 
者固多,而间有不效者。后拟得此方,莫不随手奏效。盖金铃子佐以玄胡索,虽能开气分之郁,而实不能化气。 
所谓化气者,无事开破,能使气之郁者,融化于无形,方中之乳香、没药是也。去玄胡索,加三棱、莪术者, 
因玄胡索性过猛烈,且其开破之力,多趋下焦,不如三棱、莪术性较和平,且善于理肝也。用甘草者,所以防 
金铃子有小毒也。此方不但治胁疼甚效,凡心腹作疼,而非寒凉者,用之皆甚效验。 
附录∶ 
直隶盐山李××来函∶ 
仲冬,刘××兄,病左胁掀疼,诸治无效,询方于弟。授以活络效灵丹方,服之不应,因延为诊视。脉象他 
部皆微弱,惟左关沉而有力。治以金铃泻肝汤,加当归数钱。服一剂,翌日降下若干绿色粘滞之物,遂豁然而 
愈。盖此汤原注明治胁下掀疼,由此知兄所拟方各有主治,方病相投,莫不神效也。 


<篇名>4.活络效灵丹
属性:治气血凝滞, 癖 瘕,心腹疼痛,腿疼臂疼,内外疮疡,一切脏腑积聚,经络湮淤。 
当归(五钱) 丹参(五钱) 生明乳香(五钱) 生明没药(五钱) 
上药四味作汤服。若为散,一剂分作四次服,温酒送下。 
腿疼加牛膝。臂疼加连翘。妇女瘀血腹疼,加生桃仁(带皮尖作散服炒用)、生五灵脂。疮红肿属阳者,加金银花、 
知母、连翘。白硬属阴者,加肉桂、鹿角胶(若恐其伪可代以鹿角霜)。疮破后生肌不速者,加生黄 、知母(但加黄 
恐失于热)、甘草。脏腑内痈,加三七(研细冲服)、牛蒡子。 
一人,年三十许。当脐忽结 瘕,自下渐长而上,其初长时稍软,数日后即硬如石,旬日长至心口。向愚 
询方,自言凌晨冒寒,得于途间,时心中有惊恐忧虑,遂觉其气结而不散。按∶此病因甚奇,然不外气血凝滞。 
为制此方,于流通气血之中,大具融化气血之力,连服十剂全消。以后用此方治内外疮疡,心腹四肢疼痛,凡 
病之由于气血凝滞者,恒多奇效。 
高××年近五旬。资禀素羸弱。一日访友邻村,饮酒谈宴,彻夜不眠,时当季冬,复清晨冒寒,步行旋里。 
行至中途,觉两腿酸麻,且出汗,不能行步,因坐凉地歇息,至家,遂觉腿痛, 
用热砖熨之疼益甚。其人素知医,遂自服发汗之药数剂,病又增剧,因服药过热,吐血数口,大便燥结,延愚 
诊视。见其仰卧屈膝,令两人各以手托其两腿,忽歌忽哭,疼楚之态万状,脉弦细,至数微数。因思此证,热 
砖熨而益疼者,逼寒内陷也;服发汗药而益疼者,因所服之药,散肌肉之寒,不能散筋骨之寒,且过汗必伤气 
血,血气伤,愈不能胜病也。遂用活络效灵丹,加京鹿角胶四钱(另炖兑服)、明天麻二钱,煎汤饮下,左腿 
遂愈。而右腿疼如故,遂复用原方,以虎骨胶易鹿角胶,右腿亦出凉气如左而愈。 
一少妇,左胁起一疮,其形长约五寸,上半在乳,下半在肋,皮色不变,按之甚硬,而微热于他处。延医 
询方,调治两月不效,且渐大于从前。后愚诊视,阅其所服诸方,有遵林屋山人治白疽方治者,有按乳痈治者。 
愚晓病家曰∶此证硬而色白者,阴也。按之微热者,阴中有阳也。统观所服诸方,有治纯阴阳之方,无治半阴 
半阳之方,勿怪其历试皆不效也。用活络效灵丹,俾作汤服之,数剂见轻,三十剂后,消无芥蒂。 
一妇人年五十许。脑后发一对口疮。询方于愚,时初拟出活络效灵丹方,即书而予之,连服十剂全愈。 
一妇人,年五十余。项后筋缩作疼,头向后仰,不能平视,腰背强直,下连膝后及足跟大筋皆疼,并牵周 
身皆有疼意。广延医者延医,所用之药,不外散风、和血、润筋、通络之品。两载无效,病转增剧,卧不能起, 
起不能坐,饮食懒进。后愚诊视,其脉数而有力,微有弦意,知其为宗筋受病。治以活络效灵丹,加生薏米八 
钱,知母、玄参、白芍各三钱,连服三十剂而愈。 
盖筋属于肝,独宗筋属胃,此证因胃腑素有燥热,致津液短少,不能荣养宗筋。夫宗筋为筋之主,故宗筋 
拘挛,而周身牵引作疼也。薏米性味冲和,善能清补脾胃,即能荣养宗筋。又加知 
母、玄参,以生津滋液,活络效灵丹,以活血舒筋,因其脉微弦,恐其木盛侮土,故又加芍药以和肝,即以扶脾胃也。 
薏米主筋急拘挛《神农本草经》原有明文。活络效灵丹中加薏米,即能随手奏效。益叹《神农本草经》 
之精当,为不可及。 
活络效灵丹,治心腹疼痛,无论因凉、因热、气郁、血郁皆效。同里有一少年,脐下疼甚剧。医者投以 
温药益甚,昼夜号呼不止。又延他医,以药下之稍轻,然仍昼夜呻吟,继又服药数剂,亦不见效。适愚自津门 
旋里,诊其脉,两尺洪实。询其得病之由,言夜晚将寝觉饥,因食冷饼一块,眠起遂疼。晓之曰,此虽由于食 
凉物,然其疼非凉疼,乃下焦先有蕴热,又为凉物所迫,其热愈结而不散也。投以活络效灵丹,加龙胆草、川 
楝子各四钱,一剂而愈。 
或问∶此证医者曾用药下之,何以其下焦之郁热,不随之俱下?答曰∶热在大肠者,其热可随降药俱下, 
然又必所用之下药为咸寒之品,若承气汤是也。今其热,原郁于奇经冲任之中,与大肠无关,冲任主血,而活 
络效灵丹诸药品,皆善入血分,通经络,故能引龙胆、楝子直入冲任,而消解其郁热。况其从前所服之下药, 
原非咸寒之品,是以从前不效,而投以此药,则随手奏效也。 
附录∶ 
直隶青县张××来函∶ 
族兄×,年三十余,素强壮无病。壬戌中秋,因在田间掘壑,劳苦过甚,自觉气力不支,即在壑中吃烟 
休息,少缓须臾又复力作。至晚归家时,途中步行,觉两腿酸木不仁。及至夜间,两腿抽疼甚剧。适生在里, 
其弟扣门求为往治。诊其脉,迟滞而细,号呼不已,气逆不顺,身冷,小溲不利。遂用活络效灵丹方, 
加白芍三钱、桂枝尖二钱、生姜三片。一剂腿疼大减,小便即利,身冷亦退。再剂,霍然全愈。 
又∶天津王媪,年五十七岁,右膝盖部发炎,红热肿疼,食减不眠。其嗣××延为诊视。至其家,闻病者呼 
号不止,口称救命。其右脉洪数有力,心悸头眩,舌苔白而腻,大便三日未行,小便赤热。此足征湿热下注。 
予以活络效灵丹,加生石膏六钱,知母、怀牛膝、生薏米冬四钱,甘草梢一钱,嘱服一剂。次日自能来寓,其 
疼减肿消,夜已成寐,尚云右臂酸疼。又即原方加青连翘、金银花、油松节各二钱,服之全愈。 


<篇名>5.活络祛寒汤
属性:治经络受寒,四肢发搐,妇女多有此证。 
生黄 (五钱) 当归(四钱) 丹参(四钱) 桂枝尖(二钱) 生杭芍(三钱) 生明乳香(四钱) 生明没 
药 
(四钱) 生姜(三钱) 
寒甚者,加干姜三钱。 
证寒在经络,不在脏腑。经络多行于肌肉之间,故用黄 之温补肌肉者为君,俾其形体壮旺,自能胜邪。 
又佐以温经络、通经络诸药品,不但能祛寒,且能散风,此所谓血活风自去也。风寒既去,血脉活泼,其搐焉 
有不止者乎? 


<篇名>6.健运汤
属性:治腿疼、臂疼因气虚者。亦治腰疼。 
生黄 (六钱) 野台参(三钱) 当归(三钱) 寸麦冬(三钱,带心) 知母(三钱) 生明乳香(三钱) 
生明没药(三钱) 莪术(一钱) 三棱(一钱) 
此方减麦冬、知母三分之一,合数剂为一剂,轧细炼蜜为丸,名健运丸,治同前证。 
从来治腿疼臂疼者,多责之风寒湿痹,或血瘀、气滞、痰涎凝滞。不知人身之气化壮旺流行,而周身痹者、 
瘀者、滞者,不治自愈,即偶有不愈,治之亦易为功也。愚临证体验以来,知元气素盛之人,得此病者极少。 
故凡遇腿疼、臂疼,历久调治不愈者,补其元气以流通之,数载沉 ,亦可随手奏效也。 


<篇名>7.振中汤
属性:治腿疼、腰疼,饮食减少者。 
于白术(六钱,炒) 当归身(二钱) 陈皮(二钱) 浓朴(钱半) 生明乳香(钱半) 生明没药(钱半) 
此方重用白术以健补脾胃,脾胃健则气化自能旁达。且白术主风寒湿痹,《神农本草经》原有明文,又辅 
以通活气血之药,不惟风寒湿痹开,而气血之痹而作疼者,亦自开也。 
一室女腿疼,几不能步,治以拙拟健运汤而愈。次年旧病复发,又兼腰疼,再服前方不效。诊其脉,右关 
甚濡弱,询其饮食减少,为制此汤,数剂,饮食加多,二十剂后,腰疼腿疼皆愈。 
一媪,年近七旬。陡然腿疼,不能行动,夜间疼不能寐。其家人迎愚调治,谓脉象有力,当是火郁作疼。 
及诊其脉,大而且弦,问其心中亦无热意。愚曰∶此脉非有火之象,其大也,乃脾胃过虚,真气外泄也。其弦 
也,乃肝胆失和,木盛侮土也。治以振中汤,加人参、白芍、山萸肉(去净核)各数钱,补脾胃之虚,即以抑 
肝胆之盛,数剂而愈。 


<篇名>8.曲直汤
属性:治肝虚腿疼,左部脉微弱者。 
萸肉(一两,去净核) 知母(六钱) 生明乳香(三钱) 生明没药(三钱) 当归(三钱) 丹参(三钱) 
服药数剂后,左脉仍不起者,可加续断三钱,或更加生黄 三钱,以助气分亦可。觉凉者,可减知母。 
脾虚可令人腿疼,前方已详其理,深于医学人大抵皆能知之。至肝虚可令人腿疼,方书罕言,即深于医学 
者,亦恒不知。曾治一人,年三十许,当大怒之后,渐觉腿疼,日甚一日,两月后,卧床不能转侧。医者因其 
得之恼怒之余,皆用舒肝理气之药,病转加剧。后愚诊视,其左脉甚微弱,自言凡疼甚之处皆热。因恍悟《内 
经》谓“过怒则伤肝”,所谓伤肝者,乃伤肝经之气血,非必郁肝经之气血也,气血伤,则虚弱随之,故其脉 
象如斯也。其所以腿疼且觉热者,因肝主疏泄,中藏相火(相火生于命门寄于肝胆),肝虚不能疏泄,相火即 
不能逍遥流行于周身,以致郁于经络之间,与气血凝滞,而作热作疼,所以热剧之处,疼亦剧也。为制此汤, 
以萸肉补肝,以知母泻热,更以当归、乳香诸流通血气之药佐之,连服十剂,热愈疼止,步履如常。 
安东友人刘××,年五十许。其左臂常觉发热,且有酸软之意。医者屡次投以凉剂,发热如故,转觉脾胃 
消化力减少。后愚诊之,右脉和平如常,左脉微弱,较差于右脉一倍。询其心中,不觉凉热。知其肝木之气虚 
弱,不能条畅敷荣,其中所寄之相火,郁于左臂之经络,而作热也。遂治以曲直汤,加生黄 八钱,佐萸肉以 
壮旺肝气,赤芍药三钱,佐当归、丹参诸药以流通经络,服两剂,左脉即见起,又服十剂全愈。 
奉天王××,年四十余,两胁下连腿作疼,其疼剧之时,有如锥刺,且尿道艰涩,滴沥不能成溜,每小便 
一次,须多半点钟,其脉亦右部如常,左部微弱。亦投以曲直汤,加生黄 八钱,续断三钱,一剂其疼减半, 
小便亦觉顺利。再诊之,左脉较前有力。又按原方略为加减,连服二十余剂,胁与腿之疼皆愈,小便亦通利如 
常。盖两胁为肝之部位,肝气壮旺上达,自不下郁而作 
疼。至其小便亦通利者,因肾为二便之关,肝气既旺,自能为肾行气也(古方书有肝行肾之气之语)。 

\chapter{治女科方}
<篇名>1.玉烛汤
属性:治妇女寒热往来或先寒后热,汗出热解,或月事不调,经水短少。 
生黄 (五钱) 生地黄(六钱) 玄参(四钱) 知母(四钱) 当归(三钱) 香附(三钱,醋炒) 
柴胡(一钱五分) 甘草(一钱五分) 
汗多者,以茵陈易柴胡,再加萸肉数钱。热多者,加生杭芍数钱。寒多者,加生姜数钱。 
妇女多寒热往来之证,而方书论者不一说。有谓阳分虚则头午寒,阴分虚则过午热者。夫午前阳盛,午 
后阳衰而阴又浸盛。当其盛时,虚者可以暂实。何以其时所现之病状,转与时成反比例也?有谓病在少阳则寒 
热往来,犹少阳外感之邪,与太阳并则寒,与阳明并则热者。而内伤之病,原无外邪。又何者与太阳、阳明并 
作寒热也?有谓肝虚则乍热乍寒者。斯说也,愚曾验过。遵《神农本草经》山茱萸主寒热之旨,单重用山萸肉 
(去净核)二两煎汤,服之立愈(验案在来复汤下)。然此乃肝木虚极,内风将动之候,又不可以盖寻常寒热 
也。盖人身之气化,原与时序之气化,息息相通。一日之午前,犹一岁之有春夏。而人身之阳气,即感之发动, 
以敷布于周身。妇女性多忧思,以致脏腑、经络多有郁结闭塞之处,阻遏阳气不能外达,或转因发动而内陷, 
或发动不遂,其发动排挤经络愈加闭塞。于是周身之寒作矣。迨阳气蓄极,终当愤发。而其愤发之机与抑遏之 
力,相激相荡于脏腑、经络之间,热又由兹 
而生。此前午之寒,所以变后午之热也。黄 为气分之主药,能补气更能升气。辅以柴胡之轩举,香附之宣通, 
阳气之抑遏者,皆畅发矣。然血随气行,气郁则血必瘀,故寒热往来者,其月事恒多不调,经血恒多虚损。用 
当归以调之,地黄以补之,知母、元参与甘草甘苦化阴以助之,则经血得其养矣。况地黄、知母诸凉药与黄 
温热之性相济,又为燮理阴阳、调和寒热之妙品乎。至方书有所谓日晡发热者,日晡者,申时也,足少阴肾经 
主令之候也。其人或肾经阴虚,至此而肾经之火乘时而动,亦可治以此汤。将黄 减半,地黄改用一两。有经 
闭结为 瘕,阻塞气化作寒热者,可用后理冲汤。有胸中大气下陷作寒热者,其人常觉呼吸短气,宜用拙拟升 
陷汤,方后治验之案,可以参观。 
【附方】西人铁锈鸡纳丸,治妇女经血不调,身体羸弱咳喘,或时作寒热甚效。方用铁锈、没药(忌火) 
各一钱,金鸡纳霜、花椒各五分,共为细末,炼蜜为丸六十粒。每服三粒至五粒。 
铁锈乃铁与养气化合而成。人身之血得养气而赤。铁锈中含养气,而又色赤似血,且嗅之兼有血腥之气, 
故能荣养血分,流通经脉。且人之血中,实有铁锈,以铁锈补血更有以铁补铁之妙也。金鸡纳霜有治疟之功用。 
此方中亦用之者,为其善治贫血,且又能入手、足少阳之经,以调和寒热也。又佐以花椒者,恐金鸡纳霜之性, 
偏于寒凉,而以辛热济之,使归于和平也。 
东亚人有中将汤,以调妇女经脉,恒有效验。门人高××曾开其方相寄,药品下未有分量。愚为酌定其分 
量,用之甚有功效。今将其方开列于下,以备选用。 
延胡索醋炒三钱、当归六钱、官桂二钱、甘草二钱、丁香二钱、山楂核醋炒三钱、郁金醋炒二钱、沙参 
四钱、续断酒炒三钱、肉蔻赤石脂炒三钱(去石脂不用)、苦参三钱、怀牛膝三钱,共十二味,轧作粗渣,分 
三剂。每用一剂,开水浸盖碗中约半点 
钟,将其汤饮下。如此浸服二次至第三次用水煎服。日用一剂,数剂经脉自调。此方中凉热、补破、涩滑之药 
皆有,愚所酌分量,俾其力亦适相当,故凡妇女经脉不调证,皆可服之,而以治白带证尤效。 


<篇名>2.理冲汤
属性:治妇女经闭不行或产后恶露不尽,结为 瘕,以致阴虚作热,阳虚作冷,食少劳嗽,虚证沓来。服此汤十余 
剂后,虚证自退,三十剂后,瘀血可尽消。亦治室女月闭血枯。并治男子劳瘵,一切脏腑 瘕、积聚、气郁、 
脾弱、满闷、痞胀、不能饮食。 
生黄 (三钱) 党参(二钱) 于术(二钱) 生山药(五钱) 天花粉(四钱) 知母(四钱) 
三棱(三钱) 莪术(三钱) 生鸡内金(三钱,黄者) 
用水三盅,煎至将成,加好醋少许,滚数沸服。 
服之觉闷者,减去于术。觉气弱者,减三棱、莪术各一钱。泻者,以白芍代知母,于术改用四钱。热者, 
加生地、天冬各数钱。凉者,知母、花粉各减半,或皆不用。凉甚者,加肉桂(捣细冲服)、乌附子各二钱。 
瘀血坚甚者,加生水蛭(不用炙)二钱。若其人坚壮无他病,惟用以消 瘕积聚者,宜去山药。室女与妇人 
未产育者,若用此方,三棱、莪术宜斟酌少用,减知母之半,加生地黄数钱,以濡血分之枯。若其人血分虽瘀, 
而未见 瘕或月信犹未闭者,虽在已产育之妇人,亦少用三棱、莪术。若病患身体羸弱,脉象虚数者,去三棱、 
莪术,将鸡内金改用四钱,因此药能化瘀血,又不伤气分也。迨气血渐壮,瘀血未尽消者,再用三棱、莪术未 
晚。若男子劳瘵,三棱、莪术亦宜少用或用鸡内金代之亦可。初拟此方时,原专治产后瘀血成 瘕,后以治室女月 
闭血枯亦效,又间用以治男子劳瘵亦效验,大有开胃进食,扶羸起衰之功。《内经》有四乌 鱼骨一茹芦丸, 
原是男女并治,为调血补虚之良方。此方窃师《内经》之意也。 
从来医者调气行血,习用香附,而不习用三棱、莪术。盖以其能破 瘕,遂疑其过于猛烈。而不知能破症 
瘕者,三棱、莪术之良能,非二药之性烈于香附也。愚精心考验多年,凡习用之药,皆确知其性情能力。若论 
耗散气血,香附犹甚于三棱、莪术。若论消磨 瘕,十倍香附亦不及三棱、莪术也。且此方中,用三棱、莪术以 
消冲中瘀血,而即用参、 诸药,以保护气血,则瘀血去而气血不至伤损。且参、 能补气,得三棱、莪术以 
流通之,则补而不滞,而元气愈旺。元气既旺,愈能鼓舞三棱、莪术之力以消 瘕,此其所以效也。 
人之脏腑,一气贯通,若营垒联系,互为犄角。一处受攻,则他处可为之救应。故用药攻病,宜确审病根 
结聚之处,用对证之药一二味,专攻其处。即其处气血偶有伤损,他脏腑气血犹可为之输将贯注,亦犹相连营 
垒之相救应也。又加补药以为之佐使,是以邪去正气无伤损。世俗医者,不知此理,见有专确攻病之方,若拙 
拟理冲汤者,初不审方中用意何如,但见方中有三棱、莪术,即望而生畏,不敢试用。自流俗观之,亦似慎重, 
及观其临证调方,漫不知病根结于何处,惟是混开混破。恒集若香附、木香、陈皮、砂仁、枳壳、浓朴、延胡、 
灵脂诸药,或十余味或数十味为一方。服之令人脏腑之气皆乱,常有病本可治,服此等药数十剂而竟至不治者。 
更或见有浮火虚热,而加芩、栀、蒌实之属,则开破与寒凉并用,虽脾胃坚壮者,亦断不能久服,此其贻害尤甚也。 
一妇人,年三十余。 瘕起于少腹,渐长而上。其当年长者稍软,隔年即硬如石。七年之间,上至心口, 
旁塞两肋,饮食减 
少,时觉昏愦,剧时昏睡一昼夜,不饮不食,屡次服药竟分毫无效。后愚为诊视,脉虽虚弱,至数不数,许为 
治愈,授以此方。病患自揣其病,断无可治之理,竟置不服。次年病益进,昏睡四日不醒。愚用药救醒之,遂 
恳切告之曰∶去岁若用愚方,病愈已久,何至危困若斯。然此病尚可为,甚勿再迟延也,仍为开前方。病患喜, 
信愚言,连服三十余剂,磊块皆消。惟最初所结之病根,大如核桃之巨者尚在。又加生水蛭(不宜炙)一钱, 
服数剂全愈。 
一妇人,年二十余。 瘕结于上脘,其大如橘,按之甚硬,时时上攻作疼,妨碍饮食。医者皆以为不可 
消。后愚诊视,治以此汤,连服四十余剂,消无芥蒂。 
一媪,年六旬。气弱而且郁,心腹满闷,不能饮食,一日所进谷食,不过两许,如此已月余矣。愚诊视 
之,其脉甚微细,犹喜至数调匀,知其可治。遂用此汤,将三棱、莪术各减一钱,连服数剂,即能进饮食。又 
服数剂,病遂全愈。 
奉天孙姓妇,年四十许。自幼时有 瘕结于下脘,历二十余年。 瘕之积,竟至满腹,常常作疼,心中 
怔忡,不能饮食,求为延医。因思此证,久而且剧,非轻剂所能疗。幸脉有根柢,犹可调治。遂投以理冲汤, 
加水蛭三钱。恐开破之力太过,参、 又各加一钱,又加天冬三钱,以解参、 之热。数剂后,遂能进食。服 
至四十余剂,下瘀积若干, 瘕消有强半。因有事还籍,药遂停止。阅一载,腹中之积,又将复旧,复来院求 
为延医。仍照前方加减,俾其补破凉热之间,与病体适宜。仍服四十余剂,积下数块。又继服三十余剂,瘀积 
大下。其中或片或块且有膜甚浓,若胞形。此时身体觉弱,而腹中甚松畅。恐瘀犹未净,又调以补正活血之药, 
以善其后。 
王××妻,来院求为治 瘕。自言瘀积十九年矣,满腹皆系 
硬块。亦治以理冲汤,为其平素气虚,将方中参、 加重,三棱、莪术减半。服数剂,饮食增加,将三棱、莪 
术渐增至原定分量。又服数剂,气力较壮,又加水蛭二钱、樗鸡(俗名红娘)十枚。又服二十余剂,届行经之 
期,随经下紫黑血块若干,病愈其半。又继服三十剂,届经期,瘀血遂大下,满腹积块皆消。又俾服生新化瘀 
之药,以善其后。 
一少年,因治吐血,服药失宜, 癖结于少腹(在女子为 瘕在男子为 癖)大如锦瓜。按 
之甚坚硬,其上相连有如瓜蔓一条,斜冲心口,饮食减少,形体羸弱。其脉微细稍数。治以此汤,服十余剂 
癖全消。 
附录∶ 
广西柳州宾××来函∶ 
一妇人,十七岁,自二七出嫁,未见行经。先因腹、胁作疼求为延医,投以活络效灵丹立愈。继欲调其月 
事,投以理冲汤三剂,月经亦通,三日未止。犹恐瘀血未化,改用王清任少腹逐瘀汤,亦三剂,其人从此月事 
调顺,身体强壮矣。 


<篇名>3.理冲丸
属性:治同前证。 
水蛭(一两,不用炙) 生黄 (一两半) 生三棱(五钱) 生莪术(五钱) 当归(六钱) 知母(六钱) 
生桃仁(六钱,带皮尖) 
上药七味,共为细末,炼蜜为丸桐子大,开水送服二钱,早晚各一次。 
仲景抵当汤、大黄 虫丸、百劳丸,皆用水蛭,而后世畏其性猛,鲜有用者,是未知水蛭之性也。《神农 
本草经》曰∶水蛭气味咸平无毒,主逐恶血、瘀血、月闭,破 瘕、积聚、无子、利 
水道。徐灵胎注云∶凡人身瘀血方阻,尚有生气者易治,阻之久则生气全消而难治。盖血既离经,与正气全不 
相属,投之轻药,则拒而不纳,药过峻,又转能伤未败之血,故治之极难。水蛭最善食人之血,而性又迟缓善 
入。迟缓则生血不伤,善入则坚积易破,借其力以消既久之滞,自有利而无害也。观《神农本草经》之文与徐 
氏之注,则水蛭功用之妙,为何如哉!特是徐氏所谓迟缓善入者,人多不解其理。盖水蛭行于水中,原甚迟缓。 
其在生血之中,犹水中也,故生血不伤也。着人肌肉,即紧贴善入。其遇坚积之处,犹肌肉也,故坚积易消也。 
方中桃仁不去皮尖者,以其皮赤能入血分,尖乃生发之机,又善通气分。杨玉衡《寒温条辨》曾有斯说。 
愚疑其有毒,未敢遽信。遂将带皮生桃仁,嚼服一钱,心中安然,以后始敢连皮尖用之。至于不炒用,而生用 
者,凡果中之仁,皆含生发之气,原可借之以流通既败之血也。《神农本草经百种录》注曰∶桃得三月春和之 
气以生,而花鲜明似血,故凡血瘀血枯之疾,不能调和畅达者,此能入于其中而和之散之。然其生血之功少, 
而去瘀之功多者,盖桃核本非血类,实不能有所补益。若 瘕皆已败之血,非生气不能流通,桃之生气在于仁, 
而味苦又能开泄,故能逐旧而不伤新也。夫既借其生气以流通气血,不宜炒用可知也。若入丸剂,蒸熟用之亦可。 
【附方】秘传治女子干病方,用红HT 螺(榆树内红虫大如蚕)二个,樗树(此树如椿而味臭俗名臭椿)荚 
二个,人指甲全的,壮年男子发三根。用树荚夹HT 螺、指甲以发缠之,将发面馒头如大橘者一个,开一孔,去 
中瓤俾可容药。纳药其中,仍将外皮原开下者杜孔上,木炭火煨存性为细末,用黄酒半斤炖开,兑童便半茶盅 
送服。忌腥冷、惊恐、恼怒。此方用过数次皆验,瘀血开时必吐衄又兼下血,不必惊恐,移时自愈。以治经水 
一次未来者尤效。 


<篇名>4.安冲汤
属性:治妇女经水行时多而且久,过期不止或不时漏下。 
白术(六钱,炒) 生黄 (六钱) 生龙骨(六钱,捣细) 生牡蛎(六钱,捣细) 大生地(六钱) 
生杭芍(三钱) 海螵蛸(四钱,捣细) 茜草(三钱) 川续断(四钱) 
友人刘××其长子妇,经水行时,多而且久,淋漓八九日始断。数日又复如故。医治月余,初稍见轻,继又 
不愈。延愚诊视,观所服方,即此安冲汤,去茜草、螵蛸。遂仍将二药加入,一剂即愈。又服一剂,永不反复。 
刘××疑而问曰∶茜草、螵蛸,治此证如此效验,前医何为去之?答曰∶彼但知茜草、螵蛸能通经血,而未见 
《内经》用此二药雀卵为丸,鲍鱼汤送下,治伤肝之病,时时前后血也。故于经血过多之证,即不敢用。不知 
二药大能固涩下焦,为治崩之主药也。 
一妇人,年三十余。夫妻反目,恼怒之余,经行不止,且又甚多。医者用十灰散加减,连服四剂不效。后 
愚诊视,其右脉弱而且濡。询其饮食多寡,言分毫不敢多食,多即泄泻。遂投以此汤,去黄 ,将白术改用一 
两。一剂血止,而泻亦愈。又服一剂,以善其后。 
一妇人,年二十余。小产后数日,恶露已尽,至七八日,忽又下血。延医服药,二十余日不止。诊其脉, 
洪滑有力,心中热而且渴。疑其夹杂外感,询之身不觉热,又疑其血热妄行,遂将方中生地改用一两,又加知 
母一两,服后血不止,而热渴亦如故。因思此证,实兼外感无疑。遂改用白虎加人参汤以山药代粳米。方中石 
膏重用生者三两。煎汤两盅,分两次温饮下。外感之火遂消,血亦见止。仍与安冲汤,一剂遂全愈。又服数剂, 
以善其后。 
附录∶ 
直隶青县张××来函∶ 
王氏妇,年十九岁,因殇子过痛,肝气不畅,经水行时多而且久,或不时漏下。前服逍遥、归脾等药, 
皆无效。诊其脉,左关尺及右尺皆浮弦,一息五至强。口干不思食,腰疼无力,乃血亏而有热也。遵将女科方 
安冲汤去 、术,加麦冬、霍石斛、香附米,俾服之。二剂血止,六剂后食量增加,口干腰疼皆愈。继将汤剂 
制作丸药,徐徐服之,月事亦从此调矣。 
直隶盐山孙××来函∶ 
一九二四年七月,友人张××之母,因筹办娶儿媳事劳心过度,小便下血不止,其血之来沥沥有声,请为 
诊视,举止不定,气息微弱,右脉弦细,左脉弦硬。为开安冲汤,服后稍愈。翌日晨起,忽然昏迷,其家人甚 
恐,又请诊视。其脉尚和平,知其昏迷系黄 升补之力稍过,遂仍用原方,加赭石八钱,一剂而愈。 
家族婶有下血证,医治十余年,时愈时发,终未除根。一九二六年六月,病又作,请为诊视。治以《傅 
青主女科》治老妇血崩方,遵师训加生地黄一两,一服即愈。七月,病又反复。治以安冲汤方,以其心中觉凉, 
加干姜二钱,一剂病又愈。 
斯年初秋,李姓之女,年十七岁,下血不止,面唇皆白,六脉细数。治以安冲汤,重用山萸肉,三剂而愈。 


<篇名>5.固冲汤
属性:治妇女血崩。 
白术(一两,炒) 生黄 (六钱) 龙骨(八钱, 捣细) 牡蛎(八钱, 捣细) 萸肉(八钱,去净核) 
生杭芍(四钱) 海螵蛸(四钱,捣细) 茜草(三钱) 棕边炭(二钱) 五倍子(五分,轧细药汁送服) 
脉象热者加大生地一两;凉者加乌附子二钱;大怒之后,因肝气冲激血崩者,加柴胡二钱。若服两剂不 
愈,去棕边炭,加真阿胶五钱,另炖同服。服药觉热者宜酌加生地。 
从前之方,龙骨、牡蛎皆生用,其理已详于理冲丸下。此方独用 者,因 之,则收涩之力较大,欲借 
之以收一时之功也。 
一妇人,年三十余。陡然下血,两日不止。及愚诊视,已昏愦不语,周身皆凉,其脉微弱而迟。知其气 
血将脱,而元阳亦脱也。遂急用此汤,去白芍,加野台参八钱、乌附子三钱。一剂血止,周身皆热,精神亦复。 
仍将白芍加入,再服一剂,以善其后。 
子××曾治一妇人,年四十许。骤得下血证甚剧,半日之间,即气息奄奄,不省人事。其脉右寸关微见, 
如水上浮麻,不分至数,左部脉皆不见。急用生黄 一两,大火煎数沸灌之,六部脉皆出。然微细异常,血仍 
不止。观其形状,呼气不能外出,又时有欲大便之意,知其为大气下陷也。遂为开固冲汤方,将方中黄 改用 
一两。早十一点钟,将药服下,至晚三点钟,即愈如平时(后子××在京,又治一血崩证,先用固冲汤不效,加 
柴胡二钱,一剂即愈,足见柴胡升提之力,可为治崩要药)。 
或问∶血崩之证,多有因其人暴怒,肝气郁结,不能上达,而转下冲肾关,致经血随之下注者,故其病俗 
亦名之曰气冲。兹方中多用涩补之品,独不虑于肝气郁者,有妨碍乎?答曰∶此证虽有因暴怒气冲而得者,然 
当其血大下之后,血脱而气亦随之下脱,则肝气之郁者,转可因之而开。且病急则治其标,此证诚至危急之病 
也。若其证初得,且不甚剧,又实系肝气下冲者,亦可用升肝理气之药为主,而以收补下元之药辅之也。 
【附方】《傅青主女科》,有治老妇血崩方,试之甚效。其方用生黄 一两,当归一两(酒洗),桑叶 
十四片,三七末三钱(药汁送下)水煎服,二剂血止,四剂不再发。若觉热者,用此方宜加 
生地两许。 
又∶诸城友人王××,传一治血崩秘方。用青莱菔生捣取汁,加白糖数匙,微火炖温,陆续饮至三大盅,必愈。 
又∶西药中有麦角,原霉麦上所生之小角,其性最善收摄血管,能治一切失血之证,而对于下血者用之 
尤效。角之最大者,长近寸许,以一枚和乳糖(无乳糖可代以白蔗糖)研细,可作两次服。愚常用之与止血之 
药并服,恒有捷效。 
附∶治女子血崩有两种草药 
一种为宿根之草,一根恒生数茎,高不盈尺,叶似地肤,微宽,浓则加倍,其色绿而微带苍色,孟夏开 
小白花,结实如杜梨,色如其叶,老而微黄,多生于宅畔路旁板硬之地,俗呼为HT 牛蛋,又名臭科子,然实未 
有臭味。初不知其可入药也。戊辰孟夏,愚有事回籍。有南关王氏妇,患血崩,服药不效。有人教用此草连根 
实锉碎,煮汤饮之,其病顿愈。后愚回津言及此方,门生李××谓∶“此方余素知之,若加黑豆一小握,用水、 
酒各半煎汤,则更效矣。” 
一种为当年种生之草,棵高尺余,叶圆而有尖,色深绿,季夏开小白花,五出,黄蕊,结实大如五味,状 
若小茄,嫩则绿,熟则红,老则紫黑,中含甜浆可食,俗名野茄子,有山之处呼为山茄子。奉省医者多采此草 
阴干备用,若遇血崩时,将其梗叶实共切碎煎汤服之立愈。在津曾与友人张××言及此草,张××谓,此即《本 
草纲目》之龙葵,一名天茄子,一名老鸦睛草者是也。而愚查《本草纲目》龙葵,言治吐血不止,未尝言治血 
崩。然治吐血之药,恒兼能治下血,若三七、茜草诸药是明征也。以遍地皆有之草,而能治如此重病,洵堪珍哉。 
附录∶ 
直隶青县张××来函∶ 
族姊适徐姓,年三十余。有妊流产,已旬日矣。忽然下血甚多,头晕腹胀,脉小无力。知为冲脉滑脱之征 
证,予以固冲汤,加柴胡钱半,归身二钱。服药三剂即止。俾继服坤顺至宝丹以善其后。 
直隶盐山李××来函∶ 
天津赵××妻,年四十余岁,行经过期不止,诸治不效,延弟诊视。见两部之脉皆微细无力,为开固冲汤原 
方予之,服数剂即全收功。因思如此年岁,血分又如此受伤,谅从此断生育矣。不意年余又产一子,安然无恙。 
盖因固冲汤止血兼有补血之功也。 
又∶天津张××妻,年二十四岁,因小产后血不止者绵延月余,屡经医治无效。诊其脉象,微细而数,为开 
固冲汤方,因其脉数,加生地一两。服药后,病虽见轻,而不见大功。反复思索,莫得其故。细询其药价过贱, 
忽忆人言此地药局所鬻黄 ,有真有假,今此方无显著之功效,或其黄 过劣也。改用口黄 ,连服两剂全愈。 
由斯知药物必须地道真正方效也。 


<篇名>6.温冲汤
属性:治妇人血海虚寒不育。 
生山药(八钱) 当归身(四钱) 乌附子(二钱) 肉桂(二钱,去粗皮后入) 补骨脂(三钱,炒捣) 
小茴香(二钱,炒) 核桃仁(二钱) 紫石英(八钱, 研) 真鹿角胶(二钱,另炖,同服,若恐其伪可 
代以鹿角霜三钱) 
人之血海,其名曰冲。在血室之两旁,与血室相通。上隶于胃阳明经,下连于肾少阴经。有任脉以为之担 
任,督脉为之督 
摄,带脉为之约束。阳维、阴维、阳跷、阴跷,为之拥护,共为奇经八脉。此八脉与血室,男女皆有。在男子 
则冲与血室为化精之所,在女子则冲与血室实为受胎之处。《内经》上古通天论所谓“太冲脉盛,月事以时下, 
故有子”者是也。是以女子不育,多责之冲脉。郁者理之,虚者补之,风袭者祛之,湿胜者渗之,气化不固者 
固摄之,阴阳偏胜者调剂之。冲脉无病,未有不生育者。而愚临证实验以来,凡其人素无他病,而竟不育者, 
大抵因相火虚衰,以致冲不温暖者居多。因为制温冲汤一方。其人若平素畏坐凉处,畏食凉物,经脉调和,而 
艰于生育者,即与以此汤服之。或十剂或数十剂,遂能生育者多矣。 
一妇人,自二十出嫁,至三十未育子女。其夫商治于愚。因细询其性质禀赋,言生平最畏寒凉,热时亦 
不敢食瓜果。至经脉则大致调和,偶或后期两三日。知其下焦虚寒,因思《神农本草经》谓紫石英“气味甘温, 
治女子风寒在子宫,绝孕十年无子”。遂为拟此汤,方中重用紫石英六钱,取其性温质重,能引诸药直达于冲 
中,而温暖之。服药三十余剂,而畏凉之病除。后数月遂孕,连生子女。益信《神农本草经》所谓治十年无子 
者,诚不误也。 


<篇名>7.清带汤
属性:治妇女赤白带下。 
生山药(一两) 生龙骨(六钱,捣细) 生牡蛎(六钱,捣细)海螵蛸(四钱,去净甲捣) 茜草(三钱) 
单赤带,加白芍、苦参各二钱;单白带,加鹿角霜、白术各三钱。 
带下为冲任之证。而名谓带者,盖以奇经带脉,原主合同束诸脉,冲任有滑脱之疾,责在带脉不能约束, 
故名为带也。然其病 
非仅滑脱也,若滞下然,滑脱之中,实兼有瘀滞。其所瘀滞者,不外气血,而实有因寒因热之不同。此方用龙 
骨、牡蛎以固脱,用茜草、海螵蛸以化滞,更用生山药以滋真阴固元气。至临证时,遇有因寒者,加温热之药, 
因热者,加寒凉之药,此方中意也。而愚拟此方,则又别有会心也。尝考《神农本草经》龙骨善开 瘕,牡蛎 
善消 ,是二药为收涩之品,而兼具开通之力也。乌 鱼骨即海螵蛸,茹芦即茜草,是二药为开通之品,而 
实具收涩之力也。四药汇集成方,其能开通者,兼能收涩,能收涩者,兼能开通,相助为理,相得益彰。 
一妇人,年二十余,患白带甚剧,医治年余不愈。后愚诊视,脉甚微弱。自言下焦凉甚,遂用此方,加干 
姜六钱,鹿角霜三钱,连服十剂全愈。 
又∶一媪年六旬。患赤、白带下,而赤带多于白带,亦医治年余不愈。诊其脉甚洪滑,自言心热头昏,时 
觉眩晕,已半载未起床矣。遂用此方,加白芍六钱,数剂白带不见,而赤带如故,心热、头眩晕亦如故。又加苦 
参、龙胆草、白头翁各数钱。连服七八剂,赤带亦愈,而诸疾亦遂全愈。自拟此方以来,用治带下,愈者不可 
胜数。而独载此两则者,诚以二证病因,寒热悬殊。且年少者用此方,反加大热之药,年老者用此方,反加苦 
寒之药。欲临证者,当知审证用药,不可拘于年岁之老少也。 
按∶白头翁不但治因热之带证甚效也。剖取其鲜根,以治血淋、溺血与大便下血之因热而得者甚效,诚良 
药也。是以仲景治厥阴热痢有白头翁汤也。 
带证,若服此汤未能除根者,可用此汤送服秘真丹一钱。 
带下似滞下之说,愚向持此论。后观西法,亦谓大肠病则流白痢,子宫病则流白带,其理相同。法用儿茶、 
白矾、石榴皮、没石子等水洗之。若此证之剧者,兼用其外治之法亦可。又∶其 
内治白带法,用没石子一两捣烂,水一斤半,煎至一斤,每温服一两,日三次。或研细作粉,每服五分,日二 
次亦可。又可单以之熬水洗之,或用注射器注射之。按∶没石子味苦而涩,苦则能开,涩则能敛,一药而具此 
两长,原与拙拟清带汤之意相合。且其收敛之力最胜,凡下焦滑脱之疾,或大便滑泻、或小便不禁、或男子遗 
精、或女子崩漏,用之皆效验。今之医者,多忽不知用惜哉。又东人中将汤,治白带亦甚效。玉烛汤下,载有 
其方,可采用。若以治赤带,方中官桂、丁香,宜斟酌少用,苦参宜多用。 
赤白二带,赤者多热,白者多凉。而辨其凉热,又不可尽在赤白也,宜细询其自觉或凉或热,参以脉之 
或迟或数,有力无力,则凉热可辨矣。治法宜用收涩之品,而以化瘀通滞之药佐之。清带汤,证偏热者,加生 
杭芍、生地黄;热甚者,加苦参、黄柏,或兼用防腐之药,若金银花、旱三七、鸦胆子仁皆可酌用;证偏凉者, 
加白术、鹿角胶;凉甚者,加干姜、桂附、小茴香。 
近阅《杭州医报》,载有俗传治白带便方,用绿豆芽连头根三斤,洗净,加水两大碗,煎透去渣,加生姜 
汁三两、黄蔗糖四两,慢火收膏,每晨开水冲服。约十二日服一料。服至两料必愈。按∶此方用之数次,颇有效验。 


<篇名>8.加味麦门冬汤
属性:治妇女倒经。 
干寸冬(五钱,带心) 野台参(四钱) 清半夏(三钱) 生山药(四钱,以代粳米) 生杭芍(三钱) 
丹参(三钱) 甘草(二钱) 生桃仁(二钱,带皮尖捣) 大枣(三枚,捭开) 
妇女倒经之证,陈修园《女科要旨》借用《金匮》麦门冬 
汤,可谓特识。然其方原治“火逆上气,咽喉不利”。今用以治倒经,必略为加减,而后乃与病证吻合也。 
或问,《金匮》麦门冬汤所主之病,与妇人倒经之病迥别,何以能借用之而有效验?答曰∶冲为血海, 
居少腹之两旁。其脉上隶阳明,下连少阴。少阴肾虚,其气化不能闭藏以收摄冲气,则冲气易于上干。阳明胃 
虚,其气化不能下行以镇安冲气,则冲气亦易于上干。冲中之气既上干,冲中之血自随之上逆,此倒经所由来 
也。麦门冬汤,于大补中气以生津液药中,用半夏一味,以降胃安冲,且以山药代粳米,以补肾敛冲,于是冲 
中之气安其故宅,冲中之血,自不上逆,而循其故道矣。特是经脉所以上行者,固多因冲气之上干,实亦下行 
之路,有所壅塞。观其每至下行之期,而后上行可知也。故又加芍药、丹参、桃仁以开其下行之路,使至期下 
行,毫无滞碍。是以其方非为治倒经而设,而略为加减,即以治倒经甚效,愈以叹经方之函盖无穷也。 
用此方治倒经大抵皆效,而间有不效者,以其兼他证也。曾治一室女,倒经年余不愈,其脉象微弱。投 
以此汤,服药后甚觉短气。再诊其脉,微弱益甚。自言素有短气之病,今则益加重耳。恍悟其胸中大气,必然 
下陷,故不任半夏之降也。遂改用拙拟升陷汤,连服十剂。短气愈,而倒经之病亦愈。 
一少妇,倒经半载不愈。诊其脉微弱而迟,两寸不起,呼吸自觉短气,知其亦胸中大气下陷。亦投以升 
陷汤,连服数剂,短气即愈。身体较前强壮,即停药不服。其月经水即顺,逾十月举男矣。 
或问,倒经之证,既由于冲气胃气上逆,大气下陷者,其气化升降之机正与之反对,何亦病倒经乎?答曰∶ 
此理甚微奥,人之大气,原能斡旋全身,为诸气之纲领。故大气常充满于胸中,自能运转胃气使之下降,镇摄 
冲气使不上冲。大气一陷,纲领不 
振,诸气之条贯多紊乱,此乃自然之理也。是知冲气胃气之逆,非必由于大气下陷,而大气下陷者,实可致冲 
胃气逆也。致病之因既不同,用药者岂可胶柱鼓瑟哉。 


<篇名>9.寿胎丸
属性:治滑胎。 
菟丝子(四两,炒炖) 桑寄生(二两) 川续断(二两) 真阿胶(二两) 
上药将前三味轧细,水化阿胶和为丸一分重(干足一分)。每服二十丸,开水送下,日再服。气虚者加人 
参二两,大气陷者加生黄 三两,食少者加炒白术二两,凉者加炒补骨脂二两,热者加生地二两。 
胎在母腹,若果善吸其母之气化,自无下坠之虞。且男女生育,皆赖肾脏作强。菟丝大能补肾,肾旺自 
能荫胎也。寄生能养血、强筋骨,大能使胎气强壮,故《神农本草经》载其能安胎。续断亦补肾之药。阿胶系 
驴皮所熬,最善伏藏血脉,滋阴补肾,故《神农本草经》亦载其能安胎也。至若气虚者,加人参以补气。大气 
陷者,加黄 以升补大气。饮食减少者,加白术以健补脾胃。凉者,加补骨脂以助肾中之阳(补骨脂善保胎修 
园曾详论之)。热者,加生地黄以滋肾中之阴。临时斟酌适宜,用之无不效者。 
此方乃思患预防之法,非救急之法。若胎气已动,或至下血者,又另有急救之方。曾治一少妇,其初次 
有妊,五六月而坠。后又有妊,六七月间,忽胎动下血,急投以生黄 、生地黄各二两,白术、山萸肉(去净核)、 
龙骨( 捣)、牡蛎( 捣)各一两,煎汤一大碗,顿服之,胎气遂安。将药减半,又服一剂。后举一男,强 
壮无恙。 
流产为妇人恒有之病,而方书所载保胎之方,未有用之必效者。诚以保胎所用之药,当注重于胎,以变 
化胎之性情气质,使 
之善吸其母之气化以自养,自无流产之虞。若但补助妊妇,使其气血壮旺固摄,以为母强自能荫子,此又非熟 
筹完全也。是以愚临证考验以来,见有屡次流产者,其人恒身体强壮,分毫无病;而身体软弱者,恐生育多则 
身体愈弱,欲其流产,而偏不流产。于以知∶或流产,或不流产,不尽关于妊妇身体之强弱,实兼视所受之胎 
善吸取其母之气化否也。由斯而论,愚于千百味药中,得一最善治流产之药,乃菟丝子是也。 
寿胎丸,重用菟丝子为主药,而以续断、寄生、阿胶诸药辅之,凡受妊之妇,于两月之后徐服一料,必无 
流产之弊。此乃于最易流产者屡次用之皆效。至陈修园谓宜用大补大温之剂,使子宫常得暖气,则胎自日长而有 
成,彼盖因其夫人服白术、黄芩连坠胎五次,后服四物汤加鹿角胶、补骨脂、续断而胎安,遂疑凉药能坠胎, 
笃信热药能安胎。不知黄芩之所以能坠胎者,非以其凉也。《神农本草经》谓黄芩下血闭,岂有善下血闭之药 
而能保胎者乎?盖汉、唐以前,名医用药皆谨遵《神农本草经》,所以可为经方,用其方者鲜有流弊。迨至宋、 
元以还,诸家恒师心自智,其用药或至显背《神农本草经》。是以医如丹溪,犹粗忽如此,竟用黄芩为保胎之 
药,俾用其方者不惟无益,而反有所损,此所以为近代之名医也。所可异者,修园固笃信《神农本草经》者也, 
何于用白术、黄芩之坠胎,不知黄芩之能开血闭,而但谓其性凉不利于胎乎?究之胎得其养,全在温度适宜, 
过凉之药,固不可以保胎,即药过于热,亦非所以保胎也。惟修园生平用药喜热恶凉,是以立论稍有所偏耳。 


<篇名>10.安胃饮
属性:治恶阻。 
清半夏(一两,温水淘洗两次毫无矾味然后入煎) 净青黛(三钱) 赤石脂(一两) 
用作饭小锅,煎取清汁一大碗,调入蜂蜜二两,徐徐温饮下。一次只饮一口,半日服尽。若服后吐仍未 
止或其大便燥结者,去石脂加生赭石(轧细)一两。若嫌青黛微有药味者,亦可但用半夏、赭石。 
或问,《神农本草经》谓赭石能坠胎,此方治恶阻,而有时以赭石易石脂,独不虑其有坠胎之弊乎?答 
曰∶恶阻之剧者,饮水一口亦吐出,其气化津液不能下达。恒至大便燥结,旬余不通。其甚者,或结于幽门 
(胃下口)、栏门(大小肠相接处),致上下关格不通,满腹作疼,此有关性命之证也。夫病既危急,非大力 
之药不能挽回。况赭石之性,原非开破。其镇坠之力,不过能下有形滞物。若胎至六七个月,服之或有妨碍∶ 
至恶阻之时,不过两三个月,胎体未成,惟是经血凝滞,赭石毫无破血之性,是以服之无妨。且呕吐者,其冲 
气胃气皆上逆,借赭石镇逆之力,以折其上逆之机,气化乃适得其平,《内经》所谓“有故无殒亦无殒也”。 
愚治恶阻之证,遇有上脘固结,旬日之间勺饮不能下行,无论水与药,入口须臾即吐出。群医束手诿谓不治。 
而愚放胆重用生赭石数两,煎汤一大碗,徐徐温饮下。吐止、结开、便通,而胎亦无伤(拙拟参赭镇气汤及赭 
石解下,载有详案可考也)。 


<篇名>11.大顺汤
属性:治产难,不可早服,必胎衣破后,小儿头至产门者,然后服之。 
野党参(一两) 当归(一两) 生赭石(二两,轧细) 
用卫足花子炒爆一钱作引,或丈菊花瓣一钱作引皆可,无二物作引亦可。 
或疑赭石乃金石之药,不可放胆重用。不知赭石性至和平,虽重坠下行,而不伤气血。况有党参一两以 
补气,当归一两以生 
血。且以参、归之微温,以济赭石之微凉,温凉调和愈觉稳妥也。矧产难者非气血虚弱,即气血壅滞,不能下 
行。人参、当归虽能补助气血,而性皆微兼升浮,得赭石之重坠,则力能下行,自能与赭石相助为理,以成催 
生开交骨之功也。至于当归之滑润,原为利产良药,与赭石同用,其滑润之力亦愈增也。 
族侄妇,临盆两日不产。用一切催生药,胎气转觉上逆。为制此汤,一剂即产下。 
一妇人,临产交骨不开,困顿三日,势甚危急。亦投以此汤,一剂而产。自拟得此方以来,救人多矣。放 
胆用之,皆可随手奏效。 
附∶卫足花即葵花,其子即冬葵子。缘此花若春日早种,当年即可结子。而用以催生,则季夏种之,经冬 
至明年结子者尤效,故名曰冬葵子。今药坊所鬻者,皆以丈菊子为冬葵子,殊属差误。盖古之所谓葵,与俗所 
谓向日葵者原非一种。古所谓葵即卫足花,俗呼为守足花者是也。因此花先生丛叶,自叶中心出茎,茎之下边 
尽被丛叶卫护,故曰卫足。茎高近一丈,花多红色、与木槿相似,叶大如木芙蓉。此为宿根植物,季夏下种, 
至次年孟夏始开花,结实大如钱,作扁形,其中子如榆荚,为其经冬依然发生,故其结之子名为冬葵子。须于 
鲜嫩之时采取,则多含蛋白质,故能有益于人。《圣惠方》谓采其子阴干,是当鲜嫩之时采而阴干之也。若过 
老则在科上自干,而无事阴干矣。又有一种,二、三月下种,至六月开花,其下无丛生之叶,不能卫足,而其 
茎、叶、花皆与葵无异,其治疗之功效亦大致相同,即药品中之蜀葵。《本草纲目》谓花之白者治 疟,是卫 
足葵与蜀葵皆治疟也。 
至于俗所谓向日葵者,各种本草皆未载,惟《群芳谱》载之,本名丈菊,一名西番葵,一名迎阳葵,且谓 
其性能堕胎。然 
用其堕胎之力以催生,则诚有效验,是以大顺汤用其花瓣作引也。向日葵茎长丈许,干粗如竹,叶大如 ,花 
大如盘盂,单瓣黄色,其花心成窠如蜂房。迨中心结子成熟,而周遭花瓣不凋枯。其子人恒炒食之,知其无毒, 
且知其性滑,曾单用以治淋甚效。后与鸦胆子同用(鸦胆子去皮四十粒,用丈菊子一两炒捣,煎汤送下)治花 
柳毒淋,亦甚效,然不知其能治疟也。近阅《绍兴医药学报》载卢××述葵能医疟一节,则丈菊诚可列于药品矣。 
丈菊花英,催生之力实胜于子,曾见有单用丈菊花英催生,服之即效者,惜人多不知耳。至于用卫足子催生, 
当分老嫩两种。鲜嫩卫足子,须用数两捣烂煮汁服,若用老者,当用两许微火炒裂其甲,煎汤饮之。 


<篇名>12.和血熄风汤
属性:治产后受风发搐。 
当归(一两) 生黄 (六钱) 真阿胶(四钱,不炒) 防风(三钱) 荆芥(三钱) 川芎(三钱) 
生杭芍(二钱) 红花(一钱) 生桃仁(钱半,带皮尖捣) 
此方虽治产后受风,而实以补助气血为主。盖补正气,即所以逐邪气,而血活者,风又自去也(血活风 
自去方书成语)。若产时下血过多或发汗过多,以致发搐者,此方仍不可用,为其犹有发表之药也,当滋阴养 
血,以荣其筋,熄其内风,其搐自止。若血虚而气亦虚者,又当以补气之药辅之。而补气之药以黄 为最,因 
黄不但补气,实兼能治大风也(《神农本草经》谓黄 主大风)。 
一妇人,产后七八日发搐,服发汗之药数剂不效。询方于愚,因思其屡次发汗不效,似不宜再发其汗, 
以伤其津液。遂单用阿胶一两,水融化,服之而愈。 
一妇人,产后十余日,周身汗出不止,且发搐。治以山萸肉(去净核)、生山药各一两,煎服两剂,汗 
止而搐亦愈。 
东海渔家妇,产后三日,身冷无汗,发搐甚剧。时愚游海滨,其家人造寓求方。其地隔药局甚远,而海滨 
多产麻黄,可以采取。遂俾取麻黄一握,同鱼鳔胶一具,煎汤一大碗,乘热饮之,得汗而愈。用鱼鳔胶者,亦 
防其下血过多,因阴虚而发搐,且以其物为渔家所固有也。 
一妇人,产后发汗过多,复被三层皆湿透,因致心中怔忡,精神恍惚,时觉身飘飘上至屋顶,此虚极将脱, 
而神魂飞越也。延愚诊视,见其汗出犹不止,六脉皆虚浮,按之即无。急用生山药、净萸肉各一两,生杭芍四 
钱,煎服。汗止精神亦定。翌日药力歇,又病而反复。时愚已旋里,病家复持方来询,为添龙骨、牡蛎(皆不 
用)各八钱,且嘱其服药数剂,其病必愈。孰意药坊中,竟谓方中药性过凉,产后断不宜用,且言此证系产 
后风,彼有治产后风成方,屡试屡验,怂恿病家用之。病家竟误用其方,汗出不止而脱。夫其证原属过汗所致, 
而再以治产后风发表之药,何异鸩毒。斯可为发汗不审虚实者之炯戒矣。 
《傅青主女科》曰∶产后气血暴虚,百骸少血濡养,忽然口紧牙紧,手足筋脉拘搐,类中风痫痉,虽虚 
火泛上有痰,皆当以末治之。勿执偏门,而用治风消痰方,以重虚产妇也。当用生化汤,加参、 以益其气。 
又曰,产后妇人,恶寒恶心,身体颤动,发热作渴,人以为产后伤寒也,谁知其气血两虚,正不敌邪而然乎? 
大抵人之气不虚,则邪断难入。产妇失血过多,其气必大虚,气虚则皮毛无卫,邪原易入。不必户外之风来袭 
体也,即一举一动,风可乘虚而入。然产后之风,易入亦易出,凡有外感之邪,俱不必祛风。况产后之恶寒者, 
寒由内生也。发热者,热由内弱也。身颤者,颤由气虚也。治其内寒,外寒自散。治其内弱,外热自解。壮其 
元气,而身颤自除也。 
按∶傅氏之论甚超。特其虽有外感,不必祛风二句,不无可 
议。夫产后果有外感,原当治以外感之药,惟宜兼用补气生血之药,以辅翼之耳。若其风热已入阳明之府,表 
里俱热,脉象洪实者,虽生石膏亦可用。故《金匮》有竹皮大丸,治妇人乳中虚,烦乱呕逆,方中原有石膏。 
《神农本草经》石膏治产乳,原有明文。特不宜与知母并用,又宜仿白虎加人参汤之意,重用人参,以大补元 
气,更以玄参代知母,始能托邪外出,则石膏之寒凉,得人参之温补,能逗留胃中,以化燥热,不至直趋下焦, 
而与产妇有碍也(石膏解下曾详论之)。 
【附方】《医林改错》治产后风,有黄 桃红汤,方用生黄 半斤,带皮尖生桃仁三钱捣碎,红花二钱, 
水煎服。按产后风项背反张者,此方最效。 
【附方】俗传治产后风方,当归五钱,麻黄、红花、白术各三钱,大黄、川芎、肉桂、紫菀各二钱,煎 
服。按此方效验异常,即至牙关紧闭,不能用药者,将齿拗开灌之,亦多愈者。人多畏其有大黄而不敢用,不 
知西人治产后风,亦多用破血之药。盖以产后有瘀血者多,此证用大黄以破之,所谓血活风自去也。况犹有麻、 
桂之辛热,归、术之补益以调燮之乎。 


<篇名>13.滋阴清胃汤
属性:治产后温病,阳明府实,表里俱热者。 
玄参(两半) 当归(三钱) 生杭芍(四钱) 甘草(钱半) 茅根(二钱) 
上药五味,煎汤两盅,分二次温服,一次即愈者,停后服。 
产后忌用寒凉,而温热入阳明府后,又必用寒凉方解,因此医者恒多束手。不知石膏、玄参《神农本草 
经》皆明载治产乳。是以热入阳明之重者,可用白虎加人参以山药代粳米汤,更以玄参代知母。其稍轻者,治 
以此汤,皆可随手奏效。愚用此两方, 
救人多矣。临证者当笃信《神农本草经》,不可畏石膏、玄参之寒凉也。况石膏、玄参,《神农本草经》原皆 
谓其微寒,并非甚寒凉之药也。 


<篇名>14.滋乳汤
属性:治少乳。其乳少由于气血虚或经络瘀者,服之皆有效验。 
生黄 (一两) 当归(五钱) 知母(四钱) 玄参(四钱) 穿山甲(二钱,炒捣) 六路通 
(大者三枚,捣) 王不留行(四钱,炒) 
用丝瓜瓤作引,无者不用亦可。若用猪前蹄两个煮汤,用以煎药更佳。 


<篇名>15.消乳汤
属性:治结乳肿疼或成乳痈新起者,一服即消。若已作脓,服之亦可消肿止疼,俾其速溃。并治一切红肿疮疡。 
知母(八钱) 连翘(四钱) 金银花(三钱) 穿山甲(二钱,炒捣) 栝蒌(五钱,切丝) 丹参(四钱) 
生明乳香(四钱) 生明没药(四钱) 
在德州时,有张姓妇,患乳痈,肿疼甚剧。投以此汤,两剂而愈。然犹微有疼时,怂恿其再服一两剂,以 
消其芥蒂。以为已愈,不以为意。隔旬日,又复肿疼,复求为治疗。愚曰∶此次服药不能尽消,必须出脓少许, 
因其旧有芥蒂未除,至今已溃脓也。后果服药不甚见效。遂入西医院中治疗,旬日后,其疮外破一口,医者用 
刀阔之,以期便于敷药。又旬日,内溃益甚,满乳又破七八个口,医者又欲尽阔之使通。病患惧,不敢治。强 
出院还家,复求治于愚。见其各口中皆脓、乳并流,外边实不能敷药。然内服汤药,助其肌肉速生,自能排脓 
外出,许以十日可为治愈。遂将内托生肌散,作汤药服之,每日用药一剂,煎服二次, 
果十日全愈。 
【附方】表侄刘××,从愚学医,曾得一治结乳肿疼兼治乳痈方。用生白矾、明雄黄、松萝茶各一钱半, 
共研细,分作三剂,日服一剂,黄酒送下,再多饮酒数杯更佳。此方用之屡次见效,真奇方也。若无松萝茶, 
可代以好茶叶。 


<篇名>16.升肝舒郁汤
属性:治妇女阴挺,亦治肝气虚弱,郁结不舒。 
生黄 (六钱) 当归(三钱) 知母(三钱) 柴胡(一钱五分) 生明乳香(三钱) 生明没药 
(三钱) 川芎(一钱五分) 
肝主筋,肝脉络阴器,肝又为肾行气。阴挺自阴中挺出,形状类筋之所结。病之原因,为肝气郁而下陷 
无疑也。故方中黄 与柴胡、芎 并用,补肝即以舒肝,而肝气之陷者可升。当归与乳香、没药并用,养肝即 
以调肝,而肝气之郁者可化。又恐黄 性热,与肝中所寄之相火不宜,故又加知母之凉润者,以解其热也。 
一妇人,年三十余。患此证,用陈氏《女科要旨》,治阴挺方,治之不效。因忆《傅青主女科》有治阴 
挺之方,其证得之产后。因平时过怒伤肝,产时又努力太过,自产门下坠一片,似筋非筋,似肉非肉,用升补 
肝气之药,其证可愈。遂师其意,为制此汤服之。数剂即见消,十剂全愈。 
一室女,年十五。因胸中大气下陷,二便常觉下坠,而小便尤甚。乃误认为小便不通,努力强便,阴中 
忽坠下一物,其形如桃,微露其尖,牵引腰际下坠作疼,夜间尤甚,剧时号呼不止。投以理郁升陷汤,将升麻 
加倍,二剂疼止,十剂后,其物全消。盖理郁升陷汤,原与升肝舒郁汤相似也。 


<篇名>17.资生通脉汤
属性:治室女月闭血枯,饮食减少,灼热咳嗽。 
白术(三钱,炒) 生怀山药(一两) 生鸡内金(二钱,黄色的) 龙眼肉(六钱) 山萸肉(四钱,去净核) 
枸杞果(四钱) 玄参(三钱) 生杭芍(三钱) 桃仁(二钱) 红花(钱半) 甘草(二钱) 
灼热不退者,加生地黄六钱或至一两。咳嗽者,加川贝母三钱,米壳二钱(嗽止去之)。泄泻者,去玄参, 
加熟地黄一两,云苓片二钱,或更酌将白术加重。服后泻仍不止者,可于服药之外,用生怀山药细末煮粥,搀 
入捻碎熟鸡子黄数枚,用作点心,日服两次,泻止后停服。大便干燥者,加当归、阿胶各数钱。小便不利者, 
加生车前子三钱(装袋),地肤子二钱或将芍药(善治阴虚小便不利)加重。肝气郁者,加生麦芽三钱,川芎、 
莪术各一钱。汗多者,将萸肉改用六钱,再加生龙骨、生牡蛎各六钱。 
《内经》谓“二阳之病发心脾,有不得隐曲,在女子为不月。”夫二阳者,阳明胃腑也。胃腑有病,不 
能消化饮食,推其病之所发,在于心脾。又推其心脾病之所发,在于有不得隐曲(凡不能自如者皆为不得隐曲)。 
盖心主神脾主思,人有不得隐曲,其神思郁结,胃腑必减少酸汁(化食赖酸汁,欢喜则酸汁生者多,忧思则酸 
汁生者少),不能消化饮食,以生血液,所以在女子为不月也。夫女子不月,既由于胃腑有病,不能消化饮食。 
治之者,自当调其脾胃,使之多进饮食,以为生血之根本。故方中用白术以健胃之阳,使之 动有力(饮食之 
消亦仗胃有 动)。山药、龙眼肉,以滋胃之阴,俾其酸汁多生。鸡内金原含有酸汁,且能运化诸补药之力, 
使之补而不滞。血虚者必多灼热,故用玄参、芍药以退热。又血虚者,其肝肾必虚,故用萸肉、枸杞以补其肝 
肾。甘草为补脾胃之正药,与方中萸肉并用,更有酸甘 
化阴之妙。桃仁、红花为破血之要品,方中少用之,非取其破血,欲借之以活血脉通经络也。至方后附载,因 
证加减诸药,不过粗陈梗概,至于证之更改多端,尤贵临证者,因时制宜耳。 
沧州曹姓女,年十六岁,天癸犹未至。饮食减少,身体羸瘦,渐觉灼热。其脉五至,细而无力。治以资 
生通脉汤,服至五剂,灼热已退,饮食加多。遂将方中玄参、芍药各减一钱,又加当归、怀牛膝各三钱。服至 
十剂,身体较前胖壮,脉象亦大有起色。又于方中加樗鸡(俗名红娘虫)十枚,服至七八剂,天癸遂至。遂减 
去樗鸡,再服数剂,以善其后。 
奉天马氏女,自十四岁,月事已通,至十五岁秋际,因食瓜果过多,泄泻月余方愈,从此月事遂闭。延 
医延医,至十六岁季夏,病浸增剧,因求为延医。其身形瘦弱异常,气息微喘,干嗽无痰,过午潮热,夜间尤 
甚,饮食减少,大便泄泻。其脉数近六至,微细无力。俾先用生怀山药细末八钱,水调煮作粥,又将熟鸡子黄 
四枚,捻碎搀粥中,再煮一两沸,空心时服。服后须臾,又服西药百布圣二瓦,以助其消化。每日如此两次, 
用作点心,服至四日,其泻已止。又服数日,诸病亦稍见轻。遂投以资生通脉汤,去玄参加生地黄五钱、川贝 
三钱,连服十余剂,灼热减十分之八,饮食加多,喘嗽亦渐愈。遂将生地黄换作熟地黄,又加怀牛膝五钱,服 
至十剂,自觉身体爽健,诸病皆无,惟月事犹未见。又于方中加 虫(即土鳖虫,背多横纹者真,背光滑者非 
是)五枚、樗鸡十枚,服至四剂,月事已通。遂去 虫、樗鸡,俾再服数剂,以善其后。 
马姓女十七岁。自十六岁秋际,因患右目生内障,服药不愈,忧思过度,以致月闭。自腊月服药,直至 
次年孟秋月底不愈,求为延医。其人体质瘦弱,五心烦热,过午两颧色红,灼热益甚,心中满闷,饮食少许, 
即停滞不下,夜不能寐。脉搏五 
至,弦细无力。为其饮食停滞,夜不能寐,投以资生通脉汤,加生赭石(研细)四钱,熟枣仁三钱,服至四剂, 
饮食加多,夜已能寐,灼热稍退,遂去枣仁,减赭石一钱,又加地黄五钱,丹皮三钱,服约十剂,灼热大减。 
又去丹皮,将龙眼肉改用八钱,再加怀牛膝五钱。连服十余剂,身体浸壮健。因其月事犹未通下,又加 虫五 
枚、樗鸡十枚。服至五剂,月事已通。然下者不多,遂去樗鸡、地黄,加当归五钱,俾服数剂,以善其后。 

\chapter{治眼科方}
<篇名>1.蒲公英汤
属性:治眼疾肿疼,或 肉遮睛,或赤脉络目,或目睛胀疼,或目疼连脑,或羞明多泪,一切虚火实热之证。 
鲜蒲公英(四两,根叶茎花皆用,花开残者去之,如无鲜者可用干者二两代之。) 
上一味煎汤两大碗,温服一碗。余一碗乘热熏洗(按目疼连脑者,宜用鲜蒲公英二两,加怀牛膝一两煎汤 
饮之)。 
此方得之于××,言其母尝患眼疾,疼痛异常,经延医调治,数月不愈,有高姓媪,告以此方,一次即愈。 
愚自得此方后,屡试皆效。夫蒲公英遍地皆有,仲春生苗,季春开花色正黄,至初冬其花犹有开者,状类小菊, 
其叶似大蓟,田家采取生啖,以当菜蔬。其功长于治疮,能消散痈疔毒火,然不知其能治眼疾也。使人皆知其 
治眼疾,如此神效,天下无瞽目之人矣。 
古服食方,有还少丹。蒲公英连根带叶取一斤,洗净,勿令见天日,晾干,用斗子解盐(即《神农本草经》 
大盐晒于斗之中者出山西解池)一两,香附子五钱,二味为细末,入蒲公英,水内淹一宿,分为十二团,用皮 
纸三四层裹扎定,用六一泥(即蚯蚓泥)如法固济,灶内焙干,乃以武火 
通红为度,冷定取出,去泥为末。早晚擦牙漱之,吐咽任便,久久方效。年未及八十者,服之须发反黑,齿 
落更生。年少服之,至老不衰。由是观之,其清补肾经之功可知。且其味苦,又能清心经之热,所以治眼疾甚 
效者,或以斯欤! 


<篇名>2.磨翳水
属性:治目翳遮睛。 
生炉甘石(一两) 蓬砂(八钱) 胆矾(二钱) 薄荷叶(三钱) 蝉蜕(三钱,带全足去翅土) 
上药五味,将前三味药臼捣细,再将薄荷、蝉蜕煎水一大盅,用其水和所捣药末,入药钵内研至极细,将 
浮水者随水飞出,连水别贮一器,待片时,将浮头清水,仍入钵中,和所余药渣研细,仍随水飞出,如此不计 
次数,以飞净为度。若飞过者还不甚细,可再研再飞,以极细为度。制好连水贮瓶中,勿令透气。用时将瓶 
中水药调匀,点眼上,日五六次。若目翳甚浓,已成肉螺者,加真藏 砂二分,另研调和药水中。此方效力全 
在甘石生用,然生用则质甚硬,又恐与眼不宜,故必如此研细水飞,然后可以之点眼。 


<篇名>3.磨翳散
属性:治目睛胀疼,或微生云翳,或赤脉络目,或目 溃烂,或偶因有火视物不真。 
生炉甘石(三钱) 蓬砂(二钱) 黄连(一钱) 人指甲(五分,锅焙脆无翳者不用) 
上药先将黄连捣碎,泡碗内,冷时两三日,热时一日,将泡黄连水过罗,约得清水半茶盅,再将余三味捣 
细,和黄连水入药钵中研之,如研前药之法,以极细为度。研好连水带药,用大盘盛之。白日置阴处晾之,夜 
则露之,若冬日微晒亦可。若有风尘 
时,盖以薄纸。俟干,贮瓶中,勿透气。用时凉水调和,点眼上,日三四次。若有目翳,人乳调和点之。若目 
翳大而浓者,不可用黄连水研药,宜用蝉蜕(带全足去翅土)一钱,煎水研之。盖微茫之翳,得清火之药即退。 
若其翳已遮晴,治以黄连成冰翳,而不能消矣。 


<篇名>4.明目蓬硝水
属性:治眼疾暴发红肿疼痛。或 多 肉,或渐生云翳,及因有火而眼即发干昏花者。 
蓬砂(五钱) 芒硝(三钱,硝中若不明亮用水化开澄去其中泥土) 
上药和凉水多半盅,研至融化。用点眼上,一日约点三十次。若陈目病一日点十余次。冬日须将药碗置热 
水中,候温点之。 


<篇名>5.清脑黄连膏
属性:治眼疾由热者。 
黄连(二钱) 为细末,香油调如薄糊,常常以鼻闻之,日约二三十次。勿论左右眼患证,应须两鼻孔皆闻。 
目系神经连于脑,脑部因热生炎,病及神经,必生眼疾。彼服药无捷效者,因所用之药不能直达脑部故也。 
愚悟得此理,借鼻窍为快捷方式,以直达于脑。凡眼目红肿之疾,及一切目疾之因热者,莫不随手奏效。 


<篇名>6.益瞳丸
属性:治目瞳散大昏耗,或觉视物乏力。 
萸肉(二两,去净核) 野台参(六钱) 柏子仁(一两,炒) 玄参(一两) 菟丝子(一两,炒) 
羊肝(一具,切片焙干) 
上药共为细末,炼蜜为丸桐子大。每服三钱,开水送下,日两次。 
一妇人,年三旬。瞳子散大,视物不真,不能针黹。屡次服药无效,其脉大而无力。为制此丸,服两月全愈。 


<篇名>7.羊肝猪胆丸
属性:治同前证,因有热而益甚者。 
羊肝一具,切片晒干(冬日可用,慢火焙干)。 
上一味轧细,用猪胆汁和为丸,桐子大,朱砂为衣。每服二钱,开水送下,日再服。 
按∶此方若用熊胆为丸更佳。内地鲜熊胆不易得,至干者又难辨其真伪,不如径用猪胆汁为稳妥也。 
【附方】护眉神应散 治一切眼疾。无论气蒙、火蒙、肉螺、云翳、或瞳人反背。未过十年者,皆 
见效。方用炉甘石一两 透,童便淬七次。珍珠二颗,大如绿豆以上者,纳通草中 之,珠爆即速取出。血琥 
珀三分,真梅片二分,半两钱、五铢钱(俗名马镫钱)、开元钱各一个,皆 红醋淬七次,共为细末,乳调涂 
眉上,日二三次。 
一室女。病目年余,医治无效,渐生云翳。愚为出方,服之见轻,停药仍然反复。后得此方,如法制好, 
涂数次即见轻,未尽剂而愈,妙哉。按此方若加薄荷冰二分更效。 
瞳人反背之证,最为难治,以其系目系神经病也。盖目系神经,若一边纵,一边缩,目之光线必斜,视物 
即不真。若纵、缩之距离甚大,其瞳人即可反背。治此证者,当以养其目系神经为主。此方多用金石珍贵之品, 
其中含有宝气。凡物之含有宝气者,皆善能养人筋肉,使筋肉不腐烂。目系神经,即脑气筋之连于目者。以此 
药涂眉上,中有冰片之善通窍透膜者,能引药气直 
达脑部,以养目系神经,目系神经之病者自愈。而瞳人反背及一切眼疾,亦自愈矣。 
【附方】治暴发眼便方。其眼疾初得肿疼者,用生姜三四钱、食盐一大撮,同捣烂,薄布包住,蘸新汲 
井泉水,擦上下眼皮。屡蘸屡擦,以擦至眼皮极热为度。擦完用温水将眼皮洗净。轻者一次即愈,重者一日擦 
两次亦可愈。然擦时须紧闭其目,勿令药汁入眼中。 
【附案】晋书盛彦母氏失明,躬自侍养,母食必自哺之。母病既久,至于婢使,数见捶鞭。婢愤恨,伺 
彦暂行,取蛴螬炙饴之,母食以为美,然疑是异物,密藏以示彦。彦见之,抱母恸哭,绝而复苏。母目豁然, 
从此遂愈。 
按蛴螬生粪土中,形状如蚕(俗名地蚕)遍处皆有。《神农本草经》谓,主目中淫肤、青翳、白膜。其善 
治目翳可知。内障宜油炙服之,外障宜取其汁,滴目中。 

\chapter{治咽喉方}
<篇名>咀华清喉丹
属性:治咽喉肿疼。 
大生地黄(一两,切片) 蓬砂(钱半,研细) 
将生地黄一片,裹蓬砂少许,徐徐嚼细咽之,半日许宜将药服完。 
生地黄之性能滋阴清火,无论虚热实热服之皆宜。蓬砂能润肺,清热化痰,消肿止疼。二药并用,功力甚 
大。而又必细细嚼服者,因其病在上,煎汤顿服,恐其力下趋,而病转不愈。且细细嚼咽,则药之津液常清润 
患处也。此方愚用之屡矣,随手奏效 
者不胜纪矣。 
咽喉之证,有热有凉,有外感有内伤。《白喉忌表抉微》一书,此时盛行于世。其所载之方,与所载宜 
用、宜忌之药,皆属稳善。惟其持论,与方中所用之药,有自相矛盾处。谆谆言忌表矣,而其养阴清肺汤,用 
薄荷二钱半,岂非表药乎?至于他方中,所用之葛根、连翘亦发表之品也。盖白喉之证,原亦温病之类。人之 
外肤肺主之,人之内肤三焦主之。盖此证心肺先有蕴热,外感之邪又袭三焦,而内逼心肺。则心肺之热,遂与 
邪气上并,而现证于喉。既有外邪,原宜发表,因有内热,实大忌用辛热之药发表。惟薄荷、连翘诸药,辛凉 
宣通,复与大队凉润之药并用,既能散邪,尤能清热,所以服之辄效也。若其内热炽盛,外感原甚轻者,其养 
阴清肺汤亦可用,特其薄荷,宜斟酌少用,不必定用二钱半也。至谓其喉间肿甚者加 石膏四钱,微有可议。 
夫石膏之性,生则散、 则敛。炽盛之火散之则消,敛之则实,此又不可不知也。况石膏生用,原不甚凉, 
故《神农本草经》谓微寒,又何必如此之小心乎。今将其养阴清肺汤,详录于下,以备采用。 
【附方】养阴清肺汤 大生地一两、寸麦冬六钱、生白芍四钱、薄荷二钱半、玄参八钱、丹皮四 
钱、贝母四钱、生甘草二钱。喉间肿甚者,加生石膏(原用 石膏)四钱。大便燥结者,加清宁丸二钱、玄明 
粉二钱。胸下胀闷,加神曲、焦山楂各二钱。小便短赤者,加木通、泽泻各一钱,知母二钱。燥渴者,加天冬、 
马兜铃各三钱。面赤身热,或舌苔黄色者,加金银花四钱,连翘二钱。 
白喉之证,间有服《白喉忌表抉微》诸方不效,而反加剧者。曾治一贵州人,孙××,年二十,得白喉 
证。屡经医治,不外《白喉忌表抉微》诸方加减。病日增重,医者诿谓不治。后愚为诊视,其脉细弱而数,粘 
涎甚多,须臾满口,即得吐出。知系脾 
肾两虚,肾虚气化不摄,则阴火上逆,痰水上泛。而脾土虚损,又不能制之(若脾土不虚,不但能制痰水上泛, 
并能制阴火上逆),故其咽喉肿疼,粘涎若是之多也。投以六味地黄汤,加于术,又少加苏子,连服十剂全愈。 
咽喉之证,热者居多。然亦兼有寒者,不可不知。王洪绪曰∶“咽喉之间,素分毫无病,顷刻之间,或 
疼或闷,此系虚寒阴火之证。用肉桂、炮姜、甘草各五分,置碗内浸以滚水,仍将碗置于滚水中,饮药一口, 
徐徐咽下立愈。或用乌附之片,涂以鲜蜜,火炙透至黑,取一片口含咽津,至片不甜时,再换一片,亦立愈。” 
按王氏之说,咽喉陡然疼闷者,皆系因寒。然亦有因热者,或其人素有蕴热,陡然为外感所束,或劳碌过度, 
或暴怒过度,皆能使咽喉骤觉疼闷。斯在临证者,于其人之身体、性情、动作之际,细心考验,再参以脉象之 
虚实凉热,自无差谬。若仍恐审证不确,察其病因似寒,而尤恐病因是热,可用蜜炙附子片试含一片,以细验 
其病之进退亦可。 
赵晴初曰∶鸡蛋能去喉中之风。余治一幼童喉风证,与清轻甘凉法,稍加辛药,时止时发。后有人教服 
鸡蛋,顶上针一孔,每日生吞一枚,不及十枚,病愈不复发。 
友人齐××曰∶平阳何××患喉疼,医者治以苦寒之药,愈治愈甚,渐至舌硬。后有人教用棉子油煎生鸡蛋, 
煎至外熟,里仍微生,日服二枚,未十日遂大愈。 
咽喉肿疼证,有外治异功散方甚效。其方用斑蝥一钱,真血竭、制乳香、制没药、上麝香、全蝎、大玄 
参、上梅片各分半,将斑蝥去翅足,糯米拌炒,以米色微黄为度,去糯米。用诸药共研细,瓶收贮,勿令透气。 
遇有咽喉肿疼证,将药捏作小块,如黄豆粒大,置在小膏药上,左肿贴右,右肿贴左,若左右俱肿,均贴在结 
喉(项间高骨)旁边软处。阅五六时,即揭去膏药,有水泡,用银针挑破,拭净毒水,能消肿止疼,真救急之良方也。 

\chapter{治牙疳方}
<篇名>1.古方马乳饮
属性:治青腿牙疳。 
用青白马乳,早午晚随挤随服甚效。如无青白马,杂色马亦可。若马乳自他处取来,可将碗置于开水盆中温之。 
此方出于《医宗金鉴》,其原注云∶此证自古方书罕载其名,仅传于雍正年间。陶起麟谓∶凡病腿肿色青 
者,其上必发牙疳,凡病牙疳腐血者,其下必发青腿,二者相因而至。推其病原,皆因上为阳火炎炽,下为阴 
寒闭郁,以至阴阳上下不交,各自为寒为热,凝结而生此证也。相近内地亦间有之,边外虽亦有,而不甚多, 
惟内地人初居边外,得此证者十居七八。盖内地之人,本不耐边外严寒,更不免坐卧湿地,故寒湿之痰生于下, 
致腿青肿。其病形如云片,色似茄黑,肉体顽硬,所以步履艰难也。又缘边外缺少五谷,多食牛羊等肉,其热 
与湿合蒸,瘀于胃中,毒火上熏,致生牙疳。牙龈浮肿出血,若穿腮破唇,腐烂色黑,即为危候。惟相传有服 
马乳之法,用之颇有效验云云。 
按∶此证愚未见过,友人毛××曾遇此证治愈。其方愚犹记其大概,爰列于下,以备采用。 
金银花(五钱) 连翘(三钱) 菊花(三钱) 明乳香(四钱) 明没药(四钱) 怀牛膝(五钱) 
山楂片(三钱) 真鹿角胶(四钱,捣为细末分两次用头煎二煎汤药送服) 
按此方若服之出汗,即可见愈。然方中连翘、菊花发汗之力甚微,恐服之不能出汗,当于服药之后,再服 
西药阿斯匹林一瓦,则无不出汗矣。至汗后服第二剂时,宜将菊花减半。 


<篇名>2.牙疳散
属性:甘石(二钱) 镜面朱砂(二分) 牛黄(五厘) 珍珠(五厘, ) 
共研细,日敷三次。 

<篇名>3.牙疳敷藤黄法
属性:己巳春,阅沪上《福祉医学报》载有章×之言,有误用藤黄,治愈走马牙疳之事,甚为奇异。兹特录其原文 
于下,以供医界之研究∶ 
丁卯三月,余偕友数人,偶至仁溏观优。有潘氏子,年四岁,患走马牙疳,起才三日,牙龈腐化,门牙已 
脱数枚,下唇已溃穿,其势甚剧。问尚有可救之理否?询其由,则在发麻之后。实为邪热入胃,毒火猖狂,一 
发难遏,证情危险。告以只有白马乳凉饮,并不时洗之,涂以人中白,内服大剂白虎汤,或有可救。但势已穿 
唇,效否不敢必耳。因书生石膏、生知母、生打寒水石、象贝等为方与之。其时同游者,有老医倪××,因谓之 
曰,牛黄研末,外掺腐烂之处,亦或可治。遂彼此各散。后数日,则此儿竟已痊愈,但下唇缺不能完。因询其 
用何物疗治,乃得速效若斯,则曰,用倪××说,急购藤黄屑而掺之,果然一掺腐势即定,血水不流,渐以结靥 
落痂,只三日耳。内服石膏等一方,亦仅三服。此儿获愈,因其无识,误听牛黄为藤黄。然以此一误,而竟治 
愈极重之危证。开药学中从古未有之实验,胡可以不志也。尝考李氏《本草纲目》,蔓草中曾载藤黄,而功用 
甚略。至赵恕轩《本草纲目拾遗》,言之甚详。虽曰有毒,而可为内服之品,且引《粤志》谓,其性最寒,可 
治眼疾,味酸涩,治痈肿,止血化毒,敛金疮,能除虫,同麻油、白蜡熬膏,敷金疮、汤火等伤,止疼收口, 
其效如神。而其束疮消毒之用又甚多,可知 
此药,竟是外科中绝妙良药。而世多不知用者,误于李氏《海药本草》有毒之两字。而张石顽更以能治蛀齿, 
点之即落,而附会为毒,损骨伤肾,于是畏之甚于蛇蝎,实不知石顽不可信。今之画家,常以入口,虽曰与花 
青并用,可解其毒,余以为亦理想之谈耳。既曰性寒,毒于何有?然后知能愈牙疳,正是寒凉作用。且其味酸 
涩,止血、止疼、收口、除虫皆其能治牙疳之切实发明也。 
按∶走马牙疳之原因,有内伤外感之殊。得于由内伤者轻而缓,由外感者重而急。此幼童得于麻疹之后, 
其胃中蕴有瘟毒上攻,是以三日之间,即腐烂如此。幸内服石膏、寒水石,外敷藤黄,内外夹攻,皆中要肯, 
是以其毒易消,结痂亦在三日内也。若当牙疳初起之时,但能用药消其内蕴之毒热,即外不敷药,亦可治愈。 
曾治天津于氏幼童,年六七岁,身出麻疹,旬日之外热不退,牙龈微见腐烂。其家人惧甚,恐成走马牙疳,延 
愚诊视。脉象有力而微弦,知毒热虽实,因病久者,气分有伤也。问其大便,三日未行。遂投以大剂白虎加人 
参汤,方中生石膏用三两,野党参用四钱,又加连翘数钱,以托疹毒外出。煎汤三茶盅,俾分三次温饮下。又 
用羚羊角一钱,煎水一大茶盅,分数次当茶饮之,尽剂热退而病愈。牙龈腐烂之处,亦遂自愈。 

\chapter{治疮科方}
<篇名>1.消瘰丸
属性:治瘰 。 
牡蛎(十两, ) 生黄 (四两) 三棱(二两) 莪术(二两) 朱血竭(一两) 生明乳香(一两) 
生明没药(一两) 龙胆草(二两) 玄参(三两) 浙贝母(二两) 
上药十味,共为细末,蜜丸桐子大。每服三钱,用海带五钱,洗净切丝,煎汤送下,日再服。 
瘰 之证,多在少年妇女,日久不愈,可令信水不调,甚或有因之成劳瘵者。其证系肝胆之火上升,与痰 
涎凝结而成。初起多在少阳部位,或项侧,或缺盆,久则渐入阳明部位。一颗垒然高起者为瘰,数颗历历不断 
者为 。身体强壮者甚易调治。 
此方重用牡蛎、海带,以消痰软坚,为治瘰 之主药,恐脾胃弱者,久服有碍,故用黄 、三棱、莪术以 
开胃健脾(三药并用能开胃健脾,十全育真汤下曾详言之),使脾胃强壮,自能运化药力,以达病所。且此证 
之根在于肝胆,而三棱、莪术善理肝胆之郁。此证之成,坚如铁石,三棱、莪术善开至坚之结。又佐以血竭、 
乳香、没药,以通气活血,使气血毫无滞碍,瘰 自易消散也。而犹恐少阳之火炽盛,加胆草直入肝胆以泻之, 
玄参、贝母清肃肺金以镇之。且贝母之性,善于疗郁结利痰涎,兼主恶疮。玄参之性,《名医别录》谓其散颈 
下核,《开宝本草》谓其主鼠 ,二药皆善消瘰 可知。 
血竭,色赤味辣。色赤故入血分,味辣故入气分,其通气活血之效,实较乳香、没药为尤捷。诸家本草, 
未尝言其辣,且有言其但入血分者,皆未细心实验也。然此药伪者甚多,必未研时微带紫黑,若血干之色。研 
之红如鸡血,且以置热水中则溶化,须臾复凝结水底成块者,乃为真血竭。 
曾治一少年,项侧起一瘰 ,其大如茄,上连耳,下至缺盆。求医治疗,言服药百剂,亦不能保其必愈。 
而其人家贫佣力,为人芸田,不惟无钱买如许多药,即服之亦不暇。然其人甚强壮,饮食甚多,俾于一日三餐 
之时,先用饭汤送服 牡蛎细末七八钱,一月之间消无芥蒂。 
又治一妇人,在缺盆起一瘰 ,大如小橘。其人亦甚强壮无他病,俾煮海带汤,日日饮之,半月之间,用 
海带二斤而愈。若 
身体素虚弱者,即煮牡蛎、海带,但饮其汤,脾胃已暗受其伤。盖其咸寒之性,与脾胃不宜也。 
族侄女患此证,治数年不愈。为制此方,服尽一料而愈。 


<篇名>2.消瘰膏
属性:消瘰 。 
生半夏(一两) 生山甲(三钱) 生甘遂(一钱) 生马钱子(四钱,剪碎) 皂角(三钱) 朱血竭(二钱) 
上药前五味,用香油煎枯去渣,加黄丹收膏,火候到时将血竭研细搀膏中熔化和匀,随疮大小摊作膏药。 
临用时每药一帖加麝香少许。 
友人之女年五岁。项间起瘰 数个,年幼不能服药,为制此药,贴之全愈。 
凡膏药中用黄丹,必以火炒过,然后以之熬膏,其胶粘之力始大。而麝香不早加入膏药中者,以麝香忌火也。 


<篇名>3.化腐生肌散
属性:治瘰 已溃烂者,用此药擦之。他疮破后者亦可用之。 
炉甘石(六钱, ) 乳香(三钱) 没药(三钱) 明雄黄(二钱) 蓬砂(三钱) 砂(二分) 冰片(三分) 
共研细,收贮瓶中勿令透气。日擦患处三四次,用此药长肉。将平时收口不速者,可加珍珠一分, 研 
细搀入,其 法详护眉神应散后。 


<篇名>4.内托生肌散
属性:治瘰 疮疡破后,气血亏损不能化脓生肌。或其疮数年不愈,外边疮口甚小,里边溃烂甚大,且有串至 
他处不能敷药者。 
生黄 (四两) 甘草(二两) 生明乳香(一两半) 生明没药(一两半) 生杭芍(二两) 
天花粉(三两) 丹参(一两半) 
上七味共为细末,开水送服三钱,日三次。若将散剂变作汤剂,须先将花粉改用四两八钱,一剂分作八 
次煎服,较散剂生肌尤速。 
从来治外科者,于疮疡破后不能化脓生肌者,不用八珍即用十全大补。不知此等药若遇阳分素虚之人服 
之犹可,若非阳分素虚或兼有虚热者,连服数剂有不满闷烦热,饮食顿减者乎?夫人之后天,赖水谷以生气血, 
赖气血以生肌肉,此自然之理也。而治疮疡者,欲使肌肉速生,先令饮食顿减,斯犹欲树之茂而先戕其根也。 
虽疮家阴证,亦可用辛热之品,然林屋山人阳和汤,为治阴证第一妙方,而重用熟地一两以大滋真阴,则热药 
自无偏胜之患,故用其方者,连服数十剂而无弊也。如此方重用黄 ,补气分以生肌肉,有丹参以开通之,则 
补而不滞,有花粉、芍药以凉润之,则补而不热,又有乳香、没药、甘草化腐解毒,赞助黄 以成生肌之功。 
况甘草与芍药并用,甘苦化合味同人参,能双补气血,则生肌之功愈速也。至变散剂为汤剂,花粉必加重者, 
诚以黄 煎之则热力增,花粉煎之则凉力减,故必加重而其凉热之力始能平均相济也。至黄 必用生者,因生 
用则补中有宣通之力,若炙之则一于温补,固于疮家不宜也。 
一人年二十余。因抬物用力过度,腰疼半年不愈。忽于疼处发出一疮,在脊梁之旁,微似红肿,状若复 
盂,大径七寸。疡医以为腰疼半年,始现此疮,其根蒂必深而难治。且其内外发热,饮食懒进,舌苔黄浓,脉 
象滑数。知其证兼外感实热,投以白虎加人参汤,热退能食。数日,又复虚汗淋漓,昼夜不止,遂用龙骨、牡 
蛎(皆不用 )、生杭芍、生山药各一两为方,两剂汗止。继治以清火、消肿、解毒之药,若拙拟消乳汤,去 
栝蒌加金线重楼、 
三七(冲服)之类,更加鹿角霜钱许,以引经。惟消乳汤以知母为君重八钱,兹则所用不过五六钱。外用五倍 
子、三七、枯矾、金线重楼、白芨为末,以束其根;乳香、没药、雄黄、金线重楼、三七为末,以敷其顶,皆 
用醋调之。旬日疮消三分之二,其顶甚软。遂以乌金膏(以雄黄炒巴豆仁至黑色研细名乌金膏)调香油敷其软 
处。二日,疮破出稠脓若干。将此内托生肌散改作汤剂,投之,外敷拙拟化腐生肌散。七八日间疮口长平,结 
痂而愈。 
徐灵胎治疮最重围药。以围药束住疮根,不使毒势散漫,又能阻隔周身之热力,不贯注于疮,则疮必易 
愈。愚治此疮所用束根之药,实师徐氏之意也。 


<篇名>5.洗髓丹
属性:治杨梅疮毒蔓延周身,或上至顶,或下至足,或深入骨髓,无论陈、新、轻、剧,服之皆有奇效。三四 
日间疮痂即脱落。 
净轻粉(二钱,炒至光色减去三分之二,研细,盖此药炒之则烈性少缓,若炒之过度,又恐无力,火候宜中, 
用其大片即净轻粉。) 净红粉(一钱,研细,须多带紫黑片者用之,方有效验。) 露蜂房(如拳大者一个, 
大者可用一半,小者可用两个,炮至半黑半黄色,研细,炮时须用物按之着锅。) 核桃(十个,去皮捣碎,炮 
至半黑半黄色,研细,纸包数层,压去其油,盖油多即不好为丸用。) 
上诸药用熟枣肉为丸,黄豆粒大,晒干,分三次服之。服时,须清晨空心开水送下,至午后方可饮食, 
忌腥半月。服后,口含柳棍,有痰涎即吐出,愈多吐愈好。睡时将柳棍横含,两端各系一绳,两绳之端结于脑 
后,防睡着掉落。又须将柳棍勤换,即将药服完仍须如此,必待不吐痰涎时,方可不含柳棍。其药日服一次, 
若恶心太甚者,可间日一服。制此药时,须自经手,将轻粉、红粉称极准,其秤当以库秤为定法,轻粉须称准 
后再炒。 
此方,人多有疑其服之断生育者,非也。轻粉虽烈, 之则 
烈性顿减,红粉虽性近轻粉而止用一钱,且分作三日服之,又有枣肉之甘缓以解毒,核桃仁多用至十枚,峻补 
肾经以防患,配合得宜,服之自有益无害。 
轻粉系水银同矾石升炼而成,红粉亦系水银同矾石、硝石诸药升炼而成,其质本重坠,故能深入,其成于 
升炼,故能飞扬。是以内浃骨髓,中通脏腑,外达皮肤,善控周身之毒涎,借径于阳明经络,自齿龈(上龈属 
足阳明下龈属手阳明)而出也。蜂房,能引人身之毒涎透退场门齿,且有以毒攻毒之妙用,为轻粉、红粉之佐使。 
毒涎之出者愈多,即内毒之消者愈速矣。核桃仁润而多脂,性能补骨益髓可知。且又善解疥癣之毒,其能解他 
疮之毒亦可知。加于此药之中,补正兼以逐邪,毒之深入骨髓者亦不难消除矣。至于丸以枣肉,取其甘缓之性, 
能缓二粉之猛悍,又能补助肠胃使不为毒药所伤也。 
服药之后,其牙龈必肿,间有烂者,因毒涎皆从此出故也。然内毒既清,外证不治自愈,或用甘草、蓬砂、 
金银花熬水漱之亦可。 
蜂房有三种∶有黄色大蜂其房上下恒作数层,其毒甚大不宜用。曾见有以之煎水漱牙疼者,其牙龈遂皆溃 
烂脱牙十余枚。有黄色小蜂其房甚小,房孔仅如绿豆,虽无大毒而力微,又不堪用。惟其蜂黄而兼红,大近寸 
许,恒在人家屋中垒房,俗呼为马蜂,其房入药最宜。然其房在树上者甚少,若无在树上之露蜂房,在屋中者 
亦可用,特稍宜加重耳。 
马姓,年四十余,先染淋毒,后变为梅毒,求为延医。其毒周身不现形迹,惟觉脑际沉昏颇甚,心中时或 
烦躁,骨节多有疼痛之处,所甚异者,其眉棱眼稍及手指之节多生软骨,西人亦谓系梅毒所凝结也。愚对此证, 
不敢谓其必治愈,犹幸身体不甚羸弱,遂将洗髓丹一剂俾分四次服完。歇息旬日,再服一剂,将其 
分量减三分之一。歇息旬日,又服一剂,较二次所服之分量又减三分之一,皆四日服完,其病递次消除。凡软 
骨将消者,必先发起,然后徐徐消肿,化为无有。共计四浃辰,诸病皆愈。 
又治一郝姓小孩,因食乳传染,咽喉溃烂,至不能进食,肛门亦甚溃烂,其肠胃之溃烂可知。其父来院细 
言其病状,问还有救否?答曰∶“果信用余方,仍能救。”遂与以洗髓丹六粒,俾研细水调服三次,全愈。 
又∶奉天一幼童,有遗传性梅毒,年六岁不能行,遍身起疮若小疖,愈而复发,在大连东人医院住近一年 
不愈。后来院求治,其身体羸弱,饮食甚少,先用药理其脾胃,俾能饮食。渐加以解毒之药,若金银花、连翘、 
天花粉诸品,身体渐壮,疮所发者亦渐少,然毒之根蒂仍未除也。遂将洗髓丹五分许研细(将制成丸药复研末 
者因孺子不能服丸药也),开水调服,三日服一次,仍每日服汤药一剂。后将洗髓丹服至十次,疮已不发。继 
又服汤药月余,兼用滋阴补肾之品,每剂中有核桃仁三个,取其能健骨也(食酸 齿者嚼核桃仁立愈是能健骨 
之明征),从此遂能步履行动如常童矣。观此二案,则洗髓丹奇异之功效,诚可于解梅毒药中首屈一指。且凡 
解梅毒药,无论或注射、或服药,愈后又恒肢体作疼,以其能清血中之毒,不能清骨中之毒,是以愈后其骨节 
犹疼也。因其骨中犹含有毒性,恒迟至日久而复发,或迟至十余年而复发者,若再投以此丹,则骨疼立愈,且 
以后永不反复,此又愚屡经试验而确知其然者也。 

\part{药物}
<篇名>1.石膏解
属性:石膏之质原为硫养轻钙化合而成,其性凉而能散,有透表解肌之力,为清阳明胃腑实热之圣药,无论内伤、 
外感用之皆效,即他脏腑有实热者用之亦效。《神农本草经》原谓其微寒,其寒凉之力远逊于黄连、龙胆草、 
知母、黄柏等药,而其退热之功效则远过于诸药。《神农本草经》谓其微寒,则性非大寒可知。且谓其宜于产 
乳,其性尤纯良可知。 
盖言其性不甚寒凉,可用于产后也。乃后世注《神农本草经》者,不知产乳之乳字原作生字解,而竟谓石 
膏能治妇人无乳,支离殊甚。要知产后无外感之热,石膏原不可用。若确有外感实热,他凉药或在所忌,而独 
不忌石膏,以石膏之性非大寒,乃微寒也。是以汉季张仲景所着《金匮》中有竹皮大丸,治妇人乳中虚、烦乱、 
呕逆,中有石膏。夫乳中者,生子之时也,其烦乱呕逆必有外感之实热也,此实通《神农本草经》石膏主产乳 
之义以立方也。 
石膏医者多误认为大寒而 用之,则宣散之性变为收敛(点豆腐者必 用,取其能收敛也),以治外感有 
实热者,竟将其痰火敛住,凝结不散,用至一两即足伤人,是变金丹为鸩毒也。迨至误用 石膏偾事,流俗之 
见,不知其咎在 不在石膏,转谓石膏 用之其猛烈犹足伤人,而不 者更可知矣。于是一倡百和,遂视用石 
膏为畏途,即有放胆用者,亦不过七八钱而止。夫石膏之质甚重,七 
八钱不过一大撮耳。以微寒之药,欲用一大撮扑灭寒温燎原之热,又何能有大效。是以愚用生石膏以治外感实 
热,轻证亦必至两许;若实热炽盛,又恒重用至四五两,或七八两,或单用,或与他药同用,必煎汤三四茶杯, 
分四五次徐徐温饮下,热退不必尽剂。如此多煎徐服者,欲以免病家之疑惧,且欲其药力常在上焦、中焦,而 
寒凉不至下侵致滑泻也。盖石膏生用以治外感实热,断无伤人之理,且放胆用之,亦断无不退热之理。惟热实 
脉虚者,其人必实热兼有虚热,仿白虎加人参汤之义,以人参佐石膏亦必能退热。盖诸药之退热,以寒胜热也, 
而石膏之退热,逐热外出也。是以将石膏煎服之后,能使内蕴之热息息自毛孔透出,且因其含有硫养轻,原具 
发表之性,以之煮汤又直如清水,服后其寒凉之力俱随发表之力外出,而毫无汁浆留中以伤脾胃,是以遇寒温 
之大热势若燎原,而放胆投以大剂白虎汤,莫不随手奏效。其邪实正虚者,投以白虎加人参汤,亦能奏效。 
盖石膏之所以善治寒温者,原恃其原质中之硫养轻也。若 之,其硫养轻皆飞去,所余之钙经 即变质, 
若误服之,能将人外感之痰火及周身之血脉皆为凝结锢闭。是以见有服 石膏数钱脉变结代,浸至言语不遂, 
肢体痿废者;有服 石膏数钱其证变结胸,满闷异常,永不开通者;有服 石膏数钱其周身肌肉似分界限,且 
又突起者。盖自有石膏 不伤胃之语,医者轻信其说以误人性命者实不胜计矣。故凡用生石膏者,宜买其整块 
明亮者,自监视轧细(凡石质之药不轧细,则煎不透)方的。若购自药局中难辨其 与不 ,迨将药煎成,石 
膏凝结药壶之底,倾之不出者,必系 石膏,其药汤即断不可服。 
且尝历观方书,前哲之用石膏,有一证而用至十四斤者(见《笔花医镜》);有一证而用至数十斤者 
(见《吴鞠通医案》);有产后亦重用石膏 
者(见徐灵胎医案然须用白虎加人参汤以玄参代知母生山药代粳米)。然所用者皆生石膏也。 
【附案】子××,七岁时,感冒风寒,四五日间,身大热,舌苔黄而带黑。孺子苦服药,强与之即呕吐不 
止。遂单用生石膏两许,煎取清汤,分三次温饮下,病稍愈。又煎生石膏二两,亦徐徐温饮下,病又见愈。又 
煎生石膏三两,徐徐饮下如前,病遂全愈。夫以七岁孺子,约一昼夜间,共享生石膏六两,病愈后饮食有加, 
毫无寒中之弊,则石膏果大寒乎?抑微寒乎?此系愚初次重用石膏也。故第一次只用一两,且分三次服下,犹 
未确知石膏之性也。世之不敢重用石膏者,何妨若愚之试验加多以尽石膏之能力乎? 
同邑友人赵××之妻,年近六旬得温病,脉数而洪实,舌苔黄而干,闻药气即呕吐。俾单用生石膏细末六 
两,以作饭小锅(不用药甑,恐有药味复呕吐)煎取清汤一大碗,恐其呕吐,一次只温饮一口,药下咽后,觉 
烦躁异常,病家疑药不对证。愚曰∶“非也,病重药轻故也”,饮至三次,遂不烦躁,阅四点钟尽剂而愈。 
同邑友人毛××之三子××,年三十二岁,素有痰饮,得伤寒证,服药调治而愈。后因饮食过度而复,服 
药又愈。后数日又因饮食过度而复,医治无效。四五日间,延愚诊视,其脉洪长有力,而舌苔淡白,亦不燥渴, 
食梨一口即觉凉甚,食石榴子一粒,心亦觉凉。愚舍证从脉,为开大剂白虎汤方,因其素有痰饮,加清半夏数 
钱,其表兄高××在座,邑中之宿医也,疑而问曰∶“此证心中不渴不热,而畏食寒凉如此,以余视之虽清解 
药亦不宜用,子何所据而用生石膏数两乎?”答曰∶“此脉之洪实,原是阳明实热之证,其不觉渴与热者,因 
其素有痰饮湿胜故也。其畏食寒凉者,因胃中痰饮与外感之热互相胶漆,致胃府转从其化与凉为敌也。”毛×× 
素晓医学,信用愚言,两日夜间服药十余次,共享生石膏斤余,脉始和平,愚遂旋里。隔两日复来相 
迎,言病患反复甚剧,形状异常,有危在顷刻之虑。因思此证治愈甚的,何至如此反复。既至(相隔三里强), 
见其痰涎壅盛,连连咳吐不竭,精神恍惚,言语错乱,身体颤动,诊其脉平和无病,惟右关胃气稍弱。愚恍然 
会悟,急谓其家人曰∶“此证万无闪失,前因饮食过度而复,此次又因戒饮食过度而复也。”其家人果谓有鉴 
前失,数日之间,所与饮食甚少。愚曰∶“此无须用药,饱食即可愈矣。”其家人虑其病状若此,不能进食。 
愚曰∶“无庸如此多虑,果系由饿而得之病,见饮食必然思食。”其家人根据愚言,时已届晚八句钟,至黎明进 
食三次,每次撙节与之,其病遂愈。 
友人毛××妻,年近七旬,于正月中旬,伤寒无汗。原是麻黄汤证,因误服桂枝汤,汗未得出,上焦陡觉烦 
热恶心,闻药气即呕吐,但饮石膏所煮清水及白开水亦呕吐。惟昼夜吞小冰块可以不吐,两日之间,吞冰若干, 
而烦热不减,其脉关前洪滑异常。俾用鲜梨片,蘸生石膏细末嚼咽之,遂受药不吐,服尽二两而病愈。 
一人患梅毒,在东人医院治疗二十余日,头面肿大,下体溃烂,周身壮热,谵语不省人事,东人谓毒已走 
丹不可治。其友人孙××,邀愚往东人院中为诊视。疑其证夹杂温病,遂用生石膏细末半斤,煮水一大瓶,伪作 
葡萄酒携之至其院中,托言探友,盖不欲东人知为疗治也。及入视病患,其头面肿而且红,诊其脉洪而实,知 
系夹杂温病无疑,嘱将石膏水徐徐温服。翌日,又往视,其头面红肿见退,脉之洪实亦减半,而较前加数,仍 
然昏愦谵语,分毫不省人事。所饮石膏之水尚余一半,俾自购潞党参五钱,煎汤兑所余之石膏水饮之。翌日, 
又往视之,则人事大清,脉亦和平。病患遂决意出彼院来院中调治,后十余日其梅毒亦愈。此证用潞党参者, 
取其性平不热也。 
一人,年五十,周身发冷,两腿疼痛。医者投以温补之药,其冷益甚,欲作寒战。诊其脉,甚沉伏,重按 
有力。其舌苔黄浓,小便赤涩。当时仲春,知其春温之热,郁于阳明而未发,故现此假象也。欲用白虎汤加连 
翘治之,病患闻之,骇然。愚曰∶但预购生石膏四两,迨热难忍时,煎汤饮之可乎?病者曰∶恐无其时耳。愚 
曰∶若取鲜白茅根,煎汤饮之,则冷变为热,且变为大热矣。病者仍不确信,然欲试其验否,遂剖取鲜白茅根, 
去净皮,细锉一大碗,煮数沸,取其汤,当茶饮之。有顷热发,若难忍。须臾再诊其脉,则洪大无伦矣。愚将 
所预购之四两生石膏煎汤,分三次温饮下,其热遂消。 
石膏之性,又善清瘟疹之热(参阅清疹汤后附案),又善清头面之热(参阅青盂汤后附案),又善清咽喉 
之热(参阅“详论咽喉证治法”所载治沧州友人董××一案)。 
外感痰喘,宜投以《金匮》小青龙加石膏汤。若其外感之热,已入阳明之府,而小青龙中之麻、桂、姜、 
辛诸药,实不宜用。曾治刘××,年八岁。孟秋患温病,医治十余日,病益加剧。表里大热,喘息迫促,脉象 
洪数,重按有力,知犹可治。问其大便,两日未行,投以大剂白虎汤,重用生石膏二两半,用生山药一两以代 
方中粳米。且为其喘息迫促、肺中伏邪,又加薄荷叶一钱半以清之。俾煎汤两茶盅,作两次温饮下,一剂病愈 
强半,又服一剂全愈。 
从来产后之证,最忌寒凉。而果系产后温病,心中燥热,舌苔黄浓,脉象洪实,寒凉亦在所不忌。然所用 
寒凉之药,须审慎斟酌,不可漫然相投也。愚治产后温证之轻者,其热虽入阳明之府,而脉象不甚洪实,恒重 
用玄参一两,或至二两,辄能应手奏效。若系剧者,必用白虎加人参汤方能退热。然用时须以生山药代粳米, 
玄参代知母,方为稳妥。医方篇中白虎加人参以山药代粳米汤下附有验案可参观。盖以石膏、玄参,《神农本 
草经》皆明言其 
治产乳,至知母条下则未尝言之,不敢师心自用也。 
友人毛××曾治一少妇,产后十余日,周身大热,无汗,心中热,而且渴。延医调治,病势转增。甚属危 
急。毛××诊其脉,甚洪实,舌苔黄而欲黑,撮空摸床,内风已动。治以生石膏三两,玄参一两,野台参五钱, 
甘草二钱。为服药多呕,取竹皮大丸之义,加竹茹二钱,煎汤一大碗,徐徐温饮下,尽剂而愈。观此案,则外 
感之热,直如燎原,虽在产后,岂能从容治疗乎?孙思邈曰∶智欲圆而行欲方,胆欲大而心欲小。世俗医者, 
遇此等证,但知心小,而不知胆大。岂病患危急之状,漠不关于心乎? 
在女子有因外感之热内迫,致下血不止者,亦可重用白虎加人参汤治之。邻村李氏妇,产后数日,恶露 
已尽,至七八日,忽又下血。延医服药,二十余日不止,其脉洪滑有力,心中热而且渴。疑其夹杂外感,询之 
身不觉热,舌上无苔,色似微白,又疑其血热妄行,投以凉血兼止血之药,血不止而热渴亦如故。因思此证实 
夹杂外感无疑,遂改用白虎加人参汤,方中生石膏重用三两,更以生山药代粳米煎汤三盅,分三次温饮下,热 
渴遂愈,血亦见止。又改用凉血兼止血之药而愈。 
痢证身热不休,服一切清火之药,而热仍不休者,方书多诿为不治。夫治果对证,其热焉有不休之理? 
此乃因痢证夹杂外感,其外感之热邪,随痢深陷,弥漫于下焦经络之间,永无出路,以致痢为热邪所助,日甚 
一日而永无愈期。夫病有兼证,即治之宜有兼方也,斯非重用生石膏更助以人参以清外感之热不可(通变白虎加 
人参汤后载有治王××验案可参阅)。 
表兄张××之妻高氏。年五十余,素多疾病。于季夏晨起偶下白痢,至暮十余次,秉烛后,忽然浑身大热, 
不省人事,循衣摸床,呼之不应。其脉洪而无力,肌肤之热烙手。知其系气分热 
痢,又兼受暑,多病之身不能支持,故精神昏愦如是也。急用生石膏三两、野党参四钱,煎汤一大碗,徐徐温 
饮下。至夜半尽剂而醒,痢亦遂愈,诘朝煎渣再服,其病脱然。 
上所载痢证医案,皆兼外感之热者也。故皆重用生石膏治之,非概以其方治痢证也。拙着《衷中参西录》 
中,治痢共有七方,皆随证变通用之,确有把握,前案所用之方,乃七方之一也。愚用此方治人多矣,脉证的 
确,用之自无差忒也。 
疟疾虽在少阳,而阳明兼有实热者,亦宜重用生石膏。曾治邻村李××,年四十许,疟疾间日一发,热时 
若燔,即不发之日亦觉表里俱热,舌燥口干,脉象弦长,重按甚实。此少阳邪盛,阳明热盛,疟而兼温之脉也。 
投以大剂白虎汤加柴胡三钱,服后顿觉清爽。翌晨疟即未发,又煎服前剂之半,加生姜三钱,温、疟从此皆愈。 
至脉象虽不至甚实,而按之有力,常觉发热懒食者,愚皆于治疟剂中,加生石膏两许以清之,亦莫不随手奏效也。 
石膏之性,又善治脑漏。方书治脑漏之证,恒用辛夷、苍耳。然此证病因,有因脑为风袭者,又因肝移热 
于脑者。若因脑为风袭而得,其初得之时,或可用此辛温之品散之,若久而化热,此辛温之药即不宜用,至为 
肝移热于脑,则辛温之药尤所必戒也。近治奉天郭××,得此证半载不愈。鼻中时流浊涕,其气腥臭,心热神昏,恒 
觉眩晕。其脉左右皆弦而有力,其大便恒干燥,知其肝移热于脑,其胃亦移热于脑矣。恐其病因原系风袭,先与西 
药阿斯匹林瓦许以发其汗,头目即觉清爽,继为疏方,用生石膏两半,龙胆草、生杭芍、玄参、知母、花粉各四钱, 
连翘、金银花、甘草各二钱,薄荷叶一钱。连服十剂,石膏皆用两半,他药则少有加减,其病遂脱然全愈。 
奉天一人得此证,七八日,其脉浮而有力,知其因风束生热 
也。亦先用阿斯匹林瓦许汗之。汗后,其鼻中浊涕即减,亦投以前方,连服三剂全愈。 
《神农本草经》谓石膏能治腹痛,诚有效验。曾治奉天刘××腹疼,三年不愈。其脉洪长有力,右部尤甚, 
舌心红而无皮,时觉头疼眩晕,大便干燥,小便黄涩,此乃伏气化热,阻塞奇经之经络,故作疼也。为疏方∶ 
生石膏两半,知母、花粉、玄参、生杭芍、川楝子各五钱,乳香、没药各四钱,甘草二钱,一剂疼愈强半。即 
原方略为加减,又服数剂全愈。 
愚弱冠,有本村刘氏少年,因腹疼卧病月余,昼夜号呼,势极危险。延医数人,皆束手无策。闻愚归,求 
为诊视,其脉洪长有力,盖从前之疼犹不至如斯,为屡次为热药所误,故疼益加剧耳。亦投以前方,惟生石膏 
重用二两,一剂病大轻减。后又加鲜茅根数钱,连服两剂全愈。盖此等证,大抵皆由外感伏邪窜入奇经,久而 
生热。其热无由宣散,遂郁而作疼。医者为其腹疼,不敢投以凉药,甚或以热治热,是以益治益剧。然证之凉 
热,脉自有分,即病患细心体验,亦必自觉。临证者尽心询问考究,自能得其实际也。 
石膏之性,又最宜与西药阿斯匹林并用。盖石膏清热之力虽大,而发表之力稍轻。阿斯匹林味酸性凉,最 
善达表,使内郁之热由表解散,与石膏相助为理,实有相得益彰之妙也。如外感之热,已入阳明胃腑,其人头 
疼舌苔犹白者,是仍带表证。愚恒用阿斯匹林一瓦,白蔗糖化水送服以汗之。迨其汗出遍体之时,复用生石膏 
两许,煎汤乘热饮之(宜当汗正出时饮之),在表之热解,在里之热亦随汗而解矣。若其头已不疼,舌苔微黄, 
似无表证矣,而脉象犹浮,虽洪滑而按之不实者,仍可用阿斯匹林汗之。然宜先用生石膏七八钱,或两许,煮 
汤服之,俾热势少衰,然后投以阿斯匹林,则汗既易出,汗后病亦易解也。若其热未随汗全解,仍可 
徐饮以生石膏汤,清其余热。不但此也,若斑疹之毒,郁而未发,其人表里俱热,大便不滑泻者,可用生石膏 
五六钱,煎汤冲服阿斯匹林半瓦许,俾服后,微似有汗,内毒透彻,斑疹可全然托出。若出后壮热不退,胃腑 
燥实,大便燥结者,又可多用生石膏至二三两许,煎汤一大碗(约有三四茶杯),冲阿斯匹林一瓦,或一瓦 
强,一次温饮数羹匙。初饮略促其期,迨热见退,或大便通下,尤宜徐徐少饮,以壮热全消,仍不至滑泻为度。 
如此斟酌适宜,斑疹无难愈之证矣。石膏与阿斯匹林,或前后互用,或一时并用,通变化裁,存乎其人,果能 
息息与病机相赴,功效岂有穷哉! 
用阿斯匹林治关节肿疼之挟有外感实热者,又必与石膏并用,方能立见奇效。奉天赵××之侄,年六岁。 
脑后生疮,漫肿作疼,继而头面皆肿,若赤游丹毒。继而作抽掣,日甚一日。浸至周身僵直,目不能合,亦不 
能瞬,气息若断若续,呻吟全无。其家人以为无药可治,待时而已。阅两昼夜,形状如故,试灌以勺水,似犹 
知下咽。因转念或犹可治,而彼处医者,咸皆从前延请而屡次服药无效者也。来院求为延医。其脉洪数而实, 
肌肤发热,知其夹杂温病,阳明腑证已实,势虽垂危,犹可挽回。遂用生石膏细末四两,以蒸汽水煎汤两茶杯, 
徐徐温灌之。周十二时,剂尽,脉见和缓,微能作声。又用阿斯匹林瓦半,仍以汽水所煎石膏汤,分五次送下, 
限一日夜服完。服至末二次,皆周身微见汗,其精神稍明了,肢体能微动。从先七八日不食,且不大便,至此 
可少进茶汤,大便亦通下矣。继用生山药细末煮作稀粥,调以白蔗糖,送服阿斯匹林三分之一瓦,日两次,若 
见有热,即间饮汽水所煮石膏汤。又以蜜调黄连末,少加薄荷冰,敷其头面肿处,生肌散敷其疮口破处,如此 
调养数日,病势减退,可以能言。其左边手足仍不能动,试略为屈伸,则疼不能忍。细验 
之,关节处皆微肿,按之觉疼,知其关节之间,因外感之热而生炎也。遂又用鲜茅根煎浓汤(无鲜茅根可代以 
鲜芦根),调以白蔗糖,送服阿斯匹林半瓦,日两次。俾服药后周身微似有汗,亦间有不出汗之时,令其关节 
中之炎热,徐徐随发表之药透出。又佐以健补脾胃之药,俾其多进饮食。如此旬余,左手足皆能运动,关节能 
屈伸。以后饮食复常,停药勿服,静养半月,行动如常矣。此证共享生石膏三斤,阿斯匹林三十瓦,始能完全 
治愈。愚用阿斯匹林治热性关节肿疼者多矣,为此证最险,故详记之。 
丁仲祜《西药实验谈》载,东人用阿斯匹林,治愈关节急性偻麻质斯(即热性关节肿疼)之案甚伙,而其 
证之险,皆远逊于此证。若遇此证,不能重用生石膏,尚有何药能与阿斯匹林并用,以挽回此极险之证乎?彼 
欲废弃中药者,尚其详观此案也。 
上所录诸案,其为证不同,然皆兼有外感实热者也。乃有其人纯系内伤,脏腑失和,而前哲具有特识,亦 
有重用石膏者。徐灵胎曰∶“嘉兴朱宗臣,以阳盛阴亏之体,又兼痰凝气逆。医者以温补治之,胸膈痞塞,而 
阳道痿。群医谓脾肾两亏,将恐无治,就余于山中。余视其体,丰而气旺,阳升而阴不降,诸窍皆闭。笑谓之 
曰∶此为肝肾双实证,先用清润之药,加石膏以降其逆气,后以消痰开胃之药涤其中宫,更以滋肾强阴之药镇其 
元气,阳事即通。五月后,外家即怀孕,得一女,又一年复得一男。”观此,石膏治外感兼治内伤,功用何其弘哉! 
穷极石膏之功用,恒有令人获意外之效者。曾治奉天马姓叟,年近六旬,患痔疮,三十余年不愈。后因伤 
寒证,热入阳明之府,投以大剂白虎汤数剂,其病遂愈,痔疮竟由此除根。 
奉天吕姓幼童,年五六岁,每年患眼疾六七次,皆治于东人医院。东人谓此关于禀赋,不能除根。后患瘟 
疹,毒热甚恣,投以托毒清火之品,每剂中用生石膏两半,病愈后,其眼疾亦从此 
不再反复。 
友人张××,曾治京都阎姓叟。年近七旬,素有劳疾,发则喘而且嗽。于冬日感冒风寒,上焦烦热,劳疾 
大作,痰涎胶滞,喘促异常。其脉关前洪滑,按之有力。张××治以生石膏二两以清时气之热,因其劳疾,加 
沉香五钱,以引气归肾。且以痰涎太盛,石膏能润痰之燥,不能行痰之滞,故又借其辛温之性,以为石膏之反 
佐也。一日连服二剂,于第二剂加清竹沥二钱,病若失。劳疾亦从此除根永不反复。夫劳疾至年近七旬,本属 
不治之证,而事出无心,竟以重用石膏治愈之,石膏之功用,何其神哉!愚因闻此案,心有会悟,拟得治肺劳 
黄膏方,其中亦用生石膏,服者颇有功效。 
寒温阳明府病,原宜治以白虎汤。医者畏不敢用,恒以甘寒之药清之,遇病之轻者,亦可治愈,而恒至稽 
留余热(甘寒药滞泥,故能闭塞外感热邪),变生他证。迨至病久不愈,其脉之有力者,仍可用白虎汤治之, 
其脉之有力而不甚实者,可用白虎加人参汤治之。曾治奉天一学徒。年十四五,得劳热喘嗽证。初原甚轻,医 
治数月,病势浸增,医者诿谓不治。遂来院求为诊视,其人羸弱已甚,而脉象有力,数近六至,疑其有外感伏 
热,询之果数月之前,曾患温病,经医治愈。乃知其决系外感留邪,问其心中时觉发热,大便干燥,小便黄涩, 
遂投以白虎加人参汤,去粳米加生怀山药一两,连服数剂,病若失。见者讶为奇异,不知此乃治其外感,非治 
其内伤,而能若是之速效也。 
《内经》谓“冬伤于寒,春必病温”,是言伏气为病也。乃有伏气伏于膈膜之下(《内经》所谓横连膜 
原也),逼近胃口,久而化热,不外发为温病,转上透膈膜,熏蒸肺脏,致成肺病者。若其脉有力,亦宜重用 
生石膏治之。曾治奉天赵某年四十许。始则发热懒食,继则咳嗽吐痰腥臭,医治三月,浸至不能起床。脉象滑 
实,右脉 
尤甚(伏邪之热,亦如寒温之脉,多右盛于左),舌有黄苔,大便数日一行。知系伏气为病,投以大剂白虎汤, 
以生山药代粳米,又加利痰解毒之品,三剂后病愈强半。又即其方加减,服至十余剂全愈。 
有伏气下陷于奇经诸脉中,久而化热,其热亦不能外发为温。有时随奇经之脉上升者。在女子又有热入血 
室而子宫溃烂者,爰录两案于下以证之。 
安东尉××,年二十余。时觉有热,起自下焦,上冲脑部。其脑部为热冲激,头巅有似肿胀,时作眩晕, 
心中亦时发热,大便干燥,小便黄涩。经医调治,年余无效。求其处医士李××寄函来问治法,其开来病案如此。 
曰∶“其脉象洪实,饮食照常,身体亦不软弱。”知其伏有外感热邪,因其身体不弱,俾日用生石膏细末四两, 
煮水当茶饮之,若觉凉时即停服。后二十余日,其人忽来奉,言遵示服石膏六七斤,上冲之热见轻,而大便微 
溏,因停药不服。诊其脉仍然有力,问其心中仍然发热,大便自停药后即不溏矣。为开白虎加人参汤,方中生 
石膏重用三两,以生怀山药代粳米,连服六七剂,上冲之热大减,因出院还家。嘱其至家,按原方服五六剂, 
病当除根矣。 
南皮张××妻年三十余。十年前,恒觉少腹切疼。英女医谓系子宫炎证,用药数次无效。继乃谓此病如欲除 
根,须用手术剖割,将生炎之处其腐烂者去净,然后敷药能愈。病患惧而辞之。后至奉,又延东女医治疗,用 
坐药兼内服药,数年稍愈,至壬戌夏令,病浸增剧,时时疼痛,间下脓血。癸亥正初,延愚延医。其脉弦而有 
力,尺脉尤甚。自言疼处觉热,以凉手熨之稍愈,上焦亦时觉烦躁。恍悟此证,当系曾受外感热入血室,医者 
不知,治以小柴胡汤加石膏,外感虽解,而血室之热未清。或伏气下陷入于血室,阻塞气化,久而生热,以致 
子宫生炎,浸至溃烂,脓血下注。为疏方,用金银花、乳香、没药、甘草以解其 
毒,天花粉,知母、玄参以清其热,复本小柴胡汤之义,少加柴胡提其下陷之热上出,诸药煎汤,送服三七细 
末二钱,以化腐生新。连服三剂病似稍轻,其热仍不少退。因思此证,原系外感稽留之热,非石膏不能解也。 
遂于原方中加生石膏一两,后渐加至二两,连服数剂,热退强半,疼亦大减。遂去石膏,服数剂渐将凉药减少, 
复少加健胃之品,共服药三十剂全愈。后在天津治冯氏妇此证,亦用此方。中有柴胡,即觉脓血不下行,后减 
去柴胡,为之治愈。 
《神农本草经》谓石膏治金疮,是外用以止其血也。愚尝用 石膏细末,敷金疮出血者甚效。盖多年壁上 
锻石,善止金疮出血,石膏经 与锻石相近,益见 石膏之不可内服也。 
石膏生用之功效,不但能治病,且善于治疮,且善于解毒。奉天赵××之父,年过六旬,在脐旁生痈,大径 
三寸,五六日间烦躁异常,自觉屋隘莫容。其脉左关弦硬,右关洪实,知系伏气之热与疮毒俱发也。问其大便 
数日未行,投以大剂白虎汤加金银花、连翘、龙胆草,煎汤一大碗,徐徐温饮下,连服三剂,烦躁与疮皆愈。 
又∶在籍时,本村张氏女因家庭勃 ,怒吞砒石,未移时,作呕吐。其兄疑其偷食毒物,诡言无他,惟服 
皂矾少许耳。其兄闻其言,急来询解救之方。愚曰皂矾原系硫氧与铁化合,分毫无毒,呕吐数次即愈,断无闪 
失,但恐未必是皂矾耳。须再切问之。其兄去后,迟约三点钟复来,言此时腹中绞疼,危急万分,始实言所吞 
者是砒石,非皂矾也。急令买生石膏细末二两,用凉水送下。乃村中无药铺,遂至做豆腐家买得生石膏,轧细 
末,凉水送下,腹疼顿止。犹觉腹中烧热,再用生石膏细末半斤,煮汤两大碗,徐徐饮之,尽剂而愈。后又遇 
吞火柴中毒者,治以生石膏亦愈,然以其毒缓,但煎汤饮之,无用送服其细末也。 
附录∶ 
直隶盐山孙××来函∶ 
一九二四年八月,友人张××之女,发热甚剧,来询方。为开生石膏一两半,煎汤饮之。其热仍不稍退, 
又来询方。答以多煎石膏水饮之,必能见愈。张××购石膏数两,煮汤若干,渴则饮之,数日而愈。 
直隶盐山李××来函∶ 
丁卯中秋,曾治天津傅姓少年,患温证,胃热气逆,无论饮食药物下咽即吐出。延医治疗,皆因此束手。 
弟忽忆《衷中参西录》石膏解载治毛姓媪医案,曾用此方以止呕吐,即以清胃府之大热,遂仿而用之。食梨一 
颗,蘸生石膏细末七钱余,其吐顿止,可以进食。然心中犹觉热,再投以白虎加人参汤,一剂全愈。 
江苏崇明县刁××来函∶ 
近治一伏温病,壮热烦渴,脉来洪实兼数,大解十日未行。欲透其邪,则津液已衰,恐有汗脱之虞,欲通 
其便,则并无承气确征。细思此证,乃阳明热久,真阴铄耗。遵先生重用生石膏之训,即用生石膏二两,合增 
液汤,加鲜金钗石斛、香青蒿各三钱。病家疑忌,见者皆以为药性过寒凉。余愤然曰∶“择医宜慎,任医宜专。 
既不信余药,请余何为?”病家不得已,购药一剂,俾煎汤两盅,作两次服下。而热势益炽,病家疑药不对证。 
余曰∶“此非药不对证,乃药轻不胜病耳。”遂俾将两剂并作一剂,煎汤一大碗,徐徐温饮下。移时汗出便通, 
病若失。 


\chapter{人参解}
属性:(附∶人参形状考) 
人参之种类不一,古所用之人参,方书皆谓出于上党,即今之党参是也。考《神农本草经》载,人参味甘, 
未尝言苦,今党参味 
甘,辽人参则甘而微苦,古之人参其为今之党参无疑也。特是,党参之性,虽不如辽人参之热,而其性实温而 
不凉,乃因《神农本草经》谓其微寒,后世之笃信《神农本草经》者,亦多以人参之性果然微寒,即释古方之 
用人参者,亦本微寒之意以为诠解,其用意可谓尊经矣。然古之笃信《神农本草经》而尊奉之者莫如陶弘景, 
观其所着《名医别录》,以补《神农本草经》所未备,谓人参能疗肠胃中冷,已不遵《神农本草经》以人参为 
微寒可知。因此,疑年湮代远,古经字句或有差讹,吾人生今之世,当实事求是,与古为新。今试即党参实验 
之,若与玄参等分并用,可使药性无凉热,即此可以测其热力矣(此即台党参而言,若潞党参其热稍差)。然 
辽东亦有此参,与辽人参之种类迥别,为其形状性味与党参无异,故药行名之为东党参,其功效亦与党参同。 
至于辽人参,其补力、热力皆倍于党参,而其性大约与党参相似,东人谓过服之可使脑有充血之病,其性补而 
上升可知。方书谓人参,不但补气,若以补血药辅之亦善补血。愚则谓,若辅以凉润之药即能气血双补,盖平 
其热性不使耗阴,气盛自能生血也。至《神农本草经》谓其主补五脏、安精神、定魂魄、止惊悸、除邪气、明 
目、开心、益智,无非因气血充足,脏腑官骸各得其养,自有种种诸效也。 


<篇名>附∶人参形状考
属性:人参∶无论野山、移山、种秧,其色鲜时皆白,晒干则红,浸以白冰糖水,晒干则微红,若浸之数次,虽 
晒于亦白矣。野山之参,其芦头(生苗之处,亦名露土)长而细,极长者可至二寸,细若韭莛,且多龃龉,有 
芦头短者则稍粗。至秧参之芦头,长不过七八分,其粗则过于箸矣。 
人参之鲜者,皆有粗皮,制时用线七八条作一缕为弓弦,用此弦如拉锯状,来回将其粗皮磨去,其皮色始 
光润,至皮上之横纹以细密而深者为佳。野山之参一寸有二十余纹,秧参则一寸不 
过十余纹,且其纹形破裂,有似刀划,野山参之纹则分毫无破裂。然无论野参、秧参,其纹皆系生成,非人力 
所能为也。 
人参之须以坚硬者为贵,盖野参生于坚硬土中,且多历岁月,其须自然坚硬。若秧参则人工种植,土松年 
浅,故其须甚软也。 
至于野参之性温和,秧参之性燥热,人所共知,究其所以然之故,非仅在历年之浅深也。因种秧参者多撒 
砒石末于畦中,以防虫蚁之损伤,参得砒石之气故甚燥热,是以愚于治寒温方中当用参者,从不敢投以秧参, 
恒以野党参代之,亦能立起沉 。至于西洋参,多系用秧参伪制,此愚在奉目睹,用者亦当审慎也。 
山西党参,种植者多,野生者甚少。凡野生者其横纹亦如辽人参,种植者则无横纹,或芦头下有横纹仅数 
道,且种者皮润肉肥,野者皮粗肉松,横断之中心有纹作菊花形。其芦头以粗大者为贵,名曰狮头党参,为其 
历年久远,屡次自芦头发生,故作此形。其参生于五台山者名台党参,色白而微黄,生于潞州太行紫团山者名 
潞党参,亦名紫团参,色微赤而细,以二参较之,台党参力稍大,潞党参则性平不热,以治气虚有热者甚宜。 
然潞党参野生者甚少,多系人种植者,至辽东所出之党参(为其形若党参,故俗名东党参)。状若台党参,皆 
系野生,其功用与山西之野台党参相近。 
【附案】邑中高某年四十许,于季春得温病。屡经医者调治,大热已退,精神益惫,医者诿为不治。病 
家亦以为气息奄奄,待时而已。乃迟旬日而病状如故,始转念或可挽回。迎愚诊视,其两目清白无火,竟昏愦 
不省人事,舌干如磋,却无舌苔,问之亦不能言,抚其周身皆凉,其五六呼吸之顷,必长出气一口,其脉左右 
皆微弱,至数稍迟,知其胸中大气因服开破降下药太过而下陷也。盖大气不达于脑中则神昏,大气不潮于舌本 
则舌干。神昏舌干,故问之不能言也;其周身皆凉者,大气陷后不能宣布营卫 
也;其五六呼吸之顷必长出气者,大气陷后胸中必觉短气,故太息以舒其气也。遂用野台参一两、柴胡二钱, 
煎汤灌之,一剂见轻,两剂全愈。 
外甥王××年二十时,卧病数月不愈,精神昏愦,肢体酸懒,微似短气,屡次延医服药莫审病因,用药亦无 
效验。一日忽然不能喘息,张口呼气外出而气不上达,其气蓄极下迫肛门突出,约二十呼吸之顷,气息方通, 
一昼夜间如是者八九次。诊其脉关前微弱不起,知其胸中大气下陷,不能司肺脏呼吸之枢机也。遂投以人参一 
两,柴胡三钱,知母二钱,一剂而呼吸顺,又将柴胡改用二钱,知母改用四钱,再服数剂宿病亦愈。 
愚治大气下陷多重用生黄 ,取其补气兼能升气也。而此案与前案皆重用参者,因一当外感之余,津液铄 
耗,人参兼能滋津液;一当久病之余,元气亏损,人参兼能固元气也。 
吐血过多者,古方恒治以独参汤,谓血脱者先益其气也。然吐血以后,多虚热上升,投以独参汤恐转助其 
虚热,致血证仍然反复。愚遇此等证,亦恒用人参而以镇坠凉润之药辅之(保元寒降汤一案可参阅)。 
人参之性,用之得宜,又善利小便(济阴汤后一案可参阅)。 


\chapter{西洋参解}
属性:西洋参∶味甘微苦,性凉。能补助气分,兼能补益血分,为其性凉而补,凡欲用人参而不受人参之温补者, 
皆可以此代之。惟白虎加人参汤中之人参,仍宜用党参而不可代以西洋参,以其不若党参具有升发之力,能助 
石膏逐邪外出也。且《神农本草经》谓人参味甘,未尝言苦,适与党参之味相符,是以古之人参,即今之党参, 
若西洋参与高丽参,其味皆甘而兼苦,故用于古方不宜也。西洋参产于法兰西国,外带粗皮则色黄,去粗皮则 
色白,无论或黄或白,以多有横纹者为真。愚用此参,皆用黄皮多横纹 
者,因伪造者能造白皮西洋参,不能造黄皮西洋参也。 


<篇名>4.黄解
属性:黄 ∶性温,味微甘。能补气,兼能升气,善治胸中大气(即宗气,为肺叶 辟之原动力)下陷。《神农 
本草经》谓主大风者,以其与发表药同用,能祛外风,与养阴清热药同用,更能熄内风也。谓主痈疽、久败疮 
者,以其补益之力能生肌肉,其溃脓自排出也。表虚自汗者,可用之以固外表气虚。小便不利而肿胀者,可用 
之以利小便。妇女气虚下陷而崩带者,可用之以固崩带。为其补气之功最优,故推为补药之长,而名之曰 也。 
【附案】沧州董氏女,年二十余。胸胁满闷,心中怔忡,动则自汗,其脉沉迟微弱,右部尤甚,为其脉迟, 
疑是心肺阳虚,询之不觉寒凉,知其为胸中大气下陷也。其家适有预购黄 一包,俾用一两煎汤服之。其族兄 
××在座,其人颇知医学,疑药不对证。愚曰∶“勿多疑,倘有差错,余职其咎。”服后,果诸病皆愈。其族兄 
疑而问曰∶“《神农本草经》黄 原主大风,有透表之力,生用则透表之力益大,与自汗证不宜,其性升而能 
补,有膨胀之力,与满闷证不宜,今单用生黄 两许,而两证皆愈,并心中怔忡亦愈,其义何居?”答曰∶ 
“黄 诚有透表之力,气虚不能逐邪外出者,用于发表药中,即能得汗,若其阳强阴虚者,误用之则大汗如雨 
不可遏抑。惟胸中大气下陷,致外卫之气无所统摄而自汗者,投以黄 则其效如神。至于证兼满闷而亦用之者, 
确知其为大气下陷,呼吸不利而作闷,非气郁而作闷也。至于心与肺同悬胸中,皆大气之所包举,大气升则心 
有所根据,故怔忡自止也。”继加桔梗二钱,知母三钱,又服两剂以善其后。 
一妇人产后四五日,大汗淋漓,数日不止,情势危急,气息奄奄,其脉微弱欲无,问其短气乎?心中怔忡 
且发热乎?病患不 
能言而颔之。知其大气下陷,不能吸摄卫气,而产后阴分暴虚,又不能维系阳分,故其汗若斯之脱出也。遂用 
生黄 六钱,玄参一两,净萸肉、生杭芍各五钱,桔梗二钱,一剂汗减,至三剂诸病皆愈。从前五六日未大便, 
至此大便亦通下。 
邑王氏女,年二十余,心中寒凉,饮食减少,延医服药,年余无效,且益羸瘦。后愚诊视,其左脉微弱不 
起,断为肝虚证。其父知医,疑而问曰∶“向延医延医,皆言脾胃虚弱,相火衰损,故所用之方皆健脾养胃, 
补助相火,曾未有言及肝虚者,先生独言肝虚,但因左脉之微弱乎?抑别有所见而云然乎?”答曰∶“肝脏之 
位置虽居于右,而其气化实先行于左,试问病患,其左半身必觉有不及右半身处,是其明征也。”询之,果觉 
坐时左半身下坠,卧时不敢向左侧,其父方信愚言,求为疏方。遂用生黄 八钱,柴胡、川芎各一钱,干姜三 
钱,煎汤饮下,须臾左侧即可安卧,又服数剂,诸病皆愈。惟素有带证尚未除,又于原方加牡蛎数钱,服数剂 
带证亦愈。其父复疑而问曰∶“黄 为补肺脾之药,今先生用以补肝,竟能随手奏效,其义何居?”答曰∶ 
“肝属木而应春令,其气温而性喜条达,黄 之性温而上升,以之补肝原有同气相求之妙用。愚自临证以来, 
凡遇肝气虚弱不能条达,用一切补肝之药皆不效,重用黄 为主,而少佐以理气之品,服之复杯即见效验,彼 
谓肝虚无补法者,原非见道之言也。” 
《神农本草经》谓黄 主大风者,诚有其效(参阅“论肢体痿废之原因及治法”中傅××妻治案)。 
《神农本草经》谓黄 主久败疮,亦有奇效。奉天张××,年三十余。因受时气之毒,医者不善为之清解, 
转引毒下行,自脐下皆肿,继又溃烂,睾丸露出,少腹出孔五处,小便时五孔皆出尿。为疏方∶生黄 、花粉 
各一两,乳香、没药、银花、甘草各 
三钱,煎汤连服二十余剂。溃烂之处,皆生肌排脓出外,结疤而愈,始终亦未用外敷生肌之药。 
黄 之性,又善利小便。(参阅曲直汤下王姓治案) 
黄 不但能补气,用之得当,又能滋阴。本村张媪年近五旬,身热劳嗽,脉数至八至,先用六味地黄丸 
加减煎汤服不效,继用左归饮加减亦不效。踌躇再四忽有会悟,改用生黄 六钱,知母八钱,煎汤服数剂,见 
轻,又加丹参、当归各三钱,连服十剂全愈。盖虚劳者多损肾,黄 能大补肺气以益肾水之上源,使气旺自能 
生水,而知母又大能滋肺中津液,俾阴阳不至偏胜,而生水之功益普也。至数剂后,又加丹参、当归者,因血 
痹虚劳,《金匮》合为一门,治虚劳者当防其血有痹而不行之处,故加丹参、当归以流行之也。 
黄 之性,又善治肢体痿废,然须细审其脉之强弱,其脉之甚弱而痿废者,西人所谓脑贫血证也。盖人之 
肢体运动虽脑髓神经司之,而其所以能司肢体运动者,实赖上注之血以涵养之。其脉弱者,胸中大气虚损,不 
能助血上升以养其脑髓神经,遂致脑髓神经失其所司,《内经》所谓“上气不足,脑为之不满”也。拙拟有加 
味补血汤、干颓汤,方中皆重用黄 。凡脉弱无力而痿废者,多服皆能奏效。若其脉强有力而痿废者,西人所 
谓脑充血证,又因上升之血过多,排挤其脑髓神经,俾失所司,《内经》所谓“血菀(同郁)于上,为薄厥” 
也。如此等证,初起最忌黄 ,误用之即凶危立见。迨至用镇坠收敛之品,若拙拟之镇肝熄风汤、建瓴汤治之。 
其脉柔和而其痿废仍不愈者,亦可少用黄 助活血之品以通经络,若服药后,其脉又见有力,又必须仍辅以镇 
坠之品,若拙拟之起痿汤,黄 与赭石、 虫诸药并用也。 
黄 升补之力,尤善治流产、崩带。西傅家庄王××妻,初次受妊,五月,滑下二次,受妊至六七月时, 
觉下坠见血。求为治 
疗,急投以生黄 、生地黄各二两,白术、净萸肉、 龙骨、 牡蛎各一两,煎汤一大碗顿服之,胎气遂安, 
又将药减半,再服一剂以善其后。至期举一男,强壮无恙。 
沈阳朱××,黎明时来院扣门,言其妻因行经下血不止,精神昏愦,气息若无。急往诊视,六脉不全仿佛微 
动,急用生黄 、野台参、净萸肉各一两, 龙骨、 牡蛎各八钱,煎汤灌下,血止强半,精神见复,过数点 
钟将药剂减半,又加生怀山药一两,煎服全愈。 
邑刘氏妇,年二十余,身体羸弱,心中常觉寒凉,下白带甚剧,屡治不效,脉甚细弱,左部尤甚。投以生 
黄、生牡蛎各八钱,干姜、白术、当归各四钱,甘草二钱,数剂全愈。盖此证因肝气太虚,肝中所寄之相火 
亦虚,因而气化下陷,湿寒下注而为白带。故重用黄 以补肝气,干姜以助相火,白术扶土以胜湿,牡蛎收涩 
以固下,更加以当归之温滑,与黄 并用,则气血双补,且不至有收涩太过之弊(在下者因而竭之),甘草之 
甘缓,与干姜并用,则热力绵长,又不至有过热僭上之患,所以服之有捷效也。 
按∶炉心有氢气,人腹中亦有氢气,黄者能引氢气上达于肺,与吸入之氧气相合而化水,又能鼓胃中津 
液上行,又能统摄下焦气化,不使小便频数,故能治消渴。玉液汤、滋 饮,皆治消渴之方,原皆重用黄 。 
黄 入汤剂,生用即是熟用,不必先以蜜炙。若丸散剂中宜熟用者,蜜炙可也。若用治疮疡,虽作丸散, 
亦不宜炙用。王洪绪《外科证治全生集》曾详言之。至于生用发汗、熟用止汗之说,尤为荒唐。盖因气分虚陷 
而出汗者,服之即可止汗,因阳强阴虚而出汗者,服之转大汗汪洋。若气虚不能逐邪外出者,与发表药同服, 
亦能出汗。是知其止汗与发汗不在生、熟,亦视用之者何如耳。 
附录∶ 
柳河仲××来函∶ 
庚午季秋,偶觉心中发凉,服热药数剂无效。迁延旬日,陡觉凉气上冲脑际,顿失知觉,移时始苏。日三 
四发。屡次延医延医不愈。乃病不犯时,心犹清白,遂细阅《衷中参西录》,忽见夫子治坐则左边下坠,睡时不 
敢向左侧之医案,断为肝虚。且谓黄 与肝木有同气相求之妙用,遂重用生黄 治愈。乃恍悟吾睡时亦不能左 
侧,知病源实为肝虚,其若斯之凉者,肝中所寄之相火衰也。爰用生箭 二两,广条桂五钱,因小便稍有不利, 
又加椒目五钱。煎服一剂,病大见愈。遂即原方连服数剂,全愈。 

<目录>二、药物
<篇名>5.山萸肉解
属性:山萸肉∶味酸性温。大能收敛元气,振作精神,固涩滑脱。因得木气最浓,收涩之中兼具条畅之性,故 
又通利九窍,流通血脉,治肝虚自汗,肝虚胁疼腰疼,肝虚内风萌动,且敛正气而不敛邪气,与他酸敛之药不 
同,是以《神农本草经》谓其逐寒湿痹也。其核与肉之性相反,用时务须将核去净,近阅医报有言核味涩,性 
亦主收敛,服之恒使小便不利,椎破尝之,果有有涩味者,其说或可信。 
山茱萸得木气最浓,酸收之中,大具开通之力,以木性喜条达故也。《神农本草经》谓主寒湿痹,诸家 
本草,多谓其能通利九窍,其性不但补肝,而兼能利通气血可知,若但视为收涩之品,则浅之乎视山茱萸矣。 
【附案】一人年四十余,外感痰喘,愚为治愈。但脉浮力微,按之即无。愚曰∶“脉象无根,当服峻补之 
剂,以防意外之变。”病家谓病患从来不受补药,服之则发狂疾,峻补之药,实不敢用。 
愚曰∶“既畏补药如是,备用亦可。”病家根据愚言。迟半日忽发喘逆,又似无气以息,汗出遍体,四肢逆冷, 
身躯后挺,危在顷刻。急用净萸肉四两,爆火煎一沸,即饮下,汗与喘皆微止。又添水再煎数沸饮下,病又见 
愈。复添水将原渣煎透饮下,遂汗止喘定,四肢之厥逆亦回。 
邻村李××,年二十余,素伤烟色,偶感风寒,医者用表散药数剂治愈。间日,忽遍身冷汗,心怔忡异常, 
自言气息将断,急求为调治。诊其脉浮弱无根,左右皆然。愚曰∶“此证虽危易治,得萸肉数两,可保无虞。” 
急取净萸肉四两,人参五钱。先用萸肉二两煎数沸,急服之,心定汗止,气亦接续,又将人参切作小块,用所 
余萸肉煎浓汤,送下病若失。 
一人年四十八,大汗淋漓,数日不止,衾褥皆湿,势近垂危,询方于愚。俾用净萸肉二两,煎汤饮之,汗 
遂止。翌晨,迎愚诊视,其脉沉迟细弱,而右部之沉细尤甚,虽无大汗,遍体犹湿。疑其胸中大气下陷,询之, 
果觉胸中气不上升,有类巨石相压,乃恍悟前次之大汗淋漓,实系大气陷后,卫气无所统摄而外泄也,遂用生 
黄一两,萸肉、知母各三钱,一剂胸次豁然,汗亦尽止,又服数剂以善其后。 
奉天友人田××妻,年五十余,素有心疼证,屡服理气活血之药,未能除根。一日反复甚剧,服药数剂, 
病未轻减。田××见既济汤后,载有张××所治心疼医案,心有会悟,遂用其方加没药、五灵脂各数钱,连服数 
剂全愈,至此二年,未尝反复。由是观之,萸肉诚得木气最浓,故味虽酸敛,而性仍条畅,凡肝气因虚不能条 
畅而作疼者,服之皆可奏效也。 
山萸肉之性,又善治内部血管,或肺络破裂,以致咳血吐血久不愈者(补络补管汤下载有医案宜参观)。 
山萸肉之性,又善熄内风。族家嫂,产后十余日,周身汗出 
不止,且四肢发搐,此因汗出过多而内风动也。急用净萸肉、生山药各二两,俾煎汤服之,两剂愈。 
至外感之邪不净而出汗者,亦可重用山萸肉以敛之。邑张××之子,年十八九,因伤寒服表药太过,汗出 
不止,心中怔忡,脉洪数不实,大便数日未行。为疏方,用净萸肉、生山药、生石膏各一两,知母、生龙骨、 
生牡蛎各六钱,甘草二钱,煎服两剂全愈。 
门生万××,曾治一壮年男子,因屡经恼怒之余,腹中常常作疼。他医用通气、活血、消食、祛寒之药, 
皆不效。诊其脉左关微弱,知系怒久伤肝,肝虚不能疏泄也。遂用净萸肉二两,佐以当归、丹参、柏子仁各数 
钱,连服数剂,腹疼遂愈。后凡遇此等证,投以此方皆效。 
山茱萸之核原不可入药,以其能令人小便不利也。而僻处药坊所卖山茱萸,往往核与肉参半,甚或核多于 
肉。即方中注明去净核,亦多不为去,误人甚矣。斯编重用山茱萸治险证之处甚多。凡用时愚必自加检点,或 
说给病家检点,务要将核去净,而其分量还足,然后不至误事。又山萸肉之功用,长于救脱,而所以能固脱者, 
因其味之甚酸,然间有尝之微有酸味者,此等萸肉实不堪用。用以治险证者,必须尝其味极酸者然后用之,方 
能立建奇效。 

<目录>二、药物
<篇名>6.白术解
属性:白术∶性温而燥,气香不窜,味苦微甘微辛。善健脾胃,消痰水,止泄泻。治脾虚作胀,脾湿作渴,脾弱 
四肢运动无力,甚或作疼。与凉润药同用,又善补肺;与升散药同用,又善调肝;与镇安药同用,又善养心; 
与滋阴药同用,又善补肾,为后天资生之要药。故能于肺、肝、肾、心四脏皆能有所补益也。 
【附案】一妇人年三十许,泄泻半载,百药不效,脉象濡弱,右关尤甚,知其脾胃虚也,俾用生白术轧细 
焙熟,再用熟枣肉 
六两,和为小饼,炉上炙干,当点心服之,细细嚼咽,未尽剂而愈。 
一妇人因行经下血不止,服药旬余无效,势极危殆。诊其脉象浮缓,按之即无,问其饮食不消,大便滑泻。 
知其脾胃虚甚,中焦之气化不能健运统摄,下焦之气化因之不固也。遂于治下血药中加白术一两,生鸡内金一 
两,服一剂血即止,又服数剂以善其后。 
一人年二十二,喘逆甚剧,脉数至七至,投以滋阴兼纳气降气之剂,不效。后于方中加白术数钱,将药煎 
出,其喘促亦至极点,不能服药,将药重温三次,始强服下,一剂喘即见轻,连服数剂全愈。后屡用其方以治 
喘证之剧者,多有效验。 
一少年咽喉常常发干,饮水连连不能解渴。诊其脉微弱迟濡,当系脾胃湿寒,不能健运,以致气化不升也。 
投以四君子汤加干姜、桂枝尖,方中白术重用两许,一剂其渴即止。 

<目录>二、药物
<篇名>7.赭石解
属性:赭石∶色赤,性微凉。能生血兼能凉血,而其质重坠,又善镇逆气,降痰涎,止呕吐,通燥结,用之得当, 
能建奇效。其原质为铁养化合而成,其结体虽坚而层层如铁锈(铁锈亦铁养化合),生研服之不伤肠胃,即服 
其稍粗之末亦与肠胃无损。且生服则养气纯全,大能养血,故《神农本草经》谓其治赤沃漏下;《日华诸家本 
草》谓其治月经不止也。若 用之即无斯效, 之复以醋淬之,尤非所宜。且性甚和平,虽降逆气而不伤 
正气,通燥结而毫无开破,原无需乎 也。其形为薄片,迭迭而成,一面点点作凸形,一面 
点点作凹形者,方堪入药。 
赭石为铁养化合,性同铁锈,原不宜 。徐灵胎谓若 之复用醋淬,即能伤肺。此书诸方中有赭石者, 
皆宜将生赭石轧细用之。 
【附案】邻村迟某,年四十许,当上脘处发疮,大如核桃,破后调治三年不愈。疮口大如钱,自内溃烂, 
循胁渐至背后,每日 
自背后排挤至疮口流出脓水若干。求治于愚,自言患此疮后三年未尝安枕,强卧片时,即觉有气起自下焦, 
上逆冲心。愚曰∶“此即子疮之病根也。”俾用生芡实一两,煮浓汁送服生赭石细末五钱,遂可安卧。又 
服数次,彻夜稳睡。盖气上逆者乃冲气之上冲,用赭石以镇之,芡实以敛之,冲气自安其宅也。继用活络 
效灵丹,加生黄 、生赭石各三钱煎服,日进一剂,半月全愈。 
邻村毛姓少年,于伤寒病瘥后,忽痰涎上壅,杜塞咽喉,几不能息。其父知医,用手大指点其天突穴 
(宜指甲贴喉,指端着穴,向下用力,勿向内用力),息微通,急迎愚调治。遂用香油二两,炖热调麝香一 
分灌之,旋灌旋即流出痰涎若干。继用生赭石一两,人参六钱,苏子四钱,煎汤,徐徐饮下,痰涎顿开。 
周姓妇,年三十许,连连呕吐,五六日间,勺水不存,大便亦不通行,自觉下脘之处疼而且结,凡 
药之有味者入口即吐,其无味者须臾亦复吐出,医者辞不治。后愚诊视其脉有滑象,上盛下虚,疑其有妊, 
询之月信不见者五十日矣,然结证不开,危在目前,《内经》谓“有故无殒亦无殒也”。遂单用赭石二两,煎 
汤饮下,觉药至结处不能下行,复返而吐出。继用赭石四两,又 
重罗出细末两许,将余三两煎汤,调细末服下,其结遂开,大便亦通,自此安然无恙,至期方产。 
或问∶赭石,《名医别录》谓其坠胎,今治妊妇竟用赭石如此之多,即幸而奏效,岂非行险之道乎?答 
曰∶愚生平治病,必熟筹其完全而后为疏方,初不敢为孤注之一掷也。赭石质重,其镇坠之力原能下有形 
滞物,若胎至六七个月时,服之或有妨碍,至受妊之初,因恶阻而成结证,此时其胞室之中不过血液凝结, 
赭石毫无破血之弊,且有治赤沃与下血不止之效,重用之亦何妨乎?况此证五六日间,勺饮不能下行,其 
气机之上逆,气化之壅滞,已至极点,以赭石以降逆开壅,不过调脏腑之气化使之适得 
其平,又何至有他虞乎? 
广平县吕××妻,年二十余,因恶阻呕吐甚剧。九日之间饮水或少存,食物则尽吐出。延为诊视,脉象有 
力,舌有黄苔,询其心中发热,知系夹杂外感,遂先用生石膏两半,煎汤一茶杯,防其呕吐,徐徐温饮下, 
热稍退。继用生赭石二两,煎汤一大茶杯,分两次温饮下,觉行至下脘作疼,不复下行转而上逆吐出, 
知其下脘所结甚坚,原非轻剂所能通。亦用生赭石细末四两,从中再罗出极细末一两,将余三两煎汤, 
送服其极细末,其结遂开,从此饮食顺利,及期而产。 
一室女,中秋节后,感冒风寒,三四日间,胸膈满闷,不受饮食,饮水一口亦吐出,剧时恒以手自 
挠其胸。脉象滑实,右部尤甚,遂单用生赭石细末两半,俾煎汤温饮下,顿饭顷仍吐出。 
盖其胃口皆为痰涎壅滞,药不胜病,下行不通复转而吐出也。遂更用赭石四两,煎汤一大碗,分三次陆续 
温饮下,胸次遂通,饮水不吐。翌日,脉象洪长,其舌苔从先微黄,忽变黑色,又重用白虎汤 
连进两大剂,每剂用生石膏四两,分数次温饮下,大便得通而愈。 
曲××,年三十余,得瘟病,两三日恶心呕吐,五日之间饮食不能下咽,来院求为延医。其脉浮弦,数 
近六至,重按无力,口苦心热,舌苔微黄。因思其脉象浮弦者,阳明与少阳合病也,二经之病机相并上冲, 
故作呕吐也,心热口苦者,内热已实也,其脉无力而数者,无谷气相助又为内热所迫也。因思但用生赭石 
煮水饮之,既无臭味,且有凉镇之力,或可不吐。遂用生赭石二两,煎水两茶杯,分二次温饮下,饮完仍复 
吐出,病患甚觉惶恐,加以久不饮食,形状若莫可支持。愚曰∶“无恐,再用药末数钱,必能立止 
呕吐。”遂单用生赭石细末五钱,开水送服,觉恶心立止,须臾胸次通畅,进薄粥一杯,下行顺利。从 
此饮食不复呕吐,而心中犹发热,舌根肿胀,言语不利,又用生石膏一 
两,丹参、乳香、没药、连翘各三钱,连服两剂全愈。 
癸亥秋,李××之族姊,周身灼热,脉象洪实,心中烦躁怔忡,饮食下咽即呕吐,屡次所服之药,亦皆 
呕吐不受。视其舌苔黄浓,大便数日未行,知其外感之热已入阳明之府,又挟胃气上逆,冲气上冲也。为疏 
方,用生赭石细末八钱,生石膏细末两半,蒌仁一两,玄参、天冬各六钱,甘草二钱,将后五味煎汤一大 
茶杯,先用开水送服赭石细末,继将汤药服下,遂受药不吐,再服一剂全愈。 
一妇人,年近五旬,得温病,七八日表里俱热,舌苔甚薄作黑色,状类舌斑,此乃外感兼内亏之证。医 
者用降药两次下之,遂发喘逆。令其子两手按其心口,即可不喘。须臾又喘,又令以手紧紧按住,喘又 
少停。诊其脉尺部无根,寸部摇摇,此将脱之候也。时当仲夏,俾用生鸡子黄四枚,调新汲井泉水服之,喘稍 
定。可容取药。遂用赭石细末二钱同生鸡子黄二枚,温水调和服之,喘遂愈,脉亦安定。继服 
参赭镇气汤以善其后。 
友人毛××曾治一妇人,胸次郁结,饮食至胃不能下行,时作呕吐。毛××用赭石细末六钱,浓煎人 
参汤送下,须臾腹中如爆竹之声,胸次胃中俱觉通豁,至此饮食如常。 
友人高××曾治一人,上焦满闷,艰于饮食,胸中觉有物窒塞。医者用大黄、蒌实陷胸之品十余剂,转 
觉胸中积满,上至咽喉,饮水一口即溢出。高××用赭石二两,人参六钱为方煎服, 
顿觉窒塞之物降至下焦。又加当归、肉苁蓉,再服一剂,降下瘀滞之物若干,病若失。 
友人李××曾治一人,寒痰壅滞胃中,呕吐不受饮食,大便旬日未行。用人参八钱、干姜六钱、赭石 
一两,一剂呕吐即止。又加当归五钱,大便得通而愈。 
门人高××曾治一叟,年七十余,得呃逆证,兼小便不通,剧时觉杜塞咽喉,息不能通,两目上翻,身躯后挺,更 
医数人治 
不效。高××诊其脉浮而无力,遂用赭石、台参、生山药、生芡实、牛蒡子为方投之,呃逆顿愈。又加竹茹服一剂, 
小便亦通利。 
历观以上诸治验案,赭石诚为救颠扶危之大药也。乃如此良药,今人罕用,间有用者,不过二三钱,药不胜 
病,用与不用同也。且愚放胆用至数两者,非卤莽也。诚以临证既久,凡药之性情能力及宜轻宜重之际, 
研究数十年,心中皆有定见,而后敢如此放胆,百用不至一失。且赭石所以能镇逆气,能下有形瘀滞 
者,以其饶有重坠之力,于气分实分毫无损。况气虚者又佐以人参,尤为万全之策也。 
参、赭并用,不但能纳气归原也,设于逆气上干,填塞胸臆,或兼呕吐,其证之上盛下虚者,皆可参、 
赭并用以治之。 
《内经》谓阳明厥逆、喘咳、身热、善惊、衄、呕血。黄坤载衍《内经》之旨,谓“血之失于便溺者,太 
阴之不升也;亡于吐衄者,阳明之不降也。”是语,深明《内经》者也。盖阳明胃气,以息息下降为顺, 
时或不降,则必壅滞转而上逆,上逆之极,血即随之上升而吐衄作矣。治吐衄之证,当以降胃为主, 
而降胃之药,实以赭石为最效。然胃之所以不降,有因热者,宜降之以赭石,而以蒌仁、白芍诸药佐之; 
其热而兼虚者,可兼佐以人参;有因凉者,宜降以赭石而以干姜、白芍诸药佐之(因凉犹用白芍者,防干姜之热 
侵肝胆也,然吐衄之证,由于胃气凉而不降者甚少);其凉而兼虚者,可兼佐以白术;有因下焦虚损,冲 
气不摄上冲胃气不降者,宜降以赭石而以生山药、生芡实诸药佐之,有因胃气不降,致胃中血管破裂, 
其证久不愈者,宜降以赭石而以龙骨、牡蛎、三七诸药佐之。无论吐衄之证,种种病因不同,疏方皆以 
赭石为主,而随证制宜,佐以相当之药品,吐衄未有不愈者。 
奉天张××,年近四旬,陡然鼻中衄血甚剧,脉象关前洪 
滑,两尺不任重按,知系上盛下虚之证,自言头目恒不清爽,每 
睡醒舌干无津,大便甚燥,数日一行。为疏方∶赭石、生地黄、生山药各一两,当归、白芍、生龙骨、生 
牡蛎、怀牛膝各五钱,煎汤送服旱三七细末二钱(凡用生地治吐衄者,皆宜佐以三七,血止后不至淤血留于经络 
),一剂血顿止。后将生地减去四钱,加熟地、枸杞各五钱,连服数剂,脉亦平和。 
伤寒下早成结胸,瘟疫未下亦可成结胸。所谓结胸者,乃外感之邪与胸中痰涎互相凝结,滞塞 
气道,几难呼吸也。仲景有大陷胸汤、丸,原为治此证良方,然因二方中皆有甘遂,医者不敢轻 
用,病家亦不敢轻服,一切利气理痰之药,又皆无效,故恒至束手无策。向愚治此等证,俾用新炒蒌仁四 
两,捣碎煮汤服之,恒能奏效。后拟得一方,用赭石、蒌仁各二两,苏子六钱(方名荡胸汤), 
用之代大陷胸汤、丸,屡试皆能奏效。若其结在胃口,心下满闷,按之作疼者,系小陷胸汤证,又可将方 
中分量减半以代小陷胸汤,其功效较小陷胸汤尤捷。自拟此方以来,救人多矣,至寒温之证已 
传阳明之府,却无大热,惟上焦痰涎壅滞,下焦大便不通者,亦可投以此方(分量亦宜斟酌少用),上清其 
痰,下通其便,诚一举两得之方也。 
至寒温之证,不至结胸及心下满闷,惟逆气挟胃热上冲,不能饮食,并不能受药者,宜赭石与 
清热之药并用。曾治奉天安××之妻,年四十余,临产双生,异常劳顿,恶心呕吐,数日不能 
饮食,服药亦恒呕吐,精神昏愦,情势垂危,群医辞不治。延愚诊视,其脉洪实,面有火,舌苔黄 
浓,知系产后温病,其呕吐若是者,阳明府热已实,胃气因热而上逆也。遂俾用玄参两半,赭 
石一两,同煎服,一剂即热退呕止,可以受食。继用玄参、白芍、连翘以清其余热,病遂全愈。至放胆 
用玄参而无所顾忌者,以玄参原宜于产乳,《神农本草经》有明文也。 
下有实寒,上有浮热之证,欲用温热之药以祛其寒,上焦恒格拒不受,惟佐以赭石使之速于下行, 
直达病所,上焦之浮热转能因之下降。曾治邻村刘某,因房事后恣食生冷,忽然少腹抽 
疼,肾囊紧缩,大便不通,上焦兼有烦热。医者投以大黄附子细辛汤,上焦烦热益甚,两胁疼胀,便结囊缩, 
腹疼如故。病家甚觉惶恐,求为诊视。其脉弦而沉,两尺之沉尤甚,先用醋炒葱白熨其脐及脐下,腹 
中作响,大有开通之意,囊缩腹疼亦见愈,便仍未通。遂用赭石二两,乌附子五钱,当归、苏子各一两, 
煎汤饮下,即觉药力下行,过两句钟俾煎渣饮之,有顷,降下结粪若干,诸病皆愈。 
膈食之证,千古难治之证也。《伤寒论》有旋复代赭石汤,原治伤寒发汗,若吐若下解后,心下痞硬、噫气 
不除者。周扬俊、喻嘉言皆谓治膈证甚效。然《神农本草经》谓旋复花味咸,若真好旋复花实 
咸而兼有辛味(敝邑武帝台污所产旋复花咸而辛),今药坊间所鬻旋复花皆甚苦,实不堪用。是以愚治膈 
证,恒用其方去旋复花,将赭石加重,其冲气上冲过甚,兼大便甚干结者,赭石恒用至两许,再加当归、柿 
霜、天冬诸药以润燥生津,且更临时制宜,随证加减,治愈者不胜录。盖此证因胃气衰弱,不能撑悬贲门, 
下焦冲气又挟痰涎上冲,以杜塞之,是以不受饮食。故用人参以壮胃气,气壮自能撑悬贲门,使之 
宽展;赭石以降冲气,冲降自挟痰涎下行,不虑杜塞,此方之所以效也。若药局间偶有咸而且辛之旋复花,亦 
可斟酌加入,然加旋复花又须少减赭石也。此证有因贲门肿胀,内有瘀血致贲门窄小者,宜于方 
中加苏木、 虫(俗名土鳖)各二钱。 
头疼之证,西人所谓脑气筋病也。然恒可重用赭石治愈。近在奉天曾治何姓女,年二十余岁,每日至巳 
头疼异常,左边尤甚,过午则愈。先经东人治之,投以麻醉脑筋之品不效。后求为诊视, 
其左脉浮弦有力者,系少阳之火挟心经之热,乘阳旺之时而上升 
以冲突脑部也。为疏方∶赭石、龙骨、牡蛎、龟板、萸肉、白芍各六钱,龙胆草二钱,药料皆用生者, 
煎服一剂,病愈强半,又服两剂全愈。隔数日,又治鞠姓妇,头疼亦如前状,仍投以此方两剂全愈。 
癫狂之证,亦西人所谓脑气筋病也,而其脑气筋之所以病 
者,因心与脑相通之道路(心有四支血脉管通脑)为痰火所充塞也。愚恒重用赭石二两,佐以大黄、朴硝、半 
夏、郁金,其痰火甚实者,间或加甘遂二钱(为末送服),辄能随手奏效。诚以赭石重坠之力,能引痰火下行, 
俾心脑相通之路毫无滞碍,则脑中元神,心中识神自能相助为理,而不至有神明瞀乱之时也。在奉天曾治一人, 
年近三旬,癫狂失心,屡经中、西医治疗,四载分毫无效。来院求为延医,其脉象沉实,遂投以上所拟方, 
每剂加甘遂二钱五分,间两日一服(凡药中有甘遂,不可连服)。其不服汤药之二日,仍用赭石、朴硝细末各 
五钱,分两次服下,如此旬余而愈。 
痫疯之证,千古难治之证也。西人用麻醉脑筋之品,日服数次,恒可强制不发,然亦间有发时,且服之累 
年不能除根,而此等药常服,又有昏精神、减食量之弊。庚申岁,在奉天因延医此等证,研究数方,合用之, 
连治数人皆愈。一方用赭石六钱,于术、酒曲(用神曲则无效且宜生用)、半夏、龙胆草、生明没药各三钱, 
此系汤剂;一方用真黑铅四两,铁锅内熔化,再加硫黄细末二两,撒于铅上,硫黄皆着,急用铁铲拌炒之,铅 
经硫黄烧炼,皆成红色,因拌炒结成砂子,取出凉冷,碾轧成饼者(系未化透之铅)去之,余者再用乳钵研极 
细末,搀朱砂细末与等分,再少加蒸熟麦面(以仅可作丸为度),水和作丸,半分重(干透足半分);一方用 
西药臭剥、臭素、安母纽谟各二钱,抱水过鲁拉尔一钱,共研细搀蒸熟麦面四钱,水和为丸,桐子大。上药早 
晚各服西药十四瓦,午时服铅硫朱砂丸十二丸,日服药三次,皆煎汤剂送下,汤药一剂可煎三次,以递送三次 
所服丸药,如此服药月余,痫风可以除根。《内经》云∶“诸风掉眩,皆属于肝”,肝经风火挟痰上冲,遂致 
脑气筋顿失其所司,周身抽掣,知觉全无,赭石含有铁质,既善平肝,而其降逆之力又能协同黑铅、朱砂以坠 
痰镇惊,此其所以效 
也。而必兼用西药者,因臭剥、臭素诸药,皆能强制脑筋以治病之标,俾目前不至反复,而后得徐以健脾、利 
痰、祛风、清火之药以铲除其病根也。 
人之廉于饮食者,宜补以健脾之药,而纯用健补脾脏之品,恒多碍于胃气之降,致生胀满,是以补脾者宜 
以降胃之药佐之,而降胃之品又恒与气分虚弱者不宜。惟赭石性善降胃,而分毫不伤气分,且补药性多温,易 
生浮热,赭石性原不凉,而能引热下行(所以诸家本草多言其性凉)。是以愚习用赭石,不但以之降胃也,凡 
遇有虚热之证,或其人因热痰嗽,或其人因热怔忡,但问其大便不滑泻者,方中加以赭石,则奏效必速也。 
内中风之证,忽然昏倒不省人事,《内经》所谓“血之与气并走于上”之大厥也。亦即《史记》扁鹊传 
所谓“上有绝阳之络下有破阴之纽”之尸厥也。此其风非外来,诚以肝火暴动与气血相并,上冲脑部(西人剖 
验此证谓脑部皆有死血,或兼积水),惟用药镇敛肝火,宁熄内风,将其上冲之气血引还,其证犹可挽回,此 
《金匮》风引汤所以用龙骨、牡蛎也。然龙骨、牡蛎,虽能敛火熄风,而其性皆涩,欠下达之力,惟佐以赭石 
则下达之力速,上逆之气血即可随之而下。曾治奉天杜××,忽然头目眩晕,口眼歪斜,舌强直不能发言,脉象 
弦长有力,左右皆然,视其舌苔白浓微黄,且大便数日不行,知其证兼内外中风也。俾先用阿斯匹林瓦半,白 
糖水送下以发其汗,再用赭石、生龙骨、生牡蛎、蒌仁各一两,生石膏两半,菊花、连翘各二钱,煎汤,趁其 
正出汗时服之,一剂病愈强半,大便亦通。又按其方加减,连服数剂全愈。 
邻村韩姓媪,年六旬。于外感病愈后,忽然胸膈连心下突胀,腹脐塌陷,头晕项强,妄言妄见,状若疯 
狂,其脉两尺不见,关前摇摇无根,数至六至,此下焦虚惫,冲气不摄,挟肝胆浮热上干脑部乱其神明也。遂 
用赭石、龙骨、牡蛎、山药、地黄(皆用生者) 
各一两,野台参、净萸肉各八钱,煎服一剂而愈。又少为加减再服一剂以善其后。 
邻村刘××,年三十许,因有恼怒,忽然昏倒不省人事,牙关紧闭,唇齿之间有痰涎随呼气外吐,六脉闭塞 
若无。急用作嚏之药吹鼻中,须臾得嚏,其牙关遂开。继用香油两余,炖温调麝香末一分,灌下,半句钟时稍 
醒悟能作呻吟,其脉亦出,至数五至余,而两尺弱甚,不堪重按。知其肾阴亏损,故肝胆之火易上冲也。遂用 
赭石、熟地、生山药各一两,龙骨、牡蛎、净萸肉各六钱,煎服后豁然顿愈。继投以理肝补肾之药数剂,以善 
其后。 
按∶此等证,当痰火气血上壅之时,若人参、地黄、山药诸药,似不宜用,而确审其系上盛下虚,若扁鹊 
传所云云者,重用赭石以辅之,则其补益之力直趋下焦,而上盛下虚之危机旋转甚速,莫不随手奏效也。 
附录∶ 
直隶青县张××来函∶ 
族嫂年三十余岁,身体甚弱,于季春忽患头疼,右边疼尤剧,以致上下眼睑皆疼,口中时溢涎沫,唾吐满 
地,经血两月未见,舌苔粘腻,左脉弦硬而浮,右脉沉滑。知系气血两虚,内有蕴热,挟肝胆之火上冲头目, 
且有热痰杜塞中焦也。为疏方用赭石解下所载治何姓女之方加减,生赭石细末六钱,净山萸肉五钱,野台参、 
生杭芍、生龟板、当归身各三钱。一剂左边疼顿减,而右边之疼如故。遂用前方加丹皮二钱,赭石改用八钱。 
服后不但头疼悉愈,且口内涎沫亦无,惟月经仍未见,又改用赭石至一两,加川芎二钱。服下,翌日月事亦通。 
夫赭石向在药物中为罕用之品,而此方用之以治头疼,以治痰涎杜塞,以治月事不见,皆能随手奏效,实赭石 
之力居多。 

<目录>二、药物
<篇名>8.山药解
属性:山药∶色白入肺,味甘归脾,液浓益肾。能滋润血脉,固摄气化,宁嗽定喘,强志育神,性平可以常服多 
服。宜用生者煮汁饮之,不可炒用,以其含蛋白质甚多,炒之则其蛋白质焦枯,服之无效。若作丸散,可轧细 
蒸熟用之(医方篇一味薯蓣饮后,附有用山药治愈之验案数则可参观)。 
【附案】法库万××之母,自三十余岁时,即患痰喘咳嗽,历三十年百药不效,且年愈高,病亦愈进,至 
一九二一年春,又添发烧、咽干、头汗出、食不下等证。延医诊视,云是痰盛有火,与人参清肺汤加生地、丹 
皮等味,非特无效,反发热如火,更添泄泻,有不可终日之势。后忽见《衷中参西录》一味薯蓣饮,遂用生 
怀山药四两,加玄参三钱,煎汤一大碗,分数次徐徐温服,一剂即见效,至三剂病愈强半,遂改用生怀山药细 
末一两,煮作粥服之,日两次,间用开胃药,旬余而安,宿病亦大见轻,大约久服宿病亦可除根。又∶万××妻, 
大便泄泻数年不愈,亦服山药粥而愈。 
按∶山药之功效,一味薯蓣饮后曾详言之。至治泄泻,必变饮为粥者,诚以山药汁本稠粘,若更以之作 
粥,则稠粘之力愈增,大有留恋肠胃之功也。忆二十年前,岁试津门,偶患泄泻,饮食下咽,觉与胃腑不和, 
须臾肠中作响,遂即作泻。浓煎甘草汤,调赤石脂细末,服之不效。乃用白粳米,慢火煮烂熟作粥,尽量食之, 
顿觉脾胃舒和,腹中亦不作响,泄泻遂愈。是知无论何物作粥,皆能留恋肠胃。而山药性本收涩,故煮粥食之, 
其效更捷也。且大便溏泻者,多因小便不利。山药能滋补肾经,使肾阴足,而小便自利,大便自无溏泻之患。 

<目录>二、药物
<篇名>9.地黄解
属性:鲜地黄∶性寒,微苦微甘。最善清热、凉血、化瘀血、生新 
血,治血热妄行、吐血、衄血、二便因热下血。其中含有铁质,故晒之、蒸之则黑,其生血、凉血之力,亦赖 
所含之铁质也。 
干地黄(即药局中生地黄)∶经日晒干,性凉而不寒,生血脉,益精髓,聪明耳目,治骨蒸劳热,肾虚生热。 
熟地黄(用鲜地黄和酒,屡次蒸晒而成)∶其性微温,甘而不苦,为滋阴补肾主药。治阴虚发热,阴虚不 
纳气作喘,劳瘵咳嗽,肾虚不能漉水,小便短少,积成水肿,以及各脏腑阴分虚损者,熟地黄皆能补之。 
【附案】邻村李媪,年七旬,劳喘甚剧,十年未尝卧寝。俾每日用熟地煎汤当茶饮之,数日即安卧,其 
家人反惧甚,以为如此改常,恐非吉兆,而不知其病之愈也。 
侯××之子,五岁,因服凉泻之药太过,致成慢惊,胃寒吐泻,常常螈 ,精神昏愦,目睛上泛,有危 
在顷刻之象。为处方,用熟地黄二两,生山药一两,干姜、附子、肉桂各二钱,净萸肉、野台参各三钱,煎汤 
一杯半,徐徐温饮下,吐泻螈 皆止,精神亦振,似有烦躁之意,遂去干姜加生杭芍四钱,再服一剂全愈。 
一童子,年十四五,伤寒已过旬日,大便滑泻不止,心中怔忡异常,似有不能支持之状。脉至七至,按 
之不实。医者辞不治。投以熟地、生山药、生杭芍各一两,滑石八钱,甘草五钱,煎汤一大碗,徐徐温饮下, 
亦尽剂而愈。 
统观以上诸案,冯氏谓地黄大补肾中元气之说,非尽无凭。盖阴者阳之守,血者气之配,地黄大能滋阴 
养血,大剂服之,使阴血充足,人身元阳之气,自不至上脱下陷也。 
用熟地治寒温,恒为医家所訾。然遇其人真阴太亏,不能支持外感之热者,于治寒温药中,放胆加熟地 
以滋真阴,恒能挽回人命于顷刻。曾治一室女,资禀素羸弱,得温病五六日,痰喘甚剧。治以《金匮》小青龙 
汤加石膏,一剂喘顿止。时届晚八点钟,一夜安稳。至寅时喘复作,不若从前之剧,而精神恍惚,心 
中怔忡。再诊其脉,如水上浮麻不分至数,按之即无,此将脱之候也。取药不暇,幸有预购山药两许,急煎服 
之,病少愈。此际已疏方取药,方系熟地四两、生山药一两、野台参五钱。而近处药局无野台参,并他参亦鬻 
尽。再至他处,又恐误事。遂单煎熟地、山药饮之,病愈强半。一日之内,按其方连进三剂,病遂全愈。 
按∶此证原当用拙拟来复汤,其方重用山萸肉以收脱。而当时愚在少年,其方犹未拟出,亦不知重用萸 
肉。而自晨至暮,共服熟地十二两,竟能救此垂危之证,熟地之功用诚伟哉。又此证初次失处,在服小青龙汤 
后,未用补药。愚经此证后,凡遇当用小青龙汤而脉稍弱者,服后即以补药继之。或加人参于汤中,恐其性热, 
可将所加之石膏加重。 
《张氏八阵》、赵氏《医贯》、《冯氏锦囊》皆喜重用熟地,虽外感证,亦喜用之。其立言诚有偏处。 
然当日必用之屡次见效,而后笔之于书。张氏书中载有∶治一老年伤寒,战而不汗,翌日届其时,犹有将汗之 
意。急与一大剂八味地黄汤以助其汗。服后,遂得大汗,阅数时周身皆凉,气息甚微,汗犹不止。精神昏昏, 
复与原汤一剂,汗止而精神亦复。夫用其药发汗,即用其药止汗,运用之妙,颇见慧心。又赵氏书中谓∶六味 
地黄汤能退寒温之实热,致贻后世口实。然其言亦非尽不验。忆昔乙酉、丙戌数年间之寒温病,热入阳明府后, 
凡于清解药中,能重用熟地以滋阴者,其病皆愈。此乃一时气运使然,不可笔之于书以为定法也。 
又∶冯氏所着本草,谓熟地能大补肾中元气,此亦确论。凡下焦虚损,大便滑泻,服他药不效者,单服 
熟地即可止泻。然须日用四五两,煎浓汤服之亦不作闷(熟地少用则作闷多用转不闷),少用则无效。至陈修 
园则一概抹倒,直视熟地为不可用,岂能知熟地哉。寒温传里之后,其人下焦虚惫太甚者,外邪恒直趋下焦作 
泄泻,亦非重用熟地不能愈。癸巳秋,一女年三十许,得温病,十余日,势至 
垂危,将舁于外。同坐贾××谓愚知医,主家延为诊视。其证昼夜泄泻,昏不知人,呼之不应,其脉数至七至, 
按之即无。遂用熟地黄二两,生山药、生杭芍各一两,甘草三钱,煎汤一大碗,趁温徐徐灌之,尽剂而愈。 


\chapter{甘草解}
属性:(附∶甘草反鲢鱼之质疑) 
甘草∶性微温,其味至甘。能解一切毒性。甘者主和,故有调和脾胃之功,甘者主缓,故虽补脾胃而实非 
峻补。炙用则补力较大,是以方书谓胀满证忌之。若轧末生服,转能通利二便,消胀除满。若治疮疡亦宜生用, 
或用生者煎服亦可。仲景有甘草泻心汤,用连、苓、半夏以泻心下之痞,即用甘草以保护心主,不为诸药所伤 
损也。至白虎汤用之,是借其甘缓之性以缓寒药之侵下。通脉汤、四逆汤用之,是借其甘缓之性,以缓热药之 
僭上。与芍药同用,能育阴缓中止疼,仲景有甘草芍药汤。与干姜同用,能逗留其热力使之绵长,仲景有甘草 
干姜汤。与半夏、细辛诸药同用,能解其辛而且麻之味,使归和平。惟与大戟、芫花、甘遂、海藻相反,余药 
则皆相宜也。 
古方治肺痈初起,有单用粉甘草四两,煮汤饮之者,恒有效验。愚师其意,对于肺结核之初期,咳嗽吐 
痰,微带腥臭者,恒用生粉甘草为细末,每服钱半,用金银花三钱煎汤送下,日服三次,屡屡获效。若肺病已 
久,或兼吐脓血,可用粉甘草细末三钱,浙贝母、三七细末各钱半,共调和为一日之量,亦用金银花煎汤送下。 
若觉热者,可再加玄参数钱,煎汤送服。皮黄者名粉甘草,性平不温,用于解毒清火剂中尤良。 
愚拟治霍乱两方,一为急救回生丹,一为卫生防疫宝丹,二方中皆重用甘草,则甘草之功用可想也。然 
亦多赖将甘草轧细生用,未经蜜炙、水煮耳。诚以暴病传染,皆挟有毒瓦斯流行,生用 
则其解毒之力较大,且甘草熟用则补,生用则补中仍有流通之力,故于霍乱相宜也。至于生用能流通之说,可 
以事实征之。 
【附案】开原王姓幼童,脾胃虚弱,饮食不能消化,恒吐出,且小便不利,周身漫肿,腹胀大,用生甘草 
细末与西药百布圣各等分,每服一钱,日三次,数日吐止便通,肿胀皆消。 
铁岭友人魏××,其地多甘草,魏××日以甘草置茶壶中当茶叶冲水饮之,旬日其大小便皆较勤,遂不敢饮。 
后与愚觌面,为述其事,且问甘草原有补性,何以通利二便?答曰∶“甘草熟用则补,生用则通,以之置茶壶 
中虽冲以开水,其性未熟,仍与生用相近故能通也。” 
门生李××言,曾有一孺子患腹疼,用暖脐膏贴之,后其贴处溃烂,医者谓多饮甘草水可愈。复因饮甘草水 
过多,小便不利,身肿腹胀,再延他医治之,服药无效。其地近火车站,火车恒装卸甘草,其姊携之拾甘草嚼 
之,日以为常,其肿胀竟由此而消。观此,则知甘草生用、熟用,其性竟若是悬殊,用甘草者,可不于生、熟 
之间加之意乎? 


<篇名>附∶甘草反鲢鱼之质疑
属性:近阅《遁园医案》(长沙萧琢如着)载鲢鱼反甘草之事。谓当逊清末叶,医士颜××笃实人也,一日告余, 
曾在某邑为人治病,见一奇案,令人不解。有一农家人口颇众,冬月塘涸取鱼,煮食以供午餐,丁壮食鱼且尽, 
即散而赴工。妇女童稚数人复取鱼烹治佐食。及晚,有一妇初觉饱闷不适,卧床歇息,众未介意。次日呼 
之不起,审视则已僵矣。举家惊讶,莫明其故。再四考查,自进午餐后并未更进他种食物,亦无纤芥事故,乃 
取前日烹鱼之釜细察视之,除鱼汁骨肉外,惟存甘草一条约四五寸许。究问所来,据其家妇女云,小孩啼哭每 
以甘草与食,釜中所存必系小儿所遗 
落者。又检所烹之鱼,皆系鲢鱼,并非毒物。且甘草亦并无反鲢鱼之说,矧同食者若干人,何独一人偏受其灾。 
顷刻邻里咸集。又久之,其母家亦至。家人据实以告众,一少年大言于众曰∶“甘草鲢鱼同食毙命,千古无此 
奇事,岂得以谎言搪塞?果尔,则再用此二物同煮,与我食之。”言已,即促同来者照办,并亲自手擎二物置 
釜中。烹熟,取盘箸陈列席间,旁人疑阻者辄怒斥之,即席大啖,并笑旁观者愚暗胆怯。届晚间固无甚痛苦, 
亦无若何表示,至次晨则僵卧不起矣。由斯其母家嫌疑解释。按鲢鱼为常食之物,甘草又为药中常用之品,苟 
此二物相反,疏方用甘草时即当戒其勿食鲢鱼。 

\chapter{朱砂解}
属性:朱砂∶味微甘性凉,为汞五硫一化合而成。性凉体重,故能养精神、安魂魄、镇惊悸、熄肝风;为其色 
赤入心,能清心热,使不耗血,故能治心虚怔忡及不眠;能消除毒菌,故能治暴病传染、霍乱吐泻;能入肾导引肾 
气上达于心,则阴阳调和,水火既济,目得水火之精气以养其瞳子,故能明目;外用之,又能敷疮疡疥癞诸毒。 
邹润安曰∶凡药所以致生气于病中,化病气为生气也。凡用药取其禀赋之偏,以救人阴阳之偏胜也。是 
故药物之性,未有不偏者。徐洄溪曰∶药之用,或取其气,或取其味,或取其色,或取其形,或取其质,或取 
其性情,或取其所生之时,或取其所成之地。愚谓∶丹砂,则取其质与气与色为用者也。质之刚是阳,内含汞 
则阴气之寒是阴,色纯赤则阳,故其义为阳抱阴,阴承阳,禀自先天,不假作为。人之有生以前,两精相搏即 
有神,神根据于精乃有气,有气而后有生,有生而后知识具以成其魂,鉴别昭以成其魄,故凡精气失其所养,则 
魂魄遂不安,欲养之安之,则舍阴阳紧相抱持,密相承接之丹砂又奚取乎?然谓主身体五脏 
百病,养精神,安魂魄,益气明目何也?夫固以气寒,非温煦生生之具,故仅能于身体五脏百病中,养精神安 
魂魄益气明目耳。若身体五脏百病中,其不必养精神安魂魄益气明目者,则不必用丹砂也。血脉不通者,水中 
之火不继续也,烦满消渴者,火中之水失滋泽也,中恶腹痛阴阳不相保抱,邪得乘间以入,毒瓦斯疥 诸疮,阳 
不畜阴而反灼阴,得惟药之阳抱阴,阴涵阳者治之,斯阳不为阴贼,阴不为阳累,诸疾均可已矣。按此为邹氏 
释《神农本草经》之文,可谓精细入微矣。 
【附案】壬寅秋月,霍乱流行。友人毛××之侄,受此证至垂危,衣冠既毕,舁之床上。毛××见其仍有微息, 
遂研朱砂钱许,和童便灌之,其病由此竟愈。 
一女子得霍乱至垂危,医者辞不治,时愚充教员于其处,求为延医,亦用药无效。适有摇铃卖药者,言能 
治此证,亦单重用朱砂钱许,治之而愈。愚从此知朱砂善化霍乱之毒菌。至己未在奉天拟得急救回生丹、卫生 
防疫宝丹两方,皆重用朱砂,治愈斯岁之患霍乱者不胜纪,传之他省亦救人甚伙,可征朱砂之功效神奇矣。然 
须用天产朱砂方效,若人工所造朱砂(色紫成大块作锭形者,为人工所造朱砂),止可作颜料用,不堪入药。 

<目录>二、药物
<篇名>12.鸦胆子解
属性:鸦胆子∶俗名鸭蛋子,即苦参所结之子。味极苦,性凉。为凉血解毒之要药,善治热性赤痢(赤痢间有凉者), 
二便因热下血,最能清血分之热及肠中之热,防腐生肌,诚有奇效。愚生平用此药治愈至险之赤痢不胜纪,用 
时去皮,每服二十五粒,极多至五十粒,白糖水送下。 
鸭蛋子味甚苦,服时若嚼破,即不能下咽。若去皮时破者,亦不宜服。恐服后若下行不速,或作恶心呕吐。 
故方书用此药, 
恒以龙眼肉包之,一颗龙眼肉包七数,以七七之数为剂。然病重身强者,犹可多服,常以八八之粒为剂。然亦 
不必甚拘。 
鸭蛋子连皮捣细,醋调,敷疔毒甚效,立能止疼。其仁捣如泥,可以点痣。拙拟毒淋汤又尝重用之,以治 
花柳毒淋。其化瘀解毒之力如此,治痢所以有奇效也。 

<目录>二、药物
<篇名>13.龙骨解
属性:龙骨∶味淡,微辛,性平。质最粘涩,具有翕收之力(以舌舐之即吸舌不脱,有翕收之力可知),故能收 
敛元气、镇安精神、固涩滑脱。凡心中怔忡、多汗淋漓、吐血衄血、二便下血、遗精白浊、大便滑泻、小便不 
禁、女子崩带,皆能治之。其性又善利痰,治肺中痰饮咳嗽,咳逆上气,其味微辛,收敛之中仍有开通之力, 
故《神农本草经》谓其主泻利脓血,女子漏下,而又主 瘕坚结也。龙齿与龙骨性相近,而又饶镇降之力,故 
《神农本草经》谓主小儿大人惊痫,癫疾狂走,心下结气,不能喘息也。 
愚于忽然中风肢体不遂之证,其脉甚弦硬者,知系肝火肝风内动,恒用龙骨同牡蛎加于所服药中以敛戢 
之,至脉象柔和其病自愈,拙拟镇肝熄风汤、建瓴汤,皆重用龙骨,方后皆有验案可参观。 
龙骨若生用之,凡心中怔忡、虚汗淋漓、经脉滑脱、神魂浮荡诸疾,皆因元阳不能固摄,重用龙骨,借其 
所含之元阴以翕收此欲涣之元阳,则功效立见。若 用之,其元阴之气因 伤损,纵其质本粘涩, 后其粘涩 
增加,而其翕收之力则顿失矣。用龙骨者用其粘涩,诚不如用其吸收也。明乎此理,则龙骨之不宜 益明矣。 
王洪绪《外科证治全生集》谓∶“用龙骨者宜悬之井中经宿而后用之”,是可谓深知龙骨之性,而善于用之者 
矣。愚用龙骨约皆生用,惟治女子血崩,或将流产,至极危时恒用 者,取其涩力稍胜以收一时之功也。 
陈修园曰∶痰,水也,随火而上升,龙骨能引逆上之火泛滥之水下归其宅,若与牡蛎同用,为治痰之神品, 
今人止知其性涩以收脱,何其浅也。 

<目录>二、药物
<篇名>14.牡蛎解
属性:牡蛎∶味咸而涩,性微凉。能软坚化痰,善消瘰 ,止呃逆,固精气,治女子崩带。《神农本草经》谓其 
主温疟者,因温疟但在足少阳,故不与太阳相并为寒,但与阳明相并为热。牡蛎能入其经而祛其外来之邪。主 
惊恚怒气者,因惊则由于胆,怒则由于肝,牡蛎咸寒属水,以水滋木,则肝胆自得其养。且其性善收敛有保合 
之力,则胆得其助而惊恐自除,其质类金石有镇安之力,则肝得其平而恚怒自息矣。至于筋原属肝,肝不病而 
筋之或拘或缓者自愈,故《神农本草经》又谓其除拘缓也。 
牡蛎所消之瘰 ,即《神农本草经》所谓 。而其所以能消 者,非因其咸能软坚也。盖牡蛎之原质, 
为炭酸钙化合而成,其中含有沃度(亦名海典),沃度者善消瘤赘瘰 之药也(医方篇消瘰丸下附有验案可参观)。 
龙骨、牡蛎,若专取其收涩可以 用。若用以滋阴,用以敛火,或取其收敛,兼取其开通者(二药皆敛 
而能开),皆不可 。 
若作丸散,亦可 用,因 之则其质稍软,与脾胃相宜也。然宜存性,不可过 ,若入汤剂仍以不 为佳。 
今用者一概 之,殊非所宜。 

<目录>二、药物
<篇名>15.石决明解
属性:石决明∶味微咸,性微凉。为凉肝镇肝之要药。肝开窍于目,是以其性善明目,研细水飞作敷药,能除目 
外障,作丸散内服,熊消目内障(消内障丸散优于汤剂)。为其能凉肝,兼能镇肝,故善治脑中 
充血作疼作眩晕,因此证多系肝气肝火挟血上冲也。是以愚治脑充血证,恒重用之至两许。其性又善利小便, 
通五淋,盖肝主疏泄为肾行气,用决明以凉之镇之,俾肝气肝火不妄动自能下行,肾气不失疏泄之常,则小便 
之难者自利,五淋之涩者自通矣。此物乃鳆甲也,状如蛤,单片附石而生,其边有孔如豌豆,七孔九孔者佳, 
宜生研作粉用之,不宜 用。 

<目录>二、药物
<篇名>16.玄参解
属性:玄参∶色黑,味甘微苦,性凉多液。原为清补肾经之药,中心空而色白(此其本色,药局多以黑豆皮水染 
之,则不见其白矣),故又能入肺以清肺家燥热,解毒消火,最宜于肺病结核、肺热咳嗽。《神农本草经》谓 
其治产乳余疾,因其性凉而不寒,又善滋阴,且兼有补性(凡名参者皆含有补性),故产后血虚生热及产后寒 
温诸证,热入阳明者,用之最宜。愚生平治产后外感实热,其重者用白虎加人参汤以玄参代方中知母,其轻者 
用拙拟滋阴清胃汤,亦可治愈。诚以产后忌用凉药,而既有外感实热,又不得不以凉药清之,惟石膏与玄参, 
《神农本草经》皆明载治产乳,故敢放胆用之。然用石膏又必加人参以辅之,又不敢与知母并用,至滋阴清胃 
汤中重用玄参,亦必以四物汤中归芍辅之,此所谓小心放胆并行不悖也。《神农本草经》又谓,玄参能明目, 
诚以肝开窍于目,玄参能益水以滋肝木,故能明目,且目之所以能视者,在瞳子中神水充足,神水固肾之精华 
外现者也。以玄参与柏实、枸杞并用,以治肝肾虚而生热视物不了了者,恒有捷效也。又外感大热已退,其人 
真阴亏损、舌干无津、胃液消耗、口苦懒食者,愚恒用玄参两许,加潞党参二三钱,连服数剂自愈。 

<目录>二、药物
<篇名>17.当归解
属性:当归∶味甘微辛,气香,液浓,性温。为生血、活血之主 
药,而又能宣通气分,使气血各有所归,故名当归。其力能升(因其气浓而温)能降(因其味浓而辛),内润 
脏腑(因其液浓而甘),外达肌表(因其味辛而温)。能润肺金之燥,故《神农本草经》谓其主咳逆上气;能 
缓肝木之急,故《金匮》当归芍药散,治妇人腹中诸疼痛;能补益脾血,使人肌肤华泽;生新兼能化瘀,故能 
治周身麻痹、肢体疼痛、疮疡肿疼;活血兼能止血,故能治吐血、衄血(须用醋炒取其能降也),二便下血 
(须用酒炒取其能升也);润大便兼能利小便,举凡血虚血枯、阴分亏损之证,皆宜用之。惟虚劳多汗、大便 
滑泻者,皆禁用。 
当归之性虽温,而血虚有热者,亦可用之,因其能生血即能滋阴,能滋阴即能退热也。其表散之力虽微, 
而颇善祛风,因风着人体恒致血痹,血活痹开,而风自去也。至于女子产后受风发搐,尤宜重用当归,因产后 
之发搐,半由于受风,半由于血虚(血虚不能荣筋),当归既能活血以祛风,又能生血以补虚,是以愚治此 
等证,恒重用当归一两,少加散风之品以佐之,即能随手奏效。 
【附案】一少妇,身体羸弱,月信一次少于一次,浸至只来少许,询问治法。时愚初习医未敢疏方,俾每 
日单用当归八钱煮汁饮之,至期所来经水遂如常,由此可知当归生血之效也。 
一人年四十余,得溺血证,自用当归一两酒煮饮之而愈。后病又反复,再用原方不效,求为延医,愚俾单 
用去皮鸦胆子五十粒,冰糖化水送下而愈。后其病又反复,再服鸦胆子方两次无效,仍用酒煮当归饮之而愈。 
夫人犹其人,证犹其证,从前治愈之方,后用之有效有不效者,或因血证之前后凉热不同也,然即此亦可知当 
归之能止下血矣。 

<目录>二、药物
<篇名>18.芍药解
属性:芍药∶味苦微酸,性凉多液(单煮之其汁甚浓)。善滋阴养血,退热除烦,能收敛上焦浮越之热下行自小 
便泻出,为阴虚有热小便不 
利者之要药。为其味酸,故能入肝以生肝血;为其味苦,故能入胆而益胆汁;为其味酸而兼苦,且又性凉,又 
善泻肝胆之热,以除痢疾后重(痢后重者,皆因肝胆之火下迫),疗目疾肿疼(肝开窍于目)。与当归、地黄 
同用,则生新血;与桃仁、红花同用,则消瘀血;与甘草同用则调和气血,善治腹疼;与竹茹同用,则善止吐 
衄;与附子同用,则翕收元阳下归宅窟。惟力近和缓,必重用之始能建功。 
芍药原有白、赤二种,以白者为良,故方书多用白芍。至于化瘀血,赤者较优,故治疮疡者多用之,为其 
能化毒热之瘀血不使溃脓也。白芍出于南方,杭州产者最佳,其色白而微红,其皮则红色又微重。为其色红白 
相兼,故调和气血之力独优。赤芍出于北方关东三省,各山皆有,肉红皮赤,其质甚粗,若野草之根,故张隐 
庵、陈修园皆疑其非芍药花根。愚向亦疑之,至奉后因得目睹,疑团方释,特其花叶皆小,且花皆单瓣,其花 
或粉红或紫色,然无论何色,其根之色皆相同。 
【附案】一童子年十五六岁,于季春得温病,经医调治,八九日间大热已退,而心犹发热,怔忡莫支,小 
便不利,大便滑泻,脉象虚数,仍似外邪未净,为疏方,用生杭芍二两,炙甘草一两半,煎汤一大碗徐徐温饮 
下,尽剂而愈。夫《神农本草经》谓芍药益气,元素谓其止泻利,即此案观之洵不误也。然必以炙草辅之,其 
功效乃益显。 
按∶此证原宜用拙拟滋阴清燥汤,原有芍药六钱,甘草三钱,又加生怀山药、滑石各一两,而当时其方犹 
未拟出,但投以芍药、甘草幸亦随手奏效。二方之中,其甘草一生用一炙用者,因一则少用之以为辅佐品,借 
以调和药之性味,是以生用;一则多用之至两半,借其补益之力以止滑泻,是以炙用,且《伤寒论》原有芍药 
甘草汤为育阴之妙品,方中芍药、甘草各四两,其甘草亦系炙用也。 
邻村周××,年二十余,得温病,医者用药清解之,旬日其热不退。诊其脉左大于右者一倍,按之且有力。 
夫寒温之热传入阳明,其脉皆右大于左,以阳明之脉在右也。即传入少阳厥阴,其脉亦右大于左,因既挟有外 
感实热,纵兼他经,仍以阳明为主也。此证独左大于右,乃温病之变证,遂投以小剂白虎汤(方中生石膏止用五 
钱),重加生杭芍两半,煎汤两茶杯顿饮之,须臾小便一次甚多,病若失。 
邻村霍氏妇,周身漫肿,腹胀小便不利,医者治以五皮饮不效。其脉数而有力,心中常觉发热,知其阴分 
亏损,阳分又偏盛也。为疏方,用生杭芍两半,玄参、滑石、地肤子、甘草各三钱,煎服一剂即见效验,后即 
方略为加减,连服数剂全愈。 
奉天陈某,年四十余,自正月中旬,觉心中发热懒食,延至暮春,其热益甚,常常腹疼,时或泄泻,其脉 
右部弦硬异常,按之甚实,舌苔微黄。知系外感伏邪,因春萌动,传入胃府,久而化热,而肝木复乘时令之旺 
以侮克胃土,是以腹疼且泄泻也。其脉象不为洪实而现弦硬之象者,因胃土受侮,亦从肝木之化也。为疏方, 
用生杭芍、生怀山药、滑石、玄参各一两,甘草、连翘各三钱,煎服一剂,热与腹疼皆愈强半,可以进食,自 
服药后大便犹下两次。诊其脉象已近和平,遂将方中芍药、滑石、玄参各减半,又服一剂全愈。 
陈姓妇,年二十余,于季春得温病,四五日间延为延医。其证表里俱热,脉象左右皆洪实,腹中时时切疼, 
大便日下两三次、舌苔浓而微黄,知外感邪热已入阳明之府,而肝胆乘时令木气之旺,又挟实热以侮克中土, 
故腹疼而又大便勤也。亦投以前方,加鲜茅根三钱,一剂腹疼便泻即止,又服一剂全愈。观此二案,《伤寒 
论》诸方,腹痛皆加芍药,不待疏解而自明也。至于茅根入药必须鲜者方效,若无鲜者可不用。 
一妇人年三十许,因阴虚小便不利,积成水肿甚剧,大便亦旬日不通。一老医投以八正散不效,友人高 
××为出方,用生白芍六两,煎汤两大碗,再用生阿胶二两融化其中,俾病患尽量饮之,老医甚为骇疑,高× 
×力主服之,尽剂而二便皆通,肿亦顿消。后老医与愚睹面为述其事,且问此等药何以能治此等病?答曰∶“ 
此必阴虚不能化阳,以致二便闭塞,白芍善利小便,阿胶能滑大便,二药并用又大能滋补真阴,使阴分充足以 
化其下焦偏盛之阳,则二便自能利也。” 
子××,治一水肿证,其人年六旬,二便皆不通利,心中满闷,时或烦躁,知其阴虚积有内热,又兼气分不 
舒也。投以生白芍三两,橘红、柴胡各三钱,一剂二便皆通。继服滋阴理气少加利小便之药全愈。 

<目录>二、药物
<篇名>19.芎解
属性:芎 ∶味辛、微苦、微甘,气香窜,性温。温窜相并,其力上升、下降、外达、内透无所不至。故诸家本 
草,多谓其能走泄真气,然无论何药,皆有益有弊,亦视用之何如耳。其特长在能引人身清轻之气上至于脑, 
治脑为风袭头疼、脑为浮热上冲头疼、脑部充血头疼。其温窜之力,又能通活气血,治周身拘挛,女子月闭无 
子。虽系走窜之品,为其味微甘且含有津液,用之佐使得宜,亦能生血。 
四物汤中用芎 ,所以行地黄之滞也,所以治清阳下陷时作寒热也。若其人阴虚火升,头上时汗出者,芎 
即不宜用。 
【附案】友人郭××妻,产后头疼,或与一方当归、芎 各一两煎服即愈。此盖产后血虚兼受风也。愚生平 
用芎 治头疼不过二三钱。 
一人年三十余,头疼数年,服药或愈,仍然反复,其脉弦而 
有力,左关尤甚,知其肝血亏损肝火炽盛也。投以熟地、柏实各一两,生龙骨、生牡蛎、龙胆草、生杭芍、枸杞 
各四钱,甘草、芎 各二钱,一剂疼止,又服数剂永不反复。 
又治一人,因脑为风袭头疼,用川芎、菊花各三钱,煎汤服之立愈。 

<目录>二、药物
<篇名>20.大黄解
属性:大黄∶味苦,气香,性凉。能入血分,破一切瘀血。为其气香故兼入气分,少用之亦能调气,治气郁作疼。 
其力沉而不浮,以攻决为用,下一切 瘕积聚。能开心下热痰以愈疯狂,降肠胃热实以通燥结,其香窜透窍之 
力又兼利小便(大黄之色服后入小便,其利小便可知)。性虽趋下而又善清在上之热,故目疼齿疼,用之皆为 
要药。又善解疮疡热毒,以治疔毒尤为特效之药(疔毒甚剧,他药不效者,当重用大黄以通其大便自愈)。其 
性能降胃热,并能引胃气下行,故善止吐衄,仲景治吐血、衄血有泻心汤,大黄与黄连、黄芩并用。《神农本 
草经》谓其能“推陈致新”,因有黄良之名。仲景治血痹虚劳,有大黄 虫丸,有百劳丸,方中皆用大黄,是 
真能深悟“推陈致新”之旨者也。 
《金匮》泻心汤,诚为治吐血、衄血良方,惟脉象有实热者宜之。若脉象微似有热者,愚恒用大黄三钱, 
煎汤送服赤石脂细末四五钱。若脉象分毫无热,且心中不觉热者,愚恒用大黄细末、肉桂细末各六七分,用开 
水送服即愈。 
凡气味俱浓之药,皆忌久煎,而大黄尤甚,且其质经水泡即软,煎一两沸药力皆出,与他药同煎宜后人, 
若单用之开水浸服即可,若轧作散服之,一钱之力可抵煎汤者四钱。 
大黄之力虽猛,然有病则病当之,恒有多用不妨者。是以治癫狂其脉实者,可用至二两,治疔毒之毒热甚 
盛者,亦可用至两许。盖用药以胜病为准,不如此则不能胜病,不得不放胆多用也。 
【附案】愚在籍时,曾至邻县治病,其地有杨氏少妇,得奇疾,赤身卧帐中,其背肿热,若有一缕着身, 
即觉热不能忍,百药无效。后有乘船自南来赴北闱乡试者,精通医术,延为诊视。言系阳毒,俾用大黄十斤, 
煎汤十碗,放量饮之,数日饮尽,竟霍然全愈。为其事至奇,故附记之。 


\chapter{朴硝、硝石解}
属性:(附∶朴硝炼玄明粉法) 
朴硝∶味咸,微苦,性寒。禀天地寒水之气以结晶,水能胜火,寒能胜热,为心火炽盛有实热者之要药。疗 
心热生痰,精神迷乱、五心潮热,烦躁不眠。且咸能软坚,其性又善消,故能通大便燥结,化一切瘀滞。咸入 
血分,故又善消瘀血,治妊妇胎殇未下。外用化水点眼,或煎汤熏洗,能明目消翳,愈目疾红肿。《神农本草 
经》谓炼服可以养生,所谓炼者,如法制为玄明粉,则其性尤良也。然今时之玄明粉,鲜有如法炼制者,凡药 
房中所鬻之玄明粉,多系风化朴硝,其性与朴硝无异。 
【附案】一少年女子,得疯疾癫狂甚剧,屡次用药皆未能灌下。后为设方,单用朴硝当盐,加于菜蔬中 
服之,病患不知,月余全愈。 
法库门生万××治一少女疯狂,强灌以药,竟将药碗咬破,仍未灌下。万××素阅《衷中参西录》,知此 
方,遂用朴硝和鲜莱菔作汤,令病患食之,数日全愈。 
奉天刘××,年四十余,得结证,饮食行至下脘复转而吐出,无论服何药亦如兹,且其处时时切疼,上 
下不通者已旬日矣。俾用朴硝六两,与鲜莱菔片同煮,至莱菔烂熟捞出,又添生片再煮,换至六七次,约用莱 
菔七八斤,将朴硝咸味借莱菔提之将尽,余浓汁四茶杯,每次温饮一杯,两点钟一次,饮至三次其结已开,大 
便通下。其女时息痢疾,俾饮其余,痢疾亦愈。 
奉天于姓妇,年近五旬,因心热生痰,痰火瘀滞,烦躁不眠,五心潮热,其脉象洪实。遂用朴硝和炒熟麦 
面炼蜜为丸,三钱重,每丸中约有朴硝一钱,早晚各服一丸,半月全愈。盖人多思虑则心热气结,其津液亦 
恒随气结于心下,经心火灼炼而为热痰。朴硝咸且寒,原为心经对宫之药,其咸也属水,力能胜火,而又寒能 
胜热,且其性善消,又能开结,故以治心热有痰者最宜。至于必同麦面为丸者,以麦为心谷,心脏有病以朴硝 
泻之,即以麦面补之,补破相济为用,则药性归于和平,而后可久服也。 
硝石即焰硝,俗名火硝。味辛微咸,性与朴硝相近,其寒凉之力逊于朴硝,而消化之力胜于朴硝,若与皂 
矾同用,善治内伤黄胆,消胆中结石、膀胱中结石(即石淋)及钩虫病(钩虫及胆石病,皆能令人成黄胆医论 
篇中有“论黄胆有内伤外感及内伤外感之兼证并详治法”,载有审定《金匮》硝石矾石散方,可参观)。 


<篇名>附∶朴硝炼玄明粉法
属性:用朴硝炼玄明粉,原用莱菔。然此法今人不讲久矣。至药局所鬻者,乃风化硝,非玄明粉也。今并载其法, 
以备参观。 
其法∶于冬至后,用洁净朴硝十斤,白莱菔五斤切片,同入锅中,用水一斗五升,煮至莱菔烂熟,将莱菔 
捞出。用竹筛一个,铺绵纸二层,架托于新缸之上,将硝水滤过。在庭露三日,其硝凝于缸边,将余水倾出, 
晒干。将硝取出,用砂锅熬于炉上,融化后,搅以铜铲,熬至将凝,用铲铲出。再装以瓷罐,未满者寸许,盖 
以瓦片。用钉三个,钉地作鼎足形,钉头高二寸,罐置其上,用砖在罐周遭,砌作炉形,多留风眼,炉砖离罐 
三寸。将木炭火置于炉中,罐四围上下都被炭火壅培,以 至硝红为度。次日取出,再用绵纸铺于净室地上, 
将硝碾细,用绢罗筛于纸上浓一分。将户牖皆遮蔽勿透风,三日后取出,其硝洁白如粉,轻虚成片。其性最能 
降火化痰,清利脏腑。怪证服之可蠲, 
狂躁用之即愈。搜除百病,安敛心神。大人服二、三钱,小儿服五分至一钱,用白汤或葱汤融化,空心服之。 
服药之日,不宜食他物,惟饮稀粥。服二三次后、自然精神爽健,脏腑调和,津液顿生,百病如失。惟久病泄 
泻者,服之不宜。 

<目录>二、药物
<篇名>22.浓朴解
属性:浓朴∶味苦辛,性温。治胃气上逆,恶心呕哕,胃气郁结胀满疼痛,为温中下气之要药。为其性温味又兼 
辛,其力不但下行,又能上升外达,故《神农本草经》谓其主中风伤寒头痛,《金匮》浓朴麻黄汤,用治咳而 
脉浮。与橘、夏并用,善除湿满;与姜、术并用,善开寒痰凝结;与硝、黄并用,善通大便燥结;与乌药并用, 
善治小便因寒白浊。味之辛者属金,又能入肺以治外感咳逆;且金能制木,又能入肝、平肝木之横恣以愈胁下 
掀疼;其色紫而含有油质,故兼入血分,甄权谓其破宿血,古方治月闭亦有单用之者。诸家多谓其误服能脱元 
气,独叶香岩谓“多用则破气,少用则通阳”,诚为确当之论。 
【附案】一少妇因服寒凉开胃之药太过,致胃阳伤损,饮食不化,寒痰瘀于上焦,常常短气,治以苓桂术 
甘汤加干姜四钱、浓朴二钱,嘱其服后若不觉温暖,可徐徐将干姜加重。后数月见其家人,言干姜加至一两二 
钱、浓朴加至八钱,病始脱然。问何以并将浓朴加重?谓“初但将干姜加重则服之觉闷,后将浓朴渐加重至八 
钱始服之不觉闷,而寒痰亦从此开豁矣。”由是观之,元素谓∶寒胀之病,于大热药中兼用浓朴,为结者散之 
之神药,诚不误也。 
愚二十余岁时,于仲秋之月,每至申、酉时腹中作胀,后于将作胀时,但嚼服浓朴六七分许,如此两日, 
胀遂不作。服浓朴辛以散之,温以通之,且能升降其气化是以愈耳。 
愚治冲气上冲,并挟痰涎上逆之证,皆重用龙骨、牡蛎、半夏、赭石诸药以降之、镇之、敛之,而必少 
用浓朴以宣通之,则冲气痰涎下降,而中气仍然升降自若无滞碍。 

<目录>二、药物
<篇名>23.麻黄解
属性:麻黄∶味微苦,性温。为发汗之主药。于全身之脏腑经络,莫不透达,而又以逐发太阳风寒为其主治之 
大纲。故《神农本草经》谓其主中风伤寒头痛诸证,又谓其主咳逆上气者,以其善搜肺风兼能泻肺定喘也。谓 
其破 瘕积聚者,以其能透出皮肤毛孔之外,又能深入积痰凝血之中,而消坚化瘀之药可偕之以奏效也。且其 
性善利小便,不但走太阳之经,兼能入太阳之府,更能由太阳而及于少阴(是以伤寒少阴病用之),并能治疮 
疽白硬,阴毒结而不消。 
太阳为周身之外廓,外廓者皮毛也,肺亦主之。风寒袭人,不但入太阳,必兼入手太阴肺经,恒有咳嗽 
微喘之证。麻黄兼入手太阴为逐寒搜风之要药,是以能发太阳之汗者,不仅麻黄,而《伤寒论》治太阳伤寒无 
汗,独用麻黄汤者,治足经而兼顾手经也。 
凡利小便之药,其中空者多兼能发汗,木通、 蓄之类是也。发汗之药,其中空者多兼能利小便,麻黄、 
柴胡之类是也。伤寒太阳经病,恒兼入太阳之腑(膀胱),致留连多日不解,麻黄治在经之邪,而在腑之邪亦 
兼能治之。盖在经之邪由汗而解,而在腑之邪亦可由小便而解,彼后世用他药以代麻黄者,于此义盖未之审也。 
受风水肿之证,《金匮》治以越婢汤,其方以麻黄为主,取其能祛风兼能利小便也。愚平素临证用其 
方服药后果能得汗,其小便即顿能利下,而肿亦遂消。特是,其方因麻黄与石膏并用,石膏之力原足以监制 
麻黄,恒有服之不得汗者,今变通其方,于 
服越婢汤之前,先用白糖水送服西药阿斯匹林一瓦半,必能出汗,趁其正出汗时,将越婢汤服下,其汗出 
必益多,小便亦遂通下。 
东人××博士,用麻黄十瓦,煎成水一百瓦,为一日之量,分三次服下,治慢性肾炎小便不利及肾脏萎 
缩小便不利,用之有效有不效,以其证之凉热虚实不同,不知用他药佐之以尽麻黄之长也。试观《金匮》水 
气门越婢汤,麻黄辅以石膏,因其脉浮有热也(脉浮故系有风实亦有热),麻黄附子汤辅以附子,因其脉沉而 
寒也。通变化裁,息息与病机相符,是真善用麻黄者矣。 
古方中有麻黄,皆先将麻黄煮数沸吹去浮沫,然后纳他药,盖以其所浮之沫发性过烈,去之所以使其性 
归和平也。 
麻黄带节发汗之力稍弱,去节则发汗之力较强,今时用者大抵皆不去节,至其根则纯系止汗之品,本 
是一物,而其根茎之性若是迥殊,非经细心实验,何以知之。 
陆九芝谓∶“麻黄用数分,即可发汗,此以治南方之人则可,非所论于北方也。盖南方气暖,其人肌肤 
薄弱,汗最易出,故南方有麻黄不过钱之语;北方若至塞外,气候寒冷,其人之肌肤强浓,若更为出外劳碌, 
不避风霜之人,又当严寒之候,恒用七八钱始能汗者。夫用药之道,贵因时、因地、因人,活泼斟酌以胜病为 
主,不可拘于成见也。” 

<目录>二、药物
<篇名>24.柴胡解
属性:柴胡∶味微苦,性平。禀少阳生发之气,为足少阳主药,而兼治足厥阴。肝气不舒畅者,此能舒之;胆 
火甚炽盛者,此能散之;至外感在少阳者,又能助其枢转以透膈升出之,故《神农本草经》谓其主寒热,寒热者 
少阳外感之邪也。又谓其主心腹肠胃中结气,饮食积聚,诚以五行之理,木能疏土,为柴胡善达少阳之木 
气,则少阳之气自能疏通胃土之郁,而其结气饮食积聚自消化也。 
《神农本草经》柴胡主寒热,山茱萸亦主寒热。柴胡所主之寒热,为少阳外感之邪,若伤寒疟疾是也,故 
宜用柴胡和解之,山萸肉所主之寒热,为厥阴内伤之寒热,若肝脏虚极忽寒忽热,汗出欲脱是也,故宜用山萸 
肉补敛之。二证之寒热虽同,而其病因判若天渊,临证者当细审之,用药慎勿误投也。 
柴胡非发汗之药,而多用之亦能出汗。小柴胡汤多用之至八两,按今时分量计之,且三分之(古方一煎三 
服,故可三分)一剂可得八钱。小柴胡汤中如此多用柴胡者,欲借柴胡之力升提少阳之邪以透膈上出也。然多 
用之又恐其旁行发汗,则上升之力不专,小柴胡汤之去渣重煎,所以减其发汗之力也。 
或疑小柴胡汤既非发汗之药,何以《伤寒论》百四十九节服柴胡汤后有汗出而解之语?不知此节文义,原 
为误下之后服小柴胡汤者说法。夫小柴胡汤,系和解之剂,原非发汗之剂,特以误下之后,胁下所聚外感之邪, 
兼散漫于手少阳三焦,因少阳为游部,手、足少阳原相贯彻也。此时仍投以小柴胡和解之,则邪之散漫于三焦 
者,遂可由手少阳外达之经络作汗而解,而其留于胁下者,亦与之同气相求,借径于手少阳而汗解,故于发热 
汗出上,特加一却字,言非发其汗而却由汗解也。然足少阳之由汗解原非正路,乃其服小柴胡汤后,胁下之邪 
欲上升透膈,因下后气虚不能助之通过,而其邪之散漫于手少阳者,且又以同类相招,遂于蓄极之时而开旁通 
之路,此际几有正气不能胜邪气之势,故必先蒸蒸而振,大有邪正相争之象,而后发热汗出而解,此即所谓战 
而后汗也。观下后服柴胡汤者,其出汗若是之难,则足少阳之病由汗解,原非正路益可知也。是以愚生平临证, 
于壮实之人用小柴胡汤时,恒减去人参,而于经医误下之后者,若用小柴胡汤必用人参以助其战胜之力。 
用柴胡以治少阳外感之邪,不必其寒热往来也。但知其人纯系外感,而有恶心欲吐之现象,是即病在少阳, 
欲借少阳枢转之机透膈上达也。治以小柴胡可随手奏效,此病机欲上者因而越之也。又有其人不见寒热往来, 
亦并不喜呕,惟频频多吐粘涎,斯亦可断为少阳病,而与以小柴胡汤。盖少阳之去路为太阴湿土,因包脾之脂 
膜原与板油相近,而板油亦脂膜,又有同类相招之义,此少阳欲传太阴,而太阴湿土之气经少阳之火铄炼,遂 
凝为粘涎频频吐出,投以小柴胡汤,可断其入太阴之路,俾由少阳而解矣。又∶柴胡为疟疾之主药,而小心过 
甚者,谓其人若或阴虚燥热,可以青蒿代之。不知疟邪伏于胁下两板油中,乃足少阳经之大都会,柴胡能入其 
中,升提疟邪透膈上出,而青蒿无斯力也。若遇阴虚者,或热入于血分者,不妨多用滋阴凉血之药佐之;若遇 
燥热者,或热盛于气分者,不妨多用润燥清火之药佐之。是以愚治疟疾有重用生地、熟地治愈者,有重用生石 
膏、知母治愈者,其气分虚者,又有重用参、 治愈者,然方中无不用柴胡也。 
【附案】一人年过四旬,胁下掀疼,大便七八日未行,医者投以大承气汤,大便未通而胁下之疼转甚。其 
脉弦而有力,知系肝气胆火恣盛也,投以拙拟金铃泻肝汤加柴胡、龙胆草各四钱,服后须臾大便通下,胁疼顿 
愈。审是,则《神农本草经》谓“柴胡主肠胃中饮食积聚,推陈致新”者,诚非虚语也。且不但能通大便也, 
方书通小便亦多有用之者,愚试之亦颇效验。盖小便之下通,必由手少阳三焦,三焦之气化能升而后能降,柴 
胡不但升足少阳实兼能升手少阳也。 

<目录>二、药物
<篇名>25.桂枝解
属性:桂枝∶味辛微甘,性温。力善宣通,能升大气(即胸之宗气),降逆 
气(如冲气肝气上冲之类),散邪气(如外感风寒之类)。仲景苓桂术甘汤用之治短气,是取其能升也;桂枝 
加桂汤用之治奔豚,是取其能降也;麻黄、桂枝、大小青龙诸汤用之治外感,是取其能散也。而《神农本草经》 
论牡桂(即桂枝),开端先言其主咳逆上气,似又以能降逆气为桂枝之特长,诸家本草鲜有言其能降逆气者, 
是用桂枝而弃其所长也。 
小青龙汤原桂枝、麻黄并用,至喘者去麻黄加杏仁而不去桂枝,诚以《神农本草经》原谓桂枝主吐吸 
(吐吸即喘),去桂枝则不能定喘矣。乃医者皆知麻黄泻肺定喘,而鲜知桂枝降气定喘,是不读《神农本草经》 
之过也。桂枝善抑肝木之盛使不横恣,又善理肝木之郁使之条达也。为其味甘,故又善和脾胃,能使脾气之陷 
者上升,胃气之逆者下降,脾胃调和则留饮自除,积食自化。其宣通之力,又能导引三焦下通膀胱以利小便 
(小便因热不利者禁用,然亦有用凉药利小便而少加之作向导者),惟上焦有热及恒患血证者忌用。 
桂枝非发汗之品,亦非止汗之品,其宣通表散之力,旋转于表里之间,能和营卫、暖肌肉、活血脉,俾 
风寒自解,麻痹自开,因其味辛而且甘,辛者能散,甘者能补,其功用在于半散半补之间也。故服桂枝汤欲得 
汗者,必啜热粥,其不能发汗可知;若阳强阴虚者,误服之则汗即脱出,其不能止汗可知。 
按∶《伤寒论》用桂枝,皆注明去皮,非去枝上之皮也。古人用桂枝,惟取当年新生嫩枝,折视之内外 
如一,皮骨不分,若见有皮骨可以辨者去之不用,故曰去皮,陈修园之侄鸣岐曾详论之。 
徐灵胎谓,受风有热者,误用桂枝则吐血,是诚确当之论。忆曾治一媪,年六旬,春初感冒风寒,投 
以发表之剂,中有桂枝数钱,服后即愈。其家人为其方灵,贴之壁上。至孟夏,复受感冒,自用其方取药服之, 
遂致吐血,经医治疗始愈。盖前所受者寒风,后所受者热风,故一则宜用桂枝,一则忌用桂枝,彼用桂枝汤以 
治温病者可不戒哉!特是,徐氏既知桂枝误用可致吐血, 
而其《洄溪医案》中载,治一妇人外感痰喘证,其人素有血证,时发时止,发则微嗽(据此数语断之,其血证 
当为咳血),因痰喘甚剧,病急治标,投以小青龙汤而愈。 
用小青龙汤治外感痰喘,定例原去麻黄加杏仁,而此证则当去桂枝留麻黄,且仿《金匮》用小青龙汤之法, 
再加生石膏方为稳安。盖麻黄、桂枝,皆能定喘,而桂枝动血分,麻黄不动血分,是以宜去桂枝留麻黄,再借 
石膏凉镇之力以预防血分之妄动,乃为万全之策,而当日徐氏用此方,未言加减,岂略而未言乎?抑用其原方 
乎?若用其原方,病虽治愈,亦几等孤注之一掷矣。 

<目录>二、药物
<篇名>26.三七解
属性:三七∶味苦微甘,性平(诸家多言性温,然单服其末数钱,未有觉温者)。善化瘀血,又善止血妄行,为 
吐衄要药。病愈后不至瘀血留于经络证变虚劳(凡用药强止其血者,恒至血瘀经络成血痹虚劳)。兼治二便下 
血,女子血崩,痢疾下血鲜红(宜与鸦胆子并用)久不愈,肠中腐烂,浸成溃疡,所下之痢色紫腥臭,杂以脂 
膜,此乃肠烂欲穿(三七能化腐生新,是以治之)。为其善化瘀血,故又善治女子 瘕,月事不通,化瘀血而 
不伤新血,允为理血妙品。外用善治金疮,以其末敷伤口,立能血止疼愈。若跌打损伤,内连脏腑经络作疼痛 
者,外敷、内服奏效尤捷,疮疡初起肿疼者,敷之可消(当与大黄末等分,醋调敷)。 
《本草备要》所谓,近出一种,叶似菊艾而劲浓有歧尖,茎有赤棱,夏秋开花,花蕊如金丝,盘纽可爱, 
而气不香,根小如牛蒡,味甘,极易繁衍,云是三七,治金疮折伤血病甚效者,是刘寄奴非三七也。 
三七之性,既善化血,又善止血,人多疑之,然有确实可征之处。如破伤流血者,用三七末擦之则其血立 
止,是能止血也;其破处已流出之血,着三七皆化为黄水,是能化血。 
【附案】本邑高姓童子,年十四五岁,吐血甚剧,医治旬日无效,势甚危急。仓猝遣人询方,俾单用三七 
末一两,分三次服下,当日服完其血立止。 
本庄黄氏妇,年过四旬,因行经下血不止,彼时愚甫弱冠,为近在比邻,延为诊视,投以寻常治血崩之药 
不效,病势浸至垂危。后延邻村宿医高××,投以《傅青主女科》中治老妇血崩方,一剂而愈。其方系黄 、当 
归各一两,桑叶十四片,煎汤送服三七细末三钱。后愚用此方治少年女子血崩亦效,惟心中觉热,或脉象有 
热者,宜加生地黄一两。 
奉天王姓少年,素患吐血,经医调治已两月不吐矣。而心中发闷,发热,时觉疼痛,廉于饮食,知系吐血 
时医者用药强止其血,致留瘀血为恙也。为疏方,用滋阴养血健胃利气之品,煎汤送服三七细末二钱,至二煎 
仍送服二钱,四剂后又复吐血,色多黑紫,然吐后则闷热疼痛皆减。知为吉兆,仍与前方,数剂后又吐血一次, 
其病从此竟愈,此足征三七化瘀之功也。 
乙丑孟夏末旬,愚寝室窗上糊纱一方以透空气,夜则以窗帘障之。一日寝时甚热,未下窗帘。愚睡正当窗, 
醒时觉凉风扑面袭入右腮,因睡时向左侧也。至午后右腮肿疼,知因风袭,急服西药阿斯匹林汗之。乃汗出已 
透,而肿疼依然。迟至翌晨,病又加剧,手按其处,连牙床亦肿甚,且觉心中发热。于斯连服清火、散风、活 
血消肿之药数剂。心中热退,而肿疼仍不少减,手抚之肌肤甚热。遂用醋调大黄细末屡敷其上,初似觉轻。迟 
半日仍无效,转觉其处畏凉。因以热水沃巾熨之,又见轻。乃屡熨之,继又无效。因思未受风之先,头面原觉 
发热,遽为凉风所袭,则凉热之气凝结不散。因其中凉热皆有,所以乍凉之与热相宜则觉轻,乍热之与凉相宜 
亦觉轻也。然气凝则血滞肿疼,久不愈必将化脓。遂用山甲、皂刺、乳香、没药、粉草、连翘诸药迎而治之。 
服两剂仍分毫无效。浸至其疼彻骨,夜不能眠。踌躇再四,恍悟三七外敷,善止金疮作疼,以其善化瘀血也。 
若内服之,亦当使瘀血之聚者速化而止疼。遂急取三七细末二钱服之,约数分钟其疼已见轻,逾一句钟即疼 
愈强半矣。当日又服两次,至翌晨已不觉疼,肿亦见消。继又服两日,每日三次,其肿消无芥蒂。 
丙寅季春,表侄刘××,右腿环跳穴处,肿起一块,大如掌,按之微硬,皮色不变,继则渐觉肿处骨疼, 
日益加重。及愚诊视时,已三阅月矣。愚因思其处正当骨缝,其觉骨中作疼者,必其骨缝中有瘀血也。俾日用 
三七细末三钱,分作两次服下。至三日,骨已不疼。又服数日,其外皮色渐红而欲腐。又数日,疮顶自溃,流 
出脓水若干,遂改用生黄 、天花粉各六钱,当归、甘草各三钱,乳香、没药各一钱。连服十余剂,其疮自内生 
肌排脓外出,结痂而愈。按此疮若不用三七托骨中之毒外出,其骨疼不已,疮毒内陷,或成附骨疽为不治之证。 
今因用三七,不但能托骨中之毒外出,并能化疮中之毒使速溃脓(若早服三七并可不溃脓而自消),三七之治 
疮,何若斯之神效哉!因恍悟愚之右腮肿疼时,其肿疼原连于骨,若不服三七将毒托出,必成骨槽风证无疑也。 
由此知凡疮之毒在于骨者,皆可用三七托之外出也。 
天津胡氏妇,信水六月未通,心中发热,胀闷。治以通经之药,数剂通下少许。自言少腹仍有发硬一块 
未消。其家适有三七若干,俾为末,日服四五钱许,分数次服下。约服尽三两,经水大下,其发硬之块亦消矣。 
审斯,则凡人腹中有坚硬之血积,或妇人产后恶露未尽结为 瘕者,皆可用三七徐消之也。 
天津刘××,偶患大便下血甚剧。西医注射以止血药针,其血立止。而血止之后,月余不能起床,身体 
酸软,饮食减少。其脉芤而无力,重按甚涩。因谓病家曰∶“西人所注射者,流动麦角膏也。其收缩血管之力 
甚大,故注射之后,其血顿止。然止后宜急 
服化瘀血之药,则不归经之血,始不至凝结于经络之间为恙。今但知止血,而不知化血,积之日久必成劳瘵, 
不仅酸软减食已也。然此时尚不难治,下其瘀血即愈矣。”俾日用三七细末三钱,空心时分两次服下。服至三 
次后,自大便下瘀血若干,色紫黑。从此每大便时,必有瘀血随下。至第五日,所下渐少。至第七日,即不见 
瘀血矣。于斯停药不服。旬日之间,身体复初。由斯观之,是三七一味即可代《金匮》之下瘀血汤,且较下瘀 
血汤更稳妥也。 
附录∶ 
山东沂水刘××来函∶ 
仲夏,杨姓女,年七岁,患疳疾兼大便下血,身形羸弱,不思饮食,甚为危险。前所服中西治疳积之药若 
干,均无效,来寓求治。后学查看腹部,其回血管现露,色青微紫,腹胀且疼,两颧发赤,潮热有汗,目睛白 
处有赤丝,口干不渴,六脉沉数,肌肤甲错,毛发焦枯。审证辨脉,知系瘀血为恙也。踌躇再四,忽忆及向阅 
《衷中参西录》,见先生论用三七之特殊功能,历数诸多奇效,不但善于止血,且更善化瘀血。遂俾用三七研 
为精粉,每服七分,朝夕空心时各服一次,服至五日,而大便下血愈。又服数日,疳疾亦愈。用三七一味,治 
愈中、西诸医不能治之大病,药性之妙用,真令人不可思议矣。 

<目录>二、药物
<篇名>27.滑石解
属性:滑石∶色白味淡,质滑而软,性凉而散。《神农本草经》谓其主身热者,以其微有解肌之力也,谓其主癃 
闭者,以其饶有淡渗之力也;且滑者善通窍络,故又主女子乳难;滑而能散,故又主胃中积聚;因热小便不 
利者,滑石最为要药,若寒温外感诸证,上焦燥热下焦滑泻无度,最为危险之候,可用滑石与生山药各两许,煎汤 
服之,则上能清热,下能止泻,莫不随手奏效(有附案载于滋阴清燥汤下可参观)。 
外感大热已退而阴亏脉数不能自复者,可于大滋真阴药中(若熟地黄、生山药、枸杞之类)少加滑石,则 
外感余热不至为滋补之药逗留,仍可从小便泻出,则其病必易愈。若与甘草为末(滑石六钱,甘草一钱,名六 
一散,亦名天水散)服之,善治受暑及热痢;若与赭石为末服之,善治因热吐血、衄血;若其人蕴有湿热、周 
身漫肿、心腹膨胀、小便不利者,可用滑石与土狗研为散服之,小便通利肿胀自消,至内伤阴虚作热,宜用六 
味地黄汤以滋阴者,亦可少加滑石以代苓、泽,则退热较速。盖滑石虽为石类,而其质甚软,无论汤剂丸散, 
皆与脾胃相宜,故可加于六味汤中以代苓、泽。其渗湿之力,原可如苓、泽行熟地之滞泥,而其性凉于苓、 
泽,故又善佐滋阴之品以退热也。 
天水散,为河间治暑之圣药,最宜于南方暑证。因南方暑多挟湿,滑石能清热兼能利湿,又少加甘草以和 
中补气(暑能伤气),是以用之最宜。若北方暑证,不必兼湿,甚或有兼燥,再当变通其方,滑石、生石膏各 
半,与甘草配制,方为适宜。 

<目录>二、药物
<篇名>28.牛膝解
属性:牛膝∶味甘微酸,性微温。原为补益之品,而善引气血下注,是以用药欲其下行者,恒以之为引经。故善 
治肾虚腰疼、腿疼,或膝疼不能屈伸,或腿痿不能任地,兼治女子月闭血枯,催生下胎。又善治淋疼,通利 
小便,此皆其力善下行之效也。然《名医别录》又谓其除脑中痛,时珍又谓其治口疮齿痛者何也?盖此等 
证,皆因其气血随火热上升所致,重用牛膝引其气血下行,并能引其浮越之火下行,是以能愈也。愚因悟得此 
理,用以治脑充血证,伍以赭石、龙骨、牡蛎诸重坠收敛之品,莫不随手奏效,治愈者不胜纪矣。为其性专下 
注,凡下焦气化不固,一切滑脱诸证皆忌之。此药怀产者佳,川产者有紫、白两种色,紫者佳。 
【附案】在辽宁时,曾治一女子,月信期年未见,方中重用牛膝一两,后复来诊,言服药三剂月信犹未 
见,然从前曾有脑中作疼病,今服此药脑中清爽异常,分毫不觉疼矣。愚闻此言,乃知其脑中所以作疼者,血 
之上升者多也。今因服药而不疼,想其血已随牛膝之引而下行,遂于方中加 虫五枚,连服数剂,月信果通。 
友人袁××,素知医,时当季春,牙疼久不愈,屡次服药无效。其脉两寸甚实,俾用怀牛膝、生赭石各 
一两,煎服后,疼愈强半,又为加生地黄一两,又服两剂,遂霍然全愈。 

<目录>二、药物
<篇名>29.远志解
属性:远志∶味酸微辛,性平。其酸也能 ,其辛也能辟,故其性善理肺,能使肺叶之 辟纯任自然,而肺中之 
呼吸于以调,痰涎于以化,即咳嗽于以止矣。若以甘草辅之,诚为养肺要药,至其酸敛之力,入肝能敛戢肝火, 
入肾能固涩滑脱,入胃又能助生酸汁,使人多进饮食,和平纯粹之品,夫固无所不宜也。若用水煎取浓汁,去 
渣重煎,令其汁浓若薄糊,以敷肿疼疮疡及乳痈甚效,若恐其日久发酵,每一两可加蓬砂二钱溶化其中。 
愚初次细嚼远志尝之,觉其味酸而实兼有矾味,西人谓其含有林檎酸,而林檎酸中固无矾也。后乃因用 
此药,若末服至二钱可作呕吐,乃知其中确含有矾味,因悟矾能利痰,其所以能利痰者,亦以其含有矾味也。 
矾能解毒,《本草纲目》谓其解天雄、附子、乌头毒,且并能除疮疡肿疼者,亦以其兼有矾味也。是以愚用此 
药入汤剂时,未尝过二钱,恐多用之亦可作呕吐也。 

<目录>二、药物
<篇名>30.龙胆草解
属性:龙胆草∶味苦微酸,性寒,色黄属土,为胃家正药。其苦也 
能降胃气、坚胃质,其酸也能补益胃中酸汁、消化饮食,凡胃热气逆、胃汁短少、不能食者,服之可以开胃进 
食,西人浑以健胃药称之,似欠精细。为其微酸属木,故又能入胆肝、滋肝血、益胆汁、降肝胆之热使不上炎, 
举凡目疾、吐血、衄血、二便下血、惊痫、眩晕、因肝胆有热而致病者,皆能愈之。其泻肝胆实热之力,数倍 
于芍药,而以敛戢肝胆虚热,固不如芍药也。 

<目录>二、药物
<篇名>31.半夏解
属性:半夏∶味辛,性温,有毒。凡味辛之至者,皆禀秋金收降之性,故力能下达为降胃安冲之主药。为其能降 
胃安冲,所以能止呕吐,能引肺中、胃中湿痰下行,纳气定喘。能治胃气厥逆、吐血、衄血(《内经》谓阳明 
厥逆衄呕血,阳明厥逆,即胃气厥逆也)。惟药局因其有毒,皆用白矾水煮之,相制太过,毫无辛味,转多矾 
味,令人呕吐,即药局所鬻之清半夏中亦有矾,以之利湿痰犹可,若以止呕吐及吐血、衄血,殊为非宜。愚治 
此等证,必用微温之水淘洗数次,然后用之,然屡次淘之则力减,故须将分量加重也。 
愚因药局半夏制皆失宜,每于仲春季秋之时,用生半夏数斤,浸以热汤,日换一次,至旬日,将半夏剖 
为两瓣,再入锅中,多添凉水煮一沸,速连汤取出,盛盆中,候水凉,净晒干备用。每用一两,煎汤两茶盅, 
调入净蜂蜜二两,徐徐咽之。无论呕吐如何之剧,未有不止者。盖古人用半夏,原汤泡七次即用,初未有用白 
矾制之者也。 
【附案】邻村王姓童子,年十二三岁,忽晨起半身不能动转,其家贫无钱购药,赠以自制半夏,俾为末 
每服钱半,用生姜煎汤送下,日两次,约服二十余日,其病竟愈。盖以自制半夏辛味犹存,不但能利痰,实有 
开风寒湿痹之力也。 
东洋××××日,英国军医×××屡屡吐,绝食者久矣。其 
弟与美医××氏协力治疗之,呕吐卒不止,乞诊于余,当时已认患者为不起之人,但求余一决其死生而已。美 
医××氏等遂将患者之证状及治疗之经过,一一告余。余遂向两氏曰∶余有一策,试姑行之。遂辞归检查汉法 
医书,制小半夏加茯苓汤,贮瓶令其服用,一二服后奇效忽显,数日竟回撤消有之康健。至今半夏浸剂,遂为 
一种之镇呕剂,先行于医科大学,次及于各病院与医家。 
按∶此证若用大半夏汤加赭石尤效,因吐久则伤津伤气,方中人参能生津补气,加赭石以助之,力又专于 
下行也。若有热者,可再加天冬佐之,若无自制半夏,可用药局清半夏两许,淘净矾味入煎。 

<目录>二、药物
<篇名>32.栝蒌解
属性:栝蒌∶味甘,性凉。能开胸间及胃口热痰,故仲景治结胸有小陷胸汤,栝蒌与连、夏并用;治胸痹有栝蒌 
薤白等方,栝蒌与薤、酒、桂、朴诸药并用,若与山甲同用,善治乳痈(栝蒌两个,山甲二钱煎服);若与赭 
石同用,善止吐衄(栝蒌能降胃气胃火故治吐衄)。若但用其皮,最能清肺、敛肺、宁嗽、定喘(须用新鲜者 
方效);若但用其瓤(用温水将瓤泡开,拣出仁,余煎一沸,连渣服之),最善滋阴、润燥、滑痰、生津,若 
但用其仁(须用新炒熟者,捣碎煎服),其开胸降胃之力较大,且善通小便。 
【附案】邻村高××子,年十三岁,于数日之间,痰涎郁于胸中,烦闷异常,剧时气不上达,呼吸即停, 
目翻身挺,有危在顷刻之状。连次用药,分毫无效,敢乞往为诊视,施以良方。时愚有急务未办,欲迟数点钟 
再去,彼谓此病已至极点,若稍迟延恐无及矣。于是遂与急往诊视,其脉关前浮滑,舌苔色白,肌肤有热,知 
其为温病结胸,俾用栝蒌仁四两,炒熟(新炒者其气香而能通)捣碎,煎汤两茶盅,分两次温饮下,其病顿愈。 
隔数日,其邻高 
姓童子,亦得斯证,俾用新炒蒌仁三两,苏子五钱,煎服,亦一剂而愈。盖伤寒下早成结胸,温病未经下亦可 
成结胸,有谓栝蒌力弱,故小陷胸汤中必须伍以黄连、半夏始能建功者,不知栝蒌力虽稍弱,重用之则转弱为 
强,是以重用至四两,即能随手奏效,挽回人命于顷刻也。 

<目录>二、药物
<篇名>33.天花粉解
属性:天花粉∶栝蒌根也,色白而亮者佳,味苦微酸,性凉而润。清火生津,为止渴要药(《伤寒论》小柴胡汤, 
渴者去半夏加栝蒌根,古方书治消渴亦多用之)。为其能生津止渴,故能润肺,化肺中燥痰,宁肺止嗽,治肺 
病结核。又善通行经络,解一切疮家热毒,疔痈初起者,与连翘、山甲并用即消,疮疡已溃者,与黄 、甘草 
(皆须用生者)并用,更能生肌排脓,即溃烂至深旁串他处,不能敷药者,亦可自内生长肌肉,徐徐将脓排出 
(有案附载黄 解下可参观)。大凡藤蔓之根,皆能通行经络,而花粉又性凉解毒,是以有种种功效也。 

<目录>二、药物
<篇名>34.干姜解
属性:干姜,味辛,性热。为补助上焦、中焦阳分之要药。为其味至辛,且具有宣通之力,与浓朴同用,治寒饮 
杜塞胃脘,饮食不化;与桂枝同用,治寒饮积于胸中,呼吸短气;与黄 同用,治寒饮渍于肺中,肺痿咳嗽; 
与五味子同用,治感寒肺气不降,喘逆迫促;与赭石同用,治因寒胃气不降,吐血、衄血;与白术同用, 
治脾寒不能统血,二便下血,或脾胃虚寒,常作泄泻;与甘草同用,能调其辛辣之味,使不刺戟,而其温补之 
力转能悠长。《神农本草经》谓其逐风湿痹,指风湿痹之偏于寒者而言也,而《金匮》治热瘫痫,亦用干姜, 
风引汤中与石膏、寒水石并用者是也。此乃取其至辛之味,以开气血之凝滞也。有谓炮黑则性热,能助相火 
者,不知炮之则味苦,热力即减,且其气轻浮,转不能下达。 
徐灵胎曰∶“凡味浓之药主守,气浓之药主散,干姜气味俱浓,故散而能守。夫散不全散,守不全守,则 
旋转于经络脏腑之间,驱寒除湿、和血通气所必然矣,故性虽猛峻,不妨服食。” 
【附案】邻村高某年四十余,小便下血,久不愈。其脉微细而迟,身体虚弱恶寒,饮食减少。知其脾胃虚 
寒,中气下陷,黄坤载所谓血之亡于便溺者,太阴不升也。为疏方∶干姜、于术各四钱,生山药、熟地各六钱, 
乌附子、炙甘草各三钱,煎服一剂血见少,连服十余剂全愈。 

<目录>二、药物
<篇名>35.生姜解
属性:将鲜姜种于地中,秋后剖出去皮晒干为干姜;将姜上所生之芽种于地中,秋后剖出其当年所生之姜为生姜。 
是以干姜为母姜,生姜为子姜,干姜老而生姜嫩也。为生姜系嫩姜,其味之辛、性之温,皆亚于干姜,而所 
具生发之气则优于干姜,故能透表发汗。与大枣同用,善和营卫,盖借大枣之甘缓,不使透表为汗,惟旋转于 
营卫之间,而营卫遂因之调和也。其辛散之力,善开痰理气,止呕吐,逐除一切外感不正之气。若但用其皮, 
其温性稍减,又善通利小便。能解半夏毒及菌蕈诸物毒。食料中少少加之,可为健胃进食之品。疮家食之,致 
生恶肉,不可不知。 

<目录>二、药物
<篇名>36.附子、乌头、天雄解
属性:附子∶味辛,性大热。为补助元阳之主药,其力能升能降,能内达能外散,凡凝寒锢冷之结于脏腑、着于 
筋骨、痹于经络血脉者,皆能开之,通之。而温通之中,又大具收敛之力,故治汗多亡阳(汗多有亡阳亡阴之 
殊,亡阳者身凉,亡阴者身热,临证时当审辨。凉亡阳者,宜附子与萸肉、人参并用;热亡阴者,宜生地与萸 
肉、人参并用), 
肠冷泄泻,下焦阳虚阴走,精寒自遗,论者谓善补命门相火,而服之能使心脉跳动加速,是于君相二火皆能大 
有补益也。 
种附子于地,其当年旁生者为附子,其原种之附子则成乌头矣。乌头之热力减于附子,而宣通之力较优, 
故《金匮》治历节风有乌头汤;治心痛彻背、背痛彻心有乌头赤石脂丸,治寒疝有乌头煎、乌头桂枝汤等方。 
若种后不旁生附子,惟原种之本长大,若蒜之独头无瓣者,名谓天雄,为其力不旁溢,故其温补力更大而独能 
称雄也。今药局中所鬻之乌附子,其片大而且圆者即是天雄,而其黑色较寻常附子稍重,盖因其力大而色亦稍 
变也。附子、乌头、天雄,皆反半夏。 
【附案】一少妇上焦满闷烦躁,不能饮食,绕脐板硬,月信两月未见。其脉左右皆弦细。仲景谓双弦者寒, 
偏弦者饮,脉象如此,其为上有寒饮、下有寒积无疑。其烦躁者腹中寒气充溢,迫其元阳浮越也。投以理饮汤, 
去桂枝加附子三钱,方中芍药改用五钱,一剂满闷烦躁皆见愈。又服一剂能进饮食,且觉腹中凉甚,遂去芍 
药将附子改用五钱,后来又将干姜减半,附子加至八钱,服逾十剂,大便日行四五次,所下者多白色冷积, 
汤药仍日进一剂,如此五日,冷积泻尽,大便自止。再诊其脉,见有滑象,尺部较甚,疑其有妊,俾停药勿服, 
后至期果生子。夫附子原有殒胎之说,此证服附子如此之多,而胎固安然无恙,诚所谓“有故无殒亦无殒也”。 

<目录>二、药物
<篇名>37.肉桂解
属性:肉桂∶味辛而甘,气香而窜,性大热纯阳。为其为树身近下之皮,故性能下达,暖丹田,壮元阳,补相 
火。其色紫赤,又善补助君火,温通血脉,治周身血脉因寒而痹,故治关节腰肢疼痛及疮家白疽。木得桂则枯, 
且又味辛属金,故善平肝木,治肝气 
横恣多怒,若肝有热者,可以龙胆草、芍药诸药佐之。《神农本草经》谓其为诸药之先聘通使,盖因其香窜之 
气内而脏腑筋骨,外而经络腠理,倏忽之间莫不周遍,故诸药不能透达之处,有肉桂引之,则莫不透达也。 
按∶附子、肉桂,皆气味辛热,能补助元阳,然至元阳将绝,或浮越脱陷之时,则宜用附子而不宜用肉桂。 
诚以附子但味浓,肉桂则气味俱浓,补益之中实兼有走散之力,非救危抉颠之大药,观仲景《伤寒论》少阴诸 
方,用附子而不用肉桂可知也。 
肉桂气味俱浓,最忌久煎。而坊间又多捣为细末,数沸之后,药力即减,况煎至数十沸乎。至于石膏气味 
俱淡,且系石质,非捣细煎之,则药力不出,而坊间又多不为捣细。是以愚用石膏,必捣为细末然后煎之。若 
用肉桂,但去其粗皮,而以整块入煎。至药之类肉桂、类石膏者,可以肉桂、石膏为例矣。 
肉桂味辣而兼甜,以甜胜于辣者为佳,辣胜于甜者次之。然约皆从生旺树上取下之皮,故均含有油性, 
皆可入药,至其薄浓不必计也。若其味不但不甚甜,且不甚辣,又兼甚干枯者,是系枯树之皮,不可用也。 

<目录>二、药物
<篇名>38.知母解
属性:知母∶味苦,性寒,液浓而滑。其色在黄、白之间,故能入胃以清外感之热,伍以石膏可名白虎(二药再加 
甘草、粳米和之,名白虎汤,治伤寒温病热入阳明)。入肺以润肺金之燥,而肺为肾之上源,伍以黄柏兼能滋 
肾(二药少加肉桂向导,名滋肾丸),治阴虚不能化阳,小便不利。为其寒而多液,故能壮水以制火,治骨蒸 
劳热,目病 肉遮掩白睛。为其液寒而滑,有流通之性,故能消疮疡热毒肿疼。《神农本草经》谓主消渴者, 
以其滋阴壮水而渴自止也;谓其主肢体浮肿者,以其寒滑能通利水道而肿自消也;谓其益气者,以其能除食气 
之壮火而气自得其益也。 
知母原不甚寒,亦不甚苦,尝以之与黄 等分并用,即分毫不觉凉热,其性非大寒可知。又以知母一两 
加甘草二钱煮饮之,即甘胜于苦,其味非大苦可知。寒苦皆非甚大,而又多液是以能滋阴也。有谓知母但能退 
热,不能滋阴者,犹浅之乎视知母也。是以愚治热实脉数之证,必用知母,若用黄 补气之方,恐其有热不 
受者,亦恒辅以知母,惟有液滑能通大便,其人大便不实者忌之。 

<目录>二、药物
<篇名>39.天门冬解
属性:天冬∶味甘微辛,性凉,津液浓浓滑润。其色黄兼白,能入肺以清燥热,故善利痰宁嗽;入胃以消实热, 
故善生津止渴。津浓液滑之性,能通利二便、流通血脉、畅达经络,虽为滋阴之品,实兼能补益气分。 
《神农本草经》谓“天冬主暴风湿偏痹,强骨髓”二语,经后世注解,其理终未透彻。愚尝嚼服天门冬毫 
无渣滓,尽化津液,且觉兼有人参气味,盖其津浓液滑之中,原含有生生之气,其气挟其浓滑之津液以流行 
于周身,而痹之偏于半身者可除,周身之骨得其濡养而骨髓可健。且入药者为天冬之根,乃天冬之在内者也。 
其外生之蔓多有逆刺,若无逆刺者,其皮又必涩而戟手。天冬之物原外刚内柔也,而以之作药则为柔中含刚, 
是以痹遇其柔中之刚,则不期开而自开,骨得其柔中之刚,不惟健骨且能健髓也。至《名医别录》谓其“保定 
肺气,益气力,冷而能补”诸语,实亦有以见及此也。 
附录∶ 
湖北天门县崔××来函∶ 
向染咳嗽,百药不效,后每服松脂一钱,凉茶送服,不但咳嗽全愈,精神比前更强。迨读《衷中参西录》 
药物解,知天冬含有人参性味,外刚内柔,汁浆浓润,遂改服天冬二钱,日两次,今已三年,觉神清气爽, 
气力倍增,远行不倦,皮肤发润,面上瘢痕全消。 

<目录>二、药物
<篇名>40.麦门冬解
属性:麦冬∶味甘,性凉,气微香,津液浓浓,色兼黄白。能入胃以养胃液,开胃进食,更能入脾以助脾散精于 
肺,定喘宁嗽,即引肺气清肃下行,统调水道以归膀胱。盖因其性凉、液浓、气香,而升降濡润之中,兼具开 
通之力,故有种种诸效也,用者不宜去心 

<目录>二、药物
<篇名>41.黄连解
属性:黄连∶味大苦,性寒而燥。为苦为火之味,燥为火之性,故善入心以清热,心中之热清,则上焦之热皆清, 
故善治脑膜生炎、脑部充血、时作眩晕、目疾肿疼、 肉遮睛(目生云翳者忌用),及半身以上赤游丹毒。其 
色纯黄,能入脾胃以除实热,使之进食(西人以黄连为健胃药,盖胃有热则恶心懒食,西人身体强壮且多肉食, 
胃有积热故宜黄连清之),更由胃及肠,治肠 下利脓血。为其性凉而燥,故治湿热郁于心下作痞满(仲景小 
陷胸汤,诸泻心汤皆用之),女子阴中因湿热生炎溃烂。 
徐灵胎曰∶“苦属火性宜热此常理也。黄连至苦而反至寒,则得火之味与水之性,故能除水火相乱之病, 
水火相乱者湿热是也。是故热气目痛、 伤、泪出、目不明,乃湿热在上者;肠 、腹痛、下利,乃湿热在中 
者;妇人阴中肿痛,乃湿热在下者,悉能除之矣。凡药能去湿者必增热,能除热者必不能去湿,惟黄连能以苦 
燥湿,以寒除热,一举而两得焉。” 
黄连治目之功不必皆内服也。愚治目睛胀疼者,俾用黄连淬水,乘热屡用棉花瓤蘸擦眼上,至咽中觉苦 
乃止,则胀疼立见轻。又治目疾红肿作疼者,将黄连细末调以芝麻油,频频闻于鼻中,亦能立见效验。 

<目录>二、药物
<篇名>42.黄芩解
属性:黄芩∶味苦性凉。中空,最善清肺经气分之热,由脾而下通三焦,达于膀胱以利小便。又善入脾胃清热, 
由胃而下及于肠,以治肠 下利脓血。又善入肝胆清热,治少阳寒热往来(大小柴胡汤皆用之)。兼能调气, 
无论何脏腑,其气郁而作热者,皆能宣通之。又善清躯壳之热,凡热之伏藏于经络散漫于腠理者,皆能消除之。 
治肺病、肝胆病、躯壳病,宜用枯芩(即中空之芩);治肠胃病宜用条芩(即嫩时中不空者亦名子芩)。究之, 
皆为黄芩,其功用原无甚差池也。 
李濒湖曰∶“有人素多酒欲,病少腹绞痛不可忍,小便如淋诸药不效,偶用黄芩、木通、甘草三味,煎服 
遂止。”按∶黄芩治少腹绞痛,《名医别录》原明载之,由此见古人审药之精非后人所能及也。然必因热气所 
迫致少腹绞痛者始可用,非可概以之治腹痛也。又须知太阴腹痛无热证,必少阳腹痛始有热证,《名医别录》 
明标之曰“少腹绞痛”,是尤其立言精细处。 
濒湖又曰∶“余年二十时,因感冒咳嗽既久,且犯戒,遂病骨蒸发热,肤如火燎,每日吐痰碗许,暑月烦 
渴,寝食俱废,六脉浮洪,遍服柴胡、麦冬、荆沥诸药,月余益剧,皆以为必死矣。先君偶思李东垣治肺热如 
火燎,烦躁引饮而昼盛者气分热也,宜一味黄芩汤,以泻肺经气分之火。遂按方用片芩一两,水二盅煎一盅顿 
服,次日身热尽退,而痰嗽皆愈,药中肯 ,如鼓应桴,医中之妙有如此哉。”观濒湖二段云云,其善清气分 
之热,可为黄芩独具之良能矣。 

<目录>二、药物
<篇名>43.白茅根解
属性:白茅恨∶味甘,性凉,中空有节,最善透发脏腑郁热,托痘疹之毒外出;又善利小便淋涩作疼、因热小 
便短少、腹胀身肿;又能入肺清热以宁嗽定喘;为其味甘,且鲜者嚼之多液,故能入胃滋阴以生津止渴,并治 
肺胃有热、咳血、吐血、衄血、小便下血,然必用鲜者其效方着。春前秋后剖用之味甘,至生苗盛茂时,味即 
不甘,用之亦有效验,远胜干者。 
茅针∶即茅芽,初发犹未出土,形如巨针者,其性与茅根同,而稍有破血之力。凡疮溃脓未破者,将茅 
针煮服其疮即破,用一针破一孔,两针破两孔。 
【附案】一人年近五旬,受温疹之毒传染,痧疹遍身,表里壮热,心中烦躁不安,证实脉虚,六部不起, 
屡服清解之药无效,其清解之药稍重,大便即溏。俾用鲜茅根六两,煮汤一大碗顿服之,病愈强半,又服一次 
全愈。 
一西医得温病,头疼壮热,心中烦躁,自服西药退热之品,服后热见退,旋又反复。其脉似有力,惟在 
浮分、中分,俾用鲜茅根四两、滑石一两,煎三四沸,取汤服之,周身得微汗,一剂而诸病皆愈。 
一妇人年近四旬,因阴虚发热,渐觉小便不利,积成水肿,服一切通利小便之药皆无效。其脉数近六至, 
重按似有力,问其心中常觉烦躁,知其阴虚作热,又兼有实热,以致小便不利而成水肿也。俾用鲜茅根半斤, 
煎汤两大碗,以之当茶徐徐温饮之,使药力昼夜相继,连服五日,热退便利,肿遂尽消。 

<目录>二、药物
<篇名>44.苇茎、芦根解
属性:苇与芦原系一物,其生于水边干地,小者为芦,生于水深之 
处,大者为苇。芦因生于干地,其色暗绿近黑,故字从卢(卢即黑色),苇因生于水中,其形长大有伟然之意, 
故字从韦。《千金》苇茎汤∶薏苡仁、瓜瓣(即甜瓜瓣)各半升,桃仁五十枚,苇茎切二升,水二斗煮取五升, 
去渣纳前药三味,煮取二升,服一升,当有所见吐脓血。释者谓苇用茎不用根者,而愚则以为不然。根居于水 
底,是以其性凉而善升,患大头瘟者,愚常用之为引经要药(无苇根者,可代以荷叶),是其上升之力可至脑 
部而况于肺乎?且其性凉能清肺热,中空能理肺气,而又味甘多液,更善滋阴养肺,则用根实胜于用茎明矣。 
今药局所鬻者名为芦根,实即苇根也。其善发痘疹者,以其有振发之性也;其善利小便者,以其体中空且生水 
中自能行水也;其善止吐血衄血者,以其性凉能治血热妄行,且血亦水属(血中明水居多),其性能引水 
下行,自善引血下行也。其性颇近茅根,凡当用茅根而无鲜者,皆可以鲜芦根代之也。 

<目录>二、药物
<篇名>45.鲜小蓟根解
属性:鲜小蓟根∶味微辛,气微腥,性凉而润。为其气腥与血同臭,且又性凉濡润,故善入血分,最清血分之 
热。凡咳血、吐血、衄血、二便下血之因热者,服者莫不立愈。又善治肺病结核,无论何期用之皆宜,即单用 
亦可奏效。并治一切疮疡肿疼,花柳毒淋,下血涩疼,盖其性不但能凉血止血,兼能活血解毒,是以有以上种 
种诸效也。其凉润之性,又善滋阴养血,治血虚发热,至女子血崩赤带,其因热者用之亦效。 
按∶小蓟各处皆有,俗名刺尔菜(小蓟原名刺蓟),又名青青菜,山东俗名萋萋菜,奉天俗名枪刀菜,因 
其多刺如枪刀也。其叶长二寸许,宽不足一寸,叶边多刺,叶上微有绒毛,其叶皆在茎上,其茎紫色高尺许, 
茎端开紫花,花瓣如绒丝,其大如钱作圆形状,若小绒球,其花叶皆与红花相似,嫩时可作羹,其根与茎叶皆可 
用,而根之性尤良。剖取鲜者捣烂,取其自然汁冲开水服之。若以入煎剂不可久煎,宜保存其新鲜之性,约煎 
四五沸即取汤饮之。又其茎中生虫即结成疙疸,状如小枣,其凉血之力尤胜,若取其鲜者十余枚捣烂,开水冲 
服,以治吐血、衄血之因热者尤效。今药局中有以此为大蓟者,殊属差误。用时宜取其生农田之间嫩而白者。 
【附案】一少年素染花柳毒,服药治愈,惟频频咳嗽,服一切理嗽药皆不效。经西医验其血,谓仍有毒, 
其毒侵肺,是以作嗽。询方于愚,俾用鲜小蓟根两许,煮汤服之,服过两旬,其嗽遂愈。 
一少年每年吐血,反复三四次,数年不愈。诊其脉,血热火盛,俾日用鲜小蓟根二两,煮汤数盅,当茶饮 
之,连饮二十余日,其病从此除根。 


\chapter{大麦芽解}
属性:(附∶麦苗善治黄胆) 
大麦芽∶性平,味微酸(含有稀盐酸,是以善消)。能入脾胃,消化一切饮食积聚,为补助脾胃药之辅佐 
品(补脾胃以参、术、 为主,而以此辅之)。若与参、术、 并用,能运化其补益之力,不至作胀满。为其 
性善消化,兼能通利二便,虽为脾胃之药,而实善舒肝气(舒肝宜生用,炒用之则无效)。夫肝主疏泄为肾行 
气,为其力能舒肝,善助肝木疏泄以行肾气,故又善于催生。至妇人之乳汁为血所化,因其善于消化,微兼破 
血之性,故又善回乳(无子吃乳欲回乳者,用大麦芽二两炒为末,每服五钱白汤下)。入丸散剂可炒用,入汤剂 
皆宜生用。 
【附案】一妇人年三十余,气分素弱,一日忽觉有气结于上脘,不能上达亦不下降,俾单用生麦芽一两, 
煎汤饮之,顿觉气息通顺。 
一妇人年近四旬,胁下常常作疼,饮食入胃常停滞不下行,服药数年不愈,此肝不升胃不降也。为疏方, 
用生麦芽四钱以升 
肝,生鸡内金二钱以降胃,又加生怀山药一两以培养脏腑之气化,防其因升之、降之而有所伤损,连服十余剂, 
病遂全愈。 

<篇名>附∶麦苗善治黄胆
属性:内子王氏,生平不能服药,即分毫无味之药亦不能服。于乙丑季秋,得黄胆症,为开好服之药数味,煎汤, 
强令服之,下咽即呕吐大作,将药尽行吐出。友人张某谓,可用鲜麦苗煎汤服之。遂采鲜麦苗一握,又为之加 
滑石五钱,服后病即轻减,又服一剂全愈。盖以麦苗之性,能疏通肝胆,兼能清肝胆之热,犹能消胆管之炎, 
导胆汁归小肠也。因悟得此理后,凡遇黄胆症,必加生麦芽数钱于药中,亦奏效颇着。然药铺中麦芽皆干者, 
若能得鲜麦芽,且长至寸余用之,当更佳。或当有麦苗时,于服药之外,以麦苗煎汤当茶饮之亦可。 


<篇名>47.茵陈解
属性:茵陈者,青蒿之嫩苗也。秋日青蒿结子,落地发生,贴地大如钱,至冬霜雪满地,萌芽无恙,甫经立春即 
勃然生长,宜于正月中旬采之。其气微香,其味微辛微苦,秉少阳最初之气,是以凉而能散。《神农本草经》 
谓其善治黄胆,仲景治疸证,亦多用之。为其禀少阳初生之气,是以善清肝胆之热,兼理肝胆之郁,热消郁开, 
胆汁入小肠之路毫无阻隔也。《名医别录》谓其利小便,除头热,亦清肝胆之功效也。其性颇近柴胡,实较柴 
胡之力柔和,凡欲提出少阳之邪,而其人身弱阴虚不任柴胡之升散者,皆可以茵陈代之。 
【附案】一人因境多拂逆,常动肝气、肝火,致脑部充血作疼。治以镇肝、凉肝之药,服后周身大热, 
汗出如洗,恍悟肝为将军之官,中寄相火,用药强制之,是激动其所寄之相火而起反动 
力也。即原方为加茵陈二钱,服后即安然矣。 
一少年常患头疼,诊其脉肝胆火盛,治以茵陈、川芎、菊花各二钱,一剂疼即止。又即原方为加龙胆草二 
钱,服两剂觉头部轻爽异常,又减去川芎,连服四剂,病遂除根。 


<篇名>48.莱菔子解
属性:(附∶胡莱菔英能解砒石毒) 
莱菔子∶生用味微辛、性平,炒用气香性温。其力能升、能降,生用则升多于降,炒用则降多于升,取其 
升气化痰宜用生者,取其降气消食宜用炒者。究之,无论或生或炒,皆能顺气开郁、消胀除满,此乃化气之品, 
非破气之品,而医者多谓其能破气,不宜多服、久服,殊非确当之论。盖凡理气之药,单服久服,未有不伤气 
者,而莱菔子炒熟为末,每饭后移时服钱许,借以消食顺气,转不伤气,因其能多进饮食,气分自得其养也。 
若用以除满开郁,而以参、 、术诸药佐之,虽多服、久服,亦何至伤气分乎。 
【附案】一人年五旬,当极忿怒之余,腹中连胁下突然胀起,服诸理气开气之药皆不效。俾用生莱菔子一 
两,柴胡、川芎、生麦芽各三钱,煎汤两盅,分三次温服下,尽剂而愈。 
一人年二十五六,素多痰饮,受外感,三四日间觉痰涎凝结于上脘,阻隔饮食不能下行,须臾仍复吐出。 
俾用莱菔子一两,生熟各半,捣碎煮汤一大盅,送服生赭石细末三钱,迟点半钟,再将其渣重煎汤一大盅,仍 
送服生赭石细末三钱,其上脘顿觉开通,可进饮食,又为开辛凉清解之剂,连服两剂全愈。 

<篇名>附∶胡莱菔英能解砒石毒
属性:邑东境褚姓,因夫妻反目,其妻怒吞砒石。其夫出门未归,夜间砒毒发作,觉心中热渴异常。其锅中有泡 
干胡莱菔英之水若干,犹微温,遂尽量饮之,热渴顿止,迨其夫归犹未知也。隔 
旬,其夫之妹,在婆家亦吞砒石,急遣人来送信,其夫仓猝将往视之。其妻谓,将干胡莱菔英携一筐去,开水 
浸透,多饮其水必愈,万无一失。其夫问何以知之,其妻始明言前事。其夫果亦用此方,将其妹救愈。然所用 
者,是秋末所晒之干胡莱菔英,在房顶迭次经霜,其能解砒毒或亦借严霜之力欤?至鲜胡莱菔英亦能解砒毒否, 
则犹未知也。 

<目录>二、药物
<篇名>49.枸杞子、地骨皮解
属性:枸杞子∶味甘多液,性微凉。为滋补肝肾最良之药,故其性善明目,退虚热,壮筋骨,除腰疼,久服有 
益,此皆滋补肝肾之功也。乃因古有隔家千里,勿食枸杞之谚,遂疑其能助阳道,性或偏于温热。而愚则谓其 
性决不热,且确有退热之功效,此从细心体验而得,原非凭空拟议也。 
愚自五旬后,脏腑间阳分偏盛,每夜眠时,无论冬夏床头置凉水一壶,每醒一次,觉心中发热,即饮凉 
水数口,至明则壶中水已所余无几。惟临睡时,嚼服枸杞子一两,凉水即可少饮一半,且晨起后觉心中格外镇 
静,精神格外充足。即此以论枸杞,则枸杞为滋补良药,性未必凉而确有退热之功效,不可断言乎? 
或问∶枸杞为善滋阴故能退虚热,今先生因睡醒而觉热,则此热果虚热乎?抑实热乎?答曰∶余生平胖 
壮,阴分不亏,此非虚热明矣。然白昼不觉热,即夜间彻夜不睡,亦不觉热,惟睡初醒时觉心中发热,是热生 
于睡中也,其不同于泛泛之实热又明矣。此乃因睡时心肾自然交感而生热,乃先天元阳壮旺之现象,惟枸杞能 
补益元阴,与先天元阳相济,是以有此功效。若谓其仅能退虚热,犹浅之乎视枸杞矣。 
【附方】金髓煎 
枸杞子,逐日择红熟者,以无灰酒浸之,蜡纸封固,勿令泄 
气,两月足,取入砂盆中,研烂滤取汁,同原浸之酒入银锅内,慢火熬之,不住箸搅,恐粘住不匀,候成饧, 
净瓶密贮。每早温酒服二大匙,夜卧再服,百日身轻气壮。 
地骨皮∶即枸杞根上之皮也。其根下行直达黄泉,禀地之阴气最浓,是以性凉长于退热。为其力优于下行 
有收敛之力,是以治有汗骨蒸,能止吐血、衄血,更能下清肾热,通利二便,并治二便因热下血。且其收敛下 
行之力,能使上焦浮游之热因之清肃,而肺为热伤作嗽者,服之可愈。是以诸家本草,多谓其能治嗽也。惟肺 
有风邪作嗽者忌用,以其性能敛也。 

<目录>二、药物
<篇名>50.海螵蛸、茜草解
属性:乌 骨即海螵蛸,芦茹即茜草。详阅诸家本草,载此二药之主治,皆谓其能治崩带,是与《内经》用二药 
之义相合也。又皆谓其能消 瘕,是又与《内经》用二药之义相反也。本草所载二药之性,如此自相矛盾,令 
后世医者并疑《内经》之方而不敢轻用,则良方几埋没矣。而愚对于此二药,其能治崩带洵有确实征验,其能 
消瘕与否,则又不敢遽断也。 
《内经》有四乌 骨一芦茹丸,治伤肝之病,时时前后血。方用乌 骨四芦茹一丸,以雀卵如小豆大, 
每服五丸,鲍鱼汤送下。 
海螵蛸为乌贼鱼骨,其鱼常口中吐墨,水为之黑,故能补益肾经,而助其闭藏之用。友人孙××妻经水行 
时多而且久。孙××用微火,将海螵蛸煨至半黑、半黄为末,用鹿角胶化水送服,一次即愈,其性之收涩可知。 
茜草一名地血,可以染绛,《内经》名茹芦,即茹芦根也。 
【附案】忆在籍时,曾治沧州董姓妇人,患血崩甚剧。其脉象虚而无力,遂重用黄 、白术,辅以龙骨、 
牡蛎、萸肉诸收涩之品,服后病稍见愈,遂即原方加海螵蛸四钱,茜草二钱,服后其 
病顿愈,而分毫不见血矣。愚于斯深知二药止血之能力,遂拟得安冲汤、固冲汤二方,于方中皆用此二药。 
本邑一少妇,累年多病,身形羸弱,继又下白带甚剧,屡经医治不效。诊其脉迟弱无力,自觉下焦凉甚, 
治以清带汤,为加干姜六钱、鹿角胶三钱、炙甘草三钱,连服十剂全愈。统以上经验观之,则海螵蛸、茜草之 
治带下不又确有把握哉。至其能消 瘕与否,因未尝单重用之,实犹欠此经验而不敢遽定也。 

<目录>二、药物
<篇名>51.罂粟壳解
属性:罂粟壳∶即罂粟花所结之子外包之壳也。其所结之子形如罂,中有子如粟,可作粥,甚香美(炒之则香), 
故名其外皮为罂粟壳,药局间省文曰米壳。其味微酸,性平。其嫩时皮出白浆可制鸦片,以其犹含鸦片之余气, 
故其性能敛肺、涩肠、固肾,治久嗽、久痢、遗精、脱肛、女子崩带。嗽、痢初起及咳嗽兼外感者,忌用。 
罂粟壳,治久嗽、久痢,诚有效验,如虚劳咳嗽证,但用山药、地黄、枸杞、玄参诸药以滋阴养肺,其 
嗽不止者,加罂粟壳二三钱,则其嗽可立见轻减,或又少佐以通利之品,若牛蒡、射干诸药尤为稳妥。至于久 
痢,其肠中或有腐烂,若用三七、鸦胆子,化其腐烂,而其痢仍不止者,当将罂粟壳数钱,与山药、芍药诸药 
并用,连服数剂,其痢可全愈。 

<目录>二、药物
<篇名>52.竹茹解
属性:竹茹∶味淡,性微凉。善开胃郁,降胃中上逆之气使之下行(胃气息息下行为顺),故能治呕吐、止吐 
血、衄血(皆降胃之功)。《金匮》治妇人乳中虚、烦乱呕逆,有竹皮大丸,竹皮即竹茹也。为其为竹之皮, 
且凉而能降,故又能清肺利痰,宣通三焦水道下通膀胱,为通利小便之要药,与叶同功而其力尤胜于叶。又善 
清肠中之热, 
除下痢后重腹疼。为其凉而宣通,损伤瘀血肿疼者,服之可消肿愈疼,融化瘀血。醋煮口漱,可止齿龈出血。 
须用嫩竹外边青皮,里层者力减。 
【附案】族家婶母,年四旬,足大指隐白穴处,忽然破裂出血,且色紫甚多,外科家以为疔毒,屡次服药 
不效。时愚甫习医,诊其脉洪滑有力,知系血热妄行,遂用生地黄两半、碎竹茹六钱,煎汤服之,一剂血止, 
又服数剂,脉亦平和。盖生地黄凉血之力,虽能止血,然恐止后血瘀经络致生他病,辅以竹茹宣通消瘀,且其 
性亦能凉血止血,是以有益而无弊也。 
友人刘××之女,得温病,邀愚往视。其证表里俱热,胃口满闷,时欲呕吐,舌苔白而微黄,脉象洪滑,重 
按未实,问其大便,昨行一次微燥,一医者欲投以调胃承气汤,疏方尚未取药。愚曰∶“此证用承气汤尚早”, 
遂另为疏方,用生石膏一两、碎竹茹六钱、青连翘四钱,煎汤服后,周身微汗,满闷立减,亦不复欲呕吐,从 
前小便短少,自此小便如常,其病顿愈。 

<目录>二、药物
<篇名>53.沙参解
属性:沙参∶味淡微甘,性凉。色白,质松,中空,故能入肺,清热滋阴,补益肺气,兼能宣通肺郁,故《神农 
本草经》谓其主血积,肺气平而血之上逆者自消也。人之魂藏于肝,魄藏于肺,沙参能清补肺脏以定魄,更能 
使肺金之气化清肃下行,镇戢肝木以安魂,魂魄安定,惊恐自化,故《神农本草经》又谓主惊气也。 
徐灵胎曰∶“肺主气,故肺家之药气胜者为多。但气胜之品必偏于燥,而能滋肺者又腻滞而不清虚,惟沙 
参为肺家气分中理血药,色白体轻,疏通而不燥,润泽而不滞,血阻于肺者,非此不能清也。” 
沙参以体质轻松,中心空者为佳,然必生于沙碛之上,土性 
松活,始能如此。渤海之滨,沙碛绵亘,纯系蚌壳细末,毫无土质,其上所长沙参,粗如拇指,中空大于藕孔。 
其味且甘于他处沙参,因其处若三四尺深即出甜水,是以所长之沙参,其味独甘,鲜嚼服之,大能解渴,故以 
治消渴尤良。其叶光泽如镜,七月抽茎开白花,纯禀金气,肺热作嗽者,用之甚效,洵良药也。 
【附案】近族曾孙女××,自动失乳,身形羸弱,自六七岁时恒发咳嗽,后至十一二岁嗽浸增剧,概服治 
嗽药不效。愚俾用生怀山药细末熬粥,调以白糖令适口,送服生鸡内金细末二三分,或西药百布圣二瓦,当点 
心服之,年余未间断。劳嗽虽见愈,而终不能除根。诊其脉,肺胃似皆有热,遂俾用北沙参轧为细末,每服 
二钱,日两次。服至旬余,咳嗽全愈。然恐其沙参久服或失于凉,改用沙参三两、甘草二两,共轧细,亦每服 
二钱,以善其后。 

<目录>二、药物
<篇名>54.连翘解
属性:连翘∶味淡微苦,性凉。具升浮宣散之力,流通气血,治十二经血凝气聚,为疮家要药。能透表解肌,清 
热逐风,又为治风热要药。且性能托毒外出,又为发表疹瘾要药。为其性凉而升浮,故又善治头目之疾,凡头 
疼、目疼、齿疼、鼻渊或流浊涕成脑漏证,皆能主之。为其味淡能利小便,故又善治淋证,溺管生炎。 
仲景方中所用之连招,乃连翘之根,即《神农本草经》之连根也。其性与连翘相近,其发表之力不及 
连翘,而其利水之力则胜于连翘,故仲景麻黄连轺赤小豆汤用之,以治瘀热在里,身将发黄,取其能导引湿热 
下行也。 
连翘诸家皆未言其发汗,而以治外感风热,用至一两必能出汗,且其发汗之力甚柔和,又甚绵长。曾治 
一少年风温初得,俾单用连翘一两煎汤服,彻夜微汗,翌晨病若失。 
连翘善理肝气,既能舒肝气之郁,又能平肝气之盛。曾治一媪,年过七旬,其手连臂肿疼数年不愈,其脉 
弦而有力,遂于清热消肿药中,每剂加连翘四钱,旬日肿消疼愈,其家人谓媪从前最易愤怒,自服此药后不但 
病愈,而愤怒全无,何药若是之灵妙也!由是观之,连翘可为理肝气要药矣。 

<目录>二、药物
<篇名>55.川楝子解
属性:川楝子∶大如栗者是川楝子,他处楝子小而味苦,去核名金铃子。味微酸、微苦,性凉。酸者入肝,苦者 
善降,能引肝胆之热下行自小便出,故治肝气横恣,胆火炽盛,致胁下掀疼。并治胃脘气郁作疼,木能疏土也。 
其性虽凉,治疝气者恒以之为向导药,因其下行之力能引诸药至患处也。至他处之苦楝子,因其味苦有小毒, 
除虫者恒用之。 

<目录>二、药物
<篇名>56.薄荷解
属性:薄荷∶味辛,气清郁香窜,性平,少用则凉,多用则热(如以鲜薄荷汁外擦皮肤少用殊觉清凉,多用即觉 
灼热)。其力能内透筋骨,外达肌表,宣通脏腑,贯串经络,服之能透发凉汗,为温病宜汗解者之要药。若少 
用之,亦善调和内伤,治肝气胆火郁结作疼,或肝风内动,忽然痫痉螈 ,头疼目疼,鼻渊鼻塞,齿疼咽喉肿 
疼,肢体拘挛作疼,一切风火郁热之疾,皆能治之。痢疾初起挟有外感者,亦宜用之,散外感之邪,即以清肠 
中之热,则其痢易愈。又善消毒菌(薄荷冰善消霍乱毒菌薄荷亦善消毒菌可知),逐除恶气,一切霍乱痧证, 
亦为要药。为其味辛而凉,又善表疹瘾,愈皮肤瘙痒,为儿科常用之品。 
温病发汗用薄荷,犹伤寒发汗用麻黄也。麻黄服后出热汗,热汗能解寒,是以宜于伤寒;薄荷服后出凉汗, 
凉汗能清温,是以宜于温病。若以麻黄发温病之汗,薄荷发伤寒之汗,大抵皆不 
能出汗,即出汗亦必不能愈病也。 
薄荷古原名苛,以之作蔬,不以之作药,《神农本草经》、《名医别录》皆未载之,至唐时始列于药品, 
是以《伤寒论》诸方未有用薄荷者。然细审《伤寒论》之方,确有方中当用薄荷,因当时犹未列入药品,即当 
用薄荷之方,不得不转用他药者。试取伤寒之方论之,如麻杏甘石汤中之麻黄,宜用薄荷代之,盖麻杏甘石汤, 
原治汗出而喘无大热,既云无大热,其仍有热可知,有热而犹用麻黄者,取其泻肺定喘也。然麻黄能泻肺定喘, 
薄荷亦能泻肺定喘(薄荷之辛能抑肺气之盛,又善搜肺风),用麻黄以热治热,何如用薄荷以凉治热乎?又如 
凡有葛根诸汤中之葛根,亦可以薄荷代之,盖葛根原所以发表阳明在经之热,葛根之凉不如薄荷,而其发表之 
力又远不如薄荷,则用葛根又何如用薄荷乎?斯非背古训也,古人当药物未备之时,所制之方原有不能尽善尽 
美之处,无他,时势限之也。吾人当药物既备之时,而不能随时化裁与古为新,是仍未会古人制方之意也。医界 
之研究伤寒者,尚其深思愚言哉。 

<目录>二、药物
<篇名>57.茯苓、茯神解
属性:茯苓∶气味俱淡,性平。善理脾胃,因脾胃属土,土之味原淡(土味淡之理,徐灵胎曾详论之),是 
以《内经》谓淡气归胃,而《慎柔五书》上述《内经》之旨,亦谓味淡能养脾阴。盖其性能化胃中痰饮为 
水液,引之输于脾而达于肺,复下循三焦水道以归膀胱,为渗湿利痰之主药。然其性纯良,泻中有补,虽为渗 
利之品,实能培土生金,有益于脾胃及肺。且以其得松根有余之气,伏藏地中不外透生苗,故又善敛心气之浮 
越以安魂定魄,兼能泻心下之水饮以除惊悸,又为心经要药。且其伏藏之性,又能敛抑外越之水气转而下注, 
不使作汗透出,兼为止汗之要药也。其抱根而生者为茯神,养心之力,较胜于茯苓。茯苓若入煎剂,其切作块者,终日 
煎之不透,必须切薄片,或捣为末,方能煎透。 
【附录】友人竹××曰∶“嵊县吴氏一家,以种苓为业。春间吴氏之媳病,盖产后月余,壮热口渴不引饮, 
汗出不止,心悸不寐,延余往治。病患面现红色,脉有滑象,急用甘草、麦冬、竹叶、柏子仁、浮小麦、大 
枣煎饮不效;继用酸枣仁汤,减川芎加浮小麦、大枣,亦不效;又用归脾汤加龙骨、牡蛎、萸肉则仍然如故。 
当此之时,余束手无策,忽一人进而言曰∶‘何不用补药以缓之’,余思此无稽之谈,所云补药者,心无见识 
也,姑漫应之。时已届晚寝之时,至次日早起,其翁奔告曰∶‘予媳之病昨夜用补药医痊矣。’余将信将疑, 
不识补药究系何物。乃翁持渣来见,钵中有茯苓四五两,噫!茯苓焉,胡为云补药哉?余半晌不能言。危坐思 
之,凡病有一线生机,皆可医治。茯苓固治心悸之要药,亦治汗出之主药。仲景治伤寒汗出而渴者五苓散,不 
渴者茯苓甘草汤。伤寒厥而心下悸者宜先治水,当服茯苓甘草汤。可知心悸者汗出过多,心液内涸,肾水上救 
入心则悸,余药不能治水,故用茯苓以镇之。是证心悸不寐,其不寐由心悸而来,即心悸亦从汗出而来,其壮 
热口渴不引饮、脉滑,皆有水气之象,今幸遇种苓家,否则汗出不止,终当亡阳,水气凌心,必当灭火,是谁 
之过欤?余引咎而退。”观竹××此论,不惜暴一己之失,以为医界说法,其疏解经文之处,能将仲景用茯苓 
之深意,彰彰表出,固其析理之精,亦见其居心之浓也。 
湖北天门县崔××来函云∶一九三○年,李姓妇,头目眩晕、心中怔忡、呕吐涎沫,有时觉气上冲,昏 
愦不省人事。他医治以安神之药无效,继又延医十余人皆服药无效,危险已至极点。生诊其脉,浮而无力,视 
其形状无可下药。恍悟《衷中参西录》茯苓解中,所论重用茯苓之法,当可挽回此证。遂俾单用茯苓一两煎汤 
服之,服后甫五分钟,病即轻减,旋即煎渣再服,益神 
清气爽,连服数剂,病即全愈。后每遇类此证者,投此方皆可奏效。 

<目录>二、药物
<篇名>58.木通解
属性:木通∶味苦性凉。为藤蔓之梗,其全体玲珑通彻,故能贯串经络,通利九窍。能泻上焦之热,曲曲引之下 
行自水道达出,为利小便清淋浊之要药。其贯串经络之力,又能治周身拘挛,肢体痹疼,活血消肿,催生通乳, 
多用亦能发汗。 
愚平素不喜用苦药,木通诸家未尝言苦,而其味实甚苦。因虑人嫌其苦口难服,故于木通未尝独用重用, 
以资研究,近因遇一肢体关节肿疼证,投以清热利湿活血之品,更以西药阿斯匹林佐之,治愈。适法库门生万 
××来奉,因向彼述之,万××曰∶《医宗金鉴》治三痹(行痹痛痹着痹)有木通汤方,学生以治痛痹极有效验, 
且服后必然出汗,曾用数次皆一剂而愈。”愚曰∶“我亦见其方,但未尝试用,故不知如此神效,既效验如此, 
当急录出以公诸医界。”爰列其方于下∶ 
【木通汤】用木通一味,不见水者(其整者皆未见水,捣碎用)二两,以长流水二碗煎一碗,热服取微汗, 
不愈再服,以愈为度。若其痛上下左右流走相移者,加羌活、防风以祛风邪;其痛凉甚者,有汗加附子,无汗 
加麻黄以去寒邪;其痛重着难移者,加防己以胜湿邪。其所应加之药,不可过三钱,弱者俱减半服。 

<目录>二、药物
<篇名>59.蒲黄解
属性:蒲黄∶味淡微甘微辛,性凉。善治气血不和、心腹疼痛、游风肿疼、颠仆血闷(用生蒲黄半两,煎汤 
灌下即醒)、痔疮出血(水送服一钱,日三次)、女子月闭腹痛、产后瘀血腹疼,为其有活血化瘀之力,故有 
种种诸效。若炒熟用之(不宜炒黑),又善治吐血、咳血、衄血、二便下血、女子 
血崩带下。外用治舌胀肿疼,甚或出血,一切疮疡肿疼,蜜调敷之(皆宜用生者),皆有捷效。为其生于水中, 
且又味淡,故又善利小便。 
邹润安曰∶“《金匮》用蒲灰散,利小便治厥而为皮水,解者或以为香蒲,或以为蒲席烧灰,然香蒲但 
能清上热,不云能利水,败蒲席,《名医别录》主筋溢恶疮,亦非利水之物。蒲黄,《神农本草经》主利小便, 
且《本事方》、《芝隐方》,皆述其治舌胀神验,予亦曾治多人,毫丝不爽,不正合治水之肿于皮乎?夫皮 
水为肤腠间病,不应有厥,厥者下焦病也。膀胱与肾为表里,膀胱以水气归皮,致小便不利,气阻而成寒热,则 
肾亦承其弊为之阴壅而阳不得达,遂成厥焉。病本在外,非可用温,又属皮水,无从发散,计惟解心腹膀胱之 
寒热,使小便得利,又何厥逆之有,以是知其为蒲黄无疑也。曰蒲灰者,蒲黄之质,固有似于灰也。” 
蒲黄诚为妙药,失笑散用蒲黄、五灵脂等分生研,每用五钱,水、酒各半,加醋少许,煎数沸连渣服之, 
能愈产后腹疼于顷刻之间。人多因蒲黄之质甚软,且气味俱淡,疑其无甚力量而忽视之,是皆未见邹氏之论, 
故不能研究《神农本草经》主治之文也。 

<目录>二、药物
<篇名>60.三棱、莪术解
属性:三棱∶气味俱淡,微有辛意。莪术∶味微苦,气微香,亦微有辛意。性皆微温,为化瘀血之要药。以治 
男子 癖,女子 瘕,月闭不通,性非猛烈而建功甚速。其行气之力,又能治心腹疼痛,胁下胀疼,一切血凝 
气滞之证。若与参、术、 诸药并用,大能开胃进食,调血和血。若细核二药之区别,化血之力三棱优于莪术, 
理气之力莪术优于三棱。 
药物恒有独具良能,不能从气味中窥测者,如三棱、莪术性近和平,而以治女子瘀血,虽坚如铁石亦能徐 
徐消除,而猛烈开 
破之品转不能建此奇功,此三棱、莪术独具之良能也。而耳食者流,恒以其能消坚开瘀,转疑为猛烈之品而不 
敢轻用,几何不埋没良药哉。 
三棱、莪术,若治陡然腹胁疼痛,由于气血凝滞者,可但用三棱、莪术,不必以补药佐之;若治瘀血积久 
过坚硬者,原非数剂所能愈,必以补药佐之,方能久服无弊。或用黄 六钱,三棱、莪术各三钱,或减黄 三 
钱,加野台参三钱,其补破之力皆可相敌,不但气血不受伤损,瘀血之化亦较速,盖人之气血壮旺,愈能驾驭 
药力以胜病也。 

<目录>二、药物
<篇名>61.乳香、没药解
属性:乳香∶气香窜,味淡,故善透窍以理气。没药∶气则淡薄,味则辛而微酸,故善化瘀以理血。其性皆微温, 
二药并用为宣通脏腑流通经络之要药。故凡心胃胁腹肢体关节诸疼痛皆能治之。又善治女子行经腹疼,产后瘀 
血作疼,月事不以时下。其通气活血之力,又善治风寒湿痹,周身麻木,四肢不遂及一切疮疡肿疼,或其疮硬 
不疼。外用为粉以敷疮疡,能解毒、消肿、生肌、止疼,虽为开通之品,不至耗伤气血,诚良药也。 
乳香、没药不但流通经络之气血,诸凡脏腑中,有气血凝滞,二药皆能流通之。医者但知其善入经络,用 
之以消疮疡,或外敷疮疡,而不知用之以调脏腑之气血,斯岂知乳香、没药者哉。 
乳香、没药,最宜生用,若炒用之则其流通之力顿减,至用于丸散中者,生轧作粗渣入锅内,隔纸烘至半 
熔,候冷轧之即成细末,此乳香、没药去油之法。 

<目录>二、药物
<篇名>62.常山解
属性:常山∶性凉,味微苦。善消脾中之痰,为治疟疾要药(疟疾皆系脾中 
多痰,凡久疟胁下有硬块名疟母者,皆系脾胀兼有痰也)。少服,则痰可徐消,若多服即可将脾中之痰吐出, 
为其多服即作呕吐,故诸家本草皆谓其有毒,医者用之治疟,亦因此不敢多用,遂至有效有不效。若欲用之必 
效,当效古人一剂三服之法,用常山五六钱,煎汤一大盅,分五六次徐徐温饮下,即可不作呕吐,疟疾亦有八九可愈。 
【附案】一九一七年,时当仲夏,愚因劳碌过度,兼受暑,遂至病疟。乃于不发疟之日清晨,用常山八 
钱,煎汤一大碗,徐徐温饮之,一次止饮一大口,饮至日夕而剂尽,心中分毫未觉难受,而疟亦遂愈。后遂变 
汤剂为丸剂,将常山轧细过罗,水泛为丸,桐子大,每服八分,一日之间自晨至暮服五次,共服药四钱,疟亦 
可愈。若病发时热甚剧者,可用生石膏一两煎汤,初两次服药时,可用此汤送服。西人谓病疟者有疟虫,西药金 
鸡纳霜,善除疟虫故善治疟,常山想亦善除疟虫之药品欤? 

<目录>二、药物
<篇名>63.山楂解
属性:山楂∶味至酸微甘,性平。皮赤肉红黄,故善入血分为化瘀血之要药。能除 癖 瘕、女子月闭、产后 
瘀血作疼(俗名儿枕疼)。为其味酸而微甘,能补助胃中酸汁,故能消化饮食积聚,以治肉积尤效。其化瘀之 
力,更能蠲除肠中瘀滞,下痢脓血,且兼入气分以开气郁痰结,疗心腹疼痛。若以甘药佐之(甘草蔗糖之类,酸 
甘相合,有甲己化土之义),化瘀血而不伤新血,开郁气而不伤正气,其性尤和平也。 
女子至期,月信不来,用山楂两许煎汤,冲化红蔗糖七八钱服之即通,此方屡试屡效。若月信数月不通者, 
多服几次亦通下。 
痢疾初得者,用山楂一两,红白蔗糖各五钱,好毛尖茶叶钱半,将山楂煎汤,冲糖与茶叶在盖碗中,浸片 
时,饮之即愈。 
附∶《本草纲目》山楂后载有两方∶一方治肠风下血,若用 
凉药、热药、补脾药俱不效者,独用干山楂为末,艾叶煎汤调下,应手即愈;一方治痘疹干黑危困者,用山楂 
为末,紫草煎酒调服一钱。按∶此二方皆有效验,故附载之。 

<目录>二、药物
<篇名>64.石榴解
属性:石榴∶有酸、甜二种,以酸者为石榴之正味,故入药必须酸者。其性微凉,能敛戢肝火,保合肺气,为治 
气虚不摄肺劳喘嗽之要药。又为治肝虚风动相火浮越之要药。若连皮捣烂煮汤饮之,又善治大便滑泻、小便不 
禁、久痢不止、女子崩带,以其皮中之液最涩,故有种种诸效也。 
【附案】门生高××之父,曾向愚问治泄泻方,语以酸石榴连皮捣烂,煮服甚效。后岁值壬寅,霍乱盛 
行,有甫受其病泄泻者,彼与以服酸石榴方,泄泻止而病亦遂愈。盖霍乱之上吐下泻,原系肝木挟外感之毒克 
伐脾胃,乃当其病势犹未横恣,急以酸石榴敛戢肝木,使不至助邪为虐致吐泻不已,则元气不漓,自可以抗 
御毒菌,况酸石榴之味至酸,原有消除毒菌之力乎(凡味之至酸者,皆善消)!古方治霍乱多用木瓜,取其酸 
能敛肝也,酸石榴之酸远胜木瓜,是以有效也。 
邻村张氏妇,年过四旬,素息肺劳喘嗽,夜不安枕者已数年矣。无论服何药皆无效验。一晚偶食酸石榴, 
觉夜间喘嗽稍轻,从此每晚服之,其喘嗽日轻,一连服过三月,竟脱然无累矣。 

<目录>二、药物
<篇名>65.龙眼肉解
属性:龙眼肉∶味甘,气香,性平。液浓而润,为心脾要药。能滋生心血(凡药之色赤液浓而甘者,皆能生血), 
兼能保合心气(甘而且香者皆能助气),能滋补脾血(味甘归脾),兼能强健脾胃(气香能醒脾),故能治思 
虑过度,心脾两伤(脾主思,过思则伤脾),或心虚怔忡、寝不成寐,或脾虚泄泻,或脾虚不 
能统血,致二便下血。为其味甘能培补脾土,即能有益肺金(土生金),故又治肺虚劳嗽、痰中带血,食之甘 
香适口,以治小儿尤佳。 
【附案】一少年心中怔忡,夜不能寐,其脉弦硬微数,知其心脾血液短也,俾购龙眼肉,饭甑蒸熟,随便 
当点心,食之至斤余,病遂除根。 
一六七岁童子,大便下血,数月不愈,服药亦无效。亦俾蒸熟龙眼肉服之,约日服两许,服旬日全愈。 

<目录>二、药物
<篇名>66.柏子仁解
属性:柏子仁∶味微甘微辛,气香性平,多含油质。能补助心气,治心虚惊悸怔仲;能涵濡肝木,治肝气横恣胁 
疼;滋润肾水,治肾亏虚热上浮。虽含油质甚多,而性不湿腻,且气香味甘实能有益脾胃,《神农本草经》谓其 
除风湿痹,胃之气化壮旺,由中四达而痹者自开也。其味甘而兼辛,又得秋金肃降之气,能入肺宁嗽定喘,导 
引肺气下行。统言之,和平纯粹之品,于五脏皆有补益,故《神农本草经》谓安五脏也。宜去净皮,炒香用之, 
不宜去油。 
《神农本草经》谓柏实能安五脏,而实于肝脏尤宜也。曾治邻村毛姓少年,其肝脏素有伤损,左关脉独微 
弱,一日忽胁下作疼,俾单用柏子仁一两,煎汤服之立愈。观此,则柏子仁善于理肝可知矣。 

<目录>二、药物
<篇名>67.大枣解
属性:大枣∶味甘微辛,性温。其津液浓浓滑润,最能滋养血脉、润泽肌肉、强健脾胃、固肠止泻、调和百药 
能缓猛药健悍之性,使不伤脾胃。是以十枣汤、葶苈大枣汤诸方用之。若与生姜并 
用,为调和营卫之妙品,是以桂枝汤、柴胡汤诸方用之。《内经》谓其能安中者,因其味至甘能守中也。又谓 
其能通九窍者,因其津液滑润且微有辛味,故兼有通利之能也。谓其补少气少津液者,为其味甘能益气,其津 
液浓浓滑润,又能补人身津液之不足也。虽为寻常食品,用之得当能建奇功。 
周伯度曰∶“生姜味辛色黄,由阳明入卫,大枣味甘色赤,由太阴入营。其能入营由于甘中有辛,惟其甘 
守之力多,得生姜乃不至过守;生姜辛通之力多,得大枣乃不至过通,二药并用所以为和营卫主剂。” 
《神农本草经》名之为大枣者,别于酸枣仁之小枣也。凡枣之酸者皆小,甘者皆大,而大枣又非一种,约 
以生食不脆、干食肉多、味极甘者为入药之品。若用为服食之物,而日日食之者,宜先用水将枣煮两三沸,迟 
一点钟将枣捞出(此时尝其煮枣之水甚苦,故先宜将苦水煮出),再用饭甑上蒸熟,则其味甘美,其性和平, 
可以多服久服,不至生热。 
【附案】邑中友人赵××,身体素羸弱,年届五旬,饮食减少,日益消瘦,询方于愚,俾日食熟大枣数十枚, 
当点心用之。后年余觌面貌较前丰腴若干,自言∶“自闻方后,即日服大枣,至今未尝间断,饮食增于从前三 
分之一,是以身形较前强壮也。” 
表叔高××,年过五旬,胃阳不足,又兼肝气郁结,因之饮食减少,时觉满闷,服药半载,毫无效验。适 
愚远游还里,觌面谈及,俾用大枣六斤,生姜一斤,切片,同在饭甑蒸熟,臼内捣如泥,加桂枝尖细末三两, 
炒熟麦面斤半,和匀捏成小饼,炉上炙干,随意当点心服之,尽剂而愈。 

<目录>二、药物
<篇名>68.胡桃解
属性:胡桃(亦名核桃)∶味微甘,气香,性温。多含油质,将油 
榨出,须臾即变黑色。为滋补肝肾、强健筋骨之要药,故善治腰疼腿疼,一切筋骨疼痛。为其能补肾,故能固 
齿牙、乌须发,治虚劳喘嗽、气不归元、下焦虚寒、小便频数、女子崩带诸证。其性又能消坚开瘀,治心腹疼 
痛、砂淋、石淋、杜塞作疼、肾败不能漉水、小便不利。或误吞铜物,多食亦能消化(试与铜钱同嚼,其钱 
即碎,能化铜可知)。又善消疮疽及皮肤疥癣头上白秃,又能治疮毒深入骨髓,软弱不能步履。 
【附案】一幼童,五龄犹不能行,身多疮疡,治愈复发,知其父素有梅毒,此系遗传性病在骨髓也。为 
疏方,每剂中用胡桃仁八钱,佐以金银花、白鲜皮、土茯苓、川贝母、玄参、甘草诸药,如此方少有加减,服 
药二十余剂,其疮皆愈,从此渐亦能行步矣。 
附∶古方治虚寒喘嗽、腰腿酸痛,用胡桃仁二十两烂研,补骨脂十两酒蒸为末,密调如饴,每晨酒服一大 
匙,不能饮者热水调服。汪 庵谓,补骨脂属火,入心包命门能补相火以通君火、暖丹田、壮元阳;胡桃属木, 
能通命门、利三焦、温肺润肠、补养气血,有木火相生之妙。愚常用之以治下焦虚寒之证,诚有奇效。 
上方加杜仲一斤,生姜炒蒜四两,同为丸,名青娥丸。治肾虚腰疼,而此方不但治肾虚腰疼也,以治虚寒 
腿疼亦极效验。曾治一媪年过六旬,腿疼年余不愈,其脉两尺沉细,俾日服青娥丸月余全愈。若虚寒之甚者, 
可于方中加生硫黄三两。 

<目录>二、药物
<篇名>69.五味子解
属性:五味子∶性温,五味俱备,酸、咸居多。其酸也能敛肺,故《神农本草经》谓主咳逆上气;其咸也能滋肾, 
故《神农本草经》谓其强阴益男子精。其酸收之力,又能固摄下焦气化,治五更泄 
泻、梦遗失精,及消渴小便频数,或饮一溲一,或饮一溲二。其至酸之味,又善入肝,肝开窍于目,故五味 
子能敛瞳子散大。然其酸收之力甚大,若咳逆上气挟有外感者,须与辛散之药同用(若干姜、生姜、麻黄、细 
辛诸药),方能服后不至留邪。凡入煎剂宜捣碎,以其仁之味辛与皮之酸味相济,自不至酸敛过甚,服之作胀满也。 
邹润安曰∶“《伤寒论》中,凡遇咳者,总加五味子、干姜,义甚深奥,经云‘脾气散精,上归于肺,’ 
是故咳虽肺病,而其源实主于脾,惟脾家所散上归之精不清,则肺家通调水道之令不肃,后人治咳但知润肺消 
痰,不知润肺则肺愈不清,消痰则转能伤脾,而痰之留于肺者究莫消也。干姜温脾肺,是治咳之来路,来路清 
则咳之源绝矣;五味使肺气下归于肾,是治咳之去路,去路清则气肃降矣。合两药而言,则为一开一阖,当开 
而阖是为关门逐盗;当阖而开则恐津液消亡,故小青龙汤及小柴胡汤、真武汤、四逆散之兼咳者皆用之,不嫌 
其表里无别也。” 

<目录>二、药物
<篇名>70.萆解
属性:萆 ∶味淡,性温。为其味淡而温,故能直趋膀胱温补下焦气化,治小儿夜睡遗尿,或大人小便频数, 
致大便干燥。其温补之性,兼能涩精秘气,患淋证者禁用,醒脾升陷汤后曾详论之。 
萆 为治失溺要药不可用之治淋。《名医别录》谓萆 治阴萎、失溺、老人五缓。盖失溺之证实因膀胱 
之括约筋少约束之力,此系筋缓之病,实为五缓之一,萆 善治五缓,所以治之。拙拟醒脾升陷汤中,曾重用 
萆以治小便频数不禁,屡次奏效,是萆 为治失溺之要药可知矣。乃萆 厘清饮竟用之以治膏淋,何其背 
谬若是?愚在籍时,邻村有病淋者,医者投以萆 厘清饮,两剂,其人小便滴沥不通。再服各种利小便药,皆 
无效。后延愚延医,已至十日,精神昏愦,毫无知觉,脉数近十至,按 
之即无,因谓其家人曰∶“据此脉论,即小便通下,亦恐不救。”其家人恳求甚切,遂投以大滋真阴之剂,以 
利水之药佐之。灌下移时,小便即通,床褥皆湿。再诊其脉,微细欲无,愚急辞归。后闻其人当日即亡。近又 
在津治一淋证,服药十剂已愈,隔两月病又反复,时值愚回籍,遂延他医治疗,方中亦重用萆 。服两剂,小 
便亦滴沥不通,服利小便药亦无效。遂屡用西法引溺管兼服利小便之药,治近一旬,小便少通滴沥,每小便一 
次,必须两小时。继又服滋阴利水之药十剂始全愈。 

<目录>二、药物
<篇名>71.鸡内金解
属性:鸡内金∶鸡之脾胃也,其中原含有稀盐酸,故其味酸而性微温,中有瓷、石、铜、铁皆能消化,其善化 
瘀积可知。《内经》谓“诸湿肿满,皆属于脾”,盖脾中多回血管,原为通彻玲珑之体,是以居于中焦以升降 
气化,若有瘀积,气化不能升降,是以易致胀满。用鸡内金为脏器疗法,若再与白术等分并用,为消化瘀积之 
要药,更为健补脾胃之妙品,脾胃健壮,益能运化药力以消积也。且为鸡内金含有稀盐酸,不但能消脾胃之积, 
无论脏腑何处有积,鸡内金皆能消之,是以男子 癖、女之 瘕,久久服之皆能治愈。又凡虚劳之证,其经络 
多瘀滞,加鸡内金于滋补药中,以化其经络之瘀滞而病始可愈。至以治室女月信一次未见者,尤为要药,盖以 
其能助归、芍以通经,又能助健补脾胃之药,多进饮食以生血也。 
女子干血劳之证,最为难治之证也,是以愈者恒少。惟善用鸡内金者,则治之多能奏效。愚向为妇女治病, 
其廉于饮食者,恒白术与鸡内金并用。乃有两次遇有此药者,一月间月信来三次,恍悟此过用鸡内金之弊也。 
盖鸡内金善化瘀血,即能催月信速于下行也。然月信通者服之,或至过通,而月信之不通者服 
之,即不难下通。况《内经》谓“中焦受气取汁,变化而赤,是为血。”血之来源,原在脾胃能多消饮食。鸡 
内金与白术并用,原能健脾胃以消饮食也。况脾为后天资生之本,居中央以灌溉四旁。此证之多发劳嗽者,脾 
虚肺亦虚也,多兼灼热者,脾虚而肾亦虚也。再加山药、地黄、枸杞诸药以补肺滋肾,有鸡内金以运化之,自 
能变其浓浓之汁浆为精液,以灌注于肺肾也。迨至服药日久,脏腑诸病皆愈,身体已渐撤消,而月信仍不至者, 
不妨再加 虫、水蛭诸药。如嫌诸药之猛悍,若桃仁、红花亦可以替代。然又须多用补正之药品以驾驭之,始 
能有益而无害也。愚向曾本此意拟一方,名资生通脉汤,后列用其方治愈之案数则,可参观也。 
《内经》谓“女子二七天癸至”,所谓二七者,十四岁也。然必足年足月十四岁,是则室女月信之通, 
当在年十五矣。若是年至十五月信不通,即当预为之防。宜用整条生怀山药,轧细过罗,每用一两或八钱,煮 
作茶汤,调以蔗糖令适口,以之送服生鸡内金细末五分许,当点心用之。日两次,久则月信自然通下。此因山 
药善养血,鸡内金善通血也。若至因月信不通,饮食减少,渐觉灼热者,亦可治以此方,鸡内金末宜多用至一 
钱,服茶汤后再嚼服天冬二三钱。 
至于病又加重,身体虚弱劳嗽,宜用拙拟资生通脉汤。此方之后,载有数案,且用此方各有加减,若服 
资生通脉汤,病虽见愈月信仍不至者,可参观所附案中加减诸方。 
上所论诸方之外,愚有新拟之方,凡服资生通脉汤病见愈而月信不见者,可用生怀山药四两,煮浓汁, 
送眼生鸡内金细末三钱。所余山药之渣,仍可水煮数次,当茶饮之。久之月信必至,盖鸡内金生用,为通月信 
最要之药,而多用又恐稍损气分,故又多用山药至四两,以培气分也。 
【附案】沈阳龚××,年三十岁,胃脘有硬物杜塞,已数年矣。饮食减少,不能下行,来院求为延医,其 
脉象沉而微弦,右部尤甚,为疏方,用鸡内金一两,生酒曲五钱,服数剂硬物全消。 
奉天史××,年近四旬,为腹有积聚,久治不愈,来院求为延医。其积在左胁下大径三寸,按之甚硬,时 
或作疼,呃逆气短,饮食减少,脉象沉弦。此乃肝积肥气之类。俾用生鸡内金三两,柴胡一两,共为末,每服一 
钱半,日服三次,旬余全愈。 
奉天秦××,年三十余,胃中满闷,不能饮食,自觉贲门有物窒碍,屡经医治,分毫无效。脉象沉牢,为 
疏方∶鸡内金六钱,白术、赭石各五钱,乳香、没药、丹参各四钱,生桃仁二钱,连服八剂全愈。 
奉天宋氏女,年十九岁,自十七岁时,胃有瘀滞作疼,调治无效,浸至不能饮食。脉象沉而无力,右部尤 
甚,为疏方∶鸡内金一两,生酒曲、党参各五钱,三棱、莪术、知母各三钱,樗鸡(俗名红娘子)十五个,服 
至八剂,大小二便皆下血,胃中豁然,其疼遂愈。 
盐山李氏妇,年三旬,胃脘旧有停积数年不愈,渐大如拳甚硬,不能饮食。左脉弦细,右脉沉濡,为疏方∶ 
鸡内金八钱,生箭耆六钱,三棱、莪术、乳香、没药各三钱,当归、知母各四钱,连服二十余剂积全消。 
友人毛××治一孺子,自两三岁时腹即胀大,至五六岁益加剧,面目黄瘦,饮食减少,俗所谓大肚痞也。毛 
××见拙拟期颐饼方后载,若减去芡实,可治小儿疳积痞胀、大人 瘕积聚,遂用其方(方系生鸡内金细末三两, 
白面半斤,白沙糖不拘多少,和作极薄小饼,烙至焦熟,俾作点心服之),月余全愈。 

<目录>二、药物
<篇名>72.穿山甲解
属性:穿山甲∶味淡,性平。气腥而窜,其走窜之性无微不至,故 
能宣通脏腑、贯彻经络、透达关窍,凡血凝、血聚为病皆能开之。以治疔痈,放胆用之,立见功效。并能治 
瘕积聚、周身麻痹、二便闭塞、心腹疼痛。若但知其长于治疮,而忘其他长,犹浅之乎视山甲也。 
疔痈初起未成脓者,愚恒用山甲、皂刺各四钱,花粉、知母各六钱,乳香、没药各三钱,全蜈蚣三条,服 
之立消。以治横 (鱼口便毒之类)亦极效验。其已有脓而红肿者,服之红肿即消,脓亦易出。至 瘕积聚、 
疼痛麻痹、二便闭塞诸证,用药治不效者,皆可加山甲作向导。友人黄××谓,身上若有血箭证,或金伤出血 
不止者,敷以山甲末立止,屡次用之皆效,蛤粉炒透用,惟以之熬膏药用生者。 

<目录>二、药物
<篇名>73.蜈蚣解
属性:蜈蚣∶味微辛,性微温。走窜之力最速,内而脏腑,外而经络,凡气血凝聚之处皆能开之。性有微毒,而 
转善解毒,凡一切疮疡诸毒皆能消之。其性尤善搜风,内治肝风萌动、癫痫眩晕、抽掣螈 、小儿脐风;外治 
经络中风、口眼歪斜、手足麻木。为其性能制蛇,故又治蛇症及蛇咬中毒。外敷治疮甲(俗名鸡眼为末敷之以 
生南星末醋调、敷四周),用时宜带头足,去之则力减,且其性原无大毒,故不妨全用也。 
【附案】奉天陈××之幼子,年五岁,周身壮热,四肢拘挛,有抽掣之状,渴嗜饮水,大便干燥,知系 
外感之热,引动其肝经风火上冲脑部,致脑气筋妄行,失其主宰之常也。投以白虎汤,方中生石膏用一两,又 
加薄荷叶一钱,钩藤钩二钱,全蜈蚣二条,煎汤一盅,分两次温饮下,一剂而抽掣止、拘挛舒,遂去蜈蚣,又 
服一剂热亦退净。 
奉天那姓幼子,生月余,周身壮热抽掣,两日之间不食乳、 
不啼哭,奄奄一息,待时而已。来院求治。知与前证仿佛,为其系婴孩,拟用前方将白虎汤减半,为其抽掣甚 
剧,薄荷叶、钩藤钩、蜈蚣其数仍旧,又加全蝎三个,煎药一盅,不分次数徐徐温灌之,历十二小时,药灌已 
而抽掣愈,食乳知啼哭矣。翌日,又为疏散风清热镇肝之药,一剂全愈。隔两日其同族又有三岁幼童,其病状 
与陈姓子相似,即治以陈姓子所服药,一剂而愈。 
奉天吴姓男孩,生逾百日,周身壮热,时作抽掣,然不甚剧,投以白虎汤,生石膏用六钱,又加薄荷叶一 
钱,蜈蚣一条,煎汤分三次灌下,尽剂而愈。此四证皆在暮春上旬,相隔数日之间,亦一时外感之气化有以使 
之然也。 
一人年三十余,陡然口眼歪斜,受病之边目不能瞬,用全蜈蚣二条为末,以防风五钱煎汤送服,三剂全愈。 
有病噎膈者,服药无效,偶思饮酒,饮尽一壶而病愈。后视壶中有大蜈蚣一条,恍悟其病愈之由,不在酒 
实在酒中有蜈蚣也。盖噎膈之证,多因血瘀上脘,为有形之阻隔(西人名胃癌,谓其处凸起如山石之有岩也), 
蜈蚣善于开瘀,是以能愈。观于此,则治噎膈者,蜈蚣当为急需之品矣。为其事甚奇,故附记于此。 

<目录>二、药物
<篇名>74.水蛭解
属性:水蛭∶味咸,色黑,气腐,性平。为其味成,故善入血分;为其原为噬血之物,故善破血;为其气腐,其 
气味与瘀血相感召,不与新血相感召,故但破瘀血而不伤新血。且其色黑下趋,又善破冲任中之瘀,盖其破瘀 
血者乃此物之良能,非其性之猛烈也。《神农本草经》谓主妇人无子,因无子者多系冲任瘀血,瘀血去自能有 
子也。特是,其味咸为水味,色黑为水色,气腐为水气,纯系水之精华生成,故最宜生用,甚忌火炙。 
凡破血之药,多伤气分,惟水蛭味咸专入血分,于气分丝毫 
无损。且服后腹不觉疼,并不觉开破,而瘀血默消于无形,真良药也。愚治妇女月闭 瘕之证,其脉不虚弱者, 
恒但用水蛭轧细,开水送服一钱,日两次。虽数年瘀血坚结,一月可以尽消。 
水蛭、虻虫皆为破瘀血之品。然愚尝单用以实验之,虻虫无效,而水蛭有效。以常理论之,凡食血之物, 
皆能破血。然虻虫之食血以嘴,水蛭之食血以身。其身与他物紧贴,即能吮他物之血,故其破瘀血之功独优。 
近世方书,多谓水蛭必须炙透方可用,不然则在人腹中,能生殖若干水蛭害人,诚属无稽之谈。曾治一妇 
人,经血调和,竟不产育。细询之,少腹有 瘕一块。遂单用水蛭一两,香油炙透,为末。每服五分,日两次, 
服完无效。后改用生者,如前服法。一两犹未服完, 瘕尽消,逾年即生男矣。惟气血亏损者,宜用补助气血 
之药佐之。 
或问,同一水蛭也,炙用与生用,其功效何如此悬殊?答曰∶此物生于水中,而色黑(水色)味咸(水味) 
气腐(水气),原得水之精气而生。炙之,则伤水之精气,故用之无效。水族之性,如龙骨、牡蛎、龟板大抵 
皆然。故王洪绪《外科证治全生集》谓用龙骨者,宜悬于井中,经宿而后用之,其忌火可知,而在水蛭为尤 
甚。特是水蛭不炙,为末甚难,若轧之不细,晒干再轧或纸包置炉台上令干亦可。此须亲自检点,若委之药坊, 
至轧不细时,必须火焙矣。西人治火热肿疼,用活水蛭数条,置患处,复以玻璃杯,使吮人毒血,亦良法也。 

<目录>二、药物
<篇名>75.蝎子解
属性:蝎子∶色青,味咸(本无咸味,因皆腌以盐水,故咸),性微温。善入肝经,搜风发汗,治痉痫抽掣,中 
风口眼歪斜,或周身麻痹,其性虽毒,转善解毒,消除一切疮疡,为蜈蚣之伍药,其力相得益彰也。 
此物所含之毒水即硫酸也,其入药种种之效力,亦多赖此。中其毒螫者,敷以西药重曹或碱,皆可解之, 
因此二者皆能制酸也。 
【附案】本村刘氏女,颔下起时毒甚肿硬,抚之微热,时愚甫弱冠,医学原未深造,投药两剂无甚效验。 
后或授一方,用壁上全蝎七个,焙焦为末,分两次用黄酒送下,服此方三日,其疮消无芥蒂。盖墙上所得之蝎 
子,未经盐水浸腌,其力浑全,故奏效尤捷也。 
邻庄张马村一壮年,中风半身麻木,无论服何药发汗,其半身分毫无汗。后得一方,用药局中蝎子二两, 
盐炒轧细,调红糖水中顿服之,其半身即出汗,麻木遂愈。然未免药力太过,非壮实之人不可轻用。 

<目录>二、药物
<篇名>76.蝉蜕解
属性:蝉蜕∶无气味,性微凉。能发汗,善解外感风热,为温病初得之要药。又善托隐疹外出,有皮以达皮之力, 
故又为治隐疹要药。与蛇蜕并用,善治周身癞癣瘙痒。若为末单服,又善治疮中生蛆,连服数次其蛆自化。为 
其不饮食而时有小便,故又善利小便;为其为蝉之蜕,故又能脱目翳也。 
蝉亦止小儿夜啼,又善医音哑。忆一九三六年秋,余友姚××,偶为外感所袭,音哑月余,余为拟方, 
用净蝉蜕(去足土)二钱,滑石一两,麦冬四钱,胖大海五个,桑叶、薄荷叶各二钱,嘱其用水壶泡之代茶饮, 
一日音响,二日音清,三日全愈。以后又用此方治愈多人,屡试屡验。 
受业孙××谨识 

<目录>二、药物
<篇名>77.羚羊角解
属性:羚羊角∶性近于平不过微凉。最能清大热,兼能解热中之大 
毒。且既善清里,又善透表,能引脏腑间之热毒达于肌肤而外出,疹之未出,或已出而速回者,皆可以此表之, 
为托表麻疹之妙药。即表之不出而毒瓦斯内陷者,服之亦可内消。又善入肝经以治肝火炽盛,至生眼疾,及患吐 
衄者之妙药。所最异者性善退热却不甚凉,虽过用之不致令人寒胃作泄泻,与他凉药不同。此乃具有特殊之良 
能,非可以寻常药饵之凉热相权衡也。或单用之,或杂他药中用,均有显效。今特将所用羚羊角治愈之病十余 
则,详录于下以征明之。 
【附案】壬寅之岁,季春夜半,表弟刘××之子,年六岁,于数日间出疹,因其苦于服药,强与之即作呕吐, 
所以未求诊视。今夜忽大喘不止,有危在顷刻之势,不知还可救否。遂与同往视之,见其不但喘息迫促,且精 
神恍惚,肢体骚扰不安;脉象摇摇而动,按之无根;其疹出第三日即靥,微有紫痕,知其毒火内攻,肝风已动 
也。因思熄风、清火,且托毒外出,惟羚羊角一味能兼擅其长,且色味俱无,煎汤直如清水,孺子亦不苦服。 
遂急取羚羊角三钱煎汤,视其服下,过十余分钟即安然矣。 
奉天王××之孙女,年五六岁,患眼疾。先经东医治数日不愈,延为诊视。其两目 肉长满,遮掩目睛,分 
毫不露,且疼痛异常,号泣不止。遂单用羚羊角二钱,俾急煎汤服之。时已届晚九点钟,至夜半已安然睡去, 
翌晨 肉已退其半。又煎渣服之,全愈。盖肝开窍于目,羚羊角性原属木,与肝有同气相求之妙,故善入肝经 
以泻其邪热,且善伏肝胆中寄生之相火,为眼疾有热者无上妙药。 
奉天韩姓媪,年六十余,臂上生疔毒,外科不善治疗,致令毒火内攻,热痰上壅,填塞胸臆,昏不知人。 
有东医数人为治移时不愈,气息益微。延为诊视,知系痰厥。急用蓬砂五钱,煮至融化,灌下三分之二,须臾呕 
出痰涎若干,豁然顿醒。而患处仍肿 
疼,其疔生于左臂,且左脉较右脉洪紧,知系肝火炽盛,发为肿毒也。遂投以清火解毒之剂,又单将羚羊角二 
钱煎汤兑服,一剂而愈。 
奉天王××之幼女,年五岁,因出疹倒靥过急,毒火内郁,已过旬日,犹大热不止,其形体病久似弱,而 
脉象确有实热,且其大便干燥,小便黄赤,知非轻剂所能治愈。将为疏方,为开羚羊角二钱,生石膏二两,煎 
汤一大盅,俾徐徐饮下。连服两剂,全愈。 
奉天刘××之幼女,年四岁,于孟夏时胸腹之间出白痧若干,旋即不见,周身壮热,精神昏愦,且又泄泻, 
此至危之候也。为疏方∶生怀山药、滑石各八钱,连翘、生杭芍各三钱,蝉蜕、甘草各二钱,羚羊角一钱 
(另煎兑服),煎汤一大盅,和羚羊角所煎之汤,共盅半,分三次温服下,其白痧复出,精神顿爽,泻亦遂止。 
继又用解毒清火之品调之,全愈。 
奉天马××之幼子,年四岁,因出疹靥急,来院求为延医。其状闭目喘促,精神昏昏,呼之不应,周身壮 
热,大便数日未行。断为疹毒内攻,其神明所以若斯昏沉,非羚羊角、生石膏并用不可。遂为疏方∶生石膏一 
两,玄参、花粉各六钱,连翘、金银花各三钱,甘草二钱,煎汤一大盅,又用羚羊角二钱煎汤半盅,混合,三 
次温服下,尽剂而愈。 
奉天陈姓女,年六七岁,疹后旬余灼热不退,屡服西药不效。后愚视之,脉象数而有力,知其疹毒之余热 
未清也。俾单用羚羊角一钱煎汤饮之,其热顿愈。 
天津俞××之幼子,年四岁,出疹三日,似靥非靥,周身壮热,渴嗜饮水,其精神似有恍惚不稳之意,其脉 
象有力,摇摇而动。恐其因热发痉,为开清热托毒之方,加羚羊角一钱以防其发痉,购药至,未及煎而痉发, 
且甚剧,遂将羚羊角与诸药同时各 
煎,取汤混合,连连灌下,其痉即愈。又将其方去羚羊角,再煎服一剂,全愈。 
沧州张××,来院询方,言其家有周岁小儿出疹,延医调治数日,其疹倒靥皆黑斑,有危在旦夕之势, 
不知尚可救否。细询之,知毒热内陷。为开羚羊角一钱及玄参、花粉、连翘各数钱,俾将羚羊角另煎汤半茶盅, 
与余三味所煎之汤兑服,一剂而愈。 
沧州赵××幼子,年五岁,因感受温病发痉,昏昏似睡,呼之不应,举家惧甚,恐不能救。其脉甚有力, 
肌肤发热。因晓之曰∶“此证因温病之气循督脉上行,伤其脑部,是以发痉,昏昏若睡,即西人所谓脑脊髓炎 
也。病状虽危,易治也。”遂单用羚羊角二钱,煎汤一盅,连次灌下,发痉遂愈,而精神亦明了矣。继用生石 
膏、玄参各一两,薄荷叶、连翘各一钱,煎汤一大盅,分数次温饮下,一剂而脉静身凉矣。盖痉之发由于督脉, 
因督脉上统脑髓神经也(督脉实为脑髓神经之根本)。羚羊之角乃其督脉所生,是以善清督脉与神经之热也。 
沧州刘××之幼子,甫周岁,发生扁桃体炎喉证,不能食乳,剧时有碍呼吸,目睛上泛。急用羚羊角一钱, 
煎汤多半杯,灌下,须臾呼吸通顺,食乳如常。 
沧州李氏妇,年二十余,因在西医院割瘰 ,住其院中,得伤寒证甚剧,西医不能治。延往诊视,其喘 
息迫促,脉数近七至,确有外感实热,而重诊无力,因其割瘰 已至三次,屡次闻麻药,大伤气分故也,其心 
中觉热甚难支,其胁下疼甚。急用羚羊角二钱,煎一大盅,调入生鸡子黄三枚,服下,心热与胁疼顿止。继投 
以大剂白虎加人参汤,每剂煎汤一大碗,仍调入生鸡子黄三枚,分数次温服下,连服二剂全愈。 
内子王氏生平有病不能服药,闻药气即思呕吐。偶患大便下血甚剧,时愚自奉还籍,彼自留奉,因粗 
识药性,且知羚羊角毫 
无药味,自用羚羊角一钱煎汤服之,立愈。 
友人毛××,善治吐衄闻名。其治吐衄之方,多用羚羊角。曾询其立方之义。××谓∶吐衄之证多因冲气上 
冲,胃气上逆,血即随之妄行。其所以冲胃冲逆者,又多为肝火、肝气之激发,用羚羊角以平肝火、肝气,其 
冲气不上冲,胃气不上逆,血自不妄行而归经矣。愚深韪斯论,遇吐衄证仿用之,果效验异常。 
所可虑者,羚羊角虽为挽回险证之良药,然其价昂贵,愚因临证细心品验,遇当用羚羊角之证,原可以他 
药三种并用代之,其药力不亚羚羊角,且有时胜于羚羊角,则鲜茅根、生石膏与西药阿斯匹林并用是也。今爰 
将此三药并用之分量酌定于下,且为定一方名,以便于记忆。 
【甘露清毒饮】 
鲜茅根(六两,去净皮切碎) 生石膏(两半,捣细) 阿斯匹林(半瓦) 
将前二味煎汤一大碗,分三次送服阿斯匹林,两点钟服一次。若初次服药后遍身出汗,后两次阿斯匹林宜 
少服,若分毫无汗,又宜稍多服。以服后微似有汗者方佳。至石膏之分量,亦宜因证加减,若大便不实者宜少 
用,若泻者石膏可不用,待其泻止便实仍有余热者,石膏仍可再用。 
壬申正月中旬,子××两臂及胸间肉皮微发红,咽喉微疼,疑将出疹,又强被友人挽去,为治小儿发疹。 
将病治愈,归家途中又受感冒,遂觉周身发冷,心中发热。愚适自津还籍,俾用生石膏细末一两,煎汤送服阿 
斯匹林一瓦,周身得汗,发冷遂愈,心中之热亦轻,皮肤则较前益红。迟半日又微觉发冷,心中之热更增剧, 
遂又用生石膏细末二两,煎汤送服阿斯匹林半瓦。服后微解肌,病又见愈。迟半日仍反复如故,且一日之间下 
大便两次,知其方不可再用。时地冻未解,遣人用开冻利器,剖取鲜茅 
根六两,煎汤一大碗,分三次服,每次送服阿斯匹林三分之一瓦。服后未见汗而周身出疹若干,病愈十分之八 
九,喉已不疼。隔两日觉所余之热又渐增重,且觉头目昏沉,又剖取鲜茅根八两,此时因其热增,大便已实, 
又加生石膏两半,共煎汤一大碗,仍分三次送服阿斯匹林如前。上半身又发出白泡若干,病遂全愈。观此可知 
此三药并用之妙,诚可代羚羊角矣。后返津时,值瘟疹流行,治以此方,皆随手奏效。 
附录∶ 
唐山赵××来函∶ 
小女一年有余,于季夏忽大便两三次带有粘滞,至夜发热,日闭目昏睡,翌晨手足筋惕肉 。后学断其 
肝风已动。因忆先生论羚羊角最善清肝胆之火,且历数其奇异之功效,真令人不可思议。为急购羚羊角尖一钱, 
上午九点煎服,至十一点周身得微汗,灼热即退。为其药甚珍贵,又将其渣煎服三次,筋惕亦愈。继服滋阴清 
燥汤一剂,泻痢均愈。 

<目录>二、药物
<篇名>78.血余炭解
属性:血余者,发也,不 则其质不化,故必 为炭然后入药。其性能化瘀血、生新血有似三七,故善治吐血、 
衄血。而常服之又可治劳瘵,因劳瘵之人,其血必虚而且瘀,故《金匮》谓之血痹虚劳。人之发,原人心血所 
生,服之能自还原化,有以人补人之妙,则血可不虚,而其化瘀之力,又善治血痹,是以久久服之,自能奏效。 
其性又能利小便(《金匮》利小便之方,有膏发煎),以人之小便半从血管渗出,血余能化瘀血生新血,使血 
管流通故有斯效。其化瘀生新之力,又善治大便下血腥臭,肠中腐烂,及女子月信闭塞,不以时至。 
制血余炭法∶用壮年剃下之发,碱水洗净,再用清水淘去碱味,晒干用铁锅炮至发质皆化为膏,晾冷, 
轧细,过罗,其发质未尽化者,可再炮之。 

<目录>二、药物
<篇名>79.指甲解
属性:指甲∶一名筋退,乃筋之余也。剪碎炮焦,研细用之。其味微咸,具有开破之性,疮疡将破未破者,敷之 
可速破。内服能催生下胎衣,鼻嗅之能止衄血,点眼上能消目翳。愚自制有磨翳药水,目翳浓者,可加指甲末 
与诸药同研以点目翳,屡次奏效。 

\part{医论}
<篇名>1.六经总论
属性:伤寒治法以六经分篇,然手足各有六经,实则十二经也。手足之经既有十二,而《伤寒论》但分为六经 
者何也?按《内经》之论十二经也,凡言某经而不明言其为手经、足经者皆系足经,至言手经则必明言其为 
手某经。盖人之足经长、手经短,足经大、手经小,足经原可以统手经,但言足经而手经亦恒寓其中 
矣。《伤寒论》之以六经分篇,此遵《内经》定例,寓手经于足经中也。彼解《伤寒论》者,谓其所言之 
六经皆系足经,是犹未明仲景着伤寒之深意也。 
经者,气血流通之处也。人之脏腑与某经相通,即为某经之府,其流通之气血原由府发出,而外感之内 
侵遂多以府为归宿。今将手足十二经及手足十二经之府详列于下。 
手足虽有十二经,其名则分为六经,因手足经之名原相同也。其经有阴有阳,其阳经分太阳、阳明、 
少阳,其阴经分太阴、少阴、厥阴。其阴阳之经原互相表里,太阳与少阴为表里,阳明与太阴为表里,少 
阳与厥阴为表里。凡互为表里者,因其阴阳之经并行,其阳行于表,阴行于里也。至于经之分属于府者, 
足太阳经之府在膀胱,足少阴经之府在肾,足阳明经之府在胃,足太阴经之府在脾,足少阳经之府在胆,足 
厥阴经之府在肝,此足之三阴、三阳经与府也。 
手之太阳经其府在小肠,手之少阴经其府在心,手之阳明经 
其府在大肠,手之太阴经其府在肺,手之少阳经其府在三焦,手之厥阴经其府在心胞,此手之三阴、三阳经与府也。 
阳经为阴经之表,而太阳经又为表中之表。其经之大都会在背,而实则为周身之外廓,周身之营血卫气 
皆赖其卫护保合,且具有充分之热力,为营卫御外感之内侵,是以《内经》名之为巨阳。推原其热力之由来, 
不外君、相二火,君火生于心之血脉与肺相循环,而散热于胸中大气(一名宗气)以外通于营卫,此如日丽中天 
有阳光下济之热也,是以其经名为太阳。相火生于肾中命门,肾原属水,中藏相火,其水火蒸热之气,由膀 
胱连三焦之脂膜以透达于身之外表,此犹地心水火之气(地中心有水火之气)应春令上透地面以 
生热也,为其热力发于水中,故太阳之经又名太阳寒水之经也。为太阳经之热力生于君、相二火,是以 
其经不但以膀胱为府,而亦以胸中为府,观《伤寒论》陷胸诸汤、丸及泻心诸汤,皆列于太阳篇中可知也。 
至于人病伤寒,其六经相传之次第,详于《内经》素问热论篇,谓“人之伤于寒也则为病热,一日巨阳受之, 
故头项痛、腰脊强;二日阳明受之,阳明主肌肉,其脉侠(同夹)鼻络于目,故身热目疼,而鼻干不得卧也; 
三日少阳受之,少阳主胆,其脉循胁络于耳,故胸胁痛而耳聋;三阳经络皆受其病而未入于脏者故可汗而已; 
四日太阴受之,太阴脉布胃中络于嗌(咽喉),故腹满而嗌干;五日少阴受之,少阴脉贯肾络于肺,系舌本, 
故口燥舌干而渴,六日厥阴受之,厥阴之脉循阴器而络于肝,故烦满而囊缩。”经络受病入于府者,故可下 
而已,此《内经》论六经相传之次第也。至《伤寒论》六经之次序,皆以《内经》为法,而未明言其日传一 
经,至愚生平临证之实验,见有伤寒至旬日,病犹在太阳之府者,至他经相传之日期,亦无一定,盖《内经》 
言其常,而病情之变化恒有出于常例之外者,至传至某经,即现某经之病状,此又不尽然,推原其所 
以然之故,且加以生平临证之实验, 
知传至某经即现某经之病状者,多系因其经先有内伤也。若无内伤则传至某经恒有不即现某经之病时,此在 
临证者细心体察耳。 
至于六经之命名,手足皆同,然有因手经发源之府而命名者,有因足经发源之府而命名者。如太阳 
经名为太阳寒水之经,此原因足太阳之府命名,而手太阳亦名太阳寒水之经者,是以足经而连带其手经也。他 
如阳明经名为阳明燥金之经,是因手阳明之府命名(手阳明府大肠属金,其互为表里之肺亦属金),而 
足阳明经亦名阳明燥金之经者,是以手经而连带其足经也。少阳经名为少阳相火之经,此因足少阳之府命名(胆 
中寄有相火),而手少阳经亦名为少阳相火之经者,是以足经而连带其手经也。太阴经名为太阴湿土之经,此 
因足太阴之府命名(脾为湿土),而手太阴经亦名太阴湿土之经者,是以足经而连带其手经也。少阴经名为少 
阴君火之经,此因手少阴之府命名(心为君火),而足少阴经亦名少阴君火之经者,是以手经而连带 
其足经也。厥阴经名为厥阴风木之经,此因足厥阴之府命名(肝属木而主风),而手厥阴经亦名厥阴风木之经 
者,是以足经而连带其手经也。此手足十二经可并为六经之义也。 

<目录>三、医论
<篇名>2.太阳病桂枝汤证
属性:病名伤寒,而太阳篇之开端,实中风、伤寒、风温并列,盖寒气多随风至,是中风者伤寒之诱起也。无论 
中风、伤寒,入阳明后皆化为温,是温病者伤寒之归宿也。惟其初得之时,中风、伤寒、温病,当分三种治 
法耳。为中风为伤寒之诱起,是以太阳篇开始之第一方为桂枝汤,其方原为治中风而设也。 
《伤寒论》原文∶太阳病,发热,汗出,恶风,脉缓者(缓脉与迟脉不同,脉搏以一息四至为准,脉迟者 
不足四至,若缓脉则至数不改似有懒动之意),名为中风。 
《伤寒论》原文∶太阳中风,阳浮而阴弱(脉法关前为阳,关后为阴,其浮脉见于关前,弱脉见于关后,浮者 
着手即得,弱者不任重按)。阳浮者热自发,阴弱者汗自出,啬啬恶寒。 
(单弱不胜寒之意),淅淅恶风(为风所伤恒畏风声之意),翕翕发热(其热蕴而不散之意),鼻鸣干 
呕者,桂枝汤主之。 
【桂枝汤方】桂枝三两去皮,芍药三两,炙甘草二两,生姜三两,大枣十二枚擘。上五味 咀,以水 
七升微火煮取三升,去滓,适寒温,服一升。服须臾,啜热稀粥一升余以助药力,温复令一时许,遍体 微 
似有汗者益佳,不可令如水流漓,病必不除。若一服汗出病瘥(愈也),停后服,不必尽剂;若不汗,更服 
根据前法;又不汗,后服小促其间,半日许,令三服尽;若病重者,一日一夜服,周时观之;服一剂尽,病 
证犹在者,更作服;若汗不出者乃服至二三剂。禁生冷、粘滑、肉面、五辛、酒酪、臭恶等物。 
人之营卫皆在太阳部位,卫主皮毛,皮毛之内有白膜一层名为腠理,腠理之内遍布微丝血管即营也。其 
人若卫气充盛,可为周身之外围,即受风不能深入(此受风,不可名为中风),其人恒多汗闭不出, 
迨其卫气流通,其风自去,原可不药而愈也。至桂枝汤所主之证,乃卫气虚弱,不能护卫其营分,外感之风 
直透卫而入营,其营为风邪所伤,又乏卫之保护,是以易于出汗。其发热者,因营分中之微丝血管原有自心传 
来之热,而有风以扰之,则更激发其热也。其恶风者,因卫虚无御风之力,而病之起点又由于风也。推原其 
卫气不能卫护之故,实由于胸中大气之虚损。《灵枢》五味篇曰∶“谷始入于胃,其精微者,先出于胃之 
两焦,以溉五脏,别出两行营卫之道,其大气之抟而不行者,积于胸中,命曰气海。”由斯观之,营卫原与 
胸中大气息息相通,而大气实为营卫内部之大都会,愚临证实验以来,见有大气虚者,其营卫即不能护卫于 
外而汗出淋漓,夫大气原赖水谷之气时时培养,观服桂枝汤者当啜热粥以助药力,此不惟助其速于出汗, 
实兼欲助胸中大气以固营卫之本源也。 
或问∶桂枝汤提纲中原谓阴弱者汗自出,未尝言阳弱者汗自 
出也。夫关后为阴主血,关前为阳主气,桂枝汤证,其弱脉惟见于关后,至关前之脉则见有浮象,未见其弱, 
而先生竟谓桂枝汤证之出汗,实由于胸中大气之弱,不显与提纲中之言相背乎?答曰∶凡受风之脉多见于关 
前,提纲中所谓阳浮者,其关前之脉因受风而浮也,所谓阴弱者,知其未病之先其脉原弱,至病后而仍 
不改其弱也,由斯而论,其未病之先,不但关后之脉弱,即关前之脉亦弱,既病之后,其关前脉之弱者转为 
浮脉所掩,而不见其弱耳。然其脉虽浮,必不任重按,是浮中仍有弱也,特古人立言尚简,未尝细细明言耳。 
是以愚用桂枝汤时,恒加黄 以补其胸中大气,加薄荷以助其速于出汗,不至若方后所云,恒服药多次 
始汗也。又宜加天花粉助芍药以退热(但用芍药退热之力恒不足),即以防黄 服后能助热也(黄 天花粉等 
分并用,其凉热之力相敌,若兼用之助芍药清热,分量又宜多用)。若遇干呕过甚者,又宜加半夏以治其 
呕,惟此时药局所鬻之半夏,多制以矾(虽清半夏亦有矾),若用以止呕,必须用微温之水淘净矾味,用之方效。 
愚治桂枝汤证,又有屡用屡效之便方,较用桂枝汤殊为省事,方用生怀山药细末两半或一两,凉水调和 
煮成稀粥一碗,加白糖令适口,以之送服西药阿斯匹林一瓦,得汗即愈。 
桂枝汤证之出汗,不过间有出汗之时,非时时皆出汗也,故必用药再发其汗,始能将外感之风邪逐出。然 
风邪去后,又虑其自汗之病不愈,故方中山药与阿斯匹林并用,一发汗、一止汗也,至于发汗与止汗之药并 
用而药力两不相妨者,此中原有深义,盖药性之入人脏腑,其流行之迟速原迥异,阿斯匹林之性其发汗最速,而 
山药止汗之力则奏效稍迟,是以二药虽一时并用,而其药力之行则一先一后,分毫不相妨碍也。 

<目录>三、医论
<篇名>3.太阳病麻黄汤证
属性:《伤寒论》原治伤寒之书,而首论中风者,因中风亦可名为伤寒也(《难经》曰∶“伤寒有五;有中风, 
有伤寒,有湿温,有热病,有温病”)。然究与真伤寒不同,盖中风病轻伤寒病重,为其重也,而治之者 
必须用大有力之药,始能胜任,所谓大有力者,即《伤寒论》中之麻黄汤是也。今试论 
麻黄汤证及麻黄汤制方之义,并详论用麻黄汤时通变化裁之法。 
《伤寒论》原文∶太阳病,或已发热,或未发热,必恶寒,体痛,呕逆,脉阴阳俱紧者,名曰伤寒。 
又原文∶太阳病头疼,发热,身疼,腰痛,骨节疼痛,恶风,无汗而喘者,麻黄汤主之。 
脉象阴阳俱紧,实为伤寒之确征。然紧脉之状最难形容,惟深明其病理,自不难想象而得。脉生于心, 
心一动而外输其血,周身之脉即一动,动则如波浪之有起伏,以理言之,凡脉之力大者,其起伏之势自应 
愈大。至紧脉其跳动若有力而转若无所起伏,究其所以然之故,实因太阳为外卫之阳,因为寒所袭,逼之 
内陷与脉相并,则脉得太阳蕴蓄之热,原当起伏有力以成响应之势,而寒气紧缩之力,又复逼压其脉道使不 
能起伏,是以指下诊之似甚有力而竟直穿而过,且因其不得起伏,蓄极而有左右弹之势,此紧脉真象也。 
至麻黄汤证,全体作疼痛者,以筋骨不禁寒气之紧缩也(铁条经严寒则缩短寒气紧缩之力可知)。其发 
热者,身中之元阳为寒气闭塞不能宣散而增热也。其无汗恶风者,汗为寒闭内蕴之热原欲借汗透出,是 
以恶风也。其作喘者,因手太阴肺经与卫共主皮毛,寒气由皮毛入肺,闭其肺中气管,是以不纳气而作喘。 
然深究其作喘之由,犹不但此也,人之胸中亦太阳之部位也,其中间所积大气,原与 
外表之卫气息息相通,然大气即宗气,《灵枢》谓∶“宗气积于胸 
中,出于喉咙,以贯心脉而行呼吸。”夫大气既能以贯心脉,是营血之中亦大气所流通也,伤寒之证,其 
营卫皆为外寒所束,则大气内郁必膨胀而上逆冲肺,此又喘之所由来也。 
【麻黄汤方】麻黄三两,桂枝三两去皮,甘草一两炙,杏仁七十个去皮尖。上四味以水九升,先 
煮麻黄减二升,去上沫,纳诸药煮取二升半,去渣温服八合(一升十合),复取微似汗,不须啜 
粥,余如桂枝法将息。 
麻黄发汗力甚猛烈,先煮之去其浮沫,因其沫中含有发表之猛力,去之所以缓麻黄发表之性也。麻黄不 
但善于发汗,且善利小便,外感之在太阳者,间有由经入府而留连不去者(凡太阳病多日不解者,皆是由 
经入府),以麻黄发其汗,则外感之在经者可解,以麻黄利其小便,则外感之由经入府者,亦可分消也。且 
麻黄又兼入手太阴能泻肺定喘,俾外感之由皮毛窜入肺者(肺主皮毛),亦清肃无遗。是以发太阳之汗者不但 
麻黄,而仲景定此方时独取麻黄也。桂枝味辛性温,亦具有发表之力,而其所发表者,惟在肌肉之间,故 
善托肌肉中之寒外出,且《神农本草经》谓其主上气咳逆吐吸(吸气甫入即吐出),是桂 
枝不但能佐麻黄发表,兼能佐麻黄入肺定喘也。杏仁味苦性温,《神农本草经》亦谓其主咳逆上气,是亦能 
佐麻黄定喘可知,而其苦降之性又善通小便,能佐麻黄以除太阳病之留连于府者,故又加之以为佐使也。至 
于甘草之甘缓,能缓麻黄发汗之猛烈,兼能解杏仁之小毒,即以填补(甘草属土能填补)出汗后之汗腺空 
虚也。药止四味,面面俱到,且又互相辅助,此诚非圣手莫办也。 
附∶用麻黄汤之变通法 
人之禀赋随天地之气化为转移,古今之气化或有不同,则今人与古人之禀赋,其强弱浓薄偏阴偏阳 
之际不无差池,是以古方用于今日,正不妨因时制宜而为之变通加减也。 
麻黄汤原用解其外寒,服后遍体汗出,恶寒既愈,有其病从此遂愈者,间有从此仍不愈,后浸发热 
而转为阳明证者,其故何也?愚初为人诊病时,亦未解其故。后乃知服麻黄汤汗出后,其营卫内陷之热若还 
表随汗消散,则其病即愈。若其热不复还表而内陷益深,其热必将日增,此即太阳转阳明之病也。悟得此理 
后,再用麻黄汤时,必加知母数钱以解其内陷之热,主治伤寒无汗,服后未有不愈者矣(医方篇中有麻黄 
加知母汤可参观)。大青龙汤治伤寒无汗烦躁,是胸中先有内热,无所发泄,遂郁而作烦躁,故于解表药中 
加石膏以清内热。然麻黄与石膏并用,间有不汗之时。若用麻黄加知母汤,将知母重加数钱,其寒润之性入肺 
中化合而为汗,随麻黄以达于外,而烦躁自除矣。 
上所论者,麻黄汤原宜加知母矣。而间有不宜加者,此又不 
得不斟酌也。间有其人阳分虚者,又当于麻黄汤中加补气之药以助之出汗。 
一人年近四旬,身体素羸弱,于季冬得伤寒证,医者投以麻 
黄汤汗无分毫,求为延医,其脉似紧而不任重按,遂于麻黄汤中加生黄 、天花粉各五钱,一剂得汗而愈。 
一人亦年近四旬,初得外感,经医甫治愈,即出门作事,又重受外感,内外俱觉寒凉,头疼气息微喘, 
周身微形寒战,诊其脉六部皆无,重按亦不见,愚不禁骇然,问其心中除觉寒凉外别无所苦,知犹可治, 
不至有意外之虑,遂于麻黄汤原方中为加生黄 一两,服药后六脉皆出,周身得微汗,病遂愈。 
一人,年过三旬,身形素羸弱,又喜吸鸦片。于冬令得伤寒证,因粗通医学,自服麻黄汤,分毫无汗。 
求为诊视,脉甚微细,无紧象。遂即所用原方,为加生黄 五钱。服后得汗而愈。此二 
证皆用麻黄汤是不宜加知母,宜加黄 者也。 
尝治一少年,于季冬得伤寒证,其人阴分素亏,脉近六至, 
且甚弦细,身冷恶寒,舌苔淡白。延医诊视,医者谓脉数而弱,伤寒虽在初得,恐不可用麻黄强发其汗。此时 
愚应其近邻之聘,因邀愚至其家,与所延之医相商。愚曰∶“麻黄发汗之力虽猛,然少用则无妨,再辅以补正 
之品,自能稳妥奏功矣。”遂为疏方∶麻黄钱半,桂枝尖一钱,杏仁、甘草各钱半,又为加生怀山药、 
北沙参各六钱。嘱其煎汤服后,若至两点钟不出汗,宜服西药阿斯匹林二分许以助其出汗。后果如此服之, 
周身得汗而愈矣。 
曾治邻村李姓少年,得伤寒证已过旬日,表证未罢,时或恶寒,头犹微疼,舌苔犹白,心中微觉发热, 
小便色黄,脉象浮弦,重按似有力,此热入太阳之腑(膀胱)也。投以麻黄汤,为加知母八钱,滑石六钱, 
服后一汗而愈。 
此证虽在太阳之表与腑,实已连阳明矣。故方中重用知母以清阳明之热,而仍用麻黄解其表,俾其余 
热之未尽清者仍可由汗而消散,此所以一汗而愈也。至于《伤寒论》中载有其病重还太阳者,仍宜以麻黄汤 
治之,而愚遇此证,若用麻黄汤时亦必重加知母也。 
麻黄汤证有兼咽喉疼者,宜将方中桂枝减半,加天花粉六钱,射干三钱,若其咽喉疼而且肿者,麻黄 
亦宜减半,去桂枝再加生蒲黄三钱以消其肿,然如此加减,凉药重而表药轻,若服后过点半钟不出汗时,亦 
服西药阿斯匹林瓦许以助其汗;若服后汗仍不出时,宜阿斯匹林接续再服,以汗出为目标;若能遍体皆微 
见汗,则咽喉之疼肿皆愈矣。 
麻黄汤证,若遇其人素有肺劳病者,宜于原方中加生怀山药、天门冬各八钱。 
麻黄汤证,若遇其人素有吐血病者,虽时已愈,仍宜去桂枝以防风二钱代之(吐血之证最忌桂枝),再 
加生杭芍三钱(按古之一两约折为今之三钱,且将一次所煎之汤分作三剂,则一剂之中当有麻黄三钱),然又 
宜因时、因地、因人细为斟酌,不必定以三钱为准也。如温和之时,汗易出少用麻黄即能出汗。严寒之 
时,汗难出必多用麻黄始能出汗,此因时也。又如大江以南之人,其地气候温暖,人之生于其地者,其肌肤浅 
薄,麻黄至一钱即可出汗,故南方所出医书有用麻黄不过一钱之语。至黄河南北,用麻黄约可以三钱为率。 
至东三省人,因生长于严寒之地,其肌肤颇强浓,须于三钱之外再将麻黄加重始能得汗,此因地也。至于地 
无论南北,时无论寒燠,凡其人之劳碌于风尘,与长居屋中者,其肌肤之浓薄强弱原自不同,即其汗之易 
出不易出,或宜多用麻黄,或宜少用麻黄,原不一致,此因人也。用古人之方者,岂可胶柱鼓瑟哉! 

<目录>三、医论
<篇名>4.太阳与阳明合病麻黄汤证
属性:《伤寒论》原文∶太阳与阳明合病,喘而胸满者不可下,宜麻黄汤主之。 
太阳与阳明合病,是太阳表证未罢,而又兼阳明之热也。其喘者风寒由皮毛袭肺也。其胸满者胸中大 
气因营卫闭塞,不能宣通而生 胀也。其言不可下者,因阳明仍连太阳,下之则成结胸,且其胸本发满,成 
结胸尤易,矧其阳明之热,仅在于经,亦断无可下之理,故谆谆以不可下示戒也。仍治以麻黄汤,是开其 
太阳而使阳明初生之热随汗而解也。 
证兼阳明,而仍用麻黄汤主治,在古人禀赋敦浓,淡泊寡欲,服之可以有效。今人则禀赋薄弱,嗜 
好日多,强半阴亏,若遇此等证时,宜以薄荷代方中桂枝。若其热稍剧,而大便实者,又宜 
酌加生石膏(宜生用不可 用理详白虎汤下)数钱,方能有效。 

<目录>三、医论
<篇名>5.太阳温病麻杏甘石汤证
属性:至于温病,在上古时,原与中风、伤寒统名之为伤寒,是以 
秦越人《难经》有伤寒有五之说。至仲景着《伤寒论》,知温病初 
得之治法,原与中风、伤寒皆不同,故于太阳篇首即明分为三项, 
而于温病复详细论之,此仲景之医学,较上古有进步之处也。 
《伤寒论》原文∶太阳病,发热而渴,不恶寒者,为温病。若发汗已,身灼热者,名曰风温。风温 
为病,脉阴阳俱浮,自汗出,身重,多眠睡,息必鼾,语言难出。 
论温病之开端,亦冠以太阳病三字者,因温病亦必自太阳(此是足太阳非手太阳,彼谓温病入手经 
不入足经者,果何所据也)入也。然其化热最速,不过数小时即侵入阳明,是以不觉恶寒转发热而渴也。治 
之者不知其为温病,而误以热药发之,竟至汗出不解而转增其灼热,则即此不受热药之发表,可确定其名为风温 
矣。其脉阴阳俱浮者象风之飘扬也,自汗出者热随浮脉外透也,身重者身体经热酸软也,多眠睡者精 
神经热昏沉也,语言难出者,上焦有热而舌肿胀也。 
风温之外,又有湿温病与伏气化热温病,而提纲中止论风温 
者,因湿温及伏气化热之温,其病之起点亦恒为风所激发,故皆可以风温统之也。 
提纲中论风温之病状详矣,而提纲之后未列治法,后世以为憾事。及反复详细推之,乃知《伤寒论》 
中原有治温病之方,特因全书散佚,后经叔和编辑而错简在后耳。尝观《伤寒论》第六十二节云∶“发汗后, 
不可更行桂枝汤,汗出而喘,无大热者,可与麻黄杏仁甘草生石膏汤。”今取此节与温病提纲对观,则此 
节之所谓发汗后,即提纲之所谓若发汗也,此节之所谓喘,即提纲之所谓息必鼾也,由口息而喘者,由鼻息即 
鼾矣,此节之所谓无大热,即提纲之所谓身灼热也,盖其灼热犹在外表,心中仍无大热也, 
将此节之文与温病提纲一一比较,皆若合符节。 
夫中风、伤寒、温病特立三大提纲,已并列于篇首,至其后则于治中风治伤寒之方首仍加提纲,以 
彼例此,确知此节之文原为温病之方,另加提纲无疑,即麻杏甘石汤为治温病之方无疑 
也。盖当仲景时,人之治温病者,犹混温病于中风、伤寒之中,于病初得时,未细审其发热不恶寒,而以 
温热之药发之,是以汗后不解。或见其发热不恶寒,误认为病已传里,而竟以药下之,是以百六十三节, 
又有下后不可更行桂枝汤云云。所稍异者,一在汗后,一在下后,仲景恐人见其汗出再误认为桂枝证,故 
切戒其不可更行桂枝汤,而宜治以麻杏甘石汤。盖伤寒定例,凡各经病证误服他药后,其原病犹在者,仍 
可投以正治之原方,是以百零三节云,凡柴胡汤病证而下之,若柴胡证不罢者复与小柴胡汤。以此例彼,知 
麻杏甘石汤为救温病误治之方,实即治温病初得之主方,而欲用此方于今日,须将古方之分量稍有变通。 
【麻黄杏仁甘草石膏汤原方】 麻黄四两去节,杏仁五十个去皮尖,甘草二两,石膏八两碎绵裹。 
上四味以水七升,先煮麻黄减二升去上沫,纳诸药煮取二升,去渣温服一升。 
方中之义,用麻黄协杏仁以定喘,伍以石膏以退热,热退其汗自止也。复加甘草者,取其甘缓之性,能 
调和麻黄、石膏,使其凉热之力溶和无间以相助成功,是以奏效甚捷也。 
此方原治温病之汗出无大热者,若其证非汗出且热稍重者,用此方时,原宜因证为之变通,是以愚 
用此方时,石膏之分量恒为麻黄之十倍,或麻黄一钱、石膏一两,或麻黄钱半、石膏两半。遇有不出汗者, 
恐麻黄少用不致汗,服药后可服西药阿斯匹林瓦许以助其汗。若遇热重者,石膏又可多用。曾治白喉证及烂 
喉痧证(烂喉痧证必兼温病、白喉证,亦多微兼外感),麻黄用一钱,石膏恒重至二两,喉 
证最忌麻黄,而能多用石膏以辅弼之,则不惟不忌,转能借麻黄之力立见奇功也。 
至于肺病之起点,恒有因感受风温,其风邪稽留肺中化热铄 
肺,有时肺中作痒,即连连喘嗽者,亦宜投以此汤,清其久蕴之 
风邪,连服数剂其肺中不作痒,嗽喘自能减轻,再徐治以润肺清火利痰之剂,而肺病可除矣。盖此麻杏甘 
石汤之用处甚广,凡新受外感作喘嗽,及头疼、齿疼、两腮肿疼,其病因由于外感风热 
者皆可用之,惟方中药品之分量,宜因证变通耳。 
【附记】林××,年近五旬,因受风温,虽经医治愈,而肺中余热未清,致肺阴铄耗,酿成肺病,屡经 
医治无效。其脉一息五至,浮沉皆有力,自言喉连肺际,若觉痒则咳嗽顿发,剧时连嗽数十声,周身汗出, 
必吐出若干稠痰其嗽始止。问其心中常觉发热,大便燥甚,四五日一行。因悟其肺际作痒,即顿发咳嗽 
者,必其从前病时风邪由皮毛袭入肺中者,至今犹未尽除也。因其肺中风热相助为虐,宜以麻黄祛其风, 
石膏清其热,遂为开麻杏甘石汤方,麻黄用钱半,生石膏用两半,杏仁三钱,甘草二钱,煎服一剂,咳嗽顿 
愈。诊其脉仍有力,又为开善后之方,用生山药一两,北沙参、天花粉、天冬各五钱,川贝、射干、苏 
子、甘草各二钱,嘱其多服数剂,肺病可从此除根。后阅旬日,林××又求诊视,言先生去后,余服所开善 
后方,肺痒咳嗽仍然反复,遂仍服第一次方,至今已连服十剂,心中热已退,仍分毫不觉药凉,肺痒咳嗽皆 
愈,且饮食增加,大便亦不甚干燥。闻其所言,诚出愚意料之外也。再诊其脉已不数,仍似有力,遂将方 
中麻黄改用一钱,石膏改用一两,杏仁改用二钱,又加生怀山药六钱,俾煎汤接续服之,若服之稍觉凉时,即 
速停止,后连服七八剂似稍觉凉,遂停服,肺病从此竟愈。 
按:治肺劳投以麻黄杏仁甘草石膏汤,且用至二十余剂,竟将肺劳治愈,未免令阅者生疑,然此中固 
有精细之理由在也。盖肺病之所以难愈者,为治之者但治其目前所现之证,而不深究其病因也。如此证原 
以外感受风成肺劳,且其肺中作痒,犹有风邪存留肺中,且为日既久则为锢闭难出之风邪,非麻黄不能开发其 
锢闭之深,惟其性偏于热,于肺中蕴有实热者不宜,而重用生石膏以辅弼之,既可解麻黄之热,更可清肺中 
久蕴之热,以治肺热有风劳嗽者,原为正治之方,故服之立时见功。至于此药,必久服始能拔除病根,且 
久服麻黄、石膏而无流弊者,此中又有理由在,盖深入久锢之风邪,非屡次发之不能透,而伍以多量之石 
膏以为之反佐,俾麻黄之力惟旋转于肺脏之中,不至直达于表而为汗,此麻黄久服无弊之原因也。至石膏性 
虽寒凉,然其质重气轻,煎入汤剂毫无汁浆(无汁浆即是无质),其轻而且凉之气,尽随麻黄发表之 
力外出,不复留中而伤脾胃,此石膏久服无弊之原因也。所遇之证,非如此治法不愈,用药即不得不如此也。 

<目录>三、医论
<篇名>6.太阳病大青龙汤证
属性:(附∶脉微弱汗出恶风及筋惕肉 治法) 
有太阳中风之脉,兼见太阳伤寒之脉者,大青龙汤所主之证是也。 
《伤寒论》原文∶太阳中风,脉浮紧,发热,恶寒,身疼痛,不汗出而烦躁,大青龙汤主之。若 
脉微弱,汗出恶风者,不可服,服则厥逆,筋惕肉 ,此为逆也。 
【大青龙汤方】麻黄六两去节,桂枝二两去皮,甘草二两炙,杏仁五十个去皮尖,生姜三两切, 
大枣十二枚擘。石膏如鸡子大碎(如鸡子大当有今之三两)。 
上七味,以水九升先煮麻黄减二升,去上沫,纳诸药煮取三升,去滓温服一升,取微似汗,汗出多者温 
粉扑之。一服汗者停后服。汗多亡阳遂虚,恶风烦躁,不得眠也。 
此大青龙汤所主之证,原系胸中先有蕴热,又为风寒锢其外表,致其胸中之蕴热有蓄极外越之势。而 
其锢闭之风寒,而犹恐芍药苦降酸敛之性,似于发汗不宜,而代以石膏,且多用之以浓 
其力,其辛散凉润之性,既能助麻、桂达表,又善化胸中蕴蓄之热为汗,随麻、桂透表而出也,为有云腾致雨 
之象,是以名为大青龙也。至于脉微弱汗出恶风者,原系胸中大气虚损,不能固摄卫气,即使有热亦是虚阳 
外浮,若误投以大青龙汤,人必至虚者益虚,其人之元阳因气分虚极而欲脱,遂致肝风萌动而筋惕肉 也。夫 
大青龙汤既不可用,遇此证者自当另有治法,拟用生黄 、生杭芍各五钱,麻黄钱半,煎汤一次服下,此 
用麻黄以逐其外感,黄 以补其气虚,芍药以清其虚热也。为方中有黄 以补助气分,故麻黄仍可少用也。若其 
人已误服大青龙汤,而大汗亡阳,筋惕肉 者,宜去方中麻黄加净萸肉一两。 
《伤寒论》原文∶伤寒脉浮缓,身不疼,但重,乍有轻时,无少阴证者,大青龙汤发之。 
细思此节之文,知所言之证原系温病,而节首冠以伤寒二字者,因中风、温病在本书之定例,均可名为 
伤寒也。凡外感之脉多浮,以其多兼中风也。前节言伤寒脉浮紧,是所中者为凛冽之寒风,是中风兼伤寒也。 
后节言伤寒脉浮缓,知所中者非凛冽之寒风,当为柔和之温风,既中柔和之温风,则即成风温矣。是以病 
为伤寒必胸中烦躁而后可用石膏,至温病其胸中不烦躁,亦恒可用石膏,且其身不疼但重,伤寒第六节温病 
提纲中,原明言身重此明征也。况其证乍有轻时,若在伤寒必不复重用石膏,惟温病虽有轻时,亦可重用石 
膏。又伤寒初得有少阴证,若温病则始终无少阴证(少阴证有寒有热,此言无少阴证,指少阴之寒证而言,少 
阴寒证断不可用大青龙汤,至少阴热证,原为伏气化热窜入少阴,虽在初得亦可治以大青龙汤,此又不可不知), 
此尤不为伤寒而为温病之明征也。由此观之,是此节原为治温病者说法,欲其急清燥热以存真阴为先务也。至愚 
用此方治温病时,恒以薄荷代方中桂枝,尤为稳妥。 
凡发汗所用之药,其或凉或热,贵与病适宜。其初得病寒者 
宜用热药发其汗,初得病热者宜用凉药发其汗。如大青龙汤证, 
若投以麻黄汤则以热济热,恒不能出汗,即或出汗其病不惟不解,转益增烦躁,惟于麻、桂汤中去芍药, 
重加石膏多于麻、桂数倍,其凉润轻散之性,与胸中之烦躁化合自能作汗,矧有麻黄之 
善透表者以助之,故服后复杯之顷,即可周身得汗也。 
曾治一人冬日得伤寒证,胸中异常烦躁,医者不识为大青龙汤证,竟投以麻黄汤,服后分毫无汗,胸中 
烦躁益甚,自觉屋隘莫能容,诊其脉洪滑而浮,治以大青龙汤,为加天花粉八钱,服后五分钟,周身汗出 
如洗,病若失。 
或问∶服桂枝汤者,宜微似有汗,不可令如水流漓,病必不除,服麻黄汤者,复取微似汗,知亦不可令汗 
如水流漓也。今于大青龙汤中加花粉,服汤后竟汗出如洗而病若失者何也?答曰∶善哉问也,此中原有妙理, 
非此问莫能发之。凡伤寒、温病,皆忌伤其阴分,桂枝汤证与麻黄汤证,禁过发汗者恐伤其阴分也。 
至大青龙汤证,其胸中蕴有燥热,得重量之石膏则化合而为汗,其燥热愈深者,化合之汗愈多,非尽量透发 
于外,其燥热即不能彻底清肃,是以此等汗不出则已,出则如时雨沛然莫可遏抑。盖麻黄、桂枝等汤,皆 
用药以祛病,得微汗则药力即能胜病,是以无事过汗以伤阴分。至大青龙汤乃合麻、桂为一方,又去芍药之酸 
收,益以石膏之辛凉,其与胸中所蕴之燥热化合,犹如冶红之铁沃之以水,其热气自然蓬勃四达,此乃调燮 
其阴阳,听其自汗,此中精微之理,与服桂枝、麻黄两汤不可过汗者,迥不侔也。 
或问∶大青龙汤证,当病之初得何以胸中即蕴此大热?答曰∶此伤寒中伏气化热证也(温病中有伏气化 
热,伤寒中亦有伏气化热)。因从前所受外寒甚轻,不能遽病,惟伏藏于三焦脂膜之中,阻塞升降之气化,久 
而化热,后又因薄受外感之激动,其热陡发,窜入胸中空旷之府, 
不汗出而烦躁,夫胸中原为太阳之府,为其犹在太阳,是以其热虽甚而仍可汗解也。 

<目录>三、医论
<篇名>7.太阳病小青龙汤证
属性:《伤寒论》原文∶伤寒表不解,心下有水气,干呕,发热而 
咳,或渴,或利,或噎,或小便不利,少腹满,或喘者,小青龙汤主之。 
水散为气,气可复凝为水。心下不曰停水,而曰有水气,此 
乃饮水所化之留饮,形虽似水而有粘滞之性,又与外感互相胶漆,是以有以下种种诸病也。干呕者水气粘滞 
于胃口也,发热者水气变为寒饮,迫心肺之阳外越也,咳者水气浸入肺中也,渴者水气不能化津液上潮也,利 
者水气溜入大肠作泻也,噎者水气变为寒痰梗塞咽喉也,小便不利少腹满者,水气凝结膨胀于下焦 
也,喘者肺中分支细管皆为水气所弥漫也。 
【小青龙汤原方】麻黄三两去节,桂枝三两去皮,芍药三两,五味子半升,干姜三两切,甘草三 
两炙,细辛三两,半夏半升汤洗。 
上八味,以水一斗先煮麻黄,减二升去上沫,纳诸药煮取三升,去滓温服一升。若微利者,去麻黄,加荛 
花如鸡子大熬(炒也),令赤色;若渴者,去半夏加栝蒌根三两;若噎者,去麻黄 
加附子一枚炮;若小便不利少腹满者,去麻黄加茯苓四两;若喘者去麻黄加杏仁半升。 
按∶荛花近时无用者,《医宗金鉴》注,谓系芫花之类,攻水之力甚峻,用五分可令人下数十次,当 
以茯苓代之。又噎字,注疏家多以呃逆解之,字典中原有此讲法,然观其去麻黄加附子, 
似按寒痰凝结梗塞咽喉解法,方与所加之药相宜。 
【后世所用小青龙汤分量】麻黄二钱,桂枝尖二钱,芍药三钱,五味子钱半,干姜一钱,甘草钱 
半,细辛一钱,半夏二钱,煎一盅作一次服。 
小青龙汤所兼主诸病,喘居其末,而后世治外感痰喘者,实以小青龙汤为主方,是小青龙汤为外感中治 
痰饮之剂,实为理肺之剂也。肺主呼吸,其呼吸之机关在于肺叶之 辟,其 辟之机自如,喘病自愈。是以 
陈修园谓∶小青龙汤当以五味、干姜、细辛为主药,盖五味子以司肺之 ,干姜以司肺之辟,细辛以发动 
其辟活泼之机,故小青龙汤中诸药皆可加减,独此三味不可加减。按∶陈氏此论甚当,至其谓细辛能 
发动 辟活泼之灵机,此中原有妙理。盖细辛人皆知为足少阴之药,故伤寒少阴证多用之,然其性实能引足少 
阴与手少阴相交,是以少阴伤寒,心肾不交而烦躁者宜用之,又能引诸药之力上达于脑,是以阴寒头疼者 
必用之,且其含有龙脑气味,能透发神经使之灵活,自能发动肺叶 辟之机使灵活也。 
邹润安谓∶凡风气寒气,根据于精血、津液、便溺、涕唾以为 
患者,并能曳而出之,使相离而不相附,审斯则小青龙汤中之用细辛,亦所以除水气中之风寒也。 
仲景之方,用五味即用干姜,诚以外感之证皆忌五味,而兼痰嗽者尤忌之,以其酸敛之力甚大,能将 
外感之邪锢闭肺中永成劳嗽,惟济之以干姜至辛之味,则无碍。诚以五行之理,辛能胜酸,《内经》有明 
文也。徐氏《本草百种注》中论之甚详。而愚近时临证品验,则另有心得,盖五味之皮虽酸,其仁则含有辛 
味,以仁之辛济皮之酸,自不至因过酸生弊,是以愚治劳嗽,恒将五味捣碎入煎,少佐以射干、牛蒡诸药 
即能奏效,不必定佐以干姜也。 
特是医家治外感痰喘喜用麻黄,而以小青龙汤治外感之喘,转去麻黄加杏仁,恒令用者生疑。近见有 
彰明登诸医报而议其非者,以为既减去麻黄,将恃何者以治外感之喘乎?不知《神农本草经》谓桂枝主上 
气咳逆,吐吸,是桂枝原能降气定喘也。诚以喘虽由于外感,亦恒兼因元气虚损不能固摄,麻黄虽能定喘,其得 
力处在于泻肺,恐于元气素虚者不宜,是以不取麻黄之泻肺,但 
取桂枝之降肺,更加杏仁能降肺兼能利痰祛邪之品以为之辅佐,是以能稳重建功也。 
愚初为人诊病时,亦不知用也。犹忆岁在乙酉,邻村李××,三十余,得外感痰喘证,求为延医。其 
人体丰,素有痰饮,偶因感冒风寒,遂致喘促不休,表里俱无大热,而精神不振,略一合目即昏昏如睡,胸 
膈又似满闷,不能饮食,舌苔白腻,其脉滑而濡,至数如常。投以散风清火利痰之剂,数次无效。继延他 
医数人延医,皆无效。迁延日久,势渐危险,复商治于愚。愚谂一老医皮××,年近八旬,隐居渤海之滨,为 
之介绍延至。诊视毕,曰∶“此易治,小青龙汤证也。”遂开小青龙汤原方,加杏仁三钱,仍用麻黄一钱。 
一剂喘定。继用苓桂术甘汤加天冬、浓朴,服两剂全愈。 
愚从此知小青龙汤之神妙。自咎看书未到,遂广阅《伤寒论》诸家注疏,至喻嘉言《尚论篇》论小 
青龙汤处,不觉狂喜起舞,因叹曰∶“使愚早见此名论,何至不知用小青龙汤也。”从此以后,凡遇外感喘 
证可治以小青龙汤者,莫不投以小青龙汤。而临证细心品验,知外感痰喘之挟热者,其肺必胀,当仿《金匮》 
用小青龙汤之加石膏,且必重加生石膏方效。迨至癸巳,李××又患外感痰喘,复求愚为延医,其证脉大略 
如前,而较前热盛。投以小青龙汤去麻黄,加杏仁三钱,为其有热又加生石膏一两。服后,其喘立止。药力 
歇后,而喘仍如故。连服两剂皆然。此时皮姓老医已没,无人可以质正,愚方竭力筹思,将为变通其方,其 
岳家沧州为送医至,愚即告退。后经医数人,皆延自远方,服药月余,竟至不起。 
愚因反复研究,此证非不可治,特用药未能吻合,是以服药终不见效。徐灵胎谓∶“龙骨之性,敛 
正气而不敛邪气”,故《伤 
寒论》方中,仲景于邪气未尽者,亦用之。外感喘证服小青龙汤愈而仍反复者,正气之不敛也。遂预拟一方, 
用龙骨、牡蛎(皆不 )各一两以敛正气,苏子、清半夏各五钱以降气利痰,名之曰从龙汤,谓可用于小 
青龙汤之后。 
平均小青龙汤之药性,当以热论。而外感痰喘之证又有热者十之八九,是以愚用小青龙汤三十余年,未 
尝一次不加生石膏。即所遇之证分毫不觉热,亦必加生石膏五六钱,使药性之凉热归于平均。若遇证之觉热, 
或脉象有热者,则必加生石膏两许或一两强。若因其脉虚用人参于汤中者,即其脉分毫无热,亦必加生 
石膏两许以辅之,始能受人参温补之力。至其证之或兼烦躁,或表里壮热者,又宜加生石膏至两半或至二两, 
方能奏效。盖如此多用石膏,不惟治外感之热且以解方中药性之热也。为有石膏以监制麻黄,若遇脉之实者, 
仍宜用麻黄一钱,试举一案以征明之。 
堂姊丈褚××,体丰气虚,素多痰饮,薄受外感,即大喘不止,医治无效,旬日喘始愈,偶与愚言及,若 
甚恐惧。愚曰∶此甚易治,顾用药何如耳。《金匮》小青龙加石膏汤,为治外感痰喘之神方,辅以拙拟从 
龙汤,则其功愈显,若后再喘时,先服小青龙汤加石膏,若一剂喘定,继服从龙汤一两剂,其喘必不反 
复。若一剂喘未定,小青龙加石膏汤可服至两三剂,若犹未全愈,继服从龙汤一两剂必能全愈。若服小青龙 
加石膏汤,喘止旋又反复,再服不效者,继服从龙汤一两剂必效。遂录两方赠之,褚××甚欣喜如获异珍。 
后用小青龙汤时,畏石膏不敢多加,虽效实无捷效,偶因外感较重喘剧,连服小青龙两剂,每剂加生石 
膏三钱,喘不止而转增烦躁。急迎为诊视,其脉浮沉皆有力,遂即原方加生石膏一两,煎汤服后其喘立止, 
烦躁亦愈,继又服从龙汤两剂以善其后。 
按∶小青龙汤以驱邪为主,从龙汤以敛正为主。至敛正之药,惟重用龙骨、牡蛎,以其但敛正气而不 
敛邪气也(观《伤寒论》中仲景用龙骨牡蛎之方可知)。又加半夏、牛蒡以利痰,苏子以降气,芍药清热兼 
利小便,以为余邪之出路,故先服小青龙汤病减去十之八九,即可急服从龙汤以收十全之功也。 
寒温中,皆有痰喘之证,其剧者甚为危险。医者自出私智治之,皆不能效,惟治以小青龙汤,或治以 
小青龙加石膏汤,则可随手奏效。然寒温之证,兼喘者甚多,而有有痰无痰,与虚实轻重之分,又不必定用 
小青龙汤也。今即愚所经验者,缕析条分,胪列于下,以备治外感作喘者之采用。 
(一)气逆迫促,喘且呻,或兼肩息者,宜小青龙汤减麻黄之半,加杏仁。热者加生石膏。 
(二)喘状如前,而脉象无力者,宜小青龙汤去麻黄,加杏 
仁,再加人参、生石膏。若其脉虚而兼数者,宜再加知母。 
(三)喘不至呻,亦不肩息,唯吸难呼易,苦上气,其脉虚而无力或兼数者,宜拙拟滋阴清燥汤。 
(四)喘不甚剧,呼吸无声,其脉实而至数不数者,宜小青龙汤原方加生石膏。若脉数者,宜减麻黄 
之半,加生石膏、知母。 
(五)喘不甚剧,脉洪滑而浮,舌苔白浓,胸中烦热者,宜拙拟寒解汤。服后自然汗出,其喘即愈。 
(六)喘不甚剧,脉象滑实,舌苔白浓,或微兼黄者,宜白虎汤少加薄荷叶。 
(七)喘而发热,脉象洪滑而实,舌苔白或兼黄者,宜白虎汤加栝蒌仁。 
(八)喘而发热,其脉象确有实热,至数兼数,重按无力者,宜白虎加人参,再加川贝、苏子。若 
虚甚者,宜以生山药代粳米。 
(九)喘而结胸者,宜酌其轻重,用《伤寒论》中诸陷胸汤、丸,或拙拟荡胸汤以开其结,其喘自愈。 
(十)喘而烦躁,胸中满闷,不至结胸者,宜越婢加半夏汤,再加栝蒌仁。若在暑热之时,宜以薄荷叶 
代方中麻黄。 
至于麻黄汤证恒兼有微喘者,服麻黄汤原方即愈。业医者大抵皆知,似无庸愚之赘言。然服药后喘虽 
能愈,不能必其不传阳明。惟于方中加知母数钱,则喘愈而病亦必愈。 
若遇脉象虚者,用小青龙汤及从龙汤时,皆宜加参,又宜酌加天冬,以调解参性之热,然如此佐以 
人参、天冬,仍有不足恃之时。曾治一人年近六旬,痰喘甚剧,脉则浮弱不堪重按,其心中则颇觉烦躁,投 
以小青龙汤去麻黄加杏仁,又加生石膏一两,野台参四钱,天冬六钱,俾煎汤一次服下,然仍恐其脉虚不能胜 
药,预购生杭萸肉(药局中之山萸肉多用酒拌蒸熟令色黑,其酸敛之性大减,殊非所宜)三两,以备不时 
之需。乃将药煎服后,气息顿平,阅三点钟,忽肢体颤动,遍身出汗,又似作喘,实则无气以息,心怔忡 
莫支,诊其脉如水上浮麻,莫辨至数,急将所备之萸肉急火煎数沸服下,汗止精神稍定,又添水煮透,取浓 
汤一大盅服下,脉遂复常,怔忡喘息皆愈。继于从龙汤中加萸肉一两,野台参三钱,天冬六钱,煎 
服两剂,痰喘不再反复。 
按∶此证为元气将脱,有危在顷刻之势,重用山萸肉即可随手奏效者,因人之脏腑惟肝主疏泄, 
人之元气将脱者,恒因肝脏疏泄太过,重用萸肉以收敛之,则其疏泄之机关可使之顿停,即元气可以不脱,此 
愚从临证实验而得,知山萸肉救脱之力十倍于参、 也。因屡次重用之,以挽回人命于顷刻之间,因名 
之为回生山茱萸汤。 
其人若素有肺病常咳血者,用小青龙汤时,又当另有加减,宜去桂枝留麻黄,又宜于加杏仁、 
石膏之外,再酌加天冬数钱,盖 
咳血及吐衄之证,最忌桂枝,而不甚忌麻黄,以桂枝能助血分之热也。忆岁在癸卯,近族舅母刘媪,年过 
五旬,曾于初春感受风寒,愚为诊视疏方中有桂枝,服后一汗而愈,因其方服之有效,恐其或失,粘于壁上 
以俟再用。至暮春又感受风温,遂取其方自购药服之,服后遂至吐血,治以凉血降胃之药,连服数剂始愈。 

<目录>三、医论
<篇名>8.太阳病旋复花代赭石汤证
属性:心下停有水气可作干呕咳喘,然水气仍属无形不至于痞硬也。乃至伤寒或因汗吐下伤其中焦正气,致 
冲气肝气皆因中气虚损而上干,迫搏于心下作痞硬,且其外呼之气必噫而后出者,则非小青龙汤所能治矣, 
而必须治以旋复花代赭石汤。 
《伤寒论》原文∶伤寒发汗,若吐若下解后,心下痞硬,噫气不除者,旋复代赭石汤主之。 
【旋复代赫石汤方】旋复花三两,人参二两,生姜五两切,代赭石一两,大枣十二 
枚擘,甘草三两炙,半夏半升洗。 
上七味,以水一斗煮取六升,去滓再煮取三升,温服一升,日三服。 
人之胃气,其最重之责任在传送饮食,故以息息下行为顺。乃此证因汗吐下伤其胃气,则胃 
气不能下行,或更转而上逆。下焦之冲脉(为奇经八脉之一),原上隶阳明,因胃气上逆,遂至引动冲气上 
冲,更助胃气上逆。且平时肝气原能助胃消食,至此亦随之上逆,团结于心下痞而且硬,阻塞呼吸之气不能 
上达,以致噫气不除,噫气者强呼其气外出之声也。此中原有痰涎与气相凝滞,故用旋复花之逐痰水除胁 
满者,降胃兼以平肝,又辅以赭石、半夏降胃即以镇冲,更伍以人参、甘草、大枣、生姜以补助胃气之虚,与 
平肝降胃镇冲之品相助为理,奏功自易易也。 
赭石最善平肝、降胃、镇冲,在此方中当得健将,而只用一 
两,折为今之三钱,三分之则一剂中只有一钱,如此轻用必不能见效。是以愚用此方时,轻用则六钱,重用 
则一两,盖如此多用,不但取其能助旋复、半夏以平肝、降胃、镇冲也,且能助人参以辅助正气。盖人参 
虽善补气,而实则性兼升浮,惟借赭石之重坠以化其升浮,则人参补益之力下行可至涌泉,非然者但知用人参 
以补气,而其升浮之性转能补助逆气,而分毫不能补助正气,是用之不如不用也。是以愚从屡次经验以来,知 
此方中之赭石,即少用亦当为人参之三倍也。 
旋复花,《神农本草经》谓其味咸,主结气、胁下满、惊悸、除水。为其味咸,有似朴硝,故有软坚下 
行之功,是以有以上种种之功效。敝邑(盐山)武帝台污所产旋复花,其味咸而且辛,用 
以平肝、降胃、开痰、利气诚有殊效。 
王姓童子,十二三岁,于晨起忽左半身手足不遂,知其为痰瘀经络,致气血不能流通也。时蓄有自 
制半夏若干,及所采武帝台旋复花若干,先与以自制半夏,俾为末徐徐服之,服尽六两病愈弱半,继与以武 
帝台旋复花,俾其每用二钱半,煎汤服之,日两次,旬日全愈。盖因其味咸而兼辛,则其利痰开瘀之力当 
益大,是以用之有捷效也。夫咸而兼辛之旋复花,原为罕有之佳品,至其味微咸而不甚苦者,药局中容或 
有之,用之亦可奏效。若并此种旋复花亦无之,用此方时,宜将方中旋复花减半,多加赭石数钱,如此变通 
其方亦权可奏效也。 
或问∶人之呼吸惟在肺中,旋复代赭石汤证,其痞硬在于心下,何以妨碍呼吸至噫气不除乎?答曰∶肺 
者发动呼吸之机关也,至呼吸气之所及,非仅在于肺也,是以肺管有分支下连于心,再下则透膈连于肝,再下 
则由肝连于包肾之脂膜以通于胞室(胞室男女皆有),是以女子妊子其脐带连于胞室,而竟能母呼子亦呼, 
母吸子亦吸,斯非气能下达之明征乎?由斯知心下痞硬,所阻之 
气虽为呼吸之气,实自肺管分支下达之气也。 

<目录>三、医论
<篇名>9.太阳病大陷胸汤证
属性:又有痰气之凝结,不在心下而在胸中者。其凝结之痰气,填 
满于胸膈,至窒塞其肺中之呼吸几至停止者,此为结胸之险证,原非寻常药饵所能疗治。 
《伤寒论》原文∶太阳病脉浮而动数。浮则为风,数则为热。动则为痛,数则为虚。头痛发热,微盗 
汗出,而反恶寒者,表未解也。医反下之,动数变迟,膈内拒痛,胃中空虚,客气动 
膈,短气躁烦,心中懊 ,阳气内陷,心下因硬,则为结胸,大陷胸汤主之。 
脉浮热犹在表,原当用辛凉之药发汗以解其表,乃误认为热已入里,而以药下之,其胸中大气因下而虚, 
则外表之风热即乘虚而入,与上焦痰水互相凝结于胸膺之间,以填塞其空旷之府,是以成结胸之证。不但觉 
胸中满闷异常,即肺中呼吸亦觉大有滞碍。其提纲中既言其脉数则为热,而又言数则为虚者,盖人阴分不虚 
者,总有外感之热,其脉未必即数,今其热犹在表,脉之至数已数,故又因其脉数,而断其为虚也。至于因 
结胸而脉变为迟者,非因下后热变为凉也,盖人之脏腑中有实在瘀积,阻塞气化之流通者,其脉恒现迟象,是 
以大承气汤证,其脉亦迟也。膈内拒痛者,胸中大气与痰水凝结之气,互相撑胀而作痛,按之则其痛益甚, 
是以拒按也。胃中空虚,客气动膈者,因下后胃气伤损,气化不能息息下行(胃气所以传送饮食故以息息下 
行为顺),而与胃相连之冲脉(冲脉之上源与胃相连),其气遂易于上干,至鼓动膈膜而转排挤呼吸之气,使 
不得上升是以短气也。烦躁者,因表热内陷于胸中,扰乱其心君之火故烦躁也。 
懊者,上干之气欲透膈而外越故懊 也。 
【大陷胸汤方】大黄六两去皮,芒硝一升,甘遂一钱匕为末。 
上三味,以水六升先煮大黄,取二升,去渣,纳芒硝煮一、两沸,纳甘遂末,温服一升,得快利、 
止后服,所谓一钱匕者,俾匕首作扁方形,将药末积满其上,重可至一钱耳。 
结胸之证,虽填塞于胸中异常满闷,然纯为外感之风热内陷,与胸中素蓄之水饮结成,纵有客气上干至 
于动膈,然仍阻于膈而未能上达,是以若枳实、浓朴,一切开气之药皆无须用。惟重用大黄、芒硝以开痰而 
清热,又虑大黄、芒硝之力虽猛,或难奏效于顷刻,故又少佐以甘遂,其性以攻决为用,异常迅速,与 
大黄、芒硝化合为方,立能清肃其空旷之府使毫无障碍,制此方者乃霹雳手段也。 
甘遂之性,《神农本草经》原谓其有毒。忆愚初学医时,曾遍尝诸药以求其实际,一日清晨嚼服生甘 
遂一钱,阅一点钟未觉瞑眩,忽作水泻连连下行近十次,至巳时吃饭如常,饭后又泻数次,所吃之饭皆泻出, 
由此悟得利痰之药,当推甘遂为第一。后以治痰迷心窍之疯狂,恒恃之成功,其极量可至一钱强,然非其 
脉大实,不敢轻投,为其性至猛烈,是以大陷胸汤中所用之甘遂,折为今之分量,一次所服者只一分五厘, 
而能导引大黄、芒硝直透结胸病之中坚,俾大黄、芒硝得施其药力于瞬息之顷,此 
乃以之为向导,少用即可成功,原无须乎多也。 
甘遂之性,原宜作丸散,若入汤剂,下咽即吐出,是以大陷胸汤方必将药煎成,而后纳甘遂之末于其中也。 
甘遂之性,初服之恒可不作呕吐,如连日服即易作呕吐,若此方服初次病未尽除而需再服者,宜加 
生赭石细末二钱,用此汤药送服,即可不作呕吐。 
用大陷胸汤治结胸原有捷效,后世治结胸证敢用此方者,实百中无二三。一畏方中甘遂有毒,一疑 
提纲论脉处,原明言数则为虚,恐不堪此猛烈之剂。夫人之畏其方不敢用者,愚实难以相 
强,然其方固可通变也。《伤寒论》大陷胸汤之前,原有大陷胸丸,方系大黄半斤,葶苈半升熬,杏仁半 
升去皮尖熬黑,芒硝半升。 
上四味,捣筛二味,次纳杏仁、芒硝,研如脂,和散,取如 
弹丸一枚,别捣甘遂末一钱匕、白蜜二合,水二升,煮取一升,温顿服之。 
此方所主之证,与大陷胸汤同,因其兼有颈强如柔痉状,故于大陷胸汤中加葶苈、杏仁,和以白蜜,连渣 
煮服,因其病上连颈欲药力缓缓下行也。今欲于大陷胸汤中减去甘遂,可将大陷胸丸中之葶苈及前治噫气不 
除方中之赭石,各用数钱加于大陷胸汤中,则甘遂不用亦可奏效。夫赭石饶有重坠之力前已论之,至葶 
苈则味苦善降,性近甘遂而无毒,药力之猛烈亦远逊于甘遂,其苦降之性,能排逐溢于肺中之痰水使之迅 
速下行,故可与赭石共享以代甘遂也。 
至大陷胸汤如此加减用者,若犹畏其力猛,愚又有自拟之方以代之,即医方篇之荡胸汤是也。此荡胸汤 
方不但无甘遂,并无大黄,用以代大陷胸汤莫不随手奏效,故敢笔之于书以公诸医界也。 

<目录>三、医论
<篇名>10.太阳病小陷胸汤证
属性:(附∶白散方) 
《伤寒论》大陷胸汤后,又有小陷胸汤以治结胸之轻者,盖其证既轻,治之之方亦宜轻矣。 
《伤寒论》原文∶小结胸病,正在心下,按之则痛,脉浮滑者,小陷胸汤主之。按心下之处,注疏 
家有谓在膈上者,有谓在膈下者,以理推之实以膈上为对。盖膈上为太阳部位,膈下则非太阳部位。且小结 
胸之前(百三十九节)谓∶“太阳病重发汗,而复下之,不大便五、六日,舌上燥而渴,日晡所小有潮热, 
从心下至少腹,硬满而痛不可近者,大陷胸汤主之,”观此大陷胸汤所主之病,亦 
有从下之文,则知心上仍属胸中无疑义也。 
【小陷胸汤方】黄连一两,半夏半升汤洗,栝蒌实大者一枚。 
上三味,以水六升,先煮栝蒌取三升,去渣,纳诸药,煮取二升,去渣,分温三服。 
此证乃心君之火炽盛,铄耗心下水饮结为热痰(脉现滑象,是以知为热痰,若 
但有痰而不热,当现为濡象矣),而表阳又随风内陷,与之互相胶漆,停滞于心下为痞满,以杜塞心下经络, 
俾不流通,是以按之作痛也。为其病因由于心火炽盛,故用黄连以宁熄心火,兼以解火热之团结,又佐 
以半夏开痰兼能降气,栝蒌涤痰兼以清热,其药力虽远逊于大陷胸汤,而以分消心下之痞塞自能胜任有余也。 
然用此方者,须将栝蒌细切,连其仁皆切碎,方能将药力煎出。 
此证若但痰饮痞结于心下,而脉无滑热之象者,可治以拙拟荡胸汤,惟其药剂宜斟酌减轻耳。 
小结胸之外,又有寒实结胸,与小结胸之因于热者迥然各异,其治法自当另商。《伤寒论》谓∶宜治 
以三物小陷胸汤。又谓∶白散亦可服。三物小陷胸汤《伤寒论》中未载,注疏家或疑即小陷胸汤,谓系从治 
之法。不知所谓从治者,如纯以热治凉,恐其格拒不受,而于纯热之中少用些些凉药为之作引也,若纯以凉 
治凉,是犹冰上积冰,其凝结不益坚乎?由斯知治寒实结胸,小陷胸汤断不可服,而白散可用也。爰录其方于下。 
【白散方】桔梗三分,巴豆一分去皮心熬黑研如脂,贝母三分。 
上三味为散,纳巴豆,更于臼中杵之,以白饮和服,强人半钱匕,羸者减半,病在膈上必吐,在 
膈下必利,不利进热粥一杯,利过不止,进冷粥一杯。 
按∶方中几分之分,当读为去声,原无分量多少,如方中桔 
梗、贝母各三分,巴豆一分,即桔梗、贝母之分量皆比巴豆之分量多两倍,而巴豆仅得桔梗及贝母之分量三 
分之一也。巴豆味辛性热以攻下为用,善开冷积,是以寒实结胸当以此为主药,而佐以桔梗、贝母者,因枯 
梗不但能载诸药之力上行,且善开通肺中诸气管使呼吸通畅也。至贝母为治嗽要药,而实善开胸膺之间痰 
气郁结。至巴豆必炒黑而后用者,因巴豆性至猛烈,炒至色黑可减其猛烈之性,然犹不敢多用,所谓半钱匕 
者,乃三药共和之分量,折为今之分量为一分五厘,其中巴豆之分量仅二厘强,身形羸弱者又宜少用,可谓 
慎之又慎也。 
白散方中桔梗、贝母,其分量之多少无甚关系,至巴豆为方中主药,所用仅二厘强,纵是药力猛烈, 
亦难奏效,此盖其分量传写有误也,愚曾遇有寒实结胸,但用巴豆治愈一案,爰详细录出以征明之。 
一人年近三旬,胸中素多痰饮,平时呼吸其喉间恒有痰声。时当孟春上旬,冒寒外出,受凉太过,急急 
还家,即卧床上,歇息移时,呼之吃饭不应,视之有似昏睡,呼吸之间痰声漉漉,手摇之使醒,张目不能言, 
自以手摩胸际,呼吸大有窒碍。延医治之,以为痰厥,概治以痰厥诸方皆无效。及愚视之,抚其四肢冰 
冷,其脉沉细欲无,因晓其家人曰∶此寒实结胸证,非用《伤寒论》白散不可。遂急购巴豆去皮及心,炒 
黑捣烂,纸裹数层,压去其油(药局中名为巴豆霜,恐药局制不如法,故自制之),秤准一分五厘,开水送 
下,移时胸中有开通之声,呼吸顿形顺利,可作哼声,进米汤半碗。翌晨又服一剂,大便通下,病大轻减,脉 
象已起,四肢已温,可以发言,至言从前精神昏愦似无知觉,此时觉胸中似满闷。遂又为开 
干姜、桂枝尖、人参、浓朴诸药为一方,俾多服数剂以善其后。 
如畏巴豆之猛烈不敢轻用,愚又有变通之法,试再举一案以明之。 
一妇人年近四旬,素患寒饮,平素喜服干姜、桂枝等药。时当严冬,因在冷屋察点屋中家具为时甚久, 
忽昏仆于地,舁诸床上,自犹能言,谓适才觉凉气上冲遂至昏仆,今则觉呼吸十分努力气息始通,当速用 
药救我,言际忽又昏愦,气息几断。时愚正在其村为他家治病,急求为诊视,其脉微细若无,不足四至,询 
知其素日禀赋及此次得病之由,知其为寒实结胸无疑,取药无及,急用胡椒(辛热之品能开寒结)三钱捣碎, 
煎两三沸,徐徐灌下,顿觉呼吸顺利,不再昏厥。遂又为疏方,干姜、生怀山药各六钱,白术、当归各四钱, 
桂枝尖、半夏、甘草各三钱,浓朴、陈皮各二钱,煎服两剂,病愈十之八九。又即原方略为加减,俾 
多服数剂以善其后。 

<目录>三、医论
<篇名>11.太阳病大黄黄连泻心汤证
属性:诸陷胸汤、丸及白散之外,又有泻心汤数方,虽曰泻心实亦治 
胸中之病,盖陷胸诸方所治者,胸中有形之痰水为病,诸泻心汤所治之病,胸中无形之气化为病也。 
《伤寒论》原文∶心下痞,按之濡,其脉关上浮者,大黄黄连泻心汤主之。 
【大黄黄连泻心汤方】大黄二两,黄连一两。 
上二味,以麻沸汤二升渍之,须臾绞去渣,分温再服。 
人之上焦如雾。上焦者膈上也,所谓如雾者,心阳能蒸腾上焦之湿气作云雾而化水,缘三焦脂膜以下达 
于膀胱也。乃今因外感之邪气深陷胸中,与心火蒸腾之气抟结于心下而作痞,故用黄连以泻心火,用大黄以 
除内陷之外邪,则心下之痞者开,自能还其上焦如雾之常矣。至于大黄、黄连不用汤煮,而俱以麻沸汤渍之 
者,是但取其清轻之气以治上,不欲取其重浊之汁以攻下也。 

<目录>三、医论
<篇名>12.太阳病附子泻心汤证
属性:(附∶自拟变通方) 
心下痞病,有宜并凉、热之药为一方,而后能治愈者,《伤寒论》附子泻心汤所主之病是也。试再详论之。 
《伤寒论》原文∶心下痞,而复恶寒汗出者,附子泻心汤主之。 
【附子泻心汤方】大黄二两,黄连、黄芩各一两,附子一枚炮去皮破(别煮取汁)。 
上四味,切前三味以麻沸汤二升渍之,须臾绞去滓,纳附子汁,分温再服。 
附子泻心汤所主之病,其心下之痞与大黄黄连泻心汤所主之病同,因其复恶寒,且汗出,知其外卫之阳 
不能固摄,且知其阳分虚弱不能抗御外寒也。夫太阳之根底在于下焦水府,故于前方中加附子以补助水府之元 
阳,且以大黄、黄连治上,但渍以麻沸汤,取其清轻之气易于上行也。以附子治下,则煎取浓汤,欲其 
重浊之汁易于下降也。是以如此寒热殊异之药,浑和为剂,而服下热不妨寒,寒不妨热,分途施治,同时 
奏功,此不但用药之妙具其精心,即制方之妙亦几令人不可思议也。 
附子泻心汤之方虽妙,然为其大寒大热并用,医者恒不敢轻试。而愚对于此方原有变通之法,似较平易易 
用。其方无他,即用黄 以代附子也。盖太阳之府原有二,一在膀胱、一在胸中,而胸中所积之大气,实与太 
阳外表之卫气有息息密切之关系。气原属阳,胸中大气一虚,不但外卫之气虚不能固摄,其外卫之阳,亦遂 
因之衰微而不能御寒,是以汗出而且恶寒也。用黄 以补助其胸中大气,则外卫之气固,而汗可不出,即外 
卫之阳亦因之壮旺而不畏寒矣。盖用附子者,所以补助太阳下焦之府;用黄 
者所以补助太阳上焦之府,二府之气化原互相流通也。爰审定 
其方于下,以备采用。 
大黄三钱、黄连二钱、生箭 三钱。 
前二味,用麻沸汤渍取清汤多半盅,后一味,煮取浓汤少半盅,浑和作一次温服。 
或问∶凡人脏腑有瘀,恒忌服补药,因补之则所瘀者益锢闭也,今此证既心下瘀而作痞,何以复用 
黄以易附子乎?答曰∶凡用药开瘀,将药服下必其脏腑之气化能营运其破药之力始能奏效,若但知重用破 
药以破瘀,恒有将其气分破伤而瘀转不开者,是以人之有瘀者,固忌服补气之药,而补气之药若与开破之药同 
用,则补气之药转能助开破之药,俾所瘀者速消。 

<目录>三、医论
<篇名>13.太阳病炙甘草汤证
属性:陷胸、泻心诸方,大抵皆治外感之实证,乃有其证虽属外感, 
而其人内亏实甚者,则《伤寒论》中炙甘草汤所主之证是也。 
《伤寒论》原文∶伤寒脉结代,心动悸,炙甘草汤主之。 
脉之跳动,偶有止时,其止无定数者为结,言其脉结而不行,是以中止也;止有定数者曰代,言其 
脉至此即少一跳动,必需他脉代之也。二脉虽皆为特别病脉,然实有轻重之分,盖结脉止无定数,不过其脉偶 
阻于气血凝滞之处,而有时一止,是以为病犹轻;至代脉则止有定数,是脏腑中有一脏之气内亏,不能外 
达于脉之部位,是以为病甚重也。其心动悸者,正其结代脉之所由来也。 
【炙甘草汤方】甘草四两炙,生姜三两切,桂枝三两去皮,人参二两,生地黄一斤,阿胶二两,麦门 
冬半升,麻子仁半升,大枣三十枚擘。 
上九味,以清酒七升,水八升,先煮八味取三升,去滓纳胶,烊化消尽,温服一升,日三服,一名复脉汤。 
炙甘草汤之用意甚深,而注疏家则谓,方中多用富有汁浆之药。为其心血亏少,是以心中动悸以致脉 
象结代,故重用富有汁浆之药,以滋补心血,为此方中之宗旨,不知如此以论此方,则浅之乎视此方矣。试观 
方中诸药,惟生地黄(即干地黄)重用一斤,地黄原补肾药也,惟当时无熟地黄,多用又恐其失于寒凉,故 
煮之以酒七升、水八升,且酒水共十五升,而煮之减去十二升,是酒性原热,而又复久煮,欲变生地黄之凉性 
为温性者,欲其温补肾脏也。盖脉之跳动在心,而脉之所以跳动有力者,实赖肾气上升与心气相济,是以伤寒 
少阴病,因肾为病伤,遏抑肾中气化不能上与心交,无论其病为凉为热,而脉皆微弱无力,是明征也。由斯 
观之,是炙甘草汤之用意,原以补助肾中之气化,俾其壮旺上升,与心中之气化相济救为要着也。至其滋补 
心血,则犹方中兼治之副作用也,犹此方中所缓图者也。 
方中人参原能助心脉跳动,实为方中要药,而只用二两,折为今之六钱,再三分之一,剂中只有人参 
二钱,此恐分量有误,拟加倍为四钱则奏效当速也。然人参必用党参,而不用辽参,盖辽参有热性也。 
脉象结代而兼有阳明实热者,但治以炙甘草汤恐难奏功,宜借用白虎加人参汤,以炙甘草汤中生地黄 
代方中知母,生怀山药代方中粳米。曾治一叟,年近六旬,得伤寒证,四五日间表里大热,其脉象洪而不实, 
现有代象,舌苔白而微黄,大便数日未行。为疏方,用生石膏三两,大生地一两,野台参四钱,生怀山药六 
钱,甘草三钱,煎汤三盅,分三次温饮下,将三次服完,脉已不 
代,热退强半,大便犹未通下,遂即原方减去石膏五钱,加天冬八钱,仍如从前煎服,病遂全愈。 
炙甘草汤虽结代之脉并治,然因结轻代重,故其制方之时注 
重于代,纯用补药。至结脉恒有不宜纯用补药,宜少加开通之药 
始与病相宜者。近曾在津治一钱姓壮年,得伤寒证,三四日间延为诊视,其脉象洪滑甚实,或七八动一止, 
或十余动一止,其止皆在左部,询其得病之由,知系未病之前曾怒动肝火,继又出门感寒,遂得斯病,因 
此知其左脉之结乃肝气之不舒也。为疏方,仍白虎加人参汤加减,生石膏细末四两,知母八钱,以生山药代 
粳米用六钱,野台参四钱,甘草三钱,外加生莱菔子四钱捣碎,煎汤三盅,分三次温服下。结脉虽除,而 
脉象仍有余热,遂即原方将石膏减去一两,人参、莱菔子各减一钱,仍如前煎服,其大便从前四日未通,将 
药三次服完后,大便通下,病遂全愈。此次所用之方中不以生地黄代知母者,因地黄之性与莱菔子不相宜也。 
愚治寒温证不轻用降下之品,其人虽热入阳明之府,若无大便燥硬,欲下不下之实征,亦恒投以大剂 
白虎汤清其热,热清大便恒自通下。是以愚日日临证,白虎汤实为常用之品,承气汤恒终岁不一用也。 
一叟,年过六旬,大便下血,医治三十余日病益进,日下血十余次,且多血块,精神昏愦。延 
为诊视,其脉洪实异常,至数不数,惟右部有止时,其止无定数乃结脉也。其舌苔纯黑,知系外感大实之证, 
从前医者但知治其便血,不知治其外感实热可异也。投以白虎加人参汤,方中生石膏重用四两,为其下血日 
久,又用生山药一两以代方中粳米,取其能滋阴补肾,兼能固元气也。煎汤三盅,分三次温服下,每次送服 
广三七细末一钱,如此日服一剂,两日血止,大便犹日行数次,脉象之洪实大减,而其结益甚,且腹中觉胀。 
询其病因,知得于恼怒之后,遂改用生莱菔子五钱,而佐以白芍、滑石、天花粉、甘草诸药(外用鲜白茅根切 
碎四两煮三四沸,取其汤以代水煎药),服一剂胀消,脉之至数调匀,毫无结象而仍然有力, 
大便滑泻已减半,再投以拙拟滋阴清燥汤(方系生怀山药、滑石 
各一两,生杭芍六钱,甘草三钱),一剂泻止,脉亦和平。观上 
所录二案,知结脉现象未必皆属内亏,恒有因气分不舒,理其气即可愈者。 
又有脉非结代,而若现雀啄之象者,此亦气分有所阻隔也。曾治一少妇素日多病,于孟春中旬得伤寒, 
四五日表里俱壮热,其舌苔白而中心微黄,毫无津液,脉搏近六至,重按有力,或十余动之后,或二十余动 
之后,恒现有雀啄之象,有如雀之啄粟,恒连二三啄也。其呼吸外出之时,恒似有所龃龉而不能畅舒。细 
问病因,知其平日司家中出入账目,其姑察账甚严,未病之先,因账有差误,曾被责斥,由此知其气息不顺 
及脉象之雀啄,其原因皆由此也。问其大便自病后未行,遂仍治以前案钱姓方,将生石膏减去一两,为其津 
液亏损,为加天花粉八钱,亦煎汤三盅分三次温服下,脉象已近和平,至数调匀如常,呼吸亦顺,惟大便 
犹未通下,改用滋阴、润燥、清火之品,服两剂大便通下全愈。 

<目录>三、医论
<篇名>14.太阳病桃核承气汤证
属性:以上所论伤寒太阳篇,诸方虽不一致,大抵皆治太阳在经之 
病者也。至治太阳在府之病其方原无多,而治太阳府病之至剧者,则桃核承气汤是也。试再进而详论之。 
《伤寒论》原文∶太阳病不解,热结膀胱,其人如狂,血自 
下,下者愈。其外不解者尚未可攻,当先解其外。外解已,但少腹急结者,乃可攻之,宜桃核承气汤。 
【桃核承气汤方】桃仁五十个去皮尖,桂枝二两去皮,大黄四两去皮,芒硝二两,甘草二两炙。 
上五味,以水七升,煮取二升半,去滓纳芒硝,更上火微沸,下火,先食温服五合,日三服,当微利。 
此证乃外感之热,循三焦脂膜下降结于膀胱,膀胱上与胞室 
之脂膜相连,其热上蒸,以致胞室亦蕴有实热血蓄而不行,且其热由任脉上窜,扰乱神明,是以其人如狂也。 
然病机之变化无穷,若其胞室之血蓄极而自下,其热即可随血而下,是以其病可愈。若其血蓄不能自下,且 
有欲下不下之势,此非攻之使下不可。惟其外表未解,或因下后而外感之热复内陷,故又宜先解其 
外表而后可攻下也。 
大黄∶味苦、气香、性凉,原能开气破血,为攻下之品,然无专入血分之药以引之,则其破血之力仍 
不专,方中用桃仁者,取其能引大黄之力专入血分以破血也。徐灵胎云∶桃花得三月春和之气以生,而花色 
鲜明似血,故凡血郁、血结之疾,不能自调和畅达者,桃仁能入其中而和之散之,然其生血之功少,而去瘀之 
功多者何也?盖桃核本非血类,故不能有所补益,若瘀血皆已败之血,非生气不能流通,桃之生气在于仁,而 
味苦又能开泄,故能逐旧而不伤新也。至方中又用桂枝者,亦因其善引诸药入血分, 
且能引诸药上行以清上焦血分之热,则神明自安而如狂者可愈也。 
特是,用桃核承气汤时,又须细加斟酌,其人若素日少腹恒觉 胀,至此因外感之激发,而 胀益甚 
者,当防其素有瘀血,若误用桃核承气汤下之,则所下者,必紫色成块之血,其人血下之后,十中难救一二。 
若临证至不得已必须用桃核承气汤时,须将此事帮助以免病家之误会也。 
热结膀胱之证,不必皆累及胞室蓄血也。人有病在太阳旬余不解,午前稍轻,午后则肢体酸懒、头 
目昏沉、身似灼热、转畏寒凉、舌苔纯白、小便赤涩者,此但热结膀胱而胞室未尝蓄血也。此当治以经府双 
解之剂,宜用鲜白茅根锉细二两,滑石一两,共煮五六沸取清汤一大盅,送服西药阿斯匹林瓦许,周身得 
汗,小便必然通利,而太阳之表里俱清矣。 

<目录>三、医论
<篇名>15.太阳阳明合病桂枝加葛根汤证
属性:伤寒之传经,自太阳而阳明,然二经之病恒互相连带,不能 
划然分为两界也。是以太阳之病有兼阳明者,此乃太阳入阳明之渐也,桂枝加葛根汤所主之病是也。 
《伤寒论》原文∶太阳病,项背强KT KT (音 ),反汗出恶风者,桂枝加葛根汤主之。 
【桂枝加葛根汤方】桂枝二两去皮,芍药二两,甘草二两炙,生姜三两切,大枣十二枚擘,葛根四两。 
上六味,以水七升纳诸药煮取三升,去滓温服一升,不须啜粥,余如桂枝法将息及禁忌。 
太阳主皮毛,阳明主肌肉,人身之筋络于肌肉之中,为其热在肌肉,筋被热铄有拘挛之意,有似短羽之 
鸟,伸颈难于飞举之状,故以KT KT 者状之也。至葛根性善醒酒(葛花尤良,古有葛花醒酲汤),其凉而能 
散可知。且其能鼓胃中津液上潮以止消渴,若用以治阳明之病,是借阳明府中之气化,以逐阳明 
在经之邪也,是以其奏效自易易也。 

<目录>三、医论
<篇名>16.太阳阳明合病葛根汤证
属性:桂枝加葛根汤是治太阳兼阳明之有汗者。至太阳兼阳明之无汗者,《伤寒论》又另有治法。其方即葛根汤。 
《伤寒论》原文,太阳病,项背强KT KT ,无汗恶风者,葛根汤主之。 
【葛根汤方】葛根四两,麻黄三两去节,桂枝二两去皮,芍药二两,甘草二两炙,生姜三两切,大枣 
十二枚擘。 
上七味 咀,以水一斗,先煮麻黄、葛根减二升,去沫纳诸 
药,煎取三升,去渣温服一升,复取微似汗,不须啜粥,余如桂枝汤法将息及禁忌。 
陈古愚曰∶桂枝加葛根汤与此汤,俱治太阳经 之病,太阳之经 在背,经云∶“邪入于 ,腰脊 
乃强。”师于二方皆云治项背KT KT ,KT KT 者,小鸟羽短,欲飞不能飞,而伸颈之象也。但 
前方治汗出,是邪从肌腠而入 ,故主桂枝;此方治无汗,是邪从肤表而入 ,故主麻黄。然邪既入 , 
肌腠亦病,方中取桂枝汤全方加葛根、麻黄,亦肌表两解之治,与桂枝二麻黄一汤同意 
而用却不同,微乎微乎! 

<目录>三、医论
<篇名>17.阳明病葛根黄连黄芩汤证
属性:上所论二方,皆治太阳与阳明合病之方也。乃有其病原属太阳,误治之后,而又纯属阳明者,葛根黄芩黄 
连汤所主之病是也。 
《伤寒论》原文∶太阳病桂枝证,医反下之,利遂不止,脉促 
者,表未解也,喘而汗出者,葛根黄芩黄连汤主之。 
【葛根黄连黄芩汤方】葛根半斤,甘草二两炙,黄芩三两,黄连三两。 
上四味,以水八升,先煮葛根减二升,纳诸药煮取二升,去渣,分温再服。 
促脉与结、代之脉皆不同,注疏诸家多谓,脉动速时一止者曰促。夫促脉虽多见于速脉之中,而实非止 
也。譬如,人之行路,行行且止,少停一步复行,是结、代也。又譬如人之奔驰,急急速走,路中偶遇不平, 
足下恒因有所龃龉,改其步武,而仍然奔驰不止,此促脉也。是以促脉多见于速脉中也。凡此等脉,多因 
外感之热内陷,促其脉之跳动加速,致脉管有所拥挤,偶现此象,名之为促,若有人催促之使然也。故方 
中重用芩、连,化其下陷之热,而即用葛根之清轻透表者,引其化而欲散之热尽达于外,则表里俱清矣。且喘 
为肺病,汗为心液,下陷之热既促脉之跳动改其常度,复迫心肺之阳外越,喘而且汗,由斯知方中芩、连, 
不但取其能清外感内陷之热,并善清心肺之热,而汗喘自愈也。 
况黄连性能浓肠,又为治下利之要药乎。若服药后,又有余热利不止者,宜治以拙拟滋阴宣解汤。 
陆九芝曰∶温热之与伤寒所异者,伤寒恶寒,温热不恶寒耳。恶寒为太阳主证,不恶寒为阳明主证,仲 
景于此分之最严。恶寒而无汗用麻黄,恶寒而有汗用桂枝,不恶寒而有汗且恶热者用葛根。阳明之葛根,即太 
阳之桂枝也,所以达表也。葛根黄连黄芩汤中之芩、连,即桂枝汤中之芍药也,所以安里也。桂枝协麻黄 
治恶寒之伤寒,葛根协芩、连治不恶寒之温热,其方为伤寒、温热之分途,任后人审其病之为寒为热而分 
用之。尤重在芩、连之苦,不独可降可泻,且合苦以坚之之义,坚毛窍可以止汗,坚肠胃可以止利,所以葛 
根黄芩黄连汤又有下利不止之治,一方而表里兼清,此则药借病用,本不专为下利设也。乃后人视此方若舍 
下利一证外,更无他用者何也! 
用此方为阳明温热发表之药可为特识,然葛根发表力甚微,若遇证之无汗者,当加薄荷叶三钱,始能透 
表出汗,试观葛根汤治项背强KT KT 无汗恶风者,必佐以麻、桂可知也。当仲景时薄荷尚未入药,前曾 
论之。究之清轻解肌之品,最宜于阳明经病之发表,且于温病初得者,不仅薄荷,若连翘、蝉蜕其性皆与 
薄荷相近,而当仲景时,于连翘止知用其根(即连轺赤小豆汤中之连轺)以利小便,而犹不知用连翘以发表。 
至于古人用蝉,但知用蚱蝉,是连其全身用之,而不知用其退有皮以达皮之妙也。盖连翘若单用一两,能 
于十二小时中使周身不断微汗。若止用二三钱于有薄荷剂中,亦可使薄荷发汗之力绵长。至蝉蜕若单用三钱 
煎服,分毫不觉有发表之力,即可周身得微汗,且与连翘又皆为清表温疹之妙品以辅佐薄荷奏功,故因 
论薄荷而连类及之。 
【附录】后世用葛根黄芩黄连汤分量 
葛根(四钱) 甘草(一两炙) 黄芩(一钱五分) 黄连(一钱五分) 
不下利者,去黄连加知母三钱。无汗者,加薄荷叶、蝉蜕各钱半。 

<目录>三、医论
<篇名>18.阳明病白虎汤证
属性:上所论有葛根诸方,皆治阳明在经之病者也。至阳明在府之 
病,又当另议治法,其治之主要,自当以白虎汤为称首也。 
《伤寒论》原文∶伤寒脉浮滑,此表有热里有寒,白虎汤主之。(此节载太阳篇) 
此脉象浮而且滑,夫滑则为热入里矣,乃滑而兼浮,是其热未尽入里,半在阳明之府,半在阳明之经 
也。在经为表,在府为里,故曰表有热里有寒。《内经》谓热病者皆伤寒之类也。又谓人之伤于寒也,则为病 
热。此所谓里有寒者,盖谓伤寒之热邪已入里也。陈氏之解原如斯,愚则亦以为然。至他注疏家有谓此寒热二 
字,宜上下互易,当作外有寒里有热者,然其脉象既现浮滑,其外表断不至恶寒也。有谓此寒字当系痰之误, 
因痰寒二音相近,且脉滑亦为有痰之征也。然在寒温,其脉有滑象,原主阳明之热已实,且足征病者气血 
素充,治亦易愈。若因其脉滑,而以为有痰,则白虎汤岂为治痰之剂乎。 
《伤寒论》原文∶三阳合病,腹满身重,难以转侧,口不仁 
而面垢,谵语,遗尿。发汗则谵语,下之则额上生汗,手足逆冷。若自汗出者,白虎汤主之。(此节载阳明篇) 
证为三阳合病,乃阳明外连太阳内连少阳也。由此知三阳会合以阳明为中间,三阳之病会合,即以 
阳明之病为中坚也。是以其主病之方,仍为白虎汤,势若帅师以攻敌,以全力捣其中坚, 
而其余者自瓦解。 
《伤寒论》原文∶伤寒脉滑而厥者,里有热也,白虎汤主之。(此节载厥阴篇) 
脉滑者阳明之热传入厥阴也。其脉滑而四肢厥逆者,因肝主疏泄,此证乃阳明传来之热郁于肝中,致肝 
失其所司,而不能疏泄、是以热深厥亦深也。治以白虎汤,热消而厥自回矣。 
或问∶伤寒传经之次第,原自阳明而少阳,三传而后至厥阴,今言阳明之热传入厥阴,将勿与经旨有 
背谬乎?答曰∶白虎汤原为治阳明实热之正药,其证非阳明之实热者,仲景必不用白虎汤。此盖因阳明在经 
之热,不传于府(若入府则不他传矣)而传于少阳,由少阳而为腑脏之相传(如由太阳传少阴,即腑脏相传, 
《伤寒论》少阴篇∶麻黄附子细辛汤所主之病是也),则肝中传入阳明实热矣。究之,此等证其左右两关必皆 
现有实热之象,盖此阳明在经之热,虽由少阳以入厥阴,必仍有余热入于阳明之府,俾其府亦蕴有实热,故 
可放胆投以白虎汤,而于胃府无损也。 
【白虎汤方】知母六两,石膏一斤打碎,甘草二两炙,粳米六合。 
上四味,以水一斗,煮米熟汤成,去滓,温服一升,日三服。 
方中重用石膏为主药,取其辛凉之性,质重气轻,不但长于清热,且善排挤内蕴之热息息自毛孔达出也。 
用知母者,取其凉润滋阴之性,既可佐石膏以退热,更可防阳明热久者之耗真阴也。用甘草者,取其甘缓之 
性,能逗留石膏之寒凉不至下趋也。用粳米者,取其汁浆浓郁能调石膏金石之药使之与胃相宜也。药 
止四味,而若此相助为理,俾猛悍之剂归于和平,任人放胆用之,以挽回人命于垂危之际,真无尚之良方也。 
何犹多畏之如虎而不敢轻用哉? 
白虎汤方,三见于《伤寒论》。一在太阳篇,治脉浮滑;一在阳明篇,治三阳合病自汗出者,一 
在厥阴篇,治脉滑而厥。注 
家于阳明条下,谓苟非自汗,恐表邪抑塞,亦不敢卤莽而轻用白虎汤。自此说出,医者遇白虎汤证,恒因其 
不自汗出即不敢用,此误人不浅也。盖寒温之证,邪愈深入则愈险。当其由表入里,阳明之府渐实,急投以大 
剂白虎汤,皆可保完全无虞。设当用而不用,由胃实以至肠实而必须降下者,已不敢保其完全无虞也。 
况自汗出之文惟阳明篇有之,而太阳篇但言脉浮滑,厥阴篇但言脉滑而厥,皆未言自汗出也。由是知其脉 
但见滑象,无论其滑而兼浮,滑而兼厥,皆可投以白虎汤。经义昭然,何医者不知尊经,而拘于注家之谬说也? 
白虎汤所主之病,分载于太阳阳明厥阴篇中,惟阳明所载未言其脉象何如,似令人有未惬意之处。 
然即太阳篇之脉浮而滑及厥阴篇之脉滑而厥推之,其脉当为洪滑无疑,此当用白虎汤之正脉也。故治伤寒者,临 
证时若见其脉象洪滑,知其阳明之府热已实,放胆投以白虎汤必无差谬,其人将药服后,或出凉汗而愈,或 
不出汗其热亦可暗消于无形。若其脉为浮滑,知其病犹连表,于方中加薄荷叶一钱,或加连翘、蝉蜕各一钱, 
服后须臾即可由汗解而愈。其脉为滑而厥也,知系厥阴肝气不舒,可用白茅根煮汤以之煎药,服后须臾厥回, 
其病亦遂愈。此愚生平经验所得,故敢确实言之,以补古书所未备也。 
近世用白虎汤者,恒恪守吴氏四禁。所谓四禁者,即其所着《温病条辨》白虎汤后所列禁用白虎汤之 
四条也。然其四条之中,显有与经旨相反之两条,若必奉之为金科玉律,则此救颠扶 
危挽回人命之良方,几将置之无用之地。 
吴鞠通原文∶白虎本为达热出表,若其人脉浮弦而细者,不 
可与也;脉沉者,不可与也;不渴者,不可与也;汗不出者,不可与也,常须识此,勿令误也。 
按∶前两条之不可与,原当禁用白虎汤矣。至其第三条谓不 
渴者不可与也,夫用白虎汤之定例,渴者加人参,其不渴者即服白虎汤原方,无事加参可知矣。吴氏以为不 
渴者不可与,显与经旨相背矣。且果遵吴氏之言,其人若渴即可与以白虎汤,而亦无事加参矣,不又显与渴 
者加人参之经旨相背乎?至其第四条谓汗不出者不可与也,夫白虎汤三见于《伤寒论》,惟阳明篇中所主 
之三阳合病有汗,其太阳篇所主之病及厥阴篇所主之病,皆未见有汗也。仲圣当日未见有汗即用白虎汤,而 
吴氏则于未见有汗者禁用白虎汤,此不又显与经旨相背乎?且石膏原具有发表之性,其汗不出者不正可借 
以发其汗乎?且即吴氏所定之例,必其人有汗且兼渴者始可用白虎汤,然阳明实热之证,渴而兼汗出者,十 
人之中不过一二人,是不几将白虎汤置之无用之地乎?夫吴氏为清季名医,而对于白虎汤竟误设禁忌若此, 
彼盖未知石膏之性也。及至所着医案,曾治何姓叟,手足拘挛,因误服热药所致,每剂中用生石膏八两,服 
近五十日始愈,计用生石膏二十余斤。又治赵姓中焦留饮,上泛作喘,每剂药中皆重用生石膏,有一剂 
药中用六两八两者,有一剂中用十二两者,有一剂中用至一斤者,共服生石膏近百斤,其病始愈。以观其《温 
病条辨》中,所定白虎汤之分量生石膏止用一两,犹煎汤三杯分三次温饮下者,岂不天壤悬殊哉?盖吴氏先 
着《温病条辨》,后着《吴氏医案》,当其着《温病条辨》时,因未知石膏之性,故其用白虎汤慎重若 
此;至其着《吴氏医案》时,是已知石膏之性也,故其能放胆重 
用石膏若此,学问与年俱进,故不失其为名医也。 
人之所以重视白虎汤而不敢轻用者,实皆未明石膏之性也。夫自古论药之书,当以《神农本草经》为 
称首,其次则为《名医别录》。《神农本草经》创于开天辟地之圣神,洵堪为论药性之正宗,至《名医别录》 
则成于前五代之陶弘景,乃取自汉以后及五代以前名医论药之处而集为成书,以为《神农本草经》之辅翼 
(弘景曾以朱书本经、墨书别录为一书),今即《神农本草经》及《名医别录》之文而细为研究之。 
《神农本草经》石膏原文∶气味辛,微寒,无毒,主治中风 
寒热、心下逆气、惊喘、口干、舌焦、不能息、腹中坚痛、除邪鬼、产乳、金疮。 
按∶后世本草,未有不以石膏为大寒者,独《神农本草经》以为微寒,可为万古定论。为其微寒, 
是以白虎汤中用至一斤,至《吴氏医案》治痰饮上泛作喘,服石膏近百斤而脾胃不伤也。其言主中风者,夫中 
风必用发表之药,石膏既主之则性善发表可知,至其主寒热、惊喘、口干、舌焦、无事诠解。至其能治心下逆 
气、腹中坚痛,人或疑之,而临证细心品验,自可见诸事实也。曾治一人,患春温阳明府热已实,心下胀满 
异常,投以生石膏二两、竹茹碎末五钱,煎服后,顿觉药有推荡之力,胀满与温病皆愈。又尝治一人,少腹 
肿疼甚剧,屡经医治无效,诊其脉沉洪有力,投以生石膏三两、旱三七二钱(研细冲服)、生蒲黄三钱,煎 
服两剂全愈。此证即西人所谓盲肠炎也,西人恒视之为危险难治之病,而放胆重用生石膏即可随手奏效。至 
谓其除邪鬼者,谓能治寒温实热证之妄言妄见也。治产乳者,此乳字当作生字解(注疏家多以乳字作乳汁解者非是), 
谓妇人当生产之后,偶患寒温实热,亦不妨用石膏,即《金匮》谓妇人乳中虚烦乱、呕逆、安中益气,竹皮大 
丸主之者是也(竹皮大丸中有石膏)。治金疮者,人若为刀斧所伤,掺以生石膏细末,立能止血且能消肿愈疼也。 
《名医别录》石膏原文∶石膏除时气、头疼身热、三焦大 
热、肠胃中结气、解肌发汗、止消渴、烦逆、腹胀暴气、咽痛,亦可作浴汤。 
按∶解肌者,其力能达表,使肌肤松畅,而内蕴之热息息自 
毛孔透出也。其解肌兼能发汗者,言解肌之后,其内蕴之热又可 
化汗而出也。特是,后世之论石膏者,对于《神农本草经》之微寒既皆改为大寒,而对于《名医别录》之解 
肌发汗,则尤不相信。即如近世所出之本草,若邹润安之《本经疏证》、周伯度之《本草思辨录》,均可为卓 
卓名著,而对于《名医别录》谓石膏能解肌发汗亦有微词。 
愚浮沉医界者五十余年,尝精细体验白虎汤之用法,若阳明之实热,一半在经,一半在府,或其热虽入 
府而犹连于经,服白虎汤后,大抵皆能出汗,斯乃石膏之凉与阳明之热化合而为汗以达于表也。若犹虑其或 
不出汗,则少加连翘、蝉蜕诸药以为之引导,服后复杯之顷,其汗即出,且汗出后其病即愈,而不复有外 
感之热存留矣。若其阳明之热已尽入府,服白虎汤后,大抵出汗者少,不出汗者多,其出汗者热可由汗而解, 
其不出汗者其热亦可内消。盖石膏质重气轻,其质重也可以逐热下行,其气轻也可以逐热上出,俾胃府之 
气化升降皆湛然清肃,外感之热自无存留之地矣。 
石膏之发汗,原发身有实热之汗,非能发新受之风寒也。曾治一人,年近三旬,于春初得温病,医者以 
温药发其汗,汗出而病益加剧,诊其脉洪滑而浮,投以大剂白虎汤,为加连翘、蝉蜕各钱半,服后遍体得凉 
汗而愈。然愈后泄泻数次,后过旬日又重受外感,其脉与前次相符,乃因前次服白虎汤后作泄泻,遂改用 
天花粉、玄参各八钱,薄荷叶、甘草各二钱,连翘三钱,服后亦汗出遍体,而其病分毫不减,因此次所出 
之汗乃热汗非凉汗也。不得已遂仍用前方,为防其泄泻,以生怀山药八钱代方中粳米,服后仍遍体出凉 
汗而愈。由此案观之,则石膏之妙用,有真令人不可思议者矣。 
重用石膏以发汗,非仅愚一人之实验也。邑中友人刘××, 
肺热劳喘,热令尤甚,时当季夏,病犯甚剧,因尝见愚重用生石 
膏治病,自用生石膏四两,煎汤一大碗顿饮下,周身得凉汗,劳喘骤见轻,隔一日又将石膏如前煎饮,病又 
见轻,如此隔日一饮石膏汤,饮后必然出汗,其病亦随之降序,饮过六次,而百药难愈之痼疾竟霍然矣。 
后刘××与愚相遇,因问石膏如此凉药,何以能令人发汗?愚曰∶石膏性善发汗,《名医别录》载有明文, 
脏腑蕴有实热之人,服之恒易作汗也。此证因有伏气化热,久留肺中不去,以致肺受其伤,屡次饮石膏汤 
以逐之,则久留之热不能留,遂尽随汗出而消解无余矣。 
用石膏以治肺病及劳热,古人早有经验之方,因后世未知石膏之性,即见古人之方亦不敢信,是以后 
世无用者。其方曾载于王焘《外台秘要》,今特详录于下,以备医界之采取。 
《外台秘要》原文∶治骨蒸劳热久嗽,用石膏纹如 针者一斤,粉甘草一两,研细如面,日以水调三、 
四服,言其无毒有大益,乃养命上药,不可忽其贱而疑其寒。《名医别录》言陆州杨士丞女, 
病骨蒸,内热外寒,众医不能瘥,处州吴医用此方而体遂凉。 
按∶书中所载杨氏女亦伏气化热病。凡伏气化热之病,原当治以白虎汤,脉有数象者,白虎加人参汤, 
医者不知如此治法,是以久不瘥。吴医治以石膏、甘草粉,实为白虎汤之变通用法。乃有其证非如此变通用之而 
不能愈者(必服石膏面始能愈),此愚治伏气化热临证之实验,爰录一案于下,以明用古方者原宜因证变通也。 
一人年近四旬,身形素强壮,时当暮春,忽觉心中发热,初未介意,后渐至大小便皆不利,屡次延医 
服药,病转加剧,腹中胀满,发热益甚,小便犹可通滴沥,而大便则旬余未通矣,且又觉其热上逆,无论所服 
何药,下咽即吐出,因此医皆束手无策。后延愚为诊视,其脉弦长有力,重按甚实,左右皆然,视其舌苔浓 
而已黄,且多芒刺,知为伏气化热,因谓病者曰,欲此病愈非治 
以大剂白虎汤不可。病者谓我未受外感何为服白虎汤?答曰,此 
伏气化热证也,盖因冬日或春初感受微寒,未能即病,所受之寒伏藏于三焦脂膜之中,阻塞升降之气化,久 
而生热,至春令已深,而其所伏之气更随春阳而化热,于斯二热相并,而脏腑即不胜其灼热矣,此原与外感深入 
阳明者治法相同,是以宜治以白虎汤也。病者闻愚言而颔之,遂为开白虎汤方,方中生石膏用三两,为其呕 
吐为加生赭石细末一两,为其小便不利为加滑石六钱,至大便旬余不通,而不加通大便之药者,因赭石与石 
膏并用,最善通热结之大便也。俾煎汤一大碗,徐徐温饮下,服后将药吐出一半,小便稍通,大便未通下。 
翌日即原方将石膏改用五两,赭石改用两半,且仿白虎加人参汤之义,又加野台参三钱,复煎汤徐徐温饮下, 
仍吐药一半,大便仍未通下。于是变汤为散,用生石膏细末一两,赭石细末四钱和匀,为一日之量,鲜白茅 
根四两,煎汤分三次将药末送服,服后分毫未吐,下燥粪数枚,小便则甚畅利矣。翌日更仿白虎加人参汤之 
义,又改用野党参五钱,煎汤送服从前药末,又下燥粪数枚,后或每日如此服药,歇息一日不服药,约计共 
服生石膏细末斤许,下燥粪近百枚,病始霍然全愈。其人愈后,饮食增加,脾胃分毫无伤,则石膏之功用及石 
膏之良善可知矣。愚用石膏治大便之因热燥结者实多次矣,或单用石膏细末,或少佐以赭石细末,莫不 
随手奏效,为此次所用石膏末最多,故特志之。 

<目录>三、医论
<篇名>19.阳明病白虎加人参汤证
属性:白虎汤之外,又有白虎加人参汤,以辅白虎汤之所不逮。其 
方五见于《伤寒论》,今试约略录其数节以为研究之数据。 
《伤寒论》原文∶服桂枝汤大汗出后,大烦渴不解,脉洪大者,白虎加人参汤主之。 
【白虎加人参汤方】知母六两,石膏一斤碎绵裹,甘草二两炙,粳米六合,人参三两。 
上五味,以水一斗,煮米熟汤成,去滓,温服一升,日三服。 
服桂枝汤原取微似有汗,若汗出如水流漓,病必不解,此谓服桂枝汤而致大汗出,是汗出如水流漓也。 
因汗出过多,大伤津液,是以大烦大渴、脉洪大异常,以白虎汤解其热,加人参以复其津液而病可愈矣。 
《伤寒论》原文∶伤寒病若吐若下后,七、八日不解,热结在里,表里俱热,时时恶风,大渴,舌上 
干燥而烦,欲饮水数升者,白虎加人参汤主之。 
所谓若吐、若下者,实因治失其宜,误吐、误下,是以吐下后而病不愈也。且误吐则伤其津液,误下则 
伤其气分,津液伤损可令人作渴,气分伤损,不能助津液上潮更可作渴,是以欲饮水数升也。白虎汤中 
加人参,不但能生津液,且能补助气分以助津液上潮,是以能立建奇功也。 
《伤寒论》原文∶伤寒脉浮,发热无汗,其表不解者,不可 
与白虎汤。渴欲饮水,无表证者,白虎加人参汤主之。 
凡服白虎汤之脉,皆当有滑象脉,滑者中有热也。此节之脉象但浮,虽曰发热,不过其热在表,其不 
可与以白虎汤之实际,实在于此。乃因节中有无汗及表不解之文,而后世之治伤寒者,或谓汗不出者,不可 
用白虎汤,或谓表不解者,不可用白虎汤,至引此节之文以为征据,而不能连上数句汇通读之,以重误古人。 
独不思太阳篇中白虎汤证,其脉浮滑,浮非连于表乎?又不思白虎汤证三见于《伤寒论》,惟阳明篇白虎汤 
证,明言汗出,而太阳篇与厥阴篇之所载者,皆未言有汗乎?至于其人欲饮水数升,且无寒束之表证,是 
其外感之热皆入于里,灼耗津液,令人大渴,是亦宜急救以白虎加人参汤而无可迟疑也。 
白虎加人参汤所主之证,或渴、或烦、或舌干,固由内陷之热邪所伤,实亦由其人真阴亏损也。人参, 
补气之药,非滋阴之药, 
而加于白虎汤中,实能于邪火炽盛之时立复真阴,此中盖有化合之妙也。曾治一人,患伤寒热入阳明之府, 
脉象有力而兼硬,时作谵语,按此等脉原宜投以白虎加人参汤,而愚时当少年,医学未能深造,竟与以大剂 
白虎汤,俾分数次温饮下,翌日视之热已见退,而脉搏转数,谵语更甚。乃恍然悟会,改投以白虎加人参 
汤煎一大剂,分三次徐徐温饮下,尽剂而愈。盖白虎汤证其脉宜见滑象,脉有硬象即非滑矣,此中原有阴亏 
之象,是以宜治以白虎加人参汤,而不可但治以白虎汤也。自治愈此案之后,凡遇其人脉数或弦硬,或年过五 
旬,或在劳心劳力之余,或其人身形素羸弱,即非在汗吐下后,渴而心烦者,当用白虎汤时,皆宜加人 
参,此立脚于不败之地,战则必胜之师也。 
同邑友人李××,曾治一阳明府实证,其脉虽有力而数逾六至,李××先投以白虎汤不效,继因其脉 
数加玄参、沙参以滋其阴分仍不效,询方于愚。答曰∶此白虎加人参汤证也。李××谓,此证非在汗吐下后, 
且又不渴不烦,何为用白虎加人参汤?愚曰∶用古人之方,当即古人立方之意而推展变通之,凡白虎汤所 
主之证,其渴与烦者,多因阴分虚损,而脉象数者独非阴分虚损乎?李××闻愚言而心中会悟,改投以白 
虎加人参汤一剂而愈。 
推展白虎加人参汤之用法,不必其人身体虚弱或有所伤损也。忆愚年三旬时,曾病伏气化热,五心烦热, 
头目昏沉,舌苔白浓欲黄,且多芒刺,大便干燥,每日用生石膏数两煮水饮之,连饮数日,热象不退,因思 
或药轻不能胜病,乃于头午用生石膏五两煮水饮下,过午又用生石膏五两煮水饮下,一日之间共服生 
石膏十两,而心中分毫不觉凉,大便亦未通下。踌躇再四,精思其理,恍悟此必伏气之所入甚深,原当补助 
正气,俾吾身之正气壮旺,自能逐邪外出也。于斯欲仿白虎加人参汤之义,因无确实 
把握,犹不敢遽用大剂,就已所预存之药,用生石膏二两,野台 
参二钱,甘草钱半,适有所轧生怀山药粗渣又加少许,煎汤两蛊,分三次温饮下,饮完晚间即觉清爽,一夜 
安睡,至黎明时少腹微疼,连泻三次,自觉伏气之热全消,再自视舌苔,已退去一半,而芒刺全无矣。夫 
以常理揆之,加人参于白虎汤中,必谓能减石膏之凉力,而此次之实验乃知人参反能助石膏之凉力,其理 
果安在乎?盖石膏煎汤,其凉散之力皆息息由毛孔透达于外,若与人参并用,则其凉散之力,与人参补益之 
力互相化合,能旋转于脏腑之间,以搜剔深入之外邪使之净尽无遗,此所以白虎加人 
参汤,清热之力远胜于白虎汤也。 
愚生平治寒温实热,用白虎加人参汤时,恒多于用白虎汤时,而又恒因证制宜,即原方少有通变,凡 
遇脉过六至者,恒用生怀山药一两以代方中粳米,盖以山药含蛋白质甚多,大能滋阴补肾,而其浓郁之汁浆 
又能代粳米调胃也。若遇阳明之热既实,而其人又兼下痢者,恒用生杭芍一两以代方中知母,因芍药善清 
肝热以除痢疾之里急后重,而其凉润滋阴之性又近于知母也。若妇人产后患寒温实热者,亦以山药代粳米, 
又必以玄参八钱,以代方中知母,因山药既可补产后之肾虚,而玄参主产乳余疾,《神农本草经》原有 
明文也(《神农本草经》中石膏、玄参皆主产乳,知母未言治产乳,不敢师心自用、轻以苦寒之药施于产后也)。 
且玄参原非苦寒之品,实验之原甘而微苦(《神农本草经》谓其味苦者,当系后世传写之误),是 
以虽在产后可放胆用之无碍也。 
愚治寒温之证,于阳明肠实大便燥结者,恒投以大剂白虎汤,或白虎加人参汤,往往大便得通而愈,且 
无下后不解之虞。间有服药之后大便未即通下者,而少投以降下之品,或用玄明粉二三钱和蜜冲服,其大便即 
可通下。盖因服白虎汤及服白虎加人参汤后,壮热已消,燥结已润,自易通下也。 
有外感之实热日久不退,致其人气血两亏,危险迫于目前, 
急救以白虎加人参汤,其病只愈一半,必继服他种补益之药始能 
全愈者,今试详述一案以征明之∶ 
一幼女年九岁,于季春上旬感受温病,医者以热药发之,服后分毫无汗,转觉表里大热,盖已成白虎汤 
证也。医者不知按方施治,迁延二十余日,身体 羸,危险之朕兆歧出,其目睛上窜,几至不见,筋惕肉 , 
周身颤动,时作嗳声,间有喘时,精神昏愦,毫无知觉,其肌肤甚热,启其齿见舌缩而干,苔薄微黄,其脉 
数逾六至,左部弦细而浮,不任重按,右部亦弦细而重诊似有力,大便旬日未行,此久经外感之热灼耗,致 
气血两虚,肝风内动,真阴失守,元气将脱之候也。宜急治以白虎加人参汤,再辅以滋阴固气之品,庶可 
救愈,特虑病状若此,汤药不能下咽耳。其家人谓偶与以勺水或米汤犹知下咽,想灌以药亦如下 
咽也,于斯遂为疏方。 
【处方】生石膏细末二两,野台参三钱,生怀山药六钱,生 
怀地黄一两,生净萸肉一两,甘草二钱,共煎汤两大盅,分三次温饮下。 
此方即白虎加人参汤以生地黄代知母,生山药代粳米,而又加山萸肉也。此方若不加萸肉,为愚常用之方, 
以治寒温证当用白虎加人参汤而体弱阴亏者。今重加山萸肉一两者,诚以人当元气不固之时,恒因肝脏之疏泄 
而上脱,此证目睛之上窜,乃显露之朕兆(当属于肝),重用萸肉以收敛肝脏之疏泄,元气即可不脱。且 
喻嘉言谓上脱之证,若但知重用人参,转令人气高不返,重用萸肉为之辅弼,自无斯弊,可稳重建功。将药 
三次服完,目睛即不上窜,身体安稳,嗳声已止,气息已匀,精神较前明了,而仍不能言,大便犹未通下,肌 
肤犹热,脉数已减,不若从前之浮弦,右部重诊仍似有力,遂即原方略为加减,俾再服之。 
【第二方】生石膏细末两半,野台参三钱,生怀地黄一两,生净萸肉六钱,天冬六钱,甘草二钱, 
煎汤两盅,分两次温饮 
下,每饮一次,调入生鸡子黄一枚。 
目睛已不上窜而犹用萸肉者,诚以此证先有嗳气之病,是其气难于上达也。凡气之难于上达者,须防 
其大便通后,气或下脱,故用萸肉以预防之。至于鸡子黄,化学家谓其含有副肾髓 
质,即善滋真阴,生用之又善润大便是以加之。 
此药日服一剂,服两日热已全退,精神之明了似将撤消,而仍不能言,大便仍未通下,间有努力欲便 
之状。诊其脉热象已静且微弱,拟用灌肠法通其大便。先用野台参三钱,萸肉、天冬各四钱,煎汤服下。然 
后用灌肠法以通其大便,安然通下。仍不能言,细诊其脉微弱益甚,右部关前之脉几至不见。乃恍悟其所以 
不能言者,胸中大气下陷也,升补其胸中大气,使之上达于舌本必能言矣。 
【第三方】生箭 三钱,野台参三钱,生怀山药一两,大甘枸杞一两,北沙参一两,天冬六钱,寸 
冬带心六钱,升麻一钱,桔梗钱半,共煎汤一盅半,分两次温服下。此方连服两剂,遂能 
言语,因方中重用滋阴之药以培养其精神,而精神亦复常矣。 

<目录>三、医论
<篇名>20.阳明病三承气汤证
属性:白虎汤及白虎加人参汤两方,皆治足阳明有实热者也。至热入手阳明之府,致大便因热燥结,其燥结 
愈甚者,蕴蓄之热必愈深,此非开其燥结其热固不能消也。若斯则攻下之剂,若承气汤诸方在所必需矣。 
《伤寒论》原文∶阳明病脉迟,虽汗出不恶寒者,其身必重,短气腹满而喘,有潮热者,此外欲解, 
可攻里也。手足 然而汗出者,此大便已硬也,大承气汤主之。若汗多,微发热恶寒者,外来未也,其热 
不潮,未可与承气汤。若腹大满不通者,可与小承气汤,微和胃气,勿令大泄下。 
首句为阳明病脉迟,此见阳明病脉迟为当下之第一明征也。而愚初度此句之义,以为凡伤寒阳明之当 
下者,若其脉数,下后恒至不解,此言脉迟,未必迟于常脉,特表明其脉不数,无虑其下后不解耳。迨至阅历 
既久,乃知阳明病当下之脉原有迟者。然其脉非为迟缓之象,竟若蓄极而通,有迟而突出之象。盖其脉之 
迟,因肠中有阻塞也。其迟而转能突出者,因阳明火盛,脉原有力,有阻其脉之力而使之迟者,正所以激 
其脉之力而使有跳跃之势也。如此以解脉迟,则脉迟之当下之理自明也。 
然愚临证实验以来,知阳明病既当下,其脉迟者固可下,即其脉不迟而亦不数者,亦可下。惟脉数及六 
至则不可下,即强下之,病必不解,或病更加剧。而愚对于此等证,原有变通之下法,即白虎加人参汤, 
将石膏不煎入汤中,而以所煎之汤将石膏送服者是也。愚因屡次用此方奏效,遂名之为白虎承气汤,爰详 
录之于下,以备医界采用。 
生石膏八钱捣细,大潞党参三钱,知母八钱,甘草二钱,粳米二钱。药共五味,将后四味煎汤一盅半, 
分两次将生石膏细末用温药汤送下。服初次药后,迟两点钟,若腹中不见行动,再服第二次。若腹中已见 
行动,再迟点半钟大便已下者,停后服。若仍未下者,再将第二次药服下。至若其脉虽数而洪滑有力者,用 
此方时亦可不加党参。 
愚从前遇寒温证之当下而脉象数者,恒投以大剂白虎汤,或白虎加人参汤,其大便亦可通下。然生石 
膏必须用至四五两,煎一大碗,分数次温服,大便始可通下。间有服数剂后大便仍不通 
下者,其人亦恒脉净身凉,少用玄明粉二三钱和蜜冲服,大便即可通下。然终不若白虎承气汤用之较便也。 
生石膏若服其研细之末,其退热之力一钱可抵煎汤者半两。 
若以之通其大便,一钱可抵煎汤者一两。是以方中只用生石膏八 
钱,而又慎重用之,必分两次服下也。 
寒温阳明病,其热甚盛者,投以大剂白虎汤,其热稍退,翌日恒病仍如故。如此反复数次,病家遂疑 
药不对证,而转延他医,因致病不起者多矣。愚后拟得此方,凡遇投以白虎汤见效旋又反复者,再为治时 
即用石膏为末送服。其汤剂中用五六两者,送服其末不过一两,至多至两半,其热即可全消矣。 
【大承气汤方】大黄四两酒洗,浓朴半斤炙去皮,枳实五枚炙,芒硝三合。 
上四味,以水一斗,先煮二物,取五升,去滓,纳大黄,煮取二 
升,去滓,纳芒硝,更上微火一两沸,分温再服,得下,余勿服。 
大承气汤方,所以通肠中因热之燥结也。故以大黄之性善攻下,且善泻热者为主药。然药力之行必恃 
脏腑之气化以斡旋之,故佐以朴、实以流通肠中郁塞之气化,则大黄之攻下自易为力矣。用芒硝者,取其性 
寒味咸,善清热又善软坚,且兼有攻下之力,则坚结之燥粪不难化为溏粪而通下矣。方中之用意如此,药味无 
多,实能面面精到,而愚对于此方不无可疑之点,则在其药味分量之轻重也。 
《神农本草经》谓大黄能推陈致新,是以有黄良之名,在阳明蕴有实热大便燥结者,原宜多用。至 
浓朴不过为大黄之辅佐品,竟重用至半斤,较大黄之分量为加倍,若按一两为今之三钱折算,复分两次服之, 
则一次所服之药,当有浓朴一两二钱。夫浓朴气温味辛,若多用之,能损人真气,为人所共知,而其性又 
能横行达表,发出人之热汗。忆愚少时,曾治一阳明实热大便燥结证,方中用大黄三钱,服后大便未通下, 
改延他医,方中重用浓朴一两,服后片时出热汗遍体,似喘非喘,气弱不足以息,未逾半日而亡矣。此诚可为 
前车之鉴也。是以愚谓此方之分量必有差误,愚疑此方浓朴之分量,当亦如小承气汤为大黄分量之半,其原本 
或为浓朴之分量半大黄,大抵由此半字而误为半斤也。 
再者,本节原文以阳明病脉迟五字开端,所谓脉迟者,言其脉象虽热而至数不加数也(非谓其迟于平脉)。 
此乃病者身体素壮,阴分尤充足之脉。病候至用大承气汤时,果能有如此脉象,投以大承气汤原方,亦可随 
手奏效。而今之大承气汤证如此脉象者,实不多见也。此乃半关天时,半关人事,实为古今不同之点。即 
浓朴之分量原本如是,医者亦当随时制宜为之通变化裁,方可为善师仲景之人。非然者,其脉或不迟而数,但 
用硝、黄降之,犹恐降后不解,因阴虚不能胜其燥热也,况更重用浓朴以益其燥热乎?又或其脉纵不数,而 
热实脉虚,但用硝、黄降之,犹恐降后下脱,因其气分原亏,不堪硝、黄之推荡也,况敢重用浓朴同枳实以 
破其气乎?昔叶香岩用药催生,曾加梧桐叶一片作引,有效之者,转为香岩所笑。或问其故,香岩谓∶“余用 
梧桐叶一片时,其日为立秋,取梧桐一叶落也。非其时,将用梧桐叶何为?”由斯知名医之治病,莫不因时制 
宜,原非胶柱鼓瑟也。是以愚用承气汤时,大黄、芒硝恒皆用至七八钱,浓朴、枳实不过用二钱。或仿调胃 
承气汤之义,皆减去不用,外加生赭石细末五六钱,其攻下之力不减大承气原方,而较诸原方用之实为稳 
妥也。至其脉象数者,及脉象虽热而重按无力者,又恒先投以大剂白虎加人参汤,煎汤一大碗,分数 
次温饮下,以化胃中燥热,而由胃及肠即可润其燥结,往往有服未终剂,大便即通下者。且下后又无虞其不 
解,更无虑其下脱也。其间有大便未即通下者,可用玄明粉三钱,或西药留苦四钱,调以蜂蜜,开水冲服。或 
外治用猪胆汁导法,或用食盐(若用熬火硝所出之盐更佳)融水灌肠,皆可通下。至通下之后,亦无不愈者。 
【小承气汤方】大黄四两酒洗,浓朴二两炙去皮,枳实三枚大者炙。 
上三味,以水四升,煮取一升二合,去滓,分温二服。初服汤当 
更衣,不尔者尽饮之。若更衣者,勿服之。 
大承气汤所主之病,大肠中有燥粪,是以用芒硝软坚以化其燥粪。小承气汤所主之病为腹大满不通,是其 
病在于小肠而上连于胃,是以但用大黄、朴实以开通其小肠,小肠开通下行,大便不必通下,即通下亦不至多, 
而胃中之食可下输于小肠,是以胃气得和也。此大、小承气汤用法之分别也。而二承气汤之外,又有调 
胃承气汤,更可连类论及之。 
【调胃承气汤方】大黄四两去皮清酒浸,甘草二两炙,芒硝半升。 
上二味, 咀,以水三升,煮取一升,去滓,纳芒硝,再上火微煮令沸,少少温服之。 
大黄虽为攻下之品,原善清血分之热,心中发烦实为血分有热也。大黄浸以清酒,可引其苦寒之性上行 
以清心之热而烦可除矣。证无大便燥结而仍用芒硝者,《内经》谓热淫于内治以咸寒。芒硝味咸性寒,实 
为心家对宫之药(心属火,咸属水故为心家对宫之药),其善清心热,原有专长,故无大便燥结证而亦加 
之也。用甘草者,所以缓药力之下行,且又善调胃也。不用朴、实者,因无大便燥结及腹满之证也。 
承气汤虽有三方,而小承气及调胃承气,实自大承气变化而出。《伤寒论》所载三承气主治之证不胜录, 
然果洞悉三方之各有用意,及三方药力轻重各有区别,且所主之病虽有上、中、下之分,而究之治上可及于 
中,治中可及于下,分治之中,仍有连带关系,自能凡遇宜用承气汤证,斟酌其宜轻宜重,分别施治而无差谬矣。 
至于愚用承气汤之经过,又恒变化多端,不拘拘于三承气汤中之药味也。今试举数案以征明之。 
大承气汤所主之证,原宜脉迟,其有脉不迟而洪实有力者, 
亦不妨用。惟其脉不迟而转数,若因大便燥结,而遽投以大承气汤,其脉之无力者,恒因大便通后而虚脱。 
其脉之有力者,下后纵不至虚脱,其病亦必不能愈,所谓降后不解也。凡遇此等脉,必设法将其脉数治愈, 
然后再通其大便。 
曾治一叟,年近六旬,因外感之热过甚,致大便旬日未通,其脉数逾六至,心中烦热,延医数人,皆不 
敢用降下之剂,然除降下外,又别无治法。愚诊其脉象虽数,重按甚实,遂先投以大剂白虎加人参汤,每 
剂分三次温服下,连服两剂,壮热全消,脉已不数,大便犹未通下,继用净芒硝细末三钱,蜂蜜一两,开水 
冲服,大便通下,病遂愈。 
曾治一少年,因外感实热,致大便燥结,旬余未下,其脉亦数逾六至,且不任重按,亦投以白虎加人 
参汤,以生地黄代方中知母,生山药代方中粳米,煎汤一大碗,俾分多次徐徐温饮下。初服一剂,脉数见缓, 
遂即原方略为减轻,俾再煎服,拟后服至脉象复常,再为通其大便,孰意次剂服完而大便自通下矣。且大 
使通下后,外感之实热亦消解无余矣。此直以白虎加人参汤代承气汤也,自治愈此病之后,凡遇有证之可下而 
可缓下者,恒以白虎汤代承气,或以白虎加人参汤代承气,其凉润下达之力,恒可 
使大便徐化其燥结,无事用承气而自然通下,且下后又无不解之虞也。 
治一少妇,于大怒之余感冒伤寒,热传阳明,大便燥结,医者两次投以大承气皆吐出。诊其脉弦长有力, 
盖脉现弦长,无论见于何部,皆主肝火炽盛,此不受药之所以然也。遂于大承气汤中将朴、实减轻( 
朴实各用钱半),加生杭芍、生赭石各一两,临服药时,又恐药汤入口即吐出,先用白开水送服生赭石细末 
三钱,继将药服下,阅三点钟大便通下而病即愈矣。 
又治一人素伤烟色,平日大便七八日一行,今因受外感实 
热,十六七日大便犹未通下,心中烦热,腹中胀满,用洗肠法下燥粪少许,而胀满烦热如旧,医者谓其气 
虚脉弱,不敢投降下之药。及愚诊之,知其脉虽弱而火则甚实,遂用调胃承气汤加野台参四钱,生赭石、天门 
冬各八钱,共煎汤一大碗,分三次徐徐温饮下,饮至两次,腹中作响,觉有开通之意,三次遂不敢服,迟 
两点钟大便通下,内热全消,霍然愈矣。 
有服承气汤后,大便之燥结不下,继服些许他药而燥结始下者,试再举两案以明之。 
邑中名医刘××,愚初学医时,家中常延之,一日,见先生治一伤寒热入阳明大便燥结证,从前医者, 
投以大承气汤两剂不下,继延先生治之,单用威灵仙三钱,煎汤服后大便通下,病亦遂愈。愚疑而问曰∶威 
灵仙虽能通利二便,以较硝、黄攻下之力实远不如,乃从前服大承气汤两剂大便不下,何先生只用威灵 
仙三钱而大便即下乎?答曰∶其中原有妙理,乃前后所用之药相借以成功也。盖其从前所服之大承气汤两剂, 
犹在腹中,因其脏腑之气化偶滞,药力亦随之停顿,借威灵仙走窜之力以触发之,则硝、黄力之停顿者,可陡 
呈其开通攻决之本性,是以大便遂通下也。是威灵仙之于硝、黄,犹如枪炮家导火之线也。愚闻如此妙 
论,顿觉心地开通,大有会悟,后有仿此医案之时,亦随手奏效。 
因并录之于此,由此知医学虽贵自悟,亦必启发之有自也。 
邻村霍××,当怒动肝火之余感受伤寒,七八日间腹中胀满,大便燥结,医者投以大承气汤,大便未 
通下,肋下转觉疼不可支。其脉左部沉弦有力,知系肝经气郁火盛,急用柴胡三钱,生麦芽一两,煎汤服后, 
至半点钟肋下已不觉疼,又迟一点余钟,大便即通下。大便下后,腹即不胀,而病脱然全愈矣。 
此案实仿前案之义,亦前后药力相借以通大便也。盖肾为二 
便之关,肝行肾之气,肝又主疏泄,大便之通与不通,实于肝有 
关系也。调其肝郁,即可以通行大便,此中原有至理。至于调肝用柴胡而又必佐以生麦芽者,因麦芽生用亦 
善调肝者也。且柴胡之调肝,在于升提,生麦芽之调肝,在于宣通,若因肝不舒但用柴胡以升提之,恐初服 
下时肋下之疼将益剧。惟柴胡之升提,与麦芽之宣通相济以成调肝气之功,则肝气之郁者自开,遏者自 
舒,而徐还其疏泄之常矣。且柴胡之性不但善调肝气也,《神农本草经》谓柴胡主心腹肠胃中结气,饮 
食积聚,寒热邪气,推陈致新。三复《神农本草经》之文,是柴胡不但善于调肝,兼能消胀满通大便矣。然 
柴胡非降下之药也,其于大便之当通者,能助硝黄以通之,若遇脾胃之气下溜大便泄泻者,伍以 、术转能 
升举脾胃之气以止泄泻,柴胡诚妙药也哉。善于用柴胡者,自能深悟此中之妙理也。 
至于妊妇外感热实,大便燥结者,承气汤亦不妨用,《内经》所谓“有故无殒亦无殒也”。然此中须有斟 
酌,以上所列方中诸药,芒硝断不可用。至赭石则三月以前可用,三月以后不可用。其余虽皆可用,然究宜 
先以白虎汤或白虎加人参汤代承气,即不能完全治愈,后再用承气时亦易奏效也。曾治一妇人,妊过 
五月,得伤寒证,八九日间脉象洪实,心中热而烦躁,大便自病后未行,其脐上似有结粪,按之微疼,因其 
内热过甚,先用白虎加人参汤清之,连服两剂内热颇见轻减,而脐上似益高肿,不按亦疼,知非服降下之药 
不可也。然从前服白虎加人参汤两剂,知其大便虽结不至甚燥,治以降下之轻剂当可奏效,为疏方,用大 
黄、野台参各三钱,真阿胶(不炒另炖兑服)、天冬各五钱,煎汤服下,即觉脐上开通,过一点钟,疼处即 
不疼矣。又迟点半钟,下结粪十余枚,后代溏粪,遂觉霍然全愈,后其胎气亦无所损,届期举子矣。至方中之 
义∶大黄能下结粪,有人参以驾驭之,则不至于伤胎。又辅以阿胶,取其既善保胎,又善润肠,则大便之燥者可 
以不燥矣。用天冬者,取其凉润微辛之性(细嚼之实有辛味),最能下行以润燥开瘀,兼以解人参之热也。 

<目录>三、医论
<篇名>21.阳明病茵陈蒿汤证
属性:阳明原属燥金,其为病也多燥热,白虎、承气诸方,皆所以解阳明之燥热也。然燥热者阳明恒有之正 
病,而有时间见湿热为病,此阳明之变病也。其变病果为何病?阳明篇中诸发黄之证是也。试再进而详论之。 
《伤寒论》原文∶阳明病,发热汗出者,此为热越,不能发黄也。但头汗出,身无汗,剂颈而还, 
小便不利,渴引水浆者,此为瘀热在里,身必发黄,茵陈蒿汤主之。 
阳明病发热汗出者,热外越而湿亦随之外越,即不能发黄。若其热不外越而内蕴,又兼其人小便不利,且 
饮水过多,其湿与热必至化合而生黄,是以周身必发黄也。主以茵陈蒿汤者,以茵陈蒿汤善除湿热也。 
【茵陈蒿汤方】茵陈蒿六两,栀子十四枚擘,大黄二两去皮。 
上三味,以水一斗,先煮茵陈减六升,纳二味,煮取三升,去滓,分温三服,小盒饭利,尿如皂角汁, 
色正赤,一宿腹减,黄从小便去也。 
茵陈,性寒味苦,具有生发之气,寒能胜热,苦能胜湿,其生发之气能逐内蕴之湿热外出,故可为湿 
热身黄之主药。佐以栀子、大黄者,因二药亦皆味苦性寒也,且栀子能屈曲引心火下行以利小便。大黄之色 
能直透小便(凡服大黄者,其小便即为大黄之色,是大黄能利小便之明征),故少用之亦善利小便。至茵陈 
虽具有升发之性,《名医别录》亦谓其能下利小便,三药并用,又能引内蕴之热自小便泻出,是以服之 
能随手奏效也。 
《伤寒论》原文∶伤寒七、八日,身黄如橘子色,小便不利, 
腹微满者,茵陈蒿汤主之。 
身黄如橘而腹满,小便不利,此因湿热成病可知,故亦治以茵陈蒿汤也。 

<目录>三、医论
<篇名>22.阳明病栀子柏皮汤证
属性:《伤寒论》原文∶伤寒身黄发热,栀子柏皮汤主之。 
此节示人,但见其身黄发热,即无腹满小便不利诸证,亦直可以湿热成病断之也。 
【栀子柏皮汤方】栀子十五个擘,甘草一两炙,黄柏二两。 
上三味,以水四升,煮取一升半,去滓,分温再服。 
此方之用意,欲以分消上中下之热也。是以方中栀子善清上焦之热,黄柏善清下焦之热,加甘草与三药 
并用,又能引之至中焦以清中焦之热也。且栀子、黄柏皆过于苦寒,调以甘草之甘,俾其苦寒之性味少变, 
而不至有伤于胃也。 

<目录>三、医论
<篇名>23.阳明病麻黄连轺赤小豆汤证
属性:《伤寒论》原文∶伤寒瘀热在里,身必发黄,麻黄连轺赤小豆汤主之。 
【麻黄连轺赤小豆汤方】麻黄二两去节,赤小豆一升,连轺二两,杏仁四十个去皮尖,大枣十二枚, 
生梓白皮一升,生姜二两切,甘草二两炙。 
上八味,以潦水一斗,先煮麻黄再沸,去上沫,纳诸药,煮取三升,分温三服,半日服尽。 
按∶连轺非连翘,乃连翘根也。其性凉能泻热,兼善利湿,后世改用连翘则性不同矣。赤小豆,即 
作饭之小豆,形如绿豆而色赤者,非南来之红豆也。梓白皮,药局无鬻者,有梓树处自加 
之可也。陈修园云,若无梓白皮,可以茵陈代之。 
身发黄与黄胆不同。黄胆为胆汁妄行于血中,仲景书中虽未明言,而喻嘉言《寓意草》于钱小鲁案中 
曾发明之,彼时西人谓胆汁溢于血中之说,犹未入中国也。至身发黄之病,猝成于一两日间,其非胆汁溢于 
血分可知矣。茵陈为治热结黄胆之要药,《神农本草经》载有明文,仲景治身发黄亦用之者,诚以二证之成, 
皆由于湿热,其湿热由渐而成则为黄胆,其湿热因外感所束,仓猝而成则为身发黄,是以皆可以茵陈蒿治之也。 
身发黄之证,不必皆湿热也。阳明篇七十六节云∶“伤寒发汗 
已,身目为黄,所以然者,寒湿在里不解故也,以为不可下也,于寒湿中求之。” 
王和安曰∶黄为油热色,油中含液而包脉孕血,液虚血燥则热甚为阳黄,身黄发热之栀子柏皮证也。油 
湿血热相等而交蒸,为小便不利,身黄如橘之茵陈蒿证也。油寒膜湿,郁血为热,则寒湿甚而为阴黄,即茵 
陈五苓证也。病有热而治从寒湿,玩以为二句,语气之活自可想见。盖以为不可下,明见有可下之热黄也, 
在于寒湿中求之,言治法求之寒湿,明见黄证不纯为寒湿也。凡一证二因者,治从其甚,可于二语见之。 
上王氏之论甚精细,而愚于此节之文则又别有会悟,试引从前治愈之两案以明之。 
曾治一人受感冒,恶寒无汗,周身发黄,以麻黄汤发之,汗出而黄不退。细诊其脉,左部弦而无力,右 
部濡而无力,知其肝胆之阳不振,而脾胃又虚寒也。盖脾胃属土,土色本黄,脾胃有病,现其本色,是以其病 
湿热也,可现明亮之黄色,其病湿寒也,亦可现黯淡之黄色。观此所现之黄色,虽似黯淡而不甚黯淡者, 
因有胆汁妄行在其中也。此盖因肝胆阳分不振,其中气化不能宣通胆汁达于小肠化食,以致胆管闭塞,胆 
汁遂蓄极妄行,溢于血分而透黄色,其为黄色之根源各异,竟相并以呈其象,是以其发 
黄似黯淡而非黯淡也。审病既确,遂为拟分治左右之方以治之。 
生箭 六钱,桂枝尖二钱,干姜三钱,浓朴钱半,陈皮钱半,茵陈二钱。 
上药六味,共煎汤一大盅温服。 
方中之义∶用黄 以助肝胆之阳气,佐以桂枝之辛温,更有开通之力也。用干姜以除脾胃之湿寒, 
辅以浓朴能使其热力下达。更辅以陈皮,能使其热力旁行,其热力能布 充周,脾胃之寒湿自除也。用茵 
陈者,为其具有升发之性,实能打开胆管之闭塞,且其性能利湿,更与姜、桂同用,虽云苦寒而亦不觉其苦 
寒也。况肝胆中寄有相火,肝胆虽凉,相火之寄者仍在,相火原为龙雷之火,不可纯投以辛热之剂以触发之, 
少加茵陈,实兼有热因寒用之义也。 
又治一人,时当仲秋,寒热往来,周身发黄,心中烦热,腹中又似觉寒凉,饮食不甚消化,其脉左部 
弦硬,右部沉濡,心甚疑之,问其得病之由,答云,不知。因细问其平素之饮食起居,乃知因屋宇窄隘,六 
七月间皆在外露宿,且其地多潮湿,夜间雾露尤多。乃恍悟此因脏腑久受潮湿,脾胃属土,土为太阴,湿郁 
久则生寒,是以饮食不能消化。肝胆属木,木为少阳,湿郁久则生热,又兼有所寄之相火为之熏蒸,以致 
胆管肿胀闭塞,是以胆汁妄行,溢于血中而身黄也。舌上微有白苔,知其薄受外感,侵入三焦,三焦原为手少 
阳与足少阳并为游部,一气贯通,是以亦可作寒热,原当以柴胡和解之,其寒热自已,茵陈性近柴胡,同为少 
阳之药,因其身发黄,遂用茵陈三钱以代柴胡,又加连翘、薄荷叶、生姜各三钱,甘草二钱,煎汤服后,周身 
得汗(足少阳不宜发汗手少阳宜发汗),寒热往来愈,而发黄如故。于斯就其左右之脉寒热迥殊者,再拟一方治之。 
茵陈三钱,栀子三钱,干姜三钱,白术三钱炒,浓朴二钱, 
焰硝五分研细。 
上六味,将前五味煎汤一大盅,乘热纳硝末融化服之。 
方中之义∶用栀子、茵陈以清肝胆之热,用干姜、白术、浓朴以除脾胃之寒,药性之凉热迥然不同, 
而汇为一方自能分途施治也。用焰硝者,因胆管之闭塞,恒有胆石阻隔,不能输其胆汁 
于小肠,焰硝之性善消,即使胆管果有胆石,服之亦不难消融也, 

<目录>三、医论
<篇名>24.阳明病猪苓汤证
属性:发黄之证,多成于湿热,诸治发黄之方,皆治湿热之方也。 
乃有本阳明病,其人蕴有湿热而不发黄者,自当另议治法,而阳明篇中亦曾载其治方矣。 
《伤寒论》原文∶若脉浮发热,渴欲饮水,小便不利者,猪苓汤主之。 
此节所谓脉浮者,乃病入阳明,而犹连太阳之府也。盖太阳之病,在经脉浮,在府亦脉浮,此因太阳 
之府蕴有实热,以致小便不利,而热之入于阳明者,不能由太阳之府分消其热下行,转上逆而累及于肺,是 
以渴欲饮水也。治以猪苓汤,是仍欲由太阳之府分消其热也。 
【猪苓汤方】猪苓去皮、茯苓、阿胶、滑石、泽泻各一两。 
上五味,以水四升,先煮四味取二升,去滓,纳下阿胶,烊消,温服七合,日三服。 
猪苓、茯苓,皆为渗淡之品,而猪苓生于枫下,得枫根阴柔之气,以其性善化阳,以治因热小便不利 
者尤宜,故用之为主药。用泽泻者,因其能化水气上升以止渴,而后下降以利小便也。用滑石者,其性可 
代石膏,以清阳明之实热,又能引其热自小便出也。用阿胶者,因太阳之府原与少阴相连,恐诸利水之 
药或有损于少阴,故加阿胶大滋真阴之品,以助少阴之气化也。 

<目录>三、医论
<篇名>25.阳明病四逆汤证
属性:总计阳明篇中之病证,大抵燥而且热也,其有不燥而转湿者,此阳明之变证也。于治发黄诸方,曾发 
明之矣。更有不热而反寒者,此亦阳明之变证也。夫病既寒矣,必须治以热剂,方为 
对证之药,是则温热之剂,又宜讲求矣。 
《伤寒论》原文∶脉浮而迟,表热里寒,下利清谷者,四逆汤主之。 
外感之着人,恒视人体之禀赋为转移,有如时气之流行,受病者或同室、同时,而其病之偏凉、偏热, 
或迥有不同。盖人之脏腑素有积热者,外感触动之则其热益甚;其素有积寒者,外感触动 
之则其寒亦益甚也。明乎此则可与论四逆汤矣。 
【四逆汤方】甘草二两炙,干姜两半,附子一枚生用去皮破八片。 
上三味以水三升,煮取一升二合,去滓,分温再服,强人可大附子一枚、干姜三两。 
干姜为温暖脾胃之主药,伍以甘草,能化其猛烈之性使之和平,更能留其温暖之力使之常久也。然 
脾胃之温暖,恒赖相火之壮旺,附子色黑入肾,其非常之热力,实能补助肾中之相火,以浓脾胃温暖之本 
源也。方名四逆者,诚以脾主四肢,脾胃虚寒者,其四肢常觉逆冷,服此药后,而四肢之厥逆可回也。 
方中附子,注明生用,非剖取即用也。因附子之毒甚大,种附子者,将附子剖出,先以盐水浸透,至 
药局中又几经泡制,然后能用,是知方中所谓附子生用者,特未用火炮熟耳。 

<目录>三、医论
<篇名>26.少阳病提纲及汗吐下三禁
属性:阳明之热,已入府者,不他传矣。若犹在经,而未入于府 
者,仍可传于少阳。而少阳确实之部位,又须详为辨析也。夫太阳主外,阳明主里,而介于太阳、阳明之 
间者,少阳也。少阳外与太阳相并则寒,内与阳明相并则热,是以少阳有病而寒热往来也。由此而论,则传经 
之次第,当由太阳而少阳,由少阳而阳明,而《内经》竟谓一日巨阳(即太阳)受之,二日阳明受之;三日 
少阳受之者何也?盖他手、足同名之经各有界限,独少阳主膜,人身之膜无不相通。膜有连于太阳者,皮肤 
下腠理之白膜也。膜有连于阳明者,肥肉、瘦肉间之膜也。此为手少阳经以三焦为府者也( 
三焦亦是膜,发源于命门,下焦为包肾络肠之膜,中焦为包脾连胃之膜,上焦为心下膈膜及心肺一系相连之膜 
)。又两胁之下皆板油,包其外者亦膜也,此为足少阳之膜以胆为府者也。由此知介于太阳、阳明之间者,手少 
阳也;传经在阳明之后者,足少阳也。太阳传阳明原自手少阳经过,而《伤寒论》未言及者,以其重足 
经,不重手经也。总之,手、足少阳之膜原相联系,即手、足少阳之气化原相贯通,是以《内经》谓少阳 
为游部(游部者,谓其中气化自手经至足经,自足经至手经游行无定也),更由此知所谓与太阳相并者,为 
手少阳腠理之膜也,与阳明相并者,为足少阳板油之膜也,以其相近故能相并也。能明 
乎此,即可与论少阳篇之病矣。 
《伤寒论》原文∶少阳之为病,口苦,咽干,目眩也。 
《伤寒论》原文∶少阳中风,两耳无所闻,目赤胸中满而烦者,不可吐下,吐下则悸而惊。 
《伤寒论》原文∶伤寒脉弦细,头痛发热者,属少阳。少 
阳不可发汗,发汗则谵语,此属胃。胃和则愈,胃不和,烦而悸。 
此节所言之证,乃少阳病之偏于热者也。弦细,固为少阳之脉,观提纲中谆谆以胃和、胃不和为重要 
之点,想自阳明传少阳时,其外感之热仍有一半入府,而非尽传于少阳。脉虽弦细,重 
按必然甚实,此原当为少阳、阳明合病也。愚遇此等证脉时,恒将 
柴胡汤方中药味减半(惟人参与甘草不减),外加生石膏一两,知母五钱(此为白虎加人参汤与小柴胡汤各用一半), 
则少阳之病可解,其胃中之热亦可尽清,而不至有胃不和之虞矣。此节合上节,为少阳病汗、吐、下三禁, 
凡治少阳病者当切记之。 

<目录>三、医论
<篇名>27.少阳病小柴胡汤证
属性:《伤寒论》原文∶伤寒五六日中风,往来寒热,胸胁苦满,默默不欲饮食,心烦喜呕。或胸中烦而不呕, 
或渴,或腹中痛,或胁下痞硬,或心下悸,小便不利,或不渴,身有微热,或咳者,与小柴胡汤主之。(此节载太阳篇) 
【小柴胡汤方】柴胡半斤,黄芩三两,人参三两,甘草三两炙,半夏半升洗,生姜三两切,大枣十二枚擘。 
上七味,以水一斗二升,煮取六升,去滓,再煎取三升,温服一升,日三服。 
用小柴胡汤加减法∶若胸中烦而不呕,去半夏、人参,加栝蒌实一枚;若渴者,去半夏,加人参,合前 
成四两半,栝蒌根四两,若腹中痛者,去黄芩,加芍药三两,若胁下痞硬,去大枣,加牡蛎四两;若心下悸 
小便不利者,去黄芩,加茯苓四两;若不渴外有微热者,去人参,加桂枝三两,温复取微汗愈;若咳者, 
去人参、大枣、生姜,加五味子半升、干姜二两。 
【附录】后世用小柴胡汤分量 
柴胡(八钱) 黄芩(三钱) 人参(三钱) 甘草(三钱) 
清半夏(四钱) 生姜(三钱切) 大枣(四枚擘) 
小柴胡证喜呕者,不必作呕吐也,但常常有欲呕之意,即为喜呕。是以愚治伤寒,遇有觉恶心而微寒热 
往来者,即投以小柴胡汤,一剂而愈。此《伤寒论》所谓∶“伤寒中风,有柴胡证, 
但见一证便是,不必悉具”也。 
方中重用柴胡,正所以助少阳之枢转以引邪外出也。犹恐其枢转之力或弱,故又助以人参,以浓其上升 
之力,则少阳之邪直能随少阳之气透膈上出矣。用半夏者,因其生当夏半,能通阴阳、和表里,且以病本 
喜呕,而又升以柴胡、助以人参,少阳虽能上升,恐胃气亦因之上逆,则欲呕之证仍难愈,用半夏协同甘草、 
姜、枣降胃兼以和胃也。用黄芩者,以其形原中空,故善清躯壳之热,且亦以解人参之偏热也。 
小柴胡汤证,原忌发汗,其去滓重煎者,原所以减柴胡发表之力,欲其但上升而不外达也。乃太阳篇 
一百零三节,服小柴胡汤后,竟有发热汗出之文,读《伤寒论》者,恒至此而生疑,注疏家亦未 
见有详申其义者,今试录其原文细研究之。 
《伤寒论》原文∶凡柴胡汤证而下之,若柴胡证不罢者,复与柴胡汤,必蒸蒸而振,却发热汗出而解。 
服小柴胡汤,以引少阳之邪透膈上出而无事出汗,原为小柴胡汤证治法之正则。然药力之上升透膈颇难, 
必赖其人之正气无伤,药借正气以营运之而后可以奏效。至误下者,足少阳之邪多散漫于手少阳三焦脂膜之 
中,仍投以小柴胡汤,其散漫于手少阳者,遂可借其和解宣通之力,达于太阳而汗解矣。其留于胁下板 
油中者,因误降伤气,无力上达,亦遂借径于手少阳而随之汗解,故于汗出上特加一却字,言非发其汗而却 
由汗解,此乃因误下之后而使然,以明小柴胡汤原非发汗之药也。其汗时必发热蒸蒸而振者,有战而后汗 
意也。盖少阳之病由汗解,原非正路,而其留于胁下之邪作汗解尤难,乃至服小柴胡汤后,本欲上透膈膜,因 
下后气虚,不能由上透出,而其散漫于手少阳者,且又以同类相招,遂于蓄极之时而开旁通之路,此际几有 
正气不能胜邪气之势,故汗之先必发热而振动,此小柴胡汤方中所以有人参之助也。是 
以愚用此方时,于气分壮实者,恒不用人参,而于误服降药后及气虚者,则必用人参也。 
人身之膜原,无处不相联系,女子之胞室亦膜也。其质原两膜相合,中为夹室,男女皆有,男以化精,女 
以通经,故女子之胞室亦曰血室。当其经水初过之时,适有外感之传经者乘虚袭 
入,致现少阳证病状,亦宜治以小柴胡汤,《伤寒论》中亦曾详论之矣。 
《伤寒论》原文∶妇人中风,七、八日续得寒热,发作有时,经水适断者,此为热入血室。其血必 
结,故使如疟状,发作有时,小柴胡汤主之。 
伤寒之病既自阳明传少阳矣,间有遵少阳之法治之,其证复转阳明者,此虽仅见之证,亦宜详考治法。 
《伤寒论》原文∶服柴胡汤已,渴者属阳明,以法治之。 
喻嘉言曰∶风寒之邪,从阳明而传少阳,起先不渴,里证未具,及服小柴胡汤已,重加口渴,则邪还 
阳明,而当调胃以存津液矣。然不曰攻下,而曰以法治之,意味无穷。盖少阳之寒热往来,间有渴证,倘 
少阳未罢而恣言攻下,不自犯少阳之禁乎?故见少阳重转阳明之证,但云以法治之,其法维何?即发汗利小便 
已,胃中躁烦,实大便难之说也。若未利其小便,则有猪苓、五苓之法,若津液热炽,又有人参白虎之法, 
仲景圆机活泼,人存政举,未易言矣。 
少阳证,不必皆传自阳明也。其人若胆中素有积热,偶受外感,即可口苦、心烦、寒热往来,于柴 
胡汤中加生石膏、滑石、生杭芍各六钱,从小便中分消其热,服后即愈。若其左关甚有力者,生石膏可用至一 
两(小柴胡汤证宜加石膏者甚多,不但此证也),自无转阳明之虞也。 
小柴胡汤本为平和之剂,而当时医界恒畏用之,忌柴胡之升 
提也。即名医若叶天士,亦恒于当用柴胡之处避而不用,或以青 
蒿代之。诚以古今之人,禀赋实有不同,古人禀质醇浓,不忌药之升提,今人体质多上盛下虚,上焦因多有 
浮热,见有服柴胡而头疼目眩者,见有服柴胡而齿龈出血者,其人若素患吐血及脑充血证者,尤所忌服。至 
愚用小柴胡汤时,恒将原方为之变通,今试举治验之数案以明之。 
同庄张××,少愚八岁,一方之良医也。其初习医时,曾病少阳伤寒,寒热往来,头疼发热,心中烦而 
喜呕,脉象弦细,重按有力。愚为疏方调治,用柴胡四钱,黄芩、人参、甘草、半夏各三钱,大枣四枚,生 
姜三大片,生石膏一两,俾煎汤一大盅服之。张××疑而问曰∶此方乃小柴胡汤外加生石膏也,按原方中分 
量,柴胡半斤以一两折为今之三钱计之,当为二两四钱,复三分之,当为今之八钱,今方中他药皆用其原 
分量,独柴胡减半,且又煎成一盅服之,不复去滓重煎,其故何也?弟初习医,未明医理,愿兄明以教我也! 
答曰∶用古人之方,原宜因证、因时,为之变通,非可胶柱鼓瑟也。此因古今气化,略有不同,即人之禀 
赋遂略有差池,是以愚用小柴胡汤时,其分量与药味,恒有所加减。夫柴胡之性,不但升提,实原兼有发 
表之力,古法去滓重煎者,所以减其发表之力也。今于方中加生石膏一两以化其发表之力,即不去滓重煎, 
自无发表之虞,且因未经重煎,其升提之力亦分毫无损,是以止用一半,其力即能透膈上出也。放心服之, 
自无差谬。张××果信用愚言,煎服一剂,诸病皆愈。 
邻村刘姓妇人,得伤寒少阳证,寒热往来无定时,心中发热,呕吐痰涎,连连不竭,脉象沉弦。为开小 
柴胡汤原方,亦柴胡减半用四钱,加生石膏一两,云苓片四钱。有知医者在座,疑而问曰∶少阳经之证,未 
见有连连吐粘涎不竭者,今先生用小柴胡汤,又加石膏、茯苓,将勿不但为少阳经病,或又兼他经之病 
乎?答曰∶君之问诚然也,此乃少阳病而连太阴也。少阳之去路 
原为太阴之经,太阴在腹为湿土之气,若与少阳相并,则湿热化合,即可多生粘涎,故于小柴胡汤中加石膏、 
茯苓,以清少阳之热,即以利太阴之湿也。知医者闻之,甚为叹服。遂将此方煎服,两剂全愈。 
在辽宁曾治一妇人,寒热往来,热重寒轻,夜间恒作 语,其脉沉弦有力。因忆《伤寒论》谓妇人热入 
血室证,“昼日明了,暮则 语”,遂细询之,因知其初受外感三四日,月信忽来,至月信断后遂变斯证。据 
所云云,知确为热入血室,是以其脉沉弦有力也。遂为开小柴胡原方,将柴胡减半,外加生黄 二钱、 
川芎钱半,以升举其邪之下陷,更为加生石膏两半,以清其下陷之热,将小柴胡如此变通用之,外感之 
邪虽深陷,实不难逐之使去矣。将药煎服一剂,病愈强半,又服一剂全愈。 
按∶热入血室之证,其热之甚者,又宜重用石膏二三两以清其热,血室之中,不使此外感之热稍有存 
留始无他虞。愚曾治有血室溃烂脓血者数人,而究其由来,大抵皆得诸外感之余,其为热入血室之遗恙可 
知矣。盖当其得病之初,医者纵知治以小柴胡汤,其遇热之剧者,不知重用石膏以清血室之热,遂致酿成危险 
之证,此诚医者之咎也。医者有治热入血室之证者,尚其深思愚言哉。 

<目录>三、医论
<篇名>28.少阳病大柴胡汤证
属性:柴胡汤证,有但服小柴胡不能治愈,必治以大柴胡汤始能治愈者,此病欲借少阳之枢转,外出而阻 
于阳明之阖,故宜于小柴胡汤中兼用开降阳明之品也。 
《伤寒论》原文∶太阳病过经十余日,反二、三下之,后四、五 
日柴胡证仍在者,先与小柴胡。呕不止,心下急,郁郁微烦者,为未解也,与大柴胡汤下之则愈。 
【大柴胡汤方】柴胡半斤,黄芩三两,芍药三两,半夏半升洗,生姜五两切,枳实四两炙,大枣十二枚擘。 
上七味,以水一斗二升,煮取六升,去滓再煎,温服一升,日三服。一方用大黄二两。 
《伤寒论》大柴胡汤,少阳兼阳明之方也。阳明胃府有热,少阳之邪又复挟之上升,是以呕不止,心下 
急,郁郁微烦。欲用小柴胡汤提出少阳之邪,使之透膈上出,恐其补胃助热而减去人参,更加大黄以降其热, 
步伍分明,出奇致胜,此所以为百战百胜之师也。乃后世畏大黄之猛,遂易以枳实。迨用其方不效,不 
得不仍加大黄,而竟忘去枳实,此大柴胡一方,或有大黄或无大黄之所由来也。此何以知之?因此方所主之 
病宜用大黄,不宜用枳实而知之。盖方中以柴胡为主药,原欲升提少阳之邪透膈上出,又恐力弱不能直达,故小 
柴胡汤中以人参助之。今因证兼阳明,故不敢复用人参以助热,而更加大黄以引阳明之热下行,此 
阳明与少阳并治也。然方名大柴胡,原以治少阳为主,而方中既无人参之助,若复大黄、枳实并用,既破其 
血,又破其气,纵方中有柴胡,犹能治其未罢之柴胡证乎?盖大黄虽为攻下之品,然偏于血分,仍于气分无 
甚伤损,即与柴胡无甚龃龉,至枳实能损人胸中最高之气,其不宜与柴胡并用明矣。愚想此方当日原但加 
大黄,后世用其方者,畏大黄之猛烈,遂易以枳实,迨用其方不效,不得不仍加大黄,而竟忘去枳实,此 
为大柴胡或有大黄或无大黄,以致用其方者恒莫知所从也。以后凡我同人,有用此方者,当以加大黄去枳实 
为定方矣。究之,古今之气化不同,人身之强弱因之各异,大柴胡汤用于今日,不惟枳实不可用,即大黄 
亦不可轻用,试举两案以明之。 
邑诸生刘××,其女适邑中某氏,家庭之间,多不适意,于 
季秋感冒风寒,延其近处医者治不愈。刘××邀愚往诊,病近一 
旬,寒热往来,其胸中满闷烦躁皆甚剧,时作呕吐,脉象弦长有力,愚语刘××曰∶此大柴胡汤证也,从 
前医者不知此证治法,是以不愈。刘××亦以愚言为然,遂为疏方,用柴胡四钱,黄芩、芍药、半夏各三钱, 
生石膏两半碎,竹茹四钱,生姜四片,大枣四枚,俾煎服。刘××疑而问曰∶大柴胡汤原有大黄、枳实,今 
减去之,加石膏、竹茹,将勿药力薄弱难奏效乎?答曰∶药之所以能愈病者,在对证与否,不在其力之强 
弱也,宜放胆服之,若有不效,余职其咎。病患素信愚,闻知方中有石膏,亦愿急服,遂如方煎服一剂,须臾 
觉药有推荡之力,胸次顿形开朗,烦躁呕吐皆愈。刘××疑而问曰∶余疑药力薄弱不能奏效,而不意其奏效更 
捷,此其理将安在耶?答曰∶凡人得少阳之病,其未病之先,肝胆恒有不舒,木病侮土,脾胃亦恒先受其扰。 
迨其阳明在经之邪,半入于府半传于少阳,于斯,阳明与少阳合病,其热之入于府中者,原有膨胀之力,复 
有肝胆以扰之,其膨胀之热,益逆行上干而凌心,此所以烦躁与胀满并剧也。小柴胡汤去人参原可舒其肝胆, 
肝胆既舒,自不复扰及脾胃,又重用石膏,以清入府之热,俾其不复膨胀上干,则烦躁与满闷自除也。况又加 
竹茹之开胃止呕者以辅翼之,此所以奏效甚捷也。 
又治一人,年逾弱冠,禀赋素羸弱。偶于初夏,因受感冒病于旅邸,求他医治疗,将近一旬,病犹 
未愈。后愚诊视,其父正为病患煎药,视其方乃系发表之剂,及为诊视,则白虎汤证也。嘱其所煎之药,千 
万莫服。其父求为疏方,因思病者禀赋素弱,且又在劳心之余,若用白虎汤原宜加人参,然其父虽信愚,而其 
人实小心过度,若加人参,石膏必须多用,或因此不敢径服,况病者未尝汗下,且又不渴,想但用白虎汤 
不加人参亦可奏效。遂为开白虎汤原方,酌用生石膏二两,其父犹嫌其多。愚曰∶此因君 
平素小心特少用耳,非多也。又因脉有数象,外加生地黄一两以 
滋其阴分,嘱其煎汤两盅,分两次温饮下,且嘱其若服后热未尽退,其大便不滑泻者,可即原方仍服一剂。 
迨愚旋里后,其药止服一剂,热退十之八九,虽有余热未清,不敢再服。迟旬日大便燥结不下,两腿微肿,拟 
再迎愚诊视,适有其友人某,稍知医学,谓其腿肿系为前次重用生石膏二两所伤。其父信友人之言,遂改延 
他医,见其大便燥结,投以降下之剂,方中重用大黄八钱,将药服下,其人即不能语矣。其父见病势垂危, 
急遣人迎愚,未及诊视而亡矣。夫此证之所以便结腿肿者,因其余热未清,药即停止也。乃调养既失之于前, 
又误药之于后,竟至一误再误,而不及挽救,使其当时不听其友之盲论,仍迎愚为延医,或再投以白虎 
汤,或投以白虎加人参汤,将石膏加重用之,其大便即可因服凉润之药而通下,大便既通,小便自利,腿之 
肿者不治自愈矣。就此案观之,则知大柴胡汤中用大黄,诚不如用石膏也(重用白虎汤即可代承气,曾 
于前节论承气汤时详言之)。盖愚当成童时,医者多笃信吴又可,用大剂承气汤以治阳明府实之证,莫不随 
手奏效。及愚业医时,从前之笃信吴又可者,竟恒多偾事,此相隔不过十余年耳,况汉季至今千余年哉。盖愚在 
医界颇以善治寒温知名,然对于白虎汤或白虎加人参汤,旬日之间必用数次,而对于承气汤恒终岁未尝一用也。 

<目录>三、医论
<篇名>29.少阳篇三阳合病之治法
属性:少阳篇,有三阳并病之证,提纲中详其病状而未列治法,此或有所遗失欤?抑待后人遇此证自为拟方欤? 
愚不揣固陋,本欲拟一方以补之,犹恐所拟者未必有效,今试即其所载病状以研究其病情,再印征以生平所 
治之验案,或于三阳合病之治法,可得其仿佛欤。 
《伤寒论》原文∶三阳合病,脉浮大,上关上,但欲眠睡,目合则汗。 
陶华氏谓,此节所言之病,当治以小柴胡加葛根、芍药。而愚对于此证有治验之案二则,又不拘拘于 
小柴胡汤中加葛根、芍药也。试详录二案于下,以质诸医界。 
一人年过三旬,于初春患伤寒证,经医调治不愈。七八日间延为诊视。头疼,周身发热,恶心欲吐,心 
中时或烦躁,头即有汗而身上无汗,左右脉象皆弦,右脉尤弦而有力,重按甚实,关前且甚浮。即此脉论,其 
左右皆弦者,少阳也,右脉重按甚实者,阳明也,关前之脉浮甚者,太阳也,此为三阳合病无疑。其既有 
少阳病而无寒热往来者,缘与太阳、阳明相并,无所为往无所为来也。遂为疏方∶生石膏、玄参各一两,连 
翘三钱,茵陈、甘草各二钱,俾共煎汤一大盅顿服之,将药服后,俄顷汗出遍体,近一点钟,其汗始竭,从 
此诸病皆愈。其兄颇通医学,疑而问曰∶此次所服药中分毫无发表之品,而服后竟由汗解而愈者何也?答 
曰∶出汗之道,在调剂其阴阳,听其自汗,非可强发其汗也,若强发其汗,则汗后恒不能愈,且转至增剧者 
多矣。如此证之三阳相并,其病机本欲借径于手太阴之络而外达于皮毛,是以右脉之关前独浮也,乃因其重 
按有力,知其阳明之积热,犹团结不散,故用石膏、玄参之凉润者,调剂其燥热,凉热化合,自能作汗, 
又少加连翘、茵陈(可代柴胡)以宣通之,遂得尽随病机之外越者,达于皮毛而为汗解矣,此其病之所以愈也。 
又治一人,年近三旬,因长途劳役,感冒甚重,匆匆归家,卧床不起。经医延医,半月病益加剧。及愚 
视之,见其精神昏愦。 语不休,肢体有时惕动不安,其两目直视,似无所见,其周身微热,而间有发潮热 
之时,心中如何,询之不能自言,其大便每日下行皆系溏粪,其脉左右皆弦细而浮,数逾六至,重按即 
无。根据此证之肢体惕动、两目直视,且间发潮热者,少阳也;精神昏愦、 语不休者,阳明也;其脉弦 
而甚浮者,乃自少阳还 
太阳也,是以谓之三阳合病也。《伤寒论》少阳篇所谓三阳合病,然《伤寒论》中所言者,是三阳合病之实 
证,而此症乃三阳合病之虚证,且为极虚之证。凡三阳合病以病已还表,原当由汗而解,此病虽虚,亦当由 
汗而解也。特以脉数无根,真阴大亏,阳升而阴不能应,是以不能化合而为汗耳。治此证者,当先置外感于不 
问,而以滋培其真阴为主,连服数剂,俾阴分充足,自能与阳气化合而为汗,汗出而病即愈矣。若但知病须 
汗解,当其脉数无根之时,即用药强发其汗,无论其汗不易出也,即服后将汗发出,其人几何不虚脱也。 
愚遂为开生地黄、熟地黄、生山药、大枸杞各一两,玄参、沙参、净萸肉各五钱,煎汤一大碗,分 
两次温饮下。此药一日夜间连进两剂。翌晨再诊其脉,不足六至,精神亦见明了,自服药后大便未行,遂 
于原方中去萸肉,加青连翘二钱,服后周身得汗,病若失。 

<目录>三、医论
<篇名>30.太阴病提纲及意义
属性:病由少阳而愈者,借少阳之枢转而外出也。乃有治不如法,其病不能借少阳之枢转外出,而转由腔上 
之膜息息透入腹中,是由少阳而传太阴也。夫病既传于太阴,其病情必然变易,自当另 
议治法,是则太阴经发现之病状与其治法,又当进而研究矣。 
《伤寒论》原文∶太阴之为病,腹满而吐,食不下,自利益甚,时腹自痛,若下之,必胸中结硬。 
脾为太阴之府,其处重重油脂包裹,即太阴之经也。盖论其部位,似在中焦之内,惟其处油脂独浓于 
他处,是太阴之经虽与三焦相连,而实不与三焦相混也。且《难经》谓脾有散膏半斤,即西人所谓甜肉汁,原 
系胰子团结而成,方书谓系脾之副脏,其分泌善助小肠化食,实亦太阴经之区域也。为其经居于腹之中 
间,是以腹满为太阴经之的病。其吐食、自利者,此经病而累及于府,脾病不能运化饮食,是以吐利交作也。 
其腹痛者,因病在太阴,中焦郁满而气化不通也。下之必胸中结硬者,因下后脾气下陷,不能散精以 
达于肺(《内经》谓脾气散精,以达于肺),遂致郁于胸中而为结硬也。 
此节提纲甚详,而未言治法,及下节汇通观之,可自得其治法矣。 
《伤寒论》原文∶太阴中风,四肢烦疼,阳微阴涩而长者,为欲愈。 
唐容川曰∶此节言太阴中风,脉若阳大而阴滑,则邪盛内陷矣。今阳不大而微,阴涩而又见长者,乃 
知微涩是邪不盛,不是正气虚。长是正气足,不嫌其微涩,故为欲愈也。 
一人,年甫弱冠,当仲春之时,因伏气化热窜入太阴,腹中胀满,心中烦躁,两手肿疼,其脉大而濡,两 
尺重按颇实。因思腹中者太阴之部位也,腹中胀满乃太阴受病也,太阴之府为脾,脾主四肢,因伏气化热 
窜入太阴,是以两手肿疼也,其两足无恙者,因窜入太阴者,原系热邪,热之性喜上行,是以手病而足不 
病也。为其所受者热邪,是以觉烦躁也。因忆《伤寒论》太阴篇有谓“太阴中风,四肢烦疼,阳微阴涩而长 
者,为欲愈”今此证所现之脉,正与欲愈之脉相反,是不得不细商治法也。为疏方,用生莱菔子、生鸡内 
金各三钱以开其胀满,滑石、生杭芍各六钱以清其烦躁,青连翘、生蒲黄各四钱以愈其两手肿疼,按方煎服 
两剂,诸病皆愈。诚以太阴之病原属湿热,其湿热之郁蒸于上者,服此汤后得微汗而解,其湿热之陷溺于 
下者,服此汤后亦可由小便分利而解矣。若执此案之方以治前节所言之病,于方中加法半夏三钱,则在 
上之吐可止,再加生山药八钱,下焦之利亦可愈,至方中之连翘、蒲黄,不但能治手肿疼,即腹中作痛服之亦能 
奏效,将方中药味,略为增加以治前节之病,亦可随手治愈也。 

<目录>三、医论
<篇名>31.太阴病桂枝汤证
属性:太阴之病,有时可由汗解者,然必须病机有外越之势,原非强发其汗也。 
《伤寒论》原文∶太阴病脉浮者,可发汗,宜桂枝汤。 
脉浮者,乃太阴之病机外越,原可因其势而导之,故可服桂枝汤以发其汗也。若其脉之浮而有力者,宜 
将桂枝减半(用钱半),加连翘三钱,盖凡脉有浮热之象者,过用桂枝,恒有失血之虞,而连翘之性凉而 
宣散,凡遇脉象之浮而有力者,恒得之即可出汗,故减桂枝之半而加之以发汗也。恐其汗不出者,服药后亦可 
啜粥,若间有太阴腹满之本病者,可加生莱菔子三钱,盖莱菔子 
生用,其辛辣之味不但可以消胀满,又可助连翘发汗也。 

<目录>三、医论
<篇名>32.太阴病宜四逆辈诸寒证
属性:太阴自少阳传来原无寒证,乃有其脏本素有寒积,经外感传入而触发之,致太阴外感之证不显,而 
惟显其内蓄之寒凉以为病者,是则不当治外感,惟宜治内伤矣。 
《伤寒沦》原文∶自利不渴者,属太阴,以其脏有寒故也。当温之,宜四逆辈。 
陈修园曰∶自利者,不因下而利也。凡利则津液下注,多见 
口渴,惟太阴湿土之为病不渴,至于下利者当温之,而浑言四逆辈,所包括之方原甚广。 
王和安谓∶温其中兼温其下宜四逆,但温其中宜理中、吴茱萸,寒结宜大建中汤。湿宜真武汤,渴者 
宜五苓散,不渴而滑宜赤石脂禹余粮汤。而愚则谓甘草干姜汤、干姜附子汤、茯苓四逆汤诸方,皆可因证选用也。 

<目录>三、医论
<篇名>33.太阴病坏证桂枝加芍药汤及桂枝加大黄汤证
属性:太阴之证,不必皆由少阳传来也,又间有自太阳传来者。然自少阳传来,为传经次第之正传,自太阳传 
来则为误治之坏证矣。 
《伤寒论》原文∶本太阳病,医反下之,因而腹满时痛者,属太阴也,桂枝加芍药汤主之;大实痛者,桂 
枝加大黄汤主之。 
太阳病误下之后,外感之邪固可乘虚而入太阴,究之,脾土骤为降下所伤,肝木即乘虚而侮脾土,腹中 
之满而且痛,实由肝脾之相龃龉也。桂枝原为平肝、和脾(气香能醒脾,辛温之性,又善开脾瘀)之圣药, 
而辅以芍药、甘草、姜、枣,又皆为柔肝扶脾之品,是桂枝汤一方,若免去啜粥,即可为治太阴病之正药也。 
至于本太阳证,因误下病陷太阴,腹满时痛,而独将方中芍药加倍者,因芍药善治腹痛也。试观仲景用小柴胡 
汤,腹痛者去黄芩加芍药,通脉四逆汤腹痛者,去葱加芍药此明征也。若与甘草等分同用,为甘草芍药汤, 
原为仲景复阴之方,愚尝用之以治外感杂证、骤然腹痛(须审其腹痛非凉者),莫不随手奏效。惟其所用之分 
量,芍药倍于甘草是为适宜,盖二药同用原有化合之妙,此中精微固不易窥测也。且二药如此并用,大有开 
通之力,则不惟能治腹痛,且能除腹满也。 
惟此方中芍药加倍为六两,甘草仍为二两,似嫌甘草之力薄弱, 
服后或难速效,拟将甘草亦加重为三两,应无药性偏重之弊欤。 
【桂枝加芍药汤方】桂枝三两,芍药六两,甘草二两炙,生姜三两切,大枣十二枚擘。 
上五味,以水七升,煮取三升,去滓,分温三服。 
【桂枝加大黄汤方】即前方加大黄二两。 

<目录>三、医论
<篇名>34.少阴病提纲及意义
属性:中焦脂膜团聚之处,脾居其中,斯为太阴,前已言之。而下焦 
脂膜团聚之处,肾居其中,故名少阴。少阴之府在肾,少阴之经 
即团聚之脂膜也。为其与中焦团聚之处相连,是以外感之传递,可由太阴而传入少阴也。 
《伤寒论》原文∶少阴之为病,脉微细,但欲寐也。 
少阴之病,有凉有热。说者谓,若自太阴传来,是阳明、少阳之邪顺序传入少阴则为热证,若外感之 
邪直中真阴则为寒证者。而愚临证实验以来,知少阴病之凉者原非直中,乃自太阳传来为表里之相传,亦为 
腑脏之相传(膀胱),因太阳之府相连之脂膜,原与包肾之脂膜相通也。其间有直中者,或因少阴骤虚之时, 
饮食寒凉而得,此不过百中之一二,其治法原当另商也。至少阴病之热者,非必自传经而来,多由伏气化热 
入少阴也。所谓伏气者,因其素受外寒甚轻,不能即病,其所受之寒气伏于三焦脂膜之中,阻塞气化之升 
降而化热(气化因阻塞而生热伏气即可与之相合而化热),恒因少阴之虚损,伏气即乘虚而窜入少阴,此 
乃少阴之热病初得即宜用凉药者也。 
至无论其病之或凉或热而脉皆微细者,诚以脉之跳动发于心,而脉之所以跳动有力者,又关于肾。 
心肾者,水火之根源也,心肾之气相济,则身中之气化自然壮旺,心肾之气若相离,身中之气化遽形衰惫。 
少阴有病者,其肾气为外邪遏抑不能上升以济心,是以无论病之为凉为热,其脉象皆微细无力也。其但欲 
寐者,因心肾之气不交,身中之气化衰惫,精神必然倦懒,是以常常闭目以静自休息。又因肾气不能上达 
以吸引心阳下潜,是以虽闭目休息不能成寐,而为但欲寐之状也。 
《伤寒论》原文∶少阴病,欲吐不吐,心烦但欲寐,五、六日自利而渴者,属少阴也。虚故引水自救, 
若小便色白者,少阴病形悉具,小便白者以下焦虚有寒,不能制水,故令色白也。 
张拱端曰∶少阳为阳枢,少阴为阴枢。少阴欲吐不吐者,以 
少阴有水复有火,水火之气循环上下不利,故欲吐不吐也。少阳喜呕者,以内外之气由焦膜中行,焦膜 
不利则气难于出入,是以逆于胃而为呕,呕则气少畅,故喜呕,此少阴欲吐、少阳喜呕之所以然也。又太阴、 
少阴俱有自利证,少阴自利而渴,从少阴本热之化也。太阴自利不渴,从太阴本湿之化也。若治少阴上焦口渴 
之实热,罔顾及下焦下利之虚寒,则下利不止矣。故凡对于水火分病,则当用寒热之药分治之。对于水火合 
病,无妨用寒热之药合治之。本论用方有纯于寒有纯于热,复有寒热并用者,即此理也。 
──本节未列治法,张氏谓上有实热下有虚寒,宜用寒热之药函问∶师答曰∶宜用生地一两,生杭 
芍五钱,附子二钱,干姜二钱,细辛一钱,计五味,不宜用石膏。 高××注 
《伤寒论》原文∶少阴病脉紧,至七、八日自下利,脉暴微,手足反温,脉紧反去者,为欲解也。虽烦下利必自愈。 
少阴之中有水有火,肾左右两枚水也,肾系命门所生之相火,少阴中之火也。外寒自太阳透入少阴,与少 
阴中之水气相并,以阻遏其元阳,是以脉现紧象,紧者寒也,乃阴盛阳衰逼阳不得宣布之象也。迨阳气蓄之 
既久,至七、八日又重值太阳、阳明主气之候,命门之火因蓄极而暴发,遂迫阴寒自下利外出,脉之紧 
者亦暴微。盖脉紧原阳为阴迫,致现弦而有力之象,至暴微是由紧而变为和缓,未必甚微,与紧相较则见 
其微矣。且其手足反温,此为元阳已回之兆无疑,治少阴中之寒病者,原以保护其元阳为主,此时或有心 
烦之病,实因相火暴发,偶有浮越于上者,此益足征元阳之来复也,是以知其必愈也。 
陈修园曰∶此言少阴得阳热之气而解也。余自行医以来,每遇将死之证,必以大药救之,忽而发烦下利, 
病家怨而更医,医家亦诋前医之误,以搔不着疼痒之药居功,余反因热肠受谤,甚矣名医之不可为也。 
愚年少时,初阅《伤寒论浅注》至此,疑修园之言,似近自为掩饰。迨医学研究既久,又加以临证实验, 
乃知修园之言诚不诬也。后又见常德张拱端所着《伤寒论会参》,亦谓修园之言诚然,且谓余治一人, 
服药后下利苦烦,又喜哈哈,似癫非癫,数时病愈,亦与此节烦利自愈一例也。而愚则谓,若遇少阴阴寒险 
证,欲用药以回其阳时,不妨预告病家,阳回之后恒现下利心烦之象,自能免病家之生疑也。 
子××按∶数年前余在里处,曾治一少阴寒证,服药后下利发烦而愈。一九三三年腊月,在津又治阎×× 
少阴寒证,服茴香、干姜等药久不愈,乃询方于余,俾单服生硫黄如枣大,食前服,每日三次,至五六 
日忽下利日二三次,骇而问余。余曰∶此寒结得硫黄之热而开,《伤寒论》所谓虽烦下利必自愈者是也。后数 
日利果止,其病亦愈。 
《伤寒论》原文∶少阴病下利,若利自止,恶寒而蜷卧,手足温者,可治。 
张拱端曰∶以上三节,俱少阴阴寒之病,前两节手足温,第三节自烦欲去衣被,均为阳回之候,均 
为自愈、可治之证。可见治少阴伤寒以阳为主,不特阴证见阳脉者生,即阴病见阳证亦为易愈。论中恶寒而 
蜷之蜷字,足供阴寒在内之考察,何也?大凡阴寒之病,俱有屈曲身体之形,其屈曲之理,实关系于督、任 
二脉,盖以督统诸阳行于背脊,任统诸阴行于胸腹,阴寒在内屈曲身体者,伸背之阳以抑阴也,阳热在 
内直腰张胸者,伸腹之阴以济阳也。如天气热人必张胸,天气寒人必拘急,观其伸阳以自救,则蜷之属于阴 
寒其理可得矣。故阳盛则作痉,阴盛则蜷卧,理所必然也。至于自烦欲去衣被,是阴得阳化故为可治。 
张氏论督任相助之理,以释本节中之蜷卧颇为精细,而愚于张氏所论之外,则更别有会心也。推 
坎离相济,阴阳互根之理,人之心肾相交,即能生热(心肾相交能补助元阳故能生热),而心肾之相交每在呼 
气外出之时也。盖当呼气外出之时,其心必然下降,其肾必然上升(此可默自体验),此际之一升一降 
而心肾交矣。是乃呼吸间自然之利益,以为人身热力之补助也(试观睡时恒畏冷,以人睡着则呼吸慢,热力即 
顿形不足,是明征也)。人之畏冷身蜷卧者,是其心肾欲相交以生热也(此中有无思无虑自然而然之天机)。 
至于病热,其身恒后挺,是心肾欲相远,防其相交以助热也。 
《伤寒论》原文∶少阴病吐利,手足不逆冷,反发热者不死,脉不至者,灸少阴七壮。 
陈修园谓∶宜灸太 二穴。张拱端谓,亦可灸复溜二穴。而愚则谓,若先灸太 二穴,脉仍不应,可再 
灸复溜二穴,灸时宜两腿一时同灸。太 二穴,在足内踝后五分,跟骨上动脉中,复溜二穴, 
在内踝上二寸,大骨后侧陷中,此与太 同为少阴生脉之源。 

<目录>三、医论
<篇名>35.少阴病麻黄附子细辛汤证
属性:《伤寒论》原文∶少阴病,始得之,反发热脉沉者,麻黄附子细辛汤主之。 
此外感之寒凉,由太阳直透少阴,乃太阳与少阴合病也。为少阴与太阳合病,是以少阴已为寒凉所伤,而 
外表纵有发热之时,然此非外表之壮热,乃恶寒中之发热耳。是以其脉不浮而沉,盖少阴之脉微细,微细原 
近于沉也。故用附子以解里寒,用麻黄以解外寒,而复佐以辛温香窜之细辛,既能助附子以解里 
寒,更能助麻黄以解外寒,俾其自太阳透入之寒,仍由太阳作汗而解,此麻黄附子细辛汤之妙用也。 
【麻黄附子细辛汤方】麻黄二两去节,细辛二两,附子一枚炮去皮破八片。 
上三味,以水一斗,先煮麻黄减二升,去上沫,纳诸药,煮取三升,去滓,温服一升,日三服。 
方中细辛二两,折为今之六钱,复三分之一剂中仍有二钱, 
而后世对于细辛有服不过钱之说,张隐庵曾明辩其非。二钱非不可用,而欲免病家之疑,用一钱亦可奏效。盖 
凡宜发汗之病,其脉皆浮,此独脉沉,而欲发其汗,故宜用细辛辅之,至谓用一钱 
亦可奏效者,因细辛之性原甚猛烈,一钱亦不为少矣。 
此方若少阴病初得之,但恶寒不发热者,亦可用。曾治一少年,时当夏季,午间恣食西瓜,因夜间失眠, 
遂于食余当窗酣睡,值东风骤至,天气忽变寒凉,因而冻醒,其未醒之先,又复梦中遗精,醒后遂觉周身寒凉 
抖战,腹中隐隐作疼,须臾觉疼浸加剧。急迎为延医,其脉微细若无,为疏方,用麻黄二钱,乌附子三钱,细 
辛一钱,熟地黄一两,生山药、净萸肉各五钱,干姜三钱,公丁香十粒,共煎汤服之,服后温复,周身得微 
汗,抖战与腹疼皆愈。此于麻黄附子细辛汤外而复加药数味者,为其少阴暴虚腹中疼痛也。 
李××,夏日得少阴伤寒,用麻黄附子细辛汤,加生山药、大熟地二味治愈。 

<目录>三、医论
<篇名>36.少阴病黄连阿胶汤证
属性:《伤寒论》原文∶少阴病得之二、三日以上,心中烦,不得卧,黄连阿胶汤主之。 
二、三日以上,即一日也,合一二三日而浑言之即初得也。细绎其文,是初得即为少阴病,非自他经传 
来也。其病既非自他经来,而初得即有热象者,此前所谓伏气化热而窜入少阴者也。盖凡伏气化热之后,恒 
因薄受外感而猝然发动,至其窜入之处,又恒因其脏腑素有虚损,伏气即乘虚而入。由斯而论,则此节之所 
谓少阴病,乃少阴病中之肾虚兼热者也。夫大易之象,坎上离下为既济,坎为肾而在上者,此言肾当 
上济以镇心也,离为心而在下者,此言心当下济以暖肾也。至肾素虚者,其真阴之气不能上 
济以镇心,心火原有摇摇欲动之机,是以少阴之病初得,肾气为伏气所阻,欲上升以济心尤难,故他病之现 
象犹未呈露,而心中已不胜热象之烦扰而不能安卧矣,是以当治以黄连阿胶汤也。 
【黄连阿胶汤】黄连四两,黄芩一两,芍药二两,鸡子黄二枚,阿胶三两。 
上五味,以水五升,先煮三味取二升,去滓,纳胶烊尽,小 
冷,纳鸡子黄,搅令相得,温取七合,甘三服。 
黄连味苦入心,性凉解热,故重用之以解心中发烦,辅以黄芩,恐心中之热扰及于肺也,又肺为肾之 
上源,清肺亦所以清肾也。芍药味兼苦酸,其苦也善降,其酸也善收,能收降浮越之阳,使之下归其宅,而 
性凉又能滋阴,兼能利便,故善滋补肾阴,更能引肾中外感之热自小便出也。阿胶其性善滋阴,又善潜 
伏,能直入肾中以生肾水。鸡子黄中含有副肾髓质之分泌素,推以同气相求之理,更能直入肾中以益肾水, 
肾水充足,自能胜热逐邪以上镇心火之妄动,而心中发烦自愈矣。 
或问∶提纲明言心中烦而不能卧,夫心与肾共为少阴,使其心之本体热而生烦,其人亦恒不能安卧,此 
虽为手少阴,亦可名为少阴病也,何先生独推本于肾,由肾病而累及于心乎?答曰∶凡曰少阴病者,必脉象 
微细,开端提纲中已明言之矣。若谓其病发于心,因心本体过热而发烦,则其脉必现浮洪之象,今其心虽 
有热,而脉象仍然微细(若脉非微细而有更改者,本节提纲中必言明此定例也),则知其病之源 
不在于心而在于肾可知,其心中发烦不得卧,实因肾病而累及于心,更可知也。 
此节所言之病,原系少阴病初得无大热者,故治以黄连阿胶汤已足清其热也。若其为日既久,而热 
浸加增,或其肾经素有蕴热,因有伏气之热激发之,则其热益甚,以致心肾皆热,其壮热充 
实于上下,又非此汤所能胜任矣。愚遇此等证,则恒用白虎加人 
参汤,以玄参代知母、山药代粳米,又加鲜茅根,生鸡子黄,莫不随手奏效,用之救人多矣,因名之为 
坎离互根汤,详录其方之分量及煎法于下。 
生石膏(三两细末) 玄参(一两) 生怀山药(八钱) 甘草(三钱) 
野台参(四钱) 鲜白茅根(六两洗净切碎) 生鸡子黄(三枚) 
上共六味,先将茅根煎三四沸,去滓,纳余药五味,煎汤三盅,分三次温服,每服一次,调入鸡子黄一枚。 
方中之意∶石膏、人参并用,不但能解少阴之实热,并能于邪热炽盛之时立复真阴,辅以茅根更能助肾 
气上升与心火相济也。至于玄参,性凉多液,其质轻松,原善清浮游之热,而心之烦躁可除,其色黑入肾, 
又能协同鸡子黄以滋肾补阴,俾少阴之气化壮旺,自能逐邪外出也。 
或问∶外感之伏气,恒受于冬日,至春日阳升随春日之阳而化热,是以温病多有成于伏气化热者,至伤 
寒约皆在于冬日,何亦有伏气化热者乎?答曰∶伏气化热,原有两种化法。伏气冬日受之,伏于三焦脂膜 
之中,迟至春日随春日之阳生而化热,此伏气化热之常也。乃有伏气受于冬日,其所伏之处,阻塞腹内升降 
之气化,其气化因阻塞而生热,伏气亦可随之化热,此伏气化热之变也。迨其化热之后,或又微受外感而 
触发之,其触发之后,又恒因某经素有虚损,乘虚而窜入其经,此所以伤寒病中亦有伏气化热者也。 

<目录>三、医论
<篇名>37.少阴病当灸及附子汤证
属性:《伤寒论》原文∶少阴病得之一、二日,口中和,其背恶寒者,当灸之,附子汤主之。 
陈修园曰∶此宜灸鬲关二穴以救太阳之寒,再灸关元一穴以 
助元阳之气。 
元阳存于何处?盖人身有两气海,《内经》谓膈上为气海,此后天之气海,所藏者宗气也(即胸中大气)。 
哲学家以脐下为气海,此先天之气海,所藏者祖气,即元气也。人身之元阳,以元气为体质,元气即以元阳 
为主宰,诚以其能斡旋全身则为元气,能温暖全身则为元阳,此元阳本于先天,原为先天之君火,以命门之 
相火为之辅佐者也(此与以心火为君火,以肝中所寄之少阳相火为相火者,有先天后天之分)。至下焦气海之 
形质,原为脂膜及胰子团结而中空,《医林改错》所谓,形如倒提鸡冠花者是也。人生结胎之始先生此物, 
由此而下生督脉,上生任脉,以生全身,故其处最为重要之处,实人生性命之根也。有谓人之元气、元阳藏贮 
于胞室者,不知胞室若在女子,其中生疮溃烂,原可割而去之,若果为藏元气、元阳之处,岂敢为之割去乎? 
《伤寒论》原文∶少阴病,身体痛,手足寒,骨节痛,脉沉者,附子汤主之。 
【附子汤方】附子二枚炮去皮破八片,茯苓二两,人参二两,白术四两,芍药三两。 
上五味,以水八升,煮取三升,去滓,温服一升,日三服。 

<目录>三、医论
<篇名>38.少阴病桃花汤证
属性:《伤寒论》原文∶少阴病,下利便脓血者,桃花汤主之。 
《伤寒论》原文∶少阴病,二、三日至四、五日,腹痛,小便不利,下脓血者,桃花汤主之。 
少阴之病寒者居多,故少阴篇之方亦多用热药。此二节之文,未尝言寒,亦未尝言热。然桃花汤之药, 
则纯系热药无疑也。乃释此二节者,疑下利脓血与小便不利必皆属热,遂强解桃花汤中药性,谓石脂性凉, 
而重用一斤,干姜虽热,而只用一两,合用之仍当以凉论者。然试取石脂一两六钱、干姜一钱煎服,或凉 
或热必能自觉,药性岂可重误乎?有谓此证乃大肠因热腐烂致成溃疡,故下脓血。《神农本草经》谓石脂 
能消肿去瘀,故重用一斤以治溃疡,复少用干姜之辛烈,以消溃疡中之毒菌。然愚闻之,毒菌生于热者,惟 
凉药可以消之,黄连、苦参之类是也;生于凉者,惟热药可以消之,干姜、川椒之类是也。桃花汤所主之下脓 
血果系热毒,何以不用黄连、苦参佐石脂,而以干姜佐石脂乎?虽干姜只用一两,亦可折为今之三钱,虽分 
三次服下,而病未愈者约必当日服尽。夫一日之间服干姜三钱,其热力不为小矣,而以施之热痢下脓血者,有 
不加剧者乎?盖下利脓血原有寒证,即小便不利亦有寒者。注疏诸家疑便脓血及小便不利皆为热证之发现,遂 
不得不于方中药品强为之解,斯非其智有不逮,实因临证未多耳。 
辽宁何××,年三十许,因初夏在外地多受潮湿,下痢脓血相杂,屡治不愈。后所下者渐变紫色,有 
似烂炙,杂以脂膜,腹中切痛,医者谓此因肠中腐败,故所下如此,若不能急为治愈,则肠将断矣。何×× 
闻之惧甚,遂乘火车急还辽宁,长途辛苦,至家,病益剧,下痢无度,而一日止食稀粥少许,求为延医。 
其脉微弱而沉,左三部几不见,问其心中自觉饮食不能消化,且觉上有浮热,诸般饮食皆懒下咽,下痢一昼夜 
二十余次,每欲痢时,先觉腹中坠而且疼,细审病因,确系寒痢无疑,其所下者如烂炙,杂以脂膜者,是其肠 
中之膜,诚然腐败随痢而下也。西人谓此证为肠溃疡,乃赤痢之坏证,最为危险,所用之药有水银基制品, 
而用于此证实有不宜。即愚平素所遇肠溃疡证,亦恒治以金银花、旱三七、鸭胆子诸药,对于此证亦不宜。 
盖肠溃疡证多属于热,而此证独属于寒,此诚肠溃疡证之仅见者也。遂俾用生硫黄细末,掺熟面少许为小丸, 
又重用生山药、熟地黄、龙眼肉,煎浓汤送服,连服十余剂,共服生硫黄二两半(日服药一剂,头煎次煎约 
各送服生硫黄八分许),其痢始愈。 
此证脉微弱而沉,少阴之脉也,下者如烂炙兼脂膜,较下脓血为尤甚矣。使其初得下脓血时,投以桃 
花汤不即随手可愈乎?乃至病危已至极点,非桃花汤所能胜任,故仍本桃花汤之义,以硫黄代干姜(上 
焦有浮热者忌干姜不忌硫黄),用生山药、熟地黄、龙眼肉以代石脂(病入阴虚,石脂能固下不能滋阴,山 
药诸药能固下兼能滋阴),如此变通,仍不失桃花汤之本义,是以多服十余剂亦能奏效也。至此节之下节,下 
利不止,下脓血,又添腹痛,小便不利证,亦桃花汤主之。盖小便不利因寒者亦恒有之,故投以桃花汤亦能愈也。 
奉天石××,忽然小便不通。入西医院治疗,西医治以引溺管,小便通出。有顷,小便复存蓄若干。西 
医又纳以橡皮管,使久在其中,有溺即通出。乃初虽稍利,继则小便仍不能出,遂来院求为延医。其脉弦 
迟细弱;自言下焦疼甚且凉甚。知其小便因凉而凝滞也。为拟方用人参、椒目、怀牛膝各五钱,附子、肉 
桂、当归各三钱,干姜、小茴香、威灵仙、甘草、没药各二钱。连服三剂,腹疼及便闭皆愈。遂停汤药,俾日 
用生硫黄细末钱许分两次服下,以善其后。方中之义∶人参、灵仙并用,可治气虚小便不利;椒目、桂、附、 
干姜并用,可治因寒小便不利;又佐以当归、牛膝、茴香、没药、甘草诸药,或润而滑之,或引而下 
之,或馨香以通窍,或温通以开瘀,或和中以止疼,众药相济为功,所以奏效甚速也。观此小便不利及《伤 
寒论》少阴症桃花汤后所附下利脓血治验之案皆为寒证,非热证也明矣。 
【桃花汤方】赤石脂一斤(一半全用,一半筛末),干姜一两,粳米一升。 
上三味,以水七升,煮米令熟,去滓,温服七合,纳赤石脂末方寸匕,日三服,若一服愈,余勿服。 
石脂原为土质,其性微温,故善温养脾胃。为其具有土质, 
颇有粘涩之力,故又善治肠 下脓血。又因其生于两石相并之夹 
缝,原为山脉行气之处,其质虽粘涩,实兼能流通气血之瘀滞,故方中重用之以为主药。至于一半煎汤一半末 
服者,因凡治下利之药,丸散优于汤剂,且其性和平,虽重用一斤犹恐不能胜病,故又用一半筛其细末,纳 
汤药中服之也。且服其末,又善护肠中之膜,不至为脓血凝滞所伤损也。用干姜者,因此证其气血因寒 
而瘀,是以化为脓血,干姜之热既善祛寒,干姜之辛又善开瘀也。用粳米者,以其能和脾胃,兼能利小便,亦 
可为治下利不止者之辅佐品也。 

<目录>三、医论
<篇名>39.少阴病吴茱萸汤证
属性:《伤寒论》原文∶少阴病,吐利,手足厥冷,烦躁欲死者,吴茱萸汤主之。 
柯韵伯曰∶少阴病,吐利、烦躁、四逆者死。四逆者四肢厥冷兼臂、胫而言也,此云手足是指掌而言, 
四肢之阳犹在也。 
【吴茱萸汤】吴茱萸一升汤洗七遍,人参三两切,生姜六两切,大枣十二枚擘。 
上四味,以水七升,煮取二升,去滓,温服七合,日三服。 
吴茱萸汤之实用,乃肝胃同治之剂也。至于此证烦躁欲死,非必因肝邪盛极,实因寒邪阻塞而心肾不 
交也。盖人心肾之气,果分毫不交,其人即危不旋踵,至于烦躁欲死,其心肾几分毫不交矣。夫心肾之所 
以相交者,实赖脾胃之气上下通行,是以少阴他方中皆用干姜,而吴茱萸汤中则重用生姜至六两,取其 
温通之性,能升能降(生姜善发汗,是其能升,善止呕吐,是其能降),以开脾胃凝滞之寒邪,使脾 
胃之气上下通行,则心肾自能随脾胃气化之升降而息息相通矣。 

<目录>三、医论
<篇名>40.少阴病苦酒汤证
属性:《伤寒论》原文∶少阴病,咽中伤,生疮,不能语言,声不 
出者,苦酒汤主之。 
【苦酒汤】半夏洗破如枣核十四枚,鸡子一枚去黄,内上苦酒,着鸡子壳中。 
上两味,纳半夏,着苦酒中,以鸡子壳着刀环中,安火上,令三沸,去滓,少少含咽之,不瘥,更作三剂。 
按∶苦酒即醋也,又方中枣核当作枣仁,不然,破半夏如枣 
核大十四枚,即鸡子空壳亦不能容,况鸡子壳中犹有鸡子清与苦酒乎? 
古用半夏皆用生者,汤洗七次即用,此方中半夏宜用生半夏先破之,后用汤洗,始能洗出毒涎。 
唐容川曰∶此节所言生疮,即今之喉痈、喉蛾,肿塞不得出声,今有用刀针破之者,有用巴豆烧焦烙 
之者,皆是攻破之使不壅塞也。仲景用生半夏正是破之也,余亲见治重舌敷生半夏立即消破,即知咽喉肿闭 
亦能消而破之矣。且半夏为降痰要药,凡喉肿则痰塞,此仲景用半夏之妙。正是破之又能去痰,与后世刀针、 
巴豆等方较见精密,况兼蛋清之润,苦酒之泻,真妙法也。 

<目录>三、医论
<篇名>41.少阴病白通汤证及白通加猪胆汁汤证
属性:《伤寒论》原文∶少阴病,下利,白通汤主之。 
【白通汤方】葱白四茎,干姜一两,附子一枚生用去皮破八片。 
上三味,以水三升煮取一升,去滓,分温再服。 
下利固系少阴有寒,然实与脾胃及心脏有关,故方中用附子 
以暖肾,用干姜以暖脾胃,用葱白以通心肾之气,即引心君之火下济(天道下济而光明),以消肾中之寒也。 
《伤寒论》原文∶少阴病,下利脉微者,与白通汤。利不 
止,厥逆无脉,干呕烦者,白通加猪胆汁汤主之。服汤脉暴出者 
死,微续者生。 
【白通加猪胆汁汤方】葱白四茎,干姜一两,附子一枚生用去皮破八片,人尿五合,猪胆汁一合。 
以上三味,以水三升,煮取一升,去滓,纳胆汁、人尿,和令相得,分温再服。若无胆,亦可用。 
此节较前节所言之病为又重矣。而于白通汤中加人尿、猪胆汁,即可挽回者,此中原有精微之理在也。 
人尿原含有脏腑自然之生气,愚友毛××之侄病霍乱,六脉皆闭,两目已瞑,气息已无,舁诸床上,毛××以 
手掩其口鼻觉仿佛仍有呼吸,灌水少许,似犹知下咽。乃急用现接之童便,和朱砂细末数分灌之,须臾顿醒, 
则人尿之功效可知矣。至于猪胆汁,以人之生理推之,原少阳相火之所寄生,故其味甚苦,此与命门相火原 
有先后天之分,当此元阳衰微、命门相火将绝之时,而以后天助其先天,西人所谓脏器疗法也。且人尿与猪 
胆汁之性皆凉,加于热药之中以为引导,则寒凉凝聚之处自无格拒,此又从治之法也。 
其脉暴出者,提纲中以为不治,以其将脱之脉象已现也。而愚临证数十年,于屡次实验中,得一救脱 
之圣药,其功效远过于参 ,而自古至今未有发明,其善治脱者其药非他,即山萸肉一味大剂煎服也。盖无 
论上脱、下脱、阴脱、阳脱、奄奄一息,危在目前者,急用生净萸肉(药局中恒有将酒浸萸肉蒸熟者,用之无效) 
三两,急火煎浓汁一大碗,连连温饮之,其脱即止,脱回之后,再用萸肉二两,生怀山药一两,真野台参五 
钱煎汤一大碗,复徐徐温饮之,暴脱之证约皆可救愈。想此节所谓脉暴出者用之亦可愈也。 

<目录>三、医论
<篇名>42.少阴病真武汤证
属性:《伤寒论》原文∶少阴病,二、三日不已,至四、五日,腹痛, 
小便不利,四肢沉重疼痛,自下利者,此为有水气。其人或咳, 
或小便利,或下利,或呕者,真武汤主之。 
【真武汤方】茯苓三两,芍药三两,生姜三两切,白术二两,附子一枚炮去皮破八片。 
上五味,以水八升,煮取三升,去滓,温服七合,日三服。若咳者,加五味子半升,细辛、干姜各一两; 
若小便利者,去茯苓,若下利者,去芍药,加干姜二两;若呕者,去附子,加生姜,足前成半斤。 
罗东逸曰∶夫人一身制水者脾,主水者肾也。肾为胃关,聚水而从其类,倘肾中无阳,则脾之枢机虽运, 
而肾之关门不开,水即欲行以无主制,故泛溢妄行而有是证也。用附子之辛温壮肾之元阳,则水有所主矣。白术 
之温燥,创建中土,则水有所制矣。生姜之辛散,佐附子以补阳,于补水中寓散水之意。茯苓之渗淡,佐白 
术以建土,于制水中寓利水之道焉。而尤重在芍药之苦降,其旨甚微。盖人身阳根于阴,若徒以辛热补阳,不 
少佐以苦降之品,恐真阳飞越矣。芍药为春花之殿,交夏而枯,用之以极亟收散漫之阳气而归根。下利减芍药 
者,以其苦降涌泻也。加干姜者,以其温中胜寒也。水寒伤肺则咳,加细辛、干姜者,胜水寒也。加五 
味子者,收肺气也。小便利者,去茯苓,恐其过利伤肾也。呕者,去附子倍生姜,以其病非下焦,水停于胃, 
所以不须温肾以行水,只当温胃以散水,且生姜功能止呕也。 

<目录>三、医论
<篇名>43.少阴病通脉四逆汤证
属性:《伤寒论》原文∶少阴病,下利清谷,里寒外热,手足厥逆,脉微欲绝,身反不恶寒,其人面赤色, 
腹痛,或干呕,或咽痛,或利止脉不出者,通脉四逆汤主之。 
【通脉四逆汤】甘草二两炙,附子大者一枚生用去皮破八片,干姜三两。 
上三味,以水三升,煮取一升二合,去滓,分温再服,其脉即渐而出者愈(非若暴出者之自无而忽有、 
既有而仍无,如灯火之回焰也)。面赤色者,加葱九茎,腹中痛者,去葱,加芍药二两;呕者,加生姜二两; 
咽痛者,去芍药,加桔梗一两;利止脉不出者,去桔梗,加人参二两。 
太阳篇四逆汤中干姜两半,以治汗多亡阳之证。至通脉四逆汤药味同前,惟将干姜加倍,盖因寒盛 
脉闭,欲借辛热之力开凝寒以通脉也。面赤者加葱九茎(权用粗葱白切上九寸即可),盖面赤乃阴寒在 
下,逼阳上浮,即所谓戴阳证也。加葱以通其上下之气,且多用同于老阳之数,则阳可下归其宅矣。而愚 
遇此等证,又恒加芍药数钱,盖芍药与附子并用,最善收敛浮越之元阳下降也。 

<目录>三、医论
<篇名>44.少阴病大承气汤证
属性:《伤寒论》原文∶少阴病,自利清水,色纯青,心下必痛,口干燥者,急下之,宜大承气汤。 
此证乃伏气之热窜入肝肾二经也。盖以肾主闭藏,肝主疏泄,肾为二便之关,肝又为肾行气,兹因 
伏气之热,窜入肾兼窜入肝,则肝为热助疏泄之力太过,即为肾行气之力太过,致肾关失其闭藏之用,而下利 
清水。且因肝热而波及于胆,致胆汁因热妄行,随肝气之疏泄而下纯青色之水。于斯,肾水因疏泄太过而 
将竭,不能上济以镇心火,且肝木不得水气之涵濡,则在下既过于疏泄,在上益肆其横恣,是以心下作痛口 
中干燥也。此宜急下之,泻以止泻,则肾中之真阴可回,自能上济以愈口中干燥、心下作痛也。 
此节之前有“少阴病得之二,三日,口燥咽干者,急下之,宜大承气汤。”及后节“少阴病六、七日, 
腹胀不大便者,急下之,宜大承气汤。”想此二节,仲师亦皆言急下,若不急下,当亦若纯下青水 
者,其危险即在目前。若仲师者,宜其为医中之圣也。 
按∶方书有奇恒痢,张隐庵谓,系三阳并至,三阴莫当,九窍皆塞,阳气旁溢,咽干喉塞痛,并于阴 
则上下无常,薄为肠僻,其脉缓小迟涩,血温身热者死,热见七日者死。盖因阳气偏盛,阴气受伤,是以脉 
小迟涩,此证宜急用大承气汤泻阳养阴,缓则无效。夫奇恒痢病,未知所下者奚似,而第即其脉象缓小迟 
涩,固与少阴病之脉微细者同也。其咽干喉塞,痛并于阴,又与此节之心下痛、口中干燥者同也。隐庵谓宜急 
服大承气汤,又与此节之急下之宜大承气者同也。是奇恒痢者,不外少阴下利之范 
围,名之为奇恒痢可也,名之为少阴下利亦无不可也。 
《伤寒论》原文∶少阴病,下利,脉微涩,呕而汗出,必数更衣,反少者,当温其上,灸之。 
注家谓∶宜灸百会穴。 

<目录>三、医论
<篇名>45.厥阴病提纲及意义
属性:传经之次第,由少阴而厥阴。厥阴者,肝也,肝为厥阴之府,而肝膈之下垂,与包肾之脂膜相连者,即厥 
阴之经也。为其经与少阴经之脂膜相连,是以由少阴可传于厥阴。厥者逆也,又尽也,少阴自少阳、太 
阴传来,而复逆行上传于肝,且经中气化之相传至此,又复阴尽而阳生也,是以名为厥阴也。 
《伤寒论》原文∶厥阴之为病,消渴,气上撞心,心中疼热,饥而不欲食,食则吐蛔,下之利不止。 
《内经》谓∶“厥阴之上,风气主之,中见少阳。”少阳者,肝中所寄之少阳相火也。为肝中寄有相火, 
因外感之激发而暴动,是以消渴。相火挟肝气上冲,是以觉气上撞心,心中疼且热也。凡人之肝热者,胃中亦 
恒有热,胃中有热能化食,肝中有热又恒欲呕,是以饥而不欲食。至于肠中感风木兼少阳之气化, 
原能生蛔,因病后懒食,肠中空虚,蛔无所养,偶食少许,蛔闻食味则上来,是以吐蛔也。至误下之利不止 
者,因肝受外感正在不能疏泄之时(经谓肝主疏泄),适有降下之药为向导,遂至为肾过于行气 
(肝行肾之气)而疏泄不已。 

<目录>三、医论
<篇名>46.厥阴病乌梅丸证
属性:《伤寒论》原文∶伤寒脉微而厥,至七、八日,肤冷,其人躁无暂安时者,此为脏厥,非蛔厥也。蛔 
厥者,其人当吐蛔,今病者静,而复时烦者,此为脏寒,蛔上入膈,故烦,须臾复止,得食而 
呕,又烦者,蛔闻食臭出,其人当自吐蛔,蛔厥者,乌梅丸主之,又主久利。 
【乌梅丸方】乌梅三百枚,细辛六两,干姜十两,黄连一 
斤,当归四两,附子六两去皮炮,蜀椒四两炒出汗,人参六两,黄柏六两,桂枝六两去皮。 
上十味,异捣筛,合治之,以苦酒渍乌梅一宿,去核,蒸之五升米下,饭熟,捣成泥,和药令相得, 
内臼中,与蜜杵二千下,丸如梧桐子大,先食饮,服十丸,日三服,稍加至二十丸。禁生冷、滑物、臭食等。 
厥阴一篇,病理深邃,最难疏解。注家以经文中有阴阳之气,不相顺接之语,遂以经解经,于四肢 
之厥逆,即以阴阳之气不相顺接解之,而未有深究其不相顺接之故,何独在厥阴一经者。盖肝主疏泄,原为风木 
之脏,于时应春,实为发生之始。肝膈之下垂者,又与气海相连,故能宣通先天之元气,以敷布于周 
身,而周身之气化,遂无处不流通也。至肝为外感所侵,其疏泄之力顿失,致脏腑中之气化不能传达于外, 
是以内虽蕴有实热,而四肢反逆冷,此所谓阴阳之气不相顺接也。至于病多呕吐者, 
亦因其疏泄之力外无所泻,遂至蓄极而上冲胃口,此多呕吐之所 
以然也。又胃为肝冲激不已,土为木伤,中气易漓,是以间有除中之病。除中者,脾胃之气已伤尽,而危 
在目前也。至于下利亦未必皆因藏寒,其因伏气化热窜入肝经,遏抑肝气太过,能激动其疏泄之力上冲,亦 
可激动其疏泄之力下注以成下利,然所利者必觉热而不觉凉也。试举一治验之案以明之。 
辽宁刘××,寓居天津,年近四旬,于孟秋得吐泻证,六日之间勺饮不存,一昼夜间下利二十余次,病 
势危急莫支。延为延医,其脉象微细,重按又似弦长,四肢甚凉,周身肌肤亦近于凉,而心中则甚觉发热,所 
下利者亦觉发热,断为系厥阴温病,在《伤寒论》中即为厥阴伤寒(《伤寒论》开端处,曾提出温病,后则 
浑名之为伤寒)。惟其呕吐殊甚,无论何药,入口即吐出,分毫不能下咽,实足令医者束手耳。因问之曰,心 
中既如此发热,亦想冰吃否?答曰,想甚,但家中人驳阻不令食耳。愚曰,此病已近垂危,再如此吐泻一昼 
夜,即仙丹不能挽回,惟用冰膏搀生石膏细末服之,可以止吐,吐止后泻亦不难治矣。遂立主买冰淇淋若干, 
搀生石膏细末两许服之,服后病见愈,可服稀粥少许,下利亦见少。翌日复为诊视,四肢已不发凉,身亦 
微温,其脉大于从前,心中犹觉发热,有时仍复呕吐。俾再用生石膏细末一两,搀西瓜中服之,呕吐从 
此遂愈。翌日再诊其脉,热犹未清,心中虽不若从前之大热,犹思食凉物,懒于饮食,其下利较前已愈强半。 
遂为开白虎加人参汤,方中生石膏用二两,野台参三钱,用生杭芍六钱以代知母,生山药六钱以代粳米,甘草 
则多用至四钱,又加滑石六钱,方中如此加减替代者,实欲以之清热,又欲以之止利也。俾煎汤两 
盅,分两次温饮下,病遂全愈。此于厥阴温病如此治法,若在冬令,遇厥阴伤寒之有实热者,亦可如此治法。 
盖厥阴一经,于五行属木,其性原温,而有少阳相火寄生其间,则温而热矣。若再 
有伏气化热窜入,以激动其相火,原可成极热之病也。夫石膏与 
冰膏、西瓜并用,似近猛浪,然以愚之目见耳闻,因呕吐不止而废命者多矣,况此证又兼下利乎?此为救 
人之热肠所迫,于万难挽救之中,而拟此挽救之奇方,实不暇计其方之猛浪也。若无冰 
膏、西瓜时,或用鲜梨切片,蘸生石膏细末服之,当亦不难下咽而止呕吐也。 

<目录>三、医论
<篇名>47.厥阴病白虎汤证
属性:《伤寒论》原文∶伤寒脉滑而厥者,里有热,白虎汤主之。 
太阳篇白虎汤证,脉浮滑是表里皆有热也。此节之白虎汤证,脉滑而厥,是里有热表有寒也,此所谓 
热深厥深也。愚遇此等证,恒先用鲜白茅根半斤切碎,煮四五沸,取汤一大碗,温饮下,厥回身热,然后 
投以白虎汤,可免病家之疑,病患亦敢放胆服药。若无鲜茅根时,可以药局中干茅根四两代之。若不用茅根 
时,愚恒治以白虎加人参汤,盖取人参能助人生发之气,以宣通内热外出也。 

<目录>三、医论
<篇名>48.厥阴病当归四逆汤及加吴茱萸生姜汤证
属性:《伤寒论》原文∶手足厥寒,脉细欲绝者,当归四逆汤主 
之。若其人内有久寒者,宜当归四逆加吴茱萸生姜汤。 
沈尧封曰∶叔和释脉法,细极谓之微,即此之脉细欲绝,即与脉微相浑。不知微者,薄也,属阳气虚, 
细者小也,属阴血虚,薄者未必小,小者未必薄也。盖荣行脉中,阴血虚则实其中者少,脉故小;卫行脉外, 
阳气虚则约乎外者怯,脉故薄。况前人用微字,多取薄字意,试问∶“微云淡河汉”,薄乎?细乎? 
故少阴论中脉微欲绝,用通脉四逆主治回阳之剂也。此之脉细欲 
绝,用当归四逆主治补血之剂也。两脉阴阳各异,岂堪混释! 
【当归四逆汤方】当归三两,桂枝三两去皮,芍药三两,细 
辛三两,大枣二十五枚擘,甘草二两炙,通草二两。 
上七味,以水八升,煮取三升,去滓,温服一升,日三服。 
【当归四逆加吴茱萸生姜汤方】即前方加吴茱萸二升,生姜半斤切,以水六升、清酒六升,和煮取 
五升,去滓,分温五服。 
王和安曰∶厥阴经气来自足少阴经,宣于手太阴经,成循环不息之常度。若以血寒自郁于脏,脉象应有 
弦凝之征。今脉细欲绝,可知少阴经气来源先虚,及复本经受脏寒之感,则虚寒转甚,细而欲绝也。治以当归 
四逆汤,意在温肝通郁,而必以桂枝、白芍疏浚经气之源,细辛、通草畅达经气之流,内有凝寒,重加吴 
萸、生姜,温经通气,仍加入原方以全其用,解此,则治经气之定义可三反矣。 

<目录>三、医论
<篇名>49.厥阴病白头翁汤证
属性:《伤寒论》原文∶热利下重者,白头翁汤主之。 
【白头翁汤方】白头翁二两,黄连、黄柏、秦皮各三两。 
上四味,以水七升,煮取二升,去滓,温服一升,不愈更服一升。 
《医宗金鉴》注曰∶三阴俱有下利证,自利不渴属太阴,自利渴属少阴。惟厥阴下利,属寒者厥而不渴, 
下利清谷;属热者消渴,下利后重,便利脓血。此热利下重,乃郁热奔逼广肠、魄门重滞难 
出。初痢用此法以寒治热,久痢则宜用乌梅丸,随所利而从治之,调其气使之平也。 
白头翁汤所主之热利下重,当自少阴传来,不然则为伏气化热窜入厥阴,其证虽热,而仍非外感大实之 
热,故白头翁汤可以胜任。乃有病在阳明之时,其病一半入府,一半由经而传于少阳,即由少阳入厥阴而为腑 
脏之相传。则在厥阴者既可成厥阴热利之下重,而阳明府中稽留之热,更与之相助而为虐,此非但用白头 
翁汤所能胜任矣。愚遇此等证,恒将白头翁、秦皮加于白虎加人 
参汤中,则莫不随手奏效也。 
曾治一中年妇人,于孟春感冒风寒,四、五日间延为延医。其左脉弦而有力,右脉洪而有力,舌苔白而 
微黄,心中热而且渴,下利脓血相杂,里急后重,一昼夜二十余次,即其左右之脉象论之,断为阳明、厥阴合 
并病。有一医者在座,疑而问曰∶凡病涉厥阴,手足多厥逆,此证则手足甚温何也?答曰∶此其所以与阳明 
并病也,阳明主肌肉,阳明府中有热,是以周身皆热,而四肢之厥逆,自不能于周身皆热时外现也。况厥阴之 
病,即非杂以阳明,亦未必四肢皆厥逆乎?医者深韪愚言,与病家皆求速为疏方,遂为立方如下∶ 
生石膏(三两捣细) 生杭芍(八钱) 生怀山药(八钱) 野台参(四两) 
白头翁(八钱) 秦皮(六钱) 天花粉(八钱) 甘草(三钱) 
上药八味,共煎三盅,分三次温饮下。 
方中之义,是合白虎加人参汤与白头翁汤为一方,而又因证加他药也。白虎汤中无知母者,方中芍药可 
代知母也。盖芍药既能若知母之退热滋阴,而又善治下利者之后重也。无粳米者,方中生山药可代粳米也,盖 
山药汁浆浓郁,既可代粳米和胃,而其温补之性,又能助人参固下也,至于白头翁汤中无黄连、黄柏者, 
因与白虎汤并用,有石膏之寒凉,可省去连、柏也。又外加天花粉者,因其病兼渴,天花粉偕同人参最善 
生津止渴。将此药三次服完,诸病皆减三分之二。再诊其脉仍有实热未清,遂于原方中加滑石五钱,利其小 
便,正所以止其大便,俾仍如从前煎服,于服汤药之外,又用鲜白茅根半斤煎汤当茶,病遂全愈。 

<目录>三、医论
<篇名>50.不分经之病理中丸证、竹叶石膏汤证
属性:伤寒病六经分治之外,又有不分经之病,附载于伤寒分经之 
后者,又宜择其紧要者,详为诠解,而后学治伤寒者,自能应变 
无穷也。 
《伤寒论》原文∶大病瘥后,喜唾,久不了了者,胸上有寒,当以丸药温之,宜理中丸。 
【理中丸方】人参、甘草、白术、干姜各三两。 
上四味,捣筛为末,蜜丸如鸡子黄大,以沸汤数合,和一丸,研碎,温服之,日三服,夜二服,腹 
中未热,益至三、四丸,然不及汤。汤法以四物根据两数切,用水八升,煮取三升,去滓,温服一升,日三服。 
附加减法∶若脐上筑者,肾气动也,去术,加桂四两,吐多者,去术,加生姜三两;下多者,还用术;悸 
者,加茯苓二两;渴欲饮水者,加术,足前成四两半;腹中疼者,加人参,足前成四两半;寒者,加干姜, 
足前成四两半;腹满者,去术,加附子一枚。服汤后如食顷,饮热粥一升许,微自温,勿发揭衣被。 
此病时服凉药太过,伤其胃中之阳,致胃阳虚损不能运化脾脏之湿,是以痰饮上溢而喜唾,久不了了 
也。故方中用人参以回胃中之阳,其补益之力,且能助胃之 动加数,自能运化脾中之湿使之下行。而又 
辅以白术,能健脾又能渗湿。干姜以能暖胃又能助相火以生土。且又加甘草以调和诸药,使药力之猛者,得甘 
草之缓而猛力悉化,使药性之热者,得甘草之甘而热力愈长也。至于方后诸多加减,又皆各具精义,随诸证 
之变化,而遵其加减诸法,用之自能奏效无误也。 
《伤寒论》原文∶伤寒解后,虚羸少气,气逆欲吐者,竹叶石膏汤主之。 
【竹叶石膏汤方】竹叶二把,石膏一斤,半夏半升洗,麦门冬一升,人参三两,甘草二两炙,粳米半升。 
上七味,以水一斗,煮取六升,去滓,纳粳米,煮米熟汤成去米,温服一升,日三服。 
前节是病时过用凉药,伤其阳分;此节是病时不能急用凉药以清外感之热,致耗阴分。且其大热虽退, 
仍有余热未清,是以虚羸少气,气逆欲吐,此乃阴虚不能恋阳之象,又兼有外感之余热为之助虐也。故方 
中用竹叶、石膏以清外感之热,又加人参、麦冬协同石膏以滋阴分之亏,盖石膏与人参并用,原有化合 
之妙,能于余热未清之际立复真阴也。用半夏者,降逆气以止吐也。用甘草、粳米者,调和胃气以缓石药下 
侵也。自常情观之,伤寒解后之余热,何必重用石膏,以生地、玄参、天冬、麦冬诸药,亦 
可胜任,然而甘寒留邪,可默酿痨瘵之基础,此又不可不知也。 

<目录>三、医论
<篇名>51.温病遗方
属性:《伤寒论》中原有温病,浑同于六经分篇之中,均名之为伤寒,未尝明指为温病也。况温病之原因各 
殊,或为风温,或为湿温,或为伏气成温,或为温热,受病之因既不同,治法即宜随证各异。有谓温病入手经 
不入足经者,有谓当分上中下三焦施治者,皆非确当之论,斟酌再四,惟仍按《伤寒论》六经分治乃为近是。 
太阳经 
有未觉感冒,身体忽然酸软,懒于动作,头不疼,肌肤不热,似稍畏风,舌似无苔而色白,脉象微浮, 
至数如常者,此乃受风甚轻,是以受时不觉也,宜用轻清辛凉之剂发之。 
【处方】薄荷叶三钱,连翘三钱,大葱白三寸。 
上药三味,共煎汤七、八沸,取清汤一大盅温服下,周身得汗即愈。 
薄荷之成分,含有薄荷脑,辛凉芬芳,最善透窍,内而脏腑,外而皮毛,凡有风邪匿藏,皆能逐之外出, 
惟其性凉,故于感受温风者最宜。惟煮汤服之,宜取其轻清之气,不宜过煎 
(过煎即不能发汗),是以以之煎汤,只宜七八沸。若与难煎之药同煎,后入可也。连翘为轻清宣散之品, 
其发汗之力不及薄荷,然与薄荷同用,能使薄荷发汗之力悠长(曾治一少年受感冒,俾单用连翘一两,煮两汤服之, 
终宵微汗不竭,病遂愈,其发汗之力和缓兼悠长可知)。葱之形中空,其味微辣微甘,原微具发表之性,以旋 
转于营卫之间,故最能助发表之药以调和营卫也。 
有受风较重,不但酸软懒动,且觉头疼,周身骨节皆疼,肌肤热,不畏风,心中亦微觉发热,脉象浮数似 
有力,舌苔白浓,宜于前方中去葱白,加天花粉八钱以清热,加菊花二钱以治头疼,惟煎汤时薄荷宜后入。 
有其人预有伏气化热,潜伏未动,后因薄受外感之触动,其伏气陡然勃发,一时表里俱热,其 
舌苔白浓,中心似干,脉象浮而有洪象,此其病虽连阳明而仍可由太阳汗解也。 
【处方】生石膏一两捣细,天花粉一两,薄荷叶钱半,连翘钱半。 
上药四味,煎汤一大盅,温服得汗即愈,薄荷叶煎时宜后入。 
或问∶此方重用石膏、花粉,少用薄荷、连翘,以为发表之剂,特恐石膏、花粉监制薄荷、连翘太过,服 
后不能作汗耳。答曰∶此方虽为发表之剂,实乃调剂阴阳,听其自汗,而非强发其汗也。盖此证原为伏 
气化热,偶为外感触动,遂欲达于表而外出,而重用凉药与之化合,犹如水沃冶红之铁,其蓬勃四达之热 
气原难遏抑。而复少用薄荷、连翘,为之解其外表之阻隔,则腹中所化之热气,自夺门而出作汗而解矣。且 
此等汗,原不可设法为之息止,虽如水流漓而断无亡阴、亡阳之虞,亦断无汗后不解之虞。此方原与拙拟寒 
解汤相似。二方任用其一,果能证脉无误,服后复杯之顷,即可全身得汗。间有畏石膏之凉,将其药先服一半 
者,服后亦可得汗,后再服其所余,则分毫无汗矣。因其热已化 
汗而出,所余之热无多也。即此之前后分服,或出汗或不出汗, 
可不深悟此药发汗之理乎?况石膏原具有发表之力也。 
有其人身体酸懒,且甚觉沉重,头重懒抬,足重懒举,或周身肌肤重按移时,微似有痕,或小便不利,其 
舌苔白而发腻,微带灰色,其脉浮而濡,至数如常者,此湿温也。其人或久居潮湿之地,脏腑为湿气所侵,或 
值阴雨连旬,空气之中含水分过度,或因饮食不慎,伤其脾胃,湿郁中焦,又复感受风邪,遂成斯证, 
宜用药外解其表,内利其湿则病愈矣。 
【处方】薄荷叶三钱,连翘三钱,小苍术三钱,黄芩三钱,木通二钱。 
上药五味,先将后四味水煎十余沸,再入薄荷煎七、八沸,取清汤一大盅,温服之。若小便不利者,于 
用药之外,用鲜白茅根六两,去皮切碎,水煎四、五沸,取其清汤以之当茶,渴则饮之。 
若其人肌肤发热,心中亦微觉热者,宜去苍术加滑石八钱。 
有温病初得作喘者,其肌肤不恶寒而发热,心中亦微觉发热,脉象浮而长者,此乃肺中先有痰火,又 
为风邪所袭也。宜用《伤寒论》麻杏甘石汤,而更定其分量之轻重。 
【更定麻杏甘石汤方】生石膏一两捣细,麻黄一钱,杏仁二钱去皮,甘草钱半。 
上四味,共煎汤一大盅(不先煎麻黄吹去浮沫者,因所用只一钱,而又重用生石膏以监制之也)温服。 
若服后过点半钟,汗不出者,宜服西药阿斯匹林一瓦。若不 
出汗,仍宜再服,以服至出汗为度。盖风邪由皮毛而入,仍使之由皮毛而出也。 
有温病旬日不解,其舌苔仍白,脉仍浮者,此邪入太阳之府也,其小便必发黄。宜于发表清热药中, 
加清膀胱之药,此分解法也。今拟二方于下,以便用者相热之轻重而自斟酌用之。 
【处方】滑石一两,连翘三钱,蝉蜕去土足三钱,地肤子三 
钱,甘草二钱。 
上药五味,共煎一大盅,温服。 
【又方】生石膏捣细一两,滑石八钱,连翘三钱,蝉蜕去土足三钱,地肤子三钱,甘草二钱。 
上药六味,共煎汤一大盅,温服。 
有温病至七、八日,六经已周,其脉忽然浮起,至数不数,且有大意者,宜用辛凉之剂助之达表而汗解。 
【处方】玄参一两,寸麦冬带心五钱,连翘二钱,菊花二钱,蝉蜕去土足二钱。 
上药五味,共煎汤一大盅,温服。用玄参者,恐温病日久伤阴分也。 
有温病多日,六经已周,脉象浮数而细,关前之浮尤甚,其头目昏沉,恒作 语,四肢且有扰动不安 
之意,此乃外感重还太阳欲作汗也。其所欲汗而不汗者,因阴分太亏,不能上济以应阳也。此证若因脉浮而 
强发其汗,必凶危立见,宜用大滋真阴之品,连服数剂,俾脉之数者渐缓,脉之细者渐大,迨阴气充长, 
能上升以应其阳,则汗自出矣。 
【处方】生地黄一两,生怀山药一两,玄参一两,大甘枸杞一两,生净萸肉六钱,柏子仁六钱,生 
枣仁六钱捣碎,甘草三钱。 
上药八味,水煎一大碗,候五分钟,调入生鸡子黄二枚,徐徐温饮之,饮完一剂再煎一剂,使昼夜药 
力相继不断,三剂之后,当能自汗。若至其时,汗仍不出者,其脉不似从前之数细,可仍煎此药送服西药阿 
斯匹林一瓦,其汗即出矣。 
或问∶山萸肉原具酸敛之性,先生所定来复汤尝重用之以治汗出不止,此方原欲病者服之易于出汗,何方 
中亦用之乎?答曰∶此中理甚精微,当详细言之。萸肉为养肝熄风之要药,此证四肢 
之骚扰不安,其肝风固已动也,此方中用萸肉之本意也。若虑用 
之有妨于出汗,是犹未知萸肉之性。盖萸肉之味至酸,原得木气最全,是以酸敛之中,大具条畅之性, 
《神农本草经》谓其逐寒湿痹是明征也。为其味酸敛也,故遇元气不能固摄者,用之原可止汗;为其性条 
畅也,遇肝虚不能疏泄者,用之又善出汗。如此以用萸肉,是皆得之临证实验之余,非但凭诸理想而云然也。 
若果服药数剂后,其脉渐有起色,四肢不复扰动,即去萸肉亦无妨,其开始服药时,萸肉则断不能去也。 
有未病之先,心中常常发热,后为外感触发,则其热益甚,五心烦躁,头目昏沉,其舌苔白浓,且 
生芒刺,其口中似有辣味,其脉浮数有力者,此伏气化热已入心包,而又为外感束其外表,则内蕴之热益甚, 
是以舌有芒刺且觉发辣也。宜用凉润清散之剂,内清外解,遍体得透汗则愈矣。 
【处方】鲜地黄一两,玄参一两,天花粉一两,知母五钱,寸麦冬带心五钱,西药阿斯匹林两瓦。 
上药先煎前五味,取清汤两大盅,先温服一大盅,送服阿斯匹林一瓦。若服一次后汗未出,热亦未消者, 
可再温服一盅,送服阿斯匹林一瓦。若汗已出热未尽消者,药汤可如前服法,阿斯匹林宜斟酌少服。 

<目录>三、医论
<篇名>52.伤寒风温始终皆宜汗解说
属性:伤寒初得宜用热药发其汗,麻黄、桂枝诸汤是也。风温初得宜用凉药发其汗,薄荷、连翘、蝉蜕诸药 
是也。至传经已深,阳明热实,无论伤寒、风温,皆宜治以白虎汤。而愚用白虎汤时,恒加薄荷少许,或连 
翘、蝉蜕少许,往往服后即可得汗。即但用白虎汤,亦恒有服后即汗者。因方中石膏原有解肌发表之力,故 
其方不但治阳明府病,兼能治阳明经病,况又少加辛凉之品引之,以由经达表,其得汗自易易也。拙 
拟寒解汤后载有医案可参 
阅。该方原治寒温证周身壮热,心中热而且渴,舌苔白而欲黄,其脉洪滑或兼浮,或头犹觉疼,或周身犹有 
拘束之意者。果如方下所注证脉,服之复杯可汗,勿庸虑其不效也。盖脉象洪滑,阳明府热已实,原是 
白虎汤证。至洪滑兼浮,舌苔犹白,是仍有些些表证未罢。故方中重用石膏、知母以清胃府之热,复少用 
连翘、蝉蜕之善达表者,引胃中化而欲散之热仍还于表,作汗而解。斯乃调剂阴阳,听其自汗,非强发其汗也。 
至其人气体弱者,可用补气之药助之出汗。寒解汤加潞党参即可(寒解汤下载有治一叟年七旬,素有劳疾, 
薄受外感即发喘逆一案可参阅)。 
若阴分虚损者,可用滋阴之药助之出汗。若熟地、玄参、生 
山药、枸杞之类大润之剂峻补真阴,济阴以应其阳,设病有还表之机,必汗出而愈。 
至其人阳分阴分俱虚,又宜并补其阴阳以助之出汗。张景岳曾治一叟得伤寒证,战而不汗。于其翌日 
发战之时,投以大剂八味地黄汤,须臾战而得汗。继因汗多亡阳,身冷汗犹不止,仍投以原汤,汗止病亦遂 
愈。用其药发汗,即用其药止汗,是能运用古方入于化境者也。 
至少阳证为寒热往来,其证介于表里之间,宜和解不宜发汗矣。然愚对于此证,其热盛于寒者,多因 
证兼阳明,恒于小柴胡汤中加玄参八钱,以润阳明之燥热。其阳明之燥热化而欲散,自能还于太阳而作汗, 
少阳之邪亦可随汗而解。其寒盛于热者,或因误服降下药虚其气分,或因其气分本素虚,虽服小柴胡汤不能 
提其邪透膈上出,又恒于小柴胡汤中加薄荷叶二钱,由足少阳引入手少阳,借径于游部(手足少阳合为游部) 
作汗而解。此即《伤寒论》所谓“柴胡证具,而以他药下之,柴胡证仍在者,复与小柴胡汤,必 
蒸蒸而振,却发热汗出而解也。”然助以薄荷则出汗较易,即由汗解不必蒸蒸而振,致有战汗之状也。 
至于当用承气之证,却非可发汗之证矣。然愚临证经验以来,恒有投以三承气汤,大便犹未降下而即得 
汗者。盖因胃府之实热既为承气冲开,其病机自外越也。若降之前未尝得汗,既降之后亦必于饮食之时屡 
次些些得汗,始能脉净身凉。若降后分毫无汗,其热必不能尽消,又宜投以竹叶石膏汤,或白虎加人参 
汤,将其余热消解将尽,其人亦必些些汗出也。此所谓伤寒、风温始终皆宜汗解也。 

<目录>三、医论
<篇名>53.论冬伤于寒春必病温及冬不藏精春必病温治法
属性:尝读《内经》有“冬伤于寒,春必病温”之语,此中原有深义,非浅学人所易窥测也。乃笃信西说者, 
据病菌潜伏各有定期之说,谓病菌传于人身,未有至一月而始发动者,况数月乎?因此一倡百和,遂谓《内 
经》皆荒渺之谈,分毫不足凭信。不知毒瓦斯之传染有菌,而冬令严寒之气,为寒水司天之正气,特其气严寒 
过甚,或人之居处衣服欠暖,或冒霜雪而出外营生,即不能御此气耳。是以寒气之中人也,其重者实时成病, 
即冬令之伤寒也。其轻者,微受寒侵,不能即病,由皮肤内侵,潜伏于三焦脂膜之中,阻塞气化之升降流通, 
即能暗生内热。迨至内热积而益深,又兼春回阳生触发其热,或更薄受外感以激发其热,是以其热自 
内暴发而成温病,即后世方书所谓伏气成温也。 
至于治之之法,有清一代名医多有谓此证不宜发汗者。然仍宜即脉证之现象而详为区别。若其脉象虽 
有实热,而仍在浮分,且头疼、舌苔犹白者,仍当投以汗解之剂。然宜以辛凉发汗,若薄荷叶、连翘、蝉蜕诸 
药,且更以清热之药佐之。若拙拟之清解汤。凉解汤、寒解汤三方,斟酌病之轻重,皆可选用也。此乃先 
有伏气又薄受外感之温病也。 
若其病初得即表里壮热,脉象洪实,其舌苔或白而欲黄者,宜 
投以白虎汤,再加宣散之品若连翘、茅根诸药。如此治法,非取汗解,然恒服药后竟自汗而解。即或服药后不 
见汗,其病亦解。因大队寒凉之品与清轻宣散之品相并,自能排逐内蕴之热,息息自腠理达于皮毛以透出也( 
此乃伏气暴发自内达外之温病春夏之交多有之)。盖此等证皆以先有伏气,至春深萌动欲发,而又或因暴怒, 
或因劳心劳力过度,或因作苦于烈日之中,或因酣眠于暖室内,是以一发表里即壮热。治之者,只可宣散清 
解,而不宜发汗也。此冬伤于寒春必病温之大略治法也。 
《内经》又谓∶“冬不藏精,春必病温。”此二语不但为西医所指摘,即中医对此节经文亦恒有疑意。 
谓冬不藏精之人,若因肾虚而寒入肾中,当即成少阴伤寒,为直中真阴之剧证,何能迟至春令而始成温病? 
不知此二句经文原有两解,其所成之温病亦有两种,至其治法又皆与寻常治法不同。今试析言之,并详其治法。 
冬不藏精之人,其所患之温病,有因猝然感冒而成者。大凡病温之人,多系内有蕴热,至春阳萌动之时, 
又薄受外感拘束,其热即陡发而成温。冬不藏精之人,必有阴虚,所生之热积于脏腑,而其为外感所拘束而发 
动,与内蕴实热者同也。其发动之后,脉象多数,息多微喘,舌上微有白苔,津液短少,后或干黄,或 
舌苔渐黑,状如斑点(为舌苔甚薄若有若无故见舌皮变黑),或频饮水不能解渴,或时入阴分益加潮热。此 
证初得其舌苔白时,亦可汗解,然须以大滋真阴之药辅之。愚治此证,恒用连翘、薄荷叶各三钱,玄参、 
生地黄各一两,煎汤服之,得汗即愈。若服药后汗欲出仍不能出,可用白糖水送服西药阿斯匹林二分许, 
其汗即出。或单将玄参、生地黄煎汤,送服阿斯匹林一瓦,亦能得汗。若至热已传里,舌苔欲黄,或至黄而兼 
黑,脉象数而有力,然按之弦硬,非若阳明有实热者之洪滑,此阴虚热实之象,宜治以白虎加人参汤,更以生 
地黄代知母,生山药代粳米,煎一大剂,取汤一大碗,分多次温饮下(拙着伤寒温病同用方后载有此方, 
附载治愈之案若干。可参观也 )。 
又有因伏气所化之热先伏藏于三焦脂膜之中,迨至感春阳萌动而触发,其发动之后,恒因冬不藏精者 
其肾脏虚损,伏气乘虚而窜入少阴。其为病状∶精神短少,喜偃卧,昏昏似睡,舌皮干,毫无苔,小便短赤,其 
热郁于中而肌肤却无甚热。其在冬令,为少阴伤寒,即少阴证,初得宜治以黄连阿胶汤者也。在春令,即为少 
阴温病。而愚治此证,恒用白虎加人参汤,以生地黄代知母,生怀山药代粳米,更先用鲜白茅根三两煎汤以 
之代水煎药,将药煎一大剂,取汤一大碗,分三次温饮下,每饮一次调人生鸡子黄一枚。初饮一次后,其脉 
当见大,或变为洪大,饮至三次后,其脉又复和平,而病即愈矣。此即冬不藏精春必病温者之大略治法也。 
上所论各种温病治法,原非凭空拟议也,实临证屡用有效,而后敢公诸医界同人也。 
有温病初得即表里大热,宜治以白虎汤或白虎加人参汤者。其证发现恒在长夏,或在秋夏之交。而愚 
生平所遇此等证,大抵在烈日之中,或田间作苦,或长途劳役,此《伤寒论》所谓 病也,亦可谓之暑温也。 
其脉洪滑有力者,宜用白虎汤。若脉虽洪大而按之不实者,宜用白虎加人参汤。又皆宜煎一大剂,分数次 
温饮下,皆可随手奏效。 
伏气化热成温病者,大抵因复略有感冒,而后其所化之热可陡然成温,表里俱觉壮热。不然者,虽伏 
气所化之热深入阳明之府,而无外感束其表,究不能激发其肌肉之热。是以治之者恒不 
知其为伏气化热,放胆投以治温病之重剂,是以其热遂永留胃府致生他病。今试举一案以明之∶ 
天津刘××,于壬申正月上旬,觉心中时时发热,而周身又甚畏冷。时愚回籍,因延他医延医,服药 
二十余剂,病转增剧,二便皆闭。再服他药,亦皆吐出。少进饮食,亦恒吐出。此际愚 
适来津,诊其脉,弦长有力,然在沉分。知其有伏气化热,其热 
不能外达于表,是以心中热而外畏冷,此亦热深厥深之象也。俾先用鲜茅根半斤切碎,水煮三四沸,视茅根 
皆沉水底,其汤即成。取清汤三杯,分三次服,每服一次,将土狗三个捣为末,生赭石三钱亦为细末,以茅 
根汤送下。若服过两次未吐,至三次赭石可以不用。及将药服后,呕吐即止,小便继亦通下。再诊其 
脉,变为洪长有力,其心中仍觉发热,外表则不畏冷矣。其大便到此已半月未通下。遂俾用大潞参五钱煎汤, 
送服生石膏细末一两。翌晨大便下燥粪数枚,黑而且硬。再诊其脉,力稍缓,知心中犹觉发热。又俾用潞党 
参四钱煎汤,送服生石膏细末八钱。翌晨又下燥粪二十余枚,仍未见溏粪。其心中不甚觉热,脉象仍似 
有力,又俾用潞党参三钱煎汤,送服生石膏细末六钱。又下燥粪十余枚,后则继为溏粪,病亦从此全愈矣。 
盖凡伏气化热窜入胃府,非重用石膏不解,《伤寒论》白虎汤原为治此证之的方也。然用白虎汤之例,汗 
吐下后皆加人参,以其虚也。而此证病已数旬,且频呕吐,其元气之虚可知,故以人参煎汤送石膏,此亦仿 
白虎加人参汤之义也。至石膏必为末送服者,以其凉而重坠之性善通大便,且较水煮但饮其清汤者,其退热 
之力又增数倍也。是以凡伏气化热,其积久所生之病,有成肺病者,有成喉病者,有生眼疾者,有患齿疼者, 
有病下痢者,有病腹疼者(即盲肠炎),其种种病因若皆由于伏气化热,恒有用一切凉药其病皆不能愈,而 
投以白虎汤或投以白虎加人参汤,再因证加减,辅以各病当用之药,未有不随手奏效者。此治伏气化热之大 
略也。至于拙着全书中,所载伏气化热之病甚多,其治法亦各稍有不同,皆可参观。 

<目录>三、医论
<篇名>54.温病之治法详于伤寒论解
属性:伤寒、温病之治法始异而终同。至其病之所受,则皆在于足 
经而兼及于手经。乃今之论寒温者,恒谓伤寒入足经不入手经,温 
病入手经不入足经。夫人之手足十二经原相贯通,谓伤寒入足经不入手经者,固为差谬,至谓温病入手经 
不入足经者,尤属荒唐。何以言之?《伤寒论》之开始也,其第一节浑言太阳之为病,此太阳实总括中风、伤 
寒、温病在内,故其下将太阳病平分为三项,其第二节论太阳中风,第三节论太阳伤寒(四节五节亦论伤寒 
当归纳于第三节中),第六节论太阳温病,故每节之首皆冠以太阳病三字。此太阳为手太阳乎?抑为足太 
阳乎?此固无容置辩者也。由斯知,中风、伤寒、温病皆可以伤寒统之(《难经》谓伤寒有五中风温病皆在其中), 
而其病之初得皆在足太阳经,又可浑以太阳病统之也。盖所谓太阳之为病者,若在中风、伤寒,其头痛、项 
强、恶寒三证可以并见。若在温病,但微恶寒即可为太阳病(此所谓证不必具但见一证即可定为某经病也), 
然恶寒须臾即变为热耳。 
曾治一人,于季春夜眠之时因衾薄冻醒,遂觉周身恶寒,至前午十句钟表里皆觉大热,脉象浮洪,投 
以拙拟凉解汤一汗而愈。又尝治一人,于初夏晨出被雨,遂觉头疼周身恶寒,至下午一句钟即变为大热,渴 
嗜饮水,脉象洪滑,投以拙拟寒解汤亦一汗而愈。至如此凉药而所以能发汗者,为其内蕴之燥热与凉润之药化 
合,自然能发汗,又少用达表之品以为之引导,故其得汗甚速,汗后热亦尽消也。此二则,皆温病也,以其 
初得犹须臾恶寒,故仍可以太阳病统之。即其化热之后病兼阳明,然亦必先入足阳明,迨至由胃及 
肠,大便燥结,而后传入手阳明,安得谓温病入手经不入足经乎。 
由斯知,《伤寒论》一书,原以中风、伤寒、温病平分三项,特于太阳首篇详悉言之,以示人以入手 
之正路。至后论治法之处,则三项中一切诸证皆可浑统于六经,但言某经所现之某种病宜治以某方,不复别 
其为中风、伤寒、温病,此乃纳繁于简之法,亦即提纲挚领之法也。所尤当知者,诸节中偶明言中风者, 
是确指中风而言。若明言为伤寒者,又恒统中风、温病而言。以 
伤寒二字为三项之总称,其或为中风,或为伤寒,或为温病,恒 
于论脉之处有所区别也。至于六经分编之中,其方之宜于温病者不胜举,今将其显然可见者约略陈之于下。 
一为麻杏甘石汤。其方原治汗出而喘无大热者。以治温病,不必有汗与喘之兼证也,但其外表未解,内 
有蕴热者即可用。然用时须斟酌其热之轻重,热之轻者,麻黄宜用钱半,生石膏宜用六钱。若 
热之重者,麻黄宜用一钱,生石膏宜用一两。至愚用此方时,又恒以薄荷叶代麻黄(薄荷叶代麻黄时其分量 
宜加倍),服后得微汗,其病即愈。盖薄荷叶原为温病解表最良之药,而当仲师时犹未列于药品,故当日不用也。 
一为大青龙汤。《伤寒论》中用大青龙汤者有二节。一为第三十七节。其节明言太阳中风脉浮紧。夫 
《伤寒论》首节论太阳之脉曰浮,原统中风、伤寒而言。至第二节则言脉缓者为中风,是其脉为浮中之缓也, 
第三节则言脉阴阳俱紧者为伤寒,是其脉为浮中之紧也。今既明言中风,其脉不为浮缓而为浮紧,是中风 
病中现有伤寒之脉,其所中者当为凛冽之寒风,而于温病无涉也。一为第三十八节。细审本节之文,知其 
确系温病。何以言之?以脉浮缓、身不疼、但重、无少阴证也。盖此节开端虽明言伤寒,仍是以伤寒二字为 
中风、伤寒、温病之总称。是以伤寒初得脉浮紧,温病初得脉浮缓。伤寒初得身多疼,温病初得身恒不 
疼而但重(《伤寒论》第六节温病提纲中原明言身重)。伤寒初得恒有少阴证,温病则始终 
无少阴证(少阴证有寒有热,此指少阴之寒证言,为无少阴寒证,所以敢用大青龙汤,若少阴热证温 
病中恒有之,正不妨用大青龙汤矣)。此数者皆为温病之明征也。况其病乍有轻时,若在伤寒必不复重用 
生石膏,惟系温病则仍可重用生石膏如鸡子大,约有今之四两,因温病当以清燥热救真阴为急务也。至愚用此 
方时,又恒以连翘代桂枝。虽桂枝、连翘均能逐肌肉之外感,而一则性热,一则性凉,温病宜凉不宜热,故用 
桂枝不如用连翘。而当日仲师不用者,亦因其未列入药品也(《伤寒论》方中所用之连轺是连翘根能利水不能发 
汗)。况大青龙汤中桂枝 
之分量,仅为麻黄三分之一,仲师原因其性热不欲多用也。 
一为小青龙汤。其方外能解表,内能涤饮,以治外感痰喘诚有奇效,中风、伤寒、温病皆可用。然宜酌 
加生石膏,以调麻、桂、姜、辛之热方效。是以《伤寒论》小青龙汤无加石膏之例,而《金匮》有小青龙 
加石膏汤,所以补《伤寒论》之未备也。至愚用此汤时,遇挟有实热者,又恒加生石膏至一两强也。 
一为小柴胡汤。其方中风、伤寒病皆可用。而温病中小柴胡汤证,多兼呕吐粘涎,此少阳之火与太 
阴之湿化合而成也(少阳传经之去路为太阴)。宜于方中酌加生石膏数钱或两许,以清少阳之火,其粘涎 
自能化水从小便中出。夫柴胡既能引邪上出,石膏更能逐热下 
降,如此上下分消,故服药后无事汗解,即霍然全愈也。 
以上所述诸方,大抵宜于温病初得者也。至温病传经已深,若清燥热之白虎汤、白虎加人参汤, 
通肠结之大小承气汤,开胸结之大、小陷胸汤,治下利之白头翁汤、黄芩汤,治发黄之茵陈栀子柏皮等汤, 
及一切凉润清火育阴安神之剂,皆可用于温病者,又无庸愚之赘语也。 
至于伏气之成温者,若《内经》所谓“冬伤于寒,春必病温”、“冬不藏精,春必病温”之类,《伤寒 
论》中非无其证,特其证现于某经,即与某经之本病无所区别。仲师未尝显为指示,在后世原难明辨。且其治 
法与各经之本病无异,亦无须乎明辨也。惟其病在少阴则辨之甚易。何者?因少阴之病,寒热迥分 
两途,其寒者为少阴伤寒之本病,其热者大抵为伏气化热之温病也。若谓系伤寒入少阴久而化热,何以少阴 
病两三日,即有宜用黄连阿胶汤、大承气汤者?盖伏气皆伏于三焦脂膜之中,与手足诸经皆有贯通之路,其 
当春阳化热而萌动,恒视脏腑虚弱之处以为趋向,所谓“邪之所凑,其气必虚”也。其人或因冬不藏精,少 
阴之脏必虚,而伏气之化热者即乘虚而入,遏抑其肾气不能上升与 
心气相接续,致心脏跳动无力,遂现少阴微细之脉。故其脉愈微细,而所蕴之燥热愈甚。用黄连以清少阴之 
热,阿胶、鸡子黄以增少阴之液,即以助少阴肾气之上达,俾其阴阳之气相接续,脉象必骤有起色,而内陷 
之邪热亦随之外透矣。至愚遇此等证时,又恒师仲师之意而为之变通,单用鲜白茅根四两,锉碎,慢火煎两 
三沸,视茅根皆沉水底,其汤即成,去渣取清汤一大碗,顿服下,其脉之微细者必遽变为洪大有力之象。再 
用大剂白虎加人参汤,煎汤三茶杯,分三次温饮下,每服一次调入生鸡子黄一枚, 
其病必脱然全愈。用古不必泥古,仲师有知,亦当不吾嗔也。 

<目录>三、医论
<篇名>55.论伤寒温病神昏谵语之原因及治法
属性:伤寒温病皆有谵语神昏之证,论者责之阳明胃实。然又当详辨其脉象之虚实,热度之高下,时日之浅深, 
非可概以阳明胃实论也。 
其脉象果洪而有力,按之甚实者,可按阳明胃实治之。盖胃腑之热上蒸,则脑中之元神,心中之识神 
皆受其累,是以神昏谵语,不省人事,或更大便燥结,不但胃实,且又肠实,阻塞肾气不能上交于心, 
则亢阳无制,心神恍惚,亦多谵妄,或精神不支,昏愦似睡。若斯者,可投以大剂白虎汤,遵《伤寒论》一煎 
三服之法,煎汤三盅,分三次温饮下。其大便燥结之甚者,可酌用大、小承气汤(若大便燥结不甚者但投以大剂 
白虎汤大便即可通下),其神昏谵语自愈也。 
有脉象确有实热,其人神昏谵语,似可用白虎汤矣,而其脉或兼弦、兼数,或重按仍不甚实者,宜治以白 
虎加人参汤。曾治一农家童子,劳力过度,因得温病。脉象弦而有力,数近六至。谵语不休,所言皆劳力之 
事。本拟治以白虎加人参汤,因时当仲夏,且又童年少阳之体,遂先与以白虎汤。服后脉搏力减,而谵 
语益甚。幸其大便犹未通下,急改用白虎加人参汤,将方中人参 
加倍,煎汤三茶杯,分三次温饮下,尽剂而愈。盖脉象弦数,真 
阴必然亏损,白虎加人参汤能于邪热炽盛之中滋其真阴,即以退其邪热。盖当邪热正炽时,但用玄参、沙参、 
生地诸药不能滋阴,因其不能胜邪热,阴分即无由滋长也。惟治以白虎加人参汤,则滋阴退热一举两得,且 
能起下焦真阴与上焦亢甚之阳相济,是以投之有捷效也。 
其证若在汗吐下后,脉虽洪实,用白虎汤时亦宜加人参。曾治一人,温病之热传入阳明,脉象洪实有力, 
谵语昏瞀。投以大剂白虎汤,热退强半,脉力亦减,而其至数转数,一息六至,谵语更甚。细询其病之经过,言 
数日前因有梅毒服降药两次。遂急改用白虎加人参汤,亦倍用人参(此两案中用白虎加人参汤皆将人参倍加者, 
因从前误用白虎汤也,若开首即用白虎加人参汤,则人参无事加倍矣),煎汤三杯,分三次温饮下,亦尽剂而愈。 
有伏气为病,因肾虚窜入少阴,遏抑肾气不能上升与心相济,致心脉跳动无力,燥热郁中不能外透,闭目 
昏昏似睡,间作谵语。此在冬为少阴伤寒之热证,在春为少阴温病。宜治以大剂白虎加人参汤,用鲜白茅根煮 
水以之煎药,取汤三盅,分数次饮下自愈。 
有患寒温者,周身壮热,脉象洪实,神昏不语。迨用凉药清之,热退脉近和平,而仍然神昏或谵语者, 
必兼有脑髓神经病,当继用治脑髓神经之药。曾治一学校学生,温病热入阳明,脉象甚实,神昏不语,卧床 
并不知转侧。用白虎汤清之,服两剂后热退十之七八,脉象之洪实亦减去强半,自知转侧,而精神仍不明 
了。当系温病之热上蒸,致其脑膜生炎而累及神经也。遂改用小剂白虎加人参汤,又加羚羊角二钱(另煎兑服), 
一剂而愈。又治一幼童,得温病三日,热不甚剧,脉似有力,亦非洪实,而精神竟昏 
昏似睡,不能言语。此亦温病兼脑膜炎也。因其温病甚轻,俾但用羚羊角钱半煎汤服之,其病霍然顿愈。 
有寒温之病,传经已遍,将欲作汗,其下焦阴分虚损,不能 
与上焦之阳分相济以化汗,而神昏谵语者。曾治一壮年,仲夏长 
途劳役,因受温病已过旬日,精神昏愦,谵语不省人事,且两手乱动不休,其脉弦而浮,一息近六至,不任 
循按,两尺尤甚。投以大滋真阴之品,若玄参、生地黄、生山药、甘枸杞、天门冬之 
类,共为一大剂煎服,一日连进二剂,当日得汗而愈。 
有寒温之病服开破降下之药太过,伤其胸中大气,迨其大热已退,而仍然神昏或谵语者。曾治一壮年得 
温病,延医服药二十余日,外感之热尽退,精神转益昏沉。及愚视之,周身皆凉,奄奄一息,呼之不应,舌干 
如磋,毫无舌苔,其脉象微弱而迟,不足四至,五六呼吸之顷必长出气一次。此必因服开降之药太过, 
伤其胸中大气也。盖胸中大气因受伤下陷,不能达于脑中,则神昏;不能上潮于舌本,则舌干;其周身皆凉者, 
大气因受伤不能宣布于营卫也;其五六呼吸之顷必长出气一次者,因大气伤后不能畅舒,故太息以舒其气也。 
遂用野台参一两,柴胡一钱,煎汤灌之。连服两剂全愈。又治一少年,于初春得伤寒,先经他医治愈,后因饮 
食过度,病又反复,投以白虎汤治愈。隔三日,陡然反复甚剧,精神恍惚,肢体颤动,口中喃喃皆不成语。诊 
其脉,右部寸关皆无力而关脉尤不任循按。愚曰此非病又反复,必因前次之过食病复,而此次又戒饮食过 
度也。饱食即可愈矣。其家人果谓有鉴前失,数日所与饮食甚少,然其精神昏愦若斯,恐其不能饮食。愚曰 
果系因饿而成之病,与之食必然能食。然仍须撙节与之,多食几次可也。其家人果根据愚言,十小时中连与饮 
食三次,病若失。盖人胸中大气原借水谷之气以为培养,病后气虚,又乏水谷之气以培养之,是以胸中大气 
虚损而现种种病状也。然前案因服开降之药伤其大气,故以补气兼升气之药治之。后案因 
水谷之气缺乏虚其大气,故以饮食治之。斯在临证者精心体验,息息与病机相符耳。 
有温而兼疹,其毒热内攻瞀乱其神明者。曾治一少年,温病 
热入阳明,连次用凉药清之,大热已退强半,而心神骚扰不安,合目恒作谵语。其脉有余热,似兼紧象。因 
其脉象热而兼紧,疑其伏有疹毒未出。遂投以小剂白虎汤,送服羚羊角细末一钱,西药阿斯匹林二分。表出 
痧粒满身而愈。又治一幼女患温疹,其疹出次日即靥,精神昏昏似睡,时有惊悸,脉象数而有力。投以白虎 
汤加羚羊角钱半(另煎兑服),用鲜芦根三两煮水以之煎药,取汤两茶 
盅,分三次温饮下,其疹得出,病亦遂愈。 
有其人素多痰饮,其寒温之热炽盛与痰饮互相胶漆以乱其神明者。栝蒌解下附有治验之案可参观。 
有温疫传染之邪由口鼻传入,自肺传心,其人恒无故自笑,精神恍惚,言语错乱,妄言妄见者。曾治一 
媪患此证,脉象有力,关前摇摇而动。投以拙拟护心至宝丹,一剂而愈。以上所谓 
寒温诸证,其精神昏愦谵语之原因及治法大略已备。至于变通化裁,相机制宜,又在临证者之精心研究也。 

<目录>三、医论
<篇名>56.论吴又可达原饮不可以治温病
属性:北方医者治温病,恒用吴又可达原饮,此大谬也。吴氏谓崇祯辛巳,疫气流行,山东、浙江南北两道感 
者尤多,遂着《瘟疫论》一书。首载达原饮,为治瘟疫初得之方,原非治温病之方也。疫者,天地戾气,其 
中含有毒菌,遍境传染若役使然,故名为疫。因疫多病热,故名为瘟疫(病寒者名为寒疫),瘟即温也。是 
以方中以逐不正之气为主。至于温病,乃感时序之温气,或素感外寒伏于膜原,久而化热,乘时发动,其中原 
无毒菌,不相传染。治之者惟务清解其热,病即可愈。若于此鉴别未精,本系温病而误投以达原饮,其方中 
槟榔开破之力既能引温气内陷,而浓朴、草果之辛温开散大能耗阴助热,尤非病温者所宜(病温者多阴虚 
尤忌耗阴之药),虽有知母、芍药、黄芩各一钱,其凉力甚轻,是以用此方治温病 
者,未有见其能愈者也。且不惟不能愈,更有于初病时服之即陡然变成危险之证者,此非愚之凭空拟议,诚 
有所见而云然也。 
愚初习医时,曾见一媪,年过六旬,因伤心过度,积有劳疾,于仲春得温病。医者投以达原饮,将方中 
草果改用一钱,谓得汗则愈。乃服后汗未出而病似加重,医者遂将草果加倍,谓服后必然得汗。果服后头面 
汗出如洗,喘息大作,须臾即脱。或疑此证之偾事,当在服达原饮将草果加重,若按其原方分量,草果 
只用五分,即连服数剂亦应不至汗脱也。答曰∶草果性甚猛烈,即五分亦不为少。愚尝治脾虚泄泻服药不效, 
因思四神丸治五更泻甚效,中有肉果,本草谓其能健脾涩肠,遂用健补脾胃之药煎汤送服肉果末五分。须 
臾觉心中不稳,六脉皆无,迟半点钟其脉始见。恍悟病患身体虚弱,不胜肉果辛散之力也。草果与肉果性 
原相近,而其辛散之力更烈于肉果,虽方中止用五分,而与槟榔、浓朴并用,其猛烈之力固非小矣。由 
斯观之,达原饮可轻用哉! 

<目录>三、医论
<篇名>57.论革脉之形状及治法
属性:革脉最为病脉中之险脉,而人多忽之,以其不知革脉之真象,即知之亦多不知治法也。其形状如按鼓革, 
外虽硬而中空,即弦脉之大而有力者。因其脉与弦脉相近,是以其脉虽大而不洪(无起伏故不洪),虽有力 
而不滑(中空故不滑)。即此以揣摩此脉,其真象可得矣。其主病为阴阳离绝,上下不相维系,脉至如此,病 
将变革(此又革脉之所以命名),有危在顷刻之势。丁卯在津,治愈革脉之证数次。惟有一媪八旬有六,治 
之未愈,此乃年岁所关也。今特将其脉之最险者详录一则于下,以为治斯证者之嚆矢。 
外孙王××,年五十,身体素羸弱,于仲夏得温病。心中热 
而烦躁,忽起忽卧,无一息之停。其脉大而且硬,微兼洪象。其 
舌苔薄而微黑,其黑处若斑点。知其内伤与外感并重也。其大便四日未行,腹中胀满,按之且有硬处。其家 
人言,腹中满硬系宿病,已逾半载,为有此病,所以身形益羸弱。因思宿病宜从缓治,当以清其温热为急务。 
为疏方用白虎加人参汤,方中石膏用生者两半,人参用野台参五钱,又以生山药八钱代方中粳米,煎汤两盅, 
分三次温饮下。一剂,外感之热已退强半,烦躁略减,仍然起卧不安,而可睡片时。脉之洪象已无,而大硬 
如故。其大便尤未通下,腹中胀益甚。遂用生赭石细末、生怀山药各一两,野台参六钱,知母、玄参各五钱, 
生鸡内金钱半。煎汤服后,大便通下。迟两点钟,腹中作响,觉瘀积已开,连下三次,皆系陈积,其证陡变, 
脉之大与硬,较前几加两倍,周身脉管皆大动,几有破裂之势,其心中之烦躁,精神之骚扰,起卧之频频不 
安,实有不可言语形容者。其家人环视惧甚,愚毅然许为治愈。遂急开净萸肉、生龙骨各两半,熟地黄、生山 
药各一两,野台参、白术各六钱,炙甘草三钱。煎汤一大碗,分两次温饮下,其状况稍安,脉亦见敛。当日按 
方又进一剂,可以安卧。须臾,其脉渐若瘀积未下时,其腹亦见软,惟心中时或发热。继将原方去白术, 
加生地黄八钱。日服一剂。三剂后,脉象已近平和,而大便数日未行,且自觉陈积未净,遂将萸肉、龙骨各 
减五钱,加生赭石六钱,当归三钱。又下瘀积若干。其脉又见大,遂去赭石、当归,连服十余剂全愈。 

<目录>三、医论
<篇名>58.论脑充血之原因及治法
属性:脑充血病之说倡自西人,而浅见者流恒讥中医不知此病,其人盖生平未见《内经》者也。尝读《内经》 
至调经论,有谓“血之与气,并走于上,则为大厥,厥则暴死,气反则生,不反则 
死”云云,非即西人所谓脑充血之证乎?所有异者,西人但言充 
血,《内经》则谓血之与气并走于上。盖血必随气上升,此为一定之理。而西人论病皆得之剖解之余,是以 
但见血充脑中,而不知辅以理想以深究病源,故但名为脑充血也。至《内经》所谓“气反则生,不反则死” 
者,盖谓此证幸有转机,其气上行之极,复反而下行,脑中所充之血应亦随之下行,故其人可生;若其气 
上行不反,升而愈升,血亦随之充而愈充,脑中血管可至破裂,所以其人死也。又《内经》厥论篇谓“巨阳之 
厥则肿首,头重不能行,发为 (眩也)仆”、“阳明之厥,面赤而热,妄言妄见”、“少阳之厥,则暴 
聋颊肿而热”,诸现象皆脑充血证也。推之秦越人治虢太子尸厥,谓“上有绝阳之络,下有破阴之纽”者,亦 
脑充血证也。特是古人立言简括,恒但详究病源,而不细论治法。然既洞悉致病之由,即自拟治法不难也。愚 
生平所治此证甚多,其治愈者,大抵皆脑充血之轻者,不至血管破裂也。今略举数案于下,以备治斯证者之参考。 
在奉天曾治一人,年近五旬,因处境不顺,兼劳碌,渐觉头疼,日浸加剧,服药无效,遂入西人医院。治 
旬日,头疼不减,转添目疼。又越数日,两目生翳,视物不明。来院求为延医。其脉左部洪长有力,自言脑 
疼彻目,目疼彻脑,且时觉眩晕,难堪之情莫可名状。脉证合参,知系肝胆之火挟气血上冲脑部,脑中 
血管因受冲激而膨胀,故作疼;目系连脑,脑中血管膨胀不已,故目疼,生翳且眩晕也。因晓之曰∶“此脑 
充血证也。深考此证之原因,脑疼为目疼之根;而肝胆之火挟气血上冲,又为脑疼之根。欲治此证,当清火、 
平肝、引血下行,头疼愈而目疼、生翳及眩晕自不难调治矣。”遂为疏方,用怀牛膝一两,生杭芍、生 
尤骨、生牡蛎、生赭石各六钱,玄参、川楝子各四钱,龙胆草三钱,甘草二钱,磨取铁锈浓水煎药。服一剂, 
觉头目之疼顿减,眩晕已无。即方略为加减,又服两剂,头疼目疼全愈,视物亦较 
真。其目翳原系外障,须兼外治之法,为制磨翳药水一瓶,日点眼上五六次,徐徐将翳尽消。 
在沧州治一人,六十四岁,因事心甚懊 ,于旬日前即觉头疼,不以为意。一日晨起,忽仆于地,状 
若昏厥,移时苏醒,左手足遂不能动,且觉头疼甚剧。医者投以清火通络之剂,兼法王勋臣补阳还五汤之义, 
加生黄 数钱,服后更觉脑中疼如锥刺,难忍须臾。求为诊视,其脉左部弦长,右部洪长,皆重按甚实。 
询其心中,恒觉发热。其家人谓其素性嗜酒,近因心中懊 ,益以烧酒浇愁,饥时恒以酒代饭。愚曰,此 
证乃脑充血之剧者,其左脉之弦长,懊 所生之热也。右脉之洪长,积酒所生之热也。二热相并,挟脏腑气 
血上冲脑部。脑部中之血管若因其冲激过甚而破裂,其人即昏厥不复醒,今幸昏厥片时苏醒,其脑中血管当不 
至破裂。或其管中之血隔血管渗出,或其血管少有罅隙,出血少许而复自止。其所出之血着于司知觉之神经, 
则神昏;着于司运动之神经,则痿废。此证左半身偏枯,当系脑中血管所出之血伤其司右边运动之神经也。 
医者不知致病之由,竟投以治气虚偏枯之药,而此证此脉岂能受黄 之升补乎?此所以服药后而头疼益 
剧也。遂为疏方,亦约略如前。为其右脉亦洪实,因于方中加生石膏一两,亦用铁锈水煎药。服两剂,头疼 
全愈,脉已和平,左手足已能自动。遂改用当归、赭石、生杭芍、玄参、天冬各五钱,生黄 、乳香、没药 
各三钱,红花一钱,连服数剂,即扶杖能行矣。方中用红花者,欲以化脑中之瘀血也。为此时脉已和平,头已 
不疼,可受黄 之温补,故方中少用三钱,以补助其正气,即借以助归、芍、乳、没以流通血脉,更可调 
玄参、天冬之寒凉,俾药性凉热适均,而可多服也。 
上所录二案,用药大略相同,而皆以牛膝为主药者,诚以牛 
膝善引上部之血下行,为治脑充血证无上之妙品,此愚屡经试验 
而知,故敢贡诸医界。而用治此证,尤以怀牛膝为最佳。 

<目录>三、医论
<篇名>59.论脑充血证可预防及其证误名中风之由
属性:(附∶建瓴汤) 
脑充血证即《内经》之所谓厥证,亦即后世之误称中风证,前论已详辩之矣。而论此证者谓其猝发于 
一旦,似难为之预防。不知凡病之来皆预有朕兆,至脑充血证,其朕兆之发现实较他证为尤显著。且有在数 
月之前,或数年之前,而其朕兆即发露者。今试将其发现之朕兆详列于下∶ 
(一)其脉必弦硬而长,或寸盛尺虚,或大于常脉数倍,而毫无缓和之意。 
(二)其头目时常眩晕,或觉脑中昏愦,多健忘,或常觉疼,或耳聋目胀。 
(三)胃中时觉有气上冲,阻塞饮食不能下行,或有气起自下焦,上行作呃逆。 
(四)心中常觉烦躁不宁,或心中时发热,或睡梦中神魂飘荡。 
(五)或舌胀、言语不利,或口眼歪斜,或半身似有麻木不遂,或行动脚踏不稳、时欲眩仆,或自觉 
头重足轻,脚底如 棉絮。 
上所列之证,偶有一二发现,再参以脉象之呈露,即可断为脑充血之朕兆也。愚十余年来治愈此证颇多, 
曾酌定建瓴汤一方,服后能使脑中之血如建瓴之水下行,脑充血之证自愈。爰将其方详列于下,以备医界采用。 
【建瓴汤】 
生怀山药(一两) 怀牛膝(一两) 生赭石(八钱轧细) 生龙骨(六钱捣细) 
生牡蛎(六钱捣细) 生怀地黄(六钱) 生杭芍(四钱) 柏子仁(四钱) 
磨取铁锈浓水以之煎药。 
方中赭石必一面点点有凸,一面点点有凹,生轧细用之方 
效。若大便不实者去赭石,加建莲子(去心)三钱。若畏凉者,以熟地易生地。 
在津曾治迟××之母,年七旬有四,时觉头目眩晕,脑中作疼,心中烦躁,恒觉发热,两臂觉撑 
胀不舒,脉象弦硬而大,知系为脑充血之朕兆,治以建瓴汤。连服数剂,诸病皆愈,惟脉象虽不若从前之大, 
而仍然弦硬。因苦于吃药,遂停服。后月余,病骤反复。又用建瓴汤加减,连服数剂,诸病又愈。脉象仍未和 
平,又将药停服。后月余,病又反复,亦仍用建瓴汤加减,连服三十余剂,脉象和平如常,遂停 
药勿服,病亦不再反复矣。 
天津王姓叟,年过五旬,因头疼、口眼歪斜,求治于西人医院,西人以表测其脉,言其脉搏之力已达 
百六十毫米汞柱,断为脑充血证,服其药多日无效,继求治于愚。其脉象弦硬而大,知其果系脑部充血,治 
以建瓴汤,将赭石改用一两,连服十余剂,觉头部清爽,口眼之歪斜亦愈,惟脉象仍未复常。复至西人医院 
以表测脉,西医谓较前低二十余毫米汞柱,然仍非无病之脉也。后晤面向愚述之,劝其仍须多多服药,必服 
至脉象平和,方可停服。彼觉病愈,不以介意。后四阅月未尝服药。继因有事出门,劳碌数旬,甫归后又 
连次劳累,一旦忽眩仆于地而亡。观此二案,知用此方以治脑充血者,必服至脉象平和,毫无弦硬之意, 
而后始可停止也。 
友人朱钵文,未尝业医而实精于医。尝告愚曰∶“脑充血证,宜于引血下行药中加破血之药以治之。” 
愚闻斯言,恍有悟会。如目疾其疼连脑者,多系脑部充血所致,至眼科家恒用大黄以泻其热,其脑与目即 
不疼,此无他,服大黄后脑充血之病即愈故也。夫大黄非降血兼能破血最有力之药乎?由斯知凡脑充血证 
其身体脉象壮实者,初服建瓴汤一两剂时,可酌加大黄数钱。其 
身形脉象不甚壮实者,若桃仁、丹参诸药,亦可酌加于建瓴汤中也。 
天津于氏少妇,头疼过剧,且心下发闷作疼,兼有行经过多症,以建瓴汤加减治愈。 
至唐宋以来名此证为中风者,亦非无因。尝征以平素临症实验,知脑充血证恒因病根已伏于内,继又风 
束外表,内生燥热,遂以激动其病根,而猝发于一旦。是以愚临此证,见有夹杂外感之热者,恒于建瓴汤中 
加生石膏一两;或两三日后见有阳明大热、脉象洪实者,又恒治以白虎汤或白虎加人参汤,以清外感之热, 
而后治其脑充血证。此愚生平之阅历所得,而非为唐宋以来之医家讳过也。然究之此等证,谓其为中风兼脑 
充血则可,若但名为中风仍不可也。迨至刘河间出,谓此证非外袭之风,乃内生之风,实因五志过极,动火 
而猝中。大法以白虎汤、三黄汤沃之,所以治实火也;以逍遥散疏之,所以治郁火也;以通圣散、凉膈 
散双解之,所以治表里之邪火也,以六味汤滋之,所以壮水之源以制阳光也;以八味丸引之,所谓从治之法, 
引火归原也;又用地黄饮子治舌喑不能言,足废不能行。此等议论,似高于从前误认脑充血为中风者一筹。 
盖脑充血证之起点,多由于肝气肝火妄动。肝属木能生风,名之为内中风,亦颇近理。然因未悟《内经》 
所谓血之与气并走于上之旨,是以所用之方,未能丝丝入扣,与病证吻合也。至其所载方中有防风、柴 
胡、桂、附诸品,尤为此证之禁药。 
《金匮》有风引汤除热瘫痫。夫瘫既以热名,明其病因热而得也。其证原似脑充血也。方用石药六味,多 
系寒凉之品,虽有干姜、桂枝之辛热,而与大黄、石膏、寒水石、滑石并用,药性混合,仍以凉论(细按之桂枝干 
姜究不宜用)。且诸石性皆下沉,大黄性尤下降, 
原能引逆上之血使之下行。又有龙骨、牡蛎与紫石英同用,善敛冲气,与桂枝同用,善平肝气。肝冲之气 
不上干,则血之上充者自能徐徐下降也。且其方虽名风引,而未尝用祛风之药,其不以热瘫痫为中风明矣。 
特后世不明方中之意,多将其方误解耳。拙拟之建瓴汤,重用赭石、龙骨、牡蛎,且有加石膏之时,实窃师 
风引汤之义也(风引汤方下之文甚简,似非仲景笔墨,故方书多有疑此系后世加入者,故方中之药品不纯)。 
【附录】湖北天门崔××来函∶张港一人患脑充血证,忽然仆地,上气喘急,身如角弓,两目直视。 
全家惶恐,众医束手,殓服已备,迎为延医。遵建瓴汤原方治之,一剂病愈强半。后略 
有加减,服数剂,脱然全愈。按∶镇肝熄风汤,实由建瓴汤加减而成。 

<目录>三、医论
<篇名>60.论脑贫血治法
属性:(附∶脑髓空治法) 
脑贫血者,其脑中血液不足,与脑充血之病正相反也。其人常觉头重目眩,精神昏愦,或面黄唇白、或呼 
吸短气、或心中怔忡。其头与目或间有作疼之时,然不若脑充血者之胀疼,似因有收缩之感觉而作疼。其 
剧者亦可猝然昏仆,肢体颓废或偏枯。其脉象微弱,或至数兼迟。西人但谓脑中血少,不能荣养脑筋,以 
致脑失其司知觉、司运动之机能。然此证但用补血之品,必不能愈。《内经》则谓“上气不足,脑为之不 
满”,此二语实能发明脑贫血之原因,并已发明脑贫血之治法。盖血生于心,上输于脑(心有四血脉管通脑)。 
然血不能自输于脑也。《内经》之论宗气也,谓宗气积于胸中,以贯心脉,而行呼吸,由此知胸中宗气,不 
但为呼吸之中枢,而由心输脑之血脉管亦以之为中枢。今合《内经》两处之文参之,知所谓上气者,即宗 
气上升之气也。所谓上气不足脑为之不满者,即宗气不能贯心脉以助之上升,则脑中气血皆不足 
也。然血有形而气无形,西人论病皆从实验而得,故言血而不言 
气也。因此知脑贫血治法固当滋补其血,尤当峻补其胸中宗气。以助其血上行。持此以论古方,则补血汤重 
用黄 以补气、少用当归以补血者,可为治脑贫血之的方矣。今录其方于下并详论其随证宜加之药品。 
生箭 一两、当归三钱。呼吸短气者,加柴胡、桔梗各二钱。不受温补者,加生地、玄参各四钱。素畏 
寒凉者,加熟地六钱、干姜三钱。胸有寒饮者,加干姜三钱、广陈皮二钱。 
《内经》∶“上气不足,脑为之不满”二语,非但据理想象也,更可实征诸囟门未合之小儿。《灵枢》五 
味篇谓“大气抟于胸中,赖谷气以养之,谷不入半日则气衰,一日则气少”,大气即宗气也。观小儿慢惊 
风证,脾胃虚寒,饮食不化,其宗气之衰可知。更兼以吐泻频频,虚极风动,其宗气不能助血上升以灌注于脑 
更可知。是以小儿得此证者,其囟门无不塌陷,此非上气不足头为不满之明征乎?王勉能谓∶“小儿慢惊风 
证,其脾胃虚寒,气血不能上朝脑中,既有贫血之病,又兼寒饮填胸,其阴寒之气上 
冲脑部,激动其脑髓神经,故发痫痉”,实为通论。 
方书谓∶真阴寒头疼证,半日即足损命。究之此证实兼因宗气虚寒,不能助血上升,以致脑中贫血乏气, 
不能御寒,或更因宗气虚寒之极而下陷,呼吸可至顿停,故至危险也。审斯,知欲治此证,拙拟回阳升 
陷汤可为治此证的方矣。若细审其无甚剧之实寒者,宜将干姜减半,或不用亦可。 
《内经》论人身有四海,而脑为髓海。人之色欲过度者,其脑髓必空,人之脑髓空者,其人亦必头 
重目眩,甚或猝然昏厥,知觉运动俱废,因脑髓之质原为神经之本源也。其证实较脑贫血尤为紧要。治之者, 
宜用峻补肾经之剂,加鹿角胶以通督脉。并宜清心寡欲,按此服药不辍,还精补脑之功自能收效于数旬中也。 

<目录>三、医论
<篇名>61.论脑贫血痿废治法
属性:(附∶干颓汤、补脑振痿汤) 
肢体痿废,而其病因实由于脑部贫血也。按生理之实验,人之全体运动皆脑髓神经司之,虽西人之说,而 
洵可确信。是以西人对于痿废之证皆责之于脑部。而实有脑部充血与脑部贫血之殊。盖脑髓神经原借血为濡润 
者也,而所需之血多少,尤以适宜为贵。彼脑充血者,血之注于脑者过多,力能排挤其脑髓神经,俾失所司。 
至脑贫血者,血之注于脑者过少,无以养其脑髓神经,其脑髓神经亦恒至失其所司。至于脑中之所以贫血, 
不可专责诸血也,愚尝读《内经》而悟其理矣。 
《内经》谓∶“上气不足,脑为之不满,耳为之苦鸣,头为之倾,目为之眩。”夫脑不满者,血少也。因脑 
不满而贫血,则耳鸣、头目倾眩即连带而来,其剧者能使肢体痿废不言可知。是西人脑贫血可致痿废之说原 
与《内经》相符也。然西医论痿废之由,知因脑中贫血,而《内经》更推脑中贫血之由,知因上气不足。夫 
上气者何?胸中大气也(亦名宗气)。其气能主宰全身,斡旋脑部,流通血脉。彼脑充血者,因肝胃气逆, 
挟血上冲,原于此气无关,至脑贫血者,实因胸中大气虚损,不能助血上升也。是以欲治此证者,当以补 
气之药为主,以养血之药为辅,而以通活经络之药为使也。爰本此义拟方于下。 
【干颓汤】治肢体痿废,或偏枯,脉象极微细无力者。 
生箭 (五两) 当归(一两) 甘枸杞果(一两) 净杭萸肉(一两) 
生滴乳香(三钱) 生明没药(三钱) 真鹿角胶(六钱捣碎) 
先将黄 煎十余沸,去渣。再将当归、枸杞、萸肉、乳香、没药入汤同煎十余沸,去渣。入鹿角胶末 
融化取汤两大盅,分两次温饮下, 
方中之义,重用黄 以升补胸中大气,且能助气上升,上达 
脑中,而血液亦即可随气上注。惟其副作用能外透肌表,具有宣 
散之性,去渣重煎,则其宣散之性减,专于补气升气矣。当归为生血之主药,与黄 并用,古名补血汤,因气 
旺血自易生,而黄 得当归之濡润,又不至燥热也。萸肉性善补肝,枸杞性善补肾,肝肾充足,元气必然壮 
旺,元气者胸中大气之根也。且肝肾充足则自脊上达之督脉必然流通,督脉者又脑髓神经之根也。且 
二药皆汁浆稠润,又善赞助当归生血也。用乳香、没药者,因二药善开血痹,血痹开则痿废者久瘀之经络自 
流通矣。甩鹿角胶者,诚以脑既贫血,其脑髓亦必空虚,鹿角其所熬之胶善补脑髓,脑髓足则脑中贫血之病 
自易愈也。此方服数十剂后身体渐渐强壮,而痿废仍不愈者,可继服后方。 
【补脑振痿汤】治肢体痿废偏枯,脉象极微细无力,服药久不愈者。 
生箭 (二两) 当归(八钱) 龙眼肉(八钱) 杭萸肉(五钱) 
胡桃肉(五钱) 虫(三枚大者) 地龙(三钱去净土) 生乳香(三钱) 
生没药(三钱) 鹿角胶(六钱) 制马钱子末(三分) 
共药十一味,将前九味煎汤两盅半,去渣,将鹿角胶入汤内融化,分两次送服制马钱子末一分五厘。 
此方于前方之药独少枸杞,因胡桃肉可代枸杞补肾,且有强健筋骨之效也。又尝阅《沪滨医报》,谓 
脑中血管及神经之断者,地龙能续之。愚则谓必辅以 虫,方有此效。盖蚯蚓(即地龙) 
善引, 虫善接(断之能自接),二药并用能将血管神经之断者引而接之,是以方中又加此二味也。加 
制马钱子者,以其能 动神经使灵活也。此方与前方若服之觉热者,皆可酌加天花粉、天冬各数钱。 
天津于××,年过四旬,自觉呼吸不顺,胸中满闷,言语动作皆渐觉不利,头目昏沉,时作眩晕。延 
医治疗,投以开胸理气之品,则四肢遽然痿废。再延他医,改用补剂而仍兼用开气之品, 
服后痿废加剧,言语竟不能发声。愚诊视其脉象沉微,右部尤不任循按,知其胸中大气及中焦脾胃之气皆虚 
陷也。于斯投以拙拟升陷汤加白术、当归各三钱。服两剂,诸病似皆稍愈,而脉象仍如旧。因将 、术、当 
归、知母各加倍,升麻改用钱半,又加党参、天冬各六钱,连服三剂,口可出声而仍不能言,肢体稍能运 
动而不能步履,脉象较前有起色似堪循按。因但将黄 加重至四两,又加天花粉八钱,先用水六大盅将黄 
煎透,去渣,再入他药,煎取清汤两大盅,分两次服下,又连服三剂,勉强可作言语,然恒不成句,人扶之 
可以移步。遂改用干颓汤,惟黄 仍用四两,服过十剂,脉搏又较前有力;步履虽仍需人,而起卧可自 
如矣;言语亦稍能达意,其说不真之句,间可执笔写出,从前之头目昏沉眩晕者,至斯亦见轻。俾继服补 
脑振痿汤,嘱其若服之顺利,可多多服之,当有脱然全愈之一日也。 
按∶此证其胸满闷之时,正因其呼吸不顺也。其呼吸之所以不顺,因胸中大气及中焦脾胃之气皆虚而下 
陷也。医者竟投以开破之药,是以病遽加重。至再延他医,所用之药补多开少,而又加重者,因气分当虚 
极之时,补气之药难为功,破气之药易生弊也。愚向治大气下陷证,病患恒自觉满闷,其实非满闷,实短气 
也,临证者细细考究,庶无差误。 

<目录>三、医论
<篇名>62.论心病治法
属性:心者,血脉循环之枢机也。心房一动则周身之脉一动,是以心机亢进,脉象即大而有力,或脉搏更甚数; 
心脏麻痹,脉象即细而无力,或脉搏更甚迟。是脉不得其平,大抵由心机亢进与心脏麻痹而来也。于以 
知心之病虽多端,实可分心机亢进、心脏麻痹为二大纲。 
今试先论心机亢进之病∶有因外感之热炽盛于阳明胃府之 
中,上蒸心脏,致心机亢进者,其脉象洪而有力,或脉搏加数。可用大剂白虎汤以清其胃。或更兼肠有燥粪, 
大便不通者,酌用大、小承气汤以涤其肠,则热由下泻,心机之亢进者自得其平矣。 
有下焦阴分虚损,不能与上焦阳分相维系,其心中之君火恒至浮越妄动,以致心机亢进者,其人常苦 
眩晕,或头疼、目胀、耳鸣,其脉象上盛下虚,或摇摇无根,至数加数。宜治以加味左归饮。方用大熟地、大 
生地、生怀山药各六钱,甘枸杞、怀牛膝、生龙骨、生牡蛎各五钱,净萸肉三钱,云苓片一钱。此壮水 
之源以制浮游之火,心机之亢者自归于和平矣。 
有心体之阳素旺,其胃腑又积有实热,复上升以助之,以致心机亢进者,其人脉虽有力,而脉搏不数, 
五心恒作灼热。宜治以咸寒之品(《内经》谓热淫于内治以咸寒),若朴硝、太阴玄精石及西药硫苦皆 
为对证之药(每服少许日服三次久久自愈)。盖心体属火,味之咸者属水,投以咸寒之品,是以寒胜热水胜火也。 
人之元神藏于脑,人之识神发于心。识神者思虑之神也。人常思虑,其心必多热,以人之神明属阳, 
思虑多者,其神之阳常常由心发露,遂致心机因热亢进,其人恒多迷惑。其脉多现滑实之象,因其思虑所生 
之热恒与痰涎互相胶漆,是以其脉滑而有力也。可用大承气汤(浓朴宜少用),以清热降痰;再加赭石(生 
赭石两半轧细同煎)、甘遂(甘遂一钱研细调药汤中服)以助其清热降痰之力。药性虽近猛烈,实能稳 
建奇功,而屡试屡效也。 
有心机亢进之甚者,其鼓血上行之力甚大,能使脑部之血管至于破裂,《内经》所谓血之与气并走于上 
之大厥也,亦即西人所谓脑充血之险证也。推此证之原因,实由肝木之气过升,肺金之气又失于肃降,则 
金不制木,肝木之横姿遂上干心脏,以致心机亢进。若更兼冲气上冲,其脉象之弦硬有力更迥异乎寻常矣。当 
此证之初露朕兆时,必先脑中作疼,或间觉眩晕,或微觉半身不 
利,或肢体有麻木之处。宜思患预防,当治以清肺、镇肝、敛冲之剂,更重用引血下行之药辅之,连服十 
余剂或数十剂,其脉象渐变柔和,自无意外之患。向因此证方书无相当之治法,曾拟得建瓴汤一方,屡次用 
之皆效。即不能治之于预,其人忽然昏倒,须臾能自苏醒者,大抵脑中血管未甚破裂,急服此汤,皆可保其 
性命。连服数剂,其头之疼者可以全愈,即脑中血管不复充血,其从前少有破裂之处亦可自愈,而其肢体之痿 
废者亦可徐徐见效。此方原用铁锈水煎药,若刮取铁锈数钱,或多至两许,与药同煎服更佳。 
有非心机亢进而有若心机亢进者,怔忡之证是也。心之本体,原长发动以营运血脉,然无病之人初不觉 
其动也,惟患怔忡者则时觉心中跳动不安。盖人心中之神明原以心中之气血为凭根据,有时其气血过于虚损, 
致神明失其凭根据,虽心机之动照常,原分毫未尝亢进,而神明恒若不任其震撼者。此其脉象多微细,或脉搏 
兼数。宜用山萸肉、酸枣仁、怀山药诸药品以保合其气;龙眼肉、熟地黄、柏子仁诸药以滋养其血;更宜用 
生龙骨,牡蛎、朱砂(研细送服)诸药以镇安其神明。气分虚甚者可加人参,其血分虚而且热者可加生地黄。 
有因心体肿胀,或有瘀滞,其心房之门户变为窄小,血之出入致有激荡之力,而心遂因之觉动者,此似心机 
亢进而亦非心机亢进也。其脉恒为涩象,或更兼迟。宜治以拙拟活络效灵丹加生怀山药、龙眼肉各一两,共 
煎汤服。或用节菖蒲三两,远志二两,共为细末,每服二钱,红糖冲水送下,日服三次,久当自愈。因菖蒲 
善开心窍,远志善化瘀滞(因其含有稀盐酸),且二药并用实善调补心脏,而送以红糖水者,亦所以助其血 
脉流通也。 
至心脏麻痹之原因,亦有多端,治法亦因之各异。如伤寒温病之白虎汤证,其脉皆洪大有力也,若不 
实时投以白虎汤,脉洪大有力之极,又可渐变为细小无力,此乃由心机亢进而转为心脏 
麻痹。病候至此,极为危险,宜急投以大剂白虎加人参汤,将方中人参加倍,煎汤一大碗,分数次温饮下, 
使药力相继不断,一日连服二剂,庶可挽回。盖外感之热,传入阳明,其热实脉虚者,原宜治以白虎加人参 
汤(是以伤寒汗吐下后用白虎汤时皆加人参)。然其脉非由实转虚也。至其脉由实转虚,是其心脏为热所伤而 
麻痹,已成坏证,故用白虎加人参汤时宜将人参加倍,助其心脉之跳动,即可愈其心脏之麻痹也。 
有心脏本体之阳薄弱,更兼胃中积有寒饮,溢于膈上,凌逼心脏之阳,不能用事,其心脏渐欲麻痹,脉 
象异常微细,脉搏异常迟缓者,宜治以拙拟理饮汤,连服十余剂,寒饮消除净尽,心脏之阳自复其初,脉之微 
弱迟缓者亦自复其常矣。此证间有心中觉热、或周身发热、或耳鸣欲聋之种种响应象,须兼看理饮汤后 
所载治愈诸案,临证诊断,自无差误。 
有心脏为传染之毒菌充塞以至于麻痹者,霍乱证之六脉皆闭者是也。治此证者,宜治其心脏之麻痹,更 
宜治其心脏之所以麻痹,则兴奋心脏之药,自当与扫除毒菌之药并用,如拙拟之急救回生丹、卫生防疫宝丹 
是也。此二方中用樟脑所升之冰片,是兴奋心脏以除其麻痹也。二方中皆有朱砂、薄荷冰,是扫除毒菌以治 
心脏之所以麻痹也。是以无论霍乱之因凉因热,投之皆可奏效也(急救回生丹药性微凉,以治因热之霍 
乱尤效;卫生防疫宝丹其性温,无论病因凉热用之皆有捷效)。 
有心中神明不得宁静,有若失其凭根据,而常惊悸者,此其现象若与心脏麻痹相反,若投以西药麻醉之品, 
亦可取效于一时。而究其原因,实亦由心体虚弱所致,惟投以强心之剂,乃为根本之治法。当细审其脉,若 
数而兼滑者,当系心血虚而兼热,宜用龙眼肉、熟地黄诸药补其虚,生地黄、玄参诸药泻其热,再用生龙 
骨、牡蛎以保合其神明,镇靖其魂魄,其惊悸自除矣。其脉微弱无力者,当系心气虚而莫支,宜用参、术、 
诸药以补其气,兼 
用生地黄、玄参诸滋阴药以防其因补生热,更用酸枣仁、山萸肉以凝固其神明,收敛其气化,其治法与前条 
脉弱怔忡者大略相同。特脉弱怔忡者,心机之发动尤能照常,而此则发动力微,而心之本体又不时颤动,犹 
人之力小任重而身颤也,其心脏弱似较怔忡者尤甚矣。 
有其惊悸恒发于夜间,每当交睫甫睡之时,其心中即惊悸而醒,此多因心下停有痰饮,心脏属火,痰 
饮属水,火畏水迫,故作惊悸也。宜清痰之药与养心之药并用,方用二陈汤加当归、菖蒲、远志,煎汤送服 
朱砂细末三分。有热者加玄参数钱,自能安枕稳睡而无惊悸矣。 

<目录>三、医论
<篇名>63.论肺病治法
属性:(附∶清金二妙丹、清肺三妙丹、治肺病便方) 
肺病之因,有内伤外感之殊。然无论内伤外感,大抵皆有发热之证,而后酿成肺病。诚以肺为娇脏,且 
属金,最畏火刑故也。有如肺主皮毛,外感风邪,有时自皮毛袭入肺脏,阻塞气化,即暗生内热。而皮毛为 
风邪所束,不能由皮毛排出炭气,则肺中不但生热,而且酿毒,肺病即由此起点。其初起之时,或时 
时咳嗽,吐痰多有水泡,或周身多有疼处,舌有白苔,或时觉心中发热,其脉象恒浮而有力。可先用西药阿 
斯匹林一瓦,白糖冲水送下,俾周身得汗;继用玄参、天花粉各五钱,金银花、川贝母各三钱,硼砂八分( 
研细分两次送服),粉甘草细末三钱(分两次送服),煎汤服。再每日用阿斯匹林一瓦,分三次服,白糖水送 
下,勿令出汗,此三次中或一次微有汗者亦佳。如此服数日,热不退者,可于汤药中加生石膏七八钱。若 
不用石膏,或用汤药送服西药安知歇貌林半瓦亦可。 
若此时不治,病浸加剧,吐痰色白而粘,或带腥臭,此时亦 
可先用阿斯匹林汗之。然恐其身体虚弱,不堪发汗,宜用生怀山药一两或七八钱煮作茶汤,送服阿斯匹林 
半瓦,俾服后微似有汗即可。仍用前汤药送服粉甘草细末、三七细末各一钱,煎渣时再送服二药如前。仍 
兼用阿斯匹林三分之一瓦,白糖冲水送下,或生怀山药细末四五钱煮茶汤送下,日两次。其嗽不止者,可用山 
药所煮茶汤送服川贝细末三钱。山药煮作茶汤,其味微酸,欲其适口可少调以白糖或柿霜皆可。若不欲 
吃茶汤者,可用生山药片,将其分量加倍,煮取清汤,以代茶汤饮之。 
若当此时不治,以后病又加剧,时时咳吐脓血,此肺病已至三期,非寻常药饵所能疗矣。必用中药极 
贵重之品,若徐灵胎所谓用清凉之药以清其火,滋润之药以养其血,滑降之药以祛其痰,芳香之药以通其气, 
更以珠黄之药解其毒,金石之药填其空,兼数法而行之,屡试必效。又邑中曾××,精医术,尝告愚曰∶ 
“治肺痈惟林屋山人《外科证治全生集》中犀黄丸最效,余用之数十年,治愈肺痈甚多。”后愚至奉天, 
遇肺痈咳吐脓血服他药不愈者,俾于服汤药之外兼服犀黄丸,果如曾××所言,效验异常。清凉华盖饮后有 
案,可参观。至所服汤药,宜用前方加牛蒡子、栝蒌仁各数钱以泻其脓,再送服三七细末二钱以止其血。至 
于犀黄丸配制及服法,皆按原书,兹不赘。 
有外感伏邪伏膈膜之下,久而入胃,其热上熏肺脏,以致成肺病者,其咳嗽吐痰始则稠粘,继则腥臭, 
其舌苔或白而微黄,其心中燥热,头目昏眩,脉象滑实,多右胜于左。宜用生石膏一两,玄参、花粉、生怀 
山药各六钱,知母、牛蒡子各三钱,煎汤,送服甘草、三七细末如前。再用阿斯匹林三分之一瓦,白糖水 
送服,日两次。若其热不退,其大便不滑泻者,石膏可以加重。曾治奉天徐姓叟病肺,其脉弦长有力,迥异 
寻常,每剂药中用生石膏四两,连服数剂,脉始柔和。由斯观之,药以胜病为准,其分量轻 
重,不可预为限量也。若其脉虽有力而至数数者,可于前方中石膏改为两半,知母改为六钱,再加潞党参 
四钱。盖脉数者其阴分必虚,石膏、知母诸药虽能退热,而滋阴仍所非长,辅之以参,是仿白虎加人参汤 
之义,以滋其真阴不足(凉润之药得人参则能滋真阴),而脉之数者可变为和缓也。若已咳嗽吐脓血者,亦 
宜于服汤药外兼服犀黄丸。 
至于肺病由于内伤,亦非一致。有因脾胃伤损,饮食减少,土虚不能生金,致成肺病者。盖脾胃虚损 
之人,多因肝木横恣,侮克脾土,致胃中饮食不化精液,转多化痰涎,溢于膈上,粘滞肺叶作咳嗽,久则 
伤肺,此定理也。且饮食少则虚热易生,肝中所寄之相火,因肝木横恣,更挟虚热而刑肺,于斯上焦恒觉烦 
热,吐痰始则粘滞,继则腥臭,胁下时或作疼,其脉弦而有力,或弦而兼数,重按不实。方用生怀山药一两, 
玄参、沙参、生杭芍、柏子仁炒不去油各四钱,金银花二钱,煎汤,送服三七细末一钱,西药百布圣二瓦。 
汤药煎渣时,亦如此送服。若至咳吐脓血,亦宜服此方,兼服犀黄丸。或因服犀黄丸,减去三七亦可。 
至百布圣,则不可减去,以其大有助脾胃消化之力也。然亦不必与汤药同时服,每于饭后迟一句钟服之更佳。 
有因肾阴亏损而致成肺病者。盖肾与肺为子母之脏,子虚必吸母之气化以自救,肺之气化即暗耗。且肾为 
水脏,水虚不能镇火,火必妄动而刑金。其人日晚潮热,咳嗽,懒食,或干咳无痰,或吐痰腥臭,或兼喘促, 
其脉细数无力。方用生山药一两,大熟地、甘枸杞、柏子仁各五钱,玄参、沙参各四钱,金银花、川贝各三钱, 
煎汤送服甘草、三七细末如前。若咳吐脓血者,去熟地,加牛蒡子、蒌仁各三钱,亦宜兼服犀黄丸。若服 
药后脉之数者不能渐缓,亦可兼服阿斯匹林,日两次,每次三分之一瓦。 
盖阿斯匹林之性既善治肺结核,尤善退热,无论虚热实热,其脉 
象数者服之,可使其至数渐缓。然实热服之,汗出则热退,故可服至一瓦。若虚热,不宜出汗,但可解肌, 
服后或无汗,或微似有汗,方能退热,故一瓦必须分三次服。若其人多汗者,无论虚热实热,皆分毫不宜用。 
若其人每日出汗者,无论其病因为内伤、外感、虚热、实热,皆宜于所服汤药中加生龙骨、生牡蛎、净山萸肉 
各数钱。或研服好朱砂五分,亦可止汗,盖以汗为心液,朱砂能凉心血,故能止汗也。 
有其人素患吐血衄血,阴血伤损,多生内热;或医者用药失宜,强止其血,俾血瘀经络亦久而生热,以 
致成肺病者。其人必心中发闷发热,或有疼时,廉于饮食,咳嗽短气,吐痰腥臭,其脉弦硬,或弦而兼数。 
方用生怀山药一两,玄参、天冬各五钱,当归、生杭芍、乳香、没药各三钱,远志、甘草、生桃仁 
(桃仁无毒,宜带皮生用,因其皮红能活血也,然须明辨其果为桃仁,不可误用带皮杏仁)各二钱,煎汤, 
送服三七细末钱半,煎渣时亦送服钱半。盖三七之性,不但善止血,且善 
化瘀血也。若咳吐脓血者,亦宜于服汤药之外兼服犀黄丸。 
或问∶桔梗能引诸药入肺,是以《金匮》治肺痈有桔梗汤。此论肺病诸方何以皆不用桔梗?答曰∶桔 
梗原提气上行之药,病肺者多苦咳逆上气,恒与桔梗不相宜,故未敢加入方中。若其人 
虽病肺而不咳逆上气者,亦不妨斟酌用之。 
或问∶方书治肺痈,恒于其将成未成之际,用皂荚丸或葶苈大枣汤泻之,将肺中之恶浊泻去,而后易于调 
治。二方出自《金匮》,想皆为治肺良方。此论中皆未言及,岂其方不可采用乎?答曰∶二方之药性近猛烈, 
今之病肺者多虚弱,是以不敢轻用。且二方泻肺,治肺实作喘原是正治。至泻去恶浊痰涎,以防肺中腐烂, 
原是兼治之证。其人果肺实作喘且不虚弱者,葶苈大枣汤愚曾用过数次,均能随手奏效。皂荚丸实未尝用,因 
皂荚性热,与肺病之热者不宜也。至欲以泻浊防腐,似不必用此猛烈之品,若拙拟 
方中之硼砂、三七及乳香、没药,皆化腐生新之妙品也。况硼砂善治痰厥,曾治痰厥半日不醒,用硼砂四钱, 
水煮化灌下,吐出稠痰而愈。由斯知硼砂开痰泻肺之力,固不让皂荚、葶苈也。所可贵者,泻肺脏之实,即 
以清肺金之热,润肺金之燥,解肺金之毒(清热润燥解毒皆硼砂所长)。人但知口中腐烂者漱以硼砂则愈( 
冰硼散善治口疮),而不知其治肺中之腐烂亦犹治口中之腐烂也。且拙制有安肺宁嗽丸, 
治肺郁痰火作嗽,肺结核作嗽,用之数年,屡建奇效,此丸药中实亦硼砂之功居多也。 
或问∶古有单用甘草四两煎汤治肺痈者,今所用治肺病诸方中,其有甘草者皆为末送服,而不以之 
入煎者何也?答曰∶甘草最善解毒泻热,然生用胜于熟用。因生用则其性平,且具有开通 
之力,熟用则其性温,实多填补之力。故其解毒泻热之力,生胜于熟。夫炙之为熟,水煮之亦为熟,若 
入汤剂是仍煎熟用矣,不若为末服之之为愈也。且即为末服,又须审辨,盖甘草轧细颇难,若轧之不细, 
而用火炮焦再轧,则生变为熟矣。是以用甘草末者,又宜自监视轧之。再者,愚在奉时曾制有清金二妙丹,方 
用粉甘草细末二两,远志细末一两,和匀,每服钱半,治肺病劳嗽甚有效验。肺有热者,可于每二妙丹一两 
中加好朱砂细末二钱,名为清肺三妙丹。以治病肺结核咳嗽不止,亦极有效。然初 
服三四次时,宜少加阿斯匹林,每次约加四分之一瓦,或五分之一瓦。若汗多,可无加也。 
【治肺病便方】鲜白茅根去皮锉碎一大碗,用水两大碗煎两沸,候半点钟,视其茅根不沉水底, 
再煎至微沸。候须臾茅根皆沉水底,去渣,徐徐当茶温饮之。 
鲜小蓟根二两,锉细,煮两三沸,徐徐当茶温饮之,能愈肺病吐脓血者。 
白莲藕一斤,切细丝,煮取浓汁一大碗,再用柿霜一两融化 
其中,徐徐温饮之。以上寻常土物,用之皆能清减肺病。恒有单用一方,浃辰之间即能治愈肺病者。医方中 
有将鲜茅根、鲜小蓟根、鲜藕共切碎煮汁饮之,名为三鲜饮,以治因热吐血者甚效, 
而以治肺病亦有效。若再调以柿霜更佳。 
拙拟宁嗽定喘饮,亦治肺病之妙品也,而肺病之咳而兼喘者服之尤宜。 
北沙参细末,每日用豆腐浆送服二钱,上焦发热者送服三钱,善治肺病及肺劳喘嗽。 
又∶西药有橄榄油,性善清肺,其味香美,病肺者可以之代香油,或滴七八滴于水中服之亦佳。 
饮食宜淡泊,不可过食炮炙浓味及过咸之物,宜多食菜蔬若藕、鲜笋、白菜、莱菔、冬瓜,果品若西 
瓜、梨、桑堪、苹果、荸荠、甘蔗皆宜。不宜桃、杏。忌烟酒及一切辛辣之物。又忌一切 
变味,若糟鱼、松花蛋、卤虾油、酱豆腐、臭豆腐之类,亦不宜食。 

<目录>三、医论
<篇名>64.总论喘证治法
属性:俗语云喘无善证,诚以喘证无论内伤外感,皆为紧要之证也。然欲究喘之病因,当先明呼吸之枢机何脏 
司之。喉为气管,内通于肺,人之所共知也,而吸气之入,实不仅入肺,并能入心,入肝,入冲任,以及 
于肾。何以言之?气管之正支入肺,其分支实下通于心,更透膈而下通于肝(观肺心肝一系相连可知),由 
肝而下更与冲任相连以通于肾。倘曰不然,何以妇人之妊子者,母呼而子亦呼,母吸而子亦吸乎?呼吸之 
气若不由气管分支通于肝,下及于冲任与肾,何以子之脐带其根蒂结于冲任之间,能以脐承母之呼吸之气,而 
随母呼吸乎?是知肺者发动呼吸之机关也。喘之为病,《神农本草经》名为吐吸,因吸入之气内不能容,而速吐出 
也。其不容纳之故,有由于肺者,有由于肝肾者。试先以由于肝肾者言之。 
肾主闭藏,亦主翕纳,原所以统摄下焦之气化,兼以翕纳呼吸之气,使之息息归根也。有时肾虚不能统摄 
其气化,致其气化膨胀于冲任之间,转挟冲气上冲,而为肾行气之肝木(方书谓肝行肾之气),至 
此不能疏通肾气下行,亦转随之上冲,是以吸入之气未受下焦之翕纳,而转受下焦之冲激,此乃喘之所由来, 
方书所谓肾虚不纳气也。当治以滋阴补肾之品,而佐以生肝血、镇肝气及镇冲、降逆之药。方用大怀熟地、生 
怀山药各一两,生杭芍、柏子仁、甘枸杞、净萸肉、生赭石细末各五钱,苏子、甘草各二钱。热多 
者可加玄参数钱。汗多者可加生龙骨、生牡蛎各数钱。 
有肾虚不纳气,更兼元气虚甚,不能固摄,而欲上脱者,其喘逆之状恒较但肾虚者尤甚。宜于前方中去 
芍药、甘草,加野台参五钱,萸肉改用一两,赭石改用八钱。服一剂喘见轻,心中觉热者,可酌加天冬数钱。 
或用拙拟参赭镇气汤亦可。有因猝然暴怒,激动肝气、肝火,更挟冲气上冲,胃气上逆,迫挤肺之吸气 
不能下行作喘者,方用川楝子、生杭芍、生赭石细末各六钱,浓朴、清夏、乳香、没药、龙胆草、桂枝尖、苏 
子、甘草各二钱,磨取铁锈浓水煎服。以上三项作喘之病因,由于肝肾者也,而其脉象则有区别。阴虚不纳 
气者,脉多细数;阴虚更兼元气欲脱者,脉多上盛下虚;肝火肝气挟冲气胃气上冲者,脉多硬弦而 
长。审脉辨证,自无差误也。 
至喘之由于肺者,因肺病不能容纳吸入之气,其证原有内伤外感之殊。试先论肺不纳气之由于内伤者。一 
一辟,呼吸自然之机关也。至问其所以能呼吸者,固赖胸中大气为之斡旋,又赖肺叶具有活泼机能,以遂 
其辟之用。乃有时肺脏受病,肺叶之 辟活泼者变为易 难辟,而成紧缩之性。暑热之时其紧数稍 
缓,犹可不喘,一经寒凉,则喘立作矣。此肺劳之证,多发于寒凉之时也。宜用生怀山药轧细,每用两许 
煮作粥,调以蔗白糖,送服西药百布圣七八分。盖肺叶紧缩者,以其中津液减少,血脉凝滞也。有山药蔗糖 
以润之,百布圣以化之(百布圣为小猪小牛之胃液制成故善化),久当自愈。其有顽痰过盛者,可再用蓬砂细末 
二分,与百布圣同送服。若外治,灸其肺 穴亦有效,可与内治之方并用。若无西 
药百布圣处,可代以生鸡内金细末三分,其化痰之力较百布圣尤强。 
有痰积胃中,更溢于膈上,浸入肺中,而作喘者。古人恒用葶苈大枣泻肺汤或十枣汤下之,此乃治标 
之方,究非探本穷源之治也。拙拟有理痰汤,连服十余剂,则此证之标本皆清矣。至方中之义,原方下论之甚 
详,兹不赘。若其充塞于胸膈胃府之间,不为痰而为饮,且为寒饮者(饮有寒热,热饮脉滑,其人多有神经 
病,寒饮脉弦细,概言饮为寒者非是),其人或有时喘,有时不喘,或感受寒凉病即反复者,此上焦之阳分虚 
也,宜治以《金匮》苓桂术甘汤,加干姜三钱,浓朴、陈皮各钱半,俾其药之热力能胜其寒,其饮自化而下 
行,从水道出矣。又有不但上焦之阳分甚虚,并其气分亦甚虚,致寒饮充塞于胸中作喘者,其脉不但弦细,且 
甚微弱,宜于前方中加生箭 五钱,方中干姜改用五钱。壬戌秋,严××为其友问二十六七年寒饮结胸, 
时发大喘,极畏寒凉,曾为开去此方(方中生箭 用一两干姜用八钱非极虚寒之证不可用此重剂), 
连服十余剂全愈。方中所以重用黄 者,以其能补益胸中大气,俾大气壮旺自能运化寒饮下行也。上所 
论三则,皆内伤喘证之由于肺者也。 
至外感之喘证,大抵皆由于肺。而其治法,实因证而各有所宜。人身之外表,卫气主之,卫气本于胸 
中大气,又因肺主皮毛,与肺脏亦有密切之关系。有时外表为风寒所束,卫气不能流 
通周身,以致胸中大气无所输泄,骤生膨胀之力,肺悬胸中,因 
受其排挤而作喘。又因肺与卫气关系密切,卫气郁而肺气必郁,亦可作喘。此《伤寒论》麻黄汤所主之证,多 
有兼喘者也。然用麻黄汤时,宜加知母数钱,汗后方无不解之虞。至温病亦有初得作喘者,宜治以薄荷叶、牛 
蒡子各三钱,生石膏细末六钱,甘草二钱,或用麻杏甘石汤方亦可,然石膏万勿 用,而其分量又宜数 
倍于麻黄(石膏可用至一两麻黄治此证多用不过二钱)。此二证之喘同而用药迥异者,因伤寒之脉浮紧, 
温病之脉洪滑也。 
有外感之风寒内侵,与胸间之水气凝滞,上迫肺气作喘者,此《伤寒论》小青龙汤证也。当必效《金匮》 
之小青龙加石膏法,且必加生石膏至两许,用之方效。又此方加减定例,喘者去麻黄,加杏仁。而愚用此方治 
喘时,恒加杏仁,而仍用麻黄一钱。其脉甚虚者,又宜加野台参数钱。更定后世所用小青龙汤分量,可参观也。 
又拙拟从龙汤方,治服小青龙汤后喘愈而仍反复者。用之曾屡次奏效。上所论两则治外感作喘之大略也。 
有其人素有劳疾喘嗽,少受外感即发,此乃内伤外感相并作 
喘之证也,宜治以拙拟加味越婢加半夏汤。因其内伤外感相并作喘,故所用之药亦内伤外感并用。 
特是上所论之喘,其病因虽有内伤、外感、在肝肾、在肺之殊,约皆不能纳气而为吸气难,即《神农 
本草经》所谓吐吸也。乃有其喘不觉吸气难而转觉呼气难者,其病因由于胸中大气虚而下陷,不能鼓动肺脏以 
行其呼吸,其人不得不努力呼吸以自救,其呼吸迫促之形状有似乎喘,而实与不纳气之喘有天渊之分。设或辨 
证不清,见其作喘,复投以降气纳气之药,则凶危立见矣。然欲辨此证不难也,盖不纳气之喘,其剧者必然 
肩息(肩上耸也);大气下陷之喘,纵呼吸有声,必不肩息,而其肩益下垂。即此二证之脉论,亦迥不同, 
不纳气作喘者,其脉多数,或尺弱寸强;大气下陷之喘,其脉多迟而无力,尺脉或略胜于寸脉。察其状而审其 
脉,辨之固百不失一也。其治法当用拙拟升陷汤,以升补其胸中大气,其喘自愈。 
有大气下陷作喘,又兼阴虚不纳气作喘者,其呼吸皆觉困难,益自强为呼吸而呈喘状,其脉象微弱无力, 
或脉搏略数,或背发紧而身心微有灼热。宜治以生怀山药一两,玄参、甘枸杞各六钱,生箭 四钱,知母、 
桂枝尖各二钱,煎汤服。方中不用桔梗、升、柴者,恐与阴虚不纳气有碍也。上二证之喘,同中有异,升 
陷汤后皆有验案可参观也。 
有肝气胆火挟冲胃之气上冲作喘,其上冲之极至排挤胸中大气下陷,其喘又顿止,并呼吸全无,须臾忽 
又作喘,而如斯循环不已者,此乃喘证之至奇者也。曾治一少妇,因夫妻反目得此证,用桂枝尖四钱,恐 
其性热,佐以带心寸冬三钱,煎汤服下,即愈。因读《神农本草经》桂枝能升大气兼能降逆气,用之果效如 
影响。夫以桂枝一物之微,而升陷降逆两擅其功,此诚天之生斯使独也。 

<目录>三、医论
<篇名>65.论胃病噎膈(即胃癌)治法及反胃治法
属性:(附∶变质化瘀丸) 
噎膈之证,方书有谓贲门枯干者,有谓冲气上冲者,有谓痰瘀者,有谓血瘀者。愚向谓此证系中气衰弱, 
不能撑悬贲门,以致贲门缩如藕孔(贲门与大小肠一气贯通,视其大便若羊矢,其贲门大小肠皆缩小可知), 
痰涎遂易于壅滞,因痰涎壅滞冲气更易于上冲,所以不能受食。向曾拟参赭培气汤一方,仿仲景旋复代赭石 
汤之义,重用赭石至八钱,以开胃镇冲,即以下通大便(此证大便多艰),而即用人参以驾驭之,俾气化 
旺而流通,自能撑悬贲门使之宽展,又佐以半夏、知母、当归、天冬诸药,以降胃、利痰、润燥、生津, 
用之屡见效验。迨用其方既久,效者与不效者参半,又有初用其方治愈,及病又反复再 
服其方不效者。再三踌躇,不得其解,亦以为千古难治之证,原 
不能必其全愈也。后治一叟,年近七旬,住院月余,已能饮食,而终觉不脱然。迨其回家年余,仍以旧证病 
故,濒危时吐出脓血若干,乃恍悟从前之不能脱然者,系贲门有瘀血肿胀也,当时若方中加破血之药,或 
能全愈。盖愚于瘀血致噎之证,素日未有经验,遂至忽不留心。后读吴鞠通、杨素园论噎膈,亦皆注重瘀血 
之说,似可为从前所治之叟亦有瘀血之确征。而愚于此案,或从前原有瘀血,或以后变为瘀血,心中仍有游 
移。何者?以其隔年余而后反复也。迨辛酉孟夏阅天津《卢氏医学报》百零六期,谓胃癌由于胃瘀血,治此 
证者兼用古下瘀血之剂,屡屡治愈,又无再发之 ,觉胸中疑团顿解。盖此证无论何因,其贲门积有瘀血者 
十之七八。其瘀之重者,非当时兼用治瘀血之药不能愈。其瘀之轻者,但用开胃降逆之药,瘀血亦可些 
些消散,故病亦可愈,而究之瘀血之根蒂未净,是以有再发之 也。 
古下瘀血之方,若抵当汤、抵当丸、下瘀血汤、大黄 虫丸诸方,可谓能胜病矣。而愚意以为欲治此证, 
必中、西之药并用,始觉有把握。盖以上诸方治瘀血虽有效,以消瘤赘恐难见效。西医名此证为胃癌,所 
谓癌者因其处起凸若山之有岩也。其中果函有瘀血,原可用消瘀血之药消之。若非函有瘀血,但用消瘀血之 
药,即不能消除。夫人之肠中可生肠蕈,肠蕈即瘤赘也。肠中可生瘤赘,即胃中亦可生瘤赘。而消瘤赘之药, 
惟西药沃剥即沃度加HT 谟最效,此其在变质药中独占优胜之品也。今愚合中、西药品,拟得一方于下,以备试用。 
【变质化瘀丸】 
旱三七(一两细末) 桃仁(一两炒熟细末) 硼砂(六钱细末) 粉甘草(四钱细末) 
西药沃剥(十瓦) 百布圣(二十瓦) 
上药六味调和,炼蜜为丸,二钱重。服时含化,细细咽津。 
今拟定治噎膈之法∶无论其病因何如,先服参赭培气汤两三剂,必然能进饮食。若以后愈服愈见效, 
七八剂后,可于原方中加桃仁、红花各数钱,以服至全愈为度。若初服见效,继服则不能递次见效者,可于 
原方中加三棱二钱, 虫钱半;再于汤药之外,每日口含化服变质化瘀丸三丸或四丸,久久当有效验。若其 
瘀血已成溃疡,而脓未尽出者,又宜投以山甲、皂刺、乳香、没药、花粉、连翘诸药,以消散之。 
此证之脉若见滑象者,但服参赭培气汤必愈。而服过五六剂 
后,可用药汤送服三七细末一钱,煎渣服时亦如此。迨愈后自无再发之 矣。 
王孟英谓,以新生小鼠新瓦上焙干,研末,温酒冲服,治噎 
膈极有效。盖鼠之性能消 瘕,善通经络,故以治血瘀贲门成噎膈者极效也。 
有一人患噎膈,偶思饮酒,饮尽一壶而脱然病愈。验其壶中,有蜈蚣一条甚巨,因知其病愈非由于饮酒, 
实由于饮煮蜈蚣之酒也。闻其事者质疑于愚。此盖因蜈蚣善消肿疡,患者必因贲门瘀血成疮致噎,故饮 
蜈蚣酒而顿愈也。欲用此方者,可用无灰酒数两(白酒黄酒皆可不宜用烧酒)煮全蜈蚣三条饮之。 
总论破瘀血之药,当以水蛭为最。然此物忌炙,必须生用之方有效。乃医者畏其猛烈,炙者犹不敢用, 
则生者无论矣。不知水蛭性原和平,而具有善化瘀血之良能。若服以上诸药而病不愈者,想系瘀血凝结甚固, 
当于服汤药、丸药之外,每用生水蛭细末五分,水送服,日两次。若不能服药末者,可将汤药中 虫减 
去,加生水蛭二钱。 
【附录】唐××登医志原文∶读杂志第四期张锡纯君论治噎膈,阐发玄微,于此证治法别开径面,卓 
见名言,实深钦佩。及 
又读侯××(西医)反胃论(见第三中学第二期杂志中),谓病原之最重要者,乃幽门之发生胃癌,妨碍 
食物入肠之道路。初时胃力尚佳,犹能努力排除障碍,以输运食物于肠。久而疲劳,机能愈弱,病势益 
进,乃成反胃。按∶其引西医之论反胃,言其原因同于噎膈,可以治噎膈之法治之,固属通论。然即愚 
生平经验以来,反胃之证原有两种,有因幽门生癌者;有因胃中虚寒兼胃气上逆、冲气上冲者。其幽门生癌 
者,治法原可通于噎膈。若胃中虚寒兼气机冲逆者,非投以温补胃府兼降逆镇冲之药不可。且即以胃中生 
癌论,贲门所生之癌多属瘀血,幽门所生之癌多属瘤赘。瘀血由于血管凝滞,瘤赘由于腺管肥大。治法亦宜各 
有注重,宜于参赭培气汤中加生鸡内金三钱,三棱二钱;于变质化瘀丸中加生水蛭细末八钱,再将西药沃剥改 
作十五瓦,蜜为丸,桐子大,每服三钱。日服两次。而后幽门所生之癌,若为瘤赘,可徐消,即为瘀血亦不难 
消除。又治噎膈便方,用昆布二两洗净盐,小麦二合,用水三大盏,煎至小麦烂熟,去渣,每服不拘时饮一 
小盏;仍取昆布不住口含两三片咽津,极效。按此方即用西药沃度加HT 谟之义也。盖西药之沃度加HT 谟原由 
海草烧灰制出,若中药昆布、海藻、海带皆含有沃度加HT 谟之原质者也。其与小麦同煮服者,因昆布味咸性 
凉,久服之恐与脾胃不宜,故加小麦以调补脾胃也。 
此方果效,则人之幽门因生瘤赘而反胃者,用之亦当有效也。 

<目录>三、医论
<篇名>66.论胃气不降治法
属性:阳明胃气以息息下行为顺。为其息息下行也,实时时借其下行之力,传送所化饮食达于小肠,以化乳糜; 
更传送所余渣滓,达于大肠,出为大便。此乃人身气化之自然,自飞门以至魄门,一气营运而无所窒碍者也。 
乃有时胃气不下行而转上逆,推其致病之由,或因性急多怒,肝胆气逆上干;或因肾虚不摄,冲中气 
逆上冲,而胃受肝胆冲气之排挤,其势不能下行,转随其排挤之力而上逆。迨至上逆习为故常,其下行 
之能力尽失,即无他气排挤之时,亦恒因蓄极而自上逆。于斯饮食入胃不能传送下行,上则为胀满,下则为便 
结,此必然之势也。而治之者,不知其病因在胃府之气上逆不下降,乃投以消胀之药,药力歇而胀满依然。 
治以通便之剂,今日通而明日如故,久之兼证歧出,或为呕哕,或为呃、为逆,或为吐衄,或胸膈烦热,或头 
目眩晕,或痰涎壅滞,或喘促咳嗽,或惊悸不寐,种种现证头绪纷繁,则治之愈难。即间有知其致病之由在 
胃气逆而不降者,而所用降胃之药若半夏、苏子、蒌仁、竹茹、浓朴、枳实诸品,亦用之等于不用也。 
而愚数十年经验以来,治此证者不知凡几,知欲治此证非重用赭石不能奏效也。盖赭石对于此证,其特长 
有六∶其重坠之力能引胃气下行,一也,既能引胃气下行,更能引胃气直达肠中以通大便,二也;因其饶有 
重坠之力,兼能镇安冲气使不上冲,三也;其原质系铁养化合,含有金气,能制肝木之横恣,使其气不上 
干,四也;为其原质系铁养化合,更能引浮越之相火下行,而胸膈烦热、头目眩晕自除,五也;其力能降胃 
通便,引火下行,而性非寒凉开破,分毫不伤气分,因其为铁养化合转能有益于血分(铁养化协议于铁锈故 
能补血中之铁锈),六也。是以愚治胃气逆而不降之证,恒但重用赭石,即能随手奏效也。 
丙寅季春,愚自沧州移居天津。有郭××者,年近三旬,造寓求诊。自言心中常常满闷,饮食停滞胃 
中不下,间有呕吐之时,大便非服通利之品不行,如此者年余,屡次服药无效,至今病未增剧,因饮食减少 
则身体较前羸弱矣。诊其脉,至数如常,而六部皆有郁象。因晓之曰∶“此胃气不降之证也,易治耳。但 
重用赭石数剂即可见效也。”为疏方,用生赭石细末一两,生怀山药、炒 
怀山药各七钱,全当归三钱,生鸡内金二钱,浓朴、柴胡各一 
钱。嘱之曰∶“此药煎汤日服一剂,服至大便日行一次再来换方。” 
时有同县医友李××在座,亦为诊其脉,疑而问曰∶“凡胃气不降之病,其脉之现象恒弦长有力。 
今此证既系胃气不降,何其六脉皆有郁象,而重按转若无力乎?”答曰∶“善哉问也,此中颇有可研究之 
价值。盖凡胃气不降之脉,其初得之时,大抵皆弦长有力,以其病因多系冲气上冲,或更兼肝气上干。冲气上 
冲,脉则长而有力;肝气上干,脉则弦而有力;肝冲并见,脉则弦长有力也。然其初为肝气冲气之所迫,其胃 
府之气不得不变其下行之常而上逆,迨其上逆既久,因习惯而成自然,即无他气冲之干之,亦恒上逆而不能 
下行。夫胃居中焦,实为后天气化之中枢。故胃久失其职,则人身之气化必郁,亦为胃久失其职,则人 
身之气化又必虚,是以其脉之现象亦郁而且虚也。为其郁也,是以重用赭石以引胃气下行,而佐以浓朴以通 
阳(叶天士谓浓朴多用则破气,少用则通阳),鸡内金以化积,则郁者可开矣。为其虚也,是以重用山药生、 
熟各半,取其能健脾兼能滋胃(脾湿胜不能健运,宜用炒山药以健之,胃液少不能化食,宜用生山药以滋之), 
然后能受开郁之药,而无所伤损。用当归者,取其能生血兼能润便补虚,即以开郁也。用柴胡者,因人身之 
气化左宜升、右宜降,但重用镇降之药,恐有妨于气化之自然,故少加柴胡以宣通之,所以还其气化之常 
也。”李××闻之,深韪愚言。后其人连服此药八剂,大便日行一次,满闷大减,饮食加多。遂将赭石改 
用六钱,柴胡改用五分,又加白术钱半。连服十剂全愈。阅旬日,李××遇有此证,脉亦相同,亦重用赭石治愈。 

<目录>三、医论
<篇名>67.论吐血衄血之原因及治法
属性:《内经》厥论篇谓“阳明厥逆衄呕血”,此阳明指胃腑而言 
也。盖胃腑以熟腐水谷,传送饮食为职,其中气化,原以息息下 
行为顺。乃有时不下行而上逆,胃中之血亦恒随之上逆。其上逆之极,可将胃壁之膜排挤破裂,而成呕血 
之证;或循阳明之经络上行,而成衄血之证。是以《内经》谓阳明厥逆衄呕血也。由此知∶无论其证之或虚 
或实,或凉或热,治之者,皆当以降胃之品为主。而降胃之最有力者,莫赭石若也,故愚治吐衄之证,方中 
皆重用赭石,再细审其胃气不降之所以然,而各以相当之药品辅之。兹爰将所用之方,详列于下。 
【平胃寒降汤】治吐衄证脉象洪滑重按甚实者,此因热而胃气不降也。 
生赭石(一两轧细) 栝蒌仁(一两炒捣) 生杭芍(八钱) 嫩竹茹(三钱细末) 牛蒡子 
(三钱捣碎) 甘草(钱半) 
此拙拟寒降汤,而略有加减也。服后血仍不止者,可加生地黄一两,三七细末三钱(分两次用头煎二煎之汤送服)。 
吐衄之证,忌重用凉药及药炭强止其血。因吐衄之时,血不归经,遽止以凉药及药炭,则经络瘀塞,血 
止之后,转成血痹虚劳之证。是以方中加生地黄一两,即加三七之善止血兼善化瘀血者以辅之也。 
【健胃温降汤】治吐衄证脉象虚濡迟弱,饮食停滞胃口,不能下行,此因凉而胃气不降也。 
生赭石(八钱轧细) 生怀山药(六钱) 白术(四钱炒) 干姜(三钱) 
清半夏(三钱温水淘净矾味) 生杭芍(二钱) 浓朴(钱半) 
此方原名温降汤,兹则于其分量略有加减也。方中犹用芍药者,防肝中所寄之相火不受干姜之温热也。 
吐衄之证因凉者极少,愚临证四十余年,仅遇两童子,一因 
凉致胃气不降吐血,一因凉致胃气不降衄血,皆用温降汤治愈, 
其详案皆载原方之后,可参观。 
【泻肝降胃汤】治吐衄证左脉弦长有力,或肋下胀满作疼, 
或频作呃逆,此肝胆之气火上冲胃腑,致胃气不降而吐衄也。 
生赭石(八钱捣细) 生杭芍(一两) 生石决明(六钱捣细) 栝蒌仁(四钱炒捣) 
甘草(四钱) 龙胆草(二钱) 净青黛(二钱) 
此方因病之原因在胆火肝气上冲,故重用芍药、石决明及龙 
胆、青黛诸药,以凉之、镇之。至甘草多用至四钱者,取其能缓肝之急,兼以防诸寒凉之药伤脾胃也。 
【镇冲降胃汤】治吐衄证右脉弦长有力,时觉有气起在下 
焦,上冲胃腑,饮食停滞不下,或频作呃逆,此冲气上冲,以致胃不降而吐衄也。 
生赭石(一两轧细) 生怀山药(一两) 生龙骨(八钱捣细) 生牡蛎(八钱捣细) 
生杭芍(三钱) 甘草(二钱) 广三七(细末二钱分两次用头煎二煎之汤送服) 
方中龙骨、牡蛎,不但取其能敛冲,且又能镇肝,因冲气上冲之由,恒与肝气有关系也。 
【滋阴清降汤】治吐衄证失血过多,阴分亏损,不能潜阳而作热,不能纳气而作喘,甚或冲气因虚 
上干,为呃逆、眩晕、咳嗽,心血因不能内荣,为怔忡、惊悸、不寐,脉象浮数重按无力者。 
生赭石(八钱轧细) 生怀山药(一两) 生地黄(八钱) 生龙骨(六钱捣细) 
生牡蛎(六钱捣细) 生杭芍(四钱) 甘草(二钱) 广三七(细末二钱分两次用头煎二煎之汤送服) 
此方即清降汤,加龙骨、牡蛎、地黄、三七也。原方所主之 
病,原与此方无异,而加此数味治此病尤有把握。此因临证既多,屡次用之皆验,故于原方有所增加也。 
【保元清降汤】治吐衄证血脱气亦随脱,言语若不接续,动则作喘,脉象浮弦,重按无力者。 
生赭石(一两轧细) 野台参(五钱) 生地黄(一两) 生怀山药(八钱) 净萸肉(八钱) 生 
龙骨(六钱捣细) 生杭芍(四钱) 广三七(细末三钱分两次用头煎二煎之汤送服) 
此方曾载吐衄门,而兹则略有加减也。 
【保元寒降汤】治吐衄证血脱气亦随脱,喘促咳逆,心中烦热,其脉上盛下虚者。 
生赭石(一两轧细) 野台参(五钱) 生地黄(一两) 知母(八钱) 净萸肉(八钱) 生龙骨(六 
钱捣细) 生牡蛎(六钱捣细) 生杭芍(四钱) 广三七(细末三钱捣分两次用头煎二煎药汤送服) 
此方亦载于吐衄门中,而兹则略有更改也。至于医方所载此二方之原方,非不可用,宜彼宜此之间,细 
为斟酌可也。 
上所列诸方,用之与病因相当,大抵皆能奏效。然病机之呈露多端,病因即随之各异,临证既久,所治 
愈吐衄之验案,间有不用上列诸方者,如拙拟秘红丹及补络补管汤等方后各案,可互相参观。 
吐衄证最忌黄 、升、柴、桔梗诸药,恐其能助气上升血亦 
随之上升也。若确知病系宗气下陷,可以放胆用之,然必佐以龙骨、牡蛎,以固血之本源,始无血随气升之虞也。 
然吐衄证之因宗气下陷者极少,愚临证四十余年,仅遇赵姓一人,再四斟酌,投以升陷汤加龙骨、牡蛎 
治愈,然此方实不可轻试也。近津沽有张姓,年过三旬,患吐血证,医者方中有柴胡二钱,服后遂大吐不止。 
仓猝迎愚诊视,其脉弦长有力,心中发热,知系胃气因热不降也。所携药囊中,有生赭石细末约两余, 
俾急用水送服强半。候约十二分钟,觉心中和平,又送服其余,其吐顿止。继用平胃寒降汤调之,全愈。是知 
同一吐血证也,有时用柴胡而愈,有时用柴胡几至误人性命,审证时岂可不细心哉。 
至于妇女倒经之证,每至行经之期,其血不下行而上逆作吐衄者,宜治以四物汤去川芎,加怀牛膝、 
生赭石细末,先期连服数剂可愈。然其证亦间有因气陷者,临证时又宜细察。曾治一室女吐血,及一少妇 
衄血,皆系倒行经证,其脉皆微弱无力,气短不足以息,少腹时有气下坠,皆治以他止血之药不效,后再三斟 
酌,皆投以升陷汤,先期连服,数日全愈。总之,吐衄之证,大抵皆因热而气逆,其因凉气逆者极少,即兼 
冲气肝气冲逆,亦皆挟热,若至因气下陷致吐衄者,不过千中之一二耳。 
天津赵××,年近三旬,病吐血,经医治愈,而饮食之间若稍食硬物,或所食过饱,病即反复。诊其 
六脉和平,重按似有不足,知其脾胃消化弱,其胃中出血之处,所生肌肉犹未撤消,是以被食物撑挤,因 
伤其处而复出血也。斯当健其脾胃,补其伤处,吐血之病庶可除根。为疏方用生山药、赤石脂各八钱, 龙 
骨、 牡蛎、净萸肉各五钱,白术、生明没药各三钱,天花粉、甘草各二钱。按此方加减,服之旬余,病遂 
除根。此方中重用石脂者,因治吐衄病凡其大便不实者,可用之以代赭石降胃。盖赭石能降胃而兼能通大便, 
赤石脂亦能降胃而转能固大便,且其性善保护肠胃之膜,而有生肌之效,使胃膜因出血而伤者可速愈也。 
或问∶方书治吐衄之方甚多,今详论吐衄治法,皆系自拟,岂治吐衄成方皆无可取乎?答曰∶非也。 
《金匮》治吐衄有泻心汤,其方以大黄为主,直入阳明,以降胃气,佐以黄芩,以清肺 
金之热,俾其清肃之气下行,以助阳明之降力,黄连以清心火之 
热,俾其亢阳默化潜伏,以保少阴之真液,是泻之适所以补之也。凡因热气逆吐衄者,至极危险之时用之, 
皆可立止。血止以后,然后细审其病因,徐为调补未晚也。然因方中重用大黄,吐衄者皆不敢轻服,则良方 
竟见埋没矣。不知大黄与黄连并用,但能降胃,不能通肠,虽吐衄至身形极虚,服后断无泄泻下脱之 
弊。乃素遇吐衄证,曾开此方两次,病家皆不敢服,遂不得已另拟平胃寒降汤代之,此所以委曲以行其救 
人之术也。 
《金匮》有柏叶汤方,为治因寒气逆以致吐血者之良方也。故其方中用干姜、艾叶以暖胃,用马通 
汁以降胃,然又虑姜、艾之辛热,宜于脾胃,不宜于肝胆,恐服药之后,肝胆所寄之相火妄动,故又用柏叶 
之善于镇肝且善于凉肝者以辅之。此所谓有节制之师,先自立于不败之地,而后能克敌致胜也。至后世薛立斋 
谓,因寒吐血者,宜治以理中汤加当归,但知暖胃,不知降胃,并不知镇肝凉肝,其方远逊于柏叶汤矣。 
然此时有喜服西药,恒讥中药为不洁,若杂以马通汁,将益嫌其不洁矣,是以愚另拟健胃温降汤以代之也。 
近时医者治吐衄,喜用济生犀角地黄汤。然其方原治伤寒胃火热盛以致吐血、衄血之方,无外感而吐 
衄者用之,未免失于寒凉,其血若因寒凉而骤止,转成血痹虚劳之病。至愚治寒温吐衄者,亦偶用其方, 
然必以其方煎汤送服三七细末二钱,始不至血瘀为恙。若其脉左右皆洪实者,又宜加羚羊角二钱,以泻肝胆之 
热,则血始能止。 
至葛可久之十灰散,经陈修园为之疏解,治吐衄者亦多用之。夫以药炭止血,原为吐衄者所甚忌, 
犹幸其杂有大黄炭(方下注灰存性即是炭),其降胃开瘀之力犹存,为差强人意耳。其方遇吐 
衄之轻者,或亦能奏效,而愚于其方,实未尝一用也。至于治吐衄便方,有用其吐衄之血 作炭服者, 
有用发 (即剃下之短发) 作炭 
服者,此二种炭皆有化瘀生新之力,而善止血,胜于诸药之炭但能止血而不能化瘀血以生新血者远矣。 
方书有谓血脱者,当先益其气,宜治以独参汤。然血脱须有分别,若其血自二便下脱,其脉且微弱无 
力者,独参汤原可用。若血因吐衄而脱者,纵脉象微弱,亦不宜用。夫人身之阴阳原相维系,即人身之气血 
相维系也。吐衄血者因阴血亏损,维系无力,原有孤阳浮越之虞,而复用独参汤以助其浮越,不但其气易上奔 
(喻嘉言谓气虚欲脱者但服人参转令气高不返),血亦将随之上奔而复吐衄矣。是拙拟治吐 
衄方中,凡用参者,必重用赭石辅之,使其力下达也。 
寻常服食之物,亦有善止血者,鲜藕汁、鲜莱菔汁是也。曾见有吐衄不止者,用鲜藕自然汁一大盅温饮 
之(勿令熟),或鲜莱菔自然汁一大盅温饮之,或二汁并饮之,皆可奏效。 
有堂兄××,年五旬,得吐血证,延医治不效,脉象滑动,按之不实。时愚年少,不敢轻于疏方,遂 
用鲜藕、鲜白茅根各四两,切碎,煎汤两大碗,徐徐当茶饮之,数日全愈。自言未饮此汤时,心若虚悬无着, 
既饮之后,若以手按心还其本位,何其神妙如是哉!隔数日,又有邻村刘姓少年患吐血证,其脉象有力,心 
中发热,遂用前方,又加鲜小蓟根四两,如前煮汤饮之,亦愈。因名前方为二鲜饮,后方为三鲜饮。 
至于咳血之证,上所录医案中间或连带论及,实非专为咳血发也。因咳血原出于肺,其详细治法皆载 
于治肺病方中,兹不赘。 

<目录>三、医论
<篇名>68.论痢证治法
属性:(附∶开胃资生丹) 
唐容川曰∶“《内经》云∶‘诸呕吐酸,暴注下迫,皆属于热。’下迫与吐酸同言,则知其属于肝热 
也。仲景于下利后重便脓血者,亦详于厥阴篇中,皆以痢属肝经也。盖痢多发于秋,乃 
肺金不清,肝木遏郁。肝主疏泄,其疏泄之力太过,则暴注里急,有不能待之势。然或大肠开通,则 
直泻下矣。乃大肠为肺金之腑,金性收涩,秋日当令,而不使泻出,则滞塞不得快利,遂为后重。是以治痢 
者,开其肺气,清其肝火,则下痢自愈。”,此论甚超妙,其推详痢之原因及治痢之法,皆确当。愚今特 
引伸其说,复为详悉言之。盖木虽旺于春,而其发荣滋长实在于夏。故季夏六月为未月,未者,木重叶也, 
言木至此旺之极也。而肝脏属木,故于六月亦极旺。肝木过旺而侮克脾土,是以季夏多暴注下泻之证,而痢证 
甚少,因肺金犹未当令,其收涩之力甚微也。即其时偶有患痢者,亦多系湿热酿成,但利湿清热,病即可愈。 
是以六一散为治暑痢之定方,而非所论于秋日之痢也。迨至已交秋令,金气渐伸,木气渐敛,人之脏腑原可 
安于时序之常,不必发生痢证也。惟其人先有蕴热,则肝木乘热恣肆,当敛而不敛,又于饮食起居之间感受 
寒凉,肺金乘寒凉之气,愈施其肃降收涩之权,则金木相犯,交迫于肠中,而痢作矣。是知痢之成也,固 
由于金木相犯,而金木之相犯,实又因寒火交争之力以激动之也。若唐氏所谓开肺清肝,原为正治之法。然 
止可施于病之初起,非所论于痢病之已深也。且统观古今治痢之方,大抵皆用之于初期则效,用之于末期则 
不效。今特将痢证分为数期,详陈其病之情状及治法于下。 
痢之初得也,时时下利脓血,后重,肠疼,而所下脓则甚稠,血则甚鲜,腹疼亦不甚剧,脉之滑实者, 
可用小承气汤加生杭芍四钱,甘草二钱下之。盖方中朴、实原可开肺;大黄、芍药又善清肝;且浓朴温而黄、 
芍凉,更可交平其寒热,以成涤肠荡滞之功;加甘草者,取其能调胃兼能缓肝,即以缓承气下降之力也。 
其脉按之不实者,可治以拙拟化滞汤。若当此期不治,或治 
以前方而仍不愈,或迁延数旬或至累月,其腹疼浸剧,所下者虽未甚改色,而间杂以脂膜,其脉或略数或 
微虚,宜治以拙拟燮理汤。愚生平用此方治愈之人甚多,无论新痢、久痢皆可用。 
用上方虽新痢、久痢皆可奏效,而其肠中大抵未至腐烂也。乃有腹中时时切疼后重,所下者多如烂炙,杂 
以脂膜,是其肠中已腐烂矣,当治以拙拟通变白头翁汤。方中之意∶用白头翁、秦皮、芍药、生地榆以清热; 
三七、鸦胆子以化瘀生新,治肠中腐烂,而又重用生山药以滋其久耗之津液,固其已虚之气化,所以奏效甚 
捷也。愚在奉时,有王××下痢甚剧,曾以此方治愈,其详案载此方之后可考也。至素有鸦片嗜好者,无论 
其痢之初得及日久,皆宜治以此方,用之屡建奇功。至地榆方书多炒炭用之,而此方生用者,因生用性凉, 
善保人之肌肤,使不因热溃烂。是以被汤火伤肌肤者,用生地榆为末,香油调敷立愈。痢之热毒侵入肠中 
肌肤,久至腐烂,亦犹汤火伤人肌肤至溃烂也,此地榆之所以生用也。至白头翁汤原方,原白头翁、秦皮 
与黄连、黄柏并用,方中药品若此纯用苦寒者,诚以其方本治厥阴热痢,原挟有伤寒实 
热。今用以治痢久肠中腐烂,故不得不为变通也。 
上之痢证,又可治以拙拟生化丹。为其虚甚,加生怀山药一两。先用白糖水送服三七、鸦胆子各一半,再 
将余四味煎汤服。至煎渣服时,仍先用白糖水送服所余之三七、鸦胆子,再煎服汤药。盖痢证至此,西人谓 
之肠溃疡,不可但以痢治,宜半从疮治,是以用金银花、粉甘草以解疮家之热毒;三七、鸦胆子以化瘀生 
新;而鸦胆子味至苦,且有消除之力(捣膏能点疣),又可除痢证传染之毒菌;用芍药泄肝火,以治痢之本 
病;又恐其痢久伤阴,及下焦气化不固,是以又重用生山药以滋阴液固气化,此所以投之必 
效也(医方篇本方后载有医案可参观)。当愚初拟此方时,犹未见西人肠溃疡之说。 
及后见西书,其所载治法,但注重肠溃疡,而不知兼用药清痢之 
本源,是以不如此方之效也。 
有下痢日久,虚热上蒸,饮食减少,所下者形如烂炙,杂以脂膜,又兼腐败之色,腥臭异常,腹中时时 
切疼益甚者,此腹中生机将断,其为病尤重矣。宜治以前方,再加潞党参、天门冬各三钱。此用参以助 
其生机,即用天冬以调剂参之热也。 
有原素伤烟色,肾经虚惫,复下痢日久,肠中欲腐烂,其下焦之气化愈虚脱而不能固摄者,宜治以拙拟 
三宝粥。方中之意∶用三七、鸦胆子以治肠中之腐烂,用山药粥以补下焦之虚脱也。 
有下痢或赤、或白、或赤白参半,后重腹疼,表里俱觉发热,服凉药而热不退,痢亦不愈,其脉确有 
实热者。此等痢证原兼有外感之热,其热又实在阳明之腑,非少阴篇之桃花汤所能愈,亦非厥阴篇之白头翁汤 
所能愈也。惟治以拙拟通变白虎加人参汤则随手奏效。痢证身热不休,服清火药而热亦不休者,方书多诿为 
不治。然治果对证,其热焉有不休之理?此诚因外感之热邪随痢深陷,永无出路,以致痢为热邪所助,日甚 
一日,而永无愈期。治以此汤,以人参助石膏,能使深陷之热邪徐徐上升外散,消解无余,加以芍药、甘草 
以理后重腹疼,生山药以滋阴固下,连服数剂,热退而痢亦遂愈。方中之药原以芍药代知母,生山药代粳米, 
与白虎加人参汤之原方犹相仿佛,故曰通变白虎加人参汤也。愚生平用此方治愈此等痢证甚多(医方篇本方后 
载有数案可参观也)。 
此外感之热与痢相并,最为险证。尝见东人×××着有《赤痢新论》,大为丁仲祜所推许。然其中载有 
未治愈之案二则∶一体温至38.7℃,脉搏至百一十至,神识蒙昏,言语不清,舌肿大干燥,舌苔剥离,显然 
夹杂外感之实热可知,乃东人不知以清其外感实热为要务,而惟日注射以治痢之血清,竟至不救;其二发剧 
热,夜发躁狂之举动,后则时发谵语,体温达40.2℃,此又显然有外感之大热也。案中未载治法,想 
其治法,亦与前同,是以亦 
至不救。设此二证若治以拙拟之通变白虎加人参汤,若虑病重药轻,可将两剂并作一剂,煎汤四五茶杯,分多 
次徐徐温饮下,病愈不必尽剂,其热焉有不退之理?大热既退,痢自随愈。而东人见不及此者,因东人尽弃旧 
日之中学,而专尚西学也。盖中、西医学原可相助为理,而不宜偏废,吾国果欲医学之振兴,固非 
沟通中、西不可也。 
上所论之痢证乃外感之热已入阳明之腑者也。然痢证初得,恒有因外感束缚而激动其内伤者,临证者宜 
细心体察。果其有外感束缚也,宜先用药解其外感,而后治痢;或加解表之药于治痢药中;或用治痢药煎汤 
送服西药阿斯匹林瓦许亦可解表。设若忽不加察,则外感之邪随痢内陷,即成通变白虎加人参汤所主之险 
证,何如早治为愈也。 
痢证虽为寒热凝滞而成,而论者多谓白痢偏寒,赤痢偏热。然此为痢证之常,而又不可概论也。今 
试举治愈之案以明之。同庄张××妻,年近六旬,素多疾病。于季夏晨起,偶下白痢,至暮十余次。秉烛后, 
忽周身大热,昏不知人,循衣摸床,呼之不应,其脉洪而无力,肌肤之热烙手。知其痢因伤暑而成,且多病 
之身不禁暑热之熏蒸,所以若是昏沉也。急用生石膏三两,野台参四钱,煎汤一大碗,俾徐徐温饮下,至 
夜半尽剂而醒。诘朝煎渣再服,热退痢亦遂愈。此纯系白痢而竟若是之热也。 
有前后连两次病痢,其前后寒热不同者,为细诊其脉,前后迥异,始能用药各得其宜,无所差误。曾 
治刘××,于初秋得痢证甚剧。其痢脓血稠粘,脉象弦细,重诊仍然有力。治以通变白头翁汤,两剂全愈。隔 
旬余,痢又反复,自用原方治之,病转增剧,复来院求诊。其脉弦细兼迟,不任循按,知其已成寒痢,所 
以不受原方也。俾用生怀山药细末煮粥,送服小茴香细末一钱、生硫黄细末四分, 
数次全愈。又如三宝粥方后治愈卢氏妇一案亦然。 
二案皆随病机之转移,而互治以凉热之药,自能随手奏效。故治之者宜于临证时细心研究,息息与病机相符也。 
有痢证,上热下凉,所用之药宜上下分途,以凉治上,以热治下者。曾治天津张姓媪,年近五旬,于孟 
秋患痢,两旬不愈。所下者赤痢杂以血水,后重腹疼,继则痢少泻多,亦兼泻血水,上焦烦热,噤口不食, 
闻食味即恶心欲呕,头目眩晕,不能起床,其脉关前浮弦,重诊不实,两尺则微弱无根,一息五至,病患自觉 
心中怔忡,精神恍惚,似难支持,此乃虚极将脱之兆也。遂急用净萸肉、生怀山药各一两,大熟地、龙眼肉、 
白龙骨各五钱,生杭芍、云苓片、炙甘草各二钱,俾煎汤两盅,分两次温服下。初服一次,心神即觉安稳。尽 
剂后,少进饮食,泻痢亦少止。又即原方加生地黄四钱,炙甘草改用三钱,煎汤两盅,分两 
次温服下,每服一次送服生硫黄细末二分半,日服一剂,数日全愈。 
至于暑天热痢,宜治以六一散,前已言之。然南方之暑热兼湿,用六一散诚为至当;北方之暑热恒 
不兼湿,且有兼燥之时,若用六一散时,原当有所变通。愚尝拟得一方,用之甚效。方用滑石、生石膏各五 
钱,朱砂、粉甘草细末各二钱,薄荷冰一分,共和匀,每服二钱,开水送下。热甚痢剧者,一日可服五六次。 
名之曰加味益元散,盖以六一散加朱砂为益元散,兹则又加石膏、薄荷冰也。 
按∶暑热之痢恒有噤口不食者,而治以加味益元散,即可振兴其食欲。若非暑热之痢而亦不思饮食者, 
宜用朱砂、粉甘草细末等分,少加薄荷冰,每服一钱,竹茹煎汤送下,即可思食。盖此等证多因肝胆之火挟 
胃气上逆,其人闻食味即恶心欲呕,所以不能进食,用朱砂以降胃镇肝,甘草以和胃缓肝,竹茹以平其逆 
气,薄荷冰以散其郁热,所以服之即效也。因此方屡次奏功,遂 
名之曰开胃资生丹。 
有当暑热之时,其肝胆肠胃先有蕴热,又更奔走作劳于烈日之中,陡然下痢,多带鲜血,其脉洪大者。 
宜治以大剂白虎汤,煎数盅,分数次温饮下,每次送服鸦胆子仁三十粒。若其脉虽洪大而按之虚者,宜治以大 
剂白虎加人参汤,送服鸦胆子仁。 
有痢久清阳下陷者,即胸中大气因痢下陷也。其病情常觉下坠腹疼(此气分下陷迫其下焦腹疼),或痢或 
泻,多带虚气,呼吸短气,或兼有寒热往来,其脉象迟弱者,宜治以拙拟升陷汤,去知母,加生怀山药六钱, 
白头翁三钱。盖原方之意∶原用生箭 以升补胸中大气,而以柴胡、桔梗、升麻之善升清阳者以辅之,更加 
知母以调剂黄 之热也。兹因下焦泻痢频频,气化不固,故以白头翁易知母,而更以山药辅之。因知母之 
性寒而滑,白头翁之性凉而涩,其凉也能解黄 之热,其涩也能固气化之脱,且为治痢要药,伍 
以山药,又为止泻之要药也。 
诸痢之外又有所谓休息痢者,其痢大抵皆不甚重而不易除根,治愈恒屡次反复,虽迁延日久而犹可支持, 
有若阿米巴痢之轻者,至累年累月不愈而犹可支持也。或此等痢即阿米巴痢欤?须待后实验。然其所以屡次 
反复者,实因有原虫伏于大小肠曲折之处,是以愈而复发,惟用药除净其原虫则不反复矣。至除之之 
法,证之近于热者,可用鸦胆子仁,以治痢之药佐之;近于凉者,可用硫黄末,而以治痢之药佐之。再者,无 
论或热或凉,所用药中皆宜加木贼一钱,为其性善平肝,又善去肠风止血,故后世本草谓其善治休息痢也。 
其脾胃不健壮者,又宜兼用健补脾胃之药以清痢之上源,自能祓除病根也。 
痢证又有日下痢频频,其肠中仍有燥结,必去其燥结而痢始愈者,此固属罕见之证,而治痢者实不可 
不知也。表弟刘××,年二十四岁,于中秋下痢,脓血稠粘,一日十五六次,腹疼后重 
甚剧。治以化滞汤,连服两剂,下痢次数似少减,而后重腹疼如旧。细诊其脉,尺部重按甚实,疑其肠 
有结粪,投以小承气汤加生杭芍数钱,下燥粪长约四寸,后重腹疼顿愈十之八九。再与以化滞汤一剂,病若失。 
治痢最要药品,其痢之偏热者,当以鸦胆子为最要之药,其痢之偏寒者,当以硫黄为最要之药,以此二药 
皆有消除痢中原虫之力也。此二种药,上所录方案中已屡言之。今再论之。 
鸦胆子,一名鸭蛋子,其性善凉血,止血,兼能化瘀生新。凡痢之偏于热者,用之皆有捷效,而以治 
下鲜血之痢,泻血水之痢,则尤效。鸦胆子又善清胃腑之热,凡胃脘有实热充塞、噤口不食者,服之即可 
进食。服时须去其硬皮,若去皮时其中仁破者,即不宜服,因破者服后易消,其苦味遽出,恒令人呕吐,是 
以治痢成方,有用龙眼肉包鸦胆子仁囫囵吞服者。药局中秘方,有将鸦胆子仁用益元散为衣,名之为菩提 
丹者,是皆防其入胃即化出其苦味也,若以西药局中胶囊盛之吞服,虽破者亦可用。 
硫黄原禀火之精气,其挟有杂质者有时有毒,若其色纯黄,即纯系硫质,分毫无毒,为补相火暖下焦 
之主药。痢证下焦凉者,其上焦恒有虚热,硫黄质重,其热力直达下焦而不至助上焦之虚热。且痢之寒者 
虽宜治以热药,而仍忌温补收涩之品。至硫黄,诸家本草谓其能使大便润、小便长,西人谓系轻泻药之品,是 
其性热而能通,故以治寒痢最宜也。愚屡次品验此药,人之因寒作泻者,服之大抵止泻之时多。更有五 
更泻证,服他药不效,而放胆服硫黄即愈者。又间有本系因寒作泻,服硫黄而泻转剧者,惟与干姜、白术、 
五味等药同用,则确能治因寒作泻而无更泻之弊。古方书用硫黄皆系制用,然制之则热力减,必须多服,有时 
转因多服而生燥,实不如少服生者之为愈也。且择其纯系硫质者 
用之,原分毫无毒,亦无须多方制之也。至其用量,若以治寒 
痢,一次可服二三分,极量至五六分,而以治他证,则不在此例。曾治邻村张氏妇,年二十余,胃寒作吐, 
所吐之食分毫不能消化(凡食后半日吐出不消化者皆系胃寒),医治半年无效,虽投以极热之药亦分毫 
不觉热,脉甚细弱,且又沉迟。知其胃寒过甚,但用草木之品恐难疗治。俾用生硫黄细末一两,分作十二包, 
先服一包,过两句钟不觉热,再服一包。又为开汤剂干姜、炙甘草各一两,乌附子、广油桂、补骨脂、于 
术各五钱,浓朴二钱,日煎服一剂。其硫黄当日服至八包,犹不觉热,然自此即不吐食矣。后数日,似 
又反复,遂于汤剂中加代赭石细末五钱,硫黄仍每日服八包,其吐又止。连服数日,觉微热,俾将硫黄减半, 
汤剂亦减半,惟赭石改用三钱。又服二十余日,其吐永不反复。愚生平用硫黄治病,以此证所用之量为最大。 
至于西药中硫黄三种,其初次制者名升华硫黄,只外用于疮疡,不可内服。用升华硫黄再制之,为 
精制硫黄,用精制硫黄再制之为沉降硫黄,此二种硫黄可以内服。然欲其热力充足,服之可以补助元阳、 
温暖下焦,究不若择纯质生硫黄服之之为愈也。 

<目录>三、医论
<篇名>69.论肠结治法
属性:肠结最为紧要之证,恒于人性命有关。或因常常呕吐,或因多食生冷及硬物,或因怒后饱食,皆可 
致肠结,其结多在十二指肠及小肠间,有结于幽门者。其证有腹疼者,有呕吐者,尤为难治。因投以开结 
之药,不待药力施展而即吐出也。亦有病本不吐,因所服之药行至结处不能通过,转而上逆吐出者。是以治 
此证者,当使服药不使吐出为第一要着。愚于此证吐之剧者,八九日间杓饮不存,曾用赭石细末五两,从中 
又罗出极细者一两,将所余四两煎汤,送服极细者,其吐止而结亦遂开。若结证在极危急之 
时,此方宜放胆用之。虽在孕妇恶阻呕吐者,亦可用之(赭石解参赭镇气汤后载 
有数案可参观),有谓孕妇恶阻,无论如何呕吐,与性命无关者,乃阅历未到之言也。 
有患此证急欲通下者,愚曾用赭石细末三两、芒硝五钱,煎汤送服甘遂细末钱半,服后两点半钟其结 
即通下矣。后有医者得此方以治月余之肠结证,亦一剂而愈。后闻此医自患肠结,亦用此方煎汤先服一半,甘 
遂亦送下一半,药力下行,结不能开,仍复吐出;继服其余一半,须臾仍然吐出,竟至不起。由此知用药一 
道,过于放胆固多失事,若过于小心亦多误事也。况甘遂之性,无论服多服少,初次服之尚可不吐;若 
连次服之,虽佐以赭石,亦必作吐。是以拙拟荡胸加甘遂汤,原用大剂大承气汤加赭石二两煎汤,送服甘遂 
细末二钱。方下注云∶若服一剂不愈者,须隔三日方可再服。此固欲缓服以休养其正气,实亦防其连服致吐 
也。至于赭石可如此多用者,以其原质为铁养化合,性甚和平,且善补血,不伤气分,虽多用于人无损也。 
特是药局中赭石,必火 醋激然后轧细,如此制法,则养气不全,不如径用生者之为愈也。况其虽为石类, 
与铁锈相近(铁锈亦铁养化合),即服生赭石细末,亦于人肠胃毫无伤损。若嫌上方中甘遂之性过猛烈者, 
本书载有硝菔通结汤方,药性甚稳善,惟制此药时,略费手续。方后载有治验两则,后又遇此证多次,皆以 
此方治愈。 

<目录>三、医论
<篇名>70.论结胸治法
属性:结胸之证,有内伤外感之殊。内伤结胸,大抵系寒饮凝于贲门之间,遏抑胃气不能上达,阻隔饮食不 
能下降。当用干姜八钱,赭石两半,川朴、甘草各三钱开之。其在幼童,脾胃阳虚,寒饮填胸,呕吐饮食成 
慢惊,此亦皆寒饮结胸证。可治以庄在田《福幼编》逐寒荡惊汤。若用其方寒痰仍不开,呕吐仍不能止者,可 
将方中胡椒倍用二钱。若非寒饮结胸,或为顽痰结胸,或为热痰 
结胸者,阻塞胸中之气化不能升降,甚或有碍呼吸,危在目前,欲救其急,可用硼砂四钱开水融化服之, 
将其痰吐出。其为顽痰者,可再用栝蒌仁二两,苦葶苈三钱(袋装)煎汤饮之,以涤荡其痰。其为热痰者, 
可于方中加芒硝四钱。有胸中大气下陷,兼寒饮结胸者,其证尤为难治。(曾治一赵姓媪,案详回阳升陷 
汤后,可参阅)。 
至于外感结胸,伤寒与温病皆有。伤寒降早可成结胸,温病即非降早亦可成结胸,皆外感之邪内陷与 
胸中痰饮互相胶漆也。无论伤寒、温病,其治法皆可从同。若《伤寒论》大陷胸汤及大陷胸丸,俱为治 
外感结胸良方,宜斟酌病之轻重浅深,分别用之。至拙拟之荡胸汤,亦可斟酌加减,以代诸陷胸汤、丸。 
有内伤结胸与外感结胸相并而成一至险之结胸证者。在奉天时曾治郝××,年四十余,心下痞闷杜塞, 
饮食不能下行,延医治不效。继入东人医院,治一星期,仍然无效。浸至不能起床,吐痰腥臭,精神昏愦。 
再延医诊视,以为肺病已成,又兼胃病,不能治疗。其家人惶恐无措,迎愚延医。其脉左右皆弦,右部则 
弦而有力,其舌苔白浓微黄,抚其肌肤发热,问其心中亦觉热,思食凉物,大便不行者已四五日,自言心中满 
闷异常,食物已数日不进,吐痰不惟腥臭,且又觉凉。愚筹思再四,知系温病结胸。然其脉不为洪而有力,而 
为弦而有力,且所吐之痰臭而且凉者何也?盖因其人素有寒饮,其平素之脉必弦,其平素吐痰亦必 
凉(平素忽不自觉,今因病温咽喉发热觉痰凉耳),因有温病之热与之混合,所以脉虽弦而 
仍然有力,其痰虽凉,而为温病之热熏蒸,遂至腥臭也。为疏方用蒌仁、生赭石细末各一两,玄参、知母 
各八钱,苏子、半夏、党参、生姜各四钱,煎汤冲服西药留苦四钱。一剂胸次豁然,可进饮食,右脉较前柔 
和,舌苔变白,心中犹觉发热,吐痰不臭,仍然觉凉。遂将原方前四味皆减半,加当归三钱,服后大便通 
下,心中益觉通豁。惟有时觉有凉痰自下发动,逆行上冲,周身 
即出汗。遂改用赭石、党参、干姜各四钱,半夏、白芍各三钱。川朴、五味、甘草各二钱,细辛一钱,连服 
数剂,寒痰亦消矣。此证原寒饮结胸与温病结胸相并而成,而初次方中但注重温病结胸,惟生姜一味为治寒 
饮结胸之药。因此二病之因,一凉一热,原难并治。若将方中之生姜改为干姜,则温病之热必不退。至若 
生姜之性虽热,而与凉药并用实又能散热。迨至温病热退,然后重用干姜以开其寒饮。此权其病势之缓急先 
后分治,而仍用意周匝,不至顾此失彼,是以能循序奏效也。 

<目录>三、医论
<篇名>71.论霍乱治法
属性:霍乱为最险要紧急之证,且其证分阴阳,阴证宜治以温药,阳证宜治以凉药,设或辨证不清,而凉热 
误投,必凶危立见。即辨证清矣,而用药凉热不爽,亦未必能救其强半也。己未孟秋,奉天霍乱盛行,吐泻转 
筋,甚者脉闭,身冷如冰,而心中发热,嗜饮凉水。愚断为阳证,而拟得急救回生丹一方,药性虽凉,然 
善发汗,且善解毒,能使内伏之毒热透表外出,而身之凉者可温,脉之闭者可现,服此方者,大抵皆愈。 
继又拟得卫生防疫宝丹方,于前方之中加辛香温通之药两味,俾其药性凉热适均,日服数十粒可暗消病根于 
无形。若含数粒,可省视病患不受传染。时有刘××见病者卧街头,吐泻转筋,病势垂危,而刘××适带有 
卫生防疫宝丹,与以数十粒,复至茶馆寻开水半盏,俾送下,须臾吐泻转筋皆愈,而可起坐矣。继有尚××, 
来院购防疫之药,即将卫生防疫宝丹二百包与之。其煤矿工人患霍乱者,或服八十粒,或服一百二十粒,皆完 
全救愈,由斯知卫生防疫宝丹之于霍乱,既可防之于未然,又可制之于既发,其功效亦不减急救回生丹也。 
【《时行伏阴刍言》李××评语原文】辛酉六月三十日,余方 
就诊戚家,不意长儿××(现年十二)大泻不止,及余回家,而吐亦作 
矣。其脉尤紧而迟,四末微麻,头疼,身热,无汗,口渴,此伏阴而兼外感也,遂投以先生所创之急救回 
生丹。小儿此证虽属伏阴,因有兼证,须兼解表,且先生谓此丹服之可温复得汗,故与之。从此可知无论伏 
阴霍乱,其病初起时,可先与此丹,令其得汗以减其势,而后再分途治之可也(若但系伏阴证先与以先生所 
制卫生防疫宝丹更妙)。乃服药后,须臾汗出,吐泻之势亦稍缓。继与以漂苍术三钱,枳壳二钱,浓朴钱半,西 
砂仁、广陈皮、炙甘草、苏叶各一钱,薄荷八分,加生姜、大枣,煎汤服之,未尽剂而愈。 
按∶其子兼外感,所以身热口渴。若但为伏阴,初则吐泻,继则身冷、转筋、目眶塌陷,无一不 
与霍乱相同,惟心中不觉发热,且四肢有拘急之象耳。斯实仿佛阴证霍乱,与《伤寒论》所载之霍乱相似, 
故其书所载复阳消阴法即系附子理中汤。今李××于其初得,谓可治以急救回生丹,且谓若治以卫生防疫宝 
丹更妙。盖卫生防疫宝丹,初服下觉凉,继则终归于热,因冰片、薄荷冰皆性热用凉也,况细辛、白芷原 
属温热之品,是以此丹之妙用,在上能清,在下能温耳。至急救回生丹,无辛、芷之热,朱砂又加重, 
药性似偏于凉矣,然朱砂原汞硫化合,凉中含有热性,况冰片、薄荷冰亦加多,发汗甚捷,服后无论新受之 
外感,久伏之邪气,皆可由汗透出。由斯观之,若果系阳证霍乱,即放 
胆投以急救回生丹,必能回生。若不能断其为阴为阳,即投以卫生防疫宝丹,亦无不效也。 
卫生防疫宝丹多服亦可发汗,无论霍乱因凉因热,用之皆效,并治一切暴病痧证,头疼,心烦,四 
肢作疼,泄泻,痢疾,呃逆(治此证尤效)。若无病者,每饭后服二十粒,能使饮食速消,饭量骤加, 
实为健胃良药。且每日服之,尤能预防一切杂证,不受传染。 
霍乱之证,有但用上二方不效者,其吐泻已极,奄奄一息将脱 
者是也。方书有谓霍乱为脱疫者,实指此候。此时无论病因为凉为 
热,皆当急用人参八钱以复其阳,生山药一两、生杭芍六钱以滋其阴,山萸肉八钱以敛肝气之脱(此证吐泻之 
始,肝木助邪侮土,吐泻之极而肝气转先脱将肝气敛住而元气可固),炙甘草三钱以和中气之漓,赭石细末 
四钱引人参之力下行即以防其呕吐,朱砂、童便(先用温热童便送服朱砂细末五分再煎服前药)以交其心肾。 
此方名急救回阳汤,实阴阳俱补也。心中觉热者,加天冬六七钱。身凉、脉不见、心中分毫不觉热者,去 
芍药,加乌附子一钱。若心中犹觉热,虽身凉脉闭,不可投以热药。汗多者,萸肉可用至两余。方中人参,若 
用野台参,即按方中分量,若用野山参,分量宜减半,另炖兑服。按∶此方当用于吐泻既止之后,若其势 
虽垂危,而吐泻犹未止,仍当审其凉热,用前二方,以清内毒,然后以此方继之。其服药距离时间,约在多 
半点钟。曾治奉天寇姓媪,霍乱吐泻一日夜,及愚诊视时,吐泻已止,周身皆凉,六脉闭塞,精神 
昏愦,闭目无声,而呼之仍有知觉,且恒蹙其额,知霍乱之毒犹扰乱于其心中也。问其吐泻时情状,常觉 
心中发热,频频嗜饮凉水,知其确系阳证。先与以急救回生丹三分之一,和温开水灌下。迟半点钟,视其形 
状较安,仍身凉无脉,俾煎急救回阳汤一剂,徐徐灌下,且嘱其服药以后,且不时少与以温开水。至翌晨,复 
为诊视,身热脉出,已能言语,仍自言心中热甚。遂用玄参二两,潞参一两,煎汤一大碗,俾徐 
徐温饮下,尽剂而愈。详观此案,当知用急救回阳汤之方针矣。 
上所拟治霍乱三方,急救回生丹宜于霍乱之偏热者;卫生防疫宝丹,宜于霍乱之偏凉者;急救回阳 
汤以救霍乱之将脱者。治霍乱之方,似已略备。然霍乱中间有大凉、大热之证,似宜另商治法,今更进而申 
论之。《伤寒论》之论霍乱也,主于寒,且主于大寒,若理中加附子、通脉四逆加人参诸方,皆治大寒 
之药也。然其各节中多言恶寒,四肢拘急,厥冷,或吐利汗出,或寒多不用水, 
必其病象中现如此形状,且脉象沉细欲无者,方可酌用《伤寒论》 
中诸方以急回其阳。阳回之后,间有觉烦热者,又宜急服凉润滋阴之药,以善其后。盖阳回其心脏跳动有力, 
则脉可复,身可热,吐泻亦可止。因其从前吐泻过剧,伤其阴分,是以阳回之后恒有觉烦热者,故又宜服凉 
润滋阴之药以善其后也。然此等证极少,愚经历霍乱多次,所治若此等证者不过四五焉。 
至霍乱之大热者,则恒有之。忆昔壬寅孟秋,邑中霍乱盛行,按凉治者多偾事,按热治者亦愈否参半, 
惟放胆恣饮新汲井泉水者皆愈,愚则重用羚羊角治愈此证若干。后愚问恣饮井泉水愈者数人,皆言彼时虽吐 
泻交作,脉微身凉,而心中则热而且渴,须臾难忍,惟恣饮凉水可稍解,饮后须臾复吐出,又须再饮,过半 
日凉水饮近一大水桶,热渴渐愈而吐泻亦止矣。按∶此原当饮以冰水,或食冰块,而乡村无冰, 
故以井泉水代之。 
丁卯季夏,天气炎热非常,愚临睡时偶食西瓜数块,睡至黎明,觉心中扰乱恶心,连吐三次,继 
又作泻。急服急救回生丹钱许,心中稍安。须臾病又如旧,且觉心中发热,火气上腾,右腿转筋,而身不凉, 
脉不闭。自知纯系热证。《千金方》治霍乱用治中汤(即理中汤),转筋者加石膏,是霍乱之兼热者原可重 
用石膏也。遂煎白虎加人参汤一大剂,服后病又稍愈。须臾仍然反复,心中热渴,思食冰。遂买冰若干, 
分作小块吞之,阅点半钟,约食冰二斤,热渴、吐泻俱止,而病若失矣。此虽因食凉物激动伏暑 
之热,然吐泻转筋非霍乱而何也?上二案皆证之大热者也,若无井泉水与无冰之处,可用鲜梨片或西瓜蘸 
生石膏细末食之,此愚治寒温之病阳明大热且呕不受药者之方也。究之其病发动之时,其大凉者仍宜先服卫 
生防疫宝丹,其大热者仍宜先服急救回生丹,因此二药皆能除毒菌、助心脏,使心脏不至受毒麻痹,病自 
无危险也。 
申济人(顺义县人)曰,“霍乱有阴阳之辨。若于六、七月间,或栖 
当楼窗,或夜卧露地,忽患上吐下泻、两腿筋抽、眼窝青、唇黑、身凉、有汗、脉沉伏者,此阴证也,急以针 
刺尺泽、少泽、委中(此穴宜深寸许)、十宣,若吐泻不止,刺中脘、水分,其病立愈。若身热、无汗、脉 
沉紧、腹疼甚、呕而不得上出、胀而不得下泻,此阳证也,急用针刺少商、委中、尺泽,腹疼不止,刺气海、 
章门、足三里,根据法灸刺,无不愈者。 
按∶此论辨阴阳之证颇精确。其谓阴证腿筋抽者,非转筋也,即《伤寒论》所谓四肢拘急也。若转筋, 
则阴阳之证皆有矣。其谓眼窝青、唇黑者,斯实阴证之明征。其谓身凉、脉沉伏者,阳证亦间有之,然阴证至 
此时恒恶寒,身欲浓复;阳证则始终不恶寒,即复以单被亦不欲。至其谓阴证有汗,阳证无汗,此论最确。又 
其论阴证,未言腹疼,论阳证则言腹疼甚,盖阳证邪正相争,仍有抗拒之力,其吐不得吐、泻不得泻者必然 
腹疼,即吐泻频频者亦恒腹疼,至阴证则邪太盛,正太衰,毫无抗拒之力,初得或犹有腹疼者,至吐泻数 
次后即不腹疼矣。至其以腹疼、吐不能吐、泻不能泻,名为干霍乱者,专属于阳证,尤具有特识。所论针刺十 
余穴皆为治此证要着,即不谙针灸者亦宜单习此十余穴,以备不时之需,且临时果能针药并用,证愈必速。总 
之证无论凉热,凡验其病原虫若蝌蚪形而曲其尾者,皆霍乱也。又天津鲍××曰∶“余遇纯阴霍乱,分毫不 
觉热者,恒用大块生姜切成方片,密排脐上两层,抟艾绒如枣大灸之,其吐泻转筋可立止。” 

<目录>三、医论
<篇名>72.论肝病治法
属性:(附∶新拟和肝丸) 
肝为厥阴,中见少阳,且有相火寄其中,故《内经》名为将军之官,其性至刚也。为其性刚,当有 
病时恒侮其所胜,以致脾胃受病,至有胀满、疼痛、泄泻种种诸证。因此方书有平肝之 
说,谓平肝即所以扶脾。若遇肝气横恣者,或可暂用而不可长用。因肝应春令,为气化发生之始,过平 
则人身之气化必有所伤损也。有谓肝于五行属木,木性原善条达,所以治肝之法当以散为补(方书谓肝以敛为 
泻以散为补)。散者即升发条达之也,然升散常用,实能伤气耗血,且又暗伤肾水以损肝木之根也。 
有谓∶肝恶燥喜润。燥则肝体板硬,而肝火肝气即妄动;润则肝体柔和,而肝火肝气长宁静。是以方书 
有以润药柔肝之法。然润药屡用,实与脾胃有碍,其法亦可暂用而不可长用。然则治肝之法将恶乎宜哉? 
《内经》谓∶“厥阴不治,求之阳明”。《金匮》谓∶“知肝之病,当先实脾。”先圣后圣,其揆 
如一,此诚为治肝者之不二法门也。惜自汉、唐以还,未有发明其理者。独至黄坤载,深明其理谓∶“肝气 
宜升,胆火宜降。然非脾气之上行,则肝气不升,非胃气之下行,则胆火不降。”旨哉此言,诚窥《内经》、 
《金匮》之精奥矣。由斯观之,欲治肝者,原当升脾降胃,培养中宫,俾中宫气化敦浓,以听肝木之自理。 
即有时少用理肝之药,亦不过为调理脾胃剂中辅佐之品。所以然者,五行之土原能包括金木水火四行,人之脾 
胃属土,其气化之敷布,亦能包括金木水火诸脏腑。所以脾气上行则肝气自随之上升,胃气下行则胆火自 
随之下降也。又《内经》论厥阴治法,有“调其中气,使之和平”之语。所谓调其中气者,即升脾降胃 
之谓也。所谓使之和平者,即升脾降胃而肝气自和平也。至仲景着《伤寒论》,深悟《内经》之旨,其厥阴 
治法有吴茱萸汤;厥阴与少阳脏腑相根据,乃由厥阴而推之少阳治法,有小柴胡汤。二方中之人参、半夏、大 
枣、生姜、甘草,皆调和脾胃之要药也。且小柴胡汤以柴胡为主药,而《神农本草经》谓其主肠胃中结气, 
饮食积聚,寒热邪气,推陈致新。三复《神农本草经》之文,则柴胡实亦为阳明胃府之药,而兼治少 
阳耳。欲治肝胆之病者,易弗祖《内经》而师仲景哉! 
独是,肝之为病不但不利于脾,举凡惊痫、癫狂、眩晕、脑充血诸证西人所谓脑气筋病者,皆与肝经有 
涉。盖人之脑气筋发源于肾,而分派于督脉,系淡灰色之细筋。肝原主筋,肝又为肾行气,故脑气筋之病实与 
肝脏有密切之关系也。治此等证者,当取五行金能制木之理,而多用五金之品以镇之,如铁锈、铅灰、 
金银箔、赭石(赭石铁养化合亦含有金属)之类,而佐以清肝润肝之品,若羚羊角、青黛、芍药、龙胆草、牛膝 
(牛膝味酸入肝,善引血火下行,为治脑充血之要药,然须重用方见奇效)诸药,俾肝经风定火熄,而 
脑气筋亦自循其常度,不至有种种诸病也。若目前不能速愈者,亦宜调补脾胃之药佐之,而后金属及寒 
凉之品可久服无弊。且诸证多系挟有痰涎,脾胃之升降自若而痰涎自消也。 
有至要之证,其病因不尽在肝,而急则治标,宜先注意于肝者,元气之虚而欲上脱者是也。其病状多大 
汗不止,或少止复汗,而有寒热往来之象。或危极至于戴眼,不露黑睛;或无汗而心中摇摇,需人按住;或 
兼喘促。此时宜重用敛肝之品,使肝不疏泄,即能杜塞元气将脱之路。至汗止、怔忡、喘促诸疾暂愈, 
而后徐图他治法。宜重用山茱萸净肉至二两(《神农本草经》山萸肉主治寒热即指此证), 
敛肝即以补肝,而以人参、赭石、龙骨、牡蛎诸药辅之。拙着来复汤后载有本此法挽回垂绝之证数则,可参阅也。 
究之肝胆之为用,实能与脾胃相助为理。因五行之理,木能侮土,木亦能疏土也。曾治有饮食不能 
消化,服健脾暖胃之药百剂不效。诊其左关太弱,知系肝阳不振,投以黄 (其性温升肝木之性亦温升有同气 
相求之义,故为补肝之主药)一两,桂枝尖三钱,数剂而愈。又治黄胆,诊其左关特弱,重用黄 煎汤,送 
服《金匮》黄胆门硝石矾石散而愈。若是皆其明征也。且胆汁入于小肠,能助小肠消化食物,此亦木 
能疏土之理。盖小肠虽属火,而实与胃腑一体相连,故亦可作土 
论。胆汁者,原由肝中回血管之血化出,而注之于胆,实得甲乙木气之全,是以在小肠中能化胃中不能化 
之食,其疏土之效愈捷也。又西人谓肝中为回血管会合之处,或肝体发大,或肝内有热,各管即多凝滞壅胀。 
由斯知疏达肝郁之药,若柴胡、川芎、香附、生麦芽、乳香、没药皆可选用;而又宜佐以活血之品,若 
桃仁、红花、樗鸡、 虫之类,且又宜佐以泻热之品,然不可骤用大凉之药,恐其所瘀之血得凉而凝,转 
不易消散,宜选用连翘、茵陈、川楝子、栀子(栀子为末,烧酒调敷,善治跌打处青红肿疼,能消瘀血可知) 
诸药,凉而能散,方为对证。 
肝体木硬,宜用柔肝之法。至柔肝之药,若当归、芍药、柏子仁、玄参、枸杞、阿胶、鳖甲皆可选用,而 
亦宜用活血之品佐之。而活血药中尤以三七之化瘀生新者为最紧要之品,宜煎服汤药之外,另服此药细末日 
三次,每次钱半或至二钱。则肝体之木硬者,指日可柔也。 
《内经》谓∶“肝苦急,急食甘以缓之。”所谓苦急者,乃气血忽然相并于肝中,致肝脏有急迫难缓 
之势,因之失其常司。当其急迫之时,肝体亦或木硬,而过其时又能复常。故其治法,宜重用甘缓之药以缓 
其急,其病自愈,与治肝体长此木硬者有异。曾阅《山西医志》二十四期,有人过服燥热峻烈之药,骤发痉 
厥,角弓反张,口吐血沫。乔××遵《内经》之旨,但重用甘草一味,连煎服,数日全愈,可谓善读《内经》 
者矣。然此证若如此治法仍不愈者,或加以凉润之品,若羚羊角、白芍,或再加镇重之品,若朱砂(研细送服) 
、铁锈,皆可也。 
【新拟和肝丸】治肝体木硬,肝气郁结,肝中血管闭塞,及肝木横恣侮克脾土。其现病或胁下胀疼,或 
肢体串疼,或饮食减少,呕哕,吞酸,或噫气不除,或呃逆连连,或头疼目胀、眩晕、痉痫,种种诸证。 
粉甘草(五两细末) 生杭芍(三两细末) 青连翘(三两细末) 广肉桂(两半去粗皮细末) 
冰片(三钱细末) 薄荷冰(四钱细末) 片朱砂(三两细末) 
上药七味,将前六味和匀,水泛为丸,梧桐子大,晾干(不宜晒),用朱砂为衣,勿余剩。务令坚实光 
滑始不走味。每于饭后一点钟服二十粒至三十粒,日再服。病急剧者,宜空心服;或于服两次 
之后,临睡时又服一次更佳。若无病者,但以为健胃消食药,则每饭后一点钟服十粒即可。 
数年来肝之为病颇多,而在女子为尤甚。医者习用香附、青皮、枳壳、延胡开气之品,及柴胡、川芎升 
气之品。连连服之,恒有肝病未除,元气已弱,不能支持,后遇良医,亦殊难为之挽救,若斯者良可慨也。 
此方用甘草之甘以缓肝;芍药之润以柔肝;连翘以散其气分之结(尝单用以治肝气郁结有殊效);冰片、 
薄荷冰以通其血管之闭;肉桂以抑肝木之横恣;朱砂以制肝中之相火妄行。且合之为丸,其味辛香甘美,能 
醒脾健胃,使饮食加增。又其药性平和,在上能清,在下能温(此药初服下觉凉及行至下焦则又变为温性)。 
故凡一切肝之为病,服他药不愈者,徐服此药,自能奏效。 

<目录>三、医论
<篇名>73.论黄胆有内伤外感及内伤外感之兼证并详治法
属性:黄胆之证,中说谓脾受湿热,西说谓胆汁滥行,究之二说原可沟通也。黄胆之载于方书者,原有内伤、 
外感两种,试先以内伤者言之。内伤黄胆,身无热而发黄,其来以渐,先小便黄,继则眼黄,继则周身皆黄, 
饮食减少,大便色白,恒多闭塞,乃脾土伤湿(不必有热)而累及胆与小肠也。盖人身之气化由中焦而升降, 
脾土受湿,升降不能自如以敷布其气化,而肝胆之气化遂因之湮瘀(黄坤载谓肝胆之升降由于脾胃确有至理), 
胆囊所藏之汁亦因之湮瘀而蓄极妄行,不注于小肠以化食,转溢于血中而周身发黄。是以仲景治内伤黄胆 
之方,均是胆脾兼顾。试观《金匮》黄胆门,其小柴胡汤显为治少阳胆经之方无论矣。他如治谷疸之茵陈 
蒿汤,治酒疸之栀子黄柏汤,一主以茵陈,一主以栀子,非注重清肝胆之热,俾肝胆消其炎肿而胆汁得由正 
路以入于小肠乎?至于硝石矾石方,为治女劳疸之的方,实可为治内伤黄胆之总方。其方硝石(俗名火硝 
亦名焰硝)矾石等分为散,大麦粥汁和服方寸匕(约重一钱),日三服,病随大小便去。小便正黄色,大便正 
黑色是也。特是方中矾石,释者皆以白矾当之,不无遗议。考《神农本草经》矾石一名羽涅,《尔雅》又名涅 
石。徐氏说文释涅字,谓黑土在水中,当系染黑之色。矾石既名为涅石,亦当为染黑色所需之物,岂非今之 
皂矾乎?是知皂矾、白矾,古人皆名为矾石。而愚临症体验以来,知以治黄胆,白矾之功效,诚不如皂矾。 
盖黄胆之证,中法谓由脾中蕴蓄湿热,西法谓由胆汁溢于血中。皂矾退热燥湿之力,不让白矾,故能去脾中 
湿热。而其色绿且青(亦名绿矾又名青矾),能兼入胆经,借其酸收之味,以敛胆汁之妄行。且此物化学家 
原可用硫酸水化铁而成。是知矿中所产之皂矾,亦必多含铁质。尤可借金铁之余气,以镇肝胆之本也。硝石性 
寒,能解脏腑之实热,味咸入血分,又善解血分之热。且其性善消,遇火即燃,又多含养气。人身之血,得 
养气则赤。又借硝石之消力,以消融血中之渣滓,则血之因胆汁而色变者,不难复于正矣。矧此证大便难者 
甚多,得硝石以软坚开结,湿热可从大便而解。而其咸寒之性,善清水腑之热,即兼能使湿热自小便解也。至 
用大麦粥送服者,取其补助脾胃之土以胜湿,而其甘平之性,兼能缓硝矾之猛峻,犹白虎汤中之用粳米也。 
按∶原方矾石下注有烧字,盖以矾石酸味太烈,制为枯矾则 
稍和缓。而愚实验以来,知径用生者,其效更速。临证者,相其身体强弱,斟酌适宜可也。 
或曰∶硝石、朴硝性原相近,仲景他方皆用朴硝,何此方独用硝石?答曰∶朴硝味咸,硝石则咸而兼辛, 
辛者金之味也。就此一方观之,矾石既含有铁质,硝石又具有金味,既善理脾中之湿热,又善制胆汁之妄行, 
中、西医学之理,皆包括于一方之中,所以为医中之圣也。且朴硝降下之力多,硝石消融之力多(理详砂淋丸 
下)。胆汁之溢于血中者,布满周身难尽降下,实深赖硝石之善消融也。又朴硝为水之精华结聚,其咸寒之 
性,似与脾湿者不宜。硝石遇火则燃,兼得水中真阳之气。其味之咸不若朴硝,且兼有辛味。 
似能散湿气之郁结,而不致助脾湿也。 
特是《金匮》治内伤黄胆,虽各有主方,而愚临证经验以来,知治女劳疸之硝石矾石散不但治女劳疸 
甚效,即用以治各种内伤黄胆,亦皆可随手奏效。惟用其方时,宜随证制宜而善为变通耳。 
按∶硝石矾石散原方,用硝石、矾石等分为散,每服方寸匕(约重一钱),大麦粥送下。其用大 
麦粥者,所以调和二石之性,使之与胃相宜也。至愚用此方时,为散药难服,恒用炒熟大麦面,或 
小麦面亦可,与二石之末等分,和水为丸,如五味子大,每服二钱,随证择药之相宜者,数味煎汤送下( 
因药中已有麦面为丸,不必再送以大麦粥)。其有实热者,可用茵陈、栀子煎汤送服。有食积者,可用生鸡 
内金、山楂煎汤送服。大便结者,可用大黄、麻仁煎汤送服。小便闭者,可用滑石、生杭芍煎汤送服。恶 
心呕吐者,可用赭石、青黛煎汤送服。左脉沉而无力者,可用生黄 、生姜煎汤送服。右脉沉而无力者,可用 
白术、陈皮煎汤送服。其左右之脉沉迟而弦,且心中觉凉,色黄黯者,附子、干姜皆可加入汤药之中。脉浮有 
外感者,可先用甘草煎汤送服西药阿斯匹林一瓦,出汗后再用甘草汤送服丸药。又凡服此丸药而 
嫌其味劣者,皆可于所服汤药中加甘草数钱以调之。 
又黄胆之证,西人谓恒有胆石阻塞胆囊之口,若尿道之有淋石也。硝石、矾石并用,则胆石可消。又西 
人谓小肠中有钩虫亦可令人成黄胆。硝石、矾石并用,则钩虫可除。此所以用此统治 
内伤黄胆,但变通其送服之汤药,皆可随手奏效也。 
至外感黄胆,约皆身有大热。乃寒温之热,传入阳明之府,其热旁铄,累及胆脾,或脾中素有积湿,热 
入于脾与湿合,其湿热蕴而生黄,外透肌肤而成疸;或胆中所寄之相火素炽,热入于胆与火并,其胆管因热 
肿闭,胆汁旁溢混于血中,亦外现成疸。是以仲景治外感黄胆有三方,皆载于《伤寒论》阳明篇,一为茵陈 
蒿汤,二为栀子柏皮汤,三为麻黄连翘赤小豆汤,皆胆脾并治也。且统观仲景治内伤、外感黄胆之方,皆 
以茵陈蒿为首方。诚以茵陈蒿性凉色青,能入肝胆,既善泻肝胆之热,又善达肝胆之 
郁,为理肝胆最要之品,即为治黄胆最要之品。 
至愚生平治外感黄胆,亦即遵用《伤寒论》三方。而于其热甚者,恒于方中加龙胆草数钱。又用麻黄连 
翘赤小豆汤时,恒加滑石数钱。恐连翘利水之力不足,故加滑石以助之。若其证为白虎汤或白虎加人参汤 
证及三承气汤证,而身黄者,又恒于白虎承气中,加茵陈蒿数钱。其间有但用外感诸方不效者,亦可用外感 
诸方煎汤,送服硝石矾石散。 
黄胆之证又有先受外感未即病,迨酿成内伤而后发现者。岁在乙丑,客居沧州,自仲秋至孟冬一方多有 
黄胆证。其人身无大热,心中满闷,时或觉热,见饮食则恶心,强食之恒作呕吐,或食后不能下行,剧者 
至成结证,又间有腹中觉凉,食后饮食不能消化者。愚共治六十余人,皆随手奏效。其脉左似有热,右多郁 
象,盖其肝胆热而脾胃凉也。原因为本年季夏阴雨连旬,空气之中所含水分过度,人处其中,脏腑为 
湿所伤。肝胆属木,禀少阳之性,湿郁久则生热,脾胃属土,禀太阴之性,湿郁久则生寒,此 
自然之理也。为木因湿郁而生热,则胆囊之口肿胀,不能输其汁于小肠以化食,转溢于血分,色透肌表而 
发黄。为土因湿郁而生寒,故脾胃火衰,不能熟腐水谷,运转下行,是以恒作胀满,或成结证。为疏方用 
茵陈、栀子、连翘各三钱,泻肝胆之热,即以消胆囊之肿胀;浓朴、陈皮、生麦芽各二钱,生姜五钱开脾胃之 
郁,即以祛脾胃之寒;茯苓片、生薏米、赤小豆、甘草各三钱,泻脏腑之湿,更能培土以胜湿,且重用甘草 
即以矫茵陈蒿之劣味也(此证闻茵陈之味多恶心呕吐故用甘草调之)。服一剂后,心中不觉热者,去栀子, 
加生杭芍三钱,再服一剂。若仍不能食者,用干姜二钱以代生姜。若心中不觉热转觉凉者,初服即不用栀子, 
以干姜代生姜。凉甚者,干姜可用至五六钱。呕吐者,加赭石六钱或至一两。服后吐仍不止者,可先用开水 
送服赭石细末四五钱,再服汤药。胃脘肠中结而不通者,用汤药送服牵牛(炒熟)为末三钱,通利后即减 
去。如此服至能进饮食,即可停药。黄色未退,自能徐消。此等 
黄胆,乃先有外感内伏,酿成内伤,当于《伤寒论》、《金匮》所载之黄胆以外另为一种矣。 
或问∶医学具有科学性质,原贵证实,即议论之间,亦贵确有实据。仲景治黄胆虽云胆脾并治,不过 
即其所用之药揣摩而得。然尝考之《伤寒论》,谓“伤寒脉浮而缓,手足自温,是为系在太阴,太阴者,身 
当发黄。”是但言发黄证由于脾也。又尝考之《金匮》,谓“寸口脉浮而缓,浮则为风,缓则为痹,痹非 
中风,四肢苦烦,脾色必黄,瘀热以行”,是《金匮》论黄胆亦责重脾也。夫古人立言原多浑括,后世注 
疏宜为详解。当西医未来之先,吾中华方书之祖述仲景者,亦有显然谓黄胆病由于胆汁溢于血中者乎?答曰∶有 
之。明季喻嘉言着《寓意草》,其论钱小鲁嗜酒成病,谓胆之热汁满而溢于外,以渐渗于经络,则身目俱 
黄,为酒疸之病云云。岂非显然与西说相同乎?夫西人对于此证 
必剖验而后知,喻氏则未经剖验而已知。非喻氏之智远出西人之上,诚以喻氏最深于《金匮》、《伤寒论》, 
因熟读仲景之书,观其方中所用之药而有所会心也。由斯观之,愚谓仲景治黄胆原胆脾并治者,固非无稽之谈也。 
附录∶ 
徐××云∶《金匮》硝石矾石散方,原治内伤黄胆,张锡纯氏之发明功效卓然大着。至矾石即皂矾, 
张石顽亦曾于《本经达源》论及,而先生则引《神农本草经》兼名涅石、《尔雅》又名羽涅,即一涅字,知 
其当为皂矾,又即其服药后大便正黑色,愈知其当为皂矾,可谓具有特识。又于临证之时,见其左脉细弱者, 
知系肝阳不能条畅,则用黄 、当归、桂枝尖诸药煎汤送服;若见其右脉濡弱者,知系脾胃不能健运,则用白 
术、陈皮、薏米诸药煎汤送服,不拘拘送以大麦粥,此诚善用古方,更能通变化裁者也。友人史××,治一 
妇人病黄病五六年,肌肤面目俱黄,癸亥秋感受客邪,寒热往来,周身浮肿。史××与柴胡桂枝汤和 
解之,二剂肿消,寒热不作。遂配硝石矾石散一剂,俾用大麦粥和服。数日后复来云∶此药入腹似难容受, 
得无有他虑否?史××令放胆服之∶“倘有差错,吾愿领咎。”又服两剂其黄尽失。史××欣然述之于予。 
予曰∶“仲圣之方固属神矣,苟非张先生之审定而阐发之,则亦沉潜汨没,黯淡无光耳。”噫!古人创方固 
难,而今人用方亦岂易易哉! 

<目录>三、医论
<篇名>74.论水臌气臌治法
属性:(附∶表里分消汤) 
水臌、气臌形原相近。《内经》谓“按之 而不起者,风水也。”愚临证品验以来,知凡水证,以 
手按其肿处成凹,皆不能随手而起。至气臌,以手重按成凹,则必随手而起。惟单腹胀 
病,其中水臌、气臌皆有,因其所郁气与水皆积腹中,不能外透肌肉,按之亦不成凹,似难辨其为水、为气。 
然水臌必然小便短少,气臌必觉肝胃气滞,是明征也。今试进论其治法。 
《金匮》论水病,分风水、皮水、正水、石水。谓风水、皮水脉浮,正水、石水脉沉。然水病之剧者, 
脉之部位皆肿,必重按之成凹其脉方见,原难辨其浮沉。及观其治法,脉浮者宜发汗,恒佐以凉润之药;脉 
沉者宜利小便,恒佐以温通之药。是知水肿原分凉热,其凉热之脉,可于有力、无力辨之。愚治此证, 
对于脉之有力者,亦恒先发其汗,曾拟有表里分消汤,爰录其方于下。 
【表里分消汤】麻黄三钱,生石膏、滑石各六钱,西药阿斯匹林一瓦。将前三味煎汤,送服阿 
斯匹林。若服药一点钟后不出汗者,再服阿斯匹林一瓦。若服后仍不出汗,还可再服,当以汗出为目标。 
麻黄之性,不但善于发汗,徐灵胎谓能深入积痰凝血之中,凡药力所不到之处,此能无微不至,是以 
服之外透肌表,内利小便,水病可由汗、便而解矣。惟其性偏于热,似与水病之有热者不宜,故用生石膏以 
解其热。又其力虽云无微不至,究偏于上升,故又用滑石引之以下达膀胱,即其利水之效愈捷也。至用西药阿 
斯匹林者,因患此证者,其肌肤为水锢闭,汗原不易发透,多用麻黄又恐其性热耗阴,阿斯匹林善发汗,又 
善清热,故可用为麻黄之佐使,且其原质存于杨柳皮液中,原与中药并用无碍也。 
若汗已透,肿虽见消,未能全愈者,宜专利其小便。而利小便之药,以鲜白茅根汤为最效,或与车前并 
用,则尤效。忆辛酉腊底,有邻村学生毛××,年二十,得水肿证,医治月余,病益剧,头面周身皆肿,腹如 
抱瓮,夜不能卧,根据壁喘息,盖其腹之肿胀异常,无容息之地,其气几不能吸入故作喘也。其脉六部细 
数,心中发热,小便不利,知其病久阴虚,不能化阳,致有此证。俾命人力剖冻地,取鲜茅根,每日用鲜茅 
根六两,锉碎,和水三大碗,以小锅煎一沸,即移置炉旁,仍近炉眼徐徐温之,待半点钟,再煎一沸,犹如 
前置炉旁,须臾茅根皆沉水底,可得清汤两大碗,为一日之量,徐徐当茶温饮之。再用生车前子数两, 
自炒至微熟,三指取一撮,细细嚼咽之,夜间睡醒时亦如此,嚼服一昼夜,约尽七八钱。如此二日,小便 
已利,其腹仍膨胀板硬。俾用大葱白三斤,切作丝,和醋炒至将熟,乘热裹以布,置脐上熨之。若凉,则仍 
置锅中,加醋少许炒热再熨。自晚间熨至临睡时止,一夜小便十余次,翌晨按其腹如常人矣。盖茅根如此 
煎法,取其新鲜凉润之性大能滋阴清热(久煎则无此效)。阴滋热清,小便自利。若遇证之轻者,但用 
徐服车前子法亦可消肿,曾用之屡次奏功矣。 
此证虽因病久阴虚,究非原来阴虚。若其人平素阴虚,以致小便不利,积成水肿者,宜每用熟地黄两半, 
与茅根同煎服。若恐两沸不能将地黄煎透,可先将地黄煮十余沸,再加茅根同煮。 
至车前子,仍宜少少嚼服,一日可服四五钱。 
至于因凉成水臌者,其脉必细微迟弱,或心中觉凉,或大便泄泻。宜用花椒目六钱,炒熟捣烂,煎汤 
送服生硫黄细末五分。若服后不觉温暖,可品验加多,以服后移时微觉温暖为度。盖利小便之药多凉,二 
药乃性温能利小便者也。若脾胃虚损,不能运化水饮者,宜治以健脾降胃之品,而以利小便之药佐之。 
总之,水臌之证,未有小便通利而成者。是以治此证者, 
当以利小便为要务。治小便不利,可参阅拙拟治癃闭诸方各案。 
有因胞系了戾,致小便不通者。其证偶因呕吐咳逆,或侧卧 
欠伸,仍可通少许,俗名为转胞病。孕妇与产后及自高坠下者,间 
有此病。拙拟有升麻黄 汤,曾用之治愈数人,此升提胞系而使之转正也。 
华元化有通小便秘方,愚知之而未尝试用。后阅杭报,见萧××言用其方加升麻一钱,曾治愈其令妹二 
日一夜小便不通及陶姓男子一日夜小便不通,皆投之即效,方系人参、莲子心、车前子、王不留行各三钱,甘 
草一钱,肉桂三分,白果十二枚。按∶方中白果,若以治咳嗽,可连皮捣烂用之,取其皮能敛肺也;若 
以利小便,宜去皮捣烂用之,取其滑而能降也。 
至于气臌,多系脾有瘀滞所致。盖脾为后天之主,居中央以运四旁,其中原多回血管,以流通气化。若有 
瘀滞以阻其气化,腹中即生胀满,久则积为气臌,《内经》所谓诸湿肿满皆属脾也。拙拟有鸡 汤,曾用 
之屡次奏效。方中之意∶用鸡内金以开脾之瘀;白术以助脾之运;柴胡、陈皮以升降脾气;白芍以利小便, 
防有蓄水,生姜以通窍络兼和营卫也。统论药性,原在不凉不热之间。然此证有偏凉者,则桂、附、干姜可 
以酌加;有偏热者,则芩、连、栀子可以酌加。若其脉证皆实,服药数剂不见愈者,可用所煎药汤送服黑丑头 
次所轧细末钱半,服后大便通行,病即稍愈。然须服原方数日,日用一次,连用恐伤气分。此水臌气臌治 
法之大略也(医方篇中载有治水臌气臌诸方案宜参观)。 
<目录>三、医论
<篇名>75.论血臌治法
属性:水臌、气臌之外,又有所谓血臌者,其证较水臌、气臌尤为难治。然其证甚稀少,医者或临证数十年不 
一遇,即或遇之,亦止认为水臌、气臌,而不知为血臌。是以方书鲜有论此证者,诚以此证之肿胀形状,与 
水臌、气臌几无以辨,所可辨者,其周身之回血管紫纹外现耳。 
血臌之由,多因努力过甚,激动气血,或因暴怒动气,血随 
气升,以致血不归经,而又未即吐出泻出,遂留于脏腑,阻塞经络,周身之气化因之不通,三焦之水饮因 
之不行,所以血臌之证初起,多兼水与气也。迨至瘀血渐积渐满,周身之血管皆为瘀血充塞,其回血管 
肤浅易见,遂呈紫色,且由呈紫色之处,而细纹旁达,初则两三处,浸至遍身皆是紫纹。若于回血 
管紫色初见时,其身体犹可支持者,宜先用《金匮》下瘀血汤加野台参数钱下之。其腹中之瘀血下后,可再 
用药消其血管中之瘀血,而辅以利水理气之品。程功一月,庶可奏效。若至遍身回血管多现紫色,病候至此, 
其身体必羸弱已甚,即投以下瘀血汤,恐瘀血下后转不能支持,可用拙拟化瘀通经散,再酌加三七末服之, 
或用利水理气之药煎汤送服,久之亦可奏效。若腹中瘀血已下,而周身之紫纹未消者,可用丹参、三七末各一 
钱,再用山楂四钱煎汤,冲红糖水送服,日两次,久自能消。 
《金匮》下瘀血汤∶大黄三两(当为今之九钱),桃仁三十个, 虫二十枚去足熬(炒也)。上三味末 
之,炼蜜和为四丸,以酒一升(约四两强)煮一丸,取八合顿服之,瘀血下如豚肝。按∶此方必先为丸而后 
作汤服者,是不但服药汁,实兼服药渣也。盖如此服法,能使药之力缓而且大,其腹中瘀久之血,可一服尽 
下。有用此方者,必按此服法方效。又杏仁之皮有毒,桃仁之皮无毒,其皮色红,活血之力尤大,此方桃仁, 
似宜带皮生用。然果用带皮生桃仁时,须审辨其确为桃仁,勿令其以带皮之杏仁误充。 
究之,病血臌者,其身体犹稍壮实,如法服药,原可治愈。若至身体羸弱者,即能将其瘀治净,而 
转有危险,此又不可不知。临证时务将此事言明,若病家恳求,再为治之未晚也。 

<目录>三、医论
<篇名>76.论肾弱不能作强治法
属性:《内经》谓∶“肾者作强之官,伎巧出焉。”盖肾之为用,在 
男子为作强,在女子为伎巧。然必男子有作强之能,而后女子有伎巧之用也。是以欲求嗣续者,固当调养女 
子之经血,尤宜补益男子之精髓,以为作强之根基。彼方书所载助肾之药,若海马、獭肾、蛤蚧之类,虽能 
助男子一时之作强,实皆为伤肾之品,原不可以轻试也。惟鹿茸方书皆以为补肾之要品,然止能补肾中之 
阳,久服之亦能生弊。惟用鹿角所熬之胶,《神农本草经》谓之白胶,其性阴阳俱补,大有益于肾脏。是以 
白胶在《神农本草经》列为上品,而鹿茸止列于中品也。曾治一人,年近五旬,左腿因受寒作疼,教以日用 
鹿角胶三钱含化服之。阅两月复觌面,其人言服鹿角胶半月,腿已不疼。然自服此药后,添有兴阳之病,因 
此辍服。愚曰∶“此非病也,乃肾脏因服此而壮实也。”观此,则鹿角胶之为用可知矣。若其人相火衰甚, 
下焦常觉凉者,可与生硫黄并服。鹿角胶仍含化服之。又每将饭之先,服生硫黄末三 
分,品验渐渐加多,以服后移时微觉温暖为度。 
《难经》谓命门之处,男以藏精,女以系胞。胞即胞室,与肾系同连于命门。西人之生理新发明家谓其 
处为副肾髓质,又谓其处为射精之机关,是中、西之说同也。又谓副肾髓质之分泌素名副肾碱,而鸡子黄中实 
含有此物,可用以补副肾碱之缺乏。此说愚曾实验之,确乎可信。方用生鸡子黄两、三枚,调开水服 
之,勿令熟,熟则无效。 
愚曾拟一强肾之方,用建莲子去心为末,焙熟。再用猪、羊脊髓和为丸桐子大。每服二钱,日两次。 
常服大有强肾之效,因名其方为强肾瑞莲丸。盖凡物之有脊者,其脊中必有一袋,即督脉也。其中所藏之液, 
即脊髓,亦即西人所谓副肾碱,所以能助肾脏作强。且督脉之袋上通于脑,凡物之角与脑相连,鹿角最大, 
其督脉之强可知。是用鹿角胶以补肾,与用猪、羊脊髓以补肾其理同也。 
肾主骨。胡桃仁最能补肾。人之食酸 齿者,食胡桃仁即愈,因齿牙为骨之余,原肾主之,故有斯效, 
此其能补肾之明征也。古方以治肾经虚寒,与补骨脂并用,谓有木火相生之妙(胡桃属木补骨脂属火), 
若肾经虚寒,泄泻、骨痿、腿疼用之皆效,真佳方也。 
枸杞亦为强肾之要药,故俗谚有“隔家千里,勿食枸杞”之 
语。然素有梦遗之病者不宜单服、久服,以其善兴阳也,惟与山萸肉同服,则无斯弊。 
紫稍花之性,人皆以为房术之药,而不知其大有温补下焦之功。凡下焦虚寒泄泻,服他药不愈者,恒服 
紫稍花即能愈,其能大补肾中元气可知。久久服之,可使全体强壮。至服之上焦觉热者,宜少佐以生地黄。 
然宜作丸散,不宜入汤剂煎服。曾治一人,年过四旬,身形羸弱,脉象细微,时患泄泻,房事不能作 
强。俾用紫稍花为末,每服二钱半,日两次;再随便嚼服枸杞子五六钱。两月之后,其身形遽然强壮,泄 
泻痿废皆愈。再诊其脉,亦大有起色。且从前觉精神脑力日浸衰减,自服此药后则又觉日浸增加矣。 

<目录>三、医论
<篇名>77.论冲气上冲之病因病状病脉及治法
属性:冲气上冲之病甚多,而医者识其病者甚少。即或能识此病,亦多不能洞悉其病因,而施以相当之治法。 
冲者,奇经八脉之一,其脉在胞室之两旁,与任脉相连,为肾脏之辅弼,气化相通。是以肾虚之人,冲气多 
不能收敛,而有上冲之弊。况冲脉之上系原隶阳明胃府,因冲气上冲,胃府之气亦失其息息下行之常 
(胃气以息息下行为常),或亦转而上逆,阻塞饮食,不能下行,多化痰涎,因腹中膨闷、哕气、呃逆连连 
不止,甚则两肋疼胀、头目眩晕。其脉则弦硬而长,乃肝脉之现象也。盖冲气上冲之证,固由于肾 
脏之虚,亦多由肝气恣横,素性多怒之人,其肝气之暴发,更助 
冲胃之气上逆,故脉之现象如此。治此证者,宜以敛冲、镇冲为主,而以降胃、平肝之药佐之。其脉象数 
而觉热者,宜再辅以滋阴退热之品。愚生平治愈此证已不胜纪,近在沧州连治愈数人,爰将治愈之案详列 
于下,以备参观。 
沧州安××,年十八九,胸胁满闷,饮食减少,时作哕逆,腹中漉漉有声,盖气冲痰涎作响也,大 
便干燥,脉象弦长有力。为疏方,用生龙骨、牡蛎、代赭石各八钱,生山药、生芡实各六钱,半夏、生杭芍 
各四钱,芒硝、苏子各二钱,浓朴、甘草各钱半。一剂后,脉即柔和。按方略有加减,数剂全愈。 
陈修园谓龙骨、牡蛎为治痰之神品,然泛用之多不见效,惟以治此证之痰,则效验非常。因此等痰涎, 
原因冲气上冲而生,龙骨、牡蛎能镇敛冲气,自能引导痰涎下行也。盖修园原谓其能导引逆上之火、泛滥之 
水,下归其宅,故能治痰。夫火逆上,水泛滥,其中原有冲气上冲也。 
赵××,因有冲气上冲病,自言患此病已三年,百方调治,毫无效验。其病,脉情状大略与前案同, 
惟无痰声漉漉,而尺脉稍弱。遂于前方去芒硝,加柏子仁、枸杞子各五钱。连服数剂全愈。 
沧州一叟,年七十四岁,性浮躁,因常常忿怒,致冲气上冲,剧时觉有气自下上冲,杜塞咽喉,有危在 
顷刻之势,其脉左右皆弦硬异常。为其年高,遂于前第二方中加野台参三钱,一剂见轻。又服一剂,冲 
气遂不上冲。又服数剂以善其后。为治此证多用第二方加减,因名为降胃镇冲汤。 

<目录>三、医论
<篇名>78.论火不归原治法
属性:方书谓下焦之火生于命门,名为阴分之火,又谓之龙雷之火,实肤浅之论也。下焦之火为先天之元阳, 
生于气海之元气。盖就其能撑持全身论,则为元气;就其能温暖全身论,则为元 
阳。此气海之元阳,为人生命之本源,无论阴分、阳分之火,皆于此肇基。气海,纯系脂膜护绕抟结而成。 
其脂膜旁出一条,与脊骨自下数第七节相连。夹其七节两旁,各有一穴,《内经》谓七节之旁中有小心也。而 
气海之元阳由此透入脊中,因元阳为生命之本,故于元阳透脊之处谓之命门。由斯观之,命门之实用, 
不过为气海司管钥之职,下焦之火,仍当属于气海之元阳。论下焦之火上窜不归原,亦气海元阳之浮越也。 
然其病浑名火不归原,其病因原有数端,治法各有所宜,爰详细胪列于下。 
有气海元气虚损,不能固摄下焦气化,致元阳因之浮越者。其脉尺弱寸强,浮大无根。其为病,或头目 
眩晕,或面红耳热,或心热怔忡,或气粗息贲。宜治以净萸肉、生山药各一两,人参、玄参、代赭石、生 
龙骨、生牡蛎各五钱。心中发热者,酌加生地黄、天冬各数钱,补而敛之,镇而安之,元阳自归其宅也。 
方中用赭石者,因人参虽饶有温补之性,而力多上行,与赭石并用,则力专下注,且赭石重坠之性,又善佐 
龙骨、牡蛎以潜阳也。有下焦真阴虚损,元阳无所系恋而浮越者。其脉象多弦数,或重按无力。其证时作 
灼热,或口苦舌干,或喘嗽连连。宜用生山药、熟地黄各一两,玄参、生龙骨、生牡蛎、生龟板、甘枸杞各 
五钱,生杭芍三钱,生鸡内金、甘草各钱半。此所谓壮水之主,以制阳光也。 
若其下焦阴分既虚,而阳分亦微有不足者,其人上焦常热,下焦间有觉凉之时,宜治以《金匮》崔氏八味 
丸,以生地易熟地(原方干地黄即是药局中生地),更宜将茯苓、泽泻分量减三分之二,将丸剂一料, 
分作汤药八剂服之。 
有气海元阳大虚,其下焦又积有沉寒锢冷,逼迫元阳如火之将灭,而其焰转上窜者。其脉弦迟细弱,或 
两寸浮分似有力。其为证∶心中烦躁不安,上焦时作灼热,而其下焦转觉凉甚,或常 
作泄泻。宜用乌附子、人参、生山药各五钱,净萸肉、胡桃肉各四钱,赭石、生杭芍、怀牛膝各三钱,云 
苓片、甘草各钱半。泄泻者宜去赭石。此方书所谓引火归原之法也。方中用芍药者,非以解上焦之热,以其与 
参、附并用,大能收敛元阳,下归其宅。然引火归原之法,非可概用于火不归原之证,必遇此等证与脉, 
然后可用引火归原之法,又必须将药晾至微温,然后服之,方与上焦之燥热无碍。 
有因冲气上冲兼胃气上逆,致气海元阳随之浮越者。其脉多弦长有力,右部尤甚,李士材脉诀歌括所谓 
直上直下也。其证觉胸中满闷烦热,时作呃逆,多吐痰涎,剧者觉痰火与上冲之气杜塞咽喉,几不能息。 
宜治以拙拟降胃镇冲汤(方见“论冲气上冲之病因病状病脉及治法”一节), 
俾冲、胃之气下降,而诸病自愈矣。 
有因用心过度,心中生热,牵动少阳相火(即胆肝中所寄之相火),上越且外越者。其脉寸关皆有力,多 
兼滑象,或脉搏略数。其为病∶心中烦躁不安,多生疑惑,或多忿怒,或觉热起胁下,散于周身。治用生怀 
山药细末六七钱,煮作粥,晨间送服芒硝三钱,晚送服西药臭剥两瓦。盖芒硝咸寒,善解心经之热,以开心 
下热痰(此证心下多有热痰);臭剥性亦咸寒,能解心经之热,又善制相火妄动。至送以山药粥者,因咸寒 
之药与脾胃不宜,且能耗人津液,而山药则善于养脾胃、滋津液,用之送服硝、剥,取其相济以成功, 
犹《金匮》之硝石矾石散送以大麦粥也。 
有因心肺脾胃之阳甚虚,致寒饮停于中焦,且溢于膈上,逼迫心肺脾胃之阳上越兼外越者。其脉多弦迟 
细弱,六部皆然,又间有浮大而软,按之豁然者。其现证∶或目眩耳聋,或周身发热,或觉短气,或咳喘, 
或心中发热,思食鲜果,而食后转觉心中胀满病加剧者。宜用拙拟理饮汤。服数剂后,心中不觉热、转觉凉 
者,去芍药。或觉气不足者,加生箭 三钱。此证如此治法,即 
方书所谓用温燥健补脾胃之药可以制伏相火。不知其所伏者非相火,实系温燥之药能扫除寒饮,而心肺脾胃 
之阳自安其宅也。 
上所列火不归原之证,其病因虽不同,而皆系内伤。至外感之证,亦有火不归原者,伤寒、温病中之戴 
阳证是也。其证之现状∶面赤、气粗、烦躁不安,脉象虽大,按之无力,又多寸盛尺虚。此乃下焦虚寒 
孤阳上越之危候,颇类寒温中阴极似阳证。然阴极似阳,乃内外异致,戴阳证乃上下异致也。宜用《伤寒论》 
通脉四逆汤,加葱、加人参治之(原方原谓面赤者加葱,面赤即戴阳证)。 
特是戴阳之证不一。使果若少阴脉之沉细,或其脉非沉细而按之指下豁然毫无根柢,且至数不数者, 
方可用通脉四逆汤方。若脉沉细而数或浮大而数者,其方即断不可用。曾治王××,年四十余,身形素虚, 
伤寒四五日间,延为诊视。其脉关前洪滑,两尺无力。为开拙拟仙露汤,因其尺弱嘱其将药徐徐饮下,一次 
只温饮一大口,防其寒凉侵下焦也。病家忽愚所嘱,竟顿饮之,遂致滑泻数次,多带冷沫,上焦益觉烦躁, 
鼻如烟熏,面如火炙,其关前脉大于从前一倍,数至七至。知其已成戴阳之证,急用野台参一两,煎汤八分 
茶盅,兑童便半盅(须用五岁以下童子便),将药碗置凉水盆中,候冷顿饮之。又急用知母、玄参、生地各 
一两,煎汤一大碗候用。自服参后,屡诊其脉。过半点钟,脉象渐渐收敛,脉搏似又加数,遂急用候服之药 
炖极热,徐徐饮下,一次只饮药一口,阅两点钟尽剂,周身微汗而愈。 
按∶此证上焦原有燥热,因初次凉药顿服,通过病所,直达下焦,上焦燥热仍留。迨下焦滑泻,元阳 
上浮,益助上焦之热,现种种热象,脉数七至。此时不但姜、附分毫不敢用,即单用人参,上焦之燥热亦必 
格拒不受。故以童便之性下趋者佐之,又复将药候至极凉顿服下。迨迟之有倾,脉象收敛,至数加数,是下 
焦得参温补之力而元阳收回,其上焦因参反激之力而燥热益增 
也。故又急用大凉大润之药,乘热徐徐饮之,以清上焦之燥热, 
而不使其寒凉之性复侵下焦。此于万难用药之际,仍欲用药息息吻合,实亦费尽踌躇矣。 

<目录>三、医论
<篇名>79.论腰疼治法
属性:方书谓∶“腰者肾之府,腰疼则肾将惫矣。”夫谓腰疼则肾将惫,诚为确论。至谓腰为肾之府,则尚 
欠研究。何者?凡人之腰疼,皆脊梁处作疼,此实督脉主之。督脉者,即脊梁中之脊髓袋,下连命门穴处, 
为人之副肾脏(是以不可名为肾之府)。肾虚者,其督脉必虚,是以腰疼。治斯证者,当用补肾之剂,而引 
以入督之品。曾拟益督丸一方,徐徐服之,果系肾虚腰疼,服至月余自愈。 
【益督丸】杜仲四两酒浸炮黄,菟丝子三两酒浸蒸熟,续断二两酒浸蒸熟,鹿角胶二两,将前三味为 
细末,水化鹿角胶为丸,黄豆粒大。每服三钱,日两次。服药后,嚼服熟胡桃肉一枚。 
诸家本草皆谓,杜仲宜炒断丝用,究之将杜仲炒成炭而丝仍不断,如此制法殊非所宜。是以此方中惟用 
生杜仲炮黄为度。胡桃仁原补肾良药,因其含油质过多,不宜为丸,故于服药之后单服之。 
若证兼气虚者,可用黄 、人参煎汤送服此丸。若证兼血虚者,可用熟地、当归煎汤送服此丸。 
有因瘀血腰疼者,其人或过于任重,或自高坠下,或失足闪跌,其脊梁之中存有瘀血作疼。宜治以 
活络效灵丹,加 虫三钱,煎汤服,或用葱白作引更佳。 
李××,腰疼数年不愈,为之延医。其疼剧时心中恒觉满闷,轻时则似疼非疼,绵绵不已;亦恒数 
日不疼。其脉左部沉弦,右部沉牢。自言得此病已三年,服药数百剂,其疼卒未轻 
减。观从前所服诸方,虽不一致,大抵不外补肝肾强筋骨诸药, 
间有杂以祛风药者。因思《内经》谓通则不痛,而此则痛则不通也。且即其脉象之沉弦、沉牢,心中恒 
觉满闷,其关节经络必有瘀而不通之处可知也。爰为拟利关节通络之剂,而兼用补正之品以辅助之。 
生怀山药(一两) 大甘枸杞(八钱) 当归(四钱) 丹参(四钱) 生明没药(四钱) 生 
五灵脂(四钱) 穿山甲(二钱炒捣) 桃仁(二钱) 红花(钱半) 虫(五枚) 广三七(二钱捣细) 
药共十一味。先将前十味煎汤一大盅,送服三七细末一半。至煎渣再服时,仍送服其余一半。 
此药服至三剂,腰已不疼,心中亦不发闷,脉较前缓和,不专在沉分。遂即原方去山甲,加胡桃肉四钱。 
连服十剂,自觉身体轻爽。再诊其脉,六部调匀,腰疼遂从此除根矣。就此证观之,凡其人身形不羸弱而 
腰疼者,大抵系关节经络不通;其人显然羸弱而腰疼者,或肝肾有所亏损而然也。 
在妇女又恒有行经时腰疼者,曾治一人,年过三旬,居恒呼吸觉短气,饮食似畏寒凉。当行经时觉腰 
际下坠作疼。其脉象无力,至数稍迟。知其胸中大气虚而欲陷,是以呼吸气短,至行经时因气血下注大气亦随 
之下陷,是以腰际觉下坠作疼也。为疏方用生箭 一两,桂枝尖、当归、生明没药各三钱。连服七八剂, 
其病遂愈。 
又治一妇人行经腰疼且兼腹疼,其脉有涩象,知其血分瘀也。治以当归、生鸡内金各三钱, 
生明没药、生五灵脂、生箭 、天花粉各四钱,连服数剂全愈。 

<目录>三、医论
<篇名>80.论肢体痿废之原因及治法
属性:(附∶起痿汤、养脑利肢汤) 
《内经》谓∶“五脏有病,皆能使人痿。”至后世方书,有谓 
系中风者,言风中于左,则左偏枯而痿废,风中于右则右偏枯而痿废。有谓系气虚者,左手足偏枯痿废,其 
左边之气必虚,右手足偏枯痿废,其右边之气必虚。有谓系痰瘀者。有谓系血瘀者。有谓系风寒湿相并而 
为痹,痹之甚者即令人全体痿废。因痰瘀血瘀及风寒湿痹皆能阻塞经络也。乃自脑髓神经司知觉运动之说倡自 
西人,遂谓人之肢体痿废皆系脑髓神经有所伤损。而以愚生平所经验者言之,则中西之说皆不可废。今试 
列举素所经验者于下,以征明之。 
忆在籍时,曾见一猪,其两前腿忽不能动,须就其卧处饲之,半月后始渐愈。又旬余解此猪,见其 
肺上新愈之疮痕宛然可辨,且有将愈未尽愈者。即物测人,原可比例,此即《内经》所谓因肺热叶焦发为痿 
者也。由斯知五脏有病皆使人痿者,诚不误也。 
在奉天曾治一妇人,年近三旬,因夏令夜寝当窗,为风所袭,遂觉半身麻木,其麻木之边,肌肤消瘦, 
浸至其一边手足不遂,将成偏枯。其脉左部如常,右部则微弱无力,而麻木之边适在右。此因风袭经络, 
致其经络闭塞不相贯通也。不早祛其风,久将至于痿废。为疏方用生箭 二两,当归八钱(用当归者取其血活 
风自去也),羌活、知母、乳香、没药各四钱,全蝎二钱,全蜈蚣三条。煎服一剂即见轻,又服数剂全愈。 
此中风能成痿废之明征也。 
在本邑治一媪,年过六旬,其素日气虚,呼吸常觉短气。偶因劳力过度,忽然四肢痿废,卧不能起,呼 
吸益形短气,其脉两寸甚微弱,两尺重按仍有根柢,知其胸中大气下陷,不能斡旋全身也,为疏方用生箭 
一两,当归、知母各六钱,升麻、柴胡、桔梗各钱半,乳香、没药各三钱,煎服一剂,呼吸即不短气,手足 
略能屈伸。又即原方略为加减,连服数剂全愈,此气虚成痿废之明征也。 
在本邑治一媪,年五旬,于仲冬之时忽然昏倒不知人,其胸 
中似有痰涎,大碍呼吸。诊其脉,微细欲无,且甚迟缓。其家人谓其平素常觉心中发凉,咳吐粘涎。知其胸 
中素有寒饮,又感冬日严寒之气,其寒饮愈凝结杜塞也。急用胡椒三钱捣碎,煎两三沸,取浓汁多半杯灌下, 
呼吸顿形顺利。继用干姜六钱,桂枝尖、当归各三钱,连服三剂,可作呻吟,肢体渐能运动,而左手足仍不 
能动。继治以助气消痰活络之剂,左手足亦渐复旧。此痰瘀能成痿废之明征也。 
在本邑治一室女,素本虚弱,医者用补敛之药太过,月事闭塞,两腿痿废,浸至抑搔不知疼痒,其 
六脉皆有涩象,知其经络皆为瘀血闭塞也。为疏方用拙拟活络效灵丹,加怀牛膝五钱,红花钱半, 虫五个。 
煎服数剂,月事通下,两腿已渐能屈伸,有知觉。又为加生黄 、知母各三钱,服数剂后,腿能任地。然此 
等证非仓猝所能全愈,俾将汤剂作为丸剂,久久服之,自能脱然。此血瘀能成痿废之明征也。 
族兄××,冬令两腿作疼,其腿上若胡桃大疙瘩若干。自言其少时恃身体强壮,恒于冬令半冰半水之 
中捕鱼。一日正在捕鱼之际,朔风骤至,其寒彻骨,遂急还家歇息,片时两腿疼痛不能任地,因卧热炕上, 
复以浓被。数日后,觉其疼在骨,皮肤转麻木不仁,浸至两腿不能屈伸。后经医调治,兼外用热烧酒糠熨之, 
其疼与木渐愈,亦能屈伸,惟两腿皆不能伸直。有人教坐椅上,脚踏圆木棍来往,令木棍旋转,久之腿 
可伸直。如法试演,迨至春气融和,两腿始恢撤消状。然至今已三十年,每届严寒之时,腿仍觉疼,必服 
热药数剂始愈。至腿上之疙瘩,乃当时因冻凝结,至今未消者也。愚曰∶“此病犹可除根。然其寒在骨,非 
草木之品所能奏效,必须服矿质之药,因人之骨中多函矿质也。”俾先用生硫黄细末五分,于食前服之,日两 
次,品验渐渐加多,以服后觉心中微温为度。果用此方将腿疼之病除根。此风寒湿痹能 
成痿废之明征也。 
至西人谓此证关乎脑髓神经者,愚亦确有经验。原其神经之所以受伤,大抵因脑部充血所致。盖脑部充 
血之极,可至脑中血管破裂。至破裂之甚者,管中之血溢出不止,其人即昏厥不复苏醒。若其血管不至破裂, 
因被充血排挤隔管壁将血渗出,或其血管破裂少许,出血不多而自止,其所出之血若粘滞于左边司运动 
之神经,其右边手足即痿废;若粘滞其右边司运动之神经,其左边之手足即痿废。因人之神经原左右互相 
管摄也。此证皆脏腑气血挟热上冲,即《内经》所谓血之与气并走于上之大厥也。其人必有剧烈之头疼,其 
心中必觉发热,其脉象必然洪大或弦长有力。《内经》又谓此证“气反则生,不反则死”,盖气反则气下 
行,血亦下行,血管之未破裂者,不再虞其破裂,其偶些些破裂者,亦可因气血之下行而自愈;若其气不反, 
血必随之上升不已,将血管之未破裂者可至破裂,其已破裂者更血流如注矣。愚因细参《内经》之旨,而悟 
得医治此证之方,当重用怀牛膝两许,以引脑中之血下行,而佐以清火降胃镇肝之品,俾气与火不复相并上 
冲。数剂之后,其剧烈之头疼必愈,脉象亦必和平。再治以化瘀之品以化其脑中瘀血,而以宣通气血、畅 
达经络之药佐之,肢体之痿废者自能徐徐愈也。特是因脑充血而痿废者,本属危险之证,所虑者辨证不清,当 
其初得之时若误认为气虚而重用补气之品,若王勋臣之补阳还五汤,或误认为中风而重用发表之品,若《千 
金》之续命汤,皆益助其气血上行,而危不旋踵矣。至用药将其脑充血治愈,而其肢体之痿废或仍不愈, 
亦可少用参 以助其气分,然必须用镇肝、降胃、清热、通络之药辅之,方 
能有效。因拟两方于下,以备采用。 
【起痿汤】治因脑部充血以致肢体痿废,迨脑充血治愈,脉象和平,而肢体仍痿废者。徐服此药,久自能愈。 
生箭 (四钱) 生赭石(六钱轧细) 怀牛膝(六钱) 天花粉(六钱) 
玄参(五钱) 柏子仁(四钱) 生杭芍(四钱) 生明没药(三钱) 
生明乳香(三钱) 虫(四枚大的) 制马钱子末(二分) 
共药十一味。将前十味煎汤,送服马钱子末。至煎渣再服时,亦送服马钱子末二分。 
【养脑利肢汤】治同前证,或服前方若干剂后肢体已能运动而仍觉无力者。 
野台参(四钱) 生赭石(六钱轧细) 怀牛膝(六钱) 天花粉(六钱) 
玄参(五钱) 生杭芍(四钱) 生明乳香(三钱) 生明没药(三钱) 
威灵仙(一钱) 虫(四枚大的) 制马钱子末(二分) 
共药十一味。将前十味煎汤,送服马钱子末。至煎渣再服 
时,亦送服马钱子末二分。上所录二方,为愚新拟之方,而用之颇有效验,恒能随手建功,试举一案以明之。 
天津贺某,得脑充血证,左手足骤然痿废,其脉左右皆弦硬而长,其脑中疼而且热,心中异常烦躁。 
投以建瓴汤,为其脑中疼而且热,更兼烦躁异常,加天花粉八钱。连服三剂后,觉左半身筋骨作疼,盖 
其左半身从前麻木无知觉,至此时始有知觉也。其脉之弦硬亦稍愈。遂即原方略为加减,又服数剂,脉象已近 
和平,手足稍能运动,从前起卧转身皆需人,此时则无需人矣。于斯改用起痿汤,服数剂,手足之运动渐 
有力,而脉象之弦硬又似稍增,且脑中之疼与热从前服药已愈,至此似又微觉疼热,是不受黄 之升补也。 
因即原方将黄 减去,又服数剂,其左手能持物,左足能任地矣,头中亦分毫不觉疼热。再诊其脉已和平 
如常,遂又加黄 ,将方中花粉改用八钱,又加天冬八钱,连服六剂可扶杖徐步, 
仍觉乏力。继又为拟养脑利肢汤,服数剂后,心中又似 
微热,因将花粉改用八钱,又加带心寸麦冬七钱,连服十剂全愈。 
按∶此证之原因不但脑部充血,实又因脑部充血之极而至于溢血。迨至充血溢血治愈,而痿废仍 
不愈者,因从前溢出之血留滞脑中未化,而周身经络兼有闭塞处也。是以方中多用通气化血之品。又恐久 
服此等药或至气血有损,故又少加参 助之,且更用玄参、花粉诸药以解参 之热,赭石、牛膝诸药以防参 
之升,可谓熟筹完全矣。然服后犹有觉热之时,其脉象仍有稍变弦硬之时,于斯或减参 ,或多加凉药, 
精心酌斟,息息与病机相赴,是以终能治愈也。至于二方中药品平均之实偏于凉,而服之犹觉热者,诚以 
参之性可因补而生热,兼以此证之由来,又原因脏腑之热挟气血上冲也。 

<目录>三、医论
<篇名>81.论四肢疼痛其病因凉热各异之治法
属性:从来人之腿疼者未必臂疼,臂疼者未必腿疼,至于腿臂一时并疼,其致疼之因,腿与臂大抵相同矣。 
而愚临证四十余年,治愈腿臂一时并疼者不胜记。独在奉曾治一媪,其腿臂一时并疼,而致腿疼臂疼 
之病因则各异,今详录其病案于下。 
奉天佟姓媪,年五十七岁,于仲冬渐觉四肢作疼,延医服药三十余剂,浸至卧床不能转侧,昼夜疼痛 
不休。至正月初旬,求为诊视,其脉左右皆浮而有力,舌上微有白苔,知其兼有外感之热也。西药阿斯匹林 
善发外感之汗,又善治肢体疼痛,俾用一瓦半,白糖水送下,以发其汗。翌日视之,自言汗后疼稍愈,能自 
转侧。而其脉仍然有力,遂投以连翘、花粉、当归、丹参、白芍、乳香、没药诸药,两臂疼愈强半,而腿疼 
则加剧。自言两腿得热则疼减,若服热药其疼当愈。于斯又改用当归、牛膝、续断、狗脊、骨碎补、没药、 
五加皮诸药,服两剂后腿疼见愈,而臂疼又加剧。是一人之身,腿畏凉、臂畏热也。夫腿既畏凉,其疼也 
必因有凝结之凉;臂既畏热,其疼也必因有凝结之热。筹思再三,实难疏方。细诊其脉,从前之热象已无, 
其左关不任重按。恍悟其上热下凉者,因肝木稍虚,或肝气兼有郁滞,其肝中所寄之相火不能下达,所以两 
腿畏凉;其火郁于上焦,因肝虚不能敷布,所以两臂畏热。向曾治友人刘××左臂常常发热,其肝脉虚 
而且郁,投以补肝兼舒肝之剂而愈,以彼例此,知旋转上热下凉之机关,在调补其肝木而已。遂又为疏方用净 
萸肉一两,当归、白芍各五钱,乳香、没药、续断各四钱,连翘、甘草各三钱,每日煎服一剂。又俾于每 
日用阿斯匹林一瓦分三次服下。数日全愈。方中重用萸肉者,因萸肉得木气最全,酸敛之中大具条畅 
之性,是以善补肝又善舒肝。《神农本草经》谓其逐寒湿痹,四肢之作疼,亦必有痹而不通之处也。 
况又有当归、白芍、乳香、没药以为之佐使,故能奏效甚捷也。 

<目录>三、医论
<篇名>82.论治偏枯者不可轻用补阳还五汤
属性:今之治偏枯者多主气虚之说,而习用《医林改错》补阳还五汤。然此方用之有效有不效,更间有服之 
即偾事者,其故何也?盖人之肢体运动原脑髓神经为之中枢,而脑髓神经所以能司运动者,实赖脑中血管为 
之濡润,胸中大气为之斡旋。乃有时脑中血管充血过度,甚或至于破裂,即可累及脑髓神经,而脑髓神经遂 
失其司运动之常职;又或有胸中大气虚损过甚,更或至于下陷,不能斡旋脑髓神经,而脑髓神经亦恒失其司运 
动之常职。此二者,一虚一实,同为偏枯之证,而其病因实判若天渊。设或药有误投,必至凶危立见。是以 
临此证者,原当细审其脉,且细询其未病之先状况何如。若其脉细弱无力,或时觉呼吸短气,病发之后并无心 
热头疼诸证,投以补阳还五汤,恒见效。即不效,亦必不至有何弊病。若其脉洪大有力,或弦硬有力,更 
预有头疼眩晕 
之病,至病发之时,更觉头疼眩晕益甚,或兼觉心中发热者,此必上升之血过多,致脑中血管充血过甚,隔管 
壁泌出血液,或管壁少有罅漏流出若干血液,若其所出之血液,粘滞左边司运动之神经,其右半身即偏枯,若 
粘滞右边司运动之神经,其左半身即偏枯。此时若投以拙拟建瓴汤,一二剂后头疼眩晕即愈。继续服 
之,更加以化瘀活络之品,肢体亦可渐愈。若不知如此治法,惟确信王勋臣补阳还五之说,于方中重用黄 , 
其上升之血益多,脑中血管必将至破裂不止也,可不慎哉!如以愚言为不然,而前车之鉴固有医案可征也。 
邑某君,年过六旬,患偏枯原不甚剧。欲延城中某医治之,不遇。适有在津之老医初归,造门自荐。服 
其药后,即昏不知人,迟延半日而卒。后其家人持方质愚,系仿补阳还五汤,重用黄 八钱。知其必系脑 
部充血过度以致偏枯也,不然服此等药何以偾事哉? 
又尝治直隶王××,其口眼略有歪斜,左半身微有不利,时作头疼,间或眩晕,其脉象洪实,右部尤甚, 
知其系脑部充血。问其心中,时觉发热。治以建瓴汤,连服二十余剂全愈。王××愈后甚喜,而转念忽有所 
悲,因告愚曰∶“五舍弟从前亦患此证,医者投以参 之剂,竟至不起。向以为病本不治,非用药有所错误, 
今观先生所用之方,乃知前方固大谬也。”统观两案及王××之言,则治偏枯者不可轻用补阳还五汤,不 
愈昭然哉!而当时之遇此证者,又或以为中风而以羌活、防风诸药发之,亦能助其血益上行,其弊与误用参 
者同也。盖此证虽有因兼受外感而得者,然必其外感之热传入阳明,而后激动病根而猝发,是以 
虽挟有外感,亦不可投以发表之药也。 

<目录>三、医论
<篇名>83.论鼻渊治法
属性:《内经》谓“胆移热于脑则辛 鼻渊。” 者,鼻通脑之径 
路也。辛 ,则 中觉刺戟也。鼻渊者,鼻流浊涕如渊之不竭也。盖病名鼻渊,而其病灶实在于 ,因 
中粘膜生炎,有似腐烂,而病及于脑也。其病标在上,其病本则在于下,故《内经》谓系胆之移热。 
而愚临证品验以来,知其热不但来自胆经,恒有来自他经者。而其热之甚者,又恒来自阳明胃腑。胆经之热, 
大抵由内伤积热而成。胃腑之热,大抵由伏气化热而成。临证者若见其脉象弦而有力,宜用药清其肝胆之热, 
若胆草、白芍诸药,而少加连翘、薄荷、菊花诸药辅之,以宣散其热,且以防其有外感拘束也。若见其脉象 
洪而有力,宜用药清其胃腑之热,若生石膏、知母诸药,亦宜少加连翘、薄荷、菊花诸药辅之。且浊涕常流, 
则含有毒性,若金银花、甘草、花粉诸药皆可酌加也。若病久阴虚,脉有数象者,一切滋阴退热之药皆可 
酌用也。后世方书治此证者,恒用苍耳、辛夷辛温之品,此显与经旨相背也。夫经既明言为胆之移热,则不 
宜治以温药可知。且明言 辛鼻渊,不宜更用辛温之药助其 益辛,更可知矣。即使证之初得者,或因外感 
拘束,宜先投以表散之药,然止宜辛凉而不可用辛温也。是以愚遇此证之脉象稍浮者,恒先用西药阿斯匹林 
瓦许汗之,取其既能解表又能退热也。拙着石膏解中,载有重用生石膏治愈此证之案数则,可以参观。又此证 
便方,用丝瓜蔓煎汤饮之,亦有小效。若用其汤当水煎治鼻渊诸药,其奏效当尤捷也。 

<目录>三、医论
<篇名>84.详论咽喉证治法
属性:咽喉之证,有内伤外感,或凉或热,或虚或实,或有传染或无传染之殊。今试逐条详论之于下。 
伤寒病恒兼有咽喉之证。《伤寒论》阳明篇第二十节云∶“阳明病但头眩,不恶寒,故能食而咳,其 
人必咽痛。若不咳者,咽亦不痛。”按此节但言咽痛,未言治法。乃细审其文义,是由太阳初传阳 
明,胃腑之热犹未实(是以能食),其热兼弥漫于胸中(胸中属太阳当为阳明病连太阳), 
上熏肺脏,所以作咳,更因咳而其热上窜,所以咽痛。拟治以白虎汤去甘草加连翘、川贝母。 
少阴篇第三节,“病患脉阴阳俱紧,反汗出者,亡阳也,此属少阴,法当咽痛。”此节亦未列治法。 
按少阴脉微细,此则阴阳俱紧,原为少阴之变脉。紧脉原不能出汗,因其不当出汗者而反自汗,所以知其 
亡阳。其咽痛者,无根之阳上窜也。拟用大剂八味地黄汤,以芍药易丹皮,再加苏子、牛膝,收敛元阳归根以 
止汗,而咽痛自愈也。 
【加减八味地黄汤】 
大怀熟地(一两) 净萸肉(一两) 生怀山药(八钱) 生杭芍(三钱) 大云苓片(二钱) 泽泻( 
钱半) 乌附子(二钱) 肉桂(二钱去粗皮后入) 怀牛膝(三钱) 苏子(二钱研炒) 
煎汤盅半,分两次温服。 
少阴篇第三十节云∶“少阴病,下利、咽痛、胸满、心烦者,猪肤汤主之。”按此证乃少阴之热弥漫于 
三焦也。是以在上与中,则为咽痛烦满,因肾中真阴不能上升与阳分相济,所以多生燥热也;在下,则为 
下利,因脏病移热于腑,其膀胱瘀滞,致水归大肠而下利也。至治以猪肤汤者,以猪为水蓄,其肤可熬胶, 
汁液尤胜,原能助肾阴上升与心阳调剂以化燥热。而又伍以白蜜之凉润,小粉之冲和,熬之如粥,服后能 
留恋于肠胃,不致随下利泻出,自能徐徐敷布其气化,以清三焦弥漫之热也。 
少阴篇第三十一节云∶“少阴病二、三日,咽痛者,可与甘草汤。不瘥者与桔梗汤。”此亦少阴病之 
热者也。用甘草汤取其能润肺利咽,而其甘缓之性又能缓心火之上炎,则下焦之燥热可 
消也。用桔梗汤者,取其能升提肾中之真阴,俾阴阳之气互相接 
续,则上焦之阳自不浮越以铄肺熏咽,且其上达之力又善散咽喉之郁热也。按后世治咽喉证者皆忌用桔梗, 
然果审其脉为少阴病之微细脉,用之固不妨也。况古所用之桔梗皆是苦桔梗,其性能升而兼能降,实具有开通之力也。 
少阴篇第三十二节云∶“少阴病,咽中伤生疮,不能言语,声不出者,苦酒汤主之。”按少阴之脉原络 
肺上循喉咙,是以少阴篇多兼有咽喉之病。至治以苦酒汤,唐氏为苦酒与半夏同用可 
使咽中之疮速破。苦酒即今之醋,醋调生半夏末外敷原可消疮,不必皆攻之使破之。 
厥阴篇第九节云∶“伤寒先厥后发热,下利必自止,而反汗出,咽中痛,其喉为痹。”按此节之咽痛, 
以多亡阴也,与少阴篇之汗出亡阳者原互相对照。盖其人之肝脏蕴有实热,因汗出过多耗其阴液,其热遂上 
窜郁于咽中而作痛,故曰其咽为痹。痹者热与气血凝滞不散也。仲师当日未言治法,而愚思此证当用酸敛 
之药以止其汗,凉润之药以复其液,宣通之药以利其咽,汇集为方,庶可奏功。爰将所拟之方详录于下。 
【敛阴泻肝汤】 
生杭芍(两半) 天花粉(一两) 射干(四钱) 浙贝母(四钱捣碎) 酸石榴(一个连皮捣烂) 
同煎汤一盅半,分两次温服下。 
上所录伤寒兼咽喉病者六节,伤寒中之咽喉证大略已备。而 
愚临证多年,知伤寒兼病咽喉又有出于六节之外者,试举治验之案一则明之。 
愚在奉时治一朱姓学生,患伤寒三四日,蜷卧昏昏似睡,间作谵语,呼之眼微开,舌上似无苔,而 
舌皮甚干,且有黑斑,咽喉疼痛,小便赤而热,大便数日未行,脉微细兼沉,心中时觉发 
热,而肌肤之热度如常。此乃少阴伤寒之热证,因先有伏气化热,乘肾脏虚损而窜入少阴,遏抑肾气不能 
上达,是以上焦燥热而舌斑咽痛也,其舌上有黑斑者,亦为肾虚之现象。至其病既属热而脉微细者,诚以脉 
发于心,肾气因病不能上达与心相济,其心之跳动即无力,此所以少阴伤寒无论或凉或热其脉皆微细也。 
遂为疏方∶生石膏细末二两,生怀山药一两,大潞参六钱,知母六钱,甘草二钱,先用鲜茅根二两煮水,以 
之煎药,取清汤三盅,每温服一盅调入生鸡子黄一枚。服药一次后,六脉即起。服至二次,脉转洪大。服至 
三次,脉象又渐和平,精神亦复,舌干咽痛亦见愈。翌日即原方略为加减,再服一剂,诸病全愈。按上 
所用之方,即坎离互根汤。方之细解详于本方后,兹不赘。 
至于温病,或温而兼疹,其兼咽喉证者尤多。方书名其证为烂喉痧,其证多系有传染之毒菌。治之者, 
宜注意清其温热,解其疹毒,其咽喉之证亦易愈。试举治验之案以明之。 
戌辰在津,有宋××长子××患温疹兼喉证。医者皆忌重用凉药,服其药数剂,病转增剧。继延愚为 
诊视,其脉洪长有力,纯乎阳明胃腑蕴有实热;其疹似靥未靥;视其咽喉两旁红,微有烂处;心中自觉热甚; 
小便短赤;大便三日未行。为开大剂白虎汤,加连翘四钱,薄荷叶钱半以托疹外出。方中石膏重用生者四 
两,将药煎汤三盅,分三次温饮下,病大见愈,而脉仍有力,咽喉食物犹疼。继又用原方,先取鲜白茅根 
二两煮水以煎药,仍分三次服下,尽剂而愈,大便亦通下。后其次子亦患温疹喉证,较其兄尤剧。仍治以 
前方,初次即用茅根汤煎药,药方中生石膏初用三两,渐加至五两始愈。继其幼女年七岁亦患温疹喉证,较其 
两兄尤重,其疹周身成一个,肉皮皆红(俗谓此等疹皆不能治愈)。亦治以前方,为其年幼方中生石膏初用 
二两,后加至六两,其热稍退而喉痛不减,其大便六日未行,遂单用净芒硝俾淬水服下,大便即 
通,其热大减,喉痛亦愈强半。再诊其脉虽仍有力,实有浮而还 
表之象,遂用西药阿斯匹林一瓦,因病机之外越而助其出汗。果服后周身得汗,霍然全愈。 
温疹之证。西人名为猩红热,有毒菌传染,原不易治,而兼咽喉证者治之尤难。仲景所谓“阳毒为病,面 
赤斑斑如锦纹,咽喉痛,唾脓血”者,当即此证。近世方书中又名为烂喉痧,谓可治以《伤寒论》麻杏甘 
石汤。然麻杏甘石汤中石膏之分量原为麻黄之二倍。若借用其方则石膏之分量当十倍于麻黄(石膏一两麻黄一 
钱);其热甚者,石膏之分量又当二十倍于麻黄(石膏二两麻黄一钱),然后用之无弊。 
沧州友人董××,年过三旬,初则感冒发颐,继则渐肿而下延至胸膺,服药无效。时当中秋节后,淋 
雨不止,因病势危急,冒雨驱车迎愚。既至,见其颔下连颈,壅肿异常,抚之硬而且热,色甚红,纯是一 
团火毒之气,下肿已至心口;其牙关不开,咽喉肿疼,自牙缝进水半口,必以手掩口,十分用力始能下 
咽;且痰涎填满胸中,上至咽喉,并无容水之处,进水少许,必换出痰涎一口;且觉有气自下上冲,常作 
呃逆;其脉洪滑而长,重按有力,一分钟约近九十至;大便数日未行。愚曰∶“此俗所称虾蟆瘟也。其毒热 
炽盛,盘踞阳明之府,若火之燎原,必重用生石膏清之,乃可缓其毒热之势。”从前医者在座,谓曾用生石 
膏一两,毫无功效。愚曰∶“石膏乃微寒之药,《神农本草经》原有明文,仅用两许何能清此炽盛之热毒。” 
遂为疏方,用生石膏四两,清半夏四钱,金线重楼三钱,连翘二钱,射干二钱。煎服后,觉药停胸间不下,其 
热与肿似有益增之势。知其证兼结胸,火热无下行之路,故益上冲也。复急取生石膏四两,赭石三两,又 
煎汤服下,仍觉停于胸间。又急取赭石三两,蒌仁二两,芒硝八钱,又煎汤饮下,胸中仍不开通。此时 
咽喉益肿,再饮水亦不能下咽,病家惶恐无措。愚晓之曰∶“余所以连次亟亟用药者,正 
为此病肿势浸长,恐稍缓则药不能进。今其胸中既贮如许多药,断无不下行之理。药下行则结开便通,毒 
火随之下降,而上焦之肿热必消矣。”时当晚十点钟,至夜半觉药力下行,黎明下燥粪若干,上焦肿热觉轻, 
水浆可进,晨饭时牙关亦微开,服茶汤一碗。午后肿热又渐增,抚其胸,热又烙手,脉仍洪实。意其燥粪 
必未尽下,遂投以大黄四钱,芒硝五钱,又下燥粪,兼有溏粪,病遂大愈。而肿处之硬者仍不甚消,胸间 
抚之犹热,脉象亦仍有余热。又用生石膏四两,金银花、连翘各五钱,煎汤一大碗,分数次温饮下,日服 
一剂,三日全愈。按此病实温疫(疫有寒温两种而寒者甚少),确有传染至猛至烈之毒菌,是以难治。又 
按此证当二次用药时,若加硝、黄于药中,早通其大便,或不至以后如此危险,而当时 
阅历未深,犹不能息息与病机相赴也。 
有白喉证,其发白或至腐烂,实为传染病之一端。其证大抵先有蕴热,则易受传染。为其证内伤为重,宜 
用凉润滋阴清火之品,而忌用表散之剂。然用辛凉之药以散其火郁,若薄荷、连翘诸药固所不忌也。《白喉 
忌表抉微》中之养阴清肺汤、神仙活命汤二方,原为治白喉良方。而神仙活命汤中宜加连翘三钱;热甚者可 
将方中生石膏加倍,或加两倍;若大便不通者,大黄、芒硝皆可酌加。白喉之病,又恒有与烂喉痧相并者( 
参观医案中温疹兼喉痧治沈姓学生病案)。 
又《灵枢》痈疽篇谓∶“痈发于嗌中,名曰猛疽,猛疽不治,化为脓,脓不泻,塞咽,半日死。”按 
此证即后世所谓截喉痈。初起时,咽喉之间红肿甚剧,宜用消疮之药散之,兼用扁针刺之 
使多出血。若待其脓成而后泻之,恐不容待其成脓即有危险也。 
【消肿利咽汤】 
天花粉(一两) 连翘(四钱) 金银花(四钱) 丹参(三钱) 
射干(三钱) 玄参(三钱) 乳香(二钱) 没药(二钱) 
炙山甲(钱半) 薄荷叶(钱半) 
脉象洪实者加生石膏一两,小便不利者加滑石六钱,大便不通者加大黄三钱。 
咽喉之证热者居多,然亦间有寒者(参观“论喉证治法”内治刘姓童子一案)。又咽 
喉两旁微高处,西人谓之扁桃腺,若红肿西人谓之扁桃腺炎。若其处屡次红肿,渐起疙瘩,服清火药则微消, 
或略有感冒,或稍有内热复起者,此是扁桃腺炎已有根蒂,非但服药所能愈,必用手术割去之,再投以清 
火消肿之药,始能除根。若不割去,在幼童可累其身体之发达。 
《金匮》谓妇人咽中如有炙脔(吐之不出吞之不下俗谓之梅核气病),此亦咽喉证 
之一也。按∶此证注疏家谓系痰气阻塞咽喉之中,然此证实兼有冲气之冲也。原方半夏浓朴汤主之,是以半夏 
降冲,浓朴开气,茯苓利痰,生姜、苏叶以宣通其气化。愚用此方时,恒加赭石数 
钱,兼针其合谷,奏效更速(此证不但妇人男子亦间有之)。 

<目录>三、医论
<篇名>85.论喉证治法
属性:愚弱冠时已为人疏方治病,然因年少,人多不相信。值里中有病喉者,延医治疗,烦愚作陪,病者 
喉肿甚,呼吸颇难,医者犹重用发表之剂,而所用发表之药又非辛凉解肌,愚甚不以为然,出言驳之。医者谓 
系缠喉风证,非发透其汗不能消肿。病家信其说,误服其药,竟至不救。后至津门应试,值《白喉忌表抉微》 
书新出,阅之。见其立论以润燥滋阴清热为主,惟少加薄荷、连翘以散郁热,正与从前医者所用之药相反。 
因喜而试用其方,屡奏功效。后值邑中患喉证者颇多,用《白喉忌表抉微》治法,有效有不效。观喉中,不必 
发白,恒红肿异常。有言此系烂喉痧者,又或言系截喉痈者,大抵系一时之疠气流行而互相传染也。其病初 
得脉多浮而微数,或浮而有力,久则兼有洪象,此喉证兼瘟病也。 
此时愚年近三旬,临证恒自有见解,遇脉之初得浮数有力者,重 
用玄参、花粉以清其热,牛蒡、连翘以利其喉,再加薄荷叶二钱以透其表,类能奏效。其为日既深,脉象 
洪而有力者,又恒用白虎汤加银花、连翘、乳香、没药治愈。为其有截喉痈之名,间有加炙山甲,以消其 
痈肿者。其肿处甚剧,呼吸有窒碍者,恒先用铍针刺出恶血,俾肿消然后服药,针药并施,其奏功亦愈速。 
然彼时虽治愈多人,而烂喉痧、截喉痈之名究未见诸书也。后读《灵枢》痈疽篇谓“痈发于嗌中,名曰猛疽, 
猛疽不治,化为脓,脓不泻,塞咽,半日死。”经既明言痈发嗌中,此后世截喉痈之名所由来也。至谓不 
泻其脓则危在目前,是针刺泻脓原为正治之法,即不待其化脓,针刺以出其恶血亦可为正治之法矣。又阅《伤 
寒论》,“阳毒之为病面赤斑斑如锦纹,咽喉痛,唾脓血,五日可治,七日不可治。”王孟英解曰∶“阳毒 
即后世之烂喉痧耳。”是烂喉痧衍之伤寒,而相传已久,截喉痈即烂喉痧之重者也。盖白喉与烂喉痧证均有 
外感,特白喉证内伤重而外感甚轻,其外来之邪惟袭入三焦,三焦色白,是以喉现白色,故方中宣散之品但 
少用薄荷、连翘已能逐邪外出。至烂喉痧原即《伤寒论》之阳毒,其中挟有瘟毒之气,初得之时,原宜重用宣 
散之品,然宜散以辛凉,而断不可散以温热,且又宜重用凉药以佐之。此为喉证之大略也。 
而愚临证数十年,知喉证中原有诸多变证,今详录二则以备参观。 
愚在籍时,有刘姓童子,年逾十龄,咽喉肿疼,心中满闷杜塞,剧时呼吸顿停,两目上翻,身躯后挺。 
然其所以呼吸顿停者,非咽喉杜塞,实觉胸膈杜塞也。诊其脉微细而迟,其胸膈常觉发凉,有时其凉上冲, 
即不能息而现目翻身挺之象。即脉审证,知系寒痰结胸无疑。其咽喉肿疼者,寒痰充溢于上焦,迫其心肺 
之阳上浮也。为拟方∶生赭石细末一两,干姜、乌附子各三钱,浓朴、陈皮各钱半。 
煎服一剂,胸次顿觉开通,咽喉肿疼亦愈强半。又服两剂全愈。 
奉天孙××,年二十,得喉证。屡经医治,不外《白喉忌表 
抉微》诸方加减,病日增重,医者诿谓不治。后愚为诊视,其脉细弱而数,粘涎甚多,须臾满口,即得吐出。 
知系脾肾两虚,肾虚则气化不摄,阴火上逆,痰水上泛,而脾土虚损又不能制之(若脾土不虚不但能制痰水上 
泛并能制阴火上逆),故其咽喉肿疼,粘涎若斯之多也。投以六味地黄汤加于术,又少加苏子。连服十剂全愈。 

<目录>三、医论
<篇名>86.详论猩红热治法
属性:自入夏以来,各处发生猩红热,互相传染。天气炎热而病益加多加剧,治不如法,恒至不救。夫猩红 
热非他,即痧疹而兼温病也。尝实验痧疹之证,如不兼温病,其将出未出之先,不过微有寒热,或头微疼,或 
眼胞微肿,或肢体微酸懒,或食欲不振。其疹既出之后,其表里虽俱觉发热,而实无炽盛之剧热。治之者 
始终投以清表(痧疹始终宜用表药然宜表以辛凉不宜表以温热)解毒之剂,无不愈者。即或始终不服药,听其 
自出自靥,在一星期间亦可自愈。此以其但有疹毒之热,而无温病之热相助为虐,故其病易愈耳。 
至于疹而兼温者,则与斯迥异。其初病之时疹犹未出,即表里壮热,因疹毒之热尚未萌芽,而温病之 
热已炽盛也。治之者宜将薄荷、连翘、蝉蜕诸托表之药,与玄参、沙参、天花粉诸清里之药并用。其连翘 
可用三钱,薄荷叶、蝉蜕可各用钱半,玄参、沙参、花粉可各用五钱,再少加金银花、甘草解毒。若虑其痧疹 
不能透达,可用鲜茅根二两(如无可代以鲜芦根)水煮数沸,取清汤数盅,以之代水煎药,煎汤一大盅, 
温服,其疹必完全透出矣。或以外更用鲜茅根数两煎四五沸以其汤代茶,更佳。 
若其痧疹虽皆透发于外,而火犹炽盛,且深入阳明之府,其舌从前白者至此则渐黄,心中烦热异常,或 
气粗微喘,鼻翅 动,或神昏谵语,脑膜生炎,其大便干燥,小便赤涩,此乃阳明 
胃腑大实之候。而欲治阳明胃腑之实热,《伤寒论》白虎汤原为 
千古不祧之良方。为其兼有疹毒,可于方中加连翘二钱,羚羊角一钱(另煎兑服或锉细末送服或以金银花二 
钱代之),再用鲜茅根或鲜芦根煮汤,以之代水煎药。方中若用生石膏二两,可煎汤两盅,分两次温服。若用 
生石膏三两,可煎汤三盅,分三次温服。一剂热未清者,可服至数剂,以服后热退,大便仍不滑泻为度。 
若其胃腑虽有大热,因小便不利而大便滑泻者,白虎汤又不可骤服。宜先用滑石、生怀山药各一两,生 
杭芍八钱,连翘、蝉蜕各钱半,甘草三钱(此方即拙拟滋阴宣解汤),煎汤一大盅服之,其滑泻当即 
止。泻止之后,热犹不退者,宜于初次方中加滑石六钱,服之以退其热,仍宜煎汤数盅,徐徐温服。至于 
大热已退,疹已见靥,而其余热犹盛者,宜再治以滋阴清热解毒之剂,而仍少加托表之药佐之。方用玄参 
八钱,沙参、花粉各五钱,连翘、金银花、鲜芦根各三钱,甘草二钱,可连服数剂。其热降序,药剂亦宜随之 
降序,迨服至其热全消停服。以上诸方,若遇证兼喉痧者,宜于方中加射干、生蒲黄各三钱。惟治大便滑 
泻方中不宜加。可外用硼砂、生寒水石各二钱,梅片、薄荷冰各一分,共研细吹喉中。 
猩红热本非危险之证,而所以多危险者,以其证现白虎汤证时,医者不敢放胆用白虎汤治之也。至愚治 
此证时,不但胃腑大实之候可放胆投以大剂白虎汤;即当其疹初见点,其人表里壮热,脉象浮洪,但问其大 
便实者,恒用生石膏一两或两半煎汤,送服西药阿斯匹林二分,周身得微汗,其疹全发出而热亦退矣。 
曾治一六七岁幼女,病温半月不愈。其脉象数而有力,肌肤热而干涩,其心甚烦躁,辗转床上不能 
安卧。疑其病久阴亏,不堪外感之灼热,或其痧疹之毒伏藏未能透出,是以其病之现状若斯。问其大便,三 
日未行。投以大剂白虎加人参汤,以生山药代粳米,又为加连翘二钱,蝉蜕一钱,煎汤两盅,分数次温饮下。 
连服二剂,大便通下,大热已退,心中仍骚扰不安。再诊其脉, 
已还浮分,疑其余热可作汗解,遂用阿斯匹林一瓦和白糖冲水服之,周身得微汗,透出白痧若干,病遂愈。由 
斯知阿斯匹林原可为透发痧疹之无上妙药。而石膏质重气轻原亦具透表之性,又伍以最善发表之阿斯匹林,其 
凉散之力尽透于外,化作汗液而不复留中(石膏煮水毫无汁浆是以不复留中),是以胃腑之热未实而亦可用 
也。愚临证五十年,治此证者不知凡几,其始终皆经愚一人治者,约皆能为之治愈也。 
天津许姓学生,年八岁,于庚申仲春出疹,初见点两日即靥。家人初未介意。迟数日,忽又发热,其父 
原知医,意其疹毒未透,自用药表之,不效。延他医治疗,亦无效,延愚诊视,其脉象细数有力,肌肤甚热, 
问其心中,亦甚热,气息微喘,干咳无痰,其咽喉觉疼,其外咽喉两旁各起疙瘩大如桃核之巨者,抚 
之则疼,此亦疹毒未透之所致也。且视其舌苔,已黄,大便数日未行,知其阳明府热已实,必须清热与表散 
之药并用,方能有效。遂为疏方∶鲜茅根半斤(切碎),生石膏二两(捣细),西药阿斯匹林一瓦半。先将 
茅根、石膏水煮四五沸,视茅根皆沉水底,其汤即成。取清汤一大碗,分三次温饮下,每饮一次,送服阿斯匹 
林半瓦。初次饮后,迟两点钟再饮第二次。若初服后即出汗,后二次阿斯匹林宜少用。如法将药服完,翌日 
视之,上半身微见红点,热退强半,脉亦较前平和,喉疼亦稍轻,其大便仍未通下。遂将原方茅根改用五两, 
石膏改用两半,阿斯匹林改用一瓦,仍将前二味煎汤分三次送服阿斯匹林。服后疹出见多,大便通下, 
表里之热已退十之八九,咽喉之疼又轻,惟外边疙瘩则仍旧。愚恐其所出之疹仍如从前之靥急,俾每日用鲜 
茅根四两以之煮汤当茶外,又用金银花六钱,甘草三钱,煎汤一大杯,分三次温服,每次送梅花点舌丹一 
丸(若在大人可作两次服每次送服二丸)。如此四日,疙瘩亦消无芥蒂矣。 
按∶此证脉仅细数有力,原非洪大有力,似石膏可以少用,而方中犹用生石膏二两及两半者,因与若干 
之茅根同煮,而茅根之渣可以减去石膏之力也。再此证若于方中多用羚羊角数钱,另煎汤兑药中服之,亦可 
再将疹表出。而其价此时太昂,无力之家实办不到,是以愚拟得茅根、石膏、阿斯匹林并用以代之。凡证 
之宜用羚羊角者,可将此三味为方治之也。且此三味并用,又有胜于但用羚羊角之时也( 
羚羊角解下有治愈之案可参观)。 

<目录>三、医论
<篇名>87.论鼠疫之原因及治法
属性:(附∶坎离互根汤) 
自鼠疫之证流毒甚烈,医者对于此证未之前闻,即治疗此证未有专方,致患此证者百中难愈二三, 
良可慨也。不知此证发生之初,原是少阴伤寒中之热证类,至极点始酝酿成毒,互相传染。今欲知此 
证之原因及治法,须先明少阴伤寒之热证。 
尝读《伤寒论》少阴篇,所载之证有寒有热,论者多谓寒水之气直中于少阴,则为寒证;自三阳传来,则 
为热证。执斯说也,何以阴病两三日即有用黄连阿胶汤及大承气汤者?盖寒气侵人之重者,若当时窜入阴为 
少阴伤寒之寒证。其寒气侵人之轻者,伏于三焦脂膜之中,不能使人即病,而阻塞气化之流通,暗 
生内热,后因肾脏虚损,则伏气所化之热即可乘虚而入肾。或肾中因虚生热,与伏气所化之热相招引,伏气 
为同气之求,亦易入肾,于斯虚热实热,相助为虐,互伤肾阴,致肾气不能上潮于心,多生烦躁(此少阴病有心 
中烦躁之理)。再者,心主脉,而属火,必得肾水之上济,然后阴阳互根,跳动常旺;今既肾水不上潮,则 
阴阳之气不相接续,失其互根之妙用,其脉之跳动多无力(此少阴病无论寒热其脉皆微细之理)。人 
身之精神与人身之气化原相凭根据,今因阴阳之气不相接续,则精神失其凭根据,遂不能振作而昏昏欲睡 
(此少阴病但欲寐之理)。且肾阴之气既不能上潮以濡润上焦,则上焦必干而发热,口舌无津,肺脏因干 
热而咳嗽,咽喉因干热而作痛,此皆少阴之兼证,均见于少阴篇者也。《内经》谓∶“冬伤于寒,春必病 
温”,此言伏气化热为病也。然其病未必入少阴也。《内经》又谓∶“冬不藏精,春必病温”,此则多系 
伏气化热乘虚入少阴之病,因此病较伏气入他脏而为病者难于辨认,且不易治疗,故于冬伤于寒春必温病 
之外,特为明辨而重申之也。盖同是伏气发动,窜入少阴为病,而有未届春令先发于冬令者,则为少阴伤 
寒,即系少阴伤寒之热证,初得之即宜用凉药者也;其感春阳之萌动而后发,及发于夏,发于秋者,皆可为 
少阴温病,即温病之中有郁热,其脉象转微细无力者也。其病虽异而治法则同也。既明乎此,试再进而论鼠疫。 
鼠疫之证初起,其心莫不烦躁也;其脉不但微细,恒至数兼迟(间有初得脉洪数者乃鼠疫之最轻者); 
其精神颓败异常,闭目昏昏,不但欲睡,且甚厌人呼唤;其口舌不但发干,视其舌上,毫无舌苔,而舌 
皮干亮如镜;其人不但咳嗽咽痛,其肺燥之极,可至腐烂,呕吐血水(奉天人言辛亥年此证垂危时多呕吐血水)。 
由斯而论,鼠疫固少阴热证之至重者也。虽其成鼠疫之后,酿为毒菌,互相传染,变证歧出,有为结 
核性者,有为败血性者。而当其起点之初,大抵不外上之所述也,然此非愚之凭空拟议也,试举一案以征之∶ 
一九二一年,黑龙江哈尔滨一带鼠疫盛行,奉天防范甚严,未能传染入境。惟银行之间互相交通,鼠疫之 
毒菌因之有所传染。有银行施××,年三十余,发生肺炎性鼠疫,神识时明时愦,恒作谵语,四肢逆冷, 
心中发热,思食凉物,小便短赤,大便数日未行。其脉沉细而迟,心虽发热,而周身肌肤之热度无异 
常人,且闭目昏昏似睡,呼之眼微开,此诚《伤寒论》少阴篇所谓但欲寐之景象也。其舌上无苔,干亮如镜, 
喉中亦干甚,且微觉疼,时作干咳,此乃因燥生热,肾气不能上达,阴阳不相接续, 
故证象、脉象如此,其为鼠疫无疑也。此证若燥热至于极点,肺 
叶腐烂,咳吐血水,则不能治矣。犹幸未至其候,急用药调治,尚可挽回。其治之之法,当以润燥清热为主, 
又必须助其肾气,使之上达,与上焦之阳分相接续而成坎离相济之实用,则脉变洪大,始为吉兆。爰为疏方于下∶ 
生石膏(三两捣细) 知母(八钱) 玄参(八钱) 生怀山药(六钱) 野台参(五钱) 甘草(三钱) 
共煎汤三茶盅,分三次温饮下。 
按∶此方即白虎加人参汤以山药代粳米,而又加玄参也。方中之意∶用石膏以清外感之实热;用山药、知 
母、玄参以下滋肾阴、上润肺燥;用人参者,诚以热邪下陷于少阴,遏抑肾气不能上达,而人参补而兼升之 
力既能助肾气上达,更能助石膏以逐除下陷之热邪,使之上升外散也;且凡阴虚兼有实热者,恒但用白 
虎汤不能退热,而治以白虎加人参汤始能退热,是人参与石膏并用,原能立复真阴于邪热炽盛之时也。 
将药三次服完,身热,脉起,舌上微润,精神亦明了,惟大便犹未通下,内蕴之热犹未尽清。俾即原方 
再服一剂,其大便遂通下,余热亦遂尽消矣。为此证无结核败血之现象,而有肺燥、 
舌干、喉疼之征,故可名之为肺炎性鼠疫也。 
后又治一人,其病之状况大致皆与前证同,惟其脉之沉细及咽喉之干疼则较前尤甚,仍投以前方,俾用 
鲜白茅根煎汤,以之代水煎药,及将药煎成,又调入生鸡子黄同服。服后效验异常, 
因名其方为坎离互根汤。爰将其方详细录出,以备医界之采用。 
【坎离互根汤】 
生石膏(三两捣细) 知母(八钱) 玄参(八钱) 野台参(五钱) 
生怀山药(五钱) 甘草(二钱) 鸡子黄(三枚) 鲜茅根(四两切碎) 
先将茅根煎数沸,视茅根皆沉水底,取其汤以之代水,煎方 
中前六味,取汤三盅,分三次温服下。每服一次,调入生鸡子黄一枚。此方比前方多鸡子黄,而又以茅根 
汤煎药者,因鸡子黄生用善滋肾润肺,而茅根禀少阳最初之气,其性凉而上升,能发起脉象之沉细也。上方, 
乃取《伤寒论》少阴篇黄连阿胶汤与太阳篇白虎加人参汤之义,而合为一方也。黄连阿胶汤原黄连、黄芩、芍 
药、阿胶、鸡子黄并用。为此时无真阿胶,故以玄参代之;为方中有石膏、知母,可以省去黄连、黄芩诸药。 
西人谓鸡子黄中含有副肾髓质之分泌素,故能大滋肾中真阴,实为黄连阿胶汤中之主药,而不以名汤者,以 
其宜生调入而不可煎汤也。是以单用此一味,而黄连阿胶汤之功用仍在。至于白虎加人参汤中去粳米, 
而以生山药代之,以山药之性既能和胃(原方用粳米亦取其和胃),又能助玄参、鸡子黄滋肾也。用白虎汤 
以解伏气之热,而更加人参者,取人参与石膏并用,最善生津止渴,以解寒温之燥热,而其补益之 
力,又能入于下焦,以助肾气之上达,俾其阴阳之气相接续,其脉之微细者可变为洪大,而邪可外透矣。 
继又服之,脉之洪大者渐臻于和平,而病即全愈矣。 
咳嗽者,加川贝母三钱。咽喉疼者,加射干三钱。呕吐血水者,加三七细末二钱,犀角、羚羊角细末各 
一钱,三味和匀,分三次送服,或但用三七亦可。其大便不实者,宜斟酌缓服。若大便滑泻者,非下焦有寒, 
实因小便不利,宜服拙拟滋阴清燥汤,滑泻止后,再服前方,又宜将方中石膏减作二两,生山药倍作一 
两,缓缓与服。其脉象间有不微细迟缓,而近于洪数者,此乃鼠疫之最轻者,治以此方,一服当即速愈。总之, 
此证燥热愈甚,则脉愈迟弱,身转不热。若服药后脉起身热,则病机已向愈矣。 
愚初治此证时,曾但用白虎加人参汤,以生山药代粳米,治愈后,拟得此方,奏效尤捷。 
或疑寒温之证皆不传染,鼠疫既为少阴寒温证之剧者所成, 
何以独易传染?不知其传染之毒菌,皆生于病终不愈,甚至脏腑溃败,或因阴阳之气久不接续,血脉之流 
通可至闭塞,因闭塞而成腐败,此皆足以酿毒以相传染也,少阴寒温之未变鼠疫者,其剧不至此,所以 
不传染也。至此证之因传染而成者,其毒愈酝酿而愈甚,即病不甚剧而亦可传他人。所以此病偶有见端, 
即宜严为防范也。 
此证之传变,又分数种。后观哈尔滨斯年报告之病状,实甚复杂,今录其原文于下,以备参考。 
一九二一年春,哈尔滨报告文。斯年鼠疫之病状∶染后三日至七日,为潜伏期。先有头痛、眩晕、食欲不 
振、倦怠、呕吐等前驱证。或有不发前驱证者。继则恶寒、战栗,忽发大热,达39℃一40℃以上,或稽留, 
或渐次降下,淋巴管发生肿胀。在发热前或发热之一二日内,即发肿块一个,有时一侧同发两个, 
如左股腺与左腋窝腺而并发是也。该肿块或化脓,或消散,殊不一定。大部沉嗜眠睡(此即少阴证之但欲寐之 
理),夜间每发谵语。初期多泄泻二三次。尿含蛋白(此伤肾之征)。病后一二日,肝脾常见肥大。轻证三四 
日体温下降可愈。重证二日至七日多心脏麻痹(其脉象细微同于少阴病脉可知)。 
此证可分腺肿性、败血性、肺炎性百斯笃(即鼠疫)三种。腺肿百斯笃最占多数,一处或各处之淋巴管并 
其周遭组织俱发炎证。其鼠 腺及大腿上三角部之淋巴腺尤易罹之。腋窝腺及头部腺次之。又间侵后头腺、 
肘前后腺、耳前后腺、膝 腺等。其败血性百斯笃,发大如小豆之斑,疼痛颇甚,且即变为脓 ,或更 
进而变坏疽性溃疡,又有诱起淋巴腺炎者。肺炎性百斯笃之证,剧烈殊甚,一如加答儿性肺炎或格 
鲁布肺炎,咯出之痰中含有百斯笃菌,乃最猛恶者也。 
上段述鼠疫之情状,可为详悉尽致,而竟未言及治法,想西医对于此证并无确实之治法也。且其谓轻 
证三四日体温下降可愈;至其重证,体温不下降,岂不可用药使之下降?至言重证垂危,恒至心脏麻痹,推 
其麻痹之由,即愚所谓肾气不上达于心,其阴阳之气不相接续,心脏遂跳动无力,致脉象沉迟细弱也。此 
证若当其大热之初,急投以坎离互根汤,既能退热,又能升达肾气,其心脏得肾气之助,不至麻痹,即不难 
转危为安也。又其谓大部沉嗜眠睡,与愚所经历者之状似昏睡,皆有少阴病但欲寐之现象,亦足征愚谓此证 
系伏气化热入肾变成者,原非无稽之谈也。特是愚前用之方,因在奉天未见传染之毒,所以治法不备。 
后阅《山西医志》,载有厦门吴锡璜《鼠疫消弭及疗法》一篇,其用药注重解毒,实能匡愚所不逮,爰 
详录之于下,以备治斯证者之采取。 
原文∶疫菌既染,危险万状。大略分为腺鼠疫、肺鼠疫二种。其为证也,先犯心脏,使心力衰弱; 
凡脉搏如丝,即为疫毒侵犯心脏唯一之确据。其次体温速升,头痛眩晕,或作呕吐,渐渐意识朦胧,陷于昏 
睡谵语,状态痴呆,行步蹒跚,眼结膜强度充血,舌带白色,如锻石撒上,或污紫如熟李,颈腺、腋窝、大 
腿上近阴处起肿胀疼痛,剧烈者三日即死。其神气清者,可多迁 
延数日。寻常用方,有效有不效。兹将历试有效者,详细录出,以公诸医界。 
【王孟英治结核法】 
初起用王孟英治结核方合神犀丹多服累效。方用金银花二两,蒲公英二两,皂刺钱半,粉甘草一钱。 
呕者,去甘草,加鲜竹茹一两,若无鲜竹茹,可以净青黛三钱代之。大便秘、热重者,加大黄三钱, 
水煎合神犀丹服。如仍不止,用藏红花二钱煎汤,送服真熊胆二分,即止。此方用蒲公 
英、金银花、皂刺合神犀丹,不但解毒,兼能解血热、散血滞,实为治鼠疫结核之圣药。若白泡疔,本 
方去皂刺,加白菊花一两。兼黑痘,用神犀丹、紫金锭间服。 
达樵云∶“病者发头疼,四肢倦怠,骨节禁锢,或长红点,或发丹疹,或呕或泻,舌干喉痛,间有猝 
然神昏、痰涌、窍闭者,此系秽毒内闭,毒瓦斯攻心,宜用芳香辟秽、解毒护心之品,辟秽驱毒饮主之。” 
【辟秽驱毒饮】 
西牛黄(八分研冲) 人中黄(三钱) 九节菖蒲(五分) 靛叶(钱半) 
忍冬蕊(五钱鲜者蒸露亦可) 野郁金(一钱) 
水煎服。如见结核,或发斑,或生疔,加藏红花八分、桃仁三钱、熊胆四分(送服)。大渴引饮,汗多, 
加犀角、金汁。神昏谵语,宜用至宝丹或安宫牛黄丸,开水和服,先开内窍。此证初起,不可即下,审其 
口燥,神昏,热炽,有下证者,先辟秽解毒,然后议下,每获效。下法用大黄煮汤,泡紫雪丹五分良。忌 
早用大苦大寒,以致冰闭。若脉道阻滞,形容惨淡,神气模糊,恶核痛甚者,宜用解毒活血汤。 
连翘(三钱) 柴胡(二钱) 葛根(二钱) 生地(五钱) 
赤芍(三钱) 红花(五钱) 桃仁(八钱) 川朴(一钱后下) 
当归(钱半) 甘草(二钱) 苏木(二两) 
轻证初起,每三点钟服一次。危证初起,两点钟服一次,或合数剂熬膏,连连服之。或热,或渴,或 
出汗,或吐血,加生石膏一两,芦根汁一杯,和药膏服,并多服羚羊角及犀角所磨之汁。孕妇加桑寄生一两、 
黄芩一两,略减桃仁、红花。 
热甚口燥无津,脉象洪数,唇焦大渴者,用清瘟败毒饮。项 
肿者,俗名虾蟆瘟,用普济消毒饮(二方俱见《温热经纬》),多服必效。吐红涎 
者,鲜芦根取汁和服。便秘者,加大黄三钱。 
《千金方》曰∶“恶核病者,肉中忽有核,累大如李核,小如豆粒,皮肉 痛,壮热 索,恶寒是也。 
与诸疮根瘰 结筋相似。其疮根瘰 因疮而生,似缓无毒。恶核病猝然而起,有毒,若不治,入腹烦闷杀人。 
皆由冬受温风,至春夏有暴寒相搏气结成此毒也。”观此论所谓恶核,似即系鼠疫之恶核。观其所谓冬受温 
风,至春夏又感寒而发,又似愚所谓伏气化热下陷少阴,由寒温而变鼠疫也。盖伏气化热之后,恒有因薄 
受外感而后发者。由斯知鼠疫之证,自唐时已有,特无鼠疫之名耳。 
汉皋友人冉雪峰《鼠疫问题解决》,谓水不济火则为阳燥,火不蒸水则又为阴燥,火衰不交于水固为 
阴燥,水凝自不与火交亦为阴燥。鼠疫之病,阴凝成燥,燥甚化毒之为病也。又谓∶他证以脉 
洪数为热进,微弱为热退,此证则以微弱为热进,洪数为热退,皆与愚所论少阴证可变鼠疫,其病情脉状莫 
不吻合。至冉雪峰所着之书,详悉精微,无理不彻,无法不备,洵可为治鼠疫者之金科玉律,而拙论中未采 
用其方者,正以全书之方皆宜遵用,非仅可采用其一二也。欲研究鼠疫之治法者,取冉雪峰之书与拙论参观可也。 
香山友人刘蔚楚,治鼠疫结核之剧者,曾重用麝香六分,作十余次,用解毒活血清火之药煮汤,连连送 
下而愈。冉雪峰治鼠疫方中,亦有用所煮汤药送服麝香,以通络透毒者,又可补吴锡璜方中所未备也。 
栾州友人朱钵文告愚曰∶“余有善消鼠疫结核之方,用川大黄五钱,甘草五钱,生牡蛎六钱(捣碎), 
栝蒌仁四十粒(捣碎),连翘三钱。煎汤服之,其核必消。”按∶此方大黄五钱似近猛 
烈,而与甘草等分并用,其猛烈之性已化为缓和矣,所以能稳善建功也。 
绍兴何廉臣所编《全国名医验案类编》,最推重广东罗氏芝园,谓其经验弘富,细心揣摹,剖察病情 
如老吏断狱,罗列治法如名将谈兵,以活血去瘀之方,划清鼠疫主治界限,允推卓识,爰为 
节述其因、证、方、药,俾后学有所取法。 
【一探原因】城市污秽必多,郁而成 ,其毒先见。乡村污秽较少,郁而成 ,其毒次及。故热毒重 
蒸,鼠先受之,人随感之,由毛孔气管入达于血管,所以血壅不行也。血已不行,渐红渐肿,微痛微热,结 
核如瘰 ,多见于颈胁腌膀大腿之间,亦见于手足头面腹背,尔时体虽不安,犹可支持,病尚浅也。由浅而 
深,愈肿愈大,邪气与正气相搏,而热作矣。热作而见为头痛身痹,热甚而见为大汗作渴,则病已重矣。 
【二辨证候】鼠疫初起,有先恶寒者,有不恶寒者,既热之后即不恶寒,有先核而后热者,有先热而 
后核者,有热核同见者,有见核不见热者,有见热不见核者,有汗有不汗者,有渴有不渴者,皆无不头痛、 
身痛、四肢酸痹,其兼见者疗疮、 、疹、衄、嗽、咯、吐,甚则烦躁、懊 、昏谵、癫狂、痞满、腹 
痛、便结旁流、舌焦起刺、鼻黑如煤、目瞑耳聋、骨痿足肿、舌唇裂裂、脉厥体厥,种种恶证,几难悉数, 
无非热毒迫血成瘀所致。然其间亦有轻重∶核小、色白、不发热,为轻证。核小而红、头微痛、身微热、体 
微酸,为稍重证。单核红肿、大热、大渴、头痛、身痛、四肢酸痹,为重证。或陡见热渴痛痹四证,或 
初恶寒旋见四证,未见结核,及舌黑起刺,循衣摸床,手足摆舞,脉厥体厥,与疫证盛时,忽手足抽搐, 
不省人事,面身红赤,不见结核,感毒最盛,坏人至速,皆至危证。 
【三论治法方药】古方如普济消毒饮、银翘败毒散,近方如银翘散、代赈普济散等,虽皆能清热解毒, 
而无活血去瘀之药,用之多不效。惟王清任活血解毒汤∶桃仁八钱去皮尖打,红花 
五钱,当归钱半,川朴一钱,柴胡一钱,连翘三钱,赤芍三钱,生地五钱,葛根一钱,生甘草一钱。方以 
桃仁为君,而辅以当归,去瘀而通壅;连、芍为臣,而兼以地,清热而解毒,朴、甘为佐使,疏气而和药, 
气行则血通;柴、葛以解肌退热而拒邪,邪除则病愈。惟其对证用药,故能投无不效。盖此证热毒本也,瘀 
血标也,而标实与本同重。故标本未甚者,原方可愈。标本已甚者,传表宜加白虎;传里宜加承气;毒 
甚宜加羚、犀。如连进后,汗出热清,可减除柴、葛;毒下瘀少,可减轻桃、红;其他当随证加减。轻证 
照原方一服。稍重证,日夜二服,加金银花、竹叶各二钱;如口渴微汗,加石膏五钱,知母三钱。重证、 
危证、至危证,于初起恶寒,照原方服,柴胡、葛根各加一钱;若见大热,初加金银花、竹叶各三钱,西 
藏红花一钱,危证钱半,或加紫草三钱,苏木三钱。疔疮,加紫花地丁三钱,洋菊叶汁一杯冲。小便不利,加 
车前子三钱。痰多加川贝母三钱,生莱菔汁两杯冲。若痰壅神昏,又非前药可治,当加鲜石菖蒲汁一瓢 
冲,鲜竹沥两瓢冲,或礞石滚痰丸三钱包煎。若见癫狂,双剂合服,加重白虎,并竹叶心、羚角、犀角、 
西藏红花各三钱。血从上逆,见衄咯等证,加犀角、丹皮各三钱,鲜茅根、鲜芦根各四两。见 加石膏一两, 
知母五钱,元参二钱,犀角二钱。见疹加金银花、牛蒡子各三钱,竹叶、大青叶、丹皮各二钱。老弱幼 
小,急进只用单剂,日夜惟二服,加石膏,大黄减半。所加各药,小儿皆宜减半。五、六岁,一剂同煎,分 
二次服。重危之证,一剂作一服。幼小不能服药,用针刺结核三四刺,以如意油调经验涂核散(山慈菇三钱, 
真青黛一钱,生黄柏钱半,浙贝钱半,赤小豆二钱,共研细末)日夜频涂十余次可愈。妇女同治。惟孕 
妇加黄芩、桑寄生各三钱以安胎。初起即宜急服,热甚尤宜急进, 
热久胎必坠。若疑桃仁、红花坠胎,可改用紫草、紫背天葵各三 
钱。惟宜下者除芒硝。以上诸法,俱从屡次试验得来。证以强壮者为多,故于人属强壮,毒盛热旺,每于 
重危之证,必加羚角、犀角、西藏红花,取其见效较捷耳。无如人情多俭,富者闻而退缩,贫者更可知矣。 
兹为推展,分别热盛毒盛两途,随证加药,亦足以治病。如初系热盛之证,加石膏、知母、淡竹叶或螺 
靥菜(或名雷公根)、龙胆草、白茅根之类,便可清热。如兼有毒盛之证,加金银花、牛蒡子、人中黄之类, 
便可以解毒。若热毒入心包,羚角、犀角、藏红花虽属紧要,然加生竹叶心、生灯心、黄芩、栀子、麦冬心、 
莲子心、元参心之类,便可除心包之热毒。若热毒入里,加大黄、朴硝、枳壳以泻之,便可去肠胃之热毒。 
平潭友人李健颐,着有《鼠疫新篇》一书,蒙赠一册。论鼠疫之病,谓系有一种黑蚁传染于鼠,再传 
于人。其中所载之医案治法,莫不精良,而遇此证之热甚者,恒放胆重用生石膏,有一剂而用至八两者, 
有治愈一证而用至二斤强者,可为有胆有识。爰录其治愈之案一则,以为治斯病者之标准。 
平潭蔡××,年五十八岁,初起恶寒,旋即发热,热甚口渴,手足痹疼,胯下赘生一核,热痛非常,胸 
胀呕血,目赤神昏,脉数苔黄。因其先触睹死鼠,毛窍大开,毒瓦斯传入血管,潜伏体内;复因外感春阳之 
气而为引线,是以胃热则呕逆,肺伤则喷血,热深内窜肺络,肺与心近,影响阻碍,心不守舍,故昏迷谵语。 
此证涉危笃,急宜清胃、泻肺、攻毒、解热重剂急进,庶能挽救。方拟用加减解毒活血汤加石膏、芦根。 
荆芥穗(三钱) 连翘(三钱) 金银花(五钱) 浙贝母(三钱) 生地黄(五钱) 赤芍药(三 
钱) 桃仁(五钱) 川红花(三钱) 紫草(三钱) 生石膏(二两捣细) 鲜芦根(一两) 
雄黄精(一钱) 冰片(五分) 
将前十一味煎汤两盅,分两次温服。后二味共研细末,分两 
次用汤药送服。 
将药连服二剂,呕平血止,热退胸舒。将原方减雄黄,加锦纹大黄五钱,以泻胃中余毒,服两剂,诸恙悉解。 

<目录>三、医论
<篇名>88.论女子瘕治法
属性:(附∶化瘀通经散) 
女子 瘕,多因产后恶露未净,凝结于冲任之中,而流走之新血,又日凝滞其上以附益之,遂渐积 
而为 瘕矣。 者,有实可征,在一处不移。瘕者,犹可移动,按之或有或无,若有所假托。由斯而论, 
固甚于瘕矣。此证若在数月以里,其身体犹强壮,所结之 瘕犹未甚坚,可用《金匮》下瘀血汤下之。然必如 
《金匮》所载服法,先制为丸,再煎为汤,连渣服之方效。 
若其病已逾年,或至数年, 瘕积将满腹,硬如铁石,月信闭塞,饮食减少,浸成痨瘵,病势至此,再 
投以下瘀血汤,必不能任受;即能任受,亦不能将瘀血通下。惟治以拙拟理冲汤补破之药并用,其身形弱者 
服之,更可转弱为强。即十余年久积之 瘕,硬如铁石,久久服之,亦可徐徐尽消。本方后附载有治愈之 
案若干,可参观也。近在津门,用其方因证加减,治愈 瘕数人。爰录一案于下,以为治斯病之粗规。 
天津张氏妇,年近四旬,自言∶“五年之前,因产后恶露未净,积为硬块,其大如橘,积久渐大。 
初在脐下,今则过脐已三四寸矣。其后积而渐大者,按之犹软。其初积之块,则硬如铁石,且觉其处甚凉。初 
犹不疼,自今年来渐觉疼痛。从前服药若干,分毫无效,转致饮食减少,身体软弱,不知还可治否?”言 
之似甚惧者。愚曰∶“此勿忧,保必愈。”因问其月信犹通否。言从前犹按月通行,今虽些许通行,已不 
按月,且其来浸少,今已两月未见矣。诊其脉,涩而无力,两尺尤弱。爰为疏方∶生黄 
四钱,党参、白术、当归、生山药、三棱、莪术、生鸡内金各 
三钱,桃仁、红花、生水蛭各二钱, 虫五个,小茴香钱半。煎汤一大盅,温服。将药连服四剂,腹已不疼, 
病处已不觉凉,饮食加多,脉亦略有起色。遂即原方去小茴香,又服五剂,病虽未消而周遭已渐软。惟上 
焦觉微热。因于方中加玄参三钱,樗鸡八枚。又连服十余剂,其 瘕全消。 
然 瘕不必尽属瘀血也。大抵瘀血结为 瘕者,其人必碍生育,月信恒闭。若其人不碍生育,月信亦屡 
见者,其 瘕多系冷积。其身形壮实者,可用炒熟牵牛头次所轧之末三钱下之。所下之积恒为半透明白色,状 
若绿豆粉所熬之糊。若其身形稍弱者,亦可用黄 、人参诸补气之药煎汤,送服牵牛末。若畏服此峻攻 
之药者,亦可徐服丸药化之。方用胡椒、白矾各二两,再用炒熟 
麦面和之为丸,桐子大。每服钱半,日两次。服至月余,其 瘕自消。 
若其处觉凉者,多服温暖宣通之药,其积亦可下。曾治张氏妇寒积,重用附子一案,详附子解后,可参阅。 
无论血瘀冷积,日服真鹿角胶四五钱(分两次炖化服之),日久亦可徐消。盖鹿角胶原能入冲任以 
通血脉,又能入督脉以助元阳,是以无论瘀血冷积,皆能徐为消化也。 
近又拟一消 瘕兼通经闭方。用炒白术、天冬、生鸡内金等分,为细末。以治 瘕坚结及月事不通, 
每服三钱,开水送下,日再服。若用山楂片三钱煎汤,冲化红蔗糖三钱,以之送药,更佳。因用之屡有效验,爰 
名为化瘀通经散。 
鸡内金原饶有化瘀之力,能化瘀当即善消 瘕。然向未尝单用之以奏效也。因所拟理冲汤中原有生鸡内 
金三钱,方后注云∶若虚弱者,宜去三棱、莪术,将鸡内金改用四钱。鸡内金之消 瘕,诚不让三棱、莪 
术矣。夫能消 瘕,即能通月信,此原一定之理,然未经临证实验,不敢但凭理想确定也,后来津治杨氏女, 
因患瘰 过服寒凉开散之药,伤其脾胃,以致食后胀满,不能消化,重用温补脾胃之剂,加生鸡内金二钱, 
以运化药力。后服数剂来更方,言病甚见愈,惟初服此药之夜,经即通下,隔前经期未旬日耳。因其病已 
见愈,闻此言未尝注意,更方中仍有生鸡内金二钱。又服数剂,来求更方,言病已全愈,唯一月之内,行经 
三次,后二次在服药之后,所来甚少,仍乞再为调治。愚恍悟此诚因用鸡内金之故。由此可确知鸡内金通经之 
力。因忆在奉时,曾治宋氏女,胃有瘀积作疼,方中重用生鸡内金,服数剂后,二便下血而愈。此固见 
鸡内金消瘀之力,实并见鸡内金通经之力也。总前后数案参观,鸡内金消瘀通经之力,洵兼擅其长矣。此 
方中伍以白术者,恐脾胃虚弱,不任鸡内金之开通也。更辅以天冬者,恐阴虚有热,不受白术之温燥也。然 
鸡内金必须生用,方有效验,若炒熟用之则无效矣。 

<目录>三、医论
<篇名>89.论用药以胜病为主不拘分量之多少
属性:尝思用药所以除病,所服之药病当之,非人当之也(惟用药不对病者则人当之而有害矣)。乃有所用之药 
本可除病,而往往服之不效,间有激动其病愈加重者,此无他,药不胜病故也。病足以当其药而绰有余 
力,药何以能除病乎?愚感于医界多有此弊,略举前贤之医案数则、时贤之医案数则及拙治之医案数则,以 
贡诸医界同人。 
明李士材治鲁藩阳极似阴证,时方盛暑,寝门重闭,密设毡帷,身复貂被,而犹呼冷。士材往视之 
曰∶“此热证也。古有冷水灌顶法,今姑通变用之。”乃以生石膏三斤煎汤三碗,作三次 
服。一服去貂被,再服去毡帷,服至三次体蒸流汗,遂呼进粥,病若失矣。 
清道光间,归安江涵暾着《笔花医镜》,内载治一时疫发斑案,共享生石膏十四斤,其斑始透。 
吴鞠通治何姓叟,手足拘挛,误服桂附人参熟地等补阳,以致面赤,脉洪数,小便闭,身重不能转侧, 
手不能上至鬓,足蜷曲丝毫不能移动。每剂药中重用生石膏半斤,日进一剂,服至三月后,始收全功。 
又∶治蛊胀,无汗,脉象沉弦而细。投以《金匮》麻黄附子甘草汤行太阳之阳,即以泻厥阴之阴。麻黄 
去节,重用二两,熟附子两六钱,炙甘草二钱,煎汤五饭碗。先服半碗得汗至眉;二次汗至眼;约每次其 
汗下出寸许。每次服药后,即啜鲤鱼热汤以助其汗。一昼夜饮完药二剂,鲤鱼汤饮一锅,汗出至膝上,未能 
过膝。脐以上肿尽消,其腹仍大,小便不利。改用五苓散。初服不效,将方中肉桂改用新鲜紫油安边青花桂 
四钱,又加辽人参三钱,服后小便大通,腹胀遂消。 
山东海丰,清咸丰时有杨氏少妇,得奇疾∶脊背肿热,赤身卧帐中,若有一缕着身,即热不能支。适有 
宜兴苏先生乘海船赴北闱,经过其处。其人精医术,延为诊视,断为阳毒,俾用大黄 
十斤,煎汤十斤,放量陆续饮之,尽剂而愈。 
萧琢如着《遁园医案》。其案中最善用《伤寒论》、《金匮》诸方。载有治其从妹腹中寒凉作疼, 
脉象沉迟而弦紧,每剂中重用乌附子二两,连服附子近二十斤,其病始愈。 
又∶治余某妻,左边少腹内有块,常结不散,痛时则块膨胀如拳,手足痹软,遍身冷汗,不省人事,脉 
象沉紧,舌苔白浓而湿滑,面色暗晦。与通脉四逆汤,乌附子八钱;渐增至四两。煎汤一大碗,分数次饮下。 
内块降序,证亦皆见轻。病患以为药既对证,遂放胆煎好一剂顿服下,顷之面热如醉,手足拘挛,舌尖 
麻,已而呕吐,汗出,其病脱然全愈。 
刘蔚楚着《遇安斋证治丛录》,其中用大剂治愈险证尤多。 
如其治极重鼠疫,用白虎汤,生石膏一剂,渐加至斤余。治产后 
温热,用白虎加人参汤,一剂中用生石膏半斤,连服十余剂始愈。治阳虚汗脱,用术附汤,每剂术用四两, 
渐加至一斤,天雄用二两,渐加至半斤。如此胆识,俱臻极顶,洵堪为挽回重病者之不二法程也。 
至于愚生平用大剂挽回重证之案甚多,其已载于医方篇中,兹不复赘。惟即从前未登出者略录数则, 
以质诸医界同人。 
奉天王姓妇,受妊恶阻呕吐,半月勺水不存,无论何药下咽即吐出,势极危险。爰用自制半夏二两( 
自制者中无矾味善止呕吐)、生赭石细末半斤、生怀山药两半,共煎汤八百瓦药瓶一瓶(约二十两强)或凉饮 
温饮,随病患所欲,徐徐饮下,二日尽剂而愈。夫半夏、赭石皆为妊妇禁药,而愚如此放胆用之毫无顾忌者, 
即《内经》所谓“有故无殒亦无殒也”。然此中仍另有妙理,详参赭镇气汤下,可参观。 
西安县张××腿疼,其人身体强壮,三十未娶,两腿肿疼,胫骨处尤甚。服热药则加剧,服凉药 
则平平,医治年余无效。其脉象洪实,右脉尤甚;其疼肿之处皆发热,断为相火炽盛,小便必稍有不利,因 
致湿热相并下注。宜投以清热利湿之剂。初用生石膏二两,连翘茅根各三钱,煎汤服。后渐加至石膏半斤, 
连翘茅根仍旧,日服两剂,其第二剂石膏减半。如此月余,共计用生石膏十七斤,疼与肿皆大轻减;其饮 
食如常,大便日行一次,分毫未觉寒凉。嘱其仍服原方,再十余剂当脱然全愈矣。 
奉天刘某,因常受锅炉之炙热,阴血暗耗,脏腑经络之间皆蕴有热性,至仲春又薄受外感,其热陡发, 
表里俱觉壮热,医者治以滋阴清热之药,十余剂分毫无效。其脉搏近六至,右部甚实,大便两三日一行,知其 
阳明府热甚炽又兼阴分虚损也。投以大剂白虎加人参汤,生石膏用四两,人参用六钱,以生山药代方 
中粳米,又加玄参、天冬各一两,煎汤一大碗,分三次温饮下,日 
进一剂。乃服后其热稍退,药力歇后仍如故。后将石膏渐加至半斤,一日连进二剂,如此三日,热退十之八 
九,其大便日下一次,遂改用清凉滋阴之剂,数日全愈。共计所用生石膏已八斤强矣。 
愚在籍时曾治一壮年癫狂失心,六脉皆闭,重按亦分毫不见(于以知顽痰能闭脉)。投以大承气汤 
加赭石二两,煎汤送服甘遂细末三钱(此方名荡痰加甘遂汤以治癫狂之重者,若去甘遂名荡痰汤以治癫狂之 
轻者,二方救人多矣)。服后大便未行。隔数日(凡有甘遂之药不可连日服之,连服必作呕吐)将药剂加重, 
大黄赭石各用三两,仍送服甘遂三钱,大便仍无行动。遂改用巴豆霜五分,单用赭石细末四两煎 
汤送下,间三日一服(巴豆亦不可连服,若连服则肠胃腐烂矣)。每服后大便行数次,杂以成块之痰若干。 
服至两次,其脉即出。至五次,痰净,其癫狂遂愈。复改用清火化瘀之药,服数剂以善其后。 

<目录>三、医论
<篇名>90.论治疔宜重用大黄
属性:(附∶大黄扫毒汤) 
疮疡以疔毒为最紧要,因其毒发于脏腑,非仅在于经络。其脉多见沉紧。紧者毒也,紧在沉部,其毒 
在内可知也。至其重者,发于鸠尾穴处,名为半日疔,言半日之间即有关于人性命也。若系此种疔毒,当于 
未发现之前,其人或心中怔忡,或鸠尾处隐隐作疼,或其处若发炎热,似有漫肿形迹,其脉象见沉紧者,即宜 
预防鸠尾穴处生疔,而投以大剂解毒清血之品。其大便实者,用大黄杂于解毒药中下之,其疔即可暗消于 
无形。此等疔毒,若待其发出始为疗治,恒有不及治者矣。 
至若他处生疔,原不必如此预防,而用他药治之不效者,亦宜重用大黄降下其毒。忆愚少时,见同里 
患疔者二人,一起于脑后,二日死;一起于手三里穴,三日死。彼时愚已为人疏方治病,而声名未孚于乡里, 
病家以为年少无阅历,不相延也。后愚堂侄女于口角生疔,疼痛异常,心中忙乱。投以清热解毒药不效,脉 
象沉紧,大便三日未行。恍悟寒温之证,若脉象沉洪者,可用药下之,以其热在里也。今脉象沉紧,夫紧 
为有毒(非若伤寒之紧脉为寒也),紧而且沉,其毒在里可知。律以寒温脉之沉洪者可下其热,则疔 
毒脉之沉紧者当亦可下其毒也,况其大便三日未行乎。遂为疏方∶大黄、天花粉各一两,皂刺四钱,穿山甲、 
乳香、没药(皆不去油)各三钱,薄荷叶一钱,全蜈蚣三大条。煎服一剂,大便通下,疼减心安。遂去大黄, 
又服一剂,全愈。方用大黄通其大便,不必其大便多日未行,凡脉象沉紧,其大便不滑泻者,皆可用。若身 
体弱者,大黄可以斟酌少用。愚用此方救人多矣,因用之屡建奇效,遂名之为大黄扫毒汤。 
友人朱钵文传一治疔方,大黄、甘草各一两,生牡蛎六钱,栝蒌仁四十粒捣碎,疔在上者川芎三钱作引, 
在两臂者桂枝尖三钱作引,在下者怀牛膝三钱作引。煎服立愈。身壮实者,大黄可斟酌多用。此亦重 
用大黄,是以奏效甚捷也(“答陈××疑《内经》十二经有名无质”一节中有刺疔法,宜参观)。 

\part{医话}
<篇名>1.临症随笔
属性:盐山范××,年五十余,素有肺痨,发时咳嗽连连,微兼喘促。仲夏末旬,喘发甚剧,咳嗽昼夜不止,且 
呕血甚多。延医服药十余日,咳嗽呕血,似更加剧,惫莫能支。适愚自沧回籍,求为延医,其脉象洪而微数, 
右部又实而有力,视其舌苔白浓欲黄,问其心中甚热,大便二三日一行,诊毕,断曰∶此温病之热,盘 
据阳明之府,逼迫胃气上逆,因并肺气上逆,所以咳喘连连,且屡次呕血也。治病宜清其源,若将温病之热 
治愈,则咳喘、呕血不治自愈矣。其家人谓,从前原不觉有外感,即屡次延医服药,亦未尝言有外感,何 
以先生独谓系温病乎?答曰∶此病脉象洪实,舌苔之白浓欲黄,及心中之发热,皆为温病之显征。其初不觉有 
外感者,因此乃伏气化热而为温病。其受病之原因,在冬令被寒,伏于三焦脂膜之中,因春令阳盛化热而发 
动,窜入各脏腑为温病。亦有迟至夏秋而发者,其症不必有新受之外感,亦间有薄受外感不觉,而伏气即因 
之发动者,《内经》所谓∶“冬伤于寒,春必病温”者,此也。遂为疏方∶ 
生地(二两) 生石膏(一两) 知母(八钱) 甘草(一钱) 
广犀角(三钱另煎兑服) 三七(二钱细末用水送服) 
煎汤两茶盅,分三次温饮下,一剂而诸病皆愈。又改用玄参、贝母、知母、花粉、甘草、白芍诸药,煎 
汤服。另用水送服三七末钱许,服两剂后,俾用生山药末煮粥,少加白糖,每次送 
服赭石细末钱许,以治其从前之肺痨。若觉热时,则用鲜白茅根 
四五两,切碎煮两三沸,当茶饮之。如此调养月余,肺痨亦大见愈。 
按∶吐血之症,原忌骤用凉药,恐其离经之血得凉而凝,变为血痹虚劳也。而此症因有温病之壮热, 
不得不用凉药以清之,而有三七之善化瘀血者以辅之,所以服之而有益无弊也。 
盐山王××,年近六旬,自孟夏患痢,延医服药五十余剂,痢已愈而病转加剧,卧床昏昏有危在旦夕之 
虞。此际适愚自沧回籍,求为延医,其脉左右皆洪实,一息五至,表里俱觉发热,胁下连腹,疼痛异常。其 
舌苔白浓,中心微黄,大便二三日一行。愚曰∶“此伏气化热而为温病也。当其伏气化热之初,肠为 
热迫,酝酿成痢与温俱来。然温为正病,痢为兼病。医者但知治其兼病,而不知治其正病,痢虽愈而温益重。 
绵延六十余日,病者何以堪乎?”其家人曰∶“先生之论诚然,特是既为温病,腹胁若是疼痛者何也?将勿 
腹中有郁积乎?”答曰∶“从前云大便两三日一行,未必腹有郁积。以脉言之,凡温病之壮热,大抵现 
于右脉,因壮热原属阳明胃府之脉,诊于右关也,今左部之脉亦见洪实,肝胆之火必炽盛,而肝木之气,即 
乘火之炽盛而施其横恣,此腹胁所以作疼也。”遂为开大剂白虎加人参汤,方用生石膏四两,人参六钱以 
滋阴分。为其腹胁疼痛,遵伤寒方例,加生杭芍六钱,更加川楝子六钱,疏通肝胆之郁热下行,以辅芍药之 
不逮。令煎汤三茶盅,分三次温饮下。降下粘滞之物若干。持其便盆者,觉热透盆外,其病顿愈,可以进食。 
隔二日腹胁又微觉疼,俾用元明粉四钱,净蜜两半,开水调服,又降下粘滞之物若干,病自此全愈。 
愚孙,年九岁,于正月下旬感冒风寒,两三日间,表里俱觉 
发热。诊其脉象洪实,舌苔白浓。问其大便两日未行,小便色 
黄。知其外感之实热,已入阳明之府。为疏方∶ 
生石膏(二两) 知母(六钱) 连壳(三钱) 薄荷叶(钱半) 甘草(二钱) 
晚六点时煎汤两茶盅,分两次服下,翌晨热退强半。因有事他出,临行嘱煎渣与服。阅四日来信言,仍 
不愈。按原方又服一剂,亦不见轻。斯时,头面皆肿,愚遂进城往视,见其头面肿甚 
剧,脉象之热较前又盛,舌苔中心已黄,大便三日未行。为疏方∶ 
生石膏(四两) 玄参(一两) 连壳(三钱) 银花(三钱) 甘草(三钱) 
煎汤三茶盅,又将西药阿斯匹林三分,融化汤中,分三次温服下。头面周身微汗,热退肿 
消,继服清火养阴之剂两剂以善其后。 
又邻村李姓少年,亦同时得大头瘟症,医治旬日,病益剧,亦求愚治。其头面连项皆肿,心中烦躁不能 
饮食,其脉象虽有热,而重按无力。盖其旧有鸦片嗜好,下元素虚,且大便不实,不敢投以大凉之剂。为疏方∶ 
玄参(一两) 花粉(五钱) 银花(五钱) 薄荷(钱半) 甘草(钱半) 
煎汤一大盅,送服阿斯匹林二分,头面周身皆出汗,病遂脱然全愈。 
邻村孙××,年三十许,自初夏得喘症。动则作喘,即安居呼吸亦似迫促,服药五十余剂不愈。医者 
以为已成肺痨诿为不治。闻愚回籍求为延医,其脉浮而滑,右寸关尤甚,知其风与痰 
互相胶漆滞塞肺窍也。为开麻杏甘石汤∶ 
麻黄三钱、杏仁三钱、生石膏一两、甘草钱半,煎汤送服苦葶苈子(炒熟)二钱,一剂而喘定,继又服 
利痰润肺少加表散之 
剂,数服全愈。 
邻村刁××,年二十余,自孟冬得喘症。迁延百余日,喘益加剧,屡次延医服药,分毫无效。其脉浮 
而无力,数近六至,知其肺为风袭,故作喘。病久阴虚,肝肾不能纳气,故其喘浸剧也。即其脉而论,此时肺 
中之风邪犹然存在,欲以散风之药祛之,又恐脉数阴虚益耗其阴分。于是用麻黄三钱,而佐以生山药 
二两,临睡时煎服,夜间得微汗,喘愈强半。为脉象虚数,不敢连用发表之剂,俾继用生山药末八钱煮粥, 
少调白糖,当点心用,日两次,若服之觉闷,可用粥送服鸡内金末五分,如此服药约半月,喘又见轻。再 
诊其脉,不若从前之数,仍投以从前汤药方,又得微汗,喘又稍轻,又服山药粥月余全愈。 
沧县王媪,年七旬有一,于仲冬胁下作疼,恶心呕吐,大便燥结。服药月余,更医十余人,病浸加剧。 
及愚诊视时,不食者已六七日,大便不行者已二十余日。其脉数五至余,弦而有力,左右皆然。舌苔满布, 
起芒刺,色微黄。其心中时觉发热,偶或作渴,仍非燥渴。胁下时时作疼,闻食味则欲呕吐,所以不能进 
食。小便赤涩短少。此伤寒之热已至阳明之府,胃与大肠皆实,原是承气汤症。特其脉虽有力,然自弦硬中 
见其有力,非自洪滑中见其有力(此阴虚火实之脉),且数近六至,又年过七旬,似不堪承气 
之推荡。而愚有变通之法,加药数味于白虎汤中,则呕吐与胁疼皆止,大便亦可通下矣。病家闻之,疑而问 
曰∶“先生之论诚善,然从前医者皆未言有外感,且此病初起,亦未有头疼恶寒外征,何以竟成伤寒传府之重 
症?”答曰∶此乃伏气为病也。大约此外感,受于秋冬之交,因所受甚轻,所以不觉有外感,亦未能即 
病。而其所受之邪,伏于膜原之间,阻塞气化,暗生内热,遂浸养成今日之病。观此舌苔微黄,且有芒刺, 
岂非有外感之显征乎?遂为疏方∶ 
生石膏(两半) 生山药(一两) 知母(五钱) 赭石(五钱) 
川楝子(五钱) 生杭芍(四钱) 甘草(二钱) 
煎汤两盅,分三次温服下。因其胁疼甚剧,肝木不和,但理以芍药、川楝,仍恐不能奏效,又俾用羚 
羊角一钱,另煎汤当茶饮之,以平肝泻热。当日将药服完,次晨复诊,脉象已平,舌上芒刺已无,舌苔变白 
色已退强半,胁疼亦大见愈,略思饮食,食稀粥一中碗,亦未呕吐,惟大便仍未通下。疏方再用天冬、玄 
参、沙参、赭石各五钱,甘草二钱,西药硫酸镁二钱(冲服),煎服后,大便遂通下,诸病皆愈。为其 
年高病久,又俾服滋补之药数剂以善其后。 
此症之脉,第一方原当服白虎加人参汤,为其胁下作疼,所以不敢加人参,而权用生山药一两,以代 
白虎汤中之粳米,其养阴固气之力,又可以少代人参也。又赭石重坠下行,似不宜与石膏并用,以其能迫石 
膏寒凉之力下侵也。而此症因大肠甚实,故并用无妨,且不仅以之通燥结,亦以之镇呕逆也。 
沧县李氏妇,年近三旬,月事五月未行,目胀头疼甚剧,诊其脉近五至,左右皆有力,而左脉又弦硬 
而长,心中时觉发热,周身亦有热时,知其脑部充血过度,是以目胀头疼也。盖月事不行,由于血室,而血 
室为肾之副脏,实借肝气之疏泻以为流通,方书所谓肝行肾之气也。今因月事久瘀,肝气不能由下疏泻而专 
于上行,矧因心肝积有内热,气火相并,迫心中上输之血液迅速过甚,脑中遂受充血之病。惟重用牛膝,佐 
以凉泻之品,化血室之瘀血以下应月事,此一举两得之法也。遂为疏方∶ 
怀牛膝(一两) 生杭芍(六钱) 玄参(六钱) 龙胆草(二钱) 
丹皮(二钱) 生桃仁(二钱) 红花(二钱) 
一剂目胀头疼皆愈强半,心身之热,已轻减。又按其方略为加减,连服数剂,诸病皆愈,月事亦通下。 
天津李氏妇,年过四旬,患痢三年不愈,即稍愈旋又反复。其痢或赤或白或赤白参半,且痢而兼泻,其 
脉迟而无力。平素所服之药,宜热不宜凉,其病偏于凉可知。俾先用生山药细末,日日煮粥服之,又每日 
嚼服蒸熟龙眼肉两许,如此旬日,其泻已愈,痢已见轻。又俾于服山药粥时,送服生硫黄细末三分,日两 
次,又兼用木贼一钱,淬水当茶饮之,如此旬日,其痢亦愈。 
奉天吕姓童子,年五岁,于季夏初旬,周身发热,至下午三句钟时,忽又发凉,须臾凉已,其热愈烈,此 
温而兼疟也。东医治以金鸡纳霜,数日病不少减。盖彼但知治其间歇热,不知治其温热,其温热不愈,间歇 
热亦不愈。及愚视之,羸弱已甚,饮水服药,辄呕吐,大便数日未行,脉非洪大,而重按有力。知其阳明 
之热已实,其呕吐者,阳明兼少阳也。为兼少阳,所以有疟疾。为拟方∶ 
生石膏(三两) 生赭石(六钱) 生山药(六钱) 碎竹茹(三钱) 甘草(三钱) 
煎汤一盅半,分三次温饮下。将药饮完未吐,一剂大热已退,大便亦通。至翌日复作寒热,然较轻矣。 
投以硫酸规泥涅二分强,分三次用白糖水送下,寒热亦愈。 
奉天马姓幼女,于午节前得温病,医治旬日病益增剧,周身灼热,精神恍惚,烦躁不安,情势危殆,其 
脉确有实热,而至数嫌其过数。盖因久经外感灼热而阴分亏损也。遂用生石膏两半、生山药一两(单用此二味, 
取其易服),煮浓汁两茶盅,徐徐与之。连尽两剂,灼热已退,从前两日未大便,至此大便亦通,而仍有烦 
躁不安之意。遂用阿斯匹林二分,同白糖钱许,开水冲化服之,周身微汗,透出白痧满身而愈。 
或问∶外感之症,在表者当解其表,由表而传里者当清其里。今此症先清其里,后复解其表者何也?答 
曰∶子所论者治伤 
寒则然也。而温病恒表里毗连,因此表里之界线不清。其症有当日得之者,有表未罢而即传于里者,有传里 
多日而表症仍未罢者。究其所以然之故,多因此症内有伏气,又薄受外感,伏气因感而发。一则自内而外, 
一则自外而内,以致表里混淆。后世治温者,恒不以六经立论,而以三焦立论,彼亦非尽无见也。是以 
愚对于此症有重在解表,而兼用清里之药者,有重在清里而兼用解表之药者,有其症似犹可解表,因脉 
数烦躁,遂变通其方,先清其里而后解其表者。如此则服药不至瞑眩,而其病亦易愈也。下列所治之案,盖 
准此义。试观解表于清里之后,而白痧又可表出,是知临症者,原可变通因心,不必拘于一端也。 
病者 刘××,年二十五岁,寄居天津。 
病名 脏腑瘀血。 
病候 其先偶患大便下血甚剧,西医于静脉管中注射以流动麦角膏其血立止。而血止之后已月余矣,仍 
不能起床,但觉周身酸软无力。饮食不能恢撤消量,仅如从前之半。大小便亦照常,而惟觉便时不顺利。 
其脉搏至数如常,芤而无力,重按甚涩,左右两部皆然。 
诊断 此因下血之时,血不归经,行血之道路紊乱,遽用药止之,则离经之血,瘀于脏腑经络之间。 
盖麦角止血之力甚大,愚尝嚼服其小者一枚,陡觉下部会阴穴处有抽掣之力,其最能收闭血管可知。此症因其 
血管收闭之后,其瘀血留滞于脏腑之间,阻塞气化之流行。致瘀不去而新不生,是以周身酸软无力,饮食 
减少,不能起床也。此症若不急治,其周身气化阻塞日久,必生灼热。灼热久之,必生咳嗽,或成肺病,或 
成痨瘵,即难为调治矣。今幸为日未久,灼热咳嗽未作,则调治固易易也。 
疗法 当以化其瘀血为目标。将瘀血化尽,身中气化还其流 
通之常,其饮食必然增加,身体自能撤消矣。 
处方 (旱三七细末三钱) 为一日之量,分两次服,空心时开水送下。 
效果 服药数次后,自大便下瘀血若干,其色紫黑。后每大便时,必有瘀血若干,至第五日下血渐少, 
第七日便时不见瘀血矣。遂停服药,后未旬日,身体即健康如初矣。 
病者 王××,年四十九岁。 
病名 温病兼泄泻。 
病因 丙寅仲春来津。其人素吸鸦片,立志蠲除,因致身弱。于仲夏晚间,乘凉稍过,遂得温病,且兼泄泻。 
病候 表里俱壮热。舌苔边黄、中黑,甚干。精神昏愦,时作谵语。小便短涩,大便一日夜四五次,带 
有粘滞。其臭异常,且含有灼热之气,其脉左右皆洪长。重诊欠实,至数略数,两呼吸间可九至。 
诊断 此纯系温病之热,阳明与少阳合病也。为其病在阳明,故脉象洪长;为其兼入少阳,故小便 
短少,致水归大便而滑泻。为其身形素弱,故脉中虽挟有外感之实热,而仍重按不实也。 
疗法 当泻热兼补其正,又大剂徐徐服之,方与滑泻无碍也。 
处方 生石膏(三两细末) 生山药(一两) 大生地(两半) 生杭芍(八钱) 甘草(三钱) 野台参(五 
钱) 
煎汤三大盅,徐徐温饮下。一次只饮一大口,时为早六点钟,限至晚八点时服完。此方即白虎加人参汤, 
以生山药代粳米,以生地代知母,而又加白芍也。以白虎汤清阳明之热,为其脉不实故加人参;为其滑泻 
故以生山药代粳米;生地代知母,为其少阳之府有热;致小便不利而滑泻,所以又加白芍以清少阳之 
热,即以利小便也。 
效果 所备之药,如法服完。翌晨精神顿爽,大热已退,滑泻亦见愈,脉象已近平和。因泻仍不止,又 
为疏方,用生山药一两、滑石一两、生杭芍五钱、玄参五钱、甘草三钱(此即拙拟之滋阴清燥汤加玄参也) 
一剂泻止,脉静身凉,脱然全愈。 
病者 胡××之幼子,年三岁。 
病名 间歇热。 
病因 先因失乳,饮食失调,泄泻月余,甫愈,身体虚弱,后又薄受外感,遂成间歇热。 
病候 或昼或夜发灼无定时,热近两点钟,微似有汗,其热始解。如此循环不已,体益虚弱。 
诊断 此乃内伤、外感相并而为间歇热。盖外感之症,在少阳可生间歇热;内伤之病,在厥阴亦生 
间歇热(肝虚者,恒寒热往来)。 
疗法 证虽兼内伤外感,原宜内伤外感并治,为治外感用 
西药,取孺子易服;治内伤用中药,先后分途施治,方为稳妥。 
处方 (安知歇貌林一瓦) 为一日之量,分作三次,开水化服。将此药服完后,其灼必减轻,继 
用生地八钱,煎汤一茶杯,分多次徐徐温饮下,灼热当全愈。但用生地者,取其味甘易服也。 
效果 先将安知歇貌林服下,每服一次,周身皆微有凉汗,其灼热果见轻减。翌日,又将生地煎汤, 
如法服完,病即霍然愈矣。盖生地虽非补肝虚正药,而能滋肾水以生肝,更能凉润肝血,则肝得其养,其肝 
之虚者,自然转虚为强矣。 
病者 卢姓,盐山人。 
病因 孟秋天气犹热,开窗夜寝受风,初似觉凉,翌日即大热成温病。 
病候 初次延医服药,竟投以麻、桂、干姜、细辛大热之剂。服后心如火焚,知误服药,以箸探喉, 
不能吐。热极在床上乱滚,症甚危急。急来迎愚,及至,言才饮凉水若干,病热稍愈。然犹呻吟连声,不 
能安卧。诊其脉近七至,洪大无伦,右部尤甚。舌苔黄浓,大便三日未行。 
诊断 此乃阳明胃府之热已实,又误服大热之剂,何异火上添油,若不急用药解救,有危在目前之 
虞。幸所携药囊中有自制离中丹(系用生石膏一两、朱砂二分制成),先与以五钱,俾用温开水送下,过半点 
钟,心中之热少解,可以安卧。俾再用五钱送服,须臾呻吟亦止。再诊其脉,较前和平。此时可容取药, 
宜再治以汤剂以期全愈。 
处方 生石膏(三两) 知母(一两) 生山药(六钱) 玄参(一两) 
甘草(三钱) 煎汤三盅,分三次温饮下。 
效果 当日将药服完,翌日则脉静身凉,大便亦通下矣。 
奉天宫某,年三十余,胸中满闷,常作呃逆,连连不止,调治数年,病转加剧。其脉洪滑有力,关前 
尤甚,知其心火炽盛,热痰凝郁上焦也。遂用朴硝四两、白矾一两,掺炒熟麦面四两,炼蜜为丸,三钱重,每 
服一丸,日两次,服尽一料全愈。盖朴硝味原咸寒,禀寒水之气,水能胜火,寒能治热,为治心有实热者 
之要品。《内经》所谓“热淫于内,治以咸寒”也。用白矾者,助朴硝以消热痰也。调以炒熟麦面者,诚 
以麦为心谷,以防朴硝白矾之过泻伤心,且炒之则气香归脾,又能防硝矾之不宜于脾胃也。 

<目录>四、医话
<篇名>2.诊余随笔
属性:西人谓∶胆汁渗入十二指肠,能助小肠消化食物。此理《内经》未尝言之,似为中医疏忽之处,不知后 
世名医曾言之矣。吴鞠通《医医病书》曰∶“胆无出路,借小肠以为出路。”此非谓胆汁能入小肠乎?至于 
胆汁能化食之说,中医书中亦早寓其理。《神农本草经》之论柴胡也,谓“能去肠胃中结气,饮食积聚,寒 
热邪气,推陈致新。”夫柴胡为少阳胆经之主药,而其功效多见于肠胃者,为其善理肝胆,使胆汁流通无滞, 
自能入于肠中消化食物积聚,以成推陈致新之功也。至于徐灵胎注《神农本草经》则以。“木能疏土”解之, 
是谓肝胆属木、脾胃属土。徐氏既云“木能疏土”,是明谓肝胆能助肠胃化食,而胆汁能助小肠化食之 
理,即在其中矣。 
或问∶“太阳病,发热恶寒,热多寒少,脉微弱者,此无阳也,不可发汗,宜桂枝二越婢一汤。夫既曰 
无阳,何以复用石膏?既曰不可发汗,何以复用麻黄?”答曰∶“人之血分属阴,气分属阳,无阳从脉微弱 
看出,是言其气分不足也。盖证既热多寒少,其脉原当有力,若脉果有力时,可直投以越婢汤矣,或麻 
杏甘石汤。今因其气分虚而脉象微弱,故用桂枝助其脉(凡脉之微弱者服桂枝则脉大),以托肌肉中外感之 
邪外出,随麻黄以达于皮毛也。其云不可发汗者,盖症止宜解肌。麻黄发汗之力虽猛,然所用甚少, 
且有石膏凉之,芍药敛之,是以服药之后,止为解肌之小汗,而不至于为淋漓之大汗也。” 
肺脏下无透窍,而吸入之养气,实能隔肺胞息息通过,以养胸中大气,由胸中大气以敷布于全身。而 
其吸入之气,又自喉管分支下达于心,由心及肝,由肝至冲任交会之处,以及于肾。故肝肾之气化收敛, 
自能容纳下达之气,且能导引使之归根。有时 
肝肾阴虚,其气化不能固摄,则肝气忿急,可透膈以干大气,肾气膨胀,可挟冲气上冲。则肝气可挟所寄之相火上逆, 
肾气可挟副肾脏之冲气上逆。于是逆气上干,排挤胸中喉中皆不能容受外气则喘作矣。 
肺劳咳嗽,最为难治之症。愚向治此症,惟用生怀山药条(切片者,皆经水泡,不如用条),轧细过罗, 
每用两许,煮作茶汤,调以糖,令适口,以之送服川贝细末。每日两次,当点心服之。若其脾胃消化 
不良或服后微觉满闷者,可将黄色生鸡内金,轧成细末,每用二三分与川贝同送服。若觉热时,可嚼服天冬。 
此方曾治愈肺痨作喘者若干人,且能令人健壮。 
【离中丹】治肺病发热,咳吐脓血,兼治暴发眼疾,红肿作痛,头痛齿痛,一切上焦实热之症。 
生石膏(二两细末) 甘草(六钱细末) 朱砂末(一钱半) 
共和匀,每服一钱,日再服,白水送。热甚者,一次可服钱半。咳嗽甚者,方中加川贝五钱。咳血多者, 
加三七四钱。大便不实者,将石膏去一两,加滑石一两,用生山药面熬粥,送服此丹。若阴虚作喘者,亦宜山 
药粥送服。至于山药面熬粥自五钱可至一两。 
下焦寒凉泄泻及五更泻者,皆系命门相火虚衰。确能补助相火之药,莫如硫黄,且更莫如生硫黄。为其为 
石质之药,沉重下达耳。不经水煮火烁,而其热力全也(硫黄无毒,其毒即其热,故可生用)。然愚向用 
硫黄治寒泻症,效者固多,兼有服之泻更甚者,因本草原谓其大便润、小便长,岂以其能润大便即可作泻乎? 
后阅西人药性书,硫黄原列于轻泻药药中。乃知其服后间作泻者,无足怪也。且其所谓轻泻药者,与中医说所谓大 
便润者,原相通也。于斯再用硫黄时,于石质药中,择一性温且饶有收涩之力者佐之,即无斯弊。 
且但热下焦而性不僭上,胜于但知用桂附者远矣。若于方中再少 
加辛香之品,引其温暖之力以入奇经,更可治女子血海虚寒不孕。 
【坎中丹】治下焦寒凉泄泻及五更泻。 
硫黄(纯黄者一两) 赤石脂(一两) 
共为细末和匀。每服五分,食前服,一日两次。不知则渐渐 
加多,以服后移时微觉温暖为度。若以治女子血海虚寒不孕者,宜于方中加炒熟小茴香末二钱。 
孙××按∶治此症时,恒加肉桂少许与石脂硫黄同服,且其止腹痛之力甚速也。 
【逐风通痹汤】治风袭肌肉经络,初则麻木不仁,浸至肢体关节不利。 
生箭 (六钱) 麻黄(三钱) 全当归(五钱) 丹参(三钱) 
乳香(三钱) 没药(三钱) 全蝎(二钱) 
脉象迟弱无力恶寒者,将黄 重用一两,再照加乌头二三钱;脉象有力恶热者,以薄荷易麻黄,再加 
天花粉一两。初服以遍体皆得微汗为佳,至汗后再服,宜将麻黄减半,或止用一钱;筋骨软弱者,加明天麻 
三钱;口眼歪斜者,加蜈蚣二条,其病剧者,可加三条。此风中身之外廓,未入于脏腑也。是以心中无病, 
而病在于肌肉、肢体、经络、关节之处。《内经》风论篇谓∶“风气与太阳俱入行诸脉俞,散于分肉之间, 
与卫气相干,其道不利,故使肌肉愤 而有疡,卫气有所凝而不行,故其肉有不仁也。”又《内经》痹论 
曰∶“风、寒、湿三气杂至,合而为痹也。其风气胜者为行痹,寒气胜者为痛痹,湿气胜者为着痹。”据《 
内经》二节之文观之,则风袭人之肌肉经络,可使麻木不仁,浸至肢体关节不利可知也。是以方中以黄 为主 
药,取其能升补胸中大气以通于卫气,自能逐风外出。故《神农本草经》谓∶黄 能主大风,而又以最善 
发表之麻黄辅之。一则扶正以祛邪,一则发汗以透邪,二药相济为用,其逐风之力虽猛,而实 
不至伤正气也。至 
当归、丹参、乳没、全蝎诸药,或活血以祛风,或通络以祛风,皆所以赞助黄 、麻黄以成功也。至于病 
偏凉者加乌头,更将黄 增重;病偏热者加花粉,更以薄荷易麻黄,此随病机之所宜,以细为调剂,不使 
服药后有觉凉觉热之龃龉也。筋骨软弱者加明天麻,取其能壮筋骨兼能祛风也;口眼歪斜者加蜈蚣,取其善理 
脑髓神经,而有牵正口眼之力也。 
曾治一人,夏月开轩当窗而寝,为风所袭,其左半身即觉麻木,肌肉渐形消瘦,左手足渐觉不遂,为 
拟此方。其病偏于左,又加鹿角胶二钱作引(若偏于右宜用虎骨胶作引),一剂周身得汗,病愈强半,即方 
略为加减,又服二剂全愈。后屡试其方莫不随手奏效。 
各处庭院中,多有络石与 HT ,此二种皆木本藤蔓类,而皆可入药。络石∶蔓粗而长,叶若红薯,其 
节间出须,须端作爪形,经雨露濡湿,其爪遂粘于砖石壁上,俗呼为爬山虎,即药局中之络石藤也。本草又 
名为石龙藤,其性善治喉痹肿塞,用鲜者两半,煎汤一盅,细细呷之,少顷即通。其性又善通经络,同续 
断、菟丝子煮酒(须用黄酒不宜用烧酒),日日饮之。或单用络石,煮酒饮之,善治周身拘挛,肢体作痛。若 
与狗脊、猴姜煮酒饮之,善治腰疼。若兼腿疼者,宜加牛膝。《名医别录》又谓∶此物久服能轻身、明目、润 
泽、好颜色、不老。诚如《名医别录》之所云云,则每日以之煮汤当茶饮之,其为益不亦多乎? 
HT ∶蔓类络石而稍细,花叶若鸡爪形,又多分歧,以其须缠于高树之枝柯上。其藤中多通气细孔,截 
断吹之有浆出,可擦疮疡肿毒。其性亦善治淋,煎汤当茶,最善止渴。取其叶捣汁饮之,善治呕哕。其所结之 
实,大如广红豆,形圆色红而亮,中有浆微甘微酸,其功用能止渴,益气力,悦颜色。俗传有谓其善解 
砒石毒者,然未见其出载,此则待质高明也。 
愚于诸药多喜生用,欲存其本性也。有如石膏为硫养轻钙化 
合,若 之则硫养轻皆飞去,其凉散之力顿失,而所余之钙经 即变质,断不可服。故斯编之中于生石膏之 
能救人, 石膏之能伤人,反复论之,再三致意,以其关于人命之安危甚重也。又如赭石原铁养化合,其 
重坠凉镇之力最善降胃止血,且又能生血,分毫不伤气分。至药局中所鬻之赭石,必 以煤火,则铁养分 
离,即不能生血,且更淬之以醋,转成开破之性,多用之即可令人泄泻。又如赤石脂原系粉末,宜兴茶壶 
即用此烧成。为其质同粉末,有粘滞之性,研细服之可保护肠胃之内膜,善治大便泄泻。而津沽药局中竟将 
石脂为细末,水和为泥,捏作小饼, 以煤火,即与宜兴壶瓦无异,若为末服之,其伤人脾胃也必矣。又 
如山萸肉,其酸温之性能补肝、敛肝、治肝虚自汗以固元气之将脱,实能挽回人命于至危之候。药局多酒 
浸蒸黑用之,其敛肝固气之力顿减矣。如此者实难枚举。此所以愚于药品多喜生用以存其本性也。 
药有非制过不可服者,若半夏、附子、杏仁诸有毒之药皆是也。虽古方中之附子,亦偶生用,实系卤水 
淹透,未经炮熟之附子,亦非采取即用也。凡此等药,方中虽未注明如何炮制,坊间亦必为制至无毒。若其 
药本无毒,原可生用者,斯编方中若未注明制用,皆宜生用。有用斯编之方者,甚勿另加制法,致失药之本性也。 
威灵仙、柴胡诸药,原是用根。坊间恒杂以茎叶,医者不知甄别,即可误事。细辛之叶,其功用亦不 
如根,故李濒湖《本草纲目》亦谓用根。至樗白皮与桑白皮,亦皆用根上之皮,其真伪尤属难辨,用者必自采 
取方的。如樗根白皮,大能固涩下焦。而带皮樗枝煎汤,又能通大便。俗传便方,大便不通者,用带皮樗 
枝七节,每节长寸许,煎汤服之甚效。其枝与根性之相异如此,用者可不慎哉。 
煎时易沸之药,医者须预告病家。如知母若至五六钱,微火煎之亦沸,若至一两几不能煎。然此药 
最易煎透,先将他药煎十余沸,再加此药,敞开药罐盖,略煎数沸,其汤即成。至若山药、 
阿胶诸有汁浆之药,龙骨、牡蛎、石膏、滑石、赭石诸捣末之药,亦皆易沸。大凡煎药,其初滚最易沸。 
煎至将滚时,须预将药罐之盖敞开,以箸搅之。迨沸过初滚,其后仍沸,敞盖煎之无妨,若不沸者,始可盖 
而煎之。盖险急之证,安危止争此药一剂。故古之医者,药饵必经己手修制,即煎汤液,亦必亲自监视也。 

<目录>四、医话
<篇名>3.自述治愈牙疼之经过
属性:愚素无牙疼病。丙寅腊底,自津回籍,因感冒风寒,觉外表略有拘束,抵家后又眠于热炕上,遂陡觉心 
中发热,继而左边牙疼。因思解其外表,内热当消,牙疼或可自愈。服西药阿斯匹林一瓦半(此药原以一 
瓦为常量),得微汗,心中热稍退,牙疼亦觉轻。迟两日,心中热又增,牙疼因又剧。方书谓上牙龈属足 
阳明,下牙龈属手阳明,愚素为人治牙疼有内热者,恒重用生石膏少佐以宣散之药清其阳明,其牙疼即愈。 
于斯用生石膏细末四两,薄荷叶钱半,煮汤分两次饮下,日服一剂。两剂后,内热已清,疼遂轻 
减。翌日因有重证应诊远出,时遍地雪深三尺,严寒异常,因重受外感,外表之拘束甚于初次,牙疼因又 
增剧,而心中却不觉热。遂单用麻黄六钱(愚身体素强壮是以屡次用药皆倍常量非可概以之治他人也), 
于临睡时煎汤服之。未得汗。继又煎渣再服,仍未得汗。睡至夜半始得汗,微觉肌肤松畅,而牙疼如故。剧时 
觉有气循左侧上潮,疼彻辅颊,且觉发热。有时其气旁行,更疼如锥刺。恍悟此证确系气血挟热上冲,滞于 
左腮,若再上升至脑部,即为脑充血矣。遂用怀牛膝、生赭石细末各一两煎汤服之,其疼顿愈,分毫不复 
觉疼,且从前 
头面畏风,从此亦不复畏风矣。盖愚向拟建瓴汤用治脑充血证甚效,方中原重用牛膝、赭石,今单 
用此二药以治牙疼,更捷如影响,此诚能为治牙疼者别开一门径矣,是以详志之。 
附录∶ 
唐山赵××来函∶ 
二小儿年十二岁,右边牙疼,连右腮亦肿疼。因读先生自述治愈牙疼之经过,知腮肿系外感受风,牙 
疼系胃火炽盛,遂先用西药阿斯匹林一瓦。服后微见汗。继用生石膏二两,薄荷叶钱半,连服三剂,全愈。 

<目录>四、医话
<篇名>4.虚劳温病皆忌橘红说
属性:半夏、橘红皆为利痰之药,然宜于湿寒之痰,不宜于燥热之痰,至阴虚生热有痰,外感温热有痰,尤所 
当忌。究之伍药得宜,半夏或犹可用,是以《伤寒论》竹叶石膏汤、《金匮》麦门冬汤皆用之。至橘红则无 
论伍以何药,皆不宜用。试略举数案于下以明之。 
本邑于姓媪,劳热喘嗽,医治数月,病益加剧,不能起床,脉搏近七至,心中热而且干,喘嗽连连,势 
极危险。所服之方,积三十余纸,曾经六七医生之手,而方中皆有橘红。其余若玄参、沙参、枸杞、天冬、 
贝母、牛蒡、生熟地黄诸药,大致皆对证,而其心中若是之热而干者,显系橘红之弊也。愚投以生怀山药一 
两,玄参、沙参、枸杞、龙眼肉、熟地黄各五钱,川贝、甘草各二钱,生鸡内金钱半。煎服一剂,即不觉干。 
即其方略为加减,又服十余剂,全愈。 
又∶治奉天李××,得风温证,发热、头疼、咳嗽。延医服药一剂,头疼益剧,热嗽亦 
不少减。其脉浮洪而长,知其阳明经 
府皆热也。视所服方,有薄荷、连翘诸药以解表,知母、玄参诸药以清里,而杂以橘红三钱,诸药之功尽为 
橘红所掩矣。为即原方去橘红,加生石膏一两,一剂而愈。 
又∶治沧州孙××肺脏受风,咳嗽吐痰。医者投以散风利痰之剂,中有毛橘红二钱,服后即大口吐血, 
咳嗽益甚。其脉浮而微数,右部寸关皆有力。投以《伤寒论》麻杏甘石汤,方中生石膏用一两,麻黄用一钱, 
煎汤送服旱三七细末二钱。一剂血止。又去三七,加丹参三钱,再服一剂,痰嗽亦愈。方中加丹参者, 
恐其经络中留有瘀血,酿成异日虚劳之证,故加丹参以化之。 
统观以上三案,橘红为虚劳温病之禁药,不彰彰可考哉!而 
医者习惯用之,既不能研究其性于平素,至用之病势增进,仍不知为误用橘红所致,不将梦梦终身哉! 

<目录>四、医话
<篇名>5.鳖甲、龟板不可用于虚弱之证
属性:《神农本草经》论鳖甲主心腹 瘕坚积。《金匮》鳖甲煎丸用之以消疟母(胁下硬块)。药局又皆以 
醋炙,其开破肝经之力尤胜。向曾单用鳖甲末三钱,水送服,以治久疟不愈,服后病者觉怔忡异 
常,移时始愈,由斯知肝虚弱者,鳖甲诚为禁用之品也。又龟板《神农本草经》亦主 瘕,兼开湿痹。后世 
佛手散用之,以催生下胎。尝试验此药,若用生者,原能滋阴潜阳,引热下行,且能利小便(是开湿痹之效)。 
而药局中亦皆用醋炙之,若服至一两,必令人泄泻,其开破之力虽逊于鳖甲,而与鳖甲同用以误治虚弱之证, 
实能相助为虐也。乃后世方书用此二药以治虚劳之证者甚多,即名医如吴鞠通,其治温邪深入下焦,热深 
厥深,脉细促,心中 大动,此邪实正虚,肝风煽动将脱,当用白虎加人参汤,再加龙骨、牡蛎,庶可 
挽回,而吴氏竟治以三甲复脉汤,方中鳖甲、龟板并用,虽有牡蛎之收涩,亦将何补?此乃名医之偶失检点也。 
愚曾治一媪年近五旬,患温病半月不愈。其左脉弦硬有真气不敛之象,右脉近洪而不任重按,此邪实证 
虚也,为拟补正祛邪之剂。病者将药饮一口,嫌其味苦不服。再延他医,为开三甲复脉汤方,略有加减,服 
后烦躁异常,此心肾不交,阴阳将离也。医者犹不省悟,竟于原方中加大黄二钱,服后汗出不止。此时若 
重用山萸肉二两,汗犹可止,汗止后,病仍可治,惜该医见不及此,竟至误人性命也。 

<目录>四、医话
<篇名>6.目疾由于伏气化热者治法
属性:目疾有实热之证,其热屡服凉药不解,其目疾亦因之久不愈者,大抵皆因伏气化热之后,而移热于 
目也。丙寅季春,李××,纺纱厂学徒,病目久不愈。眼睑红肿, 肉遮睛,觉目睛胀疼甚剧,又兼耳聋 
鼻塞,见闻俱废,跬步须人扶持。其脉洪长甚实,左右皆然。其心中甚觉发热,舌有白苔,中心已黄,其从前 
大便原燥,因屡服西药大便日行一次。知系冬有伏寒,感春阳而化热,其热上攻,目与耳鼻皆当其冲也。拟 
用大剂白虎汤以清阳明之热,更加白芍、龙胆草兼清少阳之热。病患谓厂中原有西医,不令服外人药,今 
因屡服其药不愈,偷来求治于先生,或服丸散犹可,断乎不能在厂中煎服汤药。愚曰∶“此易耳。我有自 
制治眼妙药,送汝一包,服之眼可立愈。”遂将预轧生石膏细末两半与之,嘱其分作六次服,日服三次,开水 
送下,服后又宜多喝开水,令微见汗方好。持药去后,隔三日复来,眼疾已愈十之八九,耳聋鼻塞皆愈, 
心中已不觉热,脉已和平。复与以生石膏细末一两,俾仍作六次服。将药服尽全愈。至与以生石膏细末而 
不明言者,恐其知之即不敢服也。后屡遇因伏气化热病目者,治以此方皆效。 

<目录>四、医话
<篇名>7.天水散治中暑宜于南方北方用之宜稍变通
属性:河间天水散(即六一散),为清暑之妙药。究之南方用之最为适宜;若北方用之,原宜稍为变通。盖南方 
之暑多挟湿,故宜重用滑石,利湿即以泻热。若在北方,病暑者多不挟湿,或更挟有燥气,若亦重用滑石以利 
其湿,将湿去而燥愈甚,暑热转不易消也。愚因是拟得一方,用滑石四两,生石膏四两,粉甘草二两,朱砂 
一两,薄荷冰一钱,共为细末,每服二钱,名之曰加味天水散。以治北方之暑病固效,以治南方之暑病,亦 
无不效也。方中之义∶用滑石、生石膏以解暑病之热;而石膏解热兼能透表,有薄荷冰以助之,热可自肌肤散 
出,滑石解热兼能利水,有甘草以和之(生甘草为末服之最善利水,且水利而不伤阴),热可自小便泻出, 
又恐暑气内侵,心经为热所伤,故仿益元散之义加朱砂(天水散加朱砂名益元散)以凉心血,即以镇安神明, 
使不至怔忡瞀乱也。 
人受暑热未必即病,亦恒如冬令伏气伏于膜原,至秋深感凉气激薄而陡然暴发,腹疼作泻。其泻也,暴 
注下迫,恒一点钟泻十余次。亦有吐泻交作者。其甚者,或两腿转筋。然身不凉,脉不闭,心中惟觉热甚, 
急欲饮凉食冰者,此仍系暑热为病,实与霍乱不同。丁卯季夏,暑热异常,中秋节后发现此等证甚多,重 
用生石膏煎汤送服益元散,其病即愈。腹中疼甚者,可用白芍、甘草(益元散中甘草甚少故加之)与石膏同煎 
汤,送服益元散。若泻甚者,可用生山药、甘草与石膏同煎汤,送服益元散;或用拙拟滋阴清燥汤加生石膏两 
余或二两,同煎服,病亦可愈。其欲食冰者,可即与之以冰,欲饮井泉凉水者,可即与之以井泉水,听其尽量 
食之饮之无碍也。且凡吐不止者,若欲食冰,听其尽量食之,其吐即可止,腹疼下泻亦可并愈。其间有不并愈 
者,而其吐既止,亦易用药为之调治也。 

<目录>四、医话
<篇名>8.治幼年温热证宜预防其出痧疹
属性:幼年温热诸证,多与痧疹并至。然温热之病,初得即知。至痧疹初得,其毒恒内伏而外无现象,或迟至 
多日始出;又或不能自出,必俟服托表之药而后能出。若思患预防,宜于治温热之时,少用清表痧疹之药。不 
然,恐其毒盘结于内,不能发出,其温热之病亦不能愈也。愚临证数十年,治愈温热兼痧疹者不胜计,莫 
不于治温热药中,时时少加以清表痧疹之品,以防痧疹之毒内蕴而不能透出。故恒有温热之病,经 
他医治疗旬日不愈,势极危险,后经愚为延医,遂发出痧疹而愈者。今略登数案于下,以为征实。 
奉天马氏幼女,年六七岁,得温病,屡经医治,旬余病势益进,亦遂委之于命,不复治疗。适其族 
家有幼子得险证,经愚治愈,因转念其女病犹可治,殷勤相求。其脉象数而有力,肌肤热而干涩,卧床上展转 
不安,其心中似甚烦躁。以为病久阴亏,不堪外感之灼热,或其痧疹之毒伏藏于内,久未透出,是以其病之 
现状如是也。问其大便,数日一行。遂为疏方,生石膏细末二两,潞党参四钱,玄参、天冬、知母、生怀山 
药各五钱,连翘,甘草各二钱,蝉蜕一钱,煎汤两盅,分数次温饮下。连服二剂,大热已退,大便通下,其精 
神仍似骚扰不安。再诊其脉,较前无力而浮。疑其病已还表,其余热当可汗解,用西药阿斯匹林二分强, 
和白蔗糖水冲服下。周身微汗,透出白痧若干而愈。乃知其从前展转骚扰不安者,因其白痧未发出也。为 
每剂中皆有透表之品,故其病易还表,而其痧疹之毒复亦易随发汗之药透出也。 
又∶奉天刘××之幼女,年五岁,周身发热,上焦躁渴,下焦滑泻,迁延日久,精神昏愦,危至极点, 
脉象数而无力,重诊即无。为疏方用生怀山药一两,滑石八钱,连翘、生杭芍、甘草 
各三钱,蝉蜕、羚羊角(此一味另煎当水饮之煎至数次尚有力)各一钱半,煎汤一盅半, 
分三次温服下,周身发出白痧,上焦烦渴、下焦滑泻皆愈。 
按∶此方即滋阴宣解汤加羚羊角也。凡幼年得温热病即滑泻者,尤须防其痧疹之毒内伏不能外出( 
滑泻则身弱恒无力托痧疹之毒外出),此方既能清热止泻,又能表毒外出,所以一药而愈也。 
奉天王××子,年二十八岁,周身发热,出白痧甚密。经医调治失宜,迁延至旬日,病益加剧。医者 
又欲用大青龙汤减去石膏,王××疑其性热,不敢用,延愚为之延医。其周身发热,却非大热,脉数五至,似 
有力而非洪实,舌苔干黑,言语不真,其心中似怔忡,又似烦躁,自觉难受莫支。其家人谓其未病之时,实劳 
心过度,后遂得此病。参之脉象病情,知其真阴内亏,外感之实热又相铄耗,故其舌干如斯,心中之怔忡 
烦躁又如斯也。问其大便,数日未行,似欲便而不能下通。遂疏方用生石膏细末三两,潞党参五钱,生山药 
五钱,知母、天花粉各八钱,连翘、甘草各二钱,生地黄一两半,蝉蜕一钱,俾煎汤三盅,分三次温饮 
下,又嘱其服药之后,再用猪胆汁少调以醋,用灌肠器注射之,以通其大便,病家果皆如所嘱。翌日视之, 
大便已通下,其灼热、怔忡、烦躁皆愈强半,舌苔未退而干黑稍瘥。又将凉方减石膏之半,生地黄改用一两。 
连服三剂,忽又遍身出疹,大便又通下,其灼热怔忡烦躁始全愈。恐其疹出回急,复为开清毒托表之 
药,俾服数剂以善其后。 
按∶此证既出痧矣,原不料其后复出疹,而每剂药中皆有透表之品者,实恐其蕴有痧毒未尽发出也。 
而疹毒之终能发出,实即得力于此。然非临时细细体察,拟方时处处周密,又何能得此意外之功效哉! 
又∶此证非幼科,亦因温而兼疹,故连类及之,且俾人知温而兼疹之证,非独幼科有之,即 
壮年亦间有之也。 

<目录>四、医话
<篇名>9.痫疯治法
属性:(附∶愈痫丸、息神丸) 
痫疯最为难治之证,因其根蒂最深,故不易治耳。愚平素对于此证,有单用磨刀水治愈者;有单用熊 
胆治愈者;有单用芦荟治愈者;有用磁朱丸加赭石治愈者。然如此治法,效者固多,不效者亦恒有之,仍觉 
对于此证未有把握。后治奉天王氏妇,年近三旬,得痫疯证,医治年余不愈,浸至每日必发,且病势较重。 
其证甫发时作狂笑,继则肢体抽掣,昏不知人。脉象滑实,关前尤甚。知其痰火充盛,上并于心,神不守舍, 
故作狂笑;痰火上并不已,迫激脑筋,失其所司,故肢体抽掣,失其知觉也。先投以拙拟荡痰汤,间日一剂。 
三剂后,病势稍轻,遂改用丸药,硫化铅、生赭石、芒硝各二两,朱砂、青黛、白矾各一两,黄丹五钱,共 
为细末,复用生怀山药四两为细末,焙熟,调和诸药中,炼蜜为丸二钱重。当空心时,开水送服一丸,日 
两次。服至百丸全愈。 
奉天刘姓学生,素患痫疯。愚曾用羚羊角加清火、理痰、镇肝之药治愈。隔二年,证又反复,再投以 
原方不效。亦与以此丸,服尽六十丸全愈。 
沈阳县乡间童子,年七八岁,夜间睡时骚扰不安,似有抽掣之状,此亦痫疯也。亦治以此丸,服至四十丸全愈。 
此丸不但治痫疯,又善治神经之病。奉天赵××,年五十许,数年头迷心乱,精神恍惚,不由自主,屡 
次医治不愈。亦治以此丸,惟方中白矾改为硼砂,仍用一两,亦服至百丸全愈。因 
此丸屡用皆效,遂名此丸为愈痫丸。而以硼砂易白矾者,名为息神丸。 
附∶制硫化铅法 用真黑铅、硫黄细末各一斤。先将铅入铁 
锅中熔化,即将硫黄末四五两撒在铅上,黄即发焰,急用铁铲拌 
炒,所熔之铅即结成砂子。其有未尽结者,又须将硫黄末接续撒其上,勿令火熄,仍不住拌熔化之铅,尽 
结成砂子为度。待晾冷,所结砂子色若铅灰,入药钵细研为粉。去其研之成饼者,所余之 
粉用芒硝半斤,分三次冲水,将其粉煮过三次,然后入药。 

<目录>四、医话
<篇名>10.小儿痉病治法
属性:小儿为少阳之体,是以或灼热作有惊骇,其身中之元阳,恒挟气血上冲以扰其脑部,致其脑筋妄行, 
失其所司而痉证作矣。痉者其颈项硬直也,而或角弓反张,或肢体抽掣,亦皆盖其中矣。此证治标之药中, 
莫如蜈蚣(宜用全的)。用治标之药以救其急,即审其病因,兼用治本之药以清其源,则标本并治,后 
自不反复也。 
癸亥季春,愚在奉天,旬日之间,遇幼童温而兼痉者四人。愚皆以白虎汤治其温,以蜈蚣治其痉, 
其痉之剧者,全蜈蚣用至三条,加白虎汤中同煎服之,分数次饮下,皆随手奏效(其详案皆在药物蜈蚣解 
下,又皆少伍以他药,然其紧要处全在白虎汤蜈蚣并用)。 
又∶乙丑季夏,愚在籍,有张姓幼子患暑温兼痉,其痉发时,气息皆闭,日数次,灼热又甚剧,精神 
异常昏愦,延医数人皆诿为不治。子××投以大剂白虎汤,加全蜈蚣三条,俾分三次饮下,亦一剂而愈。 
丙寅季春,天津俞姓童子病温兼出疹,周身壮热,渴嗜饮水,疹出三日,似靥非靥,观其神情,恍 
惚不安,脉象有力,摇摇而动,似将发痉。为开白虎汤加羚羊角钱半(另煎兑服,此预防其发痉所以未用 
蜈蚣)。药未及煎,已抽搐大作。急煎药服下,顿愈。 
至痉之因惊骇得者,当以清心、镇肝、安魂、定魄之药与蜈蚣并用,若朱砂、铁锈水、生龙骨、生 
牡蛎诸药是也。有热者,加羚羊角、青黛。有痰者,加节菖蒲、胆南星。有风者,加全 
蝎、僵蚕。气闭塞及牙关紧者,先以药吹鼻得嚏,后灌以汤药。 

<目录>四、医话
<篇名>11.癞证治法
属性:癞之为证,方书罕载。愚初亦以为犹若疥癣不必注意也。自戊午来奉天诊病,遇癞证之剧者若干,有 
患证数年,费药资甚巨不能治愈者,经愚手,皆服药数剂全愈。后有锦州县戎××患此证,在其本地服药无 
效,来奉求为延医,服药六剂即愈。隔三年,其证陡然反复,先起自面上,状若顽癣,搔破则流黄水, 
其未破之处,皮肤片片脱落,奇痒难熬,歌哭万状。在其本处服药十余日,分毫无效,复来奉求为延医。其脉 
象洪实,自言心中烦躁异常,夜间尤甚,肤愈痒而心愈躁,彻夜不眠,若再不愈,实难支持。遂为疏方,用 
蛇蜕四条,蝉蜕、僵蚕、全蝎、甘草各二钱,黄连、防风各三钱,天花粉六钱,大枫子十二粒,连皮捣碎。为 
其脉洪心躁,又为加生石膏细末两半。煎汤两茶盅,分两次温饮下。连服三剂,面上流黄水处皆结痂,其有 
旧结之痂皆脱落,瘙痒烦躁皆愈强半,脉之洪实亦减半。遂去石膏,加龙胆草三钱。服一剂,从前周身之似有 
似无者,其癞亦皆发出作瘙痒。仍按原方连服数剂,全愈。至方中之药,诸药皆可因证加减,或用或不 
用,而蛇蜕则在所必需,以其既善解毒(以毒攻毒),又善去风,且有以皮达皮之妙也。若畏大枫子有毒, 
不欲服者,减去此味亦可。 

<目录>四、医话
<篇名>12.治梦遗法
属性:梦遗之病,最能使人之肾经虚弱。此病若不革除,虽日服补肾药无益也。至若龙骨、牡蛎、萸肉、金樱 
诸固涩之品,虽服之亦恒有效,而究无确实把握。此乃脑筋轻动妄行之病,惟西药若 
臭剥抱水诸品,虽为麻醉脑筋之药,而少用之实可以安靖脑筋。 
若再与龙骨、牡蛎诸药同用,则奏效不难矣。愚素有常用之方,爰录于下∶ 
龙骨一两, 牡蛎一两,净萸肉二两,共为细末,再加西 
药臭剥十四瓦,炼蜜为百丸。每临睡时服七丸,服至两月,病可永愈。 

<目录>四、医话
<篇名>13.肢体受寒疼痛可熨以坎离砂及坎离砂制法
属性:药局中所鬻坎离砂,沃之以醋自能发热,以熨受寒腿疼及臂疼,颇有效验,而医者犹多不知其所以然 
之故。究其实际,不外物质化合之理也。按此砂纯系用铁屑制成。其制法将铁屑 红,即以醋喷灭之,晾 
干收贮。用时复以醋拌湿,即能生热。盖火非养气不着,当铁屑 红之时,铁屑中原具有养气,经醋喷灭,其 
养气即永留铁中。况养气为酸素,醋味至酸,其含养气颇多,以之喷灭 红之铁,醋中之养气亦尽归铁中。 
用时再沃之以醋,其从前所蕴之养气,遂感通发动而生热。以熨因寒痹疼之处,不惟 
可以驱逐凝寒,更可流通血脉,以人之血脉得养气则赤,而血脉之瘀者可化也。 

<目录>四、医话
<篇名>14.牛肉反红荆之目睹
属性:敝邑多红荆,而县北泊庄尤多,各地阡塍皆有荆丛绕护。乙巳季春,牛多瘟死,剥牛者弃其脏腑,但食 
其肉,未闻有中毒者。独泊庄因食牛肉,同时中毒者二百余人,迎愚为之解救,既至(相距七里许)死者已 
三人矣。中毒之现象∶发热、恶心、瞑眩、脉象紧数。投以黄连、甘草、金银花、天花粉诸药,皆随手 
奏效。细询其中毒之由,缘洗牛肉于溪中,其溪中多浸荆条,水色变红,洗后复晾于荆条闸极上,至煮肉 
时又以荆为薪,及鬻此肉,食者皆病,食多则病剧,食少则病轻耳。愚闻此言,因恍忆“老牛 
反荆花”原系邑中相传古语,想邑中古人必有中此毒者,是以其语至今留诒,人多知之。特其事未经见,虽 
知之亦淡然若忘耳。然其相反之理,究系何因,须俟深于化学人研究也。因又忆曩时阅小说,见有田家妇 于 
田间,行荆芥中,以 之饭有牛肉,食者遂中毒。疑荆芥即系红荆之讹,不然岂牛肉反荆花,而又反荆 
芥耶?医界诸大雅,有能确知之者,又期不吝指教。 

<目录>四、医话
<篇名>15.驳方书贵阳抑阴论
属性:尝思一阴一阳,互为之根,天地之气化也。人禀天地之气化以生,即人身各具一小天地,其气化何独 
不然。是以人之全身,阴阳互相维系,上焦之阳藏于心中血,中焦之阳涵于胃液,下焦之阳存于肾水,凡心 
血、胃液、肾水皆阴也。充类言之,凡全身津液脂膏脉腺存在之处,即元阳留蓄之处。阳无阴则飞越,阴无 
阳则凝滞。阳盛于阴则热,阴盛于阳则冷。由斯知阴阳偏盛则人病,阴阳平均则人安,阴阳相维则人生,阴 
阳相离则人死。彼为贵阳抑阴之论者,竟谓阳一分未尽则人不死,阴一分未尽则人不仙,斯何异梦中说梦也。 
然此则论未病之时,阴阳关于人身之紧要,原无轩 也。若论已病,又恒阳常有余,阴常不足(朱丹溪曾有此论)。 
医者当调其阴阳,使之归于和平,或滋阴以化阳,或泻阳以保阴,其宜如此治者,又恒居十之八九。倘曰 
不然,试即诸病征之。 
病有内伤外感之殊,而外感实居三分之二。今先以外感言之,伤寒、温病、疫病皆外感也,而伤寒中于 
阴经,宜用热药者,百中无二三也;温病则纯乎温热,已无他议;疫病虽间有寒疫,亦百中之一二也。他 
如或疟,或疹,或痧证,或霍乱,亦皆热者居多,而暑 之病更无论矣。 
试再以内伤言之。内伤之病,虚劳者居其半,而劳字从火, 
其人大抵皆阴虚阳盛,究之亦非真阳盛,乃阴独虚致阳偏盛耳。他如或吐衄,或淋痢,或肺病、喉病、眼疾, 
或黄胆,或水病、肿胀、二便不利,或嗽,或喘,或各种疮毒,以上诸证,已为内伤之大凡,而阳盛阴虚者 
实为十之八九也。世之业医者,能无于临证之际,以急急保其真阴为先务乎?即其病真属阳虚,当用补 
阳之药者,亦宜少佐以滋阴之品;盖上焦阴分不虚而后可受参、 ,下焦阴分不虚而后可受桂、附也。 

<目录>四、医话
<篇名>16.阅刘华封《烂喉痧证治辨异》书后
属性:丙寅中秋后,接到刘华封自济南寄赠所着《烂喉痧证治辨异》一书。细阅一过,其辩证之精,用药之妙, 
立论之通,于喉证一门实能令人起观止之叹。咽喉为人身紧要之处,而论喉证之书向无善本。自耐修子托 
之鸾语,着《白喉忌表抉微》,盛行于一时,初则用其方效者甚多,继而用其方者有效有不效,更有用 
之不惟不效而病转增剧者。于斯议论纷起,有谓白喉不忌表散,但宜表以辛凉,而不可表以温热者。又有谓 
白喉原宜表散,虽麻黄亦可用,但不可与升提之药并用者。按其人或有严寒外束不得汗,咽喉疼而不肿者,原 
可用麻黄汤解其表。然麻黄可用,桂枝不可用。若用麻黄汤时,宜去桂枝,加知母、连翘。至升提之药,惟忌 
用升麻。若桔梗亦升提之药,而《伤寒论》有桔梗汤治少阴病咽痛,因其能开提肺气散其咽喉郁热也。若 
与凉药并用,又能引凉药之力至咽喉散热。惟咽喉痛而且肿者,似不宜用。又有于《白喉忌表抉微》一书痛 
加诋毁,谓其毫无足取者。而刘华封则谓白喉证原分两种,耐修子所谓白喉忌表者,内伤之白喉也。其病因 
确系煤毒、纸烟及过服煎炒辛热之物,或贪色过度,以致阴液亏损虚火上炎所致,用药养阴清肺原为正治。 
其由外感传染者,为烂喉痧,喉中亦有白腐,乃系天行时气入于阳明,上 
蒸于肺,致咽喉溃烂,或兼有疹子,正是温热欲出不得所致,正宜疏通发表使毒热外出。二证之辨∶白喉则 
咽中干,喉痧则咽中多痰涎。白喉止五心烦热,喉痧则浑身大热云云。诚能将此二证,一内因,一外因, 
辨别极精。及至后所载治喉痧诸方,详分病之轻重浅深,而措施咸宜,洵为喉科之金科玉律也。惟其言今日 
之好人参难得,若用白虎加人参汤及小柴胡汤,方中人参可以沙参代之,似非确论。盖小柴胡汤中之人参或 
可代以沙参,若当下后小柴胡汤证仍在者,用小柴胡汤时,亦不可以沙参代人参。至白虎加人参汤,若其 
热实脉虚者,以沙参代人参其热必不退,此愚由经验而知,非想当然尔之谈也。且古方中人参即系今之党参, 
原非难得之物。若恐人工种植者不堪用,凡党参之通体横纹者(若胡莱菔之纹)皆野生之参也。至其后论 
喉证原有因下焦虚寒迫其真阳上浮致成喉证者,宜治以引火归原之法,洵为见道之言。 

<目录>四、医话
<篇名>17.答严××代友问痰饮治法
属性:详观来案,知此证乃寒饮结胸之甚者。拙拟理饮汤,原为治此证的方,特其药味与分量宜稍为更改耳。 
今拟一方于下,以备采择。方用生箭 一两,干姜八钱,于术四钱,桂枝尖、茯苓片、炙甘草各三钱,浓朴、 
陈皮各二钱,煎汤服。方中之义∶用黄 以补胸中大气,大气壮旺,自能运化水饮,仲景所谓“大气 
一转,其气(指水饮之气)乃散”也,而黄 协同干姜、桂枝,又能补助心肺之阳,使心肺阳足,如日丽 
中天,阴霾自开;更用白术、茯苓以理脾之湿,浓朴、陈皮以通胃之气,气顺湿消,痰饮自除;用炙甘草者, 
取其至甘之味,能调干姜之辛辣,而干姜得甘草,且能逗留其热力,使之绵长,并能缓和其热力,使不猛烈也。 
按∶此方即《金匮》苓桂术甘汤,加黄 、干姜、浓朴、陈皮, 
亦即拙拟之理饮汤去芍药也。原方之用芍药者,因寒饮之证,有迫其真阳外越,周身作灼,或激其真阳上窜, 
目眩耳聋者,芍药酸敛苦降之性,能收敛上窜外越之元阳归根也(然必与温补之药同用方有此效)。 
此病原无此证,故不用白芍。至黄 在原方中,原以痰饮既开、自觉气不足者加之。兹则开始即重用黄 者, 
诚以寒饮固结二十余年,非有黄 之大力者,不能斡旋诸药以成功也。 
又按∶此方大能补助上焦之阳分,而人之元阳,其根柢实在于下,若更兼服生硫黄,以培下焦之阳, 
则奏效更速。所言东硫黄亦可用,须择其纯黄者方无杂质,惟其热力减少,不如中硫黄耳。其用量,初次可 
服细末一钱,不觉热则渐渐加多。一日之极量,可至半两,然须分四五次服下。不必与汤药同时服,或先或后均可。 
【附原问】敝友患寒饮喘嗽,照方治疗未效。据其自述病因,自二十岁六月遭兵燹,困山泽中,绝饮 
食五日夜,归家急汲井水一小桶饮之,至二十一岁六月,遂发大喘。一日夜后,饮二陈汤加干姜、细辛、 
五味渐安。从此痰饮喘嗽,成为痼疾。所服之药,大燥大热则可,凉剂点滴不敢下咽。若误服之,即胸气急 
而喘作,须咳出极多水饮方止。小便一点钟五六次,如白水。若无喘,小便亦照常。饮食无论肉味菜蔬, 
俱要燥热之品。粥汤、菜汤概不敢饮。其病情喜燥热而恶冷湿者如此。其病状暑天稍安,每至霜降后朝朝发 
喘,必届巳时吐出痰饮若干,始稍定。或饮极滚之汤,亦能咳出痰饮数口,胸膈略宽舒。迄今二十六七载 
矣。近用黎芦散吐法及十枣汤等下法,皆出痰饮数升,证仍如故。《金匮》痰饮篇及寒水所关等剂,服过数 
十次,证亦如故。想此证既能延岁月,必有疗法,乞先生赐以良方,果能祓除病根,感佩当无既也。又 
《衷中参西录》载有服生硫黄法,未审日本硫黄可服否? 
服药愈后谢函∶接函教,蒙授妙方,治疗敝友奇异之宿病,连服四五剂,呼吸即觉顺适。后又 
照方服七八剂,寒饮消除,喘证全愈。 

<目录>四、医话
<篇名>18.答翁××问呃逆气郁治法
属性:详观一百十一号(绍兴医药学星期报)所登之案,其呃逆终不愈者,以其虚而兼郁也。然观其饱时加重, 
饥时见轻,知病因之由于郁者多,由于虚者少。若能令其分毫不郁,其呃当止。郁开呃止,气化流通,虽 
有所虚,自能渐渐撤消。特是理虚中之郁最为难事,必所用之药分毫不伤气化,俾其郁开得一分,其气 
化自能撤消一分,始克有效。拙拟医方篇中载有卫生防疫宝丹,原系治霍乱急证之方,无论其证因凉因热, 
皆屡试屡验。后有沈阳赵××,得温病甚剧,舁至院中求为延医,数日就愈,忽作呃逆,昼夜不止, 
服药无效。因思卫生防疫宝丹,最善行气理郁,俾一次服五十粒,呃逆顿止。又数日有陈姓患呃逆证,旬日 
不止,眠食俱废,精神疲惫,几不能支。亦治以卫生防疫宝丹,俾服八十粒,亦一次即愈。由斯知卫生防疫 
宝丹,治呃逆确有把握,无论其为虚、为郁,用之皆可奏效也。盖方中冰片、薄荷冰为透窍通气之妙 
药,而细辛善降逆气,白芷善达郁气,朱砂能镇冲气之冲逆,甘草能缓肝气之忿激,药非为呃逆专方,而无 
一味非治呃逆必需之品,是以投之皆效也。若其人下元虚甚者,可浓煎生山药汁送服。其挟热者,白芍、麦冬 
煎汤送服。其挟寒者,干姜、浓朴煎汤送服。愚用之数十次,未有不随手奏效者。若仓猝不暇作丸 
药,可为末服之。 

<目录>四、医话
<篇名>19.答金××问治吐血后咳嗽法
属性:详观病案,知系因吐血过多,下焦真阴亏损,以致肾气不敛,冲气上冲。五更乃三阳升发之时,冲气 
上冲者必益甚,所以脑筋跳动,喘嗽加剧也。欲治此症,当滋阴纳气,敛冲镇肝,方 
能有效。爰拟方于下以备酌用∶ 
生山药(一两) 大熟地(一两) 净萸肉(六钱) 怀牛膝(六钱) 柏子仁(六钱) 生 
龙骨(四钱) 生牡蛎(四钱) 生赭石(四钱) 生鸡内金(二钱) 玄参(二钱) 炙甘草(二钱) 
日服一剂,煎渣重服。 

<目录>四、医话
<篇名>20.答胡××问小儿暑天水泻及由泻变痢由疟转痢之治法
属性:小儿少阳之体,不堪暑热,恒喜食凉饮冷以解暑,饮食失宜,遂多泄泻,泻多亡阴,益至燥渴多饮,而 
阴分虚损者,其小溲恒不利,所饮之水亦遂尽归大肠,因之泄泻愈甚,此小儿暑天水泻所以难治也。而所拟 
之方,若能与证吻合,则治之亦非难事。方用生怀山药一两,滑石八钱,生杭芍六钱,甘草三钱,煎 
汤一大盅,分三次温饮下。一剂病减,再剂全愈矣。方中之意∶山药滋真阴,兼固其气;滑石泻暑热, 
兼利其水,甘草能和胃,兼能缓大便,芍药能调肝,又善利小便;肝胃调和其泄泻尤易愈也。此方即拙拟滋 
阴清燥汤。原治寒温之证,深入阳明之府,上焦燥热,下焦滑泻。而小儿暑天水泻,其上焦亦必燥热,是以宜 
之。至于由泻变痢,由疟转痢者,治以此方,亦能随手奏效。何者?暑天热痢,最宜用天水散;方中滑石, 
甘草同用,固河间之天水散也。又可治以芍药甘草汤;方中白芍、甘草同用,即仲景之芍药甘草汤也。且由 
泻变痢,由疟转痢者,其真阴必然亏损,气化必不固摄,而又重用生山药为之滋阴固气化,是以无论由泄 
变痢,由疟转痢者皆宜。若服此药间有不效者,可加白头翁三钱,因白头翁原为治热痢之要药也。 

<目录>四、医话
<篇名>21.答章××问腹内动气冲气症治法
属性:观此症,陡有气自脐上冲至胸腔,集于左乳下跳动不休。夫 
有气陡起于脐上冲者,此奇经八脉中冲脉发出之气也。冲脉之原,上隶于胃,而胃之大络虚里,贯膈络肺出 
于左乳下为动脉。然无病者其动也微,故不觉其动也。乃因此冲气上冲犯胃,且循虚里之大络贯膈络肺,复出 
于左乳下与动脉相并,以致动脉因之大动,人即自觉其动而不安矣。当用降冲、敛冲、镇冲、补冲之 
药以治病源,则左乳下之动脉,自不觉其动矣。爰拟两方于下∶ 
生山药(八钱) 生牡蛎(八钱) 生赭石末(四钱) 生芡实(四钱) 
清半夏(足四钱中有矾须用温水淘净晒干) 柏子仁(四钱炒捣不去油) 寸麦冬(三钱) 
上药七味,磨取铁锈浓水煎药。 
又方∶用净黑铅半斤,用铁勺屡次熔化之,取其屡次熔化所余之铅灰若干,研细过罗。再将熔化所余 
之铅秤之,若余有四两,复用铁勺熔化之。化后,用硫黄细末两半,撒入勺中,急以铁铲炒拌之,铅经硫黄灼 
炼,皆成红色,因炒拌结成砂子。晾冷、轧细,过罗,中有轧之成饼者,系未化透之铅,务皆去净。二药各 
用一两,和以炒熟麦面为丸(不宜多掺,以仅可作成丸为度),如桐子大。每服六七丸或至十余丸(以服后觉药 
力下行,不至下坠为度),用生山药末五六钱,煮作稀粥送下,一日再服。以上二方单用、同用皆可。 

<目录>四、医话
<篇名>22.答章××代友问病案治法
属性:详观病案,知系胃阴亏损,胃气上逆,当投以滋胃液,降胃气之品。然病久气虚,又当以补气之药佐之。 
爰拟方于下,放胆服之,必能止呕吐,通大便。迨至饮食不吐,大便照常,然后再拟他方。方用∶生赭石 
二两,生山药一两,潞党参五钱,天冬八钱,共煎汤两茶杯,分三次温服下。渣煎一杯半,再分两次温服 
下。一剂煎两次,共分五次服,日尽一剂,三剂后吐必止,便必顺。用此方者,赭石千万不可减轻。若 
此药服之觉凉者,可加生姜四五片或初服时加生姜四五片亦可。 

<目录>四、医话
<篇名>23.答庞××问大便脱肛治法
属性:脱肛之症,用曼陀罗煎浓汤洗之甚效。仆常用鲜曼陀罗四五斤,煎取浓汁两三大碗。再以其汁煎萸肉二 
三两,取浓汁一大碗。再用党参二两,轧细末调汁中,晒干。每用四五钱,水煎融化洗之,数次可全愈。 

<目录>四、医话
<篇名>24.答徐××读《伤寒论》质疑四则
属性:古人之书不可不信,又不可尽信。愚不揣固陋,敢将徐××所疑《伤寒论》四则,反复陈之。 
第一疑∶在太阳下编第二十节。其节为病在太阳之表,而不知汗解,反用凉水 之、灌之,其外感之寒 
已变热者,经内外之凉水排挤,不能出,入郁于肉中而烦热起粟,然其热在肌肉,不在胃府,故意欲饮水而 
不渴,治宜文蛤散。夫文蛤散乃蛤粉之未经 炼者也。服之,其质不化,药力难出,且虽为蛤壳,而实则 
介虫之甲,其性沉降,达表之力原甚微,借以消肉上之起粟似难奏功。故继曰∶“若不瘥者,与五苓散。”其 
方取其能利湿兼能透表,又能健运脾胃以助利湿透表之原动力,其病当瘥矣。然又可虑者,所灌之凉水过多, 
与上焦外感之邪互相胶漆而成寒实结胸,则非前二方所能治疗矣。故宜用三物小陷胸汤或白散。夫白散之辛 
温开通,用于此证当矣。至于三物小陷胸汤,若即系小陷胸汤,用于此证,以寒治寒,亦当乎?注家谓此 
系反治之法。夫反治者,以热治寒,恐其 格而少用凉药为引,以为热药之反佐,非纯以凉药治寒也。盖注 
者震摄于古人之隆名,即遇古书有舛错遗失之处,亦必曲为原护,不知此正所以误古人而更贻误后 
人也。是以拙着《衷中参西录》,于古方之可确信者,恒为之极力表彰,或更通变化裁,推行尽致, 
以穷其妙用;于其难确信者,则 
恒姑为悬疑,以待识者之论断。盖欲为医学力求进化,不得不如斯也。 
此节中三物小陷胸汤,唐容川疑其另为一方,非即小陷胸汤。 
然伤寒太阳病实鲜有用水 、水灌之事,愚疑此节非仲景原文也。 
第二疑∶在太阳下编三十二节。其节为∶“太阳病,医发汗,遂发热恶寒,因复下之,心下痞,表里俱 
虚,阴阳气并竭,无阳则阴独,复加烧针,因胸烦,面色青黄,肤 者难治,今色微黄,手足温者,易治。” 
按此节文义,必有讹遗之字。阴阳气并竭句,陈氏释为阴阳气不交,甚当。至无阳则阴独句,鄙意以为 
独下当有结字。盖言误汗误下,上焦阳气衰微,不能宣通,故阴气独结于心下而为痞也。 
第三疑∶在太阳下编五十四节。其节为伤寒脉浮滑。夫滑则热入里矣,乃滑而兼浮,是其热未尽入里,半 
在阳明之府,半在阳明之经也。在经为表,在府为里,故曰表有热,里有寒。《内经》谓“热病者,皆伤寒 
之类也。”又谓“人之伤于寒也,则为病热。”此所谓里有寒者,盖谓伤寒之热邪已入里也。陈氏之解 
原如斯,愚则亦以为然。至他注疏家,有谓此寒热二字宜上下互易,当作外有寒里有热者。然其脉象既现 
浮滑,其外表断不至恶寒也。有谓此寒字当系痰之误,因痰寒二音相近,且脉滑亦为有痰之证也。然在寒温, 
其脉有滑象原主阳明之热已实,且足征病者气血素充,治亦易愈。若因其脉滑而以为有痰,则白虎汤岂为 
治痰之剂乎? 
第四疑∶在阳明篇第七十六节。其节为病患无表里证,盖言无头痛项强恶寒之表证,又无腹满便硬之 
里证也。继谓发热七八日虽脉浮数者可下之,此数语殊令人诧异。夫脉浮宜汗,脉数忌下,人人皆知,况其脉 
浮数并见而竟下之,其病不愈而脉更加数也必矣。故继言假令已下脉数不解云云。后则因消谷善饥,久不 
大便而复以抵当汤下之。夫寒温之证脉数者,必不思饮食,未见有消谷善饥者。且即消谷善饥,不大便,何 
以见其必有瘀血,而轻投以抵当汤乎?继则又言若脉数仍不解而下不止云云,是因一下再下而其人已下脱也。 
夫用药以解其脉数,其脉数未解,而转致其下脱,此其用药诚为节节失宜,而犹可信为仲景之原文乎? 
试观阳明篇第三十一节,仲景对于下证如何郑重。将两节文对观,则此节为伪作昭然矣。夫古经之中,犹不 
免伪作(如尚书之今文),至方术之书,其有伪作也原无足深讶。所望注疏家审为辨别而批 
判之,不至贻误于医界,则幸甚矣! 

<目录>四、医话
<篇名>25.答徐××阳旦汤之商榷
属性:阳旦汤即桂枝加桂汤再加附子,盖此系他医所治之案,其失处在证原有热,因脚挛误认为寒,竟于 
桂枝中增桂加附,以致汗出亡阳,遂至厥逆,仲景因门人之问,重申之而明其所以厥逆之故,实因汗出亡阳。 
若欲挽回此证使至夜半可愈,宜先急用甘草干姜汤以回其阳,虽因汗多损液以致咽干,且液伤而大便燥结成阳 
明之谵语,亦不暇顾。迨夜半阳回脚伸,惟胫上微拘急,此非阳之未回,实因液伤不能濡筋也。故继服芍 
药甘草汤以复其津液,则胫上拘急与咽喉作干皆愈。更用承气汤以通其大便,则谵语亦遂愈也。所用之药息 
息与病机相符,故病虽危险可挽回也。 

<目录>四、医话
<篇名>26.答王××问《神州医药学报》何以用真武汤治其热日夜无休止立效
属性:《伤寒论》真武汤乃仲景救误治之方。其人本少阴烦躁,医者误认为太阳烦躁而投以大青龙汤,清之散 
之太过,遂至其人真阳欲脱,而急用真武汤以收回其欲脱之元阳,此真武汤之正用 
也。观《神州医药学报》所述之案,原系外感在半表半里,中无 
大热,故寒热往来,脉象濡缓,而投以湿温之剂,若清之散之太过,证可变为里寒外热(即真寒假热),其 
元阳不固较少阴之烦躁益甚,是以其热虽日夜无休止,口唇焦而舌苔黄腻,其脉反细数微浮而濡也。若疑脉 
数为有热,而数脉与细浮濡三脉并见实为元阳摇摇欲脱之候,犹火之垂垂欲灭也。急用真武汤以迎回元阳,俾 
复本位,则内不凉而外不热矣。是投以真武汤原是正治之法,故能立建奇功,此中原无疑义也。特其语 
气激昂,务令笔锋摇曳生姿,于病情之更改,用药之精义皆未发明,是以阅者未能了然也。 

<目录>四、医话
<篇名>27.答刘××问七伤
属性:(1)大饱伤脾∶因脾主运化饮食,饮食太饱,脾之运化力不足以胜之,是以受伤。其作噫者,因脾 
不运化,气郁中焦,其气郁极欲通,故噫以通之;其欲卧者,因脾主四肢,脾伤四肢酸懒,是以欲卧;其色黄 
者,因脾属土。凡人之五脏,何脏有病,即现何脏所属之本色。此四诊之中,所以望居首也。 
(2)大怒气逆伤肝∶因肝属木,木之条上达,木之根下达。为肝气能上达,故能助心气之宣通(肝系下 
连气海,上连心,故能接引气海中元气上达于心)。为肝气能下达,故能助肾气之疏泄(肾主闭藏,有肝气以疏 
泄之,二便始能通顺)。大怒,其气有升无降,甚而至于横行,其中所藏之相火,亦遂因之暴动 
(相火生于命门,寄于肝胆,游行于三焦),耗其血液,所以伤肝而血即少。肝开窍于 
目,目得血而能视,肝伤血少,所以其目暗也。 
(3)形寒饮冷伤肺∶因肺为娇脏,冷热皆足以伤之也。盖肺主皮毛,形寒则皮毛闭塞,肺气不能 
宣通,遂郁而生热,此肺之因热而伤也。饮冷则胃有寒饮留滞,变为饮邪,上逆于肺而为 
悬饮,此肺之因冷而伤也。肺主气,开窍于鼻,有病则咳,肺伤,所以气少、咳嗽、鼻鸣也。 
(4)忧愁思虑伤心∶因人之神明藏于脑,故脑为精明之府(《内经》脉要精微论),而发出在心, 
故心为君主之官(《内经》灵兰秘典),神明属阳,阳者主热。忧愁思虑者,神明常常由心发露,心血必因 
热而耗,是以伤心也。心伤,上之不能充量输血于脑,下之不能充量输血于肝,脑中之神失其凭借,故苦惊 
喜忘,肝中之魂,失其护卫,故夜不能寐,且肝中血少,必生燥热,故又多怒也。 
(5)强力入房久坐湿地伤肾∶因肾有两枚,皆属于水,中藏相火,为真阴中之真阳,以统摄下 
焦真阴真阳之气。强力入房则伤阴,久坐湿地则伤阳,肾之真阴真阳俱伤,所以伤肾。肾伤则呼吸之时,不能纳 
气归根,所以短气。腰者肾之腑,肾伤所以腰疼。骨者肾所主,肾伤所以脚骨作疼。至于厥逆下冷,亦肾中水 
火之气,不能敷布之故也。 
(6)风雨寒暑伤形∶因风雨寒暑,原天地之气化,虽非若疠疫不正之气,而当其来时或过于猛烈,即 
与人身之气化有不宜。乃有时为时势所迫,或自不经意,被风雨寒暑之气侵,其身体气弱,不能捍御,则 
伤形矣。形伤则发落,肌肤枯槁,此犹木伤其本,而害及枝叶也。 
(7)大恐惧不节伤志∶因志者为心之所之,必以中正之官辅之,此志始百折不回。中正之官者,胆也, 
若过恐惧,则胆失其司,即不能辅心以成志,所以伤志。志伤,则心有所图而畏首畏尾,所以恍惚不乐也。 

<目录>四、医话
<篇名>28.答刘××问湿温治法之理由
属性:湿温病状,纯系湿热郁中,致经络闭塞,故其外虽觉寒凉,而内则小便短涩赤黄也。为小便难,水气 
必多归大肠,所以兼泄泻也。其肢体酸痛者,湿而兼风也。胸膈痞满者,湿气挟饮也。 
欲治此症,甚属易易,用滑石两许煎汤,送服阿斯匹林一片半, 
汗出即愈。盖二药一发汗,一利水,可令内蕴之湿,由汗与小便而解。且二药之性皆凉,其热亦可随之而解。 
阿斯匹林又善愈关节疼痛也。余用此方,连治数人,皆一汗而愈。若热剧者,滑石 
或多用,或加生石膏数钱与滑石同煎,亦莫不随手奏效也。 

<目录>四、医话
<篇名>29.答刘××问肝与脾之关系及肝病善作疼之理
属性:(附∶肝脾双理丸) 
肝脾者,相助为理之脏也。人多谓肝木过盛可以克伤脾土,即不能消食。不知肝木过弱不能疏通脾土, 
亦不能消食。盖肝之系下连气海,兼有相火寄生其中。为其连气海也,可代元气布化,脾胃之健运实资其 
辅助。为其寄生相火也,可借火以生土,脾胃之饮食更赖之熟腐。故曰肝与脾相助为理之脏也。特是肝为 
厥阴,中见少阳,其性刚果,其气条达,故《内经》灵兰秘典名为将军之官。有时调摄失宜,拂其条达之性, 
恒至激发其刚果之性而近于横恣,于斯脾胃先当其冲,向之得其助者,至斯反受其 
损。而其横恣所及,能排挤诸脏腑之气致失其和,故善作疼也。 
于斯,欲制肝气之横恣,而平肝之议出焉。至平之犹不足制其横恣,而伐肝之议又出焉。所用之药,若 
三棱、莪术、青皮、延胡、鳖甲诸品,放胆杂投,毫无顾忌,独不思肝木于时应春,为气化发生之始,若植物 
之有萌芽,而竟若斯平之伐之,其萌芽有不挫折毁伤者乎?岂除此平肝伐肝之外,别无术以医肝乎?何 
以本属可治之证,而竟以用药失宜者归于不治乎?愚近拟得肝脾双理丸,凡肝脾不和,饮食不消,满闷胀疼, 
或呃逆嗳气呕吐,或泄泻,或痢疾,或女子月事不调,行经腹疼,关于肝脾种种诸 
证,服之莫不奏效。爰录其方于下。 
【肝脾双理丸】 
甘草(十两细末) 生杭芍(二两细末) 广条桂(两半去粗皮细末) 川紫朴(两半细末) 
薄荷冰(三钱细末) 冰片(二钱细末) 朱砂(三两细末) 
上药七味。将朱砂一两与前六味和勺,水泛为丸,桐子大,晾干(忌晒),用所余二两朱砂为衣,勿 
令余剩,上衣时以糯米浓汁代水,且令坚实光滑方不走气。其用量∶常时调养,每服二十粒至三十粒;急用 
除病时,可服至百粒,或一百二十粒。 

<目录>四、医话
<篇名>30.答徐××问腹疼治法
属性:少年素有 癖,忽然少腹胀疼。屡次服药,多系开气行气之品,或不效,或效而复发。脉象无力。以 
愚意见度之,不宜再用开气行气之药。近在奉天有治腹疼二案,详录于下,以备参考。 
一为门生张××,少腹素有寒积,因饮食失慎,肠结,大便不下,少腹胀疼,两日饮食不进。用蓖麻 
油下之,便行三次而疼胀如故。又投以温暖下焦之剂,服后亦不觉热,而疼胀如故。细诊其脉,沉而无力。 
询之,微觉短气。疑系胸中大气下陷,先用柴胡二钱煎汤试服,疼胀少瘥。遂用生箭 一两,当归、党参各三 
钱,升麻、柴胡、桔梗各钱半,煎服一剂,疼胀全消,气息亦顺,惟觉口中发干。又即原方去升麻、党参,加 
知母三钱,连服数剂全愈。 
一为奉天史姓学生,少腹疼痛颇剧,脉左右皆沉而无力。疑为气血凝滞,治以当归、丹参、乳香、没 
药各三钱,莱菔子二钱,煎服后疼益甚,且觉短气。再诊其脉,愈形沉弱。遂改用升陷汤一剂而愈。此亦大 
气下陷,迫挤少腹作疼,是以破其气则疼益甚,升举其气则疼自愈也。 
若疑因有 癖作疼,愚曾经验一善化 癖之法。忆在籍时,有人问下焦虚寒治法,俾日服鹿角胶三钱, 
取其温而且补也。后月余晤面,言服药甚效,而兼获意外之效∶少腹素有积聚甚硬, 
前竟忘言,因连服鹿角胶已尽消。盖鹿角胶具温补之性,而又善通血脉,林屋山人阳和汤用之以消硬疽, 
是以有效也。又尝阅喻氏《寓意草》,载有袁聚东痞块危证治验,亦宜参观。 

<目录>四、医话
<篇名>31.答宗弟××问右臂疼治法
属性:据来案云云,臂疼当系因热。而愚再三思之,其原因断乎非热。或经络间因瘀生热,故乍服辛凉之品似 
觉轻也。盖此证纯为经络之病,治之者宜以经络为重,而兼顾其脏腑,盖欲药力由脏腑而达经络也。西人治 
急性关节疼痛,恒用阿斯匹林。然用其药宜用中药健运脾胃通行经络之品辅之。又细阅素服之方皆佳,所 
以不见效者,大抵因少开痹通窃之药耳。今拟一方于下∶ 
于白术(此药药局中多用麸炒殊非所宜,当购生者自炒熟,其大小片分两次炒之轧细)取净末一两,乳香、 
没药(二药须购生者轧成粗渣,隔纸在锅内烘融化,取出晾干轧细)各取净末四钱,朱血竭(此药未研时外皮作 
黑色,若研之色若朱砂者方真)研细三钱,当归身(纸裹置炉旁候干轧细)净末七钱,细辛、香白 
芷细末各钱半,冰片(用樟脑升成者,不必用梅片)、薄荷冰细末各三分,诸药和匀,贮瓶密封。每服一钱 
半,络石藤(俗名爬山虎,能蔓延砖壁之上,其须自粘于壁上不落者方真)煎汤送服,日两次。方中之义∶以 
白术健脾开痹为主(《神农本草经》谓白术逐风寒湿痹),佐以白芷去风,细辛去寒,当归、乳香、没药、血 
竭以通气活血,冰片、薄荷冰以透窍即以通络。且脾主四肢,因其气化先行于右(右关候脾脉是明征),故 
右臂尤为脾之所主。丁氏《化学本草》谓没药善养脾胃,其温通之性不但能治气血痹疼,更可佐白术以健补 
脾胃,故于此证尤宜也。至阿斯匹林,初次宜服半瓦,以得微汗为度,以后每日服两次,撙节服之,不 
必令其出汗,宜与自制末药相间服之,或先或后皆可(后接来函按法治愈)。 

<目录>四、医话
<篇名>32.答周××为母问疼风证治法
属性:详观病案,曾患两膝肿疼,愈而复发,膝踝趾骨皆 热肿 
痛,连臀部亦肿,又兼目痛。此诚因心肝皆有郁热,而关节经络之间又有风湿热相并,阴塞血脉之流通,故作 
肿疼也。后见有胡××、张××皆有答复,所论病因及治法又皆尽善尽美,似无庸再为拟议。然愚从前治此等 
证,亦纯用中药,后阅东人医报见治急性偻麻质斯(即热性历节风),喜用西药阿斯匹林,载有历治诸案可考 
验,后乃屡试其药,更以中药驾驭之,尤效验异常。在奉曾治一幼童得此证,已危至极点,奄奄一息,数日 
未断,舁至院中亦治愈(详案在石膏解下)。由斯知西药之性近和平,试之果有效验,且洞悉其 
原质者,固不妨与中药并用也。爰拟方于下,以备采择∶ 
阿斯匹林一瓦半,生怀山药一两,鲜茅根去净皮切碎二两,将山药茅根煎汤三茶杯,一日之间分三次 
温服,每次送服阿斯匹林半瓦。若服一次周身得汗后,二次阿斯匹林可少用。至翌日三次皆宜少用。以一日 
间三次所服之阿斯匹林有一次微似有汗即可,不可每次皆有汗也。如此服之,大约两旬即可愈矣。 

<目录>四、医话
<篇名>33.答××女士问疼经治法
属性:详观病案,知系血海虚寒,其中气化不宣通也。夫血海者,冲脉也,居脐之两旁,微向下,男女皆有。 
在女子则上承诸经之血,下应一月之信,有任脉以为之担任,带脉以为之约束。阳维、阴维、阳跷、阴跷, 
为之拥护,督脉为之督摄,《内经》所谓女子二七,太冲脉盛,月事以时下者此也。有时其中气化虚损 
或兼寒凉,其宣通主力微,遂至凝滞而作疼也。而诸脉之担任拥护、督摄者,亦遂连带而作疼也。斯当 
温补其气化而宣通之,其疼自止。爰拟方于下∶ 
全当归(一两) 生乳香(一两) 生没药(一两) 小茴香(一两炒熟) 
鱼鳔胶(一两猪脂炸脆) 川芎(五钱) 甘松(五钱,此药原香郁,若陈腐者不用亦可) 
共为细末。每服二钱五分,用真鹿角胶钱半,煎汤送下,日服两次。 

<目录>四、医话
<篇名>34.答王××问时方生化汤
属性:当归之味甘胜于辛,性温虽能助热,而濡润多液,又实能滋阴退热,原不可但以助热论。故《神农本 
草经》谓可治温疟,且谓煮汁饮之尤良,诚以煮汁则其液浓浓,濡润之功益胜也。其性虽流通活血,而用之得 
当亦能止血。友人王××曾小便溺血,用黄酒煮当归一两饮之而愈。后其症反复,再服原方不效,问治于 
仆,俾用鸦胆子去皮五十粒,白糖水送服而愈。继其症又反复,用鸦胆子又不效,仍用酒煎当归法治愈。 
又傅青主治老妇血崩,用黄 、当归各一两,桑叶十四片,煎汤送服三七细末三钱,甚效。又单用醋 
炒当归一两煎服,治血崩亦恒有效。是当归可用以活血,亦可用以止血,故其药原名“文无”,为其能使气 
血各有所归,而又名当归也。产后血脉淆乱,且兼有瘀血,故可谓产后良药。至川芎其香窜之性,虽甚于 
当归,然善升清阳之气。凡清阳下陷作寒热者,用川芎治之甚效,而产后又恒有此证。 
同邑赵姓之妇,因临盆用力过甚,产后得寒热症,其家人为购生化汤二剂服之病顿愈。盖其临盆努力 
之时,致上焦清阳下陷,故产后遂发寒热,至服生化汤而愈者,全赖川芎升举清阳之力也。旬余寒热又作,其 
叔父××知医,往省视之,谓系产后瘀血为恙又兼受寒,于活血化瘀药中,重加干姜。数剂后,寒热益甚, 
连连饮水,不能解渴。当时仲夏,身热如炙,又复严裹浓被,略以展动即觉冷气侵肤。后仆诊视,左脉沉 
细欲无,右脉沉紧皆有数象,知其上焦清阳之气下陷,又为热药所伤也。从前服生化汤, 
借川芎升举之力而暂愈,然川芎能升举清阳,实不能补助清阳之气 
使之充盛,是以愈而又反复也。为疏方黄 、玄参各六钱,知母八钱(时已弥月,故可重用凉药),柴胡、 
桔梗各钱半,升麻一钱,一剂而寒热已,又少为加减,服数剂全愈。由是观之,川芎亦产后之要药也。吴鞠 
通、王士雄之言皆不可奉为定论。惟发热汗多者,不宜用耳。至包氏所定生化汤,大致亦顺适。惟限于四点 
钟内服完三剂,未免服药过多。每次冲入绍酒一两,其性过热,又能醉人,必多有不能任受者。仆于妇人产 
后用生化汤原方,加生怀山药数钱,其大便难者,加阿胶数钱,俾日服一剂,连服三日停止,亦必不至 
有产后病也。 

<目录>四、医话
<篇名>35.答吴××问病
属性:所问妇人血淋之症,因日久损其脾胃,饮食不化,大便滑泄,且血淋又兼砂淋,洵为难治之症。今拟一方∶ 
生山药一斤轧细末,每用八钱,加生车前子二钱同煮作粥, 
送服三七细末、生鸡内金细末各五分,每日两次,当点心用之,日久可愈。 
方中之意,用山药、车前煮粥以治泄泻。而车前又善治淋疼,又送服三七以治血淋,内金以消砂淋,且 
鸡内金又善消食,与山药并用,又为健补脾胃之妙品也。惟内金生用则力大,而稍有破气之副作用,若气分过 
虚时,宜先用生者轧细,焙熟用之。若服药数日而血淋不见轻者,可用荜澄茄细末一分,加西药哥拜拔油一 
分同服。又此症大便不止,血淋亦无从愈,若服山药、车前粥而 
泻不止,可将熟鸡子黄二三枚捻碎,调在粥中,再煮一两开服之。 

<目录>四、医话
<篇名>36.答徐××问其妻荡漾病治法
属性:详观所述病案,谓脉象滑动,且得之服六味地黄丸之余,其为热痰郁于中焦,以致胃气上逆,冲气 
上冲,浸成上盛下虚之症无疑。为其上盛下虚,所以时时有荡漾之病也。法当利痰、清火、 
降胃、敛冲,处一小剂,久久服之,气化归根,荡漾自愈。拟方于下∶ 
清半夏(三钱) 柏子仁(三钱) 生赭石(三钱轧末) 生杭芍(三钱) 生芡实(一两) 生姜(三片) 
磨生铁锈浓水煎药。 
方中之意,用半夏、赭石以利痰、坠痰,即以降胃,安冲。用芡实以固下焦气化,使药之降者、坠者,有 
所底止,且以收敛冲气,而不使再上冲也。用芍药以清肝火、利小便,即以开痰之去路。用柏子仁以养肝血、 
滋肾水,即以调半夏之辛燥。用生姜以透窍络,通神明,即以为治痰药之佐使。至用铁锈水煎药者, 
诚以诸风眩晕,皆属于肝,荡漾即眩晕也。此中必有肝风萌动,以助胃气冲气之上升不已,律以金能制木之 
理,可借铁锈之金气以镇肝木,更推以铁能重坠,引肝中所寄龙雷之火下降也。况铁锈为铁与养气化合而成, 
最善补养人之血分,强健人之精神,即久久服之,于脏腑亦无不宜也。 

<目录>四、医话
<篇名>37.答郭××问小儿耳聋口哑治法
属性:小儿之耳聋口哑,乃连带相关之证也。盖小儿必习闻大人之言,而后能言;故小儿当未能言时或甫能言 
时,骤然耳聋不闻,必至哑不能言。是以治此证者,当专治其耳聋。然耳聋之证有可治者,有不可治者。其 
不可治者,耳膜破也。其可治者,耳中核络有窒塞也。用灵磁石一块口中含之,将细铁条插耳内,磁铁之 
气相感,如此十二日,耳之窒塞当通。若仍不通,宜口含铁块, 
耳际塞磁石,如此十二日耳中之窒塞当通矣。 

<目录>四、医话
<篇名>38.答王××质疑
属性:犀黄,诚如兄言为西黄之误。盖牛黄之好者,出于高丽,因 
高丽之牛大,故所出之黄亦最美(从前高丽清心丸甚佳,以其有牛黄也),特别之曰,东牛黄,而其价亦 
较昂;青海西藏之地,亦多出牛黄,其成色亚于东牛黄,故又别之曰,西牛黄,而其地原有犀,遂又误西 
为犀也。紫石英,弟恒用之,治女子不育甚效。其未经 者,其色紫而透彻,大小皆作五棱者佳。盖白石英 
属阴,紫石英属阳,阴者宜六棱,阳者宜五梭。至钟乳石,蛇含石,皆未用过,不敢置论。 

<目录>四、医话
<篇名>39.答黄××问接骨方并论及接筋方
属性:接骨之方甚多,然求其效而速者,独有一方可以公诸医界。 
方用甜瓜子、生菜子各一两,小榆树的鲜嫩白皮一两,再加真芝麻油一两,同捣如泥,敷患处,以布缠 
之。不过半点钟,觉骨接上即去药,不然恐骨之接处起节。自得此方后,门人李××曾用以治马甚效,想 
用以治人亦无不效也。且试验可在数刻之间,设有不效,再用他方亦未晚也。 
人之筋骨相着,然骨以刚而易折,筋以韧而难断,是以方书中治接骨之方甚伙,而接筋之方甚鲜也。 
诸家本草多言旋复花能续断筋,《群芳谱》谓 根能续断筋。 根愚未试过,至旋复花邑中有以之治牛马断筋 
者,甚效。其方初则秘而不传,当耕地之时,牛马多有因惊骇奔逸被犁头铲断腿上筋者,敷以所制之药, 
过两旬必愈。后愚为其家治病,始详言其方。且言此方受之异人,本以治人,而以治物类亦无不效。因 
将其方详录于下。 
方用旋复花细末五六钱,加白蔗糖两许,和水半茶杯同熬成膏。候冷加麝香少许(无麝香亦可),摊 
布上,缠伤处。至旬日,将药揭下,筋之两端皆长一小疙瘩。再换药一帖,其两小疙瘩即连为 
一,而断者续矣。若其筋断在关节之处,又必须设法闭住,勿令其关节屈伸,筋方能续。 
按∶《外台秘要》有急续断筋方,取旋复花根洗净捣敷创上。 
日一,二易,瘥止,是取其鲜根捣烂用之也。因药局无旋复花根,是以后世用者权用其花,想性亦相近,故能奏效。 

<目录>四、医话
<篇名>40.答任××问病三则
属性:(1)问治蛇咬法 
《验方新编》治蛇咬法,用吸烟筒中油子,凉水冲出冷冻饮料之。按此方甚验,设犹不效,可用其相畏之 
物治之,蛇之所畏者,蜈蚣、雄黄也。拟方用全蜈蚣三条,雄黄二钱,共为末,分三包。每用一包,甘草、 
蚤休各二钱,煎汤送下,日服二次,旬日当愈。 
(2)问治顽癣法及足底痒治法 
大枫子去皮,将仁捣如泥,加白砒细末少许(少少的),和猪脂调膏敷之,此剧方也。又用鲜曼陀罗熬 
膏(梗叶花实皆可用),加鸦胆子细末(去皮研细),调和作膏药贴之,此为和平方。足底痒可用蛇蜕三 
条,甘草二钱,煎水饮之。再将渣重煎熏洗,半月可愈。 
(3)问喉症治法 
初秋时,用大西瓜一个(重约七八斤)开一口,装入硼砂、火硝细末各一斤,仍将开下之皮堵上,将西 
瓜装于新出窑之瓦罐中(瓦罐须未经水湿者),将罐口严封,悬于不动烟火不通空气之静室中。过 
旬日,视罐外透出白霜,扫下。每霜一两,调入薄荷冰二分,瓶贮,勿令泄气,遇红肿喉症,点之即消。 

<目录>四、医话
<篇名>41.答黄××问创伤及跌打损伤外敷内服止疼化瘀方
属性:外敷用生赤石脂细末、旱三七细末等分,和匀敷之,立能止血、止疼。内服用旱三七细末二钱、西药 
臭剥细末二分,同服下,立能化瘀止疼。 
孙××按∶凡创伤跌打损伤,用白附止痛粉甚佳。今将该方列下∶ 
白附子(六两) 白芷(五钱) 羌活(五钱) 防风(五钱) 南星(五钱均生用共轧末) 
青肿者童便调涂,破则干撒之,虽肾子破出,可能立止痛、生肌、止血、去瘀,且不忌风,真良方也。 

<目录>四、医话
<篇名>42.答陈××疑《内经》十二经有名无质
属性:天下之妙理寓于迹象之中,实超于迹象之外,彼拘于迹象以索解者,纵于理能窥其妙,实未能穷其 
极妙也。如九十六号(绍兴星期报)陈××,因研究剖解之学人于十二经之起止莫能寻其迹象, 
遂言《内经》所言十二经无可考据。非无据也,因其理甚玄妙,超于迹象之外,非常识所能索解也。夫《内 
经》之灵枢,原名《针经》,故欲究十二经之奥妙,非精针灸者不能得其实际。愚于针灸非敢言精,而尝与友 
人卢××(辽阳人最精针灸得之祖传)谈及此事,卢××谓斯可即余针愈疔毒之案以征明之。庚申八月间,族 
妹左手少阳经关冲穴生疔,至二日疼甚,为刺耳门二穴立愈。关冲为手少阳经之所起,耳门为手少阳经之所 
止也。又辛酉七月中,族中男孙七岁,在右足太阴经隐白穴生疔,三日肿至膝下,疼甚剧,取右三阴交及公孙 
二穴刺之,立愈。隐白穴为足太阴经之所起,公孙三阴交为足太阴经之所历也。设若刺其处仍不愈者,刺太阴 
经止处之大包穴,亦无不愈矣。又于辛酉八月间,本村田姓妇在手阳明二间穴生疔,肿过手腕,为刺曲池、 
迎香二穴,当时疼立止,不日即消。二间虽非阳明经起之处,距经起处之商阳穴不过二寸,曲池则经历之处, 
迎香则经止之处也。又于九月中,学生吴××在手太阴经太渊穴生疔,红肿之线已至侠气户,木不知疼, 
恶心呕吐,诊其脉象洪紧,右寸尤甚,知系太阴之毒火所发,为 
刺本经尺泽、中府及肺俞,患处觉疼,恶心呕吐立止,红线亦立 
回,半日全愈。太渊距本经起处之少商穴不过三寸强,中府则本经之所起也,尺泽则本经之所历也,肺俞 
则本经之所注也。由是观之,疔生于经之起处,刺经之止处;生于经之止处,刺经之起 
处,皆可随手奏效。则经之起处与止处非有一气贯通之妙,何以神效如是哉? 

<目录>四、医话
<篇名>43.治疯犬伤方
属性:疯犬伤证甚为危险,古方用斑蝥虽能治愈,然百日之内忌见水,忌闻锣声,忌食诸豆,忌行 麻之地 
及手摩 麻,又须切忌房事百日。犯以上所忌,其证仍反复,如此,保养甚不易也。歙县友人胡××,深 
悯患此证者不易挽救,曾登《绍兴医报》征求良方,继有江东束××登报相告,谓曾用《金匮》下瘀 
血汤治愈二人。又继有江西黄××登报相告,谓系异人传授一方,用大蜈蚣一条、大黄一两、甘草一两,煎 
汤服,甚验。如服后病者稍安静,未几又发,再根据此方续服,病必愈,乃可止。后附有治验之案二则, 
皆疯已发动服此药治愈者。 
按∶此方诚为至善良方。胡××谓∶俗传冬令蛇藏土洞,口衔或泥或草,迨至春日出蛰,口吐所衔 
之物,犬嗅之即成疯犬,此理可信。盖犬性善嗅,有殊异之气味,辄喜嗅之,是以独中其毒。而疯后咬人,是 
蛇之毒递传于人也,方中用蜈蚣一条,则蛇毒可解矣。又此证,束氏谓曾用《金匮》下瘀血汤治愈两人,由 
斯知此证必有瘀血,下之则可愈。方中用大黄一两,其瘀血当可尽下,又加甘草一两,既善解毒,又能缓大 
黄之峻攻,此所以为良方也。然此方善矣,而未知愈后亦多禁忌否?若仍然有禁忌,是善犹未尽善也。而愚在 
奉天时,得其地相传之方,凡用其方者,服后即脱然无累,百无禁忌,真良方也。其方用片灰(即枪药之轧成片者, 
系硫黄火硝木炭制成)三钱、鲜枸杞根三两,煎汤送下。必自小便下恶浊之物若干而 
愈。愈后惟禁房事旬日。然药不可早服,必被伤后或五六日,或七八日,觉内风萌动,骚扰不安,然后服 
之方效。此乃屡试屡效之方,万无闪失也。枸杞根即药中之地骨皮,然地骨但用根上之皮,兹则连皮中之木用之。 
又∶吴县友人陆××,于丁卯中秋相遇于津门,论及此证。陆××言,凡疯狗脊骨中皆有毒虫,若将其脊 
骨中脂膜刮下,炮作炭服之,可自二便中下恶浊之物,即愈。有族孙患此证,治以此 
方,果愈。然所虑者,啮人之疯犬,未必能获之也。 
又∶无锡友人周小农,曾登《山西医学杂志》,论治疯犬咬伤之方。谓岁己丑,象邑多疯犬,遭其害 
者治多无效。适有耕牛亦遭此患而毙。剖其腹,有血块大如斗,黧紫,搅之蠕蠕然动,一方惊传异事。有 
张君者,晓医理,闻之悟曰∶“仲景云‘瘀热在里,其人发狂。’又云‘其人如狂者,血证谛也,下血狂 
乃愈。’今犯此证者,大抵如狂如癫,得非瘀血为之乎?不然,牛腹中何以有此怪物耶?吾今得其要矣。”于 
斯用仲景下瘀血汤治之。不论证之轻重,毒之发与未发,莫不应手而愈。转以告人,百不失一。其所用之方, 
将古时分量折为今时分量,而略有变通。方用大黄三钱,桃仁七粒,地鳖虫去足炒七个,共为细末,加蜂蜜 
三钱,用酒一茶碗煎至七分,连渣服之。如不能饮酒者,水、酒各半煎服亦可。服后二盒饭下恶浊之物。日 
进一剂,迨二便如常,又宜再服两剂,总要大、小便无纤毫恶浊为度。服此药者,但忌房事数日,其余则一概 
不忌。若治小儿,药剂减半。妊妇亦可放胆服之,切莫忌较。按∶服此方果如上所云云,诚为佳方。 
【××附记】∶同邑友人张××据周××云∶其戚某,得一治疯犬咬伤秘法。其方系用白雄鸡一只,取 
其嘴,及腿之下截,连爪,及其胆,肫皮,翅尖翎,尾上翎。加银朱三钱,鳔须三寸,用绵纸三、四张裹之, 
缟麻扎紧,用香油四两浸透,以火燃之,余油亦浇其上,烧为炭,研末,黄 
酒送服,通身得汗即愈。愈后除忌房事旬日外,余无所忌,屡试屡验。 

<目录>四、医话
<篇名>44.解砒石毒兼解火柴毒方
属性:初受其毒者,在胃上脘,用生石膏一两,生白矾五钱共轧细,先用鸡子清七枚调服一半即当吐出。若犹未 
吐或吐亦不多,再用生鸡子清七枚调服余一半,必然涌吐。吐后若有余热,单用生石膏细末四两,煮汤两 
大碗,将碗置冰水中或新汲井泉水中,俾速冷分数次饮下,以热消为度。若其毒已至中脘,不必用吐药, 
可单用生石膏细末二三两,如前用鸡子清调服,酌热之轻重或两次服完,或三次四次服完,毒解不必尽剂。且 
热消十之七八即不宜再服石膏末,宜仍如前煮生石膏汤饮之,以消其余热。若其毒已至下脘,宜急导之下行 
自大便出,用生石膏细末二两,芒硝一两,如前用鸡子清调服,毒甚者一次服完,服后若有余热,可如 
前饮生石膏汤。此方前后虽不同,而总以石膏为主,此乃以石治石,以石之凉者治石之热者。愚用此方救人多 
矣,虽在垂危之候,放胆用之,亦可挽救。 

附录∶崔××来函介绍三方
<篇名>(1)外伤甚重救急方
属性:【神授普济五行妙化丹】治外伤甚重,其人呼息已停,或因惊吓而猝然罔觉,甚至气息已断,急用 
此丹一厘,点大眼角,男左女右;再用三分,以开水吞服。其不知服者,开水冲药灌之,须臾即可苏醒。并 
治一切暴病、霍乱、痧证、小儿痉痫、火眼、牙疳、红白痢疾等证,皆效,爰录其方于下。 
火硝(八两) 皂矾(二两) 明雄黄(一两) 辰砂(三钱) 真梅片(二钱) 
共为极细末,瓶贮勿令泄气。 
戊辰冬,本镇有吴姓幼童,年六岁,由牛马厂经过,一牛以角 入幼童口中,破至耳边,血流不止,幼 
童已死。此童无祖无父,其祖母及其母闻之,皆吓死,急迎为挽救。即取食盐炒热熨丹田,用妙化丹点大眼 
角,幼童即活。再用妙化丹点其祖母及其母大眼角,须臾亦活。再用灰锰氧将幼童伤处内外洗净,外以胶 
布贴之,加绑扎。内食牛乳,三日后视之,已生肌矣。又每日用灰锰氧冲水洗之,两旬全愈,愈后并无疤痕。 
又∶一九一七年四月中旬,潜邑张港一妇人,二十余岁,因割麦争界,言语不周,被人举足一踢, 
仆地而死。经数医生,有用吹鼻者,有用鹅换气者,有用乌梅擦牙者,百方千方,种种无效,求为往视。其身 
冷如冰,牙关紧闭,一日有余矣,而其胸犹微温。急用妙化丹点其大眼角;用食盐二斤炒热,作两包,熨其 
丹田,轮流更换,得暖气以助生气。二炷香之久,牙关已开,遂用红糖冲开水服之,即活。 
又∶乙丑季夏上旬,曾治刘××,年过四旬,因分家起争,被其弟用刀伤脐下,其肠流出盈盆,忽然 
上气喘急,大汗如雨。经数医延医,皆无把握,因迎生速往诊视。观其形状危险,有将脱之势,遂急用生黄 
、净萸肉、生山药各一两,固其气以防其脱。煎汤服后,喘定汗止。查看其肠已破,流有粪出,遂先用灰 
锰氧冲水,将粪血洗净。所破之肠,又急用桑根白皮作线为之缝好,再略上磺碘,将其肠慢慢纳进。再用洋 
白线将肚皮缝好。又用纱布浸灰锰氧水中,候温,复其上,用白士林少调磺碘作药棉,复其上,用绷带扎住, 
一日一换。内服用《衷中参西录》内托生肌散,变为汤剂,一日煎渣再服。三星期全愈。 
按∶此证未尝用妙化丹,因其伤重而且险,竟能救愈,洵堪 
为治此重伤者之表准,故连类及之。且所用内托生肌散,为愚治 
疮毒破后生肌之方,凡疮破后溃烂、不速生肌者,用之最效。若 
欲将散剂变为汤剂,宜先将天花粉改为四两,一剂分作八剂,一日之间煎渣再服,其生肌之力较服散药尤效。 

附录∶崔××来函介绍三方
<篇名>(2)服食松脂法
属性:《抱朴子》内篇载有上党赵姓身患癞病,历年不愈。后遇异人指示,服松脂百日,癞病全愈。初不 
知松脂为何物,后参阅群书,知松脂即是松香。解毒、除湿、消肿、止痛、生肌、化痰,久服轻身延年,辟谷 
不饥。《万国药方》久咳丸,系松脂、甘草并用。向曾患咳嗽,百药不效,后每服松脂干末一钱,用凉茶 
送服,月余咳嗽全愈,至今十年,未尝反复,精神比前更强壮。观此,松脂实有补髓健骨之力。 
又,丁卯夏,川鄂战争,救一兵士,子弹由背透胸出,由伤处检出碎骨若干,每日令食牛乳、山药,数 
日饮食稍进,口吐臭脓,不能坐立。后每日令服松脂两次,每次一钱,三日后臭脓已尽,伤口内另长新骨。 
月余伤口全平,行步如常。 
又一兵士李××,过食生冷,身体浮肿,腹大如箕,百药罔效。令每日服松脂三钱,分三次服下,五日全愈。 
乡村一男子,患肝痈,溃破,医治五年不愈,溃穿二孔,日出臭水碗许,口吐脓血,臭气异常。戊辰孟 
夏,迎为延医,视其形状,危险万分,辞而不治。再三悬求,遂每早晚令服松脂一 
钱,五日臭脓减少,疮口合平,照前服之,半月全愈。 
又有患肺痈者,服林屋山人犀黄丸不效,而服松脂辄效者,难以枚举矣。 

附录∶崔××来函介绍三方
<篇名>(3)止咳方
属性:家母年五十时患咳嗽,百药不效,严冬时,卧不安枕。遇一 
老医,传授一方,系米壳四两,北五味三钱,杏仁去皮炒熟五钱,枯矾二钱,共为细末,炼蜜为丸,梧桐 
子大,每服二十丸,白糖开水送下。吞服数日,病若失,永不复发。家母生于甲辰, 
现年八十有六,貌若童颜。以后用此丸疗治咳嗽全愈者,笔难悉述。 
以上二、三方,皆为寻常药品,而能愈此难愈之大证,且又屡试屡效,诚佳方也。 

\part{医案}
\chapter{虚劳喘嗽门}
<篇名>1.虚劳证阳亢阴亏
属性:天津张媪,年九十二岁,得上焦烦热病。 
病因 平素身体康强,所禀元阳独旺,是以能享高年。至八旬后阴分浸衰,阳分偏盛,胸间恒觉烦 
热,延医服药多用滋阴之品始愈。迨至年过九旬,阴愈衰而阳愈亢,仲春阳气发生烦热,旧病反复甚剧。 
证候 胸中烦热异常,剧时若屋中莫能容,恒至堂中,当户久坐以翕收庭中空气。有时,觉心为 
热迫怔忡不宁。大便干燥四五日一行,甚或服药始通。其脉左右皆弦硬,间现结脉,至数如常。 
诊断 证脉细参,纯系阳分偏盛阴分不足之象。然所以享此大年,实赖元阳充足。此时阳虽偏盛,当 
大滋真阴以潜其阳,实不可以苦寒泻之。至脉有结象,高年者虽在所不忌,而究系气分有不足之处,宜 
以大滋真阴之药为主,而少加补气之品以调其脉。 
处方 生怀山药(一两) 玄参(一两) 熟怀地黄(一两) 生怀地黄(八钱) 天冬(八钱) 甘草 
(二钱) 大甘枸杞(八钱) 生杭芍(五钱) 野台参(三钱) 赭石(六钱轧细) 生鸡内金(二钱黄色的捣) 
共煎三大盅,为一日之量,徐徐分多次温饮下。 
方解 方中之义,重用凉润之品以滋真阴,少用野台参三钱以调其脉。犹恐参性温升不宜于上焦之烦 
热,又倍用生赭石以引之下行,且此证原艰于大便,赭石又能降胃气以通大便也。 
用鸡内金者,欲其 
助胃气以运化药力也;用甘草者,以其能缓脉象之弦硬,且以调和诸凉药之性也。 
效果 每日服药一剂至三剂,烦热大减,脉已不结,且较前柔和。遂将方中玄参、生地黄皆改 
用六钱,又加龙眼肉五钱,连服五剂,诸病皆愈。 


<篇名>2.虚劳兼劳碌过度
属性:天津宁氏妇,年近四旬,素病虚劳,偶因劳碌过甚益增剧。 
病因 处境不顺,家务劳心,饮食减少,浸成虚劳,已病倒卧懒起床矣。又因讼事,强令 
公堂对质,劳苦半日,归家病大加剧。 
证候 卧床闭目,昏昏似睡,呼之眼微开不发言语,有若能言而甚懒于言者。其面色似有浮热 
,体温38·8℃,问其心中发热乎?觉怔忡乎?皆颔之。其左脉浮而弦硬,右脉浮而芤,皆不任重 
按,一息六至。两日之间,惟少饮米汤,大便数日未行,小便亦甚短少。 
诊断 即其脉之左弦右芤,且又浮数无根,知系气血亏极有阴阳不相维系之象。是以阳气上浮 
而面热,阳气外越而身热,此乃虚劳中极危险之证也。所幸气息似稍促而不至于喘,虽有 
咳嗽亦不甚剧,知尤可治。斯当培养其气血,更以收敛气血之药佐之,俾其阴阳互相维系, 
即可安然无虞矣。 
处方 野台参(四钱) 生怀山药(八钱) 净萸肉(八钱) 生龙骨(八钱捣碎) 大甘枸杞( 
六钱) 甘草(二钱) 生怀地黄(六钱) 玄参(五钱) 沙参(五钱) 生赭石(五钱轧细) 生杭芍(四钱) 
共煎汤一大盅,分两次温饮下。 
复诊 将药连服三剂,已能言语,可进饮食,浮越之热已敛,体温度下降至37.6℃,心中 
已不发热,有时微觉怔忡,大便通下一次,小便亦利,遂即原方略为加减俾再服之。 
处方 野台参(四钱) 生怀山药(一两) 大甘枸杞(八钱) 净萸肉(六钱) 
生怀地黄(五钱) 甘草(二钱) 玄参(五钱) 沙参(五钱) 生赭石(四钱轧细) 生杭 
芍(三钱) 生鸡内金(钱半黄色的捣) 
共煎汤一大盅,温服。 
方解 方中加鸡内金者,因虚劳之证,脉络多瘀,《金匮》所谓血痹虚劳也。用鸡内金以化其 
血痹,虚劳可以除根,且与台 
参并用,又能运化参之补力不使作胀满也。 
效果 将药连服四剂,新得之病全愈,其素日虚劳未能尽愈。俾停服汤药,日用生怀山药 
细末煮粥,少加白糖当点心服之。每服时送服生鸡内金细末少许以善其后。 


<篇名>3.肺劳咳嗽由于伏气化热所伤证
属性:沈阳高××,三十二岁。因伏气化热伤肺,致成肺劳咳嗽证。 
病因 腊底感受寒凉,未即成病,而从此身不见汗。继则心中渐觉发热,至仲春其热加甚, 
饮食懒进,发生咳嗽,浸成肺劳病。 
证候 其咳嗽昼轻夜重,时或咳而兼喘,身体羸弱,筋骨酸疼,精神时昏愦,腹中觉饥而饮 
食恒不欲下咽。从前惟心中发热,今则日 时身恒觉热。大便燥,小便短赤,脉左右皆弦长,右部 
重按有力,一息五至。 
诊断 此病之原因,实由伏气化热久留不去。不但伤肺而兼伤及诸脏腑也。按此证自述,因腊 
底受寒,若当时即病,则为伤 
寒矣。乃因所受之寒甚轻,不能即病,惟伏于半表半里三焦脂膜之中,阻塞气化之升降流通,是以 
从此身不见汗,而心渐发热。迨时至仲春,阳气萌动,原当随春阳而化热以成温病(《内经》谓“冬伤于 
寒,春必病温”),乃其所化之热又非如温病之大热暴发能自里达表,而惟缘三焦脂膜散漫于诸 
脏腑,是以胃受其热而懒于饮食,心受其热而精神昏愦,肾受其热而阴虚潮热,肝受其热而筋骨 
酸疼,至肺受其热而咳嗽吐痰,则又其显然者也。治此证者,当以清其伏气之热为主,而以滋养津液药辅之。 
处方 生石膏(一两捣碎) 党参(三钱) 天花粉(八钱) 玄参(八钱) 生杭芍(五钱) 甘草 
(钱半) 连翘(三钱) 滑石(三钱) 鲜茅根(三钱) 射干(三钱) 生远志(二钱) 
共煎汤一大盅半,分两次温服。若无鲜茅根,可以鲜芦根代之。 
方解 方中之义,用石膏以清伏气之热,而助之以连翘、茅根,其热可由毛孔透出;更辅之以 
滑石、杭芍,其热可由水道泻出;加花粉、玄参者,因石膏但能清实热,而花粉、玄参兼能 
清虚热也;用射干、远志者,因石膏能清肺宁嗽,而佐以射干、远志,更能利痰定喘也;用甘草者, 
所以缓诸凉药之下趋,不欲其寒凉侵下焦也;至加党参者,实仿白虎加人参汤之义,因身体虚弱者, 
必石膏与人参并用,始能逐久匿之热邪外出也。 
复诊 将药连服四剂,热退三分之二,咳嗽吐痰亦愈强半,饮食加多,脉象亦见缓和。知其 
伏气之热已消,所余者惟阴虚之热也,当再投以育阴之方,俾多服数剂自能全愈。 
处方 生怀山药(一两) 大甘枸杞(八钱) 玄参(五钱) 生怀地黄(五钱) 沙参( 
五钱) 生杭芍(三钱) 生远志(二钱) 川贝母(二钱) 生鸡内金(钱半黄色的捣) 甘草(钱半) 
共煎汤一大盅温服。方中加鸡内金者,不但欲其助胃消食,兼欲借之以化诸药之滞泥也。 
效果 将药连服五剂,病遂全愈。而夜间犹偶有咳嗽之时,俾停服汤药,日用生怀山药细末煮 
作粥,调以白糖当点心服之以善其后。 


<篇名>4.虚劳咳嗽兼外感实热证
属性:抚顺一童,九岁,因有外感实热久留不去,变为虚劳咳嗽证。 
病因 从前曾受外感,热入阳明。医者纯用甘寒之药清之,致病愈之后,犹有些些余热稽留脏 
腑,久之阴分亏耗,浸成虚劳咳嗽证。 
证候 心中常常发热,有时身亦觉热,懒于饮食,咳嗽频吐痰涎,身体瘦弱。屡服清热宁 
嗽之药,即稍效病仍反复,其脉象弦数,右部尤弦而兼硬。 
诊断 其脉象弦数者,热久涸阴血液亏损也。其右部弦而兼硬者,从前外感之余热,犹留滞 
于阳明之腑也。至其咳嗽吐痰,亦热久伤肺之现象也。欲治此证,当以清其阳明余热为初步,热 
清之后,再用药滋养其真阴,病根自不难除矣。 
处方 生石膏(两半捣细) 大潞参(三钱) 玄参(五钱) 生怀山药(五钱) 鲜茅根(三 
钱) 甘草(二钱) 
共煎汤一盅半,分两次温饮下。若无鲜茅根时,可用鲜芦根代之。 
方解 此方即白虎加人参汤以玄参代知母,生山药代粳米,而又加鲜茅根也。盖阳明久郁之邪 
热,非白虎加人参汤不能清之,为其病久阴亏,故又将原方少为变通,使之兼能滋阴 
也。加鲜茅根者,取其具有升发透达之性,与石膏并用,能清热兼能散热也。 
复诊 将药煎服两剂,身心之热大减,咳嗽吐痰已愈强半,脉象亦较前和平。知外邪之热已清, 
宜再用药专滋其阴分,俾阴分充足自能尽消其余热也。 
处方 生怀山药(一两) 大甘枸杞(八钱) 生怀地黄(五钱) 玄参(四钱) 
沙参(四钱) 生杭芍(三钱) 生远志(二钱) 白术(二钱) 生鸡内金(二钱黄色的捣) 甘草(钱半) 
共煎汤一盅温服。 
效果 将药连服三剂,饮食加多,诸病皆愈。 
方解 陆九芝谓∶“凡外感实热之证,最忌但用甘寒滞泥之药治之。其病纵治愈,亦恒稽留 
余热;永锢闭于脏腑之中,不能消散,致热久耗阴,浸成虚劳,不能救药者多矣。”此诚见道之 
言也。而愚遇此等证,其虚劳不至过甚,且脉象仍有力者,恒治以白虎加人参汤,复略为变通,使 
之退实热兼能退虚热,约皆可随手奏效也。 


<篇名>5.劳热咳嗽
属性:邻村许姓学生,年十八岁,于季春得劳热咳嗽证。 
病因 秉性刚强,劳心过度;又当新婚之余,或年少失保养,迨至春阳发动,渐成劳热咳嗽证。 
证候 日晡潮热,通夜作灼,至黎明得微汗其灼乃退。白昼咳嗽不甚剧,夜则咳嗽不能安枕。 
饮食减少,身体羸瘦,略有动作即气息迫促。左右脉皆细弱,重按无根,数逾七至。夫脉一息七至, 
即难挽回,况复逾七至乎?犹幸食量犹佳,大便干燥(此等证忌滑泻),知犹可治。拟治以峻补真阴 
之剂,而佐以收 
敛气化之品。 
处方 生怀山药(一两) 大甘枸杞(八钱) 玄参(六钱) 生怀地黄(六钱) 沙参(六钱) 
甘草(三钱) 生龙骨(六钱捣碎) 净萸肉(六钱) 生杭芍(三钱) 五味子(三钱捣碎) 牛蒡子(三钱捣碎) 
共煎汤一大盅,温服。 
方解 五味入汤剂,药局照例不捣。然其皮味酸,核味辛,若囫囵入煎则其味过酸,服之恒有 
满闷之弊。故徐灵胎谓宜与干姜之味辛者同服。若捣碎入煎,正可惜其核味之辛以济皮味之酸,无 
事伍以干姜而亦不发满闷。是以欲重用五味以治嗽者,当注意令其捣碎,或说给病家自检点。至于甘草 
多用至三钱者,诚以此方中不但五味酸,萸肉亦味酸,若用甘草之至甘者与之化合,可增加其补益之力 
(如酸能 齿,得甘则不 齿是明征),是以多用至三钱。 
复诊 将药连服三剂,灼热似见退,不复出汗,咳嗽亦稍减,而脉仍七至强。因恍悟此脉之数,不 
但因阴虚,实亦兼因气虚,犹若力小而强任重者其体发颤也。拟仍峻补其真阴,再辅以补气之品。 
处方 生怀山药(一两) 野台参(三钱) 大甘枸杞(六钱) 玄参(六钱) 生怀地黄(六 
钱) 甘草(三钱) 净萸肉(五钱) 天花粉(五钱) 五味子(三钱捣碎) 生杭芍(三钱) 射 
干(二钱) 生鸡内金(钱半黄色的捣) 
共煎一大盅温服。为方中加台参恐服之作闷,是以又加鸡内金以运化之。且凡虚劳之甚者,其 
脉络间恒多瘀滞,鸡内金又善化经络之瘀滞也。 
三诊 将药连服四剂,灼热咳嗽已愈十之七八,脉已缓至六至,此足征补气有效也。爰即 
原方略为加减,多服数剂,病自除根。 
处方 生怀山药(一两) 野台参(三钱) 大甘枸杞(六钱) 玄参(五钱) 生怀地 
黄(五钱) 甘草(二钱) 天冬(五钱) 净萸肉(五钱) 生杭芍(三钱) 川贝母(三钱) 
生远志(二钱) 生鸡内金(钱半黄色的捣) 
共煎一大盅温服。 
效果 将药连服五剂,灼热咳嗽全愈,脉已复常,遂停服汤剂。 
俾日用生怀山药细末煮作茶汤,兑以鲜梨自然汁,当点心服之,以善其后。 


<篇名>6.肺劳喘嗽遗传性证
属性:天津陈××,年十八岁。自幼得肺劳喘嗽证。 
病因 因其母素有肺劳病,再上推之,其外祖母亦有斯病。是以自幼时,因有遗传性亦患此病。 
证候 其证,初时犹轻,至热时即可如常人,惟略有感冒即作喘嗽。治之即愈,不治则两 
三日亦可自愈。至过十岁则渐加重,热时亦作喘嗽,冷时则甚于热时,服药亦可见轻,旋即 
反复。至十六七岁时,病又加剧,屡次服药亦无效,然犹可支持也。迨愚为诊视,在一九三○年仲冬, 
其时病剧已难支持,昼夜伏几,喘而且嗽,咳吐痰涎,连连不竭,无论服何中药,皆分毫无 
效。惟日延西医注射药针一次,虽不能止咳喘而可保当日无虞。诊其脉左右皆弦细,关前微浮,两尺重 
按无根。 
诊断 此等证原因,肺脏气化不能通畅,其中诸细管即易为痰涎滞塞,热时肺胞松缓,故病犹 
轻,至冷时肺胞紧缩,是以其病加剧。治之者当培养其肺中气化,使之 辟有力,更疏瀹其肺 
中诸细管,使之宣通无滞,原为治此病之正则也。而此证两尺之脉无根,不但其肺中有病,其肝 
肾实亦有病,且病因又为遗传性,原非一蹴所能治愈,当分作数步治之。 
处方 生怀山药(一两) 大甘枸杞(一两) 天花粉(三钱) 天冬(三钱) 生杭芍(三钱) 
细辛(一钱) 射干(三钱) 杏仁(二钱去皮) 五味子(二钱捣碎) 葶苈子(二钱微炒) 广三七(二钱捣细) 
药共十一味,前十味煎汤一大盅,送服三七末一钱,至煎渣再服时仍送服余一钱。 
方解 方中用三七者,恐肺中之气窒塞,肺中之血亦随之凝滞,三七为止血妄行之圣药,更为流 
通瘀血之圣药,故于初步药中加之。 
复诊 将药连服四剂,咳喘皆愈三分之二,能卧睡两三点钟。其脉关前不浮,至数少减,而两 
尺似无根,拟再治以纳气归肾之方。 
处方 生怀山药(一两) 大甘枸杞(一两) 野党参(三钱) 生赭石(六钱轧细) 生怀地黄( 
六钱) 生鸡内金(钱半黄色的捣) 净萸肉(四钱) 天花粉(四钱) 天冬(三钱) 牛蒡子(三钱 
捣碎) 射干(二钱) 
共煎汤一大盅温服。 
方解 参之性补而微升,惟与赭石并用,其补益之力直达涌泉。况咳喘之剧者,其冲胃之气恒 
因之上逆,赭石实又为降胃镇冲之要药也。至方中用鸡内金者,因其含有稀盐酸,原善化肺管 
中之瘀滞以开其闭塞,又兼能运化人参之补力不使作满闷也。 
三诊 将药连服五剂,咳喘皆愈,惟其脉仍逾五至,行动时犹觉气息微喘,此乃下焦阴分犹 
未充足,不能与阳分相维系也。此当峻补其真阴,俾阴分充足自能维系其阳分,气息自不上奔矣。 
处方 生怀山药(一两) 大甘枸杞(一两) 熟怀地黄(一两) 净萸肉(四钱) 
玄参(四钱) 生远志(钱半) 北沙参(四钱) 怀牛膝(三钱) 
大云苓片(二钱) 苏子(二钱炒捣) 牛蒡子(二钱捣碎) 生鸡内金(钱半) 
共煎汤一大盅温服。 
效果 将药连服八剂,行走动作皆不作喘,其脉至数已复常。从此停服汤药,俾日用生怀山 
药细末,水调煮作茶汤,少调以生梨自然汁,当点心用之以善其后。 


<篇名>7.肺劳痰喘
属性:天津徐××,年三十四岁,得肺劳痰喘证。 
病因 因弱冠时游戏竞走,努力过度伤肺,致有喘病,入冬以来又兼咳嗽。 
证候 平素虽有喘证,然安养时则不犯,入冬以来,寒风陡至,出外为风所袭,忽发咳嗽。咳 
嗽不已,喘病亦发,咳喘相助为虐,屡次延医,服药不愈,夜不能卧。其脉左部弦细而 
硬,右部濡而兼沉,至数如常。 
诊断 此乃气血两亏,并有停饮之证,是以其左脉弦细者,气虚也。弦细兼硬者,肝血虚津液 
短也。其右脉濡者,湿痰留饮也。濡而兼沉者,中焦气化亦有所不足也。其所以喘而且嗽 
者,亦痰饮上溢之所迫致也。拟用小青龙汤,再加滋补之药治之。 
处方 生怀山药(一两) 当归身(四钱) 天冬(四钱) 寸麦冬(四钱) 
生杭芍(三钱) 清半夏(三钱) 桂枝尖(二钱五分) 五味子(二钱捣碎) 
杏仁(二钱去皮) 干姜(钱半) 细辛(一钱) 甘草(钱半) 生姜(三片) 
共煎一大盅温饮下。 
方解 凡用小青龙汤,喘者去麻黄加杏仁,此定例也。若有外感之热者,更宜加生石膏,此 
证无外感之热,故但加二冬以解姜桂诸药之热。 
复诊 将药煎服一剂,其喘即愈,又继服两剂,咳嗽亦愈强半,右脉已不沉,似稍有力,左脉 
仍近弦硬,拟再以健胃养肺滋生血脉之品。 
处方 生怀山药(一两) 生百合(五钱) 大枸杞子(五钱) 天冬(五钱) 当归 
身(三钱) 苏子(钱半炒捣) 川贝母(三钱) 白术(三钱炒) 生薏米(三钱捣碎) 生远志 
(二钱) 生鸡内金(钱半黄色的捣) 甘草(钱半) 
共煎汤一大盅温服。 
效果 将药连服四剂,咳嗽全愈,脉亦调和如常矣。 


<篇名>8.肺劳喘咳
属性:天津罗××,年三十四岁,得肺劳喘嗽病。 
病因 数年之前,曾受肺风发咳嗽,治失其宜,病虽暂愈,风邪锢闭肺中未去,致成肺劳喘嗽证。 
证候 其病在暖燠之时甚轻,偶发喘嗽一半日即愈,至冬令则喘嗽连连,必至天气暖和时 
始渐愈。其脉左部弦硬,右部濡滑,两尺皆重按无根。 
诊断 此风邪锢闭肺中,久而伤肺,致肺中气管滞塞,暖时肌肉松缓,气管亦随之松缓,其呼 
吸犹可自如;冷时肌肉紧缩,气管亦随之紧缩,遂至吸难呼易而喘作,更因痰涎壅滞而嗽作 
矣。其脉左部弦硬者,肝肾之阴液不足也。右部濡滑者,肺胃中痰涎充溢也。两尺不任重按 
者,下焦气化虚损,不能固摄,则上焦之喘嗽益甚也。欲治此证,当先宣通其肺,俾气管之 
郁者皆开后,再投以滋阴培气,肺肾双补之剂以祓除其病根。 
处方 麻黄(钱半) 天冬(三钱) 天花粉(三钱) 牛蒡子(三钱捣碎) 杏仁(二钱去皮 
捣碎) 甘草(钱半) 苏子(二钱炒捣) 生远志(二钱去心) 
生麦芽(二钱) 生杭芍(二钱) 细辛(一钱) 
共煎汤一大盅,温服。 
复诊 将药煎服两剂,喘嗽皆愈,而劳动时仍微喘。其脉左部仍似弦硬,右部仍濡,不若 
从前之滑,两尺犹虚,此病已去而正未复也。宜再为谋根本之治法,而投以培养之剂。 
处方 野台参(三钱) 生赭石(八钱轧细) 生怀山药(一两) 熟怀地黄(一两) 
生怀地黄(一两) 大云苓片(二钱) 大甘枸杞(六钱) 天冬(六钱) 净萸肉(五钱) 
苏子(三钱炒捣) 牛蒡子(三钱捣碎) 
共煎一大盅温服。 
方解 人参为补气主药,实兼具上升之力。喻嘉言谓。“气虚欲上脱者专用之转气高不返。” 
是以凡喘逆之证,皆不可轻用人参,惟重用赭石以引之下行,转能纳气归肾,而下焦之气 
化,遂因之壮旺而固摄。此方中人参、赭石并用,不但欲导引肺气归肾,实又因其两 
尺脉虚,即借以培补下焦之气化也。 
效果 将药连服十余剂,虽劳动亦不作喘。再诊其脉,左右皆调和无病,两尺重按 
不虚,遂将赭石减去二钱,俾多服以善其后。 


<篇名>9.肺劳喘嗽兼不寐证
属性:天津于姓媪,年近五旬,咳嗽有痰微喘,且苦不寐。 
病因 夜间因不能寐,心中常觉发热,久之,则肺脏受伤,咳嗽多痰,且微作喘。 
证候 素本夜间不寐,至黎明时始能少睡。后因咳嗽不止,痰涎壅盛,且复作喘,不能安卧, 
恒至黎明亦不能睡。因之心中发热益甚,懒于饮食,大便干燥,四五日一行,两旬之间大形 
困顿,屡次服药无效。其脉左部弦而无力,右部滑而无力,数逾五至。 
诊断 此真阴亏损,心肾不能相济,是以不眠。久则心血耗散,心火更易妄动以上铄肺金,是 
以咳嗽有痰作喘。治此证者,当以大滋真阴为主,真阴足则心肾自然相交,以水济火而火不妄 
动;真阴足则自能纳气归根,气息下达,而呼吸自顺。且肺肾为子母之脏,原相连属,子虚有损于母,子 
实即有益于母,果能使真阴充足,则肺金既不受心火之铄耗,更可得肾阴之津润,自能复其清肃 
下行之常,其痰涎咳嗽不治自愈也。若更辅以清火润肺化痰宁嗽之品,则奏效当更捷矣。 
处方 沙参(一两) 大枸杞(一两) 玄参(六钱) 天冬(六钱) 
生赭石(五钱轧细) 甘草(二钱) 生杭芍(三钱) 川贝母(三钱) 牛蒡子(一钱捣碎) 
生麦芽(三钱) 枣仁(三钱炒捣) 射干(二钱) 
共煎汤一大盅,温服。 
复诊 将药连服六剂,咳喘痰涎愈十分之八,心中已不发热,食欲已振,夜能睡数时,大便亦 
不甚燥。诊其脉至数复常,惟六部重按仍皆欠实,左脉仍有弦意。拟再峻补其真阴以除病 
根,所谓上病取诸下也。 
处方 生怀山药(一两) 大枸杞(一两) 辽沙参(八钱) 生怀地黄(六钱) 
熟怀地黄(六钱) 甘草(二钱) 生赭石(六钱轧细) 净萸肉(四钱) 
生杭芍(三钱) 生麦芽(三钱) 生鸡内金(针半黄色的捣) 
共煎汤一大盅,温服。 
效果 将药连服二剂,诸病皆愈,俾用珠玉二宝粥常常当点心服之,以善其后。 
或问 两方中所用之药,若滋阴、润肺、清火、理痰、止嗽诸品,原为人所共知,而两方 
之中皆用赭石、麦芽,且又皆生用者其义何居?答曰∶胃居中焦,原以传送饮食为专职,是 
以胃中之气,以息息下行为顺,果其气能息息下行,则冲气可阻其上冲,胆火可因之下降,大便 
亦可按时下通,至于痰涎之壅滞,咳嗽喘逆诸证,亦可因之降序,而降胃之药,固莫赭石若也。至于 
麦芽,炒用之善于消食,生用之则善于升达肝气。人身之气化原左升右降,若但知用赭石降胃,其重坠 
下行之力或有碍于肝气之上升,是以方中用赭石降胃,即用麦芽升肝,此所以顺气化之自然,而还 
其左升右降之常也。 


<篇名>10.肺病咳嗽吐血
属性:天津张××,年二十六岁,得肺病咳嗽吐血。 
病因 经商劳心,又兼新婚,失于调摄,遂患劳嗽。继延推拿者为 
推拿两日,咳嗽分毫未减,转添吐血之证。 
证候 连声咳嗽不已,即继以吐血。或痰中带血,或纯血无痰,或有咳嗽兼喘。夜不能卧,心 
中发热,懒食,大便干燥,小便赤涩。脉搏五至强,其左部弦而无力,右部浮取似有力, 
而尺部重按豁然。 
处方 生怀山药(一两) 大潞参(三钱) 生赭石(六钱轧细) 生怀地黄(六钱) 
玄参(六钱) 天冬(五钱) 净萸肉(五钱) 生杭芍(四钱) 射干(二钱) 甘草(二钱) 
广三七(二钱轧细) 
药共十一味,将前十味煎汤一大盅,送服三七末一半,至煎渣重服时,再送服其余一半。 
复诊 此药服两剂后,血已不吐,又服两剂,咳嗽亦大见愈,大小便已顺利,脉已有根,不若 
从前之浮弦。遂即原方略为加减,俾再服之。 
处方 生怀山药(一两) 大潞参(三钱) 生赭石(六钱轧细) 生怀地黄(六钱) 
大甘枸杞(六钱) 甘草(二钱) 净萸肉(五钱) 沙参(五钱) 
生杭芍(二钱) 射干(二钱) 广三七(钱半轧细) 
药共十一味,将前十味煎汤一大盅,送服三七末一半,至煎渣重服时,再送其余一半。 
效果 将药连服五剂,诸病皆愈,脉已复常,而尺部重按仍欠实。遂于方中加熟怀地黄五钱, 
俾再服数剂以善其后。 


<篇名>11.肺病咳吐脓血
属性:天津叶××,年三十二岁,得肺病咳吐脓血。 
病因 其未病之前数月,心中时常发热,由此浸成肺病。 
证候 初觉发热时,屡服凉药,热不减退,大便干燥,小便短赤,后则渐生咳嗽,继则痰中 
带血,继则痰血相杂,又继则脓血相杂。诊其脉左部弦长,右部洪长,皆重按颇实。 
疹断 此乃伏气化热,窜入阳明之腑。医者不知病因,见其心中发热,而多用甘寒滞腻之品, 
稽留其热,俾无出路。久之,上熏肺部,至肺中结核因生咳嗽,溃烂遂吐脓血,斯必先清其胃腑之 
热,使不复上升熏肺而后肺病可愈。特是,此热为伏气之热所化,原非轻剂所能消除,当先投以 
治外感实热之剂。 
处方 生石膏(两半捣细) 大潞参(三钱) 生怀山药(六钱) 天花粉(六钱) 金银花( 
四钱) 鲜芦根(四钱) 川贝母(三钱) 连翘(二钱) 甘草(二钱) 广三七(二钱轧细) 
药共十味,将前九味煎汤一大盅,送服三七末一钱,至煎渣再服时,仍送服余一钱。 
方解 此方实仿白虎加人参汤之义而为之变通也。方中以天花粉代知母,以生山药代粳米,仍与 
白虎加人参汤无异,故用之以清胃腑积久之实热。而又加金银花、三七以解毒,芦根、 
连翘以引之上行,此肺胃双理之剂也。 
复诊 将药连服三剂,脓血已不复吐,咳嗽少愈,大便之干燥,小便之短赤亦见愈。惟心中仍 
觉发热,脉象仍然有力,拟再投以清肺泻热之剂。 
处方 天花粉(八钱) 北沙参(五钱) 玄参(五钱) 鲜芦根(四钱) 川贝母( 
三钱) 牛蒡子(三钱捣碎) 五味子(二钱捣细) 射干(三钱) 甘草(二钱轧细) 
药共九味,将前八味煎汤一大盅,送服甘草末一钱,至煎渣再服时,仍送服余一钱。方中五味子, 
必须捣碎入煎,不然则服之恒多发闷;方中甘草,无论红者黄者,皆可用至轧之不细时,切忌锅炮, 
若炮则其性即变,非此方中用甘草之意矣。用此药者,宜自监视轧之,或但罗取其头次所轧之末亦可。 
效果 将药连服五剂,诸病皆愈,惟心中犹间有发热之时,脉象较常脉似仍有力。为善后计, 
俾用生怀山药轧细,每用七八钱或两许,煮作茶汤,送服离中丹钱许或至钱半(多少宜自酌), 
当点心用之。后此方服阅两月,脉始复常,心中亦不复发热矣。离中丹为愚自制之方,即益元 
散方以生石膏代滑石也。盖滑石宜于湿热,石膏宜于燥热,北方多热而兼燥者,故将 
其方变通之,凡上焦有实热者,用之皆有捷效。 
或问 伏气化热,原可成温,即无新受之外感,而忽然咸温病者是也。此证伏气所化之热,何 
以不成温病而成肺病?答曰∶伏气之侵人,伏于三焦脂膜之中,有多有少,多者化热重,少 
者化热轻,化热重者当时即成温病,化热轻者恒循三焦脂膜而窜入各脏腑。愚临证五十年,细心体 
验,知有窜入肝胆病目者,窜入肠中病下痢者,有窜入肾中病虚劳者,窜入肺 
中病咳嗽久而成肺病者,有窜入胃中病吐衄而其热上熏亦可成肺病者,如此证是也。是以此证心中 
初发热时,医者不知其有伏气化热入胃,而泛以凉药治之,是以不效,而投以白虎加人参汤即随手奏 
效。至于不但用白虎汤而必用白虎加人参汤者,诚以此证已阅数月,病久气化虚损,非人参与石膏并 
用,不能托深陷之热外出也。 


<篇名>12.肺病咳吐痰血
属性:天津乔××,年三十余,得咳吐痰血病。 
病因 前因偶受肺风,服药失宜,遂息咳嗽,咳嗽日久,继患咳血。 
证候 咳嗽已近一年,服药转浸加剧,继则痰中带血,又继则间有呕血之时,然犹不至于倾吐。 
其心中时常发热,大便时常燥结,幸食欲犹佳,身形不至羸弱,其脉左部近和平,右部 
寸关俱有滑实之象。 
诊断 证脉合参,知系从前外感之热久留肺胃,金畏火刑,因热久而肺金受伤,是以咳嗽;至于 
胃腑久为热铄,致胃壁之膜腐烂连及血管,是以呕血;至其大便恒燥结者,因其热下输肠中,且因胃 
气因热上逆失其传送之职也。治此证者,当以清肺胃之热为主,而以养肺降胃之药辅之。 
处方 生石膏(二两细末) 粉甘草(六钱细末) 镜面朱砂(二钱细末) 
共和匀每服一钱五分。 
又方 生怀山药(一两) 生赭石(八钱轧细) 天冬(六钱) 玄参(五钱) 沙参( 
五钱) 天花粉(五钱) 生杭芍(四钱) 川贝母(三钱) 射干(二钱) 儿茶(二钱) 甘草( 
钱半) 广三七(二钱轧细) 
共药十二味,将前十一味煎汤送服三七一钱,至煎渣再服时 
再送服一钱。每日午前十点钟服散药一次,临睡时再服一次,汤药则晚服头煎,翌晨服次煎。 
效果 服药三日,咳血吐血皆愈。仍然咳嗽,遂即原方去沙参加生百合五钱、米壳钱半,又服四 
剂,咳嗽亦愈,已不发热,大便已不燥结。俾将散药惟头午服一次,又将汤药中赭石减 
半,再服数剂以善后。 

\chapter{气病门}
<篇名>1.大气下陷兼小便不禁
属性:天津陈××,三十五岁,于孟冬得大气下陷兼小便不禁证。 
病因 禀赋素弱,恒觉呼吸之气不能上达,屡次来社求诊,投以拙拟升陷汤,即愈。后以出外劳 
碌过度,又兼受凉,陡然反复甚剧,不但大气下陷,且又小便不禁。 
证候 自觉胸中之气息息下坠,努力呼之犹难上达,其下坠之气行至少腹,小便即不能禁,且觉 
下焦凉甚,肢体无力,其脉左右皆沉濡,而右部寸关之沉濡尤甚。 
诊断 此胸中大气下陷之剧者也。此证因大气虚陷,心血之循环无力,是以脉象沉濡而迟,肺 
气之呼吸将停,是以努力呼气外出而犹难上达。不但此也,大气虽在膈上,实能斡旋全身统摄 
三焦,今因下陷而失位无权,是以全身失其斡旋,肢体遂酸软无力,三焦失其统摄,小便遂泄泻不禁。 
其下焦凉甚者,外受之寒凉随大气下陷至下焦也。此证之危已至极点,当用重剂升举其下陷之大气, 
使复本位,更兼用温暖下焦之药,祛其寒凉庶能治愈。 
处方 野台参(五钱) 乌附子(四钱) 生怀山药(一两) 
煎汤一盅温服,此为第一方。 
又方 生箭 (一两) 生怀山药(一两) 白术(四钱炒) 净萸肉(四钱) 
萆(二钱) 升麻(钱半) 柴胡(钱半) 
共煎药一大盅,温服。此为第二方。先服第一方,后迟一点半钟即服第二方。 
效果 将药如法各服两剂,下焦之凉与小便之不禁皆愈,惟呼吸犹觉气分不足,肢体虽不酸软, 
仍觉无力。遂但用第二方,将方中柴胡减去,加桂枝尖钱半,连服数剂,气息已顺。又将方中升麻、 
桂枝,皆改用一钱,服至五剂,身体健康如常,遂停药勿服。 
或问 此二方前后相继服之,中间原为时无多,何妨将二方并为一方?答曰∶凡欲温暖下焦之药, 
宜速其下行,不可用升药提之。若将二方并为一方,附子与升、柴并用,其上焦必生烦躁,而下 
焦之寒凉转不能去。惟先服第一方,附子得人参之助,其热力之敷布最速,是以为时虽无多,下焦之寒 
凉已化其强半;且参附与山药并用,大能保合下焦之气化,小便之不禁者亦可因之收摄,此时下焦 
受参附山药之培养,已有一阳来复,徐徐上升之机。已陷之大气虽不能因之上升,实已有上升之根 
基。遂继服第二方,黄 与升柴并用,升提之力甚大,借之以升提下陷之大气,如人欲登高山则或推之, 
或挽之,纵肢体软弱,亦不难登峰造极也。且此一点余钟,附子之热力已融化于下焦,虽遇升柴之 
升提,必不至上升作烦躁,审斯则二方不可相并之理由,及二方前后继服之利益不昭然乎。 


<篇名>2.大气下陷
属性:天津李××,年三十二岁,拉洋车为业,得大气下陷证。 
病因 腹中觉饥,未吃饭,枵腹奔走七八里,遂得此病。 
证候 呼吸短气,心中发热,懒食,肢体酸懒无力,略有动作,即觉气短不足以息。其脉左部 
弦而兼硬,右部则寸关皆沉而无力。 
诊断 此胸中大气下陷,其肝胆又蕴有郁热也。盖胸中大气,原为后天宗气,能代先天元气主 
持全身,然必赖水谷之气以养之。此证因忍饥劳力过度,是以大气下陷,右寸关之沉而无 
力其明征也。其举家数口生活皆赖一人劳力,因气陷不能劳力继将断炊,肝胆之中遂多 
起急火,其左脉之弦而兼硬是明征也。治之者当用拙拟之升陷汤,升补其胸中大气,而辅以 
凉润之品以清肝胆之热。 
处方 生箭 (八钱) 知母(五钱) 桔梗(二钱) 柴胡(二钱) 
升麻(钱半) 生杭芍(五钱) 龙胆草(二钱) 
共煎汤一大盅,温服。 
效果 将药连服两剂,诸病脱然全愈。 


<篇名>3.大气下陷身冷
属性:天津宋氏妇,年四旬,于仲夏得大气下陷,周身发冷证。 
病因 禀赋素弱,居恒自觉气分不足,偶因努力搬运重物,遂觉呼吸短气,周身发冷。 
证候 呼吸之间,恒觉气息不能上达,时当暑热,着夹衣犹觉寒凉,头午病稍轻,午后则 
渐剧,必努力始能呼吸,外被大氅犹或寒战,饮食少许,犹不消化。其脉关前沉细欲无,关后 
差胜亦在沉分,一息不足四至。 
诊断 此上焦心肺之阳虚损,又兼胸中大气下陷也。为其心肺阳虚,是以周身恶寒而饮 
食不化,为其胸中大气下陷,是以呼吸短气,头午气化上升之时是以病轻,过午气化下降之时所 
以增剧也。拟治以回阳升陷汤加党参之大力者以补助之。 
处方 生箭 (八钱) 野台党参(四钱) 干姜(四钱) 当归身(四钱) 
桂枝尖(三钱) 甘草(二钱) 
共煎汤一大盅,温服。 
效果 将药连服三剂,气息已顺,而兼有短气之时,周身已不发冷,惟晚间睡时仍须浓复,饮食 
能消化,脉象亦大有起色。遂即原方去党参,将干姜、桂枝皆改用二钱,又加生怀山药 
八钱,俾再服数剂,以善其后。 
帮助 心为君火,全身热力之司命,肺与心同居膈上,一系相连,血脉之循环又息息相通,是以与心 
相助为理,同主上焦之阳气。然此气虽在上焦,实如日丽中天,照临下土,是以其热力透至中焦, 
胃中之饮食因之熟腐,更透至下焦,命门之相火因之生旺,内温脏腑,外暖周身,实赖此阳气为布护 
宣通也。特是,心与肺皆在胸中大气包举之中,其布护宣通之原动力,实又赖于大气。此证心肺之 
阳本虚,向赖大气为之保护,故犹可支持,迨大气陷而失其保护,遂致虚寒之象顿呈。此方以升补 
胸中大气为主,以培养心肺之阳为辅,病药针芥相投,是以服之辄能奏效也。 


<篇名>4.大气陷兼消食
属性:李××,年二十六岁,得大气下陷兼消食证。 
病因 其未病之前二年,常觉呼吸短气,初未注意。继因校中功课劳心短气益剧,且觉食量倍增,因成 
消食之证。 
证候 呼吸之间,觉吸气稍易而呼气费力,夜睡一点钟许,即觉气不上达,须得披衣起坐,迟移时,气 
息稍顺,始能再睡。一日之间,进食四次犹饥,饥时若不急食,即觉怔忡。且心中常觉发热,大便干 
燥,小便短赤,其脉浮分无力,沉分稍 
实,至数略迟。 
诊断 此乃胸中大气下陷,兼有伏气化热因之成消食也。为其大气下陷,是以脉象浮分无力,为 
其有伏气化热,是以其沉分犹实,既有伏气化热矣,而脉象转稍迟者,因大气下陷之脉原多 
迟也。盖胃中有热者,恒多化食,而大气下陷其胃气因之下降甚速者,亦恒能多食。今既病大气 
下陷,又兼伏气化热,侵入胃中,是以日食四次犹饥也。此宜升补其胸中大气, 
再兼用寒凉之品以清其伏气所化之热,则短气与消食原不难并愈也。 
处方 生箭 (六钱) 生石膏(一两捣细) 天花粉(五钱) 知母(五钱) 玄参(四钱 
)升麻(钱半) 柴胡(钱半) 甘草(钱半) 
共煎汤一大盅温服。 
复诊 将药连服四剂,短气已愈强半,发热与消食亦大见愈,遂即原方略为加减俾再服之。 
处方 生箭 (六钱) 天花粉(六钱) 知母(六钱) 玄参(六钱) 
净萸肉(三钱) 升麻(钱半) 柴胡(钱半) 甘草(钱半) 
共煎汤一大盅,温服。 
方解 方中去石膏者,以伏气所化之热所余无多也。既去石膏而又将花粉、知母诸凉药加重者,因 
花粉诸药原用以调剂黄 之温补生热,而今则兼用之以清伏气所化之余热,是以又加重也。至于前 
方之外,又加萸肉者,欲以收敛大气之涣散,俾大气之已升者不至复陷,且又以萸肉得木气最浓,酸敛之 
中大具条畅之性,虽伏气之热犹未尽消,而亦不妨用之也。 
效果 将药又连服四剂,病遂全愈。俾停服汤药,再用生箭 、 
天花粉等分轧为细末,每服三钱,日服两次以善其后。 
或问 脉之迟数,恒关于人身之热力,热力过盛则脉数,热力微 
弱则脉迟,此定理也。今此证虽有伏气化热,因大气下陷而 
脉仍迟,何以脉之迟数与大气若斯有关系乎?答曰∶胸中大气亦名宗气,为其实用能斡旋全身,故曰 
大气,为其为后天生命之宗主,故又曰宗气。《内经》谓宗气积于胸中以贯心脉而行呼吸,深思《内 
经》之言,知肺叶之 辟,固为大气所司,而心机之跳动,亦为大气所司也。今因大气下陷而失 
其所司,是以不惟肺受其病,心机之跳动亦受其病而脉遂迟也。 


<篇名>5.大气陷兼疝气
属性:天津陈××,年三十八岁,得大气下陷兼疝气证。 
病因 初因劳心过度,浸觉气分不舒,后又因出外办事劳碌过甚,遂觉呼吸短气,犹不 
以为意也。继又患疝气下坠作疼,始来寓求为延医。 
证候 呼吸之际,常觉气短似难上达,劳动时则益甚。夜间卧睡一点钟许,即觉气分不舒, 
披衣起坐移时将气调匀,然后能再睡。至其疝气之坠疼,恒觉与气分有关,每当呼吸不利时, 
则疝气之坠疼必益甚。其脉关前沉而无力,右部尤甚,至数稍迟。 
诊断 即此证脉参之,其呼吸之短气,疝气之下坠,实皆因胸中大气下陷也。此气一陷则肺脏 
之辟失其斡旋,是以呼吸短气,三焦之气化失其统摄,是以疝气下坠。斯当升补其下陷 
之大气,俾仍还其本位,则呼吸之短气,疝气之坠疼自皆不难愈矣。 
处方 生箭 (六钱) 天花粉(六钱) 当归(三钱) 荔枝核(三钱) 
生明没药(三钱) 生五灵脂(三钱) 柴胡(钱半) 升麻(钱半) 小茴香(一钱炒捣) 
共煎汤一大盅,温饮下。 
复诊 将药连服三剂,短气之病已大见愈,惟与人谈话多时,仍觉短气。其疝气已上升,有 
时下坠亦不作疼,脉象亦大有起色。此药已对证,而服药之功候未到也。爰即原方略为加 
减,俾再服之。 
处方 生箭 (六钱) 天花粉(六钱) 净萸肉(四钱) 当归(三钱) 荔枝核(三 
钱) 生明没药(三钱) 生五灵脂(三钱) 柴胡(钱半) 升麻(钱半) 广砂仁(一钱捣碎) 
共煎一大盅温服。 
效果 将药连服四剂,呼吸已不短气,然仍自觉气分不足,疝气亦大轻减,犹未全消。遂即 
原方去萸肉,将柴胡、升麻皆改用一钱,又加党参、天冬各三钱,俾多服数剂以善其后。 


<篇名>6.冲气上冲兼奔豚
属性:天津张××,年四十五岁,得冲气上冲兼奔豚证。 
病因 初秋之时,患赤白痢证,医者两次用大黄下之,其痢愈而变为此证。 
证候 每夜间当丑寅之交,有气起自下焦挟热上冲,行至中焦觉闷而且热,心中烦乱,迟十数 
分钟其气上出为呃,热即随之消矣。其脉大致近和平,惟两尺稍浮,按之不实。 
诊断 此因病痢时,连服大黄下之,伤其下焦气化,而下焦之冲遂挟肾中之相火上冲也。其在 
丑寅之交者,阳气上升之时也。宜用仲师桂枝加桂汤加减治之。 
处方 桂枝尖(四钱) 生怀山药(一两) 生芡实(六钱捣碎) 清半夏(四钱水洗三次) 
生杭芍(四钱) 生龙骨(四钱捣碎) 生牡蛎(四钱捣碎) 生麦芽(三钱) 
生鸡内金(二钱黄色的捣) 黄柏(二钱) 甘草(二钱) 
共煎汤一大盅,温服。 
效果 将药煎服两剂,病愈强半,遂即原方将桂枝改用三钱,又 
加净萸肉、甘枸杞各四钱,连服三剂全愈。 
帮助 凡气之逆者可降,郁者可升,惟此证冲气挟相火上冲,则升降皆无所施。桂枝一药而 
升降之性皆备,凡气之当升者遇之则升,气之当降者遇之则降,此诚天生使独而为不可思议 
之妙药也。山药、芡实,皆能补肾,又皆能敛戢下焦气化;龙骨、牡蛎,亦收敛之品,然敛正 
气而不敛邪气,用于此证初无收敛过甚之虞,此四药并用,诚能于下焦之气化培养而 
镇安之也。用芍药、黄柏者,一泻肾中之相火,一泻肝中之相火,且桂枝性热,二药性凉,凉 
热相济,方能奏效。用麦芽、鸡内金者,所以运化诸药之力也。用甘草者,欲以缓肝之急,不使 
肝木助气冲相火上升也。至于服药后病愈强半,遂减轻桂枝加萸肉、枸杞者,俾肝肾壮旺自能扫除病根。 


<篇名>7.胃气不降
属性:大城王××妻,年近四旬,时常呕吐,大便迟下,数年不愈。 
病因 其人禀性暴烈,处境又多不顺,浸成此证。 
证候 饭后每觉食停胃中,似有气上冲阻其下行,因此大便恒至旬日始下。至大便多日不下 
时,则恒作呕吐,即屡服止呕通便之药,下次仍然如故。求为延医,其脉左右皆弦,右脉弦 
而且长,重诊颇实,至数照常。 
诊断 弦为肝脉,弦而且长则冲脉也。弦长之脉,见于右部,尤按之颇实,此又为胃气上逆 
之脉。肝胃冲三经之气化皆有升无降,宜其下焦便秘而上焦呕吐也。此当治以泻肝、降胃、 
镇冲之剂,其大便自顺,呕吐自止矣。 
处方 生赭石(两半轧细) 生杭芍(六钱) 柏子仁(六钱) 生怀山药(六钱) 
天冬(六钱) 怀牛膝(五钱) 当归(四钱) 生麦芽(三钱) 
茵陈(二钱) 甘草(钱半) 
共煎汤一大盅,温服。 
效果 服药一剂,大便即通下,即原方略为加减,又服数剂,大 
便每日一次,食后胃中已不觉停滞,从此病遂除根。 
或问 麦芽生用能升肝气,茵陈为青蒿之嫩者亦具有升发之力,此证即因脏腑之气有升 
无降,何以方中复用此二药乎?答曰∶肝为将军之官,中寄相火,其性最刚烈,若强制之,恒 
激发其反动之力;麦芽、茵陈,善舒肝气而不至过于升提,是将顺肝木之性使之柔和,不至起反动力也。 


<篇名>8.肝气郁兼胃气不降
属性:天津姚××,年五十二岁,得肝郁胃逆证。 
病因 劳心太过,因得斯证。 
证候 腹中有气,自下上冲,致胃脘满闷,胸中烦热,胁下胀疼,时常呃逆,间作呕吐。大便 
燥结,其脉左部沉细,右部则弦硬而长,大于左部数倍。 
诊断 此乃肝气郁结,冲气上冲,更迫胃气不降也。为肝气郁结,是以左脉沉细,为冲气上冲, 
是以右脉弦长,冲脉上隶阳明,其气上冲不已,易致阳明胃气不下降。此证之呕吐呃逆,胃脘 
满闷,胸间烦热,皆冲胃之气相并冲逆之明征也。其胁下胀疼,肝气郁结之明征也。其大便燥结者, 
因胃气原宜息息下行,传送饮食下为二便,今其胃气既不下降,是以 
大便燥结也。拟治以舒肝降胃安冲之剂。 
处方 生赭石(一两轧细) 生怀山药(一两) 天冬(一两) 寸麦冬(六钱去心) 
清半夏(四钱水洗三次) 碎竹茹(三钱) 生麦芽(三钱) 茵陈(二钱) 
川续断(二钱) 生鸡内金(二钱黄色的捣) 甘草(钱半) 
煎汤一大盅,温服。 
方解 肝主左而宜升,胃主右而宜降,肝气不升则先天之气化不能由肝上达,胃气不降则后天之 
饮食不能由胃下输,此证之病根,正因当升者不升,当降者不降也。故方中以生麦芽、茵陈以升肝;生 
赭石、半夏、竹茹以降胃,即以安冲;用续断者,因其能补肝,可助肝气上升也;用生山药二冬者,取 
其能润胃补胃,可助胃气下降也,用鸡内金者,取其能化瘀止疼,以营运诸药之力也。 
复诊 上方随时加减,连服二十余剂,肝气已升,胃气已降,左 
右脉均已平安,诸病皆愈。惟肢体乏力,饮食不甚消化,拟再治以补气健胃之剂。 
处方 野台参(四钱) 生怀山药(一两) 生赭石(六钱轧细) 天冬(六钱) 
寸麦冬(六钱) 生鸡内金(三钱黄色的捣) 生麦芽(三钱) 甘草(钱半) 
煎汤一大盅,温服。 
效果 将药煎服三剂,饮食加多,体力渐复。于方中加枸杞五钱,白术三钱,俾再服数剂以善其后。 
帮助 身之气化,原左升右降,若但知用赭石降胃,不知用麦芽升肝,久之,肝气将有郁遏之 
弊,况此证之肝气原郁结乎?此所以方中用赭石,即用麦芽,赭石生用而麦芽亦生用也。 
且诸家本草谓麦芽炒用者为丸散计也,若入汤剂何须炒用,盖用生者煮汁饮之,则消食之力愈大也。 
或问 升肝之药,柴胡最效,今方中不用柴胡而用生麦芽者,将毋别有所取乎?答曰∶柴胡升提 
肝气之力甚大,用之失宜,恒并将胃气之下行者提之上逆。曾有患阳明厥逆吐血者,初不甚剧。医者误 
用柴胡数钱即大吐不止,须臾盈一痰盂,有危在顷刻之惧,取药无及,适备有生赭石细末若干,俾急用 
温开水送下,约尽两半,其血始止,此柴胡并能提胃气上逆之明征也。况此证之胃气原不降乎? 
至生麦芽虽能升肝,实无妨胃气之下降,盖其萌芽发生之性,与肝木同气相求,能 
宣通肝气之郁结,使之开解而自然上升,非若柴胡之纯于升提也。 


<篇名>9.胃气不降
属性:掖县任××妻,年五旬,得胃气不降证。 
原因 举家人口众多,因其夫在外,家务皆自操劳,恒动肝火,遂得此证。 
证候 食后停滞胃中,艰于下行,且时觉有气挟火上冲,口苦舌胀,目眩耳鸣,恒有呃欲呕逆 
或恶心,胸膈烦闷,大便六七日始行一次,或至服通利药始通,小便亦不顺利。其脉左部 
弦硬,右部弦硬而长,一息搏近五至,受病四年,屡次服药无效。 
诊断 此肝火与肝气相并,冲激胃腑,致胃腑之气不能息息下行传送饮食,久之,胃气不但不能 
下行,且更转而上逆,是以有种种诸病也。宜治以降胃理冲之品,而以滋阴清火之药辅之。 
处方 生赭石(两半轧细) 生怀山药(一两) 生杭芍(六钱) 玄参(六钱) 
生麦芽(三钱) 茵陈(二钱) 生鸡内金(二钱黄色的捣) 甘草(钱半) 
共煎汤一大盅,温服。 
效果 每日服药一剂,三日后大便日行一次,小便亦顺利。上焦诸病亦皆轻减,再诊其脉,颇 
见柔和。遂将赭石减去五钱,又加柏子仁五钱,连服数剂,霍然全愈。 

\chapter{血病门}
<篇名>1.吐血证
属性:天津张××,年三十五岁,得吐血证,年余不愈。 
病因 禀性褊急,劳心之余又兼有拂意之事,遂得斯证。 
证候 初次所吐甚多,屡经医治,所吐较少,然终不能除根。每日或一次或两次,觉心中有热 
上冲,即吐血一两口。因病久身羸弱,卧床不起,亦偶有扶起少坐之时,偶或微喘,幸食 
欲犹佳,大便微溏,日行两三次,其脉左部弦长,重按无力,右部大而芤,一息五至。 
诊断 凡吐血久不愈者,多系胃气不降,致胃壁破裂,出血之处不能长肉生肌也。再即此脉论 
之,其左脉之弦,右脉之大,原现有肝火浮动挟胃气上冲之象,是以其吐血时,觉有热上 
逆,至其脉之弦而无力者,病久而气化虚也。大而兼芤者,失血过多也。至其呼吸有时或喘, 
大便日行数次,亦皆气化虚而不摄之故。治此证者,当投以清肝、降胃、培养气血、固摄气化之剂。 
处方 赤石脂(两半) 生怀山药(一两) 净萸肉(八钱) 生龙骨(六钱捣碎) 
生牡蛎(六钱捣碎) 生杭芍(六钱) 大生地黄(四钱) 甘草(二钱) 广三七(二钱) 
药共九味,将前八味煎汤送服三七末。 
方解 降胃之药莫如赭石,此愚治吐衄恒用之药也。此方中独重用赤石脂者,因赭石为铁养化 
合其重坠之力甚大,用之虽善降胃,而其力达于下焦,又善通大便,此证大便不实,赭石 
似不宜用;赤石脂之性,重用之亦能使胃气下降,至行至下焦,其粘滞之力又能固涩大便,且 
其性能生肌,更可使肠壁破裂出血之处早愈,诚为此证最宜之药也。 
效果 将药煎服两剂,血即不吐,喘息已平,大便亦不若从前之勤,脉象亦较前和平,惟心 
中仍有觉热之时。遂即原方将生地黄改用一两,又加熟地黄一两,连服三剂,诸病皆愈。 


<篇名>2.咳血兼吐血证
属性:堂侄女××,适邻村王氏,年三十岁。于乙酉仲春,得吐血证。 
病因 因家务自理,劳心过度,且禀赋素弱,当此春阳发动之时,遂病吐血。 
证候 先则咳嗽痰中带血,继则大口吐血,其吐时觉心中有热上冲,一日夜吐两三次, 
剧时可吐半碗。两日之后,觉精神气力皆不能支持,遂急迎愚延医。自言心中摇摇似将上脱,两 
颧发红,面上发热,其脉左部浮而动,右部浮而濡,两尺无根,数逾五至。 
诊断 此肝肾虚极,阴分阳分不相维系,而有危在顷刻之势。遂急为出方取药以防虚脱。 
处方 生怀山药(一两) 生怀地黄(一两) 熟怀地黄(一两) 净萸肉(一两) 生赭石(一两轧细) 
急火煎药取汤两盅,分两次温服下。 
效果 将药甫煎成未服,又吐血一次,吐后忽停息闭目 然罔觉。诊其脉跳动仍旧,知能 
苏醒,约四分钟呼吸始续,两次将药服下,其血从此不吐。俾即原方再服一剂,至第三剂即原方加 
潞党参三钱、天冬四钱,连服数剂,身形亦渐撤消。继用生怀山药为细面,每用八钱煮作茶汤,少 
调以白糖,送服生 
赭石细末五分,作点心用之以善其后。 


<篇名>3.吐血兼咳嗽
属性:天津王××,年二十四岁,得咳嗽吐血证。 
病因 禀赋素弱,略有外感,即发咳嗽,偶因咳嗽未愈,继又劳心过度,心中发热,遂至吐血。 
证候 先时咳嗽犹轻,失血之后则嗽益加剧。初则痰中带血,继则大口吐血,心中发热,气息微 
喘,胁下作疼,大便干燥。其脉关前浮弦,两尺重按不实,左右皆然,数逾五至。 
诊断 此证乃肺金伤损,肝木横恣,又兼胃气不降,肾气不摄也。为其肺金受伤,是以咳嗽痰中 
带血;为胃气不降,是以血随气升,致胃中血管破裂而大口吐血;至胁下作疼,乃肝木横 
恣之明证;其脉上盛下虚,气息微喘,又肾气不摄之明征也。治之者,宜平肝、降胃、润肺、 
补肾,以培养调剂其脏腑,则病自愈矣。 
处方 生怀山药(一两) 生赭石(六钱轧细) 生怀地黄(一两) 生杭芍(五钱) 
天冬(五钱) 大甘枸杞(五钱) 川贝母(四钱) 生麦芽(三钱) 
牛蒡子(三钱捣碎) 射干(二钱) 广三七(三钱细末) 粉甘草(二钱细末) 
药共十二味,将前十味煎汤一大盅,送服三七、甘草末各一半,至煎渣再服,仍送服其余一半。 
效果 服药一剂,吐血即愈,诸病亦轻减。后即原方随时为之加 
减,连服三十余剂,其嗽始除根,身体亦渐壮健。 


<篇名>4.吐血兼咳嗽
属性:天津孙××,年二十八岁,得吐血兼咳嗽证。 
病因 因事心中着急起火,遂致吐血咳嗽。 
证候 其吐血之始,至今已二年矣。经医治愈,屡次反复,少有操劳,心中发热即复吐血。又 
频作咳嗽,嗽时吐痰亦恒带血。肋下恒作刺疼,嗽时其疼益甚,口中发干,身中亦间有 
灼热,大便干燥。其脉左部弦硬,右部弦长,皆重按不实,一息搏近五至。 
诊断 此证左脉弦硬者,阴分亏损而肝胆有热也,右部弦长者,因冲气上冲并致胃气上逆也。为 
其冲冲胃逆,是以胃壁血管破裂以至于吐血咳血也。其脉重按不实者,血亏而气亦亏 
也。至于口无津液,身或灼热,大便干燥,无非血少阴亏之现象。拟治以清肝、降胃、滋阴、化瘀之剂。 
处方 生赭石(八钱轧细) 生怀地黄(一两) 生怀山药(一两) 生杭芍(六钱) 
玄参(五钱) 川楝子(四钱捣碎) 生麦芽(三钱) 川贝母(三钱) 甘草(钱半) 广三七(二钱细末) 
药共十味,将前九味煎汤一大盅,送服三七末一半,至煎渣重服时,再送服其余一半。 
方解 愚治吐血,凡重用生地黄,必用三七辅之,因生地黄最善凉血,以治血热妄行,犹恐妄行 
之血因凉而凝,瘀塞于经络中也。三七善化瘀血,与生地黄并用,血止后自无他虞;且此证 
肋下作疼,原有瘀血,则三七尤在所必需也。 
复诊 将药连服三剂,吐血全愈,咳嗽吐痰亦不见血,肋疼亦愈强半,灼热已无,惟口中仍 
发干,脉仍有弦象。知其真阴犹亏也,拟再治以滋补真阴之剂。 
处方 生怀山药(一两) 生怀地黄(六钱) 大甘枸杞(六钱) 生杭芍(四钱) 
玄参(四钱) 生赭石(四钱轧细) 生麦芽(二钱) 甘草(二钱) 广三七(二钱细末) 
服法如前。 
效果 将药连服五剂,病全愈,脉亦复常,遂去三七,以熟地黄易生地黄,俾多服数剂以善其后。 


<篇名>5.吐血证
属性:天津冯××,年三十二岁,得吐血证久不愈。 
病因 因劳心劳力过度,遂得此证。 
证候 吐血已逾二年,治愈,屡次反复。病将发时,觉胃中气化不通,满闷发热,大便滞塞, 
旋即吐血,兼咳嗽多吐痰涎。其脉左部弦长,右部长而兼硬,一息五至。 
诊断 此证当系肝火挟冲胃之气上冲,血亦随之上逆,又兼失血久而阴分亏也。为其肝火炽盛, 
是以左脉弦长;为其肝火挟冲胃之气上冲,是以右脉长而兼硬;为其失血久而真阴亏损,是以其 
脉既弦硬(弦硬即有阴亏之象)而又兼数也。此宜治以泻肝降胃之剂,而以大滋真阴之药佐之。 
处方 生赭石(一两轧细) 玄参(八钱) 大生地(八钱) 生怀山药(六钱) 
栝蒌仁(六钱炒捣) 生杭芍(四钱) 龙胆草(三钱) 川贝母(三钱) 
甘草(钱半) 广三七(二钱细末) 
药共十味,先将前九味煎汤一大盅,送服三七细末一半,至煎渣重服时,再送服其余一半。 
效果 每日煎服一剂,初服后血即不吐,服至三剂咳嗽亦愈,大便顺利。再诊其脉,左右皆 
有和柔之象,问其心中闷热全无。遂去蒌仁、龙胆草,生山药改用一两,俾多服数剂,吐 
血之病可从此永远除根矣。 


<篇名>6.吐血证
属性:天津张姓,年过三旬,偶患吐血证。 
病因 其人性嗜酒,每日必饮,且不知节。初则饮酒过量即觉胸 
间烦热,后则不饮酒时亦觉烦热,遂至吐血。 
证候 其初吐血之时,原不甚剧,始则痰血相杂,因咳吐出。即或纯吐鲜血,亦不过一日数口, 
继复因延医服药,方中有柴胡三钱,服药半点钟后,遂大吐不止,仓猝迎愚往视。及至,则所吐之血 
已盈痰盂,又复连连呕吐,若不立为止住,实有危在目前之惧。幸所携药囊中有生赭石细末一包,俾先 
用温水送下五钱,其吐少缓须臾,又再送下五钱遂止住不吐。诊其脉弦而芤,数逾五至,其左寸 
摇摇有动意,问其心中觉怔忡乎?答曰∶怔忡殊甚,几若不能支持。 
诊断 此证初伤于酒,继伤于药,脏腑之血几于倾囊而出。犹幸速为立止,宜急服汤药以养 
其血,降其胃气保其心气,育其真阴,连服数剂,庶其血不至再吐。 
处方 生怀山药(一两) 生赭石(六钱轧细) 玄参(六钱) 生地黄(六钱) 
生龙骨(六钱捣碎) 生牡蛎(六钱捣碎) 生杭芍(五钱) 酸枣仁(四钱炒捣) 
柏子仁(四钱) 甘草(钱半) 广三七(三钱细末) 
此方将前十味煎汤,三七分两次用,头煎及二煎之汤送服。 
效果 每日服药一剂,连服三日血已不吐,心中不复怔忡。再诊 
其脉芤动皆无,至数仍略数,遂将生地黄易作熟地黄,俾再服数剂以善其后。 


<篇名>7.大便下血
属性:天津袁××,年三十二岁,得大便下血证。 
病因 先因劳心过度,心中时觉发热,继又因朋友宴会,饮酒过度遂得斯证。 
证候 自孟夏下血,历六月不止,每日六七次,腹中觉疼即须入 
厕,心中时或发热,懒于饮食。其脉浮而不实有似芤脉,而不若芤脉之硬,两尺沉分尤虚,至数微数。 
诊断 此证临便时腹疼者,肠中有溃烂处也。心中时或发热者,阴虚之热上浮也。其脉近芤者, 
失血过多也。其两尺尤虚者,下血久而阴亏,更兼下焦气化不固摄也。此宜用化腐生 
肌之药治其肠中溃烂,滋阴固气之药固其下焦气化,则大便下血可愈矣。 
处方 生怀山药(两半) 熟地黄(一两) 龙眼肉(一两) 净萸肉(六钱) 
樗白皮(五钱) 金银花(四钱) 赤石脂(四钱研细) 甘草(二钱) 
鸦胆子仁(八十粒成实者) 生硫黄(八分细末) 
药共十味,将前八味煎汤,送服鸦胆子、硫黄各一半,至煎 
渣再服时,仍送服其余一半,至于硫黄生用之理,详于敦复汤下。 
方解 方中鸦胆子、硫黄并用者,因鸦胆子善治下血,而此证之脉两尺过弱,又恐单用之失于 
寒凉,故少加硫黄辅之,况其肠中脂膜,因下血日久易至腐败酿毒,二药之性皆善消除毒 
菌也。又其腹疼下血,已历半载不愈,有似东人×××所谓阿米巴赤痢,硫黄实又为治阿米巴赤痢 
之要药也。 
复诊 前药连服三剂,下血已愈,心中亦不发热,脉不若从前之 
浮,至数如常。而其大便犹一日溏泻四五次,此宜投以健胃固肠之剂。 
处方 炙箭 (三钱) 炒白术(三钱) 生怀山药(一两) 龙眼肉(一两) 
生麦芽(三钱) 建神曲(三钱) 大云苓片(二钱) 
共煎汤一大盅温服。 
效果 将药连服五剂,大便已不溏泻,日下一次,遂停服汤药。 
俾用生怀山药细末煮作粥,调以白糖,当点心服之以善其后。 


<篇名>8.大便下血
属性:高××,年三十六岁,得大便下血证。 
病因 冷时出外办事,寝于寒凉屋中,床衾又甚寒凉遂得斯证。 
证候 每日下血数次,或全是血,或兼有大便,或多或少,其下时多在夜间,每觉腹中作疼,即 
须入厕,夜间恒苦不寐,其脉迟而芤,两尺尤不堪重按,病已二年余,服温补下元药则 
稍轻,然终不能除根,久之,则身体渐觉羸弱。 
诊断 此下焦虚寒太甚,其气化不能固摄而血下陷也。视其从前所服诸方,皆系草木之品,其 
质轻浮,温暖之力究难下达,当以矿质之品温暖兼收涩者投之。 
处方 生硫黄(半斤色纯黄者) 赤石脂(半斤纯系粉末者) 
将二味共轧细过罗,先空心服七八分,日服两次,品验渐渐加多,以服后移时微觉腹中温暖为度。 
效果 后服至每次二钱,腹中始觉温暖,血下亦渐少。服至旬 
余,身体渐壮,夜睡安然,可无入厕。服至月余,则病根祓除矣。 
方解 按硫黄之性,温暖下达,诚为温补下焦第一良药,而生用之尤佳,惟其性能润大便(本草谓 
其能使大便润、小便长,西医以为轻泻药药),于大便滑泻者不宜,故辅以赤石脂之粘腻收涩,自有益而无弊矣。 


<篇名>9.大便下血
属性:崔童,年十三岁,得大便下血证。 
病因 仲夏天热,赛球竞走,劳力过度,又兼受热,遂患大便下血。 
证候 每日大便,必然下血,便时腹中作疼,或轻或剧,若疼剧时,则血之下者必多,已年 
余矣。饮食减少,身体羸弱,面目黄白无血色,脉搏六至,左部弦而微硬,右部濡而无力。 
诊断 此证当因脾虚不能统血,是以其血下陷至其腹,所以作疼,其肠中必有损伤溃烂处也。当 
用药健补其脾胃,兼调养其肠中溃烂。 
处方 生怀山药(一两) 龙眼肉(一两) 金银花(四钱) 甘草(三钱) 广三 
七(二钱半轧细末) 鸦胆子(八十粒去皮拣其仁之成实者) 
共药六味,将前四味煎汤,送服三七、鸦胆子各一半,至煎渣再服时,仍送服其余一半。 
效果 将药如法服两次,下血病即除根矣。 


<篇名>10.大便下血
属性:阜城杜××,年四十五岁,得大便下血证。 
病因 因劳心过度,每大便时下血,服药治愈。因有事还籍,值夏季暑热过甚,又复劳心过度, 
旧证复发,屡治不愈。遂来津入西医院治疗,西医为其血在便后,谓系内痔,服药血仍 
不止,因转而求治于愚。 
证候 血随便下,且所下甚多,然不觉疼坠,心中发热懒食,其脉左部弦长,右部洪滑。 
诊断 此因劳心生内热而牵动肝经所寄相火,致肝不藏血而兼与溽暑之热相并,所以血妄行也。 
宜治以清心凉肝兼消暑热之剂,而少以培补脾胃之药佐之。 
处方 生怀地黄(一两) 白头翁(五钱) 龙眼肉(五钱) 生怀山药(五钱) 
知母(四钱) 秦皮(三钱) 黄柏(二钱) 龙胆草(二钱) 甘草(二钱) 
共煎汤一大盅,温服。 
复诊 上方煎服一剂,血已不见,服至两剂,少腹觉微凉。再诊其脉,弦长与洪滑之象皆减退, 
遂为开半清半补之方以善其后。 
处方 生怀山药(一两) 熟怀地黄(八钱) 净萸肉(五钱) 龙眼肉(五钱) 
白头翁(五钱) 秦皮(三钱) 生杭芍(三钱) 地骨皮(三钱) 
甘草(二钱) 
共煎汤一大盅,温服。 
效果 将药煎服一剂后,食欲顿开,腹已不疼,俾即原方多服数剂,下血病当可除根。 


<篇名>11.瘀血短气
属性:盐山刘××,年二十五岁,得瘀血短气证。 
病因 因出外修工,努力抬重物,当时觉胁下作疼,数日疼愈,仍觉胁下有物妨碍呼吸。 
证候 身形素强壮,自受病之后,迟延半载,渐渐羸弱,常觉右胁之下有物阻碍呼吸之气,与 
人言时恒半句而止,候至气上达再言,若偶忿怒则益甚,脉象近和平,惟稍弱不能条畅。 
诊断 此因努力太过,致肝经有不归经之血瘀经络之间,阻塞气息升降之道路也。喜其脉 
虽稍弱,犹能支持,可但用化瘀血之药,徐徐化其瘀结,气息自能调顺。 
处方 广三七(四两) 
轧为细末,每服钱半,用生麦芽三钱煎汤送下,日再服。 
方解 三七为止血妄行之圣药,又为化瘀血之圣药,且又化瘀血不伤新血,单服久服无碍,此 
乃药中特异之品,其妙处直不可令人思议。愚恒用以消积久之瘀血,皆能奏效。至麦芽原 
为消食之品,生煮服之则善舒肝气,且亦能化瘀者也。是以用之煎汤,以送服三七也。 
效果 服药四日后,自鼻孔中出紫血一条,呼吸较顺,继又服至药尽,遂脱然全愈。 
或问 人之呼吸在于肺,今谓肝经积有瘀血,即可妨碍呼吸,其 
义何居?答曰∶按生理之学,人之呼吸可达于冲任,方书又谓呼出心肺,吸入肝肾,若谓呼吸皆在 
于肺,是以上两说皆可废也。盖心、肺、肝,原一系相连,下又连于冲任,而心肺相连之系,其中 
原有两管,一为血脉管,一为回血管,血脉管下行,回血管上行。肺为发动呼吸之机关,非呼吸即限 
于肺也,是以吸入之气可由血脉管下达,呼出之气可由回血管上达,无论气之上达下达,皆从 
肝经过,是以血瘀肝经,即有妨于升降之气息也。据斯以论呼吸之关于肺者固多,而 
心肺相连之系亦司呼吸之分支也。 

\chapter{脑充血门}
<篇名>1.脑充血头疼
属性:京都谈××,年五十二岁,得脑充血头疼证。 
病因 因劳心过度,遂得脑充血头疼证。 
证候 脏腑之间恒觉有气上冲,头即作疼,甚或至于眩晕,其夜间头疼益甚,恒至疼不能寐。医治 
二年无效,浸至言语謇涩,肢体渐觉不利,饮食停滞胃口不下行,心中时常发热, 
大便干燥。其脉左右皆弦硬,关前有力,两尺重按不实。 
诊断 弦为肝脉,至弦硬有力无论见于何部,皆系有肝火过升之弊。因肝火过升,恒引动冲 
气胃气相并上升,是以其脏腑之间恒觉有气上冲也。人之血随气行,气上升不已,血即随之上升不已, 
以致脑中血管充血过甚,是以作疼。其夜间疼益剧者,因其脉上盛下虚,阴分原不充足,是以夜则 
加剧,其偶作眩晕亦职此也。至其心常发热,肝火炽其心火亦炽也。 
其饮食不下行,大便多干燥者,又皆因其冲气挟胃气上升, 
胃即不能传送饮食以速达于大肠也。其言语肢体蹇涩不利者,因脑中血管充血过甚,有妨碍于 
司运动之神经也。此宜治以镇肝、降胃、安冲之剂,而以引血下行兼清热滋阴之药辅之。又须知 
肝为将军之官,中藏相火,强镇之恒起其反动力,又宜兼用舒肝之药,将顺其性之作引也。 
处方 生赭石(一两轧细) 生怀地黄(一两) 怀牛膝(六钱) 大甘枸杞(六钱) 
生龙骨(六钱捣碎) 生牡蛎(六钱捣碎) 净萸肉(五钱) 生杭芍(五钱) 
茵陈(二钱) 甘草(二钱) 
共煎汤一大盅,温服。 
复诊 将药连服四剂,头疼已愈强半,夜间可睡四五点钟,诸病亦皆见愈,脉象之弦硬已减, 
两尺重诊有根,拟即原方略为加减俾再服之。 
处方 生赭石(一两轧细) 生怀地黄(一两) 生怀山药(八钱) 怀牛膝(六钱) 
生龙骨(六钱捣碎) 生牡蛎(六钱捣碎) 净萸肉(五钱) 生杭芍(五钱) 
生鸡内金(钱半黄色的捣) 茵陈(钱半) 甘草(二钱) 
共煎汤一大盅,温服。 
三诊 将药连服五剂,头已不疼,能彻夜安睡,诸病皆愈。惟办事,略觉操劳过度,头仍作 
疼,脉象犹微有弦硬之意,其心中仍间有觉热之时,拟再治以滋阴清热之剂。 
处方 生怀山药(一两) 生怀地黄(八钱) 玄参(四钱) 北沙参 
(四钱) 生杭芍(四钱) 净萸肉(四钱) 生珍珠母(四钱捣碎) 生石决明 
(四钱捣碎) 生赭石(四钱轧细) 怀牛膝(三钱) 生鸡内金(钱半黄色的捣) 甘草(二钱) 
共煎汤一大盅,温饮下。 
效果 将药连服六剂,至经理事务时,头亦不疼,脉象已和平如常。遂停服汤药,俾日用生山 
药细末,煮作茶汤调以白糖令适口,送服生赭石细末钱许,当点心服之以善其后。 


<篇名>2.脑充血头疼
属性:天津李氏妇,年过三旬,得脑充血头疼证。 
病因 禀性褊急,家务劳心,常起暗火,因得斯证。 
证候 其头疼或左或右,或左右皆疼,剧时至作呻吟。心中常常发热,时或烦躁,间有眩 
晕之时,其大便燥结非服通下药不行。其脉左右皆弦硬而长,重诊甚实,经中西医延医二年,毫无功效。 
诊断 其左脉弦硬而长者,肝胆之火上升也;其右脉弦硬而长者,胃气不降而逆行,又兼冲气上 
冲也。究之,左右脉皆弦硬,实亦阴分有亏损也。因其脏腑之气化有升无降,则血随气升者 
过多,遂至充塞于脑部,排挤其脑中之血管而作疼,此《内经》所谓血之与气,并走于上之厥证也。亦 
即西人所谓脑充血之证也。其大便燥结不行者,因胃气不降,失其传送之职也。其心中发烦 
躁者,因肝胃之火上升也。其头部间或眩晕者,因脑部充血过甚,有碍于神经也。此宜清其脏腑 
之热,滋其脏腑之阴,更降其脏腑之气,以引脑部所充之血下行,方能治愈。 
处方 生赭石(两半轧细) 怀牛膝(一两) 生怀山药(六钱) 生怀地黄(六钱) 
天冬(六钱) 玄参(五钱) 生杭芍(五钱) 生龙齿(五钱捣碎) 生石决明( 
五钱捣碎) 茵陈(钱半) 甘草(钱半) 
共煎汤一大盅,温服。 
方解 赭石能降胃平肝镇安冲气。其下行之力,又善通大便燥 
结而毫无开破之弊。方中重用两半者,因此证大便燥结过甚,非服药不能通下也。盖大便不通,是 
以胃气不下降,而肝火之上升冲气之上冲,又多因胃气不降而增剧。是治此证者,当以通其大便为 
要务,迨服药至大便自然通顺时,则病愈过半矣。牛膝为治腿疾要药,以其能引气血下行也。而 
《名医别录》及《千金翼方》,皆谓其除脑中痛,盖以其能引气血下行,即可轻减脑中之充血也。 
愚生平治此等证必此二药并用,而又皆重用之。用玄参、天冬、芍药者,取其既善退热兼能滋阴也。 
用龙齿、石决明者,以其皆为肝家之药,其性皆能敛戢肝火,镇熄肝风,以缓其上升之势也。用 
山药、甘草者,以二药皆善和胃,能调和金石之药与胃相宜,犹白虎汤用甘草粳米之义,而山药且 
善滋阴,甘草亦善缓肝也。用茵陈者,因肝为将军之官,其性刚果,且中寄相火,若但用药平之 
镇之,恒至起反动之力,茵陈最能将顺肝木之性,且又善泻肝热,李氏《本草纲目》谓善治头痛,是 
不但将顺肝木之性使不至反动,且又为清凉脑部之要药也。诸药汇集为方,久服之自有殊效。 
复诊 将药连服二十余剂(其中随时略有加减),头已不疼,惟夜失眠时则仍疼,心中发热、 
烦躁皆无,亦不复作眩晕,大便届时自行,无须再服通药,脉象较前和平而仍有弦硬之意,此宜注 
意滋其真阴以除病根。 
处方 生赭石(一两轧细) 怀牛膝(八钱) 生怀山药(八钱) 生怀地黄(八钱) 
玄参(六钱) 大甘枸杞(六钱) 净萸肉(五钱) 生杭芍(四钱) 
柏子仁(四钱) 生麦芽(三钱) 甘草(二钱) 
共煎汤一大盅,温服。方中用麦芽者,借以宣通诸药之滞腻也。且麦芽生用原善调和肝气, 
亦犹前方用茵陈之义也。 
效果 将药又连服二十余剂(亦随时略有加减),病遂全愈,脉象亦和平如常矣。 


<篇名>3.脑充血头疼
属性:天津于氏妇,年二十二岁,得脑充血头疼证。 
病因 其月信素日短少,不调,大便燥结,非服降药不下行,浸至脏腑气化有升无降,因成斯证。 
证候 头疼甚剧,恒至夜不能眠,心中常觉发热,偶动肝火即发眩晕,胃中饮食恒停滞不消, 
大便六七日不行,必须服通下药始行。其脉弦细有力而长,左右皆然,每分钟八十至,延 
医延医历久无效。 
诊断 此因阴分亏损,下焦气化不能固摄,冲气遂挟胃气上逆,而肝脏亦因阴分亏损水不滋 
木,致所寄之相火妄动,恒助肝气上冲。由斯脏腑之气化有升无降,而自心注脑之血为上升 
之气化所迫,遂至充塞于脑中血管而作疼作晕也。其饮食不消大便不行者,因冲胃之气皆逆也; 
其月信不调且短少者,因冲为血海,肝为冲任行气,脾胃又为生血之源,诸经皆失其常司,是以 
月信不调且少也;《内经》谓∶“血菀(同郁)于上,使人薄厥”,言为上升之气血逼薄而厥也。此 
证不急治则薄厥将成,宜急治以降胃、镇冲、平肝之剂,再以滋补真 
阴之药辅之,庶可转上升之气血下行不成薄厥也。 
处方 生赭石(一两轧细) 怀牛膝(一两) 生怀地黄(一两) 大甘枸杞(八钱) 
生怀山药(六钱) 生杭芍(五钱) 生龙齿(五钱捣碎) 生石决明(五钱捣碎) 
天冬(五钱) 生鸡内金(二钱黄色的捣) 苏子(二钱炒捣) 茵陈(钱半) 甘草(钱半) 
共煎汤一大盅,温服。 
复诊 将药连服四剂,诸病皆见轻,脉象亦稍见柔和。惟大便六 
日仍未通行,因思此证必先使其大便如常,则病始可愈,拟 
将赭石加重,再将余药略为加减以通其大便。 
处方 生赭石(两半轧细) 怀牛膝(一两) 天冬(一两) 黑芝麻(八钱炒捣) 
大甘枸杞(八钱) 生杭芍(五钱) 生龙齿(五钱捣碎) 生石决明(五钱捣碎) 
苏子(三钱炒捣) 生鸡内金(钱半黄色的捣) 甘草(钱半) 净柿霜(五钱) 
药共十二味,将前十一味煎汤一大盅,入柿霜融化温服。 
三诊 将药连服五剂,大便间日一行,诸证皆愈十之八九,月信适来,仍不甚多,脉象仍有 
弦硬之意,知其真阴犹未充足也。当即原方略为加减,再加滋阴生血之品。 
处方 生赭石(一两轧细) 怀牛膝(八钱) 大甘枸杞(八钱) 龙眼肉(六钱) 
生怀地黄(六钱) 当归(五钱) 玄参(四钱) 沙参(四钱) 
生怀山药(四钱) 生杭芍(四钱) 生鸡内金(一钱黄色的捣) 甘草(二钱) 生姜(三钱) 大枣(三枚掰开) 
共煎汤一大盅,温服。 
效果 将药连服四剂后,心中已分毫不觉热,脉象亦大见和平,大便日行一次,遂去方中玄 
参、沙参,生赭石改用八钱,生怀山药改用六钱,俾多服数剂以善其后。 


<篇名>4.脑充血兼腿痿弱
属性:天津崔××,年三十八岁,得脑充血兼两腿痿弱证。 
病因 出门采买木料,数日始归,劳心劳力过度,遂得斯证。 
证候 其初常觉头疼,时或眩晕,心中发热,饮食停滞,大便燥结,延医治疗无效。一日早 
起下床,觉痿弱无力,痿坐于地,人扶起坐床沿休息移时,自扶杖起立,犹可徐步,然时 
恐颠仆。其脉左部弦而甚硬,右部弦硬且长。 
诊断 其左脉弦硬者,肝气挟火上升也。右脉弦硬且长者,胃气上逆更兼冲气上冲也。因其脏腑间 
之气化有升无降,是以血随气升充塞于脑部作疼作眩晕。其脑部充血过甚,或自微细血管溢血于 
外,或隔血管之壁,些些渗血于外,其所出之血,若着于司运动之神经,其重者可使肢体痿废,其轻 
者亦可使肢体软弱无力。若此证之忽然痿坐于地者是也。至其心中之发热,饮食之停滞,大便之 
燥结,亦皆其气化有升无降之故,此宜平肝、清热、降胃、安冲,不使脏腑之气化过升, 
且导引其脑中过充之血使之下行,则诸证自愈矣。 
处方 生赭石(一两轧细) 怀牛膝(一两) 生怀地黄(一两) 生珍珠母(六钱捣碎) 
生石决明(六钱捣碎) 生杭芍(五钱) 当归(四钱) 龙胆草(二钱) 
茵陈(钱半) 甘草(钱半) 
共煎汤一大盅,温服。 
复诊 将药连服七剂,诸病皆大见愈,脉象亦大见缓和,惟其步履之间仍须用杖,未能复常,心中 
仍间有发热之时。拟即原方略为加减,再佐以通活血脉之品。 
处方 生赭石(一两轧细) 怀牛膝(一两) 生怀地黄(一两) 生杭芍(五钱) 
生珍珠母(四钱捣碎) 生石决明(四钱捣碎) 丹参(四钱) 生麦芽(三钱) 
土鳖虫(五个) 甘草(一钱) 
共煎汤一大盅温服。 
效果 将药连服八剂,步履复常,病遂全愈。 


<篇名>5.脑充血兼痰厥
属性:天津骆××,年四十九岁,得脑充血兼痰厥证。 
病因 平素常患头晕,间有疼时,久则精神渐似短少,言语渐形 
謇涩,一日外出会友,饮食过度,归家因事有拂意,怒动肝火,陡然昏厥。 
证候 闭目昏昏,呼之不应,喉间痰涎杜塞,气息微通。诊其脉左右皆弦硬而长,重 
按有力,知其证不但痰厥实素有脑充血病也。 
诊断 其平素头晕作疼,即脑充血之现证也。其司知觉之神经为脑充血所伤,是以精神短少。其 
司运动之神经为脑充血所伤,是以言语謇涩。又凡脑充血之人,其脏腑之气多上逆,胃气逆则 
饮食停积不能下行,肝气逆则痰火相并易于上干,此所以因饱食动怒而陡成痰厥也。此其危险即 
在目前,取药无及当先以手术治之。 
手术 治痰厥之手术,当以手指点其天突穴处(详见“治痰点天突穴法”),近八分钟许,即 
咳嗽呕吐。约吐出痰涎饮食三碗许,豁然顿醒,自言心中发热,头目胀疼,此当继治其脑部充血以求全 
愈。拟用建瓴汤方治之,因病脉之所宜而略为加减。 
处方 生赭石(一两轧细) 怀牛膝(一两) 生怀地黄(一两) 天花粉(六钱) 
生杭芍(六钱) 生龙骨(五钱捣碎) 生牡蛎(五钱捣碎) 生麦芽(三钱) 
茵陈(钱半) 甘草(钱半) 
磨取生铁锈浓水,以之煎药,煎汤一盅,温服下。 
复诊 将药服三剂,心中已不发热,头疼目胀皆愈,惟步履之时觉头重足轻,脚底如踏棉絮。 
其脉象较前和缓似有上盛下虚之象,爰即原方略为加减,再添滋补之品。 
处方 生赭石(一两轧细) 怀牛膝(一两) 生怀地黄(一两) 大甘枸杞(八钱) 
生杭芍(六钱) 净萸肉(六钱) 生龙骨(五钱捣碎) 生牡蛎(五钱捣碎) 
柏子仁(五钱炒捣) 茵陈(钱半) 甘草(钱半) 
磨取生铁锈浓水以之煎药,煎汤一大盅,温服。 
效果 将药连服五剂,病遂脱然全愈。将赭石、牛膝、地黄皆改用八钱,俾多服数剂以善其后。 


<篇名>6.脑充血兼偏枯
属性:天津孙××,年四十六岁,得脑充血证遂至偏枯。 
病因 禀性褊急,又兼处境不顺,恒触动肝火致得斯证。 
证候 未病之先恒觉头疼,时常眩晕。一日又遇事有拂意,遂忽然昏倒,移时醒后,左手足皆不 
能动,并其半身皆麻木,言语謇涩。延医服药十个月,手略能动,其五指则握而不伸,足可任地而不 
能行步,言语仍然謇涩,又服药数月病仍如故。诊其脉左右皆弦硬,右部似尤甚,知虽服药年余,脑 
充血之病犹未除也。问其心中发热乎?脑中有时觉疼乎?答曰∶心中有时觉有热上冲胃口,其热再 
上升则脑中可作疼,然不若病初得时脑疼之剧也。问其大便两三日一行,证脉相参,其 
脑中犹病充血无疑。 
诊断 按此证初得,不但脑充血实兼脑溢血也。其溢出之血,着于左边司运动之神经,则右半 
身痿废,着于右边司运动之神经,则左半身痿废,此乃交叉神经以互司其身之左右也。想其得病 
之初,脉象之弦硬,此时尤剧,是以头疼眩晕由充血之极而至于溢血,因溢血而至于残废也。即现时之 
证脉详参,其脑中溢血之病想早就愈,而脑充血之病根确未除也。宜注意治其脑充血,而以通活经 
络之药辅之。 
处方 生怀山药(一两) 生怀地黄(一两) 生赭石(八钱研细) 怀牛膝(八钱) 
生杭芍(六钱) 柏子仁(四钱炒捣) 白术(三钱炒) 滴乳香(三钱) 
明没药(三钱) 土鳖虫(四大个捣) 生鸡内金(钱半黄色的捣) 茵陈(一钱) 
共煎汤一大盅,温服。 
复诊 将药连服七剂,脑中已不作疼,心中间有微热之时,其左半身自觉肌肉松活,不若从前 
之麻木,言语之謇涩稍愈,大便较前通顺,脉之弦硬已愈十之七八,拟再注意治其左手足之痿废。 
处方 生箭 (五钱) 天花粉(八钱) 生赭石(六钱轧细) 怀牛膝(五钱) 
滴乳香(四钱) 明没药(四钱) 当归(三钱) 丝瓜络(三钱) 
土鳖虫(四大个捣) 地龙(二钱去土) 
共煎汤一大盅,温服。 
三诊 将药连服三十余剂(随时略有加喊),其左手之不伸者已能伸,左足之不能迈步者今已 
举足能行矣。病患问从此再多多服药可能撤消否?答曰∶此病若初得即治,服药四十余剂即能脱 
然,今已迟延年余,虽服数百剂亦不能保全愈,因关节经络之间瘀滞已久也。然再多服数十剂,仍 
可见愈,遂即原方略为加减,再设法以 动其神经补助其神经当更有效。 
处方 生箭 (六钱) 天花粉(八钱) 生赭石(六钱轧细) 怀牛膝(五钱) 
滴乳香(四钱) 明没药(四钱) 当归(三钱) 土鳖虫(四大个捣) 
地龙(二钱去土) 真鹿角胶(二钱轧细) 广三七(二钱轧细) 制马钱子末(三分) 
药共十二味,先将前九味共煎汤一大盅,送服后三味各一半,至煎渣再服时,仍送服其余一半。 
方解 方中用鹿角胶者,因其可为左半身引经,且其角为督脉所生,是以其性善补益脑髓以滋 
养脑髓神经也,用三七者,关节经络间积久之瘀滞,三七能融化之也。用制马钱子者,以 
其能 动神经使灵活也。 
效果 将药又连服三十余剂,手足之举动皆较前便利,言语之謇涩亦大见愈,可勉 
强出门作事矣。遂俾停服汤药,日用生怀 
山药细末煮作茶汤,调以白糖令适口,送服黄色生鸡内金细末三分许。当点心用之以善其后。此欲用山 
药以补益气血,少加鸡内金以化瘀滞也。 
帮助 按脑充血证,最忌用黄 ,因黄 之性补而兼升,气升则血必随之上升,致脑中之血充而 
益充,排挤脑中血管可至溢血,甚或至破裂而出血,不可救药者多矣。至将其脑充血之病治愈,而 
肢体之痿废仍不愈者,皆因其经络瘀塞血脉不能流通也。此时欲化其瘀塞,通其血脉,正不妨以黄 辅之, 
特是其脑中素有充血之病,终嫌黄 升补之性能助血上升,故方中仍加生赭石、牛膝,以防血之上 
升,即所以监制黄 也。又虑黄 性温,温而且补即能生热,故又重用花粉以调剂之也。 

\chapter{肠胃病门}
<篇名>1.噎膈
属性:天津盛××,年五旬,得噎膈证。 
病因 处境恒多不顺,且又秉性褊急,易动肝火,遂得斯证。 
证候 得病之初期,觉饮食有不顺时,后则常常如此,始延医为调治,服药半年,更医十余人 
皆无效验。转觉病势增剧,自以为病在不治,已停药不服矣。适其友人何××劝其求愚为之延医, 
其六脉细微无力,强食饼干少许,必嚼成稀糜方能下咽,咽时偶觉龃龉即作呕吐,带出痰涎若干。惟饮 
粳米所煮稠汤尚无阻碍,其大便燥结如羊矢,不易下行。 
诊断 杨素园谓∶“此病与失血异证同源,血之来也暴,将胃壁之 
膜冲开则为吐血;其来也缓,不能冲开胃膜,遂瘀于上脘之处, 
致食管窄隘即成噎膈。”至西人则名为胃癌,所谓癌者,如山石之有岩,其形凸出也。此与杨 
氏之说正相符合,其为瘀血致病无疑也。其脉象甚弱者,为其进食甚少气血两亏也。至其便结如羊 
矢,亦因其饮食甚少,兼胃气虚弱不输送下行之故也。此宜化其瘀血兼引其血下行,而更辅以培养气血之品。 
处方 生赭石(一两轧细) 野台参(五钱) 生怀山药(六钱) 天花粉(六钱) 
天冬(四钱) 桃仁(三钱去皮捣) 红花(二钱) 土鳖虫(五枚捣碎) 广三七(二钱捣细) 
药共九味,将前八味煎汤一大盅,送服三七末一半,至煎渣再服时,再送服其余一半。 
方解 方中之义,桃仁、红花、土鳖虫、三七诸药,所以消其瘀血也。重用生赭石至一两,所 
以引其血下行也。用台参、山药者,所以培养胃中之气化,不使因服开破之药而有伤损也。用天冬 
、天花粉者,恐其胃液枯槁,所瘀之血将益干结,故借其凉润之力以滋胃液,且即以防台参之因补生热也。 
效果 将药服至两剂后,即可进食,服至五剂,大便如常。因将赭石改用八钱,又服数剂,饮 
食加多,仍觉胃口似有阻碍不能脱然。俾将三七加倍为四钱,仍分两次服下,连进四剂, 
自大便泻下脓血若干,病遂全愈。 
帮助 按噎膈之证,有因痰饮而成者,其胃口之间生有痰囊(即喻氏《寓意草》中所谓窠囊),本方 
去土鳖虫、三七,加清半夏四钱,数剂可愈。有因胃上脘枯槁痿缩致成噎膈者,本方去土鳖虫、三七,将 
赭石改为八钱,再加当归、龙眼肉、枸杞子各五钱,多服可愈。有因胃上脘生瘤赘以致成噎膈者( 
“论胃病噎膈治法及反胃治法”中曾详论),然此证甚少,较他种噎膈亦甚难治,盖瘤赘之生,恒 
有在胃之下脘成反胃者,至生于胃之上脘成噎膈者,则百中无一二也。 


<篇名>2.反胃吐食
属性:天津陈××,年五十六岁,得反胃吐食证,半年不愈。 
病因 初因夏日多食瓜果致伤脾胃,廉于饮食,后又因处境不顺心多抑郁,致成反胃之证。 
证候 食后消化力甚弱,停滞胃中不下行,渐觉恶心,久之,则觉有气自下上冲,即将饮食 
吐出。屡经医诊视,服暖胃降气之药稍愈,仍然反复,迁延已年余矣。身体羸弱,脉弦长, 
按之不实,左右皆然。 
诊断 此证之饮食不能消化,固由于脾胃虚寒,然脾胃虚寒者,食后恒易作泄泻,此则食不 
下行而作呕吐者,因其有冲气上冲,并迫其胃气上逆也。当以温补脾胃之药为主,而以降胃 
镇冲之药辅之。 
处方 生怀山药(一两) 白术(三钱炒) 干姜(三钱) 生鸡内金(三钱黄色的捣) 
生赭石(六钱轧细) 炙甘草(二钱) 
共煎汤一大盅,温服。 
效果 将药煎服后,觉饮食下行不复呕吐,翌日头午,大便下两次,再诊其脉不若从前之 
弦长,知其下元气化不固,不任赭石之镇降也。遂去赭石加赤石脂五钱(用头煎和次煎之汤,分两次送服 
)、苏子二钱,日煎服一剂,连服十剂霍然全愈。盖赤石脂为末送服, 
可代赭石以降胃镇冲,而又有固涩下焦之力,故服后不复滑泻也。 


<篇名>3.胃脘疼闷
属性:天津徐氏妇,年近三旬,得胃脘疼闷证。 
病因 本南方人,久居北方,远怀乡里,归宁不得,常起忧思, 
因得斯证。 
证候 中焦气化凝郁,饮食停滞艰于下行,时欲呃逆,又苦不能上达,甚则蓄极绵绵作疼。 
其初病时,惟觉气分不舒,服药治疗三年,病益加剧,且身形亦渐羸弱,呼吸短气,口无津液,时 
常作渴,大便时常干燥,其脉左右皆弦细,右脉又兼有牢意。 
诊断 《内经》谓脾主思,此证乃过思伤脾以致脾不升胃不降也。为其脾气不上升,是以口无 
津液,呃逆不能上达;为其胃气不降,是以饮食停滞,大便干燥。治之者当调养其脾胃,俾 
还其脾升胃降之常,则中焦气化舒畅,疼胀自愈,饮食加多而诸病自除矣。 
处方 生怀山药(一两) 大甘枸杞(八钱) 生箭 (三钱) 
生鸡内金(三钱黄色的捣) 生麦芽(三钱) 玄参(三钱) 天花粉(三钱) 
天冬(三钱) 生杭芍(二钱) 桂枝尖(钱半) 生姜(三钱) 
大枣(三枚掰开) 
共煎汤一大盅,温服。 
方解 此方以山药、枸杞、黄 、姜、枣培养中焦气化,以麦芽升脾(麦芽生用善升),以鸡 
内金降胃(鸡内金生用善降),以桂枝升脾兼以降胃(气之当升者遇之则升,气之当降者遇之则降),又 
用玄参,花粉诸药,以调剂姜、桂、黄 之温热,则药性归于和平,可以久服无弊。 
复诊 将药连服五剂,诸病皆大轻减,而胃疼仍未脱然,右脉仍有牢意。度其疼处当有 
瘀血凝滞,拟再于升降气化药中加消瘀血之品。 
处方 生怀山药(一两) 大甘枸杞(八钱) 生箭 (三钱) 玄参(三钱) 
天花粉(三钱) 生麦芽(三钱) 生鸡内金(二钱黄色的捣) 
生杭芍(二钱) 桃仁(二钱去皮炒捣) 广三七(二钱轧细) 
药共十味,将前九味煎汤一大盅,送服三七末一半,至煎渣再服时,仍送服其余一半。 
效果 将药连服四剂,胃中安然不疼,诸病皆愈,身形渐强壮。脉象已如常人,将原方再服数剂以善其后。 


<篇名>4.冷积腹疼
属性:大城王××,年五十岁,少腹冷疼,久服药不愈。 
病因 自幼在家惯睡火炕,后在津栖处寒凉,饮食又多不慎,遂得此证。 
证候 其少腹时觉下坠,眠时须以暖水袋熨脐下,不然则疼不能寐。若屡服热药,上焦即觉 
烦躁,已历二年不愈。脉象沉弦,左右皆然,至数稍迟。 
诊断 即其两尺沉弦凉而且坠论之,知其肠中当有冷积,此宜用温通之药下之。 
处方 与以自制通彻丸(系用牵牛头末和水为丸如秫米粒大)三钱,俾于清晨空心服下。 
效果 阅三点钟,腹中疼似加剧,须臾下如绿豆糊所熬凉粉者若干。疼坠脱然全愈,亦不觉凉。 
继为开温通化滞之方,俾再服数剂以善其后。 


<篇名>5.肠结腹疼
属性:天津李××,年二十五岁,于仲春得腹结作疼证。 
病因 偶因恼怒触动肝气,遂即饮食停肠中,结而不下作疼。 
证候 食结肠中,时时切疼,二十余日大便不通。始犹少进饮食,继则食不能进,饮水一口亦 
吐出。延医服药,无论何药下咽亦皆吐出,其脉左右皆微弱,犹幸至数照常,按之犹有根柢,知犹可救。 
疗法 治此等证,必止呕之药与开结之药并用,方能直达病所,又必须内外兼治,则久停之结庶可下行。 
处方 用硝菔通结汤,送服生赭石细末,汤分三次服下(每五十分钟服一次),共送服赭石末两半,外 
又用葱白四斤切丝,醋炒至极热,将热布包熨患处,凉则易之。又俾用净萸肉二两,煮 
汤一盅,结开下后饮之,以防虚脱。 
效果 自晚八点钟服,至夜半时将药服完,炒葱外熨,至翌日早八点钟下燥粪二十枚,后继以溏 
便。知其下净,遂将萸肉汤饮下,安然全愈。若虚甚者,结开欲大便时,宜先将萸肉汤服下。 


<篇名>6.肠结腹疼兼外感实热
属性:沈阳张姓媪,年过六旬,肠结腹疼,兼心中发热。 
病因 素有肝气病,因怒肝气发动,恒至大便不通,必服泻药始 
通下。此次旧病复发而呕吐不能受药,是以病久不愈。 
证候 胃下脐上似有实积,常常作疼,按之则疼益甚,表里俱觉发热,恶心呕吐。连次 
延医服药,下咽须臾即吐出,大便不行已过旬日,水浆不入者七八日矣。脉搏五至,左右脉象皆 
弱,独右关重按似有力,舌有黄苔,中心近黑,因问其得病之初 
曾发冷否?答云∶旬日前曾发冷两日,至三日即变为热矣。 
诊断 即此证脉论之,其阳明胃腑当蕴有外感实热,是以表里俱热,因其肠结不通,胃气不能 
下行,遂转而上行与热相并作呕吐。治此证之法,当用镇降之药止其呕,咸润之药开其结, 
又当辅以补益之品,俾其呕止、结开,而正气无伤始克有济。 
处方 生石膏(一两轧细) 生赭石(一两轧细) 玄参(一两) 潞参(四钱) 
芒硝(四钱) 生麦芽(二钱) 茵陈(二钱) 
共煎汤一大盅,温服。 
效果 煎服一剂,呕止结开,大便通下燥粪若干,表里热皆轻减,可进饮食。诊其脉仍有余热未 
净,再为开滋阴清热之方,俾服数剂以善其后。 

\chapter{头部病门}
<篇名>1.头疼
属性:天津李姓,得头疼证,日久不愈。 
病因 其人素羸弱,因商务操劳遇事又多不顺,心肝之火常常妄动,遂致头疼。 
证候 头疼不起床者已逾两月,每日头午犹轻,过午则浸加重,夜间疼不能寐,鸡鸣后疼又 
渐轻可以少睡,心中时或觉热,饮食懒进。脉搏五至,左部弦长,关脉犹弦而兼硬,右脉则稍和平。 
诊断 即此脉象论之,显系肝胆之热上冲脑部作疼也。宜用药清肝火、养肝阴、镇肝逆,且兼用 
升清降浊之药理其脑部。 
处方 生杭芍(八钱) 柏子仁(六钱) 玄参(六钱) 生龟板(六钱轧细) 
龙胆草(三钱) 川芎(钱半) 甘菊花(一钱) 甘草(三钱) 
共煎汤一大盅,温服。 
效果 服药一剂,病愈十之七八,脉象亦较前和平,遂将龙胆草减去一钱,又服两剂全愈。 
或问 川芎为升提气分之品,今其头疼既因肝胆之热上冲,复用川芎以升提之,其热不益上冲乎? 
何以服之有效也?答曰∶川芎升清气者也,清气即轻气也。按化学之理,无论何种 
气,若在轻气之中必然下降,人之脏腑原有轻气,川芎能升轻气上至脑中,则脑中热浊之气自然 
下降,是以其疼可愈也。 


<篇名>2.目病干疼
属性:天津崔××,年三十四岁,患眼干,间有时作疼。 
病因 向因外感之热传入阳明之府,服药多甘寒之品,致外感之 
邪未净,痼闭胃中永不消散,其热上冲遂发为眼疾。 
证候 两目干涩,有时目睛胀疼,渐至视物昏花,心中时常发热,二便皆不通顺,其脉左右皆 
有力,而右关重按有洪实之象,屡次服药已近二年,仍不少愈。 
诊断 凡外感之热传里,最忌但用甘寒滞泥之药,痼闭其外感之邪不能尽去,是以陆九芝谓如 
此治法,其病当时虽愈,后恒变成痨瘵。此证因其禀赋强壮,是以未变痨瘵而发为眼疾,医者 
不知清其外感之余热,而泛以治眼疾之药治之,是以历久不愈也。愚有自制离中丹,再佐以清热托表 
之品,以引久蕴之邪热外出,眼疾当愈。 
处方 离中丹(一两) 鲜芦根(五钱) 鲜茅根(五钱) 
药共三味,将后二味煎汤三杯,分三次温服,每次服离中丹三钱强,为一日之量,若二种 
鲜根但有一种者,可倍作一两用之。 
效果 将药如法服之,至第三日因心中不发热,将离中丹减半,又服数日眼之干涩疼胀皆愈,二便亦顺利。 


<篇名>3.牙疼
属性:天津王姓,年三十余,得牙疼病。 
病因 商务劳心,又兼连日与友宴饮,遂得斯证。 
证候 其牙疼甚剧,有碍饮食,夜不能寐,服一切治牙疼之药不效,已迁延二十余日矣。 
其脉左部如常,而右部弦长,按之有力。 
诊断 此阳明胃气不降也。上牙龈属足阳明胃,下牙龈属手阳明大肠。究之,胃气不降肠中之气 
亦必不降,火随气升,血亦因之随气上升并于牙龈而作疼,是以牙疼者牙龈之肉多肿 
热也。宜降其胃气兼引其上逆之血下行,更以清热之药辅之。 
处方 生赭石(一两轧细) 怀牛膝(一两) 滑石(六钱) 甘草(一钱) 
煎汤服。 
效果 将药煎服一剂,牙疼立愈,俾按原方再服一剂以善其后。 
帮助 方书治牙疼未见有用赭石牛膝者,因愚曾病牙疼以二药治愈,后凡遇胃气不降致牙疼者, 
方中必用此二药。其阳明胃腑有实热者,又恒加生石膏数钱。 

\chapter{肢体疼痛门}
<篇名>1.胁疼
属性:天津陈××,年六旬,得胁下作疼证。 
病因 因操劳过度,遂得胁下作疼病。 
证候 其疼或在左胁或在右胁或有时两胁皆疼,医者治以平肝、舒肝、柔肝之法皆不效。迁延年 
余,病势浸增,疼剧之时。觉精神昏愦。其脉左部微细,按之即无,右脉似近和平,其 
搏动之力略失于弱。 
诊断 人之肝居胁下,其性属木,原喜条达,此因肝气虚弱不能条达,故郁于胁下作疼也。其 
疼或在左或在右者,《难经》云,肝之为脏其治在左,其藏在右胁右肾之前并胃,着于胃 
之第九椎(《医宗金鉴》刺灸篇曾引此数语,今本《难经》不知被何人删去)。所谓脏者,肝脏所 
居之地也,谓治者肝气所行之地也。是知肝虽居右而其气化 
实先行于左。其疼在左者,肝气郁于所行之地也;其疼在右者,肝气郁于所居之地也;其疼剧时 
精神昏愦者,因肝经之病原与神经有涉也(肝主筋,脑髓神经为灰白色之筋,是以肝经之病与神经有涉)。治 
此证者,当以补助肝气为主。而以升肝化郁之药辅之。 
处方 生箭 (五钱) 生杭芍(四钱) 玄参(四钱) 滴乳香(三钱炒) 
明没药(三钱不炒) 生麦芽(三钱) 当归(三钱)川芎(二钱) 甘草(钱半) 
共煎汤一大盅,温服。 
方解 方书有谓肝虚无补法者,此非见道之言也。黄 为补肝之主药,何则?黄 之性温而能升, 
而脏腑之中秉温升之性者肝木也,是以各脏腑气虚,黄 皆能补之。而以补肝经之气虚,实更有同 
气相求之妙,是以方中用之为主药。然因其性颇温,重用之虽善补肝气,恐并能助肝火,故以芍药、玄 
参之滋阴凉润者济之。用乳香、没药者以之融化肝气之郁也。用麦芽、芎 者以之升达肝气之郁也。 
究之,无论融化升达,皆通行其经络使之通则不痛也。用当归者以肝为藏血之脏,既补其气,又欲 
补其血也。且当归味甘多液,固善生血,而性温味又兼辛,实又能调和气分也。用甘草者以其能 
缓肝之急,而甘草与芍药并用,原又善治腹疼,当亦可善治胁疼也。 
再诊 将药连服四剂,胁疼已愈强半,偶有疼时亦不甚剧。脉象左部重按有根,右部亦较前有力, 
惟从前因胁疼食量减少,至此仍未增加,拟即原方再加健胃消食之品。 
处方 生箭 (四钱) 生杭芍(四钱) 玄参(四钱) 于白术(三钱) 
滴乳香(三钱炒) 明没药(三钱不炒) 生麦芽(三钱) 当归(三钱) 
生鸡内金(二钱黄色的捣) 川芎(二钱) 甘草(钱半) 
共煎汤一大盅,温服。 
三诊 将药连服四剂,胁下已不作疼,饮食亦较前增加,脉象左右皆调和无病,惟自觉两腿筋 
骨软弱,此因病久使然也。拟再治以舒肝、健胃、强壮筋骨之剂。 
处方 生箭 (四钱) 生怀山药(四钱) 天花粉(四钱) 胡桃仁(四钱) 
于白术(三钱) 生明没药(三钱) 当归(三钱) 生麦芽(三钱) 
寸麦冬(三钱) 生鸡内金(二钱黄色的捣) 真鹿角胶(三钱) 
药共十一味,将前十味煎汤一大盅,再将鹿角胶另用水炖化和匀,温服。 
效果 将药连服十剂,身体浸觉健壮,遂停服汤药,俾用生怀山药细末七八钱,或至一两,凉水 
调和煮作茶汤,调以蔗糖令其适口,当点心服之。服后再嚼服熟胡桃仁二三钱,如此调养,宿病可以永愈。 


<篇名>2.胁下疼兼胃口疼
属性:天津齐××,年五旬,得胁下作疼,兼胃口疼病。 
病因 素有肝气不顺病,继因设买卖赔累,激动肝气,遂致胁下作疼,久之胃口亦疼。 
证候 其初次觉疼恒在申酉时,且不至每日疼,后浸至每日觉疼,又浸至无时不疼。屡次延 
医服药,过用开破之品伤及脾胃,饮食不能消化,至疼剧时恒连胃中亦疼。其脉左部沉弦 
微硬,右部则弦而无力,一息近五至。 
诊断 其左脉弦硬而沉者,肝经血虚火盛而肝气又郁结也。其右脉弦而无力者,土为木伤, 
脾胃失其蠕动健运也。其胁疼之起点在申酉时者,因肝属木申酉属金,木遇金时其气化益遏 
抑不舒也。《内经》谓,“厥阴不治,求之阳明。”夫厥阴为 
肝,阳明为胃,遵《内经》之微旨以治此证,果能健补脾胃,俾中焦之气化营运无滞,再少 
佐以理肝之品,则胃疼可愈,而胁下之疼亦即随之而愈矣。 
处方 生怀山药(一两) 大甘枸杞(六钱) 玄参(五钱) 寸麦冬(四钱带心) 
于白术(三钱) 生杭芍(三钱) 生麦芽(三钱) 桂枝尖(二钱) 
龙胆草(二钱) 生鸡内金(二钱黄色的捣) 浓朴(钱半) 甘草(钱半) 
共煎汤一大盅,温服。 
复诊 将药连服四剂,胃中已不作疼,胁下之疼亦大轻减,且不至每日作疼,即有疼时亦 
须臾自愈。脉象亦见和缓,遂即原方略为加减俾再服之。 
处方 生怀山药(一两) 大甘枸杞(六钱) 玄参(四钱) 寸麦冬(四钱带心) 
于白术(三钱) 生杭芍(三钱) 当归(三钱) 桂枝尖(二钱) 龙胆 
草(二钱) 生鸡内金(二钱黄色的捣) 醋香附(钱半) 甘草(钱半) 生姜(二钱) 
共煎汤一大盅,温服。 
效果 将药连服五剂,胁下之疼霍然全愈,肝脉亦和平如常矣。遂停服汤药,俾日用生怀山药 
细末两许,水调煮作茶汤,调以蔗糖令适口,以之送服生鸡内金细末二分许,以善其后。 
或问 理肝之药莫如柴胡,其善舒肝气之郁结也。今治胁疼两方中皆用桂枝而不用柴胡,将 
毋另有取义?答曰∶桂枝与柴胡虽皆善理肝,而其性实有不同之处。如此证之疼肇于胁下, 
是肝气郁结而不舒畅也,继之因胁疼累及胃中亦疼,是又肝木之横恣而其所能胜也。柴胡能 
舒肝气之郁,而不能平肝木之横恣,桂枝其气温升(温升为木气),能舒肝气之郁结则胁疼可 
愈,其味辛辣(辛辣为金味),更能平肝木横恣则胃疼亦可愈也。 
惟其性偏于温,与肝血虚损有热者不宜,故特加龙胆草以调剂之,俾其性归和平而后用之,有益 
无损也。不但此也,拙拟两方之要旨,不外升肝降胃,而桂枝之妙用,不但为升肝要药,实又为 
降胃要药。《金匮》桂枝加桂汤,治肾邪奔豚上干直透中焦,而方中以桂枝为主药,是其能降胃之 
明征也。再上溯《神农本草经》,谓桂枝主上气咳逆及吐吸(吸不归根即吐出,即后世所谓喘也),是 
桂枝原善降肺气,然必胃气息息下行,肺气始能下达无碍。细绎经旨,则桂枝降胃之功用,更可借 
善治上气咳逆吐吸而益显也。盖肝升胃降,原人身气化升降之常,顺人身自然之气化而调养之, 
则有病者自然无病,此两方之中所以不用柴胡皆用桂枝也。 


<篇名>3.胁疼
属性:邻村李姓妇,年近四旬,得胁下疼证。 
病因 平素肝气不舒,继因暴怒,胁下陡然作疼。 
证候 两胁下掀疼甚剧,呻吟不止,其左胁之疼尤甚,倩人以手按之则其疼稍愈,心中时 
觉发热,恶心欲作呕吐,脉左右两部皆弦硬。 
诊断 此肝气胆火相助横恣,欲上升而不能透膈,郁于胁下而作疼也。当平其肝气泻其胆火,其疼自愈。 
处方 川楝子(八钱捣碎) 生杭芍(四钱) 生明没药(四钱) 生麦芽(三钱) 
三棱(三钱) 莪术(三钱) 茵陈(二钱) 龙胆草(二钱) 连翘(三钱) 
磨取生铁锈浓水,煎药取汤一大盅,温服。 
方解 方中川楝、芍药、龙胆,引气火下降者也。茵陈、生麦芽,引气火上散者也。三棱、莪 
术,开气火之凝结,连翘、没药,消气火之弥漫,用铁锈水煎药者,借金之余气,以镇 
肝胆之木也。 
效果 煎服一剂后其疼顿止,而仍觉气分不舒,遂将川楝、三棱、莪术各减半,再加柴胡 
二钱,一剂全愈。 


<篇名>4.腰疼
属性:天津李××,年三十四岁,得腰疼证。 
病因 劳心过度,数日懒食,又勉强远出操办要务,因得斯证。 
证候 其疼剧时不能动转,轻时则似疼非疼绵绵不已,亦恒数日不疼,或动气或劳力时则疼 
剧。心中非常发闷,其脉左部沉弦,右部沉牢,一息四至强。观其从前所服之方,虽不一 
致,大抵不外补肝肾强筋骨诸药,间有杂似祛风药者,自谓得病之初,至今已三年,服 
药数百剂,其疼卒未轻减。 
诊断 《内经》谓通则不痛,此证乃痛则不通也。肝肾果系虚弱,其脉必细数,今左部沉弦, 
右部沉牢,其为腰际关节经络有瘀而不通之气无疑,拟治以利关节通经络之剂。 
处方 生怀山药(一两) 大甘枸杞(八钱) 当归(四钱) 丹参(四钱) 生明没 
药(四钱) 生五灵脂(四钱) 穿山甲(二钱炒捣) 桃仁(二钱去皮捣碎) 红花(钱半) 土鳖 
虫(五枚捣碎) 广三七(二钱轧细) 
药共十一味,先将前十味煎汤一大盅,送服三七细末一半,至煎渣重服时,再送其余一半。 
效果 将药连服三剂腰已不疼,心中亦不发闷,脉象虽有起色,仍未复常,遂即原方去山甲加川续 
断、生杭芍各三钱,连服数剂,脉已复常,自此病遂除根。 
帮助 医者治病不可预有成见,临证时不复细审病因。方书谓腰者肾之府,腰疼则肾脏衰惫,又 
谓肝主筋肾主骨,腰疼为筋骨之病,是以肝肾主之。治腰疼者因先有此等说存于胸中, 
恒多用补肝肾之品。究之,此在由于肝肾虚者甚少,由于气血瘀者颇多,若因努力任重而腰疼 
者尤多瘀证。曾治一人因担重物后腰疼,为用三七、土鳖虫等分共为细末,每服二钱,日两次,服三 
日全愈。又一人因抬物用力过度,腰疼半年不愈,忽于疼处发出一疮,在脊梁之旁,微似红肿,状若 
复盂,大径七寸。疡医以为腰疼半年始发现此疮,其根蒂必深,不敢保好,转求愚为治疗,调治两 
旬始愈(详案载内托生肌散后)。然使当腰初觉疼之时,亦服三七、土鳖以开其瘀,又何至有 
后时之危险乎。又尝治一妇,每当行经之时腰疼殊甚,诊其脉气分甚虚,于四物汤中加黄 八钱, 
服数剂而疼愈,又一妇腰疼绵绵不止,亦不甚剧,诊其脉知其下焦虚寒,治以温补下焦之药,又于 
服汤药之外,俾服生硫黄细末一钱,日两次,硫黄服尽四两,其疼除根。是知同是腰疼而其致病之因 
各异,治之者安可胶柱鼓瑟哉。 


<篇名>5.腿疼
属性:邻村窦××,年过三旬,于孟冬得腿疼证。 
病因 禀赋素弱,下焦常畏寒凉,一日因出门寝于寒凉屋中,且铺盖甚薄,晨起遂病腿疼。 
证候 初疼时犹不甚剧,数延医服药无效,后因食猪头肉其疼陡然加剧,两腿不能任地,夜则 
疼不能寐,其脉左右皆弦细无力,两尺尤甚,至数稍迟。 
诊断 此证因下焦相火虚衰,是以易为寒侵,而细审其脉,实更兼气虚不能充体,即不能达于 
四肢以运化药力,是以所服之药纵对证亦不易见效也。此当助其相火祛其外寒,而更加补益 
气分之药,使气分壮旺自能营运药力以胜病也。 
处方 野党参(六钱) 当归(五钱) 怀牛膝(五钱) 胡桃仁(五钱) 
乌附子(四钱) 补骨脂(三钱炒捣) 滴乳香(三钱炒) 明没药(三钱不炒) 威灵仙(钱半) 
共煎汤一大盅,温服。 
复诊 将药连服五剂,腿之疼稍觉轻而仍不能任地,脉象较前似稍有力。问其心中服此热药多 
剂后仍不觉热,因思其疼在于两腿,当用性热质重之品,方能引诸药之力下行以达病所。 
处方 野党参(五钱) 怀牛膝(五钱) 胡桃仁(五钱) 乌附子(四钱) 
白术(三钱炒) 补骨脂(三钱炒捣) 滴乳香(三钱炒) 明没药(三钱不炒) 生硫黄(一钱研细) 
药共九味,将前八味煎汤一大盅,送服硫黄末五分,至煎渣再服时,又送服所余五分。 
效果 将药连服八剂,腿疼大见轻减,可扶杖行步,脉象已调和无病,心中微觉发热,俾停服 
汤药,每日用生怀山药细末七八钱许,煮作茶汤,送服青娥丸三钱,或一次或两次皆可, 
后服至月余,两腿分毫不疼,步履如常人矣。 
或问 猪肉原为寻常服食之物,何以因食猪头肉而腿疼加剧乎? 
答曰∶猪肉原有苦寒有毒之说,曾见于各家本草。究之,其肉非苦寒,亦非有毒,而猪头之肉实 
具有咸寒开破之性,是以善通大便燥结,其咸寒与开破皆与腿之虚寒作疼者不宜也,此所以 
食猪头肉后而腿之疼加剧也。 

\chapter{肿胀门}
<篇名>1.受风水肿
属性:邑北境刘氏妇,年过三旬,因受风得水肿证。 
病因 时当孟夏,农家忙甚,将饭炊熟,复自 田间,因作饭时受热出汗,出门时途间受风, 
此后即得水肿证。 
证候 腹中胀甚,头面周身皆肿,两目之肿不能开视,心中发热,周身汗闭不出,大便干燥,小 
便短赤。其两腕肿甚不能诊脉,按之移时,水气四开,始能见脉。其左部弦而兼硬, 
右部滑而颇实,一息近五至。 
诊断 《金匮》辨水证之脉,谓风水脉浮,此证脉之部位肿甚,原无从辨其脉之浮沉,然即其 
自述,谓于有汗受风之后,其为风水无疑也。其左脉弦硬者,肝胆有郁热也,其右脉滑而 
实者,外为风束胃中亦浸生热也。至于大便干燥,小便短赤,皆肝胃有热之所致也。当 
用《金匮》越婢汤加减治之。 
处方 生石膏(一两捣细) 滑石(四钱) 生杭芍(四钱) 麻黄(三钱) 
甘草(二钱) 大枣(四枚掰开) 生姜(二钱) 西药阿斯匹林(一瓦) 
中药七味,共煎汤一大盅,当煎汤将成之时,先用白糖水将西药阿斯匹林送下,候周身 
出汗(若不出汗仍可再服一瓦),将所煎之汤药温服下,其汗出必益多,其小盒饭利,肿即可消矣。 
复诊 如法将药服完,果周身皆得透汗,心中已不发热,小便遂利,腹胀身肿皆愈强半,脉象 
已近和平,拟再治以滋阴利水之剂以消其余肿。 
处方 生杭芍(六钱) 生薏米(六钱捣碎) 鲜白茅根(一两) 
药共三味,先将前二味水煎十余沸,加入白茅根,再煎四五沸,取汤一大盅,温服。 
效果 将药连服十剂,其肿全消,俾每日但用鲜白茅根一两,煎数沸当茶饮之以善其后。 
或问 前方中用麻黄三钱原可发汗,何必先用西药阿斯匹林先发 
其汗乎?答曰∶麻黄用至三钱虽能发汗,然有石膏、滑石、 
芍药以监制之,则其发汗之力顿减,况肌肤肿甚者,汗尤不易透出也。若因其汗不易出,拟复多加 
麻黄,而其性热而且燥,又非所宜。惟西药阿斯匹林,其性凉而能散,既善发汗 
又善清热,以之为麻黄之前驱,则麻黄自易奏功也。 
或问 风袭人之皮肤,何以能令人小便不利积成水肿?答曰∶小便出于膀胱,膀胱者太阳之腑也。 
袭入之风由经传腑,致膀胱失其所司,是以小便不利。麻黄能祛太阳在腑之风,佐以石膏、滑石,更 
能清太阳在腑之热,是以服药汗出而小便自利也。况此证肝中亦有蕴热,《内经》谓“肝热病者小便先 
黄”,是肝与小便亦大有关系也。方中兼用芍药以清肝热,则小便之利者当益利。至于薏米、茅根, 
亦皆为利小便之辅佐品,汇集诸药为方,是以用之必效也。 

<篇名>2.阴虚水肿
属性:邻村霍氏妇,年二十余,因阴虚得水肿证。 
病因 因阴分虚损,常作灼热,浸至小便不利,积成水肿。 
证候 头面周身皆肿,以手按其肿处成凹,移时始能撤消。日晡潮热,心中亦恒觉发热。小便赤涩,一 
日夜间不过通下一次。其脉左部弦细,右部弦而微硬,其数六至。 
诊断 此证因阴分虚损,肾脏为虚热所伤而生炎,是以不能漉水以利小便。且其左脉弦细,则肝之 
疏泄力减。可致小便不利,右脉弦硬,胃之蕴热下溜,亦可使小便不利,是以积成水肿也。宜治以大 
滋真阴之品,俾其阴足自能退热,则肾炎可愈,胃热可清。肝木得肾水之涵濡,而其疏泄之力亦自充 
足,再辅以利小便之品作向导,其小便必然通利,所积之水肿亦不难徐消矣。 
处方 生怀山药(一两) 生怀地黄(六钱) 生杭芍(六钱) 玄参(五钱) 
大甘枸杞(五钱) 沙参(四钱) 滑石(三钱) 
共煎汤一大盅,温服。 
复诊 将药连服四剂,小便已利,头面周身之肿已消弱半,日晡之热已无,心中仍有发热之时, 
惟其脉仍数逾五至,知其阴分犹未充足也。仍宜注重补其真阴而少辅以利水之品。 
处方 熟怀地黄(一两) 生杭芍(六钱) 生怀山药(五钱) 大甘枸杞(五钱) 
柏子仁(四钱) 玄参(四钱) 沙参(三钱) 生车前子(三钱装袋) 
大云苓片(二钱) 鲜白茅根(五钱) 
药共十味,先将前九味水煎十余沸,再入鲜白茅根,煎四五沸取汤一大盅,温服。若无鲜白 
茅根,可代以鲜芦根。至两方皆重用芍药者,因芍药性善滋阴,而又善利小便,原为阴 
虚小便不利者之主药也。 
效果 将药连服六剂,肿遂尽消,脉已复常,遂停服汤药,俾日用生怀山药细末两许,熬 
作粥,少兑以鲜梨自然汁,当点心服之以善其后。 

<篇名>3.风水有痰
属性:辽宁马××,年五旬,得受风水肿兼有痰证。 
病因 因秋末远出,劳碌受风遂得斯证。 
证候 腹胀,周身漫肿,喘息迫促,咽喉膺胸之间时有痰涎杜塞,舌苔淡白,小便赤涩短少, 
大便间日一行,脉象无火而微浮,拟是风水,当遵《金匮》治风水之方治之。 
处方 生石膏(一两捣细) 麻黄(三钱) 甘草(二钱) 生姜(二钱) 
大枣(四枚掰开) 西药阿斯匹林(三分) 
药共六味,将前五味煎汤一大盅,冲化阿斯匹林,温服被复取汗。 
方解 此方即越婢汤原方加西药阿斯匹林也。当时冬初,北方天气寒凉汗不易出,恐但服越 
婢汤不能得汗,故以西药之最善发汗兼能解热者之阿斯匹林佐之。 
复诊 将药服后,汗出遍体,喘息顿愈,他证如故,又添心中热渴不思饮食。诊其脉仍无火象, 
盖因痰饮多而湿胜故也。斯当舍脉从证,而治以清热之重剂。 
处方 生石膏(四两捣细) 天花粉(八钱) 薄荷叶(钱半) 
共煎汤一大碗,俾分多次徐徐温饮下。 
三诊 将药服后,热渴痰涎皆愈强半,小便亦见多,可进饮食,而漫肿腹胀不甚见轻。斯宜注重 
利其小便以消漫肿,再少加理气之品以消其腹胀。 
处方 生石膏(一两捣细) 滑石(一两) 地肤子(三钱) 丈菊子(三钱捣碎) 
海金沙(三钱) 槟榔(三钱) 鲜茅根(三钱) 
共煎汤一大盅半,分两次温服下。 
丈菊,俗名向日葵。究之,向日葵之名当属之卫足花,不可以名丈菊也。丈菊子,《本 
草纲目》未收,因其善治淋疼利小便,故方中用之。 
效果 将药煎服两剂,小便大利,肿胀皆见消,因将方中石膏、滑石、槟榔皆减半,连服三剂病全愈。 

\chapter{黄胆门}
<篇名>1.黄胆兼外感
属性:天津苏媪,年六十六岁,于仲春得黄胆证。 
病因 事有拂意,怒动肝火,继又薄受外感,遂遍身发黄成疸证。 
证候 周身黄色如橘,目睛黄尤甚,小便黄可染衣,大便色白而 
干,心中发热作渴,不思饮食。其脉左部弦长有力且甚硬,右部脉亦有力而微浮,舌苔薄而白无津液。 
诊断 此乃肝中先有蕴热,又为外感所束,其热益甚,致胆管肿胀,不能输其胆汁于小肠,而 
溢于血中随血运遍周身,是以周身无处不黄。迨至随血营运之余,又随水饮渗出归于膀胱, 
是以小便亦黄。至于大便色白者,因胆汁不入小肠以化食,大便中既无胆汁之色也。《金匮》有硝 
石矾石散,原为治女劳疸之专方,愚恒借之以概治疸证皆效,而煎汤送服之药须 
随证更改。其原方原用大麦粥送服,而此证肝胆之脉太盛,当用泻肝胆之药煎汤送之。 
处方 净火硝(一两研细) 皂矾(一两研细) 大麦面(二两焙热,如无可代以小麦面) 
水和为丸,桐子大,每服二钱,日两次。此即硝石矾石散而变散为丸也。 
汤药 生怀山药(一两) 生杭芍(八钱) 连翘(三钱) 滑石(三钱) 
栀子(二钱) 茵陈(二钱) 甘草(二钱) 
共煎汤一大盅,送服丸药一次,至第二次服丸药时,仍煎此汤药之渣送之。再者此证舌苔犹白,右 
脉犹浮,当于初次服药后迟一点钟,再服西药阿斯匹林一瓦,俾周身得微汗以解其未罢之表证。 
复诊 将药连服四剂,阿斯匹林服一次已周身得汗,其心中已不若从前之渴热,能进饮食,大便 
已变黑色,小便黄色稍淡,周身之黄亦见退,脉象亦较前和缓。俾每日仍服丸药两次, 
每次服一钱五分,所送服之汤药方则稍为加减。 
汤药 生怀山药(一两) 生杭芍(六钱) 生麦芽(三钱) 
茵陈(二钱) 鲜茅根(三钱,茅根无鲜者可代以鲜芦根) 龙胆草(二钱) 甘草(钱半) 
共煎汤,送服丸药如前。 
效果 将药连服五剂,周身之黄已减三分之二,小便之黄亦日见清减,脉象已和平如常。遂俾停 
药勿服,日用生怀山药、生薏米等分轧细,煮作茶汤,调入鲜梨、鲜荸荠自然汁,当点 
心服之,阅两旬病遂全愈。 
或问 黄胆之证,中法谓病发于脾,西法谓病发于胆。今此案全从病发于胆论治,将勿中法谓 
病发于脾者不可信欤?答曰∶黄胆之证有发于脾者有发于胆者,为黄胆之原因不同,是以仲圣治 
黄胆之方各异,即如硝石矾石散,原治病发于胆者也。其矾石若用皂矾,固为平肝胆要药,至硝石确 
系火硝,其味甚辛,辛者金味,与矾石并用更可相助为理也。且西人谓有因胆石成黄胆者,而 
硝石矾石散,又善消胆石。有因钩虫成黄胆者,而硝石矾石散,并善除钩虫,制方之妙诚不可 
令人思议也。不但此也,仲圣对于各种疸证多用茵陈,因最善入少阳之府以清热、舒郁、消肿、 
透窍,原为少阳之主药。仲圣若不知黄胆之证兼发于胆,何以若斯喜用少阳之药乎?是以至 
明季南昌喻氏出,深窥仲圣用药之奥旨,于治钱小鲁酒疸一案,直谓胆之热汁溢于外,以渐渗于经络 
则周身俱黄云云,不已显然揭明黄胆有发于胆经者乎? 

<篇名>2.黄胆
属性:王××,年三十二岁,于季秋得黄胆证。 
病因 出外行军,夜宿帐中,勤苦兼受寒凉,如此月余,遂得黄胆证。 
证候 周身黄色甚暗似兼灰色,饮食减少,肢体酸懒无力,大便一日恒两次似完谷不化, 
脉象沉细,左部更沉细欲无。 
诊断 此脾胃肝胆两伤之病也,为勤苦寒凉过度,以致伤其脾胃,是以饮食减少完谷不化;伤其 
肝胆,是以胆汁凝结于胆 
管之中,不能输肠以化食,转由胆囊渗出,随血流行于周身 
而发黄。此宜用《金匮》硝石矾石散以化其胆管之凝结,而以健脾胃补肝胆之药煎汤送服。 
处方 用硝石矾石散所制丸药,每服二钱,一日服两次,用后汤药送服。 
汤药 生箭 (六钱) 白术(四钱炒) 桂枝尖(三钱) 生鸡内金(二钱黄色的捣) 甘草(二钱) 
共煎汤一大盅,送服丸药一次,至第二次服丸药时,仍煎此汤药之渣送之。 
复诊 将药连服五剂,饮食增加,消化亦颇佳良,体力稍振,周身 
黄退弱半,脉象亦大有起色。俾仍服丸药一次服一钱五分,日两次,所送服之汤药宜略有加减。 
汤药 生箭 (六钱) 白术(三钱炒) 当归(三钱) 生麦芽(三钱) 
生鸡内金(二钱黄色的捣) 甘草(二钱) 
共煎汤一大盅,送服丸药一次。至第二次服丸药时,仍煎此汤药之渣送服。 
效果 将药连服六剂,周身之黄已退十分之七,身形亦渐强壮,脉象已复其常。俾将丸药减去一 
次,将汤药中去白术加生怀山药五钱,再服数剂以善其后。 

<篇名>3.黄胆
属性:天津范××,年三十二岁,得黄胆证。 
病因 连日朋友饮宴,饮酒过量,遂得斯证。 
证候 周身面目俱黄,饮食懒进,时作呕吐,心中恒觉发热,小便黄甚,大便白而干涩,脉象 
左部弦而有力,右部滑而有力。 
诊断 此因脾中蕴有湿热,不能助胃消食,转输其湿热于胃,以致胃气上逆(是以呕吐),胆 
火亦因之上逆(黄坤载谓,非胃气下降,则胆火不降),致胆管肿胀不能输其汁于小肠以化食,遂 
溢于血中而成黄胆矣。治此证者,宜降胃气,除脾湿,兼清肝胆之热则黄胆自愈。 
处方 生赭石(一两轧细) 生薏米(八钱捣细) 茵陈(三钱) 栀子(三钱) 
生麦芽(三钱) 竹茹(三钱) 木通(二钱) 槟榔(二钱) 甘草(二钱) 
煎汤服。 
效果 服药一剂,呕吐即止,可以进食,又服两剂,饮食如常,遂停药,静养旬日间黄胆皆退净。 

\chapter{痢疾门}
<篇名>1.痢疾转肠溃疡
属性:沧县杨××,年三十五岁,于季秋因下痢成肠溃疡证。 
病因 因业商赔累歇业,心中懊 ,暗生内热,其肝胆之热,下迫致成痢疾。痢久不愈,又转为肠溃疡。 
证候 其初下痢时,后重腹疼,一昼夜十七八次,所下者赤痢多带鲜血,间有白痢。延 
医治疗阅两月,病益加剧。所下者渐变为血水,杂以脂膜,其色腐败,其气腥臭,每腹中一觉疼即 
须入厕,一昼夜二十余次,身体羸弱,口中发干,心中怔忡,其脉左右皆弦细,其左部则 
弦而兼硬,一分钟九十二至。 
诊断 此乃因痢久不愈,肠中脂膜腐败,由腐败而至于溃烂,是 
以纯下血水杂以脂膜,即西人所谓肠溃疡也。其脉象弦细 
者,气血两亏也。其左脉细而硬者,肝肾之阴亏甚也。其口干心中怔忡者,皆下血过 
多之所致也。此宜培养其气血而以解毒化瘀生新之药佐之。 
处方 龙眼肉(一两) 生怀山药(一两) 熟地黄(一两) 金银花(四钱) 
甘草(三钱) 广三七(三钱轧细) 
药共六味,将前五味煎汤,送服三七末一半,至煎渣再服时,仍送服其余一半。 
方解 龙眼肉为补益脾胃之药,而又善生心血以愈怔忡,更善治肠风下血,治此证当为主药。 
山药亦善补脾胃,而又能上益肺气下固肾气,其所含多量之蛋白质,尤善滋阴养血,凡气 
血两虚者,洵为当用之药。熟地黄不但补肾阴也,冯楚瞻谓能大补肾中元气,要亦气血双补 
之品也。此三味并用,久亏之气血自能渐复,气血壮旺自能长肌肉排腐烂。又佐以金银 
花甘草以解毒,三七以化瘀生新,庶能挽回此垂危之证也。 
复诊 将药煎服三剂,病大见愈,一昼夜大便三四次,间见好粪,心中已不怔忡,脉象犹 
弦而左部不若从前之硬。因所服之药有效,遂即原方略为加减,又服数剂,其大便仍一日数 
次,血粪相杂,因思此证下痢甚久,或有阿米巴毒菌伏藏于内,拟方中加消除此毒菌之药治之。 
处方 龙眼肉(一两) 生怀山药(一两) 熟地黄(一两) 甘草(三钱) 
生硫黄(八分研细) 鸦胆子(成实者六十粒去皮) 
药共六味,将前四味煎汤一大盅,送服鸦胆子硫黄末各一半,至煎渣再服时,仍送服其余一半。 
方解 方中用鸦胆子、硫黄者,因鸦胆子为治血痢要药,并善治二便下血;硫黄为除阿米巴痢之 
毒菌要药,二药并用,则凉热相济,性归和平奏效当速也。 
三诊 将药煎服两剂,其大便仍血粪相杂一日数行。因思鸦胆子 
与硫黄并用虽能消除痢中毒菌,然鸦胆子化瘀之力甚大,硫黄又为润大便之药(本草谓其能使大便润、 
小便长,西人以硫黄为轻下药),二药虽能消除痢中毒菌,究难使此病完全除根,拟去此二药, 
于方中加保护脂膜固涩大便之品。 
处方 龙眼肉(一两) 生怀山药(一两) 大熟地黄(一两) 赤石脂(一两捣细) 
甘草(三钱) 广三七(三钱轧细) 
药共六味,将前五味煎汤一大盅,送服三七细末一半,至煎渣再服时,仍送服其余一半。 
效果 将药连服五剂,下血之证全愈,口中已不发干,犹日下溏粪两三次,然便时腹中分毫 
不疼矣。俾用生怀山药轧细末,每用两许煮作茶汤,调以白糖令适口,当点心服之,其大便久自能固。 


<篇名>2.痢疾
属性:天津张姓幼女,年五岁,于孟秋得痢证。 
病因 暑日恣食瓜果,脾胃有伤,入秋以来则先泻后痢。 
证候 前因泄泻旬日,身体已羸弱,继又变泻为痢,日下十余次,赤白参半,下坠腹疼。屡次服 
药不愈,身益羸弱,其脉象亦弱,而左脉之力似略胜于右。 
诊断 按其左右脉皆弱者,气血两虚也。而左脉之力似略胜于右脉者,知其肝胆虚而挟热,是 
以痢久不愈。然此热非纯系实热,不可用过凉之药,因其虚而挟热,其虚又不受补,是必 
所用之补品兼能泻热,俾肝胆之虚热皆愈而痢自愈矣。 
处方 鸭肝一具,调以食料,烹熟服之,日服二次。 
效果 如法将鸭肝烹食两日全愈,此方愚在辽宁得之友人齐× 
×。尝阅李氏《本草纲目》,鸭肉性凉善治痢,鸭蛋之腌咸 
者亦善治痢,而未尝言及鸭肝。然痢之为病,多系肝火下迫肠中,鸭肉凉想鸭肝亦凉,此证先 
泻后痢,身体羸弱,其肝经热而且虚可知,以鸭肝泻肝之热,即以鸭肝补肝之虚,此 
所谓脏器疗法,是以奏效甚速也。且又香美适口,以治孺子之苦于服药者为尤宜也。 


<篇名>3.痢疾
属性:天津郑××,年五旬,于孟秋得下痢证。 
病因 连日劳心过度,心中有热,多食瓜果,遂至病痢。 
证候 腹疼后重,下痢赤白参半,一日夜七八次,其脉左部弦而 
有力,右部浮而濡重按不实,病已八日,饮食减少,肢体酸软。 
诊断 证脉合参,当系肝胆因劳心生热,脾胃因生冷有伤,冷热相搏,遂致成痢。当清其肝 
胆之热,兼顾其脾胃之虚。 
处方 生怀山药(一两) 生杭芍(一两) 当归(六钱) 炒薏米(六钱) 
金银花(四钱) 竹茹(三钱碎者) 甘草(三钱) 生姜(三钱) 
共煎汤一大盅,温服。 
复诊 服药两剂,腹疼后重皆除,下痢次数亦减,且纯变为白痢。再诊脉左部已和平如常,而 
右部之脉仍如从前,斯再投以温补脾胃之剂当愈。 
处方 生怀山药(一两) 炒薏米(五钱) 龙眼肉(五钱) 山楂片(三钱) 
干姜(二钱) 生杭芍(二钱) 
共煎汤一大盅,温服。 
效果 将药煎汤服两剂痢遂全愈。 
帮助 按欲温补其脾胃而复用芍药者,防其肝胆因温补复生热 
也。用山楂片者,以其能化白痢之滞,且与甘草同用则酸甘化合,实有健运脾胃之功效也。 


<篇名>4.噤口痢
属性:天津施××,五十六岁,得噤口痢证。 
病因 举家数口,寄食友家不能还乡,后友家助以资斧令还乡,道路又复不通,日夜焦思,频 
动肝火,时当孟秋,心热贪凉,多食瓜果,致患下痢。 
证候 一日夜下痢十五六次,多带鲜血,后重甚剧,腹偶觉疼即须入厕,便后移时疼始稍愈, 
病已五日,分毫不能进食,唯一日之间强饮米汤数口。其脉左部弦而硬,右部弦而浮,其 
搏五至,心中发热常觉恶心。 
诊断 此肝火炽盛,肝血虚损,又兼胃气挟热上逆,是以下痢甚 
剧,而又噤口不食也。当治以滋阴、清热、平肝、降胃之品。 
处方 生杭芍(一两) 生怀山药(一两) 滑石(七钱) 白头翁(五钱) 
秦皮(三钱) 碎竹茹(三钱) 甘草(三钱) 鸦胆子(成实者五十粒去皮) 
先用白糖水囫囵送服鸭胆子仁,再将余药煎汤一大盅,温服下。 
复诊 将药如法服两剂,痢中已不见鲜血,次数减去三分之二。其脉左部较前和平,右部则 
仍有浮弦之象,仍然不能饮食,心中仍然发热,然不若从前之恶心,此宜用药再清其胃腑必然能食矣。 
处方 生怀山药(两半) 生石膏(两半捣细) 生杭芍(六钱) 白头翁(四钱) 
秦皮(二钱) 甘草(二钱) 
共煎汤一大盅,分两次温服。 
效果 将药煎服一剂,即能进食,痢已不见,变作泄泻,日四五 
次,俾用生怀山药细末煮作粥,少调以白糖服之,三日全愈。 
或问 石膏为治外感实热之药,今此证未夹杂外感,何以方中亦用之?答曰∶石膏为治阳明胃 
腑有实热者之圣药,初不论其为外感非外感也。盖阳明胃气以息息下行为顺,若有热则其气多不下行 
而上逆,因其胃气挟热上逆,所以多恶心呕吐不思饮食,若但知清其热而不知降其气,治之恒不易见效。 
惟石膏性凉质重(虽煎为汤,仍有沉重之力),其凉也能清实热,其重也能镇气逆,是以凡胃气 
挟实热上逆令人不思饮食者,服之可须臾奏效。若必谓石膏专治外感实热,不可用治内伤实热,则近 
代名医徐氏、吴氏医案中皆有重用石膏治愈内伤实热之案,何妨取以参观乎? 

\chapter{大小便病门}
<篇名>1.泄泻兼发灼
属性:天津胡××,年四十二岁,于孟秋得泄泻兼灼热病。 
病因 其兄因痢病故,铺中之事及为其兄殡葬之事,皆其一人经 
理,哀痛之余,又兼心力俱瘁,遂致大便泄泻周身发热。 
证候 一日夜泻十四五次,将泻时先腹疼,泻后疼益甚,移时始 
愈,每过午一点钟,即觉周身发热,然不甚剧,夜间三点钟后,又渐愈,其脉六部皆弱,两尺尤甚。 
诊断 按此证系下焦虚寒及胸中大气虚损也。盖下焦寒甚者,能迫下焦之元阳上浮,胸中大 
气虚甚者,恒不能收摄,致卫气外浮,则元阳之上浮与卫气之外浮相并,即可使周身发热。 
其发在过午者,因过午则下焦之阴寒益盛,而胸中大气益虚也(胸中大气乃上焦之阳气,过午阴 
盛,是以大气益虚)。此本虚寒泄泻之证,原不难治,而医者因其过午身热,皆不敢投以温补,是以屡治不愈。拟 
治以大剂温补之药,并收敛其元阳归其本源,则泄泻止而灼热亦愈矣。 
处方 白术(五钱炒) 熟怀地黄(一两) 生怀山药(一两) 净萸肉(五钱) 
干姜(三钱) 乌附子(三钱) 生杭芍(三钱) 云苓片(二钱) 炙甘草(三钱) 
共煎汤一大盅,温服。 
复诊 服药一剂,身热即愈,服至三剂,泄泻已愈强半,脉象亦 
较前有力,遂即原方略为加减俾再服之。 
处方 白术(六钱炒) 熟怀地黄(一两) 生怀山药(一两) 净萸肉(五钱) 
龙眼肉(五钱) 干姜(四钱) 乌附子(四钱) 云苓片(二钱) 炙甘草(三钱) 
效果 将药连服十余剂,病遂全愈。 
帮助 大队温补药中复用芍药者,取其与附子并用,能收敛元阳归根于阴,且能分利小便则泄 
泻易愈也。至后方去芍药者,因身已不热元阳已归其宅,且泄泻已就愈,仍有茯苓以利其 
小便,无须再用芍药也。 


<篇名>2.小便白浊
属性:天津李××,年二十六岁,得小便白浊证。 
病因 于季秋乘大车还家,中途遇雨,衣服尽湿,夜宿店中,又 
披衣至庭中小便,为寒风所袭,遂得白浊之证。 
证候 尿道中恒发刺痒,每小便完时有类精髓流出数滴。今已三阅月,屡次服药无效,颇觉身 
体衰弱,精神短少,其脉左部弦硬,右部微浮重按无力。 
诊断 《内经》谓肾主蛰藏,肝主疏泄,又谓风气通于肝,又谓肝行肾之气。此证因风寒内袭 
入肝,肝得风助,其疏泄之力愈大,故当小便时,肝为肾行气过于疏泄,遂致肾脏失其蛰 
藏之用,尿出而精亦随之出矣。其左脉弦硬者,肝脉挟风之象,其右脉浮而无力者,因病久而气 
血虚弱也。其尿道恒发刺痒者,尤显为风袭之明征也。此宜散其肝风,固其肾气, 
而更辅以培补气血之品。 
处方 生箭 (五钱) 净萸肉(五钱) 生怀山药(五钱) 生龙骨(五钱捣碎) 
生牡蛎(五钱捣碎) 生杭芍(四钱) 桂枝尖(三钱) 生怀地黄(三钱) 甘草(钱半) 
共煎汤一大盅,温服。 
方解 方中以黄 为主者,因《神农本草经》原谓黄 主大风,是以风之入脏者,黄 能逐之 
外出,且其性善补气,气盛自无滑脱之病也。桂枝亦逐风要药,因其性善平肝,故尤善逐肝家 
之风,与黄 相助为理则逐风之力愈大也。用萸肉、龙骨、牡蛎者,以其皆为收敛之品,又皆善 
收敛正气而不敛邪气,能助肾脏之蛰藏而无碍肝风之消散,药物解中论之详矣。用山药者,以 
其能固摄下焦气化,与萸肉同为肾气丸中要品,自能保合肾气不使虚泻也。用芍药、地黄者,欲以调 
剂黄 、桂枝之热,而芍药又善平肝,地黄又善补肾,古方肾气丸以干地黄为主药,即今之 
生地黄也。用甘草者,取其能缓肝之急,即能缓其过于疏泄之力也。 
效果 将药连服三剂,病即全愈,因即原方去桂枝以熟地易生地,俾再服数剂以善其后。 


<篇名>3.小便因寒闭塞
属性:辽宁石××,年三十二岁,于仲冬得小便不通证。 
病因 晚饭之后,食梨一颗,至夜站岗又受寒过甚,遂致小便不通。 
证候 病初得时,先入西医院治疗。西医治以引溺管小便通出, 
有顷小便复存蓄若干,西医又纳以橡皮引溺管,使久在其中有尿即通出。乃初虽稍利,继则小便 
仍不出,遂求为延医。其脉弦细沉微,不足四至,自言下焦疼甚且凉甚,知其小便 
因受寒而凝滞也,斯当以温热之药通之。 
处方 野党参(五钱) 椒目(五钱炒捣) 怀牛膝(五钱) 乌附子(三钱) 
广肉桂(三钱) 当归(三钱) 干姜(二钱) 小茴香(二钱) 
生明没药(二钱) 威灵仙(二钱) 甘草(二钱) 
共煎一大盅,温服。 
方解 方中之义,人参、灵仙并用,可治气虚小便不通。椒目与桂、附、干姜并用,可治因寒 
小便不通。又佐以当归、牛膝、茴香、没药、甘草诸药,或润而滑之,或引而下之,或 
辛香以透窍,或温通以开瘀,或和中以止疼,众药相济为功,自当随手奏效也。 
效果 将药煎服一剂,小便通下,服至三剂,腹疼觉凉全愈,脉 
已复常。俾停服汤药,日用生硫黄钱许研细,分作两次服,以善其后。 
帮助 诸家本草,皆谓硫黄之性能使大便润小便长,用于此证, 
其暖而能通之性适与此证相宜也。 

\chapter{不寐病门}
<篇名>1.心虚不寐
属性:天津徐××,年六十六岁,于季春得不寐证。 
病因 因性嗜吟咏,暗耗心血,遂致不寐。 
证候 自冬令间有不寐之时,未尝介意,至春日阳生病浸加剧, 
迨至季春恒数夜不寐,服一切安眠药皆不效。精神大为衰惫,心中时常发热,懒于饮食,勉强加餐, 
恒觉食停胃脘不下行。大便干燥,恒服药始下。其脉左部浮弦,右脉尤弦而兼硬,一息五至。 
诊断 其左脉浮弦者,肝血虚损,兼肝火上升也,阴虚不能潜阳,是以不寐。其右脉弦而 
兼硬者,胃中酸汁短少更兼胃气上逆也。酸汁少则不能化食,气上逆则不能息息下行传送饮食, 
是以食后恒停胃脘不下。而其大便之燥结,亦即由胃腑气化不能下达所致。治此证者,宜清肝火、生 
肝血、降胃气、滋胃汁,如此以调养肝胃,则夜间自能安睡,食后自不停滞矣。 
处方 生怀山药(一两) 大甘枸杞(八钱) 生赭石(六钱轧细) 玄参(五钱) 
北沙参(五钱) 生杭芍(五钱) 酸枣仁(四钱炒捣) 生麦芽(三钱) 
生鸡内金(钱半黄色的捣) 茵陈(钱半) 甘草(二钱) 
共煎一大盅,温服。 
复诊 将药煎服两剂,夜间可睡两三点钟,心中已不发热,食量 
亦少加增,大便仍滞,脉象不若从前之弦硬,遂即原方略为加减俾再服之。 
处方 生怀山药(一两) 大甘枸杞(八钱) 生赭石(六钱轧细) 玄参(五钱) 
北沙参(五钱) 酸枣仁(四钱炒捣) 龙眼肉(三钱) 生杭芍(三钱) 
生鸡内金(钱半黄色的捣) 生远志(钱半) 茵陈(一钱) 甘草(钱半) 
共煎汤一大盅,温服。 
效果 将药连服三剂,夜间安睡如常,食欲已振,大便亦自然通下。惟脉象仍有弦硬之意,遂 
将方中龙眼肉改用八钱,俾多服数剂以善其后。 
帮助 人禀天地之气化以生,是以上焦之气化为阳,下焦之气化为阴。当白昼时,终日言语 
动作,阴阳之气化皆有消耗,实赖向晦燕息以补助之。诚以人当睡时,上焦之阳气下降潜藏与下焦 
之阴气会合,则阴阳自能互根,心肾自然相交。是以当熟睡之时,其相火恒炽盛暗动(得心阳之助), 
此心有益于肾也。至睡足之时,精神自清爽异常(得肾阴之助),此肾有益于心也。由斯知 
人能寐者,由于阳气之潜藏,其不能寐者,即由于阳气之浮越,究其所以浮越者,实因脏腑 
之气化有升无降也。是以方中重用赭石以降胃镇肝,即以治大便燥结,且其色赤质重, 
能入心中引心阳下降以成寐,若更佐以龙骨、牡蛎诸收敛之品以镇安精神,则更可稳睡。而方 
中未加入者,因其收涩之性与大便燥结者不宜也。又《内经》治目不得瞑,有半夏秫米汤原甚 
效验,诚以胃居中焦,胃中之气化若能息息下行,上焦之气化皆可因之下行。半夏善于降胃,秫米善于 
和胃,半夏与秫米并用,俾胃气调和顺适不失下行之常,是以能令人瞑目安睡。方中赭石与山 
药并用,其和胃降胃之力实优于半夏秫米,此乃取古方之义而通变化裁,虽未显用古方而不啻用古方也。 


<篇名>2.不寐兼惊悸
属性:表兄赵××之妻,年近三旬,得不寐证,兼心中恒惊悸。 
病因 因家中诸事皆其自理,劳心过度,因得不寐兼惊悸病。 
证候 初苦不寐时,不过数日偶然,其过半夜犹能睡,继则常常如此,又继则彻夜不寐。一连 
七八日困顿已极,仿佛若睡,陡觉心中怦怦而动,即暮然惊醒,醒后心犹怔忡,移时始定。 
心常发热,呼吸似觉短气,懒于饮食,大便燥结,四五日始一行。其脉左部弦硬,右部近滑,重 
诊不实,一息数近六至。 
诊断 此因用心过度,心热耗血,更因热生痰之证也。为其血液因热暗耗,阴虚不能潜阳,是 
以不寐,痰停心下,火畏水刑(心属火痰属水),是以惊悸。其呼吸觉短气者,上焦凝滞之痰碍气 
之升降也。其大便燥结者,火盛血虚,肠中津液短也。此宜 
治以利痰、滋阴、降胃、柔肝之剂,再以养心安神之品辅之。 
处方 生赭石(八钱轧细) 大甘枸杞(八钱) 生怀地黄(八钱) 生怀山药(六钱) 
栝蒌仁(六钱炒捣) 天冬(六钱) 生杭芍(五钱) 清半夏(四钱) 
枣仁(四钱炒捣) 生远志(二钱) 茵陈(钱半) 甘草(钱半) 朱砂(二分研细) 
药共十三味,将前十二味煎汤一大盅,送服朱砂末。 
复诊 将药连服四剂,心中已不觉热,夜间可睡两点钟,惊悸已愈十之七八,气息亦较前 
调顺,大便之燥结亦见愈,脉象左部稍见柔和,右部仍有滑象,至数稍缓,遂即原方略为加减 
俾再服之。 
处方 生赭石(八钱轧细) 大甘枸杞(八钱) 生怀地黄(八钱) 生怀山药(六钱) 
龙眼肉(五钱) 栝蒌仁(五钱炒捣) 玄参(五钱) 生杭芍(五钱) 
枣仁(四钱炒捣) 生远志(二钱) 甘草(二钱) 
共煎汤一大盅,温服。 
效果 将药连服六剂,彻夜安睡,诸病皆愈。 

\chapter{痫痉癫狂门}
<篇名>1.痫风兼脑充血
属性:天津陈××,年三十八岁,得痫风兼脑充血证。 
病因 因肝火素盛,又在校中任讲英文,每日登堂演说,时间过长。劳心劳力皆过度,遂得斯证。 
证候 其来社求诊时,但言患痫风,或数日一发,或旬余一发,其发必以夜,亦不自觉,惟 
睡醒后其舌边觉疼,有咬破之处,即知其睡时已发痫风,其日必精神昏愦,身体酸懒。诊其脉 
左右皆弦硬异常,因问其脑中发热或作疼,或兼有眩晕之时乎?答曰∶此三种病脑中皆有,余 
以为系痫风之连带病,故未言及耳。愚曰∶非也,是子患痫风兼患脑充血也。 
诊断 按痫风之证,皆因脑髓神经失其所司,而有非常之变动,其脑部若充血过甚者,恒至排 
挤脑髓神经,使失其常司也。此证既患痫风,又兼脑部充血,则治之者自当以先治其脑部充血为急务。 
处方 治以拙拟镇肝熄风汤,为其兼患痫风加全蜈蚣大者三条,盖镇肝熄风汤原为拙拟治脑 
充血之主方,而蜈蚣又善治痫风之要药也。 
复诊 前方连服十剂,脑部热疼眩晕皆除。惟脉仍有力,即原方 
略为加减,又服十剂则脉象和平如常矣。继再治其痫风。 
处方 治以拙拟愈痫丹,日服两次,每次用生怀山药五钱煎汤送下。 
效果 服药逾两月旧病未发,遂停药勿服,痫风从此愈矣。 


<篇名>2.受风螈
属性:天津董姓幼女,年三岁,患螈 病。 
病因 暮春气暖着衣过浓,在院中 戏,出汗受风,至夜间遂发螈 。 
证候 剧时闭目昏昏,身躯后挺,两手紧握,轻时亦能明了,而 
舌肿不能吮乳,惟饮茶汤及代乳粉。大便每日溏泻两三次, 
如此三昼夜不愈,精神渐似不支,皮肤发热,诊其脉亦有热象。 
诊断 此因春暖衣浓,肝有郁热,因外感激发其热上冲脑部,排挤脑髓神经失其运动之常度,是以 
发搐。法当清其肝热,散其外感,兼治以镇安神经之药其病自愈。 
处方 生怀山药(一两) 滑石(八钱) 生杭芍(六钱) 连翘(三钱) 
甘草(三钱) 全蜈蚣(两条大者) 朱砂(二分细末) 
药共七味,将前六味煎汤一盅,分数次将朱砂徐徐温送下。 
效果 将药煎服一剂,螈 已愈,其头仍向后仰,左手仍拳曲不舒,舌肿已消强半,可以吮乳, 
大便之溏已愈。遂即原方减滑石之半,加玄参六钱,煎服后左手已不拳曲,其头有后仰 
之意,遂减去方中滑石,加全蝎三个,服一剂全愈。 

<篇名>3.慢脾风
属性:辽宁侯姓幼子,年七岁,于季秋得慢脾风证。 
病因 秋初病疟月余方愈,愈后觉左胁下痞硬,又屡服消瘀之品,致脾胃虚寒不能化食,浸至吐 
泻交作,兼发抽掣。 
证候 日 潮热,两颧发红,昏睡露晴,手足时作抽掣,剧时督脉紧而头向后仰(俗名角弓反张), 
无论饮食药物服后半点钟即吐出,且带出痰涎若干,时作泄泻,其脉象细数无力。 
诊断 疟为肝胆所受之邪,木病侮土,是以久病疟者多伤脾胃。此证从前之左胁下痞硬,脾 
因受伤作胀也。而又多次服消导开破之品,则中焦气化愈伤,以致寒痰留饮积满上溢,迫激 
其心肺之阳上浮,则面红外越而身热,而其病本实则凉也。其不受饮食者,为寒痰所阻也; 
其兼泄泻者,下焦之气化不固也;其手足抽掣者,血虚不能荣筋养肝,则肝风内动而筋 
紧缩也;抽掣剧时头向后仰者,不但督脉因寒紧缩,且以督 
脉与神经相连,督脉病而脑髓神经亦病,是以改其常度而妄行也。拟先用《福幼编》逐寒荡 
惊汤开其寒痰,俾其能进饮食斯为要务。 
处方 胡椒(一钱) 干姜(一钱) 肉桂(一钱) 丁香(十粒, 
四味共捣成粗渣)高丽参(一钱) 甘草(一钱) 
先用灶心土三两煮汤澄清,以之代水,先煎人参、甘草七八沸,再入前四味同煎三四沸,取 
清汤八分杯,徐徐灌之。 
此方即逐寒荡惊汤原方加人参、甘草也。原方干姜原系炮用,然炮之则其气轻浮,辣变为苦, 
其开通下达之力顿减,是以不如生者。特是生用之则苛辣过甚,故加甘草和之,且 
能逗留干姜之力使绵长也。又加人参者,欲以补助胸中大气以运化诸药之力,仲师所谓大气一转, 
其结(即痰饮)乃散也。 
又此方原以胡椒为主,若遇寒痰过甚者,可用至钱半。又此物在药局中原系背药,陈久 
则力减,宜向食料铺中买之。 
复诊 将药服后呕吐即止,抽掣亦愈,而潮热泄泻亦似轻减,拟 
继用《福幼编》中加味理中地黄汤,略为加减俾服之。 
处方 熟怀地黄(五钱) 生怀山药(五钱) 焦白术(三钱) 大甘枸杞(三钱) 
野党参(二钱) 炙箭 (二钱) 干姜(二钱) 生杭芍(二钱) 
净萸肉(二钱) 肉桂(一钱后入) 红枣(三枚掰开) 炙甘草(一钱) 胡桃(一个用仁捣碎) 
共煎汤一大盅,分多次徐徐温服下。 
方解 此方之药为温热并用之剂,热以补阳,温以滋阴,病本寒凉是以药宜温热,而独杂以性 
凉之芍药者,因此证凉在脾胃,不在肝胆,若但知暖其脾胃,不知凉其肝胆,则肝胆因 
服热药而生火,或更激动其所寄之相火,以致小便因之不利,其大便必益泄泻,芍药能凉肝胆, 
尤善利小便,且尤善 
敛阳气之浮越以退潮热,是以方中特加之也。 
《福幼编》此方干姜亦系炮用,前方中之干姜变炮为生,以生者善止呕吐也。今呕吐已止,而 
干姜复生用者,诚以方中药多滞腻,犹恐因之生痰,以干姜生用之苛辣者开通之,则滞腻可化,而干 
姜苛辣过甚之性,即可因与滞腻之药并用而变为缓和,此药性之相合而化亦即相得益彰也。 
此方原亦用灶心土煎汤以之代水煎药,而此时呕吐已止,故可不用。然须知灶心土含碱质甚多, 
凡柴中有碱质者烧余其碱多归灶心土,是以其所煮之汤苦咸,甚难下咽,愚即用时恒以灶圹红土 
代之。且灶心土一名伏龙肝,而雷 谓用此土勿误用灶下土,宜用灶额中赤土,此与灶圹中红土无异, 
愚从前原未见其说,后得见之,自喜拙见与古暗合也。 
效果 将药连服两剂,潮热与泄泻皆愈,脉象亦较前有力。遂去白术,将干姜改用一钱,又服两剂全愈。 

<篇名>4.慢脾风
属性:辽宁张××幼孙,年四岁,得慢脾风证。 
病因 秋初恣食瓜果,久则损伤脾胃,消化力减犹不知戒,中秋节后遂成慢脾风证。 
证候 食欲大减,强食少许犹不能消化,医者犹投以消食开瘀之剂,脾胃益弱,浸至吐泻交作,间 
发抽掣,始求愚为诊视,周身肌肤灼热,其脉则微细欲无,昏睡露睛,神气虚弱。 
诊断 此证因脾胃虚寒,不能熟腐水谷消化饮食,所以作吐泻。且所食之物不能融化精微以生气 
血,惟多成寒饮,积于胃中溢于膈上,排挤心肺之阳外出,是以周身灼热而脉转微细, 
此里有真寒外作假热也。其昏睡露睛者,因眼胞属脾胃,其 
脾胃如此虚寒,眼胞必然紧缩,是以虽睡时而眼犹微睁也。其肢体抽掣者,因气血亏损,不能上达于脑 
以濡润斡旋其脑髓神经(《内经》谓上气不足则脑为之不满。盖血随气升,气之上升者少, 
血之上升亦少。可知观囟门未合之小儿,患此证者,其囟门必然下陷,此实脑为不满之明证,亦即气血不能上达之明征 
也),是以神经失其常司而肢体有时抽掣也。此当投以温暖之剂,健补脾胃以消 
其寒饮,诸病当自愈。 
处方 赤石脂(一两研细) 生怀山药(六钱) 熟怀地黄(六钱) 焦白术(三钱) 
乌附子(二钱) 广肉桂(二钱去粗皮后入) 干姜(钱半) 大云苓片(钱半) 
炙甘草(二钱) 高丽参(钱半捣为粗末) 
药共十味,将前九味煎汤一大盅,分多次徐徐温服,每次皆送服参末少许。 
方解 方中重用赤石脂者,为其在上能镇呕吐,在下能止泄泻 
也。人参为末送服者,因以治吐泻丸散优于汤剂,盖因丸散之渣滓能留恋于肠胃也。 
效果 将药服完一剂,呕吐已止,泻愈强半,抽掣不复作,灼热亦大轻减,遂将干姜减去,白 
术改用四钱,再服一剂,其泻亦止。又即原方将附子减半,再加大甘枸杞五钱,服两剂病遂全愈。 
帮助 按此证若呕吐过甚者,当先用《福幼编》逐寒荡惊汤开其寒饮,然后能受他药,而此证 
呕吐原不甚剧,是以未用。 

<篇名>5.将成慢脾风
属性:邻村赵姓幼男,年八岁,脾胃受伤,将成慢脾风证。 
病因 本系农家,田园种瓜看守其间,至秋日瓜熟,饥恒食瓜当饭,因之脾胃受伤,显露慢脾风朕兆。 
证候 食后,饮食不化恒有吐时,其大便一日三四次,多带完 
谷,其腿有时不能行步,恒当行走之时委坐于地,其周身偶有灼热之时,其脉左部弦细,右部 
虚濡,且至数兼迟。 
诊断 此证之吐而且泻及偶痿废不能行步,皆慢脾风朕兆也。况其周身偶或灼热,而脉转弦 
细虚濡,至数且迟,此显系内有真寒外有假热之象。宜治以大剂温补脾胃之药,俾脾胃健旺 
自能消化饮食,不复作吐作泻,久之则中焦气化舒畅,周身血脉贯通,余病自愈。 
处方 生怀山药(一两) 白术(四钱炒) 熟怀地黄(四钱) 龙眼肉(四钱) 
干姜(三钱) 生鸡内金(二钱黄色的捣) 生杭芍(二钱) 甘草(二钱) 
共煎汤一大盅,分两次温服下。 
复诊 将药煎服两剂,吐泻灼热皆愈,惟行走时犹偶觉腿有不利,因即原方略为加减,俾多服数剂当全愈。 
处方 生怀山药(一两) 熟怀地黄(四钱) 龙眼肉(四钱) 胡桃仁(四钱) 
白术(三钱炒) 川续断(三钱) 干姜(二钱) 生鸡内金(二钱黄色的捣) 
生杭芍(钱半) 甘草(钱半) 
共煎汤一大盅,分两次温服。 
效果 将药煎服两剂,病遂全愈,因切戒其勿再食生冷之物,以防病之反复。 

<篇名>6.癫狂失心
属性:都××,年三旬,得癫狂失心证。 
病因 心郁生热,因热生痰,遂至癫狂失心。 
证候 言语错乱,精神昏瞀,时或忿怒,时或狂歌,其心中犹似烦躁,夜不能寐,恒以手自挠 
其胸,盖自觉发闷也。问之亦不能答,观其身形似颇强壮,六脉滑实,两寸尤甚,一息五至。 
诊断 人之元神在脑,识神在心,心脑息息相通,其神明自湛然长醒。生理学家谓心有四支血管通脑, 
此即神明往来于心脑之路也。此证之脉其关前之滑实太过,系有热痰上壅将其心脑相通之路杜塞,遂至神 
明有所隔碍,失其常性,此癫狂失心之所由来也。治之者当投以开通重坠之剂,引其痰火下行,其四支血 
管为痰所瘀者,复其流通之旧,则神明之往来自无所隔碍,而复湛然长醒之旧矣。 
处方 生赭石(两半轧细) 川大黄(八钱) 清半夏(五钱) 芒硝(四钱) 
药共四味,先将赭石半夏煎十余沸,加入大黄煎两三沸,取汤一大盅,入芒硝融化温服。 
方解 方中重用赭石者,其重坠之性能引血管中之瘀痰下行也。 
复诊 三日服药一次(凡降下之药不可连服,须俟其正气稍缓再服),共服三次,每次服药 
后通下大便两三次,似有痰涎随下,其精神较前稍明了,诊其脉仍有滑实之象,身体未见衰弱,拟 
再投以较重之剂,盖凡癫狂之甚者,非重剂治之不能愈也。 
处方 生赭石(二两轧细) 川大黄(一两) 芒硝(四钱) 甘遂(钱半细末) 
药共四味,先煎赭石十余沸,入大黄煎两三沸,取汤一大盅,入芒硝融化,将服时再调入甘遂末。 
三诊 将药如法煎服一剂,下大便五六次,带有痰涎若干,中隔两日又服药一次(药中有甘遂,必须三 
日服一次,不然必作呕吐),又下大便五六次,中多兼痰块挑之不开,此所谓顽痰也。从此精神大见 
明了,脉象亦不复滑实矣,拟改用平和之剂调治之。 
处方 生怀山药(一两) 生杭芍(六钱) 清半夏(四钱) 石菖蒲(三钱) 
生远志(二钱) 清竹沥(三钱) 镜面砂(三分研细) 
药共七味,将前五味煎汤一大盅,调入竹沥送服朱砂细末。 
效果 将药如法煎服数剂,病遂全愈。 

<篇名>7.神经锗乱
属性:天津黄××,年二十岁,得神经错乱病。 
病因 因心中忿郁,久之遂致神经错乱。 
证候 心中满闷发热,不思饮食,有时下焦有气上冲,并觉胃脘之气亦随之上冲,遂致精神 
昏瞀,言语支离,移时觉气消稍顺,或吐痰数口,精神遂复旧。其左脉弦而硬,右脉弦而 
长,两尺皆重按不实,一息五至。 
诊断 此乃肝火屡动,牵引冲气胃气相并上冲,更挟痰涎上冲以滞塞于喉间并冲激其脑部,是以 
其神经错乱而精神言语皆失其常也。其左脉弦硬者,肝血虚而火炽盛也。右脉弦长者,冲气挟胃气上 
冲之现象也。方书论脉有直上直下冲脉昭昭之语,所谓直上直下者,即脉弦且长之形状也。其两尺不实 
者,下焦之气化不固也,因下焦有虚脱之象,是以冲气易挟胃气上冲也。此当治以降胃、敛冲、镇 
肝之剂,更兼用凉润滋阴之品,以养肝血,清肝热,庶能治愈。 
处方 生赭石(一两轧细) 灵磁石(五钱轧细) 生怀山药(八钱) 生龙骨(八钱捣碎) 
生杭芍(六钱) 玄参(五钱) 柏子仁(五钱) 云苓片(三钱) 
清半夏(三钱) 石菖蒲(三钱) 生远志(二钱) 镜面砂(三分研细) 
药共十二味,将前十一味煎汤一大盅,送服朱砂细末。 
复诊 将药连服四剂,满闷发热皆大见愈,能进饮食,有时气复上冲而不复上干神经至于 
错乱,左右之脉皆较前平和,而尺部仍然欠实,拟兼用培补下元之品以除病根。 
处方 生赭石(一两轧细) 熟怀地黄(八钱) 生怀山药(八钱) 大甘枸杞(六钱) 
净萸肉(五钱) 生杭芍(四钱) 玄参(四钱) 云苓片(二钱) 
共煎汤一大盅,温服。 
效果 将药连服六剂,诸病皆愈,脉亦复常。 
或问 地黄之性粘腻生痰,胃脘胀满,有痰者多不敢用,今重用之何以能诸病皆愈?答曰∶用药 
如用兵,此医界之恒言也,如宋八字军最弱,刘 将之即为劲卒,遂能大败金人奏顺昌 
之捷,以斯知兵无强弱,在用之者何如耳。至用药亦何独不然,忆曾治一李姓媪,胃口 
满闷有痰,其脉上盛下虚,投以肾气丸作汤服,为加生赭石八钱,服后觉药有推荡之力,须 
臾胸次豁然,肾气丸非重用地黄者乎?然如此用药非前无师承而能有然也。《金匮》云∶短气有微饮 
当从小便去之,苓桂术甘汤主之,肾气丸亦主之。夫饮即痰也,气短亦近于满 
闷,而仲师竟谓可治以肾气丸,愚为于《金匮》曾熟读深思,故临证偶有会心耳。 

\chapter{伤寒门}
<篇名>1.伤寒兼脑膜炎
属性:盐山李××,年六旬,于季冬患伤寒兼脑膜生炎。 
病因 素有头昏证,每逢上焦有热,精神即不清爽,腊底偶冒风寒病传阳明,邪热内炽,则脑膜 
生炎,累及神明失其知觉。 
证候 从前医者治不如法,初得时未能解表,遂致伤寒传里,阳明腑实,舌苔黄而带黑,其干 
如错,不能外伸,谵语不休,分毫不省人事,两目直视不瞬。诊其脉两手筋惕不安,脉象 
似有力而不实,一息五至,大便四日未行,小便则溺时不知。 
诊断 此乃病实脉虚之证,其气血亏损难抗外邪,是以有种种危 
险之象。其舌苔黑而干者,阳明热实津液不上潮也;其两目 
直视不瞬者,肝火上冲而目发胀也;其两手筋惕不安者,肝热血耗而内风将动也;其谵语不省 
人事者,固有外感之邪热过盛,昏其神明,实亦由外感之邪热上蒸,致脑膜生炎,累 
及脑髓神经也。拟用白虎加人参汤,更辅以滋补真阴之品,庶可治愈。 
处方 生石膏(五两捣细) 生怀地黄(二两) 野台参(八钱) 天花粉(八钱) 
北沙参(八钱) 知母(六钱) 生杭芍(六钱) 生怀山药(六钱) 
甘草(四钱) 荷叶边(一钱) 
共煎汤三盅,分三次温服下,每服一盅调入生鸡子黄两枚。方中不用粳米者,以生山药可代 
粳米和胃也;用生鸡子黄者,以其善熄肝风之内动也;用荷叶者,以善引诸凉药之力 
直达脑中以清脑膜之炎也。 
再诊 将药如法煎服,翌晨下大便一次,舌苔干较愈,而仍无津液,精神较前明了而仍有谵语 
之时,其目已不直视而能瞬,诊其脉筋惕已愈强半,至数较前稍缓,其浮分不若从前有 
力,而重按却比从前有根底,此皆佳兆也。拟即前方略为加减,清其余热即以复其真阴,庶可全愈。 
处方 生石膏(四两捣细) 生怀地黄(二钱) 野台参(八钱) 大甘枸杞(一两) 
生怀山药(一两) 天花粉(八钱) 北沙参(八钱) 知母(六钱) 
生杭芍(六钱) 甘草(四钱) 
共煎汤三盅。为其大便已通,俾分多次徐徐温饮下,一次只饮一大口。 
效果 阅十点钟将药服完,精神清爽,诸病皆愈。 
帮助 按治脑膜炎证,羚羊角最佳,而以治筋惕不安亦羚羊角最效,以其上可清头脑下可熄 
肝风之萌动也。然此药价太昂,僻处药局又鲜真者,是以方中未用,且此证虽兼有脑膜炎 
病,实因脏腑之邪热上蒸,清其邪热则脑膜炎自愈,原不必 
注重于清脑也。 


<篇名>2.伤寒脉闭
属性:天津张××,年三十八岁,于季冬得伤寒证,且无脉。 
病因 旬日前曾感冒风寒,经医治愈,继出门作事,又感风寒遂得斯病。 
证候 内外俱觉寒凉,头疼,气息微喘,身体微形寒战,六脉皆无。 
诊断 盖其身体素弱,又在重感之余,风寒深入阻塞经络,是以脉闭。拟治以麻黄汤,再重加补 
气之药,补其正气以逐邪外出,当可奏效。 
处方 麻黄(三钱) 生箭 (一两) 桂枝尖(二钱) 杏仁(二钱去皮) 甘草(二钱) 
先煎麻黄数沸,吹去浮沫,再入余药同煎汤一大盅,温服,被复取微汗。 
效果 服药后周身得汗,其脉即出,诸病皆愈。 
帮助 按此证或疑系少阴伤寒,因少阴伤寒脉原微细,微细之至可至于无也。而愚从太阳治者,因 
其头疼、微喘、寒战,皆为太阳经之现象,而无少阴证蜷卧、但欲寐之现象也。是以 
于麻黄汤中,重加生黄 一两,以助麻、桂成功,此扶正即以逐邪也。 


<篇名>3.伤寒脉闭
属性:天津李姓童子,年十四岁,得伤寒脉闭证。 
病因 其左肋下素有郁气,发动时辄作疼,一日发动疼剧,头上汗出,其汗未解,出冒风寒, 
遂得斯证。 
证候 头疼、身冷、恶寒、无汗、心中发热,六脉皆闭。 
诊断 因其素有肋下作疼之病,身形羸弱;又当汗出之时感冒风寒,则风寒之入者必深,是 
以脉闭身寒;又肋下素有郁气,其肝胆之火必然郁滞,因外感所束激动其素郁之火,所以心 
中觉热。法当以发表之药为主,而以清热理郁兼补正之药佐之。 
处方 麻黄(二钱) 玄参(六钱) 生怀山药(六钱) 野台参(二钱) 
生鸡内金(二钱) 天花粉(五钱) 甘草(钱半) 
先煎麻黄数沸,吹去浮沫,再入诸药同煎一大盅,温服取汗,若不出汗时,宜再服西药阿斯匹林一瓦以助其汗。 
效果 服药两点钟,周身微发热,汗欲出不出,遂将阿斯匹林服下,须臾汗出遍体,翌日复诊,其 
脉已出,五至无力,已不恶寒,心中仍觉发热,遂去麻黄,将玄参、山药皆改用一两, 
服至三剂后;心中已不发热,遂将玄参、天花粉各减半,再服数剂以善其后。 


<篇名>4.少阴伤寒
属性:天津李××,年三十二岁,于夏季得伤寒证。 
病因 午间恣食瓜果,因夜间失眠,遂食余酣睡,值东风骤至天气忽变寒凉,因而冻醒,其未醒 
之时又复梦中遗精,醒后遂觉周身寒凉抖战,腹中又复隐隐作疼,惧甚,遂急延为诊视。 
证候 迨愚至为诊视时,其寒战腹疼益甚,其脉六部皆微细欲无,知其已成直中少阴之伤寒也。 
诊断 按直中少阴伤寒为麻黄附子细辛汤证,而因在梦遗之后, 
腹中作疼,则寒凉之内侵者益深入也,是宜于麻黄附子细辛汤中再加温暖补益之品。 
处方 麻黄(二钱) 乌附子(三钱) 细辛(一钱) 熟地黄(一两) 
生怀山药(五钱) 净萸肉(五钱) 干姜(三钱) 公丁香(十粒) 
煎汤一大盅,温服,温复取汗,勿令过度。 
效果 将药服后,过一点钟,周身微汗,寒战与腹疼皆愈。 
或问 麻黄附子细辛汤证,伤寒始得发热脉沉也,今斯证寒战脉沉细,夫寒战与发热迥异矣, 
何以亦用麻黄附子细辛汤乎?答曰∶麻黄附子细辛汤证,是由太阳传少阴也,为其病传少阴是以 
脉沉,为其自太阳传少阴是以太阳有响应之力而发热。此证昼眠冻醒,是自太阳传少阴,又因恣食寒凉 
继而昼寝梦遗,其寒凉又直中少阴,内外寒凉夹攻,是以外寒战而内腹疼,太阳虽为表阳亦无响应之 
力也。方中用麻黄以逐表寒,用附子以解里寒,用细辛以通融表里,使表里之寒尽化;又因其少阴新 
虚,加熟地黄、萸肉、山药以补之,养正即以除邪也,又因其腹疼知寒侵太深,又加干姜、丁香助附 
子、细辛以除之,寒邪自无遁藏也。方中用意周匝,是以服之即效。至于麻黄发汗止二钱者,因当 
夏令也,若当冬令则此证必须用四钱方能出汗,此用药因时令而有异也。至若在南方虽当冬令用麻 
黄二钱亦能发汗,且南方又有麻黄不过钱之说,此又用药因地点而有异也。 


<篇名>5.伤寒兼有伏热证
属性:辽宁马××,年五十一岁,得伤寒兼有伏热证。 
病因 因买卖赔钱,家计顿窘,懊悔不已,致生内热;孟冬时因受风,咳嗽有痰微喘,小便不利, 
周身漫肿。愚为治愈,旬日之外,又重受外感,因得斯证。 
证候 表里大热,烦躁不安,脑中胀疼,大便数日一行甚干燥,舌苔白浓,中心微黄,脉极洪实, 
左右皆然,此乃阳明腑实之证。凡阳明腑实之脉,多偏见于右手,此脉左右皆洪实 
者,因其时常懊悔,心肝积有内热也,其脑中胀疼者,因心 
与肝胆之热挟阳明之热上攻也。当用大剂寒凉微带表散,清其阳明胃腑之热,兼以清其心肝之热。 
处方 生石膏(四两捣细) 知母(一两) 甘草(四钱) 粳米(六钱) 青连翘(三钱) 
共作汤煎至米熟,取汤三盅,分三次温服下,病愈勿尽剂。 
方解 此方即白虎汤加连翘也,白虎汤为伤寒病阳明腑热之正药,加连翘者取其色青入肝,气 
轻入心,又能引白虎汤之力达于心肝以清热也。 
效果 将药三次服完,其热稍退,翌日病复还原,连服五剂,将生石膏加至八两,病仍如故,大 
便亦不滑泻,病家惧不可挽救,因晓之曰∶石膏原为平和之药,惟服其细末则较有力,听吾用 
药勿阻,此次即愈矣。为疏方,方中生石膏仍用八两,将药煎服之后,又用生石膏细末二两,俾蘸梨 
片徐徐嚼服之,服至两半,其热全消,遂停服,从此病愈,不再反复。 
附记 此案曾登于《全国名医验案类编》,何廉臣评此案云∶“日本××××谓∶‘石膏非大 
剂则无效,故白虎汤、竹叶石膏汤及其他石膏诸方,其量皆过于平剂。世医不知此意为小剂 
用之,譬如一杯水救一车薪之火,宜乎无效也。’吾国善用石膏者,除长沙汉方之外, 
明有缪氏仲淳,清有顾氏松园、余氏师愚、王氏孟英,皆以善治温热名,凡治阳明实热之证, 
无不重用石膏以奏功。今用石膏由四两加至八两,似已骇人听闻,然连服五、六剂热仍如故,大 
便亦不滑泻,迨外加石膏细末梨片蘸服又至两半,热始全消而病愈,可见石膏为凉药中纯 
良之品,世之畏石膏如虎者,可以放胆而不必怀疑也。” 

\chapter{温病门}
<篇名>1.温病兼大气下陷
属性:天津康××幼女,年九岁,于孟秋得温病兼大气下陷。 
病因 因得罪其母惧谴谪,藏楼下屋中,屋窗四敞,卧床上睡着,被风吹袭遂成温病。 
证候 初得病时服药失宜,热邪内陷,神昏不语,后经中西医多位延医二十余日,病益加剧,医者 
见病危已至极点,皆辞不治。继延愚为诊视,其两目上窜,几不见黑睛,精神昏愦,毫无知觉,身体 
颤动不安,时作嗳声,其肌肤甚热,启其齿见其舌缩而干,苔薄微黄,偶灌以水或米汤犹知下咽,其气 
息不匀,间有喘时,其脉数逾六至,左部细而浮,不任重按,右部亦弦细,重诊似有力,大便旬日未行。 
诊断 此外感之热久不退,灼耗真阴,以致肝脏虚损,木燥生风 
而欲上脱也。当用药清其实热,滋其真阴,而更辅以酸收敛肝之品,庶可救此极危之证。 
处方 生石膏(二两轧细) 野台参(三钱) 生怀地黄(一两) 净萸肉(一两) 
生怀山药(六钱) 甘草(二钱) 
共煎汤两大盅,分三次温饮下,每次调入生鸡子黄一枚。 
方解 此方即白虎加人参汤,以生地黄代知母,生山药代粳米,而又加萸肉也。此方若不加 
萸肉为愚常用之方,以治寒温证当用白虎加人参汤而体弱阴亏者,今加萸肉借以收敛肝气之将 
脱也。至此方不用白虎汤加减,而必用白虎加人参为之加减者,因病至此际,非加人参于 
白虎汤中,不能退其深陷之热,复其昏愦之神明也。此理参观药物人参解后所附医案自明。 
复诊 将药三次服完,目睛即不上窜,身体安稳不复颤动,嗳声已止,气息已匀,精神较前 
明了而仍不能言,大便犹未通下,肌肤犹热,脉数已减,不若从前之浮弦,而右部重诊仍 
似有力,遂即原方略为加减,俾再服之。 
处方 生石膏(两半轧细) 野台参(三钱) 生怀地黄(一两) 净萸肉(六钱) 
天冬(六钱) 甘草(二钱) 
共煎汤两盅,分两次温饮下,每次调入生鸡子黄一枚。 
三诊 日服药一剂,连服两日,热已全退,精神之明了似将撤消而仍不能言,大便仍未通下,间 
有努力欲便之象,遂用灌肠法以通其便。再诊其脉六部皆微弱无力,知其所以不能言 
者,胸中大气虚陷,不能上达于舌本也。宜于大剂滋补药中,再加升补气分之品。 
处方 生怀山药(一两) 大甘枸杞(一两) 沙参(一两) 天冬(六钱) 
寸麦冬(六钱) 生箭 (三钱) 野台参(三钱) 升麻(一钱) 桔梗(一钱) 
共煎汤一盅半,分两次温服下。 
效果 将药煎服两剂,遂能言语,因即原方去升麻减沙参之半,再加萸肉、生麦芽各三钱,再服数剂以善后。 
帮助 医者救危险将脱之证喜用人参,而喻嘉言谓气若上脱,但知重用人参转令人气高不返,必 
重用赭石辅之始能奏效,此诚千古不磨之论也。此方中之用人参原非用其救脱,因此证 
真阴大亏,惟石膏与人参并用,独能于邪火炽盛之时立复真阴,此白虎加人参汤之实用也。至 
于萸肉,其补益气分之力远不如参,而其挽救气分之上脱则远胜于参。诚以肝主疏泄,人之元气甚 
虚者,恒因肝之疏泄过甚而上脱,重用萸肉以敛肝使之不复疏泄,则元气之欲上脱者即可不脱,此愚屡 
次用之奏效而确知其然者也。 


<篇名>2.温病兼气虚气郁
属性:天津迟氏妇,年二十二岁,于季秋得温病。 
病因 其素日血分不调,恒作灼热,心中亦恒发热,因热贪凉,薄受外感,即成温病。 
证候 初受外感时,医者以温药发其汗,汗出之后,表里陡然大热,呕吐难进饮食,饮水亦 
恒吐出,气息不调,恒作呻吟,小便不利,大便泄泻日三四次,其舌苔薄而黄,脉象似有力 
而不实,左部尤不任重按,一分钟百零二至,摇摇有动象。 
诊断 其胃中为热药发表所伤,是以呕吐,其素日阴亏,肝肾有热,又兼外感之热内迫,致小 
便不利水归大肠是以泄泻,其舌苔薄而黄者,外感原不甚剧(舌苔薄,亦主胃气虚),而治以滋阴、 
清热、上止呕吐、下调二便之剂。 
处方 生怀山药(一两) 滑石(八钱) 生杭芍(八钱) 生怀地黄(六钱) 
清半夏(五钱温水洗三次) 碎竹茹(三钱) 生麦芽(三钱) 净青黛(二钱) 
连翘(二钱) 甘草(三钱) 鲜茅根(四钱) 
药共十一味,先将前十味水煎十余沸,再入茅根同煎七八沸,其汤即成,取清汤两盅,分三次温 
饮下。服医药后防其呕吐可口含生姜一片,或于煎药时加生姜三片亦可。至药局中 
若无鲜茅根,可用干茅根两半煎汤,以之代水煎药。 
方解 方中之义,山药与滑石并用,一滋阴以退热而能固大便,一清火以退热而善利小便; 
芍药与甘草并用,为甘草芍药汤,仲师用之以复真阴,而芍药亦善利小便,甘草亦善补大 
便,汇集四味成方,即拙拟之滋阴清燥汤也。以治上有燥热下焦滑泻之证,莫不随手奏效。半 
夏善止呕吐,然必须洗净矾味(药局清半夏亦有矾),屡洗之则药力减,是以用至五钱。竹茹亦 
善止呕吐,其碎者为竹之皮,津沽药局名为竹茹粉,其止呕之力较整者为优。至于青黛、生姜亦 
止呕吐之副品也。用生麦芽、鲜茅根者,以二药皆善利小便,而又善达肝木之郁以调气分也。用生 
地黄者,以其为滋补真阴之主药,即可为治脉数动摇者之要药也。 
复诊 将药煎服一剂,呕吐与泄泻皆愈,小便已利,脉象不复摇摇,仍似有力,至数未减,其表里 
之热稍退,气息仍似不顺,舌苔仍黄,欲投以重剂以清其热,犹恐大便不实,拟再治以清解之剂。 
处方 生怀地黄(一两) 玄参(八钱) 生杭芍(六钱) 天花粉(六钱) 
生麦芽(三钱) 鲜茅根(三钱) 滑石(三钱) 甘草(三钱) 
共煎汤一大盅,分两次温服下。 
三诊 将药煎服后,病又见轻,家人以为病愈无须服药矣,至翌日晚十一点钟后,见其面红,精 
神昏愦,时作呻吟,始知其病犹未愈。及愚诊视时,夜已过半,其脉左右皆弦硬而长,数近七至,两 
目直视,其呻吟之声,似阻隔不顺,舌苔变黑,问其心中何如?自言热甚,且觉气息不接续,此其气分 
虚而且郁,又兼血虚阴亏,而阳明之热又炽盛也。其脉近七至者,固为阴虚有热之象,而正气虚损 
不能抗拒外邪者,其脉亦恒现数象,至其脉不为洪滑而为弦硬者,亦气血两亏邪热炽盛之现象也。拟 
用白虎加人参汤,再加滋阴理气之品,盖此时大便已实,故敢放胆治之。 
处方 生石膏(五两轧细) 野台参(六钱) 知母(六钱) 天花粉(六钱) 
玄参(六钱) 生杭芍(五钱) 生莱菔子(四钱捣碎) 生麦芽(三钱) 
鲜茅根(三钱) 粳米(三钱) 甘草(三钱) 
共煎汤一大碗,分四次温饮下,病愈不必尽剂。 
效果 将药分四次服完,热退强半,精神已清,气息已顺,脉象较前缓和,而大便犹未通下, 
因即原方将石膏改用四两,莱菔子改用二钱,如前煎服,服至三次后,大便通下,其热全退,遂停后服。 
帮助 愚用白虎加人参汤,或以玄参代知母(产后寒温证用之)、或以芍药代知母(寒温兼下痢 
者用之)、或以生地黄代知母(寒温兼阴虚者用之)、或以生山药代粳米(寒温热实下焦气化不固者用之、 
产后寒温证用之),又恒于原方之外,加生地黄、玄参、沙参诸药以生津液,加鲜茅根、芦根、生 
麦芽诸药以宣通气化,初未有加莱菔子者,惟此证之气分虚而且郁,白虎汤中加人参可补其 
气分之虚,再加莱菔子更可理其气分之郁也。至于莱菔子必须生用者,取其有升发之力也。又须 
知此证不治以白虎汤而必治以白虎加人参汤者,不但为其气分虚也,凡人外感之热炽盛,真阴 
又复亏损,此乃极危险之证,此时若但用生地黄、玄参诸滋阴之品不能奏效,即将此 
等药加于白虎汤中亦不能奏效,惟生石膏与人参并用,独能于邪热炽盛之时立复真阴,此所 
以伤寒汗吐下后与渴者治以白虎汤时,仲圣不加他药而独加人参也。 


<篇名>3.温病兼吐泻腿抽
属性:族侄××,年五十三岁,于仲春下旬得温病兼吐泻,腿筋抽缩作疼。 
病因 素为腿筋抽疼病,犯时即卧床不能起,一日在铺中,旧病陡发,急乘车回寓,因腿疼 
出汗在路受风,遂成温病,继又吐泻交作。 
证候 表里俱壮热,呕吐连连不止,饮水少许亦吐出,一日夜泻十余次。得病已三日,小便 
滴沥全无,腿疼剧时恒作号呼,其脉左部浮弦似有力,按之不实。右部则弦长有力,重按甚 
硬,一息逾五至。 
诊断 此证因阴分素亏血不荣筋,是以腿筋抽疼。今又加以外感之壮热,传入阳明以灼耗其 
阴分,是以其脉象不为洪滑有力而为弦硬有力,此乃火盛阴亏之现象也。其作呕吐者,因其 
右脉弦硬且长,当有冲气上冲,因致胃气不下行而上逆也。其小便不利大便滑泻者,因阴虚 
肾亏不能漉水,水归大肠是以下焦之气化不能固摄也。当用拙拟滋阴宣解汤以清热、滋 
阴、调理二便,再加止呕吐及舒筋定疼之品辅之。 
处方 生怀山药(一两) 滑石(一两) 生杭芍(一两) 清半夏(四钱温水淘三次) 
碎竹茹(三钱) 净青黛(二钱) 连翘(钱半) 蝉蜕(钱半) 
甘草(三钱) 全蜈蚣(大者一条为末) 
药共十味,将前九味煎汤一大盅,送服蜈蚣细末,防其呕吐 
俾分三次温服,蜈蚣末亦分三次送服,服后口含生姜片以防恶心。 
方解 方中用蝉蜕者,不但因其能托邪外出,因蝉之为物饮而不食,有小便无大便,是以其蜕亦有 
利小便固大便之力也。用蜈蚣者,因其原善理脑髓神经,腿筋之抽疼,固由于肝血虚损不能荣 
筋,而与神经之分支在腿者,实有关系,有蜈蚣以理之,则神经不至于妄行也。 
复诊 将药服后呕吐未止,幸三次所服之药皆未吐出,小便通下两次,大便之泻全止,腿 
疼已愈强半,表里仍壮热,脉象仍弦长有力。为其滑泻已愈,拟放胆用重剂以清阳明之热,阳 
明胃之热清,则呕吐当自止矣。 
处方 生石膏(三两捣细) 生怀山药(两半) 生怀地黄(一两) 生杭芍(五钱) 
滑石(五钱) 碎竹茹(三钱) 甘草(三钱) 
共煎汤一大碗,分四次温饮下。 
方解 按用白虎汤之定例,凡在汗吐下后当加人参。此方中以生 
地黄代知母、生山药代粳米,与石膏、甘草同用,斯亦白虎汤也。而不加人参者,以其吐犹未止,加之恐助 
胃气上升,于斯变通其方,重用生山药至两半,其冲和稠粘之液,既可代粳米和胃,其培脾滋肾之功,又可代 
人参补益气血也。至于用白虎汤而复用滑石、芍药者,因二药皆善通利小便,防其水饮仍归大肠也。且芍药 
与甘草同用名甘草芍药汤,仲圣用以复真阴,前方之小便得通,实芍药之功居多(阴虚小便不利者,必重用芍 
药始能奏效)。矧弦为肝脉,此证之脉象弦硬,肝经必有炽盛之热,而芍药能生肝血、退肝热,为柔肝之 
要药,即为治脉象弦硬之要药也。 
三诊 将药分四次服完,表里之热退强半,腿疼全愈,脉象亦较前缓和,惟呕吐未能全愈,犹恶 
心懒进饮食,幸其大便犹固。俾先用生赭石细末两半,煎汤一盅半,分三次温饮下,饮至第二次后,觉 
胃脘开通,恶心全无,遂将赭石停饮,进稀米粥一大瓯,遂又为疏方以清余热。 
处方 生石膏(一两捣细) 生怀山药(一两) 生怀地黄(一两) 生杭芍(六钱) 甘草(二钱) 
共煎汤两盅,分两次温服下。 
效果 将药两次服完,表里之热全消,大便通下一次,病遂脱然全愈。惟其脉一息犹五至,知 
其真阴未尽复也。俾用生怀山药轧细过罗,每用七八钱或两许,煮粥调以蔗糖,当点心服之。若 
服久或觉发闷,可以送服西药百布圣五分,若无西药处,可用生鸡内金细末三分代之。 


<篇名>4.温病少阴证
属性:刘××,二十五岁,于季春得温病。 
病因 自正二月间,心中恒觉发热,懒于饮食,喜坐房阴乘凉,薄受外感,遂成温病。 
证候 初得病时,延近处医者延医,阅七八日病势益剧,精神昏愦,闭目蜷卧,似睡非睡,懒 
于言语,咽喉微疼,口唇干裂,舌干而缩,薄有黄苔欲黑,频频饮水不少濡润,饮食懒进, 
一日之间,惟强饮米汤瓯许,自言心中热而且干,周身酸软无力,抚其肌肤不甚发热,体温37. 8℃ 
其脉六部皆微弱而沉,左部又兼细,至数如常,大便四日未行,小便短少赤涩。 
诊断 此伏气触发于外,感而成温,因肾脏虚损而窜入少阴也。《内经》谓∶“冬伤于寒,春必病 
温”,此言冬时所受之寒甚轻,不能实时成为伤寒,恒伏于三焦脂膜之中,阻塞气化之升降,暗生内热,至春 
阳萌动之时,其所生之热恒激发于春阳而成温。然此等温病未必入少阴也。《内经》又谓∶“冬不藏精, 
春必病温”,此言冬不藏精之人,因阴虚多生内热,至春令阳回其内热必益加增,略为外感激发,即可 
成温病。而此等温病亦未必入少阴也。惟其人冬伤于寒又兼冬不藏精,其所伤之寒伏于三焦,随春阳而化热, 
恒因其素不藏精乘虚而窜入少阴,此等证若未至春令即化热窜入少阴,则为少阴伤寒,即伤寒少阴证二三日 
以上,宜用黄连阿胶汤者也。若已至春令始化热窜入少阴,当可名为少阴温病,即温病中内有实热,脉转微细 
者也。诚以脉生于心,必肾阴上潮与心阳相济,而后其跳动始有力。盖此证因温邪窜入少阴,俾心肾 
不能相济,是以内虽蕴有实热,而脉转微细,其咽喉疼者,因少阴之脉上通咽喉,其热邪循经上逆也。其唇 
裂舌干而缩者,肾中真阴为邪热遏抑不能上潮,而心中之亢阳益妄动上升以铄耗其津液也。至于心中发热且发 
干,以及大便燥结小便赤涩,亦无非阴亏阳亢之所致。为其肾阴心阳不能相济为功,是以精神昏愦,闭目蜷 
卧,烦人言语,此乃热邪深陷气化隔阂之候,在温病中最为险证。正不可因其脉象无火,身 
不甚热,而视为易治之证也。愚向拟有坎离互根汤可为治此病的方,今将其方略为加减俾与病候相宜。 
处方 生石膏(三两轧细) 野台参(四钱) 生怀地黄(一两) 生怀山药(八钱) 
玄参(五钱) 辽沙参(五钱) 甘草(三钱) 鲜茅根(五钱) 
药共八味,先将前七味煎十余沸,再入鲜茅根煎七八沸其汤即成。取清汤三盅,分三次温服下,每 
服一次调入生鸡子黄一枚。此方若无鲜茅根,可用干茅根两半,水煮数沸,取其汤代水煎药。 
方解 温病之实热,非生石膏莫解,辅以人参并能解邪实正虚之热,再辅以地黄、山药诸滋阴 
之品,更能解肾亏阴虚之热。且人参与滋阴之品同用,又能助肾阴上潮以解上焦之燥热。用鸡子黄者, 
化学家谓鸡子黄中含有副肾髓质之分泌素,为滋补肾脏最要之品也。用茅根者,其凉而能散,用之作引, 
能使深入下陷之邪热上出外散以消解无余也。 
复诊 将药三次服完,周身之热度增高,脉象较前有力,似近洪滑,诸病皆见轻减,精神已振。 
惟心中仍觉有余热,大便犹未通下,宜再以大剂凉润之药清之,而少佐以补气之品。 
处方 生石膏(一两轧细) 大潞参(三钱) 生怀地黄(一两) 玄参(八钱) 
辽沙参(八钱) 大甘枸杞(六钱) 甘草(二钱) 鲜茅根(四钱) 
药共八味,先将前七味煎十余沸,再入茅根煎七八沸其汤即成。取清汤两大盅,分两次温服下,每服一 
次调入生鸡子黄一枚。 
效果 将药连服两剂,大便通下,病遂全愈。 
帮助 此证之脉象沉细,是肾气不能上潮于心,而心肾不交也。 
迨服药之后,脉近洪滑,是肾气已能上潮于心而心肾相交也。为其心肾相交,是以诸病 
皆见轻减,非若寻常温病其脉洪大为增剧也。 


<篇名>5.温病结胸
属性:天津张姓叟,年近五旬,于季夏得温热结胸证。 
病因 心有忿怒,继复饱食,夜眠又当窗受风,晨起遂觉头疼发热,心下痞闷,服药数次病益进。 
证候 初但心下痞闷,继则胸膈之间亦甚痞塞,且甚烦热,其脉左部沉弦,右部沉牢。 
诊断 寒温下早成结胸,若表有外感,里有瘀积,不知表散药与消积药并用,而专事开破以 
消其积,则外感乘虚而入亦可成结胸。审证察脉,其病属结胸无疑,然其结之非剧,本陷胸 
汤之义而通变治之可也。 
处方 病者旬余辍工,家几断炊,愚怜其贫,为拟简便之方,与以自制通彻丸(即牵牛轧取头次末, 
水泛为小丸)五钱及自制离中丹两半,俾先服通彻丸三钱,迟一点半钟,若不觉药力猛烈,再服下所余 
二钱,候须臾再服离中丹三钱,服后多饮开水,俾出汗。若痞塞开后,仍有余热者,将所余离中 
丹分数次徐徐服之,每服后皆宜多饮开水取微汗。 
效果 如法将两种药服下,痞塞与烦热皆愈。 


<篇名>6.温病结胸
属性:天津赵××,年四十二岁,得温病结胸证。 
病因 季春下旬,因饭后有汗出受风,翌日头疼,身热无汗,心中发闷,医者外散其表热,内 
攻其发闷,服药后表未汗解而热与发闷转加剧。医者见服药无效,再疏方时益将攻破之药 
加重,下大便一次,遂至成结胸证。 
证候 胸中满闷异常,似觉有物填塞,压其气息不能上达,且发 
热嗜饮水,小便不利,大便日溏泻两三次。其脉左部弦长, 
右部中分似洪而重按不实,一息五至强。 
诊断 此证因下早而成结胸,又因小便不利而致溏泻,即其证脉合参,此乃上实下虚外感之热 
兼挟有阴虚之热也。治之者宜上开其结,下止其泻,兼清其内伤外感之热庶可奏效。 
处方 生怀山药(一两五钱) 生莱菔子(一两捣碎) 滑石(一两) 生杭芍(六钱) 
甘草(三钱) 
共煎汤一大盅,温服。 
复诊 服药后上焦之结已愈强半,气息颇形顺适,灼热亦减,已不感渴,大便仍溏,服药后下 
一次,脉象较前平和仍微数,遂再即原方略加减之。 
处方 生怀山药(一两五钱) 生莱菔子(八钱捣碎) 滑石(八钱) 生杭芍(五钱) 甘草(三钱) 
先用白茅根(鲜者更好)、青竹茹各二两,同煎数沸,取汤以之代水煎药。 
效果 将药煎服后,诸病皆愈,惟大便仍不实,俾每日用生怀山药细末两许,水调煮作茶汤, 
以之送服西药百布圣五分,充作点心以善其后。 


<篇名>7.温病
属性:天津俞××,年过四旬,于孟夏得温病。 
病因 与人动气争闹,头面出汗为风所袭,遂成温病。 
证候 表里俱发热,胸膈满闷有似结胸,呼吸甚觉不利,夜不能寐,其脉左右皆浮弦有力,舌 
苔白浓,大便三日未行。 
诊断 此病系在太阳而连及阳明少阳也。为其病在太阳,所以脉浮;为其连及阳明,所以按 
之有力;为其更连及少阳,是以脉浮有力而又兼弦也。其胸膈满闷呼吸不利者,因其怒气溢 
于胸中,挟风邪痰饮凝结于太阳部位也。宜外解太阳之表, 
内清阳明之热,兼和解其少阳,更开荡其胸膈,方为完全之策。 
处方 生石膏(二两捣细) 蒌仁(二两炒捣) 生莱菔子(八钱捣碎) 天花粉(六钱) 
苏子(三钱炒捣) 连翘(三钱) 薄荷叶(二钱) 茵陈(二钱) 龙胆草(二钱) 甘草(二钱) 
共煎汤一大盅,温服后,复衾取微汗。 
效果 服药后阅一小时,遍身得汗,胸次豁然,温热全消,夜能安睡,脉已和平如常,惟大便 
犹未通下,俾但用西药旃那叶一钱,开水浸服两次,大便遂通下。 


<篇名>8.风温
属性:邑北境赵××,年近三旬,于孟秋得风温病。 
病因 孟秋下旬,农人忙甚,因劳力出汗过多,复在树阴乘凉过度,遂得风温病。 
证候 胃热气逆,服药多呕吐。因此屡次延医服药,旬余无效。及愚诊视,见其周身壮热,心中 
亦甚觉热,五六日间饮食分毫不进,大便数日未行。问何不少进饮食?自言有时亦思饮食,然一切 
食物闻之皆臭恶异常,强食之即呕吐,所以不能食也。诊其脉弦长有力,右部微有洪象,一息五至。 
诊断 即此证脉相参,知其阳明腑热已实,又挟冲气上冲,所以不能进食,服药亦多呕也。欲治 
此证当以清胃之药为主,而以降冲之药辅之。则冲气不上冲,胃气亦必随之下降,而呕 
吐能止即可以受药进食矣。 
处方 生石膏(三两捣细) 生赭石(一两轧细) 知母(八钱) 潞党参(四钱) 
粳米(三钱) 甘草(二钱) 
共煎汤一大碗,分三次温服下。 
方解 此方乃白虎加人参汤又加赭石,为其胃腑热实故用白虎汤,为其呕吐已久故加人参,为其 
冲胃上逆故又加赭石也。 
效果 将药三次服完,呕吐即止,次日减去赭石,又服一剂,大便通下,热退强半。至第三日减 
去石膏一两,加玄参六钱,服一剂,脉静身凉,而仍分毫不能饮食,憎其臭味如前。愚晓 
其家人曰∶此病已愈,无须用药,所以仍不饮食者,其胃气不开也。胃之食物莫如莱菔,可用鲜 
莱菔切丝香油炒半熟,而以葱酱作汤勿过熟,少调以绿豆粉俾服之。至汤作熟时,病患仍不肯服, 
迫令尝少许,始知香美,须臾服尽两碗,从此饮食复常。病患谓其家人曰∶吾从前服药十余剂,病未见 
愈,今因服莱菔汤而霍然全愈,若早知莱菔汤能如此治病,则吾之病不早愈乎?其家人不觉失笑。 
附记 曾记弱冠时,比邻有病外感痰喘者,延邑中老医皮××,投以小青龙汤一剂喘即愈,然觉 
胸中似有雾气弥漫不能进食。皮××曰,此乃湿气充盛,是以胃气不开也,此当投以开胃之剂。为疏 
方,用《金匮》苓桂术甘汤,煎服后未半刻,陡觉胸中阴霾顿开,毫无障碍,遂能进食,见者皆惊其 
用药之神奇。夫皮君能如此用药,诚无愧名医之目。而益叹经方之神妙,诚有不可令人思议者矣。 
此因一用莱菔,一用古方,均开胃于顷刻之间,故附志之。 


<篇名>9.风温兼伏气化热
属性:天津陈××,年四十六岁,得风温兼伏气化热病。 
病因 因有事乘京奉车北上时,当仲夏归途受风,致成温热病。 
证候 其得病之翌日,即延为诊视,起居如常,惟觉咽喉之间有热上冲,咳嗽吐痰音微哑,周身 
似拘束酸软。脉象浮而微滑,右关重按甚实,知其证虽感风成温,而其热气之上冲咽 
喉,实有伏气化热内动也。若投以拙拟寒解汤原可一汗而愈。然当此病之初起而遽投以石膏重剂, 
彼将疑而不肯服矣。遂迁就为之拟方。盖医以救人为目标,正不妨委曲以行其道也。 
处方 薄荷叶(三钱) 青连翘(三钱) 蝉蜕(二钱) 知母(六钱) 
玄参(六钱) 天花粉(六钱) 甘草(二钱) 
共煎汤一大盅,温服。 
复诊 翌日复延为诊视,言服药后周身得微汗,而表里反大热,咳嗽音哑益甚,何以服如此凉药 
而热更增加,将毋不易治乎?言之若甚恐惧者。诊其脉洪大而实,左右皆然,知非重用石膏不可。因 
谓之曰∶此病乃伏气化热,又兼有新感之热,虽在初得亦必须用石膏清之方能治愈。若果能用生石膏 
四两,今日必愈,吾能保险也。问石膏四两一次全服乎?答曰∶非也。可分作数次服,病愈则停 
服耳。为出方,盖因其有恐惧之心,故可使相信耳。 
处方 生石膏(四两捣细) 粳米(六钱) 
共煎汤至米熟,取汤四盅,分四次徐徐温饮下。病愈不必尽剂,饮至热退而止。大便若有滑泻,尤宜将药急停服。 
复诊 翌日又延为诊视,相迎而笑曰∶我今热果全消矣,惟喉间似微觉疼,先生可再为治之。问 
药四盅全服乎?答曰∶全服矣。当服至三盅后,心犹觉稍热,是以全服,且服后并无大 
便滑泻之病,石膏真良药也。再诊其脉已平和如常,原无须服药,问其大便,三日犹未下行。为开 
滋阴润便之方,谓服至大便通后,喉疼亦必自愈,即可停药勿服矣。 


<篇名>10.温病兼痧疹
属性:天津舒××,年四十五岁,于仲夏得温病兼痧疹。 
病因 舒××原精医术,当温疹流行之时,屡次出门为人诊病,受其传染因得斯病。 
证候 其前数日皆系自治,屡次服表疹清热之药,疹已遍身出齐而热仍不退,因求愚为延医。其 
表里俱觉发热,且又烦躁异常,无片时宁静,而其脉则微弱不起,舌苔薄而微黄,大便 
日行一次不干不溏,小便赤涩短少。 
诊断 此证当先有伏气化热,因受外感之传染而激发,缘三焦脂膜窜入少阴遏抑肾气,不能 
上与心火相济,是以舌苔已黄,小便短赤,阳明腑热已实,而其脉仍然无力也。其烦躁异常 
者,亦因水火之气不相交也。此虽温病,实与少阴伤寒之热者无异,故其脉亦与少阴伤寒之 
脉同。当治以白虎加人参汤,将原方少为变通,而再加托表疹毒之品辅之。 
处方 生石膏(二两捣细) 大潞参(四钱) 天花粉(八钱) 生怀山药(八钱) 
鲜茅根(四钱) 甘草(二钱) 
共煎汤两盅分两次温服下。 
方解 此方即白虎加人参汤以花粉代知母,生山药代粳米,而又加鲜茅根也。花粉与知母,皆 
能清热,而花粉于清热之外又善解毒,山药与粳米皆能和胃,而山药于和胃之外又能滋 
肾。方中之义,用白虎汤以治外感实热,如此变通则兼能清其虚热解其疹毒,且又助以人参 
更可治证实脉虚之热,引以鲜茅根并可治温病下陷之热也。 
复诊 将药煎服一剂,热退强半,烦躁亦大轻减,可安睡片时。至翌日过午,发热烦躁又如 
旧,脉象仍然无力,因将生石膏改用三两,潞参改用五钱,俾煎汤三盅,分三次温饮下。每 
饮一次,调入生鸡子黄一枚。服后其病亦见愈,旋又反复,且其大便一日两次,知此寒凉 
之药不可再服。乃此时愚恍然会悟,得治此证之的方矣。 
处方 鲜白茅根(六两切碎) 
添凉水五盅,在炉上煎一沸,即将药罐离开炉眼,约隔三寸许,迟十分钟再煎一沸,又离开炉眼, 
再迟十分钟,视其茅根皆沉水底其汤即成。若茅根不沉水底,可再煎一沸,约可 
取清汤三盅,乘热顿饮之以得微汗方佳。 
效果 此方如法服两剂,其病脱然愈矣。 
帮助 按此证其伏气之化热,固在三焦,而毒菌之传染,实先受于上焦,于斯毒热相并随上焦之 
如雾而弥漫于全身之脏腑经络不分界限。茅根凉而能散,又能通达经络脏腑无微不至。 
惟性甚平和,非多用不能奏效。是以一剂重用至六两,其凉散之力,能将脏腑经络间之毒热尽数排 
出(茅根能微汗利小便,皆其排出之道路),毒热清肃,烦躁自除矣。愚临证五十年,用白虎加人参汤时 
不知凡几,约皆随手奏效。今此证两次用之无效,而竟以鲜白茅根收其功,此非愚所素知,乃因 
一时会悟后则屡次用之皆效,故特详之以为治温疹者开一法门也。若其脉象洪滑甚 
实者,仍须重用石膏清之,或石膏、茅根并用亦可。又按白茅根必须用鲜者,且必如此煎法方效。但 
根据之成功多用可至十两,少用亦须至四两,不然此证前两方中皆有茅根四钱未见效验,其宜多用可 
知矣。又药局中若无鲜者,可自向洼中剖之,随处皆有。若剖多不能一时皆用,以湿土埋之永久不坏。 


<篇名>11.温病兼劳力过度
属性:族弟××,年三十八岁,于孟夏来津于旅次得温病。 
病因 时天气炎热,途中自挽鹿车,辛苦过力,出汗受风,至津遂成温病。 
证候 表里俱觉甚热,合目恒谵语,所言多劳力之事。舌苔白浓,大便三日未行,脉象左部弦硬,右部 
洪实而浮,数逾五 
至。 
诊断 此证因长途炎热劳碌,脏腑间先有积热,又为外感所袭,则其热陡发。其左脉弦硬者, 
劳力过度肝肾之阴分有伤也。右部洪实者,阳明之腑热已实也。其洪实兼浮者,证犹连表 
也。拟治以白虎加人参汤以玄参代知母,生山药代粳米,更辅以透表之药以引热外出。 
处方 生石膏(三两捣细) 大潞参(四钱) 玄参(一两) 生怀山药(六钱) 
甘草(三钱) 西药阿斯匹林(一瓦) 
将前五味共煎汤两大盅,先温服一盅,迟半点钟将阿斯匹林用开水送下,俟汗出后再将所 
余一盅分两次温服下。 
效果 将药服一盅后,即不作谵语,须臾将阿斯匹林服下,遍体 
得汗,继又将所余之汤药徐徐服下,其病霍然全愈。 
帮助 白虎汤中以石膏为主药,重用至三两,所以治右脉之洪实也;于白虎汤中加人参更以玄 
参代知母,生山药代粳米,退热之中大具滋阴之力(石膏、人参并用,能于温寒大热之际,立复真阴), 
所以治左脉之弦硬也。用药如用兵,料敌详审,步伍整齐,此所以战则必胜也。至于脉象 
兼浮,知其表证未罢,犹可由汗而解,遂佐以阿斯匹林之善透表者以引之出汗,此所谓因其病机而利 
导之也。若无阿斯匹林之处,于方中加薄荷叶一钱,连翘二钱,亦能出汗。 


<篇名>12.温病兼下痢
属性:天津范姓媪,年过五旬,得温病兼下痢证。 
病因 家务劳心,恒动肝火,时当夏初,肝阳正旺,其热下迫,遂患痢证。因夜间屡次入厕, 
又受感冒兼发生温病。 
证候 表里皆觉发热,时或作渴,心中烦躁,腹中疼甚剧,恒作 
呻吟。昼夜下痢十余次,旬日之后系纯白痢,其舌苔浓欲 
黄,屡次延医服药,但知治痢且用开降之品,致身体虚弱卧 
不能起,其脉左右皆弦而有力,重按不实,搏近五至。 
诊断 此病因肝火甚盛,兼有外感之热已入阳明,所以脉象弦而有力。其按之不实者,因从 
前服开降之药过多也。其腹疼甚剧者,因弦原主疼,兹则弦而且有力,致腹中气化不和故疼 
甚剧也。其烦躁者,因下久阴虚,肾气不能上达与心相济,遂不耐肝火温热之灼耗,故觉烦躁也。 
宜治以清温凉肝之品,而以滋阴补正之药辅之。 
处方 生杭芍(一两) 滑石(一两) 生怀山药(一两) 天花粉(五钱) 
山楂片(四钱) 连翘(三钱) 甘草(三钱) 
共煎汤一大盅,温服。 
复诊 将药煎服一剂,温热已愈强半,下痢腹疼皆愈,脉象亦见和缓,拟再用凉润滋阴之剂,以清其余热。 
处方 生怀山药(一两) 生杭芍(六钱) 天花粉(五钱) 生怀地黄(五钱) 
玄参(五钱) 山楂片(三钱) 连翘(二钱) 甘草(二钱) 
共煎汤一大盅,温服。 
效果 将药连服两剂,病遂全愈。惟口中津液短少,恒作渴,运动乏力,俾用生怀山药细末煮作 
茶汤,兑以鲜梨自然汁,当点心服之,日两次,浃辰之间当即可撤消矣。盖山药原善滋阴,而其补 
益之力又能培养气化之虚耗。惟其性微温,恐与病后有余热者稍有不宜,借鲜梨自然汁之凉润 
以相济为用,则为益多矣。 


<篇名>13.温病兼脑膜炎
属性:天津侯姓幼男,年八岁,得热病兼脑膜炎。 
病因 蒙学暑假乍放,幼童贪玩,群在烈日中HT 戏,出汗受风,遂得斯证。 
证候 闭目昏昏,呼之不应,周身灼热无汗,其脉洪滑而长,两寸尤盛。其母言病已三日, 
昨日犹省人事,惟言心中发热,至夜间即昏无知觉。然以水灌之犹知下咽,问其大便三日未行。 
诊断 此温热之病,阳明腑热已实,其热循经上升兼发生脑膜炎也。脑藏神明主知觉,神经因 
热受伤,是以知觉全无,宜投以大剂白虎汤以清胃腑之热,而复佐以轻清之品,以引药之 
凉力上行,则脑中之热与胃腑之热全清,神识自明了矣。 
处方 生石膏(三两捣细) 知母(八钱) 连翘(三钱) 茵陈(钱半) 
甘草(三钱) 粳米(五钱) 
煎至米熟其汤即成。取清汁三茶杯,徐徐分三次温服,病愈无须尽剂。 
效果 服至两次已明了能言,自言心中犹发热,将药服完,其热遂尽消,霍然全愈。 


<篇名>14.温热泄泻
属性:天津钱姓幼男,年四岁,于孟秋得温热兼泄泻,病久不愈。 
病因 季夏感受暑温,服药失宜,热留阳明之腑,久则灼耗胃 
阴,嗜凉且多嗜饮水,延至孟秋,上热未清,而下焦又添泄泻。 
证候 形状瘦弱已极,周身灼热,饮食少许则恶心欲呕吐。小便不利,大便一昼夜十余次,多系稀水, 
卧不能动,哭泣无声,脉数十至且无力(四岁时,当以七至为正脉),指纹现淡红色,已透气关。 
诊断 此因外感之热久留耗阴,气化伤损,是以上焦发热懒食, 
下焦小便不利而大便泄泻也。宜治以滋阴、清热、利小便兼固大便之剂。 
处方 生怀山药(一两五钱) 滑石(一两) 生杭芍(六钱) 甘草(三钱) 
煎汤一大盅,分数次徐徐温服下。 
方解 此方即拙拟滋阴清燥汤也。原方生山药是一两,今用两半者,因此幼童瘦弱已极,气化 
太虚也。方中之义,山药与滑石同用,一利小便,一固大便,一滋阴以退虚热,一泻火以 
除实热。芍药与甘草同用,甘苦化合,味近人参,能补益气化之虚损。而芍药又善滋肝肾以利小 
便,甘草又善调脾胃以固大便,是以汇集而为一方也。 
效果 将药连服两剂,热退泻止,小便亦利,可进饮食,惟身体羸瘦不能遽复。俾用生怀 
山药细末七八钱许,煮作粥,调以白糖,作点心服之。且每次送西药百布圣一瓦,如此将养月余始胖壮。 


<篇名>15.温病兼虚热
属性:山西高××,年二十八岁,客居天津,于仲秋得温病。 
病因 朋友招饮,饮酒过度,又多喝热茶,周身出汗,出外受风。 
证候 周身骨节作疼,身热39.4℃,心中热而且渴,舌苔薄而微黄。大便干燥,小便短赤, 
时或干嗽,身体酸软殊甚,动则弦晕,脉数逾五至,浮弦无力。自始病至此已四十日矣, 
屡次延医服药无效。 
诊断 此证乃薄受外感,并非难治之证。因治疗失宜,已逾月而外表未解,内热自不能清。病 
则懒食,又兼热久耗阴,遂由外感之实热,酿成内伤之虚热,二热相并,则愈难治矣。斯 
当以大滋真阴之药为主,而以解表泻热之药佐之。 
处方 生怀山药(一两) 生怀地黄(一两) 玄参(一两) 沙参(六钱) 
生杭芍(六钱) 大甘枸杞(五钱) 天冬(五钱) 天花粉(五钱) 滑石(三钱) 甘草(三钱) 
共煎汤一大碗,分三次温饮下,其初饮一次时,先用白糖水送服西药阿斯匹林半瓦,然后服汤药。 
复诊 初服药一次后,周身得汗,骨节已不觉疼,二次三次继续服完,热退强半,小便通畅, 
脉已不浮弦,跳动稍有力,遂即原方略为加减,俾再服之。 
处方 生怀山药(一两) 生怀地黄(八钱) 玄参(六钱) 沙参(六钱) 
大甘枸杞(六钱) 天门冬(六钱) 滑石(三钱) 甘草(二钱) 真阿胶(三钱捣碎) 
药共九味,先将前八味煎汤两大盅,去渣入阿胶融化,分两次温服。其服初次时,仍先用白糖 
水送服阿斯匹林三分之一瓦。此方中加阿胶者,以其既善滋阴,又善润大便之干燥也。 
效果 将药先服一次,周身又得微汗,继将二分服下,口已不渴,其日大便亦通下,便 
下之后,顿觉精神清爽,灼热全无,病遂从此愈矣。 
帮助 方中重用大队凉润之品,滋真阴即以退虚热,而复以阿斯匹林解肌、滑石利小便者,所以 
开实热之出路也。至于服阿斯匹林半瓦,即遍身得汗者,因体虚者其汗易出,而心有燥 
热之人,得凉药之濡润亦恒自出汗也。 


<篇名>16.温病体虚
属性:辽宁刘××幼子,年七岁,于暮春得温病。 
病因 因赴澡塘洗澡,汗出未竭,遽出冒风,遂成温病。病初得时,医者不知,用辛凉之药解 
饥,而竟用温热之药为发其汗,迨汗出遍体,而灼热转剧。又延他医遽以承气下之,病 
尤加剧,因其无可下之证而误下也。从此不敢轻于服药,迟 
延数日见病势浸增,遂延愚为诊视,其精神昏愦间作谵语,气息微喘,肌肤灼热。问其心中亦甚 
觉热,唇干裂有凝血,其舌苔薄而黄,中心干黑,频频饮水不能濡润。其脉弦而有 
力,搏近六至,按之不实,而左部尤不任重按,其大便自服药下后未行。 
诊断 此因误汗误下,伤其气化,兼温热既久阴分亏耗,乃邪实正虚之候也。宜治以大剂白虎 
加人参汤。以白虎汤清其热,以人参补其虚,再加滋阴之品数味,以滋补阴分之亏耗。 
处方 生石膏(四两捣细) 知母(一两) 野党参(五钱) 大生地黄(一两) 
生怀山药(七钱) 玄参(四钱) 甘草(三钱) 
共煎汤三大盅,分三次温饮下。病愈者勿须尽剂,热退即停服。白虎加人参汤中无粳米者, 
因方中有生山药可代粳米和胃也。 
效果 三次将药服完,温热大减,神已清爽。大使犹未通下,心中犹觉发热,诊其脉仍似有力, 
遂将原方去山药仍煎三盅,俾徐徐温饮下,服至两盅大便通下,遂停药勿服,病全愈。 


<篇名>17.温热腹疼兼下痢
属性:天津张姓媪,年过五旬,先得温病,腹疼即又下痢。 
病因 因其夫与子相继病,故屡次伤心,蕴有内热,又当端阳节后,天气干热非常,遂得斯证。 
证候 腹中搅疼,号呼辗转不能安卧,周身温热,心中亦甚觉热,为其卧不安枕,手足扰动, 
脉难细诊,其大致总近热象,其舌色紫而干,舌根微有黄苔,大便两日未行。 
诊断 此乃因日日伤心,身体虚损,始则因痛悼而脏腑生热,继则因热久耗阴而更生虚热,继 
又因时令之燥热内侵与内蕴之热相并,激动肝火下迫腹中,是以作疼,火热炽盛,是以表 
里俱觉发热。此宜清其温热,平其肝火,理其腹疼,更宜防其腹疼成痢也。 
处方 先用生杭芍一两、甘草三钱,煎汤一大盅,分两次温服。每次送服卫生防疫宝丹四十粒,约 
点半钟服完两次,腹已不疼。又俾用连翘一两、甘草三钱,煎汤一大盅,分作三次温 
服。每次送服拙拟离中丹三钱,嘱约两点钟温服一次。 
复诊 翌日晚三点钟,复为诊视,闭目昏昏,呼之不应。其家人言,前日将药服完里外之热皆 
觉轻减,午前精神颇清爽,午后又渐发潮热,病势一时重于一时。前半点钟呼之犹知答应, 
兹则大声呼之亦不应矣。又自黎明时下脓血,至午后已十余次,今则将近两点钟未见下矣。诊 
其脉左右皆似大而有力,重按不实,数近六至,知其身体本虚,又因屡次下痢,更兼 
外感实热之灼耗,是以精神昏愦,分毫不能支持也。拟放胆投以大剂白虎加人参汤,复即 
原方略为加减,俾与病机适宜。 
处方 生石膏(三两捣细) 野台参(五钱) 生杭芍(一两) 生怀地黄(一两) 
甘草(三钱) 生怀山药(八钱) 
共煎汤三盅,分三次徐徐温服下。 
此方系以生地黄代原方中知母,生山药代原方中粳米,而又加芍药。以芍药与方中甘 
草并用,即《伤寒论》中甘草芍药汤,为仲圣复真阴之妙方。而用于此方之中,又善治后重 
腹疼,为治下痢之要药也。 
复诊 将药三次服完后,时过夜半,其人豁然省悟,其家人言自诊脉疏方后,又下脓血数次,至 
将药服完,即不复下脓血矣。再诊其脉,大见和平,问其心中,仍微觉热,且觉心中 
怔忡不安。拟再治以凉润育阴之剂,以清余热,而更加保合气化之品,以治其心中怔忡。 
处方 玄参(一两) 生杭芍(六钱) 净萸肉(六钱) 生龙骨(六钱捣碎) 
生牡蛎(六钱捣碎) 沙参(四钱) 酸枣仁(四钱炒捣) 甘草(二钱) 
共煎汤两盅,分两次温服。每服一次,调入生鸡子黄一枚。 
效果 将药连服三剂,余热全消,心中亦不复怔忡矣。遂停服汤药,俾用生怀山药细末一两弱,煮 
作茶汤少兑以鲜梨自然汁,当点心服之以善其后。 
帮助 温而兼痢之证,愚治之多矣,未有若此证之剧者。盖此证腹疼至辗转号呼不能诊脉,不 
但因肝火下迫欲作痢也,实兼有外感毒疠之气以相助为虐。故用芍药以泻肝之热,甘草之 
缓肝之急,更用卫生防疫宝丹以驱逐外侵之邪气。迨腹疼已愈,又恐其温热增剧,故又俾用连翘、 
甘草煎汤,送服离中丹以清其温热,是以其证翌日头午颇见轻。若即其见轻时而早为之诊脉服药,原 
可免后此之昏沉,乃因翌日相延稍晚,竟使病势危至极点,后幸用药得宜,犹能挽回,然亦险矣。谚 
有“走马看伤寒”,言其病势更改之速也。至治温病亦何独不然哉。又此证过午所以如此加剧者,亦 
以其素本阴虚,又自黎明下痢脓血多次,则虚而益虚,再加以阴亏之虚热,与外感之实热相并, 
是以其精神即不能支持。所赖方中药味无多,而举凡虚热实热及下痢所生之热,兼顾无遗,且又煎一 
大剂分三次温饮下,使药力前后相继,此古人一煎三服之法。愚遵此法以挽回险证救人多矣。 
非然者则剂轻原不能挽回重病,若剂重作一次服病患又将不堪。惟将药多煎少服, 
病愈不必尽剂,此以小心行其放胆,洵为挽回险病之要着也。 


<篇名>18.温病兼下痢
属性:袁姓妇,年三十六岁,得温病兼下痢证。 
病因 仲秋乘火车赴保定归母家省视,往来辛苦,路间又兼受风,遂得温病兼患下痢。 
证候 周身壮热,心中热而且渴,下痢赤多白少,后重腹疼,一昼夜十余次,舌苔白浓,中心 
微黄,其脉左部弦硬,右部洪实,一息五至。 
诊断 此风温之热已入阳明之腑,是以右脉洪实,其炽盛之肝火下迫肠中作痢,是以左脉弦硬。 
夫阳明脉实而渴者,宜用白虎加入参汤,因其肝热甚盛,证兼下痢,又宜以生山药代粳米 
以固下焦气化,更辅以凉肝调气之品,则温与痢庶可并愈。 
处方 生石膏(三两捣细) 野党参(四钱) 生怀山药(一两) 生杭芍(一两) 知母( 
六钱) 白头翁(五钱) 生麦芽(四钱) 甘草(四钱) 
将药煎汤三盅,分三次温饮下。 
复诊 将药分三次服完,温热已退强半,痢疾已愈十之七八,腹已不疼,脉象亦较前和平,遂即原方 
略为加减俾再服之。 
处方 生石膏(二两捣细) 野台参(三钱) 生怀山药(八钱) 生杭芍(六钱) 
知母(五钱) 白头翁(五钱) 秦皮(三钱) 甘草(三钱) 
共煎汤两盅,分两次温服下。 
效果 将药煎服两剂,诸病皆愈,惟脉象似仍有余热,胃中似不开通懒于饮食。俾用鲜梨、鲜藕、 
莱菔三者等分,切片煮汁,送服益元散三钱许,日服两次,至三次则喜进饮食,脉亦和平如常矣。 
帮助 凡温而兼痢之证,最为难治。盖温随下痢深陷而永无出路,即痢为温热所灼而益加疼坠, 
惟石膏与人参并用,能升举下陷之温邪,使之徐徐上升外散。而方中生山药一味,在 
白虎汤中能代粳米以和胃,在治痢药中又能固摄下焦气化,协同芍药、白头翁诸药以润肝滋肾, 
从容以奏肤功也。至于麦芽炒用之为消食之品,生用之不但消食实能舒发肝气,宣散 
肝火,而痢病之后重可除也。至后方加秦皮者,取其性本苦寒,力善收涩,借之以清热补虚,原 
为痢病将愈最宜之品。是以《伤寒论》白头翁汤中亦借之以清厥阴热痢也。 


<篇名>19.温病兼下痢
属性:天津姚姓媪,年六旬有二,于孟秋得温病兼下痢。 
病因 孟秋天气犹热,且自觉心中有火,多食瓜果,又喜当风乘凉,遂致病温兼下痢。 
证候 周身灼热,心中热且渴,连连呻吟不止,一日夜下痢十二三次,赤白参半,后重腹疼, 
饮食懒进,恶心欲呕,其脉左部弦而兼硬,右部似有力而重按不实,数近六至。延医治疗近旬日,病益加剧。 
诊断 其左脉弦而兼硬者,肝血虚而胆火盛也。其右脉似有力而重按不实者,因其下痢久而气化 
已伤,外感之热又侵入阳明之腑也。其数六至者,缘外感之热灼耗已久,而其真阴大有 
亏损也。证脉合参,此乃邪实正虚之候。拟用拙定通变白虎加人参汤,及通变白头翁汤二方相并治之。 
处方 生石膏(二两捣细) 野台参(四钱) 生怀山药(一两) 生杭芍(一两) 
白头翁(四钱) 金银花(四钱) 秦皮(二钱) 生地榆(二钱) 
甘草(二钱) 广三七(二钱轧细) 鸦胆子(成实者五十粒去皮) 
共药十一味,先用白糖水送服三七、鸦胆子各一半,再将余药煎汤两盅,分两次温服下。至煎渣 
再服时,亦先服所余之三七、鸦胆子。 
复诊 将药煎服日进一剂,服两日表里之热皆退,痢变为泻,仍稍带痢,泻时仍觉腹疼后重 
而较前轻减,其脉象已近平和,此宜以大剂温补止其泄泻,再少辅以治痢之品。 
处方 生怀山药(一两) 炒怀山药(一两) 龙眼肉(一两) 大云苓片(三钱) 
生杭芍(三钱) 金银花(三钱) 甘草(二钱) 
共煎汤一大盅,温服。 
效果 将药煎服两剂,痢已净尽而泻未全愈,遂即原方去金银 
花、芍药,加白术三钱,服两剂其泻亦愈。 


<篇名>20.暑温兼泄泻
属性:天津侯姓学徒,年十三岁,得暑温兼泄泻。 
病因 季夏天气暑热,出门送药受暑,表里俱觉发热,兼头目眩晕。服药失宜,又兼患泄泻。 
证候 每日泄泻十余次,已逾两旬,而心中仍觉发热懒食,周身酸软无力,时或怔忡,小 
便赤涩发热,其脉左部微弱,右部重按颇实,搏近六至。 
诊断 此暑热郁于阳明之腑,是以发热懒食,而肝肾气化不舒, 
是以小便不利致大便泄泻也。当清泻胃腑,调补肝肾,病当自愈。 
处方 生怀山药(两半) 滑石(一两) 生杭芍(六钱) 净萸肉(四钱) 
生麦芽(三钱) 甘草(三钱) 
共煎汤一大盅,温服。 
复诊 服药一剂泻即止,小便通畅,惟心中犹觉发热,又间有怔忡之时,遂即原方略为加减俾再服之。 
处方 生怀山药(一两) 生怀地黄(一两) 净萸肉(八钱) 生杭芍(六钱) 
生麦芽(二钱) 甘草(二钱) 
共煎汤一大盅,温服。 
效果 将药连服两剂,其病霍然全愈。 
帮助 初次所用之方,即拙拟之滋阴清燥汤加山萸肉、生麦芽也。 
从来寒温之热传入阳明,其上焦燥热下焦滑泻者,最为难治, 
因欲治其上焦之燥热,则有碍下焦之滑泻;欲补其下焦之滑泻,则有碍上焦之燥热,是以医者对之恒至束手。 
然此等证若不急为治愈,则下焦滑泻愈久,上焦燥热必愈甚,是以本属可治之证,因稍为迟延竟至不可救 
者多矣。惟拙拟之滋阴清燥汤,山药与滑石并用,一补大便,一利小便。而山药多液,滑石性凉,又善清上焦之 
燥热,更辅以甘草、芍药以复其阴(仲景谓作甘草芍药汤以复其阴),阴复自能胜燥热,而芍药又善利小便, 
甘草亦善调大便,汇集四味为方,凡遇证之上焦燥热下焦滑泻者,莫不随手奏效也。间有阳明热实,服药后 
滑泻虽止而燥热未尽清者,不妨继服白虎汤。其热物理虚者,或服白虎加人参汤,若虑其复作滑泻,可于 
方中仍加滑石三钱,或更以生山药代粳米煎取清汤,一次只饮一大口,徐徐将药服完,其热全消,亦不至复 
作滑泻。愚用此法救人多矣,滋阴清燥汤后,附有治愈多案可参观也。至此案方中加萸肉、生麦芽 
者,因其肝脉弱而不舒,故以萸肉补之,以生麦芽调之,所以遂其条达之性也。至于第二方中为泻止小便已 
利,故去滑石。为心中犹怔忡,故将萸肉加重。为犹有余热未清,故又加生地黄。因其余热无多,如此治法已 
可消除净尽,无须服白虎汤及白虎加人参汤也。 


<篇名>21.温病
属性:武清县孙××,年三十三岁,于孟秋时得温病。 
病因 未病之前,心中常觉发热,继因饭后有汗,未暇休息,陡有急事冒风出门,致得温病。 
证候 表里俱觉壮热,嗜饮凉水、食凉物,舌苔白浓,中心已黄,大便干燥,小便短赤,脉象 
洪长有力,左右皆然,一分钟七十八至。 
诊断 此因未病之先已有伏气化热,或有暑气之热内伏,略为外感所激,即表里陡发壮热,一 
两日间阳明府热已实,其脉之洪长有力是明征也。拟投以大剂白虎汤,再少佐以宣散之品。 
处方 生石膏(四两捣细) 知母(一两) 鲜茅根(六钱) 青连翘(三钱) 
甘草(三钱) 粳米(三钱) 
共煎汤三盅,分三次温服下。 
复诊 将药分三次服完,表里之热分毫未减,脉象之洪长有力亦仍旧,大便亦未通下。此非 
药不对证,乃药轻病重药不胜病也。夫石膏之性《神农本草经》原谓其微寒,若遇阳明大热之 
证,当放胆用之。拟即原方去连翘加天花粉,再将石膏加重。 
处方 生石膏(六两) 知母(一两) 天花粉(一两) 鲜茅根(六钱) 
甘草(四钱) 粳米(四钱) 
共煎汤三大盅,分三次温服下。 
复诊 将药分三次服完,下燥粪数枚,其表里之热仍然不退,脉象亦仍有力。愚谓孙××曰∶余 
生平治寒温实热证,若屡次治以大剂白虎汤而其热不退者,恒将方中石膏研极细,将余药 
煎汤送服即可奏效。今此证正宜用此方,孙××亦以为然。 
处方 生石膏(二两研极细) 生怀山药(二两) 甘草(六钱) 
将山药、甘草煎汤一大碗,分多次温服。每次送服石膏末二钱许,热退勿须尽剂,即其热未尽退, 
若其大便再通下一次者,亦宜将药停服。 
效果 分六次将汤药饮完,将石膏送服强半,热犹未退,大便亦未通下,又煎渣取汤两盅,分数 
次送服石膏末,甫完,陡觉表里热势大增。时当夜深,不便延医。孙××自持其脉弦硬异常,因常 
阅《衷中参西录》,知脉虽有力而无洪滑之象者,用白虎汤时皆宜加人参,遂急买高丽参五钱,煮汤顿饮下, 
其脉渐渐和缓,热亦渐退,至黎明其病霍然全愈矣。 
帮助 按伤寒定例,凡用白虎汤若在汗吐下后及渴者,皆宜加人参。细询此证之经过始知曾发大 
汗一次,此次所服之药虽非白虎汤原方,实以山药代粳米,又以石膏如此服法,其力之大,可以不用知母 
是其方亦白虎汤也。若早加党参数钱,与山药、甘草同煎汤以送服石膏,当即安然病愈。乃因一时疏 
忽,并未见及,犹幸病者自知医理以挽回于末路。此虽白虎汤与人参前后分用之,仍不啻同时并用之也。 
此证加人参于白虎汤中其益有三∶发汗之后人之正气多虚,人参大能补助正气,俾正气壮旺自能运化药 
力以胜邪,其为益一也;又发汗易伤津液,津液伤则人之阴分恒因之亏损。人参与石膏并用,能于邪热炽 
盛之时滋津液以复真阴,液滋阴复则邪热易退,其为益二也;又用药之法,恒热因凉用凉因热用,《内经》所 
谓伏其所因也。此证用山药、甘草煎汤送服石膏之后,病则纯热,药则纯凉,势若冰炭不兼容, 
是以其热益激发而暴动。加人参之性温者以为之作引,此即凉因热用之义,为凉药中有热药引之以消热, 
而后热不格拒转与化合,热与凉药化合则热即消矣,此其为益三也。统此三益观之,可晓然于此病之所以愈, 
益叹仲圣制方之妙。即约略用之,亦可挽回至险之证也。 


<篇名>22.温病兼项后作疼
属性:李姓媪,年八旬有三,于孟夏得温病,兼项后作疼。 
病因 饭后头面有汗,忽隔窗纱透入凉风,其汗遂闭,因得斯证。 
证候 项疼不能转侧,并不能俯仰,周身发灼热,心中亦热,思凉物,脉象左部弦而长,右部则弦 
硬有力,大便干燥,小便 
短少。 
诊断 此因汗出腠理不闭,风袭风池、风府,是以项疼,因而成风温也。高年之脉,大抵弦细, 
因其气虚所以无甚起伏,因其血液短少,是以细而不濡,至于弦硬而长有力,是显有温热 
之现象也。此当清其实热而辅以补正兼解表之品。 
处方 生石膏(一两轧细) 野台参(三钱) 生怀地黄(一两) 生怀山药(五钱) 
玄参(三钱) 沙参(三钱) 连翘(二钱) 
西药阿斯匹林一瓦,先将阿斯匹林用白糖水送下,继将中药煎汤一大盅,至甫出汗时,即将汤药乘热服下。 
效果 如法将药服下后,周身得汗,表里之热皆退,项之疼大减,而仍未脱然。俾每日用阿斯 
匹林一瓦强(约三分),分三次用白糖水送下,隔四点钟服一次。若初次服后微见汗者,后 
两次宜少服,如此两日,项疼全愈。盖阿斯匹林不但能发汗去热,且能为热性关节疼痛之最妙药也。 


<篇名>23.温病兼胁疼
属性:天津李××,年三十八岁,于孟冬上旬得温病。 
病因 其妻于秋间病故,子女皆幼,处处须自经管,伤心又兼劳心,遂致暗生内热,薄受外感,遽成温病。 
证候 初得时,即表里俱热,医者治以薄荷、连翘、菊花诸药,服后微见汗,病稍见轻。至再 
诊时,病患自觉呼吸短气,此气郁不舒也,医者误以为气虚,遂于清热药中加党参以补其 
气,服后右胁下陡然作疼,彻夜不能卧,亦不能眠,心中发热,舌苔白浓,大便四日未行,其左 
右脉皆弦,右部尤弦而有力,一分钟八十二至。 
诊断 凡脉象弦者主疼,又主血液短少,此证之右胁非常疼痛, 
原为证脉相符,而其伤心劳心以致暗生内热者,其血液必然 
伤损,此亦证脉相符也。其右脉弦而有力者,外感之热已入阳明之府也。拟治以白虎汤而辅以开郁滋阴之品。 
处方 生石膏(二两轧细) 知母(八钱) 玄参(八钱) 天冬(八钱) 
川楝子(五钱捣碎) 生莱菔子(五钱捣碎) 连翘(三钱) 甘草(二钱) 粳米(三钱) 
共煎汤两大盅,分两次温服下。 
复诊 将药服完,热退强半,胁疼已愈三分之二,脉象变为浮弦,惟胸膈似觉郁闷,大便犹未 
通下。再治以宽胸清热润燥之剂,为其脉浮有还表之象,宜再少加透表之药以引之外 
出,其病当由汗而解。 
处方 糖栝蒌(二两切碎) 生石膏(一两捣细) 知母(五钱) 玄参(五钱) 
连翘(三钱) 川楝子(四钱捣碎) 甘草(二钱) 
共煎汤两盅,分二次温服下。其服完两次之后,迟一点钟再服西药阿斯匹林一瓦。温复以取微汗。 
效果 如法将药服完,果周身皆得微汗,病若失,其大便亦通下矣。 


<篇名>24.风温兼喘促
属性:辽宁赫姓幼子,年五岁,得风温兼喘促证。 
病因 季春下旬,在外边HT 戏,出汗受风,遂成温病。医治失宜,七八日间又添喘促。 
证候 面红身热,喘息极迫促,痰声漉漉,目似不瞬。脉象浮滑,重按有力。指有紫纹,上透 
气关,启口视其舌苔白而润。问其二便,言大便两日未行,小便微黄,然甚通利。 
诊断 观此证状况已危至极点,然脉象见滑,虽主有痰亦足征阴 
分充足。且视其身体胖壮,知犹可治,宜用《金匮》小青龙加石膏汤,再加杏仁、川贝以利其肺气。 
处方 麻黄(一钱) 桂枝尖(一钱) 生杭芍(三钱) 清半夏(二钱) 
杏仁(二钱去皮捣碎) 川贝母(二钱捣碎) 五味子(一钱捣碎) 干姜(六分) 
细辛(六分) 生石膏(一两捣细) 
共煎汤一大盅,分两次温服下。 
方解 《金匮》小青龙加石膏汤,原治肺胀咳而上气烦躁而喘,然其石膏之分量,仅为麻桂三分 
之二(《金匮》小青龙加石膏汤,其石膏之分量原有差误,曾详论之),而此方中之生石膏则十倍于 
麻桂,诚以其面红身热,脉象有力,若不如此重用石膏,则麻、桂、姜、辛之热,即不能用 
矣。又《伤寒论》小青龙汤加减之例,喘者去麻黄加杏仁,今加杏仁而不去麻黄者,因重用生 
石膏以监制麻黄则麻黄即可不去也。 
复诊 将药服尽一剂,喘愈强半,痰犹壅盛,肌肤犹灼热,大便犹未通下,脉象仍有力, 
拟再治以清热利痰之品。 
处方 生石膏(二两捣细) 栝蒌仁(二两炒捣) 生赭石(一两轧细) 
共煎汤两盅,分三次徐徐温饮下。 
效果 将药分三次服完,火退痰消,大便通下,病遂全愈。 
帮助 此案曾登于《全国名医验案类编》,何廉臣评此案云∶“风温犯肺胀喘促,小儿尤多, 
病最危险,儿科专家,往往称为马脾风者此也。此案断定为外寒束内热,仿《金匮》小 
青龙加石膏汤,再加贝母开豁清泄,接方用二石蒌仁等清镇滑降而痊。先开后降,步骤井然。 
惟五岁小儿能受如此重量,可见北方风气刚强,体质茁实,不比南方人之体质柔弱 
也。正惟能受重剂,故能奏速功。” 
观何廉臣评语,虽亦推奖此案,而究嫌药量过重,致有南 
北分别之设想。不知此案药方之分量若作一次服,以治五岁孺子诚为过重。若分作三次服,则无论南北,凡 
身体胖壮之孺子皆可服也。试观近今新出之医书,治产后温病,有一剂用生石膏半斤者矣,曾见于刘蔚楚 
君《遇安斋证治丛录》,刘君原广东香山人也。治鼠疫病亦有一剂用生石膏半斤者矣,曾见于李健颐君《鼠 
疫新篇》,李君原福建平潭人也。若在北方治此等证,岂药之分量可再加增乎?由此知医者之治病用药,不可 
定存南北之见也。且愚亦尝南至汉皋矣,曾在彼处临证处方,未觉有异于北方,惟用发表之剂则南方出汗较 
易,其分量自宜从轻。然此乃地气寒暖之关系,非其身体强弱之关系也。既如此,一人之身则冬时发汗与 
夏时发汗,其所用药剂之轻重自迥殊也。 
尝细验天地之气化,恒数十年而一变。仲景当日原先着《伤寒论》,后着《金匮要略》,《伤寒论》小 
青龙汤,原有五种加法,而独无加石膏之例。因当时无当加石膏之病也。至着《金匮》时,则有小青龙加石 
膏汤矣,想其时已现有当加石膏之病也。忆愚弱冠时,见医者治外感痰喘证,但投以小青龙汤原方即可治愈。 
后数年愚临证遇有外感痰喘证,但投以小青龙汤不效,必加生石膏数钱方效。又迟数年必加生石膏两许,或 
至二两方效。由斯知为医者当随气化之转移,而时时与之消息,不可拘定成方而不知变通也。 


<篇名>25.秋温兼伏气化热
属性:天津徐姓媪,年五十九岁,于中秋上旬得温病,带有伏气化热。 
病因 从前原居他处,因迁居劳碌,天气燥热,有汗受风,遂得斯病。 
证候 晨起,觉周身微发热兼酸懒不舒,过午,陡觉表里大热,且其热浸增。及晚四点钟往 
视时,见其卧床闭目,精神昏昏。呻吟不止。诊其脉左部沉弦,右部洪实,数近六至。问 
其未病之前,曾有拂意之事乎?其家人曰∶诚然,其禀性褊急,恒多忧思,且又易动肝火。欲见其 
舌苔,大声呼数次,始知启口,视其舌上似无苔而有肿胀之意,问其大便,言素恒干燥。 
诊断 其左脉沉弦者,知其肝气郁滞不能条达,是以呻吟不止,此欲借呻吟以舒其气也。其右脉 
洪实者,知此证必有伏气化热,窜入阳明,不然则外感之温病,半日之间何至若斯之 
剧也。此当用白虎汤以清阳明之热,而以调气舒肝之药佐之。 
处方 生石膏(二两捣细) 知母(八钱) 生莱菔子(三钱捣碎) 青连翘(三钱) 
甘草(二钱) 粳米(四钱) 
共煎汤两盅,分两次温服。 
方解 莱菔子为善化郁气之药。其性善升亦善降,炒用之则降多于升,生用之则升多于降。 
凡肝气之郁者宜升,是以方中用生者。至于连翘,原具有透表之力,而用于此方之中,不但 
取其能透表也,其性又善舒肝,凡肝气之郁而不舒者,连翘皆能舒之也。是则连翘一味,既可佐 
白虎以清温热,更可辅莱菔以开肝气之郁滞。 
复诊 将药两次服完,周身得汗,热退十之七八,精神骤然清爽。左脉仍有弦象而不沉,右脉 
已无洪象而仍似有力,至数之数亦减。问其心中仍有觉热之时,且腹中知饥而懒于进 
食,此则再宜用凉润滋阴之品清其余热。 
处方 玄参(一两) 沙参(五钱) 生杭芍(四钱) 生麦芽(三钱) 
鲜茅根(四钱) 滑石(三钱) 甘草(二钱) 
共煎汤一大盅,温服。方中用滑石者,欲其余热自小便泻出也。 
效果 将药连服两剂,大便通下,其热全消,能进饮食,脉象亦和平矣。而至数仍有数象,俾 
再用玄参两半,潞参三钱,煎服数剂以善其后。 


<篇名>26.温病兼呕吐
属性:天津刘××,年三十二岁,于季夏得温热病,兼呕吐不受饮食。 
病因 因在校中宿卧,一日因校中无人,其衾褥被人窃去,追之 
不及,因努力奔跑,周身出汗,乘凉歇息,遂得斯病。 
证候 心中烦热,周身时时汗出,自第二日,呕吐不受饮食。今已四日,屡次服药亦皆吐出,即 
渴时饮水亦恒吐出。舌苔白浓,大便四日未行。其脉左部弦硬,右部弦长有力,一息五至。 
诊断 其脉左部弦硬者,肝胆之火炽盛也。右部弦长者,冲气挟胃气上冲也。弦长而兼有力者, 
外感之热已入阳明之府也。此证因被盗怒动肝气,肝火上冲,并激动冲气挟胃气亦上冲,而外感之热又 
复炽盛于胃中以相助为虐,是以烦热汗出不受饮食而吐药吐水也。此当投以清热镇逆之剂。 
处方 生石膏(二两细末) 生赭石(六钱细末) 镜面朱砂(五钱细末) 
和匀分作五包,先送服一包,过两点钟再送服一包,病愈即停服,不必尽剂。方用散剂不用汤 
剂者止呕吐之药丸散优于汤剂也。 
效果 服至两包,呕吐已愈,心中犹觉烦热。服至四包,烦热全愈,大便亦通下矣。 
帮助 石膏为石质之药,本重坠且又寒凉,是以白虎汤中以石膏 
为主,而以甘草缓之,以粳米和之,欲其服后留恋于胃中,不至速于下行。故用石膏者,忌再与 
重坠之药并用,恐其寒凉侵下焦也,并不可与开破之药同用,因开破之药力原下行也。乃今因肝气胆火相并 
上冲,更激动冲气挟胃气上冲,且更有外感之热助之上冲,因致脏腑之气化有升无降,是以饮 
食与药至胃中皆不能存留,此但恃石膏之寒凉重坠原不能胜任,故特用赭石之最有压力者以辅之。此 
所以旋转脏腑中之气化,而使之归于常也。设非遇此等证脉,则石膏原不可与赭石并用也。 


<篇名>27.温病兼呕吐
属性:天津杨姓媪,年过五旬,于季春得温病兼呕吐。 
病因 家庭勃溪,激动肝胆之火,继因汗出受风,遂得此证。 
证候 表里壮热,呕吐甚剧,不能服药,少进饮食亦皆吐出。舌 
苔白浓,中心微黄。大便三日未行。其脉左部弦长,右部洪长,重按皆实。 
诊断 此少阳阳明合病也。为其外感之热已入阳明胃府,是以表里俱壮热,而舌苔已黄,为其激 
动之火积于少阳肝胆,是以其火上冲频作呕吐。治此证者欲其受药不吐,当变汤剂为 
散,且又分毫无药味,庶可奏效。 
处方 生石膏(一两细末) 鲜梨(两大个) 
将梨去皮,切片,蘸石膏末,细细嚼服。 
复诊 将梨片与石膏末嚼服一强半未吐,迟两点钟又将所余者服完,自此不复呕吐,可进饮食,大 
便通下一次。诊其脉犹有余热,问其心中亦仍觉热,而较前则大轻减矣。拟改用汤剂。以清其未尽之热。 
处方 生石膏(一两捣细) 生杭芍(八钱) 玄参(三钱) 沙参(三钱) 
连翘(二钱) 甘草(二钱) 鲜白茅根(三钱) 
药共七味,先将前六味水煎十余沸,入鲜白茅根再煎三四沸,取汤一大盅,温服。 
效果 将药如法煎服一剂,热又减退若干,脉象已近和平,遂即 
原方将石膏改用六钱,芍药改用四钱,又服一剂,病遂全愈。 
或问 石膏为清阳明之主药,此证原阳明少阳均有实热,何以用石膏但清阳明之热而病即可愈?答 
曰∶凡药服下,原随气血流行无处不到。石膏虽善清阳明之热,究之,凡脏腑间蕴有 
实热,石膏皆能清之。且凡呕吐者皆气上逆也,石膏末服,其石质之重坠大能折其上逆之气使之 
下行,又有梨片之甘凉开胃者以辅之,所以奏效甚捷也。若当秋夏之交无鲜梨时,可以西瓜代之。 


<篇名>28.温病兼衄血便血
属性:天津陈姓童子,年十五岁,于仲秋得温病,兼衄血便血。 
病因 初因周身发热出有斑点,有似麻疹。医用凉药清之,斑点即回,连服凉药数剂,周身 
热已退,而心中时觉烦躁。逾旬日因薄受外感,其热陡然反复。 
证候 表里壮热,衄血两次,小便时或带血。呕吐不受饮食,服药亦多吐出。心中自觉为热 
所灼,怔忡莫支。其脉摇摇而动,数逾五至,左右皆有力,而重按不实。舌苔白而欲黄, 
大便三日未行。本拟投以白虎加人参汤,恐其服后作呕。 
处方 生石膏(三两细末) 生怀山药(二两) 
共煎汤一大碗,俾徐徐温饮下。为防其呕吐,一次只饮一大口,限定四小时将药服完。 
方解 凡呕吐之证,饮汤则吐,服粥恒可不吐。生山药二两煎取浓汁与粥无异,且无药味,服 
后其粘滞之力自能留恋于胃中。且其温补之性,又能固摄下焦以止便血,培养心气以治 
怔忡也。而以治此温而兼虚之证,与石膏相伍为方,以石膏清其温,以山药补其虚,虽非白 
虎加人参汤,而亦不啻白虎加人参汤矣。 
复诊 翌日复诊,热退十之七八,心中亦不怔忡,少进饮食亦不呕吐,衄血便血皆愈。脉象力减,至数仍数。 
处方 玄参(二两) 潞参(五钱) 连翘(五钱) 
效果 仍煎汤一大碗,徐徐温饮下,尽剂而愈,大便亦即通下。 
方解 盖其大热已退而脉仍数者,以其有阴虚之热也。玄参、潞参并用,原善退阴虚作热,而犹 
恐其伏有疹毒,故又加连翘以托之外出也。 
帮助 此证若能服药不吐,投以大剂白虎加人参汤,大热退后其脉即可不数。乃因其服药呕吐, 
遂变通其方,重用生山药二两与生石膏同煎服。因山药能健脾滋肾,其补益之力虽不如 
人参,实有近于人参处也。至大热退后,脉象犹数,遂重用玄参二两以代石膏,取其能滋真阴兼能 
清外感余热,而又伍以潞参、连翘各五钱。潞参即古之人参。此由白虎加人参之 
义化裁而出,故虚热易退,而连翘又能助玄参凉润之力外透肌肤,则余热亦易清也。 


<篇名>29.温疹
属性:天津杨姓幼子,年四岁,于季春发生温疹。 
病因 春暖时气流行,比户多有发生此病者,因受传染。 
证候 周身出疹甚密,且灼热异常。闭目昏昏,时作谵语。气息迫促,其唇干裂紫黑,上多凝血。 
脉象数而有力。大便不实, 
每日溏泻两三次。 
诊断 凡上焦有热之证,最忌下焦滑泻。此证上焦之热已极,而其大便又复溏泻,欲清其热,又恐 
其溏泻益甚,且在发疹,更虞其因溏泻毒内陷也。是以治此证者,当上清其热下止其 
泻,兼托疹毒外出,证候虽险,自能治愈。 
处方 生怀山药(一两) 滑石(一两) 生石膏(一两捣细) 生杭芍(六钱) 
甘草(三钱) 连翘(三钱) 蝉蜕(钱半去土) 
共煎一大盅,分多次徐徐温饮下。 
效果 分七八次将药服完。翌日视之其热大减,诸病皆见愈。惟不能稳睡,心中似骚扰不安, 
其脉象仍似有力。遂将方中滑石、石膏皆减半,煎汤送安宫牛黄丸半丸,至煎渣再服时, 
又送服半丸,病遂全愈。 


<篇名>30.温疹兼喉痧
属性:天津沈姓学生,年十六岁,于仲春得温疹兼喉痧证。 
病因 因在体育场中游戏,努力过度,周身出汗为风所袭,遂得斯病。 
证候 初病时微觉恶寒头疼,翌日即表里俱壮热,咽喉闷疼。延医服药病未见轻,喉中疼闷似加 
剧,周身又复出疹,遂延愚为延医。其肌肤甚热,出疹甚密,连无疹之处其肌肤亦红,诚 
西人所谓猩红热也。其心中亦自觉热甚,其喉中扁桃腺处皆红肿,其左边有如榆荚一块发白。自 
言不惟饮食疼难下咽,即呼吸亦甚觉有碍。诊其脉左右皆洪滑有力,一分钟九十八 
至。愚为刺其少商出血,复为针其合谷,又为拟一清咽、表疹、泻火之方俾服之。 
处方 生石膏(二两捣细) 玄参(六钱) 天花粉(六钱) 射干(三钱) 
牛蒡子(三钱捣碎) 浙贝母(三钱) 青连翘(三钱) 鲜芦根(三钱) 
甘草(钱半) 粳米(三钱) 
共煎汤两大盅,分两次温服下。 
复诊 翌日过午复为诊视,其表里之热皆稍退,脉象之洪滑亦稍减,疹出又稍加多。从前三日 
未大便,至此则通下一次。再视其喉,其红肿似加增,白处稍大,病患自言此时饮水必须 
努力始能下咽,呼吸之滞碍似又加剧。愚曰∶此为极危险之病,非刺患处出血不可。遂用圭 
式小刀,于喉左右红肿之处,各刺一长口放出紫血若干,遽觉呼吸顺利。拟再投以清 
热消肿托表疹毒之剂。 
处方 生石膏(一两捣细) 天花粉(六钱) 赤芍(三钱) 板蓝根(三钱) 
牛蒡子(三钱捣细) 生蒲黄(三钱) 浙贝母(三钱) 青连翘(三钱) 鲜芦根(三钱) 
共煎一大盅半,分两次温服。 
方解 赤芍药,张隐庵、陈修园皆疑是山中野草之根,以其纹理甚粗,与园中所植之芍 
药根迥异也。然此物出于东三省,愚亲至其地,见山坡多生此种芍药,开单瓣红花,其花小于寻 
常芍药花约三倍,而其叶则确系芍药无疑。盖南方亦有赤芍药,而其根仍白,兹则花赤其根 
亦赤,是以善入血分活血化瘀也。又浙贝治嗽,不如川贝,而以之治疮,浙贝似胜于川 
贝,以其味苦性凉能清热解毒也。 
效果 将药连服两剂,其病脱然全愈。 
帮助 《内经》灵枢痈疽篇谓∶“痈发于嗌中,名曰猛疽,猛疽不治,化为脓,脓不泻,塞 
咽半日死。”此证咽喉两旁红肿日增,即痈发嗌中名为猛疽者也。其脓成不泻则危在目前, 
若其剧者必俟其化脓而后泻之,又恒有迫不及待之时,是以 
此证因其红肿已甚有碍呼吸,急刺之以出其紫血而红肿遂愈,此所谓防之于预也。且化脓而后泻之,其疮 
口恒至溃烂,若未成脓而泻,其紫血所刺之口半日即合矣。 
喉证原有内伤外感之殊,其内伤者虽宜注重清热, 
亦宜少佐以宣散之品。如《白喉忌表抉微》方中之用薄荷、连翘是也。由外感者虽不忌用表散之品,然宜 
表散以辛凉,不宜表散以温热,若薄荷、连翘、蝉蜕、芦根诸药,皆表散之佳品也。或有谓喉证若由于外感,虽 
麻黄亦可用者,然用麻黄必须重用生石膏佐之。若《伤寒论》之麻杏甘石汤,诚为治外感喉证之佳方也。 
特是,其方原非治喉证之方,是以方中石膏仅为麻黄之两倍,若借以治外感喉证,则石膏当十倍于麻黄。若遇 
外感实火炽盛者,石膏尤宜多加方为稳妥。是以愚用此方以治外感喉证时,麻黄不过用至一钱,而生石膏 
恒用至两余,或重用至二两也。然此犹论喉证之红肿不甚剧者,若至肿甚有碍呼吸,不惟麻黄不可用,即 
薄荷亦不可用,是以治此证方中止用连翘、芦根也。以上所论者,无论内伤外感,皆咽喉证之属热 
者也。而咽喉中之变证,间有真寒假热者,又当另议治法。 


<篇名>31.温病兼喉痧痰喘
属性:天津马××,年二十八岁,于季秋得温病兼喉痧痰喘证。 
病因 初因外出受风感冒甚微,医者用热药发之,陡成温病,而喉病喘病遂同时发现。 
证候 表里俱壮热,喘逆咳嗽,时吐痰涎,咽喉左边红肿作疼(即西人所谓扁桃体炎)。其外 
边项左侧亦肿胀,呼吸皆有窒碍。为其病喉且兼喘逆,则吸气尤形困难,必十分努力始能将气吸入。 
其舌苔白而薄,中心微黄。小便赤涩,大便四日未行。其脉左右皆弦长,右部重诊有力,一分钟九十六至。 
诊断 此乃外感之热已入阳明之府,而冲气又挟胃气肝火上冲也。为其外感之热已入阳明之府, 
是以右脉之力胜于左脉,为其冲气挟胃气肝火上冲,是以左右脉皆弦长。病现喘逆及咽喉 
肿疼,其肿痛偏左者,正当肝火上升之路也。拟治以麻杏甘石汤,兼加镇冲降胃纳气利痰之 
品以辅之,又宜兼用针刺放血以救目前之急。 
处方 麻黄(一钱) 生石膏(二两捣细) 生赭石(一两轧细) 生怀山药(八钱) 
杏仁(三钱去皮炒捣) 连翘(三钱) 牛蒡子(三钱捣碎) 射干(二钱) 甘草(一钱) 
共煎汤两盅,分两次温服。 
又于未服药之前,用三棱针刺其两手少商出血,用有尖小刀刺其咽喉肿处,开两小口令其 
出血,且用硼砂、西药盐酸盖理,融以三十倍之水,俾其含漱。又于两手合谷处为之行 
针。其咽喉肿处骤然轻减,然后服药。 
复诊 将药服后,其喘顿愈强半,呼吸似无妨碍,表里之热亦愈强半。脉象亦较前平和,其右 
部仍然有力。胸膈似觉郁闷,有时觉气上冲,仍然咳嗽,大便犹未通下。拟再治以开郁降 
气清热理嗽之剂。 
处方 糖栝蒌(二两切碎) 生石膏(一两捣细) 生赭石(五钱轧细) 生杭芍(三钱) 
川贝母(三钱) 碎竹茹(三钱) 牛蒡子(三钱捣碎) 
共煎汤一大盅,温服。 
效果 将药煎服一剂,大便通下,诸病皆愈。唯一日之间犹偶有 
咳嗽之时,俾用川贝母细末和梨蒸食之以善其后。 
帮助 凡用古人成方治病,其药味或可不动,然必细审其药之分 
量或加或减,俾与病机相宜。如麻杏甘石汤原方,石膏之分 
量仅为麻黄之两倍,而此证所用麻杏甘石汤则石膏之分量二十倍于麻黄矣。盖《伤寒论》之 
麻杏甘石汤原非为治喉证而设,今借之以治喉证。原用麻黄以散风定喘,又因此证之喉肿太甚, 
有碍呼吸,而方中犹用麻黄,原为行险之道,故麻黄仅用一钱,而又重用生石膏二两以监制之。且于 
临服药时先用刀开其患处,用针刺其少商与合谷,此所以于险中求稳也。尝闻友人杨××言, 
有一名医深于《伤寒论》,自着有《注解伤寒论》之书行世,偶患喉证,自服麻杏甘石汤竟至 
不起,使其用麻杏甘石汤时,亦若愚所用者如此加减,又何患喉证不愈乎?纵使服药不能即愈, 
又何至竟不起乎?由此知非古人之方误人。麻杏甘石汤,原为发汗后及下后汗出而喘无大热者 
之的方,原未言及治喉证也。而欲借之以治喉证,能勿将药味之分量为之加减乎?尝总核《伤寒论》 
诸方用于今日,大抵多稍偏于热,此非仲景之不善制方也。自汉季至今,上下相隔已一千六 
百余年,其天地之气化,人生之禀赋,必有不同之处,是以欲用古方皆宜细为斟酌也。 


<篇名>32.温病兼喉疼
属性:天津胡××,年五十四岁,于仲秋感受温病兼喉疼证。 
病因 劳心过度,暗生内热。且日饮牛乳两次作点心,亦能助热, 
内热上潮,遂觉咽喉不利,至仲秋感受风温,陡觉咽喉作疼。 
证候 表里俱觉发热,咽喉疼痛,妨碍饮食。心中之热时觉上冲,则咽喉之疼即因之益甚。 
周身酸懒无力,大便干燥,脉象浮滑而长,右关尤重按有力,舌上白苔满布。 
诊断 此证脉象犹浮,舌苔犹白,盖得病甫二日,表证犹未罢也。而右关重按有力,且 
时觉有热上冲咽喉者,是内伤外感相并而为病也。宜用重剂清其胃腑之热,而少佐以解表之 
品,表解里清,喉之疼痛当自愈矣。 
处方 生石膏(四两捣细) 西药阿斯匹林(一瓦) 
单将生石膏煎汤一大盅,乘热将阿斯匹林融化其中服之。因阿斯匹林实为酸凉解肌之妙药,与大 
量之石膏并用,服后须臾其内伤外感相并之热,自能化汗而解也。 
效果 服后约半点钟,其上半身微似有汗,而未能遍身透出,迟一点钟,觉心中之热不复上冲, 
咽喉疼痛轻减。时在下午一点钟,至晚间临睡时,仍照原方再服一剂,周身皆得透汗,安睡一夜, 
翌晨,诸病若失矣。 


<篇名>33.温病兼阴虚
属性:邻村高××,年二十五岁,于仲夏得温病。 
病因 仲夏上旬,麦秋将至,远出办事,又欲急回收麦,长途趋行于烈日之中。辛苦殊甚,因 
得温病。其叔父××与其表叔毛××皆邑中名医,又皆善治温病。二人共治旬日无效,盖 
因其劳力过甚,体虚不能托病外出也。 
证候 愚诊视时,其两目清白,竟无所见,两手循衣摸床,乱动不休,谵语无伦,分毫不省人事。 
其大便从前滑泻,此时虽不滑泻,每月仍溏便一两次,脉象浮而无力,右寸之浮尤甚,两尺按 
之即无,一分钟数至一百二十至。舌苔薄黄,中心干而微黑。 
诊断 此证两目清白无火,而竟无所见者,肾阴将竭也。其两手乱动不休者,肝风已动也。病 
势至此,危险已至极点。幸喜脉浮为病还在太阳,右寸浮尤甚,又为将汗之兆。其所以将 
汗而不汗者,人身之有汗,如天地之有雨,天地阴阳和而后雨,人身亦阴阳和而后汗。此证两 
尺脉甚弱,阳升而阴不应,是以不能作汗。当用大滋真阴之品,济阴以应其阳必能 
自汗,汗出则病愈矣。然非强发其汗也,强发其汗则汗出必脱。调剂阴阳以听其自汗,是以汗出必愈也。 
处方 熟怀地黄(二两) 生怀山药(一两) 玄参(一两) 大甘枸杞(一两) 
甘草(三钱) 真阿胶(四钱) 
药共六味,将前五味煎汤一大碗去渣,入阿胶融化,徐徐分数次温饮下。 
效果 时当上午十点钟,将药煎服至下午两点钟将药服完。形状较前安静,再诊其脉颇有起色。俾 
再用原方煎汤一大碗,陆续服之,至秉烛时遍身得透汗,其病霍然愈矣。此案曾载于《全国名 
医验案类编》,何廉臣对于此案似有疑意,以为诚如案中所述病况,实为不可挽救之证也。故今将 
此案又登斯编,以征此案之事实。 
帮助 尝实验天地之气化,恒数十年而一变,医者临证用药,即宜随气化而转移,因病者所得之 
病已先随气转移也。愚未习医时,见医者治伤寒温病,皆喜用下药,见热已传里其大便稍实者,用 
承气汤下之则愈,如此者约二十年。及愚习医学时,其如此治法者则恒多偾事,而愚所阅之医书,又皆系 
赵氏《医贯》、《景岳全书》、《冯氏锦囊》诸喜用熟地之书,即外感证亦多喜用之。愚之治 
愈此证,实得力于诸书之讲究。而此证之外,又有重用熟地治愈寒温之坏证,诸多验 
案(地黄解后载有数案可参观)。此乃用药适与时会,故用之有效也。且自治愈此证之后,毛××、 
高××深与愚相契,亦仿用愚方而治愈若干外感之虚证,而一变其从前之用药矣。后至愚年过 
四旬,觉天地之气化又变,病者多系气分不足,或气分下陷,外感中亦多兼见此证,即用白虎 
汤时多宜加人参方效。其初得外感应发表时,亦恒为加黄 方效。如是者又有年。 
乃自一九二一年以来,病多亢阳,宜用大剂凉润之药济阴以 
配其阳,其外感实热之证,多宜用大剂白虎汤,更佐以凉润之品。且人脏腑之气化多有升无降,或 
脑部充血,或夜眠不寐,此皆气化过升之故,亦即阳亢无制之故。治之者宜镇安其气化,潜藏其 
阳分,再重用凉润之药辅之,而病始可治。此诚以天地之气化又有转移,人所生之病即随之转移,而医 
者之用药自不得不随之转移也。由此悟自古名医所着之书,多有所偏者非偏也,其所逢之时气化不同 
也。愚为滥竽医界者已五十年,故能举生平之所经历而细细陈之也。 


<篇名>34.温病兼喘胀
属性:邑中王××之女,年十五岁,于仲春得温病久不愈。 
病因 仲春上旬,感受风温,医者延医失宜,迁延旬余,病益增剧,医者诿为不治,始延愚为诊视。 
证候 心下胀满甚剧,喘不能卧,自言心中干甚,似难支持。其舌苔白而微黄。小便赤少,大便 
从前滑泻,此时虽不滑泻,然仍每日下行。脉搏一息五至强,左部弦而有力,右部似大而有力, 
然皆不任重按。 
诊断 此其温病之热,本不甚剧。因病久真阴亏损致小便不利,所饮之水停于肠胃则胀满,迫 
于心下则作喘。其心中自觉干甚,固系温病之热未清,亦足征其真阴亏损阴精不能上奉也 
(《内经》谓阴精上奉,其人寿)。当滋其真阴,利其小便,真阴足则以水济火,而心中自然不干; 
小便利则水从下消,而胀满喘促自愈。至于些些温病之余热,亦可皆随小便泻出而不治自愈矣。 
处方 鲜白茅根去净皮及节间细根(六两锉碎),用水三大碗,煎一沸,俟半点钟,视其茅根 
若不沉水底,再煎一沸,至茅根皆沉水底其汤即成。去渣当茶,徐徐温饮之。 
效果 如法煎饮茅根两日,其病霍然全愈。盖白茅根凉润滋阴, 
又善治肝肾有热,小便不利,且具有发表之性,能透温病之热外出。一药而三善备,故单用之而 
能立建奇功也。然必剖取鲜者用之,且复如此煎法(过煎则无效)方能有效。 
凡药之性,能利水者多不 
能滋阴,能下降者多不能上升,能清里者多不能达表。惟茅根既善滋阴,又善利水,既善引 
水气下行,又善助肾阴上升。且内清脏腑之热,外托肌表之邪,而尤善清肺利痰定其喘逆。 


<篇名>35.温病兼虚热
属性:邑城东刘氏女,年十五岁,于季春患温病久不愈。 
病因 因天气渐热,犹勤纺织,劳力之余出外乘凉,有汗被风遂成温病。 
证候 初得周身发热,原宜辛凉解肌,医者竟用热药发之,汗未出而热益甚,心中亦热而且渴。此 
时若用大剂白虎加人参汤清之,病亦可愈,而又小心不敢用。惟些些投以凉润小剂,迁延二十余日, 
外感之热似渐退。然午前稍轻而午后则仍然灼热,且多日不能饮食,形体异常清瘦。左脉弦细无根,右 
部关脉稍实,一息六至。舌苔薄而微黄,毫无津液。大便四五日一行,颇干燥。 
诊断 此因病久耗阴,阴虚生热,又兼外感之热留滞于阳明之府未尽消也。当以清外感之热为主, 
而以滋补真阴之药辅之。 
处方 生石膏(一两捣细) 野党参(三钱) 生怀地黄(一两) 生怀山药(一两) 
生杭芍(四钱) 滑石(三钱) 甘草(三钱) 
共煎汤一大盅,分两次温服下。 
复诊 将药煎服两剂后,外感之热已退,右关脉已平和,惟过午 
犹微发热,此其阴分犹虚也。当再滋补其阴分。 
处方 玄参(一两) 生怀山药(一两) 甘枸杞(五钱大者) 生杭芍(五钱) 
滑石(二钱) 熟地黄(一两) 生鸡内金(一钱黄色的捣) 甘草(二钱) 
共煎一大盅,分两次温服。 
效果 日服药一剂,连服三日,灼热全愈。 
帮助 按此方于大队滋阴药中犹少加滑石者,恐外感之热邪未尽,引之自小便出也。愚凡治 
外感之热兼有虚热者,恒生山药与滑石并用,泻热补虚一举两得。至上有外感燥热而下焦复滑 
泻者,用之以清热止泻(宜各用一两),尤屡次奏效。二药相伍,原有化合之妙用,若再加 
芍药、甘草,即拙拟之滋阴清燥汤,可参观也。 


<篇名>36.温病兼吐血
属性:沧州,吴姓媪,年过七旬,偶得温病兼患吐血。 
病因 年岁虽高,家庭事务仍自操劳,因劳心过度,心常发热, 
时当季春,有汗受风,遂得温病,且兼吐血。 
证候 三四日间表里俱壮热,心中热极之时恒吐血一两口,急饮新汲井泉水其血即止。舌苔 
白浓欲黄,大便三日未行。脉象左部弦长,右部洪长,一息五至。 
诊断 此证因家务劳心过度,心肝先有蕴热,又兼外感之热传入阳明之府。两热相并,逼 
血妄行,所以吐血。然其脉象火热虽盛,而正犹不虚,虽在高年,知犹可治。其治法当以清胃腑 
之热为主,而兼清其心肝之热,俾内伤外感之热俱清,血自不吐矣。 
处方 生石膏(三两轧细) 生怀地黄(一两五钱) 生怀山药(一两) 生杭芍(一两) 
知母(三钱) 甘草(三钱) 乌犀角(一钱五分) 广三七(二钱轧细) 
药共八味,将前六味煎汤三盅,犀角另煎汤半盅和匀,分三次温服下。每服药一次,即送服三七末三分之一。 
效果 将药三次服完,血止热退,脉亦平和,大便犹未通下,俾 
煎渣再服,犀角亦煎渣取汤,和于汤药中服之,大便通下全愈。 
帮助 愚平素用白虎汤,凡年过六旬者必加人参,此证年过七旬而不加人参者,以其证兼吐血也。为 
不用人参,所以重用生山药一两,取其既能代粳米和胃,又可代人参稍补益其正气也。 


<篇名>37.温病兼冲气上冲
属性:奉天郑××,年五十二岁,于季春得温病,兼冲气自下上冲。 
病因 其人素有痰饮,偶有拂意之事,肝火内动,其冲气即挟痰饮上涌,连连呕吐痰水。季春之 
时,因受感冒成温病。温热内传,触动冲气又复上冲。 
证候 表里俱壮热,嗜饮凉水,痰涎上泛,屡屡咳吐,呃逆哕气,连连不除,两胁作胀。舌 
苔白浓,而中心微黄。大便三日未行。其脉左部弦硬而长,右部洪滑而长,皆重按有力。 
此温病之热,已入阳明之府,又兼肝火挟冲气上冲也。是以其左脉弦硬为肝火炽盛,其弦硬而 
长即为冲脉上冲之现象也;其右脉洪滑,为温热已入阳明胃腑,其洪滑而长,亦冲 
气上冲之现象也。因冲脉虽居于上,而与阳明厥阴皆有连带之关系也。欲治此证,当重用白虎汤以 
清阳明之热,而以泻肝降冲理痰之品辅之。 
处方 生石膏(三两捣细) 生赭石(一两轧细) 生龙骨(八钱捣碎) 生牡蛎(八钱捣碎) 
白知母(八钱) 生杭芍(六钱) 清半夏(三钱) 浓朴(钱半) 
甘草(二钱) 粳米(四钱) 
共煎汤三盅,分三次温饮下。 
效果 将药分三次服完,热退气平,痰涎亦减十之七八,脉象亦 
近平和。其大便犹未通下,遂即原方将石膏、龙骨、牡蛎各减半,再煎服一剂,大便通下,病全愈。 
帮助 方书用石膏未有与赭石并用者,即愚生平用石膏亦未尝与赭石并用,恐其寒凉之性与赭石 
之重坠者并用,而直趋下焦也。然遇有当用之病则病当之,非人当之。有如此证,不重用石膏则阳 
明之大热不除,不重用赭石则上逆之冲气莫制,此所以并用之而无妨碍也。设若此证,但阳明热实而无 
冲气上逆,服此药后其大盒饭即通下,或更至于滑泻。而阳明胃腑之热转难尽消,为其兼有冲气 
上逆,故必俟服之第二剂大便始能通下,此正所谓病当之,非人当之之明征也。龙骨、牡蛎之性, 
皆善镇肝敛冲,以之治痰原非所长,而陈修园谓龙骨、牡蛎同用,能引逆上之火泛滥之水下归其 
宅,为治痰之神品。其所谓痰,皆逆上之火泛滥之水所成,即此证之冲气上冲痰饮上泛者是也。是 
以方中龙骨、牡蛎各重用八钱,辅翼赭石以成降逆消痰之功,而非可泛以之治痰也。至于二药必生用 
者,非但取其生则性凉能清热也,《伤寒论》太阳篇用龙骨、牡蛎者三方,皆表证未罢,后世解者 
谓,龙骨、牡蛎,敛正气而不敛邪气,是以仲师于表证未罢者亦用之。然三方中之龙骨、牡蛎下 
皆未注有 字,其生用可知,虽其性敛正气不敛邪气,若 之则其性过涩,亦必于 
外感有碍也。且 之则其气轻浮不能沉重下达以镇肝敛冲,更可知矣。 

\chapter{疟疾门}
<篇名>1.疟疾兼阴虚
属性:天津吴××,年三十二岁,于仲秋病疟久不愈。 
病因 厂中作工,歇人不歇机器,轮流恒有夜勤。暑热之时,彻夜不眠,辛苦有火,多食凉物,入秋遂发疟疾。 
证候 其疟初发时,寒热皆剧,服西药金鸡纳霜治愈。旬日疟复发如前,又服金鸡纳霜治愈。 
七八日疟又发,寒轻热重,服金鸡纳霜不愈,服中药治疟汤剂亦不愈,迁延旬余,始求为 
延医。自言疟作时发热固重,即不发疟之日身亦觉热,其脉左右皆弦而无力,数逾五至,知 
其阴分阳分俱虚,而阴分之虚尤甚也。此当培养其气血而以治疟之药辅之。 
处方 玄参(一两) 知母(六钱) 天冬(六钱) 潞参(三钱) 
何首乌(三钱) 炙鳖甲(三钱) 常山(钱半酒炒) 柴胡(钱半) 茵 
陈(钱半) 生姜(三钱) 大枣(三个掰开) 
此方于发疟之前一夕煎服,翌晨煎渣再服,又于发疟之前四点钟,送服西药盐酸规尼涅(即金鸡纳霜, 
以盐酸制者)半瓦。 
效果 将药如法服之,一剂疟即不发。而有时身犹觉热,脉象犹数,知其阴分犹虚也。俾用 
玄参、生怀山药各一两,生姜三片,大枣三枚,同煎服,以服至身不发热时停服。 


<篇名>2.疟疾兼脾胀
属性:天津张××,年十九岁,学生,于孟秋病疟,愈而屡次反复。 
病因 其人性笃于学,当溽暑放假之时,仍自补习功课,劳心过度,又复受热过度,兼又多食 
瓜果以解其热,入秋遂发疟疾。 
证候 自孟秋中旬病疟,服西药金鸡纳霜治愈,后旬日反复,又服金鸡纳霜治愈,后又反复, 
服金鸡纳霜无效。以中药治愈,隔旬余病又反复。服中西药皆无效,因来社求治于愚。 
其脉洪滑而实,右部尤甚,自觉心中杜塞满闷,常觉有热上攻,其病疟时则寒热平均,皆不甚剧,其大便四日未行。 
诊断 此胃间积有热痰,又兼脾作胀也。方书谓久疟在肋下结有硬块名疟母,其块不消疟即不 
愈。而西人实验所结之块确系脾脏胀大,此证之自觉满闷,即脾脏胀大也。又方书谓无痰 
不作疟,是以治疟之方多用半夏、常山以理其痰,此证之自觉满闷且杜塞,又时有热上攻, 
实为热痰充塞于胃脘也。治之者宜消其脾之胀大,清其胃之热痰,兼以治疟之品辅之。 
且更可因其大便不通,驱逐脾之病下行自大便泻出,其病疟之根柢可除矣。 
处方 川大黄(四钱) 生鸡内金(三钱黄色的捣) 清半夏(三钱) 常山(钱半酒炒) 
柴胡(钱半) 茵陈(钱半) 甘草(钱半) 净芒硝(钱半) 
药共八味,将前七味煎汤一盅,冲芒硝服之。 
其疟每日一发,在下午七点钟。宜于午前早将药服下,至午后两三点钟时,再服金鸡纳霜半瓦。 
效果 前午十点钟将药服下,至午后一点时下大便两次,其心中已不觉闷热杜塞,迟至 
两点将西药服下,其日疟遂不发,俾再用生怀山药一两,熟莱菔子二钱,生鸡内金钱半煎汤,日 
服一剂,连服数日以善其后。 


<篇名>3.疟疾兼暑热
属性:天津徐姓媪,年近五旬,于季夏得疟疾。 
病因 勤俭持家,中馈事多躬操,且宅旁设有面粉庄,其饭亦由家出,劳而兼暑,遂至病疟。 
证候 其病间日一发,先冷后热,其冷甚轻,其热甚剧。恶心懒食,心中时常发热,思 
食凉物。其脉左部弦硬,右部洪实。大便干燥,小便赤涩,屡次服药无效。 
诊断 此乃肝胆伏有疟邪,胃腑郁有暑热,暑热疟邪相并而为寒热往来,然寒少热多,此方书 
所谓阳明热疟也。宜祛其肝胆之邪,兼清其胃腑之热。 
处方 生石膏(一两研细) 
均分作三包,其未发疟之日,头午用柴胡二钱煎汤送服一包,隔半日许再用开水送服一包,至次 
日前发疟五小时,再用生姜三钱煎汤送服一包。 
效果 将药按期服完后,疟疾即愈,心中发热、懒食亦愈。盖石膏善清胃热,兼能清肝胆之 
热,初次用柴胡煎汤送服者,所以和解少阳之邪也。至三次用生姜煎汤送服者,是防其疟疾 
将发与太阳相并而生寒也。 


<篇名>4.疟痢兼证
属性:天津刘××,年三十二岁,于季秋患疟又兼下痢。 
病因 因需车孔亟,机轮坏处,须得急速收拾,忙时恒彻夜不眠,劳苦过甚,遂至下痢,继又病疟。 
证候 其痢赤白参半,一昼夜十余次,下坠腹疼,其疟间日一发,寒轻热重,其脉左右皆有弦 
象,而左关独弦而有力。 
诊断 此证之脉,左右皆弦者,病疟之脉,大抵如此。其左关独弦而有力者,其病根在肝胆也。 
为肝胆有外受之邪是以脉现弦象,而病疟为其所受之邪为外感之热邪,是以左关脉象弦 
而有力,其热下迫肠中而下痢。拟清肝胆之热,散其外感之邪,则疟痢庶可同愈。 
处方 生杭芍(一两) 山楂片(三钱) 茵陈(二钱) 生麦芽(二钱) 
柴胡(钱半) 常山(钱半酒炒) 草果(钱半捣碎) 黄芩(钱半) 甘草(二钱) 生姜(三片) 
煎汤一大盅,于不发疟之日晚间服之,翌晨煎渣再服一次。 
效果 将药如法服后,疟痢皆愈。又为开生怀山药一两,生杭芍三钱,黄色生鸡内金一钱,俾日 
煎服一剂,以滋阴、培气、化瘀,连服数日以善其后。 

\chapter{霍乱门}
<篇名>1.霍乱兼转筋
属性:天津王××,年三十八岁,于季冬得霍乱证。 
病因 厂中腊底事务烦杂,劳心过度,暗生内热,又兼因怒激动肝火,怒犹未歇,遽就寝睡,至 
一点钟时,觉心中扰乱,腹中作疼,移时则吐泻交作,遂成霍乱。 
证候 心中发热而渴,恶心怔忡,饮水须臾即吐,腹中时疼时止,疼剧时则下泻,泻时异常 
觉热,偶有小便,热亦如斯,有时两腿筋转,然不甚剧,其脉象无力,却无闭塞之象。 
诊断 霍乱之证,恒有脉象无火而其实际转大热者,即或脉闭身冷显露寒凉之象,亦不可遽 
以凉断。此证脉象不见有热,而心中热而且渴,二便尤甚觉热,其为内蕴实热无疑。至其脉 
不见有热象者,以心脏因受毒麻痹,而机关之启闭无力也。拟用大剂寒凉清其内热,而辅 
以解毒消菌之品。 
处方 生石膏(三两捣细) 生杭芍(八钱) 清半夏(五钱温水淘三次) 生怀山药(五钱) 
嫩竹茹(三钱碎的) 甘松(二钱) 甘草(三钱) 
共煎汤三盅,分三次温服下。每次送服卫生防疫宝丹五十粒。 
方载后方中。甘松亦名甘松香,即西药中之缬草也。《本草纲目》谓马氏《开宝本草》,载 
其主恶气,卒心腹痛满。西 
人谓其善治转筋,是以为治霍乱要药。且其性善熏劳瘵,诚有解毒除菌之力也。 
复诊 将药分两次服完,吐泻、腹疼、转筋诸证皆愈。惟心中犹觉热作渴,二便仍觉发热。诊其脉 
较前有力,显呈有火之象。盖其心脏至此已不麻痹,启闭之机关灵活,是以脉象更改也。其犹觉热 
与渴者,因系余火未清,而吐泻之甚者最足伤阴,阴分伤损,最易生热,且善作渴,此不可但治以泻火 
之凉药也,拟兼投以大滋真阴之品。 
处方 生怀山药(一两) 大甘枸杞(一两) 北沙参(一两) 离中丹(五钱) 
药共四味,将前三味煎汤一大盅,送服离中丹一半,迟四点钟再将药渣煎汤一大盅,送服其余一半。 
效果 将药分三次服完,热退渴止,病遂全愈。 
帮助 霍乱之证,原阴阳俱有。然愚五十年经验以来,知此证属阳,而宜治以凉药者十居其八;此 
证属阴,而宜治以热药者十居其一;此证属半阴半阳,当凉热之药并用,以调剂其阴阳者,又十 
居其一。而后世论者,恒以《伤寒论》所载之霍乱为真霍乱,至于以凉药治愈之霍乱,皆系假霍乱,不知 
《伤寒论》对于霍乱之治法亦非专用热药也。有如其篇第七节云,霍乱头痛、发热、身疼痛、热多, 
欲饮水者五苓散主之。寒多,不用水者理中丸主之。夫既明言热多寒多,是显有寒热可分也。 
虽所用之五苓散中亦有桂枝而分量独轻,至泽泻、茯苓、猪苓其性皆微凉,其方原不可以热论也。且用 
显微镜审察此病之菌,系弯曲杆形,是以此证无论凉热,惟审察其传染之毒菌,现弯曲杆形即 
为霍乱无疑也。至欲细审此病之凉热百不失一,当参观霍乱方,及论霍乱治法篇,自能临证无误。 


<篇名>2.霍乱吐泻
属性:天津李姓媪,年过六旬,于仲夏得霍乱证。 
病因 天气炎热,有事出门,道途受暑,归家又复自炊,多受炭气,遂病霍乱。 
证候 恶心呕吐,腹疼泄泻,得病不过十小时,吐泻已十余次 
矣。其手足皆凉,手凉至肘,足凉至膝,心中则觉发热,其脉沉细欲无,不足四至。 
诊断 此霍乱之毒菌随溽暑之热传入脏腑也。其心脏受毒菌之麻痹,跳动之机关将停,是 
以脉沉细且迟;其血脉之流通无力,不能达于四肢,是以手足皆凉;其毒菌侵入肠胃,俾肠 
胃之气化失和,兼以脏腑之正气与侵入之邪气,互相格拒,是以恶心腹疼,吐泻交作;其 
心中发热者固系夹杂暑气,而霍乱之属阳者,即不夹杂暑气,亦恒令人心中发热也。此宜 
治以解毒清热之剂。 
处方 卫生防疫宝丹(百六十粒) 离中丹(四钱) 益元散(四钱) 
先将卫生防疫宝丹分三次用开水送服,约半点多钟服一次,服完三次,其恶心腹疼当愈,呕 
吐泄泻亦当随愈。愈后若仍觉心中热者,再将后二味药和匀,亦分三次用开水送服。每 
一点钟服一次,热退者不必尽服。 
效果 将卫生防疫宝丹分三次服完,果恶心、呕吐、腹疼、泄泻皆愈。而心中之热,未见轻减, 
继将离中丹、益元散和匀,分三次服完,其热遂消,病全愈。 


<篇名>3.霍乱脱证
属性:辽宁寇姓媪,年过六旬,得霍乱脱证。 
病因 孟秋下旬染霍乱,经医数人调治两日,病势垂危。 
证候 其证从前吐泻交作,至此吐泻全无。奄奄一息,昏昏似睡,肢体甚凉,六脉全无。询之 
犹略能言语,惟觉心中发热难受。 
诊断 此证虽身凉脉闭,而心中自觉发热,仍当以热论。其所以身凉脉闭者,因霍乱之毒菌窜入 
心脏,致心脏行血之机关将停,血脉不达于周身,所以内虽蕴热而仍身凉脉闭也。此当 
用药消其毒菌,清其内热,并以助心房之跳动,虽危险仍可挽回。 
处方 镜面朱砂(钱半) 粉甘草(一钱细面) 冰片(三分) 薄荷冰(二分) 
共研细末,分作三次服,病急者四十分钟服一次,病缓者一点钟服一次,开水送下。 
复诊 将药末分三次服完,心热与难受皆愈强半。而脉犹不出,身仍发凉,知其年过花甲,吐 
泻两日,未进饮食,其血衰惫已极,所以不能鼓脉外出以温暖于周身。 
处方 野台参(一两) 生怀地黄(一两) 生怀山药(一两) 净萸肉(八钱) 甘草(三钱蜜炙) 
煎汤两大盅,分两次温服。 
方解 方中之义,用台参以回阳,生怀地黄以滋阴,萸肉以敛肝之脱(此证吐泻之始,肝木助邪 
侮土、至吐泻之极,而肝气转先脱),炙甘草以和中气之漓。至于生山药其味甘性温,可助台参回阳, 
其汁浆稠润又可助地黄滋阴。且此证胃中毫无谷气,又可惜之以培养脾胃,俾脾胃运化诸药有力也。 
效果 将药两次服完,脉出周身亦热,惟自觉心中余火未清,知其阴分犹亏不能潜阳也。又 
用玄参、沙参、生山药各六钱,煎汤服下,病遂全愈。 


<篇名>4.霍乱暴脱证
属性:邑北境刘氏妇,年近四旬,得霍乱暴脱证。 
病因 受妊五六个月,时当壬寅秋令,霍乱盛行,因受传染,吐 
泻一昼夜,病似稍愈,而胎忽滑下。自觉精神顿散,心摇摇似不能支持。遂急延为诊视。 
证候 迨愚至欲为诊视,则病势大革,殓服已备,着于身将舁诸床,病家辞以不必入视。愚曰∶此 
系暴脱之证,一息尚存,即可挽回。遂入视之,气息若无,大声呼之亦不知应,脉象 
模糊如水上浮麻,莫辨至数。 
诊断 此证若系陈病状况,至此定难挽回,惟因霍乱吐泻已极,又复流产,则气血暴脱, 
故仍可用药挽救。夫暴脱之证,其所脱者元气也。凡元气之上脱必由于肝(所以人之将脱者,肝风先动), 
当用酸敛之品直趋肝脏以收敛之。即所以杜塞元气上脱之路,再用补助气分之药辅之。虽病势垂 
危至极点,亦可挽回性命于呼吸之间。 
处方 净杭萸肉(二两) 野党参(一两) 生怀山药(一两) 
共煎汤一大盅,温服 
方虽开就而药局相隔数里,取药迫不及待,幸其比邻刘××是愚表兄,有愚所开药方,取药二 
剂未服,中有萸肉共六钱,遂急取来暴火煎汤灌之。 
效果 将药徐徐灌下,须臾气息稍大,呼之能应,又急煎渣灌下,较前尤明了。问其心中何如, 
言甚难受,其音惟在喉间,细听可辨。须臾药已取到,急煎汤两茶杯,此时已自能服药。俾分三次温 
服下,精神顿复,可自动转。继用生山药细末八钱许,煮作茶汤,调以白糖,令其适口当点心服之。 
日两次,如此将养五六日以善其后。 
帮助 按人之气海有二,一为先天之气海,一为后天之气海。《内经》论四海之名,以膻中 
(即膈上)为气海,所藏者大气,即宗气也,养生家及针灸家皆以脐下为气海,所藏者元气, 
即养生家所谓祖气也。此气海之形状,若倒提鸡冠花形,纯系脂膜结成而中空(剖解猪腹者, 
名之为鸡冠油),肝脏下垂之脂膜与之相连,是以元气之上行,原由肝而敷布,而元气之上脱,亦即 
由肝而疏泄也(《内经》谓肝主疏泄)。惟重用萸肉以酸敛防其疏泄,借以杜塞元气上脱之路, 
而元气即可不脱矣。所最足明征者,若初次即服所开之方以治愈此证,鲜不谓人参之功居多,乃 
因取药不及,遂单服萸肉,且所服者只六钱即能建此奇功。由此知萸肉救脱之力,实远胜人参。 
盖人参以救无气之下脱,犹足恃,而以救元气之上脱,若单用之转有气高不返之弊(说见俞氏《寓意 
草》),以其性温而兼升也。至萸肉则无论上脱下脱,用之皆效。盖元气之上脱由于肝,其下脱 
亦由于肝,诚以肝能为肾行气(《内经》谓肝行肾之气),即能泻元气自下出也。为其 
下脱亦由于肝,故亦可重用萸肉治之也。 
或问 同为元气之脱何以辨其上脱下脱?答曰∶上脱与下脱,其外现之证可据以辨别者甚多。今 
但即脉以论,如此证脉若水上浮麻,此上脱之征也。若系下脱其脉即沉细欲无矣。且元气上脱 
下脱之外,又有所谓外脱者。周身汗出不止者是也。萸肉最善敛汗,是以萸肉亦能治之。来复 
汤及山萸肉解后载有治验之案数则,可参观也。 

\chapter{妇女科}
<篇名>1.怀妊受温病
属性:何姓妇,年三十二岁,受妊五月,于孟秋感受温病。 
病因 怀妊畏热,夜眠当窗,未上窗幔,自窗纱透风,感冒成温。 
证候 初病时调治失宜,温热传里,阳明府实,延医数人皆言病原当用大凉之药,因怀妊实不 
敢轻用,继延愚为诊视,见其面红气粗,舌苔白浓,中心已黄,大便干燥,小便短赤。诊 
其脉左右皆洪滑而实,一息五至强。 
诊断 据此证状脉象观之,不但阳明胃府之热甚实,即肝胆之热亦甚盛。想其未病之前必曾怒 
动肝火,若不急清其热,势将迫血妄行,危险即在目前。治以白虎加人参汤,以白虎汤解 
其热,加参以保其胎,遂为疏方俾急服之。 
处方 生石膏(三两捣细) 野党参(四钱) 生怀地黄(一两) 生怀山药(一两) 
生杭芍(五钱) 甘草(三钱) 
共煎汤三盅,分三次温服下。 
方解 按此方虽非白虎加人参汤原方,而实以生地黄代知母,以生山药代粳米,而外加 
芍药也。盖知母、地黄同能滋阴退热,而知母性滑,地黄则饶有补肾之力,粳米与山药皆有浓汁能 
和胃,而粳米汁浓而不粘,山药之汁浓而且粘,大有固肾之力。如此通变原方,自于胎妊 
大有益也。外加芍药者,欲借之以清肝胆之热也。 
复诊 将药分三次服完,翌日午前大便通下一次,热已退十之七八,脉象已非洪实,仍 
然有力,心中仍觉发热,拟再用凉润滋阴之品清之。 
处方 玄参(一两) 生怀地黄(一两) 天花粉(五钱) 生杭芍(五钱) 
鲜茅根(四钱) 甘草(二钱) 
共煎汤两盅,分两次温服下。 
效果 将药煎服两剂,病遂霍然全愈。 
帮助 凡外感有热之证,皆右部之脉盛于左部之脉,至阳明府实之证,尤必显然于右部见之。 
因胃府之脉原候于右关也。今此证为阳明府实,其右部之脉洪滑而实宜矣。而左部之脉亦 
现此象,是以知其未病之先肝中先有郁热,继为外感之热所激,则勃然发动而亦现洪滑而实之脉象也。 


<篇名>2.受妊呕吐
属性:天津王氏妇,年二十六岁,受妊后,呕吐不止。 
病因 素有肝气病,偶有拂意,激动肝气,恒作呕吐。至受妊后,则呕吐连连不止。 
证候 受妊至四十日时,每日必吐,然犹可受饮食,后则吐浸加重,迨至两月以后勺水不存。 
及愚诊视时,不能食者已数日矣。困顿已极,不能起床。诊其脉虽甚虚弱,仍现滑象,至 
数未改,惟左关微浮,稍似有力。 
诊断 恶阻呕吐,原妊妇之常,兹因左关独浮而有力,知系肝气胆火上冲,是以呕吐特甚。有谓 
恶阻呕吐虽甚剧无碍者,此未有阅历之言。愚自行道以来,耳闻目睹,因此证偾事者已 
有多人,甚勿忽视。此宜急治以镇肝降胃之品,不可因其受妊而不敢放胆用药也。 
处方 生赭石(两半轧细) 党参(三钱) 生怀山药(一两) 生怀地黄(八钱) 
生杭芍(六钱) 大甘枸杞(五钱) 净萸肉(四钱) 青黛(三钱) 清半夏(六钱) 
药共九味,先将半夏用温水淘三次,将矾味淘净,用煮菜小锅煮取清汤一盅,调以面粉煮作茶汤, 
和以白糖令其适口,服下其吐可止。再将余药八味煎汤一大盅,分三次温服。 
复诊 将药连服两剂,呕吐即止。精神气力稍振,可以起坐,其脉左关之浮已去,六部皆近和平。惟 
仍有恶心之时,懒于饮食,拟再治以开胃、理肝、滋阴、清热之剂。 
处方 生怀山药(一两) 生杭芍(五钱) 冬瓜仁(四钱捣碎) 北沙参(四钱) 
碎竹茹(三钱) 净青黛(二钱) 甘草(二钱) 
共煎汤一大盅,分两次温服下。 
效果 将药连服三剂,病遂全愈,体渐撤消,能起床矣。 


<篇名>3.怀妊得温病兼痰喘
属性:天津董姓妇,年三十四岁,怀妊,感受温病兼有痰作喘。 
病因 受妊已逾八月,心中常常发热。时当季春,喜在院中乘凉,为风袭遂成此证。 
证候 喘息有声,呼吸迫促异常,昼夜不能少卧,心中烦躁。舌苔白浓欲黄。左右寸脉皆洪 
实异常,两尺则按之不实,其数八至。大便干燥,小便赤涩。 
诊断 此证前因医者欲治其喘,屡次用麻黄发之。致其元气将脱,又兼外感之热已入阳明。其 
实热与外感之气相并上冲,是以其脉上盛下虚,喘逆若斯迫促,脉七至即为绝脉,今竟 
八至恐难挽回。欲辞不治而病家再三恳求,遂勉为拟方。以清其热,止其喘,挽救其气化之将脱。 
处方 净萸肉(一两) 生怀地黄(一两) 生龙骨(一两捣碎) 生牡蛎(一两捣碎) 
将四味煎汤,送服生石膏细末三钱,迟五点钟若热犹不退。 
煎渣再服,仍送服生石膏细末三钱。 
复诊 服药头煎次煎后,喘愈强半,遂能卧眠,迨至黎明胎忽滑下,且系死胎。再诊其脉较前 
更数,一息九至,然不若从前之滑实,而尺脉则按之即无。其喘似又稍剧,其心中烦躁依旧,且觉 
怔忡,不能支持。此乃肝肾阴分大亏,不能维系阳分而气化欲涣散也。当峻补肝肾之阴兼清外感未尽之余热。 
处方 生怀山药(六两) 玄参(两半) 熟鸡子黄(六个捻碎) 真西洋参(二钱捣为粗末) 
先将山药煎十余沸,再入玄参、鸡子黄煎汤一大碗,分多次徐徐温饮下。每饮一次,送服洋参 
末少许,饮完再煎渣取汤接续饮之,洋参末亦分多次送服,勿令余剩。 
三诊 翌日又为诊视,其脉已减去三至为六至,尺脉按之有根,知其病已回生。问其心中已不 
怔忡,惟其心中犹觉发热,此非外感之热,乃真阴未复之热也。当纯用大滋真阴之品以复其阴。 
处方 玄参(三两) 生怀山药(两半) 当归(四钱) 真西洋参(二钱捣为粗末) 
将前三味共煎汤一大碗,分多次温饮下。每饮一次送服洋参末少许。 
四诊 前方服一剂,心中已不觉热,惟腹中作疼,问其恶露所下甚少,当系瘀血作疼。治以化瘀 
血之品,其疼当自愈。 
处方 生怀山药(一两) 当归(五钱) 怀牛膝(五钱) 生鸡内金(二钱黄色的捣) 
桃仁(二钱) 红花(钱半) 真西洋参(二钱捣为粗末) 
将前六味共煎汤一大盅,送服洋参末一半,至煎渣服时再送服余一半。 
效果 前方日服一剂,服两日病遂全愈。 
或问 他方用石膏皆与诸药同煎,此证何以独将石膏为末送服?答曰∶石膏原为石质重坠之品, 
此证之喘息迫促,呼吸惟在喉间,分毫不能下达,几有将脱之势。石膏为末服之,欲借其重坠之力以 
引气下达也。且石膏末服,其退热之力一钱可抵半两,此乃屡经自服以试验之。而确能知其如斯,此证一 
日服石膏末至六钱,大热始退。若用生石膏三两,同诸药煎汤,病家将不敢服,此为救人计,不 
得不委曲以行其术也。 
或问 产后忌用寒凉,第三方用于流产之后,方中玄参重用三两,独不虑其过于苦寒乎?答曰∶玄 
参细嚼之其味甘而微苦,原甘凉滋阴之品,实非苦寒之药。是以《神农本草经》谓其微寒,善治产乳 
余疾,故产后忌用凉药而玄参则毫无所忌也。且后世本草谓大便滑泻者忌之,因误认其为苦寒也。而 
此证服过三两玄参之后,大便仍然干燥,则玄参之性可知矣。 
或问 此证之胎已逾八月,即系流产,其胎应活,何以产下竟为死胎?答曰∶胎在腹中,原有 
脐呼吸,实借母之呼吸以为呼吸,是以凡受妊者其吸入之气,可由任脉以达于胎儿脐中。此证因吸入 
之气分毫不能下达,则胎失所荫,所以不能资生也。为其不能资生,所以下降,此非因服药而下降也。 


<篇名>4.怀妊受温病兼下痢
属性:天津张氏妇,年近三旬,怀妊,受温病兼下痢。 
病因 受妊已六个月,心中恒觉发热,继因其夫骤尔赋闲,遂致 
激动肝火,其热益甚,又薄为外感所束,遂致温而兼痢。 
证候 表里俱壮热无汗,心中热极,思饮冰水,其家人不敢予。舌苔干而黄,频饮水不濡润,腹中常 
觉疼坠,下痢赤多白少,间杂以鲜血,一昼夜十余次。其脉左部弦长,右部洪滑,皆重诊有力,一息五至。 
诊断 其脉左部弦长有力者,肝胆之火炽盛也。惟其肝胆之火炽 
盛下迫,是以不但下痢赤白,且又兼下鲜血,腹疼下坠。为其右部洪滑有力,知温热已入阳明之府, 
是以舌苔干黄,心为热迫,思饮冰水。所犹喜者脉象虽热,不至甚数,且又流利无滞,胎气可保无恙 
也。宜治以白虎加人参汤以解温病之热,而更重用芍药以代方中知母,则肝热能清而痢亦可愈矣。 
处方 生石膏(三两捣细) 大潞参(五钱) 生杭芍(一两) 粳米(五钱) 甘草(三钱) 
共煎汤三盅,分三次温饮下。 
复诊 将药分三次服完,表里之热已退强半,痢愈十之七八,腹中疼坠亦大轻减,舌苔由黄变白,已有津 
液,脉象仍然有力 
而较前则和缓矣。遂即原方为之加减俾再服之。 
处方 生石膏(二两捣细) 大潞参(三钱) 生怀山药(八钱) 生杭芍(六钱) 
白头翁(四钱) 秦皮(三钱) 甘草(二钱) 
共煎汤三盅,分三次温饮下。 
方解 按此方即白虎加人参汤与白头翁汤相并为一方也。为方中有芍药、山药,是以白虎加 
人参汤中可省去知母、粳米;为白虎加人参汤中之石膏,可抵黄连、黄柏,是以白头翁汤中止 
用白头翁、秦皮,合用之则一半治温,一半治痢,安排周匝,步伍整齐,当可奏效。 
效果 将药如法服两剂,病遂全愈。 
或问 《伤寒论》用白虎汤之方定例,汗吐下后加人参,渴者加人参。此案之证非当汗吐 
下后,亦未言渴,何以案中两次用白虎皆加人参乎?答曰∶此案证兼下痢,下痢亦下之类也。 
其舌苔干黄毫无津液,舌干无液亦渴之类也。且其温病之热,不但入胃,更随下痢陷至下焦永无 
出路。惟人参与石膏并用,实能升举其下陷之温热而清解消散之,不至久留下焦以耗真阴。况 
此证温病与下痢相助为虐,实有累于胎气,几至于莫能支,加人参于白虎汤中,亦所以保其胎气 
使无意外之虞也。 


<篇名>5.产后下血
属性:天津李氏妇,年近四旬,得产后下血证。 
病因 身形素弱,临盆时又劳碌过甚,遂得斯证。 
证候 产后未见恶露,纯下鲜血。屡次延医服药血终不止。及愚诊视,已二十八日矣。其 
精神衰惫,身体羸弱,周身时或发灼,自觉心中怔忡莫支。其下血剧时腰际疼甚,呼吸常觉短 
气,其脉左部弦细,右部沉虚,一分钟八十二至。 
诊断 即此脉证细参,当系血下陷气亦下陷。从前所服之药,但知治血,不知治气,是以屡次 
服药无效。此当培补其气血,而以收敛固涩之药佐之。 
处方 生箭 (一两) 当归身(一两) 生怀地黄(一两) 净萸肉(八钱) 
生龙骨(八钱捣碎) 桑叶(十四片) 广三七(三钱细末) 
药共七味,将前六味煎汤一大盅,送服三七末一半,至煎渣再服时,仍送服其余一半。 
方解 此乃傅青主治老妇血崩之方。愚又为之加生地黄、萸肉、龙骨也。其方不但善治老妇血 
崩,即用以治少年者亦效。初但用其原方,后因治一壮年妇人患血崩甚剧,投以原方不效, 
且服药后心中觉热,遂即原方为加生地黄一两则效。从此,愚再用其方时,必加生地黄一两, 
以济黄 之热,皆可随手奏效。今此方中又加萸肉、龙骨者,因其下血既久,下焦之气化不能 
固摄,加萸肉、龙骨所以固摄下焦之气化也。 
复诊 服药两剂,下血与短气皆愈强半,诸病亦皆见愈,脉象亦有起色。而起坐片时自觉筋骨 
酸软,此仍宜治以培补气血,固摄下焦气化,兼壮筋骨之剂。 
处方 生箭 (一两) 龙眼肉(八钱) 生怀地黄(八钱) 净萸肉(八钱) 
胡桃肉(五钱) 北沙参(五钱) 升麻(一钱) 鹿角胶(三钱) 
药共八味,将前七味煎汤一大盅,鹿角胶另炖化兑服。方中加升麻者,欲以助黄 升补气分使之 
上达,兼以升提血分使不下陷也。 
三诊 将药连服三剂,呼吸已不短气,而血分则犹见少许,然非鲜血而为从前未下之恶露,此吉兆也。若 
此恶露不下,后必为恙。且又必须下净方妥,此当兼用化瘀之药以催之速下。 
处方 生箭 (一两) 龙眼肉(八钱) 生怀地黄(八钱) 生怀山药(六钱) 
胡桃肉(五钱) 当归(四钱) 北沙参(三钱) 鹿角胶(四钱) 
广三七(三钱细末) 
药共九味,先将前七味煎汤一大盅,鹿角胶另炖化兑汤药中,送服三七末一半,至煎渣再服时,仍将 
所余之鹿角胶炖化兑汤药中,送服所余之三七末。 
方解 按此方欲用以化瘀血,而不用桃仁、红花诸药者,恐有妨于从前之下血也。且此方中原有 
善化瘀血之品,鹿角胶、三七是也。盖鹿角之性原善化瘀生新,熬之成胶其性仍在。前此之恶露 
自下,实多赖鹿角胶之力,今又助之以三七,亦化瘀血不伤新血之品。连服数剂,自不难将恶露尽化也。 
效果 将药连服五剂,恶露下尽,病遂全愈。 


<篇名>6.产后手足抽掣
属性:天津于氏妇,年过三旬,于产后得四肢抽掣病。 
病因 产时所下恶露甚少,至两日又分毫恶露不见,迟半日遂发抽掣。 
证候 心中发热,有时觉气血上涌,即昏然身驱后挺,四肢抽掣。其腹中有时作疼,令人揉之则 
少瘥,其脉左部沉弦,右部沉涩,一息四至强。 
诊断 此乃肝气胆火,挟败血上冲以瘀塞经络,而其气火相并上冲不已,兼能妨碍神经,是以昏 
然后挺而四肢作抽掣也。当降其败血,使之还为恶露泻出,其病自愈。 
处方 怀牛膝(一两) 生杭芍(六钱) 丹参(五钱) 玄参(五钱) 
苏木(三钱) 桃仁(三钱去皮) 红花(二钱) 土鳖虫(五大个捣) 红娘虫(即樗鸡,六大个捣) 
共煎汤一盅,温服。 
效果 此药煎服两剂,败血尽下,病若失。 


<篇名>7.产后瘕
属性:邑城西韩氏妇,年三十六岁,得产后 瘕证。 
病因 生产时恶露所下甚少,未尝介意,迟至半年遂成 瘕。 
证候 初因恶露下少,弥月之后渐觉少腹胀满。因系农家,时当麦秋忙甚,未暇延医服药。又 
迟月余则胀而且疼,始服便方数次皆无效。后则疼处按之觉硬,始延医服药,延医月余,其疼似减轻 
而硬处转见增大,月信自产后未见。诊其脉左部沉弦,右部沉涩,一息近五至。 
诊断 按生理正则,产后两月,月信当见;有孩吃乳,至四月亦当见矣。今则已半载月信未见,因其 
产后未下之恶露,结 瘕于冲任之间,后生之血遂不能下为月信,而尽附益于其上,俾其日有增长,是 
以积久而其硬处益大也。是当以消 瘕之药消之,又当与补益之药并用,使之消 瘕而不至有伤气化。 
处方 生箭 (五钱) 天花粉(五钱) 生怀山药(五钱) 三棱(三钱) 
莪术(三钱) 当归(三钱) 白术(二钱) 知母(二钱) 
生鸡内金(二钱黄色的捣) 桃仁(二钱去皮) 
共煎汤一大盅,温服。 
复诊 将药连服六剂,腹已不疼,其硬处未消,按之觉软,且从前食量减少,至斯已复其旧。 
其脉亦较前舒畅,遂即原方为之加减俾再服之。 
处方 生箭 (五钱) 天花粉(五钱) 生怀山药(四钱) 三棱(三钱) 
莪术(三钱) 怀牛膝(三钱) 野党参(三钱) 知母(三钱) 
生鸡内金(二钱黄色的捣) 生水蛭(二钱捣碎) 
共煎汤一大盅,温服。 
效果 将药连服十五六剂(随时略有加减),忽下紫黑血块若干,病遂全愈。 
帮助 妇女 瘕治愈者甚少,非其病之果难治也。《金匮》下瘀血汤,原可为治妇女 瘕之主方。 
特其药性猛烈,原非长服之方。于 瘕初结未坚硬者,服此药两三次或可将病消除。若至累月累年, 
瘕结如铁石,必须久服,方能奏效者,下瘀血汤原不能用。乃医者亦知下瘀血汤不可治坚结之 瘕, 
遂改用桃仁,红花、丹参、赤芍诸平和之品;见其 瘕处作疼,或更加香附、延胡、青皮、木香诸 
理气之品,如此等药用之以治坚结之 瘕,可决其虽服至百剂,亦不能奏效。然仗之奏效则不足, 
伤人气化则有余。若视为平和而连次服之,十余剂外人身之气化即暗耗矣。此所以治 瘕者十中难 
愈二三也。若拙拟之方其三棱、莪术、水蛭,皆为消 瘕专药。即鸡内金人皆用以消食,而以消 
瘕亦甚有力。更佐以参、 、术诸补益之品,则消 瘕诸药不虑其因猛烈而伤人。且又用花粉、知 
母以调剂补药之热,牛膝引药下行以直达病所,是以其方可久服无弊。而坚结之 瘕即可徐徐消除 
也。至于水蛭必生用者,理冲丸后论之最详。且其性并不猛烈过甚,治此证者,宜放胆用之以挽救人命。 


<篇名>8.血闭成瘕
属性:邻庄刘氏妇,年二十五岁,经血不行,结成 瘕。 
病因 处境不顺,心多抑郁,以致月信渐闭,结成 瘕。 
证候 瘕初结时,大如核桃,屡治不消,渐至经闭后则 瘕浸长。三年之后大如复盂,按之 
甚硬。渐至饮食减少,寒热往来,咳嗽吐痰,身体羸弱,亦以为无可医治待时而已。后忽闻 
愚善治此证,求为诊视。其脉左右皆弦细无力,一息近六至。 
诊断 此乃由经闭而积成 瘕,由 瘕而浸成虚劳之证也。此宜先注意治其虚劳,而以消 瘕之品辅之。 
处方 生怀山药(一两) 大甘枸杞(一两) 生怀地黄(五钱) 玄参(四钱) 
沙参(四钱) 生箭 (三钱) 天冬(三钱) 三棱(钱半) 
莪术(钱半) 生鸡内金(钱半黄色的捣) 
共煎汤一大盅,温服。 
方解 方中用三棱、莪术,非但以之消 瘕也。诚以此证廉于饮食,方中鸡内金固能消食,而 
三棱、莪术与黄 并用,更有开胃健脾之功。脾胃健壮,不但善消饮食,兼能运化药力使病速愈也。 
复诊 将药连服六剂,寒热已愈,饮食加多,咳嗽吐痰亦大轻减。 瘕虽未见消,然从前时或 
作疼今则不复疼矣。其脉亦较前颇有起色。拟再治以半补虚劳半消 瘕之方。 
处方 生怀山药(一两) 大甘枸杞(一两) 生怀地黄(八钱) 生箭 (四钱) 
沙参(四钱) 生杭芍(四钱) 天冬(四钱) 三棱(二钱) 
莪术(二钱) 桃仁(二钱去皮) 生鸡内金(钱半黄色的捣) 
共煎一大盅,温服。 
三诊 将药连服六剂,咳嗽吐痰皆愈。身形已渐强壮,脉象又较前有力,至数复常。至此虚劳 
已愈,无庸再治。其 瘕虽未见消,而较前颇软。拟再专用药消之。 
处方 生箭 (六钱) 天花粉(五钱) 生怀山药(五钱) 三棱(三钱) 
莪术(三钱) 怀牛膝(三钱) 潞党参(三钱) 知母(三钱) 
桃仁(二钱去皮) 生鸡内金(二钱黄色的捣) 生水蛭(二钱捣碎) 
共煎汤一大盅,温服。 
效果 将药连服十二剂,其瘀血忽然降下若干,紫黑成块,杂以脂膜, 瘕全消。为其病积 
太久,恐未除根,俾日用山楂片两许,煮汤冲红蔗糖,当茶饮之以善其后。 


<篇名>9.产后温病
属性:天津李氏妇,年二十七岁,于中秋节后得温病。 
病因 产后六日,更衣入厕,受风。 
证候 自厕返后,觉周身发冷,更数小时,冷已又复发热,自用生姜,红糖煎汤乘热饮之,周 
身得汗稍愈,至汗解而其热如故。迁延两日热益盛,心中烦躁作渴。急延愚为诊视,见其满面火 
色,且微喘,诊其脉象洪实,右部尤甚,一分钟九十三至。舌苔满布白而微黄,大便自病后未行。 
诊断 此乃产后阴虚生内热,略为外感拘束而即成温病也。其心中烦躁而渴者,因产后肾阴 
虚损,不能上达舌本,且不能与心火相济也。其微喘者,因肾虚不能纳气也。其舌苔白而微 
黄者,热已入阳明之府也。其脉洪实兼数者,此阳明府热已实,又有阴虚之象也。宜治以 
白虎加人参汤更少为变通之,方于产后无碍。 
处方 生石膏(三两捣细) 野台参(四钱) 玄参(一两) 生怀山药(八钱) 甘草(三钱) 
共煎汤三盅,分三次温饮下。 
方解 按此方即白虎加人参汤,以玄参代知母,生山药代粳米也。伤寒书中用白虎汤之定例, 
汗吐下后加人参,以其虚也;渴者加人参,以其津液不上潮也,至产后则虚之尤虚,且又作渴, 
其宜加人参明矣。至以玄参代知母者,因玄参《神农本草经》原谓其治产乳余疾也。以生山 
药代粳米者,因山药之甘温既能代粳米和胃,而其所含多量之蛋白质,更能补益产后者之肾虚 
也。如此变通,其方虽在产后用之,可毫无妨碍,况石膏《神农本草经》原谓其微寒,且明载其主产乳乎。 
复诊 服药一剂,热退强半,渴喘皆愈。脉象已近和平,大便犹未通下。宜大滋真阴以退其 
余热,而复少加补气之药佐之。诚以气旺则血易生,即真阴易复也。 
处方 玄参(二钱) 野党参(五钱) 
共煎汤两盅,分两次温饮下。 
效果 将药煎服两剂,大便通下,病遂全愈。 


<篇名>10.流产后满闷
属性:天津张姓妇年二十六岁,流产之后胃脘满闷,不能进食。 
病因 孕已四月,自觉胃口满闷,倩人以手为之下推,因用力下推至脐,遂至流产。 
证候 流产之后,忽觉气血上涌充塞胃口,三日之间分毫不能进 
食。动则作喘,头目眩晕,心中怔忡,脉象微弱,两尺无根。 
诊断 此证因流产后下焦暴虚,肾气不能固摄冲气,遂因之上冲。夫冲脉原上隶阳明胃府, 
其气上冲胃气即不能下降(胃气以息息下行为顺),是以胃中胀满,不能进食。治此等证者,若用 
开破之药开之,胀满去而其人或至于虚脱。宜投以峻补之剂,更用重镇之药辅之以引之下行,则 
上之郁开而下焦之虚亦即受此补剂之培养矣。 
处方 大潞参(四钱) 生赭石(一两轧细) 生怀山药(一两) 熟怀地黄(一两) 
玄参(八钱) 净萸肉(八钱) 紫苏子(三钱炒捣) 生麦芽(三钱) 
共煎汤一大盅,分两次温服下。 
方解 按方中用生麦芽,非取其化食消胀也。诚以人之肝气宜升,胃气宜降,凡用重剂降胃, 
必须少用升肝之药佐之,以防其肝气不舒。麦芽生用原善舒肝,况其性能补益胃中酸 
汁,兼为化食消胀之妙品乎。 
效果 将药煎服一剂,胃中豁然顿开,能进饮食,又连服两剂,喘与怔忡皆愈。 


<篇名>11.月闭兼温疹靥急
属性:天津杨氏女,年十五岁,先患月闭,继又染温疹靥急。 
病因 自十四岁月信已通,后因肝气不舒,致月信半载不至,继又感发温疹,初见点即靥。 
证候 初因月信久闭,已发热瘦弱,懒于饮食,恒倦卧终日不起。继受温疹,寒热往来,其寒 
时觉体热减轻,至热时,较从前之热增加数倍,又加以疹初见点即靥,其毒热内攻。心中烦躁怔忡,剧 
时精神昏愦,恒作谵语,舌苔白而中心已黄,毫无津液。大便数日未行,其脉觉寒时似近闭塞,觉热 
时又似洪大而重按不实,一息五至强。 
诊断 此证因阴分亏损将成痨瘵,又兼外感内侵,病连少阳,是以寒热往来,又加以疹毒之热,不 
能外透而内攻,是以烦躁怔忡,神昏谵语,此乃内伤外感两剧之证也。宜用大剂滋其 
真阴清其毒热,更佐以托疹透表之品当能奏效。 
处方 生石膏(二两捣细) 野台参(三钱) 玄参(一两) 生怀山药(一两) 
大甘枸杞(六钱) 知母(四钱) 连翘(三钱) 蝉蜕(二钱) 
茵陈(二钱) 僵蚕(钱半) 鲜芦根(四钱) 
共煎汤三盅,分三次温饮下。嘱其服一剂热不退时,可即原方再服,若服至大便通下且微溏时,即宜停药勿服。 
复诊 将药煎服两剂,大热始退,不复寒热往来,疹未表出而心已不烦躁怔忡。知其毒由 
内消,当不变生他故。大便通下一次亦未见溏,再诊其脉已近和平,惟至数仍数,和其外感已愈十 
之八九,而真阴犹未复也。拟再滋补其真阴,培养其血脉,俾其真阴充足,血脉调和,月信自然通顺而不愆期矣。 
处方 生怀山药(一两) 大甘枸杞(一两) 玄参(五钱) 地骨皮(五钱) 
龙眼肉(五钱) 北沙参(五钱) 生杭芍(三钱) 生鸡内金(钱半黄色的捣) 
甘草(二钱) 
共煎汤一大盅,温服。 
三诊 将药连服四剂,饮食增加,精神较前振作,自觉诸病皆无,惟腹中间有疼时,此月信欲 
通而未能即通也。再诊其脉已和平四至矣。知方中凉药宜减,再少加活血化瘀之品。 
处方 生怀山药(一两) 大甘枸杞(一两) 龙眼肉(六钱)当归(五钱) 
玄参(三钱) 地骨皮(三钱) 生杭芍(三钱) 生鸡内金(钱半黄色的捣) 
土鳖虫(五个大者捣) 甘草(钱半) 生姜(三片) 
共煎汤一大盅,温服。 
效果 此药连服十剂,腹已不疼,身形已渐胖壮,惟月信仍未至,俾停药静候。旬日后月 
信遂见,因将原方略为加减,再服数剂以善其后。 
或问 方书治温疹之方,未见有用参者。开首之方原以治温疹为急务,即有内伤亦当从缓 
治之,而方中用野台参者其义何居?答曰∶《伤寒论》用白虎汤之例,汗吐下后加人参,以其 
虚也;渴者加人参,以其气虚不能助津液上潮也。令此证当久病内亏之余,不但其血分虚损,其 
气分亦必虚损。若但知用白虎汤以清其热,不知加参以助之,而热转不清,且更有病转加剧之 
时(观白虎加人参以山药代粳米汤后附载医案可知)。此证之用人参,实欲其热之速退也。且此证疹 
靥之急,亦气分不足之故。用参助石膏以清外感之热,即借其力以托疹毒外出,更可借之以补从前 
之虚劳。是此方中之用参,诚为内伤外感兼顾之要药也。 
或问 凡病见寒热往来者,多系病兼少阳,是以治之者恒用柴胡以和解之。今方中未用柴胡,而 
寒热往来亦愈何也?答曰∶柴胡虽能和解少阳,而其升提之力甚大。此证根本已虚,实 
不任柴胡之升提。方中茵陈其性凉而能散,最能宣通少阳之 
郁热,可为柴胡之代用品。实为少阳病兼虚者无尚之妙药也。况又有芦根亦少阳药,更可与 
之相助为理乎?此所以不用柴胡亦能愈其寒热往来也。 


<篇名>12.处女经闭
属性:天津陈氏女,年十七岁,经通忽又半载不至。 
病因 项侧生有瘰 ,服药疗治,过于咸寒,致伤脾胃,饮食减少,遂至经闭。 
证候 午前微觉寒凉,日加申时,又复潮热,然不甚剧。黎明时或微出汗,咳嗽有痰,夜间 
略甚,然仍无妨于安眠。饮食消化不良,较寻常减半。心中恒觉发热思食凉物,大便干燥, 
三四日一行。其脉左部弦而微硬,右部脉亦近弦,而重诊无力,一息搏逾五至。 
诊断 此因饮食减少,生血不足以至经闭也。其午前觉凉者,其气分亦有不足,不能乘阳气 
上升之时而宣布也。至其晚间之觉热,则显为血虚之象。至于心中发热,是因阴虚生内热 
也。其热上升伤肺易生咳嗽,胃中消化不良易生痰涎,此咳嗽又多痰也。其大便燥结者,因脾 
胃伤损失传送之力,而血虚阴亏又不能润其肠也。左脉弦而兼硬者,心血虚损不能润肝滋肾也。 
右脉弦而无力者,肺之津液胃之酸汁皆亏,又兼肺胃之气分皆不足也。拟治以资生通脉汤,复即 
原方略为加减,俾与证相宜。 
处方 白术(三钱炒) 生怀山药(八钱) 大甘枸杞(六钱) 龙眼肉(五钱) 
生怀地黄(五钱) 玄参(四钱) 生杭芍(四钱) 生赭石(四钱轧细) 
当归(四钱) 桃仁(二钱) 红花(钱半) 甘草(二钱) 
共煎汤一大盅,温服。 
复诊 将药连服二十余剂(随时略有加减),饮食增多,身形健壮,诸病皆愈。惟月信犹未通, 
宜再注意通其月信。 
处方 生水蛭(一两轧为细末) 生怀山药(半斤轧为细末) 
每用山药末七钱,凉水调和煮作茶汤,加红蔗糖融化,令其适口,以之送服水蛭末六分, 
一日再服,当点心用之,久则月信必通。 
效果 按方服过旬日,月信果通下,从此经血调和无病。 
方解 按水蛭《神农本草经》原无炙用之文,而后世本草谓若不炙即用之,得水即活,殊为荒 
唐之言。尝试用此药,先用炙者无效,后改用生者,见效甚速。其性并不猛烈,惟稍有刺激 
性。屡服恐于胃不宜,用山药煮粥送服,此即《金匮》硝石矾石散送以大麦粥之义也。且山 
药饶有补益之力,又为寻常服食之品,以其粥送水蛭,既可防其开破伤正,且又善于调和胃腑也。 


<篇名>13.血崩证
属性:天津徐姓妇,年十八岁,得血崩证。 
病因 家庭不和,激动肝火,因致下血不止。 
证候 初时下血甚多,屡经医治,月余血虽见少,而终不能止。 
脉象濡弱,而搏近五至。呼吸短气,自觉当呼气外出之时, 
稍须努力,不能顺呼吸之自然。过午潮热,然不甚剧。 
诊断 此胸中大气下陷,其阴分兼亏损也。为其大气下陷,所以呼气努力,下血不止,为其 
阴分亏损,所以过午潮热。宜补其大气,滋其真阴,而兼用升举固涩之品方能治愈。 
处方 生箭 (一两) 白术(五钱炒) 大生地(一两) 龙骨(一两 捣) 
牡蛎(一两 捣) 天花粉(六钱) 苦参(四钱) 黄柏(四钱) 
柴胡(三钱) 海螵蛸(三钱去甲) 茜草(二钱) 
西药麦角中者一个,搀乳糖五分,共研细,将中药煎汤两大盅,分两次服,麦角末亦分两次送服。 
效果 煎服一剂,其血顿止,分毫皆无,短气与潮热皆愈。再为开调补气血之剂,俾服数剂以善其后。 



\end{document}