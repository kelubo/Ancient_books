% 神农本草经
% 神农本草经.tex

\documentclass[12pt,UTF8]{ctexbook}

% 设置纸张信息。
\usepackage[a4paper,twoside]{geometry}
\geometry{
	left=25mm,
	right=25mm,
	bottom=25.4mm,
	bindingoffset=10mm
}

% 设置字体,并解决显示难检字问题。
\xeCJKsetup{AutoFallBack=true}
\setCJKmainfont{SimSun}[BoldFont=SimHei, ItalicFont=KaiTi, FallBack=SimSun-ExtB]

% 目录 chapter 级别加点(.)。
\usepackage{titletoc}
\titlecontents{chapter}[0pt]{\vspace{3mm}\bf\addvspace{2pt}\filright}{\contentspush{\thecontentslabel\hspace{0.8em}}}{}{\titlerule*[8pt]{.}\contentspage}

% 设置 part 和 chapter 标题格式。
\ctexset{
%	part/name= {卷,},
%	part/number={\chinese{part}},
	chapter/name={卷,},
	chapter/number={\chinese{chapter}},
	section/name={},
	section/number={}
}

% 设置古文原文格式。
\newenvironment{yuanwen}{\bfseries\zihao{4}}

% 设置署名格式。
\newenvironment{shuming}{\hfill\bfseries\zihao{4}}

% 注脚每页重新编号,避免编号过大。
\usepackage[perpage]{footmisc}

\title{\heiti\zihao{0} 神农本草经}
\author{}
\date{}

\begin{document}
	
\maketitle
\tableofcontents
	
\frontmatter

\chapter{前言}

《神农本草经》简称《本经》,为中医四大经典著作之一,为我国现存最早的药物学专著。全书共收载中药365味。

此书大约成书于东汉,为东汉以前许多医药学家经验的总结。原书早已亡佚,但被《本草经集注》以红字为标志,《证类本草》以白字为标志而保存下来。明清以来,许多学者以《证类本草》为主,并从中钩沉出来,而形成多种《本经》辑本,进而又有多种注释本。选择以清·顾观光的《神农本草经》辑本为底本,并参照孙本、卢本、周本进行校对。

\chapter{顾氏自序}

李濒湖云:“神农古本草,凡三卷三品,共三百六十五种,首有名例数条,至陶氏作《别录》,乃拆分各部,而三品亦移改,又拆出青葙\footnote{xi\=ang}、赤小豆二条(按《本经》目录,青葙子在下品,非后人拆出也。疑“葙”当作“蘘\footnote{r\'ang}”)。故有三百六十七种,逮乎唐宋屡经变易旧制莫考。”(此上并李氏语)。今考《本经》三品不分部数,上品一百二十种,中品一百二十种,下品一百二十五种(见《本经》名例),品各一卷,又有序录一卷,故梁·《七录》云三卷,而陶氏《别录》云四卷,韩保昇谓《神农本草》上中下并序录合四卷是也。梁·陶隐居《名医别录》始分玉、石、草、木三品为三卷,虫、兽、果、菜、米、食,有名未用三品为三卷,又有序录一卷,合为七卷,故《别录》序后云:“《本草经》卷上,序药性之原本,论病名之形诊,题记品录,详览施用;《本草经》卷中,玉、石、草、木三品;《本草经》卷下,虫、兽、果、菜、米、食三品,有名未用三品,右三卷其中下二卷,药合七百三十种,各别有目录,并朱墨杂书并子注,今大书分为七卷。”(以上并陶氏语)。盖陶氏《别录》仍沿用《本经》上、中、下三卷之名,而中下二卷并以三品,分为子卷,《唐本草》讥其草木同品,虫兽共条,披览既难,图绘非易是也。《别录》于《本经》诸条间有并析,如胡麻《经》云叶名青蘘,即在胡麻条下,而《别录》乃分之(《本经》目录无青蘘),中品葱薤\footnote{xi\`e},下品胡粉、锡镜鼻,并各自为条,而《别录》乃合之,以此类推,凡《证类本草》三品与《本经》目录互异者,疑皆陶氏所移,李濒湖所谓拆分各部,移改三品者是也。青蘘之分,盖自《别录》始,(《唐本草》注云,《本经》在草部上品,即指《别录》原次言之。),赤小豆之分,则自《唐本草》始,是为三百六十七种,《唐本草》退姑活,别覉\footnote{j\=i}、石下长卿、翘根、屈草、淮木于有名未用,故云三百六十一种(见《别录》序后,《唐本草》注。),宋本草又退彼子于有名未用,故云三百六十种(见《补注》总叙后),今就《证类本草》三品计之,上品一百四十一种,中品一百十三种,下品一百二十五种,已与《本经》名例绝不相符,又有人部一种,有名未用七种并不言于三品何属,李濒湖所谓屡经变易,旧制莫考者是也。李氏《纲目》世称为集大成,以今考之《本经》,而误注《别录》者四种(萆薢\footnote{b\`i xi\`e}、葱、薤、杏仁。);从《本经》拆出而误注他书者二种(土蜂、桃蠹虫);原无经文而误注《本经》者一种(绿青);明注《本经》,而经文混入《别录》者三种(葈\footnote{x\v{i}}耳实、鼠妇、石龙子);经文混入《别录》,而误注《别录》者六种(王不留行、龙眼、肤青、姑活、石下长卿、燕屎);《别录》混入经文,而误注《本经》者四种(升麻、由跋、赭魁、鹰屎白)。夫以濒湖之愽洽而舛\footnote{chu\v{a}n}误至此,可见著书难,校书亦复不易,《开宝本草》序云,朱字墨字无本得同,旧注新注其文互缺,则宋本已不能无误,又无论濒湖矣,今去濒湖二百余载,古书亡佚殆尽,幸而《证类本草》灵光岿然,又幸而《纲目》卷二具载《本经》目录,得以寻其原委,而析其异同,《本经》三百六十五种之文,章章可考,无阙佚,无羡衍,岂非天之未丧斯文,而留以有待乎。近孙渊如尝辑是书,刊入问经堂中,惜其不考《本经》目录,故三品种数,显与名例相违,缪仲淳、张路玉辈,未见《证类本草》,而徒据《纲目》以求经文,尤为荒陋。大率考古者不知医,业医者不知古,遂使赤文绿字埋没于陈编蠹简之中,不及今而亟为搜辑,恐数百年后,《证类》一书又复亡佚,则经文永无完璧之期矣。爰于繙\footnote{f\=an}阅之余,重为甄录其先后,则以《本经》目录定之,仍用韩氏之说,别为序录一卷,而唐宋类书所引有出《证类》外者,亦备录焉,为考古计,非为业医计也,而非邃于古而明于医者,恐其闻之而骇,且惑也。
~\\

\begin{shuming}
甲辰九月霜降日顾观光识
\end{shuming}

\chapter{邵序}

内容:《记》曰∶医不三世,不服其药。郑康成曰∶慎物齐也。孔冲远引旧说云∶三世者,一 
曰《黄帝针灸》,二曰《神农本草》,三曰《素女脉诀》。康成《周礼注》亦曰∶五药,草、 
木、虫、石、谷也。其治合之齐,则存乎神农子仪之术,是《礼记》注所谓慎物齐者,犹言 
治合之齐,指本草诸书而言也。冲远既引旧说,复疑其非郑义过矣。《汉书》引本草方术而 
《艺文志》缺载,贾公彦引《中经簿》,有《子仪本草经》一卷,不言出于神农。至隋《经 
籍志》,始载《神农本草经》三卷,与今分上、中、下三品者相合,当属汉以来旧本。《隋志》 
又载雷公《本草集注》四卷,《蔡邕本草》七卷,今俱不传。自《别录》以后,累有损益升 
降,随时条记,或传合本文,不相别白。据陆元朗《经典释文》所引,则经文与名医所附益 
者,合并为一,其来旧矣。孙君伯渊偕其从子因《大观本草》黑白字书,厘正《神农本经》 
三卷,又据《太平御览》引《经》云∶生山谷生川泽者,定为本文,其有预章、朱崖、常山、 
奉高,郡县名者,定为后人羼入。释《本草》者,以吴普本为最古,散见于诸书征引者,缀 
集之以补《大观》本所未备,疏通古义,系以考证,非澹雅之才,沉郁之思,未易为此也。 
古者协阴阳之和,宣羸缩之节,凡夫含声负气,以及倒生旁达, 飞蠕动之伦,胥尽其性, 
遇物能名,以达于利用,生生之具,儒者宜致思焉。《淮南王书》曰∶地黄主属骨,而甘草 
主生肉之药也。又曰∶大戟去水,葶苈愈张,用之不节,乃反为病。《论衡》曰∶治风用风, 
治热用热,治边用蜜丹;《潜夫论》曰∶治疾当真人参,反得支罗服;当得麦门冬,反蒸横 
麦,已而不识真,合而服之,病以浸剧。斯皆神农之绪言,惟其赡涉者博,故引类比方,悉 
符药论。后儒或忽为方技家言,渔猎所及,又是末师而非往古,甚至经典所载鸟兽草木,亦 
辗转而昧其名,不已慎乎!《后汉书·华佗传》∶吴普从佗学,依准佗疗,多所全济,佗以五 
禽之戏别传。又载魏明帝使普为禽戏,普以其法语诸医,疑其方术相传,别有奇文异数。今 
观普所释本草,则神农、黄帝、岐伯、雷公、桐君、医和、扁鹊,以及后代名医之说,靡不 
赅载,则其多所全济,由于稽考之勤,比验之密,而非必别有其奇文异数。信乎!非读三世 
书者,不可服其药也。世俗所传,黄帝、神农、扁鹊之书,多为后人窜易,余愿得夫闳览博 
物者为之是正也。因孙君伯仲校定《本草》,而发其端。至其书考证精审 
余姚邵晋涵序

\chapter{张序}

内容:儒者不必以医名,而知医之理,则莫过于儒者。春秋时,和与缓,神于医者也。其通《周易》,辨皿虫之义,医也,而实儒也。世之言医者,必首推神农。然使神农非与太乙游, 
则其传不正,非作赭鞭钩 ,巡五岳四渎,则其识不广;非以土地所生万千类,验其能治 
与 
历试之,亲尝之,亦仅与商贾市贩等耳,于医乎何与?吾故曰∶神农,千古之大儒也。考《 
崇文总目》,载《食品》一卷,《五脏论》一卷,皆系之神农。其本久不传,传之者,《神农 
本草经》耳!而亦无专本。唐审元裒辑之,《书录解题》谓之《大观本草》,《读书志》谓之 
《证 
类本草》。阙后缪雍有《疏》,卢之颐有《乘雅半偈》,皆以《本经》为之主。然或参以臆说, 
或益以衍断,解愈纷,义愈晦,未有考核精审,卓然有所发明者。则证古难,证古而折衷于 
至是,为尤难。孙渊如观察,偕其从子凤卿,辑《神农本草经》三卷。于吴普《名医》外, 
益以《说文》、《尔雅》、《广雅》、《淮南子》、《抱朴子》诸书,不列古方,不论脉证,而古圣 
殷殷治世之意,灿然如列眉。孔子曰∶多识于鸟兽草木之名。又曰∶致知在格物。则是书也, 
非徒医家之书,而实儒家之书也,其远胜于希雍之诸人也固宜。或以《本草》之名,始见《汉 
书·平帝纪》《楼护传》,几有疑于《本草经》者。然神农始尝百草,始有医药,见于《三皇 
纪》矣;因三百六十五种注释为七卷,见于陶隐居《录》矣;增一百十四种,广为二十卷, 
《唐本草》宗之;增一百三十三种,孟昶复加厘定,《蜀本草》又宗之。至郡县,本属后人 
所附益,《经》但云生山谷、生川泽耳。《洪范》以康宁为福,《雅颂》称寿考万年,又何疑 
于久服轻身延年,为后世方士之说哉?大抵儒者之嗜学如医然,渊源,其脉也;复审,其胗 
视也。辨邪正,定是非,则温寒平热之介也。观察方闻缀学,以鸿儒名,海内求其着述者, 
如金膏水碧之珍。凤卿好博闻,研丹吮墨,日以儒为事,则上溯之羲皇以前,数千年如一日, 
非嗜之专且久而能然耶?顾吾独怪是编中,无所谓治书癖 
嘉庆四年太岁在己未冬十月望日宣城张炯撰于瞻园之灌术庄

\chapter{孙序}

内容:《神农本草经》三卷,所传白字书,见《大观本草》。 
按∶《嘉 补注》序云∶所谓《神农本经》者,以朱字;《名医》因神农旧条而有增补者, 
以墨字间于朱字。《开宝重定》序云∶旧经三卷,世所流传,《名医别录》,互为编纂。至梁 
贞白先生陶弘景,乃以《别录》参其《本经》,朱墨杂书,时谓明白。据此,则宋所传黑白 
字书,实陶弘景手书之本。自梁以前,神农、黄帝、岐伯、雷公、扁鹊,各有成书,魏吴普 
见之,故其说药性主治,各家殊异。后人纂为一书,然犹有旁注,或朱、墨字之别,《本经》 
之文以是不乱。旧说,本草之名,仅见《汉书·平帝纪》及《楼护传》。予按∶《艺文志》有 
《神农黄帝食药》七卷,今本伪为《食禁》,贾公彦《周礼》医师疏引其文,正作《食药》, 
宋人不考。遂疑《本草》非《七略》中书。贾公彦引《中经簿》,又有《子仪本草经》一卷, 
疑亦此也。梁《七录》有《神农本草》三卷,其卷数不同者,古今分合之异。神农之世,书 
契未作,说者以此疑《经》,如皇甫谧言,则知四卷成于黄帝。陶弘景云,轩辕以前,文本 
未传,药性所主,尝以识识相因。至于桐、雷,乃着在于编简,此书当与《素问》同类,其 
言良是。且《艺文志》,农、兵、五行、杂占、经方、神仙诸家,俱有神农书。大抵述作有 
本,其传非妄。是以《博物志》云∶太古书今见存,有《神农经》、《春秋传注》。贾逵以《三 
坟》为三皇之书,神农预其列。《史记》言∶秦始皇不去医药卜筮之书,则此《经》幸与《周 
易》并存。颜之推《家训》乃云∶《本草》神农所述,而有豫章、朱崖、赵国、常山、奉高、 
真定、临淄、冯翊等郡县名,出诸药物,皆由后人所羼,非本文。陶弘景亦云∶所出郡县, 
乃后汉时制,疑仲景、云化等所记。按∶薛综注《张衡赋》引《本草经》∶太一禹余粮,一 
名石脑,生山谷。是古本无郡县名。《太平御览》引《经》上云∶生山谷或川泽,下云生某 
山某郡。明生山谷,《本经》文也;其下郡县,《名医》所益。今《大观本》俱作黑字。或合 
其文,云某山川谷,某郡川泽,恐传写之误,古本不若此。仲景、元化后,有吴普、李当之, 
皆修此经。当之书,世少行用。《魏志·华佗传》,言普从佗学。隋《经籍志》称《吴普本草》 
,梁有六卷。《嘉 本草》云∶普修《神农本草》成四百四十一种。唐《经籍志》尚存六卷。 
今广内不复存,惟诸书多见引据。其说药性,寒温五味最为详悉,是普书宋时已佚,今其文 
惟见掌禹锡所引《艺文类聚》《初学记》《后汉书注》《事类赋》诸书。《太平御览》引据尤多, 
足补《大观》所缺,重是别录前书,因采其文附于《本经》,亦略备矣。其普所称,有神农 
说者,即是《本经》,《大观》或误作黑字,亦据增其药物,或数浮于三百六十五种,由后人 
以意分合,难以定之。其药名,有禹余粮、王不留行、徐长卿、鬼督邮之属不类太古时文。 
按字书以禹为虫,不必夏禹。其余名号,或系后人所增,或声音传述,改古旧称之致。又《经》 
有云∶宜酒渍者。或以酒非神农时物,然《本草衍义》已据《素问》首言“以妄为常,以酒 
为浆”,谓酒自黄帝始。又按∶《文选注》引《博物志》,亦云“杜康作酒”。王着《与杜康绝 
交书》曰∶康,字仲宁,或云黄帝时人,则俱不得疑《经》矣。孔子云∶述而不作,信而好 
古。又云∶多误于鸟兽草木之名。今儒家拘泥耳目,未能及远,不睹医经、本草之书;方家 
循守俗书,不察古本药性异同之说,又见明李时珍作《本草纲目》,其名已愚,仅取《大观》 
本,割裂旧文,妄加增驳,迷误后学。予与家凤卿集成是书,庶以辅冀完经,启蒙方伎,略 
以所知,加之考证。《本经》云∶上药,本上经;中药,本中经;下药,本下经,是古以玉 
石草木等上、中、下品分卷。而序录别为一卷。陶序朱书云∶《本草经》卷上注云∶序药性 
之源本,论病名之形论。卷中云∶玉石、草木三品。卷下云∶虫、兽、果、菜、米,合三品, 
此名医所改,今依古为次。又《帝王世纪》及陶序称四卷者,掌禹锡云∶按旧本亦作四卷。 
韩保升又云∶《神农本草》上、中、下并序录,合四卷。若此,则三、四之异,以有序录。 
则《抱朴子》《养生要略》,《太平御览》所引《神农经》,或云问于太乙子,或引太乙子云云, 
皆《经》所无,或亦在序录中,后人节去之耳。至其经文或以痒为“养”、“创”为“疮”、“淡” 
为“痰”、“注”为“蛀”、“沙”为“砂”、“兔”为“菟”之类,皆由传写之误,据古订正, 
勿嫌惊俗 
也。其辨析物类,引据诸书,本之《毛诗》、《尔雅》、《说文》、《方言》、《广雅》诸子杂家, 
则凤卿增补之力俱多云。 
阳湖孙星衍撰

着本草者,代有明哲矣,而求道者必推本于神农,以为神圣之至诚尽性,其兴物以全民, 
义至精而用至大也。历三代之世以迄秦汉,守其书而传习之,盖无敢违其教者。自陶贞白 
杂入《名医别录》,朱墨分书,其书无传本矣。至宋以降,朱墨互淆,其书无真本矣。纷纭 
散乱,千有余岁,好古者乃欲一一收拾以复其旧,亦难矣哉。故灵胎徐氏有《本草百种录》, 
修园陈氏有《本草经读》,各于经旨有所发明。不愧述者,要止体厥功能,以便世用。而于 
三品之全物,卒阙焉而无闻,久之乃得顾氏辑本,复于同郡石埭徐氏借得孙氏辑本二书,皆 
以用,一无所发。盖孙氏本非知医者,此无足怪。乃于名物形状,亦徒罗列富有,莫正是非 
。如水萍则 并列。柳华则柽杞同称。如此之类,未可殚举。然而备录前文。以待来哲之 
论定胜九谷长,其可实谷而苗草耶。二种出入,嫌入于妄作矣。尤异者,孙顾二书,同出大 
观,而三品互殊,几于十二。顾氏诋孙不考《本经》目录,故三品种数,显与名例相违。夫 
《本经》目录,载在《李氏纲目》第二卷,昭昭者也。孙氏之辑此书,不可谓不勤者矣,独 
于此忽焉而不一寓目耶?岂谓《本经》久无真本,安所得其目录?李氏所述不足据耶。然而 
名 
例相违又何也。夫数典者经生之空谈,而无与于医之实用者也。天下无无用之物,而患无用 
物之人。物无不乐效用于人,而人每至于负物。是书也,苟不求所以用之,即名物品数,尽 
如神农之旧,而何所济于世古圣垂教之深心,历代贤士表章之盛意,其在是耶。用药一用兵 
也 
抑独何哉?学海虑古籍之湮也,亟为刊布而叙其梗概如此,以见舍顾而从孙者,亦取征引之 
富赡耳。至于名象之是非,功用之变化,在善读者之自得之矣。 

时光绪辛卯秋仲建德周学海 之记
	
\mainmatter

\chapter{序录}

\begin{yuanwen}
上\footnote{张本《证类本草》其上有“本草经卷上,本草经卷中,本草经卷下”十五字。}药一百二十种为君,主养命以应天,无毒,多服、久服不伤人,欲轻身益气,不老延年者,本上经。
\end{yuanwen}

上等的药物有一百二十种当作君药,主要能调养性命来与天相应和,没有毒,服量大,长期服不会损伤人。要想使身体轻便灵巧,增添气力,身不衰老,寿命延长的人,要依据《本经》的卷上。

\begin{yuanwen}
中药一百二十种为臣,主养性以应人,无毒有毒,斟酌其宜。欲遏\footnote{\`e}病补虚羸\footnote{l\'ei}者,本中经。
\end{yuanwen}

中等的药有一百二十种当作臣药,主要能调养性情来与人相应和,有的没毒性,有的有毒性,使用时要考虑它们相适宜的情况。想断绝(消除)疾病,以修补虚损消瘦疲劳的人,应依据《本经》的卷中。

\begin{yuanwen}	
下药一百二十五种为佐使,主治病以应地,多毒,不可久服,欲除寒热邪气,破积聚,愈疾者,本下经。
\end{yuanwen}

下等的药有一百二十五种当作佐使药,主要是治疗疾病,来与地相应和,多数有毒,不能长期服用。想祛除寒、热邪气,攻克积聚,使疾病痊愈的药,应依据《本经》的卷下。

\begin{yuanwen}
三品合三百六十五种,法三百六十五度\footnote{行度,度次(轫迹)。《书·尧典》:“朞三百有六旬有六日,以闰四月定四时成岁”。孔颖达疏:“周天三百六十五度四分度之,而日月行一度。”《后汉书·显宗孝明帝纪》:“正仪度。”李贤注:“度,谓日有星辰之行度也。”}。一度应一日,以成一岁,倍其数合七百三十名也。
\end{yuanwen}

三品加起来有三百六十五种,标准的三百六十五天的日月星辰之行度,用一行度来应和一天,就构成一年,它的倍数加起来为七百三十种。

【备注】此条文掌禹锡认为不是《本经》条文,因《本经》共三百六十五味,何来七百三十种?显属朱墨错乱,使他文误入正文。故“倍其数合七百三十名也”十字当为《别录》文。

\begin{yuanwen}	
药\footnote{《千金》其上有“凡”字。}有药物一百二种作君药,有君臣佐使,以相\footnote{选择。《周礼·考工记·矢人》:“几相符。”郑玄注:“相,犹择也。”《三国演义》:“良禽相木而棲,贤臣择主而事。”}宣\footnote{即皇帝命令的文告。}摄\footnote{摄政,辅佐。《字汇》:“摄,佐也。”王引之述闻:“摄,即佐也。”}合\footnote{即配合。}和\footnote{即协调。}\footnote{《千金》其下有“者”字。},宜一君、二臣、三佐、五使\footnote{三佐、五使:森本作“五佐”。},又可一君、三臣、九佐使也。
\end{yuanwen}

药物有君臣佐使,就选择象征着下诏书的皇帝作君药,辅佐皇帝的作臣药,配合君臣的作佐药,能协调的作使药。应该是一味君药,二味臣药,三味佐药,五味使药,还可以用一味君药,三味臣药,九味佐使药。

\begin{yuanwen}	
药有阴阳配合,子母兄弟,根茎\footnote{敦煌《集注》作“叶”。}花\footnote{孙本、森本、敦煌《集注》并作“华”。}实、草石\footnote{《纲目》作“苗皮”。据上下文义,“草石”当作“苗皮”。}骨肉。
\end{yuanwen}

药物有阴阳属性匹配结合的关系,它们有着子母兄弟样的关系,如根与茎、花与实、草与石、骨与肉。

\begin{yuanwen}
有单行\footnote{行,用。《论语·卫灵公》:“行夏之时。”皇侃义疏:“谓用夏家时节以行事也。”单行,即单用一味药物。}者,有相须\footnote{相,互相。须,用。《正字通·页部》:“须,用也。”《论衡·自纪》:“事有所不须。”相须,即相互为用。}者,有相使\footnote{使,命令,派遣;支使;支配;役使。《说文·人部》:“使,伶也”。桂馥义证:“伶也者,通作令。”《左传·桓公五年》:“郑伯使祭足劳王。”《韩非子·喻老》:“重则能使轻,静则能使躁。”王先慎集解引王先谦曰:“重可御轻,静可镇躁,使之谓也。”《论语·学而》:“使民以时。”相使,相互支配,或相互支使。}者,有相畏\footnote{畏,畏惧,死。《吕氏春秋·劝学》:“曾点使曾参,过期而不至,人皆见曾总曰:‘无乃畏邪。’。”高诱注:“畏,犹死也。”《礼记·檀弓上》:“死而不吊者三:畏、厌、溺。”孙希旦集解:“畏,谓被迫胁而恐惧自裁者。”王夫之注:“畏,兵死。”此取后义。相畏,若据“有毒宜制,可用相畏”,其义为相互消除其毒性。或去掉单一我的毒性。}者,有相恶\footnote{text}者,有相反\footnote{text}者,有相杀\footnote{text}者。凡此七情,合和视之,当用相须相使者良,勿用相恶相反者。若有毒宜制,可用相畏相杀者,不尔勿合用也。
\end{yuanwen}

\begin{yuanwen}	
药有酸、咸、甘、苦、辛五味,又有寒、热、温、凉四气,及有毒无毒,阴干暴\footnote{p\`u}干,采造时月,生熟吐地所出,真伪陈新,并各有法。
\end{yuanwen}



\begin{yuanwen}	
药性有宜丸者,宜散者,宜水煮者,宜酒浸者,宜膏渍者,亦有一物兼宜者,亦有不可入汤酒者,并随药性,不得违越。
\end{yuanwen}



\begin{yuanwen}	
凡欲治病,先察其源,先候病机。五脏未虚,六腑未竭,血脉未乱,精神未散,服药必活。若病已成,可得半愈。病势已过,命将难全。
\end{yuanwen}



\begin{yuanwen}	
若用毒药疗病,先起如黍粟,病去即止。不去,倍之;不去,十之。取去为度。
\end{yuanwen}



\begin{yuanwen}	
治寒以热药,治热以寒药。饮食不消,以吐下药。鬼注蛊\footnote{g\v{u}}毒,以毒药。痈肿疮瘤,以疮药。风湿,以风湿药。各随其所宜。
\end{yuanwen}



\begin{yuanwen}	
病在胸膈以上者,先食后服药。病在心腹以下者,先服药而后食。病在四肢血脉者,宜空腹而在旦。病在骨髓者,宜饱满而在夜。
\end{yuanwen}



\begin{yuanwen}	
夫大病之主,有中风、伤寒、寒热、温疟、中恶、霍乱、大腹、水肿、肠澼\footnote{p\`i}下利、大小便不通、奔豚、上气、咳逆、呕吐、黄疸\footnote{d\v{a}n}、消渴、留饮、癖\footnote{p\v{i}}食、坚积癥瘕\footnote{zh\=eng ji\v{a}}、惊邪、癫痫、鬼注、喉痹\footnote{b\`i}、齿痛、耳聋、目盲、金创、踒\footnote{w\=o}折、痈肿、恶疮、痔瘘、瘿瘤、男子五劳七伤、虚乏羸瘦,女子带下崩中、血闭阴蚀,虫蛇蛊毒所伤,此大略宗兆。其间变动枝叶,各宜依端绪以取之。
\end{yuanwen}

\chapter{上品}

\section{丹沙}

\begin{yuanwen}
味甘,微寒。主身体五脏百病,养精神,安魂魄,益气,明目,杀精魁邪恶鬼。久服通神明,不老。能化为汞,生山谷。
\end{yuanwen}

(《太平御览》引∶多有生山谷三字,《大观》本作生符 
陵山谷。俱作黑字。考生山谷是经文,后人加郡县耳。宜改为白字,而以郡县为黑字。下皆 
仿此)。 
《吴普本草》曰∶丹沙,神农∶甘;黄帝∶苦,有毒;扁鹊∶苦;李氏∶大寒,或生武 
陵,采无时,能化未成水银,畏磁石,恶咸水(《太平御览》)。 
《名医》曰∶作末,名真朱。光色如云母,可折者良。生符陵山谷。采无时。 
案∶《说文》云∶丹,巴越之赤石也。象采丹井,象丹形,古文作日,亦作彤、沙、水 
散石也。 ,丹沙所化为水银也。《管子·地数篇》云∶山上有丹沙者,其下有 金。《淮 
南子·地形训》云∶赤矢,七百岁,生赤丹;赤丹,七百岁,生赤 。高诱云∶赤丹、丹 
沙 
传》云∶赤斧,能作水 ,炼丹,与硝石服之。按∶金石之药,古人云久服轻身、延年者, 
谓当避谷,绝人道,或服数十年,乃效耳。今人和肉食服之,遂多相反,转以成疾,不可疑 
古书之虚证。 

\section{云母}

\begin{yuanwen}
味甘,平。主身皮死肌,中风寒热,如在车船上。除邪气,安五脏,益子精,明目。久服轻身延年。一名云珠,一名云花,一名云英,一名云液,一名云沙,一名磷石。生山谷。 
\end{yuanwen}

《名医》曰∶生太山、齐卢山及琅邪、北定山石间,二月采(此录《名医》说者,即 
是仲景、元化及普所说,但后人合之,无从别耳,亦以补普书不备也)。 
案∶《列仙传》云∶方回,炼食云母。《抱朴子·仙药篇》云∶云母有五种∶五色并具 
而多青者,名云英,宜以春服之;五色并具而多赤者,名云珠。宜以夏服之;五色并具而多 
白者,名云液,宜以秋服之;五色并具而多黑者,名云母,宜以冬服之;但有青、黄二色者 
,名云沙,宜以季夏服之;晶晶纯白,名磷石,可以四时长服之也。李善《文选注》、引《 
异物志》∶云母,一名云精,人地万岁不朽。《说文》无磷字。《玉篇》云∶磷,薄也,云母 
之别名。 

\section{玉泉}

\begin{yuanwen}
味甘,平。主治五脏百病。柔筋强骨,安魂魄,长肌肉,益气。久服耐寒暑(《御览》引 
耐字多作能,古通),不饥渴,不老神仙。人临死服五斤,死三年,色不变。一名玉朼\footnote{b\v{i}}(《御 
览》引作玉浓。《初学记》引云,玉桃,服之长生不死。《御览》又引云∶玉桃,服之长生不 
死。若不得早服之,临死日服之,其尸毕天地不朽,则杜疑当作桃)。生山谷。
\end{yuanwen}

《吴普》曰∶玉泉,一名玉屑,神农、岐伯、雷公∶甘;李氏∶平。畏冬华,恶青竹 
(《御览》)。白玉杜如白头公(同上。《事类赋》引云∶白玉体如白首翁)。 
案∶《周礼》∶玉府、王斋,则供食玉。郑云∶玉是阳精之纯者,食之以御水气。郑 
司农云∶王斋,当食玉屑。《抱朴子·仙药篇》云∶玉,可以乌米酒及地榆酒化之为水,亦 
可以葱浆消之为 ,亦可饵以为丸,亦可烧以为粉,服之,一年以上,入水不沾,入火 
,刃之不伤,百毒不犯也。不可用已成之器,伤人无益,当得璞玉,乃可用也。得于阗国白 
玉,尤善。其次,有南阳徐善亭部界界中玉,及日南卢容水中玉,亦佳。 

\section{石钟乳}

\begin{yuanwen}
味甘,温。主咳逆上气,明目益精,安五脏,通百节,利九窍,下乳汁(《御览》引云∶一名留公乳。《大观本》作一名公乳。黑字)。生山谷。
\end{yuanwen}

《吴普》曰∶钟乳,一名虚中。神农∶辛;桐君、黄帝、医和∶甘;扁鹊∶甘,无毒( 
《御览》引云∶李氏,大寒)。生山谷(《御览》引云∶太山山谷),阴处岸下,溜汁成(《御 
览》引作溜汁所成聚),如乳汁,黄白色,空中相通,二月、三月采,阴干(凡《吴普本草》 
,掌禹锡所引者,不复注,惟注其出《御览》诸书者)。 
《名医》曰∶一名公乳,一名芦石,一名夏石。生少室及太山,采无时。 
案∶《范子计然》云∶石钟乳,出武都,黄白者,善(凡引《计然》,多出《艺文类聚 
》、《文选注》、《御览》及《大观本草》)。《列仙传》云∶ 疏,煮石髓而服之,谓之石钟乳, 
钟,当为潼。说云∶乳汁也;钟,假音字。 

\section{涅石}

\begin{yuanwen}
味酸,寒。主寒热泄利,白沃阴蚀,恶疮目痛,坚筋骨齿。炼饵服之,轻身不老,增年。一名羽涅\footnote{ni\`e}。生山谷。
\end{yuanwen}

内容:(旧作矾石,据郭璞注,《山海经》引作涅石) 
 
《吴普》曰∶矾石,一名羽涅,一名羽泽,神农、岐伯∶酸;扁鹊∶咸;雷公∶酸,无 
毒,生河西,或陇西,或武都、石门,采无时;岐伯∶久服伤人骨(《御览》)。 
《名医》曰∶一名羽泽,生河西,及陇西、武都、石门,采无时。 
案∶《说文》无矾字,《玉篇》云∶矾,石也;涅,矾石也。《西山经》云∶女床之山, 
其阴多涅石。郭璞云∶即矾石也,楚人名为涅石,秦名为羽涅也,《本草经》亦名曰涅石也, 
《范子计然》云∶矾石出武都。《淮南子· 真训》云∶以涅染缁。高诱云∶ 
旧涅石作矾石,羽涅作羽涅, 
非。 

\section{硝石}

\begin{yuanwen}
味苦,寒。主五脏积热,胃胀闭。涤去蓄结饮食,推陈致新,除邪气。炼之如膏,久服轻身(《御览》引云∶一名芒硝。《大观本》作黑字)。生山谷。
\end{yuanwen}

《吴普》曰∶硝石,神农∶苦;扁鹊∶甘(丹出掌禹锡所引,亦见《御览》者,不箸所 
出)。 
《名医》曰∶一名芒硝,生益州,及五都、陇西、西羌,采无时。 
案∶《范子计然》云∶硝石,出陇道,据《名医》,一名芒硝,又别出芒硝条,非。《北 
山经》云∶京山,其阴处有元 ,疑 即硝异文。 


\section{朴硝}
p\`o
\begin{yuanwen}
味苦,寒。主百病,除寒热邪气,逐六腑积聚,结固留癖。能化七十二种石。炼饵服之,轻身神仙。生山谷。
\end{yuanwen}
 
《吴普》曰∶朴硝石,神农、岐伯、雷公∶无毒,生益州,或山阴。入土,千岁不变。 
炼之不成,不可服(《御览》)。 
《名医》曰∶一名硝石朴,生益州,有盐水之阳,采无时。 
案∶《说文》云∶朴,木皮也,此盖硝石外裹如玉璞耳。旧作硝,俗字。 


\section{滑石}
\begin{yuanwen}
味甘,寒。主身热泄澼,女子乳难,癃闭,利小便,荡胃中积聚寒热,益精气。久服轻身、耐饥、长年。生山谷。
\end{yuanwen}

《名医》曰∶一名液石,一名共石,一名脱石,一名番石,生赭阳,及太山之阴,或掖 
北,白山山,或卷山。采无时。 
案∶《范子计然》云∶滑石,白滑者,善。《南越志》云∶ 城县出 石,即滑石也。 

\section{石胆}

味酸,寒。主明目,目痛;金创,诸痫痉;女子阴蚀痛,石淋寒热,崩中下血, 
诸邪毒瓦斯,令人有子。炼饵服之,不老,久服,增寿神仙。能化铁为铜,成金银(《御览 
》引作合成)。一名毕石,生山谷。 
《吴普》曰∶石胆,神农∶酸,小寒;李氏∶小寒;桐君∶辛,有毒;扁鹊∶苦,无毒 
(《御览》引云∶一名黑石,一名铜勒,生羌道或名青山,二月瘐子、辛丑采)。 
《名医》曰∶一名黑石,一名棋石,一名铜勒,生羌道、羌里、句青山。二月瘐子、辛 
丑日采。 
案∶《范子计然》云∶石胆,出陇西羌道。陶弘景云∶《仙经》一名立制石,《周礼》疡 
医∶凡疗疡,以五毒攻之;郑云∶今医方有五毒之药,作之合黄 ,置石胆、丹沙、雄黄 
、矾石、磁石其中,烧之三日三夜,其烟上着,以鸡羽扫取之以注创,恶肉破骨则尽出,《 
图经》曰∶故翰林学士杨亿尝笔记直史馆杨 ,有疡生于颊,人语之,依郑法合烧,药成。 
注之疮中,遂愈。信古方攻病之速也。 


\section{空青}
\begin{yuanwen}
	
\end{yuanwen}
内容:味甘,寒。主眚盲耳聋。明目,利九窍,通血脉,养精神。久服,轻身、延年、不老。 
能化铜、铁、铅、锡作金。生山谷。 
《吴普》曰∶空青,神农、甘。一经∶酸。久服,有神仙玉女来时,使人志高(《御览 
》)。 
《名医》曰∶生益州及越 山有铜处,铜精熏则生空青,其腹中空,三月中旬采,亦无 
时。 
案∶《西山经》云∶皇人之山,其下多青;郭璞云∶空青,曾青之属。《范子计然》云 
∶空青,出巴郡。《司马相如赋》云∶丹青。张揖云∶青,青 也。颜师古云∶青 ,今 
之丹青也。 


\section{曾青}
\begin{yuanwen}
	
\end{yuanwen}
内容:味酸,小寒。主目痛,止泪,出风痹,利关节,通九窍,破症坚积聚。久服轻身 
能化金、铜,生山谷。 
《名医》曰∶生蜀中及越 。采无时。 
案∶《管子·揆度篇》云∶秦明山之曾青;《荀子》云∶南海,则有曾青。杨 注∶曾青, 
铜之精。《范子计然》云∶曾出宏农豫章,白青,出新涂。青色者,善。《淮南子·地形 
训》云∶青天八百岁,生青曾。高诱云∶曾青,青石也。 


\section{禹余粮}
\begin{yuanwen}
	
\end{yuanwen}
内容:味甘,寒。主咳逆,寒热烦满,下(《御览》有痢字)赤白,血闭症瘕,大热。炼饵服 
之, 
不饥、轻身、延年。生池泽及山岛中。 
《名医》曰∶一名白余粮,生东海及池泽中。 
案∶《范子计然》云∶禹余粮出河东;《列仙传》云∶赤斧,上华山取禹余粮;《博物志》 
云∶世传昔禹治水,弃其所余食于江中,而为药也。按∶此出《神农经》,则禹非夏禹之禹, 
或本名白余粮,《名医》等移其名耳。 


\section{太乙余食}
\begin{yuanwen}
	
\end{yuanwen}
内容:味甘,平。主咳逆上气,症瘕、血闭、漏下,余邪气。久服,耐寒暑、不饥,轻身、飞 
行千里、神仙(《御览》引作若神仙)。一名石脑,生山谷。 
《吴普》曰∶太一禹余粮,一名禹哀,神农、岐伯、雷公∶甘,平;李氏∶小寒;扁鹊 
∶甘,无毒。生太山上,有甲;甲中有白,白中有黄,如鸡子黄色,九月采,或无时。 
《名医》曰∶生太白。九月采。 
案∶《抱朴子·金丹篇》云∶《灵丹经》用丹沙、雄黄、雌黄、石硫黄、曾青、矾石、磁 
石、戎盐、太一禹余粮,亦用六一泥及神室祭醮合之,三十六日成。 


\section{白石英}
\begin{yuanwen}
	
\end{yuanwen}
内容:味甘,微温。主消渴,阴痿不足,咳逆(《御览》引作呕),胸膈间久寒,益气,除风湿 
痹(《御览》引作阴淫痹)。久服,轻身(《御览》引作身轻健)、长年。生山谷。 
《吴普》曰∶白石英,神农∶甘,岐伯、黄帝、雷公、扁鹊∶无毒。生太山。形如紫石 
英,白泽,长者二、三寸,采无时(《御览》引云∶久服,通日月光)。 
《名医》曰∶生华阴及太山。 
案∶《司马相如赋》有白附。苏林云∶白附,白石英也,司马山云∶出鲁阳山。 


\section{紫石英}
\begin{yuanwen}
	
\end{yuanwen}
内容:味甘,温。主心腹咳逆(《御览》引作呕逆),邪气,补不足,女子风寒在子宫,绝孕十 
年无子。久服,温中、轻身、延年。生山谷。 
《吴普》曰∶紫石英,神农、扁鹊味甘,平;李氏大寒;雷公大温;岐伯∶甘,无毒, 
生太山或会稽,采无时,欲令如削,紫色达头如樗蒲者。 
又曰∶青石英,形如白石英,青端赤后者,是;赤石英,形如白石英,赤端白后者是, 
赤泽有光,味苦,补心气;黄石英,形如白石英,黄色如金,赤端者,是;黑石英,形如白 
石英,黑泽有光(《御览》掌禹锡引此节文)。 
《名医》曰∶生太山,采无时。 
青石、赤石、黄石、白石、黑石脂等 味甘,平。主黄胆,泄利,肠癖脓血,阴蚀,下 
血,赤白,邪气,痈肿,疽痔,恶创,头疡,疥瘙。久服,补髓益气,肥健,不饥,轻身、 
延年。五石脂,各随五色补五脏,生山谷中。 
《吴普》曰∶五色石脂,一名青、赤、黄、白、黑等。青符,神农∶甘;雷公∶酸,无 
毒;桐君∶辛,无毒;李氏∶小寒,生南山,或海涯,采无时。赤符,神农、雷公∶甘;黄 
帝、扁鹊∶无毒;李氏∶小寒,或生少室,或生太山,色绛,滑如脂。黄符,李氏∶小寒; 
雷公苦,或生嵩山,色如豚脑、雁雏,采无时。白符,一名随髓,岐伯、雷公∶酸,无毒 
;李氏∶小寒;桐君∶甘,无毒;扁鹊∶辛,或生少室天娄山,或太山。黑符,一名石泥, 
桐君∶甘,无毒,生洛西山空地。 
《名医》曰∶生南山之阳,一本作南阳,又云∶黑石脂,一名石涅,一名石墨。案∶《吴 
普》引神农甘云云,五石脂各有条,后世合为一条也;《范子计然》云∶赤石脂,出河东, 
色赤者,善。《列仙传》云∶赤须子,好食石脂。 


\section{白青}
内容:味甘,平。主明目,利九窍,耳聋,心下邪气,令人吐,杀诸毒、三虫。久服,通神明, 
轻身、延年、不老。生山谷。 
《吴普》曰∶神农∶甘,平;雷公∶酸,无毒。生豫章,可消而为铜(《御览》)。 
《名医》曰∶生豫章,采无时。 
案∶《范子计然》云∶白青,出巴郡。 


\section{扁青}
内容:味甘,平。主目痛,明目,折跌,痈肿,金创不疗,破积聚,解毒瓦斯(《御览》引作辟 
毒),利精神。久服,轻身、不老。生山谷。 
《吴普》曰∶扁青,神农、雷公∶小寒,无毒,生蜀郡,治丈夫内绝,令人有子(《御 
览》引云∶治痈脾风痹。久服,轻身)。 
《名医》曰∶生朱崖、武都、朱提,采无时。 
案∶《范子计然》云∶扁青,出宏农、豫章。 
上,玉、石,上品一十八种,旧同。 


\section{菖蒲}
内容:味辛,温。主风寒湿痹,咳逆上气,开心孔,补五脏,通九窍,明耳目,出声音。久服, 
轻身、不忘、不迷或,延年。一名昌阳(《御览》引云∶生石上,一寸九节者,久服轻身云 
云。《大观本》,无生石上三字,有云一寸九节者良,作黑字),生池泽。 
《吴普》曰∶菖蒲,一名尧韭(《艺文类聚》引云∶一名昌阳)。 
《名医》曰∶生上洛及蜀郡严道,五月十二日采根,阴干。 
案∶《说文》云∶ ,菖蒲也,益州生。《广雅》云∶邛,昌阳,菖蒲也。《周礼》云∶ 
菖本。郑云∶菖本,菖蒲根,切之四寸为菹。《春秋左传》云∶食以菖 。杜预云∶菖 , 
菖蒲菹。《吕氏春秋》云∶冬至后五旬七日,菖始生。菖者,百草之先,于是始耕。《淮南子·说 
山训》云∶菖羊,去蚤虱而来蛉穷;高诱云∶菖羊,菖蒲;《列仙传》云∶商邱子胥食菖蒲 
根,务光服蒲韭根,《离骚·草木疏》云,沈存中云∶所谓兰荪 


\section{鞠华}
内容:味苦,平。主风,头眩肿痛,目欲脱,泪出,皮肤死肌,恶风湿痹。久服,利血气,轻 
身、耐老、延年。一名节华,生川泽及田野。 
《吴普》曰∶菊华,一名白华(《初学记》),一名女华,一名女茎。 
《名医》曰∶一名日精,一名女节,一名女华,一名女茎,一名更生,一名周盈,一 
名傅延年,一名阴成,生雍州。正月,采根,三月,采叶;五月,采茎;九月,采花;十一 
月,采实。皆阴干。 
郭璞云∶今之秋华,菊。则 、 、 ,皆秋华,惟今作菊。《说文》以为大菊瞿麦, 
假音 
用之也。 


\section{人参}
内容:味甘,微寒。主补五脏,安精神,定魂魄,止惊悸,除邪气,明目、开心、益智。久服, 
轻身、延年。一名人衔,一名鬼盖。生山谷。 
《吴普》曰∶人参,一名土精,一名神草,一名黄参,一名血参,一名人微,一名玉精, 
神农∶甘,小寒;桐君、雷公∶苦;岐伯、黄帝∶甘,无毒;扁鹊∶有毒。生邯郸。三月生 
叶,小兑,核黑,茎有毛,三月、九月采根,根有头、足、手,面目如人(《御 
《名医》曰∶一名神草,一名人微,一名土精,一名血参,如人形者,有神。生上党及 
辽东。二月、四月、八月上旬,采根。竹刀刮,曝干,无令见风。 
案∶《说文》云∶参,人参,药草,出上党。《广雅》云∶地精,人参也;《范子计然》 
云∶人参,出上党,状类人者,善。刘敬叔《异苑》云∶人参,一名土精,生上党者, 
佳。人形皆具,能作儿啼。 


\section{天门冬}
内容:味苦,平。主诸暴风湿偏痹,强骨髓,杀三虫,去伏尸。久服,轻身、益气、延年。一 
名颠勒(《尔雅》注引云∶门冬,一名满冬,今无文)。生山谷。 
《名医》曰∶生奉高山,二月、七月、八月采根,曝干。 
案∶《说文》云∶墙,墙蘼,满冬也;《中山经》云∶条谷之山,其草多宜冬。《尔雅 
》云∶墙蘼,满冬。《列仙传》云∶赤须子食天门冬;《抱朴子·仙药篇》云∶天门冬,或 
名地门冬,或名筵门冬,或名颠棘,或名淫羊食,或名管松。 


\section{甘草}
内容:味甘,平。主五脏六腑寒热邪气,坚筋骨,长肌肉,倍力,金创 ,解毒。久服,轻身、 
延年(《御览》引云∶一名美草,一名密甘,《大观本》作黑字)。生川谷。 
《名医》曰∶一名密甘,一名美草,一名蜜草,一名 草。生河西积沙山及上郡。二月、 
八月除日,采根,曝干,十日成。 
案∶《说文》云∶ ,甘草也; ,大苦也;苦,大甘苓也;《广雅》云∶美草,甘草 
也。《毛诗》云∶隰有苓。《传》云∶苓,大苦。《尔雅》云∶ ,大苦。郭璞云∶今甘草 
,蔓延生;叶似荷,青黄;茎赤黄,有节,节有枝相当。或云 似地黄,此作甘,省字。 
苓,通。 


\section{干地黄}
内容:味甘,寒。主折跌绝筋,伤中,逐血痹,填骨髓,长肌肉,作汤,除寒热积聚,除痹, 
生者尤良。久服,轻身、不老。一名地髓。生川泽。 
《名医》曰∶一名 ,一名芑,生咸阳、黄土地者,佳,二月八日采根,阴干。 
案∶《说文》云∶ ,地黄也。《礼》曰∶ 毛牛藿、羊 、豕薇;《广雅》云∶地髓 
,地黄也。《尔雅》云∶ ,地黄。郭璞云∶一名地髓,江东呼 。《列仙传》云∶吕尚服 
地髓。 


\section{术}
内容:味苦,温。主风寒湿痹、死肌、痉、疸。止汗,除热,消食,作煎饵。久服,轻身、延 
年、不饥。一名山蓟(《艺文类聚》引作山筋),生山谷。 
《吴普》曰∶术,一名山连,一名山芥,一名天苏,一名山姜(《艺文类聚》)。 
《名医》曰∶一名山姜,一名山连,生郑山、汉中、南郑,二月、三月、八月、九月采 
根,曝干。 
案∶《说文》云∶术,山蓟也;《广雅》云∶山姜,术也。白术,牡丹也。《中山经》 
云∶首山草多术。郭璞云∶术,山蓟也;《尔雅》云∶术,山蓟;郭璞云∶今术似蓟, 
而生山中。《范子计然》云∶术,出三辅,黄白色者,善。《列仙传》云∶涓子好饵术。《 
抱朴子·仙药篇》云∶术,一名山蓟,一名山精。故《神药经》曰∶必欲长生,长服山精。 
味辛,平。主续绝伤,补不足,益气力,肥健。汁去面 ,久服,明目、轻身、延年。 
一名菟芦,生川泽。 
《吴普》曰∶菟丝,一名玉女,一名松萝,一名鸟萝,一名鸭萝,一名复实,一名赤网 
,生山谷(《御览》)。 
《名医》曰∶一名菟缕,一名唐蒙,一名玉女,一名赤网,一名菟累。生朝鲜田野,蔓 
延草木之上,色黄而细,为赤网,色浅而大,为菟累。九月采实,暴干。 
案《说文》云∶蒙玉女也《广雅》云∶菟邱,菟丝也,女萝,松萝也;《尔雅》云∶唐 
蒙, 
女萝。菟丝。又云∶蒙,玉女;《毛诗》云∶爱采唐矣。《传》云∶唐蒙,菜名。又茑与女萝。 
《传》云∶女萝、菟丝,松萝也。陆玑云∶今菟丝蔓连草上生,黄赤如金,今合药,菟丝子 
是也,非松萝,松萝,自蔓松上,枝正青,与菟丝异。《楚词》云∶被薜荔兮带女萝。王逸 
云∶女萝,菟丝也。《淮南子》云∶千秋之松,下有茯苓,上有菟丝。高诱注云∶茯苓,千 
岁松脂也。菟丝生其上而无根。旧作菟,非。 


\section{牛膝}
内容:味苦,酸(《御览》作辛)。主寒(《御览》作伤寒)湿痿痹,四肢拘挛,膝痛不可屈伸, 
逐血气,伤热火烂,堕胎。久服,轻身、耐老(《御览》作能老)。一名百倍,生川谷。 
《吴普》曰∶牛膝,神农∶甘;一经∶酸;黄帝、扁鹊∶甘;李氏,温。雷公∶酸,无 
毒。生河内或临邛。叶如夏蓝;茎本赤。二月、八月采(《御览》)。 
《名医》曰∶生河内及临朐。二月、八月、十月采根,阴干。 
案∶《广雅》云∶牛茎,牛膝也;陶弘景云∶其茎有节,似膝,故以为名也。膝,当为 
膝。 


\section{充蔚子}
内容:味辛,微温。主明目益精,除水气。久服轻身,茎主瘾疹痒,可作浴汤。一名益母,一 
名益明,一名大札。生池泽。 
《名医》曰∶一名贞蔚,生海滨,五月采。 
案∶《说文》云∶ ,萑也。《广雅》云∶益母,充蔚也。《尔雅》云∶萑, 。郭璞 
云∶今茺蔚也。《毛诗》云∶中谷有 。《传》云∶ , 也。陆玑云∶旧说及魏博士济阴 
周元明,皆云 是也。《韩诗》及三苍说,悉云益母,故曾子见益母而感。刘歆曰∶ , 
臭秽。臭秽,即茺蔚也。旧作茺,非。 


\section{女萎}
内容:味甘,平。主中风暴热,不能动摇,跌筋结肉,诸不足。久服,去面黑 ,好颜色、润 
泽,轻身、不老。生山谷。 
《吴普》曰∶女萎,一名葳蕤,一名玉马,一名地节,一名虫蝉,一名乌萎,一名荧, 
一名玉竹,神农∶苦;一经∶甘;桐君、雷公、扁鹊∶甘,无毒;黄帝∶辛。生太山山谷。 
叶青黄相值,如姜。二月、七月采。治中风暴热。久服,轻身(《御览》)。一名左眄。久服 
,轻身、耐老(同上)。 
《名医》曰∶一名荧,一名地节,一名玉竹,一名马熏,生太山及邱陵,立春后采,阴 
干。 
案∶《尔雅》云∶荧,委萎;郭璞云∶药草也,叶似竹,大者如箭,竿,有节,叶狭而 
长 
录》有萎蕤,而为用正同,疑女萎即葳蕤也,惟名异耳;陈藏器云∶《魏志·樊阿传》∶青 
粘,一名黄芝,一名地节。此即葳蕤。 


\section{防葵}
内容:味辛,寒。主疝瘕肠泄,膀胱热结,溺不下。咳逆,温疟,癫痫,惊邪狂走。久服,坚 
骨髓、益气、轻身。一名梨盖。生川谷。 
《吴普》曰∶房葵,一名梨盖,一名爵离,一名房苑,一名晨草,一名利如,一名方盖 
,神农∶辛,小寒;桐君、扁鹊∶无毒;岐伯、雷公、黄帝∶苦,无毒。茎叶如葵,上黑黄 
。二月生根,根大如桔梗,根中红白。六月,花白,七月、八月,实白,三月三日采根( 
《御览》)。 
《名医》曰∶一名房慈,一名爵离,一名农果,一名利茹,一名方盖,生临淄,及嵩高 
太山少室,三月三日采根,曝干。 
案∶《博物志》云∶防葵,与野狼毒相似。 


\section{柴胡}
内容:味苦,平。主心腹,去肠胃中结气,饮食积聚,寒热邪气,推陈致新。久服,轻身、明 
目、益精。一名地熏。 
《吴普》曰∶茈葫,一名山菜,一名茹草,神农、岐伯、雷公∶苦,无毒,生冤句。二 
月、八月采根(《御览》)。 
《名医》曰∶一名山菜,一名茹草。叶,一名芸蒿,辛香可食,生宏农及冤句。二月、 
八月采根,曝干。 
案∶《博物志》云∶芸蒿,叶似邪蒿,春秋有白 ,长四、五寸,香美可食。长安及河 
内并有之。《夏小正》云∶正月采芸。《月令》云∶仲春,芸始生;《吕氏春秋》云∶菜之美 
者,华阳之芸,皆即此也。《急就篇》有云∶颜师古注云∶即今芸蒿也,然则是此茈胡叶矣。 
茈、柴,前声相转。《名医》别录前胡条,非。陶弘景云∶《本经》上品有茈胡而无此。晚来 
医乃用之。 


\section{麦门冬}
内容:味甘,平。主心腹结气,伤中、伤饱,胃络脉绝,羸瘦短气。久服,轻身、不老、不饥。 
生川谷及堤阪。 
《吴普》曰∶一名马韭,一名舋冬,一名忍冬,一名忍陵,一名不死药,一名仆垒,一 
名随脂(《太平御览》引云∶一名羊韭,秦,一名马韭,一名禹韭,韭;越,一名羊齐,一 
名麦韭,一名禹韭,一名舋韭,一名禹余粮),神农、岐伯∶甘,平;黄帝、桐君、雷公∶ 
甘,无毒;李氏∶甘,小温;扁鹊∶无毒。生山谷肥地。叶如韭,肥泽丛生。采无时,实青 
黄。 
《名医》曰∶秦,名羊韭;齐,名麦韭;楚,名马韭;越,名羊蓍,一名禹葭,一名禹 
余粮,叶如韭,冬夏长生,生函谷肥土、石间久废处。二月、三月、八月、十月采,阴干。 
案∶《说文》云∶ , 冬草。《中山经》云∶青要之山,是多仆累,据《吴普》说,即 
麦门冬也。忍、 ,垒、累,音同,陶弘景云∶实如青珠,根似 麦,故谓麦门冬。 


\section{独活}
内容:味苦,平。主风寒所击,金疮,止痛,贲豚,痫痉,女子疝瘕。久服,轻身、耐 
老。一名羌活,一名羌青,一名扩羌使者。生川谷。 
《吴普》曰∶独活,一名胡王使者,神农、黄帝∶苦,无毒。八月采。此药有风花不 
动,无风独摇(《御览》)。 
《名医》曰∶一名胡王使者,一名独摇者。此草,得风不摇,无风自动。生雍州,或陇 
西南安。二月、八月采根,曝干。 
案∶《列仙传》云∶山图服羌活、独活,则似二名。护羌、胡王,皆羌字缓声,犹专诸 
为专设诸,庚公差为瘐公之斯,非有义也。 


\section{车前子}
内容:味甘,寒,无毒。主气癃,止痛,利水道小便,除湿痹。久服,轻身、耐老。一名当道 
(《御览》有云∶一名牛舌,《大观本》作牛遗,黑字)。生平泽。 
《名医》曰∶一名 ,一名虾蟆衣,一名牛遗,一名胜 ,生真定邱陵阪道中,五月 
五日采,阴干。 
案∶《说文》云∶ ,一曰 。一名马 ,其实如李,令人宜子,《周书》所说,《广 
雅》云∶当道,马 也;《尔雅》云∶ ,马 ;马 ,车前。郭璞云∶今车前草,大叶 
长穗,好生道边,江东呼为虾蟆衣。又 ,牛 。孙炎云∶车前,一名牛边,《毛诗》云∶ 
采采 ,《传》云∶ ,马 ;马 ,车前也。陆玑云∶马 ,一名车前,一名当道。 
喜在牛迹中生,故曰车前当道也,今药中车前子是也,幽州人谓之牛舌草。 


\section{木香}
内容:味辛。主邪气,辟毒疫温鬼,强志。主淋露(《御览》引云∶主气不足。《大观本》作黑 
字)。久服,不梦寤魇寐(《御览》引云∶一名密青。一名∶轻身,致神仙,《大观本》俱作 
黑字)。生山谷。 
《名医》曰∶一名蜜香,生永昌。 


\section{署豫}
内容:(旧作薯蓣,《御览》作署豫,是) 
味甘,温。主伤中,补虚羸,除寒热邪气,补中,益气力,长肌肉。长肌肉。久服,耳 
目聪明,轻身、不饥、延年。一名山芋,生山谷。 
《吴普》曰∶薯蓣,一名诸署(《御览》作署豫,作诸署,《艺文类聚》亦作诸)。齐越 
,名山芋,一名修脆,一名儿草(《御览》引云,秦楚,名玉延,齐越,名山芋;郑赵,名 
山芋,一名玉延)。神农∶甘,小温;桐君、雷公∶甘(御引作苦),无毒。或生临朐钟山。 
始生,赤茎细蔓;五月,华白;七月,实青黄,八月,熟落,根中白,皮黄,类芋(《御览》 
引云∶二月、八月采根。恶甘遂)。 
《名医》曰∶秦楚名玉延,郑越名土诸。生嵩高,二月、八月采根,曝干。 
案∶《广雅》云∶玉延,薯豫,署蓣也。《北山经》云∶景山草多薯豫。郭璞云∶根似羊 
蹄,可食,今江南单呼为薯,语有轻重耳;《范子计然》云∶薯豫,本出三辅,白色者,善; 
《本草衍义》云∶山药,上一字犯宋英庙讳,下一字曰蓣,唐代宗名豫,故改下一字为 


\section{薏苡仁}
内容:味甘,微寒。主筋急,拘挛不可屈神,风湿痹,下气。久服,轻身、益气。其根,下三 
虫。一名解蠡。生平泽及田野。 
《名医》曰∶一名屋 ,一名起实,一名赣。生真定。八月采实;采根,无时。 
案∶《说文》云∶KT ,KT ,一曰KT 英。赣,一曰薏 。《广雅》云∶赣,起实 
,KT 目也。《吴越春秋》∶鲧娶于有莘氏之女,名曰女嬉,年壮未孳,嬉于砥山,得薏苡 
面而吞之,意若为人所感,因而妊孕。《后汉书·马援传》∶援在交趾,常饵薏苡实,用能轻 
身、省欲 


\section{泽泻}
内容:味甘,寒。主风寒湿痹,乳难。消水,养五脏,益气力,肥健。久服,耳目聪明,不饥、 
延、轻身,面生光,能行水上。一名水泻,一名芒芋,一名鹄泻。生池泽。 
《名医》曰∶生汝南,五、六、八月采根,阴干。 
案∶《说文》云∶ ,水写也;《尔雅》云∶ 。郭璞云∶今泽泻,又 ,牛肤。郭璞 
云∶《毛诗传》云水 也,如续断,寸寸有节,拔之可复。《毛诗》云∶言采其 。《传》云∶ 
,水写也。陆玑云∶今泽写也。其叶如车前草大,其味亦相似,徐州广陵人食之。 


\section{远志}
内容:味苦,温。主咳逆伤中,补不足,除邪气,利九窍,益智慧,耳目聪明,不忘,强志倍 
力。久服,轻身、不老。叶名小草,一名棘菀(陆德明《尔雅音义》引作苋),一名 绕(《御 
览》作要绕),一名细草。生川谷。 
《名医》曰∶生太山及冤句。四月采根、叶,阴干。 
案∶《说文》云∶ ,棘 也。《广雅》云∶ 苑,远志也。其上谓之小草。《尔雅》云∶ 
绕, 。郭璞云∶今远志也,似麻黄,赤华,叶锐而黄。 


\section{龙胆}
内容:味苦涩。主骨间寒热,惊痫邪气,续绝伤,定五脏,杀蛊毒。久服,益智、不忘,轻身、 
耐老。一名陵游,生山谷。 
《名医》曰∶生齐朐及冤句。二月、八月、十一月、十二月采根,阴干。 


\section{细辛}
内容:味辛,温。主咳逆,头痛脑动,百节拘挛,风湿。痹痛、死肌。久服,明目、利九窍, 
轻身、长年。一名小辛。生山谷。 
《吴普》曰∶细辛,一名细草(《御览》引云∶一名小辛)。神农、黄帝、雷公、桐君∶ 
辛,小温;岐伯∶无毒;李氏∶小寒。如葵叶,色赤黑,一根一叶相连(《御览》引云∶三 
月、八月采根)。 
《名医》曰∶生华阴。二月、八月采根,阴干。 
案∶《广雅》云∶细条、少辛,细辛也。《中山经》云∶浮戏之山,上多少辛。郭璞云 
∶细辛也。《管子·地员篇》云∶小辛,大蒙。《范子计然》云∶细辛,出华阴,色白者, 
善。 


\section{石斛}
内容:味甘,平。主伤中,除痹,下气,补五脏虚劳、羸瘦,强阴。久服,浓肠胃、轻身、延 
年。一名林兰(《御览》引云∶一名禁生。《观本》作黑字),生山谷。 
《吴普》曰∶石斛,神农∶甘,平;扁鹊∶酸;李氏∶寒(《御览》)。 
《名医》曰∶一名禁生,一名杜兰,一名石 。生六安水傍石上。七月、八月采茎,阴 
干。 
案∶《范子计然》云∶石斛,出六安。 


\section{巴戟天}
内容:味辛,微温。主大风邪气,阴痿不起,强筋骨,安五脏,补中,增志 
《名医》曰∶生巴郡及下邳。二月、八月采根,阴干。 


\section{白英}
内容:味甘,寒。主寒热、八疽、消渴,补中益气。久服,轻身、延年。一名谷菜(元本误作 
黑字),生山谷。 
《名医》曰∶一名白草。生益州。春,采叶;夏,采茎;秋,采花;冬,采根。 
案∶《尔雅》云∶苻,鬼目。郭璞云∶今江东有鬼目草,茎似葛,叶圆而毛,子如耳 
也,赤色丛生。《唐本》注白英云∶此鬼目草也。 


\section{白蒿}
内容:味甘,平。主五脏邪气,风寒湿痹,补中益气,长毛发,令黑,疗心悬、少食、常饥。 
轻身、耳目聪明、不老。生川泽。 
《名医》曰∶生中山,二月采。 
案∶《说文》云∶蘩,白蒿也;艾,冰台也。《广雅》云∶蘩母,蒡 也。《尔雅》云∶ 
艾,冰台。郭璞云∶今艾,白蒿。《夏小正》云∶二月采蘩。《传》云∶蘩,由胡。由胡者, 
繁母也。繁母者,旁勃也。《尔雅》云∶蘩,皤蒿。郭璞云∶白蒿。又蘩,由胡,郭璞云∶ 
未详。《毛诗》云∶于以采蘩。《传》云∶蘩,皤蒿也。又采蘩祁祁。《传》云∶蘩,白蒿也; 
陆玑云∶凡艾,白色者,为皤蒿;《楚词》王逸注云∶艾,白蒿也。按∶皤、白,音义皆相 
近。艾,是药名,《本草经》无者,即白蒿是也。《名医》别出艾条,非。 


\section{赤箭}
内容:味辛,温。主杀鬼精物、蛊毒恶气。久服,益气力,长阴、肥健,轻身、增年。一名离 
母,一名鬼督邮。生川谷。 
《吴普》曰∶换督邮,一名神草,一名阎狗。或生太山,或少室。茎、箭赤,无叶,根 
如芋子。三月、四月、八月采根,日干。治痈肿(《御览》)。 
《名医》曰∶生陈仓雍州,及太山少室,三月、四月、八月采根,曝干。 
案∶《抱朴子》云∶按∶仙方中,有合离草,一名独摇,一名离母,所以谓之合离、离 
母者,此草为物,下根如芋魁,有游子十二枚周环之,去大魁数尺,虽相须,而实不相连, 
但以气相属耳,别说云∶今医家见用天麻,即是此赤箭根。 


\section{奄闾子}
内容:(旧作庵闾,《御览》作奄闾,是) 
味苦,微寒。主五脏瘀血,腹中水气,胪张留热,风寒湿痹,身体诸痛。久服,轻身、 
延年、不老。生川谷。 
《吴普》曰∶奄闾,神农、雷公、桐君、岐伯∶苦,小温,无毒;李氏∶温,或生上 
党,叶青浓两相当,七月,花曰;九月,实黑,七月、九月、十月采,驴马食,仙去(《御 
览》)。 
《名医》曰∶ 食之,神仙。生雍州,变生上党及道边,十月采根,阴干。 
案∶《司马相如赋》有奄闾,张揖云∶奄闾,蒿也。子可治疾。 


\section{析子}
内容:味辛,微温。主明目,目痛泪出,除痹,补五脏,益精光。久服,轻身,不老。一名蔑 
析,一名大蕺,一名马辛。生川泽及道旁。 
《吴普》曰∶析 ,一名析目,一名荣冥,一名马 。雷公、神农、扁鹊∶辛;李氏∶ 
小温,四月采干。二十日,生道旁。得细辛,良。畏干姜、苦参、荠实,神农∶无毒。生野 
田,五月五日采,阴干。治腹胀(《御览》)。 
《名医》曰∶一名大荠,生咸阳。四月、五月采,曝干。 
案∶《说文》云∶ ,析 ,大荠也;《广雅》云∶析 ,马辛也,《尔雅》云∶析 、 
大荠。郭璞云∶荠,叶细,俗呼之曰老荠。旧作菥,非。 


\section{蓍实}
内容:味苦,平。主益气,充肌肤,明目、聪慧、先知。久服,不饥、不老、轻身。生山谷。 
《吴普》曰∶蓍实,味苦、酸,平,无毒,主益气,充肌肤,明目、聪慧、先知,久服 
,不饥、不老、轻身,生少室山谷。八月、九月采实,曝干(《御览》)。 
《名医》曰∶生少室,八月、九月采实,晒干。 
案∶《说文》云∶蓍,蒿属,生千岁,三百茎。《史记·龟策传》云∶蓍,百茎共一根 


\section{赤芝}
内容:味苦,平。主胸中结,益心气,补中,增智慧,不忘。久食,轻身、不老、延年、神仙。 
一名丹芝。黑芝味咸,平。主癃,利水道,益肾气,通九窍,聪察。久食,轻身、不老、延 
年、 


\section{青芝}
内容:味酸,平。主明目,补肝气,安精魂,仁恕,久食,轻身、不老、延年、神仙。一名龙 
芝。 


\section{白芝}
内容:味辛,平。主咳逆上气,益肺气,通利口鼻,强志意,勇悍,安魄。久食,轻身、不老、 
延年、神仙。一名玉芝。 


\section{黄芝}
内容:味甘,平。主心腹五邪,益脾气,安神,忠信和乐。久食,轻身、不老、延年、神仙。 
一名金芝。 


\section{紫芝}
内容:味甘,温。主耳聋,利关节,保神,益精气,坚筋骨,好颜色。久服,轻身、不老、延 
年。一名木芝。生山谷(旧作六种,今并)。 
《吴普》曰∶紫芝,一名木芝。 
《名医》曰∶赤芝,生霍山,黑芝,生恒山,青芝,生太山,白芝,生华山;黄芝,生 
嵩山;紫芝,生高夏地上,色紫,形如桑(《御览》),六芝,皆无毒,六月、八月采。 
案∶《说文》云∶芝,神草也。《尔雅》云∶茵芝。郭璞云∶芝,一岁三华,瑞草;《礼 
记》则云∶芝 。卢植注云∶芝,木芝也。《楚词》云∶采三秀于山间。王逸云∶三秀,谓 
芝草。《后汉书·华佗传》,有漆叶青面散,注引佗传曰∶青面者,一名地节,一名黄芝,主 
理五脏,益精气,本《字书》无面字,相传音女廉反;《列仙传》云∶吕尚服泽芝。《抱朴子·仙 
药篇》云∶赤者,如珊瑚;白者,如截肪,黑者,如泽漆;青者,如翠羽;黄者,如紫金。 
而皆光明洞彻,如坚冰也。 


\section{卷柏}
内容:味辛,温。生山谷。主五脏邪气,女子阴中寒热痛,症瘕、血闭、绝子。久服,轻身、 
和 
《吴普》曰∶卷柏,神农∶辛;桐君、雷公∶甘(《御览》引云∶一名豹足,一名求股, 
一名万岁,一名神枝、时,生山谷)。 
《名医》曰∶一名豹足,一名求股,一名交时,生常山,五月、七月采,阴干。 
案∶《范子计然》云∶卷柏,出三辅。 


\section{蓝实}
内容:味苦,寒。主解诸毒,杀蛊 、注鬼、螫毒。久服,头不白、轻身。生平泽。 
《名医》曰∶其茎叶可以染青,生河内。 
案∶《说文》云∶ ,马蓝也。蓝,染青草也。《尔雅》云∶ ,马蓝;郭璞云∶今 
大叶冬蓝也。《周礼》掌染草,郑注云∶染草,蓝茜,象头之属。《夏小正》∶五月启灌蓝。 
《毛诗》云∶终朝采蓝。《笺》云∶蓝,染草也。 


\section{芎}
内容:味辛,温。主中风入脑,头痛,寒痹,筋挛缓急,金创,妇人血闭无子。生川谷。 
《吴普》曰∶芎 (《御览》引云∶一名香果),神农、黄帝、岐伯、雷公∶辛,无毒, 
扁鹊∶酸,无毒,李氏∶生温,熟寒,或生胡无桃山阴,或太山(《御览》作或斜谷西岭, 
或太山)。叶香细青黑,文赤如 本,冬夏丛生,五月花赤,七月实黑,茎端两叶,三月采 
。根有节,似马衔状。 
《名医》曰∶一名胡 ,一名香果。其叶,名蘼芜。生武功斜谷西岭,三月、四月采根 
,曝干。 
案∶《说文》云∶菅,菅 ,香草也。芎,司马相如说∶或从弓;《春秋左传》云∶有山 
鞠穷 
乎。杜预云∶鞠穷所以御湿。《西山经》云∶号山,其草多芎 。郭璞云∶芎 ,一名江蓠。 
《范子计然》云∶芎 生始无,枯者,善(有脱字)。《司马相如赋》有芎 。司马贞引 
司马彪云∶芎 ,似 本;郭璞云∶今历阳呼为江蓠。 


\section{蘼芜}
内容:味辛,温。主咳逆,定惊气,辟邪恶,除蛊毒鬼注,去三虫,久服通神。一名薇芜。生 
川泽。 
《吴普》曰∶麋芜,一名芎 (《御览》)。 
《名医》曰∶一名茳蓠,芎 苗也,生雍州及冤句,四月、五月采叶,曝干。 
案∶《说文》云∶麋,麋芜也。蓠,茳蓠,麋芜。《尔雅》云∶靳 ,麋芜。郭璞云∶香 
草,叶小如委状。《淮南子》云∶似蛇床,《山海经》云∶臭如麋芜。《司马相如赋》有江蓠、 
麋芜。司马贞引樊光云∶ 本,一名麋芜,根名勒芷。 


\section{黄连}
内容:味苦,寒。主热气目痛、 伤泣出,明目(《御览》引云∶主茎伤。《大观本》无),肠 
,腹痛下利,妇人阴中肿痛。久服,令人不忘。一名王连。生川谷。 
《吴普》曰∶黄连,神农、岐伯、黄帝、雷公∶苦,无毒;李氏∶小寒。或生蜀郡、太 
山之阳(《御览》)。 
《名医》曰∶生巫阳及蜀郡、太山,二月、八月采。 
案∶《广雅》云∶王连,黄连也。《范子计然》云∶黄连,出蜀郡,黄肥坚者,善。 


\section{络石}
内容:味苦,温。主风热、死肌、痈伤,口干、舌焦,痈肿不消,喉舌肿,水浆不干。 
久服,轻身,明目、润泽、好颜色,不老、延年。一名石鲮。生川谷。 
《吴普》曰∶落石,一名鳞石,一名明石,一名县石,一名云华,一名云珠,一名云 
英,一名云丹,神农∶苦,小温;雷公∶苦,无毒;扁鹊、桐君∶甘,无毒;李氏∶小寒。 
云药中君。采无时(《御览》)。 
《名医》曰∶一名石蹉,一名略石,一名明石,一名领石,一名县石,生太山或石山之 
阴,或高山岩石上,或生人间,正月采。 
案∶《西山经》云∶上申之山多硌石,疑即此。郭璞云∶硌,磊硌大石儿,非也;《唐 
本》注云∶俗名耐冬,山南人谓之石血,以其包络石木而生,帮名络石。《别录》谓之石龙 
藤,以石上生者,良。 


\section{蒺藜子}
内容:味苦,温。主恶血,破症结积聚,喉痹,乳难。久服,长肌肉,明目、轻身。一名旁通, 
一名屈人,一名止行,一名豺羽,一名升推(《御览》引云∶一名君水香,《大观本》无文)。 
生平泽,或道旁。 
《名医》曰∶一名即藜,一名茨,生冯翊。七月、八月采实,曝干。 
案∶《说文》云∶荠,蒺黎也。《诗》曰∶墙上有荠,以茨为茅苇,开屋字。《尔雅》 
云∶墙上有茨。《传》云 
∶茨,蒺藜也,旧本作蒺藜,非。 


\section{黄芪}

原文作“耆”。耆,《广韵》:“渠脂切,平脂群,脂部。” 芪,《广韵》:“巨支切,平支群,支部。” 据此,耆为芪的通假字。

\begin{yuanwen}
味甘,微温。主痈疽久败创,排脓止痛,大风痢疾,五痔鼠 ,补虚,小儿百病。一名 
戴糁。生山谷。 
《名医》曰∶一名戴椹,一名独椹,一名芰草,一名蜀脂,一名百本,生蜀郡白水汉中 
,二月、十月采,阴干。
\end{yuanwen}
 


\section{肉松蓉}
内容:味甘,微温。主五劳七伤,补中,除茎中寒热痛,养五脏,强阴,益精气,多子,妇人 
症瘕。久服,轻身。生山谷。 
《吴普》曰∶肉苁蓉,一名肉松蓉,神农、黄帝∶咸;雷公∶酸,小温(《御览》作李 
氏∶小温),生河西(《御览》作东)山阴地,长三、四寸,丛生,或代郡(《御览》下有雁 
门 
二字)。二月到八月采(《御览》引云∶阴干用之)。 
《名医》曰∶生河西及代郡雁门,五月五日采,阴干。 
案∶《吴普》云∶一名肉松蓉,当是古本,蓉,即是容字,俗写苁蓉,非正字也。陶弘 
景云∶是野马精落地所生,生时似肉,旧作肉苁蓉,非。 


\section{防风}
内容:味苦,温,无毒。主大风、头眩痛,恶风风邪,目盲无所见,风行周身,骨节疼痹(《御 
览》作痛),烦满。久服,轻身。一名铜芸(《御览》作芒)。生川泽。 
《吴普》曰∶防风,一名回云,一名回草,一名百枝,一名 根,一名百韭,一名百 
种,神农、黄帝、岐伯、桐君、雷公、扁鹊∶甘,无毒,李氏∶小寒,或生邯郸上蔡,正月 
生叶,细圆,青黑黄白;五月花黄;六月实黑。三月、十月采根,晒干,琅邪者,良(《御 
览》)。 
《名医》曰∶一名茴草,一名百枝,一名屏风,一名 根,一名百蜚。生沙苑,及邯郸 
、琅邪、上蔡。二月、十月采根,曝干。 
案∶《范子计然》云∶防风,出三辅。白者,善。 


\section{蒲黄}
内容:味甘,平。主心腹、膀胱寒热,利小便,止血,消瘀血。久服,轻身、益气力,延年、 
神仙。生池泽。 
《名医》曰∶生河东,四月采。 
案∶《玉篇》云∶ ,谓今蒲头有台,台上有重台,中出黄,即蒲黄。陶弘景云∶此即 
蒲厘花上黄粉也,《仙经》亦用此,考《尔雅》苻离,其上 ,苻离与蒲厘声相近,疑即 


\section{香蒲}
内容:味甘,平。主五脏心下邪气,口中烂臭,坚齿,明目,聪耳。久服,轻身、耐老(《御 
览》作能老)。一名睢(《御览》云睢蒲)。生池泽。 
《吴普》曰∶睢,一名睢石,一名香蒲,神农、雷公∶甘。生南海,池泽中(《御览》)。 
《名医》曰∶一名醮,生南海。 
案∶《说文》云∶菩,草也。《玉篇》云∶菩,香草也。又音蒲。《本草图经》云∶香 
蒲,蒲黄苗也,春初生嫩叶,未出水时,红白色,茸茸然,《周礼》以为菹。 


\section{}续断
内容:味苦,微温。主伤寒,补不足,金创痈伤,折跌,续筋骨,妇人乳难(《御览》作乳痈, 
云崩中、漏血,《大观本》作黑字)。久服,益气力。一名龙豆,一名属折。生山 
《名医》曰∶一名接骨,一名南草,一名槐。生常山。七月、八月采,阴干。 
案∶《广雅》云∶ ,续断也。《范子计然》云∶续断,出三辅∶《桐君药录》云∶续 
断,生蔓延,叶细,茎如荏大,根本黄白,有汁。七月、八月采根。 


\section{漏芦}

原序录中作“卢”,正文作“芦”,今据改。

\begin{yuanwen}
味甘,咸寒。主皮肤热、恶创、疽痔、湿痹,下乳汁。久服,轻身益气,耳目聪 
明,不老、延年。一名野兰。生山谷。 
《名医》曰∶生乔山,八月采根,阴干。 
案∶《广雅》云∶飞廉,漏芦也;陶弘景云∶俗中取根,名鹿骊。 
\end{yuanwen}

\section{营实}
内容:味酸,温。主痈疽恶创,结肉跌筋,败创,热气,阴蚀不疗,利关节。一名墙薇 
,一名墙麻,一名牛棘。生川谷。 
《吴普》曰∶蔷薇,一名牛勒,一名牛膝,一名蔷薇,一名山枣(《御览》)。 
《名医》曰∶一名牛勒,一名蔷蘼,一名山棘,生零陵及蜀郡,八月、九月采,阴干。 
案∶陶弘景云∶即是墙薇子。 


\section{}天名精
内容:味甘,寒。主瘀血、血瘕欲死、下血。止血,利小便。久服轻身、耐老。一名麦句姜, 
一名虾蟆蓝,一名豕首。生川泽。 
《名医》曰∶一名天门精,一名玉门精,一名彘颅,一名蟾蜍兰,一名觐,生平原。五 
月采。 
案∶《说文》云∶ ,豕首也;《尔雅》云∶ , ,豕首。郭璞云∶今江东呼 首, 
可以 蚕蛹。陶弘景云∶此即今人呼为 ,《唐本》云∶鹿活草是也。《别录》一名天 
蔓菁,南文呼为地松;掌禹锡云∶陈藏器别立地菘条,后人不当仍其谬。 


\section{}决明子
内容:味咸,平。主青盲、目淫、肤赤、白膜、眼赤痛、泪出。久服,益精光(《太平御览》 
引作理目珠精,理,即治字),轻身。生川泽。 
《吴普》曰∶决明子,一名草决明,一名羊明(《御览》)。 
《名医》曰∶生龙门,石决明,生豫章,十月采,阴干百日。 
案∶《广雅》云∶羊 ,英光也。又决明,羊明也;《尔雅》云∶ ,英光。 
郭璞云∶英,明也。叶黄锐,赤花,实如山茱萸。陶弘景云∶形似马蹄决明。 


\section{}丹参
内容:味苦,微寒。主心腹邪气,肠鸣幽幽如走水,寒热积聚;破症除瘕,止烦满,益 
气。一名却蝉草。生川谷。 
《吴普》曰∶丹参,一名赤参,一名木羊乳,一名却蝉草。神农、桐君、黄帝、 
雷公、扁鹊∶苦,无毒;李氏∶小寒。岐伯∶咸。生桐柏,或生太山山陵阴。茎花小方如荏 
,毛、根赤,四月花紫,五月采根,阴干,治心腹痛(《御览》)。 
《名医》曰∶一名赤参,一名木羊乳,生桐柏山及太山,五月采根,曝干。 
案∶《广雅》云∶却蝉,丹参也。 


\section{}茜根
内容:味苦,寒。主寒湿,风痹,黄胆,补中。生川谷。 
《名医》曰∶可以染绛。一名地血,一名茹 ,一名茅搜,一名茜。生乔山。二月 
、三月采根,阴干。 
案∶《说文》云∶茜,茅搜也。搜,茅搜,茹 。人血所生,可以染绛,从草从鬼。《 
广雅》云∶地血,茹 ,茜也。《尔雅》云∶茹 ,茅鬼;郭璞云∶今茜也,可以染绛。 
《毛诗》云∶茹 在阪。《传》云∶茹 ,茅搜也。陆玑云∶一名地血,齐人谓之茜,徐州 
人谓之牛蔓。徐广注《史记》云∶茜,一名红蓝,其花染绘,赤黄也。按∶《名医》别出红 
蓝条,非。 


\section{}飞廉
内容:味苦,平。主骨节热,胫重酸疼。久服,令人身轻。一名飞轻(已以上四字,原本黑字)。 
生川泽。 
《名医》曰∶一名伏兔,一名飞雉,一名木禾,生河内,正月采根;七月、八月采花。 
阴干。 
案∶《广雅》云∶伏猪,木禾也。飞廉,漏芦也。陶弘景云∶今既别有漏芦,则非。此 
别名耳。 


\section{}五味子
内容:味酸,温。主益气,咳逆上气,劳伤羸瘦。补不足,强阴,益男子精(《御览》引云∶ 
一名会及。《大观本》作黑字)。生山谷。 
《吴普》曰∶五味子,一名元及(《御览》)。 
《名医》曰∶一名会及,一名元及。生齐山及代郡,八月采实,阴干。 
案∶《说文》云∶ ,猪也。 , 猪草也。 , 也。《广雅》云∶会及,五味也。 
《尔 
雅》云∶ , 。郭璞云∶五味也。蔓生子,丛在茎头。《抱朴子·仙药篇》云∶五味者, 
五行之精,其子有五味。移门子服五味子十六年,色如玉女,入水不沾,入火不灼也。 


\section{}旋华
内容:味甘,温。主益气,去面 (《御览》作 )黑,色媚好(《御览》作令人色悦泽)。其 
根 
味辛。主腹中寒热邪气,利小便。久服,不饥、轻身。一名筋根华,一名金沸(《御览》引 
云∶一名美草,《大观》本作黑字)。生平泽。 
《名医》曰∶生豫州,五月采,阴干。 
案∶陶弘景云∶东人呼为山姜,南人呼为美草;《本草衍义》云∶世又谓之鼓子花。 


\section{兰草}
内容:味辛,平。主利水道,杀蛊毒,辟不祥。久服,益气、轻身、不老、通神明。一名水香。 
生池泽。 
《名医》曰∶生大吴,四月、五月采。 
案∶《说文》云∶兰,香草也;《广雅》云∶ ,兰也;易∶其臭如兰。郑云∶兰,香草 
也。《夏小正》∶五月蓄兰。《毛诗》云∶方秉 分。《传》云∶ ,兰也。陆玑云∶ ,即兰, 
香草也,其茎、叶似药草泽兰。《范子计然》云∶大兰,出汉中三辅;兰,出河东宏农,白 
者善。元杨齐贤注李白诗引《本草》云∶兰草、泽兰,二物同名。兰草,一名水香,云都梁 
是也。《水经》∶零陵郡,都梁县西小山上,有 水,其中悉生兰草,绿叶紫茎;泽兰,如薄 
荷,微香,荆湘岭南人家多种之,与兰大抵相类。颜师古以兰草为泽兰,非也 


\section{}蛇床子
内容:味苦,平。主妇人阴中肿痛,男子阳痿、湿痒,除痹气,利关节,癫痫恶创。久服,轻 
身。一名蛇米。生川谷及田野。 
《吴普》曰∶蛇床,一名蛇珠(《御览》)。 
《名医》曰∶一名蛇粟,一名虺床,一名思盐,一名绳毒,一名枣棘,一名墙蘼,生临 
淄。五月采实,阴干。 
案∶《广雅》云,蛇粟,马床,蛇床也;《尔雅》云∶盱虺床。《淮南子·汜论训》云∶ 
乱人者,若蛇床之与蘼芜。 


\section{}地肤子
内容:味苦,寒。主膀胱热,利小便,补中,益精气。久服,耳目聪明、轻身、耐老。一名地 
葵(《御览》引云∶一名地华,一名地脉。《大观本》无一名地华四字;脉,作麦,皆黑字)。 
生平泽及田野。 
《名医》曰∶一名地麦,生荆州,八月、十月采实,阴干。 
案∶《广雅》云∶地葵,地肤也;《列仙传》云∶文宾服地肤;郑樵云∶地肤,曰落帚 
,亦曰地扫;《尔雅》云∶ ,马帚,即此也;今人亦用为帚。 


\section{}景天
内容:味苦,平。主大热、火创、身热烦,邪恶气。花,主女人漏下赤白,轻身、明目。一名 
戒火,一名慎火(《御览》引云∶一名水母。《大观本》作黑字,水作火)。生川谷。 
案∶陶弘景云∶今人皆盆养之于屋上,云以辟火。 


\section{}茵陈
内容:(《御览》作茵蒿) 
味苦,平。主风湿寒热邪气,热结、黄胆。久服,轻身、益气、耐老(《御览》作能老)。 
生邱陵阪岸上。 
《吴普》曰∶因尘,神农、岐伯、雷公∶苦,无毒;黄帝∶辛,无毒。生田中,叶如蓝, 
十一月采(《御览》)。 
《名医》曰∶白兔食之,仙。生太山。五月及立秋采,阴干。 
案∶《广雅》云∶因尘,马先也;陶弘景云∶《仙经》云∶白蒿,白兔食之仙,而今茵陈 
乃云此,恐非耳。陈藏器云∶茵陈,经冬不死,因旧苗而生,故名茵陈,后加蒿字也。据此, 
知旧作茵陈蒿,非;又按∶《广雅》云∶马先,疑即马新蒿,亦白蒿之类。 


\section{}杜若
内容:味辛,微温。主胸胁下逆气,温中,风入脑户,头肿痛,多涕泪出。久服,益精(《艺 
文类聚》引作益气)、明目、轻身。一名杜衡(《艺文类聚》引作蘅,非)。生川泽。 
《名医》曰∶一名杜连,一名白连,一名白苓,一名若芝,生武陵及冤句,二月、八月 
采根,曝干。 
案∶《说文》云∶若,杜若,香草;《广雅》云∶楚蘅,杜蘅也;《西山经》云∶于帝之 
上有草焉,其状如葵,其臭如蘼芜,名曰杜蘅。《尔雅》云∶杜,土卤。郭璞云∶杜蘅也, 
似葵而香。《楚词》云∶采芳州兮杜若。《范子计然》云∶杜若,生南郡汉中。又云∶秦蘅, 
出于陇西天水。沈括《补笔谈》云∶杜若,即今之高良姜。后人不识,又别出高良姜条,按∶ 
《经》云∶一名杜蘅,是《名医》别出杜蘅条,非也。蘅,正字,俗加草。 


\section{}沙参
内容:味苦,微寒。主血积惊气,除寒热,补中,益肺气。久服利人。一名知母。生川谷。 
《吴普》曰∶白沙参,一名苦心,一名识美,一名虎须,一名白参,一名志取,一名文 
虎。神农、黄帝、扁鹊∶无毒;岐伯∶咸;李氏∶小寒。生河内川谷,或般阳渎山,三月生 
,如葵,叶青,实白如芥,根大白如芜菁,三月采(《御览》)。 
《名医》曰∶一名苦心,一名志取,一名虎须,一名白参,一名识美,一名文希,生河 
内及冤句、般阳续山,二月、八月采根,曝干。 
案∶《广雅》云∶苦心,沙参也,其蒿,青蓑也。《范子计然》云∶白沙参,出洛阳白 
者,善。 


\section{}白兔藿
内容:味苦,平。主蛇虺,蜂虿, 狗,菜、肉、蛊毒,注。一名白葛。生山谷。 
《吴普》曰∶白兔藿,一名白葛谷(《御览》)。 
《名医》曰∶生交州。 
案∶陶弘景云∶都不闻有识之者,都富似葛耳,《唐本》注云∶此草荆襄山谷大有,俗 
谓之白葛。 


\section{}徐长卿
内容:味辛,温。主鬼物、百精、蛊毒,疫疾,邪恶气,温疟。久服,强悍、轻身。 
一名鬼督邮。生山谷。 
《吴普》曰∶徐长卿,一名石下长卿,神农、雷公∶辛。或生陇西。三月采(《御览 
《名医》曰∶生太山及陇西,三月采。 
案∶《广雅》云∶徐长卿,鬼督邮也;陶弘景云∶鬼督邮之名甚多,今俗用徐长卿者, 
其根正如细辛,小短扁扁尔,气亦相似。 


\section{}石龙刍
内容:味苦,微寒。主心腹邪气,小便不利,淋闭,风湿,鬼注,恶毒。久服,补虚 
羸,轻身。耳目聪明,延年。一名龙须,一名草续断,一名龙珠。生山谷。 
《吴普》曰∶龙刍,一名龙多,一名龙须,一名续断,一名龙本,一名草毒,一名龙 
华,一名悬莞,神农、李氏∶小寒;雷公、黄帝∶苦,无毒;扁鹊∶辛,无毒,生梁州。七 
月七日采(《御览》此条,误附续断)。 
《名医》曰∶一名龙华,一名悬莞,一名草毒,生梁州湿地,五月、七月采茎,曝干。 
须也,似莞而细。生山石穴中。茎列垂,可以为席。《别录》云∶一名之宾。郑樵云∶《尔 
雅》所谓 鼠莞也。旧作 ,非。 


\section{}薇衔
内容:味苦,平。主风湿痹、历节痛,惊痫、吐舌、悸气,贼风,鼠 ,痈肿。一名糜衔。生 
川泽。 
《吴普》曰∶薇 ,一名糜痹,一名无颠,一名承膏,一名丑,一名无心(《御览》) 
。《名医》曰∶一名承膏,一名承肌,一名无心,一名无颠。生汉中及冤句、邯郸,七月 
采茎、叶,阴干。 


\section{}云实
内容:味辛,温。主泄利(旧作痢,《御览》作泄利)肠 ,杀虫蛊毒,去邪恶结气,止痛、 
除热,平,主见鬼精物;多食,令人狂走。久服,轻身、通神明。生川谷。 
《吴普》曰∶云实,一名员实,一名天豆,神农∶辛,小温,黄帝∶咸;雷公∶苦。叶 
如麻,两两相值,高四、五尺,大茎空中,六月花,八月、九月实,十月采(《御览》)。 
《名医》曰∶一名员实,一名云英,一名天豆,生河间。十月采,曝干。 
案∶《广雅》云∶天豆,云实也。 


\section{}王不留行
内容:味苦,平。主金创,止血逐痛,出刺,除风痹内寒。久服,轻身、耐老(《御览》作能 
老)、增寿。生山谷。 
《吴普》曰∶王不留行,一名王不流行。神农∶苦,平;岐伯、雷公∶甘。三月、八月 
采(《御览》)。 
案∶郑樵云∶王不留行,曰禁宫花,曰剪金花,叶似花,实作房。 


\section{}升麻
内容:味甘,辛(《大观本》作甘,平)。主解百毒,杀百老物殃鬼,辟温疾、障邪毒蛊。久服, 
不夭(《大观本》作∶主解百毒,杀百精老物殃鬼,辟瘟疫瘴气、邪气虫毒。此用《御览》 
文)。一名周升麻(《大观本》作周麻)。生山谷(旧作黑字。据《吴普》有云∶神农∶甘。 
则《本经》当有此,今增入)。 
《吴普》曰∶升麻,神农∶甘(《御览》)。 
《名医》曰∶生益州,二月、八月采根,晒干。 
案∶《广雅》云∶周麻,升麻也(此据《御览》)。 


\section{}青
内容:味甘,寒。主五脏邪气,风寒湿痹,益气,补脑髓,坚筋骨。久服,耳目聪明、不饥、 
不老、增寿。巨胜苗也。生川谷(旧在米谷部,非)。 
《吴普》曰∶青襄,一名梦神,神农∶苦;雷公∶甘(《御览》)。 
《名医》曰∶生中原。 
案∶《抱朴子·仙药篇》云∶《孝经·援神契》曰∶巨胜、延年,又云∶巨胜,一名胡 
麻,饵服之,不老、耐风湿、补衰老也。 


\section{}姑活
内容:味甘,温。主大风邪气,湿痹寒痛。久服,轻身、益寿、耐老。一名冬葵子(旧在《唐 
本草》中,无毒,今增)。 
《名医》曰∶生河东。 
案∶《水经注》解县引《神农本草》云∶地有固活、女疏、铜芸、紫苑之族也。陶弘景 
云∶方药亦无用此者,乃有固活丸,即是野葛一名。此又名冬葵子,非葵菜之冬葵子,疗体 
乖异。 


\section{}别羁
内容:味苦,微温。主风寒湿痹,身重,四肢疼酸,寒邪历节痛。生川谷(旧在《唐本草》中, 
无毒,今增)。 
《名医》曰∶一名别枝,一名别骑,一名鳖羁。生蓝田。二月、八月采。案∶陶弘景云 
∶方家时有用处,今俗亦绝耳。 


\section{}屈草
内容:味苦。主胸胁下痛,邪气,腹间寒热阴痹。久服,轻身、益气、耐老(《御览》作补益、 
能老)。生川泽(旧在《唐本草》中,无毒,今增)。 
《名医》曰∶生汉中,五月采。 
案∶陶弘景云∶方药不复用,俗无识者。 


\section{}淮木
内容:味苦,平。主久咳上气,肠中虚羸,女子阴蚀、漏下赤白沃。一名百岁城中木。生山谷 
(旧在《唐本草》中,无毒,今增)。 
《吴普》曰∶淮木,神农、雷公∶无毒,生晋平阳河东平泽。治久咳上气,伤中羸虚, 
补中益气(《御览》)。 
《名医》曰∶一名炭木,生太山,采无时。 
案∶李当之云∶是樟树上寄生树,大衔枝在肌肉,今人皆以胡桃皮当之,非也;桐君云 
∶生上洛,是木皮,状如浓朴,色似桂白,其理一纵一横,今市人皆削乃以浓朴,而无正纵 
横理,不知此复是何物,莫测真假,何者为是也。 
上草,上品七十三种,旧七十二种,考门芝当一升,升麻当白字;米谷部误入青襄;《 
唐本草》六种,姑活、屈草、淮木,皆当入此。 


\section{}牡桂
内容:味辛,温,主上气咳逆,结气,喉痹吐吸。利关节,补中益气。久服通神、轻身、不老。 
生山谷。 
《名医》曰∶生南海。 
案∶《说文》云∶桂,江南木,百药之长, 桂也。《南山经》云∶招摇之山多桂;郭 
璞云∶桂,叶似枇杷,长二尺余,广数寸,味辛,白花,丛生山峰,冬夏常青,间无杂木。 
《尔雅》云∶ ,木桂。郭璞云∶今人呼桂皮浓者,为木桂,及单名桂者,是也。一名肉桂 
,一名桂枝,一名桂心。 


\section{}菌桂
内容:味辛,温。主百病,养精神,和颜色,为诸药先聘通使。久服,轻身、不老,面生光华, 
媚好常如童子。生山谷。 
《名医》曰∶生交址桂林岩崖间,无骨,正圆如竹,立秋采。 
案∶《楚词》云∶杂申椒与菌桂兮;王逸云∶ 、桂,皆香木。《列仙传》云∶范蠡好 
服桂。 


\section{}松脂
内容:味苦,温。主疽,恶创,头疡,白秃,疥瘙风气。五脏,除热。久服,轻身、不老、延 
年。一名松膏,一名松肪。生山谷。 
《名医》曰∶生太山。六月采。 
案∶《说文》云∶松木也,或作窠。《范子计然》云∶松脂,出陇西,松胶者,善。 


\section{}槐实
内容:味苦,寒。主五内邪气热,止涎唾,补绝伤,五痔,火创,妇人乳瘕,子藏急痛。生平 
泽。 
《名医》曰∶生河南。 
案∶《说文》云∶槐木也;《尔雅》云∶ ,槐,大叶而黑。郭璞云∶槐树叶大色黑者 
,名为 。又守宫槐叶,昼聂宵炕。郭璞云∶槐叶,昼日聂合,而夜炕布者,名为守宫槐。 
味苦,寒。主五内邪气,热中消渴,周痹。久服,坚筋骨、轻身、不老(《御览》作耐 
老)。一名杞根,一名地骨,一名枸杞,一名地辅。生平泽。 
《吴普》曰∶枸杞,一名枸己,一名羊乳(《御览》)。 
《名医》曰∶一名羊乳,一名却暑,一名仙人杖,一名西王母杖。生常山及诸邱陵阪岸 
。冬采根,春、夏采叶,秋采茎、实,阴干。 
案∶《说文》云∶继,枸杞也。杞,枸杞也。《广雅》云∶地筋,枸杞也。《尔雅》云 
∶杞,枸 。郭璞云∶今枸杞也。《毛诗》云∶集子苞杞。《传》云∶杞,枸 也。陆玑云 
∶苦杞秋熟,正赤,服之,轻身、益气。《列仙传》云∶陆通食橐卢木实;《抱朴子·仙药 
篇》云∶象紫,一名托卢是也,或名仙人杖,或云西王母杖,或名天门精,或名却老,或名 
地骨,或名枸杞也。 


\section{}柏实
内容:味甘,平。主惊悸,安五脏,益气,除湿痹。久服,令人悦泽美色,耳目聪明,不饥、 
不老,轻身、延年。生山谷。 
《名医》曰∶生太山,柏叶尤良,田四时各依方面采,阴干。 
案∶《说文》云∶柏,鞠也;《广雅》云∶栝,柏也;《尔雅》云∶柏, 熟。郭璞云 
《礼记》曰∶鬯日以 。《范子计然》云∶柏脂,出三辅。上,升价七千;中,三千一斗。 
味甘,平。主胸胁逆气(《御览》作疝气),忧恚,惊邪,悸心下结痛,寒热烦满咳逆, 
口焦舌干,利小便。久服,安魂、养神,不饥、延年。一名茯菟(《御览》作茯神。案∶原 
本云∶其有抱根者,名茯神。作黑字)。生山谷。 
《吴普》曰∶茯苓通神,桐君∶甘;雷公、扁鹊∶甘,无毒。或生茂州大松根下,入地 
三丈一尺。二月七日采(《御览》)。 
《名医》曰∶其有抱根者,名茯神,生太山大松下,二月、八月采,阴干。 
案∶《广雅》云∶茯神,茯苓也;《范子计然》云∶茯苓,出嵩高三辅。《列仙传》云 
∶昌容采茯苓,饵而食之;《史记》褚先生云∶《传》曰∶下有伏灵,上有菟丝,所谓伏灵 
者,在菟丝之下,状似飞鸟之形,伏灵者,千岁松根也,食之不死,《淮南子·说林训》云 
∶茯苓掘,菟丝死。旧作茯,非。 


\section{}榆皮
内容:味甘,平。主大小便不通,利水道,除邪气。久服,轻身、不饥。其实尤良。一名零榆。 
生山谷。 
《名医》曰∶生 川,三月采皮,取白,曝干;八月采实。 
案∶《说文》云∶榆,白 榆也。《广雅》云∶柘榆,梗榆也。《尔雅》云∶榆,白 
。郭璞云∶ 榆,先生叶,却着荚,皮色白。又枢 ,郭璞云∶今云刺榆。《毛诗》云∶ 
东门之 ;《传》云∶ ,白榆也。又山有 ,《传》云∶枢, 也。陆玑云∶其针刺如柘 
,其叶如榆,渝为茹,美滑如白榆之类,有十种,叶皆相似,皮及木理异矣。 


\section{}酸枣
内容:味酸,平。主心腹寒热,邪结气聚,四肢酸疼,湿痹。久服,安五脏,轻身、延年。生 
川泽。 
《名医》曰∶生河东,八月采实,阴干,四十日成。 
案∶《说文》云∶ ,酸枣也。《尔雅》云∶ ,酸枣。郭璞云∶味小实酢。孟子云∶养 
其棘。赵岐云∶ 棘,小棘,所谓酸枣是也。 


\section{}木
内容:味苦,寒。主五脏、肠胃中结热,黄胆,肠痔,止泄利,女子漏下赤白,阴阳蚀创,一 
名檀桓。生山谷。 
《名医》曰∶生汉中及永昌。 
案∶《说文》云∶檗,黄木也, 木也。《司马相如赋》有 。张揖云∶檗木,可染者 
,颜师古云∶ ,黄薜也。 


\section{}干漆
内容:味辛,温,无毒。主绝伤,补中,续筋骨,填髓脑,安五脏,五缓六急,风寒湿痹。生 
漆,去长虫。久服,轻身、耐老。生川谷。 
《名医》曰∶生汉中,夏至后采,干之。 
案∶《说文》云∶ ,木汁,可以 物。象形。 如水滴而下,以漆为漆水字。《周礼 
》载师云∶漆林之征。郑元云∶故书漆林为 林。杜子春云∶当为漆林。 


\section{}五加皮
内容:孕辛,温。主心腹疝气,腹痛,益气疗 ,小儿不能行,疸创阴蚀。一名豺漆。 
《名医》曰∶一名豺节,生汉中及冤句。五月、十月采茎,十月采根,阴干。 
案∶《大观本草》引东华真人《煮石经》云∶舜常登苍梧山。曰∶厥金玉之香草,朕用 
偃息正道,此乃五加皮。鲁定公母单服五加酒,以致不死。 


\section{}蔓荆实
内容:味苦,微寒。主筋骨间寒热痹、拘挛,明目坚齿,利九窍,去白虫。久服,轻身、耐老, 
小荆实亦等。生山谷。 
《名医》曰∶生河间、南阳、冤句,或平寿都乡高岸上,及田野中。八月、九月采实, 
阴干。 
案∶《广雅》云∶牡荆,蔓荆也;《广志》云∶楚荆也。牡荆,蔓荆也。据牡、曼,声 
相近,故《本经》于蔓荆,不载所出州土,以其见牡荆也。今或别为二条,非。 


\section{}辛夷
内容:味辛,温。主五脏身体寒风,头脑痛,面 。久服下气、轻身、明目、增年、耐老。一 
名辛矧(《御览》作引),一名侯桃,一名房木。生川谷。 
《名医》曰∶九月采实,曝干。 
案∶《汉书·杨雄赋》云∶列新雉于林薄。师古云∶新雉,即辛夷耳,为树甚大,其木 
枝叶皆芳,一名新矧。《史记·司马相如传》∶杂以流夷。注《汉书音义》曰∶流夷,新 
夷也。陶弘景云∶小时气辛香,即《离骚》所呼辛夷者。陈藏笔器云∶初发如笔,北人呼为 
木 


\section{}桑上寄生
内容:味苦,平。主腰痛,小儿背强,痈肿,安胎,充肌肤,坚发齿,长须眉。其实,明目、 
轻身、通神。一名寄屑,一名寓木,一名宛童。生川谷。 
《名医》曰∶一名茑。生宏农桑树上。三月三日,采茎,阴干。 
案∶《说文》云∶茑,寄生也。《诗》曰∶茑与女萝,或作KT 。《广雅》云∶宛童,寄 
生也。寄生KT 也又寄屏,寄生也。《中山经》云∶龙山上多寓木。郭璞云∶寄生也。《尔 
雅》云∶寓木宛童。郭璞云∶寄生树,一名茑。《毛诗》云∶茑与女萝。《传》云∶茑,寄生 
山也;陆 


\section{}杜仲
内容:味辛,平。主腰脊痛,补中,益精气,坚筋骨,强志,除阴下痒湿,小便余沥。久服, 
轻身、耐老。一名思仙。生山谷。 
《吴普》曰∶杜仲,一名木绵,一名思仲(《御览》)。 
《名医》曰∶一名思肿,一名木绵。生上虞及上党、汉中,二月、五月、六月、 
九月采皮。 
案∶《广雅》云∶杜仲,曼榆也。《博物志》云∶杜仲,皮中有丝,折之则见。 


\section{}女贞实
内容:味苦,平。主补中,安五脏,养精神,除百疾。久服,肥健、轻身、不老。生山谷。 
《名医》曰∶生武陵,立冬采。 
案∶《说文》云∶桢,刚木也。《东山经》云∶太山上多桢木。郭璞云∶女桢也,叶冬 
不凋。《毛诗》云∶南山有杞。陆玑云∶木杞,其树如樗(陈藏器作栗),一名狗骨,理白滑 
,其子为木虻子,可合药,《司马相如赋》有女贞。师古曰∶女贞树,冬夏常青,未尝凋落 
,苦有节操,故以名为焉。陈藏器云∶冬青也。 


\section{}木兰
内容:味苦,寒。主身大热在皮肤中,去面热、赤 、酒 ,恶风 疾,阴下痒湿。明耳目。 
一名林兰。生川谷。 
《名医》曰∶一名杜兰,皮似桂而香。生零陵及太山。十二月采皮,阴干。 
案∶《广雅》云∶木栏,桂栏也。刘逵注《蜀都赋》云∶木兰,大树也,叶似长生,冬 
夏荣,常以冬花。其实如小柿,甘美。南人以为梅,其皮可食。颜师古注《汉书》云∶皮似 
椒而香,可作面膏药。 


\section{}蕤核
内容:味甘,温,主心腹邪气,明目,目赤痛伤泪出。久服,轻身、益气、不饥。生川谷。 
《吴普》曰∶蕤核,一名 。神农、雷公∶甘,平,无毒。生池泽。八月采。补中,强 
志,明目,久服不饥(《御览》)。 
《名医》曰∶生函谷及巴西。 
案∶《说文》云∶ ,白 , 。《尔雅》云∶ ,白妥。郭璞云∶ ,小木,丛生有 
实如耳 ,紫赤可啖。《一切经音义》云∶本草作蕤,今 核是也。 


\section{}橘柚
内容:味辛,温。主胸中瘕热逆气,利水谷。久服,去臭、下气、通神。一名橘皮。生川谷(旧 
在果部,非)。 
《名医》曰∶生南山、江南。十月采。 
案∶《说文》云∶橘果,出江南,柚条也,似橙而酢。《尔雅》云∶柚条。郭璞云∶似 
橙实酢,生江南。禹贡云∶厥包,橘柚。伪孔云∶大曰橘,小曰柚。《列子·汤问篇》云∶ 
吴楚之国有木焉,其名为柚,碧树而冬生,实丹而味酸,食其皮汁,已愤厥之疾。《司马 
相如赋》有橘柚,张揖曰∶柚,即橙也,似橘而大,味酢皮浓。 
上木,上品二十种,旧一十九种。考果部,橘柚当入此。 


\section{}发
内容:味苦,温。主五癃,关格不通,利小便水道,疗小儿痫、大人痉,仍自还神化。 
案∶《说文》云∶发根也, 也, 也,或作 。《毛诗》云∶不屑, 也,《笺 
》云∶ , 也。《仪礼》云∶主妇被锡,注云被锡,读为 ,古者或剔贱者、刑者之发 
,以被妇人之 为饰,因名 焉。李当之云∶是童男发。据汉人说∶发 ,当是剃荆人发 
,或童男发。《本经》不忍取人发用之,故用剃余也。方家至用天灵盖害及枯骨,卒不能治 
病。古人所无矣。 
上人,一种,旧同。 


\section{}龙骨
内容:味甘,平。主心腹鬼注,精物老魁,咳逆,泄利脓血,女子漏下症瘕坚结,小儿热气惊 
痫。齿∶主小儿、大人惊痫 疾狂走,心下结气,不能喘息,诸痉,杀精物。久服,轻身、 
通神明、延年。生山谷。 
《吴普》曰∶龙骨,生晋地山谷阴大水所过处,是龙死骨也。青白者,善。十二月采, 
或无时。龙骨,畏干漆、蜀椒、理石。龙齿,神农、李氏∶大寒,治惊痫,久服,轻身(《 
御览》、《大观本》节文)。 
《名医》曰∶生晋地及太山、岩水岸上穴中死龙处,采无时。 
案∶《范子计然》云∶龙骨,生河东。 


\section{}麝香
内容:味辛,温。主辟恶气,杀鬼精物,温疟,蛊毒,痫痉,去三虫。久服除邪,不梦寤厌寐。 
生川谷。 
《名医》曰∶生中台及益州、雍州山中,春分取之。生者益良。 
案∶《说文》云∶麝,如小麋,脐有香,黑色獐也(《御览》引多三字)。《尔雅》云∶ 
麝父麇足。郭璞云∶脚似麇,有香。 


\section{}牛黄
内容:味苦,平。主惊痫,寒热热盛狂痉,除邪逐鬼。生平泽。 
《吴普》曰∶牛黄,味苦,无毒。牛出入呻(《御览》作鸣吼)者,有之。夜有光(《御 
览》作夜视有光),走(《御览》有牛字)角中;牛死,入胆中,如鸡子黄(《后汉书》延笃 
传 
注)。 
《名医》曰∶生晋地。于牛得之,即阴干百日,使时躁,无令见日月光。 


\section{}熊脂
内容:味甘,微寒。主风痹不仁,筋急,五脏腹中积聚,寒热羸瘦,头疡白秃,面 。久服, 
强志、不饥、轻身。生山谷。 
《名医》曰∶生雍州,十一月取。 
案∶《说文》云∶熊兽似豕,山居,冬蛰。 


\section{}白胶
内容:味甘,平。主伤中劳绝,腰痛,羸瘦,补中益气,女人血闭无子,止痛、安胎。久服, 
轻身、延年。一名鹿角胶。 
《名医》曰∶生云中,煮鹿角作之。 
案∶《说文》云∶胶,昵也,作之以皮。《考工记》云∶鹿胶青白,牛胶火赤。郑云∶ 
皆谓煮,用其皮,或用角。 


\section{}阿胶
内容:味甘,平。主心腹内崩,劳极,洒洒如疟状,腰腹痛,四肢酸疼,女子下血,安胎。久 
服,轻身、益气,一名傅致胶。 
《名医》曰∶生平东郡,煮牛皮作之。出东阿。 
案∶二胶,《本经》不着所出,疑《本经》但作胶,《名医》增白字、阿字,分为二条 
上兽,上品六种。旧同。 


\section{}丹雄鸡
内容:味甘,微温。主女人崩中漏下,赤白沃,补虚温中,止血,通神,杀毒,辟不祥。头∶ 
主杀鬼,东门上者尤良,肪∶主耳聋。肠∶主遗溺。 裹黄皮∶主泄利。尿白∶主消渴, 
伤寒寒热。黑雌鸡∶主风寒湿痹,五缓六急,安胎。翮羽∶主下血闭。鸡子∶主除热,火疮 
痫痉,可作虎魄神物。鸡白蠹∶肥脂。生平泽。 
《吴普》曰∶丹鸡卵,可作琥珀(《御览》)。 
《名医》曰∶生朝鲜。 
案∶《说文》云∶鸡,知时畜也。籀文作鸡。肪,肥也。肠,大小肠也。 ,鸟 ; 
,鸟胃也。 ,粪也。羽茎也。翮羽,鸟长毛也。此作 ,省文。尿即 字古文,从,亦 
假音字也。 


\section{}雁肪
内容:味甘,平。主风挛拘急,偏枯,气不通利。久服,益气、不饥、轻身、耐老。一名 肪。 
生池泽。 
《吴普》曰∶雁肪,神农、岐伯、雷公∶甘,无毒(《御览》有 肪二字,当作一名 
肪)。杀诸石药毒(《御览》引云∶采无时)。 
《名医》曰∶生江南,取无时。 
案∶《说文》云∶雁,鹅也。 ,舒凫也。《广雅》云∶ 鹅,仓 雁也。凫 ,鸭也 
。《尔雅》云∶舒雁,鹅。郭璞云∶《礼记》曰∶出如舒雁,今江东呼 。又舒凫。 ,郭 
璞云∶鸭也。《方言》云∶雁自关而东,谓之 鹅;南楚之外,谓之鹅,或谓之仓 。据《 
说文》云∶别有雁,以为鸿雁字,鸭字,鸭,即雁之急音,此雁肪,即鹅、鸭脂也。当作 
雁字。《名医》不晓,别出 肪条,又出白鸭、鹅条,反疑此为鸿雁,何其谬也。陶、苏皆 
乱说之。 
上禽,上品二种,旧同。 


\section{}石蜜
内容:味甘,平。主心腹邪气,诸惊痉痫,安五脏,诸不足,益气补中,止痛解毒,除众病, 
和百药。久服,强志、轻身、不饥、不老。一名石饴。生山谷。 
《吴普》曰∶石蜜,神农、雷公∶甘,气平。生河源或河梁(《御览》又一引云∶生武 
都山谷)。 
《名医》曰∶生武都河源及诸山石中。色白如膏者,良。 
案∶《说文》云∶蜜蜂。甘饴也。一曰螟子,或作蜜。《中山经》云∶平逢之山多沙石 
,实惟蜂蜜之庐。郭璞云∶蜜,赤蜂名。《西京杂记》云∶南越王献高帝石蜜五斛。《玉篇 
》云∶蝇螽,甘饴也。苏恭云∶当去石字。 


\section{}蜂子
内容:味甘,平。主风头,除蛊毒,补虚羸伤中。久服,令人光泽、好颜色,不老,大黄蜂子∶ 
主心腹复满痛,轻身益气。土蜂子∶主痈肿。一名蜚零。生山谷。 
《名医》曰∶生武都。 
案∶《说文》云∶蜂,飞虫螫人者。古文省作螽。《广雅》云∶蠓 ,蜂也。又土蜂, 
也。《尔雅》云∶土蜂,郭璞云∶今江南大蜂。在地中作房者,为土蜂;啖其子,即 
马蜂,今荆巴间呼为 。又木蜂,郭璞云∶似土蜂而小,在树上作房,江东亦呼为木蜂,又 
食其子。《礼记·檀弓》云∶范,则冠。郑云∶范,蜂也。《方言》云∶蜂,燕赵之间,谓之 
,其小者,谓之 ,或谓之蚴蜕;其大而蜜,谓之壶蜂。郭璞云∶今黑蜂,穿竹木作 
孔,亦有蜜者,或呼笛师。按∶蜂,名为范者,声相近,若《司马相如赋》以泛为枫。《左 
传》 即汛汛也。 


\section{}蜜蜡
内容:味甘,微温。主下利脓血,补中,续绝伤金创。益气、不饥、耐老。生山谷。 
《名医》曰∶生武都蜜房木石间。 
案∶《西京杂记》云∶南越王献高帝蜜蜡二百枚。《玉篇》云∶蜡,蜜滓。陶弘景云∶ 
白蜡生于蜜中,故谓蜜蜡。《说文》无蜡字。张有云∶腊,别蜡,非。旧作蜡,今据改。 


\section{}牡蛎
内容:味咸,平。主伤寒寒热,温疟洒洒,惊恚怒气,除拘缓鼠 ,女子带下赤白。久服,强 
骨节、杀邪气、延年。一名蛎蛤。生池泽。 
《名医》曰∶一名牡蛤。生东海。采无时。 
案∶《说文》云∶虿,蚌属,似 ,微大,出海中,今民食之。读苦赖。又云∶蜃属, 
有三,皆生于海。蛤蛎,千岁雀所化,秦谓之牡蛎。 


\section{}龟甲
内容:味咸,平。主漏下赤白,破症瘕、 疟,五痔、阴蚀,湿痹,四肢重弱,小儿囟不合。 
久服,轻身、不饥。一名神屋。生池泽。 
《名医》曰∶生南海及湖水中。采无时。 
案∶《广雅》云∶介,龟也。高诱注《淮南》云∶龟壳,龟甲也。 


\section{}桑螵蛸
内容:味咸,平。主伤中,疝瘕,阴痿,益精生子,女子血闭腰痛,通五淋,利小便水道。一 
名蚀疣。生桑枝上。采,蒸之。 
《吴普》曰∶桑蛸条,一名(今本脱此二字)蚀疣,一名害焦,一名致。神农∶咸,无 
毒(《御览》)。 
《名医》曰∶螳螂子也。二月、三月采,火炙。 
案∶《说文》云∶蜱,蜱蛸也。或作蜱蛸。虫蛸,螳螂子。《广雅》云∶ ,乌,涕 
冒焦,螵蛸也。《尔雅》云∶不过螳 ,其子蜱蛸。郭璞云∶一名 焦,螳 卵也。《范子 
计然》云∶螵蛸,出三辅,上价三百。旧作螵,声相近,字之误也。《玉篇》云∶蜱,同螵 


\section{}海蛤
内容:味苦,平。主咳逆上气,喘息烦满,胸痛寒热。一名魁蛤。 
《吴普》曰∶海蛤,神农∶甘,岐伯∶甘;扁鹊∶咸。大节头有文,文如磨齿,采无时 
《名医》曰∶生南海。 
案∶《说文》云∶蛤,蜃属。海蛤者,百岁燕所化;魁蛤,一名复累,老服翼所化。《 
尔雅》云∶魁陆。郭璞云∶《本草》云∶魁,状如海蛤,圆而浓朴,有理纵横,即今之蚶也 
。《周礼》鳖人供 。郑司农云∶ ,蛤也。杜子春云∶ , 也。《周书》王会云∶东越 
海蛤。孔晃云∶蛤,文蛤。按∶《名医》别出海蛤条,云一名魁陆,一名活东,非。 


\section{}文蛤
内容:主恶疮,蚀(《御览》作除阴蚀)五痔(《御览》下有大孔出血。《大观》本作黑字)。 
《名医》曰∶生东海,表有文,采无时。 
蠡鱼(《初学记》引作鳢鱼) 味苦,寒。主湿痹,面目浮肿,下大水,一名 鱼。生 
池 
案∶《说文》云∶KT , 也。 ,KT 也。读若裤栊 。《广雅》云∶鲡,鲡 也。 
《尔雅》云∶鳢。郭璞注 也。《毛诗》云∶鲂鳢。《传》云∶鳢 也。据《说文》云∶鳢, 
也,与KT 不同。而毛苌、郭璞以鲷释鳢,与许不合。然《初学记》引此亦作鳢,盖二 
字音同,以致讹舛,不可得祥。《广雅》又作鲡,亦间 。又《广志》云∶豚鱼,一名 (《御 
衍义》曰∶蠡鱼,今人谓之黑鲤鱼,道家以为头有星为厌。据此诸说,若作鲤字,《说文》 
所云 ,《广志》以为江豚,《本草衍义》以为黑鲤鱼;若作鲤字,《说文》以为 ,《广雅》 
以为鳗鲡,陆玑以为鲍鱼,说各不同,难以详究。 


\section{}鲤鱼胆
内容:味苦,寒。主目热赤痛青盲,明目。久服,强悍、益志气志气。生池泽。 
《名医》曰∶生九江,采无时。 
案∶《说文》云∶鲤, 也; ,鲤也。《尔雅》云∶鲤 。舍人云∶鲤,一名 。郭 
璞注鲤云∶今赤鲤鱼。注 云∶大鱼似鲟。《毛诗》云∶ 鲔发发。《传》云∶ ,鲤也 
。据此,知郭璞别为二,非矣。《古今 
注》云∶兖州人呼赤鲤为赤骥,谓青鲤为青马,黑鲤为元驹,白鲤为白骐,黄鲤为黄雉。 
上虫、鱼。上品一十种,旧同。 


\section{藕实茎}

原序录中无“茎”字,正文有,今据补。

内容:味甘,平。主补中养神,益气力,除百疾。久服,轻身、耐老、不饥、延年。 
一名水芝丹。生池泽。 
《名医》曰∶一名莲。生汝南。八月采。 
案∶《说文》云∶藕,夫渠根;莲,夫渠之实也;茄,夫渠茎。《尔雅》云∶荷,芙 
渠。郭璞云∶别名芙蓉,江东呼荷;又其茎茄;其实莲。郭璞云∶莲,谓房也,又其根 
,藕。 


\section{}大枣
内容:味甘,平。主心腹邪气,安中养脾,肋十二经,平胃气,通九窍,补少气、少津液、身 
中不足,大惊,四肢重,和百药。久服,轻身、长年。叶覆麻黄,能令出汗。生平泽。 
《吴普》曰∶枣主调中,益脾气,令人好颜色,美志气(《大观本草》引《吴氏本草》)。 
《名医》曰∶一名干枣,一名美枣,一名良枣。八月采,曝干。生河东。 
案∶《说文》云∶枣,羊枣也。《尔雅》云∶遵羊枣。郭璞云∶实小而圆,紫黑色,今 
俗呼之为羊矢枣。又洗大枣。郭璞云∶今河东猗氏县出大枣也,如鸡卵。 


\section{}葡萄
内容:味甘,平。主筋骨湿痹,益气、倍力、强志,令人肥健、耐饥、忍风寒。久食,轻身、 
不老、延年。可作酒。生山谷。 
《名医》曰∶生陇西五原敦煌。 
案∶《史纪·大宛列传》云∶大宛左右,以葡萄为酒,汉使取其实来,于是天子始种苜 
蓿、葡萄,肥饶也,或疑《本经》不合有葡萄,《名医》所增,当为黑字。然《周礼》场 
人云∶树之果 ,珍异之物。郑玄云∶珍异,葡萄、枇杷之属,则古中国本有此,大宛种类 
殊常,故汉特取来植之。旧作葡,据《史记》作蒲。 


\section{}蓬
内容:味酸,平。主安五脏,益精气,长阴令坚,强志倍力,有子。久服,轻身、不老一名 
复盆。生平泽。 
《吴普》曰∶ 盆,一名决盆(《御览》)。《甄氏本草》曰∶复盆子,一名马 ,一名 
陆荆(同上)。 
《名医》曰∶一名陵 ,一名阴药。生荆山及冤句。 
案∶《说文》云∶ ,木也; ,缺盆也。《广雅》云∶ 盆,陆英,莓也。《尔雅》 
云∶ , 盆。郭璞云∶复盆也,实似莓而小,亦可食。《毛诗》云∶葛苗苗之。陆玑云∶ 
一名巨瓜,似燕 ,亦连蔓,叶似艾,白色,其子赤,可食。《列仙传》云∶昌容食蓬 根 
。李当之云∶即是人所食莓。陶弘景云∶蓬 ,是根名;复盆,是实名。 


\section{}鸡头实
内容:味苦,平。主湿痹,腰脊膝痛,补中,除暴疾,益精气,强志,令耳目聪明。久服,轻 
身、不饥、耐老、神仙。一名雁啄实。生池泽。 
《名医》曰∶一名芡,生雷泽。八月采。 
案∶《说文》云∶芡,鸡头也。《广雅》云∶ 芡,鸡头也。《周礼》笾人∶加笾之实 
,芡。郑元云∶芡,鸡头也。《方言》云∶ 芡,鸡头也,北燕谓之 ;青徐淮泗之间谓之 
芡;南楚江湘之间谓之鸡头,或谓之雁头,或谓之乌头。《淮南子·说山训》云∶鸡头,已 
。高诱云∶水中芡,幽州谓之雁头。《古今注》云∶叶似荷而大,叶上蹙绉如沸,实有芒 
刺,其中有米,可以度饥,即今茑子也。 
上果,上品五种。旧六种,今以橘、柚入木。 


\section{}胡麻
内容:味甘,平。主伤中虚羸,补五内(《御览》作脏),益气力,长肌肉,填髓脑。久服,轻 
身、不老。一名巨胜。叶,名青 。生川泽。 
《吴普》曰∶胡麻,一名方金。神农、雷公∶甘,无毒。一名狗虱,立秋采。 
《名医》曰∶一名狗虱,一名方茎,一名鸿藏。生上党。 
案∶《广雅》云∶狗虱,巨胜,藤 ,胡麻也。《孝经·援神契》云∶ 胜延年。宋均 
云∶世以 胜为苟杞子。陶弘景云∶本生大宛,故曰胡麻。按∶《本经》已有此,陶说非也 
。且与麻贲并列,胡之言大,或以叶大于麻。故名之。 


\section{}麻贲
内容:味辛,平。主五劳七伤,利五脏,下血,寒气。多食,令人见鬼狂走。久服,通神明、 
轻身。一名麻勃。麻子∶味甘,平。主补中益气,肥健、不老、神仙。生川谷。 
《吴普》曰∶麻子中仁,神农、岐伯∶辛;雷公、扁鹊∶无毒。不欲牡蛎、白薇。先藏 
地中者,食,杀人。麻蓝,一名麻贲,一名青欲,一名青葛。神农∶辛;岐伯∶有毒;雷公 
∶甘。畏牡蛎、白薇。叶上有毒,食之杀人。麻勃,一名花。雷公∶辛,无毒。畏牡蛎(《 
御览》)。 
《名医》曰∶麻勃,此麻花上勃勃者。七月七日采,良。子,九月采。生太山。 
案∶《说文》云∶麻与秫同,人所治在屋下, 麻也, 实也,或作 ,荸,麻母也 
。,KT 也,以贲为杂 香草。《尔雅》云∶ , 实, 麻。孙炎云∶ ,麻子也。郭 
璞云∶别二名。又KT ,麻母,郭璞云∶苴,麻盛子者。《周礼》笾朝事之笾,其实 。郑 
云∶ , 实也。郑司农云∶麻实曰 。《淮南子·齐俗训》云∶胡人见 ,不知其可以为 
布。高诱云∶ ,麻实也。据此则弘景以为牡麻无实,非也。《唐本》以为麻实,是。 
上米、谷,上品二种。旧三种。今以青 入草。 


\section{}冬葵子
内容:味甘,寒。主五脏六腑寒热、羸瘦、五癃,利小便。久服,坚骨、长肌肉、轻 
《名医》曰∶生少室山。十二月采之。 
案∶《说文》云∶KT ,古文终,葵菜也。《广雅》云∶ ,葵也。考KT 与终形相近, 
当即《尔雅》 葵。《尔雅》云∶ 葵,繁露。郭璞云∶承露也,大茎小叶,花紫黄色。《 
本草图经》云∶吴人呼为繁露,俗呼胡燕支,子可妇人涂面及作口脂。按∶《名医》别有落 
葵条,一名繁露,亦非也。陶弘景以为终冬至春作子,谓之冬葵,不经甚矣。 


\section{}苋实
内容:味甘,寒。主青盲,明目,除邪,利大小便,去寒热。久服,益气力、不饥、轻身。一 
名马苋。 
《名医》曰∶一名莫实。生淮阳及田中,叶如蓝。十一月采。 
案∶《说文》云∶苋,苋菜也。《尔雅》云∶蒉,赤苋。郭璞云∶今苋叶之赤茎者。李 
当之云∶苋实,当是今白苋。《唐本注》云∶赤苋,一名 ,今名莫实,字误。 


\section{}瓜蒂
内容:味苦,寒。主大水身面四肢浮肿,下水,杀蛊毒,咳逆上气,及食诸果,病在胸腹中, 
皆吐下之。生平泽。 
《名医》曰∶生蒿高。七月七日采,阴干。 
案∶《说文》云∶瓜, 也,象形;蒂,瓜当也。《广雅》云∶水芝,瓜也。陶弘景云∶ 
甜瓜蒂也。 


\section{}瓜子
内容:味甘,平。主令人阅泽,好颜色,益气不饥。久服,轻身、耐老。一名水芝(《御览》 
作土芝)。生平泽。 
《吴普》曰∶瓜子,一名瓣。七月七日采,可作面脂(《御览》)。 
《名医》曰∶一名白瓜子。生蒿高。冬瓜仁也,八月采。 
案∶《说文》云∶瓣,瓜中实。《广雅》云∶冬瓜 也,其子谓之瓤。陶弘景云∶白, 
当为甘,旧有白字。据《名医》云∶列中白瓜子,则本名当无。 


\section{}苦菜

原序录中作“菜”,正文作“采”。采通菜,《说文通州定声·赜部》:“采,假借为菜。”《周礼·春官·大胥》:“舍采合舞。”郑玄注:“采读为菜。”

内容:味苦,寒。主五脏邪气,厌谷,胃痹。久服,安心益气,聪察少卧,轻身、耐老。一名 
荼草,一名选。生川谷。 
《名医》曰∶一名游冬。生益州山陵道旁,凌冬不死。三月三日采,阴干。 
案∶《说文》云∶荼,苦菜也。《广雅》云∶游冬,苦菜也。《尔雅》云∶荼,苦菜; 
又,苦荼。郭璞云∶树小如栀子,冬生叶,可煮作羹,今呼早莱者为荼,晚取者为茗, 
一名 ,蜀人名之苦菜。陶弘景云∶此即是今茗,茗,一名荼,又令人不眠,亦凌冬不凋而 
兼其止。生益州。《唐本》注驳之,非矣。选与 ,音相近。 
上菜,上品五种。旧同。 

\chapter{中品}

\section{}雄黄
内容:味苦,平、寒。主寒热,鼠 恶创,疽痔死肌,杀精物、恶鬼、邪气、百虫毒,胜五兵。 
炼食之,轻身、神仙。一名黄食石。生山谷。 
《吴普》曰∶雄黄,神农∶苦。山阴有丹雄黄,生山之阳,故曰雄,是丹之雄, 
所以名雄黄也。 
《名医》曰∶生武都敦煌之阳。采无时。 
案∶《西山经》云∶高山其下多雄黄。郭璞云∶晋太兴三年,高平郡界有山崩,其中出 
数千斤雄黄。《抱朴子·仙药篇》云∶雄黄,当得武都山所出者,纯而无杂,其赤如鸡冠, 
光明晔晔,可用耳;其但纯黄似雄黄,色无赤光者,不任以作仙药,可以合理病药耳。 


\section{}石流黄
内容:(流,旧作硫。《御览》引作流,是) 
味酸,温。主妇人阴蚀,痈痔恶血,坚筋骨,除头秃,能化金银铜铁奇物(《御览》引 
云∶石流 
《吴普》曰∶硫黄,一名石留黄。神农、黄帝、雷公∶咸,有毒;医和、扁鹊∶ 
苦,无毒。或生易阳,或河西。或五色,黄,是潘水石液也(潘,即矾古字),烧令有紫焰 
者, 
八月、九月采,治妇 
《名医》曰∶生东海牧羊山,及太山河西山。矾石液也。 
案∶《范子计然》∶石流黄,出汉中。又云∶刘冯饵石流黄而更少。刘逵注《吴都赋 
》云∶流黄,土精也。 


\section{}雌黄
内容:味辛,平。主恶创头秃痂疥,杀毒虫虱,身痒,邪气、诸毒。炼之。久服,轻身、增年、 
不老。生山谷。 
吴普曰∶磁石,一名磁君。 
《名医》曰∶生武都,与雄黄同山生。其阴山有金,金精熏,则生雌黄。采无时。 


\section{}水银
内容:味辛,寒。主疥 痂疡、白秃,杀皮肤中虱,堕胎,除热,杀金、银、铜、锡毒。熔化 
还复为丹,久服,神仙、之死。生平土。 
《名医》曰∶一名汞。生符陵,出于丹砂。 
案∶《说文》云∶ ,丹沙所作为水银也。《广雅》云∶水银谓之汞。《淮南子·地形 
训》云∶白 ,九百岁,生白 ;白 ,九百岁,生白金。高诱云∶白 ,水银也。 


\section{}石膏
内容:味辛,微寒。主中风寒热,心下逆气惊喘,口干苦焦,不能息,腹中坚痛,除邪鬼,产 
乳,金创。生山谷。 
《名医》曰∶一名细石。生齐山及齐卢山、鲁蒙山。采无时。 


\section{}磁石
内容:味辛,寒。主周痹风湿,肢节中痛,不可持物,洗洗酸消,除大热烦满及耳聋。一名元 
石,生山谷。 
《吴普》曰∶磁石,一名磁君。 
《名医》曰∶一名处石。生太山及慈山山阴;有铁处,则生其阳。采无时。 
案∶《北山经》云∶灌题之山,其中多磁石。郭璞云∶可以取铁。《管子·地数篇》云∶ 
山上有磁石者,下必有铜。《吕氏春秋·精通篇》云∶磁石召铁。《淮南子·训》云∶磁石能 
引铁。只作慈,旧作磁,非。《名医》别出元石条,亦非。 


\section{}凝水石
内容:味辛,寒。主身热,腹中积聚、邪气,皮中如火烧,烦满。水饮之,久服,不饥。一名 
白水石。生山谷。 
《吴普》曰∶神农∶辛;岐伯、医和、扁鹊∶甘,无毒;李氏∶大寒。或生邯郸。采 
无时。如云母色(《御览》引云∶一名寒水石)。 
《名医》曰∶一名寒水石,一名凌水石,盐之精也。生常案及凝山,又中水县邯郸。 
《范子计然》云∶水石,出河东。色泽者,善。 


\section{}阳起石
内容:味咸,微温。主崩中漏下,破子藏中血,症瘕结气,寒热腹痛,无子,阴痿不起(《御 
览》引作阴阳不合),补不足(《御览》引有句挛二字)。一名白石。生山谷。 
《吴普》曰∶阳起石,神农、扁鹊∶酸,无毒;桐君、雷公、岐后∶咸,无毒;李氏∶ 
小寒,或生太山(《御览》引云∶或阳起云。采无时)。 
《名医》曰∶一名石生,一名羊起石,云母根也,生齐山及琅邪,或云山、阳起山、采 
无时。 


\section{}孔公孽
内容:味辛,温。主伤食不化,邪结气,恶创,疽 琅邪,利九窍,下乳汁(《御览》引云∶ 
一名通石。《大观本》作黑字)。生山谷。 
《吴普》曰∶孔公孽,神农∶辛;岐伯∶咸;扁鹊∶酸,无毒。色青黄。 
《名医》曰∶一名通石,殷孽根也。青黄色。生梁山。 


\section{}殷孽
内容:味辛,温。主烂伤瘀血,泄利寒热,鼠寒 ,症瘕结气。一名姜石。生山谷(按∶此当 
与孔公孽为一条)。 
《名医》曰∶钟乳根也。生越国,又梁山及南海,采无时。 


\section{}铁精
内容:平,主明目,化铜。铁落∶味辛,平。主风热恶创,疡疽创痂,疥气在皮肤中。铁∶主 
坚肌耐痛。生平泽(旧为三条,今并)。 
《名医》曰∶铁精,一名铁液。可以染色。生牧羊及 城或析城。采无时。 
案∶《说文》云∶铁,黑金也,或省作铁,古文作 。 


\section{}理石
内容:味辛,寒。主身热,利胃解烦,益精明目,破积聚,去三虫。一名石立制石。生山谷。 
《名医》曰∶一名饥石,如石膏,顺理而细。生汉中及卢山。采无时。 


\section{}长石
内容:味辛,寒。主身热,四肢寒厥,利小便,通血脉,明目,去翳眇,下三虫,杀蛊毒。久 
服,不饥。一名方石。生山谷。 
《吴普》曰∶长石,一名方石,一名直石。生长子山谷。如马齿,润泽,玉色长 
鲜。服之,不饥(《御览》)。 
《名医》曰∶一名土石,一名直石。理如马齿,方而润泽,玉色。生长子山及 
太山临淄,采无时。 


\section{}肤青
内容:味辛,平。主蛊毒及蛇、菜、肉诸毒,恶创。生川谷。 
《名医》曰∶一名推青,一名推石,生益州。 
案∶陶弘景云∶俗方及《仙经》,并无用此者,亦相与不复识。 
上玉石,中品一十四种。旧十六种。考铁落、铁,宜与铁精为一。 


\section{}干姜
内容:味辛,温。主胸满咳逆上气,温中止血,出汗,逐风湿痹,肠 ,下利。生者,尤良。 
久服,去臭气、通神明。生川谷。 
《名医》曰∶生楗为及荆、扬州。九月采。 
案∶《说文》云∶姜,御湿之菜也。《广雅》云∶ ,廉姜也。《吕氏春秋·本味篇》 
和之美者,阳朴之姜。高诱 
注∶阳朴,地名,在蜀郡。司马相如《上林赋》,有茈姜云 


\section{}耳实
内容:味甘,温。主风头寒痛,风湿周痹,四肢拘挛痛,恶肉死肌。久服益气,耳目聪明,强 
志轻身。一名胡 ,一名地葵。生 
《名医》曰∶一名 ,一名常思,生安陆及六安田野,实熟时采。 
案∶《说文》云∶ ,卷耳也;苓,卷耳也。《广雅》云∶苓耳, ,常 ,胡 , 
耳也。《尔雅》云∶苍耳,苓耳。郭璞云∶江东呼为常 ,形似鼠耳,丛生如盘。《毛诗》 
云∶采采卷耳。《传》云∶卷耳,苓耳也。陆玑云∶叶青,白色,似胡荽,白花,细茎蔓生 
。可煮为茹,滑而少味;四月中生子,正如妇人耳 ,今或谓之耳 草。郑康成谓是白胡荽 
,幽州人谓之爵耳。《淮南子·览冥训》云∶位贱尚 。高诱云∶ 者, 耳,菜名也。幽 
冀谓之檀菜,雒下谓之胡 。 


\section{}葛根
内容:味甘,平。主消渴,身大热,呕吐,诸痹,起阴气,解诸毒,葛谷,主下利十岁以上。 
一名鸡齐根。生种谷。 
《吴普》曰∶葛根,神农∶甘。生太山(《御览》)。 
《名医》曰∶一名鹿藿,一名黄斤。生汶山。五月采根,曝干。 


\section{栝蒌根}

原序录中作“括”,正文作“栝”,今据改。

味苦,寒。主消渴,身热烦满,大热,补虚安中,续绝伤。一名地楼。生川谷及山阴。 
《吴普》曰∶栝蒌,一名泽巨,一名泽姑(《御览》)。 
《名医》曰∶一名果裸,一名天瓜,一名泽姑。实,名黄瓜。二月、八月采根,曝干, 
三十日成。生宏农。 
案∶《说文》云∶ , 蒌,果也。《广雅》云∶王白, 也(当为王 )。《尔雅》 
云∶果裸之实,栝蒌。郭璞云∶今齐人呼之为天瓜。《毛诗》云∶果裸之实,亦施于宇。《 
传》云∶果裸,栝蒌也。《吕氏春秋》云∶王善生。高诱云∶善,或作瓜, 也。案∶《 
吕氏春秋》善字,乃 之误。 


\section{}苦参
内容:味苦,寒。主心腹结气,症瘕积聚,黄胆,溺有余沥,逐水,除痈肿,补中明目,止泪。 
一名水槐,一名苦识。生山谷及田野。 
《名医》曰∶一名地槐,一名菟槐,一名骄槐,一名白茎,一名虎麻,一名芩茎,一名 
禄曰,一名陵郎。生汝南。三月、八月、十月采根,曝干。 


\section{}当归
内容:味甘,温。主咳逆上气,温疟、寒热,洗在皮肤中(《大观本》,洗音癣),妇人漏下绝 
子,诸恶创疡、金创。煮饮之。一名干归。生川谷。 
《吴普》曰∶当归,神农、黄帝、桐君、扁鹊∶甘,无毒;岐伯、雷公∶辛,无毒;李 
氏∶小温。或生羌胡地。 
《名医》曰∶生陇西。二月、八月采根,阴干。 
案∶《广雅》云∶山靳,当归也。《尔雅》云∶薜,山靳。郭璞云∶今似靳而粗大。又 
薜,白靳,郭璞云∶即上山靳。 
《范子计然》云∶当归,出陇西。无枯者,善。 


\section{}麻黄
内容:味苦,温。主中风,伤寒头痛,温疟,发表出汗,去邪热气,止咳逆上气,除寒热,破 
症坚积聚。一名龙沙。 
《吴普》曰∶麻黄,一名卑相,一名卑坚。神农、雷公∶苦,无毒;扁鹊∶酸,无毒; 
李氏∶平。或生河东。四月、立秋采(《御览》)。 
《名医》曰∶一名卑相,一名卑盐。生晋地及河东。立秋采茎,阴干令青。 
案∶《广雅》云∶龙沙,麻黄也。麻黄茎,狗骨也。《范子计然》云∶麻黄,出汉中三 
辅。 


\section{}通草
内容:(《御览》作草) 
味辛,平。主去恶虫,除脾胃寒热,通利九窍、血脉、关节,令人不忘。一名附支。生 
山谷。 
《吴普》曰∶通草,一名丁翁,一名附支。神农、黄帝∶辛;雷公∶苦。生石城山谷, 
叶菁蔓延。止汗,自正月采(《御览》)。 
《名医》曰∶一名丁翁。生石城及山阳。正月采枝,阴干。 
案∶《广雅》云∶附支, 草也。《中山经》云∶升山,其草多寇脱。郭璞云∶寇脱草 
,生南方,高丈许,似荷叶,而茎中有瓤正白,零陵人植而日灌之,以为树也。《尔雅》云 
∶离南,活 。郭璞注同。又倚商,活脱。郭璞云∶即离南也。《范子计然》云∶ 草,出 
三辅。 


\section{}芍药
内容:味苦,平。主邪气腹痛,除血痹,破坚积、寒热、疝瘕,止痛,利小便,益气(《艺文 
类聚》引云∶一名白术。《大观本》作黑字)。生川谷及 
《吴普》曰∶芍药,神农∶苦;桐君∶甘,无毒;岐伯∶咸;李氏∶小寒;雷公 
∶酸。一名甘积,一名解仓,一名诞,一名余容,一名白术。三月三日采(《御览》)。 
《名医》曰∶一名白术,一名余容,一名犁食,一名解食,一名铤。生中岳。二 
月、八月采根,曝干。 
案∶《广雅》云∶挛夷,芍药也;白术,牡丹也。《北山经》云∶绣山,其草多芍药。 
郭璞云∶芍药,一名辛夷,亦香草属。《毛诗》云∶赠之以芍药。《传》云∶芍药,香草。《范 
子计然》云∶芍药,出三辅。崔豹《古今注》云∶芍药有三种∶有草芍药,有木芍药。木有 
花,大则色深,俗呼为牡丹,非 


\section{}蠡实
内容:味甘,平。主皮肤寒热,胃中热气,寒湿痹,坚筋骨,令人嗜食。久服,轻身。花、叶∶ 
去白虫。一名剧草,一名三坚,一名豕首。生川谷。 
《吴普》曰∶蠡实,一名剧草,一名三坚,一名剧荔华(《御览》), 
一名泽蓝,一名豕首。神农、黄帝∶甘,辛,无毒。生宛句。五月采(同上)。 
《名医》曰∶一名荔实。生河东。五月采实,阴干。 
案∶《说文》云,荔,草也,似蒲而小,根可作刷。《广雅》云∶马KT ,荔也。《月令 
》云∶仲冬之月,荔挺出。郑云∶荔挺,马薤也。高诱注《淮南子》云∶荔马,荔草也。《通 
俗文》云∶一名马兰。颜之推云∶此物河北平泽率生之,江东颇多,种于阶庭,但呼为早蒲, 
故不识马薤。 


\section{}瞿麦
内容:味苦,寒。主关格,诸癃结,小便不通,出刺,决痈肿,明目去翳,破胎堕子,下闭血。 
一名巨句麦。生川谷。 
《名医》曰∶一名大菊,一名大兰。生大山。立秋采实,阴干。 
案∶《说文》云∶蘧,蘧麦也。菊、大菊,蘧麦。《广雅》云∶茈威、陵苕,蘧麦也。 
《尔雅》云∶大菊,蘧麦。郭璞云∶一名麦句姜,即瞿麦。陶弘景云∶子颇似麦,故名瞿麦 


\section{}元参
内容:味苦,微寒。主腹中寒热积聚,女子产乳余疾,补肾气,令人目明。一名重台。 
《吴普》曰∶元参,一名鬼藏,一名正马,一名重台,一名鹿腹,一名端,一名元台, 
神农、桐君、黄帝、雷公、扁鹊∶苦,无毒;岐伯∶咸;李氏∶寒。或生冤朐山阳。二月生 
叶如梅毛,四四相植似芍药,黑茎方高四、五尺,花赤,生枝间,四月,实黑(《御览》)。 
、四月采根,曝干。 
案∶《广雅》云∶鹿肠,元参也。《范子计然》云∶元参,出三辅。青色者,善。 


\section{}秦艽
内容:味苦,平。主寒热邪气,寒湿,风痹,肢节痛,下水,利小便。生山谷。 
《名医》曰∶生飞乌山。二月、八月采根,曝干。 
案∶《说文》云∶KT ,草之相 者,《玉篇》作KT ,居包切,云秦艽,药艽同。萧 
炳 


\section{}百合
内容:味甘,平。主邪气腹张,心痛,利大小便,补中益气。生川谷。 
《吴普》曰∶百合,一名重迈,一名中庭。生冠朐及荆山(《艺文类聚》引云∶一名 
重匡)。 
《名医》曰∶一名重箱,一名摩罗,一名中逢花,一名强瞿。生荆州。二月、八月采根 
,曝干。 
案∶《玉篇》云∶蹯,百合蒜也。 


\section{}知母
内容:味苦,寒。主消渴热中,除邪气,肢体浮肿,下水,补不足,益气。一名 母,一名连 
母,一名野蓼,一名地参,一名水参,一名水浚,一名货母,一名 母。生川谷。 
《吴普》曰∶知母,神农、桐君∶无毒。补不足,益气(《御览》引云∶一名提母)。 
《名医》曰∶一名女雷,一名女理,一名儿草,一名鹿列,一名韭蓬,一名儿踵 
草,一名东根,一名水须,一名沈燔,一名薅。生河内。二月、八月采根,曝干。 
案∶《说文》云∶ , 母也。 ,苋藩也,或从爻作薅。《广雅》云∶ 母、儿踵, 
东根也。《尔雅》云∶薅, 藩。郭璞云∶生山上。叶如韭,一曰 母。《范子计然》云 
∶母,出三辅,黄白者,善。《玉篇》作 母。 


\section{}贝母
内容:味辛,平。主伤寒烦热,淋沥,邪气,疝瘕,喉痹,乳难,金创,风痉。一名空草。 
《名医》曰∶一名药实,一名苦花,一名苦菜,一名商( 字)草,一名勤母。生晋地。 
十月采根,曝干。 
案∶《说文》云∶ ,贝母也。《广雅》云∶贝父,药实也。《尔雅》云∶ ,贝母。 
郭璞云∶根如小贝,圆而白花,叶似韭。《毛诗》云∶言采其虻。《传》云∶虻,贝母也。 
陆玑云∶其叶如栝蒌而细小,其子在根下如芋子,正白,四方连累相着有分解也。 


\section{}白芷
内容:味辛,温。主女人漏下赤白,血闭,阴肿,寒热,风头侵目泪出。长肌肤,润泽,可作 
面脂。一名芳香。生川谷。 
《吴普》曰∶白芷,一名 ,一名苻离,一名泽芬,一名 (《御览》)。 
《名医》曰∶一名白芷,一名 ,一名莞,一名苻离,一名泽芬。叶,一名 麻,可作 
浴汤。生河东下泽。二月、八月采根,曝干。 
案∶《说文》云,芷, 也; ,楚谓之篱,晋谓之 ,齐谓之芷。《广雅》云∶白芷 
,其叶谓之药。《西山经》云∶号山,其草多药 。郭璞云∶药,白芷别名; ,香草也。 
《淮南子·修务训》云∶身苦秋药被风。高诱云∶药,白芷,香草也。王逸注《楚词》云∶ 
药,白芷。按∶《名医》一名莞云云,似即《尔雅》莞,苻离,其上鬲。而《说文》别有 
,夫离也。 ,夫蓠上也。是非一草。舍人云∶白蒲,一名苻离,楚谓之莞,岂蒲与芷相似 
,而《名医》误合为一乎。或《说文》云∶楚谓之蓠,即夫篱也,未可得详。旧作芷,非。 


\section{}淫羊藿
内容:味辛,寒。主阳痿绝伤,茎中痛,利小便,益气力,强志。一名刚前。生山谷。 
《吴普》曰∶淫羊藿,神农、雷公∶辛;李氏∶小寒。坚骨(《御览》)。 
《名医》曰∶生上山郡阳山。 


\section{}黄芩
内容:味苦,平。主诸热黄胆,肠 泄利,逐水,下血闭,恶创,恒蚀火疡。一名腐肠。生川 
谷。 
《吴普》曰∶黄芩,一名黄文,一名妒妇,一名虹胜,一名红芩,一名印头,一 
名内虚,神农、桐君、黄帝、雷公、扁鹊∶苦,无毒;李氏∶小温。二月生赤黄叶,两两四 
四相值,茎空中或方员,高三、四尺,四月花紫红赤,五月实黑、根黄。二月至九月采(《御 
览》) 
《名医》曰∶一名空肠,一名内虚,一名黄文,一名红芩,一名妒妇。生秭归及 
冤句。三月三日采根,阴干。 
案∶《说文》云∶ ,黄 也。《广雅》云∶ 
云∶黄芩,出三辅。色黄者, 
善。 


\section{}狗脊
内容:味苦,平。主腰背,强关机,缓急,周痹寒湿,膝痛。颇利老人。一 
《吴普》曰∶狗脊,一名狗青,一名赤节。神农,桐君、黄帝、岐伯、雷公 
、扁鹊∶甘,无毒;李氏∶小温。如萆 ,茎节如竹,有刺,叶圆赤,根黄白,亦如竹根, 
毛有刺。《岐伯经》云∶茎 
《名医》曰∶一名强膂,一名扶盖,一名扶筋,生常山,二月、八月采根,曝干。 
案∶《广雅》云∶菝 ,狗脊也。《玉篇》云∶菝 ,狗脊根也。《名医》别出菝 条, 
非。 


\section{}石龙芮
内容:味苦,平。主风寒湿痹,心腹邪气,利关节,止烦满。久服,轻身、明目、不老。一名 
鲁果能(《御览》作食果),一名地椹。生川泽石边。 
《吴普》曰∶龙芮,一名姜苔,一名天豆。神农∶苦,平;岐伯∶酸;扁鹊、李氏∶大 
寒;雷公∶咸,无毒。五月五日采(《御览》)。 
《名医》曰∶一名石能,一名彭根,一名天豆。生太山,五月五日采子,二月、 
八月采皮,阴干。 
案∶《范子计然》云∶石龙芮,出三辅。色黄者,善。 


\section{}茅根
内容:味甘,寒。主劳伤虚羸,补中益气,除瘀血、血闭、寒热,利小便。其苗,主下水。一 
名兰根,一名茹根。生山谷、田野。 
《名医》曰∶一名地管,一名地筋,一名兼杜。生楚地,六月采根。 
案∶《说文》云∶茅,菅也;菅,茅也。《广雅》云∶菅,茅也。《尔雅》云∶白华, 
野菅。郭璞云∶菅,茅属。《诗》云∶白华菅兮,白茅束兮。《传》云∶白华,野菅也,已沤, 
为菅。 


\section{}紫菀
内容:味苦,温。主咳逆上气,胸中 
《吴普》曰∶紫菀,一名青苑(《御览》)。 
《名医》曰∶一名紫茜,一名青苑。生房陵及真定邯郸。二月、三月采根,阴干 
案∶《说文》云∶菀,茈菀,出汉中房陵。陶弘景云∶白者,名白菀。《唐本》 
注云∶白菀,即女菀也。 


\section{}紫草
内容:味苦,寒。主心腹邪气,五疸,补中益气,利九窍,通水道。一名紫丹,一名紫芙(《御 
览》引云∶一名地血。《大观本》无文)。生川谷。 
《吴普》曰∶紫草节赤。二月花。(《御览》)。 
《名医》曰∶生砀山及楚地。三月采根,阴干。 
案∶《说文》云∶茈,草也;藐,茈草也, 草也,可以染留黄。《广雅》云∶茈 ,茈 
草也。《山海经》云∶劳山多 
茈草。郭璞云∶一名紫 ,中染紫也。《尔雅》云∶藐,茈草。郭璞云∶可以染紫。 


\section{}败酱
内容:味苦,平。主暴热火创、赤气,疥瘙,疸痔,马鞍热气。一名鹿肠。生川谷。 
《名医》曰∶一名鹿首,一名马草,一名泽败,生江夏,八月采根,曝干。 
案∶《范子计然》云∶败酱,出三辅。陶弘景云∶气如败酱,故以为名。 


\section{}白藓
内容:味苦,寒。主头风,黄胆,咳逆,淋沥,女子阴中肿痛,湿痹死肌,不可屈伸、起止、 
行步。生川谷。 
《名医》曰∶生上谷及冤句。四月、五月采根,阴干。 
案∶陶弘景云∶俗呼为白羊藓,气息正似羊膻,或名白膻。 


\section{}酸酱
内容:味酸,平。主热烦满,定志益气,利水道。产难,吞其实立产。一名醋酱。生川泽。 
《吴普》曰∶酸酱,一名酢酱(《御览》)。 
《名医》曰∶生荆楚及人家田园中。五月采,阴干。 
案∶《尔雅》云∶ ,寒酱。郭璞云∶今酸酱草,江东呼曰苦 。 


\section{}紫参
内容:味苦,辛,寒。主心腹积聚,寒热邪气,通九窍,利大小便。一名牡蒙。生山谷。 
《吴普》曰∶伏蒙,一名紫参,一名泉戎,一名音腹,一名伏菟,一名重伤。神 
农、黄帝∶苦;李氏∶大寒。生河西山谷或宛句商山。圆聚生,根黄赤月文,皮黑中紫,五 
月花 
《名医》曰∶一名众戎,一名童肠,一名马行。生河西及冤句。三月采根,火炙 
案∶《范子计然》云∶紫参,出三辅。赤青色者,善。 


\section{}本
内容:味辛,温。主妇人疝瘕,阴中寒、肿痛,腹中急,除风头痛,长肌肤,悦颜色。一名鬼 
卿,一名地新。生山谷。 
《名医》曰∶一名微茎,生崇山,正月、二月采根,曝干,三十日成。 
案∶《广雅》云∶山芷,蔚香, 本也。《管子·地员篇》云∶五臭畴生 本。《荀子 
·大略篇》云∶兰芷 本,渐于蜜醴,一佩易之。樊光注《尔雅》云∶ 本,一名蘼芜,根 
名靳芷。旧作 ,非。 


\section{}石苇
内容:味苦,平。主劳热邪气,五癃闭不通,利小便水道。一名石 。生山谷石上。 
《名医》曰∶一名石皮,生华阴山谷。不闻水及人声者,良。二月采叶,阴干。 


\section{}萆
内容:味苦,平。主腰背痛,强骨节,风寒湿、周痹,恶创不瘳,热气。生山谷。 
《名医》曰∶一名赤节。生真定。八月采根,曝干。 
案∶《博物志》云∶菝 ,与萆 相乱。 


\section{白薇}

原序录中作“微”,正文作“薇”,今据改。

味苦,平。主暴中风,身热肢满,忽忽不知人,狂惑,邪气,寒热酸 ,温疟洗洗,发 
作有时。生川谷。 
《名医》曰∶一名白幕,一名薇草,一名春草,一名骨美。生平原。三月三日 
采根,阴干。 

原序录中作“䘖”,疑为“衔”的异体,“御”的讹写,今据改。


\section{}水萍
内容:味辛,寒。主暴热身痒(《艺文类聚》《初学记》痒,此是),下水气,胜酒,长须发(《艺 
文类聚》作乌发),消渴。久服,轻身。一名水华(《艺文类聚》引云∶一名水廉)。生池泽。 
《吴普》曰∶水萍,一名水廉。生泽水上。叶员小,一茎一叶,根入水。五月花白,三 
月采,晒干(《御览》)。 
《名医》曰∶一始水白,一名水苏。生雷泽。三月采,曝干。 
案∶《说文》云∶苹, 也,无根,浮水而生者。萍,苹也。KT ,大萍也。《广雅》 
云∶ 
KT,萍也。《夏小正》云∶七月湟潦生苹。《尔雅》云∶萍, 。郭璞云∶水中浮萍,江 
东谓之KT 。又其大者, 。《毛诗》云∶于以采 。《传》云∶ ,大萍也。《范子计然》 
曰∶水萍,出三辅。色青者,善。《淮南子·原道训》云∶萍树根于水。高诱云∶萍,大 
也。 


\section{}王瓜
内容:味苦,寒。主消渴内痹瘀血,月闭,寒热,酸疼,益气,俞聋。一名土瓜。生平泽。 
《名医》曰∶生鲁地田野及人家坦墙间。三月采根,阴干。 
案∶《说文》云∶ ,王 也。《广雅》云∶葵茹、瓜 ,王瓜也。《夏小正》云∶四月 
王 
郭璞云∶钩, 也,一名王瓜,实如 瓜,正赤,味苦。《月令》∶王瓜生。郑元云∶《月 
令》云王 生。孔颖达云∶疑王 ,则王瓜也。《管子·地员篇》剽土之次曰五沙,其种大 
细 ,白茎青秀以蔓。《本草图经》云∶大 ,即王 也。芴,亦谓之土瓜,自 


\section{}地榆
内容:味苦,微寒。主妇人乳 痛,七伤、带下病,止痛,除恶肉,止汗,疗金创(《御览》 
上云∶主消酒。又云∶明目。《大观本草》消酒作黑字,而无明目)。生山谷。 
《名医》曰∶生桐柏及冤句。二月、八月采根,曝干。 
案∶《广雅》云∶ ,地榆也。陶弘景云∶叶似榆而长,初生布地,而花、子紫黑色, 
如豉,故名玉豉。 


\section{}海藻
内容:味苦,寒。主瘿瘤气,颈下核,破散结气、痈肿、症瘕、坚气,腹中上下鸣,下十二水 
肿,一名落首。生池泽。 
《名医》曰∶一名 。生东海。七月七日采,曝干。 
案∶《说文》云∶ ,水草也,或作藻。《广雅》云∶海萝,海藻也。《尔雅》云∶薅 
,海藻也。郭璞云∶药草也。一名海萝,如乱发,生海中。《本草》云∶又 ,石衣。郭璞 
云∶水苔也,一名石发,江东食之,或曰 。叶似KT 而大,生水底也,亦可食。 


\section{}泽兰
内容:味苦,微温。主乳妇内衄(《御览》作衄血),中风余疾,大腹水肿,身面四肢浮肿,骨 
节中水,金创、痈肿、创脓。一名虎兰,一名龙枣。生大泽傍。 
《吴普》曰∶泽兰,一名水香,神农、黄帝、岐伯、桐君∶酸,无毒;李氏∶温。生下 
地水傍。叶如兰,二月生,香,赤节,四叶相值枝节间。 
《名医》曰∶一名虎蒲。生汝南。三月三日采,阴干。 
案∶《广雅》云∶虎兰,泽兰也。 


\section{}防己
内容:味辛,平。主风寒温疟,热气诸痈,除邪,利大小便。一名解离(《御览》作石解引云∶ 
通腠理,利九窍。《大观本》六字黑)。生川谷。 
《吴普》曰∶木防己,一名解离,一名解燕。神农∶辛;黄帝、岐伯、桐君∶苦,无毒 
;李氏∶大寒。如艿,茎蔓延,如艽,白根外黄似结梗,内黑又如车辐解。二月、八月、十 
月采根(《御览》)。 
《名医》曰∶生汉中。二月、八月采根,阴干。 
案∶《范子计然》云∶防己,出汉中旬阳。 


\section{}款冬花
内容:味辛,温。主咳逆上气,善喘、喉痹,诸惊痫,寒热邪气。一名橐吾(《御览》作石), 
一名颗东(《御览》作颗冬),一名虎须,一名免奚。生山谷。 
《吴普》曰∶款冬,十二月花黄白(《艺文类聚》)。 
《名医》曰∶一名氏冬,生常山及目常水傍。十一月采花,阴干。 
案∶《广雅》云∶苦萃,款 也。《尔雅》云∶菟奚,颗 。郭璞云∶款冬也。紫赤 
,生水中。《西京杂记》云∶款冬,花于严冬。传咸《款冬赋》序曰∶仲冬之月,冰凌积雪 
,款冬独敷花艳。 


\section{}牡丹
内容:味苦辛,寒。主寒热,中风、螈 、痉,惊痫邪气,除症坚,瘀血留舍肠胃,安五脏, 
疗痈创。一名鹿韭,一名鼠姑。生山谷。 
《吴普》曰∶牡丹,神农、岐伯∶辛;李氏∶小寒;雷公、桐君∶苦,无毒;黄帝∶苦 
,有毒。叶如蓬相植,根如柏黑,中有核。二月采,八月采,晒干。人食之,轻身、益寿( 
《御览》)。 
《名医》曰∶生巴郡及汉中。二月、八月采根,阴干。 
案∶《广雅》云∶白术 


\section{}马先蒿
内容:味平。主寒热鬼注,中风湿痹,女子带下病,无子。一名马尿蒿。生川泽。 
《名医》曰∶生南阳。 
案∶《说文》云∶蔚,牡蒿也。《广雅》云∶因尘,马先也。《尔雅》云∶蔚,牡 。 
郭璞云∶无子者。《毛诗》云∶匪莪伊芳蔚。《传》云∶ ,牡 也。陆玑云∶三月始生;七 
月花,花似胡麻花而紫赤;八月为角,角似小豆,角锐而长。一名马新蒿。案∶新、先,声 
相近。 


\section{}积雪草
内容:味苦,寒。主大热,恶创,痈疽,浸淫赤 ,皮肤赤,身热。生川谷。 
《名医》曰∶生荆州。 
案∶陶弘景云∶荆楚人以叶如钱,谓为地钱草。徐仪《药图》名连钱草。《本草图经》 
云∶咸、洛二京亦有,或名胡薄荷。 


\section{}女菀
内容:(《御览》作苑) 
味辛,温。主风洗洗,霍乱泄利,肠鸣,上下无常处,惊痫,寒 
《吴普》曰∶女菀,一名白菀,一名识女苑(《御览》)。 
《名医》曰∶一名白菀,一名织女菀,一名茆。生汉中。正月、二月采,阴干。 
案∶《广雅》云∶女肠,女菀也。 


\section{}王孙
内容:味苦,平。主五脏邪气,寒湿痹,四肢疼酸,膝冷痛。生川谷。 
《吴普》曰∶黄孙,一名王孙,一名蔓延,一名公草,一名海孙。神农、雷公∶苦,无 
毒;黄帝∶甘,无毒。生西海山谷及汝南城郭垣下。蔓延,赤文,茎叶相当(《御览》)。 
《名医》曰∶吴,名白功草,楚,名王孙;齐,名长孙。一名黄孙,一名黄昏, 
一名海孙,一名蔓延,生海西及汝南城郭下。 
案∶陶弘景云∶今方家皆呼王昏,又云壮蒙。 


\section{}蜀羊泉
内容:味苦,微寒。主头秃恶创,热气疥,瘙痂,癣虫,疗龋齿。生川谷。 
《名医》曰∶一名羊泉,一名饴。生蜀郡。 
案∶《广雅》云∶ 姑,艾但鹿何,泽 


\section{}爵床
内容:味咸,寒。主腰脊痛,不得着床,俯仰艰难,除热,可作浴汤。生川谷及田野。 
《吴普》曰∶爵床,一名爵卿(《御览》)。 
《名医》曰∶生汉中。 
案∶别本注云∶今人名为香苏。 


\section{}假苏
内容:味辛,温。主寒热鼠 ,瘰 生创,破结聚气,下瘀血,除湿痹。一名鼠 。生川泽(旧 
在菜部,今移)。 
《吴普》曰∶假苏,一名鼠实,一名姜芥也(《御览》),名荆芥,叶似落藜而细,蜀 
中生啖之(《蜀本》注)。 
《名医》曰∶一名姜芥。生汉中。 
案∶陶弘景云∶即荆芥也,姜、荆,声讹耳。先居草部中。令人食之,录在菜部中也。 
味甘,寒,平(《御览》作味苦,平)。主下热气,益阴精,令人而说好,明目。久服, 
轻身、耐老。生平泽(旧在《唐本》退中,今移)。 
《吴普》曰∶翘根,神农、雷公∶甘,有毒。三月、八月采,以作蒸,饮酒病患(《御 
览》)。 
《名医》曰∶生蒿高,二月、八月采。 
案∶陶弘景云∶方药不复用,俗无识者。 
上草,中品四十九种。旧四十六种。考菜部假苏及《唐本》退中翘根,宜入此。 


\section{}桑根白皮
内容:味甘,寒。主伤中、五劳六极、羸瘦,崩中脉绝,补虚益气。叶∶主除寒热出汗。桑耳 
黑者∶主女子漏下赤白汗,血病,症瘕积聚,阴补,阴阳寒热,无子。五木耳名糯,益气、 
不饥、轻身、强志。生山谷。 
《名医》曰∶桑耳,一名桑菌,一名木麦。生犍为。六月多雨时采,即曝干。 
案∶《说文》云∶桑,蚕所食叶。木 ,木耳也。蕈,桑 。《尔雅》云∶桑瓣有葚栀 
。舍人云∶桑树,一半有葚,半无葚,名栀也。郭璞云∶瓣,半也。又女桑, 桑,郭璞云 
∶今俗呼桑树,小而条长者,为女桑树。又 山桑,郭璞云∶似桑材中作弓及草辕。又桑柳 
槐条,郭璞云∶阿那垂条。 


\section{}竹叶
内容:味苦,平。主咳逆上气溢,筋急恶疡,杀小虫。根∶作汤,益气止渴,补虚下气。汁∶ 
主风 。实∶通神明,轻身、益气。 
《名医》曰∶生益州。 
案∶《说文》云∶竹,冬生草也。象形,下 者,箬,箬也。 


\section{}吴茱萸
内容:(《御览》引无吴字,是) 
味辛,温。主温中,下气,止痛,咳逆,寒热,除湿、血痹,逐风邪,开凑(旧作腠, 
《御 
览》作涛,是)理。根∶杀三虫。一名 ,生山谷。 
《名医》曰∶生冤句。九月九日采,阴干。 
案∶《说文》云∶茱,茱萸,属。萸,茱萸也。煎,茱萸,《汉律》∶会稽献 一 
斗。《广雅》云∶ 、 、档、樾、茱萸也。《三苍》云∶ ,茱萸也(《御览》)。 
《尔雅》云∶椒、 、丑 。郭璞云∶茱萸子,聚生成房貌,今江东亦呼 ,似茱萸而小 
,赤色。《礼记》云∶三牲用 。郑云∶ 煎,茱萸也,《汉律》会稽献焉,《尔雅》谓之 
。《范子计然》云∶茱萸,出三辅。陶弘景云∶《礼记》名 ,而作俗中呼为 子。当是 
不识 字,似杂字,仍似相传。 


\section{}卮子
内容:(旧作栀,《艺文类聚》及《御览》引作支,是) 
味苦,寒。主五内邪气,胃中热气,面赤、酒泡、 鼻,白赖、赤癞,创疡。一名木丹。 
生川谷。 
《名医》曰∶一名樾桃。生南阳。九月采实,曝干。 
案∶《说文》云∶栀,黄木可染者。《广雅》云∶栀子, 桃也。《史记·货殖传》云 
∶巴蜀地饶卮。《集解》云∶徐广白∶音支,烟支也;紫,赤色也。据《说文》当为栀。 


\section{}芜荑
内容:味辛。主五内邪气,散皮肤骨节中淫淫温行毒,去三虫,化食。一名无姑,一名殿塘(《 
御览》引云∶逐寸白,散腹中温温喘息。《大观本》作黑字)。生川谷。 
《名医》曰∶一名殿塘。生晋山。三月采实,阴干。 
案∶《说文》云∶梗,山 榆,有束荚,可为芜荑者。《广雅》云∶山榆,母估也。《 
尔雅》云∶ 荑, 。郭璞云∶一名白蒉,又无姑,其实夷。郭璞云∶无姑,姑榆也。生 
山中,叶圆而浓,剥取皮合渍之,其味辛香,所谓芜荑。《范子计然》云∶芜荑在地,赤心 
者,善。 


\section{}枳实
内容:味苦,寒。主大风在皮肤中,如麻豆苦痒(《御览》作痰,非)。除寒热结,止利(旧作 
痢,《御览》作利,是)。长肌肉,利五脏,益气、轻身。生川泽。 
《吴普》曰∶枳实,苦。雷公∶酸,无毒;李氏∶大寒。九月、十月采,阴干(《御览 
《名医》曰∶生河内,九月、十月采,阴干。 
案∶《说文》云∶枳木似橘。《周礼》云∶橘渝淮而化为枳。沈括《笔谈》云∶六朝 
以前,医方唯有枳实,无枳壳,后人用枳之小、嫩者,为枳实;大者,为枳壳。 


\section{}浓朴
内容:味苦,温。主中风、伤寒、头痛、寒热,惊悸气,血痹死肌,去三虫。 
《吴普》曰∶浓朴,神农、岐伯、雷公∶苦,无毒;李氏∶小温(《御览》引云∶一名 
浓皮。生交址)。 
《名医》曰∶一名浓皮,一名赤朴。其树名榛,其子名逐。生交址冤句。九月、 
十月采皮,阴干。 
案∶《说文》云∶朴,木皮也,榛木也。《广雅》云∶重皮,浓朴也。《范子计然》云 
∶浓朴,出宏农,按∶今俗以榛为亲,不知是浓朴。《说文》榛栗,字作亲。 


\section{}秦皮
内容:味苦,微寒。主风寒湿痹,洗洗寒气,除热,目中青翳、白膜。久服,头不白、轻身。 
生川谷。 
《吴普》曰∶芩皮,一名秦皮。神农、雷公、黄帝、岐伯∶酸,无毒;李氏∶小,或生 
冤句水边。二月、八月采(《御览》)。 
《吴普》曰∶一名岑皮,一名石檀。生庐江及冤句。二月、八月采皮,阴干。 
案∶《说文》云∶ ,青皮木,或作KT 。《淮南子·倜真训》云∶ 木,色青翳。高 
诱 
云∶ 木,苦历木也。生于山,剥取其皮,以水浸之,正青,用洗眼,愈人目中肤翳。据《 
吴普》云∶岑皮,名秦皮,《本经》作秦皮者,后人以俗称改之,当为岑皮。 


\section{}秦菽
内容:味辛,温。主风邪气,温中,除寒痹,坚齿发、明目。久服,轻身、好颜色、耐老、增 
年、通神。生川谷。 
《名医》曰∶生太山及秦岭上,或琅邪。八月、九月采实。 
案∶《说文》云∶菽,菽 , 菽。菽 实裹如裘者,椒似茱萸,出《淮南》。《广雅 
云∶ 椒,茱萸也。《北山经》 
云∶ 
景山多秦椒。郭璞云∶子似椒而细叶草也。《尔雅》 
云∶ ,大椒。郭璞云∶今椒树丛生实大者,名为 。又椒 ,丑 。郭璞云∶ 萸子聚成 
房貌。今江示亦呼 KT ,似茱萸而小,赤色。《毛诗》云∶椒聊之实。《传》云∶椒聊, 
椒 
也。陆玑云∶椒树,似茱萸,有针刺,叶坚而滑泽,蜀人作茶,吴人作茗,皆合煮其叶以为 
香。《范子计然》云∶秦椒,出天水陇西,细者,善。《淮南子·人间训》云∶申椒、杜 
,美人之所怀服。旧作椒,非。据《山海经》有秦椒,生闻喜景山,则秦,非秦地之秦 
也。 
味酸,平。主心下邪气,寒热,温中,逐寒湿痹,去三虫。久服,轻身。一名蜀枣。生 
山谷。 
《吴普》曰∶山茱萸,一名魃实,一名鼠矢,一名鸡足。神农、黄帝、雷公、扁鹊∶酸, 
无毒;岐伯∶辛;一经∶酸。或生冤句、琅邪,或东海承县。叶如梅,有刺毛,二月,花如 
杏;四月,实如酸枣,赤;五月采实(《御览》)。 
《名医》曰∶一名鸡足,一名魃实,生汉中及琅邪、冤句、东海承县。九月、十月采实 
,阴干。 


\section{}紫葳
内容:味酸(《御览》作咸),微寒。主妇人产乳余疾,崩中,症瘕血闭,寒热羸瘦,养胎。生 
川谷。 
《吴普》曰∶紫葳,一名武威,一名瞿麦,一名陵居腹,一名鬼目,一名茏华。神农、 
雷公∶酸;岐伯∶辛;扁鹊∶苦、咸;黄帝∶甘,无毒。如麦根黑。正月、八月采。或生真 
定(《御览》)。 
《名医》曰∶一名陵苕,一名茏华。生西海及山阳。 
案∶《广雅》云∶茈葳,陵苕,蘧麦也。《尔雅》云∶苕,陵苕。郭璞云∶一名陵时。 
《本草》云∶又黄华,白华, 。郭璞云∶苕、花,色异,名亦不同。《毛诗》云∶苕 
之华。《传》云∶苕,陵苕也。《范子计然》云∶紫葳,出三辅。李当之云∶是瞿麦根。据 
李说与《广雅》合。而《唐本》注引《尔雅》注有一名凌霄四字,谓即凌霄花,陆玑以为鼠 
尾,疑皆非,故不采之。 


\section{}猪苓
内容:味甘,平。主 疟,解毒蛊注(《御览》作蛀)不祥,利水道。久服,轻身、耐老(《御 
览》作能老)。一名 猪尿。生山谷。 
《吴普》曰∶猪苓,神农,甘;雷公∶苦,无毒(《御览》引云∶如茯苓,或生冤句, 
八月采)。 
《名医》曰∶生衡山及济阴冤句。二月、八月采,阴干。 
案∶《庄子》云∶豕零。司马彪注作豕囊,云∶一名猪苓,根似猪卵,可以治渴。 


\section{}白棘
内容:味辛,寒。主心腹痛,痈肿渍脓,止痛。一名棘针。生川谷。 
《名医》曰∶一名棘刺。生雍州。 
案∶《说文》云∶棘,小枣丛生者。《尔雅》云∶髦颠棘。孙炎云∶一名白棘。李当之 
云∶此是酸枣树针,今人用天门冬苗代之,非是真也。案∶《经》云∶天门冬,一名颠勒。 
勒、棘,声相近,则今人用此,亦非无因也。 


\section{}龙眼
内容:味甘,平。主五脏邪气,安志厌食。久服,强魂、聪明、轻身、不老,通神明。一名益 
智。生山谷。 
《吴普》曰∶龙眼,一名益智。《要术》∶一名比目(《御览》)。 
《名医》曰∶其大者,似槟榔。生南海松树上。五月采,阴干。 
案∶《广雅》云∶益智,龙眼也。刘达注《吴都赋》云∶龙眼,如荔枝而小,圆如弹丸 
,味甘,胜荔枝,苍梧、交址、南海、合浦,皆献之,山中之家亦种之。 


\section{}松罗
内容:味苦,平。主嗔怒邪气,止虚汗、头风,女子阴寒、肿病。一名女萝。生山谷。 
《名医》曰∶生熊耳山。 
案∶《广雅》云∶女萝,松萝也。《毛诗》云∶茑与女萝。《传》云∶女萝、菟丝,松 
萝也。陆玑云∶松萝,自蔓松上,枝正青,与菟丝异。 


\section{}卫矛
内容:味苦,寒。主女子崩中下血,腹满汗出,除邪,杀鬼毒、虫注。一名鬼箭。生山谷。 
《吴普》曰∶鬼箭,一名卫矛。神农、黄帝、桐君∶苦,无毒。叶,如桃如羽,正月、 
二月、七月采,阴干。或生田野(《御览》)。 
《名医》曰∶生霍山。八月采,阴干。 
案∶《广雅》云∶鬼箭,神箭也。陶弘景云∶其茎有三羽,状如箭羽。 


\section{}合欢
内容:味甘,平。主安五脏,利心志(《艺文类聚》作和心志,《御览》作和心气)。令人献乐 
无忧。久服,轻身、明目、所欲。生山谷。 
《名医》曰∶生益州。 
案∶《唐本》注云∶或曰合昏,欢、昏,音相近。《日华子》云∶夜合。 
上木,中品一十七种,旧同。 


\section{}白马茎
内容:味咸,平。主伤中脉绝,阴不起,强志益气,长肌肉,肥健,生子。眼∶主惊痫,腹满, 
疟疾,当杀用之。悬蹄∶主惊邪,螈 ,乳难,辟恶气、鬼毒、蛊注、不祥。生 
《名医》曰∶生云中。 


\section{}鹿茸
内容:味甘,温,主漏下恶血,寒热,惊痫,益气强志,生齿不老。角,主恶创痈肿,逐邪恶 
气,留血在阴中。 
《名医》曰∶茸,四月、五月解角时取,阴干,使时躁。角,七月采。 


\section{}牛角
内容:下闭 
血,瘀血疼痛,女人带下血。髓∶补中,填骨髓。久服,增年。胆∶可 
案∶《说文》云∶ ,角中骨也。 


\section{}羚羊角
内容:味咸,温。主青盲,明目,杀疥虫,止寒泄,辟恶鬼虎野狼,止惊悸。久服,安心、益气、 
轻身。生川谷。 
《名医》曰∶生河西。取无时。 
案∶《说文》云∶羚夏羊。牝,曰羚。《尔雅》云∶羊牝,羚。郭璞云∶今人便以 、 
羚,为黑白羊名。 


\section{}牡狗阴茎
内容:味咸,平。主伤中,阴痿不起,令强、热、大、生子,除女子带下十二疾。一名狗精。 
胆∶主明目。 
《名医》曰∶六月上伏取,阴干百日。 


\section{}羚羊角
内容:味咸,寒。主明目,益气,起阴,去恶血注下,辟蛊毒恶鬼不祥,安心气,常不大厌寐。 
生川谷。 
《名医》曰∶生石城及华阴山,采无时。 
案∶《说文》云∶羚,大羊而细角。《广雅》云∶美皮,冷角。《尔雅》云;弘,大羊 
。郭璞云∶羚羊,似羊而大,角园锐,好在山崖间。陶弘景云∶《尔雅》名 羊。据《说 
文》云∶苋,山羊细角也。《尔雅》云∶ ,如羊。郭璞云∶ ,似吴羊而大角。角椭,出 
西方。苋,即 正字。然《本经》羚字,实羚字俗写,当以羚为是。《尔雅》释文引本草 
,作羚。 


\section{}犀角
内容:味苦,寒。主百毒虫注,邪鬼、障气,杀钩吻、鸩羽、蛇毒,除不迷或厌寐。久服轻血。 
生山谷。 
《名医》曰∶生永昌及益州。 
案∶《说文》云∶犀,南徼外牛,一角在鼻,一角在顶,似豕。《尔雅》云∶犀,似 
豕。郭璞云∶形似水牛,猪头大腹;痹脚,脚有三蹄,黑色;三角,一在顶上,一在鼻上, 
一在额上。鼻上者,即食角也。小而不椭,好食棘。亦有一角者。《山海经》云∶琴鼓之山, 
多白犀。郭璞云∶此与辟寒、蠲忿、辟尘、辟暑诸犀,皆异种也。《范子计然》云∶犀角, 
出南郡,上价八千,中三千,下一千。 
上兽,中品七种。旧同。 


\section{}燕屎
内容:味辛,平。主蛊毒鬼注,逐不祥邪气,破五癃,利小便。生平谷。 
《名医》曰∶生高山。 
案∶《说文》云∶燕,元鸟也。尔口,布翅,枝尾,象形。作巢,避戊己,乙元鸟也。 
齐鲁谓之乙,取其名自呼,象形。或作乱。《尔雅》云∶燕乱。《夏小正》云∶二月来降, 
燕乃睇。《传》云∶燕,乙也,九月陟元鸟,蛰。《传》云∶元鸟者,燕也。 


\section{}天鼠屎
内容:味辛,寒。主面痈肿,皮肤洗洗时痛,肠中血气,破寒热积聚,除惊悸。一名鼠 ,一 
名石肝。生山谷。 
《名医》曰∶生合浦。十月、十二月取。 
案∶李当之云∶即伏翼屎也。李云∶天鼠,《方言》一名仙鼠。 
案∶今本《方言》云∶或谓之老鼠,当为天字之误也。 
上禽,中品二种。旧同。 


\section{}猥皮
内容:味苦,平。主五痔阴蚀,下血赤白五色,血汁不止,阴肿、痛引要背。酒煮杀之。生川 
谷。 
《名医》曰∶生楚山田野。取无时。 
案∶《说文》云∶KT ,似豪猪者,或作猥。《广雅》云∶虎王,猥也。《尔雅》云∶ 
汇,毛刺。郭璞云∶今谓状似鼠。《淮南子·说山训》云∶鹊矢中猥。 


\section{}露蜂房
内容:味苦,平。主惊痫 ,寒热邪气, 疾,鬼精蛊毒,肠痔。火熬之,良。一名蜂肠。 
生山谷。 
《名医》曰∶一名百穿,一名蜂 。生 柯。七月七日采,阴干。 
案∶《淮南子·汜论训》云∶蜂房不容卵。高诱云∶房巢也。 


\section{}鳖甲
内容:味咸,平。主心腹症瘕坚积、寒热,去痞、息肉、阴蚀、痔、恶肉。生池泽。 
《名医》曰∶生丹阳,取无时。 
案∶《说文》云∶鳖,甲虫也。 


\section{}蟹
内容:味咸,寒。主胸中邪气、热结痛, 僻而肿。败漆烧之,致鼠。生池泽。 
《名医》曰∶生伊芳洛诸水中,取无时。 
案∶《说文》云∶蟹,有二敖八足旁行,非蛇鳝之穴无所庇。或作 , 蟹也。《荀子 
·勤学扁》云∶蟹,六跪而二螯,非蛇 之穴无所寄托。《广雅》云∶晡蟹, 也。《 
尔雅》云∶ ,小者 。郭璞云∶或曰即 也,似蟹而小。 


\section{}柞蝉
内容:味咸,寒。主小儿惊痫、夜啼, 病,寒热。生杨柳上。 
《名医》曰∶五月采,蒸干之。 
案∶《说文》云∶蝉以旁鸣者,蜩蝉也。《广雅》云∶KT ,蝉也;复育,蜕也。旧 
作蚱蝉。《别录》云∶蚱者,鸣蝉也。壳,一名 蝉。又名伏 ,案∶蚱,即柞字。《周礼·考 
工记》云∶侈,则柞。郑元云∶柞,读为咋咋然之咋,声大外也。《说文》云∶诸,大声也, 
音同柞,今据作柞。柞蝉,即五月鸣蜩之蜩。《夏小正》云∶五月良蜩鸣。《传》∶良蜩也, 
五采具。《尔雅》云∶蜩,螂蜩。《毛诗》云∶如蜩。《传》云∶蜩,蝉也。《方言》云∶楚, 
谓之蜩;宋、卫之间,谓之螗蜩;陈郑之间,谓之螂蜩;秦、晋之间,谓之蝉;海岱之间, 
谓之KT 。《论衡》云∶蝉,生于复育,开背而出。而《玉篇》云∶蚱蝉,七月生。陶弘景∶ 
音蚱作笮,云 蝉,是为《月令》人寒蝉,《尔雅》所云 矣,《唐本》注非之也。 


\section{}蛴螬
内容:味咸,微温。主恶血、血瘀(《御览》作血瘴),痹气,破折、血在胁下坚满痛,月闭, 
目中淫肤,青翳白膜。一名 蛴。生平泽。 
《名医》曰∶一名 齐,一名勃齐。生河内人家积粪草中。取无时。反行者,良。 
案∶《说文》云 , 螬也;蝤,蝤 也;蝎,蝤 也。《广雅》云∶蛭 , ,地蚕, 
蠹,蛴螬。《尔雅》云∶ ,蛴螬。郭璞云∶在粪土中。又蝤蛴,蝎。郭璞云∶在木中。今 
虽通名蝎,所在异。又蝎, 。郭璞云∶木中囊虫。蝎,桑蠹,郭璞云∶即拮掘。《毛诗》 
云∶领如蝤蛴。《传》云∶蝤蛴,蝎虫也。《方言》云∶蛴螬,谓之 。自关而东,谓之蝤蛴, 
或谓之蚕 ,梁益之间,谓之 ,或谓之蝎,或谓之蛭 ;秦晋之间,谓之蠹,或谓之天蝼。 
《列子·天瑞篇》云∶乌足根为蛴螬。《博物志》云∶蛴螬以背行,快于足用。《说文》无 
字,当借 为之。声相近,字之误也。 


\section{}乌贼鱼骨
内容:味咸,微温。主女子漏下赤白经汁,血闭,阴蚀肿痛,寒热症瘕,无子。生池泽。 
《名医》曰∶生东海。取无时。 
案∶《说文》云∶ ,乌 ,鱼名,或作鲫。《左思赋》有乌贼。刘逵注云∶乌贼鱼, 
腹中有墨。陶弘景云∶此是 乌所化作,今其口脚具存,犹相似尔。 


\section{}白僵蚕
内容:味咸。主小儿惊痫夜啼,去三虫,减黑 ,令人面色好,男子阴疡病。生平泽。 
《名医》曰∶生 川。四月取自死者。 
案∶《说文》云∶蚕任丝也。《淮南子·说林训》云∶蚕,食而不饮,二十二日而化。 
《博物志》云∶蚕三化,先孕而后交。不交者,亦生子,子后为 ,皆无眉目,易伤,收采 
亦薄。《玉篇》作僵蚕,正当为僵,旧作 ,非。 


\section{}鱼甲
内容:味辛,微温,主心腹症瘕、伏坚、积聚、寒热,女子崩中,下血五色,小腹阴中相引痛, 
创疥死肌。生池泽。 
《名医》曰∶生南海。取无时。 
案∶《说文》云∶鳝,鱼石,皮可为鼓。鼍蜥,水虫似蜥,易长大。陶弘景云∶蛇,即 
鼍甲也。 


\section{}樗鸡
内容:味苦,平,主心腹邪气,阴痿,益精强志,生子,好色、补中、轻身。生川谷。 
《名医》曰∶生河内樗树上。七月采,曝干。 
案∶《广雅》云∶樗鸠,樗鸡也。《尔雅》云∶ ,天鸡。李巡云∶一名酸鸡。郭璞云 
∶小虫,黑身赤头,一名莎鸡,又曰樗鸡。《毛诗》云∶六月莎鸡振羽。陆玑云∶莎鸡,如 
蝗而斑色,毛翅数重,某翅正赤,或谓之天鸡。六月中,飞而振羽,索索作声。幽州人谓之 
蒲错,是也。 


\section{}蛞蝓
内容:味咸,寒。主贼风 僻,轶筋及脱肛,惊痫挛缩。一名陵蠡。生池泽。 
《名医》曰∶一名土 ,一名附 。生大山及阴地沙石垣下。八月取。 
案∶《说文》云∶蝓,虎蝓也。蠃,一名虎蝓。《广雅》云∶蠡蠃,蜗牛, 蝓也。《 
中山经》云∶青要之山,是多仆累。郭璞云∶仆累,蜗牛也。《周礼》鳖人,祭祀供蠃。郑 
云∶蠃, 蝓。《尔雅》云∶ 蠃, 蝓。郭璞云∶即蜗牛也。《名医》曰∶别出蜗牛条,非。 
旧作蛞,《说文》云所无。据《玉篇》云∶蛞,蛞东,知即活东异文,然则当为活。 


\section{}石龙子
内容:味咸,寒。主五癃邪结气,破石淋,下血,利小便水道。一名蜥易,生川谷。 
《吴普》曰∶石龙子,一名守宫,一名石蜴,一名石龙子(《御览》)。 
《名医》曰∶一名山龙子,一名守宫,一名石蜴。生平阳及荆山石间。五月取,着石上, 
令干。 
案∶《说文》云∶蜥,虫之蜥蜴也。易,蜥易, 蜓,守宫也,象形。 ,在壁,曰 
蜓;在草,曰蜥易,或作 、荣 蛇∶医以注鸣者。《广雅》云∶蛤解, ,蚵 ,蜥 
蜴也。《尔雅》云∶蝾螈,蜥蜴;蜥蜴, 蜓;蜓 ,守宫也。《毛诗》云∶胡为虺蜴。《传》 
云∶蜴,螈也。陆玑云∶虺蜴,一名蝾螈,蜴也,或谓之蛇医,如蜥蜴,青绿色,大如指, 
形状可恶。《方言》云∶守宫,秦晋、西夏谓之守宫,或谓之 ,或谓之蜥易,其在泽中 
者,谓之易蜴;南楚谓之蛇医,或谓之蝾螈;东齐、海岱谓之 ;北燕谓之祝蜓;桂林之 
中,守宫大者而能鸣,谓之蛤蚧。 


\section{}木虻
内容:味苦,平。主目赤痛, 伤泪出,瘀血血闭,寒热酸 ,无子。一名魂常。生川泽。 
《名医》曰∶生汉中。五月取。 
案∶《说文》云∶虻,啮人飞虫。《广雅》云∶ ,虻也,此省文。《淮南子·齐俗训》 
云∶水虿,为 芒。高诱云∶青蛉也。又《说文训》云∶虻,散积血。 


\section{}蜚虻
内容:味苦,微寒。主逐瘀血,破下血积、坚痞症瘕、寒热,通利血脉及九窍。生川谷。 
《名医》曰∶生江夏。五月取。腹有血者,良。 


\section{}蜚廉
内容:味咸,寒。主血瘀(《御览》引云∶逐下血)、症坚、寒热,破积聚,喉咽痹,内寒,无 
子。生川泽。 
《吴普》曰∶蜚廉虫,神农、黄帝云∶治妇人寒热(《御览》)。 
《名医》曰∶生晋阳及人家屋间。立秋采。 
案∶《说文》云∶ ,卢, 也。蜚,臭虫,负 也。 ,KT 也。《广雅》云∶飞 
, 
飞蠊也。《尔雅》云∶蜚, 。郭璞云∶即负盘臭虫。《唐本》注云∶汉中人食之下气,名 
曰石姜,一名卢 ,一名负盘,占作蠊。据邢 疏引此作廉。 


\section{}虫
内容:味咸,寒。主心腹寒热洗洗,血积症瘕,破坚,下血闭,生子大良。一名土鳖。生川泽。 
《名医》曰∶一名土鳖。生河东及沙中、人家墙壁下、土中湿处。十月,曝干。 
案∶《说文》云∶ 虫,属 ,目 也。《广雅》云∶负 , 也。《尔雅》云∶草 
虫,负 。郭璞云∶常羊也。《毛诗》云∶ 草虫。《传》云∶草虫,常羊也。陆玑云∶ 
小大长短如蝗也。奇音,青色,好在茅草中。 


\section{}伏翼
内容:味咸,平。主目瞑,明目,夜视有精光。久服,令人喜乐,媚好无忧。一名蝙蝠。生川 
谷(旧部禽部,今移)。 
《吴普》曰∶伏翼,或生人家屋间。立夏后,阴干,治目冥,令人夜视有光(《艺文类 
聚》)。 
《名医》曰∶生太山及人家屋间。立夏后采,阴干。 
案∶《说文》云∶蝙,蝙蝠也;蝠,蝙蝠,服翼也。《广雅》云∶伏翼,飞鼠,仙鼠, 
也。《尔雅》云∶蝙蝠,服翼。《方言》云∶蝙蝠,自关而东,谓之伏翼,或谓之飞鼠 
,或谓之老鼠,或谓之仙鼠;自关而西,秦陇之间,谓之蝙蝠;北燕谓之 。李当之云∶ 
即天鼠。 
上虫、鱼,中品一十七种。旧十六种。考禽部伏翼宜入此。 


\section{}梅实
内容:味酸,平。主下气,除热、烦、满,安心,肢体痛,偏枯不仁,死肌,去青黑志,恶疾。 
生川谷。 
《吴普》曰∶梅实(《大观本草》作核),明目,益气(《御览》)、不饥(《大观本草》 
引《吴氏本草》)。 
《名医》曰∶生汉中。五月采,火干。 
案∶《说文》云∶KT ,干梅之属,或作KT 。某,酸果也。以梅为楠。《尔雅》云∶梅楠 
。郭璞云∶似杏,实酢,是以某注梅也。《周礼》∶笾人馈食,笾,其实干KT 。郑云∶干 
KT 
,干梅也。有桃诸、梅诸,是其干者。《毛诗》疏云∶梅暴为腊,羹 中,人含之,以香 
口(《大观本草》)。 
上果,中品一种。旧同。 


\section{}大黄豆卷
内容:味甘,平。主湿痹,筋挛,膝痛。 
生大豆 涂痈肿,煮汁饮,杀鬼毒,止痛。 
赤小豆 主下水,排痈肿脓血。生平泽。 
《吴普》曰∶大豆黄卷,神农、黄帝、雷公∶无毒。采无时。去面 。得前胡、乌啄 
杏子、牡蛎、天雄、鼠屎,共蜜和,佳。不欲海藻、龙胆。此法,大豆初出黄土芽是也。生 
大豆,神农、岐伯∶生、熟,寒。九月采。杀乌豆毒,并不用元参。赤小豆,神农、黄帝 
∶咸;雷公∶甘。九月采(《御览》)。 
《名医》曰∶生大山。九月采。 
案∶《说文》云∶椒,豆也,象豆生之形也; ,小椒也。藿,椒之少也。《广雅》云∶ 
大豆,椒也;小豆, 也;豆角,谓之荚;其叶,谓之藿。《尔雅》云∶戎叔,谓之荏叔。 
孙炎支∶大豆也。 


\section{}粟米
内容:味咸,微寒。主养肾气,去胃、脾中热,益气。陈者,味苦,主胃热,消渴,利小便(《大 
观本草》作黑字,据《吴普》增)。 
《吴普》曰∶陈粟,神农、黄帝∶苦,无毒。治脾热、渴。粟,养肾气(《御览》)。 
案∶《说文》云∶粟,嘉谷实也。孙炎注《尔雅》粢稷云∶粟也,今关中人呼小米为粟 
米,是。 


\section{}黍米
内容:味甘,温。主益气补中,多热、令人烦(《大观本》作黑字,据《吴普》增)。 
《吴普》曰∶黍,神农∶甘,无毒。七月取,阴干。益中补气(《御览》)。 
案∶《说文》云∶黍,禾属而粘者。以大暑而种,故谓之黍。孔子曰∶黍,可为酒,禾 
入水也。《广雅》云∶粢,黍稻,其采谓之禾。《齐氏要术》引记胜之书曰∶黍,忌丑。又 
曰∶黍,生于巳,壮于酉,长于戌,老于亥,死于丑,恶于丙午,忌于丑寅卯。按∶黍,即 
糜之种也。 
上米、谷,中品三种。旧二种,大、小豆为二,无粟米、黍米。今增。 


\section{}蓼实
内容:味辛,温。主明目温中,耐风寒,下水气,面目浮肿,痈疡。马蓼,去肠中蛭虫,轻身。 
生川泽。 
《吴普》曰∶蓼实,一名天蓼,一名野蓼,一名泽蓼(《艺文类聚》)。 
《名医》曰∶生雷泽。 
案∶《说文》云∶蓼,辛菜,蔷虞也。蔷,蔷虞,蓼。《广雅》云∶荭,茏, ,马蓼孔。 
《尔雅》云∶墙虞,蓼。郭璞云∶虞蓼,泽萝。又荭,茏古。其大者,归。郭璞云∶俗呼荭 
草为茏鼓,语转耳。《毛诗》云∶隰有游龙。《传》云∶龙,红草也。陆玑云∶一名马蓼,叶 
大而赤色,生 


\section{}葱实
内容:味辛,温。主明目,补中不足。其茎可作汤,主伤寒寒热,出汗,中风面目肿。 


\section{}KT
内容:味辛,温。主金创,创败,轻身、不饥、耐老。生平泽。 
《名医》曰∶生鲁山。 
案∶《说文》云∶KT ,菜也,叶似韭。《广雅》云∶韭,KT ,荞,其华谓之菁。《尔 
雅》云∶ 
KT鸿荟。郭璞云∶即KT 菜也。又,劲山贲。陶弘景云∶葱、KT 异物,而今共条。《本 
经》既 
无韭,以其同类,故也。 
味辛,微温。主下气,辟口臭,去毒,辟恶。久服,通神明、轻身、耐老。生池泽。 
《吴普》曰∶芥 ,一名水苏,一名劳祖(《御览》)。 
《名医》曰∶一名鸡苏,一名劳祖,一名芥 ,一名芥苴。生九真,七月采。 
案∶《说文》云∶苏,桂荏也。《广雅》云∶芥 ,水苏也。《尔雅》云∶苏,桂,荏。 
郭璞云∶苏,荏类,故名桂荏。《方言》云∶苏,亦荏也。关之东西,或谓之苏,或谓 
之荏; 
周郑之间,谓之公贲;沅湘之南,谓之 ,其小者,谓之KT 柔。按∶KT 柔,即香薷也。 
亦名香 。《名医》别出香薷条,非。今紫苏、薄荷等,皆苏类也。《名医》俱别出之。 
上菜,中品三种。旧四种,考葱实,宜与KT 同条,今并假苏,宜入草部。 

\chapter{下品}

\begin{yuanwen}


锻石 矾石 铅丹 粉锡锡镜鼻 代赭 戎盐大盐 卤盐 白垩 冬灰 青琅 ( 
上玉、石,下品八种。旧一十二种) 附子 乌头 天雄 半夏 虎掌 鸢尾 大黄 葶苈 
桔梗 莨荡子 草蒿 旋复花 藜芦 钩吻 射干 蛇合 恒山 蜀漆 甘遂 白蔹 青 
葙子 藿菌 白芨 大戟 泽漆 茵芋 贯众 荛花 牙子 羊踯躅 商陆 羊蹄 蓄 
野狼毒 白头翁 鬼臼 羊桃 女青 连翘 闾茹 乌韭 鹿藿 蚤休 石长生 陆英 荩草 
牛 夏枯草 芫花(上草,下品四十九种。旧四十八种) 巴豆 蜀椒 皂荚 柳花 楝 
实郁李仁 莽草 雷丸 桐叶 梓白皮 石南 黄环 溲疏 鼠李 药实根 栾花 蔓椒 
(上木,下品一十七种。旧一十八种) 豚卵 麇脂 鼠 六畜毛蹄甲(上兽,下品四种。 
旧同) 蛤蟆 马刀 蛇蜕 蚯蚓 蜈蚣 水蛭 班苗 贝子 石蚕 雀瓮 蜣螂 
蝼蛄 马陆 地胆 鼠妇 荧火 衣鱼(上虫、鱼,下品一十九种。旧一十八种)桃核仁 杏 
核仁(上,下品三种。旧同)腐婢(上米、谷,下品一种。旧同)苦瓠 水靳(上菜,下品 
二种。旧同) 彼子(上一 
种,未祥) 
附《吴普本草》 
\end{yuanwen}

\section{}锻石
内容:味辛,温。主疽疡、疥瘙、热气,恶创、癞疾、死肌,堕眉,杀痔虫,去黑子息肉。一 
名恶 。生山谷。 
《名医》曰∶一名希 。生中山。 
按∶恶灰,疑当为垩灰。希、石,声之缓急。 


\section{}石
内容:味辛,大热。主寒热,鼠 蚀创,死肌,风痹,腹中坚。一名青分石,一名立制石,一 
名固羊石(《御览》引云∶除热,杀百兽。《大观 
《吴普》曰∶白巩石,一名鼠乡。神农、岐伯∶辛,有毒;桐 
君∶有毒 
名太白,一名泽乳,一名食盐。又云∶李氏∶大寒,主温热)。 
《名医》曰∶一名白巩石,一名太白石,一名泽乳,一名食盐 
案∶《说文》云∶巩,毒石也,出汉中。《西山经》云∶皋涂之山,有白石焉,其名曰巩, 
可以毒鼠。 
《范子计然》云∶巩石,出汉中。色白者,善。《淮南子·地形训》云∶白天,九百岁, 
生 
白巩。高诱云∶白巩,巩石也。又《说林训》云∶人,食巩石而死;蚕,食之而肥。高诱云∶ 
出阴山。一曰能杀鼠。案∶《 


\section{}铅丹
内容:味辛,微寒,主土逆胃反,惊痫 疾,除热下气。炼化还成九光。久服,通神明(《御 
览》引作吐下,云久服成仙)。生平泽 
《名医》曰∶一名铅华。生蜀郡。 
案∶《说文》云∶铅,青金也。陶弘景云∶即今熬铅所作黄丹也。 


\section{}粉锡
内容:味辛,寒。主伏尸毒螫,杀三虫。一名解锡。锡镜鼻∶主女子血闭,症瘕,伏肠,绝孕。 
生山谷(旧作二种,今并)。 
《名医》曰∶生桂阳。 
案∶《说文》云∶锡,银、铅之间也。 


\section{}代赭
内容:味苦,寒。主鬼注、贼风、蛊毒,杀精物恶鬼,腹中毒邪气,女子赤沃漏下。一名须丸。 
生山谷。 
《名医》曰∶一名血师,生齐国,赤红青色如鸡冠,有泽。染爪甲,不渝者,良。采无 
时。 
案∶《说文》云∶赭,赤土也。《北山经》云∶少阳之山,其中多美赭。《管子·地数篇》 
云∶山上有赭者,其下有铁。《范子计然》云∶石赭,出齐郡,赤色者,善;蜀赭,出蜀郡。 
据《元和郡县志 


\section{}戎盐
内容:主明目、目痛,益气、坚肌骨,去毒蛊。大盐∶令人吐(《御览》引云∶主肠胃结热。《大 
观本》作黑字)。卤盐∶味苦,寒,主大热,消渴狂烦,除邪及下蛊毒,柔肌肤(《御览》引 
云∶一名寒石,明目益气)。生池泽(旧作三种,今并)。 
《名医》曰∶戎盐,一名胡盐。生胡盐山,及西羌、北地、酒泉、福禄城东南角。北海, 
青;南海,赤。十月采。大盐,生邯郸,又河东。卤盐,生河东盐池。 
案∶《说文》云∶盐,咸也。古者宿沙初作煮海盐。卤,西方咸地也。从西省,象盐形 
,安定有卤县。东方,谓之斥;西方,谓之卤盐。河东盐池,袤五十一里,广七里,周百十 
六里。《北山经》云∶景山南望盐贩之泽。郭璞云∶即解县盐池也。今在河东猗氏县。案∶ 
在山西安邑运城。 


\section{}白垩
内容:味苦,温。主女子寒热、症瘕、目闭、积聚。生山谷。 
《吴普》曰∶白垩,一名白 (《一切经音义》)。 
《名医》曰∶一名白善。生邯郸。采无时。 
案∶《说文》云∶垩,白涂也。《中山经》云∶葱聋之山,是多白垩。 


\section{}冬灰
内容:味辛,微温。主黑子,去疣、息肉、疽蚀、疥瘙。一名藜灰。生川泽。 
《名医》曰∶生方谷。 


\section{}青琅
内容:味辛,平。主身痒、火创,痈伤、疥瘙、死肌。一名石珠。生平泽。 
《名医》曰∶一名青珠,生蜀郡,采无时。 
案∶《说文》云∶琅 ,似珠者,古文作。《禹贡》云∶雍州贡与KT 琳琅 。郑云∶ 
琅 
上玉、石,下品九种。旧十二种,粉锡、锡镜鼻为二,戎盐、大盐、卤盐为非,三考 
当各为一。 


\section{}附子
内容:味辛,温。主风寒咳逆邪气,温中,金创,破症坚积聚,血瘕,寒湿, (《御览》作 
痿) 拘挛,脚痛不能行步(《御览》引云∶为百药之长。《大观本》作黑字)。生山 
《吴普》曰∶附子,一名莨,神农∶辛;岐伯、雷公∶甘,有毒;李氏∶苦,有毒,大 
温。或生广汉。八月采。皮黑,肥白(《御览》)。 
《名医》曰∶生楗为及广汉东。月采,为附子;春采,为乌头(《御览》)。 
案∶《范子计然》云∶附子,出蜀武都中。白色者,善。 


\section{}乌头
内容:味辛,温。主中内、恶风洗洗,出汗,除寒湿痹,咳逆上气,破积聚、寒热。其汁,煎 
之,名射罔,杀禽兽。一名奚毒,一名即子,一名乌喙。生山谷。 
《吴普》曰∶乌头,一名莨,一名千狄,一名毒公,一名卑负(《御览》作果负),一名 
耿子。神农、雷公、桐君、黄帝∶甘,有毒。正月始生,叶浓,茎方,中空,叶四四相当, 
与蒿相似。又云∶乌喙,神农、雷公、桐君、黄帝∶有毒;李氏∶小寒,十月采,形如乌 
头,有两岐相合,如乌之喙,名曰乌喙也。所畏、恶、使,尽与乌头同。一名 子,一名莨 
。神农、岐伯∶有大毒;李氏∶大寒。八月采,阴干。是附子角之大者,畏、恶与附子同( 
《御览》。 
《名医》曰∶生郎陵。正月、二月采,阴干。长三寸以上,为天雄。 
按∶《说文》云∶ ,乌喙也。《尔雅》云∶芨,堇草。郭璞云∶即乌头也,江东呼为堇。 
《范子计然》云∶乌头,出三辅中,白者,善。《国语》云∶骊姬置堇于肉。韦昭云∶堇, 
乌头也。《淮南子·主术训》云∶莫凶于鸡毒,高诱云∶鸡毒,乌头也。按∶鸡毒,即奚毒; 
即子,即 


\section{}天雄
内容:味辛,温,主大风,寒湿痹,沥节痛,拘挛缓急,破积聚,邪气,金创,强筋骨,轻身 
健行。一名白幕(《御览》引云∶长阴气,强志,令人武勇,力作不倦。《大观本》作黑字)。 
生山谷。 
《名医》曰∶生少室。二月采根,阴干。 
案∶《广雅》云∶ ,奚毒,附子也。一岁,为 子;二岁,为乌喙;三岁,为附子; 
四岁,为乌头;五岁,为天雄。《淮南子·缪称训》云∶天雄,乌喙,药之凶毒也。良医以 
活 


\section{}半夏
内容:味辛,平。主伤寒寒热,心下坚,下气,喉咽肿痛,头眩胸胀,咳逆肠鸣,止汗。一名 
地文, 
一名水玉(以上八字,原本黑字)。生川谷。 
《吴普》曰∶半夏,一名和姑,生微邱,或生野中。叶三三相偶,二月始生,白花、员, 
上(《御览》)。 
《名医》曰∶一名示姑。生槐里。五月、八月采根,曝干。 
案∶《月令》云∶二月半夏生。《范子计然》云∶半夏,出三辅。色白者善。《列仙传》 
云∶赤松子服水玉以教神农。疑即半夏别名。 


\section{}虎掌
内容:味苦,温。主心痛寒热,结气、积聚、伏梁,伤筋、痿、拘缓,利水道。生山谷。 
《吴普》曰∶虎掌,神农、雷公∶苦,无毒;岐伯、桐君∶辛,有毒。立秋九月 
采之(《御览》引云∶或生太山,或宛朐)。 
《名医》曰∶生汉中及冤句。二月、八月采,阴干。 
案∶《广雅》云∶虎掌,瓜属也。 


\section{}鸢尾
内容:味苦,平。主蛊毒邪气,鬼注,诸毒,破症瘕积聚,去水,下三虫。生山谷。 
《吴普》曰∶鸢尾,治蛊毒(《御览》)。 
《名医》曰∶一名乌园。生九嶷山。五月采。 
案∶《广雅》云∶鸢尾,乌 ,射干也(疑当作鸢尾,乌园也;乌 ,射干也。是二物) 
。《唐本》注云∶与射干全别。 


\section{}大黄
内容:味苦,寒。主下瘀血、血闭、寒热,破症瘕积聚,留饮宿食,荡涤肠胃,推陈致新,通 
利水杀(《御览》,此下有道字),调中化食,安和五脏,生山谷。 
《吴普》曰∶大黄,一名黄良,一名火参,一名肤如,神农、雷公∶苦,有毒;扁鹊∶ 
苦,无毒;李氏∶小寒,为中将军。或生蜀郡北部,或陇西。二月花生,生黄赤叶,四四相 
当,黄茎高三尺许;三月,花黄;五月,实黑。三月采根,根有黄汁,切,阴干(《御览》)。 
《名医》曰∶一名黄良,生河西及陇西。二月、八月采根,火干。 
案∶《广雅》云∶黄良,大黄也。 


\section{}亭历
内容:(旧作葶苈,《御览》作亭历) 
味辛,寒,主症瘕、积聚、结气,饮食、寒热,破坚。一名大室,一名大适。生平泽及 
田野。 
《名医》曰∶一名下历,一名 蒿。生 城。立夏后,采实阴干。得酒,良。 
案∶《说文》云∶ ,亭历也。《广雅》云∶狗荠、大室,亭苈也。《尔雅》云∶ ,亭 
历。郭璞云∶实、叶皆似芥,《淮南子·缪称训》云∶亭历愈张。《西京杂记》云∶亭历 
于盛夏。 


\section{}桔梗
内容:味辛,微温。主胸胁痛如刀刺,腹满,肠鸣幽幽,惊恐悸气(《御览》引云∶一 
名利如。《大观本》作黑字)。生山谷。 
《吴普》曰∶桔梗,一名符扈,一名白芍,一名利如,一名梗草,一名卢如。神农、医 
和∶苦,无毒;扁鹊、黄帝∶咸;岐伯、雷公∶甘,无毒;李氏∶大寒。叶如荠 ,茎如笔 
管,紫赤。二月生(《御览》)。 
《名医》曰∶一名利如,一名房图,一名白药,一名梗草,一名荠 。生蒿高及冤句。 
二、八月采根,曝干。 
案∶《说文》云∶桔,桔梗,药名。《广雅》云∶犁如,桔梗也。《战国策》云∶今求 
柴胡,及之睾黍梁父之阴,则 车而载耳。桔梗于沮泽,则累世不得一焉,《尔雅》云∶ , 
。郭璞云∶荠 


\section{}莨荡子
内容:味苦,寒。主齿痛出虫,肉痹拘急,使人健行,见鬼。多食,令人狂走。久服,轻身、 
走及奔马、强志、益力、通神。一名横唐。生川谷。 
《名医》曰∶一名行唐。生海滨及壅州。五月采子。 
案∶《广雅》云∶ 萍,KT 荡也。陶弘景云∶今方家多作野狼 。旧作菪。案∶《说文》 
无 
菪、 字。《史记·淳于意传》云∶淄川王美人怀子而不乳,引以莨荡药一撮。《本草图经》 
引作浪荡 


\section{}草蒿
内容:味苦,寒。主疥瘙、痂痒、恶创,杀虫,留热在骨节间,明目。一名青蒿,一名方溃。 
生川泽。 
《名医》曰∶生华阴。 
案∶《说文》云∶蒿, 也; ,香蒿也,或作 。《尔雅》云∶蒿 。郭璞云∶今人呼 
青蒿香中炙啖者为 。《史记·司马相如传》∶ 。注《汉书音义》曰∶ ,蒿也。陶弘 
景 
云∶即今青蒿。 


\section{}旋复花
内容:味咸,温。主结气、胁下满、惊悸,除水,去五脏间寒热,补中下气。一名金沸草,一 
名盛椹。生川谷。 
《名医》曰∶一名戴椹。生平泽。五月采花,晒干,二十日成。 
案∶《说文》云∶ ,盗庚也。《尔雅》云∶ ,盗庚。郭璞云∶旋复似菊。 


\section{}藜芦
内容:(《御览》作梨芦) 
味辛,寒。主蛊毒,咳逆,泄 
葱苒。生山谷。 
《吴普》曰∶藜芦,一名葱葵,一名丰芦,一名蕙葵(《御览》引云∶一名山葱,一名 
公苒)。神农、雷公∶辛,有毒(《御览》引云∶玄黄帝∶有毒);岐伯∶咸,有毒;李氏∶ 
大寒,大毒;扁鹊∶苦,有毒,大寒。叶、根小相 
《名医》曰∶一名葱 ,一名山葱。生太山。三月采根,阴干。 
案∶《广雅》云∶藜芦,葱苒也。《范子计然》云∶藜芦,出河东,黄白者,善。《尔 
雅》云∶ ,山葱,疑非此。 


\section{}钩吻
内容:(《御览》作 ) 
味辛,温。主金创乳 ,中恶风,咳逆上气,水肿,杀鬼注(旧作疰,《御览》作注, 
是) 
蛊毒。一名野葛。生山谷。 
《吴普》曰∶秦,钩肠,一名毒根,一名野葛。神农∶辛;雷公∶有毒,杀人。生南越 
山,或益州,叶如葛,赤茎大如箭、方,根黄。或生会稽东治,正月采(《御览》)。 
《名医》曰∶生傅高山及会稽东野。 
案∶《广雅》云∶莨,钩吻也。《淮南子·说林训》云∶蝮蛇螫人,敷以和堇,则愈。 
高诱云∶和堇,野葛,毒药。《博物志》云∶钩吻毒,桂心、葱叶,沸,解之。陶弘景云∶ 
或云钩吻是毛莨。沈括《补笔谈》云∶闽中人,呼为吻莽,亦谓之野葛;岭南人,谓之胡蔓 
;俗谓之断肠草。此草,人间至毒之物,不入药用。恐本草所出别是一物,非此钩吻也。 


\section{}射干
内容:味苦,平。主咳逆上气,喉痹咽痛不得消息,散急气,腹中邪逆,食饮大热。一名乌扇, 
一名乌蒲。生川谷。 
《吴普》曰∶射干,一名黄远也(《御览》)。 
《名医》曰∶一名乌 ,一名乌吹,一名草姜。生南阳田野。三月三日采根,阴干。 
案∶《广雅》云∶鸢尾,乌 ,射干也。《荀子·劝学篇》云∶西方有木焉,名曰射干 
,茎长四寸。《范子计然》云∶射干根如 安定。 


\section{}蛇合
内容:(原注云,合是含字) 
味苦,微寒。主惊痫寒热邪气,除热,金创,疽痔鼠 ,恶创,头疡。一名蛇衔。生山 
谷。 
《名医》曰∶生益州。八月采,阴干。 
按∶《本草图经》云∶或云是雀瓢,即是萝摩之别名。据陆玑云∶芄兰,一名萝摩,幽 
州谓之雀瓢,则即《尔雅》 ,芄兰也。《唐本草》别出萝摩条,非。又,见女青。 


\section{}恒山
内容:(旧作常山,《御览》作恒山,是) 
味苦,寒。主伤 
《吴普》曰∶恒山,一名漆叶。神农、岐伯∶苦;李氏∶大寒;桐君∶辛,有毒。二月、 
八月采。 
《名医》曰∶生益州及汉中。八月采根,阴干。 
案∶《后汉书·华佗传》云∶佗授以漆叶青粘散∶漆叶屑一斗,青粘十四两,以是为率 
,言久服去三虫,利五脏,轻体,使人头不白。 


\section{}蜀漆
内容:味辛,平。主疟及咳逆寒热,腹中症坚、痞结、积聚,邪气、蛊毒、鬼注(旧作疰,《御 
览》作蛀)。生川谷。 
《吴普》曰∶蜀漆叶,一名恒山。神农、岐伯、雷公∶辛,有毒;黄帝∶辛;一经∶酸 
。如漆叶蓝青相似,五月采(《御览》)。 
《名医》曰∶生江陵山及蜀汉中常山。苗也,五月采叶,阴干。 
案∶《广雅》云∶恒山,蜀漆也。《范子计然》云∶蜀漆,出蜀郡。 


\section{}甘遂
内容:味苦,寒。主大腹疝瘕,腹满,面目浮肿,留饮宿食,破症坚积聚,利水谷道。一名主 
田。生川谷。 
《吴普》曰∶甘遂,一名主田,一名白泽,一名重泽,一名鬼丑,一名陵 ,一名甘 
槁,一名甘泽。神农、桐君∶苦,有毒。歧伯、雷公∶有毒。须二月、八月采(《御览》)。 
案∶《广雅》云∶陵泽,甘遂也。《范子计然》云∶甘遂,出三辅。 


\section{}白蔹
内容:味苦,平。主痈肿疽创,散结气,止痛除热,目中赤,小儿惊痫,温疟,女子阴中肿痛。 
一名免核,一名白草。生山谷。 
《名医》曰∶一名白根,一名昆仑。生衡山,二月、八月采根,曝干。 
案∶《说文》云∶ ,白 也,或作蔹。《毛诗》云∶蔹蔓于野。陆玑疏云∶蔹,似栝 
蒌,叶盛而细, 
热。《尔雅》云∶KT ,菟 
一名菟核。 
即此矣。 


\section{}青葙子
内容:味苦,微寒。主邪气,皮肤中热,风瘙身痒,杀三虫,子∶名草决明,疗唇口青。一名 
草蒿,一名萋蒿。生平谷。 
《名医》曰∶生道旁,三月三日采茎、叶,阴干;五月六日采子。 
案∶《魏略》云∶初平中有青牛先生,常服青葙子。葙,当作箱字。 


\section{}菌
内容:味咸,平。主心痛,温中,去长患、白 、蛲虫、蛇螫毒,症瘕、诸虫。一名 芦,生 
池泽。 
《名医》曰∶生东海及渤海、章武。八月采,阴干。 
案∶《尔雅》云∶滇灌,茵芝。《文选》注引作菌。《声类》云∶滇灌,茵芝也,疑即 
此灌菌,或一名滇,一名芝,未敢定之。 


\section{}白芨
内容:(《御览》作芨) 
味苦,平。主痈肿、恶创、败疽,伤阴,死肌,胃中邪气,贼风鬼击,痱缓不收。一名 
甘根,一名连及草。生川谷。 
《吴普》曰∶神农∶苦;黄帝∶辛;李氏∶大寒;雷公∶辛,无毒。茎叶似生姜、藜芦 
。十月花,直上,紫赤,根白连。二月、八月、九月采。 
《名医》曰∶生北山及冤句,及越山。 
案∶《隋羊公服黄精法》云∶黄精,一名白芨,亦为黄精别名。今《名医》别出黄精条 


\section{}大戟
内容:味苦,寒。主蛊毒、十二水,肿满急痛,积聚,中风,皮肤疼痛,吐逆。一名 钜(案∶ 
此无生川泽三字者,古或与泽漆为一条)。 
《名医》曰∶生常山。十二月采根,阴干。 
案∶《尔雅》云∶荞, 钜。郭璞云∶今药草之戟也。《淮南子·缪尔训》云∶大戟 


\section{}泽漆
内容:味苦,微寒。主皮肤热,大腹、水气,四肢面目浮肿,丈夫阴气不足。生川泽。 
《名医》曰∶一名漆茎,大戟苗也。生太山。三月三日、七月七日采茎、叶,阴 
案∶《广雅》云∶黍茎,泽漆也。 


\section{}茵芋
内容:味苦,温。主五脏邪气,心腹寒热,羸瘦如疟状,发作有时,诸关节风 
《吴普》曰∶茵芋,一名卑共。微温,有毒。状如莽草而细软(《御览》)。 
《名医》曰∶一名莞草,一名卑共,生太山。三月三日采叶,阴干。 


\section{}贯众
内容:味苦,微寒。主腹中邪热气,诸毒,杀三虫。一名贯节,一名贯渠,一名百头(《御览》) 
作白),一名虎卷,一名扁符。生山谷。 
《吴普》曰∶贯众,一名贯来,一名贯中,一名渠母,一名贯钟,一名柏芹,一名药藻, 
一名扁符,一名黄钟。神农、岐伯∶苦,有毒;桐君、扁鹊∶苦;一经∶甘,有毒;黄帝∶ 
咸,酸;一经∶苦,无毒。叶黄,两两相对;茎,黑毛聚生。冬夏不老。四月花,八月实, 
黑聚相连,卷旁行生。三月、八月采根,五月采药(《御览》)。 
《名医》曰∶一名伯萍,一名药藻。此谓草鸱头。生元山及冤句、少室山。二月、八月 
采根,阴干。 
案∶《说文》云∶ 草也。《广雅》云∶贯节,贯众也。《尔雅》云∶泺,费众。郭璞云 
∶叶,圆锐;茎,毛黑。布地,冬夏不死。一名贯渠。又上云∶扁符,止。郭璞云∶未祥。 
据《经》云∶一名篇符,即此也。《尔雅》当云∶篇符,止;泺,贯众。 


\section{}荛花
内容:味苦,平,寒。主伤寒温疟,下十二水,破积聚、大坚、症瘕,荡涤肠胃中留癖饮食、 
寒热邪气,利水道,生川谷。 
《名医》曰∶生咸阳及河南中牟。六月采花,阴干。 


\section{}牙子
内容:味苦,寒。主邪气、热气,疥瘙、恶疡、创痔,去白虫。一名野狼牙,生川谷。 
《吴普》曰∶野狼牙,一名支兰,一名野狼齿,一名犬牙,一名抱子。神农、黄帝∶苦,有 
毒;桐君∶或咸;岐伯、雷公、扁鹊∶无毒。生冤句。叶青,根黄赤,六月、七月花, 
八月实黑。正月、八月采根(《御览》)。 
《名医》曰∶一名野狼齿,一名野狼子,一名犬牙。生淮南及冤句。八月采根,曝干。 
案∶《范子计然》云∶野狼牙,出三辅。色白者,善。 


\section{}羊踯躅
内容:味辛,温。主贼风在皮肤中,淫淫痛,温疟,恶毒,诸痹。生川谷。 
《吴普》曰∶羊踯躅花,神农、雷公∶辛,有毒。生淮南。治贼风、恶毒,诸邪气(《 
御览》)。 
《名医》曰∶一名玉支,生太行山及淮南山。三月采花,阴干。 
案∶《广雅》云∶羊踯躅,英光也。《古今注》云∶羊踯躅花,黄羊食之,则死;羊见之, 
则踯躅分散,故名羊踯躅。陶弘景云∶花苗似鹿葱。 


\section{}商陆
内容:味辛,平。主水张、疝瘕、痹,熨除痈肿,杀鬼精物。一名 根,一名夜呼。生川谷。 
《名医》曰∶如人行者,有神。生咸阳。 
案∶《说文》∶ 草,枝枝相值,叶叶相当。《广雅》云∶常蓼,马尾,商陆也。《尔 
雅》云∶ ,马尾。郭璞云∶今关西亦呼为陆也。盖 ,即 俗字; 
商,即 假音。 


\section{}羊蹄
内容:味苦,寒。主头秃、疥瘙,除热,女子阴蚀(《御览》此四字作无字)。一名东方宿,一 
名连虫陆,一名鬼目,生川泽。 
《名医》曰∶名蓄。生陈留。 
案∶《说文》云∶ 草也,读若厘。 ,厘草也。芨,堇草也。《广雅》云∶ ,羊蹄 
也。《毛诗》云∶言采其 。陆德明云∶本又作蓄。陆玑云∶今人谓之羊蹄。陶弘景云∶今 
人呼秃菜,即是蓄音之伪。《诗》云∶言采其蓄。案∶陆英,疑即此草之花,此草一名连虫 
陆,又陆英,即蒴 ,一名 也。亦苦、寒。 


\section{}蓄
内容:味辛,平。主浸淫、疥瘙、疽痔,杀三虫(《御览》引云∶一名篇竹。《大观本》无文)。 
生山谷。 
《吴普》曰∶ 蓄,一名蓄辩,一名 蔓(《御览》)。 
《名医》曰∶生东莱。五月采,阴干。 
案∶《说文》云∶ , 也, , 也,藩水 。藩,读若督。《尔雅》云∶竹, 
蓄。郭璞云∶似水藜,赤茎节。好生道旁。可食,又杀虫。《毛诗》云∶绿竹猗猗。《传 
》云∶竹, 竹也。《韩诗》藩云∶藩, 也。《石经》同。 


\section{}野狼毒
内容:味辛,平。主咳逆上气,破积聚、饮食、寒热,水气、恶创,鼠 、疽蚀,鬼精、蛊毒, 
杀飞鸟、走兽,一名续毒。生山谷。 
《名医》曰∶生秦亭及奉高。二月、八月采根,阴干。 
案∶《广雅》云∶野狼毒也(疑上脱续毒二字)。《中山经》云∶大KT 之山有草焉,其状 
如 
蓍而毛,青花而白实,其名曰 ,服之不夭,可以为腹病。 


\section{}白头翁
内容:味苦,温。主温疟、易狂、寒热、症瘕积聚、瘿气,逐血、止痛,疗金疮。一名野丈人, 
一名胡王使者。生山谷。 
《吴普》曰∶白头翁,一名野丈人,一名奈河草。神农、扁鹊∶苦,无毒。生嵩山川谷 
。破气狂寒热,止痛(《御览》)。 
《名医》曰∶一名奈河草,生高山及田野。四月采。 
案∶陶弘景云∶近根处有白茸,状似人白头,故以为名。 


\section{}鬼臼
内容:味辛,温。主杀蛊毒鬼注、精物,辟恶气不祥,逐邪,解百毒。一名爵犀,一名马目毒 
公,一名九臼。生山谷。 
《吴普》曰∶一名九臼,一名天臼,一名雀犀,一名马目公,一名解毒。生九真山谷及 
冤句,二月、八月采根(《御览》)。 
《名医》曰∶一名天臼,一名解毒,生九真及冤句,二月、八月采根。 


\section{}羊桃
内容:味苦,寒。主 热,身暴赤色,风水积聚,恶疡,除小儿热。一名鬼桃,一名羊肠。生 
川谷。 
《名医》曰∶一名苌楚,一名御弋,一名铫弋。生山林及田野,二月采,阴干。 
案∶《说文》云∶苌,苌楚,铫弋,一名羊桃。《广雅》云∶鬼桃、铫弋,羊桃也。《 
中山经》云∶丰山多羊桃,状如桃而方,茎可以为皮张。《尔雅》云∶长楚,姚 。郭璞云 
∶今羊桃也,或曰鬼桃。叶似桃;花白;子如小麦,亦似桃。《毛诗》云∶隰有苌楚。《传 
》云∶苌楚,铫弋也。陆玑云∶今羊桃是也,叶长而狭,花紫赤色,其枝、茎弱,过一尺, 
引蔓于草上。今人以为汲灌,重而善没,不如杨柳也。近下根,刀切其皮,着热灰中,脱之 
,可韬笔管。 


\section{}女青
内容:味辛,平。主蛊毒,逐邪恶气,杀鬼温疟,辟不祥。一名雀瓢(《御览》作闾)。 
《吴普》曰∶女青,一名霍由祗。神农、黄帝∶辛(《御览》)。 
《名医》曰∶蛇衔根也。生朱崖,八月采,阴干。 
案∶《广雅》云∶女青,乌葛也。《尔雅》云∶,芄兰。郭璞云∶ 芄蔓生。断之, 
有白汁,可啖。《毛诗》云∶芄兰之支。《传》云∶芄,兰草也。陆玑云∶一名萝摩。幽州 
人谓之雀瓢。《别录》云∶雀瓢白汁,注虫蛇毒,即女青苗汁也。《唐本草》别出萝摩条, 


\section{}连翘
内容:味苦,平。主寒热、鼠 ,瘰 、痈肿,恶创,瘿瘤,结热蛊毒。一名异翘,一名兰花, 
一名轵,一名三廉。生山谷。 
《名医》曰∶一名折根,生太山,八月采,阴干。 
案∶《尔雅》云∶连,异翘。郭璞云∶一名连苕,又名连本草云。 


\section{}兰茹
内容:(《御览》作闾,是) 
味辛,寒。主蚀恶肉、败创、死肌,杀疥虫,排脓恶血,除大风 
,善忘不乐。生川 
谷。 
《吴普》曰∶闾茹,一名离楼,一名屈居。神农∶辛;岐伯∶酸、咸,有毒;李氏∶大 
寒。二月采。叶圆黄,高四、五尺。叶四四相当。四月花黄,五月实黑、根黄,有汁,亦同 
黄。三月、五月采根。黑头者,良(《御览》)。 
《名医》曰∶一名屈据,一名离娄,生代郡。五月采,阴干。 
案∶《广雅》云∶屈居,芦茹也。《范子计然》云∶闾茹,出武都。黄色者,善。 


\section{}乌韭
内容:味甘,寒。主皮肤往来寒热,利小肠膀胱气。生山谷石上。 
案∶《广雅》云∶昔邪,乌韭也,在屋,曰昔邪;在墙,曰垣衣。《西山经》云∶萆荔, 
状如乌韭。《唐本》注云∶即石衣也,亦名石苔,又名石发。按∶《广雅》又云∶石发,石衣 
也, 


\section{}鹿藿
内容:味苦,平。主蛊毒,女子腰腹痛,不乐,肠痈、瘰 (《御览》作历)、疡气。生山谷。 
《名医》曰∶生汶山。 
案∶《说文》云∶ ,鹿藿也,读若剽。《广雅》云∶ ,鹿藿也。《尔雅》云∶ ,鹿 
藿。其实, 。郭璞云∶今鹿豆也。叶似大豆,根黄而香,蔓延生。 


\section{}蚤休
内容:味苦,微寒。主惊痫、摇头弄舌,热气在腹中, 疾痈创,阴蚀,下三虫,去蛇毒。一 
名蚩休。生川谷。 
《名医》曰∶生山及冤句。 
案∶郑樵云∶蚤休,曰螫休,曰重楼金绵,曰重台,曰草甘遂,今人谓之紫河车。服 
食家所用,而茎叶亦可爱。多植庭院间。 


\section{}石长生
内容:味咸,微寒。主寒热、恶创、火热,辟鬼气不祥(《御览》作辟恶气、不祥、鬼毒)。一 
名丹草(《御览》引云丹沙草)。生山谷。 
《吴普》曰∶石长生,神农∶苦;雷公∶辛;一经∶甘。生咸阳(《御览》)。 
《名医》曰∶生咸阳。 


\section{}陆英
内容:味苦,寒。主骨间诸痹,四肢拘挛、疼酸,膝寒痛,阴痿,短气不足,脚肿。生川谷。 
《名医》曰∶生熊耳及冤句。立秋采。又曰∶蒴 ,味酸,温,有毒。一名堇(今本误 
作堇),一名芨。生田野。春夏采叶;秋冬采茎、根。 
案∶《说文》云∶堇草也。读若厘。芨, 草也,读若急。 ,厘草也。《广雅》云 
盆,陆英莓也。《尔雅》云∶芨 草。《唐本》注陆英云∶此物,蒴 是也。后人不识, 
浪出蒴 条。今注云∶陆英,味苦、寒,无毒;蒴 ,味酸、温,有毒,既此不同。难谓一 
种,盖其类尔。 


\section{}荩草
内容:味苦,平,主久咳上气、喘逆,久寒,惊悸,痂疥、白秃、疡气,杀皮肤小虫。 
《吴普》曰∶王刍,一名黄草。神农、雷公曰∶生太山山谷。治身热邪气,小儿身热气 
(《御览》)。 
《名医》曰∶可以染黄,作金公,生青衣。九月、十月采。 
案∶《说文》云∶荩草也。KT ,王刍也。《尔雅》云∶KT ,王刍。郭璞云∶KT , 
蓐也,今呼鸱脚莎。《毛诗》云∶绿竹猗猗。《传》云∶KT ,王刍也。《唐本》注云∶荩草, 
俗名KT 蓐草,《尔雅》所谓王刍。 


\section{}牛扁
内容:味甘,微寒。主身皮创热气,可作浴汤,杀牛虱小虫,又疗牛病。生川谷。 
《名医》曰∶生桂阳。 
案∶陶弘景云∶太常贮,名扁特,或名扁毒。 


\section{}夏枯草
内容:味苦,辛,主寒热、瘰 、鼠 、头创,破症,散瘿、结气,脚肿,湿痹。轻身。一名 
夕句,一名乃东。生川谷。 
《名医》曰∶一名燕面。生蜀郡。四月采。 


\section{}芫花
内容:味辛,温。主咳逆上气,喉鸣、喘,咽肿、短气,蛊毒、鬼疟,疝瘕、痈肿,杀虫鱼。 
一名去水。生川谷(旧 
《吴普》曰∶芫花,一名去水,一名败花,一名儿草根,一名黄大戟。神农、黄 
帝∶有毒;扁鹊、岐伯∶苦;李氏∶大寒。二月生,叶青,加浓则黑。华有紫、赤、白者。 
三月实落尽,叶乃生。三月、五月采花。芫花根,一名赤芫根。神农、雷公∶苦,有毒。生 
邯郸,九月、八月采,阴干。久服,令人泄。可用毒鱼(《御览》,亦见《图经》节文)。 
《名医》曰∶一名毒鱼,一名杜芫。其根,名蜀桑,可用毒鱼。生淮源。三月三日采药, 
阴干。 
案∶《说文》云∶芫,鱼毒也。《尔雅》云∶ ,鱼毒。郭璞云∶ ,大木。子,似栗 
,生南方,皮浓,汁赤,中藏卵果。《范子计然》云∶芫花,出三辅。《史记·仓公传》∶ 
临淄女子病蛲瘕,饮以芫花一撮,出蛲可数升,病已。颜师古注《急就篇》云∶郭景纯说, 
误耳。其生南方,用藏卵果,自别一 木,乃左思所云绵 、 栌者耳,非毒鱼之 。 
上草,下品四十九种。旧四十八种,考木部芫华宜入此。 


\section{}巴豆
内容:味辛,温。主伤寒、温疟、寒热,破症瘕、结聚、坚积,留饮、痰癖。大腹水张 
,荡炼五脏六腑,开通闭塞,利水谷道,去恶内,除鬼毒蛊注邪物(《御览》作鬼毒邪注), 
杀虫鱼。一名巴叔(占作椒,《御览》作菽)。生川谷。 
《吴普》曰∶巴豆,一名巴菽,神农、岐伯、桐君∶辛,有毒;黄帝∶甘,有毒;李氏 
∶主温热寒。叶如大豆。八月采(《御览》)。 
《名医》曰∶生巴郡,八月采,阴干,用之,去心皮。 
案∶《广雅》云∶巴菽,巴豆也。《列仙传》云∶元俗饵巴豆。《淮南子·说林训》云∶ 
鱼食巴菽而死,人食之而肥。 


\section{}蜀菽
内容:味辛,温。主邪气、咳逆,温中,逐骨节,皮肤死肌,寒湿痹痛,下气。久服之,头不 
白、轻身、增年,生川谷。 
《名医》曰∶一名巴椒,一名 。生武都及巴郡。八月采实,阴干。 
案∶《范子计然》云∶蜀椒,出武都。赤色者,善。陆玑云∶蜀人作荼,又见秦椒,即 
《尔雅》 。陶弘景云∶俗呼为 。 


\section{}皂荚
内容:味辛、咸,温。主风痹、死肌、邪气,风头、泪出,利九窍,杀精物。生川谷。 
《名医》曰∶生壅州及鲁邹县。如猪牙者,良。九月、十月采,阴干。 
案∶《说文》云∶荚,草实。《范子计然》云∶皂荚,出三辅。上价一枚一钱。《广志》 
曰∶鸡栖子,皂荚也(《御览》)。皂,即草省文。 


\section{}柳花
内容:味苦,寒。主风水黄胆,面热、黑。一名柳絮。叶∶主马疥痂创;实∶主溃痈,逐脓血; 
子汁∶疗渴。生川泽。 
《名医》曰∶生琅邪。 
案∶《说文》云∶柳,小杨也;柽,河柳也,杨木也。《尔雅》∶柽,河柳。郭璞云∶ 
今河旁赤茎小杨,又旄泽柳。郭璞云∶生泽中者,又杨,蒲柳。郭璞云∶可以为箭,《左传 
》所谓董泽之蒲。《毛诗》云∶无折我树杞。《传》云∶杞,木名也。陆玑云∶杞,柳属也。 


\section{}楝实
内容:味苦,寒。主温疾伤寒,大热烦狂,杀三虫、疥疡,利小便水道。生山谷。 
《名医》曰∶生荆山。 
案∶《说文》云∶楝,木也。《中山经》云∶其实如楝。郭璞云∶楝,木名。子如指头, 
白而粘,可以浣衣也。《淮南子·时则训》云∶七月,其树楝。高诱云∶楝实,凤凰所食, 
今雒城旁有楝树。实,秋熟。 


\section{}郁李仁
内容:味酸,平。主大腹水肿,面目四肢浮肿,利小便水道。根∶主齿龈肿,龋齿,一名爵李。 
生坚齿川谷。 
《吴普》曰∶郁李,一名雀李,一名车下李,一名棣(《御览》)。 
《名医》曰∶一名车下李,一名棣。生高山及邱陵上。五月、六月采根。 
案∶《说文》云∶棣,白棣也。《广雅》云∶山李,雀其郁也。《尔雅》云∶常棣,棣 
。郭璞云∶今关西有棣树,子如樱桃,可食。《毛诗》云∶六月食郁。《传》云∶郁,棣属 
。刘稹《毛诗·义问》云∶常棣之树,高五、六尺;其实大如李,正赤,食之甜。又《诗 
》云∶常棣之花。《传》云∶常棣,棣也。陆玑云∶奥李,一名雀李,一曰李下李,所在山 
中皆有。其花,或白或赤,六 
月中熟大,子如李子,可食。沈括《补笔谈》云∶晋宫阁铭曰∶华 
十四株, 李一侏。 


\section{}莽草
内容:味辛,温。主风头痈肿、乳痈、疝瘕,除结气、疥瘙(《御览》有痈疮二字),杀虫鱼。 
生山谷。 
《吴普》曰∶莽草,一名春草。神农∶辛;雷公、桐君∶苦,有毒。生上谷山谷中或冤 
句。五月采。治风(《御览》)。 
《名医》曰∶一名 ,一名春草。生上谷及冤句。五月采叶,阴干。 
案∶《中山经》云∶朝歌之山有草焉,名曰莽草,可以毒鱼。又 山有木焉,其状如棠 
而赤,叶可以毒鱼。《尔雅》云∶ ,春草。郭璞云∶一名芒草。《本草》云∶《周礼》云 
∶翦氏掌除蠹物,以薰草莽之。《范子计然》云∶莽草,出三辅者,善。陶弘景云∶字亦作 
两。 
雷丸(《御览》作雷公丸) 味苦,寒。主杀三虫,逐毒瓦斯、胃中热,利丈夫,不利女 
子 
。作摩膏,除小儿百病(《御览》引云∶一名雷矢。《大观本》作黑字)。生山谷。 
《吴普》曰∶雷丸,神农∶苦;黄帝、岐伯、桐君∶甘,有毒;扁鹊∶甘,无毒;李氏 
∶大寒(《御览》引云∶一名雷实。或生汉中。八月采)。 
《名医》曰∶一名雷矢,一名雷实。生石城及治中土中。八月采根,曝干。 
案∶《范子计然》云∶雷矢,出汉中。色白者,善。 


\section{}桐叶
内容:味苦,寒。主恶蚀、创着阴皮,主五痔,杀三虫。花∶主传猪创,饲猪,肥大三倍。生 
山谷。 
《名医》曰∶生桐柏山。 
案∶《说文》云∶桐,荣也;梧,梧桐木,一名梓。《尔雅》云∶榇梧,郭璞云∶今梧 
桐。又荣桐木,郭璞云∶即梧桐。《毛诗》云∶梧桐生矣。《传》云∶梧桐,柔木也。 


\section{}梓白皮
内容:味苦,寒。主热,去三虫。叶∶捣,传猪创,饲猪,肥大三倍。生山谷。 
《名医》曰∶生河内。 
案∶《说文》云∶梓,楸也,或作 ,椅梓也。楸,梓也; ,楸也。《尔雅》云∶槐 
,小叶曰 。郭璞云∶槐,为楸楸;当细叶者,为 ;又大而 ,楸。郭璞云∶老乃皮粗 
,者为楸。又椅梓,郭璞云∶即楸。《毛诗》云∶椅,桐梓漆。《传》云∶椅,梓属。 
陆玑云∶梓者,楸之疏理白色而生子者,梓、梓关;桐皮,曰椅。 


\section{}石南
内容:味辛,苦。主养肾气、内伤、阴衰,利筋骨皮毛。实∶杀蛊毒,破积聚,逐风痹。一名 
鬼目。生山谷。 
《名医》曰∶生华阴。二月、四月采实,阴干。 


\section{}黄环
内容:味苦,平。主蛊毒、鬼注、鬼魅、邪气在脏中,除咳逆寒热。一名凌泉,一名大就。生 
山谷。 
《吴普》曰∶蜀,黄环,一名生刍,一名根韭。神农、黄帝、岐伯、桐君、扁鹊∶辛; 
一经∶味苦,有毒。二月生。初出正赤,高二尺;叶黄,员端、大茎,叶有汗,黄白。五月 
实员,三月采根。根黄,从理如车辐,。解治蛊毒(《御览》)。 
《名医》曰∶生蜀郡。三月采根,阴干。 
案∶《蜀都赋》有黄环。刘逵云∶黄环,出蜀郡。沈括《补笔谈》云∶黄环,即今朱藤 
也。天下皆有,叶如槐,其花穗悬紫色如葛,花可作菜食,火不熟,亦有小毒。京师人家园 
圃中,作大架种之,谓之紫藤花者,是也。 


\section{}溲疏
内容:味辛,寒。主身皮肤中热,除邪气,止遗溺,可作浴汤。生山谷及田野、故邱虚地。 
《名医》曰∶一名巨骨,生熊耳山,四月采。 
案∶李当之云∶溲疏,一名杨栌,一名牡荆,一名空疏,皮白,中空,时时有节。子 
似枸杞。子冬日熟,色赤,味甘、苦。 


\section{}鼠李
内容:主寒热瘰 创。生田野。 
《吴普》曰∶鼠李,一名牛李(《御览》)。 
《名医》曰∶一名牛李,一名鼠梓,一名啤。采无时。 
案∶《说文》云∶ ,鼠梓木。《尔雅》云∶鼠梓。郭璞云∶楸属也,今江东有虎 
梓。《毛诗》云∶北山有 。《传》云∶ ,鼠梓。据《名医》名鼠梓,未知是此否?《唐 
本》注云∶一名赵李,一名皂李,一名乌槎。 


\section{}药实根
内容:味辛,温。主邪气,诸痹疼酸,续绝伤,补骨髓。一名连木。生山谷。 
《名医》曰∶生蜀郡。采无时。 
案∶《广雅》云∶贝父,药实也。 


\section{}栾花
内容:味苦,寒。主目痛、泪出、伤 ,消目肿。生川谷。 
《名医》曰∶生汉中。五月采。 
案∶《说文》云∶栾木,似栏。《山海经》云∶云雨之山,有木名栾,黄木赤枝青叶, 
群帝焉取药。《白虎通》云∶诸侯墓树柏;大夫栾;土,槐。沈括《补笔谈》云∶栾有 
一种,树生,其实可作数珠者,谓之木栾,即本草栾花是也。 


\section{}蔓椒
内容:味苦,温。主风寒湿痹、 节疼,除四肢厥气、膝痛。一名家椒。生川谷及邱家间。 
《名医》曰∶一名猪椒,一名彘椒,一名狗椒。生云中。采茎、根,酿酒。 
案∶陶弘景云∶俗呼为 ,以椒 小,不香尔。一名稀椒。可以蒸病出汗也。 
上木,下品一十七种。旧十八种,今移芫花入草。 


\section{}豚卵
内容:味苦,温。主惊痫、 疾,鬼注、蛊毒,除寒热,贲豚、五癃,邪气、挛缩。一名 
豚颠。悬蹄∶主五痔、伏热在肠、肠痈、内蚀。 
案∶《说文》云∶ ,小豕也。从豸省,象形,从又,持内以给祭祀,篆文作豚。《方 
言》云∶猪,其子或谓之豚,或谓之 。,吴扬之间,谓之猪子。 


\section{}麋脂
内容:味辛,温。主痈肿、恶创、死肌,寒、风、湿痹,四肢拘缓不收,风头,肿气,通腠理。 
一名官脂。生山谷。 
《名医》曰∶生南山及雀淮南边。十月取。 
案∶《说文》云∶麋,鹿属,冬至解其角。《汉书》云∶刘向以为∶麋之为言,迷也。 
盖牝兽之淫者也。 


\section{}鼠
内容:主堕胎,令人产易。生平谷。 
《名医》曰∶生山都。 
案∶《说文》云∶ ,鼠形,飞走且乳之鸟也。籀文作 。《广雅》云∶KT KT , 
飞也。 
陶弘景云∶是鼯鼠,一名飞生见。《尔雅》云∶鼯鼠,夷由 


\section{}六畜毛蹄甲
内容:味咸,平。主鬼注、蛊毒,寒热、惊痫, 、狂走。骆驼毛,尤良。 
案∶陶弘景云∶六畜,谓马、牛、羊、猪、狗、鸡也;蹄,即踯省文。 
上兽,下品四种。旧同。 


\section{}虾蟆
内容:味辛,寒。主邪气,破症坚,血、痈肿、阴创。服之,不患热病。生池泽。 
《名医》曰∶一名蟾蜍,一名 ,一名去甫,一名苦 。生江湖。五月五日取,阴干 
。东行者,良。 
案∶《说文》云∶虾,虾蟆也;蟆,虾蟆也; ,虾蟆也; , ,詹诸也。其鸣詹 
诸;其皮 ;其行 ,或作 。 ,詹诸也。《夏小正》传云∶蜮也者,长股也, 
或曰屈造之属也。《诗》曰∶得此 ,言其行 , ,詹诸,以鸣者。虾蟆也。郭璞 
云∶似虾蟆,居陆地。《淮南》谓之去蚊。又 蟆,郭璞云∶蛙类。《周礼》云∶蝈氏。郑司 
农云∶蝈,读为蜮。蜮,虾蟆也。元谓蝈,今御所食蛙也。《月令》云∶仲夏之月,反舌无 
声。蔡邕云∶今谓之虾蟆。薛君《韩诗》注云∶戚放蟾蜍。高诱注《淮南子》云∶蟾, 
也。又蝈,虾蟆也。又蟾蜍,虾蟆。又鼓造,一曰虾蟆。《抱朴子·内篇》云∶或问,魏武 
帝曾收左元放而桎梏之,而得自然解脱,以何法乎?《抱朴子》曰∶以自解去父血。 


\section{}马刀
内容:味辛,微寒(《御览》有补中二字。《大观本》黑字)。主漏下赤白,寒热,破石淋,杀 
禽兽、贼鼠。生池泽。 
《吴普》曰∶马刀,一名齐蛤。神农、岐伯、桐君∶咸,有毒;扁鹊∶小寒,大毒。生 
池泽、江海。采无时也(《御览》)。 
《名医》曰∶一名马蛤。生江湖及东海。采无时。 
案∶《范子计然》云∶马刀,出河东。《艺文类聚》引《本经》云∶文蛤,表有文。又 
曰马刀,一曰名蛤,则岂古本与文蛤为一邪? 


\section{}蛇蜕
内容:味咸,平。主小儿百二十种惊痫、螈 、 疾、寒热、肠痔,虫毒,蛇痫。火熬之,良。 
一名龙子衣,一名蛇符,一名龙子单衣,一名弓皮。生川谷及田野。 
《吴普》曰∶蛇蜕,一名龙子单衣,一名弓皮,一名蛇附,一名蛇筋,一名龙皮,一名 
龙单衣(《御览》)。 
《名医》曰∶一名龙子皮。生荆州。五月五日、十五日取之,良。 
案∶《说文》云∶它,虫也。从虫而长,象冤,曲 尾形。或作蛇蜕,蛇蝉所解皮也。 
《广雅》云∶蝮 蜕也。《中山经》云∶来山多空夺。郭璞云∶即蛇皮脱也。 


\section{}蚯蚓
内容:味咸,寒。主蛇瘕,去三虫、伏尸、鬼注、蛊毒,杀长虫。仍自化作水。生平土 
《吴普》曰∶蚯蚓,一名白颈螳 ,一名附引(《御览》)。 
《名医》曰∶一名土龙。二月取,阴干。 
案∶《说文》云∶ ,侧行者,或作蚓, 也。《广雅》云∶蚯蚓,蜿 ,引无也 
。《尔雅》云∶ 蚓, 蚕。郭璞云∶即 也,江东呼寒蚓,旧作蚯,非。《吕氏春秋》 
、《淮南子》邱蚓出,不从虫。又《说文训》云∶ ,无筋骨之强。高诱注∶ ,一名蜷 
也。旧又有白颈二字,据《吴普》古本当无也。 

<目录>卷三
\section{}下经
内容:味辛,平。主久聋、咳逆、毒瓦斯,出刺出汗。生川谷。 
《名医》曰∶一名土蜂。生熊耳及 柯,或人屋间。 
案∶《说文》云∶KT ,KT 蠃,蒲卢,细要土蜂也。或作螺蠃,螺,蠃也。《广雅》 
云∶ 
土蜂, 也。《尔雅》∶土蜂。《毛诗》云∶螟蛉有子,螺蠃负之。《传》云∶螺蠃, 
卢也。《礼记》云∶夫政也者,蒲卢也。郑云∶蒲卢,果蠃,谓土蜂也。《方言》云∶蜂, 
其小者,谓之 ,或谓之蚴蜕。《说文》无 字,或当为医。 


\section{}蜈蚣
内容:味辛,温。主鬼注、蛊毒,啖诸蛇、早、鱼毒,杀鬼物、老精、温疟,去三虫(《御览》 
引云∶一名至掌。《大观本》在水蛭下)。生川谷。 
《名医》曰∶生大吴江南。赤头足者,良。 
案∶《尔雅》云∶ 蛆,吴公也。 


\section{}水蛭
内容:味咸,平。主逐恶血、瘀血、月闭(《御览》作水闭),破血瘕积聚,无子,利水道。生 
池泽。 
《名医》曰∶一名 ,一名至掌。生雷泽。五月、六月采,曝干。 
案∶《说文》云∶蛭,虮也; ,蛭 ,至掌也。《尔雅》云∶蛭虮。郭璞云∶今江东 
呼水中蛭虫入人肉者,为虮。又蛭 、至掌,郭璞云∶未详,据《名医》,即蛭也。 


\section{}班苗
内容:味辛,寒。主寒热、鬼注蛊毒、鼠 恶创、疽蚀死肌,破石癃。一名龙尾。生川 
《吴普》曰∶斑猫,一名斑蚝,一名龙蚝,一名斑苗,一名胜发,一名盘蛩,一名 
晏青。神农∶辛;岐伯∶咸;桐君∶有毒;扁鹊∶甘,有大毒。生河内川谷,或生水石。 
《名医》曰∶生河东。八月取,阴干。 
案∶《说文》云∶ , 蝥,毒虫也。《广雅》云∶ 晏青也。《名医》别出芫青条 
,非。芫、晏,音相近也。旧作猫,俗字。据吴氏云∶一名班苗,是也。 


\section{}贝子
内容:味咸,平。主目翳、鬼注虫毒、腹痛、下血、五癃,利水道。烧用之,良。生池泽。 
《名医》曰∶一名贝齿。生东海。 
案∶《说文》云∶贝,海介虫也。居陆,名飙;在水,名 ,象形。《尔雅》云 
,小者, 。郭璞云∶今细贝,亦有紫色,出日南。又 ,小而椭。郭璞云∶即上小贝。 


\section{}石蚕
内容:味咸,寒。主五癃,破五淋,堕胎,内解结气,利水道,除热。一名水虱。生池泽。 
《吴普》曰∶石蚕,亦名沙虱。神农、雷公∶酸,无毒。生汉中。治五淋,破随内结气,利 
水道,除热(《御览》)。 
《名医》曰∶生江汉。 
案∶《广雅》云∶水虱, 也。《淮南万毕术》云∶沙虱,一名蓬活,一名地脾。《御 
览》虫豸部引李当之云∶类虫,形如老蚕。生附石。《广志》云∶皆虱,虱色赤,大过虮。 
在水中,入人皮中,杀人,与李似不同。 


\section{}雀瓮
内容:味甘,平。主小儿惊痫,寒热结气,蛊毒鬼注。一名躁舍。 
《名医》曰∶生汉中。采,蒸之。生树枝间,蛄 房也。八月取。 
案∶《说文》云∶蛄,蛄斯黑也。《尔雅》云∶ ,蛄 。郭璞云∶ 属也。今青州人 
呼为蛄 。按∶《本经》名为雀瓮者,瓮与蛹,音相近,以其如雀子,又如茧虫之蛹,因 
呼之。 


\section{}蜣螂
内容:味咸,寒。主小儿惊痫、螈 ,腹胀寒热,大人疾狂易。一名 蜣。火熬之,良。生池 
泽。 
《名医》曰∶生长沙。五月五日取,蒸,藏之。 
案∶《说文》云∶ ,渠 。一曰天杜。《广雅》云∶天杜,蜣螂也。《尔雅》云∶ 蜣, 
蜣螂。郭璞云∶黑甲虫,啖粪土。《玉篇》∶蜣、螂同。《说文》无蜣字。渠 ,即 蜣,音 
之缓急。 


\section{}蝼蛄
内容:味咸,寒。主产难,出肉中刺(《御览》作刺在肉中),溃痈肿,下哽噎(《御览》作咽), 
解毒,除恶创。一名蟪蛄(《御览》作蟪蛄),一名天蝼 
《名医》曰∶生东城。夏至取,曝干。 
案∶《说文》云∶蠹,蝼蛄也;蝼,蝼蛄也;蛄,蝼蛄也。《广雅》云∶炙鼠、津姑、蝼 
蜮、蟓蛉、蛞蝼,蝼蛄也。《夏小正》云∶三月, 则鸣。KT ,天蝼也。《尔雅》云∶ , 
天蝼。郭璞云∶蝼蛄也。《淮南子·时则训》云∶孟夏之月,蝼蝈鸣。高诱云∶蝼,蝼蛄也。 
《方言》云∶蛄诣,谓之杜格;蝼蛞,谓之蝼蛞,或谓之蟓蛉。南楚谓之杜狗,或谓之 蝼。 
陆玑《诗疏》云∶《本草》又谓蝼蛄为石鼠,今无文。 


\section{}马陆
内容:味辛,温。主腹中大坚症,破积聚、息肉、恶创、白秃。一名百足。生川谷。 
《吴普》曰∶一名马轴(《御览》)。 
《名医》曰∶一名马轴。生元菟。 
案∶《说文》云∶蠲,马蠲民。从虫、皿,益声; ,象形。明堂《月令》曰∶腐草为 
蠲。《广雅》云∶蛆 ,马 ,马 也。又马践,蛆也。《尔雅》云∶ ,马践。郭璞云∶马 
蠲匀,俗呼马 。《淮南子·时则训》云∶季夏之日,腐草化为。 。高诱云∶ ,马 也。 
幽冀谓之秦渠。又《汜论训》云∶ ,足众,而走不若蛇。已《兵略训》云∶若研之足。高 
诱云∶ ,马 也。《方言》云∶马 ,北燕谓之蛆渠。其大者,谓之马蚰。《博物志》云∶ 
马,一名百足,中断成两段,各行而去。 


\section{}地胆
内容:味辛,寒。主鬼注、寒热,鼠蝼,恶创、死肌,破症瘕,堕胎。一名 青。生川谷。 
《吴普》曰∶地胆,一名元青,一名杜龙,一名青虹(《御览》)。 
《名医》曰∶一名青 。生汶山,八月取。 
案∶《广雅》云∶地胆,蛇要,青 ,青 也。陶弘景云∶状如大马蚁,有翼。伪者, 
即班猫所化,状如大豆。 


\section{}鼠妇
内容:味酸,温。主气癃不得小便,女人月闭、血症,痫症、寒热,利水道。一名负蟠,一名 
威。生平谷。 
《名医》曰∶一名 。生魏郡及人家地上。五月五日取。 
案∶《说文》云∶ , 威,委黍;委黍,鼠妇也;蟠,鼠负也。《尔雅》云∶蟠,鼠 
负。郭璞云∶瓮器底虫。又 威,委黍。郭璞云∶旧说,鼠妇别名。《毛诗》云∶伊芳威在室 
。《传》云∶伊芳威,委黍也。陆玑云∶在壁根下,瓮底中生,似白鱼。 


\section{}荧火
内容:味辛,微温。主明目,小儿火创伤,热气、蛊毒、鬼注,通神。一名夜光(《御览》引 
云∶一名熠耀,一名即照。《大观本》作黑字)。生池泽。 
《吴普》曰∶荧火,一名夜照,一名熠耀,一名救火,一名景天,一名据火,一名挟火 
(《艺文类聚》)。 
《名医》曰∶一名放光,一名熠耀,一名即照。生阶地。七月七日收,阴干。 
案∶《说文》云∶ ,兵死及牛马之血为磷,鬼火也,从炎舛。《尔雅》云∶荧火,即 
照。郭璞云∶夜飞,腹下有火。《毛诗》云∶熠耀宵行。传云∶熠耀,磷也;磷,荧火也 
《月令》云∶季夏之月,腐草化为荧。郑元云∶萤飞虫,萤火也。据毛苌以萤为磷,是也。 
《说文》无萤字,当以磷为之。《尔雅》作荧,亦是。旧作萤,非。又按∶《月令》∶腐草 
为萤,当是蠲字假音。 


\section{}衣鱼
内容:味咸,温,无毒。主妇人疝瘕,小便不利(《御览》作泄利),小儿中风(《御览》作头 
风)、项强(《御览》作疆),背起摩之。一名白鱼。生平泽。 
《吴普》曰∶衣中白鱼。一名 (《御览》)。 
《名医》曰∶一名 。生咸阳。 
案∶《说文》云∶ ,白鱼也。《广雅》云∶白鱼, 鱼也。《尔雅》云∶ ,白鱼。郭 
璞云∶衣、书中虫,一名蚋鱼。 
上虫、鱼,下品一十八种。旧同。 


\section{}桃核仁
内容:味苦,平。主症血、血闭、瘕邪,杀小虫。桃花∶杀注恶鬼,令人好颜色。桃凫∶微温。 
主杀百鬼精物(《初学记》引云∶枭桃在树不落,杀百鬼)。桃毛∶主下血瘕寒热,积寒无子。 
桃蠹∶杀鬼邪恶不祥。生川谷。 
《名医》曰∶桃核,七月采,取仁,阴干;花,三月三日采,阴干;桃凫,一名桃奴, 
一名枭景。是实着树不落。实中者,正月采之;桃蠹,食桃树虫也。生太山。 
案∶《说文》云∶桃,果也。《玉篇》云∶桃,毛果也。《尔雅》云∶桃李丑核。郭璞 
云∶子中有核仁。孙炎云∶桃李之实,类皆有核。 


\section{}杏核仁
内容:味甘,温。主咳逆上气,雷鸣,喉痹下气,产乳,金创、寒心、贲豚。生川谷。 
《名医》曰∶生晋山。 
案∶《说文》云∶杏,果也。《管子·地员篇》云∶五沃之土,其木宜杏。高诱注《淮南 
子》云∶杏,有窍在中。 
上果,下品二种。旧同。 


\section{}腐婢
内容:味辛,平。主 疟,寒热、邪气,泄利,阴不起,病酒,头痛。生汉中。 
《吴普》曰∶小豆花,一名腐婢(旧作付月,误)。神农∶甘,毒。七月采,阴干四十 
日。 
治头痛,止渴(《御览》)。 
《名医》曰∶生汉中。即小豆花也。七月采,阴干。 
上米、谷,下品一种。旧同。 


\section{}苦瓠
内容:味苦,寒。主大水,面目四肢浮肿,下水,令人叶。生川泽。 
《名医》曰∶生晋地。 
案∶《说文》云∶瓠匏,匏瓠也。《广雅》云∶匏,瓠也。《尔雅》云∶瓠,栖瓣。《 
毛诗》云∶瓠有苦叶。《传》云∶匏,谓之瓠。又九月断壶。《传》云∶壶,瓠也。《古今 
注》云∶瓠,壶芦也。壶芦,瓠之无柄者。瓠,有柄者。又云∶瓢,瓠也。其 ,曰匏。 
瓠则别名。 


\section{}水靳
内容:味甘,平。主女子赤沃,止血养精,保血脉,益气,令人肥健、嗜食。一名水英。生池 
泽。 
《名医》曰∶生南海。 
案∶《说文》云∶芹,楚葵也;近,菜类也。《周礼》有近菹。《尔雅》云∶芹,楚葵。 
郭璞云∶今水中芹菜。《字林》云∶芹草,生水中。根,可缘器。又云∶ 菜,似蒜, 
生水 
中。 
上菜,下品二种。旧同。 


\section{}彼子
内容:味甘,温。主腹中邪气,去三虫、蛇螫、蛊毒、鬼注、伏尸。生山谷(旧在《唐本》退 
中)。 
《名医》曰∶生永昌。 
案∶陶弘景云∶方家,从来无用此者。古今诸医及药家,子不复识。又,一名熊子,不 
知其形何类也。掌禹锡云∶树,似杉;子,如槟榔。《本经》虫部云∶彼子,芝注云∶彼字 
合从木。《尔雅》云∶彼,一名 。 
三品,合三百六十五种,法三百六十度,一度应一日,以成一岁(倍其数,合七百三十 
名也)。 
掌禹锡曰∶本草例,《神农本经》以朱书,《名医别录》以墨书。《神农》药三百六十五 
种,今此言倍其数,合七百三十名,是并《名医别录》副品而言也。则此下节《别录》之文 
也,当作墨书矣。盖传定浸久,朱墨错乱之所致耳。 
案∶禹锡说,是也,改为细字。 
药有君、臣、佐、使,以相宣摄合和宜。用一君,二臣,三佐,五使;又可一君,三臣 
,九佐、使也。 
药有阴阳配合,子母兄弟,根茎花实,草石骨肉;有单行者,有相须者,有相使者,有 
相畏者,有相恶者,有相反者,有相杀者。凡此七精,合和时之,当用相须、相使者良,勿 
用相恶、相反者。若有毒宜制,可用相畏、相杀者。不尔,勿合用也。 
药有酸、咸、甘、苦、辛五味,又有寒、热、温、凉四气,及有毒无毒,阴干曝干,采 
造时月,生熟土地所出,真伪陈新,并各有法。 
药性有宜丸者,宜散者,宜水煮者,宜酒渍者,宜膏煎者;亦有一物兼宜者;亦有不可 
入汤酒者,并随药性,不得违越。 
欲疗病,先察其原,先候病机。五脏未虚,六腑未竭,血脉未乱,精神未散,服药必活 
。若病已成,可得半愈;病势已过,命将难全。 
若用毒药疗病,先起如黍粟,病去,即止。不去,倍之;不去,十之。取去为度。 
疗寒,以热药;疗热,以寒药;饮食不消,以吐下药;鬼注蛊毒,以毒药;痈肿创瘤, 
以创药;风湿,以风湿药。各随其所宜。 
病在胸膈以上者,先食,后服药;病在心腹以下者,先服药而后食;病在四肢血脉者, 
宜空腹而在旦;病在骨髓者,宜饱满而在夜。 
夫大病之主,有中风伤寒,寒热温疟,中恶霍乱,大腹水肿,肠 下利,大小便不通; 
贲肫上气,咳逆呕吐;黄疽消渴,留饮癖食,坚积症瘕,惊邪 痫;鬼注喉痹、齿痛, 
耳聋目盲;金创 折,痈肿恶创,痔 瘿瘤;男子五劳七伤,虚乏羸瘦;女子带下崩中,血 
闭阴蚀;虫蛇蛊毒所伤。此大略宗兆。其间变动枝叶,各宜依端绪以取之。 
上药,令人身安命延,升天神仙,遨游上下,役使万灵,体生毛羽,行厨立至(《抱朴 
子·内篇》引《神农经》,据《太平御览》校)。 
中药,养性;下药,除病。能令毒虫不加,猛兽不犯,恶气不行,众妖并辟(《抱朴子 
·内篇》引《神农经》)。 
太一子曰∶凡药,上者,养命;中者,养性;下者,养病(《艺文类聚》引《本草经》)。 
太一子曰∶凡药,上者,养命;中药,养性;下药,养病。神农乃作赭鞭、钩 (尺制 
切)。从六阴阳,与太乙外(巡字)五岳四渎,土地所生草石,骨肉心灰,皮,毛羽,万千 
类 
,皆鞭问之,得其所能治主,当其五味,一日(一字旧误作百)七十毒(《太平御览》引《本 
草经》)。 
神农稽首再拜,问于太乙子曰∶曾闻之时寿过百岁,而徂落之咎,独何气使然也?太乙 
子曰∶天有九门,中道最良。神农乃从其尝药,以拯救人命。(《太平御览》引《神农本草 
》)。 
按∶此诸条,与今《本经》卷上文略相似,诸书所引,较《本经》文多。又云是太一子 
说,今无者,疑后节之。其云赭鞭、钩 ,当是煮辨、候制之假音,鞭问之,即辨问之。无 
怪说也。 
药物有大毒,不可入口鼻耳目者,即杀人。一曰钩吻(卢氏曰∶阴地黄精,不相连,根 
苗独生者,是也),二曰鸱(状如雌鸡,生山中),三曰阴命(赤色,着木县其子,生海中), 
四曰内童(状如鹅,亦生海中),五曰鸩羽(如雀,墨头赤喙),六曰 (生海中,雄曰 , 
雌 
曰也。《博物志》引《神农经》)。 
药种有五物∶一曰野狼毒,占斯解之;二曰巴头,藿汁解之;三曰黎,卢汤解之;四曰天 
雄、乌头,大豆解之;五曰班茅,戎盐解之。毒菜害小儿,乳汁解,先食饮二升(《博物志》 
引《神农经》)。 
五芝及饵丹砂、玉札、曾青、雄黄、雌黄、云母、太乙禹余粮,各可单服之,皆令人飞 
行、长生(《抱朴子·内篇》引《神农四经》)。 
春夏为阳,秋冬为阴(《文选》注引《神农本草》)。 
春为阳,阳温,生万物(同上)。 
黄精与术,饵之却粒;或遇凶年,可以绝粒。谓之米脯(《太平御览》引《抱朴子》、《神 
农经》)。 
五味,养精神,强魂魄。五石,养髓,肌肉肥泽。诸药,其味酸者,补肝、养心,除肾 
病;其味苦者,补心、养脾,除肝病;其味甘者,补肺、养脾,除心病;其味辛者,补肺、 
养肾,除脾病;其味咸者,补肺,除肝病。故五味,应五行;四体,应四时。夫人性生于四 
时,然后命于五行,以一补身,不死命神。以母养子,长生延年;以子守母,除病究年(《太 
平御览》引《养生要略》、《神农经》)。 
案∶此诸条,当是玉石、草木三品前总论,而后人节去。 


\section{}附∶《吴氏本草》十二条
内容:龙眼 一名益智,一名比目(《齐民要术》)。 
鼠尾 一名劲,一名山陵翘。治痢也(《太平御览》)。 
满阴实 生平谷或圃中。延蔓如瓜叶,实如桃。七月采。止渴延年(《太平御览》)。 
千岁垣中肤皮,得姜、赤石胎,治(《太平御览》)。 
小华 一名结草(《太平御览》)。 
木瓜 生夷陵(《太平御览》)。 
谷树皮 治喉闭。一名楮(《太平御览》)。 
樱桃 味甘。主调中益气,令人好颜色,美志气。一名朱桃,一名麦英也(《艺文类聚》)。 
李核 治仆僵。花,令人好颜色(《太平御览》)。 
大麦 一名 麦。五谷之大盛,无毒,治消渴,除热,益气。食蜜为使。麦种∶一名小 
麦。无毒。治利而不中 (《太平御览》)。 
豉,益人气(《太平御览》)。 
晖日,一名鸩羽(《太平御览》)。 


\section{}附∶诸药制使
内容:唐慎微曰∶《神农本经》相使,正各一种,冀以《药对》参之,乃有两三。 


\section{}玉、石,上部
内容:玉泉 畏款冬花。 
玉屑 恶鹿角。 
丹砂 恶磁石,畏咸水。 
曾青 畏菟丝子。 
石胆 水英为使;畏牡桂、菌桂、芫花、辛夷白。 
钟乳 蛇床子为使,恶牡丹、牡蒙、元石、畏紫石英、 草。 
云母 泽泻为使;畏蛇甲及流水。 
消石 口为使;恶苦参、苦菜,畏女菀。 
朴硝 畏麦句姜。 
芒硝 石苇为使,恶麦句姜。 
矾石 甘草为使;畏牡蛎。 
滑石 石苇为使;恶曾青。 
紫石英 长石为使,畏扁青、附子,不欲蛇甲、黄连、麦句姜。 
白石英 恶马目毒公。 
赤石脂 恶大黄;畏芜花。 
黄石脂 曾青为使;恶细辛;畏蜚蠊。 
太一余粮 杜仲为使;畏铁落、菖蒲、贝母。 
玉、石,中部 
水银 畏磁石。 
殷孽 恶防己;畏木。 
孔公孽 木兰为使;恶细辛。 
阳起石 桑螵蛸为使;恶泽泻、菌桂、雷丸、蛇蜕皮;畏菟丝子。 
石膏 鸡子为使;恶莽草毒公。 
凝水石 畏地榆;解巴豆毒。 
磁石 柴胡为使;畏黄石脂;恶牡丹、莽草。 
元石 恶松脂、柏子仁、菌桂。 
理石 滑石为使;恶麻黄。 


\section{}玉、石,下部
内容:矾石 得火良;棘针为使;恶虎掌、毒公、 屎、细辛、水。 
青琅 得水银良;畏鸡骨;杀锡毒。 
特生矾石 得火良;畏水。 
代赭 畏天雄。 
方解石 恶巴豆。 
大盐 漏芦为使。 


\section{}草药,上部
内容:六芝 薯蓣为使;得发良;恶常山;畏扁青、茵陈。 
术 防风、地榆为使。 
天门冬 垣衣、地黄为使;畏曾青。 
麦门冬 地黄、车前为使;恶款冬、苦瓠;畏苦参、青 。 
女萎、蕤主 畏卤KT 咸。 
干地黄 得麦门冬、清酒,良;恶贝母;畏元荑。 
菖蒲 秦花、秦皮为使;恶地胆、麻黄。 
泽泻 畏海蛤、文蛤。 
远志 得茯苓、冬葵子、龙骨,良;杀天雄、附子毒;畏珍珠、蜚蠊、藜芦。 
齐蛤 薯蓣,紫芝为使,恶甘遂。 
石斛 陆英为使;恶凝水石、巴豆;畏白僵蚕、雷丸。 
菊花 术、枸杞根、桑根、白皮,为使。 
甘草 术、干漆、苦参为使;恶远志;反甘遂、大戟、芫花、海藻。 
人参 茯苓为使;恶溲疏;反藜芦。 
牛膝 恶荧火、龟、陆英;畏白。 
细辛 曾青、东根为使;恶野狼毒、山茱萸、黄 ;畏滑石、硝石;反藜芦。 
独活 蠡石为使。 
柴胡 半夏为使;恶皂荚;畏女苑、藜芦。 
子 荆子、薏苡仁为使。 
菥 子 得荆子、细辛,良;恶干姜、苦参。 
龙胆 贯众为使;恶防葵、地黄。 
菟丝子 得酒良;薯蓣、松脂为使;恶 茵。 
巴戟天 复盆子为使;恶朝生、雷丸、丹参。 
蒺藜子 乌头为使。 
沙参 恶防己;反藜芦。 
防风 恶干姜、藜芦、白蔹、芫花;杀附子毒。 
络石 杜仲、牡丹为使,恶铁落;畏菖蒲、贝母。 
黄连 黄芩、龙骨、理石为使;恶菊花、芫花、元参、白藓皮;畏款冬;胜乌头;解巴 
豆毒。 
丹参 味咸水,反藜芦。 
天名精 垣衣为使。 
决明子 蓍实为使;恶大麻子。 
续断 地黄为使;恶雷丸。 
芎 白芷为使。 
黄 恶龟甲。 
杜若 得辛夷、细辛,良;恶柴胡、前胡。 
蛇床子 恶牡丹、巴豆、贝母。 
茜根 畏鼠姑。 
飞蠊 得乌头,良;恶麻黄。 
薇衔 得秦皮,良。 
五味子 苁蓉为使;恶葳蕤;胜乌头。 


\section{}草药,中部
内容:当归 恶兰茹;畏菖蒲、海藻、牡蒙。 
秦艽 菖蒲为使。 
黄芩 山茱萸,龙骨为使;恶葱实;畏丹砂、牡丹、藜芦。 
芍药 须丸为使;恶石斛、芒硝;畏石、鳖甲,小蓟;反藜芦。 
干姜 秦椒为使;恶黄连、黄芩、天鼠屎;杀半夏、莨菪毒。 
本 畏 茹。 
麻黄 浓朴为使;恶辛夷、石苇。 
葛根 杀野葛、巴豆、百药毒。 
前胡 半夏为使;恶皂荚;畏藜芦。 
贝母 浓朴、白薇为使;恶桃花;畏秦艽、矾石、莽草;反乌头。 
栝蒌 枸杞为使;恶干姜;畏牛膝、干漆;反乌头。 
元参 恶黄 、干姜、大枣、山茱萸;反藜芦。 
苦参 元参为使;恶贝母、漏芦、菟丝子;反藜芦。 
石龙芮 大戟为使;畏蛇蜕、吴茱萸。 
萆 薏苡为使;畏葵根、大黄、柴胡、牡蛎、前胡。 
石苇 滑石、杏仁为使,得菖蒲,良。 
狗脊 萆 为使;恶败酱。 
瞿麦 草、牡丹为使;恶螵蛸。 
白芷 当归为使;恶旋复花。 
紫菀 款冬为使;恶天雄、瞿麦、雷丸、远志;畏茵陈。 
白藓皮 恶螵蛸、桔梗、茯苓、萆 。 
白薇 恶黄 、大黄、大戟、干姜、干漆、大枣、山茱萸。 
紫参 畏辛夷。 
淫羊藿 薯蓣为使。 
款冬花 杏仁为使;得紫菀,良;恶皂荚、硝石、元参;畏贝母、辛夷、麻黄、黄芩、 
黄连、黄 、青葙。 
牡丹 畏菟丝子。 
防己 殷孽为使;恶细辛;畏萆 ;杀雄黄毒。 
女苑 畏卤咸。 
泽兰 防己为使。 
地榆 得发良;恶麦门冬。 
海藻 反甘草。 


\section{}草药,下部
内容:大黄 黄芩为使。 
桔梗 节皮为使;畏白芨;反龙胆、龙眼。 
甘遂 瓜蒂为使;恶远志;反甘草。 
葶苈 榆皮为使;得酒良;恶僵蚕、石龙芮。 
芫花 决明为使;反甘草。 
泽漆 小豆为使;恶薯蓣。 
大戟 反甘草。 
钩吻 半夏为使;恶黄芩。 
藜芦 黄连为使;反细辛、芍药、五参;恶大黄。 
乌头、乌喙 莽草为使,反半夏、栝蒌、贝母、白蔹、白芨;恶藜芦。 
天雄 远志为使,恶腐婢。 
附子 地坦为使;恶蜈蚣;畏防风、甘草、黄 、人参、乌韭、大豆。 
贯众 菌为使。 
半夏 射干为使;恶皂荚;畏雄黄、生姜、干姜、秦皮、龟甲;反乌头。 
蜀漆 栝楼为使;恶贯众。 
虎掌 蜀漆为使;畏莽草。 
野狼牙 芜荑为使;恶枣肌、地榆。 
常山 畏玉札。 
白芨 紫石英为使;恶理石、李核仁、杏仁。 
白蔹 代赭为使;反乌头。 
菌 得酒,良;畏鸡子。 
茹 甘草为使;恶麦门冬。 
荩草 畏鼠妇。 
夏枯草 土瓜为使。 
野狼毒 大豆为使;恶麦句姜。 
鬼臼 畏衣。 


\section{}木药,上部
内容:茯苓,茯神 马间为使;恶白蔹;畏牡蒙、地榆、雄黄、秦艽、龟甲。 
杜仲 恶蛇蜕、元参。 
柏实 牡蛎、桂心、瓜子为使;畏菊花、羊蹄、诸石、面曲。 
干漆 半夏为使;畏鸡子。 
蔓荆子 恶乌头、石膏。 
五加皮 远志为使;畏蛇皮、元参。 
孽木 恶干漆。 
辛夷 芎 为使;恶五石脂;畏菖蒲、蒲黄、黄连、石膏、黄环。 
酸枣仁 恶防己。 
槐子 景天为使。 
牡荆实 防己为使;恶石膏。 


\section{}木药,中部
内容:浓朴 干姜为使;恶泽泻、寒水石、硝石。 
山茱萸 蓼实为使;恶桔梗、防风、防己。 
吴茱萸 蓼实为使;恶丹参、硝石、白垩;畏紫石英。 
秦皮 大戟为使;恶茱萸。 
占斯 解野狼毒毒。 
栀子 解踯躅毒。 
秦椒 恶栝蒌、防葵;畏雌黄。 
桑根白皮 续断、桂心、麻子为使。 


\section{}木药,下部
内容:黄环 鸢尾为使;恶茯苓、防己。 
石南 五加皮为使。 
巴豆 芫花为使;恶 草;畏大黄、黄连、藜芦;杀斑蝥毒。 
栾花 决明为使。 
蜀椒 杏仁为使,畏款冬。 
溲疏 漏芦为使。 
皂荚 柏实为使;恶麦门冬;畏空青、人参、苦参。 
雷丸 荔实、浓朴为使;恶葛根。 


\section{}兽,上部
内容:龙骨 得人参、牛黄,良;畏石膏。 
龙角 畏干漆、蜀椒、理石。 
牛黄 人参为使,恶龙骨、地黄、龙胆、蜚蠊;畏牛膝。 
白胶 得火,良;畏大黄。 
阿胶 得火,良;畏大黄。 


\section{}兽,中部
内容:犀角 松子为使;恶 菌、雷丸。 
羚羊角 菟丝子为使。 
鹿茸 麻勃为使。 
鹿角 杜仲为使。 


\section{}兽,下部
内容:麋脂 畏大黄。 
伏翼 苋实、云实为使。 
天鼠屎 恶白蔹、白薇。 


\section{}虫、鱼,上部
内容:蜜蜡 恶芫花、齐蛤。 
蜂子 畏黄芩、芍药、牡蛎。 
牡蛎 贝母为使;得甘草、牛膝、远志、蛇床,良;恶麻黄、吴茱萸、辛夷。 
桑螵蛸 畏旋复花。 
海蛤 蜀漆为使;畏狗胆、甘遂、芫花。 
龟甲 恶沙参、蜚蠊。 


\section{}虫、鱼,中部
内容:皮 得酒良;畏桔梗、麦门冬。 
蜥蜴 恶硫黄、斑蝥、芜荑。 
露蜂房 恶干姜、丹参、黄芩、芍药、牡蛎。 
虫 畏皂荚、菖蒲。 
蛴螬 蜚蠊为使,恶附子。 
龟甲 恶矾石。 
蟹 杀莨菪毒、漆毒。 
鱼甲 蜀漆为使;畏狗胆、甘遂、芫花。 
乌贼鱼骨 恶白蔹、白芨。 


\section{}虫、鱼,下部
内容:蜣螂 畏羊角,石膏。 
蛇蜕 畏磁石及酒。 
斑蝥 马刀为使;畏巴豆、丹参、空青;恶肤青。 
地胆 恶甘草。 
马刀 得水良。 


\section{}果,上部
内容:大枣 杀乌头,毒。 


\section{}菜,上部
内容:冬葵子 黄芩为使。 
葱实,解藜芦毒。 


\section{}米,上部
内容:麻 、麻子 畏牡蛎、白薇;恶茯苓。 


\section{}米,中部
内容:大豆及黄卷 恶五参、龙胆;得前胡、乌喙、杏仁、牡蛎,良;杀乌头毒。 
大麦 蜜为使。 
上二百三十一种,有相制使,其余皆无(三十四种续添,案∶当云三十 
立冬之日,菊、卷柏先生;时为阳起石、桑螵蛸。凡十物使,主二百草,为之长。 
立春之日,木兰、射干先生。为柴胡、半夏使。主头痛,四十五节。 
立夏之日,蜚蠊先生。为人参、茯苓使。主腹中。七节,保神守中。 
夏至之日,豕道、茱萸先生;为牡蛎、乌喙使。主四肢,三十二节。 
立秋之日,白芷、防风先生。为细辛、蜀漆使。主胸背二十四节。 
(原注∶上此五条,出《药对》中,义旨渊深,非俗所究。虽莫可遵用,而是主统之本 
,故亦载之。) 

\section{录《本草经》书后己丑}
\end{document}