% 西游记
% 西游记.tex

\documentclass[a4paper,12pt,UTF8,twoside]{ctexbook}

% 设置纸张信息。
\RequirePackage[a4paper]{geometry}
\geometry{
	%textwidth=138mm,
	%textheight=215mm,
	%left=27mm,
	%right=27mm,
	%top=25.4mm, 
	%bottom=25.4mm,
	%headheight=2.17cm,
	%headsep=4mm,
	%footskip=12mm,
	%heightrounded,
	inner=1in,
	outer=1.25in
}

% 设置字体,并解决显示难检字问题。
\xeCJKsetup{AutoFallBack=true}
\setCJKmainfont{SimSun}[BoldFont=SimHei, ItalicFont=KaiTi, FallBack=SimSun-ExtB]

% 目录 chapter 级别加点(.)。
\usepackage{titletoc}
\titlecontents{chapter}[0pt]{\vspace{3mm}\bf\addvspace{2pt}\filright}{\contentspush{\thecontentslabel\hspace{0.8em}}}{}{\titlerule*[8pt]{.}\contentspage}

% 设置 chapter 标题格式。
\usepackage{varwidth}
\ctexset{
	chapter/name={第,回},
	chapter/titleformat= \chaptertitleformat
}
\newcommand\chaptertitleformat[1]{
	\begin{varwidth}
		[t]{.7\linewidth}#1
	\end{varwidth}
}

% 设置古文原文格式。
\newenvironment{yuanwen}{\bfseries}

% 设置署名格式。
\newenvironment{shuming}{\hfill\bfseries\zihao{4}}

\title{\heiti\zihao{0} 西游记}
\author{吴承恩}
\date{明}

\begin{document}

\maketitle
\tableofcontents

\frontmatter

\chapter{前言}

《西游记》的故事在唐玄奘取经之后,经历数百年的演变,逐渐发展成型,最迟于明代集撰成为小说。现存最早的百回本《西游记》约刊行于明嘉靖至万历年间,为明刊金陵世德堂本(简称世本)。

世本现存四套,一套藏于中国,三套藏于日本,后者中直到近年才有天理本和浅野本两套得以在国内公开。

明代李评本、唐僧本、杨闽斋本、闽斋堂本,清代新说本、证道书等。

\mainmatter

\chapter[灵根育孕源流出\ 心性修持大道生]{灵根育孕源流出\\心性修持大道生}

\begin{yuanwen}
诗曰:

\begin{verse}
混沌未分天地乱,茫茫渺渺无人见。

自从盘古破鸿蒙,开辟从兹清浊辨。

覆载群生仰至仁,发明万物皆成善。

欲知造化会元\footnote{会元:会、元,皆为时间单位,来自宋代思想家邵雍提出的一种宇宙观,以元、会、运、世为单位推演宇宙变化。三十年为一世,十二世为一运,三十运为一会,十二会为一元。每一元即为一次宇宙生灭周期。下文“盖闻天地之数”一节,即在解释这一观念。}功,须看西游释厄传。
\end{verse}

盖闻天地之数,有十二万九千六百岁为一元。将一元分为十二会,乃子、丑、寅、卯、辰、巳、午、未、申、酉、戌、亥之十二支也。每会该一万八百岁。且就一日而论:子时得阳气,而丑则鸡鸣;寅不通光,而卯则日出;辰时食后,而巳则挨排;日午天中,而未则西蹉\footnote{倾斜、下坠。};申时晡\footnote{b\=u}而日落酉;戌黄昏而人定亥。\footnote{“子时”句:古用“夜半、鸡鸣、平旦、日出、食时、隅中、日中、日昳、晡时、日入、黄昏、人定”命名十二时辰,与十二地支命名一一相对。}譬于大数\footnote{自然法则;气数。},若到戌会之终,则天地昏曚\footnote{m\'eng}而万物否[4]矣。再去五千四百岁,交亥会之初,则当黑暗,而两间人物俱无矣,故曰混沌。又五千四百岁,亥会将终,贞下起元[5],近子之会,而复逐渐开明。邵康节[6]曰:“冬至子之半,天心无改移。一阳初动处,万物未生时。”[7]到此,天始有根。再五千四百岁,正当子会,轻清上腾,有日,有月,有星,有辰。日、月、星、辰,谓之四象。故曰,天开于子。又经五千四百岁,子会将终,近丑之会,而遂渐坚实。《易》曰:“大哉乾元!至哉坤元!万物资生,乃顺承天。”至此,地始凝结。再五千四百岁,正当丑会,重浊下凝,有水,有火,有山,有石,有土。水、火、山、石、土,谓之五形。故曰,地辟于丑。又经五千四百岁,丑会终而寅会之初,发生万物。历曰:“天气下降,地气上升;天地交合,群物皆生。”至此,天清地爽,阴阳交合。再五千四百岁,正当寅会,生人,生兽,生禽,正谓天地人,三才[8]定位。故曰,人生于寅。

感盘古开辟,三皇治世,五帝定伦,世界之间,遂分为四大部洲:曰东胜神洲,曰西牛贺洲,曰南赡部洲,曰北俱芦洲。[9]这部书单表东胜神洲。海外有一国土,名曰傲来国。国近大海,海中有一座名山,唤为花果山。此山乃十洲之祖脉,三岛之来龙,[10]自开清浊而立,鸿蒙判后而成。真个好山!有词赋为证。赋曰:

势镇汪洋,威宁瑶海。势镇汪洋,潮涌银山鱼入穴;威宁瑶海,波翻雪浪蜃离渊。木火方隅高积土[11],东海之处耸崇巅。丹崖怪石,削壁奇峰。丹崖上彩凤双鸣;削壁前麒麟独卧。峰头时听锦鸡鸣,石窟每观龙出入。林中有寿鹿仙狐,树上有灵禽玄鹤。瑶草奇花不谢,青松翠柏长春。仙桃常结果,修竹每留云。一条涧壑藤萝密,四面原堤草色新。正是百川会处擎天柱,万劫无移大地根。

那座山正当顶上,有一块仙石。其石有三丈六尺五寸高,有二丈四尺围圆。三丈六尺五寸高,按周天三百六十五度;二丈四尺围圆,按政历[12]二十四气。上有九窍八孔,按九宫八卦。四面更无树木遮阴,左右倒有芝兰相衬。盖自开辟以来,每受天真地秀,日精月华,感之既久,遂有灵通之意。内育仙胞,一日迸裂,产一石卵,似圆球样大。因见风,化作一个石猴。五官俱备,四肢皆全。便就学爬学走,拜了四方[13]。目运两道金光,射冲斗府[14]。惊动高天上圣大慈仁者玉皇大天尊玄穹高上帝,驾座金阙云宫灵霄宝殿,聚集仙卿,见有金光焰焰,即命千里眼、顺风耳开南天门观看。二将果奉旨出门外,看的真,听的明,须臾回报道:“臣奉旨观听金光之处,乃东胜神洲海东傲来小国之界,有一座花果山,山上有一仙石,石产一卵,见风化一石猴,在那里拜四方,眼运金光,射冲斗府。如今服饵水食[15],金光将潜息矣。”玉帝垂赐恩慈曰:“下方之物,乃天地精华所生,不足为异。”

那猴在山中,却会行走跳跃,食草木,饮涧泉,采山花,觅树果;与狼虫为伴,虎豹为群,獐鹿为友,猕猿为亲;夜宿石崖之下,朝游峰洞之中。真是“山中无甲子[16],寒尽不知年”。一朝天气炎热,与群猴避暑,都在松阴之下顽耍。你看他一个个:

跳树攀枝,采花觅果。抛弹子,邷麽儿[17]。跑[18]沙窝,砌宝塔。赶蜻蜓,扑蜡[19]。参老天,拜菩萨。扯葛藤,编草帓[20]。捉虱子,咬又掐。理毛衣,剔指甲。挨的挨,擦的擦。推的推,压的压。扯的扯,拉的拉。青松林下任他顽,绿水涧边随洗濯。

一群猴子耍了一会,却去那山涧中洗澡。见那股涧水奔流,真个似滚瓜涌溅。古云:“禽有禽言,兽有兽语。”众猴都道:“这股水不知是那里的水。我们今日赶闲无事,顺涧边往上溜头[21]寻看源流,耍子去耶!”喊一声,都拖男挈[22]女,唤弟呼兄,一齐跑来。顺涧爬山,直至源流之处,乃是一股瀑布飞泉。但见那:

一派白虹起,千寻雷浪飞[23]。海风吹不断,江月照还依。

冷气分青嶂,余流润翠微。潺湲[24]名瀑布,真似挂帘帷。

众猴拍手称扬道:“好水!好水!原来此处远通山脚之下,直接大海之波。”又道:“那一个有本事的,钻进去寻个源头出来,不伤身体者,我等即拜他为王。”连呼了三声,忽见丛杂中跳出一个石猴,应声高叫道:“我进去!我进去!”好猴!也是他:

今日芳名显,时来大运通。有缘居此地,天遣入仙宫。

你看他瞑目蹲身,将身一纵,径跳入瀑布泉中,忽睁睛抬头观看,那里边却无水无波,明明朗朗的一架桥梁。他住了身,定了神,仔细再看,原来是座铁板桥。桥下之水,冲贯于石窍[25]之间,倒挂流出去,遮闭了桥门。却又欠身上桥头,再走再看,却似有人家住处一般,真个好所在。但见那:

翠藓堆蓝,白云浮玉,光摇片片烟霞。虚窗静室,滑凳板生花。乳窟[26]龙珠倚挂,萦回满地奇葩。锅灶傍崖存火迹,樽罍[27]靠案见肴渣。石座石床真可爱,石盆石碗更堪夸。

又见那:

一竿两竿修竹,三点五点梅花。几树青松常带雨,浑然像个人家。

看罢多时,跳过桥中间,左右观看,只见正当中有一石碣[28]。碣上有一行楷书大字,镌着:

花果山福地,水帘洞洞天。

石猿喜不自胜,急抽身往外便走,复瞑目蹲身,跳出水外,打了两个呵呵道:“大造化[29]!大造化!”众猴把他围住,问道:“里面怎么样?水有多深?”石猴道:“没水!没水!原来是一座铁板桥。桥那边是一座天造地设的家当。”众猴道:“怎见得是个家当?”石猴笑道:“这股水乃是桥下冲贯石窍,倒挂下来遮闭门户的。桥边有花有树,乃是一座石房。房内有石锅、石灶、石碗、石盆、石床、石凳,中间一块石碣,上镌着‘花果山福地,水帘洞洞天’,真个是我们安身之处。里面且是宽阔,容得千百口老小。我们都进去住,也省得受老天之气。这里边:

刮风有处躲,下雨好存身。霜雪全无惧,雷声永不闻。

烟霞常照耀,祥瑞每蒸薰。松竹年年秀,奇花日日新。”

众猴听得,个个欢喜,都道:“你还先走,带我们进去,进去!”石猴却又瞑目蹲身,往里一跳,叫道:“都随我进来!进来!”那些猴有胆大的,都跳进去了;胆小的,一个个伸头缩颈,抓耳挠腮,大声叫喊,缠一会,也都进去了。跳过桥头,一个个抢盆夺碗,占灶争床,搬过来,移过去,正是猴性顽劣,再无一个宁时,只搬得力倦神疲方止。

石猿端坐上面道:“列位呵,‘人而无信,不知其可’[30]。你们才说有本事进得来,出得去,不伤身体者,就拜他为王。我如今进来又出去,出去又进来,寻了这一个洞天与列位安眠稳睡,各享成家之福,何不拜我为王?”众猴听说,即拱伏无违。一个个序齿排班[31],朝上礼拜,都称“千岁大王”。自此,石猿高登王位,将“石”字儿隐了,遂称“美猴王”。有诗为证。诗曰:

三阳交泰[32]产群生,仙石胞含日月精。

借卵化猴完大道,假他名姓配丹成。

内观不识因无相,外合明知作有形。

历代人人皆属此,称王称圣任纵横。

美猴王领一群猿猴、猕猴、马猴等,分派了君臣佐使,朝游花果山,暮宿水帘洞,合契同情[33],不入飞鸟之丛,不从走兽之类,独自为王,不胜欢乐。是以:

春采百花为饮食,夏寻诸果作生涯。

秋收芋栗延时节,冬觅黄精[34]度岁华。

美猴王享乐天真,何期有三五百载。一日,与群猴喜宴之间,忽然忧恼,堕下泪来。众猴慌忙罗拜道:“大王何为烦恼?”猴王道:“我虽在欢喜之时,却有一点儿远虑,故此烦恼。”众猴又笑道:“大王好不知足!我等日日欢会,在仙山福地,古洞神洲,不伏麒麟辖,不伏凤凰管,又不伏人间王位所拘束,自由自在,乃无量之福,为何远虑而忧也?”猴王道:“今日虽不归人王法律,不惧禽兽威严,将来年老血衰,暗中有阎王老子管着,一旦身亡,可不枉生世界之中,不得久注天人之内?”众猴闻此言,一个个掩面悲啼,俱以无常为虑。

只见那班部中,忽跳出一个通臂猿猴[35],厉声高叫道:“大王若是这般远虑,真所谓道心[36]开发也!如今五虫[37]之内,惟有三等名色[38]不伏阎王老子所管。”猴王道:“你知那三等人?”猿猴道:“乃是佛与仙与神圣三者,躲过轮回,不生不灭,与天地山川齐寿。”猴王道:“此三者居于何所?”猿猴道:“他只在阎浮世界[39]之中,古洞仙山之内。”猴王闻之,满心欢喜,道:“我明日就辞汝等下山,云游海角,远涉天涯,务必访此三者,学一个不老长生,常躲过阎君之难。”噫!这句话,顿教跳出轮回网,致使齐天大圣成。众猴鼓掌称扬,都道:“善哉!善哉!我等明日越岭登山,广寻些果品,大设筵宴送大王也。”

次日,众猴果去采仙桃,摘异果,刨山药,劚[40]黄精,芝兰香蕙,瑶草奇花,般般件件,整整齐齐,摆开石凳石桌,排列仙酒仙肴。但见那:

金丸珠弹,红绽黄肥。金丸珠弹,腊樱桃色真甘美;红绽黄肥,熟梅子味果香酸。鲜龙眼肉甜皮薄,火荔枝核小囊红。林檎碧实连枝献,枇杷缃苞[41]带叶擎。兔头梨子鸡心枣,消渴除烦更解酲[42]。香桃烂杏,美甘甘似玉液琼浆;脆李杨梅,酸荫荫如脂酥膏酪。红瓤黑子熟西瓜,四瓣黄皮大柿子。石榴裂破,丹砂粒现火晶珠;芋栗剖开,坚硬肉团金玛瑙。胡桃银杏可传茶,椰子葡萄能做酒。榛松榧柰[43]满盘盛,橘蔗柑橙盈案摆。熟煨山药,烂煮黄精。捣碎茯苓并薏苡[44],石锅微火漫炊羹。人间纵有珍馐味,怎比山猴乐更宁?

群猴尊美猴王上坐,各依齿序排于下边,一个个轮流上前,奉酒,奉花,奉果,痛饮了一日。次日,美猴王早起,教:“小的们,替我折些枯松,编作筏子,取个竹竿作篙,收拾些果品之类,我将去也。”果独自登筏,尽力撑开,飘飘荡荡,径向大海波中,趁天风,来渡南赡部洲地界。这一去,正是那:

天产仙猴道行隆,离山驾筏趁天风。

飘洋过海寻仙道,立志潜心建大功。

有分有缘休俗愿,无忧无虑会元龙。

料应必遇知音者,说破源流万法通。

也是他运至时来,自登木筏之后,连日东南风紧,将他送到西北岸前,乃是南赡部洲地界。持篙试水,偶得浅水,弃了筏子跳上岸来,只见海边上有人捕鱼打雁、穵[45]蛤淘盐。他走近前,弄个把戏装个虎[46],吓得那些人丢筐弃网,四散奔跑。将那跑不动的拿住一个,剥了他的衣裳,也学人穿在身上,摇摇摆摆,穿州过府,在于市廛[47]中,学人礼,学人话。朝餐夜宿,一心里访问佛仙神圣之道,觅个长生不老之方。见世人都是为名为利之徒,更无一个为身命者。正是那:

争名夺利几时休,早起迟眠不自由。

骑着驴骡思骏马,官居宰相望王侯。

只愁衣食耽[48]劳碌,何怕阎君就取勾[49]。

继子荫孙图富贵,更无一个肯回头。

猴王参访仙道,无缘得遇。在于南赡部洲,串长城,游小县,不觉八九年余。忽行至西洋大海,他想着海外必有神仙。独自个依前作筏,又飘过西海,直至西牛贺洲地界。登岸遍访多时,忽见一座高山秀丽,林麓幽深。他也不怕狼虫,不惧虎豹,登在山顶上观看。果是好山:

千峰排戟,万仞开屏。日映岚光轻锁翠,雨收黛色冷含青。瘦藤缠老树,古渡界幽程。奇花瑞草,修竹乔松。修竹乔松,万载常青欺[50]福地;奇花瑞草,四时不谢赛蓬瀛。幽鸟啼声近,源泉响溜清。重重谷壑芝兰绕,处处巉崖[51]苔藓生。起伏峦头龙脉好,必有高人隐姓名。

正观看间,忽闻得林深之处有人言语,急忙趋步,穿入林中,侧耳而听,原来是歌唱之声。歌曰:

观棋柯烂[52],伐木丁丁[53],云边谷口徐行。卖薪沽酒,狂笑自陶情。苍径秋高,对月枕松根,一觉天明。认旧林,登崖过岭,持斧断枯藤。收来成一担,行歌市上,易米三升。更无些子争竞,时价平平。不会机谋巧算,没荣辱,恬淡延生。相逢处,非仙即道,静坐讲《黄庭》。

美猴王听得此言,满心欢喜道:“神仙原来藏在这里!”即忙跳入里面,仔细再看,乃是一个樵子,在那里举斧砍柴。但看他打扮非常:

头上戴箬笠,乃是新笋初脱之箨[54]。身上穿布衣,乃是木绵捻就之纱。腰间系环绦,乃是老蚕口吐之丝。足下踏草履,乃是枯莎搓就之爽[55]。手执衠钢[56]斧,担挽火麻[57]绳。扳松劈枯树,争似此樵能。

猴王近前叫道:“老神仙!弟子起手[58]。”那樵汉慌忙丢了斧,转身回礼道:“不当人[59]!不当人!我拙汉衣食不全,怎敢当‘神仙’二字?”猴王道:“你不是神仙,如何说出神仙的话来?”樵夫道:“我说甚么神仙话?”猴王道:“我才来至林边,只听的你说:‘相逢处,非仙即道,静坐讲《黄庭》。’《黄庭》乃道德真言,非神仙而何?”

樵夫笑道:“实不瞒你说,这个词名做《满庭芳》,乃一神仙教我的。那神仙与我舍下相邻。他见我家事劳苦,日常烦恼,教我遇烦恼时,即把这词儿念念,一则散心,二则解困。我才有些不足处思虑,故此念念。不期被你听了。”猴王道:“你家既与神仙相邻,何不从他修行?学得个不老之方,却不是好?”樵夫道:“我一生命苦,自幼蒙父母养育,至八九岁才知人事,不幸父丧,母亲居孀。再无兄弟姊妹,只我一人,没奈何,早晚侍奉。如今母老,一发不敢抛离。却又田园荒芜,衣食不足,只得斫[60]两束柴薪,挑向市廛之间,货[61]几文钱,籴[62]几升米,自炊自造,安排些茶饭,供养老母,所以不能修行。”

猴王道:“据你说起来,乃是一个行孝的君子,向后必有好处。但望你指与我那神仙住处,却好拜访去也。”樵夫道:“不远,不远。此山叫做灵台方寸山[63],山中有座斜月三星洞[64]。那洞中有一个神仙,称名须菩提祖师。那祖师出去的徒弟也不计其数,见今还有三四十人从他修行。你顺那条小路儿,向南行七八里远近,即是他家了。”猴王用手扯住樵夫道:“老兄,你便同我去去。若还得了好处,决不忘你指引之恩。”樵夫道:“你这汉子,甚不通变。我方才这般与你说了,你还不省?假若我与你去了,却不误了我的生意?老母何人奉养?我要斫柴,你自去,自去。”

猴王听说,只得相辞。出深林,找上路径,过一山坡,约有七八里远,果然望见一座洞府。挺身观看,真好去处!但见:

烟霞散彩,日月摇光。千株老柏,万节修篁。千株老柏,带雨半空青冉冉;万节修篁,含烟一壑色苍苍。门外奇花布锦,桥边瑶草喷香。石崖突兀青苔润,悬壁高张翠藓长。时闻仙鹤唳,每见凤凰翔。仙鹤唳时,声振九皋[65]霄汉远;凤凰翔起,翎毛五色彩云光。玄猿白鹿随隐见,金狮玉象任行藏。细观灵福地,真个赛天堂!

又见那洞门紧闭,静悄悄杳无人迹。忽回头,见崖头立一石碑,约有三丈余高,八尺余阔,上有一行十个大字,乃是:

灵台方寸山,斜月三星洞。

美猴王十分欢喜,道:“此间人果是朴实,果有此山此洞。”看够多时,不敢敲门。且去跳上松枝梢头,摘松子吃了顽耍。

少顷间,只听得呀的一声,洞门开处,里面走出一个仙童,真是丰姿英伟,像貌清奇,比寻常俗子不同。但见他:

髽髻[66]双丝绾,宽袍两袖风。貌和身自别,心与相俱空。

物外长年客,山中永寿童。一尘全不染,甲子任翻腾。

那童子出得门来,高叫道:“甚么人在此搔扰?”猴王扑的跳下树来,上前躬身道:“仙童,我是个访道学仙之弟子,更不敢在此搔扰。”仙童笑道:“你是个访道的么?”猴王道:“是。”童子道:“我家师父正才下榻,登坛讲道,还未说出原由,就教我出来开门,说:‘外面有个修行的来了,可去接待接待。’想必就是你了?”猴王笑道:“是我,是我。”童子道:“你跟我进来。”

这猴王整衣端肃,随童子径入洞天深处观看:一层层深阁琼楼,一进进珠宫贝阙,说不尽那静室幽居。直至瑶台之下。见那菩提祖师端坐在台上,两边有三十个小仙侍立台下。果然是:

大觉金仙[67]没垢姿,西方妙相祖菩提。

不生不灭三三行,全气全神万万慈。

空寂自然随变化,真如本性任为之。

与天同寿庄严体,历劫[68]明心大法师。

美猴王一见,倒身下拜,磕头不计其数,口中只道:“师父!师父!我弟子志心朝礼!志心朝礼!”祖师道:“你是那方人氏?且说个乡贯姓名明白,再拜。”猴王道:“弟子乃东胜神洲傲来国花果山水帘洞人氏。”祖师喝令:“赶出去!他本是个撒诈捣虚之徒,那里修甚么道果!”猴王慌忙磕头不住道:“弟子是老实之言,决无虚诈。”祖师道:“你既老实,怎么说东胜神洲?那去处到我这里,隔两重大海,一座南赡部洲,如何就得到此?”猴王叩头道:“弟子飘洋过海,登界游方[69],有十数个年头,方才访到此处。”

祖师道:“既是逐渐行来的,也罢。你姓甚么?”猴王又道:“我无性。人若骂我,我也不恼;若打我,我也不嗔,只是赔个礼儿就罢了。一生无性。”祖师道:“不是这个性。你父母原来姓甚么?”猴王道:“我也无父母。”祖师道:“既无父母,想是树上生的?”猴王道:“我虽不是树上生,却是石里长的。我只记得花果山上有一块仙石,其年石破,我便生也。”

祖师闻言暗喜,道:“这等说,却是个天地生成的。你起来走走我看。”猴王纵身跳起,拐呀拐的,走了两遍。祖师笑道:“你身躯虽是鄙陋,却像个食松果的猢狲。我与你就身上取个姓氏,意思教你姓‘猢’。‘猢’字去了个兽旁,乃是个古月。古者,老也;月者,阴也。老阴不能化育[70]。教你姓‘狲’倒好。‘狲’字去了兽旁,乃是个子系[71]。子者,儿男也;系者,婴细也。正合婴儿之本论。教你姓‘孙’罢。”猴王听说,满心欢喜,朝上叩头道:“好!好!好!今日方知姓也。万望师父慈悲,既然有姓,再乞赐个名字,却好呼唤。”祖师道:“我门中有十二个字,分派起名,到你乃第十辈之小徒矣。”猴王道:“那十二个字?”祖师道:“乃‘广、大、智、慧、真、如、性、海、颖、悟、圆、觉’十二字。排到你,正当‘悟’字。与你起个法名叫做‘孙悟空’,好么?”猴王笑道:“好!好!好!自今就叫做孙悟空也!”正是:

鸿蒙初辟原无姓,打破顽空[72]须悟空。

毕竟不知向后修些甚么道果,且听下回分解。
\end{yuanwen}

\chapter[悟彻菩提真妙理\ 断魔归本合元神]{悟彻菩提真妙理\\断魔归本合元神}
\chapter[四海千山皆拱伏\ 九幽十类尽除名]{四海千山皆拱伏\\九幽十类尽除名}
\chapter[官封弼马心何足\ 名注齐天意未宁]{官封弼马心何足\\名注齐天意未宁}
\chapter[乱蟠桃大圣偷丹\ 反天宫诸神捉怪]{乱蟠桃大圣偷丹\\反天宫诸神捉怪}
\chapter[观音赴会问原因\ 小圣施威降大圣]{观音赴会问原因\\小圣施威降大圣}
\chapter[八卦炉中逃大圣\ 五行山下定心猿]{八卦炉中逃大圣\\五行山下定心猿}
\chapter[我佛造经传极乐\ 观音奉旨上长安]{我佛造经传极乐\\观音奉旨上长安}
\chapter[陈光蕊赴任逢灾\ 江流僧复仇报本]{陈光蕊赴任逢灾\\江流僧复仇报本}
\chapter[袁守诚妙算无私曲\ 老龙王拙计犯天条]{袁守诚妙算无私曲\\老龙王拙计犯天条}
\chapter[二将军宫门镇鬼\ 唐太宗地府还魂]{二将军宫门镇鬼\\唐太宗地府还魂}
\chapter[还受生唐王遵善果\ 度孤魂萧瑀正空门]{还受生唐王遵善果\\度孤魂萧瑀正空门}
\chapter[玄奘秉诚建大会\ 观音显相化金蝉]{玄奘秉诚建大会\\观音显相化金蝉}
\chapter[陷虎穴金星解厄\ 双叉岭伯钦留僧]{陷虎穴金星解厄\\双叉岭伯钦留僧}
\chapter[心猿归正\ 六贼无踪]{心猿归正\\六贼无踪}
\chapter[蛇盘山诸神暗佑\ 鹰愁涧意马收缰]{蛇盘山诸神暗佑\\鹰愁涧意马收缰}
\chapter[观音院僧谋宝贝\ 黑风山怪窃袈裟]{观音院僧谋宝贝\\黑风山怪窃袈裟}
\chapter[孙行者大闹黑风山\ 观世音收伏熊罴怪]{孙行者大闹黑风山\\观世音收伏熊罴怪}
\chapter[观音院唐僧脱难\ 高老庄大圣降魔]{观音院唐僧脱难\\高老庄大圣降魔}
\chapter[云栈洞悟空收八戒\ 浮屠山玄奘受心经]{云栈洞悟空收八戒\\浮屠山玄奘受心经}
\chapter[黄风岭唐僧有难\ 半山中八戒争先]{黄风岭唐僧有难\\半山中八戒争先}
\chapter[护法设庄留大圣\ 须弥灵吉定风魔]{护法设庄留大圣\\须弥灵吉定风魔}
\chapter[八戒大战流沙河\ 木叉奉法收悟净]{八戒大战流沙河\\木叉奉法收悟净}
\chapter[三藏不忘本\ 四圣试禅心]{三藏不忘本\\四圣试禅心}
\chapter[万寿山大仙留故友\ 五庄观行者窃人参]{万寿山大仙留故友\\五庄观行者窃人参}
\chapter[镇元仙赶捉取经僧\ 孙行者大闹五庄观]{镇元仙赶捉取经僧\\孙行者大闹五庄观}
\chapter[孙悟空三岛求方\ 观世音甘泉活树]{孙悟空三岛求方\\观世音甘泉活树}
\chapter[尸魔三戏唐三藏\ 圣僧恨逐美猴王]{尸魔三戏唐三藏\\圣僧恨逐美猴王}
\chapter[花果山群妖聚义\ 黑松林三藏逢魔]{花果山群妖聚义\\黑松林三藏逢魔}
\chapter[脱难江流来国土\ 承恩八戒转山林]{脱难江流来国土\\承恩八戒转山林}
\chapter[邪魔侵正法\ 意马忆心猿]{邪魔侵正法\\意马忆心猿}
\chapter[猪八戒义激猴王\ 孙行者智降妖怪]{猪八戒义激猴王\\孙行者智降妖怪}
\chapter[平顶山功曹传信\ 莲花洞木母逢灾]{平顶山功曹传信\\莲花洞木母逢灾}
\chapter[外道迷真性\ 元神助本心]{外道迷真性\\元神助本心}
\chapter[魔头巧算困心猿\ 大圣腾那骗宝贝]{魔头巧算困心猿\\大圣腾那骗宝贝}
\chapter[外道施威欺正性\ 心猿获宝伏邪魔]{外道施威欺正性\\心猿获宝伏邪魔}
\chapter[心猿正处诸缘伏\ 劈破旁门见月明]{心猿正处诸缘伏\\劈破旁门见月明}
\chapter[鬼王夜谒唐三藏\ 悟空神化引婴儿]{鬼王夜谒唐三藏\\悟空神化引婴儿}
\chapter[婴儿问母知邪正\ 金木参玄见假真]{婴儿问母知邪正\\金木参玄见假真}
\chapter[一粒金丹天上得\ 三年故主世间生]{一粒金丹天上得\\三年故主世间生}
\chapter[婴儿戏化禅心乱\ 猿马刀圭木母空]{婴儿戏化禅心乱\\猿马刀圭木母空}
\chapter[心猿遭火败\ 木母被魔擒]{心猿遭火败\\木母被魔擒}
\chapter[大圣殷勤拜南海\ 观音慈善缚红孩]{大圣殷勤拜南海\\观音慈善缚红孩}
\chapter[黑河妖孽擒僧去\ 西洋龙子捉鼍回]{黑河妖孽擒僧去\\西洋龙子捉鼍回}
\chapter[法身元运逢车力\ 心正妖邪度脊关]{法身元运逢车力\\心正妖邪度脊关}
\chapter[三清观大圣留名\ 车迟国猴王显法]{三清观大圣留名\\车迟国猴王显法}
\chapter[外道弄强欺正法\ 心猿显圣灭诸邪]{外道弄强欺正法\\心猿显圣灭诸邪}
\chapter[圣僧夜阻通天水\ 金木垂慈救小童]{圣僧夜阻通天水\\金木垂慈救小童}
\chapter[魔弄寒风飘大雪\ 僧思拜佛履层冰]{魔弄寒风飘大雪\\僧思拜佛履层冰}
\chapter[三藏有灾沉水宅\ 观音救难现鱼篮]{三藏有灾沉水宅\\观音救难现鱼篮}
\chapter[情乱性从因爱欲\ 神昏心动遇魔头]{情乱性从因爱欲\\神昏心动遇魔头}
\chapter[心猿空用千般计\ 水火无功难炼魔]{心猿空用千般计\\水火无功难炼魔}
\chapter[悟空大闹金【山兜】洞\ 如来暗示主人公]{悟空大闹金【山兜】洞\\如来暗示主人公}
\chapter[禅主吞餐怀鬼孕\ 黄婆运水解邪胎]{禅主吞餐怀鬼孕\\黄婆运水解邪胎}
\chapter[法性西来逢女国\ 心猿定计脱烟花]{法性西来逢女国\\心猿定计脱烟花}
\chapter[色邪淫戏唐三藏\ 性正修持不坏身]{色邪淫戏唐三藏\\性正修持不坏身}
\chapter[神狂诛草寇\ 道昧放心猿]{神狂诛草寇\\道昧放心猿}
\chapter[真行者落伽山诉苦\ 假猴王水帘洞誊文]{真行者落伽山诉苦\\假猴王水帘洞誊文}
\chapter[二心搅乱大乾坤\ 一体难修真寂灭]{二心搅乱大乾坤\\一体难修真寂灭}
\chapter[唐三藏路阻火焰山\ 孙行者一调芭蕉扇]{唐三藏路阻火焰山\\孙行者一调芭蕉扇}
\chapter[牛魔王罢战赴华筵\ 孙行者二调芭蕉扇]{牛魔王罢战赴华筵\\孙行者二调芭蕉扇}
\chapter[猪八戒助力破魔王\ 孙行者三调芭蕉扇]{猪八戒助力破魔王\\孙行者三调芭蕉扇}
\chapter[涤垢洗心惟扫塔\ 缚魔归正乃修身]{涤垢洗心惟扫塔\\缚魔归正乃修身}
\chapter[二僧荡怪闹龙宫\ 群圣除邪获宝贝]{二僧荡怪闹龙宫\\群圣除邪获宝贝}
\chapter[荆棘岭悟能努力\ 木仙庵三藏谈诗]{荆棘岭悟能努力\\木仙庵三藏谈诗}
\chapter[妖邪假设小雷音\ 四众皆遭大厄难]{妖邪假设小雷音\\四众皆遭大厄难}
\chapter[诸神遭毒手\ 弥勒缚妖魔]{诸神遭毒手\\弥勒缚妖魔}
\chapter[拯救驼罗禅性稳\ 脱离秽污道心清]{拯救驼罗禅性稳\\脱离秽污道心清}
\chapter[朱紫国唐僧论前世\ 孙行者施为三折肱]{朱紫国唐僧论前世\\孙行者施为三折肱}
\chapter[心主夜间修药物\ 君王筵上论妖邪]{心主夜间修药物\\君王筵上论妖邪}
\chapter[妖魔宝放烟沙火\ 悟空计盗紫金铃]{妖魔宝放烟沙火\\悟空计盗紫金铃}
\chapter[行者假名降怪犼\ 观音现像伏妖王]{行者假名降怪犼\\观音现像伏妖王}
\chapter[盘丝洞七情迷本\ 濯垢泉八戒忘形]{盘丝洞七情迷本\\濯垢泉八戒忘形}
\chapter[情因旧恨生灾毒\ 心主遭魔幸破光]{情因旧恨生灾毒\\心主遭魔幸破光}
\chapter[长庚传报魔头狠\ 行者施为变化能]{长庚传报魔头狠\\行者施为变化能}
\chapter[心猿钻透阴阳窍\ 魔王还归大道真]{心猿钻透阴阳窍\\魔王还归大道真}
\chapter[心神居舍魔归性\ 木母同降怪体真]{心神居舍魔归性\\木母同降怪体真}
\chapter[群魔欺本性\ 一体拜真如]{群魔欺本性\\一体拜真如}
\chapter[比丘怜子遣阴神\ 金殿识魔谈道德]{比丘怜子遣阴神\\金殿识魔谈道德}
\chapter[寻洞擒妖逢老寿\ 当朝正主救婴儿]{寻洞擒妖逢老寿\\当朝正主救婴儿}
\chapter[姹女育阳求配偶\ 心猿护主识妖邪]{姹女育阳求配偶\\心猿护主识妖邪}
\chapter[镇海寺心猿知怪\ 黑松林三众寻师]{镇海寺心猿知怪\\黑松林三众寻师}
\chapter[姹女求阳\ 元神护道]{姹女求阳\\元神护道}
\chapter[心猿识得丹头\ 姹女还归本性]{心猿识得丹头\\姹女还归本性}
\chapter[难灭伽持圆大觉\ 法王成正体天然]{难灭伽持圆大觉\\法王成正体天然}
\chapter[心猿妒木母\ 魔主计吞禅]{心猿妒木母\\魔主计吞禅}
\chapter[木母助威征怪物\ 金公施法灭妖邪]{木母助威征怪物\\金公施法灭妖邪}
\chapter[凤仙郡冒天止雨\ 孙大圣劝善施霖]{凤仙郡冒天止雨\\孙大圣劝善施霖}
\chapter[禅到玉华施法会\ 心猿木土授门人]{禅到玉华施法会\\心猿木土授门人}
\chapter[黄狮精虚设钉钯宴\ 金木土计闹豹头山]{黄狮精虚设钉钯宴\\金木土计闹豹头山}
\chapter[师狮授受同归一\ 盗道躔禅静九灵]{师狮授受同归一\\盗道躔禅静九灵}
\chapter[金平府元夜观灯\ 玄英洞唐僧供状]{金平府元夜观灯\\玄英洞唐僧供状}
\chapter[三僧大战青龙山\ 四星挟捉犀牛怪]{三僧大战青龙山\\四星挟捉犀牛怪}
\chapter[给孤园问古谈因\ 天竺国朝王遇偶]{给孤园问古谈因\\天竺国朝王遇偶}
\chapter[四僧宴乐御花园\ 一怪空怀情欲喜]{四僧宴乐御花园\\一怪空怀情欲喜}
\chapter[假合形骸擒玉兔\ 真阴归正会灵元]{假合形骸擒玉兔\\真阴归正会灵元}
\chapter[寇员外喜待高僧\ 唐长老不贪富惠]{寇员外喜待高僧\\唐长老不贪富惠}
\chapter[金酬外护遭魔毒\ 圣显幽魂救本原]{金酬外护遭魔毒\\圣显幽魂救本原}
\chapter[猿熟马驯方脱壳\ 功成行满见真如]{猿熟马驯方脱壳\\功成行满见真如}
\chapter[九九数完魔刬尽\ 三三行满道归根]{九九数完魔刬尽\\三三行满道归根}
\chapter[径回东土\ 五圣成真]{径回东土\\五圣成真}













\backmatter

后记
注释
西游故事设定集
宇宙周演示意图
五行八卦示意图
四大部洲示意图
天宫分布示意图
地府分布示意图
灵山分布示意图
周天种类图
神魔关系图谱
取经路线及八十一难手册

\end{document}