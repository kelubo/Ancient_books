% 千字文
% 千字文.tex

\documentclass[a4paper,12pt,UTF8,twoside]{ctexbook}

% 设置纸张信息。
\RequirePackage[a4paper]{geometry}
\geometry{
	%textwidth=138mm,
	%textheight=215mm,
	%left=27mm,
	%right=27mm,
	%top=25.4mm, 
	%bottom=25.4mm,
	%headheight=2.17cm,
	%headsep=4mm,
	%footskip=12mm,
	%heightrounded,
	inner=1in,
	outer=1.25in
}

% 设置字体,并解决显示难检字问题。
\xeCJKsetup{AutoFallBack=true}
\setCJKmainfont{SimSun}[BoldFont=SimHei, ItalicFont=KaiTi, FallBack=SimSun-ExtB]

% 目录 chapter 级别加点(.)。
\usepackage{titletoc}
\titlecontents{chapter}[0pt]{\vspace{3mm}\bf\addvspace{2pt}\filright}{\contentspush{\thecontentslabel\hspace{0.8em}}}{}{\titlerule*[8pt]{.}\contentspage}

% 设置 part 和 chapter 标题格式。
\ctexset{
	chapter/name={},
	chapter/number={}
}

% 设置古文原文格式。
\newenvironment{yuanwen}{\bfseries\zihao{4}}

% 设置署名格式。
\newenvironment{shuming}{\hfill\bfseries\zihao{4}}

% 注脚每页重新编号,避免编号过大。
\usepackage[perpage]{footmisc}

\title{\heiti\zihao{0} 千字文}
\author{周兴嗣}
\date{南朝梁 · 公元 502--549 年}

\begin{document}

\maketitle
\tableofcontents

\frontmatter
\chapter{前言}

南朝梁武帝时期(公元502--549年),员外散骑侍郎周兴嗣奉皇命从王羲之书法中选取1000个字,编纂成文,是为《千字文》。文中1000字本来不得有所重复,但周兴嗣在编纂文章时,却重复了一个“洁”字(洁、絜为同义异体字)。因此,《千字文》实际只运用了999字。

除周兴嗣版《千字文》之外,另有《续千字文》(侍其玮著)、《叙古千字文》(胡寅著)、《新千字文》(高占祥、赵缺著)等不同版本的《千字文》。其中,“高、赵版《新千字文》”被认为是超越“周版《千字文》”的真正经典之作。

《千字文》在中国古代的童蒙读物中,是一篇承上启下的作品。它那优美的文笔,华丽的辞藻,是其他任何一部童蒙读物都无法望其项背的。

《千字文》以儒学理论为纲、穿插诸多常识,用四字韵语写出,很适于儿童诵读,后来就成了中国古代教育史上最早、最成功的启蒙教材。宋明以后直至清末,《千字文》与《三字经》、《百家姓》一起,构成了我国人民最基础的“三、百、千”启蒙读物。旧有打油诗云:“学童三五并排坐,天地玄黄喊一年”,此之谓也!不仅汉民族用作儿童启蒙教材,一些兄弟民族也使用,甚至传到了日本。

同时,《千字文》在中国文化史上也有独特地位,是历代各流派书法家进行书法创作的重要载体。隋唐以后,凡著名书法家均有不同书体的《千字文》作品传世。 

公元六世纪初,南朝梁武帝时期在建业(今南京)刻印问世的《千字文》被公认为世界使用时间最长、影响最大的儿童启蒙识字课本,比唐代出现的《百家姓》和宋代编写的《三字经》还早。《千字文》可以说是千余年来最畅销、读者最广泛的读物之一。

《千字文》乃四言长诗,首尾连贯,音韵谐美。以“天地玄黄,宇宙洪荒”开头,“谓悟助者,焉哉手也”结尾。全文共250句,每四字一句,字不重复,句句押韵,前后贯通,内容有条不紊的介绍了天文、自然、修身养性、人伦道德、地理、历史、农耕、祭祀、园艺、饮食起居等各个方面。

\mainmatter

\begin{yuanwen}
天地玄\footnote{玄青,深黑色。古人认为天为玄青色,地为黄色。}黄,宇宙洪荒\footnote{混沌蒙昧的状态,此处指宽阔辽远。}。日月盈\footnote{充满。}昃\footnote{z\`e,太阳偏西。},辰\footnote{日、月、星的统称。}宿\footnote{xi\`u,中国古代将天上某些星的集合体称为宿,共二十八宿,分别为东方七宿,即角、亢、氐、房、心、尾、箕;南方七宿,即井、鬼、柳、星、张、翼、轸;西方七宿,即奎、娄、胃、昴、毕、觜、参;北方七宿,即斗、牛、女、虚、危、室、壁。}列张。
\end{yuanwen}

玄,天也;黄,地之色也;洪,大也;荒,远也;宇宙广大无边。

太阳有正有斜,月亮有缺有圆;星辰布满在无边的太空中。

\begin{yuanwen}
寒来暑往,秋收冬藏。

闰\footnote{一回归年的时间为365天5时48分46秒。阳历把一年定为365天,所余的时间约每四年积累成一天,加在二月里;农历把一年定为354天或355天,所余的时间约每三年积累成一个月,加在一年里。这样的方法,在历法上叫作闰。}余成岁,律吕\footnote{古代用竹管制成的校正乐律的器具,以管的长短来确定音的不同高度。从低音管算起,成奇数的六个管叫作“律”,又叫阳律;成偶数的六个管叫作“吕”,又叫阴律。后来用“律吕”作为音律的统称。}调阳。

云腾致雨,露结为霜。
\end{yuanwen}

寒暑循环变换,来了又去,去了又来;秋季里忙着收割,冬天里忙着储藏。

积累数年的闰余并成一个月,放在闰年里;古人用六律六吕来调节阴阳。

云气升到天空,遇冷就形成雨;露水碰上寒夜,很快凝结为霜。


天空青,大地黄,无边无际宇宙茫茫。日出日落,月圆月缺,星辰布满天上。
简评
开篇气势不凡,以浩茫天空为背景,以广袤大地作舞台,充分展示出宇宙神秘的内涵、悠远的历史和无穷的魅力。
相关链接
盘古开天地:传说天地本是混沌一片,盘古就生活在混沌中。有一天,盘古睁开眼睛,见眼前漆黑一团,遂用斧头朝四周砍去,天地于是分开。

寒来暑往,时令循环,秋天收获,冬季储藏。历法以闰余积成闰年,音乐用律吕调节阴阳。乌云升空化作雨,露水遇冷凝为霜。


时光的变换,无不遵循着一定的规律;自然的变化,亦是遵循着基本的法则。



\begin{yuanwen}
金生丽水,玉出昆冈。

剑号巨阙\footnote{qu\`e},珠称夜光。

果珍李柰\footnote{n\`ai},菜重芥姜。

海咸河淡,鳞潜羽翔。
\end{yuanwen}

金子生于金沙江底,玉石出自昆仑山岗。

最有名的宝剑叫“巨阙”,最贵重的明珠叫“夜光”。

果子中最珍贵的是李和柰,蔬菜中最看重的是芥和姜。

海水咸,河水淡;鱼儿在水中潜游,鸟儿在空中飞翔。

\begin{yuanwen}
龙师火帝,鸟官人皇。

始制文字,乃服衣裳。

推位让国,有虞陶唐。

吊民伐罪,周发殷汤。

坐朝问道,垂拱平章。

爱育黎首,臣伏戎羌。

遐\footnote{xi\'a}迩\footnote{\v{e}r}一体,率宾归王。

鸣凤在竹,白驹食场。

化被草木,赖及万方。
\end{yuanwen}

龙师、火帝、鸟官、人皇:这都是上古时代的帝皇官员。

有了仓颉,开始创造了文字,有了嫘祖,人们才穿起了遮身盖体的衣裳。

唐尧、虞舜英明无私,主动把君位禅让给功臣贤人。

安抚百姓,讨伐暴君,有周武王姬发和商君成汤。

贤君身坐朝廷,探讨治国之道,垂衣拱手,和大臣共商国事。

他们爱抚、体恤老百姓,四方各族人都归附向往。

远远近近都统一在一起,全都心甘情屈服贤君。

凤凰在竹林中欢鸣,白马在草场上觅食,国泰民安,处处吉祥。

贤君的教化覆盖大自然的一草一木,恩泽遍及天下百姓。

\begin{yuanwen}
盖此身发,四大五常。

恭惟鞠养,岂敢毁伤。

女慕贞洁,男效才良。

知过必改,得能莫忘。

人的身体发肤分属于“四大”,一言一动都要符合“五常”。

恭蒙父母亲生养爱护,不可有一丝一毫的毁坏损伤。

女子要思慕那些为人称道的贞妇洁女,男子要效法有德有才的贤人。

知道自己有过错,一定要改正;适合自己干的事,不要放弃。

罔谈彼短,靡恃\footnote{sh\`i}己长。

信使可覆,器欲难量。

墨悲丝染,诗赞羔羊。

景行维贤,克念作圣。

不要去谈论别人的短处,也不要依仗自己有长处就不思进取。

诚实的话要能经受时间的考验;器度要大,让人难以估量。

墨子为白丝染色不褪而悲泣,「诗经」中因此有「羔羊」篇传扬。

高尚的德行只能在贤人那里看到;要克制私欲,努力仿效圣人。

德建名立,形端表正。

空谷传声,虚堂习听。

祸因恶积,福缘善庆。

养成了好的道德,就会有好的名声;就像形体端庄,仪表也随之肃穆一样。

空旷的山谷中呼喊声传得很远,宽敞的厅堂里说话声非常清晰。

祸害是因为多次作恶积累而成,幸福是由于常年行善得到的奖赏。

尺璧非宝,寸阴是竞。

资父事君,曰严与敬。

孝当竭力,忠则尽命。

临深履薄,夙兴温凊。

一尺长的璧玉算不上宝贵,一寸短的光阴却值得去争取。

供养父亲,待奉国君,要做到认真、谨慎、恭敬。

对父母孝,要尽心竭力;对国君忠,要不惜献出生命。

要“如临深渊,如履薄冰”那样小心谨慎;要早起晚睡,让父母冬暖夏凉。

似兰斯馨,如松之盛。

能这样去做,德行就同兰花一样馨香,同青松一样茂盛。

川流不息,渊澄取映。

还能延及子孙,像大河川流不息;影响世人,像碧潭清澄照人。

容止若思,言辞安定。

仪态举止要庄重,看上去若有所思;言语措辞要稳重,显得从容沉静。

笃初诚美,慎终宜令。

无论修身、求学、重视开头固然不错,认真去做,有好的结果更为重要。

荣业所基,籍甚无竟。

有德能孝是事业显耀的基础,这样的人声誉盛大,传扬不已。

学优登仕,摄职从政。

学习出色并有余力,就可出仕做官,担任一定的职务,参与国家的政事。

存以甘棠,去而益咏。

召公活着时曾在甘棠树下理政,他过世后老百姓对他更加怀念歌咏。

乐殊贵贱,礼别尊卑。

选择乐曲要根据人的身份贵贱有所不同;采用礼节要按照人的地位高低有所区别。

上和下睦,夫唱妇随。

长辈和小辈要和睦相处,夫妇要一唱一随,协调和谐。

外受傅训,入奉母仪。

在外面要听从师长的教诲,在家里要遵守母亲的规范。

诸姑伯叔,犹子比儿。

对待姑姑、伯伯、叔叔等长辈,要像是他们的亲生子女一样。

孔怀兄弟,同气连枝。

兄弟之间要非常相爱,因为同受父母血气,犹如树枝相连。

交友投分,切磨箴\footnote{zh\=en}规。

结交朋友要意相投,学习上切磋琢磨,品行上互相告勉。

仁慈隐恻,造次弗离。

仁义、慈爱,对人的恻隐之心,在最仓促、危急的情况下也不能抛离。

节义廉退,颠沛匪亏。

气节、正义、廉洁、谦让的美德,在最穷困潦倒的时候也不可亏缺。

性静情逸,心动神疲。

品性沉静淡泊,情绪就安逸自在;内心浮躁好动,精神就疲惫困倦。

守真志满,逐物意移。

保持纯洁的天性,就会感到满足;追求物欲享受,天性就会转移改变。

坚持雅操,好爵自縻\footnote{m\'i}。

坚持高尚铁情操,好的职位自然会为你所有。

都邑华夏,东西二京。

古代的都城华美壮观,有东京洛阳和西京长安。

背邙\footnote{m\'ang}面洛,浮渭据泾。

东京洛阳背靠北邙山,南临洛水;西京长安左跨渭河,右依泾水。

宫殿盘郁,楼观飞惊。

宫殿盘旋曲折,重重迭迭;楼阁高耸如飞,触目惊心。

图写禽兽,画彩仙灵。

宫殿上绘着各种飞禽走兽,描画出五彩的天仙神灵。

丙舍旁启,甲帐对楹。

正殿两边的配殿从侧面开启,豪华的账幕对着高高的楹柱。

肆筵\footnote{y\'an}设席,鼓瑟吹笙。

宫殿中大摆宴席,乐人吹笙鼓瑟,一片歌舞升平的景象。

升阶纳陛,弁\footnote{bi\`an}转疑星。

登上台阶进入殿堂的文武百官,帽子团团转,像满天的星星。

右通广内,左达承明。

右面通向用以藏书的广内殿,左面到达朝臣休息的承明殿。

既集坟典,亦聚群英。

这里收藏了很多的典籍名著,也集着成群的文武英才。

杜稿钟隶,漆书壁经。

书殿中有杜度的草书、钟繇的隶书,还有漆写的古籍和孔壁中的经典。

府罗将相,路侠槐卿。

宫廷内将想依次排成两列,宫廷外大夫公卿夹道站立。

户封八县,家给千兵。

他们每户有八县之广的封地,配备成千以上的士兵。

高冠陪辇,驱毂\footnote{g\v{u}}振缨。

他们戴着高高的官帽,陪着皇帝出游,驾着车马,帽带飘舞着,好不威风。

世禄侈\footnote{ch\v{i}}富,车驾肥轻。

他们的子孙世代领受俸禄,奢侈豪富,出门时轻车肥马,春风得意。

策功茂实,勒碑刻铭。

朝廷还详尽确实地记载他们的功德,刻在碑石上流传后世。

磻\footnote{p\'an}溪伊尹,佐时阿衡。

周武王磻溪遇吕尚,尊他为“太公望”;伊尹辅佐时政,商汤王封他为“阿衡”。

奄\footnote{y\v{a}n}宅曲阜,微旦孰营。

周成王占领了古奄国曲阜一带地面,要不是周公旦辅政哪里能成?

桓公匡合,济弱扶倾。

齐桓公九次会合诸侯,出兵援助势单力薄和面临危亡的诸侯小国。

绮回汉惠,说感武丁。

汉惠帝做太子时靠绮里季才幸免废黜,商君武丁感梦而得贤相传说。

俊乂\footnote{y\`i}密勿,多士寔\footnote{sh\'i}宁。

能人治政勤勉努力,全靠许多这样的贤士,国家才富强安宁。

晋楚更霸,赵魏困横。

晋、楚两国在齐之后称霸,赵、魏两国因连横而受困于秦。

假途灭虢\footnote{gu\'o},践土会盟。

晋献公向虞国借路去消灭虢国;晋文公在践土与诸侯会盟,推为盟主。

何遵约法,韩弊烦刑。

萧何遵循简约刑法的精神制订九律,韩非却受困于自己所主张的严酷刑法。

起翦颇牧,用军最精。

秦将白起、王翦,赵将廉颇、李牧,带兵打仗最为高明。

宣威沙漠,驰誉丹青。

他们的声威远传到沙漠边地,美誉和画像一起流芳后代。

九州禹迹,百郡秦并。

九州处处有留有大禹治水的足迹,全国各郡在秦并六国后归于统一。

岳宗泰岱,禅主云亭。

五岳中人们最尊崇东岳泰山,历代帝王都在云山和亭山主持禅礼。

雁门紫塞,鸡田赤诚。

名关有北疆雁门,要塞有万里长城,驿站有边地鸡田,奇山有天台赤城。

昆池碣石,钜\footnote{j\`u}野洞庭。

赏池赴昆明滇池,观海临河北碣石,看泽去山东钜野,望湖上湖南洞庭。

旷远绵邈,岩岫\footnote{xi\`u}杳\footnote{y\v{a}o}冥。

江河源远流长,湖海宽广无边。名山奇谷幽深秀丽,气象万千。

治本于农,务兹稼穑\footnote{s\`e}。

治国的根本在发展农业,要努力做好播种收获这些农活。

俶\footnote{ch\`u}载南亩,我艺黍稷\footnote{j\`i}。

一年的农活该开始干起来了,我种上小米,又种上高粱。

税熟贡新,劝赏黜陟\footnote{zh\`i}。

收获季节,用刚熟的新谷交纳税粮,官府应按农户的贡献大小给予奖励或处罚。

孟轲敦素,史鱼秉直。

孟轲夫子崇尚纯洁,史官子鱼秉性刚直。

庶几中庸,劳谦谨敕\footnote{ch\`i}。

做人要尽可能合乎中庸的标准,勤奋、谦逊、谨慎,懂得规劝告诫自己。

聆音察理,鉴貌辨色。

听别人说话,要仔细审察是否合理;看别人面孔,要小心辨析他的脸色。

贻厥\footnote{ju\'e}嘉猷\footnote{y\'ou},勉其祗\footnote{zh\=i}植。

要给人家留下正确高明的忠告或建议,勉励别人谨慎小心地处世立身。

省\footnote{x\v{i}ng}躬讥诫,宠增抗极。

听到别人的讥讽告诫,要反省自身;备受恩宠不要得意忘形,对抗权尊。

殆辱近耻,林皋\footnote{g\=ao}幸即。

知道有危险耻辱的事快要发生,还不如归隐山林为好。

两疏见机,解组谁逼。

疏广疏受预见到危患的苗头才告老还乡,哪里有谁逼他们除下官印?

索居闲处,沉默寂寥。

离群独居,悠闲度日,整天不用多费唇舌,清静无为岂不是好事。

求古寻论,散虑逍遥。

想想古人的话,翻翻古人的书,消往日的忧虑,乐得逍遥舒服。

欣奏累遣,戚谢欢招。

轻松的事凑到一起,费力的事丢在一边,消除不尽的烦恼,得来无限的快乐。

渠荷的历,园莽抽条。

池里的荷花开得光润鲜艳,园中的草木抽出条条嫩枝。

枇杷晚翠,梧桐蚤凋。

枇杷到了岁晚还是苍翠欲滴,梧桐刚刚交秋就早早地凋谢了。

陈根委翳\footnote{y\`i},落叶飘摇。

陈根老树枯倒伏,落叶在秋风里四处飘荡。

游鹍\footnote{k\=un}独运,淩摩绛霄。

寒秋之中,鲲鹏独自高飞,直冲布满彩霞的云霄。

耽读玩市,寓目囊箱。

汉代王充在街市上沉迷留恋于读书,眼睛注视的都是书袋和书箱。

易輶\footnote{y\'ou}攸畏,属耳垣\footnote{yu\'an}墙。

说话最怕旁若无人,毫无禁忌;要留心隔着墙壁有人在贴耳偷听。

具膳餐饭,适口充肠。

安排一日三餐的膳食,要适合各位的口味,能让大家吃饱。

饱饫\footnote{y\`u}烹宰,饥厌糟糠。

的时候自然满足于大鱼大肉,饿的时候应当满足于粗菜淡饭。

亲戚故旧,老少异粮。

亲属、朋友会面要盛情款待,老人、小孩的食物应和自己不同。

妾御绩纺,侍巾帷房。

小妾婢女要管理好家务,尽心恭敬地服待好主人。

纨\footnote{w\'an}扇圆洁,银烛炜煌。

绢制的团扇像满月一样又白又圆,银色的烛台上烛火辉煌。

昼眠夕寐,蓝笋象床。

白日小憩,晚上就寝,有青篾编成的竹席和象牙雕屏的床榻。

弦歌酒宴,接杯举觞。

奏着乐,唱着歌,摆酒开宴;接过酒杯,开怀畅饮。

矫手顿足,悦豫且康。

情不自禁地手舞足蹈,真是又快乐又安康。

嫡后嗣续,祭祀烝尝。

子孙继承了祖先的基业,一年四季的祭祀大礼不能疏忘。

稽颡\footnote{s\v{a}ng}再拜,悚惧恐惶。

跪着磕头,拜了又拜;礼仪要周全恭敬,心情要悲痛虔诚。

笺牒简要,顾答审详。

给别人写信要简明扼要,回答别人问题要详细周全。

骸垢想浴,执热愿凉。

身上有了污垢,就想洗澡,好比手上拿着烫的东西就希望有风把它吹凉。

驴骡犊特,骇跃超骧。

家里有了灾祸,连牲畜都会受惊,狂蹦乱跳,东奔西跑。

诛斩贼盗 (zhūzhǎnzéidào),捕获叛亡。

对抢劫、偷窃、反叛、逃亡的人要严厉惩罚,该抓的抓,该杀的杀。

布射僚丸,嵇\footnote{j\=i}琴阮箫。

吕布擅长射箭,宜僚有弄丸的绝活,嵇康善于弹琴,阮籍能撮口长啸。

恬笔伦纸,钧巧任钓。

蒙恬造出毛笔,蔡伦发明造纸,马钧巧制水车,任公子垂钓大鱼。

释纷利俗,并皆佳妙。

他们的技艺有的解人纠纷,有的方便群众,都高明巧妙,为人称道。

毛施淑姿,工颦妍笑。

毛嫱、西施年轻美貌,哪怕皱着眉头,也像美美的笑。

年矢每催,曦晖朗曜。

可惜青春易逝,岁月匆匆催人渐老,只有太阳的光辉永远朗照。

璿\footnote{xu\'an}玑悬斡\footnote{w\`o},晦魄环照。

高悬的北斗随着四季变换转动,明晦的月光洒遍人间每个角落。

指薪修祜\footnote{h\`u},永绥吉劭\footnote{sh\`ao}。

行善积德才能像薪尽火传那样精神长存,子孙安康全靠你留下吉祥的忠告。

矩步引领,俯仰廊庙。

如此心地坦然,方可以昂头迈步,应付朝廷委以的重任。

束带矜庄,徘徊瞻眺。

如此无愧人生,尽可以整束衣冠,庄重从容地高瞻远望。

孤陋寡闻,愚蒙等诮\footnote{qi\`ao}。

这些道理孤陋寡闻就不会明白,只能和愚味无知的人一样空活一世,让人耻笑。

谓语助者,焉哉乎也。

编完「千字文」乌发皆白,最后剩下“焉、哉、乎、也”这几个语气助词。
\end{yuanwen}

\end{document}