% 千字文
% 千字文.tex

\documentclass[a4paper,12pt,UTF8,twoside]{ctexbook}

% 设置纸张信息。
\RequirePackage[a4paper]{geometry}
\geometry{
	%textwidth=138mm,
	%textheight=215mm,
	%left=27mm,
	%right=27mm,
	%top=25.4mm, 
	%bottom=25.4mm,
	%headheight=2.17cm,
	%headsep=4mm,
	%footskip=12mm,
	%heightrounded,
	inner=1in,
	outer=1.25in
}

% 设置字体,并解决显示难检字问题。
\xeCJKsetup{AutoFallBack=true}
\setCJKmainfont{SimSun}[BoldFont=SimHei, ItalicFont=KaiTi, FallBack=SimSun-ExtB]

% 目录 chapter 级别加点(.)。
\usepackage{titletoc}
\titlecontents{chapter}[0pt]{\vspace{3mm}\bf\addvspace{2pt}\filright}{\contentspush{\thecontentslabel\hspace{0.8em}}}{}{\titlerule*[8pt]{.}\contentspage}

% 设置 part 和 chapter 标题格式。
\ctexset{
	chapter/name={},
	chapter/number={}
}

% 设置古文原文格式。
\newenvironment{yuanwen}{\bfseries\zihao{4}}

% 设置署名格式。
\newenvironment{shuming}{\hfill\bfseries\zihao{4}}

\title{\heiti\zihao{0} 千字文}
\author{周兴嗣}
\date{南朝梁 · 公元 502 - 549 年}

\begin{document}

\maketitle
\tableofcontents

\frontmatter
\chapter{前言}

南朝梁武帝时期(公元 502 - 549 年),员外散骑侍郎周兴嗣奉皇命从王羲之书法中选取1000个字,编纂成文,是为《千字文》。文中1000字本来不得有所重复,但周兴嗣在编纂文章时,却重复了一个“洁”字(洁、絜为同义异体字)。因此,《千字文》实际只运用了999字。

除周兴嗣版《千字文》之外,另有《续千字文》(侍其玮著)、《叙古千字文》(胡寅著)、《新千字文》(高占祥、赵缺著)等不同版本的《千字文》。其中,“高、赵版《新千字文》”被认为是超越“周版《千字文》”的真正经典之作。

《千字文》在中国古代的童蒙读物中,是一篇承上启下的作品。它那优美的文笔,华丽的辞藻,是其他任何一部童蒙读物都无法望其项背的。

《千字文》以儒学理论为纲、穿插诸多常识,用四字韵语写出,很适于儿童诵读,后来就成了中国古代教育史上最早、最成功的启蒙教材。宋明以后直至清末,《千字文》与《三字经》、《百家姓》一起,构成了我国人民最基础的“三、百、千”启蒙读物。旧有打油诗云:“学童三五并排坐,天地玄黄喊一年”,此之谓也!不仅汉民族用作儿童启蒙教材,一些兄弟民族也使用,甚至传到了日本。

同时,《千字文》在中国文化史上也有独特地位,是历代各流派书法家进行书法创作的重要载体。隋唐以后,凡著名书法家均有不同书体的《千字文》作品传世。 

公元六世纪初,南朝梁武帝时期在建业(今南京)刻印问世的《千字文》被公认为世界使用时间最长、影响最大的儿童启蒙识字课本,比唐代出现的《百家姓》和宋代编写的《三字经》还早。《千字文》可以说是千余年来最畅销、读者最广泛的读物之一。

《千字文》乃四言长诗,首尾连贯,音韵谐美。以“天地玄黄,宇宙洪荒”开头,“谓悟助者,焉哉手也”结尾。全文共250句,每四字一句,字不重复,句句押韵,前后贯通,内容有条不紊的介绍了天文、自然、修身养性、人伦道德、地理、历史、农耕、祭祀、园艺、饮食起居等各个方面。

\mainmatter

\chapter{正文}

\begin{yuanwen}

天地玄黄,宇宙洪荒。

日月盈昃\footnote{z\`e},辰宿\footnote{xi\`u}列张。

寒来暑往,秋收冬藏。

闰余成岁,律吕调阳。

云腾致雨,露结为霜。

\end{yuanwen}

玄,天也;黄,地之色也;洪,大也;荒,远也;宇宙广大无边。

太阳有正有斜,月亮有缺有圆;星辰布满在无边的太空中。

寒暑循环变换,来了又去,去了又来;秋季里忙着收割,冬天里忙着储藏。

积累数年的闰余并成一个月,放在闰年里;古人用六律六吕来调节阴阳。

云气升到天空,遇冷就形成雨;露水碰上寒夜,很快凝结为霜。

\begin{yuanwen}

金生丽水,玉出昆冈。

剑号巨阙\footnote{qu\`e},珠称夜光。

果珍李柰\footnote{n\`ai},菜重芥姜。

海咸河淡,鳞潜羽翔。

\end{yuanwen}

金子生于金沙江底,玉石出自昆仑山岗。

最有名的宝剑叫“巨阙”,最贵重的明珠叫“夜光”。

果子中最珍贵的是李和柰,蔬菜中最看重的是芥和姜。

海水咸,河水淡;鱼儿在水中潜游,鸟儿在空中飞翔。

\begin{yuanwen}

龙师火帝,鸟官人皇。

始制文字,乃服衣裳。

推位让国,有虞陶唐。

吊民伐罪,周发殷汤。

坐朝问道,垂拱平章。

爱育黎首,臣伏戎羌。

遐\footnote{xi\'a}迩\footnote{\v{e}r}一体,率宾归王。

鸣凤在竹,白驹食场。

化被草木,赖及万方。

\end{yuanwen}

龙师、火帝、鸟官、人皇:这都是上古时代的帝皇官员。

有了仓颉,开始创造了文字,有了嫘祖,人们才穿起了遮身盖体的衣裳。

唐尧、虞舜英明无私,主动把君位禅让给功臣贤人。

安抚百姓,讨伐暴君,有周武王姬发和商君成汤。

贤君身坐朝廷,探讨治国之道,垂衣拱手,和大臣共商国事。

他们爱抚、体恤老百姓,四方各族人都归附向往。

远远近近都统一在一起,全都心甘情屈服贤君。

凤凰在竹林中欢鸣,白马在草场上觅食,国泰民安,处处吉祥。

贤君的教化覆盖大自然的一草一木,恩泽遍及天下百姓。

\begin{yuanwen}

盖此身发,四大五常。

人的身体发肤分属于“四大”,一言一动都要符合“五常”。

恭惟鞠养,岂敢毁伤。

恭蒙父母亲生养爱护,不可有一丝一毫的毁坏损伤。

女慕贞洁,男效才良。

女子要思慕那些为人称道的贞妇洁女,男子要效法有德有才的贤人。

知过必改,得能莫忘。

知道自己有过错,一定要改正;适合自己干的事,不要放弃。

罔谈彼短 (wǎngtánbǐduǎn),靡恃己长 (míshìjǐcháng)。

不要去谈论别人的短处,也不要依仗自己有长处就不思进取。

信使可覆 (xìnshǐkěfù),器欲难量 (qìyùnánliáng)。

诚实的话要能经受时间的考验;器度要大,让人难以估量。

墨悲丝染 (mòbēisīrǎn),诗赞羔羊 (shīzàngāoyáng)。

墨子为白丝染色不褪而悲泣,「诗经」中因此有「羔羊」篇传扬。

景行维贤 (jǐngxíngwéixián),克念作圣 (kèniànzuòshèng)。

高尚的德行只能在贤人那里看到;要克制私欲,努力仿效圣人。

德建名立 (déjiànmínglì),形端表正 (xíngduānbiǎozhèng)。

养成了好的道德,就会有好的名声;就像形体端庄,仪表也随之肃穆一样。

空谷传声 (kōnggǔchuánshēng),虚堂习听 (xūtángxítīng)。

空旷的山谷中呼喊声传得很远,宽敞的厅堂里说话声非常清晰。

祸因恶积 (huòyīnèjí),福缘善庆 (fúyuánshànqìng)。

祸害是因为多次作恶积累而成,幸福是由于常年行善得到的奖赏。

尺璧非宝 (chǐbìfēibǎo),寸阴是竞 (cùnyīnshìjìng)。

一尺长的璧玉算不上宝贵,一寸短的光阴却值得去争取。

资父事君 (zīfùshìjūn),曰严与敬 (yuēyányǔjìng)。

供养父亲,待奉国君,要做到认真、谨慎、恭敬。

孝当竭力 (xiàodāngjiélì),忠则尽命 (zhōngzéjìnmìng)。

对父母孝,要尽心竭力;对国君忠,要不惜献出生命。

临深履薄 (línshēnlǚbáo),夙兴温凊 (sùxīngwēnqìng)。

要“如临深渊,如履薄冰”那样小心谨慎;要早起晚睡,让父母冬暖夏凉。

似兰斯馨 (sìlánsīxīn),如松之盛 (rúsōngzhīshèng)。

能这样去做,德行就同兰花一样馨香,同青松一样茂盛。

川流不息 (chuānliúbùxī),渊澄取映 (yuānchéngqǔyìng)。

还能延及子孙,像大河川流不息;影响世人,像碧潭清澄照人。

容止若思 (róngzhǐruòsī),言辞安定 (yáncíāndìng)。

仪态举止要庄重,看上去若有所思;言语措辞要稳重,显得从容沉静。

笃初诚美 (dǔchūchéngměi),慎终宜令 (shènzhōngyìlìng)。

无论修身、求学、重视开头固然不错,认真去做,有好的结果更为重要。

荣业所基 (róngyèsuǒjī),籍甚无竟 (jíshènwújìng)。

有德能孝是事业显耀的基础,这样的人声誉盛大,传扬不已。

学优登仕 (xuéyōudēngshì),摄职从政 (shèzhǐcóngzhèng)。

学习出色并有余力,就可出仕做官,担任一定的职务,参与国家的政事。

存以甘棠 (cúnyǐgāntáng),去而益咏 (qùéryìyǒng)。

召公活着时曾在甘棠树下理政,他过世后老百姓对他更加怀念歌咏。

乐殊贵贱 (lèshūguìjiàn),礼别尊卑 (lǐbiézūnbēi)。

选择乐曲要根据人的身份贵贱有所不同;采用礼节要按照人的地位高低有所区别。

上和下睦 (shànghéxiàmù),夫唱妇随 (fūchàngfùsuí)。

长辈和小辈要和睦相处,夫妇要一唱一随,协调和谐。

外受傅训 (wàishòufùxùn),入奉母仪 (rùfèngmǔyí)。

在外面要听从师长的教诲,在家里要遵守母亲的规范。

诸姑伯叔 (zhūgūbóshú),犹子比儿 (yōuzǐbǐér)。

对待姑姑、伯伯、叔叔等长辈,要像是他们的亲生子女一样。

孔怀兄弟 (kǒnghuáixiōngdì),同气连枝 (tóngqìliánzhī)。

兄弟之间要非常相爱,因为同受父母血气,犹如树枝相连。

交友投分 (jiāoyǒutóufēn),切磨箴规 (qiēmózhēnguī)。

结交朋友要意相投,学习上切磋琢磨,品行上互相告勉。

仁慈隐恻 (réncíyǐncè),造次弗离 (zàocìfúlí)。

仁义、慈爱,对人的恻隐之心,在最仓促、危急的情况下也不能抛离。

节义廉退 (jiéyìliántuì),颠沛匪亏 (diānpèifěikuī)。

气节、正义、廉洁、谦让的美德,在最穷困潦倒的时候也不可亏缺。

性静情逸 (xìngjìngqíngyì),心动神疲 (xīndòngshénpí)。

品性沉静淡泊,情绪就安逸自在;内心浮躁好动,精神就疲惫困倦。

守真志满 (shǒuzhēnzhìmǎn),逐物意移 (zhúwùyìyí)。

保持纯洁的天性,就会感到满足;追求物欲享受,天性就会转移改变。

坚持雅操 (jiānchíyǎcāo),好爵自縻 (hǎojuézìmí)。

坚持高尚铁情操,好的职位自然会为你所有。

都邑华夏 (dūyìhuáxià),东西二京 (dōngxīèrjīng)。

古代的都城华美壮观,有东京洛阳和西京长安。

背邙面洛 (bèimángmiànluò),浮渭据泾 (fúwèijùjīng)。

东京洛阳背靠北邙山,南临洛水;西京长安左跨渭河,右依泾水。

宫殿盘郁 (gōngdiànpányù),楼观飞惊 (lóuguānfēijīng)。

宫殿盘旋曲折,重重迭迭;楼阁高耸如飞,触目惊心。

图写禽兽 (túxiěqínshòu),画彩仙灵 (huàcǎixiānlíng)。

宫殿上绘着各种飞禽走兽,描画出五彩的天仙神灵。

丙舍旁启 (bǐngshèpángqǐ),甲帐对楹 (jiázhàngduìyíng)。

正殿两边的配殿从侧面开启,豪华的账幕对着高高的楹柱。

肆筵设席 (sìyánshèxí),鼓瑟吹笙 (gǔsèchuīshēng)。

宫殿中大摆宴席,乐人吹笙鼓瑟,一片歌舞升平的景象。

升阶纳陛 (shēngjiēnàbì),弁转疑星 (biànzhuànyíxīng)。

登上台阶进入殿堂的文武百官,帽子团团转,像满天的星星。

右通广内 (yòutōngguǎngnèi),左达承明 (zuǒdáchéngmíng)。

右面通向用以藏书的广内殿,左面到达朝臣休息的承明殿。

既集坟典 (jìjíféndiǎn),亦聚群英 (yìjùqúnyīng)。

这里收藏了很多的典籍名著,也集着成群的文武英才。

杜稿钟隶 (dùgǎozhōnglì),漆书壁经 (qīshūbìjīng)。

书殿中有杜度的草书、钟繇的隶书,还有漆写的古籍和孔壁中的经典。

府罗将相 (fǔluójiāngxiàng),路侠槐卿 (lùxiáhuáiqīng)。

宫廷内将想依次排成两列,宫廷外大夫公卿夹道站立。

户封八县 (hùfēngbāxiàn),家给千兵 (jiājǐqiānbīng)。

他们每户有八县之广的封地,配备成千以上的士兵。

高冠陪辇 (gāoguānpéiniǎn),驱毂振缨 (qūgǔzhènyīng)。

他们戴着高高的官帽,陪着皇帝出游,驾着车马,帽带飘舞着,好不威风。

世禄侈富 (shìlùchǐfù),车驾肥轻 (chējiàféiqīng)。

他们的子孙世代领受俸禄,奢侈豪富,出门时轻车肥马,春风得意。

策功茂实 (cègōngmàoshí),勒碑刻铭 (lèbēikèmíng)。

朝廷还详尽确实地记载他们的功德,刻在碑石上流传后世。

磻溪伊尹 (pánxīyīyǐn),佐时阿衡 (zuǒshíāhéng)。

周武王磻溪遇吕尚,尊他为“太公望”;伊尹辅佐时政,商汤王封他为“阿衡”。

奄宅曲阜 (yǎnzháiqūfù),微旦孰营 (wēidànshúyíng)。

周成王占领了古奄国曲阜一带地面,要不是周公旦辅政哪里能成?

桓公匡合 (huángōngkuānghé),济弱扶倾 (jìruòfúqīng)。

齐桓公九次会合诸侯,出兵援助势单力薄和面临危亡的诸侯小国。

绮回汉惠 (qǐhuíhànhuì),说感武丁 (yuègǎnwǔdīng)。

汉惠帝做太子时靠绮里季才幸免废黜,商君武丁感梦而得贤相传说。

俊乂密勿 (jùnyìmìwù),多士寔宁 (duōshìshíníng)。

能人治政勤勉努力,全靠许多这样的贤士,国家才富强安宁。

晋楚更霸 (jìnchǔgēngbà),赵魏困横 (zhàowèikùnhéng)。

晋、楚两国在齐之后称霸,赵、魏两国因连横而受困于秦。

假途灭虢 (jiǎtúmièguó),践土会盟 (jiàntǔhuìméng)。

晋献公向虞国借路去消灭虢国;晋文公在践土与诸侯会盟,推为盟主。

何遵约法 (hézūnyuēfǎ),韩弊烦刑 (hánbìfánxíng)。

萧何遵循简约刑法的精神制订九律,韩非却受困于自己所主张的严酷刑法。

起翦颇牧 (qǐjiǎnpōmù),用军最精。

秦将白起、王翦,赵将廉颇、李牧,带兵打仗最为高明。

宣威沙漠 (xuānwēishāmò),驰誉丹青 (chíyùdānqīng)。

他们的声威远传到沙漠边地,美誉和画像一起流芳后代。

九州禹迹 (jiǔzhōuyǔjì),百郡秦并 (bǎijùnqínbìng)。

九州处处有留有大禹治水的足迹,全国各郡在秦并六国后归于统一。

岳宗泰岱 (yuèzōngtàidài),禅主云亭 (chánzhǔyúntíng)。

五岳中人们最尊崇东岳泰山,历代帝王都在云山和亭山主持禅礼。

雁门紫塞,鸡田赤诚。

名关有北疆雁门,要塞有万里长城,驿站有边地鸡田,奇山有天台赤城。

昆池碣石 (kūnchíjiéshí),钜野洞庭 (jùyědòngtíng)。

赏池赴昆明滇池,观海临河北碣石,看泽去山东钜野,望湖上湖南洞庭。

旷远绵邈 (kuàngyuǎnmiánmiǎo),岩岫杳冥 (yánxiùyǎomíng)。

江河源远流长,湖海宽广无边。名山奇谷幽深秀丽,气象万千。

治本于农 (zhìběnyúnóng),务兹稼穑 (wùzījiàsè)。

治国的根本在发展农业,要努力做好播种收获这些农活。

俶载南亩 (chùzǎinánmǔ),我艺黍稷 (wǒyìshǔjì)。

一年的农活该开始干起来了,我种上小米,又种上高粱。

税熟贡新 (shuìshúgòngxīn),劝赏黜陟 (quànshǎngchùzhì)。

收获季节,用刚熟的新谷交纳税粮,官府应按农户的贡献大小给予奖励或处罚。

孟轲敦素,史鱼秉直。

孟轲夫子崇尚纯洁,史官子鱼秉性刚直。

庶几中庸,劳谦谨敕 (láoqiānjǐnchì)。

做人要尽可能合乎中庸的标准,勤奋、谦逊、谨慎,懂得规劝告诫自己。

聆音察理 (língyīnchálǐ),鉴貌辨色 (jiànmàobiànsè)。

听别人说话,要仔细审察是否合理;看别人面孔,要小心辨析他的脸色。

贻厥嘉猷 (yíjuéjiāyóu),勉其祗植 (miǎnqízhīzhí)。

要给人家留下正确高明的忠告或建议,勉励别人谨慎小心地处世立身。

省躬讥诫 (xǐnggōngjījiè),宠增抗极 (chǒngzēngkàngjí)。

听到别人的讥讽告诫,要反省自身;备受恩宠不要得意忘形,对抗权尊。

殆辱近耻 (dàirǔjìnchǐ),林皋幸即 (língāoxìngjí)。

知道有危险耻辱的事快要发生,还不如归隐山林为好。

两疏见机 (liǎngshūjiànjī),解组谁逼 (jièzǔshuíbī)。

疏广疏受预见到危患的苗头才告老还乡,哪里有谁逼他们除下官印?

索居闲处 (suǒjūxiánchù),沉默寂寥 (chénmòjìliào)。

离群独居,悠闲度日,整天不用多费唇舌,清静无为岂不是好事。

求古寻论 (qiúgǔxúnlùn),散虑逍遥 (sǎnlǜxiāoyáo)。

想想古人的话,翻翻古人的书,消往日的忧虑,乐得逍遥舒服。

欣奏累遣 (xīnzòulěiqiǎn),戚谢欢招 (qīxièhuānzhāo)。

轻松的事凑到一起,费力的事丢在一边,消除不尽的烦恼,得来无限的快乐。

渠荷的历 (qúhédelì),园莽抽条 (yuánmǎngchōutiáo)。

池里的荷花开得光润鲜艳,园中的草木抽出条条嫩枝。

枇杷晚翠 (pípáwǎncuì),梧桐蚤凋 (wútóngzǎodiāo)。

枇杷到了岁晚还是苍翠欲滴,梧桐刚刚交秋就早早地凋谢了。

陈根委翳 (chéngēnwěiyì),落叶飘摇 (luòyèpiāoyáo)。

陈根老树枯倒伏,落叶在秋风里四处飘荡。

游鹍\footnote{k\=un}独运,淩摩绛霄。

寒秋之中,鲲鹏独自高飞,直冲布满彩霞的云霄。

耽读玩市,寓目囊箱。

汉代王充在街市上沉迷留恋于读书,眼睛注视的都是书袋和书箱。

易輶\footnote{y\'ou}攸畏,属耳垣\footnote{yu\'an}墙。

说话最怕旁若无人,毫无禁忌;要留心隔着墙壁有人在贴耳偷听。

具膳餐饭,适口充肠。

安排一日三餐的膳食,要适合各位的口味,能让大家吃饱。

饱饫\footnote{y\`u}烹宰,饥厌糟糠。

的时候自然满足于大鱼大肉,饿的时候应当满足于粗菜淡饭。

亲戚故旧,老少异粮。

亲属、朋友会面要盛情款待,老人、小孩的食物应和自己不同。

妾御绩纺 (qièyùjìfǎng),侍巾帷房。

小妾婢女要管理好家务,尽心恭敬地服待好主人。

纨\footnote{w\'an}扇圆洁,银烛炜煌 (yínzhúwěihuáng)。

绢制的团扇像满月一样又白又圆,银色的烛台上烛火辉煌。

昼眠夕寐 (zhòumiánxīmèi),蓝笋象床 (lánsǔnxiàngchuáng)。

白日小憩,晚上就寝,有青篾编成的竹席和象牙雕屏的床榻。

弦歌酒宴 (xiángējiǔyàn),接杯举觞 (jiébēijǔshāng)。

奏着乐,唱着歌,摆酒开宴;接过酒杯,开怀畅饮。

矫手顿足 (jiǎoshǒudùnzú),悦豫且康 (yuèyùqiěkāng)。

情不自禁地手舞足蹈,真是又快乐又安康。

嫡后嗣续 (díhòusìxù),祭祀烝尝 (jìsìzhēngcháng)。

子孙继承了祖先的基业,一年四季的祭祀大礼不能疏忘。

稽颡再拜 (jīsǎngzàibài),悚惧恐惶 (sǒngjùkǒnghuáng)。

跪着磕头,拜了又拜;礼仪要周全恭敬,心情要悲痛虔诚。

笺牒简要 (jiāndiéjiǎnyào),顾答审详 (gùdáshěnxiáng)。

给别人写信要简明扼要,回答别人问题要详细周全。

骸垢想浴 (hàigòuxiǎngyù),执热愿凉 (zhírèyuànliáng)。

身上有了污垢,就想洗澡,好比手上拿着烫的东西就希望有风把它吹凉。

驴骡犊特 (lǘluódútè),骇跃超骧 (hàiyuèchāoxiāng)。

家里有了灾祸,连牲畜都会受惊,狂蹦乱跳,东奔西跑。

诛斩贼盗 (zhūzhǎnzéidào),捕获叛亡。

对抢劫、偷窃、反叛、逃亡的人要严厉惩罚,该抓的抓,该杀的杀。

布射僚丸,嵇\footnote{j\=i}琴阮箫。

吕布擅长射箭,宜僚有弄丸的绝活,嵇康善于弹琴,阮籍能撮口长啸。

恬笔伦纸 (tiánbǐlúnzhǐ),钧巧任钓 (jūnqiǎorèndiào)。

蒙恬造出毛笔,蔡伦发明造纸,马钧巧制水车,任公子垂钓大鱼。

释纷利俗,并皆佳妙。

他们的技艺有的解人纠纷,有的方便群众,都高明巧妙,为人称道。

毛施淑姿 (máoshīshūzī),工颦妍笑 (gōngpínyánxiào)。

毛嫱、西施年轻美貌,哪怕皱着眉头,也像美美的笑。

年矢每催 (niánshǐměicuī),曦晖朗曜 (xīhuīlǎngyào)。

可惜青春易逝,岁月匆匆催人渐老,只有太阳的光辉永远朗照。

璿玑悬斡 (xuánjīxuánwò),晦魄环照 (huìpòhuánzhào)。

高悬的北斗随着四季变换转动,明晦的月光洒遍人间每个角落。

指薪修祜 (zhǐxīnxiūhù),永绥吉劭 (yǒngsuíjíshào)。

行善积德才能像薪尽火传那样精神长存,子孙安康全靠你留下吉祥的忠告。

矩步引领 (jùbùyǐnlǐng),俯仰廊庙 (fǔyǎnglángmiào)。

如此心地坦然,方可以昂头迈步,应付朝廷委以的重任。

束带矜庄 (shùdàijīnzhuāng),徘徊瞻眺 (páihuáizhāntiào)。

如此无愧人生,尽可以整束衣冠,庄重从容地高瞻远望。

孤陋寡闻,愚蒙等诮\footnote{qi\`ao}。

这些道理孤陋寡闻就不会明白,只能和愚味无知的人一样空活一世,让人耻笑。

谓语助者,焉哉乎也。

编完「千字文」乌发皆白,最后剩下“焉、哉、乎、也”这几个语气助词。
\end{yuanwen}

\end{document}