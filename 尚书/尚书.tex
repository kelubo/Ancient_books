% 尚书
% 尚书.tex

\documentclass[a4paper,12pt,UTF8,twoside]{ctexbook}

% 设置纸张信息。
\RequirePackage[a4paper]{geometry}
\geometry{
	%textwidth=138mm,
	%textheight=215mm,
	%left=27mm,
	%right=27mm,
	%top=25.4mm, 
	%bottom=25.4mm,
	%headheight=2.17cm,
	%headsep=4mm,
	%footskip=12mm,
	%heightrounded,
	inner=1in,
	outer=1.25in
}

% 设置字体,并解决显示难检字问题。
\xeCJKsetup{AutoFallBack=true}
\setCJKmainfont{SimSun}[BoldFont=SimHei, ItalicFont=KaiTi, FallBack=SimSun-ExtB]

% 目录 chapter 级别加点(.)。
\usepackage{titletoc}
\titlecontents{chapter}[0pt]{\vspace{3mm}\bf\addvspace{2pt}\filright}{\contentspush{\thecontentslabel\hspace{0.8em}}}{}{\titlerule*[8pt]{.}\contentspage}

% 设置 part 和 chapter 标题格式。
\ctexset{
	part/name= {第,卷},
	part/number={\chinese{part}},
	chapter/name={第,篇},
	chapter/number={\chinese{chapter}}
}

% 设置古文原文格式。
\newenvironment{yuanwen}{\bfseries\zihao{4}}

\title{\heiti\zihao{0} 尚书}
\author{}
\date{}

\begin{document}

\maketitle
\tableofcontents

\frontmatter
\chapter{前言}

《尚书》,最早书名为《书》,是一部追述古代事迹著作的汇编。分为《虞书》、《夏书》、《商书》、《周书》。因是儒家五经之一,又称《书经》。

现在通行的《十三经注疏》本《尚书》,就是《今文尚书》和伪《古文尚书》的合编本。现存版本中真伪参半。西汉学者伏生口述的二十八篇《尚书》为今文《尚书》,鲁恭王在拆除孔子故宅一段墙壁时,发现的另一部《尚书》,为古文《尚书》。西晋永嘉年间战乱,今、古文《尚书》全都散失了。东晋初,豫章内史梅赜给朝廷献上了一部《尚书》,包括《今文尚书》33篇,以及伪《古文尚书》25篇 。

《尚书》列为重要核心儒家经典之一,历代儒家研习之基本书籍。 “尚”即“上”,《尚书》就是上古的书,它是我国最早的一部历史文献汇编。

\mainmatter

\part{虞书}

\chapter{尧典}

\begin{yuanwen}
昔在帝尧,聪明文思,光宅天下。将逊于位,让于虞舜,作《尧典》。
\end{yuanwen}

从前唐尧称帝的时候,耳聪目明,治理天下有计谋,他的光辉充满天下。在他即将逊位,禅让给虞舜的时候,作了《尧典》。

\begin{yuanwen}
曰若稽古,帝尧,曰放勋,钦、明、文、思、安安。允恭克让,光被四表,格于上下。克明俊德,以亲九族。九族既睦,平章百姓。百姓昭明,协和万邦。黎民于变时雍。
\end{yuanwen}

评价万古帝尧,帝尧名字叫做放勋,他恭敬节俭,明察是非,善理天下,道德纯备,温和宽容。他忠实不懈,又能让贤,光辉普照四方,思虑至于天地。他能发扬大德,使家族亲密和睦。家族和睦以后,又辨明其他各族的政事。众族的政事辨明了,又协调万邦诸侯,天下众民因此也就相递变化友好和睦起来。

\begin{yuanwen}
乃命羲和,钦若昊天,历象日月星辰,敬授民时。分命羲仲,宅嵎夷,曰暘谷。寅宾出日,平秩东作。日中,星鸟,以殷仲春。厥民析,鸟兽孳尾。申命羲叔,宅南交,曰明都。平秩南讹,敬致。日永,星火,以正仲夏。厥民因,鸟兽希革。分命和仲,宅西,曰昧谷。寅饯纳日,平秩西成。宵中,星虚,以殷仲秋。厥民夷,鸟兽毛毨。申命和叔,宅朔方,曰幽都。平在朔易。日短,星昴,以正仲冬。厥民隩,鸟兽氄毛。帝曰:“咨!汝羲暨和。期三百有六旬有六日,以闰月定四时,成岁。允厘百工,庶绩咸熙。”
\end{yuanwen}

尧于是命令羲氏与和氏,专职从事天文工作,推算日月星辰运行的规律,制定出历法,从而告诉人们依照时令节气从事生产活动。任命羲仲居住在今山东东部滨海地区,传说中日出的地方。迎接升起的太阳,注意东边星星升起的顺序。以昼夜平分的那天作为春分,并以鸟星见于南方正中之时作为划分仲春的根据。昼夜长短相等,南方朱雀七宿黄昏时出现在天的正南方,依据这些确定仲春时节。这时,人们分散在田野,鸟兽开始生育繁殖。又命令羲叔,住在南方的交趾,辨别测定太阳往南运行的情况,恭敬地迎接太阳向南回来。白昼时间最长,东方苍龙七宿中的火星黄昏时出现在南方,依据这些确定仲夏时节。这时,人们住在高处,鸟兽的羽毛稀疏。又命令和仲,住在西方的昧谷,恭敬地送别落日,辨别测定太阳西落的时刻。昼夜长短相等,北方玄武七宿中的虚星黄昏时出现在天的正南方,依据这些确定仲秋时节。这时,人们又回到平地上居住,鸟兽换生新毛。又命令和叔,住在北方的幽都,辨别观察太阳往北运行的情况。白昼时间最短,西方白虎七宿中的昴星黄昏时出现在正南方,依据这些确定仲冬时节。这时,人们住在室内,鸟兽长出了柔软的细毛。尧说:“啊!你们羲氏与和氏啊,一周年是三百六十六天,要用加闰月的办法确定春夏秋冬四季而成一岁。由此规定百官的事务,许多事情就都兴办起来。”

\begin{yuanwen}
帝曰:“畴咨若时登庸?”
尧帝说:“善治四时之职的是谁啊?我要提升任用他。”

放齐曰:“胤子朱启明。”
放齐说:“您的儿子丹朱很开明。”

帝曰:“吁!嚣讼可乎?”
尧帝说:“唉!他说话虚妄,又好争辩,可以吗?”

帝曰:“畴咨若予采?”
尧帝说:“善于处理我们政务的是谁呢?”

驩兜曰:“都!共工方鸠僝功。”
驩兜说:“啊!共工防救水灾已具有成效啊。”

帝曰:“吁!静言庸违,象恭滔天。”
尧帝说:“唉!他貌似恭敬,而违背道德天性。”

帝曰:“咨!四岳,汤汤洪水方割,荡荡怀山襄陵,浩浩滔天。下民其咨,有能俾乂?”
尧帝说:“啊!四方的诸侯啊!滔滔的洪水普遍危害人们,水势奔腾包围了山岭,淹没了丘陵,浩浩荡荡,弥漫接天。臣民百姓都在叹息,有能使洪水得到治理的吗?”

佥曰:“于!鲧哉。”
人们都说:“啊!鲧吧。”

帝曰:“吁!咈哉,方命圮族。”
尧帝说:“唉!他违背人意,不服从命令,危害族人。”

岳曰:“异哉!试可乃已。”
四方诸侯之长说:“起用吧!试试可以,就用他。”

帝曰,“往,钦哉!”九载,绩用弗成。
尧帝说:“去吧,鲧!要谨慎啊!”过了九年,成效不好。

帝曰:“咨!四岳。朕在位七十载,汝能庸命,巽朕位?”
尧帝说:“啊!四方的诸侯啊!我在位已经七十年了,你们当中谁能够顺应天命接替我的位置啊?”

岳曰:“否德忝帝位。”
四方诸侯之长说:“我们德行鄙陋,不配登上帝位。”

曰:“明明扬侧陋。”
尧帝说:“可以明察贵戚中的贤才,也可以推举地位低微的贤德之人。”

师锡帝曰:“有鳏在下,曰虞舜。”
众人提议说:“在民间有一个穷困的人,他的名字叫虞舜。”

帝曰:“俞?予闻,如何?”
尧帝说:“是的,我也听说过这个人,他的德行怎么样呢?”

岳曰:“瞽子,父顽,母嚚,象傲;克谐以孝,烝烝乂,不格奸。”
四方诸侯之长回答说:“他是乐官瞽叟的儿子。他的父亲心术不正,他的母亲总是说谎,他的弟弟非常傲慢,但是舜却能同他们和谐相处。因他的孝心醇厚,治理国务不至于坏吧!”

帝曰:“我其试哉!女于时,观厥刑于二女。”
\end{yuanwen}
尧帝说:“我试试考验考验他吧!把我的两个女儿嫁给舜,通过两个女儿考察他的德行。”

\begin{yuanwen}
厘降二女于妫汭,嫔于虞。
于是命令在妫河的弯曲处举行婚礼,将两个女儿嫁给虞舜。

帝曰:“钦哉!”
尧帝说:“以后你就敬慎地处理政务吧!”

慎微五典,五典克从。纳于百揆,百揆时叙。宾于四门,四门穆穆。纳于大麓,烈风雷雨弗迷。
舜诚心诚意地推行德教,教导臣民以父义、母慈、兄友、弟恭、子孝五种美德指导自身的行为,臣民都可以听从这种教导而不违背。舜总理百官,百官都能承顺。舜在明堂四门迎接四方宾客,四方宾客都肃然起敬。舜担任守山林的官,在暴风雷雨的恶劣天气也不迷误。

帝曰:“格!汝舜。询事考言,乃言凪可绩,三载。汝陟 帝位。”
尧帝说:“来吧!舜,我同你谋划政事,经过三年的考验,你的确取得不少成绩,你现在可以登上天子的大位了!”

舜让于德,弗嗣。
\end{yuanwen}
舜觉得自己德行不够,要让给有德的人,推让不肯继承。

\chapter{舜典}

虞舜侧微,尧闻之聪明,将使嗣位,历试诸难,作《舜典》。
虞舜出身微寒。尧听说他智慧豁达,明辨事理,想让他承继大位,用很多难题对他进行考验。作了《舜典》记录这些情况。

曰若稽古,帝舜曰重华,协于帝。浚哲文明,温恭允塞,玄德升闻,乃命以位。慎徽五典,五典克从;纳于百揆,百揆时叙;宾于四门,四门穆穆;纳于大麓,烈风雷雨弗迷。帝曰:“格!汝舜。询事考言,乃言厎可绩,三载。汝陟帝位。”舜让于德,弗嗣。
考察古代的历史,帝舜名叫重华,圣明与帝尧相合。他智慧深邃,温和谦逊的美德充满天地之间,他潜心加强自身道德修养,朝堂上的官员都听说过他,于是他被授予官职。舜真诚善意地履行父义、母慈、兄友、弟恭、子孝这五种伦理道德规范,使人们都能遵循这五种伦理道德规范。尧又命舜总理部落联盟一切政务,各种政务都处理的井井有条。又命舜在明堂门口欢迎觐见的四方部落首领,来朝的宾客都肃然起敬。又让舜深入大山丛林中,他在暴风雷雨中也不迷失道路,君主尧说:“来吧,舜!三年来我询问了你的政事活动,考察了你的言论,我认为你可以取得功业,可以继承帝位了。”舜谦让于有德的人,不肯继承帝位。

正月上日,受终于文祖。在璿玑玉衡,以齐七政。肆类于上帝,禋于六宗,望于山川,遍于群神。辑五瑞。既月乃日,觐四岳群牧,班瑞于群后。
正月初一,舜在尧的太庙接受了禅让的册命。舜继位后,他观察了北斗七星的运行规律,列出了七项政事。接着,举行祭天大典,向天帝报告继承帝位的事,又祭祀了天地四时,祭祀山川和群神。随后又聚敛了诸侯的五种圭玉,选择吉月吉日,召见四方诸侯,把圭玉颁发给各位君长。

岁二月,东巡守,至于岱宗,柴。望秩于山川,肆觐东后。协时月正日,同律度量衡。修五礼、五玉、三帛、二生、一死贽。如五器,卒乃复。
这年二月,舜来到东方巡视,在泰山举行了祭祀泰山之礼。对于其他山川,都按地位尊卑依次举行了祭祀,然后,接受了东方诸侯君长的朝见。协调春夏秋冬四时的月份,确定天数,统一音律、度、量、衡。制定了公侯伯子男朝聘的礼节、五种瑞玉、三种不同颜色的丝绸、活羊羔、活雁、死野鸡,分别作为诸侯、卿大夫和士朝见时的贡物。而五种瑞玉,朝见完毕后,仍然还给诸侯。

五月南巡守,至于南岳,如岱礼。八月西巡守,至于西岳,如初。十有一月朔巡守,至于北岳,如西礼。归,格于艺祖,用特。
五月,舜又来到南方巡视,到达南岳,举行了和祭祀泰山一样的祭祀礼。八月,舜到西方巡视,在华山举行了和泰山一样的祭祀礼。十一月初一,舜到北方巡视,所行的礼节同在西岳一样。回来后,到尧的太庙祭祀,用一头牛作祭品。

五载一巡守,群后四朝。敷奏以言,明试以功,车服以庸。
之后就每五年巡视一次,诸侯在四岳朝见天子。向天子报告自己的业绩;天子也认真地考察诸侯国的政治得失,把车马衣服赐予那些有功的诸侯。

肇十有二州,封十有二山,浚川。
舜划定十二州的疆界,在十二州的名山上封土为坛举行祭祀,又疏通了河道。

象以典刑,流宥五刑,鞭作官刑,扑作教刑,金作赎刑。眚灾肆赦,怙终贼刑。钦哉,钦哉,惟刑之恤哉!
舜又在器物上刻画五种常用的刑罚。用流放的办法宽恕犯了五刑的罪人,用鞭打作为官的刑罚,用木条打作为学校的刑罚,用铜作为赎罪的刑罚。因过失犯罪,就赦免他;有所依仗不知悔改,就要施加刑罚。谨慎啊,谨慎啊,刑罚要慎重啊!

流共工于幽州,放欢兜于崇山,窜三苗于三危,殛鲧于羽山,四罪而天下咸服。二十有八载,帝乃殂落。百姓如丧考妣,三载,四海遏密八音。月正元日,舜格于文祖,询于四岳,辟四门,明四目,达四聪。“咨,十有二牧!”曰,“食哉惟时!柔远能迩,惇德允元,而难任人,蛮夷率服。”
于是把共工流放到幽州,把驩兜流放到崇山,把三苗驱逐到三危,把鲧流放到羽山,并命令他至死不得回朝。这四个人处罚了,天下的人都心悦诚服。舜辅助尧帝二十八年后,尧帝逝世了。人们好象死了父母一样地悲痛,三年间,全国上下停止了乐音。明年正月的一个吉日,舜到了尧的太庙,与四方诸侯君长谋划政事,打开明堂四门宣布政教,使四方见得明白,听得通彻。“啊,十二州的君长!”舜帝说:“生产民食,必须依时!安抚远方的臣民,爱护近处的臣民,亲厚有德的人,信任善良的人,而又拒绝邪佞的人,这样,边远的外族都会服从。”

舜曰:“咨,四岳!有能奋庸熙帝之载,使宅百揆亮采,惠畴?”
舜帝说:“啊!四方诸侯的君长!有谁能奋发努力、发扬光大尧帝的事业,使居百揆之官辅佐政事呢?”

佥曰:“伯禹作司空。”
都说:“伯禹现在作司空。”

帝曰:“俞,咨!禹,汝平水土,惟时懋哉!”
舜帝说:“好啊!禹,你曾经平定水土,还要努力做好百揆这件事啊!”

禹拜稽首,让于稷、契暨皋陶。
禹跪拜叩头,让给稷、契和皋陶。

帝曰:“俞,汝往哉!”
舜帝说:“好啦,还是你去吧!”

帝曰:“弃,黎民阻饥,汝后稷,播时百谷。”
舜帝说:“弃,人们忍饥挨饿,你主持农业,教人们播种各种谷物吧!”

帝曰:“契,百姓不亲,五品不逊。汝作司徒,敬敷五教,在宽。”
舜帝说:“契,百姓不亲,父母兄弟子女都不和顺。你作司徒吧,谨慎地施行五常教育,要注意宽厚。”

帝曰:“皋陶,蛮夷猾夏,寇贼奸宄。汝作士,五刑有服,五服三就。五流有宅,五宅三居。惟明克允!”
舜帝说:“皋陶,外族侵扰我们中国,抢劫杀人,造成外患内乱。你作狱官之长吧,五刑各有使用的方法,五种用法分别在野外、市、朝三处执行。五种流放各有处所,分别住在三个远近不同的地方。要明察案情,处理公允!”

帝曰:“畴若予工?”
舜帝说:“谁能当好掌管我们百工的官?”

佥曰:“垂哉!”
都说:“垂啊!”

帝曰:“俞,咨!垂,汝共工。”
舜帝说:“好啊!垂,你掌管百工的官吧!”

垂拜稽首,让于殳斨暨伯与。”
垂跪拜叩头,让给殳斨和伯与。

帝曰:“俞,往哉!汝谐。”
舜帝说:“好啦,去吧!你同他们一起去吧!”

帝曰:“畴若予上下草木鸟兽?”
舜帝说:“谁掌管我们的山丘草泽的草木鸟兽呢?”

佥曰:“益哉!”
都说:“益啊!”

帝曰:“俞,咨!益,汝作朕虞。”
舜帝说:“好啊!益,你担任我的虞官吧。”

益拜稽首,让于朱虎、熊罴。
益跪拜叩头,让给朱虎和熊罴。

帝曰:“俞,往哉!汝谐。”
舜帝说:“好啦,去吧!你同他们一起去吧!”

帝曰:“咨!四岳,有能典朕三礼?”
舜帝说:“啊!四方诸侯的君长,有谁能主持我们祭祀天神、地祗、人鬼的三礼呢?”

佥曰:“伯夷!”
都说:“伯夷!”

帝曰:“俞,咨!伯,汝作秩宗。夙夜惟寅,直哉惟清。”
舜帝说:“好啊!伯,你作掌管祭祀的礼官吧。要早晚恭敬行事,又要正直、清明。”

伯拜稽首,让于夔、龙。
伯夷跪拜叩头,让给夔和龙。

帝曰:“俞,往,钦哉!”
舜帝说:“好啦,去吧!要谨慎啊!”

帝曰:“夔!命汝典乐,教胄子,直而温,宽而栗,刚而无虐,简而无傲。诗言志,歌永言,声依永,律和声。八音克谐,无相夺伦,神人以和。”
舜帝说:“夔!我现在任命你为主持乐官,教导年轻人,让他们正直温和,宽大坚栗,刚毅不粗暴,简约不傲慢。诗是表达思想感情的,歌是唱出来的语言,五声是根据所唱而制定的,六律是和谐五声的。八类乐器的声音可以和谐演奏,不使它们乱了次序,如此,即便天地之人听到也会感到和谐、快乐。”

夔曰:“于!予击石拊石,百兽率舞。”
夔说:“啊!让我们敲着石磬,奏起乐来,让那些无知的鸟兽们都跳起来吧!”

帝曰:“龙,朕{即土}谗说殄行,震惊联师。命汝作纳言,夙夜出纳朕命,惟允!”
舜帝说:“龙!我厌恶谗毁的言论和贪残的行为,会使我的民众震惊。我任命你做纳言的官,早晚传达我的命令,转告下面的意见,应当真实!”

帝曰:“咨!汝二十有二人,钦哉!惟时亮天功。”
舜帝说:“啊!你们二十二人,要谨慎啊!要好好领导天下大事啊!”

三载考绩,三考,黜陟幽明,庶绩咸熙。分北三苗。 舜生三十征,庸三十,在位五十载,陟方乃死。
舜帝三年考察一次政绩,考察三次后,罢免昏庸的官员,提拔贤明的官员,于是,许多工作都兴办起来了。又分别对三苗之族作了安置。舜三十岁时被征召,施政二十年,在帝位五十年,在巡狩南方时才逝世。

帝厘下土,方设居方,别生分类。作《汩作》、《九共》九篇、《槀饫》。

\chapter{大禹谟}

皋陶矢厥谟,禹成厥功,帝舜申之。作《大禹》、《皋陶谟》、《益稷》。
皋陶陈述他的谋略,禹陈述他的功绩,舜帝很重视他们的言论。史官根据他们的议论写作了《大禹谟》、《皋陶谟》和《益稷》。

曰若稽古,大禹曰文命敷于四海,祗承于帝。曰:“后克艰厥后,臣克艰厥臣,政乃乂,黎民敏德。”
传说:若考古时候的大禹,又名文命,他的功德广布到了四海。他曾经接受帝舜的征询,发表自己的见解道:“为君的能知道为君的艰难,为臣的能知道为臣的艰难,那么,政事就能治理好,人民也就会迅速修德了。”

帝曰:“俞!允若兹,嘉言罔攸伏,野无遗贤,万邦咸宁。稽于众,舍己从人,不虐无告,不废困穷,惟帝时克。”
帝舜道:“这话不错。确实像这样,那么,好主意就不会被搁置不用,贤才就不会被遗弃在田野之间,万邦都会太平。凡事都考察群众的意见,常常放弃自己不正确的意见,听从别人正确的意见;为政不虐待无告的穷人,用人不忽视卑贱的贤才,这只有帝尧的时候才能做到。”

益曰:“都,帝德广运,乃圣乃神,乃武乃文。皇天眷命,奄有四海为天下君。”
益插嘴道:“嗨!帝尧的道德广大而又能运用,真是圣哲神明,能武能文,所以皇天特别照顾他,命他统治四海,为天下的大君。”

禹曰:“惠迪吉,从逆凶,惟影响。”
禹说:“凡是顺道从善的就得福,逆道从恶的就得祸,这真像影随形、响应声一样!”

益曰:“吁!戒哉!儆戒无虞,罔失法度。罔游于逸,罔淫于乐。任贤勿贰,去邪勿疑。疑谋勿成,百志惟熙。罔违道以干百姓之誉,罔咈百姓以从己之欲。无怠无荒,四夷来王。”
益说:“咦!可得警戒这一点啊!只有时刻警戒自己,才能免于后忧。不要破坏法规制度,不要优游流于放纵,不要过度玩乐;任用贤才不要三心二意,铲除邪恶不要犹豫不决,谋划尚有疑问就不要勉强施行。这样,你心中的一切思虑都会通明透亮了。不要违反正道去求取百姓的称誉,也不要不顾百姓的意见去满足自己的欲望。思想不怠惰,政事不荒废,那么,四夷都会来归附你的。”

禹曰:“於!帝念哉!德惟善政,政在养民。水、火、金、木、土、谷,惟修;正德、利用、厚生、惟和。九功惟叙,九叙惟歌。戒之用休,董之用威,劝之以九歌俾勿坏。”
禹说:“帝!您要记住啊!修德主要表现在搞好政事,而为政的中心在于养育人民。水火金木土谷这六府要修治好,端正人民品德、丰富人民财用、改善人民生活这三件事要互相配合。这九个方面的功业都要安排得有秩序,有了秩序,人民自然欢欣鼓舞,歌功颂德了。对于勤劳的人,要用美好的前景去诱导他们;对于怠惰的人,要用刑罚去督责他们;而当人民受到德泽感到欢欣的时候,就要及时鼓励他们开展歌咏活动,使之乐而忘芳,干劲不衰。”

帝曰:“俞!地平天成,六府三事允治,万世永赖,时乃功。”
帝舜道:“讲得对!现在水土治平,万物得以成长,六府三事确实治理得很有秩序,万世以后都要仰赖那时你的大功啊!”

帝曰:“格,汝禹!朕宅帝位三十有三载,耄期倦于勤。汝惟不怠,总朕师。”
帝舜道:“禹,你来!我居帝位已经三十三年了,如今已到老耄昏聩的时期,掌握这样烦忙的政事委实感到疲倦。你平日是从不懈怠的,今后要接替我总管众民啊!”

禹曰:“朕德罔克,民不依。皋陶迈种德,德乃降,黎民怀之。帝念哉!念兹在兹,释兹在兹,名言兹在兹,允出兹在兹,惟帝念功。”
禹连忙答道:“我的品德不能胜任,人民不会依从我的,不如皋陶勇往力行,积极种德,德泽普及下民,民众都怀念他。帝!您可要顾念他啊!他平日一心挂念的就在于种德这件事还有欠缺,偶而放下心来也就在于这件事有了成绩;他经常在口头上谈论的就在于这件事,真诚出自内心的也就在于这件事。所以说,您可得要顾念他的大功啊!”

帝曰:“皋陶,惟兹臣庶,罔或干予正。汝作士,明于五刑,以弼五教。期于予治,刑期于无刑,民协于中,时乃功,懋哉。”
帝舜于是转向皋陶说:“皋陶!现在广大臣民没有一个敢触犯法纪的,这是由于你任我的士师,能够正确运用五刑来辅助五教,期望使我的政事达到治理的境地。要用刑罚来达到消灭刑罚的目的,使人民都能走上正道,那时你的功劳就大了!”

皋陶曰:“帝德罔愆,临下以简,御众以宽;罚弗及嗣,赏延于世。宥过无大,刑故无小;罪疑惟轻,功疑惟重;与其杀不辜,宁失不经;好生之德,洽于民心,兹用不犯于有司。”
皋陶回答道:“帝!您的德行毫无过差,对下边的要求简明扼要,治理民众非常宽大;刑罚不牵连子女,而奖赏却延及后世;对偶然的过失,再大也给以宥赦,对明知故犯的罪恶,再小也处以刑罚;罚罪有疑问就从轻发落,赏功有疑问却从重给奖;与其杀害无辜的人,宁可犯不执行常法的过失:这种好生的美德,已经融洽到人民心里,因此,人民都能守规矩,不犯官家的法纪。”

帝曰:“俾予从欲以治,四方风动,惟乃之休。”
帝舜道:“使我能如愿以治理人民,四方都听从我的命令,好像草木随风而动,这都是你做的好事啊!”

帝曰:“来,禹!降水儆予,成允成功,惟汝贤。克勤于邦,克俭于家,不自满假,惟汝贤。汝惟不矜,天下莫与汝争能。汝惟不伐,天下莫与汝争功。予懋乃德,嘉乃丕绩,天之历数在汝躬,汝终陟元后。人心惟危,道心惟微,惟精惟一,允执厥中。无稽之言勿听,弗询之谋勿庸。可爱非君?可畏非民?众非元后,何戴?后非众,罔与守邦?钦哉!慎乃有位,敬修其可愿,四海困穷,天禄永终。惟口出好兴戎,朕言不再。”
然后,舜又转回来对禹说:“来,禹!当年天降洪水来警戒我,能够言行一致,既在治平水土中成就功业,又在民众中建立威信的,就数你最贤;既能勤劳为邦,又能节俭持家,不自满自大,这也数你最贤。你正因为不自逞能,所以天下没有一个人敢与你争能;你正因为不自居功,所以天下没有一个人敢与你争功。我真诚赞美你的品德,嘉许你的大功。天命已经降落到你的身上,你终将升任大君。人心是危险难安的,道心却微妙难明。惟有精心体察,专心守住,才能坚持一条不偏不倚的正确路线。没有考核事实的言语不要听,没有征询群众意见的主意不要用。可爱的不是君而是民,可畏的不是民而是君失其道。民众没有大君他们又爱戴谁呢?大君没有民众就无人跟他守邦了。一定要谨慎啊!认真对待你所居的大位,切实做好你想要做的每件事。如果四海百姓都至于穷困不堪,那你做大君的天禄也就永远终结了。只有这张嘴,最爱惹是生非,讲话可得慎重啊!我要讲的都已讲完,没有什么再要讲的了。”

禹曰:“枚卜功臣,惟吉之从。”
禹还是谦让道:“那么,就一个个功臣来占卜,看谁的卜兆最吉就由谁来接位。”

帝曰:“禹!官占惟先蔽志,昆命于元龟。朕志先定,询谋佥同,鬼神其依,龟筮协从,卜不习吉。”禹拜稽首,固辞。
帝舜道:“禹!我们占卜公事,是先由于心有疑难掩蔽,然后才去请问大龟的。现在我的意志早已先定了,并经征询众人的意见都一致赞同,相信鬼神必定依从,龟筮也必定是吉了。占卜是不会重复出现吉兆的,用不着再卜了。”但是,禹还是稽首拜辞。

帝曰:“毋!惟汝谐。”
帝舜最后断然地说:“不!只有你合适。”

正月朔旦,受命于神宗,率百官若帝之初。
正月初一日,禹在尧庙里接受了摄政的委命,率领百官行礼,像当年舜受命摄政时一样。

帝曰:“咨,禹!惟时有苗弗率,汝徂征。”
然后,舜对禹说道:“禹!跟你商量一下,现时只有三苗不遵从我们的教令了,你去征伐他们。”

禹乃会群后,誓于师曰:“济济有众,咸听朕命。蠢兹有苗,昏迷不恭,侮慢自贤,反道败德,君子在野,小人在位,民弃不保,天降之咎,肆予以尔众士,奉辞伐罪。尔尚一乃心力,其克有勋。”
禹于是大会各邦群后及其率领的人众,宣誓于众道:“整齐众多的勇士们!都来听我的命令:这无知盲动的三苗,执迷不悟,傲慢自大,违反正道,败坏常德。致使君子被遗弃在野,而小人却窃居高位,把人民抛弃不顾,因此,上天降灾于他们。我今天是用你们群后众士之力,奉天命去罚他们的罪。你们还须齐心合力,才能成就功勋。”

三旬,苗民逆命。益赞于禹曰:“惟德动天,无远弗届。满招损,谦受益,时乃天道。帝初于历山,往于田,日号泣于旻天,于父母,负罪引慝。祗载见瞽叟,夔夔斋栗,瞽亦允若。至諴感神,矧兹有苗。”
战事进行了三十天,苗民仍然负隅顽抗,不肯听命。益就向禹建议道:“只有道德的力量才能感动天地,再远的地方也能达到。满招损,谦受益,常常就是天道。帝舜早年受父母虐待,一个人在历山耕田,苦不堪言。但他日日号哭涕泣,仍然呼喊苍天,呼喊父母,总是诚心自责,把罪错全部承担,从不怨天怨父母。有事去见瞽瞍的时候,总是端端正正,战战兢兢。在这种时候,连顽固的瞽瞍也真能通情达理了。常言至诚感神,何况有苗?”

禹拜昌言曰:“俞!”班师振旅。帝乃诞敷文德,舞干羽于两阶,七旬有苗格。
禹连忙下拜,接受了这个好意见,说:“讲得对!”立即停战,整队班师而归。从此,帝舜也接受了益和禹的建议,大布文德,在朝堂两阶之间举行大规模的舞蹈,人们举着战争中用的盾牌和雉尾,载歌载舞,表示偃武修文。七十天之后,有苗终于自动前来归附了。

\chapter{皋陶谟}

曰若稽古皋陶曰:“允迪厥德,谟明弼谐。”禹曰:“俞,如何?”皋陶曰:“都!慎厥身,修思永。惇叙九族,庶明励翼,迩可远在兹。”禹拜昌言曰:“俞!”
相传,皋陶和禹曾在舜帝面前讨论国家政事。皋陶说:“诚信地遵循尧帝的德行,君主就能做到决策英明,群臣也可以同心同德。”禹曰:“是啊!但是是怎么做到的呢?”皋陶说:“啊!要谨慎其身,坚持不懈地努力提升自身的修养。要使近亲宽厚顺从,使贤人勉力辅佐,由近及远,完全在于从这里做起。”禹听了这番精当的言论,拜谢说:“对呀!”

皋陶曰:“都!在知人,在安民。”禹曰:“吁!咸若时,惟帝其难之。知人则哲,能官人。安民则惠,黎民怀之。能哲而惠,何忧乎欢兜?何迁乎有苗?何畏乎巧言令色孔壬?”
皋陶说:“最重要的还是知人善任,把臣民治理好。”禹说:“哎呀!事实虽是这样,但是想要完全做到这些,只怕连先王都难以做到。知人善任才算得上是明智之人,有智慧才能用人得当。能够把臣民治理好,便是给他们以恩惠,这样臣民当然会把恩惠记在心里。既然聪明而有恩德,还怕什么(马雚)兜,何必迁徙流放苗民,又何必害怕讲那些花言巧语、献媚取宠的坏人呢?”

皋陶曰:“都!亦行有九德。亦言,其人有德,乃言曰,载采采。”禹曰:“何?”
皋陶说:“啊!检验人的行为大约有九种美德。检验了言论,如果那个人有德,就告诉他说,可做点工作。”禹问:“什么叫九德呢?”

皋陶曰:“宽而栗,柔而立,愿而恭,乱而敬,扰而毅,直而温,简而廉,刚而塞,强而义。彰厥有常,吉哉!日宣三德,夙夜浚明有家;日严祗敬六德,亮采有邦。翕受敷施,九德咸事,俊乂在官。百僚师师,百工惟时,抚于五辰,庶绩其凝。无教逸欲,有邦兢兢业业,一日二日万几。无旷庶官,天工,人其代之。天叙有典,敕我五典五惇哉!天秩有礼,自我五礼有庸哉!同寅协恭和衷哉!天命有德,五服五章哉!天讨有罪,五刑五用哉!政事懋哉懋哉!”“天聪明,自我民聪明。天明畏,自我民明威。达于上下,敬哉有土!”
皋陶说:“宽宏而又坚栗,柔顺而又卓立,谨厚而又严恭,多才而又敬慎,驯服而又刚毅,正直而又温和,简易而又方正,刚正而又笃实,坚强而又合宜,要明显地任用具有九德的好人啊!“天天表现出三德,早晚认真努力于家的人,天天庄严地重视六德,辅助政事于国的人,一同接受,普遍任用,使具有九德的人都担任官职,那么在职的官员就都是才德出众的人了。各位官员互相效法,他们都想处理好政务,而且顺从君王,这样,各种工作都会办成。“治理国家的人不要贪图安逸和私欲,要兢兢业业,因为情况天天变化万端。不要虚设百官,上天命定的工作,人应当代替完成。上天规定了人与人之间的常法,要告诫人们用父义、母慈、兄友、弟恭、子孝的办法,把这五者敦厚起来啊!上天规定了人的尊卑等级,推行天子、诸侯、卿大夫、士和庶人这五种礼制,要经常啊!君臣之间要同敬、同恭,和善相处啊!上天任命有德的人,要用天子、诸侯、卿、大夫、士五等礼服表彰这五者啊!上天惩罚有罪的人,要用墨、劓、剕、宫、大辟五种刑罚处治五者啊!政务要努力啊!要努力啊!“上天的视听依从臣民的视听。上天的赏罚依从臣民的赏罚。天意和民意是相通的,要谨慎啊,有国土的君王!”

皋陶曰:“朕言惠可厎行?”禹曰:“俞!乃言厎可绩。”皋陶曰:“予未有知,思曰赞赞襄哉!”
皋陶问:“我的话可以得到实行吗?”禹说:“当然!你的话可以得到实行并且获得成功。”皋陶说:“我并不懂得什么,我想赞扬佐助帝德啊!”

\chapter{益稷}

帝曰:“来,禹!汝亦昌言。”禹拜曰:“都!帝,予何言?予思日孜孜。”皋陶曰:“吁!如何?”禹曰:“洪水滔天,浩浩怀山襄陵,下民昏垫。予乘四载,随山刊木,暨益奏庶鲜食。予决九川,距四海,浚畎浍距川;暨稷播,奏庶艰食鲜食。懋迁有无,化居。烝民乃粒,万邦作乂。”皋陶曰:“俞!师汝昌言。”
舜帝说:“来吧,禹!你也发表你的意见吧。”禹拜谢说:“啊!君王,我该说什么呢?我只想每天认真的做事。”皋陶说:“啊!究竟怎么样呢?”禹说:“大水弥漫接天,浩浩荡荡地包围了山顶,漫没了丘陵,老百姓沉没陷落在洪水里。我乘坐四种运载工具,沿着山路砍削树木作为路标,同伯益一起把新杀的鸟兽肉送给百姓们。我疏通了九州的河流,使它们流到四海,挖深疏通了田间的大水沟,使它们流进大河。同后稷一起播种粮食,把百谷、鸟兽肉送给老百姓,让他们互通有无,调剂馀缺。于是,百姓们就安定下来了,各个诸侯国开始得到了治理。”皋陶说:“好啊!你的这番话真好啊。”

禹曰:“都!帝,慎乃在位。”帝曰:“俞!禹曰:“安汝止,惟几惟康。其弼直,惟动丕应徯志,以昭受上帝,天其申命用休。”
禹说:“啊!舜帝。你要诚实地对待你的在位的大臣。”舜帝说:“是啊!”禹说:“要安静你的心意,考虑天下的安危。用正直的人做辅佐,只要你行动,天下就会大力响应。依靠有德的人指导接受上天的命令,上天就会再三用休美赐予你。”

帝曰:“吁!臣哉邻哉!邻哉臣哉!”禹曰:“俞!”
舜帝说:“唉!靠大臣啊四邻啊!靠四邻啊大臣啊!”禹说:“对呀!”

帝曰:“臣作朕股肱耳目。予欲左右有民,汝翼。予欲宣力四方,汝为。予欲观古人之象,日、月、星辰、山、龙、华虫、,作会;宗彝、藻、火、粉米、黼、黻,絺绣,以五采彰施于五色,作服,汝明。予欲闻六律五声八音,在治忽,以出纳五言,汝听。予违,汝弼,汝无面从,退有后言。钦四邻!庶顽谗说,若不在时,侯以明之,挞以记之,书用识哉,欲并生哉!工以纳言,时而颺之,格则承之庸之,否则威之。”
舜帝说:“大臣作我的股肱耳目。我想帮助百姓,你辅佐我。我想用力治理好四方,你帮助我。我想显示古人衣服上的图象,用日、月、星辰、山、龙、雉六种田形绘在上衣上;用虎、水草、火、白米、黑白相间的斧形花纹、黑青相间的“己”字花纹绣在下裳上。用五种颜料明显地做成五种色彩不同的衣服,你要做好。我要听六种乐律、五种声音、八类乐器的演奏,从声音的哀乐考察治乱,取舍各方的意见,你要听清,我有过失,你就辅助我。你不要当面顺从,背后又去议论。要敬重左右辅弼的近臣!至于一些愚蠢而又喜欢谗毁、谄媚的人,如果不能明察做臣的道理,要用射侯之礼明确地教训他们,用鞭打警戒他们,用刑书记录他们的罪过,要让他们共同上进!任用官吏要根据他所进纳的言论,好的就称颂宣扬,正确的就进献上去以便采用,否则就要惩罚他们。”

禹曰:“俞哉!帝光天之下,至于海隅苍生,万邦黎献,共惟帝臣,惟帝时举。敷纳以言,明庶以功,车服以庸。谁敢不让,敢不敬应?帝不时敷,同,日奏,罔功。无若丹朱傲,惟慢游是好,傲虐是作。罔昼夜頟頟,罔水行舟。朋淫于家,用殄厥世。予创若时,娶于涂山,辛壬癸甲。启呱呱而泣,予弗子,惟荒度土功。弼成五服,至于五千。州十有二师,外薄四海,咸建五长,各迪有功,苗顽弗即工,帝其念哉!”帝曰:“迪朕德,时乃功,惟叙。”
禹说:“好啊!舜帝,普天之下,至于海内的众民,各国的众贤,都是您的臣子,您要善于举用他们。依据言论广泛地接纳他们,依据工作明确地考察他们,用车马衣服酬劳他们。这样,谁敢不让贤,谁敢不恭敬地接受您的命令?帝不善加分别,好的坏的混同不分,虽然天天进用人,也会劳而无功。“没有象丹朱那样傲慢的,只喜欢懒惰逸乐,只作戏谑,不论白天晚上都不停止。洪水已经退了,他还要乘船游玩,又成群地在家里淫乱,因此不能继承尧的帝位。我为他的这些行为感到悲伤。我娶了涂山氏的女儿,结婚四天就治水去了。后来,启生下来呱呱地啼哭,我顾不上慈爱他,只忙于考虑治理水土的事。我重新划定了五种服役地带,一直到五千里远的地方。每一个州征集三万人,从九州到四海边境,每五个诸侯国设立一个长,各诸侯长领导治水工作。只有三苗顽抗,不肯接受工作任务,舜帝您要为这事忧虑啊!”舜帝说:“宣扬我们的德教,依时布置工役,三苗应该会顺从。”

皋陶方祗厥叙,方施象刑,惟明。
皋陶正敬重那些顺从的,对违抗的,正示以刑杀的图象警戒他们,三苗的事应当会办好。

夔曰:“戛击鸣球、搏拊、琴、瑟、以咏。”祖考来格,虞宾在位,群后德让。下管鼗鼓,合止柷敔,笙镛以间。鸟兽跄跄;箫韶九成,凤皇来仪。夔曰:“于!予击石拊石,百兽率舞。”
夔说:“敲起玉磬,打起搏拊,弹起琴瑟,唱起歌来吧。”先祖、先父的灵魂降临了,我们舜帝的宾客就位了,各个诸侯国君登上了庙堂互相揖让。庙堂下吹起管乐,打着小鼓,合乐敲着柷,止乐敲着敔,笙和大钟交替演奏,扮演飞禽走兽的舞队踏着节奏跳舞,韶乐演奏了九次以后,扮演凤凰的舞队出来表演了。夔说:“唉!我轻敲重击着石磬,扮演百兽的舞队都跳起舞来,各位官长也合着乐曲一同跳起来吧!”

庶尹允谐,帝庸作歌。曰:“敕天之命,惟时惟几。”乃歌曰:“股肱喜哉!元首起哉!百工熙哉!”皋陶拜手稽首颺言曰:“念哉!率作兴事,慎乃宪,钦哉!屡省乃成,钦哉!”乃赓载歌曰:“元首明哉,股肱良哉,庶事康哉!”又歌曰:“元首丛脞哉,股肱惰哉,万事堕哉!”帝拜曰:“俞,往钦哉!”
帝因此作歌。说:“勤劳天命,这样子就差不多了。”于是唱道:“大臣欢悦啊,君王奋发啊,百事发达啊!”皋陶跪拜叩头继续说:“要念念不忘啊!统率起兴办的事业,慎守你的法度,要认真啊!经常考察你的成就,要认真啊!”于是继续作歌说:“君王英明啊!大臣贤良啊!诸事安康啊!”又继续作歌说:“君王琐碎啊!大臣懈怠啊!诸事荒废啊!”舜帝拜谢说:“对啊!我们去认真干吧!”

\part{夏书}

\chapter{禹贡}

禹别九州,随山浚川,任土作贡。禹敷土,随山刊木,奠高山大川。
禹分别土地的疆界,行走高山砍削树木作为路标,以高山大河奠定界域。

冀州:既载壶口,治梁及岐。既修太原,至于岳阳;覃怀厎绩,至于衡漳。厥土惟白壤,厥赋惟上上错,厥田惟中中。恒、卫既从,大陆既作。岛夷皮服,夹右碣石入于河。
冀州:从壶口开始施工后,就治理梁山和它的支脉。太原治理好了以后,又治理到太岳山的南面。覃怀一带的治理取得了成效,又到了横流入河的漳水。这州的土是白壤,赋税是第一等,也夹杂着第二等,这里的田地是第五等。恒水、卫水已经顺着河道而流,大陆泽也已治理了。岛夷的人用皮服来进贡,先接近右边的碣石山,再进入黄河。

济河惟兖州。九河既道,雷夏既泽,灉、沮会同。桑土既蚕,是降丘宅土。厥土黑坟,厥草惟繇,厥木惟条。厥田惟中下,厥赋贞,作十有三载乃同。厥贡漆丝,厥篚织文。浮于济、漯,达于河。
济水与黄河之间是兖州:黄河下游的九条支流疏通了,雷夏也已经成了湖泽,澭水和沮水会合流进了雷夏泽。栽种桑树的地方都已经养蚕,于是人们从山丘上搬下来住在平地上。这里的土质又黑又肥,这里的草是茂盛的,这里的树是修长的。这里的田地是第六等,赋税是第九等,耕作了十三年才与其它八个州相同。这里的贡物是漆和丝,还有用竹筐装着的彩绸。进贡的物品从济水、漯水乘船到黄河。

海岱惟青州。嵎夷既略,潍、淄其道。厥土白坟,海滨广斥。厥田惟上下,厥赋中上。厥贡盐絺,海物惟错。岱畎丝、枲、铅、松、怪石。莱夷作牧。厥篚厴丝。浮于汶,达于济。
渤海和泰山之间是青州:嵎夷治理好以后,潍水和淄水也已经疏通了。这里的土又白又肥,海边有一片广大的盐碱地。这里的田是第三等,赋税是第四等。这里进贡的物品是盐和细葛布,海产品多种多样。还有泰山谷的丝、大麻、锡、松和奇特的石头。莱夷一带可以放牧。这里进贡的物品是用筐装的柞蚕丝。进贡的船只从汶水通到济水。

海、岱及淮惟徐州。淮、沂其乂,蒙、羽其艺,大野既猪,东原厎平。厥土赤埴坟,草木渐包。厥田惟上中,厥赋中中。厥贡惟土五色,羽畎夏翟,峄阳孤桐,泗滨浮磬,淮夷蠙珠暨鱼。厥篚玄纤、缟。浮于淮、泗,达于河。
黄海、泰山及淮河之间是徐州:淮河、沂水治理好以后,蒙山、羽山一带已经可以种植了,大野泽已经停聚着深水,东原地方也获得治理。这里的土是红色的,又粘又肥,草木不断滋长而丛生。这里的田是第二等,赋税是第五等。进贡的物品是五色土,羽山山谷的大山鸡,峄山南面的特产桐木,泗水边上的可以做磬的石头,淮夷之地的蚌珠和鱼。还有用筐子装着的黑色的细绸和白色的绢。进贡的船只从淮河、泗水,到达与济水相通的荷泽。

淮海惟扬州。彭蠡既猪,阳鸟攸居。三江既入,震泽厎定。筱簜既敷,厥草惟夭,厥木惟乔。厥土惟涂泥。厥田唯下下,厥赋下上,上错。厥贡惟金三品,瑶、琨筱、簜、齿、革、羽、毛惟木。島夷卉服。厥篚织贝,厥包桔柚,锡贡。沿于江、海,达于淮、泗。
淮河与黄海之间是扬州:彭蠡泽已经汇集了深水,南方各岛可以安居。三条江水已经流入大海,震泽也获得了安定小竹和大竹已经遍布各地,这里的草很茂盛,这里的树很高大。这里的土是潮湿的泥。田是第九等,赋是第七等,杂出第六等。进贡的物品是金、银、铜、美玉、美石、小竹、大竹、象牙、犀皮、鸟的羽毛、旄牛尾和木材。东南沿海各岛的人穿着草编的衣服。这一带把贝锦放在筐子里,把橘柚包起来作为贡品。这些贡品沿着长江、黄海到达淮河、泗水。

荆及衡阳惟荆州。江、汉朝宗于海,九江孔殷,沱、潜既道,云土、梦作乂。厥土惟涂泥,厥田惟下中,厥赋上下。厥贡羽、毛、齿、革惟金三品,杶、干、栝、柏,砺、砥、砮、丹惟箘簵、楛,三邦厎贡厥名。包匦菁茅,厥篚玄纁玑组,九江纳锡大龟。浮于江、沱、潜、汉,逾于洛,至于南河。
荆山与衡山的南面是荆州:长江、汉水象诸侯朝见天子一样奔向海洋,洞庭湖的水系大定了,沱水、潜水疏通以后,云梦泽一带可以耕作了。这里的土是潮湿的泥,这里的田是第八等,赋是第三等。这里的贡物是羽毛、旄牛尾、象牙、犀皮和金、银、铜,椿树、柘树、桧树、柏树,粗磨石、细磨石、造箭镞的石头、丹砂和细长的竹子、楛木。三个诸侯国进贡他们的名产,包裹好了的杨梅、菁茅,装在筐子里的彩色丝绸和一串串的珍珠。九江进贡大龟。这些贡品从长江、沱水、潜水、汉水到达汉水上游,改走陆路到洛水,再到南河。

荆河惟豫州。伊、洛、瀍、涧既入于河,荥波既猪。导菏泽,被孟猪。厥土惟壤,下土坟垆。厥田惟中上,厥赋错上中。厥贡漆、枲,絺、紵,厥篚纤、纩,锡贡磬错。浮于洛,达于河。
荆山、黄河之间是豫州:伊水、瀍水和涧水都已流入洛水,又流入黄河,荥波泽已经停聚了大量的积水。疏通了菏泽,并在孟猪泽筑起了堤防。这里的土是柔软的壤土,低地的土是肥沃的黑色硬土。这里的田是第四等,赋税是第二等,杂出第一等。这里的贡物是漆、麻、细葛、纻麻,用篚装的绸和细绵,又进贡治玉磬的石头。进贡的船只从洛水到达黄河。

华阳、黑水惟梁州。岷、嶓既艺,沱、潜既道。蔡、蒙旅平,和夷厎绩。厥土青黎,厥田惟下上,厥赋下中,三错。厥贡璆、铁、银、镂、砮磬、熊、罴、狐、狸、织皮,西倾因桓是来,浮于潜,逾于沔,入于渭,乱于河。
华山南部到怒江之间是梁州:岷山、嶓冢山治理以后,沱水、潜水也已经疏通了。峨嵋山、蒙山治理后,和夷一带也取得了治理的功效。这里的土是疏松的黑土,这里的田是第七等,赋税是第八等,还杂出第七等和第九等。这里的贡物是美玉、铁、银、刚铁、作箭镞的石头、磬、熊、马熊、狐狸、野猫。织皮和西倾山的贡物沿着桓水而来。进贡的船只行于潜水,然后离船上岸陆行,再进入沔水,进到渭水,最后横渡渭水到达黄河。

黑水、西河惟雍州。弱水既西,泾属渭汭,漆沮既从,沣水攸同。荆、岐既旅,终南、惇物,至于鸟鼠。原隰厎绩,至于猪野。三危既宅,三苗丕叙。厥土惟黄壤,厥田惟上上,厥赋中下。厥贡惟球、琳、琅玕。浮于积石,至于龙门、西河,会于渭汭。织皮昆仑、析支、渠搜,西戎即叙。
黑水到西河之间是雍州:弱水疏通已向西流,泾河流入渭河之湾,漆沮水已经会合洛水流入黄河,沣水也向北流同渭河会合。荆山、岐山治理以后,终南山、惇物山一直到鸟鼠山都得到了治理。原隰的治理取得了成绩,至于猪野泽也得到了治理。三危山已经可以居住,三苗就安定了。这里的土是黄色的,这里的田是第一等,赋税是第六等。这里的贡物是美玉、美石和珠宝。进贡的船只从积石山附近的黄河,到达龙门、西河,与从渭河逆流而上的船只会合在渭河以北。织皮的人民定居在昆仑、析支、渠搜三座山下,西戎各族就安定顺从了。

导岍及岐,至于荆山,逾于河;壶口、雷首至于太岳;厎柱、析城至于王屋;太行、恒山至于碣石,入于海。
开通了岍山和岐山的道路,到达荆山,越过黄河。又开通壶口山、雷首山,到达太岳山。又开通厎柱山、析城山,到达王屋山。又开通太行山、恒山,到达碣石山,从这里进入渤海。

西倾、朱圉、鸟鼠至于太华;熊耳、外方、桐柏至于陪尾。
开通西倾山、朱圉山、鸟鼠山,到达太华山。又开通熊耳山、外方山、桐柏山,到达陪尾山。

导嶓冢,至于荆山;内方,至于大别。
开通嶓冢山到达荆山。开通内方山到达大别山。

岷山之阳,至于衡山,过九江,至于敷浅原。
开通岷山的南面到达衡山,过洞庭湖到达庐山。

导弱水,至于合黎,馀波入于流沙。
疏通弱水到合黎山,下游流到沙漠。

导黑水,至于三危,入于南海。
疏通黑水到三危山,流入南海。

导河、积石,至于龙门;南至于华阴,东至于厎柱,又东至于孟津,东过洛汭,至于大伾;北过降水,至于大陆;又北,播为九河,同为逆河,入于海。
疏导黄河,从积石山开始,到达龙门山;再向南到达华山的北面;再向东到达厎柱山;又向东到达孟津;又向东经过洛水与黄河会合的地方,到达大分伾山;然后向北经过降水,到达大陆泽;又向北,分成九条支流,再会合成一条逆河,流进大海。

嶓冢导漾,东流为汉,又东,为沧浪之水,过三澨,至于大别,南入于江。东,汇泽为彭蠡,东,为北江,入于海。
从嶓冢山开始疏导漾水,向东流成为汉水;又向东流,成为沧浪水;经过三澨水,到达大别山,向南流进长江。向东,来汇的水叫彭蠡泽;向东,称为北江,流进大海。

岷山导江,东别为沱,又东至于澧;过九江,至于东陵,东迆北,会于汇;东为不江,入于海。
从岷山开始疏导长江,向东另外分出一条支流称为沱江;又向东到达澧水;经过洞庭湖,到达东陵;再向东斜行向北,与淮河会合;向东称为中江,流进大海。

导沇水,东流为济,入于河,溢为荥;东出于陶丘北,又东至于菏,又东北,会于汶,又北,东入于海。
疏导沇水,向东流就称为济水,流入黄河,河水溢出成为荥泽;又从定陶的北面向东流,再向东到达菏泽县;又向东北,与汶水会合;再向北,转向东,流进大海。

导淮自桐柏,东会于泗、沂,东入于海。
从桐柏山开始疏导淮河,向东与泗水、沂水会合,向东流进大海。

导渭自鸟鼠同穴,东会于沣,又东会于泾,又东过漆沮,入于河。
从鸟鼠同穴山开始疏导渭水,向东与沣水会合,又向东与泾水会合;又向东经过漆沮水,流入黄河。

导洛自熊耳,东北,会于涧、瀍;又东,会于伊,又东北,入于河。
从熊耳山开始疏导洛水,向东北,与涧水、沣水会合;又向东,与伊水会合;又向东北,流入黄河。

九州攸同,四隩既宅,九山刊旅,九川涤源,九泽既陂,四海会同。六府孔修,庶土交正,厎慎财赋,咸则三壤成赋。中邦锡土、姓,祗台德先,不距朕行。
九州由此统一了:四方的土地都已经可以居住了,九条山脉都伐木修路可以通行了,九条河流都疏通了水源,九个湖泽都修筑了堤防,四海之内进贡的道路都畅通无阻了。水火金木土谷六府都治理得很好,各处的土地都要征收赋税,并且规定慎重征取财物赋税,都要根据土地的上中下三等来确定它。中央之国赏赐土地和姓氏给诸侯,敬重以德行为先,又不违抗我的措施的贤人。

五百里甸服:百里赋纳总,二百里纳銍,三百里纳秸服,四百里粟,五百里米。
国都以外五百里叫做甸服。离国都最近的一百里缴纳连秆的禾;二百里的,缴纳禾穗;三百里的,缴纳带稃的谷;四百里的,缴纳粗米;五百里的缴纳精米。

五百里侯服:百里采,二百里男邦,三百里诸侯。
甸服以外五百里是侯服。离甸服最近的一百里替天子服差役;二百里的,担任国家的差役;三百里的,担任侦察工作。

五百里绥服:三百里揆文教,二百里奋武卫。
侯服以外五百里是绥服。三百里的,考虑推行天子的政教;二百里的,奋扬武威保卫天子。

五百里要服:三百里夷,二百里蔡。
绥服以外五百里是要服。三百里的,要和平相处;二百里的,要遵守王法。

五百里荒服:三百里蛮,二百里流。
要服以外五百里是荒服。三百里的,维持隶属关系;二百里的,进贡与否流动不定。

东渐于海,西被于流沙,朔南暨声教讫于四海。禹锡玄圭,告厥成功。
东方进至大海,西方到达沙漠,北方、南方连同声教都到达外族居住的地方。于是禹被赐给玄色的美玉,表示大功告成了。

\chapter{甘誓}

启与有扈战于甘之野,作《甘誓》。
启与有扈氏将在甘这个地方进行大战,启作了一篇誓辞叫做《甘誓》。

大战于甘,乃召六卿。
甘这个地方将要进行一场大规模的战争,夏王启于是召见了六军的将领。

王曰:“嗟!六事之人,予誓告汝:有扈氏威侮五行,怠弃三正,天用剿绝其命,今予惟恭行天之罚。左不攻于左,汝不恭命;右不攻于右,汝不恭命;御非其马之正,汝不恭命。用命,赏于祖;弗用命,戮于社,予则孥戮汝。”
王说:“啊!六军的将士们,我告诫你们:有扈氏轻慢洪范这一大法,废弃正德、利用、厚生三大政事,因此,上天要收回他的大命。现在我奉行天地大命前去惩罚他们。兵车左边的兵士不善于射箭,你们就是不奉行我的命令;车右的兵士不善于用戈矛刺杀敌人,你们也是不奉行我的命令;驾车的兵士违反驭马的规则,你们也是不奉行我的命令。执行命令的,我会在先祖的神位面前赏赐你们;不执行命令的,我会在社神的神位面前惩罚你们,将你们变作奴隶,或者是杀掉你们。”

\chapter{五子之歌}

太康失邦,昆弟五人须于洛汭,作《五子之歌》。
康的国家被灭了,他的兄弟五人流浪到洛汭,在该处作了《五子之歌》。

太康尸位,以逸豫灭厥德,黎民咸贰,乃盘游无度,畋于有洛之表,十旬弗反。有穷后羿因民弗忍,距于河,厥弟五人御其母以从,徯于洛之汭。五子咸怨,述大禹之戒以作歌。
太康虽然处在尊位却不理朝事,又贪图安逸享乐,丧失天子应该具备的德行,所以众民都怀着二心;面对这样的情形,他竟还不会改,仍然沉迷玩乐,到洛水的南面打猎,百天还不回来。有穷国的君主羿,趁着百姓对太康的不满,在河北抵御太康,不让他回国。太康的五个弟弟,侍奉他们的母亲,在洛水湾等待太康。这时五人都埋怨太康,因此叙述大禹的教导而写了歌诗。

其一曰:“皇祖有训,民可近,不可下,民惟邦本,本固邦宁。予视天下愚夫愚妇一能胜予,一人三失,怨岂在明,不见是图。予临兆民,懔乎若朽索之驭六马,为人上者,奈何不敬?”
其中一首说:“伟大的祖先曾有明训,百姓可以亲近而不可看轻;百姓是国家的根本,根本牢固,国家才会安宁。我看天下的人,愚夫愚妇都能够胜过我。一人多次犯错,但依然不知道悔悟,百姓的怨恨,难道要明显地表现出来才会感觉到吗?应当考察它还未形成之时。我治理兆民,恐惧得像用坏索子驾着六匹马;做君主的人怎么能不敬不怕?”

其二曰:“训有之,内作色荒,外作禽荒。甘酒嗜音,峻宇雕墙。有一于此,未或不亡。”
其中第二首说:“训诫中有这些话:在宫中沉迷女色,在外沉迷游猎翱翔;喜欢喝酒和爱听音乐,修筑高大殿宇又雕饰宫墙。身为国君,这些事只要有一桩,国家就会灭亡。”

其三曰:“惟彼陶唐,有此冀方。今失厥道,乱其纪纲,乃厎灭亡。”
其中第三首说:“那陶唐氏的尧皇帝,曾经据有冀州这地方。现在废弃他的治国之道,破坏了他所建立的法度,就会灭亡!”

其四曰:“明明我祖,万邦之君。有典有则,贻厥子孙。关石和钧,王府则有。荒坠厥绪,覆宗绝祀!”
其中第四首说:“我们英明睿智的先祖大禹,是万国之君。他建立的典章与法度,留给了后代子孙。征赋和计量平均,王家府库丰殷。现在废弃破坏这些法度,就断绝祭祀又危及宗亲!”

其五曰:“呜呼曷归?予怀之悲。万姓仇予,予将畴依?郁陶乎予心,颜厚有忸怩。弗慎厥德,虽悔可追?”
其中第五首说:“唉!哪里才是我们的归宿?我的心情十分悲痛!万姓都仇恨我们,我们将依靠谁?我的心思郁闷,内心惭愧。不愿慎行祖德,即使懊悔,却不知是否有补救之策?”

\chapter{胤征}

羲和湎淫,废时乱日,胤往征之,作《胤征》。
夏朝时,掌管日月运行的羲和的后代沉湎于淫乱,胤前往征讨,在大战之前作了《胤征》来鼓舞士气。

惟仲康肇位四海,胤侯命掌六师。羲和废厥职,酒荒于厥邑,胤后承王命徂征。告于众曰:“嗟予有众,圣有谟训,明征定保,先王克谨天戒,臣人克有常宪,百官修辅,厥后惟明明,每岁孟春,遒人以木铎徇于路,官师相规,工执艺事以谏,其或不恭,邦有常刑。”“惟时羲和颠覆厥德,沈乱于酒,畔官离次,俶扰天纪,遐弃厥司,乃季秋月朔,辰弗集于房,瞽奏鼓,啬夫驰,庶人走,羲和尸厥官罔闻知,昏迷于天象,以干先王之诛,《政典》曰:‘先时者杀无赦,不及时者杀无赦。’今予以尔有众,奉将天罚。尔众士同力王室,尚弼予钦承天子威命。火炎昆冈,玉石俱焚。天吏逸德,烈于猛火。歼厥渠魁,胁从罔治,旧染污俗,咸与维新。呜呼!威克厥爱,允济;爱克厥威,允罔功。其尔众士懋戒哉!”
仲康开始登上王位统治四海时,命令胤侯为大司马掌管六军。羲氏与和氏放弃他的职守,回到自己的封地嗜酒荒乱。胤侯接受王命,去征伐羲和。胤侯告戒军众说:“啊!众位官长。圣人有谟有训,清楚明白的指明了定国安邦的事。先王能谨慎对待上天的警戒,大臣能遵守常法,百官修治职事辅佐君主,君主就明而又明。每年孟春之月,宣令官员用木铎在路上宣布教令,官长互相规劝,百工依据他们从事的技艺进行谏说。他们有不奉行的,国家将有常刑。“这个羲和颠倒他的行为,沉醉在酒中,背离职位,开始搞乱了日月星辰的运行历程,远远放弃他所司的事。前些时候季秋月的朔日,日月不会合于房,出现日食,乐官进鼓而击,啬夫奔驰取币以礼敬神明,众人跑着供役。羲和主管其官却不知道这件事,对天象昏迷无知,因此触犯了先王的诛罚。先王的《政典》说:历法出现先于天时的事,杀掉无赦;出现后于天时的事,杀掉无赦。“现在我率领你们众长,奉行上天的惩罚。你等众士要与王室同心协力,辅助我认真奉行天子的庄严命令!大火燃烧昆仑山时,美玉和顽石都遭到毁灭;天王的官吏如有过恶行为,害处将比猛火更甚。消灭那些为恶的大首领,协从的人不要惩治;旧时染有污秽习俗的人,都允许更新。“啊!从严治军胜过对士兵的爱惜,一定能成功;爱惜胜过从严要求,作战就不能成功。你等众士要努力戒慎呀!”

\part{商书}

\chapter{汤誓}

伊尹相汤伐桀,升自陑,遂与桀战于鸣条之野,作《汤誓》。
伊尹辅佐商汤讨伐夏桀,队伍开拔到山西永济一带,于是与夏桀在鸣条之野展开大战。战前作动员令,即为《汤誓》。

王曰:“格尔众庶,悉听朕言,非台小子,敢行称乱!有夏多罪,天命殛之。今尔有众,汝曰:『我后不恤我众,舍我穑事而割正夏?”予惟闻汝众言,夏氏有罪,予畏上帝,不敢不正。今汝其曰:‘夏罪其如台?’夏王率遏众力,率割夏邑。有众率怠弗协,曰:‘时日曷丧?予及汝皆亡。’夏德若兹,今朕必往。”
王说:“来吧!诸位,都听我说。不是我小子敢行作乱!是因为夏朝犯下许多罪行,上天命令我去讨伐它。现在你们当中肯定有人会说:‘我们的君王不怜悯我们众人,荒废我们的农事,怎么还能去纠正他人呢?’我虽然理解你们的话,但是夏氏有罪,我畏惧上天,不敢不去征伐啊!现在你们会问:‘夏桀的罪行究竟怎么样呢?’夏王耗尽民力,残酷地剥削压迫自己的百姓。民众对他不满,怠慢不恭,同他的态度也不友好,他们说:‘这个太阳什么时候消失呢?我们愿意同你一起灭亡。’夏国的政治已经败坏到这种程度,现在我一定要去讨伐他。

“尔尚辅予一人,致天之罚,予其大赉汝!尔无不信,朕不食言。尔不从誓言,予则孥戮汝,罔有攸赦。”
“你们只需要辅助我,奉行上天对夏的惩罚,我将重重地赏赐你们!你们不要不相信,我不会说话不算话。如果你们不遵守誓言,我就会把你们降成奴隶,或者杀死你们,不会有所赦免。”

\chapter{仲虺之诰}

成汤放桀于南巢,惟有惭德。曰:“予恐来世以台为口实。”
成汤放逐夏桀使他住在南巢,心里有些惭愧。他说:“我怕后世拿我作为话柄。”仲虺于是向汤作了解释。

仲虺乃作诰,曰:“呜呼!惟天生民有欲,无主乃乱,惟天生聪明时乂,有夏昏德,民坠涂炭,天乃锡王勇智,表正万邦,缵禹旧服。兹率厥典,奉若天命。夏王有罪,矫诬上天,以布命于下。帝用不臧,式商受命,用爽厥师。简贤附势,实繁有徒。肇我邦于有夏,若苗之有莠,若粟之有秕。小大战战,罔不惧于非辜。矧予之德,言足听闻。惟王不迩声色,不殖货利。德懋懋官,功懋懋赏。用人惟己,改过不吝。克宽克仁,彰信兆民。乃葛伯仇饷,初征自葛,东征,西夷怨;南征,北狄怨,曰:‘奚独后予?’攸徂之民,室家相庆,曰:‘徯予后,后来其苏。’民之戴商,厥惟旧哉!佑贤辅德,显忠遂良,兼弱攻昧,取乱侮亡,推亡固存,邦乃其昌。德日新,万邦惟怀;志自满,九族乃离。王懋昭大德,建中于民,以义制事,以礼制心,垂裕后昆。予闻曰:‘能自得师者王,谓人莫己若者亡。好问则裕,自用则小’。呜呼!慎厥终,惟其始。殖有礼,覆昏暴。钦崇天道,永保天命。”
仲虺说:“啊!上天生养人民,人人都有情欲,天下没有君主,人民就会乱作一团,因此上天又生出聪明的人来治理他们。夏桀行为昏乱,人民处于水深火热的困境之中;上天于是赋予勇敢和智慧给大王,让您做万国的表率,继承大禹长久的事业。您现在要遵循大禹的常法,顺从上天的大命!夏王桀有罪,他假托上天的意旨,欺骗百姓。上天因此认为他不善,要我商家承受天命,使我们教导他的众庶。慢待贤明的人,趋炎附势,这种人很多。我们国家刚开始建立时,夏桀对待我们就像对待杂草一般。老百姓和大人物都战慄恐惧,都担心自己无罪受罚;但是你不同,你是有德之人,只要说出话来,就有人信从。大王不近声色,不聚货财;德盛的人用官职劝勉他,功大的人用奖赏劝勉他;勇于采纳他人的意见,改正过错毫不吝惜;能宽能仁,昭信于万民。从前葛国国君,恩将仇报,杀掉我们前往葛国救灾的人,我们的征伐从葛国开始。大王东征时西方的人们就怨恨我们;南征时北方的人就埋怨我们。他们说:为什么把征伐我们的国君摆在后面?但凡被我们征伐的国家,百姓都在家互相庆贺。他们说:君主啊,君主来临,我们就可以兴盛了!天下人民对于商的爱戴,已经很久了!佑助贤德的诸侯,显扬忠良的诸侯;兼并懦弱的,讨伐昏暗的,夺取荒乱的,轻慢走向灭亡的。推求灭亡的道理,以巩固自己的生存,国家就将昌盛。“德行日日革新,天下万国就会怀念;志气自满自大,亲近的九族也会离散。大王要努力显扬大德,对人民建立中道,用义裁决事务,用礼制约思想,把宽裕之道传给后人。我听说能够自己求得老师的人就会为王,以为别人不及自己的人就会灭亡。爱好问,知识就充裕;只凭自己,闻见就狭小。“啊!慎终的办法,在于善谋它的开始。扶植有礼之邦,灭亡昏暴之国;敬重上天这种规律,就可以长久保持天命了。”

\chapter{汤诰}

汤既黜夏命,复归于亳,作《汤诰》。
商汤流放夏桀以后,率部队回到亳都,然后作《汤诰》训示诸侯。

王归自克夏,至于亳,诞告万方。王曰:“嗟!尔万方有众,明听予一人诰。惟皇上帝,降衷于下民。若有恒性,克绥厥猷惟后。夏王灭德作威,以敷虐于尔万方百姓。尔万方百姓,罹其凶害,弗忍荼毒,并告无辜于上下神祇。天道福善祸淫,降灾于夏,以彰厥罪。肆台小子,将天命明威,不敢赦。敢用玄牡,敢昭告于上天神后,请罪有夏。聿求元圣,与之戮力,以与尔有众请命。上天孚佑下民,罪人黜伏,天命弗僭,贲若草木,兆民允殖。俾予一人辑宁尔邦家,兹朕未知获戾于上下,栗栗危惧,若将陨于深渊。凡我造邦,无从匪彝,无即慆淫,各守尔典,以承天休。尔有善,朕弗敢蔽;罪当朕躬,弗敢自赦,惟简在上帝之心。其尔万方有罪,在予一人;予一人有罪,无以尔万方。呜呼!尚克时忱,乃亦有终。”
汤王在战胜夏桀后回来,到了亳邑,大告万方诸侯。汤王说:“啊!众多国家的民众,请认真地听取我的命令。伟大的上天,将美好的德行降于下界人民。让民众长久的保持这种好的品德,只有君王才能够做到。夏王丧失道德滥用刑罚,对百姓施行虐政。所有的百姓都遭受这样的灾难,痛苦不堪,于是你们向上天祈求。天道福佑善人,惩罚坏人,在夏国降下灾难,来显露他的罪过。所以我奉行天命明法,不敢宽恕夏桀的罪行。敢用黑色公牛向天神后土祷告,请求惩治夏桀。于是寻求那伟大的圣人,与你们一起向上天祈求。上天相信并且保护下界的百姓,罪人夏桀被流放斥退失去天子之位,遭到应得的惩罚,上天的旨意是不会出错的,如此,整个国家都焕然一新了。上天使我和睦安定你们的国家,这回伐桀我不知道得罪了天地没有,惊恐畏惧,像要落到深渊里一样。凡我建立的诸侯,不要施行非法,不要追求安乐;要各自遵守常法,以接受上天的福禄。你们有善行,我不敢掩盖;罪过在我自身,我不敢自己宽恕,因为这些在上天心里都明明白白。你们万方有过失,原因都在于我;我有过失,绝不会让你们被牵连。哎!如果能够做到这些的话,就会获得成功。”

\chapter{伊训}

成汤既没,太甲元年,伊尹作《伊训》、《肆命》、《徂后》。
成汤去世以后,后来太甲继承了帝位,这一年伊尹作《伊训》、《肆命》、《徂后》教导太甲。

惟元祀十有二月乙丑,伊尹祠于先王。奉嗣王祗见厥祖,侯、甸群后咸在,百官总已以听冢宰。伊尹乃明言烈祖之成德,以训于王。曰:“呜呼!古有夏先后,方懋厥德,罔有天灾。山川鬼神,亦莫不宁,暨鸟兽鱼鳖咸若。于其子孙弗率,皇天降灾,假手于我有命,造攻自鸣条,朕哉自亳。惟我商王,布昭圣武,代虐以宽,兆民允怀。今王嗣厥德,罔不在初,立爱惟亲,立敬惟长,始于家邦,终于四海。呜呼!先王肇修人纪,从谏弗咈,先民时若。居上克明,为下克忠,与人不求备,检身若不及,以至于有万邦,兹惟艰哉!敷求哲人,俾辅于尔后嗣,制官刑,儆于有位。曰:‘敢有恒舞于宫,酣歌于室,时谓巫风,敢有殉于货色,恒于游畋,时谓淫风。敢有侮圣言,逆忠直,远耆德,比顽童,时谓乱风。惟兹三风十愆,卿士有一于身,家必丧;邦君有一于身,国必亡。臣下不匡,其刑墨,具训于蒙士。’呜呼!嗣王祗厥身,念哉!圣谟洋洋,嘉言孔彰。惟上帝不常,作善降之百祥,作不善降之百殃。尔惟德罔小,万邦惟庆;尔惟不德罔大,坠厥宗。”
太甲元年十二月乙丑日,伊尹在祖庙祭祀先王,侍奉嗣王恭敬地拜见先祖。各诸侯都在祭祀行列,朝廷内百官也都听从百官之长的命令。伊尹于是明白地叙述成汤的大德,来教导太甲。伊尹说:“啊!从前夏代的君主,当他勉力施行德政的时候,这里就没有发生天灾,山川的鬼神也没有不安宁的,连同鸟兽鱼鳖各种动物的生长都很顺遂。到了他的子孙后代,不遵循先人的德政,于是上天降下灾祸,借我们先王之手,让奉行征讨夏朝的命令他。上天有命:先从夏桀讨伐;我们的先祖成汤就从亳都执行。只有我商王宣明德威,用宽厚代替暴虐,所以天下百姓相信我、怀念我。现在我王嗣行成汤的美德,一定要从心在开始培养自身德行!行爱于亲人,行敬于长上,从家和国开始,最终推广到天下。啊!先王努力讲求做人的纲纪,听从谏言而不违反,顺从前贤的活;处在上位能够明察,为臣下能够尽忠;结交人不求全责备,检点自己好像来不及一样。因此达到拥有万国,这是很难的呀!又普求贤智,使他们辅助你们后嗣;还制订《官刑》来警戒百官。《官刑》上说:敢有经常在宫中舞蹈、在房中饮酒酣歌的,这叫做巫风。敢有贪求财货女色、经常游乐田猎的,这叫做淫风。敢有轻视圣人教训、拒绝忠直谏戒、疏远年老有德、亲近顽愚童稚的,这叫做乱风。这些三风十过,卿士身上有一种,他的家一定会丧失;国君身上有一种,他的国一定会灭亡。臣下不匡正君主,要受到墨刑。这些对于下士也要详细教导。啊!嗣王当以这些教导警戒自身,念念不忘呀!圣谟美好,嘉训很明啊!上天的眷顾不常在一家,作善事的,就赐给百福;做恶事就会降下灾难。修德不论多小,只要做了,天下的人都会感到庆幸;你行不善,即使不大,也会使国家覆灭。”

\chapter{太甲上}

太甲既立,不明,伊尹放诸桐。三年复归于亳,思庸,伊尹作《太甲》三篇。
太甲继承帝位以后,不明事理,伊尹把他放逐到桐宫。三年后太甲回到亳地,思考需要的道理,伊尹作《太甲》三篇。

惟嗣王不惠于阿衡,伊尹作书曰:“先王顾諟天之明命,以承上下神祇。社稷宗庙,罔不祗肃。天监厥德,用集大命,抚绥万方。惟尹躬克左右厥辟,宅师,肆嗣王丕承基绪。惟尹躬先见于西邑夏,自周有终。相亦惟终;其后嗣王罔克有终,相亦罔终,嗣王戒哉!祗尔厥辟,辟不辟,忝厥祖。”
嗣王太甲不听从伊的劝告,伊尹上书给王说:“先王成汤顾念上天的命令是正确的,因此供奉上下神祇、宗庙社稷无不恭敬严肃。上天看到汤的善政,因此降下重大使命,让他治理安定四方。我能亲辅助君主建功立业,让百姓安居乐业,所以王您才能继承先王的基业。我亲眼见到西方夏邑的君主,用忠信取得成就,大臣们也就能够保持忠信因而善终;他们后继的王不能取得成就,辅相大臣也没有成就。王您一定要警戒呀!应当敬重你做君主的法则,做君主而不尽君道,将会羞辱自己的祖先。”

王惟庸罔念闻。伊尹乃言曰:“先王昧爽丕显,坐以待旦。旁求俊彦,启迪后人,无越厥命以自覆。慎乃俭德,惟怀永图。若虞机张,往省括于度则释。钦厥止,率乃祖攸行,惟朕以怿,万世有辞。”
王仍然像往常一样,没有将这些话放在心上。于是伊尹就说:“先王在天将明未明的时候,就思考国事,坐着等待天明。遍寻有才能的贤士来做官员,教导后人,不要忘记先祖的教导以自取灭亡。您要慎行俭约的美德,怀着长久的计谋。好象虞人张开了弓,还要去察看箭尾符合法度以后,才发射一样;您要重视自己所要达到的目的,遵行你的祖先的措施!这样我就高兴了,千秋万世您将会得到美好的声誉。”

王未克变。伊尹曰:“兹乃不义,习与性成。予弗狎于弗顺,营于桐宫,密迩先王其训,无俾世迷。王徂桐宫居忧,克终允德。”
太甲不依然不改变自己的行为。伊尹对群臣说:“王这是不义的行为。习惯将同生性相结合,我不能轻视不顺教导的人。要在桐营建造宫殿,让他聆听先王的教诲,莫让他终身迷误。”嗣王去桐宫,处在忧伤的环境,能够成就诚信的美德。

\chapter{太甲中}

惟三祀十有二月朔,伊尹以冕服奉嗣王归于亳,作书曰:“民非后,罔克胥匡以生;后非民,罔以辟四方。皇天眷佑有商,俾嗣王克终厥德,实万世无疆之休。”
太甲继位第三年,十二月初一,伊尹戴着礼帽穿着礼服迎接嗣王太甲回到亳都,给王上书说:“百姓没有君主,不能互相匡正而生活;君主没有百姓,就不能治理四方。上天顾念帮助商朝,让王能培养出高尚的品德,实在是万代无疆之美啊!”

王拜手稽首曰:“予小子不明于德,自厎不类。欲败度,纵败礼,以速戾于厥躬。天作孽,犹可违;自作孽,不可逭。既往背师保之训,弗克于厥初,尚赖匡救之德,图惟厥终。”
王拜跪叩头说:“我不知道什么是德,自己招致不善。多欲就败坏法度,放纵就败坏礼制,因此给自身召来了罪过。上天造成的灾祸,还可回避;自己造成的灾祸,就无法逃避了。以前我违背老师的教训,没有在开始就开个好头;还望依靠您的匡救的恩德,才能得到好的结果。”

伊尹拜手稽首曰:“修厥身,允德协于下,惟明后。先王子惠困穷,民服厥命,罔有不悦。并其有邦厥邻,乃曰:『徯我后,后来无罚。』王懋乃德,视乃厥祖,无时豫怠。奉先思孝,接下思恭。视远惟明;听德惟聪。朕承王之休无斁。”
伊尹跪拜叩头,说:“提高自身的修养,又用诚信的美德和谐臣下,这就是明君应该做的。先王成汤爱护穷困的百姓,所以百姓服从他的教导,没有不高兴的。连邻国百姓也拥护他,便说:等待我们的君主吧,我们的君主来了,就没有祸患了。王想要培养自身的美德,不妨看看先祖的所作所为,不可有顷刻的安乐懈怠。事奉先人,当思孝顺;接待臣下,当思恭敬。观察远方要眼明,顺从有德要耳聪。能够这样,我享受王的幸福就会没有止境。”

\chapter{太甲下}

伊尹申诰于王曰:“呜呼!惟天无亲,克敬惟亲。民罔常怀,怀于有仁。鬼神无常享,享于克诚。天位艰哉!德惟治,否德乱。与治同道,罔不兴;与乱同事,罔不亡。终始慎厥与,惟明明后。先王惟时懋敬厥德,克配上帝。今王嗣有令绪,尚监兹哉。若升高,必自下,若陟遐,必自迩。无轻民事,惟艰;无安厥位,惟危。慎终于始。有言逆于汝心,必求诸道;有言逊于汝志,必求诸非道。呜呼!弗虑胡获?弗为胡成?一人元良,万邦以贞。君罔以辩言乱旧政,臣罔以宠利居成功,邦其永孚于休。”
伊尹再次告诫王说:“哎!上天不偏袒任何人,能够恭敬的做事,上天就会爱护;百姓没有经常归附的君主,他们归附仁爱的君主;鬼神不会固定享受谁的祭祀,只享受诚信的人的祭祀。处在天子的位置是很困难的呀!任用有贤德的人,则天下大治,不用有德的人则天下大乱。与治世之君走相同的道路,国家就会兴盛;与乱世之军走相同的道路,国家没有不灭亡的。终和始都慎择自己的同事,就是英明的君主。先王因此勉力敬修自己的德行,所以能够匹配上天。现在我王继续享有好的基业,希望看到这一点呀!如果升高,一定要从下面开始;如果行远,一定要从近处开始。不要轻视人民的事务,要想到它的难处;不要苟安君位,要想到它的危险。慎终要从开头做起啊!有些话不顺你的心意,一定要从道义来考求;有些话顺从你的心意,一定要从不道义来考求。啊!不思考,怎么会有收获?不做事,怎么会成功?天子贤明,天下诸侯国都会尊敬。君主不要使用巧辩扰乱旧政,臣下不要凭仗骄宠和利禄而安居成功。这样,国家将永久保持在美好之中。”

\chapter{咸有一德}

伊尹作《咸有一德》。
伊尹写下了《咸有一德》。

伊尹既复政厥辟,将告归,乃陈戒于德。曰:“呜呼!天难谌,命靡常。常厥德,保厥位。厥德匪常,九有以亡。夏王弗克庸德,慢神虐民。皇天弗保,监于万方,启迪有命,眷求一德,俾作神主。惟尹躬暨汤,咸有一德,克享天心,受天明命,以有九有之师,爰革夏正。非天私我有商,惟天佑于一德;非商求于下民,惟民归于一德。德惟一,动罔不吉;德二三,动罔不凶。惟吉凶不僭在人,惟天降灾祥在德。今嗣王新服厥命,惟新厥德。终始惟一,时乃日新。任官惟贤材,左右惟其人。臣为上为德,为下为民。其难其慎,惟和惟一。德无常师,主善为师。善无常主,协于克一。俾万姓咸曰:‘大哉王言。’又曰:‘一哉王心’。克绥先王之禄,永厎烝民之生。呜呼!七世之庙,可以观德。万夫之长,可以观政。后非民罔使;民非后罔事。无自广以狭人,匹夫匹妇,不获自尽,民主罔与成厥功。”
伊尹已经把政权归还给太甲,将要告老回到他的私邑,于是陈述纯一之德,告戒太甲。伊尹说:“唉!上天难信,天命无常。经常修德,可以保持君位;修德不能经常,九州因此就会失掉。夏桀不能经常修德,怠慢神明,虐待人民。皇天不安,观察万方,开导佑助天命的人,眷念寻求纯德的君,使他作为百神之主。只有伊尹自身和成汤都有纯一之德,能合天心,接受上天的明教,因此拥有九州的民众,于是革除了夏王的虐政。这不是上天偏爱我们商家,而是上天佑助纯德的人;不是商家求请于民,而是人民归向纯德的人。德纯一,行动起来无不吉利;德不纯一,行动起来无不凶险。吉和凶不出差错,虽然在人;上天降灾降福,却在于德啊!现在嗣王新受天命,要更新自己的品德;始终如一而不间断,这样就能日日更新。任命官吏当用贤才,任用左右大臣当用忠良。大臣协助君上施行德政,协助下属治理人民;对他们要重视,要慎重,当和谐,当专一。德没有不变的榜样,以善为准则就是榜样;善没有不变的准则,协合于能够纯一的人就是准则。要使万姓都说:重要呀!君王的话。又说:纯一呀!君王的心。这样,就能安享先王的福禄,长久安定众民的生活。啊呀!供奉七世祖先的宗庙,可以看到功德;万夫的首长,可以看到行政才能。君主没有人民就无人任用,人民没有君主就无处尽力。不可自大而小视人,小视人就不能尽人的力量。平民百姓如果不得各尽其力,人君就没有人帮助建立功勋。”

\chapter{盘庚上}

盘庚五迁,将治亳殷,民咨胥怨。作《盘庚》三篇。
到盘庚时商朝面临第五次迁都,他决定把国都从亳迁到殷。这一决定受到贵族们的反对,影响很坏,于是盘庚写下三篇《《盘庚》。

盘庚迁于殷,民不适有居,率吁众戚出矢言曰:“我王来,即爰宅于兹,重我民,无尽刘。不能胥匡以生,卜稽,曰其如台?先王有服,恪谨天命,兹犹不常宁;不常厥邑,于今五邦。今不承于古,罔知天之断命,矧曰其克从先王之烈?若颠木之有由蘖,天其永我命于兹新邑,绍复先王之大业,厎绥四方。”
盘庚将把都城迁到殷。臣民不愿往那个处所,相率呼吁一些贵戚大臣出来,向他们陈述意见。臣民说:“我们的君王迁来,既已改居在这里,是看重我们臣民,不使我们受到伤害。现在我们不能互相救助,以求生存,用龟卜稽考一下,将怎么样呢?先王有事,敬慎地遵从天命。这里还不能长久安宁吗?不能长久住在一个地方,到现在已经五个国都了!现在不继承先王敬慎天命的传统,就不知道老天所决定的命运,更何况说能继承先王的事业呢?好像倒伏的树又长出了新枝、被砍伐的残余又发出嫩芽一样,老天将使我们的国运在这个新都奄邑延续下去,继续复兴先王的大业,安定天下。”

盘庚斅于民,由乃在位以常旧服,正法度。曰:“无或敢伏小人之攸箴!”王命众,悉至于庭。
盘庚开导臣民,又教导在位的大臣遵守旧制、正视法度。他说:“不要有人敢于凭借小民的谏诫,反对迁都!”于是,王命令众人,都来到朝廷。

王若曰:“格汝众,予告汝,训汝猷,黜乃心,无傲从康。
王这样说:“来吧,你们各位,我要告诉你们,开导你们。可克制你们的私心,不要傲求安。

古我先王,亦惟图任旧人共政。王播告之,修不匿厥指。王用丕钦;罔有逸言,民用丕变。今汝聒聒,起信险肤,予弗知乃所讼。非予自荒兹德,惟汝含德,不惕予一人。予若观火,予亦拙谋作,乃逸。
从前我们的先王,也只是谋求任用旧臣共同管理政事。施行先王的教令,他们不隐瞒教的旨意,先王因此敬重他们。他们没有错误的言论,百姓们因此也大变了。现在你们拒绝我的意,自以为是,起来申说危害虚浮的言论,我不知道你们争辩的意图。并不是我自己放弃了任用旧人的美德,而是你们包藏好意而不施给我。我对当前形势像看火一样地清楚,我如果又不善于谋划和行动,那就错了。

若网在纲,有条而不紊;若农服田力穑,乃亦有秋。汝克黜乃心,施实德于民,至于婚友,丕乃敢大言汝有积德。乃不畏戎毒于远迩,惰农自安,不昏作劳,不服田亩,越其罔有黍稷。
好像把网结在纲上,才能有条理而不紊乱;好像农民从事田间劳动,只有努力耕种,才会大有收成。你们能克制私心,把实际的好处施给百姓,以至于亲戚朋友,于是才敢扬言你们有积德。如果你们不怕远近会出现大灾害,像懒惰的农民一样自求安逸,不努力操劳,不从事田间劳动,就会没有黍稷。

汝不和吉言于百姓,惟汝自生毒,乃败祸奸宄,以自灾于厥身。乃既先恶于民,乃奉其恫,汝悔身何及!相时憸民,犹胥顾于箴言,其发有逸口,矧予制乃短长之命!汝曷弗告朕,而胥动以浮言,恐沈于众?若火之燎于原,不可向迩,其犹可扑灭。则惟汝众自作弗靖,非予有咎。
你们不向老百姓宣布我的善言,这是你们自生祸害,即将发生灾祸邪恶,而自己害自己。假若已经引导人们做了坏事,而又承受那些痛苦,你们悔恨自己又怎么来得及?看看这些小人吧,他们尚且顾及规劝的话,顾及发出错误言论,何况我掌握着你们或短或长的生命呢?你们为什么不亲自告诉我,却用些无稽之谈互相鼓动,恐吓煽动民众呢?好像大火在原野上燃烧一样,不能面向,不能接近,还能够扑灭吗?这都是你们众人自己做了不好的事,不是我有过错。

迟任有言曰:‘有惟求旧,器非求旧,惟新。’古我先王暨乃祖乃父胥及逸勤,予敢动用非罚?世选尔劳,予不掩尔善。兹予大享于先王,尔祖其从与享之。作福作灾,予亦不敢动用非德。
迟任说过:‘人要寻求旧的,器物不要寻求旧的,要新。’过去我们的先王同你们的祖辈父辈共同勤劳,共享安乐,我怎么敢对你们施行不恰当的刑罚呢?世世代代都会说到你们的功劳,我不会掩盖你们的好处。现在我要祭祀我们的先王,你们的祖先也将跟着享受祭祀。赐福降灾,我也不敢动用不恰当的赏赐或惩罚。

予告汝于难,若射之有志。汝无侮老成人,无弱孤有幼。各长于厥居。勉出乃力,听予一人之作猷。
我在患难的时候告诉你们,要像射箭有箭靶一样,你们不能偏离我。你们不要轻视成年人,也不要看不起年幼的人。你们各人领导着自己的封地,努力使出你们的力量,听从我一人的谋划。

无有远迩,用罪伐厥死,用德彰厥善。邦之臧,惟汝众;邦之不臧,惟予一人有佚罚。凡尔众, 其惟致告:自今至于后日,各恭尔事,齐乃位,度乃口。 罚及尔身,弗可悔。”
没有远和近的分别,我用刑罚惩处那些坏的,用赏赐表彰那些好的。国家治理得好,是你们众人的功劳;国家治理得不好,是我有过有罪。你们众人,要思考我告诫的话:从今以后,各人认真地做好你们的事情,加速你们的布置,闭上你们的口,不许乱说。否则,惩罚到你们身上,后悔也可不能啊!”

\chapter{盘庚中}

盘庚作,惟涉河以民迁。乃话民之弗率,诞告用亶。其有众咸造,勿亵在王庭,盘庚乃登进厥民。曰:“明听朕言,无荒失朕命!呜呼!古我前后,罔不惟民之承保。后胥戚鲜,以不浮于天时。殷降大虐,先王不怀厥攸作,视民利用迁。汝曷弗念我古后之闻?承汝俾汝惟喜康共,非汝有咎比于罚。予若吁怀兹新邑,亦惟汝故,以丕从厥志。
盘庚作了君主以后,计划渡过黄河带领臣民迁移。于是,集合了那些不服从的臣民,用至诚普告他们。那些民众都来了,旗帜在王庭飘扬。盘庚于是登上高处,招呼他们靠前一些。盘庚说:“你们要听清楚我的话,不要忽视我的命令!啊!从前我们的先王,没有谁不想顺承和安定人民。君王清楚大臣也明白,因此没有被天灾所惩罚。从前上天盛降大灾,先王不安于自己所作的都邑,考察臣民的利益而迁徙。你们为什么不想想我们先王的这些传闻呢?我顺从你们喜欢安乐和稳定的心愿,反对你们有灾难而陷入刑罚。我若呼吁你们安居在这个新都,也是关心你们的祸灾,并且远遵先王的意愿吗?

今予将试以汝迁,安定厥邦。汝不忧朕心之攸困,乃咸大不宣乃心,钦念以忱动予一人。尔惟自鞠自苦,若乘舟,汝弗济,臭厥载。尔忱不属,惟胥以沈。不其或稽,自怒曷瘳?汝不谋长以思乃灾,汝诞劝忧。今其有今罔后,汝何生在上?
现在我打算率领你们迁移,使国家安定。你们不体谅我内心的困苦,你们的心竟然都很不和顺,很想用些不正确的话来动摇我。你们自己搞得走投无路,自寻烦恼,譬如坐在船上,你们不渡过去,这将会把事情搞坏。你们诚心不合作,那就只有一起沉下去。不能协同一致,只是自己怨怒,又有什么好处呢?你们不作长久打算,不想想灾害,你们普遍安于忧患。这样下去,将会有今天而没有明天了,你们怎么能生活在这个地面上呢?

今予命汝,一无起秽以自臭,恐人倚乃身,迂乃心。予迓续乃命于天,予岂汝威,用奉畜汝众。
现在我命令你们同心同德,不要传播谣言来败坏自己,恐怕有人会使你们的身子不正,使你们心地歪邪。我向上天劝说延续你们的生命,我哪里是要虐待你们啊,我是要帮助你们、养育你们众人。

予念我先神后之劳尔先,予丕克羞尔,用怀尔,然。失于政,陈于兹,高后丕乃崇降罪疾,曰‘曷虐朕民?’汝万民乃不生生,暨予一人猷同心,先后丕降与汝罪疾,曰:‘曷不暨朕幼孙有比?’故有爽德,自上其罚汝,汝罔能迪。古我先后既劳乃祖乃父,汝共作我畜民,汝有戕则在乃心!我先后绥乃祖乃父,乃祖乃父乃断弃汝,不救乃死。
我想到我们神圣的先王曾经烦劳你们祖先,我才把使你们安定的意见贡献给你们;然而如果耽误了事,长久居住在这里,先王就会重重地降下罪责,问道:‘为什么虐待我的臣民?’你们万民如果不去谋生,不和我同心同德,先王也会对你们降下罪责,问道:‘为什么不同我的幼孙亲近友好?’因此,有了过错,上天就将惩罚你们,你们不能长久。从前我们的先王已经烦劳你们的祖先和父辈,你们都作为我养育的臣民,你们内心却又怀着恶念!我们的先王将会告诉你们的祖先和父辈,你们的祖先和父辈就会断然抛弃你们,不会挽救你们的死亡。

兹予有乱政同位,具乃贝玉。乃祖乃父丕乃告我高后曰:‘作丕刑于朕孙!’迪高后丕乃崇降弗祥。
现在我有乱事的大臣,聚集财物。你们的祖先和父辈于是就会告诉我们的先王说:‘对我们的子孙用大刑吧!’于是,先王就会重重地降下刑罚。

呜呼!今予告汝:不易!永敬大恤,无胥绝远!汝分猷念以相从,各设中于乃心。乃有不吉不迪,颠越不恭,暂遇奸宄,我乃劓殄灭之,无遗育,无俾易种于兹新邑。
啊!现在我告诉你们:不要轻举妄动!要永远警惕大的忧患,不要互相疏远!你们应当考虑顺从我,各人心里都要和和善善。假如有人不善良,不走正道,违法不恭,欺诈奸邪,胡作非为,我就要断绝消灭他们,不留他们的后代,不让他们这些坏人在这个新国都里延续种族。

往哉!生生!今予将试以汝迁,永建乃家。”
去吧,去谋生吧!现在我将率领你们迁徙,永久建立你们的家园。”

\chapter{盘庚下}

盘庚既迁,奠厥攸居,乃正厥位,绥爰有众。曰:“无戏怠,懋建大命!今予其敷心腹肾肠,历告尔百姓于朕志。罔罪尔众,尔无共怒,协比谗言予一人。古我先王将多于前功,适于山。用降我凶,德嘉绩于朕邦。今我民用荡析离居,罔有定极,尔谓朕曷震动万民以迁?肆上帝将复我高祖之德,乱越我家。朕及笃敬,恭承民命,用永地于新邑。肆予冲人,非废厥谋,吊由灵。各非敢违卜,用宏兹贲。
盘庚迁都以后,定好住的地方,才决定宗庙朝廷的位置,然后告诫众人。盘庚说:“不要戏乐、懒惰,努力传达我的教命吧!现在我诚心把我的意思告诉你们各位官员。我不会惩罚你们众人,你们也不要共同发怒,联合起来,毁谤我一个人。从前我们的先王想光大前人的功业,迁往山地。因此减少了洪水给我们的灾祸,在我国获得了好效果。现在我们的臣民由于洪水动荡奔腾而流离失所,没有固定的住处,你们反而问我为什么要惊动众人而迁徙!现在上天要兴复我们高祖的美德,光大我们的国家。我急切、笃实、恭谨地遵从上天的意志,奉命延续你们的生命,率领你们长远居住在新都。所以我这个年轻人,不是敢于废弃你们的谋划,是要善于遵行上天的谋度;不是敢于违背卜兆,是要发扬光大上天这一美好的指示。

呜呼!邦伯师长百执事之人,尚皆隐哉!予其懋简相尔念敬我众。朕不肩好货,敢恭生生。鞠人谋人之保居,叙钦。今我既羞告尔于朕志若否,罔有弗钦!无总于货宝,生生自庸。式敷民德,永肩一心。”
啊!各位诸侯、各位官长以及全体官员,你们都要考虑考虑啊!我将要尽力考察你们惦念尊重我们民众的情况。我不会任用贪财的人,只任用经营民生的人。对于那些能养育民众并能谋求他们安居的人,我将依次敬重他们。现在我已经把我心里的好恶告诉你们了,不要有不顺从的!不要聚敛财宝,要经营民生以自立功勋!要把恩惠施给民众,永远能够与民众同心!”

\chapter{说命上}

高宗梦得说,使百工营求诸野,得诸傅岩,作《说命》三篇。
殷之贤王高宗,在梦中得到一个贤相,他的名字叫“说”。群臣之内没此人,他便让百官依据他所梦之处寻找,在傅岩寻找到说,写下《说命》三篇。

王宅忧,亮阴三祀。既免丧,其惟弗言,群臣咸谏于王曰:“呜呼!知之曰明哲,明哲实作则。天子惟君万邦,百官承式,王言惟作命,不言臣下罔攸禀令。”
殷高宗武丁居父丧,信任冢宰默默不言,已经三年。丧期满后,他还是不论政事。于是群臣纷纷向王进谏说:“哎!通晓事理的叫做圣名睿智之人,圣名睿智的人才可制作法则。天子统治天下,文武百官都按照天子的命令行事。天子的话就是命令,天子不说,臣下就无从接受教命。”

王庸作书以诰曰:“以台正于四方,惟恐德弗类,兹故弗言。恭默思道,梦帝赉予良弼,其代予言。”乃审厥象,俾以形旁求于天下。说筑傅岩之野,惟肖。爰立作相。王置诸其左右。
王写下诏书告诫群臣说:“要我做四方的表率,而我惟恐自己的德行不好,所以不敢说话。我恭敬沉默思考治国的办法,梦见上天赐给我一位贤良的辅佐,让他代替我发言。”于是详细画出了他的形像,派遣人拿着图像到天下普遍寻找。傅说在傅岩之野筑土,同图像上的人非常相似。于是王立他为相,把他安置在自己的左右。”

命之曰:“朝夕纳诲,以辅台德。若金,用汝作砺;若济巨川,用汝作舟楫;若岁大旱,用汝作霖雨。启乃心,沃朕心,若药弗瞑眩,厥疾弗瘳;若跣弗视地,厥足用伤。惟暨乃僚,罔不同心,以匡乃辟。俾率先王,迪我高后,以康兆民。呜呼!钦予时命,其惟有终。”
王命令他说:“请早晚进谏,来帮助我修德!比如铁器,要用你作磨石;比如渡大河,要用你作船和桨;比如年岁大旱,要用你作霖雨。敞开你的心泉来灌溉我的心吧!药物不猛烈,疾病就无法治好;如果赤脚走路,脚就会很容易受伤。希望你和你的同僚,齐心协力来辅佐你的君王,使他依从先王的道路,按照先祖的治国之道治理国家,使天下百姓安居乐业。啊!重视我的这个命令,要考虑取得成绩!”

说复于王曰:“惟木从绳则正,后从谏则圣。后克圣,臣不命其承,畴敢不祗若王之休命?”
傅说回答说:“木依从绳墨砍削就会正直,君主依从谏言行事就会圣明。君主能够圣明,臣下不必等待教命就将奉行,谁敢不恭敬顺从我王的英明的命令呢?”

\chapter{说命中}

惟说命总百官,乃进于王曰:“呜呼!明王奉若天道,建邦设都,树后王君公,承以大夫师长,不惟逸豫,惟以乱民。惟天聪明,惟圣时宪,惟臣钦若,惟民从乂。惟口起羞,惟甲胄起戎,惟衣裳在笥,惟干戈省厥躬。王惟戒兹,允兹克明,乃罔不休。惟治乱在庶官。官不及私昵,惟其能;爵罔及恶德,惟其贤。虑善以动,动惟厥时。有其善,丧厥善;矜其能,丧厥功。惟事事,乃其有备,有备无患。无启宠纳侮,无耻过作非。惟厥攸居,政事惟醇。黩予祭祀,时谓弗钦。礼烦则乱,事神则难。”
傅说接受王命统领文武百官,向王进言说:“啊!古代明王顺从天道,建立邦国,设置都城,树立侯王君公,又以大夫众长辅佐他们,并不是为了让王安逸,而是为了治理天下百姓。上天聪明公正,圣明的君主善于效法它,臣下敬顺它,人民就顺从治理了。号令轻易发出会引起羞辱;甲胄轻用会引起战争;衣裳放在箱子里不用来奖励,会损害自己;干戈藏在府库里不用来讨伐,会伤害自身。王您一定要严肃谨慎地对待这些事情!如果您这样做了,政治就无不美好了。治和乱在于众官。官职不可授予亲近,当授予能者;爵位不可赐给坏人,当赐给贤人。考虑妥善而后行动,行动又当适合时机。夸自己美好,就会失掉美好,夸自己能干,就会失去成功。做事情,要有准备,有准备才没有后患。不要开宠幸的途径而受侮辱;不要以改过为耻而形成大非。这样思考所担任的事,政事就不会杂乱。轻慢对待祭祀,这叫不敬。礼神繁琐就会乱,这样,事奉鬼神就难了。”

王曰:“旨哉!说。乃言惟服。乃不良于言,予罔闻于行。”说拜稽首曰:“非知之艰,行之惟艰。王忱不艰,允协于先王成德,惟说不言有厥咎。”
王说:“好呀!傅说。你所说的这些都应当施行。你如果不善于进言,那我就没有办法听到并付诸行动了。”傅说跪拜叩头,说道:“懂这些道理并不困难,而是实行它很困难。王只要认可这些道理就不会困难,就真合于先王的盛德;我如果不说,那就是我的过失了。”

\chapter{说命下}

王曰:“来!汝说。台小子旧学于甘盘,既乃遯于荒野,入宅于河。自河徂亳,暨厥终罔显。尔惟训于朕志,若作酒醴,尔惟麴糵;若作和羹,尔惟盐梅。尔交修予,罔予弃,予惟克迈乃训。”
王说:“来呀!你傅说。我以前向甘盘学习过,但就没过多久我就避到荒野,居住在河洲,后来又从河洲回到亳都,但是我的学业还是没有什么长进。你当顺从我想学的志愿,比如作甜酒,你就是那曲蘖;比如作羹汤,你就是那盐和梅。你要多方指正我,不要抛弃我;我也一定会按照你教导执行的去。”

说曰:“王,人求多闻,时惟建事,学于古训乃有获。事不师古,以克永世,匪说攸闻。惟学,逊志务时敏,厥修乃来。允怀于兹,道积于厥躬。惟学学半,念终始典于学,厥德修罔觉。监于先王成宪,其永无愆。惟说式克钦承,旁招俊乂,列于庶位。”
傅说说:“王!人们要求增多知识,是想建立事业。要学习古训,才才会有所收获;建立事业却不效法古训,而能长治久安的,我傅说的是不赞同这种说法的。学习要心志谦逊,务必时刻努力,如此学习方能有所进步。相信和记住这些,道德在自己身上将积累增多。教人是学习的一半,思念终和始取法于学习,道德的增长就会不知不觉了。借鉴先王的成法,将永久没有失误;我傅说因此能够敬承你的意旨,广求贤俊,把他们安排在各种职位上。”

王曰:“呜呼!说,四海之内,咸仰朕德,时乃风。股肱惟人,良臣惟圣。昔先正保衡作我先王,乃曰:‘予弗克俾厥后惟尧舜,其心愧耻,若挞于市。’一夫不获,则曰时予之辜。佑我烈祖,格于皇天。尔尚明保予,罔俾阿衡专美有商。惟后非贤不乂,惟贤非后不食。其尔克绍乃辟于先王,永绥民。”
王说:“啊!傅说。天下的人都敬仰我的德行,这都是你的教化所致。手足完备就是成人,良臣具备就是圣君。从前先贤伊尹使我的先王兴起,他这样说:我不能使我的君王做尧舜,我心惭愧耻辱,好比在闹市受到鞭打一样。一人不得其所,他就说:这是我的罪过。他辅助我的烈祖成汤受到皇天赞美。你要勉力扶持我,不要让伊尹专美于我商家!君主得不到贤人就不会治理,贤人得不到君主就不会被录用。你要能让你的君主继承先王,长久安定人民。”

说拜稽首曰:“敢对扬天子之休命。”
说跪拜叩头,说:“请让我报答宣扬天子的美好教导!”\\

\chapter{高宗肜日}

高宗祭成汤,有飞雉升鼎耳而雊,祖己训诸王,作《高宗肜日》、《高宗之训》。
高宗(武丁)祭先帝成汤,有一只野鸡在鼎耳上鸣叫。祖己训王,并作《高宗肜日》、《高宗之训》两文。

高宗肜日,越有雊雉。祖己曰:“惟先格王,正厥事。”乃训于王。曰:“惟天监下民,典厥义。降年有永有不永,非天夭民,民中绝命。民有不若德,不听罪。天既孚命正厥德,乃曰:‘其如台?’呜呼!王司敬民,罔非天胤,典祀无丰于昵。”
在祭祀高宗的第二天,又举行了盛大的祭祀典礼,有一只野鸡在鼎耳上鸣叫。祖己说:“要先宽解君王的心,然后纠正他祭祀的事。”于是开导祖庚。祖己说:“上天考察下民,赞美他们合宜行事。上天赐给人的寿命有长有短,并不是上天故意让人缩短寿命,而是有些人自己的行为不合义理,断绝了自己的生命。有些人有不好的品德,有不顺从天意的罪过。上天就降下惩罚纠正他们不好的品德,他却说:‘要怎么样呢?’”啊!王啊,您继承王位之后,要恭敬的对待您的臣民,他们都是老天的后代,在祭祀的时候,祭品不要过于丰盛啦!

\chapter{西伯戡黎}

殷始咎周,周人乘黎。祖伊恐,奔告于受,作《西伯戡黎》。
殷商开始责怪周国的时候,是在周人凌驾黎国之时。祖伊惊恐,赶紧报告于商纣王,史官记录作成《西伯戡黎》。

西伯既戡黎,祖伊恐,奔告于王。曰:“天子!天既讫我殷命。格人元龟,罔敢知吉。非先王不相我后人,惟王淫戏用自绝。故天弃我,不有康食。不虞天性,不迪率典。今我民罔弗欲丧,曰:‘天曷不降威?’大命不挚,今王其如台?”
周文王打败了黎国以后,祖伊十分恐慌,赶紧跑来告诉纣王。祖伊说:“天子,上天恐怕要终止我们殷商的国运了!占卜的圣人用大龟来卜,始终没有遇上吉兆,不是先王不扶助我们后人,而是大王沉溺于酒色之中而自绝于先王。所以上天抛弃了我们,不让我们得到糟糠之食。大王不揣度天性,不遵循法律。如今天下百姓没有不希望国家灭亡的,他们说:‘老天为什么还不降威罚呢?’天命不再归向我们了,现在大王将要怎么办呢?”

王曰:“呜呼!我生不有命在天?”
纣王说:“哎!我的一生不有福命在天吗?”

祖伊反曰:“呜呼!乃罪多,参在上,乃能责命于天?殷之即丧,指乃功,不无戮于尔邦!”
祖伊反驳说:“唉!您的过失太多,又懒惰懈怠,高高在上,这些上天都了解,难道还能向上天祈求福命吗?国家就要灭亡了,这从你的所作所为就可以看出,怎么能够不被周国消灭呢?”

\chapter{微子}

殷既错天命,微子作诰父师、小师。
殷商废弃了天命,微子作诰与父师、少师商议。

微子若曰:“父师、少师!殷其弗或乱正四方。我祖厎遂陈于上,我用沈酗于酒,用乱败厥德于下。殷罔不小大好草窃奸宄。卿士师师非度。凡有辜罪,乃罔恒获,小民方兴,相为敌仇。今殷其沦丧,若涉大水,其无津涯。殷遂丧,越至于今!”
微子这样说:“父师、少师!殷商真的没有办法继续治理国家了吗?我们的先祖成汤制定了常法在先,而纣王却沉溺于酒色中,因淫乱而败坏成汤的美德在后。殷商的臣民无不抢夺偷盗、犯法作乱,官员们连法度都不遵守。对于犯罪的人不逮捕加以惩罚,百姓们难以忍受都纷纷起来反抗。现在殷商恐怕要灭亡了,就好像要渡过大河,却找不到渡口和河岸。殷商法度丧亡,竟到了这个地步!”

曰:“父师、少师,我其发出狂?吾家耄逊于荒?今尔无指告,予颠隮,若之何其?”
微子说:“父师、少师,我将被废弃而出亡在外呢?还是住在家中安然避居荒野呢?现在你们不指点我,殷商就会灭亡,你们是否告诉我应该怎么办啊?”

父师若曰:“王子!天毒降灾荒殷邦,方兴沈酗于酒,乃罔畏畏,咈其耇长旧有位人。
父师这样说:“王子!上天降下大灾要灭亡我们殷商,而君臣上下沉溺在酒色中,却不惧怕老天的威力,违背年高德劭的旧时大臣。

今殷民乃攘窃神祗之牺牷牲用以容,将食无灾。
现在,臣民竟然偷盗祭祀天地神灵的贡物,把它们藏起来,或是饲养,或是吃掉,都没有罪。

降监殷民,用乂仇敛,召敌仇不怠。罪合于一,多瘠罔诏。
再向下看看殷民,官军用杀戮和重刑横征暴敛,招致民怨也不放宽。这些罪行都在国君一人身上,痛苦不堪的国民无处申诉无处申诉。

商今其有灾,我兴受其败;商其沦丧,我罔为臣仆。诏王子出,迪我旧云刻子。王子弗出,我乃颠隮。自靖,人自献于先王,我不顾行遁。”
殷商现在或许会有灾祸呢,我们应该马上行动,铲除祸端;殷商或许会灭亡呢,我不做敌人的奴隶。我劝告王子出去,我早就说过,箕子和王子不出去,我们殷商就会灭亡。自己拿定主意吧!人人各自去对先王作出贡献,我不再顾虑了,将要出走。”

\part{周书}

\chapter{泰誓}

惟十有一年,武王伐殷。一月戊午,师渡孟津,作《泰誓》三篇。
周武王十一年,武王挥师讨伐商纣,一月戊午,渡过孟津,作《泰誓》三篇,训诫诸军。

周书·泰誓上
泰誓上

惟十有三年春,大会于孟津。
周武王十三年春天,诸侯大会于河南孟津。

王曰:“嗟!我友邦冢君越我御事庶士,明听誓。惟天地万物父母,惟人万物之灵。亶聪明,作元后,元后作民父母。今商王受,弗敬上天,降灾下民。沈湎冒色,敢行暴虐,罪人以族,官人以世,惟宫室、台榭、陂池、侈服,以残害于尔万姓。焚炙忠良,刳剔孕妇。皇天震怒,命我文考,肃将天威,大勋未集。肆予小子发,以尔友邦冢君,观政于商。惟受罔有悛心,乃夷居,弗事上帝神祗,遗厥先宗庙弗祀。牺牲粢盛,既于凶盗。乃曰:‘吾有民有命!’罔惩其侮。天佑下民,作之君,作之师,惟其克相上帝,宠绥四方。有罪无罪,予曷敢有越厥志?同力,度德;同德,度义。受有臣亿万,惟亿万心;予有臣三千,惟一心。商罪贯盈,天命诛之。予弗顺天,厥罪惟钧。予小子夙夜祗惧,受命文考,类于上帝,宜于冢土,以尔有众,厎天之罚。天矜于民,民之所欲,天必从之。尔尚弼予一人,永清四海,时哉弗可失!”
武王说:“啊!我的友邦大君和我的治事大臣、众士们,请清楚地听取我的誓言。天地是万物的父母,人是万物中的灵秀。真聪明的人就作大君,大君作人民的父母。现在商王纣不尊敬上天,降祸灾给下民。他嗜酒贪色,竟然施行暴虐,用灭族的严刑惩罚人,凭世袭的方法任用人。宫室呀,台榭呀,陂池呀,奢侈的衣服呀,他用这些东西来残害你们万姓人民。他烧杀忠良,解剖孕妇。皇天动了怒,命令我的文考文王严肃进行上天的惩罚,可惜大功没有完成。从前我小子姬发和你们友邦大君到商邦考察政治,看到了商纣没有悔改的心,他竟然傲慢不恭,不祭祀上天神祗,遗弃他的祖先宗庙而不祭祀。牺牲和粢盛等祭物,也被凶恶盗窃的人吃尽了。他却说:‘我有人民有天命!’不改变他侮慢的心意。上天帮助下民,为人民建立君主和师长,应当能够辅助上天,爱护和安定天下。对待有罪和无罪的人,我怎么敢违反上天的意志呢?力量相同就衡量德,德相同就衡量义。商纣有臣亿万,是亿万条心,我有臣子三千,只是一条心。商纣的罪恶,象穿物的串子已经穿满了,上天命令我讨伐他;我如果不顺从上天,我的罪恶就会跟商纣相等。我小子早夜敬慎忧惧。在文考庙接受了伐商的命令,我又祭告上天,祭祀大社,于是率领你们众位,进行上天的惩罚。上天怜闵人民,人民的愿望,上天一定会依从的。你们辅助我吧!要使四海之内永远清明。时机啊,不可失去呀!”

周书·泰誓中
泰誓中

惟戊午,王次于河朔,群后以师毕会。王乃徇师而誓曰:“呜呼!西土有众,咸听朕言。我闻吉人为善,惟日不足。凶人为不善,亦惟日不足。今商王受,力行无度,播弃犁老,昵比罪人。淫酗肆虐,臣下化之,朋家作仇,胁权相灭。无辜吁天,秽德彰闻。惟天惠民,惟辟奉天。有夏桀弗克若天,流毒下国。天乃佑命成汤,降黜夏命。惟受罪浮于桀。剥丧元良,贼虐谏辅。谓己有天命,谓敬不足行,谓祭无益,谓暴无伤。厥监惟不远,在彼夏王。天其以予乂民,朕梦协朕卜,袭于休祥,戎商必克。受有亿兆夷人,离心离德。予有乱臣十人,同心同德。虽有周亲,不如仁人。天视自我民视,天听自我民听。百姓有过,在予一人,今朕必往。我武维扬,侵于之疆,取彼凶残。我伐用张,于汤有光。勖哉夫子!罔或无畏,宁执非敌。百姓懔懔,若崩厥角。呜呼!乃一德一心,立定厥功,惟克永世。”
一月二十八日戊午,周武王驻兵在黄河之北,诸侯率领他们的军队都会合了。武王于是巡视军队并且告戒他们。武王说:“啊!西方各位诸侯,请都听我的话。我听说好人做好事,整天地做还是时间不够;坏人做坏事,也是整天地做还是时间不够。现在商王纣,力行不合法度的事,放弃年老的大臣,亲近有罪的人,过度嗜酒,放肆暴虐。臣下也受到他的影响,各结朋党,互为仇敌;挟持权柄,互相诛杀。无罪的人呼天告冤,秽恶的行为公开传闻。上天惠爱人民,君主遵奉上天。夏桀不能顺从天意,流毒于天下。上天于是佑助和命令成汤,使他降下废黜夏桀的命令。纣的罪恶超过了夏桀,他伤害善良的大臣,杀戮谏争的辅佐,说自己有天命,说敬天不值得实行,说祭祀没有益处,说暴虐没有害处。他的鉴戒并不远,就在夏桀身上。上天该使我治理人民,我的梦符合我的卜兆,吉庆重叠出现,讨伐商国一定会胜利。商纣有亿兆平民,都离心离德;我有拨乱的大臣十人,都同心同德。纣虽有至亲的臣子,比不上我周家的仁人。上天的看法,出自我们人民的看法,上天的听闻,出自我们人民的听闻。老百姓有所责难于我,我一定要依从民意前往讨伐。我们的武力要发扬,要攻到商国的疆土上,捉到那些残暴之徒;我们的付伐要进行,这个事业比成汤的还辉煌呀!努力吧!将士们。不可出现不威武的情况,宁愿你们保持没有对手的思想。百姓危惧不安,他们向我们叩头求助,额角响得就象山崩一样呀!啊!你们要一心一德建功立业,就能够长久安定人民。”

周书·泰誓下
泰誓下

时厥明,王乃大巡六师,明誓众士。
时在戊午的明天,周武王大规模巡视六军,明告众将士。

王曰:“呜呼!我西土君子。天有显道,厥类惟彰。今商王受,狎侮五常,荒怠弗敬。自绝于天,结怨于民。斫朝涉之胫,剖贤人之心,作威杀戮,毒痡四海。崇信奸回,放黜师保,屏弃典刑,囚奴正士,郊社不修,宗庙不享,作奇技淫巧以悦妇人。上帝弗顺,祝降时丧。尔其孜孜,奉予一人,恭行天罚。古人有言曰:‘抚我则后,虐我则仇。’独夫受洪惟作威,乃汝世仇。树德务滋,除恶务本,肆予小子诞以尔众士,殄歼乃仇。尔众士其尚迪果毅,以登乃辟。功多有厚赏,不迪有显戮。呜呼!惟我文考若日月之照临,光于四方,显于西土。惟我有周诞受多方。予克受,非予武,惟朕文考无罪;受克予,非朕文考有罪,惟予小子无良。”
王说:“啊!我们西方的将士。上天有明显的常理,它的法则应当显扬。现在商王纣轻慢五常,荒废怠情无所敬畏,自己弃绝于上天,结怨于人民。斫掉冬天清晨涉水者的脚胫,剖开贤人的心,作威作恶,杀戮无罪的人,毒害天下。崇信奸邪的人,逐退师保大臣,废除常刑,囚禁和奴役正士。祭天祭社的大礼不举行,宗庙也不享祀。造作奇技荒淫新巧的事物来取悦妇人。上天不依,断然降下这种丧亡的诛罚。你们要努力帮助我,奉行上天的惩罚!古人有言说:‘抚爱我的就是君主,虐待我的就是仇敌。’独夫商纣大行威虐,是你们的大仇。建立美德务求滋长,去掉邪恶务求除根,所以我小子率领你们众将上去歼灭你们的仇人。你们众将士要用果敢坚毅的精神来成就你们的君主!功劳多的将有重赏,不用命的将有明显的惩罚。啊!我文考文王的明德,象日月的照临一样,光辉普及四方,显著表现在西土,因此我们周国特别被众方诸侯所亲近。这次如果我战胜了纣,不是我勇武,是因为我的文考没有过失;如果纣战胜了我,不是我的文考有过失,是因为我这小子不好。”

\chapter{牧誓}

武王戎车三百两,虎贲三百人,与商战于牧野,作《牧誓》。
周武王率戎车三百辆,虎贲三百人,与商战于牧野,作《牧誓》。

时甲子昧爽,王朝至于商郊牧野,乃誓。
在甲子日的黎明时刻,周武王率领军队来到商国都城郊外的牧野,在这里举行誓师大会。

王左杖黄钺,右秉白旄以麾,曰:“逖矣,西土之人!”
武王左手拿着黄色的青铜大斧,右手拿着白色旄牛尾指挥,说:“将士们,你们千里迢迢来到这里辛苦了!”

王曰:“嗟!我友邦冢君御事,司徒、司马、司空,亚旅、师氏,千夫长、百夫长,及庸,蜀、羌、髳、微、卢、彭、濮人。称尔戈,比尔干,立尔矛,予其誓。”
武王说:“啊!我们友邦的国君以及所有官员和司徒、司马、司空,亚旅、师氏,千夫长、百夫长,以及庸、蜀、羌、髳、微、卢、彭、濮的人们,举起你们的戈,排列好你们的盾,竖起你们的矛,我要宣誓了。”

王曰:“古人有言曰:‘牝鸡无晨;牝鸡之晨,惟家之索。’今商王受惟妇言是用,昬弃厥肆祀弗答,昬弃厥遗王父母弟不迪,乃惟四方之多罪逋逃,是崇是长,是信是使,是以为大夫卿士。俾暴虐于百姓,以奸宄于商邑。今予发惟恭行天之罚。
武王说:“古人曾经说过说:‘母鸡不应该在早晨啼叫的;如果母鸡在早晨啼叫,那么这个家庭就会落败。’现在商王纣只是听信妇人的话,抛弃了对祖宗的祭祀而不闻不问,轻视并遗弃他的同祖的兄弟不用,反而只对四方重罪逃亡的人,推崇、尊敬、信任、使用,让他们做大夫、卿士一类的官。使他们残暴对待老百姓,在商国做违法乱纪的事情。现在,我姬发奉行上天的旨意来讨伐商纣。。

今日之事,不愆于六步、七步,乃止齐焉。勖哉夫子!不愆于四伐、五伐、六伐、七伐,乃止齐焉。勖哉夫子!尚桓桓如虎、如貔、如熊、如罴,于商郊弗迓克奔,以役西土,勖哉夫子!尔所弗勖,其于尔躬有戮!”
今天的战斗,行军时不超过六步、七步,就要停下来整顿一下。将士们,要努力啊!刺击时不超过四次、五次、六次、七次,就要停下来整齐一下。努力吧,将士们!希望你们威武雄壮,像虎、貔、熊、罴一样,在商都的郊外与敌人大战一场。不要迎击跑来投降的人,以便让他们帮助我们周国。努力吧,将士们!你们如果不努力,就会对你们自身有所惩罚!”

\chapter{武成}

武王伐殷。往伐归兽,识其政事,作《武成》。
武王征伐殷国,征伐归来把兽类放归山林,作《武成》。

惟一月壬辰,旁死魄。越翼日,癸巳,王朝步自周,于征伐商。
一月壬辰日,月亮大部分无光。到明天癸巳日,武王早晨从周京出发,前往征伐殷国。

厥四月,哉生明,王来自商,至于丰。乃偃武修文,归马于华山之阳,放牛于桃林之野,示天下弗服。
四月间,月亮开始放出光辉,武王从商国归来,到了丰邑。于是停止武备,施行文教,把战马放归华山的南面,把牛放回桃林的旷野, 向天下表示不用它们。

丁未,祀于周庙,邦甸、侯、卫,骏奔走,执豆、笾。越三日,庚戌,柴、望,大告武成。
四月丁未日,武王在周庙举行祭祀,建国于甸服、侯服、卫服的诸侯都忙于奔走,陈设木豆、竹笾等祭器。到第三天庚戌日,举行柴祭来祭天,举行望祭来祭山川,大力宣告伐商武功的成就。

既生魄,庶邦冢君暨百工,受命于周。
月亮已经生出光辉的时候,众国诸侯和百官都到周京来接受王命。

王若曰:“呜呼,群后!惟先王建邦启土,公刘克笃前烈,至于大王肇基王迹,王季其勤王家。我文考文王克成厥勋,诞膺天命,以抚方夏。大邦畏其力,小邦怀其德。惟九年,大统未集,予小子其承厥志。厎商之罪,告于皇天、后土、所过名山、大川,曰:‘惟有道曾孙周王发,将有大正于商。今商王受无道,暴殄天物,害虐烝民,为天下逋逃主,萃渊薮。予小子既获仁人,敢祗承上帝,以遏乱略。华夏蛮貊,罔不率俾。恭天成命,肆予东征,绥厥士女。惟其士女,篚厥玄黄,昭我周王。天休震动,用附我大邑周。惟尔有神,尚克相予以济兆民,无作神羞!既戊午,师逾孟津。癸亥,陈于商郊,俟天休命。甲子昧爽,受率其旅若林,会于牧野。罔有敌于我师,前途倒戈,攻于后以北,血流漂杵。一戎衣,天下大定。乃反商政,政由旧。释箕子囚,封比干墓,式商容闾。散鹿台之财,发钜桥之粟,大赉于四海,而万姓悦服。”
武王这样说:“啊!众位君侯。我的先王建立国家开辟疆土,公刘能修 前人的功业。到了太王,开始经营王事。王季勤劳王家。我文考文王能够成就其功勋,大受天命,安抚四方和中夏。大国畏惧他的威力,小国怀念他的恩德,诸侯归附九年而卒,大业没有完成。我小子将继承他的意愿。我把商纣的罪恶,曾经向皇天后土以及所经过的名山大川禀告说:‘有道的曾孙周王姬发,对商国将有大事。现在商王纣残暴无道,弃绝天下百物,虐待众民。他是天下逃亡罪人的主人和他们聚集的渊薮。我小子得到了仁人志士以后,冒昧地敬承上天的意旨,以制止乱谋。华夏各族和蛮貊的人民,无不遵从,我奉了上天的美命,所以我向东征讨,安定那里的士女。那里的士女,用竹 筐装着他们的黑色黄色的丝绸,求见我周王。他们被上天的休美震动了,因而归附了我大国周啊!你等神明庶几能够帮助我,来救助亿万老百姓,不要发生神明羞恶的事!’到了戊午日,军队渡过孟津。癸亥日,在商郊布好军阵,等待上天的美命。甲子日清早,商纣率领他如林的军队,来到牧野会战。他的军队对我军没有抵抗,前面的士卒反戈向后面攻击,因而大败,血流之多简直可以漂起木杵。一举讨伐殷商,而天下大安了。我于是反掉商王的恶政,政策由旧。解除箕子的囚禁,修治比干的坟墓,致敬于商容的里门。散发鹿台的财货,发放鉅桥的粟,向四海施行大赏,天下万民都心悦诚服。”

列爵惟五,分土惟三。建官惟贤,位事惟能。重民五教,惟食、丧、祭。惇信明义,崇德报功。垂拱而天下治。
武王设立爵位为五等,区分封地为三等。建立官长依据贤良,安置众吏依据才能。注重人民的五常之教和民食、丧葬、祭祀,重视诚信,讲明道 义;崇重有德的,报答有功的。于是武王垂衣拱手而天下安治了。

\chapter{洪范}

武王胜殷,杀受,立武庚,以箕子归。作《洪范》。
周武王战胜了殷纣王, 杀了纣王,立武庚,把纣王的庶兄箕子迎接回来。作《洪范》。

惟十有三祀,王访于箕子。王乃言曰:“呜呼!箕子。惟天阴骘下民,相协厥居,我不知其彝伦攸叙。”
周文王十三年,武王询问箕子。武王就说道:“啊!箕子,上天庇荫安定下民,使他们和谐共处,我不知道那治国的常理究竟有哪些?”

箕子乃言曰:“我闻在昔,鲧堙洪水,汩陈其五行。帝乃震怒,不畀‘洪范’九畴,彝伦攸斁。鲧则殛死,禹乃嗣兴,天乃锡禹‘洪范’九畴,彝伦攸叙。
箕子就回答说:“我听说从前,鲧采取堵塞方法治理洪水,结果将水、火、木、金、土的排列扰乱了。上天震怒,没有将治国的九种大法赐给鲧,治国的常理因此败坏了。后来,鲧被流放后死了,禹便继承了他父亲的事业。上天就把九种大法赐给了禹,治国的常理因此定了下来。

初一曰五行,次二曰敬用五事,次三曰农用八政,次四曰协用五纪,次五曰建用皇极,次六曰乂用三德,次七曰明用稽疑,次八曰念用庶征,次九曰向用五福,威用六极。
第一是五行。第二是认真做好五方面的事情。第三是努力施行八种政务。第四是根据日月运行的情况制定历法。第五是建立最高法则。第六是治理使用三种品德的人。第七是用卜筮来排除疑惑。第八是经常注意使用各种征兆。第九是凭五福鼓励臣民,凭六极警戒臣民。

一、五行:一曰水,二曰火,三曰木,四曰金,五曰土。水曰润下,火曰炎上,木曰曲直,金曰从革,土爰稼穑。润下作咸,炎上作苦,曲直作酸,从革作辛,稼穑作甘。
一、五行:一是水,二是火,三是木,四是金,五是土。水向下润湿,火向上燃烧,木可以弯曲伸直,金属可以顺从人的要求改变形状,土可以种植百谷。向下润湿的水产生咸味,向上燃烧的火产生苦味,可曲可直的木产生酸味,顺从人意而改变形状的金属产生辣味,种植的百谷产生甜味。

二、五事:一曰貌,二曰言,三曰视,四曰听,五曰思。貌曰恭,言曰从,视曰明,听曰聪,思曰睿。恭作肃,从作乂,明作哲,聪作谋,睿作圣。
二、五事:一是态度,二是言论,三是观察,四是听闻,五是思考。态度要恭敬,言论要正当,观察要明白,听闻要广远,思考要通达。容貌恭敬就能严肃,言论正当就能治理,观察明白就能昭晰,听闻广远就能善谋,思考通达就能圣明。

三、八政:一曰食,二曰货,三曰祀,四曰司空,五曰司徒,六曰司寇,七曰宾,八曰师。
三、八种政务:一是管理粮食,二是管理财货,三是管理祭祀,四是管理居民,五是管理教育,六是治理盗贼,七是管理朝觐,八是管理军事。

四、五纪:一曰岁,二曰月,三曰日,四曰星辰,五曰历数。
四、五种记时方法:一是年,二是月,三是日,四是星辰的出现情况,五是日月运行所经历的周天度数。

五、皇极:皇建其有极。敛时五福,用敷锡厥庶民。惟时厥庶民于汝极。锡汝保极:凡厥庶民,无有淫朋,人无有比德,惟皇作极。凡厥庶民,有猷有为有守,汝则念之。不协于极,不罹于咎,皇则受之。而康而色,曰:‘予攸好德。’汝则锡之福。时人斯其惟皇之极。无虐茕独而畏高明,人之有能有为,使羞其行,而邦其昌。凡厥正人,既富方谷,汝弗能使有好于而家,时人斯其辜。于其无好德,汝虽锡之福,其作汝用咎。无偏无陂,遵王之义;无有作好,遵王之道;无有作恶,遵王之路。无偏无党,王道荡荡;无党无偏,王道平平;无反无侧,王道正直。会其有极,归其有极。曰:皇,极之敷言,是彝是训,于帝其训,凡厥庶民,极之敷言,是训是行,以近天子之光。曰:天子作民父母,以为天下王。
五、君王的统治准则:君王应该建立至高无上的原则。掌握五福,用来普遍地赏赐给臣民。这样,臣民就会拥护最高的法则。向贡献您保持法则的方法:凡是臣下不允许结党营私,百官不能有私相比附的行为,必须将天子指定的法则视为最高准则。凡是臣下有计谋有作为有操守的,您要记着他们。行为不合法则,但没有犯罪的人,你就宽恕他们;假若他们和悦温顺地说:“我遵行美德。”您就赐给他们好处,于是,臣民就会想着最高法则。不虐待无依无靠的人,而又不畏显贵,臣下这样有才能有作为,就要让他献出他的才能,国家就会繁荣昌盛。凡那些百官之长,既然富有经常的俸禄,您不能使他们对国家有好处,于是臣民就要责怪您了。对于那些没有好德行的人,您即使赐给他们好处,将会使您受到危害。不要不平,不要不正,要遵守王令;不要作私好,要遵守王道;不要作威恶,要遵行正路。不要行偏,不要结党,王道坦荡;不要结党,不要行偏,王道平平;不要违反,不要倾侧,王道正直。团结那些守法之臣,归附那些执法之君。君王,对于皇极的广泛陈述,要宣扬教导,天帝就顺心了。凡是百官,对于皇极的敷言,要遵守实行,用来接近天子的光辉。天子作臣民的父母,因此才做天下的君王。

六、三德:一曰正直,二曰刚克,三曰柔克。平康,正直;强弗友,刚克;燮友,柔克。沈潜,刚克;高明,柔克。惟辟作福,惟辟作威,惟辟玉食。臣无有作福、作威、玉食。臣之有作福、作威、玉食,其害于而家,凶于而国。人用侧颇僻,民用僭忒。
六、三种品德:一是正直,二是过于刚强,三是过于柔弱。中正和平,就是正直;强不可亲就是刚克;和顺可亲就是柔克。应当抑制刚强不可亲近的人,推崇和顺可亲的人。只有君王才能作福,只有君王才能作威,只有君王才能享用美物。臣子不许有作福、作威、美食的情况。假若臣子有作福、作威、美食的情况,就会害及您的家,乱及您的国。百官将因此倾侧不正,百姓也将因此发生差错和疑惑。

七、稽疑:择建立卜筮人,乃命卜筮。曰雨,曰霁,曰蒙,曰驿,曰克,曰贞,曰悔,凡七。卜五,占用二,衍忒。立时人作卜筮,三人占,则从二人之言。汝则有大疑,谋及乃心,谋及卿士,谋及庶人,谋及卜筮。汝则从,龟从,筮从,卿士从,庶民从,是之谓大同。身其康强,子孙其逢,吉。汝则从,龟从,筮从,卿士逆,庶民逆吉。卿士从,龟从,筮从,汝则逆,庶民逆,吉。庶民从,龟从,筮从,汝则逆,卿士逆,吉。汝则从,龟从,筮逆,卿士逆,庶民逆,作内吉,作外凶。龟筮共违于人,用静吉,用作凶。
七、用卜筮解决疑惑的方法:选择建立掌管卜筮的官员,教导他们卜筮的方法。龟兆有的叫做雨,有的叫做霁,有的叫做蒙,有的叫做驿,有的叫做克;卦象有的叫做贞,有的叫做悔,共计有七种。龟兆用前五种,占筮用后两种,根据这些推演变化,决定吉凶。设立这种官员进行卜筮。三个人占卜,就听从两个人的说法。你若有重大的疑难,你自己要考虑,再与卿士商量,再与庶民商量,再与卜筮官员商量。你自己同意,龟卜同意,筮占同意,卿士同意,庶民同意,这就叫大同。样,自己的身体会健康强壮,子孙后代也会兴旺大吉。你赞同,龟卜赞同,蓍筮赞同,而卿士反对,庶民反对,也吉利。卿士赞同,龟卜赞同,蓍筮赞同,你反对,庶民反对,也吉利。庶民赞同,龟卜赞同,蓍筮赞同,你反对,卿士反对,也吉利。你赞同,龟卜赞同,蓍筮反对,卿士反对,庶民反对,在国内行事就吉利,在国外行事就不吉利。龟卜、蓍筮都与人意相违,不做事就吉利,做事就凶险。

八、庶征:曰雨,曰暘,曰燠,曰寒,曰风。曰时五者来备,各以其叙,庶草蕃庑。一极备,凶;一极无,凶。曰休征;曰肃、时雨若;曰乂,时暘若;曰晰,时燠若;曰谋,时寒若;曰圣,时风若。曰咎征:曰狂,恒雨若;曰僭,恒暘若;曰豫,恒燠若;曰急,恒寒若;曰蒙,恒风若。曰王省惟岁,卿士惟月,师尹惟日。岁月日时无易,百谷用成,乂用明,俊民用章,家用平康。日月岁时既易,百谷用不成,乂用昏不明,俊民用微,家用不宁。庶民惟星,星有好风,星有好雨。日月之行,则有冬有夏。月之从星,则以风雨。
八、一些征兆:下雨,天晴,温暖,寒冷,刮风。一年中这五种天气齐备,各自按照时序发生,百草与庄稼就会长得很茂盛。一种天气过多就不好;一种天气过少,也不好。君王行为美好的征兆:一种是严肃恭敬,就像及时降雨的喜人;天下治理得好就好比天气及时晴朗;一种叫做明智,就好比气候及时温暖;一种叫善谋,就像寒冷即将来到;一种叫圣明,就像风及时吹来。君王行为坏的征兆:一叫狂妄,就象久雨的愁人;一叫不信,就象久晴的愁人;一叫逸豫,就象久暖的愁人;一叫严急,就象久寒的愁人;一叫昏昧,就象久风的愁人。君王之所视察,就象一年包括四时;卿士就象月,统属于岁;众尹就象日,统属于月。假若岁、月、日、时的关系没有改变,百谷就因此成熟,政治就因此清明,杰出的人才因此显扬,国家因此太平安宁。假若日、月、岁、时的关系全都改变,百谷就因此不能成熟,政治就因此昏暗不明,杰出的人才因此不能重用,国家因此不得安宁。百姓好比星星,有的星喜欢风,有的星喜欢雨。太阳和月亮的运行,就有冬天和夏天。月亮顺从星星,就要用风和雨润泽他们。

九、五福:一曰寿,二曰富,三曰康宁,四曰攸好德,五曰考终命。六极:一曰凶、短、折,二曰疾,三曰忧,四曰贫,五曰恶,六曰弱。
九、五种幸福:一是长寿,二是富贵,三是健康平安,四是遵行美德,五是长寿善终。六种不幸的事:一是短命早死,二是疾病,三是忧愁,四是贫穷,五是邪恶,六是懦弱。”

\chapter{旅獒}

西旅献獒,太保作《旅獒》。
西方旅国向武王进献猛犬,太保召公作《旅獒》。

惟克商,遂通道于九夷八蛮。西旅厎贡厥獒,太保乃作《旅獒》,用训于王。曰:“呜呼!明王慎德,西夷咸宾。无有远迩,毕献方物,惟服食器用。王乃昭德之致于异姓之邦,无替厥服;分宝玉于伯叔之国,时庸展亲。人不易物,惟德其物!德盛不狎侮。狎侮君子,罔以尽人心;狎侮小人,罔以尽其力。不役耳目,百度惟贞。玩人丧德,玩物丧志。志以道宁,言以道接。不作无益害有益,功乃成;不贵异物贱用物,民乃足。犬马非其土性不畜,珍禽奇兽不育于国,不宝远物,则远人格;所宝惟贤,则迩人安。呜呼!夙夜罔或不勤,不矜细行,终累大德。为山九仞,功亏一篑。允迪兹,生民保厥居,惟乃世王。”
武王打败商纣以后,便向周围国家开通道路。西方旅国来贡献那里的大犬,太保召公于是写了《旅獒》,用来劝谏武王。召公说:“啊!圣明的王敬重德行,所以四周的民族都来归顺。不论远近,都贡献些各方的物产,但只是些可供衣食器用的东西。明王于是昭示这些贡品给异姓的国家,使他们不要荒废职事;分赐宝玉给同姓的国家,用这些东西展示亲爱之情。人们并不轻视那些物品,只以德意看待那些物品。德盛的人不轻易侮慢。轻易侮慢官员,就不可以使人尽心;轻易侮慢百姓,就不可以使人尽力。不被歌舞女色所役使,百事的处理就会适当。戏弄人就丧德,戏弄物就丧志。自己的意志,要依靠道来安定;别人的言论,要依靠道来接受。不做无益的事来妨害有益的事,事就能成;不重视珍奇物品,百姓的用物就能充足。犬马不是土生土长的不养,珍禽奇兽不收养于国。不宝爱远方的物品,远人就会来;所重的是贤才,近人就安了。啊!早晚不可有不勤的时候。不注重细行,终究会损害大德,比如筑九仞高的土山,工作未完只在于一筐土。真能做到这些,则人民就安其居,而周家就可以世代为王了。”

\chapter{金縢}

武王有疾,周公作《金縢》。
武王身体有恙,周公作《金縢》,请代武王死。

既克商二年,王有疾,弗豫。二公曰:“我其为王穆卜。”周公曰:“未可以戚我先王?”公乃自以为功,为三坛同墠。为坛于南方,北面,周公立焉。植璧秉珪,乃告太王、王季、文王。
周灭商后的第二年,武王重病,身身体不适。太公、召公说:“我们为王恭敬地卜问吉凶吧!”周公说:“不可以向我们先王祷告吗?”周公就把自身作为抵押,清除一块土地,在祭场上筑起三座祭坛。又在三坛的南方筑起一座台子。周公面向北方站在台上,手持玉珪,然后向太王、王季、文王祷告。

史乃册,祝曰:“惟尔元孙某,遘厉虐疾。若尔三王是有丕子之责于天,以旦代某之身。予仁若考能,多材多艺,能事鬼神。乃元孙不若旦多材多艺,不能事鬼神。乃命于帝庭,敷佑四方,用能定尔子孙于下地。四方之民罔不祗畏。呜呼!无坠天之降宝命,我先王亦永有依归。今我即命于元龟,尔之许我,我其以璧与珪归俟尔命;尔不许我,我乃屏璧与珪。”
史官将周公祷告的词写在了典册上,祝词说:“你们的长孙姬发,得了很严重的病。假若你们三位先王这时在天上有助祭的职责,就让我姬旦代替他去吧!我柔顺又乖巧伶俐,多才多艺,善于奉事鬼神。你们的长孙不如我多材多艺,也不能奉事鬼神。而且他在天帝那里接受了任命,正在统治四方,因此能够在人间安定你们的子孙,天下的老百姓也无不敬畏他。唉!不要丧失上天降给的宝贵使命,我们的先王也就永远有所归依。现在我就要通过龟卜来接受你们的命令了,如果你们答应,我就拿着璧和珪死去,等待你们的命令;你们不答应,我就收起璧和珪,不敢再请了。”

乃卜三龟,一习吉。启籥见书,乃并是吉。公曰:“体!王其罔害。予小子新命于三王,惟永终是图;兹攸俟,能念予一人。”
于是卜问三龟,占卜结果都重复出现吉兆。周公说:“根据兆形,王的病不会有什么危险。我新向三位先王祷告,只图国运长远;现在期待的,是先王能够俯念我的诚心。”

公归,乃纳册于金滕之匮中。王翼日乃瘳。
周公回去,把册书放进金属束着的匣子中。第二天,周武王的病就好了。

武王既丧,管叔及其群弟乃流言于国,曰:“公将不利于孺子。”周公乃告二公曰:“我之弗辟,我无以告我先王。”周公居东二年,则罪人斯得。于后,公乃为诗以贻王,名之曰《鸱鴞》。王亦未敢诮公。
武王死后,管叔和他的几个弟弟就在国内散布谣言。说:“周公将会对成王不利。”周公就告诉太公、石公说:“我不摄政,我将无辞告我先王。”周公留在东方两年,罪人就捕获了。后来,周公写了一首诗送给成王,叫它为《鸱鸮》。结果,成王只是不敢责备周公。

秋,大熟,未获,天大雷电以风,禾尽偃,大木斯拔,邦人大恐。王与大夫尽弁以启金滕之书,乃得周公所自以为功代武王之说。二公及王乃问诸史与百执事。对曰:“信。噫!公命我勿敢言。”
秋天,百谷成熟,还没有收获,天空出现雷电与大风。庄稼都倒伏了,大树都被拔起,国人非常恐慌。周成王和大夫们都戴上礼帽,打开金属束着的匣子,于是得到了周公以自身为质、请代武王的祝辞。太公、召公和成王就询问众史官以及许多办事官员。他们回答说:“确实的。唉!周公告诫我们不能说出来。”

王执书以泣,曰:“其勿穆卜!昔公勤劳王家,惟予冲人弗及知。今天动威以彰周公之德,惟朕小子其新逆,我国家礼亦宜之。”王出郊,天乃雨,反风,禾则尽起。二公命邦人凡大木所偃,尽起而筑之。岁则大熟。
成王拿着册书哭泣,说:“不要敬卜了!过去,周公勤劳王室,我这年轻人来不及了解。现在上天动怒来表彰周公的功德,我小子要亲自去迎接,我们国家的礼制也应该这样。”成王走出郊外,天就下着雨,风向也反转了,倒伏的庄稼又全部伸起来。太公、召公命令国人。凡大树所压的庄稼,要全部扶起来,又培好根,这一年的年成却是个大丰收。

\chapter{大诰}

武王崩,三监及淮夷叛,周公相成王,将黜殷,作《大诰》。
周武王去世后,三个监国管叔、蔡叔和商纣王的儿子武庚联合淮南少数民族叛乱,这时周公是成王的相国,他们想废除殷商,作了《大诰》。

王若曰:“猷大诰尔多邦越尔御事,弗吊天降割于我家,不少延。洪惟我幼冲人,嗣无疆大历服。弗造哲,迪民康,矧曰其有能格知天命!
王这样说:“啊!此时我要遍告你们各诸侯国君主和你们的办事大臣。不幸啊!上天给我们国家降下灾祸,灾祸依然在继续发展。我现在代替我年幼的侄儿继承远大悠久的王业。可是,我没有遇到明智的人,能令我们的臣民安居乐业,更何况说是了解天命之人?

已!予惟小子,若涉渊水,予惟往求朕攸济。敷贲敷前人受命,兹不忘大功。予不敢于闭。
唉!我的处境就像是渡过深渊,我应当前往寻求我能渡过去的办法。摆下占卜用的大龟吧,让它来宣布先人们是如何在上天那里接受大命的,这样的大恩,是永世不敢忘记的!

天降威,用宁王遗我大宝龟,绍天明。即命曰:‘有大艰于西土,西土人亦不静,越兹蠢。殷小腆诞敢纪其叙。天降威,知我国有疵,民不康,曰:予复!反鄙我周邦,今蠢今翼。日,民献有十夫予翼,以于敉宁、武图功。我有大事,休?’朕卜并吉。”
在上天降下灾难的时刻我不敢将其隐藏,用文王留给我们的大宝龟卜问天命。我向大龟祷告说:‘在西方有大灾难,西方人会陷入水深火热中,有的人就越发蠢蠢欲动了。殷商的余孽竟敢组织他的残余力量,想要恢复统治。上天降下灾祸,他们知道我们国家有困难,人们不安宁。他们说:我们要复国!反而图谋我们周国,现在他们发动叛乱了。这些天有十位贤者来帮助我,我要和他们前往完成文王、武王所谋求的功业。我们将有战事,会吉利吗?’我的卜兆全都吉利。

肆予告我友邦君越尹氏、庶士、御事、曰:‘予得吉卜,予惟以尔庶邦于伐殷逋播臣。’尔庶邦君越庶士、御事罔不反曰:‘艰大,民不静,亦惟在王宫邦君室。越予小子考,翼不可征,王害不违卜?’
“所以我告诉我的友邦国君和各位大臣说:‘我现在得到了吉卜,打算联合众诸侯国去讨伐殷商那些叛乱的罪人。’你们各位国君和各位大臣没有不反对说:‘困难很大,老百姓不安宁,也有在王室和邦君室的人。我们这些小子考虑,或许不可征讨吧,大王为什么不违背龟卜呢?’

肆予冲人永思艰,曰:於戲!允蠢,鳏寡哀哉!予造天役,遗大投艰于朕身,越予冲人,不卬自恤。义尔邦君越尔多士、尹氏、御事绥予曰:‘无毖于恤,不可不成乃宁考图功!’
“现在我深深地考虑着艰难,我说:‘唉!确实惊扰了苦难的人民,真痛心啊!我受天命的役使,上天把艰难的事重托给我,我不暇只为自身忧虑。你们众位邦君与各位大臣应该安慰我说:‘不要被忧患吓倒,不可不完成您文王所谋求的功业!’

已!予惟小子,不敢替上帝命。天休于宁王,兴我小邦周,宁王惟卜用,克绥受兹命。今天其相民,矧亦惟卜用。呜呼!天明畏,弼我丕丕基!”
“唉!我小子不敢废弃天命。上天嘉惠文王,振兴我们小小的周国,当年文王只使用龟卜,能够承受这天命。现在上天帮助老百姓,何况也是使用龟卜呢?啊!天命可畏,你们辅助我们伟大的事业吧!”

王曰:“尔惟旧人,尔丕克远省,尔知宁王若勤哉!天閟毖我成功所,予不敢不极卒宁王图事。肆予大化诱我友邦君,天棐忱辞,其考我民,予曷其不于前宁人图功攸终?天亦惟用勤毖我民,若有疾,予曷敢不于前宁人攸受休毕!”
王说:“你们是老臣,你们多能远知往事,你们知道文王是如何勤劳的啊!上天慎重地告诉我们成功的办法,我不敢不快速完成文王图谋的事业。现在我劝导我们友邦的君主:上天用诚信的话帮助我们,要成全我们的百姓,我们为什么不对前文王图谋的功业谋求完成呢?上天也想施加勤苦给我们老百姓,好像有疾病,我们怎敢不对前文王所受的疾病好好攘除呢?”

王曰:“若昔朕其逝,朕言艰日思。若考作室,既底法,厥子乃弗肯堂,矧肯构?厥父菑,厥子乃弗肯播,矧肯获?厥考翼其肯曰:予有后弗弃基?肆予曷敢不越卬敉宁王大命?若兄考,乃有友伐厥子,民养其劝弗救?”
王说:“像往日讨伐纣王一样,我将要前往,我想说些艰难日子里的想法。好像父亲建屋,已经确定了办法,他的儿子却不愿意打地基,况且愿意盖屋吗?他的父亲新开垦了田地,他的儿子却不愿意播种,况且愿意收获吗?这样,他的父亲或许会愿意说,我有后人不会废弃我的基业吗?所以我怎敢不在我自己身上完成文王伟大的使命呢?又好比兄长死了,却有人群起攻击他的儿子,为民长上的难道能够相劝不救吗?”

王曰:“呜呼!肆哉尔庶邦君越尔御事。爽邦由哲,亦惟十人迪知上帝命越天棐忱,尔时罔敢易法,矧今天降戾于周邦?惟大艰人诞邻胥伐于厥室,尔亦不知天命不易?
王说:“啊!努力吧,你们诸位邦君和各位官员。使国家清明要用明智的人,现在也有十个人引导我们知道天命和上天辅助诚信的道理,你们不能轻视这些!何况现在上天已经给周国降下了定命呢?那些发动叛乱的大罪人,勾结邻国,同室操戈。你们也不知天命不可改变吗?

予永念曰:天惟丧殷,若穑夫,予曷敢不终朕亩?天亦惟休于前宁人,予曷其极卜?敢弗于从率宁人有指疆土?矧今卜并吉?肆朕诞以尔东征。天命不僭,卜陈惟若兹。”
“我长时间考虑着:上天要灭亡殷国,好像农夫一样,我怎敢不完成我的田亩工作呢?上天也想嘉惠我们先辈文王,我们怎能放弃吉卜呢?怎敢不前去重新巡视文王美好的疆土呢?更何况今天的占卜都是吉兆呢?所以我要大规模地率领你们东征,天命不可不信,卜兆的指示应当遵从呀!”

\chapter{微子之命}

成王既黜殷命,杀武庚,命微子启代殷后,作《微子之命》。
周成王已废黜殷商的国运,诛杀武庚,诰命天下由微子启代替武庚为殷侯,史叙其事,作《微子之命》。

王若曰:“猷!殷王元子。惟稽古,崇德象贤。统承先王,修其礼物,作宾于王家,与国咸休,永世无穷。呜呼!乃祖成汤克齐圣广渊,皇天眷佑,诞受厥命。抚民以宽,除其邪虐,功加于时,德垂后裔。尔惟践修厥猷,旧有令闻,恪慎克孝,肃恭神人。予嘉乃德,曰笃不忘。上帝时歆,下民祗协,庸建尔于上公,尹兹东夏。钦哉,往敷乃训,慎乃服命,率由典常,以蕃王室。弘乃烈祖,律乃有民,永绥厥位,毗予一人。世世享德,万邦作式,俾我有周无斁。呜呼!往哉惟休,无替朕命。”
成王这样说:“啊!殷王的长子微子。你一定要稽考古代,有尊崇盛德、效法先贤的制度,就是说,效法贤王,整治典礼文物,作王家的贵宾,和友邦一起共享政德之美,世代绵长,无穷无尽。啊!你的祖先成汤,能够肃敬、圣明、广大、深远,被皇天顾念佑助,承受了天命。他用宽和的办法安治臣民,除掉邪恶暴虐之徒。功绩施展于当时,德泽流传于后裔。你履行成汤的治道,老早有美名。谨慎能孝,恭敬神和人。我赞美你的美德,以为纯厚而不可忘。上天依时享受你的祭祀,下民对你敬爱和睦,因此立你为上公,治理这块东夏地区。要敬重呀!前去发布你的政令。谨慎对待你的上公职位与使命,遵循常法以保卫周王室。宏扬你烈祖的治道,规范你的人民,长久安居上公之位,辅助我一人。这样,你的世世子孙会享受你的功德,万邦诸侯会以你为榜样,服从我周王室而不厌倦。“啊!前去吧,要好好地干!不要废弃我的诰命。”

\chapter{康诰}

成王既伐管叔、蔡叔,以殷余民封康叔,作《康诰》、《酒诰》、《梓材》。
成王平定管叔、蔡叔之后,把原商国的地盘和人民封给康叔管理,拟了三篇发言稿:《康诰》、《酒诰》、《梓材》。

惟三月哉生魄,周公初基作新大邑于东国洛,四方民大和会。侯、甸、男邦、采、卫百工、播民,和见士于周。周公咸勤,乃洪大诰治。
那是三月十六日,周公在东国洛汭举行新建都城的奠基仪式,四方人民都有幸参加了这个盛会,由近致远各大小斑竹、工人、农民,都荣幸地表达了愿意服从周国的管理,周公一一答谢,接着大声公布了成王的命令:

王若曰:“孟侯,朕其弟,小子封。惟乃丕显考文王,克明德慎罚;不敢侮鳏寡,庸庸,祗祗,威威,显民,用肇造我区夏,越我一、二邦以修我西土。惟时怙冒,闻于上帝,帝休,天乃大命文王。殪戎殷,诞受厥命越厥邦民,惟时叙,乃寡兄勖。肆汝小子封在兹东土。
王如是说:“孟侯这个最高的爵位,我要封给你了,年轻的封,希望你光大你逝去的父亲文王(的功德),能彰显仁德、慎用刑罚;不能欺侮孤老、寡母,在人民面前平易、恭敬、谦虚,以创造我小小华夏,让我们的大邦、小国都井井有条。我们西岐的繁盛,仰赖先帝的名望,先帝去后,上天把伟大的使命交给了文王;灭亡强大的殷国、光荣地继承这个使命、让国家和人民有条不紊,是你的王兄(武王)尽的力。要效法(先王),年轻的封,在这东边的国土上。”

王曰:“呜呼!封,汝念哉!今民将在祗遹乃文考,绍闻衣德言。往敷求于殷先哲王用保乂民,汝丕远惟商者成人宅心知训。别求闻由古先哲王用康保民。宏于天,若德,裕乃身不废在王命!”
王说:“啊,封,你一定要记着,如今民众信奉的是一心追随你的父亲——文王,继续聆听、依赖他伟大的教导。来到了殷的故土,就要广泛地寻求殷商过去圣明国王在治理臣民方面的方式方法。你要永远将德高望重的老人放在心上,宽容管理训教普通百姓。另外还要多多了解古代的先知圣王的治国之道,让人民安居乐业。让那些遗训像天空那么博大宽容,以德服人,你呢,就算没有辜负我对你的期望。”

王曰:“呜呼!小子封,恫瘝乃身,敬哉!天畏棐忱;民情大可见,小人难保。往尽乃心,无康好逸,乃其乂民。我闻曰:‘怨不在大,亦不在小;惠不惠,懋不懋。’
王说:“啊,年轻的封,民众的疾苦,你要重视哟!上天是很可怕的,它是否帮助你常常通过民众表达出来。去到那里一定要尽力,不要贪图安逸的生活,这样才能治理好你的臣民。我听说:“臣民的恨,不存在大,也不存在小;爱不爱,就存在大不大”(恨不论大小,都是恨;爱就图个大,小了就不叫爱)。

已!汝惟小子,乃服惟弘王应保殷民,亦惟助王宅天命,作新民。”
好了,你是年轻人,应该弘扬我(的期望),让百姓安居乐业,也算是帮我顺应天命、让百姓振作。”

王曰:“呜呼!封,敬明乃罚。人有小罪,非眚,乃惟终自作不典;式尔,有厥罪小,乃不可不杀。乃有大罪,非终,乃惟眚灾:适尔,既道极厥辜,时乃不可杀。”
王说,“啊,封,先仁后罚。一个人犯了错,不是大问题,就改掉它就好;继续犯小错,就不得不处罚。有大错误,不改,那就是大灾难来临了),甚至还说是很小很小的错误,这就要加大力度处罚。”

王曰:“呜呼!封,有叙时,乃大明服,惟民其敕懋和。若有疾,惟民其毕弃咎。若保赤子,惟民其康乂。非汝封刑人杀人,无或刑人杀人。非汝封又曰劓刵人,无或劓刵人。”
王说:“啊,封,有句话说得好啊,就是以德为首:希望人民正直、昌盛、和睦;像厌恶疾病那样,希望人们全部厌恶罪过;像保护宝宝那样,希望人民安定美满。不是你封叔要惩人罚人,是国法难容(是国家惩人罚人);不是你要批评人,是国家批评人。”

王曰:“外事,汝陈时臬司师,兹殷罚有伦。”又曰:“要囚,服念五、六日至于旬时,丕蔽要囚。”
王说:“首先要做的,你颁布完善的法律使众人信服,这样所有的处罚就公正了。”又说:“复审犯人,斟酌五六天,到了巡视的时候,好好断理、复审犯人。”

王曰:“汝陈时臬事罚。蔽殷彝,用其义刑义杀,勿庸以次汝封。乃汝尽逊曰时叙,惟曰未有逊事。
王说:“你颁布完善的法律进行处罚,所有的断案都用(与其罪过)相应的刑罚,不能草率。你封,你要竭力谦逊,要彬彬有礼,(又)希望没有逃避的现象。

已!汝惟小子,未其有若汝封之心。朕心朕德,惟乃知。
好了,你是年轻人,没得有如你封的意的(如果有让你不满意的话),我的心我的真(我的真心),你懂的。

凡民自得罪:寇攘奸宄,杀越人于货,暋不畏死,罔弗憝。
凡是人因有罪,如:抢劫偷盗,杀人为财,不怕死的;无不恨。”

王曰:“封,元恶大憝,矧惟不孝不友。子弗祗服厥父事,大伤厥考心;于父不能字厥子,乃疾厥子。于弟弗念天显,乃弗克恭厥兄;兄亦不念鞠子哀,大不友于弟。惟吊兹,不于我政人得罪,天惟与我民彝大泯乱,曰:乃其速由文王作罚,刑兹无赦。
王说:“封,最坏的人大家都恨,何况是不孝无情。儿子不敬听老子的话,大伤老子的心;作为父亲不愿养儿子,因此恨儿子;作为弟弟不知天高(地厚),不能恭敬兄长;哥哥也不讲慈爱,对弟弟很不友好。就悲哀了这,不是我们管人的没管得好(有罪的话),就是上天要我们人伦大混乱。那就快快引用文王创制的法律,上刑罚、不要赦免。

不率大戛,矧惟外庶子、训人惟厥正人越小臣、诸节。乃别播敷,造民大誉,弗念弗庸,瘝厥君,时乃引恶,惟朕憝。已!汝乃其速由兹义率杀。亦惟君惟长,不能厥家人越厥小臣、外正;惟威惟虐,大放王命;乃非德用乂。汝亦罔不克敬典,乃由裕民,惟文王之敬忌;乃裕民曰:‘我惟有及。’则予一人以怿。”
不遵守国家大法的,也有诸侯国的庶子、训人和正人、小臣、诸节等官员。竟然另外发布政令,告谕百姓,大大称誉不考虑不执行国家法令的人,危害国君;这就助长了恶人,我怨恨他们。唉!你就要迅速根据这些条例捕杀他们。也有这种情况,诸侯不能教育好他们的家人和内外官员,作威肆虐,完全放弃王命;这些人就不可用德去治理。你也不要不能崇重法令。前往教导老百姓,要思念文王的赏善罚恶;前往教导老百姓说:‘我们只求继承文王。’那么,我就高兴了。”

王曰:“封,爽惟民迪吉康,我时其惟殷先哲王德,用康乂民作求。矧今民罔迪,不适;不迪,则罔政在厥邦。”
王说:“封,多么希望人民走向幸福安康,我们时常要惦记弘扬先王的圣德,把让人民安居乐业作为追求。何况如今人民还没有步入正轨,不去引导,就跟没管理这个国家一样。”

王曰:“封,予惟不可不监,告汝德之说于罚之行。今惟民不静,未戾厥心,迪屡未同,爽惟天其罚殛我,我其不怨。惟厥罪无在大,亦无在多,矧曰其尚显闻于天。”
王说:“封,我不得不借句话来告诉你:‘爱失去了比惩罚还要痛苦。’如今人民不安宁,民心还没稳定,国家仍然没有统一,多么希望上天惩罚我,我也没有怨言,因为罪不在大小、也不在多少(都是罪过),何况我的罪过又大又明显、天(肯定)晓得了。”

王曰:“呜呼!封,敬哉!无作怨,勿用非谋非彝蔽时忱。丕则敏德,用康乃心,顾乃德,远乃猷,裕乃以;民宁,不汝瑕殄。”
王说:“啊,封,要注意呀,不要有怨言,不要用、不要想、不要经常掩藏(自己的)善良真诚,大显仁德,用来安你的心、呵护你的德、增长你的智慧、光大你的作为,人民安宁,你的瑕疵就消失了。”

王曰:“呜呼!肆汝小子封。惟命不于常,汝念哉!无我殄享,明乃服命,高乃听,用康乂民。”
王说:“啊,要效法(先王),年轻的封,生命不是永恒的,你要记住啊!我们要努力奉献,仁就是使命(使用生命),德就是(生命的)快乐,让人民安居乐业。”

王若曰:“往哉!封,勿替敬,典听朕告,汝乃以殷民世享。”
王如是说:“去吧,封,不要放弃慎用刑罚,听从我的劝告,你就会被万民永远爱戴。”

\chapter{酒诰}

王若曰:“明大命于妹邦。乃穆考文王肇国在西土。厥诰毖庶邦庶士越少正御事,朝夕曰:‘祀兹酒。惟天降命,肇我民,惟元祀。天降威,我民用大乱丧德,亦罔非酒惟行;越小大邦用丧,亦罔非酒惟辜。’
王这样说:“要在殷朝旧都宣布一项重大教命。当初,穆考文王在西方创立了这个国家。他早晚告戒各国诸侯、各位卿士和各级官员说:‘只有在祭祀时,才饮酒。’上天降下教令,劝勉我们臣民,只有在祭祀的时候才能饮酒。上天降下惩罚,我们臣民大乱失德,也无非是酒造成的罪行;有些诸侯国灭亡了,那也是众民饮酒过度带来的灾祸。

文王诰教小子有正有事:无彝酒。越庶国:饮惟祀,德将无醉。惟曰我民迪小子惟土物爱,厥心臧。聪听祖考之遗训,越小大德。
文王还告诫官员和子孙,不要经常喝酒。告诫各诸侯国君,只有在祭祀时才可以饮酒,并要用德扶持,不要喝醉了。文王还告诫臣民要教导子孙珍惜粮食,使我们的思想善良。我们一定要牢记前辈的常训,发扬他们的美德!

小子惟一妹土,嗣尔股肱,纯其艺黍稷,奔走事厥考厥长。肇牵车牛,远服贾用,孝养厥父母,厥父母庆,自洗腆,致用酒。
殷朝旧都的殷民们,你们一定要用你们的手足力量,专心种植黍稷,勤勉地奉事你们的父兄。农事完毕以后,可以牵牛赶车,到外地去从事贸易,孝顺赡养父母;父母一定会高高兴兴地为你们置办美好丰盛的膳食,这时就可以饮酒。

庶士有正越庶伯君子,其尔典听朕教!尔大克羞耇惟君,尔乃饮食醉饱。丕惟曰尔克永观省,作稽中德,尔尚克羞馈祀。尔乃自介用逸,兹乃允惟王正事之臣。兹亦惟天若元德,永不忘在王家。”
各级官员们,希望你们能经常听从我的教导!只要你们能好好的侍奉长辈和君主,你们就能喝醉吃饱。我想说的是:你们要长久地观察自己,使自己的言行符合中正的美德。你们能够参加国君举行的祭祀。你们如果自己限制行乐饮酒,这样就能长期成为王家的治事官员。你们是上天嘉奖的有德之人,将永远不会被王家忘记。”

王曰:“封,我西土棐徂,邦君御事小子尚克用文王教,不腆于酒,故我至于今,克受殷之命。”
王说:“封啊,我们西土辅导帮助诸侯和官员,常常能够遵从文王的教导,不多饮酒,所以我们到今天,能够接受重大的使命。”

王曰:“封,我闻惟曰:‘在昔殷先哲王迪畏天显小民,经德秉哲。自成汤咸至于帝乙,成王畏相惟御事,厥棐有恭,不敢自暇自逸,矧曰其敢崇饮?越在外服,侯甸男卫邦伯,越在内服,百僚庶尹惟亚惟服宗工越百姓里居,罔敢湎于酒。不惟不敢,亦不暇,惟助成王德显越,尹人祗辟。’
王说:“封啊,我听到有人说:‘过去,殷的先人明王畏惧天命和百姓,施行德政,保持恭敬。从成汤延续到帝乙,明君贤相都考虑着治理国事,他们颁布政令很认真,不敢自己安闲逸乐,何况敢聚众饮酒呢?在外地的侯、甸、男、卫的诸侯,在朝中的各级官员、宗室贵族以及退住在家的官员,没有人敢酣乐在酒中。不但不敢,他们也没有闲暇,他们只想助成王德使它显扬,助成长官重视法令。’

我闻亦惟曰:‘在今后嗣王,酣,身厥命,罔显于民祗,保越怨不易。诞惟厥纵,淫泆于非彝,用燕丧威仪,民罔不衋伤心。惟荒腆于酒,不惟自息乃逸,厥心疾很,不克畏死。辜在商邑,越殷国灭,无罹。弗惟德馨香祀,登闻于天;诞惟民怨,庶群自酒,腥闻在上。故天降丧于殷,罔爱于殷,惟逸。天非虐,惟民自速辜。’”
我听到也有人说:‘在近世的商纣王,好酒,以为有命在天,不明白臣民的痛苦,安于怨恨而不改。他大作淫乱,游乐在违反常法的活动之中,因宴乐而丧失了威仪,臣民没有不悲痛伤心的。商纣王只想放纵于酒,不想自己制止其淫乐。他心地狠恶,不能以死来畏惧他。他作恶在商都,对于殷国的灭亡,没有忧虑过。没有明德芳香的祭祀升闻于上天;只有老百姓的怨气、只有群臣私自饮酒的腥气升闻于上。所以,上天对殷邦降下了灾祸,不喜欢殷国,就是淫乐的缘故。上天并不暴虐,是殷民自己招来了罪罚。”

王:“封,予不惟若兹多诰。古人有言曰:‘人无于水监,当于民监。’今惟殷坠厥命,我其可不大监抚于时!
王说:“封啊,我不想如此多告了。古人有话说:‘人不要只从水中察看,应当从民情上察看。’现在殷商已丧失了他的福命,我们难道可以不大大地省察这个事实!

予惟曰:“汝劼毖殷献臣、侯、甸、男、卫,矧太史友、内史友、越献臣百宗工,矧惟尔事服休,服采,矧惟若畴,圻父薄违,农夫若保,宏父定辟,矧汝,刚制于酒。’
我想告诉你:“你要慎重告诫殷国的贤臣,侯、甸、男、卫的诸侯,又朝中记事记言的史官,贤良的大臣和许多尊贵的官员,还有你的治事官员,管理游宴休息和祭祀的近臣,还有你的三卿,讨伐叛乱的圻父,顺保百姓的农父,制定法度的宏父:‘你们要强行断绝饮酒!’

厥或诰曰:‘群饮。’汝勿佚。尽执拘以归于周,予其杀。又惟殷之迪诸臣惟工,乃湎于酒,勿庸杀之,姑惟教之。有斯明享,乃不用我教辞,惟我一人弗恤弗蠲,乃事时同于杀。”
假若有人报告说:‘有人群聚饮酒。’你不要放纵他们,要全部逮捕起来送到周京,我将杀掉他们。又殷商的辅臣百官酣乐在酒中,不用杀他们,暂且先教育他们。有这样明显的劝戒,若还有人不遵从我的教令,我不会怜惜,不会赦免,处治这类人,同群聚饮酒者一样,要杀。”

王曰:“封,汝典听朕毖,勿辩乃司民湎于酒。”
王说:“封啊,你要经常听从我的告诫,不要使你的官员酣乐在酒中。”

\chapter{梓材}

王曰:“封,以厥庶民暨厥臣达大家,以厥臣达王惟邦君,汝若恒越曰:我有师师、司徒、司马、司空、尹、旅。”曰:‘予罔厉杀人。’亦厥君先敬劳,肆徂厥敬劳。肆往,奸宄、杀人、历人,宥;肆亦见厥君事、戕败人,宥。
王说:“封啊,从殷的老百姓和他们的官员到卿大夫,从他们的官员到诸侯和国君,你要顺从常典。告诉我们的各位官长、司徒、司马、司空、大夫和众士说:‘我们不滥杀无罪的人。’各位邦君也当以敬重慰劳为先,努力去施行那些敬重慰劳人民的事吧!往日,内外作乱的罪犯、杀人的罪犯、虏人的罪犯,要宽恕;往日,洩露国君大事的罪犯、残坏人体的罪犯,也要宽恕。

王启监,厥乱为民。曰:‘无胥戕,无胥虐,至于敬寡,至于属妇,合由以容。’王其效邦君越御事,厥命曷以?‘引养引恬。’自古王若兹,监罔攸辟!惟曰:若稽田,既勤敷菑,惟其陈修,为厥疆畎。若作室家,既勤垣墉,惟其涂塈茨。若作梓材,既勤朴斫,惟其涂丹雘。
王者建立诸侯,大率在于教化人民。他说:‘不要互相残害,不要互相暴虐,至于鳏夫寡妇,至于孕妇,要同样教导和宽容。’王者教导诸侯和诸侯国的官员,他的诰命是用什么呢?就是‘长养百姓,长安百姓’。自古君王都像这样监督,没有什么偏差!我想:好像作田,既已勤劳地开垦、播种,就应当考虑整治土地,修筑田界,开挖水沟。好比造房屋,既已勤劳地筑起了墙壁,就应当考虑完成涂泥和盖屋的工作。好比制作梓木器具,既已勤劳地剥皮砍削,就应当考虑完成彩饰的工作。

今王惟曰:先王既勤用明德,怀为夹,庶邦享作,兄弟方来。亦既用明德,后式典集,庶邦丕享。皇天既付中国民越厥疆土于先王,肆王惟德用,和怿先后为迷民,用怿先王受命。已!若兹监,惟曰欲至于万年,惟王子子孙孙永保民。”
现在我们王家考虑:先王既已努力施行明德,来作洛邑,各国都来进贡任役,兄弟邦国也都来了。又是已经施行了明德,诸侯就依据常例来朝见,众国才来进贡。上天既已把中国的臣民和疆土都付给先王,今王也只有施行德政,来和悦、教导殷商那些迷惑的人民,用来完成先王所受的使命。唉!像这样治理殷民,我想你将传到万年,同王的子子孙孙永远保有殷民。”

\chapter{召诰}

成王在丰,欲宅洛邑,使召公先相宅,作《召诰》。
周成王在丰,打算居住到洛邑,便派召公先去视察可居住之地,史官据此作《召诰》。

惟二月既望,越六日乙未,王朝步自周,则至于丰。
二月十六日以后,到第六天乙未日,成王早晨从镐京出发,到了丰邑。

惟太保先周公相宅,越若来三月,惟丙午朏。越三日戊申,太保朝至于洛,卜宅。厥既得卜,则经营。越三日庚戌,太保乃以庶殷攻位于洛汭。越五日甲寅,位成。
太保召公在周公之前抵达之前到达洛地视察营建的地址。到了三月初三,新月初现光辉。又过了三日到戊申这天,周公早晨到达了洛地,对选好的地址进行了占卜。得了吉兆,就开始规划起来。又过了三天到庚戌这天,太保便率领众多殷民,在洛水与黄河汇合的地方测定新邑的位置。又过了五日到了甲寅这天,基地就建成了。

若翼日乙卯,周公朝至于洛,则达观于新邑营。越三日丁巳,用牲于郊,牛二。越翼日戊午,乃社于新邑,牛一,羊一,豕一。
到了第二天的乙卯日,周公早晨到达洛地,就全面视察新邑的区域。到第三天丁巳,在南郊杀了两头牛,用牲祭祀上天。到第二天戊午,又在新邑举行祭地的典礼,用了一头牛、一头羊和一头猪。

越七日甲子,周公乃朝用书命庶殷侯甸男邦伯。厥既命殷庶,庶殷丕作。
到第七天甲子,周公就在早晨用诰书命令殷民以及侯、甸、男各国诸侯营建洛邑。下达完命令后,殷民就大规模的开始动工。

太保乃以庶邦冢君出取币,乃复入锡周公。曰:“拜手稽首,旅王若公诰告庶殷越自乃御事:呜呼!皇天上帝,改厥元子兹大国殷之命。惟王受命,无疆惟休,亦无疆惟恤。呜呼!曷其奈何弗敬?
太保于是同各邦国的诸侯取出了币帛,再入内进献给周公。太保说:“请接受我们的礼拜,向大王禀告,我已经把周公您的告诫转达给众多殷人和那些治事诸臣。啊!上天改变了殷国的大命,让他们不能再继续统治天下。王接受了任命,美好无穷无尽,忧患也无穷无尽。哎!怎么能够不谨慎啊?”

天既遐终大邦殷之命,兹殷多先哲王在天,越厥后王后民,兹服厥命。厥终,智藏瘝在。夫知保抱携持厥妇子,以哀吁天,徂厥亡,出执。呜呼!天亦哀于四方民,其眷命用懋。王其疾敬德!
上天结束了殷国的大命,殷国许多圣明的先王也都认命,因此殷商后来的君王和臣民,才能够享受着天命。到了纣王的末年,明智的人隐藏了,普通人都离开家做苦役。人们只知护着、抱着、牵着、扶着他们的妻子儿女,悲切地呼告上天,诅咒纣王,希望殷国赶快灭亡,能够早日脱离困境。啊!上天也哀怜四方的老百姓,它眷顾百姓的命运因此更改殷命。大王要赶快认真施行德政呀!

相古先民有夏,天迪从子保,面稽天若;今时既坠厥命。今相有殷,天迪格保,面稽天若;今时既坠厥命。今冲子嗣,则无遗寿耇,曰其稽我古人之德,矧曰其有能稽谋自天?
观察古时候的夏朝百姓,上天让知道天道的人来教导他们顺从慈保,努力考求天意,现在还是丧失了王命。现在在观察殷人,上天教导他们要爱护百姓,他们努力也考求天意,现在也已经丧失了王命。如今年幼的成万继承了王位,没有可靠地老人辅助,考求我们古代先王的德政,何况说有能考求无意的人呢?

呜呼!有王虽小,元子哉。其丕能諴于小民。今休:王不敢后,用顾畏于民碞;王来绍上帝,自服于土中。旦曰:‘其作大邑,其自时配皇天,毖祀于上下,其自时中乂;王厥有成命治民。’今休。
啊!王虽然年幼,却是国家的天子!要特别能够和悦老百姓。现在可喜的是:王不敢迟缓营建洛邑,对殷民的艰难险阻常常顾念和畏惧;王来卜问上天,打算亲自在洛邑治理他们。“姬旦对我说:‘要营建洛邑,要从这里以始祖后稷配天,谨慎祭祀天地,要从这个中心地方统治天下;王已经有定命治理人民了。’

王先服殷御事,比介于我有周御事,节性惟日其迈。王敬作所,不可不敬德。
现在可喜的是:王重视使用殷商旧臣,让他们顺从我们并亲近我们周王朝的治事官员,使他们和睦的感情一天天地增长。成王也应恭敬谨慎,以身作则,不可不敬重德行。”

我不可不监于有夏,亦不可不监于有殷。我不敢知曰,有夏服天命,惟有历年;我不敢知曰,不其延。惟不敬厥德,乃早坠厥命。我不敢知曰,有殷受天命,惟有历年;我不敢知曰,不其延。惟不敬厥德,乃早坠厥命。今王嗣受厥命,我亦惟兹二国命,嗣若功。
我们不可以不以夏朝为鉴,也不可以不以殷国为鉴。我不敢知晓说,夏接受天命有长久时间;我也不敢知晓说,夏的国运能够经历长久。我只知道他们不重视行德,才过早失去了他们的福命。我不敢知晓说,殷接受天命有长久时间;我也不敢知晓说,殷的国运不会延长。我只知道他们不重视行德,才过早失去了他们的福命。现今大王继承了治理天下的大命,我们也该思考这两个国家的命运,继承他们的功业。

王乃初服。呜呼!若生子,罔不在厥初生,自贻哲命。今天其命哲,命吉凶,命历年;知今我初服,宅新邑。肆惟王其疾敬德?王其德之用,祈天永命。
王是刚刚处理国家政事。啊!好像教养小孩一样,没有不在开初教养时,就亲自传给他明哲的教导的。现今上天该给予明哲,给予吉祥,给予永年;因为上天知道我王初理国事时,就住到新邑来了。现在王该加快认真推行德政!王该用德政,向上天祈求长久的福命。

其惟王勿以小民淫用非彝,亦敢殄戮用乂民,若有功。其惟王位在德元,小民乃惟刑用于天下,越王显。上下勤恤,其曰我受天命,丕若有夏历年,式勿替有殷历年。欲王以小民受天永命。”
愿王不要与老百姓一般肆行非法的事,也不要用杀戮的方式来治理老百姓,如此才会有功绩。愿王立于德臣之首,让老百姓效法施行于天下,发扬王的美德。君臣上下勤劳忧虑,也许可以说,我们接受的大命会象夏代那样久远,不止殷代那样久远,愿君王和臣民共同接受好上天的永久大命。”

拜手稽首,曰:“予小臣敢以王之仇民百君子越友民,保受王威命明德。王末有成命,王亦显。我非敢勤,惟恭奉币,用供王能祈天永命。”
召公跪拜叩头说:“我这小臣和殷的臣民以及友好的臣民,会安然接受王的威严命令,宣扬王的大德。王也终于决定营建洛邑,王的大德便因此更加光显了。我不敢慰劳王,只想恭敬奉上币帛,以供王去好好祈求上天的永久福命。”

\chapter{洛诰}

召公既相宅,周公往营成周,使来告卜,作《洛诰》。
召公已经勘察了宗庙、宫室、朝市的建筑地址,周公前往营建洛邑,派遣使者请周成王来洛邑,把所卜得的吉兆报告给周成王,史官写了《洛诰》。

周公拜手稽首曰:“朕复子明辟。王如弗敢及天基命定命,予乃胤保大相东土,其基作民明辟。予惟乙卯,朝至于洛师。我卜河朔黎水,我乃卜涧水东,瀍水西,惟洛食;我又卜瀍水东,亦惟洛食。伻来以图及献卜。”
周公行礼之后说:“我告诉您治理洛邑的重大政策。你却谦逊不敢举行继位大典。我要在太保之后,全面视察了洛邑,你必须开始学会做一个圣明的君主。我在乙卯日这天,早晨的时候到了洛邑。我先占卜了黄河北方的黎水地区,我又占卜了涧水以东、瀍水以西地区,仅有洛地吉利。我又占卜了瀍水以东地区,也仅有洛地吉利。于是请您来商量,且献上卜兆。”

王拜手稽首曰:“公不敢不敬天之休,来相宅,其作周配,休!公既定宅,伻来,来,视予卜,休恒吉。我二人共贞。公其以予万亿年敬天之休。拜手稽首诲言。”
成王跪拜叩头,回答说:“公不敢不敬重上天赐给的福庆,亲自勘察地址,将营建与镐京相配的新邑,很好啊!公既已选定地址,使我来,我来了,又让我看了卜兆,我为卜兆并吉而高兴。让我们二人共同承当这一吉祥。愿公领着我永远敬重上天赐给的福庆!跪拜叩头接受我公的教诲。”

周公曰:“王,肇称殷礼,祀于新邑,咸秩无文。予齐百工,伻从王于周,予惟曰:‘庶有事。’今王即命曰:‘记功,宗以功作元祀。’惟命曰:‘汝受命笃弼,丕视功载,乃汝其悉自教工。’
周公说:“王啊,你在开始举行殷礼接见诸侯,在新邑举行祭祀,都已安排得有条不紊了。我率领百官,让他们在镐京听取王的意见,我想说您能够同官员一起举行祭祀的大事。现在王命令道:‘记下功绩,让宗人率领功臣举行大祭祀就可以了。’王又有命令道:‘你接受先王遗命,督导辅助,你全面查阅记功的书,那么你尽力教导百官就行了。’

孺子其朋,孺子其朋,其往!无若火始焰焰;厥攸灼叙,弗其绝。厥若彝及抚事如予,惟以在周工往新邑。伻向即有僚,明作有功,惇大成裕,汝永有辞。”
“王啊!您应该同官员一同到洛邑去!不要像火刚开始燃烧时那样气势很弱;那燃烧的馀火,决不可让它熄灭。您要像我一样顺从常法,汲汲主持政事,率领在镐京的官员到洛邑去。使他们各就其职,勉力建立功勋,重视大事,完成大业。您就会永远获得美誉。”

公曰:“已!汝惟冲子,惟终。汝其敬识百辟享,亦识其有不享。享多仪,仪不及物,惟曰不享。惟不役志于享,凡民惟曰不享,惟事其爽侮。乃惟孺子颁,朕不暇听。
周公说:“唉!您虽然是个年轻人,但是也要考虑完成先王未竟的功业。您应该认真考察诸侯的享礼,也要考察其中没有享礼的诸侯。享礼注重礼节,假如礼节不正确,就算贡享很多,也没有用。因为诸侯对享礼不诚心,老百姓就会认为可以不享。这样,政事将会错乱怠慢。我急想您来分担政务,我已经没有这么多时间来管理这么多事情啊!

朕教汝于棐民,彝汝乃是不蘉,乃时惟不永哉!笃叙乃正父罔不若予,不敢废乃命。汝往敬哉!兹予其明农哉!被裕我民,无远用戾。”
我教给您辅导百姓的法则,您假如不努力办这些事,您的善政就不会推广啊!全像我一样监督诠叙您的官长,他们就不敢废弃您的命令了。您到新邑去,要认真啊!现在我们要奋发努力啊!去教导好我们的百姓。远方的人因此也就归附了。”

王若曰:“公!明保予冲子。公称丕显德,以予小子扬文武烈,奉答天命,和恒四方民,居师;惇宗将礼,称秩元祀,咸秩无文。惟公德明光于上下,勤施于四方,旁作穆穆,迓衡不迷。文武勤教,予冲子夙夜毖祀。”王曰:“公功棐迪,笃罔不若时。”
王这样说:“公啊!您竭尽全力辅佐我这个年幼的人。扬伟大光显的功德,使我继承文王、武王的事业,尊奉上天的教诲,使四方百姓安居乐业,并定居在洛邑;隆重举行大礼,办理好盛大的祭祀,都有条不紊。您的功德光照天地,勤劳施于四方,普遍推行美好的政事,虽然遭受到很多不好的事情却不产生差错。文武百官努力实行您的教化,我这年幼无知的人就早夜慎重进行祭祀好了。”王说:“公善于辅导,我真的无不顺从。”

王曰:“公!予小子其退,即辟于周,命公后。四方迪乱未定,于宗礼亦未克敉,公功迪将,其后监我士师工,诞保文武受民,乱为四辅。”王曰:“公定,予往已。以功肃将祗欢,公无困哉!我惟无斁其康事,公勿替刑,四方其世享。”
王说:“公啊!我这年轻人就要回去,在镐京就位了,请公继续治理洛。四方经过教导治理,还没有安定,宗礼也没有完成,公善于教导扶持,要继续监督我们的各级官员,安定文王、武王所接受的殷民,做我的辅佐大臣。”王说:“公留下吧!我要往镐京去了。公要好好地迅速地进行敬重和睦殷民的工作,公不要让我危困呀!我当不懈地学习政事,公要不停地示范,四方诸侯将会世世代代来到周国朝享了。”

周公拜手稽首曰:“王命予来承保乃文祖受命民,越乃光烈考武王弘朕恭。孺子来相宅,其大惇典殷献民,乱为四方新辟,作周恭先。曰其自时中乂,万邦咸休,惟王有成绩。予旦以多子越御事笃前人成烈,答其师,作周孚先。’考朕昭子刑,乃单文祖德。
周公行礼后说:“王命我承担治理你祖父文王从上天那里接受的保护臣民的任务,以及光大你父亲的遗训大法。王来视察洛邑的时候,要使殷商贤良的臣民都惇厚守法,制定了治理四方的新法,作了周法的先导。我曾经说过:‘要是从这九州的中心进行治理,万国都会喜欢,王也会有功绩。我姬旦率领众位卿大夫和治事官员,经营先王的成业,集合众人,作修建洛邑的先导。’实现我告诉您的这一法则,就能发扬光大先祖文王的美德。

伻来毖殷,乃命宁予以秬鬯二卣。曰明禋,拜手稽首休享。予不敢宿,则禋于文王、武王。惠笃叙,无有遘自疾,万年厌于乃德,殷乃引考。王伻殷乃承叙万年,其永观朕子怀德。”
您派遣使者来洛邑慰劳殷人,又送来两樽黍酒。使者传达王命说:‘明洁地举行祭祀,要跪拜叩头庆幸地献给文王和武王。’我祈祷说:‘愿我很顺遂,不要遇到罪疾,万年饱受您的德泽,殷事能够长久成功。’‘愿王使殷民能够顺从万年,永远像我们的百姓一样心怀大德不敢叛逆。’”

戊辰,王在新邑烝,祭岁,文王騂牛一,武王騂牛一。王命作册逸祝册,惟告周公其后。
戊辰这天,成王在洛邑举行冬祭,向先王报告岁事,用一头红色的牛祭文王,也用一头红色的牛祭武王。成王命令作册官名字叫逸的宣读册文,报告文王、武王,周公将继续住在洛邑。

王宾杀禋咸格,王入太室,裸。王命周公后,作册逸诰,在十有二月。惟周公诞保文武受命,惟七年。
助祭诸侯在杀牲祭祀先王的时候都来到了,成王命令周公继续治理洛邑,作册官名字叫逸的将这件大事告喻天下,在十二月。周公留居洛邑担任文王、武王所受的大命,在成王七年。

\chapter{多士}

惟三月,周公初于新邑洛,用告商王士。
周成王七年三月,周公初往新都洛邑,用成王的命令告诫殷商的旧臣。

王若曰:“尔殷遗多士,弗吊旻天,大降丧于殷,我有周佑命,将天明威,致王罚,敕殷命终于帝。肆尔多士!非我小国敢弋殷命。惟天不畀允罔固乱,弼我,我其敢求位?惟帝不畀,惟我下民秉为,惟天明畏。
王这样说:“你们这些殷商的旧臣!纣王不敬重上天,他把灾祸降给殷国。我们周国佑助天命,奉行上天的明威,执行王者的诛罚,宣告殷的国命被上天终绝了。现在,你们众位官员啊!不是我们小小的周国敢于取代殷命,是上天不把大命给予那信诬怙恶的人,而辅助我们,我们岂敢擅求王位呢?正因为上天不把大命给予信诬怙恶的人,我们下民的所作所为,应当敬畏天命。

我闻曰:上帝引逸,有夏不适逸;则惟帝降格,向于时夏。弗克庸帝,大淫泆有辞。惟时天罔念闻,厥惟废元命,降致罚;乃命尔先祖成汤革夏,俊民甸四方。
我听说:‘上天制止游乐。’夏桀不节制游乐,上天就降下教令,劝导复桀。他不能听取上天的教导,大肆游乐,并且怠慢。因此,上天也不念不问,而考虑废止夏的大命,降下大罚;上天于是命令你们的先祖成汤代替夏桀,命令杰出的人才治理四方。

自成汤至于帝乙,罔不明德恤祀。亦惟天丕建,保乂有殷,殷王亦罔敢失帝,罔不配天其泽。
从成汤到帝乙,没有人不力行德政,慎行祭祀。也因为上天树立了安治殷国的贤人,殷的先王也没有人敢于违背天意,也没有人不配合上天的恩泽。

在今后嗣王,诞罔显于天,矧曰其有听念于先王勤家?诞淫厥泆,罔顾于天显民祗,惟时上帝不保,降若兹大丧。惟天不畀不明厥德,凡四方小大邦丧,罔非有辞于罚。”
当今后继的纣王,很不明白上天的意旨,何况说他又能听从、考虑先王勤劳家国的训导呢?他大肆淫游泆乐,不顾天意和民困,因此,上天不保佑了,降下这样的大丧乱。上天不把大命给予不勉行德政的人,凡是四方小国大国的灭亡,无人不是怠慢上天而被惩罚。”

王若曰:“尔殷多士,今惟我周王丕灵承帝事,有命曰:‘割殷’告敕于帝。惟我事不贰适,惟尔王家我适。予其曰惟尔洪无度,我不尔动,自乃邑。予亦念天,即于殷大戾,肆不正。”
王这样说:“你们殷国的众臣,现在只有我们周王善于奉行上天的使命,上天有命令说:‘夺取殷国,并报告上天。’我们讨伐殷商,不把别人作为敌人,只把你们的王家作为敌人。我怎么会料想到你们众官员太不守法,我并没有动你们,动乱是从你们的封邑开始的。我也考虑到天意仅仅在于夺取殷国,于是在殷乱大定之后,便不治你们的罪了。”

王曰:“猷!告尔多士,予惟时其迁居西尔,非我一人奉德不康宁,时惟天命。无违,朕不敢有后,无我怨。
王说:“啊!告诉你们众官员,我因此将把你们迁居西方,并不是我执行教导不安静,这是天命。不可违背天命,我不敢迟缓执行天命,你们不要怨恨我。

惟尔知,惟殷先人有册有典,殷革夏命。今尔又曰:‘夏迪简在王庭,有服在百僚。’予一人惟听用德,肆予敢求于天邑商,予惟率肆矜尔。非予罪,时惟天命。”
你们知道,殷人的祖先有书册有典籍,记载着殷国革了夏国的命。现在你们又说:‘当年夏的官员被选在殷的王庭,在百官之中都有职事。’我只接受、使用有德的人。现在我从大邑商招来你们,我是宽大你们和爱惜你们。这不是我的差错,这是天命。”

王曰:“多士,昔朕来自奄,予大降尔四国民命。我乃明致天罚,移尔遐逖,比事臣我宗多逊。”
王说:“殷的众臣,从前我从奄地来,对你们管、蔡、商、奄四国臣民广泛地下达过命令。我然后明行上天的惩罚,把你们从远方迁徙到这里,近来你们服务和臣属我们周族很恭顺。”

王曰:“告尔殷多士,今予惟不尔杀,予惟时命有申。今朕作大邑于兹洛,予惟四方罔攸宾,亦惟尔多士攸服奔走臣我多逊。尔乃尚有尔土,尔用尚宁干止,尔克敬,天惟畀矜尔;尔不克敬,尔不啻不有尔土,予亦致天之罚于尔躬!今尔惟时宅尔邑,继尔居;尔厥有干有年于兹洛。尔小子乃兴,从尔迁。”
王说:“告诉你们殷商的众臣,现在我不杀害你们,我想重申这个命令。现在我在这洛地建成了一座大城市,我是由于四方诸侯没有地方朝贡,也是由于你们服务奔走臣属我们很恭顺的缘故。你们还可以保有你们的土地,你们还会安宁下来。你们能够敬慎,上天将会对你们赐给怜爱;你们假如不能敬慎,你们不但不能保有你们的土地,我也将会把老天的惩罚加到你们身上。现在你们应当好好地住在你们的城里,继续做你们的事业。你们在洛邑会有安乐会有丰年的。从你们迁来洛邑开始,你们的子孙也将兴旺发达。”

王曰:“又曰时予,乃或言尔攸居。”
王说:“顺从我!顺从我!才能够谈到你们长久安居下来。”

\chapter{无逸}

周公作《无逸》。
周公作《无逸》。

周公曰:“呜呼!君子,所其无逸。先知稼穑之艰难,乃逸,则知小人之依。相小人,厥父母勤劳稼穑,厥子乃不知稼穑之艰难,乃逸乃谚。既诞,否则侮厥父母曰:‘昔之人无闻知。’”
周公说:“啊!君子在位,切不可贪图享乐。要先了解耕种收获的艰难,如此处在逸乐的境地,就会知道老百姓的艰辛了。看那些百姓,他们的父母勤劳地耕种收获,他们的儿子却不知道耕种收获的艰难,便安逸的享受起来。时间久了,行为就会十分放肆,于是就轻视侮慢他们的父母说:‘年纪大的人什么都不懂。’”

周公曰:“呜呼!我闻曰:昔在殷王中宗,严恭寅畏天命,自度治民,治民祗惧,弗敢荒宁。肆中宗之享国七十有五年。其在高宗,时旧劳于外,爰暨小人。作其即位,乃或亮阴,三年不言。其惟不言,言乃雍。不敢荒宁,嘉靖殷邦。至于小大,无时或怨。肆高宗之享国五十年有九年。其在祖甲,不义惟王,旧为小人。作其即位,爰知小人之依,能保惠于庶民,不敢侮鳏寡。肆祖甲之享国三十有三年。自时厥后,立王生则逸。生则逸,不知稼穑之艰难,不闻小人之劳,惟耽乐之从。自时厥后,亦罔或克寿。或十年,或七八年,或五六年,或四三年。”
周公说:“啊!我听说:过去殷王中宗,庆正敬畏,以天命作为自己的准则,治理百姓都是敬慎恐惧,从不敢懈怠,贪图享乐。所以中宗在位七十五年。在高宗,这个人长期在外服役,惠爱老百姓。等到他即位,便又听信冢宰沉默不言,三年不轻易说话。因为他不轻易说话,有时说出来就能使人和悦。他不敢荒废、安逸,善于安定殷国。从老百姓到群臣,没有怨恨他的。所以高宗在位五十九年。在祖甲,他以为代兄称王不合情理,逃亡民间,做过很久的平民百姓。等到他即位后,就知道老百姓的痛苦,能够安定和爱护众民,对于鳏寡无依的人也不敢轻慢。所以祖甲在位三十三年。从这以后,在位的殷王生来就安闲逸乐,生来就安闲逸乐,不知耕种收获的艰难,不知老百姓的劳苦,只是追求过度的逸乐。从这以后,在位的殷王也没有能够长寿的。有的十年,有的七、八年,有的五、六年,有的三、四年。”

周公曰:“呜呼!厥亦惟我周太王、王季,克自抑畏。文王卑服,即康功田功。徽柔懿恭,怀保小民,惠鲜鳏寡。自朝至于日中昃,不遑暇食,用咸和万民。文王不敢盘于游田,以庶邦惟正之共。文王受命惟中身,厥享国五十年。”
周公说:“啊!只有我们周的太王、王季做事的时候谨慎小心。文王安于卑贱的工作,从事过开通道路、耕种田地的劳役。他和蔼、仁慈、善良、恭敬,在他的治理下,百姓安居乐业。那是他总是从早晨到中午,到下午,他没有闲暇吃饭,要使万民生活和谐。文王不敢乐于嬉游、田猎,不敢使众国只是进献赋税,供他享乐。文王中年受命为君,在位五十年。”

周公曰:“呜呼!继自今嗣王,则其无淫于观、于逸、于游、于田,以万民惟正之共。无皇曰:‘今日耽乐。’乃非民攸训,非天攸若,时人丕则有愆。无若殷王受之迷乱,酗于酒德哉!”
周公说:“啊!从今以后的继位君王,希望你不要沉迷在观赏、安逸、嬉游和田猎之中,不可只是使老百姓进献赋税供他享乐。不要自我宽解说:‘今天先享受享受再说。’这样子,就不是老百姓所赞成的,也不是上天所喜爱的,这样的人就有罪过了。不要象商纣王那样迷惑昏乱,把酗酒作为酒德啊!”

周公曰:“呜呼!我闻曰:‘古之人犹胥训告,胥保惠,胥教诲,民无或胥譸张为幻。’此厥不听,人乃训之,乃变乱先王之正刑,至于小大。民否则厥心违怨,否则厥口诅祝。”
周公说:“啊!我听说:‘古时的人还能互相劝导,互相爱护,互相教诲,所以老百姓没有互相欺骗、互相诈惑的。’不依照这样,官员就会顺从自己的意愿,就会变动先王的正法,以至于大大小小的法令。老百姓于是就内心怨恨,就口头诅咒了。”

周公曰:“呜呼!自殷王中宗及高宗及祖甲及我周文王,兹四人迪哲。厥或告之曰:‘小人怨汝詈汝。’则皇自敬德。厥愆,曰:‘朕之愆。’允若时,不啻不敢含怒。此厥不听,人乃或譸张为幻,曰小人怨汝詈汝,则信之,则若时,不永念厥辟,不宽绰厥心,乱罚无罪,杀无辜。怨有同,是丛于厥身。”
周公说:“啊!从殷王中宗、到高宗、到祖甲、到我们的周文王,这四位君王都是圣明的君主。有人告诉他们说:‘老百姓在怨恨你咒骂你。’他们就更加敬慎自己的行为;有人举出他们的过错,他们就说:‘这件事情确实是我做错了。’不但不发怒,而且非常愿意听到这样的话,不依照这样,人们就会互相欺骗、互相诈惑。有人说老百姓在怨恨你咒骂你,你就会相信,就会象这样:不多考虑国家的法度,不放宽自己的心怀,乱罚没有罪过的人,乱杀没有罪过的人。老百姓的怨恨一旦汇合起来,就会集中到你的身上。”

周公曰:“呜呼!嗣王其监于兹。”
周公说:“啊!成王要鉴戒这些啊!”

\chapter{君奭}

召公为保,周公为师,相成王为左右。召公不说,周公作《君奭》。
周成王时,召公任太保,周公任太师,他们同时成为成王身边的辅弼大臣。召公感到不高兴,于是周公写下这篇《君奭》。

周公若曰:“君奭!弗吊天降丧于殷,殷既坠厥命,我有周既受。我不敢知曰:厥基永孚于休。若天棐忱,我亦不敢知曰:其终出于不祥。呜呼!君已曰‘时我’,我亦不敢宁于上帝命,弗永远念天威越我民;罔尤违,惟人。在我后嗣子孙,大弗克恭上下,遏佚前人光在家,不知天命不易,天难谌,乃其坠命,弗克经历。嗣前人,恭明德,在今。予小子旦非克有正,迪惟前人光施于我冲子。又曰:‘天不可信’,我道惟宁王德延,天不庸释于文王受命。”
周公说:“君奭!很不幸,上天把丧亡之祸降临给了殷商。现在殷朝已经丧失了他们的天命,而我们周室承受了上天赐予的大命。但是我不敢说周室已开始的基业就能永久地延续下去。恭敬诚恳地顺从上天的意志,我也不敢断言周朝的王业能摆脱不详的结果。唉!你曾经说过:‘依靠我们自己,不要执守天命’,我也不敢安于天命,也不敢不时时顾念上天的威严,安抚我们的人民。要想没有过错,一切在于人为。恐怕我们的后嗣子孙,不能敬天理民,不能继承发扬先王的光荣传统。身在家中,不知道天命难得,天意也难以信赖,如果他们不能继承发扬前人的光荣传统,他们将永远失去天命。现在我们应继承文王、武王的德业,恭敬地施行德政。我无法对自己的错误有所纠正,只想把文王的光辉德业传给我们的后辈。你还说过:‘上天是不可以相信的’,我们只有继承和发扬文王的美德,使之长久的保持下去,才会使上天不抛弃文王所承受的福命。”

公曰:“君奭!我闻在昔成汤既受命,时则有若伊尹,格于皇天。在太甲,时则有若保衡。在太戊,时则有若伊陟、臣扈,格于上帝;巫咸乂王家。在祖乙,时则有若巫贤。在武丁,时则有若甘盘。率惟兹有陈,保乂有殷,故殷礼陟配天,多历年所。天维纯佑命,则商实百姓王人,罔不秉德明恤,小臣屏侯甸,矧咸奔走。惟兹惟德称,用乂厥辟,故一人有事于四方,若卜筮罔不是孚。”
周公说:“君奭!我听说昔日商王成汤受了天命后,就有伊尹这样的贤臣辅助他,使他得以升配于天。太甲即位后,则有保衡的辅佐;太戊时又有贤臣伊陟、臣扈辅佐,使他们得以升配于上天。巫咸帮助商王治理国家。祖乙即位后,当时就有贤臣巫贤。武丁时期,则有贤臣甘盘的辅佐。正因为有了这些贤臣辅助治理殷商,才使殷商诸王的神灵能够配享上天的祭祀,这种礼制延续了很多年,未曾改变。上天专心帮助教导下民,于是商朝所有同姓和异姓之族莫不秉承其德业,明恤其政事。君王身边的亲近重臣及诸侯官员,也都奔走效命。诸臣因为有美德而被推举,来辅佐他们的君王。所以一旦君王向四方发出政令,天下的臣民就像信奉卜筮的灵验一样,没有人不遵从君王的教令。”

公曰:“君奭!天寿平格,保乂有殷,有殷嗣,天灭威。今汝永念,则有固命,厥乱明我新造邦。”
周公说:“君奭!天道中正平和,殷商才得以长治久安,于是商王世代相继,上天也没有给殷商降下灾祸。现在你一定要记住这个历史教训,只要我们尽心辅佐,就会有稳固的国运,以好的方法治理国家。”

公曰:“君奭!在昔上帝割申劝宁王之德,其集大命于厥躬。惟文王尚克修和我有夏;亦惟有若虢叔,有若闳夭,有若散宜生,有若泰颠,有若南宫括。”又曰:“无能往来,兹迪彝教,文王蔑德降于国人。亦惟纯佑秉德,迪知天威,乃惟时昭文王,迪见冒闻于上帝,惟时受有殷命哉!武王惟兹四人,尚迪有禄。后暨武王,诞将天威,咸刘厥敌。惟兹四人昭武王,惟冒丕单称德。今在予小子旦,若游大川,予往暨汝奭其济。小子同未在位,诞无我责,收罔勖不及。耇造德不降,我则鸣鸟不闻,矧曰其有能格?”
周公说:“君奭!在以前,上天为什么再三勉励文王修德,把治理天下的重任放在他的身上呢?因为只有我们文王才能够将国家治理好,也因为当时有虢叔、闳夭、散宜生、泰颠、南宫括等贤臣辅佐他。”周公又说:“如果没有这些贤臣辅佐文王,努力施行教化,文王也就无法施行德政给民众了。也因为文王至始至终保持美德,让百姓了解上天的威严,勉励他们使他们的功绩昭著,上天了解了他们的行为,才让他们接受殷国的大命。武王的时候,这五位贤臣中还有四人健在,后来武王征伐殷国的时候,也辅佐在旁,奋勇杀敌。由于这四人尽职尽责地辅佐武王,才使武王成就大业,所以天下人都称赞他们的美德。现在我像是在大江大河中漂游,我需要你的帮助才能渡河成功。现在我无知却担任重要职位,如果你不能提出匡正我的意见,那就没有人纠正我的不足了。您这老成有德的人不指导我,我就听不到凤凰的叫声,那我的能力怎么能够匹配上天给的责任呢?”

公曰:“呜呼!君肆其监于兹!我受命于疆惟休,亦大惟艰。告君,乃猷裕我,不以后人迷。”
周公说:“唉!您现在应该认识到这一点,我们承受天命,享受着无比的恩惠,却也面临着极大的困难。请你谋划出可以使周室兴盛的种种措施,不要让后人感到迷惑啊!"

公曰:“前人敷乃心,乃悉命汝,作汝民极。曰:‘汝明勖偶王,在亶。乘兹大命,惟文王德丕承,无疆之恤。’”
周公说:“先王曾经表明了他的心愿,他悉心教导你,命你做民众的表率。并且说:‘你们要尽心尽力地辅佐君王,负责地承担起这个重要的使命。要继承文王的圣德,并把它作为长久的计划。”

公曰:“君!告汝,朕允保奭。其汝克敬,以予监于殷丧大否,肆念我天威。予不允,惟若兹诰,予惟曰:‘襄我二人,汝有合哉?’言曰:‘在时二人。天休兹至,惟时二人弗戡。’其汝克敬德,明我俊民,在让后人于丕时。呜呼!笃棐时二人,我式克至于今日休?我咸成文王功于不怠,丕冒海隅出日,罔不率俾。”
周公说:“君!告诉你,我是非常信任你太保奭,希望你能够敬重我说的,和我一道吸取殷商亡国的教训,时时顾虑上天的惩罚。如果我不诚心,会说这些话吗?我考虑之后还要问你:‘除了我们两个人,还有比我们更同心同德的人吗?’你会说:‘正是有我们二人共辅王室,上天才降下更多的恩惠。但到了我们二人老去,就不能承担这么多的恩惠。’希望你能敬重贤良之士,提拔贤能之人,帮助后人继承先王的德业。唉!如果真的没有我们二人,我们周朝能有今日这样的美好吗?我和你成就文王的功业在于不懈怠,要使那海边日出之地,也没有人不服从我们的统治。”

公曰:“君!予不惠若兹多诰,予惟用闵于天越民。”
周公说:“君!我不是很聪慧,说了这么多,我只忧心天命和我们的人民。”

公曰:“呜呼!君!惟乃知民德亦罔不能厥初,惟其终。祗若兹,往敬用治!”
周公说:“唉!君!你知道给民众施行德政,开始时没有不好好干的,但是却很少有人能够始终如一,坚持到底。我们要记住这一点,勤劳恭敬地治理国家。”

\chapter{蔡仲之命}

蔡叔既没,王命蔡仲,践诸侯位,作《蔡仲之命》。
蔡叔度被流放后,爵位也被剥夺,蔡叔度的儿子蔡仲有德行,于是周公最后又把蔡仲封在了蔡国,并且写了《蔡仲之命》来告诫他。

惟周公位冢宰,正百工,群叔流言。乃致辟管叔于商;囚蔡叔于郭邻,以车七乘;降霍叔于庶人,三年不齿。蔡仲克庸只德,周公以为卿士。叔卒,乃命诸王邦之蔡。王若曰:“小子胡,惟尔率德改行,克慎厥猷,肆予命尔侯于东土。往即乃封,敬哉!尔尚盖前人之愆,惟忠惟孝;尔乃迈迹自身,克勤无怠,以垂宪乃后;率乃祖文王之遗训,无若尔考之违王命。皇天无亲,惟德是辅。民心无常,惟惠之怀。为善不同,同归于治;为恶不同,同归于乱。尔其戒哉!慎厥初,惟厥终,终以不困;不惟厥终,终以困穷。懋乃攸绩,睦乃四邻,以蕃王室,以和兄弟,康济小民。率自中,无作聪明乱旧章。详乃视听,罔以侧言改厥度。则予一人汝嘉。”王曰:“呜呼!小子胡,汝往哉!无荒弃朕命!”
周公位居大宰、统帅百官的时候,几个弟弟对他散布流言。周公于是到达商地,杀了管叔;囚禁了蔡叔,用车七辆把他送到郭邻;把霍叔降为庶人,三年不许录用。蔡仲能够经常重视德行,周公任用他为卿士。蔡叔死后,周公便告诉成王封蔡仲于蔡国。成王这样说:“年轻的姬胡!你遵循祖德改变你父亲的行为,能够谨守臣子之道,所以我任命你到东土去做诸侯。你前往你的封地,要敬慎呀!你当掩盖前人的罪过,思忠思孝。你要使自身迈步前进,能够勤劳不怠,用以留下模范给你的后代。你要遵循你祖父文王的常训,不要像你的父亲那样违背天命!皇天无亲无疏,只辅助有德的人;民心没有常主,只是怀念仁爱之主。做善事虽然各不相同,都会达到安治;做恶事虽然各不相同,都会走向动乱。你要警戒呀!谨慎对待事物的开初,也要考虑它的终局,终局因此不会困窘;不考虑它的终局,终将困穷。勉力做你所行的事,和睦你的四邻,以保卫周王室,以和谐兄弟之邦,而使百姓安居成业。要循用中道,不要自作聪明扰乱旧章。要审慎你的视听,不要因片面之言改变法度。这样,我就会赞美你。”成王说:“啊!年轻的姬胡。你去吧!不要废弃我的教导!”

\chapter{多方}

成王归自奄,在宗周,诰庶邦,作《多方》。
成王从奄地回来,到了宗周,训诫勉励诸侯众国,作《多方》。

惟五月丁亥,王来自奄,至于宗周。
五月丁亥这天,成王从奄地回来,到了宗周。

周公曰:“王若曰:猷告尔四国多方惟尔殷侯尹民。我惟大降尔命,尔罔不知。洪维图天之命,弗永寅念于祀,惟帝降格于夏。有夏诞厥逸,不肯慼言于民,乃大淫昏,不克终日劝于帝之迪,乃尔攸闻。厥图帝之命,不克开于民之丽,乃大降罚,崇乱有夏。因甲于内乱,不克灵承于旅。罔丕惟进之恭,洪舒于民。亦惟有夏之民叨懫日钦,劓割夏邑。天惟时求民主,乃大降显休命于成汤,刑殄有夏。惟天不畀纯,乃惟以尔多方之义民不克永于多享;惟夏之恭多士大不克明保享于民,乃胥惟虐于民,至于百为,大不克开。乃惟成汤克以尔多方简,代夏作民主。慎厥丽,乃劝;厥民刑,用劝;以至于帝乙,罔不明德慎罚,亦克用劝;要囚殄戮多罪,亦克用劝;开释无辜,亦克用劝。今至于尔辟,弗克以尔多方享天之命,呜呼!”
周公说:“成王这样说:啊!告诉你们四国、各国诸侯以及你们众诸侯国治民的长官,我给你们大下教令,你们不可昏昏不闻。夏桀夸大天命,不常重视祭祀,上天就对夏国降下了严正的命令。夏桀大肆逸乐,不肯恤问人民,竟然大行淫乱,不能用一天时间为上天的教导而努力,这些是你们所听说过的。夏桀夸大天命,不能明白老百姓归附的道理,就大肆杀戮,大乱夏国。复桀因习于让妇人治理政事,不能很好地顺从民众,无时不贪取财物,深深地毒害了人民。也由于夏民贪婪、忿戾的风气一天天盛行,残害了夏国。上天于是寻求可以做人民君主的人,就大下光明美好的使命给成汤,命令成汤消灭夏国。上天不赐给众位诸侯,就是因为那时各国首长不能常常劝导人民,夏国的官员太不懂得保护和劝导人民,竟然都对人民施行暴虐,至于各种工作都不能开展;就是因为成汤由于那时有各国邦君的选择,代替夏桀作了君主。他慎施教令,是劝勉人;他惩罚罪人,也是劝勉人;从成汤到帝乙,没有人不宣明德教,慎施刑罚,也能够用来劝勉人;他们监禁、杀死重大罪犯,也能够用来劝勉人;他们释放无罪的人,也能够用来劝勉人。现在到了你们的君王,不能够和你们各国邦君享受上天的大命,实在可悲啊!”

王若曰:“诰告尔多方,非天庸释有夏,非天庸释有殷。乃惟尔辟以尔多方大淫,图天之命屑有辞。乃惟有夏图厥政,不集于享,天降时丧,有邦间之。乃惟尔商后王逸厥逸,图厥政不蠲烝,天惟降时丧。
王这样说:“告诉你们各位邦君,并不是上天要舍弃夏国,也不是上天要舍弃殷国。是因为你们夏、殷的君王和你们各国诸侯大肆淫佚,夸大天命,安逸而又懈怠;是因为夏桀谋划政事,不在于劝勉,于是上天降下了这亡国大祸,诸侯成汤代替了夏桀;是因为你们殷商的后王安于他们的逸乐生活,谋划政事不美好,于是上天降下这亡国大祸。

“惟圣罔念作狂,惟狂克念作圣。天惟五年须暇之子孙,诞作民主,罔可念听。天惟求尔多方,大动以威,开厥顾天。惟尔多方罔堪顾之。惟我周王灵承于旅,克堪用德,惟典神天。天惟式教我用休,简畀殷命,尹尔多方。
圣人不思考就会变成狂人,狂人能够思考就能变成圣人。上天用五年时间等待、宽暇商的子孙悔改,让他继续做万民之君主,但是,无法可以使他们思考和听从天意。上天又寻求你们众诸侯国,大降灾异,启发你们众国顾念天意,你们众国也没有人能顾念它。只有我们周王善于顺从民众,能用明德,善待神、天。上天就改用休祥指示我们,选择我周王,授予伟大的使命,治理众国诸侯。

“今我曷敢多诰。我惟大降尔四国民命。尔曷不忱裕之于尔多方?尔曷不夹介乂我周王享天之命?今尔尚宅尔宅,畋尔田,尔曷不惠王熙天之命?
现在我怎么敢重复地说?我有过发布给你们四国臣民的教令,你们为什么不劝导各国臣民?你们为什么不大大帮助我周王共享天命呢?现在你们还住在你们的住处,整治你们的田地,你们为什么不顺从周王宣扬上天的大命呢?

“尔乃迪屡不静,尔心未爱。尔乃不大宅天命,尔乃悄播天命,尔乃自作不典,图忱于正。我惟时其教告之,我惟时其战要囚之,至于再,至于三。乃有不用我降尔命,我乃其大罚殛之!非我有周秉德不康宁,乃惟尔自速辜!”
你们竟然屡次教导还不安定,你们内心不顺。你们竟然不度量天命,你们竟然完全抛弃天命,你们竟然自作不法,图谋攻击长官。我因此教导过你们,我因此讨伐你们,囚禁你们,至于再,至于三。假如还有人不服从我发布给你们的命令,那么我就要重重惩罚他们!这并不是我们周国执行德教不安静,只是你们自己招致了罪过!”

王曰:“呜呼!猷告尔有方多士暨殷多士。今尔奔走臣我监五祀,越惟有胥伯小大多正,尔罔不克臬。自作不和,尔惟和哉!尔室不睦,尔惟和哉!尔邑克明,尔惟克勤乃事。尔尚不忌于凶德,亦则以穆穆在乃位,克阅于乃邑谋介。尔乃自时洛邑,尚永力畋尔田,天惟畀矜尔,我有周惟其大介赉尔,迪简在王庭。尚尔事,有服在大僚。”
王说:“啊!告诉你们各国官员和殷国的官员,到现在你们奔走效劳臣服我周国已经五年了,所有的徭役赋税和大大小小的政事,你们没有不能遵守法规的。你们自己造成了不和睦,你们应该和睦起来!你们的家庭不和睦,你们也应该和睦起来!要使你们的城邑清明,你们应该能够勤于你们的职事。你们应当不被坏人教唆,也就可以好好地站在你们的位置上,就能够留在你们的城邑里谋求美好的生活了。你们如果用这个洛邑,长久尽力耕作你们的田地,上天会怜悯你们,我们周国会大大地赏赐你们。把你们引进选拔到朝廷来;努力做好你们的职事,又将让你们担任重要官职。”

王曰:“呜呼!多士,尔不克劝忱我命,尔亦则惟不克享,凡民惟曰不享。尔乃惟逸惟颇,大远王命,则惟尔多方探天之威,我则致天之罚,离逖尔土。”
王说:“啊!官员们,如果你们不能努力信从我的教命,你们也就不能享有禄位,老百姓也将认为你们不能享有禄位。你们如果放荡邪恶,大大地违抗王命,那就是你们各国妄图试探上天的惩罚,我就要施行上天的惩罚,使你们离开你们的故土。”

王曰:“我不惟多诰,我惟祗告尔命。”又曰:“时惟尔初,不克敬于和,则无我怨。”
王说:“我不想重复地说了,我只是认真地把天命告诉你们。”王又说:“好好地谋划你们的开始吧!若不能恭敬与和睦,那么你们就不要怨我了。”

\chapter{立政}

周公作《立政》。
周公作《立政》。

周公若曰:“拜手稽首,告嗣天子王矣。”用咸戒于王曰:“王左右常伯、常任、准人、缀衣、虎贲。”
周公这样说:“跪拜叩头,报告继承天子的王。”周公因而劝诫成王说:“王要教导常伯、常任、准人、缀衣和虎贲。”

周公曰:“呜呼!休兹知恤,鲜哉!古之人迪惟有夏,乃有室大竞,吁俊尊上帝迪,知忱恂于九德之行。乃敢告教厥后曰:‘拜手稽首后矣!’曰:‘宅乃事,宅乃牧,宅乃准,兹惟后矣。谋面,用丕训德,则乃宅人,兹乃三宅无义民。’
周公说:“啊!美好的时候就知道忧虑的人,很少啊!古代的人只有夏朝的王,他们的卿大夫很强,夏王还呼吁他们长久地尊重上天的教导,使他们知道诚实地相信九德的准则。夏朝君王经常教导诸侯们:‘跪拜叩头了,诸侯们!’夏王说:‘考察你们的常任、常伯、准人,这样,才称得上君主。以貌取人,不依循德行,假若这样考察人,你们的常任、常伯和准人就没有贤人了。’

桀德,惟乃弗作往任,是惟暴德罔后。亦越成汤陟,丕厘上帝之耿命,乃用三有宅;克即宅,曰三有俊,克即俊。严惟丕式,克用三宅三俊,其在商邑,用协于厥邑;其在四方,用丕式见德。
夏桀即位后,他不用往日任用官员的法则,于是只用些暴虐的人,终于无后。到了成汤登上天位,大受上天的明命,他选用事、牧、准三宅的官,都能就三宅的职位,选用三宅的属官,也能就其属官之位。他敬念上天选用官员的大法,能够很好地任用各级官员,他在商都用这些官员和协都城的臣民,他在天下四方,用这种大法显扬他的圣德。

呜呼!其在受德,暋为羞刑暴德之人,同于厥邦;乃惟庶习逸德之人,同于厥政。帝钦罚之,乃伻我有夏,式商受命,奄甸万姓。
啊!在商王纣继位,强行把罪人和暴虐的人聚集在他的国家里;竟然用众多亲幸和失德的人,与他一起治理国家。于是上天降下灾祸惩罚他,就使我们周王代替商纣王接受上天的大命,安抚治理天下的老百姓。

亦越文王、武王,克知三有宅心,灼见三有俊心,以敬事上帝,立民长伯。立政:任人、准夫、牧、作三事。虎贲、缀衣、趣马、小尹、左右携仆、百司庶府。大都小伯、艺人、表臣百司、太史、尹伯,庶常吉士。司徒、司马、司空、亚、旅。夷、微、卢烝。三亳阪尹。
到了文王、武王,他们能够知道三宅的思想,还能清楚地看到三宅部属的思想,用敬奉上天的诚心,普通老百姓设立官长为。设立的官职是:任人、准夫、牧作为三事;有虎贲、缀衣、趣马、小尹、左右携仆以及百司庶府;有大小邦国的君主、艺人,外臣百官;有太史、尹伯;他们都是善祥的人。诸侯国的官员有司徒、司马、司空、亚旅;夷、微、卢各国没有君主;还设立了商和夏的旧都管理官员。

文王惟克厥宅心,乃克立兹常事司牧人,以克俊有德。文王罔攸兼于庶言;庶狱庶慎,惟有司之牧夫是训用违;庶狱庶慎,文王罔敢知于兹。
文王因能够度知三宅的思想,就能设立这些官员,而且能够是俊彦有德的。文王不兼管各种教令。各种狱讼案件和各种禁戒,用和不用只顺从主管官员和牧民的人;对于各种狱讼案件和各种禁戒,文王不敢过问这些。

亦越武王,率惟敉功,不敢替厥义德,率惟谋从容德,以并受此丕丕基。”
到了武王,完成了文王的事业,不敢丢弃文王的善德,谋求顺从文王宽容的美德,因此,文王和武王共同接受了这伟大的王业。

呜呼!孺子王矣!继自今我其立政。立事、准人、牧夫,我其克灼知厥若,丕乃俾乱;相我受民,和我庶狱庶慎。时则勿有间之,自一话一言。我则末惟成德之彦,以乂我受民。
啊!您现在已是君王了。从今以后,我们要这样设立官员。设立事、准人、牧夫,我们要能明白了解他们的才能,才能更好地任用他们。治理普通的百姓,平治我们各种狱讼和各种禁戒的事务,这些事务不可代替。虽然一话一言,我们终要谋于贤德的人,来治理我们的老百姓。

呜呼!予旦已受人之徽言咸告孺子王矣。继自今文子文孙,其勿误于庶狱庶慎,惟正是乂之。
啊!我姬旦把前人的美言全部告诉君王了。从今以后,继承的贤子贤孙,千万不要在各种狱讼和各种禁戒上耽误时间,这些事只让主管官员去治理。

自古商人亦越我周文王立政,立事、牧夫、准人,则克宅之,克由绎之,兹乃俾乂,国则罔有。立政用憸人,不训于德,是罔显在厥世。继自今立政,其勿以憸人,其惟吉士,用励相我国家。
从古时的商代先王到我们的周文王设立官员,设立事、牧夫、准人,就是能够考察他们,能够扶持他们,这样才让他们治理,国事就没有失误。假如设立官员,任用贪利奸佞的人,不依循于德行,于是君王终世都会没有光彩。从今以后设立官员,千万不可任用贪利奸佞的小人,应当任用善良贤能的人,用来努力治理我们的国家。

今文子文孙,孺子王矣!其勿误于庶狱,惟有司之牧夫。其克诘尔戎兵以陟禹之迹,方行天下,至于海表,罔有不服。以觐文王之耿光,以扬武王之大烈。呜呼!继自今后王立政,其惟克用常人。”
“现在,先王贤明的子孙,您已经是君王了!您不要在各种狱讼案件上耽搁,只让主管官员和牧夫去治理,您要能够治理好军队,步着大禹的足迹,遍行天下,直至海外,没有人不服从。以此显扬文王圣德的光辉,继续武王伟大的功业。啊!从今以后,继位君王设立官员,必须任用善良的人。”

周公若曰:“太史!司寇苏公式敬尔由狱,以长我王国。兹式有慎,以列用中罚。”
周公这样说:“太史!司寇苏公规定要认真地处理狱讼案件,让我们的国家长治久安。这些问题应该慎重对待,而且要使用轻重适宜的刑罚。”

\chapter{周官}

成王既黜殷命,灭淮夷,还归在丰,作《周官》。
成王既黜殷命,灭淮夷,还归在丰,作《周官》。

惟周王抚万邦,巡侯、甸,四征弗庭,绥厥兆民。六服群辟,罔不承德。归于宗周,董正治官。
周成王安抚万国,巡视侯服、甸服等诸侯,四方征讨不来朝见的诸侯,以安定天下的老百姓。六服的诸侯,无人不奉承他的德教。成王回到王都丰邑,又督导整顿治事的官员。

王曰:“若昔大猷,制治于未乱,保邦于未危。”
成王说:“顺从往日的大法,要在未出现动乱的时候制定治理的办法,在未出现危机的时候安定国家。”

曰:“唐虞稽古,建官惟百。内有百揆四岳,外有州、牧、侯伯。庶政惟和,万国咸宁。夏商官倍,亦克用乂。明王立政,不惟其官,惟其人。
成王说:“尧舜稽考古代制度,建立官职一百。内有百揆和四岳,外有州牧和侯伯。各种政策适合,天下万国都安宁。夏代和商代,官数增加一倍,也能用来治理。明王设立官员,不考虑他的官员之多,而考虑要得到贤人。

今予小子,祗勤于德,夙夜不逮。仰惟前代时若,训迪厥官。立太师、太傅、太保,兹惟三公。论道经邦,燮理阴阳。官不必备,惟其人。少师、少傅、少保,曰三孤。贰公弘化,寅亮天地,弼予一人。冢宰掌邦治,统百官,均四海。司徒掌邦教,敷五典,扰兆民。宗伯掌邦礼,治神人,和上下。司马掌邦政,统六师,平邦国。司寇掌邦禁,诘奸慝,刑暴乱。司空掌邦土,居四民,时地利。六卿分职,各率其属,以倡九牧,阜成兆民。六年,五服一朝。又六年,王乃时巡,考制度于四岳。诸侯各朝于方岳,大明黜陟。”
现在我恭敬勤奋施行德政,早起晚睡都有所不及。仰思顺从前代,建立我们的官制。设立太师、太傅、大保,这是三公。他们讲明治道,治理国家,调和阴阳。三公的官不必齐备,要考虑适当的人。设立少师、少傅、少保,叫做三孤。他们协助三公弘扬教化,敬明天地的事,辅助我一人。冢宰主管国家的治理,统帅百官,调剂四海。司徒主管国家的教育,传布五常的教训,使万民和顺。宗伯主管国家的典礼,治理神和人的感通,调和上下尊卑的关系。司马主管国家的军政,统率六师,平服邦国。司寇主管国家的法禁,治理好恶的人,刑杀暴乱之徒。司空主管国家的土地,安置士农工商,依时发展地利。六卿分管职事,各自统率他的属官,以倡导九州之牧,大力安定兆民。六年,五服诸侯来朝见一次。又隔六年,王便依时巡视,到四岳校正制度。诸侯各在所属的方岳来朝见,王对诸侯普遍讲明升降赏罚。”

王曰:“呜呼!凡我有官君子,钦乃攸司,慎乃出令,令出惟行,弗惟反。以公灭私,民其允怀。学古入官。议事以制,政乃不迷。其尔典常作之师,无以利口乱厥官。蓄疑败谋,怠忽荒政,不学墙面,莅事惟烦。戒尔卿士,功崇惟志,业广惟勤,惟克果断,乃罔后艰。位不期骄,禄不期侈。恭俭惟德,无载尔伪。作德,心逸日休;作伪,心劳日拙。居宠思危,罔不惟畏,弗畏入畏。推贤让能,庶官乃和,不和政庞。举能其官,惟尔之能。称匪其人,惟尔不任。”
成王说:“啊!各级官长,要认真谨慎的对待你们的工作,慎重对待你们发布的命令。命令发出了就要进行,不要违抗。用公正消除私情,人民将会信任归服。先学古代治法再入仕途,议论政事依据法制,政事就不会错误。你们要用周家常法作为法则,不要以巧言干扰你的官员。蓄疑不决,必定败坏计谋,怠情忽略,必定废弃政事。不学习好象向墙站着,临事就会烦乱。告诉你们各位卿士:功高由于有志,业大由于勤劳。能够果敢决断,就没有后来的艰难。居官不当骄傲,享禄不当奢侈,恭和俭是美德啊!不要行使诈伪,行德就心逸而日美,作伪就心劳而日拙。处于尊宠要想到危辱,无事不当敬畏,不知敬畏,就会进入可畏的境地。推举贤明而让能者,众官就会和谐;众官不和,政事就复杂了。推举能者在其官位,是你们的贤能;所举不是那种人,是你们不能胜任。”

王曰:“呜呼!三事暨大无,敬尔有官,乱尔有政,以佑乃辟。永康兆民,万邦惟无斁。
成王说:“啊!任人、准夫、牧三位首长和大夫们:认真对待你们的官职,治理你们的政事,来辅助你们的君主,使广大百姓长远安宁;天下万国就不会厌弃我们了。”

\chapter{君陈}

周公既沒,命君陈分正东郊成周,作《君陈》。
周公既沒,命令君陈治理东郊成周,史官作《君陈》。

王若曰:“君陈,惟尔令德孝恭。惟孝友于兄弟,克施有政。命汝尹茲东郊,敬哉!昔周公师保万民,民怀其德。往慎乃司,茲率厥常,懋昭周公之训,惟民其乂。我闻曰:‘至治馨香,感于神明。黍稷非馨,明德惟馨尔。’尚式时周公之猷训,惟日孜孜,无敢逸豫。凡人未见圣,若不克见;既见圣,亦不克由圣,尔其戒哉!尔惟风,下民惟草。图厥政,莫或不艰,有废有兴,出入自尔师虞,庶言同则绎。尔有嘉谋嘉猷,则入告尔后于內,尔乃顺之于外,曰:‘斯谋斯猷,惟我后之德。’呜呼!臣人咸若时,惟良显哉!”
成王这样说:“君陈!你有孝顺恭敬的美德。因为你孝顺父母,又友爱兄弟,就能够来处理政事。我命令你治理东郊成周,你要敬慎呀!从前周公做万民的师保,人民怀念他的美德。你前往,要慎重对待你的职务呀!遵循周公的常道,勉力宣扬周公的教导,人民就会安定。我听说:至治之世的馨香,感动神明;黍稷的香气,不是远闻的香气,明德才是远闻的香气。你要履行这一周公的教训,日日孜孜不倦,不要安逸享乐!凡人未见到圣道,好象不能见到一样;已经见到圣道,又不能遵行圣人的教导。你要戒惧呀!你是风,百姓是草,草随风而动啊!谋划殷民的政事,不要认为不难;有废除,有兴办,要反复同众人商讨,大家意见相同才能够施行。你有好的谋略跟言论,就要进入宫内告诉君王,你于是在外面顺从君主,并且说:‘这样的好谋,这样的好言,是我们君主的美德。’啊!臣下都象这样,就良好啊!”

王曰:“君陈,尔惟弘周公丕训,无依势作威,无倚法以削,宽而有制,从容以和。殷民在辟,予曰辟,尔惟勿辟;予曰宥,尔惟勿宥,惟厥中。有弗若于汝政,弗化于汝训,辟以止辟,乃辟。狃于奸宄,败常乱俗,三细不宥。尔无忿疾于顽,无求备于一夫。必有忍,其乃有济;有容,德乃大。简厥修,亦简其或不修。进厥良,以率其或不良。惟民生厚,因物有迁。违上所命,从厥攸好。尔克敬典在德,时乃罔不变。允升于大猷,惟予一人膺受多福,其尔之休,终有辞于永世。”
成王说:“君陈!你应当扬周公的大训!不要倚仗势力为非作歹,不要倚仗法律侵害百姓。要宽大而有法制,从容和谐。殷国的人民有陷入刑法的,我如果说处罚,你就不要处罚;我如果说赦免,你也不要赦免;要考虑是否符合刑罚。有人不顺从你的政事,不接受你的教导,如果处罚他能够制止他违法犯罪,才处罚。惯于做奸宄犯法的事,破坏法规,败坏风俗,这三项中的小罪,也不能宽恕。你不要忿恨愚钝无知的人,不要向一人求全责备;君王一定要有所忍耐,才能够事有所成;学会宽容,德才算是大。鉴别善良的,也鉴别有不善良的;进用那些贤良的人,来勉励那些有所不良的人。百姓纯良敦厚,又依外物而有改移;往往违背上级的教命,顺从上级的喜好。你能够敬重常法和省察自己的德行,这些人就不会不变。真的升到非常顺从的境地,我将享受大福,你也将被世人永远赞赏。”

\chapter{顾命}

成王将崩,命召公、毕公率诸侯相康王,作《顾命》。
成王即将病逝时,下令召公、毕公率领各诸侯辅佐康王,史官作《顾命》。

惟四月,哉生魄,王不怿。甲子,王乃洮颒水。相被冕服,凭玉几。乃同,召太保奭、芮伯、彤伯、毕公、卫侯、毛公、师氏、虎臣、百尹、御事。王曰:“呜呼!疾大渐,惟几,病日臻。既弥留,恐不获誓言嗣,兹予审训命汝。昔君文王、武王宣重光,奠丽陈教,则肄肄不违,用克达殷集大命。在后之侗,敬迓天威,嗣守文、武大训,无敢昏逾。今天降疾,殆弗兴弗悟。尔尚明时朕言,用敬保元子钊弘济于艰难,柔远能迩,安劝小大庶邦。思夫人自乱于威仪。尔无以钊冒贡于非几。”
四月,月亮新现光明,成王生了病。甲子这天,成王洗了头发洗了脸,太仆给王戴上王冠,披上朝服,王靠着玉几。于是会见朝臣。成王召见太保奭、芮伯、彤伯、毕公、卫侯、毛公、师氏、虎臣、百官的首长以及办事官员。王说:“啊!我的病很厉害,有危险,痛苦的日子到了。已经是临终时刻,恐怕不能郑重地讲后嗣的事了,现在,我详细地训告你们。过去,我们的先君文王、武王,放出日月般的光辉,制定法律,发布教令,臣民都努力奉行,不敢违背,因而能够讨伐殷商,成就我周国的大命。后来,幼稚的我,认真奉行天威,继续遵守文王、武王的伟大教导,不敢昏乱越轨。如今上天降下重病,几乎不能起床不能说话了。你们要勉力接受我的话,认真保护我的大儿子姬钊大渡艰难,要柔服远方,亲善近邻,安定、教导大小各国。我想众人要用礼法自治,你们不可使姬钊冒犯以至陷于非法啊!”

兹既受命,还出缀衣于庭。越翼日乙丑,王崩。太保命仲桓、南宫毛俾爰齐侯吕伋,以二干戈、虎贲百人逆子钊于南门之外。延入翼室,恤宅宗。丁卯,命作册度。
群臣已经接受教命,就退回来,拿出成王的朝服放在王庭。到了明天乙丑日,成王逝世了。太保命令仲桓和南宫毛跟从齐侯吕伋,二人分别拿着干戈,率领一百名勇士,在南门外迎接太子钊。请太子钊进入侧室,作忧居的主人,丁卯这天,命令作册制定丧礼。

越七日癸酉,伯相命士须材。狄设黼扆、缀衣。牖间南向,敷重篾席,黼纯,华玉,仍几。西序东向,敷重厎席,缀纯,文贝,仍几。东序西向,敷重丰席,画纯,雕玉,仍几。西夹南向,敷重笋席,玄纷纯,漆,仍几。越玉五重,陈宝,赤刀、大训、弘璧、琬琰、在西序。大玉、夷玉、天球、河图,在东序。胤之舞衣、大贝、鼖鼓,在西房;兑之戈、和之弓、垂之竹矢,在东房。大辂在宾阶面,缀辂在阼阶面,先辂在左塾之前,次辂在右塾之前。
到了第七天癸酉,召公命令官员布置各种器物。狄人陈设斧纹屏风和先王的礼服。门窗间朝南的位置,铺设着双层竹席,饰着黑白相间的丝织花边,陈设彩玉,用无饰的几案。在西墙朝东的位置,铺设双层细竹篾席,饰着彩色的花边,陈设花贝壳,用无饰的几案。在东墙朝西的位置,铺设双层莞席,饰着绘有云气的花边,陈设雕刻的玉器,用无饰的几案。在堂的西边夹室朝南的位置,铺设双层青竹蔑席,饰着黑丝绳连缀的花边,陈设漆器,用无饰的几案。越玉五种,宝刀、赤刀、大训,大璧、琬、琰,陈列在西墙向东的席前。大玉、夷玉、天球、河图,陈列在东墙向西的席前。胤制作的舞衣、大贝壳、大军鼓,陈列在西房。兑制作的戈、和制作的弓、垂制作的竹矢,陈列在东房。王的玉车放置在宾客们所走的台阶前,金车放置在主人走的台阶前,象车放在门左侧堂屋的前面,木车放在门右侧堂屋的前面。

二人雀弁,执惠,立于毕门之内。四人綦弁,执戈上刃,夹两阶戺。一人冕,执刘,立于东堂,一人冕,执钺,立于西堂。一人冕,执戣,立于东垂。一人冕,执瞿,立于西垂。一人冕,执锐,立于侧阶。
二人戴着赤黑色的礼帽,执三角矛,站在祖庙门里边。四人戴着青黑色的礼帽,执着戈,戈刃向前,夹着台阶,对面站在台阶两旁。一人戴着礼帽,拿着大斧,站立在东堂的前面。一人戴着礼帽,拿着大斧,站立在西堂的前面。一人戴着礼帽,拿着三锋矛,站立在东堂外边。一人戴着礼帽,拿着三锋矛,站立在西堂外边。还有一人戴着礼帽,拿着矛,站立在北堂北面的台阶上。

王麻冕黼裳,由宾阶隮。卿士邦君麻冕蚁裳,入即位。太保、太史、太宗皆麻冕彤裳。太保承介圭,上宗奉同瑁,由阼阶隮。太史秉书,由宾阶隮,御王册命。曰:“皇后凭玉几,道扬末命,命汝嗣训,临君周邦,率循大卞,燮和天下,用答扬文、武之光训。”王再拜,兴,答曰:“眇眇予末小子,其能而乱四方以敬忌天威。”乃受同瑁,王三宿,三祭,三吒。上宗曰:“飨!”太保受同,降,盥,以异同秉璋以酢。授宗人同,拜。王答拜。太保受同,祭,哜,宅,授宗人同,拜。王答拜。太保降,收。诸侯出庙门俟。
王戴着麻制的礼帽,穿着绣有斧形花纹的礼服,从西阶上来。卿士和众诸侯戴着麻制的礼帽,穿着黑色礼服,进入中庭,各人站在规定的位置上。太保、太史、太宗都戴着麻制的礼帽,穿着红色礼服。太保捧着大圭,太宗捧着酒杯和瑁,从东阶上来。太史拿着策书,从西阶走上来,进献策书给康王。太史说:“大王靠着玉几,宣布他临终的教命,命令您继承文王、武王的大训,治理领导周国,遵守大法,协和天下,以宣扬文王、武王的光明教训。”王再拜,然后起来,回答说:“我这个微末的小子,怎么能协和治理天下以敬畏天威啊?”王接受了酒杯和瑁。前进三次,祭酒三次,奠酒三次。太宗说:“请喝酒!”王喝酒后,太保接过酒杯,走下堂,洗手,又登上堂,用另外一种酒杯自斟自饮作答,然后把酒杯交给宗人,对王下拜,王也回拜。太保又从宗人那里接过酒杯,祭酒,尝酒,奠酒,然后把酒杯交给宗人,又拜。王又回拜。太保走下堂,行礼结束,诸侯卿士们都走出祖庙门,恭候康王视朝。

\chapter{康王之诰}

康王既尸天子,遂诰诸侯,作《康王之诰》。
康王既祭神主天子,于是诰命诸侯,作《康王之诰》。

王出,在应门之内,太保率西方诸侯入应门左,毕公率东方诸侯入应门右,皆布乘黄朱。宾称奉圭兼币,曰:“一二臣卫,敢执壤奠。”皆再拜稽首。王义嗣,德答拜。太保暨芮伯咸进,相揖。皆再拜稽首曰:“敢敬告天子,皇天改大邦殷之命,惟周文武诞受羑若,克恤西土。惟新陟王毕协赏罚,戡定厥功,用敷遗后人休。今王敬之哉!张惶六师,无坏我高祖寡命。”
王走出祖庙,来到应门内。太保召公率领西方的诸侯进入应门,站立在左侧,毕公率领东方的诸侯进入应门,站立在右侧,他们都穿着绣有花纹的礼服和黄朱色的韨。赞礼的官员传呼进献命圭和贡物,诸侯走上前,说:“我们这些四方的守护之臣,各自将封底上的土产献给君王。”诸侯都再拜叩头。王一一答谢。太保召公和芮伯同走向前,互相作揖后,再一同向王跪拜叩头。他们说:“恭敬地禀告天子,上天已经更改了大国殷的命运,我们周国的文王、武王大受福祥,能够安定西方。刚刚升入天庭的成王,赏罚完全合宜,能够成就文、武的功业,因此把幸福留给了后代子孙。如今我王要敬慎啊!要加强王朝的六军,不要败坏我们高祖的大命。”

王若曰:“庶邦侯、甸、男、卫,惟予一人钊报诰。昔君文武丕平,富不务咎,厎至齐信,用昭明于天下。则亦有熊罴之士,不二心之臣,保乂王家,用端命于上帝。皇天用训厥道,付畀四方。乃命建侯树屏,在我后之人。今予一二伯父尚胥暨顾,绥尔先公之臣服于先王。虽尔身在外,乃心罔不在王室,用奉恤厥若,无遗鞠子羞!”
王这样说:“侯、甸、男、卫的各位诸侯!现在我姬钊答复你们的教导。先君文王、武王很公平,仁厚慈爱,不滥施刑罚,致力实行中信,因而他们的光辉普照天下。还有像熊罴一样勇武的将士和忠心耿耿的臣子,一起安定治理我们的国家,因此,我们才被上天委以重任。上天顺从先王的治理之道,把天下交给先王治理。先王于是命令分封诸侯,树立蕃卫,眷顾我们后代子孙。现在,我们几位伯父希望你们互相爱护顾念王室,继续如你们的祖先臣服于先王。虽然你们身在朝廷之外,你们的心不可不在王室,要辅助我得到吉祥,不要把羞辱留给我!”

群公既皆听命,相楫,趋出。王释冕,反丧服。
众位大臣听完康王的诰命之后,互相作揖,快步走出退下。康王脱去吉服,返回侧室继续守丧,重新穿上丧服。

\chapter{毕命}

康王命作册毕,分居里,成周郊,作《毕命》。
周康王命令作册书,册命毕公治理成周,分别殷民善恶,区别居里疆界,安定周王都的郊区,史官写了《毕命》。

惟十有二年,六月庚午,朏。越三日壬申,王朝步自宗周,至于丰。以成周之众,命毕公保厘东郊。
康王十二年六月庚午日,新月初明。三天后,壬申日早晨,周康王离开镐京,到达丰邑,把成周的民众交给太史毕公管理,命其安治于东郊。

王若曰:“呜呼!父师,惟文王、武王敷大德于天下,用克受殷命。惟周公左右先王,绥定厥家,毖殷顽民,迁于洛邑,密迩王室,式化厥训。既历三纪,世变风移,四方无虞,予一人以宁,道有升降,政由俗革,不臧厥臧,民罔攸劝。惟公懋德,克勤小物,弼亮四世,正色率下,罔不祗师言。嘉绩多于先王,予小子垂拱仰成。”
周康王这样说:“啊!父师(毕公)。唯有我先祖文王、武王施德于天下,故而能得到殷商的王命。周公辅助先王安邦定国,管束殷商顽民,将其迁徙到洛邑,靠近王都便于监督管理。他们被逐渐周公的感化。从那时到现在,已经过了三十六年。移风易俗,世事无常。今四方安定无忧,我甚感欣喜。世道有好有坏,政教也随应时而变。如果不能褒奖善良,树立起以善为美的榜样,百姓将无向善之心。毕公您德高望重,不但能将大小事务处理的妥妥当当,而且还先后辅助过四代天子,统率群臣下属领导有方,臣下无人不敬重您,都重视您的教导。你的丰功伟绩为先王所倚重,视为左膀右臂。我小子才疏学浅,比不上先王,对您是敬仰万分,仰望您的功绩。”

王曰:“呜呼!父师,今予祗命公以周公之事,往哉!旌别淑慝,表厥宅里,彰善瘅恶,树之风声。弗率训典,殊厥井疆,俾克畏慕。申画郊圻,慎固封守,以康四海。政贵有恒,辞尚体要,不惟好异。商俗靡靡,利口惟贤,余风未殄,公其念哉!我闻曰:‘世禄之家,鲜克由礼’。以荡陵德,实悖天道。敝化奢丽,万世同流。兹殷庶士,席宠惟旧,怙侈灭义,服美于人。骄淫矜侉,将由恶终。虽收放心,闲之惟艰。资富能训,惟以永年。惟德惟义,时乃大训。不由古训,于何其训。”
康王说:“啊!父师(毕公)。今日,我把周公治理殷民的重任委托给您,您现在就去上任吧!您要识别善恶之人,对善民要加以表彰,让他荣耀乡里。奖善罚恶,树立以善为美的良好风气。对于顽固不化的人,将他们和善民隔离开,让他们住在那里继续教化,使他们懂得善恶,服从管束。你还要明确的划分出郊区与城市的分界,大力加强军事力量,从而安定天下。为政者当重视前人定下的常法,发布的政令应当突出重点,不要标新立异。殷商遗民奢侈之风甚胜,以善辩为贤。虽然经多年的整治,收效甚微,时至今日此歪风仍未断绝。您可得想想办法啊!我听说:‘世代为官享有禄位的人家多为二世祖,很少有人能循规蹈矩。他们放荡不羁,仗势欺人,欺辱有德之人,实在是有违天地正道。这种败坏的风气,世代相同。’殷商的士族们,享受先人的福泽太久了,已经堕落了。他们凭仗强大的势力,灭绝德义,穿着奢侈无度,而且骄横放荡,目中无人,无人管束则行恶一生。这些人已经无药可救,即使加以惩戒也只能他们收敛一时,但是很难让他们改过自新,重新做人。对于有钱有势又能接受我朝管束的人,自当让其福寿绵绵。重视德,重视义,这是天下的大训;如果连这个古训都不听,那么他们还会听什么呢?”

王曰:“呜呼!父师,邦之安危,惟兹殷士。不刚不柔,厥德允修。惟周公克慎厥始,惟君陈克和厥中,惟公克成厥终。三后协心,同厎于道,道洽政治,泽润生民,四夷左衽,罔不咸赖,予小子永膺多福。公其惟时成周,建无穷之基,亦有无穷之闻。子孙训其成式,惟乂。呜呼!罔曰弗克,惟既厥心;罔曰民寡,惟慎厥事。钦若先王成烈,以休于前政。”
周康王说:“啊!父师(毕公)。教化殷民责任重大,关乎国家安危,不可不慎重。施政当刚柔相济,有赏有罚,如此方能政令通达。当初,周公谨慎的教化殷民;接着,周公之子君陈和谐治理殷民,使其与我朝和睦相处;如今,就要靠您毕公完成这教化的最终使命。三位齐心协力,先后教导殷民,治理殷民,政治清明,如春风化雨,润泽百姓。四方的少数民族,也受到您的福泽,我这个年轻人也托您的福,永远享受大福。您要理好成周殷民事宜,建立我周王朝万世基业。功成则永享美名,流芳百世。后世子孙遵从您毕公制定的治国方略,天下就该安定了。啊!您不要谦虚的说,不能胜任此重任,应当尽心尽力的去做;不要说百姓少,当慎重政事。认真的治理好先王的大业,要超越前人,使它更加美好!”

\chapter{君牙}

穆王命君牙,为周大司徒,作《君牙》。
周穆王下令君牙任大司徒,史官作《君牙》。

王若曰:“呜呼!君牙,惟乃祖乃父,世笃忠贞,服劳王家,厥有成绩,纪于太常。惟予小子嗣守文、武、成、康遗绪,亦惟先正之臣,克左右乱四方。心之忧危,若蹈虎尾,涉于春冰。今命尔予翼,作股肱心膂,缵乃旧服。无忝祖考,弘敷五典,式和民则。尔身克正,罔敢弗正,民心罔中,惟尔之中。夏暑雨,小民惟曰怨咨:冬祁寒,小民亦惟曰怨咨。厥惟艰哉!思其艰以图其易,民乃宁。呜呼!丕显哉,文王谟!丕承哉,武王烈!启佑我后人,咸以正罔缺。尔惟敬明乃训,用奉若于先王,对扬文、武之光命,追配于前人。”
穆王这样说:“啊!君牙。你的祖父和你的父亲,世世纯厚忠正;尽心尽力地辅助周王室,很有成绩,已经记录在画有日月的旗子上。年幼的我将继承守文、武、成、康的遗业,也想让先王的臣子能够辅助我治理四方。任大才弱,我心里的忧虑危惧,就像踩着虎尾和走着春天的冰。现在我命令你辅助我,作我的心腹重臣。要继续你旧日的行事,不要累及你的祖考!广泛宣扬五典之教,将和谐友善作为人民的准则。只要你自身能正,人民就不敢不正;民心没有标准,只考虑你的标准。夏天炎热多雨,百姓只是怨恨嗟叹;冬天严寒,百姓也只是怨恨嗟叹。治民艰难呀!你要想到他们的艰难,因而谋求那些治理的办法,人民才会安宁。啊!光明呀!我们文王的谋略;相承呀!我们武王的功业。它可以启示佑助我们后人,使我们都依从正道而无邪缺。你当不懈地宣扬你的教训,以此恭顺于先王。你当报答并发扬文王、武王善良光辉的天命,以德政之美去追随和超越您的前人。”

王若曰:“君牙,乃惟由先正旧典时式,民之治乱在兹。率乃祖考之攸行,昭乃辟之有乂。”
穆王这样说:“君牙!你当奉行先正的旧典善法,人民治乱的关键,就在这里。你应当遵循你祖父和父亲的行为,让君王的治世之功可以彰显于天下。”

\chapter{冏命}

王若曰:“伯冏!惟予弗克于德,嗣先人宅丕后。怵惕惟厉,中夜以兴,思免厥愆。昔在文、武,聪明齐圣,小大之臣,咸怀忠良。其侍御朴从,罔匪正人,以旦夕承弼厥辟,出入起居,罔有不钦,发号施令,罔有不臧,下民祗若,万邦咸休。惟予一人无良,实赖左右前后有位之士,匡其不及。绳愆纠谬,格其非心,俾克绍先烈。今予命汝作大正,正于群仆侍御之臣,懋乃后德,交修不逮;慎简乃僚,无以巧言令色,便辟侧媚,其惟吉士。仆臣正,厥后克正,仆臣谀,厥后自圣;后德惟臣,不德惟臣。尔无昵于憸人,充耳目之官,迪上以非先王之典;非人其吉,惟货其吉;若时瘝厥官;惟尔大弗克只厥辟,惟予汝辜。”
穆王这样说:“伯冏!我不优于道德。继承先人处在大君的位置,戒惧会有危险,甚至半夜起来,想法子避免过失。从前在文王、武王的时候,他们聪明、通达、圣明,小臣大臣都怀着忠良之心。他们的侍御近臣,没有人不是正人,用他们早晚侍奉辅佐他们的君主,所以君主出入起居,没有不敬慎的事;发号施令,也没有不好的。百姓敬重顺从君主的命令,天下万国也都喜欢。我没有好的德行,实在要依赖左右前后的官员,匡正我的不到之处。纠正过错,端正我不正确的思想,使我能够继承先王的功业。今天我任命你作太仆长,领导群仆、侍御的臣子。你们要勉励你们的君主增修德行,共同医治我不够的地方。你要慎重选择你的部属,不要任用巧言令色、阿谀奉承的人,要都是贤良正士。仆侍近臣都正,他们的君主才能正;仆待近臣谄媚,他们的君主就会自以为圣明。君主有德,由于臣下,君主失德,也由于臣下。你不要亲近小人,充当我的视听之官,不要引导君上违背先王之法。如果不以贤人最善,只以货财最善,象这样,就会败坏我们的官职,就是你大大地不能敬重你的君主;我将惩罚你。”

王曰:“呜呼!钦哉!永弼乃后于彝宪。”
穆王说:“啊!要认真呀!要长久用常法辅助你的君主。”

\chapter{吕刑}

吕命穆王训夏赎刑,作《吕刑》。
周穆王命吕侯依照夏朝的赎刑作《吕刑》

惟吕命,王享国百年,耄荒,度作刑,以诘四方。王曰:“若古有训,蚩尤惟始作乱,延及于平民,罔不寇贼,鸱义,奸宄,夺攘,矫虔。苗民弗用灵,制以刑,惟作五虐之刑曰法。杀戮无辜,爰始淫为劓、刵、椓、黥。越兹丽刑并制,罔差有辞。民兴胥渐,泯泯棼棼,罔中于信,以覆诅盟。虐威庶戮,方告无辜于上。上帝监民,罔有馨香德,刑发闻惟腥。皇帝哀矜庶戮之不辜,报虐以威,遏绝苗民,无世在下。乃命重、黎,绝地天通,罔有降格。群后之逮在下,明明棐常,鳏寡无盖。
吕侯被命为卿时,穆王在位已经很长时间了,他年纪已经很大了。但还是广泛谋求制定刑法,来制约四方诸侯。王说:“古代有遗训,当时蚩尤开始作乱,波及到平民百姓。无不寇掠贼害,内外作乱,争夺窃盗,诈骗强取。苗民不遵守政令,于是就制定刑罚来制服,制定了五种酷刑作为法律。慢慢就开始杀害无罪的人,开始滥用劓、刖、椓、黥等刑罚。于是,施行杀戮,抛弃法制,不减免无罪的人。苗民互相欺诈,社会变得混乱不堪,没有公平正义,以致背叛誓约。许多遭受冤屈的无辜百姓向上天申诉,上天了解实情后,发现没有芬芳的德政,刑法所发散的只有腥气。上天怜惜那些无辜受罚的平民百姓,就用威罚处置施行虐刑的人,制止和消灭行虐的苗民,使他们没有后嗣留在世间。又命令重和黎,禁止地民和天神相互感通,神和民再不能升降来往了。高辛、尧、舜相继在下,都显用贤德的人,扶持常道,于是孤苦之人的苦情,没有壅蔽了。

皇帝清问下民鳏寡有辞于苗。德威惟畏,德明惟明。乃命三后,恤功于民。伯夷降典,折民惟刑;禹平水土,主名山川;稷降播种,家殖嘉谷。三后成功,惟殷于民。士制百姓于刑之中,以教祗德。穆穆在上,明明在下,灼于四方,罔不惟德之勤,故乃明于刑之中,率乂于民棐彝。典狱非讫于威,惟讫于富。敬忌,罔有择言在身。惟克天德,自作元命,配享在下。”
尧皇帝清楚地听到下民和孤寡对苗民的怨言。于是提拔贤人,贤人所惩罚的,人都畏服,贤人所尊重的,人们没有不尊重的。命令三位大臣慎重地为民服务。伯夷制定了法律,用刑罚来管理百姓;大禹平治水土,负责名山大川;后稷教导百姓播种,尽心竭力帮助百姓种植庄稼。三位国君功业达成后,就富足了老百姓。士师又用公正的刑罚制御百官,教导臣民敬重德行。尧皇帝恭敬在上,群臣在下也致力于明察建立功业,政治清明,光辉普照天下,没有人不努力遵守美好的德行。所以公平的施用刑罚,治理老百姓以扶持常道。主管刑罚的官,不能只树立自己的威严,而是要仁厚,造福万民。又敬、又戒,自身不说坏话。他们肩负上天仁爱的美德,自己造就了好命,所以配天在下享有禄位。”

王曰:“嗟!四方司政典狱,非尔惟作天牧?今尔何监?非时伯夷播刑之迪?其今尔何惩?惟时苗民匪察于狱之丽,罔择吉人,观于五刑之中;惟时庶威夺货,断制五刑,以乱无辜,上帝不蠲,降咎于苗,苗民无辞于罚,乃绝厥世。”
王说:“啊!四方的诸侯们,你们身上不是肩负着治理万民的重任吗?现在,你们重视什么呢?难道不是这伯夷施行刑罚的道理吗?现在你们要用什么作为惩戒呢?就是苗民不详察狱事的施行,不选择善良的人,监察五刑的公正,就是任用虚张威势,掠夺财物的人,裁决五刑,乱罚无罪,上天不加赦免,降灾给苗民,苗民对上天的惩罚无话可说,于是断绝了他们的后嗣。”

王曰:“呜呼!念之哉。伯父、伯兄、仲叔、季弟、幼子、童孙,皆听朕言,庶有格命。今尔罔不由慰曰勤,尔罔或戒不勤。天齐于民,俾我一日,非终惟终,在人。尔尚敬逆天命,以奉我一人!虽畏勿畏,虽休勿休。惟敬五刑,以成三德。一人有庆,兆民赖之,其宁惟永。”
王说:“啊!你们要记住这个教训啊!伯父、伯兄、仲叔、季弟以及年幼的子孙们,都听从我的话,这样就能享受好的天命了。如今你们没有人不喜欢慰劳说勤劳了,你们没有人制止自己的懒惰。上天治理下民,暂时任用我们,不成与成,完全在人。你们可要恭敬地接受天命,来辅助我!虽然遇到可怕的事,不要害怕;虽然可以休息,也不要休息,希望慎用五刑,养成这三种德行。一人办了好事,万民都受益,国家的安宁就会长久了。

王曰:“吁!来,有邦有土,告尔祥刑。在今尔安百姓,何择,非人?何敬,非刑?何度,非及?两造具备,师听五辞。五辞简孚,正于五刑。五刑不简,天于五罚;五罚不服,正于五过。五过之疵:惟官,惟反,惟内,惟货,惟来。其罪惟均,其审克之!
王说:“啊!来,诸侯国君和各位大臣,我将少用刑罚、注重德教的详细制度告知给你们。如今你们安定百姓,要选择什么呢,难道不应该是德才兼备的贤人吗?要慎重什么呢,难道不是刑罚吗?要考虑什么呢,难道不是公平的审理案件吗?原告和被告都在场了,官员就审查五刑的讼辞;如果讼辞核实可信,就用五刑来处理。如果用五刑处理不能核实,就用五罚来处理;如果用五罚处理也不可从,就用五过来处理。五过的弊端是:法官畏权势,报恩怨,谄媚内亲,索取贿赂,受人请求。发现上述弊端,法官的罪就与罪犯相同,你们必须详细察实啊!

五刑之疑有赦,五罚之疑有赦,其审克之!简孚有众,惟貌有稽。无简不听,具严天威。墨辟疑赦,其罚百锾,阅实其罪。劓辟疑赦,其罪惟倍,阅实其罪。剕辟疑赦,其罚倍差,阅实其罪。宫辟疑赦,其罚六百锾,阅实其罪。大辟疑赦,其罚千锾,阅实其罪。墨罚之属千。劓罚之属千,剕罚之属五百,宫罚之属三百,大辟之罚其属二百。五刑之属三千。
如果五刑判决存疑,可以直接赦免;同样地,如果发现五罚存疑,也能够赦免。这些都要详细地进行审核。罪行经过再三审核,有多人作证,还要对细节的地方进行详细察验,这样才能够判定刑罚。如果案情无法核实,则不必受理。刑狱之事要谨慎审核,那是因为要畏惧上天的威严,必须要能够谨慎恭敬地对待。。判处墨刑感到可疑,可以从轻处治,罚金一百锾,要核实其罪行。判处劓刑感到可疑,可以从轻处治,罚金二百锾,要核实其罪行。判处剕刑感到可疑,可以从轻处治,罚金五百锾,要核实其罪行。判处宫刑感到可疑,可以从轻处治,罚金六百锾,要核实其罪行。判处死刑感到可疑,可以从轻处治,罚金一千锾,要核实其罪行。墨罚的条目有一千,劓罚的条目有一千,荆罚的条目有五百,宫罚的条目有三百,死罪的刑罚,其条目有二百。五种刑罚的条目共有三千。

上下比罪,无僭乱辞,勿用不行,惟察惟法,其审克之!上刑适轻,下服;下刑适重,上服。轻重诸罚有权。刑罚世轻世重,惟齐非齐,有伦有要。罚惩非死,人极于病。非佞折狱,惟良折狱,罔非在中。察辞于差,非从惟从。哀敬折狱,明启刑书胥占,咸庶中正。其刑其罚,其审克之。狱成而孚,输而孚。其刑上备,有并两刑。”
要上下比较其罪行,不要错乱供辞,不要采取已经废除的法律,应当认真察看案情遵循刑罚,并且要认真审核!如果犯了重罪,上刑宜于减轻,就减一等处治,犯的罪行较轻,但是情节恶劣最好可以从重论处的,要用重刑来处罚。各种刑罚的轻重允许有些灵活性。刑罚时轻时重,相同或不相同,要根据实际情况来定。刑罚虽不置人死地,但受刑罚的人感到比重病还痛苦。审判案件不能只靠着巧言善辩,要能够做到善良公正,才能让判决准确无误。从矛盾处考察供词,不服从的犯人也会服从。应当怀着哀怜的心情判决诉讼案件,明白地检查刑书,互相斟酌,都要以公正为标准。当刑当罚,要详细察实啊!要做到案件判定了,人们信服;改变判决,人们也信服。刑罚贵在慎重,有时也可以把两种罪行合并考虑,只罚一种。”

王曰:“呜呼!敬之哉!官伯族姓,朕言多惧。朕敬于刑,有德惟刑。今天相民,作配在下。明清于单辞,民之乱,罔不中听狱之两辞,无或私家于狱之两辞!狱货非宝,惟府辜功,报以庶尤。永畏惟罚,非天不中,惟人在命。天罚不极,庶民罔有令政在于天下。”
王说:“啊,要谨慎地对待邢狱啊!诸侯国君以及同姓官员们,对我的话要多多畏惧借鉴。我严谨地对待刑狱处罚之事,实行德政就要善用刑法。如今上天对民众的治理,在人间会确立君主来承接天意,在审理办案的时候要能够明察秋毫而不是听取一面之词偏听偏信。老百姓的治理,无不在于公正地审理双方的诉讼词,不要对诉讼双方的诉词贪图私利啊!狱讼接受贿赂不是好事,那是获罪的事,我将以众人犯罪来论处这些人。永远可畏的是上天的惩罚,不是天道不公平,只是人们自己终结天命。上天的惩罚不加到贪赃枉法的官员身上,天下的百姓就无法享受好的政治。”

王曰:“呜呼!嗣孙,今往何监,非德?于民之中,尚明听之哉!哲人惟刑,无疆之辞,属于五极,咸中有庆。受王嘉师,监于兹祥刑。”
王说:“啊!继承天子之位的子孙们,从今以后,你们应该怎么来审理案件呢?难道不是行德吗?对于老百姓案情的判决,要明察啊!治理老百姓要运用刑罚,使无穷无尽的讼辞合于五刑,都能公正适当,就有福庆。你们接受治理我的好百姓,可要明察这种祥刑啊!”

\chapter{文侯之命}

平王锡晋文侯秬鬯、圭瓒,作《文侯之命》。
周平王赐给晋文侯酒与酒具,写了《文侯之命》。

王若曰:“父义和!丕显文、武,克慎明德,昭升于上,敷闻在下;惟时上帝,集厥命于文王。亦惟先正克左右昭事厥辟,越小大谋猷罔不率从,肆先祖怀在位。呜呼!闵予小子嗣,造天丕愆。殄资泽于下民,侵戎我国家纯。即我御事,罔或耆寿俊在厥服,予则罔克。曰惟祖惟父,其伊恤朕躬!呜呼!有绩予一人永绥在位。父义和!汝克绍乃显祖,汝肇刑文、武,用会绍乃辟,追孝于前文人。汝多修,扞我于艰,若汝,予嘉。”
王这样说:“族父义和啊!伟大光明的文王和武王,能够慎重行德,他们的圣德能够彰显升入上天,声望广布在臣民之中。于是上天降下天命给文王、武王。也因为先前的公卿大夫能够辅佐、指导、服事他们的君主,对于君主的大小谋略无不遵从,所以先祖能够安然在位。啊!不幸的是等到我继承王位的时候,上天降下了责罚。导致没有福利德泽施给老百姓,遭受外寇的侵扰。现在我的治事大臣,没有老成人长期在职,我便不能胜任了。我呼吁:‘祖辈和父辈的诸侯国君,要替我担忧啊!’啊哈!果然有促成我长安在王位的人了。族父义和啊!您能够继承您的显祖唐叔,您努力制御文武百官,用会合诸侯的方式延续了您的君主,追怀效法文王和武王。您战功赫赫,在困难的时候保护了我,像您这样,我就要赞美!”

王曰:“父义和!其归视尔师,宁尔邦。用赉尔秬鬯一卣,彤弓一,彤矢百,卢弓一,卢矢百,马四匹。父往哉!柔远能迩,惠康小民,无荒宁。简恤尔都,用成尔显德。”
王说:“族父义和啊!要回去好好治理您的臣民,安定您的邦国。现在我赐给您黑黍香酒一卣;红色的弓一张,红色的箭一百支;黑色的弓一张,黑色的箭一百支;四匹马。您回去吧!安抚远方,亲善近邻,爱护安定老百姓,不要荒废政事,贪图安逸。大力安定您的国家,以成就您显著的德行。”

\chapter{费誓}

鲁侯伯禽宅曲阜,徐、夷并兴,东郊不开。作《费誓》。
鲁侯伯侵住在曲阜,徐戎、淮夷一同叛乱,鲁国的东郊都不得安宁了。鲁侯将要征伐,作了《费誓》。

公曰:“嗟!人无哗,听命。徂兹淮夷、徐戎并兴。善敹乃甲胄,敿乃干,无敢不吊!备乃弓矢,锻乃戈矛,砺乃锋刃,无敢不善!今惟淫舍牿牛马,杜乃擭,敜乃穽,无敢伤牿。牿之伤,汝则有常刑!马牛其风,臣妾逋逃,勿敢越逐,祗复之,我商赉汝。乃越逐不复,汝则有常刑!无敢寇攘,逾垣墙,窃马牛,诱臣妾,汝则有常刑!
公说:“喂!大家不要喧哗,听我的命令。现如今淮夷、徐戎同时发生了叛乱。赶快缝制好你们的军服头盔,系好你们的盾牌,不能不准备好!准备好你们的弓箭,锻造好你们的戈矛,磨利你们的锋刃,不能不追备好!如今要将牛马从脚铐桎梏中解放出来,将不妥的计划丢掉,填上捕捉野兽的陷阱,让牛马不受到伤害。伤害了牛马,你们就要受到常刑!牛马走失了,男女奴仆逃跑了,不许离开队伍去追赶!得到了的,要恭敬送还原主,我会赏赐你们。如果你们擅自离开队伍去追赶,或者不归还原主,你们就要受到常刑!不许抢夺掠取,跨过围墙,偷窃马牛,骗其他的奴隶逃脱,如果做了这些,就一定会收到刑罚的处罚!

甲戌,我惟征徐戎。峙乃糗粮,无敢不逮;汝则有大刑!鲁人三郊三遂,峙乃桢干。甲戌,我惟筑,无敢不供;汝则有无馀刑,非杀。鲁人三郊三遂,峙乃刍茭,无敢不多;汝则有大刑!”
甲戌这天,我们要去讨伐徐戎。准备好干粮,如果达不到标准,你们将会被处以死刑!我们鲁国各地的百姓,要准备好你们的筑墙工具。甲戌这天,我们要修筑好营垒,不准不准备好;如果不准备,你们将受到终身不释放的刑罚,只是不杀头。我们鲁国的百姓,要准备你们的生草料和干草料,不许不够;如果不够,你们就要受到死刑!”

\chapter{秦誓}

\begin{yuanwen}
公曰:“嗟!我士,听无哗!予誓告汝群言之首。古人有言曰:‘民讫自若,是多盘。’责人斯无难,惟受责俾如流,是惟艰哉!我心之忧,日月逾迈,若弗云来。惟古之谋人,则曰未就予忌;惟今之谋人,姑将以为亲。虽则云然,尚猷询兹黄发,则罔所愆。” 番番良士,旅力既愆,我尚有之;仡仡勇夫,射御不违,我尚不欲。惟截截善谝言,俾君子易辞,我皇多有之! 昧昧我思之,如有一介臣,断断猗无他技,其心休休焉,其如有容。人之有技,若己有之。人之彦圣,其心好之,不啻若自其口出。是能容之,以保我子孙黎民,亦职有利哉!人之有技,冒疾以恶之;人之彦圣而违之,俾不达是不能容,以不能保我子孙黎民,亦曰殆哉! 邦之杌陧,曰由一人;邦之荣怀,亦尚一人之庆。”
\end{yuanwen}

穆公说:“啊!众大臣将士们,听着,不要喧哗!我有重要的话告诉你们。古人有话说:‘人总是贪图安逸,责难他人并不困难,但是要做到让自己受责备还能够从善如流,这就困难了。’责备别人不是难事,受到别人责备,听从它如流水一样地顺畅,这就困难啊!我心里的忧虑,在于时间过去,就不回来了。以前的谋臣,却说‘不能顺从我的教导’;现在的谋臣,秉承着我的意志,让我亲近了他们。虽说这样,但是军国一类的大事还是应该请教德高望重的老臣的想法,才不会失误。那些白发苍苍的老臣,体力已经衰了,我还是信任他们。强壮勇猛的武士,射箭和驾车都不错,我还是不能够立即任用。而对于那些善于花言巧语,容易让在位的良臣受到蛊惑变得懈怠的,我也没有时间理会!我暗暗思量着,如果有一个大臣,诚实专一而没有别的技能,他的胸怀宽广而能容人。看到别人有能力,就好像自己拥有一样那般高兴;别人美好明哲,他的心里的欢喜,又超过了他口头的称道。这样能够容人,用来保护我的子孙众民,也或许有利啊!还有一种人,别人有能力,就妒忌,就厌恶。别人美好明哲,就想方设法阻拦他让他不能通向君主。这样不能宽容人,用来也不能保护我的子孙众民,也很危险啊!国家的危险不安,由于一人;国家的繁荣安定,也许是由于一个人的贤能而造就的啊!”

\end{document}