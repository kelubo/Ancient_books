% 旧唐书
% 旧唐书.tex

\documentclass[12pt,UTF8]{ctexbook}

% 设置纸张信息。
\usepackage[a4paper,twoside]{geometry}
\geometry{
	left=25mm,
	right=25mm,
	bottom=25.4mm,
	bindingoffset=10mm
}

% 设置字体,并解决显示难检字问题。
\xeCJKsetup{AutoFallBack=true}
\setCJKmainfont{SimSun}[BoldFont=SimHei, ItalicFont=KaiTi, FallBack=SimSun-ExtB]

% 目录 chapter 级别加点(.)。
\usepackage{titletoc}
\titlecontents{chapter}[0pt]{\vspace{3mm}\bf\addvspace{2pt}\filright}{\contentspush{\thecontentslabel\hspace{0.8em}}}{}{\titlerule*[8pt]{.}\contentspage}

% 设置 part 和 chapter 标题格式。
\ctexset{
	part/name= {卷,},
	part/number={\chinese{part}},
	chapter/name={},
	chapter/number={}
}

% 设置古文原文格式。
\newenvironment{yuanwen}{\bfseries\zihao{4}}

% 设置署名格式。
\newenvironment{shuming}{\hfill\bfseries\zihao{4}}

% 注脚每页重新编号,避免编号过大。
\usepackage[perpage]{footmisc}

\title{\heiti\zihao{0} 旧唐书}
\author{刘昫}
\date{后晋}

\begin{document}
	
\maketitle
\tableofcontents
	
\frontmatter
\chapter{重刻旧唐书序}
	
	李唐氏有天下三百年,三代而降,英君明辟若唐文皇,功德固在首列。厥后子孙迭兴,虽中更丧乱,犹不失为盛朝。而玄、宪二宗至配贞观,与汉七庙同称,何也?其典章法度贻谋之善,不可及已。盖作唐史者有三人焉:吴兢、韦述、令孤峘,此皆金闺上彦,操笔石渠,而未竟一代。至石晋朝,始敕中书刘等因峘旧文,增为百九十卷,然后有唐事迹悉载无遗,而撰述详赡,妙极模写,足以上追史汉,下包魏陈,信乎史之良者,无以加于是矣。奈何宋之庆历,又出新编,大有增损,至使读者不复得睹唐朝一诏令。
	历年五百,旧书湮灭,君子不能无病诸。皇上右文弘道,化被四远,由是缙绅士夫,咸以修缉典坟为己任。此书故有刻本在吴中,惜亦未全。先任提学侍御北江闻人公闻之,慨然欲寿诸梓,与菁莪共择可托者,得苏学司训沈君,有问学干局,良儒师也。因授之,俾董厥事,且命广搜残逸,足其卷数。及募士出赀,佐经费,君鸠工堂西大舍中,无啻三十手,硃墨雠校,不舍昼夜。成未及半,而北江公以忧去,以赀不绍,白之巡抚大中丞右江欧阳公,公命掌郡事别驾钟侯助其役。未几,府主王侯至任许相以完大巡侍御西
	郭陈公尤加赞相,乐书之成,而其事则总于今任提学侍御午山冯公焉。盖学政之台,书之所由起也。工将毕,士子袁贞辈相率扣子,请先序诸后。子惟三古圣人作为经书,人极立矣,十九朝史官述为史书,往事鉴矣,去圣既远,后儒蠹经,经不可蠹,犹云翳白日,日行空自如也,吏又可以新掩旧哉!且文章之作,率视共区宇之全缺。巨唐疆域幅员万里,其广大与轩后等,是以词华蔚茂有至光焰万丈者,郎舍相踵,既出螭坳亲见,又遇刘司徒之博洽,乃克成书,其难如此,忽有改图,殆不其然。今日群公云萃,留神
	盛举,盖匪创则无以始,弗继将莫能终,至于中间经画,尤艰其任此。殆至宝将出之,幸会其数天也,伟矣哉!惜子耄矣,而不能卒业,抑不知青云士能观以否。所谓前朝国势,先贤行事,故黎命脉,班班具存,推之于政,古今一也。有能舍其新而旧是图,将来挟以为国家用,吾知事业发挥,必当炜烨峥嵘,胜常而不凡也,讵止以资见闻谈说而已哉!沈君名桐,字大材,号春波,嘉禾望族,学通壁经,累试场屋,知名,以超贡入胄监,屈就今官,其于斯绩甚勤,且出私帑不之校,斯文不坠,系其承理,之功多也。
	因并及之。时在嘉靖十七年秋仲,东吴耄生杨循吉谨序。
	
\chapter{重刊唐书序}
	
	嘉靖己亥,吴郡重刊唐书,成书凡二百卷,本纪卷二十,志三十,列传百有五十。石晋宰相涿人刘雰撰。初,御史绍兴闻人公诠视学南畿,以是书世无梓本,他日按吴,遂命郡学训导沈桐刊置学宫。工未竟而公以忧去。及是书成,以书来属徵明为叙。按唐兴,令狐德棻等始撰武德、贞观两朝国史八十卷,至吴兢,合前后为书百卷,而柳芳、韦述嗣缉之,起义宁,讫开元,仅仅百余年,而于休烈、令狐峘以次增缉,讫于建中而止,而大历、元和以后则成于崔龟。从厥后韦澳诸人又增缉之,凡为书百四十有六卷,
	而芳等又有唐历四十卷,续历二十二篇,皆当时纪载之言,非成书也。晋革唐命,昫等始因旧史,绪成此书。然《五代史》昫传不载此事,岂其书出一时史馆,而昫特以宰相领其事邪?然不可考已。或谓五代抢攘,文气卑弱,而是书纪次抚法,详略失中,不足传远。宋庆历中,诏翰林儒臣刊修之。自庆历甲申至嘉祐庚子,历十有七年,成新书二百二十五卷,视旧史削六十一传,增传三百三十有一,续撰仪卫、选举及兵及艺文四志,别撰宰相、方镇及宗室世系、宰相世系四表,所谓其事则增于前,其文则省于旧,实当时表奏之语,而第赏制词,亦谓闳博精覈,度越诸子,良以宋景文、欧阳文忠皆当时大手笔,而是书实更二公之手,故朝野尊信,而旧书遂废不行。然议者则以用字奇涩为失体,刊削诏令为太略,固不若旧书之为愈也。司马氏修《通鉴》,悉据旧史,而于新书无取焉。惟周益公称其删繁为简,变今以古,有合于所谓文省于旧之论。而刘元城顾谓事增文省,正新书之失。唐庚氏尤深斥之,乃极言旧书之佳,其所引决海救焚、引鸩止渴之语,岂直工俪而已,自是一代名言也。然则是书也其可以无传乎!虽然,不能无可议者。段秀实请辞郭晞,有吾戴吾头之语,新书省一吾字,议者以为失实,是矣。而旧史秀实传乃都不书。夫秀实大节固不以此,而此事亦卓诡可喜。柳宗元叙事尤号奇警,且郑重致词,上于史馆,若是而不得登载,则其所遗亦多矣。甚者诋韩愈文章为纰谬,谓顺宗实录繁简不当,拙于取舍,异哉,岂晁氏所谓多所阙漏,是非失实者邪!甚矣作史之难也!心术有邪正,词理有工拙,识见有浅深,而史随以异,要在传信,传著不失其实而已。今二书具在,其工拙繁简,是非得失,莫之有揜焉。彼斥新书为乱道,诚为过论,而或缘此遂废旧史,又岂可哉?此闻人公所为梓行之意也。是书尝刻于越州,卷后有教授硃倬名。倬忤秦桧,出为越州教授,当是绍兴初年,今四百年矣。其书复行,而公又出于越,其事岂偶然哉?先是书久不行,世无善本,沈君仅得旧刻数册,较全书才十之六七,于是遍访藏书之家,残章断简,悉取以从事校阅,惟审一字或数易,历三暑寒乃克就绪,其勤诚有足嘉者。因附著之。是岁三月望,前翰林待诏长洲文徵明序。
	
\chapter{重刻旧唐书序}
	
	书以纪事,谀闻为聩;事以著代,间逸则遗。是故史氏之书,与天地相为始终,《六经》相为表里,疑信并传,阙文不饰,以纪事实,以昭世代。故《六经》道明,万世宗仰,非徒文艺之夸诞而已也。《尚书》壁存,典训不篸;《鲁史》麟绝,杞宋失征。继而有作,其惟司马氏及小司马,以迨班、范诸家。八书十志,经纬天人;八志十典,
	纮维政事。藏山刊石,繁绍圣经;历汉跻隋,炳发灵宪。是故王教之要,国典之源,代有征考,若睹蓍蔡。李唐嗣兴,万目毕举,其经画之精详,维持之慎密,虽未上蹑周轨,亦足并骤汉疆。晋史臣刘昫氏者,爰集馆寮,博稽载典。纂修二十一本纪,首高祖以迄哀帝,而汶哲具昭。旁修十一志,始《礼仪》以终《刑法》,而巨细毕举。列传一千一百八十有奇,内以纪后妃之淑慝,外以悉文武之臧否。《宗室》族属,互以时叙;《外戚》、《宦官》,各以类别。《良吏》、《酷吏》,鉴戒具昭;《忠义》、《孝友》,褒论悉当。《儒学》、《文苑》,表以著达;《方伎》、《隐逸》,兼以察微。详传《列女》,以彰妇顺,分传蛮狄,以立大防。卷凡二百一十有四,统名之曰《唐书》。识博学宏,才优义正,真有唐一代之良史,秦、隋以下,罕有其俪,固后世之刑鉴具在也。有宋迭兴,分职书局,载辑《唐鉴》于祖禹,继纂《唐书》于昌朝,王、宋诸贤,相继汇辑,复成一代之新书,遂亡刘氏之旧帙。诠谬司文学,遍历辅畿。爰校《六经》,兼雠诸史,始知汉、晋以迄宋、元,皆有监本,司成甬川张公,尝奉旨校勘,总为《二十一史》。刊证谬讹,粲然明备,惟刘氏《唐书》,郁绝不传,无所考觅。积集再期,酷志刊复,苦无善本,莫可继志。窃惟古人有云:“层台云构,所缺过乎榱桷;为山霞高,不终逾乎一篑。”悯哉斯言,益用惶怵。乃旁谋学属,博访诸司,间礼儒贤,以探往籍,更历三载,竟莫有成。末复弭节姑苏,穷搜力索,吴令硃子遂得列传于光禄张氏,长洲贺子随得纪志于守溪公,遗籍俱出宋时模板。旬月之间,二美璧合,古训有获,私喜无涯。乃督同苏庠,严为校刻,司训沈子,独肩斯任,效勤四载,书幸成编。匮直千金,刻未竟业。石江欧阳公闻而助以厚镒,午山冯子、西郭陈子以迨郡邑诸长贰,咸力
	辅以终事。数百年之阙典,于是乎始有可稽矣。物之成毁,信各有数,是书之成,夫岂偶哉!肇工于嘉靖乙未,卒刻于嘉靖戊戌。珠玑璀璨,亥豕尽刊;玉薤精严,尘叶罔翳。焕新一代之旧文,遐续百王之训典,追配诸史,允备全书。因布多方,以惠多士。余姚
	闻人诠序。
	
\mainmatter
	
\part{本纪第一}
\chapter{高祖}

\begin{yuanwen}
高祖神尧大圣大光孝皇帝姓李氏,讳渊。其先陇西狄道人,凉武昭王暠七代孙也。暠生歆。歆生重耳,仕魏为弘农太守。重耳生熙,为金门镇将,领豪杰镇武川,因家焉。仪凤中,追尊宣皇帝。熙生天锡,仕魏为幢主。大统中,赠司空。仪凤中,追尊光皇帝。皇祖讳虎,后魏左仆射,封陇西郡公,与周文帝及太保李弼、大司马独孤信等以功参佐命,当时称为“八柱国家”,仍赐姓大野氏。周受禅,追封唐国公,谥曰襄。至隋文帝作相,还复本姓。武德初,追尊景皇帝,庙号太祖,陵曰永康。皇考讳昞,周安州总管、柱国大将军,袭唐国公,谥曰仁。武德初,追尊元皇帝,庙号世祖,陵曰兴宁。
\end{yuanwen}

\begin{yuanwen}
高祖以周天和元年生于长安,七岁袭唐国公。及长,倜傥豁达,任性真率,宽仁容众,无贵贱咸得其欢心。隋受禅,补千牛备身。文帝独孤皇后,即高祖从母也,由是特见亲爱,累转谯、陇、岐三州刺史。有史世良者,善相人,谓高祖曰:“公骨法非常,必为人主,愿自爱,勿忘鄙言。”高祖颇以自负。
\end{yuanwen}

\begin{yuanwen}
大业初,为荥阳、楼烦二郡太守,征为殿内少监。

九年,迁卫尉少卿。辽东之役,督运于怀远镇。及杨玄感反,诏高祖驰驿镇弘化郡,兼知关右诸军事。高祖历试中外,素树恩德,及是结纳豪杰,众多款附。时炀帝多所猜忌,人怀疑惧。会有诏征高祖诣行在所,遇疾未谒。时甥王氏在后宫,帝问曰:“汝舅何迟?”

王氏以疾对,帝曰:“可得死否?”

高祖闻之益惧,因纵酒沉湎,纳贿以混其迹焉。
\end{yuanwen}



\begin{yuanwen}
十一年,炀帝幸汾阳宫,命高祖往山西、河东黜陟讨捕。师次龙门,贼帅母端儿帅众数千薄于城下。高祖从十余骑击之,所射七十发,皆应弦而倒,贼乃大溃。

十二年,迁右骁卫将军。
\end{yuanwen}

\begin{yuanwen}
十三年,为太原留守,郡丞王威、武牙郎将高君雅为副。群贼蜂起,江都阻绝,太宗与晋阳令刘文静首谋,劝举义兵。俄而马邑校尉刘武周据汾阳宫举兵反,太宗与王威、高君雅将集兵讨之。高祖乃命太宗与刘文静及门下客长孙顺德、刘弘基各募兵,旬日间众且一万,密遣使召世子建成及元吉于河东。威、君雅见兵大集,恐高祖为变,相与疑惧,请高祖祈雨于晋祠,将为不利。晋阳乡长刘世龙知之,以告高祖,高祖阴为之备。
\end{yuanwen}

\begin{yuanwen}
五月甲子,高祖与威、君雅视事,太宗密严兵于外,以备非常。遣开阳府司马刘政会告威等谋反,即斩之以徇,遂起义兵。甲戌,遣刘文静使于突厥始毕可汗,令率兵相应。

六月甲申,命太宗将兵徇西河,下之。癸巳,建大将军府,并置三军,分为左右:以世子建成为陇西公、左领大都督,左统军隶焉;太宗为敦煌公、右领大都督,右统军隶焉。裴寂为大将军府长史,刘文静为司马,石艾县长殷开山为掾,刘政会为属,长孙顺德、刘弘基、窦琮等分为左右统军。开仓库以赈穷乏,远近响应。
\end{yuanwen}


\begin{yuanwen}
秋七月壬子,高祖率兵西图关中,以元吉为镇北将军、太原留守。癸丑,发自太原,有兵三万。丙辰,师次灵石县,营于贾胡堡。隋武牙郎将宋老生屯霍邑以拒义师。会霖雨积旬,馈运不给,高祖命旋师,太宗切谏乃止。有白衣老父诣军门
\end{yuanwen}

\begin{yuanwen}

\end{yuanwen}\begin{yuanwen}

\end{yuanwen}\begin{yuanwen}

\end{yuanwen}\begin{yuanwen}

\end{yuanwen}\begin{yuanwen}

\end{yuanwen}\begin{yuanwen}

\end{yuanwen}\begin{yuanwen}

\end{yuanwen}\begin{yuanwen}

\end{yuanwen}\begin{yuanwen}

\end{yuanwen}\begin{yuanwen}

\end{yuanwen}\begin{yuanwen}

\end{yuanwen}\begin{yuanwen}

\end{yuanwen}\begin{yuanwen}

\end{yuanwen}\begin{yuanwen}

\end{yuanwen}\begin{yuanwen}

\end{yuanwen}\begin{yuanwen}

\end{yuanwen}\begin{yuanwen}

\end{yuanwen}\begin{yuanwen}

\end{yuanwen}\begin{yuanwen}

\end{yuanwen}\begin{yuanwen}

\end{yuanwen}\begin{yuanwen}

\end{yuanwen}\begin{yuanwen}

\end{yuanwen}\begin{yuanwen}

\end{yuanwen}\begin{yuanwen}

\end{yuanwen}\begin{yuanwen}

\end{yuanwen}\begin{yuanwen}

\end{yuanwen}\begin{yuanwen}

\end{yuanwen}\begin{yuanwen}

\end{yuanwen}\begin{yuanwen}

\end{yuanwen}\begin{yuanwen}

\end{yuanwen}\begin{yuanwen}

\end{yuanwen}\begin{yuanwen}

\end{yuanwen}\begin{yuanwen}

\end{yuanwen}\begin{yuanwen}

\end{yuanwen}\begin{yuanwen}

\end{yuanwen}\begin{yuanwen}

\end{yuanwen}\begin{yuanwen}

\end{yuanwen}\begin{yuanwen}

\end{yuanwen}\begin{yuanwen}

\end{yuanwen}\begin{yuanwen}

\end{yuanwen}\begin{yuanwen}

\end{yuanwen}\begin{yuanwen}

\end{yuanwen}\begin{yuanwen}

\end{yuanwen}\begin{yuanwen}

\end{yuanwen}\begin{yuanwen}

\end{yuanwen}\begin{yuanwen}

\end{yuanwen}\begin{yuanwen}

\end{yuanwen}\begin{yuanwen}

\end{yuanwen}\begin{yuanwen}

\end{yuanwen}\begin{yuanwen}

\end{yuanwen}\begin{yuanwen}

\end{yuanwen}\begin{yuanwen}

\end{yuanwen}\begin{yuanwen}

\end{yuanwen}\begin{yuanwen}

\end{yuanwen}\begin{yuanwen}

\end{yuanwen}\begin{yuanwen}

\end{yuanwen}\begin{yuanwen}

\end{yuanwen}\begin{yuanwen}

\end{yuanwen}\begin{yuanwen}

\end{yuanwen}\begin{yuanwen}

\end{yuanwen}\begin{yuanwen}

\end{yuanwen}\begin{yuanwen}

\end{yuanwen}\begin{yuanwen}

\end{yuanwen}\begin{yuanwen}

\end{yuanwen}\begin{yuanwen}

\end{yuanwen}




曰:“余为霍山神使谒唐皇帝曰:‘八
月雨止,路出霍邑东南,吾当济师。’高祖曰:“此神不欺赵无恤,岂负我哉!”八月
辛巳,高祖引师趋霍邑,斩宋老生,平霍邑。丙戌,进下临汾郡及绛郡。癸巳,至龙门,
突厥始毕可汗遣康稍利率兵五百人、马二千匹,与刘文静会于麾下。隋骁卫大将军屈突
通镇河东,津梁断绝,关中向义者颇以为阻。河东水滨居人,竞进舟楫,不谋而至,前
后数百人。
九月壬寅,冯翊贼帅孙华、士门贼帅白玄度各率其众送款,并具舟楫以待义师。高
祖令华与统军王长谐、刘弘基引兵渡河。屈突通遣其武牙郎将桑显和率众数千,夜袭长
谐,义师不利。太宗以游骑数百掩其后,显和溃散,义军复振。丙辰,冯翊太守萧造以
郡来降。戊午,高祖亲率众围河东,屈突通自守不出,乃命攻城,不利而还。文武将吏
请高祖领太尉,加置僚佐,从之。华阴令李孝常以永丰仓来降。庚申,高祖率军济河,
舍于长春宫。三秦士庶至者日以千数,高祖礼之,咸过所望,人皆喜悦。丙寅,遣陇西
公建成、司马刘文静屯兵永丰仓,兼守潼关,以备他盗。太宗率刘弘基、长孙顺德等前
后数万人,自渭北徇三辅,所至皆下。高祖从父弟神通起兵鄠县,柴氏妇举兵于司竹,
至是并与太宗会。郿县贼帅丘师利、李仲文,盩厔贼帅何潘仁等,合众数万来降。乙亥,
命太宗自渭汭屯兵阿城,陇西公建成自新丰趣霸上。高祖率大军自下邽西上,经炀帝行
宫园苑,悉罢之,宫女放还亲属。
冬十月辛巳,至长乐宫,有众二十万。京师留守刑部尚书卫文升、右翊卫将军阴世
师、京兆郡丞滑仪挟代王侑以拒义师。高祖遣使至城下,谕以匡复之意,再三皆不报。
诸将固请围城。十一月丙辰,攻拔京城。卫文升先已病死,以阴世师、滑仪等拒义兵,
并斩之。癸亥,率百僚,备法驾,立代王侑为天子,遥尊炀帝为太上皇,大赦,改元为
义宁。甲子,隋帝诏加高祖假黄钺、使持节、大都督内外诸军事、大丞相,进封唐王,
总录万机。以武德殿为丞相府,改教为令。以陇西公建成为唐国世子;太宗为京兆尹,
改封秦公;姑臧公元吉为齐公。十二月癸未,丞相府置长史、司录已下官僚。金城贼帅
薛举寇扶风,命太宗为元帅击之。遣赵郡公孝恭招慰山南,所至皆下。癸巳,太宗大破
薛举之众于扶风。屈突通自潼关奔东都,刘文静等追擒于阌乡,虏其众数万。河池太守
萧瑀以郡降。丙午,遣云阳令詹俊、武功县正李仲衮徇巴蜀,下之。
二年春正月戊辰,世子建成为抚宁大将军、东讨元帅,太宗为副,总兵七万,徇地
东都。二月,清河贼帅窦建德僭称长乐王。吴兴人沈法兴据丹阳起兵。三月丙辰,右屯
卫将军宇文化及弑隋太上皇于江都宫,立秦王浩为帝,自称大丞相。徙封太宗为赵国公。
戊辰,隋帝进高祖相国,总百揆,备九锡之礼。唐国置丞相以下,立皇高祖已下四庙于
长安通义里第。
夏四月辛卯,停竹使符,颁银菟符于诸郡。戊戌,世子建成及太宗自东都班师。五
月乙巳,天子诏高祖冕十有二旒,建天子旌旗,出警入跸。王后、王女爵命之号,一遵
旧典。戊午,隋帝诏曰:
天祸隋国,大行太上皇遇盗江都,酷甚望夷,衅深骊北。悯予小子,奄造丕愆,哀
号永感,心情糜溃。仰惟荼毒,仇复靡申,形影相吊,罔知启处。相国唐王,膺期命世,
扶危拯溺,自北徂南,东征西怨。致九合于诸侯,决百胜于千里。纠率夷夏,大庇氓黎,
保乂朕躬,系王是赖。德侔造化,功格苍旻,兆庶归心,历数斯在,屈为人臣,载违天
命。在昔虞、夏,揖让相推,苟非重华,谁堪命禹。当今九服崩离,三灵改卜,大运去
矣,请避贤路。兆谋布德,顾己莫能,私僮命驾,须归籓国。予本代王,及予而代,天
之所废,岂其如是!庶凭稽古之圣,以诛四凶;幸值惟新之恩,预充三恪。雪冤耻于皇
祖,守禋祀为孝孙,朝闻夕殒,及泉无恨。今遵故事,逊于旧邸,庶官群辟,改事唐朝。
宜依前典,趋上尊号,若释重负,感泰兼怀。假手真人,俾除丑逆,济济多士,明知朕
意。仍敕有司,凡有表奏,皆不得以闻。
遣使持节、兼太保、邢部尚书、光禄大夫、梁郡公萧造,兼太尉、司农少卿裴之隐
奉皇帝玺绶于高祖。高祖辞让,百僚上表劝进,至于再三,乃从之。隋帝逊于旧邸。改
大兴殿为太极殿。
甲子,高祖即皇帝位于太极殿,命刑部尚书萧造兼太尉,告于南郊,大赦天下,改
隋义宁二年为唐武德元年。官人百姓,赐爵一级。义师所行之处,给复三年。罢郡置州,
改太守为刺史。丁卯,宴百官于太极殿,赐帛有差。东都留守官共立隋越王侗为帝。壬
申,命相国长史裴寂等修律令。
六月甲戌,太宗为尚书令,相国府长史裴寂为尚书右仆射,相国府司马刘文静为纳
言,隋民部尚书萧瑀、相国府司录窦威并为内史令。废隋《大业律令》,颁新格。己卯,
备法驾,迎皇高祖宣简公已下神主,祔于太庙。追谥妃窦氏为太穆皇后,陵曰寿安。庚
辰,立世子建成为后太子。封太宗为秦王,齐国公元吉为齐王。封宗室蜀国公孝基为永
安王,柱国公道玄为淮阳王,长平公叔良为长平王,郑国公神通为永康王,安吉公神符
为襄邑王,柱国德良为长乐王,上开府道素为竟陵王,上柱国博乂为陇西王,奉慈为渤
海王。诸州总管加号使持节。癸未,封隋帝为酅国公。薛举寇泾州,命秦王为西讨元帅
征之。改封永康王神通为淮安王。壬辰,加秦王雍州牧,余官如故。辛丑,内史令窦威
卒。秋七月丙午,刑部尚书萧造为太子太保。追封皇子玄霸为卫王。西突厥遣使内附。
秦王与薛举大战于泾州,我师败绩。
八月壬午,薛举死,其子仁杲复僭称帝,命秦王为元帅以讨之。丁亥,诏曰:“隋太常卿高颎、上柱国贺若弼,并抗节不阿,矫枉无挠;司隶大夫薛道衡、刑部尚书宇文弼、左翊卫将军董纯,并怀忠抱义,以陷极刑:宜从褒饰,以慰泉壤。颎可赠上柱国、郯国公,弼赠上柱国、杞国公,各令有司加谥;道衡赠上开府、临河县公,赠上开府、平昌县公,纯赠柱国、狄道县公。”又诏曰:“隋右骁卫大将军李金才、左光禄大夫李
敏,并鼎族高门,元功世胄,横受屠杀,朝野称冤。然李氏将兴,天祚有应,冥契深隐,
妄肆诛夷。朕受命君临,志存刷荡,申冤旌善,无忘寤寐。金才可赠上柱国、申国公,
敏可赠柱国、观国公。又前代酷滥,子孙被流者,并放还乡里。”凉州贼帅李轨以其地
来降,拜凉州总管,封凉王。
九月乙巳,亲录囚徒,改银菟符为铜鱼符。辛未,追谥隋太上皇为炀帝。宇文化及
至魏州,鸩杀秦王浩,僭称天子,国号许。
冬十月壬申朔,日有蚀之。李密率众来降。封皇从父弟襄武公琛为襄武王,黄台公
瑗为庐江王。癸巳,诏行傅仁均所造《戊寅历》。十一月己酉,以京师谷贵,令四面入
关者,车马牛驴各给课米,充其自食。秦王大破薛仁杲于浅水原,降之,陇右平。乙巳,
凉王李轨僭称天子于凉州。诏颁五十三条格,以约法缓刑。十二月壬申,加秦王太尉、
陕东道大行台。丁丑,封上柱国李孝常为义安王。庚子,李密反于桃林,行军总管盛彦
师追讨斩之。
二年春正月乙卯,初令文官遭父母丧者听去职。黄门侍郎陈叔达兼纳言。二月丙戌,
诏天下诸宗人无职任者,不在徭役之限,每州置宗师一人,以相统摄。丁酉,窦建德攻
宇文化及于聊城,斩之,传首突厥。闰月辛丑,刘武周侵我并州。己酉,李密旧将徐世
勣以黎阳之众及河南十郡降,授黎州总管,封曹国公,赐姓李氏。庚戌,上微行都邑,
以察氓俗,即日还宫。甲寅,贼帅硃粲杀我使散骑常侍段确,奔洛阳。
夏四月乙巳,王世充篡越王侗位,僭称天子,国号郑。辛亥,李轨为其伪尚书安兴
贵所执以降,河右平。突厥始毕可汗死。五月己卯,酅国公薨,追崇为隋帝,谥曰恭。
六月戊戌,令国子学立周公、孔子庙,四时致祭,仍博求其后。癸亥,尚书右仆射裴寂
为晋州道行军总管,以讨刘武周。秋七月壬申,置十二军,以关内诸府分隶焉。王世充
遣其将罗士信侵我谷州,士信率其众来降。西突厥叶护可汗及高昌并遣使朝贡。
九月辛未,贼帅李子通据江都,僭称天子,国号吴。沈法兴据毗陵,僭称梁王。丁
丑,和州贼帅杜伏威遣使来降,授和州总管、东南道行台尚书令,封楚王。裴寂与刘武
周将宋金刚战于介州,我师败绩,右武卫大将军姜宝谊死之。并州总管、齐王元吉惧武
周所逼,奔于京师,并州陷。乙未,京师地震。
冬十月己亥。封幽州总管罗艺为燕郡王,赐姓李氏。黄门侍郎杨恭仁为纳言。杀民
部尚书、鲁国公刘文静。乙卯,讨刘武周,军于蒲州,为诸军声援。壬子,刘武周进围
晋州。甲子,上亲祠华岳。十一月丙子,窦建德陷黎阳,尽有山东之地。淮安王神通、
左武候大将军李世勣皆没于贼。十二月丙申,永安王孝基、工部尚书独孤怀恩、总管于
筠为刘武周将宋金刚掩袭,并没焉。甲辰,狩于华山。壬子,大风拔木。
三年春正月辛巳,幸蒲州,命祀舜庙。癸巳,至自蒲州。甲午,李世勣于窦建德所
自拔归国。建德僭称夏王。二月丁酉,京师西南地有声如山崩。庚子,幸华阴。工部尚
书独孤怀恩谋反,伏诛。三月癸酉,西突厥叶护可汗、高昌王曲伯雅遣使朝贡。突厥贡
条支巨鸟。己卯,改纳言为侍中,内史令为中书令,给事郎为给事中。甲戌,内史侍郎
封德彝兼中书令。封贼帅刘孝真为彭城王,赐姓李氏。
夏四月壬寅,至自华阴。于益州置行台尚书省。甲寅,加秦王益州道行台尚书令。
秦王大破宋金刚于介州,金刚与刘武周俱奔突厥,遂平并州。伪总管尉迟敬德、寻相以
介州降。
六月壬辰,徙封楚王杜伏威为吴王,赐姓李氏,加授东南道行台尚书令。丙午,亲
录囚徒。封皇子元景为赵王,元昌为鲁王,元亨为酆王;皇孙承宗为太原王,承道为安
陆王,承乾为恆山王,恪为长沙王,泰为宜都王。
秋七月壬戌,命秦王率诸军讨王世充。遣皇太子镇蒲州,以备突厥。丙申,突厥杀
刘武周于白道。冬十月庚子,怀戍贼帅高开道遣使降,授蔚州总管,封北平郡王,赐姓
李氏。
四年春正月丁卯,窦建德行台尚书令胡大恩以大安镇来降,封定襄郡王,赐姓李氏。
辛巳,命皇太子总统诸军讨稽胡。三月,徙封宜都王泰为卫王。窦建德来援王世充,攻
陷我管州。
夏四月甲寅,封皇子元方为周王,元礼为郑王,元嘉为宋王,元则为荆王,元茂为
越王。初置都护府官员。五月己未,秦王大破窦建德之众于武牢,擒建德,河北悉平。
丙寅,王世充举东都降,河南平。秋七月甲子,秦王凯旋,献俘于太庙。丁卯,大赦天
下。废五铢钱,行开元通宝钱。斩窦建德于市;流王世充于蜀,未发,为仇人所害。甲
戌,建德余党刘黑闼据漳南反。置山东道行台尚书省于洺州。八月,兗州总管徐圆朗举
兵反,以应刘黑闼,僭称鲁王。
冬十月己丑,加秦王天策上将,位在王公上,领司徒、陕东道大行台尚书令;齐王
元吉为司空。乙巳,赵郡王孝恭平荆州,获萧铣。十一月甲申,于洺州置大行台,废洺
州都督府。庚寅,焚东都紫微宫乾阳殿。会稽贼帅李子通以其地来降。十二月丁卯,命
秦王及齐王元吉讨刘黑闼。壬申,徙封宋王元嘉为徐王。
五年春正月丙申,刘黑闼据洺州,僭称汉东王。三月丁未,秦王破刘黑闼于洺水上,
尽复所陷州县,黑闼亡奔突厥。蔚州总管、北平王高开道叛,寇易州。
夏四月庚戌,秦王还京师,高祖迎劳于长乐宫。壬申,代州总管、定襄郡王大恩为
虏所败,战死。六月,刘黑闼引突厥寇山东。置谏议大夫官员。秋七月丁亥,吴王伏威
来朝。隋汉阳太守冯盎以南越之地来降,岭表悉定。八月辛亥,以洺、荆、并、幽、交
五州为大总管府。改封恆山王承乾为中山王。葬隋炀帝于扬州。丙辰,突厥颉利寇雁门。
己未,进寇朔州。遣皇太子及秦王讨击,大败之。
冬十月癸酉,遣齐王元吉击刘黑闼于洺州。时山东州县多为黑闼所守,所在杀长吏
以应之。行军总管、淮阳王道玄与黑闼战于下博,道玄败没。十一月甲申,命皇太子率
兵讨刘黑闼。丙申,幸宜州,简阅将士。十二月丙辰,校猎于华池。庚申,至自宜州。
皇太子破刘黑闼于魏州,斩之,山东平。
六年春正月,吴王杜伏威为太子太保。二月辛亥,校猎于骊山。三月乙未,幸昆明
池,宴百官。
夏四月己未,旧宅改为通义宫,曲赦京城系囚,于是置酒高会,赐从官帛各有差。
癸酉,以尚书右仆射、魏国公裴寂为左仆射,中书令、宋国公萧瑀为右仆射,侍中、观
国公杨恭仁为吏部尚书。秋七月,突厥颉利寇朔州,遣皇太子及秦王屯并州以备之。
八月壬子,东南道行台仆射辅公祏据丹阳反,僭称宋王,遣赵郡王孝恭及岭南道大
使、永康县公李靖讨之。丙寅,吐谷浑内附。九月丙子,突厥退,皇太子班师。改东都
为洛州。高开道引突厥寇幽州。冬十月,幸华阴。
十一月,校猎于沙苑。十二月乙巳,以奉义监为龙跃宫,武功宅为庆善宫。甲寅,
至自华阴。
七年春正月己酉,封高丽王高武为辽东郡王,百济王扶余璋为带方郡王,新罗王金
真平为乐浪郡王。二月,高开道为部将张金树所杀,以其地降。丁巳,幸国子学,亲临
释奠。改大总管府为大都督府。吴王伏威薨。三月戊寅,废尚书省六司侍郎,增吏部郎
中秩正四品,掌选事。戊戌,赵郡王孝恭大破辅公祏,擒之,丹阳平。
夏四月庚子,大赦天下,颁行新律令。以天下大定,诏遭父母丧者听终制。五月,
造仁智宫于宜州之宜君县。李世勣讨徐圆朗,平之。六月辛丑,幸仁智宫。
秋七月甲午,至自仁智宫。巂州地震山崩,江水咽流。八月戊辰,突厥寇并州,京
师戒严。壬午,突厥退。乙未,京师解严。冬十月丁卯,幸庆善宫。癸酉,幸终南山,
谒老子庙。十一月戊辰,校猎于高陵。庚午,至自庆善宫。
八年春二月己巳,亲录囚徒,多所原宥。
夏四月,造太和宫于终南山。六月甲子,幸太和宫。突厥寇定州,命皇太子往幽州,
秦王往并州,以备突厥。八月,并州道总管张公谨与突厥战于太谷,王师败绩,中书令
温彦博没于贼。九月,突厥退。冬十月辛巳,幸周氏陂校猎,因幸龙跃宫。十一月辛卯,
幸宜州。庚子,讲武于同官县。改封蜀王元轨为吴王,汉王元庆为陈王。加授秦王中书
令,齐王元吉侍中。天策上将府司马宇文士及权检校侍中。十二月辛酉,至自宜州。
九年春正月丙寅,命州县修城隍,备突厥。尚书左仆射、魏国公裴寂为司空。
二月庚申,加齐王元吉为司徒。戊寅,亲祠社稷。三月辛卯,幸昆明池。夏五月辛
巳,以京师寺观不甚清净,诏曰:
释迦阐教,清净为先,远离尘垢,断除贪欲。所以弘宣胜业,修植善根,开导愚迷,
津梁品庶。是以敷演经教,检约学徒,调忏身心,舍诸染著,衣服饮食,咸资四辈。
自觉王迁谢,像法流行,末代陵迟,渐以亏滥。乃有猥贱之侣,规自尊高;浮惰之
人,苟避徭役。妄为剃度,托号出家,嗜欲无厌,营求不息。出入闾里,周旋阛阓,驱
策田产,聚积货物。耕织为生,估贩成业,事同编户,迹等齐人。进违戒律之文,退无
礼典之训。至乃亲行劫掠,躬自穿窬,造作妖讹,交通豪猾。每罹宪网,自陷重刑,黩
乱真如,倾毁妙法。譬兹稂莠,有秽嘉苗;类彼淤泥,混夫清水。又伽蓝之地,本曰净
居,栖心之所,理尚幽寂。近代以来,多立寺舍,不求闲旷之境,唯趋喧杂之方。缮采
崎岖,栋宇殊拓,错舛隐匿,诱纳奸邪。或有接延鄽邸,邻近屠酤,埃尘满室,膻腥盈
道。徒长轻慢之心,有亏崇敬之义。且老氏垂化,实贵冲虚,养志无为,遗情物外。全
真守一,是谓玄门,驱驰世务,尤乖宗旨。
朕膺期驭宇,兴隆教法,志思利益,情在护持。欲使玉石区分,薰莸有辨,长存妙
道,永固福田,正本澄源,宜从沙汰。诸僧、尼、道士、女寇等,有精勤练行、守戒律
者,并令大寺观居住,给衣食,勿令乏短。其不能精进、戒行有阙、不堪供养者,并令
罢遣,各还桑梓。所司明为条式,务依法教,违制之事,悉宜停断。京城留寺三所,观
二所。其余天下诸州,各留一所。余悉罢之。事竟不行。
六月庚申,秦王以皇太子建成与齐王元吉同谋害己,率兵诛之。诏立秦王为皇太子,
继统万机,大赦天下。八月癸亥,诏传位于皇太子。尊帝为太上皇,徙居弘义宫,改名
太安宫。
贞观八年三月甲戌,高祖宴西突厥使者于两仪殿,顾谓长孙无忌曰:“当今蛮夷率
服,古未尝有。”无忌上千万岁寿。高祖大悦,以酒赐太宗。太宗又奉觞上寿,流涕而
言曰:“百姓获安,四夷咸附,皆奉遵圣旨,岂臣之力!”于是太宗与文德皇后互进御
膳,并上服御衣物,一同家人常礼。是岁,阅武于城西,高祖亲自临视,劳将士而还。
置酒于未央宫,三品已上咸侍。高祖命突厥颉利可汗起舞,又遣南越酋长冯智戴咏诗,
既而笑曰:“胡、越一家,自古未之有也。”太宗奉觞上寿曰:“臣早蒙慈训,教以文
道;爰从义旗,平定京邑。重以薛举、武周、世充、建德,皆上禀睿算,幸而克定。三
数年间,混一区宇。天慈崇宠,遂蒙重任。今上天垂祐,时和岁阜,被发左衽,并为臣
妾。此岂臣智力,皆由上禀圣算。”高祖大悦,群臣皆呼万岁,极夜方罢。
九年五月庚子,高祖大渐,下诏:“既殡之后,皇帝宜于别所视军国大事。其服轻
重,悉从汉制,以日易月。园陵制度,务从俭约。”是日,崩于太安宫之垂拱前殿,年
七十。群臣上谥曰大武皇帝,庙号高祖。十月庚寅,葬于献陵。高宗上元元年八月,改
上尊号曰神尧皇帝。天宝十三载二月,上尊号神尧大圣大光孝皇帝。
史臣曰:有隋季年,皇图板荡,荒主燀燎原之焰,群盗发逐鹿之机,殄暴无厌,横
流靡救。高祖审独夫之运去,知新主之勃兴,密运雄图,未伸龙跃。而屈己求可汗之援,
卑辞答李密之书,决神机而速若疾雷,驱豪杰而从如偃草。洎讴谣允属,揖让受终,刑
名大刬于烦苛,爵位不逾于珝轴。由是攫金有耻,伏莽知非,人怀汉道之宽平,不责高
皇之慢骂。然而优柔失断,浸润得行,诛文静则议法不从,酬裴寂则曲恩太过。奸佞由
之贝锦,嬖幸得以掇蜂。献公遂间于申生,小白宁怀于召忽。一旦兵交爱子,矢集申孙。
匈奴寻犯于便桥,京邑咸忧于左衽。不有圣子,王业殆哉!
赞曰:高皇创图,势若摧枯。国运神武,家难圣谟。言生床笫,祸切肌肤。《鸱鸮》
之咏,无损于吾。

本纪第二 太宗上    
太宗文武大圣大广孝皇帝讳世民,高祖第二子也。母曰太穆顺圣皇后窦氏。隋开皇
十八年十二月戊午,生于武功之别馆。时有二龙戏于馆门之外,三日而去。高祖之临岐
州,太宗时年四岁。有书生自言善相,谒高祖曰:“公贵人也,且有贵子。”见太宗,
曰:“龙凤之姿,天日之表,年将二十,必能济世安民矣。”高祖惧其言泄,将杀之,
忽失所在,因采“济世安民”之义以为名焉。太宗幼聪睿,玄鉴深远,临机果断,不拘
小节,时人莫能测也。
大业末,炀帝于雁门为突厥所围,太宗应募救援,隶屯卫将军云定兴营。将行,谓
定兴曰:“必赍旗鼓以设疑兵。且始毕可汗举国之师,敢围天子,必以国家仓卒无援。
我张军容,令数十里幡旗相续,夜则钲鼓相应,虏必谓救兵云集,望尘而遁矣。不然,
彼众我寡,悉军来战,必不能支矣。”定兴从焉。师次崞县,突厥候骑驰告始毕曰:王
师大至。由是解围而遁。及高祖之守太原,太宗时年十八。有高阳贼帅魏刀兒,自号历
山飞。来攻太原,高祖击之,深入贼阵。太宗以轻骑突围而进,射之,所向皆披靡,拔
高祖于万众之中。适会步兵至,高祖与太宗又奋击,大破之。时隋祚已终,太宗潜图义
举,每折节下士,推财养客,群盗大侠,莫不愿效死力。及义兵起,乃率兵略徇西河,
克之。拜右领大都督,右三军皆隶焉,封燉煌郡公。
大军西上贾胡堡,隋将宋老生率精兵二万屯霍邑,以拒义师。会久雨粮尽,高祖与
裴寂议,且还太原,以图后举。太宗曰:“本兴大义以救苍生,当须先入咸阳,号令天
下;遇小敌即班师,将恐从义之徒一朝解体。还守太原一城之地,此为贼耳,何以自
全!”高祖不纳,促令引发。太宗遂号泣于外,声闻帐中。高祖召问其故,对曰:“今
兵以义动,进战则必克,退还则必散。众散于前,敌乘于后,死亡须臾而至,是以悲
耳。”高祖乃悟而止。
八月己卯,雨霁,高祖引师趣霍邑。太宗恐老生不出战,乃将数骑先诣其城下,举
鞭指麾,若将围城者,以激怒之。老生果怒,开门出兵,背城而阵。高祖与建成合阵于
城东,太宗及柴绍阵于城南。老生麾兵疾进,先薄高祖,而建成坠马,老生乘之,高祖
与建成军咸却。太宗自南原率二骑驰下峻坂,冲断其军,引兵奋击,贼众大败,各舍仗
而走。悬门发,老生引绳欲上,遂斩之,平霍邑。至河东,关中豪杰争走赴义。太宗请
进师入关,取永丰仓以赈穷乏,收群盗以图京师,高祖称善。太宗以前军济河,先定渭
北。三辅吏民及诸豪猾诣军门请自效者日以千计,扶老携幼,满于麾下。收纳英俊,以
备僚列,远近闻者,咸自托焉。师次于泾阳,胜兵九万,破胡贼刘鹞子,并其众。留殷
开山、刘弘基屯长安故城。太宗自趣司竹,贼帅李仲文、何潘仁、向善志等皆来会,顿
于阿城,获兵十三万。长安父老赍牛酒诣旌门者不可胜纪,劳而遣之,一无所受。军令
严肃,秋毫无所犯。寻与大军平京城。高祖辅政,受唐国内史,改封秦国公。会薛举以
劲卒十万来逼渭滨,太宗亲击之,大破其众,追斩万余级,略地至于陇坻。
义宁元年十二月,复为右元帅,总兵十万徇东都。及将旋,谓左右曰:“贼见吾还,
必相追蹑。”设三伏以待之。俄而隋将段达率万余人自后而至,度三王陵,发伏击之,
段达大败,追奔至于城下。因于宜阳、新安置熊、谷二州,戍之而还。徙封赵国公。高
祖受禅,拜尚书令、右武候大将军,进封秦王,加授雍州牧。
武德元年七月,薛举寇泾州,太宗率众讨之,不利而旋。九月,薛举死,其子仁杲
嗣立。太宗又为元帅以击仁杲,相持于折墌城,深沟高垒者六十余日。贼众十余万,兵
锋甚锐,数来挑战,太宗按甲以挫之。贼粮尽,其将牟君才、梁胡郎来降。太宗谓诸将
军曰:“彼气衰矣,吾当取之。”遣将军庞玉先阵于浅水原南以诱之,贼将宗罗并军
来拒,玉军几败。既而太宗亲御大军,奄自原北,出其不意。罗望见,复回师相拒。
太宗将骁骑数十入贼阵,于是王师表里齐奋,罗大溃,斩首数千级,投涧谷而死者不
可胜计。太宗率左右二十余骑追奔,直趣折墌以乘之。仁杲大惧,婴城自守。将夕,大
军继至,四面合围。诘朝,仁杲请降,俘其精兵万余人、男女五万口。既而诸将奉贺,
因问曰:“始大王野战破贼,其主尚保坚城,王无攻具,轻骑腾逐,不待步兵,径薄城
下,咸疑不克,而竟下之,何也?”太宗曰:“此以权道迫之,使其计不暇发,以故克
也。罗恃往年之胜,兼复养锐日久,见吾不出,意在相轻。今喜吾出,悉兵来战,虽
击破之,擒杀盖少。若不急蹑,还走投城,仁杲收而抚之,则便未可得矣。且其兵众皆
陇西人,一败披退,不及回顾,散归陇外,则折墌自虚,我军随而迫之,所以惧而降也。
此可谓成算,诸君尽不见耶?”诸将曰:“此非凡人所能及也。”获贼兵精骑甚众,还
令仁杲兄弟及贼帅宗罗、翟长孙等领之。太宗与之游猎驰射,无所间然。贼徒荷恩慑
气,咸愿效死。时李密初附,高祖令密驰传迎太宗于豳州。密见太宗天姿神武,军威严
肃,惊悚叹服,私谓殷开山曰:“真英主也。不如此,何以定祸乱乎?”凯旋,献捷于
太庙。拜太尉、陕东道行台尚书令,镇长春宫,关东兵马并受节度。寻加左武候大将军、
凉州总管。
宋金刚之陷浍州也,兵锋甚锐。高祖以王行本尚据蒲州,吕崇茂反于夏县,晋、浍
二州相继陷没,关中震骇,乃手敕曰:“贼势如此,难与争锋,宜弃河东之地,谨守关
西而已。”太宗上表曰:“太原王业所基,国之根本,河东殷实,京邑所资。若举而弃
之,臣窃愤恨。愿假精兵三万,必能平殄武周,克复汾、晋。”高祖于是悉发关中兵以
益之,又幸长春宫亲送太宗。二年十一月,太宗率众趣龙门关,履冰而渡之,进屯柏壁,
与贼将宋金刚相持。寻而永安王孝基败于夏县,于筠、独孤怀恩、唐俭并为贼将寻相、
尉迟敬德所执,将还浍州。太宗遣殷开山、秦叔宝邀之于美良川,大破之,相等仅以身
免,悉虏其众,复归柏壁。于是诸将咸请战,太宗曰:“金刚悬军千里,深入吾地,精
兵骁将,皆在于此。武周据太原,专倚金刚以为捍。士卒虽众,内实空虚,意在速战。
我坚营蓄锐以挫其锋,粮尽计穷,自当遁走。”
三年二月,金刚竟以众馁而遁,太宗追之至介州。金刚列阵,南北七里,以拒官军。
太宗遣总管李世勣、程咬金、秦叔宝当其北,翟长孙、秦武通当其南。诸军战小却,为
贼所乘。太宗率精骑击之,冲其阵后,贼众大败,追奔数十里。敬德、相率众八千来降,
还令敬德督之,与军营相参。屈突通惧其为变,骤以为请。太宗曰:“昔萧王推赤心置
人腹中,并能毕命,今委任敬德,又何疑也。”于是刘武周奔于突厥,并、汾悉复旧地。
诏就军加拜益州道行台尚书令。
七月,总率诸军攻王世充于洛邑,师次谷州。世充率精兵三万阵于慈涧,太宗以轻
骑挑之。时众寡不敌,陷于重围,左右咸惧。太宗命左右先归,独留后殿。世充骁将单
雄信数百骑夹道来逼,交抢竞进,太宗几为所败。太宗左右射之,无不应弦而倒,获其
大将燕颀。世充乃拔慈涧之镇归于东都。太宗遣行军总管史万宝自宜阳南据龙门,刘德
威自太行东围河内,王君廓自洛口断贼粮道。又遣黄君汉夜从孝水河中下舟师袭回洛城,
克之。黄河已南,莫不响应,城堡相次来降。大军进屯邙山。九月,太宗以五百骑先观
战地,卒与世充万余人相遇,会战,复破之,斩首三千余级,获大将陈智略,世充仅以
身免。其所署筠州总管杨庆遣使请降,遣李世勣率师出轘辕道安抚其众。荥、汴、洧、
豫九州相继来降。世充遂求救于窦建德。
四年二月,又进屯青城宫。营垒未立,世充众二万自方诸门临谷水而阵。太宗以精
骑阵于北邙山,令屈突通率步卒五千渡水以击之,因诫通曰:“待兵交即放烟,吾当率
骑军南下。”兵才接,太宗以骑冲之,挺身先进,与通表里相应。贼众殊死战,散而复
合者数焉。自辰及午,贼众始退。纵兵乘之,俘斩八千人,于是进营城下。世充不敢复
出,但婴城自守,以待建德之援。太宗遣诸军掘堑,匝布长围以守之。吴王杜伏威遣其
将陈正通、徐召宗率精兵二千来会于军所。伪郑州司马沈悦以武牢降,将军王君廓应之,
擒其伪荆王王行本。会窦建德以兵十余万来援世充,至于酸枣。萧瑀、屈突通、封德彝
皆以腹背受敌,恐非万全,请退师谷州以观之。太宗曰:“世充粮尽,内外离心,我当
不劳攻击,坐收其敝。建德新破孟海公,将骄卒惰,吾当进据武牢,扼其襟要。贼若冒
险与我争锋,破之必矣。如其不战,旬日间世充当自溃。若不速进,贼入武牢,诸城新
附,必不能守。二贼并力,将若之何?”通又请解围就险以候其变,太宗不许。于是留
通辅齐王元吉以围世充,亲率步骑三千五百人趣武牢。
建德自荥阳西上,筑垒于板渚,太宗屯武牢,相持二十余日。谍者曰:“建德伺官
军刍尽,候牧马于河北,因将袭武牢。”太宗知其谋,遂牧马河北以诱之。诘朝,建德
果悉众而至,陈兵氾水,世充将郭士衡阵于其南,绵互数里,鼓噪,诸将大惧。太宗将
数骑升高丘以望之,谓诸将曰:“贼起山东,未见大敌。今度险而嚣,是无政令;逼城
而阵,有轻我心。我按兵不出,彼乃气衰,阵久卒饥,必将自退,追而击之,无往不克。
吾与公等约,必以午时后破之。”建德列阵,自辰至午,兵士饥倦,皆坐列,又争饮水,
逡巡敛退。太宗曰:“可击矣!”亲率轻骑追而诱之,众继至。建德回师而阵,未及整
列,太宗先登击之,所向皆靡。俄而众军合战,嚣尘四起。太宗率史大奈、程咬金、秦
叔宝、宇文歆等挥幡而入,直突出其阵后,张我旗帜。贼顾见之,大溃。追奔三十里,
斩首三千余级,虏其众五万,生擒建德于阵。太宗数之曰:“我以干戈问罪,本在王世
充,得失存亡,不预汝事,何故越境,犯我兵锋?”建德股栗而言曰:“今若不来,恐
劳远取。”高祖闻而大悦,手诏曰;“隋氏分崩,崤函隔绝。两雄合势,一朝清荡。兵
既克捷,更无死伤。无愧为臣,不忧其父,并汝功也。”乃将建德至东都城下。世充惧,
率其官属二千余人诣军门请降,山东悉平。太宗入据宫城,令萧瑀、窦轨等封守府库,
一无所取,令记室房玄龄收隋图籍。于是诛其同恶段达等五十余人,枉被囚禁者悉释之,
非罪诛戮者祭而诔之。大飨将士,班赐有差。高祖令尚书左仆射裴寂劳于军中。
六月,凯旋。太宗亲披黄金甲,阵铁马一万骑,甲士三万人,前后部鼓吹,俘二伪
主及隋氏器物辇辂献于太庙。高祖大悦,行饮至礼以享焉。高祖以自古旧官不称殊功,
乃别表徽号,用旌勋德。
十月,加号天策上将、陕东道大行台,位在王公上。增邑二万户,通前三万户。赐
金辂一乘,衮冕之服,玉璧一双,黄金六千斤,前后部鼓吹及九部之乐,班剑四十人。
于时海内渐平,太宗乃锐意经籍,开文学馆以待四方之士。行台司勋郎中杜如晦等十有
八人为学士,每更直阁下,降以温颜,与之讨论经义,或夜分而罢。未几,窦建德旧将
刘黑闼举兵反,据洺州。
十二月,太宗总戎东讨。五年正月,进军肥乡,分兵绝其粮道,相持两月。黑闼窘
急求战,率步骑二万,南渡洺水,晨压官军。太宗亲率精骑,击其马军,破之,乘胜蹂
其步卒,贼大溃,斩首万余级。先是,太宗遣堰洺水上流使浅,令黑闼得渡。及战,乃
令决堰,水大至,深丈余,贼徒既败,赴水者皆溺死焉。黑闼与二百余骑北走突厥,悉
虏其众,河北平。时徐圆朗阻兵徐、兗,太宗回师讨平之,于是河、济、江、淮诸郡邑
皆平。十月,加左右十二卫大将军。
七年秋,突厥颉利、突利二可汗自原州入寇,侵扰关中。有说高祖云:“只为府藏
子女在京师,故突厥来,若烧却长安而不都,则胡寇自止。”高祖乃遣中书侍郎宇文士
及行山南可居之地,即欲移都。萧瑀等皆以为非,然终不敢犯颜正谏。太宗独曰:“霍
去病,汉廷之将帅耳,犹且志灭匈奴。臣忝备籓维,尚使胡尘不息,遂令陛下议欲迁都,
此臣之责也。幸乞听臣一申微效,取彼颉利。若一两年间不系其颈,徐建移都之策,臣
当不敢复言”。高祖怒,仍遣太宗将三十余骑行刬。还日,固奏必不可移都,高祖遂止。
八年,加中书令。
九年,皇太子建成、齐王元吉谋害太宗。六月四日,太宗率长孙无忌、尉迟敬德、
房玄龄、杜如晦、宇文士及、高士廉、侯君集、程知节、秦叔宝、段志玄、屈突通、张
士贵等于玄武门诛之。甲子,立为皇太子,庶政皆断决。太宗乃纵禁苑所养鹰犬,并停
诸方所进珍异,政尚简肃,天下大悦。又令百官各上封事,备陈安人理国之要。己巳,
令曰:“依礼,二名不偏讳。近代已来,两字兼避,废阙已多,率意而行,有违经典。
其官号、人名、公私文籍,有‘世民’两字不连续者,并不须讳。”罢幽州大都督府。
辛未,废陕东道大行台,置洛州都督府,废益州道行台,置益州大都督府。壬午,幽州
大都督庐江王瑗谋逆,废为庶人。乙酉,罢天策府。七月壬辰,太子左庶子高士廉为侍
中,右庶子房玄龄为中书令,尚书右仆射萧瑀为尚书左仆射,吏部尚书杨恭仁为雍州牧,
太子左庶子长孙无忌为吏部尚书,右庶子杜如晦为兵部尚书,太子詹事宇文士及为中书
令,封德彝为尚书右仆射。
八月癸亥,高祖传位于皇太子,太宗即位于东宫显德殿。遣司空、魏国公裴寂柴告
于南郊。大赦天下。武德元年以来责情流配者并放还。文武官五品已上先无爵者赐爵一
级,六品已下加勋一转。天下给复一年。癸酉,放掖庭宫女三千余人。甲戌,突厥颉利、
突利寇泾州。乙亥,突厥进寇武功,京师戒严。丙子,立妃长孙氏为皇后。己卯,突厥
寇高陵。辛巳,行军总管尉迟敬德与突厥战于泾阳,大破之,斩首千余级。癸未,突厥
颉利至于渭水便桥之北,遣其酋帅执失思力入朝为觇,自张形势,太宗命囚之。亲出玄
武门,驰六骑幸渭水上,与颉利隔津而语,责以负约。俄而众军继至,颉利见军容既盛,
又知思力就拘,由是大惧,遂请和,诏许焉。即日还宫。乙酉,又幸便桥,与颉利刑白
马设盟,突厥引退。九月丙戌,颉利献马三千匹、羊万口,帝不受,令颉利归所掠中国
户口。丁未,引诸卫骑兵统将等习射于显德殿庭,谓将军已下曰:“自古突厥与中国更
有盛衰。若轩辕善用五兵,即能北逐獯鬻;周宣驱驰方、召,亦能制胜太原。至汉、晋
之君,逮于隋代,不使兵士素习干戈,突厥来侵,莫能抗御,致遗中国生民涂炭于寇手。
我今不使汝等穿池筑苑,造诸淫费,农民恣令逸乐,兵士唯习弓马,庶使汝斗战,亦望
汝前无横敌。”于是每日引数百人于殿前教射,帝亲自临试,射中者随赏弓刀、布帛。
朝臣多有谏者,曰:“先王制法,有以兵刃至御所者刑之,所以防萌杜渐,备不虞也。
今引裨卒之人,弯弧纵矢于轩陛之侧,陛下亲在其间,正恐祸出非意,非所以为社稷计
也。”上不纳。自是后,士卒皆为精锐。壬子,诏私家不得辄立妖神,妄设淫祀,非礼
祠祷,一皆禁绝。其龟易五兆之外,诸杂占卜,亦皆停断。长孙无忌封齐国公,房玄龄
邢国公,尉迟敬德吴国公,杜如晦蔡国公,侯君集潞国公。
冬十月丙辰朔,日有蚀之。癸亥,立中山王承乾为皇太子。癸酉,裴寂食实封一千
五百户,长孙无忌、王君廓、尉迟敬德、房玄龄、杜如晦一千三百户,长孙顺德、柴绍、
罗艺、赵郡王孝恭一千二百户,侯君集、张公谨、刘师立一千户,李世勣、刘弘基九百
户,高士廉、宇文士及、秦叔宝、程知节七百户,安兴贵、安修仁、唐俭、窦轨、屈突
通、萧瑀、封德彝、刘义节六百户,钱九陇、樊世兴、公孙武达、李孟常、段志玄、庞
卿恽、张亮、李药师、杜淹、元仲文四百户,张长逊、张平高、李安远、李子和、秦行
师、马三宝三百户。十一月庚寅,降宗室封郡王者并为县公。十二月癸酉,亲录囚徒。
是岁,新罗、龟兹、突厥、高丽、百济、党项并遣使朝贡。
贞观元年春正月乙酉,改元。辛丑,燕郡王李艺据泾州反,寻为左右所斩,传首京
师。庚午,以仆射窦轨为益州大都督。三月癸巳,皇后亲蚕。尚书左仆射、宋国公萧瑀
为太子少师。丙午,诏:“齐故尚书仆射崔季舒、给事黄门侍郎郭遵、尚书右丞封孝琰
等,昔仕鄴中,名位通显,志存忠谠,抗表极言,无救社稷之亡,遂见龙逢之酷。其季
舒子刚、遵子云、孝琰子君遵,并以门遭时谴,淫刑滥及。宜从褒奖,特异常伦,可免
内侍,量才别叙。”
夏四月癸巳,凉州都督、长乐王幼良有罪伏诛。六月辛巳,尚书右仆射、密国公封
德彝薨。壬辰,太子少保宋国公萧瑀为尚书左仆射。是夏,山东诸州大旱,令所在赈恤,
无出今年租赋。秋七月壬子,吏部尚书、齐国公长孙无忌为尚书右仆射。八月戊戌,贬
侍中、义兴郡公高士廉为安州大都督。户部尚书裴矩卒。是月,关东及河南、陇右沿边
诸州霜害秋稼。
九月辛酉,命中书侍郎温彦博、尚书右丞魏徵等分往诸州赈恤。中书令、郢国公宇
文士及为殿中监。御史大夫、检校吏部尚书、参预朝政、安吉郡公杜淹署位。十二月壬
午,上谓侍臣曰:“神仙事本虚妄,空有其名。秦始皇非分爱好,遂为方士所诈,乃遣
童男女数千人随徐福入海求仙药,方士避秦苛虐,因留不归。始皇犹海侧踟蹰以待之,
还至沙丘而死。汉武帝为求仙,乃将女嫁道术人,事既无验,便行诛戮。据此二事,神
仙不烦妄求也。”尚书左仆射、宋国公萧瑀坐事免。戊申,利州都督义安王孝常、右武
卫将军刘德裕等谋反,伏诛。是岁,关中饥,至有鬻男女者。
二年春正月辛丑,尚书右仆射、齐国公长孙无忌为开府仪同三司。徙封汉王属为恪
王,卫王泰为越王,楚王祐为燕王。复置六侍郎,副六尚书事,并置左右司郎中各一人。
前安州大都督、赵王元景为雍州牧,蜀王恪为益州大都督,越王泰为扬州大都督。二月
丙戌,靺鞨内属。三月戊申朔,日有蚀之。丁卯,遣御史大夫杜淹巡关内诸州。出御府
金宝,赎男女自卖者还其父母。庚午,大赦天下。
夏四月己卯,诏骸骨暴露者,令所在埋瘗。丙申,契丹内属。初诏天下州县并置义
仓。夏州贼帅梁师都为其从父弟洛仁所杀,以城降。五月,大雨雹。六月庚寅,皇子治
生,宴五品以上,赐帛有差,仍赐天下是日生者粟。辛卯,上谓侍臣曰:“君虽不君,
臣不可以不臣。裴虔通,炀帝旧左右也,而亲为乱首。朕方崇奖敬义,岂可犹使宰民训
俗。”诏曰:
天地定位,君臣之义以彰;卑高既陈,人伦之道斯著。是用笃厚风俗,化成天下。
虽复时经治乱,主或昏明,疾风劲草,芬芳无绝,剖心焚体,赴蹈如归。夫岂不爱七尺
之躯,重百年之命?谅由君臣义重,名教所先,故能明大节于当时,立清风于身后。至
如赵高之殒二世,董卓之鸩弘农,人神所疾,异代同愤。况凡庸小竖,有怀凶悖,遐观
典策,莫不诛夷。辰州刺史、长蛇县男裴虔通,昔在隋代,委质晋籓,炀帝以旧邸之情,
特相爱幸。遂乃志蔑君亲,潜图弑逆,密伺间隙,招结群丑,长戟流矢,一朝窃发。天
下之恶,孰云可忍!宜其夷宗焚首,以彰大戮。但年代异时,累逢赦令,可特免极刑,
除名削爵,迁配驩州。
秋七月戊申,诏:“莱州刺史牛方裕、绛州刺史薛世良、广州都督府长史唐奉义、
隋武牙郎将高元礼,并于隋代俱蒙任用,乃协契宇文化及,构成弑逆。宜依裴虔通,除
名配流岭表。”太宗谓侍臣曰:“天下愚人,好犯宪章,凡赦宥之恩,唯及不轨之辈。
古语曰:‘小人之幸,君子之不幸。’‘一岁再赦,好人喑哑。’凡养稂莠者伤禾稼,
惠奸宄者贼良人。昔文王作罚,刑兹无赦。又蜀先主尝谓诸葛亮曰:‘吾周旋陈元方、
郑康成间,每见启告理乱之道备矣,曾不语赦也。’夫小人者,大人之贼,故朕有天下
已来,不甚放赦。今四海安静,礼义兴行,非常之恩,施不可数,将恐愚人常冀侥幸,
唯欲犯法,不能改过。”八月甲戌朔,幸朝堂,亲览冤屈。自是,上以军国无事,每日
视膳于西宫。癸巳,公卿奏曰:“依礼,季夏之月,可以居台榭。今隆暑未退,秋霖方
始,宫中卑湿,请营一阁以居之。”帝曰:“朕有气病,岂宜下湿。若遂来请,糜费良
多。昔汉文帝将起露台,而惜十家之产。朕德不逮于汉帝,而所费过之,岂谓为民父母
之道也。”竟不许。是月,河南、河北大霜,人饥。
九月丙午,诏曰:“尚齿重旧,先王以之垂范;还章解组,朝臣于是克终。释菜合
乐之仪,东胶西序之制,养老之义,遗文可睹。朕恭膺大宝,宪章故实,乞言尊事,弥
切深衷。然情存今古,世踵浇季,而策名就列,或乖大体。至若筋力将尽,桑榆且迫,
徒竭夙兴之勤,未悟夜行之罪。其有心惊止足,行堪激励,谢事公门,收骸闾里,能以
礼让,固可嘉焉。内外文武群官年高致仕、抗表去职者,参朝之日,宜在本品见任之
上。”丁未,谓侍臣曰:“妇人幽闭深宫,情实可愍。隋氏末年,求采无已,至于离宫
别馆,非幸御之所,多聚宫人,皆竭人财力,朕所不取。且洒扫之余,更何所用?今将
出之,任求伉俪,非独以惜费,亦人得各遂其性。”于是遣尚书左丞戴胄、给事中杜正
伦等,于掖庭宫西门简出之。
冬十月庚辰,御史大夫、安吉郡公杜淹卒。戊子,杀瀛州刺史卢祖尚。十一月辛酉,
有事于圆丘。十二月壬午,黄门侍郎王珪为侍中。
三年春正月辛亥,契丹渠帅来朝。戊午,谒太庙。癸亥,亲耕籍田。辛未,司空、
魏国公裴寂坐事免。二月戊寅,中书令、邢国公房玄龄为尚书左仆射,兵部尚书、检校
侍中、蔡国公杜如晦为尚书右仆射,刑部尚书、检校中书令、永康县公李靖为兵部尚书,
右丞魏徵为守秘书监,参预朝政。
夏四月辛巳,太上皇徙居大安宫。甲子,太宗始于太极殿听政。五月,周王元方薨。
六月戊寅,以旱,亲录囚徒。遣长孙无忌、房玄龄等祈雨于名山大川,中书舍人杜正伦
等往关内诸州慰抚。又令文武官各上封事,极言得失。已卯,大风折木。秋八月己巳朔,
日有蚀之。薛延陀遣使朝贡。
九月癸丑,诸州置医学。冬十一月丙午,西突厥、高昌遣使朝贡。庚申,以并州都
督李世勣为通汉道行军总管,兵部尚书李靖为定襄道行军总管,以击突厥。十二月戊辰,
突利可汗来奔。癸未,杜如晦以疾辞位,许之。癸丑,诏建义以来交兵之处,为义士勇
夫殒身戎阵者各立一寺,命虞世南、李伯药、褚亮、颜师古、岑文本、许敬宗、硃子奢
等为之碑铭,以纪功业。是岁,户部奏言:中国人自塞外来归及突厥前后内附、开四夷
为州县者,男女一百二十余万口。
------------------
本纪第三 太宗下    
四年春正月乙亥,定襄道行军总管李靖大破突厥,获隋皇后萧氏及炀帝之孙正道,
送至京师。癸巳,武德殿北院火。二月己亥,幸温汤。甲辰,李靖又破突厥于阴山,颉
利可汗轻骑远遁。丙午,至自温汤。甲寅,大赦,赐酺五日。民部尚书戴胄以本官检校
吏部尚书,参预朝政。太常卿萧瑀为御史大夫,与宰臣参议朝政。御史大夫、西河郡公
温彦博为中书令。三月庚辰,大同道行军副总管张宝相生擒颉利可汗,献于京师。甲申,
尚书右仆射、蔡国公杜如晦薨。甲午,以俘颉利告于太庙。
夏四月丁酉,御顺天门,军吏执颉利以献捷。自是西北诸蕃咸请上尊号为“天可
汗”,于是降玺书册命其君长,则兼称之。秋七月甲子朔,日有蚀之。上谓房玄龄、萧
瑀曰:“隋文何等主?”对曰:“克己复礼,勤劳思政,每一坐朝,或至日昃。五品已
上,引之论事。宿卫之人,传餐而食。虽非性体仁明,亦励精之主也。”上曰:“公得
其一,未知其二。此人性至察而心不明。夫心暗则照有不通,至察则多疑于物。自以欺
孤寡得之,谓群下不可信任,事皆自决,虽劳神苦形,未能尽合于理。朝臣既知上意,
亦复不敢直言,宰相已下,承受而已。朕意不然。以天下之广,岂可独断一人之虑?朕
方选天下之才,为天下之务,委任责成,各尽其用,庶几于理也。”因令有司:“诏敕
不便于时,即宜执奏,不得顺旨施行。”八月丙午,诏三品已上服紫,五品已上服绯,
六品七品以绿,八品九品以青;妇人从夫色。甲寅,兵部尚书、代国公李靖为尚书左仆
射。九月庚午,令收瘗长城之南骸骨,仍令致祭。壬午,令自古明王圣帝、贤臣烈士坟
墓无得刍牧,春秋致祭。
冬十月壬辰,幸陇州,曲赦陇、岐二州,给复一年。辛丑,校猎于贵泉谷。甲辰,
校猎于鱼龙川,自射鹿,献于大安宫。甲子,至自陇州。戊寅,制决罪人不得鞭背,以
明堂孔穴针灸之所。兵部尚书侯君集参议朝政。十二月辛亥,开府仪同三司、淮安王神
通薨。甲寅,高昌王麹文泰来朝。是岁,断死刑二十九人,几致刑措。东至于海,南至
于岭,皆外户不闭,行旅不赉粮焉。
五年正月癸酉,大蒐于昆明池,蕃夷君长咸从。丙子,亲献禽于大安宫。己卯,幸
左藏库,赐三品已上帛,任其轻重。癸未,朝集使请封禅。己酉,封皇弟元裕为郐王,
元名为谯王,灵夔为魏王,元祥为许王,元晓为密王。庚戌,封皇子愔为梁王,贞为汉
王,恽为郯王,治为晋王,慎为申王,嚣为江王,简为代王。
夏四月壬辰,代王简薨。以金帛购中国人因隋乱没突厥者男女八万人,尽还其家属。
六月甲寅,太子少师、新昌县公李纲薨。七月甲辰,遣使毁高丽所立京观,收隋人骸骨,
祭而葬之。戊申,初令天下决死刑必三覆奏,在京诸司五覆奏,其日尚食进蔬食,内教
坊及太常不举乐。九月乙丑,赐群官大射于武德殿。
冬十月,右卫大将军、顺州都督、北平郡王阿史那什钵苾卒。十二月壬寅,幸温汤。
癸卯,猎于骊山。丙午,赐新丰高年帛有差。戊申,至自温汤。
六年春正月乙卯朔,日有蚀之。二月丙戌,置三师官员。戊子,初置律学。
三月戊辰,幸九成宫。六月己亥,酆王元亨薨。辛亥,江王嚣薨。
冬十月乙卯,至自九成宫。十二月辛未,亲录囚徒,归死罪者二百九十人于家,令
明年秋末就刑。其后应期毕至,诏悉原之。是岁,党项羌前后内属者三十万口。
七年春正月戊子,诏曰:“宇文化及弟智及、司马德戡、裴虔通、孟景、元礼、杨
览、唐奉义、牛方裕、元敏、薛良、马举、元武达、李孝本、李孝质、张恺、许弘仁、
令狐行达、席德方、李覆等,大业季年,咸居列职,或恩结一代,任重一时;乃包藏凶
慝,罔思忠义,爰在江都,遂行弑逆,罪百阎、赵,衅深枭獍。虽事是前代,岁月已久,
而天下之恶,古今同弃,宜置重典,以励臣节。其子孙并宜禁锢,勿令齿叙。”是日,
上制《破阵乐舞图》。辛丑,赐京城酺三日。丁卯,雨土。乙酉,薛延陀遣使来朝。庚
寅,秘书监、检校侍中魏徵为侍中。癸巳,直太史、将仕郎李淳风铸浑天黄道仪,奏之,
置于凝晖阁。夏五月癸未,幸九成宫。八月,山东、河南三十州大水,遣使赈恤。
冬十月庚申,至自九成宫。十一月丁丑,颁新定《五经》。壬辰,开府仪同三司、
齐国公长孙无忌为司空。十二月丙辰,狩于少陵原,诏以少牢祭杜如晦、杜淹、李纲之
墓。
八年正月癸未,右卫大将军阿史那吐苾卒。辛丑,右屯卫大将军张士贵讨东、西五
洞反獠,平之。壬寅,命尚书右仆射李靖、特进萧瑀杨恭仁、礼部尚书王珪、御史大夫
韦挺、鄜州大都督府长史皇甫无逸、扬州大都督府长史李袭誉、幽州大都督府长史张亮、
凉州大都督李大亮、右领军大将军窦诞、太子左庶子杜正伦、绵州刺史刘德威、黄门侍
郎赵弘智使于四方,观省风俗。
二月乙巳,皇太子加元服。丙午,赐天下酺三日。三月庚辰,幸九成宫。五月辛未
朔,日有蚀之。丁丑,上初服翼善冠,贵臣服进德冠。七月,始以云麾将军阶为从三品。
陇右山崩,大蛇屡见。山东、河南、淮南大水,遣使赈恤。八月甲子,有星孛于虚、危,
历于氐,十一月上旬乃灭。九月丁丑,皇太子来朝。
冬十月,右骁卫大将军、褒国公段志玄击吐谷浑,破之,追奔八百余里。甲子,至
自九成宫。十一月辛未,右仆射、代国公李靖以疾辞官,授特进。丁亥,吐谷浑寇凉州。
己丑,吐谷浑拘我行入赵道德。十二月辛丑,命特进李靖、兵部尚书侯君集、刑部尚书
任城王道宗、凉州都督李大亮等为大总管,各帅师分道以讨吐谷浑。壬子,越王泰为雍
州牧。乙卯,帝从太上皇阅武于城西。是岁,龟兹、吐蕃、高昌、女国、石国遣使朝贡。
九年春三月,洮州羌叛,杀刺史孔长秀。壬午,大赦。每乡置长一人,佐二人。乙
酉,盐泽道总管高甑生大破叛羌之众。庚寅,敕天下户立三等,未尽升降,置为九等。
夏四月壬寅,康国献狮子。闰月丁卯,日有蚀之。癸巳,大总管李靖、侯君集、李
大亮、任城王道宗破吐谷浑于牛心堆。五月乙未,又破之于乌海,追奔至柏海。副总管
薛万均、薛万彻又破之于赤水源,获其名王二十人。庚子,太上皇崩于大安宫。壬子,
李靖平吐谷浑于西海之上,获其王慕容伏允。以其子慕容顺光降,封为西平郡王,复其
本国。秋七月甲寅,增修太庙为六室。
冬十月庚寅,葬高祖太武皇帝于献陵。戊申,祔于太庙。辛丑,左仆射、魏国公房
玄龄加开府仪同三司,余如故。十二月甲戌,吐谷浑西平郡王慕容顺光为其下所弑,遣
兵部尚书侯君集率师安抚之,仍封顺光子诺曷钵为河源郡王,使统其众。右光禄大夫、
宋国公萧瑀依旧特进,复令参预朝政。
十年春正月壬子,尚书左仆射房玄龄、侍中魏徵上梁、陈、齐、周、隋五代史,诏
藏于秘阁。癸丑,徙封赵王元景为荆王,鲁王元昌为汉王,郑王元礼为徐王,徐王元嘉
为韩王,荆王元则为彭王,滕王元懿为郑王,吴王元轨为霍王,豳王元凤为虢王,陈王
元庆为道王,魏王灵夔为燕王,蜀王恪为吴王,越王泰为魏王,燕王祐为齐王,梁王愔
为蜀王,郯王恽为蒋王,汉王贞为越王,申王慎为纪王。夏六月,以侍中魏徵为特进,
仍知门下省事。壬申,中书令温彦博为尚书右仆射。甲戌,太常卿、安德郡公杨师道为
侍中。己卯,皇后长孙氏崩于立政殿。冬十一月庚寅,葬文德皇后于昭陵。十二月壬申,
吐谷浑河源郡王慕容诺曷钵来朝。乙亥,亲录京师囚徒。是岁,关内、河东疾病,命医
赉药疗之。
十一年春正月丁亥朔,徙郐王元裕为邓王,谯王元名为舒王。癸巳,加魏王泰为雍
州牧、左武候大将军。庚子,颁新律令于天下。作飞山宫。甲寅,房玄龄等进所修《五
礼》。诏所司行用之。
二月丁巳,诏曰:
夫生者天地之大德,寿者修短之一期。生有七尺之形,寿以百龄为限,含灵禀气,
莫不同焉,皆得之于自然,不可以分外企也。是以《礼记》云:“君即位而为椑”。庄
周云:“劳我以形,息我以死。”岂非圣人远鉴,通贤深识?末代已来,明辟盖寡,靡
不矜黄屋之尊,虑白驹之过,并多拘忌,有慕遐年。谓云车易乘,羲轮可驻,异轨同趣,
其蔽甚矣。有隋之季,海内横流,豺狼肆暴,吞噬黔首。朕投袂发愤,情深拯溺,扶翼
义师,济斯涂炭。赖苍昊降鉴,股肱宣力,提剑指麾,天下大定。此朕之宿志,于斯已
毕。犹恐身后之日,子子孙孙,习于流俗,犹循常礼,加四重之榇,伐百祀之木,劳扰
百姓,崇厚园陵。今预为此制,务从俭约,于九嵕之山,足容棺而已。积以岁月,渐而
备之。木马涂车,土桴苇龠,事合古典,不为时用。
又佐命功臣,或义深舟楫,或谋定帷幄,或身摧行阵,同济艰危,克成鸿业,追念
在昔,何日忘之!使逝者无知,咸归寂寞;若营魂有识,还如畴曩,居止相望,不亦善
乎!汉氏使将相陪陵,又给以东园秘器,笃终之义,恩意深厚,古人岂异我哉!自今已
后,功臣密戚及德业佐时者,如有薨亡,宜赐茔地一所,及以秘器,使窀穸之时,丧事
无阙。所司依此营备,称朕意焉。
甲子,幸洛阳宫,命祭汉文帝。三月丙戌朔,日有蚀之。丁亥,车驾至洛阳。丙申,
改洛州为洛阳宫。辛亥,大蒐于广城泽。癸丑,还宫。
夏四月甲子,震乾元殿前槐树。丙寅,诏河北、淮南举孝悌淳笃,兼闲时务;儒术
该通,可为师范;文辞秀美,才堪著述;明识政体,可委字人:并志行修立,为乡闾所
推者,给传诣洛阳宫。六月甲寅,尚书右仆射、虞国公温彦博薨。丁巳,幸明德宫。己
未,定制诸王为世封刺史。戊辰,定制勋臣为世封刺史。改封任城王道宗为江夏郡王,
赵郡王孝恭为河间郡王。己巳,改封许王元祥为江王。秋七月癸未,大霪雨。谷水溢入
洛阳宫,深四尺,坏左掖门,毁宫寺十九所;洛水溢,漂六百家。庚寅,诏以灾命百官
上封事,极言得失。丁酉,车驾还宫。壬寅,废明德宫及飞山宫之玄圃院,分给遭水之
家,仍赐帛有差。丙午,修老君庙于亳州,宣尼庙于兗州,各给二十户享祀焉。凉武昭
王复近墓二十户充守卫,仍禁刍牧樵采。九月丁亥;河溢,坏陕州河北县,毁河阳中潭。
幸白司马坂以观之,赐遭水之家粟帛有差。冬十一月辛卯,幸怀州。乙未,狩于济源。
丙午,车驾还宫。十二月辛酉,百济王遣其太子隆来朝。
十二年春正月乙未,吏部尚书高士廉等上《氏族志》一百三十卷。壬寅,松、丛二
州地震,坏人庐舍,有压死者。二月乙卯,车驾还京。癸亥,观砥柱,勒铭以纪功德。
甲子,夜郎獠反,夔州都督齐善行讨平之。乙丑,次陕州,自新桥幸河北县,祀夏禹庙。
丁卯,次柳谷顿,观盐池。戊寅,以隋鹰扬郎将尧君素忠于本朝,赠蒲州刺史,仍录其
子孙。闰二月庚辰朔,日有蚀之。丙戌,至自洛阳宫。夏五月壬申,银青光禄大夫、永
兴县公虞世南卒。六月庚子,初置玄武门左右飞骑。秋七月癸酉,吏部尚书、申国公高
士廉为尚书右仆射。
冬十月己卯,狩于始平,赐高年粟帛有差。乙未,至自始平。己亥,百济遣使贡金
甲雕斧。十二月辛巳,右武候将军上官怀仁大破山獠于壁州。
十三年春正月乙巳朔,谒献陵。曲赦三原县及行从大辟罪。丁未,至自献陵。戊午,
加房玄龄为太子少师。二月丙子,停世袭刺史。三月乙丑,有星孛于毕、昴。
夏四月戊寅,幸九成宫。甲申,阿史那结社尔犯御营,伏诛。壬寅,云阳石燃者方
丈,昼如灰,夜则有光,投草木于上则焚,历年而止。自去冬不雨至于五月。甲寅,避
正殿,令五品以上上封事,减膳罢役,分使赈恤,申理冤屈,乃雨。
六月丙申,封皇弟元婴为滕王。秋八月辛未朔,日有蚀之。庚辰,立右武候大将军、
化州都督、怀化郡王李思摩为突厥可汗,率所部建牙于河北。
冬十月甲申,至自九成宫。十一月辛亥,侍中、安德郡公杨师道为中书令。十二月
丁丑,吏部尚书、陈国公侯君集为交河道行军大总管,帅师伐高昌。乙亥,封皇子福为
赵王。壬午,巂州都督王志远有罪,伏诛。诏于洛、相、幽、徐、齐、并、秦、蒲等州
并置常平仓。己丑,吐谷浑河源郡王慕容诺曷钵来逆女。壬辰,狩于咸阳。是岁,滁州
言:“野蚕食槲叶,成茧大如柰,其色绿,凡六千五百七十石。”高丽、新罗、西突厥、
吐火罗、康国、安国、波斯、疏勒、于阗、焉耆、高昌、林邑、昆明及荒服蛮酋,相次
遣使朝贡。
十四年春正月庚子,初命有司读时令。甲寅,幸魏王泰宅。赦雍州及长安狱大辟罪
已下。二月丁丑,幸国子学,亲释奠,赦大理、万年系囚,国子祭酒以下及学生高第精
勤者加一级,赐帛有差。庚辰,左骁卫将军、淮阳王道明送弘化公主归于吐谷浑。壬午,
幸温汤。辛卯,至自温汤。乙未,诏以梁皇侃、褚仲都,周熊安生、沈重,陈沈文阿、
周弘正、张机,隋何妥、刘焯、刘炫等前代名儒,学徒多行其义,命求其后。
三月戊午,置宁朔大使,以护突厥。夏五月壬戌,徙封燕王灵夔为鲁王。六月乙酉,
大风拔木。己丑,薛延陀遣使求婚。乙未,滁州野蚕成茧,凡收八千三百石。八月庚午,
新作襄城宫。癸巳,交河道行军大总管侯君集平高昌,以其地置西州。九月癸卯,曲赦
西州大辟罪。乙卯,于西州置安西都护府。冬十月己卯,诏以赠司空、河间元王孝恭,
赠陕东道大行台尚书右仆射、郧节公殷开山,赠民部尚书、渝襄公刘政会等配飨高祖庙
庭。闰月乙未,幸同州。甲辰,狩于尧山。庚戌,至自同州。丙辰,吐蕃遣使献黄金器
千斤以求婚。
十一月甲子朔,日南至。有事于圆丘。十二月丁酉,交河道旋师。吏部尚书、陈国
公侯君集执高昌王麹智盛,献捷于观德殿,行饮至之礼,赐酺三日。乙卯,高丽世子相
权来朝。
十五年春正月丁卯,吐蕃遣其国相禄东赞来逆女。丁丑,礼部尚书、江夏王道宗送
文成公主归吐蕃。辛巳,幸洛阳宫。三月戊申,幸襄城宫。庚午,发襄城宫。
夏四月辛卯,诏以来年二月有事泰山,所司详定仪制。五月壬申,并州僧道及老人
等抗表,以太原王业所因,明年登封已后,愿时临幸。上于武成殿赐宴,因从容谓侍臣
曰:“朕少在太原,喜群聚博戏,暑往寒逝,将三十年矣。”时会中有旧识上者,相与
道旧以为笑乐。因谓之曰:’他人之言,或有面谀。公等朕之故人,实以告朕,即日政
教,于百姓何如?人间得无疾苦耶?”皆奏:“即日四海太平,百姓欢乐,陛下力也。
臣等余年,日惜一日,但眷恋圣化,不知疾苦。”因固请过并州。上谓曰:“飞鸟过故
乡,犹踯躅徘徊;况朕于太原起义,遂定天下,复少小游观,诚所不忘。岱礼若毕,或
冀与公等相见。”于是赐物各有差。丙子,百济王扶余璋卒。诏立其世子扶余义慈嗣其
父位,仍封为带方郡王。
六月戊申,诏天下诸州,举学综古今及孝悌淳笃、文章秀异者,并以来年二月总集
泰山。己酉,有星孛于太微,犯郎位。丙辰,停封泰山,避正殿以思咎,命食减膳。
秋七月甲戌,孛星灭。
冬十月辛卯,大阅于伊阙。壬辰,幸嵩阳。辛丑,还宫。十一月壬戌,废乡长。壬
申,还京师。癸酉,薛延陀以同罗、仆骨、回纥、靺鞨、之众度漠,屯于白道川。命
营州都督张俭统所部兵压其东境;兵部尚书李勣为朔方行军总管,右卫大将军李大亮为
灵州道行军总管,凉州都督李袭誉为凉州道行军总管,分道以御之。十二月戊子朔,至
自洛阳宫。甲辰,李勣及薛延陀战于诺真水,大破之,斩首三千余级,获马万五千匹,
薛延陀跳身而遁。勣旋破突厥思结于五台县,虏其男女千余口,获羊马称是。
十六年春正月辛未,诏在京及诸州死罪囚徒,配西州为户;流人未达前所者,徙防
西州。兼中书侍郎、江陵子岑文本为中书侍郎,专知机密。夏六月辛卯,诏复隐王建成
曰隐太子,改封海陵剌王元吉曰巢剌主。秋七月戊午,司空、赵国公无忌为司徒,尚书
左仆射、梁国公玄龄为司空。
九月丁巳,特进、郑国公魏徵为太子太师,知门下省事如故。冬十一月丙辰,狩于
岐山。辛酉,使祭隋文帝陵。丁卯,宴武功士女于庆善宫南门。酒酣,上与父老等涕泣
论旧事,老人等递起为舞,争上万岁寿,上各尽一杯。庚午,至自岐州。十二月癸卯,
幸温汤。甲辰,狩于骊山,时阴寒晦冥,围兵断绝。上乘高望见之,欲舍其罚,恐亏军
令,乃回辔入谷以避之。是岁,高丽大臣盖苏文弑其君高武,而立武兄子藏为王。
十七年春正月戊辰,右卫将军、代州都督刘兰谋反,腰斩。太子太师、郑国公魏徵
薨。戊申,诏图画司徒、赵国公无忌等勋臣二十四人于凌烟阁。三月丙辰,齐州都督齐
王祐杀长史权万纪、典军韦文振,据齐州自守,诏兵部尚书李勣、刑部尚书刘德威发兵
讨之。兵未至,兵曹杜行敏执之而降,遂赐死于内侍省。丁巳,荧惑守心前星,十九日
而退。
夏四月庚辰朔,皇太子有罪,废为庶人。汉王元昌、吏部尚书侯君集并坐与连谋,
伏诛。丙戌,立晋王治为皇太子,大赦,赐酺三日。丁亥,中书令杨师道为吏部尚书。
己丑,加司徒、赵国公长孙无忌太子太师,司空、梁国公房玄龄太子太傅;特进、宋国
公萧瑀太子太保,兵部尚书、英国公李勣为太子詹事,仍同中书门下三品。庚寅,上亲
谒太庙,以谢承乾之过。癸巳,魏王泰以罪降爵为东莱郡王。五月乙丑,手诏举孝廉茂
才异能之士。
六月己卯朔,日有蚀之。壬午,改葬隋恭帝。丁酉,尚书右仆射高士廉请致仕,诏
以为开府仪同三司、同中书门下三品。闰月戊午,薛延陀遣其兄子突利设献马五万匹、
牛驼一万、羊十万以请婚,许之。丙子,徙封东莱郡王泰为顺阳王。秋七月庚辰,京城
讹言云:“上遣枨枨取人心肝,以祠天狗。”递相惊悚。上遣使遍加宣谕,月余乃止。
丁酉,司空、太子太傅、梁国公房玄龄以母忧罢职。八月,工部尚书、郧国公张亮为刑
部尚书,参预朝政。九月癸未,徙庶人承乾于黔州。
冬十月丁巳,房玄龄起复本职。十一月己卯,有事于南郊。壬午,赐天下酺三日。
以凉州获瑞石,曲赦凉州,并录京城及诸州系囚,多所原宥。
十八年春正月壬寅,幸温汤。
夏四月辛亥,幸九成宫。秋八月甲子,至自九成宫。丁卯,散骑常侍清苑男刘洎为
侍中,中书侍郎江陵子岑文本、中书侍郎马周并为中书令。九月,黄门侍郎褚遂良参预
朝政。冬十月辛丑朔,日有蚀之。甲辰,初置太子司议郎官员。甲寅,幸洛阳宫。安西
都护郭孝恪帅师灭焉耆,执其王突骑支送行在所。十一月壬寅,车驾至洛阳宫。庚子,
命太子詹事、英国公李勣为辽东道行军总管,出柳城,礼部尚书、江夏郡王道宗副之;
刑部尚书、郧国公张亮为平壤道行军总管,以舟师出莱州,左领军常何、泸州都督左难
当副之。发天下甲士,召募十万,并趣平壤,以伐高丽。十二月辛丑,庶人承乾死。
十九年春二月庚戌,上亲统六军发洛阳。乙卯,诏皇太子留定州监国;开府仪同三
司、申国公高士廉摄太子太傅,与侍中刘洎、中书令马周、太子少詹事张行成、太子右
庶子高季辅五人同掌机务;以吏部尚书、安德郡公杨师道为中书令。赠殷比干为太师,
谥曰忠烈,命所司封墓,葺祠堂,春秋祠以少牢,上自为文以祭之。三月壬辰,上发定
州,以司徒、太子太师兼检校侍中、赵国公长孙无忌,中书令岑文本、杨师道从。
夏四月癸卯,誓师于幽州城南,因大飨六军以遣之。丁未,中书令岑文本卒于师。
癸亥,辽东道行军大总管、英国公李勣攻盖牟城,破之。五月丁丑,车驾渡辽。甲申,
上亲率铁骑与李勣会围辽东城,因烈风发火弩,斯须城上屋及楼皆尽,麾战士令登,乃
拔之。
六月丙辰,师至安市城。丁巳,高丽别将高延寿、高惠真帅兵十五万来援安市,以
拒王师。李勣率兵奋击,上自高峰引军临之,高丽大溃,杀获不可胜纪。延寿等以其众
降,因名所幸山为驻跸山,刻石纪功焉。赐天下大酺二日。秋七月,李勣进军攻安市城,
至九月不克,乃班师。
冬十月丙辰,入临渝关,皇太子自定州迎谒。戊午,次汉武台,刻石以纪功德。十
一月辛未,幸幽州。癸酉,大飨,还师。十二月戊申,幸并州。侍中、清苑男刘洎以罪
赐死。是岁,薛延陀真珠毗伽可汗死。
二十年春正月,上在并州。丁丑,遣大理卿孙伏伽、黄门侍郎褚遂良等二十二人,
以六条巡察四方,黜陟官吏。庚辰,曲赦并州,宴从官及起义元从,赐粟帛、给复有差。
三月己巳,车驾至京师。己丑,刑部尚书、郑国公张亮谋反,诛。闰月癸巳朔,日有蚀
之。
夏四月甲子,太子太师、赵国公长孙无忌,太子太傅、梁国公房玄龄,太子太保、
宋国公萧瑀各辞调护之职,诏许之。六月,遣兵部尚书、固安公崔敦礼,特进、英国公
李勣击破薛延陀于郁督军山北,前后斩首五千余级,虏男女三万余人。秋八月甲子,封
皇孙为陈王。己巳,幸灵州。庚午,次泾阳顿。铁勒回纥、拔野古、同罗、仆骨、多滥
葛、思结、阿跌、契苾、跌结、浑、斛薛等十一姓各遣使朝贡,奏称:“延陀可汗不事
大国,部落乌散,不知所之。奴等各有分地,不能逐延陀去,归命天子,乞置汉官。”
诏遣会灵州。九月甲辰,铁勒诸部落俟斤、颉利发等遣使相继而至灵州者数千人,来贡
方物,因请置吏,咸请至尊为可汗。于是北荒悉平,为五言诗勒石以序其事。辛亥,灵
州地震有声。
冬十月,前太子太保、宋国公萧瑀贬商州刺史。丙戌,至自灵州。
二十一年春正月壬辰,开府仪同三司、申国公高士廉薨。丁酉,诏以来年二月有事
泰山。甲寅,赐京师酺三日。二月壬申,诏以左丘明、卜子夏、公羊高、谷梁赤、伏胜、
高堂生、戴圣、毛苌、孔安国、刘向、郑众、杜子春、马融、卢植、郑康成、服子慎、
何休、王肃、王辅嗣、杜元凯、范宁等二十一人,代用其书,垂于国胄,自今有事于太
学,并命配享宣尼庙堂。丁丑,皇太子于国学释菜。
夏四月乙丑,营太和宫于终南之上,改为翠微宫。五月戊子,幸翠微宫。六月癸亥,
司徒、赵国公无忌加授扬州都督。秋七月庚子,建玉华宫于宜君县之凤凰谷。庚戌,至
自翠微宫。八月壬戌,诏以河北大水,停封禅。辛未,骨利干国遣使贡名马。丁酉,封
皇子明为曹王。冬十一月癸卯,徙封顺阳王泰为濮王。十二月戊寅,左骁卫大将军阿史
那社尔、右骁卫大将军契苾何力、安西都护郭孝恪、司农卿杨弘礼为琯山道行军大总管,
以伐龟兹。是岁,堕婆登、乙利、鼻林送、都播、羊同、石、波斯、康国、吐火罗、阿
悉吉等远夷十九国,并遣使朝贡。又于突厥之北至于回纥部落,置驿六十六所,以通北
荒焉。
二十二年春正月庚寅,中书令马周卒。司徒、赵国公无忌兼检校中书令,知尚书门
下二省事。已亥,刑部侍郎崔仁师为中书侍郎,参知机务。戊戌,幸温汤。戊申,还宫。
二月,前黄门侍郎褚遂良起复黄门侍郎。中书侍郎崔仁师除名,配流连州。癸丑,西番
沙钵罗叶护率众归附,以其俟斤屈裴禄为忠武将军,兼大俟斤。戊午,以结骨部置坚昆
都督。乙亥,幸玉华宫,乙卯,赐所经高年笃疾粟帛有差。己卯,蒐于华原。
四月甲寅,碛外蕃人争牧马出界,上亲临断决,然后咸服。丁巳,右武候将军梁建
方击松外蛮,下其部落七十二所。五月庚子,右卫率长史王玄策击帝那伏帝国,大破之,
获其王阿罗那顺及王妃、子等,虏男女万二千人、牛马二万余以诣阙。使方土那罗迩娑
婆于金飚门造延年之药。吐蕃赞普击破中天竺国,遣使献捷。六月癸酉,特进、宋国公
萧瑀薨。秋七月癸卯,司空、梁国公房玄龄薨。八月己酉朔,日有蚀之。九月己亥,黄
门侍郎褚遂良为中书令。
十月癸亥,至自玉华宫。十一月戊戌,眉、邛、雅三州獠反,右卫将军梁建方讨平
之。庚子,契丹帅窟哥、奚帅可度者并率其部内属。以契丹部为松漠都督,以奚部置饶
乐都督。十二月乙卯,增置殿中侍御史、监察御史各二员,大理寺置平事十员。闰月丁
丑朔,昆山道总管阿史那社尔降处密、处月,破龟兹大拨等五十城,虏数万口,执龟兹
王诃黎布失毕以归,龟兹平,西域震骇。副将薛万彻胁于阗王伏阇信入朝。癸未,新罗
王遣其相伊赞千金春秋及其子文王来朝。是岁,新罗女王金善德死,遣册立其妹真德为
新罗王。
二十三年春正月辛亥,俘龟兹王诃黎布失毕及其相那利等,献于社庙。二月丙戌,
置瑶池都督府,隶安西都护府。丁亥,西突厥肆叶护可汗遣使来朝。三月丙辰,置丰州
都督府。自去冬不雨,至于此月己未乃雨。辛酉,大赦。丁卯,敕皇太子于金液门听政。
是月,日赤无光。
四月己亥,幸翠微宫。五月戊午,太子詹事、英国公李勣为叠州都督。辛酉,开府
仪同三司、卫国公李靖薨。己巳,上崩于含风殿,年五十二。遗诏皇太子即位于柩前,
丧纪宜用汉制。秘不发丧。庚午,遣旧将统飞骑劲兵从皇太子先还京,发六府甲士四千
人,分列于道及安化门,翼从乃入;大行御马舆,从官侍御如常。壬申,发丧。六月甲
戌朔,殡于太极殿。八月丙子,百僚上谥曰文皇帝,庙号太宗。庚寅,葬昭陵。上元元
年八月,改上尊号曰文武圣皇帝。天宝十三载二月,改上尊号为文武大圣大广孝皇帝。
史臣曰:臣观文皇帝发迹多奇,聪明神武。拔人物则不私于党,负志业则咸尽其才。
所以屈突、尉迟,由仇敌而愿倾心膂;马周、刘洎,自疏远而卒委钧衡。终平泰阶,谅
由斯道。尝试论之:础润云兴,虫鸣螽跃。虽尧、舜之圣,不能用檮杌、穷奇而治平;
伊、吕之贤,不能为夏桀、殷辛而昌盛。君臣之际,遭遇斯难,以至抉目剖心,虫流筋
擢,良由遭值之异也。以房、魏之智,不逾于丘、轲,遂能尊主庇民者,遭时也。或曰:
以太宗之贤,失爱于昆弟,失教于诸子,何也?曰:然,舜不能仁四罪,尧不能训丹硃,
斯前志也。当神尧任谗之年,建成忌功之日,苟除畏逼,孰顾分崩,变故之兴,间不容
发,方惧“毁巢”之祸,宁虞“尺布”之谣?承乾之愚,圣父不能移也。若文皇自定储
于哲嗣,不骋志于高丽;用人如贞观之初,纳谏比魏徵之日。况周发、周成之世袭,我
有遗妍;较汉文、汉武之恢弘,彼多惭德。迹其听断不惑,从善如流,千载可称,一人
而已!
赞曰:昌、发启国,一门三圣。文定高位,友于不令。管、蔡既诛,成、康道正。
贞观之风,到今歌咏。
------------------

\end{document}