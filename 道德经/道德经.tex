%-*- coding: UTF-8 -*-
% 道德经
% 道德经.tex

\documentclass[a4paper,12pt,UTF8,twoside]{ctexbook}

% 设置纸张信息。
\RequirePackage[a4paper]{geometry}
\geometry{
	%textwidth=138mm,
	%textheight=215mm,
	%left=27mm,
	%right=27mm,
	%top=25.4mm, 
	%bottom=25.4mm,
	%headheight=2.17cm,
	%headsep=4mm,
	%footskip=12mm,
	%heightrounded,
	inner=1in,
	outer=1.25in
}

% 设置字体,并解决显示难检字问题。
\xeCJKsetup{AutoFallBack=true}
\setCJKmainfont{SimSun}[BoldFont=SimHei, ItalicFont=KaiTi, FallBack=SimSun-ExtB]

% 目录 chapter 级别加点(.)。
\usepackage{titletoc}
\titlecontents{chapter}[0pt]{\vspace{3mm}\bf\addvspace{2pt}\filright}{\contentspush{\thecontentslabel\hspace{0.8em}}}{}{\titlerule*[8pt]{.}\contentspage}

% 设置 part 和 chapter 标题格式。
\ctexset{
	part/name={},
	part/number={},
	chapter/name={第,章},
	chapter/number={\chinese{chapter}}
}

% 设置古文原文格式。
\newenvironment{yuanwen}{\bfseries\zihao{4}}

\title{\heiti\zihao{0} 道德经}
\author{老子}
\date{}

\begin{document}

	\maketitle

	\tableofcontents
	
	\frontmatter
	\chapter{前言}
	
	老子,姓李名耳,字聃,一字或曰谥伯阳。华夏族, 楚国苦县厉乡曲仁里(今河南省鹿邑县太清宫镇)人,约生活于前571年至471年之间。是我国古代伟大的哲学家和思想家、道家学派创始人,被唐朝帝王追认为李姓始祖。老子故里鹿邑县亦因老子先后由苦县更名为真源县、卫真县、鹿邑县,并在鹿邑县境内留下许多与老子息息相关的珍贵文物。老子乃世界文化名人,世界百位历史名人之一,存世有《道德经》(又称《老子》),其作品的精华是朴素的辩证法,主张无为而治,其学说对中国哲学发展具有深刻影响。在道教中,老子被尊为道教始祖。老子与后世的庄子并称老庄。
	
	《老子》,又称《道德真经》《道德经》《五千言》《老子五千文》,是中国古代先秦诸子分家前的一部著作,为其时诸子所共仰,传说是春秋时期的老子李耳(似是作者、注释者、传抄者的集合体)所撰写,是道家哲学思想的重要来源。道德经分上下两篇,原文上篇《德经》、下篇《道经》,不分章,后改为《道经》37章在前,第38章之后为《德经》,并分为81章。是中国历史上首部完整的哲学著作。 
	
	\mainmatter
	
	\part{道经}
	
	\chapter{论道}
	
	道可道,非常道;名可名,非常名。无名天地之始,有名万物之母。故常无欲,以观其妙;常有欲,以观其徼(jiào)。此两者同出而异名,同谓之玄,玄之又玄,众妙之门。
	
	道,可道也,非恒道也。名,可名也,非恒名也。 “无”,名天地之始;“有”,名万物之母。 故,常“无”,欲以观其妙;常“有”,欲以观其徼。 此两者,同出而异名,同谓之玄。玄之又玄,眾妙之门。
	
	“道”如果可以用言语来表述,那它就是常“道”(“道”是可以用言语来表述的,它并非一般的“道”);“名”如果可以用文辞去命名,那它就是常“名”(“名”也是可以说明的,它并非普通的“名”)。“无”可以用来表述天地浑沌未开之际的状况;而“有”,则是宇宙万物产生之本原的命名。因此,要常从“无”中去观察领悟“道”的奥妙;要常从“有”中去观察体会“道”的端倪。无与有这两者,来源相同而名称相异,都可以称之为玄妙、深远。它不是一般的玄妙、深奥,而是玄妙又玄妙、深远又深远,是宇宙天地万物之奥妙的总门(从“有名”的奥妙到达无形的奥妙,“道”是洞悉一切奥妙变化的门径)。
	
	注释
	
	①第一个“道”是名词,指的是宇宙的本原和实质,引申为原理、原则、真理、规律等。第二个“道”是动词。指解说、表述的意思,犹言“说得出”。
	
	②恒:一般的,普通的。
	
	③第一个“名”是名词,指“道”的形态。第二个“名”是动词,说明的意思。
	
	④无名:指无形。
	
	⑤有名:指有形。
	
	⑥母:母体,根源。
	
	⑦恒:经常。
	
	⑧眇(miao):通妙,微妙的意思。
	
	⑨徼(jiao):边际、边界。引申端倪的意思。
	
	⑩谓:称谓。此为“指称”。
	
	⑾玄:深黑色,玄妙深远的含义。
	
	⑿门:之门,一切奥妙变化的总门径,此用来比喻宇宙万物的唯一原“道”的门径。
	
	\chapter{美善}
	
	天下皆知美之为美,斯恶(è)已;皆知善之为善,斯不善已。故有无相生,难易相成,长短相较,高下相倾,音声相和(hè),前后相随。是以圣人处无为之事,行不言之教,万物作焉而不辞,生而不有,为而不恃,功成而弗居。夫(fú)唯弗居,是以不去。
	
	\chapter{无为}
	
	不尚贤,使民不争;不贵难得之货,使民不为盗;不见(xiàn)可欲,使民心不乱。是以圣人之治,虚其心,实其腹;弱其志,强其骨。常使民无知无欲,使夫(fú)智者不敢为也。为无为,则无不治。
	
	\chapter{道沖}
	道冲而用之或不盈,渊兮似万物之宗。挫其锐,解其纷,和其光,同其尘。湛兮似或存,吾不知谁之子,象帝之先。
	
	
	
	\chapter{守中}
	
	天地不仁,以万物为刍(chú)狗;圣人不仁,以百姓为刍狗。天地之间,其犹橐龠(tuó	yuè)乎?虚而不屈,动而愈出。多言数(shuò)穷,不如守中。
	
	
	
	
	\chapter{谷神}
	【第六章】谷神不死,是谓玄牝(pìn),玄牝之门,是谓天地根。绵绵若存,用之不勤。 〖译文〗
	
	
	
	
	
	\chapter{无私}
	【第七章】天长地久。天地所以能长且久者,以其不自生,故能长生。是以圣人后其身而身先,外其身而
	
	身存。非以其无私邪(yé)?故能成其私。 〖译文〗
	
	
	
	
	\chapter{上善}
	【第八章】上善若水。水善利万物而不争,处众人之所恶(wù),故几(jī)于道。居善地,心善渊,与
	
	善仁,言善信,正善治,事善能,动善时。夫唯不争,故无尤。 〖译文〗
	
	
	
	
	\chapter{持盈}
	
	持而盈之,不如其已。揣(chuǎi)而锐之,不可长保。金玉满堂,莫之能守。富贵而骄,自遗(yí)其咎。功成身退,天之道。	

	\chapter{玄德}
		
	载(zài)营魄抱一,能无离乎?专气致柔,能婴儿乎?涤除玄览,能无疵乎?爱民治国,能无知(zhì)乎?天门开阖(hé),能无雌乎?明白四达,能无为乎?生之、畜(xù)之,生而不有,为而不恃,长(zhǎng)而不宰,是谓玄德。
	
	\chapter{利用}
	
	三十辐共一毂(gǔ),当其无,有车之用。埏埴(shān zhí)以为器,当其无,有器之用。凿户牖(yǒu)以为室,当其无,有室之用。故有之以为利,无之以为用。
	
	\chapter{为腹}
	
	五色令人目盲,五音令人耳聋,五味令人口爽,驰骋畋(tián)猎令人心发狂,难得之货令人行妨。是以圣人为腹不为目,故去彼取此。
	
	\chapter{宠贵}
	
	宠辱若惊,贵大患若身。何谓宠辱若惊?宠为下,得之若惊,失之若惊,是谓宠辱若惊。何谓贵大患若身?吾所以有大患者,为吾有身,及吾无身,吾有何患!故贵以身为天下,若可寄天下;爱以身为天下,若可托天下。
	
	\chapter{道纪}
	【第十四章】视之不见名曰夷,听之不闻名曰希,搏之不得名曰微。此三者不可致诘(jié),故混(
	
	hùn)而为一。其上不皦(jiǎo皎),其下不昧。绳绳(mǐn mǐn )不可名,复归于无物,是谓无状之状,无
	
	物之象。是谓惚恍。迎之不见其首,随之不见其后。执古之道,以御今之有,能知古始,是谓道纪。 〖译文〗
	
	
	
	
	\chapter{保盈}
	
	【第十五章】古之善为士者,微妙玄通,深不可识。夫唯不可识,故强(qiǎng)为之容。豫焉若冬涉川,犹
	
	兮若畏四邻,俨兮其若容,涣兮若冰之将释,敦兮其若朴,旷兮其若谷,混兮其若浊。孰能浊以静之徐清
	
	?孰能安以久动之徐生?保此道者不欲盈,夫唯不盈,故能蔽不新成。 〖译文〗
	
	
	
	\chapter{虚静}
	
	【第十六章】致虚极,守静笃(dǔ),万物并作,吾以观复。夫物芸芸,各复归其根。归根曰静,是谓复
	
	命。复命曰常,知常曰明,不知常,妄作,凶。知常容,容乃公,公乃王(wàng),王(wàng)乃天,天
	
	乃道,道乃久,没(mò)身不殆。 〖译文〗
	
	

	
	
	
	\chapter{太上}
	【第十七章】太上,下知有之。其次,亲而誉之。其次,畏之。其次,侮之。信不足焉,有不信焉。悠兮
	
	其贵言。功成事遂,百姓皆谓我自然。 〖译文〗
	
	\chapter{大道}
	
	【第十八章】大道废,有仁义;慧智出,有大伪;六亲不和,有孝慈;国家昏乱,有忠臣。 〖译文〗
	
	
	
	【
	
	\chapter{三绝}
	
	第十九章】绝圣弃智,民利百倍;绝仁弃义,民复孝慈;绝巧弃利,盗贼无有。此三者,以为文不足,
	
	故令有所属,见(xiàn)素抱朴,少私寡欲。 〖译文〗
	
	
	
	
	\chapter{绝学}
	【第二十章】绝学无忧。唯之与阿(ē),相去几何?善之与恶,相去若何?人之所畏,不可不畏。荒兮其
	
	未央哉!众人熙熙,如享太牢,如春登台。我独泊兮其未兆,如婴儿之未孩。傫傫(lěi)兮若无所归。众
	
	人皆有余,而我独若遗。我愚人之心也哉!沌沌兮!俗人昭昭,我独昏昏;俗人察察,我独闷闷。澹(
	
	dàn)兮其若海,飂(liù)兮若无止。众人皆有以,而我独顽似鄙。我独异于人,而贵食(sì)母。 〖译文〗
	
	
	
	
	\chapter{孔德}
	【第二十一章】孔德之容,惟道是从。道之为物,惟恍惟惚。惚兮恍兮,其中有象;恍兮惚兮,其中有物
	
	。窈(yǎo)兮冥兮,其中有精;其精甚真,其中有信。自古及今,其名不去,以阅众甫。吾何以知众甫之
	
	状哉?以此。 〖译文〗
	
	
	
	
	
	\chapter{全归}
	【第二十二章】曲则全,枉则直,洼则盈,敝则新,少则得,多则惑。是以圣人抱一,为天下式。不自见
	
	(xiàn)故明,不自是故彰,不自伐故有功,不自矜故长。夫唯不争,故天下莫能与之争。古之所谓曲则
	
	全者,岂虚言哉!诚全而归之。 〖译文〗
	
	
	
	\chapter{自然}
	
	【第二十三章】希言自然。故飘风不终朝(zhāo),骤雨不终日。孰为此者?天地。天地尚不能久,而况
	
	于人乎?故从事于道者,道者同于道,德者同于德,失者同于失。同于道者,道亦乐得之;同于德者,德
	
	亦乐得之;同于失者,失亦乐得之。信不足焉,有不信焉。 〖译文〗
	
	
	
	\chapter{跂跨}
	
	【第二十四章】企者不立,跨者不行,自见(xiàn)者不明,自是者不彰,自伐者无功,自矜者不长。其
	
	在道也,曰余食赘(zhuì)行。物或恶(wù)之,故有道者不处(chǔ)。 〖译文〗
	
	
	
	\chapter{混成}
	
	【第二十五章】有物混(hùn)成,先天地生。寂兮寥兮,独立不改,周行而不殆,可以为天下母。吾不知
	
	其名,字之曰道,强(qiǎng)为之名曰大。大曰逝,逝曰远,远曰反。故道大,天大,地大,王亦大。域中
	
	有四大,而王居其一焉。人法地,地法天,天法道,道法自然。 〖译文〗
	
	
	

	\chapter{重静}
		【第二十六章】重为轻根,静为躁君。是以圣人终日行不离辎(zī)重。虽有荣观(guàn),燕处超然,
	
	奈何万乘(shèng)之主,而以身轻天下?轻则失本,躁则失君。 〖译文〗
	
	
	
	
	\chapter{要妙}
	
	【第二十七章】善行无辙迹,善言无瑕谪(xiá zhé),善数(shǔ)不用筹策,善闭无关楗(jiàn)而不可
	
	开,善结无绳约而不可解。是以圣人常善救人,故无弃人;常善救物,故无弃物,是谓袭明。故善人者,
	
	不善人之师;不善人者,善人之资。不贵其师,不爱其资,虽智大迷,是谓要妙。 〖译文〗
	
	
	
	\chapter{常德}
	
	【第二十八章】知其雄,守其雌,为天下溪。为天下溪,常德不离,复归于婴儿。知其白,守其黑,为天
	
	下式。为天下式,常德不忒(tè),复归于无极。知其荣,守其辱,为天下谷。为天下谷,常德乃足,复
	
	归于朴。朴散则为器,圣人用之则为官长(zhǎng)。故大制不割。 〖译文〗
	
	

	\chapter{神器}
		
	【第二十九章】将欲取天下而为之,吾见其不得已。天下神器,不可为也。为者败之,执者失之。故物或
	
	行或随,或歔(xū)或吹,或强或羸(léi),或挫或隳(huī)。是以圣人去甚,去奢,去泰。 〖译文〗
	
	
	

	\chapter{兵强}
		【第三十章】以道佐人主者,不以兵强天下,其事好(hào)还。师之所处,荆棘生焉。大军之后,必有凶
	
	年。善有果而已,不敢以取强。果而勿矜,果而勿伐,果而勿骄,果而不得已,果而勿强。物壮则老,是
	
	谓不道,不道早已。 〖译文〗
	
	
	
	
	\chapter{佳兵}
	【第三十一章】夫佳兵者,不祥之器。物或恶(wù)之,故有道者不处(chǔ)。君子居则贵左,用兵则贵
	
	右。兵者,不祥之器,非君子之器。不得已而用之,恬淡为上,胜而不美。而美之者,是乐(yào)杀人。夫
	
	乐(yào)杀人者,则不可以得志于天下矣。吉事尚左,凶事尚右。偏将军居左,上将军居右,言以丧(
	
	sāng)礼处之。杀人之众,以哀悲泣之,战胜,以丧礼处之。 〖译文〗
	
	
	
	
	\chapter{无名}
	【第三十二章】道常无名,朴虽小,天下莫能臣也。侯王若能守之,万物将自宾。天地相合以降甘露,民
	
	莫之令而自均。始制有名,名亦既有,夫亦将知止。知止可以不殆。譬道之在天下,犹川谷之于江海。 〖译文〗
	
	
	
	\chapter{明强}
	
	【第三十三章】知人者智,自知者明。胜人者有力,自胜者强。知足者富,强行者有志,不失其所者久,
	
	死而不亡者寿。 〖译文〗
	
	
	
	\chapter{大道}
	
	【第三十四章】大道泛兮,其可左右。万物恃之而生而不辞,功成不名有,衣养万物而不为主,常无欲,
	
	可名于小;万物归焉而不为主,可名为大。以其终不自为大,故能成其大。 〖译文〗
	
	
	
	
	\chapter{大象}
	【第三十五章】执大象,天下往;往而不害,安平太。乐(yuè)与饵,过客止。道之出口,淡乎其无味,
	
	视之不足见(jiàn),听之不足闻,用之不足既。 〖译文〗
	
	
	
	
	\chapter{微明}
	【第三十六章】将欲歙(xī)之,必固张之;将欲弱之,必固强之;将欲废之,必固兴之;将欲夺之,必
	
	固与之,是谓微明。柔弱胜刚强。鱼不可脱于渊,国之利器不可以示人。 〖译文〗
	\chapter{静正}
	
		
	
	
	
	
	
	
	
	
	【第三十七章】道常无为而无不为,侯王若能守之,万物将自化。化而欲作,吾将镇之以无名之朴。无名
	
	之朴,夫亦将无欲。不欲以静,天下将自定。 〖译文〗
	
	
	
	
	
	
	\part{德经}
	
	\chapter{上德}
	【第三十八章】上德不德,是以有德;下德不失德,是以无德。上德无为而无以为,下德为之而有以为。
	
	上仁为之而无以为,上义为之而有以为,上礼为之而莫之应,则攘(rǎng)臂而扔之。故失道而后德,失德
	
	而后仁,失仁而后义,失义而后礼。夫礼者,忠信之薄(bó)而乱之首。前识者,道之华而愚之始。是以大
	
	丈夫处其厚,不居其薄(bó);处其实,不居其华。故去彼取此。 〖译文〗
	
	
	
	
	\chapter{得一}
	【第三十九章】昔之得一者,天得一以清,地得一以宁,神得一以灵,谷得一以盈,万物得一以生,侯王
	
	得一以为天下贞。其致之。天无以清将恐裂,地无以宁将恐发(fèi,“发”通“废”),神无以灵将恐歇
	
	,谷无以盈将恐竭,万物无以生将恐灭,侯王无以贵高将恐蹶(jué)。故贵以贱为本,高以下为基。是以
	
	侯王自谓孤寡不穀(谷gǔ)。此非以贱为本邪(yé)?非乎?故致数(shuò)舆(yù)无舆。不欲琭(
	
	lù)琭如玉,珞(luò)珞如石。 〖译文〗
	
	
	

	\chapter{勤用}
		【第四十章】反者,道之动;弱者,道之用。天下万物生于有,有生于无。 〖译文〗
	
	
	
	
	\chapter{闻道}
	【第四十一章】 上士闻道,勤而行之;中士闻道,若存若亡;下士闻道,大笑之,不笑不足以为道。故建
	
	言有之:明道若昧,进道若退,夷道若颣(lèi)。上德若谷,大白若辱,广德若不足,建德若偷,质真若
	
	渝(yú)。大方无隅(yú),大器晚成,大音希声,大象无形。道隐无名,夫唯道善贷且成。 〖译文〗
	
	
	
	
	\chapter{冲和}
	【第四十二章】道生一,一生二,二生三,三生万物。万物负阴而抱阳,冲气以为和。人之所恶(wù),
	
	唯孤寡不穀(谷gǔ),而王公以为称(chēng)。故物,或损之而益,或益之而损。人之所教(jiào),我
	
	亦教之。强梁者不得其死,吾将以为教父。 〖译文〗
	
	
	
	
	\chapter{至柔}
	【第四十三章】天下之至柔,驰骋天下之至坚,无有入无间,吾是以知无为之有益。不言之教,无为之益
	
	,天下希及之。 〖译文〗
	
	
	
	
	\chapter{名身}
	【第四十四章】名与身孰亲?身与货孰多?得与亡孰病? 是故甚爱必大费,多藏必厚亡。知足不辱,知止
	
	不殆,可以长久。 〖译文〗
	
	
	
	
	\chapter{清静}
	【第四十五章】大成若缺,其用不弊。大盈若冲,其用不穷。大直若屈,大巧若拙,大辩若讷。躁胜寒,
	
	静胜热。清静为天下正。 〖译文〗
	
	
	
	
	\chapter{知足}
	【第四十六章】天下有道,却走马以粪;天下无道,戎马生于郊。祸莫大于不知足,咎莫大于欲得,故知
	
	足之足,常足矣。 〖译文〗
	
	
	
	【
	\chapter{户}
	第四十七章】不出户,知天下;不窥牖,见天道。其出弥远,其知弥少。是以圣人不行而知,不见而名
	
	,不为而成。 〖译文〗
	
	

	\chapter{日损}
		
	【第四十八章】为学日益,为道日损。损之又损,以至于无为,无为而无不为。取天下常以无事,及其有
	
	事,不足以取天下。 〖译文〗
	
	
	

	\chapter{浑心}
		【第四十九章】圣人无常心,以百姓心为心。善者,吾善之;不善者,吾亦善之,德善。信者,吾信之;
	
	不信者,吾亦信之,德信。圣人在天下歙歙(xīxī),为天下浑其心。(百姓皆注其耳目),圣人皆孩之。 〖译文〗
	
	
	
	
	\chapter{摄生}
	【第五十章】出生入死。生之徒十有三,死之徒十有三。人之生动之死地,亦十有三。夫何故?以其生生
	
	之厚。盖闻善摄生者,陆行不遇兕(sì)虎,入军不被(pī)甲兵,兕无所投其角,虎无所措其爪(
	
	zhǎo),兵无所容其刃。夫何故?以其无死地。 〖译文〗
	
	
	
	
	\chapter{尊贵}
	【第五十一章】道生之,德畜(xù)之,物形之,势成之。是以万物莫不尊道而贵德。道之尊,德之贵,
	
	夫莫之命而常自然。故道生之,德畜之。长之、育之、亭之、毒之、养之、覆之。生而不有,为而不恃,
	
	长(zhǎng)而不宰,是谓玄德。 〖译文〗
	\chapter{有始}
	
	
	
	
	
	【第五十二章】天下有始,以为天下母。既得其母,以知其子;既知其子,复守其母,没(mò)身不殆。
	
	塞(sè)其兑,闭其门,终身不勤。开其兑,济其事,终身不救。见(jiàn)小曰明,守柔曰强。用其光
	
	,复归其明,无遗身殃,是为习常。 〖译文〗
	
	\chapter{}	
	
	【第五十三章】使我介然有知,行于大道,唯施(迤yí)是畏。大道甚夷,而民好径。朝(cháo)甚除,
	
	田甚芜,仓甚虚。服文彩,带利剑,厌饮食,财货有余,是为盗夸。非道也哉! 〖译文〗
	
	\chapter{}	
	
	【第五十四章】善建者不拔,善抱者不脱,子孙以祭祀不辍。修之于身,其德乃真;修之于家,其德乃余
	
	;修之于乡,其德乃长(zhǎng);修之于国,其德乃丰;修之于天下,其德乃普。故以身观身,以家观家
	
	,以乡观乡,以国观国,以天下观天下。吾何以知天下然哉?以此。 〖译文〗
	
	\chapter{}	
	
	【第五十五章】 含德之厚,比于赤子。蜂虿(chài)虺(huǐ)蛇不螫(shì),猛兽不据,攫(jué)鸟不搏
	
	。骨弱筋柔而握固。未知牝牡之合而全作,精之至也。终日号而不嗄(shà),和之至也。知和曰常,知常
	
	曰明,益生曰祥,心使气曰强。物壮则老,谓之不道,不道早已。 〖译文〗
	
	\chapter{}	
	
	【第五十六章】知(zhì)者不言,言者不知(zhì)。塞(sè)其兑,闭其门,挫其锐;解其纷,和其光
	
	,同其尘,是谓玄同。故不可得而亲,不可得而疏;不可得而利,不可得而害;不可得而贵,不可得而贱
	
	,故为天下贵。 〖译文〗
	
	\chapter{}	
	
	【第五十七章】以正治国,以奇用兵,以无事取天下。吾何以知其然哉?以此。天下多忌讳,而民弥贫;
	
	民多利器,国家滋昏;人多伎(jì)巧,奇物滋起;法令滋彰,盗贼多有。故圣人云:“我无为而民自化
	
	,我好静而民自正,我无事而民自富,我无欲而民自朴。” 〖译文〗
	
	\chapter{}	
	
	【第五十八章】其政闷闷,其民淳淳;其政察察,其民缺缺。祸兮福之所倚,福兮祸之所伏。孰知其极?
	
	其无正。正复为奇,善复为妖,人之迷,其日固久。是以圣人方而不割,廉而不刿(guì),直而不肆,光
	
	而不耀。 〖译文〗
	
	\chapter{}	
	
	【第五十九章】治人事天莫若啬(sè)。夫唯啬,是谓早服。早服谓之重(chóng)积德,重(chóng)积德则
	
	无不克,无不克则莫知其极,莫知其极,可以有国。有国之母,可以长久。是谓深根固柢(dǐ),长生久
	
	视之道。 〖译文〗
	
	\chapter{}	
	
	【第六十章】治大国若烹小鲜。以道莅(lì)天下,其鬼不神。非其鬼不神,其神不伤人;非其神不伤人
	
	,圣人亦不伤人。夫两不相伤,故德交归焉。 〖译文〗
	
	\chapter{}	
	
	【第六十一章】大国者下流。天下之交,天下之牝。牝常以静胜牡,以静为下。故大国以下小国,则取小
	
	国;小国以下大国,则取大国。故或下以取,或下而取。大国不过欲兼畜(xù)人,小国不过欲入事人,
	
	夫两者各得其所欲,大者宜为下。 〖译文〗
	
	\chapter{}	
	
	【第六十二章】道者万物之奥,善人之宝,不善人之所保。美言可以市,尊行可以加人。人之不善,何弃
	
	之有!故立天子,置三公,虽有拱璧以先驷马,不如坐进此道。古之所以贵此道者何?不曰以求得,有罪
	
	以免邪(yé)?故为天下贵。 〖译文〗
	
	\chapter{}	
	
	【第六十三章】为无为,事无事,味无味。大小多少,报怨以德。图难于其易,为大于其细。天下难事必
	
	作于易,天下大事必作于细,是以圣人终不为大,故能成其大。夫轻诺必寡信,多易必多难,是以圣人犹
	
	难之。故终无难矣。 〖译文〗
	
	\chapter{}	
	
	【第六十四章】其安易持,其未兆易谋,其脆易泮(pàn),其微易散。为之于未有,治之于未乱。合抱之
	
	木,生于毫末;九层之台,起于累土;千里之行,始于足下。为者败之,执者失之。是以圣人无为,故无
	
	败;无执,故无失。民之从事,常于几成而败之。慎终如始,则无败事。是以圣人欲不欲,不贵难得之货
	
	。学不学,复众人之所过。以辅万物之自然,而不敢为。 〖译文〗
	
	\chapter{}	
	
	【第六十五章】古之善为道者,非以明民,将以愚之。民之难治,以其智多。故以智治国,国之贼;不以
	
	智治国,国之福。知此两者,亦稽(jī)式。常知稽式,是谓玄德。玄德深矣,远矣,与物反矣,然后乃
	
	至大顺。 〖译文〗
	
	\chapter{}	
	
	【第六十六章】江海所以能为百谷王者,以其善下之,故能为百谷王。是以欲上民,必以言下之;欲先民
	
	,必以身后之。是以圣人处上而民不重,处前而民不害,是以天下乐推而不厌。以其不争,故天下莫能与
	
	之争。 〖译文〗
	
	\chapter{}	
	
	【第六十七章】天下皆谓我道大,似不肖(xiào)。夫唯大,故似不肖。若肖,久矣其细也夫。我有三宝
	
	,持而保之。一曰慈,二曰俭,三曰不敢为天下先。慈,故能勇;俭,故能广;不敢为天下先,故能成器
	
	长(zhǎng)。今舍慈且勇,舍俭且广,舍后且先,死矣!夫慈,以战则胜,以守则固,天将救之,以慈卫
	
	之。 〖译文〗
	
	\chapter{}	
	
	【第六十八章】善为士者不武,善战者不怒,善胜敌者不与,善用人者为之下。是谓不争之德,是谓用人
	
	之力,是谓配天古之极。 〖译文〗
	
	\chapter{}	
	
	【第六十九章】用兵有言,吾不敢为主而为客,不敢进寸而退尺。是谓行(xíng)无行(háng),攘
	
	(rǎng)无臂,扔无敌,执无兵。祸莫大于轻敌,轻敌几丧吾宝。故抗兵相加,哀者胜矣。 〖译文〗
	
	\chapter{}	
	
	【第七十章】吾言甚易知,甚易行,天下莫能知,莫能行。言有宗,事有君。夫唯无知,是以不我知。知
	
	我者希,则我者贵,是以圣人被(pī,“被”同“披”)褐怀玉。 〖译文〗
	
	\chapter{}	
	
	【第七十一章】知不知,上;不知知,病。夫唯病病,是以不病。圣人不病,以其病病,是以不病。 〖译文〗
	
	\chapter{}	
	
	【第七十二章】民不畏威,则大威至。无狎(xiá)其所居,无厌(yà,“厌”同“压”)其所生。夫唯不
	
	厌(yà,“厌”同“压”),是以不厌(yàn)。是以圣人自知,不自见(xiàn);自爱,不自贵。故去彼取
	
	此。 〖译文〗
	
	\chapter{}	
	
	【第七十三章】勇于敢则杀,勇于不敢则活。此两者,或利或害。天之所恶(wù),孰知其故?是以圣人
	
	犹难之。天之道,不争而善胜,不言而善应,不召而自来,繟(chǎn)然而善谋。天网恢恢,疏而不失。 〖译文〗
	
	\chapter{}	
	
	【第七十四章】民不畏死,奈何以死惧之!若使民常畏死,而为奇者,吾得执而杀之,孰敢?常有司杀者
	
	杀,夫代司杀者杀,是谓代大匠斫(zhuó)。夫代大匠斫者,希有不伤其手矣。 〖译文〗
	
	\chapter{}	
	
	【第七十五章】民之饥,以其上食税之多,是以饥。民之难治,以其上之有为,是以难治。民之轻死,以
	
	其求生之厚,是以轻死。夫唯无以生为者,是贤于贵生。 〖译文〗
	
	\chapter{}	
	\begin{yuanwen}	
	【第七十六章】人之生也柔弱,其死也坚强。万物草木之生也柔脆,其死也枯槁。故坚强者死之徒,柔弱
	
	者生之徒。是以兵强则不胜,木强则兵。强大处下,柔弱处上。 〖译文〗
	\end{yuanwen}
		
	\chapter{}	
	\begin{yuanwen}	
	【第七十七章】天之道,其犹张弓与!高者抑之,下者举之;有余者损之,不足者补之。天之道,损有余
	
	而补不足。人之道则不然,损不足以奉有余。孰能有余以奉天下?唯有道者。是以圣人为而不恃,功成而
	
	不处,其不欲见(xiàn)贤。 〖译文〗
	\end{yuanwen}
		
	\chapter{}	
	\begin{yuanwen}	
	【第七十八章】天下莫柔弱于水,而攻坚强者莫之能胜,其无以易之。弱之胜强,柔之胜刚,天下莫不知
	
	,莫能行。是以圣人云,受国之垢,是谓社稷主;受国不祥,是为天下王。正言若反。 〖译文〗
	\end{yuanwen}
		
	\chapter{}	
	\begin{yuanwen}	
	【第七十九章】和大怨,必有余怨,安可以为善?是以圣人执左契,而不责于人。有德司契,无德司彻。
	
	天道无亲,常与善人。 〖译文〗
	\end{yuanwen}
		
	\chapter{}	
	\begin{yuanwen}	
	小国寡民,使有什伯(bǎi)之器而不用,使民重(zhòng)死而不远徙(xí)。虽有舟舆,无
	
	所乘之;虽有甲兵,无所陈之;使人复结绳而用之。甘其食,美其服,安其居,乐其俗。邻国相望,鸡犬
	
	之声相闻,民至老死不相往来。 〖译文〗
	\end{yuanwen}
		
	\chapter{}
	\begin{yuanwen}
	
	信言不美,美言不信;善者不辩,辩者不善;知(zhì)者不博,博者不知(zhì)。圣人
	
	不积,既以为人,己愈有;既以与人,己愈多。天之道,利而不害。圣人之道,为而不争。〖译文〗
	\end{yuanwen}
	
\end{document}