% 笠翁对韵
% 笠翁对韵.tex

\documentclass[12pt,UTF8]{ctexbook}

% 设置纸张信息。
\usepackage[a4paper,twoside]{geometry}
\geometry{
	left=25mm,
	right=25mm,
	bottom=25.4mm,
	bindingoffset=10mm
}

% 设置字体,并解决显示难检字问题。
\xeCJKsetup{AutoFallBack=true}
\setCJKmainfont{SimSun}[BoldFont=SimHei, ItalicFont=KaiTi, FallBack=SimSun-ExtB]

% 目录 chapter 级别加点(.)。
\usepackage{titletoc}
\titlecontents{chapter}[0pt]{\vspace{3mm}\bf\addvspace{2pt}\filright}{\contentspush{\thecontentslabel\hspace{0.8em}}}{}{\titlerule*[8pt]{.}\contentspage}

% 设置 part 和 chapter 标题格式。
\ctexset{
	part/name= {},
	part/number={},
	chapter/name={,},
	chapter/number={\chinese{chapter}}
}

% 设置古文原文格式。
\newenvironment{yuanwen}{\bfseries\zihao{4}}

% 设置署名格式。
\newenvironment{shuming}{\hfill\bfseries\zihao{4}}

% 注脚每页重新编号,避免编号过大。
\usepackage[perpage]{footmisc}

\title{\heiti\zihao{0} 笠翁对韵}
\author{李渔}
\date{}

\begin{document}

\maketitle
\tableofcontents

\frontmatter
\chapter{前言、序言}

\mainmatter

% 增加空行
~\\

% 增加字间间隔,适用于三字经、诗文等。
 \qquad  

\part{卷上}

\chapter{东}

\begin{yuanwen}
天对地,雨对风。大陆对长空\footnote{宽广高远的天空。}。山花对海树\footnote{text},赤日对苍穹\footnote{text}。

雷隐隐\footnote{text},雾蒙蒙\footnote{text}。日下对天中。风高\footnote{text}秋月白,雨霁\footnote{text}jì晚霞红。

牛女二星河左右[8] ,参商两曜yào斗西东[9] 。十月塞边,飒sà 飒寒霜惊戍shù 旅[10] ;三冬江上,漫漫朔雪[11] 冷渔翁。
\end{yuanwen}

河hé 对duì 汉hàn [12] ,绿lǜ 对duì 红hóng 。雨yǔ 伯bó 对duì 雷léi 公gōng [13] 。烟yān 楼lóu 对duì 雪xuě 洞dòng [14] ,月yuè 殿diàn 对duì 天tiān 宫gōng [15] 。云yún 叆ài 叇dài [16] ,日rì 曈tóng 曚méng [17] 。蜡là 屐jī [18] 对duì 渔yú 篷péng 。过guò 天tiān 星xīng [19] 似sì 箭jiàn ,吐tǔ 魄bà 月yuè 如rú 弓gōng [20] 。驿yì 旅lǚ 客kè 逢féng 梅méi 子zǐ 雨yǔ [21] ,池chí 亭tíng 人rén 挹yì 藕ǒu 花huā 风fēng [22] 。茅máo 店diàn 村cūn 前qián ,皓hào 月yuè [23] 坠zhuì 林lín 鸡jī 唱chàng 韵yùn ;板bǎn 桥qiáo 路lù 上shàng ,青qīng 霜shuāng 锁suǒ 道dào 马mǎ 行xíng 踪zōng [24] 。

山shān 对duì 海hǎi ,华huà 对duì 嵩sōng [25] 。四sì 岳yuè 对duì 三sān 公gōng [26] 。宫gōng 花huā 对duì 禁jìn 柳liǔ [27] ,塞sài 雁yàn 对duì 江jiāng 龙lóng 。清qīng 暑shǔ 殿diàn [28] ,广guǎng 寒hán 宫gōng [29] 。拾shí 翠cuì 对duì 题tí 红hóng [30] 。庄zhuāng 周zhōu 梦mèng 化huà 蝶dié ,吕lǚ 望wàng 兆zhào 飞fēi 熊xióng [31] 。北běi 牖yǒu [32] 当dāng 风fēng 停tíng 夏xià 扇shàn ,南nán 帘lián 曝pù 日rì 省shěng 冬dōng 烘hōng [33] 。鹤hè 舞wǔ 楼lóu 头tóu ,玉yù 笛dí 弄nòng 残cán 仙xiān 子zǐ 月yuè [34] ;凤fèng 翔xiáng 台tái 上shàng ,紫zǐ 箫xiāo 吹chuī 断duàn 美měi 人rén 风fēng [35] 。


[2] 山花:山间野花。海树:岸边的树。

[3] 赤日:红日,烈日。苍穹:青天。

[4] 雷隐隐:雷声不分明的样子。隐隐,雷声或车声。

[5] 雾蒙蒙:雾迷茫的样子。

[6] 风高:风大。

[7] 霁:雨雪停止,天放晴。

[8] 牛:牵牛星。女:织女星。河:银河。

[9] 参商两曜斗西东:参和商是二十八宿中的两宿。商即辰,也即是心宿。参宿在西方,心宿居东方,古人往往把亲友久别难逢比为参商。斗,指二十八宿之一的斗宿,不是北斗。两曜,古人把日、月、五星称七曜,曜就是星。

[10] 飒飒:形容风吹动树木枝叶等的声音。戍旅:以兵卒防守边疆。

[11] 漫漫:广远无际。朔雪:北方的雪。

[12] 河:黄河。汉:汉水。由于河可以借指银河,汉也可借指银河。

[13] 雨伯、雷公:雨伯、雷公是古代神话中的雨神和雷神。雨伯原称雨师,为了属对工整,这里把师改作伯。

[14] 烟楼:耸立于烟云中之高楼。雪洞:被雪封住的山洞。

[15] 月殿:月宫。天宫:上帝或诸神在天上的住所。

[16] 叆叇:浓云蔽日之状。

[17] 曈曚:太阳将出天色微明的样子。

[18] 蜡屐:古人穿的一种底下有齿的木鞋,以蜡涂抹其上,叫蜡屐。

[19] 过天星:这里指流星。

[20] 吐魄月:魄,又作霸,月球被自身遮掩的阴影部分。古人对月的圆缺道理不理解,以为月里有只蟾蜍,是由它反复吞吐造成的。吐魄月就是刚被吐出的月,指新月,所以说它如弓。

[21] 驿旅客逢梅子雨:古代官府设立的招待往来官员的旅舍。驿旅,住在驿舍的旅客。梅子雨:即梅雨、黄梅雨。中国南部五月至七月所下的雨,因正值梅子成熟的时节,故称为梅雨。

[22] 池亭人挹藕花风:荷花香气阵阵吹来,人们在亭台上饮酒。挹,酌酒。

[23] 皓月:月光茫茫的样子。

[24] 茅店村前,皓月坠林鸡唱韵;板桥路上,青霜锁道马行踪:这一联是从晚唐温庭筠《商山早行》中“鸡声茅店月,人迹板桥霜”两句诗隐括出来的。

[25] 华对嵩:西岳华山和中岳嵩山。

[26] 四岳:传说尧时分掌四时、方岳的官。三公:古代天子以下最大的三个官员,各代的职称并不一致。四岳又释指东岳泰山、西岳华山、南岳衡山、北岳恒山。三公又释为星名。

[27] 禁柳:古代皇帝居住的城苑禁止百姓出入,所以称禁宫。禁柳即宫廷中的柳树。

[28] 清暑殿:相传三国时吴有避暑宫,夏日清凉不热。

[29] 广寒宫:神话里称月亮中的宫殿为广寒宫。

[30] 拾翠:原指拾找像翡翠一样的羽毛,后来把青年妇女采集鲜花野草也称作拾翠。题红:刘斧《青琐高议》载:唐僖宗时士人于祐,偶然中从御沟流水上拾到一片红叶,上面题有两句诗:“流水何太急,深宫尽日闲。殷勤谢红叶,好去到人间。”于祐和了两句:“曾闻叶上题红怨,叶上题诗寄阿谁?”放在上游,红叶随水又流入宫中。后于祐娶得宫中韩夫人为妻,谈及此事,其妻倍感惊异,原来当年题诗红叶的就是她。于是她又题了一首诗:“一联佳句随流水,十载幽思满素怀。今日却成鸾凤友,方知红叶是良媒。”

[31] 吕望兆飞熊:吕望,即太公望,又称姜太公。传说周文王一夜梦见飞熊进帐,经人占卜,说是将得到贤人的吉兆。第二天出猎,果然遇到姜太公。

[32] 北牖:北窗。牖,窗户。

[33] 南帘曝日省冬烘:曝,晒。曝日即晒太阳。冬烘,原意是指人头脑不清,这里借来同上句的“夏扇”对仗,就是冬天的火炉的意思。

[34] 鹤舞楼头,玉笛弄残仙子月:唐李白诗:“黄鹤楼头吹玉笛,江城五月落梅花。”《齐谐记》:“仙人子安曾驾鹤经过黄鹤楼。”楼旧址在武昌黄鹤矶上,为古时游览胜地。

[35] 凤翔台上,紫箫吹断美人风:《列仙传》载:秦穆公有女名弄玉,好道。时有人名萧史,善吹箫作鸾凤鸣。穆公把女嫁给萧史,并为他们筑了一所凤凰台。萧史教弄玉以箫吹凤鸣声,凤凰聚止其屋。一日,萧史乘龙,弄玉跨凤,双双升仙而去。




\chapter{东}
二èr 冬dōng


晨chén 对duì 午wǔ ,夏xià 对duì 冬dōng 。下xià 饷xiǎng 对duì 高gāo 舂chōng [1] 。青qīng 春chūn 对duì 白bái 昼zhòu ,古gǔ 柏bǎi 对duì 苍cāng 松sōng 。垂chuí 钓diào 客kè [2] ,荷hè 锄chú 翁wēng [3] 。仙xiān 鹤hè 对duì 神shén 龙lóng 。凤fèng 冠guān 珠zhū 闪shǎn 烁shuò ,螭chī 带dài [4] 玉yù 玲líng 珑lóng 。三sān 元yuán 及jí 第dì 才cái 千qiān 顷qǐng [5] ,一yī 品pǐn 当dāng 朝cháo 禄lù 万wàn 钟zhōng [6] 。花huā 萼è 楼lóu 间jiān ,仙xiān 李lǐ 盘pán 根gēn 调tiáo 国guó 脉mài [7] ,沉chén 香xiāng 亭tíng 畔pàn ,娇jiāo 杨yáng 擅shàn 宠chǒng 起qǐ 边biān 风fēng [8] 。

清qīng 对duì 淡dàn ,薄bó [9] 对duì 浓nóng 。暮mù 鼓gǔ 对duì 晨chén 钟zhōng [10] 。山shān 茶chá 对duì 石shí 菊jú ,烟yān 锁suǒ 对duì 云yún 封fēng 。金jīn 菡hàn 萏dàn ,玉yù 芙fú 蓉róng [11] 。绿lǜ 绮qǐ 对duì 青qīng 锋fēng [12] 。早zǎo 汤tāng 先xiān 宿sù 酒jiǔ [13] ,晚wǎn 食shí 继jì 朝zhāo 饔yōng [14] 。唐táng 库kù 金jīn 钱qián 能néng 化huà 蝶dié [15] ,延yán 津jīn 宝bǎo 剑jiàn 会huì 成chéng 龙lóng [16] 。巫wū 峡xiá 浪làng 传chuán ,云yún 雨yǔ 荒huāng 唐táng 神shén 女nǚ 庙miào [17] ;岱dài 宗zōng 遥yáo 望wàng ,儿ér 孙sūn 罗luó 列liè 丈zhàng 人rén 峰fēng [18] 。

繁fán 对duì 简jiǎn ,叠dié 对duì 重chóng 。意yì 懒lǎn 对duì 心xīn 慵yōng [19] 。仙xiān 翁wēng 对duì 释shì 伴bàn [20] ,道dào 范fàn 对duì 儒rú 宗zōng [21] 。花huā 灼zhuó 灼zhuó [22] ,草cǎo 茸róng 茸róng [23] 。浪làng 蝶dié 对duì 狂kuáng 蜂fēng 。数shù 竿gān 君jūn 子zǐ 竹zhú [24] ,五wǔ 树shù 大dà 夫fū 松sōng [25] 。高gāo 皇huáng 灭miè 项xiàng 凭píng 三sān 杰jié [26] ,虞yú 帝dì 承chéng 尧yáo 殛jí 四sì 凶xiōng [27] 。内nèi 苑yuàn 佳jiā 人rén ,满mǎn 地dì 风fēng 光guāng 愁chóu 不bù 尽jìn ;边biān 关guān 过guò 客kè ,连lián 天tiān 烟yān 草cǎo 憾hàn 无wú 穷qióng 。



* * *



[1] 下饷:下午饭。这里指下午。高舂:薄暮。

[2] 垂钓客:垂竿钓鱼的人。

[3] 荷:担着,扛着。

[4] 螭带:螭,古代传说中一种无角的龙。螭带就是带钩上雕有螭纹的玉带。

[5] 三元及第才千顷:封建科举考试,乡试第一称解元,会试第一称会元,殿试第一称状元,连续考得三个第一,就是所谓连中三元,三元及第。才千顷,形容人才学之广。

[6] 一品当朝禄万钟:古代宰相为一品官爵。禄,古代官吏的薪俸。钟,古代称粮的容积单位,每钟盛六斛四斗,万钟极言其多。

[7] 花萼楼间,仙李盘根调国脉:语出杜甫诗。唐朝皇族姓李,杜甫用这句诗比喻皇族子孙繁衍,江山永固。调脉,本指中医诊脉治病。调国脉,是说治理国家,左右国家的命运。

[8] 沉香亭畔,娇杨擅宠起边风:沉香亭,唐禁苑中的一座亭台。娇杨:指杨贵妃。擅宠,即专宠,排挤掉别人,使皇帝只对她一个人欢心。这两句讲的是:唐明皇早年宠爱杨贵妃,日夜同她饮酒作乐,不理朝政。他曾命人在沉香亭旁遍植牡丹,花开时同杨妃到亭上饮酒赏花。后来,安禄山从渔阳起兵叛乱,唐王朝自此走上了下坡路。“起边风”即指安禄山的叛乱。

[9] 薄:淡。

[10] 暮鼓、晨钟:本指寺院僧众撞钟击鼓,此指言论警策,发人深省。

[11] 菡萏、芙蓉:荷花的别称。

[12] 绿绮:相传是汉末蔡邕的琴名。青锋:剑名。

[13] 宿酒:隔夜仍使人醉而不醒的酒力。

[14] 朝饔:早饭。

[15] 唐库金钱能化蝶:《杜阳杂编》里说:唐穆宗时,殿前种千叶牡丹,开放时香气袭人,穆宗夜宴,有无数黄白蝴蝶飞集花间,天明即飞去。人们张网捕捉数百,天明都变成了金玉,后来打开宝橱,发现皆库中金银所化。

[16] 延津宝剑会成龙:传说晋代张华和雷焕在丰城地下挖出一对极为珍贵的宝剑,每人拿了一把。后来雷焕的儿子佩着剑路过延平津的时候,宝剑忽然跃入水中,变成了一条龙潜水而去。

[17] 巫峡浪传,云雨荒唐神女庙:宋玉《高唐赋》,说楚国先王曾游高唐之观,梦中见一神女,神女临行时说她是巫山之女,“旦为朝云,暮为行雨,朝朝暮暮,阳台之下”。王为立庙,号朝云庙。后人多以巫山神女故事歌咏爱情。浪传:犹如空传,意思是宋玉讲的神女不过是个寓言而已,并无其事。

[18] 岱宗遥望,儿孙罗列丈人峰:岱宗,即泰山,古人以它为群山之首,所以称它为宗。杜甫《望岳》诗:“岱宗夫如何?齐鲁青未了。”后半句也是从杜诗变化出来的。杜甫七律《望岳》的原句是:“西岳危棱竦处尊,诸峰罗立如儿孙。”不过这里描写的是西岳华山,而不是东岳泰山。丈人峰:山峰名。在泰山上,因形状像老人,所以称为丈人峰。

[19] 心慵:懒。

[20] 仙翁:称男性神仙,仙人。释伴:犹如说道侣,同修一道的伙伴。

[21] 道范:敬称他人的容颜,风范。儒宗:儒者的宗师。汉以后亦泛指为读书人所宗仰的学者。

[22] 灼灼:耀眼,光明。

[23] 茸茸:草初生的样子。

[24] 数竿君子竹:古人认为,竹劲节虚心,有君子之德。

[25] 五树大夫松:《史记》记载,秦始皇登泰山,遇到暴风雨,躲在一棵松树下避雨,于是封为“五大夫”松。

[26] 高皇:汉高祖刘邦。项:项羽。三杰:指西汉初期的张良、萧何、韩信。

[27] 虞帝承尧殛四凶:古史传说,唐尧年老时把帝位让给虞舜,舜即位后,流放了四个尧舜时代恶名昭彰的部族首领。





\chapter{三sān 江jiāng}


奇jī 对duì 偶ǒu ,只zhī 对duì 双shuāng 。大dà 海hǎi 对duì 长cháng 江jiāng 。金jīn 盘pán 对duì 玉yù 盏zhǎn ,宝bǎo 烛zhú 对duì 银yín gāng [1] 。朱zhū 漆qī 槛jiàn ,碧bì 纱shā 窗chuāng 。舞wǔ 调diào 对duì 歌gē 腔qiāng 。兴xīng 汉hàn 推tuī 马mǎ 武wǔ [2] ,谏jiàn 夏xià 著zhù 龙lóng 逄páng [3] 。四sì 收shōu 列liè 国guó 群qún 王wáng 伏fú [4] ,三sān 筑zhù 高gāo 城chéng 众zhòng 敌dí 降xiáng [5] 。跨kuà 凤fèng 登dēng 台tái ,潇xiāo 洒sǎ 仙xiān 姬jī 秦qín 弄nòng 玉yù [6] ;斩zhǎn 蛇shé 当dāng 道dào ,英yīng 雄xióng 天tiān 子zǐ 汉hàn 刘liú 邦bāng [7] 。

颜yán 对duì 貌mào ,像xiàng 对duì 庞páng [8] 。步bù 辇niǎn 对duì 徒tú 杠gàng [9] 。停tíng 针zhēn 对duì 搁gē 杼zhù [10] ,意yì 懒lǎn 对duì 心xīn 降xiáng [11] 。灯dēng 闪shǎn 闪shǎn ,月yuè 幢chuáng 幢chuáng [12] 。揽lǎn 辔pèi 对duì 飞fēi 舡xiāng [13] 。柳liǔ 堤dī 驰chí 骏jùn 马mǎ ,花huā 院yuàn 吠fèi 村cūn 尨máng [14] 。酒jiǔ 量liàng 微wēi 酡tuó 琼qióng 杏xìng 颊jiá [15] ,香xiāng 尘chén 没mò 印yìn 玉yù 莲lián 双shuāng [16] 。诗shī 写xiě 丹dān 枫fēng ,韩hán 女nǚ 幽yōu 怀huái 流liú 御yù 水shuǐ [17] ;泪lèi 弹tán 斑bān 竹zhú ,舜shùn 妃fēi 遗yí 憾hàn 积jī 湡yú 江jiāng [18] 。



* * *



[1] 银 :银白色的灯盏、烛台。

[2] 兴汉推马武:马武是汉光武帝的将军,在建立东汉王朝的斗争中起过一定的作用。

[3] 谏夏著龙逄:龙逄即关龙逄,传说是夏桀王的大臣。他见夏桀无道,淫侈暴虐,曾强力谏争,结果被夏桀处死。

[4] 四收列国群王伏:北宋初大将曹彬,他曾同潘美等将帅一道,伐灭了后蜀、南汉、南唐及北汉等五代时的地方割据政权,帮助宋太祖统一了天下。

[5] 三筑高城众敌降:初唐张仁愿,中宗朝人,曾统领朔方军与突厥族的侵扰进行斗争,使突厥不敢过山牧马。建了三座受降城以威镇北敌,从此边境安宁。

[6] 跨凤登台,潇洒仙姬秦弄玉:弄玉故事,详见一东“凤翔”二句注。

[7] 斩蛇当道,英雄天子汉刘邦:《史记·高祖本纪》记载,刘邦初起,酒醉夜行,先行者报告说有长蛇拦路,刘邦上前杀死长蛇,路遂通。后有一老太婆在斩蛇处夜哭,人们询问,她说是自己的儿子是白帝子变化为蛇,被赤帝子杀害了。

[8] 庞:面庞。

[9] 步辇对徒杠:步辇,古代皇帝乘坐的人力拉的车。徒杠,即可供徒步行走的轿子。

[10] 搁杼:放下梭子,与停针可以成对。

[11] 降:安稳、平和。心降就是心里安稳、平和。

[12] 幢幢:朦胧的样子。

[13] 揽辔:控制马匹缰绳。舡:船只。

[14] 村尨:即村狗。

[15] 酡:饮酒后脸红的样子。琼:美玉。

[16] 香尘没印玉莲双:晋石崇豪富骄奢,多蓄婢妾,布香尘于地,令诸姬行其上,以试鞋底之大小。玉莲,比喻女人的脚。

[17] 诗写丹枫,韩女幽怀流御水:见一东韵“题红”注。

[18] 泪弹斑竹,舜妃遗憾积湡江:古代神话传说,帝舜的两个妃子娥皇和女英,居住在洞庭之山,舜南巡死于苍梧之野,二妃尽日啼哭,泪洒竹上,竹尽斑,这就是今天的湘妃竹。湡江,水名。





四sì 支zhī


泉quán 对duì 石shí ,干gàn 对duì 枝zhī 。吹chuī 竹zhú 对duì 弹tán 丝sī [1] 。山shān 亭tíng 对duì 水shuǐ 榭xiè [2] ,鹦yīng 鹉wǔ 对duì 鸬lú 鹚cí [3] 。五wǔ 色sè 笔bǐ [4] ,十shí 香xiāng 词cí [5] 。泼pō 墨mò 对duì 传chuán 卮zhī [6] 。神shén 奇qí 韩hán 幹gàn 画huà [7] ,雄xióng 浑hún 李lǐ 陵líng 诗shī [8] 。几jǐ 处chù 花huā 街jiē 新xīn 夺duó 锦jǐn [9] ,有yǒu 人rén 香xiāng 径jìng 淡dàn 凝níng 脂zhī 。万wàn 里lǐ 烽fēng 烟yān ,战zhàn 士shì 边biān 关guān 争zhēng 保bǎo 塞sài ;一yī 犁lí 膏gāo 雨yǔ [10] ,农nóng 夫fū 村cūn 外wài 尽jìn 乘chéng 时shí [11] 。

葅zū 对duì 醢hǎi [12] ,赋fù 对duì 诗shī 。点diǎn 漆qī 对duì 描miáo 脂zhī 。璠fán 簪zān 对duì 珠zhū 履lǚ [13] ,剑jiàn 客kè 对duì 琴qín 师shī 。沽gū 酒jiǔ 价jià [14] ,买mǎi 山shān 资zī [15] 。国guó 色sè 对duì 仙xiān 姿zī 。晚wǎn 霞xiá 明míng 似sì 锦jǐn ,春chūn 雨yǔ 细xì 如rú 丝sī 。柳liǔ 绊bàn 长cháng 堤dī 千qiān 万wàn 树shù ,花huā 横héng 野yě 寺sì 两liǎng 三sān 枝zhī 。紫zǐ 盖gài 黄huáng 旗qí ,天tiān 象xiàng 预yù 占zhān 江jiāng 左zuǒ 地dì [16] ;青qīng 袍páo 白bái 马mǎ ,童tóng 谣yáo 终zhōng 应yìng 寿shòu 阳yáng 儿ér [17] 。

箴zhēn 对duì zàn [18] ,缶fǒu 对duì 卮zhī 。萤yíng 照zhào 对duì 蚕cán 丝sī 。轻qīng 裾jū 对duì 长cháng 袖xiù ,瑞ruì 草cǎo [19] 对duì 灵líng 芝zhī 。流liú 涕tì 策cè [20] ,断duàn 肠cháng 诗shī [21] 。喉hóu 舌shé 对duì 腰yāo 肢zhī [22] 。云yún 中zhōng 熊xióng 虎hǔ 将jiàng [23] ,天tiān 上shàng 凤fèng 凰huáng 儿ér [24] 。禹yǔ 庙miào 千qiān 年nián 垂chuí 橘jú 柚yòu [25] ,尧yáo 阶jiē 三sān 尺chǐ 覆fù 茅máo 茨cí [26] 。湘xiāng 竹zhú 含hán 烟yān ,腰yāo 下xià 轻qīng 纱shā 笼lǒng 玳dài 瑁mào ;海hǎi 棠táng 经jīng 雨yǔ ,脸liǎn 边biān 青qīng 泪lèi 湿shī 胭yān 脂zhī 。 [27]

争zhēng 对duì 让ràng ,望wàng 对duì 思sī [28] 。野yě 葛gé 对duì 山shān 栀zhī [29] 。仙xiān 风fēng 对duì 道dào 骨gǔ [30] ,天tiān 造zào [31] 对duì 人rén 为wéi 。专zhuān 诸zhū 剑jiàn [32] ,博bó 浪làng 椎zhuī [33] 。经jīng 纬wěi 对duì 干gān 支zhī 。位wèi 尊zūn 民mín 物wù 主zhǔ ,德dé 重zhòng 帝dì 王wáng 师shī 。望wàng 切qiè 不bù 妨fáng 人rén 去qù 远yuǎn ,心xīn 忙máng 无wú 奈nài 马mǎ 行xíng 迟chí 。金jīn 屋wū 闭bì 来lái ,赋fù 乞qǐ 茂mào 林lín 题tí 柱zhù 笔bǐ [34] ;玉yù 楼lóu 成chéng 后hòu ,记jì 须xū 昌chāng 谷gǔ 负fù 囊náng 词cí [35] 。



* * *



[1] 弹丝:弹奏琴瑟一类的乐器。

[2] 水榭:水上架台,台上建屋,可供人游憩。

[3] 鸬鹚:一种善于捕鱼的水鸟。

[4] 五色笔:相传南朝梁江淹,年轻时梦见晋代学者和诗人郭璞赠给他五色笔,于是才思大进,写了许多优秀诗文。晚年,又梦见郭璞讨回了五色笔,从此才情顿减,人称“江郎才尽”。后以五色笔比喻文才。

[5] 十香词:辽道宗后萧氏,小字观音,才貌双绝,后以谏猎见疏,作《同心词》自明。耶律乙辛诬后与伶人私通,假造《十香词》为证,帝竟赐后自尽。

[6] 泼墨:泼墨是绘画术语,意思是大量用墨渲染。卮:古代盛酒的器具。

[7] 神奇韩幹画:韩幹是唐代著名画家,善写人物,尤工于鞍马。传说建中初年,有人牵患有足疾的马就诊。其马毛色骨相似韩幹所画的马,为真马所无。遂牵此马绕市,巧遇韩幹,幹亦惊疑。返家后,视其所画马本,脚有一点黑缺,方知是马画通灵。

[8] 雄浑李陵诗:李陵,西汉名将李广之孙,武帝天汉二年,率步卒五千与匈奴十万骑决战,终因缺少援军,战败投降。李陵在匈奴遇到出使被扣留的苏武,后苏武南还,李陵设酒送别。其赠别苏武之诗雄浑豪爽,十分感人。

[9] 几处花街新夺锦:唐武则天驾临龙门,诏令群臣赋“明堂火珠”诗,诗先成者赐锦袍。东方虬诗先成,拜锦未坐,宋之问亦成,但写得比东方虬好。武后令夺东方虬锦袍赏给宋之问,此即所谓夺锦。

[10] 膏雨:甘霖。

[11] 乘时:利用有利时机。

[12] 葅:古代酷刑,将人剁成肉酱。醢:肉酱。

[13] 璠簪:美玉制成的簪。珠履:用珠装饰的鞋。相传战国时楚公子春申君,为了向人夸富,让他和门客都穿珠履。

[14] 沽酒价:阮咸每以百钱挂杖头,至酒市沽酒。

[15] 买山资:晋僧人支道林到深公那里去买邱山,深公曰:“未闻巢(父)、(许)由买山而隐(巢父、许由,尧时隐士)。”

[16] 紫盖黄旗,天象预占江左地:三国末年吴主孙皓时,有术士说“庚子之年,紫盖黄旗,当入于洛”,孙皓以为平晋。不料相反,庚子之年恰恰是他被俘入洛阳的一年。

[17] 青袍白马,童谣终应寿阳儿:相传南朝梁武帝时,先是大同中有童谣曰“青袍白马寿阳儿”。不久,寿阳的侯景发动叛乱,叛军中尽青袍白马,终于亡梁。

[18] 箴:古代一种以规劝、告诫为内容的文体。 :通赞,颂扬﹑称美。

[19] 轻裾:形容人在走动或舞蹈时衣襟飘扬的样子。瑞草:相传不常见的草,见则为祥兆,故称为瑞草。如蓂荚、灵芝之类。

[20] 流涕策:古代大臣献给皇帝的意见书叫策。西汉贾谊在写给汉文帝的《治安策》中有“可为痛哭,可为流涕,可为长太息”之句,因称流涕策。

[21] 断肠诗:宋代女诗人朱淑贞,相传其对婚姻不满,故诗词多幽愤哀伤情调,后人辑有《断肠诗集》《断肠词集》传世。

[22] 喉舌:泛指说话的器官,比喻要害之地。腰肢:腰身,身段。

[23] 云中:汉代北方有云中郡,在今山西北部及内蒙古一部分。熊虎将:指西汉名将魏尚,相传他做云中守时,匈奴远避,不敢近边。

[24] 天上凤凰儿:汉民歌《陇西行》有“天上何所有?历历种白榆……凤鸣何啾啾,一母将九雏”的诗句。后来多用为赞美别人儿子的话。

[25] 禹庙千年垂橘柚:语出杜甫诗《禹庙》:“禹庙空寺里,秋风落日斜。荒庭垂橘柚,古屋画龙蛇。”

[26] 尧阶三尺覆茅茨:古书记载,帝尧生活简朴,他的居室土阶三尺,茅茨不剪,采椽不斫。茨,苫房。茅茨,用茅草苫房。

[27] 湘竹含烟,腰下轻纱笼玳瑁;海棠经雨,脸边青泪湿胭脂:轻纱笼罩着腰身,好象烟雾环绕着的竹枝;脸边流下泪水,犹如雨点滴在海棠花上。

[28] 望对思:望可解作盼望,思解作思念,成对;望又可解作怨恨,思也可解作怨恨,也成对。

[29] 栀:植物名。夏开白花,有香气。果实椭圆,色黄,可入药,亦可做染料。或称为栀子。

[30] 仙风:神仙的风致。形容人的潇洒。道骨:修道者的气质。

[31] 天造:自然生成,对人为而言。

[32] 专诸剑:专诸,古代勇士名。《左传》载,春秋时,吴公子光为夺取王位,收买专诸为刺客,把匕首藏在鱼腹中,借进献食品的机会刺死了吴王僚。

[33] 博浪椎:汉代的张良,为了给被灭掉的韩国报仇,从仓海君那里请到一位大力士,携带六十公斤的大铁椎,在博浪沙地方狙击秦始皇,误中副车,未果。

[34] 金屋闭来,赋乞茂林题柱笔:汉武帝幼时,他的姑母馆陶长公主打算把自己的女儿阿娇许给他,就问:“儿欲得妇,阿娇好否?”帝曰:“若得阿娇,当以金屋贮之。”陈阿娇与汉武帝结婚后,颇得宠爱。但陈皇后嫉妒心很强,因自己未育而嫉妒卫夫人,后遭贬独居长门宫,心情悲愤。她听说司马相如很会写文章,就奉黄金百两让相如为她写一篇《长门赋》,抒写她的孤独寂寞之感和对武帝的思念。司马相如曾居住在茂陵,故称他的才思为茂陵题柱笔。题柱,司马相如初西去长安,过成都升仙桥,题柱曰:“不乘高车驷马,不过此桥。”

[35] 玉楼成后,记须昌谷负囊词:唐诗人李贺家乡濒临昌谷川,因之他的诗集称《昌谷集》,后人也称他李昌谷。相传李贺出行,常让小童背一锦囊,每得佳句,就记下投入囊中。后梦神人曰:“上帝白玉楼成,命君作记。”不久诗人就死了。





五wǔ 微wēi


贤xián 对duì 圣shèng ,是shì 对duì 非fēi 。觉jué 奥ào 对duì 参cān 微wēi [1] 。鱼yú 书shū 对duì 雁yàn 字zì [2] ,草cǎo 舍shè 对duì 柴chái 扉fēi 。鸡jī 晓xiǎo 唱chàng ,雉zhì 朝zhāo 飞fēi [3] 。红hóng 瘦shòu 对duì 绿lǜ 肥féi [4] 。举jǔ 杯bēi 邀yāo 月yuè 饮yǐn [5] ,骑qí 马mǎ 踏tà 花huā 归guī 。黄huáng 盖gài 能néng 成chéng 赤chì 壁bì 捷jié [6] ,陈chén 平píng 善shàn 解jiě 白bái 登dēng 危wēi [7] 。太tài 白bái 书shū 堂táng ,瀑pù 泉quán 垂chuí 地dì 三sān 千qiān 丈zhàng [8] ;孔kǒng 明míng 祀sì 庙miào ,老lǎo 柏bǎi 参cān 天tiān 四sì 十shí 围wéi [9] 。

戈gē 对duì 甲jiǎ ,幄wò 对duì 帏wéi 。荡dàng 荡dàng 对duì 巍wēi 巍wēi 。严yán 滩tān 对duì 邵shào 圃pǔ [10] ,靖jìng 菊jú 对duì 夷yí 薇wēi [11] 。占zhān 鸿hóng 渐jiàn [12] ,采cǎi 凤fèng 飞fēi [13] 。虎hǔ 榜bǎng 对duì 龙lóng 旗qí 。心xīn 中zhōng 罗luó 锦jǐn 绣xiù ,口kǒu 内nèi 吐tǔ 珠zhū 玑jī 。宽kuān 宏hóng 豁huò 达dá 高gāo 皇huáng 量liàng [14] ,叱chì 咤zhà 喑yìn 哑yǎ 霸bà 王wáng 威wēi [15] 。灭miè 项xiàng 兴xīng 刘liú ,狡jiǎo 兔tù 尽jìn 时shí 走zǒu 狗gǒu 死sǐ [16] ;连lián 吴wú 拒jù 魏wèi ,貔pí 貅xiū 屯tún 处chù 卧wò 龙lóng 归guī [17] 。

衰shuāi 对duì 盛shèng ,密mì 对duì 稀xī 。祭jì 服fú 对duì 朝cháo 衣yī 。鸡jī 窗chuāng 对duì 雁yàn 塔tǎ [18] ,秋qiū 榜bǎng 对duì 春chūn 闱wéi [19] 。乌wū 衣yī 巷xiàng [20] ,燕yàn 子zǐ 矶jī [21] 。久jiǔ 别bié 对duì 初chū 归guī 。天tiān 姿zī 真zhēn 窈yǎo 窕tiǎo [22] ,圣shèng 德dé 实shí 光guāng 辉huī 。蟠pán 桃táo 紫zǐ 阙quē 来lái 金jīn 母mǔ [23] ,岭lǐng 荔lì 红hóng 尘chén 进jìn 玉yù 妃fēi [24] 。霸bà 王wáng 军jūn 营yíng ,亚yà 父fù 丹dān 心xīn 撞zhuàng 玉yù 斗dǒu [25] ;长cháng 安ān 酒jiǔ 市shì ,谪zhé 仙xiān 狂kuáng 兴xìng 换huàn 银yín 龟guī [26] 。



* * *



[1] 觉奥、参微:都是弄懂深奥微小的道理的意思,多用于教学或宗教方面。

[2] 鱼书:汉乐府《饮马长城窟行》:“客从远方来,遗我双鲤鱼。呼儿烹鲤鱼,中有尺素书。”因之后来称书信为鱼书。雁字:苏武出使匈奴被拘留。汉王朝向匈奴讨还苏武,匈奴推说苏武已死。苏武的随行人员给汉使者出个主意,让他对匈奴单于说:汉天子在上林苑射得一雁,雁脚上绑着苏武的信件,说明他在某某地方。匈奴只好放了苏武。由此后来书信也称雁书、雁字。

[3] 雉朝飞:乐府古题有《雉朝飞》。

[4] 红瘦对绿肥:语出宋李清照《如梦令》:“知否,知否?应是绿肥红瘦。”

[5] 举杯邀月饮:语出李白《月下独酌》:“花间一壶酒,独酌无相亲。举杯邀明月,对影成三人。”

[6] 黄盖能成赤壁捷:黄盖,孙权手下大将,以“苦肉计”诈降曹军,成就赤壁之火攻。

[7] 陈平善解白登危:汉高祖刘邦讨伐反叛韩王信,被匈奴困于白登,七天没有粮食,形势十分危急。据说靠陈平的奇计,方才解围。

[8] 太白书堂,瀑泉垂地三千丈:语出李白《望庐山瀑布》:“飞流直下三千尺,疑是银河落九天。”

[9] 孔明祀庙,老柏参天四十围:语出杜甫《古柏行》:“孔明庙前有老柏,柯如青铜根如石。霜皮溜雨四十围,黛色参天二千尺。”

[10] 严滩:即子陵滩。见东韵“垂钓客”注。邵圃:邵平,秦时为东陵侯。秦亡,种瓜于长安,瓜美,人称东陵瓜。

[11] 靖菊:晋诗人陶潜,性爱菊,“采菊东篱下,悠然见南山”是他的名句。陶死后,谥号为靖节先生,故称靖菊。夷薇:商代末年,孤竹君的两个儿子伯夷和叔齐在周文王处养老。文王死,武王起兵伐纣。伯夷和叔齐坚决反对,阻止不成,则隐居首阳山,采薇而食,意不餐周粟,终竟饿死。

[12] 占鸿渐:《周易·渐》:“渐,女归吉。”爻辞中有“鸿渐于干”“鸿渐于磐”等话,意思是谁占得“鸿渐”一卦,嫁女是吉利的。

[13] 采凤飞:春秋时陈厉公太子陈完,逃亡到齐国,齐懿公打算把女儿许给他,占得一卦,其辞有“凤凰于飞,和鸣锵锵”的话,被认为是吉兆。后代以鸾凤比喻配偶,是这里出典。

[14] 宽宏豁达高皇量:史称刘邦宽宏豁达,心胸开阔。高皇指汉高祖刘邦。

[15] 叱咤喑哑霸王威:叱咤、喑哑都是形容人发怒的声音。楚霸王豪气盖世,所以说霸王威。

[16] 灭项兴刘,狡兔尽时走狗死:韩信帮助刘邦灭掉项羽,被封为楚王,有人告他谋反,刘邦逮捕了他,他说:“果若人言:狡兔死,走狗烹;飞鸟尽,良弓藏;敌国破,谋臣亡。天下已定,我固当烹。”走狗,春秋时越王勾践复国后,范蠡功成身退,留书给文种:“飞鸟尽,良弓藏;狡兔死,走狗烹。越王为人,长颈鸟喙,可与共患难,不可与共乐。子何不去?”文种后称病不上朝,然遭人谗言,言其意欲作乱,越王便赐剑给文种,文种自杀而亡。

[17] 貔貅:传说中的一种猛兽,这里借指勇猛的将士。卧龙:诸葛亮雄才大略,居南阳,时人送给他的雅号叫“卧龙先生”,后为蜀相。

[18] 鸡窗:晋宋处宗有一只极为宠爱的长鸣鸡,一直关在窗户边。后来鸡说人话,与处宗谈论,使处宗言谈技巧大增。后用于代指书房。雁塔:唐朝新科进士于皇帝赐宴后,须前往洛阳慈恩塔题写姓名。后比喻科举中试,金榜题名。

[19] 秋榜:秋试(乡试)后所发的榜。亦借指秋试。春闱:明、清会试都在春季,故名。

[20] 乌衣巷:六朝时金陵一个居住区,位于今南京市东南。东晋时王导、谢安等贵族多居此,故世称王谢子弟为乌衣郎。

[21] 燕子矶:地名。位于江苏省南京市北的观音山上。前临长江,形如飞燕,故名。

[22] 窈窕:形容女子摇曳多姿的样子。

[23] 蟠桃紫阙来金母:班固《汉武故事》说神人西王母来见汉武帝,拿出五个桃子,送给武帝两个,即所谓蟠桃。金母,即西王母,按五行学说,西方属金,故称金母。

[24] 岭荔红尘进玉妃:岭荔:史载唐代杨贵妃喜食荔枝,玄宗命人自岭南限七日快马送至长安。杜牧诗有“长安回望绣成堆,山顶千门次第开。一骑红尘妃子笑,无人知是荔枝来”的句子。

[25] 霸王军营,亚父丹心撞玉斗:在秦末农民大起义中,刘邦率兵攻入函谷关,占了秦都咸阳。项羽随后赶到,打算同刘邦决战。刘邦势小,只好到项羽驻军的鸿门去陪罪。项羽宴请刘邦,项羽的谋士范增几次示意杀害刘邦都没有成功。刘邦走后,范增把刘邦赠送的玉斗摔在地上,用剑击破,说:“竖子不足与谋也。”发泄他对项羽的不满。这就是有名的鸿门宴。亚父,范增年高望重,被项羽尊称为亚父。丹心,指范增对项羽的一片忠心。

[26] 长安酒市,谪仙狂兴换银龟:传说李白初到长安,拿出所作的《蜀道难》给当时的名诗人贺知章看,贺十分赞赏,称之为“谪仙”,于是解下金龟换酒,与之畅饮尽日。传说李白也曾以银龟换酒。这都表示诗人们的轻视富贵、狂放不羁。金龟、银龟,唐代官员们的佩饰,用以表示官职的级别。谪,封建时代特指贬官。





六liù 鱼yú


羹gēng





[1] 对duì 饭fàn ,柳liǔ 对duì 榆yú 。短duǎn 袖xiù 对duì 长cháng 裾jū 。鸡jī 冠guān 对duì 凤fèng 尾wěi ,芍sháo 药yào 对duì 芙fú 蕖qú 。周zhōu 有yǒu 若ruò [2] ,汉hàn 相xiàng 如rú 。王wáng 屋wū 对duì 匡kuāng 庐lú 。月yuè 明míng 山shān 寺sì 远yuǎn ,风fēng 细xì 水shuǐ 亭tíng 虚xū 。壮zhuàng 士shì 腰yāo 间jiān 三sān 尺chǐ 剑jiàn [3] ,男nán 儿ér 腹fù 内nèi 五wǔ 车chē 书shū [4] 。疏shū 影yǐng 暗àn 香xiāng ,和hé 靖jìng 孤gū 山shān 梅méi 蕊ruǐ 放fàng [5] ;轻qīng 阴yīn 清qīng 昼zhòu ,渊yuān 明míng 旧jiù 宅zhái 柳liǔ 条tiáo 舒shū [6] 。

吾wú 对duì 汝rǔ ,尔ěr 对duì 余yú 。选xuǎn 授shòu 对duì 升shēng 除chú [7] 。书shū 箱xiāng 对duì 药yào 柜guì ,耒lěi 耜sì 对duì 耰yōu 锄chú [8] 。参shēn 虽suī 鲁lǔ [9] ,回huí 不bù 愚yú [10] 。阀fá 阅yuè 对duì 阎yán 闾lǘ [11] 。诸zhū 侯hóu 千qiān 乘shèng 国guó [12] ,命mìng 妇fù 七qī 香xiāng 车jū [13] 。穿chuān 云yún 采cǎi 药yào 闻wén 仙xiān 子zǐ [14] ,踏tà 雪xuě 寻xún 梅méi 策cè 蹇jiǎn 驴lǘ [15] 。玉yù 兔tù 金jīn 乌wū ,二èr 气qì 精jīng 灵líng 为wéi 日rì 月yuè [16] ;洛luò 龟guī 河hé 马mǎ ,五wǔ 行xíng 生shēng 克kè 在zài 图tú 书shū [17] 。

欹qī [18] 对duì 正zhèng ,密mì 对duì 疏shū 。囊náng 橐tuó 对duì 苞bāo 苴jū [19] 。罗luó 浮fú 对duì 壶hú 峤qiáo [20] ,水shuǐ 曲qū 对duì 山shān 纡yū [21] 。骖cān 鹤hè 驾jià ,待dài 鸾luán 舆yú [22] 。桀jié 溺nì 对duì 长cháng 沮jū [23] 。搏bó 虎hǔ 卞biàn 庄zhuāng 子zǐ [24] ,当dǎng 熊xióng 冯féng 婕jié 妤yú [25] 。南nán 阳yáng 高gāo 士shì 吟yín 梁liáng 父fǔ [26] ,西xī 蜀shǔ 才cái 人rén 赋fù 子zǐ 虚xū [27] 。三sān 径jìng 风fēng 光guāng ,白bái 石shí 黄huáng 花huā 供gōng 杖zhàng 履lǚ [28] ;五wǔ 湖hú 烟yān 景jǐng ,青qīng 山shān 绿lǜ 水shuǐ 在zài 樵qiáo 渔yú [29] 。



* * *



[1] 羹:用肉、菜等芶芡煮成的浓汤。

[2] 周有若:有若,孔子弟子,貌似孔子。他是东周春秋时人,故称周有若。

[3] 壮士腰间三尺剑:史称汉高祖刘邦手提三尺剑起兵,因而后人常把三尺剑作为有志男儿的象征。

[4] 男儿腹内五车书:相传战国时学者惠施很有学问,“其书五车”,后来用以称人的博学。

[5] 疏影暗香,和靖孤山梅蕊放:宋林逋性恬淡好古,好作诗,隐居西湖孤山,终身不仕,不娶,以植梅养鹤为乐,世称梅妻鹤子。诗风淡远,多写隐居生活和淡泊心境,卒谥和靖先生。他写的《梅花》诗有“疏影横斜水清浅,暗香浮动月黄昏”的句子,一向为人称道。

[6] 轻阴清昼,渊明旧宅柳条舒:陶渊明写的《五柳先生传》,头几句是:“先生不知何许人也,亦不详其姓字,宅边有五柳树,因以为号焉。”他的诗写自己住宅的环境,有“方宅十余亩,草屋八九间;榆柳荫后檐,桃李罗堂前”的句子。

[7] 选授:量才授官。升除:即除去旧职就新职,由皇帝授予。

[8] 耒耜:翻土所用的农具。耰锄:用来平整田土或击碎土块的农具。

[9] 参虽鲁:参,曾参,孔子的弟子。孔子曾说:“柴也愚,参也鲁。”鲁,迟钝。

[10] 回不愚:回,颜回,孔子弟子颜渊的名。孔子说过:“吾与回言终日,不违,如愚。退而省其私,亦足以发,回也不愚。”

[11] 阀阅:古代官吏们的功劳、阅历。阎闾:大门楼,引申为高贵的社会地位。

[12] 诸侯千乘国:西周制度,诸侯国大者千乘。乘是战车的计量单位,一车四马叫一乘。

[13] 命妇七香车:受有封号的妇女称命妇。七香车,用多种香料涂抹的极为华贵的车。

[14] 穿云采药闻仙子:《幽明录》载,东汉时刘晨、阮肇,入天台山采药迷路,遇两仙女。

[15] 踏雪寻梅策蹇驴:策,马鞭,这里是赶着的意思。蹇驴,瘸驴。相传唐代诗人孟浩然曾骑骞驴于灞上踏雪寻梅,抒其幽兴。

[16] 玉兔金乌,二气精灵为日月:古代神话,说月中有玉兔捣药,日中有三只脚的乌鸦,因以玉兔代月,以金乌代日。古人又认为,宇宙中存在着相互斗争的阴阳二气,天地万物都是由它变化而成,日月则是二气的精华。

[17] 洛龟河马,五行生克在图书:传说伏羲时,黄河出龙马,背负图,称河图;夏禹治水时,神龟从洛水出现,背负书,称洛书。又说龟背上有九组不同点数组成的图画,禹因而排列其次第,乃成治理天下的九种大法,称为洛书。伏羲根据它们画成了八卦。汉孔安国谓河图即八卦。五行即金、木、水、火、土,古人认为它们是构成世界的五种元素。

[18] 欹:倾斜。

[19] 囊橐:盛物的袋子。大称囊,小称橐。或称有底面的叫囊,无底面的叫橐。苞苴:包裹。自上包之叫苞,自下垫之叫苴。

[20] 罗浮对壶峤:《初学记》云,罗浮二山随风雨而合离,壶桥二山逐波涛而下山。

[21] 山纡:山坳。

[22] 鹤驾、鸾舆:都是宗教传说中仙人所乘的车乘,由鹤和鸾凤驾着在空中飞行。骖,在这里是驾驶的意思。

[23] 桀溺、长沮:二人为春秋时隐士。也有人说,长和桀都是身材高大的样子,溺和沮都是污泥。长沮和桀溺就是两个身上沾泥的高个子,并不是人名。

[24] 搏虎卞庄子:卞庄子,鲁人,古代名勇士。传说他看到二虎争一牛,欲刺虎,管竖子劝说道:“两只老虎共食一牛,一定会因为肉味甘美而相互搏斗起来。两虎相斗,大者必伤,小者必死。到那时候您跟在受伤老虎的后面刺杀老虎,就能一举得到刺杀两头老虎的美名。”卞庄子听从劝告,一次刺死两只虎。故有搏双虎之名。

[25] 当熊冯婕妤:婕妤,古代宫廷中女官名。冯婕妤侍汉元帝观虎圈,有熊出,众惊走,冯独挡之,帝深嘉其勇也。

[26] 南阳高士吟梁父:诸葛亮原来隐居南阳,亲自种田,并且特别喜欢唱古曲《梁父吟》。

[27] 西蜀才人赋子虚:西蜀才人指司马相如,他写的《子虚赋》,受到汉武帝极大赞赏,叹不同时。

[28] 三径风光,白石黄花供杖履:语出陶渊明《归去来兮辞》:“三径就荒,松菊犹存。”

[29] 五湖烟景,青山绿水在樵渔:即太湖,古今著名风景区。





七qī 虞yú


红hóng 对duì 白bái ,有yǒu 对duì 无wú 。布bù 谷gǔ 对duì 提tí 壶hú [1] 。毛máo 锥zhuī [2] 对duì 羽yǔ 扇shàn ,天tiān 阙què 对duì 皇huáng 都dū 。谢xiè 蝴hú 蝶dié [3] ,郑zhèng 鹧zhè 鸪gū [4] 。蹈dǎo 海hǎi 对duì 归guī 湖hú [5] 。花huā 肥féi 春chūn 雨yǔ 润rùn ,竹zhú 瘦shòu 晚wǎn 风fēng 疏shū 。麦mài 饭fàn 豆dòu 麋mí 终zhōng 创chuàng 汉hàn [6] ,莼chún 羹gēng 鲈lú 脍kuài 竟jìng 归guī 吴wú [7] 。琴qín 调diào 轻qīng 弹tán ,杨yáng 柳liǔ 月yuè 中zhōng 潜qián 去qù 听tīng ;酒jiǔ 旗qí 斜xié 挂guà ,杏xìng 花huā 村cūn 里lǐ 共gòng 来lái 沽gū 。

罗luó 对duì 绮qǐ ,茗míng 对duì 蔬shū 。柏bǎi 秀xiù 对duì 松sōng 枯kū 。中zhōng 元yuán 对duì 上shàng 巳sì [8] ,返fǎn 璧bì 对duì 还huán 珠zhū [9] 。云yún 梦mèng 泽zé [10] ,洞dòng 庭tíng 湖hú 。玉yù 烛zhú 对duì 冰bīng 壶hú [11] 。苍cāng 头tóu 犀xī 角jiǎo 带dài ,绿lǜ 鬓bìn 象xiàng 牙yá 梳shū 。松sōng 阴yīn 白bái 鹤hè 声shēng 相xiāng 应yìng ,镜jìng 里lǐ 青qīng 鸾luán 影yǐng 不bù 孤gū [12] 。竹zhú 户hù 半bàn 开kāi ,对duì 牖yǒu 不bù 知zhī 人rén 在zài 否fǒu ;柴chái 门mén 深shēn 闭bì ,停tíng 车chē 还hái 有yǒu 客kè 来lái 无wú 。

宾bīn 对duì 主zhǔ ,婢bì 对duì 奴nú 。宝bǎo 鸭yā 对duì 金jīn 凫fú [13] 。升shēng 堂táng 对duì 入rù 室shì [14] ,鼓gǔ 瑟sè 对duì 投tóu 壶hú [15] 。觇chān 合hé 璧bì ,颂sòng 联lián 珠zhū [16] 。提tí 瓮wèng 对duì 当dāng 垆lú [17] 。仰yǎng 高gāo 红hóng 日rì 近jìn [18] ,望wàng 远yuǎn 白bái 云yún 孤gū [19] 。歆xīn 向xiàng 秘mì 书shū 窥kuī 二èr 酉yǒu [20] ,机jī 云yún 芳fāng 誉yù 动dòng 三sān 吴wú [21] 。祖zǔ 饯jiàn 三sān 杯bēi ,老lǎo 去qù 常cháng 斟zhēn 花huā 下xià 酒jiǔ ;荒huāng 田tián 五wǔ 亩mǔ ,归guī 来lái 独dú 荷hè 月yuè 中zhōng 锄chú 。

君jūn 对duì 父fù ,魏wèi 对duì 吴wú 。北běi 岳yuè 对duì 西xī 湖hú 。菜cài 蔬shū 对duì 茶chá 荈chuǎn [22] ,苣jù 藤téng 对duì 菖chāng 蒲pú [23] 。梅méi 花huā 数shù [24] ,竹zhú 叶yè 符fú [25] 。廷tíng 议yì 对duì 山shān 呼hū [26] 。两liǎng 都dū 班bān 固gù 赋fù [27] ,八bā 阵zhèn 孔kǒng 明míng 图tú [28] 。田tián 庆qìng 紫zǐ 荆jīng 堂táng 下xià 茂mào [29] ,王wáng 裒póu 青qīng 柏bǎi 墓mù 前qián 枯kū [30] 。出chū 塞sài 中zhōng 郎láng ,羝dī 有yǒu 乳rǔ 时shí 归guī 汉hàn 室shì [31] ;质zhì 秦qín 太tài 子zǐ ,马mǎ 生shēng 角jiǎo 日rì 返fǎn 燕yān 都dū [32] 。



* * *



[1] 提壶:鸟名。

[2] 毛锥:即毛笔。

[3] 谢蝴蝶:宋谢逸有蝴蝶诗百首,人呼为“谢蝴蝶”。

[4] 郑鹧鸪:唐郑谷写的《鹧鸪》诗,有“雨昏青草湖边过,花落黄陵庙里啼”一联,诗家许为最得神韵,所以被称为郑鹧鸪。

[5] 蹈海:战国时,秦兵围困赵都邯郸,魏王派客将军辛垣衍去劝说赵王,让他尊奉秦昭王为帝,秦兵自退。这事被围困在城中的齐国将士鲁仲连知道,当面批驳了辛垣衍的错误观点,说如果秦真的为帝,自己“有蹈东海而死耳,吾不忍为之民也”。归湖:春秋时范蠡帮助越王勾践灭吴后,功成身退,改名换姓,乘扁舟浮于五湖(即太湖)。

[6] 麦饭豆麋终创汉:汉光武帝刘秀初起兵,在饶阳地方遇到困难,将军冯异在滹沱河为他烧麦饭,在芜娄亭为他煮粥,使他度过难关,终于创立了东汉王朝。糜,粥。

[7] 莼羹鲈脍竟归吴:莼,莼菜,多年生水草,可做汤吃。莼羹:一种用野菜煮成的汤。鲈脍:鲈鱼切成的丝。晋时张翰,由于厌倦官场生活,见秋风起,思念起故乡吴地的莼羹、鲈鱼脍,当即弃官而去。

[8] 中元:农历七月十五日,道教以之为中元节。上巳:农历三月三日,古人称上巳节。

[9] 返璧:战国时,赵国有和氏璧,秦王托言以十五城易之,实际是强行索取。赵使蔺相如奉璧入秦,秦不给城,相如诈说璧有微瑕,请原璧归赵。还珠:相传古代合浦郡不产谷物,只有海中盛产珍珠。许多太守到任后尽力搜刮,宝珠竟然迁往它处。后孟尝君为合浦太守,清廉自奉,宝珠又回来了。

[10] 云梦泽:古代大泽名,在楚(今湖南洞庭湖一带),方九百里,后逐渐干涸,只剩下了洞庭湖。

[11] 冰壶:盛冰的玉壶。用以比喻人的清白,心地纯洁。

[12] 镜里青鸾影不孤:《异苑》载,罽(jì)宾国王买得一只鸾鸟,多年不鸣。夫人说:“听人说鸾鸟找到同类就鸣,何不让它照镜子试一试。”鸾鸟发现镜子里的影像,高声悲鸣,向天空奋力一飞,就死掉了。

[13] 宝鸭对金凫:金凫原为动物名,或称为野鸭。这里宝鸭和金凫都是指古代用来焚香的器具。

[14] 升堂对入室:古代居室建筑,室外有堂。一次孔子评价他的弟子子路,说:“由也,升堂矣,未入于室也。”意思是他已经有了一定的造诣。但还不够理想。

[15] 投壶:上古宴会时的一种游戏。宾主依次将矢投入壶中,多者为胜,少者罚饮。

[16] 觇合璧,颂联珠:古代迷信说法,日月合璧,五星联珠,是太平的征兆。觇,观测。

[17] 提瓮:汉人鲍宣的妻子桓少君喜欢打扮,鲍宣说:“这和我们的家境很不相称。”少君乃去服饰,著布衣,常提瓮出汲,并修妇道。瓮,瓦罐。当垆:卖酒。垆,放置酒器的土台,这里借指酒店。

[18] 仰高红日尽:史载晋元帝太子明帝幼时聪明,其父帝抱以临朝。恰逢有长安使者至,元帝问他:“日与长安孰近乎?”对曰:“长安近,不闻人从日边来。”次日日薄西山宴群臣,帝夸于众,明帝又以为日近。帝问其说,对曰:“举头见日(按:日指他的父亲晋元帝,这是古代崇拜皇帝的说法),不见长安。”众大奇之。

[19] 望远白云孤:狄仁杰客外忆亲曰:“白云飞处为亲所在。”

[20] 歆向秘书窥二酉:刘向、刘歆父子,都是西汉末年著名的学者,曾经多年整理皇家图书,对先秦典籍的整理、流传起了很大作用,刘歆继父业,整理六艺群书,编成《七略》。对经籍目录学有卓越贡献,为中国目录学之始。二酉,即大、小酉山,在湖南沅陵县西北。古代传说,秦时曾有人于此读书,留书千卷于山中。窥二酉,意思是读了许多古代的秘密藏书。

[21] 机云芳誉动三吴:陆机、陆云兄弟,都是西晋初年著名的文学家。吴亡后,与弟陆云至洛阳,为晋太常张华所器重,文名大噪,时称二陆。晋吴郡华亭(今江苏省松江县)人。三吴是二陆的家乡。

[22] 荈:粗茶。

[23] 苣藤:芝麻。菖蒲:植物名。习俗在端午节取叶插于檐下。

[24] 梅花数:古占法。相传为宋代邵雍所作。附会人事,以断吉凶。

[25] 竹叶符:即竹使符。汉代分与郡国守相的信符,右留京师,左留郡国。以竹箭五枚刻字制成。

[26] 廷议:古时在朝廷之上、皇帝面前论辩国事称廷议。山呼:《汉书·武帝纪》载,汉武帝登中岳嵩山,曾听到群山多次呼喊“万岁”。

[27] 两都班固赋:班固是东汉著名史学家、文学家,他曾写了《汉书》。《两都赋》是他辞赋中的代表作。

[28] 八阵孔明图:《三国志》载,孔明曾演八阵图,其遗址甚多,都在四川。八阵,古代作战阵法。

[29] 田庆紫荆堂下茂:《续齐谐记》载,京兆田真、田庆、田广三兄弟商议分居,准备把堂前一棵紫荆树也截为三段。第二天树就枯死了,兄弟大惊,说:树木同株,听说将分就死掉了,难道人还不如树吗?决定不再分居,紫荆树又活了。

[30] 王裒青柏墓前枯:王裒,晋人,其父被文帝杀死,裒攀墓柏号哭,柏忽枯。这是迷信说法。

[31] 出塞中郎,羝有乳时归汉室:中郎,指苏武。汉苏武以中郎将身份出使匈奴,被扣留,匈奴使牧羝羊,告诉他:“羝乳乃得归。”羝,公羊。乳,生羔。

[32] 质秦太子,马生角日返燕都:据《燕丹子》载,战国末年,燕太子丹为质于秦,秦国对他很无礼,于是思归故乡。向秦王恳请,秦王说:“乌鸦白头,马生角,一定放你回去。”太子丹仰天而叹,乌鸦果然白了头,低头落泪;马就生出了角。秦王不得不放他回来。后用以比喻极不可能实现的事情。





八bā 齐qí


鸾luán 对duì 凤fèng ,犬quǎn 对duì 鸡jī 。塞sài 北běi 对duì 关guān 西xī 。长cháng 生shēng 对duì 益yì 智zhì ,老lǎo 幼yòu 对duì 旄máo 倪ní [1] 。颁bān 竹zhú 策cè [2] ,剪jiǎn 桐tóng 圭guī [3] 。剥pū 枣zǎo [4] 对duì 蒸zhēng 梨lí 。绵mián 腰yāo 如rú 弱ruò 柳liǔ ,嫩nèn 手shǒu 似sì 柔róu 荑tí [5] 。狡jiǎo 兔tù 能néng 穿chuān 三sān 穴xué 隐yǐn [6] ,鹪jiāo 鹩liáo 权quán 借jiè 一yī 枝zhī 栖qī [7] 。甪lù 里lǐ 先xiān 生shēng ,策cè 杖zhàng 垂chuí 绅shēn 扶fú 少shào 主zhǔ [8] ;於wū 陵líng 仲zhòng 子zǐ ,辟bì lú 织zhī 履lǚ 赖lài 贤xián 妻qī [9] 。

鸣míng 对duì 吠fèi ,泛fàn 对duì 栖qī 。燕yàn 语yǔ 对duì 莺yīng 啼tí 。珊shān 瑚hú 对duì 玛mǎ 瑙nǎo ,琥hǔ 珀pò 对duì 玻bō 璃lí 。绛jiàng 县xiàn 老lǎo [10] ,伯bó 州zhōu 犁lí [11] 。测cè 蠡lǐ 对duì 燃rán 犀xī [12] 。榆yú 槐huái 堪kān 作zuò 荫yìn ,桃táo 李lǐ 自zì 成chéng 蹊xī [13] 。投tóu 巫wū 救jiù 女nǚ 西xī 门mén 豹bào [14] ,赁lìn 浣huàn 逢féng 妻qī 百bǎi 里lǐ 奚xī [15] 。阙què 里lǐ 门mén 墙qiáng ,陋lòu 巷xiàng 规guī 模mó 原yuán 不bù 陋lòu [16] ;隋suí 堤dī 基jī 址zhǐ ,迷mí 楼lóu 踪zōng 迹jì 亦yì 全quán 迷mí [17] 。

越yuè 对duì 赵zhào ,楚chǔ 对duì 齐qí 。柳liǔ 岸àn 对duì 桃táo 溪xī [18] 。纱shā 窗chuāng 对duì 绣xiù 户hù [19] ,画huà 阁gé 对duì 香xiāng 闺guī [20] 。修xiū 月yuè 斧fǔ [21] ,上shàng 天tiān 梯tī 。蝃dì 蝀dōng [22] 对duì 虹hóng 霓ní 。行xíng 乐lè 游yóu 春chūn 圃pǔ [23] ,工gōng 谀yú 病bìng 夏xià 畦xī [24] 。李lǐ 广guǎng 不bù 封fēng 空kōng 射shè 虎hǔ [25] ,魏wèi 明míng 得dé 立lì 为wèi 存cún 麑ní [26] 。按àn 辔pèi [27] 徐xú 行xíng ,细xì 柳liǔ [28] 功gōng 成chéng 劳láo 王wáng 敬jìng ;闻wén 声shēng 稍shāo 卧wò ,临lín 泾jīng [29] 名míng 震zhèn 止zhǐ 儿ér 啼tí 。



* * *



[1] 旄倪:老人和小孩。旄,通“耄”,老人。倪,小儿。

[2] 颁竹策:皇帝给诸侯王颁发的委任状,以竹简为之。

[3] 剪桐圭:圭,古代帝王诸侯举行礼仪时所用的玉器,上尖下方,代表官阶。相传周成王同他的小弟弟叔虞开玩笑,用桐叶剪成圭形,赠给他说,封你为侯。大臣进来贺喜,成王说:这是开玩笑。大臣说:天子无戏言。最后只好把叔虞封于唐。

[4] 剥枣:剥,同扑,打。

[5] 嫩手似柔荑:《诗经·卫风·硕人》写卫庄公夫人之美,说“手如柔荑,肤如凝脂”。荑:初生的茅芽,色白且柔嫩,用以比喻女子的手细白柔美。

[6] 狡兔能穿三穴隐:战国时,齐公子孟尝君出谋划策,谋求安稳的地位,说,狡兔有三窟,国君也应当如此。意思是多方采取措施,寻找几条出路。

[7] 鹪鹩权借一枝栖:鹪鹩,一种食小虫的极小的鸟,又名“巧妇鸟”。《庄子》上说:“鹪鹩栖树,不过一枝。”意思是容易满足。

[8] 甪里先生,策杖垂绅扶少主:汉初,商山有四个隐士,名东园公、绮里季、夏黄公、甪里先生,因为年老须发皆白,所以称四皓。相传高祖刘邦没能聘请他们出来,后高祖立吕后子惠帝为太子,继又欲以赵王如意易之。吕后用张良计,请四皓辅佐太子,帝见之曰“幸烦公等善为调护”,遂不见废。

[9] 於陵仲子,辟 织履赖贤妻:於陵仲子,即陈仲子,战国时齐国的隐士。因居于於陵,故号於陵子。《孟子》上记载他“身织屦,妻辟 ”。织屦即织草鞋。辟 ,原为剥麻,染麻。辟 指将分练过的麻搓成线。麻是古代纺织原料之一。 ,布缕,引申为织布。楚王欲以为相,不就,与妻逃去,为人灌园,妻子辟 织履。

[10] 绛县老:即绛县老人。《左传》记载,晋绛县一位老人,不知道自己究竟多大年纪,只知道出生那年初一是甲子日。人们去问师旷,师旷说,他已经七十三岁了。

[11] 伯州犁:春秋时晋国大夫伯宗之子伯嚭,因其父被杀,奔楚,为太宰。

[12] 测蠡:蠡,贝壳做的瓢。管窥天,蠡测海,喻见小也,自不量力。燃犀:烛照明察。相传燃烧犀角可以照妖,晋温峤路过渚矶,人们说水下有怪物,温峤用点燃的犀角照之,果然见到许多奇形异状的精灵。夜梦人曰:“幽明道别,何苦相逼。”这是迷信传说。后比喻洞察事理或奸邪。

[13] 桃李自成蹊:《史记·李将军传赞》:“谚曰:‘桃李不言,下自成蹊。’此言虽小,可以喻大也。”比喻一个人如果有高德美才,不用自我声张,自然得到人们的敬爱。蹊,小路。

[14] 投巫救女西门豹:战国魏文侯时,邺地三老、廷掾,与巫祝勾结,假托河伯欲娶妻,每年强选少女,投入河中,愚弄人民并榨取钱财。后西门豹为邺令,在河伯娶妇时,托言所选女子不美,要巫祝、三老去与河伯商量,另行选送,便将其投入河中,因而制止了利用迷信虐害人民的恶行。

[15] 赁浣逢妻百里奚:赁,本意为租借,这里指雇用。浣,洗。《风俗通》载,春秋时百里奚为秦相,赁一浣妇,歌曰:“百里奚,五羊皮,忆别时,烹伏雌,舂黄 ,烦扊扅,今日富贵忘我为?”问她是谁,原来是被百里奚抛弃在故乡的妻子。

[16] 阙里门墙,陋巷规模原不陋:阙里,孔子居住的里巷名。陋巷,孔子弟子颜渊所居,狭小的巷子。引申为狭窄简陋的住处。孔子曾夸奖颜渊:“一箪食,一瓢饮,在陋巷。人不堪其忧,回也不改其乐。”后来唐刘禹锡作《陋室铭》说:“君子居之,何陋之有?”意思是,只要有德者居住,陋巷也不简陋。

[17] 隋堤基址,迷楼踪迹亦全迷:隋炀帝为游江都,开凿了大运河,在两岸栽种杨柳,堤长一千三百余里,称隋堤。迷楼,传说也是隋炀帝所建,用以寻欢作乐的地方。两句的意思是:隋堤也好,迷宫也罢,都成了历史的残迹,当年的迷宫如今真的迷失荒草中了。

[18] 桃溪:指桃源。

[19] 纱窗:蒙纱的窗户。绣户:雕绘华美的门户。多指妇女居室。

[20] 画阁:彩绘华丽的楼阁。香闺:指青年女子的内室。

[21] 修月斧:传说唐代有人登嵩山,看见有人卧在道旁,问他为什么在道旁酣睡。那人回答说:“月亮由七宝合成,要由八万二千户人经常修理,我是其中的一个。”说着拿出身边的斧凿。

[22] 蝃蝀:古时称虹为蝃蝀。

[23] 春圃:春日的园圃。

[24] 夏畦:于炎夏中耕田,比喻勤苦工作。

[25] 李广不封空射虎:《史记·李将军传》:西汉李广守北平,出猎,见草中石以为虎,射之,箭没石中,以为奇。李广一生战功卓著,却不得封侯。

[26] 魏明得立为存麑:魏明帝曹叡小时候随父射猎,文帝射死母鹿,让明帝去射小鹿。明帝不肯,说:“陛下已杀其母,臣不忍复杀其子。”同时流下了眼泪。文帝于是决心让他继承王位。

[27] 按辔:勒住马。

[28] 细柳:汉代周亚夫为将军时,屯兵于细柳,军纪森严,天子欲入军营,亦须依军令行事。

[29] 临泾:西汉赤玼守原州,虏不过临泾,人常道其名以吓唬小儿,使之不敢啼哭。





九jiǔ 佳jiā


门mén 对duì 户hù ,陌mò 对duì 街jiē 。枝zhī 叶yè 对duì 根gēn 荄gāi [1] 。斗dòu 鸡jī 对duì 挥huī 麈zhǔ [2] ,凤fèng 髻jì 对duì 鸾luán 钗chāi [3] 。登dēng 楚chǔ 岫xiù [4] ,渡dù 秦qín 淮huái [5] 。子zǐ 犯fàn 对duì 夫fū 差chāi [6] 。石shí 鼎dǐng 龙lóng 头tóu 缩suō [7] ,银yín 筝zhēng 雁yàn 翅chì 排pái [8] 。百bǎi 年nián 诗shī 礼lǐ 延yán 余yú 庆qìng [9] ,万wàn 里lǐ 风fēng 云yún 入rù 壮zhuàng 怀huái [10] 。能néng 辨biàn 明míng 伦lún ,死sǐ 矣yǐ 野yě 哉zāi 悲bēi 季jì 路lù [11] ;不bù 由yóu 径jìng 窦dòu ,生shēng 乎hū 愚yú 也yě 有yǒu 高gāo 柴chái [12] 。

冠guān 对duì 履lǚ ,袜wà 对duì 鞋xié 。海hǎi 角jiǎo 对duì 天tiān 涯yá 。鸡jī 人rén 对duì 虎hǔ 旅lǚ [13] ,六liù 市shì 对duì 三sān 街jiē [14] 。陈chén 俎zǔ 豆dòu ,戏xì 堆duī 埋mái [15] 。皎jiǎo 皎jiǎo 对duì 皑ái 皑ái [16] 。贤xián 相xiàng 聚jù 东dōng 阁gé [17] ,良liáng 朋péng 集jí 小xiǎo 斋zhāi 。梦mèng 里lǐ 山shān 川chuān 书shū 越yuè 绝jué [18] ,枕zhěn 边biān 风fēng 月yuè 记jì 齐qí 谐xié [19] 。三sān 径jìng 萧xiāo 疏shū ,彭péng 泽zé 高gāo 风fēng 怡yí 五wǔ 柳liǔ ;六liù 朝cháo 华huá 贵guì ,琅láng 琊yá 佳jiā 气qì 种zhòng 三sān 槐huái [20] 。

勤qín 对duì 俭jiǎn ,巧qiǎo 对duì 乖guāi 。水shuǐ 榭xiè 对duì 山shān 斋zhāi [21] 。冰bīng 桃táo 对duì 雪xuě 藕ǒu ,漏lòu 箭jiàn [22] 对duì 更gēng 牌pái 。寒hán 翠cuì 袖xiù [23] ,贵guì 荆jīng 钗chāi [24] 。慷kāng 慨kǎi 对duì 诙huī 谐xié 。竹zhú 径jìng 风fēng 声shēng 籁lài [25] ,花huā 溪xī 月yuè 影yǐng 筛shāi [26] 。携xié 囊náng [27] 佳jiā 韵yùn 随suí 时shí 贮zhù ,荷hè 锄chú [28] 沉chén 酣hān 到dào 处chù 埋mái 。江jiāng 海hǎi 孤gū 踪zōng ,雪xuě 浪làng 风fēng 涛tāo 惊jīng 旅lǚ 梦mèng ;乡xiāng 关guān [29] 万wàn 里lǐ ,烟yān 峦luán 云yún 树shù 切qiè 归guī 怀huái 。

杞qǐ 对duì 梓zǐ ,桧guì 对duì 楷jiē 。水shuǐ 泊pō [30] 对duì 山shān 崖yá 。舞wǔ 裙qún 对duì 歌gē 袖xiù ,玉yù 陛bì 对duì 瑶yáo 阶jiē [31] 。风fēng 入rù 袂mèi ,月yuè 盈yíng 怀huái 。虎hǔ 兕sì [32] 对duì 狼láng 豺chái 。马mǎ 融róng 堂táng 上shàng 帐zhàng [33] ,羊yáng 侃kǎn 水shuǐ 中zhōng 斋zhāi [34] 。北běi 面miàn 黉hóng 宫gōng 宜yí 拾shí 芥jiè [35] ,东dōng 巡xún 岱dài 畤zhì 定dìng 燔fán 柴chái [36] 。锦jǐn 缆lǎn 春chūn 江jiāng ,横héng 笛dí 洞dòng 箫xiāo 通tōng 碧bì 落luò [37] ;华huá 灯dēng 夜yè 月yuè ,遗yí 簪zān 堕duò 翠cuì 遍biàn 香xiāng 街jiē [38] 。



* * *



[1] 根荄:植物的根。斗鸡:古时让鸡与鸡相搏斗的一种游戏。

[2] 挥麈:晋代人们清谈时,常挥麈以为谈助,后称谈论为挥麈。麈,古书上指鹿一类的动物,其尾可做拂尘,即“麈尾”。

[3] 鸾钗:鸾形的钗子。

[4] 楚岫:楚地山峦。

[5] 秦淮:河名。流经南京,是南京市名胜之一。

[6] 子犯:即狐偃,字子犯,春秋晋人。为晋文公舅,故亦称为舅犯。夫差:差,为压韵可读chā。春秋时的吴王,因父阖闾为越王勾践所败,故败困勾践于会稽,以报父仇,并率精兵北会诸侯于黄池,与晋争霸,勾践乘虚而入,遂灭吴,夫差自刭而死,在位二十三年。

[7] 石鼎:陶制的烹茶用具。龙头:当指石鼎上的龙头形装饰。

[8] 银筝:用银装饰的筝或用银字表示音调高低的筝。雁翅:当指古筝上的琴码。

[9] 诗礼:旧时常用来称读书讲究礼教的人家。余庆:指留给子孙后辈的德泽。

[10] 壮怀:豪壮的胸怀。

[11] 季路:姓仲,名由,字子路,一字季路。孔子弟子,性好勇、事亲孝。

[12] 高柴:孔子门人。遇卫难不径不窦(既不走小路,又不走孔道,不知变通)。

[13] 鸡人:职官名。于天将亮时,报时以警醒百官。虎旅:勇猛善战的军队。

[14] 六市、三街:街市。亦作三街六巷。

[15] 陈俎豆,戏堆埋:《列女传·母仪》载,孟子幼时,居近墓,习堆埋;移舍于市,又习贸易事;移学宫旁,乃习礼让,修俎豆。修俎豆,主持祭祀之礼。俎豆,古代祭祀、宴飨时,用来盛祭品的两种礼器。亦泛指各种礼器。

[16] 皎皎、皑皑:洁白的样子。

[17] 东阁:东向的小门。

[18] 越绝:《越绝书》,历史小说。记载春秋末年与战国初期吴越争霸的历史故事。

[19] 齐谐:《齐谐》,志怪书名。

[20] 三槐:宋代兵部侍郎王佑,多阴德,手植三槐于庭,自言子孙必有为三公的。其子旦后果为相,世称为三槐王氏,子孙因建三槐堂。

[21] 山斋:山中居室。

[22] 漏箭:古代漏壶中用作计时指针的箭。

[23] 翠袖:青绿色衣袖。泛指女子的装束。

[24] 荆钗:用荆木做的发钗。代指与丈夫同甘共苦的贤惠的妻子。

[25] 竹径:竹林中的小径。籁:本指从孔窍中所发出的声音,后泛指一切的声音。

[26] 筛:洒、落。

[27] 携囊:李贺系囊贮诗。

[28] 荷锄:晋人刘伶,好酒。荷锄自随曰:“醉死便可埋我。”

[29] 乡关:故乡。

[30] 水泊:湖泽。

[31] 玉陛:帝王宫殿的台阶。瑶阶:玉砌的台阶。亦用为石阶的美称。

[32] 虎兕:虎与犀牛。比喻凶恶残暴的人。

[33] 马融堂上帐:马融字季长,茂陵(今陕西省兴平县东北)人,东汉学者。从学者常千数,注《孝经》、《论语》、《诗》、《易》、《尚书》三《礼》等。马融堂前教授生徒,后设绛纱帐,置女乐。

[34] 羊侃水中斋:南朝梁羊侃,好奢侈,结舟为斋,亭馆皆备,日事游宴。

[35] 黉宫:古代学校名。拾芥:捡取地上的草芥。比喻取之极易。

[36] 岱:泰山。畤:古代祭天地五帝之处。燔柴:烧柴,祭天之礼。

[37] 锦缆:锦制的精美的缆绳。碧落:天空。

[38] 华灯:雕饰华美而光辉灿烂的灯。遗簪:指失落的簪子。香街:指繁华的街道。





十shí 灰huī


春chūn 对duì 夏xià ,喜xǐ 对duì 哀āi 。大dà 手shǒu 对duì 长cháng 才cái [1] 。风fēng 清qīng [2] 对duì 月yuè 朗lǎng ,地dì 阔kuò 对duì 天tiān 开kāi 。游yóu 阆làng 苑yuàn [3] ,醉zuì 蓬péng 莱lái [4] 。七qī 政zhèng 对duì 三sān 台tái [5] 。青qīng 龙lóng 壶hú 老lǎo 杖zhàng [6] ,白bái 燕yàn 玉yù 人rén 钗chāi [7] 。香xiāng 风fēng 十shí 里lǐ 望wàng 仙xiān 阁gé [8] ,明míng 月yuè 一yī 天tiān 思sī 子zǐ 台tái [9] 。玉yù 橘jú 冰bīng 桃táo [10] ,王wáng 母mǔ 几jǐ 因yīn 求qiú 道dào 降jiàng ;莲lián 舟zhōu 藜lí 杖zhàng [11] ,真zhēn 人rén 原yuán 为wèi 读dú 书shū 来lái 。

朝zhāo 对duì 暮mù ,去qù 对duì 来lái 。庶shù 矣yǐ 对duì 康kāng 哉zāi [12] 。马mǎ 肝gān 对duì 鸡jī 肋lèi [13] ,杏xìng 眼yǎn 对duì 桃táo 腮sāi 。佳jiā 兴xìng 适shì ,好hǎo 怀huái 开kāi 。朔shuò 雪xuě [14] 对duì 春chūn 雷léi 。云yún 移yí zhī 鹊què 观guàn [15] ,日rì 晒shài 凤fèng 凰huáng 台tái [16] 。河hé 边biān 淑shū 气qì [17] 迎yíng 芳fāng 草cǎo ,林lín 下xià 轻qīng 风fēng 待dài 落luò 梅méi [18] 。柳liǔ 媚mèi 花huā 明míng ,燕yàn 语yǔ 莺yīng 声shēng 浑hún 是shì 笑xiào ;松sōng 号háo 柏bǎi 舞wǔ ,猿yuán 啼tí 鹤hè 唳lì 总zǒng 成chéng 哀āi 。

忠zhōng 对duì 信xìn ,博bó 对duì 赅gāi 。忖cǔn 度duó 对duì 疑yí 猜cāi [19] 。香xiāng 消xiāo 对duì 烛zhú 暗àn [20] ,鹊què 喜xǐ 对duì 蛩qióng [21] 哀āi 。金jīn 花huā 报bào [22] ,玉yù 镜jìng 台tái [23] 。倒dǎo 斝jiǎ 对duì 衔xián 杯bēi [24] 。岩yán 巅diān 横héng 老lǎo 树shù ,石shí 磴dèng [25] 覆fù 苍cāng 苔tái 。雪xuě 满mǎn 山shān 中zhōng 高gāo 士shì 卧wò [26] ,月yuè 明míng 林lín 下xià 美měi 人rén 来lái [27] 。绿lǜ 柳liǔ 沿yán 堤dī ,皆jiē 因yīn 苏sū 子zǐ [28] 来lái 时shí 种zhòng ;碧bì 桃táo 满mǎn 观guàn ,尽jìn 是shì 刘liú 郎láng [29] 去qù 后hòu 栽zāi 。



* * *



[1] 大手:犹高手。指工于文辞的名家。长才:优异的才能。

[2] 风清:风轻柔而凉爽。

[3] 阆苑:阆风之苑,神话传说的仙人居地。

[4] 蓬莱:神话传说中的海上仙山之一。

[5] 七政:日、月和金、木、水、火、土五星。三台:古有灵台、时台、囿台,合称三台。

[6] 青龙壶老杖:《后汉书·费长房传》载,东汉费长房从壶公学仙,辞归,壶公给他一竹杖,说:骑之可以到家,长房到家后把杖投入葛陂,杖化为龙。

[7] 白燕玉人钗:汉武帝升平元年,建招灵阁,有女神留玉钗与帝,后化为玉燕升天。

[8] 望仙阁:南朝陈后主建。

[9] 思子台:汉武帝逼死了被诬陷的太子刘据,后来帝知其冤,作思子台。

[10] 玉橘冰桃:《汉武外传》载,王母降汉武宫中,享帝以玉橘、冰桃、雪藕。

[11] 莲舟藜杖:传说太乙真人坐莲舟,燃藜杖,降天禄阁,照刘向读书。

[12] 康哉:《尚书·益稷》:“﹝皋陶﹞乃赓载歌曰:‘元首明哉,股肱良哉,庶事康哉。’”歌词称颂君明臣良,诸事安宁。后遂以“康哉”为歌颂太平之词。

[13] 马肝:马肝味劣,比喻卑微琐碎的事。鸡肋:与鸡的肋骨一样无味。比喻没有味道或少有实惠。

[14] 朔雪:北方的雪。

[15] 鹊观:古代道观名。 鹊:鸟纲雀形目鸣禽类。

[16] 凤凰台:在江苏南京市。

[17] 淑气:温和怡人的气息。

[18] 落梅:汉应劭《风俗通》:五月有落梅风,江淮以为信风。

[19] 忖度:思量、考虑。疑猜:猜疑。古典诗词戏曲中为和韵脚常将一个词中的两个字颠倒使用。

[20] 香消:比喻女子死去。烛暗:人死去的通称。

[21] 蛩:蟋蟀的别名。

[22] 金花报:古代状元及第时寄家信报喜,称为金花报。

[23] 玉镜台:温峤娶其姑之女,以玉镜台为聘。

[24] 斝:古代青铜制的酒器,圆口,三足。衔杯:口含酒杯。多指饮酒。

[25] 石磴:以石头铺砌成的台阶。

[26] 高士卧:《后汉书·袁安传》载,袁安遇雪天在家高卧不出,人以为贤,举为孝廉。

[27] 美人来:隋赵师雄游罗浮山,日暮见一美人邀共饮,雄不觉醉卧。醒来在梅花树下,翠羽嘈唧其上,月落参横,惆怅不已。

[28] 苏子:苏轼守杭州,令西湖沿堤种桃柳,人号苏公堤,简称苏堤。

[29] 刘郎:语出刘禹锡《元和十一年自朗州召至京,戏赠看花诸君子》诗:“紫陌红尘拂面来,无人不道看花回。玄都观里桃千树,尽是刘郎去后栽。”





十shí 一yī 真zhēn


莲lián 对duì 菊jú ,凤fèng 对duì 麟lín 。浊zhuó 富fù 对duì 清qīng 贫pín [1] 。渔yú 庄zhuāng 对duì 佛fó 舍shè [2] ,松sōng 盖gài [3] 对duì 花huā 茵yīn 。萝luó 月yuè 叟sǒu [4] ,葛gě 天tiān 民mín [5] 。国guó 宝bǎo 对duì 家jiā 珍zhēn [6] 。草cǎo 迎yíng 金jīn 埒liè [7] 马mǎ ,花huā 醉zuì 玉yù 楼lóu [8] 人rén 。巢cháo 燕yàn 三sān 春chūn 尝cháng 唤huàn 友yǒu [9] ,塞sài 鸿hóng 八bā 月yuè 始shǐ 来lái 宾bīn [10] 。古gǔ 往wǎng 今jīn 来lái ,谁shuí 见jiàn 泰tài 山shān 曾céng 作zuò 砺lì [11] ;天tiān 长cháng 地dì 久jiǔ ,人rén 传chuán 沧cāng 海hǎi 几jǐ 扬yáng 尘chén [12] 。

兄xiōng 对duì 弟dì ,吏lì 对duì 民mín 。父fù 子zǐ 对duì 君jūn 臣chén 。勾gōu 丁dīng 对duì 甫fǔ 甲jiǎ [13] ,赴fù 卯mǎo 对duì 同tóng 寅yín [14] 。折zhé 桂guì 客kè [15] ,簪zān 花huā 人rén [16] 。四sì 皓hào 对duì 三sān 仁rén [17] 。王wáng 乔qiáo 云yún 外wài 舃xì [18] ,郭guō 泰tài 雨yǔ 中zhōng 巾jīn [19] 。人rén 交jiāo 好hǎo 友yǒu 求qiú 三sān 益yì [20] ,士shì 有yǒu 贤xián 妻qī 备bèi 五wǔ 伦lún [21] 。文wén 教jiào 南nán 宣xuān ,武wǔ 帝dì 平píng 蛮mán 开kāi 百bǎi 越yuè [22] ;义yì 旗qí 西xī 指zhǐ ,韩hán 侯hóu 扶fú 汉hàn 卷juǎn 三sān 秦qín [23] 。

申shēn 对duì 午wǔ ,侃kǎn 对duì 訚yín [24] 。阿ē 魏wèi 对duì 茵yīn 陈chén [25] 。楚chǔ 兰lán 对duì 湘xiāng 芷zhǐ [26] ,碧bì 柳liǔ 对duì 青qīng 筠yún [27] 。花huā 馥fù 馥fù ,叶yè 蓁zhēn 蓁zhēn [28] 。粉fěn 颈jǐng 对duì 朱zhū 唇chún 。曹cáo 公gōng 奸jiān 似sì 鬼guǐ [29] ,尧yáo 帝dì 智zhì 如rú 神shén [30] 。南nán 阮ruǎn 才cái 郎láng 差chā 北běi 富fù [31] ,东dōng 邻lín 丑chǒu 女nǚ 效xiào 西xī 颦pín [32] 。色sè 艳yàn 北běi 堂táng ,草cǎo 号hào 忘wàng 忧yōu [33] 忧yōu 甚shèn 事shì ?香xiāng 浓nóng 南nán 国guó ,花huā 名míng 含hán 笑xiào [34] 笑xiào 何hé 人rén ?



* * *



[1] 浊富:不义而富。与“清贫”相对。清贫:生活清寒贫苦。

[2] 渔庄:渔村。佛舍:寺院房舍,佛堂。

[3] 松盖:谓乔松枝叶茂密,状如伞盖。

[4] 萝月叟:月下走在藤萝盘绕的山路上的老人。萝月,萝藤间的月色。

[5] 葛天民:传说中的上古帝王,其治世不言而信,不化而行,是远古社会理想化的政治领袖人物。古人认为是理想中的自然、淳朴之世。

[6] 家珍:家中的珍贵物品。

[7] 金埒:埒即勒,马具。

[8] 玉楼:华丽的楼。

[9] 巢燕三春尝唤友:语出《诗经·小雅·伐木》:“伐木丁丁,鸟鸣嘤嘤,出自幽谷,迁于乔木。嘤其鸣矣,求其友声。”

[10] 塞鸿八月始来宾:塞北的鸿雁直到八月才会飞到南方去做客。称之为宾,因为塞北才是雁的家乡,经过中原好象客人一样。

[11] 泰山曾作砺:汉代封功臣、皇帝封爵的誓词有“黄河如带,泰山若砺。国以永宁,爰及苗裔”的话,意思是遥远无期,不可能出现的情况。砺,磨刀石。

[12] 沧海几扬尘:犹言沧海桑田。《神仙传》载,仙人麻姑在蔡经家见到王远,说自己曾见东海三为桑田,目前东海水又浅,大约要变成陆地。王远叹息说:圣人都说海中将要扬起尘土了。

[13] 勾丁:即征兵。甫甲:即补甲,补充兵员。

[14] 赴卯:古代官府把检查出勤情况叫做点卯(因为卯时日出,开始工作),赴卯犹如今天说上班。同寅:同僚。

[15] 折桂客:晋都诜举贤,对策最优,自己夸口说:“犹桂林之一枝,昆山之片玉。”后因以考试得中为折桂。

[16] 簪花人:古代殿试得中,则赏令簪花,以显其荣。

[17] 四皓:商山四皓的简称,汉初商山的四个隐士。三仁:殷商末年,有微子、箕子、比干三个贤人。三人劝谏纣王,不被采纳,纣王的庶兄微子逃往国外,叔父箕子装疯做奴隶,比干因进谏而被杀,俱以仁德见称于世。孔子评价他们说“殷有三仁”。

[18] 王乔云外舃:《后汉书》载,汉人王乔做叶县县令,有神术,每月两次朝见皇帝。皇帝对他来去这么迅速感动惊异,叫人暗地观察。有人报告,王乔每次来朝,只见有一对凫雁飞来。人们用网捕捉这双飞雁,却只捉得了一只鞋。舃,鞋。

[19] 郭泰雨中巾:汉代郭泰是个有名望的人物,一次遇雨,头巾折起一角,人们以为他是有意这样做的,很雅观,于是效之,故意把头巾折起一角,称为“宗林(郭泰字)巾”。

[20] 三益:语出《论语·季氏》:孔子曰:“益者三友,损者三友。友直、友谅、友多闻,益矣。”三益指直、谅、多闻。

[21] 五伦:古代指君臣﹑父子、兄弟﹑夫妻﹑朋友之间的五种伦理体系。

[22] 文教南宣,武帝平蛮开百越:汉武帝时,统一南方百越之地,议立南海、苍梧等九郡。文教,文明、教化。南宣,推广到南方。百越,古代散居南方各地越族的总称,居住两广、海南岛一带。如汉时有闽越、瓯越、南越、骆越等。其文化特征为断发、纹身、契臂、巢居、使舟及铸铜鼓等。亦作百粤。

[23] 义旗西指,韩侯扶汉卷三秦:在刘邦和项羽争夺天下的斗争中,韩信作为刘邦的将领,曾南北转战,立下了很大功劳。在他刚刚被举用的时候,曾劝说刘邦,略定三秦。刘邦听从他的意见,尽得关中之地,为楚汉之争的胜利打下了基础。韩侯,即韩信。三秦,战国时秦的国土,在今陕西。秦亡后,项羽把关中地分为三份,封秦降将章邯为雍王于咸阳以西,司马欣为塞王于咸阳以东,董翳为翟王于上郡,合称为三秦。

[24] 侃:和乐的样子。訚:态度庄重的样子。

[25] 阿魏、茵陈:两味中药名。

[26] 兰、芷:都是香草,产在古代楚国。湘江在楚国境内,因称芷为湘芷。屈原的诗歌中经常提到这两种香草,用它比喻品行高洁的人物。

[27] 筠:竹。

[28] 蓁蓁:茂盛的样子。

[29] 曹公奸似鬼:三国时曹操奸伪,人称奸鬼。

[30] 尧帝智如神:《史记》上说,帝尧十分聪明,“其智如神”。

[31] 南阮才郎差北富:晋洛阳阮氏家族中的阮籍和阮咸叔侄居道南,家贫而多才;其他阮姓宗族居道北,家富。七月七日,北阮晒衣服,光彩夺目。阮咸也以竹杆把大布裤衩挑了出来。人问其故,他说:“未能免俗,聊复尔耳。”

[32] 东邻丑女效西颦:《庄子》里的一则寓言说,美女西施因胸口痛,经常抚胸口皱眉。东邻丑女也学西施的样子,在人前故意卖弄,却引得人们更加讨厌她。颦,皱眉。

[33] 忘忧:萱草也名忘忧草。

[34] 含笑:花名。





十shí 二èr 文wén


忧yōu 对duì 喜xǐ ,戚qī 对duì 欣xīn 。二èr 典diǎn 对duì 三sān 坟fén [1] 。佛fó 经jīng 对duì 仙xiān 语yǔ ,夏xià 耨nòu 对duì 春chūn 耘yún [2] 。烹pēng 早zǎo 韭jiǔ ,剪jiǎn 春chūn 芹qín 。暮mù 雨yǔ 对duì 朝zhāo 云yún [3] 。竹zhú 间jiān 斜xié 白bái 接jiē [4] ,花huā 下xià 醉zuì 红hóng 裙qún 。掌zhǎng 握wò 灵líng 符fú 五wǔ 岳yuè 箓lù [5] ,腰yāo 悬xuán 宝bǎo 剑jiàn 七qī 星xīng 纹wén [6] 。金jīn 锁suǒ 未wèi 开kāi ,上shàng 相xiàng 趋qū 听tīng 宫gōng 漏lòu [7] 永yǒng ;珠zhū 帘lián 半bàn 卷juǎn ,群qún 僚liáo 仰yǎng 对duì 御yù 炉lú [8] 薰xūn 。

词cí 对duì 赋fù ,懒lǎn 对duì 勤qín 。类lèi 聚jù 对duì 群qún 分fēn [9] 。鸾luán 箫xiāo 对duì 凤fèng 笛dí ,带dài 草cǎo 对duì 香xiāng 芸yún [10] 。燕yān 许xǔ 笔bǐ [11] ,韩hán 柳liǔ 文wén [12] 。旧jiù 话huà 对duì 新xīn 闻wén 。赫hè 赫hè 周zhōu 南nán 仲zhòng [13] ,翩piān 翩piān 晋jìn 右yòu 军jūn [14] 。六liù 国guó 说shuì 成chéng 苏sū 子zǐ 贵guì [15] ,两liǎng 京jīng 收shōu 复fù 郭guō 公gōng 勋xūn [16] 。汉hàn 阙què 陈chén 书shū ,侃kǎn 侃kǎn 忠zhōng 言yán 推tuī 贾jiǎ 谊yì [17] ;唐táng 廷tíng 对duì 策cè ,岩yán 岩yán 直zhí 谏jiàn 有yǒu 刘liú fén [18] 。

言yán 对duì 笑xiào ,绩jì 对duì 勋xūn 。鹿lù 豕shǐ 对duì 羊yáng fén [19] 。星xīng 冠guān 对duì 月yuè 扇shàn [20] ,把bǎ 袂mèi 对duì 书shū 裙qún [21] 。汤tāng 事shì 葛gě [22] ,说yuè 兴xīng 殷yīn [23] 。萝luó 月yuè 对duì 松sōng 云yún 。西xī 池chí 青qīng 鸟niǎo 使shǐ [24] ,北běi 塞sài 黑hēi 鸦yā 军jūn [25] 。文wén 武wǔ 成chéng 康kāng 为wéi 一yī 代dài [26] ,魏wèi 吴wú 蜀shǔ 汉hàn 定dìng 三sān 分fēn [27] 。桂guì 苑yuàn 秋qiū 宵xiāo ,明míng 月yuè 三sān 杯bēi 邀yāo 曲qǔ 客kè [28] ;松sōng 亭tíng 夏xià 日rì ,薰xūn 风fēng 一yī 曲qǔ 奏zòu 桐tóng 君jūn [29] 。



* * *



[1] 二典对三坟:二典指《尚书》中的《尧典》《舜典》两篇。三坟,指三皇伏羲、神农、黄帝之坟,亦指三皇所著之书。此与二典相对,当指三皇所著之书。

[2] 耨:古代锄草的器具。这里当为动词,意为“锄草”,与“耘”相对。耘:锄草。

[3] 暮雨对朝云:据传楚襄王和宋玉一起游览云梦台时,宋玉对楚襄王说:“以前先王,也就是楚怀王曾经游览此地,玩累了便睡着了,梦见一位美丽动人的女子,她说是巫山之女,愿意献出自己的枕头席子给楚王享用。楚王知道弦外有音,非常高兴,立即宠幸那位巫山美女。巫山女临别之时告诉楚怀王:“妾在巫山之阳,高丘之阻。旦为朝云,暮为行雨,朝朝暮暮,阳台之下。”

[4] 竹间斜白接:晋山简为人狂放,做襄阳太守时,经常骑马出游,衣冠颠倒。当时有首民谣说:“山公时一醉,迳造高阳池。日暮倒载归,酩酊无所知。复能乘骏马,倒着白接篱。”白接,即白接篱,当时一种帽子。

[5] 掌握灵符五岳箓:道教传说,修炼到一定程度的道士,可以握三山五岳灵符,统领鬼神。箓,道士画的驱避邪魔的符号、帖子。

[6] 七星纹:宝剑上嵌饰的北斗图案。

[7] 宫漏:即铜壶滴漏,古代宫中计时的用具。

[8] 御炉:御用的香炉。

[9] 类聚对群分:《周易·系辞上》:“方以类聚,物以群分。”

[10] 带草对香芸:相传东汉末年郑康成曾在不其城东南山中教授,所居山下生一种草,叶长尺余,十分坚韧,人们叫它作“康成书带”。香芸,芸香一类的香草,俗称七里香。有特异香气,能去蚤虱,辟蠹奇验,古来藏书家多用以防蠹。

[11] 燕许笔:唐张说封为燕国公,苏颋(tǐng)封为许国公,二人以文章名世,时人称大手笔。

[12] 韩柳文:唐柳宗元、韩愈,文章绝代。

[13] 赫赫周南仲:南仲是周宣王时的大将,他曾率兵击败侵犯周国的少数民族玁狁。

[14] 翩翩:风流潇洒的样子。晋右军:即晋王羲之,著名书法家。他曾做过右军将军,所以人们称他为王右军。

[15] 六国说成苏子贵:战国时,苏秦以合纵术说服了山东六国诸侯,佩六国相印,为总约长。

[16] 两京收复郭公勋:唐郭子仪率兵平息“安史之乱”,收复了长安、洛阳两京,后以功封为汾阳王。

[17] 汉阙陈书,侃侃忠言推贾谊:西汉贾谊是个卓有远见的政治家,他曾上疏汉文帝,直切地指出汉王朝的危机,建议及早采取措施补救。侃侃,形容说话理直气壮,不慌不忙。

[18] 唐廷对策,岩岩直谏有刘 :唐文宗二年,举贤良方正百余人,在皇帝面前对策。进士刘 慷慨直言,切中时弊。但由于考官惧怕宦官的势力,不敢录取。同时对策的河南府参军李邰上疏,宁可把自己的官职让给刘 。后来因宦官的陷害,刘 终竟被贬死。刘 获得了许多正直的知识分子的同情,例如诗人李商隐就有《哭刘 》诗。岩岩,威严。

[19] 鹿豕:鹿和猪。比喻山野无知之物。羊 :相传春秋时鲁大夫季康子掘井,挖到一只瓦缸,里面有一只羊,问孔子,孔子说它是土之怪,叫 羊。

[20] 星冠:道士的帽子。月扇:团扇。形如满月,故称。

[21] 把袂:比喻把臂或握手。袂,衣袖。书裙:晋羊欣年十三,王羲之爱其才。昼卧,王羲之书其白练裙,羊欣视为珍宝,揣摩学习,因此书法遂大进。后以书裙称誉别人的书法,或指文人间的相互雅赏爱慕。

[22] 汤事葛:语出《孟子》。汤,成汤,商朝的第一个王。葛,汤时小国。传说葛伯不祀鬼神,汤曾帮助他祭祀。

[23] 说兴殷:说,傅说,商代人。传说他是奴隶,为人筑墙,后来商王武丁发现了他的才干,举以为三公。

[24] 西池青鸟使:《汉武内传》载,仙人西王母临降人间之前,先有青鸟飞来通报,后来诗词中多以青鸟为传达爱情信息的使者。西池,传说西王母住在西方昆仑山的瑶池。

[25] 北塞黑鸦军:唐李克用统领的守塞军队都穿黑色衣甲,号黑鸦军。

[26] 文武成康为一代:文、武、成、康,西周初的四个王,史称是承平之世。

[27] 魏吴蜀汉定三分:汉代以后魏、蜀、吴三国鼎立。

[28] 桂苑:栽有桂树的林园。曲客:指酒友。曲,造酒的媒质。

[29] 松亭:松间之亭。桐君:古琴名。因桐木可作琴,故以桐君为琴的代称。薰风:传说帝舜得五弦琴,作《南薰之歌》。





十shí 三sān 元yuán


卑bēi 对duì 长zhǎng ,季jì 对duì 昆kūn [1] 。永yǒng 巷xiàng 对duì 长cháng 门mén [2] 。山shān 亭tíng 对duì 水shuǐ 阁gé [3] ,旅lǚ 舍shè 对duì 军jūn 屯tún [4] 。杨yáng 子zǐ 渡dù [5] ,谢xiè 公gōng 墩dūn [6] 。德dé 重zhòng 对duì 年nián 尊zūn [7] 。承chéng 乾qián 对duì 出chū 震zhèn ,叠dié 坎kǎn 对duì 重chóng 坤kūn [8] 。志zhì 士shì 报bào 君jūn 思sī 犬quǎn 马mǎ ,仁rén 王wáng 养yǎng 老lǎo 察chá 鸡jī 豚tún [9] 。远yuǎn 水shuǐ 平píng 沙shā ,有yǒu 客kè 泛fàn 舟zhōu 桃táo 叶yè [10] 渡dù ;斜xié 风fēng 细xì 雨yǔ ,何hé 人rén 携xié 榼kē 杏xìng 花huā 村cūn [11] 。

君jūn 对duì 相xiàng ,祖zǔ 对duì 孙sūn 。夕xī 照zhào 对duì 朝zhāo 曛xūn [12] 。兰lán 台tái 对duì 桂guì 殿diàn [13] ,海hǎi 岛dǎo 对duì 山shān 村cūn 。碑bēi 堕duò 泪lèi [14] ,赋fù 招zhāo 魂hún [15] 。报bào 怨yuàn 对duì 怀huái 恩ēn 。陵líng 埋mái 金jīn 吐tǔ 气qì [16] ,田tián 种zhòng 玉yù 生shēng 根gēn [17] 。相xiàng 府fǔ 珠zhū 帘lián 垂chuí 白bái 昼zhòu ,边biān 城chéng 画huà 角jiǎo [18] 动dòng 黄huáng 昏hūn 。枫fēng 叶yè 半bàn 山shān ,秋qiū 去qù 烟yān 霞xiá 堪kān 倚yǐ 杖zhàng ;梨lí 花huā 满mǎn 地dì ,夜yè 来lái 风fēng 雨yǔ 不bù 开kāi 门mén [19] 。



* * *



[1] 昆:兄长。季:弟弟。

[2] 永巷:汉代拘禁犯罪的妃嫔宫女的地方。长门:汉宫名。

[3] 水阁:靠近水的楼阁。

[4] 旅舍:旅馆。军屯:指驻屯的军队。

[5] 杨子渡:古津渡名,在江苏江都县南。

[6] 谢公墩:山名,在江苏江宁县城北(古代金陵),晋谢安尝居半山,曾登临,故名。

[7] 年尊:年纪大。

[8] 承乾对出震,叠坎对重坤:乾、坤、坎、震,《周易》的四个卦名。乾为龙,所以继位为君称承乾。震为雷声,有发号施令的意思,所以出震是皇帝发号令。

[9] 仁王养老察鸡豚:战国思想家孟轲阐述他的仁政思想,说如果王者施仁政,“鸡豚狗彘之畜,无失其时,七十者可以食肉矣”。豚,泛指猪。

[10] 桃叶渡:在江苏南京市内秦淮河、青溪合流处。据说晋王献之有妾名桃叶,桃叶渡江,以歌送之曰“桃叶复桃叶,渡江不用楫”之语。

[11] 榼:古盛酒器皿。杏花村:在金陵。唐杜牧《清明》诗:“借问酒家何处有?牧童遥指杏花村。”后因以杏花村指卖酒之处。

[12] 夕照:傍晚的阳光。曛:本指日落时的余光。这里指早晨的昏暗的阳光。

[13] 兰台:这里指汉代皇家贮藏图书的府库,又称兰台寺。桂殿:对寺观殿宇的美称。

[14] 碑堕泪:晋羊祜为荆州都督,与东吴相对抗,甚有建树。羊祜死,襄阳民为之罢巿巷哭,为他在岘山建碑立庙,看见碑的人,莫不坠泪,因而称堕泪碑。

[15] 赋招魂:楚辞有《招魂赋》一篇,有人以为是屈原为招怀王之魂而作,有的以为是宋玉哀师屈原之死而作。还有说是屈原自招其魂。

[16] 陵埋金吐气:旧传秦始皇南巡,有望气者说,五百年后,金陵当有天子出。始皇于是埋金于金陵镇山以镇压之,故称金陵。

[17] 田种玉生根:《搜神记》载,杨伯雍家住无终山,山上无水,伯雍担水置路旁,供行人取饮。三年后,有一人饮水,送给他一斗石子,让他种。几年后,石子上生出了玉石。后其地称玉田。

[18] 画角:古管乐器,传自西羌。形如竹筒,本细末大,以竹木或皮革等制成,因表面有彩绘,故称。发声哀厉高亢,古时军中多用以警昏晓,振士气,肃军容。帝王出巡,亦用以报警戒严。

[19] 梨花满地,夜来风雨不开门:唐刘方平《春怨》诗:“寂寞空庭春欲晓,梨花满地不开门。”





十shí 四sì 寒hán


家jiā 对duì 国guó ,治zhì 对duì 安ān 。地dì 主zhǔ 对duì 天tiān 官guān [1] 。坎kǎn 男nán 对duì 离lí 女nǚ [2] ,周zhōu 诰gào 对duì 殷yīn 盘pán [3] 。三sān 三sān 暖nuǎn ,九jiǔ 九jiǔ 寒hán [4] 。杜dù 撰zhuàn 对duì 包bāo 弹tán [5] 。古gǔ 壁bì 蛩qióng 声shēng 匝zā [6] ,闲xián 亭tíng 鹤hè 影yǐng 单dān 。燕yàn 出chū 帘lián 边biān 春chūn 寂jì 寂jì ,莺yīng 闻wén 枕zhěn 上shàng 漏lòu 珊shān 珊shān [7] 。池chí 柳liǔ 烟yān 飘piāo ,日rì 夕xī 郎láng 归guī 青qīng 锁suǒ 闼tà [8] ;砌qì [9] 花huā 雨yǔ 过guò ,月yuè 明míng 人rén 倚yǐ 玉yù 阑lán 干gān 。

肥féi 对duì 瘦shòu ,窄zhǎi 对duì 宽kuān 。黄huáng 犬quǎn 对duì 青qīng 鸾luán [10] 。指zhǐ 环huán 对duì 腰yāo 带dài ,洗xǐ 钵bō 对duì 投tóu 竿gān [11] 。诛zhū 佞nìng 剑jiàn [12] ,进jìn 贤xián 冠guān [13] 。画huà 栋dòng 对duì 雕diāo 栏lán [14] 。双shuāng 垂chuí 白bái 玉yù 箸zhù [15] ,九jiǔ 转zhuǎn 紫zǐ 金jīn 丹dān [16] 。陕shǎn 右yòu 棠táng 高gāo 怀huái 召shào 伯bó [17] ,河hé 南nán 花huā 满mǎn 忆yì 潘pān 安ān [18] 。陌mò 上shàng 芳fāng 春chūn ,弱ruò 柳liǔ 当dāng 风fēng [19] 披pī 彩cǎi 线xiàn ;池chí 中zhōng 清qīng 晓xiǎo [20] ,碧bì 荷hé 承chéng 露lù [21] 捧pěng 珠zhū 盘pán 。

行xíng 对duì 卧wò ,听tīng 对duì 看kàn 。鹿lù 洞dòng [22] 对duì 鱼yú 滩tān 。蛟jiāo 腾téng 对duì 豹bào 变biàn [23] ,虎hǔ 踞jù 对duì 龙lóng 蟠pán [24] 。风fēng 凛lǐn 凛lǐn [25] ,雪xuě 漫màn 漫màn [26] 。手shǒu 辣là 对duì 心xīn 酸suān 。莺yīng 莺yīng 对duì 燕yàn 燕yàn [27] ,小xiǎo 小xiǎo 对duì 端duān 端duān [28] 。蓝lán 水shuǐ 远yuǎn 从cóng 千qiān 涧jiàn 落luò ,玉yù 山shān 高gāo 并bìng 两liǎng 峰fēng 寒hán [29] 。至zhì 圣shèng 不bù 凡fán ,嬉xī 戏xì 六liù 龄líng 陈chén 俎zǔ 豆dòu [30] ;老lǎo 莱lái 大dà 孝xiào ,承chéng 欢huān 七qī 衮gǔn 舞wǔ 斑bān 斓lán [31] 。



* * *



[1] 地主:指住在本地的人。天官:官名。《周礼》分设六官,以天官冢宰居首,总御百官。

[2] 坎男对离女:坎和离都是《周易》卦名,古人解释说坎为中男,离为中女。

[3] 周诰、殷盘:《尚书》中属于西周的文献有《洛诰》《康诰》诸篇,属于殷商的文献有《盘庚》上、中、下三篇。

[4] 三三暖,九九寒:农历三月三日,古人称上巳节。农历九月九日,古人称重阳节。

[5] 杜撰:凭空捏造之事,所谓不经之谈。包弹:宋包拯为御史中丞,弹劾不避权贵,人谓之包弹。

[6] 蛩声:蟋蟀的鸣声。匝:环绕。

[7] 漏:古代计时器,铜制有孔,可以滴水或漏沙,有刻度标志以计时间。简称“漏”。珊珊:形容衣裙玉珮的声音。

[8] 青锁闼:翰林直宿的地方,门上刻画有青色连锁花纹,因称青锁闼。闼,门。

[9] 砌:台阶。

[10] 黄犬:指晋陆机的黄耳犬。曾为陆机长途传递书信。青鸾:古代传说中凤凰一类的神鸟。

[11] 洗钵:即洗钵泉,今位于山东济南李清照纪念堂院内西北隅,为不规则泉池。投竿:投钓竿于水,即垂钓。

[12] 诛佞剑:汉朱云忠直敢谏。成帝的老师安昌侯张禹,在朝廷甚有地位,然毫无作为。朱云对成帝说:“臣愿求赐上方宝剑,断佞臣一人,以厉其余。”上问为谁,曰张禹。帝怒令斩之,云攀殿槛,槛折以免。或请易槛,上不许,存之以旌忠臣。

[13] 进贤冠:文官戴的一种帽子。

[14] 画栋:有彩绘装饰的栋梁。

[15] 白玉箸:道家得道,临终有白玉气出鼻孔,双垂如双玉箸。

[16] 九转紫金丹:古代术士把朱砂烧成水银,又把水银炼成丹药,叫做还丹。九转,形容经过许多步骤。

[17] 陕右棠高怀召伯:召虎是周宣王时的一位大臣,人们称他为召伯。他很有政绩,传说他的住处有一棵甘棠树,他走后,人们对这棵树加意保护,并且作了一首叫《甘棠》的诗歌,以资纪念。陕右,即关中地区。

[18] 河南花满忆潘安:河南疑当作河阳,潘安为河阳令,满县皆栽桃花,人曰花县。

[19] 当风:正对着风。

[20] 清晓:清晨,天刚亮的时候。

[21] 承露:承接甘露。

[22] 鹿洞:指白鹿洞。宋朱熹讲学处。

[23] 豹变:意思是君子的变化像豹一样,越来越有文采。喻润色事业,或迁喜去恶。

[24] 虎踞、龙蟠:诸葛亮论金陵的地形,说:“钟阜龙蟠,石城虎踞。”

[25] 凛凛:寒冷的样子。

[26] 漫漫:空间广远的样子。

[27] 莺莺、燕燕:钱塘范十二郎有二女,曰莺莺燕燕,为富民陆氏妾。

[28] 小小、端端:钱塘妓女苏小小,亦名简简。

[29] 蓝水远从千涧落,玉山高并两峰寒:是杜甫《九日兰田崔氏庄》一诗的腹联。

[30] 至圣不凡,嬉戏六龄陈俎豆:《史记·孔子世家》载:“孔子为儿嬉戏,常陈俎豆,设礼容。”

[31] 老莱大孝,承欢七衮舞斑斓:老莱子,传说中的古孝子,父母年迈,无以为欢,他虽也年纪很大,但仍穿上花花绿绿的幼儿服装,在父母面前嬉笑,引逗双亲开心。





十shí 五wǔ 删shān


林lín 对duì 坞wù





[1] ,岭lǐng 对duì 峦luán 。昼zhòu 永yǒng [2] 对duì 春chūn 闲xián 。谋móu 深shēn 对duì 望wàng 重zhòng ,任rèn 大dà 对duì 投tóu 艰jiān [3] 。裙qún 袅niǎo 袅niǎo [4] ,佩pèi [5] 珊shān 珊shān 。守shǒu 塞sài 对duì 当dāng 关guān [6] 。密mì 云yún 千qiān 里lǐ 合hé ,新xīn 月yuè 一yī 钩gōu 弯wān 。叔shū 宝bǎo 君jūn 臣chén 皆jiē 纵zòng 逸yì [7] ,重chóng 华huá 父fù 母mǔ 是shì 嚚yín 顽wán [8] 。名míng 动dòng 帝dì 畿jī ,西xī 蜀shǔ 三sān 苏sū 来lái 日rì 下xià [9] ;壮zhuàng 游yóu 京jīng 洛luò ,东dōng 吴wú 二èr 陆lù 起qǐ 云yún 间jiān [10] 。

临lín 对duì 仿fǎng ,吝lìn 对duì 悭qiān [11] 。讨tǎo 逆nì 对duì 平píng 蛮mán [12] 。忠zhōng 肝gān 对duì 义yì 胆dǎn ,雾wù 发fà 对duì 云yún 鬟huán 。埋mái 笔bǐ 冢zhǒng [13] ,烂làn 柯kē 山shān 。月yuè 貌mào 对duì 天tiān 颜yán 。龙lóng 潜qián 终zhōng 得dé 跃yuè [14] ,鸟niǎo 倦juàn 亦yì 知zhī 还huán [15] 。陇lǒng 树shù [16] 飞fēi 来lái 鹦yīng 鹉wǔ 绿lǜ ,池chí 筠yún 密mì 处chù 鹧zhè 鸪gū 斑bān 。秋qiū 露lù 横héng 江jiāng ,苏sū 子zǐ 月yuè 明míng 游yóu 赤chì 壁bì [17] ;冻dòng 云yún 迷mí 岭lǐng ,韩hán 公gōng 雪xuě 拥yōng 过guò 蓝lán 关guān [18] 。



* * *



[1] 坞:四面高,中间凹下的地方。

[2] 昼永:白昼漫长。

[3] 投艰:赋予重任。

[4] 袅:随风摆动的样子。

[5] 佩:古代女子头上或身上的佩饰。

[6] 当关:把守关隘。

[7] 叔宝君臣皆纵逸:南朝陈后主,名叔宝,历史上有名的荒淫皇帝。他经常召集江总、孔范等十个文人在一起饮宴,称为“狎客”,让张贵人等八名妃嫔与之交错而坐,整日纵情声色。

[8] 重华父母是嚚顽:重华是帝舜的名。相传他的父亲瞽叟和弟弟象品行都很坏,曾多次设阴谋准备把他害死。嚚顽,愚蠢而顽固。瞽,瞎。

[9] 名动帝畿,西蜀三苏来日下:三苏,指宋著名文学家苏洵和他的儿子苏轼、苏辙。他们都是四川眉山人,名震一时,人称三苏。帝畿,我国古代称靠近国都的地方。这里同句中的“日下”都指都城。

[10] 壮游京洛,东吴二陆起云间:二陆指晋文学家陆机、陆云兄弟,大有才名,人称二陆。他们在东吴亡后,都来到洛阳从政。据说一次陆云遇到荀隐,互相自我介绍,陆说:“云间陆士龙。”荀说:“日下荀鸣鹤。”云间,江苏松江县之古称。壮游,谓怀抱壮志而远游。

[11] 悭:吝啬。

[12] 讨逆:讨伐坏人。蛮:旧指南方少数民族。

[13] 埋笔冢:陈、隋间僧人智永是著名的书法家,相传他写字用笔积十八瓮,后埋成一墓,号曰“退笔冢”。

[14] 龙潜终得跃:《周易·乾卦》:“初九,潜龙勿用。”“九四,或跃在渊。”比喻人或事物由小到大、由弱到强的发展过程。

[15] 鸟倦亦知还:语出晋陶渊明《归去来兮辞》:“云无心以出岫,鸟倦飞而知还。”

[16] 陇树:陇山一带的树木。泛指边塞之树。

[17] 秋露横江,苏子月明游赤壁:元丰四年,苏轼曾月夜泛舟赤壁,作《前赤壁赋》,赋中有“少焉,月出于东山之上,徘徊于斗牛之间。白露横江,水光接天”等语。

[18] 冻云迷岭,韩公雪拥过蓝关:唐文学家韩愈,以上《谏迎佛骨表》触怒宪宗,被贬为潮州刺史,行程中至蓝关遇雪,写了一首《左迁至蓝关示侄孙湘》,“云横秦岭家何在,雪拥蓝关马不前”是诗中名句。





\part{卷下}


一yī 先xiān


寒hán 对duì 暑shǔ ,日rì 对duì 年nián 。蹴cù 踘jū [1] 对duì 秋qiū 千qiān 。丹dān 山shān 对duì 碧bì 水shuǐ ,淡dàn 雨yǔ 对duì 覃tán 烟yān [2] 。歌gē 宛wǎn 转zhuǎn [3] ,貌mào 婵chán 娟juān [4] 。雪xuě 鼓gǔ 对duì 云yún 笺jiān [5] 。荒huāng 芦lú 栖qī 南nán 雁yàn ,疏shū 柳liǔ 噪zào 秋qiū 蝉chán 。洗xǐ 耳ěr 尚shàng 逢féng 高gāo 士shì 笑xiào [6] ,折zhé 腰yāo 肯kěn 受shòu 小xiǎo 儿ér 怜lián [7] 。郭guō 泰tài 泛fàn 舟zhōu ,折zhé 角jiǎo 半bàn 垂chuí 梅méi 子zǐ 雨yǔ [8] ;山shān 涛tāo 骑qí 马mǎ ,接jiē lí 倒dào 着zhuó 杏xìng 花huā 天tiān [9] 。

轻qīng 对duì 重zhòng ,肥féi 对duì 坚jiān [10] 。碧bì 玉yù 对duì 青qīng 钱qián [11] 。郊jiāo 寒hán 对duì 岛dǎo 瘦shòu [12] ,酒jiǔ 圣shèng 对duì 诗shī 仙xiān [13] 。依yī 玉yù 树shù [14] ,步bù 金jīn 莲lián [15] 。凿záo 井jǐng 对duì 耕gēng 田tián [16] 。杜dù 甫fǔ 清qīng 宵xiāo 立lì [17] ,边biān 韶sháo 白bái 昼zhòu 眠mián [18] 。豪háo 饮yǐn 客kè 吞tūn 波bō 底dǐ 月yuè ,酣hān 游yóu 人rén 醉zuì 水shuǐ 中zhōng 天tiān [19] 。斗dòu 草cǎo 青qīng 郊jiāo [20] ,几jǐ 行háng 宝bǎo 马mǎ 嘶sī 金jīn 勒lè [21] ;看kàn 花huā 紫zǐ 陌mò [22] ,千qiān 里lǐ 香xiāng 车chē 拥yōng 翠cuì 钿diàn [23] 。

吟yín 对duì 咏yǒng ,授shòu 对duì 传chuán 。乐lè 矣yǐ 对duì 凄qī 然rán 。风fēng 鹏péng 对duì 雪xuě 雁yàn ,董dǒng 杏xìng 对duì 周zhōu 莲lián [24] 。春chūn 九jiǔ 十shí [25] ,岁suì 三sān 千qiān [26] 。钟zhōng 鼓gǔ 对duì 管guǎn 弦xián 。入rù 山shān 逢féng 宰zǎi 相xiàng [27] ,无wú 事shì 即jí 神shén 仙xian 。霞xiá 映yìng 武wǔ 陵líng 桃táo 淡dàn 淡dàn ,烟yān 荒huāng 隋suí 堤dī 柳liǔ 绵mián 绵mián [28] 。七qī 碗wǎn 月yuè 团tuán ,啜chuò 罢bà 清qīng 风fēng 生shēng 腋yè 下xià ;三sān 杯bēi 云yún 液yè [29] ,饮yǐn 余yú 红hóng 雨yǔ 晕yùn 腮sāi 边biān 。

中zhōng 对duì 外wài ,后hòu 对duì 先xiān 。树shù 下xià 对duì 花huā 前qián 。玉yù 柱zhù 对duì 金jīn 屋wū [30] ,叠dié 嶂zhàng 对duì 平píng 川chuān [31] 。孙sūn 子zǐ 策cè [32] ,祖zǔ 生shēng 鞭biān [33] 。盛shèng 席xí 对duì 华huá 筵yán [34] 。解jiě 醉zuì 知zhī 茶chá 力lì ,消xiāo 愁chóu 识shí 酒jiǔ 权quán [35] 。丝sī 剪jiǎn 芰jì 荷hé 开kāi 冻dòng 沼zhǎo [36] ,锦jǐn 妆zhuāng 凫fú 雁yàn 泛fàn 温wēn 泉quán [37] 。帝dì 女nǚ 衔xián 石shí ,海hǎi 中zhōng 遗yí 魄pò 为wéi 精jīng 卫wèi [38] ;蜀shǔ 王wáng 叫jiào 月yuè ,枝zhī 上shàng 游yóu 魂hún 化huà 杜dù 鹃juān [39] 。



* * *



[1] 蹴踘:我国古代的一种足球运动。

[2] 覃烟:袅袅直升空中的饮烟或横浮低空的烟雾。覃,长。

[3] 宛转:声音委婉而动听。

[4] 婵娟:体态柔弱的样子。

[5] 云笺:唐韦陟(zhì)用五采笺写信,由他人代笔,自己签名。由于他写的“陟”字像五朵云,因而后来人们称书信为五云笺或云笺。

[6] 洗耳尚逢高士笑:传说帝尧时,箕山有高人隐士曰巢父、许由,尧同许由商量,准备把帝位传给他。巢父听到了,以为玷污了他的耳朵,就跑到池中去洗耳。池水主人怒曰:“何污我水!”这个故事说帝尧、许由、巢父、池水主人,一个比一个更高洁。

[7] 折腰肯受小儿怜:陶渊明为彭泽令。一次,郡督邮来视察。县吏向陶渊明建议,应穿上官服迎见。陶渊明气愤地说:“吾不能为五斗米折腰,拳拳事乡里小儿!”于是弃官而去。作《归去来兮辞》。

[8] 郭泰泛舟,折角半垂梅子雨:见真韵“郭泰”句注。

[9] 山涛骑马,接 倒着杏花天:见文韵“竹间”句注。

[10] 肥对坚:肥,肥马。坚,坚车。

[11] 碧玉:南朝宋汝南王妾,甚受宠爱,后代引为娇怜的爱人的代称。青钱:唐张鷟(zhuó)甚有才名,时人称之为“青钱学士”,意思是他的文章万选万中,万无一失。

[12] 郊寒对岛瘦:郊指孟郊,岛指贾岛,唐代的两个诗人。孟郊的诗内容清苦,失之寒,贾岛的诗风格瘦峭,失之瘦,后人于是有“郊寒岛瘦”的评价。

[13] 酒圣:晋刘伶旷达放饮,又曾作《酒德颂》,后人因称之为酒圣。

[14] 依玉树:唐崔宗之,美容仪,饮酒时更见风度。杜甫诗《饮中八仙歌》说:“宗之潇洒美少年,举觞白眼望青天,皎如玉树临风前。”

[15] 步金莲:南齐东昏侯宠爱潘妃,以金为莲花贴地,令潘妃行其上,叫“步步生莲花”。后以金莲指女子纤足。

[16] 凿井、耕田:传说尧帝游于康衢,有一老人击壤而歌曰:“日出而作,日入而息,凿井而饮,耕田而食,帝力于我何有哉!”

[17] 杜甫清宵立:杜甫诗有“思家步月清宵立”句。

[18] 边韶白昼眠:汉儒边韶,字孝先,性放达,开帐授徒,常昼眠,弟子编歌嘲之曰:“边孝先,腹便便。夜读书,昼贪眠。”

[19] 豪饮客吞波底月,酣游人醉水中天:杜甫《饮中八仙歌》有“左相日兴费万钱,饮如长鲸吸百川”;“知章骑马似乘船,眼花落井水底眠”等语,形容醉人们的情态。

[20] 斗草:也称“斗百草”。一种古代游戏。竞采花草,比赛多寡优劣,常于端午行之。青郊:指春天的郊野。

[21] 金勒:金饰的带嚼口的马络头。

[22] 紫陌:指京师郊野的道路。

[23] 翠钿:妇女用宝石金银雕饰的首饰,这里即代指妇女。钿,为压韵可读tián。

[24] 董杏:《神仙传》中载,三国东吴董奉为人治病不取报酬,病重的为他栽五棵杏,轻者栽一棵,数年后共得十万余株,郁然成林。周莲:宋儒周敦颐喜爱莲花,曾写《爱莲说》一篇,盛赞此花出污泥而不染的高洁品质。

[25] 春九十:春光九十,意思是春光将尽。

[26] 岁三千:极言年寿之长。传说汉武帝时,东郊献短人东方朔,谓帝曰:“王母蟠桃,三千岁一熟,此儿已三偷之矣。”

[27] 入山逢宰相:南朝梁陶宏景隐山中,武帝常问之以国事,时人称之“山中宰相”。

[28] 烟荒隋堤柳绵绵:隋炀帝自板渚引河达淮,岸上悉种柳。见齐韵“隋堤”注。

[29] 云液:酒的美称。

[30] 玉柱:石柱的美称。金屋:华美之屋。

[31] 叠嶂:重迭的山峰。平川:广阔平坦之地。

[32] 孙子策:孙子指春秋战国时吴国孙武,著名军事家,著有《孙子》十三篇传世。

[33] 祖生鞭:东晋祖逖与朋友刘琨同寝,他们立志收复中原,每天闻鸡鸣就起床舞剑。一次祖逖先醒,闻鸡鸣,逖蹴琨曰:“此非恶声也。”琨恐曰:“祖生先吾着鞭。”意思是比自己行动得快。

[34] 华筵:丰盛的筵席。

[35] 解醉知茶力,消愁识酒权:茶力、酒权互文,即茶和酒的功效。

[36] 丝剪芰荷开冻沼:传说中隋炀帝的故事,说他曾命人用锦绢剪为荷花,遍插池苑,从中游乐。芰,古书上指菱。

[37] 锦妆凫雁泛温泉:唐玄宗的故事。相传玄宗扩建华清宫汤池,规模宏丽,汤池内以玉莲为喷泉,又缝锦绣为凫雁,放于水中,自己乘小舟从中游嬉,极尽奢欲。

[38] 帝女衔石,海中遗魄为精卫:上古神话,赤帝有女名女娃,游于东海,溺而不返,魂魄变成一种鸟,名叫精卫,常常衔木石填海中。

[39] 蜀王叫月,枝上游魂化杜鹃:上古神话传说,蜀王名杜宇,在蜀治水,自以德薄,让位给大臣鳖冷,自己隐居山林,死后化为杜鹃鸟,夜夜悲啼,啼则吐血。





二èr 萧xiāo


琴qín 对duì 管guǎn ,斧fǔ 对duì 瓢piáo 。水shuǐ 怪guài 对duì 花huā 妖yāo 。秋qiū 声shēng 对duì 春chūn 色sè ,白bái 缣jiān 对duì 红hóng 绡xiāo [1] 。臣chén 五wǔ 代dài [2] ,事shì 三sān 朝cháo [3] 。斗dǒu 柄bǐng 对duì 弓gōng 腰yāo [4] 。醉zuì 客kè 歌gē 金jīn 缕lǚ [5] ,佳jiā 人rén 品pǐn 玉yù 箫xiāo 。风fēng 定dìng 落luò 花huā 闲xián 不bù 扫sǎo ,霜shuāng 余yú 残cán 叶yè 湿shī 难nán 烧shāo 。千qiān 载zǎi 兴xīng 周zhōu ,尚shàng 父fù 一yī 竿gān 投tóu 渭wèi 水shuǐ [6] ;百bǎi 年nián 霸bà 越yuè ,钱qián 王wáng 万wàn 弩nǔ 射shè 江jiāng 潮cháo [7] 。

荣róng 对duì 悴cuì [8] ,夕xī 对duì 朝zhāo 。露lù 地dì [9] 对duì 云yún 霄xiāo 。商shāng 彝yí 对duì 周zhōu 鼎dǐng [10] ,殷yīn 濩huò 对duì 虞yú 韶sháo [11] 。樊fán 素sù 口kǒu ,小xiǎo 蛮mán 腰yāo [12] 。六liù 诏zhào 对duì 三sān 苗miáo [13] 。朝cháo 天tiān 车chē 奕yì 奕yì [14] ,出chū 塞sài 马mǎ 萧xiāo 萧xiāo [15] 。公gōng 子zǐ 幽yōu 兰lán 重chóng 泛fàn 舸gě [16] ,王wáng 孙sūn 芳fāng 草cǎo 正zhèng 联lián 镳biāo [17] 。潘pān 岳yuè 高gāo 怀huái ,曾céng 向xiàng 秋qiū 天tiān 吟yín 蟋xī 蟀shuài [18] ;王wáng 维wéi 清qīng 兴xìng ,尝cháng 于yú 雪xuě 夜yè 画huà 芭bā 蕉jiāo [19] 。

耕gēng 对duì 读dú ,牧mù 对duì 樵qiáo 。琥hǔ 珀pò 对duì 琼qióng 瑶yáo [20] 。兔tù 毫háo 对duì 鸿hóng 爪zhǎo [21] ,桂guì 楫jí 对duì 兰lán 桡ráo [22] 。鱼yú 潜qián 藻zǎo ,鹿lù 藏cáng 蕉jiāo [23] 。水shuǐ 远yuǎn 对duì 山shān 遥yáo 。湘xiāng 灵líng 能néng 鼓gǔ 瑟sè [24] ,嬴yíng 女nǚ 解jiě 吹chuī 箫xiāo [25] 。雪xuě 点diǎn 寒hán 梅méi 横héng 小xiǎo 院yuàn ,风fēng 吹chuī 弱ruò 柳liǔ 覆fù 平píng 桥qiáo 。月yuè 牖yǒu 通tōng 宵xiāo ,绛jiàng 蜡là 罢bà 时shí 光guāng 不bù 减jiǎn [26] ;风fēng 帘lián 当dāng 昼zhòu ,雕diāo 盘pán 停tíng 后hòu 篆zhuàn 难nán 消xiāo [27] 。



* * *



[1] 缣:丝绢,这里指细绢。绡:生丝,又指用生丝织的东西,这里指绸子。

[2] 臣五代:指五代时冯道,他曾历事后唐、后晋、后辽、后汉、后周,对丧君亡国毫不介意,并自号“长乐老”。旧时代拿他做没气节的典型。

[3] 事三朝:沈约事南朝宋、齐、梁三朝。

[4] 斗柄:北斗七星中排成柄状的三星。弓腰:舞女反身将腰弯如弓形,叫做弓腰。

[5] 金缕:词牌《贺新郎》的别名,或说指唐女诗人杜秋娘所作《金缕衣》。

[6] 千载兴周,尚父一竿投渭水:西周初,吕望曾隐居在渭水垂钓,后被周文王聘请为太师,辅佐武王灭殷。被周武王尊为尚父。

[7] 百年霸越,钱王万弩射江潮:传说五代时钱 为吴越王,做御潮铁柱于江中,未成而潮水大至。吴越王命以万弩射之,潮水乃退。筑土一升者,赏钱一升,名之曰钱塘。

[8] 荣:茂。悴:枯。

[9] 露地:佛教语。喻三界(欲界、色界、无色界)的烦恼俱尽,处于没有覆蔽的地方。

[10] 商彝、周鼎:指商周二代的青铜器。

[11] 濩:传说是商汤王的舞乐。韶:传说帝舜时乐名。虞即指帝舜虞氏。

[12] 樊素口,小蛮腰:樊素、小蛮都是白居易的歌伎。白有“樱桃樊素口,杨柳小蛮腰”的诗句。

[13] 六诏:“诏”是唐代我国西南少数民族对王的称呼,时有蒙嶲(xī)、越析、浪穹、澄睒(shān)、施浪、蒙舍诸诏,合称六诏。其地在今云南及四川西南部。三苗:传说尧、舜时代居住在西南的我国少数民族。

[14] 朝天车:指大臣们登朝拜见皇帝所用车乘。奕奕:有次序的样子。

[15] 出塞马萧萧:杜甫《后出塞》诗有“马鸣风萧萧”之句。萧萧,马嘶声或风声。

[16] 公子幽兰重泛舸:屈原《九歌》:“沅有芷兮澧(lǐ)有兰,思公子兮未敢言。”舸,大船。泛舸即乘船游览。

[17] 王孙芳草正联镳:刘安《招隐士》:“王孙游兮不归,春草生兮萋萋。”镳,马辔头。联镳,意思是并马而行。

[18] 潘岳高怀,曾向秋天吟蟋蟀:潘岳是晋诗人,曾写有《蟋蟀赋》。

[19] 王维清兴,尝于雪夜画芭蕉:唐王维诗、画、书都有很高造诣。据说他的山水画随意写来,不分四时,曾画雪中芭蕉。

[20] 琼瑶:美玉。

[21] 兔毫:笔名,这里指毛笔。鸿爪:指鸿雁在泥土上留下的脚印,比喻人生的阅历。

[22] 桂楫、兰桡:楫和桡都是划船撑船的工具。桂是桂树,兰指木兰。用桂和木兰制成的楫和桡,言其贵重华美。

[23] 鹿藏蕉:《列子·周穆王》:郑人有薪者,遇鹿而毙之,藏诸泥中,覆之以蕉,俄而失其处,遂以为梦,顺途而道其事。傍闻者取之,归告室人曰:薪者梦得鹿,不知其处,我今得之,彼真在梦中矣。

[24] 湘灵能鼓瑟:湘灵,尧女娥皇女英,哭舜于苍梧之野,死之为湘江之神。

[25] 嬴女解吹箫:即弄玉的故事。秦王族姓嬴,故称弄玉为嬴女。见江韵“跨凤”句注。

[26] 月牖通宵,绛蜡罢时光不减:由于月光透窗而入,即使灭掉红烛,室内仍很明亮。绛蜡:即红烛。

[27] 风帘当昼,雕盘停后篆难消:篆,指袅袅上升的香烟好像篆字一样。二句意思是,因为风帘遮掩门户,尽管雕盘中的薰香不再点燃,室内的香气也很难消失。





三sān 肴yáo


《诗shī 》对duì 《礼lǐ 》,卦guà 对duì 爻yáo 。燕yàn 引yǐn 对duì 莺yīng 调tiáo [1] 。晨chén 钟zhōng 对duì 暮mù 鼓gǔ [2] ,野yě 馔zhuàn 对duì 山shān 肴yáo [3] 。雉zhì 方fāng 乳rǔ [4] ,鹊què 始shǐ 巢cháo [5] 。猛měng 虎hǔ 对duì 神shén 獒áo [6] 。疏shū 星xīng 浮fú 荇xìng 叶yè ,皓hào 月yuè 上shàng 松sōng 梢shāo 。为wéi 邦bāng 自zì 古gǔ 推tuī 瑚hú 琏liǎn [7] ,从cóng 政zhèng 于yú 今jīn 愧kuì 斗dǒu 筲shāo [8] 。管guǎn 鲍bào 相xiāng 知zhī ,能néng 交jiāo 忘wàng 形xíng 胶jiāo 漆qī 友yǒu [9] ;蔺lìn 廉lián 有yǒu 隙xì ,终zhōng 为wéi 刎wěn 颈jǐng [10] 死sǐ 生shēng 交jiāo 。

歌gē 对duì 舞wǔ ,笑xiào 对duì 嘲cháo 。耳ěr 语yǔ 对duì 神shén 交jiāo [11] 。焉yān 鸟niǎo 对duì 亥hài 豕shǐ [12] ,獭tǎ 髓suǐ 对duì 鸾luán 胶jiāo [13] 。宜yí 久jiǔ 敬jìng ,莫mò 轻qīng 抛pāo 。一yī 气qì [14] 对duì 同tóng 胞bāo 。祭zhài 遵zūn 甘gān 布bù 被bèi [15] ,张zhāng 禄lù 念niàn 绨tí 袍páo [16] 。花huā 径jìng 风fēng 来lái 逢féng 客kè 访fǎng [17] ,柴chái 扉fēi 月yuè 到dào 有yǒu 僧sēng 敲qiāo 。夜yè 雨yǔ 园yuán 中zhōng ,一yī 颗kē 不bù 雕diāo 王wáng 子zǐ 柰nài [18] ;秋qiū 风fēng 江jiāng 上shàng ,三sān 重chóng 曾céng 卷juǎn 杜dù 公gōng 茅máo [19] 。

衙yá 对duì 舍shè [20] ,廪lǐn 对duì 庖páo [21] 。玉yù 磬qìng 对duì 金jīn 铙náo [22] 。竹zhú 林lín 对duì 梅méi 岭lǐng [23] ,起qǐ 凤fèng 对duì 腾téng 蛟jiāo [24] 。鲛jiāo 绡xiāo [25] 帐zhàng ,兽shòu 锦jǐn [26] 袍páo 。露lù 果guǒ 对duì 风fēng 梢shāo 。扬yáng 州zhōu 输shū 橘jú 柚yòu ,荆jīng 土tǔ 贡gòng 菁jīng 茅máo [27] 。断duàn 蛇shé 埋mái 地dì 称chēng 孙sūn 叔shū [28] ,渡dù 蚁yǐ 作zuò 桥qiáo 识shí 宋sòng 郊jiāo [29] 。好hǎo 梦mèng 难nán 成chéng ,蛩qióng 响xiǎng 阶jiē 前qián 偏piān 唧jī 唧jī ;良liáng 朋péng 远yuǎn 到dào ,鸡jī 声shēng 窗chuāng 外wài 正zhèng 嘐jiāo 嘐jiāo 。



* * *



[1] 燕引对莺调:引和调都是歌曲,这里指燕和莺动听的鸣声。

[2] 晨钟对暮鼓:见上卷冬韵“暮鼓”句注。

[3] 野馔、山肴:馔、肴是饭菜的统称。野馔、山肴指淡素的饭食。

[4] 雉方乳:汉鲁恭为中军令,很有政绩,蝗不入境。河南尹闻之,使人往看。见野鸡伏于桑下,儿童不捕,惊问,儿童说:“野鸡在孵卵,不要伤害它。”雉,野鸡。

[5] 鹊始巢:语出《礼记·月令》:“雁北乡,鹊始巢,雉雊,鸡乳。”

[6] 神獒:传说能听懂人语的犬叫獒。

[7] 为邦自古推瑚琏:《论语》载,一次孔子弟子子贡问老师:“我是怎样一个人?”孔子说:“你是能成器的。”又问:“我是怎样的器?”孔子说:“你是瑚琏。”瑚琏,古代宗庙盛黍稷的器皿,是祭祀的贵重礼器,比喻子贡会成为治国的人材。为邦,治理国家。

[8] 从政于今愧斗筲:《论语》载,一次子贡问,当今做官的人怎么样,孔子说:“噫,斗筲之人,何足算也!”斗筲之人,即德薄才疏的人。

[9] 管鲍相知,能交忘形胶漆友:春秋时,管仲和鲍叔牙交情非常好,患难与共,旧时代常以管鲍为朋友间的楷模。管仲,春秋初年政治家。经鲍叔牙推荐,被齐桓公任为上卿。相知,即相友好。胶漆,形容难解难分,关系极为密切。

[10] 刎颈:指发誓同死的交情。

[11] 耳语:凑近耳朵小声说话。神交:彼此慕名而没有见过面的交谊。

[12] 焉鸟对亥豕:古文之讹。焉和鸟,亥和豕,字形相近,往往造成讹误。焉鸟:谓字形相近而易讹。

[13] 獭:水獭,旧传水獭的髓是很好的滋补品,服食能益神智;相传水獭的骨髓与玉屑、琥珀屑相和,可以灭瘢痕。鸾胶:传说海上有凤麟洲,多仙人,以凤喙麟角合煎作膏,名续弦胶,能续弓弩断弦。

[14] 一气:犹云同气,指有血缘关系的亲属,多喻兄弟。

[15] 祭遵甘布被:祭遵是东汉光武帝的将军。《后汉书·祭遵传》:遵为人克己奉公,凡皇帝的赏赐一律分给士卒,家无私财,穿皮裤,盖布被,夫人裳不加缘,因而受到皇帝的敬重。

[16] 张禄念绨袍:战国时,范睢和须贾同事魏王,须贾出于嫉妒,唆使魏相治范睢几至于死。后范睢逃到秦国,改名张禄,为秦相。后须贾使秦,范睢故意穿了一身破衣服去见须贾。贾不知其为秦相,说“范叔何一寒至此”,以己绨袍赠之。不久,须贾终于知道范睢原来就是秦相张禄,吓得赶忙登门请罪。范睢说:“根据你旧日对我的态度,本当把你处死。但你送我一件袍子,看来还有点情谊,可以饶你一命。”绨,光滑厚实的丝织品。

[17] 花径风来逢客访:语出杜甫《客至》:“花径不曾缘客扫,蓬门今始为君开。”

[18] 夜雨园中,一颗不雕王子柰:《二十四孝》载:晋人王祥至孝,后母不慈,命其看护后园柰树,柰落则鞭之。祥抱树大哭,感动上天,柰一颗不落。柰,落叶小乔木,花白色,果小,是苹果的一种。

[19] 秋风江上,三重曾卷杜公茅:杜公指杜甫。杜甫居成都时,一次大风吹坏了草堂,他曾为此写作了《茅屋为秋风所破歌》,中有“八月秋高风怒号,卷我屋上三重茅”之句。

[20] 衙:旧时官舍之称。舍:居住的房子。

[21] 廪:粮仓。庖:厨房。

[22] 玉磬:古代的一种用玉或石制成的打击乐器。金铙:一种用金属制成的打击乐器。

[23] 竹林:晋时嵇康与阮籍等七人为友,蔑视礼教,狂放不羁,经常聚在竹林中啸饮清谈,时人号为“竹林七贤”。梅岭:英州司寇种梅三十株于大庾岭,故庾岭多梅。

[24] 起凤、腾蛟:都是形容文采的超拔。

[25] 鲛绡:古代神话,南海外有鲛人,住在水中,善织绩,常出卖绡,眼能泣泪成珠。鲛绡,鲛人所织的细绢。鲛,就是鲨鱼。绡,生丝,又指用生丝织的东西。

[26] 兽锦:绣有麟、豹一类野兽花纹的锦缎。

[27] 扬州输橘柚,荆土贡菁茅:《尚书》有《禹贡》篇,记述九州的山川土宜,提出扬州要贡赋桔柚,荆州要贡献菁茅。菁茅,一种草类,古人用以扎神像,灌酒其上,表示神饮,叫祼。

[28] 断蛇埋地称孙叔:孙叔敖,战国时楚国令尹,幼时见两个头的蛇,杀而埋之,回家后对母亲哭诉。母问其故,他说:“人们说遇到两头蛇的人一定会死,今天我遇到了。为了不至于让更多的人见而致死,我已杀死并且埋掉了它。”母亲说:“我儿做了好事,必有善报。”后来孙叔敖果然做了楚国的令尹。

[29] 渡蚁作桥识宋郊:迷信传说,宋郊为士人时,所居堂前有蚁穴为雨水冲毁,他编竹为桥让蚂蚁爬到了干处,据说因为有此阴德,后为状元。





四sì 豪háo


茭jiāo 对duì 茨cí ,荻dí 对duì 蒿hāo





[1] 。山shān 麓lù 对duì 江jiāng 皋gāo [2] 。莺yīng 簧huáng 对duì 蝶dié 板bǎn [3] ,麦mài 浪làng 对duì 桃táo 涛tāo [4] 。骐qí 骥jì 足zú ,凤fèng 凰huáng 毛máo [5] 。美měi 誉yù 对duì 嘉jiā 褒bāo 。文wén 人rén 窥kuī 蠹dù 简jiǎn [6] ,学xué 士shì 书shū 兔tù 毫háo [7] 。马mǎ 援yuán 南nán 征zhēng 载zài 薏yì 苡yǐ [8] ,张zhāng 骞qiān 西xī 使shǐ 进jìn 葡pú 萄táo [9] 。辩biàn 口kǒu 悬xuán 河hé ,万wàn 语yǔ 千qiān 言yán 常cháng 亹wěi 亹wěi ;词cí 源yuán 倒dào 峡xiá ,连lián 篇piān 累lěi 牍dú 自zì 滔tāo 滔tāo [10] 。

梅méi 对duì 杏xìng ,李lǐ 对duì 桃táo 。棫yù 朴pò 对duì 旌jīng 旄máo [11] 。酒jiǔ 仙xiān 对duì 诗shī 史shǐ [12] ,德dé 泽zé 对duì 恩ēn 膏gāo [13] 。悬xuán 一yī 榻tà [14] ,梦mèng 三sān 刀dāo [15] 。拙zhuō 逸yì 对duì 贵guì 劳láo 。玉yù 堂táng 花huā 烛zhú 绕rào ,金jīn 殿diàn 月yuè 轮lún 高gāo [16] 。孤gū 山shān 看kàn 鹤hè 盘pán 云yún 下xià [17] ,蜀shǔ 道dào 闻wén 猿yuán 向xiàng 月yuè 号háo [18] 。万wàn 事shì 从cóng 人rén ,有yǒu 花huā 有yǒu 酒jiǔ 应yīng 自zì 乐lè ;百bǎi 年nián 皆jiē 客kè ,一yī 丘qiū 一yī 壑hè 尽jìn 吾wú 豪háo [19] 。

台tái 对duì 省shěng ,署shǔ 对duì 曹cáo [20] 。分fēn 袂mèi 对duì 同tóng 袍páo [21] 。鸣míng 琴qín 对duì 击jī 剑jiàn ,返fǎn 辙zhé 对duì 回huí 艚cáo [22] 。良liáng 借jiè 箸zhù [23] ,操cāo 提tí 刀dāo [24] 。香xiāng 茶chá 对duì 醇chún 醪láo [25] 。滴dī 泉quán 归guī 海hǎi 大dà ,篑kuì 土tǔ 积jī 山shān 高gāo [26] 。石shí 室shì 客kè 来lái 煎jiān 雀què 舌shé [27] ,画huà 堂táng 宾bīn 至zhì 饮yǐn 羊yáng 羔gāo [28] 。被bèi 谪zhé 贾jiǎ 生shēng ,湘xiāng 水shuǐ 凄qī 凉liáng 吟yín 《 fú 鸟niǎo 》 [29] ;遭zāo 谗chán 屈qū 子zǐ ,江jiāng 潭tán 憔qiáo 悴cuì 著zhù 《离lí 骚sāo 》 [30] 。



* * *



[1] 茭、茨、荻、蒿:都是指蒿草。

[2] 山麓:山脚下。江皋:江边的高地。

[3] 莺簧:黄莺啼叫的声音美如笙簧。蝶板:蝴蝶的双翅忽开忽合好象乐器中的板。

[4] 麦浪:风吹麦田,麦子像波浪般起伏的样子。桃涛:春二三月,桃花盛开之时,河中春汛,称为桃花汛。

[5] 骐骥:良马。骐骥足,比喻人有才干。凤凰毛:凤毛麟角,喻稀有的优秀人才。

[6] 蠹简:指书籍。蠹,蛀书虫。

[7] 兔毫:用兔毛制成的笔。泛指毛笔。

[8] 马援南征载薏苡:马援是东汉的将军,他南征交趾时,曾携带数车薏苡,以防治瘴疠。薏苡,多年生草本植物,即中药苡仁。

[9] 张骞西使进葡萄:汉武帝时,张骞曾两次出使西域,使汉族和少数民族、中国和外国的文化得以交流。传说中原地区的葡萄是他由西域带回来的,留种中国。

[10] 辩口悬河,万语千言常亹亹;词源倒峡,连篇累牍自滔滔:都形容人善于谈吐。亹亹,原意是勤奋的样子,这里是言不绝口的意思。词源倒峡:谓诗文雄健有力,气势豪迈。

[11] 棫朴对旌旄:棫朴,两种灌木名,据说可点燃祭天神。《诗经·大雅》中有《棫朴》篇。棫,白桵。朴,桴木。意谓棫朴丛生,根枝茂密,共同附着。喻贤人众多,国家蕃兴。旌旄,指旗帜。

[12] 酒仙对诗史:杜甫有《饮中八仙歌》,称李白、贺知章、李琎、张旭等八人为酒仙。诗史:杜甫的许多诗,较为真实地记述了当时的社会状况,被人称为“诗史”。

[13] 德泽对恩膏:泽和膏都是指及时的好雨,因而被比作恩德。

[14] 悬一榻:后汉徐稚,字孺子,家贫,有德行,当时陈蕃为豫章太守,不接待宾客,只特设一榻待徐稚,徐来则放下,徐走后即悬起。

[15] 梦三刀:迷信传说,晋王浚夜梦梁上悬三把刀,后又增加一把,醒来问别人是何吉凶。解者曰:三刀是州字,又加一把是“益”的意思,是益州,所以您要做益州刺史了。后果守益州。

[16] 金殿:金饰的殿堂,指帝王的宫殿。月轮:指月亮。

[17] 孤山看鹤盘云下:宋林逋,隐西湖孤山,常养两鹤,纵之则飞入云霄,盘旋久之乃下。

[18] 蜀道闻猿向月号:古代四川多猿,所以民歌有“巴东三峡巫峡长,猿啼三声泪沾裳”的说法。

[19] 百年皆客,一丘一壑尽吾豪:这是一种消极的人生观,认为人生百年不过如客人一样暂住世间,应放浪山水之间,尽其豪情。

[20] 台、省、署、曹:都是古时官府的名称。

[21] 分袂:古时把离别称作分袂。袂,袖子。同袍:最早出自《诗经·秦风·无衣》:“岂曰无衣?与子同袍。王于兴师,修我戈矛,与子同仇。”后来多为军人用以互称。后亦用来泛指朋友、同僚、同学等。

[22] 返辙对回艚:返辙即回车。晋阮籍由于当时政治昏暗,心情苦闷,常酒醉后乘车出游,遇到绝路就痛哭而回。回艚:艚,就是船。晋王献之曾在雪夜乘船去访问他的老朋友戴逵,走到半路,忽然命令船只返回。人们问什么缘故,他说自己是“乘兴而来,兴尽而返”。

[23] 良借箸:楚汉战争中,汉高祖听信郦生的话,准备把诸将分封于各地为侯王。张良认为这是错误的,就在酒宴前,借席上箸一一陈说道理。箸,筷子。

[24] 操提刀:传说匈奴使者要拜谒曹操,曹操自以为相貌不扬,恐为耻笑,于是让崔琰装扮成魏王,曹操自己装扮成卫士,提刀立旁。朝见后,让人问使者对魏王的印象。使者曰,魏王相貌亦复平常,但床头捉刀人(指曹操)乃真英雄。

[25] 醇醪:味厚的美酒。

[26] 滴泉归海大,篑土积山高:都是说积少成多的意思。篑,古代盛土的筐子。

[27] 雀舌:一种名茶。

[28] 羊羔:美酒名。

[29] 被谪贾生,湘水凄凉吟 鸟:汉贾谊被黜为长沙王太傅,内心悲苦,一日有猫头鹰进宅,人皆以为不祥,他就写了一篇《 鸟赋》抒发情怀。 ,一种猫头鹰类的鸟。

[30] 遭谗屈子,江潭憔悴著《离骚》:战国时期楚国大夫、爱国诗人屈原,由于佞臣毁谤,遭到楚王贬谪,曾在湘江一带流浪,《史记·屈原贾生列传》:“披发行吟泽畔,颜色憔悴,形容枯槁。”后投汩罗江而死。《离骚》是他写作的长诗。





五wǔ 歌gē


微wēi 对duì 巨jù ,少shǎo 对duì 多duō 。直zhí 干gàn 对duì 平píng 柯kē [1] 。蜂fēng 媒méi 对duì 蝶dié 使shǐ [2] ,雨yǔ 笠lì 对duì 烟yān 蓑suō [3] 。眉méi 淡dàn 扫sǎo [4] ,面miàn 微wēi 酡tuó 。妙miào 舞wǔ 对duì 清qīng 歌gē 。轻qīng 衫shān 裁cái 夏xià 葛gé [5] ,薄bó 袂mèi 剪jiǎn 春chūn 罗luó [6] 。将jiàng 相xiàng 兼jiān 行xíng 唐táng 李lǐ 靖jìng [7] ,霸bà 王wáng 杂zá 用yòng 汉hàn 萧xiāo 何hé [8] 。月yuè 本běn 阴yīn 精jīng ,岂qǐ 有yǒu 羿yì 妻qī 曾céng 窃qiè 药yào [9] ;星xīng 为wéi 夜yè 宿xiù ,浪làng 传chuán 织zhī 女nǚ 漫màn 投tóu 梭suō [10] 。

慈cí 对duì 善shàn ,虐nüè 对duì 苛kē 。缥piāo 缈miǎo 对duì 婆pó 娑suō [11] 。长cháng 杨yáng 对duì 细xì 柳liǔ [12] ,嫩nèn 蕊ruǐ 对duì 寒hán 莎suō [13] 。追zhuī 风fēng 马mǎ [14] ,挽wǎn 日rì 戈gē [15] 。玉yù 液yè 对duì 金jīn 波bō [16] 。紫zǐ 诏zhào 衔xián 丹dān 凤fèng [17] ,黄huáng 庭tíng 换huàn 白bái 鹅é [18] 。画huà 阁gé 江jiāng 城chéng 梅méi 作zuò 调diào [19] ,兰lán 舟zhōu 野yě 渡dù 竹zhú 为wéi 歌gē [20] 。门mén 外wài 雪xuě 飞fēi ,错cuò 认rèn 空kōng 中zhōng 飘piāo 柳liǔ 絮xù [21] ;岩yán 边biān 瀑pù 响xiǎng ,误wù 疑yí 天tiān 半bàn 落luò 银yín 河hé [22] 。

松sōng 对duì 竹zhú ,荇xìng [23] 对duì 荷hé 。薜bì 荔lì [24] 对duì 藤téng 萝luó 。梯tī 云yún 对duì 步bù 月yuè [25] ,樵qiáo 唱chàng 对duì 渔yú 歌gē 。升shēng 鼎dǐng 雉zhì [26] ,听tīng 经jīng 鹅é [27] 。北běi 海hǎi 对duì 东dōng 坡pō [28] 。吴wú 郎láng 哀āi 废fèi 宅zhái [29] ,邵shào 子zǐ 乐lè 行xíng 窝wō [30] 。丽lí 水shuǐ 良liáng 金jīn 皆jiē 待dài 冶yě ,昆kūn 山shān 美měi 玉yù 总zǒng 须xū 磨mó [31] 。雨yǔ 过guò 皇huáng 州zhōu ,琉liú 璃lí 色sè 灿càn 华huá 清qīng 瓦wǎ ;风fēng 来lái 帝dì 苑yuàn ,荷hé 芰jì 香xiāng 飘piāo 太tài 液yè 波bō [32] 。

笼lóng 对duì 槛jiàn ,巢cháo 对duì 窝wō 。及jí 第dì 对duì 登dēng 科kē [33] 。冰bīng 清qīng 对duì 玉yù 润rùn [34] ,地dì 利lì 对duì 人rén 和hé [35] 。韩hán 擒qín 虎hǔ [36] ,荣róng 驾jià 鹅é [37] 。青qīng 女nǚ 对duì 素sù 娥é [38] 。破pò 头tóu 朱zhū 泚cǐ 笏hù [39] ,折shé 齿chǐ 谢xiè 鲲kūn 梭suō [40] 。留liú 客kè 酒jiǔ 杯bēi 应yīng 恨hèn 少shǎo ,动dòng 人rén 诗shī 句jù 不bù 须xū 多duō 。绿lǜ 野yě 凝níng 烟yān ,但dàn 听tīng 村cūn 前qián 双shuāng 牧mù 笛dí ;沧cāng 江jiāng 积jī 雪xuě ,惟wéi 看kàn 滩tān 上shàng 一yī 渔yú 蓑suō [41] 。



* * *



[1] 直干、平柯:挺直的树干。柯,树枝。平柯犹言横枝。

[2] 蜂媒:比喻为男女双方居间撮合或传递消息的人。蝶使:比喻男女双方情爱的媒介。

[3] 雨笠:遮雨的笠帽。烟蓑:蓑衣。

[4] 扫:描画。

[5] 夏葛:指夏天穿的葛衣。

[6] 春罗:适于春季穿的绫罗。

[7] 将相兼行唐李靖:李靖,唐初著名军事家。他曾在建立唐王朝的斗争中屡立战功,后又平突厥之叛,三定朔方,被封为卫国公。将相兼行是说他才兼文武。

[8] 霸王杂用汉萧何:楚汉战争中,萧何辅佐汉高祖定三秦,后为汉相,制作律令,对汉王朝的建立和巩固卓有贡献。霸王杂用,是说“王道”和“霸道”两用。儒家称以力假仁者为霸,以德行仁政者为王。

[9] 月本阴精,岂有羿妻曾窃药:古代神话传说,有穷国君后羿从西王母那里得到了长生药,其妻嫦娥窃之服用后飞升到月宫。本联意为,月本是阴气的精华,哪里有嫦娥飞升的事呢?

[10] 星为夜宿,浪传织女漫投梭:古代神话说,织女是天帝的孙女,整夜在那里织布。本联意为,世传牛郎织女隔天以梭相投。这种说法也是荒诞虚无的事。夜宿,夜间的星宿。浪传,胡传,乱传。

[11] 婆娑:树木或人的身躯摇曳多姿的样子。

[12] 长杨:汉宫殿名。细柳:周亚夫曾屯军细柳。

[13] 寒莎:秋天的莎草。

[14] 追风马:《淮南子》中有“以兔之走,使犬如马则逮日归(追)风”的说法,后常以追风形容马跑得快。

[15] 挽日戈:古代神话传说,楚国的鲁阳公与韩国人作战,战到天晚未分胜负,他举起戈来向太阳下令,太阳从西方退了回来,他继续战斗。

[16] 玉液:古人服食的用玉屑调成的药酒。金波:太阳照在水面或宫殿上反射回来的光线。

[17] 紫诏衔丹凤:《晋书·石季龙载记》说,当时诏书以五色纸衔木凤之口,后世遂称皇帝诏令为凤诏。又解衔丹凤:古人书信用泥封,泥上盖印,皇帝诏书则用紫泥,称为紫泥诏或紫诏,常以龙凤为图饰。

[18] 黄庭换白鹅:晋书法家王羲之喜欢山阴道士养的鹅,于是为道士写了一卷《黄庭经》做为交换条件。

[19] 画阁江城梅作调:这是对李白“黄鹤楼中吹玉笛,江城五月落梅花”两句诗的概括。梅作调,古代笛曲名有《梅花落》。

[20] 竹为歌:此指歌咏民俗风土人情的《竹枝词》。

[21] 门外雪飞,错认空中飘柳絮:晋才女谢道韫,有才辩,一次降雪,他的叔父谢安问子侄们:“大雪纷纭何所似?”谢朗说:“撒盐空中差可拟。”谢道韫说:“未若柳絮因风起。”谢安十分赞赏。

[22] 岩边瀑响,误疑天半落银河:语出李白《观庐山瀑布》:“飞流直下三千尺,疑是银河落九天。”

[23] 荇:多年生草本植物。

[24] 薜荔:南方的一种蔓生植物。

[25] 梯云:犹言登云。步月:在月光下散步。

[26] 升鼎雉:传说殷王武丁时祭祀太庙,有野鸡飞落鼎耳上而鸣,古人认为是一种祥瑞。

[27] 听经鹅:佛教传说,僧志违养鹅能听经说法。

[28] 北海:后汉孔融曾为北海太守,时人称之为北海,好宴客。他是当时著名的文人。东坡:宋代诗人苏轼,在黄冈东坡筑室,号东坡居士。

[29] 吴郎哀废宅:吴郎指唐代吴融,他曾写有《废宅》诗:“风飘碧瓦雨摧垣,却有邻人与锁门。”

[30] 邵子乐行窝:宋经学家邵雍隐居不仕,居洛阳三十年,筑“安乐窝”以居,自称安乐先生。

[31] 丽水良金皆待冶,昆山美玉总须磨:旧传金生丽水,玉出昆仑。

[32] 雨过皇州,琉璃色灿华清瓦;风来帝苑,荷芰香飘太液波:描写风雨中帝都景象。太液,即太液池,西汉时在长安掘成的人造湖。华清,即华清宫,在金陵,六朝陈时所建。

[33] 及第:指科举考试考中,特指考中进士,明清两代只用于殿试前三名。登科:科举时代应考人被录取。

[34] 冰清、玉润:晋乐广、卫玠翁婿俱有名,时人称乐广为冰清,其婿卫玠为玉润,喻人品高洁。

[35] 地利、人和:语出《孟子·公孙丑下》:天时不如地利,地利不如人和。

[36] 韩擒虎:隋朝大将,屡立战功,渡江平陈战役就是由他统帅的。

[37] 荣驾鹅:春秋时鲁昭公之大臣。

[38] 青女:传说中的霜神。素娥:即嫦娥,月色白,故又称素娥。李商隐诗:“青女素娥俱耐冷,月中霜里斗婵娟。”

[39] 破头朱泚笏:唐德宗时,京师兵变,德宗出逃,太尉朱泚欲窃位,司农卿段秀实执象笏击破其头,卒遭所害。笏,古代大臣登朝所持用以记事的手板。

[40] 折齿谢鲲梭:《晋书·谢鲲传》:“邻家高氏女有美色,鲲尝挑之,女投梭,折其两齿。”

[41] 沧江积雪,惟看滩上一渔蓑:唐柳宗元《江雪》诗:“孤舟蓑笠翁,独钓寒江雪。”





六liù 麻má


清qīng 对duì 浊zhuó ,美měi 对duì 嘉jiā 。鄙bǐ 吝lìn 对duì 矜jīn 夸kuā [1] 。花huā 须xū 对duì 柳liǔ 眼yǎn [2] ,屋wū 角jiǎo 对duì 檐yán 牙yá [3] 。志zhì 和hé 宅zhái [4] ,博bó 望wàng 槎chá [5] 。秋qiū 实shí 对duì 春chūn 华huā [6] 。乾qián 炉lú 烹pēng 白bái 雪xuě ,坤kūn 鼎dǐng 炼liàn 丹dān 砂shā [7] 。深shēn 宵xiāo 望wàng 冷lěng 沙shā 场chǎng 月yuè ,边biān 塞sài 听tīng 残cán 野yě 戍shù 笳jiā 。满mǎn 院yuàn 松sōng 风fēng ,钟zhōng 声shēng 隐yǐn 隐yǐn 为wéi 僧sēng 舍shè [8] ;半bàn 窗chuāng 花huā 月yuè ,锡xī 影yǐng 依yī 依yī 是shì 道dào 家jiā 。

雷léi 对duì 电diàn ,雾wù 对duì 霞xiá 。蚁yǐ 阵zhèn 对duì 蜂fēng 衙yá [9] 。寄jì 梅méi 对duì 怀huái 橘jú ,酿niàng 酒jiǔ 对duì 烹pēng 茶chá 。宜yí 男nán 草cǎo [10] ,益yì 母mǔ 花huā [11] 。杨yáng 柳liǔ 对duì 蒹jiān 葭jiā [12] 。班bān 姬jī 辞cí 帝dì 辇niǎn [13] ,蔡cài 琰yǎn 泣qì 胡hú 笳jiā [14] 。舞wǔ 榭xiè 歌gē 楼lóu 千qiān 万wàn 尺chǐ ,竹zhú 篱lí 茅máo 舍shè 两liǎng 三sān 家jiā [15] 。珊shān 枕zhěn [16] 半bàn 床chuáng ,月yuè 明míng 时shí 梦mèng 飞fēi 塞sài 外wài ;银yín 筝zhēng 一yī 奏zòu ,花huā 落luò 处chù 人rén 在zài 天tiān 涯yá 。

圆yuán 对duì 缺quē ,正zhèng 对duì 斜xié 。笑xiào 语yǔ 对duì 咨zī 嗟jiē [17] 。沈shěn 腰yāo 对duì 潘pān 鬓bìn [18] ,孟mèng 笋sǔn 对duì 卢lú 茶chá [19] 。百bǎi 舌shé 鸟niǎo [20] ,两liǎng 头tóu 蛇shé [21] 。帝dì 里lǐ [22] 对duì 仙xiān 家jiā 。尧yáo 仁rén 敷fū 率shuài 土tǔ ,舜shùn 德dé 被bèi 流liú 沙shā [23] 。桥qiáo 上shàng 授shòu 书shū 曾céng 纳nà 履lǚ [24] ,壁bì 间jiān 题tí 句jù 已yǐ 笼lǒng 纱shā [25] 。远yuǎn 塞sài 迢tiáo 迢tiáo ,露lù 碛qì [26] 风fēng 沙shā 何hé 可kě 极jí ;长cháng 沙shā 渺miǎo 渺miǎo ,雪xuě 涛tāo 烟yān 浪làng 信xìn 无wú 涯yá 。

疏shū 对duì 密mì ,朴pǔ 对duì 华huá 。义yì 鹘gǔ 对duì 慈cí 鸦yā [27] 。鹤hè 群qún 对duì 雁yàn 阵zhèn ,白bái 苎zhù 对duì 黄huáng 麻má [28] 。读dú 三sān 到dào [29] ,吟yín 八bā 叉chā [30] 。肃sù 静jìng 对duì 喧xuān 哗huá 。围wéi 棋qí 兼jiān 把bǎ 钓diào ,沉chén 李lǐ 并bìng 浮fú 瓜guā [31] 。羽yǔ 客kè 片piàn 时shí 能néng 煮zhǔ 石shí [32] ,狐hú 禅chán 千qiān 劫jié 似sì 蒸zhēng 沙shā [33] 。党dǎng 尉wèi 粗cū 豪háo ,金jīn 帐zhàng 笼lǒng 香xiāng 斟zhēn 美měi 酒jiǔ ;陶táo 生shēng 清qīng 逸yì ,银yín 铛chēng 融róng 雪xuě 啜chuò 团tuán 茶chá [34] 。



* * *



[1] 鄙吝:形容心胸狭窄。矜夸:夸耀。

[2] 花须:花蕊伸展如须。柳眼:柳叶如眉眼。

[3] 檐牙:檐际翘出如牙的部分。

[4] 志和宅:唐诗人张志和,肃宗朝命待诏翰林,授左金吾卫录事参军,后遭贬黜,遂不复仕。浪迹江湖,言以太虚(天)为庐,明月为伴,自号烟波钓徒。

[5] 博望槎:博望,即张骞,因奉使西域有功封博望侯。《荆楚岁时记》:“汉武帝令张骞使大夏,寻河源。乘槎经月,而至一处,见城郭和州府,室内有一女织,又见一丈夫牵牛饮河。骞问曰:‘此是何处?’答曰:‘可问严君平。’织女取榰机石与骞而还。”始知已到牛郎、织女星。槎,木筏。

[6] 秋实、春华:即春华秋实,古人比喻文采与德行。

[7] 乾炉烹白雪,坤鼎炼丹砂:都是道教说法。乾炉指男,坤鼎指女。

[8] 锡影依依是道家:僧人所持杖称锡。依依,隐隐约约的样子。

[9] 蚁阵:蚂蚁排阵而战。引申为争强斗胜。蜂衙:蜂早晚定时的聚集,如下属参谒长官于衙中,故称为蜂衙。

[10] 宜男草:即萱草,古人以为孕妇佩之可生男。

[11] 益母花:中药名。

[12] 蒹葭:即芦苇。

[13] 班姬辞帝辇:汉成帝游后苑,命班婕妤同辇,班婕妤说:“古代圣贤之君,都有名臣在旁;只有末代皇帝才亲近女色。”成帝听了很钦佩。

[14] 蔡琰泣胡笳:蔡琰,即蔡文姬,蔡邕女,汉末著名才女,早寡,汉末被虏入胡,在南匈奴生活了十二年,后被曹操赎回。传说她曾写了《胡笳十八拍》,历述她的不幸遭遇。

[15] 竹篱:用竹编的篱笆。茅舍:茅屋,草屋。

[16] 珊枕:即珊瑚枕。

[17] 咨嗟:文言叹词,叹息。

[18] 沈腰:南朝梁文学家沈约,字休文,体弱多病,腰肢纤弱。潘鬓:晋文学家潘岳,由于屡遭不幸,身体早衰,在《秋兴赋》中,他曾自伤两鬓早白,说自己三十二岁“始见二毛”。

[19] 孟笋:孟宗母病中喜吃笋,因时节正值冬季,无笋可取,宗入竹林悲泣哀叹,笋竟为之而生。后人遂用来形容人子事亲尽孝,至诚感天,并将之列入“二十四孝”中。卢茶:唐代诗人卢仝好茶成癖,诗风浪漫。

[20] 百舌鸟:鸟名,又名乌鸫(dōng)。益鸟,喙尖,毛色黑黄相杂,鸣声圆滑。

[21] 两头蛇:见肴韵“断蛇”句注。

[22] 帝里:犹言帝乡,指上帝所居之处。

[23] 尧仁敷率土,舜德被流沙:都是对尧舜的称颂。敷率土,是说遍及所有的地方。流沙,古人指中国以西极远的地区。

[24] 桥上授书曾纳履:传说张良年轻时曾遇到一位坐在下邳圯(桥)上的老人,命他到桥下去取失落的鞋,张良恭恭敬敬地做了这件事,老人很高兴,说孺子可教也,就授予他三卷兵书,并说自己就是黄石公。纳履,穿鞋。

[25] 壁间题句已笼纱:唐代王播少孤贫,客居扬州惠招寺木兰院,随僧斋食,为诸僧所不礼。后播显贵重游旧地,见昔日在该寺壁上所题诗句,僧已用碧纱盖护,因题曰:“上堂已散各西东,惭愧阇梨饭后钟。三十年来尘扑面,如今始得碧纱笼。”

[26] 碛:水中堆沙。

[27] 义鹘:鹰类鸷禽。鸷,凶猛的鸟。慈鸦:古人传说乌鸦是孝鸟,老鸟不能取食时,小鸟能反哺其母,因称慈鸦。

[28] 苎:一种麻类,皮可为纺织原料。黄麻:此指黄麻纸,唐时以黄麻纸写诏书。

[29] 读三到:古人经验,读书要眼到、口到、心到。

[30] 吟八叉:唐诗人温庭筠才思敏捷,传说他八叉其手而诗成,人呼之为温八叉。

[31] 沉李并浮瓜:古人消暑,往往置水果于冷水中,故有沉李浮瓜之说。

[32] 羽客片时能煮石:羽客,即仙人。道教说仙人能煮白石为饭。

[33] 狐禅千劫似蒸沙:佛教说法,狐禅毫无意义,犹如蒸沙土,虽历尽千劫,不能成饭。佛经云,狐禅如蒸沙,千劫不能成饭。

[34] 党尉粗豪,金帐笼香斟美酒;陶生清逸,银铛融雪啜团茶:《事文类聚》载,宋学士陶毂得党太尉家姬。一次烹雪茶,陶问姬曰:“党家有此味否?”姬曰:“彼但知坐销金帐里,共饮羊羔美酒,浅斟低唱而已。”铛,平底锅。





七qī 阳yáng


台tái 对duì 阁gé ,沼zhǎo 对duì 塘táng 。朝zhāo 雨yǔ 对duì 夕xī 阳yáng 。游yóu 人rén 对duì 隐yǐn 士shì ,谢xiè 女nǚ 对duì 秋qiū 娘niáng [1] 。三sān 寸cùn 舌shé [2] ,九jiǔ 回huí 肠cháng [3] 。玉yù 液yè 对duì 琼qióng 浆jiāng [4] 。秦qín 皇huáng 照zhào 胆dǎn 镜jìng [5] ,徐xú 肇zhào 返fǎn 魂hún 香xiāng [6] 。青qīng 萍píng [7] 夜yè 啸xiào 芙fú 蓉róng 匣xiá ,黄huáng 卷juàn [8] 时shí 摊tān 薜bì 荔lì 床chuáng 。元yuán 亨hēng 利lì 贞zhēn ,天tiān 地dì 一yī 机jī 成chéng 化huà 育yù [9] ;仁rén 义yì 礼lǐ 智zhì ,圣shèng 贤xián 千qiān 古gǔ 立lì 纲gāng 常cháng 。

红hóng 对duì 白bái ,绿lǜ 对duì 黄huáng 。昼zhòu 永yǒng 对duì 更gēng 长cháng 。龙lóng 飞fēi 对duì 凤fèng 舞wǔ ,锦jǐn 缆lǎn 对duì 牙yá 樯qiáng [10] 。云yún 弁biàn 使shǐ [11] ,雪xuě 衣yī 娘niáng [12] ,故gù 国guó 对duì 他tā 乡xiāng 。雄xióng 文wén 能néng 徙xǐ 鳄è [13] ,艳yàn 曲qǔ 为wèi 求qiú 凰huáng [14] 。九jiǔ 日rì 高gāo 峰fēng 惊jīng 落luò 帽mào [15] ,暮mù 春chūn 曲qū 水shuǐ 喜xǐ 流liú 觞shāng [16] 。僧sēng 占zhàn 名míng 山shān ,云yún 绕rào 茂mào 林lín 藏cáng 古gǔ 殿diàn ;客kè 栖qī 胜shèng 地dì [17] ,风fēng 飘piāo 落luò 叶yè 响xiǎng 空kōng 廊láng 。

衰shuāi 对duì 壮zhuàng ,弱ruò 对duì 强qiáng 。艳yàn 饰shì 对duì 新xīn 妆zhuāng [18] 。御yù 龙lóng 对duì 司sī 马mǎ [19] ,破pò 竹zhú 对duì 穿chuān 杨yáng [20] 。读dú 班bān 马mǎ [21] ,识shí 求qiú 羊yáng [22] 。水shuǐ 色sè 对duì 山shān 光guāng [23] 。仙xiān 棋qí 藏cáng 绿lǜ 橘jú [24] ,客kè 枕zhěn 梦mèng 黄huáng 粱liáng 。池chí 草cǎo 入rù 诗shī 因yīn 有yǒu 梦mèng [25] ,海hǎi 棠táng 带dài 恨hèn 为wèi 无wú 香xiāng [26] 。风fēng 起qǐ 画huà 堂táng [27] ,帘lián 箔bó [28] 影yǐng 翻fān 青qīng 荇xìng 沼zhǎo ;月yuè 斜xié 金jīn 井jǐng [29] ,辘lù 轳lú [30] 声shēng 度dù 碧bì 梧wú 墙qiáng 。

臣chén 对duì 子zǐ ,帝dì 对duì 王wáng 。日rì 月yuè 对duì 风fēng 霜shuāng 。乌wū 台tái 对duì 紫zǐ 府fǔ [31] ,雪xuě 牖yǒu 对duì 云yún 房fáng [32] 。香xiāng 山shān 社shè ,昼zhòu 锦jǐn 堂táng [33] 。蔀bù 屋wū 对duì 岩yán 廊láng [34] 。芬fēn 椒jiāo 涂tú 内nèi 壁bì [35] ,文wén 杏xìng 饰shì 高gāo 粱liáng [36] 。贫pín 女nǚ 幸xìng 分fēn 东dōng 壁bì 影yǐng [37] ,幽yōu 人rén 高gāo 卧wò 北běi 窗chuāng 凉liáng [38] 。绣xiù 阁gé 探tàn 春chūn ,丽lì 日rì 半bàn 笼lǒng 青qīng 镜jìng 色sè [39] ;水shuǐ 亭tíng 醉zuì 夏xià ,薰xūn 风fēng 常cháng 透tòu 碧bì 筒tǒng [40] 香xiāng 。



* * *



[1] 谢女:指晋代才女谢道韫,人称咏絮高才。秋娘:即杜秋娘,唐宗室李锜(qí)妾,能诗。

[2] 三寸舌:指能说善辩。史载战国时毛遂以三寸之舌,强于百万之师。

[3] 九回肠:形容人心情郁闷。

[4] 玉液、琼浆:都是道教服食的药饵。

[5] 秦皇照胆镜:传说秦始皇有照胆镜,能透视人的内脏,发现有人胆张心动,就意味着要暗害他,当即杀掉。

[6] 徐肇返魂香:《十洲记》载,西海申未洲上有大树,叶香闻数百里,煎制成膏,名返生香,死尸在地,闻之可活。又释徐肇遇苏德音,授以返魂香,燃之,能起上世亡魂。

[7] 青萍:宝剑名。

[8] 黄卷:用绢书写的书籍,此指道书或佛经。

[9] 元亨利贞,天地一机成化育:元亨利贞,是《周易·乾卦》中的一句。古人解释说:“元者善之长也,亨者嘉之会也,利者义之和也,贞者事之干也。”称为四德。二句的意思是,由于天地有此四德,才化生了万物。

[10] 锦缆对牙樯:用锦缎做缆绳,以象牙为樯橹。樯,桅杆。

[11] 云弁使:指蜻蜓。

[12] 雪衣娘:白鹦鹉。

[13] 雄文能徙鳄:潮州有鳄鱼为害,韩愈做刺史,作《祭鳄鱼文》驱之,传说鳄鱼就迁到了它地。

[14] 艳曲为求凰:汉时成都卓王孙有女文君新寡,司马相如爱上了她,作《凤求凰》曲以挑之,文君于是同他私奔。

[15] 九日高峰惊落帽:晋孟嘉为桓温之参军,九月九日游龙山,群僚毕集,有风将孟嘉帽子吹落而不觉。孙盛作文嘲笑,他即时作答,四座皆服。

[16] 暮春曲水喜流觞:晋永和上巳日农历三月初三,王羲之、王献之、谢安、孙绰诸人曾在山阴兰亭集会,于水边嬉游采兰,曲水流觞,饮酒赋诗以娱,以消除不祥,称为修禊。王羲之有《兰亭集序》记此事,文中有“暮春之初”“引以为流觞曲水”等语。

[17] 胜地:著名的景色宜人的地方。

[18] 艳饰:犹言浓妆打扮。新妆:指女子刚修饰好的仪容,或指女子新颖别致的打扮修饰。

[19] 御龙:驾御龙。传说夏时刘累曾为孔甲养龙,因赐姓为御龙氏。司马:官名,也是姓。

[20] 破竹:比喻做事顺利。穿杨:传说楚将养由基善射,百步之内,可穿杨叶。

[21] 读班马:班固作《汉书》,司马迁作《史记》。

[22] 识求羊:西汉末,蒋诩解官归桂林后,于竹林中开三条小径,惟故人求仲、羊仲从之游,不与俗人往还。

[23] 水色:水面呈现的色泽。山光:山的景色。

[24] 仙棋藏绿橘:神话故事,巴邛人家有橘树,一年忽长三枚,果实大如斗,剖之有二叟对弈。

[25] 池草入诗因有梦:传说南朝宋诗人谢灵运一次生病,因梦见族弟惠连而得“池塘生春草,园柳变鸣禽”之佳句。

[26] 海棠带恨为无香:宋彭渊林曰:吾生平五恨。一恨鱼多骨,二恨橘多酸,三恨菜性淡,四恨海棠无香,五恨曾子固不能诗。曾子固,曾巩字子固,古文“唐宋八大家”之一。

[27] 画堂:泛指华丽的堂舍。

[28] 帘箔:帘子。多以竹、苇编成。

[29] 金井:井栏上有雕饰的井。一般用以指宫庭园林里的井。

[30] 辘轳:井上的汲水器。

[31] 乌台:《汉书·朱博传》载,时御史府中列柏树,常有野乌数千栖息其上,后因称御史府(台)为乌台。紫府:道家称仙人居所。

[32] 雪牖:雪窗。云房:僧、道或隐者所居之室。

[33] 香山社:唐白居易于洛阳与胡杲(gǎo)、吉皎等八位老人结为九老会,因结于香山,故称为香山九老社。昼锦堂:北宋韩琦封魏国公,在做武康节度使时,于故乡相州修了一所殿堂,取名昼锦堂以致其荣,致仕退老其中。文学家欧阳修曾写有《昼锦堂记》,详述其事。

[34] 蔀屋:指草屋。岩廊:高大的宫殿。

[35] 芬椒涂内壁:汉代皇后所居宫室,以椒和泥涂内壁,取其香和多子之意,称椒房。

[36] 文杏饰高粱:旧题司马相如《长门赋》:“饰文杏以为梁。”后以杏梁指建筑华美。

[37] 贫女幸分东壁影:《战国策》载寓言故事,齐女与邻妇共烛而绩,妇辞之,女曰:“我贫无烛。一室之中,多不为暗,少不为明,何惜东壁余光。”邻妇觉得有理,就留下了她。唐李白诗:“愿假东壁辉,余光照贫女。”

[38] 幽人高卧北窗凉:晋代陶潜《与子俨等疏》:“常言五六月中,北窗下卧,遇凉风暂至,自谓是羲皇上人。”意思是说他自己夏日卧北窗下,每当凉风吹来,就好像回到了无忧无虑的太古时代一样。

[39] 绣阁:旧时女子闺房。青镜:即青铜镜。

[40] 碧筒:三国魏郑悫(què)取荷茎通之以盛酒,名曰碧筒杯。





八bā 庚gēng


形xíng 对duì 貌mào ,色sè 对duì 声shēng 。夏xià 邑yì 对duì 周zhōu 京jīng 。江jiāng 云yún 对duì 涧jiàn 树shù ,玉yù 磬qìng 对duì 银yín 筝zhēng [1] 。人rén 老lǎo 老lǎo [2] ,我wǒ 卿qīng 卿qīng [3] 。晓xiǎo 燕yàn 对duì 春chūn 莺yīng 。玄xuán 霜shuāng 舂chōng 玉yù 杵chǔ [4] ,白bái 露lù 贮zhù 金jīn 茎jīng [5] 。贾jiǎ 客kè 君jūn 山shān 秋qiū 弄nòng 笛dí [6] ,仙xiān 人rén 缑gōu 岭lǐng 夜yè 吹chuī 笙shēng [7] 。帝dì 业yè 独dú 兴xīng ,尽jìn 道dào 汉hàn 高gāo 能néng 用yòng 将jiàng [8] ;父fù 书shū 空kōng 读dú ,谁shuí 言yán 赵zhào 括kuò 善shàn 知zhī 兵bīng [9] 。

功gōng 对duì 业yè ,性xìng 对duì 情qíng 。月yuè 上shàng 对duì 云yún 行xíng 。乘chéng 龙lóng 对duì 附fù 骥jì [10] ,阆làng 苑yuàn 对duì 蓬péng 瀛yíng [11] 。春chūn 秋qiū 笔bǐ [12] ,月yuè 旦dàn 评píng [13] 。东dōng 作zuò 对duì 西xī 成chéng [14] 。隋suí 珠zhū 光guāng 照zhào 乘shèng [15] ,和hé 璧bì 价jià 连lián 城chéng 。三sān 箭jiàn 三sān 人rén 唐táng 将jiàng 勇yǒng [16] ,一yī 琴qín 一yī 鹤hè 赵zhào 公gōng 清qīng [17] 。汉hàn 帝dì 求qiú 贤xián ,诏zhào 访fǎng 严yán 滩tān 逢féng 故gù 旧jiù [18] ;宋sòng 廷tíng 优yōu 老lǎo ,年nián 尊zūn 洛luò 社shè 重zhòng 耆qí 英yīng [19] 。

昏hūn 对duì 旦dàn ,晦huì [20] 对duì 明míng 。久jiǔ 雨yǔ 对duì 新xīn 晴qíng 。蓼liǎo 湾wān 对duì 花huā 港gǎng ,竹zhú 友yǒu 对duì 梅méi 兄xiōng 。黄huáng 石shí 叟sǒu [21] ,丹dān 丘qiū 生shēng [22] 。犬quǎn 吠fèi 对duì 鸡jī 鸣míng 。暮mù 山shān 云yún 外wài 断duàn ,新xīn 水shuǐ 月yuè 中zhōng 平píng 。半bàn 榻tà 清qīng 风fēng 宜yí 午wǔ 梦mèng ,一yī 犁lí 好hǎo 雨yǔ 趁chèn 春chūn 耕gēng 。王wáng 旦dàn 登dēng 庸yōng ,误wù 我wǒ 十shí 年nián 迟chí 作zuò 相xiàng [23] ;刘liú fén 不bù 第dì ,愧kuì 他tā 多duō 士shì 早zǎo 成chéng 名míng [24] 。



* * *



[1] 玉磬:古代石制乐器名。银筝:用银装饰的筝或用银字表示音调高低的筝。

[2] 人老老:语出《孟子》:“老吾老,以及人之老;幼吾幼,以及人之幼。”意思是,尊敬自己的老人,从而也尊敬别人的老人;爱自己的孩子,从而也爱别人的孩子。人老老,即尊敬别人的老人。第一个“老”字作动词用。

[3] 我卿卿:卿是对人的尊称,也是对妻子的昵称。西晋大臣王戎妻呼戎曰卿。戎曰:“奈何卿我?”妻曰:“我不卿卿,谁当卿卿?”(意为我不称你为卿,还有谁称你为卿呢)?故后以“卿卿我我”作为夫妻恩爱之典。

[4] 玄霜舂玉杵:唐裴銏《传奇》中讲一个故事,下第秀才裴航,遇到仙人云翘夫人,赠诗一首曰:“一饮琼浆百感生,玄霜捣尽见云英。蓝桥便是神仙窟,何必崎岖上玉京。”后裴生经蓝桥驿,果遇一妪揖之求饮,妪使云英持瓯浆,令饮之。因诗合,欲娶云英,妪命裴购玉杵并捣药,果得玉杵。聘之,俱仙去。玄霜,传说中的仙药。

[5] 白露贮金茎:汉武帝好神仙之术,史载他曾作铜柱,上有铜仙人擎玉盘,承接夜露,据说以此露和玉屑饮之可长生。杜甫诗:“蓬莱宫阙对南山,承露金茎霄汉间。”又魏明帝亦作承露金茎,高十一丈。

[6] 贾(gǔ)客君山秋弄笛:《博异志》载,有商人吕乡筠,善吹笛,一次泊舟君山附近,遇到一位老人,合上天神乐、仙乐和自己欣赏的三支仙笛,吹奏数声,湖上风波大作。

[7] 仙人缑岭夜吹笙:传说周灵王太子晋好吹笙,作《凤凰鸣》,遇浮邱公,接上蒿山。后于七月七日乘白鹤过缑氏山头,拱手谢别时人而去。缑岭,山名,在河南。

[8] 帝业独兴,尽道汉高能用将:史载汉高祖刘邦善于用人,因而取得天下。汉高帝问韩信带兵几何?信曰:“多多益善。”帝曰:“卿何为我擒耶?”曰:“陛下不善将兵,而善将将。”

[9] 父书空读,谁言赵括善知兵:赵奢是战国时赵之名将。奢死,赵王令以其子赵括代廉颇为将。蔺相如说,赵括只能读其父的兵书,没有实际经验。赵王不听,使其率兵与秦交战。结果赵括中箭死,几十万军队都投降秦国,被秦人活埋了。

[10] 乘龙:唐杜甫《李监宅》:“门阑多喜气,女婿近乘龙。”因称女婿为乘龙快婿。附骥:靠别人的力量使自己得以发展,喻附于先辈或名人之后。

[11] 阆苑:传说中的仙境,在昆仑山上。蓬瀛:即蓬莱山,传说东海中的仙山。

[12] 春秋笔:旧说孔子作《春秋》,寓褒贬于字里行间,后称此种笔法为春秋笔。

[13] 月旦评:汉末河南许劭(shào)与其兄许靖俱有高名,好在一起甄别、评论当地人物,每月变换一次,农历初一发布公告,人们称之为“月旦评”。后称品评人物为月旦评或月旦。

[14] 东作、西成:《尚书·尧典》中有“平秩东作”“平秩西成”的话,“东作”是开始耕作,“西成”是收获之意。

[15] 隋珠光照乘:传说一次隋侯出行,遇断蛇于路,隋侯命人给蛇敷药包扎,后蛇衔径寸之珠报偿隋侯,因称隋侯珠。光照乘,是说把这种宝珠挂在车上可以照明前后。

[16] 三箭三人唐将勇:唐将薛仁贵东征与九姓突厥交战,三箭毙三人,威震军中。当时有歌谣曰:“将军三箭定天山,壮士长歌入汉关。”

[17] 一琴一鹤赵公清:宋赵汴治成都,匹马入蜀,以一琴一鹤相随,为政清廉简易。

[18] 汉帝求贤,诏访严滩逢故旧:即汉光武帝与严子陵的故事。见微韵“严滩”注。光武帝与严子陵友善,即位命访之,陵在富春江披蓑钓泽中,载以至朝,帝以故人礼敬之。尝以同寝,陵以足加腹。太史奏曰,有客星犯主座。

[19] 宋廷优老,年尊洛社重耆英:宋相文彦博,致仕后在洛阳同富弼、司马光等十三人,饮酒赋诗相乐,谓之耆英会。耆,年老。耆英,高年硕德的人。

[20] 晦:昏暗不明。

[21] 黄石叟:即汉初张良所遇仙人黄石公,曾赠给张良兵书。

[22] 丹丘生:道教传说中的仙人。丹丘,神话中的神仙之地,昼夜长明。

[23] 王旦登庸,误我十年迟作相:《宋史·王旦传》载,宋相王旦柄权十八年,死后,王钦若继为宰相。王钦若语人曰:“子明(即王旦)迟我十年作宰相。”登庸,选拔任用。

[24] 刘 不第,愧他多士早成名:见文韵“唐廷”注。





九jiǔ 青qīng


庚gēng 对duì 甲jiǎ ,己jǐ 对duì 丁dīng 。魏wèi 阙què 对duì 彤tóng 庭tíng [1] 。梅méi 妻qī 对duì 鹤hè 子zǐ [2] ,珠zhū 箔bó 对duì 银yín 屏píng [3] 。鸳yuān 浴yù 沼zhǎo ,鹭lù 飞fēi 汀tīng 。鸿hóng 雁yàn 对duì jí 鸰líng [4] 。人rén 间jiān 寿shòu 者zhě 相xiàng [5] ,天tiān 上shàng 老lǎo 人rén 星xīng [6] 。八bā 月yuè 好hǎo 修xiū 攀pān 桂guì 斧fǔ [7] ,三sān 春chūn 须xū 系jì 护hù 花huā 铃líng [8] 。江jiāng 阁gé 凭píng 临lín ,一yī 水shuǐ 净jìng 连lián 天tiān 际jì 碧bì ;石shí 栏lán 闲xián 倚yǐ ,群qún 山shān 秀xiù 向xiàng 雨yǔ 余yú 青qīng 。

危wēi 对duì 乱luàn ,泰tài 对duì 宁níng 。纳nà 陛bì 对duì 趋qū 庭tíng [9] 。金jīn 盘pán 对duì 玉yù 箸zhù ,泛fàn 梗gěng [10] 对duì 浮fú 萍píng 。群qún 玉yù 圃pǔ [11] ,众zhòng 芳fāng 亭tíng 。旧jiù 典diǎn [12] 对duì 新xīn 型xíng 。骑qí 牛niú 闲xián 读dú 史shǐ [13] ,牧mù 豕shǐ 自zì 横héng 经jīng [14] 。秋qiū 首shǒu 田tián 中zhōng 禾hé 颖yǐng [15] 重zhòng ,春chūn 馀yú 园yuán 内nèi 菜cài 花huā 馨xīn 。旅lǚ 次cì [16] 凄qī 凉liáng ,塞sài 月yuè 江jiāng 风fēng 皆jiē 惨cǎn 淡dàn ;筵yán 前qián 欢huān 笑xiào ,燕yān 歌gē 赵zhào 舞wǔ 独dú 娉pīng 婷tíng [17] 。



* * *



[1] 魏阙:高大的城阙。魏,通巍,形容高大。彤庭:指帝王宫殿。

[2] 梅妻、鹤子:宋林逋隐居西湖孤山,以梅鹤自娱。逋不娶,无子,时人说林“梅妻鹤子”。

[3] 珠箔、银屏:语出唐白居易《长恨歌》:“珠箔银屏迤逦开。”

[4] 鸰:鸟名,生活在水边,食小虫,喜欢群飞。

[5] 人间寿者相:旧时迷信,讲论骨相。寿者相,就是看上去长寿的相貌。

[6] 天上老人星:《史记·天官书》载天上有南极老人星,主寿。

[7] 八月好修攀桂斧:神话传说,汉人吴刚,因学仙有过,罚他砍月中桂树,桂树高五百尺,砍后伤口复合,所以吴刚要永远砍下去。旧时以科举登第为攀桂,考试一般定在八月,称“秋闱”。

[8] 三春须系护花铃:明代宁王爱花,尝作护花铃,蜂、鸟至则牵铃惊之。

[9] 纳陛:原意是深入殿堂的台阶,这里是登上台阶的意思。趋庭:快步走过庭院。《论语》记载:孔子的儿子孔鲤,一次趋庭而过,被孔子叫住,问他学诗学礼的情况。以后就把见父亲叫趋庭。

[10] 泛梗:《说苑》中的一则寓言,孟尝君入秦,客止之。见有木梗人谓土偶人曰:“今将大雨,子必沮坏。”答曰:“我沮,乃反吾真耳。今子,东园之桃也。刻子以为梗,雨至必浮,子泛泛不知所至矣。”孟尝君乃止。后遂以泛梗比喻到处漂流,无处安身。梗,这里指木偶。

[11] 群玉圃:传说仙人西王母居住在群玉山的瑶圃。

[12] 旧典:旧时的制度、法则。

[13] 骑牛闲读史:隋末李密好学,常将《汉书》一帙挂于牛角之上,骑牛读书。

[14] 牧豕自横经:汉公孙宏,年少时生活清贫,为人放猪,但自己勤奋学习,常带经卷读。年五十后位至丞相。

[15] 禾颖:带芒的谷穗。

[16] 旅次:旅途中小住的地方。也指旅途中暂作停留。

[17] 筵前欢笑,燕歌赵舞独娉婷:古代燕、赵多出歌伎,其人善歌舞。娉婷:舞姿优美的样子。





十shí 蒸zhēng


萍píng 对duì 蓼liǎo





[1] , jiǎo 对duì 菱líng [2] 。雁yàn 弋yì [3] 对duì 鱼yú 罾zēng 。齐qí 纨wán 对duì 鲁lǔ 绮qǐ ,蜀shǔ 锦jǐn 对duì 吴wú 绫líng [4] 。星xīng 渐jiàn 没mò ,日rì 初chū 升shēng 。九jiǔ 聘pìn 对duì 三sān 征zhēng [5] 。萧xiāo 何hé 曾céng 作zuò 吏lì [6] ,贾jiǎ 岛dǎo 昔xī 为wéi 僧sēng [7] 。贤xián 人rén 视shì 履lǚ 循xún 规guī 矩jǔ [8] ,大dà 匠jiàng 挥huī 斤jīn 校jiào 准zhǔn 绳shéng [9] 。野yě 渡dù 春chūn 风fēng ,人rén 喜xǐ 乘chéng 潮cháo 移yí 酒jiǔ 舫fǎng [10] ;江jiāng 天tiān 暮mù 雨yǔ ,客kè 愁chóu 隔gé 岸àn 对duì 渔yú 灯dēng 。

谈tán 对duì 吐tǔ ,谓wèi 对duì 称chēng 。冉rǎn 闵mǐn 对duì 颜yán 曾zēng [11] 。侯hóu 嬴yíng 对duì 伯bó 嚭pǐ [12] ,祖zǔ 逖tì 对duì 孙sūn 登dēng [13] 。抛pāo 白bái 纻zhù [14] ,宴yàn 红hóng 绫líng [15] 。胜shèng 友yǒu 对duì 良liáng 朋péng 。争zhēng 名míng 如rú 逐zhú 鹿lù [16] ,谋móu 利lì 似sì 趋qū 蝇yíng [17] 。仁rén 杰jié 姨yí 惭cán 周zhōu 不bù 仕shì [18] ,王wáng 陵líng 母mǔ 识shí 汉hàn 方fāng 兴xīng [19] 。句jù 写xiě 穷qióng 愁chóu ,浣huàn 花huā 寄jì 迹jì 传chuán 工gōng 部bù [20] ;诗shī 吟yín 变biàn 乱luàn ,凝níng 碧bì 伤shāng 心xīn 叹tàn 右yòu 丞chéng [21] 。



* * *



[1] 萍:水生植物。蓼:一年生或多年生草本植物。

[2] :疑为“茭”,茭白,菰的花径一种菌侵入后,刺激其细胞增生而成的肥大嫩茎,可作蔬菜。菱:一年生水生草本植物,果实有硬壳,有角,称“菱”或“菱角”,可食。

[3] 弋:一种尾上带绳子的箭。雁弋即射雁的这种箭。罾:一种用竹竿或木棍做的方形鱼网。

[4] 齐纨对鲁绮,蜀锦对吴绫:纨、绮、锦、绫都是名贵的丝织品;齐、鲁、蜀、吴是上述四种织品的产地。

[5] 九聘、三征:聘和征都是王朝或官府聘请的意思。九聘,多次聘请。三征,朝廷三次征召。

[6] 萧何曾作吏:史载萧何曾做沛郡的主吏椽,是管人事的小官。

[7] 贾岛昔为僧:唐诗人贾岛曾为僧人,法名无本。韩愈赏其诗才,令其还俗,劝其读书,后登进士,官长江主簿。

[8] 贤人视履循规矩:《尔雅·释言》:“履,礼也。”注:“礼可以履行也。”所以说视履成规矩。

[9] 大匠挥斤校准绳:《庄子》中的一则寓言说,郢人在鼻子尖上涂一点白土,一位石匠把父子抡得呼呼响,一下子就把泥点砍掉了,对鼻子丝毫无损。大匠,技术高超的匠人。斤,斧子的一种。

[10] 舫:船,画舫(装饰华美专供旅游用的船);酒舫,载酒或卖酒的船。

[11] 冉闵对颜曾:冉有、闵子骞、颜渊、曾参都是孔子的高足弟子。

[12] 侯嬴:战国时魏人,初为大梁(今河南开封)夷门的守门小吏,慷慨任侠,帮助信陵君窃符救赵,最后以身殉之。王维《夷门歌》专咏此事。伯嚭:即太宰嚭,春秋时楚伯州犁之孙,吴国奸臣。他受越王贿赂,劝吴王同越讲和。勾践灭吴,以伯嚭对其主不忠,杀之。

[13] 祖逖:东晋时爱国将领。见先韵“祖生鞭”注。孙登:晋初隐士。

[14] 抛白纻:宋裴思谦登第,以红笺数十幅入平康赋诗。王元之有诗云:“利市襕衫抛白纻,风流名字写红笺。”白纻,白苎麻织成的衣服。白纻襕衫,唐举子之服。

[15] 宴红绫:唐御膳以红绫饼为重。昭宗时放进士榜,得裴格等二十八人,会宴曲江,命御厨烧作红绫饼二十八枚赐之。

[16] 逐鹿:逐鹿中原,原指在战场上争夺政权。后来又有“未知鹿死谁手”的话,比喻胜负难定,这里即用此意。

[17] 趋蝇:追赶苍蝇。古有“蝇头微利”的说法,“趋蝇”是说十分不值得。

[18] 仁杰姨惭周不仕:唐狄仁杰为武后相,其姨卢氏有子,杰欲官之,姨曰:“姨止一子,不欲令事后周女主。”仁杰大惭而归。周,武则天的国号。

[19] 王陵母识汉方兴:王陵事汉,其母在楚,知汉必兴,嘱善事之。项羽令母召陵,母遂自刎。

[20] 句写穷愁,浣花寄迹传工部:这是写杜甫的事,杜拾遗曾为检校员外郎,后人称之为杜工部。晚年流落蜀中,寓居成都西郊浣花溪旁之浣花村草堂。

[21] 诗吟变乱,凝碧伤心叹右丞:王维官尚书右丞相,后人称之为王右丞。安史之乱陷身贼中,被迫为给事中。传说安禄山宴于凝碧宫,令乐人作乐,维闻而伤之,作七绝一首云:“万户伤心生野烟,百僚何日更朝天。秋槐叶落空宫里,凝碧池头奏管弦。”





十shí 一yī 尤yóu


荣róng 对duì 辱rǔ ,喜xǐ 对duì 忧yōu 。缱qiǎn 绻quǎn 对duì 绸chóu 缪móu [1] 。吴wú 娃wá 对duì 越yuè 女nǚ [2] ,野yě 马mǎ 对duì 沙shā 鸥ōu [3] 。茶chá 解jiě 渴kě ,酒jiǔ 消xiāo 愁chóu 。白bái 眼yǎn 对duì 苍cāng 头tóu [4] 。马mǎ 迁qiān 修xiū 《史shǐ 记jì 》,孔kǒng 子zǐ 作zuò 《春chūn 秋qiū 》。莘shēn 野yě 耕gēng 夫fū 闲xián 举jǔ 耜sì [5] ,渭wèi 滨bīn 渔yú 父fǔ 晚wǎn 垂chuí 钩gōu [6] 。龙lóng 马mǎ 游yóu 河hé ,羲xī 帝dì 因yīn 图tú 而ér 画huà 卦guà [7] ;神shén 龟guī 出chū 洛luò ,禹yǔ 王wáng 取qǔ 法fǎ 以yǐ 明míng 畴chóu [8] 。

冠guān 对duì 履lǚ ,舄xì [9] 对duì 裘qiú 。院yuàn 小xiǎo 对duì 庭tíng 幽yōu 。面miàn 墙qiáng 对duì 膝xī 地dì [10] ,错cuò 智zhì 对duì 良liáng 筹chóu [11] 。孤gū 嶂zhàng 耸sǒng ,大dà 江jiāng 流liú 。芳fāng 泽zé 对duì 圆yuán 丘qiū [12] 。花huā 潭tán 来lái 越yuè 唱chàng ,柳liǔ 屿yǔ 起qǐ 吴wú 讴ōu [13] 。莺yīng 懒lǎn 燕yàn 忙máng 三sān 月yuè 雨yǔ ,蛩qióng 摧cuī 蝉chán 退tuì 一yī 天tiān 秋qiū 。钟zhōng 子zǐ 听tīng 琴qín ,荒huāng 径jìng 入rù 林lín 山shān 寂jì 寂jì [14] ;谪zhé 仙xiān 捉zhuō 月yuè ,洪hóng 涛tāo 接jiē 岸àn 水shuǐ 悠yōu 悠yōu [15] 。

鱼yú 对duì 鸟niǎo ,鹡jí 对duì 鸠jiū 。翠cuì 馆guǎn 对duì 红hóng 楼lóu [16] 。七qī 贤xián 对duì 三sān 友yǒu [17] ,爱ài 日rì [18] 对duì 悲bēi 秋qiū 。虎hǔ 类lèi 狗gǒu [19] ,蚁yǐ 如rú 牛niú [20] 。列liè 辟bì [21] 对duì 诸zhū 侯hóu 。陈chén 唱chàng 临lín 春chūn 乐yuè [22] ,隋suí 歌gē 清qīng 夜yè 游yóu [23] 。空kōng 中zhōng 事shì 业yè 麒qí 麟lín 阁gé [24] ,地dì 下xià 文wén 章zhāng 鹦yīng 鹉wǔ 洲zhōu [25] 。旷kuàng 野yě 平píng 原yuán ,猎liè 士shì 马mǎ 蹄tí 轻qīng 似sì 箭jiàn ;斜xié 风fēng 细xì 雨yǔ ,牧mù 童tóng 牛niú 背bèi 稳wěn 如rú 舟zhōu 。



* * *



[1] 缱绻、绸缪:都是形容感情亲密、情意缠绵的样子。

[2] 吴娃:吴地的姑娘。娃,少女。越女:古代越国多出美女,西施其尤著者。后因以泛指越地美女。

[3] 野马:《庄子·逍遥游》中说:“野马也,尘埃也,生物之以息相吹也。”野马说的是早春大地上蒸腾的水蒸气。沙鸥:指栖息沙洲的鸥一类的水鸟。

[4] 苍头:在秦末农民大起义中,有一支义军的士卒以青巾裹头,称苍头军。后世苍头多指老年仆人。

[5] 莘野耕夫闲举耜:此句疑用伊尹故事。《吕氏春秋》说:有侁(shēn)氏女子得婴儿于空桑之中,名伊尹,长而贤,商汤王准备聘请他,有侁氏不肯,汤于是聘有侁氏女,以伊尹为陪嫁奴隶取了去,后以为相,国大治。有侁氏即有莘氏。

[6] 渭滨渔父晚垂钩:指商代末年姜尚的故事。见萧韵“千载”注。

[7] 龙马游河,羲帝因图而画卦:见鱼韵“洛龟”注。

[8] 神龟出洛,禹王取法以明畴:上古传说,夏禹曾参照洛水神龟献出的宝书,制定了“洪范九畴”。

[9] 舄:鞋。

[10] 面墙:《论语》记述孔子的话说:“人而不为《周南》《如南》,其犹正墙面而立也与?”后来“面墙”就成了思路闭塞的代用语。膝地:两膝着地。

[11] 错智对良筹:错指西汉政治家晁错,他在文帝时曾为太子家令。太子家令是主管太子府内庶务的官员,相当于太子府的总管,很有谋略,多智,大家称他为“智囊”。良指张良。良筹是说张良的高明策略。又解为汉初张良借箸筹画政事。

[12] 芳泽:泽本是妇女用的脂粉,或说内衣,后芳泽即转为女性的代称。圆丘:是古代天子祭祀天神的地方,也写作圜丘。

[13] 吴讴:吴地的民歌。

[14] 钟子听琴,荒径入林山寂寂:上古故事,俞伯牙善于弹琴,钟子期善解琴,闻伯牙鼓“高山流水”曲,遂相知好。子期死,伯牙碎琴不复鼓,谓无知音也。

[15] 谪仙捉月,洪涛接岸水悠悠:古代民间传说,诗人李白特别喜爱明月,在采石矶,一次酒醉,看到江心倒映的月影,就前去扑捉,结果溺水而死。谪:封建时代特指贬官。

[16] 翠馆:犹青楼,妓院。红楼:犹青楼。妓女所居。

[17] 七贤:晋嵇康与阮籍、山涛、向秀、阮咸、王戎、刘伶友好,常宴集于竹林之下,号为竹林七贤。三友:以三种事物为友,如松、竹、梅;琴、酒、诗;梅、石、竹等。

[18] 爱日:珍惜时间。悲秋:看到秋天草木凋零而感到伤悲。

[19] 虎类狗:东汉马援在《戒兄子严敦书》中,告诫他们说,学龙伯高,不成犹为谨慎之士,所谓刻鹄不成尚类鹜;学习豪侠好义的杜季良,不成刚为天下轻薄子,所谓画虎不成反类狗。

[20] 蚁如牛:晋殷浩患耳疾,听见床下蚂蚁动,以为是牛斗之声。

[21] 列辟:诸王侯。

[22] 陈唱临春乐:南朝陈后主荒淫,修结绮、临春、望仙阁,与张丽华、江总、孔贵嫔诸人日夜游戏、歌唱,其中以《玉树后庭花》《临春乐》为最有名。

[23] 隋歌清夜游:传说隋炀帝夏夜宴游,放萤火虫照明,歌清夜之曲;冬日剪彩为花。

[24] 空中事业麒麟阁:汉宣帝时,为了表彰功臣,将霍光、苏武等画在麒麟阁上,共十一人。“空中事业”,是说功名富贵本来是虚幻的,这是作者的消极思想。

[25] 地下文章鹦鹉洲:三国时才士祢衡,因反对曹操,被排挤到荆州,后被刘表部下黄祖(忌其才)杀害。他曾写过《白鹦鹉赋》,因此人们把它被害之处称之为鹦鹉洲。“地下文章”是说该人已死去。





十shí 二èr 侵qīn


歌gē 对duì 曲qǔ ,啸xiào 对duì 吟yín [1] 。往wǎng 古gǔ 对duì 来lái 今jīn 。山shān 头tóu 对duì 水shuǐ 面miàn ,远yuǎn 浦pǔ 对duì 遥yáo 岑cén [2] 。勤qín 三sān 上shàng ,惜xī 寸cùn 阴yīn [3] 。茂mào 树shù 对duì 平píng 林lín [4] 。卞biàn 和hé 三sān 献xiàn 玉yù [5] ,杨yáng 震zhèn 四sì 知zhī 金jīn 。青qīng 皇huáng 风fēng 暖nuǎn 吹chuī 芳fāng 草cǎo [6] ,白bái 帝dì 城chéng 高gāo 急jí 暮mù 砧zhēn [7] 。绣xiù 虎hǔ 雕diāo 龙lóng ,才cái 子zǐ 窗chuāng 前qián 挥huī 彩cǎi 笔bǐ [8] ;描miáo 鸾luán 刺cì 凤fèng ,佳jiā 人rén 帘lián 下xià 度dù 金jīn 针zhēn 。

登dēng 对duì 眺tiào ,涉shè 对duì 临lín [9] 。瑞ruì 雪xuě 对duì 甘gān 霖lín [10] 。主zhǔ 欢huān 对duì 民mín 乐lè ,交jiāo 浅qiǎn 对duì 言yán 深shēn [11] 。耻chǐ 三sān 战zhàn ,乐lè 七qī 擒qín [12] 。顾gù 曲qǔ [13] 对duì 知zhī 音yīn 。大dà 车chē 行xíng 槛jiàn 槛jiàn [14] ,驷sì 马mǎ 骤zhòu 骎qīn 骎qīn [15] 。紫zǐ 电diàn 青qīng 虹hóng 腾téng 剑jiàn 气qì [16] ,高gāo 山shān 流liú 水shuǐ 识shí 琴qín 心xīn [17] 。屈qū 子zǐ 怀huái 君jūn ,极jí 浦pǔ 吟yín 风fēng 悲bēi 泽zé 畔pàn [18] ;王wáng 郎láng 忆yì 友yǒu ,扁piān 舟zhōu 卧wò 雪xuě 访fǎng 山shān 阴yīn [19] 。



* * *



[1] 啸:撮口作声,打口哨。吟:声调抑扬地念。

[2] 遥岑:远处陡峭的小山崖。

[3] 勤三上:古人经验,认为善读者有“三上”之功,即枕上、途上、厕上。惜寸阴:东晋大将陶侃致力收复中原,朝夕运甓(pì,砖),常勉励大家说:“大禹惜寸阴,吾人当惜分阴。”寸阴,很短的时光。

[4] 平林:平原上的林木。

[5] 卞和三献玉:见庚韵“和璧”句注。卞和即和氏。

[6] 青皇:又称东皇、青帝。东方为春,古人所谓司春之神,故代指春天。

[7] 白帝城高急暮砧:唐杜甫《秋兴八首》诗:“寒衣处处催刀尺,白帝城高急暮砧。”白帝城在四川重庆市奉节县,三国刘备殁于此。砧,捣衣石,这里指砧杵之声。

[8] 绣虎雕龙,才子窗前挥彩笔:曹子建有奇才,七步成诗,人称绣虎之手。雕龙:南朝梁刘勰《文心雕龙》论古今文章的体裁和创作,有很高价值。

[9] 眺:望,往远处看。涉:到,经历。

[10] 瑞雪:应时的好雪。甘霖:久旱后下的雨;及时雨。

[11] 交浅对言深:战国时范睢说秦王,有“交疏”“言深”等语。交浅犹交疏,是说人与人的关系很一般,没有深交。言深,是讲了至关重要的意见。

[12] 三战:传说春秋时鲁国将军曹刿,曾三次兵败于齐。后来齐桓公和鲁庄公盟于柯,曹刿用匕首逼迫齐桓公,终于索回了失去的国土。七擒:传说孔明征南蛮,曾对其首领孟获七擒七纵,使孟获受到感化,最后归顺。

[13] 顾曲:《三国志·周瑜传》载,三国吴周瑜善审音律,曲有阙误,瑜必知之,知之必顾,故时人谣曰:“曲有误,周郎顾。”唐李瑞诗:“欲得周郎顾,时时误拂弦。”

[14] 大车行槛槛:语出《诗经·王风·大车》:“大车槛槛。”大车是上古载重的牛车。槛槛,车声。

[15] 驷马骤骎骎:语出《诗经·小雅·四牡》:“驾彼四骆,载骤骎骎。”驷马,上古一车四马。骤,奔驰。骎骎,马跑得很快的样子。

[16] 紫电青虹:形容宝剑的光华。

[17] 高山流水识琴心:这是关于钟子期、俞伯牙的故事。参见尤韵“钟子”注。据说一次伯牙弹琴,钟子期评论说,此曲“志在高山”;再弹,又评论说,此曲“志在流水”。琴心,琴曲的内容、主题。

[18] 屈子怀君,极浦吟风悲泽畔:见豪韵“遭谗”注。极浦,犹言远浦,远方的水滨。

[19] 王郎忆友,扁舟卧雪访山阴:见豪韵“回艚”注。





十shí 三sān 覃tán


宫gōng 对duì 阙què ,座zuò 对duì 龛kān [1] 。水shuǐ 北běi 对duì 天tiān 南nán 。蜃shèn 楼lóu 对duì 蚁yǐ 郡jùn [2] ,伟wěi 论lùn 对duì 高gāo 谈tán [3] 。遴lín 杞qǐ 梓zǐ ,树shù 楩pián 楠nán [4] 。得dé 一yī 对duì 函hán 三sān [5] 。八bā 宝bǎo 珊shān 瑚hú 枕zhěn ,双shuāng 珠zhū 玳dài 瑁mào 簪zān [6] 。萧xiāo 王wáng 待dài 士shì 心xīn 惟wéi 赤chì [7] ,卢lú 相xiàng 欺qī 君jūn 面miàn 独dú 蓝lán [8] 。贾jiǎ 岛dǎo 诗shī 狂kuáng ,手shǒu 拟nǐ 敲qiāo 门mén 行xíng 处chù 想xiǎng [9] ;张zhāng 颠diān 草cǎo 圣shèng ,头tóu 能néng 濡rú 墨mò 写xiě 时shí 酣hān [10] 。

闻wén 对duì 见jiàn ,解jiě 对duì 谙ān [11] 。三sān 橘jú 对duì 双shuāng 柑gān [12] 。黄huáng 童tóng 对duì 白bái 叟sǒu ,静jìng 女nǚ [13] 对duì 奇qí 男nán 。秋qiū 七qī 七qī [14] ,径jìng 三sān 三sān [15] 。海hǎi 色sè 对duì 山shān 岚lán [16] 。鸾luán 声shēng 何hé 哕huì 哕huì [17] ,虎hǔ 视shì 正zhèng 眈dān 眈dān [18] 。仪yí 封fēng 疆jiāng 吏lì 知zhī 尼ní 父fǔ [19] ,函hán 谷gǔ 关guān 人rén 识shí 老lǎo 聃dān [20] 。江jiāng 相xiàng 归guī 池chí ,止zhǐ 水shuǐ 自zì 盟méng 真zhēn 是shì 止zhǐ [21] ;吴wú 公gōng 作zuò 宰zǎi ,贪tān 泉quán 虽suī 饮yǐn 亦yì 何hé 贪tān [22] 。



* * *



[1] 阙:皇帝居处,借指朝廷。龛:供奉佛像、神位等的小阁子。

[2] 蜃楼:海洋上由空气折射而成的幻影,古人以为是蜃(一种大蛤蜊)气所化,称蜃楼。蚁郡:唐李公佐《南柯太守传》写汉豪士淳于棼酒醉后梦游大槐安国,被招为附马,守南柯郡。醒后发现,原来槐安国和南柯郡是一群蚂蚁的窝巢。

[3] 伟论:高明超卓的言论。高谈:侃侃而谈,大发议论。

[4] 遴杞梓:比喻选拔人才。遴,谨慎选择;杞、梓,两种木质优良的树,古人以喻优秀人材。树楩楠:比喻培养人才。树,种植。楩,木名,即黄楩树。楩、楠是两种木质优良的树,生在南方。

[5] 得一:“一”是个哲学概念。《老子》中有“昔之得一者,天得一以清,地得一以宁,神得一以灵,谷得一以盈,万物得一以生,侯王得一以为天下正”的话。函三:《易纬乾凿度》说:“《易》一名而含三义:所谓易也,变易也,不易也。”意思是:《周易》的“易”字含三方面意义:简易、变易和不变。

[6] 双珠玳瑁簪:这是汉乐府《有所思》中的一句。玳瑁,一种海龟,其甲可制作工艺品。

[7] 萧王待士心惟赤:汉光武帝初起时,曾被更始帝刘玄封为萧王。他在镇压铜马、高湖等起义军时,收降许多人,并将首领封为列侯,以收买人心。所以当时有人说:“萧王推赤心置人腹中,安得不投死乎!”

[8] 卢相欺君面独蓝:唐卢杞长得特别丑陋,史称“鬼貌蓝色”,代宗时为相,迫害忠良,盘剥百姓,干了许多坏事,人曰“蓝面鬼”。

[9] 贾岛诗狂,手拟敲门行处想:唐诗人贾岛,一次在驴背上得“鸟宿池边树,僧敲月下门”两句诗,开始想用“推”字,后改“敲”,仍觉未妥,不觉冲撞京兆尹韩愈。韩愈问明原因,想了一会,认为“敲”字好。这就是“推敲”一语的由来。

[10] 张颠草圣,头能濡墨写时酣:唐张旭,善草书,好酒,每次大醉,则呼叫狂走,或把墨水浇到头上,然后写字,时人称他为“张颠”。

[11] 解:明白。谙:了解,熟悉。

[12] 双柑:唐冯贽《云仙杂记》卷二引《高隐外书》:“晋戴颙,春日携双柑斗酒,人问何之,曰:“往听黄鹂声。此俗耳针砭,诗肠鼓吹,汝知之乎?”

[13] 静女:《诗经》篇名。静女指仪态端方的少女。

[14] 秋七七:七七是传说中的人名,姓殷。鹤林寺杜鹃花为天下第一。周宝谓殷七七曰:“闻君能顷刻开花,今方重九,花能开乎?”七七曰:“诺。”即于掌中作幻术使花开。夜间一女子曰:“妾为上帝司此花,不久即归阆苑。”此七七即代指杜鹃花。

[15] 径三三:陶渊明咏菊,“冶冶溶溶三径色,风风雨雨九秋时。”此“径三三”即代指菊花。

[16] 山岚:山中的雾气。

[17] 鸾声何哕哕:《诗经·小雅·庭燎》有“君子至止,鸾声哕哕”二句。鸾,车铃。哕,乐声。

[18] 虎视正眈眈:《周易·颐卦》中的一句。眈眈,注视的样子。

[19] 仪封疆吏知尼父:仪是春秋时卫国的地名。尼父即孔子。《论语》记载,孔子到卫国去,仪邑主管边境的“封人”要求见孔子,见过之后对孔子的学生说:“你们不要为流亡而苦恼,上天将让孔子制礼作乐。”

[20] 函谷关人识老聃:传说函谷关的令尹善天文,一次登楼四望,于东方见紫色云气,高兴地说:一定有圣人经过此地。后老子骑青牛过关。杜甫诗“东来紫气满函关”即用此典。聃,老子名李聃。

[21] 江相归池,止水自盟真是止:《宋史·万里传》载,南宋末年,江万里为相,他听说元军已得襄樊,就在自家后园凿个池塘,题名“止水”。后元军至城破,万里遂投池自杀。

[22] 吴公作宰,贪泉虽饮亦何贪:《晋书·吴隐之传》载,晋吴隐之清廉,他到广州为刺史,州城附近有泉名“贪泉”,人们说,谁饮此水都会起贪心。吴隐之故意饮了贪泉水,并作诗一首说:“古人云此水,一歃怀千金。试使夷齐饮,终当不易心。”到郡后更加廉洁自守。歃,用嘴吸取。





十shí 四sì 盐yán


宽kuān 对duì 猛měng





[1] ,冷lěng 对duì 炎yán 。清qīng 直zhí [2] 对duì 尊zūn 严yán 。云yún 头tóu 对duì 雨yǔ 脚jiǎo [3] ,鹤hè 发fà 对duì 龙lóng 髯rán [4] 。风fēng 台tái 谏jiàn [5] ,肃sù 堂táng 廉lián [6] 。保bǎo 泰tài 对duì 鸣míng 谦qiān [7] ,五wǔ 湖hú 归guī 范fàn 蠡lǐ [8] ,三sān 径jìng 隐yǐn 陶táo 潜qián [9] 。一yī 剑jiàn 成chéng 功gōng 堪kān 佩pèi 印yìn [10] ,百bǎi 钱qián 满mǎn 卦guà 便biàn 垂chuí 帘lián [11] 。浊zhuó 酒jiǔ 停tíng 杯bēi ,容róng 我wǒ 半bàn 酣hān 愁chóu 际jì 饮yǐn [12] ;好hǎo 花huā 傍bàng 座zuò ,看kàn 他tā 微wēi 笑xiào 悟wù 时shí 拈niān [13] 。

连lián 对duì 断duàn ,减jiǎn 对duì 添tiān 。淡dàn 泊bó 对duì 安ān 恬tián [14] ,回huí 头tóu 对duì 极jí 目mù ,水shuǐ 底dǐ 对duì 山shān 尖jiān 。腰yāo 袅niǎo 袅niǎo [15] ,手shǒu 纤xiān 纤xiān [16] 。凤fèng 卜bǔ 对duì 鸾luán 占zhān [17] 。开kāi 田tián 多duō 种zhòng 粟sù ,煮zhǔ 海hǎi 尽jìn 成chéng 盐yán 。居jū 同tóng 九jiǔ 世shì 张zhāng 公gōng 艺yì [18] ,恩ēn 给jǐ 千qiān 人rén 范fàn 仲zhòng 淹yān [19] 。箫xiāo 弄nòng 凤fèng 来lái ,秦qín 女nǚ 有yǒu 缘yuán 能néng 跨kuà 羽yǔ [20] ;鼎dǐng 成chéng 龙lóng 去qù ,轩xuān 臣chén 无wú 计jì 得dé 攀pān 髯rán [21] 。

人rén 对duì 己jǐ ,爱ài 对duì 嫌xián 。举jǔ 止zhǐ 对duì 观guān 瞻zhān [22] 。四sì 知zhī 对duì 三sān 语yǔ [23] ,义yì 正zhèng 对duì 辞cí 严yán 。勤qín 雪xuě 案àn ,课kè 风fēng 檐yán [24] 。漏lòu 箭jiàn 对duì 书shū 笺jiān 。文wén 繁fán 归guī 獭tǎ 祭jì [25] ,体tǐ 艳yàn 别bié 香xiāng 奁lián [26] 。昨zuó 夜yè 题tí 诗shī 更gēng 一yī 字zì [27] ,早zǎo 春chūn 来lái 燕yàn 卷juǎn 重chóng 帘lián 。诗shī 以yǐ 史shǐ 名míng ,愁chóu 里lǐ 悲bēi 歌gē 怀huái 杜dù 甫fǔ [28] ;笔bǐ 经jīng 人rén 索suǒ ,梦mèng 中zhōng 显xiǎn 晦huì 老lǎo 江jiāng 淹yān [29] 。



* * *



[1] 宽对猛:《左传》载(郑)大夫子产临终前对他的儿子说:“我死,子必为政。惟有德者能以宽服民,其次莫如猛。”宽,指仁厚。猛,指严厉。

[2] 清直:清廉正直。

[3] 云头:云彩上面。雨脚:随云飘行、长垂及地的雨丝。

[4] 鹤发:是说人发白如鹤羽,指老人。龙髯:龙的胡须。传说黄帝在鼎湖乘龙而升天,小臣扯龙髯而上,结果扯断了龙须。

[5] 风台谏:风即讽,讽谏。台,台省。谏,谏臣。古谏官所居官署称讽台。

[6] 肃堂廉:肃堂即官署。廉,阶陛之侧隅。此指廉正。

[7] 保泰对鸣谦:泰和谦是《周易》的两个卦名。保泰,意为保持安康。鸣谦是谦卦的一句爻辞,意思是以谦虚的品德为人所知。

[8] 范蠡:字少伯,佐越王勾践破吴后载西施归五湖,自号陶朱公。

[9] 三径:归隐者的家园。晋陶潜《归去来辞》:“三径就荒,松竹犹存。”

[10] 一剑成功堪佩印:战国时苏秦曾佩一剑说六国,后为纵约长,佩六国相印。

[11] 百钱满卦便垂帘:汉严君平隐居成都,以卖卜自给,每日得百钱,即闭户垂帘而授《老子》。

[12] 浊酒停杯,容我半酣愁际饮:语出杜甫诗:“艰难苦恨繁霜鬓,潦倒新亭浊酒杯。”

[13] 好花傍座,看他微笑悟时拈:佛教故事,传说在灵山会上,释迦牟尼拿出一朵花,众人都不解其意,唯独迦叶尊者露出笑颜,表示对佛的旨意有所领悟。后遂以拈花微笑表示心心相印、两心相通。拈,用手指轻轻拿着。

[14] 淡泊:对于名利淡漠,不看重。安恬:淡泊,不追求名利。

[15] 腰袅袅:形容女子腰肢柔软。

[16] 手纤纤:形容手指细而长。

[17] 凤卜对鸾占:凤卜、鸾占意同,见微韵“采凤飞”注。

[18] 居同九世张公艺:唐人张公艺,九世同居。高宗祭泰山,幸其第,问何以能此,公书百“忍”字以进之。

[19] 恩给千人范仲淹:宋范仲淹居官后,于姑苏城郊买良田千亩,建立“义庄”,以收养贫困的亲族。

[20] 箫弄凤来,秦女有缘能跨羽:见东韵“凤翔”注。

[21] 轩臣:轩辕皇帝的大臣。攀髯:传说轩辕皇帝铸鼎成,龙降,骑之上升。其臣攀龙髯欲随之升天,未得。

[22] 举止:指姿态和风度。观瞻:显露于外的形象。

[23] 四知对三语:四知见侵韵“杨震”注。三语:据《晋书》载,一次王戎问老子、孔子之道于阮瞻,阮瞻曰:“将无同。”意思是“大约差不多”。王戎听了很满意,就聘其为掾(署员),时人称阮瞻为“三语掾”。

[24] 雪案、风檐:形容读书条件很艰苦。勤和课指学习。

[25] 文繁归獭祭:早春刚刚解冻,水獭把鱼衔出水面,排列在冰上,古人以为这是獭在祭祀,称为獭祭鱼。唐诗人李商隐作诗爱用典故,经常把翻阅的书排在一旁,书册左右麟次,时人也就称他为獭祭鱼。

[26] 体艳别香奁:体艳即艳体诗,指爱情或色情诗。唐诗人韩偓喜欢写这类诗,诗集名《香奁集》,时人号为“香奁体”。香奁,妇女梳妆用的匣子。

[27] 昨夜题诗更一字:唐僧齐己作《早梅》诗,曰:“前村深雪里,昨夜数枝开。”许丁卯改为“一枝开”,时人称为“一字师”。

[28] 诗以史名,愁里悲歌怀杜甫:见豪韵“诗史”注。史名,杜甫感痛时事,发之为诗,人称为“诗史”。

[29] 笔经人索,梦中显晦老江淹:见支韵“五色笔”注。





十shí 五wǔ 咸xián


栽zāi 对duì 植zhí ,薙tì 对duì 芟shān [1] 。二èr 伯bó 对duì 三sān 监jiān [2] 。朝cháo 臣chén 对duì 国guó 老lǎo [3] ,职zhí 事shì 对duì 官guān 衔xián 。鹿lù 麌yǔ 麌yǔ [4] ,兔tù 毚chán 毚chán [5] 。启qǐ 牍dú 对duì 开kāi 缄jiān [6] 。绿lǜ 杨yáng 莺yīng xiàn 睆huǎn [7] ,红hóng 杏xìng 燕yàn 呢ní 喃nán [8] 。半bàn 篱lí 白bái 酒jiǔ 娱yú 陶táo 令lìng [9] ,一yī 枕zhěn 黄huáng 粱liáng 度dù 吕lǚ 岩yán [10] 。九jiǔ 夏xià 炎yán 飙biāo [11] ,长cháng 日rì 风fēng 亭tíng 留liú 客kè 骑jì [12] ;三sān 冬dōng 寒hán 冽liè ,漫màn 天tiān 雪xuě 浪làng 驻zhù 征zhēng 帆fān 。

梧wú 对duì 杞qǐ ,柏bǎi 对duì 杉shān 。夏xià 濩huò 对duì 韶sháo 咸xián [13] 。涧jiàn 瀍chán 对duì 溱qín 洧wěi [14] ,巩gǒng 洛luò 对duì 崤xiáo 函hán [15] 。藏cáng 书shū 洞dòng [16] ,避bì 诏zhào 岩yán [17] 。脱tuō 俗sú 对duì 超chāo 凡fán 。贤xián 人rén 羞xiū 献xiàn 媚mèi ,正zhèng 士shì 嫉jí 工gōng 谗chán 。霸bà 越yuè 谋móu 臣chén 推tuī 少shào 伯bó [18] ,佐zuǒ 唐táng 藩fān 将jiàng 重zhòng 浑hún 瑊jiān [19] 。邺yè 下xià 狂kuáng 生shēng ,羯jié 鼓gǔ 三sān 挝zhuā 羞xiū 锦jǐn 袄ǎo [20] 。江jiāng 州zhōu 司sī 马mǎ ,琵pí 琶pá 一yī 曲qǔ 湿shī 青qīng 衫shān [21] 。

袍páo 对duì 笏hù [22] ,履lǚ 对duì 衫shān 。匹pǐ 马mǎ 对duì 孤gū 帆fān 。琢zhuó 磨mó 对duì 雕diāo 镂lòu ,刻kè 划huà 对duì 镌juān 镵chán [23] 。星xīng 北běi 拱gǒng [24] ,日rì 西xī 衔xián 。卮zhī 漏lòu 对duì 鼎dǐng 馋chán [25] 。江jiāng 边biān 生shēng 桂guì 若ruò [26] ,海hǎi 外wài 树shù 都dū 咸xián [27] 。但dàn 得dé 恢huī 恢huī 存cún 利lì 刃rèn [28] ,何hé 须xū 咄duō 咄duō 达dá 空kōng 函hán [29] 。彩cǎi 凤fèng 知zhī 音yīn ,乐yuè 典diǎn 后hòu 夔kuí 须xū 九jiǔ 奏zòu [30] ;金jīn 人rén 守shǒu 口kǒu ,圣shèng 如rú 尼ní 父fǔ 亦yì 三sān 缄jiān [31] 。



* * *



[1] 薙:除去野草。芟:割草。薙、芟都是斩除野草的意思。

[2] 二伯:西周时主掌国事的两个大臣,所谓“自陕以东,周公主之;自陕以西,召公主之”。三监:武王灭殷后,封纣子武庚于商都,派自己的三个弟弟管叔、蔡叔和霍叔监督,称三监。

[3] 国老:指国之重臣。

[4] 麌:鹿成群结队的样子。

[5] 毚:狡猾。

[6] 启牍对开缄:启牍和开缄都是拆开信件的意思。

[7] 睆:即莺啼的声音。

[8] 呢喃:燕子叫声。

[9] 半篱白酒娱陶令:陶令,即陶渊明。因为他曾为彭泽令,故称。

[10] 一枕黄粱度吕岩:见阳韵“客枕”注。原故事中的吕翁和卢生,后人附会成八仙中的钟离权度化吕洞宾(吕岩),所以这里说“度吕岩”。

[11] 九夏:夏季。炎飙:热风。飙,狂风。

[12] 长日:指整天、终日。风亭:亭子。

[13] 夏濩对韶咸:见萧韵“殷濩”句注。

[14] 涧、瀍、溱、洧:古代四条河流。

[15] 巩洛对崤函:巩,古地名,洛水流经其旁。崤,崤山,山名,又叫“崤陵”,其西有函谷关,故称崤函。巩、洛、崤、函均在今河南省。

[16] 藏书洞:指传说中的二酉山,四川酉阳县翠屏山麓的小酉山石穴中,有书千卷,相传秦人读书于此,称为“二酉藏书洞”。

[17] 避诏岩:指汉初“四皓”所隐的商山,“四皓”(详见齐韵“甪里”注),高帝召之不至,故称其隐居的岩洞为“避诏岩”。

[18] 霸越谋臣推少伯:少伯,越国大夫范蠡的字。见虞韵“归湖”注。

[19] 佐唐藩将重浑瑊:浑瑊,唐王朝少数民族的著名将领,曾从李光弼、郭子仪平“安史之乱”,以功为太常卿。德宗出逃奉天,浑瑊率家人子弟从,与朱泚(cǐ)拒战,全城倚重,德宗得以保全。

[20] 邺下狂生,羯鼓三挝羞锦袄:狂生指祢衡。传说曹操欲辱祢衡,命他为鼓吏,击鼓为客人助酒兴。他不仅毫无惧色,反而脱掉衣服,敲起慷慨昂扬的“渔阳三挝”,以回敬曹操。渔阳三挝,传说中古代的鼓曲名。锦袄,代指曹操。挝,这里指敲鼓。

[21] 江州司马,琵琶一曲湿青衫:唐诗人白居易曾谪为江州司马,一次到浔阳江边送客,遇到一位流落为商人妇的琵琶女,为他弹奏了一曲,引起了他强烈的共鸣,为之流下了泪水。故作长诗《琵琶行》。其中最后两句是:“座中泪下谁最多,江州司马湿青衫。”

[22] 笏:古代大臣上朝拿着的手板,用玉、象牙或竹片制成,上面可以记事。

[23] 镌镵:都是刻削的意思。

[24] 星北拱:星指北极星,拱是拱托、环绕的意思。古人认为群星都围绕北极星而分布。

[25] 卮漏:卮,古代一种盛酒器。古语有“川源而不能实漏卮”的话,意为漏洞虽小,如不堵塞则后患无穷。鼎馋:孔子的祖先正考父为宋大夫,其家有鼎名馋鼎。馋,吃。

[26] 若:杜若,香草名。

[27] 都咸:传说中生于海外的神木。

[28] 但得恢恢存利刃:《庄子·养生主》中的一则寓言,说宋国有个庖丁,善于解牛,他的刀用了十九年,解过数千头牛,还好像新磨的一样。因为牛的关节之间是有缝隙的,而刀刃却很薄,让薄薄的刀刃通过有缝隙的关节,自然“恢恢乎其于游刃必有余地”。恢恢,宽绰的样子。

[29] 何须咄咄达空函:晋殷浩得到桓温将推荐他作尚书令的消息,非常高兴,准备回信,又怕言语不周,把信取出放进几十次,结果却寄出了空信封。后桓温将免职,他整日用手在空中乱划,连呼“咄咄怪事”。咄咄,表示惊讶的语气。

[30] 彩凤知音,乐典后夔须九奏:后夔,即夔,传说是舜的乐官,他奏起乐来,百兽起舞,凤凰也飞来。九奏,奏乐九曲。

[31] 金人守口,圣如尼父亦三缄:尼父即孔子。相传孔子入周太庙,见有铸金人,三缄其口,背后有铭文:“古之慎言人也。”三缄,封闭多层。两句的意思是,圣达如孔子,也要学习金人那样守口如瓶,讲话谨慎。

\backmatter

\end{document}