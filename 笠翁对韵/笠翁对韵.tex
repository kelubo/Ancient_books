% 笠翁对韵
% 笠翁对韵.tex

\documentclass[12pt,UTF8]{ctexbook}

% 设置纸张信息。
\usepackage[a4paper,twoside]{geometry}
\geometry{
	left=25mm,
	right=25mm,
	bottom=25.4mm,
	bindingoffset=10mm
}

% 设置字体,并解决显示难检字问题。
\xeCJKsetup{AutoFallBack=true}
\setCJKmainfont{SimSun}[BoldFont=SimHei, ItalicFont=KaiTi, FallBack=SimSun-ExtB]

% 目录 chapter 级别加点(.)。
\usepackage{titletoc}
\titlecontents{chapter}[0pt]{\vspace{3mm}\bf\addvspace{2pt}\filright}{\contentspush{\thecontentslabel\hspace{0.8em}}}{}{\titlerule*[8pt]{.}\contentspage}

% 设置 part 和 chapter 标题格式。
\ctexset{
	part/name= {},
	part/number={},
	chapter/name={,},
	chapter/number={\chinese{chapter}}
}

% 设置古文原文格式。
\newenvironment{yuanwen}{\bfseries\zihao{4}}

% 设置署名格式。
\newenvironment{shuming}{\hfill\bfseries\zihao{4}}

% 注脚每页重新编号,避免编号过大。
\usepackage[perpage]{footmisc}

\title{\heiti\zihao{0} 笠翁对韵}
\author{李渔}
\date{}

\begin{document}

\maketitle
\tableofcontents

\frontmatter
\chapter{前言、序言}

\mainmatter

% 增加空行
~\\

% 增加字间间隔,适用于三字经、诗文等。
 \qquad  

\part{卷上}

\chapter{东}

\begin{yuanwen}
天对地,雨对风。大陆对长空\footnote{宽广高远的天空。}。山花\footnote{山间野花。}对海树\footnote{成长在海边的树。},赤日\footnote{红日,烈日。}对苍穹\footnote{青天。}。雷隐隐\footnote{雷声不分明的样子。},雾蒙蒙\footnote{雾迷茫的样子。}。日下对天中。风高\footnote{风大。}秋月白,雨霁\footnote{jì,雨后或雪后天转晴。}晚霞红。牛女\footnote{牵牛、织女二星。}二星河\footnote{银河。}左右,参商\footnote{参星,在西方。商星,在东方。这里比喻彼此相隔,不得相见。}两曜\footnote{yào}斗西东\footnote{参和商是二十八宿中的两宿。商即辰,也即是心宿。参宿在西方,心宿居东方,古人往往把亲友久别难逢比为参商。斗,指二十八宿之一的斗宿,不是北斗。两曜,古人把日、月、五星称七曜,曜就是星。} 。十月塞边,飒飒\footnote{形容风吹动树木枝叶等的声音。sà}寒霜惊戍旅\footnote{守卫边疆的将士。};三冬江上,漫漫\footnote{广远无际。}朔\footnote{shu\`o,北方。}雪冷渔翁。
\end{yuanwen}

\begin{yuanwen}
河\footnote{黄河。}对汉\footnote{汉水。由于河可以借指银河,汉也可借指银河。},绿对红。雨伯对雷公\footnote{雨伯、雷公是古代神话中的雨神和雷神。雨伯原称雨师,为了属对工整,这里把师改作伯。}。烟楼\footnote{耸立于烟云中之高楼。}对雪洞\footnote{被雪封住的山洞。},月殿\footnote{月宫。}对天宫\footnote{上帝或诸神在天上的住所。}。云叆叇\footnote{ài dài,浓云蔽日之状。},日曈曚\footnote{tóng méng,太阳将出天色微明的样子。}。蜡屐\footnote{古人穿的一种底下有齿的木鞋,以蜡涂抹其上,叫蜡屐。}对渔篷。过天星\footnote{指流星。}似箭,吐魄月\footnote{魄,又作霸,月球被自身遮掩的阴影部分。古人对月的圆缺道理不理解,以为月里有只蟾蜍,是由它反复吞吐造成的。吐魄月就是刚被吐出的月,指新月,所以说它如弓。}如弓。驿旅\footnote{古代官府设立的招待往来官员的旅舍。}客逢梅子雨\footnote{即梅雨、黄梅雨。中国南部五月至七月所下的雨,因正值梅子成熟的时节,故称为梅雨。},池亭人挹藕花风\footnote{荷花香气阵阵吹来,人们在亭台上饮酒。挹,yì,酌酒。}。茅店村前,皓月\footnote{月光茫茫的样子。}坠林鸡唱韵;板桥路上,青霜锁道马行踪\footnote{这一联是从晚唐温庭筠《商山早行》中“鸡声茅店月,人迹板桥霜”两句诗隐括出来的。}。
\end{yuanwen}

\begin{yuanwen}
山对海,华\footnote{西岳华山。}对嵩\footnote{中岳嵩山。}。四岳\footnote{传说尧时分掌四时、方岳的官。四岳又释指东岳泰山、西岳华山、南岳衡山、北岳恒山。}对三公\footnote{古代天子以下最大的三个官员,各代的职称并不一致。三公又释为星名。}。宫花对禁柳\footnote{古代皇帝居住的城苑禁止百姓出入,所以称禁宫。禁柳即宫廷中的柳树。},塞雁对江龙。清暑殿\footnote{相传三国时吴有避暑宫,夏日清凉不热。},广寒宫\footnote{神话里称月亮中的宫殿为广寒宫。}。拾翠\footnote{原指拾找像翡翠一样的羽毛,后来把青年妇女采集鲜花野草也称作拾翠。}对题红\footnote{刘斧《青琐高议》载:唐僖宗时士人于祐,偶然中从御沟流水上拾到一片红叶,上面题有两句诗:“流水何太急,深宫尽日闲。殷勤谢红叶,好去到人间。”于祐和了两句:“曾闻叶上题红怨,叶上题诗寄阿谁?”放在上游,红叶随水又流入宫中。后于祐娶得宫中韩夫人为妻,谈及此事,其妻倍感惊异,原来当年题诗红叶的就是她。于是她又题了一首诗:“一联佳句随流水,十载幽思满素怀。今日却成鸾凤友,方知红叶是良媒。”}。庄周梦化蝶,吕望兆飞熊\footnote{吕望,即太公望,又称姜太公。传说周文王一夜梦见飞熊进帐,经人占卜,说是将得到贤人的吉兆。第二天出猎,果然遇到姜太公。}。北牖\footnote{北窗。牖,y\v{o}u,窗户。}当风停夏扇,南帘曝日\footnote{曝,pù,晒。曝日即晒太阳。}省\footnote{shěng}冬烘\footnote{原意是指人头脑不清,这里借来同上句的“夏扇”对仗,就是冬天的火炉的意思。}。鹤舞楼头,玉笛弄残仙子月\footnote{唐李白诗:“黄鹤楼头吹玉笛,江城五月落梅花。”《齐谐记》:“仙人子安曾驾鹤经过黄鹤楼。”楼旧址在武昌黄鹤矶上,为古时游览胜地。};凤翔台上,紫箫吹断美人风\footnote{《列仙传》载:秦穆公有女名弄玉,好道。时有人名萧史,善吹箫作鸾凤鸣。穆公把女嫁给萧史,并为他们筑了一所凤凰台。萧史教弄玉以箫吹凤鸣声,凤凰聚止其屋。一日,萧史乘龙,弄玉跨凤,双双升仙而去。}。
\end{yuanwen}

\chapter{东}

\begin{yuanwen}
晨对午,夏对冬。下饷\footnote{下午饭。这里指下午。xi\v{a}ng}对高舂\footnote{薄暮,傍晚。chōng}。青春\footnote{这里指春天。}对白昼,古柏对苍松。垂钓客\footnote{垂竿钓鱼的人。},荷\footnote{hè,担着,扛着。}锄翁。仙鹤对神龙。凤冠珠闪烁,螭带\footnote{雕有龙形的玉带。螭,chī,传说中一种没有角的龙。}玉玲珑。三元\footnote{封建科举考试,乡试第一称解元,会试第一称会元,殿试第一称状元,连续考得三个第一,就是所谓连中三元,三元及第。}及第才千顷\footnote{形容人才学之广。},一品\footnote{古代宰相为一品官爵。}当朝禄\footnote{古代官吏的薪俸。}万钟\footnote{古代称粮的容积单位,每钟盛六斛四斗,万钟极言其多。}。花萼 楼\footnote{花萼(è)楼全称花萼相辉楼,是唐代长安城中著名的建筑。唐玄宗和自己的弟兄常在此设宴饮酒。}间,仙李盘根调国脉\footnote{语出杜甫诗《冬日洛城北谒玄元皇帝庙》:“仙李盘根大,猗兰奕叶光。”。唐朝皇族姓李,杜甫用这句诗比喻皇族子孙繁衍,江山永固。调脉,本指中医诊脉治病。调国脉,是说治理国家,左右国家的命运。},沉香亭\footnote{沉香亭,唐禁苑中的一座亭台。}畔,娇杨\footnote{指杨贵妃。}擅宠\footnote{即专宠,排挤掉别人,使皇帝只对她一个人欢心。}起边风\footnote{唐明皇早年宠爱杨贵妃,日夜同她饮酒作乐,不理朝政。他曾命人在沉香亭旁遍植牡丹,花开时同杨妃到亭上饮酒赏花。后来,安禄山从渔阳起兵叛乱,唐王朝自此走上了下坡路。“起边风”即指安禄山的叛乱。}。
\end{yuanwen}

\begin{yuanwen}
清对淡,薄\footnote{b\'o,淡。}对浓。暮鼓对晨钟\footnote{本指寺院僧众撞钟击鼓,此指言论警策,发人深省。}。山茶对石菊,烟锁对云封。金菡萏\footnote{hàn dàn},玉芙蓉\footnote{菡萏、芙蓉:荷花的别称。}。绿绮\footnote{相传是汉末蔡邕的琴名。qǐ}对青锋\footnote{剑名。}。早汤\footnote{早上起来喝的醒酒汤。}先宿酒\footnote{隔夜仍使人醉而不醒的酒力。},晚食继朝饔\footnote{zh\=ao yōng,早饭。}。唐库金钱能化蝶\footnote{《杜阳杂编》里说:唐穆宗时,殿前种千叶牡丹,开放时香气袭人,穆宗夜宴,有无数黄白蝴蝶飞集花间,天明即飞去。人们张网捕捉数百,天明都变成了金玉,后来打开宝橱,发现皆库中金银所化。},延津宝剑会成龙\footnote{传说晋代张华和雷焕在丰城地下挖出一对极为珍贵的宝剑,每人拿了一把。后来雷焕的儿子佩着剑路过延平津的时候,宝剑忽然跃入水中,变成了一条龙潜水而去。}。巫峡浪传\footnote{犹如空传,意思是宋玉讲的神女不过是个寓言而已,并无其事。},云雨荒唐神女庙\footnote{宋玉《高唐赋》,说楚国先王曾游高唐之观,梦中见一神女,神女临行时说她是巫山之女,“旦为朝云,暮为行雨,朝朝暮暮,阳台之下”。王为立庙,号朝云庙。后人多以巫山神女故事歌咏爱情。};岱宗\footnote{即泰山,古人以它为群山之首,所以称它为宗。杜甫《望岳》诗:“岱宗夫如何?齐鲁青未了。”后半句也是从杜诗变化出来的。杜甫七律《望岳》的原句是:“西岳危棱竦处尊,诸峰罗立如儿孙。”不过这里描写的是西岳华山,而不是东岳泰山。}遥望,儿孙罗列丈人峰\footnote{山峰名。在泰山上,因形状像老人,所以称为丈人峰。}。
\end{yuanwen}

\begin{yuanwen}
繁对简,叠对重\footnote{ch\'ong}。意懒对心慵\footnote{yōng,懒。}。仙翁\footnote{称男性神仙,仙人。}对释伴\footnote{犹如说道侣,同修一道的伙伴。},道范\footnote{敬称他人的容颜,风范。道家的典范。}对儒宗\footnote{儒者的宗师。汉以后亦泛指为读书人所宗仰的学者。}。花灼灼\footnote{zhuó,耀眼,光明。},草茸茸\footnote{草初生的样子。}。浪蝶对狂蜂。数竿君子竹\footnote{古人认为,竹劲节虚心,有君子之德。},五树大夫松\footnote{《史记》记载,秦始皇登泰山,遇到暴风雨,躲在一棵松树下避雨,于是封为“五大夫”松。}。高皇\footnote{汉高祖刘邦。}灭项\footnote{项羽。}凭三杰\footnote{指西汉初期的张良、萧何、韩信。},虞帝承尧殛\footnote{jí}四凶\footnote{古史传说,唐尧年老时把帝位让给虞舜,舜即位后,流放了四个尧舜时代恶名昭彰的部族首领。}。内苑佳人,满地风光愁不尽 ;边关过客,连天烟草憾无穷。
\end{yuanwen}

\chapter{江}

\begin{yuanwen}
奇对偶,只对双。大海对长江。金盘对玉盏,宝烛对银釭\footnote{银白色的灯盏、烛台。釭,gāng。}。朱漆槛\footnote{ji\`an},碧纱窗。舞调对歌腔。兴汉推马武\footnote{马武是汉光武帝的将军,在建立东汉王朝的斗争中起过一定的作用。},谏夏著龙逄\footnote{龙逄即关龙逄,传说是夏桀王的大臣。他见夏桀无道,淫侈暴虐,曾强力谏争,结果被夏桀处死。逄,páng。}。四收列国群王伏\footnote{北宋初大将曹彬,他曾同潘美等将帅一道,伐灭了后蜀、南汉、南唐及北汉等五代时的地方割据政权,帮助宋太祖统一了天下。},三筑高城众敌降\footnote{初唐张仁愿,中宗朝人,曾统领朔方军与突厥族的侵扰进行斗争,使突厥不敢过山牧马。建了三座受降城以威镇北敌,从此边境安宁。}。跨凤登台,潇洒仙姬秦弄玉\footnote{弄玉故事,详见一东“凤翔”二句注。};斩蛇当道,英雄天子汉刘邦\footnote{《史记·高祖本纪》记载,刘邦初起,酒醉夜行,先行者报告说有长蛇拦路,刘邦上前杀死长蛇,路遂通。后有一老太婆在斩蛇处夜哭,人们询问,她说是自己的儿子是白帝子变化为蛇,被赤帝子杀害了。}。
\end{yuanwen}

\begin{yuanwen}
颜对貌,像对庞\footnote{面庞。}。步辇\footnote{古代皇帝乘坐的人力拉的车。辇,niǎn。}对徒杠\footnote{只能步行通过的桥。}。停针对搁杼\footnote{zhù,放下梭子,与停针可以成对。},意懒对心降\footnote{xiáng,安稳、平和。心降就是心里安稳、平和。}。灯闪闪,月幢幢\footnote{chuáng,朦胧的样子。}。揽辔\footnote{控制马匹缰绳。pèi}对飞舡\footnote{xiāng,船只。}。柳堤驰骏马 ,花院吠村尨\footnote{即村狗。尨,máng。}。酒量微酡\footnote{tuó,饮酒后脸红的样子。}琼\footnote{qióng,美玉。}杏颊\footnote{ji\'a},香尘没\footnote{mò}印玉莲双\footnote{晋石崇豪富骄奢,多蓄婢妾,布香尘于地,令诸姬行其上,以试鞋底之大小。玉莲,比喻女人的脚。}。诗写丹枫,韩女幽怀流御水\footnote{见一东韵“题红”注。};泪弹斑竹,舜妃遗憾积湡\footnote{yú}江\footnote{古代神话传说,帝舜的两个妃子娥皇和女英,居住在洞庭之山,舜南巡死于苍梧之野,二妃尽日啼哭,泪洒竹上,竹尽斑,这就是今天的湘妃竹。湡江,水名。}。
\end{yuanwen}

\chapter{支}

\begin{yuanwen}
泉对石,干对枝。吹竹对弹丝\footnote{弹奏琴瑟一类的乐器。}。山亭对水榭\footnote{水上架台,台上建屋,可供人游憩。},鹦鹉对鸬鹚\footnote{lú cí,一种善于捕鱼的水鸟。}。五色笔\footnote{相传南朝梁江淹,年轻时梦见晋代学者和诗人郭璞赠给他五色笔,于是才思大进,写了许多优秀诗文。晚年,又梦见郭璞讨回了五色笔,从此才情顿减,人称“江郎才尽”。后以五色笔比喻文才。},十香词\footnote{辽道宗后萧氏,小字观音,才貌双绝,后以谏猎见疏,作《同心词》自明。耶律乙辛诬后与伶人私通,假造《十香词》为证,帝竟赐后自尽。}。泼墨\footnote{泼墨是绘画术语,意思是大量用墨渲染。}对传卮\footnote{zhī,古代盛酒的器具。}。神奇韩幹\footnote{gàn}画\footnote{韩幹是唐代著名画家,善写人物,尤工于鞍马。传说建中初年,有人牵患有足疾的马就诊。其马毛色骨相似韩幹所画的马,为真马所无。遂牵此马绕市,巧遇韩幹,幹亦惊疑。返家后,视其所画马本,脚有一点黑缺,方知是马画通灵。},雄浑李陵诗\footnote{李陵,西汉名将李广之孙,武帝天汉二年,率步卒五千与匈奴十万骑决战,终因缺少援军,战败投降。李陵在匈奴遇到出使被扣留的苏武,后苏武南还,李陵设酒送别。其赠别苏武之诗雄浑豪爽,十分感人。}。几处花街新夺锦\footnote{唐武则天驾临龙门,诏令群臣赋“明堂火珠”诗,诗先成者赐锦袍。东方虬诗先成,拜锦未坐,宋之问亦成,但写得比东方虬好。武后令夺东方虬锦袍赏给宋之问,此即所谓夺锦。},有人香径淡凝脂。万里烽烟,战士边关争保塞;一犁膏雨\footnote{甘霖。},农夫村外尽乘时\footnote{利用有利时机。}。
\end{yuanwen}

\begin{yuanwen}
葅\footnote{zū,古代酷刑,将人剁成肉酱。}对醢\footnote{hǎi,肉酱。},赋对诗。点漆对描脂。璠簪\footnote{fán zān,美玉制成的簪。}对珠履\footnote{用珠装饰的鞋。相传战国时楚公子春申君,为了向人夸富,让他和门客都穿珠履。lǚ},剑客对琴师。沽酒价\footnote{西晋阮咸每以百钱挂杖头,至酒市沽酒。},买山资\footnote{晋僧人支道林到深公那里去买邱山,深公曰:“未闻巢(父)、(许)由买山而隐(巢父、许由,尧时隐士)。”}。国色对仙姿。晚霞明似锦,春雨细如丝。柳绊长堤千万树,花横野寺两三枝。紫盖黄旗,天象预占\footnote{zhān}江左地\footnote{三国末年吴主孙皓时,有术士说“庚子之年,紫盖黄旗,当入于洛”,孙皓以为平晋。不料相反,庚子之年恰恰是他被俘入洛阳的一年。};青袍白马,童谣终应\footnote{yìng}寿阳儿\footnote{相传南朝梁武帝时,先是大同中有童谣曰“青袍白马寿阳儿”。不久,寿阳的侯景发动叛乱,叛军中尽青袍白马,终于亡梁。}。
\end{yuanwen}

\begin{yuanwen}
箴\footnote{zhēn,古代一种以规劝、告诫为内容的文体。}对讚\footnote{zàn,通赞,颂扬﹑称美。},缶\footnote{fǒu}对卮\footnote{zhī}。萤照对蚕丝。轻裾\footnote{形容人在走动或舞蹈时衣襟飘扬的样子。jū}对长袖,瑞草\footnote{相传不常见的草,见则为祥兆,故称为瑞草。如蓂荚、灵芝之类。}对灵芝。流涕策\footnote{古时大臣们呈给皇上的谏议书。西汉贾谊在写给汉文帝的《治安策》中有“可为痛哭,可为流涕,可为长太息”之句,因称流涕策。},断肠诗\footnote{宋代女诗人朱淑贞,相传其对婚姻不满,故诗词多幽愤哀伤情调,后人辑有《断肠诗集》、《断肠词集》传世。}。喉舌\footnote{泛指说话的器官,比喻要害之地。}对腰肢\footnote{腰身,身段。}。云中\footnote{汉代北方有云中郡,在今山西北部及内蒙古一部分。}熊虎将\footnote{指西汉名将魏尚,相传他做云中守时,匈奴远避,不敢近边。},天上凤凰儿\footnote{汉民歌《陇西行》有“天上何所有?历历种白榆……凤鸣何啾啾,一母将九雏”的诗句。后来多用为赞美别人儿子的话。}。禹庙千年垂橘柚\footnote{语出杜甫诗《禹庙》:“禹庙空寺里,秋风落日斜。荒庭垂橘柚,古屋画龙蛇。”},尧阶三尺覆茅茨\footnote{古书记载,帝尧生活简朴,他的居室土阶三尺,茅茨不剪,采椽不斫。茨,cí,苫房。茅茨,用茅草苫房。}。湘竹含烟,腰下轻纱笼\footnote{lǒng}玳瑁\footnote{dài mào};海棠经雨,脸边青泪湿胭脂\footnote{轻纱笼罩着腰身,好象烟雾环绕着的竹枝;脸边流下泪水,犹如雨点滴在海棠花上。}。
\end{yuanwen}

\begin{yuanwen}
争对让,望对思\footnote{望可解作盼望,思解作思念,成对;望又可解作怨恨,思也可解作怨恨,也成对。}。野葛对山栀\footnote{zhī,植物名。夏开白花,有香气。果实椭圆,色黄,可入药,亦可做染料。或称为栀子。}。仙风\footnote{神仙的风致。形容人的潇洒。}对道骨\footnote{修道者的气质。},天造\footnote{自然生成,对人为而言。}对人为。专诸剑\footnote{专诸,古代勇士名。《左传》载,春秋时,吴公子光为夺取王位,收买专诸为刺客,把匕首藏在鱼腹中,借进献食品的机会刺死了吴王僚。},博浪椎\footnote{汉代的张良,为了给被灭掉的韩国报仇,从仓海君那里请到一位大力士,携带六十公斤的大铁椎,在博浪沙地方狙击秦始皇,误中副车,未果。}。经纬对干支\footnote{天干地支的简称,用来表示年、月、日的方法。}。位尊民物主,德重帝王师。望切不妨人去远,心忙无奈马行迟。金屋闭来,赋乞茂陵题柱笔\footnote{汉武帝幼时,他的姑母馆陶长公主打算把自己的女儿阿娇许给他,就问:“儿欲得妇,阿娇好否?”帝曰:“若得阿娇,当以金屋贮之。”陈阿娇与汉武帝结婚后,颇得宠爱。但陈皇后嫉妒心很强,因自己未育而嫉妒卫夫人,后遭贬独居长门宫,心情悲愤。她听说司马相如很会写文章,就奉黄金百两让相如为她写一篇《长门赋》,抒写她的孤独寂寞之感和对武帝的思念。司马相如曾居住在茂陵,故称他的才思为茂陵题柱笔。题柱,司马相如初西去长安,过成都升仙桥,题柱曰:“不乘高车驷马,不过此桥。”};玉楼成后,记须昌谷负囊词\footnote{唐诗人李贺家乡濒临昌谷川,因之他的诗集称《昌谷集》,后人也称他李昌谷。相传李贺出行,常让小童背一锦囊,每得佳句,就记下投入囊中。后梦神人曰:“上帝白玉楼成,命君作记。”不久诗人就死了。}。
\end{yuanwen}

\chapter{微}


\begin{yuanwen}
贤对圣,是对非。觉jué 奥ào 对参cān 微[1] 。鱼书对雁字[2] ,草舍shè 对柴扉。鸡晓唱,雉zhì 朝zhāo 飞[3] 。红瘦对绿肥[4] 。举杯邀月饮[5] ,骑马踏tà 花归。黄盖能成赤壁捷jié [6] ,陈平善解白登危[7] 。太白书堂,瀑pù 泉垂地三千丈[8] ;孔明祀庙miào ,老柏参cān 天四十围[9] 。
\end{yuanwen}

\begin{yuanwen}
戈对甲,幄\footnote{wò}对帏。荡荡对巍巍。严滩对邵圃,靖菊对夷薇\footnote{商代末年,孤竹君的两个儿子伯夷和叔齐在周文王处养老。文王死,武王起兵伐纣。伯夷和叔齐坚决反对,阻止不成,则隐居首阳山,采薇而食,意不餐周粟,终竟饿死。}。占鸿渐2] ,采凤飞[13] 。虎榜对龙旗。心中罗锦绣,口内吐珠玑。宽宏豁达高皇量[14] ,叱咤喑\footnote{y\=in}哑霸王威[15] 。灭项兴刘,狡兔尽时走狗死[16] ;连吴拒魏,貔貅\footnote{pí}屯处卧龙归[17] 。
\end{yuanwen}

\begin{yuanwen}
衰shuāi 对盛,密对稀。祭jì 服fú 对朝衣。鸡窗对雁塔tǎ [18] ,秋榜bǎng 对春闱[19] 。乌衣巷[20] ,燕子矶[21] 。久别对初归。天姿zī 真窈yǎo 窕tiǎo [22] ,圣德实光辉huī 。蟠pán 桃紫阙quē 来金母[23] ,岭荔红尘进玉妃[24] 。霸bà 王军营yíng ,亚父丹dān 心撞zhuàng 玉斗dǒu [25] ;长安ān 酒市,谪仙狂兴换huàn 银龟[26] 。
\end{yuanwen}


[1] 觉奥、参微:都是弄懂深奥微小的道理的意思,多用于教学或宗教方面。

[2] 鱼书:汉乐府《饮马长城窟行》:“客从远方来,遗我双鲤鱼。呼儿烹鲤鱼,中有尺素书。”因之后来称书信为鱼书。雁字:苏武出使匈奴被拘留。汉王朝向匈奴讨还苏武,匈奴推说苏武已死。苏武的随行人员给汉使者出个主意,让他对匈奴单于说:汉天子在上林苑射得一雁,雁脚上绑着苏武的信件,说明他在某某地方。匈奴只好放了苏武。由此后来书信也称雁书、雁字。

[3] 雉朝飞:乐府古题有《雉朝飞》。

[4] 红瘦对绿肥:语出宋李清照《如梦令》:“知否,知否?应是绿肥红瘦。”

[5] 举杯邀月饮:语出李白《月下独酌》:“花间一壶酒,独酌无相亲。举杯邀明月,对影成三人。”

[6] 黄盖能成赤壁捷:黄盖,孙权手下大将,以“苦肉计”诈降曹军,成就赤壁之火攻。

[7] 陈平善解白登危:汉高祖刘邦讨伐反叛韩王信,被匈奴困于白登,七天没有粮食,形势十分危急。据说靠陈平的奇计,方才解围。

[8] 太白书堂,瀑泉垂地三千丈:语出李白《望庐山瀑布》:“飞流直下三千尺,疑是银河落九天。”

[9] 孔明祀庙,老柏参天四十围:语出杜甫《古柏行》:“孔明庙前有老柏,柯如青铜根如石。霜皮溜雨四十围,黛色参天二千尺。”

[10] 严滩:即子陵滩。见东韵“垂钓客”注。邵圃:邵平,秦时为东陵侯。秦亡,种瓜于长安,瓜美,人称东陵瓜。

[11] 靖菊:晋诗人陶潜,性爱菊,“采菊东篱下,悠然见南山”是他的名句。陶死后,谥号为靖节先生,故称靖菊。


[12] 占鸿渐:《周易·渐》:“渐,女归吉。”爻辞中有“鸿渐于干”“鸿渐于磐”等话,意思是谁占得“鸿渐”一卦,嫁女是吉利的。

[13] 采凤飞:春秋时陈厉公太子陈完,逃亡到齐国,齐懿公打算把女儿许给他,占得一卦,其辞有“凤凰于飞,和鸣锵锵”的话,被认为是吉兆。后代以鸾凤比喻配偶,是这里出典。

[14] 宽宏豁达高皇量:史称刘邦宽宏豁达,心胸开阔。高皇指汉高祖刘邦。

[15] 叱咤喑哑霸王威:叱咤、喑哑都是形容人发怒的声音。楚霸王豪气盖世,所以说霸王威。

[16] 灭项兴刘,狡兔尽时走狗死:韩信帮助刘邦灭掉项羽,被封为楚王,有人告他谋反,刘邦逮捕了他,他说:“果若人言:狡兔死,走狗烹;飞鸟尽,良弓藏;敌国破,谋臣亡。天下已定,我固当烹。”走狗,春秋时越王勾践复国后,范蠡功成身退,留书给文种:“飞鸟尽,良弓藏;狡兔死,走狗烹。越王为人,长颈鸟喙,可与共患难,不可与共乐。子何不去?”文种后称病不上朝,然遭人谗言,言其意欲作乱,越王便赐剑给文种,文种自杀而亡。

[17] 貔貅:传说中的一种猛兽,这里借指勇猛的将士。卧龙:诸葛亮雄才大略,居南阳,时人送给他的雅号叫“卧龙先生”,后为蜀相。

[18] 鸡窗:晋宋处宗有一只极为宠爱的长鸣鸡,一直关在窗户边。后来鸡说人话,与处宗谈论,使处宗言谈技巧大增。后用于代指书房。雁塔:唐朝新科进士于皇帝赐宴后,须前往洛阳慈恩塔题写姓名。后比喻科举中试,金榜题名。

[19] 秋榜:秋试(乡试)后所发的榜。亦借指秋试。春闱:明、清会试都在春季,故名。

[20] 乌衣巷:六朝时金陵一个居住区,位于今南京市东南。东晋时王导、谢安等贵族多居此,故世称王谢子弟为乌衣郎。

[21] 燕子矶:地名。位于江苏省南京市北的观音山上。前临长江,形如飞燕,故名。

[22] 窈窕:形容女子摇曳多姿的样子。

[23] 蟠桃紫阙来金母:班固《汉武故事》说神人西王母来见汉武帝,拿出五个桃子,送给武帝两个,即所谓蟠桃。金母,即西王母,按五行学说,西方属金,故称金母。

[24] 岭荔红尘进玉妃:岭荔:史载唐代杨贵妃喜食荔枝,玄宗命人自岭南限七日快马送至长安。杜牧诗有“长安回望绣成堆,山顶千门次第开。一骑红尘妃子笑,无人知是荔枝来”的句子。

[25] 霸王军营,亚父丹心撞玉斗:在秦末农民大起义中,刘邦率兵攻入函谷关,占了秦都咸阳。项羽随后赶到,打算同刘邦决战。刘邦势小,只好到项羽驻军的鸿门去陪罪。项羽宴请刘邦,项羽的谋士范增几次示意杀害刘邦都没有成功。刘邦走后,范增把刘邦赠送的玉斗摔在地上,用剑击破,说:“竖子不足与谋也。”发泄他对项羽的不满。这就是有名的鸿门宴。亚父,范增年高望重,被项羽尊称为亚父。丹心,指范增对项羽的一片忠心。

[26] 长安酒市,谪仙狂兴换银龟:传说李白初到长安,拿出所作的《蜀道难》给当时的名诗人贺知章看,贺十分赞赏,称之为“谪仙”,于是解下金龟换酒,与之畅饮尽日。传说李白也曾以银龟换酒。这都表示诗人们的轻视富贵、狂放不羁。金龟、银龟,唐代官员们的佩饰,用以表示官职的级别。谪,封建时代特指贬官。





\chapter{鱼}

\begin{yuanwen}
羹gēng





[1] 对饭fàn ,柳对榆。短袖xiù 对长裾jū 。鸡冠对凤尾wěi ,芍sháo 药yào 对芙fú 蕖qú 。周有若ruò [2] ,汉hàn 相如。王屋对匡kuāng 庐lú 。月明山寺远,风细水亭虚xū 。壮zhuàng 士腰间三尺chǐ 剑[3] ,男儿腹内nèi 五车chē 书[4] 。疏影yǐng 暗àn 香,和靖孤gū 山梅蕊ruǐ 放fàng [5] ;轻阴清昼zhòu ,渊yuān 明旧宅zhái 柳条tiáo 舒[6] 。
\end{yuanwen}

\begin{yuanwen}
吾对汝rǔ ,尔ěr 对余。选xuǎn 授对升除[7] 。书箱对药yào 柜guì ,耒lěi 耜sì 对耰yōu 锄[8] 。参虽suī 鲁[9] ,回不愚[10] 。阀fá 阅对阎闾lǘ [11] 。诸侯hóu 千乘国[12] ,命mìng 妇七香车jū [13] 。穿云采药yào 闻仙子[14] ,踏tà 雪寻xún 梅策蹇jiǎn 驴lǘ [15] 。玉兔金乌,二气精灵为日月[16] ;洛龟河马,五行生克在图tú 书[17] 。
\end{yuanwen}

\begin{yuanwen}
欹qī [18] 对正,密对疏。囊náng 橐tuó 对苞bāo 苴jū [19] 。罗luó 浮fú 对壶hú 峤[20] ,水曲对山纡yū [21] 。骖cān 鹤hè 驾,待dài 鸾luán 舆yú [22] 。桀jié 溺nì 对长沮jū [23] 。搏虎卞biàn 庄子[24] ,当dǎng 熊冯féng 婕jié 妤yú [25] 。南阳高士吟梁父[26] ,西蜀才人赋子虚xū [27] 。三径风光,白石黄花供杖履lǚ [28] ;五湖hú 烟景jǐng ,青山绿水在樵渔[29] 。
\end{yuanwen}



[1] 羹:用肉、菜等芶芡煮成的浓汤。

[2] 周有若:有若,孔子弟子,貌似孔子。他是东周春秋时人,故称周有若。

[3] 壮士腰间三尺剑:史称汉高祖刘邦手提三尺剑起兵,因而后人常把三尺剑作为有志男儿的象征。

[4] 男儿腹内五车书:相传战国时学者惠施很有学问,“其书五车”,后来用以称人的博学。

[5] 疏影暗香,和靖孤山梅蕊放:宋林逋性恬淡好古,好作诗,隐居西湖孤山,终身不仕,不娶,以植梅养鹤为乐,世称梅妻鹤子。诗风淡远,多写隐居生活和淡泊心境,卒谥和靖先生。他写的《梅花》诗有“疏影横斜水清浅,暗香浮动月黄昏”的句子,一向为人称道。

[6] 轻阴清昼,渊明旧宅柳条舒:陶渊明写的《五柳先生传》,头几句是:“先生不知何许人也,亦不详其姓字,宅边有五柳树,因以为号焉。”他的诗写自己住宅的环境,有“方宅十余亩,草屋八九间;榆柳荫后檐,桃李罗堂前”的句子。

[7] 选授:量才授官。升除:即除去旧职就新职,由皇帝授予。

[8] 耒耜:翻土所用的农具。耰锄:用来平整田土或击碎土块的农具。

[9] 参虽鲁:参,曾参,孔子的弟子。孔子曾说:“柴也愚,参也鲁。”鲁,迟钝。

[10] 回不愚:回,颜回,孔子弟子颜渊的名。孔子说过:“吾与回言终日,不违,如愚。退而省其私,亦足以发,回也不愚。”

[11] 阀阅:古代官吏们的功劳、阅历。阎闾:大门楼,引申为高贵的社会地位。

[12] 诸侯千乘国:西周制度,诸侯国大者千乘。乘是战车的计量单位,一车四马叫一乘。

[13] 命妇七香车:受有封号的妇女称命妇。七香车,用多种香料涂抹的极为华贵的车。

[14] 穿云采药闻仙子:《幽明录》载,东汉时刘晨、阮肇,入天台山采药迷路,遇两仙女。

[15] 踏雪寻梅策蹇驴:策,马鞭,这里是赶着的意思。蹇驴,瘸驴。相传唐代诗人孟浩然曾骑骞驴于灞上踏雪寻梅,抒其幽兴。

[16] 玉兔金乌,二气精灵为日月:古代神话,说月中有玉兔捣药,日中有三只脚的乌鸦,因以玉兔代月,以金乌代日。古人又认为,宇宙中存在着相互斗争的阴阳二气,天地万物都是由它变化而成,日月则是二气的精华。

[17] 洛龟河马,五行生克在图书:传说伏羲时,黄河出龙马,背负图,称河图;夏禹治水时,神龟从洛水出现,背负书,称洛书。又说龟背上有九组不同点数组成的图画,禹因而排列其次第,乃成治理天下的九种大法,称为洛书。伏羲根据它们画成了八卦。汉孔安国谓河图即八卦。五行即金、木、水、火、土,古人认为它们是构成世界的五种元素。

[18] 欹:倾斜。

[19] 囊橐:盛物的袋子。大称囊,小称橐。或称有底面的叫囊,无底面的叫橐。苞苴:包裹。自上包之叫苞,自下垫之叫苴。

[20] 罗浮对壶峤:《初学记》云,罗浮二山随风雨而合离,壶桥二山逐波涛而下山。

[21] 山纡:山坳。

[22] 鹤驾、鸾舆:都是宗教传说中仙人所乘的车乘,由鹤和鸾凤驾着在空中飞行。骖,在这里是驾驶的意思。

[23] 桀溺、长沮:二人为春秋时隐士。也有人说,长和桀都是身材高大的样子,溺和沮都是污泥。长沮和桀溺就是两个身上沾泥的高个子,并不是人名。

[24] 搏虎卞庄子:卞庄子,鲁人,古代名勇士。传说他看到二虎争一牛,欲刺虎,管竖子劝说道:“两只老虎共食一牛,一定会因为肉味甘美而相互搏斗起来。两虎相斗,大者必伤,小者必死。到那时候您跟在受伤老虎的后面刺杀老虎,就能一举得到刺杀两头老虎的美名。”卞庄子听从劝告,一次刺死两只虎。故有搏双虎之名。

[25] 当熊冯婕妤:婕妤,古代宫廷中女官名。冯婕妤侍汉元帝观虎圈,有熊出,众惊走,冯独挡之,帝深嘉其勇也。

[26] 南阳高士吟梁父:诸葛亮原来隐居南阳,亲自种田,并且特别喜欢唱古曲《梁父吟》。

[27] 西蜀才人赋子虚:西蜀才人指司马相如,他写的《子虚赋》,受到汉武帝极大赞赏,叹不同时。

[28] 三径风光,白石黄花供杖履:语出陶渊明《归去来兮辞》:“三径就荒,松菊犹存。”

[29] 五湖烟景,青山绿水在樵渔:即太湖,古今著名风景区。





\chapter{虞}

\begin{yuanwen}
红对白,有对无。布谷对提tí 壶hú [1] 。毛máo 锥[2] 对羽扇,天阙què 对皇都。谢蝴hú 蝶dié [3] ,郑鹧zhè 鸪gū [4] 。蹈海hǎi 对归湖hú [5] 。花肥春雨润,竹瘦晚wǎn 风疏。麦饭fàn 豆dòu 麋终创chuàng 汉hàn [6] ,莼chún 羹gēng 鲈lú 脍kuài 竟归吴[7] 。琴调diào 轻弹,杨柳月中潜去qù 听;酒旗斜xié 挂guà ,杏花村里共gòng 来沽gū 。
\end{yuanwen}

\begin{yuanwen}
罗luó 对绮qǐ ,茗对蔬。柏秀xiù 对松枯kū 。中元对上巳sì [8] ,返璧对还huán 珠[9] 。云梦泽zé [10] ,洞庭湖hú 。玉烛对冰壶hú [11] 。苍头犀角jiǎo 带dài ,绿鬓bìn 象牙yá 梳。松阴白鹤hè 声相应yìng ,镜里青鸾luán 影yǐng 不孤gū [12] 。竹户hù 半开,对牖不知zhī 人在否fǒu ;柴门深闭,停车chē 还hái 有客来无。
\end{yuanwen}

\begin{yuanwen}
宾对主,婢对奴nú 。宝鸭yā 对金凫fú [13] 。升堂对入室[14] ,鼓瑟对投壶hú [15] 。觇chān 合璧,颂sòng 联珠[16] 。提tí 瓮wèng 对当垆lú [17] 。仰yǎng 高红日近[18] ,望远白云孤gū [19] 。歆向秘书窥kuī 二酉[20] ,机云芳fāng 誉动三吴[21] 。祖饯jiàn 三杯,老去qù 常斟花下酒;荒huāng 田tián 五亩,归来独荷hè 月中锄。
\end{yuanwen}

\begin{yuanwen}
君对父,魏对吴。北岳对西湖hú 。菜cài 蔬对茶荈chuǎn [22] ,苣jù 藤téng 对菖chāng 蒲pú [23] 。梅花数[24] ,竹叶符fú [25] 。廷议对山呼hū [26] 。两都班固gù 赋[27] ,八阵zhèn 孔明图tú [28] 。田tián 庆qìng 紫荆堂下茂mào [29] ,王裒póu 青柏墓前枯kū [30] 。出塞sài 中郎láng ,羝dī 有乳rǔ 时归汉hàn 室[31] ;质zhì 秦太子,马生角jiǎo 日返燕都[32] 。

\end{yuanwen}


[1] 提壶:鸟名。

[2] 毛锥:即毛笔。

[3] 谢蝴蝶:宋谢逸有蝴蝶诗百首,人呼为“谢蝴蝶”。

[4] 郑鹧鸪:唐郑谷写的《鹧鸪》诗,有“雨昏青草湖边过,花落黄陵庙里啼”一联,诗家许为最得神韵,所以被称为郑鹧鸪。

[5] 蹈海:战国时,秦兵围困赵都邯郸,魏王派客将军辛垣衍去劝说赵王,让他尊奉秦昭王为帝,秦兵自退。这事被围困在城中的齐国将士鲁仲连知道,当面批驳了辛垣衍的错误观点,说如果秦真的为帝,自己“有蹈东海而死耳,吾不忍为之民也”。归湖:春秋时范蠡帮助越王勾践灭吴后,功成身退,改名换姓,乘扁舟浮于五湖(即太湖)。

[6] 麦饭豆麋终创汉:汉光武帝刘秀初起兵,在饶阳地方遇到困难,将军冯异在滹沱河为他烧麦饭,在芜娄亭为他煮粥,使他度过难关,终于创立了东汉王朝。糜,粥。

[7] 莼羹鲈脍竟归吴:莼,莼菜,多年生水草,可做汤吃。莼羹:一种用野菜煮成的汤。鲈脍:鲈鱼切成的丝。晋时张翰,由于厌倦官场生活,见秋风起,思念起故乡吴地的莼羹、鲈鱼脍,当即弃官而去。

[8] 中元:农历七月十五日,道教以之为中元节。上巳:农历三月三日,古人称上巳节。

[9] 返璧:战国时,赵国有和氏璧,秦王托言以十五城易之,实际是强行索取。赵使蔺相如奉璧入秦,秦不给城,相如诈说璧有微瑕,请原璧归赵。还珠:相传古代合浦郡不产谷物,只有海中盛产珍珠。许多太守到任后尽力搜刮,宝珠竟然迁往它处。后孟尝君为合浦太守,清廉自奉,宝珠又回来了。

[10] 云梦泽:古代大泽名,在楚(今湖南洞庭湖一带),方九百里,后逐渐干涸,只剩下了洞庭湖。

[11] 冰壶:盛冰的玉壶。用以比喻人的清白,心地纯洁。

[12] 镜里青鸾影不孤:《异苑》载,罽(jì)宾国王买得一只鸾鸟,多年不鸣。夫人说:“听人说鸾鸟找到同类就鸣,何不让它照镜子试一试。”鸾鸟发现镜子里的影像,高声悲鸣,向天空奋力一飞,就死掉了。

[13] 宝鸭对金凫:金凫原为动物名,或称为野鸭。这里宝鸭和金凫都是指古代用来焚香的器具。

[14] 升堂对入室:古代居室建筑,室外有堂。一次孔子评价他的弟子子路,说:“由也,升堂矣,未入于室也。”意思是他已经有了一定的造诣。但还不够理想。

[15] 投壶:上古宴会时的一种游戏。宾主依次将矢投入壶中,多者为胜,少者罚饮。

[16] 觇合璧,颂联珠:古代迷信说法,日月合璧,五星联珠,是太平的征兆。觇,观测。

[17] 提瓮:汉人鲍宣的妻子桓少君喜欢打扮,鲍宣说:“这和我们的家境很不相称。”少君乃去服饰,著布衣,常提瓮出汲,并修妇道。瓮,瓦罐。当垆:卖酒。垆,放置酒器的土台,这里借指酒店。

[18] 仰高红日尽:史载晋元帝太子明帝幼时聪明,其父帝抱以临朝。恰逢有长安使者至,元帝问他:“日与长安孰近乎?”对曰:“长安近,不闻人从日边来。”次日日薄西山宴群臣,帝夸于众,明帝又以为日近。帝问其说,对曰:“举头见日(按:日指他的父亲晋元帝,这是古代崇拜皇帝的说法),不见长安。”众大奇之。

[19] 望远白云孤:狄仁杰客外忆亲曰:“白云飞处为亲所在。”

[20] 歆向秘书窥二酉:刘向、刘歆父子,都是西汉末年著名的学者,曾经多年整理皇家图书,对先秦典籍的整理、流传起了很大作用,刘歆继父业,整理六艺群书,编成《七略》。对经籍目录学有卓越贡献,为中国目录学之始。二酉,即大、小酉山,在湖南沅陵县西北。古代传说,秦时曾有人于此读书,留书千卷于山中。窥二酉,意思是读了许多古代的秘密藏书。

[21] 机云芳誉动三吴:陆机、陆云兄弟,都是西晋初年著名的文学家。吴亡后,与弟陆云至洛阳,为晋太常张华所器重,文名大噪,时称二陆。晋吴郡华亭(今江苏省松江县)人。三吴是二陆的家乡。

[22] 荈:粗茶。

[23] 苣藤:芝麻。菖蒲:植物名。习俗在端午节取叶插于檐下。

[24] 梅花数:古占法。相传为宋代邵雍所作。附会人事,以断吉凶。

[25] 竹叶符:即竹使符。汉代分与郡国守相的信符,右留京师,左留郡国。以竹箭五枚刻字制成。

[26] 廷议:古时在朝廷之上、皇帝面前论辩国事称廷议。山呼:《汉书·武帝纪》载,汉武帝登中岳嵩山,曾听到群山多次呼喊“万岁”。

[27] 两都班固赋:班固是东汉著名史学家、文学家,他曾写了《汉书》。《两都赋》是他辞赋中的代表作。

[28] 八阵孔明图:《三国志》载,孔明曾演八阵图,其遗址甚多,都在四川。八阵,古代作战阵法。

[29] 田庆紫荆堂下茂:《续齐谐记》载,京兆田真、田庆、田广三兄弟商议分居,准备把堂前一棵紫荆树也截为三段。第二天树就枯死了,兄弟大惊,说:树木同株,听说将分就死掉了,难道人还不如树吗?决定不再分居,紫荆树又活了。

[30] 王裒青柏墓前枯:王裒,晋人,其父被文帝杀死,裒攀墓柏号哭,柏忽枯。这是迷信说法。

[31] 出塞中郎,羝有乳时归汉室:中郎,指苏武。汉苏武以中郎将身份出使匈奴,被扣留,匈奴使牧羝羊,告诉他:“羝乳乃得归。”羝,公羊。乳,生羔。

[32] 质秦太子,马生角日返燕都:据《燕丹子》载,战国末年,燕太子丹为质于秦,秦国对他很无礼,于是思归故乡。向秦王恳请,秦王说:“乌鸦白头,马生角,一定放你回去。”太子丹仰天而叹,乌鸦果然白了头,低头落泪;马就生出了角。秦王不得不放他回来。后用以比喻极不可能实现的事情。





\chapter{齐}


鸾luán 对凤,犬对鸡。塞sài 北对关西。长生对益智zhì ,老幼对旄máo 倪[1] 。颁竹策[2] ,剪jiǎn 桐圭[3] 。剥pū 枣[4] 对蒸zhēng 梨lí 。绵mián 腰如弱ruò 柳,嫩nèn 手似柔róu 荑tí [5] 。狡jiǎo 兔能穿三穴隐[6] ,鹪jiāo 鹩liáo 权借jiè 一枝栖[7] 。甪lù 里先生,策杖垂绅扶少主[8] ;於wū 陵仲zhòng 子,辟lú 织履lǚ 赖lài 贤妻[9] 。

鸣对吠,泛fàn 对栖。燕语对莺yīng 啼tí 。珊瑚hú 对玛瑙nǎo ,琥珀pò 对玻璃lí 。绛县xiàn 老[10] ,伯州犁lí [11] 。测蠡lǐ 对燃rán 犀[12] 。榆槐堪kān 作荫yìn ,桃李自成蹊[13] 。投巫救女西门豹bào [14] ,赁lìn 浣huàn 逢féng 妻百里奚[15] 。阙què 里门墙qiáng ,陋lòu 巷规模mó 原不陋lòu [16] ;隋suí 堤dī 基址zhǐ ,迷楼踪zōng 迹jì 亦全迷[17] 。

越对赵,楚对齐。柳岸àn 对桃溪[18] 。纱窗对绣xiù 户hù [19] ,画阁对香闺[20] 。修月斧[21] ,上天梯tī 。蝃蝀dōng [22] 对虹霓。行乐游yóu 春圃pǔ [23] ,工谀yú 病夏畦xī [24] 。李广不封空射shè 虎[25] ,魏明得立为存麑ní [26] 。按àn 辔pèi [27] 徐xú 行,细柳[28] 功成劳王敬;闻声稍shāo 卧wò ,临泾[29] 名震zhèn 止zhǐ 儿啼tí 。



* * *



[1] 旄倪:老人和小孩。旄,通“耄”,老人。倪,小儿。

[2] 颁竹策:皇帝给诸侯王颁发的委任状,以竹简为之。

[3] 剪桐圭:圭,古代帝王诸侯举行礼仪时所用的玉器,上尖下方,代表官阶。相传周成王同他的小弟弟叔虞开玩笑,用桐叶剪成圭形,赠给他说,封你为侯。大臣进来贺喜,成王说:这是开玩笑。大臣说:天子无戏言。最后只好把叔虞封于唐。

[4] 剥枣:剥,同扑,打。

[5] 嫩手似柔荑:《诗经·卫风·硕人》写卫庄公夫人之美,说“手如柔荑,肤如凝脂”。荑:初生的茅芽,色白且柔嫩,用以比喻女子的手细白柔美。

[6] 狡兔能穿三穴隐:战国时,齐公子孟尝君出谋划策,谋求安稳的地位,说,狡兔有三窟,国君也应当如此。意思是多方采取措施,寻找几条出路。

[7] 鹪鹩权借一枝栖:鹪鹩,一种食小虫的极小的鸟,又名“巧妇鸟”。《庄子》上说:“鹪鹩栖树,不过一枝。”意思是容易满足。

[8] 甪里先生,策杖垂绅扶少主:汉初,商山有四个隐士,名东园公、绮里季、夏黄公、甪里先生,因为年老须发皆白,所以称四皓。相传高祖刘邦没能聘请他们出来,后高祖立吕后子惠帝为太子,继又欲以赵王如意易之。吕后用张良计,请四皓辅佐太子,帝见之曰“幸烦公等善为调护”,遂不见废。

[9] 於陵仲子,辟 织履赖贤妻:於陵仲子,即陈仲子,战国时齐国的隐士。因居于於陵,故号於陵子。《孟子》上记载他“身织屦,妻辟 ”。织屦即织草鞋。辟 ,原为剥麻,染麻。辟 指将分练过的麻搓成线。麻是古代纺织原料之一。 ,布缕,引申为织布。楚王欲以为相,不就,与妻逃去,为人灌园,妻子辟 织履。

[10] 绛县老:即绛县老人。《左传》记载,晋绛县一位老人,不知道自己究竟多大年纪,只知道出生那年初一是甲子日。人们去问师旷,师旷说,他已经七十三岁了。

[11] 伯州犁:春秋时晋国大夫伯宗之子伯嚭,因其父被杀,奔楚,为太宰。

[12] 测蠡:蠡,贝壳做的瓢。管窥天,蠡测海,喻见小也,自不量力。燃犀:烛照明察。相传燃烧犀角可以照妖,晋温峤路过渚矶,人们说水下有怪物,温峤用点燃的犀角照之,果然见到许多奇形异状的精灵。夜梦人曰:“幽明道别,何苦相逼。”这是迷信传说。后比喻洞察事理或奸邪。

[13] 桃李自成蹊:《史记·李将军传赞》:“谚曰:‘桃李不言,下自成蹊。’此言虽小,可以喻大也。”比喻一个人如果有高德美才,不用自我声张,自然得到人们的敬爱。蹊,小路。

[14] 投巫救女西门豹:战国魏文侯时,邺地三老、廷掾,与巫祝勾结,假托河伯欲娶妻,每年强选少女,投入河中,愚弄人民并榨取钱财。后西门豹为邺令,在河伯娶妇时,托言所选女子不美,要巫祝、三老去与河伯商量,另行选送,便将其投入河中,因而制止了利用迷信虐害人民的恶行。

[15] 赁浣逢妻百里奚:赁,本意为租借,这里指雇用。浣,洗。《风俗通》载,春秋时百里奚为秦相,赁一浣妇,歌曰:“百里奚,五羊皮,忆别时,烹伏雌,舂黄 ,烦扊扅,今日富贵忘我为?”问她是谁,原来是被百里奚抛弃在故乡的妻子。

[16] 阙里门墙,陋巷规模原不陋:阙里,孔子居住的里巷名。陋巷,孔子弟子颜渊所居,狭小的巷子。引申为狭窄简陋的住处。孔子曾夸奖颜渊:“一箪食,一瓢饮,在陋巷。人不堪其忧,回也不改其乐。”后来唐刘禹锡作《陋室铭》说:“君子居之,何陋之有?”意思是,只要有德者居住,陋巷也不简陋。

[17] 隋堤基址,迷楼踪迹亦全迷:隋炀帝为游江都,开凿了大运河,在两岸栽种杨柳,堤长一千三百余里,称隋堤。迷楼,传说也是隋炀帝所建,用以寻欢作乐的地方。两句的意思是:隋堤也好,迷宫也罢,都成了历史的残迹,当年的迷宫如今真的迷失荒草中了。

[18] 桃溪:指桃源。

[19] 纱窗:蒙纱的窗户。绣户:雕绘华美的门户。多指妇女居室。

[20] 画阁:彩绘华丽的楼阁。香闺:指青年女子的内室。

[21] 修月斧:传说唐代有人登嵩山,看见有人卧在道旁,问他为什么在道旁酣睡。那人回答说:“月亮由七宝合成,要由八万二千户人经常修理,我是其中的一个。”说着拿出身边的斧凿。

[22] 蝃蝀:古时称虹为蝃蝀。

[23] 春圃:春日的园圃。

[24] 夏畦:于炎夏中耕田,比喻勤苦工作。

[25] 李广不封空射虎:《史记·李将军传》:西汉李广守北平,出猎,见草中石以为虎,射之,箭没石中,以为奇。李广一生战功卓著,却不得封侯。

[26] 魏明得立为存麑:魏明帝曹叡小时候随父射猎,文帝射死母鹿,让明帝去射小鹿。明帝不肯,说:“陛下已杀其母,臣不忍复杀其子。”同时流下了眼泪。文帝于是决心让他继承王位。

[27] 按辔:勒住马。

[28] 细柳:汉代周亚夫为将军时,屯兵于细柳,军纪森严,天子欲入军营,亦须依军令行事。

[29] 临泾:西汉赤玼守原州,虏不过临泾,人常道其名以吓唬小儿,使之不敢啼哭。





\chapter{佳}


门对户hù ,陌mò 对街jiē 。枝叶对根gēn 荄gāi [1] 。斗dòu 鸡对挥huī 麈zhǔ [2] ,凤髻jì 对鸾luán 钗chāi [3] 。登楚岫xiù [4] ,渡秦淮[5] 。子犯fàn 对夫差chāi [6] 。石鼎dǐng 龙头缩suō [7] ,银筝zhēng 雁翅排[8] 。百年诗礼延余庆qìng [9] ,万里风云入壮zhuàng 怀[10] 。能辨明伦lún ,死矣野哉zāi 悲季路[11] ;不由yóu 径窦dòu ,生乎hū 愚也有高柴[12] 。

冠对履lǚ ,袜wà 对鞋xié 。海hǎi 角jiǎo 对天涯yá 。鸡人对虎旅lǚ [13] ,六市对三街jiē [14] 。陈俎zǔ 豆dòu ,戏堆duī 埋mái [15] 。皎jiǎo 皎jiǎo 对皑ái 皑ái [16] 。贤相聚东阁[17] ,良朋péng 集jí 小斋zhāi 。梦里山川书越绝jué [18] ,枕边风月记齐谐xié [19] 。三径萧xiāo 疏,彭péng 泽zé 高风怡五柳;六朝华贵guì ,琅láng 琊yá 佳气种zhòng 三槐[20] 。

勤对俭jiǎn ,巧对乖guāi 。水榭对山斋zhāi [21] 。冰桃对雪藕ǒu ,漏lòu 箭[22] 对更gēng 牌。寒翠袖xiù [23] ,贵guì 荆钗chāi [24] 。慷kāng 慨kǎi 对诙huī 谐xié 。竹径风声籁lài [25] ,花溪月影yǐng 筛shāi [26] 。携xié 囊náng [27] 佳韵yùn 随suí 时贮zhù ,荷hè 锄[28] 沉酣hān 到处埋mái 。江海hǎi 孤gū 踪zōng ,雪浪làng 风涛tāo 惊旅lǚ 梦;乡关[29] 万里,烟峦luán 云树切qiè 归怀。

杞qǐ 对梓,桧guì 对楷jiē 。水泊pō [30] 对山崖yá 。舞裙对歌袖xiù ,玉陛bì 对瑶阶jiē [31] 。风入袂mèi ,月盈yíng 怀。虎兕sì [32] 对狼láng 豺。马融堂上帐[33] ,羊侃kǎn 水中斋zhāi [34] 。北面黉hóng 宫宜拾芥jiè [35] ,东巡xún 岱dài 畤zhì 定燔fán 柴[36] 。锦jǐn 缆春江,横笛dí 洞箫xiāo 通碧落[37] ;华灯夜月,遗簪zān 堕duò 翠遍香街jiē [38] 。


[1] 根荄:植物的根。斗鸡:古时让鸡与鸡相搏斗的一种游戏。

[2] 挥麈:晋代人们清谈时,常挥麈以为谈助,后称谈论为挥麈。麈,古书上指鹿一类的动物,其尾可做拂尘,即“麈尾”。

[3] 鸾钗:鸾形的钗子。

[4] 楚岫:楚地山峦。

[5] 秦淮:河名。流经南京,是南京市名胜之一。

[6] 子犯:即狐偃,字子犯,春秋晋人。为晋文公舅,故亦称为舅犯。夫差:差,为压韵可读chā。春秋时的吴王,因父阖闾为越王勾践所败,故败困勾践于会稽,以报父仇,并率精兵北会诸侯于黄池,与晋争霸,勾践乘虚而入,遂灭吴,夫差自刭而死,在位二十三年。

[7] 石鼎:陶制的烹茶用具。龙头:当指石鼎上的龙头形装饰。

[8] 银筝:用银装饰的筝或用银字表示音调高低的筝。雁翅:当指古筝上的琴码。

[9] 诗礼:旧时常用来称读书讲究礼教的人家。余庆:指留给子孙后辈的德泽。

[10] 壮怀:豪壮的胸怀。

[11] 季路:姓仲,名由,字子路,一字季路。孔子弟子,性好勇、事亲孝。

[12] 高柴:孔子门人。遇卫难不径不窦(既不走小路,又不走孔道,不知变通)。

[13] 鸡人:职官名。于天将亮时,报时以警醒百官。虎旅:勇猛善战的军队。

[14] 六市、三街:街市。亦作三街六巷。

[15] 陈俎豆,戏堆埋:《列女传·母仪》载,孟子幼时,居近墓,习堆埋;移舍于市,又习贸易事;移学宫旁,乃习礼让,修俎豆。修俎豆,主持祭祀之礼。俎豆,古代祭祀、宴飨时,用来盛祭品的两种礼器。亦泛指各种礼器。

[16] 皎皎、皑皑:洁白的样子。

[17] 东阁:东向的小门。

[18] 越绝:《越绝书》,历史小说。记载春秋末年与战国初期吴越争霸的历史故事。

[19] 齐谐:《齐谐》,志怪书名。

[20] 三槐:宋代兵部侍郎王佑,多阴德,手植三槐于庭,自言子孙必有为三公的。其子旦后果为相,世称为三槐王氏,子孙因建三槐堂。

[21] 山斋:山中居室。

[22] 漏箭:古代漏壶中用作计时指针的箭。

[23] 翠袖:青绿色衣袖。泛指女子的装束。

[24] 荆钗:用荆木做的发钗。代指与丈夫同甘共苦的贤惠的妻子。

[25] 竹径:竹林中的小径。籁:本指从孔窍中所发出的声音,后泛指一切的声音。

[26] 筛:洒、落。

[27] 携囊:李贺系囊贮诗。

[28] 荷锄:晋人刘伶,好酒。荷锄自随曰:“醉死便可埋我。”

[29] 乡关:故乡。

[30] 水泊:湖泽。

[31] 玉陛:帝王宫殿的台阶。瑶阶:玉砌的台阶。亦用为石阶的美称。

[32] 虎兕:虎与犀牛。比喻凶恶残暴的人。

[33] 马融堂上帐:马融字季长,茂陵(今陕西省兴平县东北)人,东汉学者。从学者常千数,注《孝经》、《论语》、《诗》、《易》、《尚书》三《礼》等。马融堂前教授生徒,后设绛纱帐,置女乐。

[34] 羊侃水中斋:南朝梁羊侃,好奢侈,结舟为斋,亭馆皆备,日事游宴。

[35] 黉宫:古代学校名。拾芥:捡取地上的草芥。比喻取之极易。

[36] 岱:泰山。畤:古代祭天地五帝之处。燔柴:烧柴,祭天之礼。

[37] 锦缆:锦制的精美的缆绳。碧落:天空。

[38] 华灯:雕饰华美而光辉灿烂的灯。遗簪:指失落的簪子。香街:指繁华的街道。





\chapter{灰}


春对夏,喜xǐ 对哀āi 。大手对长才[1] 。风清[2] 对月朗lǎng ,地阔kuò 对天开。游yóu 阆làng 苑yuàn [3] ,醉蓬péng 莱[4] 。七政对三台[5] 。青龙壶hú 老杖[6] ,白燕玉人钗chāi [7] 。香风十里望仙阁[8] ,明月一天思子台[9] 。玉橘jú 冰桃[10] ,王母几因求道降;莲舟藜lí 杖[11] ,真人原为读书来。

朝zhāo 对暮,去qù 对来。庶矣对康kāng 哉zāi [12] 。马肝对鸡肋[13] ,杏眼yǎn 对桃腮sāi 。佳兴适,好怀开。朔shuò 雪[14] 对春雷léi 。云移鹊què 观guàn [15] ,日晒shài 凤凰台[16] 。河边淑气[17] 迎yíng 芳fāng 草,林下轻风待dài 落梅[18] 。柳媚mèi 花明,燕语莺yīng 声浑是笑;松号háo 柏舞,猿啼tí 鹤hè 唳总zǒng 成哀āi 。

忠对信xìn ,博对赅gāi 。忖cǔn 度duó 对疑猜cāi [19] 。香消xiāo 对烛暗àn [20] ,鹊què 喜xǐ 对蛩qióng [21] 哀āi 。金花报bào [22] ,玉镜台[23] 。倒斝jiǎ 对衔杯[24] 。岩巅diān 横老树,石磴dèng [25] 覆苍苔。雪满mǎn 山中高士卧wò [26] ,月明林下美měi 人来[27] 。绿柳沿堤dī ,皆jiē 因苏子[28] 来时种zhòng ;碧桃满mǎn 观guàn ,尽是刘郎láng [29] 去qù 后hòu 栽zāi 。



* * *



[1] 大手:犹高手。指工于文辞的名家。长才:优异的才能。

[2] 风清:风轻柔而凉爽。

[3] 阆苑:阆风之苑,神话传说的仙人居地。

[4] 蓬莱:神话传说中的海上仙山之一。

[5] 七政:日、月和金、木、水、火、土五星。三台:古有灵台、时台、囿台,合称三台。

[6] 青龙壶老杖:《后汉书·费长房传》载,东汉费长房从壶公学仙,辞归,壶公给他一竹杖,说:骑之可以到家,长房到家后把杖投入葛陂,杖化为龙。

[7] 白燕玉人钗:汉武帝升平元年,建招灵阁,有女神留玉钗与帝,后化为玉燕升天。

[8] 望仙阁:南朝陈后主建。

[9] 思子台:汉武帝逼死了被诬陷的太子刘据,后来帝知其冤,作思子台。

[10] 玉橘冰桃:《汉武外传》载,王母降汉武宫中,享帝以玉橘、冰桃、雪藕。

[11] 莲舟藜杖:传说太乙真人坐莲舟,燃藜杖,降天禄阁,照刘向读书。

[12] 康哉:《尚书·益稷》:“﹝皋陶﹞乃赓载歌曰:‘元首明哉,股肱良哉,庶事康哉。’”歌词称颂君明臣良,诸事安宁。后遂以“康哉”为歌颂太平之词。

[13] 马肝:马肝味劣,比喻卑微琐碎的事。鸡肋:与鸡的肋骨一样无味。比喻没有味道或少有实惠。

[14] 朔雪:北方的雪。

[15] 鹊观:古代道观名。 鹊:鸟纲雀形目鸣禽类。

[16] 凤凰台:在江苏南京市。

[17] 淑气:温和怡人的气息。

[18] 落梅:汉应劭《风俗通》:五月有落梅风,江淮以为信风。

[19] 忖度:思量、考虑。疑猜:猜疑。古典诗词戏曲中为和韵脚常将一个词中的两个字颠倒使用。

[20] 香消:比喻女子死去。烛暗:人死去的通称。

[21] 蛩:蟋蟀的别名。

[22] 金花报:古代状元及第时寄家信报喜,称为金花报。

[23] 玉镜台:温峤娶其姑之女,以玉镜台为聘。

[24] 斝:古代青铜制的酒器,圆口,三足。衔杯:口含酒杯。多指饮酒。

[25] 石磴:以石头铺砌成的台阶。

[26] 高士卧:《后汉书·袁安传》载,袁安遇雪天在家高卧不出,人以为贤,举为孝廉。

[27] 美人来:隋赵师雄游罗浮山,日暮见一美人邀共饮,雄不觉醉卧。醒来在梅花树下,翠羽嘈唧其上,月落参横,惆怅不已。

[28] 苏子:苏轼守杭州,令西湖沿堤种桃柳,人号苏公堤,简称苏堤。

[29] 刘郎:语出刘禹锡《元和十一年自朗州召至京,戏赠看花诸君子》诗:“紫陌红尘拂面来,无人不道看花回。玄都观里桃千树,尽是刘郎去后栽。”





\chapter{真}


莲对菊jú ,凤对麟。浊zhuó 富对清贫pín [1] 。渔庄对佛fó 舍shè [2] ,松盖[3] 对花茵。萝luó 月叟sǒu [4] ,葛gě 天民[5] 。国宝对家珍[6] 。草迎yíng 金埒liè [7] 马,花醉玉楼[8] 人。巢燕三春尝唤huàn 友[9] ,塞sài 鸿八月始来宾[10] 。古往今来,谁见泰山曾作砺[11] ;天长地久,人传沧海hǎi 几扬尘[12] 。

兄xiōng 对弟,吏对民。父子对君臣。勾丁对甫甲[13] ,赴卯mǎo 对同寅[14] 。折桂guì 客[15] ,簪zān 花人[16] 。四皓hào 对三仁[17] 。王乔云外舃xì [18] ,郭guō 泰雨中巾[19] 。人交好友求三益[20] ,士有贤妻备bèi 五伦lún [21] 。文教南宣xuān ,武帝平蛮mán 开百越[22] ;义旗西指zhǐ ,韩侯hóu 扶汉卷三秦[23] 。

申对午,侃kǎn 对訚yín [24] 。阿ē 魏对茵陈[25] 。楚兰对湘芷zhǐ [26] ,碧柳对青筠[27] 。花馥fù 馥,叶蓁zhēn 蓁[28] 。粉fěn 颈jǐng 对朱唇chún 。曹公奸似鬼[29] ,尧帝智zhì 如神[30] 。南阮ruǎn 才郎láng 差chā 北富[31] ,东邻丑女效西颦pín [32] 。色艳北堂,草号hào 忘忧[33] 忧甚shèn 事?香浓nóng 南国,花名含笑[34] 笑何人?



* * *



[1] 浊富:不义而富。与“清贫”相对。清贫:生活清寒贫苦。

[2] 渔庄:渔村。佛舍:寺院房舍,佛堂。

[3] 松盖:谓乔松枝叶茂密,状如伞盖。

[4] 萝月叟:月下走在藤萝盘绕的山路上的老人。萝月,萝藤间的月色。

[5] 葛天民:传说中的上古帝王,其治世不言而信,不化而行,是远古社会理想化的政治领袖人物。古人认为是理想中的自然、淳朴之世。

[6] 家珍:家中的珍贵物品。

[7] 金埒:埒即勒,马具。

[8] 玉楼:华丽的楼。

[9] 巢燕三春尝唤友:语出《诗经·小雅·伐木》:“伐木丁丁,鸟鸣嘤嘤,出自幽谷,迁于乔木。嘤其鸣矣,求其友声。”

[10] 塞鸿八月始来宾:塞北的鸿雁直到八月才会飞到南方去做客。称之为宾,因为塞北才是雁的家乡,经过中原好象客人一样。

[11] 泰山曾作砺:汉代封功臣、皇帝封爵的誓词有“黄河如带,泰山若砺。国以永宁,爰及苗裔”的话,意思是遥远无期,不可能出现的情况。砺,磨刀石。

[12] 沧海几扬尘:犹言沧海桑田。《神仙传》载,仙人麻姑在蔡经家见到王远,说自己曾见东海三为桑田,目前东海水又浅,大约要变成陆地。王远叹息说:圣人都说海中将要扬起尘土了。

[13] 勾丁:即征兵。甫甲:即补甲,补充兵员。

[14] 赴卯:古代官府把检查出勤情况叫做点卯(因为卯时日出,开始工作),赴卯犹如今天说上班。同寅:同僚。

[15] 折桂客:晋都诜举贤,对策最优,自己夸口说:“犹桂林之一枝,昆山之片玉。”后因以考试得中为折桂。

[16] 簪花人:古代殿试得中,则赏令簪花,以显其荣。

[17] 四皓:商山四皓的简称,汉初商山的四个隐士。三仁:殷商末年,有微子、箕子、比干三个贤人。三人劝谏纣王,不被采纳,纣王的庶兄微子逃往国外,叔父箕子装疯做奴隶,比干因进谏而被杀,俱以仁德见称于世。孔子评价他们说“殷有三仁”。

[18] 王乔云外舃:《后汉书》载,汉人王乔做叶县县令,有神术,每月两次朝见皇帝。皇帝对他来去这么迅速感动惊异,叫人暗地观察。有人报告,王乔每次来朝,只见有一对凫雁飞来。人们用网捕捉这双飞雁,却只捉得了一只鞋。舃,鞋。

[19] 郭泰雨中巾:汉代郭泰是个有名望的人物,一次遇雨,头巾折起一角,人们以为他是有意这样做的,很雅观,于是效之,故意把头巾折起一角,称为“宗林(郭泰字)巾”。

[20] 三益:语出《论语·季氏》:孔子曰:“益者三友,损者三友。友直、友谅、友多闻,益矣。”三益指直、谅、多闻。

[21] 五伦:古代指君臣﹑父子、兄弟﹑夫妻﹑朋友之间的五种伦理体系。

[22] 文教南宣,武帝平蛮开百越:汉武帝时,统一南方百越之地,议立南海、苍梧等九郡。文教,文明、教化。南宣,推广到南方。百越,古代散居南方各地越族的总称,居住两广、海南岛一带。如汉时有闽越、瓯越、南越、骆越等。其文化特征为断发、纹身、契臂、巢居、使舟及铸铜鼓等。亦作百粤。

[23] 义旗西指,韩侯扶汉卷三秦:在刘邦和项羽争夺天下的斗争中,韩信作为刘邦的将领,曾南北转战,立下了很大功劳。在他刚刚被举用的时候,曾劝说刘邦,略定三秦。刘邦听从他的意见,尽得关中之地,为楚汉之争的胜利打下了基础。韩侯,即韩信。三秦,战国时秦的国土,在今陕西。秦亡后,项羽把关中地分为三份,封秦降将章邯为雍王于咸阳以西,司马欣为塞王于咸阳以东,董翳为翟王于上郡,合称为三秦。

[24] 侃:和乐的样子。訚:态度庄重的样子。

[25] 阿魏、茵陈:两味中药名。

[26] 兰、芷:都是香草,产在古代楚国。湘江在楚国境内,因称芷为湘芷。屈原的诗歌中经常提到这两种香草,用它比喻品行高洁的人物。

[27] 筠:竹。

[28] 蓁蓁:茂盛的样子。

[29] 曹公奸似鬼:三国时曹操奸伪,人称奸鬼。

[30] 尧帝智如神:《史记》上说,帝尧十分聪明,“其智如神”。

[31] 南阮才郎差北富:晋洛阳阮氏家族中的阮籍和阮咸叔侄居道南,家贫而多才;其他阮姓宗族居道北,家富。七月七日,北阮晒衣服,光彩夺目。阮咸也以竹杆把大布裤衩挑了出来。人问其故,他说:“未能免俗,聊复尔耳。”

[32] 东邻丑女效西颦:《庄子》里的一则寓言说,美女西施因胸口痛,经常抚胸口皱眉。东邻丑女也学西施的样子,在人前故意卖弄,却引得人们更加讨厌她。颦,皱眉。

[33] 忘忧:萱草也名忘忧草。

[34] 含笑:花名。





\chapter{文}


忧对喜,戚对欣。二典对三坟fén [1] 。佛fó 经对仙语,夏耨nòu 对春耘[2] 。烹pēng 早韭,剪jiǎn 春芹。暮雨对朝zhāo 云[3] 。竹间斜xié 白接jiē [4] ,花下醉红裙。掌zhǎng 握wò 灵符五岳箓lù [5] ,腰悬xuán 宝剑七星纹[6] 。金锁suǒ 未开,上相趋听宫漏lòu [7] 永;珠帘半卷,群僚liáo 仰yǎng 对御炉lú [8] 薰xūn 。

词对赋,懒对勤。类聚对群分[9] 。鸾luán 箫xiāo 对凤笛dí ,带dài 草对香芸[10] 。燕许xǔ 笔[11] ,韩柳文[12] 。旧话对新闻。赫hè 赫hè 周南仲zhòng [13] ,翩翩晋右军[14] 。六国说shuì 成苏子贵guì [15] ,两京收shōu 复郭guō 公勋xūn [16] 。汉阙què 陈书,侃kǎn 侃kǎn 忠言推tuī 贾谊[17] ;唐廷对策,岩岩直zhí 谏jiàn 有刘fén [18] 。

言对笑,绩对勋xūn 。鹿豕shǐ 对羊fén [19] 。星冠对月扇[20] ,把袂mèi 对书裙[21] 。汤tāng 事葛gě [22] ,说yuè 兴殷[23] 。萝luó 月对松云。西池chí 青鸟niǎo 使[24] ,北塞sài 黑hēi 鸦yā 军[25] 。文武成康kāng 为一代[26] ,魏吴蜀汉定三分[27] 。桂guì 苑yuàn 秋宵xiāo ,明月三杯邀曲客[28] ;松亭夏日,薰xūn 风一曲奏zòu 桐君[29] 。



* * *



[1] 二典对三坟:二典指《尚书》中的《尧典》《舜典》两篇。三坟,指三皇伏羲、神农、黄帝之坟,亦指三皇所著之书。此与二典相对,当指三皇所著之书。

[2] 耨:古代锄草的器具。这里当为动词,意为“锄草”,与“耘”相对。耘:锄草。

[3] 暮雨对朝云:据传楚襄王和宋玉一起游览云梦台时,宋玉对楚襄王说:“以前先王,也就是楚怀王曾经游览此地,玩累了便睡着了,梦见一位美丽动人的女子,她说是巫山之女,愿意献出自己的枕头席子给楚王享用。楚王知道弦外有音,非常高兴,立即宠幸那位巫山美女。巫山女临别之时告诉楚怀王:“妾在巫山之阳,高丘之阻。旦为朝云,暮为行雨,朝朝暮暮,阳台之下。”

[4] 竹间斜白接:晋山简为人狂放,做襄阳太守时,经常骑马出游,衣冠颠倒。当时有首民谣说:“山公时一醉,迳造高阳池。日暮倒载归,酩酊无所知。复能乘骏马,倒着白接篱。”白接,即白接篱,当时一种帽子。

[5] 掌握灵符五岳箓:道教传说,修炼到一定程度的道士,可以握三山五岳灵符,统领鬼神。箓,道士画的驱避邪魔的符号、帖子。

[6] 七星纹:宝剑上嵌饰的北斗图案。

[7] 宫漏:即铜壶滴漏,古代宫中计时的用具。

[8] 御炉:御用的香炉。

[9] 类聚对群分:《周易·系辞上》:“方以类聚,物以群分。”

[10] 带草对香芸:相传东汉末年郑康成曾在不其城东南山中教授,所居山下生一种草,叶长尺余,十分坚韧,人们叫它作“康成书带”。香芸,芸香一类的香草,俗称七里香。有特异香气,能去蚤虱,辟蠹奇验,古来藏书家多用以防蠹。

[11] 燕许笔:唐张说封为燕国公,苏颋(tǐng)封为许国公,二人以文章名世,时人称大手笔。

[12] 韩柳文:唐柳宗元、韩愈,文章绝代。

[13] 赫赫周南仲:南仲是周宣王时的大将,他曾率兵击败侵犯周国的少数民族玁狁。

[14] 翩翩:风流潇洒的样子。晋右军:即晋王羲之,著名书法家。他曾做过右军将军,所以人们称他为王右军。

[15] 六国说成苏子贵:战国时,苏秦以合纵术说服了山东六国诸侯,佩六国相印,为总约长。

[16] 两京收复郭公勋:唐郭子仪率兵平息“安史之乱”,收复了长安、洛阳两京,后以功封为汾阳王。

[17] 汉阙陈书,侃侃忠言推贾谊:西汉贾谊是个卓有远见的政治家,他曾上疏汉文帝,直切地指出汉王朝的危机,建议及早采取措施补救。侃侃,形容说话理直气壮,不慌不忙。

[18] 唐廷对策,岩岩直谏有刘 :唐文宗二年,举贤良方正百余人,在皇帝面前对策。进士刘 慷慨直言,切中时弊。但由于考官惧怕宦官的势力,不敢录取。同时对策的河南府参军李邰上疏,宁可把自己的官职让给刘 。后来因宦官的陷害,刘 终竟被贬死。刘 获得了许多正直的知识分子的同情,例如诗人李商隐就有《哭刘 》诗。岩岩,威严。

[19] 鹿豕:鹿和猪。比喻山野无知之物。羊 :相传春秋时鲁大夫季康子掘井,挖到一只瓦缸,里面有一只羊,问孔子,孔子说它是土之怪,叫 羊。

[20] 星冠:道士的帽子。月扇:团扇。形如满月,故称。

[21] 把袂:比喻把臂或握手。袂,衣袖。书裙:晋羊欣年十三,王羲之爱其才。昼卧,王羲之书其白练裙,羊欣视为珍宝,揣摩学习,因此书法遂大进。后以书裙称誉别人的书法,或指文人间的相互雅赏爱慕。

[22] 汤事葛:语出《孟子》。汤,成汤,商朝的第一个王。葛,汤时小国。传说葛伯不祀鬼神,汤曾帮助他祭祀。

[23] 说兴殷:说,傅说,商代人。传说他是奴隶,为人筑墙,后来商王武丁发现了他的才干,举以为三公。

[24] 西池青鸟使:《汉武内传》载,仙人西王母临降人间之前,先有青鸟飞来通报,后来诗词中多以青鸟为传达爱情信息的使者。西池,传说西王母住在西方昆仑山的瑶池。

[25] 北塞黑鸦军:唐李克用统领的守塞军队都穿黑色衣甲,号黑鸦军。

[26] 文武成康为一代:文、武、成、康,西周初的四个王,史称是承平之世。

[27] 魏吴蜀汉定三分:汉代以后魏、蜀、吴三国鼎立。

[28] 桂苑:栽有桂树的林园。曲客:指酒友。曲,造酒的媒质。

[29] 松亭:松间之亭。桐君:古琴名。因桐木可作琴,故以桐君为琴的代称。薰风:传说帝舜得五弦琴,作《南薰之歌》。





\chapter{元}


卑对长zhǎng ,季对昆kūn [1] 。永巷对长门[2] 。山亭对水阁[3] ,旅lǚ 舍shè 对军屯tún [4] 。杨子渡[5] ,谢公墩dūn [6] 。德重zhòng 对年尊zūn [7] 。承乾对出震zhèn ,叠dié 坎kǎn 对重坤kūn [8] 。志zhì 士报bào 君思犬马,仁王养yǎng 老察鸡豚tún [9] 。远水平沙,有客泛fàn 舟桃叶[10] 渡;斜xié 风细xì 雨,何人携xié 榼kē 杏花村[11] 。

君对相,祖对孙。夕照对朝zhāo 曛xūn [12] 。兰台对桂guì 殿diàn [13] ,海hǎi 岛对山村。碑堕duò 泪[14] ,赋招zhāo 魂[15] 。报bào 怨yuàn 对怀恩ēn 。陵埋mái 金吐气[16] ,田tián 种zhòng 玉生根gēn [17] 。相府珠帘垂白昼zhòu ,边城画角jiǎo [18] 动黄昏hūn 。枫叶半山,秋去qù 烟霞xiá 堪kān 倚杖;梨lí 花满mǎn 地,夜来风雨不开门[19] 。


[1] 昆:兄长。季:弟弟。

[2] 永巷:汉代拘禁犯罪的妃嫔宫女的地方。长门:汉宫名。

[3] 水阁:靠近水的楼阁。

[4] 旅舍:旅馆。军屯:指驻屯的军队。

[5] 杨子渡:古津渡名,在江苏江都县南。

[6] 谢公墩:山名,在江苏江宁县城北(古代金陵),晋谢安尝居半山,曾登临,故名。

[7] 年尊:年纪大。

[8] 承乾对出震,叠坎对重坤:乾、坤、坎、震,《周易》的四个卦名。乾为龙,所以继位为君称承乾。震为雷声,有发号施令的意思,所以出震是皇帝发号令。

[9] 仁王养老察鸡豚:战国思想家孟轲阐述他的仁政思想,说如果王者施仁政,“鸡豚狗彘之畜,无失其时,七十者可以食肉矣”。豚,泛指猪。

[10] 桃叶渡:在江苏南京市内秦淮河、青溪合流处。据说晋王献之有妾名桃叶,桃叶渡江,以歌送之曰“桃叶复桃叶,渡江不用楫”之语。

[11] 榼:古盛酒器皿。杏花村:在金陵。唐杜牧《清明》诗:“借问酒家何处有?牧童遥指杏花村。”后因以杏花村指卖酒之处。

[12] 夕照:傍晚的阳光。曛:本指日落时的余光。这里指早晨的昏暗的阳光。

[13] 兰台:这里指汉代皇家贮藏图书的府库,又称兰台寺。桂殿:对寺观殿宇的美称。

[14] 碑堕泪:晋羊祜为荆州都督,与东吴相对抗,甚有建树。羊祜死,襄阳民为之罢巿巷哭,为他在岘山建碑立庙,看见碑的人,莫不坠泪,因而称堕泪碑。

[15] 赋招魂:楚辞有《招魂赋》一篇,有人以为是屈原为招怀王之魂而作,有的以为是宋玉哀师屈原之死而作。还有说是屈原自招其魂。

[16] 陵埋金吐气:旧传秦始皇南巡,有望气者说,五百年后,金陵当有天子出。始皇于是埋金于金陵镇山以镇压之,故称金陵。

[17] 田种玉生根:《搜神记》载,杨伯雍家住无终山,山上无水,伯雍担水置路旁,供行人取饮。三年后,有一人饮水,送给他一斗石子,让他种。几年后,石子上生出了玉石。后其地称玉田。

[18] 画角:古管乐器,传自西羌。形如竹筒,本细末大,以竹木或皮革等制成,因表面有彩绘,故称。发声哀厉高亢,古时军中多用以警昏晓,振士气,肃军容。帝王出巡,亦用以报警戒严。

[19] 梨花满地,夜来风雨不开门:唐刘方平《春怨》诗:“寂寞空庭春欲晓,梨花满地不开门。”





\chapter{寒}


家对国,治zhì 对安ān 。地主对天官[1] 。坎kǎn 男对离lí 女[2] ,周诰gào 对殷盘pán [3] 。三三暖,九九寒[4] 。杜撰zhuàn 对包bāo 弹[5] 。古壁蛩qióng 声匝zā [6] ,闲亭鹤hè 影yǐng 单dān 。燕出帘边春寂寂,莺yīng 闻枕上漏lòu 珊珊[7] 。池chí 柳烟飘,日夕郎láng 归青锁suǒ 闼tà [8] ;砌[9] 花雨过guò ,月明人倚玉阑干。

肥对瘦,窄zhǎi 对宽kuān 。黄犬对青鸾luán [10] 。指zhǐ 环huán 对腰带,洗钵对投竿[11] 。诛佞nìng 剑[12] ,进贤冠[13] 。画栋对雕diāo 栏[14] 。双垂白玉箸zhù [15] ,九转zhuǎn 紫金丹dān [16] 。陕shǎn 右棠高怀召伯[17] ,河南花满mǎn 忆潘pān 安ān [18] 。陌mò 上芳fāng 春,弱ruò 柳当风[19] 披pī 彩线xiàn ;池chí 中清晓[20] ,碧荷承露[21] 捧pěng 珠盘pán 。

行对卧wò ,听对看。鹿洞[22] 对鱼滩。蛟腾téng 对豹bào 变[23] ,虎踞对龙蟠pán [24] 。风凛lǐn 凛lǐn [25] ,雪漫màn 漫màn [26] 。手辣对心酸suān 。莺yīng 莺yīng 对燕燕[27] ,小小对端duān 端duān [28] 。蓝水远从千涧落,玉山高并两峰寒[29] 。至zhì 圣不凡fán ,嬉戏xì 六龄陈俎zǔ 豆dòu [30] ;老莱大孝,承欢huān 七衮gǔn 舞斑斓[31] 。


[1] 地主:指住在本地的人。天官:官名。《周礼》分设六官,以天官冢宰居首,总御百官。

[2] 坎男对离女:坎和离都是《周易》卦名,古人解释说坎为中男,离为中女。

[3] 周诰、殷盘:《尚书》中属于西周的文献有《洛诰》《康诰》诸篇,属于殷商的文献有《盘庚》上、中、下三篇。

[4] 三三暖,九九寒:农历三月三日,古人称上巳节。农历九月九日,古人称重阳节。

[5] 杜撰:凭空捏造之事,所谓不经之谈。包弹:宋包拯为御史中丞,弹劾不避权贵,人谓之包弹。

[6] 蛩声:蟋蟀的鸣声。匝:环绕。

[7] 漏:古代计时器,铜制有孔,可以滴水或漏沙,有刻度标志以计时间。简称“漏”。珊珊:形容衣裙玉珮的声音。

[8] 青锁闼:翰林直宿的地方,门上刻画有青色连锁花纹,因称青锁闼。闼,门。

[9] 砌:台阶。

[10] 黄犬:指晋陆机的黄耳犬。曾为陆机长途传递书信。青鸾:古代传说中凤凰一类的神鸟。

[11] 洗钵:即洗钵泉,今位于山东济南李清照纪念堂院内西北隅,为不规则泉池。投竿:投钓竿于水,即垂钓。

[12] 诛佞剑:汉朱云忠直敢谏。成帝的老师安昌侯张禹,在朝廷甚有地位,然毫无作为。朱云对成帝说:“臣愿求赐上方宝剑,断佞臣一人,以厉其余。”上问为谁,曰张禹。帝怒令斩之,云攀殿槛,槛折以免。或请易槛,上不许,存之以旌忠臣。

[13] 进贤冠:文官戴的一种帽子。

[14] 画栋:有彩绘装饰的栋梁。

[15] 白玉箸:道家得道,临终有白玉气出鼻孔,双垂如双玉箸。

[16] 九转紫金丹:古代术士把朱砂烧成水银,又把水银炼成丹药,叫做还丹。九转,形容经过许多步骤。

[17] 陕右棠高怀召伯:召虎是周宣王时的一位大臣,人们称他为召伯。他很有政绩,传说他的住处有一棵甘棠树,他走后,人们对这棵树加意保护,并且作了一首叫《甘棠》的诗歌,以资纪念。陕右,即关中地区。

[18] 河南花满忆潘安:河南疑当作河阳,潘安为河阳令,满县皆栽桃花,人曰花县。

[19] 当风:正对着风。

[20] 清晓:清晨,天刚亮的时候。

[21] 承露:承接甘露。

[22] 鹿洞:指白鹿洞。宋朱熹讲学处。

[23] 豹变:意思是君子的变化像豹一样,越来越有文采。喻润色事业,或迁喜去恶。

[24] 虎踞、龙蟠:诸葛亮论金陵的地形,说:“钟阜龙蟠,石城虎踞。”

[25] 凛凛:寒冷的样子。

[26] 漫漫:空间广远的样子。

[27] 莺莺、燕燕:钱塘范十二郎有二女,曰莺莺燕燕,为富民陆氏妾。

[28] 小小、端端:钱塘妓女苏小小,亦名简简。

[29] 蓝水远从千涧落,玉山高并两峰寒:是杜甫《九日兰田崔氏庄》一诗的腹联。

[30] 至圣不凡,嬉戏六龄陈俎豆:《史记·孔子世家》载:“孔子为儿嬉戏,常陈俎豆,设礼容。”

[31] 老莱大孝,承欢七衮舞斑斓:老莱子,传说中的古孝子,父母年迈,无以为欢,他虽也年纪很大,但仍穿上花花绿绿的幼儿服装,在父母面前嬉笑,引逗双亲开心。





\chapter{删}


林对坞wù[1] ,岭对峦luán 。昼zhòu 永[2] 对春闲。谋móu 深对望重zhòng ,任大对投艰[3] 。裙袅niǎo 袅niǎo [4] ,佩pèi [5] 珊珊。守塞sài 对当关[6] 。密云千里合,新月一钩弯wān 。叔宝君臣皆jiē 纵zòng 逸[7] ,重华父母是嚚yín 顽wán [8] 。名动帝畿jī ,西蜀三苏来日下[9] ;壮zhuàng 游yóu 京洛,东吴二陆起qǐ 云间[10] 。

临对仿,吝lìn 对悭qiān [11] 。讨tǎo 逆nì 对平蛮mán [12] 。忠肝对义胆dǎn ,雾发fà 对云鬟huán 。埋mái 笔冢zhǒng [13] ,烂làn 柯kē 山。月貌mào 对天颜。龙潜终得跃[14] ,鸟niǎo 倦juàn 亦知还huán [15] 。陇lǒng 树[16] 飞来鹦yīng 鹉绿,池chí 筠yún 密处鹧zhè 鸪gū 斑。秋露横江,苏子月明游yóu 赤壁[17] ;冻云迷岭,韩公雪拥yōng 过guò 蓝关[18] 。



* * *



[1] 坞:四面高,中间凹下的地方。

[2] 昼永:白昼漫长。

[3] 投艰:赋予重任。

[4] 袅:随风摆动的样子。

[5] 佩:古代女子头上或身上的佩饰。

[6] 当关:把守关隘。

[7] 叔宝君臣皆纵逸:南朝陈后主,名叔宝,历史上有名的荒淫皇帝。他经常召集江总、孔范等十个文人在一起饮宴,称为“狎客”,让张贵人等八名妃嫔与之交错而坐,整日纵情声色。

[8] 重华父母是嚚顽:重华是帝舜的名。相传他的父亲瞽叟和弟弟象品行都很坏,曾多次设阴谋准备把他害死。嚚顽,愚蠢而顽固。瞽,瞎。

[9] 名动帝畿,西蜀三苏来日下:三苏,指宋著名文学家苏洵和他的儿子苏轼、苏辙。他们都是四川眉山人,名震一时,人称三苏。帝畿,我国古代称靠近国都的地方。这里同句中的“日下”都指都城。

[10] 壮游京洛,东吴二陆起云间:二陆指晋文学家陆机、陆云兄弟,大有才名,人称二陆。他们在东吴亡后,都来到洛阳从政。据说一次陆云遇到荀隐,互相自我介绍,陆说:“云间陆士龙。”荀说:“日下荀鸣鹤。”云间,江苏松江县之古称。壮游,谓怀抱壮志而远游。

[11] 悭:吝啬。

[12] 讨逆:讨伐坏人。蛮:旧指南方少数民族。

[13] 埋笔冢:陈、隋间僧人智永是著名的书法家,相传他写字用笔积十八瓮,后埋成一墓,号曰“退笔冢”。

[14] 龙潜终得跃:《周易·乾卦》:“初九,潜龙勿用。”“九四,或跃在渊。”比喻人或事物由小到大、由弱到强的发展过程。

[15] 鸟倦亦知还:语出晋陶渊明《归去来兮辞》:“云无心以出岫,鸟倦飞而知还。”

[16] 陇树:陇山一带的树木。泛指边塞之树。

[17] 秋露横江,苏子月明游赤壁:元丰四年,苏轼曾月夜泛舟赤壁,作《前赤壁赋》,赋中有“少焉,月出于东山之上,徘徊于斗牛之间。白露横江,水光接天”等语。

[18] 冻云迷岭,韩公雪拥过蓝关:唐文学家韩愈,以上《谏迎佛骨表》触怒宪宗,被贬为潮州刺史,行程中至蓝关遇雪,写了一首《左迁至蓝关示侄孙湘》,“云横秦岭家何在,雪拥蓝关马不前”是诗中名句。





\part{卷下}

\setcounter{chapter}{0}

\chapter{先}


寒对暑,日对年。蹴cù 踘jū [1] 对秋千。丹dān 山对碧水,淡dàn 雨对覃tán 烟[2] 。歌宛wǎn 转zhuǎn [3] ,貌mào 婵c娟juān [4] 。雪鼓对云笺jiān [5] 。荒huāng 芦lú 栖南雁,疏柳噪zào 秋蝉c。洗耳ěr 尚逢féng 高士笑[6] ,折腰肯kěn 受小儿怜[7] 。郭guō 泰泛fàn 舟,折角jiǎo 半垂梅子雨[8] ;山涛tāo 骑马,接jiē lí 倒着zhuó 杏花天[9] 。

轻对重zhòng ,肥对坚[10] 。碧玉对青钱[11] 。郊寒对岛瘦[12] ,酒圣对诗仙[13] 。依玉树[14] ,步金莲[15] 。凿záo 井jǐng 对耕gēng 田tián [16] 。杜甫清宵xiāo 立[17] ,边韶sháo 白昼zhòu 眠mián [18] 。豪háo 饮客吞tūn 波底dǐ 月,酣hān 游yóu 人醉水中天[19] 。斗dòu 草青郊[20] ,几行háng 宝马嘶金勒[21] ;看花紫陌mò [22] ,千里香车chē 拥yōng 翠钿diàn [23] 。

吟对咏,授对传。乐矣对凄然rán 。风鹏péng 对雪雁,董dǒng 杏对周莲[24] 。春九十[25] ,岁三千[26] 。钟鼓对管guǎn 弦。入山逢féng 宰相[27] ,无事即jí 神仙xian 。霞xiá 映yìng 武陵桃淡dàn 淡dàn ,烟荒huāng 隋suí 堤dī 柳绵mián 绵mián [28] 。七碗wǎn 月团tuán ,啜chuò 罢bà 清风生腋下;三杯云液[29] ,饮余红雨晕yùn 腮sāi 边。

中对外,后hòu 对先。树下对花前。玉柱zhù 对金屋[30] ,叠dié 嶂对平川[31] 。孙子策[32] ,祖生鞭[33] 。盛席xí 对华筵[34] 。解醉知茶力,消xiāo 愁chóu 识酒权[35] 。丝剪jiǎn 芰jì 荷开冻沼zhǎo [36] ,锦jǐn 妆凫fú 雁泛fàn 温wēn 泉[37] 。帝女衔石,海hǎi 中遗魄pò 为精卫[38] ;蜀王叫月,枝上游yóu 魂化杜鹃juān [39] 。



* * *



[1] 蹴踘:我国古代的一种足球运动。

[2] 覃烟:袅袅直升空中的饮烟或横浮低空的烟雾。覃,长。

[3] 宛转:声音委婉而动听。

[4] 婵娟:体态柔弱的样子。

[5] 云笺:唐韦陟(zhì)用五采笺写信,由他人代笔,自己签名。由于他写的“陟”字像五朵云,因而后来人们称书信为五云笺或云笺。

[6] 洗耳尚逢高士笑:传说帝尧时,箕山有高人隐士曰巢父、许由,尧同许由商量,准备把帝位传给他。巢父听到了,以为玷污了他的耳朵,就跑到池中去洗耳。池水主人怒曰:“何污我水!”这个故事说帝尧、许由、巢父、池水主人,一个比一个更高洁。

[7] 折腰肯受小儿怜:陶渊明为彭泽令。一次,郡督邮来视察。县吏向陶渊明建议,应穿上官服迎见。陶渊明气愤地说:“吾不能为五斗米折腰,拳拳事乡里小儿!”于是弃官而去。作《归去来兮辞》。

[8] 郭泰泛舟,折角半垂梅子雨:见真韵“郭泰”句注。

[9] 山涛骑马,接 倒着杏花天:见文韵“竹间”句注。

[10] 肥对坚:肥,肥马。坚,坚车。

[11] 碧玉:南朝宋汝南王妾,甚受宠爱,后代引为娇怜的爱人的代称。青钱:唐张鷟(zhuó)甚有才名,时人称之为“青钱学士”,意思是他的文章万选万中,万无一失。

[12] 郊寒对岛瘦:郊指孟郊,岛指贾岛,唐代的两个诗人。孟郊的诗内容清苦,失之寒,贾岛的诗风格瘦峭,失之瘦,后人于是有“郊寒岛瘦”的评价。

[13] 酒圣:晋刘伶旷达放饮,又曾作《酒德颂》,后人因称之为酒圣。

[14] 依玉树:唐崔宗之,美容仪,饮酒时更见风度。杜甫诗《饮中八仙歌》说:“宗之潇洒美少年,举觞白眼望青天,皎如玉树临风前。”

[15] 步金莲:南齐东昏侯宠爱潘妃,以金为莲花贴地,令潘妃行其上,叫“步步生莲花”。后以金莲指女子纤足。

[16] 凿井、耕田:传说尧帝游于康衢,有一老人击壤而歌曰:“日出而作,日入而息,凿井而饮,耕田而食,帝力于我何有哉!”

[17] 杜甫清宵立:杜甫诗有“思家步月清宵立”句。

[18] 边韶白昼眠:汉儒边韶,字孝先,性放达,开帐授徒,常昼眠,弟子编歌嘲之曰:“边孝先,腹便便。夜读书,昼贪眠。”

[19] 豪饮客吞波底月,酣游人醉水中天:杜甫《饮中八仙歌》有“左相日兴费万钱,饮如长鲸吸百川”;“知章骑马似乘船,眼花落井水底眠”等语,形容醉人们的情态。

[20] 斗草:也称“斗百草”。一种古代游戏。竞采花草,比赛多寡优劣,常于端午行之。青郊:指春天的郊野。

[21] 金勒:金饰的带嚼口的马络头。

[22] 紫陌:指京师郊野的道路。

[23] 翠钿:妇女用宝石金银雕饰的首饰,这里即代指妇女。钿,为压韵可读tián。

[24] 董杏:《神仙传》中载,三国东吴董奉为人治病不取报酬,病重的为他栽五棵杏,轻者栽一棵,数年后共得十万余株,郁然成林。周莲:宋儒周敦颐喜爱莲花,曾写《爱莲说》一篇,盛赞此花出污泥而不染的高洁品质。

[25] 春九十:春光九十,意思是春光将尽。

[26] 岁三千:极言年寿之长。传说汉武帝时,东郊献短人东方朔,谓帝曰:“王母蟠桃,三千岁一熟,此儿已三偷之矣。”

[27] 入山逢宰相:南朝梁陶宏景隐山中,武帝常问之以国事,时人称之“山中宰相”。

[28] 烟荒隋堤柳绵绵:隋炀帝自板渚引河达淮,岸上悉种柳。见齐韵“隋堤”注。

[29] 云液:酒的美称。

[30] 玉柱:石柱的美称。金屋:华美之屋。

[31] 叠嶂:重迭的山峰。平川:广阔平坦之地。

[32] 孙子策:孙子指春秋战国时吴国孙武,著名军事家,著有《孙子》十三篇传世。

[33] 祖生鞭:东晋祖逖与朋友刘琨同寝,他们立志收复中原,每天闻鸡鸣就起床舞剑。一次祖逖先醒,闻鸡鸣,逖蹴琨曰:“此非恶声也。”琨恐曰:“祖生先吾着鞭。”意思是比自己行动得快。

[34] 华筵:丰盛的筵席。

[35] 解醉知茶力,消愁识酒权:茶力、酒权互文,即茶和酒的功效。

[36] 丝剪芰荷开冻沼:传说中隋炀帝的故事,说他曾命人用锦绢剪为荷花,遍插池苑,从中游乐。芰,古书上指菱。

[37] 锦妆凫雁泛温泉:唐玄宗的故事。相传玄宗扩建华清宫汤池,规模宏丽,汤池内以玉莲为喷泉,又缝锦绣为凫雁,放于水中,自己乘小舟从中游嬉,极尽奢欲。

[38] 帝女衔石,海中遗魄为精卫:上古神话,赤帝有女名女娃,游于东海,溺而不返,魂魄变成一种鸟,名叫精卫,常常衔木石填海中。

[39] 蜀王叫月,枝上游魂化杜鹃:上古神话传说,蜀王名杜宇,在蜀治水,自以德薄,让位给大臣鳖冷,自己隐居山林,死后化为杜鹃鸟,夜夜悲啼,啼则吐血。





\chapter{萧}


琴对管guǎn ,斧对瓢piáo 。水怪guài 对花妖。秋声对春色,白缣jiān 对红绡xiāo [1] 。臣五代[2] ,事三朝[3] 。斗dǒu 柄bǐng 对弓腰[4] 。醉客歌金缕lǚ [5] ,佳人品pǐn 玉箫xiāo 。风定落花闲不扫sǎo ,霜余残cán 叶湿难烧shāo 。千载兴周,尚父一竿投渭水[6] ;百年霸bà 越,钱王万弩nǔ 射shè 江潮[7] 。

荣对悴[8] ,夕对朝zhāo 。露地[9] 对云霄xiāo 。商shāng 彝对周鼎dǐng [10] ,殷濩huò 对虞韶sháo [11] 。樊fán 素sù 口kǒu ,小蛮mán 腰[12] 。六诏对三苗[13] 。朝天车chē 奕yì 奕[14] ,出塞sài 马萧xiāo 萧xiāo [15] 。公子幽兰重泛fàn 舸gě [16] ,王孙芳fāng 草正联镳biāo [17] 。潘pān 岳高怀,曾向秋天吟蟋蟀shuài [18] ;王维清兴,尝于雪夜画芭蕉[19] 。

耕gēng 对读,牧对樵。琥珀pò 对琼qióng 瑶[20] 。兔毫háo 对鸿爪zhǎo [21] ,桂guì 楫jí 对兰桡ráo [22] 。鱼潜藻,鹿藏蕉[23] 。水远对山遥。湘灵能鼓瑟[24] ,嬴yíng 女解吹箫xiāo [25] 。雪点寒梅横小院yuàn ,风吹弱ruò 柳覆平桥。月牖通宵xiāo ,绛蜡罢bà 时光不减jiǎn [26] ;风帘当昼zhòu ,雕diāo 盘pán 停后hòu 篆zhuàn 难消xiāo [27] 。



* * *



[1] 缣:丝绢,这里指细绢。绡:生丝,又指用生丝织的东西,这里指绸子。

[2] 臣五代:指五代时冯道,他曾历事后唐、后晋、后辽、后汉、后周,对丧君亡国毫不介意,并自号“长乐老”。旧时代拿他做没气节的典型。

[3] 事三朝:沈约事南朝宋、齐、梁三朝。

[4] 斗柄:北斗七星中排成柄状的三星。弓腰:舞女反身将腰弯如弓形,叫做弓腰。

[5] 金缕:词牌《贺新郎》的别名,或说指唐女诗人杜秋娘所作《金缕衣》。

[6] 千载兴周,尚父一竿投渭水:西周初,吕望曾隐居在渭水垂钓,后被周文王聘请为太师,辅佐武王灭殷。被周武王尊为尚父。

[7] 百年霸越,钱王万弩射江潮:传说五代时钱 为吴越王,做御潮铁柱于江中,未成而潮水大至。吴越王命以万弩射之,潮水乃退。筑土一升者,赏钱一升,名之曰钱塘。

[8] 荣:茂。悴:枯。

[9] 露地:佛教语。喻三界(欲界、色界、无色界)的烦恼俱尽,处于没有覆蔽的地方。

[10] 商彝、周鼎:指商周二代的青铜器。

[11] 濩:传说是商汤王的舞乐。韶:传说帝舜时乐名。虞即指帝舜虞氏。

[12] 樊素口,小蛮腰:樊素、小蛮都是白居易的歌伎。白有“樱桃樊素口,杨柳小蛮腰”的诗句。

[13] 六诏:“诏”是唐代我国西南少数民族对王的称呼,时有蒙嶲(xī)、越析、浪穹、澄睒(shān)、施浪、蒙舍诸诏,合称六诏。其地在今云南及四川西南部。三苗:传说尧、舜时代居住在西南的我国少数民族。

[14] 朝天车:指大臣们登朝拜见皇帝所用车乘。奕奕:有次序的样子。

[15] 出塞马萧萧:杜甫《后出塞》诗有“马鸣风萧萧”之句。萧萧,马嘶声或风声。

[16] 公子幽兰重泛舸:屈原《九歌》:“沅有芷兮澧(lǐ)有兰,思公子兮未敢言。”舸,大船。泛舸即乘船游览。

[17] 王孙芳草正联镳:刘安《招隐士》:“王孙游兮不归,春草生兮萋萋。”镳,马辔头。联镳,意思是并马而行。

[18] 潘岳高怀,曾向秋天吟蟋蟀:潘岳是晋诗人,曾写有《蟋蟀赋》。

[19] 王维清兴,尝于雪夜画芭蕉:唐王维诗、画、书都有很高造诣。据说他的山水画随意写来,不分四时,曾画雪中芭蕉。

[20] 琼瑶:美玉。

[21] 兔毫:笔名,这里指毛笔。鸿爪:指鸿雁在泥土上留下的脚印,比喻人生的阅历。

[22] 桂楫、兰桡:楫和桡都是划船撑船的工具。桂是桂树,兰指木兰。用桂和木兰制成的楫和桡,言其贵重华美。

[23] 鹿藏蕉:《列子·周穆王》:郑人有薪者,遇鹿而毙之,藏诸泥中,覆之以蕉,俄而失其处,遂以为梦,顺途而道其事。傍闻者取之,归告室人曰:薪者梦得鹿,不知其处,我今得之,彼真在梦中矣。

[24] 湘灵能鼓瑟:湘灵,尧女娥皇女英,哭舜于苍梧之野,死之为湘江之神。

[25] 嬴女解吹箫:即弄玉的故事。秦王族姓嬴,故称弄玉为嬴女。见江韵“跨凤”句注。

[26] 月牖通宵,绛蜡罢时光不减:由于月光透窗而入,即使灭掉红烛,室内仍很明亮。绛蜡:即红烛。

[27] 风帘当昼,雕盘停后篆难消:篆,指袅袅上升的香烟好像篆字一样。二句意思是,因为风帘遮掩门户,尽管雕盘中的薰香不再点燃,室内的香气也很难消失。





\chapter{肴}


《诗》对《礼》,卦guà 对爻。燕引对莺yīng 调tiáo [1] 。晨钟对暮鼓[2] ,野馔zhuàn 对山肴[3] 。雉zhì 方fāng 乳rǔ [4] ,鹊què 始巢[5] 。猛měng 虎对神獒áo [6] 。疏星浮荇xìng 叶,皓hào 月上松梢shāo 。为邦bāng 自古推tuī 瑚hú 琏liǎn [7] ,从政于今愧kuì 斗dǒu 筲shāo [8] 。管guǎn 鲍bào 相知,能交忘形胶漆友[9] ;蔺lìn 廉有隙xì ,终为刎wěn 颈jǐng [10] 死生交。

歌对舞,笑对嘲。耳ěr 语对神交[11] 。焉鸟niǎo 对亥hài 豕shǐ [12] ,獭tǎ 髓suǐ 对鸾luán 胶[13] 。宜久敬,莫mò 轻抛pāo 。一气[14] 对同胞bāo 。祭zhài 遵zūn 甘布被bèi [15] ,张zhāng 禄念绨tí 袍páo [16] 。花径风来逢féng 客访[17] ,柴扉月到有僧敲qiāo 。夜雨园中,一颗kē 不雕diāo 王子柰nài [18] ;秋风江上,三重曾卷杜公茅máo [19] 。

衙yá 对舍shè [20] ,廪lǐn 对庖páo [21] 。玉磬qìng 对金铙náo [22] 。竹林对梅岭[23] ,起qǐ 凤对腾téng 蛟[24] 。鲛绡xiāo [25] 帐,兽锦jǐn [26] 袍páo 。露果guǒ 对风梢shāo 。扬州输橘jú 柚,荆土贡gòng 菁茅máo [27] 。断蛇埋mái 地称chēng 孙叔[28] ,渡蚁作桥识宋sòng 郊[29] 。好梦难成,蛩qióng 响阶jiē 前偏唧唧;良朋péng 远到,鸡声窗外正嘐jiāo 嘐。



* * *



[1] 燕引对莺调:引和调都是歌曲,这里指燕和莺动听的鸣声。

[2] 晨钟对暮鼓:见上卷冬韵“暮鼓”句注。

[3] 野馔、山肴:馔、肴是饭菜的统称。野馔、山肴指淡素的饭食。

[4] 雉方乳:汉鲁恭为中军令,很有政绩,蝗不入境。河南尹闻之,使人往看。见野鸡伏于桑下,儿童不捕,惊问,儿童说:“野鸡在孵卵,不要伤害它。”雉,野鸡。

[5] 鹊始巢:语出《礼记·月令》:“雁北乡,鹊始巢,雉雊,鸡乳。”

[6] 神獒:传说能听懂人语的犬叫獒。

[7] 为邦自古推瑚琏:《论语》载,一次孔子弟子子贡问老师:“我是怎样一个人?”孔子说:“你是能成器的。”又问:“我是怎样的器?”孔子说:“你是瑚琏。”瑚琏,古代宗庙盛黍稷的器皿,是祭祀的贵重礼器,比喻子贡会成为治国的人材。为邦,治理国家。

[8] 从政于今愧斗筲:《论语》载,一次子贡问,当今做官的人怎么样,孔子说:“噫,斗筲之人,何足算也!”斗筲之人,即德薄才疏的人。

[9] 管鲍相知,能交忘形胶漆友:春秋时,管仲和鲍叔牙交情非常好,患难与共,旧时代常以管鲍为朋友间的楷模。管仲,春秋初年政治家。经鲍叔牙推荐,被齐桓公任为上卿。相知,即相友好。胶漆,形容难解难分,关系极为密切。

[10] 刎颈:指发誓同死的交情。

[11] 耳语:凑近耳朵小声说话。神交:彼此慕名而没有见过面的交谊。

[12] 焉鸟对亥豕:古文之讹。焉和鸟,亥和豕,字形相近,往往造成讹误。焉鸟:谓字形相近而易讹。

[13] 獭:水獭,旧传水獭的髓是很好的滋补品,服食能益神智;相传水獭的骨髓与玉屑、琥珀屑相和,可以灭瘢痕。鸾胶:传说海上有凤麟洲,多仙人,以凤喙麟角合煎作膏,名续弦胶,能续弓弩断弦。

[14] 一气:犹云同气,指有血缘关系的亲属,多喻兄弟。

[15] 祭遵甘布被:祭遵是东汉光武帝的将军。《后汉书·祭遵传》:遵为人克己奉公,凡皇帝的赏赐一律分给士卒,家无私财,穿皮裤,盖布被,夫人裳不加缘,因而受到皇帝的敬重。

[16] 张禄念绨袍:战国时,范睢和须贾同事魏王,须贾出于嫉妒,唆使魏相治范睢几至于死。后范睢逃到秦国,改名张禄,为秦相。后须贾使秦,范睢故意穿了一身破衣服去见须贾。贾不知其为秦相,说“范叔何一寒至此”,以己绨袍赠之。不久,须贾终于知道范睢原来就是秦相张禄,吓得赶忙登门请罪。范睢说:“根据你旧日对我的态度,本当把你处死。但你送我一件袍子,看来还有点情谊,可以饶你一命。”绨,光滑厚实的丝织品。

[17] 花径风来逢客访:语出杜甫《客至》:“花径不曾缘客扫,蓬门今始为君开。”

[18] 夜雨园中,一颗不雕王子柰:《二十四孝》载:晋人王祥至孝,后母不慈,命其看护后园柰树,柰落则鞭之。祥抱树大哭,感动上天,柰一颗不落。柰,落叶小乔木,花白色,果小,是苹果的一种。

[19] 秋风江上,三重曾卷杜公茅:杜公指杜甫。杜甫居成都时,一次大风吹坏了草堂,他曾为此写作了《茅屋为秋风所破歌》,中有“八月秋高风怒号,卷我屋上三重茅”之句。

[20] 衙:旧时官舍之称。舍:居住的房子。

[21] 廪:粮仓。庖:厨房。

[22] 玉磬:古代的一种用玉或石制成的打击乐器。金铙:一种用金属制成的打击乐器。

[23] 竹林:晋时嵇康与阮籍等七人为友,蔑视礼教,狂放不羁,经常聚在竹林中啸饮清谈,时人号为“竹林七贤”。梅岭:英州司寇种梅三十株于大庾岭,故庾岭多梅。

[24] 起凤、腾蛟:都是形容文采的超拔。

[25] 鲛绡:古代神话,南海外有鲛人,住在水中,善织绩,常出卖绡,眼能泣泪成珠。鲛绡,鲛人所织的细绢。鲛,就是鲨鱼。绡,生丝,又指用生丝织的东西。

[26] 兽锦:绣有麟、豹一类野兽花纹的锦缎。

[27] 扬州输橘柚,荆土贡菁茅:《尚书》有《禹贡》篇,记述九州的山川土宜,提出扬州要贡赋桔柚,荆州要贡献菁茅。菁茅,一种草类,古人用以扎神像,灌酒其上,表示神饮,叫祼。

[28] 断蛇埋地称孙叔:孙叔敖,战国时楚国令尹,幼时见两个头的蛇,杀而埋之,回家后对母亲哭诉。母问其故,他说:“人们说遇到两头蛇的人一定会死,今天我遇到了。为了不至于让更多的人见而致死,我已杀死并且埋掉了它。”母亲说:“我儿做了好事,必有善报。”后来孙叔敖果然做了楚国的令尹。

[29] 渡蚁作桥识宋郊:迷信传说,宋郊为士人时,所居堂前有蚁穴为雨水冲毁,他编竹为桥让蚂蚁爬到了干处,据说因为有此阴德,后为状元。





\chapter{豪}


茭对茨,荻dí 对蒿hāo





[1] 。山麓对江皋[2] 。莺yīng 簧对蝶dié 板bǎn [3] ,麦浪làng 对桃涛tāo [4] 。骐骥jì 足zú ,凤凰毛máo [5] 。美měi 誉对嘉褒bāo 。文人窥kuī 蠹dù 简jiǎn [6] ,学士书兔毫[7] 。马援南征zhēng 载薏yì 苡yǐ [8] ,张zhāng 骞qiān 西使进葡pú 萄[9] 。辩口kǒu 悬xuán 河,万语千言常亹wěi 亹wěi ;词源倒峡xiá ,连篇累lěi 牍自滔tāo 滔tāo [10] 。

梅对杏,李对桃。棫yù 朴pò 对旌旄máo [11] 。酒仙对诗史[12] ,德泽zé 对恩ēn 膏[13] 。悬xuán 一榻tà [14] ,梦三刀[15] 。拙zhuō 逸对贵guì 劳。玉堂花烛绕rào ,金殿diàn 月轮lún 高[16] 。孤gū 山看鹤hè 盘pán 云下[17] ,蜀道闻猿向月号[18] 。万事从人,有花有酒应yīng 自乐;百年皆jiē 客,一丘一壑hè 尽吾豪[19] 。

台对省shěng ,署对曹[20] 。分袂mèi 对同袍páo [21] 。鸣琴对击剑,返辙对回艚[22] 。良借jiè 箸zhù [23] ,操cāo 提tí 刀[24] 。香茶对醇chún 醪láo [25] 。滴dī 泉归海hǎi 大,篑kuì 土积山高[26] 。石室客来煎雀què 舌[27] ,画堂宾至饮羊羔[28] 。被bèi 谪贾生,湘水凄凉吟《 fú 鸟niǎo 》 [29] ;遭zāo 谗屈子,江潭憔悴著zhù 《离lí 骚sāo 》 [30] 。



* * *



[1] 茭、茨、荻、蒿:都是指蒿草。

[2] 山麓:山脚下。江皋:江边的高地。

[3] 莺簧:黄莺啼叫的声音美如笙簧。蝶板:蝴蝶的双翅忽开忽合好象乐器中的板。

[4] 麦浪:风吹麦田,麦子像波浪般起伏的样子。桃涛:春二三月,桃花盛开之时,河中春汛,称为桃花汛。

[5] 骐骥:良马。骐骥足,比喻人有才干。凤凰毛:凤毛麟角,喻稀有的优秀人才。

[6] 蠹简:指书籍。蠹,蛀书虫。

[7] 兔毫:用兔毛制成的笔。泛指毛笔。

[8] 马援南征载薏苡:马援是东汉的将军,他南征交趾时,曾携带数车薏苡,以防治瘴疠。薏苡,多年生草本植物,即中药苡仁。

[9] 张骞西使进葡萄:汉武帝时,张骞曾两次出使西域,使汉族和少数民族、中国和外国的文化得以交流。传说中原地区的葡萄是他由西域带回来的,留种中国。

[10] 辩口悬河,万语千言常亹亹;词源倒峡,连篇累牍自滔滔:都形容人善于谈吐。亹亹,原意是勤奋的样子,这里是言不绝口的意思。词源倒峡:谓诗文雄健有力,气势豪迈。

[11] 棫朴对旌旄:棫朴,两种灌木名,据说可点燃祭天神。《诗经·大雅》中有《棫朴》篇。棫,白桵。朴,桴木。意谓棫朴丛生,根枝茂密,共同附着。喻贤人众多,国家蕃兴。旌旄,指旗帜。

[12] 酒仙对诗史:杜甫有《饮中八仙歌》,称李白、贺知章、李琎、张旭等八人为酒仙。诗史:杜甫的许多诗,较为真实地记述了当时的社会状况,被人称为“诗史”。

[13] 德泽对恩膏:泽和膏都是指及时的好雨,因而被比作恩德。

[14] 悬一榻:后汉徐稚,字孺子,家贫,有德行,当时陈蕃为豫章太守,不接待宾客,只特设一榻待徐稚,徐来则放下,徐走后即悬起。

[15] 梦三刀:迷信传说,晋王浚夜梦梁上悬三把刀,后又增加一把,醒来问别人是何吉凶。解者曰:三刀是州字,又加一把是“益”的意思,是益州,所以您要做益州刺史了。后果守益州。

[16] 金殿:金饰的殿堂,指帝王的宫殿。月轮:指月亮。

[17] 孤山看鹤盘云下:宋林逋,隐西湖孤山,常养两鹤,纵之则飞入云霄,盘旋久之乃下。

[18] 蜀道闻猿向月号:古代四川多猿,所以民歌有“巴东三峡巫峡长,猿啼三声泪沾裳”的说法。

[19] 百年皆客,一丘一壑尽吾豪:这是一种消极的人生观,认为人生百年不过如客人一样暂住世间,应放浪山水之间,尽其豪情。

[20] 台、省、署、曹:都是古时官府的名称。

[21] 分袂:古时把离别称作分袂。袂,袖子。同袍:最早出自《诗经·秦风·无衣》:“岂曰无衣?与子同袍。王于兴师,修我戈矛,与子同仇。”后来多为军人用以互称。后亦用来泛指朋友、同僚、同学等。

[22] 返辙对回艚:返辙即回车。晋阮籍由于当时政治昏暗,心情苦闷,常酒醉后乘车出游,遇到绝路就痛哭而回。回艚:艚,就是船。晋王献之曾在雪夜乘船去访问他的老朋友戴逵,走到半路,忽然命令船只返回。人们问什么缘故,他说自己是“乘兴而来,兴尽而返”。

[23] 良借箸:楚汉战争中,汉高祖听信郦生的话,准备把诸将分封于各地为侯王。张良认为这是错误的,就在酒宴前,借席上箸一一陈说道理。箸,筷子。

[24] 操提刀:传说匈奴使者要拜谒曹操,曹操自以为相貌不扬,恐为耻笑,于是让崔琰装扮成魏王,曹操自己装扮成卫士,提刀立旁。朝见后,让人问使者对魏王的印象。使者曰,魏王相貌亦复平常,但床头捉刀人(指曹操)乃真英雄。

[25] 醇醪:味厚的美酒。

[26] 滴泉归海大,篑土积山高:都是说积少成多的意思。篑,古代盛土的筐子。

[27] 雀舌:一种名茶。

[28] 羊羔:美酒名。

[29] 被谪贾生,湘水凄凉吟 鸟:汉贾谊被黜为长沙王太傅,内心悲苦,一日有猫头鹰进宅,人皆以为不祥,他就写了一篇《 鸟赋》抒发情怀。 ,一种猫头鹰类的鸟。

[30] 遭谗屈子,江潭憔悴著《离骚》:战国时期楚国大夫、爱国诗人屈原,由于佞臣毁谤,遭到楚王贬谪,曾在湘江一带流浪,《史记·屈原贾生列传》:“披发行吟泽畔,颜色憔悴,形容枯槁。”后投汩罗江而死。《离骚》是他写作的长诗。





\chapter{歌}


微对巨,少对多。直zhí 干对平柯kē [1] 。蜂媒对蝶dié 使[2] ,雨笠对烟蓑suō [3] 。眉淡dàn 扫sǎo [4] ,面微酡tuó 。妙miào 舞对清歌。轻衫裁夏葛[5] ,薄袂mèi 剪jiǎn 春罗luó [6] 。将相兼行唐李靖[7] ,霸bà 王杂zá 用yòng 汉萧xiāo 何[8] 。月本阴精,岂qǐ 有羿妻曾窃qiè 药yào [9] ;星为夜宿xiù ,浪làng 传织女漫màn 投梭suō [10] 。

慈对善s,虐对苛kē 。缥缈miǎo 对婆pó 娑suō [11] 。长杨对细柳[12] ,嫩nèn 蕊ruǐ 对寒莎suō [13] 。追风马[14] ,挽wǎn 日戈[15] 。玉液对金波[16] 。紫诏衔丹dān 凤[17] ,黄庭换huàn 白鹅é [18] 。画阁江城梅作调diào [19] ,兰舟野渡竹为歌[20] 。门外雪飞,错认空中飘柳絮xù [21] ;岩边瀑pù 响,误疑天半落银河[22] 。

松对竹,荇[23] 对荷。薜荔[24] 对藤téng 萝luó 。梯tī 云对步月[25] ,樵唱对渔歌。升鼎dǐng 雉zhì [26] ,听经鹅é [27] 。北海对东坡pō [28] 。吴郎láng 哀āi 废宅zhái [29] ,邵子乐行窝wō [30] 。丽lí 水良金皆jiē 待冶,昆山美měi 玉总zǒng 须xū 磨mó [31] 。雨过guò 皇州,琉璃lí 色灿càn 华清瓦wǎ ;风来帝苑yuàn ,荷芰jì 香飘太液波[32] 。

笼对槛jiàn ,巢对窝wō 。及jí 第对登科kē [33] 。冰清对玉润[34] ,地利对人和[35] 。韩擒虎[36] ,荣驾鹅[37] 。青女对素sù 娥[38] 。破pò 头朱泚cǐ 笏hù [39] ,折齿chǐ 谢鲲kūn 梭suō [40] 。留客酒杯应yīng 恨少,动人诗句不须xū 多。绿野凝níng 烟,但dàn 听村前双牧笛dí ;沧江积雪,惟看滩上一渔蓑suō [41] 。



* * *



[1] 直干、平柯:挺直的树干。柯,树枝。平柯犹言横枝。

[2] 蜂媒:比喻为男女双方居间撮合或传递消息的人。蝶使:比喻男女双方情爱的媒介。

[3] 雨笠:遮雨的笠帽。烟蓑:蓑衣。

[4] 扫:描画。

[5] 夏葛:指夏天穿的葛衣。

[6] 春罗:适于春季穿的绫罗。

[7] 将相兼行唐李靖:李靖,唐初著名军事家。他曾在建立唐王朝的斗争中屡立战功,后又平突厥之叛,三定朔方,被封为卫国公。将相兼行是说他才兼文武。

[8] 霸王杂用汉萧何:楚汉战争中,萧何辅佐汉高祖定三秦,后为汉相,制作律令,对汉王朝的建立和巩固卓有贡献。霸王杂用,是说“王道”和“霸道”两用。儒家称以力假仁者为霸,以德行仁政者为王。

[9] 月本阴精,岂有羿妻曾窃药:古代神话传说,有穷国君后羿从西王母那里得到了长生药,其妻嫦娥窃之服用后飞升到月宫。本联意为,月本是阴气的精华,哪里有嫦娥飞升的事呢?

[10] 星为夜宿,浪传织女漫投梭:古代神话说,织女是天帝的孙女,整夜在那里织布。本联意为,世传牛郎织女隔天以梭相投。这种说法也是荒诞虚无的事。夜宿,夜间的星宿。浪传,胡传,乱传。

[11] 婆娑:树木或人的身躯摇曳多姿的样子。

[12] 长杨:汉宫殿名。细柳:周亚夫曾屯军细柳。

[13] 寒莎:秋天的莎草。

[14] 追风马:《淮南子》中有“以兔之走,使犬如马则逮日归(追)风”的说法,后常以追风形容马跑得快。

[15] 挽日戈:古代神话传说,楚国的鲁阳公与韩国人作战,战到天晚未分胜负,他举起戈来向太阳下令,太阳从西方退了回来,他继续战斗。

[16] 玉液:古人服食的用玉屑调成的药酒。金波:太阳照在水面或宫殿上反射回来的光线。

[17] 紫诏衔丹凤:《晋书·石季龙载记》说,当时诏书以五色纸衔木凤之口,后世遂称皇帝诏令为凤诏。又解衔丹凤:古人书信用泥封,泥上盖印,皇帝诏书则用紫泥,称为紫泥诏或紫诏,常以龙凤为图饰。

[18] 黄庭换白鹅:晋书法家王羲之喜欢山阴道士养的鹅,于是为道士写了一卷《黄庭经》做为交换条件。

[19] 画阁江城梅作调:这是对李白“黄鹤楼中吹玉笛,江城五月落梅花”两句诗的概括。梅作调,古代笛曲名有《梅花落》。

[20] 竹为歌:此指歌咏民俗风土人情的《竹枝词》。

[21] 门外雪飞,错认空中飘柳絮:晋才女谢道韫,有才辩,一次降雪,他的叔父谢安问子侄们:“大雪纷纭何所似?”谢朗说:“撒盐空中差可拟。”谢道韫说:“未若柳絮因风起。”谢安十分赞赏。

[22] 岩边瀑响,误疑天半落银河:语出李白《观庐山瀑布》:“飞流直下三千尺,疑是银河落九天。”

[23] 荇:多年生草本植物。

[24] 薜荔:南方的一种蔓生植物。

[25] 梯云:犹言登云。步月:在月光下散步。

[26] 升鼎雉:传说殷王武丁时祭祀太庙,有野鸡飞落鼎耳上而鸣,古人认为是一种祥瑞。

[27] 听经鹅:佛教传说,僧志违养鹅能听经说法。

[28] 北海:后汉孔融曾为北海太守,时人称之为北海,好宴客。他是当时著名的文人。东坡:宋代诗人苏轼,在黄冈东坡筑室,号东坡居士。

[29] 吴郎哀废宅:吴郎指唐代吴融,他曾写有《废宅》诗:“风飘碧瓦雨摧垣,却有邻人与锁门。”

[30] 邵子乐行窝:宋经学家邵雍隐居不仕,居洛阳三十年,筑“安乐窝”以居,自称安乐先生。

[31] 丽水良金皆待冶,昆山美玉总须磨:旧传金生丽水,玉出昆仑。

[32] 雨过皇州,琉璃色灿华清瓦;风来帝苑,荷芰香飘太液波:描写风雨中帝都景象。太液,即太液池,西汉时在长安掘成的人造湖。华清,即华清宫,在金陵,六朝陈时所建。

[33] 及第:指科举考试考中,特指考中进士,明清两代只用于殿试前三名。登科:科举时代应考人被录取。

[34] 冰清、玉润:晋乐广、卫玠翁婿俱有名,时人称乐广为冰清,其婿卫玠为玉润,喻人品高洁。

[35] 地利、人和:语出《孟子·公孙丑下》:天时不如地利,地利不如人和。

[36] 韩擒虎:隋朝大将,屡立战功,渡江平陈战役就是由他统帅的。

[37] 荣驾鹅:春秋时鲁昭公之大臣。

[38] 青女:传说中的霜神。素娥:即嫦娥,月色白,故又称素娥。李商隐诗:“青女素娥俱耐冷,月中霜里斗婵娟。”

[39] 破头朱泚笏:唐德宗时,京师兵变,德宗出逃,太尉朱泚欲窃位,司农卿段秀实执象笏击破其头,卒遭所害。笏,古代大臣登朝所持用以记事的手板。

[40] 折齿谢鲲梭:《晋书·谢鲲传》:“邻家高氏女有美色,鲲尝挑之,女投梭,折其两齿。”

[41] 沧江积雪,惟看滩上一渔蓑:唐柳宗元《江雪》诗:“孤舟蓑笠翁,独钓寒江雪。”





\chapter{麻}


清对浊zhuó ,美měi 对嘉。鄙吝lìn 对矜夸kuā [1] 。花须xū 对柳眼yǎn [2] ,屋角jiǎo 对檐牙yá [3] 。志和宅zhái [4] ,博望槎chá [5] 。秋实对春华[6] 。乾炉lú 烹pēng 白雪,坤kūn 鼎dǐng 炼liàn 丹dān 砂[7] 。深宵xiāo 望冷沙场chǎng 月,边塞sài 听残cán 野戍笳jiā 。满mǎn 院yuàn 松风,钟声隐隐为僧舍shè [8] ;半窗花月,锡影yǐng 依依是道家。

雷léi 对电diàn ,雾对霞xiá 。蚁阵zhèn 对蜂衙yá [9] 。寄梅对怀橘jú ,酿niàng 酒对烹pēng 茶。宜男草[10] ,益母花[11] 。杨柳对蒹葭[12] 。班姬辞帝辇niǎn [13] ,蔡cài 琰yǎn 泣胡hú 笳[14] 。舞榭歌楼千万尺chǐ ,竹篱lí 茅máo 舍shè 两三家[15] 。珊枕[16] 半床c,月明时梦飞塞sài 外;银筝zhēng 一奏zòu ,花落处人在天涯yá 。

圆对缺quē ,正对斜xié 。笑语对咨zī 嗟jiē [17] 。沈shěn 腰对潘pān 鬓bìn [18] ,孟笋sǔn 对卢lú 茶[19] 。百舌鸟niǎo [20] ,两头蛇[21] 。帝里[22] 对仙家。尧仁敷率shuài 土,舜shùn 德被bèi 流沙[23] 。桥上授书曾纳nà 履lǚ [24] ,壁间题tí 句已笼lǒng 纱[25] 。远塞sài 迢tiáo 迢tiáo ,露碛[26] 风沙何可kě 极jí ;长沙渺miǎo 渺miǎo ,雪涛tāo 烟浪làng 信xìn 无涯yá 。

疏对密,朴pǔ 对华。义鹘gǔ 对慈鸦yā [27] 。鹤hè 群对雁阵zhèn ,白苎zhù 对黄麻má [28] 。读三到[29] ,吟八叉chā [30] 。肃sù 静对喧xuān 哗。围棋兼把钓diào ,沉李并浮瓜[31] 。羽客片时能煮石[32] ,狐hú 禅千劫jié 似蒸zhēng 沙[33] 。党dǎng 尉粗cū 豪,金帐笼lǒng 香斟美měi 酒;陶生清逸,银铛chēng 融雪啜chuò 团tuán 茶[34] 。



* * *



[1] 鄙吝:形容心胸狭窄。矜夸:夸耀。

[2] 花须:花蕊伸展如须。柳眼:柳叶如眉眼。

[3] 檐牙:檐际翘出如牙的部分。

[4] 志和宅:唐诗人张志和,肃宗朝命待诏翰林,授左金吾卫录事参军,后遭贬黜,遂不复仕。浪迹江湖,言以太虚(天)为庐,明月为伴,自号烟波钓徒。

[5] 博望槎:博望,即张骞,因奉使西域有功封博望侯。《荆楚岁时记》:“汉武帝令张骞使大夏,寻河源。乘槎经月,而至一处,见城郭和州府,室内有一女织,又见一丈夫牵牛饮河。骞问曰:‘此是何处?’答曰:‘可问严君平。’织女取榰机石与骞而还。”始知已到牛郎、织女星。槎,木筏。

[6] 秋实、春华:即春华秋实,古人比喻文采与德行。

[7] 乾炉烹白雪,坤鼎炼丹砂:都是道教说法。乾炉指男,坤鼎指女。

[8] 锡影依依是道家:僧人所持杖称锡。依依,隐隐约约的样子。

[9] 蚁阵:蚂蚁排阵而战。引申为争强斗胜。蜂衙:蜂早晚定时的聚集,如下属参谒长官于衙中,故称为蜂衙。

[10] 宜男草:即萱草,古人以为孕妇佩之可生男。

[11] 益母花:中药名。

[12] 蒹葭:即芦苇。

[13] 班姬辞帝辇:汉成帝游后苑,命班婕妤同辇,班婕妤说:“古代圣贤之君,都有名臣在旁;只有末代皇帝才亲近女色。”成帝听了很钦佩。

[14] 蔡琰泣胡笳:蔡琰,即蔡文姬,蔡邕女,汉末著名才女,早寡,汉末被虏入胡,在南匈奴生活了十二年,后被曹操赎回。传说她曾写了《胡笳十八拍》,历述她的不幸遭遇。

[15] 竹篱:用竹编的篱笆。茅舍:茅屋,草屋。

[16] 珊枕:即珊瑚枕。

[17] 咨嗟:文言叹词,叹息。

[18] 沈腰:南朝梁文学家沈约,字休文,体弱多病,腰肢纤弱。潘鬓:晋文学家潘岳,由于屡遭不幸,身体早衰,在《秋兴赋》中,他曾自伤两鬓早白,说自己三十二岁“始见二毛”。

[19] 孟笋:孟宗母病中喜吃笋,因时节正值冬季,无笋可取,宗入竹林悲泣哀叹,笋竟为之而生。后人遂用来形容人子事亲尽孝,至诚感天,并将之列入“二十四孝”中。卢茶:唐代诗人卢仝好茶成癖,诗风浪漫。

[20] 百舌鸟:鸟名,又名乌鸫(dōng)。益鸟,喙尖,毛色黑黄相杂,鸣声圆滑。

[21] 两头蛇:见肴韵“断蛇”句注。

[22] 帝里:犹言帝乡,指上帝所居之处。

[23] 尧仁敷率土,舜德被流沙:都是对尧舜的称颂。敷率土,是说遍及所有的地方。流沙,古人指中国以西极远的地区。

[24] 桥上授书曾纳履:传说张良年轻时曾遇到一位坐在下邳圯(桥)上的老人,命他到桥下去取失落的鞋,张良恭恭敬敬地做了这件事,老人很高兴,说孺子可教也,就授予他三卷兵书,并说自己就是黄石公。纳履,穿鞋。

[25] 壁间题句已笼纱:唐代王播少孤贫,客居扬州惠招寺木兰院,随僧斋食,为诸僧所不礼。后播显贵重游旧地,见昔日在该寺壁上所题诗句,僧已用碧纱盖护,因题曰:“上堂已散各西东,惭愧阇梨饭后钟。三十年来尘扑面,如今始得碧纱笼。”

[26] 碛:水中堆沙。

[27] 义鹘:鹰类鸷禽。鸷,凶猛的鸟。慈鸦:古人传说乌鸦是孝鸟,老鸟不能取食时,小鸟能反哺其母,因称慈鸦。

[28] 苎:一种麻类,皮可为纺织原料。黄麻:此指黄麻纸,唐时以黄麻纸写诏书。

[29] 读三到:古人经验,读书要眼到、口到、心到。

[30] 吟八叉:唐诗人温庭筠才思敏捷,传说他八叉其手而诗成,人呼之为温八叉。

[31] 沉李并浮瓜:古人消暑,往往置水果于冷水中,故有沉李浮瓜之说。

[32] 羽客片时能煮石:羽客,即仙人。道教说仙人能煮白石为饭。

[33] 狐禅千劫似蒸沙:佛教说法,狐禅毫无意义,犹如蒸沙土,虽历尽千劫,不能成饭。佛经云,狐禅如蒸沙,千劫不能成饭。

[34] 党尉粗豪,金帐笼香斟美酒;陶生清逸,银铛融雪啜团茶:《事文类聚》载,宋学士陶毂得党太尉家姬。一次烹雪茶,陶问姬曰:“党家有此味否?”姬曰:“彼但知坐销金帐里,共饮羊羔美酒,浅斟低唱而已。”铛,平底锅。





\chapter{阳}


台对阁,沼zhǎo 对塘。朝zhāo 雨对夕阳。游yóu 人对隐士,谢女对秋娘niáng [1] 。三寸cùn 舌[2] ,九回肠[3] 。玉液对琼qióng 浆[4] 。秦皇照胆dǎn 镜[5] ,徐xú 肇zhào 返魂香[6] 。青萍[7] 夜啸芙蓉匣xiá ,黄卷juàn [8] 时摊薜bì 荔床chuáng 。元亨hēng 利贞,天地一机成化育[9] ;仁义礼智,圣贤千古立纲gāng 常。

红对白,绿对黄。昼zhòu 永对更gēng 长。龙飞对凤舞,锦jǐn 缆对牙yá 樯qiáng [10] 。云弁biàn 使[11] ,雪衣娘niáng [12] ,故gù 国对他乡。雄文能徙xǐ 鳄è [13] ,艳曲为求凰[14] 。九日高峰惊落帽mào [15] ,暮春曲水喜流觞shāng [16] 。僧占zhàn 名山,云绕rào 茂mào 林藏古殿diàn ;客栖胜地[17] ,风飘落叶响空廊láng 。

衰shuāi 对壮zhuàng ,弱ruò 对强qiáng 。艳饰对新妆[18] 。御龙对司马[19] ,破pò 竹对穿杨[20] 。读班马[21] ,识求羊[22] 。水色对山光[23] 。仙棋藏绿橘jú [24] ,客枕梦黄粱。池chí 草入诗因有梦[25] ,海棠带恨为无香[26] 。风起画堂[27] ,帘箔[28] 影yǐng 翻fān 青荇沼zhǎo ;月斜xié 金井jǐng [29] ,辘轳lú [30] 声度碧梧墙qiáng 。

臣对子,帝对王。日月对风霜。乌台对紫府[31] ,雪牖对云房fáng [32] 。香山社shè ,昼zhòu 锦jǐn 堂[33] 。蔀bù 屋对岩廊láng [34] 。芬椒涂tú 内nèi 壁[35] ,文杏饰高粱[36] 。贫pín 女幸分东壁影yǐng [37] ,幽人高卧wò 北窗凉[38] 。绣xiù 阁探tàn 春,丽日半笼lǒng 青镜色[39] ;水亭醉夏,薰xūn 风常透tòu 碧筒tǒng [40] 香。


[1] 谢女:指晋代才女谢道韫,人称咏絮高才。秋娘:即杜秋娘,唐宗室李锜(qí)妾,能诗。

[2] 三寸舌:指能说善辩。史载战国时毛遂以三寸之舌,强于百万之师。

[3] 九回肠:形容人心情郁闷。

[4] 玉液、琼浆:都是道教服食的药饵。

[5] 秦皇照胆镜:传说秦始皇有照胆镜,能透视人的内脏,发现有人胆张心动,就意味着要暗害他,当即杀掉。

[6] 徐肇返魂香:《十洲记》载,西海申未洲上有大树,叶香闻数百里,煎制成膏,名返生香,死尸在地,闻之可活。又释徐肇遇苏德音,授以返魂香,燃之,能起上世亡魂。

[7] 青萍:宝剑名。

[8] 黄卷:用绢书写的书籍,此指道书或佛经。

[9] 元亨利贞,天地一机成化育:元亨利贞,是《周易·乾卦》中的一句。古人解释说:“元者善之长也,亨者嘉之会也,利者义之和也,贞者事之干也。”称为四德。二句的意思是,由于天地有此四德,才化生了万物。

[10] 锦缆对牙樯:用锦缎做缆绳,以象牙为樯橹。樯,桅杆。

[11] 云弁使:指蜻蜓。

[12] 雪衣娘:白鹦鹉。

[13] 雄文能徙鳄:潮州有鳄鱼为害,韩愈做刺史,作《祭鳄鱼文》驱之,传说鳄鱼就迁到了它地。

[14] 艳曲为求凰:汉时成都卓王孙有女文君新寡,司马相如爱上了她,作《凤求凰》曲以挑之,文君于是同他私奔。

[15] 九日高峰惊落帽:晋孟嘉为桓温之参军,九月九日游龙山,群僚毕集,有风将孟嘉帽子吹落而不觉。孙盛作文嘲笑,他即时作答,四座皆服。

[16] 暮春曲水喜流觞:晋永和上巳日农历三月初三,王羲之、王献之、谢安、孙绰诸人曾在山阴兰亭集会,于水边嬉游采兰,曲水流觞,饮酒赋诗以娱,以消除不祥,称为修禊。王羲之有《兰亭集序》记此事,文中有“暮春之初”“引以为流觞曲水”等语。

[17] 胜地:著名的景色宜人的地方。

[18] 艳饰:犹言浓妆打扮。新妆:指女子刚修饰好的仪容,或指女子新颖别致的打扮修饰。

[19] 御龙:驾御龙。传说夏时刘累曾为孔甲养龙,因赐姓为御龙氏。司马:官名,也是姓。

[20] 破竹:比喻做事顺利。穿杨:传说楚将养由基善射,百步之内,可穿杨叶。

[21] 读班马:班固作《汉书》,司马迁作《史记》。

[22] 识求羊:西汉末,蒋诩解官归桂林后,于竹林中开三条小径,惟故人求仲、羊仲从之游,不与俗人往还。

[23] 水色:水面呈现的色泽。山光:山的景色。

[24] 仙棋藏绿橘:神话故事,巴邛人家有橘树,一年忽长三枚,果实大如斗,剖之有二叟对弈。

[25] 池草入诗因有梦:传说南朝宋诗人谢灵运一次生病,因梦见族弟惠连而得“池塘生春草,园柳变鸣禽”之佳句。

[26] 海棠带恨为无香:宋彭渊林曰:吾生平五恨。一恨鱼多骨,二恨橘多酸,三恨菜性淡,四恨海棠无香,五恨曾子固不能诗。曾子固,曾巩字子固,古文“唐宋八大家”之一。

[27] 画堂:泛指华丽的堂舍。

[28] 帘箔:帘子。多以竹、苇编成。

[29] 金井:井栏上有雕饰的井。一般用以指宫庭园林里的井。

[30] 辘轳:井上的汲水器。

[31] 乌台:《汉书·朱博传》载,时御史府中列柏树,常有野乌数千栖息其上,后因称御史府(台)为乌台。紫府:道家称仙人居所。

[32] 雪牖:雪窗。云房:僧、道或隐者所居之室。

[33] 香山社:唐白居易于洛阳与胡杲(gǎo)、吉皎等八位老人结为九老会,因结于香山,故称为香山九老社。昼锦堂:北宋韩琦封魏国公,在做武康节度使时,于故乡相州修了一所殿堂,取名昼锦堂以致其荣,致仕退老其中。文学家欧阳修曾写有《昼锦堂记》,详述其事。

[34] 蔀屋:指草屋。岩廊:高大的宫殿。

[35] 芬椒涂内壁:汉代皇后所居宫室,以椒和泥涂内壁,取其香和多子之意,称椒房。

[36] 文杏饰高粱:旧题司马相如《长门赋》:“饰文杏以为梁。”后以杏梁指建筑华美。

[37] 贫女幸分东壁影:《战国策》载寓言故事,齐女与邻妇共烛而绩,妇辞之,女曰:“我贫无烛。一室之中,多不为暗,少不为明,何惜东壁余光。”邻妇觉得有理,就留下了她。唐李白诗:“愿假东壁辉,余光照贫女。”

[38] 幽人高卧北窗凉:晋代陶潜《与子俨等疏》:“常言五六月中,北窗下卧,遇凉风暂至,自谓是羲皇上人。”意思是说他自己夏日卧北窗下,每当凉风吹来,就好像回到了无忧无虑的太古时代一样。

[39] 绣阁:旧时女子闺房。青镜:即青铜镜。

[40] 碧筒:三国魏郑悫(què)取荷茎通之以盛酒,名曰碧筒杯。





\chapter{庚}

\begin{yuanwen}
形对貌mào ,色对声。夏邑对周京。江云对涧树,玉磬qìng 对银筝zhēng [1] 。人老老[2] ,我卿卿[3] 。晓燕对春莺yīng 。玄xuán 霜舂chōng 玉杵[4] ,白露贮zhù 金茎[5] 。贾客君山秋弄nòng 笛dí [6] ,仙人缑gōu 岭夜吹笙[7] 。帝业独兴,尽道汉高能用yòng 将[8] ;父书空读,谁言赵括kuò 善知兵[9] 。
\end{yuanwen}

\begin{yuanwen}
功对业,性对情qíng 。月上对云行。乘龙对附骥jì [10] ,阆làng 苑yuàn 对蓬péng 瀛yíng [11] 。春秋笔[12] ,月旦dàn 评[13] 。东作对西成[14] 。隋suí 珠光照乘[15] ,和璧价连城。三箭三人唐将勇[16] ,一琴一鹤hè 赵公清[17] 。汉帝求贤,诏访严滩逢féng 故gù 旧[18] ;宋sòng 廷优老,年尊zūn 洛社shè 重zhòng 耆英yīng [19] 。
\end{yuanwen}

\begin{yuanwen}
昏hūn 对旦dàn ,晦huì [20] 对明。久雨对新晴qíng 。蓼liǎo 湾wān 对花港,竹友对梅兄xiōng 。黄石叟sǒu [21] ,丹dān 丘生[22] 。犬吠对鸡鸣。暮山云外断,新水月中平。半榻tà 清风宜午梦,一犁lí 好雨趁chèn 春耕gēng 。王旦dàn 登庸yōng ,误我十年迟chí 作相[23] ;刘fén 不第,愧kuì 他多士早成名[24] 。

\end{yuanwen}



[1] 玉磬:古代石制乐器名。银筝:用银装饰的筝或用银字表示音调高低的筝。

[2] 人老老:语出《孟子》:“老吾老,以及人之老;幼吾幼,以及人之幼。”意思是,尊敬自己的老人,从而也尊敬别人的老人;爱自己的孩子,从而也爱别人的孩子。人老老,即尊敬别人的老人。第一个“老”字作动词用。

[3] 我卿卿:卿是对人的尊称,也是对妻子的昵称。西晋大臣王戎妻呼戎曰卿。戎曰:“奈何卿我?”妻曰:“我不卿卿,谁当卿卿?”(意为我不称你为卿,还有谁称你为卿呢)?故后以“卿卿我我”作为夫妻恩爱之典。

[4] 玄霜舂玉杵:唐裴銏《传奇》中讲一个故事,下第秀才裴航,遇到仙人云翘夫人,赠诗一首曰:“一饮琼浆百感生,玄霜捣尽见云英。蓝桥便是神仙窟,何必崎岖上玉京。”后裴生经蓝桥驿,果遇一妪揖之求饮,妪使云英持瓯浆,令饮之。因诗合,欲娶云英,妪命裴购玉杵并捣药,果得玉杵。聘之,俱仙去。玄霜,传说中的仙药。

[5] 白露贮金茎:汉武帝好神仙之术,史载他曾作铜柱,上有铜仙人擎玉盘,承接夜露,据说以此露和玉屑饮之可长生。杜甫诗:“蓬莱宫阙对南山,承露金茎霄汉间。”又魏明帝亦作承露金茎,高十一丈。

[6] 贾(gǔ)客君山秋弄笛:《博异志》载,有商人吕乡筠,善吹笛,一次泊舟君山附近,遇到一位老人,合上天神乐、仙乐和自己欣赏的三支仙笛,吹奏数声,湖上风波大作。

[7] 仙人缑岭夜吹笙:传说周灵王太子晋好吹笙,作《凤凰鸣》,遇浮邱公,接上蒿山。后于七月七日乘白鹤过缑氏山头,拱手谢别时人而去。缑岭,山名,在河南。

[8] 帝业独兴,尽道汉高能用将:史载汉高祖刘邦善于用人,因而取得天下。汉高帝问韩信带兵几何?信曰:“多多益善。”帝曰:“卿何为我擒耶?”曰:“陛下不善将兵,而善将将。”

[9] 父书空读,谁言赵括善知兵:赵奢是战国时赵之名将。奢死,赵王令以其子赵括代廉颇为将。蔺相如说,赵括只能读其父的兵书,没有实际经验。赵王不听,使其率兵与秦交战。结果赵括中箭死,几十万军队都投降秦国,被秦人活埋了。

[10] 乘龙:唐杜甫《李监宅》:“门阑多喜气,女婿近乘龙。”因称女婿为乘龙快婿。附骥:靠别人的力量使自己得以发展,喻附于先辈或名人之后。

[11] 阆苑:传说中的仙境,在昆仑山上。蓬瀛:即蓬莱山,传说东海中的仙山。

[12] 春秋笔:旧说孔子作《春秋》,寓褒贬于字里行间,后称此种笔法为春秋笔。

[13] 月旦评:汉末河南许劭(shào)与其兄许靖俱有高名,好在一起甄别、评论当地人物,每月变换一次,农历初一发布公告,人们称之为“月旦评”。后称品评人物为月旦评或月旦。

[14] 东作、西成:《尚书·尧典》中有“平秩东作”“平秩西成”的话,“东作”是开始耕作,“西成”是收获之意。

[15] 隋珠光照乘:传说一次隋侯出行,遇断蛇于路,隋侯命人给蛇敷药包扎,后蛇衔径寸之珠报偿隋侯,因称隋侯珠。光照乘,是说把这种宝珠挂在车上可以照明前后。

[16] 三箭三人唐将勇:唐将薛仁贵东征与九姓突厥交战,三箭毙三人,威震军中。当时有歌谣曰:“将军三箭定天山,壮士长歌入汉关。”

[17] 一琴一鹤赵公清:宋赵汴治成都,匹马入蜀,以一琴一鹤相随,为政清廉简易。

[18] 汉帝求贤,诏访严滩逢故旧:即汉光武帝与严子陵的故事。见微韵“严滩”注。光武帝与严子陵友善,即位命访之,陵在富春江披蓑钓泽中,载以至朝,帝以故人礼敬之。尝以同寝,陵以足加腹。太史奏曰,有客星犯主座。

[19] 宋廷优老,年尊洛社重耆英:宋相文彦博,致仕后在洛阳同富弼、司马光等十三人,饮酒赋诗相乐,谓之耆英会。耆,年老。耆英,高年硕德的人。

[20] 晦:昏暗不明。

[21] 黄石叟:即汉初张良所遇仙人黄石公,曾赠给张良兵书。

[22] 丹丘生:道教传说中的仙人。丹丘,神话中的神仙之地,昼夜长明。

[23] 王旦登庸,误我十年迟作相:《宋史·王旦传》载,宋相王旦柄权十八年,死后,王钦若继为宰相。王钦若语人曰:“子明(即王旦)迟我十年作宰相。”登庸,选拔任用。

[24] 刘 不第,愧他多士早成名:见文韵“唐廷”注。





\chapter{青}

\begin{yuanwen}
庚gēng 对甲,己对丁。魏阙què 对彤庭[1] 。梅妻对鹤hè 子[2] ,珠箔对银屏[3] 。鸳yuān 浴沼zhǎo ,鹭飞汀。鸿雁对jí 鸰líng [4] 。人间寿者zhě 相[5] ,天上老人星[6] 。八月好修攀pān 桂guì 斧[7] ,三春须xū 系护hù 花铃[8] 。江阁凭临,一水净连天际碧;石栏闲倚,群山秀xiù 向雨余青。
\end{yuanwen}

\begin{yuanwen}
危对乱luàn ,泰对宁níng 。纳nà 陛对趋庭[9] 。金盘pán 对玉箸zhù ,泛fàn 梗gěng [10] 对浮萍。群玉圃pǔ [11] ,众zhòng 芳fāng 亭。旧典[12] 对新型。骑牛闲读史[13] ,牧豕shǐ 自横经[14] 。秋首田tián 中禾颖yǐng [15] 重zhòng ,春馀园内nèi 菜cài 花馨。旅lǚ 次cì [16] 凄凉,塞sài 月江风皆jiē 惨cǎn 淡dàn ;筵前欢huān 笑,燕歌赵舞独娉pīng 婷[17] 。

\end{yuanwen}

[1] 魏阙:高大的城阙。魏,通巍,形容高大。彤庭:指帝王宫殿。

[2] 梅妻、鹤子:宋林逋隐居西湖孤山,以梅鹤自娱。逋不娶,无子,时人说林“梅妻鹤子”。

[3] 珠箔、银屏:语出唐白居易《长恨歌》:“珠箔银屏迤逦开。”

[4] 鸰:鸟名,生活在水边,食小虫,喜欢群飞。

[5] 人间寿者相:旧时迷信,讲论骨相。寿者相,就是看上去长寿的相貌。

[6] 天上老人星:《史记·天官书》载天上有南极老人星,主寿。

[7] 八月好修攀桂斧:神话传说,汉人吴刚,因学仙有过,罚他砍月中桂树,桂树高五百尺,砍后伤口复合,所以吴刚要永远砍下去。旧时以科举登第为攀桂,考试一般定在八月,称“秋闱”。

[8] 三春须系护花铃:明代宁王爱花,尝作护花铃,蜂、鸟至则牵铃惊之。

[9] 纳陛:原意是深入殿堂的台阶,这里是登上台阶的意思。趋庭:快步走过庭院。《论语》记载:孔子的儿子孔鲤,一次趋庭而过,被孔子叫住,问他学诗学礼的情况。以后就把见父亲叫趋庭。

[10] 泛梗:《说苑》中的一则寓言,孟尝君入秦,客止之。见有木梗人谓土偶人曰:“今将大雨,子必沮坏。”答曰:“我沮,乃反吾真耳。今子,东园之桃也。刻子以为梗,雨至必浮,子泛泛不知所至矣。”孟尝君乃止。后遂以泛梗比喻到处漂流,无处安身。梗,这里指木偶。

[11] 群玉圃:传说仙人西王母居住在群玉山的瑶圃。

[12] 旧典:旧时的制度、法则。

[13] 骑牛闲读史:隋末李密好学,常将《汉书》一帙挂于牛角之上,骑牛读书。

[14] 牧豕自横经:汉公孙宏,年少时生活清贫,为人放猪,但自己勤奋学习,常带经卷读。年五十后位至丞相。

[15] 禾颖:带芒的谷穗。

[16] 旅次:旅途中小住的地方。也指旅途中暂作停留。

[17] 筵前欢笑,燕歌赵舞独娉婷:古代燕、赵多出歌伎,其人善歌舞。娉婷:舞姿优美的样子。





\chapter{蒸}

\begin{yuanwen}
萍对蓼liǎo[1] , jiǎo 对菱[2] 。雁弋[3] 对鱼罾zēng 。齐纨wán 对鲁绮qǐ ,蜀锦jǐn 对吴绫[4] 。星渐没mò ,日初升。九聘pìn 对三征zhēng [5] 。萧xiāo 何曾作吏[6] ,贾岛昔为僧[7] 。贤人视履lǚ 循xún 规矩[8] ,大匠挥斤校准zhǔn 绳shéng [9] 。野渡春风,人喜乘潮移酒舫[10] ;江天暮雨,客愁chóu 隔岸àn 对渔灯。
\end{yuanwen}

\begin{yuanwen}
谈对吐,谓对称chēng 。冉rǎn 闵mǐn 对颜曾zēng [11] 。侯hóu 嬴yíng 对伯嚭pǐ [12] ,祖逖tì 对孙登[13] 。抛pāo 白纻zhù [14] ,宴红绫[15] 。胜友对良朋péng 。争zhēng 名如逐鹿[16] ,谋móu 利似趋蝇yíng [17] 。仁杰jié 姨惭cán 周不仕[18] ,王陵母识汉方fāng 兴[19] 。句写穷qióng 愁chóu ,浣huàn 花寄迹传工部[20] ;诗吟变乱luàn ,凝níng 碧伤shāng 心叹tàn 右丞[21] 。
\end{yuanwen}


[1] 萍:水生植物。蓼:一年生或多年生草本植物。

[2] :疑为“茭”,茭白,菰的花径一种菌侵入后,刺激其细胞增生而成的肥大嫩茎,可作蔬菜。菱:一年生水生草本植物,果实有硬壳,有角,称“菱”或“菱角”,可食。

[3] 弋:一种尾上带绳子的箭。雁弋即射雁的这种箭。罾:一种用竹竿或木棍做的方形鱼网。

[4] 齐纨对鲁绮,蜀锦对吴绫:纨、绮、锦、绫都是名贵的丝织品;齐、鲁、蜀、吴是上述四种织品的产地。

[5] 九聘、三征:聘和征都是王朝或官府聘请的意思。九聘,多次聘请。三征,朝廷三次征召。

[6] 萧何曾作吏:史载萧何曾做沛郡的主吏椽,是管人事的小官。

[7] 贾岛昔为僧:唐诗人贾岛曾为僧人,法名无本。韩愈赏其诗才,令其还俗,劝其读书,后登进士,官长江主簿。

[8] 贤人视履循规矩:《尔雅·释言》:“履,礼也。”注:“礼可以履行也。”所以说视履成规矩。

[9] 大匠挥斤校准绳:《庄子》中的一则寓言说,郢人在鼻子尖上涂一点白土,一位石匠把父子抡得呼呼响,一下子就把泥点砍掉了,对鼻子丝毫无损。大匠,技术高超的匠人。斤,斧子的一种。

[10] 舫:船,画舫(装饰华美专供旅游用的船);酒舫,载酒或卖酒的船。

[11] 冉闵对颜曾:冉有、闵子骞、颜渊、曾参都是孔子的高足弟子。

[12] 侯嬴:战国时魏人,初为大梁(今河南开封)夷门的守门小吏,慷慨任侠,帮助信陵君窃符救赵,最后以身殉之。王维《夷门歌》专咏此事。伯嚭:即太宰嚭,春秋时楚伯州犁之孙,吴国奸臣。他受越王贿赂,劝吴王同越讲和。勾践灭吴,以伯嚭对其主不忠,杀之。

[13] 祖逖:东晋时爱国将领。见先韵“祖生鞭”注。孙登:晋初隐士。

[14] 抛白纻:宋裴思谦登第,以红笺数十幅入平康赋诗。王元之有诗云:“利市襕衫抛白纻,风流名字写红笺。”白纻,白苎麻织成的衣服。白纻襕衫,唐举子之服。

[15] 宴红绫:唐御膳以红绫饼为重。昭宗时放进士榜,得裴格等二十八人,会宴曲江,命御厨烧作红绫饼二十八枚赐之。

[16] 逐鹿:逐鹿中原,原指在战场上争夺政权。后来又有“未知鹿死谁手”的话,比喻胜负难定,这里即用此意。

[17] 趋蝇:追赶苍蝇。古有“蝇头微利”的说法,“趋蝇”是说十分不值得。

[18] 仁杰姨惭周不仕:唐狄仁杰为武后相,其姨卢氏有子,杰欲官之,姨曰:“姨止一子,不欲令事后周女主。”仁杰大惭而归。周,武则天的国号。

[19] 王陵母识汉方兴:王陵事汉,其母在楚,知汉必兴,嘱善事之。项羽令母召陵,母遂自刎。

[20] 句写穷愁,浣花寄迹传工部:这是写杜甫的事,杜拾遗曾为检校员外郎,后人称之为杜工部。晚年流落蜀中,寓居成都西郊浣花溪旁之浣花村草堂。

[21] 诗吟变乱,凝碧伤心叹右丞:王维官尚书右丞相,后人称之为王右丞。安史之乱陷身贼中,被迫为给事中。传说安禄山宴于凝碧宫,令乐人作乐,维闻而伤之,作七绝一首云:“万户伤心生野烟,百僚何日更朝天。秋槐叶落空宫里,凝碧池头奏管弦。”





\chapter{尤}

\begin{yuanwen}
荣对辱rǔ ,喜对忧。缱qiǎn 绻对绸chóu 缪móu [1] 。吴娃wá 对越女[2] ,野马对沙鸥ōu [3] 。茶解渴kě ,酒消xiāo 愁chóu 。白眼yǎn 对苍头[4] 。马迁修《史记》,孔子作《春秋》。莘野耕gēng 夫闲举耜sì [5] ,渭滨渔父晚wǎn 垂钩[6] 。龙马游yóu 河,羲帝因图tú 而画卦guà [7] ;神龟出洛,禹王取法fǎ 以明畴chóu [8] 。
\end{yuanwen}

\begin{yuanwen}
冠对履lǚ ,舄xì [9] 对裘。院yuàn 小对庭幽。面墙qiáng 对膝地[10] ,错智对良筹chóu [11] 。孤gū 嶂耸sǒng ,大江流。芳fāng 泽zé 对圆丘[12] 。花潭来越唱,柳屿起吴讴ōu [13] 。莺yīng 懒燕忙三月雨,蛩qióng 摧cuī 蝉退tuì 一天秋。钟子听琴,荒huāng 径入林山寂寂[14] ;谪仙捉zhuō 月,洪涛tāo 接jiē 岸水悠悠[15] 。
\end{yuanwen}

\begin{yuanwen}
鱼对鸟niǎo ,鹡jí 对鸠jiū 。翠馆guǎn 对红楼[16] 。七贤对三友[17] ,爱ài 日[18] 对悲秋。虎类狗[19] ,蚁如牛[20] 。列liè 辟[21] 对诸侯hóu 。陈唱临春乐[22] ,隋suí 歌清夜游yóu [23] 。空中事业麒麟阁[24] ,地下文章zhāng 鹦yīng 鹉洲[25] 。旷kuàng 野平原,猎liè 士马蹄tí 轻似箭;斜xié 风细雨,牧童牛背bèi 稳wěn 如舟。
\end{yuanwen}


[1] 缱绻、绸缪:都是形容感情亲密、情意缠绵的样子。

[2] 吴娃:吴地的姑娘。娃,少女。越女:古代越国多出美女,西施其尤著者。后因以泛指越地美女。

[3] 野马:《庄子·逍遥游》中说:“野马也,尘埃也,生物之以息相吹也。”野马说的是早春大地上蒸腾的水蒸气。沙鸥:指栖息沙洲的鸥一类的水鸟。

[4] 苍头:在秦末农民大起义中,有一支义军的士卒以青巾裹头,称苍头军。后世苍头多指老年仆人。

[5] 莘野耕夫闲举耜:此句疑用伊尹故事。《吕氏春秋》说:有侁(shēn)氏女子得婴儿于空桑之中,名伊尹,长而贤,商汤王准备聘请他,有侁氏不肯,汤于是聘有侁氏女,以伊尹为陪嫁奴隶取了去,后以为相,国大治。有侁氏即有莘氏。

[6] 渭滨渔父晚垂钩:指商代末年姜尚的故事。见萧韵“千载”注。

[7] 龙马游河,羲帝因图而画卦:见鱼韵“洛龟”注。

[8] 神龟出洛,禹王取法以明畴:上古传说,夏禹曾参照洛水神龟献出的宝书,制定了“洪范九畴”。

[9] 舄:鞋。

[10] 面墙:《论语》记述孔子的话说:“人而不为《周南》《如南》,其犹正墙面而立也与?”后来“面墙”就成了思路闭塞的代用语。膝地:两膝着地。

[11] 错智对良筹:错指西汉政治家晁错,他在文帝时曾为太子家令。太子家令是主管太子府内庶务的官员,相当于太子府的总管,很有谋略,多智,大家称他为“智囊”。良指张良。良筹是说张良的高明策略。又解为汉初张良借箸筹画政事。

[12] 芳泽:泽本是妇女用的脂粉,或说内衣,后芳泽即转为女性的代称。圆丘:是古代天子祭祀天神的地方,也写作圜丘。

[13] 吴讴:吴地的民歌。

[14] 钟子听琴,荒径入林山寂寂:上古故事,俞伯牙善于弹琴,钟子期善解琴,闻伯牙鼓“高山流水”曲,遂相知好。子期死,伯牙碎琴不复鼓,谓无知音也。

[15] 谪仙捉月,洪涛接岸水悠悠:古代民间传说,诗人李白特别喜爱明月,在采石矶,一次酒醉,看到江心倒映的月影,就前去扑捉,结果溺水而死。谪:封建时代特指贬官。

[16] 翠馆:犹青楼,妓院。红楼:犹青楼。妓女所居。

[17] 七贤:晋嵇康与阮籍、山涛、向秀、阮咸、王戎、刘伶友好,常宴集于竹林之下,号为竹林七贤。三友:以三种事物为友,如松、竹、梅;琴、酒、诗;梅、石、竹等。

[18] 爱日:珍惜时间。悲秋:看到秋天草木凋零而感到伤悲。

[19] 虎类狗:东汉马援在《戒兄子严敦书》中,告诫他们说,学龙伯高,不成犹为谨慎之士,所谓刻鹄不成尚类鹜;学习豪侠好义的杜季良,不成刚为天下轻薄子,所谓画虎不成反类狗。

[20] 蚁如牛:晋殷浩患耳疾,听见床下蚂蚁动,以为是牛斗之声。

[21] 列辟:诸王侯。

[22] 陈唱临春乐:南朝陈后主荒淫,修结绮、临春、望仙阁,与张丽华、江总、孔贵嫔诸人日夜游戏、歌唱,其中以《玉树后庭花》《临春乐》为最有名。

[23] 隋歌清夜游:传说隋炀帝夏夜宴游,放萤火虫照明,歌清夜之曲;冬日剪彩为花。

[24] 空中事业麒麟阁:汉宣帝时,为了表彰功臣,将霍光、苏武等画在麒麟阁上,共十一人。“空中事业”,是说功名富贵本来是虚幻的,这是作者的消极思想。

[25] 地下文章鹦鹉洲:三国时才士祢衡,因反对曹操,被排挤到荆州,后被刘表部下黄祖(忌其才)杀害。他曾写过《白鹦鹉赋》,因此人们把它被害之处称之为鹦鹉洲。“地下文章”是说该人已死去。





\chapter{侵}

\begin{yuanwen}
歌对曲,啸对吟[1] 。往古对来今。山头对水面,远浦pǔ 对遥岑cén [2] 。勤三上,惜寸cùn 阴[3] 。茂mào 树对平林[4] 。卞和三献xiàn 玉[5] ,杨震zhèn 四知金。青皇风暖吹芳fāng 草[6] ,白帝城高急jí 暮砧[7] 。绣xiù 虎雕diāo 龙,才子窗前挥huī 彩笔[8] ;描鸾luán 刺cì 凤,佳人帘下度金针。
\end{yuanwen}

\begin{yuanwen}
登对眺tiào ,涉shè 对临[9] 。瑞ruì 雪对甘霖[10] 。主欢huān 对民乐,交浅qiǎn 对言深[11] 。耻chǐ 三战zhàn ,乐七擒[12] 。顾gù 曲[13] 对知音。大车chē 行槛槛[14] ,驷sì 马骤zhòu 骎qīn 骎qīn [15] 。紫电diàn 青虹腾téng 剑气[16] ,高山流水识琴心[17] 。屈子怀君,极jí 浦pǔ 吟风悲泽zé 畔pàn [18] ;王郎láng 忆友,扁舟卧wò 雪访山阴[19] 。
\end{yuanwen}


[1] 啸:撮口作声,打口哨。吟:声调抑扬地念。

[2] 遥岑:远处陡峭的小山崖。

[3] 勤三上:古人经验,认为善读者有“三上”之功,即枕上、途上、厕上。惜寸阴:东晋大将陶侃致力收复中原,朝夕运甓(pì,砖),常勉励大家说:“大禹惜寸阴,吾人当惜分阴。”寸阴,很短的时光。

[4] 平林:平原上的林木。

[5] 卞和三献玉:见庚韵“和璧”句注。卞和即和氏。

[6] 青皇:又称东皇、青帝。东方为春,古人所谓司春之神,故代指春天。

[7] 白帝城高急暮砧:唐杜甫《秋兴八首》诗:“寒衣处处催刀尺,白帝城高急暮砧。”白帝城在四川重庆市奉节县,三国刘备殁于此。砧,捣衣石,这里指砧杵之声。

[8] 绣虎雕龙,才子窗前挥彩笔:曹子建有奇才,七步成诗,人称绣虎之手。雕龙:南朝梁刘勰《文心雕龙》论古今文章的体裁和创作,有很高价值。

[9] 眺:望,往远处看。涉:到,经历。

[10] 瑞雪:应时的好雪。甘霖:久旱后下的雨;及时雨。

[11] 交浅对言深:战国时范睢说秦王,有“交疏”“言深”等语。交浅犹交疏,是说人与人的关系很一般,没有深交。言深,是讲了至关重要的意见。

[12] 三战:传说春秋时鲁国将军曹刿,曾三次兵败于齐。后来齐桓公和鲁庄公盟于柯,曹刿用匕首逼迫齐桓公,终于索回了失去的国土。七擒:传说孔明征南蛮,曾对其首领孟获七擒七纵,使孟获受到感化,最后归顺。

[13] 顾曲:《三国志·周瑜传》载,三国吴周瑜善审音律,曲有阙误,瑜必知之,知之必顾,故时人谣曰:“曲有误,周郎顾。”唐李瑞诗:“欲得周郎顾,时时误拂弦。”

[14] 大车行槛槛:语出《诗经·王风·大车》:“大车槛槛。”大车是上古载重的牛车。槛槛,车声。

[15] 驷马骤骎骎:语出《诗经·小雅·四牡》:“驾彼四骆,载骤骎骎。”驷马,上古一车四马。骤,奔驰。骎骎,马跑得很快的样子。

[16] 紫电青虹:形容宝剑的光华。

[17] 高山流水识琴心:这是关于钟子期、俞伯牙的故事。参见尤韵“钟子”注。据说一次伯牙弹琴,钟子期评论说,此曲“志在高山”;再弹,又评论说,此曲“志在流水”。琴心,琴曲的内容、主题。

[18] 屈子怀君,极浦吟风悲泽畔:见豪韵“遭谗”注。极浦,犹言远浦,远方的水滨。

[19] 王郎忆友,扁舟卧雪访山阴:见豪韵“回艚”注。





\chapter{覃}

\begin{yuanwen}
宫对阙què ,座对龛kān [1] 。水北对天南。蜃shèn 楼对蚁郡jùn [2] ,伟wěi 论lùn 对高谈[3] 。遴lín 杞梓,树楩pián 楠[4] 。得一对函hán 三[5] 。八宝珊瑚hú 枕,双珠玳瑁mào 簪zān [6] 。萧xiāo 王待士心惟赤[7] ,卢lú 相欺君面独蓝[8] 。贾岛诗狂,手拟nǐ 敲qiāo 门行处想[9] ;张zhāng 颠diān 草圣,头能濡墨mò 写时酣hān [10] 。
\end{yuanwen}

\begin{yuanwen}
闻对见,解对谙ān [11] 。三橘jú 对双柑[12] 。黄童对白叟sǒu ,静女[13] 对奇男。秋七七[14] ,径三三[15] 。海色对山岚[16] 。鸾luán 声何哕huì 哕huì [17] ,虎视正眈dān 眈dān [18] 。仪封疆吏知尼父[19] ,函hán 谷关人识老聃dān [20] 。江相归池chí ,止zhǐ 水自盟méng 真是止zhǐ [21] ;吴公作宰,贪泉虽suī 饮亦何贪[22] 。

\end{yuanwen}

[1] 阙:皇帝居处,借指朝廷。龛:供奉佛像、神位等的小阁子。

[2] 蜃楼:海洋上由空气折射而成的幻影,古人以为是蜃(一种大蛤蜊)气所化,称蜃楼。蚁郡:唐李公佐《南柯太守传》写汉豪士淳于棼酒醉后梦游大槐安国,被招为附马,守南柯郡。醒后发现,原来槐安国和南柯郡是一群蚂蚁的窝巢。

[3] 伟论:高明超卓的言论。高谈:侃侃而谈,大发议论。

[4] 遴杞梓:比喻选拔人才。遴,谨慎选择;杞、梓,两种木质优良的树,古人以喻优秀人材。树楩楠:比喻培养人才。树,种植。楩,木名,即黄楩树。楩、楠是两种木质优良的树,生在南方。

[5] 得一:“一”是个哲学概念。《老子》中有“昔之得一者,天得一以清,地得一以宁,神得一以灵,谷得一以盈,万物得一以生,侯王得一以为天下正”的话。函三:《易纬乾凿度》说:“《易》一名而含三义:所谓易也,变易也,不易也。”意思是:《周易》的“易”字含三方面意义:简易、变易和不变。

[6] 双珠玳瑁簪:这是汉乐府《有所思》中的一句。玳瑁,一种海龟,其甲可制作工艺品。

[7] 萧王待士心惟赤:汉光武帝初起时,曾被更始帝刘玄封为萧王。他在镇压铜马、高湖等起义军时,收降许多人,并将首领封为列侯,以收买人心。所以当时有人说:“萧王推赤心置人腹中,安得不投死乎!”

[8] 卢相欺君面独蓝:唐卢杞长得特别丑陋,史称“鬼貌蓝色”,代宗时为相,迫害忠良,盘剥百姓,干了许多坏事,人曰“蓝面鬼”。

[9] 贾岛诗狂,手拟敲门行处想:唐诗人贾岛,一次在驴背上得“鸟宿池边树,僧敲月下门”两句诗,开始想用“推”字,后改“敲”,仍觉未妥,不觉冲撞京兆尹韩愈。韩愈问明原因,想了一会,认为“敲”字好。这就是“推敲”一语的由来。

[10] 张颠草圣,头能濡墨写时酣:唐张旭,善草书,好酒,每次大醉,则呼叫狂走,或把墨水浇到头上,然后写字,时人称他为“张颠”。

[11] 解:明白。谙:了解,熟悉。

[12] 双柑:唐冯贽《云仙杂记》卷二引《高隐外书》:“晋戴颙,春日携双柑斗酒,人问何之,曰:“往听黄鹂声。此俗耳针砭,诗肠鼓吹,汝知之乎?”

[13] 静女:《诗经》篇名。静女指仪态端方的少女。

[14] 秋七七:七七是传说中的人名,姓殷。鹤林寺杜鹃花为天下第一。周宝谓殷七七曰:“闻君能顷刻开花,今方重九,花能开乎?”七七曰:“诺。”即于掌中作幻术使花开。夜间一女子曰:“妾为上帝司此花,不久即归阆苑。”此七七即代指杜鹃花。

[15] 径三三:陶渊明咏菊,“冶冶溶溶三径色,风风雨雨九秋时。”此“径三三”即代指菊花。

[16] 山岚:山中的雾气。

[17] 鸾声何哕哕:《诗经·小雅·庭燎》有“君子至止,鸾声哕哕”二句。鸾,车铃。哕,乐声。

[18] 虎视正眈眈:《周易·颐卦》中的一句。眈眈,注视的样子。

[19] 仪封疆吏知尼父:仪是春秋时卫国的地名。尼父即孔子。《论语》记载,孔子到卫国去,仪邑主管边境的“封人”要求见孔子,见过之后对孔子的学生说:“你们不要为流亡而苦恼,上天将让孔子制礼作乐。”

[20] 函谷关人识老聃:传说函谷关的令尹善天文,一次登楼四望,于东方见紫色云气,高兴地说:一定有圣人经过此地。后老子骑青牛过关。杜甫诗“东来紫气满函关”即用此典。聃,老子名李聃。

[21] 江相归池,止水自盟真是止:《宋史·万里传》载,南宋末年,江万里为相,他听说元军已得襄樊,就在自家后园凿个池塘,题名“止水”。后元军至城破,万里遂投池自杀。

[22] 吴公作宰,贪泉虽饮亦何贪:《晋书·吴隐之传》载,晋吴隐之清廉,他到广州为刺史,州城附近有泉名“贪泉”,人们说,谁饮此水都会起贪心。吴隐之故意饮了贪泉水,并作诗一首说:“古人云此水,一歃怀千金。试使夷齐饮,终当不易心。”到郡后更加廉洁自守。歃,用嘴吸取。





\chapter{盐}

\begin{yuanwen}
宽kuān 对猛měng[1] ,冷对炎。清直zhí [2] 对尊zūn 严。云头对雨脚jiǎo [3] ,鹤hè 发fà 对龙髯rán [4] 。风台谏[5] ,肃sù 堂廉[6] 。保泰对鸣谦[7] ,五湖归范fàn 蠡lǐ [8] ,三径隐陶潜[9] 。一剑成功堪kān 佩pèi 印[10] ,百钱满mǎn 卦guà 便垂帘[11] 。浊zhuó 酒停杯,容我半酣hān 愁chóu 际饮[12] ;好花傍bàng 座,看他微笑悟时拈niān [13] 。
\end{yuanwen}

\begin{yuanwen}
连对断,减jiǎn 对添。淡dàn 泊对安ān 恬tián [14] ,回头对极jí 目,水底dǐ 对山尖。腰袅niǎo 袅niǎo [15] ,手纤纤[16] 。凤卜bǔ 对鸾luán 占zhān [17] 。开田tián 多种zhòng 粟sù ,煮海尽成盐。居同九世张zhāng 公艺[18] ,恩ēn 给jǐ 千人范fàn 仲zhòng 淹[19] 。箫xiāo 弄nòng 凤来,秦女有缘能跨kuà 羽[20] ;鼎dǐng 成龙去qù ,轩xuān 臣无计得攀pān 髯rán [21] 。
\end{yuanwen}

\begin{yuanwen}
人对己,爱ài 对嫌。举止zhǐ 对观瞻zhān [22] 。四知对三语[23] ,义正对辞严。勤雪案àn ,课风檐[24] 。漏lòu 箭对书笺jiān 。文繁fán 归獭tǎ 祭[25] ,体tǐ 艳别香奁lián [26] 。昨zuó 夜题tí 诗更gēng 一字[27] ,早春来燕卷重帘。诗以史名,愁chóu 里悲歌怀杜甫[28] ;笔经人索suǒ ,梦中显xiǎn 晦huì 老江淹[29] 。
\end{yuanwen}


[1] 宽对猛:《左传》载(郑)大夫子产临终前对他的儿子说:“我死,子必为政。惟有德者能以宽服民,其次莫如猛。”宽,指仁厚。猛,指严厉。

[2] 清直:清廉正直。

[3] 云头:云彩上面。雨脚:随云飘行、长垂及地的雨丝。

[4] 鹤发:是说人发白如鹤羽,指老人。龙髯:龙的胡须。传说黄帝在鼎湖乘龙而升天,小臣扯龙髯而上,结果扯断了龙须。

[5] 风台谏:风即讽,讽谏。台,台省。谏,谏臣。古谏官所居官署称讽台。

[6] 肃堂廉:肃堂即官署。廉,阶陛之侧隅。此指廉正。

[7] 保泰对鸣谦:泰和谦是《周易》的两个卦名。保泰,意为保持安康。鸣谦是谦卦的一句爻辞,意思是以谦虚的品德为人所知。

[8] 范蠡:字少伯,佐越王勾践破吴后载西施归五湖,自号陶朱公。

[9] 三径:归隐者的家园。晋陶潜《归去来辞》:“三径就荒,松竹犹存。”

[10] 一剑成功堪佩印:战国时苏秦曾佩一剑说六国,后为纵约长,佩六国相印。

[11] 百钱满卦便垂帘:汉严君平隐居成都,以卖卜自给,每日得百钱,即闭户垂帘而授《老子》。

[12] 浊酒停杯,容我半酣愁际饮:语出杜甫诗:“艰难苦恨繁霜鬓,潦倒新亭浊酒杯。”

[13] 好花傍座,看他微笑悟时拈:佛教故事,传说在灵山会上,释迦牟尼拿出一朵花,众人都不解其意,唯独迦叶尊者露出笑颜,表示对佛的旨意有所领悟。后遂以拈花微笑表示心心相印、两心相通。拈,用手指轻轻拿着。

[14] 淡泊:对于名利淡漠,不看重。安恬:淡泊,不追求名利。

[15] 腰袅袅:形容女子腰肢柔软。

[16] 手纤纤:形容手指细而长。

[17] 凤卜对鸾占:凤卜、鸾占意同,见微韵“采凤飞”注。

[18] 居同九世张公艺:唐人张公艺,九世同居。高宗祭泰山,幸其第,问何以能此,公书百“忍”字以进之。

[19] 恩给千人范仲淹:宋范仲淹居官后,于姑苏城郊买良田千亩,建立“义庄”,以收养贫困的亲族。

[20] 箫弄凤来,秦女有缘能跨羽:见东韵“凤翔”注。

[21] 轩臣:轩辕皇帝的大臣。攀髯:传说轩辕皇帝铸鼎成,龙降,骑之上升。其臣攀龙髯欲随之升天,未得。

[22] 举止:指姿态和风度。观瞻:显露于外的形象。

[23] 四知对三语:四知见侵韵“杨震”注。三语:据《晋书》载,一次王戎问老子、孔子之道于阮瞻,阮瞻曰:“将无同。”意思是“大约差不多”。王戎听了很满意,就聘其为掾(署员),时人称阮瞻为“三语掾”。

[24] 雪案、风檐:形容读书条件很艰苦。勤和课指学习。

[25] 文繁归獭祭:早春刚刚解冻,水獭把鱼衔出水面,排列在冰上,古人以为这是獭在祭祀,称为獭祭鱼。唐诗人李商隐作诗爱用典故,经常把翻阅的书排在一旁,书册左右麟次,时人也就称他为獭祭鱼。

[26] 体艳别香奁:体艳即艳体诗,指爱情或色情诗。唐诗人韩偓喜欢写这类诗,诗集名《香奁集》,时人号为“香奁体”。香奁,妇女梳妆用的匣子。

[27] 昨夜题诗更一字:唐僧齐己作《早梅》诗,曰:“前村深雪里,昨夜数枝开。”许丁卯改为“一枝开”,时人称为“一字师”。

[28] 诗以史名,愁里悲歌怀杜甫:见豪韵“诗史”注。史名,杜甫感痛时事,发之为诗,人称为“诗史”。

[29] 笔经人索,梦中显晦老江淹:见支韵“五色笔”注。





\chapter{咸}

\begin{yuanwen}
栽zāi 对植zhí ,薙tì 对芟shān [1] 。二伯对三监[2] 。朝臣对国老[3] ,职zhí 事对官衔。鹿麌yǔ 麌[4] ,兔毚chán 毚[5] 。启牍对开缄jiān [6] 。绿杨莺yīng xiàn 睆huǎn [7] ,红杏燕呢ní 喃[8] 。半篱lí 白酒娱陶令[9] ,一枕黄粱度吕lǚ 岩[10] 。九夏炎飙biāo [11] ,长日风亭留客骑[12] ;三冬寒冽liè ,漫màn 天雪浪làng 驻zhù 征zhēng 帆fān 。
\end{yuanwen}

\begin{yuanwen}
梧对杞,柏对杉。夏濩huò 对韶sháo 咸[13] 。涧瀍chán 对溱qín 洧wěi [14] ,巩gǒng 洛对崤xiáo 函hán [15] 。藏书洞[16] ,避诏岩[17] 。脱tuō 俗对超凡fán 。贤人羞献xiàn 媚mèi ,正士嫉jí 工谗。霸bà 越谋móu 臣推tuī 少伯[18] ,佐唐藩fān 将重zhòng 浑瑊jiān [19] 。邺yè 下狂生,羯jié 鼓三挝zhuā 羞锦jǐn 袄ǎo [20] 。江州司马,琵pí 琶pá 一曲湿青衫[21] 。
\end{yuanwen}

\begin{yuanwen}
袍páo 对笏hù [22] ,履对衫。匹马对孤帆。琢磨对雕镂,刻划对镌juān 镵chán [23] 。星北拱[24] ,日西衔。卮zhī 漏对鼎馋chán [25] 。江边生桂若[26] ,海外树都咸[27] 。但dàn 得恢huī 恢huī 存利刃[28] ,何须xū 咄咄达dá 空函[29] 。彩凤知音,乐典后hòu 夔kuí 须xū 九奏zòu [30] ;金人守口kǒu ,圣如尼父亦三缄jiān [31] 。
\end{yuanwen}

[1] 薙:除去野草。芟:割草。薙、芟都是斩除野草的意思。

[2] 二伯:西周时主掌国事的两个大臣,所谓“自陕以东,周公主之;自陕以西,召公主之”。三监:武王灭殷后,封纣子武庚于商都,派自己的三个弟弟管叔、蔡叔和霍叔监督,称三监。

[3] 国老:指国之重臣。

[4] 麌:鹿成群结队的样子。

[5] 毚:狡猾。

[6] 启牍对开缄:启牍和开缄都是拆开信件的意思。

[7] 睆:即莺啼的声音。

[8] 呢喃:燕子叫声。

[9] 半篱白酒娱陶令:陶令,即陶渊明。因为他曾为彭泽令,故称。

[10] 一枕黄粱度吕岩:见阳韵“客枕”注。原故事中的吕翁和卢生,后人附会成八仙中的钟离权度化吕洞宾(吕岩),所以这里说“度吕岩”。

[11] 九夏:夏季。炎飙:热风。飙,狂风。

[12] 长日:指整天、终日。风亭:亭子。

[13] 夏濩对韶咸:见萧韵“殷濩”句注。

[14] 涧、瀍、溱、洧:古代四条河流。

[15] 巩洛对崤函:巩,古地名,洛水流经其旁。崤,崤山,山名,又叫“崤陵”,其西有函谷关,故称崤函。巩、洛、崤、函均在今河南省。

[16] 藏书洞:指传说中的二酉山,四川酉阳县翠屏山麓的小酉山石穴中,有书千卷,相传秦人读书于此,称为“二酉藏书洞”。

[17] 避诏岩:指汉初“四皓”所隐的商山,“四皓”(详见齐韵“甪里”注),高帝召之不至,故称其隐居的岩洞为“避诏岩”。

[18] 霸越谋臣推少伯:少伯,越国大夫范蠡的字。见虞韵“归湖”注。

[19] 佐唐藩将重浑瑊:浑瑊,唐王朝少数民族的著名将领,曾从李光弼、郭子仪平“安史之乱”,以功为太常卿。德宗出逃奉天,浑瑊率家人子弟从,与朱泚(cǐ)拒战,全城倚重,德宗得以保全。

[20] 邺下狂生,羯鼓三挝羞锦袄:狂生指祢衡。传说曹操欲辱祢衡,命他为鼓吏,击鼓为客人助酒兴。他不仅毫无惧色,反而脱掉衣服,敲起慷慨昂扬的“渔阳三挝”,以回敬曹操。渔阳三挝,传说中古代的鼓曲名。锦袄,代指曹操。挝,这里指敲鼓。

[21] 江州司马,琵琶一曲湿青衫:唐诗人白居易曾谪为江州司马,一次到浔阳江边送客,遇到一位流落为商人妇的琵琶女,为他弹奏了一曲,引起了他强烈的共鸣,为之流下了泪水。故作长诗《琵琶行》。其中最后两句是:“座中泪下谁最多,江州司马湿青衫。”

[22] 笏:古代大臣上朝拿着的手板,用玉、象牙或竹片制成,上面可以记事。

[23] 镌镵:都是刻削的意思。

[24] 星北拱:星指北极星,拱是拱托、环绕的意思。古人认为群星都围绕北极星而分布。

[25] 卮漏:卮,古代一种盛酒器。古语有“川源而不能实漏卮”的话,意为漏洞虽小,如不堵塞则后患无穷。鼎馋:孔子的祖先正考父为宋大夫,其家有鼎名馋鼎。馋,吃。

[26] 若:杜若,香草名。

[27] 都咸:传说中生于海外的神木。

[28] 但得恢恢存利刃:《庄子·养生主》中的一则寓言,说宋国有个庖丁,善于解牛,他的刀用了十九年,解过数千头牛,还好像新磨的一样。因为牛的关节之间是有缝隙的,而刀刃却很薄,让薄薄的刀刃通过有缝隙的关节,自然“恢恢乎其于游刃必有余地”。恢恢,宽绰的样子。

[29] 何须咄咄达空函:晋殷浩得到桓温将推荐他作尚书令的消息,非常高兴,准备回信,又怕言语不周,把信取出放进几十次,结果却寄出了空信封。后桓温将免职,他整日用手在空中乱划,连呼“咄咄怪事”。咄咄,表示惊讶的语气。

[30] 彩凤知音,乐典后夔须九奏:后夔,即夔,传说是舜的乐官,他奏起乐来,百兽起舞,凤凰也飞来。九奏,奏乐九曲。

[31] 金人守口,圣如尼父亦三缄:尼父即孔子。相传孔子入周太庙,见有铸金人,三缄其口,背后有铭文:“古之慎言人也。”三缄,封闭多层。两句的意思是,圣达如孔子,也要学习金人那样守口如瓶,讲话谨慎。

\backmatter

\end{document}