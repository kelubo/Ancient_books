% 弟子规
% 弟子规.tex

\documentclass[a4paper,12pt,UTF8,twoside]{ctexbook}

% 设置纸张信息。
\RequirePackage[a4paper]{geometry}
\geometry{
	%textwidth=138mm,
	%textheight=215mm,
	%left=27mm,
	%right=27mm,
	%top=25.4mm, 
	%bottom=25.4mm,
	%headheight=2.17cm,
	%headsep=4mm,
	%footskip=12mm,
	%heightrounded,
	inner=1in,
	outer=1.25in
}

% 设置字体,并解决显示难检字问题。
\xeCJKsetup{AutoFallBack=true}
\setCJKmainfont{SimSun}[BoldFont=SimHei, ItalicFont=KaiTi, FallBack=SimSun-ExtB]

% 目录 chapter 级别加点(.)。
\usepackage{titletoc}
\titlecontents{chapter}[0pt]{\vspace{3mm}\bf\addvspace{2pt}\filright}{\contentspush{\thecontentslabel\hspace{0.8em}}}{}{\titlerule*[8pt]{.}\contentspage}

% 设置 part 和 chapter 标题格式。
\ctexset{
	chapter/name={第,篇},
	chapter/number={\chinese{chapter}},
    section/name= {},
	section/number={}
}

% 设置古文原文格式。
\newenvironment{yuanwen}{\bfseries\zihao{4}}

% 设置署名格式。
\newenvironment{shuming}{\hfill\bfseries\zihao{4}}

\title{\heiti\zihao{0} 弟子规}
\author{李毓秀}
\date{清·康熙}

\begin{document}

\maketitle
\tableofcontents

\frontmatter
\chapter{前言}

李毓秀(1647年—1729年),字子潜,号采三,出生于山东潍县\footnote{现潍坊市寒亭区。}李家营村。

清康熙年间,清朝作为少数民族统治阶层,出于长久统治的目的,主动追求对儒家思想文化的认同;包括崇儒尊孔,提倡修读四书五经,尊孔子为“大成至圣文宣先师”;大修孔庙,春秋祭孔、宣谕以孔子儒教为立国之本。康熙九年(1670年),清朝朝廷根据儒学核心制定和颁发“圣谕”十六条,作为人们的思想准则和行为规范;另外,康熙十二年(1673年)荐举山林隐逸,康熙十六年(1677年)开设明史馆,康熙十七年(1678年)荐举博学鸿词、网罗名士、弘扬儒学等举措都促进了儒家的发展。

经过屡次科举考试而不中后,放弃了对仕途的追求,从师党成,游历多年,精研《大学》、《中庸》等,且颇有建树,后创办敦复斋讲学,最终走上了“著书立说,教书育人”的道路。听众很多,被人尊称为“李夫子”,是清初著名的学者、教育家。著有《四书正伪》、《四书字类释义》、《学庸发明》、《读大学偶记》、《宋孺夫文约》、《水仙百咏》等。

在此过程中,李毓秀根据自身经历,完成了《训蒙文》。后来,贾存仁\footnote{一说贾有仁。}对该文章进行了修订并将名称改为《弟子规》。

该文共为五个部分,其首章“总叙”将孔子的话,用三字句改编而成,正文分为“入则孝,出则悌”“谨而信”“泛爱众,而亲仁”“行有余力,则以学文”四个部分,并对其进行阐释;主要列举了为人子弟在家、外出,待人、接物、处世、求学时应有的礼仪规范。

该文用佛家天台宗 " 五重玄义 " 的方法演义《弟子规》113 件事背后的义理,便于学习者能切入其中。清朝时,《弟子规》被朝廷高度重视,被定为幼学必读教材,并被誉为“开蒙养正最上乘”的读物。

主题思想
《弟子规》的中心思想围绕《论语·学而》中“弟子入则孝,出则悌,谨而信,泛爱众,而亲仁,行有余力,则以学文”而阐发 [6]
。主要包括:
1.孝亲敬长
《弟子规》全文的主题为孝。
《弟子规》开篇的“首孝悌”就强调了孝的地位和重要性。古人云:“水有源,木有本,父母者,人子之本源也。”人之所以能立于天地之间,是因为底下有根,根就是自己的父母,能不忘报答父母的养育之恩,才能对他人以及社会怀有感恩之心。对父母的关怀要从生活中的一点一滴做起。“冬则温,夏则清,晨则省,昏则定”,在生活上要关心父母,每天早晚向父母请安。“出必告,反必面”,不论是外出还是回来,都要告知父母,不让父母为孩子担心。孝敬父母,不仅要养父母之身,在物质方面满足父母的需要,让他们衣食无忧,更重要的是养父母之心,让父母快乐。“身有伤,贻亲忧,德有伤,贻亲羞”,养父母之心的另一层含义就是要把自己的事情做好,把自己的日子过好,不让父母操心。要修身养性,洁身自好,让父母因自己的德行而骄傲:要勤勉工作,报效社会,让父母因自己的成就而自豪。“父母呼,应勿缓,父母命,行勿懒”,父母呼唤,要立即应答,不可怠慢;父母交代的事情,要立刻去做,不可偷懒 [9]
。
《弟子规》从个人修为做起,具有很强的实践性和可操作性。它强调做人要从人性的原点——“孝”出发,首先修养身心,当德行充盈,在家就可以让家庭和谐,全家长幼有序,共享天伦,治理国家就能起到身先士卒、以身作则的表率作用,从而带领并影响自己的团体、国家,共同建设幸福家园,共谋和平安宁;相反,一个对自己的父母都没有孝心的人,更不会爱护别人和社会大众,这样的人即使能力再强也难以担当起社会的责任,说不定还会造成更大的危害 [9]
。
2.慎行谨言
《弟子规》中,“谨慎”一词的含义对于方今的社会秩序建设的启发可以归结为:其一,社会生活中务必要谨慎于细节:“冠必正,纽必结,袜与履,俱紧切。”人们常说:细节决定成败,社会秩序的构建,社会和谐的构造在很多情况下,都能从细枝末节中窥出,墨子说:“昔者楚灵王好士细腰,故灵王之臣皆以一饭为节,胁息然后带,扶墙然后起。”一个社会秩序的细微之处,两个社会秩序的细微之处出现扭曲,对于社会秩序的运转并无大碍,但无数个社会秩序的扭曲堆积到一处,如是楚王爱细腰,而国人多饿死的惨状就不可避免。所以《弟子规》将“谨慎”落脚在留意于细节 [10]
。
其次,良好社会秩序的确立要求社会成员务须谨慎彼此间的交往,其核心之要义即有所为而有所不为,例如交友当交益友,勿交损友,待人接物,更当小心翼翼,勿以善小而不为,勿以恶小而为之,完成个人的社会分工,安守本分,不要越俎代庖,而要持一种事不关己高高挂起的心态 [10]
。
再次,每个社会成员都要拥有一种自反省的精神,是非之心,廉耻之心,恻隐之心是儒教思想中对于人之为人的定位,并将这三种“心”作为先知先觉赋予人类的三种济世情怀。所以,在社会生活中,为了捍卫共同的社会秩序,每个人都要谨慎的解剖自己,对照着圣贤的教诲,祛除一切利小、损群体的利益诉求,推及此心用之于塑造社会秩序之平稳 [10]
。
3.讲求诚信
《弟子规》把诚信作为评价一个人德行的重要指标。“凡出言,信为先,诈与妄,奚可焉”,“事非宜,勿轻诺,苟轻诺,进退错”。儒家认为,在待人接物中,人们所说的每一句话,都要以诚信作为基础,做到“言必信,行必果”,自己做不到的事情,不可轻易许诺,因为“一言既出,驷马难追” [9]
。
4.爱众亲仁,善以待人
《弟子规》中的爱,成为一种大爱,先用“见人善,即思齐”“非圣书,屏勿视”的方式提离个人修养,然后从爱亲人开始,进而爱国家、爱整个天下。当爱推己及人,当朴素的亲人的情感发展为爱他人、爱国家、爱世界的情感时,爱的境界就得到了提升,人的精神境界也得到了升华 [9]
。
5.学习规范及能力
《弟子规》要求“行有余力,则以学文。”做到了孝、悌、谨、信、爱众、亲仁之后,才可以学习技能、知识。道德教育与文化知识教育并重,而实际上二者在实践过程中并不存在矛盾,而是并行不悖,相辅相成的 [2]
。
《弟子规》认为“不力行,但学文。长浮华,成何人。但力行,不学文。任已见,味理真。”对于孝、悌、谨、信、爱众、亲仁这些应该努力实践的德行,如果只是在学问上研究探索,却不肯亲身力行,这样最容易养成虚幻浮华的习性,《弟子规》中,“同是人,类不齐。流俗众,仁者希。”同样是人,却良莠不齐,随流俗者多,有仁德者少,人们之间的智力差距是有限的,而差距渐渐拉开,是因为人们在道德修养、人格品行上有一定的差异 [2]
。
艺术特色
《弟子规》在形式及艺术表现具有的特色为:
1.规范准则的具体可行。《弟子规》中的许多规范都具体可行。一系列要求:“步从容,立端正,揖深圆、拜恭敬,勿践阈,勿皲倚,勿箕踞,勿摇髀”,不仅向青少年学生指出了具体的站立、行走、行礼的正确姿势,而且指出一些应纠正的不良姿势,这种具体明确的准则要求易于使青少年学生理解接受 [6]
。
2.浅显生动的理性论证。《弟子规》不仅对青少年学生提出了具体可行的准则要求,而且还尽可能地作了浅显生动的理性论证,使青少年学生不仅知其然,而且知其所以然,使行为建立在理性自觉的基础上,也即不仅告诉青少年学生要怎样做或不能怎么做,而且还简单浅显地告诉了青少年学生为什么要这么做的理由。如“年方少,勿饮酒,饮酒醉,最为丑”,“道人善,即是善,人知之,愈加勉,扬人恶,即是恶,疾之甚,祸自作”,“能亲仁,无限好,德日进,过日少,不亲仁,无限害,小人进,百事坏”等。把一些简单的、普遍适用的道德认识,价值标准传授给青少年学生,不仅会给他们的行为以具体指导和规约,而且还可以帮助青少年学生树立基本的道德观念,增强道德自觉自律精神 [6]
。
3.朗朗上口的语言风格。《弟子规》采用三字一句,押韵的文字表达形式,这样节奏明快,便于记诵,更易于为青少年学生接受领会。

成文依据
《弟子规》的成文依据主要是参考古贤盛典和结合个人生活经历。
1.参考古贤盛典
作为行为规范类蒙学教材的集大成者,《弟子规》是建立在对其之前蒙学教材的借鉴与继承之上的。
(1)参照《论语·学而》
在著书立说时,作者没有脱离“圣人”的教诲另辟蹊径,而是参考圣贤著作,遵循封建社会的伦理道德,对其进行合理的发挥。李毓秀在创作《训蒙文》时以《论语》为基础,对弟子进行谆谆教诲,符合儒学常理。《论语·学而》篇中有“子曰:弟子人则孝,出则悌,谨而信,泛爱众,而亲仁,行有余力”。《弟子规》的总叙“弟子规,圣人训。首孝梯,次谨信。泛爱众,而亲仁。有余力,则学文”就是依据此句的意思改编的,并以此为基础,对弟子在家、出外、待人接物和求学等应有的礼仪以文字进行了规范 [5]
。
(2)参照《礼记·曲礼》
《弟子规》的部分内容来源于《礼记·曲礼》;《弟子规》在著述中,许多地方都参考了《曲礼》一文的章节。如:《礼记·曲礼》:“凡为人子之礼,冬温而夏清。昏定而晨省,在丑夷不争”《弟子规》:“冬则温,夏则清,晨则省,昏则定”等
。
(3)参照《童蒙须知》
《弟子规》中的部分内容参考《童蒙须知》;《弟子规》中的许多段落都是依据《童蒙须知》改编的。如:《童蒙须知》:“若父母长上有所召唤,却当疾走而前,不可舒缓”《弟子规》:“父母呼,应勿缓,父母命,行勿懒”等。
(4)其他参照书籍
有学者认为,作者创作《弟子规》时还参考了古代的许多经典著作,如程端蒙、董铢的《程董学则》,陈淳的《小学诗礼》等。
2.结合个人生活经历
《弟子规》的形成与作者本人所受到的教育及个人修养密不可分,是他对当时社会现状的感悟和对自身经历反思的结果。《弟子规》的部分内容是李毓秀结合自己所处的环境有感而发的。
李毓秀一生都只是个秀才,没有参加更高级别的考试,放弃了科考,几乎是断了人仕这条路,为什么生活在这样一种背景下的李毓秀偏偏此后再未考取更大的“功名”,依据推测,原因有二:
一是受他的老师党成的影响。从乾隆年间的《直隶绛州志》直至民国年间的《新绛县志》中都提到,李毓秀的老师党成是当时著名的名儒,李毓秀随师游历二十年,受其影响颇深。
二是李毓秀家庭条件很富裕,没有生活上的苦恼和压力。据史志记载,李毓秀出身于地主家庭的说法可能比较可靠。由于家境富裕,家资丰厚,其生活应当十分优越。正是有了一个稳定的生活环境,李毓秀才能专志于学问,著《训蒙文》。
因此,《弟子规》中一些内容也因此而来,体现了李毓秀所处的家庭环境。

局限不足
《弟子规》在当代使用的局限性和不足主要体现在:
1.行为条目与当代社会错位
随着时代的发展、语境的变化,《弟子规》成文之时给童蒙的示范行为和现实发生了一定的错位,因为该文作于清朝扬州十日等事件后的特有时代背景,如“晨则省,昏则定”,早上起来时要向父母请安,晚上要替父母铺好被子,侍奉父母安眠,当代子女不一定要一模一样地做,但需要学会体贴关心父母,这并非国学经典,应被视为封建糟粕而剔除。
所以当下在借鉴《弟子规》时,不能局限于字面行为的要求,应回到《弟子规》成文时的历史背景下,去理解它要表达的思想和精神。
2.自身内容矛盾
《弟子规》的内容有矛盾之处,这也成为使用《弟子规》的局限之一。如在“入则孝”篇中“父母呼,应勿缓,父母命,行勿懒,父母教,须敬听,父母责,须顺承”一开头就要求孩子要做一个孝顺的孩子,就算被责打的时候也要“父母责,须顺承”,但是事实父母非圣人,他们的责备有时不一定全对,所以《弟子规》在后面就说“亲憎,孝方贤,亲有过,谏使更,怡吾色,柔吾声,谏不入,悦复谏,号泣随,挞无怨”父母有错的时候作为子女的,一定要劝谏,就算因此遭到父母责打,内心也是没有怨言的。
正是这个原因,有的人认为《弟子规》很矛盾,解决不了问题,其实是没有把隐藏在《弟子规》背后的逻辑和思想弄明白。为此,学习和引进《弟子规》需要辩证地学和用,更需要超越性地教学和实践 [11]
。
3.知行不合一的社会现状
今人以为学到、知道就是做到,古人认为做到才是学到,这也成为当代使用《弟子规》的局限。《论语》中说:“学而时习之,不亦说乎”,就是学习了就要去实践就要去力行,这样才能有所收获,才会有所领悟。心学大师王阳明也说:“知之真切笃实处即是行,行之明觉精察处即是知。真知即所以为行,不行不足以为知”“知是行的主意,行是知的功夫。知是行之始,行是知之成。”《弟子规》传承了这些思想,“但学文,不力行,长浮华,成何人。”然而,当代的教育一味地强调学知识,道德教育忽视道德情感的体验和道德实践。如智然所说“西方文化和中国文化的区别:前者以为,知道,人就会去做;后者认为,“智慧的人才会去做。前者教育的重点是‘知道’,后者的教育立足点是‘做到’。前者重视‘说得对’,后者看重‘做得好’。” [11]
《弟子规》是要去实践的,是要变成孩子习惯,然后成为自然的。这也是当今使用《弟子规》的局限,因为人们已经习惯地认为学到或知道就是做到 [11]
。
4.古人求诸己,今人求诸人
古人求诸己,今人求诸人的社会现象也成为当代使用《弟子规》的局限。有学者将《弟子规》比作镜子,但是当代人们把这面镜子用来照别人,看别人哪里做得不好,哪里又有错误,用《弟子规》去指责别人,却没有将《弟子规》作为镜子对照自己的行为进行改过。《弟子规》这面镜子首先要拿来照自己而不是别人,因此不能求诸己也成为当今使用《弟子规》的局限 [11]
。.
5.思想和道德观念不一
《弟子规》部分思想中也包含着某些不适应于当代社会的消极因素,如“说话多,不如少,多易错,少易好”,在一定程度上会阻碍青少年语言表达能力的提高,与当代社会的开放交流环境对人们语言交际能力的高标准要求不相符合。另外“彼说长,此说短,不关已,莫闲管”一句也含有某种私有制社会的事不关己、高高挂起、明智保身、不分是非、不负责任的自私心理,是与现时代讲求原则性,积极开展批评与自批评的道德标准相违背的 [12]
。

\chapter{名家点评}

正面评价

中共中央党校(国家行政学院)教授刘余莉:《弟子规》是伦理道德教育的根本,也是做人的根本。

复旦大学教授钱文忠《孝经全鉴》:《弟子规》讲的是社会行为规范,让孩子知道应有的规矩,在孝顺父母、兄友弟恭中学会怎样与他人相处。其目的在于养成良好的行为习惯,培养诚敬的态度,形成仁爱的人格。

山西师范大学政法学院副教授张慧玲:《弟子规》不仅继承和发扬了古圣先贤的智慧和美德,而且还把“孝敬父母、尊敬师长、文明礼貌、尊重他人、诚实守信”等日常行为规范具体化、生活化,包含了大量思想修养、待人接物、饮食起居、生活礼规等做人的基本准则,特别讲究家庭教育与生活教育,其独具特色的伦理道德、童蒙养成的教育思想,对于当代家庭教育仍然具有重要的价值和意义。

东北林业大学教授刘经纬《思想政治教育研究》:《弟子规》的语言风格简明而不失文学意蕴,语言形式易懂而不失深刻内涵。《弟子规》用这种言语疏导的方法,在符合儿童的认知特点和认知规律的同时,又便于儿童通过朗诵背诵就获得道德理论知识,明白人生的道理,对道德产生初步的了解,从而达到道德认识的启蒙作用。

负面评价

北京师范大学教育学教授黄济:《弟子规》如同其他古典儿童读物一样,有它的历史、甚至阶级的局限,而且有的内容过于琐碎,对于儿童的限制过多,发挥儿童学习的主动性嫌少。

\chapter{后世影响}

《弟子规》是一部蒙学经典,内容浅白易懂、顺口押韵,以精炼的语言对儿童进行早期启蒙教育,受到了儿童们的喜爱;再加上全文灌输的是儒家文化的精髓,内容符合封建伦理,也受到了统治阶层的提倡,清朝统治阶级将其定为幼学必读教材。因而,《弟子规》成为清朝流行较广泛的蒙学教材,被誉为“开蒙养正最上乘”的读物。

在当代,《弟子规》不仅仅被用于幼儿教育,而且被运用到中学教育,甚至是大学教育和成人教育中,远远超出了最初的蒙学教育范围。

\chapter{版本流传}

清同治五年(1866年),为了抵制太平天国的革命影响,巩固孔孟之道在教育领城里的统治地位,贺瑞麟重刊了《弟子规》,公然声称他的目的是要恢篡所谓“圣贤”的教育,把儿童引入“圣贤之域”,为封建统治阶级培养孝子贤孙。他又将《弟子规》编入《清麓丛书》,重新出版。

在清朝末期和民国时期,《弟子规》曾几经修改,版本不一。

中华民国成立后,封建帝制被推翻,清朝末期的教育宗旨也被否定。1912年1月19日,南京临时政府教育部颁发了《普通教育暂行办法》和《普通教育暂行课程标准》:“如果学校教员遇有教科书中不合共和宗旨者,可随时删改,与此同时,废除不合共和精神的教学科目,如小学的读经科。这些规定虽是针对普通学校而言的,但对当时幼儿教育的教育内容及与此有关的课程设置的改革仍有指导作用。”传统的蒙学教材,尤其是《弟子规》这样宣扬封建伦理道德的蒙学读物被当时的统治阶级抛弃。

民国初年,虽然袁世凯推行过一阵“尊孔读经”的逆流,在全国学校恢复读经课,《弟子规》等传统蒙学读本被人们短暂地提及;但是由于袁世凯的逆行倒施,不久之后,全国教育恢复正常。直到新中国成立前,《弟子规》几乎不再见于中国幼儿教育的史料。

新中国成立后,中国的幼儿教育又发生了历史性的变化,全面贯彻民族的、科学的、大众的文化教育方针。在西方现代幼儿教育的基础上,中国开始探索发展现代幼儿教育。当时《弟子规》作为古代社会遗留下来的东西,并没有被人们重视。

“无产阶级文化大革命”时期,提出了《弟子规》是“宣扬孔孟之道的反动学生守则”“《弟子规》是为反动阶级培植奴才的黑规”“一本毒害青少年的黑书”“《弟子规》的要害是为‘克己复礼’培养接班人”等观点,对《弟子规》进行了大清扫。但是对《弟子规》的批判随着“批林批孔”运动的升级而更加剧烈,林彪为了改变中国共产党在社会主义历史时期的基本路线,对《弟子规》所散布的反动思想进行拼命进行鼓吹。直到“改革开放”之后,《弟子规》才重新走入公众的视野。

\chapter{作品争议}
修改者争议

对于《弟子规》修订者学术界一直存有争议,有学者认为是贾有仁,有学者认为是贾存仁;据考证,应该是贾存仁对《弟子规》进行了修改校订。

内容争议

自2010年以来,中国学术界、教育界和很多大众媒体就“《弟子规》是一本怎样的书”“《弟子规》到底该不该读”等问题产生了争论。

有学者认为:“《弟子规》是当时皇权、神权下的产物,满族人以几十万人统治亿万汉族人,统治者需要顺民、傀儡、奴隶,所以《弟子规》一诞生,就受到皇家大力追捧。主题就是听话,无条件的服从。只强调无条件的服从,就会产生十分有害的后果,人的本性会因此而受到严重的压抑,只会变成温顺听话的奴仆”。

对于此观点,有学者认为,《弟子规》不应该被看成洪水猛兽,也不应该被污蔑为“封建遗毒”,那种将《弟子规》与清王朝主流意识形态完全等同的想法过于的简单化和脸谱化了,既不符合实际内容,也不符合古代思想世界的实际;但《弟子规》也绝不是包治百病的良药,那种意图通过它挽救世道人心、移风易俗甚至“实现幸福人生”的美好愿望,超出了《弟子规》的正常阐释区域,存在着明显的过度阐释,有这种过度的宣传引起的公众对《弟子规》的反感不应该由文章来承担。

\mainmatter

\chapter{总叙}

\begin{yuanwen}
弟子\footnote{旧时对学生或小孩的称谓。}规,圣人\footnote{指孔子。}训。首孝弟\footnote{同“悌”,尊敬、顺从兄长。},次谨信。

泛爱众,而亲仁。有余力,则学文\footnote{文献、典籍。}。
\end{yuanwen}

弟子规,孝敬父母、友爱兄弟姐妹,其次是谨言慎行、信守承诺。

博爱大众,亲近有仁德的人。学好自己的思想道德之后,有多余精力,就应该多学多问。

\chapter{入则孝}

\begin{yuanwen}
父母呼,应勿\footnote{不要,不。}缓,父母命,行勿懒,

父母教\footnote{教育,教诲。},须敬听,父母责\footnote{责备、责骂。},须顺承\footnote{顺从地接受。}。

冬则温,夏则凊,晨则省\footnote{x\v{i}ng,问安,请安。},昏则定\footnote{服从父母安定地入睡。}。

出必告,反\footnote{同“返”,返家,返回。}必面,居有常\footnote{固定不变,保持常规。},业无变\footnote{变化,改变。}。

事虽小,勿擅为\footnote{擅自做主,盲目行动。},苟\footnote{如果,若是。}擅为,子道\footnote{为人子女之道。}亏。

物虽小,勿私藏,苟私藏,亲心伤。

亲\footnote{父母。}所好\footnote{喜好,爱好。},力\footnote{尽力,努力。}为具\footnote{置办,准备。},亲所恶,谨\footnote{认真、严肃、恭敬的态度。}为去\footnote{去掉、除去。}。

身有伤,贻\footnote{遗留,此处引申为带给。}亲忧,德有伤,贻亲羞\footnote{羞耻、耻辱。}。

亲爱我,孝何难\footnote{有什么困难,不难。},亲憎我,孝方\footnote{才。}贤。

亲有过\footnote{过错,过失。},谏使更,怡\footnote{和悦。}吾色,柔\footnote{柔和,温和。}吾声。

谏不入\footnote{听取,采纳。},悦复谏,号泣随\footnote{紧接着,跟随。},挞\footnote{鞭打。}无怨。

亲有疾\footnote{疾病。},药先尝,昼夜侍,不离床。

丧\footnote{跟死了人有关的事情。这里指守丧。}三年,常悲咽\footnote{因悲哀伤心而哭泣。},居处变\footnote{举丧期间,子女的日常生活起居应当有所变化、简化,以示孝道,如夫妻分居、禁食酒肉等。},酒肉绝\footnote{杜绝,戒除。}。

丧\footnote{丧事。}尽礼\footnote{礼节。},祭尽诚,事\footnote{对待,侍候。}死者,如事生。
\end{yuanwen}

如果父母呼唤自己,应该及时应答,不要故意拖延迟缓;如果父母交代自己去做事情,应该立刻动身去做,不要故意拖延或推辞偷懒。
父母教诲自己的时候,态度应该恭敬,并仔细聆听父母的话;父母批评和责备自己的时候,不管自己认为父母批评的是对是错,面对父母的批评都应该态度恭顺,不要当面顶撞。
冬天天气寒冷,在父母睡觉之前,应该提前为父母温暖被窝,夏天天气酷热,应该提前帮父母把床铺扇凉;早晨起床后,应该先探望父母,向父母请安问好;到了晚上,应该伺候父母就寝后,再入睡。
出门前,应该告诉父母自己的去向,免得父母找不到自己,担忧记挂;回到家,应该先当面见一下父母,报个平安;虽然子女有出息,父母会高兴,但是父母辈对子女最大的期望不是你多么有出息,而是你平平安安稳稳当当,一生没有灾秧。所以,居住的地方尽量固定,不要经常搬家,谋生的工作也不要经常更换。
事情虽小,也不要擅自作主和行动;擅自行动造成错误,让父母担忧,有失做子女的本分。
自己有什么东西,就算很小,也不要背着父母私藏。天下没有不透风的墙,如果私藏东西,即使自己很谨慎,也免不了会有被父母发现的一天,那时父母会伤心。
父母喜欢的事情,应该尽力去做;父母厌恶的事情,应该小心谨慎不要去做。
自己的身体受到伤害,必然会引起父母忧虑。所以,应该尽量爱惜自己的身体,不要让自己受到不必要的伤害。自己的名声德行受损,必然会令父母蒙羞受辱。所以,应该谨言慎行,不要让自己的名声和德行无端受损,更不要去做那种伤风败俗,自污名声,自贱德行的事情。
父母对我们态度慈爱的时候,孝敬父母恭顺父母不是什么难事;父母对我们态度不好,批评我们,埋怨我们,或者恶声恶气,厌恶我们,憎恨我们,打骂我们,甚至动刀动枪杀害我们,还能对父母心存孝意,才是难能可贵。
如果自己认为父母有过错,应该努力劝导父母改过向善,以免父母铸成更大的错误,使父母陷于不义的境地;不过要注意方法,劝导时应该和颜悦色、态度诚恳,说话的时候应该语气轻柔。
如果自己劝解的时候,父母听不进去,不要强劝,应该等父母高兴的时候再规劝,别跟父母顶撞,徒惹父母生气,还达不到规劝的效果;如果父母不听劝,又哭又闹,就暂时顺从父母;如果把父母劝恼,生气责打自己,不要心生怨恨,更不要当面埋怨。
父母亲生病时,要替父母先尝药的冷热和安全;应该尽力昼夜服侍,一时不离开父母床前。
父母去世之后,守孝三年,经常追思、感怀父母的养育之恩;生活起居,戒酒戒肉。
办理父母的丧事要合乎礼节,不可铺张浪费;祭奠父母要诚心诚意;对待去世的父母,要像生前一样恭敬。

\chapter{出则悌}

\begin{yuanwen}
兄道友\footnote{友爱。},弟道恭\footnote{恭敬。},兄弟睦\footnote{43},孝在中\footnote{44}。

财物轻,怨何生,言语忍,忿45自泯46。

或饮食,或坐走,长者先,幼者后。

长呼47人,即代叫48,人不在,己即到。

称尊长,勿呼名49,对尊长,勿见50能。

路遇长,疾趋51揖52,长无言,退恭53立。

骑54下马,乘55下车,过56犹57待58,百步余。

长者立,幼勿坐,长者坐,命乃59坐。

尊长前,声要低,低不闻,却非宜。

进必趋60,退必迟,问起对,视勿移。

事诸父61,如事父,事诸兄62,如事兄。
\end{yuanwen}


41.友:
42.恭:
43.睦:和睦。
44.在中:体现在其中。
45.忿:愤怒,怨恨。
46.泯(mǐn):泯灭,消失。
47.呼:呼喊,呼唤。
48.代叫:就代为呼唤。
49.呼名:直呼姓名。
50.见(xiàn)能:炫耀自已的才能,见,同“现”。
51.疾趋:快步走上前。
52.揖(yī):作揖,古时的一种拱手礼。
53.恭:恭敬。
54.骑:骑在马上。
55.乘(chéng):乘坐在车中。
56.过:长辈走过去。
57.犹:还,还要。
58.待:等待。这里指在原地稍等片刻。
59.乃:才能。
60.趋:走快。
61.诸父:伯父,叔父。
62.诸兄:堂兄堂弟。


兄长要友爱弟妹,弟妹要恭敬兄长;兄弟姐妹能和睦相处,父母自然欢喜,孝道就在其中了。
轻财重义,怨恨就无从生起;言语上包容忍让,忿怒自然消失。
不论饮食用餐,或就坐行走;都要年长者优先,年幼者在后。
长辈呼唤别人,应该立即代为传唤和转告;如果那个人不在,或者找不到那个人,应该及时告知长辈。
称呼尊者长辈,不应该直呼其姓名;在尊者长辈面前,应该谦虚有礼,见到尊者长辈有所不能,帮助可以,但不应该故意炫耀自己的才能,故意显示自己比尊者长辈强。
路上遇见长辈,应恭敬问好行礼;如果长辈没有说话,应退后恭敬站立一旁,等待长辈离去。
如果遇见长辈时,自己是骑马或乘车,应下马或下车问候;等待长者离开百步之远,方可续行。
长辈站着的时候,晚辈不应该坐着。具体是长辈坐下前,晚辈不应该先坐;大家都坐着的时候,长辈站起来时,晚辈也应该站起来;大家都坐着的时候,又一个长辈进来了,晚辈也应该立即站起来,以示尊敬。长辈坐定以后,晚辈应该等长辈示意自己坐下时,才可以坐。
在尊长跟前与尊长说话,或者在尊长跟前与别人说话,应该低声细气,不应该咋咋呼呼;但声音太低,交头接耳,窃窃私语,尊长听不清楚,也不合适。
到尊长面前,应快步向前;退回去时,稍慢一些才合礼节;长辈问话时,应该站起来回答,而且应该注视聆听,不应该东张西望;
对待父辈祖辈,如养父,姑父,姨父,叔父,舅父,岳父,祖父,外祖父,曾祖父,外曾祖父等等长辈,应该如同对待自己的亲生父亲一般孝顺恭敬;对待兄辈,如堂兄,表兄,族兄等兄长,应该如同对待自己的同胞兄长一样友爱尊敬。

\chapter{谨}

\begin{yuanwen}
朝起早,夜眠迟,老易至,惜此时。

晨必盥63,兼漱口,便溺回,辄64净手。

冠必正,纽必结,袜与履,俱紧切。

置冠服,有定位,勿乱顿65,致污秽66。

衣贵洁,不贵华67,上循分68,下称69家。

对饮食,勿拣择,食适可,勿过则70。

年方少,勿饮酒,饮酒醉,最为丑。

步从容,立端正,揖深圆,拜恭敬。

勿践阈71,勿跛倚72,勿箕踞73,勿摇髀74。

缓揭帘,勿有声,宽转弯,勿触棱75。

执虚76器,如执盈77,入虚室,如有人。

事勿忙,忙多错,勿畏难,勿轻略。

斗闹场,绝勿近,邪僻事,绝勿问。

将入门,问孰存,将上堂,声必扬78。

人问谁,对以名,吾与我,不分明。

用人物,须明求,傥79不问,即为偷。

借人物,及时还,人借物,有勿悭80。
\end{yuanwen}

63.盥(guàn):洗脸,洗手。
64.辄:立即。
65.顿:放置。
66.秽:弄脏。
67.华:美观。
68.分:名分,职分。
69.称:和家里的身分相称。
70.则:规定,一定的数量。
71.阙:门槛。
72.跛:偏。
73.箕距:两腿叉开蹲着或坐着。
74.髀:大腿。
75.棱:物体的棱角。
76.虚:空。
77.盈:满。装满东西。
78.扬:高,扩大。
79.傥:通“倘”,假若。
80.悭:吝啬,小气。

早上应该早起,晚上不应该过早睡;因为人生易老,所以应该珍惜时光。
早晨起床,务必洗脸梳妆、刷牙漱口;大小便回来,应该洗手。
穿戴仪容整洁,扣好衣服纽扣;袜子穿平整,鞋带应系紧。
放置衣服时,应该固定位置;衣物不要乱放乱扔,以免使家中脏乱差。
服装穿着重在整洁,不在多么华丽;一方面应该考虑自己的身份地位,另一方面应该根据家庭实力量力而行。
对待饮食,不要挑挑拣拣,嫌这嫌那;饮食吃饱吃好就行,不要过分追求美食,三餐只需吃个八分饱即可,避免过量,危害健康。
少年未成,不可饮酒;酒醉之态,最为丑陋。
走路步伐从容稳重,站立要端正;上门拜访他人时,拱手鞠躬,真诚恭敬。
不要踩在门槛上,站立不要歪斜,不要依靠在墙上;坐的时候不可以伸出两腿,腿不可乱抖动。
进出房间揭帘子、开关门的时候,应该动作轻缓,不要故意发出声响;拐弯的时候,应该绕大点圈,不要直楞楞的贴着墙角或者直角拐,这样就不会撞到物品的棱角,以致受伤,也不会因为有人在拐角处突然出现而撞在一起。
拿空器具的时候,应该像拿着里面装满东西的器具一样,端端正正,不要甩来甩去,不然会显得很轻浮;进入无人的房间,也应该像进入有人的房间一样,不可以随便。
做事的时候,即使再紧迫,也不要慌慌张张,因为忙中容易出错;不要畏惧困难,也不要草率行事。
打斗、赌博、色情等不良场所,绝对不要接近;对邪僻怪事,不要好奇过问。
将要入门之前,应先问:“有人在吗?”进入客厅之前,应先提高声音,让屋里的人知道有人来了。
屋里的人问:“是谁呀?”,应该回答名字;若回答:“是我”,让人无法分辨是谁。
想用别人的物品,应该明明白白向人请求、以征得同意;如果没有询问主人意愿,或者问了却没有征得主人同意,而擅自取用,那就是偷窃行为。
借人物品,应该及时归还;以后若有急用,再借不难。

\chapter{信}

\begin{yuanwen}
凡出言,信为先,诈\footnote{欺骗。}与妄\footnote{胡言乱语。},奚\footnote{何,怎么。}可焉?

话说多,不如少,惟\footnote{只有,只要。}其是\footnote{恰当,无误。},勿佞\footnote{会说动听的话。}巧。

奸巧语,秽污词,市井气,切戒之。

见未真,勿轻言,知未的87,勿轻传。

事非宜,勿轻诺,苟轻诺,进退错。

凡道字88,重且舒89,勿急疾,勿模糊。

彼说长,此说短,不关己,莫闲管90。

见人善,即思齐,纵去远91,以渐跻92。

见人恶,即内省93,有则改,无加警。

唯德学,唯才艺,不如人,当自砺。

若衣服,若饮食,不如人,勿生戚94。

闻过怒,闻誉乐,损友来,益友95却。

闻誉恐,闻过欣,直谅士96,渐相亲。

无心非97,名为错,有心非,名为恶。

过能改,归于无,倘掩饰,增一辜98。
\end{yuanwen}

87.的:鲜明,明白。
88.道字:说话吐字。
89.舒:流畅。
90.不关己,莫闲管:《孔子家语》“无多事,多事多患。”
91.纵:虽然。
92.跻:登,上升。
93.省:检查自己的思想和言行。
94.戚:忧患,悲哀。
95.友:友直,友量,有诚信。
96.直谅士:正直诚实的知识分子。
97.非:用作动词,做坏事。
98.辜:罪,罪过。

开口说话,诚信为先;欺骗和胡言乱语,不可使用;
话多不如话少;说话事实求是,不要妄言取巧;
不要讲奸邪取巧的话语、下流肮脏的词语;势利市井之气,千万都要戒之。
没有得知真相之前,不要轻易发表意见;不知道真相的传言,不可轻信而再次传播。
对不合理的要求,自己做不到的事情,不要轻易答应许诺;如果轻易答应,就会使自己进退两难。50、说话时吐字清楚,语速缓慢;说话不要太快、吐字模糊不清。
不要当面说别人的长处,背后说别人的短处;不关自己的是非,不要无事生非。
看见他人的善举,要立即学习看齐;纵然能力相差很远,也要努力去做,逐渐赶上。
看见别人的缺点或不良行为,要反省自己;有则改之,无则加以警惕。
只有品德学识才能技艺不如别人,应当自我激励,自我磨砺,自我提高。
如果是穿着饮食不如他人,不要攀比忧愁。
如果听到别人的批评就生气,听到别人的称赞就欢喜,坏朋友就会来找你,良朋益友就会离你而去。
听到他人称赞自己,唯恐过誉;听到别人批评自己,欣然接受,良师益友就会渐渐和你亲近;
不是存心故意做错的,称为过错;若是明知故犯的,便是罪恶。
知错改过,错误就会消失;如果掩饰过错,就是错上加错;

\chapter{泛爱众}

\begin{yuanwen}
凡是人,皆须爱,天同覆\footnote{遮盖。},地同载\footnote{承担。}。

行高者,名自高,人所重,非貌高。

才大者,望自大,人所服,非言大\footnote{夸大其词,吹嘘。}。

己有能,勿自私,人所能,勿轻訾\footnote{诋毁,怨恨。}。

勿谄富,勿骄贫\footnote{《礼记·坊记》“小人贫斯约,富斯骄,约斯盗,骄斯乱。”},勿厌故,勿喜新。

人不闲,勿事搅,人不安,勿话扰。

人有短,切莫揭,人有私,切莫说。

道人善,即是善,人知之,愈思勉。

扬人恶,即是恶,疾之甚,祸且作。

善相劝,德皆建,过不规,道两亏。

凡取与,贵分晓,与宜多,取宜少。

将加人,先问己,己不欲,即速已\footnote{停止。}。

恩欲报,怨欲忘,报怨短,报恩长。

待婢仆,身贵端\footnote{直,正。},虽贵端,慈而宽。

势服人,心不然,理服人,方无言。
\end{yuanwen}

凡是人类,皆须相亲相爱;因为同顶一片天,同住地球上。
德行高尚者,名声自然崇高;人们内心真正敬重的是德行,而不是那些表面上权势高,地位高的人。
大德大才者,威望自然高大;人们内心真正信服的德才,而不是那些嘴上谈论的大官,大人物,大财商。
自己有能力,不要自私自利,要帮助别人;他人有能力,不要嫉妒,应当欣赏学习。
不要献媚巴结富有的人,也不要在穷人面前骄纵自大;不要喜新厌旧。
别人正在忙碌,不要去打扰;别人心情不好,不要用闲言闲语去打扰。
别人的短处,切记不要去揭短;别人的隐私,切记不要去宣扬。
赞美他人的善行就是行善;别人听到你的称赞,就会更加勉励行善。
宣扬他人的恶行,就是在做恶事;对别人过分指责批评,会给自己招来灾祸;
互相劝善,德才共修;有错不能互相规劝,两个人的品德都会亏欠。
取得或给予财物,贵在分明,该取则取,该予则予;给予宜多,取得宜少。
要求别人做的事情,先反省问自己愿不愿意做,自己不愿意做的事情,应立刻停止要求,不要强求别人去做。
欲报答别人的恩情,就要忘记对别人的怨恨;应该短期抱怨、长期报恩。
对待婢女和仆人,自己要品行端正、以身作则;虽然品行端正很重要,但是仁慈宽厚更可贵;
仗势逼迫别人服从,对方难免口服心不服;以理服人,别人才会心悦诚服。

\chapter{亲仁}

\begin{yuanwen}
同是人,类不齐,流俗众,仁者希。

果仁者,人多畏,言不讳,色不媚。

能亲仁,无限好,德日进,过日少。

不亲仁,无限害,小人进,百事坏。
\end{yuanwen}

同样是人,善恶正邪,心智高低,良莠不齐;流于世俗的人众多,仁义博爱的人稀少。
如果有一位仁德的人出现,大家自然敬畏他;他直言不讳,不会察色献媚。
能够亲近有仁德的人,向他学习,是无限好的事情;他会使我们的德行与日俱增,过错逐日减少;
不肯亲近仁义君子,就会有无穷的祸害;奸邪小人就会趁虚而入,影响我们,导致整个人生的失败。

\chapter{余力学文}

\begin{yuanwen}
不力行,但学文,长浮华,成何人。

但力行,不学文,任己见,昧理真。

读书法,有三到,心眼口,信\footnote{的确,确实。}皆要。

方读此,勿慕彼,此未终,彼勿起。

宽为限,紧用功,工夫到,滞塞通。

心有疑,随札记\footnote{分条记录,作为参考的文字。},就人问,求确义。

房室清,墙壁净,几案洁,笔砚正。

墨磨偏,心不端,字不敬,心先病。

列典籍,有定处,读看毕,还原处。

虽有急,卷\footnote{卷轶,书本。}束\footnote{捆绑。}齐\footnote{整齐。},有缺坏,就补之。

非圣书\footnote{指儒家经书。},屏\footnote{除去,放弃。}勿视,蔽聪明,坏心志。

勿自暴,勿自弃,圣与贤,可驯\footnote{渐进,逐渐。}致\footnote{达到。}。
\end{yuanwen}

不能身体力行入则孝、出则弟、谨而信、泛爱众、而亲仁,纵有知识,也只是增长自己华而不实的习气,变成一个不切实际的人。
只是身体力行,不肯读书学习,就容易依著自己的偏见做事,也会看不到真理;
读书的方法有三到:眼到、口到、心到,三者缺一不可。
做学问要专一,不能一门学问没搞懂,又想搞其他学问。
读书计划要有宽限,用功要加紧;用功到了,学问就通了。
不懂的问题,记下笔记,就向良师益友请教,求的正确答案。
房间整洁,墙壁干净,书桌清洁,笔墨整齐。
墨磨偏了,心思不正,写字就不工整,心绪就不好了。
书架取书,读完之后,放归原处。
虽有急事,也要把书本收好再离开,有缺损就要修补。
不良书刊,摒弃不看,以免蒙蔽智慧和坏了心志。
遇到挫折,不要自暴自弃,通过身体力行圣贤的训诫,就可以达到圣贤的境界。

\end{document}