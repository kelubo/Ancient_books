% 诗经
% 诗经.tex

\documentclass[12pt,UTF8]{ctexbook}

% 设置纸张信息。
\usepackage[a4paper,twoside]{geometry}
\geometry{
	left=25mm,
	right=25mm,
	bottom=25.4mm,
	bindingoffset=10mm
}

% 设置字体,并解决显示难检字问题。
\xeCJKsetup{AutoFallBack=true}
\setCJKmainfont{SimSun}[BoldFont=SimHei, ItalicFont=KaiTi, FallBack=SimSun-ExtB]

% 目录 chapter 级别加点(.)。
\usepackage{titletoc}
\titlecontents{chapter}[0pt]{\vspace{3mm}\bf\addvspace{2pt}\filright}{\contentspush{\thecontentslabel\hspace{0.8em}}}{}{\titlerule*[8pt]{.}\contentspage}

% 设置 part 和 chapter 标题格式。
\ctexset{
	part/name= {第,卷},
	part/number={\chinese{part}},
	chapter/name={第,篇},
	chapter/number={\chinese{chapter}}
}

% 设置古文原文格式。
\newenvironment{yuanwen}{\bfseries\zihao{4}}

% 设置署名格式。
\newenvironment{shuming}{\hfill\bfseries\zihao{4}}

% 注脚每页重新编号,避免编号过大。
\usepackage[perpage]{footmisc}

\title{\heiti\zihao{0} 诗经}
\author{}
\date{}

\begin{document}

\maketitle
\tableofcontents

\frontmatter
\chapter{前言}

1.原文
2. 注释
3. 翻译
《诗经》是中国第一部诗歌总集。它汇集了从西周初年到春秋中期五百多年间的诗歌三百零五篇。在先秦叫做《诗》,或者取诗的数目整数叫《诗三百》,本来只是一本诗集。从汉代起,儒家学者把《诗》当作经典,尊称为《诗经》,列入“五经”之中,它原来的文学性质就变成了同政治、道德等密切相连的教化人的教科书,也称“诗教”。

《诗经》中的三百零五篇诗分为风、雅、颂三部分。“风”的意思是土风、风谣,也就是各地方的民歌民谣。“风”包括了十五个地方的民歌,即“十五国风”,共一百六十篇。“雅”是正声雅乐,是正统的宫廷乐歌。“雅”分为“大雅”(用于隆重盛大宴会的典礼—)和“小雅”(用于一般宴会的典礼),一共有一百零五篇。“颂”是祭祀乐歌,用于宫廷宗庙祭祀祖先,祈祷和赞颂神明,现存共四十篇。《诗经》的三百零五篇诗歌,广泛地反映了当时社会生活的各个方面,内容涉及政治、经济、伦理、天文、地理、外交、风俗、文艺各个方面,被誉为古代社会的人生百科全书,对后世产生过深远的影响。

《诗经》编辑成书的年代约在春秋后期,据说孔子曾经删定过〈诗经〉。到汉代,传授《诗经》的有四家。齐国辕固所传的《诗》叫《齐诗》,鲁国申培所传的《诗》叫《鲁诗》,燕国韩婴所传的《诗》叫《韩诗》,鲁国毛亨所传的《诗》叫《毛诗》。东汉时,《毛诗》得到了官方和学者们的认同,逐渐盛行,齐、鲁、韩三家《诗》逐渐衰落以至亡佚。现在我们见到的《诗经》,就是毛亨传下来的,我们这里选录的《诗经》,原文主要依据清代阮元校订的《十三经注疏》,并广泛参考了其他研究《诗经》的专著。

\mainmatter

% 增加空行
~\\

% 增加字间间隔,适用于三字经、诗文等。
 \qquad  

\part{周南}

\chapter{关雎}

男欢女爱的千古绝唱。这是一首男子追求女子的情诗。它是《诗经》中的第一篇,历来受人们重视。古代研读《诗经》的学者,多数认为“君子”指周文王,“淑女”指其妃太姒\footnote{s\`i},诗的主旨是歌颂“后妃之德”。但仔细吟咏,根本找不到后妃的影子,只是讲一位青年男子在追求美丽贤淑的姑娘。此诗形象生动地描绘出青年男子在追求自己心上人时焦虑急迫,昼思夜想难以入眠的相思情景。

\begin{yuanwen}
关关\footnote{水鸟叫声。}雎鸠\footnote{j\=u ji\=u,水鸟,一名王雎,状类凫鹥,生有定偶,常并游。},在河之洲\footnote{水中的陆地。}。窈窕\footnote{y\v{a}o ti\v{a}o,美心为窈,美状为窕。内心,外貌美好的样子。}淑\footnote{好,善。}女,君子\footnote{这里指女子对男子的尊称。}好逑\footnote{qi\'u,配偶。}。
\end{yuanwen}

关关对鸣的雎鸠,栖歇在河中沙洲。美丽贤淑的姑娘,真是君子好配偶。

\begin{yuanwen}
参差\footnote{c\=en c\=i,长短不齐的样子。}荇菜\footnote{x\`ing,一种多年生水草,夏天开黄色花,叶子浮在水面,嫩叶可食。},左右流\footnote{顺水之流而取之也。}之。窈窕淑女,寤\footnote{w\`u,睡醒。}寐\footnote{m\`ei,睡着。}求之。
\end{yuanwen}

长长短短的荇菜,左边右边不停采。美丽贤淑的姑娘,梦中醒来难忘怀。

\begin{yuanwen}	
求之不得,寤寐思\footnote{语气助词,没有实义。}服\footnote{思念。}。悠\footnote{忧思的样子。}哉悠哉,辗转\footnote{转动。}反侧\footnote{反身,侧身。翻来覆去。}。
\end{yuanwen}

美好愿望难实现,醒来梦中都想念。想来想去思不断,翻来覆去难入眠。

\begin{yuanwen}	
参差荇菜,左右采之。窈窕淑女,琴瑟\footnote{古时的弦乐器。琴,五弦或七弦乐器。瑟,二十五弦乐器。}友\footnote{友好交往,亲近。}之。
\end{yuanwen}

长长短短的荇菜,左边右边不停摘。美丽贤淑的姑娘,弹琴奏瑟表亲爱。

\begin{yuanwen}	
参差荇菜,左右芼\footnote{m\`ao,拔取。有选择之意。}之。窈窕淑女,钟\footnote{金属打击乐器。}鼓\footnote{皮革打击乐器。}乐\footnote{使其快乐。}之。
\end{yuanwen}

长长短短的荇菜,左边右边不停地择。美丽贤淑的姑娘,鸣钟击鼓让她乐起来。

\chapter{葛覃}

写已出嫁的女子准备回娘家探望父母的诗。在当时的社会,已婚女子回娘家探亲是件不容易的事,也是一件大事。所以她作了种种准备,采葛煮葛、织成粗细葛布、再做好衣服。征得公婆和师姆的同意,又洗衣、整理衣物,最后才高高兴兴地回去。

\begin{yuanwen}
葛\footnote{多年生藤本植物,花紫红色,茎可做绳,纤维可织葛布。}之覃\footnote{蔓延。}兮,施\footnote{y\`i,延及。}于中谷\footnote{即谷中。},维\footnote{语助词。}叶萋萋\footnote{茂盛的样子。}。

黄鸟\footnote{一说黄鹂,一说黄雀,又称黄栗留,身体很小。}于\footnote{语助词。}飞,集\footnote{聚集。}于灌木,其鸣喈喈\footnote{鸟鸣声。}。
\end{yuanwen}

葛草长长壮蔓藤,一直蔓延山谷中,叶子碧绿又茂盛。

黄鸟翩翩在飞翔,落在灌木树丛上,鸣叫声声像歌唱。

\begin{yuanwen}
葛之覃兮,施于中谷,维叶莫莫\footnote{茂盛的样子。}。

是\footnote{乃。}刈\footnote{y\`i,割。}是濩\footnote{hu\`o,煮。},为絺\footnote{ch\=i,细的葛纤维织的布。}为绤\footnote{x\`i,粗的葛纤维织的布。},服\footnote{穿。}之无斁\footnote{y\`i,厌倦。}。
\end{yuanwen}

葛草长长壮蔓藤,一直蔓延山谷中,叶子浓密又茂盛。

收割回来煮一煮,剥成细线织葛布,穿上葛衣真舒服。

\begin{yuanwen}
言\footnote{连词,于是。一说作语助。}告师氏\footnote{类似管家奴隶,保姆。一说女师。},言告\footnote{告假。}言归\footnote{回娘家。}。

薄\footnote{句首助词。}污\footnote{洗去污垢。}我私\footnote{内衣。},薄浣\footnote{hu\`an,洗。}我衣\footnote{指外衣。}。

害\footnote{通曷,盍,何,疑问词。}浣害否\footnote{不。},归宁\footnote{出嫁女子回娘家探视父母。}父母。
\end{yuanwen}

回去告诉我师姆,我要告假看父母。

先洗我的贴身衣,再把我的外衣洗。

洗与不洗理清楚,回家问候我父母。

\chapter{卷耳}

采摘卷耳的女子,怀念离家亲人,设想他途中种种困顿情况,以寄离思。这是一首妻子怀念远行丈夫的诗。全篇均通过采卷耳妇女的种种想象,表达对丈夫的深切思念。她想象丈夫旅途劳累,人困马乏,忧思愁苦,以酒解忧。又想象马儿累倒,仆人累病,丈夫唏嘘长叹。想象得越丰富,表达的感情越深切,越是有感人的力量。此诗对后世影响很大,清人方玉润说:“后世杜甫‘今夜鄜州月’一首,脱胎于此。”

\begin{yuanwen}
采采卷耳\footnote{又名苍耳,菊科一年生草本植物,果实呈枣核形,上有钩刺,名“苍耳子”,可做药用。嫩苗可食。},不盈顷筐\footnote{浅而易盈的竹筐。一说斜口筐。}。

嗟\footnote{ji\=e,语助词。}我怀人\footnote{想念的人。},寘\footnote{搁置,放置。}彼周行\footnote{大道。行,h\'ang。}。
\end{yuanwen}

采了又采卷耳,总是不满一浅筐。

只因想念远行人,筐儿丢在大路旁。

\begin{yuanwen}
陟\footnote{zh\`i,登上。}彼\footnote{指示代名词。}崔嵬\footnote{高而不平的土石山。嵬,w\'ei。},我马虺隤\footnote{hu\=i tu\'i,疲极而病。}。

我姑\footnote{姑且。}酌彼金罍\footnote{l\'ei,器名,青铜制,用以盛酒和水。},维\footnote{语助词。}以不永怀\footnote{长久地思念。}。
\end{yuanwen}

当我登上高山巅,骑的马儿腿发软。

且把酒杯来斟满,喝个一醉免怀念。

\begin{yuanwen}
陟彼高冈,我马玄黄\footnote{马过劳而视力模糊。马因病毛色焦枯。}。

我姑酌彼兕觥\footnote{s\`i g\=ong,一说野牛角制的酒杯,一说“觥”是青铜做的牛形酒器。},维以不永伤\footnote{长久思念。}。
\end{yuanwen}

我又登上高山冈,马儿累得毛玄黄。

且把酒杯来斟满,只为喝醉忘忧伤。

\begin{yuanwen}
陟彼砠\footnote{j\=u,有土的石山。}矣,我马瘏\footnote{t\'u,因劳致病。}矣。

我仆痡\footnote{p\=u,过度疲劳不能行之病。。}矣,云\footnote{语助词。}何\footnote{何等,那么。}吁\footnote{x\=u,忧愁。}矣。
\end{yuanwen}

我又登上土石山,我的马儿已累瘫。

仆人疲惫行走难,我的忧愁何时完。

\chapter{樛木}

樛木

南有樛木,葛藟系之。

乐只君子,福履绥之。

南有樛木,葛藟荒之。

乐只君子,福履将之。

南有樛木,葛藟萦之。

乐只君子,福履成之。

注释:祝福亲人得到福禄。

樛(音纠):木下垂曰樛。葛:多年生草本植物,花紫红色,茎可做绳,纤维可织葛布。藟(音垒):藟似葛,野葡萄之类。系:挂。只:语助。福履:福禄也。绥:安也。荒:掩;盖;覆。将:扶助也。萦(音营):缠绕。成:就也。

\chapter{螽斯}

螽斯

螽斯羽,诜诜兮。

宜尔子孙,振振兮。

螽斯羽,薨薨兮。

宜尔子孙,绳绳兮。

螽斯羽,揖揖兮。

宜尔子孙,蛰蛰兮。

注释:祝人子孙众多。

螽(音终)斯:或名斯螽,一种蝗虫。一说“斯”为语词。诜诜(音申):同莘莘,众多貌。振振(音真):盛貌。薨薨(音轰):众也。或曰形容螽斯的齐鸣。绳绳:不绝貌。揖揖(音集):会聚也。揖为集之假借。蛰蛰(音执):和集也。

\chapter{桃夭}

这是一首贺新娘的诗。全诗以桃树的枝、花、果、叶作为比兴事物,衬托出新嫁娘的年轻美丽以及成婚的快乐气氛。

\begin{yuanwen}
桃之夭夭\footnote{text},灼灼\footnote{text}其华。

之\footnote{text}子\footnote{text}于\footnote{text}归\footnote{text},宜\footnote{text}其室家\footnote{text}。
\end{yuanwen}

\begin{yuanwen}
桃之夭夭,有蕡\footnote{text}其实。

之子于归,宜其家室。
\end{yuanwen}

\begin{yuanwen}
桃之夭夭,其叶蓁蓁\footnote{text}。

之子于归,宜其家人。
\end{yuanwen}




夭夭:桃含苞貌。一说形容茂盛而艳丽,一说形容少壮的样子。灼灼(音茁):鲜明貌。华:花。归:妇人谓嫁曰归。宜:与仪通。仪,善也。室家:犹夫妇。男子有妻叫做有室,女子有夫叫做有家。有:语助词。蕡(音坟):实之盛也。蓁蓁(音真):草木茂盛貌。

\chapter{兔罝}

兔罝

肃肃兔罝,椓之丁丁。

赳赳武夫,公侯干城。

肃肃兔罝,施于中逵。

赳赳武夫,公侯好仇。

肃肃兔罝,施于中林。

赳赳武夫,公侯腹心。

注释:诗人赞扬武士的勇猛,意在讽刺诸侯豢养武士以做心腹爪牙。

肃肃:密密。罝(音居):捕兽的网。椓(音浊):敲击。丁丁(音争):伐木声。公候:周封列国爵位(公、候、伯、子、男)之尊者,泛指统制者。干:盾牌。干城,有屏障之义。逵(音魁):四通八达的大道。仇(音求):通逑。林:牧外谓之野,野外谓之林。

\chapter{芣苢}

\begin{yuanwen}
采采芣苢\footnote{text},薄\footnote{text}言\footnote{text}采之。

采采芣苢,薄言有\footnote{text}之。
\end{yuanwen}

\begin{yuanwen}
采采芣苢,薄言掇\footnote{text}之。

采采芣苢,薄言捋\footnote{text}之。
\end{yuanwen}

\begin{yuanwen}
采采芣苢,薄言袺\footnote{text}之。

采采芣苢,薄言襭\footnote{text}之。
\end{yuanwen}


注释: 劳动妇女采车前时所唱的歌曲。

采采:采而又采。芣苡(音浮以):植物名,即车前子,种子和全草入药。薄言:发语词。有:取也。掇(音多):拾取。捋(音罗):以手掌握物而脱取。袺(音洁):用衣襟兜东西。襭(音协):翻转衣襟插于腰带以兜东西。


\chapter{汉广}

\begin{yuanwen}
南有乔木\footnote{text},不可休思\footnote{text};汉有游女\footnote{text},不可求思。

汉之广矣,不可泳思\footnote{text};江之永\footnote{text}矣,不可方思\footnote{text}。
\end{yuanwen}

\begin{yuanwen}
翘翘错薪\footnote{text},言刈其楚;之子于归\footnote{text},言秣其马\footnote{text}。

汉之广矣,不可泳思;江之永矣,不可方思。
\end{yuanwen}

\begin{yuanwen}
翘翘错薪,言刈其蒌\footnote{text};之子于归,言秣其驹\footnote{text}。

汉之广矣,不可泳思;江之永矣,不可方思。
\end{yuanwen}

注释:诗人追求汉水游女,终于失望的恋歌。

休:息也。指高木无荫,不能休息。思:语助。汉:汉水,长江支流之一。游女:汉水之神。江:江水,即长江。永:水流长也。方:桴,筏。翘翘(音桥):本指鸟尾上的长羽,比喻杂草丛生。错薪:古代嫁娶必以燎炬为烛,故《诗经》嫁娶多以折薪、刈楚为兴。刈(音义):割。楚:灌木名,即牡荆。归:嫁也。秣(音莫):喂马。蒌(音楼):蒌蒿,嫩时可食,老则为薪。驹:小马。

\chapter{汝坟}

汝墳

遵\footnote{沿。}彼汝\footnote{汝河,源出河南省。}墳\footnote{f\'en,水涯,大堤。},伐其条\footnote{山楸树。一说树干(枝曰条,干曰枚)。}枚;

未见君子,惄如调饥。

遵彼汝墳,伐其条肄;

既见君子,不我遐弃。

鲂鱼赬尾,王室如燬;

虽然如燬,父母孔迩。

注释:妻子对远役丈夫的怀念。

惄(音逆):忧愁。调(音周):通朝,早晨。肄(音异):树砍后再生的小枝。遐(音狭):远。鲂鱼:鳊鱼。赬(音成):浅红色。燬:火。如火焚一样。孔:甚。迩(音而):近 。

\chapter{麟之趾}

麟\footnote{麒麟,传说动物。它有蹄不踏,有额不抵,有角不触,被古人看作至高至美的野兽,因而把它比作公子、公姓、公族的所谓仁厚、诚实。}之趾\footnote{足,指麒麟的蹄。},振振\footnote{zh\=en,诚实仁厚的样子。}公子\footnote{与公姓、公族皆指贵族子孙。},于\footnote{x\=u,通吁,叹词。}嗟麟兮。

麟之定\footnote{通颠,额。},振振公姓,于嗟麟兮。

麟之角,振振公族,于嗟麟兮。

注释:赞美贵族子孙繁衍,后人遂以麟趾喻子孙的贤能。





\part{召南}

\chapter{鹊巢}
鹊巢

维鹊有巢,维鸠居之;

之子于归,百两御之。

维鹊有巢,维鸠方之;

之子于归,百两将之。

维鹊有巢,维鸠盈之;

之子于归,百两成之。

注释:写贵族女子出嫁时的铺张奢侈。

维:发语词。鹊:喜鹊。有巢:比兴男子已造家室。鸠:一说鳲鸠(布谷鸟),自己不筑巢,居鹊的巢。贵州民间传说斑鸠不筑巢,居其他鸟类筑的巢。归:嫁。两:同辆。御(音亚):迎迓。方:占居。将(音江):送。盈:满。成:迎送成礼。
\chapter{采蘩}
采蘩

于以采蘩,于沼于沚;

于以用之,公侯之事。

于以采蘩,于涧之中;

于以用之,公侯之宫。

被之僮僮,夙夜在公;

被之祁祁,薄言还归。

注释:女子采蘩参加贵族祭祀。

于以:问词。一说语助。蘩(音繁):白蒿。生陂泽中,叶似嫩艾,茎或赤或白,根茎可食。沼:沼泽。沚(音止):水中小洲。涧:山夹水也。被(音币):首饰,亦用于编发(假发)。僮僮(音同):首饰盛貌,一说高而蓬松。祁祁(音其):形容首饰盛,一说舒迟貌。
\chapter{草虫}
草虫

喓喓草虫,趯趯阜螽;未见君子,忧心忡忡。

亦既见止,亦既觏止,我心则降。

陟彼南山,言采其蕨;未见君子,忧心惙惙。

亦既见止,亦既觏止,我心则说。

陟彼南山,言采其薇;未见君子,我心伤悲。

亦既见止,亦既觏止,我心则夷。

注释:女子怀念丈夫时的忧伤。

喓喓(音腰):虫鸣声。草虫:一种能叫蝗虫。蝈蝈儿。阜螽:一种蝗虫。 趯趯(音替):昆虫跳跃之状。忡忡(音冲):心跳。止:语助。觏(音够):遇见。陟:升;登。蕨:植物名。初生无叶,可食。惙惙(音绰):忧。愁苦的样子。说(音月):通悦。薇:草本植物。又名巢菜,或野豌豆。夷:平。心平则喜。
\chapter{采蘋}
采蘋

于以采蘋?南涧之滨;

于以采藻?于彼行潦。

于以盛之?维筐及筥;

于以湘之?维錡及釜。

于以奠之?宗室牖下;

谁其尸之?有齐季女。

注释:女子为贵族采蘋祭祀。

蘋:多年生水草。藻:水生植物。一说水豆。行潦(音杭老):水沟。筥(音举):圆形的筐。方称筐,圆称筥。湘:烹煮供祭祀用的牛羊等。錡(音奇):有足锅。釜:无足锅。牖(音有):天窗。尸:主持。古人祭祀用人充当神,称尸。齐(音斋):好貌。斋之省借。季女:少女。
\chapter{甘棠}
甘棠

蔽芾甘棠,勿翦勿伐,召伯所茇。

蔽芾甘棠,勿翦勿败,召伯所憩。

蔽芾甘棠,勿翦勿拜,召伯所说。

注释:人民怀念召伯。

蔽芾(音费):小貌。一说盛貌。甘棠:杜梨,落叶乔木,果实圆而小,味涩可食。茇(音拔):草舍。败:伐也。憩:休息。拜:拔也。说(音税):休憩,止息。
\chapter{行露}
行露

厌浥行露,岂不夙夜?谓行多露。

谁谓雀无角?何以穿我屋?

谁谓女无家?何以速我狱?

虽速我狱,室家不足!

谁谓鼠无牙?何以穿我墉?

谁谓女无家?何以速我讼?

虽速我讼,亦不女从!

注释:女子不为强暴所迫,作诗言志,以绝其人。

厌浥(音夜怡):沾湿。谓:可能是畏之假借。意指害怕行道多露。角:鸟喙。女:同汝,你。速:招,致。家:媒聘求为家室之礼也。一说婆家。墉(音拥):墙。
\chapter{羔羊}
羔羊

羔羊之皮,素丝五紽;

退食自公,委蛇委蛇。

羔羊之革,素丝五緎;

委蛇委蛇,自公退食。

羔羊之缝,素丝五总;

委蛇委蛇,退食自公。

注释:描写士大夫们的安闲生活。

紽(音驼):丝结,丝钮。食(音四):公家供卿大夫之常膳。委蛇(音尾移):大摇大摆洋洋自得。革:裘里。緎(音玉):缝也。缝:皮裘。总(音宗):纽结。
\chapter{殷其雷}
殷其雷

殷其雷,在南山之阳。

何斯违斯,莫敢或遑?

振振君子,归哉归哉!

殷其雷,在南山之侧。

何斯违斯,莫敢遑息?

振振君子,归哉归哉!

殷其雷,在南山之下。

何斯违斯,莫或遑处?

振振君子,归哉归哉!

注释:妇人盼望远役丈夫早早归来。

殷(音引):雷声也。一说喻车声。阳:山南为阳。斯:指示词。何斯,斯,此人也;违斯,斯,此地也。违:远也。 或:有。遑(音皇):闲暇。处:居也。
\chapter{摽有梅}

摽有梅,其实七兮!

求我庶士,迨其吉兮!

摽有梅,其实三兮!

求我庶士,迨其今兮!

摽有梅,顷筐塈之!

求我庶士,迨其谓之!

注释:女子希望男方及时前来求婚。

摽(音表,去声):一说坠落,一说掷、抛。七:一说非实数,古人以七到十表示多,三以下表示少。庶:众。迨(音代):及时。倾筐:撮萁之类。塈(音既):一说取,一说给。谓:一说聚会;一说开口说话;一说归,嫁。


\chapter{小星}
小星

嘒彼小星,三五在东。

肃肃宵征,夙夜在公。寔命不同!

嘒彼小星,维参与昴。

肃肃宵征,抱衾与裯。寔命不犹!

注释:位卑职微的小吏,对自己日夜奔忙的命运,发出不平的浩叹。

嘒(音慧):微光闪烁。三五:一说参三星,昴五星,指参昴。一说举天上星的数。肃肃:疾行貌。宵:夜。征:行。寔:实的异体字。是,此。参(音申):星名,二十八宿之一。昴(音卯):星名,二十八宿之一。抱:一说当读抛。抛弃。衾(音钦):被子。裯(音绸):被单。 犹:若,如,同。

\chapter{江有汜}
江有汜

江有汜,之子归,不我以!

不我以,其后也悔。

江有渚,之子归,不我与!

不我与,其后也处。

江有沱,之子归,不我过!

不我过,其啸也歌。

注释:媵女不得从嫁的怨词。一说弃妇怨词。

汜(音四):由主流分出而复汇合的河水。归:嫁。不我以:不用我。渚(音猪):小洲。不我与:不与我。处:忧。沱(音驼):江水的支流。过:至也。一说度。啸:一说蹙口出声,以抒愤懑之气,一说号哭。
\chapter{野有死麕}

野有死麇

野有死麇,白茅包之;

有女怀春,吉士诱之。

林有朴樕,野有死鹿;

白茅纯束,有女如玉。

舒而脱脱兮,

无感我帨兮,

无使尨也吠。

注释:青年男女,在郊外获得爱情。

麇(音军):獐子。比鹿小,无角。白茅:草名。吉士:男猎人。朴樕(音速):小木,灌木。纯束:捆扎。舒:一说语词,一说徐。脱脱(音兑):缓慢。感(音撼):通撼,动摇。帨(音税):佩巾,围腰。尨(音忙):多毛的狗。

\chapter{何彼秾矣}
何彼襛矣 

何彼襛矣,唐棣之华!

曷不肃雝?王姬之车。

何彼襛矣,华如桃李!

平王之孙,齐侯之子。

其钓维何?维丝伊缗。

齐侯之子,平王之孙。

注释:刺王姬出嫁,车服奢侈。

襛(音浓):繁盛貌。唐棣(音地):木名,似白杨。一说指车帷。曷(音何):何。肃:庄严。雝(音拥):雍容。平王、齐侯:指谁无定说。维:作语助。缗(音民):合股丝绳,喻男女合婚。

\chapter{驺虞}


彼茁者葭,壹发五豝,于嗟乎驺虞!

彼茁者蓬,壹发五豵,于嗟乎驺虞!

注释:称赞猎人的射击本领。

茁(音浊):草初生出地貌。葭(音家):初生的芦苇。壹:发语词。发:发矢。豝(音巴):小母猪。驺虞(音邹于):一说义兽,一说古牧猎官。蓬(音朋):草名,蒿也。豵(音宗):小猪。一岁曰豵。



\part{邶风}
柏舟、绿衣、燕燕、日月、终风、击鼓、凯风、雄雉、匏有苦叶、谷风、式微、旄丘、简兮、泉水、北门、北风、静女、新台、二子乘舟
二子乘舟 

二子乘舟,泛泛其景。

愿言思子,中心养养!

二子乘舟,泛泛其逝。

愿言思子,不瑕有害?

注释:父母悬念舟行的孩子。一说:卫宣公二子,争相为死,国人伤之而作是诗。

二子:卫宣公的两个异母子。景:通憬,远行貌。愿:思念貌。养养:忧貌。新台

新台有泚,河水瀰瀰。

燕婉之求,蘧篨不鲜。

新台有洒,河水浼浼。

燕婉之求,蘧篨不殄。

鱼网之设,鸿则离之。

燕婉之求,得此戚施。

注释:刺卫宣公筑新台强占儿媳的丑事。

泚(音此):鲜明貌。河水:黄河。瀰瀰(音米):大水茫茫。燕婉:燕,安;婉,顺。指夫妇和好。蘧篨(音渠除):鸡胸。一说蛤蟆。鲜:善。洒(音催):高峻。浼浼(音美):水盛貌。殄(音舔):善。鸿:蛤蟆。离:通罹,遭受。戚施:驼背,一说蛤蟆。静女

静女其姝,俟我於城隅。

爱而不见,搔首踟蹰。

静女其娈,贻我彤管。

彤管有炜,说怿女美。

自牧归荑,洵美且异。

匪女之为美,美人之贻。

注释:男女青年的幽期密约。一说刺卫宣公纳媳。

静:娴雅安详。姝(音书):美好。城隅:城角隐蔽处。爱:隐藏。踟躇(音池除):徘徊不定。娈:年轻美丽。彤管:一说红管的笔,一说和荑应是一物。说怿(音月义):喜悦。牧:野外。荑(音提):白茅,茅之始生也。象征婚媾。询:实在,诚然。北风

北风其凉,雨雪其雱。惠而好我,携手同行。

其虚其邪?既亟只且!

北风其喈,雨雪其霏。惠而好我,携手同归。

其虚其邪?既亟只且!

莫赤匪狐,莫黑匪乌。惠而好我,携手同车。

其虚其邪?既亟只且!

注释:情人相爱,愿在大风雪中同归去。一说卫行虐政,百姓惧祸,相携离去。

雨(音玉):作动词。雱(音旁):雪盛貌。惠:爱也。虚:宽貌。一说徐缓。邪:通徐。亟:急。只且(音居):作语助。喈(音皆):疾貌。霏:雨雪纷飞。莫赤匪狐:没有不红的狐狸。狐狸、乌鸦比喻坏人。一说古人将狐狸比喻为男性伴侣,将乌鸦视为吉祥鸟。北门

出自北门,忧心殷殷。

终窭且贫,莫知我艰。

已焉哉!天实为之,谓之何哉!

王事适我,政事一埤益我。

我入自外,室人交徧谪我。

已焉哉!天实为之,谓之何哉!

王事敦我,政事一埤遗我。

我入自外,室人交徧摧我。

已焉哉!天实为之,谓之何哉!

注释:这是一篇位卑任重,处境困穷,无处诉说的小官吏的怨诗。

窭(音巨):贫寒,艰窘。谓:犹奈也,即奈何不得之意。王事:周王的事。政事:公家的事。一:都。埤(音皮):加给。徧:同遍。谪(音哲):谴责。敦:逼迫。摧:挫也。讥讽。泉水

毖彼泉水,亦流于淇。有怀于卫,靡日不思。

娈彼诸姬,聊与之谋。

出宿于泲,饮饯于祢,女子有行,远父母兄弟。

问我诸姑,遂及伯姊。

出宿于干,饮饯于言。载脂载辖,还车言迈。

遄臻于卫,不瑕有害?

我思肥泉,兹之永叹。思须与漕,我心悠悠。

驾言出游,以写我忧。

注释:卫宣公之女许穆夫人怀念亲人,思慕祖国的诗篇。

毖(音必):泉水涌流。淇:淇水。娈:美好的样子。诸姬:同姓之女。聊:一说原,一说姑且。泲(音子)、祢(音米)、干、言:均为地名。行:指嫁。载:发语词。脂:涂车轴的油脂。辖:车轴两头的金属键。还车:回转车。迈:远。遄(音专):疾速。臻:至。瑕:远也。肥泉、须、漕:皆卫国的城邑。肥泉一说同出异归。写:除也。与卸音义同。简兮

简兮简兮,方将万舞。日之方中,在前上处。

硕人俣俣,公庭万舞。有力如虎,执辔如组。

左手执龠,右手秉翟。赫如渥赭,公言锡爵。

山有榛,隰有苓。云谁之思?西方美人。

彼美人兮,西方之人兮。

注释:刺贤人不为时用,居于伶馆。

简:一说鼓声,一说大貌。万舞:舞名。在前上处:在前列的上头。硕:大貌。俣俣(音与):大而美。辔(音配):马缰。组:丝织的宽带子。龠(音月):古乐器。三孔笛。翟(音敌):野鸡的尾羽。赫:红色。渥(音握):厚。赭:赤褐色,赭石。锡:赐。爵:青铜制酒器,用以温酒和盛酒。榛(音真):落叶灌木。花黄褐色,果实叫榛子,果皮坚硬,果肉可食。隰(音席):低下的湿地。苓(音零):一说甘草,一说苍耳,一说黄药,一说地黄。旄丘

旄丘之葛兮,何诞之节兮!

叔兮伯兮,何多日也?

何其处也?必有与也!何其久也?必有以也!

狐裘蒙戎,匪车不东。叔兮伯兮,靡所与同。

琐兮尾兮,流离之子。叔兮伯兮,褎如充耳。

注释:黎人遭流亡之苦,责卫不相救所发出的呼声。一说弃妇怨诗。

旄(音毛)丘:前高后低土山。诞(音丹):延,长。节:葛节巴。蒙戎:篷松,乱貌。匪:非。琐:小。尾:微。流离:鸟名,即枭。一说转徙离散。褎(音又):多笑貌。充耳:塞耳。古代挂在冠冕两旁的玉饰,用丝带下垂到耳门旁。式微

式微,式微!胡不归?

微君之故,胡为乎中露!

式微,式微!胡不归?

微君之躬,胡为乎泥中!

注释:人民行役,颠连困苦,对统治者发出不平的怨恨。 

式:作语助。微:昧,黄昏。微:非。中露:露中。倒文以协韵。躬:身体。谷风

习习谷风,以阴以雨。黾勉同心,不宜有怒。

采葑采菲,无以下体?德音莫违,及尔同死。

行道迟迟,中心有违。不远伊迩,薄送我畿。

谁谓荼苦?其甘如荠。宴尔新昏,如兄如弟。

泾以渭浊,湜湜其沚。宴尔新昏,不我屑矣。

毋逝我梁,毋发我笱。我躬不阅,遑恤我后!

就其深矣,方之舟之。就其浅矣,泳之游之。

何有何亡,黾勉求之。凡民有丧,匍匐求之。

不我能慉,反以我为雠,

既阻我德,贾用不售。

昔育恐育鞫,及尔颠覆。

既生既育,比予于毒。

我有旨蓄,亦以御冬。宴尔新昏,以我御穷。

有洸有溃,既诒我肄。不念昔者,伊余来塈。

注释:姑娘遭受遗弃,离家时,倾诉自己的不幸。

习习(音飒):象声词。谷风:一说东风,一说暴风,一说来自山谷的风。黾(音敏)勉:勉力。葑:蔓菁也。叶、根可食。菲:萝卜之类。无以下体:意指要叶不要根,比喻恋新人而弃旧人。迟迟:迟缓,徐行貌。违:恨也。畿(音机):指门槛。荼(音图):苦菜。荠:荠菜。宴:乐。泾、渭:河名。湜湜(音时):水清见底。沚(音止):水中小洲。一说止,沉淀。梁:捕鱼水坝。笱:捕鱼竹笼。阅:容纳。恤(音序):忧。能:通宁。匍匐:爬行。慉(音序):好,爱悦。雠(音仇):同仇。贾(音古):经商。育:长。鞠:穷。颠覆:艰难,患难。旨蓄:美菜。洸(音光):动武打人。溃(音愧):怒貌。既:尽。诒:遗。肄(音义):劳也。来:语词。塈(音系):爱。匏有苦叶

匏有苦叶,济有深涉。深则厉,浅则揭。

有瀰济盈,有鷕雉鸣。

济盈不濡轨,雉鸣求其牡。

雝雝鸣雁,旭日始旦。士如归妻,迨冰未泮。

招招舟子,人涉卬否。人涉卬否,卬须我友。

注释:妻子怀念远役在外的丈夫。

匏(音袍):葫芦之类。苦:一说苦味,一说枯。意指葫芦八月叶枯成熟,可以挖空作渡水工具。济:水名。涉:一说涉水过河,一说渡口。厉:带。一说以衣涉水,一说拴葫芦在腰泅渡。揭(音气):提起衣裳。瀰(音弥):大水茫茫。盈:满。鷕(音尾又音咬):雌山鸡叫声。不濡(音如):不,语词;濡,沾湿。轨:车轴头。雝雝(音拥):大雁叫声和谐。归妻:娶妻。迨(音带):及;乘时。泮(音盼):合,封冻。招招:摇橹曲伸之貌。卬(音昂):我。卬否:即我不走之意。须:等待。友:指爱侣。雄雉

雄雉于飞,泄泄其羽。我之怀矣,自诒伊阻。

雄雉于飞,下上其音。展矣君子,实劳我心。

瞻彼日月,悠悠我思。道之云远,曷云能来?

百尔君子,不知德行?不忮不求,何用不臧?

注释:妻子怀念远役在外的丈夫。

泄泄:缓飞貌。诒(音怡):通贻,遗留。伊:作语助。展:诚实。云:作语助。曷(音何):何,何时。百尔君子:汝众君子。忮(音至):忌恨,害也。臧(音脏):善。凯风

凯风自南,吹彼棘心。棘心夭夭,母氏劬劳。

凯风自南,吹彼棘薪。母氏甚善,我无令人。

爰有寒泉?在浚之下。有子七人,母氏劳苦。

睍睆黄鸟,载好其音。有子七人,莫慰母心。

注释:七子对母亲抚育劳苦的咏叹。

凯风:和风。一说南风。棘:落叶灌木,即酸枣。枝上多刺,开黄绿色小花,实小,味酸。心:指纤小尖刺。劬(音渠):辛苦。劬劳:操劳。薪:喻母。爰(音元):何处。睍睆(音现缓):犹“间关”,鸟叫声。一说美丽,好看。击鼓

击鼓其镗,踊跃用兵。土国城漕,我独南行。

从孙子仲,平陈与宋。不我以归,忧心有忡。

爰居爰处?爰丧其马?于以求之?于林之下。

死生契阔,与子成说。执子之手,与子偕老。

于嗟阔兮,不我活兮。于嗟洵兮,不我信兮。

注释:卫国兵士,远戍陈宋,久役不得归,怀念妻子,回忆临行与妻子诀别之词。

镗(音堂):鼓声。兵:武器,刀枪之类。土国:或役土功于国。漕:地名。平:和也,和二国之好。陈、宋:诸侯国名。孙子仲:卫国元帅。爰(音元):何处。爰居爰处?爰丧其马:有不还者,有亡其马者。契阔:聚散。成说:成言也。阔:疏也,远离别之意。洵:远。信:一说古伸字,志不得伸。一说誓约有信。终风

终风且暴,顾我则笑,谑浪笑敖,中心是悼。

终风且霾,惠然肯来,莫往莫来,悠悠我思。

终风且曀,不日有曀,寤言不寐,愿言则嚏。

曀曀其阴,虺虺其雷,寤言不寐,愿言则怀。

注释:女子对狂暴丈夫的怨恨。一说卫庄姜伤所遇不淑。

终:一说终日,一说既。暴:疾也。谑浪笑敖:戏谑。中心:心中。霾(音埋):阴霾。空气中悬浮着的大量烟尘所形成的混浊现象。曀(音壹):阴云密布有风。嚏(音替):打喷嚏。虺(音灰):形容雷声。日月

日居月诸,照临下土。

乃如之人兮,逝不古处?

胡能有定?宁不我顾。

日居月诸,下土是冒。

乃如之人兮,逝不相好。

胡能有定?宁不我报。

日居月诸,出自东方。

乃如之人兮,德音无良。

胡能有定?俾也可忘。

日居月诸,东方自出。

父兮母兮,畜我不卒。

胡能有定?报我不述。

注释:女子控诉丈夫对她的遗弃。一说卫庄姜为失宠于庄公而作。

居、诸:作语助。逝:发语词。古处:一说旧处,原来相处。一说姑处。定:止。指心定、心安。宁:一说乃;曾。一说岂;难道。冒:覆盖。畜我不卒:即好我不终。不述:不循义理。燕燕

燕燕于飞,差池其羽。之子于归,远送于野。

瞻望弗及,泣涕如雨。

燕燕于飞,颉之颃之。之子于归,远于将之。

瞻望弗及,伫立以泣。

燕燕于飞,下上其音。之子于归,远送于南。

瞻望弗及,实劳我心。

仲氏任只,其心塞渊。终温且惠,淑慎其身。

先君之思,以勖寡人。

注释:卫庄姜送归妾(陈女戴妫)。

燕燕:即燕子燕子。差池其羽:形容燕子张舒其尾翼。颉(音洁):上飞。颃(音航):下飞。伫:久立等待。南:一说野外。仲:排行第二。氏:姓氏。任:姓。只:语助词。终:既,已经。勖(音续):勉励。绿衣

绿兮衣兮,绿衣黄里。心之忧矣,曷维其已!

绿兮衣兮,绿衣黄裳。心之忧矣,曷维其亡!

绿兮丝兮,女所治兮。我思古人,俾无訧兮!

絺兮绤兮,凄其以风。我思古人,实获我心!

注释:诗人睹物伤心,感情缠绵地悼念亡妻。

衣、里、裳:上曰衣,下曰裳;外曰衣,内曰里。已:止。亡:一说通忘,一说停止。古人:故人,指亡妻。俾(音比):使。訧(音尤):同尤,过失,罪过。絺(音吃):细葛布。绤(音戏):粗葛布。凄:凉意。柏舟

泛彼柏舟,亦泛其流。耿耿不寐,如有隐忧。

微我无酒,以敖以游。

我心匪鉴,不可以茹。亦有兄弟,不可以据。

薄言往愬,逢彼之怒。

我心匪石,不可转也。我心匪席,不可卷也。

威仪棣棣,不可选也。

忧心悄悄,愠于群小。觏闵既多,受侮不少。

静言思之,寤辟有摽。

日居月诸,胡迭而微?心之忧矣,如匪浣衣。

静言思之,不能奋飞。

注释:妇人遭受遗弃,又为群小所欺,坚持真理,不甘屈服的抒愤诗。

泛:浮行,随水冲走。流:中流,水中间。耿耿:形容心中不安。隐忧:深忧。微:非,不是。鉴:铜镜。茹(音如):度,或容。据:依靠。愬(音诉):告诉。棣棣:雍容娴雅貌。选:屈挠退让貌。悄悄:忧貌。愠(音运):怨恨。觏(音够):遭逢。闵(音敏):忧伤。寤:交互。辟(音屁):捶胸。摽(音表,去声):捶,打。居、诸:语助词。迭:更动。微:指隐微无光。



\part{鄘风}
柏舟、墙有茨、君子偕老、桑中、鹑之奔奔、定之方中、蝃蝀、相鼠、干旄、载驰
载驰

载驰载驱,归唁卫侯。驱马悠悠,言至于漕。

大夫跋涉,我心则忧。

既不我嘉,不能旋反。视而不臧,我思不远。

既不我嘉,不能旋济。视而不臧,我思不閟。

陟彼阿丘,言采其蝱。女子善怀,亦各有行。

许人尤之,众樨且狂。

我行其野,芃芃其麦。控于大邦,谁因谁极?

大夫君子,无我有尤。

百尔所思,不如我所之。

注释:许穆夫人念故国覆亡,不能往救,赴漕吊唁,并陈立国大计,到漕邑为许大夫所阻,因赋诗以言志。

唁(音厌):吊失国曰唁。卫侯:指已死的卫戴公申。悠悠:远貌。大夫:指许国赶来阻止许穆夫人去卫的许臣。嘉:善。思:忧思。远:摆脱。济:止。閟(音必):慎。阿丘:有一边偏高的山丘。蝱(音萌):贝母草。采蝱治病,喻设法救国。行:道路。许人:许国的人们。尤:过。众:既是。樨:幼稚。芃(音彭):草茂盛貌。因:亲也。极:同急。干旄

孑孑干旄,在浚之郊。素丝纰之,良马四之。

彼姝者子,何以畀之?

孑孑干旟,在浚之都。素丝组之,良马五之。

彼姝者子,何以予之?

孑孑干旌,在浚之城。素丝祝之,良马六之。

彼姝者子,何以告之?

注释:赞美卫文公群臣乐于招贤纳士。

孑孑:特出之貌。指旗显眼,高挂干上。干旄(音毛):以牦牛尾饰旗杆,树于车后,以状威仪。干通竿、杆。浚:地名。纰(音皮):连缀。在衣冠或旗帜上镶边。畀(音必):给,予。旟(音于):画有鸟隼的旗。都:古时地方的区域名。旌(音京):旗的一种。挂牦牛尾于竿头,下有五彩鸟羽。告(音谷):作名词用,忠言也。一说告同予。相鼠

相鼠有皮,人而无仪!

人而无仪,不死何为?

相鼠有齿,人而无止!

人而无止,不死何俟?

相鼠有体,人而无礼!

人而无礼,胡不遄死?

注释:统治阶级用虚伪礼节欺骗人民,人民深恶痛绝,比之为鼠,给予辛辣的讽刺。

相:视也。仪:威仪也。止:容止。指守礼法的行为。遄(音船):速。蝃蝀

蝃蝀在东,莫之敢指。

女子有行,远父母兄弟。

朝隮于西,崇朝其雨。

女子有行,远父母兄弟。

乃如之人也,怀婚姻也。

大无信也,不知命也!

注释:女子找爱人,却遭毁谤。

蝃蝀(音帝东):虹,爱情与婚姻的象征。在东:暮虹出在东方。行:指出嫁。隮(音记):一说升云,一说虹。崇朝:终朝。如:往。怀:欲,想。无信:一说不守媒妁之言。不知命也:不知婚姻当待父母之命。定之方中

定之方中,作于楚宫。揆之以日,作于楚室。

树之榛栗,椅桐梓漆,爰伐琴瑟。

升彼虚矣,以望楚矣。望楚与堂,景山与京。

降观于桑,卜云其吉,终焉允臧。

灵雨既零,命彼倌人,星言夙驾,说于桑田。

匪直也人,秉心塞渊,騋牝三千。

注释:赞卫文公徙迁复国,从事建设,大兴农业,繁殖六畜,克勤克俭,使国人能得其所。

定:星宿名,又叫营室星。十月之交,定星昏中而正,宜定方位,造宫室。于:古声与为通,作为之意。楚:楚丘,地名。揆(音葵):测度。榛、栗、椅、桐、梓、漆:皆木名。椅,山桐子。虚(音区):一说故城,一说大丘。堂:楚丘旁邑。景山:大山。京:高丘。臧:好,善。灵:善。零:落雨。倌:驾车小臣。星言:晴焉。夙:早上。说于桑田:指文公关心农桑。匪:同非。直:特也。騋(音来):七尺以上的马。牝(音聘):母马。 鹑之奔奔

鹑之奔奔,鹊之彊彊。

人之无良,我以为兄?

鹊之彊彊,鹑之奔奔。

人之无良,我以为君?

注释:卫国人民讽刺卫宣公,以为鸟兽之不若。

鹑:鸟名,即鹌鹑。奔奔、彊彊(音疆):形容鹑鹊居有常匹,飞则相随的样子。我:古音我、何相通。君:指卫宣公。桑中

爰采唐矣?沫之乡矣。云谁之思?美孟姜矣。

期我乎桑中,要我乎上宫,送我乎淇之上矣。

爰采麦矣?沫之北矣。云谁之思?美孟弋矣。

期我乎桑中,要我乎上宫,送我乎淇之上矣。

爰采葑矣?沫之东矣。云谁之思?美孟庸矣。

期我乎桑中,要我乎上宫,送我乎淇之上矣。

注释:卫人讽刺贵族男女幽期密约的诗篇。

爰:于何。唐:植物名。即菟丝子,寄生蔓草,秋初开小花,子实入药。沫(音妹):卫邑名。谁之思:思念的是谁。孟:老大。孟姜:姓姜的大姑娘。姜、弋、庸,皆贵族姓。桑中:地名。要(音邀):邀约。上宫:楼也,指宫室。一说地名。淇:淇水。葑(音封):蔓菁菜。君子偕老

君子偕老,副笄六珈。

委委佗佗,如山如河,象服是宜。

子之不淑,云如之何?

玼兮玼兮,其之翟也。鬒发如云,不屑髢也;

玉之瑱也,象之挮也,扬且之皙也。

胡然而天也?胡然而帝也?

瑳兮瑳兮,其之展也,蒙彼绉絺,是绁袢也。

子之清扬,扬且之颜也。

展如之人兮,邦之媛也!

注释:赞扬贵妇人华服美饰,人极漂亮,然而本质极坏。似赞扬而实讽刺。

副:妇人的一种首饰。笄(音鸡):簪。六珈:笄饰,用玉做成,垂珠有六颗。委委佗佗(音驼):雍容自得之貌。象服:华丽的礼服。如之何:奈之何。玼(音此):花纹绚烂。翟:绣着山鸡的象服。鬒(音诊):黑发。髢(音敌):假发。瑱(音掭):冠冕上垂在两耳旁的玉。挮(音替):剃发针,发钗一类的首饰。一说可用于搔头。扬:额。皙(音希):白净。瑳(音搓):玉色鲜明洁白。展:古代后妃或命妇的一种礼服。絺(音吃):细葛布。绁袢:夏天穿的薄衫。扬:目。扬:额。颜:额。引申为面容、脸色。展:诚。媛:美女。墙有茨

墙有茨,不可扫也。中冓之言,不可道也。

所可道也,言之丑也。

墙有茨,不可襄也。中冓之言,不可详也。

所可详也,言之长也。

墙有茨,不可束也。中冓之言,不可读也。

所可读也,言之辱也。

注释:卫国人民对统治者荒淫无耻的揭露。

茨(音词):植物名,蒺藜。一年生草本植物,果实有刺。中冓(音够):内室,宫中龌龊之事。襄:除去。详:指讲话。束:捆走。读:诵也。柏舟

泛彼柏舟,在彼中河。髧彼两髦,实维我仪。

之死矢靡它。母也天只!不谅人只!

泛彼柏舟,在彼河侧。髧彼两髦,实维我特。

之死矢靡慝。母也天只!不谅人只!

注释:姑娘婚姻不得自由,向母亲倾诉她坚贞的爱情。

髧(音旦):头发下垂状。两髦(音毛):男子未成年时剪发齐眉。仪:配偶。之:到。矢:誓。靡它:无他心。只:语助词。特:配偶。慝(音特):邪恶,恶念,引申为变心。



\part{卫风}
淇奥、考盘、硕人、氓、竹竿、芄兰、河广、伯兮、有狐、木瓜
木瓜

投我以木瓜,报之以琼琚。

匪报也,永以为好也!

投我以木桃,报之以琼瑶。

匪报也,永以为好也!

投我以木李,报之以琼玖。

匪报也,永以为好也!

注释:男女相爱,互相赠答。一说卫人思报齐桓公复国厚恩而作。

琼:赤色玉。亦泛指美玉。琚(音居):佩玉。匪:非。瑶:美玉。一说似玉的美石。玖(音久):浅黑色玉石。有狐

有狐绥绥,在彼淇梁。

心之忧矣,之子无裳。

有狐绥绥,在彼淇厉。

心之忧矣,之子无带。

有狐绥绥,在彼淇侧。

心之忧矣,之子无服。

注释:女向男求爱。虽其人贫无衣裤,但仍爱他。一说妻子怀念久役不归的丈夫。

狐:一说狐喻男性。绥绥:从容独行的样子。一说行迟貌,一说多貌貌。淇:水名。梁:河梁。河中垒石而成,可以过人,可用于拦鱼。裳:上曰衣,下曰裳。厉:水深及腰,可以涉过之处。一说流水的沙滩。伯兮

伯兮朅兮,邦之桀兮。伯也执殳,为王前驱。

自伯之东,首如飞蓬。岂无膏沐?谁适为容!

其雨其雨,杲杲出日。愿言思伯,甘心首疾。

焉得谖草?言树之背。愿言思伯,使我心痗。

注释:丈夫久役不归,妻子怀念远人的抒情诗。

朅(音切):英武高大。殳(音书):古兵器,杖类。长丈二无刃。膏沐:妇女润发的油脂。杲(音稿):明亮的样子。谖草:萱草,忘忧草。背:北。指北堂。痗(音妹):忧思成病。河广

谁谓河广?一苇杭之。

谁谓宋远?跂予望之。

谁谓河广?曾不容刀。

谁谓宋远?曾不崇朝。

注释:宋人侨居卫国,思乡不得归,以诗抒发怀念之情。

河:黄河。卫国在戴公之前,都于朝歌,和宋国隔河相望。跂(音气):踮起脚跟。刀:小船。崇朝:终朝。芄兰

芄兰之支,童子佩觿。

虽则佩觿,能不我知?

容兮遂兮,垂带悸兮。

芄兰之叶,童子佩韘。

虽则佩韘,能不我甲?

容兮遂兮,垂带悸兮。

注释:人民对统治者骄横幼稚装腔作势不称其服的讽刺。

芄(音丸)兰:植物名。草本,蔓生。觿(音西):象骨制的解结用具,形同锥,也可为装饰品。成人佩饰。知:智。容:佩刀。遂:佩玉。一说容、遂,舒缓放肆之貌。悸:带下垂貌。韘(音社):象骨制的钩弦用具,著于右手拇指,射箭时用于钩弦。甲:长也。竹竿

藋藋竹竿,以钓于淇。岂不尔思?远莫致之。

泉源在左,淇水在右。女子有行,远兄弟父母。

淇水在右,泉源在左。巧笑之瑳,佩玉之傩。

淇水滺滺,桧楫松舟。驾言出游,以写我忧。

注释:卫女远嫁诸侯,欲归不能,以诗抒思父母、念故园之情。

藋藋(音替):长而尖削貌。泉源:一说水名。即百泉,在卫之西北,而东南流入淇水。瑳(音搓):玉色洁白。傩(音挪):通娜。婀娜。滺(音悠):河水荡漾之状。楫:船桨。桧、松:木名。桧,柏叶松身。氓

氓之蚩蚩,抱布贸丝。匪来贸丝,来即我谋。

送子涉淇,至于顿丘。匪我愆期,子无良媒。

将子无怒,秋以为期。

乘彼垝垣,以望复关。不见复关,泣涕涟涟。

既见复关,载笑载言。尔卜尔筮,体无咎言。

以尔车来,以我贿迁。

桑之未落,其叶沃若。于嗟鸠兮,无食桑葚;

于嗟女兮,无与士耽。士之耽兮,犹可说也;

女之耽兮,不可说也。

桑之落矣,其黄而陨。自我徂尔,三岁食贫。

淇水汤汤,渐车帷裳。女也不爽,士贰其行。

士也罔极,二三其德。

三岁为妇,靡室劳矣;夙兴夜寐,靡有朝矣。

言既遂矣,至于暴矣。兄弟不知,咥其笑矣。

静言思之,躬自悼矣。

及尔偕老,老使我怨。淇则有岸,隰则有泮。

总角之宴,言笑晏晏。信誓旦旦,不思其反。

反是不思,亦已焉哉!

注释:一个勤劳善良的妇女,哀诉她被遗弃的不幸遭遇。

氓:民,男子。蚩蚩:老实的样子。一说无知貌,一说戏笑貌。布:货币。一说布匹。即:靠近。谋:商量。顿丘:地名。愆(音千):过,误。将:愿,请。垝垣(音鬼员):破颓的墙。复关:诗中男子的住地。一说返回关来。卜:用龟甲卜吉凶。筮(音诗):用蓍草占吉凶。体:卜卦之体。咎言:凶,不吉之言。贿:财物,嫁妆。沃若:润泽貌。鸠:斑鸠。传说斑鸠吃桑葚过多会醉。耽(音沉):沉湎于爱情。说:脱。陨:坠落。徂尔:往你家,嫁与你。食贫:过贫苦生活。渐(音尖):沾湿。爽:差错。贰(音特):差错。罔极:没有准则,行为不端。二三其德:三心二意。遂:久。知:智。咥(音系):大笑貌。躬:自己,自身。淇:淇水。隰:当作湿,水名,即漯河。泮(音判):通畔,岸,水边。总角:古时儿童两边梳辫,如双角。指童年。硕人

硕人其颀,衣锦褧衣。齐侯之子,卫侯之妻。

东宫之妹,邢侯之姨,谭公维私。

手如柔荑,肤如凝脂,领如蝤蛴,齿如瓠犀。

螓首蛾眉,巧笑倩兮,美目盼兮。

硕人敖敖,说于农郊。

四牡有骄,朱幩镳镳,翟茀以朝。

大夫夙退,无使君劳。

河水洋洋,北流活活。

施罛濊濊,鱣鲔发发,葭菼揭揭。

庶姜孽孽,庶士有朅。

注释:卫庄公夫人庄姜初适卫,国人称赞她的美丽。

硕人:高大白胖的人。颀(音其):修长貌。锦:锦衣,翟衣。褧(音窘):妇女出嫁时御风尘用的麻布罩衣,即披风。东宫:指太子。私:姊妹之夫。荑(音题):白茅之芽。蝤蛴(音求其):天牛的幼虫,色白身长。瓠犀(音户西):瓠瓜子儿。螓(音秦):似蝉而小,头宽广方正。蛾眉:蚕蛾触角,细长而曲。倩:笑靥美好貌。盼:眼儿黑白分明。敖敖:身长貌。说(音税):停。农郊:近郊。一说东郊。幩(音坟):装在马口上的扇汗用具。镳镳(音标):马嚼子。一说盛多的样子。翟茀(音敌扶):以雉羽为饰的车围子。河水:黄河。活活(音郭):水流声。施:设。罛(音古):大的鱼网。濊濊(音或):撒网入水声。鱣(音沾):鳇鱼。一说赤鲤。鲔(音委):鲟鱼。一说鲤属。发发(音泼):鱼尾击水之声。一说盛貌。葭(音家):初生的芦苇。菼(音坦):初生的荻。揭揭:长貌。庶姜:指随嫁的姜姓众女。孽孽:盛饰貌。士:从嫁的媵臣。朅(音怯):勇武貌。考槃

考槃在涧,硕人之宽。

独寤寐言,永矢弗谖。

考槃在阿,硕人之薖。

独寤寐歌,永矢弗过。

考槃在陆,硕人之轴。

独寤寐宿,永矢弗告。

注释:留恋山林生活的赞歌。

考槃(音盘):盘桓之意。一说槃为木盘。硕人:美人,贤人。宽:心宽。一说貌美。阿:山阿,山的曲隅。薖(音科):貌美,引为心胸宽大。过:失也,失亦忘也。陆:高平曰陆。轴:徘徊往复。一说美貌。淇奥

瞻彼淇奥,绿竹猗猗。

有匪君子,如切如磋,如琢如磨。

瑟兮僩兮,赫兮咺兮,

有匪君子,终不可谖兮!

瞻彼淇奥,绿竹青青。

有匪君子,充耳琇莹,会弁如星。

瑟兮僩兮,赫兮咺兮,

有匪君子,终不可谖兮!

瞻彼淇奥,绿竹如箦。

有匪君子,如金如锡,如圭如璧。

宽兮绰兮,猗重较兮,善戏谑兮,不为虐兮!

注释:卫武公为周平王卿相,年过九十,深自儆惕,有文章,能纳人规谏。卫人颂其德,作此诗。

淇:淇水。奥(音玉):水边弯曲的地方。绿竹:一说绿为王芻,竹为萹蓄。猗猗(音恶,平声):通阿,长而美貌。匪:通斐,有文采貌。切、磋、琢、磨:治骨曰切,象曰磋,玉曰琢,石曰磨。均指文采好,有修养。瑟:庄严貌。僩(音县):宽大貌。赫:威严貌。咺(音宣):有威仪貌。谖(音宣):忘。琇(音秀)莹:美石,宝石。会弁(音贵变):鹿皮帽。会,鹿皮会合处,缀宝石如星。箦(音责):积的假借。茂密的样子。绰:旷达。一说柔和貌。猗(音以):通倚。重较(音虫觉):车厢上有两重横木的车子。为古代卿士所乘。戏谑:开玩笑。虐:粗暴。



\part{王风}
黍离、君子于役、君子阳阳、扬之水、中谷有蓷、兔爰、葛藟、采葛、大车、丘中有麻
丘中有麻

丘中有麻,彼留子嗟。

彼留子嗟,将其来施施。

丘中有麦,彼留子国。

彼留子国,将其来食。

丘中有李,彼留之子。

彼留之子,贻我佩玖。

注释:女子等待情人时所作的各种悬想。

留:一说留客的留,一说指刘姓。子嗟、子国:一说均是刘氏一人数名。将:请;愿;希望。施施:高兴貌。大车

大车槛槛,毳衣如菼。

岂不尔思?畏子不敢。

大车啍啍,毳衣如璊。

岂不尔思?畏子不奔。

谷则异室,死则同穴。

谓予不信,有如皦日。

注释:女子对男子表示坚贞的爱情。

大车:古代用牛拉货的车。槛槛(音砍):车轮的响声。毳(音粹)衣:古代冕服,一种绣衣。一说毡子和车衣。菼(音坦):初生的荻苇,形容嫩绿色。啍啍(音吞):重滞徐缓的样子。璊(音门):红色美玉,喻红色。一说赤苗的谷。谷:生,活着。皦(音缴):同皎,光亮。 采葛

彼采葛兮,一日不见,如三月兮!

彼采萧兮,一日不见,如三秋兮!

彼采艾兮,一日不见,如三岁兮!

注释:情人相思之词。一说朋友相念。

萧:植物名。蒿的一种,即青蒿。有香气,古时用于祭祀。三秋:通常一秋为一年,后又有专指秋三月的用法。这里三秋长于三月,短于三年,义同三季。艾:植物名。 葛藟

绵绵葛藟,在河之浒。

终远兄弟,谓他人父。

谓他人父,亦莫我顾!

绵绵葛藟,在河之涘。

终远兄弟,谓他人母。

谓他人母,亦莫我有!

绵绵葛藟,在河之漘。

终远兄弟,谓他人昆。

谓他人昆,亦莫我闻!

注释:父母兄弟离散,流离失所、寄人篱下的青年的痛苦的呼声。

绵绵:长而不绝之貌。葛、藟(音垒):藤类蔓生植物。浒(音虎):水边。远(音院):远离。顾、有、闻:皆亲爱之意也。涘(音四):水边。漘(音纯):河岸,水边。昆:兄。兔爰

有兔爰爰,雉离于罗。我生之初,尚无为;

我生之后,逢此百罹。尚寐无吪!

有兔爰爰,雉离于罦。我生之初,尚无造;

我生之后,逢此百忧。尚寐无觉!

有兔爰爰,雉离于罿。我生之初,尚无庸;

我生之后,逢此百凶。尚寐无聪!

注释:没落贵族感叹生不逢时,遭受百忧的歌篇。

爰(音缓):缓之借,逍遥自在。离:同罹,陷,遭难。罗:罗网。生之初:生之前。无为:无事。为、造、庸皆为劳役之事。无吪(音俄):不说话。一说不动。罦(音浮):一种装设机关的网,能自动掩捕鸟兽,又叫覆车网。罿(音冲):捕鸟的网。中谷有蓷

中谷有蓷,暵其乾矣。有女仳离,嘅其叹矣。

嘅其叹矣,遇人之艰难矣。

中谷有蓷,暵其修矣。有女仳离,条其啸矣。

条其啸矣,遇人之不淑矣。

中谷有蓷,暵其湿矣。有女仳离,啜其泣矣。

啜其泣矣,何嗟及矣。

注释:写凶年饥馑,室家相弃,弃妇无告的悲苦。

蓷(音推):药草名,即益母草。暵(音汉):水濡而干也。仳(音痞)离:别离。嘅(音慨):叹息。遇人之艰难矣:即嫁个好男人不容易。修:干肉。条:长也。扬之水

扬之水,不流束薪。彼其之子,不与我戍申。

怀哉怀哉,曷月予还归哉!

扬之水,不流束楚。彼其之子,不与我戍甫。

怀哉怀哉,曷月予还归哉!

扬之水,不流束蒲。彼其之子,不与我戍许。

怀哉怀哉,曷月予还归哉!

注释:周平王母家申国邻楚,数被侵伐,因遣戍守申,使人民家室离散,国人作诗讽之。

扬之水:激扬之水,喻夫。束薪:喻婚姻,在此指妻。彼其之子:指妻、子。戍申:在申地防守。甫:即吕国。蒲:蒲柳。许:许国。 君子阳阳

君子阳阳,左执簧,

右招我由房,其乐只且!

君子陶陶,左执翿,

右招我由敖,其乐只且!

注释:情人相约出游,感到乐趣无穷。

阳阳:洋洋得意。簧:笙簧。这里指笙。由房、由敖:遨游。只且(音居):语助词。陶陶:和乐貌。翿(音道):歌舞所用道具,羽毛做成。 君子于役

君子于役,不知其期,曷至哉?

鸡栖于埘,日之夕矣,羊牛下来。

君子于役,如之何勿思!

君子于役,不日不月,曷其有佸?

鸡栖于桀,日之夕矣,羊牛下括。

君子于役,苟无饥渴!

注释:妻子怀念行役无期不能归家的丈夫。

役:服劳役。曷:何时。至:归家。埘(音时):鸡舍。墙壁上挖洞做成。如之何勿思:如何不思。不日不月:没法用日月来计算时间。有佸(音又活):相会。桀:鸡栖木。括:来。苟:大概,也许。 黍离

彼黍离离,彼稷之苗。行迈靡靡,中心摇摇。

知我者谓我心忧,不知我者谓我何求。

悠悠苍天!此何人哉?

彼黍离离,彼稷之穗。行迈靡靡,中心如醉。

知我者谓我心忧,不知我者谓我何求。

悠悠苍天!此何人哉?

彼黍离离,彼稷之实。行迈靡靡,中心如噎。

知我者谓我心忧,不知我者谓我何求。

悠悠苍天!此何人哉?

注释:周大夫行役路过宗周镐京,见旧时宗庙宫室遗址,黍稷茂盛,因悲周室颠覆,乃作此诗。 

黍、稷(音蜀、记):两种农作物。离离:行列貌。靡靡:行步迟缓貌。摇摇:形容心神不安。此何人哉:致此颠覆者是什么人?噎(音耶):忧深气逆不能呼吸。



\part{郑风}
缁衣、将仲子、叔于田、大叔于田、清人、羔裘、遵大路、女曰鸡鸣、有女同车、山有扶苏、萚兮、狡童、褰裳、丰、东门之墠、风雨、子衿、扬之水、出其东门、野有蔓草、溱洧
溱洧

溱与洧,方涣涣兮。士与女,方秉蕑兮。

女曰观乎?士曰既且,且往观乎?

洧之外,洵訏且乐。

维士与女,伊其相谑,赠之以勺药。

溱与洧,浏其清矣。士与女,殷其盈兮。

女曰观乎?士曰既且,且往观乎?

洧之外,洵訏且乐。

维士与女,伊其将谑,赠之以勺药。

注释:青年男女春游之乐。一说夫妇同游之乐。

溱(音针)、洧(音伟):河名。涣涣:春水盛貌。秉:执。蕑(音坚):一种兰草。又名大泽兰,与山兰有别。訏(音虚):大。芍药:一说与今之芍药不同,一种香草。浏(音刘):水深而清。殷:众多。野有蔓草

野有蔓草,零露漙兮。

有美一人,清扬婉兮。

邂逅相遇,适我愿兮。

野有蔓草,零露瀼瀼。

有美一人,婉如清扬。

邂逅相遇,与子偕臧。

注释:情人不期而遇的喜悦。

蔓(音万):茂盛。漙(音团):形容露水多。清扬:目以清明为美,扬亦明也。婉:美好。邂逅(音谢后):不期而遇。瀼(音瓤):形容露水多。臧(音脏):好,善。 出其东门

出其东门,有女如云。

虽则如云,匪我思存。

缟衣綦巾,聊乐我员。

出其闉闍,有女如荼。

虽则如荼,匪我思且。

缟衣茹藘,聊可与娱。

注释:男子表现自己爱有所专。

思:语助词。存:一说在;一说念;一说慰籍。缟(音稿):白色;素白绢。綦(音机):暗绿色。员:语助词。闉闍(音因都):城外曲城的重门。荼:一说白茅花。且:语助词。一说慰籍。茹藘:茜草。扬之水

扬之水,不流束楚。

终鲜兄弟,维予与女。

无信人之言,人实诳女。

扬之水,不流束薪。

终鲜兄弟,维予二人。

无信人之言,人实不信。

注释:妻子劝丈夫勿信谗言。

楚:荆条。鲜(音选):少。言:流言。诳(音筐):骗。 子衿

青青子衿,悠悠我心。

纵我不往,子宁不嗣音?

青青子佩,悠悠我思。

纵我不往,子宁不来?

挑兮达兮,在城阙兮。

一日不见,如三月兮。

注释:妻子等候亲人时的焦急心情。

子:男子的美称。衿:衣领。嗣音:传音讯。挑兮达兮:往来轻疾貌。城阙:城正面夹门两旁之楼。 东门之墠

东门之墠,茹藘在阪。

其室则迩,其人甚远。

东门之栗,有践家室。

岂不尔思?子不我即。

注释:怨情人不来相聚。

墠(音善):土坪。茹藘(音如虑):草名。即茜草,可染红色。阪(音板):坡。践:成行成列。即:就。丰

子之丰兮,俟我乎巷兮,悔予不送兮。

子之昌兮,俟我乎堂兮,悔予不将兮。

衣锦褧衣,裳锦褧裳。叔兮伯兮,驾予与行。

裳锦褧裳,衣锦褧衣。叔兮伯兮,驾予与归。

注释:女子后悔没与情人同行,盼他来驾车同去。

丰:丰满,标致。俟(音四):等候。送:致女曰送,亲迎曰逆。昌:健壮,棒。将:出嫁时的迎送。锦:锦衣,翟衣。褧(音窘):妇女出嫁时御风尘用的麻布罩衣,即披风。行(音航):往。归:回。一说指女子出嫁。 褰裳

子惠思我,褰裳涉溱。

子不我思,岂无他人?

狂童之狂也且!

子惠思我,褰裳涉洧。

子不我思,岂无他士?

狂童之狂也且!

注释:男女间戏谑之辞。表现民间男女爱情。

惠:爱。褰(音千):揭起。溱(音臻):水名。狂:痴狂。也且(音居):作语助。洧(音伟):水名。即今河南省双泪河。 狡童

彼狡童兮,不与我言兮。

维子之故,使我不能餐兮。

彼狡童兮,不与我食兮。

维子之故,使我不能息兮。

注释:女子对爱人的责怨。

狡:狡猾。维:因为。 箨兮

箨兮箨兮,风其吹女。

叔兮伯兮,倡予和女。

箨兮箨兮,风其漂女。

叔兮伯兮,倡予要女。

注释:女子要求和爱人共同唱歌。

箨(音拓):脱落的木叶。倡:一说倡导,一说唱。漂:飘。要(音腰):成也,和也。 山有扶苏

山有扶苏,隰有荷华。

不见子都,乃见狂且。

山有乔松,隰有游龙。

不见子充,乃见狡童。

注释:女子与情人失约的感叹。一说刺时人美恶不辨。

扶苏:茂木。一说桑树。隰(音席):洼地。狂:狂愚的人。且(音居):助词。一说拙、钝也。游龙:植物名。即荭草。有女同车

有女同车,颜如舜华,

将翱将翔,佩玉琼琚。

彼美孟姜,洵美且都。

有女同行,颜如舜英,

将翱将翔,佩玉将将。

彼美孟姜,德音不忘。

注释:赞美同车姑娘孟姜。

同车:一说男子驾车到女家迎娶。舜:植物名,即芙蓉花,又名木槿。华、英:花。都:闲雅。行:音航。将将(音枪):即锵锵。女曰鸡鸣

女曰鸡鸣,士曰昧旦。子兴视夜,明星有烂。

将翱将翔,弋凫与雁。

弋言加之,与子宜之。宜言饮酒,与子偕老。

琴瑟在御,莫不静好。

知子之来之,杂佩以赠之。

知子之顺之,杂佩以问之。

知子之好之,杂佩以报之。

注释:猎人夫妇,相戒早起及相互爱悦。

昧旦:天色将明未明之际。明星:启明星。弋(音贻):射箭,以生丝系矢。加:射中。一说“加豆”,食器。宜:肴也。这里作动词。御:奏。静:美好。来:读为劳,抚慰之意。杂佩:玉佩。用各种佩玉构成,称杂佩。问:赠送。 遵大路

遵大路兮,掺执子之祛兮,

无我恶兮,不寁故也!

遵大路兮,掺执子之手兮,

无我丑兮,不寁好也!

注释:劳动人民爱情诗。女要求男莫要骤然丢掉旧情。

掺(音闪):执。祛(音区):袖口。寁(音趱或捷):迅速。故:故人。好:旧好。 羔裘

羔裘如濡,洵直且侯。

彼其之子,舍命不渝。

羔裘豹饰,孔武有力。

彼其之子,邦之司直。

羔裘晏兮,三英粲兮。

彼其之子,邦之彦兮。

注释:陈古以讽今,刺当时郑国无此等臣僚。

羔裘:羔羊皮裘。一说古大夫的朝服。濡(音如):湿,润泽 洵(音询):信。诚然,的确。侯:美。渝:变。豹饰:用豹皮做衣服的边。孔:甚;很。司直:负责正人过失的官吏。晏:鲜盛貌。英:裘饰。粲:光耀。彦:士的美称。 清人

清人在彭,驷介旁旁。

二矛重英,河上乎翱翔。

清人在消,驷介镳镳。

二矛重乔,河上乎逍遥。

清人在轴,驷介陶陶。

左旋右抽,中军作好。

注释:郑文公恶高克贪而无理,使将清邑之兵御敌于河上,久而不召,师散而归,高克奔陈。公子素以为高克进不按理;文公退高克不循法,作此诗加以讽刺。

清:郑国之邑。彭、消、轴:河上地名。驷:驾车的四匹马。介:甲。旁旁:盛貌。二矛:酋矛、夷矛。重英:以朱羽为矛饰,二矛树车上,遥遥相对,重叠相见。镳镳(音标):威武貌。乔:雉羽。陶陶:驱驰之貌。左旋右抽:御者在车左,执辔御马;勇士在车右,执兵击刺。旋,还车。抽,拔刀。中军:古三军为上军、中军、下军,中军将帅为主帅。 大叔于田

叔于田,乘乘马。执辔如组,两骖如舞。

叔在薮,火烈具举。襢裼暴虎,献于公所。

将叔勿狃,戒其伤女。

叔于田,乘乘黄。两服上襄,两骖雁行。

叔在薮,火烈具扬。叔善射忌,又良御忌。

抑磬控忌,抑纵送忌。

叔于田,乘乘鸨。两服齐首,两骖如手。

叔在薮,火烈具阜。叔马慢忌,叔发罕忌,

抑释冰忌,抑鬯弓忌。

注释:赞美善射、善猎、善御和善骑的青年猎人。一说同上篇刺郑庄公。

乘乘(音成盛):前一乘为动词,后为名词。四马驾车叫乘。骖(音参):车辕外侧两马。薮(音擞):沼泽地带。襢裼(音檀西):赤膊。暴虎:徒手搏虎。公所:君王的宫室。狃(音纽):反复做某事。黄:黄马。服:中央驾辕的马。忌:作语助。抑:作语助。磬(音庆)、控、纵、送:骋马曰磬,止马曰控,发矢曰纵,从禽曰送。皆言御者驰逐之貌。鸨(音保):有黑白杂毛的马。阜:旺盛。罕:稀少。冰:箭筒盖。鬯(音唱):弓囊。叔于田

叔于田,巷无居人。

岂无居人?不如叔也。洵美且仁。

叔于狩,巷无饮酒。

岂无饮酒?不如叔也。洵美且好。

叔适野,巷无服马。

岂无服马?不如叔也。洵美且武。

注释:颂扬青年猎人的豪迈和英武。一说美共叔段以刺郑庄公。

叔:一说指共叔段。于:去,往。田:打猎。洵(音询):真正的,的确。狩:冬猎。野:郊外。服马:骑马之人。一说用马驾车。将仲子

将仲子兮,无逾我里,无折我树杞。

岂敢爱之?畏我父母。

仲可怀也,父母之言亦可畏也。

将仲子兮,无逾我墙,无折我树桑。

岂敢爱之?畏我诸兄。

仲可怀也,诸兄之言亦可畏也。

将仲子兮,无逾我园,无折我树檀。

岂敢爱之?畏人之多言。

仲可怀也,人之多言亦可畏也。

注释:姑娘要求情人别来她家,以免受父母兄弟及邻居的责骂。

将(音枪):愿,请。一说发语词。仲子:相当于称为二哥。逾:越。里:邻里。二十五家为里。杞(音起):木名,即杞柳。又名榉。落叶乔木,树如柳叶,木质坚实。树桑、树檀:即桑树、檀树。倒文以协韵。 缁衣

缁衣之宜兮,敝予又改为兮。

适子之馆兮,还予授子之粲兮。

缁衣之好兮,敝予又改造兮。

适子之馆兮,还予授子之粲兮。

缁衣之蓆兮,敝予又改作兮。

适子之馆兮,还予授子之粲兮。

注释:作者对所谓“好人”的称道,并愿为他服务。

缁(音资)衣:黑色的衣服,古卿大夫居私朝之服。敝:坏。改为、改造、改作:这是随着衣服的破烂程度而说的,以见其关心。馆:客舍。粲:形容新衣鲜明的样子。一说餐的假借。蓆(音席):宽大舒适。



\part{齐风}
鸡鸣、还、著、东方之日、东方未明、南山、甫田、卢令、敝笱、载驱、猗嗟
猗嗟

猗嗟昌兮,颀而长兮,抑若扬兮。

美目扬兮,巧趋跄兮,射则臧兮。

猗嗟名兮,美目清兮,仪既成兮。

终日射侯,不出正兮,展我甥兮。

猗嗟娈兮,清扬婉兮。舞则选兮,

射则贯兮。四矢反兮,以御乱兮。

注释:刺鲁庄公虽有威仪技艺,不能防闲其母而正家庭。措辞巧妙,意在言外。

猗嗟:叹词。昌:盛。颀:长貌。抑:美貌。扬:额角丰满。巧趋:轻巧地疾走。跄(音枪):趋步摇曳生姿。名:明,昌盛之意。一说目上为名。侯:靶。正(音征):靶中心彩画处。展:诚然,真是。甥:姊妹之子为甥。一说姊妹之夫亦称甥。选:齐也。贯:中而穿革。反:复也,指箭皆射中原处。载驱

载驱薄薄,簟笰朱鞹。鲁道有荡,齐子发夕。

四骊济济,垂辔濔濔。鲁道有荡,齐子岂弟。

汶水汤汤,行人彭彭。鲁道有荡,齐子翱翔。

汶水滔滔,行人儦儦,鲁道有荡,齐子游敖。

注释:讽刺文姜回齐的无耻行为。

薄薄:车疾行声。簟(音电):方纹竹席。一说席作车门。笰(音浮):车帘。一说雉羽作的蔽覆,放在车后。 鞹(音括):光滑的皮革。一说用来作蔽覆。发夕:从傍晚出发到天亮。骊(音离):黑马。济济:美貌。辔(音佩):马缰。濔濔(音你):众多或柔和。岂弟(音凯替):欢乐。汶水:水名。汤汤(音伤):水大貌。彭彭:多貌。滔滔:水流浩荡。儦儦(音标):行貌。一说众多貌。敝笱

敝笱在梁,其鱼鲂鳏。

齐子归止,其从如云。

敝笱在梁,其鱼鲂鱮。

齐子归止,其从如雨。

敝笱在梁,其鱼唯唯。

齐子归止,其从如水。

注释:对文姜返齐荒淫无耻的秽行的讽刺。

敝笱:破旧鱼网,喻文姜。梁:捕鱼水坝。鲂(音房):鳊鱼。鳏(音官):鰔鱼。齐子归止:文姜已嫁。其从如云:一说齐襄仍纠缠不已。鱮(音序):鲢鱼。唯唯:游鱼互相追随。卢令

卢令令,其人美且仁。

卢重环,其人美且鬈。

卢重鋂,其人美且偲。

注释:赞美猎人。

卢:猎犬,大黑犬。令令:铃声。重(音虫)环:子母环。鬈(音全):美好。一说勇壮。重鋂(音梅):一个大环套两个小环。偲(音猜):多才。一说须多而美。 甫田

无田甫田,维莠骄骄。

无思远人,劳心忉忉。

无田甫田,维莠桀桀。

无思远人,劳心怛怛。

婉兮娈兮,总角丱兮。

未几见兮,突而弁兮。

注释:怀念远人,久不归家,徒劳心力。一说少女怀念少年,久不相见,及相见,已由小孩变为成人。

无田(音佃)甫田:不要耕种大田。莠:杂草;狗尾草。骄骄、桀桀:高大貌。忉忉(音刀):忧劳貌。怛怛(音达):悲伤。总角:童子将头发梳成两个髻。丱(音贯):形容总角翘起之状。弁(音辨):冠。男子二十而冠。南山

南山崔崔,雄狐绥绥。

鲁道有荡,齐子由归。

既曰归止,曷又怀止?

葛屦五两,冠緌双止。

鲁道有荡,齐子庸止。

既曰庸止,曷又从止?

蓺麻如之何?衡从其亩。

取妻如之何?必告父母。

既曰告止,曷又鞫止?

析薪如之何?匪斧不克。

取妻如之何?匪媒不得。

既曰得止,曷又极止?

注释:刺齐襄公与文姜兄妹淫乱,鲁桓公纵容文姜而不防闲,致遭杀身之祸。

崔崔:山势高峻。绥绥:求匹之貌。荡:平坦。齐子:指文姜。怀:思。一说来。屦(音具):麻、葛等制成的单底鞋。五两:五,通伍,行列也,两为一列之意。緌(音锐,二声):帽带结在下巴下面下垂的部分。 庸:用,指文姜嫁与鲁桓公。从:相从。蓺(音异):种植。衡从:横纵之异体。告:一说告于祖庙。鞫(音菊):穷,放任无束。极:恣极,放纵无束。东方未明

东方未明,颠倒衣裳。

颠之倒之,自公召之。

东方未晞,颠倒裳衣。

颠之倒之,自公令之。

折柳樊圃,狂夫瞿瞿。

不能辰夜,不夙则莫。

注释:穷苦人民当官差,应徭役,受监视,忙得早晚不宁。

公:公家。晞(音希):破晓,天刚亮。樊:藩篱,篱笆。圃:菜园。狂夫:监工。一说狂妄无知的人。瞿瞿(音去):惊顾貌。不能辰夜:指不能掌握时间。夙(音素):早。莫(音木):古暮字。东方之日

东方之日兮,彼姝者子,在我室兮。

在我室兮,履我即兮。

东方之月兮,彼姝者子,在我闼兮。

在我闼兮,履我发兮。

注释:写新婚夫妇恩爱,形影不离。

履:同蹑,放轻脚步。即:相就,亲近。一说脚迹。闼(音榻):门内。发:走去。 一说脚迹。著

俟我于著乎而。

充耳以素乎而,尚之以琼华乎而。

俟我于庭乎而。

充耳以青乎而,尚之以琼莹乎而。

俟我于堂乎而。

充耳以黄乎而,尚之以琼英乎而。

注释:姑娘见到亲迎时的未婚夫。

俟:迎候。著:通宁。门屏之间。古代婚娶亲迎的地方。乎而:方言。作语助。充耳:饰物,悬在冠之两侧。以玉制成,下垂至耳。素、青、黄:各色丝线。尚之:缀之。琼:赤玉。华、莹、英:均指玉之色泽。一说琼华、琼莹、琼英皆美石之名。还

子之还兮,遭我乎峱之间兮。

并驱从两肩兮,揖我谓我儇兮。

子之茂兮,遭我乎峱之道兮。

并驱从两牡兮,揖我谓我好兮。

子之昌兮,遭我乎峱之阳兮。

并驱从两狼兮,揖我谓我臧兮。

注释:猎人互相间的赞美。

还(音玄):轻捷貌。峱(音挠):山名。从:逐。肩:三岁的兽。揖:作揖,古礼节。儇(音宣):灵利。茂:美。牡:公兽。鸡鸣

鸡既鸣矣,朝既盈矣。

匪鸡则鸣,苍蝇之声。

东方明矣,朝既昌矣。

匪东方则明,月出之光。

虫飞薨薨,甘与子同梦。

会且归矣,无庶予子憎。

注释:妻子催促丈夫早起朝会。意在讽刺在朝者的荒淫怠惰。

朝:朝堂。一说早集。昌:盛也。意味人多。薨薨(音轰):虫聚飞貌。甘:愿。无庶予子憎:无使君臣以我故,憎恶于子。庶,众。



\part{魏风}
葛屦、汾沮洳、园有桃、陟岵、十亩之间、伐檀、硕鼠
硕鼠

硕鼠硕鼠,无食我黍!三岁贯女,莫我肯顾。

逝将去女,适彼乐土。乐土乐土,爰得我所?

硕鼠硕鼠,无食我麦!三岁贯女,莫我肯德。

逝将去女,适彼乐国。乐国乐国,爰得我直?

硕鼠硕鼠,无食我苗!三岁贯女,莫我肯劳。

逝将去女,适彼乐郊。乐郊乐郊,谁之永号?

注释:劳动人民把统治者喻为偷粮老鼠,发誓要到没有剥削的乐土去。

硕鼠:大老鼠。一说田鼠。黍:粘米,谷类。贯:事也。逝:誓。直:同值。劳:慰劳。永号:永叹。 伐檀

坎坎伐檀兮,置之河之干兮,河水清且涟猗。

不稼不穑,胡取禾三百廛兮?

不狩不猎,胡瞻尔庭有县貆兮?

彼君子兮,不素餐兮!

坎坎伐辐兮,置之河之侧兮,河水清且直猗。

不稼不穑,胡取禾三百亿兮?

不狩不猎,胡瞻尔庭有县特兮?

彼君子兮,不素食兮!

坎坎伐轮兮,置之河之漘兮,河水清且沦猗。

不稼不穑,胡取禾三百囷兮?

不狩不猎,胡瞻尔庭有县鹑兮?

彼君子兮,不素飧兮!

注释:劳动者对统治者不劳而获的讽刺。

坎坎:伐木声。干:河岸。涟:波纹。猗(音一):类似啊,语词。稼(音架):播种。穑(音色):收割。廛(音蝉):束。一说古制百亩。县:古悬字。貆(音欢):獾。一说幼小的貉。素餐:不劳而获。直:水流的直波。亿:束。特:三岁的小兽。漘(音纯):河坝。囷(音逡):束。一说圆形的谷仓。飧(音孙):晚餐。 十亩之间

十亩之间兮,

桑者闲闲兮,

行与子还兮。

十亩之外兮,

桑者泄泄兮,

行与子逝兮。

注释:采桑者之歌。一说有归农思想者的咏叹。

闲闲:宽闲貌。行:且。泄泄:同闲闲,十分悠闲的样子。一说多言;一说多人之貌。陟岵

陟彼岵兮,瞻望父兮。

父曰:嗟!予子行役,夙夜无已。

上慎旃哉!犹来!无止!

陟彼屺兮,瞻望母兮。

母曰:嗟!予季行役,夙夜无寐。

上慎旃哉!犹来!无弃!

陟彼冈兮,瞻望兄兮。

兄曰:嗟!予弟行役,夙夜无偕。

上慎旃哉!犹来!无死!

注释:征人远役,登高瞻望,想象父母兄弟对他的思念和希望。

陟(音志):登。岵(音户):有草木的山。予子:歌者想象中,其父对他的称呼。上:尚。旃(音沾):之,作语助。犹:可。屺(音起):无草木的山。季:兄弟中排行第四或最小。园有桃

园有桃,其实之肴。心之忧矣,我歌且谣。

不知我者,谓我士也骄。

彼人是哉,子曰何其?

心之忧矣,有谁知之!有谁知之!盖亦勿思!

园有棘,其实之食。心之忧矣,聊以行国。

不知我者,谓我士也罔极。

彼人是哉,子曰何其?

心之忧矣,有谁知之!有谁知之!盖亦勿思!

注释:士人处于困境,叹息知己难得。

歌、谣:曲合乐曰歌,徒歌曰谣。其:作语助。盖(音何):通盍,何。亦:作语助。棘:指酸枣。行国:到处流浪。罔极:无极,妄想。汾沮洳

彼汾沮洳,言采其莫。彼其之子,美无度。

美无度,殊异乎公路。

彼汾一方,言采其桑。彼其之子,美如英。

美如英,殊异乎公行。

彼汾一曲,言采其藚。彼其之子,美如玉。

美如玉,殊异乎公族。

注释:女子赞爱人的品质才能超过贵族将军。以刺品质低劣、游手好闲的贵族。

汾:汾水。沮洳(音居如):低湿的地方。莫:草名。即酸模,又名羊蹄菜。多年生草本,有酸味。度:衡量。殊异:优异出众。公路:官名。掌诸侯的路车。公行(音杭):官名。掌诸侯的兵车。藚(音序):药用植物,即泽泻草。多年生沼生草本,具地下球茎。公族:官名。掌诸侯的属车。葛屦

纠纠葛屦,可以履霜。

掺掺女手,可以缝裳。

要之襋之,好人服之。

好人提提,宛然左辟。

佩其象揥,维是褊心。

是以为刺。

注释:对傲慢的贵族妇女的讽刺。女仆为主人缝制衣鞋,主人大模大样,不理不睬。

纠纠:缭缭,缠绕。屦(音具):鞋。掺掺(仙):同纤纤。要(音妖):腰,作动词。一说钮襻。襋(音及):衣领,作动词。提提(音时):一说腰细貌,一说安舒貌。宛然:回转貌。左辟:左避。揥(音替):古首饰,可以搔头。类似发篦。褊(音偏)心:心地狭窄。刺:讽刺。



\part{唐风}
蟋蟀、山有枢、扬之水、椒聊、绸缪、杕杜、羔裘、鸨羽、无衣、有杕之杜、葛生、采苓
有杕之杜

有杕之杜,生于道左。

彼君子兮,噬肯适我?

中心好之,曷饮食之?

有杕之杜,生于道周。

彼君子兮,噬肯来游?

中心好之,曷饮食之?

注释:诗人表达对所爱人的感情。

杕(音地):孤零零的样子。噬(音是):发语词。一说何,曷。饮食(音印四):一说满足情爱之欲。中心:心中。采苓

采苓采苓,首阳之巅。人之为言,苟亦无信。

舍旃舍旃,苟亦无然。人之为言,胡得焉?

采苦采苦,首阳之下。人之为言,苟亦无与。

舍旃舍旃,苟亦无然。人之为言,胡得焉?

采葑采葑,首阳之东。人之为言,苟亦无从。

舍旃舍旃,苟亦无然。人之为言,胡得焉?

注释:告戒人们,切勿轻信谣言。

苓:甘草。为言:伪言。苟亦无信:不要轻信。旃:音瞻,之。无然:不要以为然。胡得:何所取。苦:苦菜。无与:勿用也。指不要理会。葑:芜菁。葛生

葛生蒙楚,蔹蔓于野。予美亡此。谁与独处!

葛生蒙棘,蔹蔓于域。予美亡此。谁与独息!

角枕粲兮,锦衾烂兮。予美亡此。谁与独旦!

夏之日,冬之夜。 百岁之后,归於其居!

冬之夜,夏之日。 百岁之后,归於其室!

注释:这是一篇悼念丈夫从军丧亡的诗。含有反战思想。

蒙:覆盖。蔹(音敛):白蔹。攀缘性草本植物,根可入药。予美:我的好人。域:坟地。角枕、锦衾:牛角枕,锦缎褥,敛诗的物品。粲、烂:灿烂。夏之日,冬之夜:夏之日长,冬之夜长,言长也。其居、其室:亡夫的墓穴。无衣

岂曰无衣七兮?

不如子之衣,安且吉兮?

岂曰无衣六兮?

不如子之衣,安且燠兮?

注释:赞美友人衣服,目的在怀念他。

七:一说七章之衣,诸侯的服饰。吉:舒适。六:一说音路,六节衣。燠(音玉):暖。鸨羽

肃肃鸨羽,集于苞栩。

王事靡盬,不能艺稷黍。

父母何怙?悠悠苍天!曷其有所?

肃肃鸨翼,集于苞棘。

王事靡盬,不能艺黍稷。

父母何食?悠悠苍天!曷其有极?

肃肃鸨行,集于苞桑。

王事靡盬,不能艺稻粱。

父母何尝?悠悠苍天!曷其有常?

注释:徭役繁重,劳动人民不能耕种以养父母的控诉。

肃肃:鸟翅扇动的响声。鸨(音保):鸟名,似雁。性不善栖木。苞栩:丛密的柞树。盬(音古):闲暇。艺:种植。怙(音户):依靠,凭恃。棘(音及):酸枣树,落叶灌木。行:飞成行。一说鸨腿;一说翅。羔裘

羔裘豹袪,自我人居居!

岂无他人?维子之故。

羔裘豹袖,自我人究究!

岂无他人?维子之好。

注释:晋人刺在位者不恤其民。一说姑娘谴责过去相好的贵族。

羔:羊之小者。袪(音区):袖口。自我人:我的人。居居:即倨倨,傲慢无礼。究究:恶也。杕杜

有杕之杜,其叶湑湑。独行踽踽。

岂无他人?不如我同父。

嗟行之人,胡不比焉?人无兄弟,胡不佽焉?

有杕之杜,其叶菁菁。独行茕茕。

岂无他人?不如我同姓。

嗟行之人,胡不比焉?人无兄弟,胡不佽焉?

注释:咏叹流浪人的苦难。

杕(音地):独特,孤零零的样子。杜:木名。赤棠。湑(音许):形容草木茂盛。踽踽(音举):孤独无依的样子。同父:同祖父的族昆弟。茕茕(音穷):孤独无依的样子。同姓:同祖昆弟。佽(音次):同情,帮助。绸缪

绸缪束薪,三星在天。今夕何夕,见此良人。

子兮子兮,如此良人何!

绸缪束刍,三星在隅。今夕何夕,见此邂逅。

子兮子兮,如此邂逅何!

绸缪束楚,三星在户。今夕何夕,见此粲者。

子兮子兮,如此粲者何!

注释:反映夫妇新婚的爱情。

绸缪(音仇谋):缠绕,捆束。犹缠绵也。束薪:喻夫妇同心,情意缠绵。三星:即参星,主要由三颗星组成。子兮:你呀。诗人兴奋自呼。刍(音除):喂牲口的青草。邂逅(音谢后):即解靓,原意爱悦,这里指志趣相投的人。粲(音灿):鲜明貌。椒聊

椒聊之实,蕃衍盈升。

彼其之子,硕大无朋。椒聊且,远条且。

椒聊之实,蕃衍盈掬。

彼其之子,硕大且笃。椒聊且,远条且。

注释:赞美妇人,以花椒喻她多子;以椒香美之。

椒:花椒,又名山椒。聊:语助词。一说聚成一串,同今之嘟噜;一说山楂。蕃衍:繁盛。朋:比。且:语助词。远条:指香气远扬。一说长长的枝条。掬(音居):两手合捧。扬之水

扬之水,白石凿凿。素衣朱襮,从子于沃。

既见君子,云何不乐?

扬之水,白石皓皓。素衣朱绣,从子于鹄。

既见君子,云何其忧?

扬之水,白石粼粼。我闻有命,不敢以告人。

注释:叙相见的欢乐。一说晋昭公国事不振,人民不安,离晋去曲沃(桓叔所封地),诗表现人民见桓叔的喜悦。

凿凿:鲜明貌。襮(音博):绣有黼文的衣领。沃:曲沃,地名。皓皓:洁白。绣:刺方领绣。鹄:曲沃的城邑。粼粼(音林):清澈貌。形容水清石净。山有枢

山有枢,隰有榆。子有衣裳,弗曳弗娄。

子有车马,弗驰弗驱。宛其死矣,他人是愉。

山有栲,隰有杻。子有廷内,弗洒弗扫。

子有钟鼓,弗鼓弗考。宛其死矣,他人是保。

山有漆,隰有栗。子有酒食,何不日鼓瑟?

且以喜乐,且以永日。宛其死矣,他人入室。

注释:讽刺贵族老爷们的贪鄙、吝啬和懒惰。

枢(音书):木名。刺榆。榆(音鱼):木名。白榆,又名枌,落叶乔木。弗曳(音义)弗娄:有好衣裳而不穿。曳,拖。娄,搂。古时裳长拖地,需提着走。宛:死貌。愉:一说音偷,取。栲(音考):木名。臭椿树。杻(音纽):木名。檍,梓属。廷:庭。内:堂与室。考:击。保:占有。漆、栗:木名。蟋蟀

蟋蟀在堂,岁聿其莫。今我不乐,日月其除。

无已大康,职思其居。好乐无荒,良士瞿瞿。

蟋蟀在堂,岁聿其逝。今我不乐,日月其迈。

无已大康,职思其外。好乐无荒,良士蹶蹶。

蟋蟀在堂,役车其休。今我不乐,日月其慆。

无已大康,职思其忧。好乐无荒,良士休休。

注释:诗人诫自己勉别人及时行乐,但要有节制,不荒废正事。

聿(音玉):作语助。莫:古暮字。除:过去。无:勿。已:过,太。大(音太)康:过于享乐。职:当。居:指人的处境。好乐:娱乐。无荒:不要过度。瞿瞿(音巨):敛也。一说惊顾貌。逝、迈:义同。去。蹶蹶(音贵):动而敏于事。役车:服役出差的车子。慆(音涛):逝去。休休:安闲自得,乐而有节貌。



\part{秦风}
车邻、驷驖、小戎、蒹葭、终南、黄鸟、晨风、无衣、渭阳、权舆
权舆

於我乎,夏屋渠渠。今也每食无馀。

于嗟乎,不承权舆。

於我乎,每食四簋。今也每食不饱。

于嗟乎,不承权舆。

注释:没落贵族留恋过去的生活,哀叹今不如昔。

於(音呜):叹词。夏屋:大俎;大的食器。渠渠:盛也。权舆:本谓草木萌芽的状态,引申为起始、初时。簋(音鬼):古代食器。青铜或陶制。渭阳

我送舅氏,曰至渭阳。

何以赠之?路车乘黄。

我送舅氏,悠悠我思。

何以赠之?琼瑰玉佩。

注释:表现外甥与舅父惜别的感情。

曰:语词,无实义。渭:水名。阳:山之南或水之北。路车:大车。乘(音剩)黄:黄马。琼:美玉。瑰(音归):次于玉的美石。无衣

岂曰无衣?与子同袍。

王于兴师,修我戈矛,与子同仇!

岂曰无衣?与子同泽。

王于兴师,修我矛戟,与子偕作!

岂曰无衣?与子同裳。

王于兴师,修我甲兵,与子偕行!

注释:反映战士的友谊。从穿衣到赴敌,愿共患难。

袍:长衣服的统称。泽:里衣。戟(己):长一丈六尺,镶有分枝的锋刃。甲:铠甲。兵:兵器。晨风

鴥彼晨风,郁彼北林。未见君子,忧心钦钦。

如何如何?忘我实多!

山有苞栎,隰有六驳。未见君子,忧心靡乐。

如何如何?忘我实多!

山有苞棣,隰有树檖。未见君子,忧心如醉。

如何如何?忘我实多!

注释:女子怀念爱人。

鴥(音玉):疾飞貌。晨风:鸟名。似鹞。郁:茂盛貌。北林:林名。一说北面的森林。钦钦;忧貌。苞:茂盛。栎(音立):落叶乔木。花黄褐色,俗名柞。六:多数之意。驳:树名。即梓榆。一说六驳如马,锯牙食虎豹。靡乐:不乐。棣(音地):木名。即唐棣。檖(音岁):木名。即赤罗,山梨。黄鸟

交交黄鸟,止于棘。谁从穆公?子车奄息。

维此奄息,百夫之特。临其穴,惴惴其栗。

彼苍者天,歼我良人。如可赎兮,人百其身!

交交黄鸟,止于桑。谁从穆公?子车仲行。

维此仲行,百夫之防。临其穴,惴惴其栗。

彼苍者天,歼我良人。如可赎兮,人百其身!

交交黄鸟,止于楚。谁从穆公?子车鍼虎。

维此鍼虎,百夫之御。临其穴,惴惴其栗。

彼苍者天,歼我良人。如可赎兮,人百其身!

注释:秦穆公死后,杀三良以殉葬。秦人痛惜三良,写此诗以示反抗。

交交:飞而往来之貌。一说鸟叫声。从:从死,即殉葬。奄息、仲行、鍼虎:子车之子。特:杰出之称。一说匹敌。惴惴(音缀):恐惧。栗:战栗。歼:消灭,杀死。人百其身:一人替三良死百次都愿意。一说以百人换其一人。防:抵挡。终南

终南何有?有条有梅。

君子至止,锦衣狐裘。

颜如渥丹,其君也哉。

终南何有?有纪有堂。

君子至止,黻衣绣裳。

佩玉将将,寿考不忘。

注释:终南山的姑娘,对进山青年表示热烈爱慕。

条:山楸。一说柚树。梅:一说楠树。其君也哉:仪貌尊严也。纪、堂:假借字,即杞、棠。黻(音服):古代礼服上,黑色与青色花纹。绣:五彩俱备的绘画。将将(音枪):佩玉之声。寿考不忘:万寿无疆。忘,亡也,已也。蒹葭

蒹葭苍苍,白露为霜。所谓伊人,在水一方。

溯洄从之,道阻且长;溯游从之,宛在水中央

蒹葭凄凄,白露未晞。所谓伊人,在水之湄。

溯洄从之,道阻且跻;溯游从之,宛在水中坻

蒹葭采采,白露未已,所谓伊人,在水之涘。

溯洄从之,道阻且右;溯游从之,宛在水中沚

注释:这是一篇美丽的情歌。想望伊人,可望而不可即,饱含无限情意。

蒹(音兼):没长穗的芦苇。葭(音家):初生的芦苇。苍苍:鲜明、茂盛貌。溯洄:逆流而上。溯游:顺流而下。晞(音西):干。跻(音鸡):登;升。坻(音迟):小渚。采采:茂盛貌。涘(音四):水边。右:不直,绕弯。沚(音止):水中的小沙滩。小戎

小戎俴收,五楘梁辀。游环胁驱,阴靷鋈续。

文茵畅毂,驾我骐馵。言念君子,温其如玉。

在其板屋,乱我心曲。

四牡孔阜,六辔在手。骐骝是中,騧骊是骖。

龙盾之合,鋈以觼軜。言念君子,温其在邑。

方何为期?胡然我念之。

俴驷孔群,厹矛鋈镦。蒙伐有苑,虎韔镂膺。

交韔二弓,竹闭绲滕。言念君子,载寝载兴。

厌厌良人,秩秩德音。

注释:妇人思念征夫,从回忆车马、武器到思念他的美德。

小戎:兵车。因车厢较小,故称小戎。俴(音剑)收:浅的车厢。俴,浅;收,轸。四面束舆之木谓之轸。五楘(音木):用皮革缠在车辕成x形。五,古文作x。梁辀(音周):曲辕。游环:活动的环。设于辕马背上。协驱:一皮条,上系于衡,后系于轸,限制骖马内入。靷(音印):引车前行的皮革。鋈(音误)续:以白铜镀的环紧紧扣住皮带。鋈,白铜;续,连续。文茵:虎皮坐垫。畅毂(音古):长毂。毂,车轮中心的圆木,中有圆孔,用以插轴。骐:青黑色如棋盘格子纹的马。馵(音住):左后蹄白或四蹄皆白的马。骝(音留):赤身黑鬣的马。騧(音瓜):黄马黑喙。骊:黑马。龙盾:画龙的盾牌。觼軜(音决纳):有舌的环,以舌穿过皮带,使骖马内辔绳固定。俴驷:披薄金甲的四马。孔群:群马很协和。厹(音求):三偶矛。鋈镦(音队):以白铜镀矛戟柄末的平底金属套。蒙:画杂乱的羽纹。伐:盾。苑(音晕):文貌。虎韔(音唱):虎皮弓囊。镂膺:在弓囊前刻金。闭:弓檠。竹制,弓卸弦后缚在弓里防损伤的用具。绲(音滚)::绳。滕:缠束。载寝载兴:睡起貌。厌厌:安静。秩秩:智也。驷驖

驷驖孔阜,六辔在手。

公之媚子,从公于狩。

奉时辰牡,辰牡孔硕。

公曰左之,舍拔则获。

游于北园,四马既闲。

輶车鸾镳,载猃歇骄。

注释:贵族游猎盛况。

驖(音铁):毛色似铁的好马。阜:肥硕。辔:马缰。媚子:亲信、宠爱的人。奉:猎人驱赶野兽以供射猎。时:是。辰:母鹿。牡:公兽。硕:肥大。左之:从左面射它。舍拔:放箭。拔:箭末。闲:通娴,熟练。輶(音由):轻便的车。鸾:铃。镳:马衔铁。猃(音险):长嘴的猎狗。歇骄:短嘴的猎狗。车邻

有车邻邻,有马白颠。

未见君子,寺人之令。

阪有漆,隰有栗。既见君子,并坐鼓瑟。

今者不乐,逝者其耋。

阪有桑,隰有杨。既见君子,并坐鼓簧。

今者不乐,逝者其亡。

注释:没落贵族士大夫,劝人及时行乐。

邻邻:同辚辚,车行声。白颠:白额,一种良马。寺人:宦者。耋(音迭):八十老人。



\part{陈风}
宛丘、东门之枌、衡门、东门之池、东门之杨、墓门、防有鹊巢、月出、株林、泽陂
泽陂

彼泽之陂,有蒲与荷。有美一人,伤如之何?

寤寐无为,涕泗滂沱。

彼泽之陂,有蒲与蕑。有美一人,硕大且卷。

寤寐无为,中心悁悁。

彼泽之陂,有蒲菡萏。有美一人,硕大且俨。

寤寐无为,辗转伏枕。

注释:女子在荷塘泽畔恋那碰到的青年。

陂(音杯):堤防,堤岸。一说水池的边沿,湖滨。伤:忧思。一说女性第一人称代名词。涕:眼泪。泗:鼻涕。蕑(音肩):兰草。卷(音全):好貌。一说勇壮;一说美鬓。悁悁(音冤):郁郁不乐。菡萏(音汗旦):荷花。俨:双下巴。株林

胡为乎株林?从夏南兮?

匪适株林,从夏南兮!

驾我乘马,说于株野。

乘我乘驹,朝食于株。

注释:陈灵公淫于夏姬,国人作此诗讽刺他。

株林:夏氏的食邑。指夏姬的住地。夏南:夏姬之子,夏征舒,字夏南。这里隐指夏姬。我:陈灵公。说(音睡):停车休息。月出

月出皎兮。佼人僚兮。

舒窈纠兮。劳心悄兮。

月出皓兮。佼人懰兮。

舒忧受兮。劳心慅兮。

月出照兮。佼人燎兮。

舒夭绍兮。劳心惨兮。

注释:月下想念一个漂亮的姑娘。

佼(音嚼):姣之借。美好貌。僚(音辽):美丽。窈纠(音咬脚):谓妇女行步舒缓。劳心:思念。悄:忧。懰(音刘):妩媚。忧受:舒迟之貌。慅(音草):忧愁,心神不安。燎:明也。一说姣美。夭绍:体态柔美。惨(音草):忧愁貌。防有鹊巢

防有鹊巢,邛有旨苕。

谁侜予美?心焉忉忉。

中唐有甓,邛有旨鷊。

谁侜予美?心焉惕惕。

注释:因爱人受人欺诳而感心忧。

防:水坝。一说堤岸;一说枋,常绿乔木,可为红色染料。邛(音穷):山丘。旨:美。苕(音条):草名。凌霄花。一说翘摇。一说苇花。侜(音舟):谎言欺骗。唐:朝堂前和宗庙门内的大路。甓(音屁):古代的砖,用以作瓦沟。鷊(音义):绶草,十样锦。惕惕:忧惧。墓门

墓门有棘,斧以斯之。

夫也不良,国人知之。

知而不已,谁昔然矣。

墓门有梅,有枭萃止。

夫也不良,歌以讯之。

讯予不顾,颠倒思予。

注释:谴责坏人。

墓门:一说陈国城名。斯:砍。谁昔:往昔,由来已久。枭:鸟名。猫头鹰。萃(音翠):草丛生貌。引申为聚集,群栖。讯:谏,劝。颠倒思予:颠倒予思。即好事说成坏事。东门之杨

东门之杨,其叶牂牂,

昏以为期,明星煌煌。

东门之杨,其叶肺肺,

昏以为期,明星皙皙。

注释:男女恋爱,约会于黄昏之后。

牂牂(音脏):风吹树叶的响声。一说茂盛貌。昏:黄昏。明星:启明星。煌煌:明亮。肺肺(音配):同牂牂。皙(音西):同煌煌。东门之池

东门之池,可以沤麻。

彼美淑姬,可以晤歌。

东门之池,可以沤苎。

彼美淑姬,可以晤语。

东门之池,可以沤菅。

彼美淑姬,可以晤言。

注释:恋爱诗,男女约会摆谈、唱歌。

池:护城河。沤:用水浸泡。晤歌:用歌声互相唱和。苎:苎麻。多年生草本植物,茎皮含纤维质,可做绳,织夏布。菅(音间):菅草。茅属,多年生草本植物,叶子细长,可做索。衡门

衡门之下,可以栖迟。

泌之洋洋,可以乐饥。

岂其食鱼,必河之鲂?

岂其取妻,必齐之姜?

岂其食鱼,必河之鲤?

岂其取妻,必宋之子?

注释:没落贵族的自我安慰之作。

衡门:横木为门,简陋的门。一说东西曰横,指陈国东西头的门。可以:一说何以。栖迟:游息。泌(音密):泉水名。洋洋:水流貌。乐饥:言清泉供欣赏,可以忘饥。东门之枌

东门之枌,宛丘之栩。

子仲之子,婆娑其下。

榖旦于差,南方之原。

不绩其麻,市也婆娑。

榖旦于逝,越以鬷迈。

视尔如荍,贻我握椒。

注释:良辰美景,青年男女会舞于市上。

枌(音坟):木名。白榆。栩(音许):柞树。子仲:陈国的姓氏。婆娑:舞蹈。榖(音古):良辰,好日子。差:择。越以:作语助。鬷(音宗):众。荍(音瞧):锦葵。草本植物,夏季开紫色或白色花。椒:花椒。宛丘

子之汤兮,宛丘之上兮。

洵有情兮,而无望兮。

坎其击鼓,宛丘之下。

无冬无夏,值其鹭羽。

坎其击缶,宛丘之道。

无冬无夏,值其鹭翿。

注释:讽刺游荡、荒淫无度者。

汤(音荡):荡之借字。游荡,放荡。宛丘:四周高中间低的土山。望:德望。一说观望;一说望祀;一说仰望。坎:击鼓声。值:持。鹭羽:舞蹈道具。缶(音否):瓦器。翿(音到):舞蹈道具。聚鸟羽于柄头,下垂如盖。



\part{桧风}
羔裘、素冠、隰有苌楚、匪风
匪风

匪风发兮,匪车偈兮。

顾瞻周道,中心怛兮。

匪风飘兮,匪车嘌兮。

顾瞻周道,中心吊兮。

谁能烹鱼?溉之釜鬵。

谁将西归?怀之好音。

注释:游子怀乡的咏叹。

匪:彼之借。发:风声。偈(音杰):疾驰貌。周道:大路。怛(音达):悲伤。飘:旋风。嘌(音飘):轻捷之状。一说疾速貌。溉(音盖):洗。釜:锅。鬵(音心,二声):大釜。怀:归。指带个好信。隰有苌楚

隰有苌楚,猗傩其枝,

夭之沃沃。乐子之无知。

隰有苌楚,猗傩其华,

夭之沃沃。乐子之无家。

隰有苌楚,猗傩其实,

夭之沃沃。乐子之无室。

注释:诗人生处乱世,自叹不如草木无知无累,无家无室。

苌楚(音常):植物名。又名羊桃,猕猴桃。猗傩(音婀娜):同婀娜,轻盈柔美貌。夭:少。沃沃:光泽。子:指苌楚。无知:一说无妻。素冠

庶见素冠兮?棘人栾栾兮,劳心抟抟兮。

庶见素衣兮?我心伤悲兮,聊与子同归。

庶见素韠兮?我心蕴结兮,聊与子如一。

注释:对家遭不幸者的同情。

庶:幸,希冀之辞。素冠:素冠之人,一说清贫之人。棘人:瘠。一说哀戚之人。栾栾:瘦瘠貌。憔悴。抟抟(音团):忧,不安貌。同归:如一。韠(音毕):朝服的蔽膝。羔裘

羔裘逍遥,狐裘以朝。

岂不尔思?劳心忉忉。

羔裘翱翔,狐裘在堂。

岂不尔思?我心忧伤。

羔裘如膏,日出有曜。

岂不尔思?中心是悼。

注释:贵族女子追念他的恋人。

逍遥、翱翔:游逛。朝(音巢):上朝。膏:脂膏。曜(音耀):发光。



\part{曹风}
蜉蝣、候人、鸤鸠、下泉
下泉

冽彼下泉,浸彼苞稂。忾我寤叹,念彼周京。

冽彼下泉,浸彼苞萧。忾我寤叹,念彼京周。

冽彼下泉,浸彼苞蓍。忾我寤叹,念彼京师。

芃芃黍苗,阴雨膏之。四国有王,郇伯劳之。

注释:曹国人对周王室的怀念。

冽(音列):寒冷。下泉:奔流而下的山泉。稂(音郎):童粱。田间害草。饩:叹息。周京:指周室京师。萧:植物名。蒿的一种,即青蒿。蓍:筮草。芃芃(音朋):茂盛。郇(音旬)伯:郇国君。鳲鸠

鳲鸠在桑,其子七兮。淑人君子,其仪一兮。

其仪一兮,心如结兮。

鳲鸠在桑,其子在梅。淑人君子,其带伊丝。

其带伊丝,其弁伊骐。

鳲鸠在桑,其子在棘。淑人君子,其仪不忒。

其仪不忒,正是四国。

鳲鸠在桑,其子在榛。淑人君子,正是国人。

正是国人,胡不万年?

注释:称颂君子的仪表风度好,意在陈古讽今。

鳲(音尸)鸠:布谷鸟。仪:仪容。结:固而不散。其带伊丝:带以素丝缘边。其弁伊骐:皮弁青黑色。正:法则。侯人

彼侯人兮,何戈与殳。彼其之子,三百赤芾。

维鹈在梁,不濡其翼。彼其之子,不称其服。

维鹈在梁,不濡其咮。彼其之子,不遂其媾。

荟兮蔚兮,南山朝隮。婉兮娈兮,季女斯饥。

注释:同情侯人,讽刺不称其服的贵族士大夫。

侯人:管迎宾送客的小武官。何:荷之省,扛。殳(音书):古兵器。棍棒类。三百赤芾(音服):言穿赤芾的人很多。赤芾,冕服之称。大夫以上高官朝服的一部分,熟皮制成,穿起来遮着两膝,又叫蔽膝。鹈(音提):鸟名。鹈鹕。咮(音宙):鸟嘴。不濡其咮,鸟嘴未湿,指不曾吃到鱼,比喻侯人受饥,幼女挨饿。不遂其媾:不配其厚禄。荟蔚(音会为):云雾弥漫貌。隮(音基):升云。一说虹。蜉蝣

蜉蝣之羽,衣裳楚楚。

心之忧矣,於我归处?

蜉蝣之翼,楚楚衣服。

心之忧矣,於我归息?

蜉蝣掘阅,麻衣如雪。

心之忧矣,於我归说?

注释:没落士大夫的哀叹。

蜉蝣之羽:以蜉蝣之羽形容衣服薄而有光泽。楚楚:鲜明貌。於(音乌):何。我:一说通何。掘阅(音穴):穿穴。阅通穴。说:(音睡):止息。



\part{豳风}
七月、鸱鸮、东山、破斧、伐柯、九罭、狼跋
狼跋

狼跋其胡,载疐其尾。

公孙硕肤,赤舄几几。

狼疐其尾,载跋其胡。

公孙硕肤,德音不瑕!

注释:赞美公孙的宽厚。

跋(音拔):践踏。胡:兽颌下垂肉。载:又。疐(音质):踩;牵绊。硕肤:体胖之象。赤舄(音细):红鞋。舄,复底鞋。几几:盛也。一说青铜饰物。瑕:过也。一说远,其前面的“不”为语词。九罭

九罭之鱼,鳟、鲂。

我觏之子,衮衣绣裳。

鸿飞遵渚,公归无所,於女信处。

鸿飞遵陆,公归不复,於女信宿。

是以有衮衣兮,

无以我公归兮,

无使我心悲兮。

注释:东人对周公的留恋。一说贵族的饮宴留客诗。

九罭(音玉):一种捕鱼的细网。鳟(音尊):鱼名。赤眼鳟。鲂:鱼名。鳊鱼。衮衣:大公、君王的礼服。衣上绣卷龙。鸿:鸟名。天鹅。女:同汝。信处、信宿:再处,再宿。有:收藏。伐柯

伐柯如何?匪斧不克。

取妻如何?匪媒不得。

伐柯伐柯,其则不远。

我觏之子,笾豆有践。

注释:言婚姻中媒人的作用。

柯:斧柄。则:法。其则不远:合乎礼法。觏:见。笾(音边):古代祭祀和宴会时盛果品的竹篾食具。豆:古代盛肉或其他食品的木制器皿。践:成行成列之状。破斧

既破我斧,又缺我斨。周公东征,四国是皇。

哀我人斯,亦孔之将。

既破我斧,又缺我锜。周公东征,四国是吪。

哀我人斯,亦孔之嘉。

既破我斧,又缺我銶。周公东征,四国是遒。

哀我人斯,亦孔之休。

注释:东征战士,经破斧缺斨的苦战,庆幸自己能得生还。

斨(音枪):方孔斧。皇:通匡。匡正。亦孔之将:死里逃生,也算大幸。将,大也。锜(音奇):凿属。一说矛属。吪(音俄):化,教化。銶(音求):凿属。一说独头斧。遒(音优):固,安。东山

我徂东山,慆慆不归。我来自东,零雨其蒙。

我东曰归,我心西悲。制彼裳衣,勿士行枚。

蜎蜎者蠋,烝在桑野。敦彼独宿,亦在车下。

我徂东山,慆慆不归。我来自东,零雨其蒙。

果臝之实,亦施于宇。伊威在室,蟏蛸在户。

町疃鹿场,熠耀宵行。不可畏也,伊可怀也。

我徂东山,慆慆不归。我来自东,零雨其蒙。

鹳鸣于垤,妇叹于室。洒扫穹窒,我征聿至。

有敦瓜苦,烝在栗薪。自我不见,于今三年。

我徂东山,慆慆不归。我来自东,零雨其蒙。

仓庚于飞,熠耀其羽。之子于归,皇驳其马。

亲结其缡,九十其仪。其新孔嘉,其旧如之何?

注释:从征兵士还乡,途中想念家乡田园荒芜,妻子悲叹的心情。

徂(音粗,二声):往。慆慆(音滔):久。蒙:微雨貌。士:事。行枚:裹腿。一说士兵行军口中衔枚(似筷),以防喧哗。蜎(音冤):蠕动貌。蠋(音竹):毛虫,桑蚕。烝(音真或征):乃。一说放置。敦:卷成一团。果臝(音裸):栝楼,又名瓜蒌。蔓生葫芦科植物。伊威:一名鼠妇,潮虫。蟏蛸(音消烧):长脚蜘蛛。町疃(音厅湍):田舍旁空地。熠(音义)耀:萤光。宵行:萤火虫。伊:是。鹳:水鸟名。垤(音碟):蚂蚁壅的土堆。聿(音玉):语助词。敦(音堆):圆的。瓜:一说瓠。栗:裂。仓庚:鸟名。黄鹂,黄莺。皇:黄白相间。驳:红白相间。缡(音离):古时女子的佩巾。鸱枭

鸱枭鸱枭,既取我子,无毁我室。

恩斯勤斯,鬻子之闵斯!

迨天之未阴雨,彻彼桑土,绸缪牖户。

今此下民,或敢侮予?

予手拮据,予所捋荼,予所畜租,

予口卒瘏,曰予未有室家!

予羽谯谯,予尾修修,予室翘翘,

风雨所漂摇,予维音哓哓!

注释:诗人以鸟筑巢养雏,历尽艰辛来代言其穷苦经历和对统治者的愤恨。

鸱枭(音吃消):鸟名。一说类似猫头鹰。斯:语词。鬻(音玉):养育。闵(音敏):同悯。一说病。迨:趁着。彻:撤;剥。桑土:桑根。拮据:操作辛劳。畜:积。租:聚。卒:尽。瘏(音涂):病苦。谯谯(音瞧):羽毛残敝。修修:羽毛枯焦。哓哓(音消):因恐惧发出的凄苦的叫声。七月

七月流火,九月授衣。

一之日毕发,二之日栗烈。

无衣无褐,何以卒岁?

三之日于耜,四之日举趾。

同我妇子,饁彼南亩,田畯至喜!

七月流火,九月授衣。

春日载阳,有鸣仓庚。

女执懿筐,遵彼微行,爰求柔桑?

春日迟迟,采蘩祁祁。

女心伤悲,殆及公子同归。

七月流火,八月萑苇。

蚕月条桑,取彼斧斨,

以伐远扬,猗彼女桑。

七月鸣鵙,八月载绩。

我朱孔阳,为公子裳。

四月秀葽,五月鸣蜩。

八月其获,十月陨箨。

一之日于貉,取彼狐狸,为公子裘。

二之日其同,载缵武功。

言私其豵,献豜于公。

五月斯螽动股,六月莎鸡振羽。

七月在野,八月在宇,

九月在户,十月蟋蟀入我床下。

穹窒熏鼠,塞向墐户。

嗟我妇子,曰为改岁,入此室处。

六月食郁及薁,七月烹葵及菽。

八月剥枣,十月获稻。

为此春酒,以介眉寿。

七月食瓜,八月断壶,九月叔苴,

采荼薪樗,食我农夫。

九月筑场圃,十月纳禾稼。

黍稷重穋,禾麻菽麦。

嗟我农夫,我稼既同,上入执宫功。

昼尔于茅,宵尔索绹。

亟其乘屋,其始播百谷。

二之日凿冰冲冲,三之日纳于凌阴。

四之日其蚤,献羔祭韭。

九月肃霜,十月涤场。

朋酒斯飨,曰杀羔羊。

跻彼公堂,称彼兕觥:万寿无疆!

注释:一幅瑰丽的农耕图。奴隶虽终岁勤苦,仍不免饥寒交迫。

流火:大火星在七月黄昏时偏离中天,自西而下。火,星名,心宿之亮星,又名大火。授衣:分发寒衣。一说女工裁寒衣。 一之日:周历正月,夏历十一月。以下二之日,三之日,四之日顺序类推。毕发(音伯):风寒盛。栗烈:凛冽。褐:粗麻或粗毛制短衣。穷人所穿。卒岁:终岁,年底。于耜(音四):整修农具。举趾:举足耕耘。妇子:妻子和小孩。饁(音夜):送饭食到田间。南亩:向阳的耕地。畯(音郡):管农事的管家。一说田神。喜:酒食。阳:温暖。仓庚:黄莺。女:女子,女奴。懿筐:采桑用的深筐。微行(音杭):小路。迟迟:缓慢。指白日渐长。殆:恐,怕。及:与。同归:指被胁迫做妾婢。萑(音环)苇:长成的荻苇。萑、苇,初生时称蒹、葭。条桑:修剪桑枝。斨(音枪):斧,受柄之孔方形。远扬:又长又高的桑枝。猗彼女桑:用索拉着采嫩桑。鵙(音局):鸟名。又名伯劳。体态华丽,嘴大锐利,鸣声洪亮。载绩:纺麻。孔阳:甚为鲜明。秀:不荣而实曰秀。葽:草名,即远志。蜩(音条):蝉。陨箨(音唾):草木之叶陨落。同:会集。缵:继续。武功:武事。一说田猎。豵(音宗):一岁的猪。豜(音间):三岁的猪。斯螽动股:螽斯以股翅相摩而鸣。莎鸡:昆虫名。即纺织娘。穹窒(音穷志):堵好墙洞。向:北窗。墐(音尽):用泥涂抹。曰:助词。同聿。改岁:除岁。郁:木名。郁李。一说樱桃。一说山楂。薁(音玉):木名。野葡萄。葵:滑菜。菽:豆类。剥:扑,打。介:乞。眉寿:人老眉长,表示寿长。断壶:摘葫芦。叔:拾取。苴(音居):麻子。荼(音涂):苦菜。樗(音初):木名。臭椿树。重:晚熟作物。穋(音路):晚种早熟的谷类。上:同尚。宫:宫室。功:事。绹(音陶):绳子。亟:急。凌阴:冰窖。蚤:取。一说通早。肃霜:下霜。一说天肃爽。涤场:十月之中,扫其场上粟麦,尽皆毕矣。朋酒:两樽酒。称:举。兕觥(音四公):犀牛角酒具。

\part{小雅}
采薇  鱼丽  鸿雁  沔水  鹤鸣  无羊  小宛  巧言  谷风  北山  无将大车  鼓钟  车舝  青蝇  菀柳  隰桑  绵蛮  苕之华  何草不黄
\part{大雅}
緜  思齐  灵台  颂  维清  天作  丰年  小毖  蒹葭  无衣  权舆  衡门  墓门  防有鹊巢  月出  泽陂  素冠  隰有长楚  蜉蝣  候人  七月  鸱鸮  破斧  伐柯  鹿鸣  常棣
\part{周颂}

\part{鲁颂}

\part{商颂}

\backmatter

\end{document}