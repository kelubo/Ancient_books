% 山海经
% 山海经.tex

\documentclass[a4paper,12pt,UTF8,twoside]{ctexbook}

% 设置纸张信息。
\RequirePackage[a4paper]{geometry}
\geometry{
	%textwidth=138mm,
	%textheight=215mm,
	%left=27mm,
	%right=27mm,
	%top=25.4mm, 
	%bottom=25.4mm,
	%headheight=2.17cm,
	%headsep=4mm,
	%footskip=12mm,
	%heightrounded,
	inner=1in,
	outer=1.25in
}

% 设置字体,并解决显示难检字问题。
\xeCJKsetup{AutoFallBack=true}
\setCJKmainfont{SimSun}[BoldFont=SimHei, ItalicFont=KaiTi, FallBack=SimSun-ExtB]

% 目录 chapter 级别加点(.)。
\usepackage{titletoc}
\titlecontents{chapter}[0pt]{\vspace{3mm}\bf\addvspace{2pt}\filright}{\contentspush{\thecontentslabel\hspace{0.8em}}}{}{\titlerule*[8pt]{.}\contentspage}

% 设置 part 和 chapter 标题格式。
\ctexset{
	chapter/name={第,卷},
	chapter/number={\chinese{chapter}}
}

% 设置古文原文格式。
\newenvironment{yuanwen}{\bfseries\zihao{4}}

% 设置署名格式。
\newenvironment{shuming}{\hfill\bfseries\zihao{4}}

% 注脚每页重新编号,避免编号过大。
\usepackage[perpage]{footmisc}

\title{\heiti\zihao{0} 山海经}
\author{}
\date{}

\begin{document}

\maketitle
\tableofcontents

\frontmatter

\mainmatter

\chapter{南山经}

主要介绍南方三大山系的地貌矿藏和怪兽珍禽,以及各大山系的山神祭祀情况。三大山系共辖有大小四十座山脉,总长度有一万六千三百八十里。

\begin{yuanwen}
南山之首曰䧿\footnote{qu\`e,同“鹊”。}山。其首曰招摇之山,临于西海之上,多桂,多金玉\footnote{这里指未经过提炼和磨制的天然金属矿物和玉石。以下同此。}。有草焉,其状如韭而青华\footnote{同“花”。},其名曰祝余,食之不饥。有木焉,其状如榖\footnote{即构树,落叶乔木,长得很高大,适应性强。木材可做器具等用,而树皮可作为桑皮纸的原料。}而黑理,其华四照,其名曰迷榖\footnote{g\v{u},迷榖是古代传说中的一种木本植物的名称。},佩之不迷。有兽焉,其状如禺\footnote{传说中的一种野兽,像猕猴而大一些,红眼睛,长尾巴。}而白耳,伏行人走,其名曰狌狌\footnote{sh\=eng。传说是一种长着人脸的野兽,也有说是猩猩的,而且它能知道往事,却不能知道未来。},食之善走。丽𪊨\footnote{j\v{i},古代传说中的地名。}之水出焉,而西流注于海,其中多育沛\footnote{水中生长的一种植物名称。},佩之无瘕疾\footnote{ji\v{a},中医学指腹内结块,即现在人所谓的蛊胀病。}。
\end{yuanwen}

南方首列山系叫做䧿山。山系的头一座是招摇山,屹立在西海岸边,生长着许多桂树,又蕴藏着丰富的金属矿物和玉石。山中有一种草,形状像韭菜却开着青色的花朵,名叫祝余,人吃了它就不感到饥饿。山中又有一种树木,形状像构树却呈现黑色的纹理,并且光华照耀四方,名叫迷穀,佩带它在身上就不会迷失方向。山中还有一种野兽,形状像猿猴但长着一对白色的耳朵,既能匍伏爬行,又能像人一样直立行走,名叫狌狌,吃了它的肉可以使人走得飞快。丽𪊨水从这座山发源,然后往西流入大海,水中有许多叫做育沛的东西,人佩带它在身上就不会生蛊胀病。

\begin{yuanwen}
又东三百里,曰堂庭\footnote{即今洞庭,在湖南境内。今洞庭湖滨人称洞庭音如堂庭。}之山,多棪\footnote{y\v{a}n,即苹果。一种乔木,结出的果实像苹果,表面红了即可吃。}木,多白猿,多水玉\footnote{古时也叫做水精,即现在所说的水晶石。因它莹亮如水,坚硬如玉,所以这样叫。},多黄金\footnote{这里指黄色的沙金,不是经过提炼了的纯金。}。
\end{yuanwen}

再往东三百里,是座堂庭山,山上生长着茂密的棪木,又有许多白色猿猴,还盛产水晶石,并蕴藏着丰富的黄金。

\begin{yuanwen}
又东三百八十里,曰猨翼之山,其中多怪兽,水多怪鱼,多白玉,多蝮虫\footnote{传说中的一种动物,也叫反鼻虫,颜色如同红、白相间的绶带纹理,鼻子上长有针刺,大的一百多斤重。这里的虫是“虺”的本字,不是昆虫之虫。},多怪蛇,多怪木,不可以上。
\end{yuanwen}

再往东三百八十里,是座猨翼山。山上生长着许多怪异的野兽,水中生长着许多怪异的鱼,还盛产白玉,有很多蝮虫,很多奇怪的蛇,很多奇怪的树木,人是不可上去的。

\begin{yuanwen}
又东三百七十里,曰杻\footnote{ni\v{u}}阳之山,其阳多赤金\footnote{就是上文所说的黄金,指未经提炼过的赤黄色沙金。},其阴多白金\footnote{即白银。这里指未经提炼过的银矿石。以下同此。}。有兽焉,其状如马而白首,其文如虎而赤尾,其音如谣\footnote{古代不用乐器伴奏的清唱。},其名曰鹿蜀\footnote{古代传说中的兽名。},佩之宜子孙。怪水出焉,而东流注于宪翼之水。其中多玄龟,其状如龟而鸟首虺\footnote{hu\v{i},毒蛇。}尾,其名曰旋龟,其音如判\footnote{剖开,一分为二。}木,佩之不聋,可以为\footnote{治理。这里是医治、治疗的意思。}底\footnote{这里与“胝”的意思相同,就是手掌或脚底因长期磨擦而生的厚皮,俗称 “老茧”。}。
\end{yuanwen}

再往东三百七十里,是杻阳山。山南面盛产黄金,山北面盛产白银。山中有一种野兽,形状像马却长着白色的头,身上的斑纹像老虎而尾巴却是红色的,吼叫的声音如同人在吟唱,名称是鹿蜀,穿戴上它的毛皮就可以多子多孙。怪水从这座山发源,然后向东流入宪翼水。水中有众多暗红色的龟,形状像普通乌龟却长着鸟一样的头和蛇一样的尾巴,名称是旋龟,叫声像劈开木头时发出的响声,佩带上它就能使人的耳朵不聋,还可以治愈脚底老茧。

\begin{yuanwen}
又东三百里,曰柢山,多水,无草木。有鱼焉,其状如牛,陵居,蛇尾有翼,其羽在魼\footnote{即“胠”的同声假借字,指腋下胁上部分。}下,其音如留牛\footnote{可能就是本书另一处所讲的犁牛。据古人讲,犁牛身上的纹理像老虎的斑纹。},其名曰鯥\footnote{l\`u,古生物学上的一种有四肢有尾有翼的爬虫。},冬死\footnote{指冬眠,也叫冬蛰。一些动物在过冬时处在昏睡不动的状态中,好像死了一般。}而夏生,食之无肿\footnote{一种皮肤和皮下组织的化脓性炎症。}疾。
\end{yuanwen}

再往东三百里,是座柢山,山间多水流,没有花草树木。有一种鱼,形状像牛,栖息在山坡上,长着蛇一样的尾巴并且有翅膀,而翅膀长在胁下,鸣叫的声音像犁牛,名称是鯥,冬天蛰伏而夏天复苏,吃了它的肉就能使人不患痈肿病。

\begin{yuanwen}
又东四百里,曰亶爰之山,多水,无草木,不可以上。有兽焉,其状如狸而有髦\footnote{下垂至眉的长发。},其名曰类,自为牝\footnote{鸟兽的雌性。这里指雌性器官。}牡\footnote{鸟兽的雄性。这里指雄性器官。},食者不妒。
\end{yuanwen}

再往东四百里,是座亶爰山,山间多水流,没有花草树木,不能攀登上去。山中有一种野兽,形状像野猫却长着长头发,名称是类,一身具有雄雌两种性器官,吃了它的肉就会使人不产生妒忌心。

\begin{yuanwen}
又东三百里,曰基山,其阳多玉,其阴多怪木。有兽焉,其状如羊,九尾四耳,其目在背,其名曰猼訑\footnote{b\'o sh\`i,一种怪兽名。},佩之不畏。有鸟焉,其状如鸡而三首六目、六足三翼,其名曰𪁺𩿧\footnote{ch\v{a}ng f\'u,一种鸟名。},食之无卧。
\end{yuanwen}

再往东三百里,是座基山,山南阳面盛产玉石,山北阴面有很多奇怪的树木。山中有一种野兽,形状像羊,长着九条尾巴和四只耳朵,眼睛也长在背上,名称是猼訑,人穿戴上它的毛皮就会不产生恐惧心。山中还有一种禽鸟,形状像鸡却长着三个脑袋、六只眼睛、六只脚、三只翅膀,名称是𪁺𩿧,吃了它的肉就会使人不感到瞌睡。

\begin{yuanwen}
又东三百里,曰青丘之山,其阳多玉,其阴多青雘\footnote{hu\`o,青色的善丹。一种颜色很好看的天然涂料。}。有兽焉,其状如狐而九尾,其音如婴儿,能食人;食者不蛊。有鸟焉,其状如鸠\footnote{即斑鸠,一种体型似鸽子的鸟。},其音若呵\footnote{大声喝叱。},名曰灌灌\footnote{传说中的一种鸟,它的肉很好吃,烤熟后更是味道鲜美。},佩\footnote{这里是插上的意思。}之不惑。英水出焉,南流注于即翼之泽。其中多赤鱬\footnote{r\'u,也叫鲵鱼,即现在所说的娃娃鱼,有四只脚,长尾巴,能上树,属两栖类动物。另有一说,类似今方头鱼,头高,呈长方形,分布在我国沿海。},其状如鱼而人面,其音如鸳鸯\footnote{一种雌雄同居同飞而不分离的鸟,羽毛色彩绚丽。},食之不疥。
\end{yuanwen}

再往东三百里,是座青丘山,山南阳面盛产玉石,山北阴面多出产青雘。山中有一种野兽,形状像狐狸却长着九条尾巴,吼叫的声音与婴儿啼哭相似,能吞食人。吃了它的肉就能使人不中妖邪毒气。山中还有一种禽鸟,形状像斑鸠,鸣叫的声音如同人在互相斥骂,名称是灌灌,把它的羽毛插在身上使人不迷惑。英水从这座山发源,然后向南流入即翼泽。泽中有很多赤鱬,形状像普通的鱼却有一副人的面孔,发出的声音如同鸳鸯鸟在叫,吃了它的肉就能使人不生疥疮。

\begin{yuanwen}
又东三百五十里,曰箕尾之山\footnote{即今黄山与天目山之合称。},其尾踆\footnote{通“蹲”,屈两膝如坐,臀部不着地。这里是坐的意思。}于东海,多沙石。汸水\footnote{f\=ang,即今昌江。}出焉,而南流注于淯\footnote{y\v{u},与蠡古音相近,即今鄱阳湖。},其中多白玉。
\end{yuanwen}

再往东三百五十里,是座箕尾山,山的尾端座落于东海岸边,沙石很多。汸水从这座山发源,然后向南流入淯水,水中多产白色玉石。

\begin{yuanwen}
凡䧿山之首,自招摇之山,以至箕尾之山,凡十山,二千九百五十里。其神状皆鸟身而龙首。其祠\footnote{祭祀。}之礼:毛\footnote{指祭祀所用的毛物,即猪、羊、狗、鸡等家养畜禽。}用一璋\footnote{古时一种顶端作斜锐角形的玉器,是在举行朝聘、祭祀、丧葬时使用的礼器之一。}玉瘗\footnote{y\`i,埋。},糈\footnote{祭神用的精米。}用稌\footnote{稻米。也有说是专指糯稻。}米、稻米,白菅\footnote{茅草的一种,叶片线形,细长,根坚韧,可做刷帚。}为席。
\end{yuanwen}

䧿山山系之首,从招摇山起,直到箕尾山止,一共是十座山,途经二千九百五十里。诸山山神的形状都是鸟的身子龙的头。祭祀山神的典礼:是把畜禽和璋一起埋入地下,祀神的米用稻米,用白茅草来做神的座席。

\begin{yuanwen}
南次二山之首,曰柜山,西临流黄,北望诸𣬉\footnote{p\'i,山名,亦水名。},东望长右。英水出焉,西南流注于赤水,其中多白玉,多丹粟\footnote{像粟粒一样的红色细沙。}。有兽焉,其状如豚,有距\footnote{雄鸡、野鸡等跖后面突出像脚趾的部分。这里指鸡的足爪。},其音如狗吠,其名曰狸力,见\footnote{同“现”。}则其县\footnote{这里泛指有人聚居的地方。}多土功。有鸟焉,其状如鸱\footnote{即鹞鹰,一种凶猛的飞禽,常捕食其它小型鸟禽。}而人手,其音如痺\footnote{类似鹌鹑。},其名曰鴸\footnote{传说是帝尧的儿子丹朱所化的鸟。帝尧把天下让给帝舜,而丹朱和三苗国人联合起兵反对,帝尧便派兵打败了他们,丹朱感到羞愧,就自投南海淹死而化作鴸鸟。},其名自号也,见则其县多放士\footnote{被放逐的人才。}。
\end{yuanwen}

南方第二列山系的首座山是柜山,西边临近流黄酆氏国,向北可以望见诸𣬉山,向东可以望见长右山。英水从这座山发源,向西南流入赤水,水中有很多白色玉石,还有很多粟粒般大小的丹沙。山中有一种野兽,形状像普通的小猪,长着一双鸡爪,叫的声音如同狗叫,名叫狸力,哪个地方出现狸力而那里就一定会有繁多的水土工程。山中还有一种鸟,形状像鹞鹰却长着人手一样的爪子,啼叫的声音如同鹌鹑,名叫鴸,它的鸣叫声就是自身名称的读音,哪个地方出现,而那里就一定会有众多的文士被流放。

\begin{yuanwen}
东南四百五十里,曰长右之山,无草木,多水。有兽焉,其状如禺\footnote{长尾猿。}而四耳,其名长右,其音如吟,见则其郡县大水。
\end{yuanwen}

往东南四百五十里,是座长右山,没有花草树木,但有很多水。山中有一种野兽,形状像长尾猿却长着四只耳朵,名称是长右,叫的声音如同人在呻吟,任何郡县一出现长右就会发生大水灾。

\begin{yuanwen}
又东三百四十里,曰尧光之山,其阳多玉,其阴多金\footnote{这里泛指金属矿物质。以下同此。}。有兽焉,其状如人而彘\footnote{zh\`i,猪。}鬣\footnote{li\`e,牲畜身上刚硬的毛。},穴居而冬蛰,其名曰猾褢\footnote{hu\'ai,一种形状像人的怪兽。},其音如斫木\footnote{zhu\'o,砍伐树木。},见则县有大繇\footnote{y\'ao,通“徭”。}。
\end{yuanwen}

再往东三百四十里,是座尧光山,山南阳面多产玉石,山北阴面多产金属。山中有一种野兽,形状像人却长有猪那样的鬣毛,冬季蛰伏在洞穴中,名称是猾褢,叫声如同砍木头时发出的响声,哪个地方出现猾褢,那里就会有繁重的徭役。

\begin{yuanwen}
又东三百五十里,曰羽山\footnote{传说中的上古帝王祝融曾奉黄帝之命,将大禹的父亲鲧杀死在羽山,一说是鲧被帝舜杀死在羽山的,所以,这座山很有名。},其下多水,其上多雨,无草木,多蝮虫。
\end{yuanwen}

再往东三百五十里,是座羽山,山下到处流水,山上经常下雨,没有花草树木,蝮虫很多。

\begin{yuanwen}
又东三百七十里,曰瞿父之山,无草木,多金玉。
\end{yuanwen}

再往东三百七十里,是座瞿父山。山上没有花草树木,但有丰富的金属矿物和玉石。

\begin{yuanwen}
又东四百里,曰句余之山,无草木,多金玉。
\end{yuanwen}

再往东四百里,是座句余山。山上没有花草树木,但有丰富的金属矿物和玉石。

\begin{yuanwen}
又东五百里,曰浮玉之山,北望具区,东望诸𣬉。有兽焉,其状如虎而牛尾,其音如吠犬,其名曰彘,是食人。苕水出于其阴,北流注于具区\footnote{就是现在江苏境内的太湖。}。其中多鮆鱼\footnote{又叫鲚鱼,鮤鱼,头生长得很长而狭薄,大的有一尺多长。}。
\end{yuanwen}

再往东五百里,是座浮玉山,向北可以望见具区泽,向东可以望见诸𣬉水。山中有一种野兽,形状像老虎却长着牛的尾巴,发出的叫声如同狗叫,名称是彘,能吃人。苕水从这座山的北麓发源,向北流入具区泽。里面生长着很多鮆鱼。

\begin{yuanwen}
又东五百里,曰成山,四方而三坛,其上多金玉,其下多青雘。𨴯水\footnote{zhu\=o,水名。}出焉,而南流注于虖勺\footnote{h\=u shu\`o,水名,古人认为即南虖沱水。},其中多黄金。
\end{yuanwen}

再往东五百里,是座成山,呈现四方形而像三层土坛,山上盛产金属矿物和玉石,山下多产青雘。𨴯水从这座山发源,然后向南流入虖勺水,水中有丰富的黄金。

\begin{yuanwen}
又东五百里,曰会稽之山,四方,其上多金玉,其下多砆石\footnote{f\=u,即武夫石,一种似玉的美石。}。勺水出焉,而南流注于湨\footnote{ju\'e,水名,疑似今之瓯江。}。
\end{yuanwen}

再往东五百里,是座会稽山,呈现四方形,山上有丰富的金属矿物和玉石,山下盛产晶莹透亮的砆石。勺水从这座山发源,然后向南流入湨水。

\begin{yuanwen}
又东五百里,曰夷山。无草木,多沙石。湨水出焉,而南流注于列涂。
\end{yuanwen}

再往东五百里,是座夷山,山上没有花草树木,到处是细沙石子。湨水从这座山发源,然后向南流入列涂水。

\begin{yuanwen}
又东五百里,曰仆勾之山,其上多金玉,其下多草木,无鸟鲁,无水。
\end{yuanwen}

再往东五百里,是座仆勾山,山上有丰富的金属矿物和玉石,山下有茂密的花草树木,但没有禽鸟野兽,也没有水。

\begin{yuanwen}
又东五百里,曰咸阴之山,无草木,无水。	
\end{yuanwen}

再往东五百里,是座咸阴山,没有花草树木,也没有水。

\begin{yuanwen}
又东四百里,曰洵山,其阳多金,其阴多玉。有兽焉,其状如羊而无口,不可杀\footnote{这里杀是死的意思。不可杀就是不能死,意思是这种兽即使不吃东西也 不能使它死去。}也,其名曰䍺\footnote{hu\`an,形状像羊的怪兽。}。洵水出焉,而南流注于阏\footnote{\`e,湖泊名。}之泽,其中多茈蠃\footnote{茈通“紫”。蠃通“螺”。茈蠃就是紫颜色的螺。}。
\end{yuanwen}

再往东四百里,是座洵山,山南阳面盛产金属矿物,山北阴面多出产玉石。山中有一种野兽,形状像普通的羊却没有嘴巴,不吃东西也能活着而不死,名称是䍺。洵水从这座山发源,然后向南流入阏泽,水中有很多紫色螺。

\begin{yuanwen}
又东四百里,曰虖勺之山,其上多梓\footnote{梓树,落叶乔木,生长较快。木材轻软,耐朽,供建筑及制家具、乐器等用。}枏\footnote{n\'an,即楠木树,常绿乔木,叶质厚,花小,核果小球形。木材富于香气,是建筑和制造器具的上等材料。},其下多荆\footnote{即牡荆,落叶灌木,小枝方形,叶对生,掌状复叶。果实称为黄荆子,可供药用。}杞\footnote{即枸杞,落叶小灌木,夏季开淡紫色花。果实是红色的,叫枸杞子,药用价值很大。}。滂水出焉,而东流注于海。
\end{yuanwen}

再往东四百里,是座虖勺山,山上到处是梓树和楠木树,山下生长许多牡荆树和枸杞树。滂水从这座山发源,然后向东流入大海。

\begin{yuanwen}
又东五百里,曰区吴之山,无草木,多沙石。鹿水出焉,而南流注于滂水。
\end{yuanwen}

再往东五百里,是座区吴山,山上没有花草树木,到处是沙子石头。鹿水从这座山发源,然后向南流入滂水。

\begin{yuanwen}
又东五百里,曰鹿吴之山,上无草木,多金石。泽更之水出焉,而南流注于滂水。水有兽焉,名曰蛊雕,其状如雕而有角,其音如婴儿之音,是食人。
\end{yuanwen}

再往东五百里,是座鹿吴山,山上没有花草树木,但有丰富的金属矿物和玉石。泽更水从这座山发源,然后向南流入滂水。水中有一种野兽,名称是蛊雕,形状像普通的雕鹰却头上长角,发出的声音如同婴儿啼哭,是能吃人的。

\begin{yuanwen}
东五百里,曰漆吴之山,无草木,多博石,无玉。处于海,东望丘山,其光载出载入,是惟日次\footnote{停留,休息。}。
\end{yuanwen}

再往东五百里,是座漆吴山,山中没有花草树木,多出产可以用作棋子的博石,不产玉石。这座山位于海之滨,在山上向东远望是片丘陵,有光影忽明忽暗,那是太阳停歇之处。

\begin{yuanwen}
凡南次二山之首,自柜山至于漆吴之山,凡十七山,七千二百里。其神状皆龙身而鸟首。其祠:毛用一璧\footnote{古时一种玉器,平圆形,正中有孔,是古代朝聘、祭祀、丧葬时使用的礼器 之一。}瘗,糈用稌。
\end{yuanwen}

南方第二列山系,从柜山起到漆吴山止,一共十七座山,途经七千二百里。诸山山神的形状都是龙的身子鸟的头。祭祀山神:是把畜禽和玉璧一起埋入地下,祀神的米用稻米。

\begin{yuanwen}
南次三山之首,曰天虞之山,其下多水,不可以上。
\end{yuanwen}

南方第三列山系的头一座山,是座天虞山,山下到处是水,人 不能上去。

\begin{yuanwen}
东五百里,曰祷过之山,其上多金玉,其下多犀\footnote{据古人说,犀的身子像水牛,头像猪头,蹄子好似象的蹄子,黑色皮毛,生 有三只角,一只长在头顶上,一只长在前额上,一只长在鼻子上。它生性爱吃荆棘,往往刺破嘴而口 吐血沫。}、兕\footnote{据古人说,兕的身子也像水牛,青色皮毛,生有一只角,身体很重,大的有三千斤。},多象。有鸟焉,其状如鵁\footnote{传说中的一种鸟,样子像野鸭子而小一些,脚长在接近尾巴的部位。}而白首,三足,人面,其名曰瞿如,其鸣自号也。泿水出焉,而南流注于海。其中有虎蛟\footnote{传说中龙的一个种类。},其状鱼身而蛇尾,其音如鸳鸯,食者不肿,可以已\footnote{停止。这里是止住、制止的意思。}痔。	
\end{yuanwen}

从天虞山往东五百里,是座祷过山,山上盛产金属矿物和玉石, 山下到处是犀、兕,还有很多大象。山中有一种禽鸟,形状像却是白色的 脑袋,长着三只脚,人一样的脸,名称是瞿如,它的鸣叫声就是自身名称的 读音。泿水从这座山发源,然后向南流入大海。水中有一种虎蛟,形状像普 通鱼的身子却拖着一条蛇的尾巴,脑袋如同鸳鸯鸟的头,吃了它的肉就能使 人不生痈肿疾病,还可以治愈痔疮。

\begin{yuanwen}
	
\end{yuanwen}
又东五百里,曰丹穴之山,其上多金玉。丹水出焉,而南流注于渤海。

有鸟焉,其状如鸡,五采而文,名曰凤皇①,首文曰德,翼文曰义,背文曰 礼,膺(y9ng)文曰仁②,腹文曰信。是鸟也,饮食自然,自歌自舞,见(xi4n) 则天下安宁。

【注释】①风皇:同“凤凰”,是古代传说中的鸟王。雄的叫“凤”,雌的叫“凰”。据古 人说,它的形状是鸡的头,蛇的脖颈,燕子的下颔,乌龟的背,鱼的尾巴,五彩颜色,高六尺左右。

②膺:胸。

【译文】再往东五百里,是座丹穴山,山上盛产金属矿物和玉石。丹水 从这座山发源,然后向南流入渤海。山中有一种鸟,形状像普通的鸡,全身 上下是五彩羽毛,名称是凤凰,头上的花纹是“德”字的形状,翅膀上的花 纹是“羲”字的形状,背部的花纹是“礼”字的形状,胸部的花纹是“仁” 字的形状,腹部的花纹是“信”字的形状。这种叫做凤凰的鸟,吃喝很自然 从容,常常是自个儿边唱边舞,一出现天下就会太平。
\begin{yuanwen}
	
\end{yuanwen}
又东五百里,曰发爽之山,无草木,多水,多白猿。汎(f4n)水出焉, 而南流注于渤海。

【译文】再往东五百里,是座发爽山,没有花草树木,到处流水,有很 多白色的猿猴。汎水从这座山发源,然后向南流入渤海。
\begin{yuanwen}
	
\end{yuanwen}
又东四百里,至于旄(m2o)山之尾。其南有谷,曰育遗,多怪鸟,凯 (k3i)风自是出①。

【注释】①凯风:南风,意思是柔和的风。

【译文】再往东四百里,便到了旄山的尾端。此处的南面有一峡谷,叫 做育遗,生长着许多奇怪的鸟,南风就是从这里吹出来的。
\begin{yuanwen}
	
\end{yuanwen}
又东四百里,至于非山之首。其上多金玉,无水,其下多蝮(f))虫(hu!)。

【译文】再往东四百里,便到了非山的顶部。山上盛产金属矿物和玉石, 没有水,山下到处是蝮虫。
\begin{yuanwen}
	
\end{yuanwen}
又东五百里,曰阳夹之山,无草木,多水。

【译文】再往东五百里,是座阳夹山,没有花草树木,到处流水。
\begin{yuanwen}
	
\end{yuanwen}
又东五百里,曰灌湘之山,上多木,无草;多怪鸟,无兽。

【译文】再往东五百里,是座灌湘山,山上到处是树木,但没有花草; 山中有许多奇怪的禽鸟,却没有野兽。
\begin{yuanwen}
	
\end{yuanwen}
又东五百里,曰鸡山,其上多金,其下多丹雘(hu^)。黑水山焉,而 南流注于海。其中有■(tu2n)鱼,其状如鲋(f))而彘(zh@)毛①,其 音如豚(t*n)②,见(xi4n)则天下大旱。

【注释】①鲋:即今鲫鱼,体侧扁,稍高,背面青褐色,腹面银灰色。彘:猪。②豚:小猪。

也泛指猪。

【译文】再往东五百里,是座鸡山,山上有丰富的金属矿物,山下盛产 丹雘。黑水从这座山发源,然后向南流入大海。水中有一种■鱼,形状像鲫 鱼却长着猪毛,发出声音如同小猪叫,它一出现就会天下大旱。
\begin{yuanwen}
	
\end{yuanwen}
又东四百里,曰令丘之山,无草木,多火。其南有谷焉,曰中谷,条风 自是出①。有鸟焉,其状如枭②(xi1o),人面四目而有耳,其名曰颙(y*), 其鸣自号也,见(xi4n)则天下大旱。

【注释】①条风:也叫调风、融风,即春天的东北风。②枭:通“鸮”,俗称猫头鹰,嘴和 爪弯曲呈钩状,锐利,两眼长在头部的正前方,眼的四周羽毛呈放射状,周身羽毛大多为褐色,散缀 细斑,稠密而松软,飞行时无声,在夜间活动。

【译文】再往东四百里,是座令丘山,没有花草树木,到处是野火。山 的南边有一峡谷,叫做中谷,东北风就是从这里吹出来的。山中有一种禽鸟, 形状像猫头鹰,却长着一副人脸和四只眼睛而且有耳朵,名称是颙,它发出 的叫声就是自身名称的读音,一出现而天下就会大旱。
\begin{yuanwen}
	
\end{yuanwen}
又东三百七十里,曰仑者之山,其上多金玉,其下多(汗)青雘(hu^)。

有木焉,其状如(穀)[榖](g^u)而赤理,其(汗)[汁]如漆,其味如饴 (y0)①,食者不饥,可以释劳②,其名曰白■(g1o),可以血玉③。

【注释】①饴:用麦芽制成的糖浆。②劳:忧。③血:这里用作动词,染的意思,就是染器 物饰品使之发出光彩。

【译文】再往东三百七十里,是座仑者山,山上有丰富的金属矿物和玉 石,山下盛产青雘。山中有一种树木,形状像一般的构树却是红色的纹理, 枝干流出的汁液似漆,味道是甜的,人吃了它就不感到饥饿,还可以解除忧 愁,名称是白■,可以用它把玉石染得鲜红。
\begin{yuanwen}
	
\end{yuanwen}
又东五百八十里,曰禺稿之山,多怪兽,多大蛇。

【译文】再往东五百八十里,是座禺稿山,山中有很多奇怪的野兽,还 有很多大蛇。
\begin{yuanwen}
	
\end{yuanwen}
又东五百八十里,曰南禺之山,其上多金玉,其下多水。有穴焉,水(出) [春]辄(zh6)入①,夏乃出,冬则闭。佐水出焉,而东南流注于海,有凤 皇、鹓(yu4n)雏(ch*)②。

【注释】①辄:即,就。②鹓雏:传说中的一种鸟,和凤凰、鸾凤是同一类。

【译文】再往东五百八十里,是座南禺山,山上盛产金属矿物和玉石, 山下到处流水。山中有一个洞穴,水在春天就流入洞穴,在夏天便流出洞穴, 在冬天则闭塞不通。佐水从这座山发源,然后向东南流入大海,佐水流经的 地方有凤凰和鹓雏栖息。
\begin{yuanwen}
	
\end{yuanwen}
凡南次三(经)[山]之首,自天虞之山以至南禺之山,凡一十四山,六 千五百三十里。其神皆龙身而人面。其祠皆一白狗祈(q0)①,糈(x()用 稌(t*)。

【注释】①祈:向神求祷。

【译文】总计南方第三列山系之首尾,从天虞山起到南禺山止,一共十 四座山,途经六千五百三十里。诸山山神都是龙的身子人的脸面。祭祀山神 全部是用一条白色的狗作供品祈祷,祀神的米用稻米。
\begin{yuanwen}
	
\end{yuanwen}
右南经之山志,大小凡四十山,万六千三百八十里①。

【注释】①据学者们的研究,这几句不是《山海经》原文,而是校勘整理本书之人所题写的。

因为底本所原有,故仍予保留并作今译。以下均同此。

【译文】以上是南方经历之山的记录,大大小小总共四十座,一万六千 三百八十里。

\chapter{西山经}

山海经卷二 西山经

西山(经)华山之首,曰钱来之山,其上多松,其下多洗石①。有兽焉, 其状如羊而马尾,名曰羬(xi2n)羊,其脂可以已腊(x9)②。

【注释】①洗石:古人说是一种在洗澡时用来擦去身上污垢的瓦石。②腊:皮肤皴皱。

【译文】西方第一列山系华山山系之首座山,叫做钱来山,山上有许多 松树,山下有很多洗石。山中有一种野兽,形状像普通的羊却长着马的尾巴, 名称是羬羊,羬羊的油脂可以护理治疗干裂的皮肤。

西四十五里,曰松果之山。濩(hu^)水出焉,北流注于渭,其中多铜 ①。有鸟焉,其名曰■(t¥ng)渠,其状如山鸡,黑身赤足,可以已■(b4o)。

② 【注释】①铜:这里指可以提炼为精铜的天然铜矿石。以下同此。②■:皮肤皱起。

【译文】从钱来山往西四十五里,是座松果山。濩水从这座山发源,向 北流入渭水,其中多产铜。山中有一种禽鸟,名称是■渠,形状像一般的野 鸡,黑色的身子和红色的爪子,可以用来治疗皮肤干皱。

又西六十里,曰太华之山①,削成而四方,其高五千仞(r6n),②其 广十里,鸟兽莫居。有蛇焉,名曰肥(■)[遗] ,六足四翼,见(xi4n) 则天下大旱。

【注释】①太华之山:就是现在陕西省境内的西岳华山。 ②仞:古时八尺为一仞。

【译文】再往西六十里,是座太华山,山崖陡峭像刀削而呈现四方形, 高五千仞,宽十里,禽鸟野兽无法栖身。山中有一种蛇,名称是肥遗,长着 六只脚和四只翅膀,一出现就会天下大旱。

又西八十里,曰小华之山,其木多荆杞(q!),其兽多■(zu¥)牛①, 其阴多磬(q0ng)石②,其阳多■(y()琈(f*)之玉③。鸟多赤鷩(bi5) ④,可以御火⑤。其草有萆(b@)荔⑥,状如乌韭,而生于石上,亦缘木而 生,食之已心痛。

【注释】①■牛:据古人讲,在小华山生长着许多山牛,体重都在一千斤左右,这就是■牛。

②磐石:是一种可以制造乐器的石头。古人用它制成的打击乐器叫做磬,一般是挂在架子上进行演奏。

③■琈:古时传说中的一种玉,具体的形状质料不清楚。④赤鷩:属于野鸡一类的禽鸟,胸部腹部都 是红色,冠子是金黄色,头是黄的,尾巴是绿的,间杂着红色羽毛,色彩鲜明。⑤御火:御在这里是 屏除、辟开的意思。御火就是辟火,意思是火不能烧及人的身子。⑥萆荔:古时传说中的一种香草。

【译文】再往西八十里,是座小华山,山上的树木大多是牡荆树和枸杞 树,山中的野兽大多是■牛,山北阴面盛产磬石,山南阳面盛产■琈玉。山 中有许多赤鷩鸟,饲养它就可以辟火。山中还有一种叫做萆荔的草,形状像 乌韭,但生长在石头上面,也攀缘树木而生长,人吃了它就能治愈心痛病。

又西八十里,曰符禺之山,其阳多铜,其阴多铁①。其上有木焉,名曰 文茎,其实如枣,可以已聋。其草多条,其状如葵②,而赤华黄实,如婴儿 舌,食之使人不惑。符禺之水出焉,而北流注于渭。其兽多葱聋③,其状如 羊而赤鬣(li6)。其鸟多(穀)[榖](m@n),其状如翠而赤喙(hu@)④, 可以御火。

【注释】①铁:这里指能够提炼成铁的天然铁矿石。以下同此。②葵:即冬葵,也叫冬寒菜, 是古代重要蔬菜之一。③葱聋:古人说是野山羊的一种。④翠:指翠鸟,又叫翡翠鸟,大小近似于燕 子,头大而身体小,嘴强硬而直,额部、枕部、背部的羽毛以苍翠,暗绿色为主,耳部的羽毛是棕黄 色,颊部、喉部的羽毛是白色,翅膀上的羽毛主要是黑褐色,胸下的羽毛是栗棕色。喙:鸟鲁的嘴。

【译文】再往西八十里,是座符禺山,山南阳面盛产铜,山北阴面盛产 铁。山上有一种树木,名称是文茎,结的果实像枣子,可以用来治疗耳聋。

山中生长的草大多是条草,形状与葵菜相似,但开的是红色花朵而结的是黄 色果实,果实的样子像婴儿的舌头,吃了它就可使人不迷惑。符禺水从这座 山发源,然后向北流入渭水。山中的野兽大多是葱聋,形状像普通的羊却长 有红色的鬣毛。山中的禽鸟大多是■鸟,形状像一般的翠鸟却是红色的嘴 巴,饲养它可以辟火。

又西六十里,曰石脆之山,其木多棕枏(n2n),其草多条①,其状如 韭,而白华黑实,食之已疥。其阳多■(y()琈(f*)之玉,其阴多铜。灌 水出焉,而北流注于禺水。其中有流、赭(zh7)②,以涂牛马无病。

【注释】①条:这里讲的条草和上文所说的条草,名称虽相同,但形状不同,实际上是两种 草。②流、赭:流即硫黄,是一种天然的矿物质,中医入药,有杀虫作用;赭即赭黄,是一种天然生 成的褐铁矿,可做黄色颜料。

【译文】再往西六十里,是座石脆山,山上的树大多是棕树和楠木树, 而草大多是条草,形状与韭菜相似,但是开的是白色花朵而结的是黑色果 实,人吃了这种果实就可以治愈疥疮。山南面盛产■琈玉,而山北面盛产铜。

灌水从这座山发源,然后向北流入禺水。这条水里有硫黄和赭黄,将这种水 涂洒在牛马的身上就能使牛马健壮不生病。

又西七十里,曰英山,其上多杻(ni()橿①(ji1ng),其阴多铁,其 阳多赤金。禺水出焉,北流注于招(sh2o)水,其中多■(b4ng)鱼,其状 如鳖,其音如羊。其阳多箭■②(m6i),其兽多■(zu¥)牛、羬(xi2n) 羊。有鸟焉,其状如鹑③(ch*n),黄身而赤喙(hu@),其名曰肥遗④, 食之已疠⑤(l@),可以杀虫。

【注释】①杻:杻树,长得近似于棣树,叶子细长,可以用来喂牛,木材能造车辋。橿:橿 树,木质坚硬,古人常用来制做车子。②箭■:一种节长、皮厚、根深的竹子,冬天可以从地下挖出 它的笋来吃。③鹑:即“鹌鹑”的简称,是一种鸟,体形像小鸡,头小尾短,羽毛赤褐色,有黄白色 条纹。雄性的鹌鹑好斗。④肥遗:这里讲的肥遗是一种鸟,而上文所说的肥遗是一种蛇,名称虽相同, 实际上却是两种动物。⑤疠:癞病,即麻疯。

【译文】再往西七十里,是座英山,山上到处是杻树和橿树,山北阴面 盛产铁,而山南阳面盛产黄金。禺水从这座山发源,向北流入招水,水中有 很多■鱼,形状像一般的鳖,发出的声音如同羊叫。山南面还生长有很多箭 竹和■竹,野兽大多是■牛、羬羊。山中有一种禽鸟,形状像一般的鹌鹑鸟, 是黄身子而红嘴巴,名称是肥遗,人吃了它的肉就能治愈麻疯病,还能杀死 体内寄生虫。

又西五十二里,曰竹山,其上多乔木,其阴多铁。有草焉,其名曰黄雚 (gu4n),其状如樗①(ch&),其叶如麻,白华而赤实,其状如赭②(zh7), 浴之已疥,又可以已胕(f*)。竹水出焉,北流注于渭,其阳多竹箭,多苍 玉。丹水出焉,东南流注于洛水,其中多水玉,多人鱼。有兽焉,其状如豚 而白毛,[毛]大如笄(j9)而黑端③,名曰豪彘④(zh@)。

【注释】①樗:即臭椿树,长得很高大,树皮灰色而不裂,小枝粗壮,羽状复叶,夏季开白 绿色花。②赭:赭石,就是现在所说的赤铁矿,即古人使用的一种黄棕色的矿物染料。③笄:即簪子, 是古人用来插住挽起的头发或连住头发上的冠帽的一种长针。④豪彘:即豪猪,俗称箭猪。

【译文】再往西五十二里,是座竹山,山上到处是高大的树木,山北面 盛产铁。山中有一种草,名称是黄雚,形状像樗树,但叶子像麻叶,开白色 的花朵而结红色的果实,果实外表的颜色像赭色,用它洗浴就可治愈疥疮, 又可以治疗浮肿病。竹水从这座山发源,向北流入渭水,竹水的北岸有很多 的小竹丛,还有许多青色的玉石。丹水也发源于这座山,向东南流入洛水, 水中多出产水晶石,又有很多人鱼。山中有一种野兽,形状像小猪却长着白 色的毛,毛如簪子粗细而尖端呈黑色,名称是豪彘。

又西百二十里,曰浮山,多盼木,枳(zh!)叶而无伤①,木虫居之。

有草焉,名曰熏(x&n)草,麻叶而方茎,赤华而黑实,臭(xi))如靡(m0) 芜②,佩之可以已疠(l@)。

【注释】①枳:枳树,也叫做“枸橘”、“臭橘”,叶子上有粗刺。复叶,小叶三片,有透 明腺点。无伤:指没有能刺伤人的尖刺。②臭:气味。蘼芜:一种香草,闻起来像兰花的气味。

【译文】再往西一百二十里,是座浮山,到处是盼木,长着枳树一样的 叶子却没有刺,树木上的虫子寄生于此。山中有一种草,名称是熏草,叶子 像麻叶却长着方方的茎干,开红色的花朵而结黑色的果实,气味像蘼芜,把 它插在身上就可以治疗麻疯病。

又西七十里,曰羭(y*)次之山,漆水出焉,北流注于渭。其上多棫(y)) 橿①(ji1ng),其下多竹箭,其阴多赤铜②,其阳多婴垣(yu2n)之玉③。

有兽焉,其状如禺而长臂,善投,其名曰嚻(xi1o)④。有鸟焉,其状如枭 (xi1o),人面而一足,曰橐(tu¥)■(f6i),冬见(xi4n)夏蛰(zh6) ⑤,服之不畏雷。

【注释】①棫:棫树,长得很小,枝条上有刺,结的果子像耳珰,红紫色,可以吃。②赤铜: 即黄铜。这里指未经提炼过的天然铜矿石。以下同此。③婴垣:一种玉石,主要可用来制做挂在脖子 上的装饰品。④嚻:一种野兽,古人说它就是猕猴,形貌与人相似。⑤蛰:动物冬眠时潜伏在土中或 洞穴中不食不动的状态。

【译文】再往西七十里,是座羭次山。漆水发源于此,向北流入渭水。

山上有茂密的棫树和橿树,山下有茂密的小竹丛,山北阴面有丰富的赤铜, 而山南阳面有丰富的婴垣玉。山中有一种野兽,形状像猿猴而双臂很长,擅 长投掷,名称是嚻。山中还有一种禽鸟,形状像一般的猫头鹰,长着人一样 的面孔而只有一只脚,叫做橐■,常常是冬天出现而夏天蛰伏,把它的羽毛 插在身上就使人不怕打雷。

又西百五十里,曰时山,无草木。逐水出焉,北流注于渭,其中多水玉。

【译文】再往西一百五十里,是座时山,没有花草树木。逐水从这座山 发源,向北流入渭水。水中有很多水晶石。

又西百七十里,曰南山,上多丹粟。丹水出焉,北流注于渭。兽多猛豹 ①,鸟多尸鸠(ji&)。② 【注释】①猛豹:传说中的一种野兽,形体与熊相似而小些,浅色的毛皮有光泽,吃蛇,还 能吃铜铁。②尸鸠:即布谷鸟。

【译文】再往西一百七十里,是座南山,到处是粟粒大小的丹沙。丹水 从这座山发源,向北流入渭水。山中的野兽大多是猛豹,而禽鸟大多是布谷 鸟。

又西百八十里,曰大时之山,上多(穀)[榖](g^u)柞(zu^)①,下 多杻(ni()僵(ji1ng),阴多银,阳多白玉。涔(qi4n)水出焉,北流注 于渭。清水出焉,南流注于汉水。

【注释】①柞:古人说就是栎树。它的木材可供建筑、器具、薪炭等用。

【译文】再往西一百八十里,是座大时山,山上有很多构树和栎树,山 下有很多杻树和僵树,山北面多出产银,而山南面有丰富的白色玉石。涔水 从这座山发源,向北流入渭水。清水也从这座山发源,却向南流入汉水。

又西三百二十里,曰嶓(b#)冢之山,汉水出焉,而东南流注于沔(mi3n); 嚻(xi1o)水出焉,北流注于汤水。其上多桃枝鉤端①,兽多犀(x9)兕(s@) 熊罴②(p0),鸟多白翰赤鷩(b6i)③。有草焉,其叶如蕙④,其本如桔 (j*)梗⑤,黑华而不实,名曰蓇(g()蓉,食之使人无子。

【注释】①桃枝:一种竹子,它每隔四寸为一节。鉤端:属于桃枝竹之类的竹子。②罴:熊 的一种。③白翰:一种鸟,就是白雉,又叫白鹇,雄性白雉鸟的上体和两翼白色,尾长,中央尾羽纯 白。这种鸟常栖高山竹林间。④蕙:蕙草,是一种香草,属于兰草之类。⑤桔梗:橘树的茎干。

【译文】再往西三百二十里,是座嶓冢山,汉水发源于此,然后向东南 流入沔水;嚻水也发源于此,向北流入汤水。山上到处是葱茏的桃枝竹和鉤 端竹,野兽以犀牛、兕、熊、罴最多,禽鸟却以白翰和赤鷩最多。山中有一 种草,叶子长得像蕙草叶,茎干却像桔梗,开黑色花朵但不结果实,名称是 蓇蓉,吃了它就会使人不生育孩子。

又西三百五十里,曰天帝之山,上多棕枏(n2n),下多菅(ji1n)蕙。

有兽焉,其状如狗,名曰溪边,席其皮者不蛊(g()①。有鸟焉,其状如鹑, 黑文而赤翁②,名曰栎(l@),食之已痔。有草焉,其状如葵,其臭(xi)) 如靡芜,名曰杜衡③,可以走马,食之已瘿(y!ng)。④  【注释】①席:这里作动词用,铺垫的意思。②翁:鸟脖子上的毛。③杜衡:一种香草。④ 瘿:一种人体局部细胞增生的疾病,一般形成囊状性的赘生物,形状、大小下一,多肉质。这里指脖 颈部所生肉瘤。

【译文】再往西三百五十里,是座天帝山,山上是茂密的棕树和楠木树, 山下主要生长茅草和蕙草。山中有一种野兽,形状像普通的狗,名称是溪边, 人坐卧时铺垫上溪边兽的皮就不会中妖邪毒气。山中又有一种禽鸟,形状像 一般的鹌鹑鸟,但长着黑色的花纹和红色的颈毛,名称是栎,人吃了它的肉 可以治愈痔疮。山中还有一种草,形状像葵菜,散发出和蘼芜一样的气味, 名称是杜衡,给马插戴上它就可以使马跑得很快,而人吃了它就可以治愈脖 子上的赘瘤病。

西南三百八十里,曰皋(g1o)涂之山,蔷(s6)水出焉,西流注于诸 资之水;涂水出焉,南流注于集获之水。其阳多丹粟,其阴多银、黄金,其 上多桂木。有白石焉,其名曰礜(y))①,可以毒鼠。有草焉,其状如藁(g3o) 茇(b2)②,其叶如葵而赤背,名曰无条,可以毒鼠。有兽焉,其状如鹿而 白尾,马脚人手而四角,名曰(■)[玃](ju6)如。有鸟焉,其状如鸱(ch9) 而人足,名曰数斯,食之已瘿。

【注释】①礜:即礜石,一种矿物,有毒。苍白二色的礜石可以入药。如果山上有各种礜石, 草木不能生长,霜雪不能积存;如果水里有各种礜石,就会使水不结冰。②藁茇:一种香草,根茎可 以入药。

【译文】往西南三百八十里,是座皋涂山,蔷水发源于此,向西流入诸 资水;涂水也发源于此,向南流入集获水。山南面到处是粟粒大小的丹沙, 山北阴面盛产银、黄金,山上到处是桂树。山中有一种白色的石头,名称是 礜,可以用来毒死老鼠。山中又有一种草,形状像藁茇,叶子像葵菜的叶子 而背面是红色的,名称是无条,可以用来毒死老鼠。山中还有一种野兽,形 状像普通的鹿却长着白色的尾巴,马一样的脚蹄、人一样的手而又有四只 角,名称是玃如。山中还有一种禽鸟,形状像鹞鹰却长着人一样的脚,名称 是数斯,吃了它的肉就能治愈人脖子上的赘瘤病。

又西百八十里,曰黄山,无草木,多竹箭。盼水出焉,西流注于赤水, 其中多玉。有兽焉,其状如牛,而苍黑大目,其名曰■(m!n)。有鸟焉, 其状如鸮(xi1o),青羽赤喙(hu@),人舌能言,名曰鹦■(m¥u)①。

【注释】①鹦■:即鹦鹉,俗称鹦哥,羽毛色彩美丽,舌头肉质而柔软,经反复训练,能模 仿人说话的声音。有许多的种类。

【译文】再往西一百八十里,是座黄山,没有花草树木,到处是郁郁葱 葱的竹丛。盼水从这座山发源,向西流入赤水,水中有很多玉石。山中有一 种野兽,形状像普通的牛,却长着苍黑色的皮毛大大的眼睛,名称是■。山 中又有一种禽鸟,形状像一般的猫头鹰,却长着青色的羽毛和红色的嘴,像 人一样的舌头能学人说话,名称是鹦■。

又西二百里,曰翠山,其上多棕枏(n2n),其下多竹箭,其阳多黄金、 玉,其阴多旄(m2o)牛、麢(l0ng)、麝(sh6)①。其鸟多鸓(l7i), 其状如鹊,赤黑而两首、四足,可以御火。

【注释】①旄牛:即牦牛。麢:即羚羊,形状像羊而大一些,角圆锐,喜好在山崖间活动。

麢,同“羚”。麝:一种动物,也叫香獐,前肢短,后肢长,蹄子小,耳朵大,体毛棕色,雌性和雄 性都没有角。雄性麝的脐与生殖孔之间有麝腺,分泌的麝香可作药用和香料用。

【译文】再往西二百里,是座翠山,山上是茂密的棕树和楠木树,山下 到处是竹丛,山南面盛产黄金、玉,山北面有很多牦牛、羚羊、麝。山中的 禽鸟大多是鸓鸟,形状像一般的喜鹊,却长着红黑色羽毛和两个脑袋、四只 脚,人养着它可以辟火。

又西二百五十里,曰騩(gu9)山,是錞(ch*n)于西海①,无草木, 多玉。淒水出焉,西流注于海,其中多采石、黄金②,多丹粟。

【注释】①錞:依附。这里是座落、高踞的意思。②采石:据古人说是一种彩色石头,就像 雌黄之类的矿物。

【译文】再往西二百五十里,是座騩山,它座落在西海边上,这里没有 花草树木,却有很多玉石。淒水从这座山发源,向西流入大海,水中有许多 采石、黄金,还有很多粟粒大小的丹沙。

凡西(经)[山]之首,自钱来之山至于騩(gu9)山,凡十九山,二千 九百五十七里。华山冢(zh%ng)也,其祠之礼:太牢①。羭(y*)山神也, 祠之用烛,斋百日以百牺②,瘗(y@)用百瑜(y*)③,汤其酒百樽④,婴 以百珪(gu9)百璧⑤。其余十七山之属,皆毛■(qu2n)用一羊祠之⑥。

烛者,百草之未灰,白席采等纯之。

【注释】①太牢:古人进行祭祀活动时,祭品所用牛、羊、猪三牲全备为太牢。②斋:古人 在祭祀前或举行典礼前清洁身体以示庄敬。牺:古代祭祀时用的纯色的牲。牲是供祭祀用的整体的家 畜。③瑜:美玉。④汤:通“烫”。⑤婴:据学者研究,婴是用玉器祭祀神的专称。珪:同“圭”, 一种玉器,长条形,上端作三角状,是古时朝聘、祭祀、丧葬所用的礼器之一。⑥毛■:指祀神所用 毛物牲畜是整体全具的。

【译文】总计西方第一列山系之首尾,自钱来山起到騩山止,一共十九 座山,途经二千九百五十七里。华山神是诸山神的宗主,祭祀华山山神的典 礼:用猪、牛、羊齐全的三牲作祭品。羭山神是神奇威灵的,祭祀羭山山神 用烛火,斋戒一百天后用一百只毛色纯正的牲畜,随一百块瑜埋入地下,再 烫上一百樽美酒,祀神的玉器用一百块玉珪和一百块玉璧。祭祀其余十七座 山山神的典礼相同,都是用一只完整的羊作祭品。所谓的烛,就是用百草制 作的火把但未烧成灰的时候,而祀神的席是用各种颜色等差有序地将边缘装 饰起来的白茅草席。

西次二(经)[山]之首,曰钤(qi2n)山,其上多铜,其下多玉,其木 多杻(ni()橿(ji1ng)。

【译文】西方第二列山系之首座山,叫做钤山,山上盛产铜,山下盛产 玉,山中的树大多是杻树和橿树。

西二百里,曰泰冒之山,其阳多金,其阴多铁。(浴)[洛]水出焉,东 流注于河①,其中多藻玉②。多白蛇。

【注释】①河:古人单称“河”或“河水”而不贯以名者,则大多是专指黄河,这里即指黄 河。但本书记述山川水流的方位走向都不甚确实,所述黄河也不例外,再加上黄河在古时屡次改道, 所以,和今天所看到的黄河不尽一致。现在译“河”或“河水”为“黄河”,只是为了使译文醒目而 有别于其它河流。以下同此。②藻玉:带有色彩纹理的美玉。

【译文】向西二百里,是座泰冒山,山南面多出产金,山北面多出产铁。

洛水从这座山发源,向东流入黄河,水中有很多藻玉,还有很多白色的水蛇。

又西一百七十里,曰数历之山,其上多黄金,其下多银,其木多杻(ni() 橿(ji1ng),其鸟多鹦■。楚水出焉,而南流注于渭,其中多白珠。

【译文】再往西一百七十里,是座数历山,山上盛产黄金,山下盛产银, 山中的树木大多是杻树和橿树,而禽鸟大多是鹦■。楚水从这座山发源,然 后向南流入渭水,水中有很多白色的珍珠。

又西北五十里,[曰]高山,其上多银,其下多青碧、雄黄①,其木多棕, 其草多竹②。泾水出焉,而东流注于渭,其中多磬(q@ng)石、青碧。

【注释】①青碧:青绿色的美玉。雄黄:也叫鸡冠石,是一种矿物,古人常用作解毒、杀虫 的药物。②竹:这里指低矮而丛生的小竹子,所以被当作草。

【译文】再往西北五十里,是座高山,山上有丰富的白银,山下到处是 青碧、雄黄,山中的树木大多是棕树,而草大多是小竹丛。泾水从这座山发 源,然后向东流入渭水,水中有很多磬石、青碧。

西南三百里,曰女床之山,其阳多赤铜,其阴多石涅(ni6)①,其兽 多虎、豹、犀(x9)、兕(s@)。有鸟焉,其状如翟(d@)而五采文②,名 曰鸾鸟③,见(xi4n)则天下安宁。

【注释】①石涅:据古人讲,就是石墨,古时用作黑色染料,也可以画眉和写字。②翟:一 种有很长尾巴的野鸡,形体也比一般的野鸡要大些。③鸾鸟:传说中的一种鸟,属于凤凰一类。

【译文】往西南三百里,是座女床山,山南面多出产黄铜,山北面多出 产石涅,山中的野兽以老虎、豹子、犀牛和兕居多。山里还有一种禽鸟,形 状像野鸡却长着色彩斑斓的羽毛,名称是鸾鸟,一出现天下就会安宁。

又西二百里,曰龙首之山,其阳多黄金,其阴多铁。苕水出焉,东南流 注于泾水,其中多美玉。

【译文】再往西二百里,是座龙首山,山南面盛产黄金,山北面盛产铁。

苕水从这座山发源,向东南流入泾水,水中有很多美玉。

又西二百里,曰鹿台之山,其上多白玉,其下多银,其兽多■(zu¥) 牛、羬(xi2n)羊、白豪①。有鸟焉,其状如雄鸡而人面,名曰凫(f*)徯 (x9),其鸣自叫也,见(xi4n)则有兵②。

【注释】①白豪:长着白毛的豪猪。②兵:军事,战斗。

【译文】 再往西二百里,是座鹿台山,山上多出产白玉,山下多出产 银,山中的野兽以■牛、羬羊、白豪居多。山中有一种禽鸟,形状像普通的 雄鸡却长着人一样的脸面,名称是凫徯,它的叫声就是自身名称的读音,一 出现则天下就会有战争。

西南二百里,曰鸟危之山,其阳多磬(q@ng)石,其阴多檀(t2n)楮 (ch()①,其中多女床②。鸟危之水出焉,西流注于赤水,其中多丹粟。

【注释】①檀:檀树,木材极香,可作器具。楮:即构树,长得很高大,皮可以制做桑皮纸。

②女床:据古人说是女肠草。

【译文】往西南二百里,是座鸟危山,山南面多出产磬石,山北面到处 是檀树和构树,山中生长着很多女肠草。鸟危水从这座山发源,向西流入赤 水,水中有许多粟粒大小的丹沙。

又西四百里,曰小次之山,其上多白玉,其下多赤铜。有兽焉,其状如 猿,而白首赤足,名曰朱厌,见(xi4n)则大兵。

【译文】再往西四百里,是座小次山,山上盛产白玉,山下盛产黄铜。

山中有一种野兽,形状像普通的猿猴,但头是白色的、脚是红色的,名称是 朱厌,一出现就会大起战事。

又西三百里,曰大次之山,其阳多垩(6),①其阴多碧,其兽多■(zu¥) 牛、麢(l0ng)羊。

【注释】①垩:能用作涂饰粉刷墙壁的泥土,有白、红、青、黄等多种颜色。

【译文】再往西三百里,是座大次山,山南面多出产垩土,山北面多出 产碧玉,山中的野兽以■牛、羚羊居多。

又西四百里,曰熏吴之山,无草木,多金玉。

【译文】再往西四百里,是座熏吴山,山上没有花草树木,而有丰富的 金属矿物和玉石。

又西四百里,曰(■)[厎](zh!)阳之山,其木多■(j@)、枏(n2n)、 豫章①,其兽多犀(x9)、兕(s@)、虎、犳(zhu¥)、■(zu¥)牛②。

【注释】①■:即水松,有刺,木头纹理很细。豫章:古人说就是樟树,也叫香樟,常绿乔 木,有樟脑香气。古代还有一种说法,认为二树在幼小时不可辨知而被人看作一种树木,其实,豫就 是枕木,章就是樟木,生长到七年以后,枕、章才能分别。②犳:据古人讲是一种身上有豹子斑纹的 野兽。

【译文】再往西四百里,是座厎阳山,山中的树木大多是水松树、楠木 树、樟树,而野兽大多是犀牛、兕、老虎、犳、■牛。

又西二百五十里,曰众兽之山,其上多■(y()琈(f*)之玉,其下多 檀(t2n)楮(ch(),多黄金,其兽多犀、兕。

【译文】再往西二百五十里,是座众兽山,山上遍布■琈玉,山下到处 是檀树和构树,有丰富的黄金,山中的野兽以犀牛、兕居多。

又西五百里,曰皇人之山,其上多金玉,其下多青、雄黄①。皇水出焉, 西流注于赤水,其中多丹粟。

【注释】①青:这里指石青,是一种矿物,可以制做蓝色染料。

【译文】再往西五百里,是座皇人山,山上有丰富的金属矿物和玉石, 山下有丰富的石青、雄黄。皇水从这座山发源,向西流入赤水,水中有很多 粟粒大小的丹沙。

又西三百里,曰中皇之山,其上多黄金,其下多蕙、棠(t2ng)①。

【注释】①棠:这里指棠梨树,结的果实似梨而小点,可以吃,味道甜酸。

【译文】再往西三百里,是座中皇山,山上多出产黄金,山下长满了蕙 草、棠梨树。

又西三百五十里,曰西皇之山,其阳多金,其阴多铁,其兽多麋(m@)、 鹿、麋(zu¥)牛①。

【注释】①麋:即麋鹿,毛色淡褐,背部较浓,腹部较浅,而雄性有角。因它的角像鹿角又 不像,头像马头又不像,身子像驴身又不像,蹄子像牛蹄又不像,所以古人又称作“四不像”。

【译文】再往西三百五十里,是座西皇山,山南面多出产金,山北面多 出产铁,山中的野兽以麋、鹿、麋牛居多。

又西三百五十里,曰莱山,其木多檀(t2n)楮(ch(),其鸟多罗罗, 是食人。

【译文】再往西三百五十里,是座莱山,山中的树木大多是檀树和构树, 而禽鸟大多是罗罗鸟,是能吃人的。

凡西次二(经)[山]之首,自钤(qi2n)山至于莱山,凡十七山,四千 一百四十里。其十神者,皆人面而马身。其七神皆人面牛身,四足而一臂, 操杖以行,是为飞兽之神。其祠之,毛用少(sh4o)牢①,白菅(ji1n)为 席,其十辈神者,其祠之,毛一雄鸡,钤而不糈(x();毛采。

【注释】①毛:指毛物,就是祭神所用的猪、鸡、狗、羊、牛等畜禽。少牢:古代称祭祀用 的猪和羊。

【译文】总计西方第二列山系之首尾,自钤山起到莱山止,一共十七座 山,途经四千一百四十里。其中十座山的山神,都是人的面孔而马的身子。

还有七座山的山神都是人的面孔而牛的身子,四只脚和一条臂,扶着拐杖行 走,这就是所谓的飞兽之神,祭祀这七位山神,在毛物中用猪、羊作祭品, 将其放在白茅草席上。另外那十位山神,祭祀的典礼,在毛物中用一只公鸡, 祭祀神时不用米作祭品;毛物的颜色要杂而不必纯一。

西次三(经)[山]之首,曰崇吾之山,在河之南,北望冢(zh%ng)遂, 南望■(y3o)之泽,西望帝之搏兽之(丘)[山],东望■(y1n)渊。有木 焉,员叶而白柎(f()①,赤华而黑理,其实如枳(zh!),食之宜子孙。

有兽焉,其状如禺而文臂,豹(虎)[尾]而善投,名曰举父。有鸟焉,其状 如凫(f*) ,而一翼一目,相得乃飞,名曰蛮蛮,见(xi4n)则天下大水。

【注释】①员:通“圆”。柎:花萼。是由若干萼片组成,处在花的外轮,起保护花芽的作 用。

【译文】西方第三列山系之首座山,叫做崇吾山,它雄居于黄河的南岸, 在山上向北可以望见冢遂山,向南可以望见■泽,向西可以望见天帝的搏兽 山,向东可以望见■渊。山中有一种树木,圆圆的叶子白色的花萼,红色的 花朵上有黑色的纹理,结的果实与枳实相似,吃了它就能使人多子多孙。山 中又有一种野兽,形状像猿猴而臂上却有斑纹,有豹子一样的尾巴而擅长投 掷,名称是举父。山中还有一种禽鸟,形状像一般的野鸭子,却只长了一只 翅膀和一只眼睛,要两只鸟合起来才能飞翔,名称是蛮蛮,一出现而天下就 会发生水灾。

西北三百里,曰长沙之山。泚(c!)水出焉,北流注于泑(y#u)水, 无草木,多青、雄黄。

【译文】往西北三百里,是座长沙山。泚水从这里发源,向北流入泑水, 山上没有花草树木,多的是石青、雄黄。

又西北三百七十里,曰不周之山①。北望诸■(b9)之山,临彼岳崇之 山,东望泑(y#u)泽,河水所潜也,其原浑浑(g(ng(n)泡泡(p2op2o) ②。爰(yu2n)有嘉果,其实如桃,其叶如枣,黄华而赤柎(f(),食之不 劳。

【注释】①不周之山:即不周山。据古人讲,因为这座山的形状有缺而不周全的地方,所以 叫不周山。山的西北部不周全,风从这里刮出,称为不周风。传说山形有缺而不周全的原因,是共工 与颛顼争帝位时发怒触撞造成的。②原:“源”的本字。水源。浑浑泡泡:形容水喷涌的声音。

【译文】再往北三百七十里,是座不周山。在山上向北可以望见诸■山, 高高的居于岳崇山之上,向东可以望见泑泽,是黄河源头所潜在的地方,那 源头之水喷涌而发出浑浑泡泡的响声。这里有一种特别珍贵的果树,结出的 果实与桃子很相似,叶子却很像枣树叶,开着黄色的花朵而花萼却是红红 的,吃了它就能使人解除烦恼忧愁。

又西北四百二十里,曰峚(m@)山,其上多丹木,员叶而赤茎,黄华而 赤实,其味如饴(y0),食之不饥。丹水出焉,西流注于稷泽,其中多白玉。

是有玉膏,其原沸沸(f6i f6i)汤汤(sh1ngsh1ng)①,黄帝是食是飨(xi3ng) ②。是生玄玉。玉膏所出,以灌丹木,丹木五岁,五色乃清,五味乃馨③。

黄帝乃取峚山之玉荣④,而投之钟山之阳。瑾(j!n)瑜之玉为良⑤,坚(粟) [栗]精密⑥,浊泽(有)而[有]光。五色发作,以和柔刚。天地鬼神,是食 是飨;君子服之,以御不祥。自峚山至于钟山,四百六十里,其间尽泽也。

是多奇鸟、怪兽、奇鱼,皆异物焉。

【注释】 ①沸沸汤汤:水腾涌的样子。②飨:通“享”。享受。③馨:芳香。④玉荣:玉 华。⑤瑾:美玉。⑥栗:坚。

【译文】再往西北四百二十里,是座峚山,山上到处是丹木,红红的茎 干上长着圆圆的叶子,开黄色的花朵而结红色的果实,味道是甜的,人吃了 它就不感觉饥饿。丹水从这座山发源,向西流入稷泽,水中有很多白色玉石。

这里有玉膏,玉膏之源涌出时一片沸沸腾腾的景象,黄帝常常服食享用这种 玉膏。这里还出产一种黑色玉石。用这涌出的玉膏,去浇灌丹木,丹木再经 过五年的生长,便会开出光艳美丽的五色花朵,结下味道香甜的五色果实。

黄帝于是就采撷峚山中玉石的精华,而投种在钟山向阳的南面。后来便生出 瑾和瑜这类美玉,坚硬而精密,润厚而有光泽。五种颜色的符彩一同散发出 来相互辉映,那就有刚有柔而非常和美。无论是天神还是地鬼,都来服食享 用;君子佩带它,能抵御妖邪不祥之气的侵袭。从峚山到钟山,长四百六十 里,其间全部是水泽。在这里生长着许许多多奇怪的禽鸟、怪异的野兽、神 奇的鱼类,都是些罕见的怪物。

又西北四百二十里,曰钟山。其子曰鼓,其状(如)人面而龙身,是与 钦■(p9)杀葆江于昆仑之阳,帝乃戮之钟山之东曰■崖。钦■化为大鹗(6) ①,其状如雕而黑文白首,赤喙(hu@)而虎爪,其音如晨鹄(h*)②,见 (xi4n)则有大兵;鼓亦化为鵕(j)n)鸟,其状如鸱(ch9),赤足而直喙 (hu@),黄文而白首,其音如鹄③,见(xi4n)则其邑(y@)大旱④。

【注释】①鹗:也叫鱼鹰,头顶和颈后羽毛白色,有暗褐色纵纹,头后羽毛延长成矛状。趾 具锐爪,趾底遍生细齿,外趾能前后转动,适于捕鱼。②晨鹄:鹗鹰之类的鸟。③鹄:也叫鸿鹄,即 天鹅,脖颈很长,羽毛白色,鸣叫的声音宏亮。④邑:这里泛指有人聚居的地方。

【译文】再往西北四百二十里,是座钟山。钟山山神的儿子叫做鼓,鼓 的形貌是人的脸面而龙的身子,他曾和钦■神联手在昆仑山南面杀死天神葆 江,天帝因此将鼓与钦■诛杀在钟山东面一个叫■崖的地方。钦■化为一只 大鹗,形状像普通的雕鹰却长有黑色的斑纹和白色的脑袋,红色的嘴巴和老 虎一样的爪子,发出的声音如同晨鹄鸣叫,一出现就有大的战争;鼓也化为 鵕鸟,形状像一般的鹞鹰,但长着红色的脚和直直的嘴,身上是黄色的斑纹 而头却是白色的,发出的声音与鸿鹄的鸣叫很相似,在哪个地方出现那里就 会有旱灾。

又西百八十里,曰泰器之山。观水出焉,西流注于流沙。是多文鳐(y2o) 鱼,状如鲤鱼,鱼身而鸟翼,苍文而白首赤喙(hu@),常行西海,游于东 海,以夜飞。其音如鸾鸡①,其味酸甘,食之已狂,见(xi4n)则天下大穰 (r2ng)②。

【注释】①鸾鸡:传说中的一种鸟。②穰:庄稼丰熟。

【译文】再往西一百八十里,是座泰器山,观水从这里发源,向西流入 流沙。这观水中有很多文鳐鱼,形状像普通的鲤鱼,长着鱼一样的身子和鸟 一样的翅膀,浑身是苍色的斑纹却是白脑袋和红嘴巴,常常在西海行走,在 东海畅游,在夜间飞行。它发出的声音如同鸾鸡鸟啼叫,而肉味是酸中带甜, 人吃了它的肉就可治好癫狂病,一出现而天下就会五谷丰登。

又西三百二十里,曰槐江之山。丘时之水出焉,而北流注于泑(y#u) 水。其中多蠃(l¥u)母,其上多青、雄黄,多藏琅(l2ng)馯(g1n)、黄 金、玉①,其阳多丹粟,其阴多采黄金银。实惟帝之平圃,神英招(sh2o) 司之,其状马身而人面,虎文而鸟翼,徇于四海,其音如(榴)[■](ch#u) ②。南望昆仑,其光熊熊,其气魂魂。西望大泽③,后稷所潜也④。其中多 玉,其阴多榣木之有若⑤。北望诸■(p0),槐鬼离仑居之,鹰鸇(zh1n) 之所宅也⑥。东望(恒)[桓]山四成,有穷鬼居之,各在一(搏)[抟](tu2n) ⑦。爰有淫水⑧,其清洛洛⑨。有天神焉,其状如牛,而八足二首马尾,其 音如勃皇,见(xi4n)则其邑有兵。

【注释】①琅玕:像玉一样的石头。②■:同“抽”。引出,提取。③大泽:后稷所葬的地 方。传说后稷出生以后,就很灵慧而且先知,到他死时,便化形而遁于大泽成为神。④后稷:周人的 先祖。相传他在虞舜时任农官,善于种庄稼。⑤■木:特别高大的树木。若:即若木,神话传说中的 树,具有奇异而神灵的特性。⑥鸇:鹞鹰一类的鸟。⑦抟:把散碎的东西捏聚成团。⑧淫水:洪水。

这里指水从山上流下时广阔而四溢的样子。⑨洛洛:形容水流声。

【译文】再往西三百二十里,是座槐江山。丘时水从这座山发源,然后 向北流入泑水。水中有很多■螺,山上蕴藏着丰富的石青、雄黄,还有很多 的琅玕、黄金、玉石,山南面到处是粟粒大小的丹沙,而山北阴面多产带符 彩的黄金白银。这槐江山确实可以说是天帝悬在半空的园圃,由天神英招主 管着,而天神英招的形状是马的身子而人的面孔,身上长有老虎的斑纹和禽 鸟的翅膀,巡行四海而传布天帝的旨命,发出的声音如同用辘轳抽水。在山 上向南可以望见昆仑山,那里火光熊熊,气势恢宏。向西可以望见大泽,那 里是后稷死后埋葬之地。大泽中有很多玉石,大泽的南面有许多榣木,而在 它上面又有若木。向北可以望见诸■山,是叫做槐鬼离仑的神仙所居住的地 方,也是鹰鸇等飞禽的栖息地。向东可以望见那四重高的桓山,有穷鬼居住 在那里,各自分类聚集于一起。这里有大水下泻,清清冷冷而汩汩流淌。有 个天神住在山中,他的形状像普通的牛,但却长着八只脚、两个脑袋并拖着 一条马的尾巴,啼叫声如同人在吹奏乐器时薄膜发出的声音,在哪个地方出 现那里就有战争。

西南四百里,曰昆仑之丘①,是实惟帝之下都,神陆吾司之。其神状虎 身而九尾,人面而虎爪;是神也,司天之九部及帝之囿(y^u)时②。有兽 焉,其状如羊而四角,名曰土蝼,是食人。有鸟焉,其状如蜂,大如鸳鸯, 名曰钦原,蠚(ru¥)鸟兽则死③,蠚木则枯。有鸟焉,其名曰鹑鸟④,是 司帝之百服。有木焉,其状如棠,黄华赤实,其味如李而无核,名曰沙棠, 可以御水,食之使人不溺。有草焉,名曰■(p0n)草,其状如葵,其味如 葱,食之已劳。河水出焉,而南流东注于无达。赤水出焉,而东南流注于汜 (f4n)天之水。洋水出焉,而西南流注于丑涂之水。黑水出焉,而西流于 大汜(y*)。是多怪鸟兽。

【注释】①昆仑之丘:即昆仑山,神话传说中天帝居住的地方。②九部:据古人解释是九域 的部界。囿:古代帝王畜养禽兽的园林。③蠚:毒虫类咬刺。④鹑鸟:传说中的凤凰之类的鸟,和上 文所说的鹑鸟即鹌鹑不同。

【译文】往西南四百里,是座昆仑山,这里确实是天帝在下界的都邑, 天神陆吾主管它。这位天神的形貌是老虎的身子却有九条尾巴,一副人的面 孔可长着老虎的爪子;这个神,主管天上的九部和天帝苑圃的时节。山中有 一种野兽,形状像普通的羊却长着四只角,名称是土蝼,是能吃人的。山中 有一种禽鸟,形状像一般的蜜蜂,大小与鸳鸯差不多,名称是钦原,这种钦 原鸟刺螫其它鸟兽就会使它们死去,刺螫树木就会使树木枯死。山中还有另 一种禽鸟,名称是鹑鸟,它主管天帝日常生活中各种器用服饰。山中又有一 种树木,形状像普通的棠梨树,却开着黄色的花朵并结出红色的果实,味道 像李子却没有核,名称是沙棠,可以用来辟水,人吃了它就能漂浮不沉。山 中还有一种草,名称是■草,形状很像葵菜,但味道与葱相似,吃了它就能 使人解除烦恼忧愁。黄河水从这座山发源,然后向南流而东转注入无达山。

赤水也发源于这座山,然后向东南流入汜天水。洋水也发源于这座山,然后 向西南流入丑涂水。黑水也发源于这座山,然后向西流到大杆山。这座山中 有许多奇怪的鸟兽。

又西三百七十里,曰乐游之山。桃水出焉,西流注于稷泽,是多白玉, 其中多■(hu2)鱼,其状如蛇而四足,是食鱼。

【译文】再往西三百七十里,是坐乐游山。桃水从这座山发源,向西流 入稷泽,这里到处有白色玉石,水中还有很多■鱼,形状像普通的蛇却长着 四只脚,是能吃鱼类的。

西水行四百里,曰流沙,二百里至于蠃(lu¥)母之山,神长乘司之, 是天之九德也。其神状如人而犳(gu#)尾①。其上多玉,其下多青石而无 水。

【注释】①犳:一种类似于豹子的野兽。

【译文】往西行四百里水路,就是流沙,再行二百里便到蠃母山,天神 长乘主管这里,他是天的九德之气所生。这个天神的形貌像人却长着犳的尾 巴。山上到处是玉石,山下到处是青石而没有水。

又西三百五十里,曰玉山①,是西王母所居也。西王母其状如人,豹尾 虎齿而善啸(xi4o)②,蓬发戴胜③,是司天之厉及五残。有兽焉,其状如 犬而豹文,其角如牛,其名曰狡,其音如吠(f6i)犬,见(xi4n)则其国 大穰(r2ng)。有鸟焉,其状如翟(d@)而赤,名曰胜(x@ng)遇,是食鱼, 其音如(录)[鹿],见(xi4n) 则其国大水。

【注释】①玉山:据古人讲,这座山遍布着玉石,所以叫做玉山。②啸:兽类长声吼叫。③ 胜:指玉胜,古时用玉制做的一种首饰。

【译文】再往西三百五十里,是座玉山,这是西王母居住的地方。西王 母的形貌与人一样,却长着豹子一样的尾巴和老虎一样的牙齿而且喜好啸 叫,蓬松的头发上戴着玉胜,是主管上天灾厉和五刑残杀之气的。山中有一 种野兽,形状像普通的狗却长着豹子的斑纹,头上的角与牛角相似,名称是 狡,发出的声音如同狗叫,在哪个国家出现就会使那个国家五谷丰登。山中 还有一种禽鸟,形状像野鸡却通身是红色,名称是胜遇,是能吃鱼类的,发 出的声音如同鹿在鸣叫,在哪个国家出现就会使那个国家发生水灾。

又西四百八十里,曰轩辕之丘①,无草木。洵水出焉,南流注于黑水, 其中多丹粟,多青、雄黄。

【注释】①轩辕之丘:即轩辕丘,传说上古帝王黄帝居住在这里,娶西陵氏女为妻,因此也 号称轩辕氏。

【译文】再往西四百八十里,是座轩辕丘,这里没有花草树木。洵水从 轩辕丘发源,向南流入黑水,水中有很多粟粒大小的丹沙,还有很多石青、 雄黄。

又西三百里,曰积石之山,其下有石门,河水冒以西[南]流。是山也, 万物无不有焉。

【译文】再往西三百里,是座积石山,山下有一个石门,黄河水漫过石 门向西南流去。这座积石山,是万物俱全的。

又西二百里,曰长留之山,其神白帝少昊(h4o)居之①。其兽皆文尾, 其鸟皆文首。是多文玉石。实惟员神磈(w7i)氏之宫②。是神也③,主司 反景(y!ng)④。

【注释】①白帝少昊:即少昊金天氏,传说中上古帝王帝挚的称号。②磈氏:即白帝少昊。

③神:指少昊。④景:通“影”。

【译文】再往西二百里,是座长留山,天神白帝少昊居住在这里。山中 的野兽都是花尾巴,而禽鸟都是花脑袋。山上盛产彩色花纹的玉石。它实是 员神磈氏的宫殿。这个神,主要掌管太阳落下西山时光线射向东方的反影。

又西二百八十里,曰章莪(6)之山,无草木,多瑶、碧。所为甚怪。

有兽焉,其状如赤豹,五尾一角,其音如击石,其名(如)[曰]狰(zh5ng)。

有鸟焉,其状如鹤,一足,赤文青质而白喙(hu@),名曰毕方①,其鸣自 叫也,见(xi4n)则其邑有讹(6)火②。

【注释】①毕方:传说是树木的精灵,形貌与鸟相似,青色羽毛,只长着一只脚,不吃五谷。

又传说是老父神,形状像鸟,两只脚,一只翅膀,常常衔着火到人家里去制造火灾。②讹火:怪火, 像野火那样莫名其妙地烧起来。

【译文】再往西二百八十里,是座章莪山,山上没有花草树木,到处是 瑶、碧一类的美玉。山里常常出现十分怪异的物象。山中有一种野兽,形状 像赤豹,长着五条尾巴和一只角,发出的声音如同敲击石头的响声,名称是 狰。山中还有一种禽鸟,形状像一般的鹤,但只有一只脚,红色的斑纹和青 色的身子而有一张白嘴巴,名称是毕方,它鸣叫的声音就是自身名称的读 音,在哪个地方出现那里就会发生怪火。

又西三百里,曰阴山。浊浴之水出焉,而南流注于蕃泽,其中多文贝。

有兽焉。其状如狸而白首,名曰天狗,其音如(榴榴)[猫猫],可以御凶。

【译文】再往西三百里,是座阴山。浊浴水从这座山发源,然后向南流 入蕃泽,水中有很多五彩斑斓的贝壳。山中有一种野兽,形状像野猫却是白 脑袋,名称是天狗,它发出的叫声与“猫猫”的读音相似,人饲养它可以辟 凶邪之气。

又西二百里,曰符惕(y2ng)之山,其上多棕枏(n2n),下多金玉。

神江疑居之。是山也,多怪雨,风云之所出也。

【译文】再往西二百里,是座符惕山,山上到处是棕树和楠木树,山下 有丰富的金属矿物和玉石。一个叫江疑的神居住于此。这座符惕山,常常落 下怪异之雨,风和云也从这里兴起。

又西二百二十里,曰三危之山,三青鸟居之①。是山也,广员百里。其 上有兽焉,其状如牛,白身四角,其豪如披蓑(su#)②,其名曰■(4o) 秵(y5),是食人。有鸟焉,一首而三身,其状如■(lu^)③,其名曰鸱 (ch9)。

【注释】①三青鸟:神话传说中的鸟,专为西王母取送食物。②豪:豪猪身上的刺。这里指 长而刚硬的毛。蓑:遮雨用的草衣。③■:与雕鹰相似的鸟,黑色斑纹,红色脖颈。

【译文】再往西二百二十里,是座三危山,三青鸟栖息在这里。这座三 危山,方圆百里。山上有一种野兽,形状像普通的牛,却长着白色的身子和 四只角,身上的硬毛又长又密好像披着蓑衣,名称是■秵,是能吃人的。山 中还有一种禽鸟,长着一个脑袋却有三个身子,形状与■鸟很相似,名称是 鸱。

又西一百九十里,曰騩(gu9)山,其上多玉而无石。神耆(q0)童居 之①,其音常如钟磐(q@ng)②。其下多积蛇。

【注释】①耆童:即老童,传说是上古帝王颛顼的儿子。②磬:古代一种乐器,用美石或玉 石雕制而成。悬挂于架上,用硬物敲击它而发出音响,悦耳动听。

【译文】再往西一百九十里,是座騩山,山上遍布美玉而没有石头。天 神耆童居住在这里,他发出的声音常常像是敲钟击磬的响声。山下到处是一 堆一堆的蛇。

又西三百五十里,曰天山,多金玉,有青、雄黄。英水出焉,而西南流 注于汤谷。有神焉,其状如黄囊(n2ng)①,赤如丹火,六足四翼,浑(h)n) 敦无面目②,是识歌舞,实为帝江(h¥ng)也③。

【注释】①囊:袋子,口袋。②浑敦:用“浑沌”,没有具体的形状。③帝江:即帝鸿氏, 据神话传说也就是黄帝。

【译文】再往西三百五十里,是座天山,山上有丰富的金属矿物和玉石, 也出产石青、雄黄。英水从这座山发源,然后向西南流入汤谷。山里住着一 个神,形貌像黄色口袋,发出的精光红如火,长着六只脚和四只翅膀,浑浑 沌沌没有面目,他却知道唱歌跳舞,原本是帝江。

又西二百九十里,曰泑(y#u)山,神蓐(r))收居之①。其上多婴短 之玉②,其阳多瑾、瑜之玉,其阴多青、雄黄。是山也,西望日之所入,其 气员,神红光之所司也③。

【注释】①蓐收:据古人解说就是金神,长着人面,虎爪子,白色毛皮,拿着■,管理太阳 的降落。②婴短之玉:就是上文羭次山一节中所记述的婴垣之玉。据今人考证,“垣”、“短”可能 都是“脰”之误。而婴脰之玉,就是可制做脖胫饰品的玉石。婴:环绕。脰:颈项。③红光:就是蓐 收。

【译文】再往西二百九十里,是座泑山,天神蓐收居住在这里。山上盛 产一种可用作颈饰的玉石,山南面到处是瑾、瑜一类美玉,而山北面到处是 石青、雄黄。站在这座山上,向西可以望见太阳落山的情景,那种气象浑圆, 由天神红光所主管。

西水行百里,至于翼望之山,无草木,多金玉。有兽焉,其状如狸,一 目而三尾,名曰讙(hu1n),其音如(■)[夺]百声①,是可以御凶,服之 已瘅(d4n)②。有鸟焉,其状如乌,三首六尾而善笑,名曰鵸(y9)■(y*), 服之使人不厌(y3n)③,又可以御凶。

【注释】①夺:竞取,争取。这里是超出,压倒的意思。②瘅:通“疸”,即黄疸病。中医 将此病症分为谷疸、酒疸、黑疸、女劳疸、黄汗五种,认为是由湿热造成的。③厌:通“魇”,梦中 遇可怕的事而呻吟、惊叫。

【译文】往西行一百里水路,便到了翼望山,山上没有花草树木,到处 是金属矿物和玉石。山中有一种野兽,形状像一般的野猫,只长着一只眼睛 却是三条尾巴,名称是讙,发出的声音好像能赛过一百种动物的鸣叫,饲养 它可以辟凶邪之气,人吃了它的肉就能治好黄疸病。山中还有一种禽鸟,形 状像普通的乌鸦,却长着三个脑袋、六条尾巴并且喜欢嘻笑,名称是鵸■, 吃了它的肉就能使人不做恶梦,还可以辟凶邪之气。

凡西次三(经)[山]之首,[自]崇吾之山至于翼望之山,凡二十三山, 六千七百四十四里。其神状皆羊身人面。其祠之礼,用一吉玉瘗(y@)①, 糈(x()用稷(j@)米②。

【注释】①吉玉:带有符彩的玉。②稷:即古代主要食用作物之一的粟,俗称谷子。

【译文】总计西方第三列山系之首尾,从崇吾山起到翼望山止,一共二 十三座山,途经六千七百四十四里。诸山山神的形貌都是羊的身子人的面 孔。祭祀山神的典礼,是把祀神的一块吉玉埋入地下,祀神的米用稷米。

西次四(经)[山]之首,曰阴山,上多(穀)[榖(g^u)],无石,其 草多茆(m3o)、蕃(f1n)①。阴水出焉,西流注于洛。

【注释】①茆:即莼菜,又叫凫葵,多年生水生草本,叶椭圆形,浮生在水面,夏季开花。

嫩叶可供食用。藩:即薠草,像莎草而大一些,生长在江湖水边,大雁吃它。

【译文】西方第四列山系之首座山,叫做阴山,山上生长着茂密的构树, 但没有石头,这里的草以莼菜、蕃草居多。阴水从这座山发源,向西流入洛 水。

北五十里,曰劳山,多茈(z!)草①。弱水出焉,而西流注于洛。

【注释】①茈草:即紫草,可以染紫色。

【译文】往北五十里,是座劳山,这里有茂盛的紫草。弱水从这座山发 源,然后向西流入洛水。

西五十里,曰罢(父)[谷]之山,洱(7r)水出焉,而西流注于洛,其 中多茈(z!)、碧①。

【注释】①茈:紫色。这里指紫色的美石。碧:青绿色。这里指青绿色的玉石。

【译文】往西五十里,是座罢谷山,洱水从这里发源,然后向西流入洛 水,水中多出产紫色美石、碧色玉石。

北百七十里,曰申山,其上多(穀)[榖(gu^)]柞(zu^),其下多杻 (ni()橿(ji1ng),其阳多金玉。区水出焉,而东流注于河。

【译文】往北一百七十里,是座申山,山上是茂密的构树和柞树,山下 是茂密的杻树和僵树,山南面还有丰富的金属矿物和玉石。区水从这座山发 源,然后向东流入黄河。

北二百里,曰鸟山,其上多桑,其下多楮(ch(),其阴多铁,其阳多 玉。辱水出焉,而东流注于河。

【译文】往北二百里,是座鸟山,山上到处是桑树,山下到处是构树, 山北面盛产铁,而山南面盛产玉石。辱水从这座山发源,然后向东流入黄河。

又北二十里,曰上申之山,上无草木,而多硌(lu^)石①,下多榛(zh5n) 楛(h))②,兽多白鹿。其鸟多当扈(h)),其状如雉(zh@)③,以其髯 (r2n)飞④,食之不眴(sh)n)目⑤。汤水出焉,东流注于河。

【注释】①硌:石头很大的样子。②榛:落叶灌木,结的果实叫榛子,近球形,果皮坚硬。

木材可做器物。楛:一种树木,形似荆而赤茎似蓍。木材可以做箭。③雉:俗称野鸡。雄性雉鸟的羽 毛华丽,颈下有一显著白色环纹。雌性雉鸟全身砂褐色,体形较小,尾也较短。善于行走,但不能长 时间飞行。肉可以食用,而尾羽可做装饰品。④髯:脖子咽喉下的须毛。⑤眴目:即瞬目,眨闪眼睛。

【译文】再往北二十里,是座上申山,山上没有花草树木,但到处是大 石头,山上是茂密的榛树和楛树,野兽以白鹿居多。山里最多的禽鸟是当扈 鸟,形状像普通的野鸡,却用髯毛当翅膀来奋起高飞,吃了它的肉就能使人 不眨眼睛。汤水从这座山发源,向东流入黄河。

又北八十里,曰诸次之山,诸次之水出焉,而东流注于河。是山也,多 木无草,鸟兽莫居,是多众蛇。

【译文】再往北八十里,是座诸次山,诸次水从这座山发源,然后向东 流入黄河。这座诸次山,到处生长着树木却不生长花草,也没有禽鸟野兽栖 居,但有许多蛇聚集在山中。

又北百八十里,曰号山,其木多漆、棕①,其草多药、虈(xi1o)芎(xi#ng) (qi¥ng)②。多汵(j9n)石③。端水出焉,而东流注于河。

【注释】①漆:这里指漆树,落叶乔木,从树干中流出的汁液可作涂料用。②药:白芷的别 名,是一种香草,根称白芷,叶子称药,统称为白芷。虈:一种香草。芎:一种香草。生长在四川 地区的叶做川芎,在茎叶还细嫩时称作蘼芜,当叶子长得宽大时称作江蓠。③汵石:一种石质柔软如 泥的石头。

【译文】再往北一百八十里,是座号山,山里的树木大多是漆树、棕树, 而草以白芷草、虈草、芎草居多。山中还盛产汵石。端水从这座山发源, 然后向东流入黄河。

又北二百二十里,曰盂山,其阴多铁,其阳多铜,其兽多白狼白虎,其 鸟多白雉(zh@)白(翟)[翠]。生水出焉,而东流注于河。

【译文】再往北二百二十里,是座盂山,山北面盛产铁,山南面盛产铜, 山中的野兽大多是白色的狼和白色的虎,禽鸟也大多是白色的野鸡和白色的 翠鸟。生水从这座山发源,然后向东流入黄河。

西二百五十里,曰白於之山,上多松柏,下多栎(l@)檀(t2n),其 兽多■(zu¥)牛、羬(xi2n)羊,其鸟多鸮(xi1o)。洛水出于其阳,而 东流注于渭;夹水出于其阴,东流注于生水。

【译文】往西二百五十里,是座白於山,山上是茂密的松树和柏树,山 下是茂密的栎树和檀树,山中的野兽大多是■牛、羬羊,而禽鸟以猫头鹰之 类的居多。洛水发源于这座山的南面,然后向东流入渭水;夹水发源于这座 山的北面,向东流入生水。

西北三百里,曰(申)[由]首之山,无草木,冬夏有雪。申水出于其上, 潜于其下,是多白玉。

【译文】往西北三百里,是座由首山,没有花草树木,而冬季夏季都有 积雪。申水从这座山上发源,潜流到山下,水中有很多白色玉石。

又西五十五里,曰泾谷之山。泾水出焉,东南流注于渭,是多白金白玉。

【译文】再往西五十五里,是座泾谷山。泾水从这座山发源,向东南流 入渭水,这里多出产白银和白玉。

又西百二十里,曰刚山,多柒木①,多■(y()琈之玉。刚水出焉,北 流注于渭。是多神■(ku0)②,其状人面兽身,一足一手,其音如钦③。

【注释】①柒木:漆树。“柒”即“漆”字。②神■:就是魑魅一类的东西,而魑魅是传说 中山泽的鬼怪。③钦:“吟”字的假借音,用呻吟之意。

【译文】再往西一百二十里,是座刚山,到处是茂密的漆树,多出产■ 琈玉。刚水从这座山发源,向北流入渭水。这里有很多神■,形状是人的面 孔野兽的身子,长着一只脚一只手,发出的声音像人呻吟。

又西二百里,至刚山之尾。洛水出焉,而北流注于河。其中多蛮蛮①, 其状鼠身而鳖首,其音如吠(f6i)犬。

【注释】①蛮蛮:属于水獭之类的动物,与上文的蛮蛮鸟同名而异物。

【译文】再往西二百里,便到了刚山的尾端。洛水就发源于此,然后向 北流入黄河。这里有很多的蛮蛮兽,形状像普通的老鼠却长着甲鱼的脑袋, 发出的声音如同狗叫。

又西三百五十里,曰英鞮(d9)之山,上多漆木,下多金玉,鸟兽尽白。

涴水出焉,而北流注于陵羊之泽。是多冉(r n)遗之鱼,鱼身蛇首六足, 其目如马耳,食之使人不眯(m@)①,可以御凶。

【注释】①眯:梦魇。

【译文】再往西三百五十里,是座英鞮山,山上生长着茂密的漆树,山 下蕴藏着丰富的金属矿物和玉石,禽鸟野兽都是白色的。涴水从这座山发 源,然后向北流入陵羊泽。水里有很多冉遗鱼,长着鱼的身子蛇的头和六只 脚,眼睛长长的像马耳朵,吃了它的肉就能使人睡觉不做恶梦,也可以辟凶 邪之气。

又西三百里,曰中曲之山,其阳多玉,其阴多雄黄、白玉及金。有兽焉, 其状如马而白身黑尾,一角,虎牙爪,音如鼓音,其名曰駮(b¥),是食虎 豹,可以御兵。有木焉,其状如棠,而员叶赤实,实大如木瓜①,名曰櫰(gu9) 木,食之多力。

【注释】 ①木瓜:木瓜树所结的果子。这种果树也叫楙(m4o)树,落叶灌木或乔木,果实 在秋季成熟,椭圆形,有香气,可以吃,也可入药。

【译文】 再往西三百里,是座中曲山,山南阳面盛产玉石,山北阴面 盛产雄黄、白玉和金属矿物。山中有一种野兽,形状像普通的马却长着白身 子和黑尾巴,一只角,老虎的牙齿和爪子,发出的声音如同击鼓的响声,名 称是駮,是能吃老虎和豹子的,饲养它可以辟兵器。山中还有一种树木,形 状像棠梨,但叶子是圆的并结红色的果实,果实像木瓜大小,名称是櫰木, 人吃了它就能增添气力。

又西二百六十里,曰邽(gu9)山。其上有兽焉,其状如牛,猬毛,名 曰穷奇,音如嗥(h2o)狗①,是食人。濛水出焉,南流注于洋水,其中多 黄贝②;蠃(lu¥)鱼,鱼身而鸟翼,音如鸳鸯,见(xi4n)则其邑大水。

【注释】①嗥:野兽吼叫。②黄贝:据古人说是一种甲虫,肉如蝌蚪,但有头也有尾巴。

【译文】再往西二百六十里,是座邽山。山上有一种野兽,形状像一般 的牛,但全身长着刺猬毛,名称是穷奇,发出的声音如同狗叫,是能吃人的。

濛水从这座山发源,向南流入洋水,水中有很多黄贝;还有一种蠃鱼,长着 鱼的身子却有鸟的翅膀,发出的声音像鸳鸯鸟鸣叫,在哪个地方出现那里就 会有水灾。

又西二百二十里,曰鸟鼠同穴之山①,其上多白虎、白玉。渭水出焉, 而东流注于河,其中多鳋(s1o)鱼,其状如鳣(zh1n)鱼②,动则其邑有 大兵。滥(ji4n)水出于其西,西流注于汉水,多■(r*)魮(p0)之鱼, 其状如覆■(di4o)③,鸟首而鱼翼鱼尾,音如磬(q@ng)石之声,是生珠 玉。

【注释】①鸟鼠同穴山:据古人讲,这座山上有一种叫做■的鸟,长得像燕子,而羽毛是黄 色的;又有一种叫做鼵的鼠,和一般的家鼠相似,但尾巴较短。它们穿地几尺深,鼠在洞穴里住,鸟 在洞穴外住,和平相处。 ②鳣鱼:一种形体较大的鱼,大的有二、三丈长,嘴长在颔下,身体上面 有甲,无鳞,肉是黄色的。③铫:即吊子,一种有把柄有流嘴的小型烹器。

【译文】再往西二百二十里,是座鸟鼠同穴山,山上有很多白色的虎、 洁白的玉。渭水从这座山发源,然后向东流入黄河,水中生长着许多鳋鱼, 形状像一般的鳣鱼,在哪个地方出没那里就会有大战发生。滥水从鸟鼠同穴 山的西面发源,向西流入汉水,水中有很多■魮鱼,形状像反转过来的铫, 但长着鸟的脑袋而鱼一样的鳍和尾巴,叫声就像敲击磬石发出的响声,是能 吐出珠玉的。

西南三百六十里,曰崦(y1n)嵫(z9)之山①,其上多丹木,其叶如 (穀)

\chapter{北山经}

山海经卷三 北山经

大``学"生:小..说 网
北山(经)之首,曰单狐之山,多机木①,其上多华草②。漨(f5ng) 水出焉,而西流注于泑(y#u)水,其中多(芘)[茈(z!)]石、文石③。

【注释】①机木:即桤(q9)木树,长得像榆树,把枝叶烧成灰撒在稻田中可作肥料用。② 华草:不详何草。③茈石:紫颜色的漂亮石头。文石:有纹理的漂亮石头。

【译文】北方第一列山系之首座山,叫做单狐山,有茂密的桤木树,也 有茂盛的华草。漨水从这座山发源,然后向西流入泑水,水中有很多紫石、 文石。

又北二百五十里,曰求如之山,其上多铜,其下多玉,无草木。滑水出 焉,而西流注于诸■(p0)之水。其中多滑鱼,其状如■(sh4n)①,赤背, 其音如梧②,食之已疣③。其中多水马,其状如马,文臂牛尾,其音如呼。

【注释】①■:即鳝鱼。俗称黄鳝,体形如蛇,又长又圆又光滑,肉味鲜美。②梧:枝梧, 也作“支吾”,用含混的言语搪塞。③疣:皮肤上的赘生物,俗称瘊子。

【译文】再往北二百五十里,是座求如山,山上蕴藏着丰富的铜,山下 有丰富的玉石,但没有花草树木。滑水从这座山发源,然后向西流入诸■水。

水中有很多滑鱼,形状像一般的鳝鱼,却是红色的脊背,发出的声音像人支 支吾吾的话语,吃了它的肉就能治好人的疣赘病。水中还生长着很多水马, 形状与一般的马相似,但前腿上长有花纹,并拖着一条牛尾巴,发出的声音 像人呼喊。

又北二百里,曰带山,其上多玉,其下多青碧。有兽焉,其状如马,一 角有错①,其名曰■(hu1n)疏,可以辟火。有鸟焉,其状如乌,五采而赤 文,名曰鵸(y9)■(y*),是自为牝(p@n)牡(m(),食之不疽(j&)。

彭水出焉,而西流注于芘湖之水,其中多儵(y¥u)鱼,其状如鸡而赤毛, 三尾六足四(首)目,其音如鹊,食之可以已忧。

【注释】①错:“厝”(cu^)的假借字。厝:磨刀石。

【译文】再往北三百里,是座带山,山上盛产玉石,山下盛产青石碧玉。

山中有一种野兽,形状像普通的马,长的一只角有如粗硬的磨石,名称是■ 疏,人饲养它可以辟火。山中还有一种禽鸟,形状像普通的乌鸦,但浑身是 带着红色斑纹的五彩羽毛,名称是鵸■,这种鵸■鸟自身有雌雄二种性器 官,吃了它的肉就能使人不患痈疽病。彭水从这座山发源,然后向西流入芘 湖水,水中有很多儵鱼,形状像一般的鸡却长着红色的羽毛,还长着三条尾 巴、六只脚、四只眼睛,它的叫声与喜鹊的鸣叫相似,吃了它的肉就能使人 无忧无虑。

又北四百里,曰谯明之山。谯水出焉,西流注于河。其中多何罗之鱼, 一首而十身,其音如吠(f6i)犬,食之已痈。有兽焉,其状如貆(hu2n) 而赤毫①,其音如(榴榴),[■■(ch#uch#u)]②,名曰孟槐,可以御 凶。是山也,无草木,多青、雄黄。

【注释】①狟:豪猪。毫:细毛,②■:同“抽”。引出,提取。

【译文】再往北四百里,是座谯明山。谯明水从这座山发源,向西流入 黄河。水中生长着很多何罗鱼,长着一个脑袋却有十个身子,发出的声音像 狗叫,人吃了它的肉就可以治愈痈肿病。山中有一种兽,形状像豪猪却长着 柔软的红毛,叫声如同用辘轳抽水的响声,名称是孟槐,人饲养它可以辟凶 邪之气。这座谯明山,没有花草树木,到处是石青、雄黄。

又北三百五十里,曰涿光之山。嚻(xi1o)水出焉,而西流注于河。其 中多鰼鰼(x0x0)之鱼,其状鹊而十翼,鳞皆在羽端,其音如鹊,可以御火, 食之不瘅。其上多松柏,其下多棕橿(ji1ng),其兽多麢(10ng)羊,其 鸟多蕃。

【注释】①蕃:不详何鸟。也有认为可能是猫头鹰之类的鸟。

【译文】再往北三百五十里,是座涿光山。嚻水从这座山发源,然后向 西流入黄河。水中生长着很多鳛鳛鱼,形状像一般的喜鹊却长有十只翅膀, 鳞甲全长在羽翅的尖端,发出的声音与喜鹊的鸣叫相似,人饲养它可以辟 火,吃了它的肉就能治好人的黄疸病。山上到处是松树和柏树,而山下到处 是棕树和橿树,山中的野兽以羚羊居多,禽鸟以蕃鸟居多。

又北三百八十里,曰虢(gu¥)山,其上多漆,其下多桐椐(q&)①。

其阳多玉,其阴多铁。伊水出焉,西流注于河。其兽多橐(tu¥)驼②,其 鸟多寓③,状如鼠而鸟翼,其音如羊,可以御兵④。

【注释】①椐:椐树,也就是灵寿木,树干上多长着肿节,古人常用来制做拐杖。②橐驼: 就是骆驼,身上有肉鞍,善于在沙漠中行走,知道水泉所在的地方,背负千斤重物而日行三百里。③ 寓:即蝙蝠之类的小飞禽。④御兵:即辟兵。兵在这里是指各种兵器的锋刃。辟兵就是兵器的尖锋利 刃不能伤及身子。

【译文】再往北三百八十里,是座虢山,山上是茂密的漆树,山下是茂 密的梧桐树和椐树,山南阳面盛产玉石,山北阴面盛产铁。伊水从这座山发 源,向西流入黄河。山中的野兽以橐驼最多,而禽鸟大多是寓鸟,形状与一 般的老鼠相似却长着鸟一样的翅膀,发出的声音像羊叫,人饲养它可以辟兵 器。

又北四百里,至于虢山之尾,其上多玉而无石。鱼水出焉,西流注于河, 其中多文贝。

【译文】再往北四百里,便到了虢山的尾端,山上到处是美玉而没有石 头。鱼水从这里发源,向西流入黄河,水中有很多花纹斑斓的贝。

又北二百里,曰丹熏之山,其上多樗(ch&)柏,其草多韭■(xi6)①, 多丹雘(hu^)。熏水出焉,而西流注于棠水。有兽焉,其状如鼠,而菟(t)) 首麋(身)[耳]②,其音如嗥(h2o)犬,以其尾飞,名曰耳鼠,食之不■ (c3i)③,又可以御百毒④。

【注释】①■:同“薤”,也叫蕌头,一种野菜,茎可食用,并能入药。②菟:通“兔”。

③■:臌胀。④百:这里表示多的意思,非实指。

【译文】再往北二百里,是座丹熏山,山上有茂密的臭椿树和柏树,在 众草中以野韭菜和野薤菜最多,还盛产丹雘。熏水从这座山发源,然后向西 流入棠水。山中有一种野兽,形状像一般的老鼠,却长着兔子的脑袋和麋鹿 的耳朵,发出的声音如同狗嗥叫,用尾巴飞行,名称是耳鼠,人吃了它的肉 就不会生膨胀病,还可以辟百毒之害。

又北二百八十里,曰石者之山,其上无草木,多瑶、碧。泚水出焉,西 流注于河。有兽焉,其状如豹,而文题白身①,名曰孟极,是善伏,其鸣自 呼。

【注释】①文:花纹。这里指野兽的皮毛因多种颜色相间杂而呈现出的斑纹或斑点。题:额 头。

【译文】再往北二百八十里,是座石者山,山上没有花草树木,但到处 是瑶、碧之类的美玉。泚水从这座山发源,向西流入黄河。山中有一种野兽, 形状像普通的豹子,却长着花额头和白身子,名称是孟极,善于伏身隐藏, 它叫的声音便是自身名称的读音。

又北百一十里,曰边春之山,多葱、葵、韭、桃、李①。杠水出焉,而 西流注于泑(y#u)泽。有兽焉,其状如禺(y))而文身,善笑,见人则卧, 名曰幽鴳(6),其鸣自呼。

【注释】①葱:山葱,又叫茖葱,一种野菜。茎生有枝格,一边拔取一边又生长起来,食之 不尽。冬天也不枯萎。桃:山桃,又叫榹(s9)桃,也叫毛桃,一种野果木。果子很小,核与果肉粘 结一起,桃仁多脂,可入药。

【译文】再往北一百一十里,是座边春山,山上到处是野葱、葵菜、韭 菜、野桃树、李树。杠水从这座山发源,然后向西流入泑泽。山中有一种野 兽,形状像猿猴而身上满是花纹,喜欢嘻笑,一看见人就假装睡着,名称是 幽鴳,它叫的声音便是自身名称的读音。

又北二百里,曰蔓联之山,其上无草木。有兽焉,其状如禺(y))而有 鬣(li6),牛尾、文臂、马蹄,见人则呼,名曰足訾(z!),其鸣自呼。

有鸟焉,群居而朋飞,其(毛)[尾]如雌雉(zh@),名曰鵁(ji1o),其 鸣自呼,食之已风。

【译文】再往北二百里,是座蔓联山,山上没有花草树木。山中有一种 野兽,形状像猿猴却长着鬣毛,还有牛一样的尾巴、长满花纹的双臂、马一 样的蹄子,一看见人就呼叫,名称是足訾,它叫的声音便是自身名称的读音。

山中又有一种禽鸟,喜欢成群栖息而又结队飞行,尾巴与雌野鸡相似,名称 是鵁。它叫的声音便是自身名称的读音,人吃了它的肉就能治好风痹病。

又北百八十里,曰单张之山,其上无草木。有兽焉,其状如豹而长尾, 人首而牛耳,一目,名曰诸犍,善咤(zh4)①,行则衔其尾,居则蟠(p2n) 其尾②。有鸟焉,其状如雉(zh@),而文首、白翼、黄足,名曰白■(y6), 食之已嗌(y@)痛③,可以已痸(zh@)④。栎(l@)水出焉,而南流注于 杠水。

【注释】①咤:怒声。这里是大声吼叫的意思。②蟠:盘曲而伏。③嗌:咽喉。④痸:痴病, 疯癫病。

【译文】再往北一百八十里,是座单张山,山上没有花草树木。山中有 一种野兽,形状像豹子却拖着一条长长的尾巴,还长着人一样的脑袋和牛一 样的耳朵,一只眼睛,名称是诸犍,喜欢吼叫,行走时就用嘴衔着尾巴,卧 睡时就将尾巴盘蜷起来。山中又有一种禽鸟,形状像普通的野鸡,却长着花 纹脑袋、白色翅膀、黄色脚,名称是白■,人吃了它的肉就能治好咽喉疼痛 的病,还可以治愈疯癫病。栎水从这座山发源,然后向南流入杠水。

又北三百二十里,曰灌题之山,其上多樗(ch&)柘①(zh6),其下多 流沙,多砥(d!)。有兽焉,其状如牛而白尾,其音如訆(ji4o)②,名曰 那父。有鸟焉,其状如雌雉(zh@)而人面,见人则跃,名曰竦(s))斯, 其鸣自呼也。匠韩之水出焉,而西流注于泑(y#u)泽,其中多磁石③。

【注释】①柘:柘树,也叫黄桑,奴柘。落叶灌木,叶子可以喂蚕,果子可以食用,树皮可 以造纸。②訆:同“叫”。大呼。③磁石:也作“慈石”,一种天然矿石,具有吸引铁、镍、钴等金 属物质的属性。俗称吸铁石,今称磁铁石。中国古代四大发明之一的指南针,就是利用磁石制做成的。

【译文】再往北三百二十里,是座灌题山,山上是茂密的臭椿树和柘树, 山下到处是流沙,还多出产磨石。山中有一种野兽,形状像普通的牛却拖着 一条白色的尾巴,发出的声音如同人在高声呼唤,名称是那父。山中还有一 种禽鸟,形状像一般的雌野鸡却长着人的面孔,一看见人就跳跃,名称是竦 斯,它叫的声音便是自身名称的读音。匠韩水从这座山发源,然后向西流入 泑泽,水中有很多磁铁石。

又北二百里,曰潘侯之山,其上多松柏,其下多榛(zh5n)楛(h)), 其阳多玉,其阴多铁。有兽焉,其状如牛,而四节生毛,名曰旄(m2o)牛。

边水出焉,而南流注于栎(1@)泽。

【译文】再往北二百里,是座潘侯山,山上是茂密的松树和柏树,山下 是茂密的榛树和楛树,山南阳面蕴藏着丰富的玉石,山北阴面蕴藏着丰富的 铁。山中有一种野兽,形状像一般的牛,但四肢关节上都有长长的毛,名称 是牦牛。边水从这座山发源,然后向南流入栎泽。

又北二百三十里,曰小咸之山,无草木,冬夏有雪。

【译文】再往北二百三十里,是座小咸山,没有花草树木,67 冬天和 夏天都有积雪。

北二百八十里,曰大咸之山,无草木,其下多玉。是山也,四方,不可 以上。有蛇名曰长蛇①,其毛如彘(zh@)豪,其音如鼓柝(tu#)②。

【注释】①长蛇:传说有几十丈长,能把鹿、象等动物吞入腹中。②鼓:击物作声。柝:是 古代巡夜人在报时间时所敲击的一种木梆子。

【译文】往北二百八十里,是座大咸山,没有花草树木,山下盛产玉石。

这座大咸山,呈现四方形,人不能攀登上去。山中有一种蛇叫做长蛇,身上 的毛与猪脖子上的硬毛相似,发出的声音像是人在敲击木梆子。

又北三百二十里,曰敦薨(h#ng)之山,其上多棕、枏(n2n),其下 多茈(z!)草。敦薨之水出焉,而西流注于泑(y#u)泽。出于昆仑之东北 隅,实惟河原。其中多赤鲑(gu9)①。其兽多兕(s@)、旄牛,其鸟多(鸤) [尸]鸠②。

【注释】①赤鲑:身体呈流线型,有小圆鳞,口大而斜,锥状牙齿,是一种冷水性的经济鱼 类。②尸鸠:就是布谷鸟。

【译文】再往北三百二十里,是座敦薨山,山上是茂密的棕树和楠木树, 山下是大片的紫草。敦薨水从这座山发源,然后向西流入泑泽。这泑泽位于 昆仑山的东北角,确实就是黄河的源头。水中有很多赤鲑。那里的野兽以兕、 牦牛最多,而禽鸟大多是布谷鸟。

又北二百里,曰少咸之山,无草木,多青碧。有兽焉,其状如牛,而赤 身、人面、马足,名曰窫(zh2)窳(y)),其音如婴儿,是食人。敦水出 焉,东流注于雁门之水,其中多■■(p6ip6i)之鱼①,食之杀人。

【注释】 ①■■:据古人说就是■鱼,又叫江豚,黑色,大小如同一百斤重的猪。

【译文】再往北二百里,是座少咸山,山上没有花草树木,到处是青石 碧玉。山中有一种野兽,形状像普通的牛,却长着红色的身子、人的面孔、 马的蹄子,名称是窫窳,发出的声音如同婴儿啼哭,是能吃人的。敦水从这 座山发源,向东流入雁门水,水中生长着很多■■鱼,人吃了它的肉就会中 毒而死。

又北二百里,曰狱法之山。瀤(hu2i)泽之出焉,而东北流注于泰泽。

其中多■(z3o)鱼,其状如鲤而鸡足,食之已疣。有兽焉,其状如犬而人 面,善投,见人则笑,其名[曰]山■(h*n),其行如风,见(xi4n)则天 下大风。

【译文】再往北二百里,是座狱法山。瀤泽水从这座山发源,然后向东 北流入泰泽。水中生长着很多■鱼,形状像一般的鲤鱼却长着鸡爪子,人吃 了它的肉就能治好赘瘤病。山中还有一种野兽,形状像普通的狗却长着人的 面孔,擅长投掷,一看见人就嘻笑,名称是山■,它走起来就像刮风,一出 现天下就会起大风。69 又北二百里,曰北岳之山,多枳(zh!)棘(j0)刚木①。有兽焉,其 状如牛,而四角、人目、彘(zh@)耳,其名曰诸怀,其音如鸣雁,是食人。

诸怀之水出焉,而西流注于嚻(xi1o)水,其中多鮨(y@)鱼,鱼身而犬首, 其音如婴儿,食之已狂②。

【注释】①枳棘:枳木和棘木,两种矮小的树。枳木像橘树而小一些,叶子上长满刺。春天 开白花,秋天成果实,果子小而味道酸,不能吃,可入药。棘木就是丛生的小枣树,即酸枣树,枝叶 上长满了刺。刚木:指木质坚硬的树,即檀木材、柘树之类。②狂:本义是说狗发疯。后来也指人的 神经错乱,精神失常。

【译文】再往北二百里,是座北岳山,山上到处是枳树酸枣树和檀、柘 一类的树木。山中有一种野兽,形状像一般的牛,却长着四只角、人的眼睛、 猪的耳朵,名称是诸怀,发出的声音如同大雁鸣叫,是能吃人的。诸怀水从 这座山发源,然后向西流入嚻水,水中有很多鮨鱼,长着鱼的身子而狗的脑 袋,发出的声音像婴儿啼哭,人吃了它的肉就能治愈疯狂病。

又北百八十里,曰浑夕之山,无草木,多铜玉。嚻(xi1o)水出焉,而 西北流注于海。有蛇一首两身,名曰肥遗,见(xi4n)则其国大旱。

【译文】再往北一百八十里,是座浑夕山,山上没有花草树木,盛产铜 和玉石。嚻水从这座山发源,然后向西北流入大海。这里有一种长着一个头 两个身子的蛇,名称是肥遗,在哪个国家出现那个国家就会发生大旱灾。

又北五十里,曰北单之山,无草木,多葱韭。

【译文】再往北五十里,是座北单山,山上没有花草树木,却生长着茂 盛的野葱和野韭菜。

又北百里,曰罴(p0)差之山,无草木,多马①。

【注释】①马:指一种野马,与一般的马相似而个头小一些。

【译文】再往北一百里,是座罴差山,没有花草树木,却有很多小个头 的野马。

又北百八十里,曰北鲜之山,是多马。鲜水出焉,而西北流注于涂(t)) 吾之水。

【译文】再往北一百八十里,是座北鲜山,这里有很多小个头的野马。

鲜水从这里发源,然后向西北流入涂吾水。

又北百七十里,曰隄(t0)山,多马。有兽焉,其状如豹而文首,名曰 狕(y1o)。隄水出焉,而东流注于泰泽,其中多龙龟①。

【注释】①龙龟:也有把龙龟看作是一种动物的,即龙种龟身的吉吊。

【译文】再往北一百七十里,是座隄山,有许多小个头的野马。山中有 一种野兽,形状像一般的豹子而脑袋上有花纹,名称是狕。隄水从这座山发 源,然后向东流入泰泽,水中有很多龙和龟。71 凡北山(经)之首,自单狐之山至于隄山,凡二十五山,五千四百九十 里,其神皆人面蛇身。其祠之:毛用一雄鸡彘(zh@)瘗(y@),吉玉用一 珪,瘗而不糈(x))。其山北人,皆生食不火之物。

【译文】总计北方第一列山系之首尾,自单狐山起到隄山止,一共二十 五座山,途经五千四百九十里,诸山山神都是人的面孔蛇的身子。祭祀山神: 把毛物中用作祭品的一只公鸡和一头猪埋入地下,在祀神的美好玉器中用一 块玉珪,只是埋入地下而不需要用米来祭祀。住在诸山北面的人,都生吃未 经火烤的食物。

北次二(经)[山]之首,在河之东,其首枕汾, 其名曰管涔(c6n) 之山。其上无木而多草,其下多玉。汾水出焉,而西流注于河。

【译文】北方第二列山系之首座山,座落在黄河的东岸,山的首端枕着 汾水,这座山叫管涔山。山上没有树木却到处是茂密的花草,山下盛产玉石。

汾水从这座山发源,然后向西流入黄河。

又(西)[北]二百五十里,曰少阳之山,其上多玉,其下多赤银①。酸 水出焉,而东流注于汾水,其中多美赭(zh7)②。

【注释】①赤银:最精最纯的银子。这里指天然含银量很高的优质银矿石。②赭:即赭石, 一种红土中含着铁质的矿物。

【译文】再往北二百五十里,是座少阳山,山上盛产玉石,山下盛产赤 银。酸水从这座山发源,然后向东流入汾水,水中有很多优良赭石。

又北五十里,曰县雍之山,其上多玉,其下多铜,其兽多闾(lǘ)麋①, 其鸟多白翟(d@)白■(y¥u)②。晋水出焉,而东南流注于汾水。其中多 鮆(z9)鱼,其状如儵(y¥u)而赤(麟)[鳞]③,其音如叱(ch@)④,食 之不(骄)[骚]。

【注释】①闾:据古人讲,是一种黑母羊,形体似驴而蹄子歧分,角如同羚羊的角,也叫山 驴。②白■:据古人讲,就是前面已说过的白翰鸟。③儵:通“鯈”,这里指的是小鱼。④叱:大声 呵斥。

【译文】再往北五十里,是座县雍山,山上蕴藏着丰富的玉石,山下蕴 藏着丰富的铜,山中的野兽大多是山驴和麋鹿;而禽鸟以白色野鸡和白翰鸟 居多。晋水从这座山发源,然后向东南流入汾水。水中生长着很多鱽鱼,形 状像小儵鱼却长着红色的鳞甲,发出的声音如同人的斥责声,吃了它的肉就 使人没有狐骚臭。

又北二百里,曰狐岐之山,无草木,多青碧。胜水出焉,而东北流注于 汾水,其中多苍玉。

【译文】再往北二百里,是座狐岐山,山上没有花草树木,到处是青石 碧玉。胜水从这座山发源,然后向东北流入汾水,水中有很多苍玉。73 又北三百五十里,曰白沙山,广员三百里,尽沙也,无草木鸟兽。鲔(w7i) 水出于其上,潜于其下,是多白玉。

【译文】再往北三百五十里,是座白沙山,方圆三百里大小,到处是沙 子,没有花草树木和禽鸟野兽。鲔水从这座山的山顶发源,然后潜流到山下, 水中有很多白玉。

又北四百里,曰尔是之山,无草木,无水。

【译文】再往北四百里,是座尔是山,没有花草树木,也没有水。

又北三百八十里,曰狂山,无草木。是山也,冬夏有雪。狂水出焉,而 西流注于浮水,其中多美玉。

【译文】再往北三百八十里,是座狂山,没有花草树木。这座狂山,冬 天和夏天都有雪。狂水从这座山发源,然后向西流入浮水,水中有很多优良 玉石。

又北三百八十里,曰诸余之山,其上多铜玉,其下多松柏。诸余之水出 焉,而东流注于旄(m2o)水。

【译文】再往北三百八十里,是座诸余山,山上蕴藏着丰富的铜和玉石, 山下到处是茂密的松树和柏树。诸余水从这座山发源,然后向东流入旄水。

又北三百五十里,曰敦头之山,其上多金玉,无草木。旄(m2o)水出 焉,而东流注于(印)[邛(qi¥ng)]泽。其中多■(b¥)马,牛尾而白身, 一角,其音如呼。

【译文】再往北三百五十里,是座敦头山,山上有丰富的金属矿物和玉 石,但不生长花草树木。旄水从这座山发源,然后向东流入邛泽。山中有很 多■马,长着牛一样的尾巴和白色身子,一只角,发出的声音如同人呼唤。

又北三百五十里,曰鉤吾之山,其上多玉,其下多铜。有兽焉,其状(如) 羊身人面,其目在腋下,虎齿人爪,其音如婴儿,名曰狍(p2o)鸮(xi1o) ①,是食人。

【注释】①狍鸮:传说中的一种怪兽,非常贪婪,不但吃人,而且在吃不完时,还要把人身 的各个部位咬碎。

【译文】再往北三百五十里,是座鉤吾山,山上盛产玉石,山下盛产铜。

山中有一种野兽,形状是羊的身子人的面孔。眼睛长在腋窝下,有着老虎一 样的牙齿和人一样的指甲,发出的声音如同婴儿哭啼,名称是狍鸮,是能吃 人的。

又北三百里,曰北嚻(xi1o)之山,无石,其阳多碧,其阴多玉。有兽 焉,其状如虎,而白身犬首,马尾彘(zh@)鬣(li6),名曰独■(g*)。

有鸟焉,其状如乌,人面,名曰■(b4n)■(m4o),宵飞而昼伏,食之已 暍(y5)①。涔(c6n)水出焉,而东流注于邛(qi¥ng)泽。

【注释】①暍:中暑,受暴热。

【译文】再往北三百里,是座北嚻山,没有石头,山南阳面多出产碧玉, 山北阴面多出产玉石。山中有一种野兽,形状像一般的老虎,却长着白色身 子狗脑袋,马的尾巴猪脖子上的硬毛,名称是独■。山中还有一种禽鸟,形 状像一般的乌鸦,却长着人的面孔,名称是■■,在夜里飞行而在白天隐伏, 吃了它的肉就能使人不中暑。涔水从这座山发源,然后向东流入邛泽。

又北三百五十里,曰梁渠之山,无草木,多金玉。脩水出焉,而东流注 于雁门。其兽多居暨(j9),其状如彙(w6i)而赤毛①,其音如豚(t*n)。

有鸟焉,其状如夸父②,四翼、一目、犬尾,名曰嚻(xi1o),其音如鹊, 食之已腹痛,可以止衕(d^ng)③。

【注释】①彙:据古人讲,这种动物长得像老鼠,红色的毛硬得像刺猬身上的刺。②夸父: 即前文所说的举父,一种长得像猕猴的野兽。③衕:腹泻。

【译文】再往北三百五十里,是座梁渠山,不生长花草树木,有丰富的 金属矿物和玉石,脩水从这座山发源,然后向东流入雁门水。山中的野兽大 多是居暨兽,形状像彙却浑身长着红色的毛,发出的声音如同小猪叫。山中 还有一种禽鸟,形状像夸父,长着四只翅膀、一只眼睛、狗一样的尾巴,名 称是嚻,它的叫声与喜鹊的鸣叫相似,人吃了它的肉就可以止住肚子痛,还 可以治好腹泻病。

又北四百里,曰姑灌之山,无草木。是山也,冬夏有雪。

【译文】再往北四百里,是座姑灌山,没有花草树木。在这座姑灌山上, 冬天夏天都有雪。

又北三百八十里,曰湖灌之山,其阳多玉,其阴多碧、多马。湖灌之水 出焉,而东流注于海,其中多■(sh4n)①。有木焉,其叶如柳而赤理。

【注释】①■:同“鱓”。即黄鳝。

【译文】再往北三百八十里,是座湖灌山,山南阳面盛产玉石,山北阴 面盛产碧玉,并有许多个头小的野马。湖灌水从这座山发源,然后向东流入 大海,水中有很多鳝鱼。山里生长着一种树木,叶子像柳树叶而有红色的纹 理。

又北水行五百里,流沙三百里,至于洹(hu2n)山,其上多金玉。三桑 生之,其树皆无枝,其高百仞①,百果树生之。其下多怪蛇。

【注释】①仞:古代的八尺为一仞。

【译文】再往北行五百里水路,然后经过三百里流沙,便到了洹山,山 上蕴藏着丰富的金属矿物和玉石。山中生长着一种三桑树,这种树都不长枝 条,树干高达一百仞,还生长着各种果树。山下有很多怪蛇。

又北三百里,曰敦题之山,无草木,多金玉。是錞(ch*n)于北海①。

【注释】①錞:依附。这里是座落、高踞的意思。

【译文】再往北三百里,是座敦题山,这里不长花草树木,但蕴藏有丰 富的金矿物和玉石。这座山座落在北海的岸边。

凡北次二(经)[山]之首,自管涔(c6n)之山至于敦题之山,凡十七 山,五千六百九十里。其神皆蛇身人面。其祠:毛用一雄鸡、彘(zh@)瘗 (y@);用一璧一珪,投而不糈(x()。

【译文】总计北方第二列山系之首尾,自管涔山起到敦题山止,一共十 七座山,途经五千六百九十里。诸山山神都是蛇的身子人的面孔。祭祀山神: 把毛物中用作祭品的一只公鸡、一头猪一起埋入地下;在祀神的玉器中用一 块玉璧和一块玉珪,一起投向山中而不用米祀神。

北次三(经)[山]之首,曰太行之山。其首曰归山,其上有金玉,其下 有碧。有兽焉,其状如麢(10ng)羊而四角,马尾而有距①,其名曰■(hu9), 善还(xu2n)②,其名自訆(ji4o)。有鸟焉,其状如鹊,白身、赤尾、六 足,其名曰■(b5n),是善惊,其鸣自詨(ji4o)③。

【注释】①距:雄鸡、野鸡等跖后面突出像脚趾的部分。这里指鸡爪子。②还:通“旋”。

旋转。③詨:叫,呼。

【译文】北方第三列山系之首座山,叫做太行山。太行山的首端叫归山, 山上出产金属矿物和玉石,山下出产碧玉。山中有一种野兽,形状像普通的 羚羊却有四只角,长着马一样的尾巴和鸡一样的爪子,名称是■,善于旋转 起舞,它发出的叫声就是自身名称的读音。山中还有一种禽鸟,形状像一般 的喜鹊,长着白身子、红尾巴、六只脚,名称是■,这种■鸟十分惊觉,它 发出的叫声就是自身名称的读音。

又东北二百里,曰龙侯之山,无草木,多金玉。決(決)之水出焉,而 东流注于河。其中多人鱼,其状如■(t0)鱼,四足,其音如婴儿,食之无 痴(ch9)疾。

【译文】再往东北二百里,是座龙侯山,不生长花草树木,有丰富的金 属矿物和玉石。決水从这座山发源,然后向东流入黄河。水中有很多人鱼, 形状像一般的■鱼,长有四只脚,发出的声音像婴儿哭啼,吃了它的肉就能 使人不得疯癫病。

又东北二百里,曰马成之山,其上多文石,其阴多金玉。有兽焉,其状 如白犬而黑头,见人则飞,其名曰天马,其鸣自訆(ji4o)。有鸟焉,其状 如乌,首白而身青、足黄,是名曰鶌(q))鶋(j&),其鸣自詨(ji4o), 食之不饥,可以已寓①。

【注释】①寓:古人认为寓即“误”字,大概以音近为义,指昏忘之病,就是现在所谓的老 年健忘症,或老年痴呆症。也有另一种意见认为寓当是“■”字的假借,指疣病,就是中医学上所谓 的千日疮,是因病毒感染而在皮肤上生出小疙瘩。

【译文】再往东北二百里,是座马成山,山上多出产有纹理的美石,山 北阴面有丰富的金属矿物和玉石。山里有一种野兽,形状像普通的白狗却长 着黑脑袋,一看见人就腾空飞起,名称是天马,它的叫声就是自身名称的读 音。山里还有一种禽鸟,形状像一般的乌鸦,却长着白色的脑袋和青色的身 子、黄色的足爪,名称是鶌鶋,它的叫声便是自身名称的读音,吃了它的肉 使人不感觉饥饿,还可以医治老年健忘症。

又东北七十里,曰咸山,其上有玉,其下多铜,是多松柏,草多茈(z!) 草。条菅(ji1n)之水出焉,而西南流注于长泽。其中多器酸①,三岁一成, 食之已疠。

【注释】 ①器酸:据古人讲,大概是一种可以吃而有酸味的东西,就像山西解州盐池所生 产的盐之类的东西。因为泽水静止而不流动,积的时间长了,就形成一种酸味的物质。

【译文】再往东北七十里,是座咸山,山上盛产玉石,山下盛产铜,这 里到处是松树和柏树,在所生长的草中以紫草最多。条菅水从这座山发源, 然后向西南流入长泽。水中多出产器酸,这种器酸三年才能收成一次,吃了 它就能治愈人的麻疯病。

又东北二百里,曰天池之山,其上无草木,多文石。有兽焉,其状如兔 而鼠首,以其背飞,其名曰飞鼠。渑(sh6ng)水出焉,潜于其下,其中多 黄垩(6)。

【译文】再往东北二百里,是座天池山,山上没有花草树木,到处是带 有花纹的美石。山中有一种野兽,形状像一般的兔子却长着老鼠的头,借助 它背上的毛飞行,名称是飞鼠。渑水从这座山发源,然后潜流到山下,水中 有很多黄色垩土。

又东三百里,曰阳山,其上多玉,其下多金铜。有兽焉,其状如牛而赤 尾,其颈■(sh6n)①,其状如句(g#u)瞿②,其名曰领胡,其鸣自詨(ji4o), 食之已狂。有鸟焉,其状如雌雉(zh@),而五采以文,是自为牝(p@n)牡 (m(),名曰象蛇,其鸣自詨(ji4o)。留水出焉,而南流注于河。其中有 ■(xi4n)父之鱼,其状如鲋(f))鱼,鱼首而彘(zh@)身,食之已呕。

【注释】①■:肉瘤。②句瞿:斗。

【译文】再往东三百里,是座阳山,山上有丰富的玉石,山下有丰富的 金铜。山中有一种野兽,形状像普通的牛而长着红尾巴,脖子上有肉瘤,像 斗的形状,名称是领胡,它发出的叫声便是自身名称的读音,人吃了它的肉 就能治愈癫狂症。山中还有一种禽鸟,形状像雌性野鸡,而羽毛上有五彩斑 斓的花纹,这种鸟一身兼有雄雌二种性器官,名称是象蛇,它发出的叫声便 是自身名称的读音。留水从这座山发源,然后向南流入黄河。水中生长着■ 父鱼,形状像一般的鲫鱼,长着鱼的头而猪的身子,人吃了它的肉可以治愈 呕吐。

又东三百五十里,曰贲闻之山,其上多苍玉,其下多黄垩,多涅(ni5) 石①。

【注释】①涅石:一种黑色矾石,可做黑色染料。矾石是一种矿物,为透明结晶体,有白、 黄、青、黑、绛五种。

【译文】再往东三百五十里,是座贲闻山,山上盛产苍玉,山下盛产黄 色垩土,也有许多涅石。

又北百里,曰王屋之山,是多石。■(ni3n)水出焉,而西北流于泰泽。

【译文】再往北一百里,是座王屋山,这里到处是石头。■水从这座山 发源,然后向西北流入泰泽。

又东北三百里,曰教山,其上多玉而无石。教水出焉,西流注于河,是 水冬干而夏流,实惟干河。其中有两山,是山也,广员三百步,其名曰发丸 之山①,其上有金玉。

【注释】①发丸之山:据古人讲,发丸山居于水中,形状像似神人所发射的两颗弹丸,所以 这样叫。

【译文】再往东北三百里,是座教山,山上有丰富的玉石而没有石头。

教水从这座山发源,向西流入黄河,这条河水到了冬季干枯而在夏季流水, 确实可说是干河。教水的河道中有两座小山,方圆各三百步,名称是发丸山, 小山上蕴藏着金属矿物和玉石。

又南三百里,曰景山,南望盐贩之泽,北望少泽。其上多草、薯(sh() ■(y*)①,其草多秦椒②,其阴多赭(zh7),其阳多玉。有鸟焉,其状 如蛇,而四翼、六目、三足,名曰酸与,其鸣自詨(ji4o),见(xi4n)则 其邑有恐。

【注释】①薯■:一种植物,根像羊蹄,可以食用,就是今天所说的山药。②秦椒:一种草, 所结的子实像花椒,叶子细长。

【译文】再往南三百里,是座景山,在山上向南可以望见盐贩泽,向北 可以望见少泽。山上生长着茂密的丛草、薯■,这里的草以秦椒最多,山北 阴面多出产赭石,山南阳面多出产玉石。山里有一种禽鸟,形状像一般的蛇, 却长有四只翅膀、六只眼睛、三只脚,名称是酸与,它发出的叫声便是自身 名称的读音,在哪个地方出现那里就会发生使人惊恐的事情。

又东南三百二十里,曰孟门之山,其上多苍玉,多金,其下多黄垩(6), 多涅(ni5)石。

【译文】再往东南三百二十里,是座孟门山,山上蕴藏有丰富的苍玉, 还盛产金属矿物,山下到处是黄色垩土,还有许多涅石。

又东南三百二十里,曰平山。平水出于其上,潜于其下,是多美玉。

【译文】再往东南三百二十里,是座平山。平水从这座山的顶上发源, 然后潜流到山下,水中有很多优良玉石。

又东二百里,曰京山,有美玉,多漆木,多竹,其阳有赤铜,其阴有玄 ■(s))①。高水出焉,南流注于河。

【注释】①玄:黑色。■:砥石。就是磨刀石。

【译文】再往东二百里,是座京山,盛产美玉,到处有漆树,遍山是竹 林,在这座山的阳面出产黄铜,山北阴面出产黑色磨石。高水从这座山发源, 向南流入黄河。

又东二百里,曰虫尾之山,其上多金玉,其下多竹,多青碧。丹水出焉, 南流注于河。薄水出焉,而东南流注于黄泽。

【译文】再往东二百里,是座虫尾山,山上有丰富的金属矿物和玉石, 山下到处是竹丛,还有很多青石碧玉。丹水从这座山发源,向南流入黄河。

薄水也从这座山发源,向东南流入黄泽。

又东三百里,曰彭■(p0)之山,其上无草木,多金玉,其下多水。蚤 (z4o)林之水出焉,东南流注于河。肥水出焉,而南流注于床水,其中多 肥遗之蛇。

【译文】再往东三百里,是座彭■山,山上不生长花草树木,有丰富的 金属矿物和玉石,山下到处流水。蚤林水从这座山发源,向东南流入黄河。

肥水也从这座山发源,然后向南流入床水,水中有很多叫做肥遗的蛇。

又东百八十里,曰小侯之山。明漳之水出焉,南流注于黄泽。有鸟焉, 其状如乌而白文,名曰鸪(g&)■(x9),食之不灂(ji4o)①。

【注释】①灂:通“■”。眼昏矇。

【译文】再往东一百八十里,是座小侯山。明漳水从这座山发源,向南 流入黄泽。山中有一种禽鸟,形状像一般的乌鸦却有白色斑纹,名称是鸪■, 吃了它的肉就能使人的眼睛明亮而不昏花。

又东三百七十里,曰泰头之山。共水出焉,南注于虖(h&)池(tu¥)。

其上多金玉,其下多竹箭①。

【注释】①箭:一种生长较小的竹子,坚硬可做箭矢。

【译文】再往东三百七十里,是座泰头山。共水从这座山发源,向南流 入虖池水。山上有丰富的金属矿物和玉石,山下到处是小竹丛。

又东北二百里,曰轩辕之山,其上多铜,其下多竹。有鸟焉,其状如枭 (xi1o)而白首,其名曰黄鸟,其鸣自詨(ji4o),食之不妒。

【译文】再往东北二百里,是座轩辕山。山上多出产铜,山下到处是竹 子。山中有一种禽鸟,形状是一般的猫头鹰却长着白脑袋,名称是黄鸟,发 出的叫声便是它自身名称的读音,吃了它的肉就能使人不生妒嫉心。

又北二百里,曰谒(y6)戾(l@)之山,其上多松柏,有金玉。沁水出 焉,南流注于河。其东有林焉,名曰丹林。丹林之水出焉,南流注于河。婴 侯之水出焉,北流注于氾(f4n)水。

【译文】再往北二百里,是座谒戾山,山上到处是松树和柏树,还蕴藏 着金属矿物和玉石。沁水从这座山发源,向南流入黄河。在这座山的东面有 一片树林,叫做丹林。丹林水便从这里发源,向南流入黄河。婴侯水也从这 里发源,向北流入氾水。

东三百里,曰沮洳之山,无草木,有金玉。濝(q@)水出焉,南流注于 河。

【译文】往东三百里,是座沮洳山,不生长花草树木,有金属矿物和玉 石。濝水从这座山发源,向南流入黄河。

又北三百里,曰神囷(q)n)之山,其上有文石,其下有白蛇,有飞虫 ①。黄水出焉,而东流注于洹(w2n)。滏水出焉,而东流注于欧水。

【注释】①飞虫:指蠛(mi6)蠓、蚊子之类的小飞虫,成群成堆地乱飞,满天蔽日。

【译文】再往北三百里,是座神囷山,山上是带有花纹的漂亮石头,山 下有白蛇,还有飞虫。黄水从这座山发源,然后向东流入洹水。滏水也从这 座山发源,向东流入欧水。

又北二百里,曰发鸠之山,其上多柘(zh6)木①。有鸟焉,其状如乌, 文首、白喙(hu@)、赤足,名曰精卫,其鸣自詨(ji4o)。是炎帝之少女 ②,名曰女娃。女娃游于东海,溺而不返,故为精卫,常衔西山之木石,以 堙(y9n)于东海③。漳水出焉,东流注于河。

【注释】①柘木:柘树,是桑树的一种,叶子可以喂养蚕,果实可以吃,树根树皮可作药用。

②炎帝:号称神农氏,传说中的上古帝王。③堙:堵塞。

【译文】再往北二百里,是座发鸠山,山上生长着茂密的柘树。山中有 一种禽鸟,形状像一般的乌鸦,却长着花脑袋、白嘴巴、红足爪,名称是精 卫,它发出的叫声就是自身名称的读音。精卫鸟原是炎帝的小女儿,名叫女 娃。女娃到东海游玩。淹死在东海里没有返回,就变成了精卫鸟,常常衔着 西山的树枝和石子,用来填塞东海。漳水从这座山发源,向东流入黄河。

又东北百二十里,曰少山,其上有金玉,其下有铜。清漳之水出焉,东 流于浊漳之水。

【译文】再往东北一百二十里,是座少山,山上出产金属矿物和玉石, 山下出产铜。清漳水从这座山发源,向东流入浊漳水。

又东北二百里,曰锡山,其上多玉,其下有砥(d!)。牛首之水出焉, 而东流注于滏水。

【译文】再往东北二百里,是座锡山,山上有丰富的玉石,山下出产磨 石。牛首水从这座山发源,然后向东流入滏水。

又北二百里,曰景山,有美玉。景水出焉,东南流注于海泽。

【译文】再往北二百里,是座景山,山上出产优良玉石。景水从这座山 发源,向东南流入海泽。

又北百里,曰题首之山,有玉焉,多石,无水。

【译文】再往北一百里,是座题首山,这里出产玉石,也有许多石头, 但没有水。

又北百里,曰绣山,其上有玉、青碧,其木多栒(x*n)①,其草多芍 (shu¥)药、芎(xi#ng)(qi#ng)②。洧(w7i)水出焉,而东流注于 河,其中有鳠(h))、黾(m!n)③。

【注释】①栒:栒树,古人常用树干部分的木材制做拐杖。②芍药:多年生草本花卉,初夏 开花,与牡丹相似、花朵大而美丽,有白、红等颜色。③鳠:鳠鱼,体态较细,灰褐色,头扁平,背 鳍、胸鳍相对有一硬刺,后缘有踞齿。黾:蛙的一种,形体同虾蟆相似而小一些,皮肤青色。

【译文】再往北一百里,是座绣山,山上有玉石、青色碧玉,山中的树 木大多是栒树,而草以芍药、芎最多。洧水从这座山发源,然后向东流入 黄河,水中有鳠鱼和黾蛙。

又北百二十里,曰松山。阳水出焉,东北流注于河。

【译文】再往北一百二十里,是座松山。阳水从这座山发源,向东北流 入黄河。

又北百二十里,曰敦与之山,其上无草木,有金玉。溹(su%)水出于 其阳,而东流注于泰陆之水;泜(d!)水出于其阴,而东流注于彭水;槐水 出焉,而东流注泜泽。

【译文】再往北一百二十里,是座敦与山,山上不生长花草树木、蕴藏 有金属矿物和玉石。溹水从敦与山的南面山脚流出,然后向东流入泰陆水; 泜水从敦与山的北面山脚流出,然后向东流入彭水;槐水也从这座山发源, 然后向东流入泜泽。

又北百七十里,曰柘(zh6)山,其阳有金玉,其阴有铁。历聚之水出 焉,而北流注于洧(w7i)水。

【译文】再往北一百七十里,是座柘山,山南阳面出产金属矿物和玉石, 山北阴面出产铁。历聚水从这座山发源,然后向北流入洧水。

又北三百里,曰维龙之山,其上有碧玉,其阳有金,其阴有铁。肥水出 焉,而东流注于皋泽,其中多礨(l7i)石①。敞铁之水出焉,而北流注于 大泽。

【注释】①礨石:礨的本义是地势突然高出的样子。礨石在这里指河道中的大石头高出水面 许多,显得突兀。

【译文】再往北三百里,是座维龙山,山上出产碧玉,山南阳面有金, 山北阴面有铁。肥水从这座山发源,然后向东流入皋泽,水中有很多高耸的 大石头。敞铁水也从这座山发源,然后向北流入大泽。

又北百八十里,曰白马之山,其阳多石玉,其阴多铁,多赤铜。木马之 水出焉,而东北流注于虖(h&)沱(tu¥)。

【译文】再往北一百八十里,是座白马山,山南阳面有很多石头和玉石, 山北阴面有丰富的铁,还多出产黄铜。木马水从这座山发源,然后向东北流 入虖沱水。

又北二百里,曰空桑之山,无草木,冬夏有雪。空桑之水出焉,东流注 于虖(h&)沱(tu¥)。

【译文】再往北二百里,是座空桑山,没有花草树木,冬天夏天都有雪。

空桑水从这座山发源,向东流入虖沱水。

又北三百里,曰泰戏之山,无草木,多金玉。有兽焉,其状如羊,一角 一目,目在耳后,其名曰■■(d^ng d^ng),其鸣自訆(ji4o)。虖(h&) 沱(tu¥)之水出焉,而东流注于溇(l¥u)水。液(y@)女之水出于其阳, 南流注于沁水。

【译文】再往北三百里,是座泰戏山,不生长花草树木,到处有金属矿 物和玉石。山中有一种野兽,形状像普通的羊,却长着一只角一只眼睛,眼 睛在耳朵的背后,名称是■■,它发出的叫声便是自身名称的读音。虖沱水 从这座山发源,然后向东流入溇水。液女水发源于这座山的南面,向南流入 沁水。

又北三百里,曰石山,多藏金玉。濩濩(hu^hu^)之水出焉,而东流注 于虖(h&)沱(tu¥);鲜于之水出焉,而南流注于虖沱。

【译文】再往北三百里,是座石山,山中有丰富的金属矿物和玉石。濩 濩水从这座山发源,然后向东流入虖沱水;鲜于水也从这座山发源,然后向 南流入虖沱水。

又北二百里,曰童戎之山。皋涂之水出焉,而东流注于溇(l¥u)液(y@) 水。

【译文】再往北二百里,是座童戎山。皋涂水从这座山发源,然后向东 流入溇液水。

又北三百里,曰高是之山。滋水出焉,而南流注于虖(h&)沱(tu¥)。

其木多棕,其草多条。滱(k^u)水出焉,东流注于河。

【译文】再往北三百里,是座高是山。滋水从这座山发源,然后向南流 入虖沱水。山中的树木大多是棕树,草大多是条草。滱水也从这座山发源, 然后向东流入黄河。

又北三百里,曰陆山,多美玉。■水出焉,而东流注于河。

【译文】再往北三百里,是座陆山,有很多优良玉石。■水从这座山发 源,然后向东流入黄河。

又北二百里,曰沂(q0)山。般(p2n)水出焉,而东流注于河。

【译文】再往北二百里,是座沂山。般水从这座山发源,然后向东流入 黄河。

北百二十里,曰燕山,多婴石①。燕水出焉,东流注于河。

【注释】①婴石:一种像玉一样的带有彩色条纹的漂亮石头。

【译文】往北一百二十里,是座燕山,出产很多的婴石。燕水从这座山 发源,向东流入黄河。

又北山行五百里,水行五百里,至于饶山。是无草木,多瑶、碧,其兽 多橐(tu¥)橐①,其鸟多鹠(li*)②。历虢(gu¥)之水出焉,而东流注 于河,其中有师鱼③,食之杀人。

【注释】①橐駞:就是骆驼。②鹠:即鸺鹠,也叫做横纹小鸮,头和颈侧及翼上覆羽暗褐色, 密布棕白色狭横斑。③师鱼:即鲵鱼,就是前面所说的人鱼。

【译文】再往北走五百里山路,又走五百里水路,便到了饶山。这座山 不生长花草树木,到处是瑶、碧一类的美玉,山中的野兽大多是骆驼,而禽 鸟大多是鸺鹠鸟。历虢水从这座山发源,然后向东流入黄河,水中有师鱼, 人吃了它的肉就会中毒而死。

又北四百里,曰乾(g1n)山,无草木,其阳有金玉,其阴有铁,而无 水。有兽焉,其状如牛而三足,其名曰(獂)[豲](hu2n)其鸣自詨(ji4o)。

【译文】再往北四百里,是座乾山,没有花草树木,山南阳面蕴藏着金 属矿物和玉石,山北阴面蕴藏着铁,但没有水流。山中有一种野兽,形状像 普通的牛却长着三只脚,名称是豲,它发出的叫声便是自身名称的读音。

又北五百里,曰伦山。伦水出焉,而东流注于河。有兽焉,其状如麋, 其(川)[州]在尾上①,其名曰罴(p0)[九]。

【注释】①州:古人注“州”为“窍”。上窍谓耳目鼻口,下窍谓前阴后阴。这里的窍是指 后阴而言,就是肛门的意思。

【译文】再往北五百里,是座伦山。伦水从这座山发源,然后向东流入 黄河。山中有一种野兽,形状像麋鹿,肛门却长在尾巴上面,名称是罴九。

又北五百里,曰碣(ji6)石之山。绳水出焉,而东流注于河,其中多 蒲夷之鱼①。其上有玉,其下多青碧。

【注释】①蒲夷之鱼:古人认为就是冉遗鱼,它的形体似蛇,有六只脚,眼睛像马的眼睛, 人吃了它的肉就不会做恶梦。

【译文】再往北五百里,是座碣石山。绳水从这座山发源,然后向东流 入黄河,水中有很多蒲夷鱼。这座山上出产玉石,山下还有很多青石碧玉。

又北水行五百里,至于雁门之山,无草木。

【译文】再往北行五百里水路,便到了雁门山,这里没有花草树木。

又北水行四百里,至于泰泽。其中有山焉,曰帝都之山,广员百里,无 草木,有金玉。

【译文】再往北行四百里水路,便到了泰泽。在泰泽中屹立着一座山, 叫做帝都山,方圆一百里,不生长花草树木,有金属矿物和玉石。

又北五百里,曰錞(ch*n)于毋(w*)逢之山,北望鸡号之山,其风如 ■(l@)①。西望幽都之山,浴水出焉。是有大蛇,赤首白身,其音如牛, 见(xi1n)则其邑大旱。

【注释】①■:急风的样子。

【译文】再往北五百里,是座錞于毋逢山,从山上向北可以望见鸡号山, 从那里吹出的风如强劲的■风。从錞于毋逢山向西可以望见幽都山,浴水从 那里流出。这座幽都山中有一种大蛇,红色的脑袋白色的身子,发出的声音 如同牛叫,在哪个地方出现那里就会有大旱灾。

凡北次三(经)[山]之首,自太行之山以至于无逢之山①,凡四十六山, 万二千三百五十里。其神状皆马身而人面者廿(ni4n)神②。其祠之:皆用 一藻(z3o)茝(zh!)瘗(y@)之③。其十四神状皆彘(zh@)身而载玉④。

其祠之:皆玉,不瘗。其十神状皆彘身而八足蛇尾。其祠之:皆用一璧瘗之。

大凡四十四神,皆用稌(t*)糈(x()米祠之。此皆不火食。

【注释】①无逢之山:即上文所说的錞于毋逢山。②廿:二十。有时也写成“艹”。③藻: 聚藻,一种香草。茝:香草,属于兰草之类。④载:通“戴”。

【译文】总计北方第三列山系之首尾,自太行山起到无逢山止,一共四 十六座山,途经一万二千三百五十里。其中有二十座山山神的形状都是马一 样的身子而人一样的面孔。祭祀这些山神:都是把用作祭品的藻和茝之类的 香草埋入地下。另外十四座山山神的形状是猪一样的身子却佩戴着玉制饰 品。祭祀这些山神:都用祀神的玉器,不埋入地下。还有十座山山神的形状 都是猪一样的身子却长着八只脚和蛇一样的尾巴,祭祀这些山神:用一块玉 壁祭祀后埋入地下。总共四十四个山神,都要用精米来祭祀。参加这项祭祀 活动的人都生吃未经火烤的食物。

右北经之山志,凡八十七山,二万三千二百三十里。

【译文】以上是北方经历之山的记录,总共八十七座山,二万三千二百 三十里。


\chapter{东山经}

山海经卷四 东山经


东山(经)之首,曰樕(s))■(zh))之山,北临乾(g1n)昧(m6i)。

食水出焉,而东北流注于海。其中多鳙鳙(y^ng y^ng)之鱼,其状如犁牛 ①,其音如彘(zh@)鸣。

【注释】①犁牛:毛色黄黑相杂的牛,像虎纹似的。

【译文】东方第一列山系之首座山,叫做樕■山,北面与乾昧山相邻。

食水从这座山发源,然后向东北流入大海。水中有很多鳙鳙鱼,形状像犁牛, 发出的声音如同猪叫。

又南三百里,曰藟(l7i)山,其上有玉,其下有金。湖水出焉,东流 注于食水,其中多活师①。

【注释】①活师:又叫活东,蝌蚪的别名,是青蛙、蛤蟆、娃娃鱼等两栖动物的幼体,头又 圆又大而尾巴细小,游泳水中。

【译文】再往南三百里,是座藟山,山上有玉,山下有金。湖水从这座 山发源,向东流入食水,水中有很多蝌蚪。

又南三百里,曰栒(x*n)状之山,其上多金玉,其下多青碧石。有兽 焉,其状如犬,六足,其名曰从从,其鸣自詨(ji4o)。有鸟焉,其状如鸡 而鼠(毛)[尾],其名曰(z9)鼠,见(xi4n)则其邑大旱。■(zh!)水 出焉,而北流注于湖水。其中多箴(zh5n)鱼,其状如儵(ch¥u)①,其喙 (hu@)如箴②,食之无疫疾。

【注释】①儵:即“鯈”字。鯈鱼,也叫白鲦(ti2o)、■(c1n)儵,一种小白鱼。体长 只有数寸,侧扁,银白色,腹面有肉棱,背鳍有硬刺。生活在江湖中。②箴:同“针”。

【译文】再往南三百里,是座栒状山,山上有丰富的金属矿物和玉石, 山下有丰富的青石碧玉。山中有一种野兽,形状像一般的狗,却长着六只脚, 名称是从从,它发出的叫声便是自身名称的读音。山中有一种禽鸟,形状像 普通的鸡却长着老鼠一样的尾巴,名称是■鼠,在哪个地方出现那里就会有 大旱灾。■水从这座山发源,然后向北流入湖水。水中有很多箴鱼,形状像 儵鱼,嘴巴像长针,人吃了它的肉就不会染上瘟疫病。

又南三百里,曰勃亝(q@)之山①,无草木,无水。

【注释】①亝:“齐”的古字。

【译文】再往南三百里,是座勃亝山,没有花草树木,也没有水。

又南三百里,曰番条之山,无草木,多沙。減(ji3n)水出焉,北流注 于海,其中多鳡(g3n)鱼①。

【注释】①鳡鱼:也叫做母鲇、竿鱼,体延长,亚圆筒形,青黄色,吻尖长,口大,眼小, 性凶猛,捕食各种鱼类。

【译文】再往南三百里,是座番条山,没有花草树木,到处是沙子。減 水从这座山发源,向北流入大海,水中有很多鳡鱼。

又南四百里,曰姑儿之山,其上多漆,其下多桑、柘(zh6)。姑儿之 水出焉,北流注于海,其中多鳡(g3n)鱼。

【译文】再往南四百里,是座姑儿山,山上有茂密的漆树,山下有茂密 的桑树、柘树。姑儿水从这座山发源,向北流入大海,水中有很多鳡鱼。

又南四百里,曰高氏之山,其上多玉,其下多箴(zh5n)石①。诸绳之 水出焉,东流注于泽,其中多金玉。

【注释】①箴石:石针是古代的一种医疗器具,用石头磨制而成,可以治疗痈肿疽疱,排除 脓血。箴石就是一种专门制做石针的石头。

【译文】再往南四百里,是座高氏山,山上盛产玉石,山下盛产箴石。

诸绳水从这座山发源,向东流入湖泽,水中有许多金属矿物和玉石。

又南三百里,曰岳山,其上多桑,其下多樗(ch&)。泺(lu^)水出焉, 东流注于泽,其中多金玉。

【译文】再往南三百里,是座岳山,山上有茂密的桑树,山下有茂密的 臭椿树。泺水从这座山发源,向东流入湖泽,水中有许多金属矿物和玉石。

又南三百里,曰犲(ch2i)山,其上无草木,其下多水,其中多堪■(x)) 之鱼。有兽焉,其状如夸父而彘(zh@)毛,其音如呼,见(xi4n)则天下 大水。

【译文】再往南三百里,是座犲山,山上不生长花草树木,山下到处流 水,水中有很多堪■鱼。山中有一种野兽,形状像猿猴却长着一身猪毛,发 出的声音如同人呼叫,一出现而天下就会发生水灾。

又南三百里,曰独山,其上多金玉,其下多美石。末涂之水出焉,而东 南流注于沔(mi3n),其中多■(ti1o)■(r¥ng),其状如黄蛇,鱼翼, 出入有光,见(xi4n)则其邑大旱。

【译文】再往南三百里,是座独山,山上有丰富的金属矿物和玉石,山 下多的是美观漂亮的石头。未涂水从这座山发源,然后向东南流入沔水,水 中有很多■■,形状与黄蛇相似,长着鱼一样的鳍,出入水中时闪闪发光, 在哪个地方出现那里就会有大旱灾。

又南三百里,曰泰山,其上多玉,其下多金。有兽焉,其状如豚(t*n) 而有珠,名曰狪狪(t#ngt#ng),其鸣自詨(ji4o)。环水出焉,东流注于 (江)[汶],其中多水玉。

【译文】再往南三百里,是座泰山,山上盛产玉,山下盛产金。山中有 一种野兽,形状与一般的猪相似而体内却有珠子,名称是狪狪,它发出的叫 声便是自身名称的读音。环水从这座山发源,向东流入汶水,水中有很多水 晶石。

又南三百里,曰竹山,錞(ch*n)于(江)[汶],无草木,多瑶、碧。

激水出焉,而东南流注于娶檀(t2n)之水,其中多茈(z!)(羸)[蠃](lu^)。

【译文】再往南三百里,是座竹山,座落于汶水边上,这座山没有花草 树木,到处是瑶、碧一类的玉石。激水从竹山发源,然后向东南流入娶檀水, 水中有很多紫色螺。

凡东山(经)之首,自樕(s))■(zh&)之山以至于竹山,凡十二山, 三千六百里。其神状皆人身龙首。祠:毛用一犬祈,■(6r)用鱼①。

【注释】①■:用牲畜作为祭品来向神祷告,想要使神听见。

【译文】总计东方第一列山系之首尾,自樕■山起到竹山止,一共十二 座山,途经三千六百里。诸山山神的形貌都是人的身子龙的头。祭祀山神: 在毛物中用一只狗作为祭品来祭祀,祷告时要用鱼。

东次二(经)[山]之首,曰空桑之山,北临食水,东望沮(j&)吴,南 望沙陵,西望湣(m0n)泽。有兽焉,其状如牛而虎文,其音如(钦)[吟], 其名曰軨軨(l0n l0n),其鸣自叫,见(xian)则天下大水。

【译文】东方第二列山系之首座山,叫做空桑山,北面临近食水,在山 上向东可以望见沮吴,向南可以望见沙陵,向西可以望见湣泽。山中有一种 野兽,形状像普通的牛却有老虎一样的斑纹,发出的声音如同人在呻吟,名 称是軨軨,它发出的叫声便是自身名称的读音,一出现而天下就会发生水 灾。

又南六百里,曰曹夕之山,其下多(穀)[榖(g^u)],而无水,多鸟 兽。

【译文】再往南六百里,是座曹夕山,山下到处是构树,却没有水流, 还有许多禽鸟野兽。

又西南四百里,曰峄(y@)皋(g1o)之山,其上多金玉,其下多白垩。

峄皋之水出焉,东流注于激女(r()之水,其中多蜃(sh6n)珧(y2o)①。

【注释】①蜃:大蛤(g2)。蛤是一种软体动物,贝壳卵圆形或略带三角形,颜色和斑纹美 丽。珧:小蚌。蚌是一种软体动物,贝壳长卵形,表面黑褐色或黄褐色,有环形。

【译文】再往西南四百里,是座峄皋山,山上有丰富的金属矿物和玉石, 山下有丰富的白垩土。峄皋水从这座山发源,向东流入激女水,水中有很多 大蛤和小蚌。

又南水行五百里,流沙三百里,至于葛山之尾,无草木,多砥(d!)砺 (l0)。

【译文】再往南行五百里水路,经过三百里流沙,便到了葛山的尾端, 这里没有花草树木,到处是粗细磨石。

又南三百八十里,曰葛山之首,无草木。澧(l!)水出焉,东流注于余 泽,其中多珠蟞(bi5)鱼,其状如(胏)[肺]而(有)[四]目,六足有珠, 其味酸甘,食之无疠(l0)。

【译文】再往南三百八十里,就是葛山的首端,这里没有花草树木。澧 水从此发源,向东流入余泽,水中有很多珠蟞鱼,形状像动物的一叶肺器官 却有四只眼睛,还有六只脚而且能吐珠子,这种珠蟞鱼的肉味是酸中带甜, 人吃了它的肉就不会染上瘟疫病。

又南三百八十里,日余峨(6)之山,其上多梓(z!)枏(n1n),其下 多荆芑(q!)①。杂余之水出焉,东流注于黄水。有兽焉,其状如菟(t)) 而鸟喙(hu@),鸱(ch9)目蛇尾,见人则眠②,名曰犰(qi*)狳(y*), 其鸣自訆(ji4o),见(xi4n)则螽(zh#ng)蝗为败③。

【注释】①芑:通“杞”。即构杞树。②眠,装死。③螽:即螽斯,蝗虫之类的昆虫,体绿 色或褐色,样子像蚱蜢,以翅摩擦发音。但对农作物的损害不如蝗虫厉害。为败:为害。

【译文】再往南三百八十里,是座余峨山,山上有茂密的梓树和楠木树, 山下有茂密的牡荆树和枸杞树。杂余水从这座山发源,向东流入黄水。山中 有一种野兽,形状像一般的兔子却是鸟的嘴,鹞鹰的眼睛和蛇的尾巴,一看 见人就躺下装死,名称是犰狳,发出的叫声便是它自身名称的读音,一出现 就会有螽斯蝗虫出现而为害庄稼。

又南三百里,曰杜父之山,无草木,多水。

【译文】再往南三百里,是座杜父山,不生长花草树木,到处流水。

又南三百里,曰耿山,无草木,多水碧①,多大蛇。有兽焉,其状如狐 而鱼翼,其名曰朱獳(r(),其鸣自訆(ji4o),见(xi4n)则其国有恐。

【注释】①水碧:就是前文所说的水玉之类,即水晶石。

【译文】再往南三百里,是座耿山,没有花草树木,到处是水晶石,还 有很多大蛇。山中有一种野兽,形状像狐狸却长着鱼鳍,名称是朱獳,发出 的叫声便是它自身名称的读音,在哪个国家出现而那个国家里就会有恐怖的 事发生。

又南三百里,曰卢其之山,无草木,多沙石。沙水出焉,南流注于涔(c6n) 水,其中多鵹(l@)鹕(h*)①,其状如鸳鸯而人足,共鸣自訆,见(xi4n) 则其国多土功。

【注释】①鵹鹕:即鹈鹕鸟,也叫做伽蓝鸟、淘河鸟、塘鸟。它的体长可达二米,羽毛多是 白色,翅大而阔,下颌底部有一大的皮囊,能伸缩,可以用来兜食鱼类动物。因为它的四趾之间有金 蹼相连,所以古人认为其足类似人脚。

【译文】再往南三百里,是座卢其山,不生长花草树木,到处是沙子石 头。沙水从这座山发源,向南流入涔水,水中有很多鹈鹕鸟,形状像一般的 鸳鸯却长着人一样的脚,发出的叫声便是它自身名称的读音,在哪个国家出 现那个国家里就会有水土工程的劳役。

又南三百八十里,曰姑射(y6)之山,无草木,多水。

【译文】再往南三百八十里,是座姑射山,没有花草树木,到处流水。

又南水行三百里,流沙百里,曰北姑射(y6)之山,无草木,多石。

【译文】再往南行三百里水路,经过一百里流沙,是座北姑射山,没有 花草树木,到处是石头。

又南三百里,曰南姑射(y6)之山,无草木,多水。

【译文】再往南三百里,是座南姑射山,没有花草树木,到处流水。

又南三百里,曰碧山,无草木,多大蛇,多碧、水玉。

【译文】再往南三百里,是座碧山,没有花草树木,有许多大蛇。还盛 产碧玉、水晶石。

又南五百里,曰缑(h¥u)氏之山,无草木,多金玉。原水出焉,东流 注于沙泽。

【译文】再往南五百里,是座缑氏山,不生长花草树木,有丰富的金属 矿物和玉石。原水从这座山发源,向东流入沙泽。

又南三百里,曰姑逢之山,无草木,多金玉。有兽焉,其状如狐而有翼, 其音如鸿雁,其名曰獙獙(b@ b@),见(xi4n)则天下大旱。

【译文】再往南三百里,是座姑逢山,没有花草树木,有丰富的金属矿 物和玉石。山中有一种野兽,形状像一般的狐狸却有翅膀,发出的声音如同 大雁鸣叫,名称是獙獙,一出现而天下就会发生大旱灾。

又南五百里,曰凫(f*)丽之山,其上多金玉,其下多箴(zh5n)石。

有兽焉,其状如狐,而九尾、九首、虎爪,名曰■(l¥ng)侄,其音如婴儿, 是食人。

【译文】再往南五百里,是座凫丽山,山上有丰富的金属矿物和玉石, 山下盛产箴石。山中有一种野兽,形状像一般的狐狸,却有九条尾巴、九个 脑袋、虎一样的爪子,名称是■侄,发出的声音如同婴儿啼哭,是能吃人的。

又南五百里,曰■(zh5n)山,南临■水,东望湖泽。有兽焉,其状如 马,而羊目、四角、牛尾,其音如嗥狗,其名曰峳峳(y#uy#u),见(xi4n) 则其国多狡客。有鸟焉,其状如凫(f*)而鼠尾,善登木,其名曰絜(xi6) 鉤(g#u),见(xi4n)则其国多疫。

【译文】再往南五百里,是座■山,南面临近■水,从山上向东可以望 见湖泽。山中有一种野兽,形状像普通的马,却长着羊一样的眼睛、四只角、 牛一样的尾巴,发出的声音如同狗叫,名称是峳峳,在哪个国家出现那个国 家里就会有很多奸猾的政客。山中还有一种禽鸟,形状像野鸭子却长着老鼠 一样的尾巴,擅长攀登树木,名称是絜鉤,在哪个国家出现那个国家里就多 次发生瘟疫病。

凡东次二(经)[山]之首,自空桑之山至于■(zh5n)山,凡十七山, 六千六百四十里。其神状皆兽身人面载觡(g6)①。其祠:毛用一鸡祈,婴 用一璧瘗(y@)②。

【注释】①载:戴。一般指将东西戴在头上。觡:骨角。专指麋、鹿等动物头上的角,这种 角的骨质与角质合而为一,没有差异,所以叫骨角。②婴:据学者研究,婴是古代人用玉器祭祀神的 专称。

【译文】总计东方第二列山系之首尾,自空桑山起到■山止,一共十七 座山,途经六千六百四十里。诸山山神的形貌都是野兽的身子人的面孔而且 头上戴着觡角。祭祀山神:在毛物中用一只鸡献祭,在祀神的玉器中用一块 玉璧献祭后埋入地下。

(又)东次三(经)[山]之首,曰尸胡之山,北望■(xi1ng)山,其 上多金玉,其下多棘。有兽焉,其状如麋而鱼目,名曰妴(w1n)胡,其鸣 自訆。

【译文】东方第三列山系之首座山,叫做尸胡山,从山上向北可以望见 ■山,山上有丰富的金属矿物和玉石,山下有茂密的酸枣树。山中有一种野 兽,形状像麋鹿却长着鱼一样的眼睛,名称是妴胡,它发出的叫声便是自身 名称的读音。

又南水行八百里,曰岐山,其木多桃李,其兽多虎。

【译文】再往南行八百里水路,是座岐山,山中的树木大多是桃树和李 树,而野兽大多是老虎。

又南水行五百里,曰诸鉤(g#u)之山,无草木,多沙石。是山也,广 员百里,多寐(w6i)鱼①。

【注释】①寐鱼:又叫嘉鱼、卷口鱼,古人称为鮇鱼。这种鱼体延长,前部亚圆筒形,后部 侧扁。体暗褐色。须二对,粗长。吻褶发达,裂如缨状。

【译文】再往南行五百里水路,是座诸鉤山,没有花草树木,到处是沙 子石头。这座山,方圆一百里,有很多寐鱼。

又南水行七百里,曰中父之山,无草木,多沙。

【译文】再往南行七百里水路,是座中父山,没有花草树木,到处是沙 子。

又东水行千里,曰胡射(y6)之山,无草木,多沙石。

【译文】再往东行一千里水路,是座胡射山,没有花草树木,到处是沙 子石头。

又南水行七百里,曰孟子之山,其木多梓(z!)桐,多桃李,其草多菌 蒲①,其兽多麋、鹿。是山也,广员百里。其上有水出焉,名曰碧阳,其中 多鳣(zh1n)鲔(w7i)②。

【注释】①菌蒲:即紫菜、石花菜、海带、海苔之类。②鳣:鳣鱼,据古人说是一种大鱼, 体形像鱏鱼而鼻子短,口在颔下,体有斜行甲,没有鳞,肉是黄色,大的有二、三丈长。鲔:鲔鱼, 据古人说就是鱏鱼,体形像鳣鱼而鼻子长,体无鳞甲。

【译文】再往南行七百里水路,是座孟子山,山中的树木大多是梓树和 桐树,还生长着茂密的桃树和李树,山中的草大多是菌蒲,山中的野兽大多 是麋、鹿。这座山,方圆一百里。有条河水从山上流出,名称是碧阳,水中 生长着很多鳣鱼和鲔鱼。

又南水行五百里,(曰)流沙(行)五百里,有山焉,曰跂(q0)踵之 山,广员二百里,无草木,有大蛇,其上多玉。有水焉,广员四十里皆涌, 其名曰深泽,其中多蠵(xi6)龟①。有鱼焉,其状如鲤,而六足鸟尾,名 曰鮯鮯(h2 h2)之鱼,其(名)[鸣]自訆。

【注释】①蠵龟:也叫赤蠵龟,据古人说是一种大龟,甲有纹彩,像玳瑁而薄一些。玳瑁是 海中动物,形似龟,背面角质板光,有褐色和淡黄色相间的花纹,大的可达数尺。

【译文】再往南行五百里水路,经过流沙五百里,有一座山,叫做踵踵 山,方圆二百里,没有花草树木,有大蛇,山上有丰富的玉石。这里有一水 潭,方圆四十里都在喷涌泉水,名称是深泽,水中有很多蠵龟。水中还生长 着一种鱼,形状像一般的鲤鱼,却有六只脚和鸟一样的尾巴,名称是鮯鮯鱼, 发出的叫声便是它自身名称的读音。

又南水行九百里,曰踇(m()隅(y*)之山,其上多草木,多金玉,多 赭(zh7)。有兽焉,其状如牛而马尾,名曰精精,其鸣自訆。

【译文】再往南行九百里水路,是座踇隅山,山上有茂密的花草树木, 有丰富的金属矿物和玉石,还有许多赭石。山中有一种野兽,形状像一般的 牛却长着马一样的尾巴,名称是精精,它发出的叫声便是自身名称的读音。

又南水行五百里,流沙三百里,至于无皋之山,南望幼海,东望榑(f)) 木①,无草木,多风。是山也,广员百里。

【注释】①榑木:即扶桑,神话传说中的神木,叶似桑树叶,长数千丈,大二十围,两两同 根生,更相依倚,而太阳就是从这里升起的。

【译文】再往南行五百里水路,经过三百里流沙,便到了无皋山,从山 上向南可以望见幼海,向东可以望见榑木,这里不生长花草树木,到处刮大 风。这座山,方圆一百里。

凡东次三(经)[山]之首,自尸胡之山至于无皋之山,凡九山,六千九 百里。其神状皆人身而羊角。其祠:用一牡(m()羊① ,(米)[糈(x()] 用黍(sh))②。是神也,见(xi4n)则风雨水为败。

【注释】①牡:鸟鲁的雄性。②黍:一种谷物,性黏,子粒供食用或酿酒。在脱皮以后,北 方人称它为黄米子。

【译文】总计东方第三列山系之首尾,自尸胡山起到无皋山止,一共九 座山,途经六千九百里。诸山山神的形貌都是人的身子却长着羊角。祭祀山 神:在毛物中用一只公羊作祭品,祀神的米用黄米。这些山神,一出现就会 起大风、下大雨、发大水而损坏庄稼。

(又)东次四(经)[山]之首,曰北号之山,临于北海。有木焉,其状 如杨,赤华,其实如枣而无核,其味酸甘,食之不疟。食水出焉,而东北流 注于海。有兽焉,其状如狼,赤首鼠目,其音如豚(t*n),名曰(猲)[獦 (g7)](狙)[狚(d4n)],是食人。有鸟焉,其状如鸡而白首,鼠足而虎 爪,其名曰鬿(q0)雀,亦食人。

【译文】东方第四列山系之首座山,叫做北号山,屹立在北海边上。山 中有一种树木,形状像普通的杨树,开红色花朵,果实与枣子相似但没有核, 味道是酸中带甜,吃了它就能使人不患上疟疾病。食水从这座山发源,然后 向东北流入大海。山中有一种野兽,形状像狼,长着红脑袋和老鼠一样的眼 睛,发出的声音如同小猪叫,名称是獦狚,是能吃人的。山中还有一种禽鸟, 形状像普通的鸡却长着白脑袋,老鼠一样的脚足和老虎一样的爪子,名称是 鬿雀,也是能吃人的。

又南三百里,曰旄山,无草木。苍体之水出焉,而西流注于展水,其中 多鱃(qi&)鱼①,其状如鲤而大首,食者不疣(y¥u)②。

【注释】①鱃鱼:即鳅鱼,也写成鱼,形状像鳝鱼,长约三、四寸,扁尾巴,青黑色,没 有鳞甲而微有黏液。常潜居河湖池沼水田的泥土中,所以俗称泥鳅或泥。②疣:同“肬”。一种小 肉瘤,即长在人体皮肤上的小疙瘩,俗称瘊子。

【译文】再往南三百里,是座旄山,没有花草树木。苍体水从这座山发 源,然后向西流入展水,水中生长着很多鱃鱼,形状像鲤鱼而头长得很大, 吃了它的肉就能使人皮肤上不生瘊子。

又南三百二十里,曰东始之山,上多苍玉。有木焉,其状如杨而赤理, 其汁如血,不实,其名曰芑(q!)①,可以服马。泚(c!)水出焉,而东北 流注于海,其中多美贝,多茈(z!)鱼,其状如鲋②,一首而十身,其臭(xi)) 如鲋(m0)③芜,食之不■(p0)④。

【注释】①芑:“杞”的假借字。②鲋:即鲫鱼,体侧扁,稍高,背面青褐色,腹面银灰色, 肉味鲜美。③臭:气味。蘪芜:就是蘼芜,一种香草,叶子像当归草的叶子,气味像白芷草的香气。

据古人讲,因为它的茎叶靡弱而繁芜,所以这样叫。④■:同“屁”。

【译文】再往南三百二十里,是座东始山,山上多出产苍玉。山中有一 种树木,形状像一般的杨树却有红色纹理,树干中的液汁与血相似,不结果 实,名称是芑,把液汁涂在马身上就可使马驯服。泚水从这座山发源,然后 向东北流入大海,水中有许多美丽的贝,还有很多茈鱼,形状像一般的鲫鱼, 却长着一个脑袋而十个身子,它的气味与蘼芜草相似,人吃了它就不放屁。

又东南三百里,曰女烝(zh5ng)之山,其上无草木。石膏水出焉,而 西注于鬲(g6)水,其中多薄鱼,其状如鳣(sh4n)鱼而一目①,其音如欧 (o()②,见(xi4n)则天下大旱。

【注释】①鳣:通“鳝”。即鳝鱼,俗称黄鳝。②欧:呕吐。

【译文】再往东南三百里,是座女烝山,山上没有花草树木。石膏水从 这座山发源,然后向西流入鬲水,水中有很多薄鱼,形状像一般的鳝鱼却长 着一只眼睛,发出的声音如同人在呕吐,一出现而天下就会发生大旱灾。

又东南二百里,曰钦山,多金玉而无石。师水出焉,而北流注于皋泽, 其中多鱃(qi&)鱼,多文贝。有兽焉,其状如豚(t*n)而有牙①,其名曰 当康,其鸣自訆,见(xi4n)则天下大穰(r2ng)。

【注释】①牙:这里指尖锐锋利而令人可怕的露出嘴唇之外的大牙齿。

【译文】再往东南二百里,是座钦山,山中有丰富的金属矿物和玉石却 没有石头。师水从这座山发源,然后向北流入皋泽,水中有很多鱃鱼,还有 很多色彩斑斓的贝。山中有一种野兽,形状像小猪却长着大獠牙,名称是当 康,它发出的叫声就是自身名称的读音,一出现而天下就要大丰收。

又东南二百里,曰子桐之山。子桐之水出焉,而西流注于余如之泽。其 中多■(hu2)鱼,其状如鱼而鸟翼,出入有光,其音如鸳鸯,见(xi4n) 则天下大旱。

【译文】再往东南二百里,是座子桐山。子桐水从这座山发源,然后向 西流入余如泽。水中生长着很多■鱼,形状与一般的鱼相似却长着禽鸟翅 膀,出入水中时闪闪发光,发出的声音如同鸳鸯鸣叫,一出现而天下就会发 生大旱灾。

又东北二百里,曰剡(y3n)山,多金玉。有兽焉,其状如彘(zh@)而 人面,黄身而赤尾,其名曰合窳(y*),其音如婴儿,是兽也,食人,亦食 虫蛇,见(xi4n)则天下大水。

【译文】再往东北二百里,是座剡山,有丰富的金属矿物和玉石。山中 有一种野兽,形状像猪却是人的面孔,黄色的身子上长着红色尾巴,名称是 合窳,发出的声音如同婴儿啼哭。这种合窳兽,是吃人的,也吃虫和蛇,一 出现而天下就会发生水灾。

又东二百里,曰太山,上多金玉、桢木①。有兽焉,其状如牛而白首, 一目而蛇尾,其名曰蜚(f7i),行水则竭,行草则死,见(xi4n)则天下 大疫。鉤(g#u)水出焉,而北流注于劳水,其中多鱃(qi&)鱼。

【注释】①桢木:即女桢,一种灌木,叶子对生,革质,卵状披针形,在冬季不凋落,四季 常青。初夏开花,是白色,果实椭圆形。

【译文】再往东二百里,是座太山,山上有丰富的金属矿物和玉石、茂 密的女桢树。山中有一种野兽,形状像一般的牛却是白脑袋,长着一只眼睛 和蛇一样的尾巴,名称是蜚,它行经有水的地方水就干涸,行经有草的地方 草就枯死,一出现而天下就会发生大瘟疫。鉤水从这座山发源,然后向北流 入劳水,水中有很多鱃鱼。

凡东次四(经)[山]之首,自北号之山至于太山,凡八山,一千七百二 十里。

【译文】总计东方第四列山系之首尾,自北号山起到太山止,一共八座 山,途经一千七百二十里。

右东经之山志,凡四十六山,万八千八百六十里。

【译文】以上是东方经历之山的记录,总共四十六座山,一万八千八百 六十里。

\chapter{中山经}

山海经卷五 中山经 -1

小.说。t。xt-天/堂
中山(经)薄山之首,曰甘枣之山。共水出焉,而西流注于河。其上多 杻(ni()木。其下有草焉,葵本而杏叶①,黄华而荚(ji2)实②,名曰箨 (tu#),可以已瞢(m2ng)③。有兽焉,其状如■(hu@)鼠而文题④,其 名曰■(n4i),食之已瘿。

【注释】①本:草木的根或茎干。这里指茎干。②荚:凡草木果实狭长而没有隔膜的,都叫 做荚。③瞢:眼目不明。④■鼠:不详何兽。

【译文】中央第一列山系薄山山系之首座山,叫做甘枣山。共水从这座 山发源,然后向西流入黄河。山上有茂密的杻树。山下有一种草,葵菜一样 的茎干杏树一样的叶子,开黄色的花朵而结带荚的果实,名称是箨,人吃了 它可以治愈眼睛昏花。山中还有一种野兽,形状像■鼠而额头上有花纹,名 称是■,吃了它的肉就能治好人脖子上的赘瘤。

又东二十里,曰历儿之山,其上多橿(ji1ng),多■(l@)木,是木 也,方茎而员叶,黄华而毛,其实如(拣)[楝(li4n)]①,服之不忘。

【注释】①楝:楝树,也叫苦楝,落叶乔木,春夏之交开花,淡紫色,核果球形或长圆形, 熟时黄色。木材坚实,易加工,供家具、乐器、建筑、农具等用。又据古人说捣碎楝树的子实可以洗 衣,而服食它可以益肾。

【译文】再往东二十里,是座历儿山,山上有茂密的橿树,还有茂密的 ■树,这种树木,茎干是方形的而叶子是圆形的,开黄色花而花瓣上有绒毛, 果实像楝树结的果实,人服用它可以增强记忆而不忘事。

又东十五里,曰渠猪之山,其上多竹。渠猪之水出焉,而南流注于河。

其中是多豪鱼,状如鲔(w7i),赤喙(hu@)尾赤羽,[食之]可以已白癣(xu3n) ①。

【注释】①癣:皮肤感染真菌后引起的一种疾病,有好多种。

【译文】再往东十五里,是座渠猪山,山上有茂盛的竹子。渠猪水从这 座山发源,然后向南流入黄河。水中有很多豪鱼,形状像一般的鲔鱼,但长 着红嘴巴和带羽毛的红尾巴,人吃了它的肉就能治愈白癣病。

又东三十五里,曰葱聋之山,其中多大谷,是多白垩(6),黑、青、 黄垩。

【译文】再往东三十五里,是座葱聋山,山中有许多又深又长的峡谷, 到处是白垩土,还有黑垩土、青垩土、黄垩土。

又东十五里,曰涹(w#)山,其上多赤铜,其阴多铁。

【译文】再往东十五里,是座涹山,山上有丰富的黄铜,山北面盛产铁。

又东七十里,曰脱扈(h))之山。有草焉,其状如葵叶而赤华,荚实, 实如棕荚,名曰植楮(ch*),可以已癙(sh()①,食之不眯(m@)②。

【注释】①癙:忧病。②眯:梦魇(y3n)。梦魇就是人在睡梦中遇见可怕的事而呻吟、惊 叫。

【译文】又往东七十里,是座脱扈山。山中有一种草,形状像葵菜的叶 子而是红花,结的是带荚的果实,果实的荚像棕树的果荚,名称是植楮,可 以用它治愈精神抑忧症,而服食它就能使人不做恶梦。

又东二十里,曰金星之山,多天婴①,其状如龙骨②,可以已痤(cu¥) ③。

【注释】①天婴:不详何种植物。②龙骨:据古人讲,在山岩河岸的土穴中常有死龙的脱骨, 而生长在这种地方的植物就叫龙骨。③痤:即痤疮,一种皮肤病。

【译文】再往东二十里,是座金星山,山中有很多天婴,形状与龙骨相 似,可以用来医治痤疮。

又东七十里,曰泰威之山。其中有谷曰枭(xi1o)谷,其中多铁。

【译文】再往东七十里,是座泰威山。山中有一道峡谷叫做枭谷,那里 盛产铁。

又东十五里,曰橿(ji1ng)谷之山,其中多赤铜。

【译文】再往东十五里,是座橿谷山,山中有丰富的黄铜。

又东百二十里,曰吴林之山,其中多葌(ji1n)草①。

【注释】①葌草:葌,同“蕑”,而蕑即兰,则葌草就是兰草。

【译文】再往东一百二十里,是座吴林山,山中生长着茂盛的兰草。

又北三十里,曰牛首之山。有草焉,名曰鬼草,其叶如葵而赤茎,其秀 如禾①,服之不忧。劳水出焉,而西流注于潏(ju6)水,是多飞鱼,其状 如鲋(f))鱼,食之已痔衕(d^ng)。

【注释】①秀:指禾类植物开花。又引申而泛指草木开花。

【译文】再往北三十里,是座牛首山。山中生长着一种草,名称是鬼草, 叶子像葵菜叶却是红色茎干,开的花像禾苗吐穗时的花絮,服食它就能使人 无忧无虑。劳水从这座山发源,然后向西流入潏水,水中有很多飞鱼,形状 像一般的鲫鱼,人吃了它的肉就能治愈痔疮和痢疾。

又北四十里,曰霍山,其木多(穀)[榖(g^u)]。有兽焉,其状如狸 ①,而白尾,有鬣(li6),名曰胐胐(p6i p6i),养之可以已忧。

【注释】①狸:俗称野猫,似狐狸而小一些,身肥胖而短一点。

【译文】再往北四十里,是座霍山,这里到处是茂密的构树。山中有一 种野兽,形状像一般的野猫,却长着白尾巴,脖子上有鬃毛,名称是胐胐, 人饲养它就可以消除忧愁。

又北五十二里,曰合谷之山,是多薝(zh1n)棘①。

【注释】①薝棘:不详何种植物。

【译文】再往北五十二里,是座合谷山,这里到处是薝棘。

又北三十五里,曰阴山,多砺(l0)石、文石。少水出焉。其中多雕棠, 其叶如榆叶而方,其实如赤菽(sh&)①,食之已聋。

【注释】①菽:本义是指大豆,引申为豆类的总称。

【译文】再往北三十五里,是座阴山,多的是粗磨石、色彩斑斓的漂亮 石头。少水从这座山发源。山中有茂密的雕棠树,叶子像榆树叶却呈四方形, 结的果实像红豆,服食它就能治愈人的耳聋病。

又东四百里,曰鼓镫(d5ng)之山,多赤铜。有草焉,名曰荣草,其叶 如柳,其本如鸡卵,食之已风。

【译文】再往东四百里,是座鼓镫山,有丰富的黄铜。山中有一种草, 名称是荣草,叶子与柳树叶相似,根茎与鸡蛋相似,人吃了它就能治愈风痹 病。

凡薄山之首,自甘枣之山至于鼓镫(d5ng)之山,凡十五山,六千六百 七十里。历儿,冢也,其祠礼:毛,太牢之具,县(xu2n)以吉玉①。其余 十三山者,毛用一羊,县(xu2n)婴用(桑封)[藻珪]②,瘗(y@)而不糈 (x()。(桑封)[藻珪]者②,(桑主)[藻玉]也,方其下而锐其上③,而 中穿之加金④。

【注释】①县:同“悬”。吉玉:古人往往在人、事、物等有关词语前贯以“吉”字,用来 表示对其美称。这里的吉玉就是一种美称,意思是美好的玉。②藻珪:即用带有色彩斑纹的玉石制成 的玉器。③锐:上小下大。这里指三角形尖角。④据学者研究,“藻珪者”以下的几句话,原本是古 人的解释性语句,不知何时窜入正文。因底本如此,今姑仍其旧。

【译文】总计薄山山系之首尾,自甘枣山起到鼓镫山止,一共十五座山, 途经六千六百七十里。历儿山,是诸山的宗主,祭祀宗主山山神:在毛物中, 用猪、牛、羊齐全的三牲作祭品,再悬挂上吉玉献祭。祭祀其余十三座山的 山神,在毛物中用一只羊作祭品,再悬挂上祀神玉器中的藻珪献祭,祭礼完 毕把它埋入地下而不用米祀神。所谓藻珪,就是藻玉,下端呈长方形而上端 有尖角,中间有穿孔并加上金饰物。

中次二(经)[山]济山之首,曰辉诸之山,其上多桑,其兽多闾(lǘ) 麋①,其鸟多鹖(h6)②。

【注释】①闾:就是前文所说的形状像驴而长着羚羊角的山驴。②鹖:鹖鸟。据古人说,鹖 鸟像野鸡而大一些,羽毛青色,长有毛角,天性勇猛好斗,绝不退却,直到斗死为止。

【译文】中央第二列山系济山山系之首座山,叫做辉诸山,山上有茂密 的桑树,山中的野兽大多是山驴和麋鹿,而禽鸟大多是鹖鸟。

又西南二百里,曰发视之山,其上多金玉,其下多砥(d!)砺(l0)。

即鱼之水出焉,而西流注于伊水。

【译文】再往西南二百里,是座发视山,山上有丰富的金属矿物和玉石, 山下多出产磨石。即鱼水从这座山发源,然后向西流入伊水。

又西三百里,曰豪山,其上多金玉而无草木。

【译文】再往西三百里,是座豪山,山上有丰富的金属矿物和玉石而没 有花草树木。

又西三百里,曰鲜山,多金玉,无草木。鲜水出焉,而北流注于伊水。

其中多鸣蛇,其状如蛇而四翼,其音如磬(q@ng),见(xi4n)则其邑大旱。

【译文】再往西三百里,是座鲜山,有丰富的金属矿物和玉石,但不生 长花草树木。鲜水从这座山发源,然后向北流入伊水。水中有很多鸣蛇,形 状像一般的蛇却长着四只翅膀,叫声如同敲磐的声音,在哪个地方出现那里 就会发生大旱灾。

又西三百里,曰阳山,多石,无草木。阳水出焉,而北流注于伊水。其 中多化蛇,其状如人面而豺(ch2i)身①,鸟翼而蛇行②,其音如叱呼,见 (xi4n)则其邑大水。

【注释】①豺:一种凶猛的动物,比狼小一些,体色一般是棕红,尾巴的末端是黑色,腹部 和喉部是白色。②蛇行:蜿蜒曲折地伏地爬行。

【译文】再往西三百里,是座阳山,到处是石头,没有花草树木。阳水 从这座山发源,然后向北流入伊水。水中有很多化蛇,形状是人的面孔却长 着豺一样的身子,有禽鸟的翅膀却像蛇一样地爬行,发出的声音如同人在呵 斥,在哪个地方出现那里就会发生水害。

又西二百里,曰昆吾之山,其上多赤铜①。有兽焉,其状如彘(zh@) 而有角,其音如号,名曰■(l^ng)蚳(ch0),食之不眯。

【注释】①赤铜:指传说中的昆吾山所特有的一种铜,色彩鲜红,如同赤火一般。用这里生 产的赤铜所制做的刀剑,是非常锋利的,切割玉石如同削泥一样。所谓神奇的昆吾之剑,就是由这种 铜打造的。

【译文】再往西二百里,是座昆吾山,山上有丰富的赤铜。山中有一种 野兽,形状像一般的猪却长着角,发出的声音如同人号啕大哭,名称是■蚳, 吃了它的肉就会使人不做恶梦。

又西百二十里,曰葌(ji1n)山。葌水出焉,而北流注于伊水。其上多 金玉,其下多青、雄黄。有木焉,其状如棠而赤叶,名曰芒(w4ng)草①, 可以毒鱼。

【注释】①芒草:又作莽草、■草,也可单称为芒,一种有毒性的草,与另一种类似于茅草 而大一些的芒草是同名异物。可能芒草长得高大如树,所以这里称它为树木,其实是草。

【译文】再往西一百二十里,是座葌山。葌水从这座山发源,然后向北 流入伊水。山上盛产金属矿物和玉石,山下盛产石青、雄黄。山中有一种树 木,形状像棠梨树而叶子是红色的,名称是芒草,能够毒死鱼。

又西一百五十里,曰独苏之山,无草木而多水。

【译文】再往西一百五十里,是座独苏山,这里没有花草树木而到处是 水流。

又西二百里,曰蔓渠之山,其上多金玉,其下多竹箭。伊水出焉,而东 流注于洛。有兽焉,其名曰马腹,其状如人面虎身,其音如婴儿,是食人。

【译文】再往西二百里,是座蔓渠山,山上有丰富的金属矿物和玉石, 山下到处是小竹丛。伊水从这座山发源,然后向东流入洛水。山中有一种野 兽,名称是马腹,形状是人一样的面孔虎一样的身子,发出的声音如同婴儿 啼哭,是能吃人的。

凡济山之首,自辉诸之山至于蔓渠之山,凡九山,一千六百七十里。其 神皆人面而鸟身。祠用毛,用一吉玉,投而不糈(x()。

【译文】总计济山山系之首尾,自辉诸山起到蔓渠山止,一共九座山, 途经一千六百七十里。诸山山神的形状都是人的面孔鸟的身子。祭祀山神要 用毛物作祭品,再用一块吉玉,把这些投向山谷而不用米祀神。

中次三(经)[山]萯(b6i)山之首,曰敖岸之山,其阳多■(y()琈 之玉,其阴多赭(zh7)、黄金。神熏池居之。是常出美玉。北望河林,其 状如茜(qi4n)如举①。有兽焉,其状如白鹿而四角,名曰夫诸,见(xi4n) 则其邑大水。

【注释】①茜:茜草,一种多年生攀援草本植物,根是黄红色,可作染料。举:即榉柳,落 叶乔木,生长得又快又高大,木材坚实,用途很广。

【译文】中央第三列山系萯山山系之首座山,叫做敖岸山,山南面多出 产■琈玉,山北面多出产赭石、黄金。天神熏池住在这里。这座山还常常生 出美玉来。从山上向北可以望见奔腾的黄河和葱郁的丛林,它们的形状好像 是茜草和榉柳。山中有一种野兽,形状像一般的白鹿却长着四只角,名称是 夫诸,在哪个地方出现那里就会发生水灾。

又东十里,曰青要之山,实惟帝之密都①。北望河曲,是多(驾)[鴐 (ji1)]鸟②。南望墠(ti2n)渚,禹父之所化③,是多仆累、蒲卢④。■ (sh5n)武罗司之⑤,其状人面而豹文,小要而白齿⑥,而穿耳以鐻(q*) ⑦,其鸣如鸣玉。是山也,宜女子。畛(zh7n)水出焉,而北流注于河。其 中有鸟焉,名曰鴢(y3o),其状如凫(f*),青身而朱目赤尾,食之宜子。

有草焉,其状如葌(ji1n),而方茎黄华赤实,其本如藁(g3o)本⑧,名 曰荀草,服之美人色。

【注释】①密都:隐密深邃的都邑。②鴐鸟:即鴐鹅,俗称野鹅。③禹父:指大禹的父亲鲧 (g(n)。相传禹是夏朝的开国国王。④仆累:即蜗牛,一种软体动物,栖息于潮湿的地方。蒲卢: 一种具有圆形贝壳的软体动物,属蛤(g6)、蚌之类。⑤■:一说是神鬼,即鬼中的神灵;一说是山 神。⑥要:“腰”的本字。⑦鐻:金银制成的耳环。⑧藁本:也叫抚芎(xi#ng)、西芎,一种香草, 根茎含挥发油,可作药用。

【译文】再往东十里,是座青要山,确实是天帝的密都。从青要山上向 北可以望见黄河的弯曲处,这里有许多野鹅。从青要山向南可以望见墠渚, 是大禹的父亲鲧变化成为黄熊的地方,这里有很多蜗牛、蒲卢。山神武罗掌 管着这里,这位山神的形貌是人的面孔却浑身长着豹子一样的斑纹,细小的 腰身洁白的牙齿,而且耳朵上穿挂着金银环,发出的声音像玉石碰击作响。

这座青要山,适宜女子居住。畛水从这座山发源,然后向北流入黄河。山中 有一种禽鸟,名称是鴢,形状像野鸭子,青色的身子却是浅红色的眼睛深红 色的尾巴,吃了它的肉就能使人多生孩子。山中生长着一种草,形状像兰草, 却是四方形的茎干黄色的花朵、红色的果实,根部像藁本的根,名称是荀草, 服用它就能使人的肤色洁白漂亮。

又东十里,曰騩(gu9)山,其上有美枣,其阴有■(y()琈之玉。正 回之水出焉,而北流注于河。其中多飞鱼①,其状如豚(t*n)而赤文,服 之不畏雷,可以御兵②。

【注释】①飞鱼:与上文所述飞鱼的形状不同,当为同名异物。②兵:指兵器的锋刃。

【译文】再往东十里,是座騩山,山上盛产味道甜美的枣子,山北阴面 还盛产■琈玉。正回水从这座山发源,然后向北流入黄河。水中生长着许多 飞鱼,形状像小猪却浑身是红色斑纹,吃了它的肉就能使人不怕打雷,还可 以辟兵器。

又东四十里,曰宜苏之山,其上多金玉,其下多蔓(居)[荆]之木①。

滽滽(r¥ngr¥ng)之水出焉,而北流注于河,是多黄贝。

【注释】①蔓荆:一种灌木,长在水边,苗茎蔓延,高一丈多,六月开红白色花,九月结成 的果实上有黑斑,冬天则叶子凋落。

【译文】再往东四十里,是座宜苏山,山上有丰富的金属矿物和玉石, 山下有繁茂的蔓荆。滽滽水从这座山流出,然后向北流入黄河,水中有很多 黄色的贝。

又东二十里,曰和山,其上无草木而多瑶、碧,实惟河之九都①。是山 也五曲,九水出焉,合而北流注于河,其中多苍玉。吉神泰逢司之②,其状 如人而虎尾,是好居于萯(b6i)山之阳,出入有光。泰逢神动天地气也。

【注释】①都:汇聚。②吉神:对神的美称,即善神的意思。

【译文】再往东二十里,是座和山,山上不生长花草树木而到处是瑶、 碧一类的美玉,确实是黄河中的九条水源所汇聚的地方。这座山盘旋回转了 五层,有九条水从这里发源,然后汇合起来向北流入黄河,水中有很多苍玉。

吉神泰逢主管这座山,他的形貌像人却长着虎一样的尾巴,喜欢住在萯山向 阳的南面,出入时都有闪光。泰逢这位吉神能兴起风云。

凡萯(b6i)山之首,自敖岸之山至于和山,凡五山,四百四十里。其 祠:泰逢、熏池、武罗皆一牡(m()羊副(p@)①,婴用吉玉。其二神用一 雄鸡瘗(y@)之。糈(x()用稌(t*)。

【注释】①副:裂开,剖开。

【译文】总计萯山山系之首尾,自敖岸山起到和山止,一共五座山,途 经四百四十里。祭祀诸山山神:泰逢、熏池、武罗三位神都是把一只公羊劈 开来祭祀,祀神的玉器要用吉玉。其余二位山神是用一只公鸡献祭后埋入地 下。祀神的米用稻米。

中次四(经)[山]厘山之首,曰鹿蹄之山,其上多玉,其下多金。甘水 出焉,而北流注于洛,其中多(泠)[汵](j9n)石①。

【注释】①汵石:一种柔软如泥的石头。

【译文】中央第四列山系厘山山系之首座山,叫做鹿蹄山,山上盛产玉, 山下盛产金。甘水从这座山发源,然后向北流入洛水,水中有很多汵石。

西五十里,曰扶猪之山,其上多礝(ru3n)石①。有兽焉,其状如貉(h6) 而人目②,其名曰貉(y0n)。虢(gu¥)水出焉,而北流注于洛,其中多瓀 (ru3n)石③。

【注释】①礝:也写成“碝”、“瓀”。礝石是次于玉一等的美石。白色的礝石如冰一样透 明,而水中的礝石是红色的。②貉:也叫狗獾,是一种野兽。外形像狐狸而体态较肥胖,尾巴较短, 尾毛蓬松,耳朵短而圆,两颊有长毛,体色棕灰。③瓀石:就是礝石。

【译文】往西五十里,是座扶猪山,山上到处是礝石。山中有一种野兽, 形状像貉却长着人的眼睛,名称是■。虢水从这座山发源,然后向北流入洛 水,水中有很多礝石。

又西一百二十里,曰厘山,其阳多玉,其阴多蒐(s#u)①。有兽焉, 其状如牛,苍身,其音如婴儿,是食人,其名曰犀(x9)渠。滽滽(r¥ngr¥ng) 之水出焉,而南流注于伊水。有兽焉,名曰■(ji6),其状如獳(n^u)犬 而有鳞②,其毛如彘(zh@)鬣(li6)。

【注释】①蒐:即茅蒐,现在称作茜(qi4n)草。它的根是紫红色,可作染料,并能入药。

②獳犬:发怒样子的狗。

【译文】再往西一百二十里,是座厘山,山南面有很多玉石,山北面有 茂密的茜草。山中有一种野兽,形状像一般的牛,全身青黑色,发出的声音 如同婴儿啼哭,是能吃人的,名称是犀渠。滽滽水从这座山发源,然后向南 流入伊水。这里还有一种野兽,名称是■,形状像獳犬却全身有鳞甲,长在 鳞甲间的毛像猪鬃一样。

又西二百里,曰箕尾之山,多(穀)[榖(g^u)],多涂石①,其上多 ■(y()琈之玉。

【注释】①涂石:就是上文所说的汵石,石质如泥一样柔软。

【译文】再往西二百里,是座箕尾山,有茂密的构树,盛产涂石,山上 还有许多■琈玉。

又西二百五十里,曰柄山,其上多玉,其下多铜。滔雕之水出焉,而北 流注于洛。其中多羬(xi2n)羊。有木焉,其状如樗(ch&),其叶如桐而 荚实,其名曰茇(b2)①,可以毒鱼。

【注释】①茇:学者认为“茇”可能是“芫”的误写。芫即芫华,也叫芫花,是一种落叶灌 木,春季先开花,后生叶,花蕾可入药,根茎有毒性。

【译文】再往西二百五十里,是座柄山,山上盛产玉,山下盛产铜。滔 雕水从这座山发源,然后向北流入洛水。山中有许多羬羊。山中还有一种树 木,形状像臭椿树,叶子像梧桐叶而结出带荚的果实,名称是茇,是能毒死 鱼的。

又西二百里,曰白边之山,其上多金玉,其下多青、雄黄。

【译文】再往西二百里,是座白边山,山上有丰富的金属矿物和玉石, 山下盛产石青、雄黄。

又西二百里,曰熊耳之山,其上多漆,其下多棕。浮濠之水出焉,而西 流注于洛,其中多水玉,多人鱼。有草焉,其状如苏而赤华①,名曰葶(t0ng) 苧(n@ng),可以毒鱼。

【注释】①苏:即紫苏,又叫山苏,一年生草本植物,茎干呈方形,叶子紫红色。枝、叶、 茎、果都可作药用。

【译文】再往西二百里,是座熊耳山,山上是茂密的漆树,山下是茂密 的棕树。浮濠水从这座山发源,然后向西流入洛水,水中有很多水晶石,还 有很多人鱼。山中有一种草,形状像苏草而开红花,名称是葶苧,是能毒死 鱼的。

又西三百里,曰牡(m()山,其上多文石,其下多竹箭、竹■(m6i)。

其兽多■(zu¥)牛、羬(xi2n)羊,鸟多赤鷩(b9e)①。

【注释】①赤鷩:即鷩雉,也叫锦鸡,像野鸡而小一些,冠子羽毛都很美,五色艳丽。和上 文所说的赤鷩同属野鸡的种类,形状大同小异,故名称上也往往混同。

【译文】再往西三百里,是座牡山,山上到处是色彩斑斓的漂亮石头, 山下到处是竹箭、竹■之类的竹丛。山中的野兽以■牛、羬羊最多,而禽鸟 以赤鷩最多。

又西三百五十里,曰讙(hu1n)举之山。雒(lu^)水出焉,而东北流 注于玄扈之水。其中多马肠之物①。此二山者,洛间也。

【注释】①其中:指玄扈山中。据《水经注?洛水》,知玄扈水发源于玄扈山。可见,此处 是有省文的。马肠:即上文所说的怪兽马腹,人面虎身,叫声如婴儿哭,吃人。

【译文】再往西三百五十里,是讙举山。雒水从这座山发源,然后向东 北流入玄扈水。玄扈山中生有很多马肠这样的怪物。在讙举山与玄扈山之 间,夹着一条洛水。

凡厘山之首,自鹿蹄之山至于玄扈之山,凡九山,千六百七十里。其神 状皆人面兽身。其祠之:毛用一白鸡,祈而不糈(x(),以采衣(y@)之①。

【注释】①衣:用作动词,穿的意思。这里是包裹的意思。

【译文】总计厘山山系之首尾,自鹿蹄山起到玄扈山止,一共九座山, 途经一千六百七十里。诸山山神的形貌都是人的面孔而兽的身子。祭祀山 神:在毛物中用一只白色鸡献祭,祀神不用米,用彩色帛把鸡包裹起来。

中次五(经)[山]薄山之首,曰苟(床)[林]之山,无草木,多怪石。

【译文】中央第五列山系薄山山系之首座山,叫做苟林山,不生长花草 树木,到处是奇形怪状的石头。

东三百里,曰首山,其阴多(穀)[榖](g^u)、柞(zu^)①,其草多 ■(zh*)、芫②。其阳多■(y()琈之玉,木多槐。其阴有谷,曰机谷, 多■(d@)鸟,其状如枭(xi1o)而三目,有耳,其音如(录)[鹿],食之 已垫③。

【注释】①柞:柞树,也叫蒙子树、凿刺树、冬青,常绿灌木,初秋开花,雌雄异株,花小, 黄白色,浆果小球形,黑色。②■:即山蓟,是一种可作药用的草,又分为苍术(zh*)、白术(zh*) 二种。苍术是多年生直立草本植物,可以入药。白术是多年生草本植物,根状茎可以入药。芫:即芫 华,其实是落叶灌木,因树形矮小,被看作草。花可以药用,根可以毒死鱼。③垫:一种因低下潮温 而引发的疾病。

【译文】往东三百里,是座首山,山北面有茂密的构树、柞树,这里的 草以■草、芫华居多。山南面盛产■琈玉,这里的树木以槐树居多。这座山 的北面有一峡谷,叫做机谷,峡谷里有许多■鸟,形状像猫头鹰却长着三只 眼睛,还有耳朵,发出的声音如同鹿鸣叫,人吃了它的肉就会治好湿气病。

又东三百里,曰县■(zh&)之山,无草木,多文石。

【译文】再往东三百里,是座县■山,没有花草树木,到处是色彩斑斓 的漂亮石头。

又东三百里,曰葱聋之山,无草木,多■(b4ng)石①。

【注释】①■石:即玤石,是次于玉石一等的石头。

【译文】再往东三百里,是座葱聋山,没有花草树木,到处是■石。

东北五百里,曰条谷之山,其木多槐桐,其草多芍(sh2o)药、■(m6n) 冬①。

【注释】①芍药:多年生草本植物,初夏开花,与牡丹花相似,可供观赏,而根茎可以入药。

■冬:俗作门冬,有两种,一是麦门冬,也叫沿阶草,多年生常绿草本植物,须根常膨大成纺锤形, 可以作药用;二是天门冬,也叫天冬草,多年生攀援草本植物,地下有簇生纺锤形肉质块根,可以作 药用。

【译文】往东北五百里,是座条谷山,这里的树木大多是槐树和桐树, 而草大多是芍药、门冬草。

又北十里,曰超山,其阴多苍玉,其阳有井①,冬有水而夏竭。

【注释】①井:井是人工开挖的,泉是自然形成的,而本书记述的山之所有皆为自然事物, 所以,这里的井当是指泉眼下陷而低于地面的水泉,形似水井,故称。

【译文】再往北十里,是座超山,山北面到处是苍玉,山南面有一眼水 泉,冬天有水而到夏天就干枯了。

又东五百里,曰成侯之山,其上多櫄(ch&n)木①,其草多(芃)[艽] (ji1o)②。

【注释】①櫄木:据古人说,这种树与高大的臭椿树相似,树干可以作车辕。②艽:就是秦 艽,一种可作药用的草。

【译文】再往东五百里,是座成侯山,山上是茂密的櫄树,这里的草以 秦艽居多。

又东五百里,曰朝歌之山,谷多美垩(6)。

【译文】再往东五百里,是座朝歌山,山谷里多出产优良垩土。

又东五百里,曰槐山,谷多金锡①。

【注释】①锡:这里指天然锡矿石,而非提炼的纯锡。以下同此。

【译文】再往东五百里,是座槐山,山谷里有丰富的金和锡。

又东十里,曰历山,其木多槐,其阳多玉。

【译文】再往东十里,是座历山,这里的树大多是槐树,山南面多出产 玉石。

又东十里,曰尸山,多苍玉,其兽多麖①(j9ng)。尸水出焉,南流注 于洛水,其中多美玉。

【注释】①麖:鹿的一种,体型较大。

【译文】再往东十里,是座尸山,到处是苍玉,这里的野兽以麖居多。

尸水从这座山发源,向南流入洛水,水中有很多优良玉石。

又东十里,曰良余之山,其上多(穀)[榖](g^u)柞(zu^),无石。

余水出于其阴,而北流注于河;乳水出于其阳,而东南流注于洛。

【译文】再往东十里,是座良余山,山上有茂密的构树和柞树,没有石 头。余水从良余山北麓流出,然后向北流入黄河;乳水从良余山南麓流出, 然后向东南流入洛水。

又东南十里,曰蛊(g()尾之山,多砺(l0)石、赤铜。龙余之水出焉, 而东南流注于洛。

【译文】再往东南十里,是座蛊尾山,盛产粗磨石、黄铜。龙余水从这 座山发源,然后向东南流入洛水。

又东北二十里,曰升山,其木多(穀)[榖(g^u)]、柞(zu^)、棘, 其草多薯(sh()■(y))蕙①,多寇脱②。黄酸之水出焉,而北流注于河, 其中多璇(xu2n)玉③。

【注释】①薯■:也叫山药。它的块茎不仅可以食用,并且可作药用。蕙:一种香草。②寇 脱:古人说是一种生长在南方的草,有一丈多高,叶子与荷叶相似,茎中有瓤,纯白色。③璇玉:古 人说是质料成色比玉差一点的玉石。

【译文】再往东北二十里,是座升山,这里的树以构树、柞树、酸枣树 居多,而草以山药、惠草居多,还有茂密的寇脱草。黄酸水从这座山发源, 然后向北流入黄河,水中有很多璇玉。

又东二十里,曰阳虚之山,多金,临于玄扈之水。

【译文】再往东二十里,是座阳虚山,盛产金,阳虚山临近玄扈水。

凡薄山之首,自苟林之山至于阳虚之山,凡十六山,二千九百八十二里。

升山,冢(zh%ng)也,其祠礼:太牢,婴用吉玉。首山,■(sh6n)也①, 其祠用稌(t*)、黑牺太牢之具、■(ni6)酿②;干儛③,置鼓;婴用一 璧。尸水,合天也,肥牲祠之;用一黑犬于上,用一雌鸡于下,刉(j9)一 牝(p@n)羊④,献血。婴用吉玉,采之,飨之。

【注释】①■:神灵。②■酿:■,俗作“■”。酒曲,酿酒用的发酵剂。■酿就是用曲■ 酿造的醴(l!)酒。这里泛指美酒。③干儛:古代在举行祭祀活动时跳的一种舞蹈。干,即盾牌,是 古代一种防御性兵器。儛,同“舞”。干儛就是手拿盾牌起舞,表示庄严隆重。④刉:亦作“刏”。

划破,割。

【译文】总计薄山山系之首尾,自苟林山起到阳虚山止,一共十六座山, 途经二千九百八十二里。升山,是诸山的宗主,祭祀升山山神的典礼:在毛 物中用猪、牛、羊齐全的三牲作祭品,祀神的玉器要用吉玉。首山,是神灵 显应的大山,祭祀首山山神用稻米、整只黑色皮毛的猪、牛、羊、美酒;手 持盾牌起舞,摆上鼓并敲击应和;祀神的玉器用一块玉璧。尸水,是上通到 天的,要用肥壮的牲畜作祭品献祭;用一只黑狗作祭品供在上面,用一只母 鸡作祭品供在下面,杀一只母羊,献上血。祀神的玉器要用吉玉,并用彩色 帛包装祭品,请神享用。

中次六(经)[山]缟(g3o)羝(d0)山之首,曰平逢之山,南望伊洛, 东望谷城之山,无草木,无水,多沙石。有神焉,其状如人而二首,名曰骄 虫,是为螫(zh6)虫①,实惟蜂、蜜之庐②。其祠之,用一雄鸡,禳(r2ng) 而勿杀③。

【注释】①螫虫:指一切身上长有毒刺能伤人的昆虫。②蜜:也是一种蜂。③禳:祭祀祈祷 神灵以求消除灾害。

【译文】中央第六列山系缟羝山山系之首座山,叫做平逢山,从平逢山 上向南可以望见伊水和洛水,向东可以望见谷城山,这座山不生长花草树 木,没有水,到处是沙子石头。山中有一山神,形貌像人却长着两个脑袋, 叫做骄虫,是所有螫虫的首领,也确实是各种蜜蜂聚集做巢的地方。祭祀这 位山神,用一只公鸡作祭品,在祈祷后放掉而不杀。

西十里,曰缟(g3o)羝(d0)之山,无草木,多金玉。

【译文】往西十里,是座缟羝山,没有花草树木,有丰富的金属矿物和 玉石。

又西十里,曰廆(gu9)山,多■(y()琈之玉。其阴有谷焉,名曰雚 (gu4n)谷,其木多柳、楮(ch(),其中有鸟焉,状如山鸡而长尾,赤如 丹火而青喙(hu@),名曰鸰(l0ng)■(y4o),其鸣自呼,服之不眯。交 觞(sh1ng)之水出于其阳,而南流注于洛;俞随之水出于其阴,而北流注 于谷水。

【译文】再往西十里,是座廆山,山上盛产■浮玉。在这座山的阴面有 一道峡谷,叫做雚谷,这里的树木大多是柳树、构树。山中有一种禽鸟,形 状像野鸡却拖着一条长长的尾巴,身上通红如火却是青色嘴巴,名称是鸰 ■,它发出的叫声便是自身名称的读音,吃了它的肉就能使人不做恶梦。交 觞水从这座山的南麓流出,然后向南流入洛水;俞随水从这座山的北麓流 出,然后向北流入谷水。

又西三十里,曰瞻诸之山,其阳多金,其阴多文石。■(xi6)水出焉, 而东南流注于洛;少水出[于]其阴,而东流注于谷水。

【译文】再往西三十里,是座瞻诸山,山南面盛产金属矿物,山北面盛 产带有花纹的漂亮石头。■水从这座山发源,然后向东南流入洛水;少水从 这座山的北麓流出,然后向东流入谷水。

又西三十里,曰娄涿(zhu#)之山,无草木,多金玉。瞻水出于其阳, 而东流注于洛;陂(b5i)水出于其阴,而北流注于谷水,其中多茈(z!) 石、文石。

【译文】再往西三十里,是座娄涿山,没有花草树木,有丰富的金属矿 物和玉石。瞻水从这座山的南麓流出,然后向东流入洛水;陂水从这座山的 北麓流出,然后向北流入谷水,水中有很多紫颜色的石头、带有花纹的漂亮 石头。

又西四十里,曰白石之山。惠水出于其阳,而南流注于洛,其中多水玉。

涧水出于其阴,西北流注于谷水,其中多麋(m6i)石、栌丹①。


山海经卷五 中山经 -2

【注释】①麋石:麋,通“眉”,眉毛。麋石即画眉石,一种可以描饰眉毛的矿石。栌丹: 栌,通“卢”。卢是黑色的意思。卢丹即黑丹沙,一种黑色矿物。

【译文】 再往西四十里,是座白石山。惠水从白石山的南麓流出,然 后向南流入洛水,水中有很多水晶石。涧水从白石山的北麓流出,向西北流 入谷水,水中有很多画眉石、黑丹沙。

又西五十里,曰谷山,其上多(穀)[榖(g^u)],其下多桑。爽水出 焉,而西北流注于谷水,其中多碧绿①。

【注释】①碧绿:据学者研究,可能指现在所说的孔雀石,色彩艳丽,可以制做装饰品和绿 色涂料。

【译文】再往西五十里,是座谷山,山上是茂密的构树,山下是茂密的 桑树。爽水从这座山发源,然后向西北流入谷水,水中有很多孔雀石。

又西七十二里,曰密山,其阳多玉,其阴多铁。豪水出焉,而南流注于 洛,其中多旋龟,其状鸟首而鳖尾,其音如判木。无草木。

【译文】再往西七十二里,是座密山,山南面盛产玉,山北面盛产铁。

豪水从这座山发源,然后向南流入洛水,水中有很多旋龟,形状像鸟一样的 头而鳖一样的尾巴,发出的声音好像劈木头声。这座山不生长花草树木。

又西百里,曰长石之山,无草木,多金玉。其西有谷焉,名曰共谷,多 竹。共水出焉,西南流注于洛,其中多鸣石①。

【注释】 ①鸣石:古人说是一种青色玉石,撞击后发出巨大鸣响,七八里以外都能听到, 属于能制做乐器的磬石之类。

【译文】 再往西一百里,是座长石山,没有花草树木,有丰富的金属 矿物和玉石。这座山的西面有一道峡谷,叫做共谷,生长许多竹子。共水从 这座山发源,向西南流入洛水,水中多产鸣石。

又西一百四十里,曰傅山,无草木,多瑶、碧。厌染之水出于其阳,而 南流注于洛,其中多人鱼。其西有林焉,名曰墦(f1n)冢(zh%ng)。谷水 出焉,而东流注于洛,其中多珚(y1n)玉①。

【注释】 ①珚玉:玉的一种。

【译文】再往西一百四十里,是座傅山,没有花草树木,到处是瑶、碧 之类的美玉。厌染水从这座山的南麓流出,然后向南流入洛水,水中有很多 人鱼。这座山的西面有一片树林,叫做墦冢。谷水从这里流出,然后向东南 流入洛水,水中有很多珚玉。

又西五十里,曰橐(tu#)山,其木多樗(ch&),多■(b6i)木①, 其阳多金玉,其阴多铁,多萧②。橐水出焉,而北流注于河。其中多脩辟之 鱼,状如黾(m7ng)而白喙(hu@)③,其音如鸱(ch9),食之已白癣。

【注释】①■木:古人说这种树在七、八月间吐穗,穗成熟后,像似有盐粉沾在上面。②萧: 蒿草的一种。③黾:青蛙的一种。

【译文】再往西五十里,是座橐山,山中的树木大多是臭椿树,还有很 多■树,山南面有丰富的金属矿物和玉石,山北面有丰富的铁,还有茂密的 萧草。橐水从这座山发源,然后向北流入黄河。水中有很多脩辟鱼,形状像 一般的蛙却长着白色嘴巴,发出的声音如同鹞鹰鸣叫,人吃了它的肉就能治 愈白癣病。

又西九十里,日常烝(zh5ng)之山,无草木,多垩。潐(qi2o)水出 焉,而东北流注于河,其中多苍玉。菑(z9)水出焉,而北流注于河。

【译文】再往西九十里,是座常烝山,没有花草树木,有多种颜色的垩 土。潐水从这座山发源,然后向东北流入黄河,水中有很多苍玉。菑水也从 这座山发源,然后向北流入黄河。

又西九十里,曰夸父之山,其木多棕枏(n2n),多竹箭,其兽多■(zu¥) 牛、羬羊,其鸟多[赤]鷩(bi5),其阳多玉,其阴多铁。其北有林焉,名 曰桃林,是广员三百里,其中多马。湖水出焉,而北流注于河,其中多珚(y1n) 玉。

【译文】 再往西九十里,是座夸父山,山中的树木以棕树和楠木树最 多,还有茂盛的小竹丛,山中的野兽,以■牛、羬羊最多,而禽鸟以赤鷩最 多,山南面盛产玉,山北面盛产铁。这座山北面有一片树林,叫做桃林,这 片树林方圆三百里,林子里有很多马。湖水从这座山发源,然后向北流入黄 河,水中多出产珚玉。

又西九十里,曰阳华之山,其阳多金玉,其阴多青、雄黄,其草多薯■, 多苦辛,其状如■(xi1o)①,其实如瓜,其味酸甘,食之已疟(nü6)。

杨水出焉,而西南流注于洛。其中多人鱼。门水出焉,而东北流注于河,其 中多玄■(s))。■(j0)姑之水出于其阴,而东流注于门水,其上多铜。

门水(出)[至]于河,七百九十里入■(lu^)水。

【注释】 ①■:同“楸”。楸树是落叶乔木,树形高大,树干端直。夏季开花,子实可作 药用,主治热毒及各种疮疥。

【译文】再往西九十里,是座阳华山,山南面有丰富的金属矿物和玉石, 山北面盛产石青、雄黄,山中的草以山药最多,还有茂密的苦辛草,形状像 楸木,结的果实像瓜,味道是酸中带甜,人服食它就能治愈疟疾。杨水从这 座山发源,然后向西南流入洛水,水中有很多人鱼。门水也从这座山发源, 然后向东北流入黄河,水中有很多黑色磨石。■姑水从阳华山北麓流出,然 后向东流入门水,■姑水两岸山间有丰富的铜。从门水到黄河,流经七百九 十里后注入■水。

凡缟(g3o)羝(d0)山之首,自平逢之山至于阳华之山,凡十四山, 七百九十里。岳在其中①,以六月祭之,如诸岳之祠法,则天下安宁。

【注释】①岳:高大的山。

【译文】 总计缟羝山山系之首尾,自平逢山起到阳华山止,一共十四 座山,途经七百九十里。有大山岳在这一山系中,在每年六月祭祀它,一如 祭祀其它山岳的方法,那么天下就会安宁。

中次七(经)[山]苦山之首,曰休与之山。其上有石焉,名曰帝台之棋 ①,五色而文,其状如鹑卵。帝台之石,所以祷百神者也,服之不蛊(g()。

有草焉,其状如蓍(sh@)②,赤叶而本丛生,名曰夙(s))条,可以为簳 (g3n)③。

【注释】①帝台:神人之名。棋:指博棋,古时一种游戏用具。②蓍:蓍草,又叫锯齿草, 蚰蜒草,多年生直立草本植物,叶互生,长线状披针形。古人取蓍草的茎作占筮之用。③簳:小竹子, 可以做箭杆。

【译文】 中央第七列山系苦山山系之首座山,叫做休与山。山上有一 种石子,是神仙帝台的棋,它们有五种颜色并带着斑纹,形状与鹌鹑蛋相似。

神仙帝台的石子,是用来祷祀百神的,人佩带上它就会不受邪毒之气侵染。

休与山还有一种草,形状像一般的蓍草,红色的叶子而根茎连结丛生在一 起,名称是夙条,可以用来做箭杆。

东三百里,曰鼓钟之山,帝台之所以觞(sh1ng)百神也①。有草焉, 方茎而黄华,员叶而三成②,其名曰焉酸,可以为毒③。其上多砺(l0), 其下多砥(d!)。

【注释】①觞:向人敬酒或自饮。这里指设酒席招待。②成:重,层。③为毒:除去毒性物 质。

【译文】往东三百里,是座鼓钟山,神仙帝台正是在此演奏钟鼓之乐而 宴会诸位天神的。山中有一种草,方形的茎干上开着黄色花朵,圆形的叶子 重叠为三层,名称是焉酸,可以用来解毒。山上多出产粗磨石,山下多出产 细磨石。

又东二百里,曰姑媱(y2o)之山。帝女死焉,其名曰女尸,化为■(y2o) 草,其叶胥(x&)成①,其华黄,其实如菟(t*)丘②,服之媚(m6i)于 人③。

【注释】①胥:相与,皆。②菟丘:即菟丝子,一年生缠绕寄生草本植物,茎细柔,呈丝状, 橙黄色,夏秋开花,花细小,白色,果实扁球形。③媚于人:这里指女子以美色讨人欢心。媚是喜爱 的意思。

【译文】 再往东二百里,是座姑媱山,天帝的女儿就死在这座山,她 的名字叫女尸,死后化成了■草,叶子都是一层一层的,花儿是黄色的,果 实与菟丝子的果实相似,女子服用了就能使她漂亮而讨人喜爱。

又东二十里,曰苦山。有兽焉,名曰山膏,其状如(逐)[豚(t*n)], 赤若丹火,善詈(l@)①。其上有木焉,名曰黄棘,黄华而员叶,其实如兰, 服之不字②。有草焉,员叶而无茎,赤华而不实,名曰无条③,服之不瘿。

【注释】 ①詈:骂,责骂。②字:怀孕,生育。③无条:与上文所述无条草的形状不一样, 属同名异物。

【译文】 再往东二十里,是座苦山。山中有一种野兽,名称是山膏, 形状像普通的小猪,身上红得如同丹火,喜欢骂人。山上有一种树木,名称 是黄棘,黄色花而圆叶子,果实与兰草的果实相似,女人服用了它就不生育 孩子。山中又有一种草,圆圆的叶子而没有茎干,开红色花却不结果实,名 称是无条,服用了它就能使人的脖子不生长肉瘤。

又东二十七里,曰堵山,神天愚居之,是多怪风雨。其上有木焉,名曰 天楄(bi1n),方茎而葵状,服者不■(y5)①。

【注释】①■:食物塞住咽喉。

【译文】再往东二十七里,是座堵山,神人天愚住在这里,所以这座山 上时常刮起怪风下起怪雨。山上生长着一种树木,名称是天楄,方方的茎干 而像葵菜形状,服用了它就能使人吃饭不噎住。

又东五十二里,曰放皋之山。明水出焉,南流注于伊水,其中多苍玉。

有木焉,其叶如槐,黄华而不实,其名曰蒙木,服之不惑。有兽焉,其状如 蜂,枝尾而反舌,善呼,其名曰文文。

【译文】再往东五十二里,是座放皋山。明水从这座山发源,向南流入 伊水,水中有很多苍玉。山中有一种树木,叶子与槐树叶相似,开黄色花却 不结果实,名称是蒙木,服用了它就能使人不糊涂。山中有一种野兽,形状 像蜜蜂,长着分叉的尾巴和倒转的舌头,喜欢呼叫,名称是文文。

又东五十七里,曰大■(k()之山,多■(y()琈之玉,多麋玉①。有 草焉,其状叶如榆,方茎而苍伤②,其名曰牛伤③,其根苍文,服者不厥④, 可以御兵。其阳狂水出焉,西南流注于伊水,其中多三足龟,食者无大疾, 可以已肿。

【注释】①麋玉:据古人说,可能就是瑂玉,一种像玉的石头。②苍伤:就是苍刺,即青色 的棘刺。③牛伤:如同说牛棘。④厥:古代中医学上指昏厥或手脚逆冷的病症,即突然昏倒,不省人 事,手脚僵硬冰冷。

【译文】再往东五十里,是座大■山,盛产■琈玉,还有许多麋玉。山 中有一种草,叶子与榆树叶相似,方方的茎干上长满刺,名称是牛伤,根茎 上有青色斑纹,服用了它就能使人不得昏厥病,还能辟兵器。狂水从这座山 的南麓流出,向西南流入伊水,水中有很多长着三只脚的龟,吃了它的肉就 能使人不生大病,还能消除痈肿。

又东七十里,曰半石之山。其上有草焉,生而秀①,其高丈余,赤叶赤 华,华而不实,其名曰嘉荣,服之者不[畏]霆②。来需之水出于其阳,而西 流注于伊水,其中多■(l*n)鱼,黑文,其状如鲋(f)),食者不睡。合 水出于其阴,而北流注于洛,多(t6ng)鱼③,状如鳜(gu@)④,居逵 (ku0)⑤,苍文赤尾,食者不痈,可以为瘘(1^u)⑥。

【注释】 ①秀:草类植物结实。这里指不开花就先结出果实。②霆:响声又震人又迅疾的 雷。③鱼:也叫瞻星鱼,体粗壮,亚圆筒形,后部侧扁,有粗糙骨板。④鳜:鳜鱼,也叫鯚花鱼、 桂鱼,体侧扁,背部隆起,青黄色,有不规则黑色斑纹,口大,下颔突出,鳞小,圆形。⑤逵:四通 八达的大路。这里指水底相互贯通着的洞穴。⑥瘘:人的脖子上生疮,长时间不愈,常常流脓水,还 生出蛆虫,古时把这种病状称作瘘。

【译文】 再往东七十里,是座半石山。山上长着一种草,一出土就结 子实,高一丈多,红色叶子红色花,开花后不结子实,名称是嘉荣,服用它 就能使人不畏惧霹雳雷响。来需水从半石山南麓流出,然后向西流入伊水, 水中生长着很多■鱼,浑身长满黑色斑纹,形状像普通的鲫鱼,人吃了它的 肉不感觉瞌睡。合水从半石山北麓流出,然后向北流入洛水,水中生长着很 多鱼,形状像一般的鳜鱼,隐居水底洞穴,浑身青色斑纹却拖着一条红尾 巴,人吃了它的肉就不患痈肿病,还可以治好瘘疮。

又东五十里,曰少室之山,百草木成囷(q&n)①。其上有木焉,其名 曰帝休,叶状如杨,其枝五衢(q*)②,黄华黑实,服者不怒。其上多玉, 其下多铁。休水出焉,而北流注于洛,其中多■(t@)鱼,状如(■)[盩 (zh^u)]蜼(w7i)而长距③,足白而对,食者无蛊(g()疾,可以御兵。

【注释】①囷:圆形谷仓。②衢:交错歧出的样子。③盩蜼:据古人说是一种与猕猴相似的 野兽。

【译文】再往东五十里,是座少室山,各种花草树木丛集像圆的谷仓。

山上有一种树木,名称是帝休,叶子的形状与杨树叶相似,树枝相互交叉着 伸向四方,开黄色花结黑色果实,服用了它就能使人心平气和不恼怒。少室 山上有丰富的玉石,山下有丰富的铁。休水从这座山发源,然后向北流入洛 水,水中有很多■鱼,形状像猕猴却有长长的像公鸡一样的爪子,白白的足 趾而相对着,人吃了它的肉就没了疑心病,还能辟兵器。

又东三十里,曰泰室之山。其上有木焉,叶状如梨而赤理,其名曰栯(y&) 木,服者不妒。有草焉,其状如■(zh*),白华黑实,泽如蘡(y9ng)薁 (y))①,其名曰■草②,服之不昧③。上多美石。

【注释】①蘡薁:一种藤本植物,俗称野葡萄。夏季开花,果实黑色,可以酿酒,也可入药。

②■草:与上文所述■草的形状不一样,当是同名异物。③昧:昏暗。引申为眼目不明。

【译文】再往东三十里,是座泰室山。山上有一种树木,叶子的形状像 梨树叶却有红色纹理,名称是栯木,人服用了它就没了嫉妒心。山中还有一 种草,形状像苍术或白术,开白色花结黑色果实,果实的光泽就像野葡萄, 名称是■草,服用了它就能使人的眼睛明亮不昏花。山上还有很多漂亮的石 头。

又北三十里,曰讲山,其上多玉,多柘(zh6)、多柏。有木焉,名曰 帝屋,叶状如椒①,反伤赤实②,可以御凶。

【注释】①椒:有三种,一种是木本植物,即花椒;一种是藤本植物,即胡椒;一种是蔬类 植物。这里指花椒,枝干有针刺,叶子坚而滑泽,果实红色,种子黑色,可以入药,也可调味。②反 伤:指倒生的刺。

【译文】再往北三十里,是座讲山,山上盛产玉石,有很多的柘树、许 多的柏树。山中有一种树木,名称是帝屋,叶子的形状与花椒树叶相似,长 着倒勾刺而结红色果实,可以辟凶邪之气。

又北三十里,曰婴梁之山,上多苍玉,錞(ch*n)于玄石①。

【注释】①錞:依附。

【译文】再往北三十里,是座婴梁山,山上盛产苍玉,而苍玉都附着在 黑色石头上面。

又东三十里,曰浮戏之山。有木焉,叶状如樗(ch&)而赤实,名曰亢 木,食之不蛊(g()。汜(s@)水出焉,而北流注于河。其东有谷,因名曰 蛇谷,上多少辛①。

【注释】①少辛:即细辛,一种药草。

【译文】再往东三十里,是座浮戏山。山中生长着一种树木,叶子的形 状像臭椿树叶而结红色果实,名称是亢木,人吃了它可以驱虫辟邪。汜水从 这座山发源,然后向北流入黄河。在浮戏山的东面有一道峡谷,因峡谷里有 很多蛇而取名叫蛇谷,峡谷上面还多产细辛。

又东四十里,曰少陉(x9ng)之山。有草焉,名曰■(g1ng)草,叶状 如葵,而赤茎白华,实如蘡(y9ng)薁(y)),食之不愚。器难之水出焉, 而北流注于役水。

【译文】 再往东四十里,是座少陉山。山中有一种草,名称是■草, 叶子形状与葵菜叶相似,又是红色的茎干白色的花,果实很像野葡萄,服食 了它就能使人增长智慧而不笨拙。器难水从这座山发源,然后向北流入役 水。

又东南十里,曰太山。有草焉,名曰梨,其叶状如(荻)[萩(qi&)] 而赤华①,可以已疽。太水出于其阳,而东南流注于役水;承水出于其阴, 而东北流注于役[水]。

【注释】①萩:一种蒿类植物,叶子是白色,像艾蒿却分杈多,茎干尤其高大,约有一丈余。

【译文】再往东南十里,是座太山。山里有一种草,名称是梨,叶子的 形状像蒿草叶而开红色花,可以用来治疗痈疽。太水从这座山的南麓流出, 然后向东南流入役水;承水从这座山的北麓流出,然后向东北流入役水。

又东二十里,曰末山,上多赤金。末水出焉,北流注于役[水]。

【译文】再往东二十里,是座末山,山上到处是黄金。末水从这座山发 源,向北流入役水。

又东二十五里,曰役山,上多白金,多铁。役水出焉,北注于河。

【译文】再往东二十五里,是座役山,山上有丰富的白银,还有丰富的 铁。役水从这座山发源,向北流入黄河。

又东三十五里,曰敏山。上有木焉,其状如荆,白华而赤实,名曰■(j@) 柏,服者不寒。其阳多■(y()琈之玉。

【译文】再往东三十五里,是座敏山。山上生长着一种树木,形状与牡 荆相似,开白色花朵而结红色果实,名称是■柏,吃了它的果实就能使人不 怕寒冷。敏山南面还盛产■琈玉。

又东三十里,曰大騩(gu9)之山,其阴多铁、美玉、青垩。有草焉, 其状如蓍(sh9)而毛,青华而白实,其名曰(穀)[榖](h7n),服之不夭, 可以为腹病。

【译文】再往东三十里,是座大騩山,山北面有丰富的铁、优质玉石、 青色垩土。山中有一种草,形状像蓍草却长着绒毛,开青色花而结白色果实, 名称是■,人服食了它就能不夭折而延年益寿,还可以医治肠胃上的各种疾 病。

凡苦山之首,自休与之山至于大騩(gu9)之山,凡十有九山,千一百 八十四里。其十六神者,皆豕(sh!)身而人面①。其祠:毛■(qu2n)用 一羊羞②,婴用一藻玉瘗(y@)③。苦山、少室、太室皆冢也。其祠之:太 牢之具,婴以吉玉。其神状皆人面而三首。其余属皆豕身人面也。

【注释】①豕:猪。②■:毛色纯一的全牲。全牲指整只的牛羊猪。羞:进献食品。这里指 贡献祭祀品。③藻玉:带有彩色纹理的玉。

【译文】总计苦山山系之首尾,自休与山起到大騩山止,一共十九座山, 途经一千一百八十四里。其中有十六座山的山神,形貌都是猪的身子而人的 面孔。祭祀这些山神:在毛物中用一只纯色的羊献祭,祀神的玉器用一块藻 玉而在祭祀后埋入地下。苦山、少室山、太室山都是诸山的宗主。祭祀这三 座山的山神:在毛物中用猪、牛、羊齐全的三牲作祭品,在祀神的玉器中用 吉玉。这三个山神的形貌都是人的面孔却长着三个脑袋。另外那十六座山的 山神都是猪的身子而人的面孔。

中次八(经)[山]荆山之首,曰景山,其上多金玉,其木多杼(sh)) 檀(t2n)①。雎(j&)水出焉,东南流注于江,其中多丹粟,多文鱼。

【注释】①杼:杼树,就是柞树。

【译文】中央第八列山系荆山山系之首座山,叫做景山,山上有丰富的 金属矿物和玉石,这里的树木以柞树和檀树最多。雎水从这座山发源,向东 南流入江水,水中有很多粟粒大小的丹沙,还生长着许多有彩色斑纹的鱼。

东北百里,曰荆山,其阴多铁,其阳多赤金,其中多犛(l!)牛①,多 豹虎,其木多松柏,其草多竹,多橘櫾(y^u)②。漳水出焉,而东南流注 于雎(j&),其中多黄金,多鲛(ji1o)鱼③。其兽多闾(lǘ)麋。

【注释】①犛牛:一种毛皮纯黑的牛,属于牦牛之类。②櫾:同“柚”。柚子与橘子相似而 大一些,皮厚而且味道酸。③鲛鱼:就是现在所说的鲨鱼,体型很大,性凶猛,能吃人。

【译文】往东北一百里,是座荆山,山北面有丰富的铁,山南面有丰富 的黄金,山中生长着许多犛牛,还有众多的豹子和老虎,这里的树木以松树 和柏树最多,这里的花草以丛生的小竹子最多,还有许多的橘子树和柚子 树。漳水从这座山发源,然后向东南流入雎水,水中盛产黄金,并生长着很 多鲨鱼。山中的野兽以山驴和麋鹿最多。

又东北百五十里,曰骄山,其上多玉,其下多青雘(hu^),其木多松 柏,多桃枝鉤端。神■(tu¥)围处之,其状如人(面)[而]羊角虎爪,恒 游于雎(j&)漳之渊,出入有光。

【译文】再往东北一百五十里,是座骄山,山上有丰富的玉石,山下有 丰富的青雘,这里的树木以松树和柏树居多,到处是桃枝和鉤端一类的丛生 小竹子。神仙■围居住在这座山中,形貌像人而长着羊一样的角,虎一样的 爪子,常常在雎水和漳水的深渊里畅游,出入时都有闪光。

又东北百二十里,曰女几之山,其上多玉,其下多黄金,其兽多豹虎, 多闾(lǘ)麋、麖(j9ng)、麂(j9)①,其鸟多白■(ji1o)②,多翟(d@), 多鸩(zh6n)③。

【注释】①麂:一种小鹿。②白■:也叫“■雉”,一种像野鸡而尾巴较长的鸟,常常是一 边飞行一边鸣叫。③鸩:鸩鸟,传说中的一种身体有毒的鸟,体形大小如雕鹰,羽毛紫绿色,长脖子 红嘴巴,吃有毒蝮蛇的头。

【译文】再往东北一百二十里,是座女几山,山上盛产玉石,山下盛产 黄金,山中的野兽以豹子和老虎最多,还有许许多多的山驴、麋鹿、麖、麂, 这里的禽鸟以白■最多,还有很多的长尾巴野鸡,很多的鸩鸟。

又东北二百里,曰宜诸之山,其上多金玉,其下多青雘(hu^)。洈(gu!) 水出焉,而南流注于漳,其中多白玉。

【译文】再往东北二百里,是座宜诸山,山上多出产金属矿物和玉石, 山下多出产青雘。洈水从这座山发源,然后向南流入漳水,水中有很多白色 玉石。

又东北二百里,曰纶山,其木多梓(z!)、枏(n2n),多桃枝,多柤 (zh1)、栗、橘、櫾(y^u)①,其兽多闾(lǚ)麈(zh()麢(l!ng)、 ■(zhu^)②。

【注释】①柤:柤树的形状像梨树,而树干、树枝都是红色的,开黄色花朵,结黑色果子。

②■:一种大鹿。■:形貌与兔子相似,却长着鹿脚,皮毛是青色。

【译文】再往东北二百里,是座纶山,在山中茂密的丛林中多的是梓树、 楠木树,又有很多丛生的桃枝竹,还有许多的柤树、栗子树、橘子树、柚子 树,这里的野兽以山驴、麈、羚羊、■最多。

又东二百里,曰陆■(gu@)之山,其上多■(y()琈之玉,其下多垩 (6),其木多杻(ni()橿(ji1ng)。

【译文】再往东二百里,是座陆■山,山下盛产■琈玉,山下盛产各种 颜色的垩土,这里的树木以杻树和橿树居多。

又东百三十里,曰光山,其上多碧,其下多(木)[水]。神计蒙处之, 其状人身而龙首,恒游于漳渊,出入必有飘风暴雨①。

【注释】①飘风:旋风,暴风。

【译文】再往东一百三十里,是座光山,山上到处有碧玉,山下到处流 水。神仙计蒙居住在这座山里,形貌是人的身子而龙的头,常常在漳水的深 渊里畅游,出入时一定有旋风急雨相伴随。

又东百五十里,曰岐(q@)山,其阳多赤金,其阴多白珉(m0n)①, 其上多金玉,其下多青雘(hu^),其木多樗(ch&)。神涉■(tu¥)处之, 其状人身而方面三足。

【注释】①珉:一种似玉的美石。

【译文】再往东一百五十里,是座岐山,山南面多出产黄金,山北面多 出产白色珉石,山上有丰富的金属矿物和玉石,山下有丰富的青雘,这里的 树木以臭椿树居多。神仙涉■就住在这座山里,形貌是人的身子而方形面孔 和三只脚。

又东百三十里,曰铜山,其上多金、银、铁,其木多(穀)[榖](g%u)、 柞(zu^)、柤(zh1)、栗、橘、櫾(y^u),其兽多犳(zhu¥)。

【译文】再往东一百三十里,是座铜山,山上有丰富的金、银、铁,这 里的树木以构树、柞树、柤树、栗子树、橘子树、柚子树最多,而野兽多是 长着豹子斑纹的犳。

又东北一百里,曰美山,其兽多兕(s@)、牛,多闾(lǘ)、麈(zh(), 多豕(sh!)、鹿,其上多金,其下多青雘(hu^)。

【译文】再往东北一百里,是座美山,山中的野兽以兕、野牛最多,又 有很多山驴、麈,还有许多野猪、鹿,山上多出产金,山下多出产青雘。

又东北百里,曰大尧之山,其木多松柏,多梓(z!)桑,多机①,其草 多竹,其兽多豹、虎、麢(l0ng)、■(zhu^)。

【注释】 ①机:机木树,就是桤(q9)木树。是一种落叶乔木,木材坚韧,生长很快,容 易成林。

【译文】 再往东北一百里,是座大尧山,在山里的树木中以松树和柏 树居多,又有众多的梓树和桑树,还有许多机木树,这里的草大多是丛生的 小竹子,而野兽以豹子、老虎、羚羊、■最多。

又东北三百里,曰灵山,其上多金玉,其下多青雘(hu^),其木多桃、 李、梅、杏。

【译文】再往东北三百里,是座灵山,山上有丰富的金属矿物和玉石, 山下盛产青雘,这里的树木大多是桃树、李树、梅树、杏树。

又东北七十里,曰龙山,上多寓木①,其上多碧,其下多赤锡②,其草 多桃枝、鉤端。

【注释】①寓木:又叫宛童,即寄生树。又分两种,叶子是圆的叫做茑木,叶子像麻黄叶的 叫做女萝。因这种植物是寄寓在其它树木上生长的,像鸟站立树上,所以称作寄生、寓木、茑木。俗 称寄生草。②锡:和本书中所记载的金、银、铜、铁等都是指未经提炼的矿石或矿沙一样,这里的锡 也是指未经提炼的锡土矿。以下同此。

【译文】再往东北七十里,是座龙山,山上到处是寄生树,还盛产碧玉, 山下有丰富的红色锡土,而草大多是桃枝、鉤端之类的小竹丛。

又东南五十里,曰衡山,上多寓木、(穀)[榖(g%u)]、柞,多黄垩 (6)、白垩。

【译文】 再往东南五十里,是座衡山,山上有许多寄生树、构树、柞 树,还盛产黄色垩土、白色垩土。

又东南七十里,曰石山,其上多金,其下多青雘(hu^),多寓木。

【译文】再往东南七十里,是座石山,山上多出产金,山下有丰富的青 雘,还有许多寄生树。

又南百二十里,曰若山,其上多■(y()琈之玉,多赭(zh6),多(邽) [封]石①,多寓木,多柘(zh6)。

【注释】①封石:据古人说是一种可作药用的矿物,味道是甜的,没有毒性。

【译文】再往南一百二十里,是座若山,山上多出产■琈玉,又多出产 赭石,也有很多封石,到处长着寄生树,还生长着许许多多的柘树。

又东南一百二十里,曰彘(zh@)山,多美石,多柘(zh6)。

【译文】 再往东南一百二十里,是座彘山,有很多漂亮的石头,到处 生长着柘树。

又东南一百五十里,曰玉山,其上多金玉,其下多碧、铁,其木多柏。

【译文】再往东南一百五十里,是座玉山,山上有丰富的金属矿物和玉 石,山下有丰富的碧玉、铁,这里的树木以柏树居多。

又东南七十里,曰彘山,其木多檀(t2n),多(邽)[封]石,多白锡。

郁水出于其上,潜于其下,其中多砥(d0)砺(l0)。

【译文】再往东南七十里,是座讙山,这里的树木大多是檀树,还盛产 封石,又多出产白色锡土。郁水从这座山顶上发源,潜流到山下,水中有很 多磨石。

又东北百五十里,曰仁举之山,其木多(穀)[榖(g^u)]柞(zu^), 其阳多赤金,其阴多赭(zh7)。

【译文】再往东北一百五十里,是座仁举山,这里的树木以构树和柞树 居多,山南面有丰富的黄金,山北面多出产赭石。

又东五十里,曰师每之山,其阳多砥(d!)砺(l0),其阴多青雘(hu^), 其木多柏,多檀(t2n),多柘(zh6),其草多竹。

【译文】再往东五十里,是座师每山,山南面多出产磨石,山北面多出 产青雘,山中的树木以柏树居多,又有很多檀树,还生长着大量柘树,而草 大多是丛生的小竹子。

又东南二百里,曰琴鼓之山,其木多(穀)[榖](g^u)、柞(zu^)、 椒、柘(zh6)①,其上多白珉,其下多洗石,其兽多豕(sh!)、鹿,多白 犀(x9),其鸟多鸩(zh6n)。

【注释】①椒:据古人说,这种椒树矮小而丛生,如果在它下面有草木生长就会被刺死。与 上文所记椒树指花椒树者似略有不同。

【译文】再往东南二百里,是座琴鼓山,这里的树木大多是构树、柞树、 椒树、柘树,山上多出产白色珉石,山下多出产洗石,这里的野兽,以野猪、 鹿最多,还有许多白色犀牛,而禽鸟大多是鸩鸟。

凡荆山之首,自景山至琴鼓之山,凡二十三山,二千八百九十里。其神 状皆鸟身而人面。其祠:用一雄鸡祈瘗(y@),用一藻圭,糈(x()用稌(t*)。

骄山,冢也。其祠:用羞酒少牢祈瘗,婴(毛)[用]一璧。

【译文】总计荆山山系之首尾,自景山起到琴鼓山止,一共二十三座山, 途经二千八百九十里。诸山山神的形貌都是鸟的身子而人的面孔。祭祀山 神:在毛物中用一只公鸡祭祀后埋入地下,并用一块藻圭献祭,祀神的米用 稻米。骄山,是诸山之宗主。祭祀骄山山神:用进献的美酒和猪、羊来祭祀 而后埋入地下,在祀神的玉器中用一块玉璧。

中次九(经)[山]岷山之首,曰女几之山,其上多石涅(ni6)①,其 木多杻(ni()僵(ji1ng),其草多菊、■(zh*)②。洛水出焉,东注于 江③。其中多雄黄,其兽多虎、豹。

【注释】①石涅:当即涅石,一种矿物,可做黑色染料。②菊:通称菊花,品种繁多,有九 百种,于是古人将其概括为两大类,一类是栽种在庭院中供观赏的,叫真菊;一类是在山野生长的, 叫野菊,别名叫苦薏。这里就是指野菊。③江:古人单称“江”或“江水”而不贯以名者,则大多是 专指长江,这里即指长江。但本书记述山丘河流的方位走向都不甚确实,所述长江也不例外,与今天 用科学方法测量出的长江不甚相符。现在译“江”或“江水”为“长江”,只是为了使译文醒目而有 别于其它江水。以下同此。

【译文】中央第九列山系岷山山系之首座山,叫做女几山,山上多出产 石涅,这里的树木以杻树、橿树居多,而花草以野菊、苍术或白术居多。洛 水从这座山发源,向东流入长江。山里到处有雄黄,而野兽以老虎、豹子最 多。

又东北三百里,曰岷山。江水出焉,东北流注于海,其中多良龟,多鼍 (tu¥)①。其上多金玉,其下多白珉。其木多梅棠,其兽多犀(x9)、象, 多夔(ku0)牛②,其鸟多翰、鷩(bi5)③。

【注释】①鼍:古人说是长得像蜥蜴,身上有花纹鳞,大的长达二丈,皮可以制做鼓用。也 就是现在所说的扬子鳄,俗称猪婆龙。②夔牛:古人说是一种重达几千斤的大牛。 ③翰:就是上文 所说的白翰鸟,野鸡的一种。鷩:就是上文所说的赤鷩鸟,现在叫锦鸡。

【译文】再往东北三百里,是座岷山。长江从岷山发源,向东北流入大 海,水中生长着许多优良的龟,还有许多鼍。山上有丰富的金属矿物和玉石, 山下盛产白色珉石。山中的树木以梅树和海棠树最多,而野兽以犀牛和大象 最多,还有大量的夔牛,这里的禽鸟大多是白翰鸟和赤鷩鸟。

wwW。xiaoshuotxt=com



山海经卷五 中山经 -3

。小%说^t*xt-天.堂
又东北一百四十里,曰崃山。江水出焉,东流注[于]大江。其阳多黄金, 其阴多麋麈(zh(),其木多檀(t2n)柘(zh6),其草多■(xi6)、韭, 多药、空夺①。

【注释】①药:指白芷,一种香草。空夺:就是上文所说的寇脱。

【译文】再往东北一百四十里,是座崃山。江水从这座山发源,向东流 入长江。山南面盛产黄金,山北面到处有麋鹿和麈,这里的树木大多是檀树 和柘树,而花草大多是野薤菜和野韭菜,还有许多白芷和寇脱。

又东一百五十里,曰崌(j*)山。江水出焉,东流注于大江,其中多怪 蛇①,多■(zh@)鱼②。其木多楢(qi&)杻③,多梅、梓(z!),其兽多 夔(ku9)牛、麢(l0ng)、■(zhu^)、犀(x9)、兕(s@)。有鸟焉, 状如鸮(xi1o)而赤身白首,其名曰窃脂,可以御火。

【注释】①怪蛇:据古人讲,有一种钩蛇长达几丈,尾巴分叉,在水中钩取岸上的人、牛、 马而吞食掉。怪蛇就指这样一类的蛇。②■鱼:不详何种鱼。③楢:一种木材刚硬的树木,可以用作 制造车子的材料。

【译文】再往东一百五十里,是座崌山。江水从这座山发源,向东流入 长江,水中生长着许多怪蛇,还有很多■鱼。这里的树木以楢树和杻树居多, 还有很多梅树与梓树,而野兽以夔牛、羚羊、■、犀牛、兕最多。山中有一 种禽鸟,形状像一般的猫头鹰却是红色的身子白色的脑袋,名称是窃脂,人 饲养它可以辟火。

又东三百里,曰高梁之山,其上多垩(6),其下多砥(d!)砺(l0), 其木多桃枝、鉤端。有草焉,状如葵而赤华、荚实、白柎(f&),可以走马。

【译文】再往东三百里,是座高梁山,山上盛产垩土,山下盛产磨石, 这里的草木大多是桃枝竹和鉤端竹。山中生长着一种草,形状像葵菜却是红 色的花朵、带荚的果实、白色的花萼,给马吃了它就能使马跑得快。

又东四百里,曰蛇山,其上多黄金,其下多垩(6),其木多栒(x*n), 多豫章,其草多嘉荣、少辛。有兽焉,其状如狐,而白尾长耳,名■(y!) 狼,见(xian)则国内有兵。

【译文】再往东四百里,是座蛇山,山上多出产黄金,山下多出产垩土, 这里的树木以栒树最多,还有许多豫章树,而花草以嘉荣、细辛最多。山中 有一种野兽,形状像一般的狐狸,却长着白尾巴和长耳朵,名称是■狼,在 哪个国家出现哪个国家就会有战争。

又东五百里,曰鬲山,其阳多金,其阴多白珉。蒲鹳(h#ng)之水出焉, 而东流注于江,其中多白玉。其兽多犀(x9)、象、熊、罴(p0),多猿、 蜼(w7i)①。

【注释】①蜼:据古人说是一种长尾巴猿猴,鼻孔朝上,尾巴分叉,天下雨时就自己悬挂在 树上,用尾巴塞住鼻孔。

【译文】再往东五百里,是座鬲山,山南面盛产金,山北面盛产白色珉 石。蒲鹳水从这座山发源,然后向东流入长江,水中有很多白色玉石。山中 的野兽以犀牛、大象、熊、罴最多,还有许多猿猴、长尾猿。

又东北三百里,曰隅阳之山,其上多金玉,其下多青雘,其木多梓(z!) 桑,其草多茈(z!)。徐之水出焉,东流注于江,其中多丹粟。

【译文】再往东北三百里,是座隅阳山,山上有丰富的金属矿物和玉石, 山下有丰富的青雘,这里的树木大多是梓树和桑树,而草大多是紫草。徐水 从这座山发源,向东流入长江,水中有许多粟粒大小的丹沙。

又东二百五十里,曰岐山,其上多白金,其下多铁,其木多梅梓(z!), 多杻(ni()楢(qi&)。減水出焉,东南流注于江。

【译文】再往东二百五十里,是座岐山,山上有丰富的白银,山下有丰 富的铁,这里的树木以梅树和梓树居多,还有许多杻树和楢树。減水从这座 山发源,向东南流入长江。

又东三百里,曰勾■(m@)之山,其上多玉,其下多黄金,其木多栎(l@) 柘(zh6),其草多芍药。

【译文】再往东三百里,是座勾■山,山上盛产玉石,山下盛产黄金, 这里的树木大多是栎树和柘树,而花草大多是芍药。

又东一百五十里,曰风雨之山,其上多白金,其下多石涅,其木多棷 (z#u)椫(sh4n)①,多杨。宣余之水出焉,东流注于江,其中多蛇。其 兽多闾(lǘ)、麋,多麈(zh()、豹、虎,其鸟多白■(ji1o)  【注释】①棷:不详何样树木。椫:椫树,也叫白理木。木质坚硬,木纹洁白,可以制做梳 子、勺子等器物。

【译文】再往东一百五十里,是座风雨山,山上多出产白银,山下多出 产石涅,这里的树木以棷:树和椫树居多,杨树也不少。宣余水从这座山发 源,向东流入长江,水中有很多水蛇。山里的野兽以山驴和麋鹿最多,还有 许多的麈、豹子、老虎,而禽鸟大多是白■。

又东二百里,曰玉山,其阳多铜,其阴多赤金,其木多豫章、楢(qi&)、 杻(ni(),其兽多豕(sh!)、鹿、麢(l0ng)、■(zhu^),其鸟多鸩(zh6n)。

【译文】再往东二百里,是座玉山,山南面多出产铜,山北面多出产黄 金,这里的树木以豫章树、楢树、杻树最多,而野兽以野猪、鹿、羚羊、■ 最多,禽鸟大多是鸩鸟。

又东一百五十里,曰熊山。有穴焉,熊之穴,恒出入神人。夏启而冬闭; 是穴也,冬启乃必有兵。其上多白玉,其下多白金。其木多樗(ch&)柳, 其草多寇脱。

【译文】再往东一百五十里,是座熊山。山中有一洞穴,是熊的巢穴, 也时常有神人出入。洞穴一般是夏季开启而冬季关闭;就是这个洞穴,如果 冬季开启就一定发生战争。山上多出产白色玉石,山下多出产白银。山里的 树木以臭椿树和柳树居多,而花草以寇脱草最多见。

又东一百四十里,曰騩(gu@)山,其阳多美玉、赤金,其阴多铁,其 木多桃枝、荆(芭)[芑](q!)。

【译文】再往东一百四十里,是座騩山,山南面盛产美玉黄金,山北面 盛产铁,这里的草木以桃枝竹、牡荆树、枸杞树最多。

又东二百里,曰葛山,其上多赤金,其下多瑊(ji1n)石①,其木多柤 (zh1)、栗、橘、櫾(y^u)、楢(qi&)、杻(ni(),其兽多麢(l0ng) ■(zhu^),其草多嘉荣。

【注释】①瑊石:是一种比玉差一等的美石。

【译文】再往东二百里,是座葛山,山上多出产黄金,山下多出产瑊石, 这里的树木以柤树、栗子树、橘子树、柚子树、楢树、杻树居多,而野兽以 羚羊和■居多,花草大多是嘉荣。

又东一百七十里,曰贾超之山,其阳多黄垩(6),其阴多美赭(zh6), 其木多柤(zh1)、栗、橘、櫾(y^u),其中多龙脩①。

【注释】①龙脩:就是龙须草,与莞草相似而细一些,生长在山石缝隙中,草茎倒垂,可以 用来编织席子。

【译文】再往东一百七十里,是座贾超山,山南面多出产黄色垩土,山 北面多出产精美赭石,这里的树木大多是柤树、栗子树、橘子树、柚子树, 山中的草以龙须草最多。

凡岷山之首,自女几山至于贾超之山,凡十六山,三千五百里。其神状 皆马身而龙首。其祠:毛用一雄鸡瘗,糈(x()用稌(t*)。文山、勾■(m0)、 风雨、騩(之)山①,是皆冢也。其祠之:羞酒,少牢具,婴(毛)[用]一 吉玉。熊山,(席)[帝]也②。其祠:羞酒,太牢具,婴(毛)[用]一璧。

干儛,用兵以禳(r2ng)③;祈,璆(qi*)冕舞④。

【注释】①文山:指岷山。②帝:主体。这里是首领的意思。③禳:祭祷消灾。④璆:同“球”。

美玉。冕:即冕服,是古代帝王、诸侯及卿大夫的礼服。这里泛指礼服。

【译文】总计岷山山系之首尾,自女几山起到贾超山止,一共十六座山, 途经三千五百里,诸山山神的形貌都是马的身子而龙的脑袋。祭祀山神:在 毛物中用一只公鸡作祭品埋入地下,祀神的米用稻米。文山、勾■山、风雨 山、騩山,是诸山的宗主。祭祀这几座山的山神:进献美酒,用猪、羊作祭 品,在祀神的玉器中用一块吉玉。熊山,是诸山的首领。祭祀这个山神:进 献美酒,用猪、牛、羊齐全的三牲作祭品,在祀神的玉器中用一块玉璧。手 拿盾牌舞蹈,为了禳除战争灾祸;祈求福祥,就穿戴礼服并手持美玉而舞蹈。

中次十(经)[山]之首,曰首阳之山,其上多金玉,无草木。

【译文】中央第十列山系之首座山,叫做首阳山,山上有丰富的金属矿 物和玉石,没有花草树木。

又西五十里,曰虎尾之山,其木多椒、椐(j&)①,多封石,其阳多赤 金,其阴多铁。

【注释】①椐:椐树,也叫灵寿木。树干上多肿节,古人用作手杖。

【译文】再往西五十里,是座虎尾山,这里树木以花椒树、椐树最多, 到处有封石,山南面有丰富的黄金,山北面有丰富的铁。

又西南五十里,曰繁缋(ku@)之山,其木多楢(qi&)杻(ni(),其 草多枝、勾①。

【注释】①枝、勾:就是上文所说的桃枝竹、鉤端竹,矮小而丛生。

【译文】再往西南五十里,是座繁缋山,这里的树木大多是楢树和杻树, 而草大多是桃枝、鉤端之类的小竹丛。

又西南二十里,曰勇石之山,无草木,多白金,多水。

【译文】再往西南二十里,是座勇石山,不生长花草树木,有丰富的白 银,到处流水。

又西二十里,曰复州之山,其木多檀(t2n),其阳多黄金。有鸟焉, 其状如鸮(xi1o),而一足彘(zh@)尾,其名曰跂(q!)踵(zh^ng),见 (xi4n)则其国大疫。

【译文】再往西二十里,是座复州山,这里的树木以檀树居多,山南面 有丰富的黄金。山中有一种禽鸟,形状像一般的猫头鹰,却长着一只爪子和 猪一样的尾巴,名称是跂踵,在哪个国家出现哪个国家就会发生大瘟疫。

又西三十里,曰楮(ch()山,多寓木,多椒、椐,多柘(zh6),多垩。

【译文】再往西三十里,是座楮山,生长着茂密的寄生树,到处是花椒 树、椐树,柘树也不少,还有大量的垩土。

又西二十里,曰又原之山,其阳多青雘(hu^),其阴多铁,其鸟多鸜 (q*)鹆(y))。

【译文】再往西二十里,是座又原山,山南面有丰富的青雘,山北面有 丰富的铁,这里的禽鸟以八哥最多。

又西五十里,曰涿(zhu¥)山,其木多(穀)[榖(g^u)]柞(zu^)杻 (ni(),其阳多■(y()琈之玉。

【译文】再往西五十里,是座涿山,这里的树木大多是构树、柞树、杻 树,山南面多出产■琈玉。

又西七十里,曰丙山,其木多梓(z!)、檀(t2n),多弞(sh7n)杻 (ni()①。

【注释】①弞杻:杻树的树干都是弯曲的,而弞杻的树干长得比较直,不同于一般的杻树。

【译文】再往西七十里,是座丙山,这里的树木大多是梓树、檀树,还 有很多弞杻树。

凡首阳山之首,自首山至于丙山,凡九山,二百六十七里。其神状皆龙 身而人面。其祠之:毛用一雄鸡瘗,糈(x()用五种之糈①。堵山②,冢也, 其祠之:少牢具,羞酒祠,婴(毛)[用]一璧瘗(y@)。騩(gu9)山,帝 也,其祠羞酒,太牢(其)[具];合巫祝二人儛③,婴一璧。

【注释】①五种之糈:指黍、稷、稻、粱、麦五种粮米。②堵山:指楮山。③巫:古代称能 以舞降神的人,即女巫。祝:古代在祠庙中主管祭礼的人,即男巫。

【译文】总计首阳山山系之首尾,自首阳山起到丙山止,一共九座山, 途经二百六十七里。诸山山神的形貌都是龙的身子而人的面孔。祭祀山神: 在毛物中用一只公鸡献祭后埋入地下,祀神的米用五种粮米。堵山,是诸山 的宗主,祭祀这个山神:用猪、羊二牲作祭品,进献美酒来祭祀,在玉器中 用一块玉璧,祀神后埋入地下。騩山,是诸山的首领,祭祀騩山山神要进献 美酒,用猪、牛、羊齐全的三牲作祭品;让女巫师和男祝师二人一起跳舞, 在玉器中用一块玉璧来祭祀。

中次一十一山(经)荆山之首,曰翼望之山。湍(zhu1n)水出焉,东 流注于济;贶(ku4ng)水出焉,东南流注于汉,其中多蛟①。其上多松柏, 其下多漆梓(z!),其阳多赤金,其阴多珉(m0n)。

【注释】①蛟:据古人说是像蛇的样子,却有四只脚,小小的头,细细的脖子,脖颈上有白 色肉瘤,大的有十几围粗,卵有瓮大小,能吞食人。

【译文】中央第十一列山系荆山山系之首座山,叫做翼望山。湍水从这 座山发源,向东流入济水;贶水也从这座山发源,向东南流入汉水,水中有 很多蛟。山上到处是松树和柏树,山下有茂密的漆树和梓树,山南面多出产 黄金,山北面多出产珉石。

又东北一百五十里,曰朝歌之山。■(w()水出焉,东南流注于荥(x0ng), 其中多人鱼。其上多梓(z!)、枏(n2n),其兽多麢(l0ng)、麋。有草 焉,名曰莽(w4ng)草①,可以毒鱼。

【注释】①莽草:就是上文所说的芒草,又叫鼠莽。

【译文】再往东北一百五十里,是座朝歌山。■水从这座山发源,向东 南流入荥水,水中生长着很多人鱼。山上有茂密的梓树、楠木材,这里的野 兽以羚羊、麋鹿最多。山中有一种草,名称是莽草,能够毒死鱼的。

又东南二百里,曰帝囷(q&n)之山,其阳多■(y()琈之玉,其阴多 铁。帝囷之水出于其上,潜于其下,多鸣蛇。

【译文】再往东南二百里,是座帝囷山,山南面有丰富的■琈玉,山北 面有丰富的铁。帝囷水从这座山顶上发源,潜流到山下,水中有很多长着四 只翅膀的鸣蛇。

又东南五十里,曰视山,其上多韭。有井焉①,名曰天井,夏有水,冬 竭。其上多桑,多美垩(6)、金、玉。

【注释】①井:也和上文所说的井一样,是指自然形成的水泉,而非人工挖掘的水井。古人 把四周高峻中间低洼的地形,或四面房屋和围墙中间的空地称作天井,因其形如井而露天。所以,这 里也把处在低洼地的水泉叫做天井。

【译文】再往东南五十里,是座视山,山上到处是野韭菜。山中有一口 井,叫做天井,夏天有水,冬天枯竭。山上有茂密的桑树,还有丰富的优良 垩土、金属矿物、玉石。

又东南二百里,曰前山,其木多槠(zh&)①,多柏,其阳多金,其阴 多赭(zh7)。

【注释】①槠:槠树,结的果实如同橡树的果实,可以吃,木质耐腐蚀,常用来作房屋的柱 子。

【译文】再往东南二百里,是座前山,这里的树木以槠树居多,还有不 少的柏树,山南面盛产金,山北面盛产赭石。

又东南三百里,曰丰山。有兽焉,其状如猿,赤目、赤喙(hu@)、黄 身,名曰雍和,见(xi4n)则国有大恐。神耕父处之,常游清泠(l0ng)之 渊,出入有光,见(xi4n)则其国为败。有九钟焉,是(知)[和]霜鸣。其 上多金,其下多(穀)[榖](g^u)、柞(zu^)、杻(ni()、橿(ji1ng)。

【译文】再往东南三百里,是座丰山。山中有一种野兽,形状像猿猴, 却长着红眼睛、红嘴巴、黄色的身子,名称是雍和,在哪个国家出现那个国 家里就会发生大恐怖。神仙耕父住在这座山里,常常在清泠渊畅游,出入时 都有闪光,在哪个国家出现那个国家就要衰败。这座山还有九口钟,它们都 应和霜的降落而鸣响。山上有丰富的金,山下有茂密的构树、柞树、杻树、 橿树。

又东北八百里,曰兔床之山,其阳多铁,其木多(藷■)[槠(zh&)芧 (x))]①,其草多鸡谷,其本如鸡卵,其味酸甘,食者利于人。

【注释】①芧:芧树,即栎(l@)树。果实叫橡子、橡斗。树皮可以饲养蚕,树叶可以做染 料。

【译文】再往东北八百里,是座兔床山,山南面有丰富的铁,山里的树 木以槠树和芧树最多,而花草以鸡谷草最多,它的根茎像鸡蛋似的,味道是 酸中带甜,服食它是对人的身体有益的。

又东六十里,曰皮山,多垩(6),多赭(zh7),其木多松柏。

【译文】再往东六十里,是座皮山,有大量的垩土,还有大量的赭石, 这里的树木大多是松树和柏树。

又东六十里,曰瑶碧之山,其木多梓(z!)枏(n2n),其阴多青雘(hu^), 其阳多白金。有鸟焉,其状如雉(zh@),恒食蜚(f7i)①,名曰鸩(zh6n) ②。

【注释】①蜚:一种有害的小飞虫,形状椭圆,散发恶臭。②鸩:鸩鸟,和上文所说的有毒 鸩鸟不是一种鸟,是同名异物。

【译文】再往东六十里,是座瑶碧山,这里的树木以梓树和楠木树最多, 山北阴面盛产青雘,山南面盛产白银。山中有一种禽鸟,形状像一般的野鸡, 常吃蜚虫,名称是鸩。

又东四十里,曰(支)[攻]离之山。(济)[淯]水出焉,南流注于汉。

有鸟焉,其名曰婴勺,其状如鹊,赤目、赤喙(hu@)、白身,其尾若勺, 其鸣自呼。多■(zu¥)牛、多羬(xi2n)羊。

【译文】再往东四十里,是座攻离山。淯水从这座山发源,向南流入汉 水。山中有一种禽鸟,名称是婴勺,形状像普通的喜鹊,却长着红眼睛、红 嘴巴、白色的身子,尾巴与酒勺的形状相似,它发出的叫声便是自身名称的 读音。这座山中还有很多■牛、羬羊。

又东北五十里,曰祑(zh@)■(di1o)之山,其上多松、柏、机、(柏) [桓](hu2n)①。

【注释】①机:即桤(q9)树。桓:桓树,树叶像柳叶,树皮是黄白色。古人说它又叫无患 子,可以洗涤衣服,除去污垢。

【译文】再往东北五十里,是座祑■山,山上有茂密的松树、柏树、桤 树、桓树。

又西北一百里,曰堇(q0n)理之山,其上多松柏,多美梓(z!),其 阴多丹雘(hu^),多金,其兽多豹虎。有鸟焉,其状如鹊,青身白喙(hu@), 白目白尾,名曰青耕,可以御疫,其鸣自叫。

【译文】再往西北一百里,是座堇理山,山上有茂密的松树柏树,还有 很多优良梓树,山北阴面多出产青雘,并且有丰富的金,这里的野兽以豹子 和老虎最多。山中有一种禽鸟,形状像一般的喜鹊,却是青色的身子白色的 嘴巴,白色的眼睛白色的尾巴,名称是青耕,人饲养它可以辟瘟疫,它发出 的叫声便是自身名称的读音。

又东南三十里,曰依轱(k&)之山,其上多杻(ni()橿(ji1ng),多 苴(zh1)①。有兽焉,其状如犬,虎爪有甲,其名曰獜(l@n),善駚(y1ng) ■(f6n),食者不风。

【注释】①苴:通“柤”。即柤树。

【译文】再往东南三十里,是座依轱山,山上有茂密的杻树和橿树,柤 树也不少。山中有一种野兽,形状像普通的狗,长着老虎一样的爪子而身上 又有鳞甲,名称是獜,擅长跳跃腾扑,吃了它的肉就能使人不患风痺病。

又东南三十五里,曰即谷之山,多美玉,多玄豹,多闾(lǘ)麈(zh(), 多麢(l0ng)■(zhu^)。其阳多珉,其阴多青雘(hu^)。

【译文】再往东南三十五里,是座即谷山,这里多出产优良玉石,有很 多黑豹,还有不少的山驴和麈,羚羊和■也很多。山南阳面盛产珉石,山北 阴面盛产青雘。

又东南四十里,曰鸡山,其上多美梓(z!),多桑,其草多韭。

【译文】再往东南四十里,是座鸡山,山上到处是优良梓树,还有茂密 的桑树,而花草以野韭菜为最多。

又东南五十里,曰高前之山。其上有水焉,甚寒而清,帝台之浆也,饮 之者不心痛。其上有金,其下有赭(zh7)。

【译文】再往东南五十里,是座高前山。这座山上有一条溪水,非常冰 凉而又特别清澈,是神仙帝台所用过的浆水,饮用了它就能使人不患心痛 病。山上有丰富的金,山下有丰富的赭石。

又东南三十里,曰游戏之山,多雘杻(ni()、橿(ji1ng)、(穀)[榖 (g^u)],多玉,多封石。

【译文】再往东南三十里,是座游戏山,这里有茂密的杻树、橿树、构 树,还有丰富的玉石,封石也很多。

又东南三十五里,曰从山,其上多松柏,其下多竹。从水出于其上,潜 于其下,其中多三足鳖,枝尾①,食之无蛊(g()(疫)[疾]。

【注释】①枝:分支的,分叉的。

【译文】再往东南三十五里,是座从山,山上到处是松树和柏树,山下 有茂密的竹丛。从水由这座山顶上发源,潜流到山下,水中有很多三足鳖, 长着叉开的尾巴,吃了它的肉就能使人不患疑心病。

又东南三十里,曰婴■(zh5n)之山,其上多松柏,其下多梓(z!)、 櫄(ch&n)①。

【注释】①櫄:又叫杶树,形状像臭椿树,树干可制做车辕。

【译文】再往东南三十里,是座婴■山,山上到处是松树柏树,山下有 茂密的梓树、櫄树。

又东南三十里,曰毕山。帝苑之水出焉,东北流注于(视)[瀙](q@n), 其中多水玉,多蛟。其上多■(y()琈之玉。

【译文】再往东南三十里,是座毕山。帝苑水从这座山发源,向东北流 入瀙水,水中多出产水晶石,还有很多蛟。山上有丰富的■琈玉。

又东南二十里,曰乐马之山。有兽焉,其状如彙(w6i),赤如丹火, 其名曰■(l@),见(xi4n)则其国大疫。

【译文】再往东南二十里,是座乐马山。山中有一种野兽,形状像一般 的刺猬,全身赤红如丹火,名称是■,在哪个国家出现那个国家里就会发生 大瘟疫。

又东南二十五里,曰葴(ji1n)山,(视)瀙水出焉,东南流注于汝水, 其中多人鱼,多蛟,多颉(ji2)①。

【注释】①颉:据古人说是一种皮毛青色而形态像狗的动物。可能就是今天所说的水獭(t3)。

【译文】再往东南二十五里,是座葴山,瀙水从这座山发源,向东南流 入汝水,水中有很多人鱼,又有很多蛟,还有很多的颉。

又东四十里,曰婴山,其下多青雘(hu^),其上多金玉。

【译文】再往东四十里,是座婴山,山下有丰富的青雘,山上有丰富的 金属矿物和玉石。

又东三十里,曰虎首之山,多苴(zh1)、椆(di1o)、椐①。

【注释】①椆:据古人说是一种耐寒冷而不凋落的树木。

【译文】再往东三十里,是座虎首山,有茂密的柤树、椆树、椐树。

又东二十里,曰婴矦之山,其上多封石,其下多赤锡。

【译文】再往东二十里,是座婴矦山,山上多出产封石,山下多出产红 色锡土。

又东五十里,曰大孰之山。杀水出焉,东北流注于(视)[瀙]水,其中 多白垩(6)。

【译文】再往东五十里,是座大孰山。杀水从这座山发源,向东北流入 瀙水,沿岸到处是白色垩土。

又东四十里,曰卑山,其上多桃、李、苴(zh1)、梓(z!),多纍(l7i) ①。

【注释】①纍:又叫做滕,古人说是一种与虎豆同类的植物。虎豆是缠蔓于树枝而生长的, 所结豆荚,成熟后是黑色,有毛刺外露,像老虎指爪,而荚中豆子有斑点,像老虎身上的斑纹,所以 又叫虎櫐(l7i)。虎櫐,即今所说的紫藤。櫐,同“蘽”,蔓生植物。

【译文】再往东四十里,是座卑山,山上有茂密的桃树、李树、柤树、 梓树,还有很多紫藤树。

又东三十里,曰倚帝之山,其上多玉,其下多金。有兽焉,状如鼣(f6i) 鼠①,白耳白喙(hu@),名曰狙(q&)如,见(xi4n)则其国有大兵。

【注释】①鼳鼠:不详何种动物。

【译文】再往东三十里,是座倚帝山,山上有丰富的玉石,山下有丰富 的金。山中有一种野兽,形状像鼣鼠,长着白耳朵白嘴巴,名称是狙如,在 哪个国家出现那个国家里就会发生大战争。

又东三十里,曰鲵(n0)山。鲵水出于其上,潜于其下,其中多美垩(6)。

其上多金,其下多青雘(hu^)。

【译文】再往东三十里,是座鲵山。鲵水从这座山顶上发源,潜流到山 下,这里有很多优良垩土。山上有丰富的金,山下有丰富的青雘。

又东三十里,曰雅山。澧(l@)水出焉,东流注于(视)[瀙]水,其中 多大鱼。其上多美桑,其下多苴(zh1),多赤金。

【译文】再往东三十里,是座雅山。澧水从这座山发源,向东流入瀙水, 水中有很多大鱼。山上有茂密的优良桑树,山下有茂密的柤树,这里还多出 产黄金。

又东五十(五)里,曰宣山。沦水出焉,东南流注于(视)[瀙]水,其 中多蛟。其上有桑焉,大五十尺,其枝四衢(q*),其叶大尺余,赤理、黄 华、青雘(f&),名曰帝女之桑。

【译文】再往东五十里,是座宣山。沦水从这座山发源,向东南流入瀙 水,水中有很多蛟。山上有一种桑树,树干合抱有五十尺粗细,树枝交叉伸 向四方,树叶方圆有一尺多,红色的纹理、黄色的花朵、青色的花萼,名称 是帝女桑。

又东四十五里,曰衡山,其上多青雘(hu^),多桑,其鸟多鸜(q*) 鹆(y()。

【译文】再往东四十五里,是座衡山,山上盛产青雘,还有茂密的桑树, 这里的禽鸟以八哥最多。

又东四十里,曰丰山,其上多封石,其木多桑,多羊桃,状如桃而方茎, 可以为皮张(zh4ng)①。

【注释】①为:治理。这里是治疗的意思。张:通“胀”。浮肿。

【译文】再往东四十里,是座丰山,山上多出产封石,这里的树木大多 是桑树,还有大量的羊桃,形状像一般的桃树却是方方的茎干,可以用它医 治人的皮肤肿胀病。

又东七十里,曰妪(y))山,其上多美玉,其下多金,其草多鸡谷。

【译文】再往东七十里,是座妪山,山上盛产优良玉石,山下盛产金, 这里的花草以鸡谷草最为繁盛。

又东三十里,曰鲜山,其木多楢(qi&)、杻(ni()、苴(zh1),其 草多■(m6n)冬①,其阳多金,其阴多铁。有兽焉,其状如膜(大)[犬] ②,赤喙(hu@)、赤目、白尾,见(xi4n)则其邑有火,名曰■(y@)即。

【注释】①■冬:就是现在称作蔷薇的蔓生植物,花,果、根都可入药或制造香料。②膜犬: 据古人说是西膜之犬,这种狗的体形高大,长着浓密的毛,性情猛悍,力量很大。

【译文】再往东三十里,是座鲜山,这里的树木以楢树、杻树、柤树最 多,花草以蔷薇最多,山南阳面有丰富的金,山北阴面有丰富的铁。山中有 一种野兽,形状像膜犬,长着红嘴巴、红眼睛、白尾巴,在哪个地方出现那 里就会有火灾,名称是■即。

又东三十里,曰章山,其阳多金,其阴多美石。皋水出焉,东流注于澧 (l@)水,其中多脃(cu@)石①。

【注释】①脃石:一种又轻又软而易断易碎的石头。脃,即“脆”的本字。

【译文】再往东三十里,是座章山,山南阳面多出产金,山北阴面多出 产漂亮的石头。皋水从这座山发源,向东流入澧水,水中有许多脃石。

又东二十五里,曰大支之山,其阳多金,其木多(穀)[榖](g^u)柞 (zu^),无草(木)。

【译文】再往东二十五里,是座大支山,山南阳面有丰富的金,这里的 树木大多是构树和柞树,但不生长草。

又东五十里,曰区吴之山,其木多苴(zh1)。

【译文】再往东五十里,是座区吴山,这里的树木以柤树为最繁盛。

又东五十里,曰声匈之山,其木多(穀)[榖] (g^u),多玉,上多封 石。

【译文】再往东五十里,是座声匈山,这里有茂密的构树,到处是玉石, 山上还盛产封石。

又东五十里,曰大騩(gu9)之山,其阳多赤金,其阴多砥(d!)石。

【译文】再往东五十里,是座大騩山,山南阳面多出产黄金,山北阴面 多出产细磨石。

又东十里,曰踵臼(ji))之山,无草木。

【译文】再往东十里,是座踵臼山,不生长花草树木。

又东北七十里,曰历石之山,其木多荆、芑(q!),其阳多黄金,其阴 多砥(d!)石。有兽焉,其状如狸,而白首虎爪,名曰梁渠,见(xi4n)则 其国有大兵。

【译文】再往东北七十里,是座历石山,这里的树木以牡荆和枸杞最多, 山南阳面盛产黄金,山北阴面盛产细磨石。山中有一种野兽,形状像野猫, 却长着白色的脑袋老虎一样的爪子,名称是梁渠,在哪个国家出现那个国家 里就会发生大战争。

又东南一百里,曰求山。求水出于其上,潜于其下,中有美赭(zh7)。

其木多苴(zh1),多■(m6i)。其阳多金,其阴多铁。

【译文】再往东南一百里,是座求山,求水从这座山顶上发源,潜流到 山下,这里有很多优良赭石。山中到处是柤树,还有矮小丛生的■竹。山南 阳面有丰富的金,山北阴面有丰富的铁。

又东二百里,曰丑阳之山,其上多椆(di1o)椐(j&)。有鸟焉,其状 如乌而赤足,名曰■(zh!)■(t*),可以御火。

【译文】再往东二百里,是座丑阳山,山上有茂密的椆树和椐树。山中 有一种禽鸟,形状像一般的乌鸦却长着红色爪子,名称是■■,人饲养它可 以辟火。

又东三百里,曰奥山,其上多柏、杻(ni()、橿(ji1ng),其阳多■ (y()琈之玉。奥水出焉,东流注于(视)[瀙]水。

【译文】再往东三百里,是座奥山,山上有茂密的松树、杻树、橿树, 山南阳面盛产■琈玉。奥水从这座山发源,向东流入瀙水。

又东三十五里,曰服山,其木多苴(zh1),其上多封石,其下多赤锡。

【译文】再往东三十五里,是座服山,这里的树木以柤树最多,山上有 丰富的封石,山下多出产红色锡土。

又东[三]百十里,曰杳(y3o)山,其上多嘉荣草,多金玉。

【译文】再往东三百一十里,是座杳山,山上到处是嘉荣草,还有丰富 的金属矿物和玉石。

又东三百五十里,曰■山,其木多楢(qi&)、檀(t2n)、杻(ni(), 其草多香。有兽焉,其状如彘(zh@),黄身、白头、白尾,名曰闻膦(1@n), 见(xi4n)则天下大风。

【译文】再往东三百五十里,是座■山,这里的树木,以楢树、檀树、 杻树最多,而草类主要是各种香草。山中有一种野兽,形状像普通的猪,却 是黄色的身子、白色的脑袋、白色的尾巴,名称是闻膦,一出现天下就会刮 起大风。

凡荆山之首,自翼望之山至于■山,凡四十八山,三千七百三十二里。

其神状皆彘身人首。其祠:毛用一雄鸡祈瘗(y@),[婴]用一珪,糈(x() 用五种之(精)[糈]。禾山①,帝也。其祠:太牢之具,羞瘗,倒毛②;[婴] 用一璧,牛无常。堵山、玉山,冢也,皆倒祠③,羞(毛)[用]少牢,婴(毛) [用]吉玉。

【注释】①禾山:这一山系并未述及禾山,不知是哪一山的误写。②倒毛:毛指毛物,即作 为祭品的牲畜。倒毛就是在祭礼举行完后,把猪、牛、羊三牲反倒着身子埋掉。③倒祠:也是倒毛的 意思。

【译文】总计荆山山系之首尾,自翼望山起到■山止,一共四十八座山, 途经三千七百三十二里。诸山山神的形貌都是猪的身子而人的头。祭祀山 神:在毛物中用一只公鸡来祭祀后而埋入地下,在祀神的玉器中用一块玉珪 献祭,祀神的米用黍、稷、稻、粱、麦五种粮米。禾山,是诸山的首领。祭 祀禾山山神:在毛物中用猪、牛、羊齐全的三牲作祭品,进献后埋入地下, 而且将牲畜倒着埋;在祀神的玉器中用一块玉璧献祭,但也不必三牲全备。

堵山、玉山,是诸山的宗主,祭祀后都要将牲畜倒着埋掉,进献的祭祀品是 用猪、羊,在祀神的玉器中要用一块吉玉。

中次十二(经)[山]洞庭山首,曰篇遇之山,无草木,多黄金。

【译文】中央第十二列山系洞庭山山系之首座山,是座篇遇山,这里不 生花草树木,蕴藏着丰富的黄金。

又东南五十里,曰云山,无草木。有桂竹①,甚毒,伤人必死②。其上 多黄金,其下多■(y&)琈之玉。

【注释】①桂竹:竹子的一种。古人说它有四、五丈高,茎干合围有二尺粗,叶大节长,形 状像甘竹而皮是红色。②伤:刺的意思。作动词用。

【译文】再往东南五十里,是座云山,不生长花草树木。但有一种桂竹, 毒性特别大,枝叶刺着人就必死。山上盛产黄金,山下盛产■琈玉。

又东南一百三十里,曰龟山,其木多(穀)[榖](g^u)、柞(zu^)、 椆(di1o)、椐(j&),其上多黄金,其下多青、雄黄,多扶竹①。

【注释】①扶竹:即邛(qi¥ng)竹。节杆较长,中间实心,可以制做手杖,所以又叫扶老 竹。

【译文】再往东南一百三十里,是座龟山,这里的树木以构树、柞树、 椆树、椐树最为繁盛,山上多出产黄金,山下多出产石青、雄黄,还有很多 扶竹。

又东七十里,曰丙山,多筀竹①,多黄金、铜、铁,无木。

【注释】①筀竹:就是桂竹。据古人讲,因它是生长在桂阳地方的竹子,所以叫做桂竹。

【译文】再往东七十里,是座丙山,有茂密的桂竹,还有丰富的黄金、 铜、铁,但没有树木。

又东南五十里,曰风伯之山,其上多金玉,其下多痠(su1n)石、文石 ①,多铁,其木多柳、杻(ni()、檀(t2n)、楮(ch()。其东有林焉, 曰莽浮之林,多美木鸟鲁。

【注释】①痠石:不详何样石头。

【译文】再往东南五十里,是座风伯山,山上有丰富的金属矿物和玉石, 山下盛产痠石、色彩斑斓的漂亮石头,还盛产铁,这里的树木以柳树、杻树、 檀树、构树最盛多。在风伯山东面有一片树林,叫做莽浮林,其中有许多的 优良树木和禽鸟野兽。

又东一百五十里,曰夫夫之山,其上多黄金,其下多青、雄黄,其木多 桑、楮(ch(),其草多竹、鸡鼓①。神于儿居之,其状人身而(身)[手] 操两蛇,常游于江渊,出入有光。

【注释】①鸡鼓:即上文所说的鸡谷草。鼓、谷二字音同而假借。

【译文】再往东一百五十里,是座夫夫山,山上多出产黄金,山下多出 产石青、雄黄,这里的树木以桑树、构树最多,而花草以竹子、鸡谷草最为 繁盛。神仙于儿就住在这座山里,形貌是人的身子却手握两条蛇,常常游玩 于长江水的深渊中,出没时都有闪光。

又东南一百二十里,曰洞庭之山,其上多黄金,其下多银铁,其木多柤 (zh1)、梨、橘、櫾(y^u),其草多葌(ji1n)、蘪(m6i)芜、芍药、 芎(xi¥ng)①。帝之二女居之,是常游于江渊。澧(l@)沅之风,交潇 (xi1o)湘之渊②,是在九江之间,出入必以飘风暴雨。是多怪神,状如人 而载蛇③,左右手操蛇。多怪鸟。

【注释】①蘪芜:一种香草,可以入药。②潇:水又清又深的样子。③载:戴。这里是缠绕 的意思。

【译文】再往东南一百二十里,是座洞庭山,山上多出产黄金,山下多 出产银和铁,这里的树木以柤树、梨树、橘子树、柚子树居多,而花草以兰 草、蘪芜、芍药、芎等香草居多。天帝的两个女儿住在这座山里,她俩常 在长江水的深渊中游玩。从澧水和沅水吹来的清风,交会在幽清的湘水渊潭 上,这里正是九条江水汇合的中间,她俩出入时都有旋风急雨相伴随。洞庭 山中还住着很多怪神,形貌像人而身上绕着蛇,左右两只手也握着蛇。这里 还有许多怪鸟。

又东南一百八十里,曰暴山,其木多棕、枏(n2n)、荆、芑(q!)、 竹、箭、■(m6i)、箘(j)n)①,其上多黄金、玉,其下多文石、铁,其 兽多麋、鹿、■(j!)②,[其鸟多]就③。

【注释】①箘:一种小竹子,可以制做箭杆。②■:同“麂”,一种小型鹿,仅雄性有角。

③就:即鹫,一种大型猛禽,属于雕鹰之类。就、鹫二字同音而假借。

【译文】再往东南一百八十里,是座暴山,在茂密的草木中以棕树、楠 木树、牡荆树、枸杞树和竹子、箭竹、■竹、箘竹居多,山上多出产黄金、 玉石,山下多出产彩色花纹的漂亮石头、铁,这里的野兽以麋鹿、鹿、麂居 多,这里的禽鸟大多是鹫鹰。

又东南二百里,曰即公之山,其上多黄金,其下多■(y()琈之玉,其 木多柳、杻(ni()、檀(t2n)、桑。有兽焉,其状如龟,而自身赤首,名 曰蛫(gu!),是可以御火。

【译文】再往东南二百里,是座即公山,山上多出产黄金,山下多出产 ■琈玉,这里的树木以柳树、杻树、檀树、桑树最多。山中生长着一种野兽, 形状像一般的乌龟,却是白身子红脑袋,名称是蛫,人饲养它可以辟火。

又东南一百五十九里,曰尧山,其阴多黄垩(6),其阳多黄金,其木 多荆、芑(q!)、柳、檀(t2n),其草多薯(sh()■、■(zh*)。

【译文】再往东南一百五十九里,是座尧山,山北阴面多出产黄色垩土, 山南阳面多出产黄金,这里的树木以牡荆树、枸杞树、柳树、檀树最多,而 草以山药、苍术或白术最为繁盛。

又东南一百里,曰江浮之山,其上多银、砥(d!)砺(l0),无草木, 其兽多豕(sh!),鹿。

【译文】再往东南一百里,是座江浮山,山上盛产银、磨石,这里没有 花草树木,而野兽以野猪、鹿居多。

又东二百里,曰真陵之山,其上多黄金,其下多玉,其木多(穀)[榖] (g^u)、柞(zu^)、柳、杻(ni(),其草多荣草。

【译文】再往东二百里,是座真陵山,山上多出产黄金,山下多出产玉 石,这里的树木以构树、柞树、柳树、杻树最多,而草大多是可以医治风痹 病的荣草。

又东南一百二十里,曰阳帝之山,多美铜,其木多橿(ji1ng)、杻、 檿(y3n)、楮(ch()①,其兽多麢(l0ng)麝(sh6)。

【注释】①檿:即山桑,是一种野生桑树,木质坚硬,可以制做弓和车辕。

【译文】再往东南一百二十里,是座阳帝山,到处是优质铜,这里的树 木大多是橿树、杻树、山桑树、楮树,而野兽以羚羊和麝香鹿最多。

又南九十里,曰柴桑之山,其上多银,其下多碧,多(泠)[汵](j9n) 石、赭(zh7),其木多柳、芑(q!)、楮(ch()、桑,其兽多麋、鹿,多 白蛇、飞蛇①。

【注释】①飞蛇:即螣(teng)蛇,也作“腾蛇”。传说是能够乘雾腾云而飞行的蛇,属于 龙一类。

【译文】再往南九十里,是座柴桑山,山上盛产银,山下盛产碧玉,到 处是柔软如泥的汵石、赭石,这里的树木以柳树、枸杞树、楮树、桑树居多, 而野兽以麋鹿、鹿居多,还有许多白色蛇、飞蛇。

又东二百三十里,曰荣余之山,其上多铜,其下多银,其木多柳、芑(q!), 其虫多怪蛇、怪虫①。

【注释】①虫:古时南方人也称蛇为虫。

【译文】再往东二百三十里,是座荣余山,山上多出产铜,山下多出产 银,这里的树木大多是柳树、枸杞树,这里的虫类有很多怪蛇、怪虫。

凡洞庭山之首,自篇遇之山至于荣余之山,凡十五山,二千八百里。其 神状皆鸟身而龙首。其祠:毛用一雄鸡、一牝(p@n)豚(t*n)刏(j9), 糈(x()用稌(t*)。凡夫夫之山、即公之山、尧山、阳帝之山,皆冢也, 其祠:皆肆瘗(y@)①,祈用酒,毛用少牢,婴(毛)[用]一吉玉。洞庭、 荣余山,神也,其祠:皆肆瘗,祈酒太牢祠,婴用圭璧十五,五采惠之②。

【注释】①肆:陈设。②惠:这里是绘的意思。惠、绘二字同音而假借。

【译文】总计洞庭山山系之首尾,自篇遇山起到荣余山止,一共十五座 山,途经二千八百里。诸山山神的形貌都是鸟的身子龙的脑袋。祭祀山神: 在毛物中宰杀一只公鸡、一头母猪作祭品,祀神的米用稻米。凡夫夫山、即 公山、尧山、阳帝山,都是诸山的宗主,祭祀这几座山的山神:都要陈列牲 畜、玉器而后埋入地下,祈神用美酒献祭,在毛物中用猪、羊二牲作祭品, 在祀神的玉器中要用吉玉。洞庭山、荣余山,是神灵显应之山,祭祀这二位 山神:都要陈列牲畜、玉器而后埋入地下,祈神用美酒及猪、牛、羊齐全的 三牲献祭,祀神的玉器要用十五块玉圭十五块玉璧,用青、黄、赤、白、黑 五样色彩绘饰它们。

右中经之山志,大凡百九十七山,二万一千三百七十一里。

【译文】以上是中央经历之山的记录,总共一百九十七座山,二万一千 三百七十一里。

大凡天下名山五千三百七十,居地,大凡六万四千五十六里。

【译文】总计天下名山共有五千三百七十座,分布在大地之东西南北中 各方,一共六万四千零五十六里。

禹曰:天下名山,经五千三百七十山,六万四千五十六里,居地也。言 其《五臧(z4ng)》①,盖其余小山甚众,不足记云。天地之东西二万八千 里,南北二万六千里,出水之山者八千里,受水者八千里,出铜之山四百六 十七,出铁之山三千六百九十。此天地之所分壤树谷也②,戈矛之所发也, 刀铩(sh1)之所起也③,能者有余,拙者不足。封于太山④,禅(sh4n) 于梁父⑤,七十二家,得失之数⑥,皆在此内,是谓国用⑦。

【注释】①五臧:即五脏。臧,通“脏”。五脏,指人的脾、肺、肾、肝、心等五种主要器 官。这里用来比喻《五臧山经》中所记的重要大山,如同人的五脏六腑似的,也是天地山海之间的五 脏。②树:种植,栽培。谷:这里泛指农作物。③铩:古代一种兵器,即铍(p9)。大矛。④封:古 时把帝王在泰山上筑坛祭天的活动称为“封”。太山:即泰山。⑤禅:古时把帝王在泰山南面的小山 梁父山上辟基祭地的活动称为“禅”。⑥数:命运。⑦据学者研究,这一段话非本书原有,是先秦人 的相传之语及注释的话,后被校勘本书的人采录而附于此。因底本原有,故一仍其旧。

【译文】大禹说:天下的名山,经历了五千三百七十座,六万四千零五 十六里,这些山分布在大地东西南北中各方。把以上山记在《五臧山经》中, 原因是除此以外的小山太多,不值得一一记述。广阔的天地从东方到西方共 二万八千里,从南方到北方共二万六千里,江河源头所在之山是八千里,江 河流经之地是八千里,出产铜的山有四百六十七座,出产铁的山有三千六百 九十座。这些是天下地上划分疆土、种植庄稼的凭借,也是戈和矛产生的缘 故,刀和铩兴起的根源,因而能干的人富裕有余,笨拙的人贫穷不足。国君 在泰山上行祭天礼,在梁父山上行祭地礼,一共有七十二家,或得或失的运 数,都在这个范围内,国家财用也可以说是从这块大地取得的。

右《五臧山经》五篇,大凡一万五千五百三字。

【译文】以上是《五臧山经》五篇,一共有一万五千五百零三个字。


\chapter{海外南经}

山海经卷六 海外南经

小.说。t/x/t天.堂
地之所载,六合之间①,四海之内,照之以日月,经之以星辰,纪之以 四时②,要之以太岁③。神灵所生,其物异形,或夭或寿,唯圣人能通其道。

【注释】①六合:古人以东、西、南、北、上、下六方为六合。②四时:古人以春、夏、秋、 冬四季为四时。③太岁:又叫岁星,即木星。木星在黄道带里每年经过一宫,约十二年运行一周天, 所以古人用以纪年。

【译文】大地所负载的,包括上下四方之间的万物,在四海以内,有太 阳和月亮照明,有大小星辰经历,又有春夏秋冬记季节,还有太岁正天时。

大地上的一切都是神灵造化所生成,故万物各有不同的形状,有的夭折而有 的长寿,只有圣明之人才能懂得其中的道理。

海外自西南陬(z#u)至东南陬者①。

【注释】①陬:角。又本书自《海外南经》以下各篇,大概最早成书时先有图画,后有文字, 而文字只是说明图画的 所以,每篇一开始都有表示方位的一句话,像本篇的“海外自西南陬至东南 陬者”一句就是。

【译文】海外从西南角到东南角的国家地区、山丘河川分别如下。

结匈国在其西南①,其为人结匈②。

【注释】①其:代指邻近结匈国的灭蒙鸟。而灭蒙鸟在结匈国的北边,参看本书《海外西经》。

②结匈:可能指现在所说的鸡胸。匈,同“胸”。

【译文】结胸国在灭蒙鸟的西南面,那里的人都长着像鸡一样尖削凸出 的胸脯。

南山在其东南①。自此山来,虫为蛇,蛇号为鱼。一曰南山在结匈东南。

【注释】①其:也是代指灭蒙鸟,否则,后面“一曰南山在结匈东南”一句就重复而多余了。

以下同此。

【译文】南山在灭蒙鸟的东南面。从这座山来的人,把虫叫做蛇,把蛇 叫做鱼。也有一种说法认为南山在结胸国的东南面。

比翼鸟在其东,其为鸟青、赤,两鸟比翼。一曰在南山东。

【译文】比翼鸟在灭蒙鸟的东面,它作为一种鸟有青色、红色间杂的羽 毛,两只鸟的翅膀配合起来才能飞翔。也有一种说法认为比翼鸟在南山的东 面。

羽民国在其东南,其为人长头,身生羽。一曰在比翼鸟东南,其为人长 颊(ji2)①。

【注释】①颊:面颊,脸的两侧。

【译文】羽民国在灭蒙鸟的东南面,那里的人都长着长长的脑袋,全身 生满羽毛。另一种说法认为羽民国在比翼鸟的东南面,那里的人都长着一副 长长的脸颊。

有神人二八,连臂,为帝司夜于此野①。在羽民东,其为人小颊赤肩, 尽十六人。

【注释】①司:视察。这里是守候的意思。

【译文】有叫二八的神人,手臂连在一起,在这旷野中为天帝守夜。这 位神人在羽民国的东面,那里的人都是狭小的脸颊和赤红的肩膀,总共有十 六个人。

毕方鸟在其东,青水西,其为鸟人面一脚。一曰在二八神东。

【译文】毕方鸟在它的东面,在青水的西面,这种鸟长着一副人的面孔 却是一只脚。另一种说法认为毕方鸟在二八神人的东面。

讙(hu1n)头国在其南,其为人人面有翼,鸟喙(hu@),方捕鱼①。

一曰在毕方东。或曰讙朱国。

【注释】①方:正在,正当。因为是配合图画的说明文字,所以出现了这种记述具体的一举 一动的词语。以下此类词语尚多。

【译文】讙头国在它的南面,那里的人都是人的面孔却有两只翅膀,还 长着鸟嘴,正在用它们的鸟嘴捕鱼。另一种说法认为讙头国在毕方鸟的东 面。还有人认为讙头国就是讙朱国。

厌火国在其(国)南,[其为人]兽身黑色,(生)火出其口中。一曰在 讙朱东。

【译文】厌火国在它的南面,那里的人都长着野兽一样的身子而且是黑 色的,火从他们的口中吐出。另一种说法认为厌火国在讙朱国的东面。

三(株)[珠]树在厌火北,生赤水上,其为树如柏,叶皆为珠。一曰其 为树若彗①。

【注释】①彗:即彗星。因为它拖着一条又长又散的尾巴就像扫帚,所以通常也称为扫帚星。

这里实际是指树的形状像一把扫帚。

【译文】三珠树在厌火国的北面,生长在赤水岸边,那里的树与普通的 柏树相似,叶子都是珍珠。另一种说法认为那里的树像彗星的样子。

三苗国在赤水东,其为人相随。一曰三毛国。

【译文】三苗国在赤水的东面,那里的人是一个跟着一个地行走。另一 种说法认为三苗国就是三毛国。

戴(zh@)国在其东,其为人黄,能操弓射蛇。一曰(臷)[盛]国在三 毛东。

【译文】臷国在它的东面,那里的人都是黄色皮肤,能操持弓箭射死蛇。

另一种说法认为盛国在三毛国的东面。

贯匈国在其东,其为人匈有窍。一曰在臷国东。

【译文】贯胸国在它的东边,那里的人都是胸膛上穿个洞。另一种说法 认为贯胸国在臷国的东面。

交胫(j@ng)国在其东,其为人交胫①。一曰在穿匈东②。

【注释】①胫:人的小腿。这里指整个腿脚。②穿匈:即贯匈国。穿、贯二字的音义相同。

【译文】交胫国在它的东面,那里的人总是互相交叉着双腿双脚。另一 种说法认为交胫国在穿胸国的东面。

不死民在其东,其为人黑色,寿[考]①,不死。一曰在穿匈国东。

【注释】①考:老。指长寿。

【译文】不死民在它的东面,那里人的都是黑色的,个个长寿,人人不 死。另一种说法认为不死民在穿胸国的东面。

(岐)[反]舌国在其东,[其为人反舌]。一曰在不死民东。

【译文】反舌国在它的东面,那里的人都是舌根在前、舌尖伸向喉部。

另一种说法认为反舌国在不死民的东面。

昆仑虚(q&)在其东①,虚(q&)四方②。一曰在(岐)[反]舌东,为 虚四方。

【注释】①虚:大丘。这里是山的意思。②虚:所在地。这里指山下底部地基。

【译文】昆仑山在它的东面,山基呈四方形。另一种说法认为昆仑山在 反舌国的东面,山基向四方延伸。

羿(y@)与凿齿战于寿华之野①,羿射杀之。在昆仑虚(q&)东。羿持 弓矢,凿齿持盾。一曰[持]戈。

【注释】①羿:神话传说中的天神。凿齿:传说是亦人亦兽的神人,有一个牙齿露在嘴外, 有五、六尺长,形状像一把凿子。

【译文】羿与凿齿在寿华的荒野交战厮杀,羿射死了凿齿。地方就在昆 仑山的东面。在那次交战中羿手拿弓箭,凿齿手拿盾牌。另一种说法认为凿 齿拿着戈。

三首国在其东,其为人一身三首。

【译文】三首国在它的东面,那里的人都是一个身子三个头。

周饶国在其东,其为人短小,冠带①。一曰焦侥国在三首东②。

【注释】①冠带:这里都作动词用,即戴上冠帽、系上衣带。②焦侥国:传说此国与周饶国 的人只有三尺高。而“焦侥”、“周饶”都是“侏儒”之声转。侏儒是短小的人。则焦侥国即周饶国, 就是现在所说的小人国。

【译文】周饶国在它的东面,那里的人都是身材矮小的,戴帽子系腰带 而整齐讲究。另一种说法认为周饶国在三首国的东面。

长臂国在其东,捕鱼水中,两手各操一鱼。一曰在焦侥东,捕鱼海中。

【译文】长臂国在它的东面,那里的人正在水中捕鱼,左右两只手各抓 着一条鱼。另一种说法认为长臂国在焦侥国的东面,那里的人是在大海中捕 鱼的。

狄山,帝尧葬于阳,帝喾(k))葬于阴①。爰有熊、罴(p0)、文虎、 蜼(w7i)、豹、离朱、视肉②。吁咽、文王皆葬其所③。一曰汤山。一曰 爰有熊、罴、文虎、蜼、豹、离朱、(■)[鸱](ch9)久、视肉、虖交④。

【注释】①帝喾:传说中的上古帝王唐尧的父亲。②离朱:可能是神话传说中的三足鸟。这 种鸟在太阳里,与乌鸦相似,但长着三只足。视肉:传说中的一种怪兽,形状像牛肝,有两只眼睛, 割去它的肉吃了后,不长时间就又重新生长出来,完好如故。③吁咽:可能指传说中的上古帝王虞舜。

文王:即周文王姬昌,是周朝开国君主。④虖交:不详何物。

【译文】狄山,唐尧死后葬在这座山的南面,帝喾死后葬在这座山的北 面。这里有熊、罴、花斑虎、长尾猿、豹子、三足乌、视肉。吁咽和文王也 埋葬在这里。另一种说法认为是在汤山。还有一种说法认为这里有熊、罴、 花斑虎、长尾猿、豹子、离朱鸟、鹞鹰、视肉、虖交。

(其)[有]范林方三百里①。

【注释】①范林:树林繁衍茂密。

【译文】有一片方圆三百里大小的范林。

南方祝融①,兽身人面,乘两龙。

【注释】①祝融:神话传说中的火神。

【译文】南方的祝融神,长着野兽的身子人的面孔,乘着两条龙。

\chapter{海外西经}

山海经卷七 海外西经

~小  说t  xt 天,堂
海外自西南陬(z#u)至西北陬者。

【译文】海外从西南角到西北角的国家地区、山丘河川分别如下。

灭蒙鸟在结匈国北,为鸟青,赤尾。

【译文】灭蒙鸟在结胸国的北面,那里的鸟是青色羽毛,拖着红色尾巴。

大运山高三百仞(r6n)①,在灭蒙鸟北。

【注释】①仞:古代的八尺合一仞。

【译文】大运山高三百仞,屹立在灭蒙鸟的北面。

大乐(y6)之野,夏后启于此儛(w()《九代》①,乘两龙,云盖三层。

左手操翳(y@)②,右手操环,佩玉璜(hu2ng)③。在大运山北。一曰大 遗之野。

【注释】①夏后启:传说是夏朝开国君主大禹的儿子,夏朝第一代国君。夏后,即夏王。儛: 同“舞”。②翳:用羽毛做的像伞形状的华盖。③璜:一种半圆形玉器。

【译文】大乐野,夏后启在这地方观看《九代》乐舞,乘驾着两条龙, 飞腾在三重云雾之上。他左手握着一把华盖,右手拿着一只玉环,腰间佩挂 着一块玉璜。大乐野就在大运山的北面。另一种说法认为夏后启观看乐舞《九 代》是在大遗野。

三身国在夏后启北,一首而三身。

【译文】三身国在夏后启所在之地的北面,那里的人都长着一个脑袋三 个身子。

一臂国在其北,一臂、一目、一鼻孔。有黄马,虎文,一目而一手①。

【注释】①手:这里指马的腿蹄。

【译文】一臂国在三身国的北面,那里的人都是一条胳膊、一只眼睛、 一个鼻孔。那里还有黄色的马,身上有老虎斑纹,长着一只眼睛和一条腿蹄。

奇(j9)肱(g#ng)之国在其北。其人一臂三目,有阴有阳,乘文马①。

有鸟焉,两头,赤黄色,在其旁。

【注释】①文马:即吉良马,白身子红鬃毛,眼睛像黄金,骑上它,寿命可达一千年。

【译文】奇肱国在一臂国的北面。那里的人都是一条胳膊和三只眼睛, 眼睛分为阴阳而阴在上阳在下,骑着名叫吉良的马。那里还有一种鸟,长着 两个脑袋,红黄色的身子,栖息在他们的身旁。

形天与帝(至此)争神①,帝断其首,葬之常羊之山。乃以乳为目,以 脐为口,操干戚以舞。

【注释】①形天:即刑天,是神话传说中一个没有头的神。形,通“刑”,割、杀之意。天 是颠顶之意,指人的头。刑天就是砍断头。所以,此神原本无名,在被断首之后才有了刑天神的名称。

【译文】刑天与天帝争夺神位,天帝砍断了刑天的头,把他的头埋在常 羊山。没了头的刑天便以乳头做眼睛,以肚脐做嘴巴,一手持盾牌一手操大 斧而舞动。

女祭、女(戚)[薎(mi6)]在其北,居两水间,(戚)[薎]操(鱼薎) [鱼■(zh@)]①,祭操俎(z()②。

【注释】①觛:就是小觯。觛是古代的一种酒器。②俎:古代祭祀时盛供品的礼器。

【译文】叫做祭的女巫、叫做薎的女巫住在刑天与天帝发生争斗之地的 北面,正好处于两条水流的中间,女巫薎手里拿着兕角小酒杯,女巫祭手里 捧着俎器。

■(c@)鸟、■(zh1n)鸟,其色青黄,所经国亡。在女祭北。■鸟人 面,居山上。一曰维鸟,青鸟、黄鸟所集。

【译文】一种■鸟、一种■鸟,它们的颜色是青中带黄,经过哪个国家 那个国家就会败亡。■鸟和■鸟栖息在女巫祭的北面。■鸟长着人的面孔, 立在山上。另一种说法认为这两种鸟统称维鸟,是青色鸟、黄色鸟聚集在一 起的混称。

丈夫国在维鸟北,其为人衣冠带剑。

【译文】丈夫国在维鸟的北面,那里的人都是穿衣戴帽而佩带宝剑的模 样。

女丑之尸,生而十日炙(zh@)杀之①。在丈夫北。以右手鄣(zh4ng) 其面②。十日居上,女丑居山之上。

【注释】①炙:烧烤。②鄣:同“障”。挡住,遮掩。

【译文】有一具女丑的尸体,她生前是被十个太阳的热气烤死的。她横 卧在丈夫国的北面。死时用右手遮住她的脸。十个太阳高高挂在天上,女丑 的尸体横卧在山顶上。

巫咸国在女丑北,右手操青蛇,左手操赤蛇。在登葆山,群巫所从上下 也。

【译文】巫咸国在女丑的北面,那里的人是右手握着一条青蛇,左手握 着一条红蛇。有座登葆山,是一群巫师来往于天上与人间的地方。

并封在巫咸东,其状如彘(zh@),前后皆有首,黑。

【译文】称作并封的怪兽在巫咸国的东面,它的形状像普通的猪,却前 后都有头,是黑色的。

女子国在巫咸北,两女子居,水周之。一曰居一门中。

【译文】女子国在巫咸国的北面,有两个女子住在这里,四周有水环绕 着。另一种说法认为她们住在一道门的中间。

轩(xu1n)辕(yu2n)之国在(此)穷山之际,其不寿者八百岁。在女 子国北,人面蛇身,尾交首上。

【译文】轩辕国在穷山的旁边,那里的人就是不长寿的也能活八百岁。

轩辕国在女子国的北面,他们长着人的面孔却是蛇的身子,尾巴盘绕在头顶 上。

穷山在其北,不敢西射,畏轩辕之丘。在轩辕国北,其丘方,四蛇相绕。

【译文】穷山在轩辕国的北面,那里的人拉弓射箭不敢向着西方射,是 因为敬畏黄帝威灵所在的轩辕丘。轩辕丘位于轩辕国北部,这个轩辕丘呈方 形,被四条大蛇相互围绕着。

(此)诸(夭)[沃]之野,鸾鸟自歌,凤鸟自舞;凤皇卵,民食之;甘 露①,民饮之:所欲自从也。百兽相与群居。在四蛇北,其人两手操卵食之, 两鸟居前导之。

【注释】①甘露:古人所谓甜美的露水,以为天下太平,则天降甘露。

【译文】有个叫做沃野的地方,鸾鸟自由自在地歌唱,凤鸟自由自在地 舞蹈;凤皇生下的蛋,那里的居民食用它;苍天降下的甘露,那里的居民饮 用它:凡是他们所想要的都能随心如意。那里的各种野兽与人一起居住。沃 野在四条蛇的北面,那里的人用双手捧着凤皇蛋正在吃,有两只鸟在前面引 导着。

龙鱼陵居在其北,状如(狸)[鲤]。一曰鰕(xi1)①。即有神圣乘此 以行九野②。一曰鳖鱼在(夭)[沃]野北,其为鱼也如鲤。

【注释】①鰕:体型大的鲵(n0)鱼叫做鰕鱼。鲵鱼是一种水陆两栖类动物,有四只脚,长 尾巴,眼小口大,生活在山谷溪水中。因叫声如同小孩啼哭,所以俗称娃娃鱼。②九:表示多数。这 里是广大的意思。

【译文】既可在水中居住又可在山陵居住的龙鱼在沃野的北面,龙鱼的 形状像一般的鲤鱼。另一种说法认为像鰕鱼。就有神圣的人骑着它遨游在广 大的原野上。还有一种说法认为鳖鱼在沃野的北面,这种鱼的形状也与鲤鱼 相似。

白民之国在龙鱼北,白身被(p0)发①。有乘黄,其状如狐,其背上有 角,乘之寿二千岁。

【注释】①被:通“披”。

【译文】白民国在龙鱼所在地的北面,那里的人都是白皮肤而披散着头 发。有一种叫做乘黄的野兽,形状像一般的狐狸,脊背上有角,人要是骑上 它就能活两千年的长寿。

肃慎之国在白民北。有树名曰雄常,(先入伐帝)[圣人代立],于此取 之①。

【注释】①圣人代立,于此取之:据古人解说,肃慎国的习俗是人们平时没衣服,一旦中原 地区有英明的帝王继立,那么,常雄树就生长出一种树皮,那里的人取它可以制成衣服穿。

【译文】肃慎国在白民国的北面。有一种树木叫做雄常树,每当中原地 区有圣明的天子继位,那里的人就取雄常树的树皮来做衣服。

长股之国在雄常北,被(p0)发。一曰长脚。

【译文】长股国在雄常树的北面,那里的人都披散着头发。另一种说法 认为长股国叫长脚国。

西方蓐(r))收①,左耳有蛇,乘两龙。

【注释】①蓐收:神话传说中的金神,样子是人面孔、虎爪子、白毛发,手执钺(y)e)斧。

【译文】西方的蓐收神,左耳上有一条蛇,乘驾两条龙飞行。

\chapter{海外北经}

山海经卷八 海外北经

小<说<t<xt>天?堂
海外自(东)[西]北陬(z#u)至(西)[东]北陬者。

【译文】海外从西北角到东北角的国家地区、山丘河川分别如下。

无(■)[启](q!)之国在长股东,为人无(■)[启]①。

【注释】①无启:无嗣。传说无启国的人住在洞穴中,平常吃泥土,不分男女,一死就埋了, 但他们的心不腐朽,死后一百二十年就又重新化成人。

【译文】无启国在长股国的东面,那里的人不生育子孙后代。

钟山之神,名曰烛阴,视为昼,瞑为夜,吹为冬,呼为夏,不饮,不食, 不息,息为风,身长千里。在无(■)[启]之东。其为物,人面,蛇身,赤 色,居钟山下。

【译文】钟山的山神,名叫烛阴,他睁开眼睛便是白昼,闭上眼睛便是 黑夜,一吹气便是寒冬,一呼气便是炎夏,不喝水,不吃食物,不呼吸,一 呼吸就生成风,身子有一千里长。这位烛阴神在无启国的东面。他的形貌是 人一样的面孔,蛇一样的身子,全身赤红色,住在钟山脚下。

一目国在其东,一目中其面而居。一曰有手足。

【译文】一目国在钟山的东面,那里的人是在脸的中间长着一只眼睛。

另一种说法认为像普通的人有手有脚。

柔利国在一目东,为人一手一足,反膝,曲足居上。一云留利之国,人 足反折。

【译文】柔利国在一目国的东面,那里的人是一只手一只脚,膝盖反长 着,脚弯曲朝上。另一种说法认为柔利国叫做留利国,人的脚是反折着的。

共工之臣曰相柳氏,九首,以食于九山。相柳之所抵,厥(ju6)为泽 溪①。禹杀相柳,其血腥,不可以树五谷(种)②。禹厥之,三仞(r6n) 三沮(j()③,乃以为众帝之台④。在昆仑之北,柔利之东。相柳者,九首 人面,蛇身而青。不敢北射,畏共工之台。台在其东,台四方,隅(y*)有 一蛇⑤,虎色⑥,首冲南方。

【注释】①厥:通“撅”。掘。②五谷:五种谷物。泛指庄稼。③三:表示多数。仞:通“■”, 充满。沮:败坏。这里是陷落的意思。④众帝:指帝尧、帝喾、帝丹朱、帝舜等传说中的上古帝王。

⑤隅:角落。⑥虎色:虎文,即老虎皮毛的颜色纹理。

【译文】天神共工的臣子叫相柳氏,有九个头,九个头分别在九座山上 吃食物。相柳氏所触动之处,便掘成沼泽和溪流。大禹杀死了相柳氏,血流 过的地方发出腥臭味,不能种植五谷。大禹挖填这地方,多次填满而多次蹋 陷下去,于是大禹便把挖掘出来的泥土为众帝修造了帝台。这帝台在昆仑山 的北面,柔利国的东面。这个相柳氏,长着九个脑袋和人的面孔,蛇的身子 而是青色。射箭的人不敢向北方射,因为敬畏共工威灵所在的共工台。共工 台在相柳的东面,台是四方形的,每个角上有一条蛇,身上的斑纹与老虎相 似,头向着南方。

深目国在其东,为人举一手。一(目)曰在共工台东。

【译文】深目国在相柳氏所在地的东面,那里的人总是举起一只手。另 一种说法认为深目国在共工台的东面。

无肠之国在深目东,其为人长而无肠。

【译文】无肠国在深目国的东面,那里的人身体高大而肚子里却没有肠 子。

聂(sh6)耳之国在无肠国东,使两文虎①,为人两手聂其耳②。县(xu2n) 居海水中③,及水所出入奇物。两虎在其东④。

【注释】①文虎:即雕虎,老虎身上的花纹如同雕画似的。②聂:通“摄”。握持。③县: 同“悬”。无所依倚。这里是孤单的意思。④两虎:即指上文的两文虎,原来图上的画像就是如此。

这种说明图上画面的物像、人形、山状、方向、位置等文字,自《海外南经》以下比比皆是。因为原 画已失,仅剩说明文字,图画与文字不能配合着视读,所以,文中所指所述往往有不甚了了之感。

【译文】聂耳国在无肠国的东面,那里的人使唤着两只花斑大虎,并且 在行走时用手托着自己的大耳朵。聂耳国在海水环绕的孤岛上,所以能看到 出入海水的各种怪物。有两只老虎在它的东面。

夸父与日逐(zh^u)走,入日。渴,欲得饮,饮于河渭,河渭不足,北 饮大泽,未至,道渴而死。弃其杖,化为邓林。

【译文】神人夸父要与太阳赛跑而追赶它,已追上了太阳。这时夸父很 渴,想要喝水,于是喝黄河和渭河中的水,喝完了两条河水还是不解渴,又 要向北去喝大泽中的水,还没走到,就渴死在半路上了。他死时所抛掉的拐 杖,变成了邓林。

(博)[夸]父国在聂(sh6)耳东,其为人大,右手操青蛇,左手操黄 蛇。邓林在其东,二树木。一曰博父。

【译文】夸父国在聂耳国的东面,那里的人身体高大,右手握着青色蛇, 左手握着黄色蛇。邓林在它的东面,其实由两棵非常大的树木形成了树林。

另一种说法认为夸父国叫博父国。

禹所积石之山在其东①,河水所入。

【注释】①禹所积石之山:是一座山,即禹所积石山。传说大禹曾疏通积石山而导引黄河水 流过。此积石山是另一座山,不是这里所说的禹所积石山。

【译文】禹所积石山在博父国的东面,是黄河流入的地方。

拘(缨)[瘿(y!ng)]之国在其东,一手把(缨)[瘿]①。一曰利(缨) [瘿]之国。

【注释】①瘿:因脖颈细胞增生而形成的囊状性赘生物,多肉质,比较大。

【译文】拘瘿国在禹所积石山的东面,那里的人常用一只手托着脖颈上 的大肉瘤。另一种说法认为拘瘿国叫做利瘿国。

寻木长千里,在拘(缨)[瘿]南,生河上西北。

【译文】有种叫做寻木的树有一千里长,在拘瘿国的南面,生长在黄河 岸上的西北方。

跂(q@)踵(zh%ng)国在拘(缨)[瘿]东①,其为人大,两足亦大。

一曰(大)[反]踵②。

【注释】①跂踵:走路时脚跟不着地。②反踵:脚是反转长的,走路时行进的方向和脚印的 方向是相反的。

【译文】跂踵国在拘瘿国的东面,那里的人都是身材高大,两只脚也非 常大。另一种说法认为跂踵国叫反踵国。

欧丝之野在(大)[反]踵东,一女子跪据树欧丝①。

【注释】①据树:据古人解说,是凭依桑树一边吃桑叶一边吐出丝,像蚕似的。这大概是图 画上的形状。欧:同“呕”。吐。

【译文】欧丝野在反踵国的东面,有一女子跪倚着桑树在吐丝。

三桑无枝,在欧丝东,其木长百仞①,无枝。

【注释】①仞:古时八尺为一仞。

【译文】三棵没有枝干的桑树,在欧丝野的东面,这种树虽高达一百仞, 却不生长树枝。

范林方三百里,在三桑东,洲环其下①。

【注释】①洲:水中可居人或物的小块陆地。

【译文】范林方圆三百里,在三棵桑树的东面,它的下面被沙洲环绕着。

务隅之山,帝颛(zhu1n)顼(x&)葬于阳①。九嫔葬于阴②。一曰爰 有熊、罴(p0)、文虎、离朱、(鸱)[鸱(ch9)]久、视肉。

【注释】①颛顼:传说中的上古帝王。②九嫔:指颛顼的九个妃嫔。

【译文】务隅山,帝颛顼埋葬在它的南面,九嫔埋葬在它的北面。另一 种说法认为这里有熊、罴、花斑虎、离朱鸟、鹞鹰、视肉怪兽。

平丘在三桑东。爰有遗玉、青(鸟)[马]、视肉、杨柳、甘柤(zh1)、 甘华①,百果所生②。(有)[在] 两山夹上谷,二大丘居中,名曰平丘。

【注释】①遗玉:据古人说是一种玉石,先由松枝在千年之后化为伏苓,再过千年之后化为 琥珀,又过千年之后化为遗玉。甘柤:传说中的一种树木,枝干都是红色的,花是黄色的,叶子是白 色的,果实是黑色的。甘华:传说中的一种树木,枝干都是红色的,花是黄色的。②百:这里表示很 多的意思,并非实指。

【译文】平丘在三棵桑树的东面。这里有遗玉、青马、视肉怪兽、杨柳 树、甘柤树、甘华树,是各种果树生长的地方。在两座山相夹的一道山谷上, 有两个大丘处于其间,叫做平丘。

北海内有兽,其状如马,名曰騊(t2o)駼(t*)。有兽焉,其名曰駮 (b¥),状如白马,锯牙,食虎豹。有素兽焉,状如马,名曰蛩蛩(qi¥ngqi¥ng)。

有青兽焉,状如虎,名曰罗罗。

【译文】北海内有一种野兽,形状像一般的马,名称是騊駼。又有一种 野兽,名称是駮,形状像白色的马,长着锯齿般的牙,能吃老虎和豹子。又 有一种白色的野兽,形状像马,名称是蛩蛩。还有一种青色的野兽,形状像 老虎,名称是罗罗。

北方禺彊(ji1ng)①,人面鸟身,珥(7r)两青蛇②,践两青蛇。

【注释】①禺彊:也叫玄冥,神话传说中的水神。②珥:插。这里指穿挂着。

【译文】北方的禺疆神,长着人的面孔而鸟的身子,耳朵上穿挂着两条 青蛇,脚底下践踏着两条青蛇。

\chapter{海外东经}

山海经卷九 海外东经

小/说.t/xt.天+
海外自东南陬(z#u)至东北陬者。

【译文】海外从东南角到东北角的国家地区、山丘河川分别如下。

(ji5)丘,爰有遗玉、青马、视肉、杨柳、甘柤(zh1)、甘华。甘 果所生,在东海。两山夹丘,上有树木。一曰嗟丘。一曰百果所在,在尧葬 东。

【译文】 丘,这里有遗玉、青马、视肉怪兽、杨柳树、甘柤树、甘华 树。结出甜美果子的树所生长的地方,就在东海边。两座山夹着 丘,上面 有树木。另一种说法认为 丘就是嗟丘。还有一种说法认为各种果树所存在 的地方,在葬埋帝尧之地的东面。

大人国在其北,为人大,坐而削(sh4o)船①。一曰在 (ji5)丘北。

【注释】①削船:削、梢二字同音假借。梢是长竿子,这里作动 词用。梢船就是用长竿子 撑船。

【译文】大人国在它的北面,那里的人身材高大,正坐在船上撑船。一 种说法认为大人国在 丘的北面。

奢比之尸在其北①,兽身、人面、大耳,珥(7r)两青蛇。一曰肝榆之 尸在大人北。

【注释】①奢比:也叫奢龙,传说中的神。

【译文】奢比尸神在大人国的北面,那里的人都长着野兽的身子、人的 面孔、大大的耳朵,耳朵上穿挂着两条青蛇。另一种说法认为肝榆尸神在大 人国的北面。

君子国在其北,衣冠带剑①,食兽,使二(大)[文]虎在旁,其人好(h4o) 让不争。有薰华草,朝生夕死。一曰在肝榆之尸北。

【注释】①衣冠:这里都作动词用,即穿上衣服戴上帽子。

【译文】君子国在奢比尸神的北面,那里的人穿衣戴帽而腰间佩带着 剑,能吃野兽,使唤的两只花斑老虎就在身旁,为人喜欢谦让而不争斗。那 里有一种薰华草,早晨开花傍晚凋谢。另一种说法认为君子国在肝榆尸神的 北面。

■■(h¥ng h¥ng)在其北①,各有两首。一曰在君子国北。

【注释】①■■:就是虹霓,俗称美人虹。据古人说,虹双出而颜色鲜艳的为雄,称作虹; 颜色暗淡的为雌,称作霓。

【译文】■■在它的北面,它的各端都有两个脑袋。另一种说法认为■ ■在君子国的北面。

朝(zh1o)阳之谷,神曰天吴,是为水伯。在■■(h¥ng h¥ng)北两 水间。其为兽也,八首人面,八足八尾,(皆)[背]青黄。

【译文】朝阳谷,有一个神人叫做天吴,就是所谓的水伯。他住在■■ 北面的两条水流中间。他是野兽形状,长着八个脑袋而是人的脸面,八只爪 子八条尾巴,背部是青中带黄的颜色。

青丘国在其北。其狐四足九尾。一曰在朝阳北。

【译文】青丘国在它的北面。那里有一种狐狸长着四只爪子九条尾巴。

另一种说法认为青丘国在朝阳谷的北面。

帝命竖亥(h4i)步①,自东极至于西极,五亿十选九千八百步②。竖 亥右手把算③,左手指青丘北。一曰禹令竖亥。一曰五亿十万九千八百步。

【注释】①竖亥:传说中一个走得很快的神人。②选:万。③算:通“筭”。古代人计数用 的筹码。

【译文】天帝命令竖亥用脚步测量大地,从最东端走到最西端,是五亿 十选九千八百步。竖亥右手拿着算筹,左手指着青丘国的北面。另一种说法 认为是大禹命令竖亥测量大地。还一种说法认为测量出五亿十万九千八百 步。

黑齿国在其北,为人黑[齿],食稻啖(d4n)蛇①,一赤一青,在其旁。

一曰在竖亥北,为人黑首,食稻使蛇,其一蛇赤②。

【注释】①啖:吃。②这段文字所述都是原画面上的图像。

【译文】黑齿国在它的北面,那里的人牙齿漆黑,吃着稻米又吃着蛇, 还有一条红蛇和一条青蛇,正围在他身旁。另一种说法认为黑齿国在竖亥所 在地的北面,那里的人是黑脑袋,吃着稻米驱使着蛇,其中一条蛇是红色的。

下有汤(y2ng)谷①。汤谷上有扶桑,十日所浴,在黑齿北。居水中, 有大木,九日居下枝,一日居上枝。

【注释】①下有:“下有”是针对“上有”而言,原图上自然画着上面有什么,但图画已不 存,而说明文字又未记述,故今不知何所指。汤谷:据古人解说,这条谷中的水很热。

【译文】下面有汤谷。汤谷边上有一棵扶桑树,是十个太阳洗澡的地方, 在黑齿国的北面。正当大水中间,有一棵高大的树木,九个太阳停在树的下 枝,一个太阳停在树的上枝。

雨师妾在其北。其为人黑,两手各操一蛇,左耳有青蛇,右耳有赤蛇。

一曰在十日北,为人黑身人面,各操一龟。

【译文】雨师妾国在汤谷的北面。那里的人全身黑色,两只手各握着一 条蛇,左边耳朵上挂有青色蛇,右边耳朵挂有红色蛇。另一种说法认为雨师 妾国在十个太阳所在地的北面,那里的人是黑色身子而人的面孔,两只手各 握着一只龟。

玄股之国在其北。其为人衣鱼食■(#u)①,使两鸟夹之。一曰在雨师 妾北。

【注释】①衣鱼:穿着用鱼皮做的衣服。食■:■也作“鸥”,即鸥鸟,在海边活动的叫海 鸥,在江边活动的叫江鸥。食■即食鸥,就是吃鸥鸟产下的蛋。

【译文】玄股国在它的北面。那里的人穿着鱼皮衣而吃鸥鸟蛋,使唤的 两只鸟在身边。另一种说法认为玄股国在雨师妾国的北面。

毛民之国在其北。为人身生毛。一曰在玄股北。

【译文】毛民国在它的北面。那里的人全身长满了毛。另一种说法认为 毛民国在玄股国的北面。

劳民国在其北,其为人黑。或曰教民。一曰在毛民北,为人面目手足尽 黑。

【译文】劳民国在它的北面,那里的人全身黑色。有的人称劳民国为教 民国。另一种说法认为劳民国在毛民国的北面,那里的人脸面眼睛手脚全是 黑的。

东方句(g#u)芒①,鸟身人面,乘两龙。

【注释】①苟芒:神话传说中的木神。

【译文】东方的句芒神,是鸟的身子人的面孔,乘着两条龙。

建平元年四月丙戊,待诏太常属臣望校治,侍中光禄勋臣龚、侍中奉车 都尉光禄大夫臣秀领主省①。

【注释】①这段文字不是《山海经》原文,而是整理者对本卷文字作完校勘工作后的署名。

建平是西汉哀帝的年号,而建平元年相当于公元前六年。秀即刘秀,原来叫刘歆,后来改名为秀,西 汉末年人,是著名的经学家、目录学家。他曾继承其父刘向的事业,领导主持整理古籍、编撰目录的 工作,成就很大。

【译文】建平元年四月丙戊日,待诏太常属臣丁望校对整理,侍中光禄 勋臣王龚、侍中奉车都尉光禄大夫臣刘秀领衔主持。

\chapter{海内南经}

山海经卷十 海内南经

大$学$生@小`说"网
海内东南陬(z#u)以西者。

【译文】海内由东南角向西的国家地区、山丘河川依次如下。

瓯居海中。闽在海中,其西北有山。一曰闽中山在海中。

【译文】瓯在海中。闽在海中,它的西北方有座山。另一种说法认为闽 地的山在海中。

三天子鄣(zh1ng)山在闽西(海)北。一曰在海中。

【译文】三天子鄣山在闽的西北方。另一种说法认为三天子鄣山在海 中。

桂林八树,在番(p1n)隅东。

【译文】桂林的八棵树很大而形成树林,处在番隅的东面。

伯虑国、离耳国、雕题国、北朐(q*)国皆在郁水南。郁水出湘陵南(海) [山]。一曰(相)[柏]虑。

【译文】伯虑国、离耳国、雕题国、北朐国都在郁水的南岸。郁水发源 于湘陵南山。另一种说法认为伯虑国叫做柏虑国。

枭(xi1o)阳国在北朐(q*)之西。其为人人面长唇,黑身有毛,反踵, 见人(笑亦)[则]笑;左手操管。

【译文】枭阳国在北朐国的西面。那里的人是人的面孔而长长的嘴唇, 黑黑的身子有长毛,脚跟在前而脚尖在后,一看见人就张口大笑;左手握着 一根竹筒。

兕(s@)在舜葬东,湘水南。其状如牛,苍黑,一角。

【译文】兕在帝舜葬地的东面,在湘水的南岸。兕的形状像一般的牛, 通身是青黑色,长着一只角。

苍梧之山,帝舜葬于阳,帝丹朱葬于阴。

【译文】苍梧山,帝舜葬在这座山的南面,帝丹朱葬在这座山的北面。

氾林方三百里①,在氾狌狌(x9ngx9ng)东。

【注释】①氾林:就是前文所说的范林。

【译文】氾林方圆三百里,在猩猩生活之地的东面。

狌狌知人名,其为兽如豕(sh!)而人面,在舜葬西。

【译文】猩猩能知道人的姓名,这种野兽的形状像一般的猪却长着人的 面孔,生活在帝舜葬地的西面。

狌狌西北有犀(x9)牛,其状如牛而黑。

【译文】猩猩的西北面有犀牛,它的形状像一般的牛而全身是黑色。

夏后启之臣曰孟涂,是司神于巴。[巴]人(请)讼(s^ng)于孟涂之所, 其衣有血者乃执之。是请生。居山上,在丹山西。丹山在丹阳南,丹阳(居) [巴] 属也。

【译文】夏朝国王启的臣子叫孟涂,是主管巴地诉讼的神。巴地的人到 孟涂那里去告状,而告状人中有谁的衣服沾上血迹的就被孟涂拘禁起来。这 样就不出现冤枉而有好生之德。孟涂住在一座山上,这座山在丹山的西面。

丹山在丹阳的南面,而丹阳是巴的属地。

窫(zh2)窳(y*)龙首,居弱水中,在狌狌知人名之西,其状如[貙] (ch&)①,龙首,食人。

【注释】①貙:一种像野猫而体型略大的野兽。

【译文】窫窳长着龙一样的头,住在弱水中,处在能知道人姓名的猩猩 的西面,它的形状像貙,长着龙头,能吃人的。

有木,其状如牛,引之有皮,若缨、黄蛇。其叶如罗①,其实如栾②, 其木若苉(#u)③,其名曰建木。在窫(zh2)窳(y*)西弱水上。

【注释】①罗:捕鸟的网。②栾:传说中的一种树木,树根是黄色的,树枝是红色的,树叶 是青色的。③苉:即刺榆树。

【译文】有一种树木,形状像牛,一拉就剥落下树皮,样子像冠帽上缨 带、黄色蛇皮。它的叶子像罗网,果实像栾树结的果实,树干像刺榆,名称 是建木。这种建木生长在窫窳所在地之西的弱水边上。

氐(d!)人国在建木西,其为人人面而鱼身,无足。

【译文】氐人国在建木所在地的西面,那里的人都长着人的面孔却是鱼 的身子,没有脚。

巴蛇食象,三岁而出其骨,君子服之,无心腹之疾。其为蛇青、黄、赤、 黑,一曰黑蛇青首,在犀牛西。

【译文】巴蛇能吞下大象,吞吃后三年才吐出大象的骨头,有才能德品 的人吃了巴蛇的肉,就不患心痛或肚子痛之类的病。这种巴蛇的颜色是青 色、黄色、红色、黑色混合间杂的。另一种说法认为巴蛇是黑色身子青色脑 袋,在犀牛所在地的西面。

旄(m2o)马,其状如马,四节有毛。在巴蛇西北,高山南。

【译文】旄马,形状像普通的马,但四条腿的关节上都有长毛。旄马在 巴蛇所在地的西北面,一座高山的南面。

匈奴、开题之国、列人之国并在西北。

【译文】匈奴国、开题国、列人国都在西北方。

\chapter{海内西经}

山海经卷十一 海内西经

小,说,t,xt,天,堂
海内西南陬(z#u)以北者。

【译文】海内由西南角向北的国家地区、山丘河川依次如下。

贰负之臣曰危①,危与贰负杀窫(zh2)窳(y*)②。帝乃梏(g))之 疏属之山③,桎(zh@)其右足④,反缚两手与发,系之山上木。在开题西 北。

【注释】①贰负:神话传说中的天神,样子是人的脸面蛇的身子。②窫窳:也是传说中的天 神,原来的样子是人的脸面蛇的身子,后被贰负及其臣子杀死而化成上文所说的样子——龙头,野猫 身,并且吃人。③梏:古代木制的手铐。这里是械系、拘禁的意思。④桎:古代拘系罪人两脚的刑具。

【译文】贰负神的臣子叫危,危与贰负合伙杀死了窫窳神。天帝便把贰 负拘禁在疏属山中,并给他的右脚戴上刑具,还用他自己的头发反绑上他的 双手,拴在山上的大树下。这个地方在开题国的西北面。

大泽方百里,群鸟所生及所解。在雁门北。

【译文】大泽方圆一百里,是各种禽鸟生卵孵化幼鸟和脱换羽毛的地 方。大泽在雁门的北面。

雁门山,雁出其间。在高柳北。

【译文】雁门山,是大雁冬去春来出入的地方。雁门山在高柳山的北面。

高柳在代北。

【译文】高柳山在代地的北面。

后稷(j9)之葬,山水环之。在氐(d!)国西①。

【注释】①氐国:就是上文所说的氐人国。

【译文】后稷的葬地,有青山绿水环绕着它。后稷葬地在氐人国的西面。

流黄酆(f5ng)氏之国,中方三百里①,有涂四方②,中有山。在后稷 葬西。

【注释】①中:域中,即国内土地的意思。②涂:通“途”。道路。

【译文】流黄酆氏国,疆域有方圆三百里大小。有道路通向四方,中间 有一座大山。流黄酆氏国在后稷葬地的西面。

流沙出钟山①,西行又南行昆仑之虚(q&)②,西南入海,黑水之山。

【注释】①流沙:沙子和水一起流行移动的一种自然现象。②虚:大丘。即指山。

【译文】流沙的发源地在钟山,向西流动而再朝南流过昆仑山,继续往 西南流入大海,直到黑水山。

东胡在大泽东。

【译文】东胡国在大泽的东面。

夷人在东胡东。

【译文】夷人国在东胡国的东面。

貊(m^)国在汉水东北。地近于燕(y1n),灭之。

【译文】貊国在汉水的东北面。它靠近燕国的边界,后来被燕国灭掉了。

孟鸟在貊国东北。其鸟文赤、黄、青,东乡(xi4nq)①。

【注释】①乡:通“向”  【译文】孟鸟在貊国的东北面。这种鸟的羽毛花纹有红、黄、青三种颜 色,向着东方。

海内昆仑之虚(q&),在西北,帝之下都。昆仑之虚,方八百里,高万 仞①。上有木禾,长五寻②,大五围。面有九井,以玉为槛(ji4n)③。面 有九门,门有开明兽守之,百神之所在④。在八隅之岩,赤水之际,非(仁) [夷]羿(y@)莫能上冈之岩⑤。

【注释】①仞:古代的八尺为一仞。②寻:古代的八尺为一寻。③槛:窗户下或长廊旁的栏 杆。这里指井栏。④百:并非实数,而是言其多。⑤夷羿,即后羿,神话传说中的英雄人物,善于射 箭,曾经射掉九个太阳,射死毒蛇猛兽,为民除害。

【译文】海内的昆仑山,屹立在西北方,是天帝在下方的都城。昆仑山, 方圆八百里,高一万仞。山顶有一棵像大树似的稻谷,高达五寻,粗细需五 人合抱。昆仑山的每一面有九眼井,每眼井都有用玉石制成的围栏。昆仑山 的每一面有九道门,而每道门都有称作开明的神兽守卫着,是众多天神聚集 的地方。众多天神聚集的地方是在八方山岩之间,赤水的岸边,不是具有像 夷羿那样本领的人就不能攀上那些山冈岩石。

赤水出东南隅,以行其东北,[西南流注南海厌火东]。

【译文】赤水从昆仑山的东南角发源,然后流到昆仑山的东北方,又转 向西南流而注到南海厌火国的东边。

河水出东北隅,以行其北,西南又入渤海,又出海外,即西而北,入禹 所导积石山。

【译文】黄河水从昆仑山的东北角发源,然后流到昆仑山的北面,再折 向西南流入渤海,又流出海外,就此向西而后往北流,一直流入大禹所疏导 过的积石山。

洋(xi2ng)水、黑水出西北隅,以东,东行,又东北,南入海,羽民 南。

【译文】洋水、黑水从昆仑山的西北角发源,然后折向东方,朝东流去, 再折向东北方,又朝南流入大海,直到羽民国的南面。

弱水、青水出西南隅,以东,又北,又西南,过毕方鸟东。

【译文】弱水、青水从昆仑山的西南角发源,然后折向东方,朝北流去, 再折向西南方,又流经毕方鸟所在地的东面。

昆仑南渊深三百仞。开明兽身大类虎而九首,皆人面,东向立昆仑上。

【译文】昆仑山的南面有一个深三百仞的渊潭。开明神兽的身子大小像 老虎却长着九个脑袋,九个脑袋都是人一样的面孔,朝东立在昆仑山顶。

开明西有凤皇、鸾(lu2n)鸟,皆戴蛇践蛇,膺(y9ng)有赤蛇。

【译文】开明神兽的西面有凤皇、鸾鸟栖息,都各自缠绕着蛇踩踏着蛇, 胸前还有红色的蛇。

开明北有视肉、珠树、文玉树、玕(y&)琪(q0)树、不死树①,凤皇、 鸾鸟皆戴瞂(f2)②,又有离朱、木禾、柏树、甘水、圣木曼兑(du@)③。

一曰挺木牙交。

【注释】①珠树:神话传说中的生长珍珠的树。文玉树:神话传说中的生长五彩美玉的树。

玕琪树:神话传说中的生长红色玉石的树。不死树:神话传说中的一种长生不死的树,人服食了它可 也长寿不老。②瞂:盾。③离朱:即太阳里的踆乌,也叫三足乌。甘水:即古人所谓的醴泉,甜美的 泉水。圣木曼兑:一种叫做曼兑的圣树,服食了它可使人圣明智慧。

【译文】开明神兽的北面有视肉怪兽、珠树、文玉树、玕琪树、不死树, 那里的凤皇、鸾鸟都戴着盾牌,还有三足乌、像树似的稻谷、柏树、甘水、 圣木曼兑。另一种说法认为圣木曼兑叫做挺木牙交。

开明东有巫彭、巫抵、巫阳、巫履、巫凡、巫相,夹窫(zh2)窳(y*) 之尸,皆操不死之药以距之①。窫窳者,蛇身人面,贰负臣所杀也。

【注释】①距:通“拒”。抗拒。

【译文】开明神兽的东面有巫师神医巫彭、巫抵、巫阳、巫履、巫凡、 巫相,他们围在窫窳的尸体周围,都手捧不死药来抵抗死气而要使他复活。

这位窫窳,是蛇的身子人的面孔,被贰负和他的臣子危合伙杀死的。

服常树,其上有三头人,伺琅(12ng)玕(g1n)树①。

【注释】①琅玕树:传说这种树上结出的果实就是珠玉。

【译文】有一种服常树,它上面有个长着三颗头的人,静静伺察着那就 在附近的琅玕树。

开明南有树鸟,六首;蛟、蝮(f))、蛇、蜼(w7i)、豹、鸟秩树①, 于表池树木②;诵鸟、鶽(s(n)、视肉③。

【注释】①蛟:像蛇的样子,但有四只脚,属于龙一类。蝮:大蛇。鸟秩树:不详何种树木。

②树:这里是动词,环绕着、排列着的意思。③诵鸟:不详何种禽鸟。鶽:雕鹰。

【译文】开明神兽的南面有种树鸟,长着六个脑袋;那里还有蛟龙、蝮 蛇、长尾猿、豹子、鸟秩树,在水池四周环绕着树木而显得华美;那里还有 诵鸟、鶽鸟、视肉怪兽。

\chapter{海内北经}

山海经卷十二 海内北经

小说t-xt天堂    
海内西北陬(z#u)以东者。

【译文】海内由西北角向东的国家地区、山丘河川依次如下。

蛇巫之山,上有人操柸(b4ng)而东向立①。一曰龟山。

【注释】①柸:即“棓”,音义同而字形异。棓,同“棒”。棍子,大棒。

【译文】蛇巫山,上面有人拿着一根棍棒向东站着。另一种说法认为蛇 巫山叫做龟山。

西王母梯几而戴胜(杖)①。其南有三青鸟,为西王母取食。在昆仑虚 (q&)北。

【注释】①梯:凭倚,凭靠。几:矮或小的桌子。胜:古时妇女的首饰。

【译文】西王母靠倚着小桌案而头戴玉胜。在西王母的南面有三只勇猛 善飞的青鸟,正在为西王母觅取食物。西王母和三青鸟的所在地是在昆仑山 的北面。

有人曰大行伯,把戈。其东有犬封国。贰负之尸在大行伯东。

【译文】有个神人叫大行伯,手握一把长戈。在他的东面有犬封国。贰 负之尸也在大行伯的东面。

犬封国曰犬戎国,状如犬。有一女子,方跪进(柸)[杯]食①。有文马 ②,缟(g3o)身朱鬣(li6)③,目若黄金,名曰吉量,乘之寿千岁。

【注释】①方:正在。原图上就是这样画的,所以用这类词语加以说明。以下此类情况尚多。

②文马:皮毛带有色彩花纹的马。③缟:白色。

【译文】犬封国也叫犬戎国,那里的人都是狗的模样。犬封国有一女子, 正跪在地上捧着一杯酒食向人进献。那里还有文马,是白色身子红色鬃毛, 眼睛像黄金一样闪闪发光,名称是吉量,骑上它就能使人长寿千岁。

鬼国在贰负之尸北,为物人面而一目。一曰贰负神在其东,为物人面蛇 身。

【译文】鬼国在贰负之尸的北面,那里的人物是人的面孔却长着一只眼 睛。另一种说法认为贰负神在鬼国的东面,他是人的面孔而蛇的身子。

蜪(t2o)犬如犬,青,食人从首始。

【译文】蜪犬的形状像一般的狗,全身是青色,它吃人是从人的头开始 吃起。

穷奇状如虎,有翼,食人从首始。所食被(p9)发①。在蜪犬北。一曰 从足。

【注释】①被发:即披发。被,通“披”。这是原图画上的样子。

【译文】穷奇的形状像一般的老虎,却生有翅膀,穷奇吃人是从人的头 开始吃。正被吃的人是披散着头发的。穷奇在蜪犬的北面。另一种说法认为 穷奇吃人是从人的脚开始吃起。

帝尧台、帝喾台、帝丹朱台、帝舜台,各二台,台四方,在昆仑东北。

【译文】帝尧台、帝喾台、帝丹朱台、帝舜台,各自有两座台,每座台 都是四方形,在昆仑山的东北面。

大蜂,其状如螽(zh#ng)①;朱蛾,其状如蛾②。

【注释】①螽:螽斯,一种昆虫,体呈绿色或褐色,样子像蚂蚱。②蛾:古人说是蚍蜉,就 是现在所说的蚂蚁。

【译文】有一种大蜂,形状像螽斯;有一种朱蛾,形状像蚍蜉。

蟜(qi2o),其为人虎文,胫(j@ng)有■(q!)①。在穷奇东。一曰 状如人,昆仑虚(q&)北所有。

【注释】①■:小腿肚子。

【译文】蟜,长着人的身子却有着老虎一样的斑纹,腿上有强健的小腿 肚子。蟜在穷奇的东面。另一种说法认为蟜的形状像人,是昆仑山北面所独 有的。

阘(t4)非,人面而兽身,青色。

【译文】阘非,长着人的面孔却是兽的身子,全身是青色。据比之尸, 其为人折颈被(p9)发,无一手。

【译文】天神据比的尸首,形象是折断了脖子而披散着头发,没了一只 手。

环狗,其为人兽首人身。一曰猬状如狗,黄色。

【译文】环狗,这种人是野兽的脑袋人的身子。另一种说法认为是刺猬 的样子而又像狗,全身是黄色。

祙(m6i)①,其为物人身、黑首、从(z^ng)目②。

【注释】①祙:即魅,古人认为物老则成魅。就是现在所说的鬼魅、精怪。②从:通“纵”。

【译文】袜,这种怪物长着人的身子、黑色脑袋、竖立的眼睛。

戎,其为人人首三角。

【译文】戎,这种人长着人的头而头上却有三只角。

林氏国有珍兽,大若虎,五采毕具,尾长于身,名曰驺(ch*)吾,乘 之日行千里。

【译文】林氏国有一种珍奇的野兽,大小与老虎差不多,身上有五种颜 色的斑纹,尾巴比身子长,名称是驺吾,骑上它可以日行千里。

昆仑虚(q&)南所,有氾林方三百里①。

【注释】①氾林:即上文所说的范林、泛林,意为树木茂密丛生的树林。

【译文】昆仑山南面的地方,有一片方圆三百里的氾林。

从(zh#ng)极之渊,深三百仞①,维冰夷恒都焉②。冰夷人面,乘两 龙。一曰忠极之渊。

【注释】①仞:古代的八尺为一仞。②维:通“惟”、“唯”。独,只有。冰夷:也叫冯(p0ng) 夷、无夷,即河伯,传说中的水神。

【译文】从极渊有三百仞深,只有冰夷神常常住在这里。冰夷神长着人 的面孔,乘着两条龙。另一种说法认为从极渊叫做忠极渊。

阳汙(y&)之山,河出其中;凌门之山,河出其中。

【译文】阳汙山,黄河的一条支流从这座山发源;凌门山,黄河的另一 条支流从这座山发源。

王子夜之尸,两手、两股、胸、首、齿,皆断异处。

【译文】王子夜的尸体,两只手、两条腿、胸脯、脑袋、牙齿,都斩断 而分散在不同地方。

舜妻登比氏生宵明、烛光,处河大泽,二女之灵能照此所方百里。一曰 登北氏。

【译文】帝舜的妻子登比氏生了宵明、烛光两个女儿,她们住在黄河边 上的大泽中,两位神女的灵光能照亮这里方圆百里的地方。另一种说法认为 帝舜的妻子叫登北氏。

盖国在鉅燕南①,倭(w#)北。倭属燕。

【注释】①鉅:通“巨”。大。这里是形容词。

【译文】盖国在大燕国的南面,倭国的北面。倭国隶属于燕国。

朝鲜在列阳东,海北山南。列阳属燕。

【译文】朝鲜在列阳的东面,北面有大海而南面有高山。列阳隶属于燕 国。

列姑射(y6)在海河州中①。

【注释】①河州:据古人说是黄河流入海中形成的小块陆地。州是水中高出水面的土地。

【译文】列姑射在大海的河州上。

(射)姑[射](y6)国在海中,属列姑射。西南,山环之。

【译文】姑射国在海中,隶属于列姑射。射姑国的西南部,高山环绕着 它。

大蟹在海中①。

【注释】①大蟹:据古人说是一种方圆千里大小的蟹。

【译文】大蟹生活在海里。

陵鱼人面①,手足,鱼身,在海中。

【注释】①陵鱼:即上文所说的人鱼、鲵鱼,俗称娃娃鱼。

【译文】陵鱼长着人的面孔,而且有手有脚,却是鱼的身子,生活在海 里。

大鯾(bi1n)居海中①。

【注释】①鯾:同“鳊”。即鲂鱼,体型侧扁,背部特别隆起,略呈菱形,像现在所说的武 昌鱼,肉味鲜美。

【译文】大鯾鱼生活在海里。

明组邑居海中①。

【注释】①明组邑:可能是生活在海岛上的一个部落。邑即邑落,指人所聚居的部落、村落。

【译文】明组邑生活在海岛上。

蓬莱山在海中①。

【注释】①蓬莱山:传说中的仙山,上面有神仙居住的宫室,都是用黄金玉石建造成的,飞 鸟走兽纯白色,远望如白云一般。

【译文】蓬莱山屹立在海中。

大人之市在海中。

【译文】大人贸易的集市在海里。

\chapter{海内东经}

山海经卷十三 海内东经

大;学,生,小,说'网
海内东北陬(z#u)以南者。

【译文】海内由东北角向南的国家地区、山丘河川依次如下。

鉅燕在东北陬(z#u)。

【译文】大燕国在海内的东北角。

国在流沙中者埻(d&n)端、玺(x!)■(hu4n),在昆仑虚(q&)东 南。一曰海内之郡,不为郡县,在流沙中。

【译文】在流沙中的国家有埻端国、玺■国,都在昆仑山的东南面。另 一种说法认为埻端国和玺■国是在海内建置的郡,不把它们称为郡县,是因 为处在流沙中的缘故。

国在流沙外者,大夏、竖沙、居繇(y2o)、月支之国。

【译文】在流沙以外的国家,有大夏国、竖沙国、居繇国、月支国。

西胡白玉山在大夏东,苍梧在白玉山西南,皆在流沙西,昆仑虚(q&) 东南。昆仑山在西胡西。皆在西北。

【译文】西方胡人的白玉山国在大夏国的东面,苍梧国在白玉山国的西 南面,都在流沙的西面,昆仑山的东南面。昆仑山位于西方胡人所在地的西 面。总的位置都在西北方。

雷泽中有雷神,龙身而人头,鼓其腹①。在吴西。

【注释】①鼓:这里是动词,即鼓动,振作。据传这位雷神只要鼓动他的肚子就会响起雷声。

【译文】雷泽中有一位雷神,长着龙的身子人的头,他一鼓起肚子就响 雷。雷泽在吴地的西面。

都州在海中。一曰郁州。

【译文】都州在海里。一种说法认为都州叫做郁州。

琅(l2ng)邪(y2)台在渤海间①,琅邪之东②。其北有山。一曰在海 间。

【注释】①琅邪台:据古人讲,琅邪台本来是一座山,高耸突起,形状如同高台,所以被称 为琅邪台。②琅邪:指春秋时越王勾践修筑的琅邪台,周长七里,用来观望东海。

【译文】琅邪山位于渤海与海岸之间,在琅邪台的东面。琅邪台的北面 有座山。另一种说法认为琅邪山在海中。

韩雁在海中①,都州南。

【注释】①韩雁:难以断定是国名,或鸟名。如果是国名,则应在海中的岛屿上。

【译文】韩雁在海中,又在都州的南面。

始鸠在海中①,(辕厉)[韩雁]南。

【注释】①始鸠:也难以断定是国名,或鸟名。

【译文】始鸠在海中,又在韩雁的南面。

会(gu@)稽(j@)山在大(楚)[越]南。

【译文】会稽山在大越的南面。

岷三江,首大江出汶山,北江出曼山,南江出高山。高山在(城)[成] 都西。入海,在长州南。浙江出三天子都,在其[蛮]东,在闽西北,入海, 馀暨南。庐江出三天子都,入江,彭泽西。一曰天子鄣。淮水出馀山,馀山 在朝阳东,义乡西。入海,淮浦北。湘水出舜葬东南陬,西环之。入洞庭下。

一曰东南西泽。汉水出鲋鱼之山,帝颛顼葬于阳,九嫔葬于阴,四蛇卫之。

濛水出汉阳西,入江,聂阳西。温水出崆峒,[峒崆]山在临汾南,入河,华 阳北。颍水出少室,少室山在雍氏南,入淮西鄢北。一曰缑氏。汝水出天息 山,在梁勉乡西南,入淮极西北。一曰淮在期思北。泾水出长城北山,山在 郁郅长垣北,(北)入渭,戏北。渭水出鸟鼠同穴山,东注河,入华阴北。

白水出蜀,而东南注江,入江州城下。沅水(山)出象郡镡城西,(入)东 注江,入下隽西,合洞庭中。赣水出聂都东山,东北注江,入彭泽西。泗水 出鲁东北,而南,西南过湖陵西,而东南注东海,入淮阴北。郁水出象郡, 而西南注南海,入须陵东南。肄水出临晋[武]西南,而东南注海,入番禺西。

潢水出桂阳西北山,东南注肄水,入敦浦西。洛水出[上]洛西山,东北注河, 入成皋(之)西。汾水出上窳北,而西南注河,入皮氏南。沁水出井陉山东, 东南注河,入怀东南。济水出共山南东丘,绝鉅鹿泽,注渤海,入齐琅槐东 北。潦水出卫皋东,东南注渤海,入潦阳。虖沱水出晋阳城南,而西至阳曲 北,而东注渤海,入(越)章武北。漳水出山阳东,东注渤海,入章武南①。

【注释】①从“岷三江”至“入章武南”这一大段文字,据学者的研究,认为不是《山海经》 原文,而是《水经》一书中的文字。但因这段文字为底本所原有,故仍保留它并作今译,唯不做注。

【译文】从岷山中流出三条江水,首先是长江从汶山流出,再者北江从 曼山流出,还有南江从高山流出。高山座落在成都的西面。三条江水最终注 入大海,入海处在长州的南面。浙江从三天子都山发源,三天子都山在蛮地 的东面,闽地的西北面,浙江最终注入大海,入海处在馀暨的南边。庐江也 从三天子都山发源,却注入长江,入江处在彭泽的西面。一种说法认为在天 子鄣。淮水从馀山发源,馀山座落在朝阳的东面,义乡的西面。淮水最终注 入大海,入海处在淮浦的北面。湘水从帝舜葬地的东南角发源,然后向西环 绕流去。湘水最终注入洞庭湖下游。一种说法认为注入东南方的西泽。汉水 从鲋鱼山发源,帝颛顼葬在鲋鱼山的南面,帝颛顼的九个嫔妃葬在鲋鱼山的 北面,有四条巨蛇卫护着它。濛水从汉阳西面发源,最终注入长江,入江处 在聂阳的西面。温水从崆峒山发源,崆峒山座落在临汾南面,温水最终注入 黄河,入河处在华阳的北面。颍水从少室山发源,少室山座落在雍氏的南面, 颍水最终在西鄢的北边注入淮水。一种说法认为在缑氏注入淮水。汝水从天 息山发源,天息山座落在梁勉乡的西南,汝水最终在淮极的西北注入淮水。

一种说法认为入淮处在期思的北面。泾水从长城的北山发源,北山座落在郁 郅长垣的北面,泾水最后流入渭水,入渭处在戏的北面。渭水从鸟鼠同穴山 发源,向东流入黄河,入河处在华阴的北面。白水从蜀地流出,然后向东南 流而注入长江,入江处在江州城下。沅水从象郡镡城的西面发源,向东流而 注入长江,入江处在下隽的西面,最后汇入洞庭湖中。赣水从聂都东面的山 中发源,向东北流而注入长江,入江处在彭泽的西面。泗水从鲁地的东北方 流出,然后向南流,再往西南流经湖陵的西面,然后转向东南而流入东海, 入海处在淮阴的北面。郁水从象郡发源,然后向西南流而注入南海,入海处 在须陵的东南面。肄水从临晋武的西南方流出,然后向东南流而注入大海, 入海处在番禺的西面。潢水从桂阳西北的山中发源,向东南流而注入肄水, 入肄处在敦浦的西面。洛水从上洛西边的山中发源,向东北流而注入黄河, 入河处在成皋的西边。汾水从上窳的北面流出,然后向西南流而注入黄河, 入河处在皮氏的南面。沁水从井陉山的东面发源,向东南流而注入黄河,入 河处在怀的东南面。济水从共山南面的东丘发源,流过鉅鹿泽,最终注入渤 海,入海处在齐地琅槐的东北面。潦水从卫皋的东面流出,向东南流而注入 渤海,入海处在潦阳。虖沱水从晋阳城南发源,然后向西流到阳曲的北面, 再向东流而注入渤海,入海处在章武的北面。漳水从山阳的东面流出,向东 流而注入渤海,入海处在章武的南面。

建平元年四月丙戌,待诏太常属臣望校治,侍中光禄勋臣龚、侍中奉车 都尉光禄大夫臣秀领主省。

【译文】建平元年四月丙戌日,待诏太常属臣丁望校对整理,侍中光禄 勋臣王龚、侍中奉车都尉光禄大夫刘秀领衔主持。

\chapter{大荒东经}

山海经卷十四 大荒东经

小_说  txt 天+堂
东海之外[有]大壑(h6)①,少昊(h4o)之国②。少昊孺(r))帝颛 (zhu1n)顼(x&)于此③,弃其琴瑟(s6)④。

【注释】①壑:坑谷,深沟。②少昊:传说中的上古帝王,名叫挚,以金德王,所以号称金 天氏。③孺:通“乳”。用乳奶喂养。这里是抚育、养育的意思。颛顼:传说中的上古帝王,号称高 阳氏,是黄帝的后代。④琴瑟:古时两种拨弦乐器。

【译文】东海以外有一深得不知底的沟壑,是少昊建国的地方。少昊就 在这里抚养帝颛顼成长,帝颛顼幼年玩耍过的琴瑟还丢在沟壑里。

有甘山者,甘水出焉,生甘渊①。

【注释】①渊:水流汇积就成为深渊。

【译文】有一座甘山,甘水从这座山发源,然后流汇成甘渊。

大荒东南隅有山,名皮母地丘。

【译文】大荒的东南角有座高山,名称是皮母地丘。

东海之外,大荒之中,有山名曰大言,日月所出。

【译文】东海以外,大荒当中,有座山叫做大言山,是太阳和月亮初出 升起的地方。

有波谷山者,有大人之国,有大人之市,名曰大人之堂①。有一大人踆 (c&n)其上②,张其两(耳)[臂]。

【注释】①大人之堂:本是一座山,因为山的形状就像是一座堂屋,所以称作大人堂。②踆: 通“蹲”。

【译文】有座波谷山,有个大人国就在这山里。有大人做买卖的集市, 就在叫做大人堂的山上。有一个大人正蹲在上面,张开他的两只手臂。

有小人国,名靖人①。

【注释】①:靖人:传说东北极有一种人,身高只有九寸,这就是靖人。靖的意思是细小的 样子。靖人即指小人。

【译文】有个小人国,那里的人被称作靖人。

有神,人面兽身,名曰梨■(l0ng)之尸。

【译文】有一个神人,长着人的面孔野兽的身子,叫做梨■尸。

有潏(ju6)山,杨水出焉。

【译文】有座潏山,杨水就是从这座山发源的。

有■(w7i)国,黍(sh()食①,使四鸟②:虎、豹、熊、罴(p0)。

【注释】①黍:一种黏性谷米,可供食用和酿酒,古时主要在北方种植,脱去糠皮就称作黄 米子。②鸟:古时鸟兽通名,这里即指野兽。以下同此。

【译文】有一个■国,那里的人以黄米为食物,能驯化驱使四种野兽: 老虎、豹子、熊、罴。

大荒之中,有山名曰合虚,日月所出。

【译文】在大荒当中,有座山叫做合虚山,是太阳和月亮初出升起的地 方。

有中容之国。帝俊生中容①,中容人食兽、木实,使四鸟:豹、虎、熊、 罴(p0)。

【注释】①帝俊:本书屡屡出现叫帝俊的上古帝王,具体所指,各有不同,而神话传说,分 歧已大,历时既久,更相矛盾,实难确指,只可疑似而已。以下同此。这里似指颛顼。中容:传说颛 顼生有才子八人,其中就有中容。

【译文】有一个国家叫中容国。帝俊生了中容,中容国的人吃野兽的肉、 树木的果实,能驯化驱使四种野兽:豹子、老虎、熊、罴。

有东口之山。有君子之国,其人衣冠带剑。

【译文】有座东口山。有个君子国就在东口山,那里的人穿衣戴帽而且 腰间佩带宝剑。

有司幽之国。帝俊生晏龙,晏龙生司幽,司幽生思土,不妻;思女,不 夫①。食黍(sh(),食兽,是使四鸟。

【注释】①思土不妻,思女不夫:神话传说他们虽然不娶亲,不嫁人,但因精气感应、魂魄 相合而生育孩子,延续后代。

【译文】有个国家叫司幽国。帝俊生了晏龙,晏龙生了司幽,司幽生了 思土,而思土不娶妻子;司幽还生了思女,而思女不嫁丈夫。司幽国的人吃 黄米饭,也吃野兽肉,能驯化驱使四种野兽。

有大阿之山者。

【译文】有一座山叫做大阿山。

大荒中有山,名曰明星,日月所出。

【译文】大荒当中有一座高山,叫做明星山,是太阳和月亮初出升起的 地方。

有白民之国。帝俊生帝鸿①,帝鸿生白民②,白民销姓,黍(sh()食, 使四鸟:虎、豹、熊、罴(p0)。

【注释】①帝俊:似指少典,传说中的上古帝王,娶有蟜氏,生黄帝、炎帝二子。帝鸿:即 黄帝,姓公孙,居轩辕之丘,所以号称轩辕氏。有土德之瑞,所以又号称黄帝。取代神农氏为天子。

②生:在本书中,“生”字的用法,并不一定都指某人诞生某人,也多指某人所生存、遗存的后代子 孙。这里就是指后代而言。以下这种用意尚多。

【译文】有个国家叫白民国。帝俊生了帝鸿,帝鸿的后代是白民,白民 国的人姓销,以黄米为食物,能驯化驱使四种野兽:老虎、豹子、熊、罴。

有青丘之国。有狐,九尾。

【译文】有个国家叫青丘国。青丘国有一种狐狸,长着九条尾巴。

有柔仆民,是维嬴土之国①。

【注释】①维:句中语助词,无意。②嬴土:肥沃的土地。

【译文】有一群人被称作柔仆民,他们所在的国土很肥沃。

有黑齿之国。帝俊生黑齿,姜姓,黍(sh()食,使四鸟。

【译文】有个国家叫黑齿国。帝俊的后代是黑齿,姓姜,那里的人吃黄 米饭,能驯化驱使四种野兽。

有夏州之国。有盖余之国。

【译文】有个国家叫夏州国。在夏州国附近又有一个盖余国。

有神人,八首人面,虎身十尾,名曰天吴。

【译文】有个神人,长着八颗头而都是人的脸面,老虎身子而十条尾巴, 名叫天吴。

大荒之中,有山名曰鞠陵于天、东极、离瞀(m4o),日月所出。[有神] 名曰折丹,东方曰折,来风曰俊,处东极以出入风。

【译文】在大荒当中,有三座高山分别叫做鞠陵于天山、东极山、离瞀 山,都是太阳和月亮初出升起的地方。有个神人名叫折丹,东方人单称他为 折,从东方吹来的风称作俊,他就处在大地的东极主管风起风停。

东海之渚(zh()中①,有神,人面鸟身,珥两黄蛇,践两黄蛇,名曰 禺■。黄帝生禺■,禺■生禺京②。禺京处北海,禺■处东海,是惟海神③。

【注释】①渚:水中的小洲。这里指海岛。②禺京:就是上文所说的风神禺■。同一神人, 一说是风神,一说是海神,大概因神话传说不同,或为一身而兼二职。③惟:句中语助词,无意。

【译文】在东海的岛屿上,有一个神人,长着人的面孔鸟的身子,耳朵 上穿挂着两条黄色的蛇,脚底下踩踏着两条黄色的蛇,名叫禺■。黄帝生了 禺■,禺■生了禺京。禺京住在北海,禺■住在东海,都是海神。

有招摇山,融水出焉。有国曰玄股,黍(sh()食,使四鸟。

【译文】有座招摇山,融水从这座山发源。有一个国家叫玄股国,那里 的人吃黄米饭,能驯化驱使四种野兽。

有(困)[因]民国,勾姓,(而)[黍]食。有人曰王亥,两手操鸟,方 食其头①。王亥托于有易、河伯仆牛②。有易杀王亥③,取仆牛。河[伯]念 有易④,有易潜出,为国于兽,方食之⑤,名曰摇民⑥。帝舜生戏⑦,戏生 摇民。

【注释】①方食其头:这是针对原画面上的图像而说的。②仆:通“朴”。大。③有易杀王 亥:据古史传说,王亥对有易族人奸淫暴虐,有易族人愤恨而杀了他。④河伯念有易:据古史传说, 王亥的继承者率兵为王亥报仇,残杀了许多有易族人,河伯同情有易族人,就帮助残存的有易族人悄 悄逃走。⑤方食之:这也是针对原画面上的图像而说的。⑥摇民:即因民国。⑦帝舜:传说中上古时 的贤明帝王。

【译文】有个国家叫因民国,那里的人姓勾,以黄米为食物。有个人叫 王亥,他用两手抓着一只鸟,正在吃鸟的头。王亥把一群肥牛寄养在有易族 人、水神河伯那里。有易族人把王亥杀死,没收了那群肥牛。河伯哀念有易 族人,便帮助有易族人偷偷地逃出来,在野兽出没的地方建立国家,他们正 在吃野兽肉,这个国家叫摇民国。另一种说法认为帝舜生了戏,戏的后代就 是摇民。

海内有两人①,名曰女丑②。女丑有大蟹③。

【注释】①两人:下面只说了一个,大概文字上有逸脱。②女丑:就是上文所说的女丑之尸, 是一个女巫。③大蟹:就是上文所说的方圆有一千里大小的螃蟹。

【译文】海内有两个神人,其中的一个名叫女丑。女丑有一只听使唤的 大螃蟹。

大荒之中,有山名曰孽(ni6)摇頵(y&n)羝(d9)。上有扶木①,柱 三百里②,其叶如芥(ji6)③。有谷曰温源谷④。汤(y2ng)谷上有扶木, 一日方至,一日方出,皆载于乌⑤。

【注释】①扶木:就是上文所说的扶桑树,太阳由此升起。②柱:像柱子般直立着。③芥: 芥菜,花茎带着叶子,而叶子有叶柄,不包围花茎。④温源谷:就是上文所说的汤谷,谷中水很热, 太阳在此洗澡。⑤乌:就是上文所说的踆(c&n)乌、离朱鸟、三足乌,异名同物,除过所长三只爪 子外,其它形状像乌鸦,栖息在太阳里。

【译文】在大荒当中,有一座山名叫孽摇頵羝。山上有棵扶桑树,高耸 三百里,叶子的形状像芥菜叶。有一道山谷叫做温源谷。汤谷上面也长了棵 扶桑树,一个太阳刚刚回到汤谷,另一个太阳刚刚从扶桑树上出去,都负载 于三足乌的背上。

有神,人面、(犬)[大]耳、兽身,珥两青蛇,名曰奢比尸。

【译文】有一个神人,长着人的面孔、大大的耳朵、野兽的身子,耳朵 上穿挂着两条青色的蛇,名叫奢比尸。

有五采之鸟①,相乡弃沙②。惟帝俊下友③。帝下两坛,采鸟是司。

【注释】①五采之鸟:即五采鸟,属鸾鸟、凤凰之类。采,通“彩”。彩色。②乡:通“向”。

弃沙:不详何意。有些学者认为“弃沙”二字是“媻娑”二字的讹误,而媻娑的意思是盤旋而舞的样 子。

③惟:句首语助词,无意。

【译文】有一群长着五彩羽毛的鸟,相对而舞,天帝帝俊从天上下来和 它们交友。帝俊在下界的两座祭坛,由这群五彩鸟掌管着。

大荒之中,有山名曰猗天苏门,日月所(生)[出]。

【译文】在大荒当中,有一座山名叫猗天苏门山,是太阳和月亮初出升 起的地方。

有壎(xu1n)民之国。有綦(j9)山。又有摇山。有綦(z6ng)山。又 有门户山。又有盛山。又有待山。有五采之鸟。

【译文】有个国家叫壎民国。有座綦山。又有座摇山。又有座■山。又 有座门户山。又有座盛山。又有座待山。还有一群五彩鸟。

东荒之中,有山名曰壑(h6)明俊疾,日月所出。有中容之国。

【译文】在东荒当中,有座山名叫壑明俊疾山,是太阳和月亮初出升起 的地方。这里还有个中容国。

东北海外,又有三青马、三骓(zhu9)①、甘华。爰有遗玉、三青鸟、 三骓、视肉、甘华、甘柤(zh1)。百谷所在②。

【注释】①骓:马的毛色青白间杂。②百谷:泛指各种农作物。百,表示多的意思,不是实 指。

【译文】在东北海外,又有三青马、三骓马、甘华树。这里还有遗玉、 三青鸟、三骓马、视肉怪兽、甘华树、甘柤树。是各种庄稼生长的地方。

有女和月母之国。有人名曰鹓(w3n),北方曰鹓,来(之)风曰■(y3n), 是处东(极)[北]隅以止日月①,使无相间出没②,司其短长。

【注释】①止:这里是控制的意思。②间:这里是错乱、杂乱的意思。

【译文】有个国家叫女和月母国。有一个神人名叫鹓,北方人称作鹓, 从那里吹来的风称作■,他就处在大地的东北角以便控制太阳和月亮,使不 要交相错乱地出没,掌握它们升起落下时间的长短。

大荒东北隅中,有山名曰凶犁土丘。应龙处南极①,杀蚩(ch9)尤与 夸父②,不得复上,故下数(shu^)旱③。旱而为应龙之状,乃得大雨。

【注释】①应龙:传说中的一种生有翅膀的龙。②蚩尤:神话传说中的东方九黎族首领,以 金作兵器,能唤云呼雨。③数:屡次,频繁。

【译文】在大荒的东北角上,有一座山名叫凶犁土丘山。应龙就住在这 座山的最南端,因杀了神人蚩尤和神人夸父,不能再回到天上,天上因没了 兴云布雨的应龙而使下界常常闹旱灾。下界的人们一遇天旱就装扮成应龙的 样子求雨,就得到大雨。

东海中有流波山,入海七千里。其上有兽,状如牛,苍身而无角,一足, 出入水则必风雨,其光如日月,其声如雷,其名曰夔(ku@)。黄帝得之, 以其皮为鼓,橛(ju6)以雷兽之骨①,声闻五百里②,以威天下。

【注释】①橛:通“撅”。敲,击打。雷兽:就是上文所说的雷神。

②闻:传。

【译文】东海当中有座流波山,这座山在进入东海七千里的地方。山上 有一种野兽,形状像普通的牛,是青苍色的身子却没有犄角,仅有一只蹄子, 出入海水时就一定有大风大雨相伴随,它发出的亮光如同太阳和月亮,它吼 叫的声音如同雷响,名叫夔。黄帝得到它,便用它的皮蒙鼓,再拿雷兽的骨 头敲打这鼓,响声传到五百里以外,用来威震天下。

\chapter{大荒南经}

山海经卷十五 大荒南经

小说-txt天堂
南海之外,赤水之西,流沙之东,有兽,左右有首,名曰■(ch))踢。

有三青兽相并,名曰双双。

【译文】在南海以外,赤水的西岸,流沙的东面,生长着一种野兽,左 边右边都有一个头,名称是■踢。还有三只青色的野兽交相合并着,名称是 双双。

有阿山者。南海之中,有氾天之山,赤水穷焉。赤水之东,有苍梧之野, 舜与叔均之所葬也①。爰有文贝、离俞、(■)[鸱]久、鹰、贾、委维、熊、 罴(p0)、象、虎、豹、狼、视肉②。

【注释】①叔均:又叫商均,传说是帝舜的儿子。帝舜南巡到苍梧而死去,就葬在这里,商 均因此留下,死后也葬在那里。上文说与帝舜一起葬于苍梧之野的是帝丹朱,和这里的说法不同,属 神话传说分歧。②文贝:即上文所说的紫贝,在紫颜色的贝壳上点缀有黑点。离俞:即上文所说的离 朱鸟。贾:据古人说是乌鸦之类的禽鸟。委维:即上文所说的委蛇。

【译文】有座山叫阿山。南海的当中,有一座氾天山,赤水最终流到这 座山。在赤水的东岸,有个地方叫苍梧野,帝舜与叔均葬在那里。这里有花 斑贝、离朱鸟、鹞鹰、老鹰、乌鸦、两头蛇、熊、罴、大象、老虎、豹子、 狼、视肉怪兽。

有荣山,荣水出焉。黑水之南,有玄蛇,食麈(zh()①。

【注释】①麈:一种体型较大的鹿。它的尾巴能用来拂扫尘土。

【译文】有一座荣山,荣水就从这座山发源的。在黑水的南岸,有一条 大黑蛇,正在吞食麈鹿。

有巫山者,西有黄鸟①。帝药②,八斋③。黄鸟于巫山,司此玄蛇。

【注释】①黄鸟:黄,通“皇”。黄鸟即皇鸟,而“皇鸟”亦作“凰鸟”,是属于凤凰一类 的鸟,与上文所说的黄鸟不一样,属同名异物。②药:指神仙药,即长生不死药。③斋:屋舍。

【译文】有一座山叫巫山,在巫山的西面有只黄鸟。天帝的神仙药,就 藏在巫山的八个斋舍中。黄鸟在巫山上,监视着那条大黑蛇。

大荒之中,有不庭之山,荣水穷焉。有人三身。帝俊妻娥皇①,生此三 身之国。姚姓,黍(sh()食,使四鸟。有渊四方,四隅皆达,北属(zh() 黑水②,南属大荒。北旁名曰少和之渊,南旁名曰从(z^ng)渊,舜之所浴 也。

【注释】①帝俊:这里指虞舜,即帝舜。②属:连接。

【译文】在大荒当中,有座不庭山,荣水最终流到这座山。这里有一种 人长着三个身子。帝俊的妻子叫娥皇,这三身国的人就是他们的后代子孙。

三身国的人姓姚,吃黄米饭,能驯化驱使四种野兽。这里有一个四方形的渊 潭,四个角都能旁通,北边与黑水相连,南边和大荒相通。北侧的渊称作少 和渊,南侧的渊称作从渊,是帝舜所洗澡的地方。

又有成山,甘水穷焉。有季禺之国,颛(zhu1n)顼(x&)之子,食黍。

有羽民之国,其民皆生毛羽。有卵民之国,其民皆生卵。

【译文】又有一座成山,甘水最终流到这座山。有个国家叫季禺国,他 们是帝颛顼的子孙后代,吃黄米饭。还有个国家叫羽民国,这里的人都长着 羽毛。又有个国家叫卵民国,这里的人都产卵而又从卵中孵化生出。

大荒之中,有不姜之山,黑水穷焉。又有贾山,汔(q@)水出焉。又有 言山。又有登备之山①。有恝恝(q@ q@)之山。又有蒲山,澧(l!)水出 焉。又有隗(w7i)山,其西有丹②,其东有玉。又南有山,漂水出焉。有 尾山。有翠山。

【注释】①登备之山:即上文所说的登葆山,巫师们凭借此山来往于天地之间,以反映民情, 传达神意。②丹:可能指丹雘,这里有省文。

【译文】在大荒之中,有座不姜山,黑水最终流到这座山。又有座贾山, 汔水从这座山发源。又有座言山。又有座登备山。还有座恝恝山。又有座蒲 山,澧水从这座山发源。又有座隗山,它的西面蕴藏有丹雘,它的东面蕴藏 有玉石。又往南有座高山,漂水就是从这座山中发源的。又有座尾山。还有 座翠山。

有盈民之国,於姓,黍(sh()食。又有人方食木叶。

【译文】有个国家叫盈民国,这里的人姓於,吃黄米饭。又有人正在吃 树叶。

有不死之国,阿姓,甘木是食①。

【注释】①甘木:即不死树,人食用它就能长生不老。

【译文】有个国家叫不死国,这里的人姓阿,吃的是不死树。

大荒之中,有山名曰去痓(c6)。南极果,北不成,去痓果①。

【注释】①从“南极果”以下三句的意义不详,可能是巫师留传下来的几句咒语。

【译文】在大荒当中,有座山叫做去室山。南极果,北不成,去室果。

南海渚中,有神,人面,珥两青蛇,践两赤蛇,曰不廷胡余。

【译文】在南海的岛屿上,有一个神,是人的面孔,耳朵上穿挂着两条 青色蛇,脚底下踩踏着两条红色蛇,这个神叫不廷胡余。

有神名曰(因)因乎,南方曰因(乎),(夸)[来]风曰(乎)民,处 南极以出入风。

【译文】有个神人名叫因乎,南方人单称他为因,从南方吹来的风称作 民,他处在大地的南极主管风起风停。

有襄山。又有重阴之山。有人食兽,曰季厘。帝俊生季厘①,故曰季厘 之国。有缗(m0n)渊。少昊生倍伐,倍伐降处缗渊②。有水四方,名曰俊 坛③。

【注释】①帝俊:这里指帝喾(k)),传说是黄帝之子玄嚣的后代,殷商王室以他为高祖, 号称高辛氏。②降:贬抑。③俊坛:据古人解说,水池的形状像一座土坛,所以叫俊坛。俊坛就是帝 俊的水池。

【译文】有座襄山。又有座重阴山。有人在吞食野兽肉,名叫季厘。帝 俊生了季厘,所以称作季厘国。有一个缗渊。少昊生了倍伐,倍伐被贬住在 缗渊。有一个水池是四方形的,名叫俊坛。

有臷(zh@)民之国。帝舜生无淫,降臷处,是谓巫臷民。巫臷民朌(f6n) 姓,食谷,不绩不经①,服也;不稼不穑(s6)②,食也。爰有歌舞之鸟, 鸾鸟自歌,凤鸟自舞。爰(yu2n)有百兽,相群爰处。百谷所聚。

【注释】①绩:捻搓麻线。这里泛指纺线。经:经线,即丝、棉、麻、毛等织物的纵线,与 纬线即各种织物的横线相交叉,就可织成丝帛、麻布等布匹。这里泛指织布。②稼:播种庄稼。穑: 收获庄稼。

【译文】有个国家叫臷民国。帝舜生了无淫,无淫被贬在臷这个地方居 住,他的子孙后代就是所谓的巫臷民。巫臷民姓朌,吃五谷粮食,不从事纺 织,自然有衣服穿;不从事耕种,自然有粮食吃。这里有能歌善舞的鸟,鸾 鸟自由自在地歌唱,凤鸟自由自在地舞蹈。这里又有各种各样的野兽,群居 相处。还是各种农作物汇聚的地方。

大荒之中,有山名曰融天,海水南入焉。

【译文】在大荒当中,有座山叫做融天山,海水从南面流进这座山。

有人曰凿齿,羿杀之。

【译文】有一个神人叫凿齿,羿射死了他。

有蜮(y))山者,有蜮民之国,桑姓,食黍(sh(),射蜮是食①。有 人方扞(y&)弓射黄蛇②,名曰蜮人③。

【注释】①蜮:据古人说是一种叫短狐的动物,像鳖的样子,能含沙射人,被射中的就要病 死。②扞:拉,张。③域人:就是域民。

【译文】有座山叫做蜮山,在这里有个蜮民国,这里的人姓桑,吃黄米 饭,也把射死的蜮吃掉。有人正在拉弓射黄蛇,名叫蜮人。

有宋山者,有赤蛇,名曰育蛇。有木生山上,名曰枫木①。枫木,蚩尤 所弃其桎(zh@)梏(g))②,是为枫木。

【注释】①枫木:古人说是枫香树,叶子像白杨树叶,圆叶而分杈,有油脂而芳香。②桎梏: 脚镣手铐。神话传说蚩尤被黄帝捉住后给他的手脚系上刑具,后又杀了蚩尤而刑具丢弃,刑具就化成 了枫香树。这与上文所说应龙杀蚩尤有所不同,属神话传说分歧。

【译文】有座山叫做宋山,山中有一种红颜色的蛇,名叫育蛇。山上还 有一种树,名叫枫木。枫木,原来是蚩尤死后所丢弃的手铐脚镣,这些刑具 就化成了枫木。

有人方齿虎尾,名曰祖(zh1)状之尸。

【译文】有个神人正咬着老虎的尾巴,名叫祖状尸。

有小人,名曰焦侥之国,幾(j9)姓,嘉谷是食。

【译文】有一个由三尺高的小人组成的国家,名叫焦侥国,那里的人姓 幾,吃的是优良谷米。

大荒之中,有山名■(xi()涂之山,青水穷焉。有云雨之山,有木名 曰栾。禹攻云雨①,有赤石焉生栾,黄本,赤枝,青叶,群帝焉取药②。

【注释】①攻:从事某项事情。这里指砍伐林木。②取药:传说栾树的花与果实都可以制做 长生不死的仙药。取药就是指采摘可制药的花果。

【译文】在大荒当中,有座山名叫■涂山,青水最终流到这座山。还有 座云雨山,山上有一棵树叫做栾。大禹在云雨山砍伐树木,发现红色岩石上 忽然生出这棵栾树,黄色的茎干,红色的枝条,青色的叶子,诸帝就到这里 来采药。

有国曰[柏服],颛顼生伯服,食黍(sh()。有鼬(y^u)姓之国。有苕 (sh2o)山。又有宗山。又有姓山。又有壑(h))山。又有陈州山。又有东 州山。又有白水山,白水出焉,而生白渊①,昆吾之师所浴也②。

【注释】①生:草木生长。引申为事物的产生、形成。这里即指形成的意思。②昆吾:传说 是上古时的一个诸侯,名叫樊,号昆吾。

【译文】有个国家叫伯服国,颛顼的后代组成伯服国,这里的人吃黄米 饭。有个鼬姓国。有座苕山。又有座宗山。又有座姓山。又有座壑山。又有 座陈州山。又有座东州山。还有座白水山,白水从这座山发源,然后流下来 汇聚成为白渊,是昆吾的师傅洗澡的地方。

有人曰张宏,在海上捕鱼。海中有张宏之国,食鱼,使四鸟。

【译文】有个人叫做张宏,正在海上捕鱼。海里的岛上有个张宏国,这 里的人以鱼为食物,能驯化驱使四种野兽。

有人焉,鸟喙(hu@),有翼,方捕鱼于海。大荒之中,有人名曰驩(hu1n) 头①。鲧(g(n)妻士敬,士敬子曰炎融,生驩头。驩头人面鸟喙,有翼, 食海中鱼,杖翼而行②。维宜■(q!)、苣(j))、穋(qi&)、杨是食③。

有驩头之国。

【注释】①驩头:又叫讙头、驩兜、讙朱、丹朱,不仅名称多异,而且事迹也有多种说法, 乃属神话或古史传说分歧。这里就是异说之一。②杖:凭倚。③维:通“惟”。与,和。宜:烹调作 为菜肴。■、苣:两种蔬菜类植物。穋:一种谷类植物。

【译文】有一种人,长着鸟的嘴,生有翅膀,正在海上捕鱼。在大荒当 中,有个人名叫驩头。鲧的妻子是士敬,士敬生个儿子叫炎融,炎融生了驩 头。驩头长着人的面孔而鸟一样的嘴,生有翅膀,吃海中的鱼,凭借着翅膀 行走。也把■、苣、穋、杨树叶做成食物吃。于是有了驩头国。

帝尧、帝喾(k))、帝舜葬于岳山①。爰有文贝、离俞、(■)[鸱]久、 鹰、[贾]、延维、视肉、熊、罴(p0)、虎、豹②;朱木,赤枝、青华、玄 实。有申山者。

【注释】①岳山:即上文所说狄山。②延维:即上文所说的委蛇、委维。

【译文】帝尧、帝喾、帝舜都葬埋在岳山。这里有花斑贝、三足乌、鹞 鹰、老鹰、乌鸦、两头蛇、视肉怪兽、熊、罴、老虎、豹子;还有朱木树, 是红色的枝干、青色的花朵、黑色的果实。有座申山。

大荒之中,有山名曰天台(高山),海水[南]入焉。

【译文】在大荒当中,有座山名叫天台山,海水从南边流进这座山中。

东(南)海之外,甘水之间,有羲(x9)和之国。有女子名曰羲和,方 (日)浴[日]于甘渊。羲和者,帝俊之妻,生十日。

【译文】在东海之外,甘水之间,有个羲和国。这里有个叫羲和的女子, 正在甘渊中给太阳洗澡。羲和这个女子,是帝俊的妻子,生了十个太阳。

有盖犹之山者,其上有甘柤(zh1),枝干皆赤,黄叶,白华,黑实。

东又有甘华,枝干皆赤,黄叶。有青马。有赤马,名曰三骓(zhu9)。有视 肉。

【译文】有座山叫盖犹山,山上生长有甘柤树,枝条和茎干都是红的, 叶子是黄的,花朵是白的,果实是黑的。在这座山的东端还生长有甘华树, 枝条和茎干都是红色的,叶子是黄的。有青色马。还有红色马,名叫三骓。

又有视肉怪兽。

有小人,名曰菌(j)n)人。

【译文】有一种十分矮小的人,名叫菌人。

有南类之山。爰有遗玉、青马、三骓(zhu9)、视肉、甘华。百谷所在。

【译文】有座南类山。这里有遗玉、青色马、三骓马、视肉怪兽、甘华 树。各种各样的农作物生长在这里。

\chapter{大荒西经}

山海经卷十六 大荒西经

西北海之外,大荒之隅,有山而不合,名曰不周(负子),有两黄兽守 之。有水曰寒暑之水。水西有湿山,水东有幕山。有禹攻共工国山。

【译文】在西北海以外,大荒的一个角落,有座山断裂而合不拢,名叫 不周山,有两头黄色的野兽守护着它。有一条水流名叫寒暑水。寒暑水的西 面有座湿山,寒暑水的东面有座幕山。还有一座禹攻共工国山。

有国名曰淑士,颛(zhu1n)顼(x&)之子。

【译文】有个国家名叫淑士国,这里的人是帝颛顼的子孙后代。

有神十人,名曰女娲(w3)之肠①,化为神,处栗(l@)广之野;横道 而处。

【注释】①女蜗:神话传说女娲是一位以神女的身份做帝王的女神人,是人的脸面蛇的身子, 一天内有七十次变化,她的肠子就化成这十位神人。

【译文】有十个神人,名叫女娲肠,就是女娲的肠子变化而成神的,在 称作栗广的原野上;他们拦断道路而居住。

有人名曰石夷,[西方曰夷],来风曰韦,处西北隅以司日月之长短。

【译文】有位神人名叫石夷,西方人单称他为夷,从北方吹来的风称作 韦,他处在大地的西北角掌管太阳和月亮升起落下时间的长短。

有五采之鸟,有冠,名曰狂鸟。

【译文】有一种长着五彩羽毛的鸟,头上有冠,名叫狂鸟。

有大泽之长山。有白氏之国。

【译文】有一座大泽长山。有一个白氏国。

西北海之外,赤水之东,有长胫之国。

【译文】在西北海以外,赤水的东岸,有个长胫国。

有西周之国,姬(j9)姓,食谷。有人方耕,名曰叔均。帝俊生后稷①, 稷降以百谷。稷之弟曰台(t2i)玺(t1i),生叔均②。叔均是代其父及稷 播百谷,始作耕。有赤国妻氏。有双山。

【注释】①帝俊:这里指帝喾(k)),名叫俊。传说他的第二个妃子生了后稷。后稷:古史 传说他是周朝王室的祖先,姓姬氏,号后稷,善于种庄稼,死后被奉祀为农神。 ②叔均:上文曾说 叔均是后稷的孙子,又说是帝舜的儿子,这里却说是后稷之弟台玺的儿子,诸说不同,乃属神话传说 分歧。

【译文】有个西周国,这里的人姓姬,吃谷米。有个人正在耕田,名叫 叔均。帝俊生了后稷,后稷把各种谷物的种子从天上带到下界。后稷的弟弟 叫台玺,台玺生了叔均。叔均于是代替父亲和后稷播种各种谷物,开始创造 耕田的方法。有个赤国妻氏。有座双山。

西海之外,大荒之中,有方山者,上有青树,名曰柜(j))格之松,日 月所出入也。

【译文】在西海以外,大荒的当中,有座山叫方山,山上有棵青色大树, 名叫柜格松,是太阳和月亮出入的地方。

西北海之外,赤水之西,有(先)[天]民之国,食谷,使四鸟。

【译文】在西北海以外,赤水的西岸,有个天民国,这里的人吃谷米, 能驯化驱使四种野兽。

有北狄之国。黄帝之孙曰始均,始均生北狄。

【译文】有个北狄国。黄帝的孙子叫始均,始均的后代子孙,就是北狄 国人。

有芒山。有桂山。有榣山,其上有人,号曰太子长琴。颛(zhu1n)顼 (x&)生老童①,老童生祝融②,祝融生太子长琴,是处榣山,始作乐风。

【注释】①老童:即上文所说的神人耆童。传说帝颛顼娶于滕■氏,叫女禄,生下老童。 ② 祝融:传说是高辛氏火正,名叫吴回,号称祝融,死后为火官之神。

【译文】有座芒山。有座桂山。有座榣山,山上有一个人,号称太子长 琴。颛琐生了老童,老童生了祝融,祝融生了太子长琴,于是太子长琴住在 榣山上,开始创作乐而风行世间。

有五采鸟三名:一曰皇鸟,一曰鸾鸟,一曰凤鸟。

【译文】有三种长着彩色羽毛的鸟:一种叫凰鸟,一种叫鸾鸟,一种叫 凤鸟。

有虫状如菟(t))①,胸以后者裸(lu%)不见,青如猿状②。

【注释】①虫:古人把人及鸟兽等动物通称为虫,如鸟类称为羽虫,兽类称为毛虫,龟类称 为甲虫,鱼类称鳞虫,人类称为裸虫。这里指野兽。菟:通“兔”。②状:这里不是指具体形状,而 是指颜色的深浅达到某程度的样子。

【译文】有一种野兽的形状与普通的兔子相似,胸脯以后部分全露着而 又分不出来,这是因为它的皮毛青得像猿猴而把裸露的部分遮住了。

大荒之中,有山名曰丰沮(j()玉门,日月所入。

【译文】在大荒的当中,有座山名叫丰沮玉门山,是太阳和月亮降落的 地方。

有灵山,巫咸、巫即、巫朌(f6n)、巫彭、巫姑、巫真、巫礼、巫抵、 巫谢、巫罗十巫,从此升降,百药爰在。

【译文】有座灵山,巫咸、巫即、巫朌、巫彭、巫姑、巫真、巫礼、巫 抵、巫谢、巫罗等十个巫师,从这座山升到天上和下到世间,各种各样的药 物就生长在这里。

(西)有[西]王母之山、壑(h6)山、海山。有沃[民]之国,沃民是处。

沃之野,凤鸟之卵是食,甘露是饮。凡其所欲,其味尽存。爰有甘华、甘柤 (zh1)、白柳、视肉、三骓(zhu9)、璇(xu2n)瑰(gu9)、瑶碧、白木、 琅(l2ng)玕、白丹、青丹①,多银、铁。鸾(凤)[鸟]自歌,凤鸟自舞, 爰有百兽,相群是处,是谓沃之野。

【注释】①三骓:皮毛杂色的马。璇:美玉。瑰:似玉的美石。白木:一种纯白色的树木。

琅玕:传说中的一种结满珠子的树。白丹:一种可作白色染料的自然矿物。青丹:一种可作青色染料 的自然矿物。

【译文】有西王母山、壑山、海山。有个沃民国,沃民便居住在这里。

生活在沃野的人,吃的是凤鸟产的蛋,喝的是天降的甘露。凡是他们心里想 要的美味,都能在凤鸟蛋和甘露中尝到。这里还有甘华树、甘柤树、白柳树, 视肉怪兽、三骓马、璇玉瑰石、瑶玉碧玉、白木树、琅玕树、白丹、青丹, 多出产银、铁。鸾鸟自由自在地歌唱,凤鸟自由自在地舞蹈,还有各种野兽, 群居相处,所以称作沃野。

有三青鸟,赤首黑目,一名曰大■(l@),一名少■,一名曰青鸟。

【译文】有三只青色大鸟、红红的脑袋黑黑的眼睛,一只叫做大■,一 只叫做少■,一只叫做青鸟。

有轩辕之台①,射者不敢西向(射),畏轩辕之台。

【注释】①轩辕之台:即上文所说的轩辕之丘,为传说中的上古帝王黄帝所居之地,故号轩 辕氏。

【译文】有座轩辕台,射箭的人都不敢向西射,因为敬畏轩辕台上黄帝 的威灵。

大荒之中,有龙山,日月所入。有三泽水①,名曰三淖(n4o),昆吾 之所食也②。

【注释】①泽:聚水的洼地。这里作动词用,汇聚的意思。②昆吾:相传是上古时的一个部 落。食:食邑,即古时做为专门供应某人或某部分人生活物资的一块地方。

【译文】大荒当中,有座龙山,是太阳和月亮降落的地方。有三个汇聚 成的大水地,名叫三淖,是昆吾族人取得食物的地方。

有人衣青,以袂(m6i)蔽面①,名曰女丑之尸②。

【注释】①袂:衣服的袖子。②女丑之尸:上文说女丑尸用右手遮住脸面,这里说是用衣袖 遮住脸面,大概因原图上的画像就不一样。

【译文】有个人穿着青色衣服,用袖子遮住脸面,名叫女丑尸。

有女子之国。

【译文】有个女子国。

有桃山。有■(m2ng)山①。有桂山。有于土山。

【注释】①■山:即上文所说的芒山。

【译文】有座桃山。还有座■山。又有座桂山。又有座于土山。

有丈夫之国。

【译文】有个丈夫国。

有弇(y1n)州之山,五采之鸟仰天,名曰鸣鸟。爰有百乐歌儛之风。

【译文】有座弇州山,山上有一种长着五彩羽毛的鸟正仰头向天而嘘, 名叫鸣鸟。因而这里有各种各样乐曲歌舞的风行。

有轩辕之国。江山之南栖为吉,不寿者乃八百岁。

【译文】有个轩辕国。这里的人把居住在江河山岭的南边当作吉利,就 是寿命不长的人也活到了八百岁。

西海陼(zh()中①,有神,人面鸟身,珥两青蛇,践两赤蛇,名曰弇 (y1n)兹。

【注释】①陼:同“渚”。水中的小块陆地。

【译文】在西海的岛屿上,有一个神人,长着人的面孔鸟的身子,耳朵 上穿挂着两条青色蛇,脚底下踩踏着两条红色蛇,名叫弇兹。

大荒之中,有山名日月山,天枢(sh&)也。吴姖天门,日月所入。有 神,人面无臂,两足反属(zh()于头(山)[上]①,名曰嘘。颛(zhu1n) 顼(x&)生老童,老童生重及黎②,帝令重献上天③,令黎(邛)[印]下地 ④。下地是生噎,处于西极,以行日月星辰之行次。

【注释】①属:接连。②重:神话传说中掌管天上事物的官员南正。黎:神话传说中管理地 下人类的官员火正。③献:用手捧着东西给人。这里是举起的意思。④印:痕迹着于其它物件上。如 在信件上加盖印章就要把印章朝下按压。所以,印可通“抑”,即抑压,按下之意。

【译文】大荒当中,有座山名叫日月山,是天的枢纽。这座山的主峰叫 吴姖天门山,是太阳和月亮降落的地方。有一个神人,形状像人而没有臂膀, 两只脚反转着连在头上,名叫嘘。帝颛顼生了老童,老童生了重和黎,帝颛 顼命令重托着天用力往上举,又命令黎撑着地使劲朝下按。于是黎来到地下 并生了噎,他就处在大地的最西端,主管着太阳、月亮和星辰运行的先后次 序。

有人反臂,名曰天虞。

【译文】有个神人反长着臂膀,名叫天虞。

有女子方浴月。帝俊妻常羲(x9),生月十有二,此始浴之。

【译文】有个女子正在替月亮洗澡。帝俊的妻子常羲,生了十二个月亮, 这才开始给月亮洗澡。

有玄丹之山。有五色之鸟,人面有发。爰有青雘(w6n)、黄■(2o), 青鸟、黄鸟,其所集者其国亡。

【译文】有座玄丹山。在玄丹山上有一种长着五彩羽毛的鸟,一副人的 面孔而且有头发。这里还有青雘、黄■,这种 青色的鸟、黄色的鸟,它们 在哪个国家聚集栖息那个国家就会灭亡。

有池,名孟翼之攻颛(zhu1n)顼(x&)之池。

【译文】有个水池,名叫孟翼攻颛顼池。

大荒之中,有山名曰鏖(2o)鏊(2o)鉅,日月所入者。

【译文】大荒当中,有座山名叫鏖鏊鉅山,是太阳和月亮降落的地方。

有兽,左右有首,名曰屏(p0ng)蓬。

【译文】有一种野兽,左边和右边各长着一个头,名叫屏蓬。

有巫山者。有壑(h6)山者。有金门之山,有人名曰黄姖之尸。有比翼 之鸟。有白鸟,青翼,黄尾,玄喙(hu9)。有赤犬,名曰天犬,其所下者 有兵。

【译文】有座叫做巫山的山。又有座叫做壑山的山。还有座金门山,山 上有个人名叫黄姖尸。有比翼鸟。有一种白鸟,长着青色的翅膀,黄色的尾 巴,黑色的嘴壳。有一种红颜色的狗,名叫天犬,它所降临的地方都会发生 战争。

西海之南,流沙之滨,赤水之后,黑水之前,有大山,名曰昆仑之丘。

有神,人面虎身,(有)文(有)尾,皆白①,处之。其下有弱水之渊环之 ②,其外有炎火之山,投物辄(zh6)然③。有人戴胜④,虎齿,有豹尾, 穴处,名曰西王母。此山万物尽有。

【注释】①白:指尾巴上点缀着白色斑点。②弱水:相传这种水轻得不能漂浮起鸿雁的羽毛。

③辄:即,就。然:“燃”的本字。燃烧。④胜:古时妇女的首饰。

【译文】在西海的南面,流沙的边沿,赤水的后面,黑水的前面,屹立 着一座大山,就是昆仑山。有一个神人,长着人的面孔、老虎的身子,尾巴 有花纹,而尾巴上尽是白色斑点,住在这座昆仑山上。昆仑山下有条弱水汇 聚的深渊环绕着它,深渊的外边有座炎火山,一投进东西就燃烧起来。有人 头上戴着玉制首饰,满口的老虎牙齿,有一条豹子似的尾巴,在洞穴中居住, 名叫西王母。这座山拥有世上的各种东西。

大荒之中,有山名曰常阳之山,日月所入。

【译文】大荒当中,有座山名叫常阳山,是太阳和月亮降落的地方。

有寒荒之国。有二人女祭、女薎(mi6)。

【译文】有个寒荒国。这里有两个神人分别叫女祭、女薎。

有寿麻之国。南岳娶州山女,名曰女虔(qi2n)。女虔生季格,季格生 寿麻。寿麻正立无景(y!ng)①,疾呼无响。爰有大暑,不可以往。

【注释】①景:“影”的本字。

【译文】有个国家叫寿麻国。南岳娶了州山的女子为妻,她的名字叫女 虔。女虔生了季格,季格生了寿麻。寿麻端端正正站在太阳下不见任何影子, 高声疾呼而四面八方没有一点回响。这里异常炎热,人不可以前往。

有人无首,操戈盾立,名曰夏耕之尸。故成汤伐夏桀(ji6)于章山①, 克之,斩耕厥(ju6)前②。耕既立,无首,■厥咎(ji))③,乃降于巫山。

【注释】①成汤:即商汤王,商朝的开国国王。夏桀:即夏桀王,夏朝的最后一位国王。② 厥:代词,这里指代成汤。③■:“走”的本字。这里是逃避的意思。厥:这里指代夏耕尸。咎:罪 责。

【译文】有个人没了脑袋,手拿一把戈和一面盾牌立着,名叫夏耕尸。

从前成汤在章山讨伐夏桀,打败了夏桀,斩杀夏耕尸于他的面前。夏耕尸站 立起来后,发觉没了脑袋,为逃避他的罪咎,于是窜到巫山去了。

有人名曰吴回①,奇(j9)左②,是无右臂。

【注释】①吴回:即上文所说的火神祝融。也有说是祝融的弟弟,亦为火正之官。属于神话 传说分歧。②奇:单数。这里指与配偶事物相对而言的单个事物。

【译文】有个人名叫吴回,只剩下左臂膀,而没了右臂膀。

有盖山之国。有树,赤皮支干①,青叶,名曰朱木。

【注释】①支:通“枝”。

【译文】有个盖山国。这里有一种树木,树皮树枝树干都是红色的,叶 子是青色的,名叫朱木。

有一臂民。

【译文】有一种只长一条臂膀的一臂民。

大荒之中,有山,名曰大荒之山,日月所入。有人焉三面,是颛(zhu1n) 顼(x&)之子,三面一臂,三面之人不死。是谓大荒之野。

【译文】大荒当中,有一座山,名叫大荒山,是太阳和月亮降落的地方。

这里有一种人的头上的前边和左边、右边各长着一张面孔,是颛顼的子孙后 代,三张面孔一只胳膊,这种三张面孔的人永远不死。这里就是所谓的大荒 野。

西南海之外,赤水之南,流沙之西,有人珥两青蛇,乘两龙,名曰夏后 开①。开上三嫔(b9n)于天②,得《九辩》与《九歌》以下。此天穆之野, 高二千仞③,开焉得始歌《九招(sh1o)》。

【注释】①夏后开:即上文所说的夏后启。因为汉朝人避汉景帝刘启的名讳,就改“启”为 “开”。②嫔:嫔、宾在古字中通用。这里作为动词,意思是做客。③仞:古代的八尺为一仞。

【译文】在西南海以外,赤水的南岸,流沙的西面,有个人耳朵上穿挂 着两条青色蛇,乘驾着两条龙,名叫夏后启。夏后启曾三次到天帝那里做客, 得到天帝的乐曲《九辩》和《九歌》而下到人间。这里就是所谓的天穆野, 高达二千仞,夏后启在此开始演奏《九招》乐曲。

有(互)[氐]人之国。炎帝之孙名曰灵恝(q@)①,灵恝生(互)[氐] 人,是能上下于天。

【注释】①炎帝:即传说中的上古帝王神农氏。因为以火德为王,所以号称炎帝,又因创造 农具教人们种庄稼,所以叫做神农氏。

【译文】有个氐人国。炎帝的孙子名叫灵恝,灵恝生了氐人,这里的人 能乘云驾雾上下于天。

有鱼偏枯,名曰鱼妇,颛(zhu1n)顼(x&)死即复苏。风道北来①, 天乃大水泉,蛇乃化为鱼,是为鱼妇②。颛顼死即复苏。

【注释】①道:从,由。②为:谓,以为。

【译文】有一种鱼的身子半边干枯,名叫鱼妇,是帝颛顼死了又立即苏 醒而变化的。风从北方吹来,天于是涌出大水如泉,蛇于是变化成为鱼,这 便是所谓的鱼妇。而死去的颛顼就是趁蛇鱼变化未定之机托体鱼躯并重新复 苏的。

有青鸟,身黄,赤足,六首,名曰■(zhu^)鸟。

【译文】有一种青鸟,身子是黄色的,爪子是红色的,长有六个头,名 叫■鸟。

有大巫山。有金之山。西南,大荒之(中)隅,有偏句、常羊之山。

【译文】有座大巫山。又有座金山。在西南方,大荒的一个角落,有偏 句山、常羊山。

按:夏后开即启,避汉景帝讳云①。

【注释】①这两句按语不是《山海经》原文,也不知是谁题写的,但为底本所有,今仍存其 旧。

【译文】按语:夏后开就是夏后启,为避汉景帝刘启的名讳而改的。

\chapter{大荒北经}

山海经卷十七 大荒北经

小_说  txt 天+堂
东北海之外,大荒之中,河水之间,附禺之山①,帝颛(zhu1n)顼(x&) 与九嫔葬焉。爰有(■)[鸱(ch9)]久、文贝、离俞、鸾鸟、(皇)[凤] 鸟、大物、小物②。有青鸟、琅(l2ng)鸟、玄鸟、黄鸟、虎、豹、熊、罴 (p0)、黄蛇、视肉、璿(xu2n)瑰(gu9)、瑶碧③,皆出(卫)于山。[卫] 丘方员三百里,丘南帝俊竹林在焉,大可为舟。竹南有赤泽水,名曰封渊④。

有三桑无枝,[皆高百仞]。丘西有沈渊⑤,颛顼所浴。

【注释】①附禺之山:上文所说的务禺山、鲋鱼山与此同为一山。附、务、鲋,皆古字通用。

②大物、小物:指殉葬的大小用具物品。

③琅鸟:白鸟。琅:洁白。玄鸟:燕子的别称。因它的羽毛黑色,所以 称为玄鸟。玄:黑色。璿:美玉。④封:大。⑤沈:深。

【译文】在东北海以外,大荒的当中,黄河水流经的地方,有座附禺山, 帝颛顼与他的九个妃嫔葬在这座山。这里有鹞鹰、花斑贝、离朱鸟、鸾鸟、 凤鸟、大物、小物。还有青鸟、琅鸟、燕子、黄鸟、老虎、豹子、熊、罴、 黄蛇、视肉怪兽、璿玉瑰石、瑶玉碧玉,都出产于这座山。卫丘方圆三百里, 卫丘的南面有帝俊的竹林,竹子大得可以做成船。竹林的南面有红色的湖 水,名叫封渊。有三棵不生长枝条的桑树,都高达一百仞。卫丘的西面有个 沈渊,是帝颛顼洗澡的地方。

有胡不与之国,烈姓,黍(sh()食。

【译文】有个胡不与国,这里的人姓烈,吃黄米。

大荒之中,有山名曰不咸。有肃慎氏之国。有蜚(f5i)蛭(zh@)①, 四翼。有虫②,兽首蛇身,名曰琴虫。

【注释】①蜚:通“飞”。蛭:环节动物,有好几种,如水蛭、鱼蛭、山蛭等。②虫:这里 指蛇。

【译文】大荒当中,有座山名叫不咸山。有个肃慎氏国。有一种能飞的 蛭,长着四只翅膀。有一种蛇,是野兽的脑袋蛇的身子,名叫琴虫。

有人名曰大人。有大人之国,厘(x9)姓,黍(sh()食。有大青蛇, 黄头,食麈(zh()。

【译文】有一种人名叫大人。有个大人国,这里的人姓厘,吃黄米。有 一种大青蛇,黄色的脑袋,能吞食大鹿。

有榆山。有鲧(g(n)攻程州之山。

【译文】有座榆山。又有座鲧攻程州山。

大荒之中,有山名曰衡天。有先民之山。有槃(p2n)木千里。

【译文】大荒当中,有座山名叫衡天。又有座先民山。有一棵盘旋弯曲 一千里的大树。

有叔歜(ch))国,颛(zhu1n)顼(x&)之子,黍(sh()食,使四鸟: 虎、豹、熊、罴(p0)。有黑虫如熊状,名曰猎猎(x@x@)。

【译文】有个叔歜国,这里的人都是颛顼的子孙后代,吃黄米,能驯化 驱使四种野兽:老虎、豹子、熊和罴。有一种形状与熊相似的黑虫,名叫猎 猎。

有北齐之国,姜姓,使虎、豹、熊、罴(p0)。

【译文】有个北齐国,这里的人姓姜,能驯化驱使老虎、豹子、熊和罴。

大荒之中,有山名曰先槛大逢之山,河济所入,海北注焉。其西有山, 名曰禹所积石。

【译文】大荒当中,有座山名叫先槛大逢山,是黄河水和济水流入的地 方,海水从北面灌注到这里。它的西边也有座山,名叫禹所积石山。

有阳山者。有顺山者,顺水出焉。有始州之国,有丹山。

【译文】有座阳山。又有座顺山,顺水从这座山发源。有个始州国,国 中有座丹山。

有大泽方千里,群鸟所解。

【译文】有一大泽方圆千里,是各种禽鸟脱去旧羽毛再生新羽毛的地 方。

有毛民之国,依姓,食黍(sh(),使四鸟。禹生均国,均国生役采, 役采生修鞈(ji2),修鞈杀绰人。帝念之,潜为之国,是此毛民。

【译文】有个毛民国,这里的人姓依,吃黄米,能驯化驱使四种野兽。

大禹生了均国,均国生了役采,役采生了修鞈,修鞈杀了绰人。大禹哀念绰 人被杀,暗地里帮绰人的子孙后代建成国家,就是这个毛民国。

有儋(d1n)耳之国,任姓,禺号子,食谷。北海之渚中,有神,人面 鸟身,珥两青蛇,践两赤蛇,名曰禺强。

【译文】有个儋耳国,这里的人姓任,是神人禺号的子孙后代,吃谷米。

在北海的岛屿上,有一个神人,长着人的面孔鸟的身子,耳朵上穿挂着两条 青色蛇,脚底下踩踏着两条红色蛇,名叫禺强。

大荒之中,有山名曰北极天柜,海水北注焉。有神,九首人面鸟身,名 曰九凤。又有神,衔蛇操蛇,其状虎首人身,四蹄长肘,名曰强良。

【译文】大荒当中,有座山名叫北极天柜山,海水从北面灌注到这里。

有一个神人,长着九个脑袋和人的面孔鸟的身子,名叫九凤。又有一个神人, 嘴里衔着蛇手中握着蛇,他的形貌是老虎的脑袋人的身子,有四只蹄子和长 长的臂肘,这名叫强良。

大荒之中,有山名曰成都载天。有人珥两黄蛇,把两黄蛇,名曰夸父。

后土生信,信生夸父。夸父不量力,欲追日景(y!ng)①,逮之于禺谷②。

将饮河而不足也,将走大泽,未至,死于此。应龙已杀蚩尤,又杀夸父③, 乃去南方处之,故南方多雨。

【注释】①景:“影”的本字。②逮:到,及。③又杀夸父:先说夸父因追太阳而死,后又 说夸父被应龙杀死,这是神话传说中的分歧。

【译文】大荒当中,有座山名叫成都载天山。有一个人的耳上穿挂着两 条黄色蛇,手上握着两条黄色蛇,名叫夸父。后土生了信,信生了夸父。而 夸父不衡量自己的体力,想要追赶太阳的光影,直追到禺谷。夸父想喝了黄 河水解渴,却不够喝,准备跑到北方去喝大泽的水,还未到,便渴死在这里 了。应龙在杀了蚩尤以后,又杀了夸父,因他的神力耗尽上不了天就去南方 居住,所以南方的雨水很多。

又有无肠之国,是任姓。无继子①,食鱼。

【注释】①无继:即上文所说的无启国。无启就是无嗣、没有子孙后代。但这里却说无肠国 人是无启国人的子孙,显然是有继,而非无继。这正合乎神话传说的神奇诡怪的性质。

【译文】又有个无肠国,这里的人姓任。他们是无继国人的子孙后代, 吃鱼类。

共工之臣名曰相繇(y2o)①,九首蛇身,自环,食于九(土)[山]。

其所■(w&)所尼②,即为源泽,不辛乃苦,百兽莫能处。禹湮(y1n)洪 水③,杀相繇,其血腥臭,不可生谷;其地多水,不可居也。禹湮之,三仞 (r6n)三沮(j()④,乃以为池,群帝因是以为台。在昆仑之北。

【注释】①相繇:即上文所说的相柳。②■:呕吐。尼:止。③湮:阻塞。④三:表示多数, 不是实指。仞:通“■”。充满。沮:败坏。这里指塌陷、陷落。

【译文】共工有一位臣子名叫相繇,长了九个头而是蛇的身子,盘旋自 绕成一团,贪婪地霸占九座神山而索取食物。他所喷吐停留过的地方,立即 变成大沼泽,而气味不是辛辣就是很苦,百兽中没有能居住这里的。大禹堵 塞洪水,杀死了相繇,而相繇的血又腥又臭,使谷物不能生长;那地方又水 涝成灾,使人不能居住。大禹填塞它,屡次填塞而屡次塌陷,于是把它挖成 大池子,诸帝就利用挖出的泥土建造了几座高台。诸帝台位于昆仑山的北 面。

有岳之山,寻竹生焉。

【译文】有座岳山,一种高大的竹子生长在这座山上。

大荒之中,有山名不句,海水[北]入焉。

【译文】大荒当中,有座山名叫不句山,海水从北面灌注到这里。

有系昆之山者,有共工之台,射者不敢北乡(xi4ng)①。有人衣青衣 ②,名曰黄帝女(魃),[妭(b2)]③。蚩尤作兵伐黄帝④,黄帝乃令应龙 攻之冀州之野。应龙畜水,蚩尤请风伯雨师⑤,纵大风雨。黄帝乃下天女曰 (魃)[妭],雨止,遂杀蚩尤。(魃)[妭]不得复上,所居不雨。叔均言之 帝,后置之赤水之北。叔均乃为田祖⑥。(魃)[妭]时亡之,所欲逐之者, 令曰:“神北行⑦!”先除水道,决通沟渎(d*)⑧。

【注释】①乡:通“向”。方向。②衣:穿。这里是动词。③女妭:相传是不长一根头发的 光秃女神,她所居住的地方,天不下雨。④兵:这里指兵器、武器。⑤风伯:神话传说中的风神。雨 师:神话传说中掌管雨水的神。⑥田祖:主管田地之神。⑦北行:指回到赤水之北。⑧渎:小沟渠。

【译文】有座山叫系昆山,上面有共工台,射箭的人因敬畏共工的威灵 而不敢朝北方拉弓射箭。有一个人穿着青色衣服,名叫黄帝女妭。蚩尤制造 了多种兵器用来攻击黄帝,黄帝便派应龙到冀州的原野去攻打蚩尤。应龙积 蓄了很多水,而蚩尤请来风伯和雨师,纵起一场大风雨。黄帝就降下名叫妭 的天女助战,雨被止住,于是杀死蚩尤。女妭因神力耗尽而不能再回到天上, 她居住的地方没有一点雨水。叔均将此事禀报给黄帝,后来黄帝就把女妭安 置在赤水的北面。叔均便做了田神。女妭常常逃亡而出现旱情,要想驱逐她, 便祷告说:“神啊请向北去吧!”事先清除水道,疏通大小沟渠。

有人方食鱼,名曰深目民之国,朌(f5n)姓,食鱼。

【译文】有一群人正在吃鱼,名叫深目民国,这里的人姓朌,吃鱼类。

有钟山者。有女子衣青衣,名曰赤水女子(献)[魃(b2)]①。

【注释】①赤水女子魃:即上文所说的被黄帝安置在赤水之北的女妭。妭,同“魃”。魃: 旱神。

【译文】有座钟山。有一个穿青色衣服的女子,名叫赤水女子魃。

大荒之中,有山名曰融父山,顺水入焉。有人名曰犬戎。黄帝生苗龙, 苗龙生融吾,融吾生弄明,弄明生白犬,白犬有牝(p@n)牡(m(),是为 犬戎,肉食。有赤兽,马状无首,名曰戎宣王尸①。

【注释】①戎宣王尸:传说是犬戎族人奉祀的神。

【译文】大荒当中,有座山名叫融父山,顺水流入这座山。有一种人名 叫犬戎。黄帝生了苗龙,苗龙生了融吾,融吾生了弄明,弄明生了白犬,这 白犬有一公一母而自相配偶,便生成犬戎族人,吃肉类食物。有一种红颜色 的野兽,形状像普通的马却没有脑袋,名叫戎宣王尸。

有山名曰齐州之山、君山、■(qi4n)山、鲜野山、鱼山。

【译文】有几座山分别叫做齐州山、君山、■山、鲜野山、鱼山。

有人一目,当面中生。一曰是威姓,少昊之子,食黍(sh()。

【译文】有一种人长着一只眼睛,这只眼睛正长在脸面的中间。一种说 法认为他们姓威,是少昊的子孙后代,吃黄米。

有[无]继(无)民,[无]继(无)民任姓,无骨子,食气、鱼。

【译文】有一种人称作无继民,无继民姓任,是无骨民的子孙后代,吃 的是空气和鱼类。

西北海外,流沙之东,有国曰中■(bi3n),颛(zhu1n)顼(x&)之 子,食黍(sh()。

【译文】在西北方的海外,流沙的东面,有个国家叫中■国,这里的人 是颛顼的子孙后代,吃黄米。

有国名曰赖丘。有犬戎国。有(神)[人],人面兽身,名曰犬戎。

【译文】有个国家名叫赖丘。还有个犬戎国。有一种人,长着人的面孔 兽的身子,名叫犬戎。

西北海外,黑水之北,有人有翼,名曰苗民。颛顼生驩(hu1n)头,驩 头生苗民,苗民厘(x9)姓,食肉。有山名曰章山。

【译文】在西北方的海外,黑水的北岸,有一种人长着翅膀,名叫苗民。

颛顼生了驩头,驩头生了苗民,苗民人姓厘,吃的是肉类食物。还有一座山 名叫章山。

大荒之中,有衡石山、九阴山、泂[灰]野之山,上有赤树,青叶赤华, 名曰若木。

【译文】大荒当中,有衡石山,九阴山、灰野山,山上有一种红颜色的 树木,青色的叶子红色的花朵,名叫若木。

有牛黎之国。有人无骨,儋耳之子。

【译文】有个牛黎国。这里的人身上没有骨头,是儋耳国人的子孙后代。

西北海之外,赤水之北,有章尾山。有神,人面蛇身而赤,[身长千里], 直目正乘①,其瞑乃晦,其视乃明,不食不寝不息,风雨是谒(y6)②。是 烛九阴③,是谓烛龙。

【注释】①乘:据学者研究,“乘”可能是“朕”字的假借音。朕:缝隙。②谒:据学者研 究,“谒”是“噎”的假借音。噎:吃饭太快而食物堵塞咽喉。这里是吞食、吞咽的意思。③九阴: 阴暗之地。

【译文】在西北方的海外,赤水的北岸,有座章尾山。有一个神人,长 着人的面孔蛇的身子而全身是红色,身子长达一千里,竖立生长的眼睛正中 合成一条缝,他闭上眼睛就是黑夜、睁开眼睛就是白昼,不吃饭不睡觉不呼 吸,只以风雨为食物。他能照耀阴暗的地方,所以称作烛龙。

\chapter{海内经}

东海之内,北海之隅,有国名曰朝鲜①。天毒②,其人水居,偎人爱(之) [人]。

【注释】①朝鲜:就是现在朝鲜半岛上的朝鲜和韩国。②天毒:据古人解说,即天竺国,有 文字,有商业,佛教起源于此国中。而天竺国就是现在的印度。但印度在南,朝鲜在北,一南一北, 相距很远,记在一处,不合情理,则文字上似有讹误或脱遗。

【译文】在东海以内,北海的一个角落,有个国家名叫朝鲜。还有一个 国家叫天毒,天毒国的人傍水而居,怜悯人慈爱人。

西海之内,流沙之中,有国名曰壑(h6)市。

【译文】在西海以内,流沙的中央,有个国家名叫壑市国。

西海之内,流沙之西,有国名曰氾(f4n)叶。

【译文】在西海以内,流沙的西边,有个国家名叫氾叶国。

流沙之西,有鸟山者,三水出焉。爰有黄金、璿(xu2n)瑰(gu9)、 丹货、银铁①,皆流于此中②。又有淮山,好水出焉。

【注释】①丹货:不详何物。②流:淌出。这里是出产、产生的意思。

【译文】流沙西面,有座山叫鸟山,三条河流共同发源于这座山。这里 所有的黄金、璿玉瑰石、丹货、银铁,全都产于这些水中。又有座大山叫淮 山,好水就是从这座山发源的。

流沙之东,黑水之西,有朝(zh1o)云之国、司彘(zh@)之国。黄帝 妻雷祖①,生昌意。昌意降处若水,生韩流。韩流擢(zhu¥)首、谨耳、人 面、豕(sh!)喙(hu@)、麟身、渠股、豚(t*n)止②,取淖(n4o)子曰 阿女③,生帝颛(zhu1n)顼(x&)。

【注释】①雷祖:即嫘祖,相传是教人们养蚕的始祖。②擢:引拔,耸起。这里指物体因吊 拉变成长竖形的样子。谨:慎重小心,谨慎细心。这里是细小的意思。渠股:即今天所说的罗圈腿。

③取:通“娶”。

【译文】在流沙的东面,黑水的西岸,有朝云国、司彘国。黄帝的妻子 雷祖生下昌意。昌意自天上降到若水居住,生下韩流。韩流长着长长的脑袋、 小小的耳、人的面孔、猪的长嘴、麒麟的身子、罗圈着双腿、小猪的蹄子, 娶淖子族人中叫阿女的为妻,生下帝颛顼。

流沙之东,黑水之间,有山名不死之山。

【译文】在流沙的东面,黑水流经的地方,有座山名叫不死山。

华山青水之东,有山名曰肇(zh4o)山。有人名曰柏[子]高,柏[子]高 上下于此,至于天。

【译文】在华山青水的东面,有座山名叫肇山。有个仙人名叫柏子高, 柏子高由这里上去下来的,直至到达天上。

西南黑水之间,有都广之野,后稷葬焉。爰有膏(g1o)菽(sh&)、膏 稻、膏黍(sh()、膏稷①,百谷自生,冬夏播琴②。鸾鸟自歌,凤鸟自儛, 灵寿实华③,草木所聚。爰有百兽,相群爰处。此草也,冬夏不死。

【注释】①膏:这里是味道美好而光滑如膏的意思。菽:豆类植物的总称。稷:谷子。②播 琴:即播种。这是古时楚地人的方言。③灵寿:即上文所说的椐树,所生枝节像竹节,粗细长短都正 好合于拐杖,不必人工制做,所以古代老人常利用这种天然拐杖。也有一种说法,认为灵寿是一种生 长在昆仑山及其附近地方的特殊树木,人吃了它结的果实就会长生不死,所以叫灵寿树。

【译文】在西南方黑水流经的地方,有一处叫都广野,后稷就埋葬在这 里。这里出产膏菽、膏稻、膏黍、膏稷,各种谷物自然成长,冬夏都能播种。

鸾鸟自由自在地歌唱,凤鸟自由自在地舞蹈,灵寿树开花结果,丛草树林茂 盛。这里还有各种禽鸟野兽,群居相处。在这个地方生长的草,无论寒冬炎 夏都不会枯死。

南海之(外)[内],黑水青水之间,有木名曰若木,若水出焉。

【译文】在南海以内,黑水青水流经的地方,有一种树木名叫若木,而 若水就从若木生长的地底下发源。

有禺中之国。有列襄之国。有灵山,有赤蛇在木上,名曰蝡(ru3n)蛇, 木食。

【译文】有个禺中国。又有个列襄国。有一座灵山,山中的树上有一种 红颜色的蛇,叫做耎蛇,以树木为食物。

有盐长之国。有人焉鸟首,名曰鸟(氏)[民]。

【译文】有个盐长国。这里的人长着鸟一样的脑袋,称作鸟民。

有九丘,以水络之,名曰陶唐之丘、(有)叔得之丘、孟盈之丘、昆吾 之丘、黑白之丘、赤望之丘、参卫之丘、武夫之丘、神民之丘。有木,青叶 紫茎,玄华黄实,名曰建木,百仞无枝①,[上]有九■(zh()②,下有九 枸(j()③,其实如麻,其叶如芒。大(t4i)皞(h4o)爰过④,黄帝所为。

【注释】①仞:古时以八尺为一仞。②■:树枝弯曲。③枸:树根盘错。④大皞:又叫太昊、 太皓,即伏羲氏,古史传说中的上古帝王,姓风。他开始画八卦,教人们捕鱼放牧,用来充作食物。

又是神话传说中的人类始祖。

【译文】有九座山丘,都被水环绕着,名称分别是陶唐丘、叔得丘、孟 盈丘、昆吾丘、黑白丘、赤望丘、参卫丘、武夫丘、神民丘。有一种树木, 青色的叶子紫色的茎干,黑色的花朵黄色的果实,叫做建木,高达一百仞的 树干上不生长枝条,而树顶上有九根弯蜒曲折的桠枝,树底下有九条盘旋交 错的根节,它的果实像麻子,叶子像芒树叶。大皞凭借建木登上天,黄帝栽 培了建木。

有窫(zh2)窳(y*),龙首,是食人。有(青)兽,人面,名曰猩猩。

【译文】有一种窫窳兽,长着龙一样的脑袋,能吃人。还有一种野兽, 长着人一样的面孔,名叫猩猩。

西南有巴国。皞(h4o)生咸鸟,咸鸟生乘厘,乘厘生后照,后照是始 为巴人。

【译文】西南方有个巴国。大皞(h4o)生了咸鸟,咸鸟生了乘厘,乘 厘生了后照,而后照就是巴国人的始祖。

有国名曰流黄辛氏,其域中方三百里,其出是(尘土)[麈](zh()。

有巴遂山,渑(sh6ng)水出焉。

【译文】有个国家名叫流黄辛氏国,它的疆域方圆三百里,这里出产一 种大鹿。还有一座巴遂山,渑水从这座山发源。

又有朱卷之国。有黑蛇,青首,食象。

【译文】又有个朱卷国。这里有一种黑颜色的大蛇,长着青色脑袋,能 吞食大象。

南方有赣(g4n)巨人,人面长(臂)[唇],黑身有毛,反踵,见人(笑 亦)[则]笑,唇蔽其面,因即逃也。

【译文】南方有一种赣巨人,长着人的面孔而嘴唇长长的,黑黑的身上 长满了毛,脚尖朝后而脚跟朝前反长着,看见人就发笑,一发笑而嘴唇便会 遮住他的脸面,人就趁此立即逃走。

又有黑人,虎首鸟足,两手持蛇,方啖(d4n)之。

【译文】还有一种黑人,长着老虎一样的脑袋禽鸟一样的爪子,两只手 握着蛇,正在吞食它。

有嬴(y0ng)民,鸟足。有封豕(sh!)。

【译文】有一种人称作嬴民,长着禽鸟一样的爪子。还有大野猪。

有人曰苗民。有神焉,人首蛇身,长如辕,左右有首,衣紫衣①,冠旃 (zh1n)冠②,名曰延维③,人主得而飨(xi3ng)食之④,伯(b4)天下 ⑤。

【注释】①衣紫衣:前一个“衣”是动词,穿的意思。后一个“衣”是名词,即衣服。②冠 旃冠:前一个“冠”是动词,戴的意思。后一个“冠”是名词,即帽子。旃:纯红色的曲柄旗。这里 仅是红色的意思,与上一句的紫色相对。③延维:即上文所说的委蛇,就是双头蛇。④人主:君主, 一国之主。飨:祭献。⑤伯:通“霸”。

【译文】有一种人称作苗民。这地方有一个神,长着人的脑袋蛇的身子, 身躯长长的像车辕,左边右边各长着一个脑袋,穿着紫色衣服,戴着红色帽 子,名叫延维,人主得到它后加以奉飨祭祀,便可以称霸天下。

有鸾鸟自歌,凤鸟自舞。凤鸟首文曰“德”,翼文曰“顺”,膺文曰“仁”, 背文曰“義”,见(xi4n)则天下和。

【译文】有鸾鸟自由自在地歌唱,有凤鸟自由自在地舞蹈。凤鸟头上的 花纹是“德”字,翅膀上的花纹是“顺”字,胸脯上的花纹是“仁”字,脊 背上的花纹是“義”字,它一出现就会使天下和平。

又有青兽如菟(t))①,名曰■(j)n)狗。有翠鸟②。有孔鸟③。

【注释】①菟:通“兔”。②翠鸟:即翡翠鸟,形状像燕子。古人说雄性的叫翡,羽毛是红 色;雌性的叫翠,羽毛是青色。实际上,翡翠鸟的羽毛有好多种颜色,不止红、青二色,所以自古以 来就做装饰品用。③孔鸟:即孔雀鸟。

【译文】又有一种像兔子的青色野兽,名叫■狗。又有翡翠鸟。还有孔 雀鸟。

南海之内,有衡山,有菌山,有桂山。有山名三天子之都。

【译文】在南海以内,有座衡山,又有座菌山,还有座桂山。还有座山 叫做三天子都山。

南方苍梧之丘,苍梧之渊,其中有九嶷(y0)山,舜之所葬。在长沙零 陵界中。

【译文】南方有一片山丘叫苍梧丘,还有一个深渊叫苍梧渊,在苍梧丘 和苍梧渊的中间有座九嶷山,帝舜就葬埋在这里。九嶷山位于长沙零陵境 内。

北海之内,有蛇山者,蛇水出焉,东入于海。有五采之鸟,飞蔽一乡, 名曰翳(y@)鸟①。又有不距之山,巧倕(ru@)葬其西②。

【注释】①翳鸟:传说是凤凰之类的鸟。②巧倕:相传是上古帝尧时代一位灵巧的工匠。

【译文】在北海以内,有座山叫蛇山,蛇水从蛇山发源,向东流入大海。

有一种长着五彩羽毛的鸟,成群地飞起而遮蔽一乡的上空,名叫翳鸟。还有 座不距山,巧倕便葬在不距山的西面。

北海之内,有反缚盗械、带戈常倍之佐①,名曰相顾之尸②。

【注释】①盗械:古时,凡因犯罪而被戴上刑具就称作盗械。戈:古代一种兵器。倍:通“背”。

背弃。佐:辅助帝王的人。②相顾之尸:也是上文所说贰负之臣一类的人。

【译文】在北海以内,有一个反绑着戴刑具、带着戈而图谋叛逆的臣子, 名叫相顾尸。

伯夷父生西岳①,西岳生先龙,先龙是始生氐(d!)羌(qi1ng),氐 羌乞姓。

【注释】①伯夷父:相传是帝颛(zhu1n)顼(x&)的师傅。

【译文】伯夷父生了西岳,西岳生了先龙,先龙的后代子孙便是氐羌, 氐羌人姓乞。

北海之内,有山,名曰幽都之山,黑水出焉。其上有玄鸟、玄蛇、玄豹、 玄虎、玄狐蓬尾。有大玄之山。有玄丘之民①。有大幽之国。有赤胫之民②。

【注释】①玄丘之民:古人说是生活在丘上的人物都是黑的。②赤胫之民:古人说是从膝盖 以下的腿部全为红色的一种人物。

【译文】北海以内,有一座山,名叫幽都山,黑水从这座山发源。山上 有黑色鸟、黑色蛇、黑色豹子、黑色老虎,有毛蓬蓬尾巴的黑色狐狸。有座 大玄山。有一种玄丘民。有个大幽国。有一种赤胫民。

有钉灵之国,其民从厀(x9)以下有毛①,马蹄善走②。

【注释】①厀:同“膝”。②走:跑。

【译文】有个钉灵国,这里的人从膝盖以下的腿部都有毛,长着马的蹄 子而善于快跑。

炎帝之孙伯陵①,伯陵同吴权之妻阿女缘妇②,缘妇孕三年,是生鼓、 延、殳(sh&)。[殳]始为侯③,鼓、延是始为钟④,为乐风。

【注释】①炎帝:即神农氏,传说中的上古帝王。②同:通“通”。通奸。吴权:传说中的 人物。③侯:练习或比赛射箭时用的箭靶。④钟:古代一种打击乐器 【译文】炎帝的孙子叫伯陵,伯陵与吴权的妻子阿女缘妇私通,阿女缘 妇怀孕三年,这才生下鼓、延、殳三个儿子。殳最初发明了箭靶,鼓、延二 人发明了钟,作了乐曲和音律。

黄帝生骆明,骆明生白马,白马是为鲧(g(n)①。

【注释】①鲧:相传是大禹的父亲。

【译文】黄帝生了骆明,骆明生了白马,这白马就是鲧。

帝俊生禺号①,禺号生淫梁②,淫梁生番禺,是始为舟。番禺生奚仲, 奚仲生吉光,吉光是始以木为车。

【注释】①帝俊:这里指黄帝。②淫梁:即上文所说的禺京。

【译文】帝俊生了禺号,禺号生了淫梁,淫梁生了番禺,这位番禺最初 发明了船。番禺生了奚仲,奚仲生了吉光,这位吉光最初用木头制做出车子。

少(sh4o)皞(h4o)生般①,般是始为弓矢。

【注释】①少皞:即上文所说的少昊号称金天氏,传说中的上古帝王。

【译文】少皞生了般,这位般最初发明了弓和箭。

帝俊赐羿彤(t¥ng)弓素矰(z5ng)①,以扶下国,羿是始去恤(x)) 下地之百艰②。

【注释】①彤:朱红色。矰:一种用白色羽毛装饰并系着丝绳的箭。②恤:体恤,周济。

【译文】帝俊赏赐给后羿红色弓和白色矰箭,用他的射箭技艺去扶助下 界各国,后羿便开始去救济世间人们的各种艰苦。

帝俊生晏龙,晏龙是[始]为琴瑟(s6)。

【译文】帝俊生了晏龙,这位晏龙最初发明了琴和瑟两种乐器。

帝俊有子八人①,是始为歌舞。

【注释】①帝俊:这里指帝舜。

【译文】帝俊有八个儿子,他们开始创作出歌曲和舞蹈。

帝俊生三身①,三身生义均②,义均是始为巧倕(ru@),是始作下民 百巧。后稷是播百谷。稷之孙曰叔均③,始作牛耕。大比赤阴④,是始为国。

禹、鲧(g(n)是始布土⑤,均定九州⑥。

【注释】①帝俊:这里也是指帝舜。②义均:就是上文所说的叔均,但说是帝舜的儿子,这 里却说是帝舜的孙子,属于神话传说的不同。③叔均:上文曾说叔均是后稷之弟台玺的儿子,这里又 说是后稷的孙子,而且和前面说的义均也分成了二人,神话传说分歧,往往有所不同。④大比赤阴: 意义不明。也有学者认为可能是后稷的生母姜嫄。“比”大概为“妣”的讹文。妣:母亲。“赤阴” 的读音与“姜嫄”相近。据古史传说,后稷被封于邰地而建国,姜嫄即居住在这里,所以下面说“是 始为国”。⑤布土:传说鲧与大禹父子二人相继治理洪水,鲧使用堵塞的方法,大禹使用疏通的方法, 都需要挖掘泥土。布即施予,施行。土即土工,治河时填土、挖土工程。⑥均:平均,均匀。引申为 度量、衡量。九州:相传大禹治理了洪水以后,把中原划分为九个行政区域,就是九州。

【译文】帝俊生了三身,三身生了义均,这位义均便是所谓的巧倕,从 此开始发明了世间的各种工艺技巧。后稷开始播种各种农作物。后稷的孙子 叫叔均,这位叔均最初发明了使用牛耕田。大比赤阴,开始受封而建国。大 禹和鲧开始挖掘泥土治理洪水,度量划定九州。

炎帝之妻,赤水之子听訞(y1o)生炎居,炎居生节并,节并生戏器, 戏器生祝融。祝融降处于江水,生共工。共工生术器,术器首方颠①,是复 土(穰)[壤],以处江水。共工生后土,后土生噎(y5)鸣,噎鸣生岁十 有二。

【注释】①颠:头顶。

【译文】炎帝的妻子,即赤水氏的女儿听訞生下炎居,炎居生了节并, 节并生了戏器,戏器生了祝融。祝融降临到江水居住,便生了共工。共工生 了术器。术器的头是平顶方形,他恢复了祖父祝融的土地,从而又住在江水。

共工生了后土,后土生了噎鸣,噎鸣生了一年中的十二个月。

洪水滔天①。鲧(g(n)窃帝之息壤以堙(y9n)洪水②,不待帝命。帝 令祝融杀鲧于羽郊。鲧复生禹③。帝乃命禹卒布土,以定九州。

【注释】①滔:漫。②息壤:神话传说中的一种能够自生自长、永不耗损的土壤。③复生: 相传鲧死了三年而尸体不腐烂,用刀剖开肚腹,就产生了禹。“复”即“腹”的同声假借字。

【译文】洪荒时代到处是漫天大水。鲧偷拿天帝的息壤用来堵塞洪水, 而未等待天帝下令。天帝派遣祝融把鲧杀死在羽山的郊野。禹从鲧的遗体肚 腹中生出。天帝就命令禹最后再施行土工制住了洪水,从而能划定九州区 域。







\backmatter

\end{document}