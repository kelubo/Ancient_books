% 论语
% 论语.tex

\documentclass[a4paper,12pt,UTF8,twoside]{ctexbook}

% 设置纸张信息。
\RequirePackage[a4paper]{geometry}
\geometry{
	%textwidth=138mm,
	%textheight=215mm,
	%left=27mm,
	%right=27mm,
	%top=25.4mm, 
	%bottom=25.4mm,
	%headheight=2.17cm,
	%headsep=4mm,
	%footskip=12mm,
	%heightrounded,
	inner=1in,
	outer=1.25in
}

% 设置字体,并解决显示难检字问题。
\xeCJKsetup{AutoFallBack=true}
\setCJKmainfont{SimSun}[BoldFont=SimHei, ItalicFont=KaiTi, FallBack=SimSun-ExtB]

% 目录 chapter 级别加点(.)。
\usepackage{titletoc}
\titlecontents{chapter}[0pt]{\vspace{3mm}\bf\addvspace{2pt}\filright}{\contentspush{\thecontentslabel\hspace{0.8em}}}{}{\titlerule*[8pt]{.}\contentspage}

% 设置 part 和 chapter 标题格式。
\ctexset{
	chapter/name={第,篇},
	chapter/number={\chinese{chapter}},
	section/name={},
	section/number={}
}

% 设置古文原文格式。
\newenvironment{yuanwen}{\bfseries\zihao{4}}

\title{\heiti\zihao{0} 论语}
\author{孔子弟子及再传弟子}
\date{春秋战国}

\begin{document}

\maketitle
\tableofcontents

\frontmatter
\chapter{前言}

《论语》是春秋时期思想家、教育家孔子的弟子及再传弟子记录孔子及其弟子言行而编成的语录体散文集,成书于战国前期。全书共20篇492章,以语录体为主,叙事体为辅,较为集中地体现了孔子及儒家学派的政治主张、伦理思想、道德观念、教育原则等。

作品多为语录,但辞约义富,有些语句、篇章形象生动,其主要特点是语言简练,浅近易懂,而用意深远,有一种雍容和顺、纡徐含蓄的风格,能在简单的对话和行动中展示人物形象。

《论语》自宋代以后,被列为“四书”之一,成为古代学校官定教科书和科举考试必读书。

《论语》内容涉及政治、教育、文学、哲学以及立身处世的道理等多方面。现存《论语》20篇,492章,其中记录孔子与弟子及时人谈论之语约444章,记孔门弟子相互谈论之语48章。

\section{作品目录}

《论语》的篇名通常取开篇前两个字作为篇名;若开篇前两个字是“子曰”,则跳过取句中的前两个字;若开篇三个字是一个词,则取前三个字。篇名与其中的各章没有意义上的逻辑关系,仅可当作页码看待。这种类型的篇题在先秦时代的典籍中比较常见。

\section{成书过程}

《论语》是孔门弟子集体智慧的结晶。早在春秋后期孔子设坛讲学时期,《论语》主体内容就已初始创成;孔子去世以后,他的弟子和再传弟子代代传授他的言论,并逐渐将这些口头记诵的语录言行记录下来,因此称为“论”;《论语》主要记载孔子及其弟子的言行,因此称为“语”。清朝赵翼解释说:“语者,圣人之语言,论者,诸儒之讨论也。”其实,“论”又有纂的意思,所谓《论语》,是指将孔子及其弟子的言行记载下来编纂成书。其编纂者主要是仲弓、子游、子夏、子贡,他们忧虑师道失传,首先商量起草以纪念老师。然后和少数留在鲁国的弟子及再传弟子完成。

清代学者崔述注意到今本《论语》前后十篇在文体和称谓上存在差异,前十篇记孔子答定公、哀公之问,皆变文称“孔子对曰”,以表示尊君。答大夫之问则称“子曰”,表示有别于君,“以辨上下而定民志”。而后十篇中的《先进》《颜渊》等篇,答大夫之问也皆作“孔子对曰”,故怀疑“前十篇皆有子、曾子门人所记,去圣未远,礼制方明;后十篇则后人所续记,其时卿位益尊,卿权益重,盖有习于当世所称而未尝详考其体例者,故不能无异同也”。又如,前十篇中孔子一般称“子”不称“孔子”,门人问学也不作“问于孔子”。而后十篇中的《季氏》《微子》多称孔子,《阳货》篇子张问仁,《尧曰》篇子张问政,皆称“问于孔子”,与《论语》其他篇不同,“其非孔氏遗书明甚,盖皆后人采之他书者”。受崔述的影响,以后学者继续从《论语》前后十篇用语、称谓的差异对其成书作出判断,有学者甚至认为《论语》最初只有单独的篇,其编定成书,要在汉代以后。

唐代陆德明《经典释文》转引郑玄注云:《论语》“仲弓、子游、子夏等撰。”这一说法在郭店简中得到旁证。郭店一号墓不晚于公元前300年。郭店简“《语丛·三》简引述《论语》,更确证该书之早”。《语丛·一》引用子思子《坊记》内容,而《坊记》还引用过《论语》的内容。“《语丛》摘录《坊记》,证明《坊记》早于战国中期之末,而《坊记》又引述《论语》,看来《论语》为孔子门人仲弓、子夏等撰定之说还是可信的。”孔子弟子中,有子代孔子,以所事孔子事之,称子并不奇怪,子指老师,对儒家学子除了师承之外亦有掌门人之意。除了孔子,有子、曾子、冉子、闵子亦称子,冉子、闵子早逝,故师承的儒家掌门,只能是曾参,故代有子者只剩曾子有可能。颜回,虽然被尊奉,但由于早死,没来得及收徒,不是弟子记载,故不称子,可能是家人所记。原宪、漆雕开,虽然收徒,世称子思子、漆雕子,但《论语》所记,亦不是弟子记载,故亦不称子,多半是师兄弟偶而提及。

《论语》既是语录体又是若干断片的篇章集合体。这些篇章的排列不一定有什么道理;就是前后两章间,也不一定有什么关联。而且这些断片的篇章绝不是一个人的手笔。《论语》一书,篇幅不多,却出现了不少次的重复的章节。其中有字句完全相同的,如“巧言令色鲜矣仁”一章,先见于《学而篇第一》,又重出于《阳货篇第十七》;“博学于文”一章,先见于《雍也篇第六》,又重出于《颜渊篇第十二》。又有基本上是重复只是详略不同的,如“君子不重”章,《学而篇第一》多出十一字,《子罕篇第九》只载“主忠信”以下的十四个字;“父在观其志”章,《学而篇第一》多出十字,《里仁篇第四》只载“三年”以下的十二字。还有一个意思,却有各种记载的,如《里仁篇第四》说:“不患莫己知,求可为也。”《宪问篇第十四》又说:“不患人之不己知,患不知人也。”《卫灵公篇第十五》又说:“君子病无能焉,不病人之不己知也。”如果加上《学而篇第一》的“人不知而不愠,不亦君子乎”,便是重复四次。这种现象只能作一个合理的推论:孔子的言论,当时弟子各有记载,后来才汇集成书。

《论语》的作者中当然有孔子的学生。《论语》的篇章不但出自孔子不同学生之手,而且还出自他不同的再传弟子之手。这里面不少是曾参的学生的记载。如《泰伯篇第八》第一章:“曾子有疾,召门弟子曰:‘启予足!启予手!《诗》云,战战兢兢,如临深渊,如履薄冰。而今而后,吾知免夫!小子!’”又如《子张篇第十九》:“子夏之门人问交于子张。子张曰:‘子夏云何?’对曰:‘子夏曰:可者与之,其不可者拒之。’子张曰:‘异乎吾所闻:君子尊贤而容众,嘉善而矜不能。我之大贤欤,于人何所不容?我之不贤欤,人将拒我,如之何其拒人也?’”这一段又像子张或者子夏的学生的记载。又如《先进篇第十一》的第五章和第十三章:“子曰:‘孝哉闵子骞,人不间于其父母昆弟之言。’”“闵子侍侧,訚訚如也;子路,行行如也;冉有、子贡,侃侃如也。子乐。”孔子称学生从来直呼其名,独独这里对闵损称字。有人说,这是“孔子述时人之言”,崔述在《论语余说》中对这一解释加以驳斥。这一章可能就是闵损的学生所追记的,因而有这一不经意的失实。至于《闵子侍侧》一章,不但闵子骞称“子”,而且列在子路、冉有、子贡三人之前,都是难以理解的,以年龄而论,子路最长;以仕宦而论,闵子更赶不上这三人。他凭什么能在这一段记载上居于首位而且得着“子”的尊称呢?合理的推论是,这也是闵子骞的学生把平日闻于老师之言追记下来而成的。
《论语》一书有孔子弟子的笔墨,也有孔子再传弟子的笔墨,其著作年代有先有后。崔述《洙泗信录》断定《论语》的少数篇章的“驳杂”。从词义的运用上可反映出《论语》的著笔先后间相距或者不止于三、五十年。

思想内容
《论语》作为儒家经典,其内容博大精深,包罗万象,《论语》的思想主要有三个既各自独立又紧密相依的范畴:伦理道德范畴——仁,社会政治范畴——礼,认识方法论范畴——中庸。仁,首先是人内心深处的一种真实的状态,这种真的极致必然是善的,这种真和善的全体状态就是“仁”。孔子确立的仁的范畴,进而将礼阐述为适应仁、表达仁的一种合理的社会关系与待人接物的规范,进而明确“中庸”的系统方法论原则。“仁”是《论语》的思想核心。

《论语》反映了孔子的教育原则。孔子因材施教,对于不同的对象,考虑其不同的素质、优点和缺点、进德修业的具体情况,给予不同的教诲,表现了诲人不倦的可贵精神。据《颜渊》记载,同是弟子问仁,孔子有不同的回答,答颜渊“克己复礼为仁”(为仁的表现之一为克己复礼,有所不为);答仲弓“己所不欲,勿施于人”(就己与人之间的关系,以欲施做答,欲是个人的主观能动性之取舍,施是个人主观能动性的实践,用好心坏心来说,要防止好心办坏事,就要慎施);答司马牛“仁者其言也讱”。颜渊学养高深,故答以“仁”学纲领,对仲弓和司马牛则答以细目。又如,孔子回答子路和冉有的同一个问题,内容完全不同。答子路的是:“又父兄在,如之何其闻斯行之。”因为“由也兼人,故退之”。答冉有的是:“闻斯行之。”因为“求也退,故进之”。这不仅是因材施教教育方法的问题,其中还饱含孔子对弟子的高度的责任心。

艺术特色
《论语》多为语录,但都辞约义富,有些语句、篇章形象生动。如《子路曾皙冉有公西华侍坐》不仅篇幅较长,而且注重记述,算得上一篇结构完整的记叙文,人物形象鲜明,思想倾向通过人物表情、动作、对话自然地显露出来,具有较强的艺术性。
孔子是《论语》描述的中心,“夫子风采,溢于格言”(《文心雕龙·征圣》);书中不仅有关于他的仪态举止的静态描写,而且有关于他的个性气质的传神刻画。此外,围绕孔子这一中心,《论语》还成功地刻画了一些孔门弟子的形象。如子路的率直鲁莽,颜回的温雅贤良,子贡的聪颖善辩,曾皙的潇洒脱俗等等,都称得上个性鲜明,能给人留下深刻印象。
《论语》的主要特点是语言简练,用意深远,有一种雍容和顺、纡徐含蓄的风格;还有就是在简单的对话和行动中展示人物形象;同时语言浅近易懂,接近口语,也是一个特点。 

作品评价

西汉刘向《别录》:“《鲁论语》二十篇,皆孔子弟子记诸善言也。”
东汉班固《汉书·艺文志》:“《论语》者,孔子应答弟子、时人,及弟子相与言而接闻于夫子之语也。当时弟子各有所记,夫子既卒,门人相与辑而论纂,故谓之《论语》。”
东汉王充《论衡·正说篇》:“初,孔子孙孔安国以教鲁人扶卿,官至荆州刺史,始曰《论语》。”
东汉刘熙《释名·释典艺》:“《论语》,记孔子与弟子所语之言也。论,伦也,有伦理也。语,叙也,叙己所欲说也。”
西晋傅玄《傅子》:“昔仲尼既没,仲弓之徒追论夫子之言,谓之《论语》。” (南梁萧统 《文选·辨命论注》引)
北宋赵普:“臣平生所知,诚不出此,昔以其半辅太祖定天下,今欲以其半辅陛下致太平。”(罗大经《鹤林玉露》卷七)
北宋邢昺《邢疏》:“直言曰言,答述曰语,散则言语可通,故此论夫子之语而谓之善言也。”
南宋朱熹《朱子语类》卷一〇五:“‘四子’,‘六经’之阶梯;《近思录》,‘四子’之阶梯。”
南宋何异孙《十一经问对》:“《论语》有弟子记夫子之言者,有夫子答弟子问,有弟子自相答者,又有时人相言者,有臣对君问者,有师弟子对大夫之问者,皆所以讨论文义,故谓之《论语》。”
清代邵懿辰 《仪宋堂后记》:“明太祖既一海内,其佐刘基 ,以‘四子书’章义试士。行之五百年不改,以至于今。”
清代俞樾《春在堂随笔》卷九:“余撰《文勤神道碑》,即据其子儒卿等所撰行状,言公年十有一,‘四子书’、‘十三经’皆卒读。”
清代薛福成 《选举论中》:“常科以待天下占毕之士,试策论;论仍以‘四子’、‘ 五经’命题,特易其体格而已;策则参问古今事。” 

后世影响

《论语》是儒家经典之一。自汉武帝“罢黜百家,独尊儒术”之后,《论语》被尊为“五经之輨辖,六艺之喉衿”,是研究孔子及儒家思想尤其是原始儒家思想的第一手资料。南宋时朱熹将《大学》《中庸》《论语》《孟子》合为“四书”,使之在儒家经典中的地位日益提高。元代延祐年间,科举开始以“四书”开科取士。此后一直到清朝末年推行洋务运动,废除科举之前,《论语》一直是学子士人推施奉行的金科玉律。
《论语》进入经书之列是在唐代。“到唐代,礼有《周礼》《仪礼》《礼记》,春秋有《左传》《公羊》《谷梁》,加上《论语》《尔雅》《孝经》,这样是十三经。”北宋政治家赵普曾有“半部《论语》治天下”之说。这从一个侧面反映出此书在中国古代社会所发挥的作用与影响之大。
《论语》中保留了一些人们对孔子师徒的批评讽刺,有的作了辩驳,有的没有回答。其驳议辩难部分对后世很有影响,如《答客难》等设为主客问答进行辩难的小赋,都从《论语》受到启发;其自我解嘲部分,表现了儒家对自我价值的肯定,对“知其不可为而为之”的积极奋进精神的赞扬。

版本流传

各种版本

《论语》成书于战国初期。因秦始皇焚书坑儒(古时称为方术士,擅长祭祀,算命等),到西汉时期仅有口头传授及从孔子住宅夹壁中所得的本子,计有三种不同的本子:鲁人口头传授的《鲁论语》二十篇;齐人口头传授的《齐论语》二十二篇,其中二十篇的章句很多和《鲁论语》相同,但是多出《问王》和《知道》两篇;从孔子住宅夹壁中发现的《古文论语》(即《古论语》)二十一篇,也没有《问王》和《知道》两篇,但是把《尧曰篇》的“子张问”另分为一篇,于是有了两个《子张篇》,篇次也和《齐论》《鲁论》不一样,文字不同的计四百多字。

《鲁论语》和《齐论语》最初各有师传,到西汉末年,安昌侯张禹先学习了《鲁论语》,后来又讲习《齐论语》,于是把两个本子融合为一,但是篇目以《鲁论语》为根据,“采获所安”,另成一论,称为《张侯论》。张禹是汉成帝的师傅,其时极为尊贵,所以他的这一个本子便为当时一般儒生所尊奉,后汉灵帝时所刻的《熹平石经》就是用的《张侯论》。此本成为当时的权威读本,据《汉书·张禹传》记载:“诸儒为之语曰:‘欲为《论》,念张文。’由是学者多从张氏,馀家寝微。”《齐论语》《古论语》不久亡佚。东汉末郑玄又以“张侯论”为底本,参照《齐论》《古论》作《论语注》,遂为《论语》定本。

孔壁中书本《论语》由孔安国定。当孔安国向汉武帝献书时,正值“巫蛊事件”,朝廷将这批书退还给孔氏,“其学于是在孔家流传”。

《古文论语》是在汉景帝时由鲁恭王刘余在孔子旧宅壁中发现的,当时并没有传授。何晏《论语集解·序》说:“《古论》,唯博士孔安国为之训解,而世不传。”《论语集解》并经常引用了孔安国的《注》。但孔安国是否曾为《论语》作训解,《集解》中的孔安国说是否伪作,陈鳣的《论语古训·自序》已有怀疑,沈涛的《论语孔注辨伪》认为就是何晏自己的伪造品,丁晏的《论语孔注证伪》又认为出于王肃之手。

东汉末年,大学者郑玄以《张侯论》为依据,参照《齐论》《古论》,作了《论语注》。在残存的郑玄《论语注》中还可以窥见鲁、齐、古三种《论语》本子的异同,然而,今天所用的《论语》本子,基本上就是《张侯论》。张禹这个人实际上够不上说是一位“经师”,只是一个无耻的政客,附会王氏,保全富贵,当时便被斥为“佞臣”,所以崔述在《论语源流附考》中竟说:“《公山》《佛肸》两章安知非其有意采之以入《鲁论》为己解嘲乎?”但是,崔述的话纵然不为无理,而《论语》的篇章仍然不能说有后人所杜撰的东西在内,顶多只是说有掺杂着孔门弟子以及再传弟子之中的不同传说而已。

《论语》的版本之争实际上就是真伪之辩。1973年河北定县八角廊出土有竹简《论语》。2016年江西南昌西汉海昏侯刘贺墓出土了约五千枚竹简,考古人员在这些竹简中发现了失传已久的《论语·知道》篇,并初步断定属《论语》的《齐论》版本。

历代注本

《论语》与《孝经》都是汉初学习者必读之书,是汉人启蒙书的一种。自汉代以来,便有不少人注解《论语》。汉朝人所注《论语》,已亡佚殆尽,今日所残存的,以郑玄注为较多,因为敦煌和日本发现了一些唐写本的残卷,估计十存六七;其他各家,在何晏《论语集解》以后,就多半只存于《论语集解》中。《十三经注疏·论语注疏》就是用三国何晏的《集解》和宋人邢昺的《疏》。至于何晏、邢昺前后还有不少专注《论语》的书,可以参看清人朱彝尊《经义考》、纪昀等《四库全书总目提要》以及唐陆德明《经典释文序录》和吴检斋《疏证》。

两千年来,为《论语》作注释的书籍不胜枚举。据统计,历代研治《论语》的专著不下三千余种。可惜的是,这些古籍亡佚者居多。流传有序且影响较大的《论语》注释性著作有:一、汉郑玄《论语注》;二、魏何晏《论语集解》;三、梁皇侃《论语义疏》;四、宋朱熹《论语集注》;五、清刘宝楠《论语正义》;六、民国程树德《论语集释》。从文献学的角度来看,其中重要的有四部:一是《论语集解》,它是两汉、三国时期经学家研究《论语》的结晶;二是《论语义疏》,它囊括了魏晋南北朝时期玄学家对《论语》的发挥;三是《论语集注》,它是两宋时期理学家《论语》精义的荟粹;四是《论语正义》,集清代考据学《论语》研究成果之大成。这四部《论语》注释代表了《论语》研究的四个阶段,同时也代表了四种研究方法,是现代研究《论语》基本资料。

《论语》是孔子及其弟子的语录结集,由孔子弟子及再传弟子编写而成,并不是某一个人的著作。孔子开创了私人讲学的风气,相传他有弟子三千,贤弟子七十二人。孔子去世后,其弟子及再传弟子把孔子及其弟子的言行语录和思想记录下来,整理编成了儒家经典《论语》。


\mainmatter
\chapter{学而}

本篇内容涉及学习、为人和修养道德等方面,也有一些论政的语录。包括“学而时习”的学习方法,孝弟为本的仁学基础,不断反省的进德手段,节用爱人、使民以时的治国手段,先道德后文化的学习进程,“无友不如己者”的交友原则,过则能改的君子气度,“慎重追远”的行孝规定,“温良恭俭让”的行己作风,安贫乐道、敏行慎言的君子之德,推己及彼、举一反三的治学能力等。

\begin{yuanwen}
子\footnote{古代对男子的尊称。}曰\footnote{本书中“子曰”都是孔子所讲的话。}:“学而时习之,不亦说\footnote{同“悦”,高兴、愉快的意思。}乎?有朋自远方来,不亦乐乎?人不知而不愠\footnote{y\`un,怨恨。},不亦君子乎?
\end{yuanwen}

孔子说:“学了,然后按时实习,不也是很高兴的吗?有志同道合的人从远方来相会,不也是很快乐的吗?别人不了解自己,自己并不生气,不也是君子吗?”

\begin{yuanwen}
有子\footnote{孔子弟子。姓有,名若。《论语》中记载孔子弟子时一般称字,只对曾参和有若全部尊称为子,据此有很多人认为《论语》一书是曾参和有若的弟子记录而成的。}曰:“其为人也孝弟\footnote{t\`i,同“悌”,遵从兄长。},而好犯上者,鲜\footnote{xi\v{a}n,少。}矣!不好犯上,而好作乱者,未之有也。君子务\footnote{致力于。}本,本立而道生。孝弟也者,其为仁之本与\footnote{同“欤(y\'u)”,疑问语气词。}!”
\end{yuanwen}

有子说:“为人孝敬父母、尊敬兄长的,却喜欢冒犯上级,这种人很少。不喜欢冒犯上级,却喜欢造反作乱,这种人从来也没有过。君子致力于根本性工作,根本确立了,正道就随之产生。孝敬父母、尊敬兄长这些内容,大概就是施行‘仁’道的基础吧!”

\begin{yuanwen}
子曰:“巧言令色\footnote{好的脸色。这里指假装和善。},鲜矣仁!”
\end{yuanwen}

孔子说:“花言巧语、面貌伪善的人,仁德是很少的。”

\begin{yuanwen}
曾子\footnote{孔子弟子。姓曾,名参,字子舆。}曰:“吾日三省\footnote{x\v{i}ng。多次反省。古代汉语中动作性动词前加数字修饰成份的,一般表示动作的频率。而“三”、“九”等数字,一般表示次数多,不必落实为具体次数。}吾身:为人谋而不忠乎?与朋友交而不信乎?传\footnote{chu\'an,老师的传教。}不习乎?”
\end{yuanwen}

曾子说:“我每天多次自我反省:替别人谋划事情是否尽心竭力呢?与朋友交往是否诚实相待呢?老师传授的学业是否认真复习了?”

\begin{yuanwen}
子曰:“道\footnote{d\v{a}o,“导”的古体字,治理。}千乘\footnote{sh\`eng}\footnote{古代用四匹马拉的一辆兵车称为一乘。春秋战国时代,国力的强盛以该国所拥有的兵车的数量来计算。孔子生活之世,“千乘之国”已算不上是诸侯大国了,所以《论语》中有“千乘之国,摄乎大国之间”的话。}之国,敬事而信,节用而爱人,使民以时\footnote{按时,这里指不违背农时。}。”
\end{yuanwen}

孔子说:“治理拥有一千辆兵车的国家,就要严肃认真地对待工作,言而有信,节约用度,关爱百姓,不在农忙时节役使百姓。”

\begin{yuanwen}
子曰:“弟子入则孝,出则悌,谨而信,泛爱众,而亲仁。行有余力,则以学文。”
\end{yuanwen}

孔子说:“年轻人,在家就要孝顺父母,出门在外就要尊敬兄长,行为谨慎,言语有信,博爱众人,亲近仁者。这些都做到之后还有余力的话,就去学习文化。”

\begin{yuanwen}
子夏\footnote{孔子弟子。姓卜,名商,字子夏。孔子弟子中有所谓“四科十哲”之说,子夏长于“文学”。}曰:“贤贤易\footnote{轻视。}色\footnote{看重德行,轻视表面的姿态。},事父母,能竭其力,事君,能致\footnote{给予,献出。}其身,与朋友交,言而有信。虽曰未学,吾必谓之学矣。”
\end{yuanwen}

子夏说:“看重实际的德行,轻视表面的姿态。侍奉父母要竭尽全力,服务君主要奉献自身,与朋友交往要说话诚实有信。这样的人,虽说没有学习过,我也一定说他学习过了。”

\begin{yuanwen}
子曰:“君子不重则不威,学则不固\footnote{固执己见。}。主忠信\footnote{以下三句与前两句意思不相连贯,又见于其他篇内,疑是错简重出于此。},无友不如己者,过则勿惮改。”
\end{yuanwen}

孔子说:“称得上君子的人,如果不庄重就没有威严,知道学习就不会自以为是、顽固不化。恪守忠诚信实的道德要求,不与道德上不如自己的人交往,有了错误就不要怕改正。”

\begin{yuanwen}
曾子曰:“慎终\footnote{终与后边的远,分别指长辈丧亡之事和对于远祖的祭祀。},追远,民德归厚矣!”
\end{yuanwen}

曾子说:“恭敬慎重地办理父母的丧事,虔诚静穆地追祭历代的祖先,老百姓的道德就会趋向敦厚了。”

\begin{yuanwen}
子禽\footnote{陈亢,字子禽。从《子张》篇的记事来看,陈亢不是孔子的弟子,他对孔子的学说总是持怀疑的态度。}问于子贡\footnote{孔子弟子。姓端木,名赐。在“四科十哲”中属“言语”。}曰:“夫子\footnote{古人对于做过大夫的男子的敬称。孔子曾是鲁国的司寇(掌管刑狱的官员),所以他的学生称他为夫子,后来沿袭成对老师的称呼。在一定的场合下,又可以专指孔子。}至于是邦也,必闻其政。求之与?抑与之与?”

子贡曰:“夫子温、良、恭、俭\footnote{约束。}、让以得之。夫子之求之也,其诸\footnote{表示不肯定的推测语气。}异乎人之求之与!”
\end{yuanwen}

子禽问子贡说:“夫子每到一个国家,一定能够听到那个国家的政治状况,是求教得来的呢?还是人家主动告诉他的呢?”

子贡说:“先生温和、善良、恭敬、谨慎。谦让,是凭着这些德性得到的。先生求取的方法,大概不同于别人求取的办法吧!”

\begin{yuanwen}
子曰:“父在,观其\footnote{指代儿子。}志。父没\footnote{m\`o,死去。},观其行;三年无改于父之道,可谓孝矣。”
\end{yuanwen}

孔子说:“父亲在世的时候,要观察儿子的志向。父亲去世之后,要观察儿子的实际行动。如果能够多年不改变父亲传下来的正道的话,就可以说是尽孝了。”

\begin{yuanwen}
有子曰:“礼之用\footnote{施行。},和为贵。先王之道,斯\footnote{此,这。}为美,小大由之。有所不行,知和而和,不以礼节之,亦不可行也。”
\end{yuanwen}

有子说:“礼的施行,以和谐为美。前代君王的治道,最可贵的地方就在这里,大事小事都遵循这个道理。如果有行不通的地方,只是知道和谐为贵的道理而一味追求和谐,不懂得用礼来节制的道理的话,也是行不通的。”

\begin{yuanwen}
有子曰:“信近于义,言可复\footnote{因循,实践。}也。恭近于礼,远耻辱也。因\footnote{依靠,凭借。}不失其亲,亦可宗\footnote{尊重,推崇而效法。}也。”
\end{yuanwen}

有子说:“许下的诺言如果合乎义的话,这样的诺言就是可以遵循实践的。恭敬的样子如果合乎礼的话,就能够避开耻辱。依靠的人中不缺少关系深的,也就可靠了。”

\begin{yuanwen}
子曰:“君子食无求饱,居无求安,敏于事而慎于言,就\footnote{靠近。}有道而正\footnote{匡正。}焉。可谓好学也已。”
\end{yuanwen}

孔子说:“君子,吃饭不贪求满足,居住不贪求安逸,做事勤敏,说话谨慎,求教于有道德的人来端正自己,这样就可以说是好学的了。”

\begin{yuanwen}
子贡曰:“贫而无谄\footnote{ch\v{a}n,巴结,奉承。},富而无骄,何如?”

子曰:“可也。未若贫而乐,富而好礼者也。”

子贡曰:“《诗》云:‘如切如磋,如琢如磨\footnote{《诗经·卫风·淇奥》中的句子。切、磋、琢、磨都是制作器物时反复修治的动作,这里用来比喻治学、修身要精益求精。}’,其斯之谓与?”

子曰:“赐也,始可与言《诗》已矣!告诸往而知来者。”
\end{yuanwen}

子贡说:“贫穷却不谄媚,富有却不骄纵,人能做到这些怎么样?”

孔子说:“可以了。但是不如贫穷却能怡然自乐,富贵却能谦逊好礼。”

子贡说:“《诗经》里说:‘像制造器物一样,切割、磋治、雕琢、打磨’,大概就是说这类反复修治、精益求精的事吧!”

孔子说:“赐呀,可以和你讨论《诗经》了,告诉你一件事,就可以推知另一件事。”

\begin{yuanwen}
子曰:“不患人之不己知,患不知人也。”
\end{yuanwen}

孔子说:“不担心别人不了解自己,担心的是自己不了解别人。”

\chapter{为政}

本篇全都是孔子的语录。提及的人则有鲁国国君、鲁国大夫、孔子弟子等,据此可以了解孔子为众人师表的情况。

本篇论及为政、教化、学习、修养、孝道等方面的内容。孔子主张德政礼治:认为治政必须以教化百姓为首任,从政必须以学习为前提,对于有疑问之事采取谨慎的态度;国君要任用正直之人来辅政,当政者都要从修养自身做起,以使社会形成普遍的道德风气:友爱、孝悌、讲信用。还指出了教学科目的特点,概述了自己为学进德的经历,提倡学思并重的学习方法,反对研习具有极端倾向的学说。

对孝道的涵义做了集中阐释:能够按照礼的规定办事,无论是父母在世时的赡养义务,还是父母去世后的悼念程序,这样就是尽孝。不要违背礼的规定,不要让父母为自己担忧,父母亲所有的担心只出现在疾病这一非人力可以控制的范围内,这也是对父母孝顺的方式。孝敬父母突出在这个“敬”字上,这种感情史人类所特有的,要在日常与长者的交往中显示出来。虽然在孔子的时代,敬顺之情明显是受到等级制的影响而产生的,时至今日则完全可以用对于长者的尊敬来代替。

谈到考察人的品性要以行动为依据。

分析了君子的特点:多才多能、堪当重任、积极实践、言行一致。

总结了政治文化世代继承的特点。

\begin{yuanwen}
子曰:“为政以德,譬如北辰\footnote{北极星。《尔雅·释天》:“北极谓之北辰。”},居其所而众星共\footnote{通“拱”,环抱、环绕之意。这里是以北辰比喻统治者,以众星比喻被统治者。}之。”
\end{yuanwen}

孔子说:“当政者运用道德来治理国政,就好像北极星,安居其所,而其他众星井然有序地环绕着它。”

\begin{yuanwen}
子曰:“《诗》\footnote{《诗经》}三百\footnote{概举整数而言。《诗经》实有三百零五篇诗,连同有题无辞的六篇笙诗,共三百一十一篇。},一言以蔽之,曰:‘思无邪\footnote{《诗经·鲁颂·駉(ji\=ong)》中的句子,孔子借用来评价《诗经》各篇思想内容的纯正。}’。”
\end{yuanwen}

孔子说:“《诗》三百篇,用一句话来总括它,就是‘思想主旨纯正无邪’。”

\begin{yuanwen}
子曰:“道\footnote{同“导”,引导。}之以政\footnote{法制,禁令。},齐\footnote{整饬(ch\`i)。}之以刑,民免\footnote{逃避。}而无耻。道之以德,齐之以礼,有耻且格\footnote{至,来。}。”
\end{yuanwen}

孔子说:“用政令来训导百姓,用刑罚来整饬百姓,百姓只会尽量地避免获罪,却没有羞耻心;用道德来引导人民,用礼教来整饬人民,人民就会有羞耻心而且归顺。”

\begin{yuanwen}
子曰:“吾十有\footnote{通“又”。古人十五岁为入学之年,《礼记·王制》“立四教”。郑玄注引《尚书传》曰:“年十五始入小学,年十八入大学。”}五而志于学,三十而立\footnote{指立身行事。《论语》一书中多有以“礼”为立身行事基本原则的说法。},四十而不惑,五十而知天命\footnote{懂得天命不可抗拒而听天由命。},六十而耳顺,七十而从心所欲,不逾矩。”
\end{yuanwen}

孔子说:“我十五岁立志于学习;三十岁能依照礼仪的要求立足于世;四十岁不再感到困惑;五十岁能乐天知命;六十岁能听得进各种不同的意见;七十岁能随心所欲地行事,而又从不超出规矩。”

\begin{yuanwen}
孟懿子\footnote{鲁国大夫。姓仲孙,名何忌。“懿”是谥(sh\`i)号(死后所得的尊号)。}问孝。子曰:“无违。”

樊迟\footnote{孔子弟子。姓樊,名须,字子迟。}御,子告之曰:“孟孙问孝于我,我对曰,‘无违’。”

樊迟曰:“何谓也?”

子曰:“生,事之以礼;死,葬之以礼,祭之以礼。”
\end{yuanwen}

孟懿子文什么是孝。孔子说:“不要违背礼的规定。”

樊迟为孔子驾御马车,孔子告诉他说:“孟孙向我询问怎样才算是孝,我回答说,‘不要违背礼的规定’。”

樊迟说:“这话是什么意思?”

孔子说:“父母在世的时候,按照礼的要求来服侍他们;去世以后,按照礼的要求来安葬他们,按照礼的要求来祭祀他们。”

\begin{yuanwen}
孟武伯\footnote{孟懿子的儿子。姓仲孙,名彘。“武”是谥号。}问孝。子曰:“父母唯其\footnote{指代子女。}疾之忧。”
\end{yuanwen}

孟武伯问什么是孝。孔子说:“父母对于子女,只为他们的疾病担忧。”

\begin{yuanwen}
子游\footnote{孔子弟子。姓言,名偃,字子游,吴人。在“四科十哲”中属“文学”。}问孝。子曰:“今之孝者,是谓能养。至于\footnote{就连,就是。表示提起另一件事。}犬马,皆能有养。不敬,何以别乎?”
\end{yuanwen}

子游问什么是孝。孔子说:“如今所谓的孝,只是就能养活父母而言。说到狗、马这些动物,都能被人饲养。如果对父母没有敬顺的心意,用什么来区别孝顺和饲养呢?”

\begin{yuanwen}
子夏问孝。子曰:“色\footnote{指敬爱和悦的容色态度。}难。有事,弟子服其劳;有酒食,先生\footnote{年长者。}馔\footnote{zhu\`an,吃喝。}。曾\footnote{乃,竟。}是以为孝乎?”
\end{yuanwen}

子夏问什么是孝。孔子说:“保持敬爱和悦的容态最难。遇有事情,年轻人替长者们效劳;遇有酒食,让给长者享用,仅仅这样就算是孝了吗?”

\begin{yuanwen}
子曰:“吾与回\footnote{孔子弟子颜回。字子渊,鲁国人。在“四科十哲”中属“德行”,是孔子所喜爱的最聪慧最有修养的一个学生。}言,终日不违\footnote{不违拗。},如愚。退\footnote{指散学回去。}而省其私\footnote{独处。这里指独自钻研和自我实践。},亦足以发\footnote{发挥。},回也不愚。”
\end{yuanwen}

孔子说:“我给颜回讲学,他整天从不表示异议,像是一个愚笨的人。等回去之后,省察他的钻研和实践,又能发挥所学的内容,颜回并不愚笨啊!”

\begin{yuanwen}
子曰:“视其所以\footnote{作为,行动。},观其所由\footnote{经由,经历。},察其所安\footnote{习。},人焉\footnote{怎样。}廋\footnote{s\=ou,隐藏。}哉?人焉廋哉?”
\end{yuanwen}

孔子说:“注意看他的所作所为,观察他的一贯经历,考察他的秉性习惯,一个人怎么能隐藏得住呢?一个人怎么能隐藏得住呢?”

\begin{yuanwen}
子曰:“温故而知新,可以为师矣。”
\end{yuanwen}

孔子说:“温习旧的知识,而能在其中获得新的体会,这样的人可以做老师了。”

\begin{yuanwen}
子曰:“君子不器。”
\end{yuanwen}

孔子说:“君子不能像器皿一样只有单一的用途。”

\begin{yuanwen}
子贡问君子。子曰:“先行其言,而后从之。”
\end{yuanwen}

子贡问怎样才能算是君子。孔子说:“先实践所要说的话,然后再把话说出来。”

\begin{yuanwen}
子曰:“君子周\footnote{合。}而不比\footnote{齐同。},小人比而不周。”
\end{yuanwen}

孔子说:“君子团结而不勾结,小人勾结而不团结。”

\begin{yuanwen}
子曰:“学而不思则罔\footnote{w\v{a}ng,无知的样子。},思而不学则殆\footnote{d\`ai,疑惑。}。”
\end{yuanwen}

孔子说:“只是学习,却不思考,就会惘然无知。只是思考,却不学习,就会疑惑不解。”

\begin{yuanwen}
子曰:“攻\footnote{从事某事,进行某项工作。}乎异端\footnote{历来的注疏多释为错误的学说或危险思想,而与孔子本人的学说相对。实际上,汉以前的古书没有以“邪说”为“异端”的记载。另外,《论语》中“异”字凡八见,多数情况下可释为“不同的”,因此此处的“异”作“不同的”解为佳。端,顶头,极。所以“异端”应该相当于“我叩其两端而竭焉”中的“两端”,也就是“过犹不及”中的“过”与“不及”这两端。},斯害也已\footnote{语气词连用,表示肯定。}。”
\end{yuanwen}

孔子说:“攻治两极的学说,这是一种祸害啊!”

\begin{yuanwen}
子曰:“由\footnote{仲由,孔子弟子,字子路,卞(bi\`an)人(今山东人)。在“四科十哲”中属“政事”。},诲女\footnote{通“汝”,第二人称代词,你。}知\footnote{同“智”。}之乎?知之为知之,不知为不知,是知也。”
\end{yuanwen}

孔子说:“由,教导你的内容都知道了吧?知道就是知道,不知道就是不知道,这才是有智慧。”

\begin{yuanwen}
子张\footnote{颛孙师,孔子弟子,字子张。}学干\footnote{求。}禄\footnote{官俸。}。子曰:“多闻阙疑,慎言其余,则寡尤\footnote{过失。};多见阙殆,慎行其余,则寡悔。言寡尤,行寡悔,禄在其中矣。”
\end{yuanwen}

子张向孔子学习求仕的方法。孔子说:“多聆听,对于有疑问的地方保留不言,其余有把握的地方,谨慎的发表意见,这样就可以少犯错。多观察,对于有疑问的地方保留不言,其余有把握的地方,谨慎地采取行动,这样就可以少后悔。言语方面少犯错误,行动方面避免后悔,官职俸禄就在这里面了。”

\begin{yuanwen}
哀公\footnote{鲁国的国君。姓姬,名蒋,公元前494-前466年在位。“哀”是谥号。}问曰:“何为则民服?”

孔子对曰:“举直错\footnote{放置。}诸枉\footnote{邪曲不正。},则民服;举枉错诸直,则民不服。”
\end{yuanwen}

鲁哀公问道:“怎么做才能使人民服从呢?”

孔子回答说:“选用正直的人,让他们居于邪曲之人的上位,这样百姓就会服从了。如果选用邪曲之人,让他们居于正直之人的上位,百姓就不会服从。”

\begin{yuanwen}
季康子\footnote{季孙肥,鲁哀公时的正卿,是当时最有权力的政治人物。“康”是谥号。}问:“使民敬、忠以\footnote{连词,和。}劝\footnote{勤勉。},如之何?”

子曰:“临之以庄则敬;孝慈则忠;举善而教不能则劝。”
\end{yuanwen}

季康子问道:“要使人民敬顺、忠诚又勤勉,应该怎么做呢?”

孔子说:“。”

\begin{yuanwen}
或\footnote{text}谓孔子曰:“子奚\footnote{text}不为政。”子曰:“《书》云:‘孝乎惟孝,友于兄弟,施\footnote{text}于有政。’是亦为政,奚其为为政?”
\end{yuanwen}

孔子说:“。”

\begin{yuanwen}
子曰:“人而\footnote{text}无信,不知其可也。大车无輗\footnote{text},小车无軏\footnote{text},其何以行之哉?”
\end{yuanwen}

孔子说:“。”

\begin{yuanwen}
子张问:“十世可知也?”子曰:“殷因\footnote{text}于夏礼,所损益可知也;周因于殷礼,所损益可知也。其或继周者,虽\footnote{text}百世可知也。”
\end{yuanwen}

孔子说:“。”

\begin{yuanwen}
子曰:“非其鬼\footnote{text}而祭之,谄也。见义不为,无勇也。”
\end{yuanwen}

孔子说:“。”

\chapter{八佾}

\begin{yuanwen}
孔子谓季氏:“八佾舞于庭,是可忍也,孰不可忍也?”
\end{yuanwen}

孔子说:“。”

\begin{yuanwen}
三家者以《雍》彻,子曰:‘相维辟公,天子穆穆’,奚取于三家之堂?”
\end{yuanwen}

孔子说:“。”

\begin{yuanwen}
子曰:“人而不仁,如礼何?人而不仁,如乐何?”
\end{yuanwen}

孔子说:“。”

\begin{yuanwen}
林放问礼之本,子曰:“大哉问!礼,与其奢也,宁俭;丧,与其易也,宁戚。”
\end{yuanwen}

孔子说:“。”

\begin{yuanwen}
子曰:“夷狄之有君,不如诸夏之亡也。”
\end{yuanwen}

孔子说:“。”

\begin{yuanwen}
季氏旅于泰山。子谓冉有曰:“女弗能救与?”对曰:“不能。”子曰:“呜呼!曾谓泰山不如林放乎?”
\end{yuanwen}

孔子说:“。”

\begin{yuanwen}
子曰:“君子无所争,必也射乎!揖让而升,下而饮。其争也君子。”
\end{yuanwen}

孔子说:“。”

\begin{yuanwen}
子夏问曰:“‘巧笑倩兮,美目盼兮,素以为绚兮’何谓也?”子曰:“绘事后素。”曰:“礼后乎?”子曰:“起予者商也,始可与言《诗》已矣。”
\end{yuanwen}

孔子说:“。”

\begin{yuanwen}
子曰:“夏礼,吾能言之,杞不足征也;殷礼吾能言之,宋不足征也。文献不足故也,足则吾能征之矣。”
\end{yuanwen}

孔子说:“。”

\begin{yuanwen}
子曰:“禘自既灌而往者,吾不欲观之矣。”
\end{yuanwen}

孔子说:“。”

\begin{yuanwen}
或问禘之说。子曰:“不知也。知其说者之于天下也,其如示诸斯乎!”指其掌。
\end{yuanwen}

孔子说:“。”

\begin{yuanwen}
祭如在,祭神如神在。子曰:“吾不与祭,如不祭。”
\end{yuanwen}

孔子说:“。”

\begin{yuanwen}
王孙贾问曰:“‘与其媚于奥,宁媚于灶’,何谓也?”子曰:“不然,获罪于天,无所祷也。”
\end{yuanwen}

孔子说:“。”

\begin{yuanwen}
子曰:“周监于二代,郁郁乎文哉!吾从周。”
\end{yuanwen}

孔子说:“。”

\begin{yuanwen}
子入太庙,每事问。或曰:“孰谓鄹人之子知礼乎?入太庙,每事问。”子闻之,曰:“是礼也。”
\end{yuanwen}

孔子说:“。”

\begin{yuanwen}
子曰:“射不主皮,为力不同科,古之道也。”
\end{yuanwen}

孔子说:“。”

\begin{yuanwen}
子贡欲去告朔之饩羊,子曰:“赐也!尔爱其羊,我爱其礼。”
\end{yuanwen}

孔子说:“。”

\begin{yuanwen}
子曰:“事君尽礼,人以为谄也。”
\end{yuanwen}

孔子说:“。”

\begin{yuanwen}
定公问:“君使臣,臣事君,如之何?”孔子对曰:“君使臣以礼,臣事君以忠。”
\end{yuanwen}

孔子说:“。”

\begin{yuanwen}
子曰:“《关雎》,乐而不淫,哀而不伤。”
\end{yuanwen}

孔子说:“。”

\begin{yuanwen}
哀公问社于宰我。宰我对曰:“夏后氏以松,殷人以柏,周人以栗,曰:使民战栗。”子闻之,曰:“成事不说,遂事不谏,既往不咎。”
\end{yuanwen}

孔子说:“。”

\begin{yuanwen}
子曰:“管仲之器小哉!”或曰:“管仲俭乎?”曰:“管氏有三归,官事不摄,焉得俭?”“然则管仲知礼乎?”曰:“邦君树塞门,管氏亦树塞门;邦君为两君
之好,有反坫。管氏亦有反坫,管氏而知礼,孰不知礼?”
\end{yuanwen}

孔子说:“。”

\begin{yuanwen}
子语鲁大师乐,曰:“乐其可知也。始作,翕如也;从之,纯如也,皦如也,绎如也,以成。”
\end{yuanwen}

孔子说:“。”

\begin{yuanwen}
仪封人请见,曰:“君子之至于斯也,吾未尝不得见也。”从者见之。出曰:“二三子何患于丧乎?天下之无道也久矣,天将以夫子为木铎。”
\end{yuanwen}

孔子说:“。”

\begin{yuanwen}
子谓《韶》:“尽美矣,又尽善也。”谓《武》:“尽美矣,未尽善也。”
\end{yuanwen}

孔子说:“。”

\begin{yuanwen}
子曰:“居上不宽,为礼不敬,临丧不哀,吾何以观之哉!”
\end{yuanwen}

孔子说:“。”

\chapter{里仁}

\begin{yuanwen}
子曰:“里仁为美。择不处仁,焉得知?”
\end{yuanwen}

孔子说:“。”

\begin{yuanwen}
子曰:“不仁者不可以久处约,不可以长处乐。仁者安仁,知者利仁。”
\end{yuanwen}

\begin{yuanwen}
子曰:“唯仁者能好人,能恶人。”
\end{yuanwen}

孔子说:“。”

\begin{yuanwen}
子曰:“苟志於仁矣,无恶也。”
\end{yuanwen}

孔子说:“。”

\begin{yuanwen}
子曰:“富与贵,是人之所欲也;不以其道得之,不处也。贫与贱,是人之所恶也;不以其道得之,不去也。君子去仁,恶乎成名?君子无终食之间违仁,造
次必于是,颠沛必于是。”
\end{yuanwen}

孔子说:“。”

\begin{yuanwen}
子曰:“我未见好仁者,恶不仁者。好仁者,无以尚之;恶不仁者,其为仁矣,不使不仁者加乎其身。有能一日用其力于仁矣乎?我未见力不足者。盖有之矣,我未见也。”
\end{yuanwen}

孔子说:“。”

\begin{yuanwen}
子曰:“人之过也,各于其党。观过,斯知仁矣。”
\end{yuanwen}

\begin{yuanwen}
子曰:“朝闻道,夕死可矣。”
\end{yuanwen}

\begin{yuanwen}
子曰:“士志于道,而耻恶衣恶食者,未足与议也。”
\end{yuanwen}

\begin{yuanwen}
子曰:“君子之于天下也,无适也,无莫也,义之与比。”
\end{yuanwen}

\begin{yuanwen}
子曰:“君子怀德,小人怀土;君子怀刑,小人怀惠。”
\end{yuanwen}

\begin{yuanwen}
子曰:“放于利而行,多怨。”
\end{yuanwen}

\begin{yuanwen}
子曰:“能以礼让为国乎?何有?不能以礼让为国,如礼何?”
\end{yuanwen}

\begin{yuanwen}
子曰:“不患无位,患所以立。不患莫己知,求为可知也。”
\end{yuanwen}

\begin{yuanwen}
子曰:“参乎!吾道一以贯之。”曾子曰:“唯。”子出,门人问曰:“何谓也?”曾子曰:“夫子之道,忠恕而已矣。”
\end{yuanwen}

\begin{yuanwen}
子曰:“君子喻于义,小人喻于利。”
\end{yuanwen}

\begin{yuanwen}
子曰:“见贤思齐焉,见不贤而内自省也。”
\end{yuanwen}

\begin{yuanwen}
子曰:“事父母几谏,见志不从,又敬不违,劳而不怨。”
\end{yuanwen}

\begin{yuanwen}
子曰:“父母在,不远游,游必有方。”
\end{yuanwen}

\begin{yuanwen}
子曰:“三年无改于父之道,可谓孝矣。
\end{yuanwen}

\begin{yuanwen}
子曰:“父母之年,不可不知也。一则以喜,一则以惧。
\end{yuanwen}

\begin{yuanwen}
子曰:“古者言之不出,耻躬之不逮也。
\end{yuanwen}

\begin{yuanwen}
子曰:“以约失之者鲜矣。
\end{yuanwen}

\begin{yuanwen}
子曰:“君子欲讷于言而敏于行。
\end{yuanwen}

\begin{yuanwen}
子曰:“德不孤,必有邻。”
\end{yuanwen}

\begin{yuanwen}
子游曰:“事君数,斯辱矣;朋友数,斯疏矣。
\end{yuanwen}


\chapter{公冶长}

\begin{yuanwen}
子谓公冶长:“可妻也,虽在缧绁之中,非其罪也!”以其子妻之。
\end{yuanwen}

\begin{yuanwen}
子谓南容:“邦有道不废;邦无道免于刑戮。”以其兄之子妻之。
\end{yuanwen}

\begin{yuanwen}
子谓子贱:“君子哉若人!鲁无君子者,斯焉取斯?”
\end{yuanwen}

\begin{yuanwen}
子贡问曰:“赐也何如?”子曰:“女,器也。”曰:“何器也?”曰:“瑚琏也。”
\end{yuanwen}

\begin{yuanwen}
或曰:“雍也仁而不佞。”子曰:“焉用佞?御人以口给,屡憎于人。不知其仁,焉用佞?”
\end{yuanwen}

\begin{yuanwen}
子使漆雕开仕,对曰:“吾斯之未能信。”子说。
\end{yuanwen}

\begin{yuanwen}
子曰:“道不行,乘桴浮于海,从我者其由与?”子路闻之喜,子曰:“由也好勇过我,无所取材。”
\end{yuanwen}

\begin{yuanwen}
孟武伯问:“子路仁乎?”子曰:“不知也。”又问,子曰:“由也,千乘之国,可使治其赋也,不知其仁也。”“求也何如?”子曰:“求也,千室之邑、百乘之家
,可使为之宰也,不知其仁也。”“赤也何如?”子曰:“赤也,束带立于朝,可使与宾客言也,不知其仁也。”
\end{yuanwen}

\begin{yuanwen}
子谓子贡曰:“女与回也孰愈?”对曰:“赐也何敢望回?回也闻一以知十,赐也闻一以知二。”子曰:“弗如也,吾与女弗如也!”
\end{yuanwen}

\begin{yuanwen}
宰予昼寝,子曰:“朽木不可雕也,粪土之墙不可杇也,于予与何诛?”子曰:“始吾于人也,听其言而信其行;今吾于人也,听其言而观其行。于予与改是
。”
\end{yuanwen}

\begin{yuanwen}
子曰:“吾未见刚者。”或对曰:“申枨。”子曰:“枨也欲,焉得刚。”
\end{yuanwen}

\begin{yuanwen}
子贡曰:“我不欲人之加诸我也,吾亦欲无加诸人。”子曰:“赐也,非尔所及也。”
\end{yuanwen}

\begin{yuanwen}
子贡曰:“夫子之文章,可得而闻也;夫子之言性与天道,不可得而闻也。”
\end{yuanwen}

\begin{yuanwen}
子路有闻,未之能行,唯恐有闻。
\end{yuanwen}

\begin{yuanwen}
子贡问曰:“孔文子何以谓之‘文’也?”子曰:“敏而好学,不耻下问,是以谓之‘文’也。”
\end{yuanwen}

\begin{yuanwen}
子谓子产:“有君子之道四焉:其行己也恭,其事上也敬,其养民也惠,其使民也义。”
\end{yuanwen}

\begin{yuanwen}
子曰:“晏平仲善与人交,久而敬之。”
\end{yuanwen}

\begin{yuanwen}
子曰:“臧文仲居蔡,山节藻棁,何如其知也?”
\end{yuanwen}

\begin{yuanwen}
子张问曰:“令尹子文三仕为令尹,无喜色,三已之无愠色,旧令尹之政必以告新令尹,何如?”子曰:“忠矣。”曰:“仁矣乎?”曰:“未知,焉得仁?”“崔
子弑齐君,陈文子有马十乘,弃而违之。至于他邦,则曰:‘犹吾大夫崔子也。’违之。之一邦,则又曰:‘犹吾大夫崔子也。’违之,何如?”子曰:“清矣。”曰
:“仁矣乎?”曰:“未知,焉得仁?”
\end{yuanwen}

\begin{yuanwen}
季文子三思而后行,子闻之曰:“再斯可矣。”
\end{yuanwen}

\begin{yuanwen}
子曰:“宁武子,邦有道,则知;邦无道,则愚。其知可及也,其愚不可及也。”
\end{yuanwen}

\begin{yuanwen}
子在陈,曰:“归与!归与!吾党之小子狂简,斐然成章,不知所以裁之。”
\end{yuanwen}

\begin{yuanwen}
子曰:“伯夷、叔齐不念旧恶,怨是用希。”
\end{yuanwen}

\begin{yuanwen}
子曰:“孰谓微生高直?或乞醯焉,乞诸其邻而与之。”
\end{yuanwen}

\begin{yuanwen}
子曰:“巧言、令色、足恭,左丘明耻之,丘亦耻之。匿怨而友其人,左丘明耻之,丘亦耻之。”
\end{yuanwen}

\begin{yuanwen}
颜渊、季路侍,子曰:“盍各言尔志?”子路曰:“愿车马、衣轻裘与朋友共,敝之而无憾。”颜渊曰:“愿无伐善,无施劳。”子路曰:“愿闻子之志。”子曰:
“老者安之,朋友信之,少者怀之。”
\end{yuanwen}

\begin{yuanwen}
子曰:“已矣乎!吾未见能见其过而内自讼者也。”
\end{yuanwen}

\begin{yuanwen}
子曰:“十室之邑,必有忠信如丘者焉,不如丘之好学也。”
\end{yuanwen}



\chapter{雍也}

\begin{yuanwen}
子曰:“雍也可使南面。”
\end{yuanwen}

\begin{yuanwen}
仲弓问子桑伯子,子曰:“可也简。”仲弓曰:“居敬而行简,以临其民,不亦可乎?居简而行简,无乃大简乎?”子曰:“雍之言然。”
\end{yuanwen}

\begin{yuanwen}
哀公问:“弟子孰为好学?”孔子对曰:“有颜回者好学,不迁怒,不贰过,不幸短命死矣,今也则亡,未闻好学者也。”
\end{yuanwen}

\begin{yuanwen}
子华使于齐,冉子为其母请粟,子曰:“与之釜。”请益,曰:“与之庾。”冉子与之粟五秉。子曰:“赤之适齐也,乘肥马,衣轻裘。吾闻之也,君子周急不继
富。”\end{yuanwen}

\begin{yuanwen}

原思为之宰,与之粟九百,辞。子曰:“毋以与尔邻里乡党乎!”
\end{yuanwen}

\begin{yuanwen}
子谓仲弓曰:“犁牛之子骍且角,虽欲勿用,山川其舍诸?”
\end{yuanwen}

\begin{yuanwen}
子曰:“回也,其心三月不违仁,其余则日月至焉而已矣。”
\end{yuanwen}

\begin{yuanwen}
季康子问:“仲由可使从政也与?”子曰:“由也果,于从政乎何有?”曰:“赐也可使从政也与?”曰:“赐也达,于从政乎何有?”曰:“求也可使从政也与?”曰:“求也艺,于从政乎何有?”
\end{yuanwen}

\begin{yuanwen}
季氏使闵子骞为费宰,闵子骞曰:“善为我辞焉。如有复我者,则吾必在汶上矣。”
\end{yuanwen}

\begin{yuanwen}
伯牛有疾,子问之,自牖执其手,曰:“亡之,命矣夫!斯人也而有斯疾也!斯人也而有斯疾也!”
\end{yuanwen}

\begin{yuanwen}
子曰:“贤哉回也!一箪食,一瓢饮,在陋巷,人不堪其忧,回也不改其乐。贤哉,回也!”
\end{yuanwen}

\begin{yuanwen}
冉求曰:“非不说子之道,力不足也。”子曰:“力不足者,中道而废,今女画。”
\end{yuanwen}

\begin{yuanwen}
子谓子夏曰:“女为君子儒,毋为小人儒。”
\end{yuanwen}

\begin{yuanwen}
子游为武城宰,子曰:“女得人焉尔乎?”曰:“有澹台灭明者,行不由径,非公事,未尝至于偃之室也。”
\end{yuanwen}

\begin{yuanwen}
子曰:“孟之反不伐,奔而殿,将入门,策其马曰:‘非敢后也,马不进也。’”
\end{yuanwen}

\begin{yuanwen}
子曰:“不有祝鮀之佞,而有宋朝之美,难乎免于今之世矣。”
\end{yuanwen}

\begin{yuanwen}
子曰:“谁能出不由户?何莫由斯道也?”
\end{yuanwen}

\begin{yuanwen}
子曰:“质胜文则野,文胜质则史。文质彬彬,然后君子。”
\end{yuanwen}

\begin{yuanwen}
子曰:“人之生也直,罔之生也幸而免。”
\end{yuanwen}

\begin{yuanwen}
子曰:“知之者不如好之者;好之者不如乐之者。”
\end{yuanwen}

\begin{yuanwen}
子曰:“中人以上,可以语上也;中人以下,不可以语上也。”
\end{yuanwen}

\begin{yuanwen}
樊迟问知,子曰:“务民之义,敬鬼神而远之,可谓知矣。”问仁,曰:“仁者先难而后获,可谓仁矣。”
\end{yuanwen}

\begin{yuanwen}
子曰:“知者乐水,仁者乐山。知者动,仁者静。知者乐,仁者寿。”
\end{yuanwen}

\begin{yuanwen}
子曰:“齐一变至于鲁,鲁一变至于道。”
\end{yuanwen}

\begin{yuanwen}
子曰:“觚不觚,觚哉!觚哉!”
\end{yuanwen}

\begin{yuanwen}
宰我问曰:“仁者,虽告之曰:‘井有仁焉。’其从之也?”子曰:“何为其然也?君子可逝也,不可陷也;可欺也,不可罔也。”
\end{yuanwen}

\begin{yuanwen}
子曰:“君子博学于文,约之以礼,亦可以弗畔矣夫。”
\end{yuanwen}

\begin{yuanwen}
子见南子,子路不说,夫子矢之曰:“予所否者,天厌之!天厌之!”
\end{yuanwen}

\begin{yuanwen}
子曰:“中庸之为德也,其至矣乎!民鲜久矣。”
\end{yuanwen}

\begin{yuanwen}
子贡曰:“如有博施于民而能济众,何如?可谓仁乎?”子曰:“何事于仁,必也圣乎!尧、舜其犹病诸!夫仁者,己欲立而立人,己欲达而达人。能近取譬
,可谓仁之方也已。”
\end{yuanwen}

\begin{yuanwen}
\chapter{述而}
子曰:“述而不作,信而好古,窃比于我老彭。”
\end{yuanwen}

\begin{yuanwen}
子曰:“默而识之,学而不厌,诲人不倦,何有于我哉?”
\end{yuanwen}

\begin{yuanwen}
子曰:“德之不修,学之不讲,闻义不能徙,不善不能改,是吾忧也。”
\end{yuanwen}

\begin{yuanwen}
子之燕居,申申如也,夭夭如也。
\end{yuanwen}

\begin{yuanwen}
子曰:“甚矣,吾衰也!久矣,吾不复梦见周公。”
\end{yuanwen}

\begin{yuanwen}
子曰:“志于道,据于德,依于仁,游于艺。”
\end{yuanwen}

\begin{yuanwen}
子曰:“自行束脩以上,吾未尝无诲焉。”
\end{yuanwen}

\begin{yuanwen}
子曰:“不愤不启,不悱不发,举一隅不以三隅反,则不复也。”
\end{yuanwen}

\begin{yuanwen}
子食于有丧者之侧,未尝饱也。
\end{yuanwen}

\begin{yuanwen}
子于是日哭,则不歌。
\end{yuanwen}

\begin{yuanwen}
子谓颜渊曰:“用之则行,舍之则藏,惟我与尔有是夫!”子路曰:“子行三军,则谁与?”子曰:“暴虎冯河,死而无悔者,吾不与也。必也临事而惧,好谋而成者也。”
\end{yuanwen}

\begin{yuanwen}
子曰:“富而可求也,虽执鞭之士,吾亦为之。如不可求,从吾所好。”
\end{yuanwen}

\begin{yuanwen}
子之所慎:齐,战,疾。
\end{yuanwen}

\begin{yuanwen}
子在齐闻《韶》,三月不知肉味,曰:“不图为乐之至于斯也。”
\end{yuanwen}

\begin{yuanwen}
冉有曰:“夫子为卫君乎?”子贡曰:“诺,吾将问之。”入,曰:“伯夷、叔齐何人也?”曰:“古之贤人也。”曰:“怨乎?”曰:“求仁而得仁,又何怨?”出,曰:“夫子不为也。”
\end{yuanwen}

\begin{yuanwen}
子曰:“饭疏食饮水,曲肱而枕之,乐亦在其中矣。不义而富且贵,于我如浮云。”
\end{yuanwen}

\begin{yuanwen}
子曰:“加我数年,五十以学《易》,可以无大过矣。”
\end{yuanwen}

\begin{yuanwen}
子所雅言,《诗》、《书》、执礼,皆雅言也。
\end{yuanwen}

\begin{yuanwen}
叶公问孔子于子路,子路不对。子曰:“女奚不曰:其为人也,发愤忘食,乐以忘忧,不知老之将至云尔。”
\end{yuanwen}

\begin{yuanwen}
子曰:“我非生而知之者,好古,敏以求之者也。”
\end{yuanwen}

\begin{yuanwen}
子不语:怪、力、乱、神。
\end{yuanwen}

\begin{yuanwen}
子曰:“三人行,必有我师焉。择其善者而从之,其不善者而改之。”
\end{yuanwen}

\begin{yuanwen}
子曰:“天生德于予,桓魋其如予何?”
\end{yuanwen}

\begin{yuanwen}
子曰:“二三子以我为隐乎?吾无隐乎尔!吾无行而不与二三子者,是丘也。”
\end{yuanwen}

\begin{yuanwen}
子以四教:文,行,忠,信。
\end{yuanwen}

\begin{yuanwen}
子曰:“圣人,吾不得而见之矣;得见君子者,斯可矣。”子曰:“善人,吾不得而见之矣,得见有恒者斯可矣。亡而为有,虚而为盈,约而为泰,难乎有恒
乎。”
\end{yuanwen}

\begin{yuanwen}
子钓而不纲,弋不射宿。
\end{yuanwen}

\begin{yuanwen}
子曰:“盖有不知而作之者,我无是也。多闻,择其善者而从之;多见而识之,知之次也。”
\end{yuanwen}

\begin{yuanwen}
互乡难与言,童子见,门人惑。子曰:“与其进也,不与其退也,唯何甚?人洁己以进,与其洁也,不保其往也。”
\end{yuanwen}

\begin{yuanwen}
子曰:“仁远乎哉?我欲仁,斯仁至矣。”
\end{yuanwen}

\begin{yuanwen}
陈司败问:“昭公知礼乎?”孔子曰:“知礼。”孔子退,揖巫马期而进之,曰:“吾闻君子不党,君子亦党乎?君取于吴,为同姓,谓之吴孟子。君而知礼,
孰不知礼?”巫马期以告,子曰:“丘也幸,苟有过,人必知之。”
\end{yuanwen}

\begin{yuanwen}
子与人歌而善,必使反之,而后和之。
\end{yuanwen}

\begin{yuanwen}
子曰:“文,莫吾犹人也。躬行君子,则吾未之有得。”
\end{yuanwen}

\begin{yuanwen}
子曰:“若圣与仁,则吾岂敢?抑为之不厌,诲人不倦,则可谓云尔已矣。”公西华曰:“正唯弟子不能学也。”
\end{yuanwen}

\begin{yuanwen}
子疾病,子路请祷。子曰:“有诸?”子路对曰:“有之。《诔》曰:‘祷尔于上下神祇。’”子曰:“丘之祷久矣。”
\end{yuanwen}

\begin{yuanwen}
子曰:“奢则不孙,俭则固。与其不孙也,宁固。”
\end{yuanwen}

\begin{yuanwen}
子曰:“君子坦荡荡,小人长戚戚。”
\end{yuanwen}

\begin{yuanwen}
子温而厉,威而不猛,恭而安。
\end{yuanwen}


\chapter{泰伯}

\begin{yuanwen}
子曰:“泰伯,其可谓至德也已矣。三以天下让,民无得而称焉。”
\end{yuanwen}

\begin{yuanwen}
子曰:“恭而无礼则劳;慎而无礼则葸;勇而无礼则乱;直而无礼则绞。君子笃于亲,则民兴于仁;故旧不遗,则民不偷。”
\end{yuanwen}

\begin{yuanwen}
曾子有疾,召门弟子曰:“启予足,启予手。《诗》云:‘战战兢兢,如临深渊,如履薄冰。’而今而后,吾知免夫,小子!”
\end{yuanwen}

\begin{yuanwen}
曾子有疾,孟敬子问之。曾子言曰:“鸟之将死,其鸣也哀;人之将死,其言也善。君子所贵乎道者三:动容貌,斯远暴慢矣;正颜色,斯近信矣;出辞气,
斯远鄙倍矣。笾豆之事,则有司存。”
\end{yuanwen}

\begin{yuanwen}
曾子曰:“以能问于不能;以多问于寡;有若无,实若虚,犯而不校。昔者吾友尝从事于斯矣。”
\end{yuanwen}

\begin{yuanwen}
曾子曰:“可以托六尺之孤,可以寄百里之命,临大节而不可夺也。君子人与?君子人也。”
\end{yuanwen}

\begin{yuanwen}
曾子曰:“士不可以不弘毅,任重而道远。仁以为己任,不亦重乎?死而后已,不亦远乎?”
\end{yuanwen}

\begin{yuanwen}
子曰:“兴于《诗》,立于礼,成于乐。”
\end{yuanwen}

\begin{yuanwen}
子曰:“民可使由之,不可使知之。”
\end{yuanwen}

\begin{yuanwen}
子曰:“好勇疾贫,乱也。人而不仁,疾之已甚,乱也。”
\end{yuanwen}

\begin{yuanwen}
子曰:“如有周公之才之美,使骄且吝,其余不足观也已。”
\end{yuanwen}

\begin{yuanwen}
子曰:“三年学,不至于谷,不易得也。”
\end{yuanwen}

\begin{yuanwen}
子曰:“笃信好学,守死善道。危邦不入,乱邦不居。天下有道则见,无道则隐。邦有道,贫且贱焉,耻也;邦无道,富且贵焉,耻也。”
\end{yuanwen}

\begin{yuanwen}
子曰:“不在其位,不谋其政。”
\end{yuanwen}

\begin{yuanwen}
子曰:“师挚之始,《关雎》之乱,洋洋乎盈耳哉!”
\end{yuanwen}

\begin{yuanwen}
子曰:“狂而不直,侗而不愿,悾悾而不信,吾不知之矣。”
\end{yuanwen}

\begin{yuanwen}
子曰:“学如不及,犹恐失之。”
\end{yuanwen}

\begin{yuanwen}
子曰:“巍巍乎!舜、禹之有天下也而不与焉。”
\end{yuanwen}

\begin{yuanwen}
子曰:“大哉尧之为君也!巍巍乎,唯天为大,唯尧则之。荡荡乎,民无能名焉。巍巍乎其有成功也,焕乎其有文章!”
\end{yuanwen}

\begin{yuanwen}
舜有臣五人而天下治。武王曰:“予有乱臣十人。”孔子曰:“才难,不其然乎?唐虞之际,于斯为盛;有妇人焉,九人而已。三分天下有其二,以服事殷。周之德,其可谓至德也已矣。”
\end{yuanwen}

\begin{yuanwen}
子曰:“禹,吾无间然矣。菲饮食,而致孝乎鬼神;恶衣服,而致美乎黻冕;卑宫室,而尽力乎沟洫。禹,吾无间然矣!”
\end{yuanwen}

\begin{yuanwen}
\chapter{子罕}
\end{yuanwen}

\begin{yuanwen}
子罕言利与命与仁。
\end{yuanwen}

\begin{yuanwen}
达巷党人曰:“大哉孔子!博学而无所成名。”子闻之,谓门弟子曰:“吾何执?执御乎,执射乎?吾执御矣。”
\end{yuanwen}

\begin{yuanwen}
子曰:“麻冕,礼也;今也纯,俭,吾从众。拜下,礼也;今拜乎上,泰也;虽违众,吾从下。”
\end{yuanwen}

\begin{yuanwen}
子绝四:毋意、毋必、毋固、毋我。
\end{yuanwen}

\begin{yuanwen}
子畏于匡,曰:“文王既没,文不在兹乎?天之将丧斯文也,后死者不得与于斯文也;天之未丧斯文也,匡人其如予何?”
\end{yuanwen}

\begin{yuanwen}
太宰问于子贡曰:“夫子圣者与,何其多能也?”子贡曰:“固天纵之将圣,又多能也。”子闻之,曰:“太宰知我乎?吾少也贱,故多能鄙事。君子多乎哉?不多也。”
\end{yuanwen}

\begin{yuanwen}
牢曰:“子云:‘吾不试,故艺。’”
\end{yuanwen}

\begin{yuanwen}
子曰:“吾有知乎哉?无知也。有鄙夫问于我,空空如也。我叩其两端而竭焉。”
\end{yuanwen}

\begin{yuanwen}
子曰:“凤鸟不至,河不出图,吾已矣夫!”
\end{yuanwen}

\begin{yuanwen}
子见齐衰者、冕衣裳者与瞽者,见之,虽少,必作,过之必趋。
\end{yuanwen}

\begin{yuanwen}
颜渊喟然叹曰:“仰之弥高,钻之弥坚。瞻之在前,忽焉在后。夫子循循然善诱人,博我以文,约我以礼,欲罢不能。既竭吾才,如有所立卓尔,虽欲从之,末由也已。”
\end{yuanwen}

\begin{yuanwen}
子疾病,子路使门人为臣。病间,曰:“久矣哉,由之行诈也!无臣而为有臣,吾谁欺?欺天乎?且予与其死于臣之手也,无宁死于二三子之手乎!且予纵
\end{yuanwen}

\begin{yuanwen}
不得大葬,予死于道路乎?”
\end{yuanwen}

\begin{yuanwen}
子贡曰:“有美玉于斯,韫椟而藏诸?求善贾而沽诸?”子曰:“沽之哉,沽之哉!我待贾者也。”
\end{yuanwen}

\begin{yuanwen}
子欲居九夷。或曰:“陋,如之何?”子曰:“君子居之,何陋之有!”
\end{yuanwen}

\begin{yuanwen}
子曰:“吾自卫反鲁,然后乐正,《雅》、《颂》各得其所。”
\end{yuanwen}

\begin{yuanwen}
子曰:“出则事公卿,入则事父兄,丧事不敢不勉,不为酒困,何有于我哉?”
\end{yuanwen}

\begin{yuanwen}
子在川上曰:“逝者如斯夫!不舍昼夜。”
\end{yuanwen}

\begin{yuanwen}
子曰:“吾未见好德如好色者也。”
\end{yuanwen}

\begin{yuanwen}
子曰:“譬如为山,未成一篑,止,吾止也;譬如平地,虽覆一篑,进,吾往也。”
\end{yuanwen}

\begin{yuanwen}
子曰:“语之而不惰者,其回也与!”
\end{yuanwen}

\begin{yuanwen}
子谓颜渊,曰:“惜乎!吾见其进也,未见其止也。”
\end{yuanwen}

\begin{yuanwen}
子曰:“苗而不秀者有矣夫,秀而不实者有矣夫。”
\end{yuanwen}

\begin{yuanwen}
子曰:“后生可畏,焉知来者之不如今也?四十、五十而无闻焉,斯亦不足畏也已。”
\end{yuanwen}

\begin{yuanwen}
子曰:“法语之言,能无从乎?改之为贵。巽与之言,能无说乎?绎之为贵。说而不绎,从而不改,吾末如之何也已矣。”
\end{yuanwen}

\begin{yuanwen}
子曰:“主忠信。毋友不如己者,过,则勿惮改。”
\end{yuanwen}

\begin{yuanwen}
子曰:“三军可夺帅也,匹夫不可夺志也。”
\end{yuanwen}

\begin{yuanwen}
子曰:“衣敝缊袍,与衣狐貉者立而不耻者,其由也与!‘不忮不求,何用不臧?’”子路终身诵之,子曰:“是道也,何足以臧?”
\end{yuanwen}

\begin{yuanwen}
子曰:“岁寒,然后知松柏之后凋也。”
\end{yuanwen}

\begin{yuanwen}
子曰:“知者不惑,仁者不忧,勇者不惧。”
\end{yuanwen}

\begin{yuanwen}
子曰:“可与共学,未可与适道;可与适道,未可与立;可与立,未可与权。”
\end{yuanwen}

\begin{yuanwen}
“唐棣之华,偏其反而。岂不尔思?室是远尔。”子曰:“未之思也,夫何远之有。”
\end{yuanwen}


\chapter{乡党}

\begin{yuanwen}
孔子于乡党,恂恂如也,似不能言者;其在宗庙朝廷,便便言,唯谨尔。
\end{yuanwen}

\begin{yuanwen}
朝,与下大夫言,侃侃如也;与上大夫言,訚訚如也。君在,踧踖如也,与与如也。
\end{yuanwen}

\begin{yuanwen}
君召使摈,色勃如也,足躩如也。揖所与立,左右手,衣前后襜如也。趋进,翼如也。宾退,必复命曰:“宾不顾矣。”
\end{yuanwen}

\begin{yuanwen}
入公门,鞠躬如也,如不容。立不中门,行不履阈。过位,色勃如也,足躩如也,其言似不足者。摄齐升堂,鞠躬如也,屏气似不息者。出,降一等,逞颜
色,怡怡如也;没阶,趋进,翼如也;复其位,踧踖如也。
\end{yuanwen}

\begin{yuanwen}
执圭,鞠躬如也,如不胜。上如揖,下如授。勃如战色,足蹜蹜如有循。享礼,有容色。私觌,愉愉如也。
\end{yuanwen}

\begin{yuanwen}
君子不以绀緅饰,红紫不以为亵服。当暑,袗絺绤,必表而出之。缁衣羔裘,素衣麑裘,黄衣狐裘。亵裘长,短右袂。必有寝衣,长一身有半。狐貉之厚以居。去丧,无所不佩。非帷裳,必杀之。羔裘玄冠不以吊。吉月,必朝服而朝。
\end{yuanwen}

\begin{yuanwen}
齐,必有明衣,布。齐必变食,居必迁坐。
\end{yuanwen}

\begin{yuanwen}
食不厌精,脍不厌细。食饐而餲,鱼馁而肉败,不食;色恶,不食;臭恶,不食;失饪,不食;不时,不食;割不正,不食;不得其酱,不食。肉虽多,不使胜食气。唯酒无量,不及乱。沽酒市脯,不食。不撤姜食,不多食。
\end{yuanwen}

\begin{yuanwen}
祭于公,不宿肉。祭肉不出三日,出三日不食之矣。
\end{yuanwen}

\begin{yuanwen}
食不语,寝不言。
\end{yuanwen}

\begin{yuanwen}
虽疏食菜羹,瓜祭,必齐如也。
\end{yuanwen}

\begin{yuanwen}
席不正,不坐。
\end{yuanwen}

\begin{yuanwen}
乡人饮酒,杖者出,斯出矣。
\end{yuanwen}

\begin{yuanwen}
乡人傩,朝服而立于阼阶。
\end{yuanwen}

\begin{yuanwen}
问人于他邦,再拜而送之。
\end{yuanwen}

\begin{yuanwen}
康子馈药,拜而受之。曰:“丘未达,不敢尝。”
\end{yuanwen}

\begin{yuanwen}
厩焚,子退朝,曰:“伤人乎?”不问马。
\end{yuanwen}

\begin{yuanwen}
君赐食,必正席先尝之。君赐腥,必熟而荐之。君赐生,必畜之。侍食于君,君祭,先饭。
\end{yuanwen}

\begin{yuanwen}
疾,君视之,东首,加朝服,拖绅。
\end{yuanwen}

\begin{yuanwen}
君命召,不俟驾行矣。
\end{yuanwen}

\begin{yuanwen}
入太庙,每事问。
\end{yuanwen}

\begin{yuanwen}
朋友死,无所归,曰:“于我殡。”
\end{yuanwen}

\begin{yuanwen}
朋友之馈,虽车马,非祭肉,不拜。
\end{yuanwen}

\begin{yuanwen}
寝不尸,居不容。
\end{yuanwen}

\begin{yuanwen}
见齐衰者,虽狎,必变。见冕者与瞽者,虽亵,必以貌。凶服者式之,式负版者。有盛馔,必变色而作。迅雷风烈,必变。
\end{yuanwen}

\begin{yuanwen}
升车,必正立,执绥。车中不内顾,不疾言,不亲指。
\end{yuanwen}

\begin{yuanwen}
色斯举矣,翔而后集。曰:“山梁雌雉,时哉时哉!”子路共之,三嗅而作。
\end{yuanwen}

\chapter{先进}

\begin{yuanwen}
子曰:“先进于礼乐,野人也;后进于礼乐,君子也。如用之,则吾从先进。”
\end{yuanwen}

\begin{yuanwen}
子曰:“从我于陈、蔡者,皆不及门也。”
\end{yuanwen}

\begin{yuanwen}
德行:颜渊,闵子骞,冉伯牛,仲弓。言语:宰我,子贡。政事:冉有,季路。文学:子游,子夏。
\end{yuanwen}

\begin{yuanwen}
子曰:“回也非助我者也,于吾言无所不说。”
\end{yuanwen}

\begin{yuanwen}
子曰:“孝哉闵子骞!人不间于其父母昆弟之言。”
\end{yuanwen}

\begin{yuanwen}
南容三复白圭,孔子以其兄之子妻之。
\end{yuanwen}

\begin{yuanwen}
季康子问:“弟子孰为好学?”孔子对曰:“有颜回者好学,不幸短命死矣,今也则亡。”
\end{yuanwen}

\begin{yuanwen}
颜渊死,颜路请子之车以为之椁。子曰:“才不才,亦各言其子也。鲤也死,有棺而无椁,吾不徒行以为之椁。以吾从大夫之后,不可徒行也。”

\end{yuanwen}

\begin{yuanwen}
	颜渊死,子曰:“噫!天丧予!天丧予!”

颜渊死,子哭之恸,从者曰:“子恸矣!”曰:“有恸乎?非夫人之为恸而谁为?”
\end{yuanwen}

\begin{yuanwen}
颜渊死,门人欲厚葬之,子曰:“不可。”门人厚葬之,子曰:“回也视予犹父也,予不得视犹子也。非我也,夫二三子也!”
\end{yuanwen}

\begin{yuanwen}
季路问事鬼神,子曰:“未能事人,焉能事鬼?”,曰:“敢问死。”曰:“未知生,焉知死?”
\end{yuanwen}

\begin{yuanwen}
闵子侍侧,訚訚如也;子路,行行如也;冉有、子贡,侃侃如也。子乐。“若由也,不得其死然。”
\end{yuanwen}

\begin{yuanwen}
鲁人为长府,闵子骞曰:“仍旧贯如之何?何必改作?”子曰:“夫人不言,言必有中。”
\end{yuanwen}

\begin{yuanwen}
子曰:“由之瑟,奚为于丘之门?”门人不敬子路,子曰:“由也升堂矣,未入于室也。”
\end{yuanwen}

\begin{yuanwen}
子贡问:“师与商也孰贤?”子曰:“师也过,商也不及。”曰:“然则师愈与?”子曰:“过犹不及。”
\end{yuanwen}

\begin{yuanwen}
季氏富于周公,而求也为之聚敛而附益之。子曰:“非吾徒也,小子鸣鼓而攻之可也。”
\end{yuanwen}

\begin{yuanwen}
柴也愚,参也鲁,师也辟,由也喭。
\end{yuanwen}

\begin{yuanwen}
子曰:“回也其庶乎,屡空。赐不受命而货殖焉,亿则屡中。”
\end{yuanwen}

\begin{yuanwen}
子张问善人之道,子曰:“不践迹,亦不入于室。”
\end{yuanwen}

\begin{yuanwen}
子曰:“论笃是与,君子者乎,色庄者乎?”
\end{yuanwen}

\begin{yuanwen}
子路问:“闻斯行诸?”子曰:“有父兄在,如之何其闻斯行之?”冉有问:“闻斯行诸?”子曰:“闻斯行之。”公西华曰:“由也问:“闻斯行诸?”子曰:‘有父兄在’;求也问:‘闻斯行诸’。子曰‘闻斯行之’。赤也惑,敢问。”子曰:“求也退,故进之;由也兼人,故退之。”
\end{yuanwen}

\begin{yuanwen}
子畏于匡,颜渊后。子曰:“吾以女为死矣!”曰:“子在,回何敢死!”
\end{yuanwen}

\begin{yuanwen}
季子然问:“仲由、冉求可谓大臣与?”子曰:“吾以子为异之问,曾由与求之问。所谓大臣者,以道事君,不可则止。今由与求也,可谓具臣矣。”曰:“然则从之者与?”子曰:“弑父与君,亦不从也。”
\end{yuanwen}

\begin{yuanwen}
子路使子羔为费宰,子曰:“贼夫人之子。”子路曰:“有民人焉,有社稷焉,何必读书然后为学。”子曰:“是故恶夫佞者。”
\end{yuanwen}

\begin{yuanwen}
子路、曾皙、冉有、公西华侍坐,子曰:“以吾一日长乎尔,毋吾以也。居则曰‘不吾知也’如或知尔,则何以哉?”子路率尔而对曰:“千乘之国,摄乎大国之间,加之以师旅,因之以饥馑,由也为之,比及三年,可使有勇,且知方也。”夫子哂之。“求,尔何如?”对曰:“方六七十,如五六十,求也为之,比及三年,可使足民。如其礼乐,以俟君子。”“赤!尔何如?”对曰:“非曰能之,愿学焉。宗庙之事,如会同,端章甫,愿为小相焉。”“点,尔何如?”鼓瑟希,铿尔,舍瑟而作,对曰:“异乎三子者之撰。”子曰:“何伤乎?亦各言其志也。”曰:“暮春者
\end{yuanwen}

\chapter{颜渊}

\begin{yuanwen}
颜渊问仁,子曰:“克己复礼为仁。一日克己复礼,天下归仁焉。为仁由己,而由人乎哉?”颜渊曰:“请问其目?”子曰:“非礼勿视,非礼勿听,非礼勿言,非礼勿动。”颜渊曰:“回虽不敏,请事斯语矣。”
\end{yuanwen}

\begin{yuanwen}
仲弓问仁,子曰:“出门如见大宾,使民如承大祭。己所不欲,勿施于人。在邦无怨,在家无怨。”仲弓曰:“雍虽不敏,请事斯语矣。”
\end{yuanwen}

\begin{yuanwen}
司马牛问仁,子曰:“仁者,其言也讱。”曰:“其言也讱,斯谓之仁已乎?”子曰:“为之难,言之得无讱乎?”
\end{yuanwen}

\begin{yuanwen}
司马牛问君子,子曰:“君子不忧不惧。”曰:“不忧不惧,斯谓之君子已乎?”子曰:“内省不疚,夫何忧何惧?”
\end{yuanwen}

\begin{yuanwen}
司马牛忧曰:“人皆有兄弟,我独亡。”子夏曰:“商闻之矣:死生有命,富贵在天。君子敬而无失,与人恭而有礼,四海之内皆兄弟也。君子何患乎无兄弟也?”
\end{yuanwen}

\begin{yuanwen}
子张问明,子曰:“浸润之谮,肤受之愬,不行焉,可谓明也已矣;浸润之谮、肤受之愬不行焉,可谓远也已矣。”
\end{yuanwen}

\begin{yuanwen}
子贡问政,子曰:“足食,足兵,民信之矣。”子贡曰:“必不得已而去,于斯三者何先?”曰:“去兵。”子贡曰:“必不得已而去,于斯二者何先?”曰:“去食。自古皆有死,民无信不立。”
\end{yuanwen}

\begin{yuanwen}
棘子成曰:“君子质而已矣,何以文为?”子贡曰:“惜乎,夫子之说君子也!驷不及舌。文犹质也,质犹文也。虎豹之鞟犹犬羊之鞟。”
\end{yuanwen}

\begin{yuanwen}
哀公问于有若曰:“年饥,用不足,如之何?”有若对曰:“盍彻乎?”曰:“二,吾犹不足,如之何其彻也?”对曰:“百姓足,君孰与不足?百姓不足,君孰与足?”
\end{yuanwen}

\begin{yuanwen}
子张问崇德、辨惑,子曰:“主忠信,徙义,崇德也。爱之欲其生,恶之欲其死;既欲其生又欲其死,是惑也。‘诚不以富,亦祗以异。’”
\end{yuanwen}

\begin{yuanwen}
齐景公问政于孔子,孔子对曰:“君君,臣臣,父父,子子。”公曰:“善哉!信如君不君、臣不臣、父不父、子不子,虽有粟,吾得而食诸?”
\end{yuanwen}

\begin{yuanwen}
子曰:“片言可以折狱者,其由也与?”子路无宿诺。
\end{yuanwen}

\begin{yuanwen}
子曰:“听讼,吾犹人也。必也使无讼乎。”
\end{yuanwen}

\begin{yuanwen}
子张问政,子曰:“居之无倦,行之以忠。”
\end{yuanwen}

\begin{yuanwen}
子曰:“博学于文,约之以礼,亦可以弗畔矣夫。”
\end{yuanwen}

\begin{yuanwen}
子曰:“君子成人之美,不成人之恶;小人反是。”
\end{yuanwen}

\begin{yuanwen}
季康子问政于孔子,孔子对曰:“政者,正也。子帅以正,孰敢不正?”
\end{yuanwen}

\begin{yuanwen}
季康子患盗,问于孔子。孔子对曰:“苟子之不欲,虽赏之不窃。”
\end{yuanwen}

\begin{yuanwen}
季康子问政于孔子曰:“如杀无道以就有道,何如?”孔子对曰:“子为政,焉用杀?子欲善而民善矣。君子之德风,小人之德草,草上之风必偃。”
\end{yuanwen}

\begin{yuanwen}
子张问:“士何如斯可谓之达矣?”子曰:“何哉尔所谓达者?”子张对曰:“在邦必闻,在家必闻。”子曰:“是闻也,非达也。夫达也者,质直而好义,察言而观色,虑以下人。在邦必达,在家必达。夫闻也者,色取仁而行违,居之不疑。在邦必闻,在家必闻。”
\end{yuanwen}

\begin{yuanwen}
樊迟从游于舞雩之下,曰:“敢问崇德、修慝、辨惑。”子曰:“善哉问!先事后得,非崇德与?攻其恶,无攻人之恶,非修慝与?一朝之忿,忘其身,以及其亲,非惑与?”
\end{yuanwen}

\begin{yuanwen}
樊迟问仁,子曰:“爱人。”问知,子曰:“知人。”樊迟未达,子曰:“举直错诸枉,能使枉者直。”樊迟退,见子夏,曰:“乡也吾见于夫子而问知,子曰:‘举直错诸枉,能使枉者直’,何谓也?”子夏曰:“富哉言乎!舜有天下,选于众,举皋陶,不仁者远矣。汤有天下,选于众,举伊尹,不仁者远矣。”
\end{yuanwen}

\begin{yuanwen}
子贡问友,子曰:“忠告而善道之,不可则止,毋自辱焉。”
\end{yuanwen}

\begin{yuanwen}
曾子曰:“君子以文会友,以友辅仁。”
\end{yuanwen}

\chapter{子路}

\begin{yuanwen}
子路问政,子曰:“先之,劳之。”请益,曰:“无倦。”
\end{yuanwen}

\begin{yuanwen}
仲弓为季氏宰,问政,子曰:“先有司,赦小过,举贤才。”曰:“焉知贤才而举之?”子曰:“举尔所知。尔所不知,人其舍诸?”
\end{yuanwen}

\begin{yuanwen}
子路曰:“卫君待子而为政,子将奚先?”子曰:“必也正名乎!”子路曰:“有是哉,子之迂也!奚其正?”子曰:“野哉,由也!君子于其所不知,盖阙如也。名不正、则言不顺,言不顺则事不成,事不成则礼乐不兴,礼乐不兴则刑罚不中,刑罚不中则民无所措手足。故君子名之必可言也,言之必可行也。君子于其言,无所苟而已矣。”
\end{yuanwen}

\begin{yuanwen}
樊迟请学稼,子曰:“吾不如老农。”请学为圃,曰:“吾不如老圃。”樊迟出。子曰:“小人哉,樊须也!上好礼,则民莫敢不敬;上好义,则民莫敢不服;上好信,则民莫敢不用情。夫如是,则四方之民襁负其子而至矣,焉用稼?”
\end{yuanwen}

\begin{yuanwen}
子曰:“诵《诗》三百,授之以政,不达;使于四方,不能专对;虽多,亦奚以为?”
\end{yuanwen}

\begin{yuanwen}
子曰:“其身正,不令而行;其身不正,虽令不从。”
\end{yuanwen}

\begin{yuanwen}
子曰:“鲁卫之政,兄弟也。”
\end{yuanwen}

\begin{yuanwen}
子谓卫公子荆,“善居室。始有,曰:‘苟合矣。’少有,曰:‘苟完矣。’富有,曰:‘苟美矣。’”
\end{yuanwen}

\begin{yuanwen}
子适卫,冉有仆,子曰:“庶矣哉!”冉有曰:“既庶矣,又何加焉?”曰:“富之。”曰:“既富矣,又何加焉?”曰:“教之。”
\end{yuanwen}

\begin{yuanwen}
子曰:“苟有用我者,期月而已可也,三年有成。”
\end{yuanwen}

\begin{yuanwen}
子曰:“‘善人为邦百年,亦可以胜残去杀矣。’诚哉是言也!”
\end{yuanwen}

\begin{yuanwen}
子曰:“如有王者,必世而后仁。”
\end{yuanwen}

\begin{yuanwen}
子曰:“苟正其身矣,于从政乎何有?不能正其身,如正人何?”
\end{yuanwen}

\begin{yuanwen}
冉子退朝,子曰:“何晏也?”对曰:“有政。”子曰:“其事也。如有政,虽不吾以,吾其与闻之。”
\end{yuanwen}

\begin{yuanwen}
定公问:“一言而可以兴邦,有诸?”孔子对曰:“言不可以若是。其几也。人之言曰:‘为君难,为臣不易。’如知为君之难也,不几乎一言而兴邦乎?”曰:“一言而丧邦,有诸?”孔子对曰:“言不可以若是其几也。人之言曰:‘予无乐乎为君,唯其言而莫予违也。’如其善而莫之违也,不亦善乎?如不善而莫之违也,不几乎一言而丧邦乎?”
\end{yuanwen}

\begin{yuanwen}
叶公问政,子曰:“近者说,远者来。”
\end{yuanwen}

\begin{yuanwen}
子夏为莒父宰,问政,子曰:“无欲速,无见小利。欲速则不达,见小利则大事不成。”
\end{yuanwen}

\begin{yuanwen}
叶公语孔子曰:“吾党有直躬者,其父攘羊,而子证之。”孔子曰:“吾党之直者异于是。父为子隐,子为父隐,直在其中矣。”
\end{yuanwen}

\begin{yuanwen}
樊迟问仁,子曰:“居处恭,执事敬,与人忠。虽之夷狄,不可弃也。”
\end{yuanwen}

\begin{yuanwen}
子贡问曰:“何如斯可谓之士矣?”子曰:“行己有耻,使于四方不辱君命,可谓士矣。”曰:“敢问其次。”曰:“宗族称孝焉,乡党称弟焉。”曰:“敢问其次
。”曰:“言必信,行必果,踁踁然小人哉!抑亦可以为次矣。”曰:“今之从政者何如?”子曰:“噫!斗筲之人,何足算也!”
\end{yuanwen}

\begin{yuanwen}
子曰:“不得中行而与之,必也狂狷乎!狂者进取,狷者有所不为也。”
\end{yuanwen}

\begin{yuanwen}
子曰:“南人有言曰:‘人而无恒,不可以作巫医。’善夫!”“不恒其德,或承之羞。”子曰:“不占而已矣。”
\end{yuanwen}

\begin{yuanwen}
子曰:“君子和而不同,小人同而不和。”
\end{yuanwen}

\begin{yuanwen}
子贡问曰:“乡人皆好之,何如?”子曰:“未可也。”“乡人皆恶之,何如?”子曰:“未可也。不如乡人之善者好之,其不善者恶之。”
\end{yuanwen}

\begin{yuanwen}
子曰:“君子易事而难说也,说之不以道不说也,及其使人也器之;小人难事而易说也,说之虽不以道说也,及其使人也求备焉。”
\end{yuanwen}

\begin{yuanwen}
子曰:“君子泰而不骄,小人骄而不泰。”
\end{yuanwen}

\begin{yuanwen}
子曰:“刚、毅、木、讷近仁。”
\end{yuanwen}

\begin{yuanwen}
子路问曰:“何如斯可谓之士矣?”子曰:“切切偲偲,怡怡如也,可谓士矣。朋友切切偲偲,兄弟怡怡。”
\end{yuanwen}

\begin{yuanwen}
子曰:“善人教民七年,亦可以即戎矣。”
\end{yuanwen}

\begin{yuanwen}
子曰:“以不教民战,是谓弃之。”
\end{yuanwen}


\chapter{宪问}


\begin{yuanwen}
宪问耻,子曰:“邦有道,谷;邦无道,谷,耻也。”“克、伐、怨、欲不行焉,可以为仁矣?”子曰:“可以为难矣,仁则吾不知也。”
\end{yuanwen}

孔子说:“。”

\begin{yuanwen}
子曰:“士而怀居,不足以为士矣。”
\end{yuanwen}

孔子说:“。”

\begin{yuanwen}
子曰:“邦有道,危言危行;邦无道,危行言孙。”
\end{yuanwen}

孔子说:“。”

\begin{yuanwen}
子曰:“有德者必有言,有言者不必有德。仁者必有勇,勇者不必有仁。”
\end{yuanwen}

孔子说:“。”

\begin{yuanwen}
南宫适问于孔子曰:“羿善射,奡荡舟,俱不得其死然;禹、稷躬稼而有天下。”夫子不答。南宫适出,子曰:“君子哉若人!尚德哉若人!”
\end{yuanwen}

孔子说:“。”

\begin{yuanwen}
子曰:“君子而不仁者有矣夫,未有小人而仁者也。”
\end{yuanwen}

孔子说:“。”

\begin{yuanwen}
子曰:“爱之,能勿劳乎?忠焉,能勿诲乎?”
\end{yuanwen}

孔子说:“。”

\begin{yuanwen}
子曰:“为命,裨谌草创之,世叔讨论之,行人子羽修饰之,东里子产润色之。”
\end{yuanwen}

孔子说:“。”

\begin{yuanwen}
或问子产,子曰:“惠人也。”问子西,曰:“彼哉,彼哉!”问管仲,曰:“人也。夺伯氏骈邑三百,饭疏食,没齿无怨言。”
\end{yuanwen}

孔子说:“。”

\begin{yuanwen}
子曰:“贫而无怨难,富而无骄易。”
\end{yuanwen}

孔子说:“。”

\begin{yuanwen}
子曰:“孟公绰为赵、魏老则优,不可以为滕、薛大夫。”
\end{yuanwen}

孔子说:“。”

\begin{yuanwen}
子路问成人,子曰:“若臧武仲之知、公绰之不欲、卞庄子之勇、冉求之艺,文之以礼乐,亦可以为成人矣。”曰:“今之成人者何必然?见利思义,见危授命,久要不忘平生之言,亦可以为成人矣。”
\end{yuanwen}

孔子说:“。”

\begin{yuanwen}
子问公叔文子于公明贾曰:“信乎,夫子不言,不笑,不取乎?”公明贾对曰:“以告者过也。夫子时然后言,人不厌其言;乐然后笑,人不厌其笑;义然后取,人不厌其取。”子曰:“其然?岂其然乎?”
\end{yuanwen}

孔子说:“。”

\begin{yuanwen}
子曰:“臧武仲以防求为后于鲁,虽曰不要君,吾不信也。”
\end{yuanwen}

孔子说:“。”

\begin{yuanwen}
子曰:“晋文公谲而不正,齐桓公正而不谲。”
\end{yuanwen}

孔子说:“。”

\begin{yuanwen}
子路曰:“桓公杀公子纠,召忽死之,管仲不死,曰未仁乎?”子曰:“桓公九合诸侯不以兵车,管仲之力也。如其仁,如其仁!”
\end{yuanwen}

孔子说:“。”

\begin{yuanwen}
子贡曰:“管仲非仁者与?桓公杀公子纠,不能死,又相之。”子曰:“管仲相桓公霸诸侯,一匡天下,民到于今受其赐。微管仲,吾其被发左衽矣。岂若匹夫匹妇之为谅也,自经于沟渎而莫之知也。”
\end{yuanwen}

孔子说:“。”

\begin{yuanwen}
公叔文子之臣大夫僎与文子同升诸公,子闻之,曰:“可以为‘文’矣。”
\end{yuanwen}

\begin{yuanwen}
子言卫灵公之无道也,康子曰:“夫如是,奚而不丧?”孔子曰:“仲叔圉治宾客,祝鮀治宗庙,王孙贾治军旅,夫如是,奚其丧?”
\end{yuanwen}

孔子说:“。”

\begin{yuanwen}
子曰:“其言之不怍,则为之也难。”
\end{yuanwen}

孔子说:“。”

\begin{yuanwen}
陈成子弑简公,孔子沐浴而朝,告于哀公曰:“陈恒弑其君,请讨之。”公曰:“告夫三子。”,孔子曰:“以吾从大夫之后,不敢不告也,君曰‘告夫三子’者
!”之三子告,不可。孔子曰:“以吾从大夫之后,不敢不告也。”
\end{yuanwen}

孔子说:“。”

\begin{yuanwen}
子路问事君,子曰:“勿欺也,而犯之。”
\end{yuanwen}

孔子说:“。”

\begin{yuanwen}
子曰:“君子上达,小人下达。”
\end{yuanwen}

孔子说:“。”

\begin{yuanwen}
子曰:“古之学者为己,今之学者为人。”
\end{yuanwen}

孔子说:“。”

\begin{yuanwen}
蘧伯玉使人于孔子,孔子与之坐而问焉,曰:“夫子何为?”对曰:“夫子欲寡其过而未能也。”使者出,子曰:“使乎!使乎!”
\end{yuanwen}

孔子说:“。”

\begin{yuanwen}
子曰:“不在其位,不谋其政。”曾子曰:“君子思不出其位。”
\end{yuanwen}

孔子说:“。”

\begin{yuanwen}
子曰:“君子耻其言而过其行。”
\end{yuanwen}

孔子说:“。”

\begin{yuanwen}
子曰:“君子道者三,我无能焉:仁者不忧,知者不惑,勇者不惧。”子贡曰:“夫子自道也。”
\end{yuanwen}

孔子说:“。”

\begin{yuanwen}
子贡方人,子曰:“赐也贤乎哉?夫我则不暇。”
\end{yuanwen}

孔子说:“。”

\begin{yuanwen}
子曰:“不患人之不己知,患其不能也。”
\end{yuanwen}

孔子说:“。”

\begin{yuanwen}
子曰:“不逆诈,不亿不信,抑亦先觉者,是贤乎!”
\end{yuanwen}

孔子说:“。”

\begin{yuanwen}
微生亩谓孔子曰:“丘何为是栖栖者与?无乃为佞乎?”孔子曰:“非敢为佞也,疾固也。”
\end{yuanwen}

孔子说:“。”

\begin{yuanwen}
子曰:“骥不称其力,称其德也。”
\end{yuanwen}

孔子说:“。”

\begin{yuanwen}
或曰:“以德报怨,何如?”子曰:“何以报德?以直报怨,以德报德。”
\end{yuanwen}

孔子说:“。”

\begin{yuanwen}
子曰:“莫我知也夫!”子贡曰:“何为其莫知子也?”子曰:“不怨天,不尤人,下学而上达。知我者其天乎!”
\end{yuanwen}

孔子说:“。”

\begin{yuanwen}
公伯寮愬子路于季孙。子服景伯以告,曰:“夫子固有惑志于公伯寮,吾力犹能肆诸市朝。”子曰:“道之将行也与,命也;道之将废也与,命也。公伯寮其如命何?”
\end{yuanwen}

孔子说:“。”

\begin{yuanwen}
子曰:“贤者辟世,其次辟地,其次辟色,其次辟言。”子曰:“作者七人矣。”
\end{yuanwen}

孔子说:“。”

\begin{yuanwen}
子路宿于石门,晨门曰:“奚自?”子路曰:“自孔氏。”曰:“是知其不可而为之者与?”
\end{yuanwen}

孔子说:“。”

\begin{yuanwen}
子击磬于卫,有荷蒉而过孔氏之门者,曰:“有心哉,击磬乎!”既而曰:“鄙哉,硁硁乎!莫己知也,斯己而已矣。深则厉,浅则揭。”子曰:“果哉!末之难矣。”
\end{yuanwen}

孔子说:“。”

\begin{yuanwen}
子张曰:“《书》云,‘高宗谅阴,三年不言。’何谓也?”子曰:“何必高宗,古之人皆然。君薨,百官总己以听于冢宰三年。”
\end{yuanwen}

孔子说:“。”

\begin{yuanwen}
子曰:“上好礼,则民易使也。”
\end{yuanwen}

孔子说:“。”

\begin{yuanwen}
子路问君子,子曰:“修己以敬。”曰:“如斯而已乎?”曰:“修己以安人。”曰:“如斯而已乎?”曰:“修己以安百姓。修己以安百姓,尧、舜其犹病诸!”
\end{yuanwen}

孔子说:“。”

\begin{yuanwen}
原壤夷俟,子曰:“幼而不孙弟,长而无述焉,老而不死,是为贼!”以杖叩其胫。
\end{yuanwen}

孔子说:“。”

\begin{yuanwen}
阙党童子将命,或问之曰:“益者与?”子曰:“吾见其居于位也,见其与先生并行也。非求益者也,欲速成者也。”
\end{yuanwen}

孔子说:“。”


\chapter{卫灵公}



\begin{yuanwen}
卫灵公问陈于孔子,孔子对曰:“俎豆之事,则尝闻之矣;军旅之事,未之学也。”明日遂行。
\end{yuanwen}

孔子说:“。”

\begin{yuanwen}
在陈绝粮,从者病莫能兴。子路愠见曰:“君子亦有穷乎?”子曰:“君子固穷,小人穷斯滥矣。”
\end{yuanwen}

孔子说:“。”

\begin{yuanwen}
子曰:“赐也,女以予为多学而识之者与?”对曰:“然,非与?”曰:“非也,予一以贯之。”
\end{yuanwen}

孔子说:“。”

\begin{yuanwen}
子曰:“由,知德者鲜矣。”
\end{yuanwen}

孔子说:“。”

\begin{yuanwen}
子曰:“无为而治者其舜也与!夫何为哉?恭己正南面而已矣。”
\end{yuanwen}

孔子说:“。”

\begin{yuanwen}
子张问行,子曰:“言忠信,行笃敬,虽蛮貊之邦,行矣。言不忠信,行不笃敬,虽州里,行乎哉?立则见其参于前也,在舆则见其倚于衡也,夫然后行。”子张书诸绅。
\end{yuanwen}

孔子说:“。”

\begin{yuanwen}
子曰:“直哉史鱼!邦有道如矢,邦无道如矢。君子哉蘧伯玉!邦有道则仕,邦无道则可卷而怀之。”
\end{yuanwen}

孔子说:“。”

\begin{yuanwen}
子曰:“可与言而不与之言,失人;不可与言而与之言,失言。知者不失人亦不失言。”
\end{yuanwen}

孔子说:“。”

\begin{yuanwen}
子曰:“志士仁人无求生以害仁,有杀身以成仁。”
\end{yuanwen}

孔子说:“。”

\begin{yuanwen}
子贡问为仁,子曰:“工欲善其事,必先利其器。居是邦也,事其大夫之贤者,友其士之仁者。”
\end{yuanwen}

孔子说:“。”

\begin{yuanwen}
颜渊问为邦,子曰:“行夏之时,乘殷之辂,服周之冕,乐则《韶》、《舞》;放郑声,远佞人。郑声淫,佞人殆。”
\end{yuanwen}

孔子说:“。”

\begin{yuanwen}
子曰:“人无远虑,必有近忧。”
\end{yuanwen}

孔子说:“。”

\begin{yuanwen}
子曰:“已矣乎!吾未见好德如好色者也。”
\end{yuanwen}

孔子说:“。”

\begin{yuanwen}
子曰:“臧文仲其窃位者与!知柳下惠之贤而不与立也。”
\end{yuanwen}

孔子说:“。”

\begin{yuanwen}
子曰:“躬自厚而薄责于人,则远怨矣。”
\end{yuanwen}

孔子说:“。”

\begin{yuanwen}
子曰:“不曰‘如之何、如之何’者,吾末如之何也已矣。”
\end{yuanwen}

孔子说:“。”

\begin{yuanwen}
子曰:“群居终日,言不及义,好行小慧,难矣哉!”
\end{yuanwen}

孔子说:“。”

\begin{yuanwen}
子曰:“君子义以为质,礼以行之,孙以出之,信以成之。君子哉!”
\end{yuanwen}

孔子说:“。”

\begin{yuanwen}
子曰:“君子病无能焉,不病人之不己知也。”
\end{yuanwen}

孔子说:“。”

\begin{yuanwen}
子曰:“君子疾没世而名不称焉。”
\end{yuanwen}

孔子说:“。”

\begin{yuanwen}
子曰:“君子求诸己,小人求诸人。”
\end{yuanwen}

孔子说:“。”

\begin{yuanwen}
子曰:“君子矜而不争,群而不党。”
\end{yuanwen}

孔子说:“。”

\begin{yuanwen}
子曰:“君子不以言举人,不以人废言。”
\end{yuanwen}

孔子说:“。”

\begin{yuanwen}
子贡问曰:“有一言而可以终身行之者乎?”子曰:“其恕乎!己所不欲,勿施于人。”
\end{yuanwen}

孔子说:“。”

\begin{yuanwen}
子曰:“吾之于人也,谁毁谁誉?如有所誉者,其有所试矣。斯民也,三代之所以直道而行也。”
\end{yuanwen}

孔子说:“。”

\begin{yuanwen}
子曰:“吾犹及史之阙文也,有马者借人乘之,今亡矣夫!”
\end{yuanwen}

孔子说:“。”

\begin{yuanwen}
子曰:“巧言乱德,小不忍,则乱大谋。”
\end{yuanwen}

孔子说:“。”

\begin{yuanwen}
子曰:“众恶之,必察焉;众好之,必察焉。”
\end{yuanwen}

孔子说:“。”

\begin{yuanwen}
子曰:“人能弘道,非道弘人。”
\end{yuanwen}

孔子说:“。”

\begin{yuanwen}
子曰:“过而不改,是谓过矣。”
\end{yuanwen}

孔子说:“。”

\begin{yuanwen}
子曰:“吾尝终日不食、终夜不寝以思,无益,不如学也。”
\end{yuanwen}

孔子说:“。”

\begin{yuanwen}
子曰:“君子谋道不谋食。耕也馁在其中矣,学也禄在其中矣。君子忧道不忧贫。”
\end{yuanwen}

孔子说:“。”

\begin{yuanwen}
子曰:“知及之,仁不能守之,虽得之,必失之。知及之,仁能守之,不庄以涖之,则民不敬。知及之,仁能守之,庄以涖之,动之不以礼,未善也。”
\end{yuanwen}

孔子说:“。”

\begin{yuanwen}
子曰:“君子不可小知而可大受也,小人不可大受而可小知也。”
\end{yuanwen}

孔子说:“。”

\begin{yuanwen}
子曰:“民之于仁也,甚于水火。水火,吾见蹈而死者矣,未见蹈仁而死者也。”
\end{yuanwen}

孔子说:“。”

\begin{yuanwen}
子曰:“当仁不让于师。”
\end{yuanwen}

孔子说:“。”

\begin{yuanwen}
子曰:“君子贞而不谅。”
\end{yuanwen}

孔子说:“。”

\begin{yuanwen}
子曰:“事君,敬其事而后其食。”
\end{yuanwen}

孔子说:“。”

\begin{yuanwen}
子曰:“有教无类。”
\end{yuanwen}

孔子说:“。”

\begin{yuanwen}
子曰:“道不同,不相为谋。”
\end{yuanwen}

孔子说:“。”

\begin{yuanwen}
子曰:“辞达而已矣。”
\end{yuanwen}

孔子说:“。”

\begin{yuanwen}
师冕见,及阶,子曰:“阶也。”及席,子曰:“席也。”皆坐,子告之曰:“某在斯,某在斯。”师冕出。子张问曰:“与师言之道与?”子曰:“然,固相师之道也。”
\end{yuanwen}

孔子说:“。”

\chapter{季氏}

\begin{yuanwen}
季氏将伐颛臾,冉有、季路见于孔子,曰:“季氏将有事于颛臾。”孔子曰:“求,无乃尔是过与?夫颛臾,昔者先王以为东蒙主,且在邦域之中矣,是社稷之臣也。何以伐为?”冉有曰:“夫子欲之,吾二臣者皆不欲也。”孔子曰:“求,周任有言曰:‘陈力就列,不能者止。’危而不持,颠而不扶,则将焉用彼相矣?且尔言过矣,虎兕出于柙,龟玉毁于椟中,是谁之过与?”冉有曰:“今夫颛臾固而近于费,今不取,后世必为子孙忧。”孔子曰:“求,君子疾夫舍曰欲之而必为之辞。丘也闻,有国有家者,不患寡而患不均,不患贫而患不安。盖均无贫,和无寡,安无倾。夫如是,故远人不服则修文德以来之,既来之,则安之。今由与求也相夫子,远人不服而不能来也,邦分崩离析而不能守也,而谋动干戈于邦内。吾恐季孙之忧不在颛臾,而在萧墙之内也。”
\end{yuanwen}

孔子说:“。”

\begin{yuanwen}
孔子曰:“天下有道,则礼乐征伐自天子出;天下无道,则礼乐征伐自诸侯出。自诸侯出,盖十世希不失矣;自大夫出,五世希不失矣;陪臣执国命,三世希不失矣。天下有道,则政不在大夫;天下有道,则庶人不议。”
\end{yuanwen}

孔子说:“。”

\begin{yuanwen}
孔子曰:“禄之去公室五世矣,政逮于大夫四世矣,故夫三桓之子孙微矣。”
\end{yuanwen}

孔子说:“。”

\begin{yuanwen}
孔子曰:“益者三友,损者三友。友直、友谅、友多闻,益矣;友便辟、友善柔、友便佞,损矣。”
\end{yuanwen}

孔子说:“。”

\begin{yuanwen}
孔子曰:“益者三乐,损者三乐。乐节礼乐、乐道人之善、乐多贤友,益矣;乐骄乐、乐佚游、乐宴乐,损矣。”
\end{yuanwen}

孔子说:“。”

\begin{yuanwen}
孔子曰:“侍于君子有三愆:言未及之而言谓之躁,言及之而不言谓之隐,未见颜色而言谓之瞽。”
\end{yuanwen}

孔子说:“。”

\begin{yuanwen}
孔子曰:“君子有三戒:少之时,血气未定,戒之在色;及其壮也,血气方刚,戒之在斗;及其老也,血气既衰,戒之在得。”
\end{yuanwen}

孔子说:“。”

\begin{yuanwen}
孔子曰:“君子有三畏:畏天命,畏大人,畏圣人之言。小人不知天命而不畏也,狎大人,侮圣人之言。”
\end{yuanwen}

孔子说:“。”

\begin{yuanwen}
孔子曰:“生而知之者上也,学而知之者次也;困而学之又其次也。困而不学,民斯为下矣。”
\end{yuanwen}

孔子说:“。”

\begin{yuanwen}
孔子曰:“君子有九思:视思明,听思聪,色思温,貌思恭,言思忠,事思敬,疑思问,忿思难,见得思义。”
\end{yuanwen}

孔子说:“。”

\begin{yuanwen}
孔子曰:“见善如不及,见不善如探汤;吾见其人矣。吾闻其语矣。隐居以求其志,行义以达其道;吾闻其语矣,未见其人也。”
\end{yuanwen}

孔子说:“。”

\begin{yuanwen}
齐景公有马千驷,死之日,民无德而称焉;伯夷、叔齐饿于首阳之下,民到于今称之。其斯之谓与?”
\end{yuanwen}

孔子说:“。”

\begin{yuanwen}
陈亢问于伯鱼曰:“子亦有异闻乎?”对曰:“未也。尝独立,鲤趋而过庭,曰:‘学《诗》乎?’对曰:‘未也。’‘不学《诗》,无以言。’鲤退而学《诗》。他日,又独立,鲤趋而过庭,曰:‘学《礼》乎?’对曰:‘未也。’‘不学《礼》,无以立。’鲤退而学《礼》。闻斯二者。”陈亢退而喜曰:“问一得三,闻《诗》,闻《礼》,又闻君子之远其子也。”
\end{yuanwen}

孔子说:“。”

\begin{yuanwen}
邦君之妻,君称之曰夫人,夫人自称曰小童;邦人称之曰君夫人,称诸异邦曰寡小君;异邦人称之亦曰君夫人。
\end{yuanwen}

孔子说:“。”

\chapter{阳货}

\begin{yuanwen}
阳货欲见孔子,孔子不见,归孔子豚。孔子时其亡也而往拜之,遇诸涂。谓孔子曰:“来,予与尔言。”曰:“怀其宝而迷其邦,可谓仁乎?”曰:“不可。”“好从事而亟失时,可谓知乎?”曰:“不可!”“日月逝矣,岁不我与!”孔子曰:“诺,吾将仕矣。”
\end{yuanwen}

孔子说:“。”

\begin{yuanwen}
子曰:“性相近也,习相远也。”
\end{yuanwen}

孔子说:“。”

\begin{yuanwen}
子曰:“唯上知与下愚不移。”
\end{yuanwen}

孔子说:“。”

\begin{yuanwen}
子之武城,闻弦歌之声。夫子莞尔而笑,曰:“割鸡焉用牛刀?”子游对曰:“昔者偃也闻诸夫子曰:‘君子学道则爱人,小人学道则易使也。’”子曰:“二三
子,偃之言是也!前言戏之耳。”
\end{yuanwen}

孔子说:“。”

\begin{yuanwen}
公山弗扰以费畔,召,子欲往。子路不说,曰:“末之也已,何必公山氏之之也?”子曰:“夫召我者而岂徒哉?如有用我者,吾其为东周乎!”
\end{yuanwen}

孔子说:“。”

\begin{yuanwen}
子张问仁于孔子,孔子曰:“能行五者于天下为仁矣。”请问之,曰:“恭、宽、信、敏、惠。恭则不侮,宽则得众,信则人任焉,敏则有功,惠则足以使人。”
\end{yuanwen}

孔子说:“。”

\begin{yuanwen}
佛肸召,子欲往。子路曰:“昔者由也闻诸夫子曰。亲于其身为不善者,君子不入也。佛肸以中牟畔,子之往也,如之何?"子曰:“然。有是言也。不曰坚乎,磨而不磷?不曰白乎,涅而不缁。吾岂匏瓜也哉?焉能系而不食?”
\end{yuanwen}

孔子说:“。”

\begin{yuanwen}
子曰:“由也,女闻六言六蔽矣乎?”对曰:“未也。”“居!吾语女。好仁不好学,其蔽也愚;好知不好学,其蔽也荡;好信不好学,其蔽也贼;好直不好学,其蔽也绞;好勇不好学,其蔽也乱;好刚不好学,其蔽也狂。”
\end{yuanwen}

孔子说:“。”

\begin{yuanwen}
子曰:“小子何莫学夫诗!诗,可以兴,可以观,可以群,可以怨:迩之事父,远之事君.多识于鸟兽草木之名。”
\end{yuanwen}

孔子说:“。”

\begin{yuanwen}
子谓伯鱼曰:“女为《周南》、《召南》矣乎?人而不为《周南》、《召南》,其犹正墙面而立也与?”
\end{yuanwen}

孔子说:“。”

\begin{yuanwen}
子曰:“礼云礼云,玉帛云乎哉?乐云乐云,钟鼓云乎哉?”
\end{yuanwen}

孔子说:“。”

\begin{yuanwen}
子曰:“色厉而内荏,譬诸小人,其犹穿窬之盗也与?”
\end{yuanwen}

孔子说:“。”

\begin{yuanwen}
子曰:“乡愿,德之贼也。”
\end{yuanwen}

孔子说:“。”

\begin{yuanwen}
子曰:“道听而涂说,德之弃也。”
\end{yuanwen}

孔子说:“。”

\begin{yuanwen}
子曰:“鄙夫可与事君也与哉?其未得之也,患得之;既得之,患失之。苟患失之,无所不至矣。”
\end{yuanwen}

孔子说:“。”

\begin{yuanwen}
子曰:“古者民有三疾,今也或是之亡也。古之狂也肆,今之狂也荡;古之矜也廉,今之矜也忿戾;古之愚也直,今之愚也诈而已矣。”
\end{yuanwen}

孔子说:“。”

\begin{yuanwen}
子曰:“巧言令色,鲜矣仁。”
\end{yuanwen}

孔子说:“。”

\begin{yuanwen}
子曰:“恶紫之夺朱也,恶郑声之乱雅乐也,恶利口之覆邦家者。”
\end{yuanwen}

孔子说:“。”

\begin{yuanwen}
子曰:“予欲无言。”子贡曰:“子如不言,则小子何述焉?”子曰:“天何言哉?四时行焉,百物生焉,天何言哉?”
\end{yuanwen}

孔子说:“。”

\begin{yuanwen}
孺悲欲见孔子,孔子辞以疾。将命者出户,取瑟而歌,使之闻之。
\end{yuanwen}

孔子说:“。”

\begin{yuanwen}
宰我问:“三年之丧,期已久矣!君子三年不为礼,礼必坏;三年不为乐,乐必崩。旧谷既没,新谷既升,钻燧改火,期可已矣。”子曰:“食夫稻,衣夫锦,于女安乎?”曰:“安!”“女安则为之!夫君子之居丧,食旨不甘,闻乐不乐,居处不安,故不为也。今女安,则为之!”宰我出,子曰:“予之不仁也!子生三年,然后免于父母之怀。夫三年之丧,天下之通丧也,予也有三年之爱于其父母乎!”
\end{yuanwen}

孔子说:“。”

\begin{yuanwen}
子曰:“饱食终日,无所用心,难矣哉!不有博弈者乎?为之犹贤乎已。”
\end{yuanwen}

孔子说:“。”

\begin{yuanwen}
子路曰:“君子尚勇乎?”子曰:“君子义以为上。君子有勇而无义为乱,小人有勇而无义为盗。”
\end{yuanwen}

孔子说:“。”

\begin{yuanwen}
子贡曰:“君子亦有恶乎?”子曰:“有恶。恶称人之恶者,恶居下流而讪上者,恶勇而无礼者,恶果敢而窒者。”曰:“赐也亦有恶乎?”“恶徼以为知者,恶
不孙以为勇者,恶讦以为直者。”
\end{yuanwen}

孔子说:“。”

\begin{yuanwen}
子曰:“唯女子与小人为难养也,近之则不孙,远之则怨。”
\end{yuanwen}

孔子说:“。”

\begin{yuanwen}
子曰:“年四十而见恶焉,其终也已。”
\end{yuanwen}

孔子说:“。”


\chapter{微子}

\begin{yuanwen}
微子去之,箕子为之奴,比干谏而死。孔子曰:“殷有三仁焉。”
\end{yuanwen}

孔子说:“。”

\begin{yuanwen}
柳下惠为士师,三黜。人曰:“子未可以去乎?”曰:“直道而事人,焉往而不三黜?枉道而事人,何必去父母之邦?”
\end{yuanwen}

孔子说:“。”

\begin{yuanwen}
齐景公待孔子曰:“若季氏,则吾不能。”以季、孟之间待之,曰:“吾老矣,不能用也。”孔子行。
\end{yuanwen}

孔子说:“。”

\begin{yuanwen}
齐人归女乐,季桓子受之,三日不朝,孔子行。”
\end{yuanwen}

孔子说:“。”

\begin{yuanwen}
楚狂接舆歌而过孔子曰:“凤兮凤兮,何德之衰?往者不可谏,来者犹可追。已而已而,今之从政者殆而!”孔子下,欲与之言,趋而辟之,不得与之言。
\end{yuanwen}

孔子说:“。”

\begin{yuanwen}
长沮、桀溺耦而耕,孔子过之,使子路问津焉。长沮曰:“夫执舆者为谁?”子路曰:“为孔丘。”曰:“是鲁孔丘与?”曰:“是也。”曰:“是知津矣。”问于桀溺,桀溺曰:“子为谁?”曰:“为仲由。”曰:“是鲁孔丘之徒与?”对曰:“然。”曰:“滔滔者天下皆是也,而谁以易之?且而与其从辟人之士也,岂若从辟世之士哉?”耰而不辍。子路行以告,夫子怃然曰:“鸟兽不可与同群,吾非斯人之徒与而谁与?天下有道,丘不与易也。”
\end{yuanwen}

孔子说:“。”

\begin{yuanwen}
子路从而后,遇丈人,以杖荷蓧。子路问曰:“子见夫子乎?”丈人曰:“四体不勤,五谷不分,孰为夫子?”植其杖而芸,子路拱而立。止子路宿,杀鸡为黍而食之,见其二子焉。明日,子路行以告,子曰:“隐者也。”使子路反见之,至则行矣。子路曰:“不仕无义。长幼之节不可废也,君臣之义如之何其废之?欲洁其身而乱大伦。君子之仕也,行其义也,道之不行已知之矣。”
\end{yuanwen}

孔子说:“。”

\begin{yuanwen}
逸民:伯夷、叔齐、虞仲、夷逸、朱张、柳下惠、少连。子曰:“不降其志,不辱其身,伯夷、叔齐与!”谓:“柳下惠、少连降志辱身矣,言中伦,行中虑,其斯而已矣。”谓:“虞仲、夷逸隐居放言,身中清,废中权。我则异于是,无可无不可。”
\end{yuanwen}

孔子说:“。”

\begin{yuanwen}
太师挚适齐,亚饭干适楚,三饭缭适蔡,四饭缺适秦,鼓方叔入于河,播鼗武入于汉,少师阳、击磬襄入于海。
\end{yuanwen}

孔子说:“。”

\begin{yuanwen}
周公谓鲁公曰:“君子不施其亲,不使大臣怨乎不以,故旧无大故则不弃也,无求备于一人。”
\end{yuanwen}

孔子说:“。”

\begin{yuanwen}
周有八士:伯达、伯适、仲突、仲忽、叔夜、叔夏、季随、季騧。
\end{yuanwen}

孔子说:“。”

\chapter{子张}

\begin{yuanwen}
子张曰:“士见危致命,见得思义,祭思敬,丧思哀,其可已矣。”
\end{yuanwen}

孔子说:“。”

\begin{yuanwen}
子张曰:“执德不弘,信道不笃,焉能为有?焉能为亡?”
\end{yuanwen}

孔子说:“。”

\begin{yuanwen}
子夏之门人问交于子张,子张曰:“子夏云何?”对曰:“子夏曰:‘可者与之,其不可者拒之。’”子张曰:“异乎吾所闻。君子尊贤而容众,嘉善而矜不能。我之大贤与,于人何所不容?我之不贤与,人将拒我,如之何其拒人也?”
\end{yuanwen}

孔子说:“。”

\begin{yuanwen}
子夏曰:“虽小道必有可观者焉,致远恐泥,是以君子不为也。”
\end{yuanwen}

孔子说:“。”

\begin{yuanwen}
子夏曰:“日知其所亡,月无忘其所能,可谓好学也已矣。”
\end{yuanwen}

孔子说:“。”

\begin{yuanwen}
子夏曰:“博学而笃志,切问而近思,仁在其中矣。”
\end{yuanwen}

孔子说:“。”

\begin{yuanwen}
子夏曰:“百工居肆以成其事,君子学以致其道。”
\end{yuanwen}

孔子说:“。”

\begin{yuanwen}
子夏曰:“小人之过也必文。”
\end{yuanwen}

孔子说:“。”

\begin{yuanwen}
子夏曰:“君子有三变:望之俨然,即之也温,听其言也厉。”
\end{yuanwen}

孔子说:“。”

\begin{yuanwen}
子夏曰:“君子信而后劳其民,未信,则以为厉己也;信而后谏,未信,则以为谤己也。”
\end{yuanwen}

孔子说:“。”

\begin{yuanwen}
子夏曰:“大德不逾闲,小德出入可也。”
\end{yuanwen}

孔子说:“。”

\begin{yuanwen}
子游曰:“子夏之门人小子,当洒扫应对进退则可矣。抑末也,本之则无,如之何?”子夏闻之,曰:“噫,言游过矣!君子之道,孰先传焉?孰后倦焉?譬诸草木,区以别矣。君子之道焉可诬也?有始有卒者,其惟圣人乎!”
\end{yuanwen}

孔子说:“。”

\begin{yuanwen}
子夏曰:“仕而优则学,学而优则仕。”
\end{yuanwen}

孔子说:“。”

\begin{yuanwen}
子游曰:“丧致乎哀而止。”
\end{yuanwen}

孔子说:“。”

\begin{yuanwen}
子游曰:“吾友张也为难能也,然而未仁。”
\end{yuanwen}

孔子说:“。”

\begin{yuanwen}
曾子曰:“堂堂乎张也,难与并为仁矣。”
\end{yuanwen}

孔子说:“。”

\begin{yuanwen}
曾子曰:“吾闻诸夫子,人未有自致者也,必也亲丧乎!”
\end{yuanwen}

孔子说:“。”

\begin{yuanwen}
曾子曰:“吾闻诸夫子,孟庄子之孝也,其他可能也;其不改父之臣与父之政,是难能也。”
\end{yuanwen}

孔子说:“。”

\begin{yuanwen}
孟氏使阳肤为士师,问于曾子。曾子曰:“上失其道,民散久矣。如得其情,则哀矜而勿喜!”
\end{yuanwen}

孔子说:“。”

\begin{yuanwen}
子贡曰:“纣之不善,不如是之甚也。是以君子恶居下流,天下之恶皆归焉。”
\end{yuanwen}

孔子说:“。”

\begin{yuanwen}
子贡曰:“君子之过也,如日月之食焉。过也人皆见之,更也人皆仰之。”
\end{yuanwen}

孔子说:“。”

\begin{yuanwen}
卫公孙朝问于子贡曰:“仲尼焉学?”子贡曰:“文武之道未坠于地,在人。贤者识其大者,不贤者识其小者,莫不有文武之道焉,夫子焉不学?而亦何常师之有?”
\end{yuanwen}

孔子说:“。”

\begin{yuanwen}
叔孙武叔语大夫于朝曰:“子贡贤于仲尼。”子服景伯以告子贡,子贡曰:“譬之宫墙,赐之墙也及肩,窥见室家之好;夫子之墙数仞,不得其门而入,不见宗庙之美、百官之富。得其门者或寡矣,夫子之云不亦宜乎!”
\end{yuanwen}

孔子说:“。”

\begin{yuanwen}
叔孙武叔毁仲尼,子贡曰:“无以为也,仲尼不可毁也。他人之贤者,丘陵也,犹可逾也;仲尼,日月也,无得而逾焉。人虽欲自绝,其何伤于日月乎?多见其不知量也。”
\end{yuanwen}

孔子说:“。”

\begin{yuanwen}
陈子禽谓子贡曰:“子为恭也,仲尼岂贤于子乎?”子贡曰:“君子一言以为知,一言以为不知,言不可不慎也。夫子之不可及也,犹天之不可阶而升也。夫子之得邦家者,所谓立之斯立,道之斯行,绥之斯来,动之斯和。其生也荣,其死也哀,如之何其可及也?”
\end{yuanwen}

孔子说:“。”

\chapter{尧曰}

\begin{yuanwen}
尧曰:“咨!尔舜!天之历数在尔躬,允执其中。四海困穷,天禄永终。”
\end{yuanwen}

孔子说:“。”

\begin{yuanwen}
舜亦以命禹。曰:“予小子履,敢用玄牡,敢昭告于皇皇后帝:有罪不敢赦,帝臣不蔽,简在帝心。朕躬有罪,无以万方;万方有罪,罪在朕躬。”周有大赉,善人是富。“虽有周亲,不如仁人。百姓有过,在予一人。”谨权量,审法度,修废官,四方之政行焉。兴灭国,继绝世,举逸民,天下之民归心焉。所重:民、食、丧、祭。宽则得众,信则民任焉,敏则有功,公则说。
\end{yuanwen}

孔子说:“。”

\begin{yuanwen}
子张问于孔子曰:“何如斯可以从政矣?”
\end{yuanwen}

孔子说:“。”

\begin{yuanwen}
子曰:“尊五美,屏四恶,斯可以从政矣。”
\end{yuanwen}

孔子说:“。”

\begin{yuanwen}
子张曰:“何谓五美?”
\end{yuanwen}

孔子说:“。”

\begin{yuanwen}
子曰:“君子惠而不费,劳而不怨,欲而不贪,泰而不骄,威而不猛。”
\end{yuanwen}

孔子说:“。”

\begin{yuanwen}
子张曰:“何谓惠而不费?”
\end{yuanwen}

孔子说:“。”

\begin{yuanwen}
子曰:“因民之所利而利之,斯不亦惠而不费乎?择可劳而劳之,又谁怨?欲仁而得仁,又焉贪?君子无众寡,无小大,无敢慢,斯不亦泰而不骄乎?君子正其衣冠,尊其瞻视,俨然人望而畏之,斯不亦威而不猛乎?”
\end{yuanwen}

孔子说:“。”

\begin{yuanwen}
子张曰:“何谓四恶?”
\end{yuanwen}

孔子说:“。”

\begin{yuanwen}
子曰:“不教而杀谓之虐;不戒视成谓之暴;慢令致期谓之贼;犹之与人也,出纳之吝谓之有司。”
\end{yuanwen}

孔子说:“。”

\begin{yuanwen}
孔子曰:“不知命,无以为君子也;不知礼,无以立也;不知言,无以知人也。”
\end{yuanwen}

孔子说:“。”

\end{document}