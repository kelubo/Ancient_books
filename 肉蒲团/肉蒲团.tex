% 肉蒲团
% 肉蒲团.tex

\documentclass[a4paper,12pt,UTF8,twoside]{ctexbook}

% 设置纸张信息。
\RequirePackage[a4paper]{geometry}
\geometry{
	%textwidth=138mm,
	%textheight=215mm,
	%left=27mm,
	%right=27mm,
	%top=25.4mm, 
	%bottom=25.4mm,
	%headheight=2.17cm,
	%headsep=4mm,
	%footskip=12mm,
	%heightrounded,
	inner=1in,
	outer=1.25in
}

% 设置字体,并解决显示难检字问题。
\xeCJKsetup{AutoFallBack=true}
\setCJKmainfont{SimSun}[BoldFont=SimHei, ItalicFont=KaiTi, FallBack=SimSun-ExtB]

% 目录 chapter 级别加点(.)。
\usepackage{titletoc}
\titlecontents{chapter}[0pt]{\vspace{3mm}\bf\addvspace{2pt}\filright}{\contentspush{\thecontentslabel\hspace{0.8em}}}{}{\titlerule*[8pt]{.}\contentspage}

% 设置 chapter 标题格式。
\usepackage{varwidth}
\ctexset{
	chapter/name={第,回},
	%chapter/number={\chinese{chapter}},
	chapter/titleformat= \chaptertitleformat
}
\newcommand\chaptertitleformat[1]{
	\begin{varwidth}
		[t]{.7\linewidth}#1
	\end{varwidth}
}

% 设置 section 标题格式。
\ctexset{
	section/name={},
	section/number={}	
}

% 设置古文原文格式。
\newenvironment{yuanwen}{\bfseries\zihao{4}}

% 设置署名格式。
\newenvironment{shuming}{\hfill\bfseries\zihao{4}}

\title{\heiti\zihao{0} 肉蒲团}
\author{李渔}
\date{}

\begin{document}

\maketitle
\tableofcontents

\frontmatter

\chapter{前言、序言}

李渔(1611-1680)《肉蒲团》内容简介:小说叙述元末的一位名叫“未央生”的书生,立志“作天下第一才子,娶天下第一美女”,于是抛下自己美貌绝伦的妻子,云游四方,猎艳寻欢。他经一次近于荒诞的手术之后,终于实现了自己的愿望,却发现妻子耐不住青春独处的寂寞,受人引诱而沦落风尘。未央生从此大彻大悟,斩断了人不人、狗不狗的“本钱”,削发为僧,终成正果。肉蒲团情节波澜起伏,环环相扣,而且互不重复,新意迭出,所以读起来颇有引人入胜之妙。尤其令人大感兴趣的是,作者对于性交一事,似乎别有天赋,每一个场面,都有独出心裁的地方,比起众多色情小说的陈陈相因、千篇一律来,那是不能同日而语了。它在西方享有盛誉,决非出于偶然。 

\mainmatter

\chapter[止淫风借淫事说法\ 谈色事就色欲开端]{止淫风借淫事说法\\谈色事就色欲开端}

词曰:

\begin{quotation}
黑发难留,朱颜易变,人生不比青松。名消利息,一派落花风。悔杀少年不乐,风流院,放逐衰翁。王孙辈,听歌金缕,及早恋芳药。世间真乐地,算来算去,还数房中。不比荣华境,欢始愁终。得趣朝朝,燕酣眠处,怕响晨钟。睁眼看,乾坤覆载,一幅大春宫。
\end{quotation}

这一首词名曰《满庭芳》。单说人生在世,朝朝劳苦事事愁烦,没有一毫受用处,还亏那太古之世开天辟地的圣人制一件男女交媾之情,与人息息劳苦解解愁烦,不至十分憔悴。照拘儒说来,妇人腰下物乃生我之门,死我之户。据达者看来,人生在世若没有这件东西,只怕头发还早白几年,寿还略少几岁。不信单看世间的和尚,有几人四五十岁头发不白的?有几人七八十岁肉身不倒的?或者说和尚虽然出家一般也有去路,或偷妇人或狎\footnote{xi\'a,玩弄。}徒弟,也与俗人一般不能保元固本,所以没寿这等。请看京里的太监,不但不偷妇人不狎徒弟,连那偷妇人狎徒弟的器械都没有了,论理就该少嫩一生,活活几百岁才是,为何面上的皱纹比别人多些?头上的白发比别人早些?名为公公实像婆婆?京师之内,只有挂长寿匾额的平人,没有起百岁牌坊的内相。

可见女色二字原于人无损,只因《本草纲目》上面不曾载得这一味,所以没有一定的注解。有说它是养人的,有说它是害人物。若照这等比验起来,不但还是养人的物事,他的药性与人参附子相同,而亦交相为用。只是一件,人参附子虽是大补之物,只宜长服,不宜多服;只可当药,不可当饭。若还不论分两,不拘时度饱吃下去,一般也会伤人。女色的利害与此一般。长服则有阴阳交济之功,多服则有水火相克之敝。当药则有宽中解郁之乐,当饭则有伤筋耗血之忧。世上之人若晓得把女色当药,不可太疏亦不可太密,不可不好亦不可酷好。未近女色之际,当思曰“此药也,非毒也,胡为惧之”,既近女色之际,当思曰“此药也,非饭也,胡为溺之”。如此则阳不亢阴不郁,岂有不益与人哉。只是一件,这种药性与人参附子件件相同,只有出产之处与取用之法又有些相反,服药者不可不知。人参附子,是道地者佳,土产者服之无益。女色,倒是土产者佳,倒地者不惟无益且能伤人。何谓土产?何谓倒地?自家的妻妾,不用远求,不消钱买,随手扯来就是,此之谓土产。任我横睡没有阻桡,随他敲门不担惊恐。既无伤于元气,又有益于宗祧\footnote{ti\=ao,远祖之庙。}。交感一翻,浑身通泰。岂不谓之养人?艳色出于朱门,娇妆必须绣户。家鸡味淡,不如野鹜\footnote{w\`u。家鸭,泛指鸭子。}新鲜,旧妇色衰,争似闺雏小艾,此之谓倒地。若是此等妇人,眠思梦想,务求必得,初以情挑,继将物赠,或逾墙而赴约,或钻穴而言私。饶伊色胆如天,倒底惊魂似鼠,虽无人见似有人来。风流汗少而恐惧汗多,儿女情长而英雄气短。试身不测之渊,立构非常之祸,暗伤阴德,显犯明条,身被杀矣。若无偿命之人,妻尚存兮。尤有失节之妇,种种利害惨不可当。可见世上人与女色二字断断不可舍近而求远,厌旧而求新。

做这部小说的人原具一片婆心,要为世人说法,劝人窒欲不是劝人纵欲,为人秘淫不是为人宣淫。看官们不可认错他的主意。既是要使人遏\footnote{\`e,阻止。}淫窒欲,为甚麽不著一部道学之书维持风化,却做起风流小说来?看官有所不知。凡移风易俗之法,要因势而利导之则其言易入。近日的人情,怕读圣经贤传,喜看稗\footnote{b\`ai}官野史。就是稗官野史里面,又厌闻忠孝节义之事,喜看淫邪诞妄之书。风俗至今日可谓蘼\footnote{m\'i}荡极矣。若还著一部道学之书劝人为善,莫说要使世上人将银买了去看,就如好善之家施舍经藏的刊刻成书,装订成套,赔了贴子送他,他还不是拆了塞瓮,就是扯了吃烟,那里肯把眼睛去看一看。不如就把色欲之事去歆\footnote{x\=in}动他,等他看到津津有味之时,忽然下几句针砭之语,使他瞿\footnote{q\'u}然叹息道“女色之可好如此,岂可不留行乐之身,常还受用,而为牡丹花下之鬼,务虚名而去实际乎?”又等他看到明彰报应之处,轻轻下一二点化之言,使他幡然大悟道“奸淫之必报如此,岂可不留妻妾之身自家受用,而为惰珠弹雀之事,借虚钱而还实债乎?”思念及此,自然不走邪路。不走邪路,自然夫爱其妻,妻敬其夫,周南召南之化不外是矣。此之谓就事论事以人治人之法。不但作稗官野史当用此术,就是经书上的圣贤亦先有行之者。不信且看战国齐宣王时孟子对齐宣王说王政。那宣王是声色货利中人,王政非其所好,只随口赞一句道:“善哉信乎。”孟子道:“王如善之,则何为不行?”宣王道:“寡人有疾,寡人好货。”孟子就把公刘好货一段去引进他。宣王又道:“寡人有疾,寡人好色。”他说到这一句已甘心做桀\footnote{ji\'e,古人名。夏朝末代君主名。}纣\footnote{zh\`ou,人名。商代最后一个君主,相传是一个暴君。}之君,只当写人不行王政的回帖了。若把人道学先生,就要正言历色规谏他色荒之事。从古帝王具有规箴\footnote{zh\=en,古代文体的一种。以劝诫为表达的主题。}:“庶人好色,则亡身;大夫好色,则失位;诸侯好色,则失国;天子好色,则亡天下”。宣王若闻此言,就使口中不说,心上毕竟回复道:“这等,寡人病入膏肓,不可救药,用先生不着了。”谁想孟子却如此反把大王好色一段风流佳话去勾住他,使他听得兴致勃然,住手不得。想太王在走马避难之时尚且带着姜女,则其生平好色一刻离不得妇人可知。如此淫荡之君,岂有不丧身亡国之理?他却有个好色之法,使一国的男子都带着妇人避难。太王与姜女行乐之时,一国的男女也在那边行乐。这便是阳春有脚天地无私的主。化了谁人不感颂他,还敢道他的不是?宣王听到此处自然心安意肯去行王政,不复再推“寡人有疾”矣。做这部小说的人得力就在于此。但愿普天下的看官买去当经史读,不可作小说观。凡遇叫“看官”处不是针砭之语,就是点化之言,须要留心体认。其中形容交媾之情,摹写房帷之乐,不无近于淫亵\footnote{xi\`e},总是要引人看到收场处,才知结果识警戒。不然就是一部橄榄书,后来总有回味?其如入口酸涩,人不肯咀嚼何?我这番形容摩写之词,只当把枣肉裹着橄榄,引他吃到回味处也莫厌。

摊头絮繁,本事下回便见。

\chapter[老头陀空张皮布袋\ 小居士受坐肉蒲团]{老头陀空张皮布袋\\小居士受坐肉蒲团}

说话元朝至和年间,括苍山中有一个头陀,法名正一,道号孤峰。他原是处州郡学一个有名诸生。只因性带善根,当其在襁褓之中不住的咿咿晤晤就像学生背书一般。父母不解其故。有个行脚僧上门抄化,见了鬟\footnote{hu\'an}抱在手中,似啼非啼似笑非笑。僧人听之,说他念的是《楞\footnote{l\'eng}严大藏真经》,此子乃高僧转世。就回他父母乞为弟子。父母以为妖言,不信。大来教他读书,过目成诵。但功名之事非其所愿,屡次弃儒学佛,被父母痛惩而止。不得已出来应试,垂髫\footnote{ti\'ao,古代指儿童的下垂的头发。}就入泮\footnote{p\`an},入泮就帮补。及至父母亡后,他待二年服阙\footnote{qu\`e},将万金家产尽散与族人。自己缝一个大皮袋,盛了木鱼经藏等物,落去头发,竟入山修行。知道者称为孤峰长老,不知道的只叫他做皮布袋和尚。与众僧不同,不但酒肉淫邪之事戒得甚坚。就于僧家本等事业之中也有三戒。那三戒是:不募缘,不讲经,不住名山。人问他为什么不募缘,他道:“学佛之事大抵要从苦行入门。须劳其筋骨,饿其体肤,使饥寒之虑日迫。饥寒之虑日迫则淫欲之念不生,淫欲之念不生则秽浊日去,清静日来。久之自然成佛。若还不耕而食,不织而衣,终日靠着施主拿来供养。腹饱则思闲步,体暖则爰\footnote{yu\'an}安眠。闲步而见可欲,安眠即成梦想。无论学佛不成,种种入地狱之事不求而自至矣。我所以自食其力,戒不募缘。”人问他为甚麽不讲经,他道:“经忏上的言语是佛菩萨说出来的,除非是佛菩萨才解得出。其余俗口讲经,尤如痴人说梦。昔陶渊明读书不求甚解。夫以中国之人读中国之书,尚且不敢求甚解,况以中国之人读外国之书,而再妄加翻译乎?我不敢求为菩萨之功臣,但免为佛菩萨之罪人而已。以此知愚守拙,戒不讲经。”人又问何不住名山,他道:“修行之人须要不见可欲,使心不乱。天下可欲之事不独声色货利。就是适体之清风,娱情之皎月,悦耳之禽鸟,可口之薇蕨,一切可爱可恋者皆是可欲。一居胜地,便有山灵水怪引我寻诗,月姊风姨搅人入定,所以如名山读书者学业不成,如名山学道者名根难净。况且哪一处名山没有烧香的女子随喜的仕官?月明翠柳之事乃前车也。我所以撇了名刹来住荒山,不过要使耳目之前无可沽滞的意思。”问者深服其言,以为从古高僧所未发。他因有此三戒,不求名而名日彰。远近之人发心皈依者甚众,他却不肯轻收弟子,要察他果有善根绝无尘念者,方才剃度。略有一毫信不过,便拒绝不收。所以出家多年,徒弟甚少,独自一个在山涧之旁构几间第屋,耕田而食,吸泉而饮。

一日,秋风萧瑟,木脱虫吟。和尚清晨起来,扫了门前落叶,换了佛前净水,装香已毕,放下蒲团,就在中堂打坐。忽有一少年书生,带两个家童走进门来。那书生的仪表生得神如秋水,态若春云。一对眼睛比他人更觉异样光焰。大约不喜正观扁思邪视,别处用不着,唯有偷看女子极是专门。他又不消近身,随你隔几十丈远,只消把眼光一瞬,便知好丑。遇者好的就把眼色一丢。那妇人若是正气的,低头而过,不着到他脸上来,这眼光就算是丢在空处了。若是那妇人与他一样毛病的,这边丢去,那边丢来,眼角上递了情书,就开交不得了。所以不论男子妇人,但生下这种眼睛就不是吉祥之兆,丧名败节皆由于此。看官们的尊目若有类此的不可不慎。彼时这书生走进来,对佛像拜了四拜,对和尚也拜了四拜,起来立在旁边。和尚起先在入定之时不便回礼,待完了工课方才走下蒲团,也深深回了四拜。叙坐已定,就问其姓名。书生道:“弟子乃远方之人,游苏浙中,别号‘未央生’。闻师父乃一代高僧两间活佛,故此斋戒前来,敬仰说话。”

你道那和尚问其姓名,他为何不称名道姓,却说起别号来?看官要晓得元来之时士风诡异,凡是读书人不喜称名道姓,俱以别号相呼。故士人都有个表德,有称为“某生”,有称为“某子”,有称为“某道人”。大约少年者称生,中年者称子,老年者称道人。那表德的字眼也各有取义,或是情之所钟,或是性之所近,随取二字以命名,只要自己明白,不必人人共晓。书生只因性耽女色,不善日而喜夜,又不喜后半夜而喜前半夜,见《诗经》上有“夜未央”之句,故此断章取意名为“未央生”。

当时和尚见他称誉太过,愧不敢当,回了几句谦逊的话。其时瓦铛之中斋饭已熟,和尚就留他吃了晨斋。两个对坐谈禅,机锋甚合。原来未央生性极聪明,凡三教九流之书无不浏览。这禅机里面别人千言万语参不透的,他只消和尚提头一句就彻底了然。和尚心下暗想道,好个有知识的男子,只怪造物赋形有错,为何把一副学佛的心胸配一个作孽的相貌?我看他行容举止分明是个大色鬼,若不把他收入皮布袋中,将来必到钻穴逾墙,酿祸闺阃\footnote{k\v{u}n,借指妇女、妻子。}。天地间不知多少妇人受其涂毒。我今日见了这悖乱之人而不为众人弥乱,非慈悲之道也。就对他道:“贫僧自出家以来阅人多矣。那些愚夫愚妇不肯向善的固不足道,就是走来参禅的学士,听法的宰官也都是些门外汉,能悟禅机的甚少。谁想居士竟有如此灵明,以此学禅不数年可登三味。人生在世,易得者是形体,难得者是性资。易过者是时光,难过者是劫数。居士带了作佛的资性来,不可走到鬼魅的路上去。何不趁此朝气未散之时,割除爱欲,遁入空门。贫僧虽是俗骨凡胎,犹堪作他山之石。果能发此大愿,力注此大因果,百年后上可配享于僧伽,下亦不至听命于罗刹。居士以为何如?”未央生道:“弟子归禅之念蓄之已久,将来少不得要归此法门。只是弟子尚有二愿未酬,难于摆脱。如今年纪尚幼,且待回去毕了二事,安享数年。到那时然后来摩顶皈依,未为晚也。”

和尚道:“请问居士有哪二愿?莫非是要策名天府,下酬所学?立功异域,上报朝廷么?”

未央生摇头道:“弟子所愿不是这二事。”

和尚道:“既不是这二事,但所愿者毕竟是何事?”

未央生道:“弟子所愿者乃是自己力量做得来的,不是妄想的事。不瞒师父说,弟子读书的记性,闻道的悟性,行文的笔性,都是最上一流。当今的名士不过是勉强记诵,移东换西,做几篇窗稿,刻一部诗文,就要树帜词坛,纵横一世了。据弟子看来那是假借,要做真名士毕竟要读尽天下异书,交尽天下奇士,游尽天下名山,然后退藏一室,著书立言传于后世。幸而挂名两榜,也替朝廷做些事业,万一文福不齐老于墉下,亦不失为千古之人。故此弟子心上有私语二句道:要做世间第一个才子,……”

和尚道:“这是第一句了。那第二句呢?”

未央生待开口又复吞声不好说出的意思。和尚道:“第二句居士既然怕讲,待贫僧替说了吧。”

未央生道:“弟子心上的事,师父那里说得出?”

和尚道:“贫僧若说不着,情愿受罚。只是说着了,居士不要假推不是。”

未央生道:“师父若说得着,不但是菩萨又是神仙了,岂敢遁词推托?”

和尚不慌不忙道:“是‘要娶天下第一位佳人’”。

未央生听了不觉目瞪口呆,定了半晌,方才答道:“师父真异人也!这两句私语是弟子心上终日念的,师父竟像听见了一般,一口就着着了。”

和尚道:“岂不闻人间私语天闻若雷乎?”

未央生道:“论起理来,情欲之言本不该对师父讲。今师父既猜着,弟子不敢瞒师父说,弟子道心尚浅,欲念方深。从古以来‘佳人才子’四个字再分不开,有了才子定该有佳人作对,有了佳人定该有才子成双。今弟子的才华且不必说,就是相貌也不差。时常引镜自照,就是潘安、卫介生在今时,弟子也不肯多让。天既生我为才子,岂不生一个女子相配?如今世上若没有佳人则已,倘或有之,求佳偶者非弟子而谁?故此弟子年过二十尚未定亲,是不肯辜负才貌的意思。待弟子回去觅着佳人成了配偶,生一子以继宗祧,那时节良愿已酬无复他想,不但自己回头,亦当劝化室人同登彼岸。师父以为何如?”

和尚听了冷笑道:“这等看来居士的念头一毫不差,只是生人造物的天公有些不是。若把一副丑陋形骸付与居士,居士具一点不昧之灵,或者能于正果。所以古来之人常有瘌疾痫症,手折足翘,因受天刑而成仙。仙人也就是这种道理。居士只因赋形之时天公忒骄纵了些,就如父母爱子一般,幼少之时唯恐损伤皮肉,恼壤性情,不忍打他一下,骂他一句。儿子大来,只说皮肉性情是天地生成的,父母养就的,所以任意去为非作歹。犯下罪来受官府之鞭笞,遭朝廷之刑戮,方恨父母骄纵太过,至有今日。这副细异皮肉、骄纵性情不是好祥瑞也。居士因你的相貌是第一个才子就要去寻第一位佳人,无论佳人可得不可得,就使得了一位,只恐这一位佳人额角上不曾注写‘第一’的两个字。若再见了强似他的,又要翻转来那好的。这一位佳人若与居士一般生性,不肯轻易嫁人要等第一个才子,居士还好娶来作妾。万一有了良人,居士何以处之?若千方百计必要求遂所愿,则种种堕地狱之事从此出矣。居士还是要堕地狱乎?上天堂乎?若甘心堕地狱,只管去寻第一位佳人。若要上天堂,请收拾了妄念,跟贫僧出家。”

未央生道:“师父说‘天堂地狱’四个字,未免有些落套,不似高僧之言。参禅的道理不过是要自悟。本来使身子立在不生不灭之处便是佛了。岂真有天堂可上乎?即使些有风流罪过亦不过玷辱名教而已,岂真有地狱可堕乎?”

和尚道:“‘为善者上天堂,作恶者堕地狱’果然是套话。只是你们读书人事事俱可脱套,唯有修身立行之事一毫也脱不得。无论天堂地狱,明明不爽。即使没有天堂,不可不以天堂为向善之阶。即使没有地狱,不可不以地狱为作恶之戒。你既言明套话,我今不说将来的阴报,只说现在的阳报,少不得又是套话。古语有云‘我不淫人妻,人不淫我妇。’这两句是极平常的套话,只是世上贪淫之人不曾有一个脱得套去,淫人妻女,妻女亦为人所淫。若要脱套,除非不奸淫则已。若要奸淫,少不得要被套话说着。居士还是要脱套乎,要入套乎?若要入套,只管去寻第一位佳人;若要脱套,请收拈\footnote{ni\=an}了妄念跟贫僧出家。”

未央生道:“师父所言讲的样样透彻。只是为愚蒙者说法,不得不讲个尽情,使他听得毛骨悚然,才知警戒。若对我辈说理亦未必如此。天公立法虽严,行法亦未尝不恕。奸淫必报者虽多,奸淫不报者亦未尝不少。若挨家逐户去访缉奸淫,淫人妻女者亦使其妻女偿人淫债,则天公亦其亵矣!总之循环之道,报施天理,大概不爽,为人不善者不可不知道,就是劝化的大题目了,何必如此纳柱?”

和尚道:“照居士这等说来,世上的奸淫亦有不报的么?只怕大公立法并不曾使人漏网。或者居士忠厚,略有使人漏网处。据贫僧看来,淫人妻女而不报者古今并没有一个。书史所载,俗口相传者,盈千累万。居士请试想之,淫人妻女是得便宜的事,肯对人说,故知道的多。妻女被淫是失便宜的事,不肯对人说,故知道的少,内中还有妻瞒其夫,女瞒其父,连自家也不知道,还说奸淫之报必无此事。直到盖棺之后,方信古语不诬,到那时节这了悟的话又对人说不出了。无论奸人的妻女,才以妻女偿人淫债。只奸淫之念一动,此时妻女之心不知不觉也就有许多忘了。譬如自家的妻女生得丑陋,夜间与他交媾不十分起兴,心上想着日间所见的标致女子,把妻子权当了他,自取其乐。焉知此时妻子心上不嫌丈夫丑陋,想着日间所见的标致男子,把丈夫权当了他,自取其乐?此等事人人有之,虽无损于冰霜之操,颇有伤于匪石之心。亦男子好淫之报也。举心动念尚且如此,何况身入其室,体压其层而鬼神不见,造物不嗔,使妻子为全节之妇乎!贫僧此言却不是套话。居士以为然否?”

未央生道:“极讲的入理,只是还要请问师父,妻女者淫了人的妻女还有妻女相报,倘若无妻女者淫了人的妻女,把甚么去还债?这大公的法度也就行不去了。还有一说,一人之妻女有限,天下之女色无穷。譬如自家只有一两个妻妾,一两个儿女,却淫了天下无限的妇人,即使妻女坏事,也就本少利多了。天公将何以处之?”

和尚听了,知他大块顽石推移不动的人,就对他道:“居士谈锋甚利,贫僧就不敢当。只是这种道理口说无凭,直待做出来方见明白。居士请自待娶了佳人之后,从肉蒲团上参悟出来,方得实际。贫僧观居士有超凡入圣之具,登岸造极之资,实不忍舍万一到豁然大悟之后,还要来见贫僧,商量归路。贫僧从明日起终朝拭目以待。”说罢,取出笺\footnote{ji\=an}纸提起笔来,写五言四句的一首偈\footnote{j\`i,佛经中的唱词(梵语音译词“偈陀”的简称)。}道:

\begin{quotation}
请抛皮布袋,去坐肉蒲团。须及生时悔,休嗟\footnote{ji\=e}已盖棺。
\end{quotation}

和尚写完递与未央生道:“粗笨头陀,不识忌讳,偈语虽然太激,实出一片婆心。屈居士留之,以为后日之验。”说完立起身来,竟像要送他的意思。未央生知道见绝,又念他是个高僧,不敢悖悖而去,只得低头陪罪道:“弟子赋性愚顽,不受教悔,望师父海涵。他日重来,尚祈收纳。”说罢依旧拜了四拜,和尚也一般回礼送他出门,分别而去。那和尚的出处言之已尽,后面只说未央生迷恋女色事,不复容叙孤峰,要知孤峰结果到末回始见。

评曰:未央生是一本戏文的正生,孤峰乃末脚也。他人执笔,定将未央生说起,引孤峰作过客。此独叙孤峰,极其详悉,使观者疑孤峰后来或有淫行,谁料却又不然。直到打座参禅才露出正意来,使人捉摸不定。此从来小说之变体,乃作者辟尽窠臼处。即使他人用此法必至题旨错乱,头绪纷然,使观者不辨谁宾谁主。此独眉眼分明,使人看到入题处俱自了然。末后数语又提清线路,不复难为观者,真老手也。

\chapter[道学翁错配风流婿\ 端庄女情移薄情郎]{道学翁错配风流婿\\端庄女情移薄情郎}

却说未央生别了孤峰,一路叽叽哝哝的埋怨道,好没来头。我二十多岁的人,一朵鲜花才开,就要教人削发修行,去寻苦吃。世上那有这样不情的人。我今日见他不过是因他是由名士出家,胸中必有别样见解,要领略他禅机,好助我的文思。谁想竟受他许多怠慢,又做一首乌龟偈赠我,教我怎当得起?我一个昂藏的丈夫,若做了官还要治天下,管万民,难道自家妻子就管不下?我今遇着好妇人,偏不肯当面错过。略做几桩风流罪犯,把自家闺门严谨,看有个男子来讨得债去。况且有妇人嫁我这样标致丈夫,就有别个男子来引诱她只怕也看不上眼。那失节之事料定是没有的。他方才那一首偈,论理就该扯碎了丢还他。只是后来相见要塞他毒口没有凭据,我且留在身边,看他后来见了,悔过不悔过。思量已定就将偈语折好藏在衣带中。

回到家里,吩咐几个伴当各路去传谕媒婆,要寻世间第一位佳人。他原是个阀阅之家,又兼才貌双全,哪一个男子不愿得他为婿,哪一个妇人不愿得他为夫?自从传谕之后,日日有几个媒婆寻他说亲。小户人家任凭他上门去相,若是大户人家要顾体面,或约在寺院中,或定在荒郊外,俩下相逢,以有心装作无意,相得分明。惹了多少妇人回去害相思,他却个个都看不上眼。有个媒婆对他道:“这等看来别的女子都不是你的对头,只有铁扉道人的小姐名叫‘玉香’,才配得你上。只是他父亲古怪,定不肯使人相,你又定要相,这事又是做不来的了。”

未央生道:“他为何叫做‘铁扉道人’?你为何见得他小姐标致?既然标致,为何不肯使人相?”

媒婆道:“这老者是有名的宿儒,做人孤介。家中有田有地无求于人,生平没有一个朋友,独自一个在家读书,随你甚么人去敲门,他只是不开。有一个贵客慕他的名去访他,敲了半日门,莫说不开,连答应也不答应。那贵客没奈何,题诗一首写在门上而去。中间有两句道:‘但知高士篷为户,谁料先生铁为扉’。他后来见了诗句道:‘铁扉两字道得不差,’他就把做别号叫做‘铁扉道人’。生平没有儿子只得一女,生得如花似玉,无人可比。又且读了一肚子书,都是父亲所教,凡诗词歌赋皆做得出。他家的闺门严谨,又不走去烧香,又不出来看会,长了一十六岁不曾出头露面,至于三姑六婆飞不进门。因昨日那老者立在门前,见我走过叫住问道:‘你莫非是做媒的么?’我答道:‘正是。’他就请我到家中指着女儿对我道:‘这是我的小姐,要招个像样的女婿当儿子养老。你可留心替我访择。’我就把相公说上,他道:‘我也闻得他的才名,但不知德行何如?’我又道:‘相公少年老成,毫无破绽。只是一件,他要亲眼相一相才肯下聘。’他听得这句话就放下脸道:‘胡说!只有扬州人家养的瘦马肯与人相,那有正经女儿许男子见面之理。’我见他说了这话不好再讲,竟自出来。故此知道这头亲事定做不成。”

未央生闻言心中暗想道:“我如今上无父母下无兄弟,明日娶了妻,心性哪一个拘管?就是自己行监坐守难道没有出门的时节?这老儿的古板如此,我若赘在他家,不消我去提防,他自家的女儿自然会照管,我就出门一世也不妨事。只是不得相一相究竟不放心,媒人的口,那里信得。就对他道:“照你说来亲事是极好的,毕竟求你设个法子,使我窥见些影响,只要大段不差也就罢了。”

媒婆道:“这个断断不能。你若不信,只好去求签问数,卜之于神。该做就做,不该做就罢。”

未央生道:“也说的是。我有个朋友,请仙判事及其灵验,待我请他来判断过了,然后回你的话。”媒人答应而去。

次日未央生斋戒沐浴,把请仙的朋友延至家中。焚香稽首,低声祝道:“弟子不为别事,只因铁扉道人之女名唤玉香。闻得他姿容绝世,要娶为妻,但属耳闻未曾目击,所以请问于大仙。果姿容绝世,弟子就与他连姻,稍不然即行谢绝。伏望大仙明白指示,勿为模糊之言,使弟子参详不出。”祝完又拜四拜,起来扶住仙栾,听其挥写。果然写出一首诗道:

\begin{quotation}
红粉丛中第一人,不须疑鬼复疑神。

只愁艳冶将淫诲,邪正关头好问津。

右其一
\end{quotation}

未央生见了这一首,心上思道:“这等看来姿色是好的,只是后一句明白说她冶容诲淫,难道这女人已被人破了瓜去不成?诗后既有‘其一’二字,毕竟还有一首,且看后作何如。”只见仙栾停了一会,又写出四句道:

\begin{quotation}
妇女贞淫挽不差,但须男子善齐家。

闭门不使青蝇入,何处飞来玉上瑕。

右其二

回道人题
\end{quotation}

未央生见了“回道人”三字知是吕纯阳的别号,心上大喜道:“此公于酒色二字极是在行,他说好毕竟是好的了。后面这一首是又我心中之疑不通,要我堤防的意思。我想这古板丈人替我拘管,料然无事。后两句明明说他铁扉之中无人钻得进的意思,不必再疑惑了。就望空拜谢了纯阳,叫人唤媒婆来。吩咐说:“仙诗判得甚好,如今不消去相瞒,竟去说亲罢了。”

媒人甚喜,走到铁扉道人家,把未央生求亲的意思述了一遍。道人道:“他起先要亲眼相亲,就是重色不重德的人了,轻薄可知。我要招个有品行的女婿,不要这等务外之人。”

那媒婆要趁媒钱,只得把巧话回复道:“他要相的意思不是为色,只怕举止轻佻,没有福相,后来不得夫人。故今访得府上的闺训甚严,小姐的阃德又备,故此心安意肯,特地央我来求亲。”

道人道见他说的近理,就许了亲约,定吉日过门完姻。

未央生虽听了媒人之话,信了仙诗之言,只因不曾相得,到底狐疑。直到成亲之夜,拜堂已毕,同入绣房,定睛细看,方才欢喜。怎见得新人的好处?有新词一首为证:

\begin{quotation}
人窈窕,浑身满面都堆俏。
都堆俏,愁容可掬,颦眉难效。

还愁不是新人料,腰肢九细如何抱?
如何抱,柔如无骨将又惊靠。

右调《忆秦娥》
\end{quotation}

怎见得新郎与新人成亲的乐处?也有新词一首为证:

\begin{quotation}
星眸合处差即盼,枕上桃花歌两瓣。

多方欲闭口脂香,却被舌功唇已绽。

娇啼歇处情何限,酥胸已透风流汗。

睁开四目互相看,两心热似红炉炭。

右调《玉楼春》
\end{quotation}

却说玉香小姐姿容虽然无双,风情未免不足,还有一二分不中丈夫的意。只因平日父训既严,母仪又肃,耳不闻淫声,目不睹邪色,所读之书不是《烈女传》就是《女孝经》,所说的话都与未央生心事相反。至于举止,不免有乃父之风,丈夫替他取个混名叫“女道学”。对他说一句调情的话就满面通红,走了开去。未央生极喜日间干事,好看阴物以助淫兴。有几次扯她脱裤,她就大喊起来,却象强奸她的一般,只得罢了。夜间干事,虽然承当,都是无可奈何的光景与见。行房的套数只好行些中庸之道,不肯标新立异。要做“隔山取火”,就说犯了背夫之嫌。要做“倒浇蜡烛”,又说倒了夫纲之礼。要搭她两脚上肩,也费许多气力。至于快活之时不肯叫死叫活,助男子的军威,就唤她心肝命肉,竟象哑妇一般,不肯答应。

未央生见她没有一毫生动之趣,甚以为苦。我今只得用些淘养的工夫,变化她出来。明日就书画铺中买一副绝巧的春宫册子,是学士赵子昂的手笔,共有三十六幅,取唐诗上三十六宫都是春的意思。拿回去与玉香小姐一同翻阅,可见男女交媾这些套数不是我创造出来的,古人先有行之者,现有赵文敏墨卷在此,取来证验。

起初拿到之时,玉香不知里面是甚麽册,接到手中揭开细看,只见开卷两页写着“汉宫遗照”四个大字。玉香想道,汉宫之中有许多贤妃淑媛,一定是些遗像,且看是怎生相貌。及到第三页,只见一个男子搂着一个妇人,赤条条在假山上干事,就不觉面红发起性来道:“这等不祥之物,是从那里取来的?玷污闺阃,快叫丫鬟拿去烧了。”

未央生一把扯住道:“这是一件古董,价值百金。我问朋友借来看的。你若赔得百金起只管拿去烧,若赔不起,好好放在这边,待我把玩一两日拿去还他。”

玉香道:“这样没正经的东西看它何用?”

未央生道:“若是没正经的事,那画工不去画他,收藏的人也不肯出重价去买他了。只因是开天辟地以来第一件正经事,所以文人墨士拿来绘以丹青,裱以绫绢,卖于书画之肆,藏于翰墨之林,使后来的人知所取法。不然阴阳交感之理渐渐沦没,将来必至夫弃其妻,妻背其夫,生生之道尽绝,直弄到人无焦类而后止。我今日借来不但自己翻阅,也要使娘子知道这种道理绝好受胎怀孕,生男育女,不致为道学令尊所误,使夫妻后来没有结果的意思。娘子怎么发起恼来?”

玉香道:“我未信这件勾当是正经事。若是正经事,当初立法的古人何不教人明明白白在日间对着人做?为何在更深夜静之时,瞒了众人就像做贼一般,才行这件勾当?即此观之,可见不是正经事。”

未央生笑道:“这等说来怪不得娘子,都是你令尊不是。把你关在家中,没有在行的女伴对汝说说风情,所以孤陋寡闻,不晓人事。你想,世上的夫妻那一对不在日里去干事?那干事不是明公正气使人知道的?若还夫妻日里不行房,这画画之人怎么晓得这些套数?怎么描写得这样入神,使人一看就动兴起来?”

玉香道:“这等,我家父母为甚么不在日间做事?”

未央生道:“请问娘子,怎见得令尊令堂不在日间做事?”

玉香道:“他们若做事,我毕竟撞着。为何我生长一十六岁并不曾撞着一次?莫说眼睛不曾看见,就是耳朵也不曾听见?”

未央生笑道:“好懵懂妇人!这桩事只是儿女看见不得,听见不得。除了儿女,其余丫鬟使婢哪一个不看见?哪一个不听见?他们要做事毕竟晓得你不在面前,把门闭了,然后上场。若被你看见就怕引动春心,思想男子,生出郁病来。故此瞒着你做。”

玉香想了一会道:“他们日里也常关门睡觉,或是干此事也未可知。只是羞人答答的,你看我我看你,如何做得出来?”

未央生道:“日里行房比夜间的快活更加十倍。其间妙处正在我看你你看我,才觉得动兴。世间只有两种夫妻断不可在日间干事。”

玉香道:“哪两种夫妻?”

未央生道:“丑陋丈夫标致妻子,此一种也。丑陋妻子标致丈夫,又一种也。”

玉香道:“为何这两种人日间做不得事?”

未央生道:“做这事全要你爱我我爱你,精神血脉彼此相交,方才会快活。若是妻子生得肌肤雪白,又娇又嫩,就像美玉琢成的一般,丈夫把他衣脱了搂在怀中,一面看一面干,自然兴高十倍。那阳物不觉又坚又硬,又粗又大了。只是女子看见男人就像鬼怪一般,身上皮肉又黑又粗。穿了衣服还不觉,此时脱了丑态毕露,掩饰不来。况与雪白肌肤相映,八分丑陋就觉有十二分。妻子看了岂不憎嫌?心上既然憎嫌就要形与词色,男子看见不知不觉坚硬的也软了,粗大的也细了。快活事不曾做得,反讨一场没趣。不如在夜里行房,还可以藏拙。这是标致妻子与丑陋丈夫干事的样子。那标致丈夫与丑陋妻子行房的情敝也与此一般,不消再讲。若是我和你这样夫妻,白对白红对红,娇嫩对娇嫩,若不在日间取乐,显一显皮肤,终日钻在被窝里面暗中摸索,可不埋没了一生,与丑陋夫妻何择?娘子不信,我和你试一试,看比夜间的滋味何如?”

玉香到此处不觉有些省悟,口里虽然不肯,心上却要顺从,但觉两腮微红,骚容已露。未央生暗想,她有些意思来了。本要下手,只是此女欲心初动,饥渴未深,若就与她做事,譬如馋汉见了饮食,信口直吞,不知咀嚼,究竟没有美处。我且熬她一熬,然后同她上场。就扯一把太师椅,自己坐了,扯她坐在怀中,揭开春宫册子一幅一幅指与她看。那册子与别的春意不同,每一幅上前半页是春宫,后半页是题跋。那题跋的话前几句是解释画面上的情形,后几句是赞画工的好处。未央生教她存想里面神情,将来才好模仿,就逐句念与她听道:

第一幅乃纵蝶寻芳之势。

跋云:女子坐太湖石上,两足分开。男手以玉麈\footnote{zh\v{u}}投入阴中,左掏右摸以探花心。此时男子妇人俱在入手之初,未逢佳竟,故眉眼开张,与寻常面目不甚相远也。

第二幅乃教蜂酿蜜之势。

跋云:女子仰卧锦褥之上,两手着实,两股悬空,以迎玉尘,使男子识花心所在,不致妄投。此时女子的神情近于饥渴,男子的面目似乎张惶,使观者代为之急,乃化工作恶处也。

第三幅乃迷鸟归林之势。

跋云:女子倚眠绣床之上,双足朝天,以两手扳住男人两股往下直舂。似乎佳境已入,能恐复迷,两下正在用工之时,精神勃勃。真有笔飞墨舞之妙也。

第四幅乃饿马奔槽之势。

跋云:女子正眠榻上,两手缠抱男子,有如束缚之形。男子以肩取他双足,玉麈尽入阴中,不得纤毫余地。此时男子妇人俱在将丢未丢之时,眼半闭而尚睁,舌将吞而复吐,两种面目一样神情。真化工之笔也。

第五幅乃双龙斗倦之势。

跋云:妇人之头倚于枕侧,两手贴伏,其软如绵。男子之头又倚于妇人颈侧,浑身贴伏,亦软如绵,乃已丢之后。香魂欲去,好梦将来,动极近静之状。但妇人双足未下,尚在男子肩臂之间,尤有一线生动之意。不然竟像一对已毙之人,使观者悟其妙境有同棺共穴之思也。

玉香看到此处不觉骚性大发。未央生又翻过一页,正要指与她看,玉香就把册子一推,立起身来道:“甚么好书,看得人不自在。起来你自己看,我要去睡了。”

未央生道:“还有好光景在后面,一起看完了同你去睡。”

玉香道:“难道明天没有日子,定要今天看完?”

未央生知她急了,就搂住亲嘴。往常亲嘴把舌头送过去,她的牙门紧闭不开,若要她伸过来一发不能够了。做过一月夫妻还不知舌长舌短。此番才靠朱唇,她的舌头已不知不觉度过两重牙门来了。

未央生道:“心肝,我和你不消上床就把这太师椅当了假山石,照册上的光景摹拟一番何如?”

玉香假意恼道:“这岂是人干的事?”

未央生道:“果然不是人干的事,乃神仙干的事。我和你权做一刻神仙。”就手解她裤带。

玉香口虽不允手却允了,搭在未央生肩上,任他把裤子脱下。只见裤裆之中湿了一大块,乃看画之时淫水流出的原故。未央生把自家裤子也脱了,扯他坐在椅上,两脚分开,将玉麈插入阴中,然后脱她上身的衣服。为甚麽起先不脱衣服,直到脱裤之后才解上衣?要晓得未央生是个在行的人,若先脱她上面衣服,她心上虽然着急外面还要怕羞,必竟有许造作。故先把要害处据了,其余的地方自然不劳而定。这是行兵擒王捣穴的道理。

玉香果然凭他把一身的衣服脱得精光,唯有脚上的褶\footnote{zh\v{e}}裤不脱。这是何故?原来褶裤里面就是足脚,妇人裹脚之时只顾下面齐整,十指未免参差,没有十分好处。况且三寸金莲毕竟要褶裤罩在上面才觉有趣。不然就是一朵无叶之花,不耐看了。所以未央生得窍只除这一件不脱。替她脱完之后把自己的衣服也尽脱下,然后大整旗枪,分开小脚架在椅上,挺起玉麈向阴中左掏右摸,也像第一幅春宫探觅花心的光景。掏摸了一会,玉香就把两手伸直抵住交椅,把阴户凑上来迎合玉麈。玉麈往左,以左承之。玉麈往右,以右承之。忽然抵着一处,觉得里面似酸非酸,似痒非痒,使人当不得又使人离不得的光景,就对未央生道:“如今只是这样罢了,不要左掏右摸,搔坏了人。”

未央生知道花心已得,就依了她。并力只攻一处,由浅入深,由宽而紧,提了数百提。又见玉香的两手不觉来在身子后面扳住两股向上,直凑与第二幅春宫的光景自然相合。未央生就把她双足提起放在肩上,以两手抱住纤腰,尽根直抵。此时玉尘更觉粗大,塞满阴中。又提了数百提,只见她星眼将朦,云鬓欲坠,却像要睡的光景。未央生扑两扑道:“心肝,我知道你要丢了。这椅上难为人,到床上去完事罢。”

玉香正在要紧头上,恐怕走上床去未免要取出玉麈来,把快活事打断了。况且此时手酸脚软动弹不得,要走也走不上床。闻他一说这一句只是闭了双眼摇头不应。未央生道:“心肝,你莫非走不动麽?”

玉香把头点一点。未央生道:“待我抱你上去就是。”竟把她双足架在手臂上。玉香双手抱住未央生,口里含了绛舌。未央生抱将起来,玉尘留在阴中并不抽出,一边行走一边抽送做个走马看花的势。

抱到床上,把玉香放倒,架起双足从头干起。再抽数百抽,玉香忽然叫道:“心肝,我要不好了!”双手紧紧搂住未央生,口里哼哼吓吓就像大病之人要绝命的一般。未央生知道阴精已生,把玉麈顶住花心用力一揉,也陪她似死。

两个抱住睡了一刻,玉香醒转来道:“我方才死了去你知道么?”

未央生道:“我怎么不知道,这不叫死叫做丢。”

玉香道:“怎么叫做丢?”

未央生道:“男有阳精女有阴精,干到快活尽头处,那精就来了。将来未来之时,浑身皮肉连骨头一齐酥麻起来,昏昏沉沉竟像睡去一般,那精才得泄。这就是丢了。那春宫第五幅就是这个模样。”

玉香道:“照你说来丢过之后还会活,竟是不死的么?”

未央生道:“男子与妇人干一次丢一次,还有阴有来得快的妇人,男子丢一次她丢几十次的。这叫做快活,那里会死!”

玉香道:“既是如此,从今后我日日要丢,夜夜要丢了。”

未央生大笑道:“何如我劝你不差么!这副春宫册子可是件宝具么?”

玉香道:“果然是件宝具。若买来放在家里常看看也是好,只怕那朋友要来取去。”

未央生道:“那是哄你的话,其实是我自己买的。”

玉香听了欢喜。两个说完起来穿了衣服再看春宫,看到兴高之处重新又干。夫妇二人从这日起分外相投,愈加恩爱。

玉香自看春宫之后,道学变做风流。夜间行房不行中庸之道,最喜标新立异。蜡烛也肯倒浇,隔山也容取火。干事之时骚声助兴的狂态渐渐在行。未央生要助她淫性,又到书铺中买了许多风月之书,如《绣塌野史》、《如意君传》、《痴婆子传》之类,共有一二十种。放在案头任她翻阅,把以前所读之书尽行束之高阁。夫妇二人枕席之欢就画三百六十幅春宫也还描写不尽。真是:

\begin{quotation}
琴瑟不足喻其和,钟鼓不能鸣其乐。
\end{quotation}

未央生至此可谓快乐之极矣,只是一件,夫妇里面虽然和谐,翁婿之间甚觉不合。为甚么原故?只因铁扉道人是个古执君子,喜质朴恶繁华,忌说风流爱讲道学。自从未央生入赘之夜见他衣服华丽,举动轻浮,心上就觉有懊恼。叹一口气道:“此子华而不实,必非有成之器。吾女失所规矣。”只是聘礼已收,朱陈已结,不可改移,只得将错就错,等他成亲后以严父拘管,把他磨炼出来,做个方正之士。所以词色之间毫不假借,莫说言语舛\footnote{chu\v{a}n}错,做事差池定要呵斥他教训他,就是行起坐卧稍有不端正处,亦要聒絮一番。

未央生少年心性,父母早亡,不曾有人拘束,那里受得这般磨难?几次要与他相抗,只怕妻子有所不安,有妨琴瑟之乐,没奈何只得隐忍。忍到后来忍不过了,心上思量道,我当初不过慕他女儿,因他不遣嫁定要招人,我所以来就他。他如何竟把太山势来压我。他那样一个腐儒我不去变化他也罢了,他反要来变化我。况且我这一个风流才子将来正要做些窃玉偷香脍炙人口的事,难道靠他一人女儿就勾我终身大事不成?都像这等拘管起来,一步路也不许乱走,一句话也不容多说,若还做出分外事来倒不问我一个死罪?我如今思量与他拗又拗不得,忍又忍不过,只有一着,除非把女儿交托与他,只说出门游学,且往别处走走。如今世上第一位佳人已被我娶着,倘若遇见第二位纵不能够娶他,便做几夜露水夫妻,了了夙缘也是好的。

主意定了,要先对玉香说过,然后请问丈人,又怕玉香贪恋枕席之欢不放我去,若先受她一番阻挠就不好再对丈人说了。只得瞒了玉香背后告丈人道:“小婿僻处山邑,孤陋寡闻,上少明师下无益友,所以学问没有长进之日。如今要拜别岳父,游艺四方,使眼界略宽,胸襟稍大。但见有明师益友之处就在那边下帷,遇了场期就到省中应试,或者博得一科两榜也不枉岳父招赘一场。不知肯容小婿去么?”

铁扉道人道:“你在我家做了半年女婿,只有这一句话才堪入耳。肯离家读书是极好的事,我为甚么不肯?”

未央生道:“岳父虽然见允,只怕令爱怪小婿寡情,新婚未几就要远出。如今照小婿的意想,只说出自岳父之心非干小婿之事,方才没有牵带,可以率意径行。”

道人道:“极说得是。”

商量定了,道人当着女儿劝未央生出门游学,未央生假意不肯,道人正颜厉色苦说一番,未央生方才依命。玉香正有得趣之时,忽然听得丈夫要去,就像小孩子要断乳一般,那里苦得过?连出门以后的欠账都要预支了去。未央生也晓得长途寂寞,一时未必有妇人到手,着力承奉。就像办酒席的一般,虽然是为客而设,也落得自家奉陪。一连几夜的绸缪,真是别人替他说不出,只好夫妻自家知道而已。到临行之时,未央生别了丈人妻子,带了家童随身而去。此后未央生奇遇尚多,静听下回分解。

评曰:说道理劝人使听者毛发俱竦,说情欲动人又令观者神魂俱荡。不知者以首鼠两端为作者病,殊不知委曲动人处正是刻意劝人处。但思玉香未看春宫以前是何等正气?既观题跋以后是何等淫欲?贞淫贵贱判于顷刻之间,皆男子导淫之过也。为丈夫者可不慎哉?

\chapter[宿荒郊客心悲寂寞\ 消长夜贼口说风情]{宿荒郊客心悲寂寞\\消长夜贼口说风情}

未央生别了丈夫妻子,出门游学。信足所至,没有一定的方向,只要有标致妇人的所在,就是他安身立命之乡。每过一府一县,定要住几日。他是个少年名士,平日极考得起,又喜结社,刻的文字最多。千里内外,凡是读书人没有一个不知道他的,所以到一处就有一处朋友拉他入社。他把作文会友当了末着,只有寻访佳人是他第一件要紧。每日清晨起来,不论大街小巷定去寻历一边。所见的都是寻常女子,再不见有天姿国色。

一日在荒郊旅店之中,两个伴当一齐生起病来,动身不得。要出门走走,没个跟随的人,怕妇人家见了不象体面,独自一个坐在下处甚觉无聊。忽见隔壁房里有个同下的客人走过来道:“相公独坐未免寂寞,小人有壶酒在那边,若不弃嫌,请过去同饮一杯何如?”

未央生道:“萍水相逢,怎好奉扰?”

那人道:“我闻得读书人是极喜脱略的,相公为何这等拘执?小人虽是下贱之人,极喜结朋友,只是相公前程远大,不敢高攀。如今同在旅店中也是难逢难遇,就屈坐一坐何妨?”

未央生正在闷极之中,巴不得扯人讲话,就应允了。同他过去,他把未央生送在上面,自己坐在旁边。未央生再三不肯,扯他对坐,那人就问姓名。未央生把自己的别号说了,也问他是何尊号。那人道:“小人是个俗子,没有别号。只有个浑名叫做‘赛昆仑’。”

未央生道:“这个尊称来的异样。为何取这三个字?”

那人道:“若说起来只怕相公害怕,不屑与小人对饮了。”

未央生道:“小弟也是豪侠之人,随你神仙鬼怪立在面前也不怕的。至于贵践贤愚一概不论,只要意气相投,有甚麽不屑!”

赛昆仑道:“这等就不妨直说了。小人平日是个做贼,能飞墙走壁,随你几千丈的高楼,几百层的厚壁,我不消些气力就直入他卧榻之中,把东西席卷出来。不盗第二日也不使他知道。人说当初有个昆仑,能飞入郭令公府中盗取红绡出来。他一生一世不过做得一次,我不知做了几百次,故此把我叫做‘赛昆仑’。”

未央生大惊道:“你既然久做此事,又出了名,人人晓得,难道不犯出事来?”

赛昆仑道:“若犯出事来就不为豪杰了。自古道‘拿贼拿脏’,脏拿不着,我就对他说,他也不敢奈何我。远近的人没有一个不奉承我,惟恐得罪了我,要算计他。我生平有些义气有‘五不偷’:遇凶不偷,遇吉不偷,相熟不偷,偷过不偷,不提防不偷。”

未央生道:“这五种名目来的有意思了,请逐件说明。”

赛昆仑道:“人家有凶事,或是生病或是居丧,或是有飞灾奇祸,他正在急难之中,我若去偷他,如火上添油,他一发当不起了。我所以不去。人家有喜事,或是嫁娶或是起盖,或是生子寿诞,他正在吉庆头上,我若去偷他,使他没有好彩头,将来做事就蹭\footnote{c\`eng}蹬\footnote{d\`eng}了。我所以不去。那一面不相识的人,我去偷他不为过。若是终日相见拱手作揖的人,我去偷他,他总不疑我,我见了他,也觉得有些惭愧。我所以不去。那财主人家金银甚多,我去下顾一次,只当打他的抽丰,何为之过?若偷过一遭,得了甜头,只管去骚扰他,就是个贪得无厌之人,这样事我也不做。那提心吊胆的人家,夜夜防贼,口里不住的说贼。他以不肖之心待我,我就以不肖之心待他。偷他一遭使他知道我的见识,不容易防的。若是宽胸大度之家,知道钱财是身外之物,不以为意,或是大门忘了不闭,或是房门设而不关,我若去偷他,就是个欺软怕硬的人了,我岂肯做他。这就叫做‘五不偷’。远近之人见我有这些好处,所以明知我是贼,不以为贼待我,反与我相处不以为辱。如今相公若还不弃,就在这里拜个弟兄,以后有用着小人处,只管效劳,就是死也肯替的。”

未央生听他说话,不觉心上叹息道,不意盗贼之中竟有这般豪杰,我若同他相处与别处还用不着,倘若遇了佳人如红绡、红拂之类,在高门大宅之中,或有消息不能相通,或身子不能出入,我就托他当了昆仑,何等不妙?思量到此不觉手舞足踏起来。后来听说要同他结拜,心上就有些踌躇,口里虽应道“极好”,心内不十分踊跃。

赛昆仑知道他心思就开口道:“相公口里决了,心上还未决,莫非怕有连累麽?无论小人高强,做贼断然不犯,就是犯了出来,死便自家死,决不扳扯无辜之人。相公不消多虑。”

未央生见他参破机关又解了疑虑,满口应承。两人各出分资办了三牲祭礼,写出年月日,就在店中歃血为盟,誓同生死。赛昆仑年长,未央生年幼,序了兄弟之称。又同享祭物,吃到半夜。要分别去睡,未央生道:“两处睡了大家都寂寞,不如同在小弟床上,抵足谈心,消此长夜何如?”

赛昆仑道:“也说得是。”两人就脱了衣服,同床而睡。

未央生才爬上床不觉就露出惯相来。口中说道:“怎么这样好所在,没有看的上的妇人!”

赛昆仑听了问道:“贤弟为何说这两句,莫非不曾娶弟妇?要各处求亲么?”

未央生道:“弟妇是娶过了。只是一个男子怎么靠得一个妇人相处到老?必竟在妻子之外,还要别寻几个相伴才好。不瞒长兄说,小弟的心性是极喜风流的,此番出来名为游学,实是为访女色。走过了许多州县,看见的妇人不是涂脂抹粉掩饰她漆黑的肌肤,就是戴翠项珠遮蔽她焦黄的头上,那里有一个妇人不消打扮,自然标致的?所以小弟看厌了,不觉说这两句。”

赛昆仑道:“贤弟差了。天下好妇人决不使人见面,那见面的决不是好妇人。莫说良家子女,就是娼妓里面,除非是极丑极陋没人爱的,方肯出来倚门卖笑。略有几分身价就坐在家中,等人去访她方肯出来,何况好人家子女,肯立在门前使人观看?你若要晓得好妇人,只除非来问我。”

未央生听了就昂起头来道:“这又奇了。长兄又不在风月场中着脚,为何晓得我那事?”

赛昆仑道:“我虽不在风月场中着脚,那风月的事,却只有我眼睛看得分明,耳朵听得分明。我且问你,天下标致的女子还是富贵人家多,贫贱人家多?”

未央生道:“自然是富贵人家多。”

赛昆仑道:“这等富贵人家标致的女子,还是脸上搽了脂粉,身上穿了衣服才看的仔细,还是洗了脂粉,脱了衣服才看得仔细?”

未央生道:“自然是洗脱去了才见本色。”

赛昆仑道:“这等就明白了。我们做贼的人,那贫贱人家自然不去,去走动的,毕竟是珠翠成行的去处,自然看见的多了。去的时节又是更深漏静之时,她或是脱了衣服坐在明月之下,或是开了帐幕睡在灯影之中。我怕她不曾睡着,不敢收拾东西,就躲在暗处,把双眼盯在她身上看她,响不响动不动,直待她睡着了,方才动手。所以看得仔细,不但面貌肌肤一毫没有躲闪,就是那牝\footnote{p\`in,雌性的。}户之高低,阴毛之多寡,也看得明白。这数百里内外的人家,哪个妇人生得好,哪个妇人生得不好,都在我肚里。你若要做这桩事,只消来问我。”

未央生起先还在被窝中侧耳而听,及至说道此处,不觉露出胸膛坐起来道:“有理。大人家女,随你甚麽人不得见,就见也不分明,惟有你们相得到。还有一说,你看了标致的妇人,又见了丰满的阴户,万一动起兴来,都怎么处?”

赛昆仑道:“起先少年的时节,见这光景也熬不住,常在暗地对着妇人打手铳,只当与她干事一般。后来见得多了,也就不以为意。看着阴户就象寻常动用的家伙,并不动情。只是见她与丈夫干起事来,口里哼哼唧唧,阴中即即作作,未免有些动兴起来。”

未央生见他说到至妙处,就拨转身子睡到一头去听。赛昆仑道:“你若不嫌亵渎,待我说一两桩为你听,未知肯听否?”

未央生道:“妙极!如得如此,真是与君一席话胜读十年书。快讲讲来。”

赛昆仑道:“我生平看见的事甚多,不知从那里说起。如今随你问一件,我就说一件罢了。”

未央生道:“请问妇人是喜干的多,是不喜干的多?”

赛昆仑道:“自然是喜干的多。大约一百个妇人只有一两个不喜干,其余都是喜干的。只是这喜干的里面有两种。有心上喜干,口里就说要干的。有心上喜干,故意装作不要干,待丈夫强她上场,然后露出本相来。这两种妇人倒是前面的一种好打发。我起先躲在暗处见她催丈夫干事,我想是个极淫之妇,通宵不倦的了。谁想抽不下几下就丢,一丢之后,精神倦怠只想睡觉,随丈夫干也罢不干也罢。惟有心上要干,假说不干的妇人,极难相处。我曾去偷一家,见丈夫扯妻子干事,妻子不肯。丈夫爬上身去,反推下来。丈夫只说是不要干,竟呼呼的睡了。那个妇人故意把身子翻来复去,要碍他醒来。见碍他不醒,又把手去摇他。谁想丈夫睡到好处,再不得醒。她就高声喊起来道:‘有贼!’若把别个做贼的,就被他吓走了。我知道她不是喊贼,是要惊醒丈夫,好起来干事。果然不出所料,只见丈夫吓醒之后,她又把巧话支吾道:‘方才是猫捉老鼠跳一下响,我误听了,只说是贼,其实不相干。’就把丈夫紧紧搂住,将牝户在阳物边挨挨擦擦。丈夫才动起兴,上身去干。初时抽送,还勉强熬住,不露骚声。抽到数百上,渐渐哼哈起来,下面淫水流不住。干到半夜,丈夫丢了,她的骚兴正发,又不好叫丈夫再干,只得装声叹气,却像有病的光景。让丈夫揉胸摸肚,不容他睡。丈夫睡不着,只得又爬上身从头干起,直到鸡鸣方才歇息。累我守了一夜,正要收拾东西天又明了,只得潜身而出。所以晓得这种妇人极难相处。”

未央生道:“请问妇人干事的时节,还是会浪的多,不会浪的多?”

赛昆仑道:“自然是会浪的多。大约十个妇人只有一两个不会浪,其余都是会浪的。只是妇人口中有三种浪法,惟有我们听得清楚,那干事的男子反不知道。”

未央生问:“哪三种?”

赛昆仑道:“初干的时节,不曾快活,心上不要浪,外面假浪起来,好等丈夫动兴。这种声气原听得出大约,口里叫出来的字字清楚。此是一种浪法。干到快活的时节,心上也浪,口里也浪,连五官四肢都浪起来。这种声气也听得出,叫出来字字模糊,上气不接下气。又是一种浪法。干到快活尽头处,精神倦了,手脚软了,要浪浪不出。这种声气在喉咙里面,就有些听不出了。我曾偷一家,见他夫妻干事,起先乱颠乱耸,响声如雷。干到后面,那妇人不响不动,宛如被男子入死了一般。我走到近处去听,只见喉咙里面咿咿呀呀似说话非说话,似叹气非叹气。我听了这光景,知道她快活极了,不觉淫兴大动,浑身酸麻,又不曾打手铳,自己的精竟流出来。所以晓得妇人又再有这一种浪法。”

未央生听到此处,也就浑身酸痒,不觉的阳精竟流了一席。还要问他别的,不想天已大明。两个起来梳洗毕,依旧对坐说些妙话。两个绸缪几日,交情愈密,未央生就对他道:“小弟生平以女色为性命,如今得遇长兄可谓三生有幸了。若不以心事相托,岂不当面错过?要求兄长把见过的妇人拣第一个标致的,生个法子使小弟经一经眼,若果然是绝色,不瞒长兄说,小弟的贱造是有红鸾照命的,生平一见了妇人,我不去寻她,她自然来寻我。到那时求长兄显个神通,成其好事何如?”

赛昆仑摇头道:“这个使不得。我生平有偷过不偷之戒。偷过了她财物尚不忍再偷,何况于有关名节的妇人?只好从今以后留心为你寻访,走到人家见有标致妇人就不偷她财物,竟走回来与你商量,做成好事,这还使得。”

未央生道:“小弟有眼不识义士,方才的话唐突多了。只是一件,既蒙金诺要替小弟留心,若果见了绝色妇人,千万不可偷她财物,忘了今日之言。诺做得事成,小弟后来自然图报。”

赛昆仑道:“这等看来,你果然有眼不识义士。我若是想你图报的人,又不如拿现在的穗了。就是你日后做官,许我打几次抽丰,那打抽丰的银子也看得见,不如我做一次盗。这样的报也可以不图。我如今许你一个标致妇人,少不得明日还你一个标致妇人。你如今既遇了我,不消到别处去,且在这边赁几间房子读书。也不要靠我一个,你若看见有好的,就自己去做事。我若看见有好的,走来报你。两路搜寻,自然遇着,决不至落空。”

未央生大喜,就央人去寻寓处。临别之时,又扯住他拜了四拜,方才分别。毕竟未央生的奇遇如何,下回便见。

评曰:赛昆仑的人品高于未央生十倍!不是未央生结交匪类,还是赛昆仑结交匪类耳?

\chapter[选手姿严造名花册\ 狗情面宽收雪鬓娘]{选手姿严造名花册\\狗情面宽收雪鬓娘}

未央生自赛昆仑别后,搬在一个庙中作寓。这庙是送子张仙的行宫,里面房间甚少,往常是不寓客的。只因未央生不惜重价,别处一两一月,他情愿出二两,道士贪图微利,所以租与他住也。为甚么肯出重价?只因本庙的张仙极其灵验,远近妇人来求子者极多。未央生要在此处做个选场,所以谋在这边作寓。

自进寓之后,每日定有几班妇女进来烧香。那烧香的妇女又与别处烧香的不同。十个之中定有一两个将就看得。这是甚么缘故?原来各处烧香的妇人,大抵老中年的多,少年的少,所以没一个看得上眼。此处烧香的妇人都是求嗣而来,老年的经水已绝,必无生理。中年的经水将绝,子兴以阑。所以进来求嗣都是少年女子,不过有一二个老成的陪来。但凡女子十四岁至二十岁这五六年中间,无论好歹,面上都有点桃花色艳,隐隐动人。所以十个之中定有一两个看得。

未央生每日早起,打扮得整整齐齐,在神座前走来走去。望见有妇人来,就躲在张仙的背后,听道士替她通诚,又看她拈香礼拜,把面庞态度看得无遗。然后攻其不备从里面闯出来。那妇人见他姿容绝世,都吃一惊,疑是自己至诚把泥塑的张仙拜活了,下来送子与我。直待他走下阶前摇摆一会,方才晓得是人。那灵魂已被活张仙勾去了。弄得那些女子心花意乱,眼角传情,都恋恋不肯回去。也有故意扔下汗巾子为表记的。

自此以后,未央生举止分外轻佻\footnote{ti\=ao,轻薄,不庄重。},精心愈加放荡,竟说世间标致女人该是我受用的。自起先入庙之时,就钉下一本袖珍册子,藏在夹袋之中,上面题四个字“广收春色”。凡是烧香女子有几分姿色就登记入册。如妇人某人,年岁若干,良人某某,住居某处,都细细写下名字。旁又用朱笔加圈,以定高下。特等三圈,上等二圈,中等一圈。每一名后面又做四六批语,形容她的好处。

那未央生怎么晓得许多妇人并丈夫姓名住处?只因妇人入庙烧香,定有个香火道士立在旁边替她通诚,就问她姓甚么名甚么,年纪多少,系那一位信士之妻,住在何坊何里。那妇人就不说,定有个家人使婢替她答应。未央生此时就记在腹中,待她去后,取出册子登记上去。不上数日,把一方的女色收罗殆尽。虽然录了许多妇女,都是一等中等的,要那三圈头竟没有一个。心上想到,我生平的志向原要娶世间第一位佳人,起先在家里娶着的只说是第一位了。如今看起来与她一样的尽多,可见还算不得第一位。我想天下的女色,岂有有了榜眼探花而无状元之理,必竟有第一位的在那边,我还不曾遇着。如今看来看去,这些妇女只好存在这边,做个备卷,若终久遇不着,亦可拿来塞责。我且姑待几日,看以后进来的何如。于是取法加严,不肯少恕。

一日,精神怠倦,正在房里睡觉,忽见家童跑进来道:“相公,快起来看标致女子。”

未央生连忙下床来,戴新巾,穿丽服,又要照照镜子,未免耽搁了一会。及至走到外面,只见两位少年女子,一个穿银红,一个穿藕色,陪伴来的是个半老佳人,都烧了香要出去了。未央生隔着许多路把那两个少年女子一看,真是巫山神女,洛浦仙颐,比往常所见的大不相同,一时不觉风颠起来。见她要走还不曾出门,就如飞赴去跪在门槛外,不住的叩头。把两个家童与香火道士皆吓得口呆,只怕妇人要发作。

谁想未央生外面虽是疯颠,心上却有主意。料那三个妇人若是肯走这条路的,知道我见她标致爱她不过,所以跪拜她,料她必不发作。若还是正气的发作起来,我只推是外面走来的人,要拜张仙求嗣,见有女眷在内,混杂不雅,所以不敢进去,跪在门外叩头。她难道晓得我寓在庙中不成?把这个计较放在胸中,立于不败之地,所以才敢如此。

果然那三个妇人不知就里,只说他是求嗣的,都缩转身去立在旁边。直等他拜完,方才举步。拜的时节,那两个少年女子虽然一般顾眄\footnote{mi\v{a}n},只是那种意思还在有意无意之间,不觉得十分出像。独不那个半老佳人,对着未央生十分做作,自己掩口不住的笑。临行之际,还把未央生瞧了几眼,方才出去。

未央生痴足半晌不能出声,将去一二里才问香火道士是哪家的女子。道士见他轻举妄动,几乎惹出事来,埋怨不了,那肯对他说。未央生要跟着轿子去追踪迹,他又知道去远了,追赶不上,只得回到房中,闷闷的坐。心上想到,这等可恨的事,那些不中意的,个个都晓得姓名住处,偏是这两个极中意的,一个也不知道下落。可惜一对绝世佳人当面错过。就取出那本册子,要添这两个上去,竟无名字可写,只得先记一笔在前,道:

某月某日遇国色二名,不知姓氏,姑就所衣之色随意命名,仿佛年齿性情开列于左,以便物色。

银红女子一名。年可十七八。察其情意,她于归未决而欲窦未开者。

批:此妇态如云行,姿同玉立。朱唇绽处,娇同解语之花。纤步移时,轻若能飞之燕。眉无忧而长蹙,信乎西子善颦。眼不倦而慵开,应是杨妃喜睡。更可爱者,赠人以心,而不赠人以物,将行无杂佩之遗。示我以意,而不示我以形,临去少秋波之转,殆女中之隐士而阃内之幽人。置之巍等,谁曰不宜?

藕色佳人一名。年可二十许。察其神气,似适人虽久而原阴未流者。

批:此妇风神绰约,意志翩跹\footnote{xi\=an}。眉无待画之痕。不烦京兆,面有难增之色。焉用何郎肌肉,介肥瘦之间,妙在瘦不可增,肥不可减。妆束居浓淡之际,妙在浓似乎浅,而淡似乎深。所可怜者,幽情郁而未舒,似常开不开之菡萏\footnote{h\`an d\`an,睡莲科莲属多年生水生草本植物。}。心事含而莫吐,怠未谢愁谢之芳菲。所贵与前,并压群芳,同称国色者也俟!

面试后再定元魁。

批评已毕,心上又想到,那个半老佳人也不减少年风致。别的且不要说,只是那双眼睛或如一件至宝了。她起先丢上许多眼色,我只因注意那两个,不曾回她一眼。如今想来甚不过意,况且与标致妇人同行,不是妯娌,定是亲戚,也就要看标致的分上宽待她几分了。她又肯帮情凑趣,引那两个顾眄我,分明是个解人。我若寻得她,何愁那两个不入鹄\footnote{g\v{u}}中?我今也把她写在册上,加一个好批评。一来报她牵卷之情,二来若寻着的时节,就把这册子送与她看,先把她奉承到了,不愁她不替我做事。就提起笔来,把国色二名的“二”字改作“三”字。因她穿服是玄衣,再添一名道:

玄色美人一名。年疑四九,姿同二八。观其体态,似欲事书疏而情甚炽者。

批:此妇幽情勃动,逸兴湍飞。腰肢比少妇虽实,眉黛与新人竞曲。腮红不减桃花,肌莹如同玉润。最销魂者,双星不动而眼波自流,闪烁如同崖下电。寸步未移而身容忽转,轻飘酷似岭头云。即与二美鼎足奚多让焉!

写完,每一个名字上圈了三圈,依旧藏在夹袋中。

从这一日起,那张仙殿上去也得,不去也得。进来的妇人看也可,不看也可。只把这三个佳人时刻放在心上,终日带了这个本子沿街去撞。再不见一毫踪影,心上想道,赛昆仑见识最高,路数又熟,为甚么不去问他?只是一件,他原许我寻一个,这几日不见,想是去寻了。我若对他说,他只道我有中意的,倒把这担子丢开了。况且没名没姓,教他哪里去查?我且放在肚里,再等几日,他或许寻一个来报我也不可知,别的东西怕多,标致妇人也不怕多了。

自此以后,每日起来不是出门问撞,就是在家死等。一日,在街上遇着赛昆仑,就扯住问道:“大哥,向日所许的事为何不见回音?莫非忘记了?”

赛昆仑道:“时刻在心,怎么会忘记。只是平常的多,绝色的少。近日才寻着,正要来报你,恰好撞着。”

未央生听了,满脸堆下笑来道:“既然如此,请到敝寓去讲。两人偕手而行,一同入寓。把家童打发出去了,两个关了房门商量好事。 不知是哪一家妇人造化,遇着这会干的男子,又不知是哪一家丈夫晦气,惹着这作孽的奸夫?看官不用猜疑,自有下回分解。

\chapter[饰短才漫夸长技\ 现小物怡笑大方]{饰短才漫夸长技\\现小物怡笑大方}

诗曰:

\begin{quotation}
不是房中作干才,休将末技惹愁胎。

暗中谁见潘安貌,阵上难施子建才。

既返迷魂归楚国,问伊何事到阳台。

生时欲带风流具,尺寸还须自剪裁。
\end{quotation}

赛昆仑坐下先问未央生道:“贤弟这一向可曾有甚么奇遇么?”

未央生怕他要卸担,只回没有。接口就问道:“长兄方才所说的是哪一家?住在哪一处?多少年纪?怎么样姿色?”

赛昆仑道:“我如今寻着的不止一个,一共有三个,只许你拣择一个。你不要贪心不足都想要,做起来这就成不得了。”

未央生心上疑惑道,我心上有三个,他口里也说三个,莫非是日前见的不成?若果然是,只要弄得一个上手,那两个自然会来,何须要他帮助?就回复道:“岂有此理!只要有一个也就够得紧了,怎敢做那贪得无厌之事!”

赛昆仑道:“这等才好。我且问你,你还是喜肥的,还是喜瘦的?”

未央生道:“妇人家的身体肥有肥的妙处,瘦有瘦的妙处。但是肥不可胜衣,瘦不可露骨。只要肥瘦得宜就好了。”

赛昆仑道:“这等说来三个都合着你意思。我再问你,你还是喜风流的,喜老实的?”

未央生道:“自然是风流的好。老实妇人睡在身边,一些兴趣也没有,倒不如独宿的干净。”

赛昆仑摇头道:“这等说来,三个都不是你的对头。”

未央生道:“怎见得那妇人老实?”

赛昆仑道:“那三个妇人皆是一般家数,若论姿色,倒有十二分,只是‘风流’二字不十分在行。”

未央生道:“这个不妨。妇人家的风情态度可以教导得来。不瞒长兄说,弟妇初来的时节也是个老实头,被小弟用几日工夫把她淘熔出来,如今竟风流不过了。只要那三个妇人姿色好,就老实些,小弟自有变化之法。”

赛昆仑道:“这也罢了。我再问你,你还是一见了面就要到手,还是肯熬几月工夫,慢慢伺候到手?”

未央生道:“不瞒长兄说,小弟平日欲火极盛,三五夜不同妇人睡就要梦遗。如今离家日久,这点欲心慌得紧了。遇不着标致女子还可以勉强支持,若遇着了,只怕就涵养不住了。”

赛昆仑道:“这等,丢了那两个,单说这一个罢。那两个是富贵人家女子,一时难到手。这一个是穷汉老婆,容易设法。我因许你这桩事,时时刻刻放在心头,遇了妇人定要仔细看看。那一日,偶从街上走过,看见这个妇人坐在门里,门外挂着一条竹帘。虽然隔着帘子看不明白,只觉得面庞之上红光灼\footnote{zhu\'o}灼,白焰腾腾,竟象珍珠宝贝,有一段光芒从里面射出来一般。再看她浑身态度,只像一幅美人图挂在帘子里面随风吹动一般。我走过去那门对面立了一会,只见一个男子从里面出来,生得粗粗笨笨,衣服褴褛,背一捆丝到市上去卖。我就去问他,邻居说他姓权,为人老实,人就因此叫他做‘权老实’。那妇人就是他妻子。

“我恐隔着帘子看不仔细,过了几日又从门首经过。她又坐在里面。我心生一计,掀开帘子闯进去,只说寻她丈夫买丝。她说男人不在家,若要买丝家里尽有,取出来看就是。说罢回身取丝出来。我见她十个指头就如藕芽一般,一双小脚还没有三寸。手脚虽然看见了,还有身上的肌肉不能看见,未知黑白何如。我又生个法子,见她架子顶上还有一捆丝,就对他道:‘这些都不好。那架子顶上的拿来看看何如?’她答应了,就擎起手臂来去拿。你晓得,此时热天,她身上穿的是单纱衫子,擎起手来的时节,那两双大袖直褪到肩头上面,不但一双手臂全然现出,连胸前的两乳也隐隐跃跃露出些影子出来。真是雪一般白,镜一般光。我生平所见的妇人,这就是第一了。我因劳她半日,不好意思,只得买了一捆丝出来。请问贤弟,这妇人你是要不要?”

未央生道:“这等说来竟是个十全的了,有甚么不要?只是这个妇人怎么就能勾见面,见了面就能勾到手?”

赛昆仑道:“不难。我如今就同你拿些银子去伺候,等她丈夫出门,依旧用前面的法闯进去买丝。你中意不中意一见就决了。我想她终日对着那个粗笨丈夫老老实实,一些情趣也没有。忽见了你,岂不动心?你略做些勾引她的光景,她若当面不恼,我回来就替你商量做事。管取三日之内定然到手。若要做长远夫妻,也都在我身上。”

未央生道:“若得如此,感恩不浅。只是一件,你既有神出鬼没的计较,又有飞墙走壁的神通,天下的事必没有难做的了。为甚么这一个就做得来,那两个全不说起?毕竟是穷汉好欺负,富贵人家不敢去惹他?”

赛昆仑道:“天下事都是穷汉好欺负,富贵人家难惹,只有偷妇人一节,倒是富贵人家好欺负,穷汉难惹。”

未央生道:“这是何故?”

赛昆仑道:“富贵人家定有三妻四妾,丈夫睡了一个,定有几个守空房。自古道饱暖思淫欲。那妇人饱食暖衣,终日无聊,单单想着这件事。到没奈何的时节,若有男子钻进被去,她还求之不得,岂肯推了出来?就是丈夫走来撞见,若是要捉住送官,又怕坏了富贵体面,若是要一齐杀死,又舍不得那样标致妇人。妇人舍不得,岂有独杀奸夫之理?所以忍气吞声,放条生路让他走了。那穷汉之家只有一个妻子,夜夜同睡,莫说那妇人饥寒劳苦不起淫心,就有淫心与男子干事,万一被丈夫撞见,那贫穷之人不顾体面,不是拿住送官,就是一同杀死。所以穷汉难惹,富贵人家好欺负。”

未央生道:“既然如此,你今日所说的事又与这议论相反?”

赛昆仑道:“不是我做的事与说的话相反,只因这一个人家与那两个人家的地位恰好相反。所以这一家好设法,那两个妇人难以到手。”

未央生道:“如今小弟心上已注意在这一边了,只是那两个妇人何妨也说一说,等小弟知道长兄的盛意,为我这样费心。”

赛昆仑道:“那两个妇人一个有二十多岁,一个有十六七岁。她两个在娘家是嫡堂姐妹,在夫家又是姻亲妯娌。夫家世代做官,只有她两人的丈夫是个秀才。哥哥叫做‘卧云生’,与那二十多岁妇人做亲四五年了。兄弟叫做‘倚云生’,与那十六七岁的妇人成亲不上三月。两人的姿色也与方才说的妇人一般。只是一样的老实,干事的时节身也不动,口也不开,看她意思竟象不喜干的光景。妇人又不好淫,丈夫又没有三妻四妾,夜夜同睡,难以算计。你除非千方百计引动她淫心,又要嗣候她丈夫不在,方才可以下手。这不是有几月工夫?不如卖丝的妇人,丈夫常不在家,容易设法。”

未央生见他说那两个妇人与日前所见之人有些相似,心上还舍不得丢开。又对他道:“长兄的主意虽不差,只是还有见不到处。你说那两个妇人老实没有淫心,必是她丈夫本钱细微,精力短少,干得她不快活,所以如此。若还遇了小弟,只怕那老实的也会不老实起来。”

赛昆仑道:“我看那两个男子本钱也不细微,精力也不短少。只是比了极粗大长远的稍逊他。我且问你,你的本钱有多少大?精力有几时长?也要见教一见教,使我知道你伎俩的深浅,好放心替你做事。”

未央生欣然道:“这个不劳长兄挂念,小弟的本钱精力也算得来。随你甚么大量妇人,定要请她吃个醉饱,方才散席。决不象酸子请客,到把饱的吃饥,醉的吃醒了。”

赛昆仑道:“这等就好。只是略说一说也不妨,贤弟往常与妇人干事大约有多少提方才得泄?”

未央生道:“小弟与妇人干事没有甚么规矩,只请她吃一个无算数就罢了。那里记得数目。”

赛昆仑道:“数目记不出,时刻是记得出的。大约耐得几更天气?”

原来未央生的本事只有半更,因要赛昆仑替他做事,恐怕说少了他要借端推诿,只得加上半更。就答应道:“小弟的力量足足支持得一更!”

塞昆仑道:“这等说来也是平常的精力,不叫做高强。若是夫妇干事,有这本领也就好了。若要隔家过舍去做偷菅\footnote{ji\=an}劫寨的事,只怕不是平等力量可以做得来的。”

未央生道:“长兄不消过虑。小弟前日买得有绝好的春方在那边,如今正为没有妇人使英雄无用武之地。只要好事做得成,到临时用些搽抹的功夫,不怕他不久。”

赛昆仑道:“春方只能使他久,不能使他大。若是本钱粗大的,用了春方就象有才学的举子,到临考时吃些人参补药,走到场屋里自然精神加倍,做得文字出来。那本钱微细的,用了春方尤如腹内空虚的秀才,到临考时就把人参补药论斤吃下去,走到场屋里也只是做不出。我今只问你这物事有多少大?有几寸长?”

未央生道:“不消说得,只还你不小就是。”

赛昆仑见他不说,就伸手去扯他的裤裆,要他脱出来看。未央生再三回避,只是不肯。赛昆仑道:“若是这等,劣兄绝不敢替你做事,若强替你做事,万一不看那妇人疼痒,被她叫喊起来,说你去强奸她,怎么了得?到那时弄出事来,倒是劣兄耽误你了。怎么使得?”

未央生见他激切,只得陪个笑脸道:“小弟的本钱也看得过,只是清天白日在朋友面前取出,觉得不雅。今长兄既然过虑,小弟只得献丑了!”就把裤带解开,取出阳物,把一双手托住,对赛昆仑掂几踮,道:“这就是小弟的微本。长兄请看。”

赛昆仑走近身去仔细一观,只见:

本身莹白,头角鲜红。根边细草蒙茸,皮里微丝隐现。量处岂无二寸,称来足有三钱。十三处子能容,二七娈\footnote{lu\'an}童最喜。临事时身坚似铁,几同绝大之蛏于;竣事后体曲如弓,颇类极粗之虾米。

赛昆仑把阳物看了一会,再不则声。未央生只说见他本钱粗大,所以吃惊,就说道:“这是疲软时如此,若到振作之后,还有可观。”

赛昆仑道:“疲软时是这等,振作时也有限。请收拾罢。”说完不觉大笑道:“贤弟为何不知分量,自家本钱没有别人三分之一,还要去偷别人的老婆!我起初见你各处寻妇人,只说定有绝大的家伙带在身边,使人见了害怕,所以不敢轻易借观。那里晓得是根肉搔头,只好放在阴毛里面擦痒,正经所在是用他不着。”

未央生道:“不瞒长兄说,小弟这贱具虽不甚魁伟,也曾有人喝彩过的,亦不至如此无用。”

赛昆仑道:“有人喝彩,必是未经破瓜的处女,不曾干事的孩童,若见了他自然要赞叹几句。除了这两种人,只怕就与我一样,不肯奉承尊具了。”

未央生道:“照长兄说来,难道世上人的肉具都大似小弟的不成?”

赛昆仑道:“
这件东西是劣兄常见之物,不止千余根。从没有第二根像尊具这般雅致。”

未央生道:“别人的且不要管,只请问那三个妇人的丈夫,他腰间之物比小弟的何如?”

赛昆仑道:“比贤弟的大也大一两倍,长也长一两倍。”

未央生笑道:“我知道长兄的话不是真言。乃不肯替小弟任事,借端推诿,如今试出来了。我且问你,那两个的或者你夜间去偷他看见了,也不可知。这个卖丝的妇人,据你说不过日间去一次,又不曾遇见他男子,怎么知道他的东西比小的长大一两倍?”

赛昆仑道:“那两个是目见的,这一个是耳闻的。我初见之时,走去问她邻舍,邻舍对我说了姓名。我又问他道:‘这样标致女子嫁了那粗蠢丈夫,不知平日相得否?’邻舍道:‘他丈夫的相貌虽然粗蠢,还亏得有一副争气的本钱,所以过得日子还不十分吵闹。’我又问道:‘他的本钱有多少大?’邻舍道:‘量便不曾替他量,只见他夏天脱了衣服那件东西在裤子里荡来荡去,就像棒槌一样,所以知道他的本钱争气。’我今日所以定要问你借观,就是为此。不然为甚么没原没故借人阳物看起来?”

未央生听了,才晓得他是真话,有些没趣起来。只得又对他道:“妇人与男子相处,也不单为色欲之事,或是怜他的才,或是爱他的貌。若是才貌不济的,就要靠本事了。小弟这两件都还去得,或者她看才貌分上恕我几分也不可知。还请长兄始终其事,不可以一短而弃所长,把为朋友的念头就中止了。”

赛昆仑道:“才貌两件是偷妇人的引子,就如药中的姜枣一般,不过借它气味,把药力引入脏腑。及至引入之后,全要药去治病,那姜枣都用不着了。男子偷妇人若没有才貌,引不得身子入门。入门之后,就要用着真本事了。难道在被窝里相面,肚子上做诗不成?若还本钱细微,精力有限的,就把才貌两件引了进去,到干事的时节,一两遭干不中意,那娇人就要生疏了。做男子的既然拚\footnote{p\`an}了性命偷着女子,也要与她心投意合相处一生半世便好。若要只图一两遭快活,为甚么费这样心机?且不要说男子偷妇人要图长久快活,就是妇人瞒丈夫偷男子,也不知费多少提防,担多少惊吓,指望要快活。若还一些受用也没有,就像雌鸡受雄的一般,里面还不曾得知就完了账,岂不坏她一生名节?贤弟不要怪我说,都像你这样的本钱,这样的精力,只要保得自家妻子不走邪路就够了。再不可痴心妄想,去坫污人家女子。今日还亏劣兄老到,相体裁衣,若还不顾长短,信手做去,使衣服大似身子,岂不坏了作料?等那妇人报怨也罢了,只怕贤弟还要怪我谋事不忠,故意寻那宽而无当的妇人来塞责。劣兄出言粗卤,贤弟不要见怪。”

未央生见他言语激烈,料想好事不成,无言可答。赛昆仑又安慰了几句,就起身辞去。未央生兴致索然,也就送他去了。他扫兴之后不知如何,直到下回是有定局。

评曰:每一番议论定有绝精的比喻,无不使人快心。如“春方乃临场补药”,“才貌乃药中引子”之类,不可胜数。虽属谐谑之语,实有至理存焉。我竟不知作者的心肝有几万几千个孔窍,而遂玲珑至此也。

\chapter[怨生成抚阳痛哭\ 思改正屈膝哀求]{怨生成抚阳痛哭\\思改正屈膝哀求}

却说未央生一团高兴,被赛昆仑说得冰冷,就像死人一般。独自坐在寓中想到,我生长二十多岁,别的物事见得也多,只有阳物其实不曾多见。平常的人藏在衣服里面,自然看不出了。只有那些年少的龙阳,脱下裤来与我干事,方才露出前伴。他的年纪轻似我,物事自然少似我,终日所见都是小似我的,所以就把我的形大了。今被他说所见之物没有一根不长大于我,这等我的竟是废物了,要他何用?只是一件,我在家中与妻子干事的时节,她一般也觉得快活。就是往常嫖女客偷丫鬟,她们一般也浪,一般也丢,若不是这件东西弄得她快活,难道她自己会浪,自己会丢不成?可见他的话究竟不是真言,还是推诿的意思。

疑了一会,又相一会。忽然了悟道,我晓得了,妻子的牝户是件混沌之物,从我开辟出来的。我的多少大,她的就多少宽;我的多少长,她的就多少深。以短投浅,以细投窄,彼此相当,所以觉得快活。譬如取耳一般,极细的消息放在极小的耳朵里面转动起来,也觉爽利。若还是宽耳朵遇着细消息,就未必然了。日前赛昆仑说妇人有心上不浪,口里假浪之法,焉知那些丫鬟女客不是因得了我的钱财,故意奉承我,心上其实不要浪,口里假浪骗我,也不可知。浪既可假,岂有丢不可假者乎?他说这话虽不可全信,也不可不信。以后遇着男子,要留心看他的阳物何如,就明白了。

从此以后,与朋友会文的时节,朋友小解,他也随去小解;朋友大便,他也跟去大便。把朋友的看一看,又把自己的看一看。果然,没有一个不雄似他的。就在路上行走,看是肩上坑上有人绊手,也定要斜着眼睛,把他的阳物看个仔细。果然个个大也大的他、长也长的他。自此比验之后,未央生的欲心也渐渐轻了,色胆也渐渐小了。心上思量道,赛昆仑的话句句是药石之言,不可不听。他还是个男子,我前日被他一番取笑,尚且满面羞愧,万一与妇人干事弄到半中间被她轻薄几句,我还是自己抽出来不干的好、还是放在里面等她呕吐出来的好?从今以后,把偷妇人的事情收拾起,老老实实干我的正经,只要弄得功名到手,拼些银子讨几个处女做妾,我自然受她奉承不受怠慢了。何须陪了精神去做烧香塑佛的事?算计以定,果然从这一日起,撇却闲情,专攻举业。看见妇人来烧香,不但不赶去看,就在外面撞见,也还要避了进来。至于街坊上行走,看见妇人,低头而过,一发不消说了。

准准熬了十余日,到半月之后,欲心难禁,色胆又大。一日,从街上走过,看见一个少年妇人把一只手揭开帘子,露出半个面庞,与对门的妇人说话。未央生远远望见,就把脚势放松,一步勾做三步走,好慢慢的听她声音、看她面貌。只见吐出来的字眼就像箫声笛韵一般,又清楚又娇媚,又轻重得宜。躲着走到门前细看她面貌态度,竟与赛昆仑所说的话件件相同。也像珍珠宝贝,也像一幅美人图在帘子里随风吹动。心上想猜,她前日所说的莫不就是此人?

相了一会,走过几家门面,故意问人道:“这边有个卖丝的人,叫做权老实,不知他在哪里?”

那人道:“你走过了。方才那帘子里面有妇人说话的就是他家。”

未央生知道果然是了,就复转身来又看个仔细,方才回到寓中。心上想道,起先,赛昆仑在我面前形容她的标致,我还不信,只道他未必识货。那里晓得是一双法眼。这一个相得不差,那一家两个的自然不消说了。有这样的佳人,又有那样的侠士肯替我出力,只因这一件东西不替我争气,把三个好机会都错过了,怎么教人恨得过。懊恼一番就把房门关上,解开裤子,取出阳物来左相一会,右相一会,不觉大怒起来,恨不得取一把快刀,登时割去,省得有名无实放在身边。又埋怨道,这都是天公的不是,你当初既要娇纵我,就该娇纵到底,为甚么定要留些缺陷?这才貌两件是中看不中用的东西,你偏生赋得完备,独有这件要紧物事舍不得做情。难道叫它长几寸大几分要你费甚么本钱不成?为何不把别人的有余,损些下来补我的不足?就说各人的形体赋定了,改移不得,何不把我自己腿上的皮肉、浑身上下的气力匀些放在上面,也就够了。为甚么把这上边的作料反匀到别处去使?人要用的有没得用,不要用的反余剩在那边,岂不是天公的过处?如今看了这样标致女子不敢动手,就像饥渴之人见了美味,口上又生了疔疮,吃不下去的一般,教人苦不苦?思量到此,不觉痛哭起来。

哭了一会,把阳物收拾过了,度到庙门前去闲步遣闷。只见照壁上一张簇新的报帖,未央生向前一看,只见上写道:

天际真人来受房术能使微阳变成巨物这四句是前面的大字,后面还有一行细字。是偶经此地,暂寓某寺某房,愿受者速来赐顾,迟则不及见矣。

未央生看了不觉大喜道,有这么样的奇事,我的阳物渺小,正没摆布,怎么就有如此的异人到这边来卖术,岂非天意?遂如飞赶进庙去,封了一封贽\footnote{zh\`i}见礼,放在拜匣中,教家童捧了,自己寻到寓处去。

只见那为术士相貌奇伟,是个童颜鹤发的老人。见他走到,拱一拱手,就问道:“尊兄要传房术么?”

未央生道:“然也。”

术士道:“尊兄所问,还是为人之学,还是为己之学?”

未央生道:“请问老先生,为人怎么样,为己怎么样?”

术士道:“若单要奉承妇人,使她快活,自己不图欢乐,这样的房术最容易传。不过吃些塞精之药,使肾水来的迟缓;再用春方搽在上面,把阳物弄麻木了,就如顽铁一般,一毫痛痒不知。这就是为人之学。若还要自家的身子与妇人一齐快活,阴物阳物皆知痛痒--抽一下,两边都要活;抵一下,两边都要死。这才叫做交相取乐,只是快活之极,妇人惟恐丢得迟,男子惟恐丢得早。要使男子越快活而越不丢;妇人越丢而越快活,这种房术最难,必须有修养的工夫到,再以药力助之,方才有这种乐处。尊兄要传,跟在下云游几年,慢慢参悟出来,方有实际。不是一朝一夕可以得去的。”

未央生道:“这等,学生不能待,还是为人之学罢了。方才见尊禀上有‘能使微阳变成巨物’这八个字,所以特来请教。不知是怎样方法才能改变?”

术士道:“做法不同,大抵要因才而施。第一,要看他本来的尺寸生得何如;第二,要于本来尺寸要扩充多少;第三,要问他熬得熬不得,拼得拼不得。定了规矩,方好下手。”

未央生道:“这三件是怎么样,都求老先生明白指教,好得学生择事而行。”

术士道:“若是本来的尺寸不短小,又于本来尺寸之外扩充不多,这种做法甚容易,连那拼得拼不得、熬得熬不得的话都不必问,只消用些药敷在上面,使它不辨寒热不知痛痒,然后把药替它薰洗,每薰一次洗一次,就要搓一次扯一次。薰之欲其长,洗之欲其大;搓之使其大,扯之使其长。如此三日三夜,就可比原来尺寸之外长大三分之一。这种做法是人所乐从的。若还本来的尺寸短少,又要于本来尺寸之外扩充得多,这种做法就要伤筋动骨了。所以要问他熬得熬不得,拼得拼不得。他若是个胆小的人,不肯做利害之事也就罢了,若还是爱风流不顾性命的,就放胆替他改造。改造之法,先用一只雄狗、一只雌狗关在空房里,它们自然交媾起来。等它们交媾不曾完事之时,就把两狗分开。那狗肾是极热之物,一入阴中长大几倍,就是精泄后还有半日扯不出来,何况不曾完事?而这时节先用快刀割断,然后割开雌狗之阴,取雌狗之肾,切为四条。连忙把本人的阳物用麻药麻了,使它不知疼痛,然后将上下两旁割开四条深缝,每一条缝内塞入带热狗肾一条,外面把收口灵丹即时敷上。只怕不善用刀,割伤肾管,将来就有不举之病,若肾管不伤,再不妨事养到一月之后,里面就像水乳交融,不复有人阳狗肾之别。再养几时,与妇人干事那种热性,就与狗肾一般。在外面看来,已比未做的时节长大几倍;收入阴中,又比在外的时节长大几倍。只当把一根阳物变做几十根了,你道那阴物里面快活不快活?”

未央生听到此处,竟像已死之人要重新转活来一般,不觉双膝跪下道:“若得如此,恩同再造。”

术士连忙扶起道:“尊兄要仿学生服事就是了,为何行此大礼?”

未央生道:“学生赋性好淫,以女色为命。无奈如先天所限,使我胸中的志愿再不能酬。如今得见异人,怎敢不行北面之礼,就好造次奉求。”说完就唤家童取礼过来,自己亲手递过去道:“些须不腆,暂为拜见之仪。待改正之后,再当奉献。”

术士道:“这桩事说便是这等说,十有九分还是做不成的。这个盛仪不敢轻领。”

未央生道:“没有甚么做不成。学生贱性是极爱风流,不顾性命的。若还改造的好,能使微阳便成巨物,将来感恩不浅。就或者用刀差错,有伤性命,也是数该如此,学生亦不敢怨。老先生不必多疑。”

术士道:“这法度在下做得惯拿得稳,用刀自无差错。只是改造之后有三件不便处,所以不敢轻易任事。须要逐件说过,若还情愿如此,才敢领命。倘三件之中有一件不情愿,就不敢相强。”未央生道:“是哪三件不便处?”术士道:“第一件不便,做过之后有三个月不可行房。一行了房,里面就要伤损,使人阳、狗肾两下分开,不但假的生不牢,连自己真的也要烂。我起先说熬得熬不得的话,就是为此。第二件不便,做过之后,除非二三十岁的妇人方能承受,未满二十者就是已经破瓜、大而生育的,初干之时也要受许多磨难。若未曾出嫁的处女干一个死一个,决无幸全之理。要做这事,除非戒了不娶头婚,不御少妇,方才使得。不然岂但本人的阴德难全,连代做之人罪过也不小也。第三件不便,做过之后,后天的人力虽然有余,那先天的原气割的时节未免泄漏了些,定然不足生男育女。即使生男育女,生出来也都是夭亡者多,长命者少。我起先所说拼得拼不得的话就是为此。我看尊兄是个青年有志的人,一来欲心太燥,熬不得三月不行房;二来色心太贪,保不得将来不幸处女;三来年事甚轻,恐怕令郎还不曾有,就有也不多。我想这三件事皆有碍于尊兄,料尊兄未必件件情愿而敢于轻试也。”

未央生道:“这三件事皆碍学生不着。老先生放心,只管替我改造就是。”术士道:“怎见得碍不着?”未央生道:“我如今在客边比在家里不同,就是不做此事尚且连夜孤眠,难道做了此事反有甚么走动不成?那第一件事是与我无碍的了。有甚么做不得?”至于结发妻子不可娶头婚,其余婢妾都可以不论。学生的荆妻已经娶过,可以不消虑得。况且女色之中极不受用的是处女,一毫人事不知,一些风情不谙,有甚么乐处?要干实事,必待二十以外、三十以内的妇人,才晓得些起承转合。与做文字的一般,一段有一段的做法,一般有一般的对法,岂是开笔的蒙童做得来?那第二件事不但于我无碍,又且与我相投了。有甚么做不得?若子息一事别人看得极重,学生看得极轻。天下的子嗣肖者少,不肖者多;孝顺者少,忤逆者多。若侥幸生个好的出来这不消论,若生个不肖不孝的出来,把家业废去,又把父亲气死,要此子何用?况且天下的人十个之中,定有一两个无子,这都是他命该绝嗣,难道也是因改造阳物,泄了原气所以绝嗣不成?我今天起了这个念头,就是个无子之兆了,又自己情愿无子,一定要割。万一命中有子,到那临割的时节原气不十分漏泄,依旧会生育男女,生出来的男女或不到夭亡也未可知。这总是意外的事,我不想,只打点做个无子的人就是了。老先生所说之事,学生熬也熬得,拼也拼得,有甚么不便?如今不消疑我,竟替学生改造就是了。”

术士道:“既然尊意甚坚,一定要做,在下不好作难。须要选个日子,或约在尊馆,或屈到小寓,必须做得隐静,不可使一人知道。若有人知道走来窃看,就不便行事了。”未央生道:“敝寓往来人杂,难行此事。不如还到尊寓来罢。”两个相约定了,术士才把贽仪收下,取出一本通书,选了日子,是个火日,阳物属火,取火旺则盛的意思。 改造日子定了,未央生千欢万喜,分别而去。他生平造孽之根皆始于此,可见天下学房术是学不得的,学了房术就要坏了心术,从未有学房术单为奉承妻子,而不淫人妻子者也。

评曰:

他人执笔定于未央生知道阳物短小,急急寻人改正。改正之后好叙淫欲之事,使看书之人精神踊跃,无枝多干少之嫌。岂肯插入不看妇人一段,使风流才子忽变为道学先生以冷观者之目?作者独于此处着意,殆有深意存焉。使未央生果于此时改弦易辙,则后来名利无伤,无妻妾偿淫之事矣。可见极恶之人,一念回头即是彼岸,不可于回头之后再转一念耳。读此书者当在此处着眼,则于枣肉之中嚼出橄榄之味,作者深心不待终篇而始见也。

\chapter{第八回 三月苦藏修良朋刮目 一番乔卖弄美妇倾心}

作者:《肉蒲团》李渔

未央生别了术士,回到寓中,独自一个睡了。就把改造阳物以后与妇人干事的光景预先揣摩起来,不觉淫兴大发,一时难禁。只得叫随身一个家童上床去睡,把他权当了妇人,恣其淫乐。

他有两个家童,一个叫做书笥,一个叫做剑鞘。书笥年十六岁,因他识几个字,未央生把一厅书籍都交给他掌管,就像个藏书的箧子一般,所以取名叫做书笥。剑鞘年十八岁,未央生有一口古剑交付他收藏,就像个护剑的套子一般,所以取名叫做剑鞘。两个人物都一样妖姣,姿色都与标致妇人一般。剑鞘不会作骄态,未央生虽不时弄他还不觉十分得意。书笥性极狡猾,与未央生行乐之时态耸驾后庭如妇人一般迎合,口里也会做些浪声,未央生最钟爱他。所以这一晚不用剑鞘,单叫他上床好发泄狂兴。

书笥等他完事之后就问道:“相公这一向单爱妇人,厌弃男子,把我们抛撇久了。为何今夜高兴,温起旧账来?”未央生道:“我今晚不是同你干事,是与你作别。”书笥道:“这么说,莫非要卖我么?”未央生道:“我怎舍得卖你,这‘作别’二字不是我同你作别,是我的阳物与你的后庭作别。”就把要改造阳物的缘故细细说了一遍。书笥道:“这等,你改造之后一根阳物有几十根大的,好去偷妇人,量我后庭想是不能承受了。”未央生道:“是。”书笥道:“你若去偷妇人,少不得要一个使唤的随身护驾。就把我带在身边,若有多余的妇人你睡不了的,赏我一个,等我尝尝女色的滋味,也不枉跟个风月主人一场。”未央生道:“这个容易。‘饱将手下无饿兵’,正经的同我睡了,那手下的丫鬟任凭你睡。莫说一个,就要几十个也有。”书笥听了欢喜道:“你的阳物既与我的后庭作别,我如今也要与你作别了。”就倒爬上身去,浇了一回本色蜡烛,方才下来。

未央生睡到第二日,就买了一只极健的雄狗,又买一只雌的相配,分作两处养在寓中。等到约定日期,叫书笥牵了,自己一同过去,又令剑鞘备一桌酒席,随后送来。那术士的寓处是个极秘密的所在,没有闲杂人往来,极好做事。当日见未央生走到,就叫他取出阳物,预先上了麻药,好待临期用刀。那麻药初搽上去就像冷水激了一下,一激之后竟像没了此物一般。掐也不知疼,搔也不觉痒。未央生放下了心,知道割的时节没有苦吃的了。

不多时,酒已送到,与术士一边吃酒,一边等雄狗与雌狗干事。那两个畜生牵到僻静处来,放在一处,它们只道是主人盛意,肯行方便,就联络起来。那里晓得是主人要借它本钱?!那两狗牵来的时节颈项里各系一条索子,未肯解去。术士见它们干到兴高之时,就令两个家童把两根牵索用力扯开。雄狗舍不得开交,口里乱吠,两只后腿紧紧夹住阴物,惟恐它开去;雌狗也舍不得开交,口里乱吠,两只后腿紧紧夹住阳物,惟恐它出去。术士手持快刀,把狗肾割断。随割开雌狗之阴,取出雄狗之肾,切分四条。就连忙把未央生阳物割开四条缝,每一条缝内托一条狗肾,带热塞进去。四条塞完,外面敷上灵丹,用汗巾包扎好了,两个依旧饮酒。

未央生这一晚就在术士寓中借宿,夜间抵足之时,又传授了许多战法。到第二日才回去将养。这三个月之中也亏他把持得定,不但不想欲事,连新改的阳物眼也不去看一看。直等过了三个月方才解去汗巾,把它刮洗出来。仔细一看,不觉大喜道:“魁梧奇伟,果然改观,有此异物,可以横行天下矣。”

又过了数日,忽见赛昆仑走来问道:“贤弟一向不出门,在寓中静坐,想举业的功夫必然长进了。”未央生道:“举业的功夫不过如此,倒是房术的功夫有长进了。”赛昆仑笑道:“资质不高,长进也有限。”未央生道:“长兄差了,士三日不见便当刮目相待,何况小弟别了三月?难道就没进益么?何不思三尺之童后来变成大汉,脱兔之师起先有若处女?只有死人的阳物只会消不会长,哪有活人的东西是人所能料定的?”赛昆仑道:“这话我不信,十三四岁的孩子那鸡巴不曾出汁就会一日大似一日,岂有二十以外之人阳物还会发作么?就发也发不多,不过论丝论毫,决无论分论寸之理。”未央生道:“莫说论丝论毫,论分论寸也不足形其所发之长大。”赛昆仑道:“岂有此理。世上只有暴发的财主,不曾见有暴发的阳物。既然如此求取出来与愚兄看一看。”未央生道:“前次取出来受兄许多怠慢,如今怎敢再献出?”赛昆仑道:“贤弟不要取笑,快取出来。若果然长进,待我奉承几句请罪就是了。”未央生道:“口中奉承也没干,除非寻件实事与它做做,一来试验它,二来鼓舞它,才见长兄作养人材的盛意。”赛昆仑道:“若真是长进了我就把前日说的事作养它。”

未央生道:“既是如此,依旧要出丑了。”就把衣服抄起系在带间,次将裤子卸下。然后把两手捧住阳物,就像波斯献宝一般,对赛昆仑道:“长进不长进,看就知了。”赛昆仑远远望见,疑是用一条驴肾挂在腰间骗我。及至近身仔细一看,方才知是真货,不觉吐舌大惊,问道:“贤弟用甚么方法就把一个极疲矮的物事弄得极雄壮起来?”未央生道:“不知甚么原故被长兄一激之后,它就平空振作,竟像要发狠争气的一般。连我自己也不能禁止。”赛昆仑道:“你不要骗我。我看皮肤上现有刀痕,四面四条又是一种颜色,毕竟是用甚么巧术造作出来。好好对我直说。”未央生被他盘驳,只得把改造的事细细说了。赛昆仑道:“贤弟好色之心坚韧至此,真不可阻挠了。我只得完备这件事罢,今日就同你撞到他家去看机会。”

未央生大喜。换了衣冠同赛昆仑出去。走到相近的所在,赛昆仑把他安顿在一处,自己先去打探消息。不多时走来回报道:“恭喜、恭喜,今夜就能成事了。”未央生道:“面也不曾见,怎么就保得今夜成事?”赛昆仑道:“我方才去问邻舍,邻舍说她丈夫往远处卖丝去了,有十几日不得回来。你如今同我走进去用心勾搭她,只要有些情意,我晚间自有办法送你进去,包管有十几夜同她快活就是了。”

未央生大喜,两人连忙走去。到了门前,赛昆仑把帘子倡起,同未央生一齐钻进去道:“权大爷在家么?”妇人道:“不在家。”赛昆仑道:“在下要买几斤丝,如今不在家怎么处?”妇人道:“别处去买罢了。”未央生就接口道:“丝怕没处买?只因一向是府上的主顾,不好去总承别人。”妇人道:“既是舍下的主顾,为甚么我不认得?”赛昆仑又接口道:“大娘,我夏天来买丝,也遇着太爷不在,是大娘亲自交易,从架子内取下来与我去的。难道就忘记了?”妇人道:“是记得有这一次。”未央生道:“既然大娘记得,可见不是空口来打价了,如今要有丝,取出来交易就是。为甚么把自家的生意推到别人家去?”妇人道:“丝便有几斤,不知你中意否。”未央生道:“府上的丝岂有不中意,还是忒好了些,怕我这酸子买不起?”妇人道:“好说,这等相公请坐了,待我取出来。”

赛昆仑就叫未央生坐在上面,自己坐在下面。上面近着妇人,待他好调情的意思。那妇人取出一捆丝来,递与未央生看。未央生还不曾接丝到手,就回复道:“这丝颜色太黄,恐怕用不得。”及至接到手仔细一看,又道:“好古怪,方才大娘拿在手里,觉得是焦黄的,如今接到我手又会白起来,这是甚么缘故?”故意想了一会又道:“这是大娘的手忒白了些,所以映得丝黄;如今我的手黑,所以把黄丝都映白了。”妇人听了这话,就把一双眼凑着未央生的手,相了一会,方说道:“相公的尊手也不叫做黑手。”说便说这一句,还是正言厉色,没有一毫嘻笑之容。赛昆仑道:“他的手比了我们的不叫做黑,若比了大娘的就不叫做白了。”妇人道:“丝既然白为何不买?”未央生道:“这是贱手映白的,可见不是真白。毕竟要与大娘的尊手一样颜色的方是好丝。求取出来看看。”赛昆仑道:“世上那有这样白丝,只要象你脸上这样颜色,它就用得过了。”妇人听了这话,又把一双眼睛凑着未央生的脸,相了一会,方才有欢喜之容,对他笑道:“只怕世上没有这样白丝。”

看官,你道她为甚么以前不笑,直到此时才笑?以前不顾眄,直到此时忽然顾眄起来?原来,这妇人是一双近视眼,隔了二尺路就看不见。起先,未央生进去,只道是寻常买卖之人,及至听见“酸子”二字,方才晓得是个秀才也。还只说是寻常人物,不把眼去相他。因为睁眼看人有些费力,所以遇见男子不大十分顾眄。但凡为妇人者,一点云雨之心,却与男子一样都是要认真做事,不肯放松的过了。若是色心太重的妇人,眼睛又能远视,看见标致男子,岂能保得不动私情?生平的节操就不能完了。所以造化赋形也有一种妙处,把这近视眼赋予她,使她除了丈夫之外,随你潘安、宋玉都看不分明,就省了许多孽障。所以,近视妇人完节的多,坏事的少,总是那双眼睛不会惹事。

这个妇人若不是把几句巧话引他眼睛上身,随你立在面前调戏到晚,她只当在云雾之中,那里晓得。只因手上一看,脸上一看,看花了心,就有些开交不得。对着未央生道:“相公当真买不买?若果然要买,我房里有一把好的,取出来看就是。”未央生道:“特地寻来,岂有不买之理。快取来看。”妇人进去一会,果然取出一捆丝来,又叫一个□□丫鬟捧了两盅茶,递与赛昆仑、未央生吃。未央生不敢吃完,留了半盅做个转奉主人之意。妇人看见,又对未央生笑了一笑,方才递出丝来。未央生接丝,就趁手把妇人捏了一把。妇人只当不知,也把指甲在未央生手上兜了一下。塞昆仑道:“这一捆果然好,买了去罢。”就把银包递与未央生。未央生照他说的价钱称了,递与妇人。 妇人道:“这银子成锭,恐怕是中看不中用的。”未央生道:“大娘若不放心,我把丝与银子都放在这边,今晚就夹开一锭,试他一试何如?不是夸嘴说,我们的银子都是表里如一的。”妇人道:“也不消如此,若果不差,下次还可交易。不然,只好做一遭主顾罢了。”赛昆仑拿着丝,催未央生回去。未央生临行,又把妇人唆了几眼,妇人虽不看见,也能领略大意,竟把眼睛收做细缝,似笑非笑的模样送他。

未央生走到寓中问塞昆仑道:“这事有八九分成了,只是今晚怎样进去?”赛昆仑道:“我细细打听过了,她家没有第二个人,只有方才那个丫鬟,才十一二岁,夜间跌倒头就睡着了。她家的房屋是看得见的,又不是楼房,又不是土穴,只消我背了你爬到她屋上,掀去几片瓦,摆去一根椽,做个从天而下罢了。”未央生道:“若还被她邻舍听见,大家捉贼起来怎么处?”赛昆仑道:“有我在身边不消多虑。只是一件,那妇人方才的话说是恐怕你中看不中用的,若还干得她不快活,就是一遭主顾了。劣兄前日的话如今可验了么。你须要自己挣扎,不要被她考倒,只进一场,到第二三场就不得进去。”未央生道:“决不至此,长兄放心。”

两个笑了一场,巴不得金乌西下,玉兔东升,好做进场举子。但不知那位试官是怎生一个考法,须得题目出来方知分晓。

评曰:

小说,寓言也。言既曰“寓”则非实事。可知此回割狗肾补人肾非有是理,盖言未央生将来所行之事,尽狗彘之事也。犹第三回与赛昆仑结盟,而且以兄事之,盖言其人品志向犹出盗贼之下也。皆深恶而痛绝之词,分明是他做狗乌龟、贼乌龟耳。世人不得认贬为褒,以虚作实,谓狗真可割而割之,贼真可交而交之,使作贼之人,反蒙作俑之谤。斯千古文人有同幸矣。

\chapter{第九回 擅奇淫偏持大礼 分馀乐反占先筹}

作者:《肉蒲团》李渔

却说权老实的妻子,名叫艳芳,是个村学究之女。自小也教她读书写字,性极聪明。父母因她姿貌出众,不肯轻易许人。十六岁上,有个考案首的童生央人作伐,父亲料他有些出息,就许了他。谁想做亲一年就害弱病而死,艳芳守过周年,方才改嫁给权老实。

此妇虽好淫,颇知大体,每见妇人有淫佚之事,就在背后笑她。尝对女伴道:“我们前世不修,做了女子,一世不出闺门,不过靠着行房之事消遣一生,难道好叫做妇人的不要好色?只是一夫一妇乃天地生成,父母配就,与他取乐自然该当。若要相处别个男人,就是越礼犯分之事,丈夫晓得要打骂,旁人知道要谈论。且无论打骂不打骂,谈论不谈论,只是这桩事体不干就罢,要干定要干个像意。毕竟是自家丈夫,要做事体两个脱衣上床,有头有脑,不慌不忙的做去,做到后来方才有些妙境。那慌忙急促之中只图草草完事,不问中窍不中窍,着题不着题,有些甚么趣味。况且饥时不点,点时不饥,就像吃饮食一般,伤饥失饱反要成病。那走邪路的女子,何不把后来相情人的眼睛留在当初择婿。若要慕虚名,拣个文雅的;若要图外貌,选个标致的;若不慕虚名,不图外貌,单要干房中的实事,只消寻个精神健旺气力勇猛的,自然不差。何须丢了自己丈夫去寻别个?”那些女伴听了都道:“过来的人,说话自然不同,句句亲切有味。”

怎见得她是过来的人?她当初做女儿的时节,也慕虚名,也图外貌,也要干实事。及至嫁了那个童生,才也有几分,貌也有几分,只道是三样俱修的了,谁想本钱竟短小不过,精力又支持不来。爬上身去肚子不曾猥得热,就要下来。艳芳是个勤力的人,那里肯容他懒惰,少不得作兴鼓舞,又要耸拥他上来。本领不济之人,经不得十分剥削,所以不上一年就害弱症而死。 他经过这一番挫折,就晓得“才貌”二字是中看不中用的东西,三者不可得兼宁可舍虚而取实。所以后来择婿,不要才貌,单选精神健旺、气力勇猛的以备实事之用。看见权老实,生的粗粗笨笨,精力如狼似虎,知道是有用之材,所以不问贫富,就嫁了他。起先还单取精力,不知他的器械何如。只说力雄气壮之夫,不必定用长枪大斧方能取胜,就是短兵薄刃亦可摧锋陷阵。那里晓得竟是一根丈八长矛,所以艳芳喜出望外,自从嫁他之后,死心塌地依靠着他,不生一毫妄念。因他生意微细,日进不多,终日替他络丝,每日有一二钱进益,故权老实得以清闲度日。

只因那一日合当有事,掀开帘子与对门妇人说话,未央生从门首经过,把她细看两番。她因眼睛近视,只看见有个人影在门前过来过去,却不知道面貌何如。谁想倒被对门妇人看了一个像意。那妇人有三十多岁,丈夫也是贩丝卖的,与权老实一同去卖,虽不合本,倒像伙计一般。这个妇人面貌虽丑,性子甚淫。一来因招牌不好,没人想她;二来因丈夫凶狠,略有差错,不是打就是骂,所以还慎法,不敢胡行。那一日,把未央生看得清清楚楚,待他去后,就走过街来对艳芳道:“方才一个绝标致男子走来走去,看你两次。你晓得么?”艳芳道:“你知道我的眼睛可是看得人见的,我坐在这边,哪一日没有几个男人隔着帘子看我,便舍他看看罢了。晓得他做甚么。”妇人道:“往常的男子,你这样人物直不得舍与他看。方才这一个,就等他看了三日三夜也是情愿的。”艳芳道:“怎么这等说,难道有十二分人才不成?”妇人道:“岂止十二分?照我看起来,竟有一百二十分。我终日立在门前,看了许多人,并不见有这样标致的。脸上皮肉,随你甚么东西没有那种白法。眉毛、眼睛、鼻头、耳朵,那一件不生得可爱?身上俊俏竟像个绢做的人物一般。就是画上画的有这般标致,也没有这样飘逸。真教人想思。”

艳芳道:“好笑大娘说得这样活现。我不信世上有这样男子,就有这样男子,他是他我是我,想他做甚么?”妇人道:“你便不想他,我看他好不想你,出神出智,好像落魂了一般。要去又舍不得去,要立又怕别人知。没奈何,只得走过去一会,又重新走转来。临去的时候又去看看。你道可怜不可怜?你不曾看见,自然不想他,我看见他,就替你患起相思病。”艳芳道:“只怕他那种光景不是为我,是为你。你自己相思不好说,得故意把我来出名。”妇人道:“我好副嘴脸,他肯为我?其实是为大娘,大娘不信,他少不得还要来走过,我远远望见他来,就知会大娘。大娘把身子立到外面,一来好看他,二来等他也好看你。”艳芳道:“且等他走过的时节再做道理。”

妇人又说许多话,方才过去。艳芳到第二三日,倒也留心要看,不想过了许多日,再不见来,也就丢开了。及至这一日,来买丝,看见这副标致面貌,自然再想起前话来。等他去后,心上想到,前日所说的莫非就是此人不成?论他外貌,果然是第一品男人,但不知内才何如。他方才有一句巧话,说今天就夹开来试他一试,虽然是说银子,却是双开二意。万一今晚当真走来,我还是拒绝的好,收留的好?终身的名节,坏与不坏,就在这一刻定局了,不可不自家斟酌。

正在踌躇,只见对门的妇人走过来道:“大娘,方才买丝的人你认得么?”艳芳道:“我不认得。”妇人道:“就是我前日说的。你难道不明白,世上那有第二个男子像这样标致的?”艳芳道:“果然标致。只是忒轻薄些,不像正人君子。”妇人道:“大娘又来道学了。世上那有正人君子肯来看妇人的?我们只取人物罢了 ,又不要他称斤两,管他轻薄不轻薄。”艳芳道:“是便是这等说,只是在人面前也该稳重些便好。方才做出许多调戏来,亏得我家主不在,若还在家,看见怎么了得?”妇人道:“怎么样调戏你?对我说说。”艳芳道:“总是不老成,说他做甚么。”

那妇人是个极淫的,听见“调戏”二字,不知怎么样要搂她亲嘴,扯她做事,就不觉摇头摆尾,把手在艳芳身上左捏一把,右敲一下,定要她说。艳芳被她缠不过,就回他道:“方才是两个人,一齐进来,难道有甚么别样?调戏不过就是说话之间眉来眼去,做些勾搭人的意思就是了。”妇人道:“这等,你也该露些好意回答他。”艳芳道:“我不骂他就够了!还有甚么好意回答他?”妇人道:“这就是你的寡情了。不要怪我说,倘这样标致女人,他那样标致男子,真是天生一对,地生一双,原该配做夫妻才是。既不能勾做夫妻也该相处,了了心愿。我想权大爷那样人物不是你的对头,一朵鲜花插在牛粪堆上,也觉可惜。他若再来,我就走过来替你做媒,若把好事干得一两遭也不枉为人在世。”

她一边讲,艳芳一边算计道,看这妇人心上爱他极了,我就要做这桩事,她住在对门,若不把些甜头到她,她岂不坏我的事?我如今不知那人的本事何如,不如让她先弄一次,只当委她考试一般。若还本事好,我然后上场,不怕这样丑妇夺了我的宠去;若还本事不济,我就一顿发作起来,赶他出去就是了,依旧不曾坏得名节,何等不妙?主意已定,就对她道:“这样事我其实不做,他若再来,倒不要大娘替我做媒,待我替大娘作伐,等你两个做几遭好事何如?”妇人道:“岂有此理。莫说大娘这句话未必出于本意,就是出于本心,我这样丑貌他那里肯要?大娘若有好意,除非你两个弄下了手,一遭两遭之后我故意撞来,大娘只说不好意思,扯我也干一遭。这还使得。”艳芳道:“我这话不是假话,有个做法在这边。我方才被他歪缠不过,要拒绝他又放不下脸来,他方才临去的时节说一句巧话,今晚就要摸来也不可知。如今你家男子与我家男子一同买卖去了,总则这里没人,你今晚竟锁了门,到我这边来睡。预先吹灭了灯,待我躲在暗处,他若果然来,你竟假充了我同他睡觉。他在暗地里那里晓得是你,只当替我做了一个人情,又保全了我的名节,不致有亏。何等不妙?”妇人道:“这等说是你许他来的了?我如今心上被你说得痒不过,要辞也辞不得了。只是一件,你为甚么许他来又不肯同他干事?从来的节妇那里有这样做法的?”艳芳道:“不是我假仁假意,定要做这掩耳盗铃之事。不瞒大娘说,房事的滋味,我也尝得透了。随你有本事的,也赶我自家的男人不上。吃过大宴席的些须东道看不上眼,荤不荤素不素,不如不吃的妙。我所以不肯累这个虚名。”妇人道:“你的主意我知道了,权大爷的本钱是一方有名的,你被大喧头喧过了,恐怕那喧周鞋的小喧,撩不着大人的鞋帮,所以要我做个探子,替你探探消息的。我想这事在我也没有甚么折本。只是一件,也要等我干个像意,不要在要紧头上,你又自己冲上阵来,使我进退不得。自古道‘斋僧不饱不如活埋’,这句话你须要记得。”艳芳道:“料想没有这等徼幸的事,你且放心。”

两个商量定了,只等临期行事。这也是那奇丑的妇人一时的造化,奉了这个美差。一个簇簇新新改造出来的喧头,是她这双皮鞋喧起。要知宽窄何如,少刻喧时便见。

\chapter{第十回 聆先声而知劲敌 留余地以养真才}

那个妇人奉了这个美差,满心欢喜。预先寻几块绢袱带在身边,好待干事之时揩抹淫水,省得湿了别人家的被褥。捱到点灯时候,忙把门锁,走过街来。

艳芳故意哄他道:“今晚竟是虚貌了,他方才寄个信来,说被人批住吃酒,脱不得身。还要别约日子。大娘且请回罢。”妇人听了,急得眼中火出,鼻内烟生。又怪艳芳不寄信转去,强她今晚来,又疑艳芳起先失口许了,如今舍不得让人,要赶人回去,自己受用。埋怨了一会,艳芳笑道:“我是哄你。如今想又要来了,只打点与他干事就是。”先烧一盆热水,同妇人净了下身,然后拿一张春塌,铺在床横头,自家睡了,好听他们干事。吩咐妇人把大门关好,悄悄立在门后,他若来必轻轻敲门,你听见敲一下就开门,放他进来。不可使他敲多次,恐怕隔壁人家听见。放他进来之后依旧把门闩好,一同到床上去睡。只是与他说话声气要放轻些,恐怕他认得出。妇人唯唯听命。艳芳就去睡着了。妇人到大门边去伺候。

等了一更多天,不见动静,只得走进房去,正要问艳芳,不想暗地之中有人搂住她亲嘴。妇人只说是艳芳假装男子和她取笑,就伸手去摸他裤裆。才伸得下去,就有一根绝大的东西把手撞了一下,方才知道是本人。就装出娇声来问道:“心肝,你从哪里进来的?”未央生道:“是从梁上下来的。”妇人道:“好个本事。如今上床去睡罢。”两人遂各自解衣服。未央生不曾解完,妇人已脱得赤条条仰睡在床上了。未央生爬上肚去,要摸着她两只脚好架上肩头,不想再寻不见。那里晓得自上床时节已高高翘在半天,献出阴户,只等阳物进来。

未央生想道,不料此妇竟是这等一个淫物,既然如此,那些温柔的家数都用不着了,只得赏她一个下马威。就把下身抬起,离阴户一尺多高,挺起阳物朝下一攻。那妇人就像杀猪一般喊起来道:“阿呀!使不得。求你放轻些。”未央生把两只手替她扒开阴户,慢慢轻轻捱擦捱擦许久,只进得一寸龟头,其余都在外面不能进入。未央生又挺起阳物朝里一攻。妇人又喊起来道:“使不得!求你用些馋唾。”未央生道:“只有弄小官用着那件东西,岂有同妇人干事要用馋唾之理?这例子破不得,还是干弄的是。”挺起阳物又向下直攻。妇人道:“使不得,你若不肯破例,请抽出来,待我自己用些罢。”未央生听了,就把阳物拔出,听她自用。妇人伸开巴掌,吐上许多唾沫,把阴物扒开,灌了一半进去,余剩的都搽在阳物上。对未央生道:“如今没事了,慢慢弄进去。”

未央生要显本事,不肯从容,把两只手捧住她两股,响的一声,将改造长大的阳物一概事攻进去。妇人又喊起来道:“怎么你们读书人倒是这样粗卤,不管人死活,一下就弄到底?如今里头着不下,快拿些出来。”未央生道:“里头着不下,难道如今在外面不成?只该叫它活动些,不要坐冷板凳就是了。”遂运动起来。起初几下,妇人还当不起,每送一次,定叫一声“阿呀”,送到数百之数,就不见则声了。及至送到百外,那妇人就有无限的骚状做出来,无限的淫声唤出来,使人禁持不住,只得一阵紧似一阵,要催他丢过了自己好丢的意思。谁想那妇人有些奸诈,明明丢了两次,问她,只说“不曾”。为甚么不说实话?只因自己是代职的,恐怕艳芳听见,说她心事已完,要来交代。未央生认作真话,再不敢丢。抽到后来,忍耐不住,也丢了一次。丢过之后又不好住手,只是没有勇往直前之气。

妇人见阳物逡巡不进,就问道:“你丢了么?”未央生怕笑他本事不济,只得也说“不曾”。起先未问之先,一下软一下,自从问了这句,竟像学生要睡,被先生打了,那读书的精神比未睡时节更加一倍,遂一连抽上几百下也不停一停。那妇人叫起来:“心肝,我丢了,我要死了!你今不要动,搂住我睡罢。”未央生方才住手,抱住酣睡。原来,妇人面貌虽丑,还亏一双脚小;肌肤虽黑,还不十分粗糙,所以黑夜认不出是替身。

却说艳芳躲在床横头,侧耳细听。起先见妇人叫疼叫苦,弄不进去,就知他的家伙长大,可以用的。又见他的干法在行,抽送有度,不像没有来历的。又见他干到中间,懈了一阵,虽有些鄙薄之意,后来见他重整军容,比入手之初更加奋勇,心上大喜道:“这等看来,分明是阃内之骁才,色中之飞将了,我今就失身与他亦可无悔。欲要趁他歇息钻进被去,说个明白,又怕他在阴暗之中不看见妇人的嘴脸,只说她好似我,还要想去弄她,况男子久战之后,若不把姿色去歆动他,未必能勾再举。就悄悄走到橱下,取起火来,先汲了几瓢水,在锅里下面点一个草把烧着,然后拿烛光走进房去。把帐一掀,绵被一揭道:“是哪一个奸贼?深夜闯入人家奸淫妇人,是何道理?快起来说个明白!?

未央生在睡梦中忽然惊醒,只说是她的丈夫躲在家中,故意等妻子同我睡了,走来捉奸,要我的银子,吓得牙齿乱斗。及至抬头一看,就是夜间所干的妇人。心上想道,难道他家又有一个不成?低下头把那同睡的妇人一看,才知道是个极丑陋之妇。一脸漆黑的癞麻,一头焦黄的短发,颜色就如火腿不曾剥洗过的一般。就大惊道:“这是哪一个?”妇人道:“你不要惊慌,我是替她做探子的,住在对门。那一日,你在门前走过,与你说的就是我。她说你容貌虽好,只怕中看不中用,恐累她偷汉的名,所以央我来试你一试。如今料想见中式了,你同她睡觉罢。我论理也该睡在这边,再讨些赏赐了去。只是旁边有打混的人,你两个就干不爽利,不若我回家去睡罢。”说完就起来,只穿一领绵袄,一条夹裤,其余衣裙物件都挂在手臂上,带了回去。临去时又对未央生道:“我的容貌虽丑,也是你的功臣。这事是我说起的,今晚与你睡这一次,一来是大娘的好意,二来也是前世的姻缘。后来若有闲空的工夫,也还同我睡睡,不要十分寡情。”说完又对艳芳拜几拜,谢了东道主人,方才出去。

未央生如醉初醒,如梦初觉,若不是赛昆仑激我改造,今日进来只好做个秦邦赴考的苏秦,不中文章,白白赶了出去。艳芳送妇人去后,把门闭好了走进房来,对未央生道:“我晓得你今夜放我不过,特寻一个替身等你,你如今与他干事一次,也消得我的账了,还不出去,在这里干甚么?”未央生道:“不但消不得账,还要加你的罪,如今已是半夜了,快些上床来睡睡。”艳芳道:“你且起来披了衣服,做一件紧要事,才好同睡。”未央生道:“除了这一桩,还有甚么紧要事?”艳芳道:“你不要管,只爬起来。”说完走到橱下,把起先温的热水汲在坐桶里,掇来放在床前。对未央生道:“快些起来,把身子洗洗,不要把别人身上的龌龊弄在我身上来。”未央生道:“有理。果然是紧要事。我方才不但干事,又同他亲嘴,若是这等说,还该漱一漱口。”正要问她取碗汲水,不想坐桶中放着一碗热水,碗上又架着一枝刷牙。未央生想道,好周至女子,若不是这一出,就是个腌脏妇人,不问清浊的了。

艳芳等他漱洗过了,自己也把下身洗濯。她下身起先已与妇人一齐净过了,为甚么又要洗濯起来?要晓得她睡在床头听他干事的时节,未免有淫水出来,恐怕未央生摸着要讥诮他,所以再洗一次。洗过了把一条湿手巾揩抹了,又在箱子里取出一条新汗巾,放在枕边。方才吹灭了灯,坐在床上。未央生搂在怀中,一边亲嘴,一边替她脱下衣服。只见两个乳峰捏来不上一把,放去竟满胸膛,总是娇而且嫩,里面没有块磊的原故。及至脱去裤子,摸着阴物,其骄嫩与乳峰一样。未央生放她睡倒。先取一双小脚架在肩头,然后提起下身,也像弄丑妇的方法远远舂进去,要等她先受苦,后来才觉得快活。不想舂进去艳芳心上只做不晓得一般。未央生思想赛昆仑的言语一字不差,若没有权老实的粗长之物,焉得有此宽大之阴?我若未经改造,只好做大仓一粒,焉能窥其底里?如今军容不足以威敌,全要看着阵势了。就把他头底下的枕头取来垫在腰下,然后按了兵法同她干起。

艳芳不曾到好处,但见他取了枕头下去,又不再取一物与她枕头,就晓得此人是个惯家了。取枕头垫腰是行房的常事,怎见得就是惯家?要晓得男女交媾之事,与行兵的道理无异,善对敌者才能用兵。男子晓得妇人的深浅方知进退。妇人知道男子的长短,才识迎送。这叫做“知彼知己,百战百胜。”男子的阳物长短不同,妇人的阴户浅深不一。阴户生得浅的,就有极长之物也无所用。抽送之际定要留有余不尽之意。若尽根直抵,则妇人不但不乐,而且痛楚。男子岂能独乐乎?若阴户生得深的,就要用着极长之物,略短些也不济事。只是阳物生定怎么长得来到其间,就要用补凑之法。腰之下股之上,定须一物衬之,使牝户高张,以就阳物,则纵送之时易于到底。故垫腰之法,惟阳短阴深者可以用之,不是说枕头乃行房必须之物也。所以男子的阳物短者可医,小者不可医。与其小而长,无宁大而短。术士替未央生改造之时,只求其大,不使其长,就是这个缘故。

如今艳芳的深,未央生的短,所以取枕头垫在下面。岂不是惯家?这种道理世上人还有知道,至于取枕头垫在腰下面,竟不取他物与妇人枕头,这种法窍就没人参得透了。妇人腰底下既有一物,若还头底下又有一物,则上身一段不过二尺多长,两头凸起,中间凹下,只当把妇人的身体拘断在下面,上面又压了一个男子,你道她气闷不气闷,辛苦不辛苦?况且妇人枕了枕头,面庞未免带反,口齿唇舌都与男子不对,极不便于亲嘴。男子要亲嘴必须鞠着身子往下面凑;妇人要亲嘴,必须便起颈项朝上面凑。碍了一个枕头,费人多少气力,所以干事之时无论垫腰不垫腰,总是颈项底下的东西断断留他不得。会干事的,将要动手,就把枕头推过一边,使她云鬓贴席,朱唇面天,五官四肢没有一件不与男子相合。上下二孔又与别的肢体不同,不惟相合而且相投,不惟相投而且相出入。男子的玉麈入于女子阴中,女子的绛舌入于男子口中,使她也有一件的便宜处。则乐事相均,而无有余不足之事矣。

未央生把一只手取枕头下去,就把一只手托住她的头颈,安顿在席上,使面孔不歪不邪,以预为亲嘴之地。所以艳芳暗喜,知道他是惯家。未央生垫腰之后,重新提起小脚放在肩头,把两只手抵住了席,放出本事尽力抽送。每一抽,定要拔出半截;每一送,定要抵个尽根。只是一件,抽便抽得急,抵却抵得缓。为甚么缘故?他恐怕下去急了要入得阴户响,恐怕邻舍人家听见,弄出事来,所以不敢放手。干了一会,那阴户里面渐渐紧凑起来,不像初干的时节汗漫无际了。未央生晓得是狗肾发作,阳物大起来的原故。就不觉精神百倍,抽送的度数愈加紧密。

艳芳起先不动声色,直到此时方才把身子扭几扭,叫一声道:“心肝,有些好意思来了。”未央生道:“我的乖肉,方才干起头,那里就有好意思?且待我干到后来,看你中意不中意。只是一件,我生平不喜干哑事,须要弄得里头响起来,才觉得动兴。只是你这房子狭窄,恐怕邻舍听见,不好放手,却怎么处?”艳芳道:“不妨。一边是空地,一边是人家的厨房,没有人宿的。你放心干就是。”未央生道:“这等就好了。”此后的干法就与前相反,抽得缓,送得急。送进去的时节,就像叫花子打肋砖,要故意使人听见好可怜见他的一般。

翻天倒地干了一阵,艳芳骚性大发,口里“心肝、儿子”叫不绝声,牝中淫水旁流横溢。未央生见她势头来得汹涌,要替她搽抹干了,重新再干,就伸手去取汗巾。不想摸到手里被艳芳抢去,不容他揩抹。这是甚么缘故?原来,她的生性也是不喜干哑事的,与未央生所好略同,但凡干事之时,淫水越来得多,响声越觉得溜亮。所以她平日干事随下面横流直淌,就把身子都浸在里边,也不许丈夫揩抹,直待完事之后,索性坐起来,把浑身上下拭个干净。这是她生平的嗜好之癖。未央生见她不肯揩抹,就悟到这个缘故,比前愈加响弄起来。又翻天倒地干了一阵,艳芳就紧紧搂住道:“心肝,我要丢了。你同我一齐睡罢。”未央生要骋本事,还不肯丢。艳芳道:“你的本事我知道了,不是有名无实的。如今不肯住手,弄了一夜,抵敌了两个妇人,也是亏你。可留些精神明夜再干。不要弄坏了人,使我没得受用。”未央生见她这几句话说得疼人,就紧紧搂住,又抽了一番。两个才一齐完事。完事之后,不曾说几句话,天已将明。艳芳怕他出去迟了被人看见,只得催他起来,自己也穿了衣服,送他出去。

从此以后,未央生晓去夜来,俱是从门里出入,再不做梁上君子了。还有几次舍不得分别,连日里也藏在家中。艳芳只推生病,不出去开门。两人青天白日一丝不穿,彼此看了雪白的肌肤,恣其淫乐。对门的丑妇隔一两夜过来一遭,未央生不好拒绝她,也时常点缀点缀,但不能饱其所欲,只好免于怨恨而已。左右邻舍有几个听见些响动的,都只说赛昆仑自己来奸她,再不想是替别人做事。未曾到晚,各家都闭户关门,不管外面闲事,惟恐赛昆仑恼他碍眼,要顺便去算计他。所以一连睡了十几夜,没有一毫惊恐。直到权老实回来之后,方才断了踪迹。

赛昆仑恐怕未央生年少心性,弄出事来,连日间也禁止他,不许到门前去窥探。宁可自己做红娘,终日托名买丝替他传消递息。权老实有几次在家,只说是生意主顾,平日与妻子交易惯的,自己倒立过一边,凭他两个说话,一味忠厚到底,不以诡谲待人。这才叫做权老实。始信天下的混名叫得一毫不差。不像自家取表德,只拣好字眼称呼。天下择交之法,不必察其为人,观其行事,只问此人叫做甚么混名,就知道交得交不得也。

评曰:

千古不传之秘,千金不易之方,尽为世人泄之,殊觉可惜!

\chapter{第十一回 穿窬豪杰浪挥金 露水夫妻成结发}

作者:《肉蒲团》李渔

诗云:

豪杰从来数绿荫,一逢知己便挥金。
衣冠亦复多豪客,何事全无念友心?

艳芳与未央生睡了十几夜,那种云雨私情正在稠密之时,被丈夫回来打断好事,苦不可言。心上想道,我起先只说天下的男子,才貌与实事决不能相兼,我所以去了才貌,单取实事。把个粗蠢东西当做宝贝一般,终日吃辛吃苦,帮他做活。那里晓得男子里面原有三件俱全的,我若不遇着这个才子,枉做了一世佳人。如今过去的日子虽不可追,后来的光阴怎肯虚度?自古道“明人不做暗事”,做妇人的不坏名节则已,既然坏了名节,索性做个决裂之人,省得身子姓张肚肠姓李。我常说从来的妇人,有红拂妓的眼、卓文君的胆,方才可以偷汉。生平只偷一次,一偷就偷到底,连那个偷字后面也改正过来,才是个女中豪杰。况且“淫奔”二字原分不开,既要淫就要奔,若度量后来奔不得,就不如省了那些孽障,做个守贞不二之人,何等不妙?为甚把名节性命去换那顷刻的欢娱?

主意定了,就写书一封寄与未央生,约要私奔。他当初在母家的时节,极喜读书写字,只因嫁做商人妇,就把笔研荒疏了,所以写的书扎如说话一般。书云:

情郎未央生赐览:

自你不来之后,我终日对了饮食吞咽不下。就勉强吃下去,不过三分之一。我如今立定主意,随你终身。你可速速料理,或是你烦赛昆仑进来盗我,或是我做红拂前来奔你。只要期定日子,约在何处等我,不致彼此相左。至嘱至嘱。

你若虑祸,踌躇不敢做此险事,就是薄悻负心之人,可写书来回我,从此绝交。以后不得再见,若还再见我,必咬你的肉,当做猪肉狗肉吃也。

馀言不尽,只此寄知。

辱爱妾艳芳敛衽拜寄

写了此书,立在门前,看见赛昆仑走过,付与他带去。又怕未央生胆小,不敢行此险着,又生一计:终日寻是寻非,与权老实争闹,使他不能相容,好做朱买臣的故事。就终日只推有病,一根丝也不络,连茶饭都要丈夫炊煮。每日清晨起来,咒骂到晚方才停息,至于干事之时,把摆布前夫的手段重新放出来,要打发他上路,好嫁三样俱全的丈夫。权老实见他日里憎嫌不过,只得竭力奉承,指望将功赎罪。谁想夜里的功劳补不得日间的过失,爬下床来,就换了一副面孔,把一个如狼似虎的丈夫不消两月,磨得骨瘦如柴,恹恹待毙。邻舍见了个个不平,只是惧怕赛昆仑,不敢说得。

权老实见妻子一向安心贴意,忽然改变起来,知她必有缘故,就在邻舍面前细问消息,说“我出门的时节,可曾有甚人在我家往来?”邻舍起先只推不知,后来见他盘问不过,又怜他是个忠厚之人,将来要死于淫妇之手,得说道:“有便有一个人在你家走动,只是不可惹他,若惹他就有不测之祸。”权老实道:“是甚人?这等厉害?”邻舍道:“就是天下驰名,人人俱怕,惯做神贼的赛昆仑。旧时在你门前经过,看见你娘子美貌,就走来问我们说‘是哪一个的妻子’,我们说是你的令正。他又说‘这样妻子嫁了那样丈夫,平日夫妻之间和睦不和睦’,我们又说是极相得的。后来见你出去卖货,走来问道‘权老实这番出去有几日才得回来’,我们只说你去卖丝,有十几日才得回来。不想那一日起,你家夜夜像有人说话一般,若是别个,我们就好出来稽查,你晓得太岁头上可是动得土的?不去惹他,尚且要来照顾,况得罪他有个不来搅扰的?又且律法没有邻舍捉奸之理,所以凭他自来自往,宿了十几夜,直待你回来方才断了这路。我说便对你说,只好放在肚里,切不可泄漏出来,招灾惹祸。就在令正面前也只宜隐忍,不可说破。恐怕走漏消息,害你性命。”

权老实道:“原来如此。今既蒙吩咐,怎敢漏泄。但他终有日落在我手里,待我拿住了他,杀头的时候,求列位高邻助我一臂之力。”邻舍道:“这都是呆话,自古道‘拿贼拿赃,拿奸拿双”,他做了一世贼,不曾被人拿着赃,难道通了奸情就被你拿着双不成?令正既被他奸,终有日被他领去,只保得不赔妆奁也就够了。”权老实道:“怎见得如此?”邻舍道:“他平素的手段你难道不知?任你高墙厚壁,他也有本事进去,何况你这几间小屋?终究被他钻进去把人领去。人既被他领去,那屋中的财物岂保得不做妆奁?你不可不堤防。”权老实听了大惊,就对邻舍跪下求他画策免祸。邻舍怜他情急,个个代他算计。有的劝他休了妻子,断绝祸根 ;有的教他带了妻子搬远处去。内中有一个老成的道:“这都不是主意。他令正虽有可出之条,却不曾拿捉赃据。把甚题目休他?赛昆仑的路数没有一处不熟,随你搬在那里去,他也会寻着。这都不是良策。依我愚见,只有将错就错之法,可以做得。你妻子既然无心靠你,留在家中也没有用。不如卖些银子用用。若卖与别人,令正决不肯去。就是塞昆仑知道,怪你断他恩爱,也要来报仇。不如就卖与他。他既然爱你令正,或者肯出一二百两也不可知。你拿了这宗银子过来别讨一个妇人理家,自然不至招灾惹祸。又得了人又保得不破财,岂不两便?”权老实道:“此计甚好。只是我自己不好去说,须得别一个对他说话便好,不如列位中那一位肯替我周全否?”邻舍道:“若肯如此不妨与事,只是卖去之后,你不可生端,说我们通同奸贼,占你妻子,这就使不得了。”权老实道:“若做得成,我身家性命都亏列位保全,怎敢做此负心之事?”众人听了就大家酌拟一个会说话的,约次日去寻赛昆仑说话。

却说未央生自与艳芳别后,害起相思病来,终日废寝忘食。欲要赛昆仑去拐她出来,又恐她丈夫缉获;欲领她远去,又想起两个特等妇人不曾弄得上手,舍不得丢了远去。心内踌躇不决。后来看见艳芳的书写得极激切,只得定了主意。就求赛昆仑拐她出来,情愿领她到远方去,使她丈夫缉访不着。赛昆仑道:“若肯如此就好处了。但权老实是个穷汉,没了老婆,那里还讨得起。凡人情倒了极处就有性命之忧,不可不替受害之人想个退步。除非带百十两银子丢在他家,然后拐出人来,使他失了一个,还好再讨一个,这等做来才不失我做英雄的本色。”未央生道:“此计虽好,只是小弟旅囊羞涩,设处不来。奈何?”赛昆仑道:“贤弟不消忧虑,我做了一生豪杰,若拼不得挥金,怎敢说此仗义的话!要银子都在我身上,你可写书回她,不拘时日,只要权老实不在家我就去拐她出来。”

未央生大喜,就写下书札,也不用文理深奥,只把几句浅话回她,省得她费解。其书云:

艳娘芳卿赐览:

别得两个月,竟像几十年,终日寝食俱废,履告昆仑求他力图,他只恐尊意不决,所以不敢轻举。因看来札,始知勾我之心坚如铁石,今已力任不辞矣。

红拂之事甚险,切不可做。既有此人出力,只做红绡可也。佳期难以预卜,典守离家之日,即是嫦娥出月之期。速赐好音,以便举事。

别话不宜,只此奉复。

真名不具

赛昆仑拿了此书送与艳芳之后,就取一百二十两银子,预先封好,好待临时带去。过了两日,忽见她邻舍走来说:“权老实生意折本,日给不敷,不能养活妻子,要转卖与人,我想你为人宽胸大度,有闲饭养人,又肯济贫扶危,所以特来作合。求你积个阴德,一来超拔此妇出来,省得她饿死;二来使权老实得些聘金,好做生意糊口。极是两便的事。”赛昆仑听了暗想道,有这样奇事?我正要去算计他,他就央人来卖与我。或者他晓得些风声,知道我替人做事,料想出不得圈套,故此来上这条路也不可知。既然如此,我要暗买不如明买了。就问邻舍道:“他既贫穷要卖妻子,不知他妻子肯去否?”邻舍道:“她在家受苦,巴不得出门。有甚不肯去。”赛昆仑道:“他要多少财礼?”邻舍道:“他要讨二百两,若不得,一百两外多些,他也就肯了。”赛昆仑道:“既然如此,就是一百二十两罢。”

邻舍见他允了,就去叫权老实亲来交易。赛昆仑初意,要教未央生做受主。后来想道,我的名声人人惧怕,不敢同我打官司。若叫他出名,后来就有官司口舌了。所以不提起未央生,只说自己要做意。权老实走到写了婚书,打了手印,邻舍押了花名,交与赛昆仑。赛昆仑取出那封银子,恰好是这些数目,又别取十两,送与邻舍做媒钱。当日就佣轿子,把艳芳抬过来,也不使未央生知道。直待他寻下房屋,置了床帐家伙,方才备办花烛,把他两个送入洞房。虽鲍叔之交情,虬髯之侠气,不过如此。只可惜把题目错认,所以算不得为豪杰也。

\chapter{第十二回 补磕头方成好事 因吃醋反结同心}

作者:《肉蒲团》李渔

未央生与艳芳做了夫妻,两个不分昼夜尽兴快乐。艳芳进门之后,经水来得一次就有了胎。未央生大喜,以为术士之言不验,一般也会生育,这根取乐之具是落得改造的。到了四五个月,艳芳的腹渐渐大了,行房之时未免碍手碍脚,不能如意。艳芳就吩咐未央生,教他权且耽搁几时,养精蓄锐,待生育之后好图大举,不要枉费了精神。从此以后,两个就分房宿歇。

未央生独睡在书房,不免静极思动,又要做起分外事来。心下想到,我目中所见的妇人,只有那两个不见姓名的是个绝色,与我新娶的这一个可以鼎足而立。怎奈不知下落,无处寻访。不得已而求其次,只好在册中遴选一个出来,暂救目前之急。就瞒着艳芳把书房门关了,取出册子来细细翻阅。

翻着一个名字叫做香云。批他的批语虽不多几句,比别个的略加厚些。这分明是第一等之第一名,比绝色的女子止争一间也。

批云:
此妇色多殊美,态有馀研。轻不留痕,肢体堪擎掌上;娇非作意,风神俨在画中。因风嗅异香,似沽花气;从旁听妙语,不数莺簧。殆色中之铮铮,闺中之娇娇者也。拔之高等,以冠群姿。

未央生看了批词,追想她面貌。记得是个二十以外、三十以内的人,神致妩媚,从前走过,觉得有一阵香气,与熏在衣上、带在身上的不同。既去之后,又在香案旁拾得一把诗扇,知道是她所遗。未央生想了数日要去踪迹他,因后来遇着特等的,就把她丢下。此时翻阅着了不觉死灰复燃,就把下面的小字细查,看她住在何处。原来与自己的住处同是一个巷名,心上大喜,忙走出去问人。

那里晓得作孽之事偏生凑巧,这个女子就是他的紧邻。只有一墙之隔,书房间壁就是她的卧房。丈夫叫做“轩轩子”,是个才高行短的秀才,年纪有五十多岁。前妻已死,香云是他的继室。轩轩子在外处馆,每一个月回来宿一两次,其余日子都在馆中宿歇。

未央生访问的实,心上暗暗喜道,这分明是前世的姻缘,神差鬼使送我住在这处同她作乐的了。忙回到家中,一边想计策,一边看形势。书房外面的墙虽然不高,是有房子隔住的,跳不过去。书房里面的墙是夹砖砌的,又有白灰粉在上面,一动就有痕迹,又不好凿孔。存想了一会就要做爬梁上屋之事。仰起头来细看,只见 屋山头上有三尺高五尺阔的一块,是砖墙砌不到,用板壁铺完的。心上喜道,既有这隙可乘,又不消想到屋上去了。只消把板壁撬去几块,那砖墙上面就可以跳得过了,有甚么难做的事? 就掇一张梯子斜靠在墙上,然后到书橱里取出一副家伙,外面是个纸匣,纸匣里面刀、斧、锯、凿样样都有,名字叫做“十件头”。未央生自买回来一件也不曾用,只说是没用的东西。那晓得天下无弃物,要偷妇人竟用着它。就带了这副家伙爬上梯去,把那板壁一看还喜得有缝可寻,就先用一把小锉将横木之上锉去二分使橇板下来的时节没有障碍。然后用小凿投入缝中用力一橇,已下来一块。一连橇下三块,就伸头过去一张。

看见一个妇人坐在马桶上小解。解完之后未曾系裤,先盖马桶。那马桶盖落在地上,伸手去取,未免屈倒纤腰,把两片美臀高高耸起,连那半截阴门也与未央生打个照面。未央生在背后看了,还不知可是本人。直等得撒上裤子,掉过脸来,仔细一认,正是当初赏鉴的人。未央生要叫她一声,一来怕被人听见;二来我在暗中,她不知我是何人,怎么肯来招接?万一发作起来,反为不便。须要设计引她上来张我,看见我面貌,不消我去仰扳她自然来俯就了。想了一会,忽然记起她当日遗下一把扇子,上面有三首唐诗,是她亲笔写的,我如今把板壁开在这边,走下梯去寻出那扇子,把上面的诗高声朗诵,她听了自然会意,比上来张我。然后用巧话挑拨她,自然一勾便上了。

算计已定,就下去开了箱子,搜寻那把诗扇。他在庙上作寓之时,烧香妇女所遗之物甚多,不止捕把扇子。拾得一件就收藏一件,又怕与别的东西混在一处,一时要寻难以寻起,又别作一箱,盖上写四个大字,取国风上一句,是“美人之贻”四字。此时开了箱子把那些哀艳之物细细拣阅。阅到一把扇子就是她的,展开一看,上面写三首绝句,乃唐朝才子李白所作,名为清平调,是唐玄宗与贵妃赏牡丹召他进宫做的。未央生不敢造次就念,先把衣冠换得齐齐整整,然后打扫喉咙,竟像昆腔戏子唱慢调的一般,逐字逐句哦出韵来,等她好仔细听,诗云:

云想衣裳花想容,春风拂栏露花浓。
若非群玉山头见,会向瑶台月下逢。(右其一)

一枝红艳露凝香,云雨巫山枉断肠。
借问汉宫谁得似,可怜飞燕倚新妆。(右其二)

名花倾国两相欢,常得君王带笑看。
解释春风无限恨,沈香亭北倚栏杆。(其三)

念过一遍不见响动,就把落款年月与写字之人的姓名当做曲子里面的介白一般,也念出来。要使她听得明白,又念了几遍。只见板壁上忽然响了一声人,像咳嗽又像叹气。未央生知道她上来了,就对着扇子埋怨道:“为这一把扇子弄得人死不死、活不活,如今扇子在这边,人在哪里?若还寻得着,不如送还她,留在这里做甚么!”

说了这话,只见板壁上有人应道:“扇子的主人现在这里,丢上来还我!”未央生抬头看见,故意吃一惊道:“原来绝世佳人就在咫尺,枉害了一向相思。这等说死不成了。”就把十步梯子并作五步跨上去,一见了面就搂住亲嘴。

香云问道:“你一向在哪里?再不能见面,如今为甚么走到这里人家,念起我扇上的诗来?”未央生道:“这就是我的寓所。我就是你的紧邻。”香云道:“既住在这里,为何一向不曾见你?”未央生道:“我是新搬来的。”香云道:“你为甚么搬到这里来?”未央生要买她的欢心,就随机应变,想出话来道:“我搬来的意思都是为你。因前日在张仙庙看见尊容,心下十分想念,见你临别之时十分顾眄,又留下扇子赠我,所以丢你不下,谋到这里来住,好与你相处的。”

香云听了微笑一笑,把手在未央生肩上轻轻打一下,道:“你原来这样有情,我错怪了你。你家里还有甚么人?”未央生道:“只有一个小妾,是朋友赠我的,其余的贱眷都在故乡,不曾带来。”香云道:“你未搬来之先,如何不到我门前走走?使我终日想你。”未央生道:“我起初问你不着,不知住在哪里。后来访知下落就搬过来就你了。”香云道:“是几时搬来的?”未央生道:“不上半年,只好四五个月了。”香云一听这一句,登时变脸就问道:“既然来了这些日子,为甚么往常不理我?”未央生见她词色不好,知道露了马脚。又把巧话支吾道:“一向只说尊夫在家,恐怕轻举妄动,贻害于你,所以只当不知道。直到今日,方晓得尊夫在馆,家里没有别人,才敢露些声色。不过谨慎的意思。难道敢忘记了你不成?” 香云听了冷笑一声,又问道:“我的扇子可还在么?”未央生道:“紧紧藏在身边,不敢遗失。”香云道:“你拿来我看。”未央生听了就下去取来,把一把汗巾子裹住,双手递过去。不想香云接到了手两三把扯得粉碎,往自己房里一丢,遂将汗巾子掷还他道:“这样薄情的人亏得不曾与你有染。从今以后两下闭交。下去吧。”就忿忿走下梯子,眼泪汪汪的哭起来。

未央生不知她为着何事,要爬下去问个明白,又怕被人撞见,只得立在上面看了她哭。正在难处之际,忽然书房门外芭蕉弄声,竟象有人走动。未央生怕是艳芳,只得上了板壁,走下梯来。心上猜疑道,这是甚么原故?又不曾有话冲撞她,为甚么使起性来?察她口气不过怪我亲近迟了,耽搁半年工夫,不曾与她作乐,要逼我去请罪的意思。但日间不好过去,待到晚上钻过去问个明白。无论她怪得有理没理,总是陪她个不是就完了账。

主意定了,挨到黄昏时候,打发艳芳睡了,来到书房。把门窗紧闭,遂爬上梯去。将日间橇动的板壁尽数除下,心上想道,她那边没有东西拔脚,二丈高的高墙如何跳得下去?欲要叫她一声,她既说了硬话,怎么肯来接引?谁想香云口嘴虽硬,心肠还软。临睡之时,原开一面之网在那边招纳她。未央生爬到墙上伸手过去一摸,只见日间所用的梯子不曾撤去,依旧放在那边,若有所待。未央生大喜,就踏着梯子悄悄爬下去。只见黑洞洞不辨东西,又悄悄摸到床前,见她不响不动,只道是睡着了。就伸手去揭她被,要把身子钻进去。

那里晓得香云此时也不曾睡着,未央生过来的时节她明明听见,只因要省些客气,所以朝里睡了,只当不知。及至他伸手来揭被,这番客气就省不得了。只得转身来,假装梦中惊醒的模样,叫道:“你是哪一个?黑地里爬到我床上来!”未央生靠着耳朵低低说道:“不是别人,就是日间与你说话的人。知道自家不是,特过来请罪。”一面说一面钻进被窝。香云紧紧裹住,不放他进去。发作道:“这样寡情的人,那个要你请罪?”未央生道:“我费尽心机谋到这边来亲近你,也不叫做寡情了。香云道:“你那双眼睛好不识货!怕没有标致的同她作乐,希罕我这等丑陋东西?”未央生道:“我家里一妾是朋友赠我的,我不得不受。娘子怎么吃起醋来?”香云道:“你同自家妻妾作乐是该当的。我怎么好吃醋?只是与我一样的人,你不该先去缠她,把我丢在九霄云外。若住在远处也罢了,只隔得一壁,叫也不叫一声,竟象不相识的一般。这样寡情的人还要人理?”

未央生道:“娘子这话是从哪里说起?我除了一妾之外,并不曾相处一个妇人。娘子为何谤起我来了?”香云道:“我且问你,某日某时张仙庙里,有三个标致妇人进去烧香,有人跪在门外磕头,可就是你么?”未央生道:“那日果有三个妇人在庙烧香。彼时我也去求神,见有一阵在内,恐怕混杂不雅,所以不好进去。就跪在门外磕头。我是拜张仙,难道拜那三个妇人不成?”香云笑道:“何如自己招出来了。既然磕头是你,还有甚么辨得?你前日躲在张仙背后偷看妇人,见有少年女子竟不怕混杂不雅,直走出来调戏他。岂有妇人在里面反怕混杂不雅,跪在门外磕头之理?这样混话骗三岁孩子也不信,反要来骗我。”

未央生听了,知道掩饰不过,只得吐露真情,好套出那三个妇人的下落。就对他笑一笑道:“不瞒娘子说,我那日磕头一半为神,一半为妇人。但不知娘子坐在家里怎么晓得这事?是哪个对你说的?”香云道:“我自有千里眼、顺风耳,何须要人说得?”未央生道:“娘子既晓得这事,必晓得那三个妇人住在哪里,叫甚么名字,丈夫叫甚么名字,索性求娘子说个明白。”香云道:“你同她相处半年怕不晓得,反来问我?”未央生道:“这话从哪里说起。我从一见之后就不曾再见。怎说与她相处半年?这冤情叫我哪里去申诉!”香云道:“你既然不曾与她相处,为甚么半年之中不见我一面?分明是她们叫你不要理我。我难道不晓得?”未央生道:“屈天屈地何曾有一些影响,娘子若不信,待我对天发誓:我若与三个妇人有一毫于染,天雷立即打死!”

香云见他发的誓愿,疑心也释了一半。就说道:“既是这等,你的罪过还可原。”未央生道:“如今我说明了,请娘子揭开被窝,放我进来睡罢。”香云道:“我的面貌不如那三个妇人生得标致,你还是去寻标致的睡,不要来缠我。”未央生道:“娘子又太谦了,怎见得你的面貌不如那三个?”香云道:“你的眼力自然不差,毕竟是她标致你不肯跪下磕头。”未央生道:“那磕头的事不过是兴之所至,偶然做出来。何曾有甚么成见。据娘子这等说,是怪我磕她的头,不曾磕你的头,所以这等牢骚不平。我如今加上利钱多磕些头,补还前日的欠账就是了。”

说完,遂跪在床前一连磕了几十个响头,把床都振动。香云才伸手下来扶他上床。未央生脱了衣服,钻进被窝。那阳物就与阴户凑着,竟象轻车熟路走过几次的一般。想来是初交之际,彼此情浓,又被客气的话耽搁了一会,到此时所以我要凑他、他要凑我,两件来而自然倾盖如故。未央生凑着之后,就把阳物直抵牝中,是要借些痛意,好煞阴中之痒。香云欲图快活,所以耐着艰难,任他抵塞。未央生见她承受得起,就放出本事,同她对垒。起先几十提,里面倒还滑溜。到半百之后,渐渐有些濡滞起来。

香云抵挡不住,就问道:“我往常与自家男人干事,都是先难后易。为甚么今日不同,反先易后难起来?”未央生道:“我的阳物与人不同,有两桩异样。第一桩是先小后大,起初象一块干粮,一入牝就渐渐大起来,竟象是浸得胀一般。第二桩是先冷后热,就象块火石,擦磨几下渐渐热起来,就象有火星要爆出的一般。只因有这两桩好处,所以不敢埋没,要来亲近娘子,求你赏鉴的意思。”香云道:“不信你身上有这样宝贝,只怕是哄人的话。就作是真的,怎么这等艰难起来?”未央生道:“如今牝内干燥不过,所以艰难。少刻有些淫水浸润,自然不象这等了。”香云道:“这等,待我敖住了疼,任你狠弄一阵,弄些淫水出来,省得里面干涩。”

未央生听了,就把双脚架在肩上,紧紧抽送。不上几十下,那阴户也滑起来,阳物也热起来。滑则不觉其疼,热则愈见其乐。香云道:“真个你方才的话不是哄我,我如今快活了。”未央生就乘势愈加狠弄,一边使她欢心,一边套她的口气道:“心肝,我这话不哄你,可见别样的话也不是哄你。你可把那三个妇人的话对我说说何妨?”香云道:“只要你真心待我,我自然会对你说。何你这等忙?”未央生道:“也说得是。”就把她的舌头紧紧含在口里,再不说话,一味哑干,足足抽了一二更天。只见香云手足冰冷,一连丢了三次,就说道:“心肝,我的精神单薄,再经不得掏掳了。搂着我睡罢。”

未央生听了,爬下身来搂住同睡。睡的时节觉得一阵异香,与那日初会时闻见的一样。就问道:“你平日熏衣服的是甚么香?这等可爱。”香云道:“我平日并不熏香,你在哪里闻得?”未央生道:“那日相见的时节,你在我面前走过,就有一阵香气。今日睡在床上,也是如此。你平日若不熏香,这一种气味是那里来的?”香云道:“这是我皮肉里面透出来气味。”未央生道:“不信皮肉里面有这样好气味,若是这等你皮肉也是一件宝贝了。”香云道:“我生平也没有别长,只有这一件与别个妇人不同。当初父母生我时,临盆之际有一朵红云飞进房来,觉得有一阵香气。及至生我下来,云便散了。这种香气再不散,常常在我身上闻出来,所以取名叫做‘香云’。若坐了不动,还不十分觉察,但是劳碌之后,有些汗出,这种气味就从毛孔里透出来,不但别人闻得出连自家也闻得出。我有这件好处也不敢埋没,前日庙中与你相逼见,你生得标致,故把扇子赠你,又把这种气味与你赏鉴,要你寻到我家来。谁想你不来,直到今日方了得心愿。”

未央生听了就把她浑身上下仔细闻,没有一个毛孔不有香气。方才晓得绝世佳人不是相得出来的。就把她紧紧抱住,一连叫了几十个心肝。香云道:“我身上的香气你都闻到了,还有一种香气更比身上的不同。索性与你赏鉴。”未央生道:“在哪一处?”香云把一只手捏着未央生的指头,朝阴户里面点一点,道:“此中的气味更自不同。你若不嫌亵渎,也去闻一闻看。”未央生缩下身子,去把鼻孔对着阴门嗅了几嗅,就爬上来道:“真宝贝,真宝贝!我如今没得说,竟死在尔身上罢。”说了这话,又把身子缩下去,扒开那件宝贝,就用舌头舔将它起来。香云道:“这怎么使得!还不快些上来。”一面说一面去扯他。越扯得急,未央生越舔得慌,把一根三寸长的舌头竟作了干事的阳物,在里面一抽一送,一来一往,与交媾无异,一见有淫水流出来,就吸在口里,吞下肚去。直舔得她丢了,连阴精都吃下去,方才爬上肚来。香云紧紧抱住道:“我的心肝,你怎么这等爱我!我如今没得说,也死在你身上罢了。”

未央生道:“照我看来,你这样佳人如今世上没有第二个了。你既有这件宝贝,你的丈夫为甚么不回来受用,终日睡在外边,使你孤眠独宿?”香云道:“他心上也要受用,只是力量不济,所以借处馆的名色在外面躲避差徭。”未央生道:“我闻他还是中年的人,怎么就这等不济?”香云道:“他少年时也是个风流子弟,极喜偷良家女子,日夜淫乐。减丧太过,到中年就没用了。”未央生道:“他少年时的力量比我如何?”香云道:“做事的伎俩虽然差不多,那有你这两桩好处。”未央生道:“我这件东西与你这件东西皆是世上没有的。如今两件宝贝凑在一处,切不可使它们分开。从今以后,夜夜要过来同你睡了。”

香云道:“你是有家小的人,怎能夜夜过来?只不要象前日的寡情也就够了。”未央生道:“不知是哪个多嘴的人到你面前来学舌,使我抱了不白之冤,到这时候还说我寡情。我若知道那个学舌的人,定要与他狠做一出。”香云道:“我老实对你说,学舌的人不是别个,就是那三位佳人。”未央生道:“这又奇了。这话若是别人说的也该没趣,难道自己不怕腼腆,竟告诉起人家来。”香云道:“不瞒你说,我与三位佳人是姻门之亲。两个年纪小的,我叫他妹子;一个年纪大的,我叫他姑娘。两个妹子更与我心投意合,竟象同胞的一般。我有心事对她们讲,她们有隐情也对我说。我那日烧香回来,见了两个,就把你生的标致,偷看我,我也爱你,丢下扇子的话告诉她们。她们两个道,既然他爱你你爱他,少不得有个寻来的日子,看你怎么样打发他?我心上也料你要寻来,立在门前等了十来日,再不见一毫踪影。后来她两个烧香回来,遇见我就问我道,你那日看见的人是怎么面貌,怎样打扮?我就把你面孔衣服对她们说。她两个道,这等说,你心上的人我今日也见过了。又问我,他既然爱你,那一日可曾对你磕头否?我说,他爱我只好在心上,那有在众人面前磕头之理?她们见我说这话,就不作声,只是笑,像个得意之貌。我疑心起来,再三盘问,她们方才把你磕头的事细细告我。一面笑一面说,果然有个骄人之貌。我一连没趣了几日,心上想道,我与他一般是初见之人,你为甚么见我就避嫌疑,诺也不唱一个?见他就疯颠起来,一些嫌疑不避壁,竟磕起头来?可见我的面貌不如她们,你就要寻也去寻她们,必不来寻我。往常我与她们是极好的姊妹,为这件事竟有些恨她们起来。所以今日于你相会,见说来了半年,直到如今方才理我,焉得不疑?后见你发誓起来,方才知道没有此事。这些戏文都是你磕头磕出来的,请问你该做不该做?”

未央生道:“原来如此也。难怪你不忿恨。但她两个既是你令妹,也只当是我的小姨了。你肯使我见她们一面,别的事不想得,只等我叫她们几声姨娘,使她们知道我们两个有了私情。她们起先把磕头的话来骄你,待我替你把不但磕头又且相与的话去骄她们。心上何如?”香云道:“这也不消,我与两个不但是姊妹,又且同盟。原说有福同享有苦同受,她们以前既不曾背我,我如今怎么又反背她们?我意欲要别你去与她们相会,使她两个也知道天地间有一种妙物,大家赏鉴赏鉴。只是我也要与你断过。你得了她们之后不可改变心肠,要像今夜这等爱我,方才使得。你改变不改变也要发个誓来。”

未央生听了,不觉手舞足蹈,一个筋斗就翻下床来,对了天地,比以前所发的誓更加狠毒。发完之后,又爬上床去,重新干起,当做央媒一般。及干到事完之后,交颈而睡。睡到天明起来,香云打发未央生依旧从梯上过去。

自此以后,两个日日见面,夜夜同床。但不知两位小姨何日到手,今且暂停。下面两回另叙别事,少不得两出戏文之后又是正生上台也。

评曰: 我观肉蒲团之奇,未有奇于此回者。初看香云使性一段,使人张紧眼,莫知所自。及至看到末幅,始知从前一段乃理之当然,非作意也。香云未经相与之先,便吃无影之醋;既同枕席之后,必抬有理之酸。此妇人之常情也。后来不怪不妒,而且以月老自居,使三段奇缘一时毕集。观者虽有急事,亦不暇理,凡看未央生如何得意也。

\chapter{第十三回 破釜焚舟除隐恨 卧薪尝胆复奸仇}

作者:《肉蒲团》李渔

却说权老实自从卖妻之后,愤恨不过,且无颜见人,就把生意不做,歇了。终日闷坐在家,拷问那十二岁丫鬟,说她与那长大汉子是几时睡起,还有甚么人替她往来做事。丫鬟起先怕主母利害,不敢多嘴。如今见主母卖去,料没有回来,就把某时睡起,某时才住,连对门丑妇过来同睡的话尽情说出,又说与他同睡的不是那个大汉,另是一个标致后生,那大汉子反是替他做事的。权老实听了这话,愈加愤恨。后来艳芳归了未央生,有人传说过来,权老实方才得了真情,就去查访未央生的来历。知道不是本处人,家中现有妻子,这是娶去做妾的。

权老实想道,若是赛昆仑自己做事,我这冤仇也不要想报,只好忍过一世,到阴司地府之中与他算帐罢了。如今奸骗之人既不是他,我这冤仇如何不报?若要与他告状,他有赛昆仑帮助,不怕没有银子用,如今官府哪个不听分上的?他若央了人情,我的官司就要输与他了。我想起来告他也无益,不如走到他故乡,访着他的住处,千方百计钻进内室之中,把他结发妻子也拿来淫了几次,方才遂我的心。他淫我妻,我淫他妻,这才叫做冤报冤、仇报仇,就是杀死他也没有这桩事痛快。主意定了,就把那十一岁的丫鬟与一应家伙物件都变卖出银子来,连那一百二十两财礼与平日贩丝的本钱,都收拾了。别了乡邻,破釜焚舟而去。

不一日,到了地头,就在饭店中歇下。次日去访未央生的住居与他家里的动静。访了半日,方才晓得事体难做,心下十分忧虑。起先,只说别人家的闺门与自己的一样,男子在家的时节自然严紧,男子出去之后就像门上少了关,可以借托事端,直进直出了。那里晓得读书的人家比做生意不同,不是三党亲戚及至交朋友即若不许跨进门槛。他那个人家又比别个读书的不同,就是三党的亲戚,至交的朋友,也不许跨进门槛。心上踌躇道,这等看来,那桩心事多应做不来了,只是既然举了此念,无论成与不成,也要尽心竭力去做一做,若万万做不来就是天意了。难道千山万水来到这里,就被“铁扉”二字吓了不成?

主意定了,就要到他前后左右赁间房子住下,早晚之间好看机会行事。谁想他住的所在,是孤孤别别一个宅子,四面都是空地,那里有个房子可以赁得。权老实相了一遍,知道这事难做,只得走回寓处。走不上四五十步,只见他宅子旁边还有一株大树,树上挂了一个木牌,牌上写了八个大字。权老实近前一看,见上面写道“荒园招垦,初种免租。”权老实看了又把大树周围相了一遍,只见野草连天一望无际。心上想道,字上所说的荒园,想就是这空地了。不知是甚么人家的,既有荒园,毕竟也有间房子与人住了才好锄种。我就去租来住在近边,终日以锄地为名好看他家的动静。

就走到附近之处去问人道:“这荒园的业主是哪一个?可有间房子租与种园的人居住么?”那人道:“荒园的业主叫做铁扉道人,就住在那孤别房子里面。只有园没有屋,是要种园之人别寻房子住的。”权老实道:“我要替他开垦,但不知他做人何如?”那人摇头道:“这人是难相遇的,若好相遇的也有人开垦,不倒如今了。”权老实道:“怎见得他难相遇?”那人道:“开荒的旧例,原该免租三年,他只肯免一年,到第二年就要交纳。这也罢了,他平日做人酸啬不过,拼不得饭食养人,一个官家也没有做他的佃户,只当他的长工,家里有生活要做去叫,又没有工钱。三年前头也有人开垦过了,只因被他差使不过,只得丢了不种。所以荒到如今。”权老实听了欢喜不过,肚里思量道,我所虑者,是不能够进门,只要进得门去,就有三分机括了。别人怕差使,我巴不得求他差使;别人要工钱,我巴不得没有工钱,正要使他用我才有妙处。只恐他女婿回来识破机关,就不妙了。我今须要别换一个姓名。他与我不曾见面,就回来也认不出我的。亦不至被他识破了。

算计已定,就改姓为“来”,名字叫做“遂心”。他原为报仇而来,取来到即遂心之意。做小说的仍称他为“权老实”,省得人看花了眼。改名之后,就写了一张租约,走去伺候。知道他家的门是从来敲不开的,只得坐在门外死等。等了一日,不见有人出来。回到寓所宿了。到次日又去。恰好,铁扉道人立在门前买豆腐点心。老实见他相貌端严,就知是本人。走上前深深作揖问道:“铁扉道人莫非就是尊号么?”道人道:“正是。你问我怎的?”权老实道:“闻得府上有一片荒园招人开垦,小人因没有生意,要替府上租来种作。”道人道:“开荒的事,不是无力之人和懒惰之人做得来的,你平日方作如何?”权老实道:“小人平时是吃苦惯的,气力也将就去得。府上若不信得我,权做几时,若还开垦不来,再换佃户就是了。”道人道:“这等,我家没有房子,你在那里居住?”权老实道:“这个不难。小人又没有妻小,不过单身一人,待我自出工本,搭一个草舍起来就可以住得。”道人道:“也好,你去写租契来。”权老实已写在身边,就把租约递过去。道人见他形体粗笨,知道是个健汉,不但园地开得来,连家里的长工也当得过了。就收了租约,随他自备工本来搭草舍。

权老实就去买几根木料,几担稻草,叫一两个泥工木作,不上半日就搭起来。虽是茅屋草舍,也觉得焕然一新。又把种园垦地的家伙办得整齐。每日清晨起来就去锄茅掘土。要使主人看见,觉得他勤谨,好乘青看顾的意思。铁扉道人有一间小阁,恰好对着荒园。行起坐卧都在这阁上。他平日起得极早,谁想权老实又早似他。他不曾下床,权老实已锄过许多地了。道人看见不住的喝彩,自己家里有费力的生活就央他去做。权老实竭力奉承,替他做事不但不要工钱,连饭也不敢吃饱。心上想道,他的女儿不知怎么样奇丑,所以厌恶他,离乡撇井去偷女色,我是睡过好妇人的,万一勾引他上场,看了那奇丑面貌,这根阳物不举,不肯替我报仇奈何?及看见一个绝美的妇人,心上虽然欢喜,还不知是与不是。后来见他丫鬟都叫小姐,方才晓得就是此人。心上又想道,这样妻子也睡得过了,为甚么丢在家中去占别人妻子?从此以后,忍心耐性,只图报仇。见他家里闺门严肃,愈加勤谨,不敢露一毫窥伺之容。在玉香面前走过,头也不敢抬,声也不敢则,竟像个诚实的人。

一连过了几个月,道人见他又勤谨又老实,又不贪嘴,心上爱他不过,因想道,前日女婿临行曾留下几两银子,教我讨一个薪水之仆。我看见别人的官家好吃懒做的多,体心得力的少,所以不敢轻讨。若像这样的人讨他一个也未为不是。我想此人穷无依倚,或者肯卖身为仆也不可知。只是一个汉子讨在家中,有两桩不便:一来怕他没有牵绊,要偷物件逃走;二来男女混杂,那里防闲的许多。我想他若肯卖身,就把一个丫鬟配他,他有妻子系住了身,自然不想逃走,就是出入之间有妻子防闲他,别样的事也就不消虑了。

主意定了,一日走去看他锄地,就问道:“你这等克勤克苦,论理就该做起人家来了。为甚么家小也不讨一房?”权老实道:“自古道‘智养千口,力养一身’,靠力养活的人,糊得口来也就够了。那里能够讨家小?”道人道:“人生一世,妻子儿女都是少不得的。你自家既不能娶亲,何不投靠一个人家有现成女子,配他一个?生得儿女出来,百年之后也有个烧钱化纸的人,多少是好。”权老实听了,知道他有接纳之心,就将计就计答道:“我想投靠人家也是难事,一来怕主人不知甘苦,终日为他做马牛,他不为功劳,又要打骂;二来怕同伴里面不能相容,他不肯替主人出力,见我赤胆忠心,就怕形他短处出来,反要主人面前离间,使我不能够安身。我常见乡宦人家有这情敝,所以不敢去投靠。”道人道:“那乡宦人家仆从甚多,上下之间情意不洽,所以有这情敝。若是不大不小的人家,手下人的好恶主人就看得出。况且同伴甚少,有甚么相容不得?譬如人家像我这模样,一进了门又有妻子配你,你肯去不肯去?”权老实道:“这是极好的,有甚么不肯去。”道人道:“老实对你说,我家少一个使唤的人,今见你勤谨老实,心上要留你,所以问你这些话。你若果然情愿,就写一张身契进来,要几两身价先对我说,待我好设处。进门之日我就把丫鬟配你。你意下何如?”老实道:“若得如此,我明天就送身契进来。只是小人平日欲心极淡,妻子有也得,没有也得,不十分思想。欲把丫鬟配我且从容些,待我做事几年,到精力衰倦的时节把来配我,也不为迟。如今这样年纪,正要为主人出力,何苦把精神气力被妇人消耗了去?至于‘身价’二字一发不消提起,我是自己卖身的,又没有父母兄弟,身价把与哪一个?只要自己有得穿、有得吃就是了。要银子何用?只是文契上不写身价怎么叫做卖身,只好在纸上随意写出多少银子就是。其实一分一文都不要主人破费。”道人听了,不觉欢喜道:“听你这些话,可见你是个忠义之仆。只是两件之中只好辞一件。身价不领,或者留在我身边,待后来做衣服穿。这还使得。若说不要妻子那就成不得了。从来卖身的人只为得一房老小,要图些夫妻之乐,你为甚么不要?身价既不领,妻子又不要,只当是毫无干涉的人,我怎么好取留你?”权老实道:“既然主人怕我心性不常,后来要去,故欲把妻子配我使我没有二心的,但我不是那样恶人,今既不放心,我承受了就是。”

两个说明白了,权老实不等第二日,当晚就写身契过去。道人也不等第二日,当晚就把丫鬟配他。从此以后,道人把草舍拆了,教他在家里宿歇。起先唤他“来遂心”,如今把“来”字削去,单唤“遂心”,配他的丫鬟叫做“如意”。眼见报仇之事有了八分,如意之名又增一遂心之兆矣。

评曰:

妙在粗笨真率之权老实而能委屈求全,迂回钻入铁扉之中,为司马相如之韵事,又妙在铁扉道人于将来之事节节虑到,究竟入权老实计中为卓王孙之后身女心。思路亦可谓奇之极、曲之至矣。

\chapter{第十四回 闭户说欢娱隔墙有耳 禁人观沐浴此地无银}

作者:《肉蒲团》李渔

却说权老实未卖身之前,那玉香小姐有许多幽郁之情,总因笔墨不闲,不曾叙得,如今方才说起他。当初正在得趣之时,被个狠心父亲把丈夫赶出去,竟像好饮的人戒了酒,知味的人断了荤,就是三五夜也熬不过,何况今年隔岁守起活寡来。实在欢娱既不可得,只好把春宫册子摆在面前观看。谁想越看越不禁止那一段淫欲之心。从此以后就把春宫册子放过一边,寻几种闲书出来消愁解闷。

看官,你道她当此之时,要消愁解闷,是甚么书好?据在下看起来别种闲书皆不中用,惟有她少年所读父亲所授的书,如《列女传》、《女孝经》之类,极是对科。若肯拿来一看,岂但消愁解闷,就是活寡也守得来,死寡也守得住。怎乃计不出此,反把丈夫所买之书,取出观玩。那丈夫所买之书都是淫词艺语,如《痴婆子传》、《绣塌野史》、《如意君传》之类,尽数翻出来细看。 只见那书上凡说男子抽送的度数,不是论万就是论千,说男子的阳物,不是赞它极大,就是夸它极长,甚至有头如蜗牛、身如剥兔,挂斗粟而不垂的。心上想道,我不信男子身上有这样雄壮的东西,我家男子的物事长不过二寸,大不过两指,干事的时节,极多不过一二百提,就要泄了。何曾有上千?自古道:“尽信书则不如无书”。这些百经的话一定是做书之人造出来的,那有这等异事。疑了一会,又想道,天下甚大,男子甚多,里面奇奇怪怪,何所不为,焉知书上的话不是实事?倘若做妇人的嫁得这样一个男子,那房帏之乐自然不可以言语形容,就是天上的神仙也不愿去做了。又把这话疑了又信。

连日爬起来,女工针指一些不做,只把这几种闲书做对头,要使心上的淫兴索性浓到极处,好等丈夫回来一齐发泄。谁想等到后面,一毫音信也没有,不由她不怨恨起来。心上想道,我前世不修,嫁着这样狠心男子,成性不上数月,一去倒丢了几年。料他那样好色的人,再没有熬到如今不走邪路之理。他既走得邪路,我也开得后门,就与别个男子相处也不为过。只可惜闺门严紧,没有男子见面。想到那个地步就把怨恨丈夫的心迁怒到父亲身上,巴不得他早些死了,好等男子进来。 及至看见权老实就像饿鹰见鸡,不论精粗美恶,只要吞得进口就是食了。起先做工的时节,虽有此心,一来见他老实,相见之际头也不抬,不好突然俯就他;二来日间进来,夜间出去,就要俯就他亦无其时。后来,听见他要卖身,心中甚喜,要想进门的头一夜就不肯放过他。不料父亲把如意配他,见他两个拜堂之后,双双进房,心上就吃起醋。伺候父亲睡了,就悄悄走去听他干事。权老实的阳物甚大,如意虽有二十多岁,只因主人至诚,不曾偷摸过他,所以还是个处子,那里能经得绝大东西。叫喊之声,啼哭之状,自然惊天动地。连窃听之人都要替她疼痛起来。权老实见她承受不起,只好草草完事。

玉香立了一会,听不出好处,也自进房睡了。到第二三夜,又去补听,也还只见其苦,不见其乐,直听到三夜之后,也自权老实的本事该当出现以前。几夜都是吹灭了灯,然后睡的,独有这一晚,灯也不吹,帐子也不放,未曾动手之先,把一根八寸多长、一手把握不来的阳物,教如意捏在手中,摩弄了一会,方才插入阴户。此时的阴户已被阳物喧大了,不像以前紧涩。权老实就放出本领来,抽送的度数竟与书上一般,不到数千不肯住手。如意从奇苦之后,忽逢奇乐,那些呼唤之声,又不觉惊天动地。以前替她疼痛之人,如今又替她快活起来。看出来的淫水,比弄出来的淫水更多。

从此以后,玉香的心已注在权老实身上。权老实自进门之后,也不老实。遇见玉香,不住把眼睛偷觑玉香,若有笑面,也把笑面相承。一日,玉香在房里洗浴,他从门外走过,无心中咳嗽一声。玉香知道是他,要引他看看肌肤,好动淫兴。故意说道:“我在这边洗澡,外面是哪一个?不要进来。”权老实知道这话是此处无银之意,就不敢拂他的盛情,把纸窗湿破一块,靠在面上张看。玉香看见窗外有人,知道是了。就把两个肉峰,一张牝户,正正的对着窗子,好等他细看。还怕要紧的去处浸在水里,看不分明,又把身子睡倒,两脚扒开,现出个正面,使他一览无遗。

睡了一会,就坐起身来,两手捧住牝户,自己看了,长叹一声,做个技痒难搔,无可奈何的意思。权老实看了,知道这妇人淫也淫到极处,熬也熬到苦处,我若进去,决不拒客了。直把房门一推,直闯进去,跪在玉香面前道:“奴辈该死。”就爬起身来把她搂住。玉香故意吃惊道:“你为何这般胆大?”权老实道:“小人卖身之意,原是要进来亲近小姐。起先还要在没有人去处诉出衷情,待小姐许了,才敢放肆。不想今日看见千金之体生得娇嫩,熬不住了,只得进来冒渎,求小姐救命。”玉香道:“据你的意思,要怎么样?难道浴盆里面好干甚么事体不成?”权老实道:“小人也知道,这个所在与这个时候,不是干得事的。只求小姐恩允过了,待我夜间来服事就是。”玉香道:“你夜间与如意同睡,她怎肯放你来?”权老实道:“她是极贪睡的,夜间干事之后,直睡到天明方醒。我今夜瞒了她来,她那里知道。”玉香道:“这等,依你就是。”权老实见她允了,就把浑身上下摸过一遍,又亲了两个嘴,约今夜开门等我,方才出去。

此时天色已晚,玉香揩干了身子,衣服也不穿,夜饭也不吃,就爬上床去,要先睡一觉,养养精神好同他干事。谁想再睡不着,捱到二更,初听见房门响,知道是他进来,就低低叫道:“遂心哥,你来么?”权老实也低低应道:“小姐,我来了。”玉香怕他在黑暗之中摸不上床,忙爬下来接引,就牵他上床,说道:“心肝 ,你的东西,我看见过了,比别人的不同,我承受不起,求你从容些。”权老实道:“千金之体,我怎敢唐突。”

口虽说这话,心内还疑她假意装娇,岂有偷妇人的男子没有绝大本钱,使自家妻子还怕疼痛之理。就把阳物对着牝户唐突起来。玉香忍不过,就恼起来道:“我吩咐你从容些,你怎么又这等急遽?”权老实见抵不进去,知道起先的话不是虚情。就陪个小心道:“不瞒小姐说,我不曾见过标致妇人。今遇小姐,心上爱你不过,巴不得早进一刻也是好的,所以用力太重,得罪了小姐。如今待我将功折罪就是了。”遂把阳物提起,在她阴户两旁东挨西擦,不敢入室,竟在腿缝之中弄送起来。你道他是甚么意思?原来是个“疏石引泉”之法。天下最滑之物,莫过于淫水,是天生地设,要使它滋阴润户的东西。唾沫虽好,那里赶得它上?凡用唾沫者皆是男子性急,等不得淫水出来,所以把口中之物纳入阴中,用那假借之法。究竟别洞之水,不若本源之水滑溜,容易入口。权老实起先也不知有此法,只因初娶艳芳之时,阳大阴小,不能入。亏得艳芳搜索枯肠,想出这种法来,把极难之事弄得极易。如今玉香的阴户,与艳芳昔日的阴户宽窄相同。权老实忽然记起这旧事,所以仍用此法,把阳物放在腿缝之中,替阴户摩肩擦背,使她里面痒不过,自然有淫水出来。淫水一来,如浅滩上的重船得了春涨,一到,自然一息千里,连篙橹之功都可以不费了。

玉香见他把腿缝认做阴户,就笑道:“你走错了路,我们往常不是这样干。”权老实道:“一毫也不错。我还你快活就是。”弄了一会,只见腿缝里面有些滑溜起来,知道淫水已至。又怕太滑,抵不着阴门,要溜到别处去,就拿住玉香的手,把阳物交与她道:“起先果然弄错了,如今摸不着真穴,求你自家点一点。”玉香就叠起阴户,把阳物凑在阴户口,吩咐道:“如今是了,你自己用力插进。”权老实挺起阳物,一直插进去。每抽一次,送进一二分。再抽二十馀抽,那根八寸多长的阳物,不知不觉已尽根进去了。玉香见他干法在行,愈加爱惜。就紧紧搂住道:“心肝,你是初近女色的人,怎么就这等知情识趣。我今爱杀你了。”权老实任事之初,得了这篇奖语,自然不肯偷安。把抽送之法,不猛不宽,不缓不急的做去。做到后面,竟使他一辞莫赞,连奖语都做不出来,方才住手。

玉香不曾尝这样滋味,十分欢喜。自此以后,夜夜少他不得,起先,还是背着如意做事,后来晓得瞒不到底,索性对她说过,明明白白的往来。玉香怕如意吃醋,尽心奉承她,名为主婢,实同大小。或是一人一夜,或是一人半夜,甚至有高兴之时,三人同睡。

在权老实的初意,原为报仇而来,指望弄上了手,睡几个月,即便抽身,不可被妇人恋住。谁想冤孽之事难以开交,当初与艳芳睡了几年,不见生子,如今与玉香一干,就成了孕。起先还不觉,及至三月后害起喜来,方才知道。千方百计寻药来打胎,再打不下。玉香对权老实哭道:“我这条性命送在你身上了,你晓得我父亲严法,一句话讲错,尚且要打骂,肯容做这恶事?明日知道,我少不得是一死。不如预先死了,还省得淘气。”说罢就要上吊起来。权老实再三苦劝。

玉香道:“你若要我不死,除非领我逃走,逃到他乡外国。一来免了后患,二来好做长远夫妻,三来肚里生出来是男是女,总是你的骨血,也省得淹死了他。你心下何如?权老实见她说得有理,就要瞒着如意做事;又恐怕她预先知觉,要说出来,只得与她商量定了,把随身衣服捆好,等铁扉道人睡了,开了大门一齐逃走。但不知她走到何方,后来怎生结果,看到十八回才知下落。

评曰:

有人看到此回,疑铁扉道人是个善士,不该有淫奔之女,天公既欲惩奸,独不欲劝善乎?余曰:不然。此等报应,正是天公不谬处。铁扉道人生平不交一友,不见一人,不免蹊刻太甚,且开荒之例,原该免租三年,他只免一年,不时呼佃户服役而不给工钱之类,皆残忍刻薄之事,安得使后来无报?所以从来狐介之士厥后反不昌者即此理也。为君子者可不慎乎?

\chapter{第十五回 同盟义议通宵乐 姊妹平分一夜欢}

作者:《肉蒲团》李渔

权老实报仇的因果按下慢表,如今且把未央生得意之事畅说一番。自这一夜搂住香云细谈往事,知道那三个美妇都是她一家,两个少年的又分外心投意合。只因话长夜短,两个又要干事,竟不曾问那三个妇人是何姓名,三个丈夫是何别号,家住在哪里。直到第二夜过去,方才补问。

香云道:“我叫她姑娘的,是花朝日生的,名字叫做‘花晨’,我们叫她晨姑。丈夫死过十年了,她心上要嫁,只因生下个遗腹子,累住了身子,不好嫁得,所以守寡。我叫她妹子那两个,是她嫡亲侄妇,大的叫做‘瑞珠’,小的叫做‘瑞玉’。瑞珠的丈夫,号‘卧云生’;瑞玉的丈夫,号‘倚云生’,两个是胞兄弟。她三个人家门户虽然个别,里面其实相通。只有我远一步,隔得几家门面。总来都在这条巷内。”

未央生听了,愈加欢喜。又记起赛昆仑前日之言说两个富贵女子,就是此人。可见贼眼与色眼一样,同是一丝不漏的。就问香云道:“昨日蒙你盛情,把两位令妹许我,但不知何时才许我相会?”香云道:“再过三五日,我就要过去,可以引你去相会。只是一件,我一去之后,就不回来,这张床不是我们作乐之处了。”未央生吃一惊道:“这是甚么缘故?你可明白说来。”香云道:“因我家丈夫在她家处馆,那兄弟两个是我丈夫的学生,文理都不齐,怕做秀才要岁考,两个一齐缘了例,目下要进京坐监,她两个是不得离先生,少不得我家丈夫要同他进去。他怕我没人照管,要接到他家,等我姊妹三个一同居住。这数日之内就要起身,所以我一去之后就不回来,只好约你到那边相会了。”未央生听了,一发喜上加喜,想三个男子一齐开去,三个女子一齐撮合,可以肆意宣淫了。果然数日之后,师徒三个一齐起身。起身之日就把香云接去。香云与未央生两个正相到好处,哪里离得长久?少不得一见之后就要透露出来,好商量定了,领他来干事。

次日,香云对瑞珠、瑞玉问道:“你两个可曾再到庙里去烧香么?”瑞玉先答道:“烧过一次就罢了,难道只管去烧?”香云道:“有那样标致男人磕你的头,就三五日去烧一次也不为过。”瑞珠道:“香倒要去烧,只是没有扇子送他。”香云道:“贤妹不要笑我,我的扇子固然折本就是。你们两个虽受他磕头,也不曾见他跟你们回来,哄你害害相思罢了。”瑞玉道:“我们两个说起这件事,也解说不出为甚么。那个男人这等虎头蛇尾,若照那样颠狂起来,就像等不得第二日,当晚就要跟来的一般。及至等到后面,一些踪影也没有。既然这等寡情,何不省了那几个头不磕也罢。”香云道:“我闻得人说,他终日在那边思想,只是寻你们不着。无可奈何了。”瑞珠道:“我们两个他未必思想,只怕对了那把扇子睹物思人,要害起相思病来。”香云道:“扇子的相思他倒果然害过,不是假话。如今倒勾过帐了。只是磕头的相思,害得沉重,一时医他不好。将来害死,只怕要你来偿命。”瑞珠、瑞玉见他这话可疑,就一齐到他脸上看他颜色何如。香云一面说一面笑,也做出一种骄人的光景。两个一齐道:“看你这样得意,莫非上了手么?”香云道:“也差不多,偏背你们与他勾账过了。”两个听见这话,就像科场后不中的举子,遇着新贵人一般,又惭愧又羡慕,变赔个笑脸道:“这等,恭喜!添了个得意的新姐夫我不曾贺你,如今新姐夫在哪里?可肯借我们看看么?”香云故意作难道:“你们都是见面过了,何须再要见?”瑞玉道:“当初是道路之人,他便磕头,我不好回礼。如今是至亲了,何妨再会,待我们回他个礼,叫声姐夫,替他亲热也是好的。”香云道:“要见有何难,我就去叫他来。只怕他一见了面,要象前日磕头的光景,疯颠起来,得罪了二位贤妹,不成体统。”瑞玉道:“他起先少人拘管,所以轻举妄动,如今有你这个吃醋的人立在面前,他怎么敢放肆。”瑞珠对了瑞玉道:“你这些话都是枉说的,他心上的人怎么舍得把与别人见面?当初结盟的话,虽说有祸同受,有福同享,如今那里依得许多。只是求他不要追吃以前的醋,把磕头的话置之不问,也就好了,怎么还想别样的事。”

香云听了,知道她发急了,就认真道:“你不要发急。我若是要独自受用,不与你们同乐,只消住在家中不肯过来,日夜同他快活就是了。何须带自己的醋到别人家吃起来?我今肯对你们说,可见不是恶意了。如今要从公酌议,定一个规矩,使见面之后,大家没有争兢,我就叫他进来,同你们相会。”瑞珠道:“若肯如此,也不枉结拜一场。就求你立个规矩,我们遵依就是了。”香云道:“我与他相处在你们之先,论起理来,就该有个妻妾之分,大小之别。凡是要占便宜,得我与你是相好的姊妹。不好这等论得,只是序齿罢了。凡日间、夜间取乐,总要自大而小,从长而幼,不许越位。就是言语之间,也要留些余地。不可以少年之所长,形老成之所短,使他有后来居上之评;不可以新交之太密,使旧好之渐疏,使我有前鱼见弃之恨。若依得这些话,自然情投意合,你们肯依不肯依?”瑞珠、瑞玉齐答道:“这议论甚是公道,只怕你不肯。我们有甚么不依?”香云道:“这等,待我写字唤他来。”就取出一幅花笺,写出两句诗道:
天台诸女伴,相约待刘郎。

写了这两句,就把签折做几折,放进笔筒里。瑞玉道:“为甚么只写两句?这诗叫做甚么体?”瑞珠道:“我晓得云姐的主意,是舍不得他搜索枯肠,留后两句待他续来,省得再写回贴的意思。你也忒熬爱他了。”香云笑一笑,把诗封好,交与丫鬟,吩咐拿到自己房里从板壁缝中丢过去,讨了回字转来。

丫鬟去后,瑞珠问道:“你是怎么法引他到家里来?如今过几夜了?”香云就把他住在隔壁,如何相会,共睡几夜,细说一遍。瑞玉道:“他的本事何如?”香云道:“若说起本事,竟要使人爱杀。你们两个只知道他的面貌标致,那里晓得他的本钱是一件至宝。从来妇人不但不曾看见过,连闻也不曾闻过。”瑞珠、瑞玉听了,一发要问,就像未考的童生,遇着考过的朋友,扯住问题目一般,是大是小,是长是短,出经不出经,给烛不给烛,件件要问道。彼时正在吃饭之后,碗碟未收,香云见他问多少长,就拈一根箸,道:“有如此箸。”见问他多少大,就拿一个茶盅,道:“有如此盅。”见他问坚硬何如,就指一碗豆腐,道:“有如此腐。”瑞珠、瑞玉笑道:“这等,是极软的了。既然如此,就要他长大何用?”香云道:“不然。天下极硬之物,莫过于豆腐。更比钢铁不同,钢铁虽然坚硬,一见火就软了。只有豆腐,放在热处越烘越硬,他的东西也是如此,是弄不软的。我所以把豆腐比他。”瑞珠、瑞玉道:“我不信有这件好宝。”香云道:“我说这话还不曾尽其所长,他另有两种妙处,我若说出,你一发不信。只好到干事时,你自己去验罢了。”瑞珠、瑞玉道:“你说就是,管我们信不信。”香云又把先小后大,先冷后热,次第形容出来,两人听了他,不觉欲火上升,耳红面赤,即刻要他来与他干事,好试他绝技。

谁想丫鬟去了半日,再不见来。原来未央生不在家。他坐在房里等候,被书笥看见,也从板壁上爬过来,两个大弄半日。直待未央生回来,把书笥丢过去,方才讨得回字转来。三人拆开一看,见他果然会心,就在原诗后面续两句道:
早修胡麻饭,相逢节馁肠。

瑞珠、瑞玉看了知道今夜是万无一失了,不胜欢喜。香云道:“今夜干事的次序,须议一个妥当,省得临事之时,个个要想争先。”瑞珠心上晓得她睡过几夜,该当让人,没有今夜就要序齿之理。心上虽然如此,口里故意谦逊道:“你方才做定规矩,自长而幼,自大而小,不消说是你起头。”香云道:“论理原该如此,只是今夜又当别论。自古道‘先入为主,后入为宾’,我同他睡了几夜,就算是主人,今夜且定宾主之礼,等你两人各睡一次,然后再序长幼。你们不要虚谦,今夜自然是珠妹起了,只是你两人还是每人一夜,睡个完全的好;还是每人半夜,睡个均匀的好?你们商议定了,回我的话就是。”瑞珠、瑞玉想了一会齐说道:“我们两人不好说得,凭家长吩咐就是。”香云道:“每人一夜觉得像意,只是难为侯缺的,还是每人半夜罢。你两人意中如何?”谁想他两人各有隐情,不好说出,只是闭口不言。香云道:“你们不说的意思我知道了,前面的一个怕他不肯尽欢,要留量去赴第二席,所以不应;后面的一个怕他是强弩之末,干事的时节没有锋芒,所以不应。我老实对你说,他的本事是一个当得几个的。”对着瑞珠道:“你就同他睡一夜,只好做半夜实事,只怕还不到半夜,就要求免,落得交下手去。”又对瑞玉道:“酒醉后来人,况且他那壶酒又分明是下半壶好吃。你两个不必狐自。”

瑞珠、瑞玉的隐情被她参破,又决下疑心,一齐应道:“依命就是。”香云遂吩咐丫鬟立在门前去等。不多一会,就把未央生领进来。瑞珠、瑞玉见他来到,假装羞怯退后一步,让香云接他。未央生对香云深深一揖,道:“请两位小妹过来相见。”香云每一只手扯住一个,同他相见。见后,瑞珠唤丫鬟拿茶,香云道:“不消唤茶,他为你两个也想得苦了,各人把口里琼果送些过去,当了茶罢。”就把两个的手交与未央生。未央生接到了手,就双双搂住,把自己的舌头先伸在瑞珠口里,等她尝了一会;又伸在瑞玉的口里,也等她尝了一会。然后把三张口合在一处,凑成一个“品”字,又把两根舌一齐含在口里,尝了一会,方才放手。

只见丫头排上夜饭,未央生上坐,香云下坐,瑞珠居左,瑞玉居右。四个吃了晚饭,将要收碗,未央生扯香云到背后去问道:“请问娘子,今夜是怎么样睡法?”香云道:“我预先替你酌定了,上半夜是瑞珠,下半夜是瑞玉。”未央生道:“这等,娘子呢?”香云道:“今夜我且恬退一夜,让她两个受用。待明夜然后轮起,照序齿一人睡一夜。但你今夜要争气些,应得我的口来就是了。”未央生道:“那个不消吩咐,只是忒难为你。”香云就叫丫鬟拿灯送未央生与瑞珠进去。自己怕瑞玉难过,陪她说了一会闲话,方才就寝。

瑞珠与未央生进房之后,就宽衣解带,上床行乐。初干之际,颇觉艰难,瑞珠想起日间的话说得好听,知有将来之乐,足偿此际之苦,所以坚忍,咬住牙关,任他冲突。时时刻刻盼他大起来,时时刻刻望他热起来。只见抽到后面,果然越弄越大,越干越热,竟像是个极大的角先生,灌了一肚滚水,塞进去一般。就是不抽不动,留在里面也是快活。方才知道日间所言不是虚誉,“至宝”二字竟可做此物的别名。就把未央生紧紧搂住道:“我的心肝,你有这样标致面孔,又有这件至宝生在身上,难道要把普天下的妇人都想死了不成?”未央生道:“弄得人死,才想得人死。心肝,你舍得一条性命,等我弄死了么?”瑞珠道:“遇着这件东西,难道还要想活不成?只是让我多干了几次,死才死得甘心。不要头一次就送我性命。”未央生就翻天倒地干起来。瑞珠的阴户虽深,花心生得极浅,只消进一二寸就挠着痒处,所以抽送之间再没得落空。抽到半千之后,就要死要活起来,口里不住的叫道:“心肝,我今要死了。求你饶了罢。”未央生要现所长,听见这话,只当听不见,力也不较,从一更干起,直干到二更,只见她四肢瘫软,口内冷气直冲,未央生知道不是劲敌,就住了手。紧紧搂住睡了一会,瑞珠醒转来道:“心肝,你怎么这么会干?如今我妹子在房里等,你过去罢。”未央生道:“黑暗暗的,我那里摸得过去?”瑞珠道:“待我叫丫鬟送你去。”就叫一个丫鬟起来,搀了未央生的手,送他过去。

那个丫鬟是个十五六岁的处子,起先听见他干事,弄得山摇地动,阴中骚痒不过,淫水不知流了多少。如今搀着未央生的手,那里放得他过。走到僻静去处,就对未央生道:“你怎么这等狠心,刚才那样好滋味,何不使我尝一尝?”就把一手搂住未央生,一手去脱自家的裤子。未央生见她情急不过,不好推辞,就叫她睡在懒榻之下,将她阴户扒开,然后取出阳物,唾沫也不搽,对了阴户直抵。那丫鬟不曾经人弄过,暗想那件东西是好吃的汤水,所以扯他弄,还愁他不肯弄。不料,他把阳物一抵,疼痛难当,就喊叫起来。未央生见她是个处子,就搽上许多唾沫,紧紧朝里又抵。她又叫喊起来道:“做不得!若再照样,一些好处也没有。为甚么我主母弄了就快活,这是何故?”未央生就把初次干起要皮破血流,直要干过十余次方才会快活,又安慰她道:“我的本钱忒大,你当不起。我有个小厮,叫做‘书笥’,他的本钱还小。明日带他来先与你干几次,然后等我干就不妨了。”

丫鬟感激不尽,就爬起来,穿好裤子,引他行走。走到瑞玉门前,只见明烛辉煌,点在房里伺候。听见外面走响,丫鬟就开房门,接他进去。未央生走到床前,叫道:“心肝,我来迟了。你不要见怪。”遂把衣服脱下,揭开被窝,爬在瑞玉肚上,挺起阳物就干。初干之时,痛楚起来,与瑞珠一般,干到好处,那种要死要活的模样,更比瑞珠不同,使人看了竟要可怜起来。这是甚么原故?因他的年纪比瑞珠小三四岁,身体也在瘦弱一边,肌肤娇嫩,竟无一物可比。就是立在阶前,尚怕随风吹倒;坐在椅上,还要东扶西靠的人,那里能经得这样干事?所以抽到数百之后,星眼微撑,朱唇半启,心上有话,口里说不出来,无非是弱体难胜,香魂欲断,若再抽一会,定有性命之忧。未央生看了,心上怜惜不过,就问道:“心肝,你经不得再弄了么?”瑞玉答应不出,只把头点一点。未央生就爬下身来,等她苏息一会,要干,又经不得再干;不干,又爱她不过,只得把她抱在肚子上面,睡到天明。

香云与瑞珠清早起来,要商量长久之策,就到瑞玉床前催未央生早起。揭开帐子一看,只见瑞玉倒在上面,未央生倒在下面,就叫醒来笑道:“今夜点灯不消买蜡烛了!”姊妹三个笑了一会,就与未央生商议道:“如今晚去夜来,终究被人看见。就是你自己家里的人见你夜夜不回,也要根究出来。怎么设法在这里住几时,连日里也不回去,不必定要干事,就是下棋、做诗,说说笑笑,也是快活的。你有这个妙法么?”未央生道:“我未来之先,就把绝妙之法算得妥当了。”三人问道:“甚么妙法?”未央生道:“我的小妾现今怀孕在身,干不得事。我昨日对她说,我离家日久,不曾回去,今趁你怀孕之时,到故乡去看看。往返只消散个月,就好转来看你分娩。省得分娩之后,又要回去,妨我们作乐的工夫。她说我这话极讲得是。我今日回去,就收拾行李出门,只说回故乡去,竟挑到你家来。这三个月之中,莫说做诗、下棋、说笑话,就是要串戏,也串得几本了。”

三个女子听了,不胜欢喜,皆言妙计。未央生道:“还有一事,要与三位商议。我身边有两个伴当,一个丢在家里,一个带他出来。只是那小介也有主人之风,若不把些甜头到他,他若走回去露出事来,却怎么处?”瑞珠道:“这个不难,我家有得是丫鬟,随他去作乐就是。不但可系伴当之身,还可塞梅香之口,省得我们男子回来要去学舌。”未央生道:“说的有理。”四人计议定了,就打发未央生回去。当晚就挑行李过来。自此后,不但未央生醉卧群芳,连随身伴当亦享温柔之福。只可惜故园春色一旦飘零,使人有不堪回首之叹耳。

评曰:

香云不吃同盟之醋,而背以钟爱之人,公之同好。虽所为出于不正,而交情亦自可取。求之男子中正不可得。今之同盟兄弟,所共图之事,亦未必尽出于正,而嫉妒之心更有甚于不同盟者。此等男子,幸不生为妇人,若为妇人,必极尽天下之淫行而后止。

\chapter{第十六回 真好事半路遭魔 活春宫连箱被劫}

作者:《肉蒲团》李渔

诗云:芳心忍负春晴日,小阁添丝绣碧罗。绣到鸳鸯针忽折,画中好事也多魔。

香云与瑞珠、瑞玉,把未央生藏在家中,依了定例,一人睡一夜。周而复始,轮了几次,未央生与旧例之外,增个新例出来,叫做“三分一统”,分睡了三夜,定要合睡一夜;合睡了一夜,又依旧轮睡三夜。使她姊妹三人,有共体连形之乐。自添新例之后,就设一张宽榻,做一个五尺的高长枕,缝一条八幅的大被。每到合睡之夜,教她姊妹三人并头而卧,自己的身子再不着席,只在三人身上滚来滚去。滚到那一个身上,兴高起来,就在那一个干起。喜得三个妇人的色量都还不高,多者不过一二百抽,少者还不上百余抽,就要丢了。中间的丢过一次,就要轮着左边的;左边的丢过一次,就好轮着右边的。只消一二更天完了正事,其余多的工夫,就好摩弄温柔,咀尝香味了。

一日,香云与瑞珠、瑞玉在背后商量道:“我们三个把这等一个神仙,一件宝贝,放在身边受用,可谓侥幸之极。只是一件,从来的好事多魔,须要在得意之时,预防失意之事,不可被外人知觉,唇播开来,使他立脚不住,就不妥了。”瑞珠道:“我家屋宇深沉,没有闲杂人进来。房中的事,外面那里晓得。就是自己的官家,也只许在二门外伺候,不容他进来就是。所怕者是一个妇人,万一被她知道,我们的好事就做不成了。”香云道:“是那一个?”瑞珠道:“就是晨姑。你晓得,她性子是好淫不过的,虽然守寡,哪一时一刻不想男人?况且那日去烧香,她看见磕头也疯颠起来,就像要跪下去,与他回拜的一般。只是不好做出。及至回来,又极口赞他标致,还说可惜不认得他。若晓得他姓名住处,定然放他不过。你说那爱慕的人,若晓得被我们藏在家中作乐,岂有不怀忌妒,暗算我们之理?一经她暗算,我们就有不测之祸,岂但好事做不得?”香云道:“说的有理,果然她是个好淫的人,这事不可不虑。”瑞珠道:“我起先怕丫鬟泄漏,如今有书笥塞了口,料想不肯传说出去。只怕她亲来看见。她往常过来的时节,不响不动,就钻进房来。那双眼睛,就像偷油的老鼠,东张西望,就像有人瞒她做事一般。如今倒要防备,第一着,实吩咐那些个丫鬟,叫她们在两边交界处轮班看着,一见她过来,就要做个暗号,或咳嗽或叫唤,我们就好藏人;第二着,要算一个藏人之处,使她撞不着、寻不出就是了。”

瑞玉道:“藏在那一处好?”三个人交相酌议,有说躲在门背后的,有说伏在床底下的,瑞珠道:“这都不是算计。她那双贼眼,好不厉害,岂有门背后及床底下藏人不被她搜出之理。”想了一会,忽然看见一只篾箱,是收藏古画的,有六尺长、二尺阔、三尺深,外面是一层竹丝,里面是一层薄板。瑞珠看了,指着道:“此物甚妙,又不大不小,将里面古画搬出,可以睡得一人。到要紧时节,把人藏在里面,她那里知道。所虑者是气闷不过,只要把里面薄板掀去两块,就不妨了。”香云与瑞玉道:“果然绝妙。”主意定了,就吩咐丫鬟叫她轮班打听,又把篾箱里面掀去两块薄板,吩咐未央生,叫他见有妇人来就睡在里面去,不可响动。自从设计之后,果然有几次过来,被丫鬟做了暗号,未央生忙躲进去,一毫也看不出。

偶然一日,那三个姊妹合该有事。在未央生匣内拾着一本册子,揭开一看,见有许多妇人的名字,美貌分等第,后列批评,都是未央生的亲笔。就问道:“这册子是几时造的?要它何用?”未央生道:“就是我寓在庙中之时,一边看见,一边登记的。要待造完之后,选几个玉笋门生出来,好做公门性交,不时去浇灌她、培植她的意思。”三个问道:“那玉笋门生如今有了不成?”未央生道:“就是三位。”三个笑道:“不信我们就当得这样品题。”未央生道:“不必多疑。”就把三个人的等第批评查出来,指与她们看。三人细细看了一遍,大家一齐得意起来。只有香云,见他的批语比两人略减些,欢喜之中,不十分满足。还亏得他未雨绸缪,怕香云看见,预先在两圈之上,加了一圈,把一等提做特等,所以香云看了,见他虽有详略之分,实无高下之别,故不以为意。

及看到后面,又有“玄色女子”一名,批评的话竟与瑞珠、瑞玉不相上下。三人见了,不觉惊骇,一齐问道:“这一位佳人,这等标致,是甚么人家的?”未央生道:“就是那一日同二位进来的,怎么就忘了?”瑞珠、瑞玉听了,不觉大笑道:“这等说,就是那个老东西了。她是何等年纪,何等面貌,竟与我们三人一齐考起特等来?有这样无赛的事。”香云道:“这等说,我们考法都不足为荣,反足为辱了,这样的批评要它做甚么,不如涂抹了罢。”未央生要暴白原情,把一人有福,带系满屋的话,说与她们听。奈何三个门生一齐鼓噪起来,竟不容主司开口。瑞珠、瑞玉道:“云姐的话极讲的是,我们一概除名,让那老门生独占鳌头罢了。”瑞珠就提起笔来,把三个人的名字、批评一齐抹去,后面批一笔道:
淮阴齿幼,绛灌年尊,不敢雁行,谨当逊位。

批过之后,就对未央生道:“这一位玉笋门生还喜得不远,那旁门里面是走得过的,请去浇灌她,我们三个不劳你培植了。”

未央生见他动了公愤,不好措办,只得低头下气,随她们驱逐,只是不理。直待她们气平之后,方才说出原情,是推你们的屋鸟之爱,要寻她做个介绍,好与列位相处,所以奉承她几句,其实不是公道批评,列位不要过责。三人听了,方才释了公愤。未央生就于释愤之后,卖笑求欢。自己先脱去衣服,睡在床上,等三人次第宽衣。正要爬在床上去,不想守门丫鬟咳嗽一声。三人知是暗号,就流水穿起衣服来,留香云在里面藏人,瑞珠、瑞玉连忙出去招接。未央生的衣服脱得最早,堆在女衣下面,寻不出来。及至众人穿完,捡出来时又穿不及,只得精赤条条爬进箱去。

且说花晨走到中堂,见了瑞珠、瑞玉,看她两个面容大有惊慌之色,心内疑惑起来,知道这三个人必有良之事了。就要闯尽卧房,察她动静。谁想她已把活跳的春宫,锁在箱子里去了。花晨走到房中,故意喝彩她道:“好几日不来,一发摆列的整齐了。”就到床前床后走了一次。连橱柜里面都去搜检一番,并不见一毫形迹。只说是自己生疑,其实没有相干。遂坐下与三人共说闲话。不料,这事到底做不完全,弄来弄去,依旧露出马脚来。起先,她三人听见咳嗽,大家慌了,只有工夫穿衣服,开书箱,急把窝藏的人塞得进去,就完得一桩事。不虑那一本册子丢案头,不曾收拾。直到说话之际,方才看见。正要去取,谁知花晨眼快,一把就捏在手中。三个人慌了手脚,一齐去夺,那里夺得过来。

香云知道不能夺来,就先放手,故意对瑞珠、瑞玉道:“不过是路上拾得一本残书,送与晨姑拿去罢了。抢它做甚么。”两人一齐放手,花晨道:“既蒙云姐见赐,待我揭开张它一张,看是甚么书。”就把身子立开,与她三人隔了一丈多路,揭开一看,看见“广收春色”四个字,只说是本春意图。急急翻到后面,先看人物,后看标题,才晓得其中意味。谁想翻来复去不见一幅春宫,都是批评的语,方才晓得是个多情才子品评佳人的册籍,比春意还好看。就把一概批评细细看去,看到一个名为“玄色佳人”,后面批语竟像为她写照的一般,就不禁动起心来。暗想这册子莫非就是庙中相遇的人做出来的不成?就翻转到前面去看题头,只见有“某时某日遇国色三人”的话,写在名字之前。再把“银红”、“藕色”的字眼想了一会,就知道是她无疑了。及至看到“淮阴齿幼,绛灌年尊”的一行批语,认得是瑞珠的笔迹,就放下脸来,把册子藏入袖中,故意叹道:“当初造字的苍颉,真是圣人。”

香云道:“怎见得?”花晨道:“他造的字,再没有一个字没解说的。譬如奸淫的‘奸’字,是三个“女”字合起来,即如你们三个女子住在一处,做出奸淫的事来一般。难道还不晓得苍颉造字的妙处?”瑞珠、瑞玉道:“我们住在一处,并不曾做出甚么事来。这话从那里说起?”花晨道:“你们既不曾做,这册子是哪里来的?”香云道:“是我过来的时节,在路上拾得的。”花晨道:“你不要骗我。我如今只问造册的人现在哪里?好好抬出来,万事干休。若还不说,我就写一封书,把这册子封在里面,寄与你们的丈夫,叫他们回来同你们说话就是了。”

三人见她词色不佳,不好与她相抗,只是推说这册真是拾来的,那里晓得造册的人姓张姓李,住在何方。花晨一面盘问,一面东看西看,心上想道,别处都相过了,只有这只画箱不曾检验。往常是开着的,为甚么忽然锁了?其中必有原故,就说道:“这事你们既不肯抬,只得暂时免究,待改日再审。只是你这箱子里有几轴古画,可开出来待我看看。”瑞珠道:“钥匙不知放在哪里,这几日尚寻不着,待寻着时开出画来送与姑娘看。”花晨道:“这等,不难。我家钥匙甚多,可以开得的。”吩咐丫鬟去取。不上一刻,取了几百把来。花晨接到手,就去开箱。她三人就像死人一般,又不好嗔,又不好拦阻,只得凭她去开。心上还妄想她钥匙凑不着,开不来。 谁想她不用第二把,头一把就开着了。揭起盖子一看,只见一个雪白男子睡在里面,腿上横着一根肉棒槌,软到极处,尚且令观者吃惊。不知他坚硬起来更作何状。花晨见了如此奇货可观,岂有不居之理,就不忍惊动他,依旧放下箱盖,把原锁锁了,对着三人发作道:“你们做的好事。这男子是几时弄进来?每人睡过几十夜?好好招出来,如若不招,我就要惊官动府,叫丫鬟去知会邻舍,说拿住奸夫,先叫他进来验一验,好连箱抬去送官。”

香云与瑞珠、瑞玉惊得面如土色,只得走到背后去商量道:“她的说话是狠意,我们若不理她,她就要弄假成真了。如今我们该走过去调停她,把这个男子放出来,公用就是了。”遂一齐走到花晨面前道:“这桩好事,原不该偏背姑娘。如今自知理亏,不敢巧辨,只求姑娘海涵。就把箱中之物送出来请罪就是了。”花晨道 :“请罪之法,该甚么样道理?倒要请呀!”香云道:“不瞒姑娘说,我们三人三股均分,如今也把姑娘派上一份。”花晨大笑道:“好个请罪的法子,你们把人藏在家中,不知睡了多少日子,到如今败露出来,方才搭我一份。难道从前睡过的,都不消追究了?”瑞珠道:“据姑娘的意思,要怎么样?”花晨道:“若要私休,只除非叫他跟我回去,随我作乐,睡睡几时,补了以前的欠数。然后把他交付出来,与你们一个一夜,重新睡起。这还可以使得。不然,只有官休之法,拼得打破饭锅,大家不吃就是了。有甚么别说?”瑞玉道:“这等,也要说个数目。或是三夜,或是五夜,就放他过来便好。”花晨道:“这个数目定不得,等我带他回去审问一番,说你们三个睡过多少夜数,我就要也睡多少夜数,然后交出来。”三个听了内心暗想,未央生爱我三人,未必肯说真话,或者少说几夜也不可知。就一齐应允道:“既然如此,他只来得一两夜,你竟带回去,审问他就是了。”

三个定议之后,就要开了箱子,放未央生出来,好随她过去。花晨怕他要逃走,就对三人道:“日间走过去,要被家人看见,不妙。我今有个妙法,连这锁也不消开,只说这一箱古画原是我家的,叫几个官家进来,连这箱连人抬了过去就是了。”说了这一句,不等他们回复,就吩咐丫鬟去叫官家。不多时,四个官家一齐唤到,把画箱撮上肩头,抬了飞走。可怜这三个姊妹,就像送棺材的孝妇一般,心上悲悲切切,只不好啼哭出来。不但舍不得这幅活春宫被人连箱劫去,还怕箱中之人被淫妇干死,有路过去,无路回来。只因书箱这件东西与棺材无异,恐怕是不详之兆也。

评曰: 看庙中相遇一回,疑是花晨之好事在瑞珠、瑞玉之先,而评花晨数语,即穿珠之线、引玉之砖也。孰意作者之心与造物之心无异,别有一种安排,决不肯由人计较,以最易得之人,反出最难得之人之后,亦可谓奇之极、幻之至矣。

\chapter{第十七回 得便宜因人瞒己 遭涂毒为己骄人}

作者:《肉蒲团》李渔

花晨把未央生抬到家里,打发管家出去之后,就开自己箱子,取出一套男衣,一顶旧巾并鞋袜,是他丈夫在日穿的,摆在书箱边。然后开了金锁,请出未央生,替他穿着。二人先见了礼,然后对坐。未央生那张利嘴,是极会骗人的。说我在庙中相见之后,终日思想,不知尊姓芳名,无由寻觅。幸得今日天假以缘,因祸得福,方才得观芳容。

花晨只因看见批评,想他果然见许,就把假话当了真言,心上欢喜,等不得到晚,两个就上床做事。她的身体虽不叫做极胖,也有八分身体。未央生才爬上身,被她紧紧抱住,亲一个嘴,叫一声“心肝”,未央生就遍体酥麻起来,觉得妇人睡过许多,未尝有此之乐。

这个甚么原故?要晓得妇人里面有中看中用二种。中看者,未必中用;中用者,未必中看。那中看的妇人要有“三宜”。哪“三宜”?宜瘦不宜肥;宜小不宜大;宜娇怯不宜强健。所以墙上画的美人,都是画瘦小娇窃的,再没有画肥大的身子,健旺的精神。凡画的美人,是画与人看的,不是把人用的。那中用的也有“三宜”:宜肥不宜瘦;宜大不宜小;宜强健不宜娇怯。怎见得中用的妇人要有这“三宜”?凡男子睡在妇人身上,一要温柔似褥;二要身体相当;三要盛载得起。瘦的妇人同石床板榻一般,睡在上面混身都要疼痛,怎能像肥胖妇人,又温又软?睡在上面不消干事,自然会麻木人的身体,最爽人的精神。所以知道瘦不如肥。与矮小妇人同睡,两下的肢体不能相当,凑着上面凑不着下面;凑着下面凑不着上面,竟像与孩子一般,那能有趣?所以知道小不如大。男子身子之轻重,多者百余斤,少者亦有七八十斤,若不是强健妇人,那里盛载得起?睡在娇怯妇人身上,心下惟恐压坏了她。追欢逐乐之事全要以适性为主,那里经得要战战兢兢?所以知道娇怯不如强健。

这等说起来,中看中用两件事竟是相反的。若能与相反之事相兼得来,这样妇人,只要有八分姿色就是十足的了。花晨年纪虽大,实能兼此二美。未央生睡在床上,花晨就露出所长,把一双嫩肩搂住他上身,一双嫩腿搂住他下身,竟像一条绵软的褥子,把他裹在中间。你说快活不快活?未央生以前所御的妇人,都在瘦小娇怯的一边,何知有此乐?所以还不曾动手,竟觉得遍体酥麻了。只因身上快活,引得下面的东西分外雄壮坚固,遂把阳物对着阴户直刺。

花晨的阴户是生育过的,里面自然宽大,不见痛楚就入佳境。只见到十抽之外,搂着未央生叫道:“心肝,快些弄。我要丢了!”未央生狠抽不上十下,又叫道:“心肝,不要动。我丢了!”未央生就把龟头抵住花心,停了一会,待她丢过之后,又弄起来。一边弄一边问道:“心肝,你的本事怎么这等不济?抽不上三十下竟自丢了?你那三位侄女多的要二三百抽,少的也要一二百抽,方才得泄。我还说她们容易打发,那里晓得妇人里面更有容易打发的。”花晨就应道:“你不要把我看容易,我是妇人里面第一个难打发的。若不到一二千抽不得我丢。就是到了一二千抽,我要丢的时节,也要费上好些气力,不是这等抽送就弄得丢。”

未央生道:“你既有这样本事,为何方才这一遭容易打发?难道是假丢,骗我不成?”花晨道:“不是假丢骗你。有个原故,因我十几年不见男子,欲火甚盛。及忽见你人物又标致、本钱又壮大,心上欢喜不过,所以才塞进去,那阴精不知不觉就出来。这是我自己丢的,不关你抽送之事。你不信,只看这一次,就不比方才了。”未央生道:“原来如此。你方才的话,我还有些不明白。你说到一二千抽,也要费好些力气,不是弄得丢,这一句说话,真正难解。莫非除了抽送之外,还有别的干法不成?”花晨道:“干法不过如此,只要加些助兴的功夫,或是弄出响声,或是说起骚话,使我听得兴起,方才会丢。若是底下没有响声,口里不说骚话,就像与哑男子干事一般,有甚么兴趣?随你一夜弄到天明,那阴精也不肯来。只是一件,我的丢法与别人不同,竟要死去一刻时辰,方才得活来。我预先对你说明,你若见我死去的时节,不要呆怕。”

未央生道:“这等说来,竟要强雄健壮,极有精力的男子方才弄得你丢。我的精力算不得头等,也还是二等前列,或者能应付你。但不知你亡过的尊夫,精力何如?”花晨道:“他的精力算不得二等,只好在三等前列。他当初也极爱偷妇人,做了许多伤伦之事。他尝对我说,别人的阴户都是肉做的,只有你的是铁打的,千方百计再弄不丢。就想出许多助兴之法,煽动我的欲火,后面干起来也就容易。不论一千二千,只是心窝快活就要丢了。”未央生道:“这等话说,那些法子是怎么样的?”花晨道:“那些法子极容易做,做来也极有趣。不过是三件事。”未央生道:“哪三件事?”

花晨就念道:“看春意、读淫书、听骚声。”未央生道:“‘看春意’、‘读淫书’,这两件事我初婚的时节都曾做过,果然是有趣的事。至于‘听骚声’这件事,不但文字不曾做过,连题目也解说不来。怎么叫做‘听骚声’?花晨道:“我生平及喜听人干事,可以助我的兴动。当初先夫在日之时,故意叫他偷丫鬟,又要他弄得极响,干得极急,等丫鬟极快活不过,叫唤起来。我听到兴浓之际,然后咳嗽一声,他就如飞走来,抱我上床,把阳物塞进去,狠舂乱捣。不可按兵法,只是一味狠野战。这等干起来,不但里面快活,连心窝里都快活。只消七八百抽,就要丢了。这个法子比看春意、读淫书更觉得有趣。” 未央生道:“这种议论甚是奇畅。只是一件,依你方才说话来,尊夫的精力也在单薄一边,怎能先弄丫鬟,后干主母?而且起先又要弄得极响,干得极急,飞搬过来的时侯,一定是强弩之末了,怎么又能再肆野战?这事我还不能信。”花晨道:“起先不要他干,另有代庖的人。就是后来野战,也要央他接济。不然,哪里支持的来。”未央生道:“那代庖的人我知道了,莫非是一位姓‘角’的么?”花晨道:“然也。这件东西,我家里最多。今日我和你初交,料想不到难丢地步。明日干事,就要用到此法了。”

未央生听了,也就不按兵法,挺起一味野战,乱来舂捣,抽了数千,自然从阴户快活到心窝里去。只见她手寒脚冷,目定口张,竟像死得一般。若不是预先说破,未央生竟要害怕。果然死了一刻时辰,方才苏醒。搂着未央生道:“心肝,你不消用代庖之物,竟把我弄丢了。这看来你的精力竟是特等,怎么说在二等前列?”

未央生道:“我册子上面取你做特等,你如今也取我做特等,何相报之速耶。”花晨道:“我正要问你,那册子上面他们三个名字是哪个涂抹的?后面一行批语是那一个添上?”

未央生不好说出,只推不知。花晨道:“你虽不肯说,我心上明白不过。那三个说我年老色衰败,还能配得她们过。把自己比做淮阴,把我比做绛灌,是个不削为伍的意思。不是我夸口说,她们的年纪虽幼小几岁,面色虽比我嫩几分,只好在面前你看看罢了。若要做起事来,恐怕还赶我老人家不上。我今忍在心里,不与她们争论,待等闲空时节,待我走过去,约她们做个胜会,一个奇男子,四个俏佳人,都要脱了衣裙,日间干事,与她们各显神通,且看是少年的好,老成的好。”未央生道:“说得有理,这个胜会不可不做。”

二人见天色暗起来,穿了衣服,丫鬟排上酒肴。花晨酒量极高,与未央生不相上下。二人猜拳行令,直饮到更初。乘了酒兴,依旧上床干事。这一晚是久旷之后,阴精易泄,不消用三种法子。到了次日起来,就把许多春意、淫书一齐搬运出来,摆在案头,好待临时翻阅。他看两个长丫鬟,年纪俱在十七八岁,都有姿色,又是已经破瓜的,承受得起,就吩咐在身边,以备助兴之用。 从此以后,朝朝取乐,夜夜追欢,都用三种成法。花晨最怕隔壁的人要来索取,追还原物,自从画箱过来之后,就把旁门锁了。随她叫唤,只是不开。叫到第五日,未央生过意不去,替她哀求。花晨没奈何,只得说要睡到七日,到第七日后送去还她。那三人见有了定期,方不叫唤。到第八日上,未央生要辞别过去,花晨还有求闺之意。亏得未央生善为说辞,方才得脱。及至开了房门,走了过去,香云姊妹三人见了大喜,就问未央生道:“你连夜的受用何如?这老东西的兴趣何如?”未央生怕她们吃醋,不敢十分赞扬,只把三种成法说与她们听,好等学样。连花晨要做胜会的话也说出来,叫她各人争气,切不可以一日之短,埋没了千日之长。

三人听了,遂暗暗商量算计花晨,未有定着,只得放下。香云道:“今日为始,又要照从前次序,每人分睡一夜何如?”瑞珠、瑞玉道:“如此极妙。”三人遂分睡三夜,到了四日,正打点要做合体联形之事,不想花晨写字过来,约她们三个做盛会,又出了一两公份,叫她们备办酒席:一面饮酒,一面干事,方才觉得有兴。三个商量道:“恰好今日是个合睡的日子,自古道‘添客不杀鸡’,就等她来大觉会聚也分不多少去。这落得做个虚人情。”立刻写字回她:“谨依来命。”

花晨的名分大,为甚么不叫侄女就姑娘,反屈姑娘就侄女?要晓得她家里有个十岁的儿子,虽然不大,也是有知识的。起先把未央生一个藏在家中不觉得,如今一男四女饮酒作乐起来,恐遮掩不住,被儿子看见不好意思。香云姊妹三个都是没有儿子的,只要关了二门就不见人影了,所以不论尊卑,情愿过来就她。

只见回字去后,过了一会,花晨就来赴会。未央生见她衣袖之中隐隐跃跃却像有物的模样,就问她道:“袖中何物?”花晨道:“是一件有趣的东西。酒色二件事都用着它,所以带来。”就取出与众人看,原来是一副春意酒牌。未央生道:“这件东西今日做胜会才好用着,如今且不要看,等到酒兴发作之时,你们各取一张,照上面的法则,同我模仿一模仿就是了。”香云道:“这等,待我四人先看一遍,看明白了,到了临期之时才好模仿。”未央生道:“也说得是。”花晨道:“我看过多次,上面的方法都是烂熟的,不得临时抱佛脚。如今立过一边,让你们看看就是。”

三人笑了一笑,就摊开牌来,逐张仔细看。看到一张,只见一个少年女子覆在太湖石上,耸起后庭,与男子干龙阳之事。三人看了一齐笑道:“这是甚么形状,为何丢了乾净事不做,做起龌龊事来?”花晨道:“是哪一张?拿来我看。”香云就递与她。她看了道:“这个干法,是从文字上面摹拟下来,难道你们不晓得?”

香云道:“是哪一篇文字?我们不曾看过,求你指教。”花晨道:“是一篇《奴要嫁传》。当初有个标致闺女,与一个俊俏书生隔墙居住。书生想这闺女,不得到手,害起相思病来。央人到闺女面前致意,说只要见得一面,就死也甘心,不敢做非礼之事。那闺女见他说得可怜,只得应允。及至相会的时节,坐在书生怀里,随他要搂就搂,要摸就摸,要亲嘴就亲嘴,只不与他干事。等他要干就回复道:‘奴要嫁人,此事不可为。’书生急不过,跪在地下哀求,她到底不允。只把‘奴要嫁’三字回他。说你求见之心不过因我生得标致,要靠一靠身体,粘一粘皮肉,我今坐在你怀中,把浑身皮肉随你摩弄,你的心事也可以完了,何须定要坏我原身,明日嫁去时节被丈夫识破此事,我一世就做不得人了,这怎么使得。书生道,男女相交,定要这三寸东西把了皮肉,方算得有情,不然终久是一对道路之人,随你身体相靠,皮肉相粘,总了不得心事,只是跪在地下哀求不肯起来。闺女被他哀求不过,只得低头暗想,想出权宜之法,就对他道:‘我是要嫁的人,这件东西断许你不得。我如今别寻一物赠你,何如?’书生道:‘除了此物,那里还有一物?’闺女道:‘除非舍前而取后,等把你三寸东西一般进了皮肉,了却这桩心事,再没得说了。’书生见她说得真切,也就不好再强,竟依这个权宜之法,把后庭当做前伴,交情起来。这个干法,就是从那篇传上摹拟下来的。这样好书,你们何不曾读过?”香云姊妹三人见她说话骄傲,心上甚是不平,就丢了酒牌不看,一齐到背后去商量。大家协力同心,要摆布她一场。

花晨与未央生隔了三日不见,胜似九秋,巴不得众人开去,好与他绸缪一番。两个就搂住亲嘴,说了许多话,那秭妹三个方才走来。叫丫鬟摆酒,未央生上座,花晨下座,香云与瑞珠、瑞玉分坐两旁。饮过数寻,花晨就叫事牌,过来各取一张,照上面行酒。香云道:“看了那件东西,只想要干事,连酒都吃不下。如今且行别令,吃到半酣,然后取它过来,照上面行酒也得,照上面行事也得,就无碍了。”未央生道:“也说得是。”瑞珠遂取出色盆来,未央生道:“掷骰费力,不如猜个状元拳,定了前后次序。如今照次序行酒,少刻就照次序行事,列位心上何如?”花晨的拳经最熟,听见这话就眉欢眼笑,巴不得要做状元,好摆布她们三个。所虑者,恐中状元干事要从她干起。她是要先听虚声,后干实事的人,那里肯当头阵。想了一会,就对道:“行事的次第,不必照依行酒,只凭状元发挥,凭她要先就先,要后就后。”

未央生道:“也说得是。”就把五个拳头一齐伸出,从未央生猜起,猜到瑞玉住。果然花晨拳高,一口就被她猜着状元,是她中去了。不等榜眼、探花出来就先发令道:“我既中状元,就是个令官,不但老儒听考,连榜眼、探花都要受我节制,如有抗令者,罚一大杯。”未央生道:“既然如此,求你把条教号令预先张挂出来,定了个规矩。”花晨道:“吃酒的数目,从状元起到探花住,吃个节节高。老儒执壶旁立,只教她斟,不许她吃。干事的先后,要与前面相反,从探花起到榜眼住,也干个节节高。老儒执巾旁立,只叫她揩,不许她干。”又对未央生道:“你如今不用考,委你做监令官,好待后面用你干事。”未央生道:“这等说,我事便有得做,酒却没得吃了。”花晨道:“你的酒数更多,状元、榜眼、探花有酒,都要你陪。只是老儒服役,不许你去待劳。代劳讨好者罚一巨杯。”未央生道:“她自己不争气,去做老儒,不干我事,凭她去受苦罢了。”香云姊妹三个侧目而视,让她发挥,不敢稍参末议。还亏她虑在事前,起先到背后去想了一个妙计,放在胸中。就对未央生道:“你既做监令,若令官不公道,你也要参劾她,不要阿谀曲从,助纣为虐。若是如此,我们就鼓噪起来,不受约束了。”花晨道:“若做得不公,不消监令参劾,你们只管公举,举得确常,我只管受罚就是。”

花晨定了条约,就除出未央生,教她姊妹三人决个胜负。却也古怪,那三个拳头恰好也照序齿之例,香云中了榜眼,瑞珠中了探花,把个经不得大干的瑞玉做了老儒。猜定之后,花晨就叫瑞玉行酒,自己一杯,香云两杯,瑞珠三杯。都是未央生陪吃。吃完之后,就叫瑞玉把酒牌洗好放在桌上,然后执巾旁立,待众人干事之际,好替她揩抹淫水。瑞玉不敢违拗,只得依令。

花晨对未央生道:“头一个限你一百抽,第二个限你二百抽,多一下,少一下都要罚酒。丢与不丢,看她造化,不累你管。干到第三个就得轮着我了,主令之人,与众人不同,不计数目,定要丢了才住,以前两个的数目,都要老儒代数,差者罚。”又对香云、瑞珠道:“你们上前揭起,揭着那一张,就依那一张的干法好与不好凭人造化,不许换牌。干事的时节,要摹仿酷肖方才中式,若有一毫不像,除罚酒外还要减去抽数。”瑞珠道:“我们做得不像,自然受罚;若令官不如式,却怎么处?”花晨道:“令官不如式,罚了三杯,重新做起,定要做到如式才住。”

瑞珠听了,就伸手去揭第一张,只见一个妇人睡在床上,两足张开,男子的身体与妇人隔开三尺,两手抵住了席,伏在上面抽送,叫做“蜻蜓点水”之势。瑞珠把酒牌呈过了堂,就脱下裤子,仰卧在床上。未央生爬上身去,仿起蜻蜓的样子,把阳物塞进阴中,不住的乱点。瑞珠要奉承令官,后面动兴,不等快活之后方才叫唤,未央生点一点,她浪一浪;点十点,她浪十浪。直浪到不点才住。

香云道:“如今临着我了”。就揭起第二张,见一个妇人睡在春榻头上,男子立着,把她双脚放在肩头,两手抵住春榻,用力推送,叫做“顺水推船”之法。香云也把酒牌呈过了堂,就睡在春榻上去,与未央生摹仿成式。她那个浪法,更比瑞珠不同,顺水推船既容易推,则顺船之水也容易出,船头上的浪声与船底下的浪声一齐澎湃起来,你说好听不好听?

花晨往常窃听骚声都是暗中摸索之事,何曾看见这快活头上。如今见了,那种淫兴比往常咳嗽的时节更不相同,大有不能姑待之意。等得香云满数之后,就立起身道:“如今轮着令官了。”就把一只手取牌,一只手插在裤裆,先去解带。及至揭起第三张一看,不觉惊慌失色,对众人道:“这一张是用不得的,只得要别换一张。”香云姊妹三个一齐鼓噪起来,先把余下的牌藏在一处,然后来看这一张。

原来就是“奴要嫁”的故事,妇人耸起后庭,与男子干龙阳的套数。为甚么这等凑巧?多少牌揭不着,偏揭这一张?原来就是她姊妹三人商量出来的计策。料想她三个毕竟轮着一个洗牌,就把这一张做了计号,要分与她。谁想她又预先号令出来,众人居先,令官落后,所以瑞玉洗牌的时节就把这一张放在第三。如今恰好取着,这也是她骄傲之报。

三个看过了牌,就催花晨脱裤。花晨抵死不肯,道:“求列位公议,这一桩事可是做得么?况他那一件东西,可是做得这一桩事么?大家想一想就是了。”三个道:“这个说不得,若是我们揭着,你可肯饶恕我们么?况且不许换牌的话,又是你说的。牌上的方法,只有你烂熟。你既知道这张用不得,何不预先除出这一张?如今揭着了,还有甚么说?快些脱裤,省得众人动手。”又对未央生道:“好个监令官,为甚么口也不开,手也不动?要你何用?”未央生道:“不是监令官徇情,其实我这件东西,她后面原当不起。还要开个赎罪之例,等她多吃了几杯酒,当了这事罢。”三人道:“你这句话,只当放屁!若是吃酒当得干事,我们起先只该吃酒,不该干事了。哪个是不顾廉耻,肯脱衣服在人面前出丑?”

未央生见她们说得词严义正,无言可对,只得求众人道:“如今我也没得说,只求刻令开一面之网,不要求全责备,等她脱下裤来,略见大意罢了。”香云、瑞玉还不肯依,要与寻常干事一般,瑞珠紫一紫眼道:“只要见得大意也就罢了。难道定要尽法不成?”未央生道:“这等还易处。”就伸手去扯花晨,替她脱裤。花晨执意不肯,被未央生苦劝不过,低头丧气,只得曲从。就把裤子解开,伏在春榻头上。未央生取出阳物,抹上涎唾,只在肛门外面抵得一抵,花晨就叫喊起来。正要立起身子不容他干,谁想这班恶少安排三双毒手等她。起先紫眼的话,是哄她脱裤,等她脱了裤子伏上春榻,就一齐走上前去,捺头的捺头,封手的封手,莫说立不起,就要把身子动一动也不能。更有一个最恶的,躲在未央生背后,等他抵着肛门的时节,就把未央生的身子着力一推。那阳物竟推进了半截,又把住未央生的身子,替他抽送。花晨就像杀猪一般,大声喊叫“饶命”。未央生道:“人命相关,不是当要的事,饶了她罢。”众人道:“她起先说令官与众人不同,不论次数,直要丢了才住,如今问她丢了不曾?”花晨连声应道:“丢了、丢了。”

众人见他狼狈已极,只得放手。花晨立起身来,就像死人一般,话也说不出,站也站不牢,只得叫丫鬟扶了回去。后来肛门臃肿,发寒发热,睡了三四天方才爬得起。从此以后心上虽怀恨,只因要做这桩勾当,不好怨恨同事之人,只得与她们相好起来,一男四女,共枕同衾,说不尽她们的乐处。

未央生出门之日,原与艳芳约以三月为期,就回来看她分娩。不想乐而忘返,等到想着期,已在三月之后。叫书笥出去打听,闻得艳芳已经分娩,一胞生下两个女儿。花晨四人办酒,与他贺喜。又作乐了几日,方送他回去。艳芳恐怕孩子累身不好作乐,就雇了两个奶娘,把孩子抱去抚养。恰好到弥月之时,未央生走到。就叫他大整旗枪,重新对垒,要严追已往的积逋。那里晓得民穷财尽,一时催征不起。这是何故?只因四五个月中,以一男而敌四女,肆意奸淫,不分昼夜,岂有不神疲力倦之理?从此以后,艳芳不能遂其欲,遂有悔恨之心矣。

评曰:

有病此回形容太过,不为奸夫淫妇留余地者,然非此回之奇淫不足起下回之惨报。纵容他处,正是难为他处。看到玉香独擅奇淫,替丈夫还债处,始觉以前数回不妨形容太过耳。

\chapter{第十八回 妻子落风尘明偿积欠 兄弟争窈窕暗索前逋}

作者:《肉蒲团》李渔

未央生得意之事按下慢表,再说他妻子玉香跟了权老实与丫鬟如意逃走,走到一处,忽然肚痛起来。她肚里的东西起先在家时节千方百计再打不下,如今走到路上受些辛苦,不觉就坠了下来。若早坠几日,岂不省了这番举动?如今逃走出来,回去不得,白白做了私奔之人,岂不是丈夫造下的冤孽带累她如此?

权老实的初意原为报仇,不是贪淫。自从拐出之后,就要卖她下水,只因有孕在身,踌躇未决。此时见她落下胎来,方才定了主意。就把主婢两个带入京师,寓在店中,寻人货卖。但凡卖良为娼,定要做个圈套,瞒了本妇,只说有亲眷在此,托他寻房居住,才好领人来看,看中了意,才好骗她入娼门。京师里面有个鸨母叫做“顾仙娘”,一见玉香就知道是桩奇货,照媒人所说的身价一天平对出来,连如意也买过去,依旧做了丫鬟服事她。

权老实卖过玉香之后,就有些过意不去,渐渐懊悔起来。心中想道,我闻得佛经上说,要知前世因今生受者,是要知后世因今生作者。是我自家妻子做了丑事,焉知不是我前世淫人妻之故?今世把妻子还人也不可知。我只该逆来顺受才是,为甚么又去淫人妻子,造起来世的孽障来?就是要报仇,既然与她睡过几夜,消了意恨也就罢了,为甚么又卖她为娼?又把她无事使女也卖下水去?权老实想到此处,不禁捶胸顿足,自家恨起自家来。想从前的事俱已做错,不可挽回,只有个忏悟今生,预修来世之法。就把卖人的银子,施舍与残疾穷苦之人,自己把头发剪去半截,做了个头陀,往各处去云游,要访真正高僧,求他剃度。后来游到括苍山中,遇着孤峰长老,知道是一尊活佛,就摩顶皈依了他,苦修二十年,成了正果。这是后话。

却说玉香堕落风尘,与如意两个走到顾先娘家,看一看动静,才晓得不是良家的光景。就是贞烈妇人跨进这重门槛也跑不出去,何况已经是失节之妇?玉香看了无可奈何,只得安心贴意,做起青楼女子的行径来。遂改名字叫□妙,取个表字,好待嫖客称呼。作者还叫他玉香,省得人看花了眼。

初到的一晚,就有个大财主来嫖。到第二日就要去,顾仙娘留他不住,他临去的时节吩咐顾仙娘道:“这位令爱容貌丰姿,件件都好,单少那三种绝计。你还应该传授她才是。我如今暂别,待你传授她会了再来请教。”说罢回去。他为甚么说出这话来?原来顾仙娘生平有三种绝技,都是妇人里面不曾讲究过的。她少年时节容貌也平常,竟享了三十余年的盛名。与她相处的都是乡绅大老,公子王孙,就到四五十岁的时节,还有富贵人去嫖她,就是为此三种绝技。第一种是俯阴就阳;第二种是耸阴接阳;第三种是舍阴助阳。她与男子干事,教男子仰面睡了,她爬上身去,把阳物插入阴中,立起来套一阵,坐下来揉一阵,又立起来套一阵。别的妇人弄了几下就腿酸脚软,动不得了。她一双膝弯竟像铁铸的一般,越弄越有力气。不但奉承男子,连自己也十分快活。这就叫做俯阴就阳,是她第一种绝技。她有时候睡在底下与男子干事,再不教男子一人着力,定要把自家身子耸动起来,男子抵一抵,她迎一迎;男子抽一抽,她让一让。不但替了男子一半气力,她自家也讨了一半便宜。若还女子不迎不送,只叫男人抽抵,何不把泥塑木雕的美人腰间控一个深孔,只要伸得阳物进去,就可以抽送得了,何须要与活人干事?所以做名妓的人要晓得这种道理,方才讨得男人欢心,图得自家快乐。这就叫做耸阴接阳,是她第二种绝技。至于舍阴助阳之法,一发玄妙,她与男子干事,再不肯使有限的阴精泄于无用之地,每丢一次,使男子受她一次之益。这是甚么样的法子?原来她与男子干事到将丢之际,就吩咐男子,教他把龟头抵住花心,不可再动;她又能使花心上小孔与龟头上小孔恰好相对,预先把吸精之法传授男子,到此时阴精一泄,就被男子吸进阳物之中,由尾闾而直上,径入丹田。这种东西的妙处,不但人参附子难与争功,就是长生不老的药,原不过如此。这种妙术是她十六岁上有个异人来嫖她,无意之中说出这道理,被她学过来,遇着有情的嫖客,就教如此如此,嫖客依她做来,无有不验。与她宿过几夜,不但精神倍加,连面上的颜色也光彩起来。人都说是仙女转世,所以教她做仙娘。这种道理既传与嫖客,那嫖客就该到自己家里去做,不须用着她了。要晓得吸精之法,虽然可传,那对着精孔之法,是传授不去的。要在干事的妇人善于凑合,这些关窍,只有她肚里明白,别的妇人那里凑合得来。妙在天下妇人皆迷,惟有她一人独悟,所以叫做绝技。

玉香初到底时节,那里晓得有这三种绝技,嫖客与她干事,见她第一种绝技尚然不会,那两种一发做不来了,就与她草草完事。睡到天明,见她美貌,舍她不得,可惜不谙此法,所以临行之际有这一番叮咛。仙娘送了嫖客出门,就骂她装娇作态,不曾奉承,把这大财主接得一夜就打发开去,以后怎么样赚钱,就要鞭打起来。玉香跪下再四哀求,仙娘方才饶了,就把这三种绝技,日夜与她讲究。自己同嫖客干事,就教她立再面前细看,会与不会,好当面指教她;她与嫖客干事,自己也坐在面前细看,是与不是,好当面提醒她。

俗语说得好,天下无难事,只怕有心人。玉香惧怕鸨母的法度,不敢不学,只消一两月工夫把三种绝技都学会了。又兼姿容秀美,笔墨精工,一时闻名动京师。没有一个乡绅大老公子王孙不来赏鉴。更有两个大老官极肯破钞,宿她一晚,定有一二十金相赠。你道这两个大老官是那个?原来就是瑞珠、瑞玉的丈夫,一个叫做卧云生,一个叫做倚云生。因在京里坐监,闻得玉香的盛名,兄弟两个争先拜访。起先是卧云生,瞒了阿弟先去嫖了几夜。后来是倚云生,瞒了阿兄也去嫖了几夜。一日兄弟两个盘问出来,遂索性把玉香包在家中,大家公用。不但兄弟同利,又且师弟同门,连香云的丈夫名为轩轩子,也时常点缀点缀。与她睡过一两夜,竟有些老当益壮起来。方才晓得玉香的阴物竟是一味补药,若取着这样妻子,竟不消躲避差徭了。

卧云生兄弟在监里坐了一年,偶然想起故乡,要回去看看妻子,就央一个人情,求大司成给假数月,大司成批了。师弟三人别了玉香一同回去。到了家中,少不得三位佳人替丈夫接风之后,就问一向在外嫖了几个女客。三位丈夫就把相处玉香的话陈说一遍,又把那三种绝技次第夸张出来。香云姊妹三个第二日起来各述所闻,都是一般诧事。瑞珠、瑞玉道:“我不信妇人之中竟有那样怪物。这等说起来,我们三个都是没用得了,这些话还是他们三个通同造出来,要激励我们用心干事的意思。”香云道:“这样事瞒不得我们相处的人,他生平见广识多,若有这一种妓妇,他毕竟晓得。等他进来,大家一问就是了。”瑞珠、瑞玉道:“也说得是。”

一日,遇着清明佳节,三个的丈夫一齐出去扫墓,要第二日回来。就叫丫鬟请未央生进去相会。一见了面,就把这疑事问他。未央生道:“天下的事奇奇怪怪,或者妓妇里面有这一种阴物也不可知。她既在京师,我终有一日遇着她,待我嫖她一夜,若对得我过的,方才是个真怪物。”四人说了一会,宿了一晚。未央生次日出来,心上想道,他们三个丈夫的话如出一口,可见这一桩事是真的了。当今之世有这样异人,何不去会她一会?况且我的精血被这四五个妇人也耗得多了,正要学个采战之法。滋补一滋补。那个妓者既有许多妙术,我只消嫖她一夜,把个吸精之法传授过来,就一生受用不尽了。主意定了,就要先回故乡看看妻子,然后进京去访那名妓。他这一去,有分教:

触翻东岳,泄不尽愤懑之胸;掬尽西江,洗不尽羞惭之色。
要知分解,就在下回。

评曰:

未央生之淫恶已造到极处,若使其妻子止于偷汉而不至于为娼,人犹不痛快。即使为娼,人心犹不痛快。即使为娼,止于接他客而不及香云姊妹之夫,人心犹不痛快。一部淫书看到头,无一人不报,稍有风流罪过之人,未有不通身汗下者,如此淫书不可不多读也!

\chapter{第十九回 孽贯已盈两处香闺齐出丑 禅机将发诸般美色尽成空}

作者:《肉蒲团》李渔

未央生临行之际,走去辞别赛昆仑,把家中之事交托与他,求他照管。赛昆仑道:“托妻寄子的事,不是轻易任的,寄子容易,托妻甚难。劣兄只好替你料理薪水,不能替你防守闺门。”未央生道:“小弟所托之事单为薪水,不虑闺门。你弟媳妇是个过来人,比初嫁丈夫的不同。天下中用的男子不过像权老实,他尚且嫌他不济,要跟小弟终身。料想男子里面没有第二个像小弟的,老兄不必过虑。”赛昆仑道:“也说得是,只要贤弟信得过劣兄,受托也不妨了。” 未央生别过赛昆仑,就写封密札寄别花晨与香云姊妹,又与艳芳绸缪了几夜,方才起身。不一日,到了故乡,走到铁扉道人门首,敲了半日不开。心上暗喜道,他门户这等森严,料想没有闲人进去,我就再迟几日回来也不妨了。直敲到晚,方才有个人影在门缝里视望,未央生晓得是铁扉道人,就叫“岳父开门,小婿回来了”。铁扉道人听见,忙把门开,接他进去。

未央生走进中堂,见过了礼,就问起居。先候岳父的台安,后问令爱的清吉。道人叹道:“老夫身体倒还粗安,只是小女自贤婿去后,就生起病来,睡卧不安,饮食不进,竟成了忧郁之症,不上一年就身故了。”说罢放声痛哭。未央生道:“怎么有这等异事?”也就痛哭起来。哭了一阵,又问“灵柩在哪里,如今葬了不曾?”道人道:“现停在冷屋里,等你回来见一见才好安葬。”未央生就走到冷屋,伏在灵柩上又重新哭了一场。

你道这口棺木是哪里来的?原来是铁扉道人见女儿跟人逃走,不好说得,一来怕乡舍取笑,二来怕女婿要人,只得买口棺木回来,封钉好了,只说女儿病故,停在家中,既可掩人之耳目,又可免女婿之追求。未央生因他平日至诚,没有虚话,所以并不疑心,反自怨不早回来,以至她忧郁而亡。就请几众高僧,做三日三夜好事,追荐亡灵,教她早生早化,不要怨恨丈夫贪恋女色,在阴间吃起醋来,做活王魁的故事。追荐之后,仍以游学为名,别了道人,往京师进发,要学滋补之方。

不一日,到了京师,安顿行李,就去访问佳人。访着住处,就去登门拜见。谁想玉香数日前被一个大老官请去,睡了数日不肯放她回来。仙娘回复了未央生,未央生只得回寓。过了两日,又去拜访,仙娘道:“小女昨日有个话来,说今日靠晚就到。”未央生听了,就送嫖金三十两,还有几件私礼,待她回来面送。仙娘收了嫖金,又道:“如今天色尚早,相公若有别事,且去一会再来,若没有别事,就在这里等。”未央生道:“我专为令爱而来,没有别事。”仙娘道:“这等,到小女房中坐下,或是看书,或是睡觉。待小女一到就来奉陪。”说罢,就领未央生进房,吩咐一个小妓教他煎茶服事。又对未央生道:“老妇有俗要去料理,不能相伴。”遂转身出来。

未央生想要将养精神,好到夜间干事,就从午刻睡起,直睡到薄暮,方才下床,取了一本书正在看,只见纱窗外有个标致妇人把他张了一张,就慌忙走开去,却像要躲避的一般。未央生就问小妓道:“方才张我的人是哪一个?”小妓道:“就是我家姊姊。”未央生看见那些光景,怕她有拒绝之心,就出来求见。 玉香起先张了一张,认得是自己丈夫,只说有心来捉她,所以慌了手脚,要同仙娘商量去路。不想走到仙娘房前,还不曾说话,就望见未央生赶来,只得对仙娘道:“此人是接不得的,不可使他见我。”就跑入仙娘房里,把门窗坚闭,声也不则。仙娘不知就里,只想她心上不爱,所以不肯接他。就去对未央生道:“小女又有信来,就依旧被他留住,不得回来。却怎么处?”未央生道:“令爱回来了。怎么是这等说?莫非怪我礼物轻微么?”仙娘道:“真是不曾回来,并无他意。”未央生道:“方才明明在窗外张我,一张就躲避开去。怎么讲这样胡话?就是有些怪我,也须与我想见一面,再把话辞我,我也是辞得去的。何须这等绝人?”

顾仙娘只是照前话回覆。未央生道“我刚才见一个妇人躲在你房里去,若果然不曾回来,待我搜一搜,若搜不着,我嫖也不嫖,礼物也不取,竟自回去。”仙娘见他说得对针,恐他搜出人又不好意思,只得对他道:“不瞒相公说,来是果然来了。只是被个作孽的男子一连掏漉了几夜,身子缺安,要将息一两夜,才好留客的意思。相公既然执意要见,待我叫她出来就是,何须搜得。”未央生道:“这等,待我亲身去请,省得说我来意不诚,又要推托。”就跟仙娘走到房门前一齐启请。仙娘道:“我儿,相公要会你,你可出来会一会。”连叫几遍,再不见则声。未央生也叫一会,不见开门。

玉香看见势头不好,想起见面之后定要惊官动府。加起刑来,少不得是一死,不如死在未见之先,还省得一场没趣。就解下束腰的带,系在梁上自尽。后未央生见门打不开,打开进去,人已吊死了。未央生看见弄出事来,要想脱身,那里有心看吊死的人是何面貌,遂转身竟走。仙娘见他逼死了人,一把扯住道:“往哪里走?我和你无冤无仇,为甚么把我养差的人活活逼死?”

正在校问之时,只见许多嫖客走到,都是些公子,往常嫖过玉香的,连日因人接去不得见面,闻她回来,大家不约而同都来看她。见被人逼死,大家怒发冲冠,就吩咐管家一齐动手,把未央生按在地下,用青柴短棍打了上千,只有致命之处不曾受伤,其余的皮肉没有一处不被他打的乌青烂熟,打过之后,就把铁练练了,锁在死人旁边。要等地方乡保同来看过,好领户主报官。

未央生起先要逃走,不看死人。如今被打得损伤,又锁在死人旁边,料想脱不得身,就把死人面貌头脑仔细一看,就大惊起来,想这面貌与我亡妻无异,难道天下的面孔竟有这样相同?看了又想,想了又看,越看越像,越想越是。不觉疑心起来,焉知不是我妻跟人逃走,岳父不好说得,买口棺木骗我也不可知。况且这妇人若还没有虚心之事,为甚么见我就躲,躲到后面见躲不脱,就寻起自尽来。想到此处,已有八分明白,又想起妻子顶门里有一灸疤,是不生头发的,我今何不验个仔细。就把她鸦髻分开,里面一看,恰好有指头大的一块,没有头发,正是她无疑了。

忽见地方乡保一齐拥进房来,查问致死来历。未央生道:“吊死之人是我妻子,被人拐骗出来,卖与仙娘接客。自己还不晓得,走来嫖她。她虚心不敢见面,所以悬梁自缢。及致锁在一处,细看面貌方认出来。我这冤枉少不得要到官伸诉,只求早些到官,就见天日了。”众人盘问仙娘,这个女子是甚么人卖与你的?仙娘不知就里,说:“他满口胡言,总是支吾的话,我这女子现有一个丫鬟相随,同时明买的。”众人道:“吊死的人不会说话,可问这丫鬟就明白了。”仙娘起身去叫如意,谁想寻了半日不见,只说她走了。那里晓得竟躲在仙娘床底下,被众人看见,一把拖出来。 原来她也是看见未央生,慌了手脚,同玉香一齐躲入房中,看见玉香吊死,未央生又打进房来,知道没有好处,所以钻在床下躲避。不想被人看见,拖了出来。众人指着未央生问道:“这个人你可认得他?”如意心上还要不认,怎奈面上的颜色,口里的声音竟替她递起认状来。众人知道有些原故再把利害的话恐吓她,她就把玉香在家与某人通奸,怀孕怕父亲知道置于死地,只得跟了某人与自己一齐逃走,谁想某人负心,卖她下水的话,细细招了一遍。

众人知道情节,就劝他两下解交,不必惊官动府。一个逼死自家妻子,料不抵命;一个明买妇人接客,料非拐带。只是这个使女问原主还要不要,若要,便赎她回去;不要,还留在这边。未央生到了这个时候,只当是已死之人,连自家身子都可以不要,巴不得早死一刻也是好的,那里还要她。就对众人道:“论理起来定该到公堂上去,求官府替我追究一番,消消隐恨才是。但恐被人传拨开去,声名不雅,不如依列位,隐忍些罢。这个使女既然做过娼妇,也不便带回,由她在这边罢了。”仙娘见他说出真情料想没有后患,就依众人处分,开了铁锁,追还嫖金,打发他出去。临去的时节还被那些嫖客骂了多少王八乌龟才走得脱身。

未央生回到寓处,棒疮发作起来,叫天叫地,喊个不住。心下想道,我起先只说别人的妻子该是我睡的,我的妻子断没得与别人睡的,所以终日贪淫,讨尽天下的便宜。那里晓得报应之理,如此神速。我睡人的妻女,人也睡我的妻子;我睡人的妻子还是私偷,人睡我的妻子竟是明做;我占人妻子还是做妾,人占我的妻子竟是为娼。这等看起来,奸淫之事,竟是做不得的。我还记得三年前孤峰长老劝我出家,我不肯从,他就把奸淫的果报说来劝我,我与他强说奸淫之事未必人人有报。如今看起来这桩事再没有不报的了。我又说一人之妻妾有限,天下之女色无穷,若是淫了无限妇人,就把一两个妻妾还债也就本少利多,不叫做吃亏了。如今打算起来,我生平所睡的妇人不上五六个,我自家妻子既做了娼,所睡的妇人不止几十个了。天下的利息那里还有重似这桩的?孤峰又说这道理口说无凭,教从肉蒲团上参悟出来,方见明白。我这几年,肉蒲团上的酸甜苦辣尝得透了,如今受这番打骂凌辱也无颜归故乡了,此时若不醒悟,更待何时?不如写一封恳切的书寄与赛昆仑,教他寻一个人家把艳芳打发出去,两个孩子,随她带去也得,留与赛昆抚养也得。我自家一个径至括苍山寻见孤峰长老,磕他一百二十个响头,陪了以前的不是,然后求他指出迷津,引归觉路,何等不妙?

主意定了,就要写书,怎奈两只手臂都被众人打伤,写不得字。将养了一月,手臂好了,就要写书,恰好赛昆仑有书寄到,拆开一看,说家中有急事,教他闻信之日,即便起身,又不说紧急事是那一桩。未央生心上疑惑,不知何事,遂盘问来人。来人道:“是二娘跟人逃走。”未央生又问:“她跟甚么人逃走?”来人道:“莫说我家不知,就是府上的丫头伴当也不晓得。只说未走之先,夜夜听见床上有些响动。及至起来又不见有个人影。一连响了十几夜,那一日清早起来,只见重门洞开,寻觅二娘,竟不知哪里去了。故此家主一面缉访,一面着小人前来追赶相公回去。”

未央生叹道,这个信来又是一番报应了。可见奸淫之债,断断是借不得的。借了一倍,还了百倍。焉知这两个女儿不是还债的种子,如今也虑不得许多,遂写一封决绝书,回覆赛昆仑道:淫姬私奔,不足为奇。悖而入者亦悖而出,此常理也。故乡之事亦复类此。自知罪恶贯盈,有此报。魔障消除之日,即道心发现之期,不当返江东,径归西土。所恨者祸胎未灭,犹存二孽于怀中,暂累故人,延其喘息,俟我见佛后,当借慧剑除之耳。单复不尽。

打发回书去后就欲起身,要把书笥带在身边,做个沙弥服事。后来想了一想,惟恐狡童在侧,又起淫心,不如不见可欲,使心不乱。竟叫书笥跟了来人也发他回去。自己收拾行李,单身独往括苍山去。

评曰:

作者本意直到此回乃见。凡看肉蒲团者,别回只看一遍,此回与下回能看三四遍者,□会看小说之人也。

\chapter{第二十回 布袋皮宽色鬼奸雄齐摄入 旃檀路阔冤家债主任相逢}

却说孤峰和尚自从放过未央生,时时刻刻埋怨道,毕竟是我法力不高,婆心不切,见了情魔色鬼走过不能收缚,任他流毒于苍生,肆恶于闺阃,乃老僧之罪也。既不能缚鬼受魔要这皮布袋何用?就拿去挂在大门外面松树梢头,又削一块小板,写几行细字,钉在松树上道:

未央生一日不至,皮布袋一日不收;皮布袋一日不烂,老和尚之心一日不死。但愿早收皮布袋,免教常坐肉蒲团。

这件东西却也古怪,自从未央生去那一日在松树上挂起,挂到如今,已是三年,不但一些不烂,反觉得比未挂之先倒硬挣起来。未央生走到时节,看见松树梢悬一个皮布袋,又看见树上有一块小板,小板上有两行小字,念了一遍,不觉痛哭起来。就把这条木板当做孤峰法像,跪在松树旁不知拜了几十拜,然后爬上树去,取了皮布袋下来,顶在头上,走入佛堂。遇着孤峰打坐,就跪在他面前,不住的磕头。从入定之初,磕到出定之后,约有三个时辰,岂止磕一百二十个响头而已。

孤峰走下蒲团,一把搀住道:“贤居士重来赐顾,就见盛情了,为何行此重礼?快请起来。”未央生道:“弟子愚蒙,悔当初不曾受得教悔,以至肆意胡行,把种种落地狱之事都做出来。如今,现在的阳报虽然受了,将来的阴报还不曾受,要求老师父哀怜,收在法座之下,使弟子忏悔前因,归依正果。不知老师父可肯收约否?”孤峰道:“既然收我皮布袋进来,我岂有不收纳之理。只恐你道念不坚,将来又有入尘之事。”未央生道:“弟子因悔恨之极,方才猛省回头。如今只当是从地狱里面逃走出来,那里还敢再去。自然没有反覆的,只求师父收纳。”孤峰道:“既然如此,收纳你就是。”未央生爬起身来,重新行礼。孤峰就拣个好日,替他落了头发。未央生告过孤峰,自取法名叫做“顽石”。一来自恨回头不早,有如顽石;二来感激孤峰善于说法,使三年不点头的顽石依旧点起头来。从此以后,立意参禅,专心悟道。

谁想少年出家到底有些不便,随你强制,淫心硬挠欲火。在日间念佛看经自然混过,睡到半夜,那孽物不知不觉就要磨起人来,不住在被窝中碍手绊脚,捺又捺它不住,放又放它不倒,只得要想个法子去安顿它。不是借指头救急,就是寻徒弟解纷,这两桩事是僧家的方便法门。未央生却不如此,他道出家之人,无论奸淫不奸淫,总要以绝欲为主。这两桩事虽然不犯条款,不丧名节,俱不能绝欲之心,与奸淫无异。况且手铳即房事之媒,男风乃妇人之渐,对假而思真,由此而及彼,此必然之势,不可不禁其初。偶然一夜,梦见花晨与香云姊妹到庵拜佛,连玉香、艳芳也在里面,未央生见了愤恨之极,就叫花晨与香云姊妹帮助他拿入,睡想转眼之间不见了玉香、艳芳两个,单单剩下四位旧交,就引他入禅房,大家脱了衣服,竟要做起胜会来。把阳物凑着阴门正要干起,被隔林犬吠忽然惊醒,方才晓得是梦。那翘然一物,竟在被窝里面东钻一下,西撞一头,要寻旧时的门户。顽石捏了这件东西,正要想个法子安顿它,又忽然止住道,我生平冤孽之根,皆由于此,它就是我的对头,如今怎么又放纵它起来。就止了妄念,要安睡一觉。

谁想翻来复去再睡不着,总为那件孽根在被里打搅。心上想道,有这件作祟之物带在身边,终久不妙,不如割去了它,杜绝将来之患。况且狗肉这件东西是佛家最忌之物,使它附与身体也不是好事。若不割去,只当是畜类,算不得是人身,就修到尽头地步,也只好转个人身,怎能成佛作祖?想到此处,不待天明,就在琉璃上点下火来,取一把切菜的薄刀。一手扭住阳物,一手拿起薄刀,恨命割下。也是他人身将转,畜运将终,割下的时节竟不觉十分疼痛。

从此以后,欲心顿绝,善念益坚。住了半年,还是泛泛修行,不曾摩顶受戒。到半年以后,聚了一二十僧,都是死心受戒,没有转念的人,请孤峰登坛说法。但凡和尚受戒,先要把生平做过的罪犯逐件自说出来,定了罪案,然后跪在佛前,求大和尚替他忏悔。若有一件不说出来,就是欺天诳佛,犯了不赦之条,随你苦修一世也成不得正果。

众僧请孤峰登坛拜毕,以入门之先后定了次第。大家分坐在两旁,孤峰把受戒的条规说了一番,就叫众僧各陈罪过,不得隐讳。顽石进门最迟坐在末席。一时轮未及他,只听得众僧里面也有杀人放火的,也有做贼奸淫的,皆自己陈告出来。后来轮着一僧,相貌粗笨,坐在顽石上首,也陈告道:“弟子生平不做恶事,只有卖身与人为仆、奸了主人之女,连她使女都拐出来,卖与青楼为妓这桩罪犯。真是死有余辜,求师父忏悔。”孤峰道:“你这罪重大,只怕忏悔不来。自古道‘万恶淫为首’,只消一个淫字也就够得紧了,怎么做出拐事来?又怎么卖她为娼?你这罪恶就有几世不得超升,我便替你忏悔,只恐菩萨不准,奈何?”和尚道:“禀告师父,这事是别人逼我做,不是我自己要做。只因那妇人的丈夫先奸我妻子,又逼我卖与他,我没有势力,敌他不过,所以逼上梁山,做了这事。其情可原,或者还可以忏悔。”

顽石听了,不觉动心,就问老师兄:“你拐他去卖的妇人叫甚么名字?是哪一家的妻子?那一家的女儿?如今在何处?”和尚道:“他是未央生之妻,铁扉道人之女,叫做玉香,丫鬟叫做如意,如今在京师接客。”未央生大惊道:“这等说来,你就是权老实了!”和尚道:“莫非你就是未央生么?”顽石道:“正是。”两个一齐走下蒲团,各赔个不是,然后对着孤峰共剖原情,各陈罪犯。孤峰大笑道:“好!冤家也有相会的日子。亏得佛菩萨慈悲,造了这条阔路,使两个冤家行走,一毫不碍。若在别路上相逢,就开交不得了。你两个罪犯原是忏悔不得,亏那两位夫人替丈夫还债,使你们的罪犯轻了许多。不然莫说修行一世,就修行十世也脱不得轮回,免不得劫数。我如今替你忏悔,求佛菩萨大舍慈悲看那两个妻子面上,宽待你们一分。”就叫两人跪在佛前,自己念起经来,替他俩忏悔。

忏悔之后,顽石又问道:“请问师父,奸淫之人既有妻子女儿,妻子还过了债,那怀抱中的幼女,也可以赦得他过,后来不还债么?”孤峰摇头道:“赦不过,赦不过。奸淫的人,除非不生女儿就罢,若生下女儿就是还债的种子。那里赦得她过。”未央生道:“不瞒师父说,弟子现有两个债种,将来定是不赦得了。弟子要别师父回去,用慧剑除了孽根,只当生来时节一盆水淹死了,不曾领起来的一般。”孤峰合掌念一声“阿弥陀佛”道:“如此恶言,不该出于你口,入于我耳。那里有受过法戒的和尚还想杀人的道理?”顽石道:“既不可杀,当用何法以处之?”孤峰道:“那两个孩子不是你的孩儿,是天公见你作恶不过,特送与你还债。古语说得好‘一善能解百恶’,你只是一心向善,没有转移,或者天公回心,替你收去,也不可知。何须用甚么慧剑?”顽石点头道:“是。”遂一心向善奉佛。

又过了半年,正在禅堂与孤峰讲话,忽见有个大汉闯进门来。顽石一看,见是赛昆仑。先参佛像,然后拜孤峰。顽石对孤峰道:“这人就是弟子的盟兄,叫做赛昆仑。是当今第一个侠士。”孤峰道:“莫非就是穿窬豪杰、生平有五不偷的人么?”顽石道:“然也。”孤峰道:“这等,是一尊贼菩萨了。贫僧何人,敢受得菩萨的拜?”就要跪下答拜。赛昆仑忙扯住道:“弟子今日到此,一来为访故人,二来为参活佛。师父若不受拜,是绝人向善之路,坚人作恶之心。可见天下人该做暗贼,不该做明贼;该做衣冠之贼,不该做穿窬之贼了。”孤峰道:“这等说,贫僧不敢回礼了。”赛昆仑又与顽石行礼,然后分宾主坐下,对孤峰叙了寒温,就立起身,要与顽石到后面去说话。顽石道:“小弟以前的事都与师父说过,家中有甚么隐情不妨面讲。”赛昆仑听了,依旧坐下道:“劣兄谋事不忠,不但不可托妻,亦且不堪寄子。今日相会甚觉无颜。”顽石道:“这等说来,想是家中的孽障有甚么原故了。”赛昆仑道:“你两位令爱,又无疾病,好好睡在床上,就一齐死了。临死之夜,两个乳母都梦见有人叫唤,说他家的账目都已算清,用你们不着,跟我回去罢。及至醒来,把孩子一摸就没用了。这事着实古怪。”顽石听了大喜,就怕自己惧怕女儿还债,师父教我一心向善,天公自然回心替你收去的话述了一遍。如今孽障消除,乃大幸之事,老兄怎么说起负托的话来。

赛昆仑闻言不觉毛骨竦然。听了一会,又道:“还有一个喜信报你。那淫妇艳芳背你逃走,其实可恨。小弟终日缉访不着。谁想被一个和尚拐去,藏在地窖中,被我无心看见,替你除了。”孤峰道:“她藏在地窖中可谓极稳的了,你怎么能看见?”赛昆仑道:“那个和尚常在三叉路口惯做谋财害命的事,我打听他有无数银子藏在地窖中。那一夜去偷他,谁想他睡在床上与妇人说话。我就躲在旁边细听,只见妇人道:‘我当初的原夫叫做权老实,虽然粗笨,倒是一马一鞍,没有别个妇人分宠。谁想赛昆仑替未央生做事,把我奸骗上手,强娶过去。他丢了自家妻子终日去走邪路,教我独守空房。弄到精力衰微,应付不来,又到远处去躲避差徭,不管家人的死活。这样的薄悻男子,我为甚么跟他?’弟子听了,知是艳芳,不觉大怒,拔出利剑掀起帐子,把两个杀了。然后点起火来,搜寻财物,约有二千多金都被弟子取来,任意挥霍,济了无数的穷人。请问师父,这两个男女该杀不该杀?这一注钱财该取不该取?”

孤峰道:“杀也该杀,取也该取,只是不该是居士杀,不该是居士取,恐天理王法上还有些说不过去,只怕阴阳二报定有所不免。”赛昆仑道:“人情痛快即是天理昭张,有何说不去?我做一世贼,不曾弄出事来,难道为这项银子就犯了王法不成?”孤峰道:“居士不要这等说,天理王法两件事都是一丝不漏的。没有一个不报,只是迟速之分。报的速的倒还轻些,报的迟的,忽然发作起来就当不起了。那和尚既犯了奸淫,那妇人既犯了私奔,天公自然会诛殛他,难道少了雷神霹雳,定要假手于人去杀他们不成?就作要假手于人,天下人个个有手,为甚么不去假他,单要借重你一个?难道只有你这手是杀得人死的不成?大权不可假人,太阿不容旁落,杀人的大事,天公能主持,使有罪之人依旧被有罪之人所杀,岂有付之不问之理。所以将来的阴报定不能免,或者比杀良善之人不同,罪略轻些也不可知。居士这桩事业既然做了一生,料想你的大名是没有一个衙门不知,没有一个官府不晓得了。你偷来的银子虽然济了穷人,别人不信,只说你藏在家中,少不得有个寻着你的日子。你往常所得的财物若果然藏在家中,还好送去买命,只怕济穷人的银子一时追不转来,就有性命之忧了。所以将来的阳报定不能免,只怕发作的迟,比初犯罪孽略重大些也不可知。”

赛昆仑平日原是些狼器的人,只因性子不好,人人惧怕他,所以善言不入于耳。如今听了这番正论,就不觉动了悔过之心,不消强逼,他竟有个反邪归正的意思。就对孤峰道:“弟子所做的事,原不是正人君子所为。只因世上有钱的人自家不肯挥霍,所以要去取些出来,替他做几件好事,只想为人,竟不想着自己。照师父说来,弟子作恶多端,阴阳二报都是不免的了。但如今从此回头,可还忏悔的去么?”孤峰指着顽石道:“他之作孽比彼还重得多。只因一心向善,就感动了天心,把还债的女儿都替他收他回去,这是你亲耳听见的话,不是贫僧附会出来的。即此一推,忏悔得去忏悔不去就知道了。”

顽石见他有向善之心,不胜之喜,就把自己三年前不受师父教训,肆意妄行,后来报应句句合着他所言,不可不以小弟为鉴。塞昆仑定了主意,就拜孤峰为师,削了头发,立志苦修二十年,成了正果。与孤峰、顽石一同坐化。

可见世上的人皆可作佛,只因被“财、色”二字缚住,不能跳脱迷津,超登彼岸。是以天堂之上,地广人稀;地狱之中,人稠地窄。上天大帝,清闻不过;阎罗天子,料理不来。总是开天辟地的圣人多事,不该生女子、设钱财,把人限到这地步。如今把这两句《四书》定他罪案,道: 始作俑者,其为圣人乎?

评曰:

开首处是感激圣人,收场处又埋怨圣人,使圣人欢喜不得,烦恼不得,真玩世之书也。仍以《四书》二句为圣人解嘲曰:知我者其为肉蒲团乎?罪我者其为肉蒲团乎?

\backmatter

\end{document}